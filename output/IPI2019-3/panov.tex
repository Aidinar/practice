\def\stat{panov}

\def\tit{ПЕРСОНАЛЬНЫЙ КОГНИТИВНЫЙ АССИСТЕНТ:\\ КОНЦЕПЦИЯ И~ПРИНЦИПЫ 
РАБОТЫ$^*$}

\def\titkol{Персональный когнитивный ассистент: концепция и~принципы 
работы}

\def\aut{И.\,В.~Смирнов$^1$, А.\,И.~Панов$^2$, А.\,А.~Скрынник$^3$, 
%В.\,А.~Исаков$^4$, 
Е.\,В.~Чистова$^4$}

\def\autkol{И.\,В.~Смирнов, А.\,И.~Панов, А.\,А.~Скрынник, Е.\,В.~Чистова}
%Е.\,В.~Чистова$^5$}

\titel{\tit}{\aut}{\autkol}{\titkol}

\index{Смирнов И.\,В.}
\index{Панов А.\,И.}
\index{Скрынник А.\,А.} 
%\index{Исаков В.\,А.}
\index{Чистова Е.\,В.}
\index{Smirnov I.\,V.}
\index{Panov A.\,I.}
\index{Skrynnik A.\,A.}
%\index{Isakov V.\,A.}
\index{Chistova E.\,V.}


{\renewcommand{\thefootnote}{\fnsymbol{footnote}} \footnotetext[1]
{Работа выполнена при частичной финансовой поддержке РФФИ (проект №\,18-29-22027).}}


\renewcommand{\thefootnote}{\arabic{footnote}}
\footnotetext[1]{Институт проблем искусственного интеллекта Федерального исследовательского центра 
<<Информатика и~управление>> Российской академии наук; Российский университет дружбы 
народов, \mbox{ivs@isa.ru}}
\footnotetext[2]{Институт проблем искусственного интеллекта Федерального исследовательского центра 
<<Информатика и~управление>> Российской академии наук; Московский фи\-зи\-ко-тех\-ни\-че\-ский 
институт (государственный университет), \mbox{pan@isa.ru}}
\footnotetext[3]{Институт проблем искусственного интеллекта Федерального исследовательского центра 
<<Информатика и~управление>> Российской академии наук, \mbox{skrynnik@isa.ru}}
\footnotetext[4]{Институт проблем искусственного интеллекта Федерального исследовательского центра 
<<Информатика и~управление>> Российской академии наук; Российский университет дружбы 
народов, \mbox{chistova@isa.ru}}

\vspace*{-12pt}
  
       
  

  \Abst{Предложена концепция когнитивного персонального ассистента. Когнитивный 
ассистент выступает виртуальным интеллектуальным агентом, обладающим своей 
собственной картиной мира (КМ) и~строящим КМ пользователя, которому он помогает 
решать различные задачи. Описана архитектура когнитивного ассистента, рассмотрены 
основные функции, которые он должен реализовывать, и~представлены основные методы 
и~технологии, которые используются при построении такого рода ассистентов. Рассмотрены 
две предметные области, в~которых использование когнитивных ассистентов наиболее 
перспективно.}
  
  \KW{когнитивный ассистент; образовательный ассистент; медицинский ассистент; 
знаковая картина мира; обработка естественного языка; диалоговая система; сценарий; 
планирование}

\DOI{10.14357/19922264190315} 
  
%\vspace*{6pt}


\vskip 10pt plus 9pt minus 6pt

\thispagestyle{headings}

\begin{multicols}{2}

\label{st\stat}
  
\section{Введение}

  Разработки в~области интеллектуальных ас\-сис\-тен\-тов~--- помощников человека 
при совершении им каждодневных задач (выбор товаров и~услуг, поиск 
в~интернете, прокладка маршрута и~навигация во время вождения, голосовое 
управление бытовыми устройствами и~т.\,п.)~--- получили в~последнее время 
новый импульс своего развития в~связи с~появлением новых методов анализа 
естественного языка и~глубокого обучения. Методы распознавания и~синтеза 
речи, лингвистического анализа и~синтеза текстов достигли сегодня такого 
уровня, который обеспечил создание промышленных голосовых помощников, 
таких как Siri, Cortana, Alexa, Алиса. 
%
Однако до сих пор не решена задача 
создания полноценного ассистента, который бы не только действовал 
реактивно в~ответ на запросы пользователя, но и~вел бы себя проактивно, 
предсказывая поведение пользователя, а~также обладал целеполаганием 
и~внутренней мотивацией по достижению поставленных перед ним целей. Во
многом это связано с~тем, что основной упор при создании современных 
ассистентов делается преимущественно на поддержание  
воп\-рос\-но-от\-вет\-ной коммуникации, т.\,е.\ создание диалоговых \mbox{систем}. 
  
  Для решения же поставленной цели генерации проактивных 
целенаправленных действий ассистент должен обладать элементами 
искусственного сознания, собственной моделью поведения и~строить 
аналогичную по сложности модель поведения пользователя на основе 
понимания разговорного языка. Только в~таком случае интеллектуальный агент 
сможет стать полноценным помощником человека, а не только голосовым 
интерфейсом для разных приложений, например поисковых.
  
  В настоящей работе предлагается концепция персонального когнитивного 
ассистента, т.\,е.\ такого виртуального ассистента, который обладал бы 
проактивным целенаправленным поведением и~моделировал поведение 
собеседника. Описана архитектура, принципы работы, основные 
функции когнитивного ассистента, предложены методы реализации этих 
функций. Возможные варианты использования когнитивного ассистента 
пред\-став\-ле\-ны для двух предметных областей: ассистирование в~процессе 
обучения пользователя (образовательный ассистент) и~ассистирование 
в~процессе поддержания здоровья пользователя (виртуальный тренер 
здоровья).

  \begin{figure*}] %fig1
   \vspace*{1pt}
    \begin{center}  
  \mbox{%
 \epsfxsize=137.411mm 
 \epsfbox{pan-1.eps}
 }
\end{center}
\vspace*{-9pt}
  \Caption{Архитектура когнитивного ассистента}
  \end{figure*}
 
 \vspace*{-14pt} 
  
\section{Архитектура когнитивного ассистента}

\vspace*{-4pt}

  На рис.~1 представлена принципиальная схема устройства и~работы 
когнитивного ассистента. Основным компонентом агента является его 
КМ, которая моделируется на основе знакового подхода~[1, 2]. 
Основным компонентом знаковой КМ служит \textit{знак}, 
представляющий собой на синтаксическом уровне описания модели  
(по~\cite{2-sm}) четверку  
$$
s=\langle n,p,m,a\rangle\,,
$$
 где $n\hm\in N$; 
$p\hm\subset P$; $m\hm\subset M$; $a\hm\subset A$. Здесь
$N$~--- \textit{множество 
имен}, представляющее собой множество слов конечной длины в~некотором 
алфавите; $P$~---\linebreak множество замкнутых атомарных формул языка исчис\-ле\-ния 
предикатов первого порядка, которое называется \textit{множеством образов}; 
$M$~--- множество значений; $A$~--- множество личностных смыслов.
  
  В случае так называемой житейской КМ, компонента образа знака 
участвует в~процессе распознавания и~категоризации. Значения представляют 
фиксированные, сценарные знания интеллектуального агента о предметной 
области и~окружающей среде, а множество личностных смыслов характеризует 
его предпочтения и~текущий деятельностный контекст. Компонента имени 
осуществляет связывание остальных компонент знака в~единое целое 
(именование).
  

  На структурном уровне описания знаковой КМ каждая компонента 
знака представляет собой множество каузальных матриц, которые 
представляют собой структурированное множество ссылок на другие знаки, 
либо элементарные компоненты (в~случае образа~--- это первичный признак 
или данные с~сенсоров, в~случае личностного смысла~--- это операционный 
состав действия). Каузальная матрица позволяет кодировать информацию для 
представления как декларативных, так и~процедурных знаний. Множество 
компонент знака образуют четыре типа каузальных сетей~--- специального 
типа семантических сетей. Моделирование функций планирования 
и~рассуждения осуществляется за счет введения понятия ак\-тив\-ности (множества 
активных знаков или каузальных матриц) и~правил распространения 
активности по различным типам сетей~[3]. В~процессе работы той или иной 
когнитивной функции формируются новые каузальные матрицы, которые 
могут затем сохраняться в~составе компонент нового знака аналогично 
сохранению опыта в~системах, основанных на прецедентах.
  
  Отдельное множество знаков в~КМ ассистента отвечает за моделирование 
КМ пользователя. Это представление ассистента о свойствах, 
возможностях и~целях другого субъекта хранится в~тех же структурах, что 
и~основная информация о~предметной об\-ласти. В~КМ ассистента 
присутствуют специальные знаки~<<Я>> и~<<Другой>> (<<Пользователь>>), 
которые позволяют ассистенту различать информацию, относящуюся 
непосредственно к~нему или другому субъекту.
  
  На основе знаковой КМ ассистент способен строить свои 
собственные планы действий, в~которые он может включать действия 
(реакцию) других субъектов, в~том числе пользователя (см.\ п.~2.2.1). Картина мира 
ассистента описывает его назначение, цели, возможные действия и~сценарии, 
личностные смыслы, оценки достижения целей. Модель КМ пользователя 
строится за счет выявления его сценариев и~личностных смыслов, ценностей, 
предпочтений, привычек и~т.\,п. Когнитивный ассистент общается 
с~собеседником с~учетом этих двух~КМ.
  
  Основными структурами на множестве знаков, которые позволяют 
генерировать проактивные действия, являются сценарии~--- переиспользуемые 
абстрактные последовательности действий и~ситуаций~\cite{2-sm}. 
Когнитивный ассистент имеет сценарии собственных действий и~сценарии 
действий пользователя. Таким образом, ассистент имеет представление 
о~последовательности действий, необходимых для решения той или иной 
задачи, что позволяет ему ассистировать пользователю при решении задач, 
подсказывая дальнейшие действия с~учетом персональных сценариев, 
характерных для конкретного пользователя.
  
  Когнитивный ассистент требует предварительного обучения и~настройки за 
счет внешних ис\-точников, по которым он формирует базовые сценарии 
деятельности, характерные для данной\linebreak предметной области. Автоматическое 
или автоматизированное пополнение КМ сценариями происходит за счет двух 
основных источников текстовой и~нетекстовой (видео, изображения, данные 
с~различных сенсоров) информации: внешние источники (интернет, коллекции 
документов из данной предметной об\-ласти и~т.\,п.)\ и~пользователь, который 
передает ас\-сис\-тен\-ту текст в~виде запросов и~описания задач либо через 
программный интерфейс, который по желанию пользователя может 
регистрировать его различные характеристики (например, био\-мет\-ри\-че\-ские 
данные, психологические черты лич\-ности~\cite{4-sm}, действия с~интерфейсом 
или действия с~другими устройствами). 

Автоматическое извлечение 
и~формирование сценариев по тексту происходит с~помощью подходов, 
описанных в~п.~2.2.2. До\-пус\-ти\-мо по\-стро\-ение КМ ассистента 
и~вручную экспертом предметной об\-ласти.
  
  Одна из важнейших особенностей ассистента~--- его способность 
генерировать текст на естественном языке и~вести связный диалог (см.\ 
п.~2.2.3).

%\vspace*{-6pt}
  
  \subsection{Режимы работы} %2.1
  
  Когнитивный ассистент работает в~нескольких режимах, автоматически 
выбирая режим в~зависимости от текущего диалога или высказывания:
  \begin{itemize}
\item диалог на свободную тему (chit-chat);
\item вопросно-ответный режим;
\item целеориентированный режим.
\end{itemize}

  В диалоговом режиме ассистент поддерживает беседу на свободную тему, 
отвечает на приветствия, спрашивает о настроении пользователя. В~этом 
режиме ассистент также выполняет простейшие просьбы, например рассказать 
анекдот или выдать прогноз погоды.
  
  В вопросно-ответном режиме ассистент находит точный ответ на 
поставленный пользователем вопрос. Вопрос может задаваться на естественном 
языке в~свободной форме. При поиске ответа предполагается учитывать 
интересы и~текущее настроение пользователя.
  
  Целеориентированный режим предполагает помощь ассистента в~решении 
конкретных задач пользователя. Этот режим в~значительной степени 
использует КМ ассистента и~пользователя и~зависит от назначения 
ассистента.

%\vspace*{-3pt}
  
  \subsection{Основные методы} %2.2.
  
 % \vspace*{-3pt}
  
  \subsubsection{Планирование поведения} %2.2.1
  
  На этапе синтеза плана деятельности когнитивный ассистент рекурсивно 
создает все возможные планы по достижению конечной ситуации, которая 
описывает целевое состояние ассистента и~пользователя. Для этого ассистентом 
рассматриваются все знаки, которые входят в~описание текущей\linebreak 
ситуации~$z_{\mathrm{sit}\mbox{-}\mathrm{cur}}$, и~с~по\-мощью процесса распространения 
активности по сети значений~\cite{3-sm, 5-sm}\linebreak активируются процедурные 
матрицы действий. Затем актуализируются матрицы действий, заменяются 
ссылки на знаки ролей и~типов объектов на ссылки конкретных объектов 
задачи. Далее следует шаг выбора действий, которые эвристически были 
оценены как наиболее подходящие в~ситуации~$z_{\mathrm{sit}\mbox{-}\mathrm{cur}}$ для 
достижения ситуации~$z_{\mathrm{sit}\mbox{-}\mathrm{goal}}$. После этого из эффектов каждого 
действия и~ссылок на знаки, которые входят в~текущую ситуацию, строится 
$z_{\mathrm{sit}\mbox{-}\mathrm{cur}+1}$, которая описывает состояние агента после применения 
действия. В~план добавляется рассматриваемое действие  
и~$z_{\mathrm{sit}\mbox{-}\mathrm{cur}}$, затем проверяется
 вхождение~$z_{\mathrm{sit}\mbox{-}\mathrm{goal}}$ 
в~$z_{\mathrm{sit}\mbox{-}\mathrm{cur}+1}$. Если матрицы текущего состояния включают 
матрицы целевого состояния, то алгоритм сохраняет найденный план как один 
из возможных; если мат\-ри\-цы целевого со\-сто\-яния
не входят, то функция поиска плана рекурсивно 
повторяется.

  
 % \vspace*{-6pt}
  
  \subsubsection{Автоматическое формирование сценариев} %2.2.2
  
  %\vspace*{-2pt}
  
   Основной задачей для когнитивного ассистента является задача 
автоматизированного формирования КМ, прежде всего сценариев 
решения задач и~сети значений знаков. Предполагается, что основным 
источником для формирования КМ служат наборы текстов. Для 
конструирования сценариев по текстам предлагается использовать подходы 
открытого извлечения информации из текстов~\cite{6-sm, 7-sm}.

  
  Сеть на значениях в~КМ состоит из концептов и~связей между ними. Концепт 
представляет собой предмет или явление в~рамках предметной области. Разные 
лексические единицы могут ссылаться на один и~тот же концепт. Например, 
<<центральное обрабатывающее устройство>> и~<<процессор>>. 
  
  Между концептами существуют два типа связей:
  \begin{enumerate}[(1)]
\item таксономические связи образуют иерархии концептов. Например, 
<<попугай~--- это птица>>. <<Птица>>~--- гипероним по отношению 
к~<<попугай>>;
\item нетаксономические связи являются предикатами, описывающими 
взаимодействие концептов. Например, <<Эксперт размечает корпус>>.
\end{enumerate}

  Основные шаги пополнения сети значений.
  \begin{enumerate}[1.]
\item  Выполняется графематический, морфологический и~синтаксический 
анализ текстов. На этапе графематического анализа происходит выделение 
предложений из текста и~выделение слов из предложений (токенизация). 
Морфологический анализ позволяет получить леммы (нормальные формы) 
и~морфологические признаки для каждого слова из текста. В~результате 
синтаксического анализа текстов генерируются синтаксические деревья 
зависимостей для каждого предложения. 
  \item Извлекаются всевозможные именные группы из синтаксических 
деревьев зависимостей с~помощью следующего алгоритма:
  \begin{itemize}
\item выполняется поиск вершины синтаксического дерева (как правило, это 
глагол);
\item происходит спуск по дереву до первого существительного;
\item найденное существительное сохраняется вместе с~потомками в~качестве 
именной группы, вплоть до первого слова, часть речи которого не входит 
в~следующий список: существительное, прилагательное, местоимение, 
числительное, имя собственное, союз, наречие, причастие.
\end{itemize}
   \item Извлекаются термины предметной области путем кластеризации 
выделенных именных групп с~учетом метрики C-value:
\begin{multline*}
  \mathrm{C}\mbox{-}\mathrm{Value}(a)={}\\
  {}=\begin{cases}
  \log_2\vert a\vert \cdot \mathrm{freq}(a)\,, &\hspace*{-33mm}\mbox{если именная группа}\\
  &\hspace*{-33mm}\mbox{не вложена  в~другие}\,;\\
  \displaystyle \log_2\vert a\vert \cdot \mathrm{freq}(a)-\fr{1}{p(T_a)}
  \,\sum\limits_{b\in T_a} \mathrm{freq}(b)&\\
& \hspace*{-10.5mm}\mbox{иначе}\,,
  \end{cases}
\end{multline*}
    где $a$~--- именная группа; $\vert a\vert$~--- число слов в~именной группе; 
    $\mathrm{freq}(a)$~--- частота 
встречаемости~$a$; $T_a$~--- именные группы, в~которые входит~$a$;
$p(T_a)$~--- число именных групп, содержащих~$a$.
  
  Данная метрика учитывает пересечения лексики между словосочетаниями 
и~позволяет выделять многословные термины.
  
  \item Для извлечения концептов выполняется клас\-те\-ри\-за\-ция полученных 
терминов. В~качестве признаков используются векторные пред\-став\-ле\-ния 
слов~\cite{8-sm}. Для словосочетаний используются усредненные векторы.
  \item С~помощью набора эвристик из полученных концептов извлекаются 
таксономические связи. Эвристики в~первую очередь опираются на предлоги, 
союзы и~морфологические признаки.
\item Кандидаты в~таксономические связи извлекаются путем поиска 
глагольных групп в~синтаксическом дереве зависимостей. Поиск вы\-полняется 
между лексическими единицами,\linebreak связанными с~концептами.
  \item На конечном этапе таксономические связи извлекаются путем 
кластеризации нескольких признаков кандидатов: векторное представление 
глагольной группы, идентификаторы концептов, между которыми найдена 
глагольная группа.
  \item  На основе ряда эвристик по тексту формируются последовательности 
найденных троек (кон\-цепт\,--\,гла\-голь\-ная груп\-па\,--\,кон\-цепт).
  \end{enumerate}
  
  В результате выявленные в~тексте последовательности троек и~представляют 
собой сценарии решения задач с~необходимыми участниками и~орудиями 
действий.
  
  \subsubsection{Вопросно-ответный режим} %2.2.3 
  
  Вопросно-ответный режим предполагается\linebreak реализовать с~помощью 
технологий Exactus~\cite{9-sm}, основанных на 
 ре\-ля\-ци\-он\-но-си\-ту\-а\-ци\-он\-ном и~се\-ман\-ти\-ко-син\-так\-си\-че\-ском 
анализе текста~[10, 11]. Ре\-ля\-ци\-он\-но-си\-ту\-а\-ци\-он\-ный анализ текста 
пред\-став\-ля\-ет семантику текста в~виде семантической сети, а семантика 
предложения или высказывания при этом пред\-став\-ля\-ет\-ся в~виде со\-во\-куп\-ности 
предикатных слов, их аргументов и~семантических ролей. Семантические сети 
строятся для вопроса и~каждого текста, в~котором может находиться 
формулировка ответа, затем происходит сопоставление семантических сетей 
вопроса и~текстов, вычисляется релевантность текстов вопросу. Таким образом, 
для работы необходим набор текстов, в~которых потенциально могут 
содержаться ответы на вопросы, т.\,е.\ ответы ищутся, а не генерируются.  
Воп\-рос\-но-от\-вет\-ный поиск реализуется в~два этапа: на первом 
выполняется семантический поиск предложений, содержащих формулировку 
ответа на поставленный вопрос; на втором этапе из предложения выделяется 
фрагмент, являющийся точным ответом на поставленный вопрос.
  
  В работе~\cite{8-sm} было показано, что учет семантической структуры 
предложения в~воп\-рос\-но-от\-вет\-ном поиске значительно повышает 
качество поиска ответов по сравнению с~лексическим критерием 
ранжирования, а также позволяет извлекать сам ответ на вопрос. В~2010~г.\ 
технология Exactus была представлена на российском семинаре по оценке 
методов информационного поиска РОМИП в~дорожке во\-прос\-но-от\-вет\-но\-го 
поиска и~показала высокие результаты по всем метрикам~\cite{12-sm}.
  
  \subsubsection{Диалоговый режим}%2.2.4
  
  Для реализации режима диалога на свободную тему на русском языке 
предлагается использовать порождающие подходы на основе различных 
нейросетевых моделей. В~работе~\cite{13-sm} показано, что добавление 
к~модели seq2seq механизма внимания повышает качество и~грамматическую 
согласованность генерируемых реплик в~диалоге. В~будущем возможна 
комбинация данного подхода к~генерации ответов и~использование баз знаний 
для того, чтобы модель оперировала более конкретными представлениями 
в~каждой области.

\section{Применение в~образовательном процессе}

  Один из вариантов использования ассистента, строящего модель 
КМ собеседника, а~также обладающего одним из вариантов КМ,~--- 
  его применение в~процессах он\-лайн-обуче\-ния. 
  
  Образовательные он\-лайн-сис\-те\-мы (к примеру, Cursera, Stepik, Logiclik, 
Examer и~др.)\ сейчас используют методы искусственного интеллекта для 
улучшения качества обучения пользователей при решении следующих задач:
\begin{enumerate}[1.]
\item Выбор образовательной траектории, подстройка блоков курса, иными 
словами, адаптация программы обучения по выбранной пользователем теме 
в~зависимости от когнитивных особенностей этого пользователя.
\item Организация frequently asked quetions (FAQ) (раздела во\-про\-сов-от\-ве\-тов) по данному курсу, 
в~котором ассистент заменяет преподавателя при ответе на стандартные 
вопросы по курсу.
\item Подсказки учителю по время подготовки или проведения урока, которые 
формируются в~зависимости от реакции аудитории на прошлые занятия, либо 
в~зависимости от когнитивных особенностей преподавателя.
\item Прокторинг~--- отслеживание поведения пользователя во время 
просмотра курса или выполнения задания с~учетом его поведенческих 
характеристик, подача ему предупреждений или советов.
\item Автоматическая проверка выполненного задания, выдача рекомендаций 
по исправлению ошибок или по выполнению дополнительного задания на одну 
из тем, по которой пользователь допустил ошибку, с~учетом его эмоциональных 
и~когнитивных особенностей.
\item Формирование советов по дальнейшему усвоению курса как для 
пользователя, так и~для преподавателя.
\item Мотивация пользователя путем генерации соответствующих его 
состоянию и~психологическим особенностям реплик или вывод его на диалог.
\end{enumerate}

  Все упомянутые выше поведенческие, психологические и~когнитивные 
особенности должны определяться когнитивным ассистентом в~рамках 
КМ пользователя, модель которой он строит. Гипотеза авторов состоит в~том, 
что весь комплекс упомянутых функций и~корректный учет всех особенностей 
пользователя системой он\-лайн-об\-ра\-зо\-ва\-ния невозможно обеспечить без 
использования специальных методов моделирования его знаковой 
КМ. При этом для реализации некоторых функций (например,~4 или~7) 
необходимо, чтобы агент сам обладал КМ (возможно, статичной, 
возможно, эволюционирующей), которая позволяла бы ему вести более 
полноценные диалоги и~демонстрировать собственные психологические 
особенности для достижения большего эффекта от общения с~пользователем.
  
  На рис.~2 приведен примерный вид интерфейса программной реализации 
такого ассистента, который выполняет роль помощника при обучении игры 
в~шахматы.
  

  
  Ассистент предварительно обучен на текстах шахматных книг (в~том числе 
детских и~художественных, в~которых встречается большее разнообразие 
простых языковых конструкций)~--- по ним он формирует базовую 
КМ потенциального пользователя и~свою собственную КМ учителя, 
используя методы п.~2.2.1. Также агент обладает функционалом классических 
шахматных программ (система\linebreak\vspace*{-12pt}

{ \begin{center}  %fig2
 \vspace*{-3pt}
\mbox{%
 \epsfxsize=79mm 
 \epsfbox{pan-2.eps}
 }


\end{center}

%\vspace*{-3pt}


\noindent
{{\figurename~2}\ \ \small{Пример интерфейса образовательного когнтивного ассистента по игре 
в~шахматы}}

}


\vspace*{12pt}


  
  \noindent
   просчета вариантов и~т.\,п.). Виды 
формируемых сценариев: шахматные розыгрыши в~стандартных позициях, 
правила оценки той или иной позиции, общие сценарии поведения учителя 
(того, кто советует, направляет, подсказывает) и~ученика (того, кто что-то 
решает, приобретает знания). Ассистент учится во время взаимодействия 
с~учеником, обновляя как свою КМ, так и~информацию о~КМ 
пользователя (какие ошибки совершает, что интересно).
  
  Информация, которую получает ассистент от пользователя: выбор  
ка\-кой-ли\-бо темы обучения (дебют, окончания, тактика, стратегия, разбор 
партий и~т.\,п.), ходы при решении предложенных задач и~время от времени 
вопросы и~ответы в~текстовом поле. Также возможно отслеживание по 
видеокамере поведения пользователя в~процессе решения задач.
  
  Основные задачи ассистента: по сформированным (и~пополняемым 
в~процессе работы) в~КМ сценариям для пользователя ассистент 
(с~учетом\linebreak особенностей своей собственной КМ) выдает 
мотивирующие реплики пользователю, задает ему вопросы, отвечает на его 
вопросы, предлагает подсказки в~различных ситуациях, дает ему советы\linebreak после 
партии, предлагает ему новые задачи в~рамках выбранного курса для 
ликвидации определенных недостатков в~знаниях пользователя, предлагает ему 
пройти необходимый ему новый курс\linebreak (например, король и~пешка против 
короля). 

\section{Применение в~здоровьесбережении}

  Другим вариантом использования когнитивного ассистента является его 
применение в~процессах здоровьесбережения. Сегодня наблюдается активный 
рост объема научных исследований и~разработок в~области 
здоровьесбережения~--- профилактики и~поддержания здоровья людей. 
Основная цель этого направления~--- анализ особенностей здоровья 
конкретного человека и~подбор и~проведение персонализированных 
профилактических мероприятий до появления первых симптомов возможных 
заболеваний. Как можно более ранний\linebreak
 прогноз возникновения заболеваний 
служит снижению риска патологии. В~настоящее время разработаны 
интеллектуальные системы, под\-дер\-жи\-вающие процесс здоровьесбережения на 
всех\linebreak стадиях~\cite{14-sm}. Основные задачи технологий здоровьесбережения 
включают: 
  \begin{itemize}
  \item [(а)] сбор данных о состоянии здоровья и~образе жизни человека из 
различных источников;
  \item[(б)] интеллектуальный анализ данных о состоянии здоровья и~образе 
жизни человека для выявления проблем со здоровьем, оценку 
персонализированных рисков ухудшения здоровья человека;
  \item[(в)] подбор персональных рекомендаций по изменению образа жизни 
конкретного человека в~зависимости от состояния его здоровья, образа жизни, 
проблемных зон и~индивидуальных особенностей;
  \item[(г)] формирование персонального плана профилактических 
мероприятий;
  \item[(д)] мотивацию человека к~выполнению рекомендаций, отслеживанию 
изменений в~образе жизни и~рисках заболеваний.
  \end{itemize}
  
  Персональный когнитивный ассистент в~виде виртуального персонального 
тренера здоровья может использоваться при решении всех указанных задач. 
Предполагается, что виртуальный тренер здоровья постоянно мониторит 
психическое и~физическое состояние пользователя, сообщая пользователю 
о~критических изменениях. Он может быть интегрирован с~различными 
гаджетами (фит\-несс-тре\-ке\-ра\-ми) и~мобильными приложениями, может 
анализировать активность в~социальных сетях, учитывать пищевые 
предпочтения пользователя с~помощью фуд-тре\-кера. 
  
  Одним из основных способов получения информации об образе жизни 
и~здоровье человека будет проактивный диалог ассистента с~пользователем. 
Например, предполагается, что ассистент сам инициирует диалог~--- 
спрашивает, как настроение сегодня, просит ответить на вопросы 
психологического теста, напоминает о необходимости запланированных 
действий, например приема пищи, отдыха, физической активности. 
Когнитивный ассистент может ответить на любой вопрос по теме 
здо\-ровье\-сбе\-ре\-же\-ния или отправить пользователя к~информации, содержащей 
ответ на вопрос, обладает эмоциями и~выражает их с~помощью смайлов или 
эмодзи, учитывает КМ и~текущее настроение пользователя при 
выражении эмоций. При формировании плана профилактических мероприятий 
ассистент также должен учитывать КМ пользователя, его 
предпочтения. 

\section{Заключение}

  Сегодня существуют много виртуальных ас\-сис\-тен\-тов и~чат-бо\-тов, 
нуждающихся в~интеллектуальной начинке, которая была бы основана на 
моделировании когнитивных функций человека\linebreak и~повышала качество работы 
ассистентов. Когнитивный ассистент отличается от аналогов наличием 
формализованных знаний о~час\-ти окру\-жа\-юще\-го мира и~способах решения 
определенных задач, а~также способностью учитывать КМ
пользователя в~процессе ассистирования ему. 
  
  В результате дальнейших работ планируется создание семейства 
интеллектуальных ассистентов и~соответствующих технологий, легко 
настраиваемых на решение новых задач. На основе предложенной концепции 
персонального когнитивного ас\-сис\-тен\-та предполагается разработка 
масштабируемой программной платформы для создания ас\-сис\-тен\-тов 
различного назначения и~их настройки под конкретные задачи и~предметные 
области. Такие ас\-сис\-тен\-ты будут способны к~общению с~человеком (или 
другим виртуальным ас\-сис\-тен\-том) на естественном языке и~смогут 
встраиваться в~различные программы (мессенджеры, социальные сети) или 
искусственные технические устройства (например,\linebreak робототехнические 
ас\-сис\-тен\-ты). Когнитивные ас\-сис\-тен\-ты наиболее востребованы в~автономных 
интел\-лек\-ту\-аль\-ных устройствах или их коалициях, действующих в~опасных 
средах с~отложенной коммуникацией с~человеком.
  
 {\small\frenchspacing
 {%\baselineskip=10.8pt
 \addcontentsline{toc}{section}{References}
 \begin{thebibliography}{99}
 
  \bibitem{2-sm}
  \Au{Осипов Г.\,С., Панов А.\,И.} Отношения и~операции в~знаковой картине 
мира субъекта поведения~// Искусственный интеллект и~принятие решений, 
2017. №\,4. С.~5--22.

 \bibitem{1-sm}
  \Au{Осипов Г.\,С., Чудова~Н.\,В., Панов~А.\,И., Кузнецова~Ю.\,М.} Знаковая 
картина мира субъекта поведения.~--- М.: Физматлит, 2018. 264~с.

  \bibitem{3-sm}
  \Au{Панов А.\,И.} Целеполагание и~синтез плана поведения когнитивным 
агентом~// Искусственный интеллект и~принятие решений, 2018. №\,2.  
С.~21--35.
  \bibitem{4-sm}
  \Au{Stankevich M., Smirnov~I., Ignatiev~N., Grigoriev~O., Kiselnikova~N.} 
Analysis of big five personality traits by processing of social media users activity 
features~// CEUR Workshop Procee., 2018.  
%Data Analytics and Management in Data Intensive Domains 2018: Selected Papers of the XX 
%International Conference (DAMDID/RCDL 2018). 
Vol.~2277. P.~162--166.
  \bibitem{5-sm}
  \Au{Киселев Г.\,А., Панов~А.\,И.} Знаковый подход к~задаче распределения 
ролей в~коалиции когнитивных агентов~// Труды СПИИРАН, 2018. №\,2. 
С.~161--187.
  \bibitem{6-sm}
  \Au{Шелманов А.\,О., Исаков~В.\,А., Станкевич~М.\,А., Смирнов~И.\,В.} 
Открытое извлечение информации из\linebreak текс\-тов. Часть~I. Постановка задачи 
и~обзор методов~// Искусственный интеллект и~принятие решений, 2018. №\,2. 
С.~47--67.
  \bibitem{7-sm}
  \Au{Шелманов А.\,О., Девяткин~Д.\,А., Исаков~В.\,А., Смирнов~И.\,В.} 
Открытое извлечение информации из текстов. Часть~II. Извлечение 
семантических отношений с~помощью машинного обучения без учителя~// 
Искусственный интеллект и~принятие решений, 2019. №\,2. С.~39--49.
  \bibitem{8-sm}
  \Au{Mikolov T., Sutskever~I., Chen~K., Corrado~G.\,S., Dean~J.} Distributed 
representations of words and phrases and their compositionality~// Adv.
Neur. Inf., 2013. Vol.~26. P.~3111--3119.
  \bibitem{9-sm}
  \Au{Шелманов А.\,О., Каменская~М.\,И., Ананьева~И.\,В., Смирнов~И.\,В.} 
Семантико-синтаксический анализ текстов в~задачах  
воп\-рос\-но-от\-вет\-но\-го поиска и~извлечения определений~// Искусственный 
интеллект и~принятие решений, 2016. №\,4. C.~47--61.
  \bibitem{10-sm}
  \Au{Осипов~Г.\,С., Смирнов~И.\,В., Тихомиров~И.\,А.}  
Ре\-ля\-ци\-он\-но-си\-ту\-а\-ци\-он\-ный метод поиска и~анализа текстов и~его 
приложения~// Искусственный интеллект и~принятие решений, 2008. №\,2. 
С.~3--10.
  \bibitem{11-sm}
  \Au{Смирнов~И.\,В., Шелманов~А.\,О., Кузнецова~Е.\,С.,\linebreak Храмоин~И.\,В.} 
Семантико-синтаксический анализ естественных языков. Часть~II. Метод 
се\-ман\-ти\-ко-син\-так\-си\-че\-ско\-го анализа текстов~// Искусственный интеллект 
и~принятие решений, 2014. №\,1. С.~11--24.
  \bibitem{12-sm}
  \Au{Завьялова О.\,С., Киселёв~А.\,А., Осипов~Г.\,С.,
  Смирнов~И.\,В., Тихомиров~И.\,А., Соченков~И.\,В.} 
Система интеллектуального поиска и~анализа информации <<EXACTUS>> на  
\mbox{РОМИП}-2010~// Тр. Российского семинара по оценке методов 
информационного поиска.~--- Казань: Казанский ун-т, 2010. 
С.~49--69.
  \bibitem{13-sm}
  \Au{Чистова Е.\,В., Шелманов~А.\,О., Смирнов~И.\,В.} Применение 
глубокого обучения к~моделированию диалога на естественном языке~// Тр. 
Института системного анализа РАН, 2019. Т.~69. №\,1. С.~105--115.
  \bibitem{14-sm}
  \Au{Grigoriev O.\,G., Kobrinskii~B.\,A., Osipov~G.\,S., Molodchenkov~A.\,I., 
Smirnov~I.\,V.} Health management system knowledge base for formation and 
support of a~preventive measures plan~// Procedia Comput. Sci., 2018. 
Vol.~145. P.~238--241.
  \end{thebibliography}

 }
 }

\end{multicols}

\vspace*{-6pt}

\hfill{\small\textit{Поступила в~редакцию 28.01.19}}

%\vspace*{8pt}

%\pagebreak

\newpage

\vspace*{-28pt}

%\hrule

%\vspace*{2pt}

%\hrule

%\vspace*{-2pt}

\def\tit{PERSONAL COGNITIVE ASSISTANT:\\ CONCEPT AND~KEY 
PRINCIPALS}


\def\titkol{Personal cognitive assistant: Concept and~key 
principals}

\def\aut{I.\,V.~Smirnov$^{1,2}$, A.\,I.~Panov$^{1,3}$, 
  A.\,A.~Skrynnik$^1$, %V.\,A.~Isakov$^1$, 
  and~E.\,V.~Chistova$^{1,2}$}

\def\autkol{I.\,V.~Smirnov, A.\,I.~Panov,    A.\,A.~Skrynnik, 
 and~E.\,V.~Chistova}

\titel{\tit}{\aut}{\autkol}{\titkol}

\vspace*{-11pt}


\noindent
  $^1$Institute of Artificial Intelligence Problems, Federal Research Center 
``Computer Science and Control'' of the\linebreak
$\hphantom{^1}$Russian Academy of Sciences; 9~60-letiya 
Oktyabrya Prosp., Moscow 117312, Russian Federation
  
  \noindent
  $^2$Friendship University of Russia (RUDN University), 6~Miklukho-Maklaya 
Str., Moscow 117198, Russian\linebreak
$\hphantom{^1}$Federation
  
  \noindent
  $^3$Moscow Institute of Physics and Technology (State University), 9~Institutskiy 
Per., Dolgoprudny, Moscow Region\linebreak
$\hphantom{^1}$141701, Russian Federation

\def\leftfootline{\small{\textbf{\thepage}
\hfill INFORMATIKA I EE PRIMENENIYA~--- INFORMATICS AND
APPLICATIONS\ \ \ 2019\ \ \ volume~13\ \ \ issue\ 3}
}%
 \def\rightfootline{\small{INFORMATIKA I EE PRIMENENIYA~---
INFORMATICS AND APPLICATIONS\ \ \ 2019\ \ \ volume~13\ \ \ issue\ 3
\hfill \textbf{\thepage}}}

\vspace*{3pt}   
  
    
  
  \Abste{The paper proposes the concept of cognitive personal assistant. The 
cognitive assistant is a virtual intelligent agent that has its own sign-based world 
model and builds a world model of the user, which it helps to solve various problems. 
The architecture of the cognitive assistant is described, the main functions that it 
should implement are considered, and the main methods and technologies that are 
used in the construction of such assistants are presented. Two subject areas in which 
the use of cognitive assistants is the most promising are considered.}
  
  \KWE{cognitive assistant; educational assistant; medical assistant; sign-based 
worldview; natural language processing; script; dialog system; planning}
  
 
  
\DOI{10.14357/19922264190315} 

%\vspace*{-14pt}

 \Ack
  \noindent
  The study was partially funded by the Russian Foundation for Basic Research (project 
No.\,18-29-22027).


%\vspace*{-6pt}

  \begin{multicols}{2}

\renewcommand{\bibname}{\protect\rmfamily References}
%\renewcommand{\bibname}{\large\protect\rm References}

{\small\frenchspacing
 {%\baselineskip=10.8pt
 \addcontentsline{toc}{section}{References}
 \begin{thebibliography}{99}
  
  \bibitem{2-sm-1}
  \Aue{Osipov, G.\,S., and A.\,I.~Panov.} 2018. Relationships and operations in agent's 
sign-based model of the world. \textit{Scientific Technical Information Processing}
45(5):1--14.
\bibitem{1-sm-1}
  \Aue{Osipov, G.\,S., A.\,I.~Panov, and N.\,V.~Chudova.} 2014. Behavior control as 
  a~function of consciousness. I.~World model and goal setting. 
  \textit{J.~Comput. Sys. Sc. Int.} 53(4):517--529.
  
  \bibitem{3-sm-1}
  \Aue{Panov, A.\,I.} 2017. Behavior planning of intelligent agent with sign world 
model. \textit{Biol. Inspir. Cogn. Arc.} 19:21--31. 
  \bibitem{4-sm-1}
  \Aue{Stankevich, M., I.~Smirnov, N.~Ignatiev, O.~Grigoriev, and N.~Kiselnikova.} 
2018. Analysis of big five personality traits by processing of social media users 
activity features. \textit{CEUR Workshop Procee.} 2277:162--166.
  \bibitem{5-sm-1}
  \Aue{Kiselev G.\,A., and A.\,I.~Panov}. 
  2018. Znakovyy podkhod k~zadache raspredeleniya 
roley v koalitsii kognitivnykh agentov [Sign-based approach to the task of role 
distribution in the coalition of cognitive agents]. \textit{Trudy SPIIRAN} [SPIIRAS 
Proceedings] 57:161--187.
  \bibitem{6-sm-1}
  \Aue{Shelmanov, A.\,O., V.\,A.~Isakov, M.\,A.~Stankevich, and I.\,V.~Smirnov.} 2018. 
Otkrytoe izvlechenie informatsii iz tekstov. Chast'~I. Postanovka zadachi i~obzor 
metodov [Open information extraction. Part~I. The task and the review of the state 
of the art]. \textit{Iskusstvennyy intellekt i~prinyatie resheniy} 
[Artificial Intelligence and  Decision Making] 2:47--67.
  \bibitem{7-sm-1}
  \Aue{Shelmanov, A.\,O., D.\,A.~Devyatkin, V.\,A.~Isakov, and I.\,V.~Smirnov.} 2019. 
Otkrytoe izvlechenie informatsii iz tekstov. Chast'~II. Izvlechenie 
semanticheskikh otnosheniy s~pomoshch'yu mashinnogo obucheniya bez uchitelya 
[Open information extraction from texts. Part~2. Extraction of semantic relations 
using unsupervised machine learning]. 
\textit{Iskusstvennyy intellekt i~prinyatie resheniy} 
[Artificial Intelligence and Decision Making] 2:39--49. 
  \bibitem{8-sm-1}
  \Aue{Mikolov, T., I.~Sutskever, K.~Chen, G.\,S.~Corrado, and J.~Dean.} 2013. 
Distributed representations of words and phrases and their compositionality. 
\textit{Adv. Neur. Inf.} 26:3111--3119.
  \bibitem{9-sm-1}
  \Aue{Shelmanov, A.\,O., M.\,I.~Kamenskaya, I.\,V.~Ananyeva, and I.\,V.~Smirnov.} 2017. 
Semantic-syntactic analysis for question answering and 
definition extraction. \textit{Scientific Technical Information Processing}
44(6):412--423.
  \bibitem{10-sm-1}
  \Aue{Osipov, G.\,S., I.\,V.~Smirnov, and I.\,A.~Tikhomirov.} 2010. 
  Relational-situational method for text search and analysis and its applications. 
  \textit{Scientific  Technical Information Processing} 6:432--437.
  \bibitem{11-sm-1}
  \Aue{Smirnov, I.\,V., A.\,O.~Shelmanov, E.\,S.~Kuznetsova, and I.\,V.~Khramoin.} 2014. 
Semantiko-sintaksicheskiy ana\-liz estestvennykh yazykov. Chast'~II. Metod 
semantiko-sintaksicheskogo analiza tekstov [Semantic-syntactic analysis of natural 
languages. Part~II. Method for semantic-syntactic analysis of texts].
\textit{Iskusstvennyy 
intellekt i~prinyatie resheniy} [Artificial Intelligence and Decision Making] 1:11--24.
  \bibitem{12-sm-1}
  \Aue{Zavjalova, O.\,S., A.\,A.~Kiselyov, G.\,S.~Osipov,
 I.\,V.~Smir\-nov, I.\,A.~Tikhomirov, and I.\,V.~Sochenkov.} 2010. 
Sistema intellektual'nogo poiska i~analiza informatsii \mbox{``EXACTUS''} na ROMIP-2010 
[The system of information search and analysis EXACTUS on ROMIP-2010]. 
\textit{Tr. Rossiyskogo seminara po otsenke metodov informatsionnogo poiska} 
[ROMIP-2010 Proceedings]. Kazan: Kazan State University. 49--69.
  \bibitem{13-sm-1}
  \Aue{Chistova, E.\,V., A.\,O.~Shelmanov, and I.\,V.~Smirnov.} 2019. 
  Primenenie 
glubokogo obucheniya k~mo\-de\-li\-ro\-va\-niyu dialoga na estestvennom yazyke [Natural 
language dialogue modelling with deep learning]. 
\textit{Proceedings of the Institute for Systems Analysis of the Russian Academy of 
Sciences} 69(1):105--115. 
  \bibitem{14-sm-1}
  \Aue{Grigoriev, O.\,G., B.\,A.~Kobrinskii, G.\,S.~Osipov, A.\,I.~Mo\-lod\-chen\-kov, and 
I.\,V.~Smirnov.} 2018. Health management system knowledge base for formation and 
support of a~preventive measures plan. \textit{Procedia Comput. Sci.} 145:238--241.
\end{thebibliography}

 }
 }

\end{multicols}

%\vspace*{-7pt}

\hfill{\small\textit{Received January 28, 2019}}

%\pagebreak

%\vspace*{-22pt}
  
  \Contr
  
  \noindent
  \textbf{Smirnov Ivan V.} (b.\ 1978)~--- Candidate of Science (PhD) in physics 
and mathematics; Head of Department, Institute of Artificial Intelligence Problems, 
Federal Research Center ``Computer Science and Control'' of the Russian Academy 
of Sciences, 9,~60-letiya Oktyabrya Prosp., Moscow 117312, Russian Federation; 
associate professor, Peoples' Friendship University of Russia (RUDN University), 
6~Miklukho-Maklaya Str., Moscow 117198, Russian Federation; \mbox{ivs@isa.ru}
  
  \vspace*{3pt}
  
  \noindent
  \textbf{Panov Aleksandr I.} (b.\ 1987)~--- Candidate of Science (PhD) in physics 
and mathematics; senior scientist, Institute of Artificial Intelligence Problems, 
Federal Research Center ``Computer Science and Control'' of the Russian Academy 
of Sciences; 9,~60-letiya Oktyabrya Prosp., Moscow 117312, Russian Federation; 
deputy head of laboratory, Moscow Institute of Physics and Technology (State 
University), 9~Institutskiy Per., Dolgoprudny, Moscow Region 141701, Russian Federation; 
\mbox{pan@isa.ru}
  
  \vspace*{3pt}
  
  \noindent
  \textbf{Skrynnik Alexey A.} (b.\ 1993)~--- junior scientist, Institute of Artificial 
Intelligence Problems, Federal Research Center ``Computer Science and Control'' of 
the Russian Academy of Sciences; 9,~60-letiya Oktyabrya Prosp., Moscow 117312, 
Russian Federation; \mbox{skrynnik@isa.ru}
  
  \vspace*{3pt}
  
  
 % \noindent
 % \textbf{Isakov Vadim A.} (b.\ 1995)~--- PhD student, Federal Research Center 
%``Computer Science and Control'' of the Russian Academy of Sciences; 9~60-letiya 
%Oktyabrya pr., Moscow 117312, Russian Federation; \mbox{visakov@isa.ru}
  
 % \vspace*{3pt}
  
  \noindent
  \textbf{Chistova Elena V.} (b.\ 1996)~--- programmer, Institute of Artificial 
Intelligence Problems, Federal Research Center ``Computer Science and Control'' of 
the Russian Academy of Sciences; 9,~60-letiya Oktyabrya Prosp., Moscow 117312, 
Russian Federation; student, Peoples' Friendship University of Russia (RUDN 
University), 6~Miklukho-Maklaya Str., Moscow 117198, Russian Federation; 
\mbox{chistova@isa.ru}
\label{end\stat}

\renewcommand{\bibname}{\protect\rm Литература}  
  