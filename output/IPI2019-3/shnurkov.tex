\def\stat{shnurkov}

\def\tit{О РЕШЕНИИ ПРОБЛЕМЫ ОПТИМАЛЬНОГО УПРАВЛЕНИЯ ЗАПАСОМ ДИСКРЕТНОГО 
ПРОДУКТА В~СТОХАСТИЧЕСКОЙ МОДЕЛИ РЕГЕНЕРАЦИИ С~НЕПРЕРЫВНО 
ПРОИСХОДЯЩИМ ПОТРЕБЛЕНИЕМ}

\def\titkol{О решении проблемы оптимального управления запасом дискретного 
продукта в~стохастической модели регенерации} % с~непрерывно  происходящим потреблением}

\def\aut{П.\,В.~Шнурков$^1$, Н.\,А.~Вахтанов$^2$}

\def\autkol{П.\,В.~Шнурков, Н.\,А.~Вахтанов}

\titel{\tit}{\aut}{\autkol}{\titkol}

\index{Шнурков П.\,В.}
\index{Вахтанов Н.\,А.}
\index{Shnurkov P.\,V.}
\index{Vakhtanov N.\,A.}


%{\renewcommand{\thefootnote}{\fnsymbol{footnote}} \footnotetext[1]
%{Работа выполнена при частичной поддержке РФФИ (проект 19-07-00187-A).}}


\renewcommand{\thefootnote}{\arabic{footnote}}
\footnotetext[1]{Национальный исследовательский университет <<Высшая школа экономики>>, 
\mbox{pshnurkov@hse.ru}}
\footnotetext[2]{Национальный исследовательский университет <<Высшая школа экономики>>, 
\mbox{Vakhtanov1997@mail.ru}}

\vspace*{-12pt}


 

\Abst{Работа представляет собой вторую и~завершающую часть исследования проб\-ле\-мы 
оптимального управ\-ле\-ния запасом дискретного продукта в~стохастической модели 
регенерации. Основным содержанием работы является вывод аналитических пред\-став\-ле\-ний 
для математического ожидания приращения функционала прибыли, полученной за период 
регенерации. При этом указанные математические ожидания находятся при всех возможных 
в~данной задаче условиях на решения, при\-ни\-ма\-емые на периоде регенерации. Полученные 
аналитические пред\-став\-ле\-ния позволяют явно выразить стационарный стоимостный показатель 
эффективности управ\-ле\-ния, который был определен в~первой час\-ти исследования. Таким 
образом, появляется воз\-мож\-ность чис\-лен\-но решить задачу оптимального управ\-ле\-ния запасом 
в~рас\-смат\-ри\-ва\-емой модели.}

\KW{управление запасом дискретного продукта; управ\-ля\-емый регенерирующий процесс; 
стационарный стоимостный показатель эффективности управ\-ления}

\DOI{10.14357/19922264190308} 
  
\vspace*{-9pt}


\vskip 10pt plus 9pt minus 6pt

\thispagestyle{headings}

\begin{multicols}{2}

\label{st\stat}


\section{Введение}

  В первой части настоящего исследования~[1] была по\-стро\-ена стохастическая 
модель управ\-ле\-ния запасом дискретного продукта в~форме регенерирующего 
процесса. Получено общее пред\-став\-ле\-ние для стационарного стоимостного 
показателя эффективности управ\-ле\-ния. Установлено, что этот показатель по 
форме пред\-став\-ля\-ет собой дроб\-но-ли\-ней\-ный интегральный функционал, 
заданный на множестве дис\-крет\-ных вероятностных распределений, каждое из 
которых определяет стратегию управ\-ле\-ния запасом или, что то же самое, 
стратегию управ\-ле\-ния введенным регенерирующим процессом. Задача 
оптимального управ\-ле\-ния по отношению к~указанному функционалу решается 
на основе тео\-ре\-мы о~безуслов\-ном экстремуме дроб\-но-ли\-ней\-но\-го 
интегрального функционала~[2, 3]. Доказано, что оптимальное управ\-ле\-ние 
является детерминированным и~определяется точ\-кой глобального экстремума 
некоторой функ\-ции от дискретного аргумента, который пред\-став\-ля\-ет собой 
параметр управ\-ле\-ния. Таким образом, проб\-ле\-ма оптимального управ\-ле\-ния 
сведена к~задаче поиска глобального экстремума этой функции.
  
  Вторая часть проведенного исследования
посвящена нахождению явного аналитического пред\-став\-ле\-ния для указанной 
целевой функции, которая формально есть не что иное, как стационарный 
стоимостный показатель эф\-фек\-тив\-ности, т.\,е.\ сред\-няя удельная прибыль, 
при детерминированном управ\-ле\-нии.
  
Чтобы найти аналитическое пред\-став\-ле\-ние средней удельной 
прибыли как функции от па\-ра\-мет\-ра управ\-ле\-ния, необходимо провести 
глубокий анализ процедуры формирования доходов и~за\-трат, связанных 
с~эволюцией ис\-сле\-ду\-емой сис\-темы.
  %
  Такая процедура существенно зависит от значения па\-ра\-мет\-ра управ\-ле\-ния. 
Проведя со\-от\-вет\-ст\-ву\-ющий анализ для всех допустимых значений па\-ра\-мет\-ра 
управ\-ле\-ния, мож\-но получить явное аналитическое пред\-став\-ле\-ние для целевой 
функции сред\-ней удельной прибыли. Именно такое пред\-став\-ле\-ние и~получено 
в~данной работе. Окончательное решение задачи оптимального управ\-ле\-ния, т.\,е.\ 
на\-хож\-де\-ние значения па\-ра\-мет\-ра управ\-ле\-ния, которое до\-став\-ля\-ет 
глобальный максимум целевой функции, может быть выполнено только 
чис\-лен\-ны\-ми методами для заданного набора исходных па\-ра\-мет\-ров 
математической модели.

\vspace*{-6pt}
  
\section{Вспомогательные теоретические результаты}

\vspace*{-2pt}

  Прежде всего, приведем обобщение известной формулы для 
математического ожидания неотрицательной случайной величины.
  
  Пусть $(\Omega, \mathcal{A}, {\sf P})$~--- исходное вероятностное пространство. 
Предположим, что на этом пространстве задана неотрицательная случайная 
величина $\eta\hm=\eta(\omega)$, $\omega\hm\in \Omega$, и~произвольное 
событие $A\hm\in \mathcal{A}$. Обозначим через $F(x)\hm= {\sf P}(\eta<x)$, $0\hm< 
x\hm<\infty$, функ\-цию распределения случайной величины~$\eta$.
  
  Известен следующий результат о~пред\-став\-ле\-нии математического ожидания 
неотрицательной случайной величины~$\eta$, при\-над\-ле\-жа\-щий 
Б.\,В.~Гне\-ден\-ко~[4]:

\noindent
  \begin{multline}
  E\eta=\int\limits_0^\infty x\,dF(x)=\int\limits_0^\infty {\sf P}(\eta\geq 
{x})\,dx={}\\
{}= \int\limits_0^\infty (1-F(x))\,dx\,.
  \label{e1-sn}
  \end{multline}
  
  Рассмотрим теперь совместное распределение $F(x;A)\hm= {\sf P}(\eta<x;A)$ 
и~соответствующее математическое ожидание
  $$
  E(\eta;A)=\int\limits_0^\infty x\,dF(x;A)\,.
  $$
  
  П.\,В.~Шнурковым доказано сле\-ду\-ющее утверж\-де\-ние о~пред\-став\-ле\-нии 
математического ожидания по совместному распределению неотрицательной 
случайной величины~$\eta$ и~события~$A$:
  \begin{equation}
  E(\eta;A)=\int\limits_0^\infty {\sf P}(\eta\geq {x};A)\,dx\,.
  \label{e2-sn}
  \end{equation}
  
  Соотношение~(\ref{e2-sn}) является обобщением формулы~(\ref{e1-sn}), 
которая непосредственно следует из~(\ref{e2-sn}) при $A\hm=\Omega$.
  
  Общая формула~(\ref{e2-sn}) будет лежать в~основе доказательства 
утверж\-де\-ния, которое в~данной работе будет называться тео\-ре\-мой~1. Это 
утверж\-де\-ние, в~свою очередь, будет использоваться для получения явных 
пред\-став\-ле\-ний для различных вспомогательных вероятностных характеристик, 
связанных с~целевым показателем управ\-ле\-ния.
  
  Перейдем к~формулировке и~доказательству указанного утверж\-де\-ния. При 
этом будем предполагать, что все вводимые сто\-ха\-сти\-че\-ские объекты 
определены на исходном вероятностном про\-стран\-ст\-ве $(\Omega, 
\mathcal{A},{\sf P})$, которое описывает проводимый случайный эксперимент.
  
  \smallskip
  
  \noindent
  \textbf{Теорема~1.}\ \textit{Рассмотрим простейший поток однородных 
событий с~па\-ра\-мет\-ром~$\lambda$. Обозначим через $\xi_1,\ldots , \xi_s$ 
интервалы меж\-ду последовательными событиями этого потока, $\zeta_s\hm= 
\sum\nolimits^s_{i=1} \xi_i$~--- момент на\-ступ\-ле\-ния $s$-го события, 
$s\hm=1,2,\ldots$}
  
  \textit{Пусть $h_r$~--- положительная случайная величина, име\-ющая 
заданную функцию распределения $F_r(z)$, которая может зависеть от 
некоторого вещественного па\-ра\-мет\-ра~$r$. Будем интерпретировать~$h_r$ как 
случайный интервал времени. Предположим, что случайная величина~$h_r$ 
при любом фиксированном~$r$ не зависит от рас\-смат\-ри\-ва\-емо\-го прос\-тей\-ше\-го 
потока и,~в~част\-ности, от случайных величин $\xi_1,\ldots ,\xi_s$. Обозначим 
через~$A_s$ случайное событие, со\-сто\-ящее в~том, что на интервале 
времени~$h_r$ произошло ровно~$s$~событий простейшего потока, а~через 
$\theta_{r,s}\hm= h_r\hm-\zeta_s$~--- случайную величину, которая пред\-став\-ля\-ет 
собой остаточную дли\-тель\-ность интервала времени~$h_r$ после наступления 
$s$-го события потока. Тогда математическое ожидание случайной 
величины~$\theta_{r,s}$, опре\-де\-ля\-емое по совместному распределению 
с~событием~$A_s$, может быть выражено сле\-ду\-ющей формулой}:

\noindent
  \begin{multline}
  \tau_{r,s}=E\left[ \theta_{r,s};A_s\right] ={}\\
  {}=\int\limits_0^\infty \fr{\lambda^s}{s!} 
\int\limits^\infty_x (z-x)^s e^{-\lambda z}\,dF_r(z)\,dx\,.
  \label{e3-sn}
  \end{multline}

\noindent
  Д\,о\,к\,а\,з\,а\,т\,е\,л\,ь\,с\,т\,в\,о\,.\ \ Заметим, что случайная величина 
$\theta_{r,s}$ определена на множестве элементарных исходов $\omega\hm\in 
\Omega$, которые соответствуют событию~$A_s$. Тогда можно 
воспользоваться формулой~(\ref{e2-sn}) для 
математического ожидания по совместному распределению случайной 
величины~$\theta_{r,s}$ и~события~$A_s$:
  \begin{equation}
  \tau_{r,s}=E\left[ \theta_{r,s};A_s\right] =\int\limits_0^\infty 
  {\sf P}\left( \theta_{r,s} > 
x, A_s\right)dx\,.
  \label{e4-sn}
  \end{equation}
  
  В то же время событие~$A_s$ можно пред\-ста\-вить сле\-ду\-ющим образом:
    $$
  A_s=\left( \zeta_s=\sum\limits^s_{i=1} \xi_i< h_r,\ \zeta_s+\xi_{s+1}>h_r\right)\,.
  $$
  
  Выразим совместное распределение случайной величины~$\theta_{r,s}$ 
и~события~$A_s$. Для любого $x\hm\geq 0$ имеем:

\vspace*{-2pt}

\noindent
  \begin{multline}
  {\sf P}\left( \theta_{r,s}>x;A_s\right) ={}\\
  {}={\sf P}\left( \theta_{r,s}>x, \zeta_s<h_r, \zeta_s+ 
\xi_{s+1}>h_r\right)={}\\
  {}={\sf P}\left( h_r-\zeta_s>x, \zeta_s<h_r, \zeta_s+\xi_{s+1}>h_r\right) ={}\\
  {}= {\sf P}\left( h_r-
\zeta_s>x, \xi_{s+1}>h_r-\zeta_s\right)\,.
  \label{e5-sn}
  \end{multline}
  
    \begin{figure*} %fig1
  \vspace*{1pt}
    \begin{center}  
  \mbox{%
 \epsfxsize=106.27mm 
 \epsfbox{shn-1.eps}
 }
\end{center}
\vspace*{-15pt}
  \Caption{Графическая иллюстрация к~утверждению тео\-ре\-мы~1}
  \end{figure*}
  
  В силу введенных предположений случайная величина~$\zeta_s$ имеет 
рас\-пре\-де\-ле\-ние Эр\-лан\-га порядка $s\hm\geq 1$ с~па\-ра\-мет\-ром~$\lambda$. Для 
этой величины известны явные пред\-став\-ле\-ния функции 
рас\-пре\-де\-ле\-ния~$G_s(y)$ и~ее полного дифференциала:

\pagebreak

\noindent
  \begin{align*}
  G_s(y)&= {\sf P}\left( \zeta_s<y\right) =1-\!\!\sum\limits_{i=0}^{s-1}\!\! \fr{(\lambda 
y)^i}{i!}\,e^{-\lambda y},\enskip s\geq 1;\\
  dG_s(y)&= \lambda \fr{(\lambda y)^{s-1}}{(s-1)!}\,e^{-\lambda y}\,dy,\   
s\geq 1.
  \end{align*}
  
  Заметим, что в~условиях доказываемой тео\-ре\-мы случайные величины $h_r$, 
$\zeta_s$ и~$\xi_{s+1}$ независимы. Зафиксируем значения случайных величин 
$h_r\hm= z$ и~$\zeta_s\hm=y$. При указанном условии для реализации события 
($h_r\hm-\zeta_s\hm> x, \xi_{s+1}\hm>h_r\hm-\zeta_s$) необходимо, чтобы 
выполнялось соотношение $z\hm-y\hm>x$, т.\,е.\ $0\hm<y\hm<z\hm-x$. В~то же 
время должно выполняться соотношение $z\hm>x$, так как иначе событие 
($h_r\hm- \zeta_s\hm>x)$ невозможно, т.\,е.\ имеет ве\-ро\-ят\-ность, рав\-ную нулю.
  
  Из указанных замечаний следует, что совместная вероятность  ${\sf P}(h_r\hm- 
\zeta_s\hm> x, \xi_{s+1}\hm>h_r\hm-\zeta_s)$ пред\-ста\-ви\-ма в~виде:
  \begin{multline}
  {\sf P}\left( h_r-\zeta_s>x, \xi_{s+1}>h_r-\zeta_s\right) ={}\\
  {}=\int\limits_x^\infty 
\int\limits_0^{z-x} e^{-\lambda(z-y)}\,dG_s(y)\,dF_r(z)={}\\
  {}=\int\limits_x^{\infty} \int\limits_0^{z-x} e^{-\lambda(z-y)}\lambda 
\fr{(\lambda y)^{s-1}}{(s-1)!}\,e^{-\lambda y} \,dy \,dF_r(z)={}\\
  {}=\int\limits_x^{\infty} \int\limits_0^{z-x} \lambda \fr{(\lambda y)^{s-1}}{(s-
1)!}\, e^{-\lambda z}\,dy\,dF_r(z)={}\\
  {}=
  \int\limits_x^\infty \fr{\lambda^s}{(s-1)!}\,e^{-\lambda z} \int\limits_0^{z-x} 
y^{s-1}\,dy\,dF_r(z)={}\\
  {}= \int\limits_x^\infty \fr{\lambda^s}{(s-1)!}\,e^{-\lambda z} \fr{(z-
x)^s}{s}\,dF_r(z)={}\\
  {}= \int\limits_x^\infty \fr{\lambda^s}{s!}\left( z-x\right)^s e^{-\lambda z} \,
dF_r(z)\,.
  \label{e6-sn}
  \end{multline}
  
  Из~(\ref{e5-sn}) и~(\ref{e6-sn}) получаем явное пред\-став\-ле\-ние для 
совместного распределения ${\sf P}(\theta_{r,s}\hm>x;A_s)$, $x\hm>0$. Остается 
применить формулу~(\ref{e4-sn}):

\noindent
  \begin{multline*}
  \tau_{r,s}=E\left[ \theta_{r,s};A_s\right] =\int\limits_0^\infty {\sf P}\left( 
\theta_{r,s}>x;A_s\right)dx={}\\
  {}=\int\limits_0^\infty {\sf P}\left( h_r-\zeta_s>x, \xi_{s+1}>h_r-\zeta_s\right)dx={}
 \\
  {}= \int\limits_0^\infty \int\limits_x^\infty 
  \fr{\lambda^s(z-x)^s}{s!}\,e^{-\lambda z}\,dF_r(z)\,dx={}\\
{}= \int\limits_0^\infty \fr{\lambda^s}{s!} \int\limits_x^\infty (z-x)^s  e^{-\lambda z}\, dF_r(z)\,dx\,.
  \end{multline*}


  Полученное соотношение совпадает с~формулой~(\ref{e3-sn}). Теорема~1 
доказана.
  
  \smallskip
  

  
  \noindent
  \textbf{Замечание~1.} В~ходе дальнейшего анализа рассматриваемой 
модели управления запасом в~роли простейшего потока событий будет 
выступать поток моментов поступления заявок на приобретение товара. 
Случайная величина~$h_r$ будет играть роль длительности за\-держ\-ки по\-став\-ки, 
распределение которой зависит от па\-ра\-мет\-ра управ\-ле\-ния~$r$. Случайная 
величина~$\theta_{r,s}$ будет пред\-став\-лять собой остаточную дли\-тель\-ность 
задержки после по\-ступ\-ле\-ния последнего на данном периоде регенерации 
заказа, имеющего фиксированный номер~$s$, считая от момента начала периода 
за\-держ\-ки. Иллюстрация реализаций со\-от\-вет\-ст\-ву\-ющих событий в~ходе эволюции 
рассматриваемой сис\-те\-мы приведена на рис.~1.
  
  \smallskip
  
  \noindent
  \textbf{Замечание~2.} Согласно принятым предположениям 
о~рас\-смат\-ри\-ва\-емой модели~\cite{1-sn}, случайная величина~$h_r$ имеет 
заданное рас\-пре\-де\-ле\-ние~$H_r(z)$. Таким образом, в~дальнейшем при 
использовании тео\-ре\-мы~1 будем полагать $F_r(z)\hm= H_r(z)$, $z\hm\geq 0$.
  
  
\section{Аналитические представления для~стоимостных 
характеристик исследуемой модели}

  Перейдем к~доказательству утверждений о~конкретных пред\-став\-ле\-ни\-ях 
вероятностных характеристик, связанных со стационарным стоимостным 
показателем эф\-фек\-тив\-ности управ\-ле\-ния. 
%
Прежде всего введем систему $\{A_s, 
s\hm=0,1,2,\ldots\}$, где $A_s\hm\in \mathcal{A}$~--- событие, состоящее в~том, 
что за период задержки в~рас\-смат\-ри\-ва\-емую сис\-те\-му по\-сту\-пи\-ло ров\-но~$s$ 
требований на продукт (клиентов). Из свойств пуассоновского потока 
и~формулы пол\-ной ве\-ро\-ят\-ности следует:
  $$
  {\sf P}\left( A_s\right) =\int\limits_0^\infty \fr{(\lambda y)^s}{s!}\,e^{-\lambda y} 
\,dH_r(y)\,,\enskip s=0,1,2,\ldots
  $$
  
  Рассмотрим случайную величину $\Delta\gamma_n$, которая пред\-став\-ля\-ет 
собой приращение аддитивного функционала прибыли на очередном периоде 
регенерации $(t_n,t_{n+1}]$ (см.~[1, разд.~4]). Поскольку сис\-те\-ма 
$\{A_s, s\hm=0,1,2,\ldots\}$ образует пол\-ную группу несовместных событий, то 
по свойству математического ожидания можно записать:
  $$
  E\left( \Delta \gamma_n\right) =\sum\limits_{s=0}^\infty E\left( \Delta\gamma_n; 
A_s\right)\,.
  $$
  
  Теперь рассмотрим соответствующие математические ожидания при 
условии, что параметр управ\-ле\-ния на данном периоде регенерации принимает 
фиксированное значение~$r$. Обозначим эти математические ожидания 
$E_r(\Delta\gamma_n)$ и~$E_r(\Delta\gamma_n;A_s)$. Используя упомянутое 
свойство математического ожидания, получим сле\-ду\-ющее исходное 
соотношение:
  \begin{equation}
  E_r\left(\Delta\gamma_n\right) =\sum\limits^\infty_{s=0} 
E_r\left(\Delta\gamma_n;A_s\right)\,.
  \label{e7-sn}
  \end{equation}
  
  Введем дополнительно две случайные величины, связанные с~периодом 
регенерации $(t_n, t_{n+1}]$: $\Delta\gamma_n^{(+)}$~--- приращение дохода 
(кратко~--- доход) и~$\Delta\gamma_n^{(-)}$~--- приращение за\-трат (кратко~--- 
за\-тра\-ты). Тогда приращение прибыли на периоде пред\-став\-ля\-ет\-ся в~виде 
$\Delta\gamma_n\hm= \Delta\gamma_n^{(+)}\hm- \Delta\gamma_n^{(-)}$.
  
  Введем также обозначения $E_r(\Delta\gamma_n^{(+)}; A_s)$ 
и~$E_r(\Delta\gamma_n^{(-)}; A_s)$ для условных математических ожиданий 
дохода и~за\-трат на периоде регенерации $(t_n,t_{n+1}]$, опре\-де\-ля\-емых по 
совместному рас\-пре\-де\-ле\-нию с~событием~$A_s$, при условии, что на 
указанном периоде принято решение~$r$. В~силу свойства ад\-ди\-тив\-ности 
величины~$\Delta\gamma_n$ имеет место соотношение:
  $$
  E_r\left(\Delta\gamma_n;A_s\right) =E_r\left( \Delta\gamma_n^{(+)}; A_s\right) 
-E_r\left( \Delta\gamma_n^{(-)};A_s\right)\,.
  $$
  
  Условные математические ожидания $E_r(\Delta\gamma_n)$ необходимы для 
получения явного аналитического вида показателя эффективности 
управ\-ле\-ния~$I_r$. 
  
  Следует отметить, что структура формирования случайных 
величин~$\Delta\gamma_n^{(+)}$ и~$\Delta\gamma_n^{(-)}$ существенно зависит 
от значения па\-ра\-мет\-ра управ\-ле\-ния~$r$ и~чис\-ла требований, поступивших за 
период за\-держ\-ки поставки, т.\,е.\ от величины~$s$. В~связи с~этим разобь\-ем 
множество возможных значений па\-ра\-мет\-ра~$r$ на сле\-ду\-ющие 
непересекающиеся подмножества и~определим на каж\-дом из них явное 
пред\-став\-ле\-ние для $E_r(\Delta\gamma_n)$ и~$I_r$:
  \begin{itemize}
  \item [$\bullet$] $r=N$;
  \item [$\bullet$] $1\leq r<N$;
  \item [$\bullet$] $r=0$;
  \item [$\bullet$] $-N_0<r\leq -1$;
  \item [$\bullet$] $r=-N_0$.
  \end{itemize}
  
  В дальнейшем в~каждой из указанных областей значений па\-ра\-мет\-ра~$r$ 
необходимо найти явные пред\-став\-ле\-ния для математических ожиданий 
$E_r(\Delta\gamma_n;A_s)$ при всех значениях $s\hm=0,1,2,\ldots$ После этого 
можно воспользоваться формулой~(\ref{e7-sn}).
  
  Ограничения по объему данной статьи не позволяют привести явные 
формулы для всех областей изменения па\-ра\-мет\-ра управ\-ле\-ния. Пол\-ностью эти 
формулы приведены и~доказаны в~приложении к~на\-сто\-ящей работе~\cite{5-sn}. 
В~рамках на\-сто\-ящей \mbox{статьи} ограничимся только некоторыми фрагментами 
вероятностного анализа, да\-ющи\-ми пред\-став\-ле\-ние о~характере данного этапа 
исследования.
  
  Рассмотрим следующий вариант: $1\hm\leq r\hm<N$, $1\hm\leq s\hm<r$ (рис.~2). 
Заметим, что если па\-ра\-мет\-ры~$r$ и~$s$ принимают значения в~указанных областях, 
то происходят сле\-ду\-ющие события: заказ на очередную пар\-тию поставки 
осуществляется после прихода некоторого чис\-ла требований, но до полного 
опус\-то\-ше\-ния склада, а~чис\-ло требований, поступивших в~период задержки 
по\-став\-ки, меньше оставшегося на складе в~момент заказа чис\-ла единиц 
продукта.



  
  В первую очередь выпишем математическое ожидание дохода за период 
регенерации по со-\linebreak\vspace*{-12pt}

{ \begin{center}  %fig2
 \vspace*{9pt}
   \mbox{%
 \epsfxsize=50.08mm 
 \epsfbox{shn-2.eps}
 }


\end{center}

\vspace*{-1pt}


\noindent
{{\figurename~2}\ \ \small{Графическая иллюстрация эволюции процесса для варианта значений 
па\-ра\-мет\-ров $1\hm\leq r\hm<N$ и~$1\hm\leq s\hm<r$}}

}

%\vspace*{9pt}

\noindent
вместному распределению с~событием~$A_s$. Получим:
  \begin{multline}
  E_r\left(\Delta \gamma_n^{(+)};A_s\right) =E_r\left( \Delta 
\gamma_n^{(+)}\vert A_s\right) {\sf P}\left( A_s\right) ={}\\
  {}=c_0 \left( N-r+s\right) \int\limits_0^\infty \fr{(\lambda y)^s}{s!}\,e^{-\lambda 
y} dH_r(y)\,.
  \label{e8-sn}
  \end{multline}
  
  Теперь необходимо проанализировать аналогичное математическое 
ожидание для расходов на периоде регенерации. Расходы в~рас\-смат\-ри\-ва\-емом 
случае обуслов\-ле\-ны хранением имеющегося товара и~приобретением 
дополнительного товара для пополнения запаса. Приведем общее 
пред\-став\-ле\-ние для этой стоимостной характеристики, добавив некоторые 
по\-яс\-не\-ния к~отдельным компонентам найденной формулы:

\noindent
%{\substack{{i=\overline{1,n}}\\ {j=\overline{1,l}}}}
  \begin{multline}
  E_r\left( \Delta\gamma_n^{(-)};A_s\right) =E_r\left( \Delta\gamma_n^{(-)}\vert 
A_s\right) {\sf P}\left(A_s\right) ={}\\
  {}=\! \Bigg[ 
  \underbrace{\sum\limits^N_{i=r-s+1} \!
\fr{c_1i}{\lambda}}_{\substack{{\mbox{{\tiny затраты}}}\\ 
{\mbox{{\tiny на\ хранение}}}\\ {\mbox{{\tiny до\ поступления}}}\\ 
{\mbox{{\tiny последнего}}}\\ 
{\mbox{{\tiny требования}}}}} + 
\underbrace{E_{1^*}\!\left(\Delta\gamma_n^{(-)}\vert 
A_s\!\right)}_{\substack{{\mbox{{\tiny затраты}}}\\ 
{\mbox{{\tiny на хранение}}}\\ 
{\mbox{{\tiny после поступления}}}\\ 
{\mbox{{\tiny последнего}}}\\  
{\mbox{{\tiny требования}}}}}
  +
  \underbrace{c_2(N-r-s)}_{\substack{
  {\mbox{{\tiny затраты}}}\\
{\mbox{{\tiny на\ пополнение}}}\\
{\mbox{{\tiny  запаса}}}
}} 
\Bigg] \!\times\\
{}\times
{\sf P}\left(A_s\right)\,.
  \label{e9-sn}
  \end{multline} 
  
  Заметим, что
  \begin{equation}
  \sum\limits^N_{i=r-s+1} \fr{c_1i}{\lambda} = \fr{c_1(N-r+s) (N + r-
s+1)}{2\lambda} \,.
  \label{e10-sn}
  \end{equation}
  
  Явное представление для математического ожидания 
$E_{1^*}(\Delta\gamma_n^{(-)};A_s)$ определяется на основе тео\-ре\-мы~1 
(формула~(\ref{e3-sn})):
  \begin{multline}
  E_{1^*}\left( \Delta\gamma_n^{(-)};A_s\right) =
  c_1(r-s)\tau_{r,s}={}\\
  {}=c_1(r-s) 
\int\limits_0^\infty \fr{\lambda^s}{s!} \int\limits_x^\infty (z-x)^s e^{-\lambda 
z}dH_r(z)\,dx\,.
  \label{e11-sn}
  \end{multline}
  
  Подставляя~(\ref{e10-sn}) и~(\ref{e11-sn}) в~соотношение~(\ref{e9-sn}), 
получаем явную формулу для математического ожидания расходов по 
совместному распределению с~событием~$A_s$:

\noindent
  \begin{multline}
  E_r\left(\Delta\gamma_n^{(-)};A_s\right) ={}\\
  {}=\left[ 
  \fr{c_1(N-r+s)(N+r-s+1)} {2\lambda} +{}\right.
\\
\left.  {}+c_2(N-r+s)
\vphantom{  \fr{c_1(N-r+s)(N+r-s+1)} {2\lambda}}
\right] \int\limits_0^\infty 
\fr{(\lambda y)^s}{s!}\,e^{-\lambda y} \,d H_r(y)+{}\\
  {}+ c_1(r-s) \int\limits_0^\infty \fr{\lambda^s}{s!} \int\limits_x^{\infty} (z-x)^s 
e^{-\lambda z} dH_r(z)\,dx\,.
  \label{e12-sn}
  \end{multline}
  
  Теперь воспользуемся формулами~(\ref{e8-sn}) и~(\ref{e12-sn}) и~получим 
окончательное выражение для математического ожидания приращения 
функционала прибыли при фиксированных значениях па\-ра\-мет\-ров~$r$ и~$s$, 
удовле\-тво\-ря\-ющих условиям $1\hm\leq r\hm<N$, $1\hm\leq s\hm<r$:
  \begin{multline}
  \hspace*{-6.91103pt}E_r\left( \Delta\gamma_n;A_s\right) = E_r \left( 
\Delta\gamma_n^{(+)};A_s\right)- E_r\left(\Delta 
\gamma_n^{(-)};A_s\right)={}\\
  {}=c_0(N-r+s)\int\limits_0^\infty \fr{(\lambda y)^s}{s!}\,e^{-\lambda y}\, dH_r(y)-
{}\\
  {}-
  \left[ \fr{c_1(N-r+s)(N+r-s+1)}{2\lambda}+ {}\right.\\
\left.  {}+c_2(N-r+s)
\vphantom{\fr{c_1(N-r+s)(N+r-s+1)}{2\lambda}}
\right] \int\limits_0^\infty 
\fr{(\lambda y)^s}{s!}\,e^{-\lambda y}\, dH_r(y)
  -{}\\
  {}- c_1(r-s) \int\limits_0^\infty \fr{\lambda^s}{s!} \int\limits_x^\infty (z-x)^s e^{-
\lambda z} \,dH_r(z)\,dx={}\\
  {}=
  (N-r+s) \left[ c_0 -\fr{c_1(N+r-s+1)}{2\lambda} -{}\right.\\
\left.  {}-c_2
\vphantom{\fr{c_1(N+r-s+1)}{2\lambda}}
\right]  \int\limits_0^\infty 
  \fr{(\lambda y)^s}{s!}\,e^{-\lambda y} \,dH_r(y) -{}\\
{}-c_1(r-s)
  \int\limits_0^\infty \fr{\lambda^s}{s!}\int\limits_x^\infty (z-x)^s e^{-\lambda z} 
\,dH_r(z)\,dx\,.
\label{e13-sn}
  \end{multline}
  
  Напомним~\cite{1-sn}, что показатель эффективности управ\-ле\-ния~$I_r$ 
представляет собой отношение математического ожидания приращения 
аддитивного функционала прибыли, полученной за период регенерации, 
к~математическому ожиданию дли\-тель\-ности периода регенерации. Оба 
указанных математических ожидания определяются при условии, что параметр 
управ\-ле\-ния принимает фиксированное значение~$r$.  В~принятых 
обозначениях
  \begin{equation}
  I_r=\fr{E_r\left(\Delta\gamma_n\right)}{E_r\left( \Delta t_n\right)}\,.
  \label{e14-sn}
  \end{equation}
%  
  При этом в~соответствии с~[1, соотношение~(1)] условное 
математическое ожидание~$E_r(\Delta t_n)$ определяется формулой:

\noindent
  $$
  E_r\left(\Delta t_n\right) =\fr{N-r}{\lambda} +\tau_r =\fr{N-r}{\lambda} + 
\int\limits_0^\infty \left( 1-H_r(y)\right)\,dy\,.
  $$
  
  В заключение данного раздела приведем также итоговое пред\-став\-ле\-ние для 
показателя эффективности~$I_r$ для варианта, когда параметр управ\-ле\-ния~$r$ 
принимает значение в~об\-ласти $1\hm\leq r\hm<N$. Заметим дополнительно, что 
выражение для $E_r(\Delta \gamma_n)$ в~формуле~(\ref{e14-sn}) 
определяется на основе соотношения~(\ref{e7-sn}) с~учетом, в~частности, 
формулы~(\ref{e13-sn}):

\vspace*{-4pt}

\noindent
  \begin{multline*}
  I_r={}\\[-6pt]
  {}=\fr{1}{(N-r)/\lambda  +\int\nolimits_0^\infty (1-H_r(y))\,dy}\! \left[ \left(
  \vphantom{\int\limits_0^\infty} 
c_0(N-r) -{}\right.\right.\\
{}-\fr{c_1(N+r+1)(N-r)}{2\lambda} -c_1 r\int\limits_0^\infty \left(1-
H_r(y)\right)dy-{}\\
\left.{}-c_2(N-r)
  \vphantom{\int\limits_0^\infty}
\right) \int\limits_0^\infty e^{-\lambda y}\,dH_r(y)+{}\\
{}+\sum\limits_{i=1}^{r-1} \left(
  \vphantom{\int\limits_0^\infty} \left(
\vphantom{\fr{c_1(N-r+i)(N+r-i+1)}{2\lambda}}
 c_0(N-r+i) -{}\right.\right.\\
 {}-\fr{c_1(N-r+i)(N+r-i+1)}{2\lambda} -{}\\
\left. {}- c_2(N-r+i)
\vphantom{\fr{c_1(N-r)}{2\lambda}}
\right)
 \int\limits_0^\infty \fr{(\lambda y)^i}{i!}\,e^{-\lambda y} \,dH_r(y) -{}\\
 \left.  {}-c_1 (r-i) \int\limits_0^\infty 
  \fr{\lambda^i}{i!}
 \int\limits_x^\infty (z-x)^i e^{-\lambda z}\, dH_r(z)\,dx\right) +{}\\
 {}+
  \left( 
  \vphantom{\fr{c_1(N+1)N}{2\lambda}}
  c_0N-\fr{c_1(N+1)N}{2\lambda} -{}\right.\\
 \left. {}-c_2N
\vphantom{\fr{c_1(N+1)N}{2\lambda}}
\right) \int\limits_0^\infty 
\fr{(\lambda y)^r}{r!}\,e^{-\lambda y} \,dH_r(y)
  +{}\\
  {}+\sum\limits_{i=r+1}^{r+N_0-1} \left( 
  \vphantom{\int\limits_0^\infty}\left( 
  \vphantom{\fr{c_3(i-r)}{2\lambda}}
  c_0(N-r+i) -\right.\right.\\
  {}-
\fr{c_1(N+1)N}{2\lambda} -
c_2(N-r+i)-{}\\
\left.{}-\fr{c_3(i-r)(i-r-1)}{2\lambda} \right) 
\int\limits_0^\infty \fr{(\lambda y)^i}{i!}\,e^{-\lambda y} dH_r(y)-{}\\
\left.{}-
  c_3(i-r)\int\limits_0^\infty \fr{\lambda^i}{i!}
   \int\limits_x^\infty (z-x)^i  
e^{-\lambda z} dH_r(z)\,dx\right) +{}\\
{}+
  \left( 
  \vphantom{\fr{c_1(N+1)N}{2\lambda}}
  c_0(N+N_0)-
  \fr{c_1(N+1)N}{2\lambda} -c_2(N+N_0) -{}\right.\\
\left.  {}-
  \fr{c_3N_0(N_0-1)}{2\lambda}  \right)
 \int\limits_0^\infty \fr{(\lambda 
y)^{r+N_0}}{(r+N_0)!}\,e^{-\lambda y} dH_r(y) -{}\\
{}-c_3 N_0
  \int\limits_0^\infty \fr{\lambda^{r+N_0}}{(r+N_0)!} 
\int\limits_x^\infty (z-x)^{r+N_0}e^{-\lambda z} dH_r(z)\,dx+{}
\end{multline*}

\noindent
\begin{multline}
{}+
  \sum\limits^\infty_{i=r+N_0+1} \left(
  \vphantom{\int\limits_0^\infty}
  \left(
  c_0(N+N_0) -\fr{c_1(N+1)N}{2\lambda} -{}\right.\right.\\
{}-c_2(N+N_0) -\fr{c_3 N_0 (2i- 2r- 
N_0-1)}{2\lambda} -{}\\
\left.{}- c_4^{(i-r-N_0)}
   \vphantom{\fr{c_1(N+1)N}{2\lambda}}
   \right) \int\limits_0^\infty \fr{(\lambda 
y)^i}{i!} \, e^{-\lambda y} dH_r(y)-{}\\
   \left.\left.{}-c_3 N_0\! \int\limits_0^\infty\! \fr{\lambda^i}{i!} 
\int\limits_x^\infty \!(z-x)^i e^{-\lambda z} dH_r(z)\,dx\right)\right].
  \label{e15-sn}
  \end{multline}
  
  \vspace*{-12pt}

\section{Решение проблемы оптимального управления}

  В настоящем исследовании получены явные аналитические пред\-став\-ле\-ния 
показателя эффективности управ\-ле\-ния~$I_r$, аналогичные  
формуле~(\ref{e15-sn}), для всех значений па\-ра\-мет\-ра оптимизации $r\hm\in 
U\hm= \{ N, N-1, \ldots , 0, -1, \ldots , -N_0\}$. Соответст\-ву\-ющие пред\-став\-ле\-ния 
пол\-ностью приведены  
в~приложении~\cite{5-sn}. Как было уста\-нов\-ле\-но в~[1,  
формулы~(12), (13)], показатель~$I_r$ совпадает 
с~основной функцией  
дроб\-но-ли\-ней\-но\-го дискретного интегрального функционала~$I_\alpha$:
  \begin{multline}
  I_r=C(r)=\fr{A(r)}{B(r)}\,,\\ 
  r\in U=\left\{ N, N-1, \ldots , 0, -1, \ldots , -
N_0\right\}\,.
  \label{e16-sn}
  \end{multline}
  
  В соответствии с~общим подходом к~решению задачи оптимального 
управ\-ле\-ния в~рас\-смат\-ри\-ва\-емой модели регенерации (см.~[1,  разд.~5]) 
оптимальное управление является детерминированным и~определяется точ\-кой 
глобального экстремума (максимума) функции~$I_r$, которая аналитически 
задается равенством~(\ref{e16-sn}). Поскольку множество допустимых 
управ\-ле\-ний~$U$ конечно, глобальный максимум основной функции 
достигается в~некоторой точке $r^*\hm\in U$. Следовательно, оптимальное 
управление существует и~совпадает с~точ\-кой глобального максимума~$r^*$. 
Задача оптимального управ\-ле\-ния в~рас\-смат\-ри\-ва\-емой стохастической модели 
сводится к~на\-хож\-де\-нию точ\-ки~$r^*$, которое можно осуществить только 
чис\-лен\-ным методом.

  \vspace*{-6pt}

\section{Заключение}

  Подведем итоги проведенного исследования в~целом. В~работе~\cite{1-sn} 
была по\-стро\-ена стохастическая модель управ\-ле\-ния дискретным запасом в~виде 
регенерирующего случайного процесса. Определен общий подход к~решению 
этой задачи на основе тео\-ре\-мы о~без\-услов\-ном экстремуме  
дроб\-но-ли\-ней\-но\-го интегрального функционала, зависящего от 
дискретного вероятностного распределения, которое характеризует стратегию 
управ\-ле\-ния. В~на\-сто\-ящей работе получены явные аналитические 
пред\-став\-ле\-ния для основной функции дроб\-но-ли\-ней\-но\-го интегрального 
функционала, который играет роль показателя эффективности управ\-ле\-ния. 
В~соответствии с~утверж\-де\-ни\-ем указанной выше тео\-ре\-мы решение исходной 
задачи управ\-ле\-ния существует и~определяется точ\-кой глобального максимума 
заданной основной функции.

{\small\frenchspacing
 {%\baselineskip=10.8pt
 \addcontentsline{toc}{section}{References}
 \begin{thebibliography}{9}
\bibitem{1-sn}
\Au{Шнурков П.\,В., Вахтанов Н.\,A.} Исследование проб\-ле\-мы оптимального 
управ\-ле\-ния запасом дискретного продукта в~стохастической модели 
регенерации с~непрерывно происходящим по\-треб\-ле\-ни\-ем и~случайной 
задержкой по\-став\-ки~// Информатика и~её применения, 2019. Т.~13. Вып.~2. 
С.~54--61.
\bibitem{2-sn}
\Au{Шнурков П.\,В.} О~решении задачи безусловного экстремума для  
дроб\-но-ли\-ней\-но\-го интегрального функционала на множестве вероятностных 
мер~// Докл. Акад. наук. Сер. Математика, 2016. Т.~470. №\,4. С.~387--392.
\bibitem{3-sn}
\Au{Шнурков П.\,В., Горшенин А.\,К., Белоусов~В.\,В.} Аналитическое решение 
задачи оптимального управ\-ле\-ния полумарковским процессом с~конечным 
множеством со\-сто\-яний~// Информатика и~её применения, 2016. Т.~10. Вып.~4. 
С.~72--88. 
\bibitem{4-sn}
\Au{Гнеденко Б.\,В.} Курс теории вероятностей.~--- М.: Либ\-ро\-ком, 2011. 488~с.
\bibitem{5-sn}
\Au{Шнурков П.\,В., Вахтанов~Н.\,А.} Приложение\linebreak к~статьям <<Исследование 
проб\-ле\-мы оптимального управ\-ле\-ния запасом дис\-крет\-но\-го продукта 
в~стохастической модели регенерации с~непрерывно происходящим 
по\-треб\-ле\-ни\-ем и~случайной задержкой по\-став\-ки>> и~<<Об оптимальном 
решении проб\-ле\-мы оптимального управ\-ле\-ния запасом дискретного продукта 
в~стохастической модели регенерации с~непрерывно происходящим 
по\-треб\-ле\-ни\-ем>>, 2019. 69~с. {\sf http://www.ipiran.ru/publications/Приложение.pdf}.
 \end{thebibliography}

 }
 }

\end{multicols}

\vspace*{-6pt}

\hfill{\small\textit{Поступила в~редакцию 01.07.19}}

\vspace*{6pt}

%\pagebreak

%\newpage

%\vspace*{-28pt}

\hrule

\vspace*{2pt}

\hrule

\vspace*{-1pt}

\def\tit{ON THE SOLUTION OF~THE~OPTIMAL CONTROL 
PROBLEM OF~INVENTORY OF~A~DISCRETE PRODUCT 
IN~THE~STOCHASTIC MODEL OF~REGENERATION 
WITH~CONTINUOUSLY OCCURING CONSUMPTION}


\def\titkol{On the solution of~the~optimal control 
problem of~inventory of~a~discrete product 
in~the~stochastic model of~regeneration}
%with~continuously occuring consumption}

\def\aut{P.\,V.~Shnurkov and N.\,A.~Vakhtanov}

\def\autkol{P.\,V.~Shnurkov and N.\,A.~Vakhtanov}

\titel{\tit}{\aut}{\autkol}{\titkol}

\vspace*{-11pt}


\noindent
National Research University Higher School of Economics, 34~Tallinskaya Str., 
Moscow 123458, Russian Federation


\def\leftfootline{\small{\textbf{\thepage}
\hfill INFORMATIKA I EE PRIMENENIYA~--- INFORMATICS AND
APPLICATIONS\ \ \ 2019\ \ \ volume~13\ \ \ issue\ 3}
}%
 \def\rightfootline{\small{INFORMATIKA I EE PRIMENENIYA~---
INFORMATICS AND APPLICATIONS\ \ \ 2019\ \ \ volume~13\ \ \ issue\ 3
\hfill \textbf{\thepage}}}

\vspace*{3pt} 



\Abste{The article is the second and final part of the research of the optimal control problem of 
inventory of a~discrete product in a stochastic regeneration model. The main content of the work is the 
derivation of analytical representations for the mathematical expectation of the increment of the 
functional of profit obtained during the regeneration period. At the same time, these mathematical 
expectations are determined under different conditions for decisions made during the regeneration 
period. The obtained analytical representations enable one to explicitly determine the stationary cost 
indicator of control efficiency, which was introduced in the first part of the research. Thus, it becomes 
possible to numerically solve the optimal control problem of inventory in the model under 
consideration.}

\KWE{inventory management of a discrete product; controlled regenerative process; stationary cost 
indicator of control efficiency}

\DOI{10.14357/19922264190308} 

%\vspace*{-14pt}

%\Ack
%\noindent


%\vspace*{-6pt}

  \begin{multicols}{2}

\renewcommand{\bibname}{\protect\rmfamily References}
%\renewcommand{\bibname}{\large\protect\rm References}

{\small\frenchspacing
 {%\baselineskip=10.8pt
 \addcontentsline{toc}{section}{References}
 \begin{thebibliography}{9}
 
 \vspace*{-2pt}
 
\bibitem{1-sn-1}
\Aue{Shnurkov, P.\,V., and N.\,A.~Vakhtanov.} 2019. Issledovanie problemy 
optimal'nogo upravleniya zapasom diskretnogo produkta v~stokhasticheskoy 
modeli regeneratsii s~nepreryvno proiskhodyashchim potrebleniem i~sluchaynoy 
za\-derzh\-koy postavki [Research of the optimal control problem of inventory of 
a~discrete product in stochastic regeneration model with continuously occurring 
consumption and random delivery delay]. \textit{Informatika i~ee Primeneniya~--- 
Inform. Appl.} 13(2):54--61.
\bibitem{2-sn-1}
\Aue{Shnurkov, P.\,V.} 2016. Solution of the unconditional extremum problem for 
a~linear-fractional integral functional on a set of probability measures. \textit{Dokl. 
Math.} 94(2):550--554.
\bibitem{3-sn-1}
\Aue{Shnurkov, P.\,V., A.\,K.~Gorshenin, and V.\,V.~Belousov.} 2016. 
Analiticheskoe reshenie zadachi optimal'nogo upravleniya polumarkovskim 
protsessom s~konechnym mnozhestvom sostoyaniy [Analytical solution of the 
optimal control task of a semi-Markov process with finite set of states]. 
\textit{Informatika i~ee Primeneniya~--- Inform. Appl.} 10(4):72--88.
\bibitem{4-sn-1}
\Aue{Gnedenko, B.\,V.} 2011. \textit{Kurs teorii veroyatnostey} [A~course in the 
theory of probability]. Moscow:  Librokom. 488~p.
\bibitem{5-sn-1}
\Aue{Shnurkov, P.\,V., and N.\,A.~Vakhtanov.} 2019. Prilozhenie k~stat'yam 
``Issledovanie problemy optimal'nogo upravleniya zapasom diskretnogo produkta 
v~sto\-kha\-sti\-che\-skoy mo\-de\-li regeneratsii s~nepreryvno proiskhodyashchim 
potrebleniem i~sluchaynoy zaderzhkoy postavki'' i~``Ob optimal'nom reshenii 
problemy optimal'nogo upravleniya zapasom diskretnogo produkta 
v~stokhasticheskoy mo\-de\-li regeneratsii s~nepreryvno proiskhodyashchim 
po\-treb\-le\-ni\-em'' [Appendix to articles ``Research of the optimal control problem of 
inventory of a discrete product in stochastic regeneration model with 
continuously occurring consumption and random delivery delay'' and ``On the 
optimal solution of the optimal control problem of inventory of a~discrete product 
in stochastic regeneration model with continuously occurring consumption'']. 
69~p. Available at: {\sf http://www.ipiran.ru/publications/Приложение.pdf} 
(accessed May~6, 2019). 
\end{thebibliography}

 }
 }

\end{multicols}

%\vspace*{-7pt}

\hfill{\small\textit{Received July 1, 2019}}

%\pagebreak

%\vspace*{-22pt}
     

\Contr

\noindent
\textbf{Shnurkov Peter V.} (b.\ 1953)~--- Candidate of Science (PhD) in physics and 
mathematics, associate professor, National Research University Higher School of Economics, 
34~Tallinskaya Str., Moscow 123458, Russian Federation; \mbox{pshnurkov@hse.ru}

\vspace*{3pt}

\noindent
\textbf{Vakhtanov Nikita A.} (b.\ 1997)~--- Master student, National Research 
University Higher School of Economics, 34~Tallinskaya Str., Moscow 123458, 
Russian Federation; \mbox{Vakhtanov1997@mail.ru}
\label{end\stat}

\renewcommand{\bibname}{\protect\rm Литература}  
      