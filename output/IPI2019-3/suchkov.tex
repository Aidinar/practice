\def\stat{suchkov}

\def\tit{НАУЧНЫЙ РЕЗУЛЬТАТ КАК~ИНФОРМАЦИОННЫЙ ОБЪЕКТ В~КОНТЕКСТЕ 
СИСТЕМЫ УПРАВЛЕНИЯ НАУЧНЫМИ СЕРВИСАМИ$^*$}

\def\titkol{Научный результат как информационный объект в~контексте 
системы управления научными сервисами}

\def\aut{А.\,П.~Сучков$^1$}

\def\autkol{А.\,П.~Сучков}

\titel{\tit}{\aut}{\autkol}{\titkol}

\index{Сучков А.\,П.}
\index{Suchkov A.\,P.}


{\renewcommand{\thefootnote}{\fnsymbol{footnote}} \footnotetext[1]
{Работа выполнена при частичной поддержке РФФИ (проект 18-29-03091).}}


\renewcommand{\thefootnote}{\arabic{footnote}}
\footnotetext[1]{Институт проблем информатики Федерального исследовательского центра <<Информатика 
и~управ\-ле\-ние>> Российской академии наук, \mbox{ASuchkov@ipiran.ru}}

%\vspace*{-2pt}

  

      \Abst{Обсуждается проблема формализации одного их важнейших понятий для 
системы научных сервисов~--- научный результат, который является квинтэссенцией всех 
научных исследований и~основным информационным объектом для процессов постановки 
научной проблемы, формулировки научных гипотез, мониторинга  
на\-уч\-но-тех\-ни\-че\-ской информации. Формализованное понятие <<научный 
результат>>, интегрирующее всю информационную структуру научного исследования, 
используется при организации значительного числа научных сервисов: сервисов 
извлечения фактов и~знаний (извлечение фактов, понятий, связей и~формализация 
фактографических данных на основе лингвистического анализа слабоструктурированной 
информации); интеллектуального поиска информации; тематического индексирования; 
наукометрического анализа; анализа фронта исследований; сервисов коммуникаций 
(сервисов работы с~научно-технической информацией
(НТИ) и~коммуникаций в~научном сообществе; интеллектуального 
анализа специализированных социальных сетей и~других средств научных 
коммуникаций); проверки заимствований. Особенно это актуально для 
междисциплинарных исследований, где обеспечивается поиск и~подбор релевантных 
научных положений и~инструментария в~смежных областях научного знания.}
      
\KW{научный результат; информационная модель; научные сервисы; алгоритмы 
идентификации; интеллектуальный поиск; междисциплинарные исследования}

\DOI{10.14357/19922264190319} 
  
%\vspace*{1pt}


\vskip 10pt plus 9pt minus 6pt

\thispagestyle{headings}

\begin{multicols}{2}

\label{st\stat}


\section{Введение}
     
     В настоящий момент развитие народного хозяйства все больше 
связывается с~внедрением подходов и~технологий цифрового развития. 
Российская наука обладает всеми необходимыми компонентами для 
рассмотрения ее в~качестве отрасли цифровой экономики: наличие развитой 
вычислительной\linebreak и~телекоммуникационной инфраструктуры, системы 
центров коллективного пользования, других\linebreak организа\-ционных структур, 
нормативной базы, высокого уровня компетенций 
и~высококвалифицированных научных коллективов. При этом все 
перечисленные компоненты объективно продвигаются по пути 
<<цифровизации>>, и~наука как отрасль экономики также становится 
<<цифровой>>.
     
     Другим не менее важным обстоятельством является объективная смена 
подходов к~проведению научных исследований, а именно: переход к~новой 
парадигме в~научных исследованиях, основанной на анализе накопленных 
данных в~конкретных предметных областях, естественно, в~формализованном 
цифровом виде. Проведение таких исследований становится неотъемлемой 
частью различных областей науки, экономики, бизнеса на основе 
инструментария интенсивного использования данных. 
     
     Научные и~образовательные организации России обладают широкой, 
распределенной по территории страны сетью центров коллективного 
пользования (ЦКП) и~уникальных научных установок (УНУ), обладающих колоссальным 
спектром научных услуг в~различных областях науки~[1]. Необходима 
систематизация этих услуг и~повышение эффективности их использования на 
основе создания современной исследовательской инфраструктуры, которая 
предо\-став\-ля\-ла бы широкий спектр возможностей по научным сервисам не 
только для научных организаций, но и~для внешних пользователей. 
     
     В ФИЦ ИУ РАН на основе платформенного подхода создается 
прототип подобной инфраструктуры, который представляет собой 
совокупность трех компонентов. Первый~--- центр компетенций, в~котором 
концентрируются знания в~конкретной области. Второй~---  
ма\-те\-ри\-аль\-но-тех\-ни\-че\-ская среда в~виде высокопроизводительного 
центра обработки данных с~гибридной архитектурой. Наконец, третий~--- это 
совокупность научных сервисов, которые создаются и~накапливаются на этой 
платформе научными подразделениями ФИЦ ИУ РАН~[2].
     
     Успешное продвижение цифровизации научных исследований во 
многом обусловлено воз\-мож\-ностью создания эффективной системы 
широкодоступных \textit{научных сервисов}. 

Научный сервис~--- 
совокупность действий (процессов) и~средств обеспечения процессов 
(ресурсов) по обслуживанию выполнения конкретных\linebreak работ и~реализации 
проектов на\-уч\-но-ис\-сле\-до\-ва\-тель\-ского и~прикладного характера путем 
сервисной деятельности и~предоставления потребителю (исследователям, 
специалистам или организациям) оборудования, расходных материалов,  
ин\-фор\-ма\-ци\-он\-но-ком\-му\-ни\-ка\-ци\-он\-ных ресурсов, 
обес\-пе\-чи\-ва\-ющих ресурсов, продуктов интеллектуальной научной 
деятельности и~об\-слу\-жи\-ва\-ющих человеческих ресурсов (субъекты сервисной 
деятельности). Результатом сервисной дея\-тель\-ности является услуга~[3]. 
Очевидно, что научный сервис опирается как на цифровые технологии 
(автоматическое и~автоматизированное предоставление услуг), так и~на 
использование интеллектуальных и~обслуживающих человеческих ресурсов. 
Система научных сервисов должна обеспечивать поддержку процессов 
автоматизированного подбора релевантных сервисов и~осуществления 
различных формальных и~неформальных коммуникаций исследователя 
и~государства, научного сообщества и~бизнеса.
     
     Научным коллективом ФИЦ ИУ РАН в~рамках проводимых 
исследований обоснованы~[3] концептуальные и~системотехнические 
подходы к~созданию системы управления научными сервисами (СУС), 
которые обеспечиваются ЦКП 
и~УНУ академических институтов.
     
     В статье обсуждается проблема формализации одного из важнейших 
понятий для системы научных сервисов~--- научный результат, который 
является квинтэссенцией всех научных исследований и~основным 
информационным объектом для процессов постановки научной проб\-ле\-мы, 
формулировки научных гипотез, мониторинга НТИ. 
Формализованное понятие <<научный результат>>, 
интегрирующее всю информационную структуру\linebreak научного исследования, 
используется при организации значительного числа научных сервисов: 
сервисов извлечения фактов и~знаний (извлечение фактов,
 понятий, связей 
и~формализация фактографических данных на основе лингвистического 
анализа слабоструктурированной информации); интеллектуального поиска 
информации; тематического индексирования; наукометрического анализа; 
анализа фронта исследований; сервисов коммуникаций (сервисов работы 
с~НТИ и~коммуникаций в~научном сообществе; интеллектуального анализа 
специализированных социальных сетей и~других средств научных 
коммуникаций); проверки заимствований, антиплагиата. Особенно это 
актуально для междисциплинарных исследований, где обеспечивается поиск и~подбор релевантных научных положений и~инструментария в~смежных 
областях научного знания.
     
     Другим важнейшим вопросом, связанным с~формализацией данного 
понятия, представляется постановка проблемы создания реестра научных 
результатов, который мог бы стать ключевым ресурсом системы научных 
сервисов. В~связи с~этим также рассматриваются задачи и~алгоритмы 
идентификации научных результатов на построенной информационной 
модели данного понятия.

\vspace*{-6pt}
     
\section{Информационная структура процессов научного 
исследования}

     В современном представлении научное исследование~--- 
взаимоувязанная совокупность процессов, включающая ряд типовых 
компонентов~[4, 5]: 
     \begin{itemize}
\item формулировка научной проблемы;
\item предварительный анализ доступной НТИ
(факты, теории, гипотезы);
\item формулировка и~сравнительный анализ исходных гипотез; 
\item планирование научных исследований; 
\item организация и~проведение эксперимента; 
\item анализ и~обобщение полученных результатов;
\item проверка исходных гипотез, принятие решений; 
\item формулирование фактов и~положений, их обосно\-ва\-ние 
и~описание (получение продукта знаний).
    \end{itemize}
    
         \begin{table*}[b]\small %tabl1
         \vspace*{-12pt}
         
     \begin{center}
     \Caption{Структура объекта <<Научный результат>>}
     \vspace*{2ex}
     
     \begin{tabular}{|l|c|p{72mm}|}
     \hline
\multicolumn{1}{|c|}{Структуры/подструктуры}&Типы&\multicolumn{1}{c|}{Описание}\\
\hline
\multicolumn{1}{|l|}{\raisebox{-18pt}[0pt][0pt]
{нр:НаучныйРезультат}}&
\multicolumn{1}{c|}{\raisebox{-18pt}[0pt][0pt]{Корневой}}&Продукт научной и~(или) научно-технической 
дея\-тель\-ности, содержащий новые знания или решения и~зафиксированный на любом 
информационном носителе\\
\hline
нр:ОбластьЗнаний&
\tabcolsep=0pt\begin{tabular}{c}Массив, текст,\\ по классификатору\\ ГРНТИ\end{tabular}&
\multicolumn{1}{l|}{\tabcolsep=0pt\begin{tabular}{l}Область научной и~(или) 
научно-технической дея-\\ тель\-ности, для междисциплинарных исследова-\\ ний~--- 
множественная\end{tabular}}\\
\hline
\multicolumn{1}{|l|}{\raisebox{-12pt}[0pt][0pt]{нр:ОбъектИсследования}}&
\multicolumn{1}{|c|}{\raisebox{-12pt}[0pt][0pt]{Текстовый}}&Материальная или мыслимая сущность, по поводу 
которой изучаются ее неизвестные свойства или сам факт существования\\
\hline
нр:СредствоИсследования&\tabcolsep=0pt\begin{tabular}{c}По классификатору,\\ текстовый\end{tabular}&
\multicolumn{1}{l|}{\tabcolsep=0pt\begin{tabular}{l}Инструменты исследования 
(метод, алгоритм, мо-\\ дель, техническое устройство и~т.\,п.)\end{tabular}}\\
\hline
нр:СвойствоОбъектаИсследования&Текст&Открытое новое свойство объекта 
исследования\\
\hline
нр:ВидРезультата&\tabcolsep=0pt\begin{tabular}{c}Структура,\\ текстовый\end{tabular}&
\multicolumn{1}{l|}{\tabcolsep=0pt\begin{tabular}{l}Иерархический классификатор научных 
результа-\\тов\end{tabular}} \\
\hline
нр:БиблиографическаяСсылка&
\tabcolsep=0pt\begin{tabular}{c}Структура,\\ текстовый\end{tabular}&
\multicolumn{1}{l|}{\tabcolsep=0pt\begin{tabular}{l}Библиографическое описание 
публикации по\\ ГОСТ Р~7.0.100-2018\end{tabular}}\\
\hline
\end{tabular}
\end{center}
\end{table*}
    
    Систематизация процессов научного исследования позволяет 
сформировать ориентировочный набор научных сервисов, который может 
лечь в~основу системы цифровизации науки. В~\cite{4-su} обосновывается 
перечень и~состав основных групп научных сервисов, которые могут быть 
использованы в~процессах научных исследований:
    \begin{itemize}
\item сервисы интеллектуального поиска информации и~мониторинга НТИ;
\item сервисы извлечения фактов и~знаний;
\item аналитические сервисы;
\item сервисы коммуникаций;
\item сервисы планирования научного исследования; 
\item сервисы доступа к~услугам ЦКП и~УНУ;
\item сервисы подготовки публикаций.
\end{itemize}

    Все эти группы научных сервисов с~необхо\-ди\-мостью должны опираться 
на формализованную информационную модель процессов научного 
исследования. Анализ информационной структуры научного исследования 
позволяет выделить ее основные информационные объекты. В~их числе:
    \begin{itemize}
\item объект исследования~--- материальная или мыс\-ли\-мая сущность, по 
поводу которой изучаются ее неизвестные свойства или сам факт 
существования;
\item субъект исследования~--- уче\-ный-ис\-сле\-до\-ва\-тель;
\item средства исследования~--- инструменты исследования (метод, 
алгоритм, модель, техническое устройство и~т.\,п.);
\item научный результат~--- обоснование неизвестных свойств или факта 
существования объекта исследования или средства исследования.
\end{itemize}

    Центральным объектом такой модели служит понятие <<результат 
научного исследования>>, который непосредственно взаимосвязан со всеми 
другими информационными объектами.

\vspace*{-6pt}
    
\section{Формализация понятия <<научный результат>>}

    Законодательно закреплено следующее определение: научный и~(или)  
на\-уч\-но-тех\-ни\-че\-ский результат~--- продукт научной и~(или)  
на\-уч\-но-тех\-ни\-че\-ской деятельности, содержащий новые знания или 
решения и~зафиксированный на любом информационном носителе.
    
    Понятие научных результатов подразделяется на два основных вида:  
во-пер\-вых, научных результатов~--- идей, сформулированных в~виде 
научных положений, и,~во-вто\-рых, результатов методического 
(методологического) и~предметного уровня~--- научного инструментария 
(методов, моделей, методик, экспериментальных установок), научных 
эффектов, результатов экспериментов, устройств, технических 
и~организационных систем и~др.~[6].
    
    Для формирования XML-мо\-де\-ли понятия <<Научный результат>> 
возможно применение сле\-ду\-ющих структур данных (табл.~1).
    

     
    ГРНТИ ({\sf http://grnti.ru})~--- Государственный руб\-ри\-ка\-тор  
НТИ (прежнее наименование~--- 
Рубрикатор ГАСНТИ)~--- представляет собой универсальную иерархическую 
классификацию областей знания, принятую для систематизации всего потока 
НТИ. На основе Рубрикатора построена сис\-те\-ма 
локальных (отраслевых, тематических, проблемных) руб\-ри\-ка\-то\-ров в~органах  
НТИ.
    
    Объект исследования определяется субъектом исследования и~обычно 
дается в~перечне ключевых слов публикации. Новое свойство объекта 
исследования формулируется на естественном языке и~не может быть 
лингвистически формализовано в~виде классификаторов, словарей терминов в~силу своей новизны.

\begin{figure*} %fig1
\vspace*{1pt}
    \begin{center}  
  \mbox{%
 \epsfxsize=121.143mm 
 \epsfbox{suc-1.eps}
 }
\end{center}
\vspace*{-9pt}
\Caption{Распределение публикаций по четырем видам научных результатов: \textit{1}~--- 
инструмент; \textit{2}~--- проблема; \textit{3}~--- объект; \textit{4}~--- обзор}
\end{figure*}
\begin{figure*}[b] %fig2
\vspace*{1pt}
    \begin{center}  
  \mbox{%
 \epsfxsize=121.089mm 
 \epsfbox{suc-2.eps}
 }
\end{center}
\vspace*{-9pt}
\Caption{Распределение публикаций по шести видам научных результатов: \textit{1}~--- 
проблема; \textit{2}~--- инструмент; \textit{3}~--- объект\&инструмент; \textit{4}~--- 
новый объект; \textit{5}~--- новые свойства; \textit{6}~---  систематизация}
\end{figure*}

    
    Так как перечень видов научных результатов формулируется впервые, 
проведено исследование корпуса публикаций научного журнала РАН~[7] за 
период времени 2007--2018~гг.\ с~целью обоснования их классификации. 
Общее число публикаций составило~534. Первоначально в~перечень типов 
научных результатов было включено четыре типа (рис.~1) и~анализ дал 
следующее распределение долей:
     \begin{itemize}
    \item 
\textit{научная проблема} (0,2\% \textit{публикаций});
    \item 
\textit{новые свойства объекта исследования} (30\%);
    \item 
\textit{инструмент исследования} (69\%);
     \item \textit{систематизация} (2,8\%).
     \end{itemize}

\begin{table*}\small %tabl2
\begin{center}
\Caption{Структура объекта <<ВидНаучногоРезультата>>}
\vspace*{2ex}

\begin{tabular}{|p{53mm}|c|p{65mm}|}
\hline
\multicolumn{1}{|c|}{Структуры/подструктуры}&Типы&\multicolumn{1}{c|}{Описание}\\
\hline
вр:ВидРезультата:&Структура, текстовый&Классификатор видов\\
\hline
  НаучноеПоложение:\newline
   \textit{НаучнаяПроблема}\newline
   \textit{   НовыйОбъектИсследования}\newline
\textit{НовыеСвойстваОбъектаИсследования}\newline
     \textit{ Объект\&Инструмент исследования}&Структура, текстовый&Научные 
результаты~--- идеи, сформулированные в~виде научных положений\\
\hline
  НаучныйИнструментарий: \newline
   \textit{ИнструментИсследования}\newline
   \textit{Систематизация}\newline&Структура, текстовый&Методы, модели, методики, 
экспериментальные установки, а также научные эффекты, результаты экспериментов, 
технические и~организационные системы. Систематизация включает в~себя аналитические 
обзоры, классификации, результаты системного анализа\\
\hline
\end{tabular}
\end{center}
\end{table*}


    Очевидно, что при такой типизации возникает существенный перекос 
в~сторону изучения инструментов исследования. Это объясняется тем, что 
в~случае, когда объектом исследования служит научный инструментарий 
(методы, модели, методики, экспериментальные установки), зачастую 
получают и~результаты о новых свойствах изучаемых материальных или 
мыслимых сущностей, т.\,е.\ научный результат содержит в~себе оба 
вышеупомянутых вида. Поэтому тип <<научный инструмент>> был разделен 
на два типа: инструмент исследования как таковой и~инструмент 
исследования с~полученными результатами в~отношении объекта 
исследования. Также тип <<новые свойства объекта исследования>> 
разделен на два: новый объект исследования и~новые свойства объекта 
исследования. Повторный анализ корпуса публикаций дал следующее 
распределение их по предложенному перечню типов научных результатов 
(рис.~2):
    \begin{itemize}
\item \textit{научная проблема} (0,4\%);
\item \textit{новый объект исследования} (0,2\%);
\item \textit{новые свойства объекта исследования} (28\%);
\item \textit{инструмент исследования} (42\%);
\item \textit{объект\&инструмент исследования} (26,6\%);
\item \textit{систематизация} (2,8\%).
\end{itemize}



    Предложенный состав классификатора можно признать 
удовлетворительным, так как он, во-пер\-вых, позволил классифицировать 
все статьи из довольно представительного корпуса публикаций за 12~лет, 
 во-вто\-рых, позволил выявить даже такие редкие виды научных 
результатов, как постановка научной проблемы и~систематизация  
и,~в-треть\-их, позволил осуществить разумное разбиение самого обширного 
раздела публикации~--- изучение научного инструментария (до~70\% 
публикаций). Формализованное описание классификатора пред\-став\-ле\-но 
в~табл.~2.
    


\section{Алгоритмы идентификации научных результатов}

    Формализация обсуждаемого понятия позволяет очертить схему 
алгоритма процесса идентификации научного результата на основе 
мониторинга доступного массива НТИ. Научные сервисы, основанные на 
этом алгоритме, могут позволить решать следующие задачи: определение 
новизны полученного результата, подбор схожих результатов по различным 
критериям, поиск результатов исследований в~смежных отраслях знаний, 
поиск научного инструментария. Можно выделить следующие примеры 
необходимых для этого источников данных, представляющих собой 
отечественные и~зарубежные базы данных (БД) НТИ:
    \begin{itemize}
\item публикации на английском языке ресурса \mbox{arXiv.org} (1~млн 250~тыс.\ 
документов) по естественным наукам;
\item англоязычная Википедия (5~млн 164~тыс.\ статей);
\item БД по американским патентам (USPTO с~2002 по 2016~гг.): 2~млн 
965~тыс.\ патентов;
\item БД по международным патентам (WIPO): 2~млн 384~тыс.\ патентов.
\end{itemize}

    Другим значимым источником в~этой области служат отечественные 
и~зарубежные электронные реферативные БД:
    \begin{itemize}
\item БД ВИНИТИ РАН;
\item электронный каталог ГПНТБ России;
\item Scopus (SciVerse Scopus)~--- библиографический индекс, Elsevier;
\item Web of Science~--- библиографический индекс, Thomson Reuters;
\item NTIS (National Technical Information Service)~--- политематическая 
БД, U.S.\ Department of Commerce;
\item Life Sciences Collection~--- БД по естественным наукам, 
Cambridge Scientific Abstracts;
\item Biological Abstracts~--- БД по биологии, Biological Abstracts 
Inc.;
\item Biotechnology \& Bioengineering~--- БД по биотехнологии 
и~биоинженерии, Cambridge Scientific Abstracts and Engineering Information Inc.;
\item Chemical Engineering and Biotechnology Abstracts~--- БД по 
химической инженерии и~биотехнологии, The Royal Society of Chemistry;
\item Compendex Plus~--- БД по техническим наукам, Engineering 
Information Inc.;
\item Corporate \& Industry Research Reports (CIRR)~--- БД
аналитических отчетов фирм, JA Micropublishing;
\item Current Contents~--- еженедельное обозрение содержания научных 
периодических изданий, Institute for Scientific Information;
\item Derwent Biotechnology Abstracts~--- БД по биотехнологии, 
Derwent Publications Ltd.;
\item MEDLINE~--- БД по медицине и~биологии, National Library of 
Medicine
\item TOXLINE~--- БД по токсикологии, National Library of 
Medicine, Swedish Nat. Chemicals Inspectorate.
\end{itemize}

    Не претендуя на полноту и~окончательность этого перечня ресурсов, 
отметим их тематическое разнообразие, значительные объемы 
и~необходимость наличия глубоких научных компетенций для эффективного 
использования таких данных. Так как понятие научного результата не 
полностью поддается формализации, а значительные объемы связанной 
с~ним информации выражены на естественном языке и~содержатся в~плохо 
структурированных массивах НТИ, первостепенную роль в~алгоритмах 
идентификации играют методы искусственного интеллекта 
и~интеллектуального поиска. Существуют известные методы 
интеллектуального поиска, например~\cite{8-su}, которые включают 
следующие:
    \begin{itemize}
    \item единый каталог документов~--- с~помощью технологий 
каталогизации и~категоризации система позволяет создать единый каталог 
документов из всех источников с~понятной структурой и~удобной 
навигацией;
      \item полнотекстовый поиск по содержанию и~атрибутам~--- при 
формировании результатов обработки запроса поиск ключевых слов ведется в~содержании документа, а также в~значениях атрибутов (полей) учетной 
карточки документа; 
          \item  поиск с~учетом морфологии~--- морфологический поиск позволяет 
найти ключевое слово в~документах не только в~строго заданном виде, но 
и~во всех его морфологических формах (с~учетом рода, числа и~склонения 
по падежам); 
         \item  поиск с~фасетными фильтрами~--- управление размером выборки 
документов с~помощью группы из нескольких фильтров (фасетов), которые 
представляют различные характеристики (тип документа, автор, дата 
создания и~др.); 
          \item  поиск с~учетом словарей синонимов~--- с~использованием словарей 
синонимов, а~также данных о семантической близости слов, полученных 
с~помощью методов дистрибутивной семантики; 
          \item  контекстный поиск~--- позволяет найти документы, содержащие 
ключевые слова, если они расположены не далее указанного расстояния друг 
от друга. 
\end{itemize}

    Алгоритм состоит из ряда последовательных шагов (рис.~3). 
    
    Во-первых, алгоритм опирается на единый реестр научных результатов, 
который используется и~пополняется в~ходе выполнения алгоритма. 
    
    Во-вторых, в~ходе интерактивного диалога формируются запросы 
различной степени формализации о~сути искомого научного результата, 
которые подаются на вход процедур интеллектуального поиска информации. 


    В-третьих, процедуры поиска формируют свои поисковые предписания 
в~соответствии с~особенностями заложенных алгоритмов искусственного 
интеллекта (морфологический и~синтаксический разбор, синонимия 
и~контекстный поиск). 
    
    В-четвертых, по этим предписаниям осуществляется поиск 
в~подключенных источниках НТИ с~использованием единого каталога 
документов. На выходе этих поисковых процедур выдается набор схожих 
научных результатов, ранжированных по степени релевантности. В~случае 
пустого множества искомых данных результат признается новым и~заносится в~единый реестр.
    
    Итак, обоснование структуры и~состава информационной модели 
понятия <<научный результат>>, а~также алгоритмов идентификации 
позволяет обеспечить новыми возможностями всю систему научных 
сервисов: интеллектуальный поиск информации; тематическое 
индексирование; наукометрический анализ; анализ фронта исследований; 
сервисы коммуникаций (сервисы работы с~НТИ
 и~коммуникаций в~научном 
сообществе; интеллектуального анализа специализированных социальных 
сетей и~других средств научных коммуникаций); проверку заимствований. 
Особенно это\linebreak\vspace*{-12pt}

\pagebreak

\end{multicols}

\begin{figure*} %fig3
\vspace*{1pt}
    \begin{center}  
  \mbox{%
 \epsfxsize=161.6mm 
 \epsfbox{suc-3.eps}
 }
\end{center}
\vspace*{-11pt}
\Caption{Схема алгоритма идентификации научного результата}
\vspace*{-3pt}
\end{figure*}

\begin{multicols}{2}


\noindent
 актуально для междисциплинарных исследований, где 
обеспечивается поиск и~подбор релевантных научных положений 
и~инструментария в~смежных областях научного знания. Сформированная 
информационная модель может лечь в~основу создания реестра научных 
результатов, который мог бы стать ключевым ресурсом системы научных 
сервисов.

\vspace*{-12pt}

{\small\frenchspacing
 {%\baselineskip=10.8pt
 \addcontentsline{toc}{section}{References}
 \begin{thebibliography}{9}
\bibitem{1-su}
\Au{Зацаринный А.\,А., Киселев~Э.\,В., Козлов~С.\,В., Колин~К.\,К.} 
Информационное пространство цифровой экономики России. 
Концептуальные основы и~проблемы формирования~/ Под общ. ред. 
А.\,А.~Зацаринного.~--- М.: ФИЦ ИУ РАН, 2018. 236~с. 
\bibitem{2-su}
\Au{Зацаринный А.\,А., Волович~К.\,И., Кондрашев~В.\,А.} Методологические 
вопросы управления научными сервисами научных и~образовательных 
организаций Российской Федерации~// Радиолокация, навигация, связь: Сб. 
трудов XXIII Междунар. научн.-технич. конф.~--- Воронеж: Вэлборн, 2017. 
Т.~1. С.~7--14.
\bibitem{3-su}
Исследование вопросов управления результатами на\-уч\-но-ис\-сле\-до\-ва\-тель\-ской 
деятельности организаций, подведомственных ФАНО России, и~научными 
сервисами сети ЦКП ФАНО.~--- М.: ФИЦ ИУ РАН, 2016. 
 Отчет о~НИР  <<Сервис-У>>. 437~с. 

\bibitem{5-su}
\Au{Ушаков Е.\,В.} Введение в~философию и~методологию науки.~--- М.: 
Экзамен, 2005. 528~с. 

\bibitem{4-su}
\Au{Зацаринный А.\,А., Кондрашев~В.\,А., Сучков~А.\,П.} Система научных 
сервисов как актуальный компонент научных исследований~// Системы 
и~средства информатики, 2019. Т.~29. №\,1. С.~23--38.

\bibitem{6-su}
Понятие, сущность и~классификация исследований. {\sf 
https://studopedia.ru/6\_66034\_ponyatie-sushchnost-i-klassifikatsiya-issledovaniy-rol-issledovaniy-v-razvitii-menedzhmenta.html}.
\bibitem{7-su}
Системы и~средства информатики, 2007--2018.
\bibitem{8-su}
Интеллектуальный поиск. {\sf https://www.naumen.ru/\linebreak products}.
 \end{thebibliography}

 }
 }

\end{multicols}

\vspace*{-12pt}

\hfill{\small\textit{Поступила в~редакцию 30.05.19}}

%\vspace*{8pt}

\pagebreak

%\newpage

\vspace*{-28pt}

%\hrule

%\vspace*{2pt}

%\hrule

%\vspace*{-2pt}

\def\tit{THE SCIENTIFIC RESULT AS~THE~INFORMATION OBJECT 
IN~THE~CONTEXT OF~THE~SCIENTIFIC SERVICES SYSTEM 
MANAGEMENT}


\def\titkol{The scientific result as~the~information object 
in~the~context of~the~scientific services system 
management}

\def\aut{A.\,P.~Suchkov}

\def\autkol{A.\,P.~Suchkov}

\titel{\tit}{\aut}{\autkol}{\titkol}

\vspace*{-11pt}


\noindent
Institute of Informatics Problems, Federal Research Center ``Computer Science and Control'' of the 
Russian Academy of Sciences, 44-2~Vavilov Str., Moscow 119333, Russian Federation 


\def\leftfootline{\small{\textbf{\thepage}
\hfill INFORMATIKA I EE PRIMENENIYA~--- INFORMATICS AND
APPLICATIONS\ \ \ 2019\ \ \ volume~13\ \ \ issue\ 3}
}%
 \def\rightfootline{\small{INFORMATIKA I EE PRIMENENIYA~---
INFORMATICS AND APPLICATIONS\ \ \ 2019\ \ \ volume~13\ \ \ issue\ 3
\hfill \textbf{\thepage}}}

\vspace*{3pt}      
      


\Abste{The article discusses the problem of formalization of one of the most important concepts 
for the system of scientific services~--- the scientific result, which is the quintessence of all 
scientific research and the main information object for the processes of formulation of scientific 
problems, formulation of scientific hypotheses, and
monitoring of scientific and technical 
information (STI). The formalized concept of ``scientific result'' which integrates the entire 
information structure of scientific research is used in the organization of a significant number of 
scientific services: services of extraction of facts and knowledge (extraction of facts, concepts, 
connections, and formalization of factual data based on linguistic analysis of semistructured 
information); intelligent information search; thematic indexing; analysis of the front of research; 
communication services (services of work with STI and communication in the scientific 
community; intellectual analysis of specialized social networks and other means of scientific 
communication; and verification of borrowings). This is especially true for interdisciplinary research 
providing search and selection of relevant scientific provisions and tools in related fields of 
scientific knowledge.}

\KWE{scientific result; information model; scientific services; identification algorithms; 
intellectual search; interdisciplinary research}



\DOI{10.14357/19922264190319} 

\vspace*{-16pt}

\Ack
\noindent
The work was partially supported by the Russian Foundation for Basic Research (project  
18-29-03091).


\vspace*{-4pt}

  \begin{multicols}{2}

\renewcommand{\bibname}{\protect\rmfamily References}
%\renewcommand{\bibname}{\large\protect\rm References}

{\small\frenchspacing
 {%\baselineskip=10.8pt
 \addcontentsline{toc}{section}{References}
 \begin{thebibliography}{9}
\bibitem{1-su-1}
\Aue{Zatsarinnyy, A.\,A., E.\,V.~Kiselev, S.\,V.~Kozlov, and K.\,K.~Kolin.} 
2018. \textit{Informatsionnoe prostranstvo tsifrovoy ekonomiki Rossii. 
Kontseptual'nye osnovy i~problemy formirovaniya} [Information space of the 
digital economy of Russia. Conceptual framework and problems of formation]. 
Moscow: FRC CSC RAS. 236~p. 
\bibitem{2-su-1}
\Aue{Zatsarinny, A.\,A., K.\,I.~Volovich, and V.\,A.~Kondrashev.} 2017. 
Metodologicheskie voprosy upravleniya nauchnymi servisami nauchnykh 
i~obrazovatel'nykh organizatsiy Rossiyskoy Federatsii [Methodological issues of 
scientific services management of scientific and educational organizations of the 
Russian Federation]~// 23rd Scientific and Technical Conference (International) 
``Radar, Navigation, Communication'' Proceedings. Voronezh. 1:7--14.
\bibitem{3-su-1}
FRC CSC RAS. 2016. Issledovanie voprosov upravleniya rezul'tatami nauchno-issledovatel'skoy
deyatel'nosti organizatsiy, podvedomstvennykh FANO Rossii, i~nauchnymi servisami
seti TsKP FANO [Study of issues related to the management of the 
results of research activities of organizations under the jurisdiction of the FASO of Russia, and 
scientific services of the network of FASO collective centers]. Moscow.
Otchet o~NIR <<Servis-U>> [Research Report ``Service-U'']. 437~p.


\bibitem{5-su-1}
\Aue{Ushakov, E.\,V.} 2005. \textit{Vvedenie v~filosofiyu i~metodologiyu nauki} 
[Introduction to philosophy and methodology of science]. Moscow: Ekzamen. 
528~p.

\bibitem{4-su-1}
\Aue{Zatsarinny, A.\,A., V.\,A.~Kondrashev, and A.\,P.~Suchkov.} 2017. Sistema 
nauchnykh servisov kak aktual'nyy komponent nauchnykh issledovaniy [The 
system of scientific services as an actual component of scientific research]. 
\textit{Sistemy i~Sredstva Informatiki~--- Systems and Means of Informatics} 
29(1):23--38.

\bibitem{6-su-1}
Concept, essence and classification of research. Available at: {\sf 
https://studopedia.ru/6\_66034\_ponyatie-sushchnost-i-klassifikatsiya-issledovaniy-rol-issledovaniy-v-razvitii-menedzhmenta.html} 
(accessed May~30, 2019).
\bibitem{7-su-1}
\textit{Sistemy i~Sredstva Informatiki~--- Systems and Means of Informatics}. 
2007--2019.
\bibitem{8-su-1}
Smart search. Available at: {\sf https://www.naumen.ru/\linebreak products/} (accessed 
May~30, 2019).
\end{thebibliography}

 }
 }

\end{multicols}

\vspace*{-7pt}

\hfill{\small\textit{Received May 30, 2019}}

%\pagebreak

\vspace*{-22pt}

\Contrl


\noindent
\textbf{Suchkov Alexander P.} (b.\ 1954)~--- Doctor of Science in technology, 
leading scientist, Institute of Informatics Problems, Federal Research Center 
``Computer Science and Control'' of the Russian Academy of Sciences,  
44-2~Vavilov Str., Moscow 119333, Russian Federation; 
\mbox{ASuchkov@ipiran.ru}
\label{end\stat}

\renewcommand{\bibname}{\protect\rm Литература}  
      
      