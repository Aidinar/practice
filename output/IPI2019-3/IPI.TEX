
\documentclass[10pt]{book}
\usepackage[utf8]{inputenc}

\usepackage{latexsym,amssymb,amsfonts,amsmath,amsxtra,dsfont,
indentfirst,shapepar,%fleqn,%
picinpar,shadow,floatflt,enumerate,multicol,colortbl,moreverb,cite,ipi}

\usepackage{rotating}
\usepackage{mathrsfs}
\usepackage[noend]{algorithmic}
\usepackage{ulem}
\usepackage{graphicx}
%\usepackage{algorithm2e}
\usepackage[linesnumbered,boxed,ruled]{algorithm2e}
%\usepackage{xypic}
\usepackage{oldgerm}
\usepackage{epic}
\usepackage{eepic}


\SetAlgorithmName{Algorithm}{алгоритм}{Список алгоритмов}

%из Дюковой

\newcommand{\algKeyword}[1]{{\bf #1}}
\newcommand{\Proc}[1]{\text{\tt #1}}
\def\CALL{\algKeyword{call}~}

\newenvironment{AlgProcedure}[1]
{
    \small
    \medskip
    %    \hrule
    \medskip
    \algKeyword{PROCEDURE} #1
    \begin{algorithmic}[1]}
    {\end{algorithmic}
    %    \hrule
    \bigskip
}

\def\CALL{\algKeyword{call}~}

%конец для Дюковой

%\RequirePackage[ruled]{algorithm}


\input{epsf}

%\nofiles

%\includeonly{avtor}             %+pdf+
%\includeonly{obchak,avtor}
%\includeonly{pred}                 %+
%\includeonly{podgot-rus-site,podgot-eng-site}  
%\includeonly{ocherk} 
%\includeonly{nekrol} 
%\includeonly{ipi-ind} 
%\includeonly{index12}
%\includeonly{toc-rus, toc-en}
%\includeonly{toc-rus}
%\includeonly{toc-en} 



%\includeonly{grusho}                   %1+pdf+авт+
%\includeonly{pavlov}                   %2+pdf+авт+
%\includeonly{zeifman}                  %3+pdf+авт+
%\includeonly{lebedev}                  %4+pdf+авт+
%\includeonly{kudr}                     %5+pdf+авт??
%\includeonly{gorsh+mart}               %6+pdf+авт+
%\includeonly{bosov}                    %7+pdf+авт+
%\includeonly{shnurkov}                 %8+pdf
%\includeonly{zaharova}                 %9+pdf
%\includeonly{gaidamaka}                %10+pdf+авт+
%\includeonly{agasand}                  %11+pdf+авт+
%\includeonly{zatsman}                  %12+pdf+авт??????????повт???
%\includeonly{nuriev}                   %13+pdf+авт+
%\includeonly{goncharov+ink}            %14+pdf+авт+
%\includeonly{panov}                    %15+pdf+авт+
%\includeonly{gor+kuz}                  %16+pdf+авт+
%\includeonly{rumov+kir}                %17+pdf
%\includeonly{burl+yak}                 %18+pdf+авт+
%\includeonly{suchkov}                  %19+pdf



%\includeonly{nekrol}             %+


%\includeonly{obchak}
%\includeonly{rekl}
%\includeonly{rekl-1}
%\includeonly{reshal}  %
%\includeonly{cover3}

\usepackage{acad}
%\usepackage{courier}
\usepackage{decor}
\usepackage{newton}
\usepackage{pragmatica}
\usepackage{zapfchan}
\usepackage{petrotex}
\usepackage{bm}                     % полужирные греческие буквы
\usepackage{upgreek}                % прямые греческие буквы
\usepackage{eufrak}
\usepackage{verbatim}

\renewcommand{\bottomfraction}{0.99}
\renewcommand{\topfraction}{0.99}
\renewcommand{\textfraction}{0.01}

\setcounter{secnumdepth}{1} %здесь - 3 + chapter = 4

\arraycolsep=1.5pt

%\usepackage[pdftex]{graphicx}

%\usepackage{oz}

%NEW COMMANDS


\renewcommand*{\hm}[1]{#1\nobreak\discretionary{}%
            {\hbox{$\mathsurround=0pt #1$}}{}} %% Дублирует знаки операций
                               %при переносе в формуле (перед знаком, который
                               %надо продублировать ставится команда \hm)

%\newcommand{\endproof}{\hfill$\Box$}
\renewcommand{\r}{\mathbb{R}}
%\newcommand{\I}{{\rm I\hspace{-0.7mm}I}}
%\newcommand{\Ikl}{{\tt{1}}\hspace*{-1.44mm}\mathtt{1}}
\newcommand{\Ik}{\mbox{{\small \tt {1}}\hspace{-1.3mm}{\tt 1}}}
\newcommand{\argmin}{\mathop{\mathrm{arg}\,\mathrm{min}}}
\newcommand{\argmax}{\mathop{\mathrm{arg}\,\mathrm{max}}}
%\newcommand{\capr}{\mathop{\cap\,}}
%\newcommand{\cupr}{\mathop{\cup\,}}
%\def\argmin{\mathop{arg\,min}}

\def\vrp{\varphi}
\def\prt{\partial}
\def\mm{{\sf M}}
\def\modnop#1{\mathop{#1}\limits_{n}}
\def\eam{\mathbin{{\mathop{=}\limits^{\mathrm{def}}}}}
\def\dey#1#2{#1 (#2)}
\def\deyc#1#2{#1 \cdot  #2}
\def\ra#1{\;\mathop{\to}\limits^{#1}\;}
\def\raz#1{\;\mathop{\longrightarrow}\limits^{\!\!\!#1}\;}
\def\ral#1{\;\mathop{\longrightarrow}\limits^{#1}\;}

\newcommand{\Nor}{\mathcal{N}}
\newcommand{\T}{\mathbb{T}}
\newcommand{\Z}{\mathbb{Z}}



\newcommand{\il}[2]{\int\limits_{#1}^{#2}}%интеграл с пределами #1 и #2

\def\sm2{\mathop {\sum\limits^{n^\Theta}\sum\limits^{n^\Theta}}}
\def\sss{\sum\limits}
\def\tr{,\,\ldots\,,\,}
\def\rk{\right]}
\def\lk{\left[}
\def\rf{\right\}}
\def\lf{\left\{}
\def\lv{\,\left\vert}
\def\rv{\right\vert\,}
\def\iii{\int\limits}
\def\iin{\int\limits_{-\infty}^\infty}
\def\rrv{\right\vert}


\def\ee{{\cal E}}
\def\ww{{\cal W}}
\def\yy{{\cal Y}}
\def\vv{{\cal V}}

\newcommand{\R}{\mathbb R}
\newcommand{\E}{\mathbb E}
\newcommand{\N}{\mathbb N}

\renewcommand{\P}{\mathbb{P}}

\newcommand{\h}{{\bf H}}
\newcommand{\p}{{\sf P}}  % вероятность

\newcommand{\e}{{\sf E}}  % мат. ожидание
\newcommand{\D}{{\sf D}}  % дисперсия
\newcommand{\eps}{\varepsilon}
\newcommand{\vp}{{\mathbf p}}
\newcommand{\vz}{{\mathbf z}}
\newcommand{\vx}{{\mathbf x}}
\newcommand{\vf}{{\mathbf f}}
\newcommand{\F}{{\mathcal F}}
\def\ap{{\mathrm{ЭР}}}
\newcommand{\ud}{\Delta_n} %uniform ditance
\newcommand{\nud}{\Delta_n(x)}
%\renewcommand{\Re}{\mathrm{Re}\,}

\newcommand{\abs}[1]{\left\vert#1\right\vert}

\newcommand{\norm}[1]{\left\Vert#1\right\Vert}
\def\da{(\Delta_t,A)}

\newcommand{\corr}{\mathrm{corr}}

\newcommand{\cov}{\mathrm{cov}}
\newcommand{\Expect}{\mathbb{E}}

\def\w{\omega}
\def\W{\Omega}

\def\inh{\int\limits_{nh}^{(n+1)h}}

\def\sumin{\sum_{i=1}^N}


\def\bxt{(Y,t)}
\def\xt{(y,t)}

\def\ovth{{\fr{\tau-nh}{h}}}
\def\ov{\overline}
\def\tm{\tilde m}
\def\tl{\tilde\lambda}
\def\tB{\widetilde B}
\def\tb{\tilde b}
\def\ld{\ldots}
\def\cd{\cdots}


\DeclareMathOperator{\sign}{sign}

%\newcommand{\gr}{{\geqslant}}


\newcommand{\g}{\mbox{\textit{g}}}

\renewcommand{\la}{\lambda}
\newcommand{\si}{\sigma}
\newcommand{\alp}{\alpha}

\newcommand{\pto}{\stackrel{P}{\longrightarrow}} % сходимость по веpоятности

\newcommand{\eqd}{\stackrel{\mathrm{d}}{=}} % равенство по pаспpеделению
\newcommand{\eqdelta}{\stackrel{\triangle}{=}} % равенство по pаспpеделению

\def\be#1{\begin{equation}\label{#1}}
\def\ee{\end{equation}}
\def\re#1{(\ref{#1})}

\def\bn{\begin{enumerate}}
\def\en{\end{enumerate}}
\def\bi{\begin{itemize}}
\def\ei{\end{itemize}}
%\def\i{\item}

%\newcommand{\kp}{\kappa}
%\def\Q{{\cal Q}} \def\H{{\cal H}}
%\newcommand{\bet}{\beta_{2+\delta}}


%\newtheorem{definition}{Определение}
%\renewcommand{\thedefinition}{\arabic{definition}.}
%END NEW COMMANDS

%\renewcommand{\baselinestretch}{1.2}

%\pagestyle{myheadings}

\setlength{\textwidth}{167mm}      % 122mm
\setlength{\textheight}{658pt}
%\setlength{\textheight}{635.6pt}
\setlength{\columnsep}{4.5mm}

\setcounter{secnumdepth}{4}

%\addtolength{\headheight}{2pt}
%\addtolength{\headsep}{-2mm}

\addtolength{\topmargin}{-7mm}  % for printing


%\hoffset=-30mm  % From Yap
\hoffset=-23mm  % From Acrobat

%\voffset=0mm % From Yap
\voffset=-5mm   % From Acrobat

%\addtolength{\evensidemargin}{-2.5mm} % for printing
%\addtolength{\oddsidemargin}{2.5mm}  % for printing

\addtolength{\evensidemargin}{-12mm} % for printing
\addtolength{\oddsidemargin}{8mm}  % for printing

%\renewcommand{\thefootnote}{\fnsymbol{footnote}}
%\renewcommand{\thefootnote}{\arabic{footnote}}
\renewcommand{\figurename}{\protect\bf Рис.}
\renewcommand{\tablename}{\protect\bf Таблица}

\newcommand{\Caption}[1]{\caption{\protect\small %\baselineskip=2.5ex
#1}}

\renewcommand{\thefigure}{\arabic{figure}}
\renewcommand{\thetable}{\arabic{table}}
\renewcommand{\theequation}{\arabic{equation}}
\renewcommand{\thesection}{\arabic{section}}

\renewcommand{\contentsname}{СОДЕРЖАНИЕ}
\newcommand{\fr}[2]{\displaystyle\frac{\displaystyle #1\mathstrut}{\displaystyle #2\mathstrut}}

%\renewcommand{\thefootnote}{\fnsymbol{footnote}}
%\newcommand{\g}{\mbox{\textit{g}}}

%\newcommand{\Caption}[1]{\caption{\protect\small\baselineskip=2ex #1}}
\newcounter{razdel}
\setcounter{razdel}{0}


\newcommand{\titel}[4]{%
\

\vspace*{5pt}

\ifodd\therazdel {\raggedright\noindent\Large\textrm\textbf
 \lineskip .75em
  \baselineskip=3.2ex #1 \par}
\vskip 1em {\noindent\large\textrm\textbf #2 \par}
\addcontentsline{toc}{subsection}{{\textrm\textbf #1}\protect\newline #2}
\def\rightheadline{\underline{\noindent\hbox to \textwidth{\hfill\small\textrm{#4}
%\hfill \large\bf\thepage
}}}
\def\leftheadline{\underline{\noindent\parbox{\textwidth}{
%\raggedleft\large\bf\thepage \hfill
\small\textit{#3}\hfill}}}
\def\leftfootline{\small{\textbf{\thepage}
\hfill ИНФОРМАТИКА И ЕЁ ПРИМЕНЕНИЯ\ \ \ том~13\ \ \ выпуск 3\ \ \ 2019}
}%
 \def\rightfootline{\small{ИНФОРМАТИКА И ЕЁ ПРИМЕНЕНИЯ\ \ \ том~13\ \ \ выпуск~3\ \ \ 2019
\hfill \textbf{\thepage}}}
\vskip 2em \setcounter{figure}{0}
\setcounter{table}{0}
\setcounter{equation}{0}
\setcounter{section}{0}
\setcounter{subsection}{0}
\setcounter{subsubsection}{0}
\setcounter{footnote}{0}
\setcounter{razdel}{0}
%\end{flushleft}
\else {
 \raggedright\noindent\Large\textrm\textbf
 \lineskip .75em
\baselineskip=3.2ex #1 \par} \vskip 1em
%\begin{flushleft}
{\noindent\large\textrm\textbf #2 \par}
\addcontentsline{toc}{subsection}{{\textrm\textbf #1}\protect\newline #2}
\def\rightheadline{\underline{\noindent\hbox to \textwidth{\hfill\small\textrm{#4}
%\hfill \large\bf\thepage
}}}
\def\leftheadline{\underline{\noindent\parbox{\textwidth}{%\raggedleft\large\bf\thepage \hfill
\small\textit{#3}\hfill}}}
\def\leftfootline{\small{\textbf{\thepage}
\hfill ИНФОРМАТИКА И ЕЁ ПРИМЕНЕНИЯ\ \ \ том~13\ \ \ выпуск~3\ \ \ 2019}
}%
 \def\rightfootline{\small{ИНФОРМАТИКА И ЕЁ ПРИМЕНЕНИЯ\ \ \ том~13\ \ \ выпуск~3\ \ \ 2019
\hfill \textbf{\thepage}}} \vskip 2em \setcounter{figure}{0}
\setcounter{table}{0} \setcounter{equation}{0} \setcounter{section}{0}
\setcounter{subsection}{0} \setcounter{subsubsection}{0}
\setcounter{footnote}{0}
%\end{flushleft}
\fi}

\newcommand{\titelr}[2]{%
\

\vspace*{5pt}

\ifodd\therazdel {\raggedright\noindent%\Large\textrm\textbf
 \lineskip .75em
  \baselineskip=3.2ex #1 \par}
\vskip 1em {\noindent\normalsize\textrm\textbf #2 \par}
\else {
 \raggedright\noindent\Large\textrm\textbf
 \lineskip .75em
\baselineskip=3.2ex #1 \par} \vskip 1em
%\begin{flushleft}
{\noindent\large\textrm\textbf #2 \par
%\noindent\normalsize\textrm\textbf #2 \par
} \fi}

\newcommand{\titele}[5]{%
\

%\vspace*{5pt}

\ifodd\therazdel {\raggedright\noindent\large
\textrm\textbf
 \lineskip .75em
%  \baselineskip=3.2ex
#1 \par}
\vskip .5em {\noindent\large\textrm\textbf #2 \par}
\vskip .5em
 {\noindent\textrm #3 \par}
\addcontentsline{toc}{subsection}{{\textrm\textbf #1}\protect\newline #2}
\def\rightheadline{\underline{\noindent\hbox to \textwidth{\hfill\small\textrm{#4}
%\hfill \large\bf\thepage
}}}
\def\leftheadline{\underline{\noindent\parbox{\textwidth}{
%\raggedleft\large\bf\thepage \hfill
\small\textrm{#5}\hfill}}}
\def\leftfootline{\small{\textbf{\thepage}
\hfill ИНФОРМАТИКА И ЕЁ ПРИМЕНЕНИЯ\ \ \ том~13\ \ \ выпуск~3\ \ \ 2019}
}%
 \def\rightfootline{\small{ИНФОРМАТИКА И ЕЁ ПРИМЕНЕНИЯ\ \ \ том~13\ \ \ выпуск~3\ \ \ 2019
\hfill \textbf{\thepage}}} \vskip 1em \setcounter{figure}{0}
\setcounter{table}{0} \setcounter{equation}{0} \setcounter{section}{0}
\setcounter{subsection}{0} \setcounter{subsubsection}{0}
\setcounter{footnote}{0} \setcounter{razdel}{0}
%\end{flushleft}
\else {
 \raggedright\noindent\large
 \textrm\textbf
 \lineskip .75em
%\baselineskip=3.2ex
#1 \par} \vskip .5em
%\begin{flushleft}
{\noindent\large\textrm\textbf #2 \par} \vskip .5em
 {\noindent\textrm #3 \par}
\addcontentsline{toc}{subsection}{{\textrm\textbf #1}\protect\newline #2}
\def\rightheadline{\underline{\noindent\hbox to \textwidth{\hfill\small\textrm{#4}
%\hfill \large\bf\thepage
}}}
\def\leftheadline{\underline{\noindent\parbox{\textwidth}{%\raggedleft\large\bf\thepage \hfill
\small\textrm{#5}\hfill}}}
\def\leftfootline{\small{\textbf{\thepage}
\hfill ИНФОРМАТИКА И ЕЁ ПРИМЕНЕНИЯ\ \ \ том~13\ \ \ выпуск~3\ \ \ 2019}
}%
 \def\rightfootline{\small{ИНФОРМАТИКА И ЕЁ ПРИМЕНЕНИЯ\ \ \ том~13\ \ \ выпуск~3\ \ \ 2019
\hfill \textbf{\thepage}}} \vskip 1em \setcounter{figure}{0}
\setcounter{table}{0} \setcounter{equation}{0} \setcounter{section}{0}
\setcounter{subsection}{0} \setcounter{subsubsection}{0}
\setcounter{footnote}{0}
%\end{flushleft}
\fi}

\def\Abst#1{
\begin{center}\small\nwt
\parbox{150mm}{%\baselineskip=2.5ex
\textbf{Аннотация:}\ \
%\hspace*{\parindent}
#1}
\end{center}}
\def\Abste#1{
\begin{center}\small\nwt
\parbox{150mm}{%\baselineskip=2.5ex
\textbf{Abstract:}\ \
%\hspace*{\parindent}
#1}
\end{center}}

\def\DOI#1{
\begin{center}\small\nwt
\parbox{150mm}{%\baselineskip=2.5ex
\textbf{DOI:}\ \
%\hspace*{\parindent}
#1}
\end{center}}

\def\Abstend#1{
\begin{center}\small\nwt
\parbox{150mm}{%\baselineskip=2.5ex
%\hspace*{\parindent}
#1}
\end{center}}


\def\KW#1{
\begin{center}\small\nwt
\parbox{150mm}{%\baselineskip=2.5ex
\textbf{Ключевые слова:}\ \ #1}
\end{center}}

\def\KWE#1{
\begin{center}\small\nwt
\parbox{150mm}{%\baselineskip=2.5ex
\textbf{Keywords:}\ \ #1}
\end{center}}


\def\KWN#1{
%\begin{center}
%\small
%\parbox{150mm}\end{center}
}

\newcommand{\Avtors}[1]{%\smallskip
%\vspace*{.5pt}
\hangindent=23pt\noindent
%\nwt
{\bfseries#1}\
}


\renewcommand{\thesubsection}{\thesection.\arabic{subsection}\hspace*{-5pt}}
\renewcommand{\thesubsubsection}{\thesubsection\hspace*{5pt}.\arabic{subsubsection}\hspace*{-3pt}}

\newcommand{\Ack}{\section*{\protect\rmfamily Acknowledgments}\noindent}
\newcommand{\Contr}{\section*{\protect\rmfamily Contributors}\noindent}
\newcommand{\Contrl}{\section*{\protect\rmfamily Contributor}\noindent}

\makeindex


\begin{document}
\Rus

\nwt
%\ptb


%\renewcommand{\contentsname}{\protect\Large\bf Содержание}

\setcounter{tocdepth}{2}

%\tableofcontents

\renewcommand{\bibname}{\protect\rmfamily Литература}
  \def\Au#1{{\it #1}}
    \def\Aue#1{{#1}}

%\newcommand{\No}{№}
  \newcommand{\tg}{\,\mathrm{tg}\,}
    \newcommand{\ctg}{\,\mathrm{ctg}\,}
  \newcommand{\arctg}{\,\mathrm{arctg}\,}

\def\forallb{\mathop{\forall}}
\def\cupb{\mathop{\cup}}
\def\existsb{\mathop{\exists}}


\newpage
\addtocounter{razdel}{1}
%\def\razd{РЕГУЛИРУЕМЫЙ ЭЛЕКТРОПРИВОД ДЛЯ ЭЛЕКТРОЭНЕРГЕТИКИ}


\setcounter{page}{3}

%   { %\Large  
   { %\baselineskip=16.6pt
   
   \vspace*{-48pt}
   \begin{center}\LARGE
   \textit{Предисловие}
   \end{center}
   
   %\vspace*{2.5mm}
   
   \vspace*{25mm}
   
   \thispagestyle{empty}
   
   { %\small 

    
Вниманию читателей журнала <<Информатика и её применения>> предлагается 
очередной тематический выпуск <<Вероятностно-статистические методы и 
задачи информатики и информационных технологий>>. Предыдущие тематические 
выпуски журнала по данному направлению вышли в 2008~г.\ (т.~2, вып.~2), 
в 2009~г.\ (т.~3, вып.~3) и в 2010~г.\ (т.~4, вып.~2). 

Статьи, собранные в данном журнале, посвящены разработке новых вероятностно-статистических 
методов, ориентированных на применение к решению конкретных задач информатики и информационных 
технологий, а также~--- в ряде случаев~--- и других прикладных задач. Проблематика, охватываемая 
публикуемыми работами, развивается в рамках научного сотрудничества между Институтом проблем 
информатики Российской академии наук (ИПИ РАН) и Факультетом вычислительной математики и 
кибернетики Московского государственного университета им.\ М.\,В.~Ломоносова в ходе работ 
над совместными научными проектами (в том числе в рамках функционирования 
Научно-образовательного центра <<Вероятностно-статистические методы анализа рисков>>). 
Многие из авторов статей, включенных в данный номер журнала, являются активными участниками 
традиционного международного семинара по проблемам устойчивости стохастических моделей, 
руководимого В.\,М.~Золотаревым и В.\,Ю.~Королевым; регулярные сессии этого семинара 
проводятся под эгидой МГУ и ИПИ РАН (в 2011~г.\ указанный семинар проводится в октябре 
в Калининградской области РФ). 

Наряду с представителями ИПИ РАН и МГУ в число авторов данного выпуска журнала входят 
ученые из Научно-исследовательского института системных исследований РАН, Института 
проблем технологии микроэлектроники и особочистых материалов РАН, Института 
прикладных математических исследований Карельского НЦ РАН, Московского 
авиационного института, Вологодского государственного педагогического университета, 
НИИММ им.\ Н.\,Г.~Чеботарева, Казанского государственного университета, Дебреценского 
университета (Венгрия).

Несколько статей выпуска посвящено разработке и применению стохастических методов и 
информационных технологий для решения различных прикладных задач. В~работе В.\,Г.~Ушакова 
и О.\,В.~Шестакова рассмотрена задача определения вероятностных характеристик случайных 
функций по распределениям интегральных преобразований, возникающих в задачах эмиссионной 
томографии. В~статье Д.\,О.~Яковенко и М.\,А.~Целищева рассмотрены некоторые вопросы 
математической теории риска и предложен новый подход к диверсификации инвестиционных 
портфелей. Работа И.\,А.~Кудрявцевой и А.\,В.~Пантелеева посвящена построению и 
исследованию математической модели, описывающей динамику сильноионизованной плазмы. 
В~статье П.\,П.~Кольцова изучается качество работы ряда алгоритмов сегментации изображений. 
Статья А.\,Н.~Чупрунова и И.~Фазекаша посвящена вероятностному анализу числа без\-оши\-бочных 
блоков при помехоустойчивом кодировании; получены усиленные законы больших чисел для указанных 
величин.

В данном выпуске традиционно присутствует тематика, весьма активно разрабатываемая в течение 
многих лет специалистами ИПИ РАН и МГУ,~--- методы моделирования и управления для 
информационно-телекоммуникационных и вычислительных систем, в частности методы 
теории массового обслуживания. В~статье А.\,И.~Зейфмана с соавторами рассматриваются 
модели обслуживания, описываемые марковскими цепями с непрерывным временем в случае 
наличия катастроф. В~работе М.\,М.~Лери и И.\,А.~Чеплюковой рассматриваются случайные 
графы Интернет-типа, т.\,е.\ графы, степени вершин которых имеют степенные распределения; 
такие задачи находят применение при исследовании глобальных сетей передачи данных. 
Работа Р.\,В.~Разумчика посвящена исследованию систем массового обслуживания специального 
вида~--- с отрицательными заявками и хранением вытесненных заявок.

Ряд статей посвящен развитию перспективных теоретических 
вероятностно-статистических методов, которые находят широкое применение в различных 
задачах информатики и информационных технологий. В~работе В.\,Е.~Бенинга, А.\,К.~Горшенина 
и В.\,Ю.~Королева рассмотрена задача статистической проверки гипотез о числе компонент 
смеси вероятностных распределений, приводится конструкция асимптотически наиболее мощного 
критерия. Результаты этой работы найдут применение в ряде прикладных задач, использующих 
математическую модель смеси вероятностных распределений (в информатике, моделировании 
финансовых рынков, физике турбулентной плазмы и~т.\,д.). В~статье В.\,Ю.~Королева, 
И.\,Г.~Шевцовой и С.\,Я.~Шоргина строится новая, улучшенная оценка точности нормальной 
аппроксимации для пуассоновских случайных сумм; как известно, указанные случайные суммы 
широко используются в качестве моделей многих реальных объектов, в том числе в информатике, 
физике и других прикладных областях. Работа В.\,Г.~Ушакова и Н.\,Г.~Ушакова посвящена 
исследованию ядерной оценки плотности распределения; эти результаты могут применяться, 
в част\-ности, при анализе трафика в телекоммуникационных системах. Серьезные приложения 
в статистике могут получить результаты работы О.\,В.~Шестакова, в которой доказаны оценки 
скорости сходимости распределения выборочного абсолютного медианного отклонения к нормальному 
закону. 

\smallskip

Редакционная коллегия журнала выражает надежду, что данный тематический  выпуск 
будет интересен специалистам в области теории вероятностей и математической статистики 
и их применения к решению задач информатики и информационных технологий.
     
     %\vfill 
     \vspace*{20mm}
     \noindent
     Заместитель главного редактора журнала <<Информатика и её 
применения>>,\\
     директор ИПИ РАН, академик  \hfill
     \textit{И.\,А.~Соколов}\\
     
     \noindent
     Редактор-составитель тематического выпуска,\\
     профессор кафедры математической статистики факультета\\
      вычислительной математики и кибернетики МГУ им.\ М.\,В.~Ломоносова,\\
     ведущий научный сотрудник ИПИ РАН,\\ 
доктор физико-математических наук \hfill
      \textit{В.\,Ю.~Королев}
     
     } }
     }

\def\stat{grusho}

\def\tit{АРХИТЕКТУРНЫЕ РЕШЕНИЯ В~ЗАДАЧЕ ВЫЯВЛЕНИЯ МОШЕННИЧЕСТВА ПРИ~АНАЛИЗЕ 
ИНФОРМАЦИОННЫХ ПОТОКОВ В~ЦИФРОВОЙ ЭКОНОМИКЕ$^*$}

\def\titkol{Архитектурные решения в~задаче выявления мошенничества при~анализе 
информационных потоков в
%~цифровой 
экономике}

\def\aut{А.\,А.~Грушо$^1$, М.\,И.~Забежайло$^2$, Н.\,А.~Грушо$^3$, 
Е.\,Е.~Тимонина$^4$}

\def\autkol{А.\,А.~Грушо, М.\,И.~Забежайло, Н.\,А.~Грушо, 
Е.\,Е.~Тимонина}

\titel{\tit}{\aut}{\autkol}{\titkol}

\index{Грушо А.\,А.}
\index{Забежайло М.\,И.}
\index{Грушо Н.\,А.}
\index{Тимонина Е.\,Е.}
\index{Grusho A.\,A.}
\index{Zabezhailo M.\,I.}
\index{Grusho N.\,A.}
\index{Timonina E.\,E.}


{\renewcommand{\thefootnote}{\fnsymbol{footnote}} \footnotetext[1]
{Работа частично поддержана РФФИ (проекты 18-29-03081 и~18-07-00274).}}


\renewcommand{\thefootnote}{\arabic{footnote}}
\footnotetext[1]{Институт проблем информатики Федерального исследовательского центра <<Информатика и~управление>> 
Российской академии наук, grusho@yandex.ru}
\footnotetext[2]{Институт проблем информатики Федерального исследовательского центра <<Информатика и~управление>> 
Российской академии наук, m.zabezhailo@yandex.ru}
\footnotetext[3]{Институт проблем информатики Федерального исследовательского центра <<Информатика и~управление>> 
Российской академии наук, info@itake.ru}
\footnotetext[4]{Институт проблем информатики Федерального исследовательского центра <<Информатика и~управление>> 
Российской академии наук, eltimon@yandex.ru}

\vspace*{-12pt}
   

 
  
  \Abst{Cформулирован подход к~исследованию некоторых видов мошенничества в~цифровой 
экономике с~использованием причинно-следственных связей. Во всех видах рассматриваемых 
мошенничеств должно наблюдаться несоответствие между целями финансовых транзакций 
и~реальной стоимостью достижения этих целей. Данные о транзакциях можно собирать, 
наблюдая информационные потоки, в~которых отражаются эти транзакции. Архитектура сбора 
данных и~их анализа может быть организована с~помощью распределенных реестров 
с~централизованным консенсусом, что позволяет создать аналог электронной бухгалтерской 
книги, фиксирующей финансово-экономическую деятельность субъектов цифровой экономики в~регионе. 
  Рассматриваемые методы выявления мошенничества основаны на противоречиях 
между действиями, описанными в~транзакциях, и~информацией, содержащейся в~планах, 
стандартах, прецедентах и~др. Рассмотрен метод, основанный на некоторой упрощенной схеме 
реализации абстрактного проекта. Для выявления противоречий необходимо проводить анализ 
от следствия к~причине, т.\,е.\ искать аномалии в~информации, описывающей порождение 
наблюдаемых следствий. 
  Показано, как в~реализации проекта можно выделять простые <<необходимые условия>> 
нарушения при\-чин\-но-след\-ст\-вен\-ных связей, т.\,е.\ множество <<необходимых условий>>, 
нарушение которых свидетельствует о наличии мошенничества. Это множество <<необходимых 
условий>> можно назвать метаданными для контроля проекта на выявление мошенничества.} 
 
 
  \KW{цифровая экономика; информационные потоки; при\-чин\-но-след\-ст\-вен\-ные связи; 
выявление мошеннических схем} 

\DOI{10.14357/19922264190204}
  
\vspace*{-4pt}


\vskip 10pt plus 9pt minus 6pt

\thispagestyle{headings}

\begin{multicols}{2}

\label{st\stat}

\section{Введение}

\vspace*{3pt}

  В работе сформулирован подход к~исследованию некоторых видов 
мошенничества в~цифровой экономике с~использованием  
при\-чин\-но-след\-ст\-вен\-ных связей. Рассматриваются три вида мошенничества, 
а именно:
  \begin{enumerate}[(1)]
\item отмыв денег; 
\item обман при выполнении договорных обязательств при реализации 
технических проектов (строительные проекты и~др.); 
\item незаконный вывод денег. 
\end{enumerate}

  Названные виды мошенничества могут быть сведены к~решению одного типа 
задач. Для отмывания денег источник должен заключать фиктивные контракты, 
в~соответствии с~которыми будут переводиться средства за заведомо ненужную 
работу и~материалы. 
  
  Мошенничество, связанное с~невыполнением договорных обязательств, связано 
со снижением качества услуг, качества и~количества закупаемых 
материалов, выполнением работ с~ненадлежащим качеством. 
  
  Вывод денег связан с~переводом средств фир\-мам-од\-но\-днев\-кам, которые 
заведомо не могут выполнить обязательства по контрактам, за которые им 
переводятся средства. 
  
  Таким образом, во всех трех видах рассматриваемых мошенничеств должно 
наблюдаться несоответствие между целями финансовых транзакций и~реальной 
стоимостью достижения этих целей. Данные о транзакциях можно собирать, 
наблюдая информационные потоки, в~которых отражаются эти транзакции. 
  
  Однако для наблюдения таких информационных потоков необходимо создавать 
архитектуру\linebreak телекоммуникационной системы, позволяющей перехватывать 
и~собирать данные о всех транзакциях. Например, такая архитектура может быть 
организована с~помощью распределенных реестров с~централизованным 
консенсусом, т.\,е.\ все информационные потоки, сформированные в~цифровой 
экономике и~несущие информацию о транзакциях, проходят через некоторый 
центральный узел, запоминающий их в~форме распределенного реестра. Такие 
реестры могут дублироваться в~аналогичных центрах различных регионов, что 
позволяет создать аналог электронной бухгалтерской книги, фиксирующей 
фи\-нан\-со\-во-эко\-но\-ми\-че\-скую деятельность субъектов цифровой экономики. Такой 
подход предложено реализовать на базе системы ситуационных центров, что 
отражено в~работах~[1, 2].
  
  Собранная из информационных потоков информация о~транзакциях, т.\,е.\ 
о~контрактах, договорах, платежах, отчетах, закупленных материалах, 
характеристиках исполнителей работ и~др., собирается в~базе данных в~указанном 
центре. Согласно теории интеллектуальных сис\-тем~[3], эту базу данных можно 
называть базой фактов (БФ). Базу фактов можно представить как бинарную мат\-ри\-цу, 
строки которой описывают характеристики, входящие в~транзакции, а столбцы 
нумеруются характеристиками. Строки матрицы будем называть 
\textit{объектами}~[4, 5]. 
  
  Рассматриваемые в~работе методы выявления мошенничества будут основаны 
на противоречиях между действиями, описанными в~транзакциях, и~информацией, 
содержащейся в~планах, стандартах, прецедентах и~др. Для нахождения 
противоречий в~архитектуре центра предусмотрена другая база данных~--- база 
знаний (БЗ)~\cite{3-gr, 6-gr}, которая устроена так же, как БФ. 
  
  Информация в~БЗ собирается на основе положительного опыта или расчетов. 
Используя БЗ, можно выводить факты нарушения при\-чин\-но-след\-ст\-вен\-ных 
связей. Нарушения при\-чин\-но-след\-ст\-вен\-ных связей будем называть 
\textit{аномалиями}. 
  
  Для упрощения дальнейшее изложение будет вестись в~рамках поиска 
противоречий при выполнении некоторого абстрактного проекта. Выявление 
аномалий будет происходить на основе фактов из БФ с~помощью знаний из БЗ 
методами искусственного интеллекта и~интеллектуального анализа 
данных~\cite{6-gr}. 

\vspace*{-10pt}
  
  \section{Модели}
  
  \vspace*{-3pt}
  
  Наиболее сложная из рассмотренных выше задач~--- выявление противоречий, 
т.\,е.\ использование БЗ для получения новых знаний и~выявление аномалий из 
полученных фактов. 
  
  Все способы выявления противоречий основаны на определении 
  причинно-следственных связей. При этом противоречия в~параметрах транзакций по 
отношению к~требуемым в~БЗ составляют сущность аномалий. 
  
   Далее будет рассмотрен метод, основанный на некоторой упрощенной схеме 
реализации абстрактного проекта. 
  
  Каждый проект имеет цель: например, цель представляет собой построение 
некоторой системы. Воспользуемся структурным подходом, который позволяет 
строить проект на основе разбиения системы на подсистемы и~определения 
взаимодействий подсистем~\cite{7-gr}. При этом каждая подсистема также 
представима структурной моделью. 
  
  Как сама система, так и~каждая ее подсистема имеют свой функционал 
и~спецификацию, па\-ра\-мет\-ры настройки и~домены параметров настройки. Кроме 
этих характеристик существует множество характеристик, связанных 
с~<<жизненным циклом>> создания системы. Сюда входят работы, ресурсы, 
сроки выполнения работ по созданию подсистем и~самой системы, стоимости 
компонентов и~материалов, стоимости работ, схемы поставок, договорные 
обязательства и~др. Все характеристики связаны между собой, поэтому можно 
говорить о стоимости и~времени изготовления структурных компонентов системы. 
  
  Одной из важнейших характеристик является смета (система смет для 
подсистем). Смета сопоставляет каждому компоненту системы стоимость его 
изготовления и~настройки. 
  
  Схема построения системы может быть пред\-став\-ле\-на диаграммой, 
изображенной на рис.~1. 

{ \begin{center}  %fig1
 \vspace*{9pt}
   \mbox{%
 \epsfxsize=79mm 
 \epsfbox{gru-1.eps}
 }


\vspace*{9pt}


\noindent
{{\figurename~1}\ \ \small{Диаграмма достижения цели}}
\end{center}
}

\vspace*{9pt}

\addtocounter{figure}{1}
  
  


  Представленная на рис.~1 диаграмма позволяет описать основные классы 
возможных противоречий при достижении цели. Противоречия возникают, когда 
данные БФ не соответствуют требуемым характеристикам. 
  
  
  \section{Потенциальные классы аномалий при~достижении цели}
  
  Выделим четыре потенциальных класса противоречий, которые показывают, 
каким образом нужно искать эти противоречия.
  
 
  Противоречие цели и~проекта (рис.~2) возникает при отсутствии обоснования 
или в~случае логического противоречия между возможностями проектируемого 
функционала и~целью системы. Отметим, что в~проект входят сроки, перечень 
работ, материалы, настройки, которые описываются соответствующими 
параметрами и~допустимыми значениями этих параметров. Проект формируется 
на основе БЗ и~расчетов, исходя из информации, полученной по аналогии 
с~другими проектами и~решениями, которые считаются апробированными. 
  
  Отметим, что цель порождает проект и~в этом смысле является причиной 
проекта. Однако для анализа противоречий необходимо двигаться по штриховой 
стрелке диаграммы (см.\ рис.~2) от проекта к~цели. В~самом деле, любой компонент 
проекта направлен на теоретическое достижение цели. Цель~--- сложный объект, 
поэтому в~проекте могут возникнуть характеристики, противоречащие хотя бы 
некоторым характеристикам цели. Это делает проект противоречивым, но вывод 
об этом может быть сделан только на уровне описания цели. 
  

  Противоречия между проектом и~его реализацией, исключая настройки 
(рис.~3), могут возникать, например, при закупке исполнителем материалов более 
низкого качества по более низким ценам, при попытках достижения требуемых 
сроков работы за счет снижения качества выполнения работ, за счет нахождения 
<<объективных>> причин для увеличения сроков работы и,~следовательно, 
увеличения цены реализации проекта. 


  Для выявления указанных противоречий необходимо двигаться по диаграмме 
(см.\ рис.~3) в~обратную сторону в~соответствии со~штриховыми стрелками. 
Действительно, выявить противоречия между характеристиками закупленных 
материалов и~требуемыми по проекту можно только при обращении к~проекту 
и~его спецификациям. Манипуляции со сроками работы также можно выявить 
только при обращении к~соответствующим расчетам в~проекте. Задержки в~сроках 
работы, связанные с~поставками материалов, можно определить только на 
предыдущем этапе диаграммы (см.\ рис.~3) в~описании проекта. 


  


  Противоречия между реализацией проекта и~его настройкой (рис.~4) возникает, 
когда не удается добиться требуемых значений параметров функционала, не 
удается обеспечить необходимый уровень\linebreak\vspace*{-12pt}

{ \begin{center}  %fig2
 \vspace*{-6pt}
   \mbox{%
 \epsfxsize=16mm 
 \epsfbox{gru-2.eps}
 }


\vspace*{6pt}


\noindent
{{\figurename~2}\ \ \small{Противоречия цели и~проекта}}
\end{center}
}

%\vspace*{9pt}

\addtocounter{figure}{1}

{ \begin{center}  %fig3
 \vspace*{6pt}
    \mbox{%
 \epsfxsize=79mm 
 \epsfbox{gru-3.eps}
 }


\end{center}

\vspace*{-2pt}


\noindent
{{\figurename~3}\ \ \small{Противоречия проекта и~его реализации (без настройки)}}
}

\vspace*{6pt}

\addtocounter{figure}{1}

{ \begin{center}  %fig4
 \vspace*{1pt}
   \mbox{%
 \epsfxsize=54.5mm 
 \epsfbox{gru-4.eps}
 }


\end{center}


\noindent
{{\figurename~4}\ \ \small{Противоречия реализации проекта и~его на\-стройки}}
}

%\vspace*{9pt}

\addtocounter{figure}{1}

{ \begin{center}  %fig5
 \vspace*{5pt}
    \mbox{%
 \epsfxsize=79mm 
 \epsfbox{gru-5.eps}
 }


\end{center}



\noindent
{{\figurename~5}\ \ \small{Противоречия цели и~достигнутой реализации проекта}}
}

\vspace*{6pt}

\addtocounter{figure}{1}

\noindent
 качества реализации проекта. Для 
определения противоречия в~настройках надо опять же двигаться по диаграмме 
(см.\ рис.~4) в~обратную сторону по штриховым стрелкам, так как для выявления 
характеристик результатов работы, которые не дают возможности реализации 
определенного функционала, необходимо иметь информацию о результатах этой 
работы. 


  



  Противоречие между целью и~достигнутой реализацией проекта (рис.~5) 
возникает, когда реализованная система не позволяет достичь цели. В~этом случае 
опять противоречие нужно искать, двигаясь от цели к~реальному достигнутому 
функционалу по штриховой стрелке (см.\ рис.~5).
  
  Суммируя положения, изложенные в~данном разделе, приходим к~выводу, что 
для выявления противоречий необходимо проводить анализ от следствия 
к~причине, т.\,е.\ искать аномалии в~информации, описывающей порождение 
наблюдаемых следствий. 
  
  
  \section{Связь противоречий и~причин}
  
  Прежде чем построить связь между причинами и~противоречиями, кратко 
опишем простейшую модель связи этих понятий. Причины и~противоречия будут 
сформулированы для представления компонентов системы как объектов, 
обладающих наборами известных характеристик~\cite{4-gr, 5-gr}. 
  
  Пусть $U\hm=\{\alpha, \beta, \ldots\}$~--- совокупность характеристик 
(пространство характеристик). Согласно~\cite{4-gr} \textit{объектом}~$O$ 
называется любое подмножество характеристик $O\hm\subseteq U$. Рассмотрим 
последовательность объектов, возможно в~различных пространствах 
характеристик. 
  
  \smallskip
  
  \noindent
  \textbf{Определение~1.}\ Объект~$P$ с~числом характеристик, большим или 
равным~2, является \textit{причиной} объекта (\textit{свойства})~$B$ в~цепочке 
наблюдаемых объектов тогда и~только тогда, когда выполнены следующие 
условия:
  \begin{enumerate}[(1)]
\item для каждого объекта~$C$, если $P\hm\subseteq C$, то $C\mapsto B$, где 
$C\mapsto B$ означает, что объект~$B$ присутствует в~объекте, следующем за 
объектом~$C$;
\item объект~$P$ является минимальным объектом, удовлетворяющим 
условию~1, а~именно: $\forall \alpha\hm\in P$ объект~$P\backslash \{\alpha\}$ 
не является причиной, т.\,е.\ $\exists C:\ \alpha\not\in C$, $P\backslash 
\{\alpha\}\hm\subseteq C$ и~$C\not\mapsto B$, где $C\not\mapsto B$ означает, 
что~$B$ не может содержаться в~объекте, следующем за объектом~$C$. 
\end{enumerate}

  Приведенное определение причины является упрощением причин, 
возникающих в~реальном мире. Например, реальные причины могут возникать\linebreak 
как совокупность характеристик из разных пространств. Одно следствие может 
порождаться разными причинами или возникать из внешних\linebreak и~ненаблюдаемых 
характеристик. Однако пред\-став\-лен\-ная далее формализация позволяет доступно 
изложить при\-чин\-но-след\-ст\-вен\-ные истоки противоречий, которые 
инициируют в~дальнейшем глубокое исследование рассматриваемых процессов.
  
  Будем считать, что для любого интересующего нас свойства~$B$ существует 
причина. Тогда справедлива следующая теорема.
  
  \smallskip
  
  \noindent
  \textbf{Теорема~1.}\ \textit{Для любого свойства~$B$ существует 
единственная причина}. 
  
  \smallskip
  
  \noindent
  Д\,о\,к\,а\,з\,а\,т\,е\,л\,ь\,с\,т\,в\,о\,.\ \ Доказательство будем вести от противного, 
т.\,е.\ предположим, что существуют две причины свойства~$B$: $P$ 
и~$P^\prime$, $P\hm\not= P^\prime$. Тогда существует $\alpha\hm\in U$, которое 
удовлетворяет одному из двух условий:
  \begin{itemize}
\item[(а)] $\alpha\in P$, $\alpha\notin P^\prime$;
\item[(б)] $\alpha\notin P$, $\alpha \in P^\prime$.
\end{itemize}

  Пусть выполняется условие~(б). Тогда $P^\prime\backslash \{\alpha\}$ не 
является причиной по условию~2 определения~1, т.\,е.\ $\exists C$ такое, что 
$\alpha\notin C$, $P^\prime\backslash \{\alpha\}\hm\subseteq C$ и~$C\not\mapsto B$. 
Но если~$B$ произошло и~$P$ его причина, то $C\mapsto B$, что противоречит 
предположению. Теорема~1 доказана.
  
  \smallskip
  
  \noindent
  \textbf{Лемма.} \textit{Если $P$~--- причина появления свойства~$B$, то 
объект~$B$ определяет существование свойства~$P$ в~объекте, 
предшествующем~$B$. }
  
  \smallskip
  
  \noindent
  Д\,о\,к\,а\,з\,а\,т\,е\,л\,ь\,с\,т\,в\,о\,.\ \ Из предположения, что у~каж\-до\-го 
свойства~$B$ есть причина, и~условия, что~$P$ является причиной~$B$, следует, 
что при появлении в~данных свойства~$B$ объект~$C$, предшествующий 
появлению~$B$, содержит как часть объект~$P$. Это следует из теоремы~1 
и~определения причины. 
  
  Докажем принцип <<необходимого условия>>, который, несмотря на простоту 
доказательства, будет играть в~дальнейшем существенную роль.
  
  \smallskip
  
  \noindent
  \textbf{Теорема~2.} \textit{Если~$P$~--- причина появления свойства~$B$ 
и~$A\hm\subseteq P$, то объект~$B$ определяет наличие свойства~$A$ 
в~объекте, предшествующем~$B$}. 
  
  \smallskip
  
  \noindent
  Д\,о\,к\,а\,з\,а\,т\,е\,л\,ь\,с\,т\,в\,о\,.\ \ Пусть в~данных имеется объект~$B$ 
и~$P\mapsto B$, тогда в~силу существования и~единственности причины~$B$ 
в~данных должен существовать объект~$C$, предшествующий~$B$ 
и~содержащий причину~$P$. Поскольку $A\hm\subseteq P$ и~$B$ содержит 
причину~$P$, то $B\mapsto A$. С~учетом леммы теорема~2 доказана.
  
  \smallskip
  
  Пусть даны пространства $U_1, U_2,\ldots$ и~имеется последовательность 
данных (процесс выполнения этапов проекта в~соответствии с~рис.~1) $A, B, 
\ldots$, где каждый объект является подмножеством некоторого 
пространства~$U_i$, $i\hm=1,\ldots$ Тогда в~объекте~$A$ присутствует 
причина~$P$ появления интересующего нас свойства~$C$ в~объекте~$B$. Пусть 
$P\hm\subseteq A$, тогда по теореме~2 $\forall \alpha\hm\in P$:  
$C\mapsto \{\alpha\}$, т.\,е.\ из появления~$C$ следует появление 
характеристики~$\alpha$ в~предшествующем объекте. Это необходимое условие 
того, что~$C$ удовлетворяет причинно-следственным связям развития процесса 
выполнения проекта. Если для~$C$ нет характеристики~$\alpha$, которую можно 
отнести к~причине~$C$, то можно считать, что нарушена  
при\-чин\-но-след\-ст\-вен\-ная связь и~$C$~--- аномальный объект. 
  
  \smallskip
  
  \noindent
  \textbf{Пример.} Если объект~$C$ состоит в~получении суммы~$a$ 
фирмой~$K$, то согласно теореме~2 в~пред\-шест\-ву\-ющем объекте должна 
существовать причина перевода суммы~$a$ на фирму~$K$. Если эта причина 
в~проекте отсутствует, то это можно считать признаком мошеннической схемы. 
Все проекты по предположению собираются из <<кубиков>>, содержащихся в~БЗ. 
Тогда можно сравнить цену объекта~$C$, породившего получение суммы~$a$, 
и~сумму, присутствующую в~смете проекта. Если разница велика, то это либо 
ошибка проекта, либо признак мошеннической схемы.
  
  \section{Поиск противоречий на~основе~принципа <<необходимых~условий>>}
   
  Как было показано в~разд.~3, нахождение противоречий соответствуют 
движению от следствия к~причине. Для каждого объекта в~наблюдаемых данных 
выявление причин его появления является трудоемкой задачей. Кроме того, при 
реализации контроля соблюдения при\-чин\-но-след\-ст\-вен\-ных связей на 
большом множестве участников экономической деятельности задача анализа 
причин становится трудоемкой. Поэтому процедуру контроля необходимо разбить 
на два этапа, где первый этап состоит в~анализе простых <<необходимых 
условий>> проявления мошенничества, когда используется хотя бы одна 
известная характеристика причины. Второй этап (в~режиме офлайн) состоит 
в~выявлении причин, позволяющих провести анализ источников мошеннических 
схем. 
  
  Один из подходов к~выбору <<необходимых условий>> состоит в~построении 
множества подцелей исходной цели проекта (структурный метод построения 
проекта~\cite{7-gr}). Каждая подцель описывается диаграммой на рис.~1, 
и~реализации подцелей должны образовывать полный функционал цели. Это 
является необходимым, но не достаточным условием достижения цели, так как 
при таком подходе отсутствует компонент согласования всех подцелей в~единую 
систему. Однако такой подход значительно упрощает анализ выполнения проекта 
на предмет поиска мошенничества. Если признаки мошенничества будут 
обнаружены в~реализации хотя бы одной из подцелей, то это значит, что 
мошенничество присутствует в~реализации всего проекта. 
  
  Аналогично в~реализации каждого этапа в~любой из подцелей можно выделять 
простые <<необходимые условия>> нарушения при\-чин\-но-след\-ст\-венн\-ых 
связей. 
  
  Таким образом, получается множество <<необходимых условий>>, нарушение 
которых свидетельствует о наличии мошенничества. Это множество 
<<необходимых условий>> можно назвать метаданными~[8, 9] для контроля 
проекта на выявление мошенничества. 
  
  
  \section{Заключение }
  
  В поиске противоречий необходимо от транзакций, соответствующих 
следствиям при\-чин\-но-след\-ст\-вен\-ных связей, переходить к~анализу причин 
наблюдаемых следствий. Это сложная задача, которая связана с~описанием причин 
определенных свойств. 
  
  В работе представлена модель, позволяющая строить множество необходимых 
условий соответствия наблюдаемого следствия вызвавшей его причине. Этот 
подход делает поиск противоречий вполне вычислимой задачей, но не гарантирует 
успех. 
  
  {\small\frenchspacing
 {%\baselineskip=10.8pt
 \addcontentsline{toc}{section}{References}
 \begin{thebibliography}{9}
\bibitem{1-gr}
\Au{Грушо А.\,А., Зацаринный~А.\,А., Тимонина~Е.\,Е.} Блокчейны цифровой экономики на базе 
системы ситуационных центров и~централизованного консенсуса~// Радиолокация, навигация, 
связь: Мат-лы XXV Междунар. научн.-технич. конф.~---
Воронеж: Издательский дом ВГУ, 2019. Т.~6. С.~183--191. 
\bibitem{2-gr}
\Au{Grusho A., Zatsarinny~A., Timonina~E.} A~system approach to information security in 
distributed ledgers on the situational centers platform.~---
Lecture notes in computer science ser.~--- Springer, 2019 
(in press).
\bibitem{3-gr}
\Au{Финн В.\,К.} Искусственный интеллект: Методология, применения, философия.~--- М.: 
Красанд, 2011. 448~с.

\bibitem{5-gr} %4
\Au{Аншаков~О.\,М., Фабрикантова~Е.\,Ф.} ДСМ-ме\-тод автоматического порождения 
гипотез: Логические и~эпистемологические основания.~--- М.: Либроком, 2009. 432~с.

\bibitem{4-gr} %5
\Au{Poelmans J., Elzinga~P., Viaene~S., Dedene~G.} Formal concept analysis in knowledge 
discovery: A~survey~// Conceptual structures: From information to intelligence~/ Eds.\ M.~Croitoru, 
S.~Ferr$\acute{\mbox{e}}$, and D.~Lukose.~--- Lecture notes in computer science 
ser.~--- Berlin--Heidelberg: Springer, 2010. Vol.~6208.  P.~139--153.

\bibitem{6-gr}
\Au{Панкратова~Е.\,С., Финн~В.\,К.} Автоматическое по\-рож\-де\-ние гипотез в~интеллектуальных 
системах.~--- М.: Либроком, 2009. 528~с. 
\bibitem{7-gr}
\Au{Денисов А.\,А., Колесников~Д.\,Н.} Теория больших систем управления.~--- Л.: Энергоиздат, 1982. 488~с.

\bibitem{9-gr}
\Au{Грушо А.\,А., Грушо Н.\,А., Забежайло~М.\,И., Смирнов~Д.\,В., Тимонина~Е.\,Е.} 
Параметризация в~прикладных задачах поиска эмпирических причин~// Информатика и~её 
применения, 2018. Т.~12. Вып.~3. С.~62--66.

\bibitem{8-gr}
\Au{Грушо А.\,А., Грушо Н.\,А., Левыкин~М.\,В., Тимонина~Е.\,Е.} Методы идентификации 
захвата хоста в~распределенной ин\-фор\-ма\-ци\-он\-но-вы\-чис\-ли\-тель\-ной сис\-те\-ме, 
защищенной с~помощью метаданных~// Информатика и~её применения, 2018. Т.~12. Вып.~4. 
С.~41--45.

 \end{thebibliography}

 }
 }

\end{multicols}

\vspace*{-3pt}

\hfill{\small\textit{Поступила в~редакцию 03.04.19}}

%\vspace*{8pt}

%\pagebreak

\newpage

\vspace*{-28pt}

%\hrule

%\vspace*{2pt}

%\hrule

%\vspace*{-2pt}

\def\tit{ARCHITECTURAL DECISIONS IN~THE~PROBLEM 
OF~IDENTIFICATION OF~FRAUD IN~THE~ANALYSIS 
OF~INFORMATION FLOWS IN~DIGITAL ECONOMY\\[-5pt]}


\def\titkol{Architectural decisions in~the~problem 
of~identification of~fraud in~the~analysis 
of~information flows in~digital economy}

\def\aut{A.\,A.~Grusho, M.\,I.~Zabezhailo, N.\,A.~Grusho, and~E.\,E.~Timonina}

\def\autkol{A.\,A.~Grusho, M.\,I.~Zabezhailo, N.\,A.~Grusho, and~E.\,E.~Timonina}

\titel{\tit}{\aut}{\autkol}{\titkol}

\vspace*{-13pt}


 \noindent
   Institute of Informatics Problems, Federal Research Center ``Computer Sciences and 
Control'' of the Russian Academy of Sciences; 44-2~Vavilov Str., Moscow 119133, 
Russian Federation

\def\leftfootline{\small{\textbf{\thepage}
\hfill INFORMATIKA I EE PRIMENENIYA~--- INFORMATICS AND
APPLICATIONS\ \ \ 2019\ \ \ volume~13\ \ \ issue\ 2}
}%
 \def\rightfootline{\small{INFORMATIKA I EE PRIMENENIYA~---
INFORMATICS AND APPLICATIONS\ \ \ 2019\ \ \ volume~13\ \ \ issue\ 2
\hfill \textbf{\thepage}}}

\vspace*{3pt}


   
     
   \Abste{An approach to a~research of some types of fraud in digital economy with the usage of relationships of 
cause and effect is formulated. In all types of the considered frauds, the discrepancy between the 
purposes of financial transactions and actual cost of achievement of these purposes
has to be observed. Data on 
transactions can be collected by observing information flows in which these transactions are reflected. 
The architecture of data collection and their analysis can be organized by means of the distributed 
ledgers with the centralized consensus that allows creating an analog of the electronic account book 
fixing financial and economic activity of subjects of digital economy in the region. 
   The methods of fraud identification considered are based on the contradictions 
between actions described in transactions and information, which is contained in plans, standards, 
precedents, etc. 
   The method based on a~simplified scheme of implementation of the abstract project is considered. 
For identification of contradictions, it is necessary to carry out the analysis from the effect to the cause, 
i.\,e., to look for anomalies in information describing the generation of the observed effects. 
   It is shown how in implementation of the project it is possible to allocate simple ``necessary 
conditions'' of violation of cause and effect relationships, i.\,e., a~set of ``necessary conditions'' 
violation of which demonstrates fraud existence. It is possible to call this set of "necessary conditions" 
by metadata for control of the project for fraud identification.} 
   
   \KWE{digital economy; information flows; relationships of reason and effect; detection of 
fraudulent schemes}
   
  

 \DOI{10.14357/19922264190204}

\vspace*{-20pt}

 \Ack
   \noindent
   The work was partially supported by the Russian Foundation for Basic Research (projects  
18-29-03081 and 18-07-00274).



%\vspace*{6pt}

  \begin{multicols}{2}

\renewcommand{\bibname}{\protect\rmfamily References}
%\renewcommand{\bibname}{\large\protect\rm References}

{\small\frenchspacing
 {\baselineskip=10.5pt
 \addcontentsline{toc}{section}{References}
 \begin{thebibliography}{9}
\bibitem{1-gr-1}
\Aue{Grusho, A.\,A., A.\,A.~Zatsarinny, and E.\,E.~Timonina.} 2019. Blokcheyny tsifrovoy ekonomiki 
na baze sistemy situatsionnykh tsentrov i~tsentralizovannogo konsensusa [Blockchains of digital 
economy on the basis of the system of the situational centres and the centralized consensus]. 
\textit{25th Scientific and Technical Conference (International) ``Radar-Location, Navigation, 
Communication'' Proceedings}. Voronezh: VSU Publs. 6:183--191.
\bibitem{2-gr-1}
\Aue{Grusho, A., A.~Zatsarinny, and E.~Timonina.} 2019 (in press). 
A~system approach to information security 
in distributed ledgers on the situational centers platform. 
Lecture notes in computer science ser. Springer.
\bibitem{3-gr-1}
\Aue{Finn, V.\,K.} 2011. \textit{Iskusstvennyy intellekt: Metodologiya, primeneniya, filosofiya} 
[Artificial intelligence: Methodology, applications, philosophy]. Moscow: KRASAND. 448~p.

\bibitem{5-gr-1}
\Aue{Anshakov, O.\,M., and E.\,F.~Fabrikantova}. 2009. \textit{DSM-metod avtomaticheskogo porozhdeniya gipotez: Logicheskie 
i~epistemologicheskie osnovaniya} [JSM-method of automatic hypothesis generation: Logical and 
epistemological]. Moscow: KD LIBROKOM. 432~p.
\bibitem{4-gr-1} %5
\Aue{Poelmans, J., P.~Elzinga, S.~Viaene, and G.~Dedene.} 2010. Formal concept analysis in 
knowledge discovery: A~survey. \textit{Conceptual structures: From information to intelligence}. 
Eds.\ M.~Croitoru, S.~Ferr$\acute{\mbox{e}}$, and D.~Lukose. Lecture notes in 
computer science ser. Berlin--Heidelberg: Springer. 6208:139--153.

\bibitem{6-gr-1}
\Aue{Pankratov, E.\,S., and V.\,K.~Finn}. 
2009. \textit{Avtomaticheskoe porozhdenie gipotez v~intellektual'nykh 
sistemakh} [Automatic hypotheses generation in intelligent systems]. Moscow: KD 
\mbox{LIBROKOM}.  528~p. 
\bibitem{7-gr-1}
\Aue{Denisov, A.\,A., and D.\,N.~Kolesnikov.} 1982. \textit{Teoriya bol'shikh 
sistem upravleniya} [Theory of big control systems]. Leningrad: Energoizdat. 488~p.

\bibitem{9-gr-1}
\Aue{Grusho, A.\,A., N.\,A.~Grusho, M.\,I.~Zabezhailo, D.\,V.~Smirnov, and 
E.\,E.~Timonina.} 2018. 
Parametrizatsiya v~prikladnykh zadachakh poiska empiricheskikh prichin 
[Parametrization in applied 
problems of search of the empirical reasons]. 
\textit{Informatika i~ee Primeneniya~--- 
Inform. Appl.} 12(3):62--66.

\bibitem{8-gr-1}
\Aue{Grusho, A.\,A., N.\,A.~Grusho, M.\,V.~Levykin, and E.\,E.~Timonina.} 2018. Metody 
identifikatsii zakhvata khosta v~raspredelennoy informatsionno-vychislitel'noy sisteme, 
zashchishchennoy s~pomoshch'yu metadannykh [Methods of identification of host capture 
in the  distributed information system which is protected on the base of meta data].
\textit{Informatika i~ee 
Primeneniya~--- Inform. Appl.} 12(4):41--45.
{ %\looseness=1

}

\end{thebibliography}

 }
 }

\end{multicols}

\vspace*{-12pt}

\hfill{\small\textit{Received April 3, 2019}}

%\pagebreak

%\vspace*{-18pt}

\Contr

\noindent
\textbf{Grusho Alexander A.} (b.\ 1946)~--- Doctor of Science in physics and 
mathematics, professor, principal scientist, Institute of Informatics Problems, 
Federal Research Center ``Computer Sciences and Control'' of the Russian 
Academy of Sciences; 44-2~Vavilov Str., Moscow 119133, Russian Federation; 
\mbox{grusho@yandex.ru} 

\vspace*{3pt}

\noindent
\textbf{Zabezhailo Michael I.} (b.\ 1956)~--- Doctor of Science in physics and 
mathematics, principal scientist, Institute of Informatics Problems, Federal Research 
Center ``Computer Sciences and Control'' of the Russian Academy of Sciences;  
44-2~Vavilov Str., Moscow 119133, Russian Federation; 
\mbox{m.zabezhailo@yandex.ru} 

\vspace*{3pt}


\noindent
\textbf{Grusho Nikolai A.} (b.\ 1982)~--- Candidate of Science (PhD) in physics 
and mathematics, senior scientist, Institute of Informatics Problems, Federal 
Research Center ``Computer Sciences and Control'' of the Russian Academy of 
Sciences; 44-2~Vavilov Str., Moscow 119133, Russian Federation; 
\mbox{info@itake.ru} 

\vspace*{3pt}


\noindent
\textbf{Timonina Elena E.} (b.\ 1952)~--- Doctor of Science in technology, 
professor, leading scientist, Institute of Informatics Problems, Federal Research 
Center ``Computer Sciences and Control'' of the Russian Academy of Sciences;  
44-2~Vavilov Str., Moscow 119133, Russian Federation; 
\mbox{eltimon@yandex.ru} 

\label{end\stat}

\renewcommand{\bibname}{\protect\rm Литература}    %1 
\def\stat{pavlov}

\def\tit{РАСЧЕТ И ОПТИМИЗАЦИЯ НЕКОТОРЫХ ХАРАКТЕРИСТИК 
ДЛЯ~МОДЕЛИ ВЫЧИСЛИТЕЛЬНОГО КОМПЛЕКСА}

\def\titkol{Расчет и оптимизация некоторых характеристик 
для модели вычислительного комплекса}

\def\autkol{И.\,В.~Павлов}
\def\aut{И.\,В.~Павлов$^1$}

\titel{\tit}{\aut}{\autkol}{\titkol}

%{\renewcommand{\thefootnote}{\fnsymbol{footnote}}\footnotetext[1]
%{Работа выполнена при поддержке РФФИ (гранты 09-07-12098, 09-07-00212-а и
%09-07-00211-а) и Минобрнауки РФ (контракт №\,07.514.11.4001).}}


\renewcommand{\thefootnote}{\arabic{footnote}}
\footnotetext[1]{Московский государственный технический университет им.\ Н.\,Э.~Баумана, 
ipavlov@bmstu.ru}

\Abst{Рассматривается проблема выбора оптимального размера пакетов при обработке 
информационных задач большого объема для модели вычислительного комплекса с 
учетом возможных отказов или сбоев элементов в процессе решения задачи. Получено 
приближенное асимптотическое решение данной проблемы для случая высоконадежных 
элементов и малого времени пересылки (загрузки) пакетов.}

\KW{оптимальный размер пакета; надежность; интенсивность отказов; время пересылки 
пакетов} 

\vskip 14pt plus 9pt minus 6pt

      \thispagestyle{headings}

      \begin{multicols}{2}

            \label{st\stat}

\section{Введение}

     Пусть имеется система, включающая в себя $l$ основных 
вычислительных элементов. В~систему поступают <<задания>>, каждое из 
которых состоит из некоторого (вообще говоря, случайного) числа   
<<элементарных задач>>, каждая из которых может выполняться 
(обрабатываться) независимо от остальных на любом из этих элементов. Для 
выполнения очередного задания, поступившего в систему, необходимо 
выполнить все составляющие его элементарные задачи. При этом в процессе 
выполнения задание разбивается на некоторое количество $n$  блоков 
(<<пакетов>>) элементарных задач равного объема $\upsilon\hm=L/n$, 
$n\hm\in N$, где $N$~--- множество допустимых значений  (например, 
$N$~---  некоторое подмножество целочисленных значений, кратных~2 
и~т.\,п.). Время~$h$ выполнения одной элементарной задачи на любом из 
элементов далее будем считать равным единице: $h\hm=1$. Соответственно, 
время выполнения одного пакета объемом~$\upsilon$ на любом из элементов 
будет численно совпадать с величиной~$\upsilon$. 
{\looseness=1

}
     
     Выполнение задания происходит путем пересылки пакетов на рабочие 
элементы и дальнейшей их обработки на этих элементах. Время пересылки 
(загрузки) пакета на элемент равно величине $\tau\hm>0$, не зависит от 
размера пакета~$\upsilon$ и от состояния других элементов. Обработка пакета 
после его загрузки на данном элементе занимает время~$\tau$ и происходит 
независимо от состояния других элементов. После завершения обработки 
очередного пакета на том или ином элементе снова происходит его загрузка в 
течение времени~$\tau$ следующим пакетом (из общей очереди всех пакетов 
данного задания) независимо от состояния (работы или загрузки) остальных 
элементов и~т.\,д. Задание считается выполненным после выполнения 
(обработки) всех составляющих его пакетов. Близкие по смыслу модели и 
процессы рассматривались ранее в~[1--5].
     
     В процессе работы любой из элементов может отказывать с постоянной 
(не зависящей от времени) функцией интенсивности отказов 
$\lambda(t)\hm\equiv \lambda$~[6, 7]. Заметим, что более близким к 
реальности было бы предположение о монотонном возрастании 
(неубывании) $\lambda(t)$ по времени. Поэтому фактически здесь 
предполагается, что, по крайней мере в течение времени выполнения одного 
задания, функция интенсивности отказов~$\lambda(t)$ меняется 
незначительно и может считаться приближенно постоянной. Такое 
допущение является естественным, по крайней мере в случае высокой 
надежности элементов, когда вероятность отказа элемента за время 
выполнения в системе одного задания достаточно мала. В~указанных 
допущениях время безотказной работы элемента имеет экспоненциальное 
распределение с функцией надежности $P(t)\hm=e^{-\lambda t}$, а вероятность 
отказа элемента за время~$h$ выполнения одной элементарной задачи равна 
величине $\lambda h\hm+ o(\lambda h)$.
     
     Одной из существенных проблем, возникающих в данной ситуации, 
является выбор оптимального размера пакета~$\upsilon$ с учетом 
возможности отказов (сбоев) элементов при выполнении задания.

\section{Модель со сбоями элементов}

     Рассмотрим случай, когда возможные отказы элементов в системе 
имеют характер <<сбоев>>. Другими словами, в результате отказа (сбоя) 
элемент сам по себе не выходит из строя и продолжает работать, но 
находящийся на нем в момент сбоя пакет считается невыполненным и после 
завершения его обработки снова ставится в очередь необработанных пакетов 
и должен быть полностью обработан заново на этом же или любом другом 
элементе.
     
     Рассмотрим сначала более простой частный случай, когда число 
элементов $l\hm=1$. Обозначим через $p\hm=\exp (-\lambda \upsilon)$ 
вероятность обработки пакета объемом~$\upsilon$ без сбоев и $q\hm=1-p$. 
Время~$\eta$ выполнения всего задания объемом~$L$ имеет вид:
     \begin{equation}
     \eta=(\upsilon+\tau) v\,,
     \label{e1p}
     \end{equation}
где $v$~--- момент (номер шага) первого достижения $n$ <<успехов>> в 
классической схеме независимых ис\-пытаний Бернулли при вероятности 
<<успеха>> (на\linebreak
одном шаге) $p\hm=\exp\left( -\lambda \upsilon\right)$. Задача 
выбора оп\-тимального размера пакета~$\upsilon$ далее сводится к 
минимизации математического ожидания E$\eta$ по параметру~$\upsilon$, 
или, учитывая равенство $\upsilon\hm= L/n$, к\linebreak
 минимизации~E$\eta$ по 
переменной $n\hm\in N$, где $n$~---  чис\-ло пакетов, на которое разбивается 
задание. Случайная величина~$v$ имеет распределение Пас\-каля 
\begin{equation}
P\left( v=m\right) = C_{m-1}^{n-1} p^n q^{m-n}\,,\quad m=n, n+1, \ldots ,
\label{e2p}
\end{equation}
с математическим ожиданием E$v=n/p$, откуда с учетом~(\ref{e1p}) следует, 
что выбор оптимального размера пакета сводится к задаче: найти
\begin{equation}
\min \left( L+n\tau\right)\exp\left( \fr{\lambda L}{n}\right)
\label{e3p}
\end{equation}
по $n\in N$. Далее оптимальный размер пакета~$\tilde{\upsilon}$ 
находится по формуле $\tilde{\upsilon} =L/\tilde{n}$, где $\tilde{n}\hm\in 
N$~--- решение задачи~(\ref{e3p}). 
     
     Оптимизационная задача~(\ref{e3p}) является цело\-чис\-лен\-ной, 
поскольку множество $N$ допустимых значений $n$ содержит только 
целочисленные точки. Введем также дополнительную <<непрерывную>> 
задачу: найти
     \begin{equation}
     \min\left( L+n\tau\right) \exp \left( \fr{\lambda L}{n}\right)
     \label{e4p}
     \end{equation}
по всем (не только целочисленным) значениям $n\hm\geq 1$. Далее 
оптимальный размер пакета~$\upsilon^*$ (без ограничения целочисленности 
$n\hm\in N$) находится как $\upsilon^*=L/n^*$, где $n^*$~--- решение 
задачи~(\ref{e4p}).
     
Теорема~1 дает точное решение оптимизационных 
задач~(\ref{e3p}) и~(\ref{e4p}). Теорема~2 дает асимптотическое выражение 
для оптимального размера пакета~$\upsilon^*$.
     
     \medskip
     
     \noindent
     \textbf{Теорема 1.} \textit{Пусть $\lambda\hm>0$, $\tau\hm>0$ и 
выполняется неравенство}
     \begin{equation}
     \tau\leq \lambda L^2\,.
     \label{e5p}
     \end{equation}
\textit{Тогда минимум}~(\ref{e4p}) \textit{достигается в единственной \mbox{точке}}
$$
n^*=L\sqrt{\fr{\lambda}{\tau}}\left[  
\sqrt{1+\fr{\lambda\tau}{4}}+\fr{\sqrt{\lambda\tau}}{2}\right]\,.
$$
\textit{Минимум}~(\ref{e3p}) \textit{достигается в одной из двух ближайших 
(слева или справа) к точке~$n^*$ целочисленных точек $n\hm\in N$.} 

     \smallskip
     
     \noindent
     Д\,о\,к\,а\,з\,а\,т\,е\,л\,ь\,с\,т\,в\,о\,.\ Введем функцию
     \begin{equation}
     f(n) =\left( L+n\tau\right) \exp\left( \fr{\lambda L}{n}\right)
     \label{e6p}
     \end{equation}
от непрерывного аргумента $n\hm\geq 1$. Нетрудно показать, что знак 
производной этой функции совпадает со знаком многочлена 
$Q(n)\hm=n^2\hm-\lambda L n -\lambda L^2/\tau$, который имеет при 
$n\hm\geq 1$ единственный корень в точке $n\hm=n^*$ и для которого 
справедливы неравенства:
\begin{align*}
Q(n)<0 &\ \ \mbox{при}\ \ 1\leq n\leq n^*\,;\\
Q(n)>0 &\  \ \mbox{при}\ \ n>n^*\,,
\end{align*}
если выполняется условие~(\ref{e5p}), откуда далее и следует теорема~1. 
Теорема доказана.

\smallskip

     \noindent
     \textbf{Теорема~2.} \textit{Пусть $\lambda\hm>0$, $\tau\hm>0$, 
$\tau\hm\leq \lambda L^2$ и $\lambda\tau\hm\rightarrow 0$. Тогда} 

\noindent
     \begin{equation}
     \upsilon^*=\sqrt{\fr{\tau}{\lambda}}\left[ 1+o(1)\right]\,.
     \label{e7p}
     \end{equation}
     
     \smallskip
     
     \noindent
     Д\,о\,к\,а\,з\,а\,т\,е\,л\,ь\,с\,т\,в\,о\ следует из теоремы~1 и равенства 
$\upsilon^*\hm=L/n^*$. 
     
     \medskip
     
     Из~(\ref{e7p}) далее следует приближенная формула для оптимального 
размера пакета при $\lambda\tau\hm\ll 1$:
     $$
     \upsilon^*\cong \sqrt{\tau\theta}\,,
     $$
где $\theta=1/\lambda$~--- математическое ожидание \mbox{времени} безотказной 
работы (средний ресурс) элемента. Другими словами, оптимальный размер 
пакета~$\upsilon^*$ приближенно равен среднему геометрическому между 
временем пересылки (загрузки)~$\tau$ и средним ресурсом элемента~$\theta$ 
(при условии $\lambda\tau\hm\ll 1$). Существенно, что оптимальное 
значение~$\upsilon^*$ не зависит от размера всего задания~$L$, который, 
вообще говоря, может быть неизвестным и случайным.
     
     Рассмотрим далее общий случай $l\hm\geq 1$ элементов.
Для рассматриваемой модели время выполнения задания

\noindent
     \begin{equation}
     \eta=\left(\upsilon+\tau\right) \left(\fr{v}{l}\right)^+\,,
     \label{e8p}
     \end{equation}
где $z^+$~---  величина~$z$, округленная вверх до ближайшего целого. 
Задача сводится к вычислению
\begin{equation}
\min E\eta
\label{e9p}
\end{equation}
по $n\in N$, после чего оптимальный размер пакета~$\tilde{\upsilon}$ находится 
по формуле $\tilde\upsilon\hm=L/\tilde{n}$, где $\tilde{n}\hm\in N$~--- решение 
задачи~(\ref{e9p}). 

\pagebreak
     
     В соответствии с~(\ref{e2p}) и (\ref{e8p})
     $$
     E\eta =\left( \fr{L}{n}+\tau\right) \sum\limits_{m=n}^\infty 
     \left (\fr{m}{l}\right)^+ C_{m-1}^{n-
1} p^n q^{m-n}\,,
     $$
откуда, учитывая, что $m C_{m-1}^{n-1}=nC_m^n$,
\begin{multline}
E\eta = \left( \fr{L}{n}+\tau\right) \sum\limits_{m=n}^\infty 
\left (\fr{m}{l}\right)^+ \fr{n}{m}\,C_m^n 
p^n q^{m-n}={}\\
{}=\left( \fr{1}{l}\right) \left( L+n\tau\right) p^n \sum\limits_{m=n}^\infty 
\fr{(m/l)^+}{m/l}\,C_m^n q^{m-n}\,.
\label{e10p}
\end{multline}

В соответствии с~(\ref{e2p}) 
\begin{equation}
E v = \!\sum\limits_{m=n}^\infty m C_{m-1}^{n-1} p^n q^{m-n} =n p^n\! 
\sum\limits_{m=n}^\infty C_m^n q^{m-n}.
\label{e11p}
\end{equation}
С другой стороны, случайная величина~$v$ является суммой~$n$ 
независимых, одинаково распределенных случайных величин, каждая из 
которых имеет геометрическое распределение с параметром~$p$ и 
математическим ожиданием $1/p$. Соответственно, $Ev \hm= n/p$, откуда с 
учетом~(\ref{e11p}) следует 
\begin{equation}
\sum\limits_{m=n}^\infty C_m^n q^{m-n} =\fr{1}{p^{n+1}}\,.
\label{e12p}
\end{equation}
Из~(\ref{e10p}) и~(\ref{e12p}) следует
$$
E\eta = \fr{1}{l}\left( L+n\tau\right) p^n \sum\limits_{m=n}^\infty \left[ 
1+\fr{(m/l)^\prime}{m/l}\right] C_m^n q^{m-n}\,,
$$
где $z^\prime=z^+-z$, откуда
\begin{equation}
E\eta =\fr{1}{l}\left(L+n\tau\right)\exp \left( \fr{\lambda L}{n}\right) 
\left(1+\delta_l(n)\right)\,,
\label{e13p}
\end{equation}
где 
\begin{equation}
\delta_l(n)=\sum\limits_{m=n}^\infty \alpha_{nm} \fr{(m/l)^\prime}{m/l}\,,
\label{e14p}
\end{equation}
где коэффициенты $\alpha_{nm} =p^{n+1} C_m^n q^{m-n}$, $p\hm= e^{-
\lambda L/n}$, $q\hm=1\hm-p$. При этом в соответствии с~(\ref{e12p}) 
\begin{equation}
\sum\limits_{m=n}^\infty \alpha_{nm}=1\,.
\label{e15p}
\end{equation}
Из~(\ref{e14p}) и (\ref{e15p}) видно, что 
\begin{equation}
0<\delta_l(n)<\fr{l}{n}\,.
\label{e16p}
\end{equation}
     
     Целевая функция~(\ref{e13p}) для общего случая $l\hm\geq 1$ 
совпадает с целевой функцией в~(\ref{e3p}) для случая $l\hm=1$ с точностью 
до множителя $(1/l)\left[ 1+\delta_l(n)\right]$, откуда с учетом~(\ref{e16p}) 
видно, что полученное выше решение для случая $l\hm=1$ практически дает 
и решение для случая $l\hm>1$, если оптимальное число пакетов $n$ 
достаточно велико.
     
     Обозначим через
     \begin{equation}
     f_l(n) =\fr{f(n)}{l}\left[ 1+\delta_l(n)\right]
     \label{e17p}
     \end{equation}
целевую функцию~(\ref{e13p})~--- среднее время выполнения задания при 
данных значениях $n$~--- чис\-ле пакетов и $l$~--- чис\-ле элементов, где 
$f(n)\hm=(L+n\tau)\exp\left(\lambda L/n\right)$~--- целевая функция~(\ref{e6p}) 
для случая $l\hm=1$.
     
     Задача выбора оптимального размера пакета~$\tilde{\upsilon}$ сводится к 
нахождению 
     \begin{equation}
     \min f_l(n) =f_l\left(\tilde{n}_l\right)\,.
     \label{e18p}
     \end{equation}
Здесь минимум берется по всем $n\hm\in N$, где $N$~--- множество 
допустимых значений~$n$ (например, $N$~--- множество це\-ло\-чис\-лен\-ных 
значений~$n$, кратных~2, лежащих в некотором допустимом диапазоне, 
и~т.\,п.). Полагаем $\tilde{\upsilon}\hm=L/\tilde{n}_l$, где 
$\tilde{n}_l$~--- решение задачи~(\ref{e18p}). Введем также дополнительную 
задачу нахождения 
\begin{equation}
\min f_l(n) =f_l(n_l^*)\,,
\label{e19p}
\end{equation}
где минимум берется по всем (не только це\-ло\-чис\-лен\-ным) значениям 
$n\hm\geq 1$. Далее оптимальный размер пакета~$\upsilon_l^*$ (без 
ограничений це\-ло\-чис\-лен\-ности $n\hm\in N$) определим по формуле 
$\upsilon_l^*\hm=L/n_l^*$, где $n_l^*$~--- решение задачи~(\ref{e19p}). Из 
выражений~(\ref{e13p})--(\ref{e16p}) далее следует теорема~3.

\medskip

\noindent
\textbf{Теорема~3.} \textit{Решение оптимизационной задачи}~(\ref{e19p}) 
\textit{удовлетворяет неравенствам}
\begin{equation}
\fr{f_l(n^*)}{1+\varepsilon}\leq \min\limits_{n\geq 1} f_l(n)\leq f_l(n^*)\,,
\label{e20p} 
\end{equation}
\textit{где $n^*$~--- решение этой задачи для случая $l\hm=1$, 
$\varepsilon\hm=\delta_l(n^*)\hm<l/n^*$. Решение оптимизационной 
задачи}~(\ref{e18p}) \textit{удовлетворяет аналогичным неравенствам}
\begin{equation}
\fr{f_l(\tilde{n})}{1+\varepsilon}\leq \min\limits_{n\in N} f_l(n)\leq 
f_l(\tilde{n})\,, 
\label{e21p}
\end{equation}
\textit{где $\tilde{n}$~--- решение этой задачи для случая} $l\hm=1$, 
$\varepsilon\hm=\delta_l(\tilde{n})\hm<l/\tilde{n}$.
     
     \medskip
     
\noindent
     Д\,о\,к\,а\,з\,а\,т\,е\,л\,ь\,с\,т\,в\,о\,.\ Равенство~(\ref{e17p}) при 
$n\hm=n^*$ имеет вид:
     \begin{equation}
     f_l(n^*) =\fr{f(n^*)}{l}\left[ 1+\delta_l(n^*)\right]\,.
     \label{e22p}
     \end{equation}
Из этого же равенства, учитывая, что $\delta_l(n)\hm>0$, следует
$$
f_l(n)\geq \fr{f(n)}{l}\,,
$$
откуда с учетом~(\ref{e22p})
$$
\min\limits_{n\geq 1} f_l(n) \geq \fr{1}{l}\min\limits_{n\geq 1} f(n) 
=\fr{f(n^*)}{l}= \fr{f_l(n^*)}{1+\delta_l(n^*)}\,,
$$
что вместе с~(\ref{e16p}) доказывает левое неравенство в~(\ref{e20p}). 
Правое неравенство очевидно. Доказательство неравенств~(\ref{e21p}) 
аналогично. Теорема доказана.

\medskip

     Таким образом, полученное решение для случая $l\hm=1$ 
практически дает решение и в случае $l\hm>1$, если число пакетов много 
больше по сравнению с количеством элементов~$l$.

\section{Заключение}
     
     Получено решение указанной выше основной проблемы (выбора 
оптимального размера пакета) для модели со сбоями элементов в 
естественной с прикладной точки зрения асимптотике, а именно для случая 
высоконадежных элементов и при малом времени пересылки пакетов. 
Существенно, что полученное решение не зависит от общего объема всего 
задания, что, в частности, позволяет использовать его в ситуации 
неопределенности, когда эта величина, вообще говоря, может быть 
неизвестной и случайной. Отметим также, что представляет интерес 
дальнейшее обобщение полученных результатов на ситуацию, когда 
различные элементы могут иметь существенно различные характеристики 
как производительности, так и надежности, а также на модель с отказами и 
восстановлением (заменой) отказавших элементов. 

{\small\frenchspacing
{%\baselineskip=10.8pt
\addcontentsline{toc}{section}{Литература}
\begin{thebibliography}{9}


\bibitem{3p} %1
\Au{Ронжин А.\,Ф., Суриков В.\,Н.}
О~времени полного перебора~// Обозр. прикл. пром. матем., 2007. Т.~14. 
№\,3. С.~506--508.

\bibitem{4p} %2
\Au{Коновалов М.\,Г., Малашенко Ю.\,Е., Назарова~И.\,А.}
Модели и методы управления заданиями в системах распределенных 
вычислительных ресурсов.~--- М.: ВЦ РАН, 2009. 110~с. (Сообщения по 
прикладной математике.)

\bibitem{5p} %3
\Au{Коновалов М.\,Г., Малашенко Ю.\,Е., Назарова~И.\,А.}
Оперативное управление потоком заданий в системе распределенных 
вычислительных ресурсов~// VI Московская междунар. конф. по 
исследованию операций: ORM-2010: Труды.~--- М.: 
МАКС Пресс, 2010. С.~301--302.

\bibitem{6p} %4
\Au{Козлов М.\,В., Малашенко Ю.\,Е., Назарова~И.\,А., Ронжин~А.\,Ф.}
Анализ режимов управления вычислительным комплексом в условиях 
неопределенности.~--- М.: ВЦ РАН, 2011. 63~с. (Сообщения по прикладной 
математике.)

\bibitem{7p} %5
\Au{Коновалов М.\,Г., Малашенко Ю.\,Е., Назарова~И.\,А.}
Управ\-ле\-ние заданиями в гетерогенных вычислительных сис\-те\-мах~// 
Известия РАН. Теория и системы управления, 2011. №\,2. С.~72--90. 

\bibitem{1p} %6
\Au{Гнеденко Б.\,В., Беляев Ю.\,К., Соловьев~А.\,Д.}
Математические методы в теории надежности.~--- М.: Наука, 1965. 524~с. 

\label{end\stat}

\bibitem{2p} %7
\Au{Gnedenko B.\,V., Pavlov I.\,V., Ushakov~I.\,A.}
Statistical reliability engineering.~--- N.Y.: John Wiley, 1999. 514~p.
 \end{thebibliography}
}
}


\end{multicols}  %2
%\newcommand{\A}{{\mathbf A}}
%\newcommand{\B}{{\mathbf B}}
%\newcommand{\la}{{\lambda}}
%\newcommand{\be}{\begin{equation}}
%\newcommand{\ee}{\end{equation}}
%\newcommand{\ber}{\begin{eqnarray}}
%\newcommand{\eer}{\end{eqnarray}}

%\newcommand{\nin}{\noindent}
%\newcommand{\non}{\nonumber}
%\newcommand{\half}{\frac{1}{2}}
%\newcommand{\quarter}{\frac{1}{4}}

\def\stat{zeifman}

\def\tit{ОБ ОДНОМ КЛАССЕ МАРКОВСКИХ СИСТЕМ ОБСЛУЖИВАНИЯ$^*$}

\def\titkol{Об одном классе марковских систем обслуживания}

\def\autkol{Я.\,А.~Сатин, А.\,И.~Зейфман, А.\,В.~Коротышева, С.\,Я.~Шоргин}
\def\aut{Я.\,А.~Сатин$^1$, А.\,И.~Зейфман$^2$, А.\,В.~Коротышева$^3$, С.\,Я.~Шоргин$^4$}

\titel{\tit}{\aut}{\autkol}{\titkol}

{\renewcommand{\thefootnote}{\fnsymbol{footnote}}\footnotetext[1]
{Исследование поддержано РФФИ, гранты 11-07-00112-а и 11-01-12026-офи-м.}}


\renewcommand{\thefootnote}{\arabic{footnote}}
\footnotetext[1]{Вологодский государственный педагогический
университет, yacovi@mail.ru}
\footnotetext[2]{Вологодский государственный педагогический университет;  
Институт проблем информатики Российской академии наук; 
Институт социально-экономического развития территорий Российской академии наук,  a\_zeifman@mail.ru}
\footnotetext[3]{Вологодский государственный педагогический
университет,  a\_korotysheva@mail.ru}
\footnotetext[4]{Институт проблем информатики Российской академии наук, SShorgin@ipiran.ru}


\Abst{Рассматриваются модели обслуживания, описываемые конечными марковскими 
цепями с непрерывным временем. При этом предполагается,  что интенсивности 
поступления и обслуживания требований не зависят от числа требований в сис\-те\-ме. 
Получены оценки скорости сходимости и устойчивости различных характеристик таких сис\-тем.}

\KW{нестационарные марковские системы
обслуживания; скорость сходимости; устойчивость; оценки}

 \vskip 14pt plus 9pt minus 6pt

      \thispagestyle{headings}

      \begin{multicols}{2}
      
            \label{st\stat}

\section{Введение}

Классы систем массового обслуживания, описываемых процессами
рождения и гибели (стационарными и нестационарными, с катастрофами)
изучались начиная с 1970-х~гг.\ многими авторами
(см., например,~[1--6]). С~помощью методов,
разработанных одним из авторов настоящей \mbox{статьи}\linebreak (подробное изложение
этих методов приведено в~[7--9]), для таких сис\-тем
удалось получить точные оценки скорости сходимости и устойчивости.

Оказывается, этот же подход можно применить и к существенно более 
общему классу систем обслуживания.

Рассмотрим систему массового обслуживания, число требований в которой 
описывается нестационарной марковской цепью с непрерывным временем и 
конечным пространством состояний, причем требования могут поступать и 
обслуживаться группами.

Пусть $X=X(t)$, $t\geq 0$,~--- число требований в системе обслуживания ($0 \hm\le X(t) \hm\le r$).

Обозначим через 
\begin{gather*}
p_{ij}(s,t)=\mathrm{Pr}\left\{ X(t)=j\left| X(s)=i\right.
\right\}\,,\\
i,j \ge 0\,,\ 0\leq s\leq t\,,
\end{gather*}
переходные вероятности
процесса $X\hm=X(t)$, а через  $p_i(t)\hm=\mathrm{Pr}\left\{ X(t) \hm=i \right\}$~---
его вероятности состояний.

Будем предполагать, что интенсивности поступления и обслуживания $k$ требований в 
момент~$t$ в сис\-те\-ме об\-слу\-жи\-ва\-ния ($\lambda_{k}(t)$ и  $\mu_{k}(t)$ соответственно)  
не зависят от числа требований, находящихся в системе в момент~$t$, являются локально 
интегрируемыми на $[0,\infty)$ функциями времени~$t$ и, кроме того, 
$\lambda_{k+1}(t) \hm\le \lambda_{k}(t)$ и  $\mu_{k+1}(t) \hm\le \mu_{k}(t)$ при всех~$k$ 
и почти при всех $t \hm\ge 0$.

Тогда для описания вероятностной динамики процесса получаем прямую систему Колмогорова в виде
\begin{equation} 
\fr{d\vp}{dt}=A(t)\vp(t)\,,
\label{ur_1}
\end{equation}
 где
 {\footnotesize
\begin{multline*}
A(t)={}\\
{}=
\begin{pmatrix}
a_{00}(t) & \mu_1(t)  & \mu_2(t)   & \mu_3(t)  & \mu_4(t) & \cdots & \mu_r(t) \\
\la_1(t)   & a_{11}(t)  & \mu_1(t)  & \mu_2(t)   & \mu_3(t)  & \cdots & \mu_{r-1}(t) \\
\la_2(t)  & \la_1(t)    & a_{22}(t)& \mu_1(t)  & \mu2(t)    &  \cdots & \mu_{r-2}(t) \\
\cdots&\cdots&\cdots&\cdots&\cdots&\cdots&\cdots \\
\la_r(t) & \la_{r-1}(t) & \la_{r-2}(t) & \cdots & \la_2(t)  & \la_1 (t)   &  a_{rr}(t)
\end{pmatrix}\,,
\end{multline*}}
причем  
$$
a_{ii}(t)=-\sum\limits_{k=1}^{i}\mu_k(t) - \sum\limits_{k=1}^{r-i} \la_{r-k}(t)\,.
$$

Далее будем обозначать через $\|\bullet\|$  $l_1$-нор\-му, т.\,е.\ 
$\|{\vx}\|\hm=\sum|x_i|$, а $\|B\| \hm= \max\limits_j \sum\limits_i |b_{ij}|$, 
если $B \hm= (b_{ij})_{i,j=0}^{r}$.
%
Тогда, в частности, имеем 
$$
\|A(t)\| \le 2\sum\limits_{k=1}^{r}(\la_{k}(t)+ \mu_k(t))
$$ 
при  всех $t \hm\ge 0$.

Через 
$$
E(t,k) = E\left\{X(t)\left|X(0)\hm=k\right.\right\}
$$ 
будем далее обозначать математическое ожидание процесса (среднее число требований) в момент~$t$ 
при условии, что в нулевой момент времени он находится в состоянии~$k$, 
а через $E_{\bf p}(t)$ обозначим математическое ожидание процесса в момент~$t$ 
при начальном распределении вероятностей состояний $\mathbf{p}(0) \hm= \mathbf{p}$.

\section{Оценки скорости сходимости}

Рассмотрим вспомогательную последовательность положительных чисел $\{d_i\}$, $i\hm=1, \dots,r$.

Положим
\begin{equation*}
d=\min\limits_{1 \le i \le r} d_i\,; \enskip 
G=\sum\limits_{i=1}^r d_i\,; \enskip W=\min\limits_k \fr{d_k}{k}\,.
%\label{2.01}
\end{equation*}

Рассмотрим величины
\begin{multline*}
\alpha_i(t)= -a_{ii}(t)+\la_{r-i+1}(t)-\sum\limits_{k=1}^{i-1}(\mu_{i-k}(t)-{}\\
{}-
\mu_i(t))\fr{d_k}{d_i}-\sum\limits_{k=1}^{r-i}(\la_k(t)-\la_{i+r-1}(t))\fr{d_{k+i}}{d_i}\,,
%\label{2.02}
\end{multline*}

\noindent
\begin{equation*}
\alpha(t)=\min\limits_{1 \le i \le r}\alpha_i(t)\,.
%\label{2.03}
\end{equation*}

\smallskip

\noindent
\textbf{Теорема~1.} \textit{Пусть существует последовательность положительных 
чисел  $\{d_j\}$ такая, что}
\begin{equation}
\int\limits_0^{\infty} \alpha(t)\, dt = + \infty\,.
\label{2.031}
\end{equation}
\textit{Тогда $X(t)$ слабо эргодичен, при
любых начальных условиях} $\mathbf{p}^*(s)$, $\mathbf{p}^{**}(s)$ 
\textit{и любых $s$, $t$, $0\le s\le t$, справедлива оценка
\begin{equation} 
\label{2.04}
\|\vp^*(t)-\vp^{**}(t)\| \le \fr{8G}{d}\,e^{-\int\limits_s^t {\alpha(u)\,du}}\,.
\end{equation}
Кроме того,  $X(t)$ имеет предельное среднее $\phi(t)$ и при любых~$k$ и $t \hm\ge 0$ справедливо неравенство}:
\begin{equation}
\label{2.05}
|E(t,k)-\phi(t)|\le \fr{4G}{W}\,e^{-\int\limits_0^t {\alpha(u)\,du}}\,.
\end{equation}


\smallskip


\noindent
Д\,о\,к\,а\,з\,а\,т\,е\,л\,ь\,с\,т\,в\,о\,.\

Пользуясь предложенным в предыдущих работах способом, 
выразим 
$$
p_0=1-\sum\limits_{1\le i \le r}{p_i}\,.
$$

Тогда получим неоднородное уравнение:
\begin{equation} 
\label{ur_per}
\fr{d\vz}{dt}= B(t)\vz(t)+\vf(t)\,, 
%\label{2.06}
\end{equation}
\noindent
где $\vf(t)=\left(\la_1, \la_2,\cdots,\la_r \right)^{\mathrm{T}}$;

\end{multicols}


\hrule

\vspace*{6pt}

\begin{equation*}
B = \left(
\begin{array}{cccccccc}
a_{11}- \la_1   & \mu_1 - \la_1   & \mu_2 - \la_1   & \mu_3 -\la_1   & \cdots& \cdots & \mu_{r-1}- \la_1  \\
\la_1 -\la_2    & a_{22} -\la_2  & \mu_1-\la_2   & \mu_2 -\la_2     & \cdots&  \cdots & \mu_{r-2} -\la_2 \\
\la_2 -\la_3    & \la_1 -\la_3   & a_{33} -\la_3  & \mu_1-\la_2   & \cdots&  \cdots & \mu_{r-3} -\la_3 \\
\cdots&\cdots&\cdots&\cdots&\cdots&\cdots&\cdots \\
\la_{r-1} -\la_r  &\la_{r-2} -\la_r & \cdots & \cdots & \la_2 -\la_r   & \la_1 -\la_r     &  a_{rr} -\la_r
\end{array}
\right)\,.
%\label{2.07}
\end{equation*}

Рассмотрим треугольную матрицу
\begin{equation*}
D=\begin{pmatrix}
d_1   & d_1 & d_1 & \cdots & d_1 \\
0   & d_2  & d_2  &   \cdots & d_2 \\
\cdots&\cdots&\cdots&\cdots&\cdots \\
0  & 0 & \cdots & 0 &  d_r
\end{pmatrix}
%\label{2.08}
\end{equation*}
и соответствующую норму $\|{\bf z}\|_{D}\hm=\|D {\bf z}\|_1$.

Тогда имеем:
\begin{equation*}
 D BD^{-1}=\left(
\begin{array}{ccccccc}
a_{11}-\la_r  &  (\mu_1-\mu_2) \fr{d_1}{d_2}  & (\mu_2-\mu_3)\fr{d_1}{d_3}  & \cdots &  (\mu_{r-1}-\mu_r)\fr{d_1}{d_r} \\
(\la_1-\la_r) \fr{d_2}{d_1} &  a_{22}-\la_{r-1}  &(\mu_1-\mu_3)\fr{d_2}{d_3}  & \cdots &  (\mu_{r-2}-\mu_r)\fr{d_2}{d_r} \\
(\la_2-\la_r) \fr{d_3}{d_1} &  (\la_1-\la_{r-1})\fr{d_3}{d_2}   &a_{33}-\la_{r-2}   & \cdots &  (\mu_{r-3}-\mu_r)
\fr{d_3}{d_r}  \\
\cdots&\cdots&\cdots&\cdots&\cdots \\
(\la_{r-1} -\la_r) \fr{d_r}{d_1} & (\la_{r-2} -\la_{r-1}) \fr{d_r}{d_2}  & (\la_{r-3} -\la_{r-2}) \fr{d_r}{d_3}  & \cdots & a_{rr}-\la_1 \\
\end{array}
\right)\,.
%\label{2.09}
\end{equation*}


\begin{multicols}{2}


Далее, оценивая логарифмическую норму оператора~$B(t)$ (см., например, 
подробное рассмотрение в~[8--10]), получаем
\begin{multline*}
\gamma \left(B(t)\right)_{1D} = \gamma \left(DB(t)D^{-1}\right)_{1}={}\\
{}=
\max \left(\vphantom{\sum\limits_{k=1}^{i-1}}
a_{ii}(t) - \la_{r-i+1}(t) + \sum\limits_{k=1}^{i-1}\left(\mu_{i-k}(t)-{}\right.\right.\\
\left.\left.{}-\mu_i(t)\right)
\fr{d_k}{d_i} +
\sum\limits_{k=1}^{r-i}(\la_k(t)-\la_{i+r-1}(t))\fr{d_{k+i}}{d_i}\right) ={}\\
{}=
 - \min \alpha_i(t) = - \alpha(t)\,.
% \label{2.10}
\end{multline*}
Тогда\\[-7.9pt]
\begin{equation*}
\|\vz^*(t)-\vz^{**}(t)\|_{1D}\le  e^{-\int\limits_s^t {\alpha(u)du}}\|\vz^*(s)-\vz^{**}(s)\|_{1D}
%\label{2.11}
\end{equation*}
для всех $0 \le s \le t$ и любых начальных условий $\vz^*(s)$, $\vz^{**}(s)$.

Теперь, учитывая оценки для сравнения норм (см., например,~\cite{z08b}), получаем:
\begin{multline*}
\|\vp^*(t)-\vp^{**}(t)\| \le 2\|\vz^*(t)-\vz^{**}(t)\| \le{}\\
{}\le  \fr{4}{d}\|\vz^*(t)-\vz^{**}(t)\|_{1D}\le{} \\
{} \le \fr{4}{d}\,e^{-\int\limits_s^t {\alpha(u)\,du}}\|\vz^*(s)-\vz^{**}(s)\|_{1D} 
\le{}\\
{}\le
 \fr{4G}{d}\,e^{-\int\limits_s^t {\alpha(u)\,du}}\|\vz^*(s)-\vz^{**}(s)\| \le{} \\
{} \le  \fr{4G}{d}\,e^{-\int\limits_s^t {\alpha(u)\,du}}\|\vp^*(s)-\vp^{**}(s)\| \le 
\fr{8G}{d}\,e^{-\int\limits_s^t {\alpha(u)\,du}} 
%\label{2.11-a}
\end{multline*}
для любых начальных условий ${\bf p^*}(s)$, ${\bf p^{**}}(s)$ и любых $s,t$, $0\hm\le s\hm\le t$.

Из слабой эргодичности процесса с конечным пространством состояний 
вытекает существование предельного среднего, начальные условия для которого можно 
в общем случае выбрать произвольно.
Для оценки средних воспользуемся неравенством, приведенным в параграфе~2.3 из~\cite{z08b}:
\begin{multline*}
\|{\bf z}\|_{1D} = d_0 \left|\sum\limits_{i=1}^{\infty} p_i \right|
+ d_1 \left|\sum\limits_{i=2}^{\infty} p_i \right| + \dots \ge{}\\
{}\ge 
 W \sum\limits_{k \ge 1} k \left|\sum\limits_{i \ge k} p_i\right| \ge \fr{W}{2}
\sum\limits_{k \ge 1} k \left|p_k\right|\,.  
%\label{2.12}
\end{multline*}
Получаем теперь
\begin{multline*}
|E(t,k)-\phi(t)|\le \fr{2}{W}\,\|\vp^*(t)-\vp^{**}(t)\|_{1D}\le {} \\
{}\le\fr{2}{W}\,e^{-\int\limits_0^t {\alpha(u)\,du}}\|{\bf e}_k -
\vp^{**}(0)\|_{1D} \le \frac{4G}{W}e^{-\int\limits_0^t
{\alpha(u)\,du}}\,,
%\label{2.13}
\end{multline*}
что и требовалось доказать.
\columnbreak

%\smallskip

\noindent
\textbf{Замечание~1.} {Положим в условиях теоремы~1 
$$
\beta(t)=\max\limits_{1 \le i \le r}\alpha_i(t)\,.
$$ 
Тогда, пользуясь внедиагональной неотрицательностью матрицы $DB(t)D^{-1}$ 
с помощью методики, описанной в~\cite{z08b, z95b}, получаем справедливость неравенства

\noindent
\begin{equation*} 
%\label{2.14}
\|\vp^*(t)-\vp^{**}(t)\| \ge \fr{d}{8G}\,e^{-\int\limits_s^t {\beta(u)\,du}}
\end{equation*}
при любых $s$, $t$, $0\le s\le t$ и уже не при любых начальных условиях~${\bf p^*}(s)$, 
${\bf p^{**}}(s)$, а таких, что  $D\left({\bf p^*}(s) \hm-{\bf p^{**}}(s)\right) \hm\ge 0.$ 
Следовательно, оценки тео\-ре\-мы~1 будут заведомо иметь точный по времени порядок, если удастся 
выбрать вспомогательную последовательность $\{d_i\}$ так, что $\alpha(t)\hm=\beta(t)$, т.\,е.\ 
все $\alpha_i(t)$ одинаковы (не зависят от индекса~$i$)}.



\smallskip

Введем теперь в рассмотрение величины

\vspace*{-1pt}

\noindent
\begin{multline*}
\zeta_i(t)= -a_{ii}(t)+\la_{r-i+1}(t)+{}\\
{}+\sum\limits_{k=1}^{i-1}\left(\mu_{i-k}(t)-
\mu_i(t)\right) \fr{d_k}{d_i}+{}\\
{}+\sum\limits_{k=1}^{r-i}\left(\la_k(t)-\la_{i+r-1}(t)\right)\fr{d_{k+i}}{d_i}\,;
%\label{2.0211}
\end{multline*}
\begin{equation*}
\chi(t)=\max\limits_{1 \le i \le r}\zeta_i(t)\,.
%\label{2.0311}
\end{equation*}

\noindent
\textbf{Замечание 2.} {В условиях теоремы~1 при любых начальных условиях 
${\bf p^*}(s)$, ${\bf p^{**}}(s)$ и любых $s,t$,  $0\le s\le t$, 
справедлива следующая двухсторонняя оценка скорости сходимости:

\vspace*{-1pt}

\noindent
\begin{multline*} 
%\label{2.041}
\!\!\!\fr{d}{4G}\,e^{-\int\limits_s^t {\chi(u)\,du}}\|\vp^*(s)-\vp^{**}(s)\| \le
 \|\vp^*(t)-\vp^{**}(t)\| \le {}\\
 {}\le\fr{4G}{d}\,e^{-\int\limits_s^t {\alpha(u)\,du}}\|\vp^*(s)-\vp^{**}(s)\|.
\end{multline*}
Таким образом, можно оценить и сверху и снизу время  вхождения 
сис\-те\-мы обслуживания в предельный режим. Более подробно о получении 
нижних оценок см., например, в~\cite{z95b, gz05}.}

\smallskip

Рассмотрим два частных случая теоремы.

\smallskip

\noindent
\textbf{Следствие 1}. \textit{Пусть при выполнении остальных условий теоремы~1 
вместо}~(\ref{2.031}) \textit{выполняется условие $\alpha(t) \hm\ge \alpha \hm> 0$ 
почти при всех $t \hm\ge 0$. Тогда вместо}~(\ref{2.04}) \textit{и}~(\ref{2.05}) 
\textit{справедливы оценки}:

\vspace*{-1pt}

\noindent
\begin{align*} 
%\label{2.15}
\|\vp^*(t)-\vp^{**}(t)\| &\le \fr{8G}{d}\,e^{-\alpha \left(t-s\right)}\,;
\\
%\label{2.16}
|E(t,k)-\phi(t)|&\le \fr{4G}{W}\,e^{- \alpha t}\,.
\end{align*}

\pagebreak

%\smallskip

Положим 
\begin{gather*}
M_0=\max\limits_{|t-s|\le 1}\int\limits_s^t \alpha(u)\,du;\\
\alpha^* = \int\limits_0^1 \alpha(t)\, dt\,; \quad
M=e^{M_0+\alpha^*}\,.
\end{gather*}
С учетом неравенства 
$$
e^{-\int\limits_s^t {\alpha(u)\,du}} \hm\le M e^{-\alpha^* (t-s)}
$$ 
получаем следующее утверждение.

\smallskip

\noindent
\textbf{Следствие~2.} \textit{Пусть все $\lambda_k(t)$ и $\mu_k(t)$ 1-пе\-ри\-одич\-ны,  
а при выполнении остальных условий теоремы~1 вместо}~(\ref{2.031}) 
\textit{выполняется условие  $\alpha^* \hm> 0$.  Тогда предельный режим (скажем, $\vp^*(t)$) 
и соответствующее ему предельное среднее $\phi^*(t)$ можно выбрать 
1-пе\-ри\-оди\-че\-ски\-ми, а вместо}~(\ref{2.04}) \textit{и}~(\ref{2.05}) 
\textit{справедливы оценки}:
\begin{equation*} 
%\label{2.17}
\|\vp(t) - \vp^*(t)\| \le \fr{8GM}{d}\,e^{-\alpha^*t}
\end{equation*}
\textit{и, кроме того,}
\begin{equation*}
|E(t,k)-\phi^*(t)|\le \fr{4GM}{W}\,e^{-\alpha^*t}
%\label{2.18}
\end{equation*}
\textit{при любом $k$ и $t \ge 0$}.



\section{Устойчивость}

Рассмотрим также <<возмущенный>> процесс обслуживания $\bar{X}\hm=\bar{X}(t)$, $t\hm\geq 0$, 
в котором интенсивности поступления и обслуживания требований также не зависят от чис\-ла 
требований в системе, обозначая его соответствующие характеристики теми же буквами с 
чертой сверху. Для прос\-то\-ты записи оценок будем предполагать, что возмущения 
<<равномерно малы>>, т.\,е.\ выполняется неравенство $\| A(t)-\bar{A}(t)\| \hm\le \varepsilon$. 
Первые результаты для нестационарных цепей с непрерывным временем получены в~\cite{z85}, 
а детальное рассмотрение для более общего случая неравномерных оценок можно без труда 
провести так же, как это сделано в~\cite{z98, ae}. Для получения требуемых равномерных 
оценок устойчивости необходима экспоненциальная эргодичность соответствующего процесса, 
т.\,е.\ существование положительных констант $N$, $a$ таких, что  для правой части~(\ref{2.04}) 
справедливо неравенство:
\begin{equation}
e^{-\int\limits_s^t {\alpha(u)\,du}} \le Ne^{-a\left(t-s\right)}\,.
\label{3.01}
\end{equation}
Оценка~(\ref{3.01}) заведомо имеет место, в частности, если выполнены условия одного из следствий 
предыду\-ще\-го параграфа.

\smallskip

\noindent
\textbf{Теорема~2.}
\textit{Пусть выполнены условия теоремы~1 и}~(\ref{3.01}). \textit{Тогда при
 любых начальных условиях ${\bf p}(s)$ и ${\bar{\bf p}}(s)$ для процессов~$X(t)$ 
 и $\bar{X}(t)$ соответственно справедливы следующие оценки устойчивости:}
\begin{align*} 
%\label{3.02}
\limsup_{t \to \infty}  \|{\bf p}(t)- \bar{\bf p}(t)\| &\le
\fr{\varepsilon(1+\ln(4GN/d))}{a}\,;
\\
% \label{3.03}
\limsup\limits_{t \to \infty}   |E_{\bf p}(t)- \bar{E}_{\bar{\bf p}(t)}|&\le 
\fr{r \varepsilon(1+\ln(4GN/d))}{a}\,.
\end{align*}


\smallskip

\noindent
Д\,о\,к\,а\,з\,а\,т\,е\,л\,ь\,с\,т\,в\,о\ основано на подходе, 
введенном для стационарных процессов в~\cite{mit03} и описанном для нестационарной 
ситуации в~\cite{z11}.
Если  при любых начальных условиях для исходного процесса справедлива оценка
\begin{equation*} 
%\label{3.04}
\|\vp(t) - \vp^*(t)\| \le ce^{-b\left(t-s\right)}\,,
\end{equation*}
то, полагая
\begin{multline*}
\beta (t, s)=\sup\limits_{ \| {\bf v} \| =1, \sum {v_i}=0}
{\|V(t,s){\bf v}(t,s)\|} ={}\\
{}= \fr{1}{2} \max_{i,j} \sum\limits_k {|p_{ik}(t,
s)-p_{jk}(t, s)|}\,, 
\end{multline*}
где $V(t, s)$~--- матрица Коши
уравнения~(\ref{ur_1}), получаем в итоге следующее неравенство:
\begin{equation*}
\|{\bf p}(t)-\bar{\bf p}(t)\| \le{}
\begin{cases}
\|{\bf p}(s)-{\bf \bar{p}}(s)\|+ (t-s)\varepsilon \,, &\\
&\hspace*{-35mm} 0<t< b^{-1} \ln \left(\fr{c}{2}\right)\,; \\
b^{-1}\left(\ln \fr{c}{2} +1-\fr{c}{2}\,e^{-b(t-s)}\right)\varepsilon +{}&\\
{}+
\fr{c}{2}\,e^{-b(t-s)} \|{\bf p}(s)-{\bf \bar{p}}(s)\|\,, &\\
&\hspace*{-30mm}t\ge b^{-1}\ln \left(\fr{c}{2}\right)
\end{cases}
%\label{3.05}
\end{equation*}
для любых начальных условий ${\bf p}(s)$ и $\bar{\bf p}(s)$.
Из неравенств~(\ref{2.04}) и~(\ref{3.01}) вытекает, что $b=a$, $c={8GN}/{d}$.  
Устремив $t \hm\to \infty$ и взяв $s\hm=0$, получаем требуемые оценки.


\smallskip

\noindent
\textbf{Замечание~3.} 
В полученную оценку устойчивости для математического ожидания процесса 
в качестве множителя входит размерность~$r$, поэтому иногда лучший результат 
удается получить при помощи другого подхода, описанного в работе~\cite{z11}.

\smallskip

Положим 
$$
S=\max\limits_{{1 \le i, j \le r}} \fr{d_i}{d_j}\,,
$$ 
и пусть числа $K, L$ таковы, что 

\noindent
$$
d_1\la_1(t) + (d_1+d_2)\la_2(t) + \dots + 
\left(\sum\limits_{1 \le i \le r}d_i\right) \la_r(t) \le K\,,
$$ 
а 

\noindent
\begin{multline*}
d_1(\la_1(t)-\bar{\la}_1(t)) + (d_1+d_2)(\la_2(t)-\bar{\la}_2(t)) + \dots\\
\dots + 
\left(\sum\limits_{1 \le i \le r}d_i\right) (\la_r(t)-\bar{\la}_r(t)) \le 
L\varepsilon
\end{multline*} 
почти при всех $t \ge 0.$

\smallskip

\noindent
\textbf{Теорема~3.}
\textit{Пусть  выполнены условия теоремы~2 и, кроме того, при всех~$k$ 
и почти всех $t \hm\ge 0$ $\la_k(t) \hm< \infty$. Тогда при любых начальных условиях 
${\bf p}(s)$ и ${\bar{\bf p}}(s)$ для процессов $X(t)$ и $\bar{X}(t)$ 
соответственно справедливо неравенство}

\noindent
\begin{equation*}
\limsup\limits_{t \to \infty}   |E_{\bf p}(t)- \bar{E}_{\bar{\bf p}(t)}|\le 
\fr{ N\varepsilon\left(L a+ 2KNS\right)}{W a \left(a-2\varepsilon S\right)}\,.
\end{equation*}


\smallskip

\noindent
Д\,о\,к\,а\,з\,а\,т\,е\,л\,ь\,с\,т\,в\,о.\
 Перепишем исходную систему~(\ref{ur_per}) для невозмущенного процесса в следующем виде:
 \noindent
 
\begin{equation*}
\fr{d\vp}{dt}=\bar{B}(t)\vp(t) + {\bf f}(t)+\left(B(t)-\bar{B}(t)\right)\vp(t)\,.
%\label{eq112-n}
\end{equation*}
Тогда

\noindent
\begin{multline*}
\vp(t)=\bar{U}(t,0)\vp(0)+\int\limits_0^t \bar{U}(t,\tau){\bf{f}}(\tau) \, d\tau+{}\\
{}+\int\limits_0^t \bar{U}(t,\tau) \left(B(\tau)-\bar{B}(\tau)\right)\vp(\tau)\, d\tau\,;
\end{multline*}

\vspace*{-9pt}

\begin{equation*}
\hspace*{-15mm}\bar{\vp}(t)=\bar{U}(t,0)\bar{\vp}(0)+\int\limits_0^t \bar{U}(t,\tau){\bf{f}}(\tau) \, d\tau,
\end{equation*}
где $U(t,s)$~--- матрица Коши для уравнения~(\ref{ur_per}).
В любой норме при одинаковых начальных условиях получаем следующую оценку:
%\noindent
\begin{multline}
 \label{3000}
\!\!\!\!\!\!\left\|\vp(t)-\bar{\vp}(t)\right\|\le \!\!\int\limits_0^t \!\!\|\bar{U}(t,\tau)\|
\left(\| B(\tau)-\bar{B}(\tau)\| \|\vp(\tau)\| +\right.\\
\left.{}+ \| \vf(\tau)-\bar{\vf}(\tau)\|\right)\,d\tau\,.\!
\end{multline}
Имеем почти при всех $t \ge 0$:
\begin{equation*}
\|B(t)-\bar{B}(t)\|_{1D}=\|D(B(t)-\bar{B}(t))D^{-1}\| \le 2S\varepsilon\,;
%\label{3002}
\end{equation*}
%
%\vspace*{-14pt}
%
%\noindent
\begin{multline*}
\|{\bf f}(t)\|_{1D} \le d_1\la_1(t) + (d_1+d_2)\la_2(t) + \dots + {}\\
{}+
\left(\sum\limits_{1 \le i \le r}d_i\right) \la_r(t) \le K\,, 
\quad \|\vf(\tau)-\bar{\vf}(\tau)\|_{1D} \le L\varepsilon\,.
%\label{3002-a}
\end{multline*}
А тогда
\begin{multline*}
\gamma(\bar{B}(t))_{1D} \le \gamma(DB(t)D^{-1})+\|B(t)-\bar{B}(t)\|_{1D} \le  {}\\
{}\le -
\alpha(t)+2S \varepsilon \,.
% \label{3003}
\end{multline*}

Оценим теперь
\begin{multline*} 
%\label{8402}
\!\|{\bf p}(t)\|_{1D} \le
\|U(t){\bf p}(0) \|_{1D} +
 \int\limits_0^t \!\!\| U(t,\tau){\bf f}(\tau)\, d\tau \|_{1D} \le {}\\
 {}\le
 N e^{-a t} \| \vp(0)\|_{1D}  + \fr{K N}{a}.
\end{multline*}

 Тогда с учетом~(\ref{3000}) получаем:
\begin{multline*} 
%\label{3004}
\left\|\vp(t)-\bar{\vp}(t)\right\|_{1D}\le N\int\limits_0^t e^{-(a - 2\varepsilon S)(t-\tau)}\times{}\\
{}\times
\left(2S\varepsilon (N e^{-a \tau} \| \vp(0)\|_{1D}  + \fr{K N}{a}) +  L\varepsilon \right)\, d\tau  \le {} \\
{}\le  o(1)+\fr{ N\varepsilon(L+{2KNS}/{a})}{a-2\varepsilon S}\,. 
\end{multline*}

\vspace*{-9pt}

\section{Примеры}

\noindent
\textbf{Пример 1.}

Рассмотрим исходный процесс обслуживания с интенсивностями 
$\la_1(t)\hm=\la_2(t)\hm=\la_3(t)\hm=\la(t) \hm= 3\hm+\sin{2\pi t}$, 
$\mu_1(t)\hm=\mu_2(t)\hm=  \mu(t) \hm= 2\hm+\cos{2\pi t}$, 
$\la_4(t)=\ldots=\la_r(t)\hm=\mu_3(t)=\ldots=\mu_r(t)\hm=0$. Выберем последовательность  
$d_k\hm=h^k$, где $0{,}82 \hm< h \hm<1$. Тогда имеем
$$
d=h^r\,; \quad G \le \fr{h}{1-h}\,; \quad W=\fr{h^r}{r}\,.
$$

Будем предполагать, что возмущенный процесс имеет такую же структуру 
мат\-ри\-цы интенсивностей, причем $|\la(t)\hm-\bar{\la}(t)| \hm\le \varepsilon$ 
и  $|\mu(t)\hm-\bar{\mu}(t)| \hm\le \varepsilon$ почти при всех $t \hm\ge 0$. 
Отметим кстати, что при этом $\| A(t)\hm-\bar{A}(t)\| \hm\le 10 \varepsilon$ почти при 
всех $t \hm\ge 0$. Рассмотрим дальнейшие оценки:
$$
S=\fr{1}{h^2}\,; \ K=4 \left(3h+2h^2+h^3\right)\,; \ L=3h+2h^2+h^3\,;
$$
$$
\alpha(t) \ge \la(t)\left(3 - h - h^2 -h^3\right)-\mu(t)\left(\fr{1}{h^2}+\fr{1}{h}-2\right)\,;
$$
$$
\alpha^*= 3\left(3 - h - h^2 -h^3\right)-2\left(\fr{1}{h^2}+\fr{1}{h}-2\right)\,;
$$


\noindent
\begin{multline*}
M_0 \le \int\limits_0^1 |\alpha(t)|\, dt \le 4\left(3 - h - h^2 -h^3\right)+{}\\
{}+
3\left(\fr{1}{h^2}+\fr{1}{h}-2\right)\,;
\end{multline*}

\vspace*{-9pt}

\noindent
$$
M=e^{\alpha^*+M_0}\,.
$$

Если, например, взять 
$h\hm=0{,}9$, то $\alpha^*\hm=0{,}992$, $M_0\hm=3{,}281$, $M\hm=71{,}737$.

Тогда получаем следующие оценки.

По следствию~2
\begin{align*}
 \|{\bf p}(t)- {\bf p^{*}}(t)\| &\le \fr{8Me^{-\alpha^*t}}{h^{r-1}(1-h)}\,;\\
|E_{\bf p}(t)-\phi^*(t)| &\le  \fr{4Mre^{-\alpha^*t}}{h^{r-1}(1-h)}\,.
\end{align*}

По теореме~2 ($N=M$, $a=\alpha^*$) с использованием оценок следствия~2
\begin{align*}
\limsup\limits_{t \to \infty} \|{\bf p}(t)- \bar{\bf p}(t)\| &\le{} \notag\\
&\hspace*{-15mm}{}\le \fr{\varepsilon(1+\ln({4M}/({h^{r-1}(1-h)})))}{\alpha^*}\,;\\
\limsup\limits_{t \to \infty}   |E_{\bf p}(t)- \bar{E}_{\bar{\bf p}(t)}| &\le \notag\\
&\hspace*{-15mm}{}\le\fr{r\varepsilon(1+\ln(4M/(h^{r-1}(1-h))))}{\alpha^*}\,.
\end{align*}

По теореме~3 с использованием оценок следствия~2
\begin{multline*}
\limsup\limits_{t \to \infty}   |E_{\bf p}(t)- \bar{E}_{\bar{\bf p}(t)}| \le {}\\
{}\le
\fr{rM\varepsilon(3h+2h^2+h^3)(\alpha^* h^2+8M)}{h^r\alpha^*(\alpha^* h^2-2\varepsilon)}\,.
\end{multline*}

\noindent

\textbf{Пример 2.}

Рассмотрим процесс с интенсивностями 
$\la_1(t)\hm=\la_2(t)\hm=\ldots=\la_r(t) \hm= \la(t) \hm= 3\hm+\sin{2\pi t}$; 
$\mu_1(t)\hm=\mu_2(t)\hm= \mu(t) \hm= 2+\cos{2\pi t}$;
$\mu_3(t)=\ldots=\mu_r(t)=0$.

Будем предполагать, что возмущенный процесс имеет такую же структуру 
мат\-ри\-цы интен\-сив\-ностей, причем $|\la(t)-\bar{\la}(t)| \hm\le \varepsilon$ и  
$|\mu(t)-\bar{\mu}(t)| \hm\le \varepsilon$ почти при всех $t \hm\ge 0$. 
При этом будем иметь $\| A(t)\hm-\bar{A}(t)\| \hm\le 2r \varepsilon$ почти при всех $t \hm\ge 0$.

Выберем последовательность $d_k\hm=1$. Тогда  
\begin{gather*}
d=1\,; \enskip G=r\,; \enskip W=\fr{1}{r}\,; \enskip S=1\,; \\
K=\fr{4r(1+r)}{2}\,; \quad L=\fr{r(1+r)}{2}\,;
\\
\alpha(t)=\la(t)\,; \ \alpha=2\,; \ \alpha^*=3\,; M_0 \le 4\,; \ M \le  e^{7}\,.
\end{gather*}

И получаем следующие оценки.

\columnbreak

По следствию~1
\begin{align*}
 \|{\bf p^*}(t)- {\bf p^{**}}(t)\| &\le 8re^{-2t}\,;\\
|E_{\bf p}(t)- \phi(t)|&\le  4r^2 e^{-2t}\,.
\end{align*}

По следствию~2
\begin{align*}
\|{\bf p}(t)- {\bf p^{*}}(t)\| &\le 8re^{7-3t}\,;
\\[6pt]
|E_{\bf p}(t)- \phi^*(t)| &\le 4r^2 e^{7-3t}\,.
\end{align*}

По теореме~2 ($N=1$, $a=\alpha$) с учетом оценок следствия~1
\begin{align*}
\limsup\limits_{t \to \infty} \|{\bf p}(t)- \bar{\bf p}(t)\| &\le 
\fr{\varepsilon(1+\ln{4r})}{2}\,;
\\[6pt]
\limsup\limits_{t \to \infty}   |E_{\bf p}(t)- \bar{E}_{\bar{\bf p}(t)}|
&\le \fr{r\varepsilon(1+\ln{4r})}{2}\,.
\end{align*}

По теореме~2 ($N=M$, $a=\alpha^*$) с учетом оценок следствия~2
\begin{align*}
\limsup\limits_{t \to \infty} \|{\bf p}(t)- \bar{\bf p}(t)\| &\le 
\fr{\varepsilon(8+\ln{4r})}{3}\,;
\\
\limsup\limits_{t \to \infty}   \left|E_{\bf p}(t)- \bar{E}_{\bar{\bf p}(t)}\right| &\le 
\fr{r\varepsilon(8 + \ln{4r})}{3}\,.
\end{align*}

По теореме~3 с учетом оценок следствия~1
\begin{equation*}
\limsup\limits_{t \to \infty}   \left|E_{\bf p}(t)- \bar{E}_{\bar{\bf p}(t)}\right| \le 
\fr{5 \varepsilon r^2 (1+r)}{4(1- \varepsilon)}\,.
\end{equation*}

По теореме~3 с учетом оценок следствия~2
\begin{equation*}
\limsup\limits_{t \to \infty}   \left|E_{\bf p}(t)- \bar{E}_{\bar{\bf p}(t)}\right| \le 
\fr{\varepsilon e^{7} r^2 (1+r) (3+8e^{7})}{6(3-2\varepsilon)}\,.
\end{equation*}

{\small\frenchspacing
{%\baselineskip=10.8pt
\addcontentsline{toc}{section}{Литература}
\begin{thebibliography}{99}

 \bibitem{b} %1
\Au{Баруча-Рид~А.\,Т.} Элементы теории марковских процессов и их
приложения.~--- М.: Наука, 1969.

\bibitem{gm}  %2
\Au{Гнеденко~Б.\,В., Макаров~И.\,П.} Свойства решений задачи с потерями
в случае периодических интенсивностей~// Дифф. уравнения, 1971.
Вып.~7. С.~1696--1698.

\bibitem{g1}   %3
\Au{Gnedenko~D.\,B.} On a generalization of Erlang formulae~// 
Zastosow. Mat., 1971. Vol.~12. P.~239--242.

\bibitem{S}  %4
\Au{Саати~Т.\,Л.} Элементы теории массового обслуживания
 и ее приложения.~--- М.: Сов. радио, 1971.

\bibitem{g}  %5
\Au{Gnedenko~B., Soloviev~A.} On the conditions of the
existence of final probabilities for a Markov process~// Math.
Operations. Stat., 1973. P.~379--390.

\bibitem{gk} %6
\Au{Гнеденко~Б.\,В., Коваленко~И.\,Н.} Введение в теорию массового
обслуживания.~--- М.: Наука, 1987.
\pagebreak

\bibitem{gz00}   %7
\Au{Granovsky~B.\,L., Zeifman~A.\,I.}  The N-limit of spectral gap of 
a class of birth-death Markov chains~//
 Appl. Stoch. Models Business Ind., 2000. Vol.~16. P.~235--248.

\bibitem{z08b}  %8
\Au{Зейфман~А.\,И., Бенинг~В.\,Е., Соколов~И.\,А.} 
Марковские цепи и модели с непрерывным временем.~--- М.: Элекс-КМ, 2008.

\bibitem{dzp} %9
\Au{Van Doorn~E.\,A., Zeifman~A.\,I., Panfilova~T.\,L.}  
Bounds and asymptotics for the rate of convergence of birth-death processes~//  
Th. Prob. Appl., 2010. Vol.~54. P.~97--113.

\bibitem{z95b}   %10
\Au{Zeifman~A.\,I.} Upper and lower bounds on the rate of
convergence for nonhomogeneous birth and death processes~//  Stoch.
Proc. Appl., 1995. Vol.~59. P.~157--173.

\bibitem{gz05}  %11
\Au{Granovsky~B.\,L., Zeifman~A.\,I.} On the lower bound of the spectrum
 of some mean-field models~// Theory Prob. Appl., 2005. Vol.~49. P.~148--155.
 
\bibitem{z85}  %12
\Au{Zeifman~A.\,I.} Stability for contionuous-time
nonhomogeneous Markov chains~// Lect. Notes Math.,  1985. Vol.~1155.
P.~401--414.

\bibitem{z98} %13
\Au{Zeifman~A.} Stability of birth and death processes~// 
J.~Math. Sci., 1998. Vol.~91. P.~3023--3031.

\bibitem{ae} %14
\Au{Андреев~Д., Елесин~М., Кузнецов~А., Крылов~Е., Зейфман~А.}
Эргодичность и устойчивость нестационарных систем обслуживания~//
Теория вероятностей и математическая статистика, 2003. Т.~68.
С.~1--11.

\bibitem{mit03} %15
\Au{Mitrophanov~A.\,Yu.} Stability and exponential convergence of continuous-time 
Markov chains~//  J. Appl. Prob., 2003. Vol.~40. P.~970--979.

\label{end\stat} 

\bibitem{z11} %16
\Au{Зейфман~А.\,И., Коротышева~А.\,В., Панфилова~Т.\,Л., Шоргин~С.\,Я.} 
Оценки устойчивости  для некоторых систем обслуживания с катастрофами~//  
Информатика и её применения, 2011. Т.~5. Вып.~3. С.~27--33.
 \end{thebibliography}
}
}


\end{multicols}        %3
\def\stat{lebedev}

\def\tit{МАКСИМУМЫ АКТИВНОСТИ В~БЕЗМАСШТАБНЫХ СЛУЧАЙНЫХ СЕТЯХ С~ТЯЖЕЛЫМИ ХВОСТАМИ$^*$}

\def\titkol{Максимумы активности в~безмасштабных случайных сетях с~тяжелыми хвостами}

\def\autkol{А.\,В.~Лебедев}
\def\aut{А.\,В.~Лебедев$^1$}

\titel{\tit}{\aut}{\autkol}{\titkol}

{\renewcommand{\thefootnote}{\fnsymbol{footnote}}\footnotetext[1]
{Работа выполнена при поддержке РФФИ, грант 11-01-00050.}}


\renewcommand{\thefootnote}{\arabic{footnote}}
\footnotetext[1]{Механико-математический факультет
    Московского государственного университета им.\ М.~В.~Ломоносова,
    avlebed@yandex.ru}
    
  %  \vspace*{-6pt}

\Abst{Рассматриваются ориентированные степенные случайные графы как
    модели информационных сетей, где каждый узел обладает случайной информационной
    активностью, распределение которой имеет тяжелый (правильно меняющийся)
    хвост. Используется модель случайного графа, в которой входящие
    степени вершин независимы и имеют распределение со степенным хвостом.
    Выведены достаточные условия, при которых максимум
    суммарных активностей (по узлу и его входящим соседям) растет асимптотически
    так же, как и максимум индивидуальных активностей, и в силу этого для них
    имеет место предельный закон Фреше.}

 %   \vspace*{-2pt}
    
    \KW{максимумы; случайные суммы; безмасштабные сети;
    степенной закон; случайный граф; тяжелый хвост; правильное изменение;
    распределение Фреше}
    
%                    \vspace*{-9pt}
    
     \vskip 14pt plus 9pt minus 6pt

      \thispagestyle{headings}

      \begin{multicols}{2}
      
            \label{st\stat}
            


\section{Введение}

    Степенными (\textit{power-law}) или безмасштабными (\textit{free-scale}) называют
    случайные графы, у которых степени вершин подчиняются асимптотически
    степенному закону (с вероятностями $p_k\hm\sim ck^{-\beta}$, $k\hm\to\infty$,
    $\beta\hm>1$).
    Активные исследования данного класса графов и их приложений
    в последнее десятилетие были инициированы работой~\cite{Bar},
    где приведен ряд интересных примеров (Интернет, электрическая сеть,
    социальная сеть киноактеров).
    С~тех пор были предложены и изучены различные модели степенных графов.
    В~одних степенной закон возникает благодаря некоторому случайному
    процессу~\cite{Bar, Komp}, в других он постулируется изначально~\cite{Pow, Reit}.
    Следует отметить, что некоторые асимптотические свойства графов при
    одинаковом распределении степеней вершин могут оказаться общими,
    а другие зависят от выбора модели.

    Степенной граф может служить моделью некоторой информационной сети.
    Например, исследования кириллического сегмента <<Живого журнала>>
    ({\sf livejournal.com})~\cite{Komp}
    показывают, что он хорошо описывается степенным графом с $\beta\hm\approx 3$.
    Пусть каждый узел этой сети обладает случайной информационной активностью
    (интенсивностью производства информации). Имеется в виду среднее количество
    информации, производимой узлом в единицу времени. Активность будем
    полагать индивидуальной характеристикой, присущей узлу. Например, речь может идти
    о пользователях, которые пишут сообщения в Интернет.
    Предположим, что активности узлов
    независимы и одинаково распределены, причем их распределение~$F$ имеет тяжелый
    (правильно меняющийся) хвост, т.\,е.\ ${\bar F}(x)\hm\sim x^{-a}L(x)$,
    $x\hm\to\infty$, $a>0$, где $L(x)$~--- медленно меняющаяся функция~\cite[\S\ 8.8]{Fel}. 
    Такое предположение находится в русле современных
    представлений о распространенности степенных законов в природе, технике и
    человеческой деятельности. Активности и степени вершин (узлов) для
    простоты будем полагать независимыми.

    Рассмотрим суммарную активность в узле (т.\,е.\ сумму его собственной и
    ближайших соседей). Например, в <<Живом журнале>>
    каждый пользователь может оставлять свои записи и читать записи
    своих друзей, объединяемые для удобства в общую <<ленту друзей>> (френдленту).
    Далее будем интересоваться вопросом: когда максимум суммарных активностей
    растет асимптотически так же, как и максимум индивидуальных
    активностей узлов? В~этом случае для максимумов легко выводится
    предельный закон Фреше
    $\Phi_a(x)\hm=\exp\{-x^{-a}\}$, $x\hm>0$~[5, \S\ 8.8; 6, \S~3.3.1].

    Для модели степенного графа, введенной в~\cite{Pow},
    этот вопрос был изучен автором в~\cite{Leb2}.
    Там число вершин степени~$k$ полагалось
    детерминированным и равным $\lfloor e^\alpha/k^\beta\rfloor$,
    $\alpha$, $\beta>0$, $1\hm\le k\hm\le e^{\alpha/\beta}$,
    а распределение на множестве графов,
    удовле\-тво\-ря\-ющих этому условию, равномерным.
    Были получены достаточные условия того,
    что максимум сумм с ростом числа узлов (при $\alpha\hm\to\infty$)
    растет асимптотически так же, как и макcимум
    индивидуальных активностей: $a\hm<\beta\hm-3/2$, если $3/2\hm<\beta\hm<3$, и
    $a\hm<\beta/2$, если $\beta\hm\ge 3$. При этом применялись
    ранее полученные автором результаты для
    общей схемы максимумов сумм независимых случайных величин~\cite{Leb1}.

    Рассмотрим теперь модель ориентированного
    случайного графа, где направления ребер соответствуют направлениям
    передачи информации. Пусть имеется $n$ вершин и заданы независимые
    неотрицательные целочисленные случайные величины $K_1,\dots, K_n$,
    имеющие одинаковое
    распределение, заданное вероятностями $p_k\hm\sim ck^{-\beta}$, $k\hm\to\infty$,
    $\beta>1$. Положим $D_i\hm=\min\{K_i,n-1\}$. Для $i$-й вершины выберем
    случайным образом (равновероятно и независимо от выбора для других
    вершин) $D_i$ различных вершин из числа остальных (кроме $i$-й) и
    выпустим из них ребра, направленные в \mbox{$i$-ю} вершину. Полученный в
    результате граф можно отнести к степенным в том смысле, что входящие
    степени вершин распределены асимптотически по степенному закону.
    Суммарной активностью в узле в данном случае будем считать сумму
    собственной активности узла и всех узлов, из которых в него
    поступает информация (его входящих соседей).

    К сожалению, метод, использованный в~\cite{Leb2}, здесь
    не работает при $\beta\hm<3$, так как второй момент
    входящей степени вершины тогда растет слишком быстро при $n\hm\to\infty$.
    Эта проблема решается с по\-мощью урезания.
    При этом получаются более сильные ограничения на
    параметры, что связано с более быст\-рым ростом максимальной (входящей)
    степени вершины в графе. Однако поскольку используются лишь
    достаточные условия, не исключено, что эти ограничения в
    будущем могут быть ослаблены.

    %Результаты \cite{Leb2} и настоящей работы были кратко изложены в \cite{Leb3}.

    Отметим, что асимптотическая эквивалентность хвостов распределений
    суммы и максимума конечного числа независимых одинаково распределенных
    случайных величин в случае тяжелых хвостов
    представляет собой давно известный факт~\cite[\S\ 8.8]{Fel},
    обусловленный тем, что основной вклад в сумму дает самое большое
    слагаемое (максимум), а сумма остальных слагаемых по сравнению с
    ним оказывается мала. Теперь обобщим это утверждение
    на модель, где имеется
    некоторый набор случайных сумм со случайными числами слагаемых
    и от сумм берется максимум. По-преж\-не\-му оказывается, что основной
    вклад (в одну или несколько сумм, а значит, и в их максимум) дает только
    одно, максимальное слагаемое. Однако для этого хвост распределения
    слагаемых должен быть достаточно тяжелым.

    Проверка наличия подобного эффекта в реальных сетях, разумеется,
    требует экспериментального исследования, выходящего
    за рамки данной работы, которая имеет теоретический характер.
    
        \vspace*{-9pt}

    
    \section{Основной результат}
    
    \vspace*{-2pt}

    Будем рассматривать сети из $n$ узлов, затем устремляя $n$ к бесконечности.
    Обозначим через $M(n)$ максимум суммарных активностей (самого узла и его
    входящих соседей), а через $M_0(n)$~--- максимум индивидуальных активностей
    узлов. Требуется определить условия, при которых
    \begin{equation}
    \label{MP}
\fr{M(n)}{M_0(n)}\stackrel{P}{\to} 1\,,\quad n\to\infty\,.
    \end{equation}

    Введем неотрицательную функцию
    $u(s)$ такую, что $s{\bar F}(u(s))\hm\to 1$, $s\hm\to\infty$.
    Заметим, что $u(s)$ заведомо существует и правильно
    меняется с показателем $1/a$, т.\,е.\ $u(s)\hm\sim s^{1/a}L_2(s)$,
    $s\to\infty$, где $L_2(s)$~--- медленно меняющаяся функция~\cite[\S\ 1.5]{Sen}.

    Тогда имеет место предельный закон для максимумов независимых случайных
    величин в случае правильно меняющихся хвостов~[5, \S~8.8; 6, \S~3.3.1]:
    $$
    \lim\limits_{n\to\infty}{\bf P}\left(\fr{M_0(n)}{u(n)}\le x\right)=\Phi_a(x)\,,\quad x>0\,,
    $$
    что в сочетании с~(\ref{MP}) дает
    \begin{equation*}
%    \label{PZ}
    \lim\limits_{n\to\infty}{\bf P}\left(\fr{M(n)}{u(n)}\le x\right)=\Phi_a(x)\,,\quad x>0\,.
    \end{equation*}
    %В этом и заключается польза соотношения (\ref{MP}).

\smallskip

\noindent
\textbf{Теорема 1.} \textit{Соотношение}~(\ref{MP}) \textit{выполняется при
    $a\hm<\beta-2$, если $2\hm<\beta\hm<3$, и при $a\hm<(\beta-1)/2$,
    если $\beta\hm\ge 3$.}

    
    \smallskip
    
    \noindent
    Д\,о\,к\,а\,з\,а\,т\,е\,л\,ь\,с\,т\,в\,о\ теоремы будет приведено в разд.~4.

    
    \section{Общая схема максимумов сумм}

    Напомним введенную в~\cite{Leb1} схему (немного изменив обозначения).
    Пусть заданы случайный процесс $\Upsilon(t)$, $t\hm\in T$,
    значениями которого являются конечные
    классы конечных подмножеств ${\bf N}$,
    и семейство $\Xi\hm=\{\xi_{i,t},i\in {\bf N},t\in T\}$
    неотрицательных случайных величин, независимых и одинаково распределенных
    при любом фиксированном значении параметра $t\hm\in T$.
    Полагаем, что $\Upsilon$ и~$\Xi$ независимы.

    Для любых $A\subset {\bf N}$, $t\hm\in T$
    обозначим максимум набора случайных величин $\{\xi_{i,t}$, $i\hm\in A\}$
    через $M_t(A)$, $r$-й максимум (т.\,е.\ чис\-ло, стоящее $r$-м с конца в
    вариационном ряду)~--- через $M_t^{(r)}$, сумму~--- через $S_t(A)$.
    Пусть 
    $$
    U(t)\hm=\bigcup\limits_{A\in\Upsilon(t)}A\,.
    $$

    Введем случайные процессы, порожденные $\Upsilon$ и~$\Xi$:
    \begin{gather*}
    Z(t)=\!\!\sup\limits_{A\in\Upsilon(t)}\!\!S_t(A)\,;\enskip
    \kappa(t)=\!\!\sup\limits_{A\in\Upsilon(t)}\!\!|A|\,;
    \enskip\nu(t)=\left|U(t)\right|\,;\\
    \mu_1(t)=M_t(U(t))\,;\quad \mu_r(t)=M_t^{(r)}(U(t))\,,
 \end{gather*}
    где через $|A|$ обозначен размер (число элементов) множества~$A$.
    Через $|\Upsilon(t)|$ обозначим число различных множеств $A\hm\in\Upsilon(t)$.

    Предполагается, что $\nu(t)<\infty$ почти наверное (п.\,н.)\
    при всех $t\in T$, откуда следует конечность п.\,н.\
    всех процессов, введенных выше.

    Рассмотрим предельное поведение $Z(t)$ при $t\hm\to\infty$.

    Пусть существует случайный процесс $\rho(t)$ со
    значениями в ${\bf Z}_+$, измеримый относительно~$\Upsilon$ и такой, что
    $\rho(t)\hm\ge 1$ при $\nu(t)\hm\ge 1$, $\rho(t)\hm\le\nu(t)$ п.\,н. при всех $t\hm\in T$.

    Обозначим через $\pi(t)$ вероятность того, что для
    множества~$B$, равновероятно выбранного среди всех подмножеств $U(t)$,
    состоящих из $\rho(t)$ элементов,
    имеет место $\sup\limits_{A\in\Upsilon(t)}|A\cap B|\hm>1$.

\smallskip

\noindent
\textbf{Теорема I.} \textit{Пусть выполнены условия
    \begin{align}
    \label{us1}
    (\kappa(t)-1)\fr{\mu_{\rho(t)}(t)}{\mu_1(t)}\stackrel{P}{\to} 0,\quad t\to\infty\,;
\\
\label{us2}
    \pi(t)\to 0\,,\quad t\to\infty\,,
    \end{align}
    тогда
    \begin{equation}
    \label{res1}
    \fr{Z(t)}{\mu_1(t)}\stackrel{P}{\to} 1\,,\quad t\to\infty\,.
    \end{equation}
    }

    Доказано также следующее свойство порядковых статистик
    в случае правильно меняющихся хвостов.
    Пусть $X_n$, $n\ge 1$, независимы и имеют распределение~$F$
    с правильно меняющимся хвос\-том ${\bar F}(x)\hm\sim x^{-a}L(x)$,
    $x\hm\to\infty$, $a\hm>0$.
    Обозначим максимум
    $X_1,\dots, X_n$ через ${\tilde X}_n$ и $r$-й максимум через
    ${\tilde X}^{(r)}_n$.

\smallskip

\noindent
\textbf{Следствие II.}  \textit{Пусть $r_n\hm\sim n^\gamma$, $n\hm\to\infty$, $0\hm<\gamma\hm<1$ и
    $0\hm<\delta\hm<\gamma/a$, тогда
    $$
    \fr{n^\delta {\tilde X}^{(r_n)}_n}{{\tilde X}_n}\stackrel{P}{\to} 0\,,\quad
    n\to\infty\,.
    $$}
    
\vspace*{-12pt}

    
\section{Приложение к случайным графам}

    Адаптируем общую схему к изучению случайных графов.
    Рассмотрим процесс~$\Upsilon$ с дискретным временем, соответствующим
    числу узлов~$n$.
    Случайные величины $\xi_{i,n}$, $1\hm\le i\hm\le n$, описывают
    информационные активности узлов. Обозначим через
    $A_i$ множество из индекса~$i$ и индексов входящих соседей
    $i$-го узла, тогда набор множеств~$A$ получается из набора
    $A_1,\dots, A_n$ удалением повторов (если они есть). Имеем
    $|A_i|\hm=D_i\hm+1$ и $\kappa(n)\hm=\max\limits_{1\le i\le n}D_i\hm+1$.
    Очевидно, $|\Upsilon(n)|\hm\le n$ и $\nu(n)\hm=n$.
    Последовательность $\rho(n)$ далее будем полагать детерминированной.
    Кроме того, в используемых обозначениях $M(n)\hm=Z(n)$, $M_0(n)\hm=\mu_1(n)$ 
    и~(\ref{res1}) эквивалентно~(\ref{MP}).

    Обозначим
    $$
    Q(n,m)=n{\bf M}\left((D+1)D{\bf I}\{D\le m-1\}\right)\,,
    $$
    где $D\stackrel{d}{=}D_1$.

\medskip

\noindent
\textbf{Лемма 1.} \textit{При $n>2$ и $\rho(n)<n$ верно неравенство}
    $$
    \pi(n)\le\fr{\rho(n)(\rho(n)-1)}{2(n-1)(n-2)}\,Q(n,m)+
    {\bf P}(\kappa(n)>m)\,.
    $$

    \smallskip
    
    \noindent
    Д\,о\,к\,а\,з\,а\,т\,е\,л\,ь\,с\,т\,в\,о\,.\
    Событие $\{\sup\limits_{A\in\Upsilon(t)}|A\cap B|\hm>1\}$ представляет собой
    объединение событий $\{|A_i\cap B|\hm>1\}$, $1\hm\le i\hm\le n$.
    Пусть для простоты $B\hm=\{1,2,\dots, \rho(n)\}$ (в противном
    случае можно перенумеровать~$A_i$). Зафиксируем входящие степени
    вершин $d_1,\dots, d_n$. Тогда для $1\hm\le i\hm\le \rho(n)$ один элемент
    множества~$B$ заведомо принадлежит~$A_i$ (а именно, индекс~$i$),
    а любой другой принадлежит с вероятностью $d_i/(n-1)$. Для
    $\rho(n)+1\hm\le i\hm\le n$ любая пара индексов из~$B$ принадлежит~$A_i$
    с ве\-ро\-ят\-ностью $d_i(d_i-1)/((n-1)(n-2))$, а всего таких пар
    $\rho(n)(\rho(n)-1)/2$. Суммируя вероятности, получаем оценку сверху:
    $$
    \fr{\rho(n)-1}{n-1}\sum\limits_{i=1}^{\rho(n)}d_i+
    \fr{\rho(n)(\rho(n)-1)}{2(n-1)(n-2)}\sum\limits_{i=\rho(n)+1}^nd_i(d_i-1)\,.
    $$
    Обозначим 
    \begin{align*}
    q_1&={\bf M}(D{\bf I}\{D\le m-1\})\,;\\
q_2&={\bf M}(D(D-1){\bf I}\{D\le m-1\})\,.
\end{align*}

Усредняя по наборам входящих
    степеней вершин в области $\kappa(n)\hm\le m$, получаем оценку сверху:
\begin{multline*}
    \fr{\rho(n)-1}{n-1}\,\rho(n)q_1+
    \fr{\rho(n)(\rho(n)-1)}{2(n-1)(n-2)}\left(n-\rho(n)\right)q_2\le{}\\
{}\le\fr{\rho(n)(\rho(n)-1)}{2(n-1)(n-2)}\,n\left(2q_1+q_2\right)={}\\
{}=
    \fr{\rho(n)(\rho(n)-1)}{2(n-1)(n-2)}\,Q(n,m)\,.
\end{multline*}
    Учитывая также вероятность события $\{\kappa(n)>m\}$, получаем
    утверждение леммы.

\medskip

\noindent
\textbf{Лемма 2.} \textit{Пусть выполнены следующие условия:}
\begin{enumerate}[(1)]
\item \textit{все $\xi_{i,n}$ имеют одинаковое распределение $F$ на~${\bf R}_+$
    с хвостом ${\bar F}(x)\sim x^{-a}L(x)$, $x\to\infty$, $a>0$};
\item
    $m\sim n^\delta$, $n\to\infty$, $\delta>0$;
\item
    $Q(n,m)=O(n^b)$, $n\to\infty$,   $0<b<2$;
\item
    $\kappa(n)=o_p(m)$, $n\to\infty$;
\item
    $a<(2-b)/(2\delta)$.
    \end{enumerate}
    \textit{Тогда верно}~(\ref{MP}).

    
    \medskip
    
    \noindent
    Д\,о\,к\,а\,з\,а\,т\,е\,л\,ь\,с\,т\,в\,о\,.\
    Можно выбрать $\gamma\hm\in (0,1)$
    так, чтобы выполнялось неравенство $a\delta\hm<\gamma\hm<(2-b)/2$.
    Положим $\rho(n)\hm=[n^\gamma]$, тогда
    по следствию~II получаем~(\ref{us1}), а по лемме~1~---~(\ref{us2}),
    так что условия теоремы~I выполняются и верно соотношение~(\ref{res1}), 
    эквивалентное~(\ref{MP}).

\smallskip

\noindent
Д\,о\,к\,а\,з\,а\,т\,е\,л\,ь\,с\,т\,в\,о\ теоремы~1.\
    Поскольку $p_k\hm\sim ck^{-\beta}$,
    $k\hm\to\infty$, то хвост распределения имеет
    асимптотику ${\bar F}_K(k)\hm\sim c_1k^{-(\beta-1)}$.
    Отсюда получаем
    $\kappa(n)\hm=O_p(n^{1/(\beta-1)})$, $n\hm\to\infty$,
    и, следовательно, $\kappa(n)\hm\sim o_p(m)$ при любом
    $\delta\hm=(1+\varepsilon)/(\beta-1)$, $\varepsilon\hm>0$. Имеем
    \begin{multline*}
    Q(n,m)=n\sum\limits_{k=1}^{m-1}(k+1)kp_k={}\\
    {}=
\begin{cases}
    O(n^{1+\delta(3-\beta)})\,,& 1<\beta<3\,;\\
    O(n\ln n)\,,& \beta=3\,;\\
    O(n)\,, & \beta>3\,.
    \end{cases}
    \end{multline*}
    При $2<\beta<3$ применяем лемму 2 с $b\hm=2\delta\hm>1\hm+\delta(3\hm-\beta)$ и,
    устремляя~$\varepsilon$ к нулю,
    получаем достаточное условие $a\hm<\beta\hm-2$ для выполнения~(\ref{MP}).
    При $\beta\hm\ge 3$ применяем лемму 2 с $b\hm=1\hm+\varepsilon$ и
    аналогично получаем достаточное условие $a\hm<(\beta-1)/2$.
    
    {\small\frenchspacing
{%\baselineskip=10.8pt
\addcontentsline{toc}{section}{Литература}
\begin{thebibliography}{9}


    \bibitem{Bar}
    \Au{Barab\'asi A., Albert R.} Emergence of scaling in random networks~//
    Science, 1999. Vol.~286. P.~509--512.

    \bibitem{Komp}
    \Au{Захаров П.} Народ-бло\-го\-но\-сец~// Компьютерра,
    2007. №\,27-28. C.~36--39. {\sf http://offline.computerra.ru/2007/ 695/327726}.

    \bibitem{Pow}
    \Au{Aiello W., Chung F., Lu~L.} A random graph model for power law
    graphs~// Experimental Math., 2001. Vol.~10. No.~1. P.~53--66.

    \bibitem{Reit}
    \Au{Reittu H., Norros~I.} On the power-law random graph model
    of massive data network~// Performance Evaluation, 2004. Vol.~55. P.~3--23.

    \bibitem{Fel}
    \Au{Феллер В.} Введение в теорию вероятностей и ее приложения. Т.~2.~---
    М.: Мир, 1984.

    \bibitem{EKM}
    \Au{Embrechts P., Kl$\ddot{\mbox{u}}$ppelberg C., Mikosh~T.} Modelling
    extremal events for insurance and finance.~--- Springer-Verlag, 2003.

    \bibitem{Leb2}
    \Au{Лебедев А.\,В.} Максимумы активности в случайных сетях в
    случае тяжелых хвостов~// Проблемы передачи информации, 2008. Т.~44.
    №\,2. С.~96--100.

    \bibitem{Leb1}
\Au{Лебедев А.\,В.} Общая схема максимумов сумм независимых
    случайных величин и ее приложения~// Математические заметки, 2005.
    Т.~77. №\,4. С.~544--550.

\label{end\stat}

    \bibitem{Sen}
\Au{Сенета Е.} Правильно меняющиеся функции.~--- М.: Наука, 1985.
 \end{thebibliography}
}
}


\end{multicols}        %4
\def\stat{kudr}

\def\tit{ПРИБЛИЖЕННЫЕ МЕТОДЫ РЕШЕНИЯ ЗАДАЧИ ДИАГНОСТИКИ ПЛОСКИМ 
ЗОНДОМ СИЛЬНОИОНИЗОВАННОЙ ПЛАЗМЫ С~УЧЕТОМ КУЛОНОВСКИХ 
СТОЛКНОВЕНИЙ}

\def\titkol{Приближенные методы решения задачи диагностики плоским 
зондом сильноионизованной плазмы} %с~учетом Кулоновских  столкновений}

\def\autkol{И.\,А.~Кудрявцева, А.\,В.~Пантелеев}
\def\aut{И.\,А.~Кудрявцева$^1$, А.\,В.~Пантелеев$^2$}

\titel{\tit}{\aut}{\autkol}{\titkol}

%{\renewcommand{\thefootnote}{\fnsymbol{footnote}}\footnotetext[1]
%{Работа поддержана Российским фондом фундаментальных исследований
%(проекты 11-01-00515а и 11-07-00112а), а также Министерством
%образования и науки РФ в рамках ФЦП <<Научные и
%научно-педагогические кадры инновационной России на 2009--2013~годы>>.}}


\renewcommand{\thefootnote}{\arabic{footnote}}
\footnotetext[1]{Московский авиационный институт, irina.home.mail@mail.ru}
\footnotetext[2]{Московский авиационный институт, avpanteleev@inbox.ru}

\vspace*{-2pt}

\Abst{Сформирована математическая модель, описывающая динамику сильноионизованной 
плазмы с учетом столкновений заряженных частиц вблизи плоского зонда. Модель включает уравнение 
Фоккера--Планка и уравнение Пуассона. Предложено два подхода к решению задачи: на основе метода 
статистических испытаний Мон\-те-Кар\-ло и на основе композиции метода крупных частиц и метода 
расщепления.} 

\vspace*{-2pt}

\KW{телекоммуникационные системы; метод Монте-Карло; метод крупных частиц; метод 
расщепления; зонд; уравнение Фоккера--Планка; уравнение Пуассона} 

\vspace*{-4pt}

 \vskip 8pt plus 9pt minus 6pt

      \thispagestyle{headings}

      \begin{multicols}{2}
      
            \label{st\stat}

\section{Введение}

В настоящее время в области телекоммуникаций все более востребованными становятся 
информационные технологии, основанные на использовании математических моделей и численных 
методов физики плазмы. Поэтому особенно актуальным является решение разнообразных задач анализа 
поведения плазмы, включающих в себя формирование новых моделей и методов их исследования. 
Помимо этого, в разработке телекоммуникационного оборудования эффективно используются 
собственно физические свойства плазмы. В~частности, изготовлена антенна, работа которой основана 
на газовом разряде низкотемпературной плазмы~[1], интенсивно ведутся разработки по созданию и 
усовершенствованию источников бесперебойного питания на основе плазменных элементов~[2, 3]. 
      
      Одним из наиболее перспективных направлений для построения систем оптической 
беспроводной связи является использование лазеров~\cite{4-k, 5-k}. В~этой связи большое внимание 
уделяется использованию плазмы при разработке импульсных сильноточных коммутаторов~\cite{6-k}, 
так как практическое применение подобных разработок требует повышения уровня надежности и 
быстродействия лазерных систем.
      
      Исследования низкотемпературной плазмы также связаны с разработками в области дальней 
космической связи, так как моделирование процессов взаимодействия заряженного тела с верхними 
слоями атмосферы позволяет предлагать способы улучшения существующих систем радиосвязи с 
космическими летательными аппаратами~\cite{7-k}. 
      
      Наряду с этим актуальными также являются задачи диагностики плазмы, поскольку перспективы 
ее использования в области телекоммуникаций после более полного изучения физических свойств 
могут значительно расшириться. 

Для диагностики плазмы применяют зондовые методы исследования~[8--11]. Эти методы относятся к 
классу контактных методов; как следствие, возникает сложность в исследовании пристеночной области 
вблизи зонда, которая характеризуется достаточно сложным распределением потенциала и функциями 
распределения, отличными от максвелловских. 

Данная работа посвящена исследованию переходного режима обтекания заряженного тела плазмой. Для 
переходного режима выполняется следующее условие: длина свободного пробега иона до столкновения 
с нейтральным атомом или другим ионом невелика по сравнению с характерными размерами тела. 
В~этом случае возникает необходимость учета столкновений заряженных частиц с нейтральными 
атомами и кулоновских столкновений. В~работах~\cite{10-k, 11-k} подробно рассмотрена модель с 
учетом столкновений заряженных частиц с нейтральными атомами. В~настоящей статье представлена 
теоретическая модель, описывающая влияния ион-ионных и ион-элек\-т\-рон\-ных столкновений на 
измеряемые характеристики плазмы, что ранее детально не исследовалось.
      
      В~рамках данной работы предлагается модель, описывающая динамику сильноионизованной 
плазмы с учетом кулоновских столкновений. Эта модель учитывает такие процессы взаимодействия, 
как перенос частиц и столкновения между заряженными частицами типа <<ион--ион>> и 
      <<ион--электрон>> под влиянием макроскопического электрического поля. Перечисленные 
процессы описываются самосогласованной системой уравнений, включающей уравнение 
      Фок\-ке\-ра--План\-ка и уравнение Пуассона~[12].
      
      Вычислительная модель задачи строится на основе двух методов: метода статистических 
испытаний Мон\-те-Кар\-ло и композиции метода крупных частиц и метода расщепления. Приведены 
результаты численного моделирования, полученные с использованием вышеперечисленных методов.

\vspace*{-4pt}

\section{Постановка задачи}

\vspace*{-2pt}

Рассматривается следующая физическая постановка зондовой задачи~[11]. В~невозмущенную 
бесконечно протяженную плазму, состоящую из электронов и однозарядных ионов, внесена большая\linebreak 
заряженная до потенциала $\varphi_p$ плоскость. Плоскость, расположенная поперек потока плазмы, 
является идеально поглощающей для электронов. Ионы при ударе о плоскость нейтрализуются. 
Предполагается, что частицы в плазме движутся под действием внешнего электрического поля, 
магнитное поле отсутствует. Концентрации ионов $n_{i\infty}$ и электронов $n_{e\infty}$, а также 
температуры данных час\-тиц~$T_{i\infty}$ 
и~$T_{e\infty}$ в невозмущенной плазме заданы. За начальные 
функции распределения обоих типов час\-тиц принимаются функции распределения Максвелла. 
      
      Требуется с учетом столкновений между заряженными частицами найти напряженность 
самосогласованного электрического поля $\vec{E}(\vec{r},t)$, функции распределения однозарядных 
ионов $f_i(\vec{r}, \vec{v}, t)$ и электронов $f_e(\vec{r}, \vec{v}, t)$, 
а также их моменты (плотности 
токов ионов и электронов  $j_i(\vec{r},t)\hm
=q\int f_i(\vec{r}, \vec{v}, t)\vec{v}\,d\vec{v}$, $j_e(\vec{r},t) 
\hm={\sf e}\int f_e(\vec{r},\vec{v},t)\vec{v}\,d\vec{v}$, где $q=Z_i{\sf e}$, $Z_i=1$~--- заряд иона, ${\sf 
e}$~--- заряд электрона; концентрации ионов и электронов $n_i(\vec{r},t)\hm=\int 
f_i(\vec{r},\vec{v},t)\,d\vec{v}$, $n_e(\vec{r},t)\hm=\int f_e(\vec{r},\vec{v}, t)\,d\vec{v}$). 
Поведение частиц во 
времени~$t$ характеризуется ра\-ди\-ус-век\-то\-ром~$\vec{r}$ и вектором скорости~$\vec{v}$.
      
      Математическая модель, соответствующая данной физической постановке задачи, имеет 
вид~\cite{11-k, 13-k}:

\noindent
      \begin{equation}
      \left.
      \begin{array}{c}
      \fr{\partial f_\alpha (\vec{r},\vec{v},t)}{\partial t}+
      \vec{v}\fr{\partial f_\alpha (\vec{r},\vec{v},t)}{ 
\partial \vec{r}}+
\fr{\vec{F}_\alpha(\vec{r},t)}{m_\alpha}\times{}\\[4pt]
{}\times\fr{\partial f_\alpha(\vec{r},\vec{v},t)}{ \partial 
\vec{v}}=
\left(\fr{\partial f_\alpha(\vec{r},\vec{v},t)}{ \partial t}\right)_{\mathrm{с}}+S_\alpha 
(\vec{r},\vec{v},t)\,;\\[6pt]
      \Delta\varphi(\vec{r},t)=-\fr{{\sf e}}{\varepsilon_0}\left( n_i(\vec{r},t)-n_e(\vec{r},t)\right)\,;\\[6pt]
      \vec{E}(\vec{r},t)=-\nabla \varphi(\vec{r},t)\,.
      \end{array}\!\!
      \right\}\!\!
      \label{e1-k}
      \end{equation}
Здесь первое уравнение~--- уравнение Фок\-ке\-ра--План\-ка для частиц сорта~$\alpha$ ($\alpha=i,e$), 
второе~--- уравнение Пуассона для самосогласованного электрического поля; 
$f_\alpha(\vec{r},\vec{v},t)$~--- функция\linebreak
распределения час\-тиц сорта~$\alpha$; $(\partial 
f_\alpha(\vec{r},\vec{v},t)/\partial t)_{\mathrm{с}}$~--- 
оператор столкновений Фок\-ке\-ра--План\-ка; 
функция~$S_\alpha(\vec{r},\vec{v},t)$ описывает источники или стоки\linebreak
 час\-тиц; 
$\vec{F}_\alpha(\vec{r},t)=q_\alpha\vec{E}(\vec{r},t)$, где $\vec{E}(\vec{r},t)$~--- напряженность 
самосогласованного электрического поля, 
$$
q_\alpha =
\begin{cases}
-{\sf e}\,, & \alpha=e\,,\\
{\sf e}\,, & \alpha=i\,;
\end{cases}
$$
$\varphi(\vec{r},t)$~--- потенциал самосогласованного электрического поля; $n_\alpha(\vec{r},t)$ ($\alpha 
\hm=i,e$)~--- концентрация частиц сорта~$\alpha$; $m_\alpha$~--- масса частицы сорта~$\alpha$; 
$\varepsilon_0$~--- электрическая постоянная. 

Оператор столкновений Фок\-ке\-ра--План\-ка имеет вид~\cite{13-k, 14-k}
\begin{multline*}
\fr{1}{\Gamma_\alpha}\left( \fr{\partial f_\alpha}{\partial t}\right)_{\mathrm{с}} 
=\fr{1}{2}\,\nabla_v\nabla_v:\left(f_\alpha\nabla_v\nabla_vg_\alpha(\vec{r},\vec{v},t)\right)-{}\\
{}-
\nabla_v\cdot\left(f_\alpha\nabla_v h_\alpha\right)\,,
\end{multline*}
где $\nabla_v\nabla_v g_\alpha(\vec{r},\vec{v},t)$~--- ковариантная тензорная производная второго ранга, 
знак двоеточия ($:$) обозначает операцию двойного суммирования:
\begin{gather*}
\Gamma_\alpha=\fr{Z_\alpha^4 {\sf e}^4}{4\pi \varepsilon_0^2 m^2_\alpha}\,\ln D_\alpha\,;
\\
D_\alpha =\fr{12\pi\varepsilon_0 kT_{\alpha\infty}}{Z_\alpha^2 {\sf e}^2}\left( \fr{\varepsilon_0 k 
T_{e\infty}}{n_{e\infty} {\sf e}^2}\right)^{1/2}\,;\\
g_\alpha (\vec{r},\vec{v},t)=\sum\limits_{b=i,e}\left( \fr{Z_b}{Z_\alpha}\right) \int f_b 
(\vec{r},{\vec{v}}^{\,\prime},t)\left\vert \vec{v}-{\vec{v}}^{\,\prime}\right\vert\,d\vec{v}^{\,\prime}\,;\\
h_\alpha (\vec{r},\vec{v},t)=\sum\limits_{b=i,e} \fr{m_\alpha+m_b}{m_b} 
\left(\fr{Z_b}{Z_\alpha}\right)
\int
\fr{f_b(\vec{r},{\vec{v}}^{\,\prime}, t)}{\vert \vec{v}-{\vec{v}}^{\,\prime}\vert}
\,d{\vec{v}}^{\,\prime}\,;\\
Z_\alpha =1\,, \quad \alpha=i,e\,.
\end{gather*}
 
К системе уравнений~(\ref{e1-k}) необходимо добавить начальные и краевые условия:
\begin{equation}
\!\left.
\begin{array}{rrl}
t=0:\ & f_\alpha(\vec{r},\vec{v},0)&=f_\alpha^{\mathrm{maksv}}\,,\enskip \alpha=i,e;\\[9pt]
\vec{r}\in \Omega_p:\ & f_\alpha(\vec{r},\vec{v},t)\big\vert_{\vec{r}\in\Omega_p}&=0\,,\enskip \alpha=i,e\,;\\[9pt]
&\varphi(\vec{r},t)\big\vert_{\vec{r}\in\Omega_p}&=\varphi_p\,;\\[9pt]
\vec{r}\in\Omega_\infty:\ & 
f_\alpha(\vec{r},\vec{v},t)\big\vert_{\vec{r}\in\Omega_\infty}&= %{}\\[9pt]
f_\alpha^{\mathrm{maksv}}\,,\enskip \alpha=i,e\,;\\[9pt]
&\varphi(\vec{r},t)\big\vert_{\vec{r}\in\Omega_\infty}&=0\,,
\end{array}\!\!
\right\}\!\!\!\!
\label{e2-k}
\end{equation}
    где 
    
    \noindent
    \begin{multline*}
    f_\alpha^{\mathrm{maksv}}=n_{\alpha\infty}\left(\fr{m_\alpha}{2k\pi T_{\alpha\infty}}\right)^{3/2}\times{}\\
    {}\times
    \exp\left( -
\fr{m_\alpha}{2kT_{\alpha\infty}}\left\vert\vec{v}-\vec{v}_\infty\right\vert^2\right)\,,
\enskip \alpha=i, e\,;
\end{multline*} 
$\Omega_p$ и $\Omega_\infty$~--- множество радиус-векторов час\-тиц, концы которых принадлежат плоскости зонда и 
границе возмущенной зоны соответственно.

Для решения поставленной задачи введем декартову систему координат таким образом, чтобы 
заряженная плоскость совпала с плоскостью~$0xz$. Тогда положение частицы в пространстве будет 
определяться координатами $x,y,z$, а скорость~--- координатами $v_x, v_y, v_z$. В~силу того что 
плоскость является бесконечно большой в сравнении с характерным размером задачи, функции 
распределения частиц будут зависеть только от переменных $y, v_y, t$.

Поставленную задачу предлагается решать независимо двумя методами. Первый метод основывается на 
методе статистических испытаний Мон\-те-Кар\-ло, второй метод является композицией метода 
расщепления и метода крупных частиц.

\section{Применение метода Монте-Карло}

Запишем самосогласованную систему уравнений~(\ref{e1-k}) и~(\ref{e2-k}) в декартовой системе 
координат с учетом сделанных предположений:
\begin{equation}
\left.
\begin{array}{l}
\fr{\partial f_\alpha}{\partial t}+
v_y\fr{\partial f_\alpha}{\partial y}+\fr{F_y^\alpha}{m_\alpha}\,\fr{\partial 
f_\alpha}{\partial v_y}=\fr{1}{2}\,\fr{\partial^2 }{\partial [v_y]^2}\times{}\\
{}\times \left( 
f_\alpha\fr{\partial^2 g_\alpha  }{\partial [v_y]^2}\right) -
\fr{\partial}{\partial v_y}\left( f_\alpha\fr{\partial h_\alpha}{\partial v_y}\right)\,,
\enskip \alpha=i,e\,;\\[6pt]
    \fr{\partial^2\varphi}{\partial y^2} =-\fr{{\sf e}}{\varepsilon_0}\left(n_i-n_e\right)\,;
    \enskip E_y=-
\fr{\partial\varphi}{\partial y}\,;\\[6pt]
\hspace*{3.1mm}    t=0:\  \hspace*{2.6mm}f_\alpha(y,v_y,0)=f_\alpha^{\mathrm{maksv}}\,,\ \alpha=i,e\,;\\[9pt]
\hspace*{2.9mm} y=0:\ \hspace*{2.8mm}f_\alpha(0,v_y,t)=0\,,\ \alpha=i,e\,;\\[9pt]
\hspace*{24.3mm}\varphi(0,t)=\varphi_p\,;\\[9pt]
y=y_\infty:\ f_\alpha(y_\infty, v_y, t)=f_\alpha^{\mathrm{maksv}}\,,\ \alpha=i,e\,;\\[9pt]
\hspace*{21.5mm}\varphi(y_\infty, t)=0\,.
\end{array}
\right \}
\label{e3-k}
\end{equation}

В полученной системе уравнений~(\ref{e3-k}) перейдем к безразмерным величинам, применив 
соотношение $X=M_X \hat{X}$, где $M_X$~--- масштаб размерной величины~$X$, $\hat{X}$~--- 
безразмерная величина~$X$. В~качестве используемых масштабов были взяты следующие: радиус 
Дебая, скорость теплового движения частиц, концентрация частиц в невозмущенной плазме, потенциал, 
возникающий при разделении зарядов в дебаевской сфере, и производные от них величины.

Система безразмерных уравнений имеет следующий вид:
%\noindent
\begin{equation}
\left.
\begin{array}{l}
\fr{\partial 
\hat{f}_\alpha}{\partial\hat{t}}+A_\alpha\fr{\partial\hat{f}_\alpha}{\partial\hat{y}}+
B_\alpha\hat{E}_y\fr{\partial\hat{f}_\alpha}{\partial \hat{v}_y}={}\\
\!{}=
\fr{\partial^2}{\partial[\hat{v}_y]^2}\left(D_\alpha 
\hat{f}_\alpha\right)-\fr{\partial}{\partial\hat{v}_y}\left(K_\alpha \hat{f}_\alpha\right),\enskip 
\alpha=i,e;\\[9pt]
\fr{\partial^2\hat{\varphi}}{\partial\hat{y}^2}=-\left(\hat{n}_i-\hat{n}_e\right)\,;\enskip \hat{e}_y=-
\fr{\partial\hat\varphi}{\partial\hat{y}}\,;\\[9pt]
\hspace*{3.1mm}\hat{t}=0:\ \hspace*{2.6mm}\hat{f}_\alpha(\hat{y},\hat{v}_y,0)=\hat{f}_\alpha^{\mathrm{maksv}}\,,\enskip \alpha-i,e\,;\\[9pt]
\hspace*{2.9mm}\hat{y}=0:\ \hspace*{2.8mm}\hat{f}_\alpha(0,\hat{v}_y,\hat{t})=0\,,\enskip \alpha=i,e\,;\\[9pt]
\hspace*{24.3mm}\hat\varphi(0,\hat{t})=\hat{\varphi}_p\,;\\[9pt]
\hat{y}=\hat{y}_\infty:\ \hat{f}_\alpha(\hat{y}_\infty, \hat{v}_y, \hat{t})=\hat{f}^{\mathrm{maksv}}_\alpha\,,\enskip 
\alpha=i,e\,;\\[9pt]
\hspace*{21.5mm}\hat\varphi(\hat{y}_\infty,\hat{t})=0\,.
\end{array}
\right\}
\label{e4-k}
\end{equation}
Здесь 

\vspace*{-2pt}

\noindent
\begin{gather*}
A_\alpha=\sqrt{\delta_\alpha }\,\hat{v}_y\,;\enskip 
B_\alpha=\sqrt{\delta_\alpha}\,\fr{z_\alpha}{2\varepsilon_\alpha}\,;\\
\delta_\alpha=\fr{\varepsilon_\alpha}{\mu_\alpha}\,;\enskip 
\varepsilon_\alpha=\fr{T_{\alpha\infty}}{T_{i\infty}}\,;\\
\mu_\alpha=\fr{m_\alpha}{m_i}\,;\enskip 
D_\alpha=A_g^\alpha\fr{\partial^2\hat{g}_\alpha}{\partial  [\hat{v}_y]^2}\,;\\
K_\alpha=A_h^\alpha \fr{\partial \hat{h}_\alpha}{\partial \hat{v}_y}\,,\enskip \alpha=i,e\,,
\end{gather*}
где $A_g^\alpha$ и $A_h^\alpha$~--- коэффициенты, определяемые характерными параметрами 
задачи~\cite{15-k}.

Поиск решения самосогласованной системы уравнений~(\ref{e4-k}) осуществляется по следующей 
схе-\linebreak ме. Вначале находятся значения напряженности\linebreak
 электрического поля по значениям потенциала, 
полученным из граничной задачи для уравнения Пуассона. Далее, используя найденные значения 
напряженности, решается уравнение Фок\-ке\-ра--План\-ка путем перехода к стохастическому 
дифференциальному уравнению (СДУ) Ито:

\noindent
\begin{multline*}
d\Theta_\alpha(\hat{t}) = a_\alpha \left(\hat{t},\Theta_\alpha(\hat{t})\right)+{}\\
{}+\sigma\left(
\hat{t},\Theta_\alpha(\hat{t})\right)\,dW(\hat{t})\,,\quad \alpha=i,e\,,
%\label{e5-k}
\end{multline*}
где 

\noindent
\begin{align*}
\Theta_\alpha(\hat{t})&=\begin{bmatrix}
\hat{y}(\hat{t})\\ \hat{v}_y(\hat{t})
\end{bmatrix}\,;\\
a_\alpha\left(\hat{t},\Theta_\alpha(\hat{t})\right)&=\begin{bmatrix}
-A_\alpha\\ -K_\alpha -B_\alpha \hat{E}_y
\end{bmatrix}\,;\\
\sigma_\alpha\left(\hat{t},\Theta_\alpha(\hat{t})\right)\sigma_\alpha^{\mathrm{T}}\left( 
\hat{t},\Theta_\alpha(\hat{t})\right)&=D_\alpha\,,\enskip \alpha=i,e\,;
\end{align*} 
$W(\hat{t})$~--- стандартный винеровский случайный процесс.
\pagebreak

Для нахождения значений вектора состояния~$\Theta_\alpha(\hat{t})$ применим явную разностную 
схему стохастического метода Эйлера~\cite{16-k}:
\begin{multline*}
\Theta_\alpha^{n+1}=\Theta_\alpha^n +h_\tau a_\alpha \left( \hat{t}_n, \Theta_\alpha^n\right)+\sigma_\alpha 
\left( \hat{t}_n, \Theta_\alpha^n\right)\Delta W_n\,,\\ 
n=0,\ldots , N\,,\ \alpha=i,e\,,
%\label{e6-k}
\end{multline*}
где $\Theta_\alpha^n$, $n=0,\ldots , N$,~--- приближенное значение вектора 
состояния~$\Theta_\alpha(\hat{t})$, $\alpha=i,e$, в момент времени $\hat{t}\hm=\hat{t}_n$, 
$\hat{t}_n\hm=n h_\tau$, $n=0,\ldots , N$; $h_\tau$~--- достаточно малый шаг интегрирования; $\Delta 
W_n$, $n=0,\ldots ,N$,~--- величина приращения винеровского процесса~$W(\hat{t})$ на отрезке $\left[ 
\hat{t}_n,\,\hat{t}_{n+1}\right]$, по определению независимая от~$\Theta_\alpha^0$, 
$\Delta W_0,\ldots , 
\Delta W_{n-1}$: $\Delta W_n\hm=W(\hat{t}_{n-1})\hm-W(\hat{t}_n)$; $\Delta W_n\hm\sim N(0,\,h_\tau)$, 
т.\,е.\ $\Delta W_n$ представляют собой гауссовские случайные величины с нулевыми математическими 
ожиданиями и дисперсиями, равными шагу интегрирования; $\Theta_\alpha^0$~--- значение вектора 
состояния $\Theta_\alpha(\hat{t})$, $\alpha\hm=i,e$, в момент времени $\hat{t}=0$, 
$\Theta_\alpha^0\hm\sim \hat{f}_\alpha^{\mathrm{maksv}}$. 

Частные производные $\partial^2\hat{g}_\alpha/\partial[\hat{v}_y]^2$ и $\partial \hat{h}_\alpha/\partial 
\hat{v}_y$, являющиеся составляющими матрицы $\sigma_\alpha (\hat{t}_n, 
\Theta_\alpha^n)\sigma_\alpha^{\mathrm{T}}(\hat{t}_n,\Theta_\alpha^n)$ и вектора $a_\alpha(\hat{t}_n, 
\Theta_\alpha^n)$ соответственно, аппроксимируются со вторым порядком точности на трехточечном 
шаблоне на основе значений~$\hat{g}_\alpha$ и~$\hat{h}_\alpha$~\cite{17-k}.
      
      В выражения для функций~$\hat{g}_\alpha$ и~$\hat{h}_\alpha$ входят интегралы, которые 
вычисляются методом Мон\-те-Кар\-ло с использованием набора значений скоростной компоненты 
вектора состояния~$\hat{v}_y$, полученных из решения СДУ Ито:
      \begin{equation*}
      \int \hat{f}_\alpha \left\vert \hat{v}_y-
\hat{v}_y^\prime\right\vert\,dv_y^\prime=M\left(\zeta\left(\hat{V}_y\right)\right)\,,
\end{equation*}
где
$$
      \zeta\left(\hat{V}_y\right)=\left\vert \hat{v}_y-\hat{V}_y\right\vert\,,\enskip \hat{V}_y\sim 
\hat{f}_\alpha\,.
  $$
      
      Для вычисления напряженности самосогласованного электрического поля $\hat{E}_y=-
\partial\hat{\varphi}/\partial\hat{y}$, входящей в вектор $a_\alpha(\hat{t}_n, \Theta_\alpha^n)$, необходимо 
аналогично аппроксимировать со вторым порядком точности производную 
$\partial\hat{\varphi}/\partial\hat{y}$ на трехточечном шаблоне с использованием значений 
потенциала~$\hat{\varphi}$~\cite{17-k}. Значения потенциала~$\hat\varphi$ находятся из решения 
уравнения Пуассона. 
      
      Граничную задачу для уравнения Пуассона 
      \begin{align*}
      \fr{\partial^2 \hat\varphi}{\partial \hat{y}^2} & = -\left(\hat{n}_i-\hat{n}_e\right)\,;\\
      \hat{\varphi}\big|_{\hat{y}=0} &=\hat{\varphi}_p\,;\\
      \hat{\varphi}\big|_{\hat{y}_\infty=0} &=0
      \end{align*}
    предлагается решать путем перехода к конечно-разностной системе с последующим ее решением 
методом прогонки~\cite{17-k}:

\noindent
\begin{gather*}
\hat{\varphi}^n_{l-1}+2\hat{\varphi}_l^n+\hat{\varphi}^n_{l+1}=
h_y\hat{\delta}_l^n\,,\enskip l=1,\ldots , 
N_y\,;\\
\hat{\delta}_l^n=-\left( \hat{n}^n_{i,l}-\hat{n}^n_{e,l}\right)\,;\enskip 
\hat{\varphi}_0=\hat{\varphi}_p\,;\enskip \hat{\varphi}_{N_y}=0\,,
\end{gather*}
где $N_y$~--- число шагов по переменной~$\hat{y}$, $h_y$~--- величина шагов разбиения по~$\hat{y}$. 
      
      Концентрации $\hat{n}_\alpha$, $\alpha=i,e$, и плотности токов частиц на зонд~$\hat{f}_\alpha$, 
$\alpha=i,e$, вычисляются согласно описанному выше методу Мон\-те-Карло.

\section{Применение метода расщепления и~метода крупных~частиц}

Решение задачи в данном случае предлагается начать с записи правой части уравнения 
Фок\-ке\-ра--План\-ка в декартовой системе координат в виде:
$$
\mathbf{Q} f_\alpha = \fr{1}{2}\,\fr{\partial^2 f_\alpha}{\partial [v_y]^2}\,\fr{\partial^2 g_\alpha}{\partial 
[v_y]^2}+\fr{\partial f_\alpha}{\partial v_y}\,\fr{\partial C_\alpha}{\partial v_y}+H_\alpha\,,\enskip 
\alpha=i,e\,,
$$  
где 
\begin{align*}
C_\alpha(\vec{r},\vec{v},t)&=
\begin{cases}
\fr{1-\gamma}{Z_i^2}\int\fr{f_e(\vec{r},{\vec{v}}^{\,\prime},t)}{|\vec{v}-{\vec{v}}^{\,\prime} |}\,d{\vec{v}}^{\,\prime}\,, 
&\alpha=i\,;\\[9pt]
\fr{Z_i^2(\gamma-1)}{\gamma}\int \fr{f_i(\vec{r},{\vec{v}}^{\,\prime}, t)}
{|\vec{v}-{\vec{v}}^{\,\prime} 
|}\,d{\vec{v}}^{\,\prime}\,, &\alpha=e\,;
\end{cases} 
\\
H_\alpha&=
\begin{cases}
4\pi \left( \fr{\gamma f_e}{Z_i^2}+f_i\right)f_i\,, & \alpha=i\,;\\[9pt]
4\pi\left(\fr{Z_i^2 f_i}{\gamma}+f_e\right)f_e\,, &\alpha=e\,.
\end{cases}
\end{align*}
Тогда при переходе к безразмерным величинам (см.\ разд.~3) система~(\ref{e1-k}) запишется 
следующим образом:
      \begin{equation}
      \left.
\!\!\begin{array}{l}
      \fr{\partial 
\hat{f}_\alpha}{\partial\hat{t}}+A_\alpha\fr{\partial\hat{f}_\alpha}{\partial\hat{y}}+
B_\alpha  \hat{E}_y
\fr{\partial\hat{f}_\alpha}{\partial\hat{v}_\alpha}=\tilde{\mathbf{Q}}\hat{f}_\alpha\,,\enskip 
\alpha=i,e;\\[9pt]
      \fr{\partial^2\hat{\varphi}}{\partial\hat{y}^2}=-\left( \hat{n}_i-\hat{n}_e\right)\,,\enskip \hat{E}_y=-
\fr{\partial\hat\varphi}{\partial\hat{y}}\,,\\[9pt]
\hspace*{3.1mm}\hat{t}=0:\ \hspace*{2.6mm}\hat{f}_\alpha(\hat{y},\hat{v}_y, 0)=\hat{f}_\alpha^{\mathrm{maksv}}\,,\enskip \alpha=i,e\,,\\[9pt]
\hspace*{2.9mm} \hat{y}=0:\ \hspace*{2.8mm}\hat{f}_\alpha(0,\hat{v}_y,\hat{t})=0\,,\enskip \alpha=i,e\,;\\[9pt]
\hspace*{24.3mm}\hat\varphi(0,\hat{t})=\hat{\varphi}_p\,;\\[9pt]
      \hat{y}=\hat{y}_\infty:\ \hat{f}_\alpha(\hat{y}_\infty, 
\hat{v}_y,\hat{t})=\hat{f}_\alpha^{\mathrm{maksv}}\,,\enskip \alpha=i,e\,;\\[9pt]
\hspace*{21.5mm}\hat{\varphi}(\hat{y}_\infty,\hat{t})=0\,,\\[9pt]
    \end{array}
\right\}\!\!
\label{e7-k}
\end{equation}
где 
\begin{gather*}
\tilde{\mathbf{Q}} \hat{f}_\alpha=D_\alpha\fr{\partial^2\hat{f}_\alpha}{\partial 
[\hat{v}_y]^2}+K_\alpha\fr{\partial\hat{f}_\alpha}{\partial\hat{v}_y}+H_\alpha\,;\\
D_\alpha=A_g^\alpha\fr{\partial^2\hat{g}_\alpha}{\partial [\hat{v}_y]^2}\,;\enskip 
K_\alpha=A_h^\alpha \fr{\partial \hat{h}_\alpha}{\partial\hat{v}_y}\,,\ \alpha=i,e\,.
\end{gather*}

Для решения системы уравнений~(\ref{e7-k}) применяется модификация метода 
расщепления~\cite{17-k}, согласно которой исходная задача разбивается на две вспомогательные. Такое 
разбиение можно осуществить, переписав уравнение Фок\-ке\-ра--План\-ка в следующем виде:
$$
\fr{\partial\hat{f}_\alpha}{\partial\hat{t}} =
\tilde{\mathbf{Q}}_1\hat{f}_\alpha+\tilde{\mathbf{Q}}_2\hat{f}_\alpha\,,
$$
где 
\begin{align*}
\tilde{\mathbf{Q}}_1\hat{f}_\alpha &=-
\left(A_\alpha\fr{\partial\hat{f}_\alpha}{\partial\hat{y}}+
B_\alpha\fr{\partial\hat{f}_\alpha}{\partial\hat{y}}
\right)\,;\\
\tilde{\mathbf{Q}}_2\hat{f}_\alpha 
&=\left(D_\alpha\fr{\partial^2\hat{f}_\alpha}{\partial[\hat{v}_y]^2}+K_\alpha\fr{\partial 
\hat{f}_\alpha}{\partial\hat{v}_y}+H_\alpha\right)\,.
\end{align*}

      Правая часть уравнения Фок\-ке\-ра--План\-ка представляет собой сумму двух операторов, 
первый из которых отвечает за перенос частиц, второй~--- за столкновения заряженных частиц. 
В~результате образуются следующие задачи, которые решаются последовательно:
      \begin{itemize}
\item первая задача:
\begin{align*}
&\fr{\partial w_\alpha(\hat{y},\hat{v}_y,\hat{t})}{\partial\hat{t}} =\mathbf{Q}_1 
w_\alpha(\hat{y},\hat{v}_y,\hat{t})\,,\enskip \alpha=i,e\,;\\[9pt]
&\fr{\partial^2\hat\varphi}{\partial\hat{y}^2}=-\left(\hat{n}_i-\hat{n}_e\right)\,;\enskip
\hat{E}_y=-
\fr{\partial\hat\varphi}{\partial\hat{y}}\,;\\[9pt]
&w_\alpha(\hat{y},\hat{v}_y,\hat{t}^n)=\hat{f}_\alpha(\hat{y},\hat{v}_y,\hat{t}^n)\,,\enskip n=0,\ldots ,N-
1\,;\\[9pt]
&\hspace{2.9mm}\hat{y}=0:\ \hspace*{2.9mm}w_\alpha(0,\hat{v}_y,\hat{t})=0\,,\enskip \alpha=i,e\,;\\[9pt]
&\hspace*{25.1mm}\hat\varphi(0,\hat{t})=\hat{\varphi}_p\,;\\[9pt]
&\hat{y}=\hat{y}_\infty:\ w_\alpha(\hat{y}_\infty, \hat{v}_y, \hat{t})=
\hat{f}_\alpha^{\mathrm{maksv}}\,,\enskip 
\alpha=i,e\,;\\[9pt]
&\hspace*{22.5mm}\hat\varphi(\hat{y}_\infty,\hat{t})=0\,;
\end{align*}
\item вторая задача:
\begin{align*}
\!\!\!\!\!\!\!\fr{\partial s_\alpha(\hat{y},\hat{v}_y,\hat{t})}{\partial \hat{t}} &=\mathbf{Q}_2 
s_\alpha(\hat{y},\hat{v}_y,\hat{t})\,, & \alpha&=i,e\,;\\
\!\!\!\!\!\!\!s_\alpha (\hat{y},\hat{v}_y,\hat{t}^n) &=w_\alpha (\hat{y},\hat{v}_y, \hat{t}^{n+1}),& n&=0,\ldots ,N-
1.
\end{align*}
\end{itemize}

Первая задача представляет собой систему безразмерных уравнений Вла\-со\-ва--Пуас\-со\-на. Для ее 
решения применяется метод крупных частиц~\cite{18-k}. Согласно этому методу решение задачи 
осуществляется путем расщепления на два этапа: на первом этапе не учитываются конвективные члены 
и решение получается обычным интегрированием на неподвижной эйлеровой сетке, а на втором этапе 
рассматривается система, которая описывает перенос частиц в лагранжевой системе координат. Кроме 
того, на первом этапе необходимо решить уравнение Пуассона для получения значений потенциала 
самосогласованного электрического поля. Для этого применяется метод, описанный в разд.~3. 

Вторая задача решается путем перехода к ко\-неч\-но-раз\-ност\-ной сис\-те\-ме. При этом частные 
производные $\partial^2\hat{g}_\alpha/\partial[\hat{v}_y]^2$ и $\partial\hat{h}_\alpha/\partial\hat{v}_y$ 
аппроксимируются со вторым порядком точности с использованием трехточечного шаблона, а 
производная $\partial s_\alpha/\partial\hat{t}$ аппроксимируется на двухточечном шаблоне с первым 
порядком точности~\cite{16-k}. К~полученной системе разностных уравнений предлагается применить 
один из классических методов решения систем линейных уравнений, например метод 
Гаусса~\cite{19-k}.
      
      Решением первой задачи является функция $w_\alpha(\hat{y}, \hat{v}_y, \hat{t}^n)$, 
$n\hm=0,\ldots ,N$, , которая дает начальное условие для второй задачи. Решая вторую задачу, находим 
функцию $s_\alpha(\hat{y},\hat{v}_y,\hat{t}^n)\hm=\hat{f}_\alpha(\hat{y},\hat{v}_y,\hat{t}^n)$, 
$n=1,\ldots ,N$, $\alpha=i,e$, которая определяет решение $\hat{f}_\alpha(\hat{y},\hat{v}_y,\hat{t}^n)$, 
$\alpha=i,e$, исходной системы~(\ref{e7-k}) для рассматриваемых моментов времени $n=1,\ldots ,N$.

Моменты функций распределения $\hat{f}_\alpha$, $\alpha=i,e$, находятся с помощью методов 
численного интегрирования, например метода трапеций~\cite{19-k}.

\section{Результаты численного моделирования}

Для двух описанных выше методов реализованы две отдельные программы в среде {Matlab~7.0}. 
Эти программы позволяют по заданным значениям концентраций и температур частиц $n_{i\infty}$, 
$n_{e\infty}$, $T_{i\infty}$ и~$T_{e\infty}$ в невозмущенной плазме, а также потенциала~$\varphi_p$, 
подаваемого на зонд, изучить эволюцию во времени плотностей тока частиц~$j_i$ и~$j_e$, концентраций 
частиц~$n_i$  и~$n_e$ в произвольной точке пространства в возмущенной зоне, а также динамику 
изменения напряженности~$E_y$ самосогласованного электрического поля во времени и пространстве.

С использованием разработанных программ проведены серии расчетных экспериментов, в которых 
значение концентраций варьировалось в пределах $n_{i\infty} \hm = n_{e\infty}\hm =10^{18}\div 
10^{22}$~м$^{-3}$. Значение температур было выбрано неизменным и равным $T_{i\infty}\hm = 
T_{e\infty}\hm=3000$~K, а значения потенциала, подаваемого на зонд, изменялись в пределах 
$\varphi_p\hm=0\div 2{,}6$~В.

На рис.~1  и~2 приведены графики изменения напряженности самосогласованного электрического
 поля (см.\ рис.~1) и плотности токов ионов (см.\linebreak\vspace*{-12pt}

\pagebreak

\end{multicols}

\begin{figure} %fig1
\vspace*{1pt}
\begin{center}
\mbox{%
\epsfxsize=162.594mm
\epsfbox{kud-1.eps}
}
\end{center}
\vspace*{-9pt}
\Caption{Динамика изменения плотности тока ионов во времени в фиксированной точке возмущенной 
зоны для значений потенциала: \textit{1}~--- $\varphi_p=-6$; 
\textit{2}~--- $\varphi_p=-16$; \textit{3}~--- $\varphi_p=- 30$ 
в случае применения методов Монте-Карло~(\textit{а}) 
и крупных частиц~(\textit{б})}
\end{figure}

\begin{figure} %fig2
\vspace*{1pt}
\begin{center}
\mbox{%
\epsfxsize=162.713mm
\epsfbox{kud-2.eps}
}
\end{center}
\vspace*{-9pt}
\Caption{Динамика изменения напряженности электрического поля во времени в фиксированной точке 
возмущенной зоны для значений потенциала: 
\textit{1}~--- $\varphi_p=-6$; \textit{2}~--- $\varphi_p=-16$; 
\textit{3}~--- $\varphi_p=-30$ в случае применения методов Монте-Карло~(\textit{а}) и
крупных частиц~(\textit{б})
}
\end{figure}

\begin{multicols}{2}

\noindent
 рис.~2) во времени в фиксированной точке пространства 
возмущенной зоны в случае применения обоих разработанных алгоритмов.


На основании полученных результатов можно отметить похожее поведение зависимостей 
напряженности электрического поля и плотности тока от времени в двух рассматриваемых случаях. 
Графики кривых сначала убывают, затем начинают возрастать, выходя в некоторый момент 
времени~$t^\prime$ (момент установления) на стационарные значения. 

Одинаковое поведение 
напряженности и плот\-ности тока можно объяснить из следующих соображений: плотность тока ионов в 
данной области пространства равна произведению концентрации ионов на их направленную скорость и 
на заряд иона. Скорость ионов, в свою очередь, зависит от заряда, массы и напряженности 
электрического поля. 
%\columnbreak

При внесении в плазму отрицательно заряженного зонда возникает электрическое поле, которое 
нарушает квазинейтральность плазмы. Для того чтобы компенсировать действие внешнего 
электрического поля, ионы устремляются к зонду, а электроны~--- от зонда. Это приводит к дисбалансу 
концентраций вблизи зонда и, как следствие, к увеличению разности потенциалов; график 
напряженности электрического поля убывает. Вскоре разделение зарядов компенсирует внешнее 
электрическое поле; график выходит на стационарное значение. 

Также можно отметить, что значения 
напряженности электрического поля и плотности тока частиц на зонд в момент установления для двух 
методов совпадают. 

Момент установления~$t^\prime$ зависит от при\-ме\-ня\-емо\-го метода решения. В~случае метода 
Мон\-те-Кар\-ло $t^\prime=3{,}5\div 4$~ед., а для метода крупных частиц совместно с методом 
расщепления $t^\prime\hm=5\div 5{,}5$~ед. Используя ко\-неч\-но-раз\-ност\-ный метод, можно 
получить динамику изменения функций распределения частиц~$f_\alpha$, $\alpha=i,e$, во времени и 
пространстве. Функции распределения позволяют наглядно представить влияние на картину 
распределения частиц вблизи зонда самой поверхности зонда и электрического поля.

\section{Заключение}
      
      В работе найдено решение задачи диагностики плоским зондом сильноионизованной плазмы с 
учетом столкновений заряженных частиц. Разработана математическая модель исследуемого явления, 
описываемая уравнениями Фок\-ке\-ра--План\-ка и Пуассона. Решение получено двумя методами:\linebreak 
статистическим и ко\-неч\-но-раз\-ност\-ным на основе\linebreak сформированных алгоритмов. Приведены 
резуль-\linebreak таты численного моделирования при различных\linebreak характерных параметрах задачи.
 Из  проведенных 
вычислительных экспериментов вытекает, что искомые величины: напряженность 
электрического поля, плотности токов частиц на зонд, концентрации частиц вблизи зонда~--- как по 
характеру зависимости, так и по числовым значениям совпадают. При применении метода 
      Мон\-те-Кар\-ло момент установления наступает быстрее по сравнению с конечно-разностным 
методом, однако конечно-разностный метод позволяет получить более наглядные результаты.

{\small\frenchspacing
{%\baselineskip=10.8pt
\addcontentsline{toc}{section}{Литература}
\begin{thebibliography}{99}

\bibitem{1-k}
\Au{Alexeff I., Anderson T.}
Experimental and theoretical results with plasma antenna~// IEEE Trans. Plasma Sci., 2006. Vol.~34. 
No.\,2. P.~166--172.

\bibitem{2-k}
\Au{Сысун В.\,И.}
Сильноионизованная низкотемпературная плазма в приборах электронной техники: Методы 
исследования, свойства, применение. Дисс. \ldots д-ра физ.-мат. наук в форме науч. докл.: 
01.04.08.~--- Пет\-ро\-за\-водск, 1996.

\bibitem{3-k}
\Au{Тухас В.\,А.}
Методология создания средств измерений и испытаний на устойчивость к кондуктивным помехам~// 
Мат-лы VI Междунар. симп. по электромагнитной совместимости и 
электромагнитной экологии.~--- СПб., 2005. С.~231--234.

\bibitem{4-k}
\Au{Гудзенко Л.\,И., Яковленко С.\,И.}
Плазменные лазеры.~--- М.: Атомиздат, 1978.  256~с.

\bibitem{5-k}
\Au{Звелто О.}
Принципы лазеров.~--- М.: Мир, 1990.  560~с.

\bibitem{6-k}
\Au{Сысун В.\,И., Хромой Ю.\,Д.}
Расширение канала мощного импульсного разряда в парах ртути~// Электронная техника, 1974. 
Сер.~4. Вып.~10. С.~80--85. 

\bibitem{7-k}
\Au{Винклер Дж.\,Р.}
Искусственные пучки частиц в космической плазме.~--- М.: Мир, 1985.  451~с.

\bibitem{8-k}
\Au{Bernstein I.\,B., Rabinowitz I.\,N.}
Theory of electrostatic probes in low-density plasma~// Phys. Fluids, 1959. Vol.~2. No.\,2. P.~112--121. 

\bibitem{9-k}
\Au{Альперт Я.\,Л., Гуревич А.\,В., Питаевский~Л.\,П.}
Искусственные спутники в разреженной плазме.~--- М.: Наука, 1964.  282~с.

\bibitem{10-k}
\Au{Чан П., Тэлбот Л., Турян~К.}
Электрические зонды в неподвижной и движущейся плазме.~--- М.: Мир, 1978.  202~с.

\bibitem{11-k}
\Au{Алексеев Б.\,В., Котельников В.\,А.}
Зондовый метод диагностики плазмы.~--- М.: Энергоатомиздат, 1989.  240~с.

\bibitem{12-k}
\Au{Пантелеев А.\,В., Кудрявцева И.\,А.}
Формирование математической модели двухкомпонентной плазмы с учетом столкновений 
заряженных частиц в случае плоского зонда~// Теоретические вопросы вычислительной техники и 
программного обеспечения: Межвузовский сб. научн. тр.~--- М.: МИРЭА, 2006. С.~11--21.

\bibitem{13-k}
\Au{Олдер Б.}
Вычислительные методы в физике плазмы.~--- М.: Мир, 1974.  111~с.

\bibitem{14-k}
\Au{Montgomery D.\,C., Tidman D.\,A.}
Plasma kinetic theory.~--- New York, 1964. 

\bibitem{15-k}
\Au{Кудрявцева И.\,А., Пантелеев А.\,В.}
Применение метода Мон\-те-Кар\-ло для анализа поведения двухкомпонентной плазмы с учетом 
столкновений между заряженными частицами~// Теоретические вопросы\linebreak
вычислительной техники и 
программного обеспечения: Межвузовский сб. научн. тр.~--- М.: МИРЭА, 2008. С.~122--128. 

\bibitem{16-k}
\Au{Семенов В.\,В., Пантелеев А.\,В., Руденко~Е.\,А., Бор\-та\-ков\-ский~А.\,С.}
Методы описания, анализа и синтеза нелинейных систем управления.~--- М.: МАИ, 1993.  312~с.

\bibitem{17-k}
\Au{Киреев В.\,И., Пантелеев А.\,В.}
Численные методы в примерах и задачах.~--- М.: Высшая школа, 2006.  480~с.

\bibitem{18-k}
\Au{Белоцерковский О.\,М., Давыдов~Ю.\,М.}
Метод крупных частиц в газовой динамике. Вычислительный эксперимент.~--- М.: Наука, 
Физматгиз, 1982.

\label{end\stat}

\bibitem{19-k}
\Au{Вержбицкий В.\,М.}
Основы численных методов.~--- М.: Высшая школа, 2002.  840~с.
 \end{thebibliography}
}
}


\end{multicols}            %5
\def\stat{gor+mart}

\def\tit{ГИБРИДНЫЕ МОДЕЛИ ЭКСТРЕМАЛЬНОГО ГРАДИЕНТНОГО БУСТИНГА 
ДЛЯ~ВОССТАНОВЛЕНИЯ ПРОПУЩЕННЫХ ЗНАЧЕНИЙ В~ДАННЫХ ОБ~ОСАДКАХ$^*$}

\def\titkol{Гибридные модели экстремального градиентного бустинга 
для~восстановления пропущенных значений в~данных} % об~осадках}

\def\aut{А.\,К.~Горшенин$^1$,  О.\,П.~Мартынов$^2$}

\def\autkol{А.\,К.~Горшенин,  О.\,П.~Мартынов}

\titel{\tit}{\aut}{\autkol}{\titkol}

\index{Горшенин А.\,К.}
\index{Мартынов О.\,П.}
\index{Gorshenin A.\,K.}
\index{Martynov O.\,P.}


{\renewcommand{\thefootnote}{\fnsymbol{footnote}} \footnotetext[1]
{Постановка задачи и~анализ полученных результатов в~данной статье проведены 
А.\,К.~Горшениным, чьи исследования поддержаны РНФ (проект 18-71-00156). 
Разработка и~программная реализация методов анализа пропущенных значений 
выполнены О.\,П.~Мартыновым.}}


\renewcommand{\thefootnote}{\arabic{footnote}}
\footnotetext[1]{Институт проблем информатики Федерального исследовательского
центра <<Информатика и~управление>> Российской академии наук; факультет
вычислительной математики и~кибернетики Московского государственного 
университета имени М.\,В.~Ломоносова, \mbox{agorshenin@frccsc.ru}}
\footnotetext[2]{Факультет вычислительной математики и~кибернетики Московского 
государственного университета им.\ М.\,В.~Ломоносова, 
\mbox{martynov.oleg.mipt@gmail.com}}

%\vspace*{-2pt}



\Abst{Проведено сравнение классического метода экстремального градиентного бустинга, 
реализованного во фреймворке {\sf XGBoost}
 ({\sf  eXtreme Gradient Boosting}, экстремальный 
градиентный бустинг) и~категориальной
 модификации {\sf CatBoost} ({\sf Categorical Boosting}, категориальный бустинг), 
 которая достаточно редко встречается в~научных 
 исследованиях. Предложены некоторые гибридные модели 
 клас\-си\-фи\-ка\-ции-ре\-грес\-сии 
 для повышения точности заполнения пропусков в~реальных данных на примере~14~станций 
 в~Германии. Достигнутая точность в~задачах классификации составила до~92\% 
 при весьма умеренных значениях ошибок прогнозов в~метрике {\sf RMSE}
 ({\sf Root Mean-Square Error}, сред\-не\-квад\-ра\-тич\-ная ошибка). 
 Гибридные методы превзошли по качеству предсказания простые модели 
 классификации и~регрессии. Развиваемые подходы могут быть успешно 
 использованы как для непосредственного анализа метеорологических данных 
 методами машинного обучения, так и~для улучшения качества предсказания 
 на основе физических моделей атмосферных процессов.}

\KW{заполнение пропусков; осадки; классификация; регрессия; градиентный бустинг; 
XGBoost; CatBoost}

\DOI{10.14357/19922264190306} 
  
%\vspace*{1pt}


\vskip 10pt plus 9pt minus 6pt

\thispagestyle{headings}

\begin{multicols}{2}

\label{st\stat}


\section{Введение}

Повышение эффективности алгоритмов машинного обучения привело к~росту их 
востребо\-ванности как в~задачах анализа результатов физических моделей 
предсказания погоды с~целью получения более точного прогноза, так и~в~качестве 
самостоятельных инструментов исследования\linebreak про\-стран\-ственно-вре\-мен\-н$\acute{\mbox{ы}}$х 
метеорологических рядов, %\linebreak 
полученных со спутников и~метеостанций. Такие %\linebreak 
наблюдения в~больших объемах поступают с~огромного числа датчиков и~зачастую 
содержат пропуски, которые могут существенным образом повлиять на качество 
обучения методов или изменить решения статистических моделей анализа различных 
метеорологических явлений, например экстремальных осадков~\cite{Gorshenin2018c}. 
Поэтому  весьма важной оказывается задача корректного заполнения пропусков в~данных.

В настоящей статье развивается подход к~обработке пропущенных значений 
для объемов осадков на основе популярного алгоритма машинного\linebreak обучения, 
называемого градиентным бустингом над деревьями решений~\cite{Friedman2001}. 
Это семейство методов, включая и~его наиболее час\-то применяемую модификацию 
\verb"XGBoost"~\cite{Chen2016}, широко используется для решения задач 
классификации и~регрессии в~значительном спектре прикладных областей. 
Например, можно упомянуть работы по предсказанию биологической активности 
лекарств~\cite{Mustapha2016}, кредитному скоррингу~\cite{Xia2017}, 
прогнозированию кризисов на финансовых рынках~\cite{Chatzis2018}, 
обнаружению дефектов в~вет\-ро\-вых турбинах~\cite{Zhang2018}, вет\-ро\-вой 
энергетике и~солнечной радиации~\cite{Aler2017,Torres-Barran2018}.

Предложенная компанией \verb"Яндекс" модификация метода градиентного 
бустинга, получившая название \verb"CatBoost"~\cite{Prokhorenkova2018}, 
несмотря на ее 
применение\linebreak в~реальных продуктах, в~научных статьях используется достаточно редко. 
В~качестве примеров можно упомянуть протеомные исследования~\cite{Ivanov2019} 
и~<<умные>> сети электроснабжения~\cite{Punmiya2019}. 
В~данной статье восполняется этот пробел для метеорологических данных, для 
которых традиционно применяется классический экстремальный градиентный 
бустинг~\cite{Korner2018,Fan2018}. Будет проведено сравнение \verb"XGBoost" 
и~\verb"CatBoost", а~также предложены некоторые гибридные модели для повышения 
точ\-ности заполнения пропусков в~реальных данных на примере~14~станций в~Германии. 
Кроме того, для извлечения признаков будет использована новая библиотека 
\verb"tsfresh"~\cite{Christ2018} для языка программирования \verb"Python".

\section{Постановка задачи}


Исходные данные представляют собой наборы признаков 
$\{\hat{x}_t\,|\,t \hm\in T\}$ и~соответствующие им измерения $\{y_t\,|\,t \hm\in T\}$ 
за некоторый временной интервал~$T$. На первом этапе наблюдения $\{y_t\,|\,t\hm \in T\}$ 
случайным образом разделяются на тренировочную $\{y_t\,|\,t \hm\in T_{\mathrm{train}}\}$ 
и~тестовую $\{y_t\,|\,t \hm\in T_{\mathrm{test}}\}$ части, причем 
\begin{equation*}
T_{\mathrm{train}} \bigcup T_{\mathrm{test}} = T\,, \quad T_{\mathrm{train}} 
\bigcap T_{\mathrm{test}} = \emptyset.
\end{equation*}

На втором этапе без увеличения объема исходных наблюдений с~помощью 
ряда специальных процедур (см.\ статью~\cite{Christ2018}, где 
также приведено описание пакета на языке \verb"Python") 
формируются дополнительные признаки, которые в~дальнейшем 
будут использованы при обучении моделей. Таким образом, 
проводится дополнение исходных признаков~$\{\hat{x}_t\}$ 
наборами функций~$\{f_i(\cdot)\}$ и~$\{g_i(\cdot)\}$ для получения расширенного 
набора~$\{\hat{x}^e_t\}$:
\begin{multline}
\label{eXtend}
\left\{\hat{x}^e_t\right\} =\left\{\hat{x}_t\right\}\bigcup 
\left\{ f_i\left(y_{t-1}, y_{t-2}, \ldots, y_{t-k+1}\right) \right\} \bigcup{}\\
{}\bigcup
\left\{ g_i\left(y_{t+1}, y_{t+2}, \ldots, y_{t+l-1}\right)\right \},
\end{multline}
который и~будет использован для обучения моделей. Отметим, 
что  в~формуле~\eqref{eXtend} величины~$k$ и~$l$ также могут рассматриваться
 в~качестве свободных параметров. Таким образом, для обучения используются 
 расширенные наборы признаков $\{\hat{x}^e_t\,|\, t \hm\in T_{\mathrm{train}}\}$ 
 и~соответствующие им измерения $\{y_t\,|\,t \hm\in T_{\mathrm{train}}\}$, а~в~качестве 
 выходных данных выступают прогнозы $\{y^{\mathrm{pred}}_t\,|\,t \hm\in T_{\mathrm{test}}\}$.

Для оценивания корректности работы методов в~данной статье будут 
использоваться несколько метрик. Для задач классификации используется 
величина \verb"ACC", определяемая выражением:
\begin{equation}
\label{ACC}
\mathrm{ACC= TPR + TNR}\,,
\end{equation}
где \verb"TPR" (\verb"True" \verb"Positive" \verb"Rate")~--- 
доля верно угаданных положительных классов, а \verb"TNR" (\verb"True Negative Rate")~--- 
доля верно угаданных отрицательных классов (подробнее они будут описаны 
в~подразд.~4.1 с~привязкой к~наличию или отсутствию осадков 
в~конкретный день). Также для сравнения бинарных классификаторов используется 
площадь под \verb"ROC"-кри\-вой (\verb"ROC AUC"~---
area under ROC (receiver operating characteristic) curve)~\cite{Huang2005}, которая 
описывает точность решения модели как вероятность принадлежности к~определенному 
классу и~задается выражением:
\begin{equation}
\label{AUC}
\mathrm{ROC\,AUC} = \int\limits_{0}^{1} \mathrm{TPR}\left(\mathrm{FPR}^{-1}(x)\right)\,dx\,,
\end{equation}
где \verb"FPR" (\verb"False Positive Rate")~-- доля неверно угаданных 
положительных классов. Она позволяет корректнее оценивать точность в~случае, 
когда один из классов является доминирующим, что вполне соответствует исходным 
данным об осадках, в~которых достаточно много нулевых значений. Кроме того, 
используется и~стандартная для непрерывных данных метрика \verb"RMSE":
\begin{equation}
\label{RMSE}
\mathrm{RMSE} = \sqrt{|T|^{-1}\sum\limits_{t\in T} \left(y_t - y^{\mathrm{pred}}_t\right)^2}\,.
\end{equation}

\begin{table*}[b]\small %tabl1
\begin{center}
\Caption{\label{Tab1} Сравнение моделей классификации}
\vspace*{2ex}


\begin{tabular}{|l|c|r|l|c|}
\hline
\multicolumn{1}{|c|}{{\bf Город} }& 
\multicolumn{1}{c|}{{\bf Станция}} & 
\multicolumn{1}{c|}{{\bf Лучшая модель}} & 
\multicolumn{1}{c|}{{\bf ACC}}& \multicolumn{1}{c|}{{\bf ROC AUC}} \\
\hline
      Берлин &   \hphantom{9}93850 &   \verb"XGBoost_Logistic" &  $86{,}11\%$ &  $95{,}64\%$ \\
      Берлин &  103810 &  \verb"CatBoost_Logistic" &  $80{,}56\%$ &  $82{,}19\%$ \\
    Доберлуг &   \hphantom{9}94900 &  \verb"CatBoost_Logistic" &  $83{,}33\%$ &  $88{,}1\%$\hphantom{9} \\
    Доберлуг &  104900 &  \verb"CatBoost_Logistic" &  $86{,}11\%$ & $ 84{,}23\%$ \\
    Хольцдорф&  104760 &   \verb"XGBoost_Logistic" &  $61{,}11\%$ &  $72{,}1\%$\hphantom{9} \\
  Линденберг &   \hphantom{9}93930 &  \verb"CatBoost_Logistic" &  $75\%$ &  $76{,}92\%$ \\
  Линденберг &  103930 &  \verb"CatBoost_Logistic" &  $86{,}11\%$ &  $86{,}55\%$ \\
   Нойруппин &   \hphantom{9}92700 &   \verb"XGBoost_Logistic" &  $72{,}22\%$ & $ 75{,}6\%$\hphantom{9} \\
   Нойруппин &  102700 &  \verb"CatBoost_Logistic" &  $69{,}44\%$ &  $66{,}9\%$\hphantom{9} \\
     Потсдам &   \hphantom{9}93790 &   \verb"XGBoost_Logistic" &  $63{,}89\%$ &  $68{,}75\%$ \\
     Потсдам &  103790 &   \verb"XGBoost_Logistic" &  $66{,}7\%$ &  $78{,}41\%$ \\
  Визенбург &  103680 &   \verb"XGBoost_Logistic" & $66{,}7\%$ &  $70{,}63\%$ \\
  Виттенберг &   \hphantom{9}94740 &   \verb"XGBoost_Logistic" &  $77{,}78\%$ &  $74{,}6\%$\hphantom{9} \\
  Виттенберг &  104740 &   \verb"XGBoost_Logistic" &  $69{,}44\%$ &  $73{,}96\%$ \\
\hline
\end{tabular}
\end{center}
\end{table*}

\section{Подготовка данных и~программная реализация методов машинного обучения}

В качестве исходных данных использованы суточные метеорологические наблюдения 
за различные периоды времени, собранные на~14~станциях в~восьми городах  
Германии\footnote{Открытая база {\sf NNDC} (NOAA's National Data Centers) 
{\sf Climate Data} 
({\sf https://www7.ncdc.noaa.gov/CDO/cdo}) 
Национального управления океанических и~атмосферных исследований (NOAA~---
National Oceanic and Atmospheric Administration), США.}. 
Анализ в~настоящей работе проводится для объемов осадков, а такие показатели, 
как средняя температура, точка росы и~средняя скорость ветра, 
включены в~модели в~качестве дополнительных признаков. Предварительно 
проведена простая регуляризация исходных данных~--- отсечение аномальных выбросов, 
связанных с~пропущенными значениями (в~имеющихся в~распоряжении рядах они заполнены 
большими величинами), на уровне~95\% выборочного квантиля. Отметим, что такое 
преобразование не повлекло изменений очевидно корректных значений в~тестовых наборах, 
следовательно, не повлияло на качество работы методов.

Для возможности верификации раз\-ра\-ба\-ты\-ва\-емых методов заполнения пропусков 
использовались интервалы наблюдений, в~которых данные были полны. 
Для искусственного внесения пропусков исходный интервал разбивался 
на непересекающиеся отрезки на некотором расстоянии друг от друга, 
затем в~каждый из них случайным образом помещалось единственное 
пропущенное значение:

\vspace*{-6pt}

\noindent
\begin{multline*}
\ldots
\underbrace{y_{t+1}, \ldots, y_{t+g}}_{\mbox{{\small 0\ пропусков}}},
\underbrace{y_{t+g+1}, \ldots, y_{t+g+s}}_{\mbox{{\small 1\ пропуск}}},\\
\underbrace{y_{t+g+s+1}, \ldots, y_{t+2g+s+1}}_{\mbox{{\small 0\ пропусков}}}
\ldots,
\end{multline*}

\vspace*{-6pt}

\noindent
где $g = \max\{k, l\}$ (величины~$k$ и~$l$ определяются из формулы~\eqref{eXtend}),
а~параметр $s$ задается следующим выражением: 

%\vspace*{-6pt}

\noindent
$$
s = \fr{{\{\mbox{длина\ интервала}\} \hm- g}}{\{\mbox{число\ пропусков}\}}-g\,. 
$$

В~целях повышения качества предсказания рассматриваемых моделей 
из тренировочной и~тес\-то\-вой частей данных были исключены значения, 
для которых нельзя было выбрать в~точности~$k$ предыду\-щих и~$l$ последующих 
значений для корректного извлечения дополнительных признаков (см.\
 разд.~2).
 {\looseness=1
 
 }

Предварительная статистическая обработка, а~так\-же результаты 
статьи~\cite{Gorshenin2018a} демонстрируют, что распределения суточных 
объемов осадков хорошо аппроксимируются с~помощью гам\-ма-рас\-пре\-де\-ле\-ния. 
Этот факт будет использован в~некоторых моделях, в~частности в~\verb"XGBoost_Gamma". 

Программные инструменты анализа данных реализованы на языке \verb"Python" 
с~использованием биб\-лио\-тек \verb"XGBoost" и~\verb"CatBoost" для со\-от\-вет\-ст\-ву\-ющих 
моделей градиентного бустинга, \verb"tsfresh" для извлечения дополнительных 
признаков, а~так\-же классических инструментов \verb"NumPy", \verb"Pandas", 
\verb"SciPy" и~\verb"scikit-learn". Все модели обучались для различных 
конфигураций па\-ра\-мет\-ров~$k$ и~$l$~\eqref{eXtend}, итоговые результаты приведены 
для $k \hm\equiv l \hm \equiv 5$, для которых получена наибольшая точность 
из всех рассмотренных комбинаций. Всего для рассмотренных станций были 
апробированы порядка~8500~различных конфигураций.

\section{Классические модели градиентного бустинга}

В задачах прогнозирования традиционно выделяются два возможных 
направления: определение попадания предсказываемого наблюдения в~некоторую 
группу (классификация) и~установление точной величины неизвестного значения 
(регрессия). В~этом разделе рассмотрим применение алгоритмов \verb"XGBoost" 
и~\verb"CatBoost" для решения каждой из них.

\setcounter{table}{2}
\begin{table*}[b]\small %tabl3
\begin{center}
\Caption{\label{Tab3} Сравнение моделей регрессии}
\vspace*{2ex}

\begin{tabular}{|l|c|l|c|c|}
\hline
\multicolumn{1}{|c|}{{\bf Город}} & 
\multicolumn{1}{c|}{{\bf Станция}} & {\bf Лучшая модель} & 
\multicolumn{1}{c|}{\bf ACC}& {\bf RMSE} \\
\hline
      Берлин &   \hphantom{9}93850 &  \verb"XGBoost_Gamma" &  $83{,}33\%$ &  $0{,}0631$ \\
      Берлин &  103810 &  \verb"XGBoost_Gamma" &  $69{,}44\%$ &  $0{,}1111$ \\
    Доберлуг &   \hphantom{9}94900 &  \verb"XGBoost_Gamma" &  $86{,}11\%$ &  $0{,}075$\hphantom{9} \\
    Доберлуг &  104900 &   \verb"CatBoost_MAE" &  $83{,}33\%$ &  $0{,}0876$ \\
    Хольцдорф&  104760 &  \verb"XGBoost_Gamma" &  $75\%$\hphantom{,99} &  $0{,}1027$ \\
  Линденберг &   \hphantom{9}93930 &  \verb"XGBoost_Gamma" & $ 75\%$\hphantom{,99} &  $0{,}0721$ \\
  Линденберг &  103930 &  \verb"XGBoost_Gamma" &  $80{,}56\%$ &  $0{,}0912$ \\
   Нойруппин &   \hphantom{9}92700 &  \verb"XGBoost_Gamma" &  $66{,}7\%$\hphantom{9} &  $0{,}061$\hphantom{9} \\
   Нойруппин &  102700 &   \verb"CatBoost_MAE" &  $66{,}7\%$\hphantom{9} &  $0{,}0643$ \\
     Потсдам &   \hphantom{9}93790 &  \verb"XGBoost_Gamma" & $66{,}7\%$\hphantom{9} &  $0{,}0910$ \\
     Потсдам &  103790 &  \verb"XGBoost_Gamma" &  $66{,}7\%$\hphantom{9} & $ 0{,}0886$ \\
  Визенбург &  103680 &  \verb"XGBoost_Gamma" & $ 61{,}11\%$ &  $0{,}1767$ \\
  Виттенберг &   \hphantom{9}94740 &   \verb"CatBoost_MAE" &  $72{,}22\%$ &  $0{,}0591$ \\
  Виттенберг &  104740 &   \verb"XGBoost_RMSE" &  $77{,}78\%$ &  $0{,}0734$ \\
\hline
\end{tabular}
\end{center}
\end{table*}

\subsection{Классификация}
%\label{SecClass}

Поставим в~соответствие измерениям осадков~$\{y_t\}$ последовательность 
$\{ c_t \hm=\mathcal {I} (y_t\hm\geqslant \varepsilon)\}$ 
для\linebreak
 некоторого наперед заданного значения~$\varepsilon$ (например,
 $\varepsilon \hm= 0{,}005$), где~$\mathcal {I}(\cdot)$ обозначает индикатор 
 соответствующего множества. Так формируется разби\-ение исходных данных на 
 два класса~--- наличие и~отсутствие осадков. Первый класс $\{ c_t \hm= 1\}$ 
 будем называть положительным, а второй $\{c_t \hm= 0\}$~--- отрицательным. 
 Модель классификации на вход получает значения~$\hat{x}^e_t$, а~в~качестве 
 результата выдает вероятность принадлежности наблюдений к~положительному 
 классу $\{p^{\mathrm{pred}}_t \hm\equiv \mathbb P(c_t \hm= 1)\}$. 
 Результирующий класс определяется как $\{ c^{\mathrm{pred}}_t \hm=\mathcal {I}
  (p^{\mathrm{pred}}_t
 \hm\geqslant 0{,}5)\}$.
 
 
% \begin{table*}
  %tabl2
\noindent
{{\tablename~2}\ \ \small{Усредненные результаты предсказания моделей классификации}}
%\label{Tab2}}
%\vspace*{2ex}

{\small
\begin{center}
\tabcolsep=13pt
\begin{tabular}{|r|c|c|}
\hline
& \multicolumn{2}{c|}{\bf Метрика}\\
\cline{2-3} %\multicolumn{1}{|c|}{\raisebox{-6pt}[0pt][0pt]{
\multicolumn{1}{|c|}{{\raisebox{6pt}[0pt][0pt]{{\bf Модель}}}} &
{\bf ACC}&   {\bf ROC AUC}\\
\hline
\verb"XGBoost_Logistic"  &  $72{,}22\%$ &  $78{,}26\%$ \\
\verb"CatBoost_Logistic" &  $71{,}03\%$ &  $75{,}96\%$ \\
\hline
\end{tabular}
\end{center}
}
%\end{table*}

\addtocounter{table}{1}

\vspace*{12pt}

Для решения задачи классификации были выбраны модели градиентного 
бустинга \verb"XGBoost_Logistic" и~\verb"CatBoost_Logistic" с~целевой функцией 
логистической регрессии~--- для них требуется меньший объем данных для обучения. 
В~табл.~\ref{Tab1} приведены результаты для лучшей из рассмотренных конфигураций 
по каждой станции в~метриках \verb"ACC"~\eqref{ACC} и~\verb"ROC AUC"~\eqref{AUC}. 

В табл.~2 приведены усредненные сразу по всем вариантам 
параметров значения точности для каж\-дой модели.

В данной задаче модель \verb"XGBoost_Logistic" продемонстрировала в~среднем 
несколько более высокие (порядка $1\%\mbox{--}3\%$) показатели точ\-ности по сравнению
 с~\verb"CatBoost_Logistic". Однако, как видно из табл.~\ref{Tab1}, для шести 
 из~$14$  станций более успешным оказалось применение категориального бустинга.

\subsection{Регрессия}

В данном случае в~качестве входных параметров используются величины~$\{\hat{x}^e_t\}$, 
а~на выходе формируется набор~$\{y^{\mathrm{pred}}_t\}$ (см.\ разд.~2). 
При этом результирующий класс имеет следующий вид: $\{ c^{\mathrm{pred}}_t \hm=
\mathcal {I} (y^{\mathrm{pred}}_t\hm\geqslant \varepsilon)\}$. 
В~качестве моделей регрессии используются \verb"XGBoost" и~\verb"CatBoost" 
с~различными целевыми функциями. В~\verb"XGBoost_RMSE" и~\verb"CatBoost_RMSE" 
используется \verb"RMSE"~\eqref{RMSE}, \verb"CatBoost_MAE"~--- 
средняя абсолютная ошибка (\verb"MAE"), а~в~\verb"XGBoost_Gamma"~--- 
гам\-ма-ре\-грес\-сия с~логарифмической связью. 
В~табл.~\ref{Tab3} приведены результаты лучшей модели для каждой станции 
в~метриках \verb"ACC"~\eqref{ACC} и~\verb"RMSE"~\eqref{RMSE}.

\setcounter{table}{3}
\begin{table*}\small %tabl4
\begin{center}
\Caption{\label{Tab4}  Сравнение результатов гибридных моделей}
\vspace*{2ex}

\begin{tabular}{|l|c|l|c|c|}
\hline
\multicolumn{1}{|c|}{{\bf Город}} & 
\multicolumn{1}{c|}{{\bf Станция}} & 
\multicolumn{1}{c|}{{\bf Лучшая модель}} & 
{\bf ACC}& {\bf RMSE} \\
\hline
      Берлин &   \hphantom{9}93850 &  \verb"XGBoost_Logistic+RMSE+Sigmoid" &  $91{,}67\%$ &  $0{,}0588$ \\
      Берлин &  103810 &          \verb"XGBoost_Logistic+RMSE" & $ 75\%$\hphantom{,99} &  $0{,}1012$ \\
   Нойруппин &   \hphantom{9}92700 &          \verb"XGBoost_Logistic+RMSE" &  $72{,}22 \%$& $ 0{,}0556$ \\
   Нойруппин &  102700 &  \verb"XGBoost_Logistic+RMSE+Sigmoid" &  $69{,}44 \%$&  $0{,}0593$ \\
     Потсдам &   \hphantom{9}93790 &  \verb"XGBoost_Logistic+RMSE+Sigmoid" &  $69{,}44\%$ &  $0{,}0697$ \\
     Потсдам &  103790 &  \verb"XGBoost_Logistic+RMSE+Sigmoid" & $69{,}44\%$ &  $0{,}0777$ \\
  Визенбург &  103680 &          \verb"XGBoost_Logistic+RMSE" &  $66{,}7\%$\hphantom{9} &  $0{,}1418$ \\
  Виттенберг &   \hphantom{9}94740 &  \verb"XGBoost_Logistic+RMSE+Sigmoid" &  $80{,}56\%$ &  $0{,}052$\hphantom{9} \\
\hline
\end{tabular}
\end{center}
\end{table*}


\section{Гибридные модели}

Сравнение результатов точности описанных в~предыдущем разделе моделей показывает, 
что модели регрессии гораздо хуже справляются с~задачей классификации. 
В~этом разделе предложены гиб\-рид\-ные модели, которые сочетают преимущества 
обоих подходов.

\begin{description}
\item[XGBoost\_Logistic+Mean.] Простую комбинированную модель можно получить, 
совместив выход классификатора~$\{c^{\mathrm{cls}}_t\}$ и~среднее значение объемов 
осадков $m \hm= |T|^{-1}\sum\nolimits_{t \in T} y_t$. Тогда выход 
комбинированной модели определим как $\{y^{\mathrm{pred}}_t \hm= c^{\mathrm{cls}}_t m\}$.
\item[XGBoost\_Logistic+RMSE.] Здесь вместо среднего значения~$m$ 
будет использоваться выход прос\-то\-го регрессора~$\{y^{\mathrm{reg}}_t\}$. Тогда выход 
новой модели определим как $\{y^{\mathrm{pred}}_t \hm= 
c^{\mathrm{cls}}_t y^{\mathrm{reg}}_t\}$.
\item[XGBoost\_Logistic+RMSE+Sigmoid.] Пусть теперь выход 
классификатора~$\{p^{\mathrm{cls}}_t\}$ 
определяет вероятность принадлежности к~положительному классу. Введем функцию 
связи вида 
$$
s(x) = \left(1 \hm+ e^{-\alpha (x \hm- \beta)}\right)^{-1},
$$ 
где~$\alpha$ и~$\beta$~--- некоторые заданные действительные коэффициенты. 
Она позволяет использовать вместо бинарного решения~$\{c^{\mathrm{cls}}_t\}$ набор 
непрерывных значений, а~кроме того, предоставляет возможность гибкого 
подбора коэффициентов, наиболее подходящих для конкретных данных. Выход 
такой модели определим как $\{y^{\mathrm{pred}}_t \hm= s\left(p^{\mathrm{cls}}_t\right) 
 y^{\mathrm{reg}}_t\}$.  Отметим, что для рассматриваемых в~статье данных 
наилучшие результаты были получены для конфигураций 
с~$\alpha \hm= 10$ и~$\beta\hm = 0{,}45$.
\end{description}

В табл.~\ref{Tab4} приведены станции, для которых были улучшены 
результаты чисто регрессионных моделей (см.\ табл.~\ref{Tab3}). 
Отметим, что этого удалось достичь для~8 из~14~станций, причем увеличение 
точ\-ности классификации составило от~2\% до~8\% в~зависимости от местоположения. 
Таким образом, использование гибридных моделей оказывается вполне оправданным 
с~точки зрения более точной обработки данных и~корректного заполнения пропусков в~них.



В табл.~5 приведены усредненные сразу по всем вариантам 
параметров значения точности как для чисто регрессионных, так и~для гибридных 
моделей в~порядке возрастания величины \verb"ACC"~\eqref{ACC}. 
Кроме того, для сравнения представлена простая модель заполнения 
средним значением, обозначенная как \verb"Mean".



Отметим, что и~величины ошибок \verb"RMSE" для непрерывных 
значений также остаются весьма умеренными.

\vspace*{6pt}

%\begin{table*}
 %tabl5

\noindent
{{\tablename~5}\ \ \small{Усредненные результаты точности моделей 
регрессии, включая гибридные}}
%\vspace*{2ex}

{\small
\begin{center}
%\tabcolsep=5.5pt
\begin{tabular}{|l|c|c|}
\hline
& \multicolumn{2}{c|}{\bf Метрика}\\
\cline{2-3}
\multicolumn{1}{|c|}{{\raisebox{6pt}[0pt][0pt]{{\bf Модель}}}} 
 & {\bf ACC}& {\bf RMSE}\\
\hline
\verb"Mean"  &  $45{,}24\%$ &  $0{,}0801$ \\
\verb"CatBoost_RMSE"                 &  $47{,}02\%$ &  $0{,}0756$ \\
\verb"XGBoost_RMSE"                  &  $58{,}73\%$ &  $0{,}0759$ \\
\verb"CatBoost_MAE"                  &  $60{,}91\%$ &  $0{,}0804$ \\
\verb"XGBoost_Gamma"                 &  $70{,}83\%$ &  $0{,}0877$ \\
\verb"XGBoost_Logistic+Mean"         &  $72{,}22\%$ &  $0{,}0808$ \\
\verb"XGBoost_Logistic+RMSE"         &  $72{,}42\%$ &  $0{,}0787$ \\
\verb"XGBoost_Logistic+RMSE+Sigmoid" &  $74{,}4\%$\hphantom{9} &  $0{,}0763$ \\
\hline
\end{tabular}
\end{center}
}
%\end{table*}

\section{Заключение}

В работе продемонстрирована возможность высокоточного заполнения 
пропусков в~данных с~использованием гибридных моделей градиентного\linebreak бустинга. 
Достигнутая точность в~задачах классификации составила до~92\% 
при весьма умеренных значениях ошибок прогнозов в~метрике \verb"RMSE". 
Гибридные методы превзошли по качеству предсказания как простые модели 
классификации, так и~регрессии.

Было проведено сравнение методов библиотек \verb"XGBoost" и~\verb"CatBoost", 
которое показало, что в~большинстве случаев для работы с~осадками более 
перспективным представляется использование экс\-тремального градиентного бустинга, однако для отдельных рядов результаты могут быть улучшены за счет применения категориальных моделей. Развиваемые подходы могут быть успешно использованы как для непосредственного анализа метеорологических данных методами машинного обучения, так и~для улучшения качества предсказания на основе физических моделей атмосферных процессов.

\bigskip
Авторы выражают особую признательность профессору В.\,Ю.~Королеву 
за полезные обсуждения в~рамках совместных исследований 
метеорологических явлений.


\vspace*{-6pt}

{\small\frenchspacing
 {%\baselineskip=10.8pt
 \addcontentsline{toc}{section}{References}
 \begin{thebibliography}{99}
 
\vspace*{-3pt}
 
\bibitem{Gorshenin2018c} 
\Au{Горшенин~А.\,К., Королев~В.\,Ю.} 
Определение экстремальности объемов осадков на основе модифицированного 
метода превышения порогового значения~// Информатика и~её применения, 2018. Т.~12. 
Вып.~4. C.~16--24.

\bibitem{Friedman2001}
\Au{Friedman~J.\,H.} 
Greedy function approximation: A~gradient boosting machine~// Ann. Stat., 
2001. Vol.~29. Iss.~5. P.~1189--1232.

\bibitem{Chen2016} 
\Au{Chen~T., Guestrin~C.} XGBoost: A~scalable tree boosting system~// 
22nd ACM SIGKDD  Conference (International) on Knowledge Discovery and Data Mining
Proceedings.~--- San Francisco, CA, USA, 2016. P.~785--794.

\bibitem{Mustapha2016}
\Au{Mustapha~I.\,B., Saeed~F.} 
Bioactive molecule prediction using extreme gradient boosting~// 
Molecules, 2016. Vol.~21. Iss.~8. Art.~No.\,983.

\bibitem{Xia2017}
\Au{Xia~Y., Liu~C., Li~Y., Liu~N.} 
A~boosted decision tree approach using Bayesian hyper-parameter optimization 
for credit scoring~// Expert Syst. Appl., 2017. Vol.~78. P.~225--241.

\bibitem{Chatzis2018}
\Au{Chatzis~S.\,P., Siakoulis~V., Petropoulos~A., Stavroulakis~E., Vlachogiannakis~N.}
 Forecasting stock market crisis events using deep and statistical machine 
 learning techniques~// Expert Syst. Appl., 2018. Vol.~112. P.~353--371.

\bibitem{Zhang2018}
\Au{Zhang~D., Qian~L., Mao~B., Huang~C., Huang~B., Si~Y.}
 A~data-driven design for fault detection of wind turbines using random
forests and XGboost~// IEEE Access, 2018. Vol.~6. P.~21020-21031.

\bibitem{Aler2017} 
\Au{Aler~R., Galvan~I.\,M., Ruiz-Arias~J.\,A., Gueymard~C.\,A.} 
Improving the separation of direct and diffuse solar radiation 
components using machine learning by gradient boosting~// Sol. Energy, 2017. 
Vol.~150. P.~558--569.

\bibitem{Torres-Barran2018} 
\Au{Torres-Barran~A., Alonso~A., Dorronsoro~J.\,R.}
 Regression tree ensembles for wind energy and solar radiation prediction~// 
 Neurocomputing, 2018. Vol.~326. P.~151--160.

\bibitem{Prokhorenkova2018}
\Au{Prokhorenkova~L., Gusev~G., Vorobev~A., Dorogush~A.\,V., Gulin~A.} 
CatBoost: Unbiased boosting with categorical features~// 
Adv. Neur. In., 2018. Vol.~31. P.~6638--6648.

\bibitem{Ivanov2019}
\Au{Ivanov~M.\,V., Levitsky~L.\,I., Bubis~J.\,A., Gorshkov~M.\,V.} 
Scavager: A~versatile postsearch validation algorithm for shotgun proteomics 
based on gradient boosting~// Proteomics, 2019. Vol.~19. Iss.~3. Art.~No.\,1800280.

\bibitem{Punmiya2019}
\Au{Punmiya~R., Choe~S.} Energy theft detection using gradient boosting 
theft detector with feature boost
engineering-based preprocessing~// IEEE T.~Smart Grid, 2019. Vol.~10. 
Iss.~2. P.~2326--2329.

\bibitem{Korner2018}
\Au{Korner~P., Kronenberg~R., Genzel~S., Bernhofer~C.} 
Introducing Gradient Boosting as a~universal gap filling tool 
for meteorological time series~// Meteorol.~Z., 2018. Vol.~27. 
Iss.~5. P.~369--376.

\bibitem{Fan2018}
\Au{Fan~J., Wang~X., Wu~L., Zhou~H., Zhang~F., Yu~X., Lu~X., Xiang~Y.} 
Comparison of Support Vector Machine and Extreme Gradient Boosting for 
predicting daily global solar radiation using temperature and precipitation 
in humid subtropical climates: A case study in China~// 
Energ. Convers. Manage., 2018. Vol.~164. P.~102--111.

\bibitem{Christ2018} 
\Au{Christ~M., Braun~N., Neuffer~J., Kempa-Liehr~A.\,W.} 
Time Series FeatuRe Extraction on basis of Scalable Hypothesis tests 
(tsfresh~--- a~Python package)~// Neurocomputing, 2018. Vol.~307. P.~72--77.

\bibitem{Huang2005} 
\Au{Huang~J., Ling~C.\,X.} Using AUC and accuracy in evaluating learning algorithms~// 
IEEE T.~Knowl. Data En., 2005. Vol.~17. Iss.~3. P.~299--310.
  
\bibitem{Gorshenin2018a}
\Au{Gorshenin~A.\,K., Korolev~V.\,Yu.} Scale mixtures of
Frechet distributions as asymptotic approximations of extreme precipitation~// 
J.~Math. Sci., 2018. Vol.~234. Iss.~6. P.~886--903.

 \end{thebibliography}

 }
 }

\end{multicols}

\vspace*{-6pt}

\hfill{\small\textit{Поступила в~редакцию 08.07.19}}

\vspace*{8pt}

%\pagebreak

%\newpage

%\vspace*{-28pt}

\hrule

\vspace*{2pt}

\hrule

%\vspace*{-2pt}

\def\tit{HYBRID EXTREME GRADIENT BOOSTING MODELS TO~IMPUTE THE~MISSING DATA 
IN~PRECIPITATION RECORDS}


\def\titkol{Hybrid extreme gradient boosting models to~impute the~missing data 
in~precipitation records}

\def\aut{A.\,K.~Gorshenin$^{1,2}$ and~O.\,P.~Martynov$^2$}

\def\autkol{A.\,K.~Gorshenin and~O.\,P.~Martynov}

\titel{\tit}{\aut}{\autkol}{\titkol}

\vspace*{-11pt}


\noindent
$^1$Institute of Informatics Problems, Federal Research Center ``Computer Science and
Control'' of the Russian\linebreak
$\hphantom{^1}$Academy of Sciences, 44-2~Vavilov Str., Moscow 119333, Russian
Federation

\noindent
$^2$Faculty of Computational Mathematics and Cybernetics, M.\,V.~Lomonosov Moscow
State University, GSP-1,\linebreak
$\hphantom{^1}$Leninskie Gory, Moscow, 119991, Russian Federation


\def\leftfootline{\small{\textbf{\thepage}
\hfill INFORMATIKA I EE PRIMENENIYA~--- INFORMATICS AND
APPLICATIONS\ \ \ 2019\ \ \ volume~13\ \ \ issue\ 3}
}%
 \def\rightfootline{\small{INFORMATIKA I EE PRIMENENIYA~---
INFORMATICS AND APPLICATIONS\ \ \ 2019\ \ \ volume~13\ \ \ issue\ 3
\hfill \textbf{\thepage}}}

\vspace*{3pt} 



\Abste{The article compares the classical method of extreme gradient boosting 
implemented in the {\sf XGBoost} (eXtreme Gradient Boosting) framework with the new modification 
{\sf CatBoost} (Categorial Boosting),
 which is rarely involved in scientific researches. Some hybrid 
 classification-regression models are proposed to improve the accuracy 
 of imputation in missing values in real data using~14~meteorological stations 
 in Germany. The achieved accuracy of the classification is up to~92\% 
 and the root-mean-square errors are quite moderate. The hybrid methods outperformed 
 both simple classification and regression models in prediction accuracy. 
 The proposed approaches can be successfully used for meteorological data 
 analysis by machine learning methods as well as for improving the forecasting 
 accuracy in physical models of atmospheric processes.}


\KWE{data imputation; precipitation; classification; regression; gradient boosting; 
XGBoost; CatBoost}



\DOI{10.14357/19922264190306} 

\vspace*{-14pt}

\Ack
\noindent
The problems formulation and analysis of results through the paper were 
performed by A.\,K.~Gorshenin whose research was supported by the 
Russian Science Foundation (project 18-71-00156). 
Machine learning algorithms for imputation of missing values were 
implemented by BSc student O.\,P.~Martynov.


%\vspace*{-6pt}

  \begin{multicols}{2}

\renewcommand{\bibname}{\protect\rmfamily References}
%\renewcommand{\bibname}{\large\protect\rm References}

{\small\frenchspacing
 {%\baselineskip=10.8pt
 \addcontentsline{toc}{section}{References}
 \begin{thebibliography}{99}
\bibitem{1-gm}
\Aue{Gorshenin,~A.\,K., and V.\,Yu.~Korolev.}
 2018. {Opredelenie ekstremal'nosti ob''emov osadkov na osnove 
 mo\-di\-fi\-tsi\-ro\-van\-no\-go metoda prevysheniya porogovogo znacheniya} 
 [Determining the extremes of precipitation volumes  based on a modified ``Peaks
  over Threshold'']. \textit{Informatika i~ee  Primeneniya~--- Inform. Appl.} 
  12(4):16--24. 

\bibitem{2-gm}
\Aue{Friedman,~J.\,H.} 2001.
 Greedy function approximation: A~gradient boosting machine. 
 \textit{Ann. Stat.} 29(5):1189--1232.

\bibitem{3-gm}
\Aue{Chen,~T., and C.~Guestrin.} 2016. XGBoost: A~scalable tree boosting system. 
\textit{22nd ACM SIGKDD  Conference (International) on Knowledge Discovery 
and Data Mining Proceedings}. San Francisco, CA. 785--794.

\bibitem{4-gm}
\Aue{Mustapha,~I.\,B., and F.~Saeed.} 2016. 
Bioactive molecule prediction using extreme gradient boosting. 
\textit{Molecules} 21(8):983.

\bibitem{5-gm}
\Aue{Xia,~Y., C.~Liu, Y.~Li, and N.~Liu.} 
2017. A~boosted decision tree approach using Bayesian hyper-parameter optimization 
for credit scoring. \textit{Expert Syst. Appl.} 78:225--241.

\bibitem{6-gm}
\Aue{Chatzis,~S.\,P., V.~Siakoulis, A.~Petropoulos, E.~Stav\-rou\-lakis, 
and N.~Vlachogiannakis.} 2018. Forecasting stock market crisis events 
using deep and statistical machine learning techniques. 
\textit{Expert Syst. Appl.} 112:353--371.

\bibitem{7-gm}
\Aue{Zhang,~D., L.~Qian, B.~Mao, C.~Huang, B.~Huang, and Y.~Si.}
 2018. A data-driven design for fault detection of wind turbines 
 using random forests and XGboost. 
 \textit{IEEE Access} 6:21020--21031.

\bibitem{8-gm}
\Aue{Aler,~R., I.\,M.~Galvan, J.\,A.~Ruiz-Arias, and C.\,A.~Gueymard.}
 2017. Improving the separation of direct and diffuse solar radiation 
 components using machine learning by gradient boosting. 
 \textit{Sol. Energy} 150:558--569.

\bibitem{9-gm}
\Aue{Torres-Barran,~A., A.~Alonso, and J.\,R.~Dorronsoro.} 
2018. Regression tree ensembles for wind energy and solar radiation prediction. 
\textit{Neurocomputing} 326:151--160.

\bibitem{10-gm}
\Aue{Prokhorenkova,~L., G.~Gusev, A.~Vorobev, A.\,V.~Dorogush, and A.~Gulin.}
 2018. CatBoost: Unbiased boosting with categorical features. 
 \textit{Adv. Neur. In.} 31:6638--6648.

\bibitem{11-gm}
\Aue{Ivanov,~M.\,V., L.\,I.~Levitsky, J.\,A.~Bubis, and M.\,V.~Gorshkov.}
 2019. Scavager: A~versatile postsearch validation algorithm for shotgun 
 proteomics based on gradient boosting. \textit{Proteomics} 19(3):1800280.

\bibitem{12-gm}
\Aue{Punmiya,~R., and S.~Choe.} 2019. 
Energy theft detection using gradient boosting theft detector with feature 
boost engineering-based preprocessing. 
\textit{IEEE T.~Smart Grid} 10(2):2326--2329.

\bibitem{13-gm}
\Aue{Korner,~P., R.~Kronenberg, S.~Genzel, and C.~Bernhofer.}
 2018. Introducing Gradient Boosting as a~universal gap filling tool 
 for meteorological time series. \textit{Meteorol.~Z.} 27(5):369--376.

\bibitem{14-gm}
\Aue{Fan,~J., X.~Wang, L.~Wu, H.~Zhou, F.~Zhang, X.~Yu, X.~Lu, and Y.~Xiang.}
 2018. Comparison of Support Vector Machine and Extreme Gradient Boosting
  for predicting daily global solar radiation using temperature and 
  precipitation in humid subtropical climates: A~case study in China. 
\textit{Energ. Convers.  Manage.} 164:102--111.

\bibitem{15-gm}
\Aue{Christ,~M., N.~Braun, J.~Neuffer, and A.\,W.~Kempa-Liehr.}
 2018. Time Series FeatuRe Extraction on basis of Scalable Hypothesis 
 tests (tsfresh~--- a~Python package). \textit{Neurocomputing} 307:72--77.
  
\bibitem{16-gm}
\Aue{Huang,~J., and C.\,X.~Ling.} 2005. 
Using AUC and accuracy in evaluating learning algorithms. 
\textit{IEEE T.~Knowl. Data En.} 17(3):299--310.

\bibitem{17-gm}
\Aue{Gorshenin,~A.\,K., and V.\,Yu.~Korolev.}
 2018. Scale mixtures of Frechet distributions as asymptotic approximations 
 of extreme precipitation. \textit{J.~Math. Sci.} 234(6):886--903.
\end{thebibliography}

 }
 }

\end{multicols}

%\vspace*{-7pt}

\hfill{\small\textit{Received July 8, 2019}}

%\pagebreak

%\vspace*{-22pt}

\Contr


\noindent
\textbf{Gorshenin Andrey K.} (b.\ 1986)~--- 
Candidate of Science (PhD) in physics and
mathematics, associate professor, leading scientist, Institute of Informatics Problems,
Federal Research Center ``Computer Science and Control'' of the Russian Academy of
Sciences, 44-2~Vavilova Str., Moscow 119333, Russian Federation;  leading scientist, Faculty
of Computational Mathematics and Cybernetics, Lomonosov Moscow State 
University, GSP-1,
Leninskie Gory, Moscow, 119991, Russian Federation; \mbox{agorshenin@frccsc.ru}

\vspace*{3pt}

\noindent
\textbf{Martynov Oleg P.} (b.\ 1996)~--- BSc student,  Faculty
of Computational Mathematics and Cybernetics, 
M.\,V.~Lomonosov Moscow State University, GSP-1,
Leninskie Gory, Moscow, 119991, Russian Federation; 
\mbox{martynov.oleg.mipt@gmail.com}
\label{end\stat}

\renewcommand{\bibname}{\protect\rm Литература}   %6
\def\stat{bosov+stef}

\def\tit{УПРАВЛЕНИЕ ВЫХОДОМ СТОХАСТИЧЕСКОЙ ДИФФЕРЕНЦИАЛЬНОЙ СИСТЕМЫ 
ПО~КВАДРАТИЧНОМУ КРИТЕРИЮ. I.~ОПТИМАЛЬНОЕ РЕШЕНИЕ МЕТОДОМ 
ДИНАМИЧЕСКОГО ПРОГРАММИРОВАНИЯ$^*$}

\def\titkol{Управление выходом стохастической дифференциальной системы 
по~квадратичному критерию. I}
%.~Оптимальное решение методом 
%динамического программирования}

\def\aut{А.\,В.~Босов$^1$, А.\,И.~Стефанович$^2$}

\def\autkol{А.\,В.~Босов, А.\,И.~Стефанович}

\titel{\tit}{\aut}{\autkol}{\titkol}

\index{Босов А.\,В.}
\index{Стефанович А.\,И.}
\index{Bosov A.\,V.}
\index{Stefanovich A.\,I.}




{\renewcommand{\thefootnote}{\fnsymbol{footnote}} \footnotetext[1]
{Работа выполнена при частичной поддержке РФФИ (проект 16-07-00677).}}


\renewcommand{\thefootnote}{\arabic{footnote}}
\footnotetext[1]{Институт проблем информатики Федерального исследовательского центра <<Информатика 
и~управление>> Российской академии наук, \mbox{AVBosov@ipiran.ru}}
\footnotetext[2]{Институт проблем информатики Федерального исследовательского центра <<Информатика 
и~управление>> Российской академии наук, \mbox{AStefanovich@frccsc.ru}}

%\vspace*{8pt}



  
  \Abst{Решается задача оптимального управления для диффузионного процесса 
Ито и~линейного управ\-ля\-емо\-го выхода. Рассматриваемая постановка близка 
к~классической ли\-ней\-но-квад\-ра\-тич\-ной гауссовской задаче управления 
(linear-quadratic Gaussian (LQG) control). Отличия состоят в~том, что состояние описывается нелинейным 
дифференциальным уравнение Ито $dy_t\hm= A_t(y_t) \,dt\hm+ \Sigma_t(y_t)\,dv_t$ 
и~не зависит от управ\-ле\-ния~$u_t$, оптимизации подлежит управ\-ля\-емый 
линейный выход $dz_t\hm= a_t y_t\,dt\hm+ b_t z_t \,dt\hm+ c_t u_t \,dt\hm+ \sigma_t\, 
dw_t$. Дополнительные обобщения внесены в~квад\-ра\-тич\-ный критерий качества 
с~целью воз\-мож\-ности постановки таких задач, как отслеживание выходом 
состояния или управ\-ле\-ни\-ем~--- линейной комбинации состояния и~выхода. Для 
решения используется метод динамического программирования. Функцию 
Беллмана позволяет найти предположение о~ее структуре вида $V_t(y,z)\hm= 
\alpha_t z^2\hm+ \beta_t(y)z \hm+\gamma_t(y)$. Решение дают три 
дифференциальных уравнения для коэффициентов~$\alpha_t$, $\beta_t(y)$ 
и~$\gamma_t(y)$. Эти уравнения со\-став\-ля\-ют оптимальное решение 
рас\-смат\-ри\-ва\-емой задачи.}
  
  \KW{стохастическое дифференциальное уравнение; оптимальное управ\-ле\-ние; 
динамическое программирование; функция Беллмана; уравнение Риккати; 
линейные уравнения параболического типа}

\DOI{10.14357/19922264180314}
  
%\vspace*{4pt}


\vskip 10pt plus 9pt minus 6pt

\thispagestyle{headings}

\begin{multicols}{2}

\label{st\stat}

\section{Введение}

     Ключевые результаты в~области оптимизации стохастических 
динамических систем, со\-став\-ля\-ющие классическую теорию управления, 
получены более~40~лет назад (такова работа~[1] в~отношении задачи 
управ\-ле\-ния ли\-ней\-но-гаус\-сов\-ски\-ми стохастическими сис\-те\-ма\-ми по 
квад\-ра\-тич\-но\-му критерию). К~классической тео\-рии следует относить 
линейные модели стохастических сис\-тем и~квадратичный критерий качества. 
Это исходный базис, на котором основано множество успешно 
исследованных и~решенных задач стохастического управ\-ле\-ния 
и~оптимизации. 

Дальнейшее развитие~--- это новые модели и~критерии, но 
прежде всего это новые методы: от тео\-рии линейных регуляторов, метода 
динамического программирования и~принципа максимума к~адаптивному 
и~минимаксному подходу, импульсному управ\-ле\-нию и~т.\,д. Множество 
инноваций как в~час\-ти моделей, так и~в~час\-ти математического аппарата, 
имевших мес\-то в~по\-сле\-ду\-ющие годы, существенно обогатили тео\-рию 
управ\-ле\-ния. Но и~до настоящего времени линейные модели и~квадратичный 
критерий, несмотря на всю справедливую критику в~отношении их 
аде\-кват\-ности и~гиб\-кости, сохраняют исследовательский интерес и~находят 
современные области приложения.
     
     Не претендуя на сколь\-ко-ни\-будь полное обосно\-ва\-ние последнего 
тезиса, приведем несколько примеров, показавшихся наиболее ин\-те\-рес\-ными. 

Так, в~[2] решается ли\-ней\-но-квад\-ра\-тич\-ная за\-да\-ча в~игровой 
постановке с~запаздыванием. В~близ\-кой по модели работе~[3] задача 
управ\-ле\-ния ставится в~терминах $H_\infty$-ро\-баст\-ности. Точнее \mbox{называть} 
эту тематику $H_2/H_\infty$-управ\-ле\-ни\-ем, и~работ по этой теме очень 
много. Аккуратности ради следует уточнить, что под линейными 
понимаются модели с~мультипликативными по состоянию воз\-му\-ще\-ниями. 

Совсем другой класс моделей, особо популярных в~по\-след\-ние годы, 
составляют скачкообразные процессы. Например, линейные уравнения 
в~сочетании с~пуассоновскими скачками в~[4] используются в~моделях, 
описывающих различные показатели функционирования сетевых протоколов 
передачи данных транспортного уровня. Телекоммуникации представляют 
в~последние годы самый популярный прикладной материал для 
исследований, работ по этой проб\-ле\-ма\-ти\-ке множество, математические 
техники привлекаются самые разные и~самые современные, но и~линейным 
моделям место находится. Еще один любопытный пример исследования 
скачкообразного процесса и~оптимизации на основе квад\-ра\-тич\-но\-го критерия 
можно найти в~[5] применительно к~задаче инвестирования на финансовом 
рынке. Наконец, упомянем еще работу~[6], подводящую итог исследований 
в~отношении классической детерминированной  
ли\-ней\-но-квад\-ра\-тич\-ной задачи с~использованием техники матричных 
неравенств.
     
     В данной работе также эксплуатируются привлекательные свойства 
линейных моделей и~квад\-ра\-тич\-но\-го критерия, причем в~стохастической 
постановке. На\-прав\-ле\-ни\-ем для обобщения \mbox{выбрана} модель динамики 
сис\-те\-мы: основные усилия на\-прав\-ле\-ны на то, чтобы сделать ее нелинейной. 
Кроме того, пред\-став\-лен\-ная постановка может рас\-смат\-ри\-вать\-ся и~как 
обобщение ранее решенной задачи в~дискретном времени~[7, 8] на время 
непрерывное. В~упомянутых работах помимо собственно модельной 
постановки важна еще и~привлекаемая прикладная об\-ласть~--- 
функционирование сложных программных сис\-тем. Результатов, 
ориентированных непосредственно на такие приложения, к~настоящему 
времени пренебрежимо мало, поэтому~[7, 8]~--- это еще и~прикладное 
обоснование рас\-смат\-ри\-ва\-емой далее задачи.
     
     Оптимизируемая динамическая сис\-те\-ма описывается двумя 
уравнениями. Состояние задается нелинейным стохастическим 
дифференциальным уравнением Ито, не содержащим управ\-ля\-емой 
переменной. Возмущение здесь описывается стандартным винеровским 
процессом, накладываются простые условия существования 
и~един\-ст\-вен\-ности решения. Поскольку состояние не управ\-ля\-ет\-ся, то уместно 
его интерпретировать как слож\-ное внешнее возмущение. Вторая 
переменная~--- управ\-ля\-емый выход~--- задается линейным стохастическим 
дифференциальным уравнением. Цель оптимизации выхода формируется 
квадратичным критерием общего вида. Формальная постановка задачи 
приведена в~сле\-ду\-ющем разделе.
     
     Для решения задачи используется метод динамического 
программирования, решается уравнение Беллмана~[9]. Соответственно, 
в~результате получаются аналитические выражения и~для оптимального 
управ\-ле\-ния, и~для значения функционала качества. Технически 
традиционный, стандартный подход к~задаче обременен, пожалуй, 
единственной проблемой~--- поиском верного пред\-став\-ле\-ния структуры 
функции Беллмана. Справиться с~этой проблемой в~большей степени удается 
за счет результата, полученного при решении дискретного по времени 
аналога рассматриваемой постановки~\cite{8-bos}. Конечные соотношения 
для оптимального решения, как и~во всех подобных задачах, включая 
классическую ли\-ней\-но-квад\-ра\-тич\-ную, содержат решения 
определенных дифференциальных уравнений (обыкновенных и~в~частных 
производных). Вывод этих уравнений и~со\-став\-ля\-ет содержание первой час\-ти 
данной работы. Во второй части будет обсуждаться их приближенное 
чис\-лен\-ное решение и~компьютерные эксперименты.
     
     Кратко обозначим основные положения, при\-вле\-ка\-емые далее 
к~решению задачи, следуя в~основном обозначениям 
и~терминологии~\cite{9-bos}, а~именно: будем рассматривать задачу 
оптимального управления в~стохастической динамической сис\-те\-ме по полной 
информации, применяя метод динамического программирования. В~качестве 
целевого функционала, опре\-де\-ля\-юще\-го качество управ\-ле\-ния $U_0^T\hm= \{ 
u_t,\ 0\leq t\leq T\}$, выступает
     \begin{equation}
     J\left(U_0^T\right)={\sf E}\left\{ \int\limits_0^T L_t \left(x_t, u_t\right)\,dt+ 
l\left(x_T\right)\right\}\,.
     \label{e1-bos}
     \end{equation}
Здесь ${\sf E}\{\cdot\}$~--- оператор математического ожидания; $x_t$~--- 
случайный процесс, описываемый стохастическим дифференциальным 
уравнением Ито
     \begin{equation}
     dx_t=m_t\left( x_t, u_t\right) dt+ \sigma_t\left( x_t\right)dW_t\,,\enskip 
x_0=X\,,
     \label{e2-bos}
     \end{equation}
где $W_t$~--- стандартный винеровский процесс подходящей раз\-мер\-ности; 
$X$~--- случайный вектор.

     $U_0^T$ будем выбирать из класса допустимых неупреждающих (по 
отношению к~$W_t$) управлений~\cite{9-bos}. Соответственно, 
относительно функций сноса и~диффузии~$m_t$ и~$\sigma_t$  
в~(\ref{e2-bos}) будем предполагать выполненными ка\-кие-ли\-бо условия 
существования сильного решения для заданного до\-пус\-ти\-мо\-го управ\-ле\-ния. 
Например, для управ\-ле\-ния с~обратной связью $u_t\hm= u_t(x_t)$ будем 
считать, что $m_t(x,u_t(x))$ и~$\sigma_t(x)$ удовлетворяют условию 
линейного рос\-та и~локальному условию Липшица по~$x$ равномерно 
по~$t$ (т.\,е.\ условиям Ито).
     
     Для поиска оптимального управления, минимизирующего $J(U_0^T)$, 
рас\-смат\-ри\-ва\-ет\-ся функция Беллмана
     \begin{equation}
     V_t(x)=\left.\mathop{\mathrm{inf}}\limits_{U_t^T} {\sf E} \left\{ \int\limits_t^T 
L_t \left( x_t, u_t\right)\,dt+l\left( x_T\right) \right\vert \mathcal{F}_t^x\right\}\,,
     \label{e3-bos}
     \end{equation}
где $\mathcal{F}_t^x$~--- $\sigma$-ал\-геб\-ра, по\-рож\-ден\-ная~$x_\tau$, 
$0\hm\leq \tau\hm\leq t$, ${\sf E}\{\cdot\vert \mathcal{F}\}$~--- оператор условного 
математического ожидания относительно~$\mathcal{F}$. Соответственно, 
в~качестве достаточного условия оп\-ти\-маль\-ности воспользуемся уравнением 
динамического программирования
\begin{multline}
\fr{\partial V_t(x)}{\partial t} +\fr{1}{2}\sum\limits^n_{i,j=1} \sigma^2_{t_{ij}}
\fr{\partial^2 V_t(x)}{\partial x_i \partial x_j}+{}\\
{}+\min\limits_u\left[  
\sum\limits^n_{i=1} m_{t_i} \fr{\partial V_t(x)}{\partial x_i} + L_t(x,u)\right] 
=0\,,\\
V_T(x)=l(x)\,,
\label{e4-bos}
\end{multline}
где $m_{t_i}$~--- $i$-й элемент век\-тор-функ\-ции~$m_t(x,u)$; 
$\sigma^2_{t_{ij}} \hm= \sum\nolimits^m_{k=1} 
\sigma_{t_{ik}}\sigma_{t_{ki}}$, $\sigma_{t_{ij}}$~--- $i$-й по строке, $j$-й 
по столб\-цу элемент мат\-рич\-ной функции~$\sigma_t(x)$; $n$ и~$m$~--- 
размерности~$x_t$ и~$W_t$ соответственно.

     Традиционно в~рамках применения метода динамического 
программирования будем предполагать, что функции~$L_t$, $l$, $m_t$ 
и~$\sigma_t$ обеспечивают существование хотя бы одного решения 
уравнения~(\ref{e4-bos}), а~следовательно, и~оптимального 
управления~$u_t^*$, $0\hm\leq t\hm\leq T$, до\-став\-ля\-юще\-го минимум 
целевому функционалу~(\ref{e1-bos}). Задача оптимизации далее получается 
путем указания конкретных выражений для~$L_t$, $l$, $m_t$ и~$\sigma_t$.

\section{Постановка задачи управления выходом}

     Рассматриваемые далее случайные функции будут предполагаться 
скалярными. Такое упрощение позволит разгрузить выкладки и~итоговые 
выражения от не самых существенных деталей.
     
     Рассмотрим стохастическую дифференциальную сис\-те\-му, со\-сто\-яние 
которой представляет диффузи\-он\-ный процесс~$y_t$, описываемый 
нелинейным стохастическим дифференциальным уравнением Ито
     \begin{equation}
     dy_t=A_t\left( y_t\right) dt +\Sigma_t \left( y_t\right) dv_t\,,\enskip 
y_0=Y\,,
     \label{e5-bos}
     \end{equation}
где $v_t$~--- стандартный (одномерный) винеровский процесс; $Y$~--- 
случайная величина с~конечным вторым моментом; функции~$A_t$ 
и~$\Sigma_t$ удовлетворяют условиям Ито:
\begin{equation*}
\left\vert A_t(y)\right\vert +\left\vert \Sigma_t(y)\right\vert \leq C(1+\vert y\vert )\ 
\mbox{для\ всех } 0\leq t\leq T\,;
\end{equation*}

\vspace*{-12pt}

\noindent
\begin{multline*}
\hspace*{-2.10051pt}\left\vert A_t\left(y_1\right) -A_t \left( y_2\right) \right\vert +\left\vert 
\Sigma_t\left( y_1\right) -\Sigma_t \left(y_2\right)\right\vert \leq
C\left\vert y_1-y_2\right\vert\\
 \mbox{для\ всех\ } 0\leq t\leq T\ \mbox{и } 
y_1,y_2\in \mathbb{R}^1\,,
\end{multline*}
обеспечивающим существование единственного сильного (потраекторного) 
решения уравнения.
     
     Будем считать, что~$y_t$ описывает состояние некоторой 
динамической системы. Соответственно, поведение этой сис\-те\-мы опишем 
выходом, линейно связанным с~со\-сто\-янием:
     \begin{equation}
     dz_t=a_t y_t \,dt+ b_t z_t \,dt+ c_t u_t \,dt+\sigma_t \,dw_t\,,\enskip
     z_0=Z\,.
     \label{e6-bos}
     \end{equation}
Здесь $w_t$~--- не зависящий от~$v_t$, $Y$ и~$Z$ стандартный (одномерный) 
винеровский процесс; $Z$~--- случайная величина с~конечным вторым 
моментом; $u_t$~--- допустимое неупреждающее управ\-ле\-ние, качество 
которого определяется целевым функционалом следующего вида:
\begin{multline}
\!\hspace*{-3.98538pt}J\left( U_0^T\right) ={\sf E}\left\{ \int\limits_0^T \!\left( S_t\left( s_ty_t-g_t z_t -h_t 
u_t\right)^2 +G_t z_t^2+{}\right.\right.\\
\left.\left.{}+ H_t u_t^2
\vphantom{S_t\left( s_ty_t-g_t z_t -h_t 
u_t\right)^2}
\right) dt+S_T\left( s_T y_T -g_T 
z_T\right)^2+G_T z_T^2
\vphantom{\int\limits_0^T}\right\}\,,
\label{e7-bos}
\end{multline}
где $S_t$, $G_t$ и~$H_t$~--- неотрицательные функции\linebreak
$0\hm\leq t\hm\leq T$. 
Такой критерий отражает физический смысл задачи распределения ресурсов 
со\-глас\-но аналогичной~(\ref{e5-bos})--(\ref{e7-bos}) задаче для дис\-крет\-но\-го 
времени, рас\-смот\-рен\-ной в~\cite{7-bos}. В~част\-ности,  
функци\-онал~(\ref{e7-bos}) поз\-во\-ля\-ет ставить задачи отслеживания
 выходом 
со\-сто\-яния сис\-те\-мы, используя сла\-га\-емое $(y_t\hm- z_t)^2$, или 
управлением~--- линейной комбинации со\-сто\-яния и~выхода, сла\-га\-емое типа\linebreak 
$(y_t\hm+ z_t\hm- u_t)^2$. Поскольку задача формулируется 
в~предположении наличия пол\-ной информации о~со\-сто\-янии~$y_t$ 
и~выходе~$z_t$ (соответствующую $\sigma$-ал\-геб\-ру 
обозначим~$\mathcal{F}_t^{y,z}$), то допустимое управ\-ле\-ние ищется 
в~классе~$\mathcal{F}_t^{y,z}$-из\-ме\-ри\-мых неупреждающих функций 
(и,~как будет показано далее, оказывается управ\-ле\-ни\-ем с~обратной связью).

     Функции~$a_t$, $b_t$, $c_t$ и~$\sigma_t$ будем предполагать 
ограниченными: $\vert a_t\vert \hm+ \vert b_t\vert \hm+\vert c_t\vert \hm+ \vert 
\sigma_t \vert \hm\leq C$ для всех $0\hm\leq t\hm\leq T$, процесс  
управления~--- допустимым не\-упреж\-да\-ющим~\cite{9-bos}, обеспечивая, 
таким образом, существование сильного решения урав\-не\-ния~(\ref{e6-bos}) 
для любого допустимого управ\-ления.
     
     Задачу составляет поиск~$u_t^*$~--- допустимого управ\-ле\-ния, 
доставляющего минимум квад\-ра\-тич\-но\-му функционалу~$J(U_0^T)$.
      
     Поставленная задача очевидным образом формулируется в~терминах 
введенных выше в~(\ref{e1-bos})--(\ref{e3-bos}) обозначений, а~именно: 
     требуется обозначить
     \begin{gather*}
      x_t=\begin{pmatrix}
     y_t\\ z_t\end{pmatrix};\quad  m_t(x_t, u_t)=\begin{pmatrix}
     A_t(y_t)\\ a_t y_t +b_t z_t +c_t u_t\end{pmatrix};\\
     \sigma_t(x_t)= \begin{pmatrix}
     \Sigma_t(y_t)& 0\\
     0& \sigma_t\end{pmatrix};\quad W_t=\begin{pmatrix}
     v_t \\ w_t\end{pmatrix}
     %     \label{e8-bos}
     \end{gather*}
для записи уравнения со\-сто\-яния типа~(\ref{e2-bos}) и
\begin{align*}
L_t(x,u)&= L_t(y,z,u) ={}\\
&\hspace*{3mm}{}=S_t\left( s_t y-g_t z -h_t u\right)^2 +G_t z^2 +H_t  u^2\,;\\
l(x)&= l(y,z) =S_T \left( S_T y-g_T z\right)^2 +G_T z^2
%\label{e9-bos}
\end{align*}
для записи целевого функционала в~виде~(\ref{e1-bos}).

     Функция Беллмана~(\ref{e3-bos}) принимает вид 
     $V_t(x)\hm= V_t(y,z)$. Для записи со\-от\-вет\-ст\-ву\-юще\-го~(\ref{e4-bos}) 
уравнения Беллмана для~$V_t(y,z)$ заметим, что
     $$
     \left( \sigma^2_{t_{ij}}\right)_{i,j=1,2}= \begin{pmatrix}
     \Sigma_t^2(y) & 0\\
     0 & \sigma_t^2\end{pmatrix}\,.
     $$
     
     С~учетом перечисленных обозначений урав\-не\-ние динамического 
программирования~(\ref{e4-bos}) принимает вид:
     \begin{multline}
     \fr{\partial V_t(y,z)}{\partial t} +\fr{1}{2}\left( \Sigma_t^2(y) \fr{\partial^2 
V_t(y,z)} {\partial y^2}+\sigma_t^2\fr{\partial^2 V_t(y,z)} {\partial 
z^2}\right)+{}\\
    {}+\min\limits_u\! \left[ A_t(y) \fr{\partial V_t(y,z)}{\partial y}+\left( a_t 
y+b_t z+c_t u\right) \fr{\partial V_t(y,z)}{\partial z} +{}\right.\hspace*{-3pt}\\
\left.{}+ S_t\left( s_t y-g_t z-h_t 
u\right)^2+G_t z^2+H_t u^2
     \vphantom{\fr{\partial V_t(y,z)}{\partial y}}\right] =0\,,\\
     V_T(y,z)=S_T\left( s_T y-g_T z\right)^2+G_T z^2\,.
     \label{e10-bos}
     \end{multline}
     Это и~есть то самое уравнение, которое требуется решить: 
существование решения данного урав\-не\-ния суть достаточное условие 
оптимальности; оптимальное управ\-ле\-ние при этом~--- точ\-ка минимума 
со\-от\-вет\-ст\-ву\-юще\-го сла\-га\-емого.
     
\section{Динамическое программирование и~оптимальное 
управление}

     В рассматриваемой постановке линейность\linebreak выхода и~квадратичность 
критерия дают те же преимущества, что и~в~классической  
ли\-ней\-но-квад\-ра\-тич\-ной задаче управ\-ле\-ния~\cite{1-bos}, а~именно: 
позволяют сразу определить вид оптимального управ\-ле\-ния и~фактические 
условия его существования. Действительно, со\-хра\-няя в~(\ref{e10-bos}) под 
знаком $\min\nolimits_u$ только члены, зависящие от~$u$, получаем
     \begin{multline*}
     \fr{\partial V_t(y,z)}{\partial t} +\fr{1}{2}\left( \Sigma_t^2(y) \fr{\partial^2 
V_t(y,z)} {\partial y^2}+\sigma_t^2\fr{\partial^2 V_t(y,z)} {\partial 
z^2}\right)+{}\\
     {}+A_t(y)\fr{\partial V_t(y,z)}{\partial y}+\left( a_t y+b_t z\right) 
\fr{\partial V_t(y,z)}{\partial z}+{}\\
{}+S_t\left( s_t y-g_t z\right)^2 +G_t z^2+{}
\end{multline*}

\noindent
\begin{multline*}
     {}+\min\limits_u \left[ \left( c_t \fr{\partial V_t(y,z)}{\partial z}-2S_t \left( 
s_t y-g_t z\right) h_t\right)u +{}\right.\\
\left.{}+\left( S_t h_t^2+H_t\right) u^2
\vphantom{\fr{\partial V_t(y,z)}{\partial z}}
\right]=0\,,
     %\label{e11-bos}
     \end{multline*}
откуда в~предположении $S_t h_t^2\hm+ H_t\hm>0$ следует, что существует 
оптимальное управ\-ле\-ние, которое определяется равенством
\begin{multline}
u_t^* = u_t^*(y,z)=-\fr{1}{2}\left( S_t h_t^2 +H_t\right)^{-1} \left( c_t 
\fr{\partial V_t(y,z)}{\partial z}-{}\right.\\
\left.{}-2S_t\left( s_t y-g_t z\right) h_t
\vphantom{\fr{\partial V_t(y,z)}{\partial z}}
\right)
\label{e12-bos}
\end{multline}
и доставляет минимум соответствующему сла\-га\-емо\-му в~урав\-не\-нии Беллмана, 
равный
$-\left( S_t h_t^2\hm+\right.$\linebreak
$\left.{}+H_t\right)^{-1} \left( c_t 
{\partial V_t(y,z)}/{\partial 
z}\hm-2S_t\left( s_t y \hm-g_t z\right) h_t \right)^2/4.
$ 
     
     Отметим, что, как и~в~классической ли\-ней\-но-квад\-ра\-тич\-ной 
задаче, управ\-ле\-ние из класса до\-пус\-ти\-мых не\-упреж\-да\-ющих получилось 
управ\-ле\-ни\-ем с~обратной связью.
     
     Таким образом, функция Беллмана описывается сле\-ду\-ющим 
дифференциальным уравнением:
     \begin{multline}
     \fr{\partial V_t(y,z)}{\partial t} +\fr{1}{2}\left( \Sigma_t^2(y) \fr{\partial^2 
V_t(y,z)} {\partial y^2}+\sigma_t^2\fr{\partial^2 V_t(y,z)} {\partial 
z^2}\right)+{}\\
     {}+ A_t(y) \fr{\partial V_t(y,z)}{\partial y}+\left( a_t y+b_t z\right) 
\fr{\partial V_t(y,z)}{\partial z}+{}\\
{}+ S_t \left( s_t y- g_t z\right)^2 +G_t z^2-
 \fr{1}{4}\left( S_t h_t^2+H_t\right)^{-1}\times{}\\
 {}\times \left( c_t \fr{\partial V_t(y,z)} 
{\partial z}-2S_t\left( s_t y -g_t z\right) h_t \right)^2=0\,.
     \label{e13-bos}
     \end{multline}
     
     Возводя в~квадрат по\-след\-нее сла\-га\-емое в~(\ref{e13-bos}), перепишем 
его в~виде:
     \begin{multline}
     \fr{\partial V_t(y,z)}{\partial t} +\fr{1}{2}\left( \Sigma_t^2(y) \fr{\partial^2 
V_t(y,z)} {\partial y^2}+\sigma_t^2\fr{\partial^2 V_t(y,z)} {\partial 
z^2}\!\right)+{}\\
{}+A_t(y) \fr{\partial V_t(y,z)}{\partial y}
+ \left( 
\vphantom{\left( S_t h_t^2 +H_t\right)^{-1}}
a_t y+b_t z+{}\right.\\
\left.{}+\left( S_t h_t^2 +H_t\right)^{-1}
 c_t S_t \left( s_t y-g_t z\right) h_t
\right) 
     \fr{\partial V_t(y,z)}{\partial z}+{}\\
     {}+\left( S_t-\left( S_t h_t^2 +H_t\right)^{-1} S_t^2 h_t^2\right)\left( s_t y -
g_t z\right)^2+{}\\
     \!\!{}+
     G_t z^2 -\fr{1}{4}\left( S_t h_t^2+H_t\right)^{-1}\! c_t^2
     \left(\fr{\partial V_t(y,z)}{\partial z}\right)^{\!2}=0\,.\!\!
     \label{e14-bos}
     \end{multline}
     
     Рассматривая полученное уравнение, заметим, что его решение может 
быть пред\-став\-ле\-но в~виде:
   \begin{equation}
     V_t(y,z)= \alpha_t z^2+\beta_t(y) z +\gamma_t(y)\,,
     \label{e15-bos}
     \end{equation}
т.\,е.\ будем искать решение при дополнительном предположении 
о~квад\-ра\-тич\-ности функции Белл\-ма\-на по переменной~$z$, и~сведем, таким 
образом, поиск оптимального решения к~уравнениям относительно функций 
$\alpha_t$, $\beta_t(y)$ и~$\gamma_t(y)$. Отметим сразу, что явный вид 
функции~$\gamma_t(y)$ для реализации оптимального управ\-ле\-ния не 
требуется, однако далее будет предложен вариант вы\-чис\-ле\-ния и~этой 
функции, что пред\-став\-ля\-ет\-ся небесполезным, поскольку позволит выполнять 
расчет минимума целевого функционала. Источником для 
предложения~(\ref{e15-bos}) является уже упоминавшаяся аналогичная 
задача для случая дис\-крет\-но\-го времени~\cite{7-bos, 8-bos}. В~той задаче 
выражение для функции Беллмана получается формально без 
дополнительных усилий. При этом форма~(\ref{e15-bos}) обнаруживается 
как свойство оптимального решения. В~рассматриваемом случае 
непрерывного времени~(\ref{e15-bos}) постулируется, а~пра\-виль\-ность 
постулата под\-тверж\-да\-ет\-ся далее ре\-зуль\-ти\-ру\-ющи\-ми уравнениями 
для~$\alpha_t$, $\beta_t(y)$ и~$\gamma_t(y)$ Кроме того, данное 
предположение пред\-став\-ля\-ет\-ся вы\-те\-ка\-ющим из линейной структуры задачи 
в~отношении переменной~$z$, в~част\-ности, тем фактом, что такой вид 
функции Беллмана обеспечивает линейность оптимального 
управ\-ле\-ния~(\ref{e12-bos}) по~$z$.

     Граничное условие при выбранном предположении~(\ref{e15-bos}) 
принимает вид:

\noindent
     \begin{multline*}
     V_T(y,z)= S_T\left( s_T y- g_T z\right)^2+G_T z^2 ={}\\[-0.5pt]
     {}=\alpha_T z^2 
+\beta_T(y) z +\gamma_T(y)\,,
    \end{multline*}
т.\,е.

\noindent
\begin{align*}
\alpha_T&= S_T g_T^2 +G_T\,;\\[-0.5pt]
\beta_T(y)&=-2S_T s_T g_T y\,;\\[-0.5pt]
\gamma_T(y)&=S_T s_T^2 y^2\,.
%\label{e16-bos}
\end{align*}
          При этом само оптимальное управ\-ле\-ние, определенное 
выражением~(\ref{e12-bos}), оказывается управ\-ле\-ни\-ем с~обратной связью 
по~$y_t$ и~$z_t$:

\noindent
     \begin{multline}
     u_t^*=u_t^*(y,z) ={}\\[-0.5pt]
     {}=
     -\fr{1}{2}\left( S_t h_t^2 +H_t\right)^{-1}
     \left( c_t \left( 2\alpha_t z +\beta_t(y)\right) +{}\right.\\[-0.5pt]
    \left. {}+2S_t\left( s_t y-g_t z\right) 
h_t\right)\,.
     \label{e17-bos}
     \end{multline}
          Подставляем $V_t(y,z)\hm= \alpha_t z^2 \hm+ \beta_t(y) 
z\hm+\gamma_t(y)$ в~(\ref{e14-bos}):

\noindent
     \begin{multline*}
     \fr{\partial \alpha_t}{\partial t}\, z^2 +
     \fr{\partial \beta_t(y)}{\partial t}\,z +
     \fr{\partial \gamma_t(y)}{\partial t}+{}\\[-0.5pt]
     {}+\fr{1}{2}\left( \Sigma_t^2(y) \left( 
\fr{\partial^2\beta_t(y)}{\partial y^2}\,z +\fr{\partial^2 \gamma_t(y)}{\partial 
y^2}\right) +2\sigma_t^2\alpha_t\right)+{}\\[-0.5pt]
 {}+A_t(y)\left(\fr{\partial \beta_t(y)}{\partial y}\,z + \fr{\partial 
\gamma_t(y)}{\partial y}\right) +{}\\[-0.5pt]
\hspace*{-0.22987pt}{}+\left( a_t y+b_t z+\left( S_t h_t^2 +H_t\right)^{-1} c_t S_t \left( s_t y-
g_t z\right) h_t\right)\times{}
\end{multline*}

\noindent
\begin{multline*}
         {}\times \left( 2\alpha_t z+\beta_t(y)\right)+{}\\
     {}+\left( S_t-\left( S_t h_t^2 +H_t\right)^{-1} S_t^2 h_t^2\right)\left( s_t y-
g_t z\right)^2+{}\\
     {}+ G_t z^2 -\fr{1}{4}\left( S_t h_t^2 +H_t\right)^{-1} c_t^2 \left( 
2\alpha_t z+\beta_t(y)\right)^2=0\,.
     \end{multline*}
          Далее выделяем слагаемые при~$z^2$, $z$ и~$z^0$
          
          \noindent
     \begin{multline*}
     \fr{\partial \alpha_t}{\partial t}\, z^2 +\fr{\partial \beta_t(y)}{\partial t}\,z +
     \fr{\partial \gamma_t(y)}{\partial 
t}+\fr{1}{2}\,\Sigma_t^2(y)\fr{\partial^2\beta_t(y)}{\partial y^2}\,z+ {}\\
{}+
\fr{1}{2}\,\Sigma_t^2(y)\fr{\partial^2\gamma_t(y)}{\partial 
y^2}+\sigma_t^2\alpha_t+A_t(y)\fr{\partial \beta_t(y)}{\partial y}\,z +{}\\
{}+A_t(y) \fr{\partial 
\gamma_t(y)}{\partial y}+{}\\
{}+ 2\alpha_t \left( b_t -\left( S_t h_t^2+H_t\right)^{-1} c_t 
S_t h_t g_t \right)z^2+{}\\
     {}+
     \left( 2\alpha_t\left( \alpha_t+\left( S_t h_t^2+H_t\right)^{-1} c_t S_t h_t 
s_t\right)y +{}\right.\\
\left.{}+\beta_t(y) \left( b_t-\left( S_t h_t^2+H_t\right)^{-1} c_t S_t h_t 
g_t\right) \right) z+{}\\
     {}+\beta_t(y)\left( a_t +\left( S_t h_t^2+H_t\right)^{-1} c_t S_t h_t s_t\right) 
y+{}\\
{}+ \left( S_t -\left( S_t h_t^2+H_t\right)^{-1} S_t^2 h_t^2\right) g_t^2 z^2-{}\\
     {}- 2\left( S_t -\left( S_t h_t^2+H_t\right)^{-1} S_t^2 h_t^2\right) s_t g_t yz 
+{}\\
{}+
     \left( S_t-\left( S_t h_t^2+H_t\right)^{-1} S_t^2 h_t^2\right) s_t^2 y^2+{}\\
     {}+G_t z^2 -\left( S_t h_t^2 +H_t\right)^{-1} c_t^2 \alpha_t^2 z^2 -{}\\
     {}-\left( 
S_t h_t^2+H_t\right)^{-1} c_t^2 \alpha_t \beta_t(y) z-{}\\
{}-
\fr{1}{4}\left( S_t h_t^2+H_t\right)^{-1}  c_t^2 \beta_t^2(y)=0\,,
     \end{multline*}
группируем их и~получаем сле\-ду\-ющие уравнения:
\begin{itemize}
\item  для~$\alpha_t$:

\noindent
\begin{multline}
\fr{\partial\alpha_t}{\partial t}+2\alpha_t\left( b_t-\left( S_t h_t^2+H_t\right)^{-1} c_t 
S_t h_t g_t\right)+{}\\
{}+ \left( S_t- \left( S_t h_t^2+H_t\right)^{-1} S_t^2 h_t^2\right) 
g_t^2+G_t-{}\\
\hspace*{-8mm}{}-\left( S_t h_t^2+H_t\right)^{-1} c_t^2 \alpha_t^2 =0\,,\enskip \alpha_T=S_T 
g_t^2+G_T\,;\!\!
\label{e18-bos}
\end{multline}
\item для $\beta_t$:

\noindent
\begin{multline}
\fr{\partial\beta_t(y)}{\partial 
t}+\fr{1}{2}\,\Sigma_t^2(y)\fr{\partial^2\beta_t(y)}{\partial y^2} 
+A_t(y)\fr{\partial \beta_t(y)}{\partial y}+{}\\
{}+ 2\alpha_t\left( a_t +\left( S_t h_t^2+H_t\right)^{-1} c_t S_t h_t s_t\right) y+{}\\
{}+
\beta_t(y)\left( b_t -\left( S_t h_t^2 +H_t\right)^{-1} c_t S_t h_t g_t\right)-{}\\
{}-2\left( S_t-\left( S_t h_t^2+H_t\right)^{-1} S_t^2 h_t^2\right) s_t g_t y-{}
\\
{}-
\left( S_t h_t^2+H_t\right)^{-1} c_t^2 \alpha_t \beta_t(y)=0\,,\\
\beta_T(y)=-2S_T s_T g_T y\,;
\label{e19-bos}
\end{multline}
\item  для $\gamma_t$:
\begin{multline}
\hspace*{-0.8pt}\fr{\partial \gamma_t(y)}{\partial t}+\fr{1}{2}\,\Sigma_t^2(y)
\fr{\partial^2 \gamma_t(y)}{\partial y^2} +\sigma_t^2 \alpha_t +A_t(y)
\fr{\partial \gamma_t(y)}{\partial y}+{}\\
{}+ \beta_t(y)\left( a_t +\left( S_t h_t^2+H_t\right)^{-1} c_t S_t h_t s_t\right) y+{}\\
{}+
\left( S_t-\left( S_t h_t^2+H_t\right)^{-1} S_t^2 h_t^2\right)  s_t^2 y^2-{}\\
{}-\fr{1}{4}\left( S_t h_t^2+H_t\right)^{-1} c_t^2 \beta_t^2(y) =0\,,\\
\gamma_T(y)=S_T s_T^2 y^2\,.
\label{e20-bos}
\end{multline}
\end{itemize}
     
     Уравнение~(\ref{e18-bos}), легко заметить, является уравнением 
Риккати, которое в~силу сформулированного выше условия   
имеет единственное неотрицательное решение для всех $0\hm\leq t\hm\leq T$. 
Этот факт требует дополнительного комментария. Для получения 
уравнения~(\ref{e18-bos}) рас\-смот\-рим исходную задачу при дополнительных 
условиях $a_t\hm=0$ и~$s_t\hm=0$ для всех $0\hm\leq t\hm\leq T$. Нетрудно 
видеть, что эти условия рассматриваемую по\-ста\-нов\-ку сводят фактически 
к~классической ли\-ней\-но-квад\-ра\-тич\-ной задаче. Имеющуюся 
в~рассматриваемой формулировке чуть более общую форму целевой 
функции (принципиального значения это обобщение, конечно, не имеет) 
сведем к~классической еще одним предположением: $S_t\hm=0$ для всех 
$0\hm\leq t\hm\leq T$. Теперь уравнение для~$\alpha_t$ принимает хорошо 
известный вид:
     \begin{equation}
     \fr{\partial \alpha_t}{\partial t}+2\alpha_t b_t +G_t- H_t^{-1} c_t^2 
\alpha_t^2=0\,,\enskip \alpha_T=G_T\,.
     \label{e21-bos}
     \end{equation}

     В таком случае, как известно~\cite{10-bos}, существует единственное 
оптимальное управление~--- линейное с~обратной связью по выходу~$z_t$, 
с~коэффициентом усиления, опи\-сы\-ва\-емым уравнением  
Риккати~(\ref{e21-bos}). Именно этот результат дают  
уравнения~(\ref{e18-bos})--(\ref{e20-bos}) и~описываемая ими функция 
Беллмана~(\ref{e15-bos}), так как из $a_t\hm=0$ и~$s_t\hm=0$ немедленно 
следует, что $\beta_t(y)\hm=0$, откуда, в~свою очередь, с~учетом 
не\-за\-ви\-си\-мости решения от~$y_t$ следует, что $\gamma_t(y)\hm=\gamma_t$, 
т.\,е.\ не зависит от~$y$ и~задается уравнением: 
     $$
     \fr{\partial \gamma_t(y)}{\partial t} +\sigma^2_t \alpha_t=0\,,\enskip 
\gamma_T=0\,.
     $$ 
     Оптимальное управ\-ле\-ние при этом 
     $$
     u_t^*= -H_t^{-1} c_t \alpha_t z_t\,,
     $$
      т.\,е.\ все полностью совпадает с~известным классическим решением.
     
     С уравнениями~(\ref{e19-bos}) и~(\ref{e20-bos}) ситуация, естественно, 
обстоит сложнее. Это линейные уравнения второго порядка параболического 
типа, поскольку\linebreak
 $\Sigma_t^2(y)\hm>0$. Фактически отсутствуют 
конструктивные условия, гарантирующие существование их\linebreak
 решений 
(требовать, чтобы все фигурирующие в~уравнениях коэффициенты были 
представлены аналитическими функциями на всем пространстве значений, 
вряд ли целесообразно), поэтому далее будем предполагать, что данные 
уравнения имеют на рас\-смат\-ри\-ва\-емом интервале $0\hm\leq t\hm\leq T$ хотя 
бы одно ограниченное решение и~именно эти условия будем рас\-смат\-ри\-вать 
как достаточные условия существования оптимального решения 
рассматриваемой задачи.
     
     Таким образом, доказана следующая тео\-рема.
     
     \smallskip
     
     \noindent
     \textbf{Теорема.}\ \textit{Пусть для диффузионного 
процесса}~(\ref{e5-bos}) \textit{выполнены условия Ито, для 
     процесса}~(\ref{e6-bos})~--- \textit{ограничены коэффициенты, 
уравнения}~(\ref{e18-bos})--(\ref{e20-bos}) \textit{имеют ограниченные 
решения для $0\hm\leq t\hm\leq T$. Тогда минимум  
функционалу}~(\ref{e7-bos}) \textit{доставляет оптимальное 
управ\-ле\-ние}~(\ref{e17-bos}), \textit{где} $y\hm= y_t$; $z\hm=z_t$.
     
\section{Заключение}

     Рассмотренная задача оптимизации в~целом близка и~по модели, и~по 
критерию к~классической ли\-ней\-но-квад\-ра\-тич\-ной постановке. 
Принципиальным отличием является нелинейная модель для описания 
со\-сто\-яния динамической сис\-те\-мы, в~которой отсутствует управ\-ля\-ющее 
воздействие.\linebreak
 Такую модель наряду с~традиционной интер\-пре\-тацией  
<<со\-сто\-яние--вы\-ход>> мож\-но понимать как\linebreak модель неконтролируемого 
слож\-но\-го внешнего воздействия. Небольшое дополнительное отличие дает 
предложенная форма квад\-ра\-тич\-но\-го критерия, поз\-во\-ля\-ющая, в~част\-ности, 
ставить такие задачи, как отслеживание выходом или управ\-ле\-ни\-ем со\-сто\-яния 
сис\-те\-мы или ее выхода.
     
     Поскольку обсуждать возможности точного решения уравнений, 
определяющих оптимальное управ\-ле\-ние, не имеет смыс\-ла, наиболее 
актуальной далее является задача их приближенного чис\-лен\-но\-го решения 
и~анализа воз\-мож\-ности практической реализации. Этому посвящена вторая 
часть данной работы, пла\-ни\-ру\-емая к~выходу в~ближайшее время.

{\small\frenchspacing
 {%\baselineskip=10.8pt
 \addcontentsline{toc}{section}{References}
 \begin{thebibliography}{99}
\bibitem{1-bos}
\Au{Athans M.} Editorial on the LQG problem~// IEEE~T. Automat. Contr., 1971. Vol.~16. 
No.\,6. P.~528--552. doi: 10.1109/TAC.1971.1099845.
\bibitem{2-bos}
\Au{Wu Z.} Forward-backward stochastic differential equations, linear quadratic stochastic 
optimal control and nonzero sum differential games~// J.~Syst. Sci. Complex., 2005. Vol.~18. 
No.\,2. P.~179--192.
\bibitem{3-bos}
\Au{Chen B.\,S., Zhang~W.} Stochastic H2/H1 control with state-dependent noise~// IEEE 
T.~Automat. Contr., 2004. Vol.~49. No.\,1. P.~45--56. doi: 10.1109/TAC.2003.821400.
\bibitem{4-bos}
\Au{Bohacek S.} A~stochastic model of TCP and fair video transmission~// IEEE 
INFOCOM, 2003. Vol.~2. P.~1134--1144. doi: 10.1109/INFCOM.2003.1208950.
\bibitem{5-bos}
\Au{Домбровский В.\,В., Объедко~Т.\,Ю.} Управление с~прогнозированием системами 
с~марковскими скачками при ограничениях и~применение к~оптимизации 
инвестиционного портфеля~// Автомат. телемех., 2011. №\,5. С.~96--112. doi: 
10.1134/S0005117911050079.
\bibitem{6-bos}
\Au{Баландин Д.\,В., Коган~М.\,М.} Оптимальное линейно-квад\-ра\-тич\-ное управление: от 
матричных уравнений к~линейным матричным неравенствам~// Автомат. телемех., 2011. 
№\,11. С.~60--69. doi: 10.1134/ S0005117911110038.
\bibitem{7-bos}
\Au{Босов А.\,В.} Обобщенная задача распределения ресурсов программной системы~// 
Информатика и~её применения, 2014. Т.~8. Вып.~2. С.~39--47. doi: 
10.14357/19922264140204.
\bibitem{8-bos}
\Au{Босов А.\,В.} Управление линейным выходом дискретной стохастической системы по 
квадратичному критерию~// Изв. РАН. Теория и~системы управления, 2016. №\,3.  
С.~19--35. doi: 10.1134/S1064230716030060.
\bibitem{9-bos}
\Au{Флеминг У., Ришел~Р.} Оптимальное управление детерминированными 
и~стохастическими системами~/ Пер. с~англ.~--- М.: Мир, 1978. 316~с. 
(\Au{Fleming~W.\,H., Rishel~R.\,W.} Deterministic and stochastic optimal control.~--- New 
York, NY, USA: Springer-Verlag, 1975. 222~p.)
\bibitem{10-bos}
\Au{Девис М.\,Х.\,А.} Линейное оценивание и~стохастическое управление~/ Пер. с~англ.~--- 
М.: Наука, 1984. 206~с. (\Au{Davis~M.\,H.\,A.} Linear estimation and stochastic control.~--- 
London: Chapman and Hall, 1977. 224~p.)

 \end{thebibliography}

 }
 }

\end{multicols}

\vspace*{-6pt}

\hfill{\small\textit{Поступила в~редакцию 30.03.18}}

\vspace*{4pt}

%\newpage

%\vspace*{-24pt}

\hrule

\vspace*{2pt}

\hrule

\vspace*{-2pt}


\def\tit{STOCHASTIC DIFFERENTIAL SYSTEM OUTPUT CONTROL 
BY~THE~QUADRATIC CRITERION.~I.~DYNAMIC\\ PROGRAMMING 
OPTIMAL SOLUTION}


\def\titkol{Stochastic differential system output control 
by~the~quadratic criterion. I.~Dynamic programming 
optimal solution}

\def\aut{A.\,V.~Bosov and~A.\,I.~Stefanovich}

\def\autkol{A.\,V.~Bosov and~A.\,I.~Stefanovich}

\titel{\tit}{\aut}{\autkol}{\titkol}

\vspace*{-11pt}


\noindent
Institute of Informatics Problems, Federal Research Center ``Computer Science 
and Control'' of the Russian Academy of Sciences, 44-2~Vavilov Str., Moscow 
119333, Russian Federation


\def\leftfootline{\small{\textbf{\thepage}
\hfill INFORMATIKA I EE PRIMENENIYA~--- INFORMATICS AND
APPLICATIONS\ \ \ 2018\ \ \ volume~12\ \ \ issue\ 3}
}%
 \def\rightfootline{\small{INFORMATIKA I EE PRIMENENIYA~---
INFORMATICS AND APPLICATIONS\ \ \ 2018\ \ \ volume~12\ \ \ issue\ 3
\hfill \textbf{\thepage}}}

\vspace*{3pt}



\Abste{The problem of optimal control for the Ito diffusion 
process and a~controlled linear output is solved. The considered 
statement is close to the classical linear-quadratic Gaussian 
control  (LQG control) problem. Differences consist in the fact 
that the state is described by the nonlinear differential Ito equation  $dy_y = A_t(y_t) 
\,dt+\Sigma_t(y_t)\,dv_t$ and does not depend on the control~$u_t$, 
optimization subject is controlled linear output 
 $dz_t=a_ty_t\,dt +b_tz_t\,dt +c_t u_t\,dt +\sigma_t \,dw_t$. 
Additional generalizations are included in the quadratic 
quality criterion for the purpose of statement such problems 
as state tracking by output or a linear combination of state 
and output tracking by control. The method of dynamic programming 
is used for the solution. 
The assumption about Bellman function in the form  $V_t(y,z)= \alpha_t 
z^2+\beta_t(y) z+\gamma_t(y)$ allows one to find it. 
Three differential equations for the coefficients $\alpha_t$,  $\beta_t(y)$,
and $\gamma_t(y)$ give the solution. 
These equations constitute the optimal solution of the problem under consideration.}

\KWE{stochastic differential equation; optimal control; dynamic programming; 
Bellman function; Riccati equation; linear differential equations of parabolic type}


\DOI{10.14357/19922264180314}

\vspace*{-12pt}

\Ack
\noindent
This work was partially supported by the Russian Science Foundation (grant  
16-07-00677).



%\vspace*{6pt}

  \begin{multicols}{2}

\renewcommand{\bibname}{\protect\rmfamily References}
%\renewcommand{\bibname}{\large\protect\rm References}

{\small\frenchspacing
 {%\baselineskip=10.8pt
 \addcontentsline{toc}{section}{References}
 \begin{thebibliography}{99}
\bibitem{1-bos-1}
\Aue{Athans, M.} 1971. Editorial on the LQG problem. \textit{IEEE~T. 
Automat. Contr.} 16(6):528--552. doi: 10.1109/ TAC.1971.1099845.
\bibitem{2-bos-1}
\Aue{Wu, Z.} 2005. Forward-backward stochastic differential equations, linear 
quadratic stochastic optimal control and\linebreak\vspace*{-12pt}

\columnbreak

\noindent
 nonzero sum differential games. 
\textit{J.~Syst. Sci. Complex.} 18(2):179--192.
\bibitem{3-bos-1}
\Aue{Chen, B.\,S. and W.~Zhang.} 2004. Stochastic H2/H1 control with  
state-dependent noise. \textit{IEEE~T. Automat. Contr.} 49(1):45--56.
doi: 10.1109/TAC.2003.821400.
\bibitem{4-bos-1}
\Aue{Bohacek, S.} 2003. A~stochastic model of TCP and fair video 
transmission. \textit{IEEE INFOCOM}. 2:1134--1144.
doi: 10.1109/INFCOM.2003.1208950.
\bibitem{5-bos-1}
\Aue{Dombrovskii, V.\,V., and T.\,Yu.~Ob''edko.} 2011. Predictive control of 
systems with Markovian jumps under constraints and its application to the 
investment portfolio optimization. \textit{Automat. Rem. Contr.}  
72(5):989--1003.
\bibitem{6-bos-1}
\Aue{Balandin, D.\,V., and M.\,M.~Kogan.} 2011. Optimal linear-quadratic 
control: From matrix equations to linear matrix inequalities. \textit{Automat. 
Rem. Contr.} 72(11):2276--2284.
\bibitem{7-bos-1}
\Aue{Bosov, A.\,V.} 2014. Obobshchennaya zadacha raspredeleniya resursov 
programmnoy sistemy [The generalized problem of software system resources 
distribution]. \textit{Informatika i~ee Primeneniya~--- Inform. Appl.}  
8(2):39--47. doi: 
10.14357/19922264140204.
\bibitem{8-bos-1}
\Aue{Bosov, A.\,V.} 2016. Discrete stochastic system linear output control 
with respect to a quadratic criterion. \textit{J.~Comput. Syst. Sc. 
Int.} 55(3):349--364.
\bibitem{9-bos-1}
\Aue{Fleming, W.\,H., and R.\,W.~Rishel.} 1975. \textit{Deterministic and 
stochastic optimal control.} New York, NY: Springer-Verlag. 222~p.
\bibitem{10-bos-1}
\Aue{Davis, M.\,H.\,A.} 1977. \textit{Linear estimation and stochastic 
control.} London: Chapman and Hall. 224~p.
\end{thebibliography}

 }
 }

\end{multicols}

\vspace*{-6pt}

\hfill{\small\textit{Received March 30, 2018}}

%\pagebreak

%\vspace*{-18pt}
     
     \Contr
     
       \noindent
       \textbf{Bosov Alexey V.} (b.\ 1969)~--- Doctor of Science in technology, 
principal scientist, Institute of Informatics Problems, Federal Research 
Center ``Computer Science and Control'' of the Russian Academy of Sciences, 
44-2~Vavilov Str., Moscow 119333, Russian Federation; 
\mbox{AVBosov@ipiran.ru}
       
       \vspace*{3pt}
       
       \noindent
       \textbf{Stefanovich Alexey I.} (b.\ 1983)~--- principal specialist, 
Institute of Informatics Problems, Federal Research Center ``Computer Science 
and Control'' of the Russian Academy of Sciences, 44-2~Vavilov Str., Moscow 
119333, Russian Federation; \mbox{AStefanovich@frccsc.ru}
\label{end\stat}

\renewcommand{\bibname}{\protect\rm Литература}       

        %7

\def\stat{shnurkov}

\def\tit{АНАЛИТИЧЕСКОЕ РЕШЕНИЕ ЗАДАЧИ ОПТИМАЛЬНОГО УПРАВЛЕНИЯ ПОЛУМАРКОВСКИМ ПРОЦЕССОМ\\ 
С~КОНЕЧНЫМ МНОЖЕСТВОМ СОСТОЯНИЙ$^*$}

\def\titkol{Аналитическое решение задачи оптимального управления полумарковским 
процессом} %с~конечным множеством состояний}

\def\aut{П.\,В.~Шнурков$^1$, А.\,К.~Горшенин$^2$, В.\,В.~Белоусов$^3$}

\def\autkol{П.\,В.~Шнурков, А.\,К.~Горшенин, В.\,В.~Белоусов}

\titel{\tit}{\aut}{\autkol}{\titkol}

\index{Шнурков П.\,В.}
\index{Горшенин А.\,К.}
\index{Белоусов В.\,В.}
\index{Shnurkov P.\,V.}
\index{Gorshenin A.\,K.}
\index{Belousov V.\,V.}


{\renewcommand{\thefootnote}{\fnsymbol{footnote}} \footnotetext[1]
{Работа выполнена при частичной поддержке РФФИ (проект 15-07-05316).}}


\renewcommand{\thefootnote}{\arabic{footnote}}
\footnotetext[1]{Национальный исследовательский университет <<Высшая школа экономики>>, 
\mbox{pshnurkov@hse.ru}}
\footnotetext[2]{Институт проблем информатики Федерального исследовательского центра <<Информатика 
и~управ\-ле\-ние>> Российской академии наук, \mbox{agorshenin@frccsc.ru}}
\footnotetext[3]{Институт проблем информатики Федерального исследовательского центра <<Информатика 
и~управление>> Российской академии наук, \mbox{vbelousov@ipiran.ru}}

%\vspace*{-6pt}

\Abst{Настоящее исследование посвящено теоретическому обоснованию нового метода 
нахождения оптимальной стратегии управления полумарковским процессом с~конечным 
множеством состояний. Рассматриваются марковские рандомизированные стратегии 
управления, определяемые конечным набором вероятностных мер, соответствующих 
каждому состоянию. Характеристикой качества управления служит стационарный 
стоимостной показатель. Данный показатель представляет собой дроб\-но-ли\-ней\-ный 
интегральный функционал от набора вероятностных мер, задающих стратегию управления. 
Для этого функционала известны явные аналитические представления подынтегральных 
функций числителя и~знаменателя. Дальнейшие результаты основываются на новой 
усиленной и~обобщенной форме теоремы об экстремуме дроб\-но-ли\-ней\-но\-го интегрального 
функционала. Доказывается, что проблемы существования оптимальной стратегии управления 
полумарковским процессом и~ее нахождения сводятся к~задаче численного исследования 
на глобальный экстремум заданной функции от конечного числа вещественных переменных.}

\KW{оптимальное управление полумарковским процессом; стационарный стоимостной 
показатель качества управления; дроб\-но-ли\-ней\-ный интегральный функционал}

\DOI{10.14357/19922264160408} 

\vspace*{9pt}


\vskip 10pt plus 9pt minus 6pt

\thispagestyle{headings}

\begin{multicols}{2}

\label{st\stat}

\section{Введение}

Теория оптимального управления марковскими и~полумарковскими случайными 
процессами интенсивно развивается с~начала 1960-х~гг. Еще в~первых 
основополагающих исследованиях рассматривались не только проблемы существования 
оптимальных стратегий управления, но и~способы нахождения этих стратегий. 

Для решения таких проблем, имеющих алгоритмическое содержание, использовались 
открытые незадолго до этого мощные методы прикладной математики: линейное 
программирование и~динамическое программирование. Отметим, прежде всего, 
классическую работу Р.~Ховарда~\cite{1}, в~которой метод динамического 
программирования был применен для решения проблемы оптимального управления 
марковским процессом с~непрерывным временем. В~дальнейшем В.\,В.~Рыков~\cite{2} 
доказал, что для аналогичной модели управления марковским процессом с~учетом 
переоценки оптимальной стратегией также является стационарная.

Важную роль в~развитии теории управления случайными процессами сыграла работа 
В.~Джевелла~\cite{3}, в~которой были впервые рассмотрены полумарковские модели 
управления для вариантов с~переоценкой и~без переоценки. Данная работа была 
переведена на русский язык и~послужила основой для многих последующих работ 
отечественных и~зарубежных специалистов. В~частности, Б.~Фокс показал~\cite{4}, 
что оптимальной стратегией управления полумарковским процессом в~варианте без 
переоценки является стационарная; аналогичные результаты были получены Э.~Денардо 
и~для варианта с~переоценкой~\cite{5}.

Среди последующих исследований алгоритмической направленности отметим работы 
Р.~Ховарда~\cite{6}, Б.~Фокса~\cite{4}, а также С.~Осаки и~Х.~Майна~\cite{7}. 
В~этих работах для нахождения оптимальных стратегий управления полумарковскими 
процессами использовался метод линейного программирования.

В 1970~г.\ была опубликована фундаментальная монография Х.~Майна и~С.~Осаки~\cite{8}, 
переведенная на русский язык в~1977~г., в~которой были систе\-ма\-ти\-зи\-ро\-ва\-ны и~изложены 
основные результаты по теории оптимального управления марковскими и~полумарковскими 
случайными процессами. Фактически данная книга стала итогом исследований по проблемам 
стохастического управления\linebreak
 за~10~лет. Отметим, что в~этой монографии рас\-смат\-ри\-ва\-лись 
марковские и~полумарковские модели управления с~конечными множествами состояний 
и~допустимых решений, принимаемых \mbox{в~каждом} состоянии. Были получены принципиальные 
тео\-ре\-ти\-че\-ские результаты, заключающиеся в~том, что оптимальные стратегии управ\-ле\-ния 
для основных видов рас\-смат\-ри\-ва\-емых моделей с~переоценкой и~без переоценки являются 
детерминированными и~стационарными. Были разработаны и~обоснованы процедуры нахождения 
оптимальных стратегий управления. В~частности, для модели управления полумарковским 
процессом без переоценки, когда множество со\-сто\-яний образует один эргодический класс, 
а~показатель качества управления пред\-став\-ля\-ет собой стационарный средний удельный 
доход (см.~[8, гл.~5, п.~5.5]), процедура поиска оптимальной рандомизированной 
стратегии осуществлялась методом линейного программирования. Обратим особое внимание 
на данный результат, поскольку аналогичная модель управления полумарковским 
процессом будет рассмотрена в~настоящей работе.

Принципиальную роль в~развитии теории стохастического управления сыграла 
монография И.\,И.~Гихмана и~А.\,В.~Скорохода~\cite{9}. В~этой книге были впервые 
систематически изложены основы теории оптимального управления случайными процессами 
с~дискретным и~непрерывным временем, включая теорию управления процессами, которые 
описываются стохастическими дифференциальными уравнениями. Отдельно были рас\-смот\-ре\-ны 
проблемы управления марковскими процессами с~дискретным временем и~скачкообразными 
марковскими процессами с~непрерывным временем. Роли множеств состояний и~допустимых 
управ\-ле\-ний играли пространства весьма общей структуры. Для широких классов функционалов 
качества управ\-ле\-ния (так называемых эволюционных функционалов в~марковских моделях 
с~дискретным временем и~интегральных функционалов накопления в~марковских моделях 
с~непрерывным временем) были доказаны теоремы о~существовании и~формах пред\-став\-ле\-ния 
оптимальных стратегий управ\-ле\-ния. Было установлено, что для однородных марковских 
моделей оптимальные стратегии управ\-ле\-ния существуют, являются стационарными 
и~детерминированными. Иначе говоря, такие стратегии задаются детерминированными 
функциями, аргументом которых является со\-сто\-яние сис\-те\-мы в~момент принятия решения, 
и~не зависящими от самого момента принятия решения. Что же касается важного вопроса 
о~формах представления этих функций, то их можно охарактеризовать следующим образом. 
Были найдены функциональные уравнения, осложненные условием экстремума, которым 
удовле\-тво\-ря\-ют упомянутые функции. По существу эти соотношения пред\-став\-ля\-ют собой 
уравнения Беллмана для соответствующих динамических стохастических моделей.

Особо отметим, что в~монографии~\cite{9} не рас\-смат\-ри\-ва\-лись проблемы управления 
полумарковскими процессами. Однако дальнейшее развитие общей теории управления 
такими процессами шло по пути, идейно намеченному в~указанной книге.

В последующие годы развитие теории управ\-ле\-ния полумарковскими процессами 
осуществля-\linebreak лось по направлению усложнения моделей и~обобщения исходных предположений. 
Например,\linebreak в~работах~\cite{10, 11} рассмотрены управляемые по\-лумарковские процессы при 
весьма общих предположениях относительно характера пространств состояний и~управлений. 
Проблемы управления исследовались по отношению к~различным видам целевых показателей, 
обобщающих упомянутый выше стационарный показатель средней удельной прибыли. В~этих 
работах доказывается, что оптимальная стратегия управления по отношению к~каж\-до\-му из 
показателей существует и~является одной и~той же стационарной детерминированной 
стратегией, определяемой некоторой функцией, заданной на множестве со\-сто\-яний процесса. 
Об этой функции известно лишь то, что она удовлетворяет некоторому интегральному 
уравнению, которое по содержанию пред\-став\-ля\-ет собой уравнение Бел\-лма\-на для 
соответствующей задачи управ\-ления.

Среди исследований, предшествовавших настоящему, отметим работу 
В.\,А.~Каштанова~[12, гл. 13]. В этом разделе коллективной монографии~\cite{12} 
автором была рассмотрена проблема оптимального управления полумарковским 
процессом с~конечным множеством состояний и~множеством возможных решений, 
которое представляет собой произвольный интервал множества вещественных чисел. 
Модель относится к~виду моделей без переоценки, показателем качества управления 
служит стационарное значение среднего удельного дохода, определяемое аналогично 
классическим работам~\cite{3, 8}. Рандомизированное управление в~каждом состоянии 
определяется в~соответствии с~вероятностным распределением, совокупность которых 
задает\linebreak
 стратегию управления. В.\,А.~Каш\-та\-но\-вым было\linebreak сформулировано утверждение о том, 
что стацио\-нарное значение среднего удельного дохода представляет собой 
дроб\-но-ли\-ней\-ный интегральный функционал от набора вероятностных распределений, 
образующих стратегию управления. При этом\linebreak ранее~[12, гл.~10; 13] было уста\-нов\-ле\-но, 
что дроб\-но-ли\-ней\-ный функционал достигает экстремума на вырожденных распределениях. 
Отсюда естест-\linebreak венно следует, что оптимальная стратегия управ\-ле-ния является 
детерминированной и~должна\linebreak определяться точкой экстремума функции, представляющей 
собой отношение подынтегральных функций чис\-ли\-те\-ля и~знаменателя данного 
дроб\-но-ли\-ней\-но\-го функционала. Однако в~\cite{12} не были получены явные 
представления для указан-\linebreak ных функций. Кроме того, приведенный в~гл.~10 
монографии~\cite{12} вариант теоремы об экстремуме дроб\-но-ли\-ней\-но\-го 
интегрального функционала требовал проверки выполнения условия существования 
этого экстремума. Такие условия указаны не были. В~связи с~этими обстоятельствами 
использовать полученные в~\cite{12} результаты для доказательства существования 
оптимальной детерминированной стратегии управ\-ле\-ния полумарковским процессом и~для 
строгого обоснования способа нахождения такой стратегии оказалось невозможным.

Настоящее исследование посвящено теоретическому обоснованию нового метода 
нахождения\linebreak оптимальной стратегии управления полумарковским процессом с~конечным 
множеством со\-сто\-яний. Рассматриваются марковские рандомизи\-рованные стратегии 
управления, определяемые конеч\-ным набором вероятностных мер, соответствующих 
каждому состоянию. Показателем качества управления служит уже упоминавшийся 
классический  показатель~--- стационарное значение средней удельной прибыли. 
Доказано, что этот показатель представляет собой дроб\-но-ли\-ней\-ный интегральный 
функционал от набора вероятностных мер, задающих стратегию управления. При этом, 
в~отличие от~\cite{12}, получены явные аналитические представления для подынтегральных 
функций числителя и~знаменателя этого дроб\-но-ли\-ней\-но\-го\linebreak
 функционала. Дальнейшие 
результаты основываются на новой усиленной и~обобщенной форме\linebreak
 теоремы об экстремуме 
дроб\-но-ли\-ней\-но\-го интегрального функционала, впервые опубликованной 
в~работе П.\,В.~Шнуркова~\cite{14}. Согласно\linebreak
 утверж\-де\-нию этой теоремы, если 
существует глобальный экстремум так называемой основной функции дроб\-но-ли\-ней\-но\-го 
функционала, которая пред\-став\-ля\-ет собой отношение подынтегральных функций чис\-ли\-те\-ля 
и~знаменателя, то существует безусловный экстремум самого дроб\-но-ли\-ней\-но\-го 
функционала, который достигается на наборе вырожденных вероятностных распределений, 
сосредоточенных в~точке глобального экстремума. В~этом случае оптимальная стратегия 
управ\-ле\-ния существует, является стационарной и~детерминированной и~определяется точкой, 
в~которой основная\linebreak функция достигает глобального экстремума. Таким\linebreak образом, проблемы 
существования оптимальной стратегии управ\-ле\-ния полумарковским процессом и~ее 
нахождения сводятся к~задаче чис\-лен\-но\-го исследования на глобальный экстремум 
заданной функции от конечного чис\-ла вещественных переменных.

\section{Общее описание модели управления полумарковским случайным процессом}

Построим модель управления полумарковским случайным процессом, следуя общему 
подходу, принятому в~классических работах~\cite{3, 8}. Пусть $\xi(t)$~--- 
случайный полумарковский процесс с~конечным множеством состояний
$X\hm=\{1,2,\ldots, N\}$, $N\hm< \infty$. Обозначим через~$t_n$, $n=0,1,2,\ldots$, 
$t_0\hm=0$, случайные моменты изменения состояний данного процесса, 
$\theta_n\hm=t_{n+1}-t_n$, $n\hm=0,1,2,\ldots$, $\xi_n\hm=\xi(t_n)\hm=\xi(t_n+0)$, 
$n\hm=0,1,2,\ldots$ (предполагается, что траектории процесса~$\xi(t)$ 
непрерывны справа). Случайная последовательность~$\{\xi_n\}$
образует цепь Маркова, вложенную в~полумарковский процесс~$\xi(t)$.
Зададим набор измеримых пространств\linebreak $(U_1, \mathscr{B}_1), 
(U_2, \mathscr{B}_2), \ldots, (U_N, \mathscr{B}_N)$, где $U_i$~--- 
множество возможных допустимых управ\-ле\-ний, $\mathscr{B}_i$~--- $\sigma$-ал\-геб\-ра 
подмножеств множества~$U_i$, вклю\-ча\-ющая в~себя все одноточечные подмножества\linebreak  
множества~$U_i$, т.\,е.\ если $u_i \hm\in U_i$, то $\{u_i\} \hm\in \mathscr{B}_i$, 
$i\hm=1,2,\ldots, N$.
Пусть $\Gamma_i$~--- некоторое множество всевозможных вероятностных мер $\Psi_i 
\hm \in \Gamma_i$, заданных на $\sigma$-ал\-геб\-ре~$\mathscr{B}_i$, $i\hm=1,2,\ldots,N$.

Поскольку идейное содержание и~свойства вероятностных мер существенно используются 
в~данной работе, укажем на некоторые фундаментальные издания, в~которых 
изложена соответствующая тео\-рия. Понятие и~основные свойства вероятностной 
меры определены и~подробно проанализированы в~книге А.\,Н.~Ширяева~\cite[гл.~II]{15}. 
Глубокое изложение основ теории вероятностных мер имеется также в~книге 
А.\,А.~Боровкова~\cite{16}. Заметим попутно, что в~книге~\cite{16} имеются разделы, 
посвященные изложению основ теории полумарковских и~регенерирующих случайных процессов. 
Из зарубежных изданий отметим фундаментальную работу П.~Хеннекена и~А.~Тортра~\cite{17}, 
основная часть которой посвящена изложению математических основ теории вероятностей.

Введем специальное понятие вырожденной вероятностной меры, которое будет часто 
использоваться в~дальнейшем. Пусть $(U_0, \mathscr{B}_0)$~--- некоторое измеримое 
пространство, $\mathscr{B}_0$~--- $\sigma$-ал\-геб\-ра подмножеств множества~$U_0$, 
включающая в~себя все одноточечные подмножества этого множества.

\medskip

\noindent
\textbf{Определение 1.}\ Вероятностная мера~$\Psi^*$, заданная 
на $\sigma$-ал\-геб\-рe~$\mathscr{B}_0$, называется вырожденной, если существует 
такой элемент $u^* \hm\in U_0$, для которого выполняются условия $\Psi^*(\{u^*\})\hm=
1$, $\Psi^*(U_0 \setminus \{u^*\})\hm=0$, где $\{u^*\}=u^*$~--- 
множество, состоящее из единственной точки $u^* \hm\in U_0$. Соответствующая 
точка $u^* \hm\in U_0$ будет называться точкой сосредоточения вырожденной 
вероятностной меры~$\Psi^*$.
Таким образом, всякая вырожденная вероятностная мера~$\Psi^*$ определяется 
своей точкой сосредоточения~$u^*$. В~дальнейшем будем использовать 
обозначение~$\Psi_{u^*}^{*}$, имея в~виду, что вырожденная вероятностная мера~$\Psi^*$ 
сосредоточена в~точке~$u^*$.
Отметим также, что вырожденная вероятностная мера~$\Psi_{u^*}^{*}$ соответствует 
детерминированной величине, которая принимает фиксированное значение $u\hm=u^*$ 
с~вероятностью, равной единице.

\medskip

Обозначим через $\Gamma_0$ множество всех  вероятностных мер, заданных 
на измеримом пространстве ($U_0, \mathscr{B}_0$), а через~$\Gamma_0^*$~--- 
множество всех вырожденных вероятностных мер, заданных на этом пространстве, 
$\Gamma_0^*\hm\in \Gamma_0$. Аналогичные обозначения будут использоваться 
и~в~дальнейшем. Заметим, что множество~$\Gamma_0^*$ находится во взаимно
 однозначном соответствии с~множеством точек сосредоточения вырожденных 
 вероятностных мер, т.\,е.\ с~множеством~$U_0$.

Пусть $\Gamma_i^{*}$~--- множество всех вырожденных мер, заданных на 
$\sigma$-ал\-геб\-ре~$\mathscr{B}_i$, $\Gamma_i^{*}\hm\subset \Gamma_i$.
Произвольная вероятностная мера~$\Psi_i$ описывает случайную величину, 
принимающую значения в~$U_i$, а вырожденная мера~$\Psi_i^*$, сосредоточенная 
в~точке~$u_i^*$, соответствует детерминированной величине $u_i^*\hm\in U_i$.
Предполагается, что соответствующие конструкции определены на всех измеримых 
пространствах управлений $(U_1, \mathscr{B}_1), (U_2, \mathscr{B}_2), \ldots, 
(U_N,\mathscr{B}_N)$.

Предположим, что управления случайным полумарковским процессом~$\xi(t)$ 
осуществляются в~моменты времени~$t_n,$ $n\hm=0,1,2,\ldots,$
непосредственно после изменения состояния процесса. Если\linebreak 
$\xi_n\hm=\xi(t_n)\hm=i \hm\in X$, то значение управления представляет 
собой случайную величину~$u_n$, принимающую значения в~множестве допустимых 
управ\-ле\-ний~$U_i$ и~описываемую вероятностной мерой (распределе\-ни\-ем 
вероятностей) $\Psi_i \hm\in \Gamma_i$.
Будем предполагать, что при фиксированном условии $\xi_n\hm=\xi(t_n)=i$ 
управ\-ле\-ние определяется независимо от прошлого поведения процесса~$\xi(t)$ 
и~вероятностная мера~$\Psi_i$,
описывающая стохастическое управление~$u_n$, зависит только от состояния $i\hm\in X$.
Тогда выбор управ\-ле\-ний в~моменты изменения состояний $\{t_n, n\hm=0,1,2,\ldots \}$ 
описывается набором вероятностных мер (распределений вероятностей) 
$(\Psi_1, \Psi_2,\ldots, \Psi_N)$, 
$\Psi_i \hm\in \Gamma_i$, $i\hm=1,2,\ldots,N$.
Назовем любой такой набор стратегией управ\-ле\-ния полумарковским процессом~$\xi(t)$. 
По своим свойствам такая стратегия является марковской, однородной 
и~рандомизированной.

Следуя классической монографии П.~Халмоша~\cite[гл.~VII]{18}, 
рассмотрим декартово произведение пространств $U\hm=U_1\times U_2\times \cdots\times U_N$ 
и~соответствующих $\sigma$-ал\-гебр $\mathscr{B}\hm=\mathscr{B}_1\times \mathscr{B}_2
\times \cdots \times\mathscr{B}_N$. Обозначим через $\Psi\hm=\Psi_1\times \Psi_2\times \cdots
\times \Psi_N$ вероятностную меру на~$(U,\mathscr{B})$, определяемую как 
произведение мер $\Psi_1,\Psi_2, \ldots , \Psi_N$, где $\Psi_i \hm\in \Gamma_i$, 
$i\hm=1,2,\ldots,N$. Обозначим также через~$\Gamma$ множество вероятностных мер~$\Psi$, 
заданных на~$(U,\mathscr{B})$, которые пред\-став\-ля\-ют собой произведение 
мер $\Psi_1,\Psi_2, \ldots , \Psi_N$, где $\Psi_i \hm\in \Gamma_i$, $i\hm=1,2,\ldots,N$.
Множество~$\Gamma$ можно отож\-де\-ст\-вить с~множеством всех стратегий управ\-ле\-ния 
полумарковским процессом~$\xi(t)$.

Проблема оптимального управления полумар\-ковским процессом~$\xi(t)$ будет в~дальнейшем 
сформулирована в~виде задачи безусловного экстремума некоторого функционала 
$I(\Psi)\hm=I(\Psi_1,\Psi_2, \ldots , \Psi_N)$, заданного на множестве 
допустимых стратегий управления. Содержание показателя качества управления~$I(\Psi)$, 
аналитическое представление для него, а~также описание множества допустимых 
стратегий управления будут приведены в~последующих разделах данной работы.

Для получения дальнейших результатов потребуются различные вероятностные 
характеристики управляемого полумарковского процесса~$\xi(t)$. Как известно из
 общей теории полумарковских процессов~\cite{19, 20}, 
 основной вероятностной характеристикой такого процесса является так называемая 
 полумарковская функция. Определим эту функцию для процесса с~управлением 
 (см.~\cite[гл.~5]{8}):
\begin{multline}
Q_{ij}(t,u)=
{\sf P}\left(\xi_{n+1}=j,\theta_n<t \mid \xi_n=i, u_n=u\right)\,,\\
t\in [0,\infty)\,,\ u\in U_i\,;\ i,j\in X=\{1,2,\ldots,N\}\,. \label{e1}
\end{multline}
Используя полумарковские функции, можно получить вероятности перехода 
управляемой цепи Маркова~$\{\xi_n\}$:
\begin{multline}
p_{ij}(u)={\sf P}\left(\xi_{n+1}=j \mid \xi_n=i, u_n=u\right)= {}\\
{}=
\lim\limits_{t\rightarrow\infty}Q_{ij}(t,u)\,,\enskip
u\in U_i\,;\enskip i,j\in X\,, 
\label{e2}
\end{multline}
а также функции распределения длительностей пребывания полумарковского 
процесса~$\xi(t)$ в~соответствующих состояниях:

\noindent
\begin{multline}
H_{i}(t,u)={\sf P}\left(\theta_n<t \mid \xi_n=i, u_n=u\right)={}\\
{}=
\sum\limits_{j\in X}Q_{ij}(t,u)\,,\enskip
t\in [0,\infty)\,,\  u\in U_i\,; \  i\in X\,. 
\label{e3}
\end{multline}

Обозначим через
\begin{multline}
T_{i}(u)=\mathbf{E}\left[\theta_n \mid \xi_n=i, u_n=u\right]={}\\
{}=
\int\limits_0^{\infty}\left[1-H_i(t,u)\right]\,dt\,,\enskip
u\in U_i\,,\ i\in X\,, 
\label{e4}
\end{multline}
математические ожидания длительностей пребывания полумарковского процесса~$\xi(t)$ 
в~каждом из состояний.

Введенные выше характеристики~(1)--(4) определены для случая, когда 
в~момент изменения состояния~$t_n$ процесс оказывается в~состоянии~$i$ 
и~принимается решение $u\hm\in U_i$. При заданной стратегии управления 
$\Psi\hm=\left(\Psi_1,\Psi_2, \ldots , \Psi_N\right)$ можно записать 
соответствующие вероятностные характеристики без условия на управление, а~именно:
\begin{multline*}
Q_{ij}(t)={\sf P}\left(\xi_{n+1}=j,\theta_n<t \mid \xi_n=i\right)={}\\
{}=
\int\limits_{U_i}Q_{ij}(t,u) \,d\Psi_i(u)\,,\enskip 
t\in [0,\infty)\,,\ i,j\in X\,; %\label{e5}
\end{multline*}

\vspace*{-12pt}

\noindent
\begin{multline}
p_{ij}={\sf P}\left(\xi_{n+1}=j \mid \xi_n=i\right)=
\int\limits_{U_i}p_{ij}(u)\, d\Psi_i(u)\,,\\  
i,j\in X\,; 
\label{e6}
\end{multline}

\vspace*{-9pt}

\noindent
\begin{equation}
T_{i}=\mathbf{E}\left[\theta_n \mid \xi_n=i\right]=
\int\limits_{U_i}T_{i}(u)\,d\Psi_i(u)\,,\enskip i\in X\,. 
\label{e7}
\end{equation}
В дальнейшем будем предполагать, что для рас\-смат\-ри\-ва\-емой 
полумарковской модели заданы вероятностные характеристики 
$p_{ij}(u)$, $u\hm\in U_i$, $i,j\hm\in X$, и~$T_i(u)$, $u\hm\in U_i$, $i\hm\in X$, 
определяемые соотношениями~(\ref{e2}) и~(\ref{e4}). 
Для фиксированной стратегии управления $\Psi\hm=(\Psi_1, \Psi_2,\ldots, \Psi_N)$ 
соответствующие вероятностные характеристики~$p_{ij}$ и~ $T_i$, $i,j\hm\in X,$ 
определены равенствами~(\ref{e6}) и~(\ref{e7}) без условий на управление.

\section{Стационарный стоимостной показатель качества управления}

Определим некоторый стоимостной аддитивный функционал, связанный 
с~рассматриваемым полумарковским процессом~$\xi(t)$. По содержанию этот функционал 
представляет собой случайный\linebreak доход или прибыль, накопленную за период времени $[0,t]$. 
Определения такого функционала приведены в~основополагающих работах~[3; 8, гл.~5].\linebreak 
Обозначим через $\widetilde{v}(t)$, $t\hm\geq 0$, значение этого аддитивного 
функционала в~момент времени~$t$; $\widetilde{v}_n\hm=\widetilde{v}(t_n\hm+0)$~--- 
соответствующее значение непосредственно после очередного момента изменения\linebreak 
состояния~$t_n$, $n\hm=0,1,2,\ldots$; $\widetilde{v}_0\hm=v_0$~--- 
заданное начальное значение в~момент $t\hm=0$. Рассмотрим величину
\begin{multline}
d_i(u)=\mathbf{E}\left[\widetilde{v}_{n+1}-\widetilde{v}_n \mid \xi_n=i\,, 
u_n=u\right]\,,\\
u\in U_i\,, \enskip i\in X\,, \label{e8}
\end{multline}
представляющую собой математическое ожидание приращения стоимостного 
аддитивного функционала за период времени между последовательными 
изменениями состояния полумарковского процесса~$\xi(t)$. Тогда соответствующее 
математическое ожидание, вычисляемое без условия на решение, 
принимаемое в~момент времени~$t_n$, представляется в~виде:
\begin{equation*}
d_i=\mathbf{E}\left[\widetilde{v}_{n+1}-\widetilde{v}_n \mid \xi_n=i\right]=
\!\int\limits_{U_i}\!d_i(u)\,d\Psi_i(u)\,,\ i\in X \,. %\label{e9}
\end{equation*}

Предположим, что для заданной стратегии управ\-ле\-ния 
$\Psi\hm=(\Psi_1,\Psi_2,\ldots,\Psi_N)$ вложенная цепь Маркова~$\{\xi_n\}$ 
имеет ровно один класс возвратных положительных состояний (по терминологии, 
принятой в~\cite{8}, такое множество состояний называется эргодическим классом). 
Как известно~\cite[гл.~VIII]{15}, данное условие является необходимым 
и~достаточным для существования единственного\linebreak стационарного распределения. 
Обозначим это стационарное распределение цепи Маркова~$\{\xi_n\}$ через 
$\pi\hm=(\pi_1, \pi_2,\ldots, \pi_N)$. Заметим, что данное\linebreak распределение зависит  
от стратегии управления $\Psi\hm=(\Psi_1,\Psi_2,\ldots,\Psi_N)$. При указанном 
условии имеет место следующий результат, называемый эргодической теоремой 
для аддитивного стоимостного функционала:
\begin{equation}
I=\lim\limits_{t\rightarrow\infty}\fr{\mathbf{E}\widetilde{v}(t)}{t}=
\fr{\sum\nolimits_{i=1}^N d_i\pi_i}{\sum\nolimits_{i=1}^N T_i\pi_i}\,. 
\label{e10}
\end{equation}

Соотношение~(\ref{e10}) доказано в~работе~\cite[гл.~5]{8}. Заметим, что аналогичные 
результаты имеют мес\-то для гораздо более общих полумарковских моделей~\cite{10, 11}.

По своему прикладному содержанию величина, определяемая соотношением~(\ref{e10}), 
представляет собой
среднюю удельную прибыль, связанную с~эволюцией системы в~стационарном
режиме. Кроме того, величина~$I$ представляет собой функционал от
набора вероятностных распределений~$\Psi_{i}$, $i\hm\in\lbrace 1,\ldots
,N\rbrace $, определяющих стратегию управле-\linebreak\vspace*{-12pt}

\pagebreak

\noindent
ния системой. 
В~дальнейшем будем рассматривать стационарный стоимостной функционал 
$I\hm=I(\Psi_{1},\Psi_{2},\ldots , \Psi_{N})$ как
показатель качества управ\-ле\-ния системой и~построенным полумарковским
процессом~$\xi (t)$.

\section{Представление стационарного показателя в~форме
дробно-линейного интегрального функционала}

В данном разделе будет приведено утверждение об аналитическом
представлении стационарного стоимостного функционала~(\ref{e10}), 
служащего критерием качества управления в~рассматриваемой задаче управления 
полумарковским процессом.

\smallskip

\noindent
\textbf{Теорема 1.} \textit{Стационарный стоимостной показатель, 
определяемый равенством}~(\ref{e10}), \textit{представляет собой дроб\-но-ли\-ней\-ный
функционал от вероятностных распределений~$\Psi_{i}(u_{i})$,
$i\hm\in\{1,\dots,N\}$. Данный функционал задается
аналитически следующей формулой:}
\begin{multline}
I=I(\Psi_{1},\ldots, \Psi_{N})={}\\
\hspace*{-2mm}{}=\!
\fr{\int\nolimits_{U_1}\!{\cdots\! 
\int\nolimits_{U_N}\!{A(u_{1},\ldots ,u_{N})d\Psi_{1}(u_{1})\cdots
\,d\Psi_{N}(u_{N})}}}{\int\nolimits_{U_1}{\!\cdots\! \int\nolimits_{U_N}\!{B(u_{1},\ldots ,u_{N})
\,d\Psi_{1}(u_{1})\ldots
d\Psi_{N}(u_{N})}}},\!\!\! \label{e11}
\end{multline}
\textit{где подынтегральные функции числителя и~знаменателя выражаются
соотношениями}:
\begin{align}
A(u_{1},\ldots
,u_{N})&={}\notag\\
&\hspace*{-20mm}{}=\sum\limits_{i=1}^{N}{d_{i}(u_{i})}{\widehat{D}}^{(i)}(u_{1}, \ldots
,u_{i-1},u_{i+1}, \ldots , u_{N})\,;  \label{e12}\\
 B(u_{1},\ldots
,u_{N})&={}\notag\\
&\hspace*{-20mm}{}=\sum\limits_{i=1}^{N}{T_{i}(u_{i})}{\widehat{D}}^{(i)}(u_{1}, \ldots
,u_{i-1},u_{i+1}, \ldots , u_{N})\,.  \label{e13}
\end{align}
\textit{В свою очередь, функции} ${\widehat{D}}^{(i)}(u_{1}, \ldots
,u_{i-1},u_{i+1}, \ldots$\linebreak $\ldots , u_{N})$, $i\hm\in\{1,\dots,N\}$, 
\textit{входящие в~правые части формул}~(\ref{e12}) и~(\ref{e13}), 
\textit{определяются следующим образом:}

\noindent
\begin{multline}
{\widehat{D}}^{(i)}(u_{1}, \ldots ,u_{i-1},u_{i+1}, \ldots , u_{N})={}
\\
{}=(-1)^{N+i+2}\sum\limits_{\alpha ^{(N),i}}{(-1)}^{\delta (\alpha
^{(N),i}) }{\widehat{D}}_{0}^{(i)}\left(\alpha ^{(N),i},u_{1}, \ldots\right.\\
\left.\ldots , u_{i-1},u_{i+1}, \ldots , u_{N}\right)\,. \label{e14}
\end{multline}
\textit{Здесь} $\alpha ^{(N),i}=(\alpha _{1}, \ldots , \alpha _{i-1},\alpha_{i+1}, \ldots , 
\alpha _{N})$~\textit{--- произвольная
перестановка чисел }$(1, \ldots , i-1, i+1, \ldots , N)$;
$\delta
(\alpha ^{(N),i})$~\textit{--- число инверсий в~перестановке} 
$\alpha ^{(N),i}$;

\noindent
\begin{multline}
{\widehat{D}}_{0}^{(i)}(\alpha ^{(N),i},u_{1}, \ldots ,u_{i-1},u_{i+1},
\ldots , u_{N})={}\\
{} ={\widetilde{p}}_{1,\alpha _{1}}\left(u_{1}\right)\cdots {\widetilde{p}}_{i-1,\alpha
_{i-1}}\left(u_{i-1}\right){\widetilde{p}}_{i+1,\alpha _{i+1}}\left(u_{i+1}\right)\cdots\\
\cdots
{\widetilde{p}}_{N,\alpha _{N}}\left(u_{N}\right)\,, 
\label{e15}
\end{multline}
где
\begin{multline}
 {\widetilde{p}}_{k,\alpha _{k}}(u_{k})=
\begin{cases}
p_{kk}(u_{k})-1,\  & \alpha _{k}=k\,; \\
p_{k,\alpha _{k}}(u_{k}),\  & \alpha _{k}\ne k, \\
\end{cases}\\
 k=1, \ldots , i-1, i+1, \ldots ,N\,. \label{e16}
 \end{multline}
\textit{Функции $p_{ij}(u_i)$, $T_{i}(u_{i})$ и~$d_{i}(u_{i})$,
$u_i\hm\in U_i$, $i,j\hm\in \{1,2,\ldots,N\}$, 
входящие в~соотношения}~(\ref{e12})--(\ref{e16}), 
\textit{определяются равенствами}~(\ref{e2}), (\ref{e4}) \textit{и}~(\ref{e8}) \textit{соответственно.}

\smallskip

\noindent
Д\,о\,к\,а\,з\,а\,т\,е\,л\,ь\,с\,т\,в\,о\ теоремы~1 
в~весьма сжатой форме приведено в~работе~\cite{21}. Читателю, интересующемуся 
более подробным обоснованием данного результата, порекомендуем обратиться к~тексту 
кандидатской диссертации А.\,В.~Иванова~\cite[гл.~3]{22}.

\smallskip

Итак, теорема~1 позволяет получить явное аналитическое представление 
для стационарного стоимостного показателя вида~(\ref{e10}) в~форме 
дроб\-но-ли\-ней\-но\-го интегрального функционала от набора\linebreak вероятностных мер 
$\Psi\hm=(\Psi_{1},\Psi_{2},\ldots , \Psi_{N})$, за\-да\-ющих стратегию управления 
полумарковским процессом~$\xi(t)$. При этом подынтегральные функции числителя 
и~знаменателя задаются формулами~(\ref{e12}), (\ref{e13}) 
и~вспомогательными равенствами~(\ref{e14})--(\ref{e16}). Таким образом, функция
\begin{equation}
C\left(u_1, u_2,\ldots, u_N\right)=\fr{A(u_1, u_2,\ldots, u_N)}{B(u_1, u_2,\ldots, u_N)}\,,
\label{e17}
\end{equation}
которая в~дальнейшем будет называться основной функцией дроб\-но-ли\-ней\-но\-го 
интегрального функционала~(\ref{e11}) и~которая будет играть важную роль 
в~дальнейшем исследовании, также явно определяется формулами~(\ref{e17}), 
(\ref{e12}), (\ref{e13}).

\section{Формальная постановка оптимизационной задачи 
и~условия существования оптимальной стратегии управления полумарковским процессом}

Будем рассматривать проблему управления полумарковским процессом~$\xi(t)$ в~форме 
экстремальной задачи
\begin{multline}
I(\Psi)=I\left(\Psi_1, \Psi_2,\ldots,\Psi_N\right)\rightarrow \mathrm{extr}\,,
\\
\Psi=\left(\Psi_1, \Psi_2,\ldots,\Psi_N\right)\in\Gamma\,. \label{e18}
\end{multline}
При этом показатель качества управления~$I(\Psi)$ представляет собой 
дроб\-но-ли\-ней\-ный интегральный функционал вида~(\ref{e11}).

Для решения экстремальной задачи~(\ref{e18}) воспользуемся некоторым утверждением 
об экстремуме дроб\-но-ли\-ней\-но\-го интегрального функционала. Прежде 
чем сформулировать данное утверждение, отметим, что в~теории оптимизации 
хорошо известны задачи, в~которых целевая функция представляет собой 
отношение двух линейных отображений, а имеющиеся ограничения также линейны. 
Такой раздел называется дроб\-но-ли\-ней\-ным программированием. Основные
 теоретические результаты данного направления изложены в~работе~\cite{23},
  там же приведена подробная библиография. В~дальнейшем потребуется некоторый 
  специальный результат о безусловном экстремуме дроб\-но-ли\-ней\-но\-го 
  интегрального функционала вида~(\ref{e11}), который был впервые сформулирован 
  в~работе~\cite{14}. Заметим, что для использования этого результата необходимо, 
  чтобы выполнялись некоторые предварительные условия, которые в~данном случае 
  можно сформулировать следующим образом:
\begin{enumerate}[1.]
\item Интегральные выражения
\begin{align*}
I_1(\Psi)&=I_1\left(\Psi_1,\Psi_2,\ldots,\Psi_N\right)={}&\\
&\hspace*{-13mm}{}=\int\limits_{U_1}\!\cdots\!
\int\limits_{U_N}\!\!A\left(u_1,\ldots ,u_N\right)\,
d\Psi_1\left(u_1\right) %d\Psi_2\left(u_2\right)
\cdots
 d\Psi_N\left(u_N\right)\,;
\\
I_2(\Psi)&=I_2\left(\Psi_1,\Psi_2,\ldots,\Psi_N\right)={}&\\
&\hspace*{-13mm}{}=\int\limits_{U_1}\!\cdots\!\int\limits_{U_N}\!\!
B\left(u_1,\ldots,u_N\right)\,
d\Psi_1\left(u_1\right)% d\Psi_2\left(u_2\right)\cdots\\
\cdots d\Psi_N\left(u_N\right)
\end{align*}
определены для всех стратегий управления $\Psi\hm=(\Psi_1, \ldots,\Psi_N)
\hm\in \Gamma$.

\item Функционал $I_2(\Psi)=I_2(\Psi_1, \ldots,\Psi_N)\hm\neq 0$ 
для всех $\Psi\hm=(\Psi_1, \ldots,\Psi_N)\hm\in \Gamma$.

\item Множество $\Gamma$ включает в~себя множество всех вырожденных 
вероятностных мер: $\Gamma^* \hm\subset \Gamma$.
\end{enumerate}

Сделаем несколько важных замечаний по поводу введенных предварительных условий.

\smallskip

\noindent
\textbf{Замечание~1.}\ Из условия~2 следует, что функция $B(u_1, u_2,\ldots, u_N)$ 
не может принимать значения разных знаков. С~учетом условия~3 
получаем, что указанная функция должна обладать \mbox{свойством} строгой 
знакопостоянности на всем множестве~$U$. С~другой стороны, если выполняется 
условие строгой знакопостоянности функции $B(u_1, u_2,\ldots, u_N), 
(u_1, u_2,\ldots, u_N)\hm\in U$, то условие~2 выполняется автоматически.

\smallskip

\noindent
\textbf{Замечание~2.}\ Если рассматривать в~качестве целевого функционала 
$I(\Psi_1, \Psi_2,\ldots,\Psi_N)$ экстремальной задачи~(\ref{e18}) 
стационарный стоимостной пока\-затель~(\ref{e10}), то функция $B(u_1,u_2,\ldots,u_N)$ 
имеет\linebreak следующее теоретическое содержание. Данная функция представляет собой условное 
математическое ожидание длительности периода времени между соседними моментами 
изменения со\-сто\-яния полумарковского процесса~$\xi(t)$ при условии, что стратегия 
его управ\-ле\-ния является детерминированной и~задается набором значений аргументов 
$(u_1,u_2,\ldots,u_N)$. Тогда условие строгой положительности функции 
$B(u_1,u_2,\ldots,u_N)$ при всех $(u_1,u_2,\ldots,u_N)\hm\in U$ является естественным 
и~фактически означает, что при любой заданной детерминированной стратегии 
управ\-ле\-ния процесс~$\xi(t)$ не имеет мгновенных со\-сто\-яний, длительность пребывания 
в~которых равна нулю.

\smallskip

\noindent
\textbf{Замечание~3.}\ Сделаем некоторые замечания, связан\-ные с~подынтегральной 
функцией числителя дроб\-но-ли\-ней\-но\-го интегрального функционала~(\ref{e11}). 
Как и~ранее, будем рассматривать в~качестве целевого функционала $I(\Psi_1, \Psi_2,\ldots,\Psi_N)$\linebreak 
экстремальной задачи~(\ref{e18}) стационарный стоимостной показатель~(\ref{e10}). 
Тогда для любого фиксированного набора значений аргументов $(u_1,u_2,\ldots,u_N)\hm\in U$ 
значение функции $A(u_1,u_2,\ldots\linebreak \ldots,u_N)$ представляет собой условное математическое
 ожидание приращения рассматриваемого стоимостного функционала, 
 происшедшее за время пребывания полумарковского процесса~$\xi(t)$ в~некотором 
 фиксированном  состоянии при условии, что стратегия управления является 
 детерминированной и~задается указанным набором $(u_1,u_2,\ldots,u_N)\hm\in U$. 
 Отметим, что в~теореме об экстремуме дроб\-но-ли\-ней\-но\-го интегрального 
 функционала, доказанной в~работе~\cite[гл.~10]{12}, 
 на подынтегральную функцию числителя накладываются условия ограниченности на 
 всем множестве значений аргумента. Для многих математических моделей и~связанных 
 с~ними задач оптимального управления такое условие является излишне ограничительным. 
 В~качестве примера можно привести модели оптимального управления запасом непрерывного 
 продукта, рассмотренные в~работах~\cite{27, 28}. 
 В~настоящем исследовании на функцию $A(u_1,u_2,\ldots,u_N)$ накладывается только 
 условие интегрируемости по любому заданному набору вероятностных мер 
 $\Psi\hm=(\Psi_1, \Psi_2,\ldots,\Psi_N)$, образующему стратегию управления 
 полумарковским процессом~$\xi(t)$ (условие~1 системы предварительных условий).

\smallskip

\noindent
\textbf{Замечание~4.} Условия~1--3 являются необходимыми для корректной 
постановки задачи безусловного экстремума дроб\-но-ли\-ней\-но\-го интегрального 
функционала. Если этот функционал служит показателем качества в~задаче оптимального 
управления случайным процессом, то необходимо добавить к~этим условиям дополнительное, 
связанное с~некоторой регулярностью самого управляемого процесса, а~именно: некоторый 
содержательный показатель, связанный с~поведением этого процесса, должен существовать 
и~быть представимым в~виде дроб\-но-ли\-ней\-но\-го интегрального функционала. 
Если потребовать, чтобы выполнялось эргодическое соотношение~(\ref{e10}), 
то можно использовать\linebreak теорему~1 и~сформулировать задачу оптимального управ\-ле\-ния 
в~виде~(\ref{e18}) для дроб\-но-ли\-ней\-но\-го\linebreak интегрального функционала~(\ref{e11}). 
Таким образом, необходимо ввести условие, обеспечивающее существование единственного 
стационарного распределения вложенной цепи Маркова и~выполнение\linebreak соотношения~(\ref{e10}). 
По аналогии с~[8, гл.~5] сформулируем это дополнительное условие в~следующем виде:
\begin{enumerate}
\setcounter{enumi}{3}
\item Для любой рассматриваемой стратегии управ\-ле\-ния $\Psi\hm=
(\Psi_1, \Psi_2,\ldots,\Psi_N)\hm\in \Gamma$ вложенная цепь Маркова 
полумарковского процесса $\xi(t)$ имеет ровно один класс возвратных 
положительных состояний.
\end{enumerate}

Теперь определим понятие допустимой стратегии управления полумарковским процессом 
с~конечным множеством состояний.

\smallskip

\noindent
\textbf{Определение~2.}\ Назовем стратегию управления 
$\Psi\hm=(\Psi_1, \Psi_2,\ldots,\Psi_N)$ 
допустимой в~данной задаче, если она удовлетворяет условиям~1--4.


\smallskip

\noindent
\textbf{Замечание~5.}\ Как следует из замечания~1, если потребовать, 
чтобы функция $B(u_1, u_2,\ldots,u_N)$ являлась строго знакопостоянной при 
всех $(u_1, u_2,\ldots,u_N)\hm\in U$, то можно считать допустимыми стратегии 
$(\Psi_1, \Psi_2,\ldots,\Psi_N)$, удовлетворяющие условиям~1, 3, 4. С~учетом замечания~2 
о~естественном характере условия строгой знакопостоянности функции $B(u_1,u_2,\ldots,u_N)$ 
при всех значениях аргументов $(u_1, u_2,\ldots,u_N)\hm\in U$ будем требовать 
выполнения этого условия в~формулировке приводимой в~дальнейшем основной 
теоремы об оптимальной стратегии управления полумарковским процессом.

\smallskip

\noindent
\textbf{Замечание~6.}\ Ниже будет сформулирована и~доказана основная 
теорема об оптимальной стра\-тегии управления полумарковским процессом с~конеч\-ным 
множеством состояний. Будем формулировать эту теорему по отношению к~экстремальной 
задаче~(\ref{e18}), в~которой целевой функционал $I(\Psi_1, \Psi_2,\ldots,\Psi_N)$ 
имеет вид дроб\-но-ли\-ней\-но\-го интегрального функционала. 
Это обстоятельство связано с~тем, что целевой функционал в~задаче 
оптимального управления необязательно должен иметь характер стационарного 
стоимостного показателя вида~(\ref{e10}). В~частности, еще в~1983~г.\ П.\,В.~Шнурковым 
было установлено~\cite{24}, что ряд показателей, связанных 
с~временем пребывания управляемого полумарковского процесса в~заданном конечном 
подмножестве состояний, имеет структуру дроб\-но-ли\-ней\-но\-го интегрального 
функционала от набора вероятностных мер, определяющих стратегию управления. 
Таким образом, рассматриваемая задача управления имеет более общий характер, 
чем задача, в~которой целевой функционал представляет собой стационарный 
стоимостной показатель вида~(\ref{e10}).






\smallskip

\noindent
\textbf{Замечание~7.}\ Если рассматривать задачу оптимального управления 
полумарковским процессом, в~кото\-рой целевой функционал не совпадает 
со стационарным стоимостным показателем~(\ref{e10}), то возможно, что могут 
потребоваться другие дополнительные условия, обеспечивающие существование этого 
показателя и~его представление в~форме~(\ref{e11}). В~связи с~этим в~формулировке 
основной теоремы будем использовать термин допустимые стратегии в~широком смысле, 
имея в~виду выполнение всех необходимых условий для каждого рассмат\-ри\-ва\-емо\-го 
показателя качества управления.

\smallskip


\noindent
\textbf{Замечание 8.} Множество допустимых стратегий может 
не совпадать с~множеством всех возможных стратегий управления. 
В~частности, допустимые стратегии могут состоять только из дискретных вероятностных 
мер $\Psi_1, \Psi_2,\ldots,\Psi_N$, т.\,е.\ таких, которые сосредоточены на дискретных 
множествах точек пространств $U_1, U_2,\ldots,U_N$.

\section{Теоретическое решение задачи оптимального управления}

Перейдем к~формулировке и~доказательству тео\-ре\-мы об 
оптимальной стратегии управ\-ле\-ния полумарковским процессом с~конечным 
множеством состояний.

\smallskip

\noindent
\textbf{Теорема~2.} \textit{Рассмотрим проблему оптимального управ\-ле\-ния 
полумарковским процессом~$\xi(t)$ в~виде экстремальной задачи}~(\ref{e18}), 
\textit{определенной на множестве допустимых стратегий $\Gamma$, 
для дроб\-но-ли\-ней\-но\-го 
функционала}~(\ref{e11}). \textit{Пусть функция $B(u_1,u_2,\ldots,u_N)$, 
входящая в~определение функционала}~(\ref{e11}),
\textit{является строго знакопостоянной (строго положительной или строго отрицательной) 
при всех значениях аргументов $(u_1,u_2,\ldots,u_N)\hm\in U$.
Тогда справедливы сле\-ду\-ющие утверждения}:
\begin{enumerate}[1.]
\item \textit{Если функция} $C(u_1,u_2,\ldots,u_N)\hm=A(u_1,u_2,\ldots$\linebreak
$\ldots,u_N)/{B(u_1,u_2,\ldots,u_N)}$ 
\textit{ограничена сверху или снизу и~достигает глобального экст\-ре\-му\-ма на множестве
$U\hm=U_1\times U_2\times \cdots \times U_N$ (максимума или минимума), 
то оптимальная стратегия управления полумарковским процессом~$\xi(t)$ существует, 
является детерминированной и~определяется
вырожденной вероятностной мерой $\Psi^*\hm\in \Gamma^*$, сосредоточенной в~точке, 
в~которой достига\-ет соответствующего экстремума функция $C(u_1,u_2,\ldots,u_N)$,
и~при этом выполняются соотношения}:
\begin{multline}  %{\substack{{i=\overline{1,n}}\\ {j=\overline{1,l}}}}
\max\limits_{\Psi \in \Gamma} I(\Psi)=
\max\limits_{\substack{{\Psi_i \in \Gamma_i\,,}\\ 
{i=\overline{1,N}}}}
I\left(\Psi_1,\Psi_2,\ldots,\Psi_N\right)={}\\
{}=
\max\limits_{\substack{{\Psi_i^* \in \Gamma_i^*,}\\ 
{i=\overline{1,N}}}}
 I\left(\Psi_1^*,\Psi_2^*,\ldots,\Psi_N^*\right)={}\\
{}=\max\limits_{(u_1,u_2,\ldots,u_N)\in U}\fr{A(u_1,u_2,\ldots,u_N)}
{B(u_1,u_2,\ldots,u_N)}\,; \label{e19}
\end{multline}

\vspace*{-12pt}

\noindent
\begin{multline*}
\min\limits_{\Psi \in \Gamma} I(\Psi)=
\min\limits_{\substack{{\Psi_i \in \Gamma_i\,,}\\ 
{i=\overline{1,N}}}} I\left(\Psi_1,\Psi_2,\ldots,\Psi_N\right)={}\\
{}=
\min\limits_{\substack{{\Psi_i^* \in \Gamma_i^*,}\\ 
{i=\overline{1,N}}}}
I\left(\Psi_1^*,\Psi_2^*,\ldots,\Psi_N^*\right)={}\\
{}=\min\limits_{(u_1,u_2,\ldots,u_N)\in U}\fr{A(u_1,u_2,\ldots,u_N)}
{B(u_1,u_2,\ldots,u_N)}\,. %\label{e20}
\end{multline*}
\item \textit{Если функция $C(u_1,u_2,\ldots,u_N)\hm=
{A(u_1,u_2,\ldots,u_N)}/{B(u_1,u_2,\ldots,u_N)}$ ограничена сверху или снизу, 
но не достигает глобального экстремума на множестве $U\hm=U_1\times U_2\times\cdots
\times U_N$,
то для любого $\varepsilon\hm > 0$ можно выбрать $\varepsilon$-оп\-ти\-маль\-ную 
детерминированную стратегию управления полумарковским процессом~$\xi(t)$, 
которая определяется вырожденной
вероятностной мерой $\Psi^{*(+)}(\varepsilon)\hm\in \Gamma^*$ или вырожденной
вероятностной мерой $\Psi^{*(-)}(\varepsilon)\hm\in \Gamma^*$, в~зависимости от 
вида экстремума (максимума или минимума) в~задаче}~(\ref{e18}). 
\textit{При этом вероятностная мера $\Psi^{*(+)}(\varepsilon)\hm\in \Gamma^*$ может быть 
сосредоточена в~любой точке $\left(u_1^{(+)}(\varepsilon),u_2^{(+)}(\varepsilon),\ldots,
u_N^{(+)}(\varepsilon)\right)$, удовлетворяющей соотношению}:
\begin{multline}
\sup\limits_{(u_1,u_2,\ldots,u_N) \in U}
\fr{A(u_1,u_2,\ldots,u_N)}{B(u_1,u_2,\ldots,u_N)}-\varepsilon <{}\\
{}<
\fr{A\left(u_1^{(+)}(\varepsilon),u_2^{(+)}(\varepsilon),\ldots,u_N^{(+)}
(\varepsilon)\right)}
{B\left(u_1^{(+)}(\varepsilon),u_2^{(+)}(\varepsilon),\ldots,u_N^{(+)}
(\varepsilon)\right)}<{}\\
{}<\sup\limits_{(u_1,u_2,\ldots,u_N) \in U}
\fr{A(u_1,u_2,\ldots,u_N)}{B(u_1,u_2,\ldots,u_N)}<\infty\,, 
\label{e21}
\end{multline}
\textit{если функция $C(u_1,u_2,\ldots,u_N)$ ограничена сверху 
и~экстремальная задача}~(\ref{e18}) 
\textit{представляет собой задачу на максимум. Аналогично вероятностная мера 
$\Psi^{*(-)}(\varepsilon)\hm\in \Gamma^*$ может быть сосредоточена в~любой точке 
$\left(u_1^{(-)}(\varepsilon),u_2^{(-)}(\varepsilon),\ldots,u_N^{(-)}(\varepsilon)
\right)$, удовлетворяющей соотношению}:

\noindent
\begin{multline*}
-\infty<\inf\limits_{(u_1,u_2,\ldots,u_N) \in U}\fr{A(u_1,u_2,\ldots,u_N)}
{B(u_1,u_2,\ldots,u_N)} <{}\\
{}<
\fr{A\left(u_1^{(-)}(\varepsilon),u_2^{(-)}
(\varepsilon),\ldots,u_N^{(-)}(\varepsilon)\right)}
{B\left(u_1^{(-)}(\varepsilon),u_2^{(-)}(\varepsilon),\ldots,
u_N^{(-)}(\varepsilon)\right)}<{}\\
{}<\inf\limits_{(u_1,u_2,\ldots,u_N) \in U}
\fr{A(u_1,u_2,\ldots,u_N)}{B(u_1,u_2,\ldots,u_N)}+\varepsilon\,, 
%\label{e22}
\end{multline*}
\textit{если функция $C(u_1,u_2,\ldots,u_N)$ ограничена снизу и~экстремальная 
задача}~(\ref{e18})  \textit{представляет собой задачу на минимум}.
\item \textit{Если функция $C(u_1,u_2,\ldots,u_N)\hm=
{A(u_1,u_2,\ldots,u_N)}/{B(u_1,u_2,\ldots,u_N)}$ не ограничена сверху 
или снизу, то оптимальной стратегии управления в~смысле
соответствующей экстремальной задачи не существует. 
При этом найдется такая последовательность вырожденных вероятностных
мер~$\Psi^{*(+)}(n)$, сосредоточенных в~точках 
$\left(u_1^{(+)}(n),u_2^{(+)}(n),\ldots,u_N^{(+)}(n)\right)$, $n\hm=1,2,\dots $, 
для которых выполняется соотношение}:
\begin{multline*}
I\left(\Psi^*(n)\right)={}\\
{}=
I\left(\Psi_1^{*(+)}(n),\Psi_2^{*(+)}(n),\ldots,\Psi_N^{*(+)}(n)\right)={}\\
{}=\fr{A\left(u_1^{(+)}(n),u_2^{(+)}(n),\ldots,u_N^{(+)}(n)\right)}
{B\left(u_1^{(+)}(n),u_2^{(+)}(n),\ldots,u_N^{(+)}(n)\right)}\to 
\infty\\
\mbox{при}\ n\to\infty\,, 
%\label{e23}
\end{multline*}
\textit{если функция $C(u_1,u_2,\ldots,u_N)$ не ограничена сверху. 
Аналогично найдется такая последовательность вырожденных вероятностных
мер~$\Psi^{*(-)}(n)$, сосредоточенных в~точках 
$\left(u_1^{(-)}(n),u_2^{(-)}(n),\ldots,u_N^{(-)}(n)\right)$, 
$n\hm=1,2,\dots $, для которых выполняется соотношение}:
\begin{multline*}
I\left(\Psi^{*(-)}(n)\right)={}\\
{}= I
\left(\Psi_1^{*(-)}(n),\Psi_2^{*(-)}(n),\ldots,\Psi_N^{*(-)}(n)\right)={}\\
{}=\fr{A\left(u_1^{(-)}(n),u_2^{(-)}(n),\ldots,u_N^{(-)}(n)\right)}
{B\left(u_1^{(-)}(n),u_2^{(-)}(n),\ldots,u_N^{(-)}(n)\right)}\to 
-\infty\\
\mbox{при}~~n\to\infty\,,  
%\label{e24}
\end{multline*}
\textit{если функция $C(u_1,u_2,\ldots,u_N)$ не ограничена \mbox{снизу}}.
\end{enumerate}
\textit{При этом сформулированные утверждения каждого пункта теоремы~$2$ 
могут выполняться как по отдельности, для одного из двух
видов экстремума, так и~совместно, для обоих видов экстремума.}

\smallskip

Прежде чем непосредственно доказывать теорему~2, докажем некоторые 
вспомогательные утверждения.

\smallskip

\noindent
\textbf{Лемма~1.}\ 
\textit{Рассмотрим дроб\-но-ли\-ней\-ный интегральный функционал 
$I(\Psi_1, \Psi_2,\ldots, \Psi_N)$ вида}~(\ref{e11}), 
\textit{заданный на некотором множестве наборов вероятностных мер 
$\Psi\hm=(\Psi_1, \Psi_2,\ldots, \Psi_N)\hm \in \Gamma$. Предположим, что на 
множестве~$\Gamma$ выполняется условие~$1$ из набора предварительных условий 
и~функция $B(u_1, u_2,\ldots, u_N)$  обладает свойством строгой знакопостоянности 
при всех $(u_1, u_2,\ldots, u_N) \hm\in U$. Тогда справедливы следующие утверждения}:
\begin{enumerate}[1.]
\item \textit{Если основная функция 
$C(u_1, u_2,\ldots, u_N)\hm={A(u_1, u_2,\ldots, u_N)}/{B(u_1, u_2,\ldots, u_N)}$ 
ограничена сверху, т.\,е.\ выполняется условие}
\begin{multline}
C\left(u_1, u_2,\ldots, u_N\right)=
\fr{A(u_1, u_2,\ldots, u_N)}{B(u_1, u_2,\ldots, u_N)}\leq {}\\
{}\leq
c_0^{(+)}<\infty \,, \enskip \left(u_1, u_2,\ldots, u_N\right) \in U\,, \label{e25}
\end{multline}
\textit{то имеет место неравенство}:
\begin{equation}
I\left(\Psi_1, \Psi_2,\ldots, \Psi_N\right)\leq c_0^{(+)} 
\label{e26}
\end{equation}
\textit{для всех} $(\Psi_1, \Psi_2,\ldots, \Psi_N) \in \Gamma$.
\item \textit{Если основная функция 
$C(u_1, u_2,\ldots, u_N)\hm={A(u_1, u_2,\ldots, u_N)}/{B(u_1, u_2,\ldots, u_N)}$ 
ограничена снизу, т.\,е.\ выполняется условие}
\begin{multline*}
C\left(u_1, u_2,\ldots, u_N\right)=\fr{A(u_1, u_2,\ldots, u_N)}{B(u_1, u_2,\ldots, 
u_N)}\geq{}\\
{}\geq c_0^{(-)}>-\infty \,, 
\left(u_1, u_2,\ldots, u_N\right) \in U\,, 
%\label{e27}
\end{multline*}
\textit{то имеет место неравенство}:
\begin{equation*}
I\left(\Psi_1, \Psi_2,\ldots, \Psi_N\right)\geq c_0^{(-)} 
%\label{e28}
\end{equation*}
\textit{для всех} $(\Psi_1, \Psi_2,\ldots, \Psi_N) \hm\in \Gamma$.
\end{enumerate}

\noindent
Д\,о\,к\,а\,з\,а\,т\,е\,л\,ь\,с\,т\,в\,о\ \ леммы~1.\ 
Докажем первое утверждение леммы. Предположим сначала, 
что функция $B(u_1, u_2,\ldots,  u_N)$ строго положительна:
\begin{equation}
B\left(u_1, u_2,\ldots, u_N\right)>0\,,\enskip
\left(u_1, u_2,\ldots, u_N\right)\in U\,. \label{e29}
\end{equation}
Заметим, что в~таком случае по свойству интеграла~\cite[гл.~V]{18}
\begin{multline}
\hspace*{-2mm}\int\limits_{U_1}\!\!\cdots\! \!\int\limits_{U_N}\!\!B(u_1, \ldots,u_N) \,
d\Psi_1(u_1)%d\Psi_2(u_2)\cdots\\
\cdots d\Psi_N(u_N)>0 \!\!\!\!\label{e30}
\end{multline}
для любого фиксированного набора $\Psi\hm=(\Psi_1, \ldots, \Psi_N)\hm\in \Gamma$.
Из неравенства~(\ref{e25}) с~уче\-том~(\ref{e29}) получаем:
\begin{multline}
\hspace*{-4mm}A\left(u_1,\ldots, u_N\right)\leq{}\\
\hspace*{-4mm}{}\leq c_0^{(+)} B\left(u_1, \ldots, u_N\right)\,, 
\left(u_1, \ldots, u_N\right)\in U\,. \label{e31}
\end{multline}
В свою очередь, из неравенства~(\ref{e31}) и~свойств интеграла следует:
\begin{multline}
\int\limits_{U_1}\!\!\cdots\! \!\int\limits_{U_N}\!\!A(u_1,\ldots, u_N) \,
d\Psi_1\left(u_1\right)%d\Psi_2\left(u_2\right)\cdots\\
\cdots d\Psi_N\left(u_N\right)\leq\\
\hspace*{-24pt}\leq 
c_0^{(+)}\!\!\int\limits_{U_1}\!\!\cdots\!\! \int\limits_{U_N}\!\!\!B\!\left(u_1,\ldots, u_N\right)
 d\Psi_1\!\left(u_1\right)\!%d\Psi_2\left(u_2\right)\cdots\\
 \cdots d\Psi_N\!\left(u_N\right)\!\! 
 \label{e32}
\end{multline}
для любого фиксированного набора $\Psi\hm=(\Psi_1, \ldots, \Psi_N)\hm\in \Gamma$. 
Но тогда из~(\ref{e32}) с~учетом~(\ref{e30}) получаем:
\begin{multline}
I(\Psi_1, \ldots, \Psi_N)={}\\
{}=
\fr{\int\nolimits_{U_1}\!\cdots\! \int\nolimits_{U_N}\!\!A\left(u_1, \ldots, u_N\right)\,
 d\Psi_1(u_1)\cdots d\Psi_N(u_N)}{
\int\nolimits_{U_1}\!\cdots\! \int\nolimits_{U_N}\!\!B\left(u_1, \ldots, u_N\right)\,
 d\Psi_1(u_1)
 \cdots d\Psi_N(u_N)}\leq{}\\
 {}\leq c_0^{(+)} 
 \label{e33}
\end{multline}
для любого фиксированного набора $(\Psi_1, \ldots\linebreak\ldots, \Psi_N)\hm\in \Gamma$.

Предположим теперь, что функция $B(u_1,\ldots, u_N)$ строго отрицательна:
\begin{equation}
B(u_1,\ldots, u_N)<0 \quad \left(u_1, \ldots, u_N\right)\in U\,. 
\label{e34}
\end{equation}
Тогда
\begin{multline}
\hspace*{-6pt}\int\limits_{U_1}\!\!\cdots\!\! \int\limits_{U_N}\!\!B\!\left(u_1,\ldots, u_N\right)\!
 d\Psi_1(u_1) \cdots d\Psi_N(u_N)<0 \!\!\!
 \label{e35}
\end{multline}
для любого фиксированного набора $(\Psi_1, \ldots\linebreak \ldots, \Psi_N)\hm\in \Gamma$.

Как и~ранее, будем исходить из неравенства~(\ref{e25}). 
При выполнении условий~(\ref{e34}) и~(\ref{e35}) характер неравенств~(\ref{e31}) 
и~(\ref{e32}) меняется на противоположный, но характер неравенства~(\ref{e33}) 
остается неизменным. Таким образом, для любой функции 
$B(u_1, u_2,\ldots, u_N)$, обладающей свойством строгой знакопостоянности, 
из условия~(\ref{e25}) следует выполнение неравенства~(\ref{e33}), 
которое совпадает с~(\ref{e26}). Первое утверждение леммы~1 доказано. 
Второе утверждение доказывается аналогично. Лемма~1 доказана.

\smallskip

\noindent
\textbf{Лемма 2.} \textit{Рассмотрим дроб\-но-ли\-ней\-ный интегральный функционал 
$I(\Psi_1, \Psi_2,\ldots, \Psi_N)$ вида}~(\ref{e11}), 
\textit{заданный на некотором множестве наборов вероятностных мер 
$\Psi\hm=(\Psi_1, \Psi_2,\ldots, \Psi_N)\hm\in \Gamma$. Предпо\-ложим, что на 
множестве~$\Gamma$ выполняется условие~$1$ из набора предварительных условий 
и~функция $B(u_1, u_2,\ldots, u_N)$ обладает свойством строгой знакопостоянности 
при всех $(u_1, u_2,\ldots, u_N)\hm\in U$. Тогда справедливы следующие утверждения}:
\begin{enumerate}[1.]
\item \textit{Если основная функция $C(u_1, u_2,\ldots, u_N)\hm=
{A(u_1, u_2,\ldots, u_N)}/{B(u_1, u_2,\ldots, u_N)}$ ограничена сверху, 
но не достигает своего максимального 
значения, то имеет место неравенство}:
\begin{multline}
I\left(\Psi_1, \Psi_2,\ldots, \Psi_N\right)<{}\\
{}< \sup\limits_{(u_1, u_2,\ldots, u_N)\in U}
 C\left(u_1, u_2,\ldots, u_N\right)<\infty \label{e36}
\end{multline}
\textit{для всех} $(\Psi_1, \Psi_2,\ldots, \Psi_N)\in \Gamma$.
\item \textit{Если основная функция $C(u_1, u_2,\ldots, u_N)\hm=
{A(u_1, u_2,\ldots, u_N)}/{B(u_1, u_2,\ldots, u_N)}$ ограничена снизу, 
но не достигает своего минимального значения, то имеет место неравенство}:
\begin{multline*}
I\left(\Psi_1, \Psi_2,\ldots, \Psi_N\right)>{}\\
{}> \inf\limits_{(u_1, u_2,\ldots, u_N)\in U} 
C\left(u_1, u_2,\ldots, u_N\right)>-\infty 
%\label{e37}
\end{multline*}
\textit{для всех} $(\Psi_1, \Psi_2,\ldots, \Psi_N)\hm\in \Gamma$.
\end{enumerate}

\noindent
Д\,о\,к\,а\,з\,а\,т\,е\,л\,ь\,с\,т\,в\,о\ \ леммы~2. 
Докажем первое утверждение леммы. Поскольку множество значений 
основной функции $C(u_1, u_2,\ldots, u_N)$ ограничено сверху, оно имеет конечную 
верхнюю грань:
$$
\exists \sup\limits_{(u_1, u_2,\ldots, u_N)\in U} 
C\left(u_1, u_2,\ldots, u_N\right)<\infty
$$
(см.~\cite[гл.~1, \S3, п.~3.4, теорема~1]{25}).

По условию функция $C(u_1, u_2,\ldots, u_N)$ не достигает своего максимального 
значения. Следовательно, выполняется неравенство:
\begin{multline}
C(u_1, u_2,\ldots, u_N)=\fr{A(u_1, u_2,\ldots, u_N)}{B(u_1, u_2,\ldots, u_N)}<{}\\
{}< 
\sup\limits_{(u_1, u_2,\ldots, u_N)\in U} C(u_1, u_2,\ldots, u_N)<\infty\,, 
\\
\left(u_1, u_2,\ldots, u_N\right)\in U\,.
\label{e38}
\end{multline}
Взяв за основу строгое неравенство~(\ref{e38}), проведем рассуждения, аналогичные тем, 
которые были проведены в~лемме~1 по отношению к~неравенству~(\ref{e25}). 
В~результате получим строгое неравенство~(\ref{e36}).

Второе утверждение леммы~2 доказывается аналогично. Лемма~2 доказана.

\noindent
Д\,о\,к\,а\,з\,а\,т\,е\,л\,ь\,с\,т\,в\,о\ 
\ теоремы~2.
Начнем с~доказательства утверждения~1. Предположим сначала, что основная 
функция $C(u_1, u_2,\ldots, u_N)={A(u_1, u_2,\ldots, u_N)}/{B(u_1, u_2,\ldots, u_N)}$ 
ограничена сверху и~достигает глобального максимума на множестве~$U$ 
в~некоторой точке $u^{(+)}\hm=\left(u^{(+)}_1,u^{(+)}_2,\ldots,u^{(+)}_N\right)\hm\in U$,
а~именно:
\begin{multline*}
\max\limits_{(u_1, u_2,\ldots, u_N)\in U} C\left(u_1, u_2,\ldots, u_N\right) = {}\\
{}=
C\left(u^{(+)}_1,u^{(+)}_2,\ldots,u^{(+)}_N\right)<\infty\,.
\end{multline*}
Тогда выполняется соотношение:
\begin{multline}
C(u_1, u_2,\ldots, u_N)=\fr{A(u_1, u_2,\ldots, u_N)}{B(u_1, u_2,\ldots, u_N)}
\leq{}\\
{}\leq C\left(u^{(+)}_1,u^{(+)}_2,\ldots,u^{(+)}_N\right)<\infty\,, 
\\
\left(u_1, u_2,\ldots, u_N\right)\in U\,.
\label{e39}
\end{multline}
Условия леммы~1 выполнены, и~можно воспользоваться ее утверждениями. 
Согласно первому из них, если выполняется неравенство~(\ref{e39}), 
то имеет место соотношение:
\begin{equation*}
I(\Psi_1, \Psi_2,\ldots, \Psi_N)\leq 
C\left(u^{(+)}_1,u^{(+)}_2,\ldots,u^{(+)}_N\right)<\infty 
%\label{e40}
\end{equation*}
для всех стратегий управления $\Psi\hm=(\Psi_1, \Psi_2,\ldots\linebreak
\ldots, \Psi_N)\hm\in \Gamma$.

Таким образом, множество значений дроб\-но-ли\-ней\-но\-го интегрального 
функционала $I(\Psi_1, \Psi_2,\ldots, \Psi_N)$ ограничено сверху при всех 
$\Psi\hm=(\Psi_1, \Psi_2,\ldots, \Psi_N)\hm\in \Gamma$. Тогда существует верхняя 
грань этого множества и~выполняется неравенство:
\begin{multline}
\sup\limits_{(\Psi_1, \Psi_2,\ldots, \Psi_N)\in \Gamma} 
I\left(\Psi_1, \Psi_2,\ldots, \Psi_N\right)\leq {}\\
{}\leq
C\left(u^{(+)}_1,u^{(+)}_2,\ldots,u^{(+)}_N\right). \label{e41}
\end{multline}
Рассмотрим детерминированную стратегию управ\-ле\-ния 
$\Psi^{*(+)}\hm=\left(\Psi_1^{*(+)}, \Psi_2^{*(+)},\ldots, \Psi_N^{*(+)}\right)$, 
в~которой каждая вероятностная мера~$\Psi_i^{*(+)}$ является вы\-рож\-ден\-ной 
и~сосредоточена в~точке $u_i^{(+)}$, $i\hm=\overline{1, N}$.
По свойству интеграла
\begin{multline}
I\left(\Psi_1^{*(+)}, \Psi_2^{*(+)},\ldots ,\Psi_N^{*(+)}\right)={}\\
{}=
C\left(u^{(+)}_1,u^{(+)}_2,\ldots,u^{(+)}_N\right). \label{e42}
\end{multline}
Из соотношений~(\ref{e41}) и~(\ref{e42}) получаем:
\begin{multline}
\sup\limits_{(\Psi_1, \Psi_2,\ldots, \Psi_N)\in \Gamma} 
I\left(\Psi_1, \Psi_2,\ldots, \Psi_N\right)\leq{}\\
{}\leq
 I\left(\Psi_1^{*(+)}, 
\Psi_2^{*(+)},\ldots, \Psi_N^{*(+)}\right). \label{e43}
\end{multline}
Заметим дополнительно, что выполняются отношения принадлежности:
\begin{equation}
\Psi^{*(+)}=\left(\Psi_1^{*(+)}, \Psi_2^{*(+)},\ldots, \Psi_N^{*(+)}\right) 
\in \Gamma^* \subset \Gamma\,. \label{e44}
\end{equation}
Из~(\ref{e44}) и~свойства верхней грани следует:
\begin{multline}
\sup\limits_{\left(\Psi_1^{*}, \Psi_2^{*},\ldots, \Psi_N^{*}\right) \in \Gamma^*} 
I\left(\Psi_1^{*}, \Psi_2^{*},\ldots, \Psi_N^{*}\right)\leq {}\\
{}\leq
\sup\limits_{\left(\Psi_1, \Psi_2,\ldots, \Psi_N\right) 
\in \Gamma} I\left(\Psi_1, \Psi_2,\ldots, \Psi_N\right)\,. 
\label{e45}
\end{multline}
Объединяя~(\ref{e42}), (\ref{e43}) и~(\ref{e45}), получаем соотношение:
\begin{multline}
\sup\limits_{\left(\Psi_1^{*}, \Psi_2^{*},\ldots, \Psi_N^{*}\right) 
\in \Gamma^*} I\left(\Psi_1^{*}, \Psi_2^{*},\ldots, 
\Psi_N^{*}\right)\leq{}\\
{}\leq \sup\limits_{\left(\Psi_1, \Psi_2,\ldots, \Psi_N\right) 
\in \Gamma} I\left(\Psi_1, \Psi_2,\ldots, \Psi_N\right)\leq{}\\
{}\leq I\left(\Psi_1^{*(+)}, \Psi_2^{*(+)},\ldots, \Psi_N^{*(+)}\right)={}\\
{}=
\fr{A\left(u^{(+)}_1,u^{(+)}_2,\ldots,u^{(+)}_N\right)}{B\left(u^{(+)}_1,u^{(+)}_2,
\ldots,u^{(+)}_N\right)}\,.
 \label{e46}
\end{multline}
Из соотношения~(\ref{e46}) с~учетом~(\ref{e44}) получаем, что максимум 
функционала $I(\Psi_1, \Psi_2,\ldots, \Psi_N)$ на множестве допустимых стратегий 
$\Psi\hm=(\Psi_1, \Psi_2,\ldots, \Psi_N)\hm\in \Gamma$ существует и~достигается 
на детерминированной стратегии $\left(\Psi_1^{*(+)}, \Psi_2^{*(+)},\ldots, 
\Psi_N^{*(+)}\right)$.

Кроме того, выполняются соотношения~(\ref{e19}). Таким образом, утверждение~1 
в~случае, когда основная функция $C(u_1, u_2,\ldots, u_N)$ достигает глобального 
максимума, доказано. Соответствующее утверждение в~случае, когда основная функция 
$C(u_1, u_2,\ldots, u_N)$ достигает глобального минимума, доказывается аналогично. 
При этом используется второе утверждение леммы~1.

\smallskip

Перейдем к~доказательству второго утверждения теоремы~2. Предположим, что основная 
функция $C(u_1, u_2,\ldots, u_N)\hm=A(u_1, u_2,\ldots$\linebreak
$\ldots, u_N)/{B(u_1, u_2,\ldots, u_N)}$ 
ограничена сверху, но не достигает глобального максимума на множестве 
$U \hm= U_1 \times U_2 \times \cdots \times U_N$. Тогда множество значений 
основной функции имеет конечную верхнюю грань:

\noindent
\begin{multline*}
C\left(u_1, u_2,\ldots, u_N\right)=\fr{A(u_1, u_2,\ldots, u_N)}
{B(u_1, u_2,\ldots, u_N)}<{}\\
{}<
\sup\limits_{(u_1, u_2,\ldots, u_N)\in U} \fr{A(u_1, u_2,\ldots, u_N)}
{B(u_1, u_2,\ldots, u_N)}<\infty\,, 
\\
\left(u_1, u_2,\ldots, u_N\right)\in U\,.
%\label{e47}
\end{multline*}
По определению верхней грани для любого фиксированного $\varepsilon \hm>0$ 
существует точка $(u_1^{(+)}(\varepsilon), u_2^{(+)}(\varepsilon),\ldots, 
u_N^{(+)}(\varepsilon))$ такая, что выполняется двойное неравенство~(\ref{e21}) 
(см.~\cite[гл.~1, \S\,3, п.~3.4]{25}). Иначе говоря, значение основной функции 
в~указанной точке лежит в~левой \mbox{$\varepsilon$-окрест}\-ности верхней грани. 
Рассмотрим детерминированную стратегию управления 
$\Psi^{*(+)}(\varepsilon)\hm=\!\left(\Psi_1^{*(+)}(\varepsilon), 
\Psi_2^{*(+)}(\varepsilon),\ldots, \Psi_N^{*(+)}(\varepsilon)\!\right)$, компонентами\linebreak 
которой являются вырожденные вероятностные меры $\Psi_1^{*(+)}(\varepsilon), 
\Psi_2^{*(+)}(\varepsilon),\ldots, \Psi_N^{*(+)}(\varepsilon)$, причем вырожденная 
мера~$\Psi_i^{*(+)}(\varepsilon)$ сосредоточена в~точке~$u_i^{(+)}(\varepsilon)$,
$i\hm=1,2,\ldots,N$.

По свойству интеграла
\begin{multline}
I\left(\Psi_1^{*(+)}(\varepsilon), \Psi_2^{*(+)}(\varepsilon),\ldots,
 \Psi_N^{*(+)}(\varepsilon)\right)={}\\
 {}=
 C\left(u_1^{(+)}(\varepsilon), u_2^{(+)}(\varepsilon),\ldots, 
 u_N^{(+)}(\varepsilon)\right)\,. 
 \label{e48}
\end{multline}
Из соотношения~(\ref{e48}) с~учетом указанного свойства основной функции получаем:
\begin{multline}
\sup\limits_{(u_1, u_2,\ldots, u_N)\in U} \fr{A(u_1, u_2,\ldots, u_N)}
{B(u_1, u_2,\ldots, u_N)}-\varepsilon<{}\\
{}< I\left(\Psi_1^{*(+)}(\varepsilon), 
\Psi_2^{*(+)}(\varepsilon),\ldots, \Psi_N^{*(+)}(\varepsilon)\right)<{}
\\
{}< \sup\limits_{(u_1, u_2,\ldots, u_N)\in U} \fr{A(u_1, u_2,\ldots, u_N)}
{B(u_1, u_2,\ldots, u_N)}<\infty\,. 
\label{e49}
\end{multline}
Заметим также, что в~рассматриваемом случае выполнены условия леммы~2. 
Воспользуемся первым утверждением этой леммы, а~именно соотношением~(\ref{e36}):
\begin{multline}
I(\Psi_1, \Psi_2,\ldots, \Psi_N)< {}\\
{}<\sup\limits_{(u_1, u_2,\ldots, u_N)
\in U} \fr{A(u_1, u_2,\ldots, u_N)}{B(u_1, u_2,\ldots, u_N)}<\infty 
\label{e50}
\end{multline}
для всех $(\Psi_1, \Psi_2,\ldots, \Psi_N)\in\Gamma$.

Из соотношений~(\ref{e49}) и~(\ref{e50}) следует, что детерминированная стратегия 
$\Psi^{*(+)}(\varepsilon)\hm=\left(\Psi_1^{*(+)}(\varepsilon), \Psi_2^{*(+)}(\varepsilon),
\ldots, \Psi_N^{*(+)}(\varepsilon)\right)$, опре\-де\-ля\-емая набором вырожденных 
вероятностных мер, сосредоточенных в~соответствующих точках 
$\left(u_1^{(+)}(\varepsilon), u_2^{(+)}(\varepsilon),\ldots, 
u_N^{(+)}(\varepsilon)\right)$, является $\varepsilon$-оп\-ти\-маль\-ной. 
Вторая часть утверждения~2 теоремы~2, связанная со свойствами нижней грани, 
доказывается аналогично.

Докажем третье утверждение теоремы~2. Предположим, что множество значений 
основной функции $C(u_1, u_2,\ldots, u_N)\hm=
A(u_1, u_2,\ldots$\linebreak $\ldots, u_N)/{B(u_1, u_2,\ldots, u_N)}$
не является ограниченным сверху на множестве $U\hm=U_1\times U_2 \times \cdots $\linebreak
$\cdots \times U_N$.
Тогда существует последовательность\linebreak точек $\left(u_1^{(+)}(n), u_2^{(+)}(n),
\ldots,u_N^{(+)}(n)\right)\hm\in U$, $n\hm=1,2,\ldots$, для которой
\begin{multline}
C\left(u_1^{(+)}(n), u_2^{(+)}(n),\ldots,u_N^{(+)}(n)\right)={}\\
{}=
\fr{A\left(u_1^{(+)}(n), u_2^{(+)}(n),\ldots,u_N^{(+)}(n)\right)}
{B\left(u_1^{(+)}(n), u_2^{(+)}(n),\ldots,u_N^{(+)}(n)\right)}
\longrightarrow \infty \,,\\
n\rightarrow \infty\,.
\label{e51}
\end{multline}
Зафиксируем некоторую последовательность точек $\left(u_1^{(+)}(n), u_2^{(+)}(n),
\ldots,u_N^{(+)}(n)\right)\hm\in U$, $n\hm=1,2,\ldots$, обладающих указанным свойством, 
и~рассмотрим последовательность детерминированных  стратегий управления 
$\Psi^{*(+)}(n)\hm=\left(\Psi_1^{*(+)}(n), \Psi_2^{*(+)}(n),\ldots, 
\Psi_N^{*(+)}(n)\right)$, $n\hm=1,2,\ldots$, определяемых набором вырожденных 
вероятностных мер, сосредоточенных в~соответствующих точках 
$\left(u_1^{(+)}(n), u_2^{(+)}(n),\ldots,u_N^{(+)}(n)\right)$, $n\hm=1,2,\ldots$ 
По свойству интеграла для любого фиксированного значения $n=1,2,\ldots$ 
выполняется равенство:
\begin{multline}
I \left(\Psi^{*(+)}(n)\right)={}\\
{}=I\left(\Psi_1^{*(+)}(n), \Psi_2^{*(+)}(n),\ldots,
 \Psi_N^{*(+)}(n)\right)={}\\
{}=\fr{A\left(u_1^{(+)}(n), u_2^{(+)}(n),\ldots,u_N^{(+)}(n)\right)}
{B\left(u_1^{(+)}(n), u_2^{(+)}(n),\ldots,u_N^{(+)}(n)\right)}\,. 
\label{e52}
\end{multline}
Из соотношений~(\ref{e51}) и~(\ref{e52}) следует, что
\begin{multline}
I\left(\Psi^{*(+)}(n)\right)={}\\
{}=I\left(\Psi_1^{*(+)}(n), \Psi_2^{*(+)}(n),\ldots, 
\Psi_N^{*(+)}(n)\right)\longrightarrow\infty\,,\\ 
n \rightarrow\infty\,.
 \label{e53}
\end{multline}
Соотношение~(\ref{e53}) означает, что множество значе\-ний дроб\-но-ли\-ней\-но\-го 
интегрального функциона\-ла $I(\Psi_1, \Psi_2,\ldots, \Psi_N)$ вида~(\ref{e11}) 
не ограничено сверху\linebreak на множестве наборов вырожденных вероятностных мер 
$\left(\Psi_1^{*(+)}(n), \Psi_2^{*(+)}(n),\ldots, \Psi_N^{*(+)}(n)\right)\hm\in\Gamma^*$, 
а~следовательно, и~на более широком\linebreak множестве наборов вероятностных 
мер $(\Psi_1, \Psi_2,\ldots$\linebreak $\ldots, \Psi_N)\hm\in\Gamma$. В~таком случае решения экстремальной 
задачи~(\ref{e18}) в~форме задачи на максимум не существует. Соответствующее утвержде\-ние 
для варианта, когда множество значений основной функции $C(u_1, u_2,\ldots,u_N)
\hm=A(u_1, u_2,\ldots$\linebreak $\ldots,u_N)/{B(u_1, u_2,\ldots,u_N)}$ 
не является ограниченным снизу, доказывается аналогично. Третье утверж\-де\-ние теоремы~2 
доказано. Тем самым тео\-ре\-ма~2 доказана полностью.

\smallskip

Применим теорему~2 для решения поставленной задачи оптимального управления. 
Из утверждения этой теоремы следует, что для доказательства су-\linebreak ществования 
оптимального управ\-ле\-ния и~его нахождения необходимо исследовать на 
глобальный экстремум основную функцию дроб\-но-ли\-ней\-но\-го интегрального 
функционала $C(u_1,u_2,\ldots,u_N)$, определяемую формулой~(\ref{e17}) с~учетом 
равенств~(\ref{e12})--(\ref{e16}). В~некоторых случаях, например когда основной 
процесс~$\xi(t)$ является регенерирующим, а~стоимостные характеристики 
модели задаются линейными функциями, такое исследование можно провести 
аналитически. Однако для подавляющего большинства полумарковских моделей 
для этого необходимо использовать численные методы.

\section{Заключение}

В заключительной части работы приведем \mbox{краткое} описание теоретической 
основы метода решения задачи оптимального управления полумарковским 
процессом с~конечным множеством состояний.

\begin{enumerate}[1.]
\item Исходная проблема оптимального управления формулируется в~виде 
экстремальной задачи~(\ref{e18}). Целевым показателем качества управ\-ле\-ния в~данной задаче 
служит величина~(\ref{e10}), которая имеет характер средней удельной прибыли.
\item Доказывается, что стационарный показатель~(\ref{e10}) представим в~виде 
дроб\-но-ли\-ней\-но\-го интегрального функционала~(\ref{e11}), для которого явно 
определяются подынтегральные функции числителя и~знаменателя, а~следовательно, 
и~основная функция данного функционала.
\item Используется теорема об экстремуме дроб\-но-ли\-ней\-но\-го интегрального 
функционала. На основании утверждений этой теоремы уста\-нав\-ли\-ва\-ет\-ся, что 
исходная задача оптимального управления сводится к~исследованию на глобальный 
экстремум основной функции этого функционала, для которой получено явное 
аналитическое представление.
\end{enumerate}

Заметим, что такое исследование задач оптимального управления 
стохастическими системами фактически уже было проведено в~ряде работ П.\,В.~Шнуркова 
и~его соавторов. В~частности, в~работе~\cite{26} была рассмотрена модель 
управления для обрывающегося процесса восстановления, описывающего функционирование 
некоторой технической системы. Задача управления решалась для различных показателей 
эффективности и~надежности этой системы, имеющих структуру дроб\-но-ли\-ней\-но\-го 
интегрального функционала.

В работах~\cite{27, 28} рассматривались модели регенерирующих процессов 
для исследования сис\-тем управления запасами. Различные показатели качества 
управления были представлены в~форме дроб\-но-ли\-ней\-ных интегральных функционалов. 
Основные функции этих функционалов были найде\-ны в~явной форме и~исследовались 
на глобальный экстремум. В~работах~\cite{21,29} рассматривалась достаточно 
сложная полумарковская модель с~конечным множеством состояний, описывающая 
сис\-те\-му управления запасом непрерывного продукта. Показатели качества управления в~этой 
модели также имели структуру дроб\-но-ли\-ней\-ных интегральных функционалов, 
для основных функций которых были найдены явные аналитические представления. 
Упомянем также работы~\cite{30, 31}, в~которых была исследована полумарковская 
модель с~дис\-крет\-но-не\-пре\-рыв\-ным фазовым пространством. Показатели 
качества управления в~этой  модели были найдены в~явной форме как функции от 
двух непрерывных параметров управления.

Фактически во всех упомянутых работах уже был использован метод решения задачи 
оптимального управления регенерирующим или полумарковским случайным процессом, 
основанный на исследовании экстремальных свойств основной функции соответствующего 
дроб\-но-ли\-ней\-но\-го интегрального функционала. Из соображений, изложенных 
во\linebreak введении, следует, что в~период написания и~пуб\-ли\-кации этих работ данный метод 
не имел стро\-гого обоснования. Однако после публикации\linebreak работы~\cite{14} и~настоящего 
исследования можно утверж\-дать, что полученные в~них результаты полностью теоретически 
обоснованы.

Таким образом, изложенный выше метод решения проблемы оптимального управления 
полумарковскими процессами с~конечными множествами состояний может быть успешно 
реализован для многих задач, рассматриваемых в~различных областях прикладной 
теории вероятностей.

Практическая реализация численной процедуры поиска оптимального решения на примере\linebreak 
полумарковской модели управления запасом непрерывного продукта (подробнее 
см.~\cite{21, 29}), ба\-зи\-ру\-юща\-яся на изложенных выше результатах (в~частности, 
теореме~1), была осуществлена А.\,К.~Горшениным и~соавторами 
в~статье~\cite{Gorshenin2015}. Коротко опишем наиболее важные аспекты этой работы.

Для решения поставленной задачи опти\-мального управления была создана 
специальная программа \verb"Inventory" на встроенном языке программирования 
пакета \verb"MATLAB", ее возможности\linebreak кратко представле\-ны в~упомянутой ранее 
\mbox{статье}~\cite{Gorshenin2015}. В~программе \verb"Inventory" реализованы функции 
для оценивания через заданные исходные параметры вероятностных и~стоимостных 
характеристик модели, которые в~дальнейшем используются для поиска значений 
основной функции дроб\-но-ли\-ней\-но\-го функционала~(\ref{e17}). Точка глобального 
экстремума этой функции и~определяет оптимальное управление.

В качестве начальных данных необходимо задание следующих параметров:
\begin{itemize}
\item спрос и~вместимость склада;
\item разбиение множества значений объема запаса;
\item вероятностные характеристики, описывающие модель пополнения запаса;
\item условные математические ожидания длительностей задержек пополнения запаса;
\item функции для характеризации затрат и~доходов.
\end{itemize}

По итогам работы программы \verb"Inventory" ряд вспомогательных функций 
представляется в~аналитической форме (в частности, с~использованием аппарата 
символьных вычислений  \verb"Symbolic Toolbox"\linebreak пакета \verb"MATLAB"), выводится 
точка глобального экстремума функции нескольких вещественных переменных~(\ref{e17}), 
найденная с~помощью применения численных и~при\-бли\-жен\-но-ана\-ли\-ти\-че\-ских\linebreak 
аппроксимаций. 
Также формируются графики оценок значений ве\-ро\-ят\-ност\-но-сто\-и\-мост\-ных 
характеристик 
и~основной функции дроб\-но-ли\-ней\-но\-го функционала~(\ref{e17}), либо трехмерных 
сечений в~случае наличия более трех параметров управления (переменных).

Функциональность пакета \verb"Inventory" может быть расширена для практической 
реализации метода решения задачи поиска оптимального управ\-ле\-ния полумарковскими 
процессами с~конечными множествами состояний, рассмотренного в~данной статье.


 {\small\frenchspacing
 {%\baselineskip=10.8pt
 \addcontentsline{toc}{section}{References}
 \begin{thebibliography}{99}
 \bibitem{1}
\Au{Ховард Р.} Динамическое программирование и~марковские процессы~/ 
Пер. с~англ.~--- М.: Сов. радио, 1964. 189~с.
(\Au{Howard~R.\,A.} Dynamic programming and Markov processes.~--- 
Cambridge, MA, USA: MIT Press, 1960. 136~p.)
\bibitem{2} 
\Au{Рыков В.\,В.} Управляемые марковские процессы с~конечными пространствами 
состояний и~управлений~// Теория вероятностей и~ее применения, 1966. Т.~11. 
Вып.~2. С.~343--351.
\bibitem{3} 
\Au{Джевелл В.} Управляемые полумарковские процессы~// Кибернетич. сборник.~--- 
М.: Мир, 1967. Вып.~4. С.~97--134.
%{\em Jewell W.\,S.} Markov-renewal programming~// Operations Research, 1963. Vol.~11. P.~938--971.
\bibitem{4} 
\Au{Fox B.} Markov renewal programming by linear fractional programming~// 
SIAM J.~Appl. Math., 1966. Vol.~14. P.~1418--1432.
\bibitem{5} 
\Au{Denardo E.\,V.} Contraction mappings in the theory underlying dinamic programming~// 
SIAM Rev., 1967. Vol.~9. P.~165--177.

\bibitem{6} 
\Au{Howard R.\,A.} Research in semi-Markovian decision structures~// 
J.~Oper. Res. Soc. Japan, 1963. Vol.~6. P.~163--199.
\bibitem{7} 
\Au{Osaki S., Mine H.} Linear programming algorithms for Markovian decision processes~//
 J.~Math. Anal. Appl., 1968. Vol.~22. P.~356--381.
\bibitem{8} 
\Au{Майн Х., Осаки С.} Марковские процессы принятия решений~/ Пер. с~англ.~--- 
М.: Наука, 1977. 176~с.
(\Au{Mine~H., Osaki~S.} 
Markovian decision processes.~--- New York, NY, USA: 
American Elsevier Publishing Co., 1970. 142~p.)
\bibitem{9} 
\Au{Гихман И.\,И., Скороход А.\,В.} Управляемые случайные процессы.~--- 
Киев: Наукова думка, 1977. 251~с.
\bibitem{10} 
\Au{Luque-Vasquez F., Herndndez-Lerma~О.} Semi-Markov control models with average costs~// 
Appl. Math., 1999. Vol.~26. No.\,3. P.~315--331.
\bibitem{11} 
\Au{Vega-Amaya O., Luque-Vasquez~F.} Sample-path average cost optimality for 
semi-Markov control processes on Borel spaces: Unbounded costs and mean holding times~// 
Appl. Math., 2000. Vol.~27. No.\,3. P.~343--367.
\bibitem{12} 
Вопросы математической теории надежности~/ Под ред. Б.\,В. Гнеденко.~--- 
М.: Радио и~связь, 1983. 376~с.
\bibitem{13} 
\Au{Барзилович Е.\,Ю., Каштанов~В.\,А.} Некоторые математические вопросы теории 
обслуживания сложных систем.~---  М.: Сов. радио, 1971. 272~с.
\bibitem{14} 
\Au{Шнурков П.\,В.} О~решении проблемы безусловного экстремума для 
дроб\-но-ли\-ней\-но\-го интегрального функционала на множестве вероятностных мер~// 
Докл. РАН. Сер. Математика, 2016. Т.~470. №\,4. C.~387--392.
\bibitem{15} 
\Au{Ширяев А.\,Н.}  Вероятность.~--- М.:~МЦНМО, 2011. Кн.~1. 552~с. Кн.~2. 968~с.
\bibitem{16} 
\Au{Боровков А.\,А.} Теория вероятностей.~--- М.: Либроком, 2009. 656~c.
\bibitem{17} 
\Au{Хеннекен П.\,Л., Тортра А.} Теория вероятностей 
и~некоторые ее приложения.~--- М.: Наука, 1974. 472~c.
\bibitem{18} 
\Au{Халмош П.} Теория меры~/ Пер. с~англ.~--- М.: ИЛ, 1953. 282~c.
(\Au{Halmos~P.} Measure theory.~--- Litton Educational Publishing, Inc. 1950. 304~p.)
\bibitem{19} 
\Au{Королюк В.\,С., Турбин~А.\,Ф.} Полумарковские процессы и~их приложения.~--- 
Киев:~Наукова думка, 1976. 184~с.
\bibitem{20} 
\Au{Janssen J., Manca R.} Applied semi-Markov processes.~--- New York,
NY, USA: Springer, 2006. 309~p.
\bibitem{21} 
\Au{Шнурков П.\,В., Иванов~А.\,В.} Анализ дискретной полумарковской модели
 управления запасом непрерывного продукта при периодическом прекращении потребления~// 
 Дискретная математика, 2014. Т.~26. Вып.~1. С.~143--154.
\bibitem{22} 
\Au{Иванов~А.\,В.} Анализ дискретной полумарковской модели
 управления запасом непрерывного продукта при периодическом прекращении 
 потребления.~--- М.: НИУ ВШЭ, 2014.  Дисс.\ \ldots\ канд. физ.-мат. наук. 120~с.
\bibitem{23}  %23
\Au{Bajalinov~E.\,B.} Linear-fractional programming. 
Theory, methods, applications and software.~--- 
Boston/\linebreak Dordrecht/London: Kluwer Academic Publs., 2003. 423~p.

\bibitem{27} %27
\Au{Шнурков П.\,В., Мельников~Р.\,В.} Оптимальное управление запасом 
непрерывного продукта в~модели регенерации~// Обозрение прикладной 
и~промышленной математики, 2006. Т.~13. Вып.~3. С.~434--452.
\bibitem{28} 
\Au{Шнурков П.\,В., Мельников~Р.\,В.} 
Исследование проб\-ле\-мы управления запасом непрерывного продукта при детерминированной 
задержке поставки~// Автоматика и~телемеханика, 2008. Т.~10. С.~93--113.


\bibitem{24}  %26
\Au{Шнурков П.\,В.} Методы исследования задач оптимального обслуживания 
в~математической теории надежности.~--- 
М.: МИЭМ, 1983.  Дисс.\ \ldots\ канд. физ.-мат. наук.

 \bibitem{25}  %25
\Au{Кудрявцев Л.\,Д.} Курс математического анализа. Т.~1.~--- 
М.: Дрофа, 2006. 704~с.

\bibitem{26} %24
\Au{Шнурков П.\,В.} Оптимальное обслуживание на периоде 
до первого отказа системы~// Применение аналитических методов в~вероятностных
 задачах.~--- Киев: Институт математики АН УССР, 1986. С.~121--129.

\bibitem{29} 
\Au{Шнурков П.\,В., Иванов~А.\,В.} Исследование задачи оптимизации в~дискретной 
полумарковской модели управления непрерывным запасом~// Вестник МГТУ им.\ 
Н.\,Э. Баумана. Сер.\ Естественные науки, 2013. Т.~3. Вып.~50. С.~62--87.
\bibitem{30} 
\Au{Shnourkoff P.\,V.} The two-element system with one 
restoring device optimum maintenance~// Stoch. Anal. Appl., 1997. 
Vol.~15. No.\,5. P.~823--837.
\bibitem{31} 
\Au{Shnourkoff P.\,V.} The two-element system optimum maintenance tills the first fail~// 
Stoch. Anal. Appl., 2001. Vol.~19. No.\,6. P.~1005--1024.
\bibitem{Gorshenin2015} 
\Au{Gorshenin~A.\,K., Belousov~V.\,V., Shnourkoff~P.\,V.,
Ivanov~A.\,V.} Numerical research of the optimal control problem in the semi-Markov 
inventory model~// AIP Conference Proceedings, 2015. Vol.~1648. {250007}. 4~p.
%\bibitem{33} {\em Горшенин А.\,К., Белоусов В.\,В., Шнурков П.\,В.} 2016. Система управления запасами на основе стохастических полумарковских моделей. Свидетельство о государственной регистрации программы для ЭВМ \textnumero 2016619021.
 \end{thebibliography}

 }
 }

\end{multicols}

\vspace*{-6pt}

\hfill{\small\textit{Поступила в~редакцию 15.07.16}}

%\vspace*{8pt}

\newpage

\vspace*{-24pt}

%\hrule

%\vspace*{2pt}

%\hrule

%\vspace*{8pt}


\def\tit{ANALYTICAL SOLUTION OF~THE~OPTIMAL CONTROL TASK OF~A~SEMI-MARKOV 
PROCESS WITH~FINITE SET OF~STATES}

\def\titkol{Analytical solution of~the~optimal control task of~a~semi-Markov 
process with~finite set of~states}

\def\aut{P.\,V.~Shnurkov$^{1}$, A.\,K.~Gorshenin$^{2}$, and~V.\,V.~Belousov$^{2}$}

\def\autkol{P.\,V.~Shnurkov, A.\,K.~Gorshenin, and~V.\,V.~Belousov}

\titel{\tit}{\aut}{\autkol}{\titkol}

\vspace*{-9pt}


    
\noindent
$^1$National Research University Higher School of Economics, 34~Tallinskaya Str., 
Moscow, 123458, Russian\linebreak
$\hphantom{^9}$Federation

\noindent
$^2$Institute of Informatics Problems, Federal Research Center 
``Computer Science and Control'' of the Russian\linebreak
$\hphantom{^9}$Academy of Sciences, 44-2~Vavilova Str., 
Moscow 119333, Russian Federation



\def\leftfootline{\small{\textbf{\thepage}
\hfill INFORMATIKA I EE PRIMENENIYA~--- INFORMATICS AND
APPLICATIONS\ \ \ 2016\ \ \ volume~10\ \ \ issue\ 4}
}%
 \def\rightfootline{\small{INFORMATIKA I EE PRIMENENIYA~---
INFORMATICS AND APPLICATIONS\ \ \ 2016\ \ \ volume~10\ \ \ issue\ 4
\hfill \textbf{\thepage}}}

\vspace*{3pt}


\Abste{The theoretical verification of the new method of finding 
the optimal strategy of control of a~semi-Markov process with finite set of states is 
presented. The paper considers Markov randomized strategies of control, determined by 
a~finite collection of probability measures, corresponding to each state. The quality 
characteristic is the stationary cost index. This index is a~linear-fractional integral 
functional, depending on collection of probability measures, giving the strategy of control. 
Explicit analytical forms of integrands of numerator and denominator of this 
linear-fractional integral functional are known. The basis of consequent results is 
the new generalized and strengthened form of the theorem about an extremum of 
a~linear-fractional integral functional. It is proved that problems of existence 
of an optimal control strategy of a~semi-Markov process and finding this strategy 
can be reduced to the task of numerical analysis of global extremum for 
the given function, depending on finite number of real arguments.}

\KWE{optimal control of a~semi-Markov process; stationary cost index of quality control; 
linear-fractional integral functional}




\DOI{10.14357/19922264160408} 

\vspace*{-16pt}

\Ack
\noindent
The research was partially supported by the Russian Foundation 
for Basic Research (project 15-07-05316).



%\vspace*{3pt}

  \begin{multicols}{2}

\renewcommand{\bibname}{\protect\rmfamily References}
%\renewcommand{\bibname}{\large\protect\rm References}

{\small\frenchspacing
 {%\baselineskip=10.8pt
 \addcontentsline{toc}{section}{References}
 \begin{thebibliography}{99}
\bibitem{1-1}
\Aue{Howard,~R.\,A.} 1960. \textit{Dynamic programming and Markov processes}. 
Cambridge, MA: MIT Press. 136~p.
\bibitem{2-1}
\Aue{Rykov,~V.\,V.} 1966. Upravlyaemye markovskie protsessy 
s~konechnymi prostranstvami sostoyaniy i~upravleniy 
[Controlled Markov processes with finite spaces of states and controls ]. 
\textit{Teoriya veroyatnostey i~ee primeneniya} 
[Theory of Probability and Its Applications] 11(2):343--351.
\bibitem{3-1}
\Aue{Jewell,~W.\,S.} 1963. Markov-renewal programming. 
\textit{Oper. Res.} 11:938--971.
\bibitem{4-1}
\Aue{Fox,~B.} 1966. Markov renewal programming by linear fractional programming. 
\textit{SIAM J.~Appl. Math.} 14:1418--1432.
\bibitem{5-1}
\Aue{Denardo, E.\,V.} 1967. Contraction mappings in the theory underlying dinamic 
programming. \textit{SIAM Rev.} 9:165--177.
\bibitem{6-1}
\Aue{Howard,~R.\,A.} 1963. Research in semi-Markovian decision structures. 
\textit{J.~Oper. Res. Soc. Japan} 6:163--199.
\bibitem{7-1}
\Aue{Osaki,~S., and H.~Mine.} 1968. Linear programming algorithms 
for Markovian decision processes. \textit{J.~Math. Anal. Appl.} 22:356--381.
\bibitem{8-1}
\Aue{Mine,~H., and S.~Osaki.} 1970. 
\textit{Markovian decision processes}. New York, NY: Elsevier. 142~p.
\bibitem{9-1}
\Aue{Gikhman,~I.\,I., and A.\,V.~Skorokhod.} 1977. 
\textit{Upravlyaemye sluchaynye protsessy} 
[Controlled random processes]. Kiev: Naukova Dumka. 251~p.
\bibitem{10-1}
\Aue{Luque-Vasquez,~F., and О.~Herndndez-Lerma.} 1999. 
Semi-Markov control models with average costs. \textit{Appl. Math.} 26(3):315--331.
\bibitem{11-1}
\Aue{Vega-Amaya,~O., and  F.~Luque-Vasquez.} 2000.  
Sample-path average cost optimality for semi-Markov control processes on Borel spaces: 
Unbounded costs and mean holding times. \textit{Appl. Math.} 27(3):343--367.
\bibitem{12-1}
Gnedenko,~B.~V., ed. 1983. 
\textit{Voprosy matematicheskoy teorii nadezhnosti} 
[Problems of the mathematical theory of reliability].  Moscow: Radio i~svyaz'. 376~p.
\bibitem{13-1}
\Aue{Barzilovich,~E.\,Yu., and V.\,A.~Kashtanov.} 1971. 
\textit{Nekotorye matematicheskie voprosy teorii obsluzhivaniya slozhnykh sistem}  
[Some mathematical questions in theory of complex systems maintenance]. 
Moscow: Sovetskoe radio. 272~p.
\bibitem{14-1}
\Aue{Shnurkov,~P.\,V.} 2016. Solution of the unconditional extremum problem for 
a~linear-fractional 
integral functional on a~set of probability measures. 
\textit{Dokl. Math.} 94(2):550--554.
\bibitem{15-1} %14
\Aue{Shiryaev,~A.\,N.} 2016. 
\textit{Probability-1}. Graduate texts in mathematics ser.
New York, NY: Springer. Vol.~95. 503~p.;
2017. \textit{Probability-2.} Vol.~900. 500~p.
\bibitem{16-1}
\Aue{Borovkov,~А.\,А.} 2009. 
\textit{Teoriya veroyatnostey} [Probability theory]. Moscow: Librokom. 656~p.
\bibitem{17-1}
\Aue{Khenneken,~P.\,L., and A.~Tortra.} 1974. 
\textit{Teoriya veroyatnostey i~nekotorye ee prilozheniya} 
[Probability theory and some of its applications]. Moscow: Nauka. 472~p.
\bibitem{18-1}
\Aue{Halmos,~P.} 1950. \textit{Measure theory}. Litton Educational Publishing. 304~p.
\bibitem{19-1}
\Aue{Korolyuk, V.\,S., and A.\,F.~Turbin.} 1976. 
\textit{Polumarkovskie protsessy i~ikh prilozheniya} 
[Semi-Markov processes and their applications]. Kiev: Naukova Dumka. 184~p.
\bibitem{20-1}
\Aue{Janssen,~J., and R.~Manca.} 2006. 
\textit{Applied semi-Markov processes}. New York, NY: Springer. 309~p.
\bibitem{21-1}
\Aue{Shnurkov,~P.\,V, and A.\,V~Ivanov.} 2015. Analysis of a~discrete semi-Markov model of continuous inventory 
control with periodic interruptions of consumption. 
\textit{Discrete Math. \mbox{Appl}.} 25(1):59--67.
\bibitem{22-1} %21
\Aue{Ivanov,~A.\,V.} 2014. Analiz diskretnoy polumarkovskoy modeli upravleniya 
zapasom nepreryvnogo produkta pri periodicheskom prekrashchenii potrebleniya 
[Analysis of a~discrete semi-Markov control model of continuous product inventory 
in a~periodic cessation of consumption].  
Moscow: Natsional'nyy Issledovatel'skiy Universitet ``Vysshaya Shkola Ekonomiki.'' 
PhD Thesis. 120~p.
\bibitem{23-1} %22
\Aue{Bajalinov,~E.\,B.} 2003. 
\textit{Linear-fractional programming. Theory, methods, applications and software}. 
Boston/\linebreak Dordrecht/London: Kluwer Academic Publs. 423~p.
\bibitem{26-1} %24
\Aue{Shnurkov,~P.\,V., and R.\,V.~Mel'nikov.} 2006. Optimal'noe upravlenie 
zapasom nepreryvnogo produkta v modeli regeneratsii [Optimal control of 
a~continuous product inventory in the regeneration model]. 
\textit{Obozrenie prikladnoy i~promyshlennoy matematiki} [Rev. Appl. Ind. Math.]
13(3):434--452.

\bibitem{25-1} %25
\Aue{Shnurkov,~P.\,V., and R.\,V.~Mel'nikov.} 2008. 
Analysis of the problem of continuous-product inventory control under deterministic 
lead time. \textit{Automat. Rem. Contr.} 69(10):1734--1751.

\columnbreak

\bibitem{24-1} %26
\Aue{Shnurkov,~P.\,V.} 1983. Metody issledovaniya zadach optimal'nogo obsluzhivaniya 
v~matematicheskoy teorii nadezhnosti [Research methods of optimal service problems 
in the mathematical theory of reliability].  
Moscow: Moskovskiy Institut Elektronnogo Mashinostroeniya.  PhD Thesis. 


\bibitem{27-1} %27
\Aue{Kudryavtsev,~L.\,D.} 2006. 
\textit{Kurs matematicheskogo analiza} 
[A~course of mathematical analysis]. Vol.~1. Moscow: Drofa. 704~p.

\bibitem{28-1}
\Aue{Shnurkov,~P.\,V.} 1986. Optimal'noe obsluzhivanie na periode do 
pervogo otkaza sistemy [The optimum service period until the first system failure]. 
\textit{Primenenie analiticheskikh metodov v~veroyatnostnykh zadachakh} 
[The application of analytical methods in probabilistic tasks]. Kiev:
Institute of Mathematics of the Academy of Sciences of the USSR. 121--129.

\bibitem{29-1}
\Aue{Shnurkov,~P.\,V., and A.\,V.~Ivanov.} 2013. Issledovanie zadachi optimizatsii 
v~diskretnoy polumarkovskoy modeli upravleniya nepreryvnym zapasom 
[Study of the optimization problem in discrete semi-Markov model of continuous 
inventory control]. \textit{Vestnik MGTU im.\ N.\,E.~Baumana. Ser. 
Estestvennye nauki} [Vestnik of MSTU named after N.\,E.~Bauman. Ser. Natural sciences] 
3(50):62--87.
\bibitem{30-1}
\Aue{Shnourkoff,~P.\,V.} 1997. The two-element system with one restoring device 
optimum maintenance.  \textit{Stoch. Anal. Appl.} 15(5):823--837.
\bibitem{31-1}
\Aue{Shnourkoff,~P.\,V.} 2001. The two-element system optimum maintenance tills 
the first fail. \textit{Stoch. Anal. Appl.} 19(6):1005--1024.
\bibitem{32-1}
\Aue{Gorshenin,~A.\,K., V.\,V.~Belousov, P.\,V.~Shnourkoff, and A.\,V.~Ivanov.}
2015. Numerical research of the optimal control problem in the semi-Markov 
inventory model. \textit{AIP Conference Proceedings} 1648:250007.
\end{thebibliography}

 }
 }

\end{multicols}

\vspace*{-3pt}

\hfill{\small\textit{Received July 15, 2016}}

\Contr

\noindent
\textbf{Shnurkov Peter V.} (b.\ 1953)~---
 Candidate of Science (PhD) in physics and mathematics, 
 associate professor, National Research University Higher School of Economics, 
 34~Tallinskaya Str., Moscow 123458, Russian Federation; \mbox{pshnurkov@hse.ru} 
 
 \vspace*{3pt}
 
 \noindent
\textbf{Gorshenin Andrey K.}  (b.\ 1986)~---
Candidate of Science (PhD) in physics and mathematics, leading scientist, 
Institute of Informatics Problems, Federal Research Center ``Computer Science 
and Control'' of the Russian Academy of Sciences, 44-2~Vavilov Str., Moscow 119333, 
Russian Federation; associate professor, Federal State Budget Educational 
Institution of Higher Education ``Moscow Technological University,'' 
78~Vernadskogo Ave., Moscow 119454, Russian Federation;
\mbox{agorshenin@frccsc.ru}

\vspace*{3pt}

\noindent
\textbf{Belousov Vasiliy V.} (b.\ 1977)~---
Candidate of Science (PhD) in technology, senior scientist, Institute of 
Informatics Problems, Federal Research Center ``Computer Science and Control'' 
of the Russian Academy of Sciences, 44-2~Vavilov Str., Moscow 119333, Russian 
Federation; \mbox{VBelousov@ipiran.ru}
\label{end\stat}


\renewcommand{\bibname}{\protect\rm Литература}   %8
\def\stat{zaharova}

\def\tit{ОЦЕНКА УРОВНЯ ЗНАЧИМОСТИ КРИТЕРИЯ ШУИРМАННА ДЛЯ~ПРОВЕРКИ ГИПОТЕЗЫ 
БИОЭКВИВАЛЕНТНОСТИ ПРИ~НАЛИЧИИ ПРОПУЩЕННЫХ ДАННЫХ$^*$}

\def\titkol{Оценка уровня значимости критерия Шуирманна для~проверки гипотезы 
биоэквивалентности} % при~наличии пропущенных данных}

\def\aut{Т.\,В.~Захарова$^1$,  А.\,А.~Тархов$^2$}

\def\autkol{Т.\,В.~Захарова,  А.\,А.~Тархов}

\titel{\tit}{\aut}{\autkol}{\titkol}

\index{Захарова Т.\,В.}
\index{Тархов А.\,А.}
\index{Zakharova T.\,V.}
\index{Tarkhov A.\,A.}


{\renewcommand{\thefootnote}{\fnsymbol{footnote}} \footnotetext[1]
{Работа выполнена при поддержке РФФИ (проект 18-07-00252).}}


\renewcommand{\thefootnote}{\arabic{footnote}}
\footnotetext[1]{Московский государственный университет им.\ М.\,В.~Ломоносова,
факультет вычислительной математики и~кибернетики; Институт 
проблем информатики Федерального исследовательского центра <<Информатика и~управление>> 
Российской академии наук, \mbox{tvzaharova@mail.ru}}
\footnotetext[2]{Московский государственный университет им.\ М.\,В.~Ломоносова, 
факультет вычислительной математики и~кибернетики, \mbox{alexeytarkhov@gmail.com}}

%\vspace*{-2pt}



\Abst{Задача проверки гипотезы биоэквивалентности имеет важное 
значение в~фармакокинетике. С~ее помощью принимают решение об 
эквивалентности воспроизведенного лекарственного препарата референтному 
лекарственному препарату. Одна из проблем исследований биоэквивалентности~--- 
наличие пропущенных данных. Так как объем исследуемых данных достаточно мал,
 то удаление данных субъекта, у~которого есть пропущенные данные, нежелательно. 
 Поэтому стоит задача оценить влияние пропущенных данных при принятии решения
  о~биоэквивалентности, а~именно: дать оценку уровня значимости.
Основным методом проверки гипотезы биоэквивалентности является 
процедура двух односторонних тес\-тов Шуирманна. В~статье дана оценка уровня 
значимости данной процедуры при наличии пропущенных данных. 
В~явном виде получена компонента оценки уровня значимости, зависящая от уровня 
полноты данных.}

\KW{биоэквивалентность; уровень значимости; ошибка первого рода;
 пропущенные данные;  процедура двух односторонних тестов Шуирманна}


\DOI{10.14357/19922264190309} 
  
\vspace*{3pt}


\vskip 10pt plus 9pt minus 6pt

\thispagestyle{headings}

\begin{multicols}{2}

\label{st\stat}

\section{Введение}

\vspace*{-3pt}

Предположим, что имеется лекарственный препа\-рат, для которого 
был проведен набор  широкомасштабных клинических исследований, 
доказавших его безопасность и~медицинскую эф\-фективность. Данный 
препарат будем называть\linebreak референтным лекарственным препаратом. 
На основе действующих веществ референтного лекарственного препарата 
могут быть созданы новые лекар\-ст\-вен\-ные препараты с~таким же 
количественным и~качественным составом. Далее будем называть 
их воспроизведенными лекарственными препаратами. Чтобы 
перенести имеющиеся сведения о безопасности и~эффективности 
референтного\linebreak лекарственного препарата на воспроизведенный 
препарат без проведения широкомасштабных исследований, исследуют 
биоэквивалентность лекарственных препаратов.


Понятие биоэквивалентности тесно связано с~понятием биодоступности.

Биодоступность~-- скорость и~степень, с~которыми действующее вещество 
или его активная часть молекулы из дозированной лекарственной формы 
всасываются и~становятся доступными в~месте действия. Два лекарственных 
препарата, содержащих одинаковое количество действующего вещества, 
считаются биоэквивалентными, если они являются фармацевтически 
эквивалентными или фармацевтически альтернативными и~их биодоступность 
(по скорости и~степени) после применения в~одинаковой молярной дозе 
укладывается в~заранее установленные допустимые пределы~\cite{defin}.


В процессе сбора данных некоторые полученные значения могут быть утеряны. 
Так как число испытуемых ограничено, то исключать данные испытуемого,
 для которого было утеряно одно значение концентрации действующего вещества,
  нерационально. Вместо этого пропущенные данные заполняют нулем или другим значением,
   полученным на основе информации о~других значениях. 
   
   При использовании недостаточно 
   точных методов заполнения данных, таких как заполнение нулем, уменьшается часть 
   значений исследуемых данных.
Хотелось бы оценить, как наличие пропущенных данных влияет на проверку 
гипотезы биоэквивалентности.


В данной работе будет рассмотрена процедура двух односторонних тестов Шуирманна, 
которая в~настоящее время используется при проверке гипотезы биоэквивалентности, 
при наличии пропусков в~исследуемых данных.



\section{Задача проверки гипотезы биоэквивалентности}

Пусть $T$~--- воспроизведенный лекарственный препарат, а~$R$ --- 
референтный лекарственный препарат и~соответственно~$\mu_T$ и~$\mu_R$~--- 
математические ожидания сравнительных характеристик для лекарственных препаратов~$T$ и~$R$.

Основными сравнительными характеристиками биоэквивалентности служат максимальная 
концентрация  в~крови~$C_{\max}$ и~площадь под кривой <<кон\-цент\-ра\-ция\,--\,вре\-мя>> 
$\mathrm{AUC}$ (от \textit{англ.}\ Area Under the Curve).

Будем следовать предположению, что рас\-смат\-ри\-ва\-емые сравнительные 
характеристики имеют логнормальное распределение~[2--4].
%\cite{schuirmann, book, article}.
Например, для $\mathrm{AUC}$:
\begin{equation*}
%\label{log-norm}
    \mathrm{AUC}_i \sim \mathrm{Log}\,N\left(a_i, \sigma_i\right),\enskip i \in \{T,R\}.
\end{equation*}

Пусть $\theta_1$ и~$\theta_2$~--- соответственно нижний и~верхний принятый 
допустимый предел признания биоэквивалентности. Следовательно, гипотеза 
о~биоэквивалентности может быть записана следующим образом:
\begin{align*}
%\left.
%\begin{array}{rl}
&    H_0: \fr{\mu_T}{\mu_R} \leqslant q \theta_1\enskip \mbox{или}\enskip  
\fr{\mu_T}{\mu_R} \ge \theta_2; \\
 &   H_A: \theta_1 <\fr{\mu_T}{\mu_R} < \theta_2.  
% \end{array}
% \right\}
 %\label{h-bio}
\end{align*}

Сделав логарифмическое преобразование, можем перейти к~следующей 
постановке рассматриваемой гипотезы:
\begin{align*}
%\left.
%\begin{array}{rl}
    &H_0': \mu'_T - \mu'_R \leqslant q \delta_1\enskip \mbox{или}\enskip  \mu'_T - \mu'_R \ge \delta_2; \\
    &H_A': \delta_1 <\mu'_T - \mu'_R < \delta_2,  
 %   \end{array}
  %  \right\}
   % \label{h-bio-log}
\end{align*}
где $\delta_1 = \ln\theta_1$ и~$\delta_2 \hm= \ln\theta_2$, а~$\mu'_T$ и~$\mu'_R$~--- 
математические ожидания логарифмов сравнительных характеристик 
для лекарственных препаратов~$T$ и~$R$.
Например, для~$\ln{\mathrm{AUC}_T}$ из свойств логнормального распределения следует, 
что $ \mu'_T \hm= a_T$.

Гипотеза $H_0'$ соответствует небиоэквивалент\-ности исследуемых 
лекарственных препаратов, в~то время как~$H_A'$ утверждает, что они 
биоэквивалентны. Выбор такого порядка основной и~альтернативной гипотез 
обусловлен тем, что в~таком случае ошибка первого рода соответствует 
признанию лекарственных средств биоэквивалентными, хотя на самом деле 
они такими и~не является. В~этом случае пациент несет риск получить препарат, 
который может не обладать такими же эффективностью и~безопасностью, как 
референтный лекарственный препарат~\cite{article}.


\section{Процедура двух односторонних тестов Шуирманна}

Разобьем гипотезы $H'_0$ и~$H'_1$ на два множества односторонних гипотез:
\begin{equation*}
\left\{
\begin{array}{rl}
    H_{01}:& \mu'_T - \mu'_R \leqslant q \delta_1;  \\[6pt]
    H_{A1}:& \mu'_T - \mu'_R > \delta_1;  
    \end{array}
    \right.
   % \label{h-bio-log1}
\end{equation*}
\begin{equation*}
\left\{
\begin{array}{rl}
    H_{02}:& \mu'_T - \mu'_R \geqslant \delta_2; \\[6pt]
    H_{A2}:& \mu'_T - \mu'_R < \delta_2.  
    \end{array}
    \right.
   % \label{h-bio-log2}
\end{equation*}
Процедура двух односторонних тестов заключается в~том, что~$H'_0$ 
отвергаем при уровне зна\-чи\-мости~$\alpha$, тем самым устанавливая 
эквивалентность~$\mu_T$ и~$\mu_R$,  только в~том случае, если отвергаются 
обе гипотезы~$H_{01}$ и~$H_{02}$ при заданном уровне зна\-чи\-мости~$\alpha$~\cite{schuirmann, book}.

Таким образом, два односторонних теста проверяются с~использованием односторонних 
t-кри\-те\-ри\-ев, т.\,е.\ 
характеристики биодоступности признаются эквивалентными, если
\begin{equation}
\left.
\begin{array}{rl}
    \hspace*{-2mm}t_1& =  \fr{\bar{Y_T} -\bar{Y_R}-\delta_1}
    {\hat\sigma_d\sqrt{{1}/{n_1} + {1}/{n_2}}} > t\left(\alpha, n_1 + n_2-2\right); \\[6pt]
        \hspace*{-2mm}t_2& =  \fr{\bar{Y_T} - \bar{Y_R}-\delta_2}
    {\hat\sigma_d\sqrt{{1}/{n_1} + {1}/{n_2}}} < -t\left(\alpha, n_1 + n_2-2\right), 
    \end{array}\!
    \right\}\!\!
    \label{t}
\end{equation}
где $n_1$ и~$n_2$~--- число субъектов в~последовательностях клинического 
исследования с~перекрестным\linebreak двухпоследовательным дизайном, 
$t(\alpha, n_1\hm + n_2-2)$~--- $(1\hm - \alpha)$-кван\-тиль центрального 
t-рас\-пре\-де\-ле\-ния с~$n_1\hm+n_2-2$ степенями свободы; $\hat\sigma_d$~--- 
обобщенная выборочная дисперсия разностей между периодами (для обоих 
последовательностей в~исследовании), которая является несмещенной оценкой~$\sigma_d$,\linebreak 
причем
$$
\sigma_d^2 = \fr{\sigma_w^2}{2}\,,
$$
где $\sigma_w$~--- внутрисубъектная вариабельность изуча\-емых параметров~\cite{article}.

Процедура двух односторонних тестов эквивалентна подходу с~построением 
доверительного интервала для разности выборочных средних, т.\,е.\ 
получению следующей интервальной оценки:

\noindent
\begin{multline*}
%\label{interv}
    \left(\bar{Y_T} - \bar{Y_R} + t\left(\alpha, n_1 + n_2-2\right)\hat\sigma_d
    \sqrt{\fr{1}{n_1} + \fr{1}{n_2}};\right.\\
    \left.   \bar{Y_T} - \bar{Y_R} - t\left(\alpha, n_1 + n_2-2\right)
    \hat\sigma_d\sqrt{\fr{1}{n_1} + \fr{1}{n_2}}\right).
\end{multline*}

Признание эквивалентности параметров биодоступности на
 уровне значимости~$\alpha$ может быть сделано, только если 
 полученный доверительный $(1\hm-2\alpha)100\%$-ный интервал для 
 $\mu'_T \hm- \mu'_R$ полностью содержится в~интервале 
 $\left(\delta_1, \delta_2\right)$~\cite{schuirmann}.


Так как рассматриваем сбалансированный дизайн, то $n_1\hm=n_2\hm=n$, 
и,~учитывая~(\ref{t}), получаем, что t-кри\-те\-рии принимают вид:
\begin{align*}
%\left.
%\begin{array}{rl}
    t'_1 &=  \fr{\bar{Y_T} - \bar{Y_R}-\delta_1}{\hat\sigma_d\sqrt{2/n}} > t(2n-2, \alpha); \\
    t'_2 &=  \fr{\bar{Y_T} - \bar{Y_R}-\delta_2}{\hat\sigma_d\sqrt{2/n}} < -t(2n-2, \alpha)  
%    \end{array}
 %   \right\}
  %  \label{t_2}
\end{align*}
и соответствующий доверительный интервал принимает вид:
\begin{multline*}
%\label{interv_2}
    \left(\bar{Y_T} - \bar{Y_R} + t(\alpha, 2n-2)\hat\sigma_d\sqrt{\fr{2}{n}};\  \right.\\
\left.     \bar{Y_T} - \bar{Y_R} - t(\alpha, 2n-2)\hat\sigma_d\sqrt{\fr{2}{n}}\right).
\end{multline*}

\vspace*{-9pt}

\section{Оценка уровня значимости при~наличии пропущенных данных}

\vspace*{-3pt}

Рассмотрим выборочное пространство $\chi$~--- пространство элементарных событий. 
Статистический критерий разбивает пространство элементарных событий~$\chi$ 
на два подмножества:
\begin{enumerate}[(1)]
\item область принятия гипотезы $\chi_0$~--- множество, состоящее 
из точек, для которых гипотеза~$H_0$ принимается;\\[-14pt]
\item  область отклонения гипотезы $\chi_A$~--- множество, 
состоящее из точек, для которых гипотеза~$H_0$ отвергается.
\end{enumerate}


Говорят, что критерий имеет уровень зна\-чи\-мости~$\alpha$, если вероятность 
наступления ошибки первого рода не превышает~$\alpha$, $0 \hm< \alpha\hm < 1$, 
для $\delta\hm \in \chi_A$:

\noindent
\begin{multline*}
{\sf P}\left\{\mbox{отклонить } H_0 \mbox{ при\ истинной}\right.\\[-1pt]
\left.\mbox{небиоэквивалентности}\right\} ={}\\[-1pt]
{}
= {\sf P}\left\{\mbox{отклонить\ } H_0, \delta \in \chi_0\right\} \leqslant \alpha\,.
\end{multline*}



Рассмотрим задачу при наличии пропусков в~данных:
пусть $q$~--- уровень полноты данных,
 т.\,е.\ 
доля данных, оставшихся от изначальных дан-\linebreak\vspace*{-12pt}

\columnbreak

\noindent
ных, $0 \hm<q\hm \leqslant 1$
($1 - q$~--- доля пропущенных данных в~выборке).



Тогда $\tilde{Y_T} = \bar{Y_T} + \ln(q)$~--- 
выборочное среднее логарифмов сравнительных характеристик для лекарственного 
препарата~$T$ при наличии пропущенных данных.


Критическая область для данной задачи имеет следующий вид:

\vspace*{-4pt}

\noindent
\begin{multline*}
\chi_A' = \left\{(\tilde{Y_T} - \bar{Y_R}, \hat{\sigma_d}): 
\delta_1 + t\left(\alpha, 2n-2\right)\hat\sigma_d\sqrt{\fr{2}{n}} <{}\right.\\
\left.{}< \tilde{Y_T} - \bar{Y_R} <  \delta_2 - t\left(\alpha, 2n-2\right)\hat\sigma_d
\sqrt{\fr{2}{n}}\right\}.
\end{multline*}

\vspace*{-2pt}

Рассмотрим функцию мощности:

\vspace*{-3pt}

\noindent
\begin{align*}
    \phi_{\hat{\sigma_d}}'(\delta) &= {\sf P}\{\mbox{отклонить } H_0 \mbox{ при истинной}\\
&    \mbox{биоэквивалентности}\}={} \\
    &{}= {\sf P}\{(\tilde{Y_T} - \bar{Y_R}, \hat{\sigma_d}): \chi_A', если \delta \in \chi_A'\}.
\end{align*}

\vspace*{-2pt}

\noindent
Фиксируем $\delta = \delta_0$, получаем:

\vspace*{-2pt}

\noindent
\begin{multline*}
    \phi_{\hat{\sigma_d}}'(\delta_0) = {} \\
{}= {\sf P}\left(\left(\tilde{Y_T} - \bar{Y_R}, \hat{\sigma_d}\right): 
\delta_1 + t(\alpha, 2n-2)\hat\sigma_d\sqrt{\fr{2}{n}} < {}\right.\\
\left.{}<\tilde{Y_T} - \bar{Y_R} <  \delta_2 - t(\alpha, 2n-2)
\hat\sigma_d\sqrt{\fr{2}{n}}\vert  \delta =\delta_0\right) ={} \\
{}={\sf P}\left( \delta_1 + t(\alpha, 2n-2)\hat\sigma_d\sqrt{\fr{2}{n}}
 < \bar{Y_T} + \ln(q) - {}\right.\\
\left. {}-\bar{Y_R} <  \delta_2 - t(\alpha, 2n-2)\hat\sigma_d
 \sqrt{\fr{2}{n}}\right) ={} \\
 {}={\sf P}\Biggl(\fr{\delta_1 + t(\alpha, 2n-2)\hat\sigma_d\sqrt{{2}/{n}}- 
\ln(q)-\delta_0}{\sigma_d\sqrt{{2}/{n}}}<{}\\
{}< \fr{\bar{Y_T} - 
\bar{Y_R}-\delta_0}{\sigma_d\sqrt{{2}/{n}}} <{}\\
{} < \fr{ \delta_2 - t(\alpha, 2n-2)\hat\sigma_d\sqrt{{2}/{n}}- 
\ln(q)-\delta_0}{\sigma_d\sqrt{{2}/{n}}}\Biggr) = {}\\
{}= \{\mbox{фиксируем } \hat\sigma_d\} = {}\\
{}=E\Biggl[{\sf P}\Biggl(\fr{\delta_1 + t(\alpha, 2n-2)\hat\sigma_d\sqrt{{2}/{n}}
- \ln(q)-\delta_0}{\sigma_d\sqrt{{2}/{n}}}< {}\\
{}<\fr{\bar{Y_T} - 
\bar{Y_R}-\delta_0}{\sigma_d\sqrt{{2}/{n}}} <{}\\
 {} < \fr{ \delta_2 - t(\alpha, 2n-2)\hat\sigma_d\sqrt{{2}/{n}}- 
\ln(q)-\delta_0}{\sigma_d\sqrt{{2}/{n}}}|\hat\sigma_d\Biggr)\Biggr] = {}
\end{multline*}

 \noindent
 \begin{multline*}
\hspace*{-4pt}{}=E\Biggl[\Phi\left(\fr{\delta_1 + t(\alpha, 2n-2)\hat\sigma_d\sqrt{{2}/{n}}-
 \ln(q)-\delta_0}{\sigma_d\sqrt{{2}/{n}}}\right) - {}\\[3pt]
 {}-
 \Phi\left( \fr{ \delta_2 - t(\alpha, 2n-2)\hat\sigma_d\sqrt{{2}/{n}}- 
 \ln(q)-\delta_0}{\sigma_d\sqrt{{2}/{n}}}\right)\Biggr], 
\end{multline*}
где $\Phi(x)$~--- функция стандартного нормального распределения.

Процедура двух односторонних тестов Шуирманна используется в~условиях 
решающего правила 80/125~\cite{schuirmann, article, hsu}. Это значит, что
 $\delta_1\hm = \ln(0,80)\hm \approx -0{,}2231$ и~$\delta_2\hm = \ln(1,25)
 \hm \approx 0{,}2231$; следовательно, $\delta_1 \hm\approx -\delta_2$. 
 Тогда видим, что функция $\phi_{\hat{\sigma_d}}'(\delta)$ сим\-мет\-рич\-на 
 относительно точки $\delta \hm= -\ln(q)$ и~достигает максимума в~этой точке.

Тогда 

\vspace*{-6pt}

\noindent
\begin{multline*}
\max_{\delta \in \chi_0} {\sf P}\{\mbox{отклонить } H_0\} \hm= 
\phi_{\hat{\sigma_d}}'(\delta_2) ={}\\
{}\{\mbox{подставим }  \delta = \delta_2,\ 
\delta_1 = -\delta_2, \sigma_d=\hat\sigma_d \} ={}\\
    {}={\sf P}\left( \fr{-2\delta_2 - \ln(q)}{\hat\sigma_d\sqrt{{2}/{n}}} 
    +  t(\alpha, 2n-2)<{}\right.\\
   \left. {}<\fr{\bar{Y_T} - \bar{Y_R}-\delta_2}{\hat\sigma_d
    \sqrt{{2}/{n}}}< \fr{- \ln(q)}{\hat\sigma_d\sqrt{{2}/{n}}}- 
    t(\alpha, 2n-2)\right) = {}\\
    {}={\sf P}\left( \fr{-2\delta_2 - \ln(q)}{\hat\sigma_d\sqrt{{2}/{n}}} + 
     t(\alpha, 2n-2)<{}\right.\\
\left.     {}<\fr{\bar{Y_T} - \bar{Y_R}-\delta_2}{\hat\sigma_d\sqrt{{2}/{n}}}<
      - t(\alpha, 2n-2)\right) + {}\\
    {}+{\sf P}\left( - t(\alpha, 2n-2)<\fr{\bar{Y_T} - \bar{Y_R}-\delta_2}
    {\hat\sigma_d\sqrt{{2}/{n}}}<{}\right.\\
\left.    {}< \fr{- \ln(q)}{\hat\sigma_d\sqrt{{2}/{n}}}- 
    t(\alpha, 2n-2)\right) \leqslant {}\\
{}\leqslant {\sf P}\left(\fr{\bar{Y_T} - \bar{Y_R}-\delta_2}{\hat\sigma_d\sqrt{{2}/{n}}}< 
- t(\alpha, 2n-2)\right) + {}\\
    {}+ {\sf P}\left( - t(\alpha, 2n-2)<\fr{\bar{Y_T} - \bar{Y_R}-\delta_2}
    {\hat\sigma_d\sqrt{{2}/{n}}}< {}\right.\\
\left.    {}<\fr{- \ln(q)}{\hat\sigma_d\sqrt{{2}/{n}}}- 
    t(\alpha, 2n-2)\right) = {}
   \\
    {}=\alpha + {\sf P}\left( - t(\alpha, 2n-2)<\fr{\bar{Y_T} - 
    \bar{Y_R}-\delta_2}{\hat\sigma_d\sqrt{{2}/{n}}}< {}\right.\\
\left.    {}<\fr{- \ln(q)}
    {\hat\sigma_d\sqrt{{2}/{n}}}- t(\alpha, 2n-2)\right) =\alpha + \alpha'. \\
\end{multline*}

\vspace*{-18pt}

\noindent
Полученная оценка показывает, что величина ошибки первого рода не превосходит
 $\alpha \hm+ \alpha'$. При чем в~работах~\cite{book, article} показано, что 
 при исполь-\linebreak\vspace*{-12pt}
 
 \columnbreak
 
 \noindent
зо\-вании двух односторонних тестов Шуирманна\linebreak
  величина 
 вероятности ошибки первого рода не превосходит~$\alpha$. 
 В~рассматриваемой постановке уровень значимости критерия повышается на 
 величину~$\alpha'$, что обусловлено наличием пропущенных данных для 
 воспроизводимого лекарственного препарата. Таким образом, риск потенциального 
 выхода на рынок небиоэквивалентного лекарственного препарата повышается.
 
 \vspace*{-15pt}

\section{Заключение}

 \vspace*{-5pt}

Процедура двух односторонних тестов Шуирманна~--- одно из основных средств 
при проверке гипотезы биоэквивалентности. При исследовании критериев принятия 
гипотезы биоэквивалентности важную роль играет оценка вероятности наступления 
ошибки первого рода. Важность ее контроля\linebreak обусловлена риском пациента получить 
препарат с~несоответствующими эффективностью и~без\-опас\-ностью.
%
В~данной статье впервые дана оценка уровня значимости процедуры двух односторонних
 тестов Шуирманна при наличии пропущенных данных. В~част\-ности, в~явном виде
  показана та ее часть, которая зависит от уровня полноты данных.
%
В практическом плане данная оценка может быть использована для корректировки 
задаваемого уровня зна\-чи\-мости при известном уровне полноты данных, чтобы 
обеспечить гарантированную эффективность и~безопасность воспроизведенных лекарств.

 \vspace*{-15pt}


{\small\frenchspacing
 { %\baselineskip=10.5pt
 \addcontentsline{toc}{section}{References}
 \begin{thebibliography}{9}
 
  \vspace*{-5pt}
  
    \bibitem{defin} 
    Правила проведения исследований биоэквивалентности лекарственных 
    средств Евразийского экономического союза. 
    {\sf 
    http://www.eurasiancommission.org/ ru/act/texnreg/deptexreg/konsultComitet/Documents/\linebreak Правила\%20БЭИ\%20итог\%2020.02.2015\%20на\%20\linebreak сайт.pdf}.
    \bibitem{schuirmann}
    \Au{Schuirmann D.\,J.}
    A~comparison of the two one-sided tests procedure and the power approach 
    for assessing the equivalence of average bioavailability~// J.~Pharmacokinet. 
    Biop., 1987. Vol.~15. P.~657--680.
    \bibitem{book}
    \Au{Chow Shein-Chung, Liu Jen-pei.}
    Design and analysis of bioavailability and bioequivalence studies.~--- 
    Chapman \& Hall/CRC, 2009. 735~p.
    \bibitem{article}
\Au{Драницына М.\,А., Захарова~Т.\,В., Ниязов~Р.\,Р.}
    Свойства процедуры двух односторонних тестов 
    для признания биоэквивалентности лекарственных препаратов~// Ремедиум. 
    Журнал о рынке лекарств и~медицинской техники, 2019. №\,3. С.~40--47.
    \bibitem{hsu} \Au{Berger R.\,L., Hsu~J.\,C.} 
    Bioequivalence trials, intersection--union tests and equivalence confidence sets~// 
    Stat. Sci., 1996. Vol.~11. No.~4. P.~283--319.
    
     \end{thebibliography}

 }
 }

\end{multicols}

\vspace*{-12pt}

\hfill{\small\textit{Поступила в~редакцию 09.05.19}}

%\vspace*{8pt}

\pagebreak

%\newpage

\vspace*{-28pt}

%\hrule

%\vspace*{2pt}

%\hrule

%\vspace*{-2pt}

\def\tit{EVALUATION OF THE SIGNIFICANCE LEVEL IN~SCHUIRMANN'S TEST FOR~CHECKING 
THE~BIOEQUIVALENCE HYPOTHESIS IN~MISSING DATA CONDITIONS}


\def\titkol{Evaluation of the significance level in~Schuirmann's test for~checking 
the~bioequivalence hypothesis in~missing data conditions}

\def\aut{T.\,V.~Zakharova$^{1,2}$ and A.\,A.~Tarkhov$^1$}

\def\autkol{T.\,V.~Zakharova and A.\,A.~Tarkhov}

\titel{\tit}{\aut}{\autkol}{\titkol}

\vspace*{-11pt}


\noindent
$^1$Department of Mathematical Statistics, Faculty of Computational Mathematics 
 and Cybernetics, M.\,V.~Lo\-mo-\linebreak
 $\hphantom{^1}$nosov Moscow State University, 1-52~Leninskiye Gory, 
 GSP-1, Moscow 119991, Russian Federation
 
 \noindent
 $^2$Institute of 
 Informatics Problems, Federal Research Center ``Computer Science and Control'' 
 of the Russian\linebreak
  $\hphantom{^1}$Academy of Sciences, 44-2~Vavilov Str., Moscow 119333, 
 Russian Federation

\def\leftfootline{\small{\textbf{\thepage}
\hfill INFORMATIKA I EE PRIMENENIYA~--- INFORMATICS AND
APPLICATIONS\ \ \ 2019\ \ \ volume~13\ \ \ issue\ 3}
}%
 \def\rightfootline{\small{INFORMATIKA I EE PRIMENENIYA~---
INFORMATICS AND APPLICATIONS\ \ \ 2019\ \ \ volume~13\ \ \ issue\ 3
\hfill \textbf{\thepage}}}

\vspace*{3pt}    



\Abste{The bioequivalence hypothesis testing is the important task 
in pharmacokinetics. It helps to make a~decision about the equivalence 
of the reproduced drug to the reference drug. One of the problems of bioequivalence 
studies is the availability of missing data. 
A~small amount of data entails the inability to delete a~data sample with 
missing data. Therefore, there is a~task to estimate the impact of missing data 
on bioequivalence testing task, in particular, to estimate the significance level. 
The main method of the bioequivalence hypothesis testing is Schuirmann's 
two one-sided tests procedure. The article shows the significance level evaluation 
of this procedure in the case of missing data. The evaluation component, depending 
on the level of data completeness, is shown in the explicit form.}


\KWE{bioequivalence; significance level; type I error; missing data; 
Schuirmann's two one-sided tests procedure}


\DOI{10.14357/19922264190309} 

%\vspace*{-14pt}

\Ack
   \noindent
   The paper was supported by the Russian Foundation for Basic Research (project  
18-07-00252).


%\vspace*{-6pt}

  \begin{multicols}{2}

\renewcommand{\bibname}{\protect\rmfamily References}
%\renewcommand{\bibname}{\large\protect\rm References}

{\small\frenchspacing
 {%\baselineskip=10.8pt
 \addcontentsline{toc}{section}{References}
 \begin{thebibliography}{9}

\bibitem{1-zah}
Pravila provedeniya issledovaniy bioekvivalentnosti 
lekarstvennykh sredstv Evraziyskogo ekonomicheskogo soyuza. Available at:
{\sf http://www.eurasiancommission.\linebreak org/ru/act/texnreg/deptexreg/konsultComitet/\linebreak Documents/Pravila\%20BEI\%20itog\%2020.02.2015\%20\linebreak na\%20sajt.pdf}
  (accessed May~8, 2019).
\bibitem{2-zah}
\Aue{Schuirmann, D.\,J.} 1987. 
A~comparison of the two one-sided tests procedure and the power approach for 
assessing the equivalence of average bioavailability.
\textit{J.~Pharmacokinet. Biop.} 15:657--680.

\columnbreak 

\bibitem{3-zah}
\Aue{Chow, Shein-Chung, and Jen-pei Liu.} 2009. \textit{Design 
and analysis of bioavailability and bioequivalence studies.} 
Chapman \& Hall/CRC. 735~p.

\vspace*{-2pt}

\bibitem{4-zah}
\Aue{Dranitsyna, M.\,A., T.\,V.~Zakharova, and R.\,R.~Niyazov.}
 2019. Svoystva protsedury dvukh odnostoronnikh testov dlya priznaniya
  bioekvivalentnosti lekarstvennykh preparatov [Properties of the 
   two-sided tests procedure for the  bioequivalence assessment of medical products]. 
  \textit{Remedium. Zh.~o~rynke lekarstv i~meditsinskoy tekhniki}
   [Remedium: J.~of the Market of Medicines and Medical Equipment] 2019(3):40--47.
   
   \vspace*{-2pt}
   
\bibitem{5-zah}
\Aue{Berger, R.\,L.,  and J.\,C.~Hsu.} 1996. 
Bioequivalence trials, intersection--union tests and equivalence confidence sets. 
\textit{Stat. Sci.} 11(4):283--319.
\end{thebibliography}

 }
 }

\end{multicols}

%\vspace*{-7pt}

\hfill{\small\textit{Received May 9, 2019}}

%\pagebreak

\vspace*{-12pt}

\Contr

\noindent
\textbf{Zakharova Tatiana V.} (b.\ 1962)~--- 
Candidate of Science (PhD) in physics and mathematics, associate professor,
 Department of Mathematical Statistics, Faculty of Computational Mathematics 
 and Cybernetics, M.\,V.~Lomonosov Moscow State University, 1-52~Leninskiye Gory, 
 GSP-1, Moscow 119991, Russian Federation; senior scientist, Institute of 
 Informatics Problems, Federal Research Center ``Computer Science and Control'' 
 of the Russian Academy of Sciences, 44-2~Vavilov Str., Moscow 119333, 
 Russian Federation; \mbox{tvzaharova@mail.ru}
 
 \vspace*{3pt}

\noindent
\textbf{Tarkhov Alexey A.} (b.\ 1995)~--- 
master student, Department of Mathematical Statistics, Faculty of Computational 
Mathematics and Cybernetics, M.\,V.~Lomonosov Moscow State University, 
1-52~Leninskiye Gory, GSP-1, Moscow 119991, Russian Federation; 
\mbox{alexeytarkhov@gmail.com}

\label{end\stat}

\renewcommand{\bibname}{\protect\rm Литература}    %9
\def\stat{gaidamaka}

\def\tit{МЕТОД РАСЧЕТА ХАРАКТЕРИСТИК ИНТЕРФЕРЕНЦИИ
ДВУХ ВЗАИМОДЕЙСТВУЮЩИХ УСТРОЙСТВ
В~БЕСПРОВОДНОЙ ГЕТЕРОГЕННОЙ СЕТИ$^*$}

\def\titkol{Метод расчета характеристик интерференции
двух взаимодействующих устройств
в~беспроводной гетерогенной сети}

\def\aut{Ю.\,В.~Гайдамака$^1$, А.\,К.~Самуйлов$^2$}

\def\autkol{Ю.\,В.~Гайдамака, А.\,К.~Самуйлов}

\titel{\tit}{\aut}{\autkol}{\titkol}

{\renewcommand{\thefootnote}{\fnsymbol{footnote}} \footnotetext[1]
{Работа выполнена при финансовой поддержке РФФИ (проекты 14-07-00090
и~15-07-03051).}}


\renewcommand{\thefootnote}{\arabic{footnote}}
\footnotetext[1]{Российский университет дружбы народов, ygaidamaka@sci.pfu.edu.ru}
\footnotetext[2]{Российский университет дружбы народов; Технологический университет г.\ Тампере, Финляндия,
aksamuylov@gmail.com}


\Abst{Одним из показателей качества функционирования современных беспроводных сетей
является отношение сигнала к~сумме интерференции и шума (SINR, Signal to Interference plus
Noise Ratio) в~беспроводных каналах связи. Поскольку значение этой характеристики
существенно зависит от расстояния между интерферирующими устройствами, задача оценки
значения SINR часто сводится к~вычислению длины одной из сторон треугольника,
в~вершинах которого находятся взаимодействующие устройства. В~данной статье решается
задача нахождения математического ожидания и~среднеквадратического отклонения
отношения сигнал/интерференция пары взаимодействующих устройств в достаточно общих
предположениях о~распределении случайных величин (с.в.)\ расстояний между
интерфери\-ру\-ющи\-ми устройствами. Предложенный метод может быть использован
в~качестве основы для анализа интерференции в~гетерогенной сети с~применением различных
беспроводных технологий, включая анализ беспроводных взаимодействий оконечных
устройств, на которые интерференция оказывает наиболее сильное воздействие.}

\KW{беспроводная сеть; LTE; интерференция; SINR; взаимодействие устройств; D2D}

\DOI{10.14357/19922264150102}


\vskip 14pt plus 9pt minus 6pt

\thispagestyle{headings}

\begin{multicols}{2}

\label{st\stat}

\section{Постановка задачи}

  В современных беспроводных сетях, построенных на базе технологии LTE
(Long Term Evolution), оценка интерференции между взаимодействующими
устройствами является одной из основных задач анализа показателей качества
функционирования~[1,~2]. Под интерференцией понимается взаимодействие
сигналов, передаваемых разными\linebreak источниками на одном и~том же канале.
Интерференция вызывает искажение сигнала рас\-смат\-ри\-ва\-емо\-го источника под
воздействием сигнала сторонне\-го источника. В~гетерогенных сетях
беспроводного взаимодействия оконечных устройств D2D
  (device-to-device)~[3], где плот\-ность интерферирующих объектов высока,
интерференция оказывает существенное влияние на принимаемый оконечным
устройством сигнал. При анализе беспроводных взаимодействий устройств
обычно рассматривается несколько источников сигнала (передатчиков),
распределенных на плоскости согласно некоторому закону~[4]. Упрощение
задачи состоит в~том, что, рассмотрев один передатчик и~оценив
характеристики интерференции на соответствующем ему приемном устройстве
(приемнике), можно предположить, что основные показатели будут идентичны
и~для остальных пар <<пе\-ре\-дат\-чик--при\-ем\-ник>>. В~данной статье
решается задача нахождения числовых характеристик отношения
сигнал/ин\-тер\-фе\-рен\-ция пары взаимодействующих устройств.

  Отношение сигнала к сумме интерференции и~шума, SINR,
  является одной из основных характеристик качества канала
  в~беспроводных сетях связи~[5--7]. Отношение сигнала к~сумме интерференции 
и~шума на стороне приемника определяется по следующей формуле:
  \begin{equation}
  \mathrm{SINR} = \fr{S}{\sigma^2 +I}\,,
  \label{e1-gai}
  \end{equation}
где $S$~--- мощность принимаемого сигнала от соответствующего
передатчика; $\sigma^2$~--- мощность шума; $I$~--- мощность принимаемого
сигнала от интерферирующих передатчиков. Согласно линейной модели~[4]
\begin{equation}
S=gl^{-\alpha}\,,
\label{e2-gai}
\end{equation}
где $g$~--- базовая мощность сигнала передатчика, соответствующего
рассматриваемому приемнику; $l$~--- расстояние между передатчиком
и~приемником; $\alpha$~--- коэффициент потерь (path loss exponent),
принимающий значение от~2 (при условии прямой видимости) до~6 (в~худшем
случае). Величина~$I$ в~знаменателе формулы~(1) соответствует суммарной
мощности сигнала от всех интерферирующих передатчиков, где каждое
слагаемое имеет вид~(2). Заметим, что принцип повторного использования
частот (frequency reuse) в~беспроводных сетях связи поколения 4G (4th
Generation) позволяет назначать одну и~ту же единицу ресурса сети (например,
один и~тот же ресурсный блок LTE) нескольким парам взаимодействующих
устройств, если интерференция не превосходит определенного стандартами
уровня.

  Рассмотрим случай, когда несколько принимающих устройств (приемников)
и~одно передающее устройство (передатчик), образующие кластер,
расположены на плоскости внутри круга радиуса~$r_0$, причем передатчик
расположен в центре круга. Такой кластер образуется, например, при
проведении интерактивного занятия преподавателя с учениками, когда можно
предположить, что передатчик располагается в центре круга, а приемники
равномерно распределены внутри круга. Для передачи данных на каждую пару
взаимодействующих устройств внутри кластера планировщиком распределения
радиоресурсов в беспроводной сети 4G назначается по одному ресурсному
блоку LTE, и тогда сигналы взаимодействующих пар не интерферируют друг
с~другом. Но если в соседнем помещении также проходит интерактивное
занятие и там использованы те же ресурсные блоки, то пары из соседних
кластеров, использующие один и тот же ресурсный блок, будут создавать
помехи друг другу. Сведем задачу к анализу взаимодействия двух пар
устройств в двух кластерах, как показано на рис.~1.



  Пару взаимодействующих устройств, для которой будем рассчитывать
показатели качества канала, назовем целевой, а соответствующую ей пару
устройств обозначим TR$_0\hm= \langle \mathrm{Tx}_0, \mathrm{Rx}_0\rangle$.
Остальные пары, которые создают помехи целевой паре 
$\mathrm{TR}_0$,\linebreak\vspace*{-12pt}
\begin{center}  %fig1
\vspace*{8pt}
\mbox{%
 \epsfxsize=77.569mm
 \epsfbox{gai-1.eps}
 }
\end{center}

\noindent
{{\figurename~1}\ \ \small{Схема взаимодействия интерферирующих устройств}}


%\vspace*{9pt}


\addtocounter{figure}{1}


\noindent
 обозначим $\mathrm{TR}_i\hm= \langle
\mathrm{Tx}_i, \mathrm{Rx}_i\rangle$ и~будем называть их интерферирующими. Расстояние
между Rx$_i$ и~Tx$_i$ обозначим $R_i$, а~расстояние между Tx$_0$ и~Tx$_i$
обозначим~$U_i$. Мощность интерферирующего сигнала от пары TR$_i$
является функцией расстояния между приемником Rx$_0$ из целевой пары и
интерферирующим передатчиком~Tx$_i$, которое обозначим~$D_i$. Угол
между прямой, соединяющей целевые передатчик~Tx$_0$ и~приемник~Rx$_0$,
и~прямой, соеди\-ня\-ющей передатчики~Tx$_0$ и~Tx$_i$,
обозначим~$\gamma_i$.

  Рассмотрим систему двух кластеров, показанную на рис.~1. В~условиях
отсутствия шума и~одинаковой базовой мощности~$g$ сигналов обоих
передатчиков искомой характеристикой является отношение
  сигнал/ин\-тер\-фе\-рен\-ция SIR для приемника~Rx$_0$, вычисляемое по
формуле:
  \begin{equation}
\mathrm{SIR}=\left( \fr{D_1}{R_0}\right)^{\alpha}\,.
  \label{e3-gai}
  \end{equation}

  Будем считать, что $R_0$, $U_i$ и~$\gamma_i$ являются
с.в.\ с~заданными функциями распределения. Задача состоит
в~нахождении числовых характеристик с.в.~SIR. Для
решения задачи в следующем разделе статьи предлагается метод нахождения
совместной плотности распределения с.в.~$R_0$ и~$D_i$, что позволяет
вычислять начальные моменты ${\sf E}[\mathrm{SIR}^n]$ с.в.~SIR.

\vspace*{-6pt}

\section{Метод расчета отношения сигнал/интерференция}

%\vspace*{-2pt}

  Как видно из формулы~(3), с.в.~SIR пропорциональна с.в.~$D_1$, которая,
в~свою очередь, зависит от с.в.~$R_0$. В~этом случае для нахождения
характеристик с.в.~SIR необходимо найти совместное распределение
с.в.~$R_0$ и~$D_1$.

  Введем обозначения $\xi_1{:=} R_0$, $\xi_2 {:=} U_1$, $\xi_3 {:=}
\gamma_1$, $\eta_1 {:=} D_1$. Тогда $w_{\xi_1,\xi_2,\xi_3}(x_1,x_2,x_3) {:=}$\linebreak
${=:}\;f_{R_0, U_1, \gamma_1}(x_1,x_2,x_3)$~--- совместная плот\-ность
распределения с.в.~$R_0$, $U_1$ и~$\gamma_1$, а~$W_{\xi_1,\eta_1}(x_1,y_1)
{:=} f_{R_0, D_1}(x_1,y_1)$~--- искомое совместное распределение с.в.~$R_0$
и~$D_1$. По теореме косинусов с.в.~$\eta_1$ является функцией с.в.~$\xi_1$,
$\xi_2$ и~$\xi_3$:
  \begin{equation}
  \eta_1=\sqrt{\xi_1^2+\xi_2^2-2\xi_1\xi_2\cos \xi_3}\,.
  \label{e4-gai}
  \end{equation}

  Следуя~\cite{8-gai, 9-gai}, введя вспомогательную
переменную $\eta_2\hm=\xi_3$, искомое распределение можно найти по
следующей формуле:
  \begin{multline}
W_{\xi_1, \eta_1} (y_1,y_2) ={}\\
{}=\sum\limits_{i=1}^2
\int\limits_{\mathrm{Y}_{3,j}}\!\!\! w_{\xi_1,\xi_2,\xi_3}\left(
y_1,\varphi_i(y_1,y_2,y_3),y_3\right) \times{}\\[-6pt]
{}\times
\left\vert \fr{\partial \varphi_j(y_1,y_2,y_3)}
{\partial y_2}\right\vert\,dy_3\,,
  \label{e5-gai}
  \end{multline}
где $\varphi_j$~--- обратное преобразование правой части формулы~(\ref{e4-gai})
относительно~$\xi_2$:
\begin{align*}
\varphi_1(y_1,y_2,y_3) &= y_1\cos y_3 +\sqrt{y_2^2-y_1^2+y_1^2\cos^2 y_3}\,;\\
\varphi_2(y_1,y_2,y_3) &= y_1\cos y_3 -\sqrt{y_2^2-y_1^2+y_1^2\cos^2 y_3}\,.
\end{align*}

  В формуле~(\ref{e5-gai}) области значений Y$_{3,j}$ переменной~$y_3$ для
$j$-й вет\-ви обратного преобразования определяются системой неравенств:
  \begin{equation}
  \left.
  \begin{array}{c}
  \varphi_j(y_1,y_2,y_3)\geq0\,;\\[6pt]
  y_1\geq 0\,;\\[6pt]
  y_2\geq 0\,;\\[6pt]
  0\leq y_3\leq 2\pi\,.
  \end{array}
  \right\}
  \label{e6-gai}
  \end{equation}

  Решая систему~(\ref{e6-gai}), нетрудно убедиться, что для первой ветви
обратного преобразования  $\mathrm{Y}_{3,1}\hm= \mathrm{Y}_{3,1}^1\cup
\mathrm{Y}_{3,1}^2\cup \mathrm{Y}_{3,1}^3$, где
  \begin{align}
  \hspace*{-2mm}\mathrm{Y}_{3,1}^1 &=\begin{cases}
  0\leq y_2\leq y_1;\\
  0\leq y_3\leq \fr{1}{2}\,\mathrm{arccos}\,\left( \fr{y_1^2-
2y_2^2}{y_1^2}\right);\end{cases}
  \label{e7-1-gai}
\\
\hspace*{-2mm}\mathrm{Y}_{3,1}^2 &= \begin{cases}
  0\leq y_2\leq y_1;\\
  2\pi -\fr{1}{2}\,\mathrm{arccos}\left( \fr{y_1^2-2y_2^2}{y_1^2}\right) \leq
y_3\leq 2\pi;
  \end{cases}\!\!\!\!\!
  \label{e7-2-gai}
  \\
\hspace*{-2mm}\mathrm{Y}_{3,1}^3 &= \begin{cases}
  y_2\geq y_1;\\
  0\leq y_3\leq 2\pi,
  \end{cases}\!\!\!\!\!\!\!\!\!
  \label{e7-3-gai}
  \end{align}
а для второй ветви  $\mathrm{Y}_{3.2}=\mathrm{Y}_{3,2}^1\cup
\mathrm{Y}_{3,2}^2$, где
\begin{align}
\label{e8-1-gai}
\hspace*{-2mm}\mathrm{Y}_{3,2}^1 &= \begin{cases}
0\leq y_2\leq y_1\,;\\
0\leq y_3\leq \fr{1}{2}\,\mathrm{arccos}\left( \fr{y_1^2-2y_2^2}{y_1^2}\right);
\end{cases}
\\
\hspace*{-2mm}\mathrm{Y}_{3,2}^2 &=\begin{cases}
0\leq y_2\leq y_1\,;\\
2\pi -\fr{1}{2}\,\mathrm{arccos} \left( \fr{y_1^2-2y_2^2}{y_1^2}\right) \leq y_3\leq
2\pi.\!\!\!\!\!\!\!\!
\end{cases}
\label{e8-2-gai}
\end{align}

  Таким образом, получена формула для вычисления совместной плотности
с.в.~$R_0$ и~$D_1$:
  \begin{multline}
  W_{\xi_1,\eta_1}(y_1,y_2) ={}\\
  {}=\sum\limits_{i=1}^2 \int\limits_{\mathrm{Y}_{3,i}}
\fr{w_{\xi_1,\xi_2,\xi_3} (y_1,\varphi_i(y_1,y_2,y_3),y_3) y_2} {\sqrt{y_2^2-
y_1^2+y_1^2\cos^2 y_3}}\,dy_3\,,
  \label{e9-gai}
  \end{multline}
где $\mathrm{Y}_{3,j}$ вычисляются по
формулам~(\ref{e7-1-gai})--(\ref{e8-2-gai}).

  В следующем разделе приведен пример численного анализа
с~использованием формул~(\ref{e7-1-gai})--(\ref{e9-gai}).

\section{Пример численного анализа}

  В рассматриваемом примере предложенный выше метод использован для
расчета начальных моментов ${\sf E}[\mathrm{SIR}^n]$ отношения сигнал/интерференция,
которые определяются следующей формулой:
  \begin{multline}
{\sf   E}[\mathrm{SIR}^n] ={}\\
{}=\int\limits_{0\leq y_1\leq r_0} \int\limits_{y_2\geq0} \left(
\fr{y_2}{y_2}\right)^{n\alpha} W_{\xi_1,\eta_1}(y_1,y_2)\,dy_2dy_1\,.
  \label{e10-gai}
  \end{multline}

  Рассматривается случай, когда целевой приемник Rx$_0$ находится внутри
круга единичного \mbox{радиуса} ($r_0\hm=1$), в центре которого расположен
передат\-чик~Tx$_0$, а~интерферирующий передатчик~Tx$_1$~--- в~кольце
вокруг передатчика~Tx$_0$ с~внутренним радиусом~$r_0$ и~внешним
радиусом~$h_0$ (рис.~2).

\begin{center}  %fig2
\vspace*{8pt}
\mbox{%
 \epsfxsize=77.111mm
 \epsfbox{gai-2.eps}
 }


\noindent
{{\figurename~2}\ \ \small{Пример взаимодействия двух устройств}}

\end{center}


\vspace*{9pt}


\addtocounter{figure}{1}


%\noindent


  Тогда с.в.~$R_0$ расстояния от целевого передатчика Tx$_0$ до
соответствующего ему приемника Rx$_0$ и~с.в.~$U_1$ расстояния от целевого
передатчика~Tx$_0$ до интерферирующего передатчика~Tx$_1$ имеют
распределения
  \begin{alignat*}{2}
  f_{R_0}(r) &= 2r\,,&\quad 0&\leq r\leq1\,;\\
  f_{U_1}(u) &= \fr{2u}{h_0^2-1}\,,&\quad 1&\leq u\leq h_0\,.
  \end{alignat*}
Будем считать, что с.в.\ угла~$\gamma_1$ равномерно распределена на отрезке
$[0,\,2\pi]$, а~коэффициент потерь в~формуле~(2) принимает значение
$\alpha\hm=2$. Приняты условные единицы измерения: например, расстояние
между взаимодействующими устройствами может измеряться в~метрах,
а~величина SIR~--- в~децибелах.

  По формулам~(\ref{e7-1-gai})--(\ref{e10-gai}) рассчитано математическое
ожидание отношения сигнал/ин\-тер\-фе\-рен\-ция ${\sf E}[\mathrm{SIR}]$, представленное в
таблице в зависимости от радиуса внешней границы кольца, внутри которого
распределены интерферирующие передатчики. В~таблице также показаны
значения математического ожидания расстояния  ${\sf E}[U_1]$ от целевого
передатчика~Tx$_0$ до интерферирующего передатчика~Tx$_1$.

%  \begin{table*}\small
  \begin{center}
  \begin{tabular}{|c|c|c|}
  \multicolumn{3}{p{48mm}}{Математическое ожидание величины~SIR}\\
  \multicolumn{3}{c}{\ }\\[-5pt]
  \hline
\ \ \ \ $h_0$\ \ \ \ &\ \ \ \ ${\sf E}[U_1]$\ \ \ \ &${\sf E}[\mathrm{SIR}]$\\
\hline
2&1,56&4,84985\\
3&2,17&7,41701\\
4&2,8\hphantom{9}&9,54562\\
5&3,44&11,30286\hphantom{9}\\
\hline
\end{tabular}
\end{center}
%\end{table*}

\begin{center}  %fig3
\vspace*{18pt}
\mbox{%
 \epsfxsize=77.754mm
 \epsfbox{gai-3.eps}
 }
 \end{center}


\noindent
{{\figurename~3}\ \ \small{Числовые характеристики отношения сигнал/ин\-тер\-фе\-рен\-ция: \textit{1}~---
${\sf E}[\mathrm{SIR}]$; \textit{2}~--- $\sigma_{\mathrm{SIR}}$}}

\vspace*{18pt}

  Также были рассчитаны математическое ожидание ${\sf E}[\mathrm{SIR}]$
  и~среднеквадратическое \mbox{отклонение} $\sigma_{\mathrm{SIR}}\hm= \sqrt{{\sf E}[\mathrm{SIR}^2]-
{\sf E}[\mathrm{SIR}]^2}$ отношения сигнал/ин\-тер\-фе\-рен\-ция, показанные на рис.~3 в
зави\-си\-мости от математического ожидания расстояния ${\sf E}[U_1]$ между
целевым передатчиком~Tx$_0$ и~интерферирующим передатчиком~Tx$_1$. Из
таблицы и~графиков видно, что с~ростом расстояния между целевым
и~интерферирующим передатчиком обе \mbox{числовые} характеристики отношения
сиг\-нал/ин\-тер\-фе\-рен\-ция растут, поскольку мощность интерферирующего сигнала
убывает. Вычисления проводились с~использованием встроенных средств
пакета программ Wolfram Mathematica~[10].



\section{Заключение}

  В настоящей статье метод преобразования с.в.\ применен для
анализа основной характеристики качества функционирования беспроводных
сетей, а~именно: отношения сигнал/ин\-тер\-фе\-рен\-ция при заданных
распределениях расстояний между интерферирующими устройствами.
Приведенный пример показывает, что чис\-лен\-ный анализ является достаточно
трудоемким даже в простейших предположениях о~распределении исходных
с.в., а~для оценки характеристик интерференции в~условиях
наличия в~беспроводной сети нескольких источников интерференции требуется
разработка приближенных методов и~имитационных моделей, как это сделано,
например, в~[11]. Задача с несколькими источниками интерференции
в~беспроводных гетерогенных сетях взаимодействующих устройств
представляется особенно актуальной ввиду быст\-ро\-го развития сетей 4G
и~принятия в~ближайшем будущем стандартов для беспроводных сетей 5G~[12].
{\looseness=1

}

  \bigskip

Авторы выражают благодарность проф.\ К.\,Е.~Самуйлову за
плодотворное обсуждение и ценные советы.


{\small\frenchspacing
 {%\baselineskip=10.8pt
 \addcontentsline{toc}{section}{References}
 \begin{thebibliography}{99}
\bibitem{1-gai}
\Au{Гайдамака~Ю.\,В., Ефимушкина~Т.\,В., Самуйлов~А.\,К., Самуйлов~К.\,Е.} Задачи
оптимального планирования межуровневого интерфейса в беспроводных сетях~//
Информатика и~её применения, 2012. Т.~6. Вып.~3. С.~75--81.
\bibitem{2-gai}
\Au{Basharin G.\,P., Gaidamaka Yu.\,V., Samouylov~K.\,E.} Mathematical theory of teletraffic and
its application to the analysis of multiservice communication of next generation networks~//
Autom. Control Comp. Sci., 2013. Vol.~47. No.\,2. P.~62--69.
\bibitem{3-gai}
\Au{Andreev S., Pyattaev A., Johnsson~K., Galinina~O., Koucheryavy~Y.} Cellular traffic
offloading onto network-assisted device-to-device connections~// IEEE Commun. Mag.,
2014. Vol.~52. No.\,4. {\sf http://ieeexplore.ieee.org/\linebreak xpl/tocresult.jsp?isnumber=6807935}.
\bibitem{4-gai}
\Au{Baccelli F., Blaszczyszyn B.} Stochastic geometry and wireless networks. Vol.~I: Theory.~---
Boston: NoW Publs. Inc., 2009. 164~p.


\bibitem{6-gai} %5
\Au{Erturk M.\,C., Mukherjee S., Ishii~H., Arslan~H.} Distributions of transmit power and SINR in
device-to-device networks~// IEEE Commun. Lett., 2013. Vol.~17. No.\,2. {\sf
http://ieeexplore.ieee.org/xpl/tocresult.jsp?isnumber=\linebreak 6472443}.

\bibitem{7-gai} %6
\Au{Kim M., Han Y., Yoon~Y., Chong~Y., Lee~H.} Modeling of adjacent channel interference in
heterogeneous wireless networks~// IEEE Commun. Lett., 2013. Vol.~17. No.\,9. {\sf
http://ieeexplore.ieee.org/xpl/tocresult.jsp?isnumber=\linebreak 6604524}.

\bibitem{5-gai} %7
\Au{Andrews J.\,G., Singh S., Ye~Q., Lin~X., Dhillon~H.\,S.} An overview of load balancing in
hetnets: Old myths and open problems~// IEEE Wirel. Commun., 2014. Vol.~21. No.\,2.
{\sf http://ieeexplore.ieee.org/xpl/tocresult.\linebreak jsp?isnumber=6812279}.


\bibitem{8-gai}
\Au{Левин Б.\,Р.} Теоретические основы статистической радиотехники.~--- 3-е изд.~--- М.:
Радио и связь, 1989. 656~с.
\bibitem{9-gai}
\Au{Mardia K., Jupp P.} Directional statistics.~--- Wiley Press, 1999. 441~p.
\bibitem{10-gai}
Wolfram Mathematica: Программное обеспечение для технических вычислений. {\sf
http://www.wolfram.\linebreak com/mathematica}.
\bibitem{11-gai}
\Au{Гайдамака Ю.\,В., Печинкин А.\,В., Разумчик~Р.\,В., Самуйлов~А.\,К., Самуйлов~К.\,Е.,
Соколов~И.\,А., Сопин~Э.\,С., Шоргин~С.\,Я.} Распределение времени выхода из множества
состояний перегрузки в системе $M\vert M\vert 1\vert \langle L,H\rangle \vert \langle
H,R\rangle$ с~гистерезисным управлением нагрузкой~// Информатика и~её применения,
2013. Т.~7. Вып.~4. С.~20--33.
\bibitem{12-gai}
\Au{Tehrani M., Uysal M., Yanikomeroglu~H.} Device-to-device communication in 5G cellular
networks: Challenges, solutions, and future directions~// IEEE Commun. Mag., 2014.
Vol.~52. No.\,5. {\sf http://ieeexplore. ieee.org/xpl/tocresult.jsp?isnumber=6815882}.
 \end{thebibliography}

 }
 }

\end{multicols}

\vspace*{-3pt}

\hfill{\small\textit{Поступила в редакцию 20.01.15}}

%\newpage

\vspace*{12pt}

\hrule

\vspace*{2pt}

\hrule

%\vspace*{12pt}

\def\tit{METHOD FOR CALCULATING NUMERICAL
CHARACTERISTICS OF~TWO DEVICES INTERFERENCE
FOR~DEVICE-TO-DEVICE COMMUNICATIONS
IN~A~WIRELESS HETEROGENEOUS NETWORK}

\def\titkol{Method for calculating numerical
characteristics of~two devices interference
for~D2D communications
in~a~wireless %heterogeneous
network}

\def\aut{Yu.~Gaidamaka$^1$ and A.~Samuylov$^{1,2}$}

\def\autkol{Yu.~Gaidamaka and A.~Samuylov}

\titel{\tit}{\aut}{\autkol}{\titkol}

\vspace*{-9pt}

 \noindent
$^1$Peoples' Friendship University of Russia,
Applied Probability and Informatics Department,
6~Miklukho-Maklaya\linebreak
$\hphantom{^1}$Str., Moscow 117198, Russian Federation

\noindent
$^2$Tampere University of Technology,
Department of Electronics and Communications Engineering,
10 Korkeak-\linebreak
$\hphantom{^1}$oulunkatu,  Tampere 33720, Finland


\def\leftfootline{\small{\textbf{\thepage}
\hfill INFORMATIKA I EE PRIMENENIYA~--- INFORMATICS AND
APPLICATIONS\ \ \ 2015\ \ \ volume~9\ \ \ issue\ 1}
}%
 \def\rightfootline{\small{INFORMATIKA I EE PRIMENENIYA~---
INFORMATICS AND APPLICATIONS\ \ \ 2015\ \ \ volume~9\ \ \ issue\ 1
\hfill \textbf{\thepage}}}

\vspace*{3pt}


\Abste{In wireless networks, one of the key performance metrics is the signal to noise ratio, SINR. As this metric
highly depends on the distance between the interfering devices, the problem of SINR estimation is often reduced to the
calculation of a triangle's side length, where the vertices represent the interacting devices. This paper addresses the
problem of calculating the numerical characteristics of the signal to interference ratio for a pair of interfering devices
determined by the known distributions of distances between the entities in question. The proposed method can be used
as a basis for analyzing heterogeneous networks, including the analysis of
device-to-device (D2D) communications as one of
the interference-limited cases.}

\KWE{wireless network; LTE; interference; SINR; D2D}




\DOI{10.14357/19922264150102}

\Ack
\noindent
The reported study was partially supported by the Russian Foundation for Basic
Research,  research projects Nos.\,14-07-00090 and
15-07-03051.



%\vspace*{3pt}

  \begin{multicols}{2}

\renewcommand{\bibname}{\protect\rmfamily References}
%\renewcommand{\bibname}{\large\protect\rm References}



{\small\frenchspacing
 {%\baselineskip=10.8pt
 \addcontentsline{toc}{section}{References}
 \begin{thebibliography}{99}
\bibitem{1-gai-1}
\Aue{Gaidamaka, Yu.\,V., T.\,V. Efimushkina, A.\,K.~Samuylov, and K.\,E.~Samouylov}. 2012.
Zadachi optimal'nogo planirovaniya mezhurovnevogo interfeysa v besprovodnykh setyakh
[Cross-layer optimization planning problems in wireless networks]. \textit{Informatika i~ee
Primeneniya}~--- \textit{Inform. Appl.} 6(3):75--81.
\bibitem{2-gai-1}
\Aue{Basharin, G.\,P., Yu.\,V. Gaidamaka, and K.\,E.~Samouylov}. 2013. Mathematical theory of
teletraffic and its application to the analysis of multiservice communication of next generation
networks. \textit{Autom. Control Comp. Sci.} 47 (2):62--69.
\bibitem{3-gai-1}
\Aue{Andreev, S., A. Pyattaev, K.~Johnsson, O.~Galinina, and Y.~Koucheryavy}. 2014. Cellular
traffic offloading onto network-assisted device-to-device connections. \textit{IEEE
Commun. Mag.} 52(4). Available at: {\sf
http://ieeexplore.ieee.\linebreak org/xpl/tocresult.jsp?isnumber=6807935} (accessed January~10, 2015).
\bibitem{4-gai-1}
\Aue{Baccelli, F., and B. Blaszczyszyn.} 2009. \textit{Stochastic geometry and wireless networks}.
Vol.~I: Theory. Boston: NoW Publs. Inc. 164~p.



January~10, 2015). %5
\bibitem{6-gai-1}
\Aue{Erturk, M.\,C., S. Mukherjee, H.~Ishii, and H.~Arslan}. 2013. Distributions of transmit
power and SINR in device-to-device networks. \textit{IEEE Commun. Lett.} 17(2).
Available at: {\sf http://ieeexplore.ieee.org/xpl/tocresult.jsp?isnumber=\linebreak 6472443} (accessed
January~10, 2015).
\bibitem{7-gai-1} %6
\Aue{Kim, M., Y. Han, Y.~Yoon, Y.~Chong, and H.~Lee}. 2013. Modeling of adjacent channel
interference in heterogeneous wireless networks. \textit{IEEE Commun. Lett.} 17(9).
Available at: {\sf http://ieeexplore.ieee.org/\linebreak xpl/tocresult.jsp?isnumber=6604524} (accessed
January~10, 2015).

\bibitem{5-gai-1} %7
\Aue{Andrews, J.\,G., S. Singh, Q.~Ye, X.~Lin, and H.\,S.~Dhillon}. 2014. An overview of load
balancing in hetnets: Old myths and open problems. \textit{IEEE Wirel. Commun.} 21(2).
Available at: {\sf http://ieeexplore.ieee.org/\linebreak xpl/tocresult.jsp?isnumber=6812279} (accessed

\bibitem{8-gai-1}
\Aue{Levin, B.\,R.} 1989. \textit{Teoreticheskie osnovy statisticheskoy radiotekhniki} [Theoretical
basis of statistical radiotechnics]. 3rd ed. Moscow: Radio and Communications. 656~p.
\bibitem{9-gai-1}
\Aue{Mardia, K., and P. Jupp}. 1999. \textit{Directional statistics}. 1st ed. Wiley Press. 441~p.
\bibitem{10-gai-1}
Wolfram mathematica: Software for technical computing. [Free access] Available at: {\sf
http://www.wolfram.\linebreak com/mathematica} (accessed December~1, 2014).
\bibitem{11-gai-1}
\Aue{Gaidamaka, Yu.\,V., A.\,V. Pechinkin, R.\,V.~Razumchik, A.\,K.~Samuylov,
K.\,E.~Samouylov, I.\,A.~Sokolov, E.\,S.~Sopin, and S.\,Ya.~Shorgin}. 2013. Raspredelenie
vremeni vykhoda iz mnozhestva sostoyaniy peregruzki v~sisteme $M\vert M\vert 1\vert \langle
L,H\rangle \vert \langle H,R\rangle$ s~gisterezisnym upravleniem nagruzkoy [The distribution of
the return time from the set of overload states to the set of normal load states in a system $M\vert
M\vert 1\vert \langle L,H\rangle \vert \langle H,R\rangle$ with hysteretic load control].
\textit{Informatika i~ee~Primeneniya}~--- \textit{Inform. Appl.} 7(4):20--33.
\bibitem{12-gai-1}
\Aue{Tehrani, M., M. Uysal, and H.~Yanikomeroglu.} 2014. Device-to-device communication in
5G cellular networks: Challenges, solutions, and future directions.
\textit{IEEE Commun. Mag.} 52(5). Available at: {\sf http://ieeexplore.\linebreak ieee.org/xpl/tocresult.jsp?isnumber=6815882}
(accessed January~10, 2015).
\end{thebibliography}

 }
 }

\end{multicols}

\vspace*{-3pt}

\hfill{\small\textit{Received January 20, 2015}}

%\vspace*{-18pt}


\Contr

\noindent
\textbf{Gaidamaka Yuliya V.} (b.\ 1971)~---
Candidate of Science (PhD) in physics and mathematics, associate
professor, Applied Probability and Informatics Department, Peoples' Friendship University of Russia,
6~Miklukho-Maklaya Str., Moscow 117198, Russian Federation;
ygaidamaka@sci.pfu.edu.ru

\vspace*{3pt}

\noindent
\textbf{Samuylov Andrey K.} (b.\ 1988)~---
PhD student, Peoples' Friendship University of Russia, Moscow 117198, Russian
Federation; researcher, Department of Electronics and Communications Engineering,  Tampere
University of Technology, 10 Korkeakoulunkatu, Tampere 33720, Finland;
aksamuylov@gmail.com

\label{end\stat}

\renewcommand{\bibname}{\protect\rm Литература}  %10

\def\stat{agasandyan}

\def\tit{МНОГОМЕРНЫЕ БАТТЕРФЛЯИ\\ В~ЗАДАЧАХ ОПТИМИЗАЦИИ ПО CC-VaR}

\def\titkol{Многомерные баттерфляи в~задачах оптимизации 
по~CC-VaR}

\def\aut{Г.\,А.~Агасандян$^1$}

\def\autkol{Г.\,А.~Агасандян}

\titel{\tit}{\aut}{\autkol}{\titkol}

\index{Агасандян Г.\,А.}
\index{Agasandyan G.\,A.}


%{\renewcommand{\thefootnote}{\fnsymbol{footnote}} \footnotetext[1]
%{Работа выполнена при поддержке Министерства науки и~высшего образования
%Российской федерации, грант №\,075-15-2020-799.}}


\renewcommand{\thefootnote}{\arabic{footnote}}
\footnotetext[1]{Федеральный исследовательский центр <<Информатика и~управление>> Российской 
академии наук, \mbox{agasand17@yandex.ru}}

\vspace*{-6pt}
 
  
  \Abst{Работа продолжает исследование технических проблем, связанных с~применением  
континуального критерия VaR (CC-VaR) на многомерных рынках опционов. 
В~предположении, что на рынке сценарными баттерфляями непосредственно не торгуют, 
разрабатывается методика получения их реп\-ли\-ка\-ции из многомерных $\alpha$-оп\-ци\-онов~--- 
многомерного обобщения обычных одномерных опционов, таких как коллы и~путы. Работа 
служит непосредственным расширением предложенного в~предыдущей работе автора 
способа, позволяющего конструировать индикаторы базиса на многомерном сценарном 
рынке комбинациями многомерных бинарных опционов. Методика основывается на 
теоремах паритета для одномерного рынка традиционных опционов и~пригодна для рынков 
произвольной размерности, но ее фактическая реализация проводится для двумерных 
рынков. Приводятся конструкции базисов из $\alpha$-оп\-ци\-онов~--- как однотипных, так 
и~смешанных естественных с~выделенным цент\-ром рынка. Теоретические пред\-став\-ле\-ния 
оптимальных портфелей в~этих базисах иллюстрируются на примере конкретного 
двумерного рынка.}
   
  \KW{базовые активы; многомерный рынок; функция рисковых предпочтений инвестора; 
континуальный критерий VaR (CC-VaR); стоимостная и~прогнозная плотности; опционы 
колл и~пут; $\alpha$-оп\-ци\-оны; сценарные баттерфляи; базисы; центр рынка; портфели 
баттерфляев}

 \DOI{10.14357/19922264230114} 
  
\vspace*{-2pt}


\vskip 10pt plus 9pt minus 6pt

\thispagestyle{headings}

\begin{multicols}{2}

\label{st\stat}
   
  \section{Введение}
  
  Проблемы применения на рынках опционов введенного автором 
континуального критерия VaR (CC-VaR) рассматриваются в~[1--5]. Настоящую 
работу можно рассматривать как продолжение исследования~[6], в~котором 
предлагались варианты репликации индикаторов базиса на многомерном 
сценарном рынке комбинациями так называемых $\zeta$-оп\-ци\-онов 
(многомерных бинарных опционов).\linebreak Здесь подобная задача решается для более 
сложных инструментов~--- многомерных аналогов одномерных базисных 
баттерфляев, которые реплицируются комбинациями так называемых
  $\alpha$-оп\-ци\-онов~--- многомерных аналогов традиционных \mbox{опционов} типа 
колл и~пут. 
  
  Основания для такого рассмотрения и~его проб\-ле\-мы, связанные 
с~применением континуального критерия VaR (CC-VaR), приведены в~[6], там 
же вводятся многие обозначения, которые используются и~здесь. 
Теоретической моделью при построении $\alpha$-рын\-ка служит также 
многомерный $\delta$-ры\-нок~\cite{5-aga, 6-aga}. 
  
  В работе для многомерных рынков опционов решаются те же проблемы 
технического характера, что и~для $\zeta$-рын\-ков~--- рынков  многомерных 
бинарных опционов. Но на этот раз в~отношении своего инструментария они 
в~большей мере напоминают проб\-ле\-мы традиционных рынков опционов, на 
которых в~отсутствие баттерфляев в~качестве объектов непосредственной 
торговли предлагается получать их в~виде комбинаций коллов и~путов. 
  
  \section{Теоретический $\alpha$-рынок и~его~свойства}
  
  Вновь рассматривается многомерный $\delta$-ры\-нок (однопериодный, 
теоретический и~идеальный)\linebreak с~$n$ ($>1$) базовыми активами, векторы цен 
которых в~конце периода $\bm{x}\hm= (x_1, x_2, \ldots, x_n)$, $x_l\hm\in {\sf X}_l \hm\subset 
\mathfrak{R}$, $l\hm\in N\hm=\{1, \ldots , n\}$, образуют $n$-мер\-ное множество 
${\mathsf X}\hm=\prod_{l\in N} {\mathsf X}_l$. На~${\mathsf X}$ заданы 
\textit{прогнозная} $p(\bm{x})$ и~\textit{стоимостная} $c(\bm{x})$ плотности, 
по\-рож\-да\-ющие вероятностные меры~${\mathsf P}\{\cdot\}$ и~${\mathsf 
C}\{\cdot\}$. 
  
  Платежная функция произвольного инструмента~$\bm{I}$ обозначается 
$\pi(\bm{x}; \bm{I})$, его рыночная сто\-и\-мость и~средний, с~точки зрения 
инвестора, доход, рас\-счи\-тан\-ные по плотностям $c(\bm{x})$ и~$p(\bm{x})$ 
соответственно, определяются соотношениями: 
  $$
  \vert \bm{I}\vert =\int\limits_{\mathsf X} \pi (\bm{x};\bm{I}) 
c(\bm{x})\,d\bm{x}\,;\enskip
  \left\| \bm{I}\right\| =\int\limits_{\mathsf X} \pi(\bm{x};\bm{I}) 
p(\bm{x})\,d\bm{x}\,.
  $$
  
  Базис рынка составляют $\delta$-ин\-стру\-мен\-ты $\bm{D}(\bm{s})$, 
$\bm{s}\hm\in {\mathsf X}$, с~обобщенной $n$-мер\-ной $\delta$-функ\-ци\-ей 
относительно~$\bm{s}$ в~качестве платежной: 

\noindent
  \begin{multline}
    \bm{D}(\bm{s}) =\prod\limits_{l\in N} \bm{D}_l(s_l)\,,\\
  \pi(\bm{x};\bm{D}(\bm{s}))=\delta(\bm{x}-\bm{s})=\prod\limits_{l\in N} 
\delta(x_l-s_l).
  \label{1-aga}
  \end{multline}
  
  Инструмент $\bm{G}$ с~произвольной измеримой платежной 
функцией~$g(\bm{x})$ и~его стоимость имеют вид: 
  \begin{align*}
  \bm{G}&= \int\limits_{\mathsf X} g(\bm{s}) \bm{D}(\bm{s})\,d\bm{s}\,;\\
  \vert \bm{G}\vert &=\int\limits_{\mathsf X} g(\bm{s}) \vert \bm{D}(\bm{s}) \vert 
\,d\bm{s}= \int\limits_{\mathsf X} g(\bm{s}) c(\bm{s}) \,d\bm{s}\,.
  \end{align*}
  
  Наряду с~<<полноправными>> $n$-мер\-ны\-ми инструментами на рынке 
присутствуют и~их $k$-мер\-ные версии, у~которых $n\hm- k$ координатных 
базовых активов пред\-став\-ле\-ны в~форме одномерных единичных без\-рис\-ко\-вых 
инструментов. 
  
  Для индикаторов $\bm{H}\{M\}$, $M\hm\subset {\mathsf X}$, без\-рис\-ко\-во\-го 
актива $\bm{U}\hm=\bm{H}\{ {\mathsf X}\}$ и~их цен 
  \begin{gather*}
  \bm{H}\{ M\}=\int\limits_M \bm{D}(\bm{s})\, d\bm{s}\,;\enskip
   \vert \bm{H}\{M\}\vert =\int\limits_M c(\bm{s})\,d\bm{s}\,;\\
   \vert \bm{U}\vert ={\mathsf C}\{ {\mathsf X}\}= \int\limits_{\mathsf X} c(\bm{s}) 
\,d\bm{s}=\fr{1}{r}\,,
   \end{gather*}
где $r$~--- приравниваемый единице безрисковый доход за период. 
  
  На \textit{одномерном} рынке опционы пут~$\bm{P}_s$ и~колл~$\bm{C}_s$ 
со страйком~$s$ задаются своими платежными функциями: 
  \begin{equation}
  \left.
  \begin{array}{rl}
  \pi(x;\bm{P}_s)&=\max (0,s-x);\\[6pt]
  \pi(x;{\bm C}_x)&=\max (0,x-s),\ x,s \in {\mathsf X}\subset \mathfrak{R}\,.
  \end{array}
  \right\}
  \label{e2-aga}
  \end{equation}
  %
  Для них выполняется формула паритета ($\bm{X}$~--- вектор базовых 
активов)
  $$
  \bm{C}_s-\bm{P}_s= \bm{X}-s\bm{U}\,.
  $$
  
  Нормированными спрэдами быка с~парой страйков $s\hm-h$, $s \hm\in 
{\mathsf X}$ ($h \hm>0$)  и~медведя с~парой страйков~$s$, $s\hm+h \hm\in 
{\mathsf X}$ служат комбинации опционов соответственно 
  \begin{equation}
  \left.
  \begin{array}{rl}
 \!\!\!\! \bm{S}_{s;h}^{\mathrm{bull}} &= \fr{1}{h}\left( \bm{C}_{s-h}-\bm{C}_s\right) =
\bm{U}+\fr{1}{h}\left( \bm{P}_{s-h}-\bm{P}_s\right)\,;\\[6pt]
 \!\! \!\! \bm{S}_{s;h}^{\mathrm{bear}} &=\fr{1}{h}\left( \bm{P}_{s+h}-\bm{P}_s\right) 
=\bm{U}+\fr{1}{h} \left( \bm{C}_{s+h}-\bm{C}_s\right)
  \end{array}\!
  \right\}\!\!
  \label{e3-aga}
  \end{equation}
с платежными функциями 
\begin{equation}
\left.
\begin{array}{rl}
 \!\!\!\!\pi\left( x;\bm{S}_{s;h}^{\mathrm{bull}}\right) &= \min \left(\! 1,\fr{1}{h}\max (0,x-(s-h))\!\right);\\[9pt]
\! \!\!\!\pi\left( x;\bm{S}_{s;h}^{\mathrm{bear}}\right) &= \min \left(\! 1,\fr{1}{h}\max (0, (s+h)-x)\!\right).
\end{array}\!
\right\}\!
\label{e4-aga}
\end{equation}
  
  Нормированные симметричные баттерфляи с~тройкой страйков $s\hm-h$, $s$, 
$s\hm+h \hm\in {\mathsf X}$ образуются комбинациями 
  \begin{multline}
  \bm{B}_{s;h} = \fr{1}{h}\left( \bm{C}_{s-h} -2\bm{C}_s +\bm{C}_{s+h}\right) 
={}\\
  {}= \fr{1}{h} \left( \bm{P}_{s-h} -2\bm{P}_s+\bm{P}_{s+h}\right) 
={}\\
{}=\bm{U}+\fr{1}{h}\left( \bm{P}_{s-h} -\bm{P}_s -
\bm{C}_s+\bm{C}_{s+h}\right)
  \label{e5-aga}
  \end{multline}
с платежными функциями 
\begin{equation}
\pi \left( x;\bm{B}_{s;h}\right) =\fr{1}{h}\max ( 0, h-\vert x-s\vert).
\label{e6-aga}
\end{equation}
  
  В комбинациях~(\ref{e3-aga}) и~(\ref{e5-aga}) безрисковый 
инструмент~$\bm{U}$ выполняет функцию маржевого инструмента 
и~применяется инвестором в~соответствии с~требованиями рынка. Формально 
верно еще одно пред\-став\-ле\-ние:
  \begin{equation*}
  \bm{B}_{s;h}=\fr{1}{h}\left( \bm{C}_{s-h} -\bm{C}_s -\bm{P}_s 
+\bm{P}_{s+h}\right)\,,
 % \label{e7-aga}
  \end{equation*}
но оно не является \textit{естественным} (страйки коллов в~комбинации ниже 
страйков путов) и~потому далее не используется. 
  
  Можно было бы рассматривать и~не создающие принципиальных трудностей 
несимметричные баттерфляи (с~неравными по длине сценариями 
и~неравномерной линейкой страйков), но они, как правило, не применяются на 
рынках и~к~тому же сильно загромождали бы изложение. 
  
  С целью алгоритмической автоматизации дальнейших построений для 
одномерных опционов $\bm{P}_s$ и~$\bm{C}_s$ вводятся также обозначения 
$\bm{O}_s^-$ (и~$\bm{O}_{0;s}$) и~$\bm{O}_s^+$ (и~$\bm{O}_{1;s})$, которые 
могут обрастать дополнительными индексами координат $l\hm\in N$: 
  \begin{equation}
  \bm{O}_{0;s}\equiv \bm{O}_s^- \equiv \bm{P}_s\,;\enskip
  \bm{O}_{1;s} \equiv \bm{O}_s^+\equiv \bm{C}_s\,.
  \label{e8-aga}
  \end{equation}
  
  \textit{Многомерным} обобщением одномерных опционов $\bm{P}_s$ 
и~$\bm{C}_s$ служат $n$-мер\-ные $\alpha$-\textit{оп\-ци\-оны} 
$\bm{A}_{\bm{\alpha};\bm{s}}$ векторного типа~$\bm{\alpha}$ и~с~векторным 
страйком $\bm{s}\hm \in \mathfrak{R}^n$, задаваемые вместе с~платежными 
функциями соотношениями 
  \begin{multline}
  {A}_{\alpha;s}=\prod\limits_{i\in N} \bm{O}_{i\beta_i;s_i}\,,\\
  \pi\left(\bm{x}, A_{\alpha;s}\right)=
  \prod_{l\in N}\pi \left(x_l; \bm{O}_{l\beta_l;s_l}\right),\ \bm{x}\in 
\mathfrak{R}^n\,,\\[6pt]
  \pi\left( x_l; \bm{O}_{l\beta_l;s_l}\right) =\omega_{l\beta_l;s_l}(x_l)={}\\[6pt]
  \hspace*{10mm}{}=\max  \left(0,\alpha_l (x_l-s_l)\right), \enskip l\in N\,.
    \label{e9-aga}
  \end{multline}
  
  Как и~в~[6], вектор~$\bm{\alpha}$ с~компонентами $\alpha_l\hm= \pm1$, $l\hm\in N$, 
в~индексах инструментов (или просто~$\pm$) определяет векторный тип  
$\alpha$-оп\-ци\-онов~$\bm{A}_{\bm{\alpha};\bm{s}}$. Вектор $\bm{\beta}\hm= 
(\bm{\alpha}\hm+1)/2$, дублирующий~$\bm{\alpha}$, вводится для удобства по 
техническим причинам и~принимает для каждого $l\hm\in N$ два значения: 
$$
\beta_l= \begin{cases}
0 &\mbox{для\ пута;}\\ 
1 & \mbox{для\ колла.}
\end{cases}
$$ 
  
  Для каждого векторного страйка~$\bm{s}$ на $n$-мер\-ном рынке могут 
котироваться $2^n$ типов $\alpha$-оп\-ци\-онов. Рынок $n$-мер\-ных  
$\alpha$-оп\-ци\-онов с~их $k$-мер\-ны\-ми версиями, $k\hm< n$, называется  
$n$-мер\-ным $\alpha$-\textit{рын\-ком}. 
  
  \section{Двумерный дискретный $\alpha$-рынок}
   
  В основе дискретного $\alpha$-рын\-ка лежит \textit{сценарный} рынок~--- 
сценарная дискретизация двумерного тео\-ре\-ти\-че\-ско\-го $\delta$-рын\-ка. Как 
и~для бинарного рынка, используется в~большей мере адаптированная 
к~двумерному случаю очевидная сис\-те\-ма обозначений, но учитывается 
и~специфика требований $\alpha$-рынка. 
  
  Цены двух базовых активов \textit{теоретического} двумерного  
$\delta$-рын\-ка обозначаются~$x$ и~$y$, страйки опционов~--- 
соответственно~$s$ и~$t$, $x, s \hm\in {\mathsf X} \hm=[a_1,b_1)\hm\subset 
\mathfrak{R}$, $y,t \hm\in {\mathsf Y}\hm=[a_2,b_2) \hm\subset \mathfrak{R}$. 
Дискретизация осуществляется равномерным разбиением множества~${\mathsf 
X}$ на~$v_1$ интервалов (сценариев), ${\mathsf Y}$~--- на $v_2$ интервалов. 
Одномерные сценарии на~${\mathsf X}$ и~${\mathsf Y}$ даются формулами: 
  \begin{multline}
  S_i= \left[ x_{i-1},x_i\right),\ x_i=a_1+ih_1,\ h_1=\fr{b_1-a_1}{v_1},\\
   i\in  \bm{I},\ x_0=a_1; \label{e10-aga}
   \end{multline}
   
   \vspace*{-12pt}
   
   \noindent
   \begin{multline}
  T_j= \left [ y_{j-1},y_j\right),\ y_j= a_2+jh_2,\ h_2=\fr{b_2-a_2}{v_2},\\ j\in 
\bm{J},\ y_0=a_2\,,
  \label{e11-aga}
\end{multline}
где $\bm{I}=\{1,2, \ldots, v_1\}$, $\bm{J}\hm=\{1,2,\ldots , v_2\}$, а~номер 
сценария совпадает с~индексом его правой границы. Двумерными сценариями 
служат прямые произведения всех пар $S_i\times T_j$, $i\hm\in \bm{I}$, $j\hm\in 
\bm{J}$. 
  
  На сценарном рынке базис образуют индикаторы сценариев 
$\bm{D}_{ij}\hm=\bm{H}\{S_i\times T_j\}$, но для $\alpha$-рын\-ка при той же структуре 
сценариев уместнее использовать иной базис~--- из баттерфляев~$\bm{B}_{ij}$, 
задаваемых с~учетом определения~(\ref{e5-aga}), но для специально 
подобранных страйков. Страйками~$s_i$ и~$t_j$ одномерных опционов 
и~упомянутых баттерфляев служат середины сценариев~(\ref{e10-aga}) 
и~(\ref{e11-aga}): 
  $$
  s_i = \fr{x_{i-1} + x_i}{2}\,,\  i\in \bm{I}; \enskip   t_j = \fr{y_{j-1} + y_j}{2},\   j\in 
\bm{J}\,. 
  $$
  %
  При этом параметр~$h$ для баттерфляев~(\ref{e5-aga}), равный длине 
сценариев, определяется в~(\ref{e10-aga}) и~\ref{e11-aga}). Для удобства 
записи формул также доопределяются параметры $s_0 \hm= a_1$, $s_{v_1+1}\hm = 
b_1$, $t_0 \hm= a_2$, $t_{v_2+1}\hm = b_2$, но они страйками не являются. 
  
  Портфель с~вектором~$\bm{g}$ весов базисных баттерфляев в~двумерном 
случае приобретает вид: 
  \begin{equation}
  \bm{G}= \sum\limits_{i\in \bm{I}, j\in \bm{J}} g_{ij} \bm{B}_{ij}\,.
  \label{e12-aga}
  \end{equation}
  
  Двумерным обобщением обычных опционов служат инструменты, 
характеризуемые парой страйков $(s_i,t_j)$, или просто $(i,j)$, 
с~дополнительным указанием типа (лучше в~терминах $\bm{\beta}\hm=(\beta_1, 
\beta_2))$: 
\begin{multline*}
\bm{A}_{\beta_1\beta_2;ij} 
=\bm{O}_{\beta_1;1,i}\bm{O}_{\beta_2;2,j}=\bm{O}_{\beta_1;i\cdot} 
\bm{O}_{\beta_2;\cdot j}\,,\\ 
i\in \bm{I}\,,\ j\in \bm{J}\,.
\end{multline*}
  %
  Также рассматриваются и~их одномерные версии, обозначаемые~$\bm{A}_{i\cdot}$ 
и~$\bm{A}_{\cdot j}$ с~маркером <<точка>> в~позиции, отведенной координате 
безрискового актива. 
  
  Для представления произвольного инструмента~$\bm{G}$~(\ref{e12-aga}) 
в~базисе из $\alpha$-оп\-ци\-онов (избыточным в~сравнении с~базисом из 
баттерфляев) все нормированные баттерфляи в~(\ref{e12-aga}) следует 
реплицировать в~терминах $\alpha$-оп\-ци\-онов. 
  
  В соответствии с~(\ref{e9-aga}) для конструирования репликаций следует 
перемножать одномерные представления сценарных баттерфляев, выбирая 
подходящие сомножители из~(\ref{e3-aga}) и~(\ref{e5-aga}). Для одномерного 
рынка с~$v$~сценариями $i\hm\in \bm{I}$ справедливы такие репликации 
баттерфляев коллами и~путами:
  \begin{multline}
  \bm{B}_i={}\\
  \!\!\!\!{}=\begin{cases}
  \bm{U}-\fr{\bm{O}_1^+ -\bm{O}_2^+}{h}=\fr{\bm{O}_2^- -\bm{O}_1^-}{h}\,, & i=1;\\
  \fr{\bm{O}_{i-1}^+-2\bm{O}_i^+ +\bm{O}_{i+1}^+}{h}={}&\\
  \hspace*{7mm}{}= \fr{\bm{O}^-_{i-1}-2\bm{O}_i^- +\bm{O}^-_{i+1}}{h}={}&\\
  \hspace*{12mm}{}= \bm{U}-\fr{\bm{O}_i^- -\bm{O}^-_{i-1}}{h} -{}&\\
  \hspace*{15mm}{}- \fr{\bm{O}_i^+  - \bm{O}^+_{i+1}}{h}\,, &\hspace*{-7mm} 1<i<v\,;\\
  \fr{\bm{O}^+_{v-1} -\bm{O}_v^+}{h}=\bm{U}-\fr{\bm{O}_v^- -\bm{O}^-_{v-
1}}{h}\,, & i=v\,.
  \end{cases}\!
  \label{e13-aga}
  \end{multline}
  
  Базисные инструменты для $i\hm=\overline{1,v}$ являются спрэдами, но их для 
удобства также называем баттерфляями (\textit{усеченными}). 
Инструмент~$\bm{U}$ в~выписанных соотношениях, как в~(\ref{e3-aga}) 
и~(\ref{e5-aga}),  выполняет функцию маржевого инструмента. 
  
  Подобно сценарным базисам для $\zeta$-рынка~\cite{6-aga} построение 
двумерных базисов из $\alpha$-оп\-ци\-онов проводится на основе одномерных 
базисов, но их элементы на этот раз выбираются из~(\ref{e13-aga}). Строятся 
три варианта репликации базисов: два однотипных (один в~путах, другой 
в~коллах) и~третий~--- смешанный естественный. Если в~базисе $v$~сценариев, 
а~центральный страйк~$i_c$, то 
  \begin{itemize}
\item однотипный базис при $\alpha\hm=-1$ (в~путах):
\begin{equation}
\left.
\begin{array}{rl}
\bm{B}_1^- &=\fr{\bm{O}_2^*-\bm{O}_1^-}{h}\,;\\[6pt]
  \bm{B}_i^- &= \fr{\bm{O}^-_{i-1} -2\bm{O}_i^- +\bm{O}^-_{i+1}}{h}\,;\\[6pt]
   \bm{B}^-_v &=\bm{U}- \fr{\bm{O}_v^- -\bm{O}^-_{v-1}}{h}\,;
   \end{array}
   \right\}
\label{e14-aga}
\end{equation}
\item однотипный базис при $\alpha\hm=+1$ (в~коллах):
\begin{equation}
\left.
\begin{array}{rl}
\bm{B}_1^+ &\equiv \bm{U} - \fr{\bm{O}_1^+ -\bm{O}_2^+}{h}\,;\\[6pt]
  \bm{B}_i^+ &= \fr{\bm{O}^+_{i-1} -2\bm{O}_i^+ +\bm{O}^+_{i+1}}{h}\,;\\[6pt]
    \bm{B}_v^+ &\equiv \fr{\bm{O}^+_{v-1} -\bm{O}_v^+}{h}\,;
    \end{array}
    \right\}
\label{e15-aga}
\end{equation}
\item смешанный естественный базис:
\begin{equation}
\left.
\begin{array}{l}
\!\!\!\bm{B}_1^m\equiv \fr{\bm{O}_2^- -  \bm{O}_1^-}{h}\,;\\[6pt]
  \!\!\!\bm{B}_i^m \equiv \fr{\bm{O}^-_{i-1} -2\bm{O}_i^- +\bm{O}^-_{i+1}}{h}\,,\ 0< i< i_c;\\[6pt]
\!\!\!\bm{B}^m_{i_c} \equiv  \bm{U}-\fr{\bm{O}^-_{i_c-1} -\bm{O}^-_{i_c} -
\bm{O}^+_{i_c} +\bm{O}^+_{i_c+1}}{h}\,;\\[6pt]
\!\!\!\bm{B}_i^m\equiv \fr{\bm{O}^+_{i-1} -2\bm{O}_i^++\bm{O}^+_{i+1}}{h_i}\,,\ i_c<i<v\,;\\[6pt]
 \!\!\!\bm{B}_v^m\equiv \fr{\bm{O}^+_{v-1} -\bm{O}_v^+}{h}\,.
\end{array}\!
\right\}\!
\label{e16-aga}
\end{equation}
  \end{itemize}
  
  \section{Формирование базисов и~платежных функций 
портфелей $\alpha$-опционов}
  
  На основе соотношений~(\ref{e14-aga})--(\ref{e16-aga}) введенные 
  в~многомерном случае произвольной размерности конструкции здесь 
переписываются для двумерного  
$\alpha$-рын\-ка в~однотипных и~смешанных вариантах. 
  
  Поскольку каждый двумерный базисный баттерфляй определяется как 
произведение двух одномерных (что соответствует перемножению платежных 
функций), его репликации двумерными\linebreak $\alpha$-оп\-ци\-она\-ми находятся 
перемножением пары подходящих представлений  
из~(\ref{e14-aga})--(\ref{e16-aga}). 
  
  \textit{Однотипная} репликация сценарных баттерфляев проводится 
  $\alpha$-оп\-ци\-она\-ми единого типа $\bm{\alpha}\hm=\{\alpha_1, \alpha_2\}$. Он фиксируется 
заранее, и~потому используются более простые соотношения~(\ref{e14-aga}) 
и~(\ref{e15-aga}), а~обозначение типа опциона опускается. 
  
  Каждое перемножение сумм одномерных опционов в~(\ref{e14-aga}) 
или~(\ref{e15-aga}) дает сумму парных произведений этих опционов, которые 
затем следует замещать согласно~(\ref{e9-aga}) эквивалентными двумерными 
$\alpha$-оп\-ци\-она\-ми по правилам 
  \begin{multline}
  1\to \bm{U}\,,\enskip \bm{O}_{1,i}\bm{O}_{2,j}\to \bm{A}_{ij}\,,\\ 
\bm{O}_{1,i}\bm{U}_2\to \bm{A}_{i\cdot}\,,\enskip \bm{U}_1 \bm{O}_{2,j}\to 
\bm{A}_{\cdot j}\,.
  \label{e17-aga}
  \end{multline}
  
  \textit{Смешанная} репликация осуществляется аналогично, но указание типа в~обозначениях необходимо, и~потому правила трансформации приобретают 
вид: 
  \begin{multline}
  1\to \bm{U}\,,\ \bm{O}_{1,i}^{\alpha_1} \bm{O}_{2,j}^{\alpha_2} \to 
\bm{A}_{ij}^{\bm {\alpha}}=\bm{A}_{\beta_1,\beta_2;ij},\\
\bm{O}_{1,i}^{\alpha_1}\to \bm{A}_{\beta_1;i,\cdot} \left( = 
\bm{A}^{\alpha_1}_{i\cdot}\right)\,,\ \bm{O}^{\alpha_2}_{2,j} \to 
\bm{A}_{\beta_2;\cdot,j}\left( =\bm{A}^{\alpha_2}_{\cdot j}\right),\\
 \beta_1,\beta_2\in  \{0,1\}\,,
\label{e18-aga}
\end{multline}
  
  На двумерном $\alpha$-рын\-ке в~соответствии с~чис\-лом возможных 
векторов~$\bm{\alpha}$ насчитываются четыре варианта однотипных базисов 
и~один смешанный (естественный с~заданным центром рынка). 
  
  Для каждого варианта с~\textit{однотипным} базисом и~оптимальным 
портфелем фиксируется тип~$\bm{\alpha}$, и~он становится типом всех  
$\alpha$-оп\-ци\-онов варианта. В~двумерном случае таких типов четыре: $\{-1, -
1\}$; $\{-1,+1\}$; $\{+1,-1\}$; $\{+1,+1\}$. Последовательным применением 
правил~(\ref{e17-aga}) ко всем страйкам для каж\-до\-го значения векторного 
параметра~$\bm{\alpha}$ находятся искомые четыре базиса. В~однотипном случае 
для каждой компоненты рынка наличествуют три качественно различных 
представления по варианту страйка~--- двум крайним и~общему внутреннему, 
и~потому их $3^2\hm=9$. В~качестве примера приводится базис для 
$\bm{\alpha}\hm=\{-1,+1\}$, т.\,е.\ в~терминах $\alpha$-оп\-ци\-онов~$\bm{A}_{01}$ 
(остальные три \textit{однотипных} базиса выписываются сходным образом), 
при этом в~списке принимается $0\hm<i \hm< v_1$, $0 \hm< j \hm< v_2$: 
  \begin{align*}
  \bm{B}_{1,1}&=\fr{\bm{A}_{1,1}- \bm{A}_{1,2} -
\bm{A}_{2,1}+\bm{A}_{2,2}}{h_1h_2}+{}\\
&\hspace*{35mm}{}+ \fr{-\bm{A}_{1,\cdot} +\bm{A}_{2,\cdot}}{h_1}\,;\\
  \bm{B}_{1,j} &= \fr{-\bm{A}_{1,j-1} +2\bm{A}_{1,j} -\bm{A}_{1,j+1}}{h_1h_2}+{}\\
  &  \hspace*{15mm} {}+
\fr{\bm{A}_{2,j-1} -2\bm{A}_{2,j} +\bm{A}_{2,j+1}}{h_1h_2}\,;\\
   \bm{B}_{1,v_2}&= \fr{-\bm{A}_{1,v_2-1} +\bm{A}_{1,v_2} +\bm{A}_{2,v_2-1}- \bm{A}_{2,v_2}}{h_1h_2}\,;
  \end{align*}
  
\noindent
  \begin{align*}
   \bm{B}_{i,1}&= \fr{ -\bm{A}_{i-1,1} +\bm{A}_{i-1,2} +2\bm{A}_{i,1}}{h_1h_2}+{}\\
  &\hspace*{15mm}{}+   \fr{ -2\bm{A}_{i,2} -\bm{A}_{i+1,1} +\bm{A}_{i+1,2}}{h_1h_2}+{}\\
  &\hspace*{25mm}{}+ \fr{ \bm{A}_{i-1,\cdot } - 2\bm{A}_{i,\cdot} +\bm{A}_{i+1,\cdot}}{h_1}\,;\\
  \bm{B}_{i,j} &=\fr{\bm{A}_{i-1,j-1} -2\bm{A}_{i-1,j} +\bm{A}_{i-1,j+1}}{h_1h_2}+{}\\
  &\hspace*{11mm}{}+ \fr{-2\bm{A}_{i,j-1} +4\bm{A}_{i,j} -2\bm{A}_{i,j+1}}{h_1h_2} +{}\\
 &\hspace*{13mm}{}+  \fr{\bm{A}_{i+1,j-1} -2\bm{A}_{i+1,j}+\bm{A}_{i+1,j+1}}{h_1h_2}\,;\\
  \bm{B}_{i,v_2}&= \fr{\bm{A}_{i-1,v_2-1} -\bm{A}_{i-1,v_2} -2\bm{A}_{i,v_2-1}}{h_1h_2} +{}\\
 & \hspace*{16mm}{}+\fr{2\bm{A}_{i,v_2} +\bm{A}_{i+1,v_2-1} -\bm{A}_{i+1,v_2}}{h_1h_2}\,;\\
  \bm{B}_{v_1,1} &=\bm{U} +\fr{ -\bm{A}_{\cdot,1} +\bm{A}_{\cdot, 2}}{h_2}+ {}\\
 & \hspace*{2mm}{}+\fr{-\bm{A}_{v_1-1,1} +\bm{A}_{v_1-1,2} +\bm{A}_{v_1,1} -\bm{A}_{v_1,2}}{h_1h_2} +{}\\
&\hspace*{37mm}{}+\fr{\bm{A}_{v_1-1,\cdot}- \bm{A}_{v_1,\cdot}}{h_1}\,;\\
  \bm{B}_{v_1,j}&= \fr{\bm{A}_{\cdot,j-1}- 2\bm{A}_{\cdot,j} +\bm{A}_{\cdot,j+1}}{h_2} +{}\\
&\hspace*{2mm}{}+\fr{\bm{A}_{v_1-1,j-1}-2\bm{A}_{v_1-1,j}+\bm{A}_{v_1-1,j+1}}{h_1h_2}+{}\\
&\hspace*{15mm}{}+\fr{ -\bm{A}_{v_1,j-1}+2\bm{A}_{v_1,j}-\bm{A}_{v_1,j+1}}{h_1h_1}\,;\\
  \bm{B}_{v_1,v_2}&= \fr{\bm{A}_{\cdot,v_2-1}- \bm{A}_{\cdot,v_2}}{h_2} 
+{}\\
&\hspace*{-7mm}{}+\fr{\bm{A}_{v_1-1, v_2-1}- \bm{A}_{v_1-1,v_2} - \bm{A}_{v_1,v_2-1} 
+\bm{A}_{v_1,v_2}}{h_1h_2}\,.
  \end{align*}
    Здесь в~индексах опционов маркер <<точка>> отмечает координату 
безрискового актива, а~под~$\bm{A}_{i,\cdot}$ и~$\bm{A}_{\cdot,j}$, как уже обсуждалось 
выше, понимаются двумерные инструменты $\bm{A}_i\times \bm{U}_2$ 
и~$\bm{U}_1\times \bm{A}_j$ соответственно. 
  
  \textit{Смешанный} базис состоит из $5^2\hm=25$ качественно различных 
вариантов представления базисных инструментов, поскольку для каждой 
компоненты рынка вариантов страйка пять: два крайних, один центральный 
и~два внутренних, ниже и~выше центра. Их перечень получается применением 
правил~(\ref{e16-aga}). Приводим лишь часть базиса, связанную с~первым по 
отношению к~центру рынка квадрантом, т.\,е.\ для $1 \hm\leq i \hm\leq i_c$, $1 
\hm\leq j \hm\leq j_c$ (прочие части образуются аналогично):
  \begin{align*}
  \bm{B}_{1,1}&=\fr{\bm{A}_{00;1,1} - \bm{A}_{00;1,2} - \bm{A}_{00;2,1} + 
\bm{A}_{00;2,2}}{h_1h_2}\,; 
\end{align*}

  \noindent
  \begin{align*}
  \bm{B}_{1,j}&=\fr{-\bm{A}_{00;1,j-1} + 2\bm{A}_{00;1,j} - \bm{A}_{00;1,j+1}}{h_1h_2} + {}\\
&\hspace*{-3mm}{}+
\fr{\bm{A}_{00;2,j-1} - 2\bm{A}_{00;2,j} + \bm{A}_{00;2,j+1}}{h_1h_2}\,,\enskip 0<j<j_c\,; \\
  \bm{B}_{1,j_c}&=\fr{- \bm{A}_{00;1,j_c-1} + \bm{A}_{00;1,j_c} + \bm{A}_{00;2,j_c - 1}}{h_1h_2} +{}\\
  &\hspace*{-3mm}{}+  
  \fr{- \bm{A}_{00;2,j_c} + \bm{A}_{01;1,j_c} - \bm{A}_{01;1,j_c+1} - \bm{A}_{01;2,j_c} }{h_1h_2}+{}\\
&\hspace*{21mm}{}+ \fr{\bm{A}_{01;2,j_c+1}}{h_1h_2} + \fr{- \bm{A}_{0;1,\cdot} + \bm{A}_{0;2,\cdot}}{h_1}\,; \\
  \bm{B}_{i,1}&=\fr{-\bm{A}_{00;i-1,1} + \bm{A}_{00;i-1,2} + 2\bm{A}_{00;i,1}}{h_1h_2}+{}\\
  &\hspace*{-4mm}{}+ \fr{ -  2\bm{A}_{00;i,2} - \bm{A}_{00;i+1,1} + \bm{A}_{00;i+1,2}}{h_1h_2}\,,\enskip   0<i<i_c\,; \\
  \bm{B}_{i,j}&= \fr{\bm{A}_{00;i-1,j-1} - 2\bm{A}_{00;i-1,j} + \bm{A}_{00;i-1,j+1}}{h_1h_2}+{}\\
  &\hspace*{2mm}{}+ \fr{- 2\bm{A}_{00;i,j-1} + 4\bm{A}_{00;i,j} - 2\bm{A}_{00;i,j+1}}{h_1h_2} +{}\\
&\hspace*{3mm}{}+\fr{\bm{A}_{00;i+1,j-1} - 2\bm{A}_{00;i+1,j} + \bm{A}_{00;i+1,j+1}}{h_1h_2}\,,\\
    & \hspace*{35mm}0<i<i_c,\enskip  0<j<j_c\,; \\
  \bm{B}_{i,j_c}&=\fr{\bm{A}_{00;i-1,j_c-1} - \bm{A}_{00;i-1,j_c} - 2\bm{A}_{00;i,j_c-1}}{h_1h_2} +{} \\
 &\hspace*{1mm}{}+\fr{2\bm{A}_{00;i,j_c} + \bm{A}_{00;i+1,j_c-1} - \bm{A}_{00;i+1,j_c}}{h_1h_2}+{}\\
 &\hspace*{2mm}{}+ \fr{ -\bm{A}_{01;i-1,j_c} + \bm{A}_{01;i-1,j_c+1} + 2\bm{A}_{01;i,j_c}}{h_1h_2}+{}\\
& {}+\fr{ - 2\bm{A}_{01;i,j_c+1} - \bm{A}_{01;i+1,j_c} + \bm{A}_{01;i+1,j_c+1}}{h_1h_2} +{}\\
&\hspace*{4mm}{}+ \fr{\bm{A}_{0;i-1,\cdot} - 2\bm{A}_{0;i,\cdot} + \bm{A}_{0;i+1,\cdot}}{h_1},\enskip    0<i<i_c\,; \\
  \bm{B}_{i_c,1}&=\fr{-\bm{A}_{00;i_c-1,1} + \bm{A}_{00;i_c-1,2} + \bm{A}_{00;i_c,1}}{h_1h_2}+{}\\
  &\hspace*{5mm}{}+ \fr{ -\bm{A}_{00; i_c,2} + \bm{A}_{10;i_c,1} - \bm{A}_{10;i_c,2}}{h_1 h_2}+{}\\
  &\hspace*{-2mm}{}+\fr{ - \bm{A}_{10;i_c+1,1} +\bm{A}_{10;i_c+1,2}}{h_1h_2} +   \fr{- \bm{A}_{0;\cdot,1} + \bm{A}_{0;\cdot,2}}{h_2}\,; \\
   \bm{B}_{i_c,j}&=\fr{\bm{A}_{00;i_c-1,j-1} - 2\bm{A}_{00;i_c-1,j} + \bm{A}_{00;i_c-1,j+1}}{h_1h_2} + {}\\
   &\hspace*{2mm}{}+   \fr{-\bm{A}_{00;i_c,j-1} + 2\bm{A}_{00;i_c,j} - \bm{A}_{00;i_c,j+1}}{h_1h_2}+{}\\
   &\hspace*{3mm}{}+ \fr{ - \bm{A}_{10;i_c,j-1} + 2\bm{A}_{10;i_c,j} - \bm{A}_{10;i_c,j+1}}{h_1h_2} +{}\\
  &\hspace*{-1mm} {}+\fr{ \bm{A}_{10;i_c+1,j-1} - 2\bm{A}_{10;i_c+1,j} + \bm{A}_{10;i_c+1,j+1}}{h_1h_2} +{}\\
&\hspace*{1mm}{}+ \fr{\bm{A}_{0;\cdot,j-1} - 2\bm{A}_{0;\cdot,j} + \bm{A}_{0;\cdot,j+1}}{h_2}\,,\enskip    0<j<j_c\,; 
 \end{align*}
  
 \noindent
  \begin{align*}
  \bm{B}_{i_c,j_c}& =1 + 
  \fr{\bm{A}_{00;i_c-1,j_c-1} - \bm{A}_{00;i_c-1,j_c}}{h_1h_2}+{}\\[2pt]
  &{}+\fr{ - \bm{A}_{00;i_c,j_c-1} + 
\bm{A}_{00;i_c,j_c} - \bm{A}_{01;i_c-1,j_c}}{h_1h_2} +{}\\[2pt]
&{}+ \fr{\bm{A}_{01;i_c-1,j_c+1} + \bm{A}_{01;i_c,j_c} - \bm{A}_{01;i_c,j_c+1}}{h_1h_2}+{}\\[2pt]
&{}+ \fr{ - \bm{A}_{10;i_c,j_c-1} + \bm{A}_{10;i_c,j_c} + \bm{A}_{10;i_c+1,j_c-1}}{h_1h_2}+{}\\[2pt]
&{}+ \fr{ -\bm{A}_{10;i_c+1,j_c} + \bm{A}_{11;i_c,j_c} - \bm{A}_{11;i_c,j_c+1}}{h_1h_2}+{}\\[2pt]
&{}+\fr{ - \bm{A}_{11;i_c+1,j_c} + \bm{A}_{11;i_c+1,j_c+1}}{h_1h_2} + {}\\[2pt]
&{}+\fr{\bm{A}_{0;i_c-1,\cdot} - \bm{A}_{0;i_c, \cdot} - \bm{A}_{1;i_c,\cdot} + \bm{A}_{1;i_c+1,\cdot}}{h_1} + {}\\[2pt]
&\hspace*{3mm}{}+\fr{\bm{A}_{0;\cdot,j_c-1} - \bm{A}_{0;\cdot,j_c} - \bm{A}_{1;\cdot,j_c} + \bm{A}_{1;\cdot,j_c+1}}{h_2}\,. 
  \end{align*}
  %
  В этом списке присутствуют обозначения инструментов~$\bm{A}$ 
с~четырьмя и~тремя индексами. В~первой группе пара индексов до точки 
с~запятой означает тип двумерного $\alpha$-оп\-ци\-она~(\ref{e18-aga}), а~после 
нее~--- его страйк. Во второй группе представлены одномерные версии 
двумерных $\alpha$-оп\-ци\-онов. Индекс до точки с~запятой означает тип опциона, 
числовой индекс после нее~--- его страйк, а~позиция маркера <<точка>> 
показывает координату безрискового актива. 
  
  Сценарные баттерфляи, полученные из $\alpha$-оп\-ци\-онов, позволяют 
произвольный инструмент на рынке представить в~виде портфеля  
$\alpha$-оп\-ци\-онов. Для нахождения его доходов следует воспользоваться 
соотношениями~(\ref{e2-aga}) с~учетом переопределения~(\ref{e8-aga}). Так, 
в~однотипном случае платежная функция портфеля находится в~соответствии 
с~(\ref{e17-aga}) по правилам: 
  \begin{multline}
  \bm{U}\to 1\,,\enskip \bm{A}_{ij}\to \omega_{1;i}(x)\omega_{2;j}(y)\,,\\[2pt] 
\bm{A}_{i\cdot}\to \omega_{1;i}(x)\,,\enskip 
 \bm{A}_{\cdot j}\to \omega_{2;j}(y)\,;
  \label{e19-aga}
  \end{multline}
  
  \vspace*{-12pt}
  
  \noindent
  \begin{multline*}
  \omega_{\beta_1;i\cdot}(x)=\max \left( 0,\alpha_1(x-s_i)\right)\,,\
  i\in \bm{I};\\[2pt]
   \omega_{2;i}(y)=\max \left( 0,\alpha_2(y-t_j)\right), \enskip j\in \bm{J}\,.
 % \label{e20-aga}
  \end{multline*}
  
  Аналогично в~соответствии с~(\ref{e18-aga}) записываются в~смешанном 
случае правила формирования платежных функций: 
  \begin{multline*}
  \bm{U}\to 1\,,\enskip \bm{A}_{\beta;ij}\to \omega_{\beta_1,1;i}(x)\omega_{\beta_2,2;j}(y),\\[2pt]
  \bm{A}_{\beta_1;i,\cdot}\to \omega_{\beta_1,1;i}(x),\enskip
  \bm{A}_{\beta_2;\cdot,j}\to \omega_{\beta_2,2;j}(y)\,;
  \end{multline*}
  
  \vspace*{-12pt}
  
  \noindent
  \begin{multline}
  \omega_{\beta_1,1;i}(x)=\max \left( 0,\alpha_1(x-s_i)\right),\enskip i\in \bm{I},\\[2pt]
  \omega_{\beta_2,2;j}(y)=\max \left( 0,\alpha_2(y-t_j)\right),\enskip j\in\bm{J}.
  \label{e21-aga}
  \end{multline}
  
  \section{Иллюстративный пример}
  
  \vspace*{-1pt}
  
  Для построения мер ${\mathsf C}\{\cdot\}$ и~${\mathsf P}\{\cdot\}$ и~их 
сравнительного анализа данные в~примере заимствуются из~[6]. Так, 
принимается ${\mathsf X}\hm=[0,1)$, ${\mathsf Y}\hm=[0,1)$, а~для ${\mathsf 
F}_{\mathsf {CX}}(x)$ и~${\mathsf F}_{\mathsf{CY}}(y)$ выбираются  
бе\-та-рас\-пре\-де\-ле\-ния с~па\-ра\-мет\-ра\-ми $\{3/2,2\}$ и~$\{3/2,3\}$ 
соответственно, для ${\mathsf F}_{\mathsf PX}(x)$ и~${\mathsf F}_{\mathsf PY}(y)$~--- $\{2,3\}$ и~$\{2,4\}$:

\vspace*{-2pt}

\noindent
  \begin{align*}
  {\mathsf F}_{\mathsf {CX}} (x) &= \fr{x^{3/2}(5-3x)}{2}\,;\\  
  {\mathsf F}_{\mathsf {CY}}(y)&= \fr{y^{3/2}(35-42y+15y^2)}{8}\,;\\
  {\mathsf F}_{\mathsf {PX}}(x) &= x^2\left(6-8x+3x^2\right);\\
  {\mathsf F}_{\mathsf {PY}}(y)&= y^2\left( 10-20y+15y^2-4y^3\right).
  \end{align*}
  %
  
  \vspace*{-2pt}
  
  \noindent
  Из них совместные функции распределения для обеих мер строятся как
  
  \vspace*{-5pt}
  
  \noindent 
  \begin{multline}
  {\mathsf F}(x,y)={}\\
\hspace*{-3mm}  {}= {\mathsf F}_{\mathsf X}(x) {\mathsf F}_{\mathsf Y}(y) \left( 
1+3\kappa \left( 1-{\mathsf F}_{\mathsf X}(x)\right) \left(1-{\mathsf F}_{\mathsf 
Y}(y)\right)\right).
  \label{e22-aga}
  \end{multline}
  
\vspace*{-1pt}
 
  Искомые двумерные функции распределения ${\mathsf F}_{\mathsf C}(x,y)$ 
и~${\mathsf F}_{\mathsf P} (x,y)$ определяются подстановкой в~(\ref{e22-aga}) 
в~качестве параметра, отвечающего за корреляционную связь компонент, 
соответственно $\kappa_c\hm=0$ и~$\kappa_p\hm=0{,}2$. Из них простым 
смешанным дифференцированием по обеим переменным находятся плотности 
$c(x,y)$ и~$p(x,y)$, но ввиду громоздкости записей они здесь не приводятся. 
  
  Двумерная дискретизация множества ${\mathsf X}\times \mathsf{Y}$ 
в~примере проводится также при $v_1\hm=6$ и~$v_2\hm=5$, а~центральным 
выбирается страйк $i_c\hm=3$, $j_c\hm=3$. 
  
  В отличие от $\zeta$-рын\-ков~[6], для которых в~целях применения 
дискретного алгоритма находились стоимости сценарных индикаторов, для  
\mbox{$\alpha$-рын}\-ков следует вычислять стоимости сценарных баттерфляев. И~потому 
для адекватного сравнения относительных доходов естественно вычислять и~их 
средние доходы. И~те и~другие, а~это векторы $\bm{c}^B$ и~$\bm{p}^B$, 
определяются интегрированием платежных функций~(\ref{e4-aga})  
и~(\ref{e6-aga}) с~плотностями $c(x,y)$ и~$p(x,y)$ соответственно. 
  
  Применением к~этим векторам дискретного алгоритма оптимизации~[6], 
основанного на процедуре Ней\-ма\-на--Пир\-со\-на~\cite{7-aga}, определяется 
вектор весов базисных баттерфляев для оптимального двумерного портфеля. 
При этом в~качестве функции рисковых предпочтений выбирается 
$\varphi(\varepsilon)\hm=\varepsilon^2$, $\varepsilon\hm\in [0,1]$.

\begin{figure*} %fig1
\vspace*{1pt}
\begin{minipage}[t]{80mm}
\begin{center}
   \mbox{%
\epsfxsize=77.328mm
\epsfbox{aga-1.eps}
}
\end{center}
\vspace*{-9pt}
\Caption{Доходы оптимального опционного портфеля при дискретизации $6\times5$}
\end{minipage}
%\end{figure*}
\hfill
%\begin{figure*} %fig2
\vspace*{1pt}
\begin{minipage}[t]{80mm}
\begin{center}
   \mbox{%
\epsfxsize=77.328mm
\epsfbox{aga-2.eps}
}
\end{center}
\vspace*{-9pt}
\Caption{Доходы сценарного портфеля при дискретизации $40\times40$}
\end{minipage}
\end{figure*}

  
  В результате  получается вектор весов 
  
  \vspace*{-4pt}
  
  \noindent
 \begin{multline*}
  \bm{g}=\{0{,}118; 0{,}159; 0{,}0113; 0{,}000219; 0{,}000008; 0{,}414;\\
   1{,}0; 0{,}228; 0{,}0151; 0{,}000989; 0{,}0739; 0{,}788; 0{,}602;\\
      0{,}0873; 0{,}00175; 0{,}0069; 0{,}309; 0{,}495; 0{,}176; 0{,}00191;
\end{multline*}

%\vspace*{-3pt}

%\pagebreak

\noindent
 \begin{multline*}
   \hspace*{-4.8848pt}0{,}000907; 0{,}0254; 0{,}0405; 0{,}0291; 0{,}00159; 0{,}0000058;\\
   0{,}0000848; 0{,}00151; 0{,}00112;  0{,}000009\}. 
  \end{multline*}
  
  \vspace*{-2pt}

  
  Он порождает оптимальный портфель~(\ref{e12-aga}) с~инвестиционной 
суммой, средним доходом и~средней доходностью соответственно 
  $A\hm=0{,}28642$,  $R\hm=0{,}364418$  и~$y\hm=0{,}272317$. 
      График его платежной функции изображен на рис.~1. Для сравнения на 
рис.~2 приведен аналогичный график для сценарного рынка при дискретизации 
$40\times40$.
  

  По понятным причинам графики платежных функций на рис.~1 и~2 
демонстрируют большее взаимное сходство, чем аналогичная пара графиков 
из~[6]. Но и~различие между собой графиков на рис.~1 и~2 пред\-став\-ля\-ет\-ся 
естественным. 
  
  Остается определить оптимальные портфели в~терминах $\alpha$-оп\-ци\-онов 
рас\-смат\-ри\-ва\-емых типов. Для нахождения каждого такого пред\-став\-ле\-ния 
в~формулу~(\ref{e12-aga}) следует для всех пар $(i,j)$ под\-став\-лять вмес\-то 
индикаторов~$\bm{B}_{ij}$ со\-от\-вет\-ст\-ву\-ющие им пред\-став\-ле\-ния  
в~$\alpha$-оп\-ци\-онах. В~результате после упрощений получаются четыре 
однотипных портфеля и~один смешанный. Так, оптимальный 
\textit{однотипный} портфель для $\bm{\alpha}\hm= \{-1, +1\}$, т.\,е.\ образованный 
путами для первого актива и~коллами~--- для второго: 

\vspace*{-2pt}

\noindent
 \begin{multline*}
  \bm{G}_{01}=0{,}000006 \bm{U} + 16{,}369\bm{A}_{11} - 35{,}107\bm{A}_{12} + {}\\
  {}+12{,}681\bm{A}_{13} + 5{,}641\bm{A}_{14} + 0{,}416\bm{A}_{15} + 1{,}774\bm{A}_{1\cdot} - {}\\
{}-12{,}549\bm{A}_{21} + 48{,}875\bm{A}_{22} - 39{,}318\bm{A}_{23} + 1{,}263\bm{A}_{24} + {}\\
{}+ 1{,}729\bm{A}_{25} - 3{,}812\bm{A}_{2\cdot} - 16{,}162\bm{A}_{31} + 9{,}717\bm{A}_{32} + {}\\
{}+21{,}364\bm{A}_{33} - 15{,}431\bm{A}_{34} + 0{,}512\bm{A}_{35} + 1{,}636\bm{A}_{3\cdot} + {}\\
{}+ 4{,}007\bm{A}_{41} - 20{,}268\bm{A}_{42} + 19{,}612\bm{A}_{43} + 3{,}707\bm{A}_{44} - {}\\
{}-7{,}058\bm{A}_{45} + 0{,}366\bm{A}_{4\cdot} + 7{,}603\bm{A}_{51} - 2{,}899\bm{A}_{52} - {}\\
{}- 13{,}593\bm{A}_{53} + 5{,}279\bm{A}_{54} + 3{,}609\bm{A}_{55} + 0,031\bm{A}_{5\cdot} + {}\\
{}+ 0{,}731\bm{A}_{61} - 0{,}318\bm{A}_{62} - 0{,}745\bm{A}_{63} - 0{,}458\bm{A}_{64} + {}
\end{multline*}

\noindent
\begin{multline*}
{}+0{,}791\bm{A}_{65} + 0{,}005\bm{A}_{6\cdot} + 0{,}0004\bm{A}_{\cdot1} + 0{,}007\bm{A}_{\cdot 2} -{}\\
{}- 0{,}009\bm{A}_{\cdot 3} - 0{,}004\bm{A}_{\cdot 4} + 0{,}006\bm{A}_{\cdot 5}.
\end{multline*}

\vspace*{-2pt}


  
  Платежные функции всех однотипных портфелей получаются по 
правилам~(\ref{e19-aga}). Проведенные расчеты под\-тверж\-да\-ют вер\-ность 
алгоритма. Все они, несмотря на внешнее различие их пред\-став\-ле\-ний, на 
идеальном рынке должны иметь единую платежную функцию с~графиком, 
пред\-став\-лен\-ным на рис.~1. 
  
  Оптимальный \textit{смешанный портфель} строится вновь по 
формуле~(\ref{e12-aga}), но в~смешанном базисе\linebreak естественного происхождения с~выделенным 
цент\-раль\-ным страйком~$(3, 3)$. В~первом квад\-ран\-те\linebreak 
(относительно цент\-ра рынка) используются $\alpha$-оп\-ци\-оны~$\bm{A}_{11}$, во 
втором~--- $\bm{A}_{01}$, в~треть\-ем~--- $\bm{A}_{00}$, в~\mbox{чет\-вер\-том}~--- 
$\bm{A}_{10}$. 
  
  Вычисления с~применением~(\ref{e12-aga}) дают оптимальный 
\textit{смешанный} портфель:

\vspace*{-2pt}
 

\noindent
\begin{multline*}
  \bm{G}_m=0{,}602\bm{U} + 16{,}369\bm{A}_{00;11} - 35{,}107\bm{A}_{00;12} + {}\\
  {}+
18{,}737\bm{A}_{00;13} - 12{,}549\bm{A}_{00;21} + 48{,}876\bm{A}_{00;22} - {}\\
{}-
36{,}326\bm{A}_{00;23} - 3{,}82\bm{A}_{00;31} - 13{,}769\bm{A}_{00;32} +{}\\
{}+ 17{,}589\bm{A}_{00;33} - 
6{,}056\bm{A}_{01;13} + 5{,}641\bm{A}_{01;14} +{}\\
{}+ 0{,}416\bm{A}_{01;15} - 2{,}992\bm{A}_{01;23} + 
1{,}263\bm{A}_{01;24} + {}\\
{}+1{,}729\bm{A}_{01;25} + 9{,}049\bm{A}_{01;33} - 6{,}903\bm{A}_{01;34} - {}\\
{}-2{,}145\bm{A}_{01;35} - 12{,}342\bm{A}_{10;31} + 23{,}486\bm{A}_{10;32} - {}\\
{}-11{,}144\bm{A}_{10;33} + 4{,}007\bm{A}_{10;41} - 20{,}268\bm{A}_{10;42} + {}\\
{}+16{,}261\bm{A}_{10;43} + 7{,}603\bm{A}_{10;51} - 2{,}899\bm{A}_{10;52} -{}\\
{}-4{,}704\bm{A}_{10;53} +  0{,}731\bm{A}_{10;61} - 0{,}318\bm{A}_{10;62} - {}\\
{}-0{,}413\bm{A}_{10;63} + 5{,}871\bm{A}_{11;33} - 8{,}528\bm{A}_{11;34} +{}\\
{}+ 2{,}657\bm{A}_{11;35} + 3{,}351\bm{A}_{11;43} + 3{,}707\bm{A}_{11;44} - {}\\
{}-7{,}058\bm{A}_{11;45} - 8{,}889\bm{A}_{11;53} + 5{,}279\bm{A}_{11;54} +{}\\
{}+ 3{,}609\bm{A}_{11;55} - 0{,}333\bm{A}_{11;63} - 0{,}458\bm{A}_{11;64} +{}
\end{multline*}

\noindent
\begin{multline*}
{}+ 0{,}791\bm{A}_{11;65} + 1{,}3\bm{A}_{0;1\cdot} + 0{,}943\bm{A}_{0;2\cdot} - 2{,}243\bm{A}_{0;3\cdot} +{}\\
{}+ 3{,}568\bm{A}_{0;\cdot 1} - 4{,}497\bm{A}_{0;\cdot 2} + 0{,}929\bm{A}_{0;\cdot 3} - 0{,}642\bm{A}_{1;3\cdot} - {}\\
{}-
2{,}085\bm{A}_{1;4\cdot} + 2{,}492\bm{A}_{1;5\cdot} + 0{,}234\bm{A}_{1;6\cdot} - 
2{,}573\bm{A}_{1;\cdot 3} +{}\\
{}+ 2{,}145\bm{A}_{1;\cdot 4} + 0{,}428\bm{A}_{1;\cdot 5}. 
  \end{multline*}
  
  \vspace*{-2pt}
  
\noindent
  В этом портфеле 56~инструментов; среди них один безрисковый актив, 
42~двумерных $\alpha$-оп\-ци\-она~$\bm{A}_{00}$, $\bm{A}_{01}$, $\bm{A}_{10}$ 
и~$\bm{A}_{11}$ и~13~одномерных версий~$\bm{A}_0, \bm{A}_1$ в~количествах 
9, 9, 12, 12 и~6, 7 соответственно. 
  
  Двумерные $\alpha$-оп\-ци\-оны снабжены четырьмя индексами; первые два из 
них (до точки с~запятой) показывают тип опциона по каждой координате 
в~терминах~$\beta$, другие два индекса~--- номера страй\-ков. 
  
  Одномерные версии снабжены двумя индексами и~маркером <<точка>>. 
Один индекс до точки с~запятой указывает тип опциона, индекс после нее~---  
номер страй\-ка, а~маркер~--- координату под\-ра\-зу\-ме\-ва\-емо\-го безрискового 
актива.
  
  Платежная функция смешанного портфеля строится по  
правилам~(\ref{e21-aga}). Ее графиком служит все тот же график на рис.~1. 
  
  \section{Заключение }
  
  В работе решена задача алгоритмического на\-хож\-де\-ния пред\-став\-ле\-ний 
многомерных баттерфляев при сценарной дискретизации рынка и~\mbox{по\-стро\-ения} 
из них базиса в~терминах $\alpha$-оп\-ци\-онов~--- многомерного обобщения 
традиционных опционов колл и~пут. На конкретном примере двумерного рынка 
продемонстрирована работа этого алгоритма и~ее результат. Подобные расчеты 
могут быть без принципиальных трудностей реализованы и~для рынков 
большей размерности. Поскольку бат\-тер\-фляи для одномерного рынка 
образуются из трех страй\-ков, а~индикаторы~--- из двух, то их многомерные 
реп\-ли\-ка\-ции получаются еще более громоздкими. Фактические расчеты, 
проведенные для $n\hm=4$, показали, что уже объем перечня базисных 
инструментов однотипного базиса, например в~терминах~$\bm{A}_{0000}$, 
аналогичного~$\bm{A}_{00}$ из разд.~4, был соразмерен всему объему 
на\-сто\-ящей работы, а~потому и~на многомерных $\alpha$-рын\-ках лучше торговать 
непосредственно бат\-тер\-фля\-ями, а~не $\alpha$-оп\-ци\-онами. 
  
{\small\frenchspacing
 {%\baselineskip=10.8pt
 %\addcontentsline{toc}{section}{References}
 \begin{thebibliography}{9}
  \bibitem{1-aga}
  \Au{Агасандян~Г.\,А.} Применение континуального критерия VaR на 
финансовых рынках.~--- М.: ВЦ РАН, 2011. 299~с. 
  \bibitem{2-aga}
  \Au{Агасандян~Г.\,А.} Континуальный критерий VaR и~оптимальный 
портфель инвестора~// Управ\-ле\-ние большими сис\-те\-ма\-ми, 2018. Вып.~73. 
С.~6--26.
  \bibitem{3-aga}
  \Au{Агасандян~Г.\,А.} Континуальный критерий VaR на сценарных рынках~// 
Информатика и~её применения, 2018. Т.~12. Вып.~1. С.~32--40. 
  \bibitem{4-aga}
  \Au{Агасандян~Г.\,А.} Вычисление показателей оптимальных по CC-VaR 
портфелей на рынках опционов~// Информатика и~её применения, 2019. Т.~13. 
Вып.~3. С.~75--84. 
  \bibitem{5-aga}
  \Au{Агасандян~Г.\,А.} Многомерные рынки опционов и~оптимизация по  
CC-VaR~// Управ\-ле\-ние большими сис\-те\-ма\-ми, 2020. Вып.~88. С.~5--25.
  \bibitem{6-aga}
  \Au{Агасандян~Г.\,А.} Многомерные бинарные рынки и~CC-VaR~// 
Информатика и~её применения, 2022. Т.~16. Вып.~2. С.~2--10.
  \bibitem{7-aga}
  \Au{Крамер~Г.} Математические методы статистики~/ Пер. с~англ.~--- М.: 
Мир, 1975. 750~с. (\Au{Cramer~H.} Mathematical methods of statistics.~--- 
Princeton, NJ, USA: Princeton University Press, 1946. 575~p.)
\end{thebibliography}

 }
 }

\end{multicols}

\vspace*{-6pt}

\hfill{\small\textit{Поступила в~редакцию 09.03.22}}

\vspace*{8pt}

%\pagebreak

%\newpage

%\vspace*{-28pt}

\hrule

\vspace*{2pt}

\hrule

%\vspace*{-2pt}

\def\tit{MULTIDIMENSIONAL BUTTERFLIES IN~PROBLEMS 
OF~OPTIMIZATION ON CC-VaR}


\def\titkol{Multidimensional butterflies in~problems 
of~optimization on CC-VaR}


\def\aut{G.\,A.~Agasandyan}

\def\autkol{G.\,A.~Agasandyan}

\titel{\tit}{\aut}{\autkol}{\titkol}

\vspace*{-8pt}


\noindent
Federal Research Center ``Computer Science and Control'' of the Russian Academy 
of Sciences, 44-2~Vavilov Str., Moscow 119333, Russian Federation


\def\leftfootline{\small{\textbf{\thepage}
\hfill INFORMATIKA I EE PRIMENENIYA~--- INFORMATICS AND
APPLICATIONS\ \ \ 2023\ \ \ volume~17\ \ \ issue\ 1}
}%
 \def\rightfootline{\small{INFORMATIKA I EE PRIMENENIYA~---
INFORMATICS AND APPLICATIONS\ \ \ 2023\ \ \ volume~17\ \ \ issue\ 1
\hfill \textbf{\thepage}}}

\vspace*{3pt} 
  
  
  


  
  \Abste{The work continues studying problems of using continuous VaR-criterion (CC-VaR) in 
financial markets. Again some technical problems are concerned. However, they emerge this time 
not in multidimensional relatively simple binary markets but in multidimensional markets that are 
an extension of one-dimensional traditional
markets of options such as calls and puts. In assumption 
that scenario butterflies are not traded in markets directly, a~method of receiving their replication 
from multidimensional options, i.\,e., $\alpha$-options, is developed. It is based on options parity 
theorems and can be applied to markets of arbitrary dimension, but actual realization is conducted
for two-dimensional markets. The bases constructions in terms of $\alpha$-options both one-type and 
natural mixed with\linebreak\vspace*{-12pt}}

\Abstend{selected market center are produced. Theoretical representations of optimal 
portfolios in these bases accompanied with the payoffs diagram are illustrated by the distinctive 
example of a two-dimensional market.}
  
  \KWE{underliers; multidimensional market; investor's risk preferences function; continuous 
VaR-criterion; cost and forecast densities; scenario indicators; bases; binary options; one-type 
portfolio; market center; mixed portfolio}
  
 \DOI{10.14357/19922264230114} 

%\vspace*{-16pt}

%\Ack
%\noindent

  

%\vspace*{4pt}

  \begin{multicols}{2}

\renewcommand{\bibname}{\protect\rmfamily References}
%\renewcommand{\bibname}{\large\protect\rm References}

{\small\frenchspacing
 {%\baselineskip=10.8pt
 \addcontentsline{toc}{section}{References}
 \begin{thebibliography}{9} 
  \bibitem{1-aga-1}
  \Aue{Agasandyan, G.\,A.} 2011. \textit{Pri\-me\-ne\-nie kon\-ti\-nu\-al'\-no\-go kri\-te\-riya VaR 
na fi\-nan\-so\-vykh ryn\-kakh} [Application of continuous VaR-criterion in financial 
markets]. Moscow: CCRAS. 299~p.
  \bibitem{2-aga-1}
  \Aue{Agasandyan, G.\,A.} 2018. Kon\-ti\-nu\-al'\-nyy kri\-te\-riy VaR i~op\-ti\-mal'\-nyy 
port\-fel' in\-ves\-to\-ra [Continuous VaR-criterion and investor's optimal portfolio]. 
\textit{Upravlenie bol'shimi sistemami} [Large-Scale Systems Control] 73:6--26.
  \bibitem{3-aga-1}
  \Aue{Agasandyan, G.\,A.} 2018. Kon\-ti\-nu\-al'\-nyy kri\-te\-riy VaR na stse\-nar\-nykh 
ryn\-kakh [Continuous VaR-criterion in scenario markets]. \textit{Informatika i~ee 
Primeneniya~--- Inform. Appl.} 12(1):32--40.
  \bibitem{4-aga-1}
  \Aue{Agasandyan, G.\,A.} 2019. Vy\-chis\-le\-nie po\-ka\-za\-te\-ley op\-ti\-mal'\-nykh po  
CC-VaR port\-fe\-ley na ryn\-kakh op\-tsi\-o\-nov [Performance estimations for  
optimal-on-CC-VaR portfolios in option markets]. \textit{Informatika i~ee 
Primeneniya~--- Inform. Appl.} 13(3):75--84.
  \bibitem{5-aga-1}
  \Aue{Agasandyan, G.\,A.} 2020. Mno\-go\-mer\-nye ryn\-ki op\-tsi\-o\-nov i~op\-ti\-mi\-za\-tsiya 
po CC-VaR [Multidimensional option markets and optimization on CC-VaR]. 
\textit{Upravlenie bol'shimi sistemami} [Large-Scale Systems Control] 88:5--25.
  \bibitem{6-aga-1}
  \Aue{Agasandyan, G.\,A.} 2022. Mno\-go\-mer\-nye bi\-nar\-nye ryn\-ki i~CC-VaR 
[Multidimensional binary markets and CC-VaR]. \textit{Informatika i~ee 
Primeneniya~--- Inform. Appl.} 16(2):2--10.
  \bibitem{7-aga-1}
  \Aue{Cramer, H.} 1946. \textit{Mathematical methods of statistics}. Princeton, 
NJ: Princeton University Press. 575~p.

\end{thebibliography}

 }
 }

\end{multicols}

\vspace*{-6pt}

\hfill{\small\textit{Received March 9, 2022}}

  
  \Contrl
  
  \noindent
  \textbf{Agasandyan Gennady A.} (b.\ 1941)~--- Doctor of Science in physics and 
mathematics, leading scientist, A.\,A.~Dorodnicyn Computing Center, Federal 
Research Center ``Computer Science and Control'' of the Russian Academy of 
Sciences, 40~Vavilov Str., Moscow 119333, Russian Federation; 
\mbox{agasand17@yandex.ru}
  

   
\label{end\stat}

\renewcommand{\bibname}{\protect\rm Литература} 
     %11
\def\stat{zatsman}

\def\tit{ТРАНСФОРМАЦИИ ОБЪЕКТОВ ПЕРВОГО И~ВТОРОГО ПОРЯДКА 
В~ЛЕКСИКОГРАФИЧЕСКОЙ ИНФОРМАЦИОННОЙ СИСТЕМЕ$^*$}

\def\titkol{Трансформации объектов первого и~второго порядка 
в~лексикографической информационной системе}

\def\aut{И.\,М.~Зацман$^1$}

\def\autkol{И.\,М.~Зацман}

\titel{\tit}{\aut}{\autkol}{\titkol}

\index{Зацман И.\,М.}
\index{Zatsman I.\,M.}


{\renewcommand{\thefootnote}{\fnsymbol{footnote}} \footnotetext[1]
{Исследование выполнено в~ФИЦ ИУ РАН за счет гранта Российского научного фонда №\,24-18-00155, {\sf 
https://rscf.ru/project/24-18-00155}. Работа выполнялась с~использованием инфраструктуры Центра 
коллективного пользования <<Высокопроизводительные вычисления и~большие данные>> (ЦКП 
<<Информатика>>) ФИЦ ИУ РАН (г.\ Москва).}}


\renewcommand{\thefootnote}{\arabic{footnote}}
\footnotetext[1]{ Федеральный исследовательский центр <<Информатика и~управление>> Российской академии наук, 
\mbox{izatsman@yandex.ru}}

\vspace*{-12pt}


  
  \Abst{Рассматриваются теоретические основания проектирования информационных 
технологий (ИТ) интеграции двуязычных словарей и~параллельных корпусов. Дано описание 
первых результатов создания третьего уровня классификации трансформаций объектов 
предметной области информатики, которую предполагается использовать при создании 
концепции лексикографической информационной системы, обеспечивающей интеграцию. 
Все сущности информатики в~статье разделены на два глобальных класса: объекты и~их 
трансформации. Для каждого такого класса конструируется своя классификация. Ранее были 
описаны два верхних уровня классификации трансформаций объектов предметной области. 
В~данной статье рассматривается третий уровень этой классификации. Основанием для 
построения самого верхнего ее уровня служило деление предметной области информатики 
на среды (ментальная, сенсорно воспринимаемая, цифровая и~ряд других сред), каждая из 
которых по определению включает объекты одной природы. Основанием для построения 
второго уровня классификации трансформаций объектов служила типология знаковых  
сис\-тем А.~Соломоника. Цель статьи состоит в~систематизации трансформаций первого 
и~второго порядка объектов предметной области на третьем уровне этой классификации. 
Основанием для систематизации служит средовая версия иерархии Акоффа.}
  
  \KW{объекты предметной области; трансформации объектов; классификация; данные; 
информация; знание; лексикографическая информационная сис\-тема}

\DOI{10.14357/19922264240211}{VZTGVV}
  
\vspace*{3pt}


\vskip 10pt plus 9pt minus 6pt

\thispagestyle{headings}

\begin{multicols}{2}

\label{st\stat}
  
\section{Введение}

\vspace*{-9pt}

  Возникновение параллельных корпусов, в~которых предложениям 
оригинального текста со\-по\-став\-ле\-ны предложения его перевода, обеспечило 
возможность контрастивного лингвистического\linebreak \mbox{анализа} на принципиально 
новом уровне полноты и~точности, недостижимом в~докорпусную эпоху. 
Пионерскими в~этой области стали работы \mbox{1990-х~гг}. Стига Йоханссона  
с~анг\-ло-нор\-веж\-ским корпусом~[1]. В России параллельные корпусы стали 
формироваться в~начале XXI~века в~рамках Национального корпуса русского 
языка~[2].
  
  Создатели двуязычных словарей используют параллельные корпусы для 
сбора материала и~эмпирической проверки своих гипотез, касающихся 
межъязы\-ко\-вой эквивалентности. Ценность параллельных корпусов 
определяется тем, что в~лингвистике этап сбора исходного материала считается 
наиболее трудоемким и~наименее творческим, а~параллельные корпусы 
позволяют значительно сэкономить время и~силы для творческого этапа 
создания словарей~[3].
 % 
  При этом двуязычные словари, создаваемые на основе исходного материала, 
извлеченного из параллельных корпусов, сейчас формируются без связей с~их 
текстами. Другими словами, онлайновые связи созданных словарей 
с~параллельными корпусами, которые служили источниками исходного 
материала, отсутствуют. 

Параллельные корпусы постоянно пополняются 
новыми текстами, в~предложениях которых можно обнаружить новые значения 
слов и~устойчивых словосочетаний. Однако при этом отсутствуют методы 
и~средства оперативного обновления словарей по корпусным данным. 
В~настоящее время проблема установления связей между двуязычными 
словарями и~параллельными корпусами (далее~--- проблема интеграции) 
находится на стадии поиска концептуальных подходов к~их интеграции на 
уровне значений.
  
  Подход к~решению проблемы интеграции, предлагаемый в~статье, учитывает 
  и~появление новых значений слов и~устойчивых словосочетаний, и~динамику 
смысловых значений, которая обусловлена развитием и~пополнением знания 
лингвистов, фиксирующих эти значения в~результате семантического анализа 
пополняемых корпусных данных. Проведенные эксперименты показали, что 
обнаружение нового лингвистического знания обусловливает и~формирование 
дефиниций новых значений, и~пересмотр уже существующих дефиниций~[4, 5].
  
  Например, в~проведенных экспериментах с~использованием ЦКП 
<<Информатика>> ФИЦ ИУ РАН фиксировалась эволюция значений немецких 
модальных глаголов, исходное состояние значений которых было описано 
в~не\-мец\-ко-рус\-ском словаре. В~экспериментальном массиве текстов как 
потенциальных источниках нового знания 16\,268 предложений содержали 
немецкие модальные глаголы и~в~2041 из них встречался глагол sollen. 
В~начале эксперимента в~словаре были описаны~12~значений этого модального 
глагола. По окончании эксперимента лингвисты обнаружили два новых его 
значения, согласовали их дефиниции и~описали эволюцию дефиниций~[6, 7].
  
  Таким образом, для решения проблемы интеграции требуется фиксировать 
новое знание, обнаруженное лингвистами в~текстовых данных параллельных 
корпусов, отслеживать эволюцию знания, представленного в~виде дефиниций 
значений слов и~устойчивых словосочетаний, и,~соответственно, 
актуализировать электронные двуязычные словари. Предлагаемый 
концептуальный подход к~интеграции, который планируется реализовать 
в~процессе проектирования лексикографической информационной сис\-те\-мы, 
фиксирующей эволюцию лингвистического знания, основан на решении 
следующих задач:\\[-14pt]
  \begin{itemize}
  \item категоризация трех базовых понятий информатики, включенных 
  в~иерархию Акоффа~[8] (данные, информация, знание), на объекты 
проектируемой сис\-те\-мы, которая необходима, чтобы фиксировать 
<<кванты>> нового знания и~отслеживать его эволюцию в~этой сис\-теме;\\[-15pt]
  \item  систематизация трансформаций объектов этой сис\-темы.\\[-14pt]
  \end{itemize}
  
  Цель статьи и~состоит в~решении двух задач: категоризации трех базовых 
понятий информатики на объекты лексикографической информационной  
сис\-те\-мы и~сис\-те\-ма\-ти\-за\-ции трансформаций первого и~второго порядка 
ее объектов.
  
  Трансформациями первого порядка, о которых сказано в~формулировке цели 
статьи, называются взаимные преобразования между двумя объектами  
сис\-те\-мы одной природы. Например, перевод в~сис\-те\-ме текста с~русского 
языка на английский относится к~ним. Трансформациями второго порядка 
и~выше называются взаимные преобразования между двумя и~более объектами 
разной природы. Например, кодирование символов текс\-та компьютерными 
кодами и~их декодирование относятся по определению к~трансформациям 
второго порядка.

%\vspace*{-9pt}
  
\section{Процессы трансформаций в~информатике}

%\vspace*{-3pt}

Процессы трансформаций, рассматриваемые в~статье, относятся к~теоретическому ядру информатики, а~не 
только к~проектированию лексикографической информационной сис\-те\-мы. Например, из трех основных 
подходов к~описанию предметной об\-ласти информатики\footnote{В статье предметная область информатики 
трактуется согласно концепции полиадического компьютинга Пола Розенблума~\cite{9-zac}.} (объектный, 
трансформационный и~синтетический) сис\-те\-ма\-ти\-за\-ция трансформаций ближе всего ко второму 
подходу. Примерами первого подхода, в~рамках которого основное внимание уделяется объектам предметной 
области информатики и~в~меньшей степени отношениям\linebreak между ними, могут служить  
работы~\cite{8-zac, 10-zac, 11-zac}; \mbox{примерами} второго подхода, в~рамках которого основное внимание 
уделяется трансформациям и~в~меньшей степени трансформируемым объектам,~---  
работы~\cite{12-zac, 13-zac}; примерами третьего, синтетического подхода, в~котором уделяется внимание 
и~объектам предметной об\-ласти информатики, и~отношениям между ними, могут служить работы~\cite{14-zac, 
15-zac, 16-zac, 17-zac, 18-zac}.

  Таким образом, для описания трансформаций объектов лексикографической 
информационной\linebreak системы предпочтительнее всего трансформационный 
подход, который упоминается и~в определениях информатики. Например, 
в~2009~г.\ П.~Деннинг и~П.~Розенблум сформулировали суть \mbox{информатики} как 
компьютинга следующим образом: <<$\ldots$информатика~--- это не просто 
алгоритмы и~структуры данных; это преобразования [трансформации] 
представлений>>~\cite{12-zac}. Чуть позже, в~контексте краткого описания 
парадигмы информатики как компьютинга, П.~Деннинг и~П.~Фриман изменили 
эту формулировку на такую: <<Центральный объект внимания в~информатике 
можно определить как информационные процессы~--- \textit{естественные или 
искусственные процессы, преобразующие информацию} (курсив мой~--- 
И.\,З.)>>~\cite{13-zac}. Согласно парадигме, предлагаемой авторами этой 
статьи, на начальном этапе проектирования автоматизированных систем 
базовыми элементами моделей их функционирования служат 
\textit{информационные про\-цессы}.
  
  Однако если 15~лет назад в~формулировке из работы~\cite{13-zac} шла речь 
о~процессах, преобразующих информацию, то в~последние~10~лет в~спектр 
процессов трансформаций все чаще стали включать процессы, преобразующие 
не только информацию, но также и~другие объекты автоматизированных 
систем, в~первую очередь данные и~знания~[19--21]. Например, Виктория 
Стодден, позиционируя науку о~данных как одну из дисциплин информатики, 
говорит, что центральный объект исследований в~науке о~данных~--- это 
<<изучение обобщаемого извлечения знания из данных>>~\cite{21-zac}. 
Увеличение и~чис\-ла объектов, и~спект\-ра процессов их трансформаций 
в~автоматизированных сис\-те\-мах обуслов\-ли\-ва\-ет не\-об\-хо\-ди\-мость 
систематизации и~объектов, и~процессов их трансформаций на начальном этапе 
проектирования сис\-тем.
  
  Для создания концепции лексикографической информационной сис\-те\-мы 
и~проектирования ИТ, обеспечивающих интеграцию 
двуязычных словарей и~параллельных корпусов, сначала выполним 
категоризацию на объекты этой сис\-те\-мы трех базовых понятий информатики 
(данные, информация, знание) в~контексте построения классификаций 
сущностей ее предметной об\-ласти.
  
  Необходимость использования классификаций информатики в~процессе 
создания концепции проиллюстрируем, используя иерархию  
Акоффа~\cite{8-zac}. Он использовал принцип их вертикального размещения 
в~иерархии снизу вверх: данные, информация и~знание. Еще в~ней есть термин 
<<мудрость>>, который в~статье не рассматривается. Такое размещение Акофф 
прокомментировал так: <<Каждое из пе\-ре\-чис\-лен\-ных понятий [кроме данных] 
содержит в~себе нижестоящие$\ldots$>>~\cite{8-zac}.
  
  Этому принципу размещения и~комментарию Акоффа свойственны 
недостатки, проанализированные, в~частности, в~работе~\cite{10-zac}. Главный 
вывод, к~которому пришла Роули после изучения иерархии Акоффа, 
заключается в~следующем: <<$\ldots$информация определяется в~терминах 
данных, знание~--- в~терминах информации$\ldots$ но существует меньше 
консенсуса в~описании трансформаций, которые преобразуют сущности, 
расположенные ниже в~иерархии, в~те, которые находятся над ними, что 
приводит к~их терминологической неопределенности>>~\cite{10-zac}. Причина 
этой неопределенности, скорее всего, в~том, что базовые понятия информатики 
включены в~иерархию Акоффа изолированно от общего контекста 
классификаций сущностей ее предметной об\-ласти.

%\vspace*{-9pt}
  
\section{Классификации сущностей информатики}


%\vspace*{-2pt}

  Все сущности предметной области информатики в~работах~[22, 23] 
разделены на два глобальных класса: ее объекты и~их трансформации. Для 
каждого такого класса была предложена своя классификация. 
В~работе~\cite{22-zac} дано описание классификации объектов предметной 
области информатики, первый уровень которой содержит базовые понятия ее 
предметной области (данные, информация, знания и~др.).  
В~работе~\cite{23-zac} дано описание двух верхних уровней классификации 
трансформаций объектов предметной об\-ласти (см.\ рисунок 
в~работе~\cite{23-zac}). Основанием для построения самого верхнего ее уровня послужило деление 
предметной области информатики на среды\footnote{В~работе~\cite{24-zac} дано описание пяти сред 
предметной области информатики (ментальная; сенсорно воспринимаемая, или информационная; 
цифровая; нейро- и~ДНК-среда), каждая из которых по определению включает объекты одной и~той же 
природы.} и~степень разнообразия природы объектов, вовлеченных в~трансформации:
\begin{itemize}
\item  первый класс верхнего уровня классификации включает 
трансформации объектов в~пределах среды только одной природы 
(трансформации первого порядка);
\item  второй класс включает трансформации объектов, относящихся 
к~двум средам разной природы (трансформации второго порядка);
\item третий и~последующие классы включают трансформации объектов, 
относящихся к~трем и~более средам разной природы (трансформации 
третьего и~более высоких порядков).
\end{itemize}

  В работе~\cite{23-zac} были приведены примеры для трех первых классов 
трансформаций, включая пример трансформаций объектов, относящихся 
к~двум средам разной природы (компьютерное кодирование символов текстов 
с~по\-мощью таб\-лиц Unicode).
  
Основанием для построения второго уровня классификации трансформаций объектов послужила типология 
знаковых сис\-тем А.~Соломоника~\cite[c.~131]{25-zac}: естественные знаковые сис\-те\-мы, образные,  
ес\-тест\-вен\-но-язы\-ко\-в$\acute{\mbox{ы}}$е,  
вер\-баль\-но-не\-сло\-вес\-ные сис\-те\-мы записи\footnote{Под системой записи понимается знаковая 
система, сочетающая вербальные знаки с~несловесными (языки нотной записи, карт, таблиц и~др.).} 
и~формализованные знаковые сис\-те\-мы, включая математические. Введем понятие обобщенного текста~--- 
это текст, который может быть создан в~любой из перечисленных знаковых систем. Тогда обобщенные тексты 
могут быть естественными, образными, ес\-тест\-вен\-но-язы\-ко\-в$\acute{\mbox{ы}}$\-ми,  
вер\-баль\-но-не\-сло\-вес\-ны\-ми и~формализованными. Второй уровень классификации трансформаций 
охватывает не все виды объектов предметной  
об\-ласти информатики, а~только перечисленные~5~видов текс\-тов и~их представления, вовлеченные 
в~процессы трансформаций в~одной или более средах вместе с~данными, знанием и~его концептами.

\begin{figure*}[b] %fig1
\vspace*{6pt}
      \begin{center}
     \mbox{%
\epsfxsize=121.191mm 
\epsfbox{zac-1.eps}
}
\end{center}
\vspace*{-6pt}
\Caption{Средовая версия иерархии Акоффа}
\end{figure*}

\section{Классификация трансформаций: построение~третьего 
уровня}

  Основанием для систематизации трансформаций первого и~второго порядка 
на третьем уровне этой классификации служит иерархия Акоффа~\cite{8-zac}, 
на основе которой и~была создана ее средов$\acute{\mbox{а}}$я версия~[26, 
27]. Для создания средов$\acute{\mbox{о}}$й версии была выполнена 
категоризация трех базовых понятий информатики (данные, информация, 
знания) на объекты лексикографической информационной сис\-те\-мы 
в~процессе создания ее концепции\linebreak (рис.~1).
  


  В отличие от классической иерархии Акоффа, в~ее 
средов$\acute{\mbox{о}}$й версии различаются три вида данных: сенсорно 
воспринимаемые, цифровые и~те данные, которые генерируются 
искусственными нейронными сетями (ИНС) в~системах искусственного интеллекта 
(далее~--- ИИ-дан\-ные). Последний вид данных необходим, например, для 
различения входа и~выхода процесса применения обученной 
ИНС в~цифровой модели генерации знания, описанию которой 
посвящена работа~\cite{27-zac}.
  
  Также предлагается различать два вида информации: сенсорно 
воспринимаемая и~цифровая. Кроме знания в~средов$\acute{\mbox{у}}$ю 
версию добавлены концепты и~ментальные образы сенсорно воспринимаемых 
данных. Последние служат промежуточной сущностью между сенсорно 
воспринимаемыми данными и~генерируемым знанием при описании процессов 
извлечения знания из текстовых данных лексикографической информационной 
системы. Описание объектов средов$\acute{\mbox{о}}$й версии иерархии 
Акоффа (см.\ рис.~1) и~отношений между ними дано в~работах~\cite{26-zac, 28-zac}.
  
  В средов$\acute{\mbox{о}}$й версии число объектов равно восьми. Если 
учитывать направления трансформаций, то между восемью объектами на 
рис.~1 она включает~16 их видов (трансформации на границе между сенсорно 
воспринимаемыми данными и~информацией, обозначенные символом~<<?>>, 
в~статье не рас\-смат\-ри\-ва\-ют\-ся). В~будущем число объектов 
в~средов$\acute{\mbox{о}}$й версии, которая выбрана как основание для 
сис\-те\-ма\-ти\-за\-ции трансформаций первого и~второго порядка, может быть 
увеличено. Для построения классификации трансформаций 
важ\-но не возможное увеличение числа объектов 
и~трансформаций между ними, а то, что их виды в~средов$\acute{\mbox{о}}$й 
версии распределены между трансформациями первого и~второго порядка. Из 
16~видов на рис.~1 шесть относятся к~трансформациям первого порядка, это\linebreak 
виды с~номерами~7, 8, 13--16 (далее~--- типология трансформаций первого 
порядка), а~десять~--- к~трансформациям второго порядка, это виды 
с~\mbox{номерами}~1--6 и~9--12 (далее~--- типология трансформаций второго 
порядка). Разместим обе типологии на третьем уровне классификации (см.\ ее 
схему на рис.~2). Перечислим виды трансформаций первой типологии, вводя 
в~скобках их краткие названия, используемые ниже на рис.~3:
  \begin{description}
  \item[\,] 7~--- членение знания на концепты с~помощью одной или нескольких 
знаковых систем (далее~--- членение знания);
  \item[\,] 8~--- формирование знания на основе концептов (формирование 
знания);
  \item[\,] 13~--- обучение ИНС;
  \end{description}
  
  \vspace*{-6pt}
  
  \pagebreak
  
  \end{multicols}
  
  \begin{figure*} %fig2
\vspace*{1pt}
      \begin{center}
     \mbox{%
\epsfxsize=127.513mm 
\epsfbox{zac-2.eps}
}
\end{center}
\vspace*{-9pt}
\Caption{Схема трех верхних уровней классификации трансформаций объектов (объединены 
по три слоя и~для второго, и~для третьего уровней этой классификации)}
\end{figure*}
  
  \begin{multicols}{2}
  
  \noindent
  \begin{description}
  \item[\,] 14~--- восстановление обучающей информации на основе 
содержания обученной ИНС (обращение ИНС);
  \item[\,] 15~--- использование обученной ИНС (использование ИНС);



  \item[\,] 16~--- восстановление исходных данных, соответствующих 
полученным результатам работы обучен\-ной ИНС (восстановление исходных данных 
по результатам ИНС).
  \end{description}
  
  
  Не все виды трансформаций 13--16 поддерживаются в~конкретных системах 
искусственного интеллекта, но с~теоретической точки зрения все их 
предлагается включить в~первую типологию для полноты спектра видов 
трансформаций.
  
  Перечислим виды трансформаций второй типологии:
  \begin{description}
  \item[\,] 1~--- декодирование цифровых данных в~компьютерных системах 
(декодирование данных);
  \item[\,]  2~--- кодирование сенсорно воспринимаемых данных (кодирование 
данных);
  \item[\,] 3~--- ментальное копирование сенсорно воспринимаемых данных 
(ментальное копирование);
  \item[\,] 4~--- восстановление сенсорно воспринимаемых данных по 
ментальным образам (восстановление по образам);
  \item[\,] 5~--- смысловая интерпретация без деления на концепты ментальных 
образов сенсорно воспринимаемых данных (смысловая интерпретация);
  \item[\,] 6~--- восстановление ментальных образов (восстановление образов);
  \item[\,] 9~--- представление концептов в~виде сенсорно воспринимаемой 
информации, например текс\-та\-ми, формулами, таблицами, рисунками и~т.\,д.\ 
(представление концептов);
  \item[\,] 10~--- понимание смысла сенсорно воспринимаемой информации 
(понимание смысла);
  \item[\,] 11~--- кодирование сенсорно воспринимаемой информации 
(кодирование информации);
\end{description}

\vspace*{-6pt}

\pagebreak

\end{multicols}

\begin{figure*} %fig3
\vspace*{1pt}
      \begin{center}
     \mbox{%
\epsfxsize=163mm 
\epsfbox{zac-3.eps}
}
\end{center}
\vspace*{-9pt}
\Caption{Схема частного случая классификации трансформаций объектов (трансформации 
пронумерованы согласно рис.~1)}
\end{figure*}

\begin{multicols}{2}

\noindent
\begin{description}

  \item[\,] 12~--- декодирование цифровой информации (декодирование 
информации).
  \end{description}
  
  Отметим, что в~существующих ИТ
  и~компьютерных системах наиболее часто используются виды 
трансформаций~13 и~15 типологии первого порядка и~1, 2, 11 и~12 типологии 
второго порядка. На рис.~2 в~первом слое третьего уровня классификации 
показаны типологии первого порядка без указания числа трансформаций в~них 
и~без детализации трансформируемых объектов.
  
  Во втором слое третьего уровня классификации условно (без названий) 
показаны типологии второго порядка. Также на рис.~2 в~третьем слое третьего 
уровня классификации условно (также без названий) показаны типологии 
третьего порядка, которые планируется рассмотреть в~отдельной статье. По 
определению они должны включать трансформации между тремя объектами 
разной природы, но средов$\acute{\mbox{а}}$я версия иерархии Акоффа 
включает трансформации только между двумя объектами разной природы. 
Поэтому потребуется другое основание для их систематизации (ранее были 
рассмотрены отдельные примеры трансформаций третьего 
порядка\footnote{Далеко не всегда трансформации третьего и~более высоких порядков можно 
рассматривать как последовательность трансформаций второго порядка. Примером этого могут 
служить трансформации в~процессе обучения пациента пользованию роботизированной рукой, 
охватывающие личностные концепты пациента, релевантные его намерениям, сигналы активности 
мозга как объекты нейросреды и~компьютерные коды~\cite{29-zac}.}~\cite{29-zac}).

\section{Классификация трансформаций: частный~случай}

  Выше было отмечено, что в~будущем число объектов 
в~средов$\acute{\mbox{о}}$й версии иерархии Акоффа может быть увеличено. 
Это означает, что увеличатся и~чис\-ло объектов, и~чис\-ло трансформаций между 
ними в~классификации трансформаций, так как эта средов$\acute{\mbox{а}}$я 
версия служит по определению основанием для систематизации 
трансформаций первого и~второго порядка. Поэтому на третьем уровне рис.~2 
указаны типологии без детализации объектов и~без указания числа 
трансформаций в~каждой из них. С~одной стороны, при таком подходе 
получаем достаточно общий вид этой классификации, так как она не зависит от 
числа объектов в~том или ином варианте средов$\acute{\mbox{о}}$й версии 
(и~это существенно упрощает рис.~2). С~другой стороны, на третьем уровне 
такой общей классификации подразумевается, но не эксплицируется природа 
трансформируемых объектов и~их возможные сочетания в~трансформациях. 

При проектировании лексикографической информационной системы важно 
эксплицировать природу трансформируемых объектов и~их возможные 
сочетания.
  %
  Поэтому в~парадигму информатики~\cite{30-zac} кроме общей 
классификации трансформаций предлагается включать и~ее частные случаи, 
эксплицирующие природу трансформируемых объектов. 

В~этом разделе 
рассмотрим один частный случай, когда используются только естественные 
знаковые сис\-те\-мы из типологии А.~Соломоника~\cite{25-zac} вместе 
с~данными, знанием и~его концептами. Чис\-ло естественных языков при этом не 
ограничено. И~этот частный случай классификации включает только три 
класса природных трансформаций (первого, второго и~третьего порядка, см.\ 
схему классификации на рис.~3).
  
  Первый и~второй уровни схемы общей классификации (см.\ рис.~2) можно 
объединить в~один уровень в~этом частном случае. Ниже этого уровня 
приведено содержание типологий первого и~второго порядка без содержания 
типологий третьего по\-рядка.




  Наполнение типологий первого и~второго порядка соответствует 
средов$\acute{\mbox{о}}$й версии иерархии Акоффа на рис.~1, содержащей 
6~видов трансформаций типологии первого порядка и~10~видов 
трансформаций типологии второго порядка (на рис.~3 стрелки указывают 
направления трансформаций согласно средов$\acute{\mbox{о}}$й версии на рис.~1).
  
  Таким образом, частный случай классификации содержит для этих двух 
типологий 16~теоретически возможных трансформаций, 6 из которых 
в~настоящее время в~существующих ИТ применяются наиболее часто: виды 
трансформаций~1, 2, 11 и~12 типологии второго порядка реализуются 
с~помощью тех или иных методов ко\-ди\-ро\-ва\-ния/де\-ко\-ди\-ро\-ва\-ния 
(например, с~использованием таблиц Unicode), а~виды трансформаций~13 и~15
 в~типологии первого порядка реализуются полностью с~по\-мощью процессов 
цифровой обработки компьютерами.
  
  Остальные виды трансформаций или применяются намного реже (это 
виды~3, 5, 7, 9 и~10), или находятся в~стадии поиска и~разработки (14 и~16) или 
в~настоящее время носят только теоретический характер, обеспечивая полноту 
первой и~второй типологий (4, 6 и~8). Знаком~<<?>> обозначены те виды 
трансформаций, которые по определению не существуют в~используемой 
парадигме информатики~\cite{30-zac}. Однако возможно, что в~других 
будущих подходах к~построению ее парадигмы эти виды трансформаций будут 
существовать.
  
\section{Заключение}

  На сегодняшний день процесс построения классификаций объектов 
предметной области информатики~\cite{22-zac} и~их  
трансформаций~\cite{23-zac} еще не завершен. Однако первые результаты их 
построения уже используются для создания концепции лексикографической 
информационной сис\-те\-мы, обеспечивающей интеграцию двуязычных 
словарей и~параллельных корпусов.
  
  \bigskip
  
  
  Автор признателен рецензентам за помощь в~улучшении статьи.
  
{\small\frenchspacing
 { %\baselineskip=10.6pt
 %\addcontentsline{toc}{section}{References}
 \begin{thebibliography}{99}
\bibitem{1-zac}
\Au{Aijmer K., Altenberg~B.} Advances in corpus-based contrastive linguistics. Studies in honour 
of Stig Johansson.~--- Amsterdam: John Benjamins, 2013. 295~p.  doi: 10.1075/scl.54.
\bibitem{2-zac}
\Au{Добровольский Д.\,О., Кретов~А.\, А., Шаров~С.\,А.} Корпус параллельных текстов~// 
Научная и~техническая информация. Сер.~2: Информационные процессы и~сис\-те\-мы, 2005. 
№\,6. С.~16--27.
\bibitem{3-zac}
\Au{Добровольский Д.\,О.} Корпус параллельных текстов и~сопоставительная 
лексикология~// Труды Института русского языка им.\ В.\,В.~Виноградова, 2015. №\,6. 
С.~413--449. EDN: VJQBHP.
\bibitem{4-zac}
\Au{Гончаров А.\,А., Зацман~И.\,М., Кружков~М.\,Г.} Эволюция классификаций 
в~надкорпусных базах данных~// Информатика и~её применения, 2020. Т.~14. Вып.~4. 
С.~108--116. doi: 10.14357/19922264200415.  
EDN: \mbox{GKWBZT}.
\bibitem{5-zac}
\Au{Гончаров А.\, А., Зацман И. \,М., Кружков~М.\, Г}. Представление новых 
лексикографических знаний в~динамических классификационных сис\-те\-мах~// 
Информатика и~её применения, 2021. Т.~15. Вып.~1. С.~86--93.  doi: 10.14357/19922264210112. EDN: OPEFXW.
\bibitem{6-zac}
\Au{Zatsman I.} Finding and filling lacunas in linguistic typologies~// 15th Forum (International) 
on Knowledge Asset Dynamics Proceedings.~--- Matera, Italy: Institute of Knowledge Asset 
Management, 2020. P.~780--793.
\bibitem{7-zac}
\Au{Zatsman I.} Three-dimensional encoding of emerging meanings in AI-systems~// 21st 
European Conference on Knowledge Management Proceedings.~--- Reading, U.K.: Academic 
Publishing International Ltd., 2020. P.~878--887.
\bibitem{8-zac}
\Au{Ackoff R.} From data to wisdom~// J.~Applied Systems Analysis, 1989. Vol.~16. No.\,1. P.~3--9.
\bibitem{9-zac}
\Au{Rosenbloom P.\,S.} On computing: The fourth great scientific domain.~--- Cambridge, MA, 
USA: MIT Press, 2013. 307~p.
\bibitem{10-zac}
\Au{Rowley J.} The wisdom hierarchy: Representations of the DIKW hierarchy~// J.~Inf. 
Sci., 2007. Vol.~33. Iss.~2. P.~163--180. doi: 10.1177/0165551506070706.
\bibitem{11-zac} 
\Au{Frick$\acute{\mbox{e}}$~M.\,H.} Data--Information--Knowledge--Wisdom (DIKW) pyramid, 
framework, continuum~// Encyclopedia of big data~/ Eds. L.~Schintler, C.~McNeely.~--- Cham: 
Springer, 2018. 4~p. doi: 10.1007/978-3-319-32001-4\_331-1.
\bibitem{12-zac}
\Au{Denning P., Rosenbloom~P.} Computing: The fourth great domain of science~// Commun. 
ACM, 2009. Vol.~52. Iss.~9. P.~27--29.
\bibitem{13-zac}
\Au{Denning P., Freeman~P.} Computing's paradigm~// Commun.  ACM, 2009. Vol.~52. 
Iss.~12. P.~28--30. doi: 10.1145/ 1610252.1610265.
\bibitem{17-zac} %14
\Au{Farradane J.} Knowledge, information, and information science~// J.~Inf. Sci., 
1980. Vol.~2. Iss.~2. P.~75--80. doi: 10.1177/01655515800020020.

\bibitem{15-zac}
\Au{Шрейдер Ю.\,А.} Информация и~знание~// Сис\-тем\-ная концепция информационных 
процессов.~--- М.: ВНИИСИ, 1988. С.~47--52.
\bibitem{16-zac}
\Au{Ingwersen P.} Information and information science~// Enclyclopaedie of library and 
information science~/ Eds. J.\,D.~McDonald, 
M.~Levine-Clark.~--- New York, NY, USA: Marcel Dekker Inc., 1992. Vol.~56. Sup.~19. 
P.~137--174.

\bibitem{14-zac} %17
Информатика как наука об информации: Информационный, документальный, 
технологический, экономический, социальный и~организационный аспекты~/ Под ред. 
Р.\,С.~Гиляревского.~--- М.: Фаир-Пресс, 2006. 592~с.

\bibitem{18-zac}
\Au{Hjorland B.} Library and information science: practice, theory, and philosophical basis~// 
Inform. Process. Manag., 2000. Vol.~36. Iss.~3. P.~501--531. doi:  
10.1016/S0306-\mbox{4573(99)00038-2}.
\bibitem{19-zac}
Deep shift~--- technology tipping points and societal impact.~--- Geneva: WE Forum, 2015. 44~p. 
{\sf http://www3.weforum.org/docs/WEF\_GAC15\_ Technological\_Tipping\_Points\_report\_2015.pdf}.
\bibitem{20-zac}
\Au{Berman F., Rutenbar~R., Hailpern~B., Christensen~H., Davidson~S., Estrin~D., 
Franklin~M., Martonosi~M., Raghavan~P., Stodden~V., Szalay~A.\,S.} Realizing the potential of 
data science~// Commun.  ACM, 2018. Vol.~61. Iss.~4. P.~67--72. doi: 10.1145/3188721.

\bibitem{21-zac}
\Au{Stodden V.} The data science life cycle: A~disciplined approach to advancing data science as 
a~science~// Commun.  ACM, 2020. Vol.~63. Iss.~7. P.~58--66. doi: 10.1145/ 3360646.


\bibitem{23-zac} %22
\Au{Зацман И.\,М.} Научная парадигма информатики: классификация трансформаций 
объектов предметной об\-ласти~// Системы и~средства информатики, 2023. Т.~33. №\,4. 
С.~126--138. doi: 10.14357/08696527230412. EDN: ZIKUWO.

\bibitem{22-zac} %23
\Au{Зацман И.\,М.} Научная парадигма информатики: классификация объектов предметной  
об\-ласти~// Информатика и~её применения, 2023. Т.~17. Вып.~4. С.~96--103. doi: 
10.14357/19922264230413. EDN: FIUQAT.

\bibitem{24-zac}
\Au{Зацман И.\,М.} О~научной парадигме информатики: верхний уровень классификации 
объектов ее предметной об\-ласти~// Информатика и~её применения, 2022. Т.~16. Вып.~4. 
С.~73--79. doi: 10.14357/ 19922264220411. EDN: XZNKVI.

\bibitem{25-zac}
\Au{Соломоник А.\,Б.} Философия знаковых систем и~язык.~--- М.: ЛКИ, 2011. 408~с.
\bibitem{26-zac}
\Au{Зацман И.\,М.} Трансформация иерархии Акоффа в~научной парадигме информатики~// 
Информатика и~её применения, 2023. Т.~17. Вып.~3. С.~107--113. doi: 
10.14357/19922264230315. EDN: UMVRRV.

\bibitem{27-zac}
\Au{Zatsman I.} Building digital spiral models of knowledge generation~// 19th Forum 
(International) on Knowledge Asset Dynamics Proceedings.~--- Matera, Italy: Arts for Business 
Institute, 2024. P.~2185--2196.
\bibitem{28-zac}
\Au{Zatsman I.} Digital spiral model of knowledge creation and encoding its dynamics~// 18th 
Forum (International) on Knowledge Asset Dynamics Proceedings.~--- Matera, Italy: Arts for 
Business Institute, 2023. P.~581--596.
\bibitem{29-zac}
\Au{Зацман И.\,М.} Интерфейсы третьего порядка в~информатике~// Информатика и~её 
применения, 2019. Т.~13. Вып.~3. С.~82--89. doi: 10.14357/19922264190312. EDN: 
EHRQLF.

\bibitem{30-zac}
\Au{Зацман И.\,М.} Научная парадигма информатики как третьей культуры~//  
На\-уч\-но-тех\-ни\-че\-ская информация. Сер.~1: Организация и~методика информационной 
работы, 2023. №\,11. С.~1--14.

\end{thebibliography}

 }
 }

\end{multicols}

\vspace*{-9pt}

\hfill{\small\textit{Поступила в~редакцию 14.04.24}}

\vspace*{4pt}

%\pagebreak

%\newpage

%\vspace*{-28pt}

\hrule

\vspace*{2pt}

\hrule



\def\tit{OBJECT TRANSFORMATIONS OF~THE~FIRST AND~SECOND ORDER
IN~A~LEXICOGRAPHIC INFORMATION SYSTEM\\[-5pt]}


\def\titkol{Object transformations of~the~first and~second order
in~a~lexicographic information system}


\def\aut{I.\,M.~Zatsman}

\def\autkol{I.\,M.~Zatsman}

\titel{\tit}{\aut}{\autkol}{\titkol}

\vspace*{-13pt}


\noindent
Federal Research Center ``Computer Science and Control'' of the Russian Academy of Sciences, 
44-2~Vavilov Str., Moscow 119133, Russian Federation


\def\leftfootline{\small{\textbf{\thepage}
\hfill INFORMATIKA I EE PRIMENENIYA~--- INFORMATICS AND
APPLICATIONS\ \ \ 2024\ \ \ volume~18\ \ \ issue\ 2}
}%
 \def\rightfootline{\small{INFORMATIKA I EE PRIMENENIYA~---
INFORMATICS AND APPLICATIONS\ \ \ 2024\ \ \ volume~18\ \ \ issue\ 2
\hfill \textbf{\thepage}}}

\vspace*{2pt}



\Abste{The theoretical foundations of the design of information technologies used for 
the integration of bilingual dictionaries and parallel corpora are considered. The 
description of the first outcomes of the creation of the third\linebreak\vspace*{-12pt}}

\Abstend{ level of object 
transformations classification in the subject domain of informatics, which is supposed 
to be used
in creating the lexicographic information system providing integration, is 
given. All the entities of informatics are divided into two global classes: objects and 
their transformations. For each such class, its own classification is constructed. 
Previously, the two upper levels of the object transformation classification in the subject 
domain have been described. The present paper discusses the third level of this classification. The 
basis for the construction of its highest level was the division of the subject domain of 
informatics into media (mental, sensory, digital, and a~number of other media), each 
of which by definition includes objects of the same nature. The Solomonick's 
typology of sign systems served as the basis for constructing the second level of the 
object transformation classification. The aim of the paper is to systematize object 
transformations of the first and second orders at the third level of this classification. 
The basis for systematization is the medium version of the Ackoff's hierarchy.}

\KWE{subject domain objects; object transformations; classification; data; 
information; knowledge; lexicographic information system}


\DOI{10.14357/19922264240211}{VZTGVV}

\vspace*{-12pt}

\Ack

\vspace*{-3pt}


\noindent
The reported study was funded by the Russian Science Foundation, project  
No.\,24-18-00155, {\sf 
https://rscf.ru/project/24-18-00155}. The research was carried out using the infrastructure of the Shared 
Research Facilities ``High Performance Computing and Big Data'' (CKP 
``Informatics'') of FRC CSC RAS (Moscow) .
   


  \begin{multicols}{2}

\renewcommand{\bibname}{\protect\rmfamily References}
%\renewcommand{\bibname}{\large\protect\rm References}

{\small\frenchspacing
 {%\baselineskip=10.8pt
 \addcontentsline{toc}{section}{References}
 \begin{thebibliography}{99} 
\bibitem{1-zac-1}
\Aue{Aijmer, K., and B.~Altenberg.} 2013. \textit{Advances in corpus-based 
contrastive linguistics. Studies in honour of Stig Johansson}. Amsterdam: John 
Benjamins. 295~p. doi: 10.1075/scl.54.
\bibitem{2-zac-1}
\Aue{Dobrovolskiy, D.\,O., A.\,A.~Kretov, and S.\,A.~Sharov.} 2005. Korpus 
parallel'nykh tekstov [Corpus of parallel texts]. \textit{Nauchnaya i~tekhnicheskaya 
informatsiya. Ser. 2. Informatsionnye protsessy i~sistemy} [Scientific and Technical 
Information. Ser.~2: Information Processes and Systems] 6:16--27.
\bibitem{3-zac-1}
\Aue{Dobrovolskiy, D.\,O.} 2015. Korpus parallel'nykh tekstov i~sopostavitel'naya 
leksikologiya [The corpus of parallel texts and contrastive lexicology]. \textit{Trudy 
Instituta russkogo yazyka im. V.\,V.~Vinogradova} [Proceedings of the 
V.\,V.~Vinogradov Russian Language Institute] 6:413--449. EDN: VJQBHP.
\bibitem{4-zac-1}
\Aue{Goncharov, A.\,A., I.\,M.~Zatsman, and M.\,G.~Kruzhkov.} 2020. Evolyutsiya 
klassifikatsiy v~nadkorpusnykh ba\-zakh dannykh [Evolution of classifications in 
supracorpora databases]. \textit{Informatika i~ee Primeneniya~--- Inform. \mbox{Appl.}}  
14(4):108--116. doi: 10.14357/19922264200415.  
EDN: GKWBZT.
\bibitem{5-zac-1}
\Aue{Goncharov, A.\,A., I.\,M.~Zatsman, and M.\,G.~Kruzhkov.} 2021. 
Predstavlenie novykh leksikograficheskikh znaniy v~dinamicheskikh 
klassifikatsionnykh sistemakh [Representation of new lexicographical knowledge in 
dynamic classification systems]. \textit{Informatika i~ee Primeneniya~--- Inform. 
Appl.} 15(1):86--93. doi: 10.14357/19922264210112. EDN: OPEFXW.
\bibitem{6-zac-1}
\Aue{Zatsman, I.} 2020. Finding and filling lacunas in linguistic typologies. 
\textit{15th Forum (International) on Knowledge Asset Dynamics Proceedings}. 
Matera, Italy: Institute of Knowledge Asset Management. 780--793.
\bibitem{7-zac-1}
\Aue{Zatsman, I.} 2020. Three-dimensional encoding of emerging meanings in  
AI-systems. \textit{21st European Conference on Knowledge Management 
Proceedings}. Reading, U.K.: Academic Publishing International Ltd. 878--887.
\bibitem{8-zac-1}
\Aue{Ackoff, R.} 1989. From data to wisdom. \textit{J.~Applied Systems Analysis} 
16(1):3--9.
\bibitem{9-zac-1}
\Aue{Rosenbloom, P.\,S.} 2013. \textit{On computing: The fourth great scientific 
domain}. Cambridge, MA: MIT Press. 307~p.
\bibitem{10-zac-1}
\Aue{Rowley, J.} 2007. The wisdom hierarchy: Representations of the DIKW 
hierarchy. \textit{J.~Inf. Sci.} 33(2):163--180. doi: 10.1177/0165551506070706.
\bibitem{11-zac-1}
\Aue{Frick$\acute{\mbox{e}}$, M.\,H.} 2018.  
Data-Information-Knowledge-Wisdom (DIKW) pyramid, framework, continuum. 
\textit{Encyclopedia of big data}. Eds. L.~Schintler and C.~McNeely. Cham: 
Springer. 4~p. doi: 10.1007/978-3-319-32001- 4\_331-1.
\bibitem{12-zac-1}
\Aue{Denning, P., and P.~Rosenbloom.} 2009. Computing: The fourth great domain 
of science. \textit{Commun. ACM} 52(9):27--29.
\bibitem{13-zac-1}
\Aue{Denning, P., and P.~Freeman.} 2009. Computing's paradigm. \textit{Commun. 
ACM} 52(12):28--30. doi: 10.1145/ 1610252.1610265.

\bibitem{17-zac-1} %14
\Aue{Farradane, J.} 1980. Knowledge, information, and information science. 
\textit{J.~Inf. Sci.} 2(2):75--80. doi: 10.1177/ 01655515800020020.

\bibitem{15-zac-1}
\Aue{Shreyder, Yu.\,A.} 1988. Informatsiya i~znanie [Information and knowledge]. 
\textit{Sistemnaya kontseptsiya in\-for\-ma\-tsi\-on\-nykh protsessov} [System concept of 
information processes]. Moscow: VNIISI. 47--52.
\bibitem{16-zac-1}
\Aue{Ingwersen, P.} 1995. Information and information science. 
\textit{Encyclopedia of library and information science}. Eds. J.\,D.~McDonald and 
M.~Levine-Clark. New York, NY: Marcel Dekker Inc. 56(19):137--174.

\bibitem{14-zac-1} %17
Gilyarevskiy, R.\,S., ed. 2006. \textit{Informatika kak nauka ob informatsii: 
informatsionnyy, dokumental'nyy, tekh\-no\-lo\-gi\-che\-skiy, ekonomicheskiy, sotsial'nyy 
i~organizatsionnyy aspekty} [Informatics as information science: Informational, 
documentary, technological, economic, social, and organizational dimensions]. 
Moscow: FAIR-PRESS. 592~p.

\bibitem{18-zac-1}
\Aue{Hjorland, B.} 2000. Library and information science: Practice, theory, and 
philosophical basis. \textit{Inform. Process. Manag.} 36(3):501--531. doi:  
10.1016/S0306-\mbox{4573(99)00038-2}.
\bibitem{19-zac-1}
Deep shift~--- technology tipping points and societal impact. 2015. \textit{World Economic 
Forum}. Geneva. 44~p. Available at: {\sf 
http://www3.weforum.org/docs/WEF\_ GAC15\_Technological\_Tipping\_Points\_report\_2015.pdf} (accessed May~20, 
2024).
\bibitem{20-zac-1}
\Aue{Berman, F., R.~Rutenbar, B.~Hailpern, H.~Christensen, S.~Davidson, 
D.~Estrin, M.~Franklin, M.~Martonosi, P.~Raghavan, V.~Stodden, and 
A.\,S.~Szalay.} 2018. Realizing the potential of data science. \textit{Commun. ACM} 
61(4):67--72. doi: 10.1145/3188721.
\bibitem{21-zac-1}
\Aue{Stodden, V.} 2020. The data science life cycle: A~disciplined approach to 
advancing data science as a~science. \textit{Commun. ACM} 
 63(7):58--66. doi: 10.1145/3360646.

\bibitem{23-zac-1} %22
\Aue{Zatsman, I.\,M.} 2023. Nauchnaya paradigma informatiki: klassifikatsiya 
transformatsiy ob''ektov predmetnoy oblasti [Scientific paradigm of informatics: 
Transformation classification of domain objects]. \textit{Sistemy i~Sredstva 
Informatiki~--- Systems and Means of Informatics} 33(4):126--138. doi: 
10.14357/08696527230412. EDN: ZIKUWO.

\bibitem{22-zac-1} %23
\Aue{Zatsman, I.\,M.} 2023. Nauchnaya paradigma informatiki: klassifikatsiya 
ob''ektov predmetnoy oblasti [Scientific paradigm of informatics: Classification of 
domain objects]. \textit{Informatika i~ee Primeneniya~--- Inform. Appl.} 
 17(4):96--103. doi: 10.14357/19922264230413. EDN: FIUQAT.
 
\bibitem{24-zac-1}
\Aue{   Zatsman, I.\,M.} 2022. O nauchnoy paradigme informatiki: verkhniy uroven' 
klassifikatsii ob''ektov ee predmetnoy oblasti [On the scientific paradigm of 
informatics: The classification high level of its objects]. \textit{Informatika i~ee 
Primeneniya~--- Inform. Appl.} 16(4):73--79. doi: 10.14357/19922264220411. EDN: 
XZNKVI.
\bibitem{25-zac-1}
\Aue{Solomonick, A.\,B.} 2011. \textit{Filosofiya znakovykh system i~yazyk} 
[Philosophy of sign systems and language]. Moscow: LKI. 408~p.
\bibitem{26-zac-1}
\Aue{Zatsman, I.\,M.} 2023. Transformatsiya ierarkhii Akoffa v~nauchnoy 
paradigme informatiki [Transformation of the Ackoff's hierarchy in the scientific 
paradigm of informatics]. \textit{Informatika i~ee Primeneniya~--- Inform. \mbox{Appl.}} 
17(3):107--113. doi: 10.14357/19922264230315. EDN: UMVRRV.
\bibitem{27-zac-1}
\Aue{Zatsman, I.} 2024. Building digital spiral models of knowledge 
generation. \textit{19th Forum (International) on Knowledge Asset Dynamics 
Proceedings}. Matera, Italy: Arts for Business Institute. 2185--2196.
\bibitem{28-zac-1}
\Aue{Zatsman, I.} 2023. Digital spiral model of knowledge creation and encoding its 
dynamics. \textit{18th Forum (International) on Knowledge Asset Dynamics 
Proceedings}. Matera, Italy: Arts for Business Institute. 581--596.
\bibitem{29-zac-1}
\Aue{Zatsman, I.\,M.} 2019. Interfeysy tret'ego poryadka v~informatike 
 [Third-order interfaces in informatics]. \textit{Informatika i~ee Primeneniya~--- 
Inform. Appl.} 13(3):82--89. doi: 10.14357/19922264190312. EDN: EHRQLF.
\bibitem{30-zac-1}
\Aue{Zatsman, I.} 2023. Scientific paradigm of informatics as a~third culture. 
\textit{Scientific Technical Information Processing} 50(4):246--258. doi: 
10.3103/S0147688223040111. EDN: CKHMYS.

\end{thebibliography}

 }
 }

\end{multicols}

\vspace*{-6pt}

\hfill{\small\textit{Received April 14, 2024}} 


\vspace*{-12pt}


\Contrl

\vspace*{-3pt}

\noindent
\textbf{Zatsman Igor M.} (b.\ 1952)~--- Doctor of Science in technology, head of 
department, Federal Research Center ``Computer Science and Control'' of the 
Russian Academy of Sciences, 44-2~Vavilov Str., Moscow 119333, Russian 
Federation; \mbox{izatsman@yandex.ru}





\label{end\stat}

\renewcommand{\bibname}{\protect\rm Литература}   %12
\def\stat{nuriev}

\def\tit{МЕТОДОЛОГИЯ КОРПУСНО-ОРИЕНТИРОВАННОГО ИССЛЕДОВАНИЯ 
В~ОБЛАСТИ КОНТРАСТИВНОЙ ПУНКТУАЦИИ$^*$\\[-5pt]}

\def\titkol{Методология корпусно-ориентированного исследования 
в~области контрастивной пунктуации}

\def\aut{В.\,А.~Нуриев$^1$, В.\,И.~Карпов$^2$}

\def\autkol{В.\,А.~Нуриев, В.\,И.~Карпов}

\titel{\tit}{\aut}{\autkol}{\titkol}

\index{Нуриев В.\,А.}
\index{Карпов В.\,И.}
\index{Nuriev V.\,A.}
\index{Karpov V.\,I.}


{\renewcommand{\thefootnote}{\fnsymbol{footnote}} \footnotetext[1]
{Работа выполнена за счет гранта Российского научного фонда (проект 23-28-00548) с~использованием инфраструктуры 
Центра коллективного пользования <<Высокопроизводительные вычисления и~большие данные>> (ЦКП 
<<Информатика>>) ФИЦ ИУ РАН (г.~Москва).}}


\renewcommand{\thefootnote}{\arabic{footnote}}
\footnotetext[1]{Федеральный исследовательский центр <<Информатика и~управление>> Российской академии наук, 
\mbox{nurieff.v@gmail.com}}
\footnotetext[2]{Институт языкознания Российской академии наук; Федеральный исследовательский центр <<Информатика 
и~управ\-ле\-ние>> Российской академии наук, \mbox{wi.karpow@gmail.com}}

\vspace*{-3pt}

  
  
    
  \Abst{Уточняется  подход к~современным исследованиям 
в~об\-ласти контрастивной пунктуации с~точки зрения методологии. С~учетом новейших достижений информатики, 
компьютерной лингвистики и~теории перевода такие исследования очевидным образом 
должны иметь кор\-пус\-но-ори\-ен\-ти\-ро\-ван\-ный характер. В~данной статье представлена 
методологическая схема подобного исследования, направленного на выявление 
межъязыковой пунктуационной асим\-мет\-рии посредством сравнения функционального 
диапазона одного и~того же знака препинания в~разных языках. Показываются основные 
методологические тенденции, характерные для этой научной об\-ласти. Внимание 
уделяется особенностям корпусной методологии при контрастивном изучении 
пунктуации. В~качестве одного из современных методологических инструментов 
предлагаются надкорпусные базы данных (НБД), раз\-ра\-ба\-ты\-ва\-емые в~ФИЦ ИУ РАН.}

%\vspace*{-6pt}
  
  \KW{контрастивная пунктуация; перевод; корпусное переводоведение; кор\-пус\-но-ори\-ен\-ти\-ро\-ван\-ное 
  исследование; параллельный корпус; надкорпусная база данных; 
межъязыковая асим\-мет\-рия; методология}

%\vspace*{-6pt}

\DOI{10.14357/19922264230213}{VBOZAO} 
  
%\vspace*{-3pt}


\vskip 10pt plus 9pt minus 6pt

\thispagestyle{headings}

\begin{multicols}{2}

\label{st\stat}
    
    \section{Введение}
    
    \vspace*{-3pt}
    
  Важность и~необходимость исследований в~области контрастивной 
пунктуации в~научной литературе отмечалась неоднократно (см., 
например,~[1--7]). Обычно эта необходимость выводится из нужд 
переводческой практики, которая предполагает при обработке письменного 
текста обязательную речемыслительную программу, связанную с~исходным 
пунктуационным компонентом и~его переносом в~сис\-те\-му переводящего 
языка. Так, Ньюмарк в~своем <<Учебнике перевода>> пишет, что 
<<пунктуация может быть мощнейшим инструментом, но ее настолько легко 
упус\-тить из виду, что я~советую переводчикам: специально сравнивайте, где 
у~вас рас\-став\-ле\-ны знаки препинания, а~где они стоят 
в~оригинале>>~\cite[с.~58]{4-nu}. В~работе <<Переводчик в~текс\-те: 
о~чтении русской литературы  
по-анг\-лий\-ски>> значение пунктуации отмечает Мей, критикуя 
англоязычных переводчиков за недостаточное внимание к~межъязыковой 
пунктуационной асим\-мет\-рии~--- за <<игнорирование отличительных 
особенностей, присущих знакам препинания>>~\cite[с.~121]{2-nu}. 
О~пунктуации в~переводе говорит Юдейл, выделяя три аспекта:
%\begin{enumerate}[(1)]
%\item 
(1)~<<знаки препинания~--- важ\-ная часть перевода, но, концентрируясь на 
общем смыс\-ле переводимого, ее час\-то не замечают>>; 
%\item 
(2)~<<изменения 
в~пунктуации при переводе могут значительно по\-вли\-ять на вы\-ра\-зи\-тель\-ность 
текс\-та, его свя\-зан\-ность и~ритм>>; 
%\item 
(3)~<<час\-то возникает впечатление, что 
литературные переводчики наделили себя правом менять границы исходного 
предложения и~пунктуационные знаки, как им 
заблагорассудится>>~\cite[с.~121]{5-nu}.
%\end{enumerate}
 Гораздо реже 
в~специализированной литературе подчеркивается роль, которую 
исследования в~об\-ласти контрастивной пунктуации играют при обучении 
иностранным языкам, в~част\-ности при обуче\-нии иноязычной письменной 
речи~\cite{7-nu}.
  
  Признавая безусловную зна\-чи\-мость данного научного на\-прав\-ле\-ния и~его 
дальнейшего развития, необходимо предметно разрабатывать методологию 
исследования в~об\-ласти контрастивной пунктуации, которая учитывала бы 
новейшие достижения информатики, компьютерной лингвистики 
и~корпусного переводоведения. Пред\-став\-ля\-ет\-ся, что такая методология 
долж\-на основываться на использовании современных информационных 
корпусных инструментов, поз\-во\-ля\-ющих автоматизированным образом 
обрабатывать пред\-ста\-ви\-тель\-ные массивы текс\-то\-вых данных, 
и,~следовательно, носить кор\-пус\-но-ори\-ен\-ти\-ро\-ван\-ный характер 
(о~корпусных данных при контрастивном изуче\-нии пунктуации  
см.~\cite{6-nu}).

%\vspace*{-6pt}
    
    \section{Методологические модели  
корпусно-ориентированного исследования контрастивной 
пунктуации}

\vspace*{-3pt}
  
  В мае 2019~г.\ в~Регенсбурге (Германия) про\-шла научная конференция 
под названием <<Punctuation Seen Internationally. System--Norm--Practice>> 
(<<Пунктуация в~мировом мас\-шта\-бе: 
 сис\-те\-ма--нор\-ма--прак\-ти\-ка>>)~--- первая конференция, пол\-ностью\linebreak 
по\-свя\-щен\-ная проб\-ле\-мам контрастивной пунктуации. Оргкомитет, собирая 
заявки на участие, справедливо отмечал, что до на\-сто\-яще\-го времени 
пунктуации едва ли уделялось внимание в~рамках \mbox{типологии}, контрастивной 
лингвистики, прагмалингвистики, а~так\-же в~исследованиях индивидуальной 
языковой манеры на фоне языкового стандарта. Сейчас появляются 
отдельные работы, где проводится сопоставительное изуче\-ние пунктуации, 
однако по-преж\-не\-му ощущается острая не\-об\-хо\-ди\-мость в~исследованиях по 
контрастивной пунктуации, которые бы учитывали типологические 
(сис\-тем\-ные), социолингвистические (нормативные) и~прагматические 
(речевые) ас\-пекты.
  
  Итогом конференции стала коллективная монография~\cite{8-nu}, 
со\-сто\-ящая из шестнадцати статей, которые пред\-став\-ля\-ют собой пио\-нер\-ские 
исследования, на\-прав\-лен\-ные на формирование целостной па\-ра\-диг\-мы 
контрастивного изучения пунктуации и~борьбу с~маргинализацией важ\-ной 
научной от\-расли. Все статьи услов\-но мож\-но разделить на~4~категории, 
первые две из которых имеют в~большей степени тео\-ре\-ти\-че\-ский характер и~связаны с~сис\-те\-мой и~нормой, а~вторые~--- более практической 
на\-прав\-лен\-ности~--- с~узусом и~освоением пунктуационных навыков. 
В~пред\-став\-лен\-ных работах доминируют два подхода к~исследованию 
конт\-растив\-ной пунк\-ту\-ации:
  \begin{enumerate}[(1)]
\item интралингвистический (контрастивный анализ знаков препинания 
и~кон\-ку\-ри\-ру\-ющих с~ними маркеров синтаксических отношений в~рамках 
одного языка)~\cite[с.~110]{9-nu};
  \item  интерлингвистический (контрастивный анализ знаков препинания 
  и~конкурирующих с~ними средств в~разных языках, конт\-растив\-ная пунктуация 
рас\-смат\-ри\-ва\-ет\-ся в~том чис\-ле как часть методики обуче\-ния неродному языку, 
например при интеграции трудовых мигрантов в~иноязычную 
среду)~\cite[с.~57--73]{10-nu}.
  \end{enumerate}
  
  Интралингвистический подход час\-то носит смешанный характер: если 
речь идет об эволюции пунктуационной сис\-те\-мы отдельно взятого языка на 
фоне развития аналогичных сис\-тем других языков, контрастивный анализ 
со\-про\-вож\-да\-ет\-ся 
 ис\-то\-ри\-ко-эти\-мо\-ло\-ги\-че\-ским~\cite[с.~187--206]{11-nu}. В~рамках 
этого подхода в~указанной монографии имеются психолингвистические 
исследования с~нетривиальным корпусным материалом. Так, 
в~статье~\cite[с.~163--186]{12-nu} корпусные данные привлекаются для 
контрастивного анализа пунктуационных предпочтений двух групп 
ис\-пы\-ту\-емых. Автор использует корпус \mbox{CoPaDocs} (Corpus of Patient 
Documents), основу которого со\-ста\-ви\-ли письма и~другие личные документы 
бывших пациентов психиатрических учреж\-де\-ний Германии на рубеже  
XIX--XX~вв. Корпус поз\-во\-ля\-ет установить, зависит ли языковое оформление 
пись\-ма от лич\-ности адресата~--- происходит ли переключение ре\-гист\-ров 
сознательно. Данный корпус создан с~целью разработки интегративной 
методики анализа языковой ва\-риа\-тивн\-ости, в~том чис\-ле и~в~об\-ласти 
пунктуации. \mbox{Изучив} специфику расстановки~12~знаков препинания, 
Эбер-Хам\-мерль приходит к~выводу, что пациенты, чей род де\-я\-тель\-ности 
прежде не был связан с~письменной сферой, использовали больше 
пунктуационных маркеров (но с~меньшей ва\-риа\-тив\-ностью), чем 
представители второй опытной группы~--- канцелярские служащие. 
В~лич\-ной переписке участники обеих групп к~знакам препинания прибегали 
гораздо реже, чем в~документах, адресованных официальным лицам.
  
 В статье~\cite[с.~57--73]{10-nu} представлено контрастивное исследование, выполненное в~интерлигвистическом 
ключе. Со\-по\-став\-ле\-ние 
пунктуации в~италь\-ян\-ском и~немецком языках здесь проводится на основе 
комплексной методологии, вклю\-ча\-ющей приемы дескриптивного, 
просодического, синтаксического и~ком\-му\-ни\-ка\-тив\-но-текс\-то\-во\-го 
анализа. Примеры приводятся из различных источников, причем 
к~корпусным данным в~статье отсылают не напрямую, а~опосредованно~--- 
через более раннюю работу~\cite{13-nu}. По мнению авторов, 
пунктуирование в~этих языках организовано по-раз\-но\-му, что объясняется 
резкими различиями в~пунктуационном узусе: если в~итальянском знаки 
препинания коммуникативно на\-гру\-же\-ны, то в~немецком они подчинены 
фор\-маль\-но-син\-так\-си\-че\-ско\-му принципу. Иначе говоря, итальянская 
пунктуация выполняет не формальную функцию, а~сигнализиру-\linebreak ет о~тон\-ких 
смыс\-ло\-вых нюансах, которых нельзя\linebreak достичь другими языковыми 
средствами (аргументативный конфликт, полифонические эффекты, 
метатекстовые комментарии). В~этом же духе\linebreak выполнена и~другая 
интерлингвистическая работа~\cite{14-nu}, по\-свя\-щен\-ная контрастивному 
исследованию многоточия и~тире в~италь\-ян\-ском и~анг\-лий\-ском языках 
и~продуктивно ис\-поль\-зу\-ющая \mbox{корпусный} метод сбора и~обработки 
эмпирических данных.
     
     Объединенные в~коллективную монографию рабо\-ты позволяют 
вывести обобщенную ме\-то\-до\-ло\-гическую схему контрастивного изуче\-ния 
пунктуации. Она имеет трехфазную структуру. Первая\linebreak фаза включает 
тео\-ре\-ти\-че\-ское описание пунктуации в~изуча\-емом языке с~привлечением 
исторических и~современных нормативных грамматик и~справочников. 
Вторая фаза на\-прав\-ле\-на на описание трансформаций в~других языках, 
оказавших существенное влияние на статус и~мес\-то пунктуации в~сис\-те\-ме 
конкретного языка. Обе фазы нацелены на создание такого 
исследовательского поля, которое поз\-во\-лит выявить значение пунктуации 
для языковой культуры. Это, в~свою очередь, долж\-но стать задачей треть\-ей 
фазы. Вторая и~\mbox{третья} фазы предполагают межъязыковое сравнение как 
функционального диапазона отдельно взятых знаков препинания, так 
и~пунктуационного репертуара в~целом. На этих стадиях применяется 
корпусный метод. Контрастивный анализ в~за\-ви\-си\-мости от по\-став\-лен\-ных 
целей и~задач наряду со знаками препинания может охватывать 
и~кон\-ку\-ри\-ру\-ющие с~ними языковые средства. На\-прав\-ле\-ние контрастивного 
исследования пунктуации может быть и~синхронным, и~диахроническим.

\vspace*{-6pt}
    
    \section{Методологические особенности  
корпусно-ориентированного исследования в~области 
контрастивной пунктуации}

\vspace*{-3pt}
  
  Особенности методологии при корпусном контрастивном изуче\-нии 
пунктуации, как, впрочем, и~при любом 
 кор\-пус\-но-ори\-ен\-ти\-ро\-ван\-ном исследовании, связаны прежде всего 
со стремлением получить непротиворечивые, валидные и~на\-деж\-ные данные. 
Электронный корпус, будучи методологически новаторским инструментом 
для получения научного знания, поз\-во\-ля\-ет, с~одной стороны, автоматическим 
образом обрабатывать большие массивы данных и~тем самым серьезно 
сокращает временные издержки на поиск эмпирического материала. 
С~другой стороны, электронные корпусные ресурсы имеют свои 
особенности, и~без их над\-ле\-жа\-ще\-го учета пользователь рискует получить 
искаженные результаты.
  
  Например, в~указанной выше работе~\cite[с.~291]{14-nu} авторы, описывая 
методологию своего исследования, отмечают, что итальянские примеры 
заимствованы из корпуса, хранящегося в~Базельском университете 
и~со\-сто\-яще\-го из двух частей~--- 33~современных  
ро\-ма\-на-бест\-сел\-ле\-ра (1~млн словоупотреблений) 
и~нехудожественных текс\-та разной на\-прав\-лен\-ности (1~млн 40~тыс.\ 
словоупотреблений), в~то время как англоязычные примеры извлечены из 
подкорпуса <<Книги и~периодические издания>> Британского 
национального корпуса (80~млн словоупотреблений). Итальянский материал, 
по словам авторов, был проанализирован весь, а~для английского из-за 
гораздо большего объема ограничились анализом случайной выборки, объем 
которой со\-по\-ста\-вим с~выборкой из итальянского корпуса. Очевидным 
образом ва\-лид\-ность выводов по результатам анализа англоязычного 
материала здесь может оказаться под вопросом в~силу методологически 
неоднородных установок применительно к~процедуре обработки данных, 
полученных по двум языкам. Примечательно к~тому же, что базельский 
корпус, в~отличие от британского, за\-крыт для общественного пользования.
  
  О подобных ограничениях рассуждает На\-двор\-ни\-ко\-ва в~своей работе, 
по\-свя\-щен\-ной корпусной методологии контрастивного изучения 
пунктуации~\cite{15-nu}, где анализируется час\-тот\-ность упо\-треб\-ле\-ния шести 
знаков препинания (запятой, точки, двоеточия, точ\-ки с~запятой, 
вопросительного и~восклицательного знака) в~английском, французском 
и~чешском языках. Для сбора данных используются со\-по\-ста\-ви\-мые 
веб-кор\-пу\-сы, моноязычные общие (референтные) и~параллельные корпусы. Цель 
автора~--- определить, какой из трех типов корпусных ресурсов наиболее 
подходит для исследований в~об\-ласти контрастивной пунктуации.
  
  Полученные данные показывают, что при изучении пунктуации показатели 
час\-тот\-ности проявляют высокую чув\-ст\-ви\-тель\-ность к~типу текс\-та; 
следовательно, веб-кор\-пу\-сы, которые, как правило, отличают стихийное 
наполнение, не\-упо\-ря\-до\-чен\-ность и~низ\-кая степень струк\-ту\-ри\-ро\-ван\-ности, не 
могут служить источником до\-сто\-вер\-ной информации об упо\-треб\-ле\-нии 
знаков препинания в~том или ином языке. Моноязычный общий корпус, 
наоборот, содержит специальную раз\-мет\-ку (морфологическую, 
синтаксическую и~т.\,д.)\ и~поз\-во\-ля\-ет гиб\-ко настраивать поиск (в~том чис\-ле 
выбирать соответствующий тип текс\-та) в~за\-ви\-си\-мости от конкретных 
исследовательских задач. Такие корпусы располагают большими массивами 
данных, поскольку призваны пред\-ста\-вить язык во всей его пол\-но\-те 
и~многообразии, что, казалось бы, обеспечивает на\-деж\-ность и~ва\-лид\-ность 
полученных результатов. Меж\-ду тем этот тип корпусов имеет существенный 
недостаток~--- ограниченную межъязыковую со\-по\-ста\-ви\-мость. Как правило, 
моноязычные общие корпусы разных языков разительно отличаются по 
объему данных и~их со\-ста\-ву и~поэтому не подходят в~качестве основного 
инструмента контрастивного исследования, а~могут служить лишь 
референтным (проверочным) источником для дополнительной верификации 
ре\-зуль\-ти\-ру\-ющих данных. Кроме того, со\-по\-ста\-ви\-тель\-ный анализ 
относительной час\-тот\-ности упо\-треб\-ле\-ния знаков препинания в~разных 
языках на основе данных, извлеченных из корпусов этого типа, так\-же имеет 
свои ограничения. Он не применим для изучения пунк\-ту\-а\-ции в~языках 
разного строя, которым для кодирования информации требуется 
количественно больше (аналитические языки типа французского) или 
меньше слов (синтетические языки типа русского). Таким образом, лучше 
всего для контрастивного изуче\-ния пунк\-ту\-а\-ции подходят параллельные 
корпусы, которые, не\-смот\-ря на свой сравнительно небольшой объем, 
пред\-став\-ля\-ют существенно больше воз\-мож\-но\-стей для качественного анализа 
упо\-треб\-ле\-ния знаков препинания и~непосредственного со\-по\-став\-ле\-ния их 
абсолютной час\-тот\-ности в~параллельных текс\-тах~--- оригинале и~переводе. 
Однако и~этот тип информационного ресурса не может служить 
универсальным исследовательским инструментом. При его использовании 
необходимо учитывать, что пунктуационные рас\-хож\-де\-ния в~исходном 
и~переводном текс\-те могут быть не результатом сис\-тем\-ных дифференциаций, 
а~возникнуть под влиянием переводческих предпочтений. Следовательно, 
чтобы избежать искажения ре\-зуль\-ти\-ру\-ющих данных, надо следовать 
некоторым методологическим принципам: %\\[-13pt] 
\begin{enumerate}[(1)]
\item данные собираются в~обоих 
переводных на\-прав\-ле\-ни\-ях; %\\[-13pt] 
\item выявленные тенденции проходят 
обязательную проверку с~по\-мощью референтного моноязычного корпуса; %\\[-13pt] 
\item контрастивное изуче\-ние пунктуации с~применением параллельных 
корпусов требует сис\-тем\-но\-го подхода в~том смыс\-ле, что в~функциональном 
диапазоне разных знаков препинания могут быть общие зоны, ука\-зы\-ва\-ющие 
на их потенциальную внут\-ри\-язы\-ко\-вую и~межъ\-язы\-ко\-вую конкуренцию. %\\[-13pt]
\end{enumerate}
    
 \vspace*{-12pt}
 
    \section{Заключение}
    
    \vspace*{-3pt}
    
  В статье представлена обобщенная методологическая схема 
  кор\-пус\-но-ори\-ен\-ти\-ро\-ван\-но\-го 
  исследования в~об\-ласти контрастивной пунктуации~--- 
от\-расли научного знания, интенсивно \mbox{раз\-ви\-ва\-ющей\-ся} и~при\-вле\-ка\-ющей 
внимание специалистов самого широкого профиля. Несмотря на то что 
появляются работы, где описываются сопоставительные исследования 
пунктуации на примере одного произведения или литературного наследия 
отдельно взятого писателя (см., например,~\cite{16-nu,17-nu}), очевидно, что 
для ка\-ких-ли\-бо существенных, круп\-но\-мас\-штаб\-ных обобщений относительно 
межъязыковой пунктуационной асимметрии и~специфики функционирования 
знаков препинания в~разных языках требуется привлечение корпусного 
материала.
  
  Дальнейшее изучение контрастивной пунктуации видится в~нескольких 
направлениях. Необходимо качественное углубление со\-по\-ста\-ви\-тель\-но\-го 
анализа, чтобы его тонкая нюансировка \mbox{поз\-во\-ли\-ла} установить, в~какой мере 
совпадает и~разнится функциональный диапазон того или иного знака 
препинания в~кон\-так\-ти\-ру\-ющих языках в~за\-ви\-си\-мости от жанровой 
при\-над\-леж\-ности текс\-та. Этот анализ целесообразно проводить комплексно, 
охватывая всю со\-во\-куп\-ность синтаксических изменений, которые влекут за 
собой отказ от исходного пунктуирования при переводе с~одного языка на 
другой. Такая ком\-плекс\-ность поможет выявить и~с~большей пол\-но\-той 
описать су\-щест\-ву\-ющие межъ\-язы\-ко\-вые структурные различия, что 
необходимо и~для переводческой практики, и~для обуче\-ния иностранным 
языкам. Требует дальнейшего уточ\-не\-ния вопрос, как на пунктуационные 
преференции переводчика влияет род\-ная языковая культура, 
пунктуационные уста\-нов\-ки которой могут меняться со временем. По мере 
наращивания опыта и~мастерства могут меняться пунктуационные 
предпочтения и~самого переводчика, и~это так\-же пред\-став\-ля\-ет определенный 
научный интерес.
  
  В заключение следует отметить, что одним из современных 
информационных инструментов корпусного исследования в~об\-ласти 
контрастивной пунктуации могут быть НБД, 
раз\-ра\-ба\-ты\-ва\-емые в~отделе~54 Федерального исследовательского цент\-ра 
<<Информатика и~управ\-ле\-ние>> Российской академии наук (о~возможностях 
НБД см.~\cite{6-nu}). В~данный момент этот методологический инструмент 
проходит апро\-ба\-цию в~контрастивном исследовании двоеточия и~многоточия в~трех языках~--- русском, французском и~немецком.

\vspace*{-9pt}
  
{\small\frenchspacing
 {\baselineskip=11.5pt
 %\addcontentsline{toc}{section}{References}
 \begin{thebibliography}{99}
 
 \vspace*{-3pt}
 
 \bibitem{4-nu} %1
\Au{Newmark P.} A~textbook of translation.~--- New York, London, Toronto, Sydney, Tokyo: Prentice 
Hall, 1988. 402~p.
 

\bibitem{2-nu} %2
\Au{May R.} The translator in the text: On reading Russian literature in English.~--- Evanston, IL, USA: 
Northwestern University Press, 1994. 209 p.
\bibitem{3-nu}
\Au{Munday J.} Systems in translation: A~systemic model for descriptive translation studies~// 
Crosscultural transgressions: Research models in translation studies II~--- historical and 
ideological issues~/ Ed. T.~Hermans.~---  Manchester, U.K.: St.\ Jerome, 2002. P.~76--92.
\bibitem{1-nu} %4
\Au{Baker M.} In other words.~--- 2nd ed.~--- London, New York: Routledge, 2011. 352~p.

\bibitem{7-nu} %5
\Au{Сигал К.\,Я.} Контрастивная пунктуация в~начале XXI века~// Язык. Текст. Дискурс: 
Научный альманах Ставропольского отделения РАЛК.~--- Ставрополь: СКФУ, 
2019.  Вып.~17. С.~69--78.
\bibitem{5-nu} %6
\Au{Youdale R.} Using computers in the translation of literary style: Challenges and 
opportunities.~--- London, New York: Routledge, 2020. 242~p.
\bibitem{6-nu} %7
\Au{Нуриев В.\,А., Кружков~М.\,Г.} Корпусные данные при контрастивном изуче\-нии 
пунктуации~// Сис\-те\-мы и~средства информатики, 2023. Т.~33. №\,1. С.~14--23. doi: 10.14357/08696527230102.

\bibitem{8-nu}
Vergleichende Interpunktion~--- comparative punctuation~/ Eds. P.~R$\ddot{\mbox{o}}$ssler, P.~Besl, A.~Saller.~--- 
Berlin, Boston: De Gruyter, 2021. 454~p.
\bibitem{9-nu}
\Au{Rinas K.} Vom genormten Satzbau zur genormten Interpunktion. Zur Funktion der 
Zeichensetzung in $\ddot{\mbox{a}}$lterer und neuerer Zeit~// Vergleichende Interpunktion~--- comparative 
punctuation~/ Eds. P.~R$\ddot{\mbox{o}}$ssler, P.~Besl, A.~Saller.~---
 Berlin, Boston: De Gruyter, 2021. P.~109--136. doi: 10.1515/9783110756319-006.
\bibitem{10-nu}
\Au{Ferrari~A., Stojmenova Weber R.} Das Komma in kontrastiver Perspektive Italienisch-Deutsch~// Vergleichende Interpunktion~--- 
comparative punctuation / Eds. P.~R$\ddot{\mbox{o}}$ssler, P.~Besl, 
A.~Saller.~--- Berlin, Boston: De Gruyter, 2021. P.~57--73. doi: 10.1515/9783110756319-003.

\columnbreak

\bibitem{11-nu}
\Au{Besch W.} Zur Entwicklung der deutschen Interpunktion seit dem sp$\ddot{\mbox{a}}$ten Mittelalter~// 
Interpretation und Edition deutscher Texte des Mittelalters. Festschrift f$\ddot{\mbox{u}}$r John Asher zum 60. 
Geburtstag~/ Eds. K.~Smits, W.~Besch, V.~Lange.~--- Berlin: Erich Schmidt, 1981. P.~187--206.
\bibitem{12-nu}
\Au{Eber-Hammerl F.} Interpunktion in historischen Patientenbriefen // Vergleichende
Interpunktion~--- comparative punctuation~/ Eds. P.~R$\ddot{\mbox{o}}$ssler, 
P.~Besl, A.~Saller.~--- Berlin, Boston: De Gruyter, 2021. P.~163--186.
\bibitem{13-nu}
\Au{Ferrari A.} Leggere la virgola. Una prima ricognizione~// Chimera Romance Corpora 
Linguistic Studies, 2017. Vol.~4. Iss.~2. P.~145--162. doi: 
10.15366/chimera2017. 4.2.001.
\bibitem{14-nu}
\Au{Pecorari F., Longo~F.} The ellipsis and the dash in Italian and English: A~contrastive 
perspective~// Vergleichende Interpunktion~--- comparative punctuation~/ Eds.
 P.~R$\ddot{\mbox{o}}$ssler, P.~Besl, A.~Saller.~--- Berlin, Boston: De Gruyter, 2021. P.~289--314.
 doi: 10.1515/9783110756319-013.
\bibitem{15-nu}
\Au{N$\acute{\mbox{a}}$dvorn$\acute{{\iota}}$kov$\acute{\mbox{a}}$~O.}
The use of English, Czech and French punctuation marks in reference, 
parallel and comparable web corpora: A~question of methodology~// 
Linguist. Prag.,  2020. Vol.~30. Iss.~2. P.~30--50. doi: 
10.14712/ 18059635.2020.1.2.
\bibitem{16-nu}
\Au{Сигал К.\,Я.} Пунктуация как средство создания эмоционального под\-текс\-та (на 
материале рассказа М.\,А.~Шолохова <<Судьба человека>> и~его переводов на английский 
язык)~// Известия РАН. Серия литературы и~языка, 2014. Т.~73. №\,6. С.~38--50.
\bibitem{17-nu}
\Au{Богданов К.\,А.} Пунктуация как мотив: многоточие и~тире~// НЛО, 2022. №\,2(174). С.~241--253.
doi: 0.53953/ 08696365\_2022\_174\_2\_241.

\end{thebibliography}

 }
 }

\end{multicols}

\vspace*{-8pt}

\hfill{\small\textit{Поступила в~редакцию 15.04.23}}

\vspace*{6pt}

%\pagebreak

%\newpage

%\vspace*{-28pt}

\hrule

\vspace*{2pt}

\hrule

\vspace*{-2pt}

\def\tit{METHODOLOGY OF~THE~CORPUS-BASED STUDIES\\ 
IN~THE~FIELD OF~CONTRASTIVE PUNCTUATION}


\def\titkol{Methodology of~the~corpus-based studies 
in~the~field of~contrastive punctuation}


\def\aut{V.\,A.~Nuriev$^1$ and~V.\,I.~Karpov$^{1,2}$}

\def\autkol{V.\,A.~Nuriev and~V.\,I.~Karpov}

\titel{\tit}{\aut}{\autkol}{\titkol}

\vspace*{-14pt}


\noindent
      $^1$Federal Research Center ``Computer Science and Control'' of the Russian 
Academy of Sciences, 44-2~Vavilov\linebreak
$\hphantom{^1}$Str., Moscow 119333, Russian Federation
      
      \noindent
      $^2$Institute of Linguistics of the Russian Academy of Sciences, 1~bld.~1 
Bolshoy Kislovsky Lane, Moscow 125009,\linebreak
$\hphantom{^1}$Russian Federation

\def\leftfootline{\small{\textbf{\thepage}
\hfill INFORMATIKA I EE PRIMENENIYA~--- INFORMATICS AND
APPLICATIONS\ \ \ 2023\ \ \ volume~17\ \ \ issue\ 2}
}%
 \def\rightfootline{\small{INFORMATIKA I EE PRIMENENIYA~---
INFORMATICS AND APPLICATIONS\ \ \ 2023\ \ \ volume~17\ \ \ issue\ 2
\hfill \textbf{\thepage}}}

\vspace*{3pt}
      
      
    
    \Abste{The paper refines the methodological approach to the contrastive 
studies of punctuation. Given the recent achievements of information science, 
computer linguistics, and translation theory, such studies are most likely to be 
corpus-based. The paper presents a~methodological model of research into 
interlingual punctuation asymmetry, the aim of which is to shed light on the 
functional scope of the same punctuation marks in different languages. It shows 
what methodological trends are characteristic of this research area. The focus is 
also on the specificities of corpus methodology in the contrastive study of 
punctuation. It is argued that one of the methodological tools, tailored specifically 
to the needs of contrastive punctuation research, may be the supracorpora 
databases developed at the Federal Research Center ``Computer Science and 
Control'' of the Russian Academy of Sciences.}
    
    \KWE{contrastive punctuation; translation; corpus-based translation studies; 
corpus-based studies; parallel corpus; supracorpora database; asymmetry between 
languages; methodology}
    
    
    
\DOI{10.14357/19922264230213}{VBOZAO}

%\vspace*{-18pt}

\Ack
    \noindent
    The research was carried out using the infrastructure of the Shared Research 
Facilities ``High Performance Computing and Big Data'' (CKP ``Informatics'') of 
FRC CSC RAS (Moscow). The research was supported by the Russian Science Foundation (project  
No.\,23-28-00548).
 
%\vspace*{4pt}

  \begin{multicols}{2}

\renewcommand{\bibname}{\protect\rmfamily References}
%\renewcommand{\bibname}{\large\protect\rm References}

{\small\frenchspacing
 {%\baselineskip=10.8pt
 \addcontentsline{toc}{section}{References}
 \begin{thebibliography}{99}
 
 \bibitem{4-nu-1} %1
\Aue{Newmark, P.} 1988. \textit{A~textbook of translation}. New York, London, Toronto, Sydney, Tokyo: 
Prentice Hall. 402~p.   

\bibitem{2-nu-1}
\Aue{May, R.} 1994. \textit{The translator in the text: On reading Russian 
literature in English}. Evanston, IL: Northwestern University Press. 209~p.
\bibitem{3-nu-1}
\Aue{Munday, J.} 2002. Systems in translation: A~systemic model for 
descriptive translation studies. \textit{Crosscultural transgressions: Research models in 
translation studies II~--- historical and ideological issues}. Ed. T.~Hermans. 
Manchester, U.K.: St.\ Jerome. 76--92.

\bibitem{1-nu-1} %4
\Aue{Baker, M.} 2011. \textit{In other words}. 2nd ed. London, New York: 
Routledge. 352~p.

\bibitem{7-nu-1} %5
\Aue{Seagal, K.\,Ya.} 2019. Kont\-ras\-tiv\-naya punk\-tu\-a\-tsiya v~na\-cha\-le XXI~veka 
[Contrastive punctuation at the beginning of the XXI century]. \textit{Yazyk. Tekst. 
Diskurs: Nauchnyy al'manakh Stavropol'skogo otdeleniya RALK} [Language. Text. 
Discourse: Scientific almanac of Stavropol Branch of the Russian Cognitive 
Linguists Association].  Stavropol': SKFU. 17:69--78.

\bibitem{5-nu-1} %6
\Aue{Youdale, R.} 2020. \textit{Using computers in the translation of literary style: 
Challenges and opportunities}. London, New York: Routledge. 242~p.
\bibitem{6-nu-1} %7
\Aue{Nuriev, V.\,A., and M.\,G.~Kruzhkov.} 2023. Kor\-pus\-nye dan\-nye pri 
kont\-ras\-tiv\-nom izu\-che\-nii punk\-tu\-a\-tsii [The parallel corpora perspective on studying 
contrastive punctuation]. \textit{Sistemy i~Sredstva Informatiki~--- Systems and Means of 
Informatics} 33(1):14--23. doi: 10.14357/08696527230102.

  \bibitem{8-nu-1}
R$\ddot{\mbox{o}}$ssler, P., P.~Besl, and A.~Saller, eds. 2021. \textit{Vergleichende 
Interpunktion~--- comparative punctuation}. Berlin, Boston: De Gruyter. 454~p.
\bibitem{9-nu-1}
\Aue{Rinas, K.} 2021. Vom genormten satzbau zur genormten interpunktion. 
Zur funktion der zeichensetzung in $\ddot{\mbox{a}}$lterer und neuerer zeit. \textit{Vergleichende 
Interpunktion~--- comparative punctuation}. Eds.\ P.~R$\ddot{\mbox{o}}$ssler, 
P.~Besl, and A.~Saller. 
Berlin, Boston: De Gruyter. 109--136. doi: 10.1515/ 9783110756319-006.
\bibitem{10-nu-1}
\Aue{Ferrari, A., and R.~Stojmenova.} 2021. Weber das komma in kontrastiver 
perspektive Italienisch-Deutsch. \textit{Vergleichende Interpunktion~--- comparative 
punctuation}. Eds. P.~R$\ddot{\mbox{o}}$ssler, P.~Besl, and A.~Saller. Berlin, Boston: De Gruyter.  
57--73. doi: 10.1515/9783110756319-003.
 \bibitem{11-nu-1}
\Aue{Besch, W.} 1981. Zur entwicklung der deutschen interpunktion seit 
dem sp$\ddot{\mbox{a}}$ten mittelalter. \textit{Interpretation und Edition deutscher Texte des Mittelalters. 
Festschrift f$\ddot{\mbox{u}}$r John Asher zum 60. Geburtstag}. Eds. K.~Smits, W.~Besch, and 
V.~Lange. Berlin: Erich Schmidt. 187--206.
 \bibitem{12-nu-1}
\Aue{Eber-Hammerl, F.} 2021. Interpunktion in historischen 
Patientenbriefen. \textit{Vergleichende Interpunktion~--- comparative punctuation}. Eds. 
P.~R$\ddot{\mbox{o}}$ssler, P.~Besl, and A.~Saller. Berlin, Boston: De Gruyter. 163--186.
\bibitem{13-nu-1}
\Aue{Ferrari, A.} 2017. Leggere la virgola. Una prima ricognizione. 
\textit{Chimera Romance Corpora Linguistic Studies} 4(2):145--162. doi: 
10.15366/chimera2017.4.2.001.
\bibitem{14-nu-1}
\Aue{Pecorari, F., and F.~Longo.} 2021. The ellipsis and the dash in Italian 
and English: A~contrastive perspective. \textit{Vergleichende Interpunktion~--- 
comparative punctuation}. Eds. P.~R$\ddot{\mbox{o}}$ssler, P.~Besl, and A.~Saller. Berlin, Boston: 
De Gruyter. 289--314. doi: 10.1515/9783110756319-013.
\bibitem{15-nu-1}
\Aue{N$\acute{\mbox{a}}$dvorn$\!\acute{\mbox{\ptb{\i}}}$kov$\acute{\mbox{a}}$,~O.} 2020. The use of English, Czech and French 
punctuation marks in reference, parallel and comparable web corpora: A~question 
of methodology. \textit{Linguist. Prag.} 30(2):30--50. doi: 
10.14712/18059635.2020.1.2.
\bibitem{16-nu-1}
\Aue{Seagal, K.\,Ya.} 2014. Punk\-tu\-a\-tsiya kak sred\-st\-vo so\-zda\-niya 
emo\-tsi\-o\-nal'\-no\-go pod\-teks\-ta (na ma\-te\-ri\-ale ras\-ska\-za M.\,A.~Sho\-lo\-kho\-va ``Sud'\-ba 
che\-lo\-ve\-ka'' i~ego pe\-re\-vo\-dov na ang\-liy\-skiy yazyk) [Punctuation as a means of 
revealing the emotional subtext (the case of Mikhail Sholokhov's short story ``The 
Fate of a~Man'' and its translations into English)]. \textit{Izvestiya RAN. Seriya literatury i~yazyka}
 [The Bulletin of the Russian Academy of Sciences: Studies in Literature 
and Language]. 73(6):38--50.
\bibitem{17-nu-1}
\Aue{Bogdanov, K.\,A.} 2022. Punk\-tu\-a\-tsiya kak mo\-tiv: mno\-go\-to\-chie i~ti\-re 
[Punctuation as a~motive: The ellipsis and the dash]. \textit{NLO} [New Literary Observer] 
2(174):241--253. doi: 0.53953/08696365\_2022\_174\_2\_241.
\end{thebibliography}

 }
 }

\end{multicols}

\vspace*{-6pt}

\hfill{\small\textit{Received April 15, 2023}} 

\vspace*{-18pt}
    
    
    \Contr
    
    
    \vspace*{-3pt}
    
    \noindent
    \textbf{Nuriev Vitaly A.} (b.\ 1980)~--- Doctor of Science in philology, leading 
scientist, Institute of Informatics Problems, Federal Research Center ``Computer 
Science and Control'' of the Russian Academy of Sciences, 44-2~Vavilov Str., 
Moscow 119333, Russian Federation; \mbox{nurieff.v@gmail.com}
    
    \vspace*{3pt}
    
    \noindent
    \textbf{Karpov Vladimir I.} (b.\ 1971)~--- Candidate of Science (PhD) in 
philology, leading scientist, Institute of Linguistics of the Russian Academy of 
Sciences, 1~bld.~1 Bolshoy Kislovsky lane, Moscow 125009, Russian Federation; 
scientist, Institute of Informatics Problems, Federal Research Center ``Computer 
Science and Control'' of the Russian Academy of Sciences, 44-2~Vavilov Str., 
Moscow 119333, Russian Federation; \mbox{wi.karpow@gmail.com}
     
      
\label{end\stat}

\renewcommand{\bibname}{\protect\rm Литература}    %13
\include{goncharov+ink}   %14
\def\stat{panov}

\def\tit{ПЕРСОНАЛЬНЫЙ КОГНИТИВНЫЙ АССИСТЕНТ:\\ КОНЦЕПЦИЯ И~ПРИНЦИПЫ 
РАБОТЫ$^*$}

\def\titkol{Персональный когнитивный ассистент: концепция и~принципы 
работы}

\def\aut{И.\,В.~Смирнов$^1$, А.\,И.~Панов$^2$, А.\,А.~Скрынник$^3$, 
%В.\,А.~Исаков$^4$, 
Е.\,В.~Чистова$^4$}

\def\autkol{И.\,В.~Смирнов, А.\,И.~Панов, А.\,А.~Скрынник, Е.\,В.~Чистова}
%Е.\,В.~Чистова$^5$}

\titel{\tit}{\aut}{\autkol}{\titkol}

\index{Смирнов И.\,В.}
\index{Панов А.\,И.}
\index{Скрынник А.\,А.} 
%\index{Исаков В.\,А.}
\index{Чистова Е.\,В.}
\index{Smirnov I.\,V.}
\index{Panov A.\,I.}
\index{Skrynnik A.\,A.}
%\index{Isakov V.\,A.}
\index{Chistova E.\,V.}


{\renewcommand{\thefootnote}{\fnsymbol{footnote}} \footnotetext[1]
{Работа выполнена при частичной финансовой поддержке РФФИ (проект №\,18-29-22027).}}


\renewcommand{\thefootnote}{\arabic{footnote}}
\footnotetext[1]{Институт проблем искусственного интеллекта Федерального исследовательского центра 
<<Информатика и~управление>> Российской академии наук; Российский университет дружбы 
народов, \mbox{ivs@isa.ru}}
\footnotetext[2]{Институт проблем искусственного интеллекта Федерального исследовательского центра 
<<Информатика и~управление>> Российской академии наук; Московский фи\-зи\-ко-тех\-ни\-че\-ский 
институт (государственный университет), \mbox{pan@isa.ru}}
\footnotetext[3]{Институт проблем искусственного интеллекта Федерального исследовательского центра 
<<Информатика и~управление>> Российской академии наук, \mbox{skrynnik@isa.ru}}
\footnotetext[4]{Институт проблем искусственного интеллекта Федерального исследовательского центра 
<<Информатика и~управление>> Российской академии наук; Российский университет дружбы 
народов, \mbox{chistova@isa.ru}}

\vspace*{-12pt}
  
       
  

  \Abst{Предложена концепция когнитивного персонального ассистента. Когнитивный 
ассистент выступает виртуальным интеллектуальным агентом, обладающим своей 
собственной картиной мира (КМ) и~строящим КМ пользователя, которому он помогает 
решать различные задачи. Описана архитектура когнитивного ассистента, рассмотрены 
основные функции, которые он должен реализовывать, и~представлены основные методы 
и~технологии, которые используются при построении такого рода ассистентов. Рассмотрены 
две предметные области, в~которых использование когнитивных ассистентов наиболее 
перспективно.}
  
  \KW{когнитивный ассистент; образовательный ассистент; медицинский ассистент; 
знаковая картина мира; обработка естественного языка; диалоговая система; сценарий; 
планирование}

\DOI{10.14357/19922264190315} 
  
%\vspace*{6pt}


\vskip 10pt plus 9pt minus 6pt

\thispagestyle{headings}

\begin{multicols}{2}

\label{st\stat}
  
\section{Введение}

  Разработки в~области интеллектуальных ас\-сис\-тен\-тов~--- помощников человека 
при совершении им каждодневных задач (выбор товаров и~услуг, поиск 
в~интернете, прокладка маршрута и~навигация во время вождения, голосовое 
управление бытовыми устройствами и~т.\,п.)~--- получили в~последнее время 
новый импульс своего развития в~связи с~появлением новых методов анализа 
естественного языка и~глубокого обучения. Методы распознавания и~синтеза 
речи, лингвистического анализа и~синтеза текстов достигли сегодня такого 
уровня, который обеспечил создание промышленных голосовых помощников, 
таких как Siri, Cortana, Alexa, Алиса. 
%
Однако до сих пор не решена задача 
создания полноценного ассистента, который бы не только действовал 
реактивно в~ответ на запросы пользователя, но и~вел бы себя проактивно, 
предсказывая поведение пользователя, а~также обладал целеполаганием 
и~внутренней мотивацией по достижению поставленных перед ним целей. Во
многом это связано с~тем, что основной упор при создании современных 
ассистентов делается преимущественно на поддержание  
воп\-рос\-но-от\-вет\-ной коммуникации, т.\,е.\ создание диалоговых \mbox{систем}. 
  
  Для решения же поставленной цели генерации проактивных 
целенаправленных действий ассистент должен обладать элементами 
искусственного сознания, собственной моделью поведения и~строить 
аналогичную по сложности модель поведения пользователя на основе 
понимания разговорного языка. Только в~таком случае интеллектуальный агент 
сможет стать полноценным помощником человека, а не только голосовым 
интерфейсом для разных приложений, например поисковых.
  
  В настоящей работе предлагается концепция персонального когнитивного 
ассистента, т.\,е.\ такого виртуального ассистента, который обладал бы 
проактивным целенаправленным поведением и~моделировал поведение 
собеседника. Описана архитектура, принципы работы, основные 
функции когнитивного ассистента, предложены методы реализации этих 
функций. Возможные варианты использования когнитивного ассистента 
пред\-став\-ле\-ны для двух предметных областей: ассистирование в~процессе 
обучения пользователя (образовательный ассистент) и~ассистирование 
в~процессе поддержания здоровья пользователя (виртуальный тренер 
здоровья).

  \begin{figure*}] %fig1
   \vspace*{1pt}
    \begin{center}  
  \mbox{%
 \epsfxsize=137.411mm 
 \epsfbox{pan-1.eps}
 }
\end{center}
\vspace*{-9pt}
  \Caption{Архитектура когнитивного ассистента}
  \end{figure*}
 
 \vspace*{-14pt} 
  
\section{Архитектура когнитивного ассистента}

\vspace*{-4pt}

  На рис.~1 представлена принципиальная схема устройства и~работы 
когнитивного ассистента. Основным компонентом агента является его 
КМ, которая моделируется на основе знакового подхода~[1, 2]. 
Основным компонентом знаковой КМ служит \textit{знак}, 
представляющий собой на синтаксическом уровне описания модели  
(по~\cite{2-sm}) четверку  
$$
s=\langle n,p,m,a\rangle\,,
$$
 где $n\hm\in N$; 
$p\hm\subset P$; $m\hm\subset M$; $a\hm\subset A$. Здесь
$N$~--- \textit{множество 
имен}, представляющее собой множество слов конечной длины в~некотором 
алфавите; $P$~---\linebreak множество замкнутых атомарных формул языка исчис\-ле\-ния 
предикатов первого порядка, которое называется \textit{множеством образов}; 
$M$~--- множество значений; $A$~--- множество личностных смыслов.
  
  В случае так называемой житейской КМ, компонента образа знака 
участвует в~процессе распознавания и~категоризации. Значения представляют 
фиксированные, сценарные знания интеллектуального агента о предметной 
области и~окружающей среде, а множество личностных смыслов характеризует 
его предпочтения и~текущий деятельностный контекст. Компонента имени 
осуществляет связывание остальных компонент знака в~единое целое 
(именование).
  

  На структурном уровне описания знаковой КМ каждая компонента 
знака представляет собой множество каузальных матриц, которые 
представляют собой структурированное множество ссылок на другие знаки, 
либо элементарные компоненты (в~случае образа~--- это первичный признак 
или данные с~сенсоров, в~случае личностного смысла~--- это операционный 
состав действия). Каузальная матрица позволяет кодировать информацию для 
представления как декларативных, так и~процедурных знаний. Множество 
компонент знака образуют четыре типа каузальных сетей~--- специального 
типа семантических сетей. Моделирование функций планирования 
и~рассуждения осуществляется за счет введения понятия ак\-тив\-ности (множества 
активных знаков или каузальных матриц) и~правил распространения 
активности по различным типам сетей~[3]. В~процессе работы той или иной 
когнитивной функции формируются новые каузальные матрицы, которые 
могут затем сохраняться в~составе компонент нового знака аналогично 
сохранению опыта в~системах, основанных на прецедентах.
  
  Отдельное множество знаков в~КМ ассистента отвечает за моделирование 
КМ пользователя. Это представление ассистента о свойствах, 
возможностях и~целях другого субъекта хранится в~тех же структурах, что 
и~основная информация о~предметной об\-ласти. В~КМ ассистента 
присутствуют специальные знаки~<<Я>> и~<<Другой>> (<<Пользователь>>), 
которые позволяют ассистенту различать информацию, относящуюся 
непосредственно к~нему или другому субъекту.
  
  На основе знаковой КМ ассистент способен строить свои 
собственные планы действий, в~которые он может включать действия 
(реакцию) других субъектов, в~том числе пользователя (см.\ п.~2.2.1). Картина мира 
ассистента описывает его назначение, цели, возможные действия и~сценарии, 
личностные смыслы, оценки достижения целей. Модель КМ пользователя 
строится за счет выявления его сценариев и~личностных смыслов, ценностей, 
предпочтений, привычек и~т.\,п. Когнитивный ассистент общается 
с~собеседником с~учетом этих двух~КМ.
  
  Основными структурами на множестве знаков, которые позволяют 
генерировать проактивные действия, являются сценарии~--- переиспользуемые 
абстрактные последовательности действий и~ситуаций~\cite{2-sm}. 
Когнитивный ассистент имеет сценарии собственных действий и~сценарии 
действий пользователя. Таким образом, ассистент имеет представление 
о~последовательности действий, необходимых для решения той или иной 
задачи, что позволяет ему ассистировать пользователю при решении задач, 
подсказывая дальнейшие действия с~учетом персональных сценариев, 
характерных для конкретного пользователя.
  
  Когнитивный ассистент требует предварительного обучения и~настройки за 
счет внешних ис\-точников, по которым он формирует базовые сценарии 
деятельности, характерные для данной\linebreak предметной области. Автоматическое 
или автоматизированное пополнение КМ сценариями происходит за счет двух 
основных источников текстовой и~нетекстовой (видео, изображения, данные 
с~различных сенсоров) информации: внешние источники (интернет, коллекции 
документов из данной предметной об\-ласти и~т.\,п.)\ и~пользователь, который 
передает ас\-сис\-тен\-ту текст в~виде запросов и~описания задач либо через 
программный интерфейс, который по желанию пользователя может 
регистрировать его различные характеристики (например, био\-мет\-ри\-че\-ские 
данные, психологические черты лич\-ности~\cite{4-sm}, действия с~интерфейсом 
или действия с~другими устройствами). 

Автоматическое извлечение 
и~формирование сценариев по тексту происходит с~помощью подходов, 
описанных в~п.~2.2.2. До\-пус\-ти\-мо по\-стро\-ение КМ ассистента 
и~вручную экспертом предметной об\-ласти.
  
  Одна из важнейших особенностей ассистента~--- его способность 
генерировать текст на естественном языке и~вести связный диалог (см.\ 
п.~2.2.3).

%\vspace*{-6pt}
  
  \subsection{Режимы работы} %2.1
  
  Когнитивный ассистент работает в~нескольких режимах, автоматически 
выбирая режим в~зависимости от текущего диалога или высказывания:
  \begin{itemize}
\item диалог на свободную тему (chit-chat);
\item вопросно-ответный режим;
\item целеориентированный режим.
\end{itemize}

  В диалоговом режиме ассистент поддерживает беседу на свободную тему, 
отвечает на приветствия, спрашивает о настроении пользователя. В~этом 
режиме ассистент также выполняет простейшие просьбы, например рассказать 
анекдот или выдать прогноз погоды.
  
  В вопросно-ответном режиме ассистент находит точный ответ на 
поставленный пользователем вопрос. Вопрос может задаваться на естественном 
языке в~свободной форме. При поиске ответа предполагается учитывать 
интересы и~текущее настроение пользователя.
  
  Целеориентированный режим предполагает помощь ассистента в~решении 
конкретных задач пользователя. Этот режим в~значительной степени 
использует КМ ассистента и~пользователя и~зависит от назначения 
ассистента.

%\vspace*{-3pt}
  
  \subsection{Основные методы} %2.2.
  
 % \vspace*{-3pt}
  
  \subsubsection{Планирование поведения} %2.2.1
  
  На этапе синтеза плана деятельности когнитивный ассистент рекурсивно 
создает все возможные планы по достижению конечной ситуации, которая 
описывает целевое состояние ассистента и~пользователя. Для этого ассистентом 
рассматриваются все знаки, которые входят в~описание текущей\linebreak 
ситуации~$z_{\mathrm{sit}\mbox{-}\mathrm{cur}}$, и~с~по\-мощью процесса распространения 
активности по сети значений~\cite{3-sm, 5-sm}\linebreak активируются процедурные 
матрицы действий. Затем актуализируются матрицы действий, заменяются 
ссылки на знаки ролей и~типов объектов на ссылки конкретных объектов 
задачи. Далее следует шаг выбора действий, которые эвристически были 
оценены как наиболее подходящие в~ситуации~$z_{\mathrm{sit}\mbox{-}\mathrm{cur}}$ для 
достижения ситуации~$z_{\mathrm{sit}\mbox{-}\mathrm{goal}}$. После этого из эффектов каждого 
действия и~ссылок на знаки, которые входят в~текущую ситуацию, строится 
$z_{\mathrm{sit}\mbox{-}\mathrm{cur}+1}$, которая описывает состояние агента после применения 
действия. В~план добавляется рассматриваемое действие  
и~$z_{\mathrm{sit}\mbox{-}\mathrm{cur}}$, затем проверяется
 вхождение~$z_{\mathrm{sit}\mbox{-}\mathrm{goal}}$ 
в~$z_{\mathrm{sit}\mbox{-}\mathrm{cur}+1}$. Если матрицы текущего состояния включают 
матрицы целевого состояния, то алгоритм сохраняет найденный план как один 
из возможных; если мат\-ри\-цы целевого со\-сто\-яния
не входят, то функция поиска плана рекурсивно 
повторяется.

  
 % \vspace*{-6pt}
  
  \subsubsection{Автоматическое формирование сценариев} %2.2.2
  
  %\vspace*{-2pt}
  
   Основной задачей для когнитивного ассистента является задача 
автоматизированного формирования КМ, прежде всего сценариев 
решения задач и~сети значений знаков. Предполагается, что основным 
источником для формирования КМ служат наборы текстов. Для 
конструирования сценариев по текстам предлагается использовать подходы 
открытого извлечения информации из текстов~\cite{6-sm, 7-sm}.

  
  Сеть на значениях в~КМ состоит из концептов и~связей между ними. Концепт 
представляет собой предмет или явление в~рамках предметной области. Разные 
лексические единицы могут ссылаться на один и~тот же концепт. Например, 
<<центральное обрабатывающее устройство>> и~<<процессор>>. 
  
  Между концептами существуют два типа связей:
  \begin{enumerate}[(1)]
\item таксономические связи образуют иерархии концептов. Например, 
<<попугай~--- это птица>>. <<Птица>>~--- гипероним по отношению 
к~<<попугай>>;
\item нетаксономические связи являются предикатами, описывающими 
взаимодействие концептов. Например, <<Эксперт размечает корпус>>.
\end{enumerate}

  Основные шаги пополнения сети значений.
  \begin{enumerate}[1.]
\item  Выполняется графематический, морфологический и~синтаксический 
анализ текстов. На этапе графематического анализа происходит выделение 
предложений из текста и~выделение слов из предложений (токенизация). 
Морфологический анализ позволяет получить леммы (нормальные формы) 
и~морфологические признаки для каждого слова из текста. В~результате 
синтаксического анализа текстов генерируются синтаксические деревья 
зависимостей для каждого предложения. 
  \item Извлекаются всевозможные именные группы из синтаксических 
деревьев зависимостей с~помощью следующего алгоритма:
  \begin{itemize}
\item выполняется поиск вершины синтаксического дерева (как правило, это 
глагол);
\item происходит спуск по дереву до первого существительного;
\item найденное существительное сохраняется вместе с~потомками в~качестве 
именной группы, вплоть до первого слова, часть речи которого не входит 
в~следующий список: существительное, прилагательное, местоимение, 
числительное, имя собственное, союз, наречие, причастие.
\end{itemize}
   \item Извлекаются термины предметной области путем кластеризации 
выделенных именных групп с~учетом метрики C-value:
\begin{multline*}
  \mathrm{C}\mbox{-}\mathrm{Value}(a)={}\\
  {}=\begin{cases}
  \log_2\vert a\vert \cdot \mathrm{freq}(a)\,, &\hspace*{-33mm}\mbox{если именная группа}\\
  &\hspace*{-33mm}\mbox{не вложена  в~другие}\,;\\
  \displaystyle \log_2\vert a\vert \cdot \mathrm{freq}(a)-\fr{1}{p(T_a)}
  \,\sum\limits_{b\in T_a} \mathrm{freq}(b)&\\
& \hspace*{-10.5mm}\mbox{иначе}\,,
  \end{cases}
\end{multline*}
    где $a$~--- именная группа; $\vert a\vert$~--- число слов в~именной группе; 
    $\mathrm{freq}(a)$~--- частота 
встречаемости~$a$; $T_a$~--- именные группы, в~которые входит~$a$;
$p(T_a)$~--- число именных групп, содержащих~$a$.
  
  Данная метрика учитывает пересечения лексики между словосочетаниями 
и~позволяет выделять многословные термины.
  
  \item Для извлечения концептов выполняется клас\-те\-ри\-за\-ция полученных 
терминов. В~качестве признаков используются векторные пред\-став\-ле\-ния 
слов~\cite{8-sm}. Для словосочетаний используются усредненные векторы.
  \item С~помощью набора эвристик из полученных концептов извлекаются 
таксономические связи. Эвристики в~первую очередь опираются на предлоги, 
союзы и~морфологические признаки.
\item Кандидаты в~таксономические связи извлекаются путем поиска 
глагольных групп в~синтаксическом дереве зависимостей. Поиск вы\-полняется 
между лексическими единицами,\linebreak связанными с~концептами.
  \item На конечном этапе таксономические связи извлекаются путем 
кластеризации нескольких признаков кандидатов: векторное представление 
глагольной группы, идентификаторы концептов, между которыми найдена 
глагольная группа.
  \item  На основе ряда эвристик по тексту формируются последовательности 
найденных троек (кон\-цепт\,--\,гла\-голь\-ная груп\-па\,--\,кон\-цепт).
  \end{enumerate}
  
  В результате выявленные в~тексте последовательности троек и~представляют 
собой сценарии решения задач с~необходимыми участниками и~орудиями 
действий.
  
  \subsubsection{Вопросно-ответный режим} %2.2.3 
  
  Вопросно-ответный режим предполагается\linebreak реализовать с~помощью 
технологий Exactus~\cite{9-sm}, основанных на 
 ре\-ля\-ци\-он\-но-си\-ту\-а\-ци\-он\-ном и~се\-ман\-ти\-ко-син\-так\-си\-че\-ском 
анализе текста~[10, 11]. Ре\-ля\-ци\-он\-но-си\-ту\-а\-ци\-он\-ный анализ текста 
пред\-став\-ля\-ет семантику текста в~виде семантической сети, а семантика 
предложения или высказывания при этом пред\-став\-ля\-ет\-ся в~виде со\-во\-куп\-ности 
предикатных слов, их аргументов и~семантических ролей. Семантические сети 
строятся для вопроса и~каждого текста, в~котором может находиться 
формулировка ответа, затем происходит сопоставление семантических сетей 
вопроса и~текстов, вычисляется релевантность текстов вопросу. Таким образом, 
для работы необходим набор текстов, в~которых потенциально могут 
содержаться ответы на вопросы, т.\,е.\ ответы ищутся, а не генерируются.  
Воп\-рос\-но-от\-вет\-ный поиск реализуется в~два этапа: на первом 
выполняется семантический поиск предложений, содержащих формулировку 
ответа на поставленный вопрос; на втором этапе из предложения выделяется 
фрагмент, являющийся точным ответом на поставленный вопрос.
  
  В работе~\cite{8-sm} было показано, что учет семантической структуры 
предложения в~воп\-рос\-но-от\-вет\-ном поиске значительно повышает 
качество поиска ответов по сравнению с~лексическим критерием 
ранжирования, а также позволяет извлекать сам ответ на вопрос. В~2010~г.\ 
технология Exactus была представлена на российском семинаре по оценке 
методов информационного поиска РОМИП в~дорожке во\-прос\-но-от\-вет\-но\-го 
поиска и~показала высокие результаты по всем метрикам~\cite{12-sm}.
  
  \subsubsection{Диалоговый режим}%2.2.4
  
  Для реализации режима диалога на свободную тему на русском языке 
предлагается использовать порождающие подходы на основе различных 
нейросетевых моделей. В~работе~\cite{13-sm} показано, что добавление 
к~модели seq2seq механизма внимания повышает качество и~грамматическую 
согласованность генерируемых реплик в~диалоге. В~будущем возможна 
комбинация данного подхода к~генерации ответов и~использование баз знаний 
для того, чтобы модель оперировала более конкретными представлениями 
в~каждой области.

\section{Применение в~образовательном процессе}

  Один из вариантов использования ассистента, строящего модель 
КМ собеседника, а~также обладающего одним из вариантов КМ,~--- 
  его применение в~процессах он\-лайн-обуче\-ния. 
  
  Образовательные он\-лайн-сис\-те\-мы (к примеру, Cursera, Stepik, Logiclik, 
Examer и~др.)\ сейчас используют методы искусственного интеллекта для 
улучшения качества обучения пользователей при решении следующих задач:
\begin{enumerate}[1.]
\item Выбор образовательной траектории, подстройка блоков курса, иными 
словами, адаптация программы обучения по выбранной пользователем теме 
в~зависимости от когнитивных особенностей этого пользователя.
\item Организация frequently asked quetions (FAQ) (раздела во\-про\-сов-от\-ве\-тов) по данному курсу, 
в~котором ассистент заменяет преподавателя при ответе на стандартные 
вопросы по курсу.
\item Подсказки учителю по время подготовки или проведения урока, которые 
формируются в~зависимости от реакции аудитории на прошлые занятия, либо 
в~зависимости от когнитивных особенностей преподавателя.
\item Прокторинг~--- отслеживание поведения пользователя во время 
просмотра курса или выполнения задания с~учетом его поведенческих 
характеристик, подача ему предупреждений или советов.
\item Автоматическая проверка выполненного задания, выдача рекомендаций 
по исправлению ошибок или по выполнению дополнительного задания на одну 
из тем, по которой пользователь допустил ошибку, с~учетом его эмоциональных 
и~когнитивных особенностей.
\item Формирование советов по дальнейшему усвоению курса как для 
пользователя, так и~для преподавателя.
\item Мотивация пользователя путем генерации соответствующих его 
состоянию и~психологическим особенностям реплик или вывод его на диалог.
\end{enumerate}

  Все упомянутые выше поведенческие, психологические и~когнитивные 
особенности должны определяться когнитивным ассистентом в~рамках 
КМ пользователя, модель которой он строит. Гипотеза авторов состоит в~том, 
что весь комплекс упомянутых функций и~корректный учет всех особенностей 
пользователя системой он\-лайн-об\-ра\-зо\-ва\-ния невозможно обеспечить без 
использования специальных методов моделирования его знаковой 
КМ. При этом для реализации некоторых функций (например,~4 или~7) 
необходимо, чтобы агент сам обладал КМ (возможно, статичной, 
возможно, эволюционирующей), которая позволяла бы ему вести более 
полноценные диалоги и~демонстрировать собственные психологические 
особенности для достижения большего эффекта от общения с~пользователем.
  
  На рис.~2 приведен примерный вид интерфейса программной реализации 
такого ассистента, который выполняет роль помощника при обучении игры 
в~шахматы.
  

  
  Ассистент предварительно обучен на текстах шахматных книг (в~том числе 
детских и~художественных, в~которых встречается большее разнообразие 
простых языковых конструкций)~--- по ним он формирует базовую 
КМ потенциального пользователя и~свою собственную КМ учителя, 
используя методы п.~2.2.1. Также агент обладает функционалом классических 
шахматных программ (система\linebreak\vspace*{-12pt}

{ \begin{center}  %fig2
 \vspace*{-3pt}
\mbox{%
 \epsfxsize=79mm 
 \epsfbox{pan-2.eps}
 }


\end{center}

%\vspace*{-3pt}


\noindent
{{\figurename~2}\ \ \small{Пример интерфейса образовательного когнтивного ассистента по игре 
в~шахматы}}

}


\vspace*{12pt}


  
  \noindent
   просчета вариантов и~т.\,п.). Виды 
формируемых сценариев: шахматные розыгрыши в~стандартных позициях, 
правила оценки той или иной позиции, общие сценарии поведения учителя 
(того, кто советует, направляет, подсказывает) и~ученика (того, кто что-то 
решает, приобретает знания). Ассистент учится во время взаимодействия 
с~учеником, обновляя как свою КМ, так и~информацию о~КМ 
пользователя (какие ошибки совершает, что интересно).
  
  Информация, которую получает ассистент от пользователя: выбор  
ка\-кой-ли\-бо темы обучения (дебют, окончания, тактика, стратегия, разбор 
партий и~т.\,п.), ходы при решении предложенных задач и~время от времени 
вопросы и~ответы в~текстовом поле. Также возможно отслеживание по 
видеокамере поведения пользователя в~процессе решения задач.
  
  Основные задачи ассистента: по сформированным (и~пополняемым 
в~процессе работы) в~КМ сценариям для пользователя ассистент 
(с~учетом\linebreak особенностей своей собственной КМ) выдает 
мотивирующие реплики пользователю, задает ему вопросы, отвечает на его 
вопросы, предлагает подсказки в~различных ситуациях, дает ему советы\linebreak после 
партии, предлагает ему новые задачи в~рамках выбранного курса для 
ликвидации определенных недостатков в~знаниях пользователя, предлагает ему 
пройти необходимый ему новый курс\linebreak (например, король и~пешка против 
короля). 

\section{Применение в~здоровьесбережении}

  Другим вариантом использования когнитивного ассистента является его 
применение в~процессах здоровьесбережения. Сегодня наблюдается активный 
рост объема научных исследований и~разработок в~области 
здоровьесбережения~--- профилактики и~поддержания здоровья людей. 
Основная цель этого направления~--- анализ особенностей здоровья 
конкретного человека и~подбор и~проведение персонализированных 
профилактических мероприятий до появления первых симптомов возможных 
заболеваний. Как можно более ранний\linebreak
 прогноз возникновения заболеваний 
служит снижению риска патологии. В~настоящее время разработаны 
интеллектуальные системы, под\-дер\-жи\-вающие процесс здоровьесбережения на 
всех\linebreak стадиях~\cite{14-sm}. Основные задачи технологий здоровьесбережения 
включают: 
  \begin{itemize}
  \item [(а)] сбор данных о состоянии здоровья и~образе жизни человека из 
различных источников;
  \item[(б)] интеллектуальный анализ данных о состоянии здоровья и~образе 
жизни человека для выявления проблем со здоровьем, оценку 
персонализированных рисков ухудшения здоровья человека;
  \item[(в)] подбор персональных рекомендаций по изменению образа жизни 
конкретного человека в~зависимости от состояния его здоровья, образа жизни, 
проблемных зон и~индивидуальных особенностей;
  \item[(г)] формирование персонального плана профилактических 
мероприятий;
  \item[(д)] мотивацию человека к~выполнению рекомендаций, отслеживанию 
изменений в~образе жизни и~рисках заболеваний.
  \end{itemize}
  
  Персональный когнитивный ассистент в~виде виртуального персонального 
тренера здоровья может использоваться при решении всех указанных задач. 
Предполагается, что виртуальный тренер здоровья постоянно мониторит 
психическое и~физическое состояние пользователя, сообщая пользователю 
о~критических изменениях. Он может быть интегрирован с~различными 
гаджетами (фит\-несс-тре\-ке\-ра\-ми) и~мобильными приложениями, может 
анализировать активность в~социальных сетях, учитывать пищевые 
предпочтения пользователя с~помощью фуд-тре\-кера. 
  
  Одним из основных способов получения информации об образе жизни 
и~здоровье человека будет проактивный диалог ассистента с~пользователем. 
Например, предполагается, что ассистент сам инициирует диалог~--- 
спрашивает, как настроение сегодня, просит ответить на вопросы 
психологического теста, напоминает о необходимости запланированных 
действий, например приема пищи, отдыха, физической активности. 
Когнитивный ассистент может ответить на любой вопрос по теме 
здо\-ровье\-сбе\-ре\-же\-ния или отправить пользователя к~информации, содержащей 
ответ на вопрос, обладает эмоциями и~выражает их с~помощью смайлов или 
эмодзи, учитывает КМ и~текущее настроение пользователя при 
выражении эмоций. При формировании плана профилактических мероприятий 
ассистент также должен учитывать КМ пользователя, его 
предпочтения. 

\section{Заключение}

  Сегодня существуют много виртуальных ас\-сис\-тен\-тов и~чат-бо\-тов, 
нуждающихся в~интеллектуальной начинке, которая была бы основана на 
моделировании когнитивных функций человека\linebreak и~повышала качество работы 
ассистентов. Когнитивный ассистент отличается от аналогов наличием 
формализованных знаний о~час\-ти окру\-жа\-юще\-го мира и~способах решения 
определенных задач, а~также способностью учитывать КМ
пользователя в~процессе ассистирования ему. 
  
  В результате дальнейших работ планируется создание семейства 
интеллектуальных ассистентов и~соответствующих технологий, легко 
настраиваемых на решение новых задач. На основе предложенной концепции 
персонального когнитивного ас\-сис\-тен\-та предполагается разработка 
масштабируемой программной платформы для создания ас\-сис\-тен\-тов 
различного назначения и~их настройки под конкретные задачи и~предметные 
области. Такие ас\-сис\-тен\-ты будут способны к~общению с~человеком (или 
другим виртуальным ас\-сис\-тен\-том) на естественном языке и~смогут 
встраиваться в~различные программы (мессенджеры, социальные сети) или 
искусственные технические устройства (например,\linebreak робототехнические 
ас\-сис\-тен\-ты). Когнитивные ас\-сис\-тен\-ты наиболее востребованы в~автономных 
интел\-лек\-ту\-аль\-ных устройствах или их коалициях, действующих в~опасных 
средах с~отложенной коммуникацией с~человеком.
  
 {\small\frenchspacing
 {%\baselineskip=10.8pt
 \addcontentsline{toc}{section}{References}
 \begin{thebibliography}{99}
 
  \bibitem{2-sm}
  \Au{Осипов Г.\,С., Панов А.\,И.} Отношения и~операции в~знаковой картине 
мира субъекта поведения~// Искусственный интеллект и~принятие решений, 
2017. №\,4. С.~5--22.

 \bibitem{1-sm}
  \Au{Осипов Г.\,С., Чудова~Н.\,В., Панов~А.\,И., Кузнецова~Ю.\,М.} Знаковая 
картина мира субъекта поведения.~--- М.: Физматлит, 2018. 264~с.

  \bibitem{3-sm}
  \Au{Панов А.\,И.} Целеполагание и~синтез плана поведения когнитивным 
агентом~// Искусственный интеллект и~принятие решений, 2018. №\,2.  
С.~21--35.
  \bibitem{4-sm}
  \Au{Stankevich M., Smirnov~I., Ignatiev~N., Grigoriev~O., Kiselnikova~N.} 
Analysis of big five personality traits by processing of social media users activity 
features~// CEUR Workshop Procee., 2018.  
%Data Analytics and Management in Data Intensive Domains 2018: Selected Papers of the XX 
%International Conference (DAMDID/RCDL 2018). 
Vol.~2277. P.~162--166.
  \bibitem{5-sm}
  \Au{Киселев Г.\,А., Панов~А.\,И.} Знаковый подход к~задаче распределения 
ролей в~коалиции когнитивных агентов~// Труды СПИИРАН, 2018. №\,2. 
С.~161--187.
  \bibitem{6-sm}
  \Au{Шелманов А.\,О., Исаков~В.\,А., Станкевич~М.\,А., Смирнов~И.\,В.} 
Открытое извлечение информации из\linebreak текс\-тов. Часть~I. Постановка задачи 
и~обзор методов~// Искусственный интеллект и~принятие решений, 2018. №\,2. 
С.~47--67.
  \bibitem{7-sm}
  \Au{Шелманов А.\,О., Девяткин~Д.\,А., Исаков~В.\,А., Смирнов~И.\,В.} 
Открытое извлечение информации из текстов. Часть~II. Извлечение 
семантических отношений с~помощью машинного обучения без учителя~// 
Искусственный интеллект и~принятие решений, 2019. №\,2. С.~39--49.
  \bibitem{8-sm}
  \Au{Mikolov T., Sutskever~I., Chen~K., Corrado~G.\,S., Dean~J.} Distributed 
representations of words and phrases and their compositionality~// Adv.
Neur. Inf., 2013. Vol.~26. P.~3111--3119.
  \bibitem{9-sm}
  \Au{Шелманов А.\,О., Каменская~М.\,И., Ананьева~И.\,В., Смирнов~И.\,В.} 
Семантико-синтаксический анализ текстов в~задачах  
воп\-рос\-но-от\-вет\-но\-го поиска и~извлечения определений~// Искусственный 
интеллект и~принятие решений, 2016. №\,4. C.~47--61.
  \bibitem{10-sm}
  \Au{Осипов~Г.\,С., Смирнов~И.\,В., Тихомиров~И.\,А.}  
Ре\-ля\-ци\-он\-но-си\-ту\-а\-ци\-он\-ный метод поиска и~анализа текстов и~его 
приложения~// Искусственный интеллект и~принятие решений, 2008. №\,2. 
С.~3--10.
  \bibitem{11-sm}
  \Au{Смирнов~И.\,В., Шелманов~А.\,О., Кузнецова~Е.\,С.,\linebreak Храмоин~И.\,В.} 
Семантико-синтаксический анализ естественных языков. Часть~II. Метод 
се\-ман\-ти\-ко-син\-так\-си\-че\-ско\-го анализа текстов~// Искусственный интеллект 
и~принятие решений, 2014. №\,1. С.~11--24.
  \bibitem{12-sm}
  \Au{Завьялова О.\,С., Киселёв~А.\,А., Осипов~Г.\,С.,
  Смирнов~И.\,В., Тихомиров~И.\,А., Соченков~И.\,В.} 
Система интеллектуального поиска и~анализа информации <<EXACTUS>> на  
\mbox{РОМИП}-2010~// Тр. Российского семинара по оценке методов 
информационного поиска.~--- Казань: Казанский ун-т, 2010. 
С.~49--69.
  \bibitem{13-sm}
  \Au{Чистова Е.\,В., Шелманов~А.\,О., Смирнов~И.\,В.} Применение 
глубокого обучения к~моделированию диалога на естественном языке~// Тр. 
Института системного анализа РАН, 2019. Т.~69. №\,1. С.~105--115.
  \bibitem{14-sm}
  \Au{Grigoriev O.\,G., Kobrinskii~B.\,A., Osipov~G.\,S., Molodchenkov~A.\,I., 
Smirnov~I.\,V.} Health management system knowledge base for formation and 
support of a~preventive measures plan~// Procedia Comput. Sci., 2018. 
Vol.~145. P.~238--241.
  \end{thebibliography}

 }
 }

\end{multicols}

\vspace*{-6pt}

\hfill{\small\textit{Поступила в~редакцию 28.01.19}}

%\vspace*{8pt}

%\pagebreak

\newpage

\vspace*{-28pt}

%\hrule

%\vspace*{2pt}

%\hrule

%\vspace*{-2pt}

\def\tit{PERSONAL COGNITIVE ASSISTANT:\\ CONCEPT AND~KEY 
PRINCIPALS}


\def\titkol{Personal cognitive assistant: Concept and~key 
principals}

\def\aut{I.\,V.~Smirnov$^{1,2}$, A.\,I.~Panov$^{1,3}$, 
  A.\,A.~Skrynnik$^1$, %V.\,A.~Isakov$^1$, 
  and~E.\,V.~Chistova$^{1,2}$}

\def\autkol{I.\,V.~Smirnov, A.\,I.~Panov,    A.\,A.~Skrynnik, 
 and~E.\,V.~Chistova}

\titel{\tit}{\aut}{\autkol}{\titkol}

\vspace*{-11pt}


\noindent
  $^1$Institute of Artificial Intelligence Problems, Federal Research Center 
``Computer Science and Control'' of the\linebreak
$\hphantom{^1}$Russian Academy of Sciences; 9~60-letiya 
Oktyabrya Prosp., Moscow 117312, Russian Federation
  
  \noindent
  $^2$Friendship University of Russia (RUDN University), 6~Miklukho-Maklaya 
Str., Moscow 117198, Russian\linebreak
$\hphantom{^1}$Federation
  
  \noindent
  $^3$Moscow Institute of Physics and Technology (State University), 9~Institutskiy 
Per., Dolgoprudny, Moscow Region\linebreak
$\hphantom{^1}$141701, Russian Federation

\def\leftfootline{\small{\textbf{\thepage}
\hfill INFORMATIKA I EE PRIMENENIYA~--- INFORMATICS AND
APPLICATIONS\ \ \ 2019\ \ \ volume~13\ \ \ issue\ 3}
}%
 \def\rightfootline{\small{INFORMATIKA I EE PRIMENENIYA~---
INFORMATICS AND APPLICATIONS\ \ \ 2019\ \ \ volume~13\ \ \ issue\ 3
\hfill \textbf{\thepage}}}

\vspace*{3pt}   
  
    
  
  \Abste{The paper proposes the concept of cognitive personal assistant. The 
cognitive assistant is a virtual intelligent agent that has its own sign-based world 
model and builds a world model of the user, which it helps to solve various problems. 
The architecture of the cognitive assistant is described, the main functions that it 
should implement are considered, and the main methods and technologies that are 
used in the construction of such assistants are presented. Two subject areas in which 
the use of cognitive assistants is the most promising are considered.}
  
  \KWE{cognitive assistant; educational assistant; medical assistant; sign-based 
worldview; natural language processing; script; dialog system; planning}
  
 
  
\DOI{10.14357/19922264190315} 

%\vspace*{-14pt}

 \Ack
  \noindent
  The study was partially funded by the Russian Foundation for Basic Research (project 
No.\,18-29-22027).


%\vspace*{-6pt}

  \begin{multicols}{2}

\renewcommand{\bibname}{\protect\rmfamily References}
%\renewcommand{\bibname}{\large\protect\rm References}

{\small\frenchspacing
 {%\baselineskip=10.8pt
 \addcontentsline{toc}{section}{References}
 \begin{thebibliography}{99}
  
  \bibitem{2-sm-1}
  \Aue{Osipov, G.\,S., and A.\,I.~Panov.} 2018. Relationships and operations in agent's 
sign-based model of the world. \textit{Scientific Technical Information Processing}
45(5):1--14.
\bibitem{1-sm-1}
  \Aue{Osipov, G.\,S., A.\,I.~Panov, and N.\,V.~Chudova.} 2014. Behavior control as 
  a~function of consciousness. I.~World model and goal setting. 
  \textit{J.~Comput. Sys. Sc. Int.} 53(4):517--529.
  
  \bibitem{3-sm-1}
  \Aue{Panov, A.\,I.} 2017. Behavior planning of intelligent agent with sign world 
model. \textit{Biol. Inspir. Cogn. Arc.} 19:21--31. 
  \bibitem{4-sm-1}
  \Aue{Stankevich, M., I.~Smirnov, N.~Ignatiev, O.~Grigoriev, and N.~Kiselnikova.} 
2018. Analysis of big five personality traits by processing of social media users 
activity features. \textit{CEUR Workshop Procee.} 2277:162--166.
  \bibitem{5-sm-1}
  \Aue{Kiselev G.\,A., and A.\,I.~Panov}. 
  2018. Znakovyy podkhod k~zadache raspredeleniya 
roley v koalitsii kognitivnykh agentov [Sign-based approach to the task of role 
distribution in the coalition of cognitive agents]. \textit{Trudy SPIIRAN} [SPIIRAS 
Proceedings] 57:161--187.
  \bibitem{6-sm-1}
  \Aue{Shelmanov, A.\,O., V.\,A.~Isakov, M.\,A.~Stankevich, and I.\,V.~Smirnov.} 2018. 
Otkrytoe izvlechenie informatsii iz tekstov. Chast'~I. Postanovka zadachi i~obzor 
metodov [Open information extraction. Part~I. The task and the review of the state 
of the art]. \textit{Iskusstvennyy intellekt i~prinyatie resheniy} 
[Artificial Intelligence and  Decision Making] 2:47--67.
  \bibitem{7-sm-1}
  \Aue{Shelmanov, A.\,O., D.\,A.~Devyatkin, V.\,A.~Isakov, and I.\,V.~Smirnov.} 2019. 
Otkrytoe izvlechenie informatsii iz tekstov. Chast'~II. Izvlechenie 
semanticheskikh otnosheniy s~pomoshch'yu mashinnogo obucheniya bez uchitelya 
[Open information extraction from texts. Part~2. Extraction of semantic relations 
using unsupervised machine learning]. 
\textit{Iskusstvennyy intellekt i~prinyatie resheniy} 
[Artificial Intelligence and Decision Making] 2:39--49. 
  \bibitem{8-sm-1}
  \Aue{Mikolov, T., I.~Sutskever, K.~Chen, G.\,S.~Corrado, and J.~Dean.} 2013. 
Distributed representations of words and phrases and their compositionality. 
\textit{Adv. Neur. Inf.} 26:3111--3119.
  \bibitem{9-sm-1}
  \Aue{Shelmanov, A.\,O., M.\,I.~Kamenskaya, I.\,V.~Ananyeva, and I.\,V.~Smirnov.} 2017. 
Semantic-syntactic analysis for question answering and 
definition extraction. \textit{Scientific Technical Information Processing}
44(6):412--423.
  \bibitem{10-sm-1}
  \Aue{Osipov, G.\,S., I.\,V.~Smirnov, and I.\,A.~Tikhomirov.} 2010. 
  Relational-situational method for text search and analysis and its applications. 
  \textit{Scientific  Technical Information Processing} 6:432--437.
  \bibitem{11-sm-1}
  \Aue{Smirnov, I.\,V., A.\,O.~Shelmanov, E.\,S.~Kuznetsova, and I.\,V.~Khramoin.} 2014. 
Semantiko-sintaksicheskiy ana\-liz estestvennykh yazykov. Chast'~II. Metod 
semantiko-sintaksicheskogo analiza tekstov [Semantic-syntactic analysis of natural 
languages. Part~II. Method for semantic-syntactic analysis of texts].
\textit{Iskusstvennyy 
intellekt i~prinyatie resheniy} [Artificial Intelligence and Decision Making] 1:11--24.
  \bibitem{12-sm-1}
  \Aue{Zavjalova, O.\,S., A.\,A.~Kiselyov, G.\,S.~Osipov,
 I.\,V.~Smir\-nov, I.\,A.~Tikhomirov, and I.\,V.~Sochenkov.} 2010. 
Sistema intellektual'nogo poiska i~analiza informatsii \mbox{``EXACTUS''} na ROMIP-2010 
[The system of information search and analysis EXACTUS on ROMIP-2010]. 
\textit{Tr. Rossiyskogo seminara po otsenke metodov informatsionnogo poiska} 
[ROMIP-2010 Proceedings]. Kazan: Kazan State University. 49--69.
  \bibitem{13-sm-1}
  \Aue{Chistova, E.\,V., A.\,O.~Shelmanov, and I.\,V.~Smirnov.} 2019. 
  Primenenie 
glubokogo obucheniya k~mo\-de\-li\-ro\-va\-niyu dialoga na estestvennom yazyke [Natural 
language dialogue modelling with deep learning]. 
\textit{Proceedings of the Institute for Systems Analysis of the Russian Academy of 
Sciences} 69(1):105--115. 
  \bibitem{14-sm-1}
  \Aue{Grigoriev, O.\,G., B.\,A.~Kobrinskii, G.\,S.~Osipov, A.\,I.~Mo\-lod\-chen\-kov, and 
I.\,V.~Smirnov.} 2018. Health management system knowledge base for formation and 
support of a~preventive measures plan. \textit{Procedia Comput. Sci.} 145:238--241.
\end{thebibliography}

 }
 }

\end{multicols}

%\vspace*{-7pt}

\hfill{\small\textit{Received January 28, 2019}}

%\pagebreak

%\vspace*{-22pt}
  
  \Contr
  
  \noindent
  \textbf{Smirnov Ivan V.} (b.\ 1978)~--- Candidate of Science (PhD) in physics 
and mathematics; Head of Department, Institute of Artificial Intelligence Problems, 
Federal Research Center ``Computer Science and Control'' of the Russian Academy 
of Sciences, 9,~60-letiya Oktyabrya Prosp., Moscow 117312, Russian Federation; 
associate professor, Peoples' Friendship University of Russia (RUDN University), 
6~Miklukho-Maklaya Str., Moscow 117198, Russian Federation; \mbox{ivs@isa.ru}
  
  \vspace*{3pt}
  
  \noindent
  \textbf{Panov Aleksandr I.} (b.\ 1987)~--- Candidate of Science (PhD) in physics 
and mathematics; senior scientist, Institute of Artificial Intelligence Problems, 
Federal Research Center ``Computer Science and Control'' of the Russian Academy 
of Sciences; 9,~60-letiya Oktyabrya Prosp., Moscow 117312, Russian Federation; 
deputy head of laboratory, Moscow Institute of Physics and Technology (State 
University), 9~Institutskiy Per., Dolgoprudny, Moscow Region 141701, Russian Federation; 
\mbox{pan@isa.ru}
  
  \vspace*{3pt}
  
  \noindent
  \textbf{Skrynnik Alexey A.} (b.\ 1993)~--- junior scientist, Institute of Artificial 
Intelligence Problems, Federal Research Center ``Computer Science and Control'' of 
the Russian Academy of Sciences; 9,~60-letiya Oktyabrya Prosp., Moscow 117312, 
Russian Federation; \mbox{skrynnik@isa.ru}
  
  \vspace*{3pt}
  
  
 % \noindent
 % \textbf{Isakov Vadim A.} (b.\ 1995)~--- PhD student, Federal Research Center 
%``Computer Science and Control'' of the Russian Academy of Sciences; 9~60-letiya 
%Oktyabrya pr., Moscow 117312, Russian Federation; \mbox{visakov@isa.ru}
  
 % \vspace*{3pt}
  
  \noindent
  \textbf{Chistova Elena V.} (b.\ 1996)~--- programmer, Institute of Artificial 
Intelligence Problems, Federal Research Center ``Computer Science and Control'' of 
the Russian Academy of Sciences; 9,~60-letiya Oktyabrya Prosp., Moscow 117312, 
Russian Federation; student, Peoples' Friendship University of Russia (RUDN 
University), 6~Miklukho-Maklaya Str., Moscow 117198, Russian Federation; 
\mbox{chistova@isa.ru}
\label{end\stat}

\renewcommand{\bibname}{\protect\rm Литература}  
      %15
\def\stat{gor+kuz}

\def\tit{ПРИМЕНЕНИЕ РЕКУРРЕНТНЫХ НЕЙРОННЫХ СЕТЕЙ\\ ДЛЯ ПРОГНОЗИРОВАНИЯ МОМЕНТОВ КОНЕЧНЫХ\\ 
НОРМАЛЬНЫХ СМЕСЕЙ$^*$}

\def\titkol{Применение рекуррентных нейронных сетей для прогнозирования моментов конечных 
нормальных смесей}

\def\aut{А.\,К.~Горшенин$^1$, В.\,Ю.~Кузьмин$^2$}

\def\autkol{А.\,К.~Горшенин, В.\,Ю.~Кузьмин}

\titel{\tit}{\aut}{\autkol}{\titkol}

\index{Горшенин А.\,К.}
\index{Кузьмин В.\,Ю.}
\index{Gorshenin A.\,K.}
\index{Kuzmin V.\,Yu.}


{\renewcommand{\thefootnote}{\fnsymbol{footnote}} \footnotetext[1]
{Работа выполнена при поддержке РФФИ (проекты 18-29-03100 и~19-07-00352) 
и~Стипендии Президента Российской Федерации молодым ученым и~аспирантам (СП-538.2018.5). Для ускорения обучения был использован гибридный высокопроизводительный вычислительный комплекс ФИЦ 
ИУ РАН: {\sf http://hhpcc.frccsc.ru}.}}


\renewcommand{\thefootnote}{\arabic{footnote}}
\footnotetext[1]{Институт проблем информатики Федерального исследовательского
центра <<Информатика и~управление>> Российской академии наук; факультет
вычислительной математики и~кибернетики Московского государственного 
университета им.\ М.\,В.~Ломоносова, \mbox{agorshenin@frccsc.ru}}
\footnotetext[2]{ООО <<Вай2Гео>>, \mbox{shadesilent@yandex.ru}}

\vspace*{-6pt}



\Abst{Проведено сравнение нейронных сетей прямого распространения и~рекуррентных 
модификаций для решения задачи построения прогнозов непрерывных значений для 
математического ожидания, дисперсии, коэффициентов асимметрии и~эксцесса 
конечных нормальных смесей. Рас\-смот\-ре\-ны~$14$~раз\-лич\-ных архитектур
нейронных сетей, включая 
и~{\sf LSTM} (Long-Short Term Memory).
%, отличающиеся числом скрытых слоев и~нейронов в~них. 
Для повышения ско\-рости обучения использованы высокопроизводительные вычислительные 
средства. Продемонстрировано, что на рассматриваемых данных наилучшие результаты 
для всех моментных характеристик в~смысле качества прогнозирования в~стандартных 
метриках (среднеквадратичная ошибка, функция потерь, средняя абсолютная ошибка) 
достигаются с~использованием двух рекуррентных архитектур~--- с~одним скрытым слоем 
из~$100$ нейронов и~тремя слоями по~$50$~нейронов.}

\KW{рекуррентные нейронные сети; прогнозирование; глубокое обучение; 
высокопроизводительные вычисления; CUDA}

\DOI{10.14357/19922264190316} 
  
%\vspace*{1pt}

%\vspace*{-2pt}


\vskip 10pt plus 9pt minus 6pt

\thispagestyle{headings}

\begin{multicols}{2}

\label{st\stat}


\section{Введение}

%\vspace*{2pt}

Одним из наиболее популярных и~востребованных 
методов анализа данных и~по настоящий день остается
EM (expectation--maximization) 
алгоритм. Обычно рассматриваются различные его модификации, однако 
общий принцип наличия E- и~\mbox{M-ша}\-гов остается без изменений. В~частности, он 
при\-меняется для анализа па\-ра\-мет\-ров смешанных вероятностных моделей в~различных 
прикладных задачах. Также можно упомянуть кластерный 
анализ~\cite{Yang2012, Cai2019,Hassen2019} и~работу с~цензурированными данными 
на основе масштабных смесей нормальных законов~\cite{Zeller2019}. 
Остается популярным на\-прав\-ле\-ние исследований, ориентированное на повышение 
скорости работы EM-ал\-го\-рит\-ма с~помощью различных модификаций~--- за счет
 внедрения искусственного шума~\cite{Osoba2013}, блочных реализаций для 
 организации параллельных вычислений~\cite{Lee2018}, приближений на основе 
 методов Мон\-те-Кар\-ло по схеме марковских цепей~\cite{Wu2019}. Актуальными 
 остаются вопросы о числе компонент в~смешанных 
 моделях~\cite{Verbeek2003,Gorshenin2011,Liu2019}, снижении размерности 
 параметрического пространства~\cite{Yu2019}.

В статье~\cite{Gorshenin2019a} для описания эволюции турбулентных процессов 
в~маг\-ни\-то\-ак\-тив\-ной высокотемпературной плазме использован метод 
скользящего разделения конечных нормальных смесей~\cite{Korolev2011}. 
Исследование моментных характеристик аппроксимирующих распределений позволило 
проанализировать нелинейную стадию развития тур\-бу\-лент\-ности, ее насыщения, 
образования вихрей и~их хаотизации.

В статье~\cite{Gorshenin2019b} исследованы вопросы построения прогнозов 
моментов конечных нормальных смесей с~помощью нейронных сетей прямого 
распространения в~смысле классической задачи кластеризации. Непрерывные 
данные разбивались на некоторые промежутки, а нейронная сеть определяла 
попадание в~тот или иной класс. К~достоинствам данного метода стоит отнести 
полученную высокую точность (вплоть до~$99{,}7\%$) прогнозирования. Однако
 точные значения моментных характеристик (т.\,е.\ решение задачи регрессии), 
 которые бы представляли особый интерес для проведения исследований в~физике 
 турбулентной плазмы, в~упомянутой работе не определялись.

В данной статье будут рассмотрены~$14$ различных топологий
нейронных сетей~\cite{Buduma2017}, 
включая рекуррентную модификацию \verb"LSTM"~\cite{Greff2017}.
Они отличаются между собой количеством скрытых слоев и~числом 
нейронов в~них. Все они предназначены для решения задачи построения прогнозов 
непрерывных значений для математического ожидания (\verb"Exp"), 
дисперсии (\verb"Var"), коэффициентов асим\-мет\-рии (\verb"Skew") и~эксцесса 
(\verb"Kurt") конечных нормальных смесей. Кроме того, в~работе обсуждается 
повышение скорости обучения нейросетей с~использованием средств гибридного 
высокопроизводительного вычислительного комплекса (ГВВК).


\section{Базовые архитектуры нейронных сетей для~решения задачи прогнозирования}
%\label{SecArchitecture}

В этом разделе опишем конфигурации нейронных сетей, выбранных 
для прогнозирования различных моментов конечных смесей нормальных законов с~функцией 
распределения следующего вида ($x \hm\in \R$, $a_i(n)\hm\in\R$, $\sigma_i(n)\hm>0$, 
$p_{i}(n)\hm\geqslant 0$, $\sum p_{i}(n)\hm=1$, $i\hm=\overline{1,k(n)}$, 
$n\hm=1,2,\ldots$):
\begin{equation*}
\label{Mixture}
F(x)=\sum\limits_{i=1}^{k(n)}
\fr{p_i(n)}{\sigma_i(n)\sqrt{2\pi}}\int\limits_{-\infty}^{x}
\exp\left\{-\fr{(y-a_i(n))^2}
{2\sigma_i^2(n)}\right\}\,dy\,.
\end{equation*}

Зависимость от параметра~$n$ (номера шага) в~этой формуле соответствует 
принятому в~методе скользящего разделения смесей (СРС-ме\-то\-де) 
изучению эволюции распределения данных во времени в~режиме сдвигающегося окна. 
В~качестве тестовых выборок будут использоваться ряды из статьи~\cite{Gorshenin2019a}.
 Явные выражения для моментных характеристик, включая матричные представления, 
 приведены в~работах~\cite{Gorshenin2016a,Gorshenin2019b}.

Для исследования моментных характеристик были рассмотрены два типа архитектур~--- 
сети прямого распространения (многослойный перцептрон) и~рекуррентные сети с~долгой 
краткосрочной памятью \verb"LSTM". Строится прогноз на~$1$~шаг, при этом в~качестве 
входных данных используются~$50$~предшествующих наблюдений. Были рассмотрены 
сле\-ду\-ющие архитектуры нейронной сети:
\begin{enumerate}[I:] %[label=(\Roman{enumi}), ref=\Roman{enumi}]
\item \label{I} один скрытый слой, $60$ нейронов;
\item \label{II} один скрытый слой, $100$ нейронов;
\item \label{III} два скрытых слоя по $20$ нейронов в~каждом;
\item \label{IV} два скрытых слоя по $50$ нейронов в~каждом;
\item \label{V} два скрытых слоя по $100$ нейронов в~каждом;
\item \label{VI} три скрытых слоя по $20$ нейронов в~каждом;
\item \label{VII} три скрытых слоя по $50$ нейронов в~каждом.
\end{enumerate}

Обучение проводилось на протяжении~$750$~эпох либо останавливалось 
при отсутствии убывания\linebreak\vspace*{-12pt}

  { \begin{center}  %fig1
 \vspace*{-4pt}
   \mbox{%
 \epsfxsize=79mm 
 \epsfbox{gor-1.eps}
 }


\end{center}


\noindent
{{\figurename~1}\ \ \small{Пример изменения величины ошибок в~процессе обучения
 (логарифмическая шкала): \textit{1}~--- RMSE; \textit{2}~--- Loss; \textit{3}~--- MSE;
 \textit{4}~--- MAE}}
}

\vspace*{11pt}

\addtocounter{figure}{1}


\noindent
 функции потерь, основанной на среднеквадратичной ошибке, 
в~течение~$35$~эпох подряд. На рис.~1 показано изменение 
величины различных использованных метрик в~зависимости от 
эпохи обучения для коэффициента эксцесса, анализируемого с~помощью 
рекуррентной модификации архитектуры~\ref{VII}. Очевидно, что в~данном случае 
нет необходимости в~значительном числе шагов, однако подобная ситуация наблюдалась 
не для всех рядов. Кроме того, через некоторое число эпох после выхода на 
локальное <<плато>> обучение может быть продолжено.



Результаты представлены для метода оптимизации \verb"Adam"~\cite{Kingma2014}; 
использование \verb"SGD"~\cite{Buduma2017} и~\verb"AdaDelta"~\cite{Zeiler2012} 
не давало ощутимых преимуществ.
В качестве функции активации сетей прямого распространения выбрана \verb"ReLU" 
(Rectified Linear Unit)~\cite{Glorot2011}. Для повышения точ\-ности 
прогнозов использовано изменение скорости обучения нейронной сети 
при достижении <<плато>> точности (метод \verb"ReduceLROnPlateau" 
библиотеки \verb"Keras"). Для рекуррентных сетей в~качестве функции активации 
применены гиперболический тангенс и~так называемая рациональная 
сигмоида $x/(1\hm + |x|)$, для которой получены лучшие результаты
 на тестовых данных. Эффект переобучения не наблюдался. Применение 
 дроп\-аут-сло\-ев~\cite{Srivastava2014} не привело к~повышению качества 
 результатов, поэтому в~окончательных вариантах архитектур они не использовались.
 
 \begin{figure*}[b] %fig2
\vspace*{9pt}
    \begin{center}  
  \mbox{%
 \epsfxsize=107.205mm 
 \epsfbox{gor-2.eps}
 }
\end{center}
\vspace*{-9pt}
\Caption{Сравнение среднеквадратичной~(\textit{а}) и~средней абсолютной~(\textit{б}) ошибок
для различных архитектур: \textit{1}~--- математическое ожидание; 
\textit{2}~--- дисперсия; \textit{3}~--- коэффициент асимметрии; \textit{4}~---
коэффициент эксцесса} 
\label{FigRMSE_MAE}
\end{figure*}

\section{Применение нейронных сетей для прогнозирования моментов}
%\label{SecAnalysis}

В данном разделе рассмотрим результаты применения описанных выше архитектур 
к~задаче прогнозирования математического ожидания, дис\-пер\-сии, коэффициентов 
асимметрии и~эксцесса\linebreak
 конечных нормальных смесей. Для сравнения результатов 
прогнозирования использованы сред\-не\-квад\-ра\-тич\-ная ошибка \verb"RMSE", 
функция потерь на основе \verb"MSE" и~$L^2$-регуляризации, а также 
средняя абсолютная ошибка \verb"MAE". Данные нормализованы (сдвинуты и~нормированы) 
так, чтобы все наблюдения принадлежали сегменту $[0, 1]$.

На рис.~\ref{FigRMSE_MAE} представлены величины ошибок\linebreak
 \verb"RMSE" и~\verb"MAE" 
моделей, полученных в~результате обучения~$14$~архитектур на основе базовых 
типов~\ref{I}--\ref{VII}. Символ~<<a>> рядом с~римскими цифрами используется 
для обозначения \verb"LSTM".



Из рис.~\ref{FigRMSE_MAE} видно, что использование рекуррентных архитектур 
во всех случаях уменьшает значение ошибки (точные величины для 
\verb"RMSE" приведены в~табл.~1, наименьшие значения в~каж\-дом столбце 
выделены полужирным шрифтом), при этом в~среднем для всех рядов~--- в~$1{,}33$ 
и~$1{,}31$ раза (\verb"RMSE" и~\verb"MAE" соответственно). Для математического 
ожидания и~коэффициента эксцесса на лучших \verb"LSTM"-ар\-хи\-тек\-ту\-рах 
ошибка в~среднем меньше на~$10\%$--$20\%$, а~для дисперсии и~коэффициента 
асимметрии~--- на $30\%$--$60\%$.  Этот эффект более наглядно проявляется 
для функции потерь (рис.~\ref{FigLoss}).



В среднем для всех рядов в~данной метрике разница составляет~$20{,}3$ раза, 
а~в~отдельных случаях (см.\linebreak\vspace*{-12pt}

\pagebreak



\end{multicols}

\begin{table}
{\small %tabl1
\begin{center}
\parbox{390pt}{\Caption{Значения метрики RMSE и~функции потерь Loss (мантиссы, порядок~--- $10^{-3}$) 
для различных архитектур}
}

%\label{TabErr} 
\vspace*{2ex}

\tabcolsep=10pt
\begin{tabular}{|c|c|c|c|c|c|c|c|c|c|c|}
\hline
%\multicolumn{1}{|c|}{\raisebox{-6pt}[0pt][0pt]{
\multicolumn{2}{|c|}{\ }&\multicolumn{8}{c|}{\bf Момент}\\
\cline{3-10}
\multicolumn{2}{|c|}{\raisebox{6pt}[0pt][0pt]{\bf Архитектура}}&\multicolumn{2}{c|}{\bf Exp}&
\multicolumn{2}{c|}{\bf Var}&\multicolumn{2}{c|}{\bf Skew}&
\multicolumn{2}{c|}{\bf Kurt}\\
\hline
{\bf Тип}&\verb"LSTM"&{\bf \verb"RMSE"}&\verb"Loss"&{\bf \verb"RMSE"}&\verb"Loss"&{\bf \verb"RMSE"}&\verb"Loss"&{\bf \verb"RMSE"}&\verb"Loss"\\
\hline
\multicolumn{1}{|c|}{\raisebox{-6pt}[0pt][0pt]{{\bf I}}}&Нет&$5{,}03$&$3{,}78$&$8{,}43$&$3{,}45$&$11{,}92$&$2{,}42$&$22{,}18$&$1{,}71$\\
%\cline{2-10}
&Да&$5{,}08$&$0{,}23$&$4{,}5$\hphantom{9}&$0{,}11$&$11{,}34$&$0{,}34$&$21{,}94$&$0{,}65$\\
\hline
\multicolumn{1}{|c|}{\raisebox{-6pt}[0pt][0pt]{{\bf II}}}&Нет&
$4{,}67$&$2{,}85$&$8{,}25$&$3$\hphantom{,99}&$11{,}5$\hphantom{9}&$2{,}97$&$22{,}18$&$2{,}07$\\
%\cline{2-10}
&Да&$\mathbf {4{,}04}$&$0{,}16$&$4{,}54$&$\mathbf{0{,}05}$&\hphantom{9}$8{,}38$&$\mathbf{0{,}14}$&$21{,}44$&$\mathbf{0{,}55}$\\
\hline
\multicolumn{1}{|c|}{\raisebox{-6pt}[0pt][0pt]{{\bf III}}}&Нет&$5{,}56$&$4{,}14$&$8{,}51$&$2{,}39$&$12{,}02$&$3{,}15$&$22{,}26$&$2{,}53$\\
%\cline{2-10}
&Да&$5{,}53$&$0{,}19$&6{,}9$\hphantom{9}$&$0{,}22$&\hphantom{9}$9{,}56$&$0{,}23$&$21{,}42$&$0{,}64$\\
\hline
\multicolumn{1}{|c|}{\raisebox{-6pt}[0pt][0pt]{{\bf IV}}}&Нет&$5{,}31$&$3{,}7$\hphantom{9}&$8{,}7$\hphantom{9}&$4{,}53$&$12{,}14$&$3{,}46$&$22{,}22$&$2{,}54$\\
%\cline{2-10}
&Да&$5{,}45$&$0{,}17$&$4{,}86$&$0{,}09$&\hphantom{9}$8{,}59$&$0{,}18$&$21{,}34$&$0{,}57$\\
\hline
\multicolumn{1}{|c|}{\raisebox{-6pt}[0pt][0pt]{{\bf V}}}&Нет&$5{,}53$&$4{,}68$&$8{,}61$&$3{,}23$&$11{,}93$&$3{,}16$&$22{,}18$&$2{,}19$\\
%\cline{2-10}
&Да&$4{,}94$&$0{,}19$&$7{,}37$&$0{,}2$\hphantom{9}&\hphantom{9}$\mathbf{7{,}72}$&$0{,}18$&$\mathbf{20{,}06}$&$\mathbf{0{,}55}$\\
\hline
\multicolumn{1}{|c|}{\raisebox{-6pt}[0pt][0pt]{{\bf VI}}}&Нет&$6{,}14$&$2{,}82$&$8{,}78$&$2{,}41$&$11{,}79$&$1{,}81$&$22{,}19$&$1{,}74$\\
%\cline{2-10}
&Да&$5{,}85$&$0{,}08$&$\mathbf{3{,}38}$&$0{,}09$&\hphantom{9}$9{,}23$&$0{,}22$&$21{,}57$&$0{,}59$\\
\hline
\multicolumn{1}{|c|}{\raisebox{-6pt}[0pt][0pt]{{\bf VII}}}&
Нет&$5{,}85$&$4{,}67$&$8{,}89$&$4{,}24$&$12{,}03$&$3{,}41$&$22{,}15$&$2{,}6$\hphantom{9}\\
%\cline{2-10}
&Да&$4{,}71$&$\mathbf {0{,}07}$&$3{,}71$&$0{,}11$&\hphantom{9}$8{,}01$&$0{,}17$&$20{,}69$&
$\mathbf{0{,}55}$\\
\hline
\end{tabular}
\end{center}}
%\vspace*{4pt}
%\end{table}
%\begin{figure} %fig3
%\renewcommand{\figurename}{\protect\bf }
\renewcommand{\tablename}{\protect\bf Рис.}
\setcounter{table}{2}
\vspace*{1pt}
    \begin{center}  
  \mbox{%
 \epsfxsize=104.157mm 
 \epsfbox{gor-3.eps}
 }
\end{center}
\vspace*{-14pt}
\Caption{Сравнение величины функции потерь для различных архитектур: 
\textit{1}~--- математическое ожидание; \textit{2}~--- 
дисперсия; \textit{3}~--- коэффициент асимметрии; \textit{4}~---
коэффициент эксцесса}
\label{FigLoss}
\vspace*{-1pt}
\end{table}

\renewcommand{\figurename}{\protect\bf Рис.}
\renewcommand{\tablename}{\protect\bf Таблица}
\setcounter{figure}{3}
\setcounter{table}{1}

\begin{multicols}{2}

\noindent
 математическое ожидание для конфигураций~\ref{VII} 
и~\ref{VII}{a}) получается более чем $65$-крат\-ное уменьшение величины 
ошибки. Отдельно можно выделить конфигурации~\ref{II}{a} и~\ref{VII}{a} (т.\,е.\ 
рекуррентные сети), в~которых для всех моментных характеристик сразу в~обеих 
метриках получены либо наименьшие среди всех, либо близкие к~этому значения. 
Таким образом, применение рекуррентных архитектур ведет к~значительному 
повышению качества обучения в~любой из рассматриваемых метрик. 

\begin{table*}[b]\small %tabl2
\vspace*{-8pt}
\begin{center}
\Caption{Время, затраченное на обучение на ГВВК (в секундах)}
\label{TabTime}
\vspace*{2ex}

\tabcolsep=10pt
\begin{tabular}{|c|c|r|r|r|r|}
\hline
\multicolumn{2}{|c|}{\bf Архитектура}&\multicolumn{4}{c|}{\bf Момент}\\
\hline
{\bf Тип}&\verb"LSTM"&\multicolumn{1}{c|}{{\bf Exp}}&
\multicolumn{1}{c|}{{\bf Var}}&
\multicolumn{1}{c|}{{\bf Skew}}&\multicolumn{1}{c|}{{\bf Kurt}}\\
\hline
\multicolumn{1}{|c|}{\raisebox{-6pt}[0pt][0pt]{{\bf I}}}&Нет&$38{,}08$&$32{,}99$&$303{,}79$&$67{,}72$\\
%\cline{2-6}
&Да&$2081{,}98$&$1932{,}59$&$1872{,}37$&$2034{,}13$\\
\hline
\multicolumn{1}{|c|}{\raisebox{-6pt}[0pt][0pt]{{\bf II}}}&Нет&$51{,}21$&$31{,}99$&$37{,}93$&$71{,}48$\\
%\cline{2-6}
&Да&$2170{,}9$\hphantom{9}&$1999{,}57$&$1985{,}41$&$1079{,}17$\\
\hline
\multicolumn{1}{|c|}{\raisebox{-6pt}[0pt][0pt]{{\bf III}}}&Нет&$45{,}61$&$49{,}82$&$56{,}26$&$80{,}56$\\
%\cline{2-6}
&Да&$3114{,}08$&$2813{,}48$&$2815{,}11$&$1362{,}48$\\
\hline
\multicolumn{1}{|c|}{\raisebox{-6pt}[0pt][0pt]{{\bf IV}}}&Нет&$63{,}83$&$36{,}16$&$57{,}23$&$80{,}04$\\
%\cline{2-6}
&Да&$3108{,}84$&$2839{,}28$&$2870{,}9$\hphantom{9}&$1680{,}07$\\
\hline
\multicolumn{1}{|c|}{\raisebox{-6pt}[0pt][0pt]{{\bf V}}}&
Нет&$58{,}79$&$40{,}7$\hphantom{9}&$61{,}33$&$79{,}21$\\
%\cline{2-6}
&Да&$3194{,}87$&$2983{,}34$&$2988{,}7$\hphantom{9}&$3159{,}44$\\
\hline
\multicolumn{1}{|c|}{\raisebox{-6pt}[0pt][0pt]{{\bf VI}}}&Нет&$73{,}9$\hphantom{9}&$42{,}52$&$71{,}38$&$61{,}83$\\
%\cline{2-6}
&Да&$3897{,}8$\hphantom{9}&$3826{,}78$&$3756{,}39$&$2435{,}18$\\
\hline
\multicolumn{1}{|c|}{\raisebox{-6pt}[0pt][0pt]{{\bf VII}}}&Нет&$71{,}72$&$275{,}35$&$68{,}82$&$61{,}96$\\
%\cline{2-6}
&Да&$3534{,}68$&$4018{,}36$&$3890{,}21$&$1909{,}27$\\
\hline
\end{tabular}
\end{center}
\end{table*}


Для сравнения архитектур определим индикатор, который представим как 
сумму отношений ошибки в~некоторой метрике на данной архитектуре к~минимальной
 ошибке прогноза для данного ряда (случай нулевой величины на практике является 
 почти недостижимым). Очевидно, что его наименьшее значение~--- $4$. 

 На тестовых данных минимальные значения данного индикатора достигаются на 
 архитектуре~\ref{II}{a} для математического ожидания, \ref{VI}{a} 
 для дисперсии и~\ref{V}{a} для коэффициентов асимметрии и~эксцесса. 
 При этом оптимальными в~общем случае можно признать конфигурации~\ref{II} 
 и~\ref{II}{a}, поскольку для них получается минимальная средневзвешенная
  сумма ошибок для сетей прямого распространения~($4{,}001$) 
  и~второе по величине значение после архитектуры~\ref{VII}{a}, 
  для которой введенный индикатор равен~$4{,}49$, среди всех рекуррентных сетей.
  
    { \begin{center}  %fig4
 \vspace*{-4pt}
   \mbox{%
 \epsfxsize=78.945mm 
 \epsfbox{gor-4.eps}
 }


\end{center}


\noindent
{{\figurename~4}\ \ \small{Коэффициент эксцесса и~прогнозы архитектуры~\ref{VII}{a}:
\textit{1}~--- данные; \textit{2}~--- модель}}
}

\vspace*{15pt}

%\addtocounter{figure}{1}



Для иллюстрации качества приближения данных обученными моделями 
на рис.~4 продемонстрированы значения для коэффициента 
эксцесса (нормализованные данные) и~аппроксимация ряда с~помощью предсказаний, 
сделанных с~применением рекуррентной архитектуры~\ref{VII}{a}.

%\vspace*{-2pt}

\section{Повышение эффективности обучения нейронных сетей за~счет 
использования высокопроизводительных вычислительных средств}
%\label{SecHPC}

В данном разделе будут обсуждаться вопросы, связанные с~временн$\acute{\mbox{ы}}$ми 
затратами на обучение базовых архитектур~\ref{I}--\ref{VII}. 
Результаты разд.~3 означают, что использование рекуррентных сетей позволило 
существенным образом повысить качество аппроксимации исходных данных. 
Однако нельзя не принимать во внимание тот факт, что усложнение конфигурации 
влечет за собой и~дополнительную вычислительную нагрузку при обучении. 
В~табл.~\ref{TabTime} приведены временн$\acute{\mbox{ы}}$е затраты для <<обычной>> 
и~\verb"LSTM"-ар\-хи\-тек\-тур, при этом не учитываются накладные расходы 
на предварительную обработку данных и~инициализацию графических видеокарт. 



В среднем сети прямого распространения обуча\-лись за~$664$~эпохи, в~то время как 
рекуррентные модификации~--- за $687$. Для рассматриваемых рядов было установлено, 
что для \verb"LSTM"-кон\-фи\-гу\-ра\-ций необходимое время обучения в~среднем 
превышает результаты для классических в~$47$~раз (минимальное значение~--- $6$~раз, 
максимальное~--- $90$) в~за\-ви\-си\-мости от архитектуры и~анализируемого ряда. Из  
отмеченных в~разд.~3 конфигураций~\ref{II}{a} и~\ref{VII}{a} 
лучшее время показывает именно~\ref{II}{a}~--- около~$30$~мин в~среднем на 
моментную характеристику, в~то время как для~\ref{VII}{a} получается более~$55$~мин. 
Таким образом, для более быстрой обработки рядов можно порекомендовать именно 
архитектуру~\ref{II}{a}.

Эффект от использования средств ГВВК проявляется прежде всего при обучении нейронных 
сетей прямого распространения. Так, среднее время на одну эпоху для 
сетей прямого распространения составляет~$2$~с для настольного решения 
(процессор \verb"Core i7", видеокарта \verb"NVIDIA GTX970M"), а~для ГВВК~--- 
$0{,}1$~с. Однако и~для рекуррентных конфигураций получено пятикратное ускорение,
 а~именно: с~$25$ до~$5$~с,~--- которое позволяет рассматривать их в~качестве инструмента 
 решения реальных прикладных задач.


\section{Заключение}

В~работе проведено сравнение результатов для~$14$~различных архитектур нейронных 
сетей в~задаче прогнозирования первых четырех моментных характеристик конечных 
смесей нормальных распределений. Выбраны две лучшие конфигурации 
(с~одним скрытым слоем из~$100$~нейронов и~с~тремя слоями по~$50$~нейронов), 
для которых получены наиболее точные значения во всех рас\-смат\-ри\-ва\-емых метриках. 
При этом на обучение конфигурации с~одним слоем затрачивается меньшее время, 
поэтому она может быть использована в~задачах, для которых более критично 
быстродействие, а~не точ\-ность аппроксимации. Использованные подходы в~достаточной 
степени универсальны, поэтому полученные архитектуры могут быть успешно 
применены для анализа временн$\acute{\mbox{ы}}$х рядов различной природы.


{\small\frenchspacing
 {%\baselineskip=10.8pt
 \addcontentsline{toc}{section}{References}
 \begin{thebibliography}{99}
\bibitem{Yang2012} \Au{Yang~M.-Sh., Lai~Ch.-Yo, Lin~C.-Y.} 
A~robust EM clustering algorithm for Gaussian mixture models~// 
Pattern Recogn., 2012. Vol.~45. Iss.~11. P.~3950--3961.

\bibitem{Cai2019} \Au{Cai~T.\,T., Ma~J., Zhang~L.} CHIME: Clustering 
of high-dimensional Gaussian mixtures with EM algorithm and its optimality~// 
Ann. Stat., 2019. Vol.~47. Iss.~3 P.~1234--1267.

\bibitem{Hassen2019} \Au{Ben Hassen~H., Masmoudi~K., Masmoudi~A.} 
Model selection in biological networks using a graphical EM algorithm~// 
Neurocomputing, 2019. Vol.~349. P.~271--280.

\bibitem{Zeller2019} \Au{Zeller~C.\,B., Cabral~C.\,R.\,B., Lachos~V.\,H., Benites~L.} 
Finite mixture of regression models for censored data based on scale mixtures of 
normal distributions~// Adv. Data Anal. Classi., 2019. Vol.~13. Iss.~1. P.~89--116.

\bibitem{Osoba2013} \Au{Osoba~O., Mitaim~S., Kosko~B.} 
The noisy Expectation--Maximization algorithm~// Fluct. Noise Lett., 2013. 
Vol.~12. Iss.~3. Art.~No.\,1350012.

\bibitem{Lee2018} \Au{Lee~S.\,X., Leemaqz~K.\,L., McLachlan~G.\,J.} 
A~block EM algorithm for multivariate skew normal and skew t-mixture
models~// IEEE~T. Neur. Net. Lear., 2018. 
Vol.~29. Iss.~11. P.~5581--5591.

\bibitem{Wu2019} \Au{Wu~D., Ma~J.} 
An effective EM algorithm for mixtures of Gaussian processes via the MCMC 
sampling and approximation~// Neurocomputing, 2019. Vol.~331. P.~366--374.

\bibitem{Verbeek2003} \Au{Verbeek~J.\,J., Vlassis~N., Krose~B.} 
Efficient greedy learning of Gaussian mixture models~// 
Neural Comput., 2003. Vol.~15. Iss.~2. P.~469--485.

\bibitem{Gorshenin2011} \Au{Горшенин~А.\,К.} 
Проверка статистических гипотез в~модели расщепления компоненты~// 
Вестник Мос\-ков\-ско\-го университета. Сер.~15: Вычислительная математика и~кибернетика, 
2011. Вып.~4. С.~26--32.

\bibitem{Liu2019}  \Au{Liu~C., Li~H.-C., Fu~K., Zhang~F., Datcu~M., Emery~W.\,J.} 
A~robust EM clustering algorithm for Gaussian mixture models~// 
Pattern Recogn., 2019. Vol.~87. P.~269--284.

\bibitem{Yu2019}  \Au{Yu~L., Yang~T., Chan~A.\,B.} 
Density-preserving hierarchical EM algorithm: Simplifying Gaussian mixture
models for approximate inference~// IEEE~T. Pattern Anal., 
2019. Vol.~41. Iss.~6. P.~1323--1337.

\bibitem{Gorshenin2019a}   \Au{Batanov~G.\,M., Borzosekov~V.\,D., 
Gorshenin~A.\,K., Kharchev~N.\,K., Korolev~V.\,Yu., Sarskyan~K.\,A.}
Evolution of statistical properties of microturbulence 
during transient process under electron cyclotron resonance heating of the 
L-2M stellarator plasma~// Plasma Phys. Contr.~F., 2019. Vol.~61. Iss.~7. 
Art.~No.\,075006.

\bibitem{Korolev2011}  \Au{Королев~В.\,Ю.} 
Ве\-ро\-ят\-но\-ст\-но-ста\-ти\-сти\-че\-ские методы декомпозиции волатильности 
хаотических процессов.~--- М.: Изд-во Моск. ун-та, 2011. 512~с.

\bibitem{Gorshenin2019b}  \Au{Gorshenin~A.\,K., Kuzmin~V.\,Yu.} 
Improved architecture of feedforward neural networks to increase accuracy of  
predictions for moments of finite normal mixtures~// 
Pattern Recogn. Image Anal., 2019. Vol.~29. No.\,1. P.~79--88.

\bibitem{Buduma2017}  \Au{Buduma~N.} Fundamentals of deep learning: 
Designing next-generation machine intelligence algorithms.~--- 
Sebastopol, CA, USA: O'Reilly Media, 2017. 298~p.

\bibitem{Greff2017}  
\Au{Greff~K., Srivastava~R.\,K., Koutnik~J.,
 Steunebrink~B.\,R., Schmidhuber~J.} 
LSTM: A~search space Odyssey~// IEEE~T. Neur. Net. 
Lear., 2017. Vol.~28. Iss.~10. P.~2222--2232.

\bibitem{Gorshenin2016a}  
\Au{Горшенин~А.\,К.} Концепция он\-лайн-ком\-плек\-са 
для стохастического моделирования реальных процессов~// Информатика и~её 
применения, 2016. Т.~10. Вып.~1. C.~72--81. 

\bibitem{Kingma2014}  
\Au{Kingma~D., Ba~J.}  
Adam: A~method for stochastic optimization~// 3rd 
 Conference (International) for Learning Representations~// arXiv:1412.6980, 2015. 13~p.

\bibitem{Zeiler2012}  
 \Au{Zeiler~M.\,D.} 
ADADELTA: An adaptive learning rate method~// arXiv:1212.5701, 2012. 6~p.

\bibitem{Glorot2011}  
\Au{Glorot~X., Bordes~A., Bengio~Y.} 
Deep sparse rectifier neural
networks~// J.~Mach. Learn. Res., 2011. Vol.~15. P.~315--323.

\bibitem{Srivastava2014}   
\Au{Srivastava~N., Hinton~G., Krizhevsky~A., 
Sutskever~I., Salakhutdinov~R.} Dropout: 
A~simple way to prevent neural networks from overfitting~// J.~Mach. Learn. Res., 
2014. Vol.~15. P.~1929--1958.
 \end{thebibliography}

 }
 }

\end{multicols}

\vspace*{-6pt}

\hfill{\small\textit{Поступила в~редакцию 04.09.19}}

%\vspace*{8pt}

%\pagebreak

\newpage

\vspace*{-28pt}

%\hrule

%\vspace*{2pt}

%\hrule

%\vspace*{-2pt}

\def\tit{APPLICATION OF RECURRENT NEURAL NETWORKS TO~FORECASTING THE~MOMENTS 
OF~FINITE NORMAL MIXTURES}


\def\titkol{Application of recurrent neural networks to~forecasting the moments 
of~finite normal mixtures}

\def\aut{A.\,K.~Gorshenin$^{1,2}$ and~V.\,Yu.~Kuzmin$^3$}

\def\autkol{A.\,K.~Gorshenin and~V.\,Yu.~Kuzmin}

\titel{\tit}{\aut}{\autkol}{\titkol}

\vspace*{-11pt}


\noindent
$^1$Institute of Informatics Problems, Federal Research Center ``Computer Science and
Control'' of the Russian\linebreak
$\hphantom{^1}$Academy of Sciences, 44-2~Vavilov Str., Moscow 119333, Russian
Federation  

\noindent
$^2$Faculty of Computational Mathematics and Cybernetics, M.\,V.~Lomonosov Moscow
State University, GSP-1,\linebreak
$\hphantom{^1}$Leninskie Gory, Moscow 119991, Russian Federation

\noindent
$^3$ ``Wi2Geo LLC'', 3-1~Mira Ave., Moscow 129090, Russian Federation


\def\leftfootline{\small{\textbf{\thepage}
\hfill INFORMATIKA I EE PRIMENENIYA~--- INFORMATICS AND
APPLICATIONS\ \ \ 2019\ \ \ volume~13\ \ \ issue\ 3}
}%
 \def\rightfootline{\small{INFORMATIKA I EE PRIMENENIYA~---
INFORMATICS AND APPLICATIONS\ \ \ 2019\ \ \ volume~13\ \ \ issue\ 3
\hfill \textbf{\thepage}}}

\vspace*{3pt} 



\Abste{The article compares the application of feedforward and recurrent 
neural networks to forecasting continuous values of expectation, variance,
 skewness, and kurtosis of finite normal mixtures. Fourteen various architectures
of neural networks are considered. 
% including the {\sf LSTM} (Long-Short Term Memory)
%  with different numbers of hidden layers and 
% neurons are considered. 
To increase training speed, the high-performance 
 computing cluster is used. It is demonstrated that the best forecasting 
 results based on standard metrics (root-mean-square error, mean absolute errors, 
 and loss function) are achieved on the two {\sf LSTM} 
 (Long-Short Term Memory) networks: with~100 neurons
  in one hidden layer and~50~neurons in each three hidden layers.}

\KWE{recurrent neural networks; forecasting; deep learning; high-performance
 computing; CUDA}


\DOI{10.14357/19922264190316} 

%\vspace*{-14pt}

\Ack
\noindent
The research is partially supported by the Russian Foundation for Basic Research
(projects 18-29-03100 and 19-07-00352) and the RF Presidential 
scholarship program (project No.\,538.2018.5). The calculations were performed 
using Hybrid high-performance computing cluster of FRC CSC RAS 
({\sf http://hhpcc.frccsc.ru}).


%\vspace*{-6pt}

  \begin{multicols}{2}

\renewcommand{\bibname}{\protect\rmfamily References}
%\renewcommand{\bibname}{\large\protect\rm References}

{\small\frenchspacing
 {%\baselineskip=10.8pt
 \addcontentsline{toc}{section}{References}
 \begin{thebibliography}{99}
\bibitem{1-gk}
\Aue{Yang,~M.-Sh., Ch.-Yo~Lai, and C.-Y.~Lin.} 2012. A~robust EM clustering algorithm for 
Gaussian mixture models. \textit{Pattern Recogn.} 45(11):3950--3961.

\bibitem{2-gk}
\Aue{Cai,~T.\,T., J.~Ma, and L.~Zhang.} 2019. CHIME: Clustering 
of high-dimensional Gaussian mixtures with em algorithm and its optimality. 
\textit{Ann. Stat.} 47(3):1234--1267.

\bibitem{3-gk}
\Aue{Ben Hassen,~H., K.~Masmoudi, and A.~Masmoudi.} 2019. 
Model selection in biological networks using a graphical EM algorithm. 
 \textit{Neurocomputing} 349:271--280.

\bibitem{4-gk}
\Aue{Zeller,~C.\,B., C.\,R.\,B.~Cabral, V.\,H.~Lachos, and L.~Benites.}
 2019. Finite mixture of regression models for censored 
 data based on scale mixtures of normal distributions. 
  \textit{Adv. Data Anal. Classi.} 13(1):89--116.

\bibitem{5-gk}
\Aue{Osoba,~O., S.~Mitaim, and B.~Kosko.} 2013. 
The noisy Expectation-Maximization algorithm. 
 \textit{Fluct. Noise Lett.} 12(3):1350012.

\bibitem{6-gk}
\Aue{Lee,~S.\,X., K.\,L.~Leemaqz, and G.\,J.~McLachlan.} 2018. 
A~block EM algorithm for multivariate skew normal and skew t-mixture models. 
 \textit{IEEE~T. Neur. Net. Lear.} 29(11):5581--5591.

\bibitem{7-gk}
\Aue{Wu,~D., J.~Ma.} 2019. An effective EM algorithm for mixtures 
of Gaussian processes via the MCMC sampling and approximation. 
 \textit{Neurocomputing} 331:366--374.

\bibitem{8-gk}
\Aue{Verbeek,~J.\,J., N.~Vlassis, and B.~Krose.} 
2003. Efficient greedy learning of Gaussian mixture models. 
 \textit{Neural Comput.} 15(2):469--485.

\bibitem{9-gk}
\Aue{Gorshenin,~A.\,K.} 2011. Testing of statistical hypotheses in the 
splitting component model.  \textit{Mosc. Univ. Comput. Math. 
Cybernetics} 35(4):176--183.

\bibitem{10-gk}
\Aue{Liu,~C., H.-C.~Li, K.~Fu, F.~Zhang, M.~Datcu, and W.\,J.~Emery.} 
2019. A~robust EM clustering algorithm for Gaussian mixture models. 
 \textit{Pattern Recogn.} 87:269--284.

\bibitem{11-gk}
\Aue{Yu,~L., T.~Yang, and A.\,B.~Chan.} 2019. 
Density-preserving hierarchical EM algorithm: Simplifying Gaussian 
mixture models for approximate inference. 
 \textit{IEEE~T. Pattern Anal.} 41(6):1323--1337.

\bibitem{12-gk}
\Aue{Batanov,~G.\,M., V.\,D.~Borzosekov, A.\,K.~Gorshenin, N.\,K.~Kharchev, 
V.\,Yu.~Korolev, and K.\,A.~Sarskyan.}  2019. Evolution of statistical properties 
of microturbulence during transient process under electron cyclotron resonance 
heating of the L-2M stellarator plasma. 
 \textit{Plasma Phys. Contr.~F.} 61(7):075006.

\bibitem{13-gk}
\Aue{Korolev,~V.\,Yu.} 2011.  \textit{Veroyatnostno-statisticheskie metody dekompozitsii 
volatil'nosti khaoticheskikh protsessov} 
[Probabilistic and statistical methods of decomposition 
of volatility of chaotic processes]. Moscow: Moscow University Publishing House. 512~p.

\bibitem{14-gk}
\Aue{Gorshenin,~A.\,K., and V.\,Yu.~Kuzmin.}
 2019. Improved architecture of feedforward neural networks to increase accuracy 
 of predictions for moments of finite normal mixtures. 
  \textit{Pattern Recog. Image Anal.} 29(1):68--77.

\bibitem{15-gk}
\Aue{Buduma,~N.} 2017. \textit{Fundamentals of deep learning: 
Designing next-generation machine intelligence algorithms}. 
 Sebastopol, CA: O'Reilly Media. 298~p.

\bibitem{16-gk}
\Aue{Greff,~K., R.\,K.~Srivastava, J.~Koutnik, B.\,R~Steunebrink, 
and J.~Schmidhuber.}  2017. LSTM: A~search space Odyssey. 
 \textit{IEEE~T. Neur. Net. Lear.} 28(10):2222--2232.

\bibitem{17-gk}
\Aue{Gorshenin,~A.\,K.} 2016. Kontseptsiya onlayn-kompleksa dlya stokhasticheskogo 
modelirovaniya real'nykh pro\-tses\-sov [Concept of online service for stochastic modeling of real 
processes].  \textit{Informatika i~ee Primeneniya~--- Inform. Appl.} 10(1):72--81.

\bibitem{18-gk}
\Aue{Kingma,~D., and J.~Ba.} 2015. Adam: A~method for stochastic optimization. 
 \textit{3rd Conference (International) for Learning Representations}. 
 arXiv:1412.6980. 13~p.

\bibitem{19-gk}
\Aue{Zeiler,~M.\,D.} 2012. ADADELTA: An adaptive learning rate method. 
arXiv:1212.5701. 6~p.

\bibitem{20-gk}
\Aue{Glorot,~X., A.~Bordes, and Y.~Bengio.} 
2011. Deep sparse rectifier neural networks.
 \textit{J.~Mach. Learn. Res.} 15:315--323.

\bibitem{21-gk}
\Aue{Srivastava,~N., G.~Hinton, A.~Krizhevsky, I.~Sutskever, and R.~Salakhutdinov.}
 2014. Dropout: A~simple way to prevent neural networks from overfitting.
 \textit{J.~Mach. Learn. Res.} 15:1929--1958.
 
 \end{thebibliography}

 }
 }

\end{multicols}

%\vspace*{-7pt}

\hfill{\small\textit{Received September 4, 2019}}

%\pagebreak

%\vspace*{-22pt}


\Contr


\noindent
\textbf{Gorshenin Andrey K.} (b.\ 1986)~--- Candidate of Science (PhD) in physics and
mathematics, associate professor, leading scientist, Institute of Informatics Problems,
Federal Research Center ``Computer Science and Control'' of the Russian Academy of
Sciences, 44-2~Vavilov Str., Moscow 119333, Russian Federation;  
leading scientist, Faculty
of Computational Mathematics and Cybernetics, M.\,V.~Lomonosov Moscow State University, GSP-1,
Leninskie Gory, Moscow 119991, Russian Federation; \mbox{agorshenin@frccsc.ru}

\vspace*{3pt}

\noindent
\textbf{Kuzmin Victor Yu.} (b.\ 1986)~--- Head of Development, ``Wi2Geo LLC,'' 
3-1~Mira
Ave., Moscow 129090, Russian Federation; \mbox{shadesilent@yandex.ru}



\label{end\stat}

\renewcommand{\bibname}{\protect\rm Литература}   %16
 \def\stat{rumov+kir}

\def\tit{МЕТОДЫ МОДЕЛИРОВАНИЯ И ВИЗУАЛЬНОГО ПРЕДСТАВЛЕНИЯ КОНФЛИКТА 
В~МАЛОМ КОЛЛЕКТИВЕ ЭКСПЕРТОВ, РЕШАЮЩИХ ПРОБЛЕМЫ (ОБЗОР)}

\def\titkol{Методы моделирования и~визуального представления конфликта 
в~малом коллективе экспертов} %, решающих проблемы (обзор)}

\def\aut{С.\,Б.~Румовская$^1$, И.\,А.~Кириков$^2$}

\def\autkol{С.\,Б.~Румовская, И.\,А.~Кириков}

\titel{\tit}{\aut}{\autkol}{\titkol}

\index{Румовская С.\,Б.}
\index{Кириков И.\,А.}
\index{Rumovskaya S.\,B.}
\index{Kirikov I.\,A.}


%{\renewcommand{\thefootnote}{\fnsymbol{footnote}} \footnotetext[1]
%{Работа частично поддержана РФФИ (проект 18-07-00274).}}


\renewcommand{\thefootnote}{\arabic{footnote}}
\footnotetext[1]{Калининградский филиал Федерального исследовательского центра <<Информатика и~управ\-ле\-ние>> 
Российской академии наук, \mbox{sophiyabr@gmail.com}}
\footnotetext[2]{Калининградский филиал Федерального исследовательского центра <<Информатика  
и~управ\-ле\-ние>> Российской академии наук, \mbox{baltbipiran@mail.ru}}

\vspace*{-14pt}


  
  
 
      

   \Abst{Малые коллективы экспертов как естественный коллективный интеллект 
поддержки принятия решений (гетерогенный коллектив) эффективно решают сложные 
проблемы. При этом такая форма взаимодействия между экспертами, как конфликт, 
порождает позитивные изменения в~коллективе: развитие группы, диагностику отношений, 
снятие напряжения, сплачивание группы, а~также способствует сохранению коллектива. 
В~человеческом мышлении огромную роль играют за\-го\-тов\-ки-схе\-мы стандартных ситуаций, 
использование которых существенно ускоряет рассуждения. Визуализация конфликтной 
ситуации делает возникшие противоречия контрастными, видимыми, предоставляя новую 
информацию для разрешения конфликтов, делая их легкоуправляемыми и~позволяя 
контролировать влияние на них субъективных предпочтений. Рас\-смот\-ре\-но понятие 
конфликта в~малых коллективах, его особенности, структура и~динамика, а~также подходы 
к~моделированию и~визуальному представлению конфликтологического аспекта групповой 
динамики экспертов, решающих проблемы.}
    
  \KW{малый коллектив экспертов; конфликт; модели конфликта; визуализация 
конфликта}

\DOI{10.14357/19922264190317} 
  
\vspace*{-1pt}


\vskip 10pt plus 9pt minus 6pt

\thispagestyle{headings}

\begin{multicols}{2}

\label{st\stat}

\section{Введение}

  Естественный коллективный интеллект (ге\-терогенный коллектив) поддержки 
принятия ре\-шений~\cite{1-r}~--- малая группа экспертов, которой\linebreak присущи 
неоднородность, разнообразие, со\-труд\-ничество, дополнительность 
и~относительность знаний. Подобные коллективы эффективно решают 
сложные проблемы. Ввиду этого малые группы, проблемы взаимодействия 
людей внутри них, а~также моделирование их взаимодействия занимают особое 
место в~широком спектре направлений современной науки. Часто 
встречающаяся форма организации малых коллективов~--- совещания, 
построенные по принципу <<круглого стола>>~\cite{3-r} \mbox{с~целью} выявления 
и~решения проб\-лем. Разного рода конфликты порождают дискуссии, глубина 
которых позволяет получить более продуманные и~согласованные решения. 
Координация работы экспертов в~группе лицом, принимающим решения (ЛПР), 
позволяет повысить качество решений, а~самоорганизация в~группе определяет 
способность чутко реагировать на изменения во внешней среде, корректируя 
свое функционирование и~при\-ни\-ма\-емые решения. 
  
  В~\cite{2-r} представлена система для опе\-ра\-тив\-но-про\-из\-вод\-ст\-вен\-но\-го 
планирования~--- моделировалась координация ЛПР (начальником 
производственного отдела) коллективной работы главного конструктора, 
главного технолога, начальника отдела материального снабжения, начальника 
электромеханического цеха, начальника отдела продаж \mbox{с~целью} повышения 
качества оперативных план-гра\-фи\-ков мелкосерийного производства. 
Относительная погрешность результатов решения задачи с~координацией~--- 
менее~1\%, а без~--- до~36\%.
  
  В~\cite{2-r} также описана модель самоорганизации\linebreak в~группе для 
транспортной логистики гибридными интеллектуальными многоагентными 
сис\-те\-ма\-ми (ГиМАС). Алгоритм функционирования сис\-те\-мы динамически 
перестраивается, вырабатывая\linebreak
 релевантный сложной задаче метод решения 
и~сокращая среднюю суммарную се\-бе\-сто\-и\-мость и~время доставки грузов 
в~день на~7,2\% и~12,13\% соответственно; среднее время по\-стро\-ения 
маршрутов уменьшилось на~23,14\%.


  
  Долгое время социологи и~психологи считали, что конфликты~--- негативное 
явление и~их надо устранять. Однако работа Г.~Зиммеля~\cite{3-r} 
способствовала развитию идеи наличия позитивных изменений, порождаемых 
конфликтами: сохранение социальной системы, развитие группы, диагностика 
отношений, снятие напряжения, сплачивание группы и~др. 
  
  Таким образом, моделирование развития и~разрешения конфликта в~рамках 
методологии ГиМАС позволит спроектировать функционирование сис\-те\-мы 
релевантно групповой динамике коллектива экспертов, решающих проб\-ле\-му, 
и~тем самым существенно повысить качество принимаемых ре\-шений. При 
этом визуализация конфликтной\linebreak си\-туации сделает возникшие противоречия 
контрастными, видимыми, предоставляя новую информацию для разрешения 
конфликтов в~реальном коллективе экспертов. В~работе рассмотрено\linebreak понятие 
конфликта в~малых коллективах, его особенности, структура и~динамика, 
а~также подходы к~моделированию и~визуальному представлению 
конфликтологического аспекта групповой динамики экспертов, решающих 
проблемы. 

\vspace*{-6pt}
  
\section{Малый коллектив экспертов: понятие и~классификация}

\vspace*{-2pt}
  
  Попытки определить малую группу экспертов сводились к~субъективному 
пониманию и~фокусировке на тех или иных сторонах группового процесса 
(определенных априори либо эмпирическим): М.~Шоу~\cite{4-r} 
(психологическая со\-став\-ля\-ющая), Р.~Браун~\cite{5-r} 
и~Г.\,М.~Андреева~\cite{6-r} (социальная составляющая). 
  
  Приведем определение, включающее в~себя психологическую и~социальную 
составляющие: <<малая группа>>~--- элементарное звено структуры 
социальных отношений, обретающее через не-\linebreak посредственные межличностные 
контакты струк-\linebreak турные, динамические, феноменологические\linebreak характеристики, 
отражающие признаки группы как целостной сис\-те\-мы социальных 
и~психологических отношений. Понятия <<малая группа>> и~<<малый 
коллектив>> идентичны.
  
  Верхняя граница размерности коллектива (нижняя~--- 2~участника) 
определяется с~двух точек зрения:
\begin{enumerate}[(1)]
\item в соответствии с~требованиями 
реализации ее основной функции~\cite{6-r}~--- верхняя граница не может быть 
обозначена априори; 
\item относительно успешного руководства группой~--- 
верхняя граница соответствует <<магическому числу>> Дж.~Миллера ($7\pm 
2$), так как при численности свыше~10~чел.\ возрастает число подгрупп 
и~вероятность противостояния ЛПР, осложняется координация.
\end{enumerate}
  
  Есть различные классификации малых коллективов~\cite{6-r}, в~частности 
английский психолог М.~Аргайл выделяет~\cite{7-r}: \textit{семью; 
подростково-юношеские коллективы; рабочие коллективы}~--- модель малых 
групп с~четкой трудовой направленностью и~доминирующими отношениями 
делового характера; \textit{комитеты} и~\textit{группы по решению  
проб\-лем}~--- модель коммуникативных малых коллективов, задача которых 
принимать эффективные решения, а их участники должны владеть навыками 
организации информационного обмена, достижения внутригруппового 
согласия и~т.\,п.; \textit{тренинговые} и~\textit{терапевтические группы}.


\vspace*{-6pt}
  
\section{Конфликт: понятие, структура и~типология}

\vspace*{-2pt}

  Конфликт~\cite{8-r, 9-r}~--- столкновение противоположно направленных 
интересов, целей, взглядов и~т.\,п.\ (обострение противоречия) при 
взаимодействии и~взаимоотношении сторон, вос\-при\-ни\-ма\-емое субъектом как 
значимая для него психологическая проблема, требующая своего разрешения 
и~вызывающая активность, направленную на его преодоление. Структура 
конфликта представлена на рис.~1~\cite{10-r, 11-r}. 



  Стороны конфликта~--- это субъекты социального взаимодействия: 
в~состоянии конфликта (основные участники, оппоненты); или же явно или 
неявно поддерживающие конфликтующих (группы\linebreak поддержки); или 
оказывающие эпизодическое влияние на конфликт (другие участники~--- 
подстрекатели, организаторы и~т.\,д.). Объект находится на пересечении 
личных или групповых интересов субъектов. Предмет конфликта~--- это 
противоречие, из-за которого и~для разрешения которого возникает конфликт. 
Образ конфликтной ситуации~--- это отображение предмета конфликта 
в~сознании субъектов. Мотивы конфликта~--- это потребности, интересы, цели 
и~т.\,д., подталкивающие субъектов к~конфликту. 
  
  Конфликты классифицируются по различным признакам  
(рис.~2)~\cite{8-r, 10-r}. 
  
  \textit{Конструктивные} возникают при объективных противоречиях, когда 
цели и~потребности сторон едины (развитие коллектива). 
\textit{Деструктивные}~--- причины их субъективны, вызывают напряженность и~разрушают коллектив.
  
  \textit{Конфликт отношений (социальный конфликт)}~--- разногласие между 
членами группы по личным\linebreak вопросам и~проблемам, не относящимся 
к~вы\-пол\-ня\-емой работе (связан с~несовместимостью и~враждебностью между 
субъектами). 



  \textit{Межличностные} конфликты бывают пяти типов (см.\ рис.~2). При этом 
конфликты внутри устойчивой подгруппы более скоротечны и~чаще\linebreak\vspace*{-12pt}

\pagebreak

\end{multicols}

\begin{figure*} %fig1
 \vspace*{1pt}
    \begin{center}  
  \mbox{%
 \epsfxsize=134.001mm 
 \epsfbox{rum-1.eps}
 }
\end{center}
\vspace*{-9pt}
\Caption{Структура конфликта}
%\end{figure*}
%\begin{figure*} %fig2
\vspace*{24pt}
    \begin{center}  
  \mbox{%
 \epsfxsize=157.144mm 
 \epsfbox{rum-2.eps}
 }
\end{center}
\vspace*{-9pt}
\Caption{Типология конфликта в~малом коллективе}
\end{figure*}

\begin{multicols}{2}


\noindent
 имеют 
иное значение и~последствие для их участников~--- расцениваются как 
конфликт <<между\linebreak своими>>, семейный конфликт.  
\textit{Микрогрупповой}~---\linebreak конфликт между индивидом и~неформальной 
подгруппой. \textit{Меж\-мик\-ро\-груп\-по\-вой}~--- конфликт между неформальными 
подгруппами в~группе. \textit{Груп\-по\-вой}~--- конфликт между индивидом 
и~группой.
  
  \textit{Конфликт интересов} обусловлен мотивационными факторами 
и~ситуацией, в~которой цели каж\-до\-го члена не совпадают с~целями других. 
\textit{Конфликт ресурсов} предполагает отсутствие соглашения между 
субъектами по поводу ресурсов. \textit{Когнитивный конфликт}~--- цели всех 
членов группы совпадают, но их позиции различны (фокус на 
интеллектуальных или оценочных проблемах, связи между внутригрупповым 
конфликтом и~дея\-тель\-ностью группы).

\vspace*{5pt}
  
  \textit{Инструментальный (деловой) конфликт}:\\[-16pt]
\begin{enumerate}[(1)]
\item \textit{конфликт 
задачи}~--- связан с~различием точек зрения на групповые цели и~задачи 
(помогают открытое об\-суж\-де\-ние и~споры);\\[-14pt]
\item \textit{конфликт процесса}~--- 
возникает по поводу проблем, возникающих относительно технологии 
и~способов решения поставленной задачи, распределения между субъектами 
ролей и~от\-вет\-ст\-вен\-ности. 
\end{enumerate}

\begin{figure*}[b] %fig3
\vspace*{1pt}
    \begin{center}  
  \mbox{%
 \epsfxsize=156.127mm 
 \epsfbox{rum-3.eps}
 }
\end{center}
\vspace*{-9pt}
    \Caption{Фазы конфликта}
    \end{figure*}
  
\section{Причины, позитивные функции и~динамика конфликта} 
  
  Есть разные точки зрения~\cite{9-r, 12-r} на классификацию причин 
конфликтов. Приведем причины по К.~Левину~\cite{9-r}: 
\begin{itemize}
\item \textit{степень 
удовлетворенности потребностей}~--- неудовлетворенные по\-треб\-но\-сти час\-то 
становятся до\-ми\-ни\-ру\-ющи\-ми, увеличивая ве\-ро\-ят\-ность конфликтов; 
\item <<\textit{пространство свободного движения}>> (достаточное  
пространство~--- условие удовлетворения индивидуальных по\-треб\-но\-стей 
и~адаптации к~группе, а~огра\-ни\-чен\-ность ведет к~рос\-ту на\-пря\-же\-ния); 
\item \textit{внешний барьер}~--- наличие или отсутствие возможностей выйти из 
неприятной ситуации (отсутствие провоцирует конфликт); 
\item \textit{совпадение 
или расхождение целей членов группы}~--- конфликты зависят от степени 
противоречия целей участников и~от их го\-тов\-ности к~компромиссу.
\end{itemize}
  
  Позитивные функции конфликта~\cite{8-r}: 
  \begin{itemize}
  \item обеспечивает уникальность 
и~автономность каждого из взаимодействующих субъектов, а~также развитие 
отношений между ними; 
\item предоставляет информацию о~воз\-мож\-но\-стях 
противодействующих субъектов; 
\item высвобождает накапливающееся внутреннее 
на\-пря\-же\-ние, сохраняя связи; 
\item актуализирует разные позиции и~мнения по 
поводу возникающих проб\-лем и~тем самым способствует поиску оптимальных 
способов их решения; 
\item усиливает груп\-по\-вую/мик\-ро\-груп\-по\-вую 
идентичность и~сплоченность.
\end{itemize}
  
  Динамика конфликта~\cite{10-r, 11-r} отражается в~двух понятиях: этапы 
конфликта и~фазы конфликта. Основные этапы конфликта: 
\begin{enumerate}[(1)]
\item \textit{возникновение объективной проблемной ситуации} (появление 
противоречия); 
\item \textit{осознание проблемы} хотя бы одним из участников 
(если разрешить проблему неконфликтными методами не получается, то 
возникает предконфликтная ситуация); 
\item \textit{начало открытого 
конфликтного взаимодействия (инцидент)}; 
\item \textit{развитие открытого 
конфликта (эскалация)}~--- открыто заявляются позиции и~выдвигаются 
требования. Завершается этот этап \textit{сбалансированным 
противодействием}~--- когда силовые методы не дают результата, 
интенсивность борьбы снижается, но действия по достижению согласия еще не 
предпринимаются; 
\item \textit{разрешение конфликта}. 
\end{enumerate}
  
  Основные фазы конфликта:
  \begin{enumerate}[(1)]
  \item  начальная фаза;
  \item  фаза подъема; 
  \item пик конфликта; 
\item фаза спада.
\end{enumerate}
Фазы могут повторяться циклически (рис.~3)~\cite{10-r}. При этом 
возможности разрешения конфликта в~каж\-дом последующем цикле сужаются.

 \begin{figure*}[b] %fig4
\vspace*{1pt}
    \begin{center}  
  \mbox{%
 \epsfxsize=162.027mm 
 \epsfbox{rum-4.eps}
 }
\end{center}
\vspace*{-9pt}
   \Caption{Карта анализа конфликта~(\textit{а}) и~метод декартовых координат~(\textit{б})}
   \end{figure*}  

  
\section{Методы моделирования конфликта}

  В открытой печати встречаются подходы к~моделированию межгрупповых 
  и~межгосударственных конфликтов~\cite{11-r, 13-r, 14-r, 15-r}, которые 
позволяют заменить непосредственный анализ конфликтов анализом свойств 
и~характеристик их моделей, а также прогнозировать и~оценивать события 
в~реальном времени. Их опыт можно перенести на моделирование конфликтов 
в~малых коллективах при их классификации в~рамках микрогрупповой 
концепции~\cite{8-r}. В~\cite{5-r} представлен развернутый обзор 
моделирования военных конфликтов и~выделяются:
\begin{enumerate}[(1)]
\item \textit{описательные 
модели военных действий}~--- основываются на методах теории вероятностей 
и~статистической теории решений (принятие\linebreak решений в~условиях 
<<природной>> неопределенности), тео\-рии на\-деж\-ности и~тео\-рии массового 
обслуживания, тео\-рии экспертных\linebreak оценок, а~так\-же качественный анализ 
соответствующих динамических систем и~исследование их структурной 
устойчивости; 
\item \textit{имитационные модели}~--- основываются на аппарате 
марковских цепей, дифференциальных уравнений (ланчестеровские модели), 
конечных автоматов или методах распределенного искусственного интеллекта; 
\item \textit{оптимизационные модели военных действий} используют аппарат 
линейного и~динамического программирования, теории оптимального 
управления, дискретной оптимизации; 
\item \textit{модели принятия решений}~--- 
индивидуального (основной акцент обычно делается на многокритериальном 
принятии решений) и~коллективного (акцент на использовании теории игр). 
\end{enumerate}
  
  \textit{Многоагентное моделирование}~--- методология, 
применяемая для поддержки принятия решений, анализа и~изучения сложных 
сис\-тем, со\-сто\-ящих из отдельных, функционирующих независимо друг от друга 
индивидов~\cite{16-r}. Изучается влияние взаимодействий индивидов на сис\-тем\-ные 
характеристики в~целом.
  
  \textit{Вероятностные распределения}~--- способ описания 
переменных через указание доли элементов совокупности с~данным значением 
переменной~\cite{17-r}.
  
  \textit{Модели целенаправленного поведения}~--- использование 
целевых функций для анализа, прогнозирования и~планирования социальных 
процессов~\cite{18-r}. Модели имеют вид задачи математического программирования.
  
  \textit{Статистические исследования зависимостей}~--- прежде 
всего это регрессионные модели, пред\-став\-ля\-ющие связь зависимых 
и~независимых переменных в~виде функциональных отношений~\cite{19-r}.
  
  \textit{Теоретические модели} предназначены для логического 
анализа тех или иных содержательных концепций, когда затруднена 
возможность измерения основных параметров и~переменных (возможные 
межгосударственные конфликты и~др.)~\cite{17-r}.
  
\section{Методы визуализации конфликтов }

  Эффективность разрешения конфликтов во многом определяется 
правильностью и~полнотой их анализа. Рассмотрим основные методы, 
пред\-ла\-га\-емые зарубежными и~отечественными психологами.
  
  \textit{Картографический анализ конфликтов  
Х.~Кор\-не\-ли\-ус и~Ш.~Фейр}~\cite{20-r}. Проблема обычно записывается в~виде 
противоречия в~центре карты (рис.~4,\,\textit{а}), которая делится по числу 
сторон конфликта. У~каж\-до\-го участника конфликта выясняются потребности, 
интересы и~опасения. Все основные элементы конфликта (проблема, позиции, 
условия, образы, интересы, потребности, опасения, исходы и~др.)\ 
упорядочиваются и~систематизируются. 
  
  \textit{Метод декартовых координат}~\cite{21-r}. Р.~Декарт считал, что 
наше сознание имеет определенную структуру и~его можно пред\-ста\-вить как 
про\-стран\-ст\-во (рис.~4,\,\textit{б}), обозначенное математическими\linebreak\vspace*{-12pt}

\pagebreak

\end{multicols}

\begin{figure*} %fig5
\vspace*{1pt}
    \begin{center}  
  \mbox{%
 \epsfxsize=130.703mm 
 \epsfbox{rum-5.eps}
 }
\end{center}
\vspace*{-9pt}
\Caption{Ресурсный анализ конфликта: конфликтограмма и~балансный лист ресурсов}
%\end{figure*}
%\begin{figure*} %fig6
\vspace*{18pt}
    \begin{center}  
  \mbox{%
 \epsfxsize=162.045mm 
 \epsfbox{rum-6.eps}
 }
\end{center}
\vspace*{-9pt}
\Caption{Модели социального поведения (\textit{а}) и~сетка  
К.~То\-ма\-са\,--\,Л.~Кил\-ме\-на~(\textit{б})}
%\vspace*{3pt}
\end{figure*}

\begin{multicols}{2}

\noindent
 символами 
$(a, b)$. Все это пространство поделено на четыре квад\-ра\-та с~по\-мощью двух 
пе\-ре\-се\-ка\-ющих\-ся прямых.
     В~каж\-дый квадрат надо записать не менее десяти ответов, которые 
со\-по\-став\-ля\-ют\-ся и~анализируются для поиска решения выхода из кон\-фликта. 
{\looseness=1

}
  
  \textit{Ресурсный анализ конфликта В.\,Н.~Ковалева (РАКК)}~\cite{22-r} 
базируется на заполнении всеми субъектами конфликтограммы 
и~балансного листа ресурсов сторон (рис.~5), которые позволяют по\-смот\-реть 
на конфликт глазами оппонента, установить воз\-мож\-ность обмена ресурсами 
и~компромисса.
  


  Межличностные отношения в~конфликте в~рамках РАКК представляются 
моделями социального поведения оппонента (рис.~6): четверть~I~--- альтруизм 
(оказание помощи другим); II~--- эгоизм (приобретение всего лучшего без 
обмена ресурсами); III~--- альтруизм (оказание помощи через избавление от 
трудностей других); IV~--- эгоизм (избавление от всего плохого). 
В~квадрате~V в~соизмеримом отношении встречаются все формы поведения. 
Завершающий этап~--- оценка сторонами того, что они отдали и~что приобрели, 
и~взаимное согласие с~полученными результатами. Для анализа стилей 
поведения личности в~конфликте психологи пользуются сеткой К.~Томаса 
и~Р.~Килмена (рис.~6,\,\textit{б})~--- она дает возможность спрогнозировать 
исходы конфликта, остроту его протекания, возможные ресурсные потери. 
  

  Затем анализируются возможные типы поведения и~реакции на них (рис.~7), 
и~каждой стороной заполняется таблица проектов предложений. 
Стороны приступают к~совместному анализу предложений и~заключают 
соглашение.
   
  \begin{figure*} %fig7
  \vspace*{1pt}
    \begin{center}  
  \mbox{%
 \epsfxsize=111.089mm 
 \epsfbox{rum-7.eps}
 }
\end{center}
\vspace*{-9pt}
  \Caption{Комплексная матрица анализа возможных типов поведения и~проекты предложений}
   \end{figure*}
  
  Также для визуализации конфликта можно использовать дерево целей 
и~дерево проблем~\cite{23-r}. Дерево целей~--- это структурированная, 
построенная по иерархическому принципу совокупность целей 
конфликтующей системы. Если представить предмет конфликта как проблему, 
то можно построить дерево проблем, а затем и~дерево решений. В~\cite{24-r} 
рассматривается диаграмма разрешения конфликтов, в~которой совместно 
используются деревья текущей реальности, будущей реальности, перехода, 
разрешения конфликтов <<Грозовая туча>>. 

\section{Заключение}

  Аналитический обзор по материалам открытой печати показал разнообразие 
подходов к~определению понятия конфликта, его классификации, а~так\-же 
причин и~функций конфликта. При этом для анализа конфликтной ситуации 
с~целью ее разрешения нет единого подхода к~визуализации субъектов 
конфликтной ситуации, их характеристик, взаимодействия и~взаимоотношений. 
Модели конфликтов строятся в~рамках различных методов и~подходов: 
описательные, имитационные, оптимизационные и~модели принятия  
решений~--- однако найдены были только модели военных конфликтов. 
Используя опыт моделирования  
последних~\cite{15-r, 16-r, 17-r, 18-r, 19-r, 20-r, 21-r} с~позиции микрогрупповой 
концепции А.\,В.~Сидоренкова~\cite{10-r}, будет разработана модель 
конфликта в~\mbox{ГиМАС.}
  
  {\small\frenchspacing
 {%\baselineskip=10.8pt
 \addcontentsline{toc}{section}{References}
 \begin{thebibliography}{99}
\bibitem{1-r}
\Au{Колесников А.\,В.} Гетерогенные естественные и~искусственные системы~// 
Интегрированные модели и~мягкие вычисления в~искусственном интеллекте~/
Под ред.\ В.\,Б.~Тарасова.~--- М.: Физматлит, 2013. Т.~1. С.~86--103.

\bibitem{3-r}
\Au{Зиммель Г.} Конфликт современной культуры~/ Пер. с~нем.~--- Птг.: Начатки знаний, 
1923. 40~с. (\Au{Simmel~G.} Der Konflikt der modernen Kultur.~--- Munchen: Duncker \& 
Humblot, 1918. 60~p.)
\bibitem{2-r}
\Au{Колесников А.\,В., Кириков~И.\,А., Листопад~С.\,В.} Гиб\-рид\-ные интеллектуальные 
системы с~самоорганизацией: координация, согласованность, спор.~--- М.: ИПИ РАН, 2014. 
189~с.

\bibitem{4-r}
\Au{Shaw M.} Group dynamics: The psychology of small group behavior.~--- New York, NY, 
USA: McGraw-Hill, 1981. 531~p.
\bibitem{5-r}
\Au{Brown R.} Group processes.~--- 2nd ed.~--- Oxford: Wiley-Blackwell, 1996. 442~p.
\bibitem{6-r}
\Au{Андреева Г.\,М.} Социальная психология.~--- М.: Аспект-пресс, 2009. 393~с.
\bibitem{7-r}
Small groups and social interaction~/ Eds. H.~Blumberg, A.~Hare, V.~Kent, M.~Davies.~--- 
Chichester: Wiley, 1983. Vol.~1. 478~p.
\bibitem{8-r}
\Au{Сидоренков А.\,В.} Конфликт в~малой группе: понятие, функции, виды и~модель~//  
Се\-ве\-ро-Кав\-каз\-ский психологический вестник, 2008. Т.~6. №\,4. С.~22--28.
\bibitem{9-r}
\Au{Гришина Н.\,В.} Психология конфликта.~--- 2-е изд.~--- СПб.: Питер, 2008. 544~с.
\bibitem{10-r}
\Au{Емельянов С.\,М.} Практикум по конфликтологии.~--- 3-е изд.~--- СПб.: Питер, 2009. 
384~с.
\bibitem{11-r}
\Au{Анцупов А.\,Я., Баклановский~С.\,В.} Конфликтология в~схемах и~комментариях.~--- 
2-е изд.~--- СПб.: Питер, 2009. 304~с.
\bibitem{12-r}
\Au{Робер М.-А., Тильман~Ф.} Психология индивида и~группы~/ Пер. с~фр.~--- М.: 
Прогресс, 1988. 256~с. (\Au{Ro\-bert~M.\,A., Tilman~F.} Psycho. 
Conna$\hat{\mbox{\!\normalsize\!\ptb{\i}}}$tre l'individu et le groupe aujourd'hui.~--- Bruxelles: Vie ouvriere, 
1980. 307~p.)

\bibitem{14-r} %13
\Au{Epstein J.\,M., Steinbruner~J.\,D., Parker~M.\,T.} Modeling civil violence: An agent-based 
computational approach~// P.~Natl. Acad. Sci. USA, 2002. Vol.~99. Suppl.~3. P.~7243--7250. doi: 
10.1073/pnas.092080199.
\bibitem{15-r} %14
\Au{Новиков Д.\,А.} Иерархические модели военных действий~// Управление большими 
системами, 2012. №\,37. С.~25--62.
\bibitem{13-r} %15
\Au{Клаус Н.\,Г., Свечкарев~В.\,П.} Агентные модели локальных этнических конфликтов (на 
примере осе\-ти\-но-ин\-гуш\-ско\-го конфликта в~селе Тарское)~// Инженерный вестник 
Дона, 2013. №\,4.  С.~138.
{\sf http://www. ivdon.ru/magazine/archive/n4y2013/1983}.

\bibitem{16-r}
\Au{Клаус Н.\,Г., Свечкарев~В.\,П.} Многоагентное моделирование конфликтных  
ситуаций.~--- Ростов на Дону: СКНЦВШ ЮФУ, 2013. 148~с.
\bibitem{17-r}
\Au{Саати Т.\,Л.} Математические модели конфликтных ситуаций~/ Пер. с~англ. под ред. 
И.\,А.~Ушакова.~--- М.: Сов. радио, 1977. 302~с. (\Au{Saaty~T.} Mathematical models of arms 
control and disarmament: Application of mathematical structures in politics.~--- John Wiley\&Sons, 
1968. 190~p.)
\bibitem{18-r}
\Au{Покорная О.\,Ю., Покорная~И.\,Ю., Прядкин~Д.\,В.} Математическое моделирование 
оптимальных стратегий в~условиях конфликта~// Молодой ученый, 2011.  \mbox{№\,4-1}.  
С.~16--19. {\sf https://moluch.ru/archive/27/2896.}
\bibitem{19-r}
\Au{Tibshirani R.} Regression shrinkage and selection via the Lasso~// J.~Roy. 
Stat. Soc.~B Met., 1996. Vol.~58. No.\,1. P.~267--288.
\bibitem{20-r}
\Au{Корнелиус Х., Фейр~Ш.} Выиграть может каждый. Как разрешать конфликты~/ Пер. 
с~англ.~--- Киев: Наукова думка, 2006. 344~с. {\sf  
http://www.conflict-resolve.org/conflictresolve.pdf}.
(\Au{Cornelius~H., Faire~S.} Everyone can win: 
How to resolve conflict.~--- West Roseville, New South Wales, Australia: 
Simon \& Schuster, 1989. 192~p.)
\bibitem{21-r}
\Au{Мириманова М.\,С.} Конфликтология.~--- 
2-е изд.~--- М.: Академия, 2004. 320~с.
\bibitem{22-r}
\Au{Ковалев В.\,Н., Лопатина~Н.\,Н., Лебеденко~Д.\,Н.} Сборник ситуационных задач по 
конфликтологии.~--- Севастополь: СФ МГУ, 2013. 91~с.
\bibitem{23-r}
\Au{Готин С.\,В., Калоша~Л.\,П.} Ло\-ги\-ко-струк\-тур\-ный подход и~его применение для 
анализа и~планирования деятельности.~--- М.: Вариант, 2007. 118~с. 
\bibitem{24-r}
\Au{Детмер У.} Теория ограничений Голдратта: Системный подход к~непрерывному 
совершенствованию~/ Пер. с~англ.~--- 4-е изд.~--- М.: Альпина Паблишер, 2014. 443~с. 
{\sf https://econ.wikireading.ru/64894}. (\Au{Dettmer~W.} Goldratt's theory of constraints: 
A~systems approach to continuous improvement.~--- ASQ Quality Press, 2006. 377~p.) 

 \end{thebibliography}

 }
 }

\end{multicols}

\vspace*{-6pt}

\hfill{\small\textit{Поступила в~редакцию 11.06.19}}

\vspace*{9pt}

%\pagebreak

%\newpage

%\vspace*{-28pt}

\hrule

\vspace*{2pt}

\hrule

%\vspace*{-2pt}

\def\tit{METHODS OF MODELING AND~VISUAL REPRESENTATION OF~A~CONFLICT 
IN~A~SMALL COLLECTIVE OF~EXPERTS SOLVING~PROBLEMS (REVIEW)}


\def\titkol{Methods of modeling and~visual representation of~a~conflict 
in~a~small collective of~experts solving problems (review)}

\def\aut{S.\,B.~Rumovskaya and~I.\,A.~Kirikov}

\def\autkol{S.\,B.~Rumovskaya and~I.\,A.~Kirikov}

\titel{\tit}{\aut}{\autkol}{\titkol}

\vspace*{-11pt}


 \noindent
   Kaliningrad Branch of the Federal Research Center ``Computer Science and 
Control'' of the Russian Academy of Sciences, 5~Gostinaya Str., Kaliningrad 
236022, Russian Federation


\def\leftfootline{\small{\textbf{\thepage}
\hfill INFORMATIKA I EE PRIMENENIYA~--- INFORMATICS AND
APPLICATIONS\ \ \ 2019\ \ \ volume~13\ \ \ issue\ 3}
}%
 \def\rightfootline{\small{INFORMATIKA I EE PRIMENENIYA~---
INFORMATICS AND APPLICATIONS\ \ \ 2019\ \ \ volume~13\ \ \ issue\ 3
\hfill \textbf{\thepage}}}

\vspace*{6pt} 
   
     
  
   \Abste{Small collectives of experts as natural collective decision support intellect (heterogeneous 
collective) solve problems effectively. In addition, the form of interaction between experts as conflict 
generates positive changes in collective such as development of the group, diagnostics of relations, tension 
reduction, and consolidation of the group and inspire saving the collective. Preset sketches of standard 
situations play a~huge role in human reasoning. The use of them highly promote reasoning. Visualization of 
conflict situation makes appeared contradictions contrast and observable, giving a new information of 
resolving the conflict. This makes them easy-to-handle and gives the opportunity to control the impaction on 
the conflict of subjective preference. The notion and particularities of the conflict in 
small collectives, its structure, dynamics, and approaches to modelling and visual presentation of the conflict 
aspect in group dynamics of experts solving problems are reviewed.}
   
   \KWE{small collective of experts; conflict; model of a conflict; visualization of a conflict}
  
\DOI{10.14357/19922264190317} 

%\vspace*{-14pt}

%\Ack
%   \noindent



%\vspace*{-6pt}

  \begin{multicols}{2}

\renewcommand{\bibname}{\protect\rmfamily References}
%\renewcommand{\bibname}{\large\protect\rm References}

{\small\frenchspacing
 {%\baselineskip=10.8pt
 \addcontentsline{toc}{section}{References}
 \begin{thebibliography}{99}
\bibitem{1-r-1}
\Aue{Kolesnikov, A.\,V.} 2013. Geterogennye estestvennye i~is\-kus\-stvennye sistemy 
[Natural and artificial heterogeneous systems]. \textit{Integrirovannye modeli
i~myagkie vychisleniya v~iskusstvennom intellekte}
[Integrated models and 
soft computing in artificial intelligence]. 
Moscow: Fizmatlit. 1:86--103. 
\bibitem{3-r-1} %2
\Aue{Simmel, G.} 1918. Der Konflikt der modernen Kultur. Munchen: Duncker \& 
Humblot. 60~p.
\bibitem{2-r-1} %3
\Aue{Kolesnikov, A.\,V., I.\,A.~Kirikov, and S.\,V.~Listopad.} 2014. 
\textit{Gibridnye intellektual'nye sistemy s~samoorganizatsiey: koordinatsiya, 
soglasovannost', spor} [Hybrid artificial systems with self-organization: 
Coordination, conformance, row]. Мoscow: IPI RAN. 189~p. 

\bibitem{4-r-1}
\Aue{Shaw, M.} 1981. \textit{Group dynamics: The psychology of small group 
behavior}. New York, NY: McGraw-Hill. 531~p.
\bibitem{5-r-1}
\Aue{Brown, R.} 1996. \textit{Group processes}. 2nd ed. Oxford: Wiley-Blackwell. 
442~p.
\bibitem{6-r-1}
\Aue{Andreeva, G.\,M.} 2009. \textit{Sotsial'naya psikhologiya} [Social psychology]. 
Moscow: Aspect-press. 393~p.
\bibitem{7-r-1}
Blumberg, H., A.~Hare, V.~Kent, and M.~Davies, eds. 1983. \textit{Small groups 
and social interaction}. Chichester:\linebreak Wiley. Vol.~1. 478~p.
\bibitem{8-r-1}
\Aue{Sidorenkov, A.\,V.} 2008. Konflikt v maloy gruppe: po\-nya\-tie, funktsii, vidy 
i~model' [Conflict in a~small group: Concept, functions, forms and model].  
\textit{Severo-Kavkazskiy psikhologicheskiy vestnik} [North Caucasian 
Psychological Annals] 6(4):22--28.
\bibitem{9-r-1}
\Aue{Grishina, N.\,V.} 2008. \textit{Psikhologiya konflikta} [Psychology of conflict]. 
SPb.: Piter. 544~p.
\bibitem{10-r-1}
\Aue{Emel'yanov, S.\,M.} 2009. \textit{Praktikum po konfliktologii} [Tutorial at 
conflictology]. SPb.: Piter. 384~p.
\bibitem{11-r-1}
\Aue{Antsupov, A.\,Ya., and S.\,V.~Baklanovskiy.} 2009. \textit{Konfliktologiya 
v~skhemakh i~kommentariyakh} [Conflictology in schemes and 
comments]. SPb.: Piter. 304~p.
\bibitem{12-r-1}
\Aue{Robert, M.\,A., and F.~Tilman.} 1980. \textit{Psycho. 
Conna$\hat{\mbox{\normalsize\!\!\ptb{\i}}}$tre l'individu et le groupe aujourd'hui}. Bruxelles: Vie 
ouvriere. 307~p.

\bibitem{14-r-1} %13
\Aue{Epstein, J.\,M., J.\,D.~Steinbruner, and M.\,T.~Parker.} 2002. Modeling civil 
violence: An agent-based computational approach. \textit{P.~Natl. Acad. Sci. USA} 
99(Suppl.~3):7243--7250. 
doi: 10.1073/pnas.092080199.
\bibitem{15-r-1} %14
\Aue{Novikov, D.\,A.} 2012. Ierarkhicheskie modeli voennykh deystviy [Hierarchical 
models of combat]. \textit{Upravlenie bol'shimi sistemami} [Control of Large 
Systems] 37:25--62.
\bibitem{13-r-1} %15
\Aue{Klaus, N.\,G., and V.\,P.~Svechkarev.} 2013. Agentnye modeli lokal'nykh 
ehtnicheskikh konfliktov (na primere ose\-ti\-no-in\-gush\-sko\-go konflikta v~sele Tarskoe) 
[Agent based modeling of the social conflict: Ossetino-Ingushskiy conflict 
in the area of village Tarskoe]. \textit{Inzhenernyy vestnik Dona} [Engineering Annals of Don] 
4:138. Available at: {\sf http://www.ivdon.ru/magazine/archive/n4y2013/ 1983/}
 (accessed 
May~27, 2019).

\bibitem{16-r-1}
\Aue{Klaus, N.\,G., and V.\,P.~Svechkarev.} 2013. \textit{Mnogoagentnoe 
modelirovanie konfliktnykh situatsiy} [Multiagent modeling of conflict situations]. 
Rostov-on-Don: NCSC HS SFEDU. 148~p.

%\columnbreak

\bibitem{17-r-1}
\Aue{Saaty, T.} 1968. \textit{Mathematical models of arms control and disarmament: 
Application of mathematical structures in politics}. John Wiley\&Sons. 190~p.
\bibitem{18-r-1}
\Aue{Pokornaya, O.\,Yu., I.\,Yu.~Pokornaya, and D.\,V.~Pryadkin.} 2011. 
Matematicheskoe modelirovanie op\-ti\-mal'\-nykh strategiy v~usloviyakh konflikta 
[Mathematical modeling of optimal strategies in conflict]. \textit{Molodoy uchenyy} 
[Young Scientist] 4-1:16--19. Available at: {\sf  
 https:// moluch.ru/archive/27/2896/} (accessed May~27, 2019).
\bibitem{19-r-1}
\Aue{Tibshirani, R.} 1996. Regression shrinkage and selection via the Lasso. 
\textit{J.~Roy. Stat. Soc.~B Met.} 58(1):267--288.
\bibitem{20-r-1}
\Aue{Cornelius, H., and S.~Faire.} 1989. \textit{Everyone can win:
How to resolve conflict}. West Roseville, New South Wales, Australia: 
Simon \& Schuster. 192~p.
\bibitem{21-r-1}
\Aue{Mirimanova, M.\,S.} 2004. \textit{Konfliktologiya} 
[Conflictology]. Moscow: Academy. 320~p.
\bibitem{22-r-1}
\Aue{Kovalev, V.\,N., N.\,N.~Lopatina, and D.\,N.~Lebedenko.} 2013. \textit{Sbornik 
situatsionnykh zadach po konfliktologii} [The collection of situational tasks on 
conflictology]. Se\-va\-sto\-pol:  SB MSU. 91~p.
\bibitem{23-r-1}
\Aue{Gotin, S.\,V., and L.\,P.~Kalosha.} 2007. \textit{Logiko-strukturnyy podkhod 
i~ego primenenie dlya analiza i~planirovaniya deya\-tel'\-nosti} [Logical-structural 
approach and its application for the analysis and planning of activities]. Moscow: 
Variant. 118~p. 
\bibitem{24-r-1}
\Aue{Dettmer, W.} \textit{Goldratt's theory of constraints: A~systems approach to 
continuous improvement.} ASQ Quality Press 2006. 377~p. Available at: {\sf 
https://econ.wikireading.ru/64894} (accessed May~27, 2019).
 \end{thebibliography}

 }
 }

\end{multicols}

%\vspace*{-7pt}

\hfill{\small\textit{Received June 11, 2019}}

%\pagebreak

%\vspace*{-22pt}
 
   
   
   
   \Contr
   
\noindent
\textbf{Rumovskaya Sophiya B.} (b.\ 1985)~--- Candidate of Sciences (PhD) in 
technology, scientist, Kaliningrad Branch of the Federal Research Center 
``Computer Science and Control'' of the Russian Academy of Sciences, 5~Gostinaya 
Str., Kaliningrad 236022, Russian Federation; \mbox{sophiyabr@gmail.com}

\vspace*{3pt}

\noindent
\textbf{Kirikov Igor A.} (b.\ 1955)~--- Candidate of Sciences (PhD) in 
technology; director, Kaliningrad Branch of the Federal Research Center 
``Computer Science and Control'' of the Russian Academy of Sciences, 5~Gostinaya 
Str., Kaliningrad 236022, Russian Federation; \mbox{baltbipiran@mail.ru} 
\label{end\stat}

\renewcommand{\bibname}{\protect\rm Литература}
      %17
\def\stat{burl+yak}

\def\tit{РАЗРАБОТКА МЕТОДА ФОРМИРОВАНИЯ ПРИЗНАКОВОГО ПРОСТРАНСТВА 
И~МОДЕЛИ ДЛЯ~ОЦЕНКИ И~ПРОГНОЗИРОВАНИЯ АНТРОПОГЕННОГО ВЛИЯНИЯ 
НА~ОКРУЖАЮЩУЮ СРЕДУ (НА~ПРИМЕРЕ ЛЕСНОГО ФОНДА 
НЕФТЕДОБЫВАЮЩЕГО РЕГИОНА)$^*$}

\def\titkol{Разработка метода формирования признакового пространства 
и~модели для~оценки и~прогнозирования}
% антропогенного влияния 
%на~окружающую среду (на~примере лесного фонда 
%нефтедобывающего региона)}

\def\aut{В.\,В.~Бурлуцкий$^1$, А.\,В.~Якимчук$^2$, А.\,В.~Мельников$^3$, 
  А.\,Л.~Царегородцев$^4$,\\  С.\,В.~Волошин$^5$}

\def\autkol{В.\,В.~Бурлуцкий, А.\,В.~Якимчук, А.\,В.~Мельников и~др.} 
%  А.\,Л.~Царегородцев$^4$,  С.\,В.~Волошин$^5$}

\titel{\tit}{\aut}{\autkol}{\titkol}

\index{Бурлуцкий В.\,В.}
\index{Якимчук А.\,В.}
\index{Мельников А.\,В.} 
\index{Царегородцев А.\,Л.}
\index{Волошин С.\,В.}
\index{Burlutskiy V.\,V.}
\index{Yakimchuk A.\,V.}
\index{Melnikov A.\,V.} 
\index{Tsaregorodtsev A.\,L.}
\index{Voloshin S.\,V.}


{\renewcommand{\thefootnote}{\fnsymbol{footnote}} \footnotetext[1]
{Работа выполнена при поддержке Научного фонда ЮГУ (проект 13-01-20/25).}}


\renewcommand{\thefootnote}{\arabic{footnote}}
\footnotetext[1]{Югорский НИИ информационных технологий, г.~Хан\-ты-Ман\-сийск, BurlutskyVV@uriit.ru}
\footnotetext[2]{Югорский государственный университет, г.~Хан\-ты-Ман\-сийск, YakimchukAV@uriit.ru}
\footnotetext[3]{Югорский НИИ информационных технологий, г.~Хан\-ты-Мансийск, andmelnikov1956@yandex.ru}
\footnotetext[4]{Югорский НИИ информационных технологий, г.~Хан\-ты-Ман\-сийск, TsaregorodtsevAL@uriit.ru}
\footnotetext[5]{Югорский государственный университет, г.~Хан\-ты-Ман\-сийск, Voloshinsv@uriit.ru}

\vspace*{-18pt}

    
        
  
  
  \Abst{Работа посвящена разработке системного метода оценки и~прогнозирования 
влияния природных и~антропогенных воздействий на окружающую среду, включающего 
процедуры преобразования исходных информационных массивов, формирования 
нейросетевой модели, ее обучения и~тестирования. Метод применен для анализа последствий 
антропогенных воздействий на окружающую среду в~Хан\-ты-Ман\-сий\-ском автономном 
округе~--- Югре.}
  
  \KW{анализ данных; машинное обучение; нейронные сети; пространственный анализ; 
географические информационные системы; риск-ори\-ен\-ти\-ро\-ван\-ный подход}

\DOI{10.14357/19922264190318} 
  
\vspace*{-3pt}


\vskip 10pt plus 9pt minus 6pt

\thispagestyle{headings}

\begin{multicols}{2}

\label{st\stat}

  
\section{Введение}

\vspace*{-6pt}

  Проблема формирования признакового пространства, математической 
модели и~подходящих алгоритмов машинного обучения для оценки 
и~прогнозирования влияния природных и~антропогенных факторов на 
окружающую среду весьма актуальна как с~теоретической, так и~с~прикладной 
точки зрения. В~особенности решение этой проб\-ле\-мы важно для 
нефтедобывающих регионов, так как производственные технологии в~них 
сопряжены с~загрязнением окружающей среды. 
  
  Первая часть исходной проблемы~--- фор\-ми\-рование признакового 
пространства~--- связана\linebreak с~доступными количественными показателями 
техногенных аварий~[1--3]. Здесь важной является трансформация исходных 
информационных массивов в~цифровые характеристики риска влияния 
антропогенных факторов.
  
  Вторая и~третья части исходной проблемы~--- модель и~алгоритм  
обучения~--- связаны друг с~другом и~сводятся к~задачам классификации, 
в~которых\linebreak используется какая-либо нейросетевая модель\linebreak в~сочетании 
с~подходящим алгоритмом обучения. В~настоящее время разработано большое 
число различных алгоритмов классификации: метод~$k$~ближайших 
соседей~[4], случайный лес~[5], стохастический градиентный спуск~[6], метод 
опорных векторов~[7]. 
  
  В данной работе для решения поставленной проблемы предлагается метод 
трансформации исходных разнородных динамически организованных 
информационных массивов, полученных путем дис\-тан\-ци\-он\-но\-го зондирования 
земной поверхности с~использованием средств 
географических информационных сис\-тем (ГИС).
  
  На основе преобразованных информационных массивов формируется 
трехслойная нейронная сеть и~проводится ее обучение и~тестирование. 
Обученная сеть используется для прогнозирования последствий природных 
и~антропогенных воздействий на окружающую среду. 
  
  Разработанные методы были применены для оценивания и~прогнозирования 
антропогенных рисков на территории Хан\-ты-Ман\-сий\-ско\-го автономного 
округа.

\begin{figure*}[b] %fig1
%\vspace*{8pt}
    \begin{center}  
  \mbox{%
 \epsfxsize=165.986mm 
 \epsfbox{bur-1.eps}
 }
\end{center}
\vspace*{-9pt}
\Caption{Распределение аварий по районам}
\end{figure*}
  

\vspace*{-12pt}
  
\section{Структуризация и~предварительная обработка разнородных 
данных}

\vspace*{-4pt}

  Все единицы информационных массивов привязаны к~ячейкам 
поверхностной сетки. Для каждого признака набора данных была определена 
шкала измерения с~учетом метода обработки: текстовые признаки были 
заменены на числовые, непрерывные значения были нормализованы по 
максимальному значению.
  
  Ключевым атрибутом набора данных служат поверхностные координаты 
соответствующего информационного объекта.
  
  Для проверки корректности заполнения базы использовались методы 
визуального анализа: гистограммы и~картографический анализ.
  
  Для визуального анализа использовался boxplot, компактно изображающий 
распределение. Такой вид диаграммы в~удобной форме показывает медиану 
(или, если нужно, среднее), нижний и~верхний квартили, минимальное 
и~максимальное значение выборки и~выбросы.

\vspace*{-6pt}
  
\section{Модель оценки рисков}

\vspace*{-4pt}

  В результате проведенного анализа была по\-строена 4-слой\-ная нейросетевая 
модель оценки рисков~[8]. В~качестве данных для входного слоя 
использовалась векторизация признаков, что позволило существенно сократить 
число входных нейронов.
  
  Число нейронов во втором слое составило~166, в~третьем~--- 83, в~четвертом~--- 40, 
и~выходной слой был с~двумя нейронами, которые фиксировали 
положительную и~отрицательную реакцию модели.
  
  Информационные массивы были разделены на три части, которые 
использовались для обучения модели, тестирования и~прогнозирования.
  
\section{Применение разработанного метода для оценки 
и~прогнозирования рисков антропогенных факторов 
в~нефтедобывающем регионе}

\vspace*{-14pt}

  В качестве подходящего полигона был избран Хан\-ты-Ман\-cий\-ский 
автономный округ~--- Югра, который лидирует среди нефтедобывающих 
регионов по добыче нефти и~газа. При этом более~95\% территории 
автономного округа представляют собой земли лесного фонда, которые 
находятся под постоянным негативным воздействием предприятий 
нефтегазового комплекса. 

\begin{figure*} %fig2
\vspace*{1pt}
    \begin{center}  
  \mbox{%
 \epsfxsize=163mm 
 \epsfbox{bur-2.eps}
 }
\end{center}
\vspace*{-9pt}
\Caption{Визуализация исходного набора данных}
%\vspace*{-6pt}
\end{figure*}




  
  Для построения модели прогнозирования рисков загрязнения 
использовались данные об авариях и~инцидентах, транспортной 
инфраструктуре, населенных пунктах~[9]. Вся информация привязывалась  
к~5-ки\-ло\-мет\-ро\-вой поверхностной сетке, и~процесс интеграции данных 
заключался в~унификации географических координат различных объектов, 
построении общих классификаторов, удалении дублирующей информации 
и~приведении к~единому виду текстовых признаков.
  
  В~результате был получен первичный массив данных о загрязнениях 
окружающей среды за период~2012--2017~гг. Объем данной информативной 
выборки составил 26\,015~записей. Общее число элементарных участков 
составило~22\,054. 
  
  Рисунок~1 изображает распределение аварий по районам, а также показывает 
медиану (или, если нужно, среднее), нижний и~верхний квартили, минимальное и~максимальное значение выборки и~выбросы.


  Объекты, привязанные к~элементарным участкам, представлены на рис.~2. 
  

  
  На следующем этапе набор данных был проанализирован на наличие 
неинформативных признаков, для этого использовался метод главных 
компонент.
  
  В результате проведенного анализа и~предобработки данных был получен 
базовый набор данных, объем которого составил~22\,054.
  
  Для тестирования модели использовался метод кросс-ва\-ли\-да\-ции 
по~$k$~блокам~[10]. Исходный набор данных разбивался на~10~одинаковых 
по размеру блоков. Из~10~блоков один оставлялся для тестирования модели, 
а~оставшиеся~9~блоков использовались как тренировочный набор. Процесс 
повторялся~10~раз, и~каждый из блоков один раз использовался как тестовый 
набор. В~итоге получилось~10~результатов, по одному на каждый блок, они 
усреднялись или комбинировались ка\-ким-либо другим способом и~дали одну 
оценку. Результаты представлены в~таблице.
  
  
  
  Средняя оценка составила~90,68.
  
  Преимущество такого способа перед случайным сэмплированием в~том, что 
все наблюдения используются и~для тренировки, и~для тестирования\linebreak\vspace*{-12pt}
  
  %\begin{table*}
{\small
  \begin{center}
  \begin{tabular}{|c|c|}
  \multicolumn{2}{p{35mm}}{Оценка методом кросс-валидации}\\
  \multicolumn{2}{c}{\ }\\[-6pt]
  \hline
\tabcolsep=0pt\begin{tabular}{c}Номер\\ модели\end{tabular}&
\tabcolsep=0pt\begin{tabular}{c}Полученная\\ оценка\end{tabular}\\
\hline
1&91,5\\
2& 90,3\\
3&90,8\\
4&91,2\\
5&89,9\\
6&90,6\\
7&91,4\\
8&89,5\\
9&91,2\\
10\hphantom{9}&90,4\\
\hline
\end{tabular}
\end{center}
}
%\end{table*}

\vspace*{12pt}
  
  \noindent
     модели, 
при этом каждое наблюдение используется для тестирования только один раз.
  
  Получившаяся карта с~прогнозированными авариями представлена на рис.~3.
  
  
\begin{figure*} %fig3
  \vspace*{1pt}
    \begin{center}  
  \mbox{%
 \epsfxsize=163mm 
 \epsfbox{bur-3.eps}
 }
\end{center}
\vspace*{-9pt}
  \Caption{Визуализация прогноза аварий}
  \end{figure*} 
  
\vspace*{-6pt} 

\section{Заключение}

\vspace*{-4pt}

  Разработан системный метод оценки и~прогнозирования природных 
и~антропогенных воздействий на окружающую среду, включающий процедуры 
преобразования исходных информационных массивов, формирования 
нейросетевой прогнозирующей модели, ее обучения и~тестирования. Метод 
применен для прогнозирования техногенных аварий в~нефтедобывающем 
регионе~--- Югре. 
  
 {\small\frenchspacing
 {%\baselineskip=10.8pt
 \addcontentsline{toc}{section}{References}
 \begin{thebibliography}{99}

\bibitem{2-bur} %1
\Au{Guikema S.\,D.} Natural disaster risk analysis for critical infrastructure systems: 
An approach based on statistical learning theory~// Reliab. Eng. 
Syst. Safe., 2009. Vol.~94. Iss.~4. P.~855--860. doi: 10.1016/j.ress.2008.09.003.
\bibitem{3-bur} %2
\Au{Шокин Ю.\,И., Москвичев~В.\,В., Ничепорчук~В.\,В.} Методика оценки 
антропогенных рисков территорий и~построения картограмм рисков 
с~использованием геоинформационных сис\-тем~// Вычислительные 
технологии, 2010. Т.~15. №\,1. С.~120--131. 
\bibitem{1-bur} %3
\Au{Гуменюк В.\,И., Кармишин~А.\,М., Киреев~В.\,А.} О~количественных 
показателях опасности техногенных аварий~// На\-уч\-но-тех\-ни\-че\-ские 
ведомости СПбГПУ, 2013. №\,2(171). С.~281--288.
\bibitem{4-bur}
\Au{Колесенков А.\,Н., Костров~Б.\,В., Ручкин~В.\,Н.} Нейронные сети 
мониторинга чрезвычайных ситуаций по данным ДЗЗ~// Известия Тульского 
государственного университета. Технические науки, 2014. №\,5. С.~220--225.
\bibitem{5-bur}
\Au{Stone C.\,J.} Consistent nonparametric regression~// Ann. Stat., 1977. 
Vol.~5. Iss.~4. P.~595--620.
\bibitem{6-bur}
\Au{Breiman L.} Random forests~// Mach. Learn., 2001. Vol.~45. Iss.~1. 
P.~5--32.
\bibitem{7-bur}
\Au{Robbins H., Siegmund~D.} A~convergence theorem for non negative almost 
supermartingales and some applications~// Optimizing methods in statistics~/
Ed. J.\,S.~Rustagi.~--- New 
York, NY, USA: Academic Press, 1971. P.~233--257.
\bibitem{8-bur}
\Au{Cortes C., Vapnik~V.} Support-vector networks~// Mach. Learn., 1995. 
Vol.~20. Iss.~3. P.~273--297.
\bibitem{9-bur}
\Au{Hong H., Pradhan~B., Jebur~M.\,N.} Spatial prediction of landslide hazard at 
the Luxi area (China) using support vector machines~// Environ. Earth 
Sci., 2016. Vol.~75. Iss.~1. P.~40.
\bibitem{10-bur}
\Au{Kohavi R.} A~study of cross-validation and bootstrap for accuracy estimation 
and model selection~// 14th Joint Conference (International) on Artificial Intelligence 
Proceedings.~--- Montr$\acute{\mbox{e}}$al, 
Qu$\acute{\mbox{e}}$bec, Canada, 1995. Vol.~2. P.~1137--1143.
 \end{thebibliography}

 }
 }

\end{multicols}

\vspace*{-6pt}

\hfill{\small\textit{Поступила в~редакцию 02.10.18}}

%\vspace*{8pt}

%\pagebreak

\newpage

\vspace*{-28pt}

%\hrule

%\vspace*{2pt}

%\hrule

%\vspace*{-2pt}

\def\tit{DEVELOPMENT OF~A~METHOD FOR~THE~FORMATION OF~ATTRIBUTE SPACE 
AND~A~MODEL FOR~THE~ASSESSMENT AND~PREDICTION OF~ANTHROPOGENIC INFLUENCE 
ON~THE~ENVIRONMENT (ON~THE~EXAMPLE OF~THE~FOREST FUND OF~THE~OIL-PRODUCING 
REGION)}


\def\titkol{Development of~a~method for~the~formation of attribute space 
and~a~model for~the~assessment and~prediction of %~anthropogenic 
influence} 
%on~the~environment (on~the~example of~the~forest Fund of~the~oil-producing  region)}

\def\aut{V.\,V.~Burlutskiy$^1$, A.\,V.~Yakimchuk$^2$, A.\,V.~Melnikov$^1$, 
A.\,L.~Tsaregorodtsev$^1$, and~S.\,V.~Voloshin$^2$}

\def\autkol{V.\,V.~Burlutskiy, A.\,V.~Yakimchuk, A.\,V.~Melnikov, et al.} 
%A.\,L.~Tsaregorodtsev$^1$, and~S.\,V.~Voloshin$^2$}

\titel{\tit}{\aut}{\autkol}{\titkol}

\vspace*{-11pt}


\noindent
  $^1$Yugra Research Institute of Information Technologies, 151~Mira Str., 
Khanty-Mansiysk 628011, Russian\linebreak
$\hphantom{^1}$Federation 
  
  \noindent
  $^2$Yugra State University, 16~Chekhova Str., Khanty-Mansiysk 628012, 
Russian Federation

\def\leftfootline{\small{\textbf{\thepage}
\hfill INFORMATIKA I EE PRIMENENIYA~--- INFORMATICS AND
APPLICATIONS\ \ \ 2019\ \ \ volume~13\ \ \ issue\ 3}
}%
 \def\rightfootline{\small{INFORMATIKA I EE PRIMENENIYA~---
INFORMATICS AND APPLICATIONS\ \ \ 2019\ \ \ volume~13\ \ \ issue\ 3
\hfill \textbf{\thepage}}}

\vspace*{3pt}  
  
  
   
  
\Abste{The work is devoted to the development of a systematic method for assessing 
and predicting the influence of natural and anthropogenic impacts on the 
environment, including the procedures for the transformation of initial data store, the 
formation of the neural network model, its training, and testing. The method is used to 
analyze the consequences of anthropogenic impacts on the environment in the 
Khanty-Mansiysk Autonomous Okrug~--- Yugra.}
  
  \KWE{data analysis; machine learning; neural networks; spatial analysis; 
geographic information systems; risk-based approach; control and supervision}
  
  
 
  
  
\DOI{10.14357/19922264190318} 

%\vspace*{-14pt}

 \Ack
  \noindent
  This work was supported by the Science  Foundation of Yugra State 
University  under grant No.\,13-01-20/25.


%\vspace*{-6pt}

  \begin{multicols}{2}

\renewcommand{\bibname}{\protect\rmfamily References}
%\renewcommand{\bibname}{\large\protect\rm References}

{\small\frenchspacing
 {%\baselineskip=10.8pt
 \addcontentsline{toc}{section}{References}
 \begin{thebibliography}{99}
  

\bibitem{2-bur-1} %1
\Aue{Guikema, S.\,D.} 2009. Natural disaster risk analysis for critical infrastructure 
systems: An approach based on statistical learning theory. \textit{Reliab.
Eng. Syst. Safe.} 94(4):855--860. doi: 10.1016/j.ress.2008.09.003.
\bibitem{3-bur-1} %2
\Aue{Shokin, Y.\,I., V.\,V.~Moskvichev, and V.\,V.~Nicheporchuk.} 2010.Metodika otsenki 
antropogennykh riskov territoriy i~postroeniya kartogramm riskov s~ispol'zovaniem 
geoinformatsionnykh system [Technique for estimation of anthropogenous risks for 
territories and construction of risks cartograms using geoinformation systems]. 
\textit{Computational Technologies} 15(1):120--131.

\bibitem{1-bur-1} %3
\Aue{Gumenyuk, V.\,I., A.\,M.~Karmishin, and V.\,A.~Kireev.} 2013. 
O~kolichestvennykh pokazatelyakh opasnosti tekhnogennykh avariy [About 
quantitative indicators of danger man-made accidents]. 
\textit{Nauchno-tekhnicheskie vedomosti SPbPU} [St.\ Petersburg State
Polytechnic University 
J.~Engineering Science Technology] 2(171):281--288. 

\bibitem{4-bur-1}
\Aue{Kolesenkov, A.\,N., B.\,V.~Kostrov, and V.\,N.~Ruchkin.} 2014. Neyronnye 
seti monitoringa chrezvychaynykh situatsiy po dannym DZZ  [Neural network 
monitoring for emergencies according ERS]. \textit{Izvestiya 
Tul'skogo gosudarstvennogo universiteta. Tekhnicheskie nauki} [Transactions of 
Tula State University. Technical Sciences] 5:220--225.
\bibitem{5-bur-1}
\Aue{Stone, C.\,J.} 1977. Consistent nonparametric regression. \textit{Ann.
Stat.} 5(4):595--620.
\bibitem{6-bur-1}
\Aue{Breiman, L.} 2001. Random forests. \textit{Mach. Learn.} 45(1):5--32.
\bibitem{7-bur-1}
\Aue{Robbins, H., and D.~Siegmund.} 1971. A~convergence theorem for non 
negative almost supermartingales and some applications. \textit{Optimizing methods 
statistics}. Ed.\ J.\,S.~Rustagi.
New York, NY: Academic Press. 233--257.
\bibitem{8-bur-1}
\Aue{Cortes, C., and V.~Vapnik.} 1995. Support-vector networks. \textit{Mach. 
Learn.} 20(3):273--297.
\bibitem{9-bur-1}
\Aue{Hong, H., B.~Pradhan, and M.\,N.~Jebur.} 2016. Spatial prediction of 
landslide hazard at the Luxi area (China) using support vector machines. 
\textit{Environ. Earth Sci.} 75(1):40.
\bibitem{10-bur-1}
\Aue{Kohavi, R.} 1995. A~study of cross-validation and bootstrap for accuracy 
estimation and model selection. \textit{14th  Joint Conference (International) on 
Artificial Intelligence Proceedings}. Montr$\acute{\mbox{e}}$al, 
Qu$\acute{\mbox{e}}$bec, Canada.  2:1137--1143. 
 \end{thebibliography}

 }
 }

\end{multicols}

%\vspace*{-7pt}

\hfill{\small\textit{Received October 2, 2018}}

%\pagebreak

%\vspace*{-22pt} 
  
  
  \Contr
  
  
  \noindent
  \textbf{Burlutskiy Vladimir V.} (b.\ 1975)~--- Candidate of Science (PhD) in 
physics and mathematics, Head of the Center of Information and Analytical Systems, 
Yugra Research Institute of Information Technologies, 151~Mira Str.,  
Khanty-Mansiysk 628011, Russian Federation; \mbox{BurlutskyVV@uriit.ru}
  
  \vspace*{3pt}
  
  \noindent
  \textbf{Yakimchuk Aleksandr V.} (b.\ 1994)~--- PhD student, Yugra State 
University, 16~Chekhova Str., Khanty-Mansiysk 628012, Russian Federation; 
\mbox{YakimchukAV@uriit.ru }
  
  \vspace*{3pt}
  
  \noindent
  \textbf{Melnikov Andrey V.} (b.\ 1956)~--- Doctor of Science in technology, 
professor, Director, Yugra Research Institute of Information Technologies, 151~Mira 
Str., Khanty-Mansiysk 628011, Russian Federation; 
\mbox{andmelnikov1956@yandex.ru} 
  
  \vspace*{3pt}
  
  \noindent
  \textbf{Tsaregorodtsev Aleksandr L.} (b.\ 1979)~--- Candidate of Science (PhD) 
in technology, First Deputy Director, Yugra Research Institute of Information 
Technologies, 151~Mira Str., Khanty-Mansiysk 628011, Russian Federation; 
\mbox{TsaregorodtsevAL@uriit.ru} 
  
  
  \vspace*{3pt}
  
  \noindent
  \textbf{Voloshin Semen V.} (b.\ 1991)~--- PhD student, Yugra State University, 
16~Chekhova Str., Khanty-Mansiysk 628012, Russian Federation; 
\mbox{voloshinsv@uriit.ru}
  
       
   
   
   
\label{end\stat}

\renewcommand{\bibname}{\protect\rm Литература}      %18
\def\stat{suchkov}

\def\tit{ИНФОРМАЦИОННО-АНАЛИТИЧЕСКАЯ АВТОМАТИЗИРОВАННАЯ 
СИСТЕМА <<МЕГАЛИТ>> В~ОПТИМИЗАЦИИ ДИАГНОСТИКИ И ЛЕЧЕНИЯ МОЧЕКАМЕННОЙ БОЛЕЗНИ}

\def\titkol{Информационно-аналитическая автоматизированная 
система <<Мегалит>> в~оптимизации диагностики} % и лечения мочекаменной болезни}

\def\autkol{М.\,П.~Кривенко, С.\,А.~Голованов,  П.\,А.~Савченко
 и др.}

\def\aut{М.\,П.~Кривенко$^1$, С.\,А.~Голованов$^2$, П.\,А.~Савченко$^3$, 
А.\,В.~Сивков$^4$,  А.\,П.~Сучков$^5$}

\titel{\tit}{\aut}{\autkol}{\titkol}

%{\renewcommand{\thefootnote}{\fnsymbol{footnote}}\footnotetext[1] {Статья 
%рекомендована к публикации в журнале Программным комитетом конференции 
%<<Электронные библиотеки: перспективные методы и технологии, электронные 
%коллекции>> (RCDL-2012).}}

\renewcommand{\thefootnote}{\arabic{footnote}}
\footnotetext[1]{Институт проблем информатики Российской академии наук, mkrivenko@ipiran.ru} 
\footnotetext[2]{Научно-исследовательский институт урологии, sergeygol124@mail.ru} 
\footnotetext[3]{Институт проблем информатики Российской академии наук, psavchenko@ipiran.ru} 
\footnotetext[4]{Научно-исследовательский институт урологии, uroinfo@yandex.ru} 
\footnotetext[5]{Институт проблем информатики Российской академии наук, asuchkov@ipiran.ru}


\Abst{В статье, первой из предполагаемой серии научных публикаций, рассматриваются 
результаты исследований по автоматизации информационных и аналитических процессов 
обследования, диагностирования и лечения мочекаменной болезни (МКБ). Существенную 
роль в создании систем диагностики МКБ играет разработка информационных технологий 
сбора клинических данных и формирования специализированных баз данных (БД). Изучена 
возможность создания и способы реализации ин\-фор\-ма\-ци\-он\-но-ана\-ли\-ти\-че\-ской 
автоматизированной системы (ИААС) по сбору, хранению и обработке клинических данных 
обследования больных, а также алгоритмизации процессов принятия решений при 
диагностике МКБ и выборе схем лечения и профилактики этого заболевания. Предложенные 
математические методы и алгоритмы могут найти применение при дальнейшем развитии 
фундаментальных научных исследований в области разработки математических методов 
моделирования ме\-ди\-ко-био\-ло\-ги\-че\-ских сис\-тем, а также при создании необходимого 
математического инструментария.}
      
\KW{информационно-аналитическая система; урология; компьютерная диагностика; схема 
лечения; схема профилактики}

\DOI{10.14357/19922264130409}

\vskip 14pt plus 9pt minus 6pt

      \thispagestyle{headings}

      \begin{multicols}{2}

            \label{st\stat}

\section{Введение}

      В настоящее время доля людей, у которых на протяжении их жизни диагностируется 
МКБ, довольно значительна и составляет в странах Западной Европы 5\%--9\%, в Канаде и 
США~--- 7\%--12\%, в странах Азии~--- 1\%--5\%~[1--4]. 
  %    
      Эпидемиологические исследования, проводимые в ряде индустриально развитых стран, 
указывают на сохранение тенденции к росту частоты возникновения МКБ 
среди населения. Так, число  впервые выявленных случаев
МКБ на 100\,000 населения за последние 
десятилетия возросло в США с 58,7 (1950--1954~гг.)\ до 85,1 (2000~г.)~\cite{4-su, 3-su}, 
в Японии~--- с~43,7 
(1965~г.)\ до~134 (2005~г.)~\cite{6-su, 5-su}, в России~--- со 123,3 (2002~г.)\ до~138,6 
(2010~г.)~[7, 8].
      
      По данным исследований~[9] с использованием БД Pediatric Health 
Information System (национальная БД, в которую включены данные об амбулаторных 
визитах, срочных госпитализациях и стационарном лечении детей из 42~детских больниц 
США) по сравнению с общим количеством госпитализированных пациентов число пациентов 
с МКБ увеличилось с 18,4 на 100\,000 населения в 1999~г.\ до 57,0 в 2008~г., годовой прирост 
составил 10,6\% ($p \hm<0{,}0001$). 
      
      В основе развития МКБ лежат характерные нарушения обмена веществ, приводящие к 
образованию камней в мочевых путях. Эти литогенные (камнеобразующие) нарушения обмена 
веществ характеризуются большим многообразием и проявляются различными 
патологическими изменениями биохимического состава крови и мочи пациента.
      
      Необходимым условием для выбора правильной тактики консервативного лечения с 
целью предупреждения повторного камнеобразования является исследование всего комплекса 
метаболических факторов риска (МФР), ответственных за развитие МКБ.
{\looseness=-1

}
      
      В этой связи большое внимание придается изуче\-нию особенностей фи\-зи\-ко-хи\-ми\-че\-ских 
па\-ра\-мет\-ров мочи, во многом определяющих вероятность образования мочевых камней~[10]. Кроме 
того, литогенные нарушения метаболизма зачастую имеют сложный многофакторный 
характер воздействия на процесс формирования камня. Это создает особые трудности для 
врача в полной и объективной оценке всех влияющих литогенных факторов обмена веществ, а 
также в принятии решения по диагностике и выбору лечебной тактики для конкретного 
больного. Отсюда возникает необходимость формирования БД анкетных и 
лабораторных исследований, систем, связанных с диагностикой МКБ и формирования базы 
знаний по профилактике и лечению этого заболевания.
      
\section{Системы компьютерной диагностики в~области урологии}

      Существенную роль в создании систем диагностики МКБ является разработка 
информационных технологий сбора клинических данных и формирования 
специализированных БД. К~ним относится упомянутая Pediatric Health Information 
System. В~ряде медицинских работ упоминается реестр по уролитиазу (БД по 
больным и результатам лечения) Юго-за\-пад\-но\-го медицинского центра Техасского 
университета: <<Retrospective data from the University of Texas Southwestern Medical Center 
\textit{Nephrolithiasis Registry} from 17~studies that dealt with physiologic and physicochemical 
effects of various magnesium and potassium salts were categorized into three groups and 
analyzed\ldots>>~[11]. Однако подробного описания данного реестра не приведено. 
      
      Задачи диагностики, дифференциальной диагностики, прогнозирования, выбора 
стратегии и тактики лечения позволяют решать экспертные медицинские системы~[12].
      
      Ряд работ посвящен использованию в урологии компьютерных диагностических систем 
на основе методов искусственных нейронных сетей (ИНС)~[13].
 Так, в онкоурологии смогли 
прогнозировать 5-лет\-нюю выживаемость пациентов, перенесших радикальную цистэктомию 
по поводу\linebreak
рака мочевого пузы\-ря~[14]. Искусственные ней\-ронные сети применили также для 
автоматизи\-рованного анализа показаний к биопсии предстательной железы~[15]. 
Методика основывалась на\linebreak 
выявлении общего прос\-тат-спе\-ци\-фи\-че\-ско\-го антигена (ПСА) и определении доли 
свободного ПСА. Чувствительность составила 95\%, специфичность~--- 34\%. При 
дополне\-нии нейросети моделью логистической регрессии специфичность возросла до 95\%. 
Искусственная нейронная сеть использовалась для выявле\-ния группы риска рака предстательной железы в сравнении с 
моделью логистической регрессии~[15]. Искусственная нейронная сеть так\-же 
продемонстрировала более точные 
прогностические возможности. Компьютерных систем диагностики именно МКБ по 
литературным данным не выявлено.
{ %\looseness=-1

}
      
      Отсюда ясно, что имеется настоятельная необходимость разработки аналитической 
системы диагностики и лечения больных МКБ в процессе их динами\-ческого наблюдения 
(мониторинге) для пред\-упреж\-де\-ния повторного камнеобразования. Отсутствие подобных 
аналитических систем для мониторинга больных МКБ послужило основанием для разработки 
опытного образца ИААС 
<<Мегалит>>. Создание системы осуществляется ИПИ РАН совместно с НИИ урологии 
Минздравсоцразвития России в рамках серии совместных на\-уч\-но-ис\-сле\-до\-ва\-тель\-ских работ.
      
\subsection*{Основные цели и~задачи создания информационно-аналитической автоматизированной системы
 <<Мегалит>>}

      \noindent
      \begin{enumerate}[1.]
      \item  Создание БД по результатам обследования пациента, включающей:\\[-15pt]
      \begin{itemize}
\item формализованные данные опроса пациента при первом и последующих визитах, 
содержащие информацию о факторах, способных оказывать влияние на возникновение и 
особенности клинического течения МКБ (lifestyle-фак\-то\-ры индивида, факторы среды, 
питания, профессии и проч.);\\[-15pt]
\item данные лабораторного обследования (результатов простого или расширенного 
лабораторного обследования).\\[-15pt] 
\end{itemize}
      \item  Создание аналитической подсистемы, обеспечивающей решение следующих 
задач:\\[-15pt]
      \begin{itemize}
\item на основании данных первичного опроса выявление наличия или отсутствия, а также 
степень риска развития МКБ и определение объема предполагаемого лабораторного 
обследования пациента (простое или расширенное обследование);\\[-15pt]
\item на основе анализа входных данных лабораторного обследования осуществление выбора 
дальнейшей тактики ведения больного~--- дополнительные виды исследования, выбор 
лечебных мероприятий (тип хирургического лечения, схема медикаментозной терапии, 
коррекция диеты и проч.);\\[-15pt]
\item реализация методов оптимального выбора (с учетом показаний и противопоказаний) вида 
хирургического лечения или схемы проведения профилактического лечения (включая прием 
специальных фармпрепаратов, рекомендации по модификации диеты и образа жизни).
\end{itemize}
\end{enumerate}

        При создании опытного образца ИААС <<Мегалит>> учитывались следующие 
требования.
      \begin{enumerate}[1.]
\item Опытный образец аналитической системы <<Мегалит>> должен иметь возможность 
ведения распределенной БД пациентов, содержащей результаты обследований, профили 
МФР и относительный индекс перенасыщенности мочи (ОИП) как 
исходные, так и измененные в результате назначенного лечения, и включать набор подсистем, 
вклю\-ча\-ющих программную реализацию разработанных методов диагностирования и лечения.
\item Данные простого лабораторного обследования пациента должны включать: 
\begin{itemize}
\item исследование химического состава мочевого камня; 
\item биохимическое исследование крови и мочи по различным параметрам; 
\item клинический анализ мочи с посевом на мик\-ро\-флору; 
\item обзорный рентгеновский снимок, сонограмму и другие виды инструментального 
обследования пациента.
\end{itemize}
%\end{enumerate}
       Данные биохимического исследования представлены величинами содержания в крови 
и моче ионов и соединений, способных приводить к образованию мочевых камней. При 
наличии патологических отклонений в биохимических исследования проводится расширенное 
лабораторное обследование.
 \item Данные расширенного лабораторного обследования включают протокол диагностики 
типа гиперкальциурии (ПД-ГКУ) (при выявлении повышен\-ной суточной экскреции кальция у 
пациента). Выполняется поэтапно, с помощью модифицированной по кальцию диеты. 
В~расширенное лабораторное обследование входит также полный диагностический протокол (ПДП)
больного МКБ. 
\item Полный диагностический протокол пред\-став\-ля\-ет собой выраженное в 
графическом виде исходное состояние обмена веществ у пациента с МКБ с выявленными 
МФР и динамику изменения показателей обмена веществ 
в результате проводимого лечения. Графическое отображение МФР и ОИП больного МКБ 
позволяет оценить степень выявленных нарушений и их динамику в процессе 
профилактического лечения и вносить в лечебную схему необходимые коррекции, также 
выбираемые по особому алгорит\-му. Выявленные при первичном обследовании МФР и ОИП 
служат основой для програм\-мно\-го выбора схем коррекции метаболических нарушений и 
предупреждения рецидивов МКБ. Коррекция включает в себя лечебные мероприятия, прием 
специальных фармпрепаратов, рекомендации по модификации диеты и образа жизни.
\item В~аналитической системе <<Мегалит>> пред\-усмат\-ри\-ва\-ет\-ся возможность ее обучения и 
настройки на основе получаемых новых данных о результатах лечения пациента (пациентов) 
на \mbox{каждом} этапе наблюдения.
\item Предусмотреть в программной реализации алгоритмов экспертного модуля: 
\begin{itemize}
\item
алгоритм оценки эффективности выбранной схемы лечения;
\item
алгоритм принятия решения по дальнейшему лечению;
\item
алгоритм поиска и выбора рациональной схемы профилактического лечения.
\end{itemize}
\item Оценить возможности разработки метода корректировки параметров подсистемы 
диагностирования и лечения на основе анализа вновь поступающих данных (обратная связь).
\item Разработанные аналитические методы и алгоритмы, реализованные в составе опытного 
образца аналитической системы <<Мегалит>>, должны пройти апробацию и тестирование в 
реальных клинических условиях. По результатам применения опытного образца должны быть 
сформулированы рекомендации по его совершенствованию и развитию.
\end{enumerate}

\begin{figure*} %fig1
   \vspace*{1pt}
 \begin{center}
 \mbox{%
 \epsfxsize=143.69mm
 \epsfbox{such-1.eps}
 }
 \end{center}
 \vspace*{-6pt}
\Caption{Структура ИААС}
\end{figure*}

\section{Основные подходы к~созданию информационно-аналитической
автоматизированной системы <<Мегалит>> и~их~реализация}
      Основные функции ИААС:
      \begin{itemize}
\item сбор и формализация данных, включая ведение реестра пациентов, системы словарей и 
справочников;
\item поддержка принятия решения по назначению и сбор данных диагностических 
исследований;
\item первичный и ретроспективный анализ тестов;
\item поддержка принятия решения по выбору схемы лечения, оценка эффективности схемы 
лечения;
\item поддержка принятия решения по дальнейшему лечению;
\item поиск и поддержка принятия решения по выбору рациональной схемы 
профилактического лечения.
\end{itemize}

      Информационно-аналитическая автоматизированная система
       <<Мегалит>> включает в себя подсистемы:
      \begin{itemize}
\item администрирования;
\item регистрации пациентов и сбора данных анкет, анамнеза;
\item ведения лингвистического обеспечения;
\item первичного обследования;
\item диагностических исследований;
\item экспертный модуль (поддержки процессов лечения).
\end{itemize}

Структурная схема ИААС <<Мегалит>> пред\-став\-ле\-на на рис.~1.


\begin{figure*}[b] %fig2
   \vspace*{1pt}
 \begin{center}
 \mbox{%
 \epsfxsize=161.589mm
 \epsfbox{such-2.eps}
 }
 \end{center}
 \vspace*{-6pt}
\Caption{Пирамида анализа данных}
\end{figure*}


      Для обеспечения возможности коллективной работы по формированию БД
системы и многопользовательского режима работы с ее аналитическим модулем она 
проектируется в виде веб-сай\-та, доступного авторизованным пользователям в сети Интернет. 
Основные базовые функции информационного сайта должны быть реализованы 
общесистемным функционалом его платформы. 
%
Таким образом, 
в про\-грам\-мно-тех\-но\-ло\-ги\-че\-ской платформе должны быть заложены следующие функции:
      \begin{enumerate}[(1)]
\item выполнение приложений~--- позволяет легко разрабатывать, развертывать различные 
приложения и управлять ими;
\item возможность совместной работы~--- позволяет отдельным пользователям и крупным 
организациям объединить свои ресурсы и работать вместе через Интернет;
\item управление содержимым~--- придает гибкость производству и управлению 
отдельными веб-уз\-ла\-ми, позволяя поставлять конечному пользователю 
приспособленное под него (персонифицированное) содержимое сайта;
\item управление пользователями~--- позволяет организации управлять пользователями, 
ресурсами и безопасностью внутри и вне системы сетевой защиты, а также предоставлять 
канал для внешних связей и проведения электронных транзакций;
\item контроль и управление про\-из\-во\-ди\-тель\-ностью~--- позволяет улучшать качество 
пользовательского интерфейса, обеспечивая:
\begin{itemize}
\item управление знаниями~--- помогает объединять внутреннюю и внешнюю 
информацию и предоставлять информацию, основанную на контекстной 
концепции;
\item поддержку поиска~--- обеспечивает клиента доступом к широкому спектру 
источников информации как внутри, так и вне сайта;
\item безопасность~--- защиту данных, приложений и транзакций;
\item стандартный www-до\-ступ к сайту~--- для технического обеспечения 
функционирования его содержимого.
\end{itemize}
\end{enumerate}
      
      Определяющими характеристиками веб-ре\-шений являются масштабируемость, 
доступность, надежность, защита данных от несанкционированного доступа, транзакционная 
целостность и распространение.
      
      Важной особенностью платформы является то, что она объединяет все необходимые 
модули, которые позволяют выполнять практически любую работу, связанную с созданием и 
обновлением сайта специалистом предметной области. 
      
      В качестве языка программирования выбран один из самых современных языков~--- 
C\#. ASP.NET~--- технология, которая является частью .NET и используется для разработки 
ин\-тер\-нет-ори\-ен\-ти\-ро\-ван\-но\-го программного обеспечения и ин\-тер\-нет-сай\-тов. 

Для работы ин\-тер\-нет-сай\-тов используется связка: операционная система Windows Server 
2008\;+\;ин\-тер\-нет-сер\-вер IIS~7.0\;+\;СУБД Ms SQL Server~2008.

\section{Концепция экспертного модуля системы}

\subsection{Основные подходы к~использованию статистических методов анализа данных 
в~урологии}

      Клинические БД содержат большое количество информации о пациентах и их 
заболеваниях. Скрытые (латентные) связи и структуры в этих данных могут быть источником 
новых медицинских знаний. К~сожалению, немногие из существующих технологий анализа 
данных оказываются непосредственно применимыми и действенными при обнаружении и 
описании этих латентных знаний, но, безусловно, универсальной из них является технология 
на принципах \textit{Data Mining}~--- извлечение скрытой информации из уже накопленных и 
пополняемых сведений об объекте исследования. Эта и ряд других сформировавшихся 
технологий, ориентированных на анализ массивов данных, терминологически пересекаются 
или оказываются взглядом на одном и том же, но с разных точек зрения (краткое освещение 
данного вопроса дано в~[17, разд.~1]). В~первую очередь речь идет о следующих 
подходах:
      \begin{itemize}
\item разведочный анализ данных~--- Exploratory Data Analysis (EDA);
\item извлечение скрытой информации из данных~--- Data Mining 
(DM);
\item обнаружение знаний в данных~--- Knowledge Discovery in Databases (KDD);
\item машинное обучение~--- Machine Learning (ML).
\end{itemize}

      Таким образом, обнаружение в данных ранее не известных, нетривиальных, 
практически полезных и доступных интерпретации знаний, необходимых для поддержки 
принятия решений в различных сферах человеческой деятельности, составляет суть DM. 
Говоря далее об анализе данных, будем понимать при этом цели, задачи, технологии, методы и 
алгоритмы, присущие DM. 
      
      На рис.~2 схематично изображена иерархия содержательной стороны анализа данных. 
Вертикальная стрелка слева показывает направление роста отдельных характеристик задач 
анализа данных в зависимости от их уровня. Примеры постановок практических задач 
приведены справа. Надо понимать, что <<восхождение>> по пирамиде анализа данных 
должно обеспечиваться обязательным существенным ростом объема используемой 
информации (данных и предположений об объектах исследования), а также глубиной 
проработки вопроса о качестве предлагаемых решений.


      
      \textbf{Основные принципы анализа данных.} Среди методов, которые 
использовались при решении проб\-ле\-мы обучения в ML, те, которые представляют\linebreak 
наибольший интерес при анализе данных (снижение размерности, оценивание распределения 
данных, регрессионный анализ, классификация, клас\-те\-ри\-за\-ция), теперь все вместе 
упоминаются как статисти\-ческое обучение. Проблема обучения делится на различные 
категории: две из них, наиболее близкие к статистике, суть контролируемое обучение или 
обучение с <<учителем>> и не\-конт\-ро\-ли\-ру\-емое, без <<учителя>>.
      
      Одна из самых важных задач в анализе данных состоит в том, чтобы оценить качество 
полученных решений, в частности точность предложенного прогноза (например, качество 
построенного классификатора). В~качестве меры точности прогноза обычно используется 
ошибка прогноза. Простейшая оценка ошибки прогноза строится с помощью тех же данных, 
которые используются для построения модели (такой вариант оценки называют самооценкой, 
оценкой переподстановки). Понятно, что в результате сформируется чрезмерно 
оптимистический взгляд на точность прогноза. 
      
      Очевидный способ улучшения состоит в обобщении: оценивать точность прогноза с 
помощью данных, независимых от тех, которые использовались для подгонки модели. 
Получить подобные независимые данные можно путем сбора новых данных. Если это 
невозможно, то имеет смысл разделить исходные данные на части и воспользоваться ими для 
решения самостоятельных задач. Обычная практика заключается в следующем: если набор 
данных достаточно велик, то необходимо использовать случайный механизм для разделения 
данных на два непересекающихся и независимых набора: 
      \begin{enumerate}[(1)]
\item данные для обучения, которые можно использовать для предварительного контроля 
данных, для формирования моделей;
\item тестирующие данные, которые будут использоваться для оценки 
качества построенной модели.
      \end{enumerate}
      
      Альтернативные методы расщепления данных для того, чтобы оценить тестовую 
ошибку, основаны на перепроверке~[16] и бут\-стреп-ме\-то\-де~[17].
     
     Суть вероятностной модели бутстреп-метода в данном случае состоит в следующем. 
Предположим, что по выборке $x\hm=(x_1,\ldots ,x_N)$ данных лабораторных исследований 
из распределения $F(u)$ оценивается значение $\vartheta\hm=\vartheta(F)$ некоторого 
функционала (например, классификатора заболеваний), заданного на семействе~$\mathbf{F}$. 
Качество оценки $\vartheta^*(X)$ характеризуется величиной
     $$
     R(\vartheta^*(X),\vartheta(F))=E_F\{L(\vartheta^*(X),\vartheta(F))\}\,,
     $$
где $L(\vartheta^*(X),\vartheta(F))$~--- потери от принятия оценки $\vartheta^*(X)$ вместо 
неизвестного значения $\vartheta(F)$. Бут\-стреп-ме\-тод позволяет оценить $ 
R(\vartheta^*(X),\vartheta(F))$ с помощью замены распределения~$F$ его некоторой оценкой 
$F^B$ и вычисления статистики $\vartheta^*$ по выборке $x^B$ объемом~$N$ из~$F^B$. 
Совокупность $x^B$ называется бут\-стреп-вы\-бор\-кой, статистика $\vartheta^*(x^B)$~--- 
бут\-стреп-реа\-ли\-за\-ци\-ей~$\vartheta^*$. 
     
     Условное распределение
     \begin{multline*}
     \mathrm{Pr}\left\{ \vartheta^*\left( X^B\right) <u\vert x_1,\ldots , x_N\right\} = {}\\
     {}=
     \int\limits_{\{y:\ \vartheta^*(y)<u\}} dF^B (y_1)\cdots dF^B(y_N)
     \end{multline*}
является бут\-стреп-оцен\-кой функции распределения $\mathrm{Pr}\left\{ 
\vartheta^*(X)<u\right\}$ статистики~$\vartheta^*$. 
     
     Процедура выбора оценки $F^B$ для~$F$ мотивируется наличием априорной 
информации. В~параметрической ситуации, когда $\mathbf{F}\hm= \left\{ F_\lambda, \, 
\lambda\in \Lambda\right\}$, оценка $F^B$ часто оказывается результатом подстановки 
вместо~$\lambda$ некоторой оценки~$\lambda^*$, т.\,е.\ $F^B\hm=F_{\lambda^*}$. 
{\looseness=1

}

Другая  ситуация относится к области непараметрической статистики. Здесь $F^B$ обычно 
оказывается эмпирической функцией распределения, т.\,е.\ каждому наблюденному значению 
(элементу исходной выборки) приписывается вероятность $1/N$. Бут\-стреп-вы\-бор\-ки тогда 
подчиняются условному полиномиальному распределению, сосредоточенному на  $x_1,\ldots , 
x_N$. 
     
     Наиболее трудную часть бутстреп-метода со\-став\-ля\-ет нахождение распределения 
$\vartheta^*(X^B)$, для чего применяются три приема:
    \begin{enumerate}[(1)]
\item прямое теоретическое вычисление;
\item аппроксимация с помощью метода статистических испытаний;
\item аппроксимация с помощью аналитических методов (например, используя разложение в 
ряд Тейлора).
\end{enumerate}

     Прямое теоретическое вычисление распределения $\vartheta^*(X^B)$ может 
осуществляться либо аналитическим путем, либо путем непосредственного перечис\-ле\-ния 
     бут\-стреп-вы\-бо\-рок и подсчета соответствующих вероятностей. Если оба приема 
недоступны (первый из-за аналитических сложностей, второй из-за вычислительных), то 
приходится прибегать к методу статистических испытаний, т.\,е.\ к повторению экспериментов 
по случайному формированию бут\-стреп-вы\-бор\-ки $x^B$ и подсчету значения 
$\vartheta^*(x^B)$.
     
     Следует обратить внимание на реальные возможности бут\-стреп-ме\-то\-да: он не 
позволяет получить новую информацию о наблюдаемых объектах, его назначение~--- 
сформировать объективное представление о свойствах использованных процедур анализа 
данных.
      
      В аналитической системе <<Мегалит>> накапливаются данные следующих типов:
      \begin{enumerate}[1.]
\item Неформализованные (неструктурированные), представленные в виде текста (например, 
текст назначения врача).
\item Формализованные:
\begin{enumerate}[{2.}1.]
\item Качественные:
\begin{enumerate}[{2.1.}1.]
\item Измеренные по шкале наименований (например, пол пациента).
\item Измеренные по порядковой шкале (например, порядковый номер сезона, когда 
обследовался пациент).
\end{enumerate}
\item Количественные:
\begin{enumerate}[{2.2.}1.]
\item Измеренные по одной из соответствующих шкал и при\-ни\-ма\-ющие значения из 
небольшого \mbox{набора} числовых значений (например, дата взятие анализов или 
количество обнаруженных у пациента камней).
\item Измеренные по одной из соответствующих шкал и принимающие значения в виде 
действительных чисел (например, уровень кальция в анализе крови пациента).
\end{enumerate}
\end{enumerate}
\end{enumerate}
      
      Приведенный систематизированный перечень встречающихся типов данных требует 
привлечения разнообразного арсенала средств, таких как лингвистический анализ (п.~1), 
статистический анализ категориальных данных (п.~2.1.1), ранговые процедуры 
(п.~2.1.2), статистический анализ на основе моделей дискретных и непрерывных 
распределений (пп.~2.2.1 и~2.2.2).
      
      Ошибки есть во всех видах БД; к сожалению, встречаются они и в данном 
случае. В~различных прикладных областях накоплен опыт (см., в частности,~[18]), 
позволяющий привести типичный перечень источников ошибок: фальсификация, неполнота, 
несогласованность, дублирование. 
      
      Те ошибки, которые легко обнаружить, вероятнее всего можно найти на стадии 
<<очистки>> данных, более же скрытые, неочевидные могут быть обнаружены только при 
анализе данных. <<Очистка>> данных обычно происходит, когда данные получены и прежде, 
чем они сохраняются в формате только для чтения в хранилище данных. В~частности, должны 
быть исключены ошибки, при которых переменные принимают значения, противоречащие 
естественным ограничениям (например, при описании химического состава камней значения 
отдельных переменных не могут превосходить 100\%). Доля подобных грубых ошибок в 
медицинских исследованиях может превышать~10\%. 
      
      Ошибки недопустимости значений должны быть описаны с помощью логических 
выражений, истинность которых проверяется на этапе <<очистки>>. В~случаях, когда их не 
удается исправить автоматически или автоматизированно, результат должен помечаться 
специальным образом. 
      
      Для данных, уже хранящихся в БД и явля\-ющих\-ся объектом анализа, могут 
быть характерны сле\-ду\-ющие проблемы: 
\begin{itemize}
\item наличие аномальных наблюдений (значения, которые 
существенно отличаются от основной массы наблюдений);
\item пропуски в данных;
\item малочисленность данных (ситуация, когда количество переменных превышает число 
наблюдений).
\end{itemize}
      
      Таким образом, статистический анализ конкретных данных является многоэтапным 
процессом, включающим планирование статистического исследования, организацию сбора 
необходимых статистических данных, первичное описание данных, оценивание характеристик 
данных, проверку статистических гипотез, анализ полученных решений, формулировку 
выводов, составление итоговых документов. 
      
      В основе принципов построения статистического вывода относительно данных лежат 
следующие положения:
      \begin{itemize}
\item при выборе семейства вероятностных распределений, описывающих данные, 
существенную роль играет предварительный анализ данных; последующий итеративный 
процесс уточнения априорных предположений направлен на построение модели, являющейся 
достаточно реалистичной и позволяющей строить содержательные выводы;
\item при построении методов анализа наряду с постановкой задач разработки оптимальных 
процедур и попыткой их решения следует не пренебрегать разумными подходами к созданию 
ка\-ких-ли\-бо процедур с последующим обязательным анализом предлагаемых решений; 
\item завершающим этапом построения методов анализа должен быть количественный или 
качественный анализ влияния на предлагаемые реше\-ния отклонений от априорных 
предположений, при этом исследование качества полученных решений реальнее всего 
проводить с по\-мощью бут\-стреп-ме\-тода.
\end{itemize}

\subsection{Принципиальные возможности создания экспертного модуля системы 
<<Мегалит>>}

      Повседневная деятельность врача требует решения задач интерпретации, диагностики, 
контроля и прогнозирования, т.\,е.\ таких задач, которые могут быть решены с помощью 
систем поддержки принятия решений. Медицина представляет одну из областей человеческой 
деятельности, где знания специалистов трудно формализуемы, однако разработка 
диагностических медицинских систем в настоящее время является актуальной задачей.
      
      При создании экспертного модуля системы <<Мегалит>>, предназначенного для 
поддержки принятия решения по диагностике заболевания, предполагается:
      \begin{itemize}
\item разработать систему представления медицинских знаний (с использованием данных 
анкет, анамнеза, данных инструментальных и лабораторных методов исследования);
\item разработать алгоритм механизма логического вывода (выполнение диагностики типа 
литогенного нарушения обмена веществ; выбора адекватной схемы лечения, оценки 
эффективности заданной схемы лечения с возможностью ее коррекции при дальнейшем 
мониторинге пациента).
\end{itemize}
      
      Система представления медицинских знаний позволяет выделять значимые для 
принятия врачебного решения или постановки медицинского диагноза данные (качественные 
или количественные). Так, анализ качественных данных анамнеза, анкетных данных позволяет 
сделать заключение о силе влияния наследственных, средовых и социальных факторов риска 
развития МКБ; потенциальной активности процесса камнеобразования. Этой же цели служат 
качественные и количественные данные, полученные при инструментальном/лабораторном 
(рентгенологическом, микробиологическом, ультразвуковом или антропометрическом) 
обследовании пациента.
      
      Большой массив количественных данных в виде числовых значений показателей 
получают при биохимическом исследовании. Именно он является основным объектом 
алгоритмизации при разработке экспертного модуля системы <<Мегалит>>. Применение 
этого модуля предназначено для объективной и более точной диагностики метаболического 
литогенного синдрома, выбора адекватной схемы лечения, качественной оценки результатов 
лечения, коррекции лечебной схемы на основе полученных данных в целях выбора 
оптимального лечебного воздействия на нарушенный обмен веществ у пациента.
      
      Учитывая, что указанные процессы являются алгоритмизуемыми, можно полагать, что 
принципиальные возможности создания экспертного модуля для системы <<Мегалит>> 
имеются.

\subsection{Качественное описание алгоритмической базы экспертного~модуля }

      Качественное описание алгоритмической базы включает: постановку задачи, описание 
входных и выходных данных, описание вход\-ных/вы\-ход\-ных форм пользовательского 
интерфейса, описание событий и реакций системы в рамках поддержки процесса оценки 
эффективности.
      
      \subsubsection*{Алгоритм оценки эффективности заданной схемы лечения}
      
      \paragraph*{Постановка задачи.} Качественно и количественно оценить эффективность 
применения выбранной схемы лечения (с выводом о продолжении ее использования в 
лечении; ее модификации в той или иной степени; замены на другую схему лечения).
      
      \paragraph*{Входные данные.} Входными данными служат те количественно измененные 
биохимические показатели, которые на предыдущем этапе про\-грам\-мно\-го анализа были 
определены (диагностированы) как характерные для данного метаболического синдрома. 
      
      \paragraph*{Выходные данные.} Выходными данными служат количественные значения 
биохимических признаков пролеченного метаболического синдрома, полученные при 
лабораторном исследовании после курса лечения. Эти данные должны быть про\-грам\-мно 
проанализированы в сравнении с их исходными (до начала лечения) значениями. 
Используется \textit{алгоритм <<Оценка качества лечебного эффекта>>}, который 
предполагает следующие варианты вывода о качестве лечения:
      \begin{itemize}
\item <<отсутствие эффекта>>;
\item <<слабо выраженный положительный эффект>>;
\item <<выраженный положительный эффект>>;
\item <<слабо выраженный отрицательный эффект>>; 
\item <<выраженный отрицательный эффект>>.
\end{itemize}
      
      \paragraph*{Выходные формы пользовательского интерфейса.} Выходные формы 
пользовательского интерфейса при этом могут быть представлены в виде таб\-ли\-цы со 
значениями биохимических признаков метаболического синдрома до и после лечения. 
Возможна опция отображения исходных данных в виде диаграммы или графика.

      \subsubsection*{Алгоритм принятия решения по дальнейшему лечению}
      
      \paragraph*{Постановка задачи.} Построить алгоритмические правила, позволяющие 
пользователю сделать вывод и принять решение об использовании данной схемы в 
дальнейшем лечении; модификации схемы в той или иной степени; замены данной схемы на 
другую схему лечения.
      
      \paragraph*{Входные данные:}
      \begin{itemize}
\item данные сравнительного анализа численных биохимических величин до и после лечения;
\item данные качественной оценки лечебного эффекта, получаемые в результате обработки 
данных сравнительного анализа численных биохимических величин до и после лечения.
     \end{itemize}
     
     При формировании и сборе данных первого типа выполняется процедура сравнения 
достигнутых в результате лечения величин значимых для данного метаболического синдрома 
показателей с их исходными значениями, диагностированными до начала лечения в ходе 
биохимического лабораторного исследования.
     
     Данные второго типа являются качественными, производными от данных первого типа. 
Эти данные представляют собой возможные варианты вывода о качестве лечения.
     
     \paragraph*{Выходные данные.} Выходными данными алгоритма принятия решения по 
дальнейшему лечению служат установленные типы рекомендаций по применявшейся схеме 
лечения: 
     \begin{itemize}
\item сохранение схемы без изменений и продолжение лечения;
\item модификация схемы трех степеней выраженности 
(незначительная, умеренная, существенная);
\item отказ от применения данной схемы и выбор новой схемы лечения.
\end{itemize}

\section{Перспективы развития информационно-аналитической автоматизированной
системы~<<Мегалит>>}

      Таким образом, разработана методологическая и техническая база для экспертной 
системы комплексной диагностики и профилактического \mbox{лечения} пациентов с МКБ. Учитывая 
не только медицинскую, но и социальную актуальность проб\-ле\-мы МКБ, а также трудности 
принятия врачебного решения в выборе адекватной тактики противорецидивного лечения 
этого заболевания, следует считать целесообразным создание технологий расширения и 
совершенствования функционала экспертного модуля системы <<Мегалит>> на основе 
анализа вновь поступающих данных с использованием принципа обратной связи.
      
      На следующих этапах планируется работа по оптимизации практического применения 
ИААС <<Мегалит>> в 
клинической урологии.
      
      Предложенные математические методы и алгоритмы найдут применение при 
дальнейшем развитии фундаментальных научных исследований в области разработки 
математических методов моделирования медико-биологических систем, а также при создании 
необходимого математического инструментария.
    %  
      В первую очередь это касается постановки развернутого диагноза, наиболее полно 
отражающего особенности метаболического типа конкретного больного (обследуемого), 
специфику функционального состояния почек и мочевых путей пациента. При этом также 
учитывается влияние различных модифицирующих факторов, таких как наличие или 
отсутствие инфекции и степени ее вы\-ра\-жен\-ности, воздействие социальных факторов, 
факторов питания, наследственности и~др. 
      
      Внедрение и практическое использование сис\-те\-мы <<Мегалит>> позволит 
сформировать представительный набор данных, на основе которого разработать методы 
выбора оптимальной лечебной тактики. Таким образом, конечные пользователи получат 
возможность дистанционной диагностики метаболических литогенных синдромов у пациента, 
оценки степени риска развития МКБ, выбора адекватных терапевтических схем лечения МКБ 
и/или профилактики рецидивов камнеобразования, ввода данных о пациенте в единый банк 
данных для последующего мониторинга и~др. 

{\small\frenchspacing
{%\baselineskip=10.8pt
\addcontentsline{toc}{section}{Литература}
\begin{thebibliography}{99}

\bibitem{1-su} %1
\Au{Ramello, A., Vitale C., Marangella D.} Epidemiology of nephrolithiasis~// J.~Nephrol., 2000. 
Vol.~13. Suppl.~3. P.~45--50.

\bibitem{4-su} %2
\Au{Trinchieri A., Coppi F., Montanari~E., Del Nero~A., Zanetti~G., Pisani~E.} Increase in the 
prevalence of symptomatic upper urinary tract stones during the last ten years~// Eur. Urol., 2000. 
Vol.~37. P.~23--25.

\bibitem{2-su} %3
\Au{Pearle M.\,S., Calhoun E.\,A., Curhan~G.\,C.} Urologic diseases in America project: 
Urolithiasis~// J.~Urology, 2005. Vol.~173. P.~848--857. 

\bibitem{3-su} %4
\Au{Lieske J.\,C., Pena de la Vega~L.\,S., Slezak~J.\,M., Bergstralh~E.\,J., Leibson~C.\,L., 
Ho~K.\,L., Gettman~M.\,T.} Renal stone epidemiology in Rochester, Minnesota: An update~// Kidney 
Int., 2006. Vol.~69. No.\,4. P.~760--764.


\bibitem{6-su} %5
\Au{Johnson C.\,M., Wilson D.\,M., O'Fallon~W.\,M., Malek~R.\,S., Kurland~L.\,T.} 
Renal stone epidemiology: A~25-year study in Rochester, Minnesota~// Kidney Int., 1979. Vol.~16. 
No.\,5. P.~624--631.

\bibitem{5-su} %6
\Au{Yasui T., Iguchi M., Suzuki~S., Kohri~K.} Prevalence and epidemiological characteristics of 
urolithiasis in Japan: National trends between 1965 and 2005~// Urology, 2008. Vol.~71. No.\,2. 
P.~209--213.

\bibitem{7-su}
Заболеваемость населения России в 2003~году: Статистические материалы.~--- М., 2004 
(электронная версия МЗ и СР РФ и ЦНИИ организации и информатизации здравоохранения 
МЗ и СР РФ). 
{\sf http:// www.minzdravsoc.ru/docs/mzsr/stat/17}.
\bibitem{8-su}
\Au{Аполихин О.\,И., Сивков А.\,В., Солнцева Т.\,В., Комарова~В.\,А.}
Анализ урологической заболеваемости в Российской Федерации в 2005--2010~годах~//
Экспериментальная и клиническая урология, 2012. №\,2. C.~4--12.
{\sf http://ecuro.ru/article/analiz-urologicheskoi-zabolevaemosti-v-rossiiskoi-federatsii-v-2005-2010-godakh}.
%\bibitem{9-su}
%Заболеваемость населения России в 2007~году: Статистические материалы.~--- М., 2008 
%(электронная версия МЗ и СР РФ и ЦНИИ организации и информатизации здравоохранения 
%МЗ и СР РФ). {\sf http:// www.minzdravsoc.ru/docs/mzsr/stat/27}.
\bibitem{11-su} %9
\Au{Routh J.\,C., Graham D.\,A., Nelson~C.\,P.} Epidemiological trends in pediatric urolithiasis at 
United States freestanding pediatric hospitals~// J.~Urology, 2010. Vol.~184. No.\,3. P.~1100--1104.

\bibitem{10-su} %10
\Au{Голованов С.\,А., Дрожжева В.\,В.}
Кристаллообразующая активность мочи при оксалатном уролитиазе~//
Экспериментальная и клиническая урология, 2010. №\,2. C.~24--29.
{\sf http://ecuro.ru/ article/kristalloobrazuyushchaya-aktivnost-mochi-pri-oksalatnom-urolitiaze}.
\bibitem{12-su} %11
\Au{Bonny O., Rubin A., Huang~Ch.-L., Frawley~W.\,H., Pak~C.\,Y.\,C., Moe~O.\,W.} Mechanism 
of urinary calcium regulation by urinary magnesium and pH~// J.~Am. Soc. Nephrol., 2008. Vol.~19. 
No.\,8. P.~1530--1537.
\bibitem{13-su} %12
\Au{Дюк В.\,А., Эмануэль В.\,Л.} Информационные технологии в 
ме\-ди\-ко-био\-ло\-ги\-че\-ских исследованиях.~--- СПб.: Питер, 2003. 525~с.
\bibitem{14-su} %13
\Au{Liew P.\,L., Lee Y.\,C., Lin~Y.\,C., \textit{et al}.} Comparison of artificial neural networks with 
logistic regression in prediction of gallbladder disease among obese patients~// Digest. Liver Dis., 2007. 
Vol.~39. No.\,4. P.~356--362.
\bibitem{15-su} %14
\Au{Bassi P., Sacco E., De Marco~V., \textit{et al}.} Prognostic accuracy of an artificial neural 
network in patients undergoing radical cystectomy for bladder cancer: A~comparison with logistic 
regression analysis~// BJU Int., 2007. Vol.~99. No.\,5. P.~1007--1012.
\bibitem{16-su} %15
\Au{Stephan C., Xu C., Finne~P., \textit{et al}.} Comparison of two different artificial neural 
networks for prostate biopsy indication in two different patient populations~// J.~Urology, 2007. 
Vol.~70. No.\,3. P.~596--601.
%\bibitem{17-su} 
%\Au{Chun F.\,K., Karakiewicz P.\,I., Briganti~A., \textit{et al}.} A~critical appraisal of logistic 
%regression-based nomograms, artificial neural networks, classification and regression-tree models, 
%look-up tables and risk-group stratification models for prostate cancer ~/ BJU Intern., 2007. Vol.~99. 
%No.\,4. P.~794--800.


\bibitem{19-su} %16
\Au{Stone M.} Cross-validatory choice and assessment of statistical predictions (with discussion)~// 
J.~Roy. Stat. Soc. B, 1974. Vol.~36. P.~111--147.

\bibitem{18-su} %17
\Au{Efron B.} Bootstrap methods: Another look at the jackknife~// Ann. Stat., 1979. Vol.~7. 
P.~1--26.

\bibitem{21-su} %18
\Au{Izenman A.\,J.} Modern multivariate statistical techniques.~--- Springer, 2008. 731~p. 
%\bibitem{20-su} %20
%\Au{Breiman L.} The 1991 census adjustment: Undercount or bad data~// Stat. Sci., 1994. 
%Vol.~9. P.~458--475.


\end{thebibliography} } }

\end{multicols}

\hfill{\small\textit{Поступила в редакцию 17.04.13}}
%\vspace*{12pt}

%\hrule

%\vspace*{2pt}

%\hrule

\newpage

\def\tit{THE INFORMATION-ANALYTICAL COMPUTER SYSTEM ``MEGALITH'' 
IN~OPTIMIZATION OF~THE~DIAGNOSIS AND~TREATMENT OF~UROLITHIASIS}

\def\titkol{The information-analytical computer system ``Megalith'' 
in~the~field of~urology}

\def\aut{M.\,P.~Krivenko$^1$, S.\,A.~Golovanov$^2$, P.\,A.~Savchenko$^1$, A.\,V.~Sivkov$^2$, 
 and~A.\,P.~Suchkov$^1$}
 
 \def\autkol{S.\,A.~Golovanov, M.\,P.~Krivenko, P.\,A.~Savchenko, et al.}


\titel{\tit}{\aut}{\autkol}{\titkol}

\vspace*{-12pt}


\noindent
$^1$Institute of Informatics 
Problems, Russian Academy of Sciences, Moscow 119333, Russian Federation\\
\noindent $^2$Research Institute of Urology, Moscow 105425, Russian Federation

\vspace*{12pt}

\def\leftfootline{\small{\textbf{\thepage}
\hfill INFORMATIKA I EE PRIMENENIYA~--- INFORMATICS AND APPLICATIONS\ \ \ 2013\ \ \ volume~7\ \ \ issue\ 4}
}%
 \def\rightfootline{\small{INFORMATIKA I EE PRIMENENIYA~--- INFORMATICS AND APPLICATIONS\ \ \ 2013\ \ \ volume~7\ \ \ issue\ 4
\hfill \textbf{\thepage}}}

\Abste{In this article, that is the first of an expected series of scientific publications, the results of 
research on automation of the information and analytical processes of the urolithic disease (ULD) 
survey, diagnosis, and treatment are discussed. A significant role in creating the systems of ULD 
diagnostics has the development of information technologies for clinical data collection and 
formation of specialized databases. The possibility of creation and the ways of realization of 
information-analytical computer system of collection, storage, and processing of the clinical data of 
patients examination, as well as programming decision-making processes in the diagnosis ULD and 
the choice of schemes of treatment and prevention of this disease has been studied. The developed 
mathematical methods and algorithms may be applied to the further fundamental scientific researches 
in the field of development of mathematical methods of medical and biological systems modeling; 
besides, they may be applied for necessary mathematical tools creation.}

\KWE{informational-analytical system; urology; computer diagnostics; treatment 
scheme; scheme of prevention} 

\DOI{10.14357/19922264130409}

%\Ack
%\noindent
%?????

\vspace*{3pt}

  \begin{multicols}{2}

\renewcommand{\bibname}{\protect\rmfamily References}
%\renewcommand{\bibname}{\large\protect\rm References}

{\small\frenchspacing
{%\baselineskip=10.8pt
\addcontentsline{toc}{section}{References}
\begin{thebibliography}{99}


\bibitem{1-su-1}
\Aue{Ramello, A., C.~Vitale, and D.~Marangella}. 2000. Epidemiology of nephrolithiasis. 
\textit{J.~Nephrol.} 13(Suppl.~3):45--50.

\bibitem{4-su-1} %2
\Aue{Trinchieri, A., F.~Coppi, E.~Montanari, A.~Del Nero, G.~Zanetti, and E.~Pisani}. 2000. 
Increase in the prevalence of symptomatic upper urinary tract stones during the last ten years. 
\textit{Eur. Urol.} 37:23--25.

\bibitem{2-su-1} %3
\Aue{Pearle, M.\,S., E.\,A.~Calhoun, and G.\,C.~Curhan}. 2005. Urologic diseases in America 
project: Urolithiasis. \textit{J.~Urology} 173:848--857. 
\bibitem{3-su-1} %4
\Aue{Lieske, J.\,C., L.\,S.~Pena de la Vega, J.\,M.~Slezak, E.\,J.~Bergstralh; C.\,L.~Leibson, 
K.\,L.~Ho, and M.\,T.~Gettman}. 2006. Renal stone epidemiology in Rochester, Minnesota: An 
update. \textit{Kidney Int.} 69(4):760--768.


\bibitem{6-su-1} %5
\Aue{Johnson, C.\,M., D.\,M.~Wilson, W.\,M.~O'Fallon, R.\,S.~Malek, and L.\,T.~Kurland}. 
1979. Renal stone epidemiology: A~25-year study in Rochester, Minnesota. \textit{Kidney 
Int.} 16(5):624--631.

\bibitem{5-su-1} %6
\Aue{Yasui, T, M.~Iguchi, S.~Suzuki, and K.~Kohri}. 2008. Prevalence and epidemiological 
characteristics of urolithiasis in Japan: National trends between 1965 and 2005. \textit{Urology}  
 71(2):209--213.

\bibitem{7-su-1} %7
Russian Ministry of Health:
Central Research Institute of Organization and Informatization of Population.
2004. Zabolevaemost' naseleniya Rossii v 2003 godu: Sta\-ti\-sti\-che\-skie materialy [Morbidity of 
population of Russia in 2003: Statistical materials]. Мoscow.  Electronic version.
{\sf http://www.minzdravsoc.ru/docs/mzsr/stat/17}.

\bibitem{8-su-1}
\Aue{Apolikhin,~O.\,I., A.\,V.~Sivkov, T.\,V.~Solntseva, and V.\,A.~Komarova.}
2012. Analysis of urological morbidity in the Russian Federation within the period of 2005--2010.
\textit{Experimental and Clinical Urology} 2:4--12.
{\sf http://ecuro.ru/en/article/analysis-urological-morbidity-russian-federation-within-period-2005-2010}.


\bibitem{11-su-1} %9
\Aue{Routh, J.\,C., D.\,A.~Graham, and C.\,P.~Nelson}. 2100. Epidemiological trends in 
pediatric urolithiasis at United States freestanding pediatric hospitals. \textit{J.~Urology} 
184(3):1100--1104.

%\bibitem{9-su-1}
%Russian Ministry of Health:
%Central Research Institute of Organization and Informatization of Population.
%2008. Zabolevaemost' naseleniya Rossii v 2007 godu: Sta\-ti\-sti\-che\-skie materialy [Morbidity of 
%population of Russia in 2007: Statistical materials]. Мoscow. Electronic version.
%{\sf http://www.minzdravsoc.ru/docs/mzsr/stat/27}.
\bibitem{10-su-1} %10
\Aue{Golovanov, S.\,A., and V.\,V.~Drozhzheva}.
2010. Crystal formation activity of urine in oxalate urolithiasis.
\textit{Experimental and Clinical Urology} 2:24--29.
{\sf http://ecuro.ru/en/article/crystal-formation-activity-urine-oxalate-urolithiasis}.



\bibitem{12-su-1} %12
\Aue{Bonny, O., A.~Rubin, Ch.-L.~Huang, W.\,H.~Frawley, C.\,Y.\,C.~Pak, and 
O.\,W.~Moe}. 2008. Mechanism of urinary calcium regulation by urinary magnesium and $pH$. 
\textit{J.~Am. Soc. Nephrol.} 19(8):1530--1537.
\bibitem{13-su-1}
\Aue{Djuk, V.\,A., and V.\,L.~Jemanujel'}. 2003. \textit{Informatsionnye tekhnologii v 
mediko-biologicheskikh issledovaniyakh} [\textit{Information technologies in medical and 
biological researches}]. St.\ Petersburg, Russia: Piter, 2003. 525~p.
\bibitem{14-su-1}
\Aue{Liew, P.\,L., Y.\,C.~Lee, Y.\,C.~Lin, \textit{et al}.} 2007. Comparison of artificial neural 
networks with logistic regression\linebreak\vspace*{-12pt}

\pagebreak

\noindent
 in prediction of gallbladder disease among obese patients. 
\textit{Digest. Liver Dis.} 39(4):356--362.

%\pagebreak


\bibitem{15-su-1}
\Aue{Bassi, P., E. Sacco, V.~De Marco, \textit{et al}.} 2007. Prognostic accuracy of an 
artificial neural network in patients undergoing radical cystectomy for bladder cancer: 
A~comparison with logistic regression analysis. \textit{BJU Int.}  99(5):1007--1012.
\bibitem{16-su-1}
\Aue{Stephan, C., C.~Xu, P.~Finne, \textit{et al}.} 2007. Comparison of two different artificial 
neural networks for prostate biopsy indication in two different patient populations. 
\textit{J.~Urology}  70(3):596--601.

%\columnbreak

\bibitem{19-su-1} %18
\Aue{Stone, M.} 1974. Cross-validatory choice and assessment of statistical predictions (with 
discussion). \textit{J.~Roy. Stat. Soc. B} 36:111--147.
%\bibitem{17-su-1}
%\Aue{Chun, F.\,K., P.\,I.~Karakiewicz, A.~Briganti, \textit{et al}.} 2007. A~critical appraisal 
%of logistic regression-based nomograms, artificial neural networks, classification and 
%regression-tree models, look-up tables and risk-group stratification models for prostate cancer. 
%\textit{BJU Intern}.  99(4):794--800.



\bibitem{18-su-1} %19
\Aue{Efron, B.} 1979. Bootstrap methods: Another look at the jackknife. \textit{Ann.  
Stat.}  7:1--26.

\bibitem{21-su-1} %17
\Aue{Izenman, A.\,J.} 2008. \textit{Modern multivariate statistical techniques}. Springer. 
731~p.


 
 
 
% \bibitem{20-su-1} %20
%\Aue{Breiman, L.} 1994. The 1991 census adjustment: Undercount or bad data.  
%\textit{Stat. Sci.}  9:458--75.
 

\end{thebibliography}
} }

\end{multicols}

\hfill{\small\textit{Received April 17, 2013}}

\Contr

\noindent
\textbf{Krivenko Michail P.} (b.\ 1946)~--- 
Doctor of Science in technology, principal scientist, Institute of Informatics 
Problems, Russian Academy of Sciences, Moscow 119333, Russian Federation;  mkrivenko@ipiran.ru

\vspace*{3pt}


\noindent\textbf{Golovanov Sergey  A.} (b.\ 1950)~--- Doctor of Science in medicine, Head of 
Laboratory, Research Institute of Urology, Moscow 105425, Russian Federation;
sergeygol124@mail.ru
 

\vspace*{3pt}

\noindent
\textbf{Savchenko Pavel A.} (b.\ 1967)~--- software engineer, Institute of Informatics 
Problems, Russian Academy of Sciences, Moscow 119333, Russian Federation;  
psavchenko@ipiran.ru

\vspace*{3pt}

\noindent
\textbf{Sivkov Andrey V.} (b.\ 1957)~--- Doctor of Science in medicine, Deputy director, 
Research Institute of Urology, Moscow 105425, Russian Federation;  uroinfo@yandex.ru

\vspace*{3pt}

\noindent
\textbf{Suchkov Alexander P.} (b. 1954)~--- Doctor of Science in technology, principal 
scientist, Institute of Informatics Problems, Russian Academy of Sciences, Moscow 119333, Russian Federation;  
asuchkov@ipiran.ru 

\label{end\stat}

\renewcommand{\bibname}{\protect\rm Литература} %19





%%%%%%%%%%%%%%%%%%%%%%%%%%%%%%%%%%%%%%%%%%%%%%%

%\def\stat{rez}
{%\hrule\par
%\vskip 7pt % 7pt
\raggedleft\Large \bf%\baselineskip=3.2ex
Р\,Е\,Ц\,Е\,Н\,З\,И\,И \vskip 17pt
    \hrule
    \par
\vskip 6pt plus 6pt minus 3pt }

%\thispagestyle{headings} %с верхним колонтитулом
%\thispagestyle{myheadings} %с нижним колонтитулом, но в верхнем РЕЦЕНЗИИ

\def\tit{НОВАЯ КНИГА И.\,Н.~СИНИЦЫНА, А.\,С.~ШАЛАМОВА <<ЛЕКЦИИ ПО ТЕОРИИ 
ИНТЕГРИРОВАННОЙ ЛОГИСТИЧЕСКОЙ ПОДДЕРЖКИ>> (М.: ТОРУС ПРЕСС, 2012. 624~с.)}

%1
\def\aut{Д.ф.-м.н., профессор С.\,Я.~Шоргин}

\def\auf{\ }

\def\leftkol{\ % РЕЦЕНЗИИ
}

\def\rightkol{ \ } 

%\def\leftkol{\ } % ENGLISH ABSTRACTS}

%\def\rightkol{\ } %ENGLISH ABSTRACTS}

%\def\leftkol{РЕЦЕНЗИИ}

%\def\rightkol{РЕЦЕНЗИИ}

\titele{\tit}{\aut}{\auf}{\leftkol}{\rightkol}
\vspace*{-18pt}


     \label{st\stat}

     \begin{multicols}{2}
     {\small
     {\baselineskip=10.1pt
     

      В книге представлено системное изложение теоретических основ одного из новейших 
направлений в \mbox{об\-ласти} экономики послепродажного обслуживания изделий наукоемкой 
продукции (ИНП) длительного пользования~--- интегрированной логистической поддержки
(ИЛП). 
{\looseness=1

}

Приведены также результаты новых работ, выполненных в Институте проблем информатики 
Российской академии наук в рамках научного направления <<Информационные технологии и 
анализ сложных сис\-тем>>.
 {%\looseness=1

}
     
      Излагаемые в книге научные подходы позво\-ляют карди\-наль\-но реформировать 
существующие системы производства и эксплуатации ИНП путем создания и внед\-ре\-ния 
методов рационального и оптимального управ\-ле\-ния процессами расходования 
вре\-мен\-н$\acute{\mbox{ы}}$х, 
мате\-ри\-аль\-ных, трудовых и других ресурсов на всех стадиях жизненного цикла изделий (ЖЦИ) по 
критериям экономической целесообразности и эф\-фек\-тив\-ности.
  {\looseness=1

}
    
      В книге приведен краткий обзор причин возник\-новения и
      развития CALS-методологии как основы 
современных международных стандартов по созданию и функционированию глобальных 
ин\-фор\-ма\-ци\-он\-но-ком\-му\-ни\-ка\-ци\-он\-ных систем, ее ключевых возможностей и эффективности 
результатов ее использования. 
Авторы %\linebreak 
предлагают ряд научных обоснований для разработки 
единой теории проектирования и управления систем ИЛП для полноценного использования 
преимуществ %\linebreak
 суще\-ст\-ву\-ющей методологии, определяют \mbox{общую} структурную схему 
комплексной системы <<ИНП-СППО>> и необходимость разработки для ее описания 
гибридных стохастических моделей.
{%\looseness=1

}

%\columnbreak
      
      Книга состоит из пяти частей, где последовательно излагается материал по каждой из 
следующих тем: <<Интегрированная логистическая поддержка>>, <<Теория гибридных 
стохастических систем и компьютерная поддержка исследований и разработок>>, <<Основы 
математического моделирования, анализа и синтеза систем послепродажного обслуживания>>, 
<<Определение и анализ показателей экспортного потенциала ИНП при проектировании>>, 
<<Задачи управления поддержкой послепродажного обслуживания>>, а также 
<<Моделирование инвестиционных процессов ИЛП в условиях неравновесных финансовых 
рынков>>. 
   
      В конце каждой главы приведены выводы и даны вопросы и задания для 
самоконтроля. В~приложениях содержатся основные определения по программам работ по 
анализу ИЛП, логистическим базам данных и компьютерным решениям, эквивалентной статистической 
линеаризации нелинейных преобразований ИЛП, справочный материал, а также развернутые 
уравнения для вероятностных характеристик.


      \def\leftkol{РЕЦЕНЗИИ}

\def\rightkol{РЕЦЕНЗИИ} 

      
      Книга заинтересует широкий круг специалистов и может быть использована научными 
проектными организациями в сфере промышленного производства ИНП. Большое количество 
иллюстраций, примеров и вопросов, обращенных к читателю, позволяет использовать книгу 
также в качестве учебного пособия для студентов и аспирантов машиностроительных, 
транспортных и~других специальностей, а также для самостоятельного изучения. 
{%\looseness=-1

}

Книга 
представляет несомненный интерес для специалистов и студентов в области прикладной 
математики и информатики.
    

}

}
\end{multicols}

%\newpage

\def\stat{authorsrus}
{%\hrule\par
%\vskip 7pt % 7pt
\raggedleft\Large \bf%\baselineskip=3.2ex
О\,Б\ \ А\,В\,Т\,О\,Р\,А\,Х \vskip 17pt
    \hrule
    \par
\vskip 21pt plus 8pt minus 4pt }


\def\tit{\ }

\def\aut{\ }

\def\auf{\ }

\def\leftkol{\ } % ENGLISH ABSTRACTS}

\def\rightkol{ОБ АВТОРАХ} %ENGLISH ABSTRACTS}

\titele{\tit}{\aut}{\auf}{\leftkol}{\rightkol}
      
            \label{st\stat}



\vspace*{24pt}

\begin{multicols}{2}




\noindent
\textbf{Архипов Олег Петрович} (р.\ 1948)~---
кандидат технических наук, директор Орловского филиала Института проб\-лем информатики
Российской академии наук
%302025, г.Орел, Московское шоссе, д.137

\vspace*{3pt}

\noindent
\textbf{Бирюкова Татьяна Константиновна} (р.\ 1968)~---
кандидат фи\-зи\-ко-ма\-те\-ма\-ти\-че\-ских наук, старший научный сотрудник Института проб\-лем информатики
Российской академии наук

\vspace*{3pt}

\noindent 
\textbf{Бобков  Сергей Геннадьевич} (р.\ 1955)~---
доктор технических наук,  заведующий отделением На\-уч\-но-ис\-сле\-до\-ва\-тель\-ско\-го 
института системных исследований Российской академии наук
%117218, Москва, Нахимовский просп., 36, к.1 

\vspace*{3pt}

\noindent \textbf{Васильев Николай Семенович} (р.\ 1952)~--- доктор 
фи\-зи\-ко-ма\-те\-ма\-ти\-че\-ских наук, профессор, 
МГТУ им.\ Н.\,Э.~Баумана 
%, Москва 105005, 2-я Бауманская ул., д.~5,

\vspace*{3pt}

\noindent
\textbf{Гершкович Максим Михайлович} (р.\ 1968)~---
старший научный сотрудник Института проб\-лем информатики
Российской академии наук

\vspace*{3pt}

\noindent 
\textbf{Дьяченко Юрий Георгиевич} (р.\ 1958)~--- кандидат технических наук, 
старший научный сотрудник Института проб\-лем информатики
Российской академии наук

\vspace*{3pt}

\noindent 
\textbf{Ерошенко Александр Андреевич} (р.\ 1989)~--- аспирант кафедры 
математической статистики факультета вычисли\-тельной математики и кибернетики 
Московского государственного университета им.\ М.\,В.~Ломоносова
%119991, Москва ГСП-1, Ленинские горы, д.\ 1, стр. 52

\vspace*{3pt}
 
\noindent 
\textbf{Захаров Виктор Николаевич} (р.\ 1948)~--- 
доктор технических наук, доцент, ученый секретарь Института проб\-лем информатики
Российской академии наук

\vspace*{3pt}

\noindent
\textbf{Зейфман Александр Израилевич} (р.\ 1954)~---
доктор фи\-зи\-ко-ма\-те\-ма\-ти\-че\-ских наук, профессор, 
заведующий кафедрой Вологодского государственного университета; 
старший научный сотрудник Института проб\-лем информатики
Российской академии наук; главный научный сотрудник ИСЭРТ Российской академии наук

\vspace*{3pt}

\noindent
\textbf{Зыкин Сергей Владимирович} (р.\ 1959)~--- 
доктор технических наук, профессор, заведующий лабораторией Института математики 
им.\ С.\,Л.~Соболева Сибирского отделения Российской академии наук, Новосибирск 
%630090, пр.\ ак.\ Коптюга, 4 

\vspace*{4pt}

\noindent
\textbf{Киреев Владимир Иванович} (р.\ 1938)~---
доктор фи\-зи\-ко-ма\-те\-ма\-ти\-че\-ских наук, профессор Московского 
государственного горного университета
%Адрес: Россия, 119991, г. Москва, Ленинский проспект, д. 6

%\columnbreak

\vspace*{4pt}

\noindent
\textbf{Козеренко Елена Борисовна} (р.\ 1959)~---
кандидат филологических наук, заведующая лабораторией Института проб\-лем информатики
Российской академии наук

\vspace*{4pt}

\noindent
\textbf{Королев Виктор Юрьевич} (р.\ 1954)~--- доктор
фи\-зи\-ко-ма\-те\-ма\-ти\-че\-ских наук, профессор кафедры математической 
статистики факультета вычисли\-тельной математики и кибернетики 
Московского государственного университета; 
ведущий научный сотрудник Института проб\-лем информатики
Российской академии наук

\vspace*{4pt}

\noindent
\textbf{Коротышева Анна Владимировна} (р.\ 1988)~---
старший преподаватель Вологодского государственного университета

\vspace*{4pt}

\noindent 
\textbf{Кун Де Турк} (р.\ 1981)~--- научный сотрудник 
исследовательской группы SMACS факультета телекоммуникаций и обработки информации
Университета Гента, Бельгия
%В-9000 Гент, Бельгия

\vspace*{4pt}

\noindent
\textbf{Лупенцов Олег Сергеевич} (р.\ 1986)~---
аспирант Омского государственного института сервиса
%Омск 644043, ул.\ Певцова 13

\vspace*{4pt}

\noindent
\textbf{Лучко Олег Николаевич} (р.\ 1961)~---
кандидат педагогических наук, профессор, заведующий кафедрой 
Омского государственного института сервиса
%Омск 644043, ул.\ Певцова 13

\vspace*{4pt}

\noindent
\textbf{Малашенко Юрий Евгеньевич} (р.\ 1946)~---
доктор фи\-зи\-ко-ма\-те\-ма\-ти\-че\-ских наук, заведующий сектором 
Вычислительного центра им.\ А.\,А.~Дородницына Российской академии наук
%Адрес: 119333, Москва, ул. Вавилова, 40,

\vspace*{4pt}

\noindent
\textbf{Маньяков Юрий Анатольевич} (р.\ 1984)~---
кандидат технических наук, научный сотрудник Орловского филиала Института проб\-лем информатики
Российской академии наук
%302025, г.Орел, Московское шоссе, д.137

\vspace*{4pt}

\noindent
\textbf{Маренко Валентина Афанасьевна} (р.\ 1951)~---
кандидат технических наук, доцент, старший научный сотрудник 
Института математики им.\ С.\,Л.~Соболева Сибирского отделения Российской академии наук
%Новосибирск 630090, пр. ак. Коптюга, 4 

\vspace*{3pt}

\noindent 
\textbf{Морозов Евсей Викторович} (р.\ 1947)~--- доктор 
фи\-зи\-ко-ма\-те\-ма\-ти\-че\-ских, профессор, ведущий научный сотрудник 
Института прикладных математических исследований Карельского научного центра Российской
академии наук; 
%%185910 Россия, Республика Карелия, г.\ Петрозаводск, ул.\ Пушкинская, 11
профессор Петрозаводского государственного университета, Петрозаводск
%185910 Россия, Республика Карелия, г.\ Петрозаводск, пр.\ Ленина, 33

%\pagebreak

\vspace*{3pt}

\noindent
\textbf{Назарова Ирина Александровна} (р.\ 1966)~---
кандидат фи\-зи\-ко-ма\-те\-ма\-ти\-че\-ских наук, 
научный сотрудник Вычислительного центра им.\ А.\,А.~Дородницына Российской академии наук 
%Адрес: 119333, Москва, ул. Вавилова, 40

\vspace*{3pt}

\noindent
\textbf{Павлов Игорь Валерианович} (р.\ 1945)~--- 
доктор фи\-зи\-ко-ма\-те\-ма\-ти\-че\-ских наук, профессор МГТУ им.\ Н.\,Э.~Баумана 
%Москва 105005, 2-я Бауманская ул., д.~5 

%\pagebreak

\vspace*{3pt}

\noindent 
\textbf{Потахина Любовь Викторовна} (р.\ 1989)~--- аспирантка
Института прикладных математических исследований Карельского научного центра
Российской академии наук; 
%%185910 Россия, Республика Карелия, г.\ Петрозаводск, ул.\ Пушкинская, 11
инженер Петрозаводского государственного университета, Петрозаводск
%185910 Россия, Республика Карелия, г.\ Петрозаводск, пр.\ Ленина, 33

\vspace*{3pt}

\noindent 
\textbf{Рождественский Юрий Владимирович} (р.\ 1952)~--- 
кандидат технических наук, заведующий сектором Института проб\-лем информатики
Российской академии наук

\vspace*{3pt}

\noindent 
\textbf{Синицын Игорь Николаевич} (р.\ 1940)~--- доктор технических наук,
профессор, заслуженный деятель\linebreak\vspace*{-12pt}

\columnbreak

\noindent
 науки РФ, заведующий отделом Института проб\-лем информатики
Российской академии наук

\vspace*{7pt}


\noindent
\textbf{Сиротинин Денис Олегович} (р.\ 1984)~---
кандидат технических наук, научный сотрудник Орловского филиала Института проб\-лем информатики
Российской академии наук
%302025, г.Орел, Московское шоссе, д.137

\vspace*{7pt}

%\columnbreak

\noindent 
\textbf{Соколов  Игорь Анатольевич} (р.\ 1954)~--- академик (действительный член) Российской 
академии наук, доктор технических наук, директор Института проб\-лем информатики
Российской академии наук

\vspace*{7pt}

\noindent
\textbf{Степченков Юрий Афанасьевич} (р.\ 1951)~---
кандидат технических наук, заведующий отделом Института проб\-лем информатики
Российской академии наук

\vspace*{7pt}

\noindent
\textbf{Сурков Алексей Викторович} (р.\ 1978)~--- 
старший научный сотрудник На\-уч\-но-ис\-сле\-до\-ва\-тель\-ско\-го 
института системных исследований Российской академии наук
%117218, Москва, Нахимовский просп., 36, к.1 

\vspace*{7pt}

\noindent 
\textbf{Шестаков Олег Владимирович} (р.\ 1976)~--- доктор 
фи\-зи\-ко-ма\-те\-ма\-ти\-че\-ских, доцент кафедры математической статистики 
факультета вычисли\-тельной математики и кибернетики Московского 
государственного университета им.\ М.\,В.~Ломоносова; 
%119991, Москва ГСП-1, Ленинские горы, д.\ 1, стр. 52
старший научный сотрудник Института проб\-лем информатики
Российской академии наук
%, Москва 119333, ул. Вавилова, д.~44, корп.~2

\vspace*{7pt}

\noindent 
\textbf{Шоргин Сергей Яковлевич} (р.\ 1952.)~--- доктор
фи\-зи\-ко-ма\-те\-ма\-ти\-че\-ских наук, профессор, заместитель директора Института 
проб\-лем информатики Российской академии наук





%%%%%%%%%%%%%%%%%%%%%%%%%%%%%%%%%%%%%%%%%%%%%%%%%%%%%%%%%%%%%%%%%%%%%%%%%%%%%%%




%\def\rightkol{ОБ АВТОРАХ}
%\def\leftkol{ОБ АВТОРАХ}

 \label{end\stat}





%\def\leftfootline{\small{\textbf{\thepage}
%\hfill ИНФОРМАТИКА И ЕЁ ПРИМЕНЕНИЯ\ \ \ том~7\ \ \ выпуск~1\ \ \ 2013}
%}%
% \def\rightfootline{\small{ИНФОРМАТИКА И ЕЁ ПРИМЕНЕНИЯ\ \ \ том~7\ \ \ выпуск~1\ \ \ 2013
%\hfill \textbf{\thepage}}}


%\thispagestyle{myheadings}



\end{multicols}

\newpage  

%\def\stat{cont}
{%\hrule\par
%\vskip 7pt % 7pt
\raggedleft\Large \bf%\baselineskip=3.2ex
А\,В\,Т\,О\,Р\,С\,К\,И\,Й\ \ У\,К\,А\,З\,А\,Т\,Е\,Л\,Ь\ \ З\,А\ \ 2\,0\,0\,7 г. \vskip 17pt
    \hrule
    \par
\vskip 21pt plus 6pt minus 3pt }

\label{st\stat}

\def\tit{\ }

\def\aut{\ }
\def\auf{\ }

\def\leftkol{\ } % ENGLISH ABSTRACTS}

\def\rightkol{\ } %ENGLISH ABSTRACTS}

\titele{\tit}{\aut}{\auf}{\leftkol}{\rightkol}


\contentsline {chapter}{\ }{Выпуск \quad Стр.} 
\contentsline {section}{\textbf{Батракова Д.\,А., Королев В.\,Ю., Шоргин С.\,Я.}\ \ Новый метод вероятностно-ста\-ти\-сти\-че\-ско\-го анализа информационных потоков в\nobreakspace {}телекоммуникационных сетях}{\qquad 1 \qquad 40} 
\contentsline {section}{\textbf{Борисов А.\,В.}\ \ Байесовское оценивание в системах наблюдения с\nobreakspace {}марковскими скачкообразными процессами: игровой подход}{\qquad 2 \qquad 65}
\contentsline {section}{\textbf{Босов А.\,В., Иванов А.\,В.}\ \ Программная инфраструктура информационного Web-пор\-тала}{\qquad 2 \qquad 50}
\contentsline {section}{\textbf{Захаров В.\,Н., Калиниченко Л.\,А., Соколов И.\,А., Ступников С.\,А.}\ \ Конструирование канонических информационных моделей для интегрированных информационных систем}{\qquad 2 \qquad 15}
\contentsline {section}{\textbf{Захаров В.\,Н., Козмидиади В.\,А.}\ \ Средства обеспечения отказоустойчивости при\-ло\-жений}{\qquad 1 \qquad 14} 
\contentsline {section}{\textbf{Иванов А.\,В.}\ \ см. Босов А.\,В.\hfill\hfill\hfill\hfill\hfill\hfill\hfill\hfill\hfill\hfill\hfill\hfill\hfill\hfill\hfill\hfill\hfill\hfill\hfill\hfill\hfill\hfill\hfill\hfill\hfill\hfill\hfill\hfill\hfill\hfill\hfill\hfill\hfill\hfill\hfill}{\ }
\contentsline {section}{\textbf{Ильин В.\,Д., Соколов И.\,А.}\ \ Символьная модель системы знаний информатики в\nobreakspace {}че\-ло\-ве\-ко-автоматной среде}{\qquad 1 \qquad 66} 
\contentsline {section}{\textbf{Калиниченко Л.\,А.}\ \ см. Захаров В.\,Н.\hfill\hfill\hfill\hfill\hfill\hfill\hfill\hfill\hfill\hfill\hfill\hfill\hfill\hfill\hfill\hfill\hfill\hfill\hfill\hfill\hfill\hfill\hfill\hfill\hfill\hfill\hfill\hfill\hfill\hfill\hfill\hfill\hfill\hfill\hfill}{\ }
\contentsline {section}{\textbf{Козеренко Е.\,Б.}\ \ Лингвистическое моделирование для систем машинного перевода и обработки знаний}{\qquad 1 \qquad 54} 
\contentsline {section}{\textbf{Козмидиади В.\,А.}\ \ см. Захаров В.\,Н.\hfill\hfill\hfill\hfill\hfill\hfill\hfill\hfill\hfill\hfill\hfill\hfill\hfill\hfill\hfill\hfill\hfill\hfill\hfill\hfill\hfill\hfill\hfill\hfill\hfill\hfill\hfill\hfill\hfill\hfill\hfill\hfill\hfill\hfill\hfill }{\ } 
\contentsline {section}{\textbf{Королев В.\,Ю.}\ \ см. Батракова Д.\,А.\hfill\hfill\hfill\hfill\hfill\hfill\hfill\hfill\hfill\hfill\hfill\hfill\hfill\hfill\hfill\hfill\hfill\hfill\hfill\hfill\hfill\hfill\hfill\hfill\hfill\hfill\hfill\hfill\hfill\hfill\hfill\hfill\hfill\hfill\hfill}{\ } 
\contentsline {section}{\textbf{Кудрявцев А.\,А., Шоргин С.\,Я.}\ \ Байесовский подход к\nobreakspace {}анализу систем массового обслуживания и\nobreakspace {}показателей надежности}{\qquad 2 \qquad 76}
\contentsline {section}{\textbf{Печинкин А.\,В., Соколов И.\,А., Чаплыгин В.\,В.}\ \ Многолинейная система массового обслуживания с конечным накопителем и ненадежными приборами}{\qquad 1 \qquad 27} 
\contentsline {section}{\textbf{Печинкин А.\,В., Соколов И.\,А., Чаплыгин В.\,В.}\ \ Стационарные характеристики многолинейной\nobreakspace {}системы массового обслуживания с\nobreakspace {}одновременными отказами приборов}{\qquad 2 \qquad 39}
\contentsline {section}{\textbf{Синицын И.\,Н.}\ \ Корреляционные методы построения аналитических информационных моделей флуктуаций полюса Земли по априорным данным}{\qquad 2 \qquad \hphantom{9}2}
\contentsline {section}{\textbf{Синицын И.\,Н.}\ \ Развитие теории фильтров Пугачева для оперативной обработки информации в стохастических системах}{{\qquad 1 \qquad \hphantom{9}3}} 
\contentsline {section}{\textbf{Соколов И.\,А.}\ \ см. Захаров В.\,Н.\hfill\hfill\hfill\hfill\hfill\hfill\hfill\hfill\hfill\hfill\hfill\hfill\hfill\hfill\hfill\hfill\hfill\hfill\hfill\hfill\hfill\hfill\hfill\hfill\hfill\hfill\hfill\hfill\hfill\hfill\hfill\hfill\hfill\hfill\hfill}{\ }
\contentsline {section}{\textbf{Соколов И.\,А.}\ \ см. Ильин В.\,Д.\hfill\hfill\hfill\hfill\hfill\hfill\hfill\hfill\hfill\hfill\hfill\hfill\hfill\hfill\hfill\hfill\hfill\hfill\hfill\hfill\hfill\hfill\hfill\hfill\hfill\hfill\hfill\hfill\hfill\hfill\hfill\hfill\hfill\hfill\hfill}{\ } 
\contentsline {section}{\textbf{Соколов И.\,А.}\ \ см. Печинкин А.\,В.\hfill\hfill\hfill\hfill\hfill\hfill\hfill\hfill\hfill\hfill\hfill\hfill\hfill\hfill\hfill\hfill\hfill\hfill\hfill\hfill\hfill\hfill\hfill\hfill\hfill\hfill\hfill\hfill\hfill\hfill\hfill\hfill\hfill\hfill\hfill}{\ } 
\contentsline {section}{\textbf{Соколов И.\,А.}\ \ см. Печинкин А.\,В.\hfill\hfill\hfill\hfill\hfill\hfill\hfill\hfill\hfill\hfill\hfill\hfill\hfill\hfill\hfill\hfill\hfill\hfill\hfill\hfill\hfill\hfill\hfill\hfill\hfill\hfill\hfill\hfill\hfill\hfill\hfill\hfill\hfill\hfill\hfill}{\ }
\contentsline {section}{\textbf{Ступников С.\,А.}\ \ см. Захаров В.\,Н.\hfill\hfill\hfill\hfill\hfill\hfill\hfill\hfill\hfill\hfill\hfill\hfill\hfill\hfill\hfill\hfill\hfill\hfill\hfill\hfill\hfill\hfill\hfill\hfill\hfill\hfill\hfill\hfill\hfill\hfill\hfill\hfill\hfill\hfill\hfill}{\ }
\contentsline {section}{\textbf{Чаплыгин В.\,В.}\ \ см. Печинкин А.\,В.\hfill\hfill\hfill\hfill\hfill\hfill\hfill\hfill\hfill\hfill\hfill\hfill\hfill\hfill\hfill\hfill\hfill\hfill\hfill\hfill\hfill\hfill\hfill\hfill\hfill\hfill\hfill\hfill\hfill\hfill\hfill\hfill\hfill\hfill\hfill}{\ } 
\contentsline {section}{\textbf{Чаплыгин В.\,В.}\ \ см. Печинкин А.\,В.\hfill\hfill\hfill\hfill\hfill\hfill\hfill\hfill\hfill\hfill\hfill\hfill\hfill\hfill\hfill\hfill\hfill\hfill\hfill\hfill\hfill\hfill\hfill\hfill\hfill\hfill\hfill\hfill\hfill\hfill\hfill\hfill\hfill\hfill\hfill}{\ }
\contentsline {section}{\textbf{Шоргин С.\,Я.}\ \ см. Батракова Д.\,А.\hfill\hfill\hfill\hfill\hfill\hfill\hfill\hfill\hfill\hfill\hfill\hfill\hfill\hfill\hfill\hfill\hfill\hfill\hfill\hfill\hfill\hfill\hfill\hfill\hfill\hfill\hfill\hfill\hfill\hfill\hfill\hfill\hfill\hfill\hfill}{\ } 
\contentsline {section}{\textbf{Шоргин С.\,Я.}\ \ см. Кудрявцев А.\,А.\hfill\hfill\hfill\hfill\hfill\hfill\hfill\hfill\hfill\hfill\hfill\hfill\hfill\hfill\hfill\hfill\hfill\hfill\hfill\hfill\hfill\hfill\hfill\hfill\hfill\hfill\hfill\hfill\hfill\hfill\hfill\hfill\hfill\hfill\hfill}{\ }
%\thispagestyle{myheadings}
\def\leftfootline{\small{\textbf{\thepage}
\hfill ИНФОРМАТИКА И ЕЁ ПРИМЕНЕНИЯ\ \ \ том~1\ \ \ выпуск~2\ \ \ 2007}
}%
 \def\rightfootline{\small{ИНФОРМАТИКА И ЕЁ ПРИМЕНЕНИЯ\ \ \ том~1\ \ \ выпуск~2\ \ \ 2007
 \hfill \textbf{\thepage}}}
 \label{end\stat} 
                     
%\def\stat{cont-e}
{%\hrule\par
%\vskip 7pt % 7pt
\raggedleft\Large \bf%\baselineskip=3.2ex
2\,0\,0\,7\ \ A\,U\,T\,H\,O\,R\ \ I\,N\,D\,E\,X \vskip 17pt
    \hrule
    \par
\vskip 21pt plus 6pt minus 3pt }

\label{st\stat}

\def\tit{\ }

\def\aut{\ }
\def\auf{\ }

\def\leftkol{\ } % ENGLISH ABSTRACTS}

\def\rightkol{\ } %ENGLISH ABSTRACTS}

\titele{\tit}{\aut}{\auf}{\leftkol}{\rightkol}


\contentsline {chapter}{\ }{Issue \quad Page} 
\contentsline {subsection}{\textbf{Batrakova D.\,A., Korolev V.\,Yu., Shorgin S.\,Ya.}\ \ A New Method for the Probabilistic and Statistical Analysis of Information Flows in Telecommunication Networks}{\qquad 1 \qquad 40} 
\contentsline {subsection}{\textbf{Borisov A.\,V.}\ \ Bayesian Estimation in\nobreakspace {}Observation Systems with\nobreakspace {}Markov Jump Processes: Game-Theoretic Approach}{\qquad 2 \qquad 65} 
\contentsline {subsection}{\textbf{Bosov A.\,V., Ivanov A.\,V.}\ \ Linguistic Simulation for Machine Translation and Knowledge Management Systems}{\qquad 2 \qquad 50} 
\contentsline {subsection}{\textbf{Chaplygin V.\,V.} see Pechinkin A.\,V.\hfill\hfill\hfill\hfill\hfill\hfill\hfill\hfill\hfill\hfill\hfill\hfill\hfill\hfill\hfill\hfill\hfill\hfill\hfill\hfill\hfill\hfill\hfill\hfill\hfill\hfill\hfill\hfill\hfill\hfill\hfill\hfill\hfill\hfill\hfill}{\ }
\contentsline {subsection}{\textbf{Chaplygin V.\,V.} see Pechinkin A.\,V.\hfill\hfill\hfill\hfill\hfill\hfill\hfill\hfill\hfill\hfill\hfill\hfill\hfill\hfill\hfill\hfill\hfill\hfill\hfill\hfill\hfill\hfill\hfill\hfill\hfill\hfill\hfill\hfill\hfill\hfill\hfill\hfill\hfill\hfill\hfill}{\ }
\contentsline {subsection}{\textbf{Ilyin V.\,D., Sokolov I.\,A.}\ \ The Symbol Model of Informatics Knowledge System in Human-Automaton Environment}{\qquad 1 \qquad 66} 
\contentsline {subsection}{\textbf{Ivanov A.\,V.} see Bosov A.\,V.\hfill\hfill\hfill\hfill\hfill\hfill\hfill\hfill\hfill\hfill\hfill\hfill\hfill\hfill\hfill\hfill\hfill\hfill\hfill\hfill\hfill\hfill\hfill\hfill\hfill\hfill\hfill\hfill\hfill\hfill\hfill\hfill\hfill\hfill\hfill}{\ }
\contentsline {subsection}{\textbf{Kalinichenko L.\,A.} see Zakharov V.\,N.\hfill\hfill\hfill\hfill\hfill\hfill\hfill\hfill\hfill\hfill\hfill\hfill\hfill\hfill\hfill\hfill\hfill\hfill\hfill\hfill\hfill\hfill\hfill\hfill\hfill\hfill\hfill\hfill\hfill\hfill\hfill\hfill\hfill\hfill\hfill}{\ }
\contentsline {subsection}{\textbf{Korolev V.\,Yu.} see Batrakova D.\,A.\hfill\hfill\hfill\hfill\hfill\hfill\hfill\hfill\hfill\hfill\hfill\hfill\hfill\hfill\hfill\hfill\hfill\hfill\hfill\hfill\hfill\hfill\hfill\hfill\hfill\hfill\hfill\hfill\hfill\hfill\hfill\hfill\hfill\hfill\hfill}{\ }
\contentsline {subsection}{\textbf{Kozerenko E.\,B.}\ \ Linguistic Simulation for Machine Translation and Knowledge Management Systems}{\qquad 1 \qquad 54} 
\contentsline {subsection}{\textbf{Kozmidiady V.\,A.} see Zakharov V.\,N.\hfill\hfill\hfill\hfill\hfill\hfill\hfill\hfill\hfill\hfill\hfill\hfill\hfill\hfill\hfill\hfill\hfill\hfill\hfill\hfill\hfill\hfill\hfill\hfill\hfill\hfill\hfill\hfill\hfill\hfill\hfill\hfill\hfill\hfill\hfill}{\ }
\contentsline {subsection}{\textbf{Kudryavtsev A.\,A., Shorgin S.\,Ya.}\ \ Bayesian Approach to Queueing Systems and Reliability Characteristics}{\qquad 2 \qquad 76} 
\contentsline {subsection}{\textbf{Pechinkin A.\,V., Sokolov I.\,A., Chaplygin V.\,V.}\ \ Multichannel Queuing System with Finite Buffer and Unreliable Servers}{\qquad 1 \qquad 27} 
\contentsline {subsection}{\textbf{Pechinkin A.\,V., Sokolov I.\,A., Chaplygin V.\,V.}\ \ Stationary Characteristics of a Multichannel Queueing System with\nobreakspace {}Simultaneous Refusals of Servers}{\qquad 2 \qquad 39} 
\contentsline {subsection}{\textbf{Shorgin S.\,Ya.} see Batrakova D.\,A.\hfill\hfill\hfill\hfill\hfill\hfill\hfill\hfill\hfill\hfill\hfill\hfill\hfill\hfill\hfill\hfill\hfill\hfill\hfill\hfill\hfill\hfill\hfill\hfill\hfill\hfill\hfill\hfill\hfill\hfill\hfill\hfill\hfill\hfill\hfill}{\ }
\contentsline {subsection}{\textbf{Shorgin S.\,Ya.} see Kudryavtsev A.\,A.\hfill\hfill\hfill\hfill\hfill\hfill\hfill\hfill\hfill\hfill\hfill\hfill\hfill\hfill\hfill\hfill\hfill\hfill\hfill\hfill\hfill\hfill\hfill\hfill\hfill\hfill\hfill\hfill\hfill\hfill\hfill\hfill\hfill\hfill\hfill}{\ }
\contentsline {subsection}{\textbf{Sinitsyn I.\,N.}\ \ Correlational Methods for Analytical Informational Models of the Earth Pole Fluctuations Design Based on a priori Data}{\qquad 2 \qquad \hphantom{9}2}
\contentsline {subsection}{\textbf{Sinitsyn I.\,N.}\ \ Development of Pugachev Filtering for Stochastic Systems}{\qquad 1 \qquad \hphantom{9}3}
\contentsline {subsection}{\textbf{Sokolov I.\,A.} see Ilyin V.\,D.\hfill\hfill\hfill\hfill\hfill\hfill\hfill\hfill\hfill\hfill\hfill\hfill\hfill\hfill\hfill\hfill\hfill\hfill\hfill\hfill\hfill\hfill\hfill\hfill\hfill\hfill\hfill\hfill\hfill\hfill\hfill\hfill\hfill\hfill\hfill}{\ }
\contentsline {subsection}{\textbf{Sokolov I.\,A.} see Pechinkin A.\,V.\hfill\hfill\hfill\hfill\hfill\hfill\hfill\hfill\hfill\hfill\hfill\hfill\hfill\hfill\hfill\hfill\hfill\hfill\hfill\hfill\hfill\hfill\hfill\hfill\hfill\hfill\hfill\hfill\hfill\hfill\hfill\hfill\hfill\hfill\hfill}{\ }
\contentsline {subsection}{\textbf{Sokolov I.\,A.} see Pechinkin A.\,V.\hfill\hfill\hfill\hfill\hfill\hfill\hfill\hfill\hfill\hfill\hfill\hfill\hfill\hfill\hfill\hfill\hfill\hfill\hfill\hfill\hfill\hfill\hfill\hfill\hfill\hfill\hfill\hfill\hfill\hfill\hfill\hfill\hfill\hfill\hfill}{\ }
\contentsline {subsection}{\textbf{Sokolov I.\,A.} see Zakharov V.\,N.\hfill\hfill\hfill\hfill\hfill\hfill\hfill\hfill\hfill\hfill\hfill\hfill\hfill\hfill\hfill\hfill\hfill\hfill\hfill\hfill\hfill\hfill\hfill\hfill\hfill\hfill\hfill\hfill\hfill\hfill\hfill\hfill\hfill\hfill\hfill}{\ }
\contentsline {subsection}{\textbf{Stupnikov S.\,A.} see Zakharov V.\,N.\hfill\hfill\hfill\hfill\hfill\hfill\hfill\hfill\hfill\hfill\hfill\hfill\hfill\hfill\hfill\hfill\hfill\hfill\hfill\hfill\hfill\hfill\hfill\hfill\hfill\hfill\hfill\hfill\hfill\hfill\hfill\hfill\hfill\hfill\hfill}{\ }
\contentsline {subsection}{\textbf{Zakharov V.\,N., Kalinichenko L.\,A., Sokolov I.\,A., Stupnikov S.\,A.}\ \ Development of Canonical Information Models for Integrated Information Systems}{\qquad 2 \qquad 15} 
\contentsline {subsection}{\textbf{Zakharov V.\,N., Kozmidiady V.\,A.}\ \ Means Providing Applications Fault Tolerance}{\qquad 1 \qquad 14} 
\def\leftfootline{\small{\textbf{\thepage}
\hfill ИНФОРМАТИКА И ЕЁ ПРИМЕНЕНИЯ\ \ \ том~1\ \ \ выпуск~2\ \ \ 2007}
}%
 \def\rightfootline{\small{ИНФОРМАТИКА И ЕЁ ПРИМЕНЕНИЯ\ \ \ том~1\ \ \ выпуск~2\ \ \ 2007
 \hfill \textbf{\thepage}}}
 \label{end\stat} 


%\end{document}

%
\def\stat{rekl}
%\label{preobr}

%\def\tit{АКАДЕМИК ПУГАЧЁВ  ВЛАДИМИР СЕМЁНОВИЧ\\
%25.03.1911--25.03.1998}


%   \vspace*{-48pt}
%   \begin{center}\LARGE
%Академик Пугачёв  Владимир Семёнович\\ (25.03.1911--25.03.1998)
%   \end{center}

   %\vspace*{2.5mm}

   \begin{center}

{\prgsh\LARGE
ЮБИЛЕИ}

\end{center}
%\hrule

\vspace*{6pt}


   \vspace*{8mm}

   \thispagestyle{empty}


%\def\stat{emel}


\section*{К 70-летию заместителя директора ИПИ РАН,\\ члена редколлегии журнала
<<Информатика и её применения>>\\ доктора технических наук В.\,И.~Будзко}

\vspace*{18pt}




          \begin{multicols}{2}

%            \label{st\stat}

\begin{center}
\vspace*{1pt}
\mbox{%
\epsfxsize=78mm
\epsfbox{bud-1.eps}
}
\end{center}

\vspace*{12pt}

      14 августа 2014~г.\ исполнилось 70~лет за\-мес\-ти\-те\-лю директора ИПИ РАН по
научной работе доктору технических наук Владимиру Игоревичу Будзко.

      Владимир Игоревич Будзко родился в г.~Москве. Высшее образование получил на факультете
элект\-рон\-но-вы\-чис\-ли\-тель\-ных устройств в Московском
ин\-же\-нер\-но-фи\-зи\-че\-ском институте
(МИФИ), который он окончил в 1968~г., после чего был на\-прав\-лен для прохождения
службы в одну из войс\-ко\-вых частей, где прошел путь от инженера до первого заместителя
командира войсковой части.

      С приходом В.\,И.~Будзко в ИПИ РАН (2001~г.)\ в институте
сформировалось новое научное на\-прав\-ле\-ние теоретических исследований~--- <<Постро\-ение
ин\-фор\-ма\-ци\-он\-но-те\-ле\-ком\-му\-ни\-ка\-ци\-он\-ных\linebreak сис\-тем
высокой до\-ступ\-ности>>. В~рамках этого
направления выполнен широкий круг фундаментальных исследований по поиску подходов и
определению принципов построения средств обеспечения доступности, конфиденциальности
и целостности современных крупномасштабных
ин\-фор\-ма\-ци\-он\-но-те\-ле\-ком\-му\-ни\-ка\-ци\-он\-ных
сис\-тем (ИТС). Разработаны основные сис\-тем\-но-тех\-ни\-че\-ские принципы и базовые
архитектурные решения построения перспективных для условий России ИТС с
централизованной обработкой и хранением информации, сочетающих в себе свойства
высокой доступности, отказо- и катастрофоустойчивости, информационной защищенности.
Определены принципы, методы и математические основы рационального построения и
оптимизации средств восстановления функционирования центров обработки данных (ЦОД)
после возникновения отказов и катастроф, передачи и хранения данных, обеспечения
информационной безопасности при достижении минимальной совокупной стоимости
владения такими системами. Результаты нашли практическое воплощение при реализации
проектов в интересах ряда отечественных государственных и негосударственных
организаций, таких как Банк России (БР), Внешторгбанк, ОАО <<ГМК <<Норильский Никель>>,
<<Газпром>>, Минэкономразвития России, Правительство Москвы, а также ряд силовых
ведомств.

      Под руководством В.\,И.~Будзко начиная с 2001~г.\ выполнен комплекс
      на\-уч\-но-ис\-сле\-до\-ва\-тель\-ских и
      опыт\-но-кон\-ст\-рук\-тор\-ских работ (свыше 100~проектов),
направленных на развитие электронной информационной технологии БР.
Разработаны концепции развития ИТС БР сначала до 2008~г., а затем до 2013~г., которые
были приняты в качестве основы проведения технической политики. За реализацию проекта
<<Катастрофоустойчивая тер\-ри\-то\-ри\-аль\-но-рас\-пре\-де\-лен\-ная
      ин\-фор\-ма\-ци\-он\-но-те\-ле\-ком\-му\-ни\-ка\-ци\-он\-ная сис\-те\-ма централизованной
обработки банковской информации>> В.\,И.~Будзко удостоен Премии Правительства РФ в
области науки и техники за 2010~г.

      В.\,И.~Будзко возглавлял и возглавляет работы по ряду других прикладных проектов,
связанных с созданием, совершенствованием и развитием крупномасштабных ИТС.

      В.\,И.~Будзко~--- генерал-майор, доктор технических наук, член-кор\-рес\-пон\-дент
Академии криптографии РФ, известный ученый в области информатики и применения
информационных технологий при построении территориально распределенных ИТС
различного назначения. Является автором свыше 250~научных работ, опубликованных в
на\-уч\-но-тех\-ни\-че\-ских и специальных изданиях.

    \thispagestyle{empty}

      В.\,И.~Будзко уделяет большое внимание подготовке научных кадров. Под его
руководством защищено 6~диссертаций на соискание ученой степени кандидата
технических наук. Свыше 30~лет он читает лекции в ИКСИ Академии ФСБ, профессор
кафедры НИЯУ МИФИ. Является членом двух диссертационных советов, главным
редактором журнала <<Системы высокой доступности>> и членом редколлегии журнала
<<Информатика и её применения>>.

      \bigskip

      Редакционный совет и Редакционная коллегия журнала <<Информатика и её
применения>> сердечно поздравляют Владимира Игоревича Будзко с 70-ле\-ти\-ем и желают
крепкого здоровья и новых научных достижений.

\end{multicols}

%%Информатика и её применения
%Том 12   Выпуск 1-4   Год 2018

\def\stat{cont}
{%\hrule\par
%\vskip 7pt % 7pt
\raggedleft\Large \bf%\baselineskip=3.2ex
А\,В\,Т\,О\,Р\,С\,К\,И\,Й\ \ У\,К\,А\,З\,А\,Т\,Е\,Л\,Ь\ \ З\,А\ \ 2\,0\,1\,8 г. \vskip 17pt
 \hrule
 \par
\vskip 21pt plus 6pt minus 3pt }

\label{st\stat}

\def\tit{\ }

\def\aut{\ }
\def\auf{\ }

\def\leftkol{\ } % ENGLISH ABSTRACTS}

\def\rightkol{\ } %АВТОРСКИЙ УКАЗАТЕЛЬ ЗА 2018 г.} %ENGLISH ABSTRACTS}

\titele{\tit}{\aut}{\auf}{\leftkol}{\rightkol}
\addcontentsline{toc}{subsection}{\textrm\textbf Авторский указатель за 2018 г.}

\vspace*{-12pt}
\vspace*{-36pt}

\noindent
{\tabcolsep=3pt
\begin{tabular}{p{397pt}cc}
&\textbf{Вып.} & \textbf{Стр.}\\[6pt]
\Avtors{Агаларов~Я.\,М.} Оптимизация объема буферной памяти узла коммутации при схеме\linebreak
\\[-12pt]
\hspace*{23pt}полного разделения памяти&4&25--32\\
\Avtors{Агасандян~Г.\,А.} Континуальный критерий VaR на сценарных рынках&1&31--39\\
\Avtors{Алешин~И.\,С.} О формальной постановке задач поиска сгущений в разреженных булевых\linebreak
\\[-12pt]
\hspace*{23pt}матрицах&1&40--48\\
\Avtors{Арутюнов~Е.\,Н., Кудрявцев~А.\,А., Титова~А.\,И.} Гамма-вейбулловский случай априорных\linebreak
\\[-12pt]
\hspace*{23pt}распределений в~байесовских моделях массового обслуживания&4&92--95\\
\Avtors{Атаева~О.\,М., Серебряков~В.\,А.} Онтология цифровой семантической библиотеки LibMeta&1&\hphantom{1}2--10\\
\Avtors{Басок~Б.\,М., Захаров~В.\,Н., Френкель~С.\,Л.} Использование вероятностной модели вычислений для тестирования одного класса готовых к~использованию программных\linebreak
\\[-12pt]
\hspace*{23pt}компонентов локальных и~сетевых систем&4&44--51\\
\Avtors{Батенков~А.\,А., Маньяков~Ю.\,А., Гасилов~А.\,В., Яковлев~О.\,А.} Математическая модель\linebreak
\\[-12pt]
\hspace*{23pt}оптимальной триангуляции&2&69--74\\
\Avtors{Бахтеев~О.\,Ю.} см.~Огальцов~А.\,В.&&\\
\Avtors{Бахтеев~О.\,Ю.} см.~Смердов~А.\,Н.&&\\
\Avtors{Борисов~А.\,В.} Фильтрация состояний марковских скачкообразных процессов по дискре-\linebreak
\\[-12pt]
\hspace*{23pt}тизованным наблюдениям&3&115--121\\
\Avtors{Босов~А.\,В., Игнатов~А.\,Н., Наумов~А.\,В.} Модель передвижения поездов и маневровых локомотивов на железнодорожной станции в приложении к оценке и анализу\linebreak
\\[-12pt]
\hspace*{23pt}вероятности бокового столкновения&3&107--114\\
\Avtors{Босов~А.\,В., Стефанович~А.\,И.} Управление выходом стохастической дифференциальной системы по квадратичному критерию. I.~Оптимальное решение методом динами-\linebreak
\\[-12pt]
\hspace*{23pt}ческого программирования&3&\hphantom{1}99--106\\
\Avtors{Бунтман~Н.\,В., Гончаров~А.\,А., Зацман~И.\,М., Нуриев~В.\,А.} Количественный анализ\linebreak
\\[-12pt]
\hspace*{23pt}результатов машинного перевода с~использованием надкорпусных баз данных&4&\hphantom{1}96--105\\
\Avtors{Бунтман~Н.\,В.} см.~Нуриев~В.\,А.&&\\
\Avtors{Быковец~Е.\,В., Лаврентьев~В.\,В., Назаров~Л.\,В.} Вероятностная модель влияния книги\linebreak
\\[-12pt]
\hspace*{23pt}заказов на процесс цены&2&29--34\\
\Avtors{Васильева~С.\,Н., Кан~Ю.\,С.} Алгоритм визуализации плоского ядра вероятностной меры&2&60--68\\
\Avtors{Виноградов~Д.\,В.} Учет предварительных оценок скорости порождения сходств спарива-\linebreak
\\[-12pt]
\hspace*{23pt}ющей цепью Маркова&1&49--54\\
\Avtors{Вышинский~Л.\,Л., Флеров~Ю.\,А., Широков~Н.\,И.} Автоматизированная система весового\linebreak
\\[-12pt]
\hspace*{23pt}проектирования самолетов&1&18--30\\
\Avtors{Гайдамака~Ю.\,В.} см.~Горбунова~А.\,В.&&\\
\Avtors{Гайдамака~Ю.\,В.} см.~Самуйлов~К.\,Е.&&\\
\Avtors{Гасилов~А.\,В.} см.~Батенков~А.\,А.,&&\\
\Avtors{Гончаров~А.\,А.} см.~Бунтман~Н.\,В.&&\\
\Avtors{Горбунова~А.\,В., Наумов~В.\,А., Гайдамака~Ю.\,В., Самуйлов~К.\,Е.} Ресурсные системы\linebreak
\\[-12pt]
\hspace*{23pt}массового обслуживания как модели беспроводных систем связи&3&48--55\\
\Avtors{Горшенин~А.\,К.} Зашумление данных конечными смесями нормальных и гамма-рас\-пре-\linebreak
\\[-12pt]
\hspace*{23pt}де\-ле\-ний с применением к задаче округления наблюдений&3&28--34\\
\Avtors{Горшенин~А.\,К.} Развитие сервисов цифровых платформ для преодоления нефинансовых\linebreak
\\[-12pt]
\hspace*{23pt}барьеров&4&106--112\\
\Avtors{Горшенин~А.\,К., Королев~В.\,Ю.} Определение экстремальности объемов осадков на основе\linebreak
\\[-12pt]
\hspace*{23pt}модифицированного метода превышения порогового значения&4&16--24\\
\Avtors{Горшенин~А.\,К.} см.~Королев~В.\,Ю.&&\\
\end{tabular}
}

\pagebreak

\def\leftkol{АВТОРСКИЙ УКАЗАТЕЛЬ ЗА 2018 г.} % ENGLISH ABSTRACTS}

\def\rightkol{АВТОРСКИЙ УКАЗАТЕЛЬ ЗА 2018 г.} %ENGLISH ABSTRACTS}

%\thispagestyle{myheadings}
\def\leftfootline{\small{\textbf{\thepage}
\hfill ИНФОРМАТИКА И ЕЁ ПРИМЕНЕНИЯ\ \ \ том~12\ \ \ выпуск~4\ \ \ 2018}
}%
 \def\rightfootline{\small{ИНФОРМАТИКА И ЕЁ ПРИМЕНЕНИЯ\ \ \ том~12\ \ \ выпуск~4\ \ \ 2018
 \hfill \textbf{\thepage}}}


\noindent
{\tabcolsep=3pt
\begin{tabular}{p{394pt}cc}
&\textbf{Вып.} & \textbf{Стр.}\\[3pt]
\Avtors{Грушо~А.\,А., Грушо~Н.\,А., Забежайло~М.\,И., Смирнов~Д.\,В., Тимонина~Е.\,Е.} Параметриза-\linebreak
\\[-12pt]
\hspace*{23pt}ция в прикладных задачах поиска эмпирических причин&3&62--66\\
\Avtors{Грушо~А.\,А., Грушо~Н.\,А., Левыкин~М.\,В., Тимонина~Е.\,Е.} Методы идентификации захвата хоста в~распределенной информационно-вычислительной сис\-те\-ме, защищенной\linebreak
\\[-12pt]
\hspace*{23pt}с~по\-мощью метаданных&4&39--43\\
\Avtors{Грушо~А.\,А., Забежайло~М.\,И., Зацаринный~А.\,А., Тимонина~Е.\,Е.} О некоторых возможностях управления ресурсами при организации проактивного противодействия\linebreak
\\[-12pt]
\hspace*{23pt}компьютерным атакам&1&62--70\\
\Avtors{Грушо~А.\,А., Тимонина~Е.\,Е., Шоргин~С.\,Я.} Иерархический метод порождения метадан-\linebreak
\\[-12pt]
\hspace*{23pt}ных для управления сетевыми соединениями&2&44--49\\
\Avtors{Грушо~Н.\,А.} см.~Грушо~А.\,А.&&\\
\Avtors{Грушо~Н.\,А.} см.~Грушо~А.\,А.&&\\
\Avtors{Дорофеева~А.\,В.} см.~Королев~В.\,Ю.&&\\
\Avtors{Егоров~А.\,Ю.} см.~Шнурков~П.\,В.&&\\
\Avtors{Егоров~А.\,Ю.} см.~Шнурков~П.\,В.&&\\
\Avtors{Жуков~Д.\,О., Хватова~Т.\,Ю., Лесько~С.\,А., Зальцман~А.\,Д.} Влияние плотности связей на кластеризацию и порог перколяции при распространении информации в~со\-ци\-аль-\linebreak
\\[-12pt]
\hspace*{23pt}ных сетях&2&90--97\\
\Avtors{Забежайло~М.\,И.} см.~Грушо~А.\,А.&&\\
\Avtors{Забежайло~М.\,И.} см.~Грушо~А.\,А.&&\\
\Avtors{Зальцман~А.\,Д.} см.~Жуков~Д.\,О.&&\\
\Avtors{Захаров~В.\,Н.} см.~Басок~Б.\,М.&&\\
\Avtors{Захаров~В.\,Н.} см.~Шанин~И.\,А.&&\\
\Avtors{Зацаринный~А.\,А., Сучков~А.\,П.} Система ситуационного управления как мультисервисная\linebreak
\\[-12pt]
\hspace*{23pt}технология в облачной среде&1&78--88\\
\Avtors{Зацаринный~А.\,А.} см.~Грушо~А.\,А.&&\\
\Avtors{Зацман~И.\,М.} Имплицированные знания: основания и технологии извлечения&3&74--82\\
\Avtors{Зацман~И.\,М.} см.~Бунтман~Н.\,В.&&\\
\Avtors{Зейфман~А.\,И.} см.~Королев~В.\,Ю.&&\\
\Avtors{Зубарев~Д.\,В.} см.~Соченков~И.\,В.&&\\
\Avtors{Игнатов~А.\,Н.} см.~Босов~А.\,В.&&\\
\Avtors{Инькова~О.\,Ю., Кружков~М.\,Г.} Статистический анализ лингвоспецифичности коннек-\linebreak
\\[-12pt]
\hspace*{23pt}торов (на материале параллельных корпусов)&3&83--90\\
\Avtors{Инькова~О.\,Ю.} см.~Нуриев~В.\,А.&&\\
\Avtors{Кан~Ю.\,С.} см.~Васильева~С.\,Н.&&\\
\Avtors{Ковалёв~С.\,П.} Теория категорий как математическая прагматика модельно-ори\-ен\-ти-\linebreak
\\[-12pt]
\hspace*{23pt}ро\-ван\-ной системной инженерии&1&\hphantom{1}95--104\\
\Avtors{Козеренко~Е.\,Б., Кузнецов~К.\,И., Романов~Д.\,А.} Семантическая обработка неструктури-\linebreak
\\[-12pt]
\hspace*{23pt}рованных текстовых данных на основе лингвистического процессора PullEnti&3&91--98\\
\Avtors{Кондранин~Е.\,С., Ушаков~В.\,Г.} Система обслуживания с~относительным приоритетом\linebreak
\\[-12pt]
\hspace*{23pt}и~профилактиками прибора&4&33--38\\
\Avtors{Коновалов~М.\,Г., Разумчик~Р.\,В.} Сравнение двух механизмов активного управления\linebreak
\\[-12pt]
\hspace*{23pt}очередью в~системе $M/D/1/N$&4&\hphantom{1}9--15\\
\Avtors{Коновалов~М.\,Г., Разумчик~Р.\,В.} Управление случайным блужданием с эталонным\linebreak
\\[-12pt]
\hspace*{23pt}стационарным распределением&3&\hphantom{1}2--13\\
\Avtors{Королев~В.\,Ю., Горшенин~А.\,К., Зейфман~А.\,И.} Новые представления обобщенного\linebreak
\\[-12pt]
\hspace*{23pt}распределения Миттаг-Леффлера в~виде смесей и~их приложения&4&75--85\\
\Avtors{Королев~В.\,Ю., Дорофеева~А.\,В.} О~неравномерных оценках точности нормальной аппроксимации для распределений некоторых случайных сумм при ослабленных\linebreak
\\[-12pt]
\hspace*{23pt}моментных условиях&4&86--91\\
\Avtors{Королев~В.\,Ю.} см.~Горшенин~А.\,К.&&\\
\Avtors{Кривенко~М.\,П.}\ Обучаемая классификация данных с учетом анализа главных компонент&3&56--61\\
\Avtors{Кривенко~М.\,П.}\ Реконструкция осей главных компонент&1&71--77\\
\Avtors{Кружков~М.\,Г.} см.~Инькова~О.\,Ю.&&\\
\end{tabular}
}

\pagebreak

\def\leftkol{АВТОРСКИЙ УКАЗАТЕЛЬ ЗА 2018 г.} % ENGLISH ABSTRACTS}

\def\rightkol{АВТОРСКИЙ УКАЗАТЕЛЬ ЗА 2018 г.} %ENGLISH ABSTRACTS}

%\thispagestyle{myheadings}
\def\leftfootline{\small{\textbf{\thepage}
\hfill ИНФОРМАТИКА И ЕЁ ПРИМЕНЕНИЯ\ \ \ том~12\ \ \ выпуск~4\ \ \ 2018}
}%
 \def\rightfootline{\small{ИНФОРМАТИКА И ЕЁ ПРИМЕНЕНИЯ\ \ \ том~12\ \ \ выпуск~4\ \ \ 2018
 \hfill \textbf{\thepage}}}


\noindent
{\tabcolsep=3pt
\begin{tabular}{p{394pt}cc}
&\textbf{Вып.} & \textbf{Стр.}\\[3pt]
\Avtors{Кудрявцев~А.\,А.} Байесовские модели баланса&3&18--27\\
\Avtors{Кудрявцев~А.\,А., Шестаков~О.\,В.} Байесовские модели тестирования больших групп\linebreak
\\[-12pt]
\hspace*{23pt}обслуживающих приборов&1&105--108\\
\Avtors{Кудрявцев~А.\,А., Шестаков О.\,В.} Минимизация ошибок вычисления вейвлет-ко\-эф\-фи-\linebreak
\\[-12pt]
\hspace*{23pt}ци\-ен\-тов при решении обратных задач&2&17--23\\
\Avtors{Кудрявцев~А.\,А.} см.~Арутюнов~Е.\,Н.&&\\
\Avtors{Кузнецов~К.\,И.} см.~Козеренко~Е.\,Б.&&\\
\Avtors{Лаврентьев~В.\,В.} см.~Быковец~Е.\,В.&&\\
\Avtors{Лебедев~А.\,В.} Максимальные ветвящиеся процессы в случайной среде&2&35--43\\
\Avtors{Левыкин~М.\,В.} см.~Грушо~А.\,А.&&\\
\Avtors{Лери~М.\,М., Павлов~Ю.\,Л.} Об устойчивости конфигурационных графов в случайной\linebreak
\\[-12pt]
\hspace*{23pt}среде&2&\hphantom{1}2--10\\
\Avtors{Лесько~С.\,А.} см.~Жуков~Д.\,О.&&\\
\Avtors{Логачев~О.\,А.} Теоретико-информационная характеризация совершенно уравновешен-\linebreak
\\[-12pt]
\hspace*{23pt}ных функций&4&70--74\\
\Avtors{Малашенко~Ю.\,Е., Назарова~И.\,А., Новикова~Н.\,М.} Анализ разрезных повреждений\linebreak
\\[-12pt]
\hspace*{23pt}в~многополюсных сетях&3&35--41\\
\Avtors{Малашенко~Ю.\,Е., Назарова~И.\,А., Новикова~Н.\,М.} Диаграммы уязвимости потоковых\linebreak
\\[-12pt]
\hspace*{23pt}сетевых систем&1&11--17\\
\Avtors{Маньяков~Ю.\,А.} см.~Батенков~А.\,А.&&\\
\Avtors{Мирзабеков~Я.\,М., Шихиев~Ш.\,Б.} Дискретный анализ в синтаксическом анализе&2&\hphantom{1}98--104\\
\Avtors{Мистрюков~А.\,В., Ушаков~В.\,Г.} Достаточные условия эргодичности приоритетных\linebreak
\\[-12pt]
\hspace*{23pt}систем массового обслуживания&2&24--28\\
\Avtors{Назаров~Л.\,В.} см.~Быковец~Е.\,В.&&\\
\Avtors{Назарова~И.\,А.} см.~Малашенко~Ю.\,Е.&&\\
\Avtors{Назарова~И.\,А.} см.~Малашенко~Ю.\,Е.&&\\
\Avtors{Наумов~А.\,В.} см.~Босов~А.\,В.&&\\
\Avtors{Наумов~В.\,А.} см.~Горбунова~А.\,В.&&\\
\Avtors{Наумов~В.\,А.} см.~Сопин~Э.\,С.&&\\
\Avtors{Новикова~Н.\,М.} см.~Малашенко~Ю.\,Е.&&\\
\Avtors{Новикова~Н.\,М.} см.~Малашенко~Ю.\,Е.&&\\
\Avtors{Нуриев~В.\,А., Бунтман~Н.\,В., Инькова~О.\,Ю.} Ошибки и~неточности машинного перевода\linebreak
\\[-12pt]
\hspace*{23pt}русских коннекторов на~французский язык&2&105--113\\
\Avtors{Нуриев~В.\,А.} см.~Бунтман~Н.\,В.&&\\
\Avtors{Огальцов~А.\,В., Бахтеев~О.\,Ю.} Автоматическое извлечение метаданных из научных\linebreak
\\[-12pt]
\hspace*{23pt}PDF-документов&2&75--82\\
\Avtors{Павлов~Ю.\,Л.} см.~Лери~М.\,М.&&\\
\Avtors{Разумчик~Р.\,В.} см.~Коновалов~М.\,Г.&&\\
\Avtors{Разумчик~Р.\,В.} см.~Коновалов~М.\,Г.&&\\
\Avtors{Романов~Д.\,А.} см.~Козеренко~Е.\,Б.&&\\
\Avtors{Самуйлов~К.\,Е., Гайдамака~Ю.\,В., Шоргин~С.\,Я.} Применение моделей случайного\linebreak
\\[-12pt]
\hspace*{23pt}блуждания при моделировании перемещения устройств в~беспроводной сети&4&2--8\\
\Avtors{Самуйлов~К.\,Е.} см.~Горбунова~А.\,В.&&\\
\Avtors{Самуйлов~К.\,Е.} см.~Сопин~Э.\,С.&&\\
\Avtors{Серебряков~В.\,А.} см.~Атаева~О.\,М.&&\\
\Avtors{Синицын~И.\,Н.} Метод интерполяционного аналитического моделирования одномерных\linebreak
\\[-12pt]
\hspace*{23pt}распределений в стохастических системах&1&55--61\\
\Avtors{Смердов~А.\,Н., Бахтеев~О.\,Ю., Стрижов~В.\,В.} Выбор оптимальной модели рекуррентной\linebreak
\\[-12pt]
\hspace*{23pt}сети в~задачах поиска парафраза&4&63--69\\
\Avtors{Смирнов~Д.\,В.} см.~Грушо~А.\,А.&&\\
\Avtors{Сопин~Э.\,С., Наумов~В.\,А., Самуйлов~К.\,Е.} Об инвариантности стационарного распределения системы массового обслуживания с ограниченными ресурсами и~с~ин\-тен-\linebreak
\\[-12pt]
\hspace*{23pt}сив\-ностями поступления и~обслуживания, зависящими от состояния системы&3&42--47\\
\Avtors{Соченков~И.\,В., Зубарев~Д.\,В., Тихомиров~И.\,А.} Эксплоративный патентный поиск&1&89--94\\
\end{tabular}
}

\pagebreak

\def\leftkol{АВТОРСКИЙ УКАЗАТЕЛЬ ЗА 2018 г.} % ENGLISH ABSTRACTS}

\def\rightkol{АВТОРСКИЙ УКАЗАТЕЛЬ ЗА 2018 г.} %ENGLISH ABSTRACTS}

%\thispagestyle{myheadings}
\def\leftfootline{\small{\textbf{\thepage}
\hfill ИНФОРМАТИКА И ЕЁ ПРИМЕНЕНИЯ\ \ \ том~12\ \ \ выпуск~4\ \ \ 2018}
}%
 \def\rightfootline{\small{ИНФОРМАТИКА И ЕЁ ПРИМЕНЕНИЯ\ \ \ том~12\ \ \ выпуск~4\ \ \ 2018
 \hfill \textbf{\thepage}}}


\noindent
{\tabcolsep=3pt
\begin{tabular}{p{394pt}cc}
&\textbf{Вып.} & \textbf{Стр.}\\[3pt]
\Avtors{Стефанович~А.\,И.} см.~Босов~А.\,В.&&\\
\Avtors{Стрижов~В.\,В.} см.~Смердов~А.\,Н.&&\\
\Avtors{Ступников~С.\,А.} см.~Шанин~И.\,А.&&\\
\Avtors{Сурина~А.\,А.} см.~Тырсин~А.\,Н.&&\\
\Avtors{Сучков~А.\,П.} см.~Зацаринный~А.\,А.&&\\
\Avtors{Сюнтюренко~О.\,В.} Финансирование фундаментальных исследований: концептуальный облик системы поддержки принятия решений с использованием методов\linebreak
\\[-12pt]
\hspace*{23pt}наукометрии и анализа данных&1&118--127\\
\Avtors{Тимонина~Е.\,Е.} см.~Грушо~А.\,А.&&\\
\Avtors{Тимонина~Е.\,Е.} см.~Грушо~А.\,А.&&\\
\Avtors{Тимонина~Е.\,Е.} см.~Грушо~А.\,А.&&\\
\Avtors{Тимонина~Е.\,Е.} см.~Грушо~А.\,А.&&\\
\Avtors{Титова~А.\,И.} см.~Арутюнов~Е.\,Н.&&\\
\Avtors{Тихомиров~И.\,А.} см.~Соченков~И.\,В.&&\\
\Avtors{Тырсин~А.\,Н., Сурина~А.\,А.} Модели управления риском в гауссовских стохастических\linebreak
\\[-12pt]
\hspace*{23pt}системах&2&50--59\\
\Avtors{Ушаков~В.\,Г.} см.~Кондранин~Е.\,С.&&\\
\Avtors{Ушаков~В.\,Г.} см.~Мистрюков~А.\,В.&&\\
\Avtors{Флеров~Ю.\,А.} см.~Вышинский~Л.\,Л.&&\\
\Avtors{Френкель~С.\,Л., Ханкин~Д.} Непрерывные обновления маршрута в~SDN с~использованием\linebreak
\\[-12pt]
\hspace*{23pt}проверки соответствия качеству обслуживания&4&52--62\\
\Avtors{Френкель~С.\,Л.} см.~Басок~Б.\,М.&&\\
\Avtors{Ханкин~Д.} см.~Френкель~С.\,Л.&&\\
\Avtors{Хватова~Т.\,Ю.} см.~Жуков~Д.\,О.&&\\
\Avtors{Шанин~И.\,А., Ступников~С.\,А., Захаров~В.\,Н.} Методы и средства обнаружения нештатных\linebreak
\\[-12pt]
\hspace*{23pt}ситуаций, возникающих на элементах жилищно-коммунальной инфраструктуры&3&67--73\\
\Avtors{Шестаков~О.\,В.} Несмещенная оценка риска стабилизированной жесткой пороговой\linebreak
\\[-12pt]
\hspace*{23pt}обработки в модели с долгосрочной зависимостью&2&11--16\\
\Avtors{Шестаков~О.\,В.} Среднеквадратичный риск пороговой обработки при случайном объеме\linebreak
\\[-12pt]
\hspace*{23pt}выборки&3&14--17\\
\Avtors{Шестаков~О.\,В.} см.~Кудрявцев~А.\,А.&&\\
\Avtors{Шестаков~О.\,В.} см.~Кудрявцев~А.\,А.&&\\
\Avtors{Широков~Н.\,И.} см.~Вышинский~Л.\,Л.&&\\
\Avtors{Шихиев~Ш.\,Б.} см.~Мирзабеков~Я.\,М.&&\\
\Avtors{Шнурков~П.\,В., Егоров~А.\,Ю.} Разработка и предварительное исследование стохастической полумарковской модели управления запасом непрерывного продукта при\linebreak
\\[-12pt]
\hspace*{23pt}постоянно происходящем потреблении&1&109--117\\
\Avtors{Шнурков~П.\,В., Егоров~А.\,Ю.} Решение проблемы оптимального управления запасом непрерывного продукта при постоянно происходящем потреблении в стохастической\linebreak
\\[-12pt]
\hspace*{23pt}полумарковской модели&2&83--89\\
\Avtors{Шоргин~С.\,Я.} см.~Грушо~А.\,А.&&\\
\Avtors{Шоргин~С.\,Я.} см.~Самуйлов~К.\,Е.&&\\
\Avtors{Яковлев~О.\,А.} см.~Батенков~А.\,А.&&\\
\end{tabular}
}

%\thispagestyle{myheadings}
\def\leftfootline{\small{\textbf{\thepage}
\hfill ИНФОРМАТИКА И ЕЁ ПРИМЕНЕНИЯ\ \ \ том~12\ \ \ выпуск~4\ \ \ 2018}
}%
 \def\rightfootline{\small{ИНФОРМАТИКА И ЕЁ ПРИМЕНЕНИЯ\ \ \ том~12\ \ \ выпуск~4\ \ \ 2018
 \hfill \textbf{\thepage}}}

 \label{end\stat}

\newpage

%Информатика и её применения
%Том 12   Выпуск 1-4   Год 2018

\def\stat{cont-e}
{%\hrule\par
%\vskip 7pt % 7pt
\raggedleft\Large \bf%\baselineskip=3.2ex
2\,0\,1\,8\ \ A\,U\,T\,H\,O\,R\ \ I\,N\,D\,E\,X \vskip 17pt
 \hrule
 \par
\vskip 21pt plus 6pt minus 3pt }

\label{st\stat}

\def\tit{\ }

\def\aut{\ }
\def\auf{\ }

\def\leftkol{\ } %2018 AUTHOR INDEX} % ENGLISH ABSTRACTS}

\def\rightkol{\ } %2018 AUTHOR INDEX} %ENGLISH ABSTRACTS}

\titele{\tit}{\aut}{\auf}{\leftkol}{\rightkol}
\addcontentsline{toc}{subsection}{\textrm\textbf 2018 Author Index}

\def\leftfootline{\small{\textbf{\thepage}
\hfill INFORMATIKA I EE PRIMENENIYA~--- INFORMATICS AND APPLICATIONS\ \ \ 2018\
\ \ volume~12\ \ \ issue\ 4}
}%
 \def\rightfootline{\small{INFORMATIKA I EE PRIMENENIYA~--- INFORMATICS AND APPLICATIONS\ \ \ 2018\ \ \ volume~12\ \ \ issue\ 4
\hfill \textbf{\thepage}}}

\vspace*{-12pt}
\vspace*{-18pt}

\noindent
{\tabcolsep=3pt
\begin{tabular}{p{396pt}cc}
&\textbf{Issue} & \textbf{Page}\\[6pt]
\Avtors{Agalarov~Yа.\,M.} Optimization of buffer memory size of switching node in mode of full memory\linebreak
\\[-12pt]
\hspace*{23pt}sharing&4&25--32\\
\Avtors{Agasandyan~G.\,A.} Continuous VaR-criterion in scenario markets&1&31--39\\
\Avtors{Aleshin~I.\,S.} On the formalization of tasks searching dense submatrices in boolean sparse\linebreak
\\[-12pt]
\hspace*{23pt}matrices&1&40--48\\
\Avtors{Arutyunov~E.\,N., Kudryavtsev~A.\,A., and~Titova~A.\,I.} Gamma-Weibull \textit{a~priori} distributions\linebreak
\\[-12pt]
\hspace*{23pt}in~Bayesian queuing models&4&92--95\\
\Avtors{Ataeva~O.\,M.} see~Serebryakov~V.\,A.&&\\
\Avtors{Bakhteev~O.\,Y.} see~Ogaltsov~A.\,V.&&\\
\Avtors{Bakhteev~O.\,Y.} see~Smerdov~A.\,N.&&\\
\Avtors{Basok~B.\,M., Zakharov~V.\,N., and~Frenkel~S.\,L.} Using a probabilistic calculation model to test\linebreak
\\[-12pt]
\hspace*{23pt}one class of ready-to-use software components of local and network systems&4&44--51\\
\Avtors{Batenkov~A.\,A., Maniakov Yu.\,A., Gasilov A.\,V., and Yakovlev O.\,A.} Mathematical model\linebreak
\\[-12pt]
\hspace*{23pt}of~optimal triangulation&2&69--74\\
\Avtors{Borisov~A.\,V.} Filtering of Markov jump processes by discretized observations&3&115--121\\
\Avtors{Bosov~A.\,V., Ignatov~A.\,N., and Naumov~A.\,V.} Model of transportation of trains and shunting\linebreak
\\[-12pt]
\hspace*{23pt}locomotives at a railway station for evaluation and analysis of side-collision probability&3&107--114\\
\Avtors{Bosov~A.\,V.\ and Stefanovich~A.\,I.} Stochastic differential system output control by the quadratic\linebreak
\\[-12pt]
\hspace*{23pt}criterion. I.~Dynamic programming optimal solution&3&\hphantom{1}99--106\\
\Avtors{Buntman~N.\,V., Goncharov~A.\,A., Zatsman~I.\,M., and~Nuriev~V.\,A.} Using supracorpora databases\linebreak
\\[-12pt]
\hspace*{23pt}for quantitative analysis of machine translations&4&\hphantom{1}96--105\\
\Avtors{Buntman~N.\,V.} see~Nuriev~V.\,A., &&\\
\Avtors{Bykovets~E.\,V.} see~Nazarov~L.\,V.&&\\
\Avtors{Dorofeeva~A.\,V.} see~Korolev~V.\,Yu.&&\\
\Avtors{Egorov~A.\,Y.} see~Shnurkov~P.\,V.&&\\
\Avtors{Egorov~A.\,Y.} see~Shnurkov~P.\,V.&&\\
\Avtors{Flerov~Yu.\,A.} see~Vyshinsky~L.\,L.&&\\
\Avtors{Frenkel~S.\,L.\ and Khankin~D.} Seamless route updates in software-defined networking via quality\linebreak
\\[-12pt]
\hspace*{23pt}of~service compliance verification &4&52--62\\
\Avtors{Frenkel~S.\,L.} see~Basok~B.\,M.&&\\
\Avtors{Gaidamaka~Yu.\,V.} see~Gorbunova~A.\,V.&&\\
\Avtors{Gaidamaka~Yu.\,V.} see~Samouylov~K.\,E.&&\\
\Avtors{Gasilov A.\,V.} see~Batenkov~A.\,A.&&\\
\Avtors{Goncharov~A.\,A.} see~Buntman~N.\,V.&&\\
\Avtors{Gorbunova~A.\,V., Naumov~V.\,A., Gaidamaka~Yu.\,V., and Samouylov~K.\,E.} Resource queuing\linebreak
\\[-12pt]
\hspace*{23pt}systems as models of wireless communication systems&3&48--55\\
\Avtors{Gorshenin~A.\,K.} Data noising by finite normal and gamma mixtures with application to~the~prob-\linebreak
\\[-12pt]
\hspace*{23pt}lem of rounded observations&3&28--34\\
\Avtors{Gorshenin~A.\,K.} Development of services of digital platforms to overcome nonfinancial barriers&4&106--112\\
\Avtors{Gorshenin~A.\,K.\ and~Korolev~V.\,Yu.} Determining the extremes of precipitation volumes based\linebreak
\\[-12pt]
\hspace*{23pt}on~the~modified ``Peaks over Threshold'' method&4&16--24\\
\Avtors{Gorshenin~A.\,K.} see~Korolev~V.\,Yu.&&\\
\Avtors{Grusho~A.\,A., Grusho~N.\,A., Levykin~M.\,V., and~Timonina~E.\,E.} Methods of identification of host\linebreak
\\[-12pt]
\hspace*{23pt}capture in a distributed information system which is protected on the basis of meta data&4&39--43\\
\Avtors{Grusho~A.\,A., Grusho~N.\,A., Zabezhailo~M.\,I., Smirnov~D.\,V., and Timonina~E.\,E.} Parametrization\linebreak
\\[-12pt]
\hspace*{23pt}in applied problems of search of empirical reasons&3&62--66\\
\end{tabular}
}
\pagebreak

\def\leftfootline{\small{\textbf{\thepage}
\hfill INFORMATIKA I EE PRIMENENIYA~--- INFORMATICS AND APPLICATIONS\ \ \ 2018\
\ \ volume~12\ \ \ issue\ 4}
}%
 \def\rightfootline{\small{INFORMATIKA I EE PRIMENENIYA~---
INFORMATICS AND APPLICATIONS\ \ \ 2018\ \ \ volume~12\ \ \ issue\ 4
\hfill \textbf{\thepage}}}

\def\leftkol{2018 AUTHOR INDEX} % ENGLISH ABSTRACTS}

\def\rightkol{2018 AUTHOR INDEX} %ENGLISH ABSTRACTS}


\noindent
{\tabcolsep=3pt
\begin{tabular}{p{395.48108pt}cc}
&\textbf{Issue} & \textbf{Page}\\[6pt]
\Avtors{Grusho~A.\,A., Timonina~E.\,E., and Shorgin~S.\,Ya.} Hierarchical method of meta data generation\linebreak
\\[-12pt]
\hspace*{23pt}for control of network connections&2&44--49\\
\Avtors{Grusho~A.\,A., Zabezhailo~M.\,I., Zatsarinny~A.\,A., and Timonina~E.\,E.} On some possibilities\linebreak
\\[-12pt]
\hspace*{23pt}of~resource management for organizing active counteraction to computer attacks&1&62--70\\
\Avtors{Grusho~N.\,A.} see~Grusho~A.\,A.&&\\
\Avtors{Grusho~N.\,A.} see~Grusho~A.\,A.&&\\
\Avtors{Ignatov~A.\,N.} see~Bosov~A.\,V.&&\\
\Avtors{Inkova~O.\,Yu.\ and Kruzhkov~M.\,G.} Statistical analysis of language specificity of connectives\linebreak
\\[-12pt]
\hspace*{23pt}based on parallel texts&3&83--90\\
\Avtors{Inkova~O.\,Yu.} see~Nuriev~V.\,A., &&\\
\Avtors{Kan~Yu.\,S.} see~Vasil'eva~S.\,N.&&\\
\Avtors{Khankin~D.} see~Frenkel~S.\,L.&&\\
\Avtors{Khvatova~T.\,Yu.} see~Zhukov~D.\,O.&&\\
\Avtors{Kondranin~E.\,S.\ and~Ushakov~V.\,G.} A~head of the line priority queue with working vacations&4&33--38\\
\Avtors{Konovalov~M.\,G.\ and Razumchik~R.\,V.} Comparison of two active queue management schemes\linebreak
\\[-12pt]
\hspace*{23pt}through the $M/D/1/N$ queue&4&\hphantom{1}9--15\\
\Avtors{Konovalov~M.\,G.\ and Razumchik~R.\,V.} Finding control policy for one discrete-time Markov\linebreak
\\[-12pt]
\hspace*{23pt}chain on [0,1] with a given invariant measure&3&\hphantom{1}2--13\\
\Avtors{Korolev~V.\,Yu.\ and~Dorofeeva~A.\,V.} On nonuniform estimates of accuracy of normal approxima-\linebreak
\\[-12pt]
\hspace*{23pt}tion for distributions of some random sums under relaxed moment conditions&4&86--91\\
\Avtors{Korolev~V.\,Yu., Gorshenin~A.\,K., and~Zeifman~A.\,I. } New mixture representations of~the~general-\linebreak
\\[-12pt]
\hspace*{23pt}ized Mittag-Leffler distribution and their applications&4&75--85\\
\Avtors{Korolev~V.\,Yu.} see~Gorshenin~A.\,K.&&\\
\Avtors{Kovalyov~S.\,P.} Category theory as a mathematical pragmatics of model-based systems engineer-\linebreak
\\[-12pt]
\hspace*{23pt}ing&1&\hphantom{1}95--104\\
\Avtors{Kozerenko~E.\,B., Kuznetsov~K.\,I., and Romanov~D.\,A.} Semantic processing of unstructured\linebreak
\\[-12pt]
\hspace*{23pt}textual data based on the linguistic processor PullEnti&3&91--98\\
\Avtors{Krivenko~M.\,P.} Principal axes reconstruction&1&71--77\\
\Avtors{Krivenko~M.\,P.} Supervised learning classification of data taking into account principal compo-\linebreak
\\[-12pt]
\hspace*{23pt}nent analysis&3&56--61\\
\Avtors{Kruzhkov~M.\,G.} see~Inkova~O.\,Yu.&&\\
\Avtors{Kudryavtsev~A.\,A.} Bayesian balance models&3&18--27\\
\Avtors{Kudryavtsev~A.\,A.\ and Shestakov~O.\,V.} Bayesian models for testing large groups of service devices&1&105--108\\
\Avtors{Kudryavtsev~A.\,A.\ and Shestakov~O.\,V.} Minimization of errors of calculating wavelet coefficients\linebreak
\\[-12pt]
\hspace*{23pt}while solving inverse problems&2&17--23\\
\Avtors{Kudryavtsev~A.\,A.} see~Arutyunov~E.\,N.&&\\
\Avtors{Kuznetsov~K.\,I.} see~Kozerenko~E.\,B.&&\\
\Avtors{Lavrentyev~V.\,V.} see~Nazarov~L.\,V.&&\\
\Avtors{Lebedev~A.\,V.} Maximal branching processes in random environment&2&35--43\\
\Avtors{Leri~M.\,M.\ and Pavlov~Yu.\,L.} On the robustness of configuration graphs in a random environment&2&\hphantom{1}2--10\\
\Avtors{Lesko~S.\,A.} see~Zhukov~D.\,O.&&\\
\Avtors{Levykin~M.\,V.} see~Grusho~A.\,A.&&\\
\Avtors{Logachev~O.\,A.} An information based criterion for perfectly balanced functions&4&70--74\\
\Avtors{Malashenko~Yu.\,E., Nazarova~I.\,A., and Novikova~N.\,M.} Analysis of cutting damages to multipolar\linebreak
\\[-12pt]
\hspace*{23pt}networks&3&35--41\\
\Avtors{Malashenko~Yu.\,E., Nazarova~I.\,A., and Novikova~N.\,M.} Diagrams of the functional vulnerability\linebreak
\\[-12pt]
\hspace*{23pt}of flow network systems&1&11--17\\
\Avtors{Maniakov Yu.\,A.} see~Batenkov~A.\,A.&&\\
\Avtors{Mirzabekov~Ya.\,M.\ and Shihiev~Sh.\,B.} Discrete analysis in parsing&2&\hphantom{1}98--104\\
\Avtors{Mistryukov~A.\,V.\ and Ushakov~V.\,G.} Sufficient ergodicity conditions for priority queues&2&24--28\\
\Avtors{Naumov~A.\,V.} see~Bosov~A.\,V.&&\\
\Avtors{Naumov~V.\,A.} see~Gorbunova~A.\,V.&&\\
\Avtors{Naumov~V.\,A.} see~Sopin~E.\,S.&&\\
\end{tabular}
}
\pagebreak

\def\leftfootline{\small{\textbf{\thepage}
\hfill INFORMATIKA I EE PRIMENENIYA~--- INFORMATICS AND APPLICATIONS\ \ \ 2018\
\ \ volume~12\ \ \ issue\ 4}
}%
 \def\rightfootline{\small{INFORMATIKA I EE PRIMENENIYA~---
INFORMATICS AND APPLICATIONS\ \ \ 2018\ \ \ volume~12\ \ \ issue\ 4
\hfill \textbf{\thepage}}}

\def\leftkol{2018 AUTHOR INDEX} % ENGLISH ABSTRACTS}

\def\rightkol{2018 AUTHOR INDEX} %ENGLISH ABSTRACTS}


\noindent
{\tabcolsep=3pt
\begin{tabular}{p{395.48108pt}cc}
&\textbf{Issue} & \textbf{Page}\\[6pt]
\Avtors{Nazarov~L.\,V., Lavrentyev~V.\,V., and Bykovets~E.\,V.} A~probability model of the influence\linebreak
\\[-12pt]
\hspace*{23pt}of~the~order book on the price process&2&29--34\\
\Avtors{Nazarova~I.\,A.} see~Malashenko~Yu.\,E.&&\\
\Avtors{Nazarova~I.\,A.} see~Malashenko~Yu.\,E.&&\\
\Avtors{Novikova~N.\,M.} see~Malashenko~Yu.\,E.&&\\
\Avtors{Novikova~N.\,M.} see~Malashenko~Yu.\,E.&&\\
\Avtors{Nuriev~V.\,A., Buntman~N.\,V., and Inkova~O.\,Yu.} Machine translation of russian connectives into\linebreak
\\[-12pt]
\hspace*{23pt}french: Errors and quality failures&2&105--113\\
\Avtors{Nuriev~V.\,A.} see~Buntman~N.\,V.&&\\
\Avtors{Ogaltsov~A.\,V.\ and Bakhteev~O.\,Y.} Automatic metadata extraction from scientific PDF documents&2&75--82\\
\Avtors{Pavlov~Yu.\,L.} see~Leri~M.\,M.&&\\
\Avtors{Razumchik~R.\,V.} see~Konovalov~M.\,G.&&\\
\Avtors{Razumchik~R.\,V.} see~Konovalov~M.\,G.&&\\
\Avtors{Romanov~D.\,A.} see~Kozerenko~E.\,B.&&\\
\Avtors{Samouylov~K.\,E., Gaidamaka~Yu.\,V., and~Shorgin~S.\,Ya.} Modeling movement of devices in\linebreak
\\[-12pt]
\hspace*{23pt}a~wireless network by random walk models&4&2--8\\
\Avtors{Samouylov~K.\,E.} see~Gorbunova~A.\,V.&&\\
\Avtors{Samouylov~K.\,Е.} see~Sopin~E.\,S.&&\\
\Avtors{Serebryakov~V.\,A.\ and Ataeva~O.\,M.} Ontology of the digital semantic library LibMeta&1&\hphantom{1}2--10\\
\Avtors{Shanin~I.\,A., Stupnikov~S.\,A., and Zakharov~V.\,N.} Methods and tools for fault detection\linebreak
\\[-12pt]
\hspace*{23pt}on~elements of housing and utility infrastructure&3&67--73\\
\Avtors{Shestakov~O.\,V.} Mean-square thresholding risk with a random sample size&3&14--17\\
\Avtors{Shestakov~O.\,V.} Unbiased risk estimate of stabilized hard thresholding in the model with\linebreak
\\[-12pt]
\hspace*{23pt}a~long-range dependence&2&11--16\\
\Avtors{Shestakov~O.\,V.} see~Kudryavtsev~A.\,A.&&\\
\Avtors{Shestakov~O.\,V.} see~Kudryavtsev~A.\,A.&&\\
\Avtors{Shihiev~Sh.\,B.} see~Mirzabekov~Ya.\,M.&&\\
\Avtors{Shirokov~N.\,I.} see~Vyshinsky~L.\,L.&&\\
\Avtors{Shnurkov~P.\,V.\ and Egorov~A.\,Y.} Development and preliminary study of a~stochastic semi-Markov model of continuous supply of product management under the condition of\linebreak
\\[-12pt]
\hspace*{23pt}constant consumption&1&109--117\\
\Avtors{Shnurkov~P.\,V.\ and Egorov~A.\,Y.} Solution to the problem of optimal control of a~stochastic semi-Markov model of continuous supply of product management under the condition\linebreak
\\[-12pt]
\hspace*{23pt}of~constantly happening consumption&2&83--89\\
\Avtors{Shorgin~S.\,Ya.} see~Grusho~A.\,A.&&\\
\Avtors{Shorgin~S.\,Ya.} see~Samouylov~K.\,E.&&\\
\Avtors{Sinitsyn~I.\,N.} Method of interpolational analytical modeling of processes in stochastic systems&1&55--61\\
\Avtors{Smerdov~A.\,N., Bakhteev~O.\,Y., and~Strijov~V.\,V.} Optimal recurrent neural network model\linebreak
\\[-12pt]
\hspace*{23pt}in~paraphrase detection&4&63--69\\
\Avtors{Smirnov~D.\,V.} see~Grusho~A.\,A.&&\\
\Avtors{Sochenkov~I.\,V., Zubarev~D.\,V., and Tikhomirov~I.\,A.} Exploratory patent search&1&89--94\\
\Avtors{Sopin~E.\,S., Naumov~V.\,A., and Samouylov~K.\,Е.} On the insensitivity of the stationary distribution\linebreak
\\[-12pt]
\hspace*{23pt}of the limited resources queuing system with state-dependent arrival and service rates&3&42--47\\
\Avtors{Stefanovich~A.\,I.} see~Bosov~A.\,V.&&\\
\Avtors{Strijov~V.\,V.} see~Smerdov~A.\,N.&&\\
\Avtors{Stupnikov~S.\,A.} see~Shanin~I.\,A.&&\\
\Avtors{Suchkov~A.\,P.} see~Zatsarinny~A.\,A.&&\\
\Avtors{Surina~A.\,A.} see~Tyrsin~A.\,N.&&\\
\Avtors{Syuntyurenko~O.\,V.} Financing of basic research: Conceptual shape of a system of support\linebreak
\\[-12pt]
\hspace*{23pt}of~decision-making with use of methods of scientometrics and analysis of data&1&118--127\\
\Avtors{Tikhomirov~I.\,A.} see~Sochenkov~I.\,V.&&\\
\Avtors{Timonina~E.\,E.} see~Grusho~A.\,A.&&\\
\Avtors{Timonina~E.\,E.} see~Grusho~A.\,A.&&\\
\end{tabular}
}
\pagebreak

\def\leftfootline{\small{\textbf{\thepage}
\hfill INFORMATIKA I EE PRIMENENIYA~--- INFORMATICS AND APPLICATIONS\ \ \ 2018\
\ \ volume~12\ \ \ issue\ 4}
}%
 \def\rightfootline{\small{INFORMATIKA I EE PRIMENENIYA~---
INFORMATICS AND APPLICATIONS\ \ \ 2018\ \ \ volume~12\ \ \ issue\ 4
\hfill \textbf{\thepage}}}

\def\leftkol{2018 AUTHOR INDEX} % ENGLISH ABSTRACTS}

\def\rightkol{2018 AUTHOR INDEX} %ENGLISH ABSTRACTS}


\noindent
{\tabcolsep=3pt
\begin{tabular}{p{395.48108pt}cc}
&\textbf{Issue} & \textbf{Page}\\[6pt]
\Avtors{Timonina~E.\,E.} see~Grusho~A.\,A.&&\\
\Avtors{Timonina~E.\,E.} see~Grusho~A.\,A.&&\\
\Avtors{Titova~A.\,I.} see~Arutyunov~E.\,N.&&\\
\Avtors{Tyrsin~A.\,N.\ and Surina~A.\,A.} A~model of risk management in Gaussian stochastic systems&2&50--59\\
\Avtors{Ushakov~V.\,G.} see~Kondranin~E.\,S.&&\\
\Avtors{Ushakov~V.\,G.} see~Mistryukov~A.\,V.&&\\
\Avtors{Vasil'eva~S.\,N.\ and Kan~Yu.\,S.} A~visualization algorithm for the plane probability measure kernel&2&60--68\\
\Avtors{Vinogradov~D.\,V.} Influence of preliminary estimates on the speed of search of similarities by\linebreak
\\[-12pt]
\hspace*{23pt}the~coupling Markov chain&1&49--54\\
\Avtors{Vyshinsky~L.\,L., Flerov~Yu.\,A., and Shirokov~N.\,I.} Computer-aided system of aircraft weight\linebreak
\\[-12pt]
\hspace*{23pt}design&1&18--30\\
\Avtors{Yakovlev O.\,A.} see~Batenkov~A.\,A.&&\\
\Avtors{Zabezhailo~M.\,I.} see~Grusho~A.\,A.&&\\
\Avtors{Zabezhailo~M.\,I.} see~Grusho~A.\,A.&&\\
\Avtors{Zakharov~V.\,N.} see~Basok~B.\,M.&&\\
\Avtors{Zakharov~V.\,N.} see~Shanin~I.\,A.&&\\
\Avtors{Zaltsman~A.\,D.} see~Zhukov~D.\,O.&&\\
\Avtors{Zatsarinny~A.\,A.\ and Suchkov~A.\,P.} The situational management system as a multiservice\linebreak
\\[-12pt]
\hspace*{23pt}technology in the cloud&1&78--88\\
\Avtors{Zatsarinny~A.\,A.} see~Grusho~A.\,A.,&&\\
\Avtors{Zatsman~I.\,M.} Implied knowledge: Foundations and technologies of explication&3&74--82\\
\Avtors{Zatsman~I.\,M.} see~Buntman~N.\,V.&&\\
\Avtors{Zeifman~A.\,I.} see~Korolev~V.\,Yu.&&\\
\Avtors{Zhukov~D.\,O., Khvatova~T.\,Yu., Lesko~S.\,A., and Zaltsman~A.\,D.} The influence of the connections' density on clusterization and percolation threshold during information distribution in social\linebreak
\\[-12pt]
\hspace*{23pt}networks&2&90--97\\
\Avtors{Zubarev~D.\,V.} see~Sochenkov~I.\,V.&&\\
\end{tabular}
}

%\thispagestyle{myheadings}
\def\leftfootline{\small{\textbf{\thepage}
\hfill INFORMATIKA I EE PRIMENENIYA~--- INFORMATICS AND APPLICATIONS\ \ \ 2018\
\ \ volume~12\ \ \ issue\ 4}
}%
 \def\rightfootline{\small{INFORMATIKA I EE PRIMENENIYA~---
INFORMATICS AND APPLICATIONS\ \ \ 2018\ \ \ volume~12\ \ \ issue\ 4
\hfill \textbf{\thepage}}}

 \label{end\stat}

\newpage

%   \vspace*{-48pt}

\begin{center}
\vspace*{6pt}
\mbox{%
\epsfxsize=53.502mm
\epsfbox{foto-1.eps}
}
\end{center}

\vspace*{6pt} %Академик


   \begin{center}
\fbox{\Large\textbf{Профессор Игорь Алексеевич Ушаков}}\\[12pt]
\textbf{\large 22.01.1935--27.02.2015}
   \end{center}


   %\vspace*{2.5mm}

   \vspace*{5mm}

   \thispagestyle{empty}

%\

%\vspace*{-12pt}


Редакционный совет и редакционная коллегия журнала <<Информатика и~её применения>> с~глубоким прискорбием извещают, что 27~февраля 2015~г.\ после тяжелой
и~продолжительной болезни скончался Игорь Алексеевич Ушаков~--- доктор технических наук, профессор, член редколлегии журнала <<Информатика и ее применения>>.

Игорь Алексеевич Ушаков окончил Московский авиационный институт, в~1963~г.\ защитил кандидатскую, а~в~1968~г.~--- докторскую диссертацию. С~1958 по 1989~гг.\ работал в~ряде научно-исследовательских организаций СССР, в~том числе руководил отделами в~НИИ АА и~ВЦ АН СССР; с 1969 по 1989 гг. преподавал в~МФТИ (был профессором, а~затем заведующим кафедрой) и~в~МЭИ. С~1989~г.~---- в~США: являлся профессором университета Дж.\ Вашингтона, университета Дж.\ Мэйсона и~Калифорнийского университета, сотрудником компаний MCI, Qualcomm и Hughes.

И.\,А.~Ушаков с момента основания журнала <<Надежность и~контроль качества>> был заместителем ответственного редактора, а~затем на протяжении многих лет членом редколлегии. В~2006~г.\ основал электронный международный журнал ``Reliability: Theory \& Application'', главным редактором которого оставался до конца жизни.

Учебниками и справочниками по теории надежности, написанными И.\,А.~Ушаковым, пользовались и~пользуются несколько поколений ученых и~специалистов в~разных странах мира.

Игорь Алексеевич всегда уделял огромное внимание работе с~молодежью; более~50 его учеников защитили докторские и~кандидатские диссертации.

И.\,А.~Ушаков вел активную научно-про\-све\-ти\-тель\-скую деятельность. В~частности, он был одним из организаторов и~руководителей Московского кабинета качества и~надежности при Политехническом музее (целью этого Кабинета было оказание консультаций работникам промышленных предприятий и~чтение курсов лекций для инженеров, занимающихся проблемой надежности). Находясь в~США, И.\,А.~Ушаков создал международный ин\-тер\-нет-фо\-рум им.\ Б.\,В.~Гнеденко, объединивший около~400~видных специалистов по приложениям теории вероятностей и~математической статистики, преимущественно в~об\-ласти теории надежности и~анализа риска, из десятков стран мира; коллективным членов этого Форума является и~наш журнал. Цели Форума~--- содействие контактам между специалистами из разных стран, организация обмена профессиональными 
новостями и~информацией (новые публикации, предстоящие события и~др.). Также необходимо отметить большое число на\-уч\-но-по\-пу\-ляр\-ных работ, опубликованных И.\,А.~Ушаковым.

И.\,А.~Ушаков обладал большим личным обаянием, имел широкий круг интересов. Все знавшие И.\,А.~Ушакова всегда будут помнить его как замечательного ученого и~прекрасного человека.

\bigskip

Редакционный совет и редакционная коллегия журнала <<Информатика и~её применения>> 
выражают глубокие соболезнования родным и близким покойного, всем, кто его знал и~работал с~ним.


%\def\stat{cont}
{%\hrule\par
%\vskip 7pt % 7pt
\raggedleft\Large \bf%\baselineskip=3.2ex
А\,В\,Т\,О\,Р\,С\,К\,И\,Й\ \ У\,К\,А\,З\,А\,Т\,Е\,Л\,Ь\ \ З\,А\ \ 2\,0\,1\,0 г. \vskip 17pt
    \hrule
    \par
\vskip 21pt plus 6pt minus 3pt }

\label{st\stat}

\def\tit{\ }

\def\aut{\ }
\def\auf{\ }

\def\leftkol{\ } % ENGLISH ABSTRACTS}

\def\rightkol{\ } %АВТОРСКИЙ УКАЗАТЕЛЬ ЗА 2010 г.} %ENGLISH ABSTRACTS}

\titele{\tit}{\aut}{\auf}{\leftkol}{\rightkol}

\vspace*{-12pt}

{\tabcolsep=3pt
\begin{tabular}{p{388pt}rr}
&\textbf{Выпуск} & \textbf{Стр.}\\[6pt]
\hangindent=23pt\noindent\textbf{Арутюнян~А.\,Р.} Моделирование влияния деформаций отпечатков пальцев на 
точность\linebreak
\vspace*{-12pt}\\
\hspace*{23pt}дактилоскопической идентификации$\dotfill$&1&51\\
\hangindent=23pt\noindent\textbf{Архипов~О.\,П., Зыкова~З.\,П.} Интеграция гетерогенной информации о цветных 
пикселях\linebreak
\vspace*{-12pt}\\
\hspace*{23pt}и их цветовосприятии$\dotfill$&4&15\\
\hangindent=23pt\noindent\textbf{Баранов~С.\,И., Френкель~С.\,Л., Захаров~В.\,Н.} Полуформальная верификация 
цифрового устройства с конвейером, основанная на использовании алгоритмических машин\linebreak
\vspace*{-12pt}\\
\hspace*{23pt}состояния$\dotfill$&4&49\\
\textbf{Бекетова~И.\,В.} см.~Каратеев~С.\,Л.&&\\
\textbf{Белоусов~В.\,В.} см.~Синицын~И.\,Н.&&\\
\hangindent=23pt\noindent\textbf{Бенинг~В.\,Е., Королев~Р.\,А.} О предельном поведении мощностей критериев в 
случае\linebreak
\vspace*{-12pt}\\
\hspace*{23pt}распределения Лапласа$\dotfill$&2&63\\
\hangindent=23pt\noindent\textbf{Бенинг~В.\,Е., Сипина~А.\,В.} Асимптотическое разложение для мощности 
критерия,\linebreak
\vspace*{-12pt}\\
\hspace*{23pt}основанного на выборочной медиане, в случае распределения Лапласа$\dotfill$&1&18\\
\textbf{Бондаренко~А.\,В.} см.~Каратеев~С.\,Л.&&\\
\hangindent=23pt\noindent\textbf{Бородина~А.\,В., Морозов~Е.\,В.} Об оценивании асимптотики вероятности 
большого\linebreak
\vspace*{-12pt}\\
\hspace*{23pt}уклонения стационарной регенеративной очереди с одним прибором$\dotfill$&3&29\\
\hangindent=23pt\noindent\textbf{Бунтман~Н.\,В., Минель~Ж.-Л., Ле~Пезан~Д., Зацман~И.\,М.} Типология и 
компьютерное\linebreak
\vspace*{-12pt}\\
\hspace*{23pt}моделирование трудностей перевода$\dotfill$&3&77\\
\textbf{Визильтер~Ю.\,В.} см.~Каратеев~С.\,Л.&&\\
\hangindent=23pt\noindent\textbf{Гавриленко~С.\,В.} Оценки скорости сходимости распределений случайных сумм с 
безгранично делимыми индексами к нормальному закону$\dotfill$&4&81\\
\hangindent=23pt\noindent\textbf{Григорьева~М.\,Е., Шевцова~И.\,Г.} Уточнение неравенства 
Каца--Берри--Эссеена$\dotfill$&2&75\\
\hangindent=23pt\noindent\textbf{Грушо~А.\,А., Грушо~Н.\,А., Тимонина~Е.\,Е.} Поиск конфликтов в политиках 
безопасности: модель случайных графов$\dotfill$&3&38\\
\textbf{Грушо~Н.\,А.} см.~Грушо~А.\,А.&&\\
\hangindent=23pt\noindent\textbf{Гудков~В.\,Ю.} Математические модели изображения отпечатка пальца на основе 
описания линий$\dotfill$&1&58\\
\textbf{Гуртов~А.\,В.} см.~Лукьяненко~А.\,С.&&\\
\textbf{Желтов~С.\,Ю.} см.~Каратеев~С.\,Л.&&\\
\hangindent=23pt\noindent\textbf{Захаров~А.\,А., Серебряков~В.\,А.} Система управления электронной библиотекой 
LibMeta$\dotfill$&4&2\\
\textbf{Захаров~В.\,Н.} см.~Баранов~С.\,И.&&\\
\textbf{Захарова~Т.\,В.} см.~Матвеева~С.\,С.&&\\
\hangindent=23pt\noindent\textbf{Зацаринный~А.\,А., Чупраков~К.\,Г.} Некоторые аспекты выбора технологии для 
постро-\linebreak
\vspace*{-12pt}\\
\hspace*{23pt}ения систем отображения информации ситуационного центра$\dotfill$&3&59\\
\textbf{Зацман~И.\,М.} см.~Бунтман~Н.\,В.&&\\
\hangindent=23pt\noindent\textbf{Зейфман~А.\,И., Коротышева~А.\,В., Сатин~Я.\,А., Шоргин~С.\,Я.} Об 
устойчивости нестаци-\linebreak
\vspace*{-12pt}\\
\hspace*{23pt}онарных систем обслуживания с катастрофами$\dotfill$&3&9\\
\textbf{Зыкова~З.\,П.} см.~Архипов~О.\,П.&&\\
\hangindent=23pt\noindent\textbf{Илюшин~Г.\,Я., Соколов~И.\,А.} Организация управляемого доступа пользователей 
к\linebreak
\vspace*{-12pt}\\
\hspace*{23pt}разнородным ведомственным информационным ресурсам$\dotfill$&1&24\\
\hangindent=23pt\noindent\textbf{Кавагучи~Ю., Ульянов~В.\,В., Фуджикоши~Я.} Приближения для статистик, 
описывающих\linebreak
\vspace*{-12pt}\\
\hspace*{23pt}геометрические свойства данных большой размерности, с оценками 
ошибок$\dotfill$&1&12\\
\hangindent=23pt\noindent\textbf{Каратеев~С.\,Л., Бекетова~И.\,В., Ососков~М.\,В., Князь~В.\,А., 
Визильтер~Ю.\,В., Бондаренко~А.\,В., Желтов~С.\,Ю.} Автоматизированный контроль 
качества цифровых\linebreak
\vspace*{-12pt}\\
\hspace*{23pt}изображений для персональных документов$\dotfill$&1&65\\
\end{tabular}
}

\pagebreak

\def\leftkol{АВТОРСКИЙ УКАЗАТЕЛЬ ЗА 2010 г.} % ENGLISH ABSTRACTS}

\def\rightkol{АВТОРСКИЙ УКАЗАТЕЛЬ ЗА 2010 г.} %ENGLISH ABSTRACTS}

{\tabcolsep=3pt
\begin{tabular}{p{388pt}rr}
&\textbf{Выпуск} & \textbf{Стр.}\\[3pt]
\hangindent=23pt\noindent\textbf{Козеренко~Е.\,Б.} Лингвистические фильтры в статистических моделях машинного\linebreak
\vspace*{-12pt}\\
\hspace*{23pt}перевода$\dotfill$&2&83\\
\hangindent=23pt\noindent\textbf{Козеренко~Е.\,Б., Кузнецов~И.\,П.} Когнитивно-лингвистические представления в 
систе-\linebreak
\vspace*{-12pt}\\
\hspace*{23pt}мах обработки текстов$\dotfill$&3&69\\
\textbf{Князь~В.\,А.} см.~Каратеев~С.\,Л.&&\\
\hangindent=23pt\noindent\textbf{Колесников~А.\,В., Солдатов~С.\,А.} Алгоритм координации для гибридной 
интеллектуальной системы решения сложной задачи оперативно-производственного\linebreak
\vspace*{-12pt}\\
\hspace*{23pt}планирования$\dotfill$&4&61\\
\hangindent=23pt\noindent\textbf{Коновалов~М.\,Г.} О планировании потоков в системах вычислительных 
ресурсов$\dotfill$&2&3\\
\textbf{Конушин~А.\,С.} см.~Конушин~В.\,С.&&\\
\hangindent=23pt\noindent\textbf{Конушин~В.\,С., Кривовязь~Г.\,Р., Конушин~А.\,С.} Алгоритм распознавания людей 
в видео-\linebreak
\vspace*{-12pt}\\
\hspace*{23pt}последовательности по одежде$\dotfill$&1&74\\
\textbf{Корепанов~Э.\, Р.} см.~Синицын~И.\,Н.&&\\
\textbf{Королев~В.\,Ю.} см.~Соколов~И.\,А.&&\\
\textbf{Королев~Р.\,А.} см.~Бенинг~В.\,Е.&&\\
\textbf{Коротышева~А.\,В.} см.~Зейфман~А.\,И.&&\\
\hangindent=23pt\noindent\textbf{Кривенко~М.\,П.} Непараметрическое оценивание элементов байесовского 
клас\-си-\linebreak
\vspace*{-12pt}\\
\hspace*{23pt}фикатора$\dotfill$&2&13\\
\textbf{Кривовязь~Г.\,Р.} см.~Конушин~В.\,С.&&\\
\textbf{Крылов~А.\,С.} см.~Павельева~Е.\,А.&&\\
\hangindent=23pt\noindent\textbf{Крылов~В.\,А.} Моделирование и классификация многоканальных дистанционных\linebreak
\vspace*{-12pt}\\
\hspace*{23pt}изображений с использованием копул$\dotfill$&4&34\\
\hangindent=23pt\noindent\textbf{Крючин~О.\,В.} Разработка параллельных эвристических алгоритмов подбора 
весовых\linebreak
\vspace*{-12pt}\\
\hspace*{23pt}коэффициентов искусственной нейтронной сети$\dotfill$&2&53\\
\hangindent=23pt\noindent\textbf{Кудрявцев~А.\,А., Шоргин~С.\,Я.} Байесовские модели массового обслуживания и 
надеж-\linebreak
\vspace*{-12pt}\\
\hspace*{23pt}ности: характеристики среднего числа заявок в системе $M\vert M \vert 1\vert 
\infty$$\dotfill$&3&16\\
\hangindent=23pt\noindent\textbf{Кузнецов~А.\,А.} Связь между временными и структурно-топологическими 
характери-\linebreak
\vspace*{-12pt}\\
\hspace*{23pt}стиками диаграмм ритма сердца здоровых людей$\dotfill$&4&39\\
\textbf{Кузнецов~И.\,П.} см.~Козеренко~Е.\,Б.&&\\
\textbf{Ле~Пезан~Д.} см.~Бунтман~Н.\,В.&&\\
\hangindent=23pt\noindent\textbf{Лукьяненко~А.\,С., Морозов~Е.\,В., Гуртов~А.\,В.} Анализ сетевого протокола с общей 
функ-\linebreak
\vspace*{-12pt}\\
\hspace*{23pt}цией расширения окна передачи сообщения при конфликтах$\dotfill$&2&46\\
\hangindent=23pt\noindent\textbf{Лямин~О.\,О.} О предельном поведении мощностей критериев в случае обобщенного\linebreak
\vspace*{-12pt}\\
\hspace*{23pt}распределения Лапласа$\dotfill$&3&47\\
\hangindent=23pt\noindent\textbf{Маркин~А.\,В., Шестаков~О.\,В.} Асимптотики оценки риска при пороговой 
обработке\linebreak
\vspace*{-12pt}\\
\hspace*{23pt}вейвлет-вейглет коэффициентов в задаче томографии$\dotfill$&2&36\\
\hangindent=23pt\noindent\textbf{Матвеева~С.\,С., Захарова~Т.\,В.} Сети массового обслуживания с наименьшей 
длиной\linebreak
\vspace*{-12pt}\\
\hspace*{23pt}очереди$\dotfill$&3&22\\
\hangindent=23pt\noindent\textbf{Матюшенко~С.\,И.} Стационарные характеристики двухканальной системы 
обслужива-\linebreak
\vspace*{-12pt}\\
\hspace*{23pt}ния с переупорядочиванием заявок и распределениями фазового типа$\dotfill$&4&68\\
\textbf{Минель~Ж.-Л.} см.~Бунтман~Н.\,В.&&\\
\textbf{Морозов~Е.\,В.} см.~Бородина~А.\,В.&&\\
\textbf{Морозов~Е.\,В.} см.~Лукьяненко~А.\,С.&&\\
\textbf{Ососков~М.\,В.} см.~Каратеев~С.\,Л.&&\\
\hangindent=23pt\noindent\textbf{Павельева~Е.\,А., Крылов~А.\,С.} Поиск и анализ ключевых точек радужной 
оболочки\linebreak
\vspace*{-12pt}\\
\hspace*{23pt}глаза методом преобразования Эрмита$\dotfill$&1&79\\
\textbf{Печинкин~А.\,В.} см.~Френкель~С.\,Л.,&&\\
\hangindent=23pt\noindent\textbf{Протасов~В.\,И.} Составление субъективного портрета с использованием 
эволюционно-\linebreak
\vspace*{-12pt}\\
\hspace*{23pt}го морфинга и квалиметрия метода$\dotfill$&1&83\\
\hangindent=23pt\noindent\textbf{Рудаков~К.\,В., Торшин~И.\,Ю.} Вопросы разрешимости задачи распознавания 
вторичной\linebreak
\vspace*{-12pt}\\
\hspace*{23pt}структуры белка$\dotfill$&2&25\\
\textbf{Сатин~Я.\,А.} см.~Зейфман~А.\,И.&&\\
\hangindent=23pt\noindent\textbf{Сейфуль-Мулюков~Р.\,Б.} Нефть как носитель информации о своем 
происхождении,\linebreak
\vspace*{-12pt}\\
\hspace*{23pt}структуре и эволюции$\dotfill$&1&41\\
\end{tabular}
}

{\tabcolsep=3pt
\begin{tabular}{p{388pt}rr}
&\textbf{Выпуск} & \textbf{Стр.}\\[6pt]
\textbf{Семендяев~Н.\,Н.} см.~Синицын~И.\,Н.&&\\
\textbf{Серебряков~В.\,А.} см.~Захаров~А.\,А.&&\\
\textbf{Синицын~В.\,И.} см.~Синицын~И.\,Н.&&\\
\hangindent=23pt\noindent\textbf{Синицын~И.\,Н., Синицын~В.\,И., Корепанов~Э.\, Р., Белоусов~В.\,В., 
Семендяев~Н.\,Н.} Оперативное построение информационных моделей движения полюса 
Земли\linebreak
\vspace*{-12pt}\\
\hspace*{23pt}методами линейных и линеаризованных фильтров$\dotfill$&1&2\\
\textbf{Сипина~А.\,В.} см.~Бенинг~В.\,Е.&&\\
\hangindent=23pt\noindent\textbf{Соколов~И.\,А.} О работах заслуженного деятеля науки Российской Федерации 
И.\,Н.~Синицына в области информационных технологий и автоматизации (к 70-летию\linebreak
\vspace*{-12pt}\\
\hspace*{23pt}со дня рождения)$\dotfill$&3&84\\
\textbf{Соколов~И.\,А.} см.~Илюшин~Г.\,Я.&&\\
\hangindent=23pt\noindent\textbf{Соколов~И.\,А., Королев~В.\,Ю.} Предисловие$\dotfill$&2&2\\
\textbf{Солдатов~С.\,А.} см.~Колесников~А.\,В.&&\\
\hangindent=23pt\noindent\textbf{Степанов~С.\,Ю.} Использование координатного метода фрагментации 
коммутаторной\linebreak
\vspace*{-12pt}\\
\hspace*{23pt}нейронной сети для сокращения трафика$\dotfill$&2&57\\
\textbf{Тимонина~Е.\,Е.} см.~Грушо~А.\,А.&&\\
\textbf{Торшин~И.\,Ю.} см.~Рудаков~К.\,В.&&\\
\textbf{Ульянов~В.\,В.} см.~Кавагучи~Ю.&&\\
\textbf{Фазекаш~И.} см.~Чупрунов~А.\,Н.&&\\
\textbf{Френкель~С.\,Л.} см.~Баранов~С.\,И.&&\\
\hangindent=23pt\noindent\textbf{Френкель~С.\,Л., Печинкин~А.\,В.} Оценка времени самовосстановления в 
цифровых\linebreak
\vspace*{-12pt}\\
\hspace*{23pt}системах после сбоев, вызываемых переходными помехами$\dotfill$&3&2\\
\textbf{Фуджикоши~Я.} см.~Кавагучи~Ю.&&\\
\hangindent=23pt\noindent\textbf{Цискаридзе~А.\,К.} Математическая модель и метод восстановления позы человека 
по\linebreak
\vspace*{-12pt}\\
\hspace*{23pt}стереопаре силуэтных изображений$\dotfill$&4&27\\
\hangindent=23pt\noindent\textbf{Чупраков~К.\,Г.} К вопросу о размещении коллективных средств отображения в 
ситуа-\linebreak
\vspace*{-12pt}\\
\hspace*{23pt}ционном зале с заданными параметрами$\dotfill$&4&89\\
\textbf{Чупраков~К.\,Г.} см.~Зацаринный~А.\,А.&&\\
\hangindent=23pt\noindent\textbf{Чупрунов~А.\,Н., Фазекаш~И.} Законы повторного логарифма для числа 
безошибочных\linebreak
\vspace*{-12pt}\\
\hspace*{23pt}блоков при помехоустойчивом кодировании$\dotfill$&3&42\\
\textbf{Шевцова~И.\,Г.} см.~Григорьева~М.\,Е.&&\\
\hangindent=23pt\noindent\textbf{Шестаков~О.\,В.} Аппроксимация распределения оценки риска пороговой 
обработки вейвлет-коэффициентов нормальным распределением при использовании 
выбо-\linebreak
\vspace*{-12pt}\\
\hspace*{23pt}рочной дисперсии$\dotfill$&4&73\\
\textbf{Шестаков~О.\,В.} см.~Маркин~А.\,В.&&\\
\textbf{Шоргин~С.\,Я.} см.~Зейфман~А.\,И.&&\\
\textbf{Шоргин~С.\,Я.} см.~Кудрявцев~А.\,А.&&\\
\end{tabular}
}

%\thispagestyle{myheadings}
\def\leftfootline{\small{\textbf{\thepage}
\hfill ИНФОРМАТИКА И ЕЁ ПРИМЕНЕНИЯ\ \ \ том~4\ \ \ выпуск~4\ \ \ 2010}
}%
 \def\rightfootline{\small{ИНФОРМАТИКА И ЕЁ ПРИМЕНЕНИЯ\ \ \ том~4\ \ \ выпуск~4\ \ \ 2010
 \hfill \textbf{\thepage}}}
 \label{end\stat}
%
%Том 10 Выпуск 1-4 Год 2016

\def\stat{cont-e}
{%\hrule\par
%\vskip 7pt % 7pt
\raggedleft\Large \bf%\baselineskip=3.2ex
2\,0\,1\,6\ \ A\,U\,T\,H\,O\,R\ \ I\,N\,D\,E\,X \vskip 17pt
 \hrule
 \par
\vskip 21pt plus 6pt minus 3pt }

\label{st\stat}

\def\tit{\ }

\def\aut{\ }
\def\auf{\ }

\def\leftkol{\ } %2016 AUTHOR INDEX} % ENGLISH ABSTRACTS}

\def\rightkol{\ } %2016 AUTHOR INDEX} %ENGLISH ABSTRACTS}

\titele{\tit}{\aut}{\auf}{\leftkol}{\rightkol}

\def\leftfootline{\small{\textbf{\thepage}
\hfill INFORMATIKA I EE PRIMENENIYA~--- INFORMATICS AND APPLICATIONS\ \ \ 2016\
\ \ volume~10\ \ \ issue\ 4}
}%
 \def\rightfootline{\small{INFORMATIKA I EE PRIMENENIYA~--- INFORMATICS AND APPLICATIONS\ \ \ 2016\ \ \ volume~10\ \ \ issue\ 4
\hfill \textbf{\thepage}}}

\vspace*{-12pt}
\vspace*{-18pt}

{\tabcolsep=2.8pt
\begin{tabular}{p{382pt}cc}
&\textbf{Issue} & \textbf{Page}\\[6pt]
\Avtors{Agalarov~M.\,Ya.} see~Agalarov~Ya.\,M.&&\\
\Avtors{Agalarov~Ya.\,M., Agalarov~M.\,Ya., and
Shorgin~V.\,S.} About the optimal threshold of queue\linebreak
\\[-12pt]
\hspace*{23pt}length in a~particular problem of profit maximization
in the $M/G/1$ queuing system&2&70--79\\
\Avtors{Alexeyevsky~D.\,A.} BioNLP ontology extraction from 
a~restricted language corpus with\linebreak
\\[-12pt]
\hspace*{23pt}context-free grammars&1&119--128\\
\Avtors{Andreev~S.\,D.} see~Gaidamaka~Yu.\,V.&&\\
\Avtors{Andreev~S.\,D.} see~Ometov~A.\,Ya.&&\\
\Avtors{Arkhipov~O.\,P., Arkhipov~P.\,O., and Sidorkin~I.\,I.} The
option to create a~local coordinate\linebreak
\\[-12pt]
\hspace*{23pt}system for synchronization of selected images&3&91--97\\
\Avtors{Arkhipov~P.\,O.} see~Arkhipov~O.\,P.&&\\
\Avtors{Belousov~V.\,V.} see~Shnurkov~P.\,V.&&\\
\Avtors{Belousov~V.\,V.} see~Shnurkov~P.\,V.&&\\
\Avtors{Bening~V.\,E.} Calculation of~the~asymptotic deficiency
of~some statistical procedures based\linebreak
\\[-12pt]
\hspace*{23pt}on~samples with~random sizes&4&34--45\\
\Avtors{Borisov~A.\,V., Bosov~A.\,V., and Miller~G.\,B.} Modeling and
monitoring of VoIP connection&2&\hphantom{1}2--13\\
\Avtors{Bosov~A.\,V.} see~Borisov~A.\,V.&&\\
\Avtors{Briukhov~D.\,O.} see~Stupnikov~S.\,A.&&\\
\Avtors{Callaos~N.\,K.\ and Seyful-Mulyukov~R.\,B.} Complexity and
its information content&1&129--139\\
\Avtors{Chertok~A.\,V., Kadaner~A.\,I., Khazeeva~G.\,T., and
Sokolov~I.\,A.} Regime switching detection\linebreak
\\[-12pt]
\hspace*{23pt}for~the~Levy driven
Ornstein--Uhlenbeck process using CUSUM methods&4&46--56\\
\Avtors{Chichagov~V.\,V.} Asymptotic expansions of mean absolute
error of uniformly minimum variance unbiased and maximum likelihood
estimators on the one-parameter exponential\linebreak
\\[-12pt]
\hspace*{23pt}family model of lattice distributions&3&66--76\\
\Avtors{Danishevsky~V.\,I.} see~Kolesnikov A.\,V.&&\\
\Avtors{Fazliev~A.\,Z.} see~Kalinichenko~L.\,A.&&\\
\Avtors{Fedoseev~A.\,A.} What is behind the concept of ``knowledge in
small packages''&3&105--110\\
\Avtors{Gaidamaka~Yu.\,V., Andreev~S.\,D., Sopin~E.\,S.,
Samouylov~K.\,E., and Shorgin~S.\,Ya.} Interference analysis
of~the~device-to-device communications model with~regard to~a~signal\linebreak
\\[-12pt]
\hspace*{23pt}propagation environment&4&\hphantom{1}2--10\\
\Avtors{Gasilov~A.\,V.} see~Yakovlev~O.\,A.&&\\
\Avtors{Goncharov~A.\,V.\ and Strijov~V.\,V.} Metric time series
classification using weighted dynamic\linebreak
\\[-12pt]
\hspace*{23pt}warping relative to centroids of classes&2&36--47\\
\Avtors{Gordov~E.\,P.} see~Kalinichenko~L.\,A.&&\\
\Avtors{Gorshenin~A.\,K.} Concept of online service for stochastic
modeling of real processes&1&72--81\\
\Avtors{Gorshenin~A.\,K.} see~Shnurkov~P.\,V.&&\\
\Avtors{Gorshenin~A.\,K.} see~Shnurkov~P.\,V.&&\\
\Avtors{Grusho~A.\,A., Grusho~N.\,A., Zabezhailo~M.\,I., and
Timonina~E.\,E.} Integration of statistical and\linebreak
\\[-12pt]
\hspace*{23pt}deterministic methods for
analysis of information security&3&2--8\\
\Avtors{Grusho~A.\,A., Zabezhailo~M.\,I., and Zatsarinny~A.\,A.} On
the advanced procedure to reduce\linebreak
\\[-12pt]
\hspace*{23pt}calculation of Galois closures&4&\hphantom{1}96--104\\
\Avtors{Grusho~N.\,A.} see~Grusho~A.\,A.&&\\
\Avtors{Havanskov~V.\,A.} see~Minin~V.\,A.&&\\
\Avtors{Inkova~O.\,Yu.} see~Zatsman~I.\,M.&&\\
\Avtors{Isachenko~R.\,V.\ and Strijov~V.\,V.} Metric learning in
multiclass time series classification\linebreak
\\[-12pt]
\hspace*{23pt}problem&2&48--57\\
\end{tabular}
}
\pagebreak

\def\leftfootline{\small{\textbf{\thepage}
\hfill INFORMATIKA I EE PRIMENENIYA~--- INFORMATICS AND APPLICATIONS\ \ \ 2016\
\ \ volume~10\ \ \ issue\ 4}
}%
 \def\rightfootline{\small{INFORMATIKA I EE PRIMENENIYA~---
INFORMATICS AND APPLICATIONS\ \ \ 2016\ \ \ volume~10\ \ \ issue\ 4
\hfill \textbf{\thepage}}}

\def\leftkol{2016 AUTHOR INDEX} % ENGLISH ABSTRACTS}

\def\rightkol{2016 AUTHOR INDEX} %ENGLISH ABSTRACTS}


{\tabcolsep=2.83pt
\begin{tabular}{p{382pt}cc}
&\textbf{Issue} & \textbf{Page}\\[6pt]
\Avtors{Kadaner~A.\,I.} see~Chertok~A.\,V.&&\\[.255pt]
\Avtors{Kalinichenko~L.\,A., Volnova~A.\,A., Gordov~E.\,P.,
Kiselyova~N.\,N., Kovaleva~D.\,A., Malkov~O.\,Yu., Okladnikov~I.\,G.,
Podkolodnyy~N.\,L., Pozanenko~A.\,S., Ponomareva~N.\,V.,
Stupnikov~S.\,A.,} \textbf{and Fazliev~A.\,Z.} Data access challenges for data
intensive\linebreak
\\[-12pt]
\hspace*{23pt}research in Russia&1& 2--22\\[.255pt]
\Avtors{Karasikov~M.\,E.\ and Strijov~V.\,V.} Feature-based
time-series classification&4&121--131\\[.255pt]
\Avtors{Khazeeva~G.\,T.} see~Chertok~A.\,V.&&\\[.255pt]
\Avtors{Khokhlov~Yu.\,S.} Multivariate fractional Levy motion and its
applications&2&\hphantom{1}98--106\\[.255pt]
\Avtors{Kirikov~I.\,A., Kolesnikov~A.\,V., Listopad~S.\,V., and
Rumovskaya~S.\,B.} Fine-grained hybrid\linebreak
\\[-12pt]
\hspace*{23pt}intelligent systems. Part 2:
Bidirectional hybridization&1&\hphantom{1}96--105\\[.255pt]
\Avtors{Kirikov~I.\,A., Kolesnikov~A.\,V., Listopad~S.\,V., and
Rumovskaya~S.\,B.} ``Virtual council''~---\linebreak
\\[-12pt]
\hspace*{23pt}source environment
supporting complex diagnostic decision making&3&81--90\\[.255pt]
\Avtors{Kiselyova~N.\,N.} see~Kalinichenko~L.\,A.&&\\[.255pt]
\Avtors{Kolesnikov A.\,V., Listopad~S.\,V., Rumovskaya~S.\,B., and
Danishevsky~V.\,I.} Informal axiomatic\linebreak
\\[-12pt]
\hspace*{23pt}theory of~the~role visual models&4&114--120\\[.255pt]
\Avtors{Kolesnikov~A.\,V.} see~Kirikov~I.\,A.&&\\[.255pt]
\Avtors{Kolesnikov~A.\,V.} see~Kirikov~I.\,A.&&\\[.255pt]
\Avtors{Kolin~K.\,K.} Humanitarian aspects of information
security&3&111--121\\[.255pt]
\Avtors{Konovalov~M.\,G.\ and Razumchik~R.\,V.} Dispatching
to~two parallel nonobservable queues using\linebreak
\\[-12pt]
\hspace*{23pt}only static
information&4&57--67\\[.255pt]
\Avtors{Korchagin~A.\,Yu.} see~Korolev~V.\,Yu.&&\\[.255pt]
\Avtors{Korchagin~A.\,Yu.} see~Korolev~V.\,Yu.&&\\[.255pt]
\Avtors{Korepanov~E.\,R.} see~Sinitsyn~I.\,N.&&\\[.255pt]
\Avtors{Korepanov~E.\,R.} see~Sinitsyn~I.\,N.&&\\[.255pt]
\Avtors{Korolev~V.\,Yu., Korchagin~A.\,Yu., and Zeifman~A.\,I.} The
Poisson theorem for Bernoulli trials\linebreak
\\[-12pt]
\hspace*{23pt}with~a~random probability
of~success and~a~discrete analog of~the~Weibull distribution&4&11--20\\[.255pt]
\Avtors{Korolev~V.\,Yu., Zeifman~A.\,I., and Korchagin~A.\,Yu.}
Asymmetric Linnik distributions as~limit\linebreak
\\[-12pt]
\hspace*{23pt}laws for~random sums
of~independent random variables with~finite variances&4&21--33\\[.255pt]
\Avtors{Koucheryavy~E.\,A.} see~Ometov~A.\,Ya.&&\\[.255pt]
\Avtors{Kovaleva~D.\,A.} see~Kalinichenko~L.\,A.&&\\[.255pt]
\Avtors{Kovalyov~S.\,P.} Metaprogramming to increase
manufacturability of large-scale software-\linebreak
\\[-12pt]
\hspace*{23pt}intensive systems&1&56--66\\[.255pt]
\Avtors{Krivenko~M.\,P.} Significance tests of feature selection for
classification&3&32--40\\[.255pt]
\Avtors{Kruzhkov~M.\,G.} see~Zalizniak~Anna~A.&&\\[.255pt]
\Avtors{Kruzhkov~M.\,G.} see~Zatsman~I.\,M.&&\\[.255pt]
\Avtors{Kudryavtsev~A.\,A.} Bayesian queueing and reliability models:
\textit{A~priori} distributions with\linebreak
\\[-12pt]
\hspace*{23pt}compact support&1&67--71\\[.255pt]
\Avtors{Kudryavtsev~A.\,A.} Characteristics dependent on the balance
coefficient in Bayesian models\linebreak
\\[-12pt]
\hspace*{23pt}with compact support of \textit{a priori}
distributions&3&77--80\\[.255pt]
\Avtors{Kudryavtsev~A.\,A.\ and Palionnaia~S.\,I.} Bayesian recurrent
model of reliability growth:\linebreak
\\[-12pt]
\hspace*{23pt}Parabolic distribution of parameters&2&80--83\\[.255pt]
\Avtors{Kudryavtsev~A.\,A.\ and Titova~A.\,I.} Bayesian queuing
and~reliability models: Degenerate-\linebreak
\\[-12pt]
\hspace*{23pt}Weibull case&4&68--71\\[.255pt]
\Avtors{Leontyev~N.\,D.\ and Ushakov~V.\,G.} Analysis of a queueing
system with autoregressive arrivals\linebreak
\\[-12pt]
\hspace*{23pt}and nonpreemptive priority&3&15--22\\[.255pt]
\Avtors{Listopad~S.\,V.} see~Kirikov~I.\,A.&&\\[.255pt]
\Avtors{Listopad~S.\,V.} see~Kirikov~I.\,A.&&\\[.255pt]
\Avtors{Listopad~S.\,V.} see~Kolesnikov A.\,V.&&\\[.255pt]
\Avtors{Malkov~O.\,Yu.} see~Kalinichenko~L.\,A.&&\\[.255pt]
\Avtors{Markov~A.\,S., Monakhov~M.\,M., and
Ulyanov~V.\,V.} Generalized Cornish--Fisher expansions\linebreak
\\[-12pt]
\hspace*{23pt}for distributions of statistics based on samples
of random size&2&84--91\\[.255pt]
\Avtors{Melnikov~A.\,K.\ and Ronzhin~A.\,F.} Generalized statistical
method of~text analysis based\linebreak
\\[-12pt]
\hspace*{23pt}on~calculation of~probability distributions
of~statistical values&4&89--95\\
\end{tabular}
}
\pagebreak

\def\leftfootline{\small{\textbf{\thepage}
\hfill INFORMATIKA I EE PRIMENENIYA~--- INFORMATICS AND APPLICATIONS\ \ \ 2016\
\ \ volume~10\ \ \ issue\ 4}
}%
 \def\rightfootline{\small{INFORMATIKA I EE PRIMENENIYA~---
INFORMATICS AND APPLICATIONS\ \ \ 2016\ \ \ volume~10\ \ \ issue\ 4
\hfill \textbf{\thepage}}}

\def\leftkol{2016 AUTHOR INDEX} % ENGLISH ABSTRACTS}

\def\rightkol{2016 AUTHOR INDEX} %ENGLISH ABSTRACTS}


{\tabcolsep=3pt
\begin{tabular}{p{381pt}cc}
&\textbf{Issue} & \textbf{Page}\\[6pt]
\Avtors{Meykhanadzhyan~L.\,A.} Stationary characteristics of the finite
capacity queueing system with\linebreak
\\[-12pt]
\hspace*{23pt}inverse service order and generalized
probabilistic priority&2&123--131\\[.23pt]
\Avtors{Miller~G.\,B.} see~Borisov~A.\,V.&&\\[.23pt]
\Avtors{Minin~V.\,A., Zatsman~I.\,M., Havanskov~V.\,A., and
Shubnikov~S.\,K.} Intensity of citation of scientific publications in
inventions on information and computer technologies patented\linebreak
\\[-12pt]
\hspace*{23pt}in Russia by domestic and foreign applicants&2&107--122\\[.23pt]
\Avtors{Monakhov~M.\,M.} see~Markov~A.\,S.&&\\[.23pt]
\Avtors{Naumov~V.\,A.\ and Samouylov~K.\,E.} On relationship
between queuing systems with resources\linebreak
\\[-12pt]
\hspace*{23pt}and Erlang networks&3&\hphantom{1}9--14\\[.23pt]
\Avtors{Okladnikov~I.\,G.} see~Kalinichenko~L.\,A.&&\\[.23pt]
\Avtors{Ometov~A.\,Ya., Andreev~S.\,D., Turlikov~A.\,M., and
Koucheryavy~E.\,A.} Performance analysis of\linebreak
\\[-12pt]
\hspace*{23pt}a wireless data
aggregation system with contention for contemporary sensor
networks&3&23--31\\[.23pt]
\Avtors{Palionnaia~S.\,I.} see~Kudryavtsev~A.\,A.&&\\[.23pt]
\Avtors{Podkolodnyy~N.\,L.} see~Kalinichenko~L.\,A.&&\\[.23pt]
\Avtors{Ponomareva~N.\,V.} see~Kalinichenko~L.\,A.&&\\[.23pt]
\Avtors{Popkova~N.\,A.} see~Zatsman~I.\,M.&&\\[.23pt]
\Avtors{Pozanenko~A.\,S.} see~Kalinichenko~L.\,A.&&\\[.23pt]
\Avtors{Razumchik~R.\,V.} see~Konovalov~M.\,G.&&\\[.23pt]
\Avtors{Ronzhin~A.\,F.} see~Melnikov~A.\,K.&&\\[.23pt]
\Avtors{Rumovskaya~S.\,B.} see~Kirikov~I.\,A.&&\\[.23pt]
\Avtors{Rumovskaya~S.\,B.} see~Kirikov~I.\,A.&&\\[.23pt]
\Avtors{Rumovskaya~S.\,B.} see~Kolesnikov A.\,V.&&\\[.23pt]
\Avtors{Samouylov~K.\,E.} see~Gaidamaka~Yu.\,V.&&\\[.23pt]
\Avtors{Samouylov~K.\,E.} see~Naumov~V.\,A.&&\\[.23pt]
\Avtors{Serebryanskii~S.\,M.} see~Tyrsin~A.\,N.&&\\[.23pt]
\Avtors{Seyful-Mulyukov~R.\,B.} see~Callaos~N.\,K.&&\\[.23pt]
\Avtors{Shestakov~O.\,V.} Statistical properties of the denoising method
based on the stabilized hard\linebreak
\\[-12pt]
\hspace*{23pt}thresholding&2&65--69\\[.23pt]
\Avtors{Shestakov~O.\,V.} The strong law of large numbers for the risk
estimate in the problem of\linebreak
\\[-12pt]
\hspace*{23pt}tomographic image reconstruction from
projections with a correlated noise&3&41--45\\[.23pt]
\Avtors{Shestakov~O.\,V.} see~Zakharova~T.\,V.&&\\[.23pt]
\Avtors{Shnurkov~P.\,V., Gorshenin~A.\,K., and Belousov~V.\,V.}
Analytical solution of~the~optimal control\linebreak
\\[-12pt]
\hspace*{23pt}task of~a~semi-Markov
process with~finite set of~states&4&72--88\\[.23pt]
\Avtors{Shnurkov~P.\,V., Zasypko~V.\,V., Belousov~V.\,V., and
Gorshenin~A.\,K.} Development of the algorithm of numerical solution
of the optimal investment control problem\linebreak
\\[-12pt]
\hspace*{23pt}in the closed dynamical model of three-sector economy&1&82--95\\[.23pt]
\Avtors{Shorgin~S.\,Ya.} see~Gaidamaka~Yu.\,V.&&\\[.23pt]
\Avtors{Shorgin~V.\,S.} see~Agalarov~Ya.\,M.&&\\[.23pt]
\Avtors{Shubnikov~S.\,K.} see~Minin~V.\,A.&&\\[.23pt]
\Avtors{Sidorkin~I.\,I.} see~Arkhipov~O.\,P.&&\\[.23pt]
\Avtors{Sinitsyn~I.\,N.} Analytical modeling of processes in stochastic
systems with complex fractional\linebreak
\\[-12pt]
\hspace*{23pt}order Bessel nonlinearities&3&55--65\\[.23pt]
\Avtors{Sinitsyn~I.\,N.} Orthogonal supoptimal filters for nonlinear
stochastic systems on manifolds&1&34--44\\[.23pt]
\Avtors{Sinitsyn~I.\,N.\ and Korepanov~E.\,R.} Normal Pugachev
conditionally-optimal filters and extra-\linebreak
\\[-12pt]
\hspace*{23pt}polators for state linear stochastic systems&2&14--23\\[.23pt]
\Avtors{Sinitsyn~I.\,N.\ and Sinitsyn~V.\,I.} Analytical modeling of
distributions in stochastic systems on\linebreak
\\[-12pt]
\hspace*{23pt}manifolds based on ellipsoidal approximation&1&45--55\\[.23pt]
\Avtors{Sinitsyn~I.\,N., Sinitsyn~V.\,I., and
Korepanov~E.\,R.} Ellipsoidal suboptimal filters for nonlinear\linebreak
\\[-12pt]
\hspace*{23pt}stochastic systems on manifolds&2&24--35\\[.23pt]
\Avtors{Sinitsyn~V.\,I.} see~Sinitsyn~I.\,N.&&\\[.23pt]
\Avtors{Sinitsyn~V.\,I.} see~Sinitsyn~I.\,N.&&\\[.23pt]
\Avtors{Skvortsov~N.\,A.} see~Stupnikov~S.\,A.&&\\[.23pt]
\Avtors{Sokolov~I.\,A.} see~Chertok~A.\,V.&&\\
\end{tabular}
}
\pagebreak

\def\leftfootline{\small{\textbf{\thepage}
\hfill INFORMATIKA I EE PRIMENENIYA~--- INFORMATICS AND APPLICATIONS\ \ \ 2016\
\ \ volume~10\ \ \ issue\ 4}
}%
 \def\rightfootline{\small{INFORMATIKA I EE PRIMENENIYA~---
INFORMATICS AND APPLICATIONS\ \ \ 2016\ \ \ volume~10\ \ \ issue\ 4
\hfill \textbf{\thepage}}}

\def\leftkol{2016 AUTHOR INDEX} % ENGLISH ABSTRACTS}

\def\rightkol{2016 AUTHOR INDEX} %ENGLISH ABSTRACTS}


{\tabcolsep=3pt
\begin{tabular}{p{382pt}cc}
&\textbf{Issue} & \textbf{Page}\\[6pt]
\Avtors{Sopin~E.\,S.} see~Gaidamaka~Yu.\,V.&&\\
\Avtors{Strijov~V.\,V.} see~Goncharov~A.\,V.&&\\
\Avtors{Strijov~V.\,V.} see~Isachenko~R.\,V.&&\\
\Avtors{Strijov~V.\,V.} see~Karasikov~M.\,E.&&\\
\Avtors{Stupnikov~S.\,A., Briukhov~D.\,O., and Skvortsov~N.\,A.}
Co-lending systemic risk analysis over\linebreak
\\[-12pt]
\hspace*{23pt}heterogeneous data collections&1&23--33\\
\Avtors{Stupnikov~S.\,A.} see~Kalinichenko~L.\,A.&&\\
\Avtors{Suchkov~A.\,P.} see~Zatsarinny~A.\,A.&&\\
\Avtors{Timonina~E.\,E.} see~Grusho~A.\,A.&&\\
\Avtors{Titova~A.\,I.} see~Kudryavtsev~A.\,A.&&\\
\Avtors{Turlikov~A.\,M.} see~Ometov~A.\,Ya.&&\\
\Avtors{Tyrsin~A.\,N.\ and Serebryanskii~S.\,M.} Recognition of
dependences on the basis of inverse\linebreak
\\[-12pt]
\hspace*{23pt}mapping&2&58--64\\
\Avtors{Ulyanov~V.\,V.} see~Markov~A.\,S.&&\\
\Avtors{Ushakov~V.\,G.} Queueing system with working vacations and
hyperexponential input stream&2&92--97\\
\Avtors{Ushakov~V.\,G.} see~Leontyev~N.\,D.&&\\
\Avtors{Volnova~A.\,A.} see~Kalinichenko~L.\,A.&&\\
\Avtors{Yakovlev~O.\,A.\ and Gasilov~A.\,V.} Speeded-up stereo
matching using geodesic support weights&3&\hphantom{1}98--104\\
\Avtors{Zabezhailo~M.\,I.} see~Grusho~A.\,A.&&\\
\Avtors{Zabezhailo~M.\,I.} see~Grusho~A.\,A.&&\\
\Avtors{Zakharova~T.\,V.\ and Shestakov~O.\,V.} Precision analysis of
wavelet processing of aerodynamic\linebreak
\\[-12pt]
\hspace*{23pt}flow patterns&3&46--54\\
\Avtors{Zalizniak~Anna~A.\ and Kruzhkov~M.\,G.} Database
of~Russian impersonal verbal constructions&4&132--141\\
\Avtors{Zasypko~V.\,V.} see~Shnurkov~P.\,V.&&\\
\Avtors{Zatsarinny~A.\,A.\ and Suchkov~A.\,P.} Systems engineering
approaches to~the~establishment of\linebreak
\\[-12pt]
\hspace*{23pt}a~system for~decision support based
on~situational analysis&4&105--113\\
\Avtors{Zatsarinny~A.\,A.} see~Grusho~A.\,A.&&\\
\Avtors{Zatsman~I.\,M., Inkova~O.\,Yu., Kruzhkov~M.\,G., and
Popkova~N.\,A.} Representation of cross-\linebreak
\\[-12pt]
\hspace*{23pt}lingual knowledge about
connectors in supracorpora databases&1&106--118\\
\Avtors{Zatsman~I.\,M.} see~Minin~V.\,A.&&\\
\Avtors{Zeifman~A.\,I.} see~Korolev~V.\,Yu.&&\\
\Avtors{Zeifman~A.\,I.} see~Korolev~V.\,Yu.&&\\
\end{tabular}
}

%\thispagestyle{myheadings}
\def\leftfootline{\small{\textbf{\thepage}
\hfill INFORMATIKA I EE PRIMENENIYA~--- INFORMATICS AND APPLICATIONS\ \ \ 2016\
\ \ volume~10\ \ \ issue\ 4}
}%
 \def\rightfootline{\small{INFORMATIKA I EE PRIMENENIYA~---
INFORMATICS AND APPLICATIONS\ \ \ 2016\ \ \ volume~10\ \ \ issue\ 4
\hfill \textbf{\thepage}}}

 \label{end\stat}

\newpage

%\def\stat{rekl}
%\label{preobr}

%\def\tit{АКАДЕМИК ПУГАЧЁВ  ВЛАДИМИР СЕМЁНОВИЧ\\
%25.03.1911--25.03.1998}


%   \vspace*{-48pt}
%   \begin{center}\LARGE
%Академик Пугачёв  Владимир Семёнович\\ (25.03.1911--25.03.1998)
%   \end{center}
   
   %\vspace*{2.5mm}
   
   \begin{center}

{\prgsh\LARGE
ОБЪЯВЛЕНИЯ О КОНФЕРЕНЦИЯХ}

\end{center}
%\hrule

\vspace*{6pt}

   
   \vspace*{10mm}
   
   \thispagestyle{empty}

\noindent
\begin{tabular}{cc}
%\begin{center}
\multicolumn{1}{c}{\raisebox{-40pt}[0pt][0pt]{\mbox{%
\epsfxsize=33mm
\epsfbox{vspu.eps}
}}}
%\end{center}
&
\tabcolsep=0pt\begin{tabular}{c}
{\prg{\Large\textbf{XII Всероссийское совещание}}}\\[6pt]
{\prg{\Large\textbf{по проблемам управления}}}\\[12pt]
{\prg{\large 16--19 июня 2014~г.}}\\[6pt] 
{\prg{\large Институт проблем управления имени В.\,А.~Трапезникова РАН}}\\[6pt]
{\prg{\large Москва, Россия}}
\end{tabular}
\end{tabular}

\vspace*{60pt}

     
 { %\large    
 XII Всероссийское совещание по проблемам управления (ВСПУ XII), посвященное 75-летию 
Института проблем управления (ИПУ) имени В.\,А.~Трапезникова РАН, проводится 16--19~июня 
2014~г.\ 
в ИПУ РАН (г.~Москва, Россия). ВСПУ XII организуется ИПУ РАН при поддержке РФФИ, Отделения 
энергетики, машиностроения, механики и процессов управления Российской академии наук, 
Российского 
национального комитета по автоматическому управлению, Академии навигации и управ\-ле\-ния 
движением, 
Научного совета РАН по комплексным проблемам управления и автоматизации, Совета по 
мехатронике и робототехнике РАН. Официальный язык Совещания~--- русский.

\vspace*{24pt}
     
     \textbf{Направления работы}
     \begin{enumerate}[1.]
\item Теория систем управления
\item Управление подвижными объектами и навигация
\item Интеллектуальные системы управления
\item Управление в промышленности, транспортом и логистикой
\item Управление системами междисциплинарной природы
\item Средства измерения, вычислений и контроля в управлении
\item Системный анализ и принятие решений в задачах управления
\item Информационные технологии в управлении
\item Проблемы образования в области управления: современное содержание и технологии обучения
\end{enumerate}

\vspace*{24pt}

     Подробная информация о Совещании находится на сайте {\sf http://vspu2014.ipu.ru}. Срок 
окончательной подачи докладов через систему подачи докладов на сайте~--- \textbf{30~ноября} 
2013~г.
}

%\include{rekl-1}

%\end{document}

%   \vspace*{-48pt}

\begin{center}
\vspace*{6pt}
\mbox{%
\epsfxsize=53.502mm
\epsfbox{foto-1.eps}
}
\end{center}

\vspace*{6pt} %Академик


   \begin{center}
\fbox{\Large\textbf{Профессор Игорь Алексеевич Ушаков}}\\[12pt]
\textbf{\large 22.01.1935--27.02.2015}
   \end{center}


   %\vspace*{2.5mm}

   \vspace*{5mm}

   \thispagestyle{empty}

%\

%\vspace*{-12pt}


Редакционный совет и редакционная коллегия журнала <<Информатика и~её применения>> с~глубоким прискорбием извещают, что 27~февраля 2015~г.\ после тяжелой
и~продолжительной болезни скончался Игорь Алексеевич Ушаков~--- доктор технических наук, профессор, член редколлегии журнала <<Информатика и ее применения>>.

Игорь Алексеевич Ушаков окончил Московский авиационный институт, в~1963~г.\ защитил кандидатскую, а~в~1968~г.~--- докторскую диссертацию. С~1958 по 1989~гг.\ работал в~ряде научно-исследовательских организаций СССР, в~том числе руководил отделами в~НИИ АА и~ВЦ АН СССР; с 1969 по 1989 гг. преподавал в~МФТИ (был профессором, а~затем заведующим кафедрой) и~в~МЭИ. С~1989~г.~---- в~США: являлся профессором университета Дж.\ Вашингтона, университета Дж.\ Мэйсона и~Калифорнийского университета, сотрудником компаний MCI, Qualcomm и Hughes.

И.\,А.~Ушаков с момента основания журнала <<Надежность и~контроль качества>> был заместителем ответственного редактора, а~затем на протяжении многих лет членом редколлегии. В~2006~г.\ основал электронный международный журнал ``Reliability: Theory \& Application'', главным редактором которого оставался до конца жизни.

Учебниками и справочниками по теории надежности, написанными И.\,А.~Ушаковым, пользовались и~пользуются несколько поколений ученых и~специалистов в~разных странах мира.

Игорь Алексеевич всегда уделял огромное внимание работе с~молодежью; более~50 его учеников защитили докторские и~кандидатские диссертации.

И.\,А.~Ушаков вел активную научно-про\-све\-ти\-тель\-скую деятельность. В~частности, он был одним из организаторов и~руководителей Московского кабинета качества и~надежности при Политехническом музее (целью этого Кабинета было оказание консультаций работникам промышленных предприятий и~чтение курсов лекций для инженеров, занимающихся проблемой надежности). Находясь в~США, И.\,А.~Ушаков создал международный ин\-тер\-нет-фо\-рум им.\ Б.\,В.~Гнеденко, объединивший около~400~видных специалистов по приложениям теории вероятностей и~математической статистики, преимущественно в~об\-ласти теории надежности и~анализа риска, из десятков стран мира; коллективным членов этого Форума является и~наш журнал. Цели Форума~--- содействие контактам между специалистами из разных стран, организация обмена профессиональными 
новостями и~информацией (новые публикации, предстоящие события и~др.). Также необходимо отметить большое число на\-уч\-но-по\-пу\-ляр\-ных работ, опубликованных И.\,А.~Ушаковым.

И.\,А.~Ушаков обладал большим личным обаянием, имел широкий круг интересов. Все знавшие И.\,А.~Ушакова всегда будут помнить его как замечательного ученого и~прекрасного человека.

\bigskip

Редакционный совет и редакционная коллегия журнала <<Информатика и~её применения>> 
выражают глубокие соболезнования родным и близким покойного, всем, кто его знал и~работал с~ним.



%\end{document}

%\include{IPPM-25}

\def\stat{cont-rus}
{%\hrule\par
%\vskip 7pt % 7pt
\vspace*{-24pt}
\raggedleft\Large \bf%\baselineskip=3.2ex
Правила подготовки рукописей  для публикации в журнале
<<Информатика~и~её~применения>> \vskip 8pt
    \hrule
    \par
\vskip 14pt plus 6pt minus 3pt }

\label{st\stat}

\def\tit{\ }

\def\aut{\ }
\def\auf{\ }

\def\leftkol{\ }
% Правила подготовки рукописей  для публикации в журнале
%<<Информатика и её применения>>

\def\rightkol{\ }
%Правила подготовки рукописей  для публикации в журнале
%<<Информатика и её применения>>}


\titele{\tit}{\aut}{\auf}{\leftkol}{\rightkol}


\vspace*{-60pt}
{ %\small

Журнал <<Информатика и её применения>>
публикует теоретические, обзорные и дискуссионные статьи,
посвященные научным исследованиям и разработкам в области
информатики и ее приложений.

Журнал издается на русском языке. По специальному решению
редколлегии отдельные статьи могут печататься на английском языке.

Тематика журнала охватывает следующие направления:
\begin{itemize}
\item теоретические основы информатики;\\[-15pt]
      \item
математические методы исследования сложных систем и процессов;\\[-15pt]
           \item
информационные системы и сети;\\[-15pt]
                \item
информационные технологии;\\[-15pt]
                     \item
архитектура и программное обеспечение вычислительных комплексов и сетей.\\[-15pt]
\end{itemize}


\noindent
\begin{enumerate}[1.]
\item В журнале печатаются статьи, содержащие результаты, ранее не опубликованные и
не предназначенные к одновременной публикации в других изданиях.

%Публикация не должна нарушать закон об авторских правах.
Публикация предоставленной автором(ами) рукописи не должна нарушать 
положений глав~69, 70 раздела~VII части~IV Гражданского кодекса, 
которые определяют права на результаты интеллектуальной деятельности 
и~средства индивидуализации, в~том числе авторские права, в~РФ.

Ответственность за нарушение авторских прав, в~случае предъявления претензий к~редакции журнала,  
несут авторы статей.



Направляя рукопись в редакцию, авторы сохраняют свои права на данную
рукопись и при этом передают учредителям и редколлегии журнала неисключительные права на
издание статьи на русском языке 
(или на языке статьи, если он отличен от рус\-ско\-го) и~на перевод ее на английский
язык, а~также на
ее распространение в России и за рубежом. 
Каждый автор должен представить в~редакцию подписанный 
с~его стороны <<Лицензионный договор о~передаче неисключительных прав 
на использование произведения>>, текст которого размещен по адресу 
{\sf http://www.ipiran.ru/publications/licence.doc}. 
Этот договор может быть пред\-став\-лен в~бумажном (в~2-х экз.)\ 
или в~электронном виде (отсканированная копия заполненного и~подписанного документа).




Редколлегия вправе запросить у авторов экспертное заключение о возможности
пуб\-ли\-ка\-ции пред\-став\-лен\-ной статьи в открытой печати.\\[-13.5pt]

\item К статье прилагаются данные автора (авторов) (см.\ п.~8). При наличии нескольких
авторов указывается фамилия автора, ответственного за переписку с редакцией.\\[-13.5pt]

\item Редакция журнала осуществляет экспертизу присланных статей в соответствии с
принятой в журнале процедурой рецензирования.

Возвращение рукописи на доработку не означает ее принятия к печати.

Доработанный вариант с ответом на замечания рецензента необходимо прислать в
редакцию.\\[-13.5pt]

\item Решение редколлегии о публикации статьи или ее отклонении сообщается авторам.

Редколлегия может также направить авторам текст рецензии на их статью. Дискуссия по
поводу отклоненных статей не ведется.\\[-13.5pt]

%\pagebreak

\item Редактура статей высылается авторам для просмотра. Замечания к редактуре должны
быть присланы авторами в кратчайшие сроки.\\[-13.5pt]

\item Рукопись предоставляется в электронном виде в форматах MS WORD (.doc или
.docx) или \LaTeX\  (.tex), дополнительно~--- в формате .pdf, на дискете, лазерном диске
или электронной почтой. Предоставление бумажной рукописи необязательно.\\[-13.5pt]

\item При подготовке рукописи в MS Word рекомендуется использовать следующие
настройки.

Параметры страницы:
формат~--- А4; ориентация~--- книжная; поля (см): внутри~--- 2,5, снаружи~--- 1,5,
сверху~--- 2, снизу~--- 2, от края до нижнего колонтитула~--- 1,3.

Основной текст: стиль~--- <<Обычный>>, шрифт~--- Times New Roman, размер~---
14~пунк\-тов, абзацный отступ~--- 0,5~см, 1,5~интервала, выравнивание~--- по ширине.

\pagebreak

\def\leftkol{Правила подготовки рукописей  для публикации в журнале
<<Информатика и её применения>>}

\def\rightkol{Правила подготовки рукописей  для публикации в журнале
<<Информатика и её применения>>}



Рекомендуемый объем рукописи~--- не свыше 10~страниц указанного формата.
При превышении указанного объема редколлегия вправе потребовать от 
автора сокращения объема рукописи.


Сокращения слов, помимо стандартных, не допускаются. Допускается минимальное
количество аббревиатур.


Все страницы рукописи нумеруются.

Шаблоны оформления представлены в интернете:

\noindent
 {\sf
http://www.ipiran.ru/journal/template\_iiep\_ssi\_2024.zip}\\[-14pt]

\item Статья должна содержать следующую информацию на {\bfseries\textit{русском и
английском языках}}:\\[-16pt]

\begin{itemize}
\item название статьи;\\[-15pt]
\item Ф.И.О.\ авторов, на английском можно только имя и фамилию;\\[-15pt]
\item место работы, с указанием почтового адреса организации и электронного адреса каждого
автора;\\[-15pt]
\item сведения об авторах, в соответствии с форматом, образцы которого
представлены на страницах:



\def\leftfootline{\small{\textbf{\thepage}
\hfill ИНФОРМАТИКА И ЕЁ ПРИМЕНЕНИЯ\ \ \ том\ 18\ \ \ выпуск\ 3\ \ \ 2024}
}%
 \def\rightfootline{\small{ИНФОРМАТИКА И ЕЁ ПРИМЕНЕНИЯ\ \ \ том\ 18\ \ \ выпуск\ 3\ \ \ 2024
\hfill \textbf{\thepage}}}



{\sf http://www.ipiran.ru/journal/issues/2013\_07\_01/authors.asp} и

{\sf http://www.ipiran.ru/journal/issues/2013\_07\_01\_eng/authors.asp};
\item аннотация (не менее 100~слов на каждом из языков). Аннотация~--- это краткое
резюме работы, которое может публиковаться отдельно. Она является основным
источником информации в~ин\-фор\-ма\-ци\-он\-ных системах и базах данных. Английская
аннотация должна быть оригинальной, может не быть дословным переводом русского
текста и должна быть написана хорошим английским языком. В~аннотации не должно
быть ссылок на литературу и, по возможности, формул;\\[-15pt]
\item ключевые слова~--- желательно из принятых в мировой
на\-уч\-но-тех\-ни\-че\-ской литературе тематических тезаурусов. Предложения не
могут быть ключевыми словами;\\[-15pt]
\item источники финансирования работы (ссылки на гранты, проекты,
поддерживающие организации и~т.\,п.).
\end{itemize}



%\pagebreak

\item  Требования к спискам литературы.\\[-14pt]

Ссылки на литературу в тексте статьи нумеруются (в квадратных скобках) и
располагаются в каждом из списков литературы в порядке  первых упоминаний. Если источник имеет DOI и/или EDN,
то их необходимо указывать.

Списки литературы представляются в двух вариантах:\\[-14pt]


\noindent
\begin{enumerate}[(1)]
\item \textbf{Список литературы к русскоязычной части}. Русские и английские
работы~---  на языке и в алфавите оригинала;\\[-14.5pt]
\item  \textbf{References}. Русские работы и работы на других языках~--- в латинской
транслитерации с переводом на английский язык; английские работы и работы на других
языках~--- на языке оригинала.
\end{enumerate}

Необходимо для составления списка ``References'' пользоваться размещенной на сайте
{\sf http://www. translit.net/ru/bgn/} бесплатной программой транслитерации русского
 текста в~латиницу. %, при этом в~за\-клад\-ке <<варианты\ldots>> следует выбратьопцию BGN.

Список литературы ``References'' приводится полностью отдельным блоком, повторяя все
позиции из списка литературы к русскоязычной части, независимо от того, имеются или
нет в нем иностранные источники. Если в списке литературы к русскоязычной части есть
ссылки на иностранные публикации, набранные латиницей, они полностью повторяются в
списке ``References''.

Ниже приведены примеры ссылок на различные виды публикаций в списке ``References''.

\def\leftfootline{\small{\textbf{\thepage}
\hfill ИНФОРМАТИКА И ЕЁ ПРИМЕНЕНИЯ\ \ \ том\ 18\ \ \ выпуск\ 3\ \ \ 2024}
}%
 \def\rightfootline{\small{ИНФОРМАТИКА И ЕЁ ПРИМЕНЕНИЯ\ \ \ том\ 18\ \ \ выпуск\ 3\ \ \ 2024
\hfill \textbf{\thepage}}}

{\small

\noindent
\textbf{Описание статьи из журнала:}

\Aue{Zagurenko, A.\,G., V.\,A.~Korotovskikh, A.\,A.~Kolesnikov, A.\,V.~Timonov, and D.\,V.~Kardymon}. 2008.
Tekhniko-ekonomicheskaya optimizatsiya dizayna gidrorazryva plasta [Technical and
economic optimization of the design
of hydraulic fracturing]. \textit{Neftyanoe hozyaystvo} [\textit{Oil Industry}] 11:54--57.

\Aue{Zhang, Z., and D.~Zhu}. 2008. Experimental research on the localized
electrochemical micromachining. \textit{Russ. J.~Electrochem.}  44(8):926--930.
{\sf doi:10.1134/S1023193508080077}.

\noindent
\textbf{Описание статьи из электронного журнала:}

\Aue{Swaminathan, V., E.~Lepkoswka-White, and B.\,P.~Rao}. 1999. Browsers or buyers in cyberspace? An
investigation of electronic factors influencing electronic exchange. \textit{JCMC}
5(2). Available at: {\sf http://www.ascusc.org/jcmc/vol5/issue2/} (accessed April~28, 2011).

\def\leftkol{Правила подготовки рукописей  для публикации в журнале
<<Информатика и её применения>>}

\def\rightkol{Правила подготовки рукописей  для публикации в журнале
<<Информатика и её применения>>}


\noindent
\textbf{Описание статьи из продолжающегося издания (сборника трудов):}

\Aue{Astakhov, M.\,V., and T.\,V.~Tagantsev}. 2006. Eksperimental'noe
issledovanie prochnosti soedineniy ``stal'--kompozit'' [Experimental study of
the strength of joints ``steel--composite'']. \textit{Trudy MGTU
``Matematicheskoe modelirovanie slozhnykh tekh\-ni\-che\-skikh sistem''}
[\textit{Bauman MSTU ``Mathematical Modeling of Complex Technical
Systems'' Proceedings}]. 593:125--130.


\pagebreak



\noindent
\textbf{Описание материалов конференций:}

\Aue{Usmanov, T.\,S., A.\,A.~Gusmanov, I.\,Z.~Mullagalin, R.\,Ju.~Muhametshina, A.\,N.~Chervyakova, and
A.\,V.~Sveshnikov}. 2007. Osobennosti proektirovaniya razrabotki mestorozhdeniy
s primeneniem gidrorazryva
plasta [Features of the design of field development with the use of hydraulic fracturing].
\textit{Trudy 6-go
Mezhdu\-na\-rod\-no\-go Simpoziuma ``Novye resursosberegayushchie tekhnologii nedropol'zovaniya i povysheniya
neftegazootdachi''} [\textit{6th  Symposium (International) ``New Energy Saving Subsoil Technologies and
the Increasing of the Oil and Gas Impact'' Proceedings}]. Moscow. 267--272.



\def\leftfootline{\small{\textbf{\thepage}
\hfill ИНФОРМАТИКА И ЕЁ ПРИМЕНЕНИЯ\ \ \ том\ 18\ \ \ выпуск\ 3\ \ \ 2024}
}%
 \def\rightfootline{\small{ИНФОРМАТИКА И ЕЁ ПРИМЕНЕНИЯ\ \ \ том\ 18\ \ \ выпуск\ 3\ \ \ 2024
\hfill \textbf{\thepage}}}



\noindent
\textbf{Описание книги (монографии, сборники):}



Lindorf, L.\,S., and L.\,G.~Mamikoniants, eds. 1972.
\textit{Ekspluatatsiya turbogeneratorov s neposredstvennym
okhlazhdeniem} [\textit{Operation of turbine generators with direct cooling}].
Moscow: Energy Publs. 352~p.


\Aue{Latyshev, V.\,N.} 2009. \textit{Tribologiya rezaniya. Kn.~1: Friktsionnye protsessy
pri rezanii metallov}
[\textit{Tribology of cutting. Vol.~1: Frictional processes in metal cutting}]. Ivanovo: Ivanovskii
State Univ. 108~p.

\def\leftkol{Правила подготовки рукописей  для публикации в журнале
<<Информатика и её применения>>}

\def\rightkol{Правила подготовки рукописей  для публикации в журнале
<<Информатика и её применения>>}

\noindent
\textbf{Описание переводной книги}
(в списке литературы к русскоязычной части необходимо указать:~/ Пер.\ с англ.~---
после названия книги, а в конце ссылки указать оригинал книги в круглых скобках):
\begin{enumerate}[1.]
\item  В русскоязычной части:

\def\leftfootline{\small{\textbf{\thepage}
\hfill ИНФОРМАТИКА И ЕЁ ПРИМЕНЕНИЯ\ \ \ том\ 18\ \ \ выпуск\ 3\ \ \ 2024}
}%
 \def\rightfootline{\small{ИНФОРМАТИКА И ЕЁ ПРИМЕНЕНИЯ\ \ \ том\ 18\ \ \ выпуск\ 3\ \ \ 2024
\hfill \textbf{\thepage}}}

\Au{Тимошенко С.\,П., Янг Д.\,Х., Уивер~У.}
Колебания в инженерном деле~/ Пер.\ с англ.~--- М.: Машиностроение, 1985. 472~с.
(\Au{Timoshenko~S.\,P., Young~D.\,H., Weaver~W.}
Vibration problems in engineering.~--- 4th ed.~--- New York, NY, USA: Wiley, 1974. 521~p.)\\[-13.5pt]
\item  В англоязычной части:

\Aue{Timoshenko, S.\,P., D.\,H.~Young, and W.~Weaver}.
1974. \textit{Vibration problems in engineering}. 4th ed. New York: 
Wiley. 521~p.
\end{enumerate}

\vspace*{-3pt}


\noindent
\textbf{Описание неопубликованного документа:}


\Aue{Latypov, A.\,R., M.\,M.~Khasanov, and V.\,A.~Baikov}.
2004 (unpubl.). Geologiya i~dobycha (NGT GiD) [Geology and production (NGT GiD)]. Certificate on official registration of the computer program
No.\,2004611198. 

\noindent
\textbf{Описание интернет-ресурса:}


Pravila tsitirovaniya istochnikov [Rules for the citing of sources]. Available at: {\sf
http://www.scribd.com/doc/1034528/} (accessed February~7, 2011).

%\pagebreak

\noindent
\textbf{Описание диссертации или автореферата диссертации:}

\Aue{Semenov, V.\,I.}
2003. Matematicheskoe modelirovanie plazmy v sisteme kompaktnyy tor [Mathematical
modeling of the plasma in the compact torus].  Moscow.  D.Sc.\ Diss. 272~p.

\Aue{Kozhunova, O.\,S.} 2009. Tekhnologiya razrabotki semanticheskogo
slovarya informatsionnogo monitoringa [Technology of development of
semantic dictionary of information monitoring system].  Moscow: IPI RAN. PhD Thesis. 23~p.


\noindent
\textbf{Описание ГОСТа:}

GOST 8.586.5-2005. 2007. Metodika vypolneniya izmereniy. Izmerenie raskhoda i~kolichestva zhidkostey i~gazov
s~pomoshch'yu standartnykh suzhayushchikh ustroystv [Method of measurement.
Measurement of flow rate and volume of liquids and gases by means of orifice devices]. Moscow:
Standardinform  Publs. 10~p.

\noindent
\textbf{Описание патента:}

\Aue{Bolshakov, M.\,V., A.\,V.~Kulakov, A.\,N.~Lavrenov, and M.\,V.~Palkin}.
2006. Sposob orientirovaniya po krenu letatel'nogo
apparata s opti\-che\-skoy golovkoy
samonavedeniya [The way to orient on the roll of aircraft with optical homing head].
Patent RF No.\,2280590.
}

\item Присланные в редакцию материалы авторам не возвращаются.\\[-13.5pt]

\item При отправке файлов по электронной почте просим придерживаться следующих
правил:
\begin{itemize}
\item указывать в поле subject (тема) название журнала и фамилию автора;\\[-13.5pt]
\item указывать в тексте письма название статьи, авторов и~журнал, в~который направляется статья;\\[-13.5pt]
\item использовать attach (присоединение);\\[-13.5pt]
\item в состав электронной версии статьи должны входить: файл, содержащий текст
статьи, и файл(ы), содержащий(е) иллюстрации.\\[-13.5pt]
\end{itemize}

\item Журнал <<Информатика и её применения>> является некоммерческим изданием.
Плата за публикацию не взимается, гонорар авторам не выплачивается.
\end{enumerate}



\def\leftfootline{\small{\textbf{\thepage}
\hfill ИНФОРМАТИКА И ЕЁ ПРИМЕНЕНИЯ\ \ \ том\ 18\ \ \ выпуск\ 3\ \ \ 2024}
}%
 \def\rightfootline{\small{ИНФОРМАТИКА И ЕЁ ПРИМЕНЕНИЯ\ \ \ том\ 18\ \ \ выпуск\ 3\ \ \ 2024
\hfill \textbf{\thepage}}}


\vspace*{-1mm}

\begin{center}

\textbf{Адрес редакции журнала <<Информатика и её применения>>:} \\




Москва 119333, ул.~Вавилова, д.~44, корп.~2, ФИЦ ИУ РАН\\[-10pt]

\

Тел.: +7\,(499)\,135-86-92\ \ Факс:  +7\,(495)\,930-45-05\\[-10pt]

 \

e-mail:   {\sf iiep@frccsc.ru} (Стригина Светлана Николаевна)\\[-10pt]

\

{\sf http://www.ipiran.ru/journal/issues/}
\end{center}
}


\def\leftkol{Правила подготовки рукописей  для публикации в журнале
<<Информатика и её применения>>}

\def\rightkol{Правила подготовки рукописей  для публикации в журнале
<<Информатика и её применения>>}


\def\leftfootline{\small{\textbf{\thepage}
\hfill ИНФОРМАТИКА И ЕЁ ПРИМЕНЕНИЯ\ \ \ том\ 18\ \ \ выпуск\ 3\ \ \ 2024}
}%
 \def\rightfootline{\small{ИНФОРМАТИКА И ЕЁ ПРИМЕНЕНИЯ\ \ \ том\ 18\ \ \ выпуск\ 3\ \ \ 2024
\hfill \textbf{\thepage}}} 
\def\stat{podg-e}
{%\hrule\par
%\vskip 7pt % 7pt
\vspace*{-24pt}
\raggedleft\Large \bf%\baselineskip=3.2ex
Requirements for manuscripts submitted to Journal
``Informatics~and~Applications'' \vskip 8pt
    \hrule
    \par
\vskip 21pt plus 6pt minus 3pt }

\label{st\stat}

\def\tit{\ }

\def\aut{\ }
\def\auf{\ }

\def\leftkol{\ }

\def\rightkol{\ }
%Requirements for manuscripts submitted to Journal
%``Informatics~and~Applications''}

\titele{\tit}{\aut}{\auf}{\leftkol}{\rightkol}

\def\leftfootline{\small{\textbf{\thepage}
\hfill INFORMATIKA I EE PRIMENENIYA~--- INFORMATICS AND APPLICATIONS\ \ \ 2019\
\ \ volume~13\ \ \ issue\ 4}
}%
 \def\rightfootline{\small{INFORMATIKA I EE PRIMENENIYA~--- INFORMATICS AND APPLICATIONS\ \ \ 2019\ \ \ volume~13\ \ \ issue\ 4
\hfill \textbf{\thepage}}}

\vspace*{-60pt}

{\small

\noindent
Journal ``Informatics and Applications'' (Inform.\ Appl.)
publishes theoretical, review, and discussion
articles on the research and development in the
field of informatics and its applications.

The journal is published in Russian.
By a special decision of the editorial
board, some articles can be published in English.


The topics covered include the following areas:
\begin{itemize}
               \item
     theoretical fundamentals of informatics; \\[-14pt]
\item
mathematical methods for studying complex systems and processes; \\[-14pt]
\item
information systems and networks;\\[-14pt]
\item
information technologies; and \\[-14pt]
\item
architecture and software of computational complexes and networks. \\[-14pt]
\end{itemize}

\noindent
\begin{enumerate}[1.]
\item The Journal publishes original articles which have not been published before and are not
intended for simultaneous publication in other editions. An article submitted to the Journal must not violate the
Copyright law. Sending the manuscript to the Editorial Board, the authors retain all rights of the
owners of the manuscript and transfer the nonexclusive rights to publish the article in Russian
(or the language of the article, if not Russian) and its distribution in Russia and abroad to the
Founders and the Editorial Board. Authors should submit a letter to the Editorial Board in the
following form:

{\bfseries\textit{Agreement on the transfer of rights to publish:}}

``\textit{We, the undersigned authors of the manuscript ``\ldots'', pass to the
Founder and the Editorial Board of the Journal ``Informatics and Applications''
the nonexclusive right to publish the manuscript of the article in Russian (or
in English) in both print and electronic versions of the Journal. We affirm
that this publication does not violate the Copyright of other persons or
organizations.}

\textit{Author(s) signature(s): (name(s), address(es), date).}

This agreement should be submitted in paper form or in the form of a scanned copy (signed by
the authors).


%The Editorial Board has the right to request from the authors an official expert conclusion that
%the submitted article has no secret data prohibited for publication. \\[-13.5pt]
\item
A submitted article should be attached with \textbf{the data on the author(s)} (see item~8). If
there are several authors, the contact person should be indicated who is responsible for
correspondence with the Editorial Board and other authors about revisions and final approval
of the proofs.\\[-13.5pt]

\item The Editorial Board of the Journal examines the article according to the established
reviewing procedure. If the authors receive their article for correction after reviewing, it does not
mean that the article is approved for publication. The corrected article should be sent to the
Editorial Board for the subsequent review and approval.\\[-13.5pt]

\item The decision on the article publication or its rejection is communicated to the authors. The
Editorial Board may also send the reviews on the submitted articles to the authors. Any
discussion upon the rejected articles is not possible.\\[-13.5pt]

\item The edited articles will be sent to the authors for proofread. The comments of the authors
to the edited text of the article should be sent to the Editorial Board as soon as possible.\\[-13.5pt]

\item The manuscript of the article should be presented electronically in the MS WORD (.doc or
.docx) or \LaTeX\ (.tex) formats, and additionally in the .pdf format. All documents
 may be sent
by e-mail or provided on a CD or diskette. A~hard copy submission is not necessary.\\[-13.5pt]

\item The recommended typesetting instructions for manuscript.

Pages parameters: format A4, portrait orientation, document margins (cm): left~--- 2.5, right~---
1.5, above~--- 2.0, below~--- 2.0, footer 1.3.

Text: font~---Times New Roman, font size~--- 14, paragraph indent~--- 0.5, line spacing~--- 1.5,
justified alignment.

The recommended manuscript size: not more than 15~pages of the specified format.
If the specified size exceeded, the editorial board is entitled to require the author
to reduce the manuscript.

Use only standard abbreviations. Avoid  abbreviations in the title and
abstract. The full term for which an abbreviation stands should precede
its first use in the text unless it is a standard unit of measurement.

All pages of the manuscript should be numbered.

The templates for the manuscript typesetting are presented on site: {\sf
http://www.ipiran.ru/journal/template.doc}.\\[-13.5pt]


%\def\leftkol{Requirements for manuscripts submitted to Journal
%``Informatics~and~Applications''}

\item The articles should enclose data both in \textbf{Russian and English}:
\begin{itemize}
\item title;\\[-13.5pt]
\item author's name and surname;\\[-13.5pt]
\item affiliation~--- organization, its address with ZIP code, city, country, and
official e-mail address;\\[-13.5pt]
\item data on authors according to the format: (see site)

{\sf http://www.ipiran.ru/journal/issues/2013\_07\_01/authors.asp}  and

{\sf  http://www.ipiran.ru/journal/issues/2013\_07\_01\_eng/authors.asp};\\[-13.5pt]

\pagebreak

\def\leftfootline{\small{\textbf{\thepage}
\hfill INFORMATIKA I EE PRIMENENIYA~--- INFORMATICS AND APPLICATIONS\ \ \ 2019\
\ \ volume~13\ \ \ issue\ 4}
}%
 \def\rightfootline{\small{INFORMATIKA I EE PRIMENENIYA~--- INFORMATICS AND APPLICATIONS\ \ \ 2019\ \ \ volume~13\ \ \ issue\ 4
\hfill \textbf{\thepage}}}


%\def\leftkol{Requirements for manuscripts submitted to Journal
%``Informatics~and~Applications''}

%\def\rightkol{Requirements for manuscripts submitted to Journal
%``Informatics~and~Applications''}



\item abstract (not less than 100 words) both in Russian and in English. Abstract is a short
summary of the article that can be published separately. The abstract is the
main source of information on the article and it could be included in leading information
systems and data bases. The abstract in English has to be an original text and should
not be an exact translation of the Russian one. Good English is required.
In abstracts, avoid references and formulae;\\[-13.5pt]
\item indexing is performed on the basis of keywords. The use of keywords from the
internationally accepted thematic Thesauri is recommended.

%\def\leftkol{Requirements for manuscripts submitted to Journal
%``Informatics~and~Applications''}

%\def\rightkol{Requirements for manuscripts submitted to Journal
%``Informatics~and~Applications''}

Important! Keywords must not be sentences;
\item Acknowledgments.
\end{itemize}

\item References. Russian references have to be presented both in English translation and Latin
transliteration (refer {\sf http://www.translit.net/ru/bgn/}).

Please take into account the following examples of Russian references appearance:

\noindent
\textbf{Article in journal:}

\Aue{Zhang, Z., and D.~Zhu}. 2008. Experimental research on the localized electrochemical
micromachining.
\textit{Rus. J.~Electrochem.}  44(8):926--930. {\sf doi:10.1134/S1023193508080077}.


\noindent
\textbf{Journal article in electronic format:}

\Aue{Swaminathan, V., E.~Lepkoswka-White, and B.\,P.~Rao}. 1999. Browsers or buyers in
cyberspace? An
investigation of electronic factors influencing electronic exchange. \textit{JCMC}
5(2). Available at: {\sf http://www.ascusc.org/jcmc/vol5/issue2/} (accessed April~28, 2011).




\noindent
\textbf{Article from the continuing publication (collection of works, proceedings):}

\Aue{Astakhov, M.\,V., and T.\,V.~Tagantsev}. 2006. Eksperimental'noe
issledovanie prochnosti soedineniy ``stal'--kompozit'' [Experimental study of
the strength of joints ``steel--composite'']. \textit{Trudy MGTU
``Matematicheskoe modelirovanie slozhnykh tekh\-ni\-che\-skikh sistem''}
[\textit{Bauman MSTU ``Mathematical Modeling of Complex Technical
Systems'' Proceedings}]. 593:125--130.

\def\leftfootline{\small{\textbf{\thepage}
\hfill INFORMATIKA I EE PRIMENENIYA~--- INFORMATICS AND APPLICATIONS\ \ \ 2019\
\ \ volume~13\ \ \ issue\ 4}
}%
 \def\rightfootline{\small{INFORMATIKA I EE PRIMENENIYA~--- INFORMATICS AND APPLICATIONS\ \ \ 2019\ \ \ volume~13\ \ \ issue\ 4
\hfill \textbf{\thepage}}}

\def\leftkol{Requirements for manuscripts submitted to Journal
``Informatics~and~Applications''}

\def\rightkol{Requirements for manuscripts submitted to Journal
``Informatics~and~Applications''}

\noindent
\textbf{Conference proceedings:}

\Aue{Usmanov, T.\,S., A.\,A.~Gusmanov, I.\,Z.~Mullagalin, R.\,Ju.~Muhametshina,
A.\,N.~Chervyakova, and
A.\,V.~Sveshnikov}. 2007. Osobennosti proektirovaniya razrabotki mestorozhdeniy
s primeneniem gidrorazryva
plasta [Features of the design of field development with the use of hydraulic fracturing].
\textit{Trudy 6-go
Mezhdu\-na\-rod\-no\-go Simpoziuma ``Novye resursosberegayushchie tekhnologii
nedropol'zovaniya i povysheniya
neftegazootdachi''} [\textit{6th  Symposium (International) ``New Energy Saving Subsoil
Technologies and
the Increasing of the Oil and Gas Impact'' Proceedings}]. Moscow. 267--272.


\noindent
\textbf{Books and other monographs:}




Lindorf, L.\,S., and L.\,G.~Mamikoniants, eds. 1972.
\textit{Ekspluatatsiya turbogeneratorov s neposredstvennym
okhlazhdeniem} [\textit{Operation of turbine generators with direct cooling}].
Moscow: Energy Publs. 352~p.


%\Aue{Latyshev, V.\,N.} 2009. \textit{Tribologiya rezaniya. Kn.~1: Frikcionnye prosessy
%pri rezanii metallov}
%[\textit{Tribology of cutting. Vol.~1: Frictional processes in metal cutting}]. Ivanovo: Ivanovskii
%State Univ. 108~p.


%\noindent
%\textbf{Unpublished material:}

%\Aue{Latypov, A.\,R., M.\,M.~Khasanov, and V.\,A.~Baikov}.
%2004. Geology and production (NGT GiD). Certificate on official registration of the computer
%program
%No.\,2004611198. (In Russian, unpubl.)

%\noindent
%\textbf{Internet-source:}

%APA Style. 2011. Available at: {\sf http://www.apastyle.org/apa-style-help.aspx} (accessed
%February~5, 2011).

%Pravila citirovaniya istochnikov [Rules for the citing of sources]. Available at: {\sf
%http://www.scribd.com/doc/1034528/} (accessed February~7, 2011).


\noindent
\textbf{Dissertation and Thesis:}

%\Aue{Semenov, V.\,I.}
%2003. Matematicheskoe modelirovanie plazmy v sisteme kompaktnyy tor. [Mathematical
%modeling of the plasma in the compact torus]. D.Sc.\ Diss. Moscow. 272~p.

\Aue{Kozhunova, O.\,S.} 2009. Tekhnologiya razrabotki semanticheskogo
slovarya informatsionnogo monitoringa [Technology of development of
semantic dictionary of information monitoring system]. PhD Thesis. Moscow: IPI RAN. 23~p.


\noindent
\textbf{State standards and patents:}

GOST 8.586.5-2005. 2007. Metodika vypolneniya izmereniy. Izmerenie raskhoda i~kolichestva
zhidkostey i gazov 
s~pomoshch'yu standartnykh suzhayushchikh ustroystv [Method of measurement.
Measurement of flow rate and volume of liquids and gases by means of orifice devices]. M.:
Standardinform
Publs. 10~p.

%\noindent
%\textbf{Patent:}

\Aue{Bolshakov, M.\,V., A.\,V.~Kulakov, A.\,N.~Lavrenov, and M.\,V.~Palkin}.
2006. Sposob orientirovaniya po krenu letatel'nogo
apparata s opti\-che\-skoy golovkoy
samonavedeniya [The way to orient on the roll of aircraft with optical homing head].
Patent RF No.\,2280590.

References in Latin transcription are presented in the original language.

References in the text are numbered according to the order of their
first appearance; the number is
placed in square brackets. All items from the reference list should be
cited.\\[-13.5pt]

\item Manuscripts and additional materials are not returned to Authors by the Editorial Board.\\[-13.5pt]

\item Submissions of files by e-mail must include:\\[-13.5pt]
\begin{itemize}
\item   the journal title and author's name in the ``Subject'' field; \\[-13.5pt]
\item   an article and additional materials have to be attached using the ``attach'' function;\\[-13.5pt]
\item   an electronic version of the article should contain the file with the text and a separate file
with figures.\\[-13.5pt]
\end{itemize}

\item ``Informatics and Applications'' journal is not a profit publication. There are no
charges for the authors as well as there are no royalties.\\[-13.5pt]
\end{enumerate}

\def\leftfootline{\small{\textbf{\thepage}
\hfill INFORMATIKA I EE PRIMENENIYA~--- INFORMATICS AND APPLICATIONS\ \ \ 2019\
\ \ volume~13\ \ \ issue\ 4}
}%
 \def\rightfootline{\small{INFORMATIKA I EE PRIMENENIYA~--- INFORMATICS AND APPLICATIONS\ \ \ 2019\ \ \ volume~13\ \ \ issue\ 4
\hfill \textbf{\thepage}}}

\def\leftkol{Requirements for manuscripts submitted to Journal
``Informatics~and~Applications''}

\def\rightkol{Requirements for manuscripts submitted to Journal
``Informatics~and~Applications''}


%\vspace*{5mm}


\begin{center}
\textbf{Editorial Board address:} \\

%ABOUT AUTHORS



FRC CSC RAS, 44, block~2, Vavilov Str., Moscow 119333, Russia\\[-10pt]

\

Ph.: +7\,(499)\,135\,86\,92,\ \ Fax: +7\,(495)\,930\,45\,05\\[-10pt]

\

 e-mail: {\sf rust@ipiran.ru} (to Prof.\ Rustem Seyful-Mulyukov)\\[-10pt]

\

 {\sf http://www.ipiran.ru/english/journal.asp}
\end{center}
 }
%\thispagestyle{myheadings}

\def\leftkol{Requirements for manuscripts submitted to Journal
``Informatics~and~Applications''}

\def\rightkol{Requirements for manuscripts submitted to Journal
``Informatics~and~Applications''}

\def\leftfootline{\small{\textbf{\thepage}
\hfill INFORMATIKA I EE PRIMENENIYA~--- INFORMATICS AND APPLICATIONS\ \ \ 2019\
\ \ volume~13\ \ \ issue\ 4}
}%
 \def\rightfootline{\small{INFORMATIKA I EE PRIMENENIYA~--- INFORMATICS AND APPLICATIONS\ \ \ 2019\ \ \ volume~13\ \ \ issue\ 4
\hfill \textbf{\thepage}}}

 \label{end\stat}

\newpage

%\vspace*{-60pt} {\small
{\baselineskip=9.1pt
\section*{Правила подготовки рукописей статей для публикации в журнале
<<Информатика и её применения>>}

\thispagestyle{empty}

 Журнал <<Информатика и её применения>> публикует
теоретические, обзорные и дискуссионные статьи, посвященные научным
исследованиям и разработкам в области информатики и ее приложений. Журнал
издается на русском языке. По специальному решению редколлегии отдельные статьи,
в виде исключения, могут печататься на английском языке.
Тематика журнала охватывает следующие направления:
\begin{itemize}
\item теоретические основы информатики; %\\[-13.5pt]
\item математические методы исследования сложных систем и процессов; %\\[-13.5pt]
\item информационные системы и сети; %\\[-13.5pt]
\item информационные технологии; %\\[-13.5pt]
\item архитектура и программное
обеспечение вычислительных комплексов и сетей.
\end{itemize}
\begin{enumerate}
\item В журнале печатаются результаты, ранее не
опубликованные и не предназначенные к одновременной публикации в других
изданиях. Публикация не должна нарушать закон об авторских правах. Направляя
свою рукопись в редакцию, авторы автоматически передают учредителям и
редколлегии неисключительные права на издание данной статьи на русском языке и
на ее распространение в России и за рубежом. При этом за авторами сохраняются
все права как собственников данной рукописи. В связи с этим авторами должно
быть представлено в редакцию письмо в следующей форме:
Соглашение о передаче права на публикацию:

\textit{<<Мы, нижеподписавшиеся, авторы рукописи <<$\qquad\qquad$>>, передаем
учредителям и редколлегии журнала <<Информатика и её применения>>
неисключительное право опубликовать данную рукопись статьи на русском языке как
в печатной, так и в электронной версиях журнала. Мы подтверждаем, что данная
публикация не нарушает авторского права других лиц или организаций. Подписи
авторов: (ф.\,и.\,о., дата, адрес)>>.}

Указанное соглашение может быть представлено 
как в бумажном виде, так и в виде отсканированной копии (с подписями авторов).


Редколлегия вправе запросить у авторов экспертное заключение о возможности
опубликования представленной статьи в открытой печати. %\\[-13.5pt]
\item Статья
подписывается всеми авторами. На отдельном листе представляются данные автора
(или всех авторов): фамилия, полные имя и отчество, телефон, факс, e-mail,
почтовый адрес. Если работа выполнена несколькими авторами, указывается фамилия
одного из них, ответственного за переписку с редакцией. %\\[-13.5pt]
\item Редакция журнала
осуществляет самостоятельную экспертизу присланных статей. Возвращение рукописи
на доработку не означает, что статья уже принята к печати. Доработанный вариант
с ответом на замечания рецензента необходимо прислать в редакцию. %\\[-13.5pt]
\item Решение
редакционной коллегии о принятии статьи к печати или ее отклонении сообщается
авторам. Редколлегия не обязуется направлять рецензию авторам отклоненной
статьи. %\\[-13.5pt]
\item Корректура статей высылается авторам для просмотра. Редакция
просит авторов присылать свои замечания в кратчайшие сроки. %\\[-13.5pt]
\item При
подготовке рукописи в MS Word рекомендуется использовать следующие настройки.
Параметры страницы: формат~--- А4; ориентация~--- книжная; поля (см): внутри~---
2,5, снаружи~--- 1,5, сверху~--- 2, снизу~--- 2, от края до нижнего
колонтитула~--- 1,3. Основной текст: стиль~--- <<Обычный>>: шрифт Times New
Roman, размер 14~пунктов, абзацный отступ~--- 0,5~см, 1,5 интервала,
выравнивание~--- по ширине. Рекомендуемый объем рукописи~--- не свыше
25~страниц указанного формата. Ознакомиться с шаблонами, содержащими примеры
оформления, можно по адресу в Интернете:
\textsf{http://www.ipiran.ru/journal/template.doc}.
\item К рукописи, предоставляемой в 2-х
экземплярах, обязательно прилагается электронная версия статьи (как правило, в
форматах MS WORD (.doc) или \LaTeX\ (.tex), а также~--- дополнительно~--- в
формате .pdf) на дискете, лазерном диске или по электронной почте. Сокращения
слов, кроме стандартных, не применяются. Все страницы рукописи должны быть
пронумерованы. %\\[-13.5pt]
\item Статья должна содержать следующую информацию на русском и
английском языках: название, Ф.И.О. авторов, места работы авторов и их
электронные адреса, подробные сведения об авторах, оформленные в соответствии с форматом, 
определяемым файлами {\sf http://www.ipiran.ru/journal/issues/2011\_05\_01/authors.asp} и 
{\sf http://www.ipiran.ru/journal/issues/2011\_01\_eng/authors.asp},
аннотация (не более 100~слов), ключевые слова. Ссылки на
литературу в тексте статьи нумеруются (в квадратных скобках) и располагаются в
порядке их первого упоминания. В~списке литературы не должно быть позиций, на которые нет ссылки в тексте статьи.
Все фамилии авторов, заглавия статей, названия
книг, конференций и~т.\,п.\ даются на языке оригинала, если этот язык
использует кириллический или латинский алфавит. %\\[-13.5pt]
\item Присланные в редакцию материалы авторам не возвращаются.
\item При отправке файлов по электронной
почте просим придерживаться следующих правил:
\begin{itemize}
\item указывать в поле subject (тема) название журнала и фамилию автора; %\\[-13.5pt]
\item использовать attach (присоединение); %\\[-13.5pt]
\item в случае больших объемов информации возможно
использование общеизвестных архиваторов (ZIP, RAR); %\\[-13.5pt]
\item в состав электронной версии статьи должны входить: файл, содержащий текст статьи, и файл(ы),
содержащий(е) иллюстрации. %\\[-13.5pt]
\end{itemize}
\item Журнал <<Информатика и её применения>> является некоммерческим изданием. 
Плата за публикацию с авторов не взимается, гонорар авторам не выплачивается.
\end{enumerate}
\thispagestyle{empty}
\textbf{Адрес редакции:} Москва 119333,
ул.~Вавилова, д.~44, корп.~2, ИПИ РАН\\
\hphantom{\textbf{Адрес редакции:} }Тел.: +7 (499) 135-86-92\ \
Факс:  +7 (495) 930-45-05\ \  E-mail:   rust@ipiran.ru }
}

%\include{ipi-ind}

%\tableofcontents

\end{document}

%\tableofcontents

%\end{document}

%\tableofcontents


\end{document}

\newcommand{\Ack}{\subsection*{\protect\large\bf Acknowledgments}}