\def\stat{zaharova}

\def\tit{ОЦЕНКА УРОВНЯ ЗНАЧИМОСТИ КРИТЕРИЯ ШУИРМАННА ДЛЯ~ПРОВЕРКИ ГИПОТЕЗЫ 
БИОЭКВИВАЛЕНТНОСТИ ПРИ~НАЛИЧИИ ПРОПУЩЕННЫХ ДАННЫХ$^*$}

\def\titkol{Оценка уровня значимости критерия Шуирманна для~проверки гипотезы 
биоэквивалентности} % при~наличии пропущенных данных}

\def\aut{Т.\,В.~Захарова$^1$,  А.\,А.~Тархов$^2$}

\def\autkol{Т.\,В.~Захарова,  А.\,А.~Тархов}

\titel{\tit}{\aut}{\autkol}{\titkol}

\index{Захарова Т.\,В.}
\index{Тархов А.\,А.}
\index{Zakharova T.\,V.}
\index{Tarkhov A.\,A.}


{\renewcommand{\thefootnote}{\fnsymbol{footnote}} \footnotetext[1]
{Работа выполнена при поддержке РФФИ (проект 18-07-00252).}}


\renewcommand{\thefootnote}{\arabic{footnote}}
\footnotetext[1]{Московский государственный университет им.\ М.\,В.~Ломоносова,
факультет вычислительной математики и~кибернетики; Институт 
проблем информатики Федерального исследовательского центра <<Информатика и~управление>> 
Российской академии наук, \mbox{tvzaharova@mail.ru}}
\footnotetext[2]{Московский государственный университет им.\ М.\,В.~Ломоносова, 
факультет вычислительной математики и~кибернетики, \mbox{alexeytarkhov@gmail.com}}

%\vspace*{-2pt}



\Abst{Задача проверки гипотезы биоэквивалентности имеет важное 
значение в~фармакокинетике. С~ее помощью принимают решение об 
эквивалентности воспроизведенного лекарственного препарата референтному 
лекарственному препарату. Одна из проблем исследований биоэквивалентности~--- 
наличие пропущенных данных. Так как объем исследуемых данных достаточно мал,
 то удаление данных субъекта, у~которого есть пропущенные данные, нежелательно. 
 Поэтому стоит задача оценить влияние пропущенных данных при принятии решения
  о~биоэквивалентности, а~именно: дать оценку уровня значимости.
Основным методом проверки гипотезы биоэквивалентности является 
процедура двух односторонних тес\-тов Шуирманна. В~статье дана оценка уровня 
значимости данной процедуры при наличии пропущенных данных. 
В~явном виде получена компонента оценки уровня значимости, зависящая от уровня 
полноты данных.}

\KW{биоэквивалентность; уровень значимости; ошибка первого рода;
 пропущенные данные;  процедура двух односторонних тестов Шуирманна}


\DOI{10.14357/19922264190309} 
  
\vspace*{3pt}


\vskip 10pt plus 9pt minus 6pt

\thispagestyle{headings}

\begin{multicols}{2}

\label{st\stat}

\section{Введение}

\vspace*{-3pt}

Предположим, что имеется лекарственный препа\-рат, для которого 
был проведен набор  широкомасштабных клинических исследований, 
доказавших его безопасность и~медицинскую эф\-фективность. Данный 
препарат будем называть\linebreak референтным лекарственным препаратом. 
На основе действующих веществ референтного лекарственного препарата 
могут быть созданы новые лекар\-ст\-вен\-ные препараты с~таким же 
количественным и~качественным составом. Далее будем называть 
их воспроизведенными лекарственными препаратами. Чтобы 
перенести имеющиеся сведения о безопасности и~эффективности 
референтного\linebreak лекарственного препарата на воспроизведенный 
препарат без проведения широкомасштабных исследований, исследуют 
биоэквивалентность лекарственных препаратов.


Понятие биоэквивалентности тесно связано с~понятием биодоступности.

Биодоступность~-- скорость и~степень, с~которыми действующее вещество 
или его активная часть молекулы из дозированной лекарственной формы 
всасываются и~становятся доступными в~месте действия. Два лекарственных 
препарата, содержащих одинаковое количество действующего вещества, 
считаются биоэквивалентными, если они являются фармацевтически 
эквивалентными или фармацевтически альтернативными и~их биодоступность 
(по скорости и~степени) после применения в~одинаковой молярной дозе 
укладывается в~заранее установленные допустимые пределы~\cite{defin}.


В процессе сбора данных некоторые полученные значения могут быть утеряны. 
Так как число испытуемых ограничено, то исключать данные испытуемого,
 для которого было утеряно одно значение концентрации действующего вещества,
  нерационально. Вместо этого пропущенные данные заполняют нулем или другим значением,
   полученным на основе информации о~других значениях. 
   
   При использовании недостаточно 
   точных методов заполнения данных, таких как заполнение нулем, уменьшается часть 
   значений исследуемых данных.
Хотелось бы оценить, как наличие пропущенных данных влияет на проверку 
гипотезы биоэквивалентности.


В данной работе будет рассмотрена процедура двух односторонних тестов Шуирманна, 
которая в~настоящее время используется при проверке гипотезы биоэквивалентности, 
при наличии пропусков в~исследуемых данных.



\section{Задача проверки гипотезы биоэквивалентности}

Пусть $T$~--- воспроизведенный лекарственный препарат, а~$R$ --- 
референтный лекарственный препарат и~соответственно~$\mu_T$ и~$\mu_R$~--- 
математические ожидания сравнительных характеристик для лекарственных препаратов~$T$ и~$R$.

Основными сравнительными характеристиками биоэквивалентности служат максимальная 
концентрация  в~крови~$C_{\max}$ и~площадь под кривой <<кон\-цент\-ра\-ция\,--\,вре\-мя>> 
$\mathrm{AUC}$ (от \textit{англ.}\ Area Under the Curve).

Будем следовать предположению, что рас\-смат\-ри\-ва\-емые сравнительные 
характеристики имеют логнормальное распределение~[2--4].
%\cite{schuirmann, book, article}.
Например, для $\mathrm{AUC}$:
\begin{equation*}
%\label{log-norm}
    \mathrm{AUC}_i \sim \mathrm{Log}\,N\left(a_i, \sigma_i\right),\enskip i \in \{T,R\}.
\end{equation*}

Пусть $\theta_1$ и~$\theta_2$~--- соответственно нижний и~верхний принятый 
допустимый предел признания биоэквивалентности. Следовательно, гипотеза 
о~биоэквивалентности может быть записана следующим образом:
\begin{align*}
%\left.
%\begin{array}{rl}
&    H_0: \fr{\mu_T}{\mu_R} \leqslant q \theta_1\enskip \mbox{или}\enskip  
\fr{\mu_T}{\mu_R} \ge \theta_2; \\
 &   H_A: \theta_1 <\fr{\mu_T}{\mu_R} < \theta_2.  
% \end{array}
% \right\}
 %\label{h-bio}
\end{align*}

Сделав логарифмическое преобразование, можем перейти к~следующей 
постановке рассматриваемой гипотезы:
\begin{align*}
%\left.
%\begin{array}{rl}
    &H_0': \mu'_T - \mu'_R \leqslant q \delta_1\enskip \mbox{или}\enskip  \mu'_T - \mu'_R \ge \delta_2; \\
    &H_A': \delta_1 <\mu'_T - \mu'_R < \delta_2,  
 %   \end{array}
  %  \right\}
   % \label{h-bio-log}
\end{align*}
где $\delta_1 = \ln\theta_1$ и~$\delta_2 \hm= \ln\theta_2$, а~$\mu'_T$ и~$\mu'_R$~--- 
математические ожидания логарифмов сравнительных характеристик 
для лекарственных препаратов~$T$ и~$R$.
Например, для~$\ln{\mathrm{AUC}_T}$ из свойств логнормального распределения следует, 
что $ \mu'_T \hm= a_T$.

Гипотеза $H_0'$ соответствует небиоэквивалент\-ности исследуемых 
лекарственных препаратов, в~то время как~$H_A'$ утверждает, что они 
биоэквивалентны. Выбор такого порядка основной и~альтернативной гипотез 
обусловлен тем, что в~таком случае ошибка первого рода соответствует 
признанию лекарственных средств биоэквивалентными, хотя на самом деле 
они такими и~не является. В~этом случае пациент несет риск получить препарат, 
который может не обладать такими же эффективностью и~безопасностью, как 
референтный лекарственный препарат~\cite{article}.


\section{Процедура двух односторонних тестов Шуирманна}

Разобьем гипотезы $H'_0$ и~$H'_1$ на два множества односторонних гипотез:
\begin{equation*}
\left\{
\begin{array}{rl}
    H_{01}:& \mu'_T - \mu'_R \leqslant q \delta_1;  \\[6pt]
    H_{A1}:& \mu'_T - \mu'_R > \delta_1;  
    \end{array}
    \right.
   % \label{h-bio-log1}
\end{equation*}
\begin{equation*}
\left\{
\begin{array}{rl}
    H_{02}:& \mu'_T - \mu'_R \geqslant \delta_2; \\[6pt]
    H_{A2}:& \mu'_T - \mu'_R < \delta_2.  
    \end{array}
    \right.
   % \label{h-bio-log2}
\end{equation*}
Процедура двух односторонних тестов заключается в~том, что~$H'_0$ 
отвергаем при уровне зна\-чи\-мости~$\alpha$, тем самым устанавливая 
эквивалентность~$\mu_T$ и~$\mu_R$,  только в~том случае, если отвергаются 
обе гипотезы~$H_{01}$ и~$H_{02}$ при заданном уровне зна\-чи\-мости~$\alpha$~\cite{schuirmann, book}.

Таким образом, два односторонних теста проверяются с~использованием односторонних 
t-кри\-те\-ри\-ев, т.\,е.\ 
характеристики биодоступности признаются эквивалентными, если
\begin{equation}
\left.
\begin{array}{rl}
    \hspace*{-2mm}t_1& =  \fr{\bar{Y_T} -\bar{Y_R}-\delta_1}
    {\hat\sigma_d\sqrt{{1}/{n_1} + {1}/{n_2}}} > t\left(\alpha, n_1 + n_2-2\right); \\[6pt]
        \hspace*{-2mm}t_2& =  \fr{\bar{Y_T} - \bar{Y_R}-\delta_2}
    {\hat\sigma_d\sqrt{{1}/{n_1} + {1}/{n_2}}} < -t\left(\alpha, n_1 + n_2-2\right), 
    \end{array}\!
    \right\}\!\!
    \label{t}
\end{equation}
где $n_1$ и~$n_2$~--- число субъектов в~последовательностях клинического 
исследования с~перекрестным\linebreak двухпоследовательным дизайном, 
$t(\alpha, n_1\hm + n_2-2)$~--- $(1\hm - \alpha)$-кван\-тиль центрального 
t-рас\-пре\-де\-ле\-ния с~$n_1\hm+n_2-2$ степенями свободы; $\hat\sigma_d$~--- 
обобщенная выборочная дисперсия разностей между периодами (для обоих 
последовательностей в~исследовании), которая является несмещенной оценкой~$\sigma_d$,\linebreak 
причем
$$
\sigma_d^2 = \fr{\sigma_w^2}{2}\,,
$$
где $\sigma_w$~--- внутрисубъектная вариабельность изуча\-емых параметров~\cite{article}.

Процедура двух односторонних тестов эквивалентна подходу с~построением 
доверительного интервала для разности выборочных средних, т.\,е.\ 
получению следующей интервальной оценки:

\noindent
\begin{multline*}
%\label{interv}
    \left(\bar{Y_T} - \bar{Y_R} + t\left(\alpha, n_1 + n_2-2\right)\hat\sigma_d
    \sqrt{\fr{1}{n_1} + \fr{1}{n_2}};\right.\\
    \left.   \bar{Y_T} - \bar{Y_R} - t\left(\alpha, n_1 + n_2-2\right)
    \hat\sigma_d\sqrt{\fr{1}{n_1} + \fr{1}{n_2}}\right).
\end{multline*}

Признание эквивалентности параметров биодоступности на
 уровне значимости~$\alpha$ может быть сделано, только если 
 полученный доверительный $(1\hm-2\alpha)100\%$-ный интервал для 
 $\mu'_T \hm- \mu'_R$ полностью содержится в~интервале 
 $\left(\delta_1, \delta_2\right)$~\cite{schuirmann}.


Так как рассматриваем сбалансированный дизайн, то $n_1\hm=n_2\hm=n$, 
и,~учитывая~(\ref{t}), получаем, что t-кри\-те\-рии принимают вид:
\begin{align*}
%\left.
%\begin{array}{rl}
    t'_1 &=  \fr{\bar{Y_T} - \bar{Y_R}-\delta_1}{\hat\sigma_d\sqrt{2/n}} > t(2n-2, \alpha); \\
    t'_2 &=  \fr{\bar{Y_T} - \bar{Y_R}-\delta_2}{\hat\sigma_d\sqrt{2/n}} < -t(2n-2, \alpha)  
%    \end{array}
 %   \right\}
  %  \label{t_2}
\end{align*}
и соответствующий доверительный интервал принимает вид:
\begin{multline*}
%\label{interv_2}
    \left(\bar{Y_T} - \bar{Y_R} + t(\alpha, 2n-2)\hat\sigma_d\sqrt{\fr{2}{n}};\  \right.\\
\left.     \bar{Y_T} - \bar{Y_R} - t(\alpha, 2n-2)\hat\sigma_d\sqrt{\fr{2}{n}}\right).
\end{multline*}

\vspace*{-9pt}

\section{Оценка уровня значимости при~наличии пропущенных данных}

\vspace*{-3pt}

Рассмотрим выборочное пространство $\chi$~--- пространство элементарных событий. 
Статистический критерий разбивает пространство элементарных событий~$\chi$ 
на два подмножества:
\begin{enumerate}[(1)]
\item область принятия гипотезы $\chi_0$~--- множество, состоящее 
из точек, для которых гипотеза~$H_0$ принимается;\\[-14pt]
\item  область отклонения гипотезы $\chi_A$~--- множество, 
состоящее из точек, для которых гипотеза~$H_0$ отвергается.
\end{enumerate}


Говорят, что критерий имеет уровень зна\-чи\-мости~$\alpha$, если вероятность 
наступления ошибки первого рода не превышает~$\alpha$, $0 \hm< \alpha\hm < 1$, 
для $\delta\hm \in \chi_A$:

\noindent
\begin{multline*}
{\sf P}\left\{\mbox{отклонить } H_0 \mbox{ при\ истинной}\right.\\[-1pt]
\left.\mbox{небиоэквивалентности}\right\} ={}\\[-1pt]
{}
= {\sf P}\left\{\mbox{отклонить\ } H_0, \delta \in \chi_0\right\} \leqslant \alpha\,.
\end{multline*}



Рассмотрим задачу при наличии пропусков в~данных:
пусть $q$~--- уровень полноты данных,
 т.\,е.\ 
доля данных, оставшихся от изначальных дан-\linebreak\vspace*{-12pt}

\columnbreak

\noindent
ных, $0 \hm<q\hm \leqslant 1$
($1 - q$~--- доля пропущенных данных в~выборке).



Тогда $\tilde{Y_T} = \bar{Y_T} + \ln(q)$~--- 
выборочное среднее логарифмов сравнительных характеристик для лекарственного 
препарата~$T$ при наличии пропущенных данных.


Критическая область для данной задачи имеет следующий вид:

\vspace*{-4pt}

\noindent
\begin{multline*}
\chi_A' = \left\{(\tilde{Y_T} - \bar{Y_R}, \hat{\sigma_d}): 
\delta_1 + t\left(\alpha, 2n-2\right)\hat\sigma_d\sqrt{\fr{2}{n}} <{}\right.\\
\left.{}< \tilde{Y_T} - \bar{Y_R} <  \delta_2 - t\left(\alpha, 2n-2\right)\hat\sigma_d
\sqrt{\fr{2}{n}}\right\}.
\end{multline*}

\vspace*{-2pt}

Рассмотрим функцию мощности:

\vspace*{-3pt}

\noindent
\begin{align*}
    \phi_{\hat{\sigma_d}}'(\delta) &= {\sf P}\{\mbox{отклонить } H_0 \mbox{ при истинной}\\
&    \mbox{биоэквивалентности}\}={} \\
    &{}= {\sf P}\{(\tilde{Y_T} - \bar{Y_R}, \hat{\sigma_d}): \chi_A', если \delta \in \chi_A'\}.
\end{align*}

\vspace*{-2pt}

\noindent
Фиксируем $\delta = \delta_0$, получаем:

\vspace*{-2pt}

\noindent
\begin{multline*}
    \phi_{\hat{\sigma_d}}'(\delta_0) = {} \\
{}= {\sf P}\left(\left(\tilde{Y_T} - \bar{Y_R}, \hat{\sigma_d}\right): 
\delta_1 + t(\alpha, 2n-2)\hat\sigma_d\sqrt{\fr{2}{n}} < {}\right.\\
\left.{}<\tilde{Y_T} - \bar{Y_R} <  \delta_2 - t(\alpha, 2n-2)
\hat\sigma_d\sqrt{\fr{2}{n}}\vert  \delta =\delta_0\right) ={} \\
{}={\sf P}\left( \delta_1 + t(\alpha, 2n-2)\hat\sigma_d\sqrt{\fr{2}{n}}
 < \bar{Y_T} + \ln(q) - {}\right.\\
\left. {}-\bar{Y_R} <  \delta_2 - t(\alpha, 2n-2)\hat\sigma_d
 \sqrt{\fr{2}{n}}\right) ={} \\
 {}={\sf P}\Biggl(\fr{\delta_1 + t(\alpha, 2n-2)\hat\sigma_d\sqrt{{2}/{n}}- 
\ln(q)-\delta_0}{\sigma_d\sqrt{{2}/{n}}}<{}\\
{}< \fr{\bar{Y_T} - 
\bar{Y_R}-\delta_0}{\sigma_d\sqrt{{2}/{n}}} <{}\\
{} < \fr{ \delta_2 - t(\alpha, 2n-2)\hat\sigma_d\sqrt{{2}/{n}}- 
\ln(q)-\delta_0}{\sigma_d\sqrt{{2}/{n}}}\Biggr) = {}\\
{}= \{\mbox{фиксируем } \hat\sigma_d\} = {}\\
{}=E\Biggl[{\sf P}\Biggl(\fr{\delta_1 + t(\alpha, 2n-2)\hat\sigma_d\sqrt{{2}/{n}}
- \ln(q)-\delta_0}{\sigma_d\sqrt{{2}/{n}}}< {}\\
{}<\fr{\bar{Y_T} - 
\bar{Y_R}-\delta_0}{\sigma_d\sqrt{{2}/{n}}} <{}\\
 {} < \fr{ \delta_2 - t(\alpha, 2n-2)\hat\sigma_d\sqrt{{2}/{n}}- 
\ln(q)-\delta_0}{\sigma_d\sqrt{{2}/{n}}}|\hat\sigma_d\Biggr)\Biggr] = {}
\end{multline*}

 \noindent
 \begin{multline*}
\hspace*{-4pt}{}=E\Biggl[\Phi\left(\fr{\delta_1 + t(\alpha, 2n-2)\hat\sigma_d\sqrt{{2}/{n}}-
 \ln(q)-\delta_0}{\sigma_d\sqrt{{2}/{n}}}\right) - {}\\[3pt]
 {}-
 \Phi\left( \fr{ \delta_2 - t(\alpha, 2n-2)\hat\sigma_d\sqrt{{2}/{n}}- 
 \ln(q)-\delta_0}{\sigma_d\sqrt{{2}/{n}}}\right)\Biggr], 
\end{multline*}
где $\Phi(x)$~--- функция стандартного нормального распределения.

Процедура двух односторонних тестов Шуирманна используется в~условиях 
решающего правила 80/125~\cite{schuirmann, article, hsu}. Это значит, что
 $\delta_1\hm = \ln(0,80)\hm \approx -0{,}2231$ и~$\delta_2\hm = \ln(1,25)
 \hm \approx 0{,}2231$; следовательно, $\delta_1 \hm\approx -\delta_2$. 
 Тогда видим, что функция $\phi_{\hat{\sigma_d}}'(\delta)$ сим\-мет\-рич\-на 
 относительно точки $\delta \hm= -\ln(q)$ и~достигает максимума в~этой точке.

Тогда 

\vspace*{-6pt}

\noindent
\begin{multline*}
\max_{\delta \in \chi_0} {\sf P}\{\mbox{отклонить } H_0\} \hm= 
\phi_{\hat{\sigma_d}}'(\delta_2) ={}\\
{}\{\mbox{подставим }  \delta = \delta_2,\ 
\delta_1 = -\delta_2, \sigma_d=\hat\sigma_d \} ={}\\
    {}={\sf P}\left( \fr{-2\delta_2 - \ln(q)}{\hat\sigma_d\sqrt{{2}/{n}}} 
    +  t(\alpha, 2n-2)<{}\right.\\
   \left. {}<\fr{\bar{Y_T} - \bar{Y_R}-\delta_2}{\hat\sigma_d
    \sqrt{{2}/{n}}}< \fr{- \ln(q)}{\hat\sigma_d\sqrt{{2}/{n}}}- 
    t(\alpha, 2n-2)\right) = {}\\
    {}={\sf P}\left( \fr{-2\delta_2 - \ln(q)}{\hat\sigma_d\sqrt{{2}/{n}}} + 
     t(\alpha, 2n-2)<{}\right.\\
\left.     {}<\fr{\bar{Y_T} - \bar{Y_R}-\delta_2}{\hat\sigma_d\sqrt{{2}/{n}}}<
      - t(\alpha, 2n-2)\right) + {}\\
    {}+{\sf P}\left( - t(\alpha, 2n-2)<\fr{\bar{Y_T} - \bar{Y_R}-\delta_2}
    {\hat\sigma_d\sqrt{{2}/{n}}}<{}\right.\\
\left.    {}< \fr{- \ln(q)}{\hat\sigma_d\sqrt{{2}/{n}}}- 
    t(\alpha, 2n-2)\right) \leqslant {}\\
{}\leqslant {\sf P}\left(\fr{\bar{Y_T} - \bar{Y_R}-\delta_2}{\hat\sigma_d\sqrt{{2}/{n}}}< 
- t(\alpha, 2n-2)\right) + {}\\
    {}+ {\sf P}\left( - t(\alpha, 2n-2)<\fr{\bar{Y_T} - \bar{Y_R}-\delta_2}
    {\hat\sigma_d\sqrt{{2}/{n}}}< {}\right.\\
\left.    {}<\fr{- \ln(q)}{\hat\sigma_d\sqrt{{2}/{n}}}- 
    t(\alpha, 2n-2)\right) = {}
   \\
    {}=\alpha + {\sf P}\left( - t(\alpha, 2n-2)<\fr{\bar{Y_T} - 
    \bar{Y_R}-\delta_2}{\hat\sigma_d\sqrt{{2}/{n}}}< {}\right.\\
\left.    {}<\fr{- \ln(q)}
    {\hat\sigma_d\sqrt{{2}/{n}}}- t(\alpha, 2n-2)\right) =\alpha + \alpha'. \\
\end{multline*}

\vspace*{-18pt}

\noindent
Полученная оценка показывает, что величина ошибки первого рода не превосходит
 $\alpha \hm+ \alpha'$. При чем в~работах~\cite{book, article} показано, что 
 при исполь-\linebreak\vspace*{-12pt}
 
 \columnbreak
 
 \noindent
зо\-вании двух односторонних тестов Шуирманна\linebreak
  величина 
 вероятности ошибки первого рода не превосходит~$\alpha$. 
 В~рассматриваемой постановке уровень значимости критерия повышается на 
 величину~$\alpha'$, что обусловлено наличием пропущенных данных для 
 воспроизводимого лекарственного препарата. Таким образом, риск потенциального 
 выхода на рынок небиоэквивалентного лекарственного препарата повышается.
 
 \vspace*{-15pt}

\section{Заключение}

 \vspace*{-5pt}

Процедура двух односторонних тестов Шуирманна~--- одно из основных средств 
при проверке гипотезы биоэквивалентности. При исследовании критериев принятия 
гипотезы биоэквивалентности важную роль играет оценка вероятности наступления 
ошибки первого рода. Важность ее контроля\linebreak обусловлена риском пациента получить 
препарат с~несоответствующими эффективностью и~без\-опас\-ностью.
%
В~данной статье впервые дана оценка уровня значимости процедуры двух односторонних
 тестов Шуирманна при наличии пропущенных данных. В~част\-ности, в~явном виде
  показана та ее часть, которая зависит от уровня полноты данных.
%
В практическом плане данная оценка может быть использована для корректировки 
задаваемого уровня зна\-чи\-мости при известном уровне полноты данных, чтобы 
обеспечить гарантированную эффективность и~безопасность воспроизведенных лекарств.

 \vspace*{-15pt}


{\small\frenchspacing
 { %\baselineskip=10.5pt
 \addcontentsline{toc}{section}{References}
 \begin{thebibliography}{9}
 
  \vspace*{-5pt}
  
    \bibitem{defin} 
    Правила проведения исследований биоэквивалентности лекарственных 
    средств Евразийского экономического союза. 
    {\sf 
    http://www.eurasiancommission.org/ ru/act/texnreg/deptexreg/konsultComitet/Documents/\linebreak Правила\%20БЭИ\%20итог\%2020.02.2015\%20на\%20\linebreak сайт.pdf}.
    \bibitem{schuirmann}
    \Au{Schuirmann D.\,J.}
    A~comparison of the two one-sided tests procedure and the power approach 
    for assessing the equivalence of average bioavailability~// J.~Pharmacokinet. 
    Biop., 1987. Vol.~15. P.~657--680.
    \bibitem{book}
    \Au{Chow Shein-Chung, Liu Jen-pei.}
    Design and analysis of bioavailability and bioequivalence studies.~--- 
    Chapman \& Hall/CRC, 2009. 735~p.
    \bibitem{article}
\Au{Драницына М.\,А., Захарова~Т.\,В., Ниязов~Р.\,Р.}
    Свойства процедуры двух односторонних тестов 
    для признания биоэквивалентности лекарственных препаратов~// Ремедиум. 
    Журнал о рынке лекарств и~медицинской техники, 2019. №\,3. С.~40--47.
    \bibitem{hsu} \Au{Berger R.\,L., Hsu~J.\,C.} 
    Bioequivalence trials, intersection--union tests and equivalence confidence sets~// 
    Stat. Sci., 1996. Vol.~11. No.~4. P.~283--319.
    
     \end{thebibliography}

 }
 }

\end{multicols}

\vspace*{-12pt}

\hfill{\small\textit{Поступила в~редакцию 09.05.19}}

%\vspace*{8pt}

\pagebreak

%\newpage

\vspace*{-28pt}

%\hrule

%\vspace*{2pt}

%\hrule

%\vspace*{-2pt}

\def\tit{EVALUATION OF THE SIGNIFICANCE LEVEL IN~SCHUIRMANN'S TEST FOR~CHECKING 
THE~BIOEQUIVALENCE HYPOTHESIS IN~MISSING DATA CONDITIONS}


\def\titkol{Evaluation of the significance level in~Schuirmann's test for~checking 
the~bioequivalence hypothesis in~missing data conditions}

\def\aut{T.\,V.~Zakharova$^{1,2}$ and A.\,A.~Tarkhov$^1$}

\def\autkol{T.\,V.~Zakharova and A.\,A.~Tarkhov}

\titel{\tit}{\aut}{\autkol}{\titkol}

\vspace*{-11pt}


\noindent
$^1$Department of Mathematical Statistics, Faculty of Computational Mathematics 
 and Cybernetics, M.\,V.~Lo\-mo-\linebreak
 $\hphantom{^1}$nosov Moscow State University, 1-52~Leninskiye Gory, 
 GSP-1, Moscow 119991, Russian Federation
 
 \noindent
 $^2$Institute of 
 Informatics Problems, Federal Research Center ``Computer Science and Control'' 
 of the Russian\linebreak
  $\hphantom{^1}$Academy of Sciences, 44-2~Vavilov Str., Moscow 119333, 
 Russian Federation

\def\leftfootline{\small{\textbf{\thepage}
\hfill INFORMATIKA I EE PRIMENENIYA~--- INFORMATICS AND
APPLICATIONS\ \ \ 2019\ \ \ volume~13\ \ \ issue\ 3}
}%
 \def\rightfootline{\small{INFORMATIKA I EE PRIMENENIYA~---
INFORMATICS AND APPLICATIONS\ \ \ 2019\ \ \ volume~13\ \ \ issue\ 3
\hfill \textbf{\thepage}}}

\vspace*{3pt}    



\Abste{The bioequivalence hypothesis testing is the important task 
in pharmacokinetics. It helps to make a~decision about the equivalence 
of the reproduced drug to the reference drug. One of the problems of bioequivalence 
studies is the availability of missing data. 
A~small amount of data entails the inability to delete a~data sample with 
missing data. Therefore, there is a~task to estimate the impact of missing data 
on bioequivalence testing task, in particular, to estimate the significance level. 
The main method of the bioequivalence hypothesis testing is Schuirmann's 
two one-sided tests procedure. The article shows the significance level evaluation 
of this procedure in the case of missing data. The evaluation component, depending 
on the level of data completeness, is shown in the explicit form.}


\KWE{bioequivalence; significance level; type I error; missing data; 
Schuirmann's two one-sided tests procedure}


\DOI{10.14357/19922264190309} 

%\vspace*{-14pt}

\Ack
   \noindent
   The paper was supported by the Russian Foundation for Basic Research (project  
18-07-00252).


%\vspace*{-6pt}

  \begin{multicols}{2}

\renewcommand{\bibname}{\protect\rmfamily References}
%\renewcommand{\bibname}{\large\protect\rm References}

{\small\frenchspacing
 {%\baselineskip=10.8pt
 \addcontentsline{toc}{section}{References}
 \begin{thebibliography}{9}

\bibitem{1-zah}
Pravila provedeniya issledovaniy bioekvivalentnosti 
lekarstvennykh sredstv Evraziyskogo ekonomicheskogo soyuza. Available at:
{\sf http://www.eurasiancommission.\linebreak org/ru/act/texnreg/deptexreg/konsultComitet/\linebreak Documents/Pravila\%20BEI\%20itog\%2020.02.2015\%20\linebreak na\%20sajt.pdf}
  (accessed May~8, 2019).
\bibitem{2-zah}
\Aue{Schuirmann, D.\,J.} 1987. 
A~comparison of the two one-sided tests procedure and the power approach for 
assessing the equivalence of average bioavailability.
\textit{J.~Pharmacokinet. Biop.} 15:657--680.

\columnbreak 

\bibitem{3-zah}
\Aue{Chow, Shein-Chung, and Jen-pei Liu.} 2009. \textit{Design 
and analysis of bioavailability and bioequivalence studies.} 
Chapman \& Hall/CRC. 735~p.

\vspace*{-2pt}

\bibitem{4-zah}
\Aue{Dranitsyna, M.\,A., T.\,V.~Zakharova, and R.\,R.~Niyazov.}
 2019. Svoystva protsedury dvukh odnostoronnikh testov dlya priznaniya
  bioekvivalentnosti lekarstvennykh preparatov [Properties of the 
   two-sided tests procedure for the  bioequivalence assessment of medical products]. 
  \textit{Remedium. Zh.~o~rynke lekarstv i~meditsinskoy tekhniki}
   [Remedium: J.~of the Market of Medicines and Medical Equipment] 2019(3):40--47.
   
   \vspace*{-2pt}
   
\bibitem{5-zah}
\Aue{Berger, R.\,L.,  and J.\,C.~Hsu.} 1996. 
Bioequivalence trials, intersection--union tests and equivalence confidence sets. 
\textit{Stat. Sci.} 11(4):283--319.
\end{thebibliography}

 }
 }

\end{multicols}

%\vspace*{-7pt}

\hfill{\small\textit{Received May 9, 2019}}

%\pagebreak

\vspace*{-12pt}

\Contr

\noindent
\textbf{Zakharova Tatiana V.} (b.\ 1962)~--- 
Candidate of Science (PhD) in physics and mathematics, associate professor,
 Department of Mathematical Statistics, Faculty of Computational Mathematics 
 and Cybernetics, M.\,V.~Lomonosov Moscow State University, 1-52~Leninskiye Gory, 
 GSP-1, Moscow 119991, Russian Federation; senior scientist, Institute of 
 Informatics Problems, Federal Research Center ``Computer Science and Control'' 
 of the Russian Academy of Sciences, 44-2~Vavilov Str., Moscow 119333, 
 Russian Federation; \mbox{tvzaharova@mail.ru}
 
 \vspace*{3pt}

\noindent
\textbf{Tarkhov Alexey A.} (b.\ 1995)~--- 
master student, Department of Mathematical Statistics, Faculty of Computational 
Mathematics and Cybernetics, M.\,V.~Lomonosov Moscow State University, 
1-52~Leninskiye Gory, GSP-1, Moscow 119991, Russian Federation; 
\mbox{alexeytarkhov@gmail.com}

\label{end\stat}

\renewcommand{\bibname}{\protect\rm Литература}  