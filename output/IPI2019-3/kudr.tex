\def\stat{kudr}

\def\tit{АПРИОРНОЕ ОБОБЩЕННОЕ ГАММА-РАСПРЕДЕЛЕНИЕ В~БАЙЕСОВСКИХ МОДЕЛЯХ 
БАЛАНСА$^*$}

\def\titkol{Априорное обобщенное гамма-распределение в~байесовских моделях 
баланса}

\def\aut{А.\,А.~Кудрявцев$^1$}

\def\autkol{А.\,А.~Кудрявцев}

\titel{\tit}{\aut}{\autkol}{\titkol}

\index{Кудрявцев А.\,А.}
\index{Kudryavtsev A.\,A.}


{\renewcommand{\thefootnote}{\fnsymbol{footnote}} \footnotetext[1]
{Работа выполнена при частичной финансовой поддержке РФФИ 
(проект 17-07-00577).}}


\renewcommand{\thefootnote}{\arabic{footnote}}
\footnotetext[1]{Московский государственный 
университет им.~М.\,В.~Ломоносова, факультет вычислительной математики 
и~кибернетики, \mbox{nubigena@mail.ru}}

%\vspace*{-2pt}



\Abst{Работа посвящена исследованию байесовских моделей баланса, предполагающих 
разбиение параметров системы на два класса: способствующих функционированию 
позитивных факторов и~препятствующих функционированию негативных факторов. 
Рассматривается индекс баланса, определяемый как отношение негативного фактора 
к~позитивному. Изучается постановка задачи, заключающаяся в~нахождении основных 
вероятностных характеристик (плотность, функция распределения, моменты) индекса 
баланса факторов, имеющих априорные обобщенные гам\-ма-рас\-пре\-де\-ле\-ния с~параметрами 
формы одного знака. Результаты формулируются в~терминах гам\-ма-экс\-по\-нен\-ци\-аль\-ной 
функции. Приводится ряд новых свойств последней. Показано, что приводимые 
утверждения легко переформулируются для масштабных смесей обобщенных 
гам\-ма-рас\-пре\-де\-ле\-ний, имеющих параметры формы разных знаков. Полученные результаты 
могут найти широкое применение в~моделях, использующих для описания процессов 
и~явлений распределения с~положительным неограниченным носителем.}

\KW{байесовский подход; обобщенное гамма-рас\-пре\-де\-ле\-ние; 
гам\-ма-экс\-по\-нен\-ци\-аль\-ная функция; модели баланса; смешанные распределения}

\DOI{10.14357/19922264190305} 
  
%\vspace*{1pt}


\vskip 10pt plus 9pt minus 6pt

\thispagestyle{headings}

\begin{multicols}{2}

\label{st\stat}

\section{ Введение}

Байесовский подход к~моделям баланса в~терминах теории массового обслуживания 
был впервые\linebreak предложен С.\,Я.~Шоргиным в~работе~\cite{Shorgin05} и~рас\-про\-стра\-нен 
на задачи повышения на\-деж\-ности \mbox{сложных} 
ин\-фор\-ма\-ци\-он\-но-те\-ле\-ком\-му\-ни\-ка\-ци\-он\-ных
сис\-тем  в~\cite{KuSoSh}.
Подробное описание байесовских моделей баланса и~сферы их применения 
в~разнообразных областях знания было дано в~статье~\cite{Ku2018}. Остановимся на 
нем вкратце.

Всевозможные аспекты современной жизни усложнились настолько, что вынуждают 
прибегать к~использованию разного рода индексов и~рейтингов для определения 
эффективности функционирования исследуемой системы. Для построения таких 
показателей целесообразно условно разделять факторы, оказывающие влияние на 
систему, на способствующие функционированию ({p-фак\-то\-ры}) и~препятствующие 
функционированию ({n-фак\-то\-ры}).
Разделение параметров на негативные и~позитивные факторы свойственно моделям из 
разных областей науки~\cite{Ku2018} от теории массового обслуживания до 
моделирования чрезвычайных ситуаций.

Вполне естественно, что функционирование исследуемой системы в~итоге зависит не 
столько от значений n- и~p-фак\-то\-ров, сколько от их соотношения~--- индекса 
баланса. При этом во избежание чрезмерных затрат на преодоление негативного 
влияния и~с целью недопущения недооценки негативного воздействия следует 
добиваться сбалансированности системы, т.\,е.\ стремиться приблизить к~единице 
индекс баланса.

С течением времени n- и~p-фак\-то\-ры, а~следова\-тельно и~индекс баланса, 
претерпевают изменения,\linebreak связанные с~неустойчивостью и~не\-опре\-де\-лен\-ностью среды, 
в~которой происходит функционирование. Это дает предпосылки для рассмотрения 
факторов и~индексов как случайных величин. Однако глобальные изменения 
окружающей среды происходят достаточно редко, поэтому законы, влияющие на 
значения факторов, можно считать неизменными. Из этого следует, что 
распределения рассматриваемых случайных величин можно полагать заданными 
априорно, что в~свою очередь обусловливает применение к~моделям баланса 
байесовского метода.

Поскольку в~байесовских моделях баланса оба смешиваемых распределения 
в~большинстве случаев имеют положительные носители, в~качестве априорных имеет 
смысл рассматривать распределения гамма-типа, зарекомендовавшие себя за 
последние столетия в~качестве адекватных моделей многих реальных процессов 
и~явлений. Стоит также отметить, что вопрос об отыскании характеристик отношения 
двух факторов (индекса баланса) в~данном случае эквивалентен задаче нахождения 
характеристик произведения двух величин из гам\-ма-клас\-са, что существенно 
расширяет область применимости рассматриваемых результатов.

\section{Гамма-экспоненциальная функция}

Приведем ряд вспомогательных инструментов.

\smallskip

\noindent
\textbf{Определение~1.}
Назовем функцию вида
\begin{multline}
\label{GEF}
{\sf Ge}_{\alpha,\, \beta} (x) = \sum\limits_{k=0}^{\infty}\fr{x^k}{k!}\,
 \Gamma(\alpha k + 
\beta)\,, \\
 x\in\mathbb{R}\,, \enskip 0\le\alpha<1\,, \enskip \beta> 0\,,
\end{multline}
гамма-экспоненциальной функ\-ци\-ей~\cite{KuTi2017}.


\smallskip

%\begin{rem}
Функция~(\ref{GEF}) обобщает на случай $\beta\hm\neq1$ преобразование, введенное 
Леруа~\cite{LeRoy1900_1} для исследования производящих функций специального 
вида. Кроме того, функцию~(\ref{GEF}) можно рассматривать (при некоторых 
допущениях) как частный случай функции Сри\-ва\-ста\-ва--То\-мовски~\cite{SrTo2009}, 
обобщающей функцию Мит\-таг-Леф\-фле\-ра~\cite{GoKiMaRo2014}.
%\end{rem}

\smallskip

\noindent
\textbf{Лемма~1.}\
\textit{ %\label{Lem_GGF_general}
Пусть $\alpha,\theta>0$, а $r,u,v\neq0$ имеют один знак. Тогда}
\begin{multline*}
\int\limits_0^{\infty}y^{r-1}e^{-(y/\alpha)^u-(y/\theta)^v} \, dy ={}\\
{}=
 \begin{cases}
   \displaystyle \fr{\theta^{r}}{|v|}{\sf Ge}_{u/v,\,r/v}\left(-
\left(\fr{\theta}{\alpha}\right)^{u}\right), &|v|>|u|;\\[9pt]
   \displaystyle \fr{\alpha^{r}}{|u|}{\sf Ge}_{v/u,\,r/u}\left(-
\left(\fr{\alpha}{\theta}\right)^{v}\right), &|u|>|v|;\\[9pt]
   \displaystyle \fr{\Gamma(r/u)}{|u|(\alpha^{-u}+\theta^{-u})^{r/u}}, 
&u = v.
 \end{cases}
\end{multline*}

\noindent
Д\,о\,к\,а\,з\,а\,т\,е\,л\,ь\,с\,т\,в\,о\,.\ \
Рассмотрим случай $r\hm>0$, $u\hm>0$ и~$v\hm>0$. Предположим, 
что $v\hm>u$. Тогда, используя теорему Лебега о~предельном переходе, получаем:
\begin{multline*}
\int\limits_0^{\infty}y^{r-1}e^{-(y/\alpha)^u-(y/\theta)^v} \, dy ={}\\
{}=
\fr{\alpha^{r}}{u}\int\limits_0^{\infty}t^{r/u-1}e^{-t-(\alpha/\theta)^vt^{v/u}} \,  
dt ={}\\
{}=\fr{\alpha^{r}}{u}\sum\limits_{k=0}^\infty\fr{(-1)^k}{k!}
\int\limits_0^{\infty}t^{r/u+k-1}e^{-(\alpha /\theta)^vt^{v/u}} \,  dt 
={}\\
{}=
\fr{\theta^{r}}{v}\sum\limits_{k=0}^\infty\fr{(-1)^k\theta^{uk}}{\alpha^{uk}k!}
\,\Gamma\left(\fr{r+uk}{v}\right).
\end{multline*}

Случай $u>v$ рассматривается аналогично. Случай $u=v$ напрямую следует из 
определения гам\-ма-функ\-ции. Так же аналогично получаются результаты для случая 
$r\hm<0$, $u\hm<0$ и~$v\hm<0$. Лемма доказана.

\smallskip

\noindent
\textbf{Замечание~1.}\
Несмотря на то что при $uv\hm<0$ и~$r\hm\neq0$ интеграл
$\int\nolimits_0^{\infty}y^{r-1}e^{-(y/\alpha)^u-(y/\theta)^v} \, dy$,
очевидно, сходится, непосредственное его вычисление связано с~принципиальными 
трудностями.


\smallskip

\noindent
\textbf{Определение~2.}\
Интегральной гамма-экс\-по\-нен\-ци\-аль\-ной функцией~\cite{KuPaSh2019_2} назовем функцию
$$
{\sf Gi}(r,s,t;\,x)=\fr{1}{t\Gamma(s)\Gamma(t)}\int\limits_{0}^{x} 
{\sf Ge}_{r,\,tr+s}\left(-z^{1/t}\right) \, dz,
$$
где $0\le r<1$, $s,t\hm>0$.


\section{Масштабные смеси обобщенных гамма-распределений}

Будем обозначать обобщенное гамма-рас\-пре\-де\-ле\-ние, или распределение 
Стейси~\cite{Stacy1962}, с~плот\-ностью
\begin{multline*}
f(x)=\fr{|v| x^{vq -1}e^{-(x/\theta)^v}}{\theta^{vq}\Gamma(q)}\,, \\ v\neq0\,, 
\enskip q>0\,,\enskip 
 \theta >0\,, \enskip x>0\,,
\end{multline*}
через $\mathrm{GG}\,(v,q,\theta)$.

Класс обобщенных гам\-ма-рас\-пре\-де\-ле\-ний до\-ста\-точно широк и~включает
экспоненциальное распре\-де\-ле\-ние~\cite{KoPoSkTu1985} ($v\hm=1$, $q\hm=1$);
$\chi^2$-рас\-пре\-де\-ле\-ние~\cite{KoPoSkTu1985} ($v\hm=1$, $q\hm=k/2$, 
$k\hm\in\mathbb{N}$,  $\theta\hm=2$);
масштабированное $\chi^2$-рас\-пре\-де\-ле\-ние~\cite{Lee2012} ($v\hm=1$, $q\hm=k/2$, 
$k\hm\in\mathbb{N}$);
распределение Эрланга~\cite{KoPoSkTu1985} ($v\hm=1$, $q\hm\in\mathbb{N}$);
гамма-распределение~\cite{KoPoSkTu1985} ($v\hm=1$);
полунормальное распределение~\cite{KlKo2003}, или
распределение максимума процесса броуновского движения~\cite{Kruglov2016} 
($v\hm=2$, $q\hm=1/2$);
распределение Рэлея~\cite{KoPoSkTu1985} ($v\hm=2$, $q\hm=1$);
распределение Макс\-вел\-ла--Больц\-ма\-на~\cite{KoPoSkTu1985} ($v\hm=2$, $q\hm=3/2$);
$\chi$-рас\-пре\-де\-ле\-ние~\cite{KoPoSkTu1985} ($v\hm=2$, $q\hm=k/2$, $k\hm\in\mathbb{N}$, 
$\theta\hm=\sqrt{2}$);
масштабированное $\chi$-рас\-пре\-де\-ление~\cite{Gelman2014} ($v\hm=2$, $q\hm=k/2$, 
$k\hm\in\mathbb{N}$);
m-распределе-\linebreak ние Накагами~\cite{Nakagami} ($v\hm=2$);
распределение\linebreak Виль\-со\-на--Хиль\-фер\-ти~\cite{WiHi1931} ($v\hm=3$);
распределение Вей\-бул\-ла--Гне\-ден\-ко~\cite{KoPoSkTu1985} ($v\hm>0$, $q\hm=1$);
обобщенное распределение Вейбулла~\cite{Lai2014} ($v\hm>0$, $q\hm\in\mathbb{N}$);
псевдовейбулловское распределение~\cite{Voda1989} ($v\hm>0$, $q\hm=1\hm+1/v$);
распределение Пирсона третьего и~пятого типов~\cite{KoPoSkTu1985} ($|v|\hm=1$);
распределение Леви с~нулевым смещением~\cite{KoPoSkTu1985} ($v\hm=-1$, $q\hm=1/2$);
обратное экспоненциальное распределение~\cite{KlKo2003} ($v\hm=-1$, $q\hm=1$);
обратное $\chi^2$-рас\-пре\-де\-ле\-ние~\cite{Gelman2014} ($v\hm=-1$, $q\hm=k/2$, 
$k\hm\in\mathbb{N}$, $\theta\hm=1/{2}$);
масштабированное обратное $\chi^2$-рас\-пре\-де\-ле\-ние~\cite{Gelman2014} ($v\hm=-1$, 
$q\hm=k/2$, $k\hm\in\mathbb{N}$);
обратное гам\-ма-рас\-пре\-де\-ле\-ние \cite{Gelman2014} ($v\hm=-1$);
обратное распределение Рэлея \cite{Panwar2015} ($v\hm=-2$, $q\hm=1$);
обратное $\chi$-рас\-пре\-де\-ле\-ние \cite{Lee2012} ($v\hm=-2$, $q\hm=k/2$, $k\hm\in\mathbb{N}$, 
$\theta\hm=1/\sqrt{2}$);
масштабированное обратное $\chi$-рас\-пре\-де\-ле\-ние~\cite{Lee2012} ($v\hm=-2$, $q\hm=k/2$, 
$k\hm\in\mathbb{N}$);
распределение Фреше~\cite{Frechet1927} ($v\hm<0$, $q\hm=1$) и~др.
%обобщенное распределение Фреше ($v<0$, $q\in\mathbb{N}$) \cite{Carolynne2013}
%обобщенное распределение экстремальных значений Фишера--Типпетта с
%неотрицательным носителем ($q=1$);


\smallskip

\noindent
\textbf{Замечание~2.}\
Если случайная величина~$\xi$ имеет распределение~$\mathrm{GG}\,(v,q,\theta)$, то случайная 
величина~$1/\xi$ имеет распределение $\mathrm{GG}\,(-v,q,1/\theta)$. Поэтому приведенные 
ниже результаты для вероятностных харак\-те\-ри\-стик отношения двух независимых 
случайных величин, имеющих обобщенное гам\-ма-рас\-пре\-де\-ле\-ние с~параметрами формы 
одного знака, несложно переформулировать для масштабных смесей обобщенных 
гам\-ма-рас\-пре\-де\-ле\-ний, име\-ющих параметры формы разных знаков, 
с~со\-от\-вет\-ст\-ву\-ющей заменой  параметров.


\bigskip

Следующее утверждение непосредственно вытекает из леммы~1.

\bigskip

\noindent
\textbf{Теорема~1.}\
\textit{Пусть независимые случайные величины~$\lambda$ и~$\mu$ имеют соответственно 
распределения $\mathrm{GG}\,(v,q,\theta)$ и~$\mathrm{GG}\,(u,p,\alpha)$, причем $uv\hm>0$. Тогда их 
отношение $\rho\hm=\lambda/\mu$ при $x\hm>0$ имеет плотность}:
\begin{multline*}
f_{\rho}(x) ={}\\
{}=\!
 \begin{cases}
   \displaystyle \fr{\abs{v}\alpha^{vq}x^{vq-
1}}{\theta^{vq}\Gamma(p)\Gamma(q)}{\sf Ge}_{v/u,\, vq/u+p}\left(-\left(\fr{\alpha 
x}{\theta}\right)^{v}\right), &\\[6pt]
&\hspace*{-18mm}|u|>|v|;\\[6pt]
   \displaystyle \fr{\abs{u}\theta^{up}x^{-up-1}}{\alpha^{up}
   \Gamma(p)\Gamma(q)}{\sf Ge}_{u/v,\, up/v+q}\left(-\left(\fr{\alpha 
x}{\theta}\right)^{-u}\right), &\\[6pt]
&\hspace*{-18mm}|v|>|u|;\\[6pt]
   \displaystyle \fr{\abs{v}(\alpha/\theta)^{vq}x^{vq-1}}{B(p,q)
   \left(1+(\alpha x/\theta)^{v}\right)^{p+q}}, &\hspace*{-13mm}u = v.
 \end{cases}
\end{multline*}


Заметим, что при $u\hm=v$ случайная величина~$\rho$ имеет обобщенное 
бе\-та-рас\-пре\-де\-ле\-ние\linebreak второго рода~\cite{McDonald1984} $\mathrm{GB2}\,(v,\theta/\alpha,q,p)$, 
частные случаи которого представляют распределение Бурра~\cite{Burr1942}, или 
распределение Сингх--Мад\-да\-ла~\cite{SiMa1976}, 
$\mathrm{GB2}\,(|v|,\theta/\alpha,1,p)$;
распределение Дагума~\cite{Dagum1977} $\mathrm{GB2}\,(|v|,\theta/\alpha,q,1)$;
распределение Ломакса~\cite{Lomax1954} $\mathrm{GB2}\,(1,\theta/\alpha,1,p)$;
F-распределение Фи\-ше\-ра--Сне\-де\-ко\-ра~\cite{KoPoSkTu1985} 
$\mathrm{GB2}\,(1,\theta/\alpha,\alpha/2,\theta/2)$
и~др.
Легко показать, что при $v\hm<0$ распределение $\mathrm{GB2}\,(v,\theta/\alpha,q,p)$ 
совпадает с~$\mathrm{GB2}\,(|v|,\theta/\alpha,p,q)$.
Таким образом, тео\-ре\-ма~1 вместе с~замечанием~2 
дают возможность выражать характеристики смесей прямых и~обратных 
гам\-ма-рас\-пре\-де\-ле\-ний, а также распределений Вейбулла и~Фреше с~равными параметрами 
формы в~терминах хорошо изученных распределений.


Несмотря на то что плотность отношения двух независимых случайных величин, 
имеющих обобщенное гам\-ма-рас\-пре\-де\-ле\-ние, выражается в~терминах специальной 
гам\-ма-экс\-по\-нен\-ци\-аль\-ной функции, моментные характеристики~$\rho$ вычисляются достаточно 
просто.

\bigskip

\noindent
\textbf{Теорема~2.}\
\textit{Пусть независимые случайные величины~$\lambda$ и~$\mu$ имеют соответственно 
распределения $\mathrm{GG}\,(v,q,\theta)$ и~$\mathrm{GG}\,(u,p,\alpha)$, причем $uv\hm>0$. Тогда для их 
отношения $\rho\hm=\lambda/\mu$ при всех $z\hm\in\mathbb{R}$ имеет место соотношение}:
\begin{multline*}
\e\rho^z=\fr{(\theta/\alpha)^{z} \Gamma(q+z/v)\Gamma(p-z/u)}
{\Gamma(q)\Gamma(p)}\,, \\
q+\fr{z}{v}>0\,, \enskip p-\fr{z}{u}>0\,.
\end{multline*}


%\smallskip

\noindent
Д\,о\,к\,а\,з\,а\,т\,е\,л\,ь\,с\,т\,в\,о\ \
 непосредственно вытекает из независимости случайных 
величин~$\lambda$ и~$\mu$, замечания~2 и~того факта, что
$$
\e\lambda^z=\fr{\theta^{z} \Gamma(q+z/v)}{\Gamma(q)},\enskip q+\fr{z}{v}>0\,.
$$

Теорема~2 дает возможность сформулировать следующее свойство 
гам\-ма-экс\-по\-нен\-ци\-аль\-ной функции. %(\ref{GEF}):

\bigskip

\noindent
\textbf{Следствие~1.}\
Для всех $z\in\mathbb{R}$ при $0\hm<v/u\hm<1$, $p-z/u\hm>0$, $q+z/v\hm>0$, $c\hm>0$ выполняется 
соотношение:
\begin{multline*}
\int\limits_0^\infty x^{z+vq-1}{\sf Ge}_{v/u,\, vq/u+p}\left(-c x^{v}\right)\,dx={}\\
{}=
\fr{\Gamma(q+z/v)\Gamma(p-z/u)}{\abs{v}c^{q+z/v}}\,,
\end{multline*}
в частности
$$
\int\limits_0^\infty x^{q-1}{\sf Ge}_{\alpha,\, \alpha q+p}\left(-c x\right)\,dx=
\fr{\Gamma(q)\Gamma(p)}{c^{q}}\,.
$$


\smallskip

Перейдем к~рассмотрению результата, касающегося описания функции распределения 
отношения двух независимых случайных величин, имеющих обобщенное гамма-распределение.

\bigskip

\noindent
\textbf{Теорема 3.}\
\textit{Пусть независимые случайные величины $\lambda$ и~$\mu$ имеют соответственно 
распределения $\mathrm{GG}\,(v,q,\theta)$ и~$\mathrm{GG}\,(u,p,\alpha)$. Тогда их отношение 
$\rho=\lambda/\mu$ при $x>0$ имеет функцию распределения}:
$$
F_{\rho}(x) =
 \begin{cases}
   \displaystyle {\sf Gi}\left(\fr{v}{u},p,q;\,\left(\fr{\alpha x}{\theta}\right)^{vq}\right), 
&u>v>0;\\[9pt]
   \displaystyle 1-{\sf Gi}\left(\fr{v}{u},p,q;\,\left(\fr{\alpha x}{\theta}\right)^{vq}\right), 
&u<v<0;\\[9pt]
   \displaystyle 1-{\sf Gi}\left(\fr{u}{v},q,p;\,\left(\fr{\alpha x}{\theta}\right)^{-up}\right), 
&v>u>0;\\[9pt]
   \displaystyle {\sf Gi}\left(\fr{u}{v},q,p;\,\left(\fr{\alpha x}{\theta}\right)^{-up}\right), 
&v<u<0.
 \end{cases}
$$


\noindent
Д\,о\,к\,а\,з\,а\,т\,е\,л\,ь\,с\,т\,в\,о\,.\ \ Пусть $u\hm>v\hm>0$. Имеем при $x\hm>0$:
\begin{multline*}
F_\rho(x)={}\\
{}=\int\limits_0^x
\fr{{v}(\alpha/\theta)^{vq}y^{vq-1}}{\Gamma(p)\Gamma(q)}\,{\sf Ge}_{v/u,\, 
vq/u+p}\!\left(-\!\left(\fr{\alpha y}{\theta}\!\right)^{v}\right)
\, dy={}\\
{}=\fr{{v}(\alpha/\theta)^{vq}}{\Gamma(p)\Gamma(q)}\int\limits_0^x
\sum\limits_{k=0}^{\infty}\fr{(-1)^ky^{vk+vq-1}}{(\theta/\alpha)^{vk}k!} \times{}\\
{}\times
\Gamma\left(\fr{ vk + vq+up}{u}\right)
\, dy=\fr{1}{q\Gamma(p)\Gamma(q)}\times{}\\
{}\times \int\limits_0^{(\alpha x/\theta)^{vq}}
\sum\limits_{k=0}^{\infty}\fr{\left(-y^{1/q}\right)^{k}}{k!} \,
\Gamma\left(\fr{ vk + vq+up}{u}\right)
\, {dy}={}\\
{}=\fr{1}{q\Gamma(p)\Gamma(q)}\int\limits_0^{(\alpha x/\theta)^{vq}}
{\sf Ge}_{v/u,\,vq/u+p}\left(-y^{1/q}\right)\, {dy}.
\end{multline*}

Пусть $u<v<0$. Имеем при $x\hm>0$:
\begin{multline*}
F_\rho(x)=-\int\limits_0^x
\fr{{v}(\alpha/\theta)^{vq}y^{vq-1}}{\Gamma(p)\Gamma(q)}\times{}\\
{}\times {\sf Ge}_{v/u,\, 
vq/u+p}\left(-\left(\fr{\alpha y}{\theta}\right)^{v}\right)
\, dy={}\\
{}=-\fr{{v}(\alpha/\theta)^{vq}}{\Gamma(p)\Gamma(q)}\int\limits_0^x
\sum\limits_{k=0}^{\infty}\fr{(-1)^ky^{vk+vq-1}}{(\theta/\alpha)^{vk}k!} \times{}\\
{}\times
\Gamma\left(\fr{ vk + vq+up}{u}\right)
\, dy=\fr{1}{q\Gamma(p)\Gamma(q)}\times{}\\
{}\times \int\limits_{(\alpha x/\theta)^{vq}}^\infty
\sum\limits_{k=0}^{\infty}\fr{\left(-y^{1/q}\right)^{k}}{k!} \,
\Gamma\left(\fr{ vk + vq+up}{u}\right)
\, {dy}={}\\
{}=1-\fr{1}{q\Gamma(p)\Gamma(q)}\int\limits_0^{(\alpha x/\theta)^{vq}}
{\sf Ge}_{v/u,\, vq/u+p}\left(-y^{1/q}\right)\, {dy}\,.
\end{multline*}

Случаи $v>u>0$ и~$v\hm<u\hm<0$ рассматриваются аналогично. Теорема доказана.

\bigskip

\noindent
\textbf{Следствие~2.}\
Для интегральной гамма-экс\-по\-нен\-ци\-аль\-ной функции справедливо соотношение:
$$
\lim\limits_{x\to\infty}{\sf Gi}\left(r,s,t;\,x\right)=1\,.
$$


\smallskip

Заметим, что при $q\hm=1$ и~$p\hm=1$ функция распределения отношения двух независимых 
случайных величин, имеющих обобщенное гам\-ма-рас\-пре\-де\-ление, может быть 
представлена через гам\-ма-экс\-понен\-ци\-аль\-ную функцию. Это дает возможность\linebreak 
исследовать масштабные смеси распределений Вейбулла и~Фреше без использования 
интегральной гам\-ма-экс\-по\-нен\-ци\-аль\-ной функции.

\bigskip

\noindent
\textbf{Следствие~3.}\
Пусть независимые случайные величины~$\lambda$ и~$\mu$ имеют соответственно 
распределения $\mathrm{GG}\,(v,q,\theta)$ и~$\mathrm{GG}\,(u,p,\alpha)$. Тогда их отношение 
$\rho\hm=\lambda/\mu$ при $x\hm>0$ имеет функцию распределения:
$$
F_{\rho}(x) =
 \begin{cases}
   \displaystyle \fr{{v}\alpha^{vq}x^{vq}}{u\theta^{vq}\Gamma(q)}\,
{\sf Ge}_{v/u,\,vq/u}\left(-\left(\fr{\alpha x}{\theta}\right)^v\right), 
&\\[6pt]
&\hspace*{-30mm}u>v>0,\ p=1;\\[6pt]
   \displaystyle 1-\fr{1}{\Gamma(p)}\,{\sf Ge}_{v/u,\,p}\left(-\left(\fr{\alpha 
x}{\theta}\right)^v\right), &\\[6pt]
&\hspace*{-30mm}u<v<0,\ q=1;\\[6pt]
   \displaystyle 1-\fr{1}{\Gamma(q)}\,{\sf Ge}_{u/v,\,q}\left(-\left(\fr{\alpha 
x}{\theta}\right)^{-u}\right), &\\[6pt]
&\hspace*{-30mm}v<u<0,\ p=1;\\[6pt]
   \displaystyle \fr{u\theta^{up}x^{-up}}{v\alpha^{up}\Gamma(p)}\,
{\sf Ge}_{u/v,\,up/v}\left(-\left(\fr{\alpha x}{\theta}\right)^{-u}\right), 
&\\[6pt]
&\hspace*{-30mm}v<u<0,\ q=1.
 \end{cases}
$$

\noindent
Д\,о\,к\,а\,з\,а\,т\,е\,л\,ь\,с\,т\,в\,о\,.\ \ Пусть $u\hm>v\hm>0$. Имеем при $x\hm>0$:
\begin{multline*}
F_\rho(x)=\fr{{v}(\alpha/\theta)^{vq}}{\Gamma(p)\Gamma(q)}\times{}\\
{}\times\int\limits_0^x
\sum\limits_{k=0}^{\infty}\fr{(-1)^ky^{vk+vq-1}}{(\theta/\alpha)^{vk}k!} 
\,\Gamma\left(\fr{ vk + vq+up}{u}\right)\, dy={}\\
\!{}=\fr{(\alpha/\theta)^{vq}}{\Gamma(p)\Gamma(q)}
\sum\limits_{k=0}^{\infty}\fr{(-1)^kx^{vk+vq}}{(\theta/\alpha)^{vk}(k+q)k!}\, 
\Gamma\left(\fr{ vk + vq+up}{u}\right)\!.\hspace*{-8.51317pt}
\end{multline*}
При $p=1$ получаем:
$$
F_\rho(x)=\fr{v(\alpha/\theta)^{vq}}{u\Gamma(q)}
\sum\limits_{k=0}^{\infty}\fr{(-1)^kx^{vk+vq}}{(\theta/\alpha)^{vk}k!} \,
\Gamma\left(\fr{ vk + vq}{u}\right).
$$
При $q=1$ получаем
\begin{multline*}
F_\rho(x)=-\fr{1}{\Gamma(p)}\times{}\\
{}\times
\sum\limits_{k=0}^{\infty}\fr{(-1)^{k+1}x^{v(k+1)}}{(\theta/\alpha)^{v(k+1)}(k+1)!} \,
\Gamma\left(\fr{v(k+1) + up}{u}\right)={}\\
{}=1-\fr{1}{\Gamma(p)}
\sum\limits_{k=0}^{\infty}\fr{(-1)^{k}x^{vk}}{(\theta/\alpha)^{vk}k!} 
\,\Gamma\left(\fr{vk + up}{u}\right).
\end{multline*}

Случай $v<u<0$ рассматривается аналогично.
Следствие доказано.

\bigskip

\noindent
\textbf{Замечание~3.}\
Следствие~3 можно переформулировать в~терминах 
гам\-ма-экс\-по\-нен\-ци\-аль\-ной функции как свойство инвариантности относительно 
интегрального преобразования при
$\alpha,c,p,q\hm>0$:
$$
\int\limits_0^x
{y^{\alpha q-1}}{\sf Ge}_{\alpha,\, \alpha q+1}\left(-{c y}^{\alpha}\right)
\, dy={x^{\alpha q}}{\sf Ge}_{\alpha,\,\alpha q}\left(-{c x}^\alpha\right);
$$

\vspace*{-12pt}


\noindent
\begin{multline*}
\int\limits_0^x
{c{\alpha}y^{-\alpha-1}}{\sf Ge}_{\alpha,\, \alpha+p}\left(-cy^{-\alpha}\right)
\, dy={}\\
{}=\Gamma(p)-{\sf Ge}_{\alpha,\,p}\left(-cx^{-\alpha}\right).
\end{multline*}

\section{Заключение}

Приведенные результаты дают возможность исследовать модели, в~которых 
используются отношения и~произведения случайных величин, имеющих обобщенное 
гам\-ма-рас\-пре\-де\-ле\-ние, в~терминах гам\-ма-экс\-по\-нен\-ци\-аль\-ной 
функции, свойства которой 
во многом унаследованы от показательной функции.

\bigskip

Автор выражает признательность В.\,Ю.~Королеву, Ю.\,Н.~Недоливко, 
С.\,И.~Палионной, В.\,В.~Саенко, А.\,И.~Титовой, Ю.\,С.~Хохлову и~С.\,Я.~Шоргину 
за содействие при подготовке публикации.

{\small\frenchspacing
 {%\baselineskip=10.8pt
 \addcontentsline{toc}{section}{References}
 \begin{thebibliography}{99}
\bibitem{Shorgin05}
\Au{Шоргин~С.\,Я.}
О~байесовских моделях массового обслуживания~// 
II~Научная сессия Института проблем информатики РАН: Тезисы докладов.~--- 
М.: ИПИ РАН, 2005. С.~120--121.

\bibitem{KuSoSh}
\Au{Кудрявцев А.\,А., Соколов~И.\,А., Шоргин~С.\,Я.} Байесовская
рекуррентная модель роста надежности: равномерное распределение
параметров~// Информатика и~её применения, 2013. Т.~7. Вып.~2.
С.~55--59.

\bibitem{Ku2018}
\Au{Кудрявцев~А.\,А.}
Байесовские модели баланса~// Информатика и~её применения, 2018. Т.~12. Вып.~3. С.~18--27.

\bibitem{KuTi2017}
\Au{Кудрявцев~А.\,А., Титова~А.\,И.}
Гам\-ма-экс\-по\-нен\-ци\-аль\-ная функция в~байесовских моделях массового обслуживания~// 
Информатика и~её применения, 2017. Т.~11. Вып.~4. С.~104--108.

\bibitem{LeRoy1900_1}
\Au{Le~Roy~\,$\acute{\mbox{\!E}}$.}
Sur les s$\acute{\mbox{e}}$ries divergentes et les fonctions 
d$\acute{\mbox{e}}$finies par un d$\acute{\mbox{e}}$veloppement de Taylor~// 
Annales de la facult$\acute{\mbox{e}}$ des sciences de Toulouse
 2e s$\acute{\mbox{e}}$rie, 1900. Vol.~2. No.~3. P.~317--384.

\bibitem{SrTo2009}
\Au{Srivastava~H.\,M., Tomovski~{\ptb{\v{Z}}}.}
Fractional calculus with an integral operator containing a generalized 
Mittag-Leffler function in the kernel~// Appl. Math. Comput., 2009. Vol.~211. P.~198--210.

\bibitem{GoKiMaRo2014}
\Au{Gorenlo~R., Kilbas~A.\,A., Mainardi~F., Rogosin~S.\,V.}
Mittag-Leffler functions, related topics and applications.~--- Berlin--Heidelberg: Springer-Verlag, 2014.\linebreak
 443~p.

\bibitem{KuPaSh2019_2}
\Au{Кудрявцев~А.\,А., Палионная~С.\,И., Шоргин~В.\,С.}
Априорное обобщенное распределение Фреше в~байесовских моделях баланса~// Системы 
и~средства информатики, 2019. Т.~29. №\,2. С.~39--45.

\bibitem{Stacy1962}
\Au{Stacy~E.\,W.}
A generalization of the gamma distribution~// 
Ann. Math. Stat., 1962. Vol.~33. P.~1187--1192.

\bibitem{KoPoSkTu1985}
\Au{Королюк~В.\,С., Портенко~Н.\,И., Скороход~А.\,В., Турбин~А.\,Ф.}
Справочник по теории вероятностей и~математической статистике.~--- 
М.: Наука, 1985.\linebreak 640~с.

\bibitem{Lee2012}
\Au{Lee~P.\,M.}
Bayesian statistics: An introduction.~--- 4th ed.~--- 
Chichester, U.K.: John Wiley \& Sons, 2012. 462~p.

\bibitem{KlKo2003}
\Au{Kleiber~C., Kotz~S.}
Statistical size distributions in economics and actuarial sciences.~--- New
York, NY, USA: John Wiley \& Sons, 2003. 332~p.

\bibitem{Kruglov2016}
\Au{Круглов~В.\,М.}
Случайные процессы. Ч.~1. Основы общей теории.~--- 2-е изд., перераб. и~доп.~--- М.: Юрайт, 2016. 276~с.

\bibitem{Gelman2014}
\Au{Gelman~A., Carlin~J.\,B., Stern~H.\,S., Dunson~D.\,B., Vehtari~A., 
Rubin~D.\,B.} Bayesian data analysis.~--- 3rd ed.~--- CRC Press, 2014. 639~p.

\bibitem{Nakagami}
\Au{Nakagami M.}
The m-distribution, a~general formula of intensity of rapid fading~// 
Statistical Methods in Radio Wave Propagation: Symposium Proceedings~/ 
Ed. W.\,C.~Homan.~--- New York, NY, USA: Pergamon Press, 1960. 
P.~3--36.

\bibitem{WiHi1931}
\Au{Wilson~E.\,B., Hilferty~M.\,M.}
The distribution of chi-square~// P.~Natl. Acad. Sci. USA, 1931. 
Vol.~17. No.\,12. P.~684--688.

\bibitem{Lai2014}
\Au{Lai~C.-D.\/}
Generalized Weibull distributions.~--- New York, NY, USA: Springer, 2014. 128~p.

\bibitem{Voda1989}
\Au{Vod{\!\ptb{\!\v{a}}}~V.\,G.}
New models in durability tool-testing: Pseudo-Weibull distribution~// 
Kybernetika, 1989. Vol.~25. No.\,3. P.~209--215.

\bibitem{Panwar2015}
\Au{Panwar~M.\,S., Sudhir~B.\,A., Bundel~R., Tomer~S.\,K.}
Parameter estimation of inverse Rayleigh distribution under\linebreak\vspace*{-12pt}

\pagebreak

\noindent
  competing risk 
model for masked data~// J.~Inst. Sci. Technol., 2015. Vol.~20. 
No.\,2. P.~122--127.



\bibitem{Frechet1927}
\Au{Fr$\acute{\mbox{e}}$chet~M.\/}
Sur la loi de probabilit$\acute{\mbox{e}}$ de l'$\acute{\mbox{e}}$cart maximum~// 
Ann. Soc. Polonaise Math., 1927. Vol.~6. P.~93--116.



\bibitem{McDonald1984}
\Au{McDonald~J.\,B.}
Some generalized functions for the size distribution of income~// 
Econometrica, 1984. Vol.~52. No.\,3. P.~647--665.

\bibitem{Burr1942}
\Au{Burr~I.\,W.}
Cumulative frequency functions~// Ann. Math. Stat., 1942. Vol.~13. Iss.~2. P.~215--232.

\bibitem{SiMa1976}
\Au{Singh~S.\,K., Maddala~G.\,S.}
A~function for size distribution of incomes~// Econometrica, 1976. Vol.~44. No.\,5. P.~963--970.

\bibitem{Dagum1977}
\Au{Dagum~C.}
A~new model of personal income-distribution-specification and estimation~// 
Econ. Appl., 1977. Vol.~30. No.\,3. P.~413--437.

\bibitem{Lomax1954}
\Au{Lomax~K.\,S.}
Business failures: Another example of the analysis of failure data~// 
J.~Am. Stat. Assoc., 1954. Vol.~49. No.\,268. 
P.~847--852.
 \end{thebibliography}

 }
 }

\end{multicols}

\vspace*{-6pt}

\hfill{\small\textit{Поступила в~редакцию 07.06.19}}

\vspace*{8pt}

%\pagebreak

%\newpage

%\vspace*{-28pt}

\hrule

\vspace*{2pt}

\hrule

%\vspace*{-2pt}

\def\tit{\textit{A~PRIORI} GENERALIZED GAMMA DISTRIBUTION IN~BAYESIAN BALANCE MODELS}


\def\titkol{\textit{A~priori} generalized gamma distribution in~Bayesian balance models}

\def\aut{A.\,A.~Kudryavtsev}

\def\autkol{A.\,A.~Kudryavtsev}

\titel{\tit}{\aut}{\autkol}{\titkol}

\vspace*{-11pt}


\noindent
Department of Mathematical Statistics, Faculty of Computational 
Mathematics and Cybernetics, M.\,V.~Lomonosov Moscow State University, 
1-52~Leninskiye Gory, GSP-1, Moscow 119991, Russian Federation


\def\leftfootline{\small{\textbf{\thepage}
\hfill INFORMATIKA I EE PRIMENENIYA~--- INFORMATICS AND
APPLICATIONS\ \ \ 2019\ \ \ volume~13\ \ \ issue\ 3}
}%
 \def\rightfootline{\small{INFORMATIKA I EE PRIMENENIYA~---
INFORMATICS AND APPLICATIONS\ \ \ 2019\ \ \ volume~13\ \ \ issue\ 3
\hfill \textbf{\thepage}}}

\vspace*{3pt}  



\Abste{The work is devoted to the study of Bayesian balance models, 
involving the division of the system parameters into two classes: supporting 
system  functioning positive factors and interfering with the functioning 
negative factors. The balance index, defined as the ratio of the negative 
factor to the positive factor, is considered. The formulation of the problem, 
which consists in finding the main probabilistic characteristics (density, 
distribution function, and moments) of the balance index of factors having 
\textit{a~priori} generalized gamma distribution with the parameters of 
the form of one sign, is studied. The results are formulated in terms of the 
gamma-exponential function. A~number of new properties of the latter are given. 
It is shown that the given statements are easily reformulated for large-scale 
mixtures of generalized gamma distributions with parameters of the form of 
different signs. The obtained results can be widely used in models, which 
describe the processes and phenomena using distributions with a~positive 
unlimited support.}

\KWE{Bayesian approach; generalized gamma distribution; gamma-exponential function; 
balance models; mixed distributions}


 
\DOI{10.14357/19922264190305} 

%\vspace*{-14pt}

\Ack
\noindent
The work was partly supported by the Russian Foundation for Basic 
Research (project 17-07-00577).


%\vspace*{-6pt}

  \begin{multicols}{2}

\renewcommand{\bibname}{\protect\rmfamily References}
%\renewcommand{\bibname}{\large\protect\rm References}

{\small\frenchspacing
 {%\baselineskip=10.8pt
 \addcontentsline{toc}{section}{References}
 \begin{thebibliography}{99}

\bibitem{1-ku}
\Aue{Shorgin, S.\,Ya.} 2005. O~bayesovskikh modelyakh massovogo obsluzhivaniya 
[On Bayesian queuing models]. 
 \textit{II Nauchnaya sessiya Instituta problem informatiki RAN: Tezisy dokladov}
  [2nd Scientific Session of the Institute of Informatics Problems of the 
  Russian Academy of Sciences: Abstracts]. Moscow: IPI RAN. 120--121.
\bibitem{2-ku}
\Aue{Kudryavtsev, A.\,A., I.\,A.~Sokolov, and S.\,Ya.~Shorgin.} 
2013. Bayesovskaya rekurrentnaya model' rosta nadezhnosti: 
ravnomernoe raspredelenie parametrov [Bayesian recurrent model of reliability growth: 
Uniform distribution of parameters]. 
 \textit{Informatika i~ee Primeneniya~--- Inform. Appl.} 7(2):55--59.
\bibitem{3-ku}
\Aue{Kudryavtsev, A.\,A.} 2018. Bayesovskie modeli balansa 
[Bayesian balance models]. 
 \textit{Informatika i~ee Primeneniya~--- Inform. Appl.} 12(3):18--27.
\bibitem{4-ku}
\Aue{Kudryavtsev, A.\,A., and A.\,I.~Titova.}
 2017. Gamma-eksponentsial'naya funktsiya v~bayesovskikh modelyakh 
 massovogo obsluzhivaniya [Gamma-exponential function in Bayesian queuing models]. 
 \textit{Informatika i~ee Primeneniya~--- Inform. Appl.} 11(4):104--108.
\bibitem{5-ku}
\Aue{Le Roy, $\acute{\mbox{E}}$}. 1900. Sur les s$\acute{\mbox{e}}$ries 
divergentes et les fonctions d$\acute{\mbox{e}}$finies par un 
d$\acute{\mbox{e}}$veloppement de Taylor. 
\textit{Annales de la facult$\acute{\mbox{e}}$ des sciences de Toulouse 2e s$\acute{\mbox{e}}$rie}
2(3):\linebreak 317--384.

\bibitem{6-ku}
\Aue{Srivastava, H.\,M., and {\ptb{\v{Z}}}.~Tomovski.} 2009. Fractional calculus with 
an integral operator containing a~generalized Mittag-Leffler function in the kernel. 
 \textit{Appl. Math. Comput.} 211:198--210.

\bibitem{7-ku}
\Aue{Gorenlo, R., A.\,A.~Kilbas, F.~Mainardi, and S.\,V.~Rogosin.}
 2014.  \textit{Mittag-Leffler functions, related topics and applications}. 
 Berlin\,--\,Heidelberg: Springer-Verlag. 443~p.

\bibitem{8-ku}
\Aue{Kudryavtsev, A.\,A., S.\,I.~Palionnaia, and V.\,S.~Shorgin.}
 2019. Apriornoe obobshchennoe raspredelenie Freshe v~bayesovskikh modelyakh 
 balansa [\textit{A~priori} generalized Frechet distribution in Bayesian balance models]. 
  \textit{Sistemy i~Sredstva Informatiki~--- Systems and Means of Informatics}
   29(2):39--45.

\bibitem{9-ku}
\Aue{Stacy, E.\,W.}
 1962. A~generalization of the gamma distribution.
  \textit{Ann. Math. Stat.} 33:1187--1192.

\bibitem{10-ku}
\Aue{Korolyuk, V.\,S., N.\,I.~Portenko, A.\,V.~Skorokhod, and A.\,F.~Turbin.}
 1985.  \textit{Spravochnik po teorii veroyatnostey i~matematicheskoy statistike} 
 [Handbook of probability theory and mathematical statistics]. 
 Moscow: Nauka. 640~p.

\bibitem{11-ku}
\Aue{Lee, P.\,M.} 2012. 
\textit{Bayesian statistics: An introduction}. 4th ed. 
Chichester, U.K.: John Wiley \& Sons. 462~p.

\bibitem{12-ku}
\Aue{Kleiber, C., and S.~Kotz.}
 2003.  \textit{Statistical size distributions in economics and actuarial sciences}. 
 New York, NY: John Wiley \& Sons. 332~p.

\bibitem{13-ku}
\Aue{Kruglov, V.\,M.}
 2016. \textit{Sluchaynye protsessy. Ch.~1. Osnovy obshchey teorii} 
 [Stochastic processes. Part~1. Bases of general theory]. 2nd ed.
  Moscow: Yurayt. 276~p.

\bibitem{14-ku}
\Aue{Gelman, A., J.\,B.~Carlin, H.\,S.~Stern, D.\,B.~Dunson, A.~Vehtari, and D.\,B.~Rubin.}
 2014.  \textit{Bayesian data analysis}. 3rd ed. CRC Press. 639~p.

\bibitem{15-ku}
\Aue{Nakagami, M.} 1960. The m-distribution, a~general formula 
of intensity of rapid fading.  \textit{Statistical Methods in 
Radio Wave Propagation: Symposium Proceedings}. 
Ed. W.\,C.~Hoffman. New York, NY: Pergamon Press. 3--36.

\bibitem{16-ku}
\Aue{Wilson, E.\,B., and M.\,M.~Hilferty.} 1931. The distribution of chi-square. 
 \textit{P.~Natl. Acad. Sci. USA} 17(12):684--688.

\bibitem{17-ku}
\Aue{Lai, C.-D.} 2014.  \textit{Generalized Weibull distributions}. New York, NY: 
Springer. 128~p.

\bibitem{18-ku}
\Aue{Vod{\mbox{\normalsize\!\ptb{\v {a}}}},~V.\,G.} 1989. New models in durability tool-testing: 
Pseudo-Weibull distribution.  \textit{Kybernetika} 25(3):209--215.

\bibitem{19-ku}
\Aue{Panwar, M.\,S., B.\,A.~Sudhir, R.~Bundel, and S.\,K.~Tomer.}
 2015. Parameter estimation of inverse Rayleigh distribution under 
 competing risk model for masked data. 
 \textit{J.~Inst. Sci. Technol.} 20(2):122--127.

\bibitem{20-ku}
\Aue{Fr$\acute{\mbox{e}}$chet, M.} 1927. 
Sur la loi de probabilit$\acute{\mbox{e}}$ de l'$\acute{\mbox{e}}$cart 
maximum. \textit{Ann. Soc. Polonaise  Math.} 6:93--116.

\bibitem{21-ku}
\Aue{McDonald, J.\,B.} 1984.
 Some generalized functions for the size distribution of income. 
 \textit{Econometrica} 52(3):647--665.

\bibitem{22-ku}
\Aue{Burr, I.\,W.}
 1942. Cumulative frequency functions.
 \textit{Ann. Math. Stat.} 13(2):215--232.

\bibitem{23-ku}
\Aue{Singh, S.\,K., and G.\,S.~Maddala.} 1976. 
A~function for size distribution of incomes. \textit{Econometrica} 44(5):963--970.

\bibitem{24-ku}
\Aue{Dagum, C.} 1977. A~new model of personal income-distribution-specification 
and estimation. 
\textit{Econ. Appl.} 30(3):413--437.

\bibitem{25-ku}
\Aue{Lomax, K.\,S.} 1954. Business failures: Another 
example of the analysis of failure data. 
\textit{J.~Am. Stat. Assoc.} 49(268):847--852.
\end{thebibliography}

 }
 }

\end{multicols}

%\vspace*{-7pt}

\hfill{\small\textit{Received June 7, 2019}}

%\pagebreak

%\vspace*{-22pt}


\Contrl

\noindent
\textbf{Kudryavtsev Alexey A.} (b.\ 1978)~--- 
Candidate of Sciences (PhD) in physics and mathematics, associate professor,
 Department of Mathematical Statistics, Faculty of Computational Mathematics 
 and Cybernetics, M.\,V.~Lomonosov Moscow State University, 1-52 Leninskiye Gory, 
 GSP-1, Moscow 119991, Russian Federation; \mbox{nubigena@mail.ru}
\label{end\stat}

\renewcommand{\bibname}{\protect\rm Литература}  