\def\stat{agasandyan}

\def\tit{ВЫЧИСЛЕНИЕ ПОКАЗАТЕЛЕЙ ОПТИМАЛЬНЫХ ПО CC-VaR ПОРТФЕЛЕЙ НА 
РЫНКАХ ОПЦИОНОВ$^*$}

\def\titkol{Вычисление показателей оптимальных по CC-VaR портфелей на 
рынках опционов}

\def\aut{Г.\,А.~Агасандян$^1$}

\def\autkol{Г.\,А.~Агасандян}

\titel{\tit}{\aut}{\autkol}{\titkol}

\index{Агасандян Г.\,А.}
\index{Agasandyan G.\,A.}


{\renewcommand{\thefootnote}{\fnsymbol{footnote}} \footnotetext[1]
{Работа выполнена при финансовой поддержке РФФИ (проект 17-01-00816).}}


\renewcommand{\thefootnote}{\arabic{footnote}}
\footnotetext[1]{Вычислительный центр им.\ А.\,А.~Дородницына Федерального исследовательского 
центра <<Информатика и~управ\-ле\-ние>> Российской академии наук, 
\mbox{agasand17@yandex.ru}}

%\vspace*{-2pt}
        
  
  
  \Abst{Работа продолжает исследования автора по проблеме применения на финансовых 
рынках континуального критерия VaR (CC-VaR). Речь идет о проецировании идей 
и~методов, разработанных для задачи инвестирования на теоретическом однопериодном 
рынке с~одним базовым активом, на дискретный по страйкам рынок опционов с~небольшим 
числом сценариев. Основное внимание уделяется методам расчета функций распределения 
дохода и~доходности, среднего дохода и~средней доходности для оптимального по CC-VaR 
портфеля опционов и~его рандомизированных версий~--- как полных, так и~частичных. 
Предлагаются эффективные по скорости и~точности алгоритмы вычислений для разных 
вариантов задач оптимизации. Исследование иллюстрируется на примере  
с~бе\-та-рас\-пре\-де\-лен\-ны\-ми рыночными ценами базового актива и~прогнозом 
инвестора, сопровождаемом графиками.}
  
  \KW{континуальный критерий VaR (CC-VaR); функция рисковых предпочтений (ф.р.п.); 
сценарии; опционы; процедура Ней\-ма\-на--Пир\-со\-на; индикаторы; баттерфляи; полная 
и~частичная рандомизации; оптимальный портфель; доход; доходность}

\DOI{10.14357/19922264190311} 
  
%\vspace*{1pt}


\vskip 10pt plus 9pt minus 6pt

\thispagestyle{headings}

\begin{multicols}{2}

\label{st\stat}
  
  \section{Введение}
  
  Работа продолжает исследования автора по проб\-ле\-ме применения на 
финансовых рынках континуального критерия VaR (CC-VaR)~[1, 2]. Речь идет 
о проецировании идей и~методов, разработанных для задачи инвестирования на 
теоретическом однопериодном рынке с~одним базовым активом, на дискретный 
рынок с~небольшим числом сценариев (уже близкий к~реальному). 
  
  В значительной мере вопросы такого проецирования на сценарные рынки 
были изучены уже в~\cite{1-aga, 2-aga, 3-aga, 4-aga}. Здесь основное внимание 
уделяется распространению аналогичных методов на дискретные по страйкам 
\textit{опционные} рынки и~их рандомизированные версии. Обсуждается 
инструментарий опционных рынков. Для него предлагается базис из 
простейших нормированных баттерфляев, в~котором можно строить 
инвестиционные портфели. 
  
  Приводятся формулы расчета основных числовых показателей оптимальных 
портфелей и~вычислительные алгоритмы, специально при\-спо\-соб\-лен\-ные к~части 
возникающих при этом задач и~демонстрирующие свою эффективность. 
Исследование иллюстрируется на примере с~бета-распределенными рыночными 
ценами базового актива и~прогнозом инвестора. Приводятся графики функций 
распределения дохода для вариантов инвестиционного портфеля. 
  
  \section{Теоретический однопериодный $\delta$-рынок 
и~опционы}
  
  Напомним основные понятия и~обозначения для теоретического 
однопериодного $\delta$-рын\-ка и~распространим его конструкции на 
опционные инструменты. Известная на начало периода цена базового 
актива~$\boldsymbol{X}$ в~его конце образует случайную 
величину~${X}$, принимающую значения~$x$ из континуального 
множества ${\sf X}\hm\subset \mathfrak{R}_+$ (или даже~$\mathfrak{R}$). На 
рынке обращаются  
$\delta$-\textit{инструменты} $\boldsymbol{D}(s)$, $s\hm\in {\sf X}$, платежная 
функция которых равна $\delta$-функ\-ции относительно~$s$: 

\vspace*{-3pt}

\noindent
$$
\pi(x;  \boldsymbol{D}(s))\hm\equiv \delta (x\hm-s)\,. 
$$

\vspace*{-3pt}
  
  Заданы \textit{прогнозная} $p(x)$ и~\textit{стоимостная}~$c(x)$, $x\hm\in 
{\sf X}$, плотности. На рынке, называемом $\delta$-\textit{рын\-ком}, можно 
торговать любым инструментом~$\boldsymbol{G}$ с~доходом, представимым 
в~виде произвольной неотрицательной измеримой функции~$g(x)$, $x\hm\in 
{\sf X}$. Ее называем \textit{платежной функцией} инструмента и~обозначаем 
$\pi (x; \boldsymbol{G})$, $x\hm\in {\sf X}$, т.\,е.\ $g(x) \hm= \pi(x; \boldsymbol{G})$; 
также $\pi(x; \boldsymbol{X}) \hm= x$. Цена инструмента~$\boldsymbol{G}$ 
обозначается $\vert\boldsymbol{G}\vert$, а средний доход от него~---
~$\|\boldsymbol{G}\|$. 
  
  В частности, инструментами $\delta$-\textit{рын\-ка} являются индикаторы 
множеств~$\boldsymbol{H}\{M\}$, $M\hm\subset {X}$, играющие роль 
\textit{базисных} инструментов для \textit{сценарных} рынков.
 Аналогичную 
роль на рассматриваемых далее рын-\linebreak\vspace*{-12pt}

\pagebreak

\noindent
ках обычных опционов (коллов и~путов) 
играют простейшие нормированные баттерфляи. В~терминах теоретического 
$\delta$-рын\-ка основные опционы, такие как коллы~$\boldsymbol{C}_s$, 
путы~$\boldsymbol{P}_s$, спрэды быка~$\boldsymbol{S}_s^+$ 
и~медведя~$\boldsymbol{S}_s^-$, баттерфляи~$\boldsymbol{B}_s$, $s\hm\in {\sf 
X}$, их платежные функции и~теоремы паритета имеют представления: 
  \begin{align*}
  \boldsymbol{C}_s&=\!\!\!\int\limits_{\{x\geq s\}}\! \!(x-s) \boldsymbol{D}(x)\,dx\,,\enskip 
  \pi\left(x; \boldsymbol{C}_s\right) = \max (0, x-s);\\
  \boldsymbol{P}_s&=\!\!\!\int\limits_{\{x<s\}}\!\! (s-x) \boldsymbol{D}(x)\,dx\,,\ \pi\left(x; 
\boldsymbol{P}_s\right) =\max(0, s-x);\\
&\hspace*{20mm}\boldsymbol{C}_s-\boldsymbol{P}_s=\boldsymbol{X}-s\boldsymbol{U};
  \end{align*}
  
  \vspace*{-12pt}
  
  \noindent
  \begin{multline}
%  \left.
%  \begin{array}{c}
  \boldsymbol{S}^+_{s;h} =\boldsymbol{C}_{s-h}- \boldsymbol{C}_s 
=h\boldsymbol{U} +\boldsymbol{P}_{s-h} -\boldsymbol{P}_s\,,\\
  \pi\left( x; \boldsymbol{S}^+_{s;h} \right)= \min \left( h,\max \left( 0,x-(s-
h)\right)\right)\,;
%\end{array}
%\right\}
\label{e1-aga}
\end{multline}
\ \vspace*{-12pt}
  
  \noindent
  \begin{multline}
  \boldsymbol{S}^-_{s;h} =\boldsymbol{P}_{s+h}-
\boldsymbol{P}_s=h\boldsymbol{U} +\boldsymbol{C}_{s+h} -
\boldsymbol{C}_s\,,\\[6pt]
  \pi\left( x; \boldsymbol{S}^-_{s;h}\right) =\min(h,\max(0,(s+h)-x))\,;
  %\end{array}
 % \right\}
 \label{e2-aga}
  \end{multline}
  
  \vspace*{-12pt}
  
  \noindent
  \begin{multline}
  \boldsymbol{B}_{s;h} =\boldsymbol{S}^+_{s;h}+\boldsymbol{S}^-
_{s;h}=\boldsymbol{C}_{s-h} -2\boldsymbol{C}_s +\boldsymbol{C}_{s+h} 
={}\\
{}=\boldsymbol{P}_{s-h} -2\boldsymbol{P}_s+\boldsymbol{P}_{s+h}={}\\
  {}= h\boldsymbol{U} +\boldsymbol{P}_{s-h} -\boldsymbol{P}_s-
\boldsymbol{C}_s +\boldsymbol{C}_{s+h}\,,\enskip\\
  \pi\left( x; \boldsymbol{B}_{s;h}\right) =\max (0,h-\vert x-s\vert )\,,
  \label{e3-aga}
  \end{multline}
где $s$~--- страйк колла и~пута, короткого колла или пута у спрэда 
и~центральный у~баттерфляя; $h$~--- расстояние между соседними страйками, 
$h\hm> 0$. (При необходимости формируются и~несимметричные баттерфляи.) 

  Из специально нормированных баттерфляев можно строить сходящиеся 
к~$\delta$-ин\-стру\-мен\-там последовательности, как это делалось  
в~\cite{4-aga} из индикаторов на сценарном рынке. И,~наконец, CC-VaR 
требует, чтобы порождаемый портфелем инвестора доход~$q$ удовлетворял 
неравенствам: 
  $$
   {\sf P}\left\{ q\geq \phi(\varepsilon)\right\} \geq 1-\varepsilon \ 
   \mbox{для \textit{всех }} \varepsilon \hm\in 
[0,1].
   $$
   Здесь ${\sf P}\{M\}$~--- вероятностная мера множества~$M$ в~соответствии 
с~прогнозом инвестора; $\phi(\varepsilon)$~--- не\-от\-ри\-ца\-тель\-ная монотонно 
возрастающая и~непрерывная \textit{функция рисковых предпочтений} (ф.р.п.)\ 
инвестора со значениями в~$[0, 1]$. 
  
  \section{Опционная дискретизация $\delta$-рынка}
  
  Напомним~\cite{2-aga, 4-aga}, что сценарная дискретизация рынка вводится 
разбиением множества ${\sf X}\hm= [x_0, x_n)$ на~$n$ сценариев $S_i\hm= 
[x_{i-1}, x_i)\hm\subset {\sf X}$, $x_{i-1}\hm< x_i$, $i\hm\in I \hm= \{1, \ldots, n\}$. 
Равномерное разбиение выделяется правилом $x_i\hm = x_0 \hm+ ih$, $h \hm= 
(x_n \hm- x_0)/n$, $i\hm\in I$. Рыночные цены базисных инструментов 
$\boldsymbol{D}_i \hm= \boldsymbol{H}\{S_i\}$, $i\hm\in I$, и~\textit{прогнозные 
вероятности} сценариев образуют векторы~$\boldsymbol{c}^S$ 
и~$\boldsymbol{p}^S$ с~компонентами 
  \begin{multline}
  c_i^S=\left\vert \boldsymbol{D}_i\right\vert =\int\limits_{x_{i-1}}^{x_i} 
c(x)\,dx\,;\\
p_i^S=\left\| \boldsymbol{D}_i \right\| =
\int\limits_{x_{i-1}}^{x_i} p(x)\,dx\,,\ i\in I\,.
  \label{e4-aga}
  \end{multline}
  
  На сценарной основе  строится и~\textit{опционная} дискретизация рынка, для 
которой страйки опционов совпадают с~серединами сценариев. С~ней 
связывается также иная система интервалов: 
  \begin{equation}
   \boldsymbol{K}=\left\{ 
    K_i=\left[ \kappa_i,\kappa_{i+1}\right)\,,\enskip
  i\in I_0\right\}\,,
  \label{e5-aga}
  \end{equation}
где $I_0=\{0\}\cup I$; $\kappa_i$, $i\hm\in I$,~--- страйки опционов:
      \begin{equation*}
   \kappa_i=\fr{x_{i-1}+x_i}{2} =\fr{i}{n}-\fr{1}{2n}\,, %\enskip i\in I\,.
     \end{equation*}
 также 
доопределяем $\kappa_0\hm=a$, $\kappa_{n+1}\hm= b$. Новые элементы 
порождают вероятностный вектор~$\boldsymbol{p}^K$ с~компонентами 
\begin{equation}
p_i^K =\int\limits_{\kappa_i}^{\kappa_{i+1}} p (x)\,dx\,,\enskip i\in I_0\,,\ 
\sum\limits_{i\in I_0} p_i^K=1\,.
\label{e6-aga}
\end{equation}
  
  Применяются еще обозначения:
  $$
  I^\prime = I\backslash \{n\} = \{1, 
\ldots , n-1\};\ I^{\prime\prime}= \{2, \ldots , n\hm-2\}.
$$ 
  
  На опционном рынке роль базисных инструментов играют 
\textit{нормированные} баттерфляи~$\boldsymbol{B}_i$, $i\hm\in I^\prime$, 
вместе с~двумя крайними \textit{нормированными} спрэдами. Ограничимся 
тремя вариантами базисов: ($i$)~только из коллов, ($ii$)~только из путов 
и~($iii$)~смешанным так называемого \textit{естественного} происхождения, 
а~именно (индекс~$i$ в~обозначениях опционов соответствует страйку~$s_i$, 
$i\hm\in I$): 
  \begin{enumerate}[($i$)]
  \item  $\boldsymbol{B}_1 = \boldsymbol{U} + (\boldsymbol{C}_2 \hm- 
\boldsymbol{C}_1)/h$,  $\boldsymbol{B}_i \hm= (\boldsymbol{C}_{i-1}\hm- 
2\boldsymbol{C}_i \hm+ \boldsymbol{C}_{i+1})/h$, $i \hm= 2,\ldots , n-1$,  
$\boldsymbol{B}_n \hm= (\boldsymbol{C}_{n-1} \hm- \boldsymbol{C}_n)/h$; 
  \item  $\boldsymbol{B}_1 \hm= (\boldsymbol{P}_2 \hm- \boldsymbol{P}_1)/h$,   
$\boldsymbol{B}_i \hm= (\boldsymbol{P}_{i-1}\hm- 2\boldsymbol{P}_i \hm+ 
\boldsymbol{P}_{i+1})/h$,  $i \hm= 2,\ldots , n\hm-1$,  $\boldsymbol{B}_n \hm= 
\boldsymbol{U} \hm+ (\boldsymbol{P}_{n-1}\hm- \boldsymbol{P}_n)/h$; 
  \item  выделяется некоторый внутренний страйк, \textit{центр рынка}, 
с~индексом~$\nu$, и~для него используется смешанный баттерфляй 
$\boldsymbol{B}_\nu \hm= \boldsymbol{U} \hm+ (\boldsymbol{P}_{\nu-1}\hm- 
\boldsymbol{P}_\nu \hm - \boldsymbol{C}_\nu \hm+ \boldsymbol{C}_{\nu +1})/h$, 
остальные базисные инструменты с~меньшими индексами состоят только из 
путов, с~б$\acute{\mbox{о}}$льшими~--- только из коллов.
  \end{enumerate}
  
  Крайние спрэды в~базисе также называем (\textit{усеченными}) 
\textit{баттерфляями}.
  
  Платежные функции базисных инструментов, которым даются специальные 
обозначения $\beta_i(x)$, $i\hm\in I$, в~соответствии  
с~(\ref{e1-aga})--(\ref{e3-aga}) принимают вид:
  \begin{align*}
  \beta_1(x)&=\max \left( 0,\min\left( 1,\, 1+\fr{s_1-x}{h}\right)\right)\,;\\
  \beta_n(x)&=\max\left( 0, \min \left( 1,\ 1-\fr{s_n-x}{h}\right)\right)\,;\\
  \beta_i(x)&=\max \left( 0,\, 1-\fr{\left\vert s_i-x\right\vert}{h}\right)\,,\enskip 
i=2,\ldots , n-1\,.
  \end{align*}
  
  Легко видеть, что для всех опционных базисных инструментов, как и~для 
сценарных, интегралы от их платежных функций равны~$h$. Кроме того, 
  $$
  \sum\limits_{i\in I} \boldsymbol{D}_i=\sum\limits_{i\in I} \boldsymbol{B}_i 
=\boldsymbol{U}\,.
  $$
   
  Интегрированием платежных функций по мерам с~плотностями~$c(x)$ 
и~$p(x)$ находятся векторы~$\boldsymbol{c}^B$ и~$\boldsymbol{p}^B$ 
с~компонентами 
  \begin{multline}
  c_i^B=\vert \boldsymbol{B}_i\vert =\int\limits_{\sf X} \beta_i(x) c(x) \,dx\,;\\
  p_i^B=\left\| \boldsymbol{B}_i\right\|=\int\limits_{\sf X} \beta_i(x) p(x)\,dx\,,\ 
i\in I\,.
  \label{e7-aga}
  \end{multline}
  %
  Величина~$c_i^B$ означает стоимость $i$-го базисного баттерфляя, 
$p_i^B$~--- средний доход, который также называем \textit{суррогатной} 
вероятностью $i$-го сценария. %, $i\hm\in I$. 
  
  \section{Алгоритм оптимизации на~рынке опционов}
  
  Для опционного рынка нахождение весов оптимального портфеля реализует 
практически тот же дискретный алгоритм, основанный на процедуре  
Ней\-ма\-на--Пир\-со\-на~\cite{5-aga}, что и~для сценарного рынка, только 
вместо векторов~$\boldsymbol{p}^S$ и~$\boldsymbol{c}^S$ применяются 
векторы~$\boldsymbol{p}^B$ и~$\boldsymbol{c}^B$, хотя назначение 
вероятностей производится тем же вектором~$\boldsymbol{p}^S$. Подчеркнем, 
что сначала речь идет о портфеле с~детерминированными весами. Имеем 
(в~сокращенном виде): 
  \begin{description}
  \item $\boldsymbol{\rho} =\boldsymbol{p}^B/\boldsymbol{c}^B$~--- вектор 
(длины $n$) относительных доходов; 
  \item $\boldsymbol{\xi} = \mathbf{O}(\boldsymbol{\rho})$~--- вектор, 
задающий на множестве сценариев позиции компонент 
вектора~$\boldsymbol{\rho}$ в~порядке их возрастания; 
  \item $\boldsymbol{\eta} = \mathbf{O}(\boldsymbol{\xi})$~--- обратный 
к~$\boldsymbol{\xi}$ вектор, задает порядок возрастания 
компонент~$\boldsymbol{\rho}$;
  \item $\boldsymbol{d} = \boldsymbol{p}^S(\boldsymbol{\xi})$~--- суперпозиция 
векторов~$\boldsymbol{p}^S$ и~$\boldsymbol{\xi}$, ее компоненты 
упорядочены по возрастанию компонент вектора~$\boldsymbol{\rho}$ 
(подчеркнем использование вектора~$\boldsymbol{p}^S$);
  \item $\mathbf{T} = \left[t_{ij} = \{1, i\leq j; t_{ij} = 0,\ i>j\}\right]$~--- 
треугольная матрица, применяемая для последовательного суммирования 
компонент векторов, начиная с~первой;
  \item $\boldsymbol{\varepsilon} = \mathbf{T}\boldsymbol{d}$~--- вектор 
кумулятивных вероятностей для вектора~$\boldsymbol{d}$; в~дополнение 
к~вектору~$\boldsymbol{\varepsilon}$ принимается также $\varepsilon_0\hm = 0$. 
  \end{description}
  
  На основе вектора~$\boldsymbol{\varepsilon}$ вводятся и~последовательные 
полуинтервалы: 
  \begin{multline}
  {\sf E}_i=\left[ \underline{e}_i, \overline{e}_i\right)\,,\enskip
  \underline{e}_i=\varepsilon_{\eta_i-1}(=0,\,\eta_i=1)\,,\\
  \overline{e}_i = \varepsilon_{\eta_i}\,,\enskip \overline{e}_i-\underline{e}_i 
=p_i^S\,,\enskip i\in I\,.
    \label{e8-aga}
  \end{multline}
  
  При назначении вектора~$\boldsymbol{b}$ (и~$\boldsymbol{g}\hm= 
\boldsymbol{b}(\boldsymbol{\eta}))$ весов портфеля предлагается вариант 
модификации~$\boldsymbol{\varepsilon}_{\mathrm{opt}}$ 
вектора~$\boldsymbol{\varepsilon}$, для чего вводится новый 
вектор~$\boldsymbol{d}^\prime$: 
  \begin{equation}
  \left.
  \begin{array}{c}
  \boldsymbol{\varepsilon}_{\mathrm{opt}} = 
\underline{\boldsymbol{e}}\left(\boldsymbol{\xi}\right)+\boldsymbol{d}^\prime\,,\
  d_i^\prime= d_i-\fr{i-1}{n-1}\,,\ i\in I\,;\\[6pt]
   \boldsymbol{b}=\phi 
(\boldsymbol{\varepsilon}_{\mathrm{opt}});\enskip  \boldsymbol{g} = 
\boldsymbol{b}(\boldsymbol{\eta})\,.
\end{array}
\right\}
  \label{e9-aga}
  \end{equation}
  
  Этот вариант связан с~опционным характером платежной функции, 
распространяется на всю область ее значений $[0, 1)$, но применяется только 
для детерминированной схемы. В~большей мере это оправдано, когда минимум 
портфельного дохода достигается на внутренних сценариях. 
  
  Основными \textit{числовыми} показателями портфелей служат его 
стоимость~$A$ (средняя, если применяется рандомизация), средний 
доход~$R$, средний относительный доход~$R /A$, средняя доходность~$y$ 
и~стандартное отклонение~$\delta$ доходности. Имеем ($(\boldsymbol{u}, 
\boldsymbol{v})$~--- скалярное произведение векторов~$\boldsymbol{u}$ 
и~$\boldsymbol{v}$) запись результатов:
  \begin{align*}
  A^B &= \left(\boldsymbol{g},\boldsymbol{c}^{B}\right) = 
\left(\boldsymbol{b},\boldsymbol{c}^B(\boldsymbol{\xi})\right);\\
  \boldsymbol{R}^B &= \left(\boldsymbol{g},\boldsymbol{p}^B\right) = 
  \left(\boldsymbol{b}, \boldsymbol{p}^B\left(\boldsymbol{\xi}\right)\right); \\
   y &=  \fr{R}{A}- 1\,.     
  \end{align*}
  
  В иллюстративных расчетах и~при построении графиков на протяжении 
большей части изложения принимается, что плотности~$p(\cdot)$ и~$c(\cdot)$ 
подчиняются бе\-та-рас\-пре\-де\-ле\-нию с~параметрами $(2, 1{,}5)$ и~$(3, 
2{,}5)$ соответственно:
  \begin{multline}
  p(x)=x^2\fr{(1-x)^{1{,}5}}{B(3,2{,}5)};\quad
  c(x)=x\fr{(1-x)^{0{,}5}}{B(2,1{,}5)}\,,\\
  x\in {\sf X}=[0,\,1)\,,
  \label{e10-aga}
  \end{multline}
где $B(\cdot\,,\,\cdot)$~--- бе\-та-функ\-ция, $B(3, 2{,}5) \hm= 19{,}6875$, $B(2, 
1{,}5) \hm= 3{,}75$; $x_0\hm = 0$; $x_n = 1$. Эту задачу считаем\linebreak\vspace*{-12pt}

\pagebreak

\noindent
 прямой, но 
при проведении некоторого сравнительного анализа приводятся и~результаты 
решения обратной задачи, в~которой плотности меняются ролями. 

  В качестве ф.р.п.\ инвестора принимается функция $\phi(\varepsilon) \hm= 
\varepsilon^2$, $\varepsilon\hm\in [0, 1]$, и~потому функция 
распределения доходов оптимального теоретического портфеля~\cite{2-aga, 4-aga}
$$
{\sf   F}_0(z) = \phi^{\leftarrow}(z)= z^{1/2},\enskip  z\in [0, 1]\,. 
$$
  
  Приближенные значения показателей теоретической континуальной модели, 
с~которыми будем соотносить получаемые далее результаты, 
получены на дискретной модели с~2000~сценариев~\cite{4-aga}: 
 \begin{alignat*}{2}
   A_0 &= 0{,}265439\,;&\enskip   R_0 &= 0{,}333729\,;\\
     y_0 &= 0{,}257275\,;&\enskip  \sigma_0 &= 
1{,}1242\,. 
\end{alignat*}
  
  Напомним~\cite{2-aga}, что для показателя~$R_0$ в~случае 
$\phi(\varepsilon)\hm= \varepsilon^\lambda$ точное теоретическое значение 
равно ($1\hm+\lambda)^{-1}$, что при $\lambda\hm = 2$ дает $1/3 \hm\approx 
0{,}333333$. 
  
  На рис.~1 приведены графики важных для оптимизации функций 
относительных доходов $\rho(x)\hm = p(x)/c(x)$ в~прямой задаче 
с~плотностями~(\ref{e10-aga})~(\textit{1}) и~обратной~(\textit{2}), 
в~которой плотности меняются ролями. 


  
  Внешний вид этих графиков говорит о том, что инвестор сталкивается 
с~игрой на волатильности. По рыночной терминологии речь идет о ее продаже 
и~покупке соответственно, правда, к~ней примешивается и~некоторая слабо 
выраженная игра на понижение в~первом случае и~на повышение~--- во втором. 
  
  В примерах используется дискретизация, как и~в~\cite{4-aga}, с~$n\hm=5$ 
сценариями как в~сценарной, так
и~в~опционной версиях. (Намеренно 
выбирается столь малое число сценариев, чтобы высветить\linebreak\vspace*{-12pt}

  { \begin{center}  %fig1
 \vspace*{18pt}
  \mbox{%
 \epsfxsize=79mm 
 \epsfbox{aga-1.eps}
 }


\end{center}


\noindent
{{\figurename~1}\ \ \small{Графики функции~$\rho$ в~прямой~(\textit{1}) и~обратной~(\textit{2}) задачах}}
}

%\vspace*{9pt}

\addtocounter{figure}{1}

\noindent
  проблемы, 
связанные именно с~дискретизацией, поскольку при возрастании~$n$ решение 
все более походит на теоретическое для континуальной схемы.)




  Интегрированием~(\ref{e4-aga}) находятся \textit{сценарные} векторы: 
  \begin{align*}
  \boldsymbol{p}^S &= \{0{,}0412; 0{,}206; 0{,}341; 0{,}309; 0{,}104\};\\
  \boldsymbol{c}^S &= \{0{,}070; 0{,}187; 0{,}263; 0{,}284; 0{,}197\}.
  \end{align*}
  %
    Применение алгоритма с~$\boldsymbol{p}^S$ и~$\boldsymbol{c}^S$ дает 
векторы \textit{сценарного} рынка: 
  \begin{gather*}
  \boldsymbol{\xi}  = \{5, 1, 4, 2, 3\};\quad \boldsymbol{\eta} = \{2, 4, 5, 3, 1\};\\
  \boldsymbol{\varepsilon} = \{0,{1}04; 0{,}145; 0{,}454; 0{,}659; 1\}.
  \end{gather*}
   
  Рандомизация проводится введением случайных весов  
портфеля~\cite{2-aga, 4-aga}:
  \begin{equation}
  \boldsymbol{G}=\sum\limits_{i\in I} \omega_i \boldsymbol{
    D}_i\,, \enskip \omega_i=\phi\left( \theta_i\right)\,,
   \label{e11-aga}
   \end{equation}
где $\theta_i\sim {\sf R}\{ {\sf E}_i\}$, $i\hm\in I$,~--- равномерно 
распределенные случайные величины на последовательных полуинтервалах 
${\sf E}_i$~(\ref{e8-aga}), для которых 
\begin{align*}
  \underline{\boldsymbol{e}} &= \{ 0{,}104; 0{,}454; 0{,}659; 0{,}145; 0\}; \\
  \overline{\boldsymbol{e}} &= \{0{,}145; 0{,}659; 1; 0{,}454; 0{,}104\}.
  \end{align*}
  
  Функция распределения ${\sf F}_\zeta(\cdot)$ доходов~$\zeta$ 
портфеля~(\ref{e11-aga}) определяется правилом:
  \begin{equation}
  \left.
  \begin{array}{rl}
{\sf F}_{\omega;i}(z) &= \fr{\phi^\leftarrow (z)-\underline{e}_i}{p_i^S}\,,
  \enskip
  \phi\left( \underline{e}_i\right) \leq z\leq \phi\left( \overline{e}_i\right)\,,\\
  &\hspace*{45mm}
  i\in I\,;\\[6pt]
{\sf F}_\zeta(z)&=\phi^\leftarrow (z)\,,\enskip z\in [0,1)\,.
  \end{array}
  \right\}
  \label{e12-aga}
  \end{equation}
  
  При рандомизации вектор~$\boldsymbol{\varepsilon}_{\mathrm{opt}}$ не существен, 
а~из-за случайности величины~$A$ для анализа относительного дохода 
потребуется специальный подход. 
  
  При применении алгоритма к~\textit{опционным} рынкам используются 
векторы, получаемые по формулам~(\ref{e7-aga}) и,~по 
необходимости,~(\ref{e6-aga}): 
  \begin{align*}
  \boldsymbol{p}^B &= \{0{,}049; 0{,}204; 0{,}333; 0{,}301; 0{,}113\};\\
  \boldsymbol{c}^B &= \{0{,}075; 0{,}185; 0{,}261; 0{,}280; 0{,}200\};\\
  \boldsymbol{p}^K &= \{0{,}006; 0{,}115; 0{,}288; 0{,}350; 0{,}220; 0{,}021\}. 
  \end{align*}
  
  В результате применения алгоритма (с~$\boldsymbol{p}^B$, 
$\boldsymbol{c}^B$ и~частично~$\boldsymbol{p}^S$) получаем:
  $$
  \boldsymbol{\xi} = \{1, 5, 2, 4, 3\};\quad 
  \boldsymbol{\eta} = \{1, 3, 5, 4, 2\}\,. 
  $$
  
  В~рассматриваемом примере эта пара векторов $(\boldsymbol{\xi}, 
\boldsymbol{\eta})$ отличается от сценарного рынка, что вполне\linebreak\vspace*{-12pt}

 { \begin{center}  %fig2
 \vspace*{-3pt}
 \mbox{%
 \epsfxsize=79mm 
 \epsfbox{aga-2.eps}
 }


\vspace*{6pt}


\noindent
{{\figurename~2}\ \ \small{Графики платежных функций  $\boldsymbol{G}^B$~(\textit{1}) 
и~$\boldsymbol{G}^S$~(\textit{2})}}
\end{center}
}

\vspace*{9pt}

\addtocounter{figure}{1}

\noindent
 возможно при 
малом числе сценариев. И~показатели~$\boldsymbol{b}$, $\boldsymbol{g}$ 
и~$A$ для опционного рынка, тем более в~связи с~(\ref{e9-aga}), должны быть 
несколько иными по сравнению со сценарным. Для детерминированной 
схемы имеем: 
  \begin{align}
  {\boldsymbol{\varepsilon}}_{\mathrm{opt}}& = \{0; 0{,}114; 0{,}299; 0{,}608; 1 \}; \notag\\
 {\boldsymbol{b}} & = \phi\left({\boldsymbol{\varepsilon}}_{\mathrm{opt}}\right) = \{0; 0{,}013; 
0{,}090; 0{,}370; 1\}; \notag\\
  {\boldsymbol{g}} & = \boldsymbol{b}(\boldsymbol{\eta}) = \{ 0{,}013; 0{,}370; 1; 
0{,}090; 0 \};\label{e13-aga}
\end{align}
\begin{equation}
\left.
\begin{array}{rlrl}
  A &= 0{,}358244;&\enskip R &= 0{,}444829;\\[6pt]
  y &= 0{,}241692;&\enskip  \sigma &= 1{,}162.
  \end{array}
  \right\}
    \label{e14-aga}
  \end{equation}
  
  На рис.~2 (для прямой задачи) приводятся графики платежных функций 
$\pi(x; \boldsymbol{G}^B)$~(\textit{1}) и~$\pi(x; \boldsymbol{G}^S)$~(\textit{2}), 
$x\hm\in [0, 1]$. Первая из функций образована $(n\hm+1)$-м 
линейным участком $\psi_i(x)$, $i\hm\in I_0$, по правилу 
 \begin{align*}
  \psi_0(x)&=g_1\,;\\
  \psi_i(x)&=\fr{g_i(\kappa_{i+1}-x)+g_{i+1}(x-\kappa_i)}{h}\,;\\ 
  \psi_n(x)&=g_n\,;
\end{align*}



  \section{Детерминированный портфель}
  
  Оптимальный детерминированный портфель задается вектором 
весов~(\ref{e13-aga}) с~нормированными баттерфляями в~качестве базисных 
инструментов: 
  $$
  \boldsymbol{G}=\sum\limits_{i\in I} g_i \boldsymbol{B}_i\,.
  $$
  
  Функцию распределения дохода $\zeta\hm= \psi(\mathrm{X})$ определяем 
в~системе интервалов~$\mathbf{K}$~(\ref{e5-aga}) для $z\hm\in [0, 1]$: 
  \begin{equation}
  {\sf F}_\zeta(z)=\Phi_0(z) +\Phi_n(z)+\sum\limits_{i=1}^{n-1}\Phi_i(z)\,,
    \label{e15-aga}
    \end{equation}
    где
$$
  \Phi_i(z)={\sf P}\left\{ \zeta<z,\ \mathrm{X}\in \mathrm{K}_i\right\}\,,\enskip 
i\in I_0\,.
  $$

Для $i = 1$ и~$n$ доходы постоянны и~равны~$g_1$ и~$g_n$ соответственно, 
и~потому (здесь и~далее ${\sf u}(w)$~---  \textit{характеристическая функция} 
множества $w\hm\geq  0$)
$$
\Phi_0(z)=p_0^K{\sf u}\left(z-g_1\right)\,;\enskip \Phi_n(z)=p_n^K{\sf u}\left( z-
g_n\right)\,.
$$
  
  Но для $ i\hm\in I^\prime$ уже сказывается опционный характер платежной 
функции, и~потому доходы и~частные функции распределения соответственно 
равны:
  \begin{align*}
  \psi_i(x)&=\fr{g_i(\kappa_{i+1}-x)+g_{i+1}(x-\kappa_i)}{h}\,,\enskip x\in 
\mathrm{K}_i\,;\\
  \Phi_i(z)&=\int\limits_{\kappa_i}^{\kappa_{i+1}} {\sf u}\left(z-\psi_i(x)\right) 
p(x)\,dx\,.
  \end{align*}
  
  Все слагаемые суммы ${\sf F}_\zeta(z)$ в~(\ref{e15-aga}) находятся прос\-тым 
интегрированием. Из функции распределения можно определять и~запись 
результатов, но в~детерминированном случае этого и~не требуется, так как она 
уже получена выше~(\ref{e14-aga}). 
  
  График ${\sf F}_\zeta$ представлен (под обозначением~${\sf F}_d$) на 
  рис.~3~(\textit{1}), на нем приводится для сравнения и~теоретический 
график~${\sf F}_0$~(\textit{2}). 


  Функция распределения \textit{относительного} дохода $\chi\hm= \zeta/A$ 
находится при этом простым увеличением аргумента в~$1/A$~раз:
  $$
  {\sf F}_\chi(z)={\sf F}_\zeta(Az)\,,\enskip z\in [0, 1/A]\,.
  $$
  
  Далее рассматриваются варианты рандомизации. В~силу специфики 
опционного рынка наряду с~полной оправдан и~вариант частичной 
рандомизации.
  
  \section{Полная рандомизация}
  
  При полной рандомизации все веса портфеля базисных 
баттерфляев~$\omega_i$, $i\hm\in I$, являются случайными величинами. 
В~основе определения функции распределения дохода вновь лежит 
пред\-став\-ле\-ние~(\ref{e15-aga}), но теперь вместо детерминированных 
весов~$g_i$ вводятся случайные величины $\omega_i\hm= \phi(\theta_i)$, 
$i\hm\in I$, и~тогда доход $\zeta\hm= \psi(x; \boldsymbol{t})$, где 
$\boldsymbol{t}\hm= \{ t_1, \ldots , t_n\}$~--- вектор значений случайного 
вектора~$\boldsymbol{\theta}$. Итак, 
  \begin{equation}
  {\sf F}_\zeta(z)=\sum\limits_{i\in I_0} \Phi_i(z)\,,
    \label{e16-aga}
    \end{equation}
    где
    $$
  \Phi_i(z)={\sf P}\left\{ \zeta\leq z,\enskip x\in \mathrm{K}_i\right\}\,,\enskip i\in I_0\,.
$$
  %
  Для $i = 0$ и~$n$ из~(\ref{e12-aga}) следует, что соответственно 
  \begin{alignat}{2}
  \Phi_0(z) &= \fr{p_0^K (\phi^\leftarrow (z)-\underline{e}_1)}{p_1^S}\,,&\  
z&\in \left[ \phi\left( \underline{e}_1\right), \phi\left( 
\overline{e}_1\right)\right)\,;\label{e17-aga}\\
  \Phi_n(z) &= \fr{p_n^K (\phi^\leftarrow (z)-\underline{e}_n)}{p_n^S}\,,&\ 
z&\in \left[ \phi\left( \underline{e}_n\right), \phi\left( 
\overline{e}_n\right)\right)\,.\!\!
\label{e17-aga-1}
\end{alignat}
 % 
  В каждом из остальных слагаемых в~(\ref{e16-aga}), т.\,е.\ для 
$\mathrm{K}_i$, $i\hm\in I^\prime$, следует проводить интегрирование по 
$x\hm\in{\sf X}$ и~по всем $n$-мер\-ным векторам~$\boldsymbol{t}$ (хотя, по 
сути, лишь по всем парам $\{t_i, t_{i+1}\}$ значений для $\{\theta_i, \theta_{i+1}\}$, 
$t_i\hm\in \mathrm{E}_i$, $t_{i+1}\hm\in \mathrm{E}_{i+1}$). Имеем: 
  \begin{multline}
  \psi_i\left(x; t_i,t_{i+1}\right)= {}\\
  {}=\fr{\phi(t_i)(\kappa_{i+1}-x) +\phi(t_{i+1}) (x-
\kappa_i)}{h}\,,\\
x\in \mathrm{K}_i\,,\enskip 
i\in I^\prime\,;
\end{multline}

\vspace*{-12pt}

\noindent
\begin{multline}
  \Phi_i(z)={}\\
  {}=\!\! \iint\limits_{\mathrm{E}_i\times \mathrm{E}_{i+1}}\!\!\! \left(
\int\limits_{\,\kappa_i}^{\kappa_{i+1}}\!\! {\sf u}\left( z-
\psi_i\left(x;t_i,t_{i+1}\right)\right) p(x)\,dx \!\right)\!\! \fr{dt_i dt_{i+1}}{p_i^S 
p_{i+1}^S},\\
i\in I^\prime\,.
  \label{e18-aga}
  \end{multline}
  Здесь численное интегрирование реализуемо для любого значения $z\hm\in 
[0, 1]$. С~его помощью и~строится график функции~(\ref{e16-aga}) (например, 
интерполяцией). Он представлен на рис.~3 под обозначением~${\sf F}_f$~(\textit{3}). 
  
  Использование стандартных программ интегрирования не работает в~более 
сложных случаях, например для \textit{относительного дохода}, когда 
требуется шестикратное интегрирование. Тем не менее для демонстрации 
общности подхода протестируем на оценивании функции  
${\sf F}_\zeta(z)$~(\ref{e16-aga}) также конт\-ро\-ли\-ру\-емый и~учитывающий 
специфику задачи детерминированный аналог метода Мон\-те-Кар\-ло, уже 
применявшийся в~\cite{4-aga} для сценарного рынка.

 { \begin{center}  %fig3
 \vspace*{9pt}
\mbox{%
 \epsfxsize=78.856mm 
 \epsfbox{aga-3.eps}
 }


\vspace*{6pt}


\noindent
{{\figurename~3}\ \ \small{Графики функций  ${\sf F}_d$~(\textit{1}),
 ${\sf F}_0$~(\textit{2}),
${\sf F}_f$~(\textit{3}) и~${\sf F}_p$~(\textit{4})}}
\end{center}
}

%\vspace*{9pt}

\addtocounter{figure}{1}


  
  С учетом рандомизации вводятся решетки значений случайных 
величин~$\theta_i$, $i\hm\in I$, определяемые параметром~$n_J$ и~множеством 
$J\hm = \{1, 2, \ldots , n_J\}$: 
  \begin{equation}
  t_{ij} =\underline{e}_i+\fr{p_i^S\left( j-1/2\right)}{n_J}\,,\enskip j\in J\,,\ i\in I\,.
  \label{e19-aga}
  \end{equation}
  
  Для построения решетки реализуемых \textit{доходов} для 
системы~$\mathbf{K}$ интервалов~(\ref{e5-aga}) (за исключением интервалов 
$\mathrm{K}_0$ и~$\mathrm{K}_n$) образуются дополнительные решетки 
значений цены базового актива~$\pi_{il}$, $l\hm\in L$, $i\hm\in I^\prime$, 
определяемые параметром~$n_L$ и~множеством $L\hm = \{0, 1, \ldots , n_L\}$, 
и~вычисляются соответ\-ст\-ву\-ющие вероятности: 
  \begin{gather*}
  r_{0,j_1} =\phi\left( t_{1,j_1}\right);\enskip 
  r_{n,j_n}=\phi\left( t_{n,j_n}\right)\,,\enskip j_1, j_n\in 
J\,;\\
 \hspace*{-10mm}r_{i,j_i, j_{i+1},l}=\phi\left( t_{i,j_i}\right) \left( 1-\fr{\left( l-
1/2\right)}{n_L}\right)+{}\\
{}+\phi\left( t_{i+1, j_{i+1}}\right) \fr{l-1/2}{n_L}\,,\enskip 
l\in L\,,\ j_i\in J\,,\ i\in I^{\prime}\,;\\
  \pi_{i,l} = \kappa_i+\fr{l-1/2}{nn_L}\,;\\
  p^X_{i,l}  =\!\!\int\limits_{\pi_{i,l}-1/(2nn_S)}^{\pi_{i,l}+1/(2nn_S)}
   \hspace*{-6mm}p(x)\,dx\,;\enskip
  p_0^K +
  \sum\limits_{\substack{{i\in I^\prime,}\\ {l\in L}}} \!\! p^X_{i,l} +p_n^K=1\,.
  \end{gather*}
  

  
  В результате получается приближение для функции распределения в~виде: 
  \begin{multline}
  {\sf F}_\zeta(z)={}\\
  {}=\fr{p_0^K}{n_J} \sum\limits_{j_1\in J} {\sf u}\left( z-
r_{0,j_1}\right) +\fr{p_n^K}{n_J}\sum\limits_{j_n\in J}  {\sf u}\left( z-
r_{n,j_n}\right)+{}\\
  {}+\sum\limits_{\substack{{i\in I^\prime,}\\ 
  {j_1, j_{i+1}\in J,}\\
  {l\in L}}} \fr{p^X_{i,l}}{n_J^2}{\sf 
u}\left( z-r_{i,j_i, j_{i+1},l}\right)\,.
  \label{e20-aga}
  \end{multline}
  
  Построенный по~(\ref{e20-aga}) график визуально не отличим от 
графика~${\sf F}_f$ на рис.~3, что косвенно подтверждает правильность обоих 
подходов. 
  
  На основе~(\ref{e20-aga}) строятся и~представления для моментов: 
  \begin{align*}
  \mu_m&= \fr{p_0^K}{n_J}\sum\limits_{j_1\in J} r^m_{0,j_1} +\fr{p_n^K}{n_J} 
\sum\limits_{j_n\in J} r^m_{n,j_n}+{}\\
&\hspace*{10mm}{}+\sum\limits_{\substack{{i\in I^\prime,}\\
{j_i, j_{i+1}\in J,}\\
{l\in  L}}} \fr{p^X_{i,l}}{n_J^2}\,r_{i,j_i, j_{i+1},l}\,,\ m=1,2;\\
  \sigma^2&= \mu_2-\mu_1^2\,.
  \end{align*}
  %
    %\substack{{i=\overline{1,n}}\\ {j=\overline{1,l}}}
  %
  По ним определяются числовые показатели портфеля:
  $$
  {\sf E}\zeta = 0{,}326928\,;\ {\sf E}\zeta^2 = 0{,}161734\,;\ \sigma = 
0{,}234204\,,
  $$
а средняя сумма ${\sf E}A$ находится далее при оценивании 
\textit{относительного} дохода.

  Требуемое для расчетов время $\sim  n n_J^2n_L$.
  
  Подобный подход без труда распространяется и~на распределение 
\textit{относительного дохода}~$\chi$. Формула~(\ref{e16-aga}) 
переписывается применительно к~случайной величине 
$$
\chi = \psi(x; \theta_1, 
\ldots , \theta_n) = \psi(x; \boldsymbol{\theta}) = 
\fr{\zeta}{A(\boldsymbol{\theta})},
$$
 где $A(\boldsymbol{\theta}) \hm= 
\sum\nolimits_{\iota\in I} c_{\iota}^B\phi(\theta_{\iota})$~--- случайная инвестиционная 
сумма со значениями $A(\boldsymbol{t}) \hm= \sum\nolimits_{\iota \in I} 
c_{\iota}^B \phi(t_{\iota})$:
  \begin{equation}
  {\sf F}_\chi(z)=\sum\limits_{i\in I_0} \Phi_i(z)\,,
    \label{e21-aga}
    \end{equation}
    где
    $$
   \Phi_i(z)={\sf P}\left\{ \chi\leq 
z\,,\ x\in \mathrm{K}_i\right\}\,,\
 i\in I_0\,,
$$
а формулы~(\ref{e17-aga})--(20) приобретают вид: 
\begin{align*}
\psi_0(x; \boldsymbol{t}) &=\fr{\phi(t_1)}{A(\boldsymbol{t})}\,,\enskip  x\in 
\mathrm{K}_0\,;\\
\psi_n(x; \boldsymbol{t}) &=\fr{\phi(t_n)}{A(\boldsymbol{t})}\,,\enskip 
x\in \mathrm{K}_n\,,\\
\psi_i(x; \boldsymbol{t})&=h \fr{\phi(t_i) (\kappa_i-x) +\phi(t_{i+1}) (x-
\kappa_{i+1})} {A(\boldsymbol{t})}\,,\\
& \hspace*{40mm}x\in \mathrm{K}_i\,,\ i\in I^\prime\,;\\
  \Phi_i(z)&=\!\!\!\!\!\int\limits_{\prod\nolimits_{i=1,\ldots , n} \mathrm{E}_i} \! \left( 
\int\limits_{\kappa_i}^{\kappa_{i+1}} \fr{{\sf u}\left( z-\psi_i(x; 
\boldsymbol{t})\right) p(x)}{A(\boldsymbol{t})}\,dx \right) \times{}\\
&\hspace*{27mm}{}\times
\fr{\prod\nolimits_{i=1,\ldots , n} dt_i}{\prod\nolimits_{i=1,\ldots , n} p_i^S}\,,\enskip i\in 
I_0\,.
  \end{align*}
  
  Решетка для инвестиционной суммы $A(\boldsymbol{t})$ в~соответствии 
с~(\ref{e19-aga}) имеет вид:
  $$
  a_{j_1,\ldots , j_n}=\sum\limits_{\iota\in I}
   c_\iota^B \phi\left(t_{\iota,j_\iota}\right)\,,\
  j_i\in J\,,\ i\in I\,.
  $$
  
  Функция распределения относительного дохода, а также его моменты 
и~дисперсия определяются приближенными формулами, фактически 
реализующими шестикратное (при $n \hm= 5$) интегрирование:

\noindent 
  \begin{multline}
  {\sf F}_\chi(z)=\fr{p_0^K}{n_J^n} 
  \sum\limits_{j_1,\ldots, j_n\in J}\hspace*{-3mm} {\sf u}\left( 
z-\fr{r_{0,j_1}}{a_{j_1,\ldots, j_n}}\right) +{}\\
{}+\fr{p_n^K}{n_J^n} 
\sum\limits_{j_1,\ldots ,j_n\in J} \hspace*{-3mm}{\sf u}\left( z-\fr{r_{n,j_n}}{a_{j_1,\ldots, 
j_n}}\right)+{}\\
  {}+\sum\limits_{\substack{{i\in I^\prime,}\\
  {j_1,\ldots , j_n\in J,}\\
  {l\in L}}}
  \fr{p^X_{i,l}}{n_J^n} 
{\sf u} \left( z-\fr{r_{i,j_i,j_{i+1},l}}{a_{j_1,\ldots ,j_n}}\right)\,;\label{e22-aga}
  \end{multline}
  
  \vspace*{-12pt}
  
  \noindent
  \begin{multline}
  \mu_m=\fr{p_0^K}{n_J^n} \sum\limits_{j_1,\ldots ,j_n\in J} 
\fr{r^m_{0,j_1}}{a^m_{j_1,\ldots , j_n}}+{}\\
{}+
  \fr{p_n^K}{n_J^n} \sum\limits_{j_1,\ldots , j_n\in J} 
\fr{r^m_{n,j_n}}{a^m_{j_1,\ldots , j_n}}+  \sum\limits_{\substack{
  {i\in I^\prime,}\\
  {j_1,\ldots , j_n\in J,}\\
  {l\in L}}} \fr{p^X_{i,l}}{n_J^n} 
\,
 \fr{r_{i,j_1, j_{i+1},l}}{a_{j_1,\ldots , j_n}}\,,\\
   m=1, 2;\enskip \sigma^2=\mu_2-\mu_1^2\,.
  \label{e23-aga}
  \end{multline}
  
  Выборочные моменты и~дисперсия определяются по~(\ref{e23-aga}):
  $$
  {\sf E}\chi = 1{,}21373; \ {\sf E}\chi^2 = 2{,}1483;\ \sigma = 0{,}821676
  $$
  (при $n_L = 8$, $n_J = 6$). 
  
Из подобных~(\ref{e23-aga}) формул также ${\sf E}A \hm= 0{,}268748$, ${\sf 
E}(1/A) \hm= 3{,}83142$. 

  Построенный по~(\ref{e22-aga}) график ${\sf F}_\chi(\cdot)$ походит на 
растянутый по оси абсцисс график~${\sf F}_\zeta(\cdot)$ с~коэффициентом, 
близким к~${\sf E}(1/A)$, и~в работе не приводится. 
  
  Требуемое для расчетов время $\sim  n n_J^nn_L$. 
  
  \section{Частичная рандомизация}
  
  Под частичной рандомизацией понимается вариант рандомизации, в~котором 
случайными становятся лишь веса~$\omega_1$ и~$\omega_n$ базисных 
баттерфляев в~портфеле для крайних страйков. Суть в~том, что уже сами 
платежные функции опционов, исключая эти страйки, реализуют некоторое 
сглаживание, но лишь для внутренних страйков. И~частичной рандомизации 
может оказаться достаточно. 
  
  При частичной рандомизации \textit{доход} $\zeta\hm=\psi(x; t_1, t_n)$. Вновь 
исходим из~(\ref{e16-aga}) и~для крайних интервалов~$\mathrm{K}_i$, $i\hm = 
0, n$, применяем формулы~(\ref{e17-aga}) и~(18): 
  \begin{align*}
  \Phi_0(z) &= p_0^K 
  \fr {\phi^{\leftarrow} (z) -\underline{e}_1}{p_1^S}\,,\ z\in \left[ \phi\left( 
\underline{e}_1\right), \phi\left( \overline{e}_1\right)\right)\,;\\
  \Phi_n(z) &=p_n^K \fr{\phi^\leftarrow (z) -\underline{e}_n}{p_n^S}\,,\ z\in \left( 
\phi\left( \underline{e}_n\right) \phi\left( \overline{e}_n\right)\right)\,.
  \end{align*}
  
  Для $i\hm\in I^{\prime\prime}$ применяются формулы детерминированной 
схемы, а для $i\hm = 1, n\hm-1$ формулы приобретают гибридный вид, 
сочетающий в~себе характерные черты как рандомизированной, так 
и~детерминированной схем. Имеем: 
  \begin{align*}
  \psi_1\left(x;t_1,t_n\right)& =h\left( \psi(t_1)\left( \kappa_1-x\right) +g_2\left( x-
\kappa_2\right)\right)\,,\\
&\hspace*{40mm}x\in \mathrm{K}_1\,;\\
  \psi_{n-1}\left( x; t_1, t_n\right) &=h\left( g_{n-1}\left( \kappa_{n-1}-x\right) 
+{}\right.\\
&\left.{}+\phi(t_n)\left( x-\kappa_n\right)\right)\,,\ x\in \mathrm{K}_{n-1}\,,\\
  \psi_i\left( x; t_1, t_n\right) &=h\left( g_i\left( \kappa_{i+1} -x\right) +g_{i+1} \left( 
x-\kappa_i\right)\right)\,,\\
&\hspace*{30mm} x\in \mathrm{K}_i\,,\ i\in I^{\prime\prime}\,,\\
  \Phi_i(z) &={}\\
&\hspace*{-20mm}{}=\iint\limits_{\mathrm{E}_1\times\mathrm{E}_2} \!\left(  
\int\limits_{\kappa_i}^{\kappa_{i+1}} {\sf u} \left( z-\psi_i(x; t_1,t_n)\right) 
p(x)\,dx\right)\! \fr{dt_1 dt_n}{p_1^S p_n^S}\,,\\
&\hspace*{47mm} i\in I^\prime\,.
  \end{align*}
  %
  На этот раз решетка~(\ref{e19-aga}) применяется лишь для крайних 
сценариев: 
  \begin{equation}
  t_{i,j} =\underline{e}_i+p_i^S\fr{\left( j-1/2\right)}{n_J}\,,\ j\in J\,,\ i=1,n\,.
  \label{e24-aga}
  \end{equation}
По ней строится и~решетка доходов: 
\begin{gather*}
r_{0,j_1} =\phi\left( t_{1,j_1}\right)\,,\enskip r_{n,j_n} =\phi\left( 
t_{n,j_n}\right)\,,\enskip  j_1, j_n\in J\,;\\
r_{1,j_1,l} =\phi \left( t_{1,j_1}\right) \left( 1- \fr{l-1/2}{n_L}\right) +g_2 \fr{l-
1/2}{n_L}\,,\\
 \hspace*{46mm}j_1\in J\,,\ l\in L\,;\\
r_{n-1, j_n, l} =g_{n-1} \left( 1-\fr{l-1/2}{n_L}\right) +\phi\left( t_{n,j_n}\right) 
\fr{l-1/2}{n_L}\,,\\
\hspace*{50mm} j_n\in J\,,\ l\in L\,;\\
r_{i,l} =g_i\left( 1-\fr{l-1/2}{n_L}\right) +g_{i+1} \fr{l-1/2}{n_L}\,,\\
 \hspace*{50mm}l\in  L\,,\enskip i\in I^{\prime\prime}\,.
\end{gather*}

  Наконец, находится аппроксимация функции распределения дохода: 
  \begin{multline}
  \hspace*{-8pt}{\sf F}_{\zeta}(z) =\!\sum\limits_{j_1\in J} \fr{p_0^K}{n_J}\,{\sf u} \left( z-
r_{0,j_1}\right) +\!\sum\limits_{j_n\in J} \fr{p_n^K}{n_J}\, {\sf u} \left( z-
r_{n,j_n}\right)+{}\\
  {}+ \sum\limits_{\substack{
  {j_1\in J,}\\
  {l\in L}}}\! \!\fr{p^X_{1,l}}{n_J}\, {\sf u}
  \left( z-
r_{1,j_1,l}\right) +\sum\limits_{\substack{{j_n\in J,}\\ {l\in L} }}\!\!
\fr{p^X_{n-1,l}}{n_J}\, {\sf u}\left( z-
r_{n-1, j_n,l}\right) +{}\\
{}+\sum\limits_{\substack{{i\in I^{\prime\prime},}\\ {l\in L}}} 
\fr{p^X_{i,l}}{n_J^2}\, {\sf u}\left( z-r_{i,l}\right)\,.
  \label{e25-aga}
  \end{multline}
 % 
  На ее основе строится график функции распределения дохода. Выборочные 
моменты и~дисперсия определяются формулами:

\noindent
  \begin{multline*}
  \mu_m=\sum\limits_{j_1\in J} \fr{p_0^K}{n_J} \,r^m_{0,j_1} 
+\sum\limits_{j_n\in J} \fr{p_n^K}{n_J}\,r^m_{n,j_n} +{}\\
{}+\sum\limits_{\substack{{j_1\in J,}\\ 
{l\in  L}}} \fr{p^X_{1,l}}{n_J} \, r^m_{1, j_1, l} +
\sum\limits_{\substack{{j_n\in J,}\\ {l\in L}}} 
\fr{p^X_{n-1,l}}{n_J}\,r^m_{n-1, j_n, l}+{}\\
  {}+ \sum\limits_{\substack{{i\in I^{\prime\prime},}\\ {l\in L}}} 
\fr{p^X_{i,l}}{n_J^2}\,r^m_{i,l}\,,\ m=1,2;\\[-6pt] 
\sigma^2=\mu_2-\mu_1^2\,.
  \end{multline*}
%      
  Из них находим ${\sf E}\zeta\hm=  0{,}436903$, ${\sf E}\zeta^2\hm = 
0{,}292945$ и~$\sigma \hm= 0{,}319469$ (при $n_L \hm= 100$ и~$n_J\hm= 50$). 
Требуемое для расчетов время $\sim  n n_J^2n_L$.
  
  График функции~(\ref{e25-aga}) представлен на рис.~3 под 
обозначением~${\sf F}_p$~(\textit{4}). 
  
  
  Для \textit{относительного дохода} имеем 
    $$
  \chi = \psi(x; \theta_1, \theta_n) 
= \fr{\zeta}{A(\theta_1,\theta_n)}\,,
$$
 где
  $A(\theta_1,\theta_n)\hm =c_1^B\phi (\theta_1)\hm+\sum\nolimits_{\iota=2,\ldots , n-1} c_\iota^B 
g_\iota\hm +c_n^B \phi(\theta_n)$~--- случайная инвестиционная сумма со значениями 
$\mathrm{A}(t_1, t_n)$. Вновь исходим из пред\-став\-ле\-ния~(\ref{e21-aga}), при этом для компонент с~$i 
\hm= 0$ и~$n$ формулы те же, что и~для полной рандомизации,  а~для $i\hm\in I^\prime$ требуются 
очевидные изменения:

\noindent
   \begin{align*}
   \psi_0(x;\boldsymbol{t}) &= \fr{\phi (t_1)}{A(\boldsymbol{t})}\,,\enskip 
   x\in \mathrm{K}_0\,;\\ 
   \psi_n(x; \boldsymbol{t})&=\fr{\psi(t_n)}{A(\boldsymbol{t})}\,,\enskip x\in \mathrm{K}_n\,;\\
      \psi_1\left(x; t_1, t_n\right) &=
      \fr{h(\phi(t_1) (\kappa_1-x)+g_2(x-\kappa_2))}{A(t_1,t_n)}\,,\\ 
      &\hspace*{40mm}x\in  \mathrm{K}_1\,;\\
   \psi_{n-1}\left( x; t_1, t_n\right) &={}\\
   &\hspace*{-22mm}{}=\fr{h(g_{n-1}(\kappa_{n-1}-x) +\phi(t_n) (x-
\kappa_n))}{A(t_1,t_n)}\,,\enskip x\in \mathrm{K}_{n-1}\,;\\
   \psi_i\left( x; t_1, t_n\right)& =\fr{h(g_i(\kappa_{i+1}-x) +g_{i+1}(x-\kappa_i))}{A(t_1,t_n)}\,,\\
&\hspace*{30mm}x\in \mathrm{K}_i,\ i\in I^{\prime\prime}\,;\\
\Phi_i(z)&={}\\
&  \hspace*{-22mm}{}=\!\!\iint\limits_{\mathrm{E}_1\times\mathrm{E}_n} \left( \int\limits_{\kappa_i}^{\kappa_{i+1}} 
{\sf u} \left( z-\phi_i(x; t_1, t_n)\right) p(x)\,dx\right)\!
\fr{dt_1dt_n}{p_1^S p_n^S}\,,\\
& \hspace*{45mm}i\in I_0\,.
   \end{align*}
   
   \vspace*{-2pt}
  
  На этот раз также образуется двумерная решетка~(\ref{e24-aga}), и~на ее 
основе формируется решетка стоимости портфеля:

\noindent
\begin{multline*}
  a_{j_1,j_n}=c_1^B\phi\left( t_{1,j_1}\right) +\sum\limits_{\iota=2,\ldots , n-1} \!\!
c_\iota^B g_\iota +c_n^B \phi \left( t_{n,j_n}\right)\,,\\ j_1, j_n\in J\,.
\end{multline*}
  
  Аппроксимацию функции распределения для~$\chi$ дает представление:
  
  \noindent
  \begin{multline*}
  {\sf F}_\chi (z) =\fr{p_0^K}{n_J^2}\sum\limits_{j_1, j_n\in J} {\sf u}\left( z-
\fr{r_{0,j_1}}{a_{j_1,j_n}}\right) +{}\\
{}+\fr{p_n^K}{n_J^2} \sum\limits_{j_1, j_n\in J} 
{\sf u} \left( z- \fr{r_{n,j_n}}{a_{j_1,j_n}}\right) +{}\\
  {}+ \sum\limits_{\substack{{j_1, j_n\in J,}\\
  { l\in L}}} \fr{p^X_{1,l}}{n_J^2}{\sf u} \left( z-
\fr{r_{1,j_1,l}}{a_{j_1,j_n}}\right)+ {}\\
{}+
\sum\limits_{\substack{{j_1, j_n\in J,}\\
{ l\in L}}} \fr{p^X_{n-1, l}}{n_J^2} {\sf u} 
\left( z-\fr{r_{n-1, j_n,l}}{a_{j_1, j_n}}\right)+{}\\
  {}+ \sum\limits_{
  \substack{{i\in I^{\prime\prime},}\\
  {j_1, j_n\in J,}\\
  {l\in L}}} \fr{p^X_{i,l}}{n_J^2} 
{\sf u} \left( z-\fr{r_{i,l}}{a_{j_1,j_n}}\right)\,.
  \end{multline*}
  
  Выборочные моменты и~дисперсия определяются формулами:
  
  \noindent
  \begin{multline*}
  \mu_m=\fr{p_0^K}{n_J^2} \sum\limits_{j_1,j_n\in J} 
\fr{r^m_{0,j_1}}{a^m_{j_1,j_n}}+ \fr{p_n^K}{n_J^2} \sum\limits_{j_1,j_n\in J} 
\fr{r^m_{n,j_n}}{a^m_{j_1,j_n}}+{}\\
{}+\sum\limits_{\substack{{j_1,j_n\in J,}\\ {l\in L} }}
\fr{p^X_{1,l}}{n_J^2}\, \fr{r^m_{1,j_1,l}}{a^m_{j_1, j_n}}+
  \sum\limits_{\substack{{j_1,j_n\in J,}\\
  { l\in L}}} \fr{p^X_{n-1,l}}{n_J^2}\, 
  \fr{r^m_{n-1, j_n, l}}{a^m_{j_1,j_n}} + {}\\
  {}+
  \sum\limits_{\substack{{i\in I^{\prime\prime},}\\
  {j_1,j_n\in J,}\\
  {l\in L}}} 
\fr{p^X_{i,l}}{n_J^2} \,\fr{r^m_{i,l} }{a^m_{j_1, j_n}}\,,\enskip
  m=1,2\,;
    \end{multline*}
    
    \vspace*{-9pt}
    
    \noindent
    $$
  \sigma^2=\mu_2-\mu_1^2\,.
  $$
   %
  Из них находим ${\sf E}\chi = 1{,}22723$, ${\sf E}\chi^2 \hm= 2{,}31133$ 
и~$\sigma\hm = 0{,}897355$ (при $n_L \hm= 50$ и~$n_J \hm= 6$), а из аналогичных 
им формул~--- также ${\sf E}A\hm = 0{,}356003$ и~${\sf E}(1/A) \hm= 
2{,}80898$. Кстати, ${\sf E}A\times {\sf E}(1/A) \hm= 1{,}0000035$, что 
косвенно подтверждает близость графиков для детерминированной и~частично 
рандомизированной схем. 
  
  Требуемое для расчетов время $\sim n n_J^2 n_L$. 
  
  \vspace*{-4pt}
  
  \section{Заключение}
  
    \vspace*{-4pt}
  
  Завершаем описание алгоритмов нахождения функций распределения 
доходов и~относительного дохода в~задаче оптимизации по CC-VaR портфеля 
опционов. Результаты расчетов на примере с~бе\-та-рас\-пре\-де\-ле\-ни\-ями 
для прямой задачи (продажи
 волатильности) проиллюстрированы на графиках 
рис.~3. 
  

  Разумеется, при дискретизации задачи в~случае небольшого числа сценариев 
результаты должны %\linebreak\vspace*{-12pt}

{ \begin{center}  %fig4
 \vspace*{-4pt}
\mbox{%
 \epsfxsize=78.3mm 
 \epsfbox{aga-4.eps}
 }


\end{center}


\noindent
{{\figurename~4}\ \ \small{Функции  ${\sf F}_d$~(\textit{1}),  ${\sf F}_d$~(\textit{2}),
${\sf F}_f$~(\textit{3}) и~${\sf F}_p$~(\textit{4}) для обратной задачи}}
}

\vspace*{12pt}

\addtocounter{figure}{1}


\noindent
 отличаться от теоретической схемы. И~интерес 
представляет сравнение полученных результатов, в~частности с~обратной 
задачей~--- покупкой волатильности. На рис.~4 представлены для нее 
аналогичные результаты, полученные тем же алгоритмом. Значительное 
расхождение результатов в~большей степени объясняется различием задач, 
очевидным по рис.~1.

  
  В прямой задаче частичная рандомизация дает практически тот же результат, 
что и~детерминированная схема (см.\ рис.~3), что легко объясняется 
незначительным вкладом в~картину доходов крайними сценариями, на которые 
частичная рандомизация только и~распространяется. А~потому выбор 
предстоит делать между детерминированной схемой и~полной рандомизацией. 
При этом полная рандомизация дает более сглаженную и~более близкую 
к~теоретической схеме кривую, а детерминированная~--- более надежную 
в~выполнении CC-VaR. 
  
  В обратной задаче (см.\ рис.~4) все три аппроксимации дают результаты, 
достаточно близкие к~тео\-ре\-ти\-че\-ским, что можно объяснять полноценным 
вкладом в~итоговый результат всех пяти сценариев. И~частичная 
рандомизация, почти совпадающая с~детерминированной кривой в~области 
пониженных доходов и~с полной рандомизированной~--- в~зоне повышенных, 
участвует в~этом в~наибольшей степени и~даже представляется наилучшей из 
трех схем. 
  
  В целом, на рынках с~малым числом сценариев при рассмотрении вариантов 
назначения весов портфеля инвестору придется при выборе варианта 
дополнительно руководствоваться критериями близости функции 
распределения к~теоретической, степени выполнения CC-VaR, средней 
доходности и~пр. 
  
{\small\frenchspacing
 {%\baselineskip=10.8pt
 \addcontentsline{toc}{section}{References}
 \begin{thebibliography}{9}
  \bibitem{1-aga}
  \Au{Agasandian G.\,A.} Optimal behavior of an investor in option market~// 
  Conference (International) on Neural Networks. The 2002 IEEE World Congress on Computational 
Intelligence.~--- Honolulu, Hawaii, 2002. P.~1859--1864.
{\looseness=1

} 
  \bibitem{2-aga}
  \Au{Агасандян Г.\,А.} Применение континуального критерия VaR на финансовых  
рынках.~--- М.: ВЦ РАН, 2011. 299~с. 



  \bibitem{3-aga}
  \Au{Агасандян Г.\,А.} Континуальный критерий VaR на многомерных рынках  
опционов.~--- М.: ВЦ РАН, 2015. 297~с. 
  \bibitem{4-aga}
  \Au{Агасандян Г.\,А.} Континуальный критерий VaR на сценарных рынках~// 
Информатика и~её применения, 2018. Т.~12. Вып.~1. С.~32--40. 
  \bibitem{5-aga}
  \Au{Крамер Г.} Математические методы статистики~/ Пер. с~англ.~--- М.: Мир, 1975. 
750~с. (\Au{Cramer~H.} Mathematical methods of statistics.~--- Princeton, NJ, USA: Princeton 
University Press, 1946. 575~p.)
 \end{thebibliography}

 }
 }

\end{multicols}

\vspace*{-9pt}

\hfill{\small\textit{Поступила в~редакцию 18.12.18}}

\vspace*{6pt}

%\pagebreak

%\newpage

%\vspace*{-28pt}

\hrule

\vspace*{2pt}

\hrule

\vspace*{-2pt}

\def\tit{PERFORMANCE ESTIMATIONS FOR~OPTIMAL-ON-CC-VaR 
PORTFOLIOS IN~OPTION MARKETS}


\def\titkol{Performance estimations for optimal-on-CC-VaR 
portfolios in~option markets}

\def\aut{G.\,A.~Agasandyan}

\def\autkol{G.\,A.~Agasandyan}

\titel{\tit}{\aut}{\autkol}{\titkol}

\vspace*{-11pt}


\noindent
  A.\,A.~Dorodnicyn Computing Center, Federal Research Center ``Computer 
Science and Control'' of the Russian Academy of Sciences, 40~Vavilov Str., Moscow 
119333, Russian Federation


\def\leftfootline{\small{\textbf{\thepage}
\hfill INFORMATIKA I EE PRIMENENIYA~--- INFORMATICS AND
APPLICATIONS\ \ \ 2019\ \ \ volume~13\ \ \ issue\ 3}
}%
 \def\rightfootline{\small{INFORMATIKA I EE PRIMENENIYA~---
INFORMATICS AND APPLICATIONS\ \ \ 2019\ \ \ volume~13\ \ \ issue\ 3
\hfill \textbf{\thepage}}}

\vspace*{3pt} 
    
   
  \Abste{The paper continues investigations of the author about using continuous VaR-criterion 
(CC-VaR) in financial markets. The problem of projecting ideas and methods elaborated for 
investments in the ideal theoretical one-period market and its discrete scenario analog onto 
a~discrete-in-strikes option market is considered. The main focus
is on the methods of 
calculating distribution function of income and return relative, and also their mean for option 
portfolios optimal on CC-VaR and their randomized versions, both full and partial. A~discrete 
optimization algorithm as the result of projecting the theoretical algorithm based on the 
Newman--Pearson procedure onto scenario market is suggested. The optimal vector of weights derived from 
this algorithm is applied to the basis of normalized simplest butterflies. If randomizing portfolios are 
admissible, then special algorithms based on the ideas of the Monte-Carlo method that determine 
distribution functions of income and return relative are suggested. The exposition is illustrated by 
examples with  
beta-distributed underlier's prices and investor's probability forecast, which deal with the problems 
of volatility selling and buying. The respective diagrams are adduced.}
  
  \KWE{continuous VaR-criterion (CC-VaR); investor's risk-preferences function (r.p.f.);  
Newman--Pearson procedure; scenarios; options; indicators; butterflies; full and partial 
randomizing; optimal portfolio; income; yield}
  
  
  
\DOI{10.14357/19922264190311} 

\vspace*{-18pt}

\Ack
  \noindent
  The work was supported by the Russian Foundation for Basic Research (project 
17-01-00816). 


%\vspace*{-6pt}

  \begin{multicols}{2}

\renewcommand{\bibname}{\protect\rmfamily References}
%\renewcommand{\bibname}{\large\protect\rm References}

{\small\frenchspacing
 {%\baselineskip=10.8pt
 \addcontentsline{toc}{section}{References}
 \begin{thebibliography}{9}
  \bibitem{1-aga-1}
  \Aue{Agasandian, G.\,A.} 2002. Optimal behavior of an investor in option market. 
  \textit{Conference (International) on Neural Networks. The 2002 IEEE World Congress on Computational 
Intelligence}. Honolulu, Hawaii. 1859--1864. 
  \bibitem{2-aga-1}
  \Aue{Agasandyan, G.\,A.} 2011. \textit{Primenenie kontinual'nogo kriteriya VaR na 
finansovykh rynkakh} [Application of continuous VaR-criterion in financial markets]. Moscow: 
CC RAS. 299~p.
  \bibitem{3-aga-1}
  \Aue{Agasandyan, G.\,A.} 2011. \textit{Kontinual'nyy kriteriy VaR na mnogomernykh rynkakh 
optsionov} [Continuous VaR-criterion in multidimensional option markets]. Moscow: CC RAS. 297~p.
  \bibitem{4-aga-1}
  \Aue{Agasandyan, G.\,A.} 2018. Kontinual'nyy kriteriy VaR na stsenarnykh rynkakh 
[Continuous VaR-Criterion in scenario markets]. 
\textit{Informatika i~ee Primeneniya~--- Inform. Appl.} 
12(1):32--40. 
  \bibitem{5-aga-1}
  \Aue{Cramer, H.} 1946. \textit{Mathematical methods of statistics}. Princeton, NJ: Princeton 
University Press. 575~p. 
 \end{thebibliography}

 }
 }

\end{multicols}

\vspace*{-6pt}

\hfill{\small\textit{Received December 18, 2018}}

%\pagebreak

\vspace*{-22pt} 
  
  \Contrl
  
  \noindent
  \textbf{Agasandyan Gennady A.} (b.\ 1941)~--- Doctor of Science in physics and 
mathematics, leading scientist, A.\,A.~Dorodnicyn Computing Center, Federal 
Research Center ``Computer Science and Control'' of the Russian Academy of 
Sciences, 40~Vavilov Str., Moscow 119333, Russian Federation; 
\mbox{agasand17@yandex.ru}
\label{end\stat}

\renewcommand{\bibname}{\protect\rm Литература}    

   

   