\def\stat{zeifman}

\def\tit{ОБ ОЦЕНКАХ СКОРОСТИ СХОДИМОСТИ ДЛЯ НЕКОТОРЫХ МОДЕЛЕЙ МАССОВОГО ОБСЛУЖИВАНИЯ\\ 
С~НЕПОЛНО ЗАДАННЫМИ ИНТЕНСИВНОСТЯМИ$^*$}

\def\titkol{Об оценках скорости сходимости для %некоторых 
моделей массового обслуживания 
с~неполно заданными интенсивностями}

\def\aut{А.\,И.~Зейфман$^1$, Я.\,А.~Сатин$^2$, К.\,М.~Киселева$^3$}

\def\autkol{А.\,И.~Зейфман, Я.\,А.~Сатин, К.\,М.~Киселева}

\titel{\tit}{\aut}{\autkol}{\titkol}

\index{Зейфман А.\,И.}
\index{Сатин Я.\,А.}
\index{Киселева К.\,М.}
\index{Zeifman A.\,I.}
\index{Satin Y.\,A.}
\index{Kiseleva K.\,M.}


{\renewcommand{\thefootnote}{\fnsymbol{footnote}} \footnotetext[1]
{Исследование выполнено за счет гранта Российского научного
фонда (проект 19-11-00020).}}


\renewcommand{\thefootnote}{\arabic{footnote}}
\footnotetext[1]{Вологодский государственный университет; Институт проб\-лем информатики
Федерального исследовательского центра <<Информатика и~управление>> 
Российской академии наук;  Вологодский научный центр Российской
академии наук, \mbox{a\_zeifman@mail.ru}}
\footnotetext[2]{Вологодский
государственный университет, \mbox{yacovi@mail.ru}}
\footnotetext[3]{Вологодский государственный университет,
\mbox{ksushakiseleva@mail.ru}}

%\vspace*{-2pt}



\Abst{Рассматриваются некоторые модели
массового обслуживания с~неполно заданными интенсивностями. Авторы
изучают систему $M_t/M_t/S$ с~произвольным числом серверов~$S$ 
и~систему $M_t/M_t/S/S$ (модель Эрланга) с~интенсивностями,
удовлетворяющими соответствующим условиям. Для получения оценок
скорости сходимости используется понятие логарифмической нормы
операторной функции и~связанные с~ней оценки нормы матрицы Коши.}

\KW{система массового обслуживания; неполно
заданные интенсивности; скорость сходимости; эргодичность;
логарифмическая норма; $M_t/M_t/S$; $M_t/M_t/S/S$}

\DOI{10.14357/19922264190303} 
  
\vspace*{6pt}


\vskip 10pt plus 9pt minus 6pt

\thispagestyle{headings}

\begin{multicols}{2}

\label{st\stat}

\section{Введение}


В работе изучаются простые нестационарные системы обслуживания
$M_t/M_t/S/S$ и~$M_t/M_t/S$ в~ситуации, когда интенсивности
поступления и~обслуживания требований являются 1-пе\-рио\-ди\-че\-ски\-ми
функциями времени и~известны не сами эти функции, а их
<<средние>>$~$значения, т.\,е.\ числа 
$$
\la^*=\int\limits_0^1\la (t)\, dt\,;\quad
\mu^*\hm=\int\limits_0^1\mu (t)\, dt\,.
$$
 Такого типа модели (в~которых известны
не сами интенсивности поступления и~обслуживания требований, 
а~ка\-кие-ли\-бо их характеристики) достаточно часто встречаются 
в~современной литературе (см., например,~[1--6]).

Оценки скорости сходимости для рассматриваемых моделей 
в~стационарной и~нестационарной ситуациях изучались во многих работах,
(см., например,~[7--17] и~приведенную в~этих работах биб\-лио\-гра\-фию).

Сами эти модели достаточно стандартны, они характеризуются числом
серверов~$S$, а~также интенсивностями поступления и~обслуживания
требований~$\lambda(t)$ и~$\mu(t)$ соответственно.

Число требований~$X(t)$ описывается неоднородным процессом
рождения и~гибели (ПРГ):
\begin{itemize}
\item для системы $M_t/M_t/S/S$ с~конечным пространством состояний $0,1.
\dots, S$ и~интенсивностями $\lambda_k(t) \hm= \lambda(t)$ и~$\mu_k(t)=k\mu(t)$;

\item  для системы $M_t/M_t/S$ со счетным  пространством состояний $0,1.
\dots, S, \dots $ и~интенсивностями $\lambda_k(t) \hm= \lambda(t)$ 
и~$\mu_k(t)\hm=\mu(t)\min(k,S)$.
\end{itemize}

Функции $\lambda(t)$ и~$\mu(t)$ предполагаются неотрицательными,
1-пе\-рио\-ди\-че\-ски\-ми и~интегрируемыми на $[0,1]$.

Обозначим через 
\begin{multline*}
p_{ij}(s,t)=\mathrm{Pr}\left\{ X(t)=j\left| X(s)=i\right.
\right\},\\ i,j \ge 0\,\ 0\leq s\leq t\,,
\end{multline*}
 переходные вероятности
процесса $X\hm=X(t)$, а~его вероятности состояний~--- через  
$$
p_i(t)=\mathrm{Pr}\left\{ X(t)=i \right\}\,.
$$


Будем обозначать  через $\|\!\bullet\!\|$   $l_1$-нор\-му вектора 
и~мат\-ри\-цы, т.\,е.\  $\|{\bf x}\|\hm=\sum|x_i|$, 
а~$\|B\| \hm= \max\nolimits_j \sum\nolimits_i
|b_{ij}|$ при $B\hm = (b_{ij})_{i,j=0}^{S}$, а~через~$\Omega$~---
множество всех стохастических векторов, т.\,е.\ множество векторов 
с~неотрицательными координатами и~единичной $l_1$-нормой.


Через $E(t,k) = E\left\{X(t)\left\vert X(0)\hm=k\right.\right\}$ будем далее
обозначать математическое ожидание процесса (среднее число
требований) в~момент~$t$ при условии, что в~нулевой момент времени
он находится в~состоянии~$k$.

Напомним, что марковская цепь~$X(t)$ называется слабо эргодичной,
если $\| {\bf p^*}(t) \hm- {\bf p^{**}}(t) \| \hm\to 0 $ при $t\hm \to
\infty$ для любых начальных условий  ${\bf p^*}(s)$, ${\bf p^{**}}(s)$
и~любом $ s \hm\ge 0$. Марковская цепь~$X(t)$ имеет предельное среднее~$\phi (t)$, 
если $E(t,k) \hm- \phi (t) \hm\to 0$ при $t \hm\to \infty$ и~любом~$k$.

\section{Оценки для системы $M_t/M_t/S/S$}

Прежде всего отметим, что слабая эргодичность~$X(t)$ эквивалентна
условию $\la^*\hm+\mu^* \hm> 0$, это вытекает, например, из следствия~1~\cite{zAT}.

Сформулируем получаемые оценки отдельно для каждого из двух случаев.

\noindent
\textbf{Теорема~1.}\
\textit{Пусть $\mu^* \hm> 0$. Тогда~$X(t)$ имеет предельный 1-пе\-рио\-ди\-че\-ский
режим  
$$
{\bf \pi}(t) =\left(\pi_0(t),\pi_1(t),\dots,\pi_S(t)\right)^\mathrm{T} 
$$ 
и~соответствующее предельное среднее 
$$
\phi(t)= \sum\limits_{k=0}^Sk\pi_k(t)
$$
 и~при любом $t \hm\ge 0$ выполняются следующие оценки скорости схо\-ди\-мости}:
\begin{equation}
\|{\bf p}^*(t) - {\bf \pi}(t)\| \le 8S e^{-\mu^* (t-1)}
 \label{022'}
\end{equation}
\textit{при любых начальных условиях};
\begin{equation}
|E(t,k) - \phi (t)|  \le 8S^2e^{-\mu^* (t-1)}
 \label{023'}
\end{equation}
\textit{при любом}~$k$.
\textit{ При $\lambda^* \hm> 0$ вместо}~(\ref{022'}) \textit{и}~(\ref{023'}) 
\textit{справедлива оценка}:
\begin{equation}
\|{\bf p^*}(t) - {\bf \pi}(t)\| \le    8S
e^{-{\lambda^*(t-1)}/{S}}. \label{053}
\end{equation}


\smallskip

Для д\,о\,к\,а\,з\,а\,т\,е\,л\,ь\,с\,т\,в\,а\
 теоремы~1 рассмотрим положительные числа~$d_i$ и~положим
\begin{alignat*}{2}
%\left.
%\begin{array}{rlrl}
d&=\mathop{\mathrm{inf}}\limits_{i \ge 1} d_i = 1\,; &\enskip W&=\inf_{i \ge 1} \fr {d_i}{i}\,;\\
\displaystyle g_i&=\sum\limits_{n=1}^i d_n; &\enskip 
\displaystyle G &= \sum\limits_{i=1}^Sd_i.
%\end{array}
%\right\}
 %\label{dW}
\end{alignat*}

Рассмотрим теперь выражения:
\begin{align}
\alpha_{k}\left( t\right) &= \lambda _k\left( t\right) +\mu
_{k+1}\left( t\right) -{}\notag\\
&{}- \fr{d_{k+1}}{d_k} \lambda _{k+1}\left(t\right) -
\fr{d_{k-1}}{d_k} \mu _k\left( t\right),
\enskip k \ge 0\,; \notag %\label{211}
\\
\beta_*\left( t\right) &= \inf_{k\geq 0} \alpha_{k}\left( t\right).
\label{212}
\end{align}

Тогда, применяя соответствующее преобразование прямой системы
Колмогорова, из следствия~1~\cite{zAT} получаем такие оценки:
\begin{align*}
\|{\bf p}^*(t) - {\bf \pi}(t)\| &\le \fr{8G}{d}\,
e^{-\int_0^t\beta_*(\tau)\,d\tau};
% \label{022'1}
\\
|E(t,k) - \phi (t)| & \le  \fr{8G}{W}\,
e^{-\int_0^t\beta_*(\tau)\,d\tau}.
 %\label{023'1}
\end{align*}


Положим вначале все $d_i\hm=1$, тогда все
$\alpha_{k}(t)\hm=\beta_*\hm=\mu(t)$, и~с~учетом неравенства
$$
e^{-\int\nolimits_0^t\mu(\tau)\,d\tau}\hm \le e^{-\int_0^{[t]}\mu(\tau)\,d\tau}
\le e^{-\mu^*(t-1)}
$$ 
получаем оценки~(\ref{022'}) и~(\ref{023'}).


Теперь, следуя ходу рассуждений теоремы~93 из~\cite{zbs}, положим
$$
d_1=1\,;\quad d_{k+1}=\fr{S-1}{S}\,d_k <1\,.
$$ 
Тогда получим все
$\alpha_{k}(t) \hm> {\la(t)}/{S}$, откуда $\beta_* \hm\ge
{\la(t)}/{S}$. При этом  $G \hm\le S$, а~$e^{-\int_0^t\la(\tau)\,d\tau} \hm\le 
e^{-\la^*(t-1)}$ и,~значит,
справедливо неравенство~(\ref{053}).

\section{Оценки для~системы $M_t/M_t/S$}

Отметим, что для наличия слабой эргодичности достаточно, чтобы
выполнялось условие $\la^* \hm< S\mu^*$.

Рассмотрим треугольную матрицу $D$:
\begin{equation*}
D=\begin{pmatrix}
d_1   & d_1 & d_1 & \cdots  \\
0   & d_2  & d_2  &   \cdots  \\
0   & 0  & d_3  &   \cdots  \\
\vdots& \vdots & \ddots & \ddots 
\end{pmatrix}
%\label{204}
\end{equation*}
и пространства  последовательностей~$l_{1D}$:
$$
l_{1D}=\left\{{\bf z} = (p_1,p_2,\ldots)^{\mathrm{T}} :\, \|{\bf z}\|_{1D}
\equiv \|D {\bf z}\| <\infty \right\}
 $$
и~$l_{1E}$:
$$
l_{1E}=\left\{{\bf z} = (p_1,p_2,\ldots)^{\mathrm{T}} :\, \|{\bf z}\|_{1E}
\equiv \sum k|p_k| <\infty \right\}.
$$



Выберем число $\delta \in \left(1,{S}/({S-1})\right)$.Тогда, как
показано в~\cite[параграф~4.2]{zbs}, в~(\ref{212}) получается
$$
\beta_*\left( t\right)  \ge \left(S\mu(t)\hm-\delta\la(t)\right)
\left(1-\delta^{-1}\right).
$$
А~теперь, применяя  теорему~1 и~следствие~1 из 
работы~\cite{Zeifman2014i}, получаем общие оценки:
\begin{align*}
\|{\bf p^*}(t) - {\bf p^{**}}(t)\|_{1D}  &\le{}\notag\\
&\hspace*{-20mm}{}\le 4 e^{-\int\limits_0^t
\beta_*(\tau)\, d\tau}\sum_{i \ge 1}g_i |p^*_i(0) - p^{**}_i(0)|\,;
%\label{215}
\\
\|{\bf p^*}(t) - {\bf p^{**}}(t)\|_{1E}& = \left|\phi(t) - E(t,k)\right| \le {}\notag\\
&\hspace*{-20mm}{}\le
\fr{4}{W}\, e^{-\int\limits_0^t
\beta_*(\tau)\, d\tau}\sum\limits_{i \ge 1}g_i |p^*_i(0) - p^{**}_i(0)|
%\label{222}
\end{align*} 
при любом $ t \ge 0$ и~любых начальных условиях ${\bf p^*}(0)$ и~${\bf p^{**}}(0)$.



Далее, имеем в~1-пе\-рио\-ди\-че\-ском случае
\begin{multline*}
e^{-\int\limits_0^t \left(S\mu(\tau)-\delta\la(\tau)\right)\left(
1-\delta^{-1}\right)\, d\tau} \le{}\\
{}\le
 e^{-\int\limits_0^{[t]}
\left(S\mu(\tau)-\delta\la(\tau)\right)\left(1-\delta^{-1}\right)\,
d\tau} e^{\int\limits_{[t]}^t
\delta\la(\tau)\left(1-\delta^{-1}\right)\, d\tau}\le{}\\
{}\le
e^{\la^*\left(\delta - 1\right)-
\left(S\mu^*-\delta\la^*\right)\left(1-\delta^{-1}\right)\left(t-1\right).
}
\end{multline*}

\noindent
\textbf{Теорема~2.}
\textit{Пусть $\la^* \hm< S\mu^*$. Тогда~$X(t)$ имеет предельный 1-периодический
режим  ${\bf \pi}(t)
\hm=\left(\pi_0(t),\pi_1(t),\dots,\right)^\mathrm{T} $ 
и~соответствующее предельное среднее $\phi(t) \hm= \sum\nolimits_{k \ge 0}
k\pi_k(t)$ и~выполняются сле\-ду\-ющие оценки скорости сходимости}:

\noindent
\begin{multline*}
\!\!\!\|{\bf p^*}(t) - {\bf \pi}(t)\|_{1D} \! \le 4 e^{\lambda^*\left(\delta -
1\right)-\left(S\mu^*-\delta\lambda^*\right)\left(1-\delta^{-1}\right)\left(t-1\right)}\times{}\\
{}\times \sum_{i \ge 1}g_i |p^*_i(0) - \pi_i(0)|\,; 
%\label{2150}
\end{multline*}

\vspace*{-18pt}

\noindent
\begin{multline*}
\left|\phi(t) - E(t,k)\right| \le{}\\[-1pt]
{}\le \fr{4g_k}{W}\,
e^{\la^*\left(\delta - 1\right)-
\left(S\mu^*-\delta\la^*\right)\left(1-\delta^{-1}\right)\left(t-1\right)}. 
%\label{2220}
\end{multline*}

\vspace*{-16pt}

\section{Примеры}

Рассмотрим модели систем обслуживания $M_t/M_t/S/S$ и~$M_t/M_t/S$  
с~интенсивностями  $\mu\hm=1$, $\la(t)\hm=\la(1+M\sin 2\pi\omega t)$ 
и~различными вариациями <<амплитуды>>~$M$ и~<<частоты>>~$\omega$. Выбор $S \hm=100$ 
в~первом примере обусловлен возможностью непосредственного
вычисления предельных характеристик этой системы. Для модели
$M_t/M_t/S$ постро\-ение предельного среднего проводится с~по\-мощью
усечений процессами с~конечным пространством состояний 
в~соответствии с~методикой~\cite{Zeifman2014i}, здесь
для простоты
взято $S\hm=10$. На приведенных рис.~1--6 показано влияние амплитуды 
и~частоты
 интенсивности поступления требований на предельные
характеристики процесса, описывающего число требований в~системе.
Интересно отметить, что <<двойное~среднее>>, т.\,е.\ величина
$\textsf{E}\hm=\int\nolimits_0^1\phi(t)\,dt$,
 по крайней мере в~пределах
получаемой точности, оказывается не зависящей от этих характеристик.


Для построения всех интересующих величин задача Коши для прямой
системы Колмогорова с~начальным условием, соответствующим ситуации
$X(0)\hm=0$, решается на  $[0, t^*]$, а~затем нужная величина 
с~точ\-ностью~$10^{-3}$ строится на отрезке $[t^*,t^*+1]$. Во втором
примере предварительно выбирается нужная размерность <<усеченного>>
процесса~$N$.

\noindent
\textbf{Пример~I.} Система $M_t/M_t/S/S$, $S\hm=100$, $\la\hm=10$. Получаем
$t^*\hm=15$. Рассмотрены случаи:

{ \begin{center}  %fig1
 \vspace*{-4pt}
  \mbox{%
 \epsfxsize=74.19mm 
 \epsfbox{zei-1.eps}
 }


\end{center}

\vspace*{-1pt}


\noindent
{{\figurename~1}\ \ \small{Среднее число требований в~системе $E(t,0)$ на
отрезке $[0,15]$: \textit{1}~--- пример~I, случай~1; \textit{2}~--- 
соответствующий однородный процесс}}

}

\vspace*{9pt}


{ \begin{center}  %fig2
 \vspace*{5pt}
 \mbox{%
 \epsfxsize=79mm 
 \epsfbox{zei-2.eps}
 }


\end{center}

\vspace*{-1pt}


\noindent
{{\figurename~2}\ \ \small{Среднее число требований в~системе $E(t,0)$ на
отрезке $[15,16]$, пример~I, амплитуды $M\hm=1$~(\textit{а})
и~0,1~(\textit{б}): \textit{1}~--- $w\hm=1$; \textit{2}~--- 5; \textit{3}~--- $w\hm=25$}}

}

\vspace*{3pt}

\begin{enumerate}[(1)]
\item  $M=1$, $\omega =1$;
\item  $M=1$, $\omega =5$;
\item  $M=1$, $\omega =25$;
\end{enumerate}

{ \begin{center}  %fig3
 \vspace*{-5pt}
\mbox{%
 \epsfxsize=75.719mm 
 \epsfbox{zei-4.eps}
 }


\end{center}

\vspace*{-3pt}


\noindent
{{\figurename~3}\ \ \small{Среднее число требований в~системе $E(t,0)$ на
отрезке $[0,20]$: \textit{1}~--- пример~II, случай~1; \textit{2}~--- 
соответствующий однородный процесс}}

}


\vspace*{9pt}




{ \begin{center}  %fig4
 \vspace*{-2pt}
 \mbox{%
 \epsfxsize=79mm 
 \epsfbox{zei-5.eps}
 }


\end{center}

\vspace*{-5pt}


\noindent
{{\figurename~4}\ \ \small{Среднее число требований в~системе $E(t,0)$ на
отрезке $[20,21]$, пример~II, амплитуды $M\hm=1$~(\textit{а})
и~0,1~(\textit{б}):
\textit{1}~--- $w\hm=1$; \textit{2}~--- 5; \textit{3}~--- $w\hm=25$}}

}


\vspace*{10pt}




\begin{enumerate}[(1)]
\setcounter{enumi}{3}
\item  $M=0{,}1$, $\omega =1$;
\item  $M=0{,}1$, $\omega =5$;
\item  $M=0{,}1$, $\omega =25$.
\end{enumerate}
При этом $\textsf{E}\hm \approx \int\nolimits_0^1\phi(t)\,dt \hm= 10{,}000$.

\smallskip

\noindent
\textbf{Пример~II.}\ Система $M_t/M_t/S$, $S\hm=10$, $\la\hm=4$. Здесь
получается $N\hm=100$, $t^*\hm=20$. Рассмотрены случаи:
\begin{enumerate}[(1)]
\item  $M=1$, $\omega =1$;
\item  $M=1$, $\omega =5$;
\item  $M=1$, $\omega =25$;
\item  $M=0{,}1$, $\omega =1$;
\item  $M=0{,}1$, $\omega =5$;
\item  $M=0{,}1$, $\omega =25$.
\end{enumerate}
При этом $\textsf{E} \approx \int\nolimits_0^1\phi(t)\,dt \hm= 4{,}006$.

{\small\frenchspacing
 {%\baselineskip=10.8pt
 \addcontentsline{toc}{section}{References}
 \begin{thebibliography}{99}
 
 \bibitem{Guo2013}  %1
\Au{Guo Y.,  Wang~Z.}  Stability of Markovian jump systems with
generally uncertain transition rates~// J.~Frankl.
Inst., 2013. Vol.~350. Iss.~9. P. 2826--2836.

\bibitem{Crawford2014}  %2
\Au{Crawford F.\,W., Minin~V.\,N.,  Suchard~M.\,A.}  
Estimation for general birth-death processes~// 
J.~Am. Stat. Assoc., 2014. Vol.~109. Iss.~506. P.~730--747.

\bibitem{Dong2015}   %3
\Au{Dong J., Whitt~W.} Stochastic grey-box modeling of queueing systems:
Fitting birth-and-death processes to data~// Queueing Syst., 2015.
Vol.~79. P.~391--426.

\bibitem{Zhu2016} %4
\Au{Zhu D.\,M., Ching~W.\,K.,  Guu~S.\,M.} 
Sufficient conditions for the ergodicity of fuzzy Markov chains~// 
Fuzzy Set.  Syst., 2016. Vol.~304. P. 82--93.

\bibitem{Cruz2017}  %5
\Au{Cruz F.\,R.\,B., Quinino~R.\,C.,  Ho~L.\,L.}  Bayesian estimation of
traffic intensity based on queue length in a~multi-server $M/M/s$
queue~// Commun. Stat. Simulat.,
2017. Vol.~46. P. 7319--7331.



\bibitem{Ho2017} %6
\Au{Ho~L.\,S.\,T., Xu~J., Crawford~F.\,W., Minin~V.\,N.,  Suchard~M.\,A.}
Birth/birth--death processes and their computable transition probabilities 
with biological applications~// J.~Math. Biol.,  2018. Vol.~76. P. 911--944.



\bibitem{z89}  %7
\Au{Зейфман А.\,И.}  Некоторые свойства системы с~потерями в~случае
переменных интенсивностей~// Автоматика и~телемеханика, 1989.
Вып.~1. С.~107--113.

\bibitem{ki90}  %8
\Au{Kijima M.} On the largest negative eigenvalue of the
infinitesimal generator associated with $M/M/n/n$ queues~//
Oper.  Res. Lett.,  1990. Vol.~9. P. 59--64.

\bibitem{z95}    %9
\Au{Zeifman A.\,I.}  Upper and lower bounds on the
rate of convergence for nonhomogeneous birth and death processes~//
Stoch.  Proc. Appl.,  1995. Vol.~59. P.
157--173.

\bibitem{FRT}  %10
\Au{Fricker C., Robert~P., Tibi~D.}  
On the rate of convergence of Erlang's model~//
 J.~Appl. Probab., 1999. Vol.~36. P.~1167--1184.

\bibitem{voit} %11 
\Au{Voit M.}  A~note of the rate of convergence to equilibrium
 for Erlang's model in the subcritical case~//  J.~Appl.
  Probab., 2000. Vol.~37. P. 918--923.

\bibitem{z06} %12
\Au{Zeifman A., Leorato~S., Orsingher~E., Satin~Ya., Shilova~G.}  
Some universal limits for nonhomogeneous birth and death
processes~// Queueing Syst., 2006. Vol.~52. P.~139--151.

\bibitem{zbs} %13
\Au{Зейфман А.\,И., Бенинг~В.\,Е., Соколов~И.\,А.}  
Марковские цепи и~модели с~непрерывным временем.~--- М.:  ЭЛЕКС-КМ, 2008. 168~с.

\bibitem{dz}  %14
\Au{Van Doorn~E.\,A., Zeifman~A.\,I.} On the speed of convergence to stationarity
of the Erlang loss system~// Queueing Syst.,  2009. Vol.~63. P.~241--252.

\bibitem{zAT} %15
\Au{Зейфман А.\,И.}  
О~нестационарной модели Эрланга~// Автоматика и~телемеханика, 2009. Вып.~12. 
С.~71--80.

\bibitem{Doorn2010} %16
\Au{Van Doorn~E.\,A., Zeifman~A.\,I.,  Panfilova~T.\,L.}  Bounds and asymptotics
for the rate of convergence of birth--death processes~//  Theor. 
Probab. Appl., 2010. Vol.~54. P. 97--113.

\bibitem{Zeifman2014i}  %17
\Au{Zeifman A.,  Satin~Ya.,  Korolev~V.,  Shorgin~S.}  On truncations for weakly
ergodic inhomogeneous birth and death processes~// Int.
J.~Appl. Math. Comp., 2014. Vol.~24. P.~503--518.
 \end{thebibliography}

 }
 }

\end{multicols}

\vspace*{-6pt}

\hfill{\small\textit{Поступила в~редакцию 06.06.19}}

\vspace*{8pt}

%\pagebreak

%\newpage

%\vspace*{-28pt}

\hrule

\vspace*{2pt}

\hrule

%\vspace*{-2pt}

\def\tit{ON THE~BOUNDS OF~THE~RATE OF~CONVERGENCE FOR~SOME~QUEUEING MODELS 
WITH~INCOMPLETELY~DEFINED~INTENSITIES}


\def\titkol{On the~bounds of~the~rate of~convergence for~some queueing models 
with~incompletely defined intensities}

\def\aut{A.\,I.~Zeifman$^{1,2,3}$, Y.\,A.~Satin$^{1}$, and~K.\,M.~Kiseleva$^{1}$}

\def\autkol{A.\,I.~Zeifman, Y.\,A.~Satin, and~K.\,M.~Kiseleva}

\titel{\tit}{\aut}{\autkol}{\titkol}

\vspace*{-11pt}


\noindent
$^1$Vologda State University,
15~Lenin Str., Vologda 160000, Russian Federation

\noindent
$^2$Institute of Informatics Problems, Federal Research Center 
``Computer Sciences and Control'' of the Russian\linebreak
$\hphantom{^1}$Academy of Sciences, 44-2~Vavilov 
Str., Moscow 119133, Russian Federation

\noindent
$^3$Vologda Research Center of the Russian Academy of Sciences,
56A Gorky Str., Vologda 160014, Russian\linebreak
$\hphantom{^1}$Federation

\def\leftfootline{\small{\textbf{\thepage}
\hfill INFORMATIKA I EE PRIMENENIYA~--- INFORMATICS AND
APPLICATIONS\ \ \ 2019\ \ \ volume~13\ \ \ issue\ 3}
}%
 \def\rightfootline{\small{INFORMATIKA I EE PRIMENENIYA~---
INFORMATICS AND APPLICATIONS\ \ \ 2019\ \ \ volume~13\ \ \ issue\ 3
\hfill \textbf{\thepage}}}

\vspace*{3pt}    


\Abste{The authors consider some queuing systems with
incompletely defined 1-periodical intensities under corresponding
conditions. The authors deal with $M_t/M_t/S$ queue for any number
of servers $S$ and $M_t/M_t/S/S$ (the Erlang model). Estimates of
the rate of convergence in weakly ergodic situation are obtained by
applying the method of the logarithmic norm of the operator of a~linear
function. The examples with exact given values of intensities and
different variations of amplitude and frequency are considered,
ergodicity conditions and estimates of the rate of convergence are
obtained for each model, and plots of the effect of intensities'
amplitude and frequency of incoming requirements on the limiting
characteristics of the process are constructed. The authors use the
general algorithm to build graphs, it is associated with solving the
Cauchy problem for the forward Kolmogorov system on the
corresponding interval, which has already been used by the authors
in previous  papers.}


\KWE{queuing systems; incompletely defined intensities; rate of convergence; ergodicity;
logarithmic norm; $M_t/M_t/S$ queue; $M_t/M_t/S/S$ queue}


\DOI{10.14357/19922264190303} 

%\vspace*{-14pt}

\Ack
\noindent
This work was financially supported by the Russian Science Foundation 
(grant No 19-11-00020).


%\vspace*{-6pt}

  \begin{multicols}{2}

\renewcommand{\bibname}{\protect\rmfamily References}
%\renewcommand{\bibname}{\large\protect\rm References}

{\small\frenchspacing
 {%\baselineskip=10.8pt
 \addcontentsline{toc}{section}{References}
 \begin{thebibliography}{99}
 
 \bibitem{6-zei}%1
\Aue{Guo, Y., and Z.~Wang.}
 2013.  Stability of Markovian jump systems with generally uncertain 
 transition rates. \textit{J.~Frankl. Inst.} 350(9):2826--2836.
 
\bibitem{1-zei} %2
\Aue{Crawford, F.\,W., V.\,N.~Minin, and M.\,A.~Suchard.}
 2014. Estimation for birth-death processes. 
 \textit{J.~Am. Stat. Assoc.} 109(506):730--747.


\bibitem{3-zei} %3
\Aue{Dong J., and W.~Whitt.}
 2015. Stochastic grey-box modeling of queueing systems: 
 Fitting birth-and-death processes to data. \textit{Queueing Syst.} 79:391--426.


\bibitem{5-zei} %4
\Aue{Zhu, D.\,M., W.\,K.~Ching, and S.\,M.~Guu.}
2016. Sufficient conditions for the ergodicity of fuzzy Markov chains. 
\textit{Fuzzy Set. Syst.} 304:82--93.

\bibitem{2-zei} %5
\Aue{Cruz, F.\,R.\,B., R.\,C.~Quinino, and L.\,L.~Ho.}
2017. Bayesian estimation of traffic intensity based on queue length 
in a~multi-server $M/M/s$ queue. 
\textit{Commun. Stat. Simulat.} 46:7319--7331.


\bibitem{4-zei} %6
\Aue{Ho, L.\,S.\,T., J.~Xu, F.\,W.~Crawford, V.\,N.~Minin, and M.\,A.~Suchard.}
 2018. Birth/birth-death processes and their computable transition probabilities 
 with biological applications. \textit{J.~Math. Biol.} 76:911--944.

\bibitem{12-zei}%7
\Aue{Zeifman, A.\,I.} 1989. Some properties of a~system with losses in the 
case of variable rates. 
\textit{Automat. Rem. Contr.} 50(1):82--87. 

\bibitem{10-zei} %8
\Aue{Kijima, M.} 1990. On the largest negative eigenvalue of the 
infinitesimal generator associated with $M/M/n/n$ queues.
\textit{Oper. Res. Lett.} 9:59--64.

\bibitem{13-zei} %9
\Aue{Zeifman, A.\,I.}
 1995. Upper and lower bounds on the rate of convergence for nonhomogeneous 
 birth and death processes. 
  \textit{Stoch. Proc. Appl.} 59:157--173.
  
  \bibitem{9-zei} %10
\Aue{Fricker C., P.~Robert, and D.~Tibi.}
 1999. On the rate of convergence of Erlang's model.
 \textit{J.~Appl. Probab.} 36:1167--1184.
  
  \bibitem{11-zei} %11
\Aue{Voit, M.} 2000. 
A~note of the rate of convergence to equilibrium for Erlang's model 
in the subcritical case. \textit{J.~Appl. Probab.} 37:918--923.

\bibitem{14-zei} %12
\Aue{Zeifman, A., S.~Leorato, E.~Orsingher, Ya.~Satin, and G.~Shi\-lo\-va.}
2006. Some universal limits for nonhomogeneous birth and death processes. 
 \textit{Queueing Syst.} 52:139--151.
 
\bibitem{15-zei} %13
\Aue{Zeifman, A.\,I., V.\,E.~Bening, and I.\,A.~Sokolov.}
 2008.  \textit{Markovskie tsepi i~modeli s~nepreryvnym vremenem} 
 [Markov chains and models with continuous time]. Moscow: ELEKS-KM Publs. 168~p.
 
 \bibitem{7-zei}%14
\Aue{Van Doorn, E.\,A., and A.\,I.~Zeifman.} 
2009. On the speed of convergence to stationarity of the Erlang loss system. 
\textit{Queueing Syst.} 63:241--252.

\bibitem{16-zei} %15
\Aue{Zeifman, A.\,I.} 
 2009. On the nonstationary Erlang loss model.
  \textit{Automat. Rem. Contr.} 70(12):2003--2012.
  
  \bibitem{8-zei} %16
\Aue{Van Doorn, E.\,A., A.\,I.~Zeifman, and T.\,L.~Panfilova.}
 2010. Bounds and asymptotics for the rate of convergence of birth--death processes. 
 \textit{Theor. Probab.  Appl.} 54:97--113.
 
\bibitem{17-zei}
\Aue{Zeifman, A., Ya.~Satin, V.~Korolev, and S.~Shorgin.}
 2014. On truncations for weakly ergodic inhomogeneous birth and death processes. 
 \textit{Int. J.~Appl. Math. Comp.} 24:503--518.

\end{thebibliography}

 }
 }

\end{multicols}

%\vspace*{-7pt}

\hfill{\small\textit{Received June 6, 2019}}

%\pagebreak

%\vspace*{-22pt}

 


\Contr

\noindent
\textbf{Zeifman Alexander I.} (b. 1954)~--- 
Doctor of Science in physics and mathematics, professor, 
Head of Department, Vologda State University, 15~Lenin Str., Vologda 160000, 
Russian Federation; 
senior scientist, Institute of Informatics Problems, Federal Research Center 
``Computer Sciences and Control'' of the Russian Academy of Sciences, 44-2~Vavilov 
Str., Moscow 119133, Russian Federation; 
principal scientist, Vologda Research Center of the Russian Academy of Sciences,
56A Gorky Str., Vologda 160014, Russian
Federation; \mbox{a\_zeifman@mail.ru}

\vspace*{3pt}

\noindent
\textbf{Satin Yacov A.} (b.\ 1978)~--- 
Candidate of Science (PhD) in physics and mathematics, associate professor, 
Vologda State University, 15~Lenin Str., Vologda 160000; \mbox{yacovi@mail.ru}

\vspace*{3pt}

\noindent
\textbf{Kiseleva Ksenia M.} (b.\ 1992)~--- 
Candidate of Science (PhD) in physics and mathematics, scientist, 
Vologda State University, 15~Lenin Str., Vologda 160000, Russian Federation; 
\mbox{ksushakiseleva@mail.ru}
\label{end\stat}

\renewcommand{\bibname}{\protect\rm Литература}  