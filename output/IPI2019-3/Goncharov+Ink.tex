\def\stat{ginch+ink}

\def\tit{МЕТОДИКА ПОИСКА ИМПЛИЦИТНЫХ ЛОГИКО-СЕМАНТИЧЕСКИХ ОТНОШЕНИЙ 
В~ТЕКСТЕ$^*$}

\def\titkol{Методика поиска имплицитных логико-семантических отношений 
в~тексте}

\def\aut{А.\,А.~Гончаров$^1$, О.\,Ю.~Инькова$^2$}

\def\autkol{А.\,А.~Гончаров, О.\,Ю.~Инькова}

\titel{\tit}{\aut}{\autkol}{\titkol}

\index{Гончаров А.\,А.}
\index{Инькова О.\,Ю.}
\index{Goncharov A.\,A.}
\index{Inkova O.\,Yu.}


{\renewcommand{\thefootnote}{\fnsymbol{footnote}} \footnotetext[1]
{Работа выполнена в~Институте проблем информатики ФИЦ ИУ РАН при поддержке РФФИ (проект по 
гранту №\,18-07-00192).}}


\renewcommand{\thefootnote}{\arabic{footnote}}
\footnotetext[1]{Институт проблем информатики Федерального исследовательского центра 
<<Информатика и~управ\-ле\-ние>> Российской академии наук, \mbox{a.gonch48@gmail.com}}
\footnotetext[2]{Институт проблем информатики Федерального исследовательского центра <<Информатика 
и~управ\-ле\-ние>> Российской академии наук, \mbox{olyainkova@yandex.ru}}

%\vspace*{-2pt}


   
   
   \Abst{Рассматривается методика поиска имплицитных логико-семантических 
отношений (ЛСО) в~текстах надкорпусной базы данных (НБД) коннекторов. На примере 
направления перевода <<рус\-ский\;$\to$\;фран\-цуз\-ский>> описаны стадии процесса 
поиска: (1)~выбор анализируемого ЛСО и~создание массива аннотаций тех коннекторов 
русского языка, которые могут считаться прототипическими средствами выражения этого 
ЛСО; (2)~анализ полученного массива аннотаций и~выявление основных переводных 
эквивалентов для данных коннекторов во французском языке; (3)~использование 
двуязычного поиска в~НБД, причем для текстов на русском языке исключаются 
коннекторы, аннотированные на первой стадии, а для текстов на французском проводится 
поиск по языковым единицам, представляющим собой наиболее частотные переводные 
эквиваленты этих коннекторов, определенные на второй стадии; (4)~аннотирование 
найденных таким образом пар фрагментов параллельных текстов; (5)~анализ 
сформированных аннотаций, дающий возможность произвести категоризацию случаев, 
когда ЛСО в~русском языке является имплицитным. Предлагаемая методика поиска в~НБД 
позволяет получить новое знание об имплицитных ЛСО.}
   
   \KW{выявление имплицитной информации; коннекторы; контрастивная лингвистика; 
корпусная лингвистика; надкорпусные базы данных; ло\-ги\-ко-се\-ман\-ти\-че\-ские отношения}

\DOI{10.14357/19922264190314} 
  
\vspace*{-2pt}


\vskip 10pt plus 9pt minus 6pt

\thispagestyle{headings}

\begin{multicols}{2}

\label{st\stat}
   
\section{Введение}

   Кросслингвистическая НБД коннекторов (подробнее о структуре НБД 
см.~[1, 2], об использу\-емой системе терминов см.~[3, 4]) разрабатывается 
в~ИПИ ФИЦ ИУ РАН на основе текстов параллельных подкорпусов 
Национального корпуса русского языка (НКРЯ) с~учетом особенностей этих 
языковых единиц, основная функция которых состоит в~выражении  
ЛСО в~тексте~[5]. 

В~процессе 
работы исследователи столкнулись со случаями, когда в~параллельных 
текстах на одном из двух языков ЛСО выражается коннектором, а на другом 
языке либо используются отличные от коннекторов средства, либо ЛСО не 
выражено никакой языковой единицей, т.\,е.\ является имплицитным. 

Такие 
случаи представляют интерес как для лингвистики, так и~для разработки 
методов автоматической обработки текстов и~повышения качества 
машинного перевода, поскольку позволяют получить новое знание 
о~языковых средствах, которые используются для выражения того или иного 
ЛСО. В~то же время выявление случаев, когда то или иное ЛСО не выражено 
в~тексте эксплицитно (или выражено не своим 
прототипическим\footnote[3]{Прототипическим считается коннектор, который 
специализируется на выражении данного ЛСО, наиболее часто упоминается как его показатель 
в~грамматиках и~словарях, а значит, может считаться репрезентативным образцом показателя 
данного ЛСО (см.\ понятие <<лучшего представителя>> в~когнитивной лингвистике и~психологии, 
например, в~работе~\cite{6-in}).} показателем), является довольно сложной 
и~трудоемкой задачей. Самый распространенный на сегодняшний день 
способ выявления таких случаев~--- сплошная ручная разметка, требующая 
значительных временн$\acute{\mbox{ы}}$х и~человеческих ресурсов~[7--10].
   
   Подобные случаи не являются, однако, редкими и~исключительными. 
Таблица~1 содержит данные по числу аннотаций переводных соответствий 
(далее~--- ПС), сформированных в~НБД, где в~одной из частей были 
проставлены признаки, ука\-зы\-ва\-ющие на отсутствие эксплицитного 
показателя ЛСО. Это явление может быть вызвано рядом факторов, описание 
которых выходит за рамки данной статьи (о~некоторых из них  
см.~\cite{11-in}). Признаки, присваиваемые аннотации при разметке во всех 
таких случаях, объединены в~кластер \textit{Zero}.
   
   
   
   
   \begin{table*}\small %tabl1
   \begin{center}
   \parbox{430pt}{\Caption{Распределение аннотаций ПС с~признаками из кластера \textit{Zero} в~НБД 
коннекторов до начала эксперимента по описываемой методике
(указано число аннотаций, сформированных в~НБД до 11.03.2019 
включительно (учитываются только кортежи первого типа, подробнее о категориях кортежей 
см.~\cite{1-in}). Аннотации в~рамках эксперимента формировались начиная с~12.03.2019)}
}
   \vspace*{2ex}
   
   \begin{tabular}{|c|c|c|c|c|}
   \hline
   Направление перевода&\tabcolsep=0pt\begin{tabular}{c}Общее число\\ аннотаций ПС\\
   до начала\\ 
эксперимента\end{tabular}&Язык&
\tabcolsep=0pt\begin{tabular}{c}Число аннотаций ПС\\ с~признаками\\ из кластера 
\textit{Zero}\end{tabular}&
\tabcolsep=0pt\begin{tabular}{c}Примерная доля \\ аннотаций ПС\\ с~признаками\\ из кластера
 \textit{Zero}\end{tabular}\\ 
\hline 
Русский\;$\to$\;французский&12\,667&Русский&\hphantom{9}94&0,7\%\\ 
&&Французский&1438\hphantom{9}&11,4\%\\
   \hline 
Французский\;$\to$\;русский&\hphantom{9\,}3049&Французский&435&14,2\%\\ 
&&Русский&\hphantom{9}64&2,1\%\\
   \hline
   \end{tabular}
   \end{center}
   \vspace*{-6pt}
   \end{table*}

   Основной причиной столь значительного чис\-лен\-но\-го перевеса аннотаций, в~которых признаки из кластера \textit{Zero} проставлены именно во 
франкоязычной части, является прежде всего тот факт, что поиск пар 
фрагментов параллельных текстов для аннотирования осуществлялся, как 
правило, в~текстах на русском языке по эксплицитной языковой единице для 
обоих направлений перевода. В~результате было 
сформировано~1438~аннотаций в~направлении  
<<рус\-ский\;$\to$\;фран\-цуз\-ский>> и~435~аннотации в~направлении 
<<фран\-цуз\-ский\;$\to$\;рус\-ский>>, где ЛСО имплицировано в~тексте на 
французском языке. Цель статьи состоит в~постановке и~решении задачи 
поиска в~НБД подобных случаев в~русскоязычном тексте, их последующего 
аннотирования и~анализа. Для эффективного решения этой задачи и~была 
разработана рассматриваемая методика.

\vspace*{-6pt}
   
\section{Исходные данные для~постановки задачи поиска}

\vspace*{-4pt}

Первые две стадии методики обеспечивают получение исходных данных для решения задачи: во-пер\-вых, 
выбор анализируемого ЛСО и~создание массива аннотаций тех коннекторов русского языка, которые могут 
считаться прототипическими средствами выражения этого ЛСО; во-вто\-рых, анализ полученного массива 
аннотаций и~выявление в~переводах на французский язык основных функционально эквивалентных 
фрагментов (ФЭФ) данных коннекторов русского языка. 
В~настоящей статье методика поиска случаев 
имплицитных ЛСО ил\-люст\-ри\-ру\-ет\-ся на примере ЛСО причины. К~основным коннекторам, выражающим 
данное отношение в~русском языке, относятся \textit{потому что}, \textit{ибо}, \textit{поскольку}, которые 
наиболее часто упоминаются в~словарях и~грамматиках  
(например,~\cite[с.~577--583]{12-in}). При поиске контекстов не учитывалось наличие запятой между 
частями коннектора \textit{потому что} и~возможное сочетание \textit{поскольку} с~компонентом 
\textit{то}, однако при аннотировании эти коннекторы фиксировались в~той форме, в~которой они 
употреблены в~тексте. Именно поэтому далее приводятся данные не по трем, а по пяти коннекторам 
русского языка, для которых в~сумме сформировано~350~аннотаций в~НБД (для направления перевода  
<<рус\-ский\;$\to$\;фран\-цуз\-ский>>): \textit{ибо} (50~аннотаций), \textit{поскольку} (46~аннотаций), 
\textit{поскольку}$\|$\textit{то}\footnote{Используемая при перечислении 
коннекторов двойная 
вертикальная черта <<$\|$>> указывает на то, что коннектор состоит из двух или более расположенных 
дистантно компонентов, каждый из которых вводит отдельный фрагмент текста; 
одиночная вертикальная 
черта~<<$\vert$>> говорит о том, что языковые единицы (элементы), составляющие коннектор или его 
компонент, разделены текстом, но находятся при этом в~рамках одного и~того же фрагмента.} 
(4~аннотации), \textit{потому что} (196~аннотаций) и~\textit{потому, что} (54~аннотации). Поскольку 
в~рамках данной статьи перечисленные коннекторы анализируются в~текстах оригинала, они предстают 
в~качестве стимулов перевода (СП). Соответствие СП и~ФЭФ будем называть моделью перевода (МП).
   
   Этот объем аннотаций оказался достаточным для выявления основных 
ФЭФ перечисленных коннекторов во французском языке. Таблица~2 
содержит информацию о~тех ПС, где коннектору русского языка 
соответствует ка\-кой-ли\-бо коннектор французского языка (т.\,е.\ случаи, 
когда перевод коннектора является конгруэнтным; о~понятиях 
<<конгруэнтный>> и~<<дивергентный>> перевод применительно 
к~коннекторам см.~\cite{13-in}). В~первом столбце пе\-ре\-чис\-ле\-ны  
коннекторы, зафиксированные хотя бы один раз как ФЭФ 
выбранных~5~коннекторов русского языка. Следующие~5~столб\-цов 
содержат число ПС, в~которых ФЭФ из первого столбца соответствует 
каждому из анализируемых СП. Последний столбец отражает общее число 
ПС для того или иного ФЭФ. Именно эта информация является 
определяющей при выборе единицы французского языка, которая будет 
использована для поиска ПС, где ЛСО причины в~русском языке выражается 
имплицитно.

   \begin{table*}\small %tabl2
   \begin{center}
   \Caption{Фрагмент таблицы по МП, где и~в~оригинале, и~в~переводе использован 
коннектор}
   \vspace*{2ex}
   
   \tabcolsep=3.8pt
   \begin{tabular}{|l|c|c|c|c|c|c|}
   \hline %\multicolumn{1}{|c|}{\raisebox{-6pt}[0pt][0pt]{
\multicolumn{1}{|c|}{\raisebox{-6pt}[0pt][0pt]
{\tabcolsep=0pt\begin{tabular}{c}ФЭФ французского языка,\\ представляющий собой
 коннектор\end{tabular}}}&\multicolumn{5}{c|}{СП 
 в~русском языке}&
 \multicolumn{1}{c|}{\raisebox{-6pt}[0pt][0pt]
 {\tabcolsep=0pt\begin{tabular}{c}Общее\\ число ПС\end{tabular}}}\\
\cline{2-6}
&ибо&поскольку&поскольку$\|$то&потому что&потому, что&\\
\hline
\hspace*{11.5mm}\textbf{1.\ car}&\textbf{35}\hphantom{9}&\textbf{5}&\textbf{0}&\textbf{65}\hphantom{9}&\textbf{0}&\textbf{105}\hphantom{9}\\
\hspace*{11.5mm}\textbf{2.\ parce 
que}&\textbf{1}&\textbf{1}&\textbf{0}&\textbf{63}\hphantom{9}&\textbf{35}\hphantom{9}&\textbf{100}\hphantom{9}\\
\hspace*{11.5mm}3.\ puisque&3&3&0&6&0&12\\
\hspace*{11.5mm}4.\ comme&0&5&2&2&0&\hphantom{9}9\\
\hspace*{11.5mm}5.\ dans la mesure o$\grave{\mbox{u}}$&0&5&0&0&0&\hphantom{9}5\\
\multicolumn{1}{|c|}{$\ldots$}&$\ldots$&$\ldots$&$\ldots$&$\ldots$&$\ldots$&$\ldots$\\
\hspace*{11.5mm}28.\ vu&0&0&0&1&0&\hphantom{9}1\\
\hline
\multicolumn{1}{|c|}{ВСЕГО}&42  (из 50)&29  (из 46)&2  (из 4)&156  (из 196)&47  (из 54)&276  (из 350)\\
\hline
\end{tabular}
\end{center}
\vspace*{-3pt}
\end{table*}
\begin{table*}[b]\small %tabl3
\vspace*{-9pt}
\begin{center}
\Caption{Пример результата двуязычного поиска, в~котором между единицами, 
включенными в~поисковый запрос, отсутствует соответствие}
\vspace*{2ex}

\begin{tabular}{|p{74mm}|p{84mm}|}
\hline
[$\ldots$] \textbf{поскольку} мне ни читать, ни тем более писать таких писем никогда не 
доводилось, я с~большим интересом пробегал их глазами, пока они мне не приелись, 
\textbf{потому что} накал Арнольдовой страсти от письма к~письму угасал, сменившись 
вскоре житейской прозой [$\ldots$].\newline
[А.~Вайнер, Г.~Вайнер. Эра милосердия (1975)]&[$\ldots$] \textbf{pour autant que} je 
n'avais jamais eu $\grave{\mbox{a}}$~lire, encore moins 
$\grave{\mbox{a}}$~$\acute{\mbox{e}}$crire, de telles lettres, je les parcourus avec beaucoup 
d'int$\acute{\mbox{e}}$r$\hat{\mbox{e}}$t, jusqu'au moment o$\grave{\mbox{u}}$ elles 
commenc$\grave{\mbox{e}}$rent $\grave{\mbox{a}}$~m'ennuyer, \textbf{car} l'ardeur de la 
passion d'Arnold diminuait de lettre en lettre pour faire place $\grave{\mbox{a}}$~une banale 
prose [$\ldots$].
\newline
\newline
[Trad.\ par J.-P.~Dussaussois et E.~Avrorine, en coll.\ avec 
\mbox{J.-G.}~Synakiewicz. Revu et 
corrig$\acute{\mbox{e}}$ par B.~Durupt (2005)]\\
\hline
\end{tabular}
\end{center}
\end{table*}

   Из табл.~2 видно, что среди~28~ФЭФ (не все из которых, впрочем, 
выражают именно ЛСО причины), упорядоченных по частотности, со 
значительным перевесом выделяются два коннектора французского языка~--- 
\textit{car} и~\textit{parce que}, употребленные в~105 и~100~случаях 
соответственно. После того как наиболее частотный ФЭФ найден (коннектор 
\textit{car}), можно переходить к~основной стадии предлагаемой  
методики~--- непосредственно поиску.

\vspace*{-9pt}

\section{Двуязычный поиск в~надкорпусной базе данных}

\vspace*{-4pt}

   
   Надкорпусная база данных коннекторов~--- кросслингвистическая, и~при 
работе с~ней имеется возможность осуществлять двуязычный поиск 
одновременно с~использованием как текстов оригинала, так и~текстов 
перевода (виды поиска, используемые в~шести существующих НБД, описаны 
в~\cite{14-in}). Для обнаружения примеров использования конкретной МП 
применяется поиск по текстам на обоих языках: так, средствами НБД можно 
\mbox{найти} все контексты, где фрагмент на русском языке содержит коннектор 
\textit{потому что}, а фрагмент на французском~--- \textit{car} (по 
состоянию на 15.06.2019 таких контекстов~221).
   
   Можно с~уверенностью утверждать лишь то, что эти контексты 
\textit{содержат} данные языковые единицы, но функциональное 
соответствие между заданными в~поисковом запросе единицами в~некоторых 
случаях может и~отсутствовать. К~примеру, если осуществить поиск пар 
фрагментов параллельных текстов, которые одновременно содержат 
\textit{поскольку} в~русской части и~\textit{car} во французской, будет 
найдено~6~таких пар, однако только в~5 из них между этими коннекторами 
есть функциональное соответствие. В~оставшейся паре используются оба 
коннектора, но \textit{поскольку} переведен при помощи \textit{pour autant 
que}, а~СП для французского \textit{car} является \textit{потому что} (см.\ 
табл.~3, где эти единицы выделены полужирным).
   

   
   Несмотря на наличие такого <<шума>>, этот вид поиска позволяет 
проверить возможность использования той или иной МП и,~если случаи ее 
использования были найдены, сформировать аннотации в~рамках НБД.
   
   Однако сама по себе функция двуязычного поиска еще не позволяет 
целенаправленно находить имплицитные и/или выраженные косвенно (ср.\ 
понятие <<альтернативные лексикализации>> в~\cite{15-in}, а~также группы 
случаев, выделяемые в~разд.~2) ЛСО. Для этих целей в~НБД реализован 
поиск, при котором ищутся такие пары контекстов, где во фрагменте на 
одном из языков употреблена ка\-кая-ли\-бо конкретная языковая единица, 
а~во фрагменте на другом языке отсутствуют частотные СП или ФЭФ этой 
единицы.

\begin{table*}\small %tabl4
\begin{center}
\Caption{Пара фрагментов, где для фрагмента на русском языке проставлен признак 
\textit{Absent FragmCNT}}
\vspace*{2ex}

\begin{tabular}{|p{78mm}|p{78mm}|}
\hline
Прокуратор начал с~того, что пригласил первосвященника на балкон, с~тем чтобы 
укрыться от безжалостного зноя, но Каифа вежливо извинился и~объяснил, что сделать 
этого не может.\newline 
[М.\,А.~Булгаков. Мастер и~Маргарита (1929--1940)]&Le procurateur 
\mbox{commen{\normalsize \!\!\ptb{\c{c}}}a} par inviter le grand pr$\hat{\mbox{e}}$tre 
$\grave{\mbox{a}}$~venir jusqu'$\grave{\mbox{a}}$ la terrasse couverte, afin de s'y abriter de 
la chaleur impitoyable, mais Ca$\ddot{\mbox{\normalsize\!\!\ptb{\i}}}$phe s'excusa poliment, en 
expliquant qu'il ne le pouvait pas, {\bfseries\textit{car on $\acute{\mbox{e}}$tait 
$\grave{\mbox{a}}$~la veille des f$\hat{\mbox{e}}$tes}}.\newline
[Trad.\ par C.~Ligny (1968)]\\
\hline
\end{tabular}
\end{center}
\end{table*}

   
Так, на основе приведенных в~разд.~2 данных по~350~аннотациям, где для выражения ЛСО причины 
используются коннекторы \textit{ибо}, \textit{поскольку} и~\textit{потому что}, было установлено, что 
французский коннектор \textit{car} чаще всего используется для выражения этого ЛСО. Исходя из 
предположения, что, будучи столь частотным средством выражения ЛСО причины во французском языке, 
\textit{car} может быть ФЭФ не только для перечисленных коннекторов русского языка, в~НБД был 
осуществлен поиск таких пар фрагментов, где во французском языке используется \textit{car}, а~в~русском 
коннекторы \textit{ибо}, \textit{поскольку}, \textit{потому что}, напротив, отсутствуют\footnote{Ввиду того, 
что в~текстах, содержащихся в~НБД, омонимия не снята, из результатов выдачи были исключены фрагменты, 
содержащие в~русскоязычной части существительное \textit{автобус} во всех формах, поскольку оно может 
переводиться существительным \textit{car} `туристический автобус', омонимичным коннектору \textit{car}, 
по которому производился поиск в~текстах на французском языке. Данная мера нацелена на сокращение 
числа нерелевантных контекстов. Также из выдачи исключались фрагменты, содержащие частицу 
\textit{ведь}, которая не аннотировалась на первой стадии описанной методики, но которая способна 
выполнять функцию коннектора и~выражать ЛСО причины, особенно в~разговорной речи.}. По этому запросу 
было найдено~444~пары фрагментов параллельных текстов, из которых аннотации ПС были сформированы 
для~381. Несмотря на то что некоторые из найденных контекстов являются нерелевантными для целей 
исследования (в первую очередь речь идет о случаях, когда в~русскоязычной части содержится  
ка\-кой-ли\-бо другой коннектор причины, не попавший в~список исключаемых из выдачи единиц), данный 
вид поиска дает возможность решить поставленную задачу. После того как работа по формированию 
аннотаций для найденных контекстов завершена, можно переходить к~их анализу.

\vspace*{-6pt}

\section{Полученные результаты}

\vspace*{-4pt}

   В~47,2\% сформированных аннотаций (180~случаев) для русскоязычной 
части был проставлен один из признаков кластера \textit{Zero} 
(перечисленных ни-\linebreak же), тогда как для всех направлений перевода, 
представ\-лен\-ных в~НБД, процент таких аннотаций в~несколько раз ниже 
(0,7\%~--- более чем в~67~раз меньше~--- для рассматриваемого на\-прав\-ле\-ния 
перевода <<рус\-ский\;$\to$\;фран\-цуз\-ский>>, см.\ табл.~1). Столь 
представительный объем аннотаций, причем для одного и~того же ЛСО, 
позволил уточнить систему признаков лингвистического аннотирования, 
используемую в~НБД для категоризации контекстов, когда ЛСО является 
имплицитным или выражено косвенно. Вместо одного признака \textit{Zero}, 
который использовался в~предыдущей методике аннотирования во всех 
случаях, когда языковая единица, передающая семантику коннектора, 
отсутствовала, в~новой методике предлагается использовать систему из трех 
признаков, отражающих характеристики описываемых контекстов 
(в~скобках приводятся количественные данные по описываемому 
материалу).
   \begin{enumerate}[1.]
\item \textit{Absent CNT}~--- коннектор, выражающий ЛСО, отсутствует, но 
может быть восстановлен без изменения синтаксической структуры 
фрагмента текста (40,7\%, 155~случаев); см.\ пример в~табл.~6 ниже.
\item \textit{Absent FragmCNT}~--- фрагмент, со\-от\-вет\-ст\-ву\-ющий фрагменту 
с~коннектором, отсутствует (2,3\%, 9~случаев); см.\ пример в~табл.~4, где 
при переводе добавлен выделенный полужирным курсивом фрагмент `так 
как был канун праздников'.


\item \textit{DifferStr FragmCNT}~--- фрагмент, соответст\-ву\-ющий фрагменту с~коннектором, имеет иную, чем в~оригинале, структуру, не допускающую 
использования коннектора (4,2\%, 16~случаев); см.\ пример в~табл.~5, где 
в~переводе фрагмент переведен как `воры легко его [замок] выворотили, так 
как дерево было гнилым'.
\end{enumerate}

\begin{table*}\small %tabl5
\begin{center}
\Caption{Пара фрагментов, где для фрагмента на русском языке проставлен признак 
\textit{DifferStr FragmCNT}}
\vspace*{2ex}

\begin{tabular}{|p{78mm}|p{78mm}|}
\hline
Воры легко выворотили замок из подгнившего дерева вместе с~петлями.\newline 
[А.~Вайнер, Г.~Вайнер. Эра милосердия (1975)]&Les voleurs l'avaient facilement 
arrach$\acute{\mbox{e}}$ \textbf{car} le bois $\acute{\mbox{e}}$tait pourri. \newline
[Trad.\ par J.-P.~Dussaussois et E.~Avrorine, en coll.\ avec J.-G.~Synakiewicz. Revu et 
corrig$\acute{\mbox{e}}$ par B.~Durupt (2005)]\\
\hline
\end{tabular}
\end{center}
\vspace*{-6pt}
%\end{table*}
%\begin{table*}\small %tabl6
\begin{center}
\parbox{400pt}{\Caption{Пример использования эксплицитного показателя ЛСО обоими переводчиками 
для передачи ЛСО, имплицированного в~оригинале}}
\vspace*{2ex}

\begin{tabular}{|l|l|}
\hline
& Ne me demandez plus de chanter,\\
& \textbf{car} je ne serais plus capable de chanter ainsi$\ldots$\\
Не просите меня петь, я не спою уже больше так$\ldots$
& [Trad.\ par A.~Adamov (1959)]\\
\cline{2-2}
[И.\,А.~Гончаров. Обломов (1848--1859)]&Ne me demandez pas de chanter,\\ 
&\textbf{car} je ne saurais plus chanter comme 
\mbox{{\normalsize\ptb{\c{c}}}a}$\ldots$\\
&[Trad.\ par L.~Jurgenson (1988)]\\
\hline
\end{tabular}
\end{center}
\vspace*{-9pt}
\end{table*}
   
   Так как лишь первый признак (\textit{Absent CNT}) отмечает случаи 
имплицитного ЛСО, то аннотации с~признаками \textit{Absent FragmCNT} 
и~\textit{DifferStr \mbox{FragmCNT}} при последующем анализе не 
рассматриваются, поскольку наличие этих признаков дает информацию не об 
аннотируемом контексте как таковом, а~о~переводческих решениях (что не 
входит в~цели исследования).
   
   Случаи косвенного выражения ЛСО причины представляется возможным 
разделить на три группы (подробнее см.~\cite{16-in}):
   \begin{enumerate}[(1)]
\item использование семантически нагруженных знаков препинания, 
а~именно двоеточия и~тире (18,4\%, 70~случаев);
\item использование грамматических конструкций и~лексических средств, не 
являющихся коннекторами, но выражающих семантику причины (16,5\%, 
63~случая);
\item использование коннекторов, для которых выражение ЛСО причины 
является нехарактерным (17,9\%, 68~случаев)\footnote{В~рамках данной статьи 
к~этой же группе относятся и~те случаи, когда в~оригинале используется коннектор, выражающий 
ЛСО, отличное от ЛСО причины, а~также случаи, когда коннектор, способный выражать ЛСО 
причины (например, коннектор \textit{а~то}), выражает ка\-кое-ли\-бо другое отношение.}.
\end{enumerate}

   Отметим, что в~рамках аналогичного, но моноязычного проекта PDTB 
   (Pen Discourse TreeBank)
многие из этих случаев рассматриваются как имплицитное ЛСО, а не его 
косвенное выражение. Авторы руководства по аннотированию 
в~PDTB~3.0~\cite{10-in} отмечают, что при восстановлении 
<<имплицитного коннектора>> разрешается модифицировать исходный 
контекст в~процессе аннотирования для того, чтобы предложение оставалось 
грамматически правильным:
   
   \textit{The Nicaraguan president}, citing (Implicit\;=\;as a~result of) 
\textbf{attacks by the U.S.-backed rebels}, \textit{suspended a~19-month-old 
cease-fire}$\ldots$\footnote{\textit{Президент Никарагуа}, ссылаясь на \textbf{нападения, 
совершенные поддерживаемыми США повстанцами}, \textit{приостановил режим 
прекращения огня, продлившийся 19~месяцев}$\ldots$ (перевод наш~--- А.\,Г., О.\,И.).}
   
   Так, в~приведенном примере~\cite[с.~35]{10-in} курсивом и~полужирным 
выделены связываемые фрагменты текста, причем форма \textit{citing} 
(<<ссылаясь на>>) исключена из их состава. После этого изменения 
и~добавления коннектора \textit{as a~result of} (<<в~результате  
че\-го-ли\-бо>>) авторы относят данный случай к~группе отношения 
причины. В~рамках НБД единицей, выражающей ЛСО, здесь считалась бы 
как раз форма \textit{citing} (грамматическое средство выражения ЛСО), что 
позволило бы не менять контекст в~зависимости от целей исследования, 
а~анализировать фрагменты реальных текстов.
   
   Отдельного рассмотрения заслуживают случаи (выходящие за рамки 
статьи), когда французскому коннектору причины \textit{car} в~русском языке 
соответствуют коннекторы, выражающие какое-либо другое ЛСО. Анализ 
таких соответствий может позволить лингвистам, с~одной стороны, провести 
более четкие границы между различными ЛСО, а~с~другой~--- определить, 
каковы возможные объективные причины разногласий среди исследователей 
относительно оснований выделения и~количества логических и~риторических 
отношений в~рамках различных подходов.
   
   При наличии в~НБД для одного и~того же текста более чем одного 
перевода имплицитное ЛСО причины может иметь эксплицитные показатели 
в~переводах. В~табл.~6 приводится контекст, где ЛСО причины в~русском 
языке устанавливается имплицитно~--- `не просите, потому что я все равно 
не спою',~--- тогда как оба переводчика употребляют французский коннектор 
\textit{car}.
   

   
   Результаты анализа аннотаций говорят о том, что случаи, когда ЛСО 
причины является имплицитным или выражено косвенными средствами, не 
являются исключительными, а встречаются в~текстах систематически. 
Несмотря на неявный характер, ЛСО верно интерпретируется реципиентом, 
что подтверждается данными переводов (в~особенности примерами, 
подобными приведенному в~табл.~6).

\vspace*{-11pt}
   
\section{Заключение}

\vspace*{-2pt}

   Апробация новой методики поиска на примере лишь одного французского 
коннектора демонстрирует целенаправленный характер выявления нового 
лингвистического знания, которое позволит сформировать типологию 
случаев, когда ЛСО выражено косвенно или является имплицитным, 
с~использованием методов контрастивного анализа в~процессе 
аннотирования. Для этого представляется целесообразным распространить 
описанную методику на другие языковые единицы (например, на \textit{parce 
que}, второй по частотности коннектор причины в~НБД) и,~шире, на другие 
ЛСО. Стимулы перевода, относящиеся к~группе косвенных средств выражения ЛСО, также 
могут стать объектом поиска на следующих этапах исследования. Более 
детальный семантический анализ результатов позволит лингвистам 
определить и~возможные причины выражения ЛСО в~тексте без 
использования их прототипических показателей. Таким образом, 
предлагаемая методика, ориентированная на использование НБД, 
предназначена для поиска и~исследования случаев, когда ЛСО в~тексте 
выражено косвенными средствами или является имплицитным, а~также для 
последующего формирования их типологии, и~была разработана в~рамках 
проекта РФФИ, цель которого состоит в~создании методов и~средств 
информатики для целенаправленного формирования новых и~развития 
существующих лингвистических типологий с~использованием параллельных 
текстов~\cite{17-in, 18-in, 19-in, 20-in}.

\vspace*{-6pt}
   
  {\small\frenchspacing
 {%\baselineskip=10.8pt
 \addcontentsline{toc}{section}{References}
 \begin{thebibliography}{99}
 
\vspace*{-3pt}

\bibitem{1-in}
\Au{Зацман И.\,М., Инькова~О.\,Ю., Кружков~М.\,Г., Попкова~Н.\,А.} Представление 
кроссязыковых знаний о~коннекторах в~надкорпусных базах данных~// Информатика и~её 
применения, 2016. Т.~10. Вып.~1. С.~106--118.
\bibitem{2-in}
\Au{Дурново А.\,А., Зацман~И.\,М., Лощилова~Е.\,Ю.} Кросс\-линг\-ви\-сти\-че\-ская база данных 
для аннотирования ло\-ги\-ко-се\-ман\-ти\-че\-ских отношений в~тексте~// Сис\-те\-мы и~средства 
информатики, 2016. Т.~26. №\,4. С.~124--137.
\bibitem{3-in}
\Au{Зализняк Анна~А., Зацман~И.\,М., Инькова~О.\,Ю.} Надкорпусная база данных 
коннекторов: построение сис\-те\-мы терминов~// Информатика и~её применения, 2017. 
Т.~11. Вып.~1. С.~100--108.
\bibitem{4-in}
\Au{Зацман И.\,М., Кружков~М.\,Г.} Надкорпусная база данных коннекторов: развитие 
системы терминов проектирования~// Системы и~средства информатики, 2018. Т.~28. 
№\,4. С.~156--167.
\bibitem{5-in}
\Au{Инькова-Манзотти О.\,Ю.} Коннекторы противопоставления во французском 
и~русском языках (сопоставительное исследование).~--- М.: Информэлектро, 2001. 434~с. 
\bibitem{6-in}
\Au{Rosch Е., Mervis~С.\,В., Gray~W.\,D., Johnson~D.\,M., Boyes-Braem~P.} Basic objects in 
natural categories~// Cognitive Psychol., 1976. Vol.~7. P.~382--439.
\bibitem{7-in}
\Au{Carlson L., Marcu~D.} Discourse tagging reference manual. 2001.  87~p. 
{\sf  ftp://128.9.176.20/isi-pubs/tr-545.pdf}. 

\bibitem{9-in}
PDTB Research Group. The Penn Discourse Treebank~2.0 Annotation Manual.~--- 
Philadelphia, PA, USA: Institute for Research in Cognitive Science, University of 
Pennsylvania, 2008. Technical 
Report IRCS-08-01. {\sf  
https://www.seas.upenn.edu/$\sim$pdtb/PDTBAPI/pdtb-annotation-manual.pdf}.

\bibitem{8-in}
\Au{Ho-Dac L.-M., P$\acute{\mbox{e}}$ry-Woodley~M.-P.} Annotation des structures 
discursives: l'exp$\acute{\mbox{e}}$rience ANNODIS~// 
4e Congr$\grave{\mbox{e}}$s Mondial de Linguistique 
\mbox{Fran{\normalsize\!\ptb{\!\c{c}}}aise}. SHS Web of Conferences~/ Eds.
F.~Neveu, P.~Blumenthal, L.~Hriba, A.~Gerstenberg, J.~Meinschaefer, 
S.~Pr$\acute{\mbox{e}}$vost.~--- Berlin, 2014. Vol.~8. P.~2647--2661.

\bibitem{10-in}
\Au{Webber B., Prasad~R., Lee~A., Joshi~A.} The Penn Discourse Treebank~3.0 Annotation 
Manual, 2019. {\sf https:// catalog.ldc.upenn.edu/docs/LDC2019T05/PDTB3-Annotation-Manual.pdf}.
\bibitem{11-in}
Семантика коннекторов: контрастивное исследование~/ Под науч. ред.
О.\,Ю.~Иньковой.~---  М.: ТОРУС ПРЕСС, 2018. 368~с.
\bibitem{12-in}
Русская грамматика: в~2~т. Том~II. Синтаксис~/ Под ред. Н.\,Ю.~Шведовой.~--- М.: Наука, 
1980. 709~с.
\bibitem{13-in}
\Au{Инькова О.\,Ю.} Аннотирование параллельных текстов: понятие <<дивергентный 
перевод>>~// Компьютерная лингвистика и~интеллектуальные технологии: По мат-лам 
ежегодной Междунар. конф. <<Диалог>>.~--- М.: РГГУ, 2019. Вып.~18(25). С.~227--238.
\bibitem{14-in}
\Au{Гончаров А.\,А., Инькова~О.\,Ю., Кружков~М.\,Г.} Методология аннотирования 
в~надкорпусных базах данных~// Системы и~средства информатики, 2019. Т.~29. №\,2. 
С.~148--160.
\bibitem{15-in}
\Au{Prasad R., Joshi A., Webber~B.} Realization of discourse relation by other means: 
Alternative lexicalizations~// 23rd Conference (International) on Computational 
Linguistics Proceedings.~--- Beijing, China, 2010. Р.~1023--1031.
\bibitem{16-in}
\Au{Гончаров А.\,А., Инькова~О.\,Ю.} Способы выражения причинных отношений 
в~русском языке: опыт анализа с~использованием кросслингвистической надкорпусной 
базы данных~// Русская грамматика: активные процессы в~языке и~речи: Междунар. 
научный симпозиум: Сб. статей, 2019 (в печати).
\bibitem{17-in}
\Au{Зацман И.\,М.} Имплицированные знания: основания и~технологии извлечения~// 
Информатика и~её применения, 2018. Т.~12. Вып.~3. С.~74--82. 
\bibitem{18-in}
\Au{Зацман И.\,М.} Стадии целенаправленного извлечения знаний, имплицированных 
в~параллельных текстах~// Системы и~средства информатики, 2018. Т.~28. №\,3.  
С.~175--188.

\bibitem{20-in}
\Au{Зацман И.\,М.} Целенаправленное развитие систем лингвистических знаний: 
выявление и~заполнение лакун~// Информатика и~её применения, 2019. Т.~13. Вып.~1. 
С.~91--98.
\bibitem{19-in}
\Au{Гончаров А.\,А., Зацман~И.\,М.} Информационные трансформации параллельных 
текстов в~задачах извлечения знаний~// Системы и~средства информатики, 2019. Т.~29. 
№\,1. С.~180--193.

 \end{thebibliography}

 }
 }

\end{multicols}

\vspace*{-9pt}

\hfill{\small\textit{Поступила в~редакцию 30.06.19}}

%\vspace*{8pt}

%\pagebreak

\newpage

\vspace*{-28pt}

%\hrule

%\vspace*{2pt}

%\hrule

%\vspace*{-2pt}

\def\tit{METHODS FOR~IDENTIFICATION OF~IMPLICIT LOGICAL-SEMANTIC RELATIONS IN TEXTS}


\def\titkol{Methods for~identification of~implicit logical-semantic relations in texts}

\def\aut{A.\,A.~Goncharov and O.\,Yu.~Inkova}

\def\autkol{A.\,A.~Goncharov and O.\,Yu.~Inkova}

\titel{\tit}{\aut}{\autkol}{\titkol}

\vspace*{-11pt}


\noindent
Institute of Informatics Problems, Federal Research Center ``Computer Science 
and Control'' of the Russian Academy of Sciences, 44-2~Vavilov Str., Moscow 
119333, Russian Federation

\def\leftfootline{\small{\textbf{\thepage}
\hfill INFORMATIKA I EE PRIMENENIYA~--- INFORMATICS AND
APPLICATIONS\ \ \ 2019\ \ \ volume~13\ \ \ issue\ 3}
}%
 \def\rightfootline{\small{INFORMATIKA I EE PRIMENENIYA~---
INFORMATICS AND APPLICATIONS\ \ \ 2019\ \ \ volume~13\ \ \ issue\ 3
\hfill \textbf{\thepage}}}

\vspace*{3pt}  




   \Abste{The paper presents methods for identification of implicit logical-semantic relations 
(LSR) in parallel texts of the Supracorpora Database (SCDB) of Connectives. The stages of the 
search process are described based on the Russian--French translations: ($i$)~selection of an LSR 
to be analyzed and creation of an array of annotations of Russian connectives considered as 
prototypical means for expressing this LSR; ($ii$)~analysis of the produced array of annotations 
and identification of common equivalents for translating Russian connectives into French; 
($iii$)~utilizing the bilingual search functions of the SCDB with exclusion of Russian connectives 
annotated during the first stage and with specification of the most frequent French language units 
identified during the second stage; ($i\nu$)~annotation of the pairs of fragments of parallel texts 
found as the result of the third stage; and
($\nu$)~analysis of the array annotations produced at the fourth 
stage in order to identify and categorize instances of implicit LSR. The proposed SCDB-based 
search methods make it possible to gather new data on implicit LSRs.}
   
   \KWE{identifying implicit information; connectives; contrastive linguistics; corpus 
linguistics; supracorpora databases; logical-semantic relations}
   
   
   
\DOI{10.14357/19922264190314} 

%\vspace*{-14pt}

\Ack
   \noindent
   The study has been conducted at the Institute of Informatics Problems, Federal Research 
Center ``Computer Science and Control'' of the Russian Academy of Sciences  
with financial aid from the Russian Foundation for Basic Research (Grant No.\,18-07-00192).


%\vspace*{-6pt}

  \begin{multicols}{2}

\renewcommand{\bibname}{\protect\rmfamily References}
%\renewcommand{\bibname}{\large\protect\rm References}

{\small\frenchspacing
 {%\baselineskip=10.8pt
 \addcontentsline{toc}{section}{References}
 \begin{thebibliography}{99}
\bibitem{1-in-1}
\Aue{Zatsman, I.\,M., O.\,Yu.~Inkova, M.\,G.~Kruzhkov, and N.\,A.~Popkova.} 
2016. Predstavlenie kross-yazykovykh znaniy o~konnektorakh v~nadkorpusnykh 
bazakh dannykh [Representation of cross-lingual knowledge about connectors in 
suprocorpora databases]. \textit{Informatika i~ee Primeneniya~--- Inform. Appl.} 
10(1):106--118.
\bibitem{2-in-1}
\Aue{Durnovo, A.\,A., I.\,M.~Zatsman, and E.\,Yu.~Loshchilova.} 2016.  
Krosslingvisticheskaya baza dannykh dlya annotirovaniya logiko-semanticheskikh 
otnosheniy v~tekste [Cross-lingual database for annotating logical-semantic 
relations in the text]. \textit{Sistemy i~Sredstva Informatiki~--- Systems and Means 
of Informatics} 26(4):124--137.
\bibitem{3-in-1}
\Aue{Zaliznyak, Anna~A., I.\,M.~Zatsman, and O.\,Yu.~Inkova.} 2017. 
Nadkorpusnaya baza dannykh konnektorov: postroenie sistemy terminov 
[Supracorpora database on connectives: Term system development]. 
\textit{Informatika i~ee Primeneniya~--- Inform. Appl.} 11(1):100--106.
\bibitem{4-in-1}
\Aue{Zatsman, I.\,M., and M.\,G.~Kruzhkov.} 2018. Nadkorpusnaya baza dannykh 
konnektorov: razvitie sistemy terminov proektirovaniya [Supracorpora database of 
connectives: Design-oriented evolution of the term system]. \textit{Sistemy 
i~Sredstva Informatiki~--- Systems and Means of Informatics} 28(4):156--167.
\bibitem{5-in-1}
\Aue{Inkova-Manzotti, O.\,Yu.} 2001. \textit{Konnektory protivopostavleniya vo 
frantsuzskom i~russkom yazykakh. Sopostavitel'noe issledovanie} [Connectives of 
opposition in French and Russian. Comparative research]. Moscow: 
Informelektro. 434~p. 
\bibitem{6-in-1}
\Aue{Rosch,~Е., С.\,В.~Mervis, W.\,D.~Gray, D.\,M.~Johnson, and  
P.~Boyes-Braem.} 1976. Basic objects in natural categories. \textit{Cognitive 
Psychol.} 7:382--439.
\bibitem{7-in-1}
\Aue{Carlson, L., and D.~Marcu.} 2001. Discourse tagging reference manual. 87~p.
Available at: {\sf ftp://128.9.176.20/isi-pubs/tr-545.pdf}
(accessed July~29, 2019). 

\bibitem{9-in-1}
PDTB Research Group. 2008. The Penn Discourse Treebank~2.0 Annotation 
Manual.  Philadelphia, PA: Institute for Research in 
Cognitive Science, University of Pennsylvania. Technical Report IRCS-08-01.
Available at: {\sf 
https://www.seas.upenn.edu/$\sim$pdtb/PDTBAPI/pdtb-annotation-manual.pdf}.

\bibitem{8-in-1}
\Aue{Ho-Dac, L.-M., and M.-P.~P$\acute{\mbox{e}}$ry-Woodley.} 2014.  
Annotation des structures discursives: l'exp$\acute{\mbox{e}}$rience ANNODIS. 
\textit{4e Congr$\grave{\mbox{e}}$s Mondial de Linguistique 
\mbox{Fran{\normalsize\!\ptb{\!\c{c}}}aise}. SHS Web of Conferences}.  Eds.\ 
F.~Neveu, P.~Blumenthal, L.~Hriba, A.~Gerstenberg, J.~Meinschaefer, 
and S.~Pr$\acute{\mbox{e}}$vost.  Berlin. 8:2647--2661. Available at: {\sf  
https://www.shs-conferences.org/articles/shsconf/pdf/2014/05/shsconf\_ cmlf14\_01286.pdf}.
\bibitem{10-in-1}
\Aue{Webber, B., R.~Prasad, A.~Lee, and A.~Joshi.} 2019. The Penn Discourse 
Treebank 3.0 Annotation Manual. Available at: {\sf 
https://catalog.ldc.upenn.edu/docs/ LDC2019T05/PDTB3-Annotation-Manual.pdf}.
\bibitem{11-in-1}
Inkova, O.\,Yu., ed. 2018. \textit{Semantika konnektorov: kontrastivnoe 
issledovanie} [Semantics of connectives: A~contrastive study]. Moscow: TORUS 
PRESS. 368~p.
\bibitem{12-in-1}
Shvedova, N.\,Yu., ed. 1980. \textit{Russkaya grammatika: v~2~t. T.~II. Sintaksis} 
[Russian Grammar. In 2~vols. Vol.~II. Syntax]. Moscow: Nauka.  709~p.
\bibitem{13-in-1}
\Aue{Inkova, O.\,Yu.} 2019. Annotirovanie parallel'nykh tekstov: ponyatie 
``divergentnyy perevod'' [Annotation of parallel texts: The concept of divergent 
translation]. \textit{Computational Linguistics and 
Intellectual Technologies. Papers from the Annual International Conference 
``Dialogue''} 18(25):227--238.
\bibitem{14-in-1}
\Aue{Goncharov, A.\,A., O.\,Yu.~Inkova, and  M.\,G.~Kruzhkov.} 2019. 
Metodologiya annotirovaniya v~nadkorpusnykh bazakh dannykh [Annotation 
methodology of supracorpora databases]. \textit{Sistemy i~Sredstva  
Informatiki~--- Systems and Means of Informatics} 29(2):148--160.
\bibitem{15-in-1}
\Aue{Prasad R., A.~Joshi, and B.~Webber.} 2010. Realization of discourse relation 
by other means: Alternative lexicalizations. \textit{23rd 
Conference (International) on Computational Linguistics Proceedings}. Beijing, 
China. 1023--1031.
\bibitem{16-in-1}
\Aue{Goncharov, A.\,A., and O.\,Yu.~Inkova.} 2019 (in print.) Sposoby 
vyrazheniya prichinnykh otnosheniy v~russkom yazyke: opyt analiza 
s~ispol'zovaniem krosslingvisticheskoy nadkorpusnoy bazy dannykh [Means of 
expressing causal relations in Russian: Analysis using a~cross-linguistic 
supracorpora database]. \textit{Russkaya grammatika: aktivnye protsessy v~yazyke 
i~rechi. Mezhdun. nauchnyy simpozium. Sbornik statey} [Russian Grammar: 
Active Processes in Language and Discourse: Scientific Symposium (International) 
Proceedings].
\bibitem{17-in-1}
\Aue{Zatsman, I.\,M.} 2018. Implitsirovannye znaniya: osnovaniya i~tekhnologii 
izvlecheniya [Implied knowledge: Foundations and technologies of explication]. 
\textit{Informatika i~ee Primeneniya~--- Inform. Appl.} 12(3):74--82.
\bibitem{18-in-1}
\Aue{Zatsman, I.\,M.} 2018. Stadii tselenapravlennogo iz\-vle\-che\-niya znaniy, 
implitsirovannykh v~parallel'nykh teks\-takh [Stages of goal-oriented discovery of 
knowledge implied in parallel texts]. \textit{Sistemy i~Sredstva Informatiki~---
Systems and Means of Informatics} 28(3):175--188.

\bibitem{20-in-1}
\Aue{Zatsman, I.\,M.} 2019. Tselenapravlennoe razvitie sistem lingvisticheskikh 
znaniy: vyyavlenie i~zapolnenie lakun [Goal-oriented development of linguistic 
knowledge systems: Identifying and filling lacunae]. \textit{Informatika i~ee 
Primeneniya~--- Inform. Appl.} 13(1):91--98.

\bibitem{19-in-1}
\Aue{Goncharov, A.\,A., and I.\,M.~Zatsman.} 2019. In\-for\-ma\-tsi\-on\-nye 
transformatsii parallel'nykh tekstov v~zadachakh izvlecheniya znaniy [Information 
transformations of parallel texts in knowledge extraction]. \textit{Sistemy 
i~Sredstva Informatiki~--- Systems and Means of Informatics} 29(1):180--193.

\end{thebibliography}

 }
 }

\end{multicols}

%\vspace*{-7pt}

\hfill{\small\textit{Received June 30, 2019}}

%\pagebreak

%\vspace*{-22pt}


  \Contr
  
  \noindent
  \textbf{Goncharov Alexander A.} (b.\ 1994)~--- junior scientist, Institute of 
Informatics Problems, Federal Research Center ``Computer Science and Control'' 
of the Russian Academy of Sciences, 44-2~Vavilov Str., Moscow 119333, Russian 
Federation; \mbox{a.gonch48@gmail.com}
  
  \vspace*{3pt}
  
  \noindent
  \textbf{Inkova Olga Yu.} (b.\ 1965)~--- Doctor of Science (PhD) in philology, 
senior scientist, Institute of Informatics Problems, Federal Research Center 
``Computer Science and Control'' of the Russian Academy of Sciences,  
44-2~Vavilov Str., Moscow 119333, Russian Federation; 
\mbox{olyainkova@yandex.ru}

\label{end\stat}

\renewcommand{\bibname}{\protect\rm Литература}  