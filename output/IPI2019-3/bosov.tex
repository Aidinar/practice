\def\stat{bosov}

\def\tit{УПРАВЛЕНИЕ ВЫХОДОМ СТОХАСТИЧЕСКОЙ ДИФФЕРЕНЦИАЛЬНОЙ СИСТЕМЫ 
ПО~КВАДРАТИЧНОМУ~КРИТЕРИЮ.\\
III.~АНАЛИЗ СВОЙСТВ ОПТИМАЛЬНОГО 
УПРАВЛЕНИЯ$^*$}

\def\titkol{Управление выходом стохастической дифференциальной системы 
по квадратичному критерию. III.~Анализ свойств} % оптимального  управления}

\def\aut{А.\,В.~Босов$^1$, А.\,И.~Стефанович$^2$}

\def\autkol{А.\,В.~Босов, А.\,И.~Стефанович}

\titel{\tit}{\aut}{\autkol}{\titkol}

\index{Босов А.\,В.}
\index{Стефанович А.\,И.}
\index{Bosov A.\,V.}
\index{Stefanovich A.\,I.}


{\renewcommand{\thefootnote}{\fnsymbol{footnote}} \footnotetext[1]
{Работа выполнена при частичной поддержке РФФИ (проект 19-07-00187-A).}}


\renewcommand{\thefootnote}{\arabic{footnote}}
\footnotetext[1]{Институт проблем информатики Федерального исследовательского центра <<Информатика 
и~управ\-ле\-ние>> Российской академии наук, \mbox{ABosov@frccsc.ru}}
\footnotetext[2]{Институт проблем информатики Федерального исследовательского центра <<Информатика 
и~управ\-ле\-ние>> Российской академии наук, \mbox{AStefanovich@frccsc.ru}}

%\vspace*{-2pt}


\Abst{Продолжено исследование задачи оптимального управления для 
диффузионного процесса Ито и~линейного управляемого выхода с~квадратичным 
критерием качества. Изучаются свойства оптимального решения, определяемого 
функцией Беллмана вида 
$V_t(y,z)\hm=\alpha_tz^2\hm+\beta_t(y)z\hm+\gamma_t(y)$, коэффициенты 
$\beta_t(y)$ и~$\gamma_t(y)$ которой описываются линейными уравнениями 
в~частных производных параболического типа. Для данных коэффициентов 
определяются альтернативные эквивалентные описания в~форме стохастических 
дифференциальных уравнений и~тео\-ре\-ти\-ко-ве\-ро\-ят\-ност\-но\-го пред\-став\-ле\-ния их 
решений, известного как уравнение А.\,Н.~Колмогорова. Показано, что 
полученное дифференциальное представление эквивалентно интегральной 
формуле  
Фейн\-ма\-на--Ка\-ца. В~перспективе полученное описание коэффициентов и, как 
следствие, решение исходной задачи управления могут использоваться для 
реализации альтернативного численного метода их расчета как результата 
имитационного моделирования решения стохастического дифференциального 
уравнения.} 

\KW{стохастическое дифференциальное уравнение; оптимальное управление; 
функция Беллмана; линейные уравнения параболического типа; уравнение 
А.\,Н.~Колмогорова; формула Фейн\-ма\-на--Каца}

\DOI{10.14357/19922264190307} 
  
%\vspace*{1pt}


\vskip 10pt plus 9pt minus 6pt

\thispagestyle{headings}

\begin{multicols}{2}

\label{st\stat}

\section{Введение}

     В работах~[1, 2] получены оптимальное и~при\-бли\-жен\-ное численные 
решения в~задаче управления линейным выходом стохастической 
дифференциальной системы по квадратичному критерию\linebreak качества, краткая 
выдержка полученного результата приведена в~разд.~2 статьи. 
Оптимизируемая динамическая система описывается двумя уравнениями: 
нелинейным стохастическим дифференциальным уравнением Ито для 
состояния и~линейным уравнением для управляемого выхода, цель 
оптимизации~--- квадратичным функционалом. Аналитическое решение 
определяет функция Беллмана вида 
$$
V_t(y,z)=\alpha_tz^2+\beta_t(y)z+\gamma_t(y)\,, 
$$
коэффициенты $\alpha_t$, $\beta_t(y)$ и~$\gamma_t(y)$ которой пред\-став\-ле\-ны 
решениями определенных дифференциальных уравнений (обыкновенного 
Риккати и~двух в~частных производных линейных параболического типа). 
Приближенное решение осуществляется численным методом расчета 
указанных коэффициентов. При этом применение сеточных методов для 
параболического уравнения осложняется тем обстоятельством, что метод 
динамического программирования дает только начальные и~не дает 
граничных условий для уравнений, определяющих $\beta_t(y)$, 
$\gamma_t(y)$, точнее, отсутствуют основания для их обосно\-ван\-но\-го выбора. 
Из-за этого граничные условия выбираются довольно волюнтаристски, 
обсуждается устойчивость решения в~отношении этих граничных условий. 
Так, при вычислении $\beta_t(y)$ рассматривались и~сравнивались два 
традиционных варианта граничных условий~--- в~задаче Дирихле: 
$$
\beta_t(y)=0\,,
$$
 т.\,е.\ условие поглощения, и~в~задаче Неймана: 
 $$
 \fr{\partial \beta_t(y)}{\partial y}=0\,,
 $$
  т.\,е.\ условие отражения, явные и~неявные схемы 
с~разными вариантами дискретизации.
     
     Получение удовлетворительного результата не сделало менее 
актуальной задачу поиска альтернативного метода для приближенного 
вычисления коэффициентов функции Беллмана в~рассматриваемой задаче, 
которой посвящена данная статья  (здесь и~далее под коэффициентами 
подразумевается пара $\beta_t(y)$, $\gamma_t(y)$, так как эффективных 
вариантов решения уравнения Риккати для~$\alpha_t$ известно 
предостаточно). Поскольку обсуждаемые коэффициенты описываются 
линейными уравнениями в~частных производных второго порядка 
параболического типа, то перспективным представлялось опираться 
в~исследовании этой задачи на известные факты, ассоциирующие такие 
уравнения с~уравнениями стохастическими. Видимо, самая известная 
ассоциация такого рода~--- это уравнение А.\,Н.~Колмогорова~[3], которое 
и~было использовано в~качестве инструмента для реализации альтернативного 
метода приближенного вычисления коэффициентов $\beta_t(y)$ 
и~$\gamma_t(y)$ Точнее, использовалось утверждение, которое определяет 
связь между решением задачи Коши для уравнения А.\,Н.~Колмогорова, 
с~одной стороны, и~тео\-ре\-ти\-ко-ве\-ро\-ят\-ност\-ным представлением ее 
терминального условия, с~другой. Адап\-ти\-ро\-ван\-ная для рассматриваемой 
задачи оптимизации формулировка этого результата приведена в~разд.~3 
статьи, а~ее применение к~уравнениям для $\beta_t(y)$ и~$\gamma_t(y)$~--- 
в~разд.~4.
     
     Связь дифференциальных уравнений в~частных производных 
и~стохастических изучается давно, существенных результатов в~этой области 
получено\linebreak
 множество. Наиболее развита тематика так называемых обратных 
стохастических уравнений, нача\-ло которой положено в~работах~[4, 5], 
некоторые\linebreak итоги имеются в~[6, 7], а~актуальные исследования\linebreak проводятся 
и~в~настоящее время. Надо отметить, что и~приложение этой тематики 
именно к~численному решению параболических уравнений также успешно 
реализовано~[8]. Так что предложенный\linebreak в~статье подход можно считать 
исследованным и~апробированным. Другое дело, что предложенный в~статье 
порядок изложения результата на основе уравнения А.\,Н.~Колмогорова 
позволяет не только получить альтернативную численную процедуру, но 
качественно описать свойства рас\-смат\-ри\-ва\-емых коэффициентов, так как 
полученное эквивалентное описание коэффициентов~--- это стохастические 
дифференциальные уравнения и~соотношение, определяющее связь между 
математическим ожиданием их решения и~искомыми коэффициентами. 
Соответственно, такое представление служит основой для численного 
решения рас\-смат\-ри\-ва\-емых параболических уравнений вероятностными 
методами, т.\,е.\ методом имитационного моделирования решения 
стохастического дифференциального уравнения. Кроме того, полученное 
дифференциальное представление имеет известный интегральный аналог~--- 
формулу Фейн\-ма\-на--Ка\-ца~[9], является частным случаем классического 
варианта этой формулы. Обсуждению данного факта посвящен разд.~5. 
     
\section{Оптимальное управление выходом}

     Аналитическое решение задачи оптимизации, исследуемое далее, 
получено в~\cite{1-bos} и~кратко описывается следующими положениями. 

Рассматривается состояние~$y_t$ стохастической дифференциальной 
системы, описываемое нелинейным стохастическим дифференциальным 
уравнением Ито:
     \begin{equation}
     dy_t=A_t\left( y_t\right) dt+\Sigma_t\left( y_t\right) dv_t\,,\enskip y_0=Y\,,
     \label{e1-bos}
     \end{equation}
где $v_t$~--- стандартный винеровский процесс; 
$Y$~--- случайная величина с~конечным вторым моментом, функции~$A_t$;
$\Sigma_t$ удовлетворяют 
условиям Ито, обес\-пе\-чи\-ва\-ющим существование единственного 
решения~\cite{10-bos}.

     С состоянием~$y_t$ линейно связан выход~$z_t$:
     \begin{equation}
     dz_t=a_ty_t\,dt+b_tz_t\,dt+c_tu_t\,dt+\sigma_t\,dw_t,\ z_0=Z,
     \label{e2-bos}
     \end{equation}
где $w_t$~--- не зависящий от~$v_t$, $Y$ и~$Z$ стандартный винеровский 
процесс; $Z$~--- случайная величина с~конечным вторым моментом;  
$u_t$~--- допустимое управление. Функции~$a_t$, $b_t$, $c_t$ и~$\sigma_t$ 
предполагаются ограниченными, процесс управления~--- допустимым 
неупреждающим~\cite{10-bos}, что обеспечивает существование решения 
уравнения~(\ref{e2-bos}) для любого допустимого управления.
     
     Используется целевой функционал следующего вида:
     \begin{multline}
     J\left( U_0^T\right) =E\!\left\{ 
     \int\limits_0^T \left( S_t \left( s_ty_t-g_tz_t-
h_tu_t\right)^2+{}\right.\right.\\
\left.\left.{}+G_tz_t^2+H_tu_t^2\right)dt+ 
S_T\left( s_T y_T -g_T z_T\right)^2+G_Tz_T^2
     \vphantom{\int\limits_0^T}
     \right\}\!,\\
     U_0^T=\left\{ u_t,\ 0\leq t\leq T\right\}\,,
     \label{e3-bos}
     \end{multline}
где $S_t$, $G_t$ и~$H_t$~--- неотрицательные ограниченные функции.   
     
     Оптимальное управление, до\-став\-ля\-ющее минимум функционалу 
$J\left(U_0^T\right)$, 
\begin{multline*}
u_t^*= -\fr{1}{2} \left( S_th_t^2+ H_t  
\right)^{-1} \left( c_t\left( 2\alpha_t z_t \hm+\beta_t(y_t)\right) +{}\right.\\
\left.{}+ 2S_t\left( s_t 
y_t - g_t z_t\right) h_t\right),
\end{multline*}
решение задачи поиска~$u_t^*$ 
обеспечивается пред\-став\-ле\-ни\-ем функции Беллмана~$V_t(y,z)$ в~виде:
     \begin{equation*}
     V_t(y,z) =\alpha_t z^2+\beta_t(y)z +\gamma_t(y)\,,
    % \label{e4-bos}
\end{equation*}
где коэффициенты $\alpha_t$, $\beta_t(y)$ и~$\gamma_t(y)$ задаются 
сле\-ду\-ющи\-ми дифференциальными уравнениями:

\noindent
\begin{multline}
\fr{\partial \alpha_t}{\partial t} +2\alpha_t\left( b_t-\left(
S_th_t^2+H_t\right)^{-1} c_t S_t 
h_t g_t\right) +{}\\
{}+\left( S_t-\left( S_t h_t^2+H_t\right)^{-1} S_t^2 h_t^2\right) g_t^2 
+G_t-{}\\
{}- \left( S_t h_t^2+H_t\right)^{-1} c_t^2 \alpha_t^2 =0\,,\enskip \alpha_T=  
S_Tg_t^2+G_T\,;
\label{e5-bos}
\end{multline}

\vspace*{-12pt}

\noindent
\begin{multline}
\fr{\partial \beta_t(y)}{\partial 
t}+\fr{1}{2}\,\Sigma_t^2(y)\fr{\partial^2\beta_t(y)}{\partial 
y^2}+A_t(y)\fr{\partial \beta_t(y)}{\partial y}+{}\\
{}+ 2\alpha_t\left( a_t+\left( S_th_t^2+H_t\right)^{-1} c_t S_t h_t 
s_t\right)y+{}\\
{}+\beta_t(y)\left(
b_t-\left( S_th_t^2+H_t\right)^{-1}c_t S_t h_t g_t\right)-{}\\
{}-2\left(S_t-\left( S_t h_t^2+H_t\right)^{-1} S_t^2h_t^2\right) s_t g_t y-{}\\
{}-\left( S_t 
h_t^2+H_t\right)^{-1}c_t^2 \alpha_t\beta_t(y)=0\,,\\ 
\beta_T(y)=-2S_T s_Tg_Ty\,;
\label{e6-bos}
\end{multline}

\vspace*{-12pt}

\noindent
\begin{multline}
\fr{\partial \gamma_t(y)}{\partial t} +\fr{1}{2}\,\Sigma_t^2(y)\fr{\partial^2 
\gamma_t(y)}{\partial y^2}+\sigma_t^2 \alpha_t +A_t(y) \fr{\partial 
\gamma_t(y)}{\partial y}+{}\\
{}+\beta_t(y)\left( a_t +\left( S_th_t^2+H_t\right)^{-1} c_t S_t h_t 
s_t\right)y+{}\\
{}+\left( S_t-\left( S_t h_t^2+H_t\right)^{-1}S_t^2 h_t^2\right) s_t^2y^2-
{}\\
{}- \fr{1}{4}\left( S_t h_t^2+H_t\right)^{-1} c_t^2 \beta_t^2(y)=0\,,\\
\gamma_T(y)=S_T s_T^2 y^2\,.
\label{e7-bos}
\end{multline}

     
     Уравнение~(\ref{e5-bos}) является уравнением Риккати и~имеет 
единственное неотрицательное решение для всех $0\hm\leq t\hm\leq T$, так 
как предполагается $S_t h_t^2\hm+H_t\hm>0$. Уравнения~(\ref{e6-bos}) 
и~(\ref{e7-bos}) представляют собой линейные уравнения в~частных 
производных второго порядка, относятся к~параболическому типу, поскольку 
$\Sigma_t^2(y)\hm>0$. Численное решение этих уравнений традиционными 
сеточными методами~\cite{2-bos} обеспечивает возможность получения 
приближенного решения рассматриваемой задачи оптимизации. Однако 
с~реализацией такого чис\-лен\-но\-го решения возникают проблемы как 
вычислительного характера, так и~алгоритмического, связанные 
с~неограниченной областью допустимых значений аргумента~$y$ и,~как 
следствие, отсутствием граничного условия, необходимого сеточному 
алгоритму.

\section{Уравнение А.\,Н.~Колмогорова}

     Помимо сложностей реализации приближенного решения полученное 
выражение для функции Беллмана представляется целесообразным 
сопроводить разъяснениями в~отношении физического (содержательного) 
смысла коэффициентов $\beta_t(y)$ и~$\gamma_t(y)$, т.\,е.\ ответить на вопрос, 
в~чем смысл уравнений~(\ref{e6-bos}) и~(\ref{e7-bos}), а~также предложить 
эффективный способ их приближенного решения. Отметим, что основную 
ценность составляет, конечно, определение коэффициента~$\beta_t(y)$, 
поскольку именно он требуется для реализации оптимального управления. 
Но и~коэффициент~$\gamma_t(y)$ заслуживает внимания в~целях анализа 
качества управления, динамики изменения целевой функции~$J(U_0^t)$. 
В~качестве такого способа видится использование имитационного 
моделирования для стохастического дифференциального уравнения, 
ассоциированного с~решаемым параболическим уравнением~(\ref{e6-bos}) 
или~(\ref{e7-bos}). Для реализации такой ассоциации будет использован 
следующий результат~\cite{3-bos}.
     
     Рассмотрим процесс $X_\tau\hm\in \mathbb{R}^n$, опи\-сы\-ва\-емый 
стохастической дифференциальной системой 
$$
dX_\tau= m_\tau \left(X_\tau\right)\,dt 
+ \sigma_\tau\left(X_\tau\right)\,dW_\tau,
$$
 где $W_\tau$~--- винеровский процесс 
подходящей размерности. Отметим, что здесь в~постановке требуется 
рассматривать многомерный случай, с~учетом скалярного~$y_t$  
в~(\ref{e1-bos}) можно ограничиться $X_\tau\hm\in \mathbb{R}^2$, что для 
дальнейшего уже несущественно. Кроме того, отметим, что формулируется 
результат в~существенно более частном виде, чем дано в~\cite{3-bos}. 
Результат общего вида не ограничивается случаем непрерывной гауссовской 
меры~$dW_\tau$ и~приводит к~интегродифференциальному уравнению, 
здесь достаточно частного случая дифференциального уравнения. 
Решение~$X_\tau$ будем рассматривать на интервале $t\hm\leq \tau\hm\leq T$ 
с~начальным условием $X_t\hm=X$. Подчеркивая это, обозначим такое 
решение~$X_\tau(t,X)$. Далее рассмотрим его в~правой точке горизонта 
управления~$X_T(t,X)$, интерпретируя как функцию переменных $(t,X)$. 
Пусть~$f(X)$~--- некоторая дважды непрерывно дифференцируемая и~вместе 
со своими частными производными первого и~второго порядка равномерно 
ограниченная функция, $F(t,X)\hm= E\{ f(X_T(t,X))\}$. Тогда 
функция~$F(t,X)$ дважды непрерывно дифференцируема по~$X$, 
дифференцируема по~$t$ и~удовлетворяет уравнению:
     \begin{multline}
\hspace*{-10pt}\fr{\partial F(t,x)}{\partial t}+\sum\limits_{i=1}^n \hspace*{-2pt}m_{t_i} 
\fr{\partial 
F(t,X)}{\partial X_i} +\fr{1}{2}\!\sum\limits^n_{i,j=1} \hspace*{-4pt}\sigma^2_{t_{ij}} \fr{ 
\partial^2 F(t,X)}{\partial X_i\partial X_j}=0,\\
     \lim\limits_{t\to T} F(t,X) =f(X)\,.
     \label{e8-bos}
     \end{multline} 
     
     Уравнение~(\ref{e8-bos}) известно как уравнение А.\,Н.~Колмогорова, 
а~данное утверждение определяет связь между решением задачи Коши 
с~терминальным условием для уравнения с~частными 
производными~(\ref{e8-bos}), с~одной стороны,  
и~тео\-ре\-ти\-ко-ве\-ро\-ят\-ност\-ным пред\-став\-ле\-ни\-ем $F(t,X)\hm= E\{ 
f(X_T(t,X))\}$ этого решения, с~другой. Кроме того, этот результат может 
служить основой для приближенного решения~(\ref{e8-bos}) методом 
имитационного моделирования (моделирование решения~$X_T(t,X)$ 
стохастического дифференциального уравнения и~метод Мон\-те-Кар\-ло для 
оценки~$F(t,X)$).
     
     Уравнение~(\ref{e8-bos}) имеет очевидное сходство 
     с~уравнениями~(\ref{e6-bos}) и~(\ref{e7-bos}). Соответственно, представляется 
обоснованным, используя~(\ref{e8-bos}), прояснить смысл решений~(\ref{e6-bos}) 
и~(\ref{e7-bos}) и~предложить метод их приближенного решения.
     
\section{Стохастическое представление коэффициентов функции 
Беллмана}

     Требуемый результат получается в~итоге несложных манипуляций, 
приводящих уравнения~(\ref{e6-bos}) и~(\ref{e7-bos}) к~виду~(\ref{e8-bos}) 
уравнения А.\,Н.~Колмогорова. Сначала приведем выкладки 
для~$\beta_t(y)$, для чего уравнение~(\ref{e6-bos}) запишем в~виде:
     \begin{multline}
     \fr{\partial\beta_t(y)}{\partial t} +A_t(y)\fr{\partial \beta_t(y)}{\partial y} 
+\fr{1}{2}\,\Sigma_t^2(y)\fr{\partial^2\beta_t(y)}{\partial y^2} +M_t y +{}\\
{}+ N_t 
\beta_t(y)=0\,,\enskip 
\beta_T(y)=-2S_T s_T g_T y\,,
     \label{e9-bos}
     \end{multline}
где 
\begin{align*}
M_t&=2\left(\alpha_t \left( a_t+\left( S_t h_t^2+H_t\right)^{-1} c_t S_t h_t s_t\right)
 -{}\right.\\
&\hspace*{10mm}\left. {}-
\left( S_t -\left( S_t h_t^2+H_t\right)^{-1} S_t^2 h_t^2\right) s_t g_t\right)\,;
\\
N_t&=b_t-\left( S_t h_t^2 +H_t\right)^{-1} c_t S_t h_t g_t -{}\\
&\hspace*{33mm}{}-\left( S_t h_t^2 
+H_t\right)^{-1} c_t^2 \alpha_t\,.
\end{align*}
     
     Введем новую функцию $\beta_t^{(0)}(y)$ следующим равенством:
     \begin{multline}
     \beta_t(y)=\beta_t^{(0)}(y)\exp 
     \left\{ -\!\int\limits_0^t\! N_\tau\,d\tau\right\}={}\\
     {}=
     \beta_t^{(0)} \exp \left\{ -i_t\right\},
          \label{e10-bos}
     \end{multline}
     где
     $$
     i_t=\int\limits_0^t N_\tau\,d\tau\,.
     $$

     
     Замена переменных (корректная, с~учетом сделанных в~рамках 
модели~(\ref{e1-bos})--(\ref{e3-bos}) предположений, гарантирующих 
ограниченность подынтегрального выражения) в~(\ref{e9-bos}) с~учетом 
\begin{align*}
     \fr{\partial i_t}{\partial t}&=N_t\,;
     \\
     \fr{\partial \beta_t(y)}{\partial t}&=\fr{\beta_t^{(0)}(y)}{\partial t} 
     \exp \left\{ 
-i_t\right\}-\beta_t^{(0)}\exp \left\{ -i_t\right\} N_t\,;
    \\
     \fr{\partial \beta_t(y)}{\partial y}&=\fr{\beta_t^{(0)}(y)}{\partial y}\exp 
\left\{-i_t\right\}\,;
     \\
     \fr{\partial^2 \beta_t(y)}{\partial 
y^2}&=\fr{\partial^2\beta_t^{(0)}(y)}{\partial y^2}\exp \left\{ -i_t\right\}
    \end{align*}
     и~дополнительного обозначения 
     $$
     I_t=\exp \left\{ -i_t\right\}
     $$ 
     дает:
     \begin{multline*}
    \hspace*{-8pt} \fr{\partial \beta_t^{(0)}(y)}{\partial t}\,I_t +A_t(y) \fr{\partial 
\beta_t^{(0)}(y)}{\partial y}\,I_t +{}\\
{}+\fr{1}{2}\,\Sigma_t^2(y)\fr{\partial^2 
\beta_t^{(0)}(y)}{\partial y^2}\,I_t +{}\\
{}+M_t y+N_t\beta_t^{(0)}(y) I_t-\beta_t^{(0)} 
(y) I_tN_t=0\,,
     \end{multline*}
откуда с~учетом $I_t^{-1}\hm= \exp \left\{i_i\right\}$ получаем:
\begin{multline}
\fr{\partial \beta_t^{(0)}(y)}{\partial t}+A_t(y)\fr{\partial 
\beta_t^{(0)}(y)}{\partial y} 
+\fr{1}{2}\,\Sigma_t^2(y)\fr{\partial^2\beta_t^{(0)}(y)}{\partial y^2}+{}\\
{}+M_t I_t^{-1} y=0\,,
\enskip
\beta_T^{(0)}(y)=-2S_T s_T g_T I_T^{-1}y\,.
\label{e11-bos}
\end{multline}

     Далее введем еще одну функцию $\beta_t^{(1)}\left(y,y^{(1)}\right)$, 
обозначив:
     \begin{equation}
     \beta_t^{(1)}\left(y, y^{(1)}\right) =\beta_t^{(0)}(y)+y^{(1)}\,.
     \label{e12-bos}
     \end{equation}
     
     Это функция времени и~двух фазовых переменных, <<старой>>~$y$ 
     и~<<но\-вой/до\-пол\-ни\-тель\-ной>>~$y^{(1)}$, частные производные которой 
связаны с~производными~$\beta_t^{(0)}(y)$ следующим образом: 
\begin{itemize}
\item     производная по времени 
     $$
     \fr{\partial\beta_t^{(1)}(y,y^{(1)})}{\partial t} =\fr{\partial 
\beta_t^{(0)}(y)}{\partial t}\,;
     $$
\item     
     градиент
     $$
     \Delta \beta_t^{(1)}=\begin{pmatrix}
     \displaystyle\fr{\partial \beta_t^{(1)}(y,y^{(1)})}{\partial y}\\
     \displaystyle\fr{\partial \beta_t^{(1)} (y,y^{(1)})}{\partial y^1}
     \end{pmatrix}=\begin{pmatrix}
     \displaystyle\fr{\partial \beta_t^{(0)}(y))}{\partial y}\\
     1
     \end{pmatrix}\,;
     $$
\item     
     гессиан 
\begin{multline*}
     \hspace*{-11pt}     \nabla \beta_t^{(1)}= \begin{pmatrix}
     \displaystyle \fr{\partial^2\beta_t^{(1)}\left(y,y^{(1)}\right)}
     {\left(\partial y\right)^2} 
     &
     \displaystyle \fr{\partial^2\beta_t^{(1)}\left(y,y^{(1)}\right)}
     {\partial y \partial 
y^{(1)}}\\[12pt]
     \displaystyle \fr{\partial^2\beta_t^{(1)}\left(y,y^{(1)}\right)}{\partial y^{(1)}  
\partial y} & 
     \displaystyle \fr{\partial^2\beta_t^{(1)}\left(y,y^{(1)}\right)}
     {\partial \left( y^{(1)}\right)^2}
     \end{pmatrix}={}\\
     {}=\begin{pmatrix}
     \displaystyle \fr{\partial^2\beta_t^{(0)}(y)}{(\partial y)^2} &0\\
     0&0
     \end{pmatrix}\,.
   \end{multline*}
   \end{itemize}
     
     С учетом сделанных замечаний уравнение~(\ref{e11-bos}) записывается в~виде:
     \begin{multline}
     \fr{\partial \beta_t^{(1)}\left(y,y^{(1)}\right)}{\partial t}+ \left(
     \begin{matrix}
     A_t(y)\\ M_t I_t^{-1} y
     \end{matrix}\right)^{\mathrm{T}} \Delta \beta_t^{(1)}+{}\\
     {}+\fr{1}{2}\left(
     \begin{matrix}
     \Sigma_t^2(y)\\ 0
     \end{matrix}
     \right)^{\mathrm{T}} \nabla \beta_t^{(1)} \left( 
     \begin{matrix}
     \Sigma_t^2(y)\\ 0\end{matrix}
     \right) =0
     \label{e13-bos}
     \end{multline}
с терминальным условием 
$$
\beta_T^{(1)}\left( y, y^{(1)}\right) = -2S_T s_T gT I_T^{-1} y+y^{(1)}\,.
$$ 

Полученное уравнение~(\ref{e13-bos})~--- это уравнение А.\,Н.~Колмогорова~(\ref{e8-bos}), которое 
порождается случайным процессом 
$$
X_\tau= \begin{pmatrix}
x_\tau \\ y_\tau^{(1)}\end{pmatrix}\,,\enskip t\leq \tau\leq T\,,
$$ 
заданным следующей стохастической дифференциальной сис\-темой:
\begin{equation}
\left.
\begin{array}{rl}
\hspace*{-2mm}dx_{\tau}&=A_\tau\left( x_\tau\right) d\tau +
\Sigma_\tau\left(x_{\tau}\right)dv_{\tau}\,,\\[6pt]
& \hspace*{27mm}t\leq \tau\leq 
T\,,\enskip x_t=y_t\,;\\[6pt]
\hspace*{-2mm}dy_{\tau}^{(1)}&=M_{\tau} I_{\tau}^{-1}x_{\tau} d\tau\,,\ 
y_t^{(1)}=\displaystyle \int\limits_0^t M_{\tau} 
I_{\tau}^{-1} y_{\tau} \,d\tau\,.
\end{array}
\right\}\!
\label{e14-bos}
\end{equation}

Полученная система очевидным образом эквивалентна исходной стохастической системе: ее 
состояние~$y_t$ описывается таким же уравнением, что и~первое уравнение~(\ref{e14-bos}) для 
$x_{\tau}$, где в~записи использован тот же самый винеровский процесс~$v_\tau$, плюс 
представляет некоторый среднеквадратичный (с.к.)\ 
ин\-те\-грал этого состояния (второе уравнение).

     Теперь, полагая в~уравнении~(\ref{e8-bos}) 
     $$
     X=\begin{pmatrix} y\\ 
y^{(1)}\end{pmatrix},
$$
 используя вместо обозначения~$X_T(t,X)$ пару 
$x_T\left(t, y, y^{(1)}\right)$ и~$y_T^{(1)}\left(t,y,y^{(1)}\right)$ и~подставляя 
$$
     F(t,X)=\beta_t^{(1)}\left( y,y^{(1)}\right)\,;
  $$
  
    
  
  \noindent
  \begin{multline*}
    f\left( x_T(t,X)\right)={}\\
    {}=\beta_T^{(1)}\left( x_T\left( t,y,y^{(1)}\right),
y_T^{(1)}\left( t,y, y^{(1)}\right)\right)  ={} \\
{}= -2S_T s_T g_T I_T^{-1} x_T \left( t,y, 
y^{(1)}\right)+ y_T^{(1)}\left(t,y,y^{(1)}\right),
    \end{multline*}
     получаем:
     
     \noindent
     \begin{multline*}
     F(t,X)=E\left\{ f\left( X_T(t,X)\right) \right\} \Rightarrow \beta_t^{(1)}\left( 
y,y^{(1)}\right) ={}\\
\hspace*{-2pt}{}= E\!\left\{\! -2S_T s_T g_T I_T^{-1} y_T\!\left( t,y,y^{(1)}\!\right) + 
y_T^{(1)}\left( t,y,y^{(1)}\right)\!\right\}.\hspace*{-6.56989pt}
     \end{multline*}
     
     Далее заметим, что согласно~(\ref{e14-bos}) величина~$x_T\left( 
t,y,y^{(1)}\right)$ не зависит от~$y^{(1)}$, поэтому далее ее следует 
обозначать~$x_T(t,y)$. Значение~$y_T^{(1)}$ можно представить в~виде: 
     $$
     y_T^{(1)}=y_t^{(1)}+\int\limits_t^T M_\tau I_\tau^{-1} x_\tau\,d\tau\,.
     $$
     
     И далее с~учетом введенных выше обозначений~(\ref{e10-bos}) 
и~(\ref{e12-bos}):
     
     \noindent
     \begin{multline}
     \beta_t(y)=I_t\beta_t^{(0)}={}\\
     \hspace*{-3mm}{}=I_tE\left\{ -2S_T s_T g_T I_T^{-1} x_T(t,y) +
     y_T^{(1)}(t,y,0)\!\right\},\!\!
     \label{e15-bos}
     \end{multline}
     где
     
     \noindent
     \begin{multline*}
     \beta_t^{(0)}(y)=\beta_t^{(1)}\left( y,y^{(1)}\right) -y^{(1)}={}\\
     {}=E\left\{ 
     -2S_T s_T g_T I_T^{-1}x_T(t,y) +y_T^{(1)}\left( t,y,y^{(1)}\right)\right\}-{}\\
     {}-
y^{(1)}=
E\left\{ -2S_T s_T g_T I_T^{-1} x_T(t,y)+y_T^{(1)}\left( 
t,y,0\right)\right\}.
     \end{multline*}
     
     В последних выражениях учитывается начальное условие, с~которым 
решается уравнение для~$y_\tau^{(1)}$ из~(\ref{e14-bos}): для 
переменной~$\beta_t^{(1)}$ интегрирование начинается 
с~величины~$y_t^{(1)}$~--- интеграла, рассчитанного для $0\hm\leq \tau\hm\leq 
t$, для переменной~$\beta_t^{(0)}$ это слагаемое обнуляется.
     
     Полученное выражение~(\ref{e15-bos}) является, во-пер\-вых, 
некоторым объяснением физического смысла коэффициента~$\beta_t(y)$, 
определяющего оптимальное управление в~рассматриваемой задаче, 
а~именно: $\beta_t(y)$, аддитивно входящее в~оптимальное 
управление~$u_t^*$, представляет собой линейную комбинацию всех 
прогнозов $x_\tau(t,y)\hm= E\left\{ y_\tau\vert \mathcal{F}_t^y\right\}$ для 
$t\hm\leq \tau\hm\leq T$, т.\,е.\ состояний от текущего момента~$t$ вплоть до 
горизонта управления~$T$. Здесь $\mathcal{F}_t^y$~--- $\sigma$-ал\-геб\-ра, 
порожденная значениями~$y_\tau$ до момента~$t$ включительно, с~учетом 
марковского свойства~$y_t$ из~(\ref{e1-bos}) формально выражает 
начальное условие $x_t\hm=y_t$ в~системе~(\ref{e14-bos}).

%\pagebreak
     
     Отметим, что аналогичное свойство обсуждается  
в~работе~\cite{11-bos}, где рассмотрена такая же точно\linebreak\vspace*{-12pt}

\pagebreak

\noindent
 задача, но 
в~дискретном времени. Второе обстоятельство в~связи 
с~выражением~(\ref{e15-bos})~--- это возможность решения 
уравнения~(\ref{e6-bos}) для расчета~$\beta_t(y)$ без использования 
каких-либо (например, сеточных) методов решения уравнений в~частных 
производных, а~с~использованием компьютерного имитационного 
моделирования, т.\,е.\ с~заменой расчетов аппроксимаций частных 
производных моделированием пучка траекторий процесса~$x_\tau$, 
описываемого уравнением~(\ref{e14-bos}), или, что то же самое, 
процесса~$y_t$, описываемого уравнением~(\ref{e1-bos}), и~набора 
дополнительных переменных, задающих фи\-гу\-ри\-ру\-ющие в~(\ref{e15-bos}) 
интегралы (с.к.-ин\-те\-грал от~$x_{\tau}$ из~(\ref{e15-bos}), задаваемый 
процессом~$y_{\tau}^{(1)}$, и~римановский интеграл~$I_t$). Формальный 
алгоритм, реализующий эту идею, будет представлен и~исследован позже.
     
     Далее, опираясь на полученный результат, можно записать 
аналогичные выражения и~для коэффициента~$\gamma_t(y)$, т.\,е.\ закрыть 
уже вопрос полностью, вычислив функцию Беллмана.
     
     Итак, рассмотрим уравнение~(\ref{e7-bos}), записав его для удобства 
в~виде:

\noindent
     \begin{multline}
     \fr{\partial \gamma_t(y)}{\partial t}+A_t(y) \fr{\partial \gamma_t(y)} 
{\partial y}+\fr{1}{2}\,\Sigma_t^2(y)\fr{\partial^2 \gamma_t(y)}{\partial y^2} 
+L_t(y) =0,\\
     \gamma_T(y)=S_T s_T^2 y^2\,,
     \label{e16-bos}
     \end{multline}
где 

\noindent
\begin{multline*}
L_t(y)=\beta_t(y)\left(a_t+ \left(S_t h_t^2+H_t\right)^{-1} c_t S_t h_t s_t\right)y+{}\\
{}+ 
\left(S_t-\left( S_th_t^2 +H_t\right)^{-1} S_t^2 h_t^2\right)s_t^2 y^2-{}\\
{}-\fr{1}{4}
\left( S_t h_t^2+H_t\right)^{-1}c_t^2 \beta_t^2(y)\,.
\end{multline*}

\noindent
     Здесь задача немного проще, чем в~(\ref{e9-bos}), так как нет 
слагаемого с~$\gamma_t(y)$, поэтому преобразования ограничатся одним 
шагом с~новой функцией, как в~(\ref{e12-bos}): 
     $\gamma_t^{(1)}\left( y,y^{(2)}\right)\hm= \gamma_t(y)\hm+ y^{(2)}$ 
и~теми же действиями, которые приведут к~уравнению 
А.\,Н.~Колмогорова~(\ref{e8-bos}), порождаемому случайным процессом
     $$
     X_\tau= \begin{pmatrix} x_\tau\\ y_{\tau}^{(2)}\end{pmatrix},\enskip
t\leq \tau\leq T\,,
$$
 заданным стохастической дифференциальной 
сис\-те\-мой:
     \begin{equation}
     \left.
     \begin{array}{rl}
     dx_\tau &= A_\tau\left( x_\tau\right)d\tau +\Sigma_{\tau}\left( 
x_\tau\right)dv_{\tau}\,,\\[6pt]
&\hspace*{22mm}t\leq \tau\leq T\,,\ x_t=y_t\,;\\[6pt]
     dy_\tau^{(2)}&=L_{\tau}\left( x_\tau\right)d\tau\,,\ \displaystyle 
     y_t^{(2)}=\int\limits_0^t L_\tau\left( y_{\tau}\right)d\tau\,,
     \end{array}
     \right\}
     \label{e17-bos}
     \end{equation}
причем процесс~$x_{\tau}$ здесь тот же, что и~в~(\ref{e14-bos}), так 
что~(\ref{e14-bos}) и~(\ref{e17-bos}) можно рассматривать как одну сис\-те\-му 
для переменной 
$$
X_\tau= \left( x_\tau, y_\tau^{(1)}, y_\tau^{(2)}\right)^{\mathrm{T}}.
$$ 
Итогом станет следующее представление для~$\gamma_t(y)$: 
\begin{equation}
\gamma_t(y)=E\left\{ S_T s_T^2 x_T^2(t,y)+y_T^{(2)}(t,y,0)\right\}\,.
\label{e18-bos}
\end{equation}

     Полученное выражение играет ту же роль, что и~(\ref{e15-bos}) для 
коэффициента~$\beta_t(y)$. Отметим, что согласно~(\ref{e18-bos}) 
прогнозированию подлежит еще\linebreak и~второй момент процесса~$x_\tau$, 
описываемого уравнением~(\ref{e14-bos}), или процесса~$y_t$. Кроме того, 
обратим внимание на возможность вычисления~$\gamma_t(y)$ вмес\-те 
с~$\beta_t(y)$ по тому же пучку траекторий (с~дополнительной 
переменной~$y_\tau^{(2)}$, пред\-став\-ля\-ющей с.к.-ин\-те\-грал 
от~$x_\tau$) состояния~$y_t$. Это замечание завершает исследование 
оптимального управления, так как наряду с~возможностью вычисления 
самого управления имеется и~возможность определения его качества:

\vspace*{-2pt}

\noindent
     \begin{multline*}
     \min\limits_{U_0^T} J\left( U_0^T\right) =J\left( \left( 
U^*\right)_0^T\right) = E\left\{ V_0\left( Y,Z\right)\right\}={}\\
{}=
     E\left\{ \alpha_0Z^2+\beta_0(Y)Z+\gamma_0(y)\right\}\,,
    % \label{e19-bos}
     \end{multline*}
     
     \vspace*{-2pt}
     
     \noindent
которое после решения уравнений~(\ref{e5-bos})--(\ref{e7-bos}) методом 
имитационного моделирования естественным будет определить усреднением 
по уже имеющемуся пучку.

\vspace*{-6pt}

\section{Представление Фейнмана--Каца коэффициентов функции 
Беллмана}

\vspace*{-3pt}

     В этом разделе приводится дополнение к~основным полученным 
результатам~--- соотношениям~(\ref{e15-bos}) и~(\ref{e18-bos}), а~именно: 
их несложные преобразования приведут к~формуле  
Фейн\-ма\-на--Ка\-ца~\cite{9-bos}, которая представляет самый известный 
результат ассоциации параболического дифференциального уравнения 
в~частных производных (линейного неоднородного второго порядка) со 
стохастической системой. С~этой целью заметим, во-пер\-вых, что, 
доопределив случайный процесс~$x_\tau$ из~(\ref{e14-bos})  
или~(\ref{e17-bos}) равенством $x_\tau\hm=y_\tau$ на интервале $0\hm\leq 
\tau\hm\leq t$, получим стохастически эквивалентные процессы~$y_t$ 
и~$x_t$ на интервале $0\hm\leq t\hm\leq T$. Подчеркивая это, в~(\ref{e14-bos}) 
и~(\ref{e17-bos}) в~диффузионном слагаемом использовано то же 
обозначение для винеровского процесса~$v_t$, что и~в~(\ref{e1-bos}), хотя 
они должны быть определены на разных пространствах. Это позволяет 
использовать в~(\ref{e15-bos}) и~(\ref{e18-bos}) вместо~$x_T(t,y)$ 
переменную~$y_T(t,y)$ и~далее в~выражениях использовать только~$y_T$, 
заменяя начальное условие $x_\tau\hm=y_t$ на условное математическое 
ожидание относительно~$\mathcal{F}_t^y$. Во-вто\-рых, вернем к~исходным 
обозначениям следующие элементы, фигурирующие в~(\ref{e15-bos}) 
и~(\ref{e18-bos}):
     \begin{align*}
     y_T^{(1)}(t,y,0)&=\int\limits_t^T M_\tau I_\tau^{-1} y_\tau\,d\tau\,;\\ 
y_T^{(2)}(t,y,0)&=\int\limits^T_t L_\tau\left( y_\tau\right)\,d\tau\,;\\
     I_t I_\tau^{-1}&=\exp \left\{ -\int\limits_0^t N_s \,ds\right\} \exp  \left\{ 
\int\limits_0^\tau N_s\,ds\right\} ={}\\
&\hspace*{28mm}{}=\exp \left\{ \int\limits_t^\tau N_s\,ds\right\}\,;\\
     I_t I_T^{-1}&=\exp\left\{ -\!\int\limits_0^t N_\tau\,d\tau\right\}\exp \left\{ 
\int\limits_0^T N_\tau\,d\tau\right\} ={}\hspace*{-16pt}\\
&\hspace*{28mm} {}=\exp \left\{ \int\limits_t^T 
N_\tau\,d\tau\right\}\,.
     \end{align*}
     
     Подставляя теперь все вместе в~(\ref{e15-bos}) и~(\ref{e18-bos}), 
получаем:
     \begin{equation}
     \left.
     \begin{array}{l}
     \hspace*{-2.8mm}\displaystyle \beta_t(y)=E\left\{ -2S_T s_T g_T \exp \left\{ \int\limits_t^T 
N_\tau\,d\tau\right\} y_T +{}\right.\\
\displaystyle\hspace*{10mm}\left.{}+\int\limits_t^T \exp \left\{ \int\limits_t^\tau 
N_s\,ds\right\} M_\tau y_\tau \,d\tau\vert \mathcal{F}_t^y\right\}\,;\\[6pt]
      \hspace*{-2.8mm}    \displaystyle \gamma_t(y)=E\left\{ S_T s_T^2 y_T^2+\int\limits_t^T 
L_\tau(y_\tau)\,d\tau \vert \mathcal{F}_t^y\right\}\,.
     \end{array}\!
     \right\}\!\!
     \label{e20-bos}
     \end{equation}
     
     Каждое из равенств~(\ref{e20-bos})~--- это формула  
Фейн\-ма\-на--Ка\-ца~\cite{9-bos}, записанная с~учетом конкретного вида 
слагаемых и~терминальных условий в~(\ref{e9-bos}) и~(\ref{e16-bos}) 
соответственно. Практическую же ценность представляет не то, в~какой 
форме записан результат~(\ref{e20-bos}) или~(\ref{e15-bos}), (\ref{e18-bos}), 
а~дифференциальная система~(\ref{e14-bos}), (\ref{e17-bos}), служащая 
основой для приближенного расчета коэффициентов~$\beta_t(y)$ 
и~$\gamma_t(y)$. 

\section{Заключение}

     В статье продолжено исследование задачи оптимизации линейного 
выхода нелинейной дифференциальной системы по квадратичному 
критерию,\linebreak
 начатое в~\cite{1-bos}. Свойства оптимального решения 
формируются коэффициентами~$\alpha_t$, $\beta_t(y)$ и~$\gamma_t(y)$, 
определенными выражением для функции Беллмана $V_t (y,z)\hm= \alpha_t 
z^2\hm+  \beta_t(y)z\hm+ \gamma_t(y)$. И~если коэффициент~$\alpha_t$, 
описываемый уравнением Риккати,\linebreak не предоставляет пространства для 
исследования, то коэффициенты~$\beta_t(y)$ и~$\gamma_t(y)$, опи\-сы\-ва\-емые 
схожими по структуре линейными неоднородными уравнениями в~част\-ных 
производных второго порядка параболического типа, обнаруживают 
любопытные свойства. Выражением этих свойств является формула  
Фейн\-ма\-на--Ка\-ца~(\ref{e20-bos}), показывающая связь исследуемых 
коэффициентов и~моментных характеристик некоторой дифференциальной 
сис\-те\-мы. Сама полученная сис\-те\-ма~(\ref{e14-bos}), (\ref{e17-bos}) со\-став\-ля\-ет 
основной прикладной результат работы, так как позволяет предложить 
альтернативный (по отношению в~традиционному сеточному методу, 
использованному для расчетов в~\cite{2-bos}) метод приближенного решения 
задачи оптимизации в~исходной модели~(\ref{e1-bos})--(\ref{e3-bos}). 
Формальное изложение этого метода и~его практическая апробация 
со\-став\-ля\-ют дальнейшую задачу исследования, планируемую к~последующей 
пуб\-ли\-ка\-ции.
     
{\small\frenchspacing
 {%\baselineskip=10.8pt
 \addcontentsline{toc}{section}{References}
 \begin{thebibliography}{99}
\bibitem{1-bos}
\Au{Босов А.\,В., Стефанович~А.\,И.} Управление выходом 
стохастической дифференциальной системы по квад\-ра\-тич\-но\-му критерию. 
I.~Оптимальное решение методом динамического программирования~// 
Информатика и~её применения, 2018. Т.~12. Вып.~3. С.~99--106.  doi: 
10.14357/19922264180314.
\bibitem{2-bos}
\Au{Босов А.\,В., Стефанович~А.\,И.} Управление выходом 
стохастической дифференциальной системы по квад\-ра\-тич\-но\-му критерию. 
II.~Численное решение уравнений динамического программирования~// 
Информатика и~её применения, 2019. Т.~13. Вып.~1. С.~9--15. doi: 
10.14357/19922264190102.
\bibitem{3-bos}
\Au{Гихман И.\,И., Скороход~А.\,В.} Теория случайных процессов. 
Т.~III.~--- М.: Наука, 1975. 496~с.
\bibitem{4-bos}
\Au{Bismut J.-M.} Conjugate convex functions in optimal stochastic control~// 
J.~Math. Anal. Appl., 1973. Vol.~44.  No.\,2. P.~384--404. doi: 
10.1016/0022-247X(73)90066-8.
\bibitem{5-bos}
\Au{Pardoux E., Peng~S.\,G.} Adapted solution of a backward stochastic 
differential equation~// Syst. Control Lett., 1990. Vol.~14. No.\,1.  
P.~55--61. doi: 10.1016/0167-6911(90)90082-6.
\bibitem{6-bos}
\Au{Delong L.} Backward stochastic differential equations with jumps and their 
actuarial and financial applications.~--- New York, NY, USA:  
Springer-Verlag, 2013. 288~p. doi: 10.1007/978-1-4471-5331-3.
\bibitem{7-bos}
\Au{Touzi N.} Optimal stochastic control, stochastic target problems, and 
backward SDEs.~--- New York, NY, USA: Springer-Verlag, 2013. 214~p. doi: 
10.1007/978-1-4614-4286-8.
\bibitem{8-bos}
\Au{Fahim A., Touzi~N., Warin~X.} A~probabilistic numerical method for fully 
nonlinear parabolic pdes~// Ann. Appl. Probab., 2011. Vol.~21. No.\,4. 
P.~1322--1364. doi: 10.1214/10-AAP723.
\bibitem{9-bos}
\Au{{\ptb{\O}}\,\,ksendal B.} Stochastic Differential equations. An introduction 
with applications.~--- New York, NY, USA: Springer-Verlag, 2003. 379~p. 
doi: 10.1007/978-3-642-14394-6.
\bibitem{10-bos}
\Au{Флеминг У., Ришел~Р.} Оптимальное управление 
детерминированными и~стохастическими системами~/ Пер. с~англ.~--- М.: 
Мир, 1978. 316~с. (\Au{Fleming~W.\,H., Rishel~R.\,W.} Deterministic and 
stochastic optimal control.~--- New York, NY, USA: Springer-Verlag, 1975. 
222~p.)
\bibitem{11-bos}
\Au{Босов А.\,В.} Управление линейным выходом дискретной 
стохастической системы по квадратичному критерию~// Изв. РАН. Теория 
и~системы управления, 2016. №\,3. С.~19--35. doi: 
10.1134/S1064230716030060.
 \end{thebibliography}

 }
 }

\end{multicols}

\vspace*{-6pt}

\hfill{\small\textit{Поступила в~редакцию 21.02.19}}

\vspace*{8pt}

%\pagebreak

%\newpage

%\vspace*{-28pt}

\hrule

\vspace*{2pt}

\hrule

%\vspace*{-2pt}

\def\tit{STOCHASTIC DIFFERENTIAL SYSTEM OUTPUT CONTROL 
BY~THE~QUADRATIC CRITERION.\\ III.~OPTIMAL CONTROL 
PROPERTIES ANALYSIS}


\def\titkol{Stochastic differential system output control 
by~the~quadratic criterion. III.~Optimal control 
properties analysis}

\def\aut{A.\,V.~Bosov and A.\,I.~Stefanovich}

\def\autkol{A.\,V.~Bosov and A.\,I.~Stefanovich}

\titel{\tit}{\aut}{\autkol}{\titkol}

\vspace*{-11pt}


\noindent
Institute of Informatics Problems, Federal Research Center ``Computer Science and Control'' of the 
Russian Academy of Sciences, 44-2 Vavilov Str., Moscow 119333, Russian Federation 

\def\leftfootline{\small{\textbf{\thepage}
\hfill INFORMATIKA I EE PRIMENENIYA~--- INFORMATICS AND
APPLICATIONS\ \ \ 2019\ \ \ volume~13\ \ \ issue\ 3}
}%
 \def\rightfootline{\small{INFORMATIKA I EE PRIMENENIYA~---
INFORMATICS AND APPLICATIONS\ \ \ 2019\ \ \ volume~13\ \ \ issue\ 3
\hfill \textbf{\thepage}}}

\vspace*{3pt}   
     

\Abste{The investigation of the optimal control problem for the Ito diffusion process 
and linear controlled output with a quadratic quality criterion is continued. The 
properties of the optimal solution defined by the Bellman function of the form 
$V_t(y,z)=\alpha_tz^2+\beta_t(y)z+\gamma_t(y)$, whose coefficients $\beta_t(y)$ 
and~$\gamma_t(y)$ are described by linear parabolic equations, are studied. For these 
coefficients, alternative equivalent descriptions are defined in the form of stochastic 
differential equations and a theoretical-to-probabilistic representation of their solutions, 
known as the Kolmogorov equation. It is shown that the obtained differential 
representation is equivalent to the Feynman--Kac integral formula. In the future, the 
obtained description of the coefficients and, as a result, the solutions of the original 
control problem can be used to implement an alternative numerical method for 
calculating them as a result of computer simulation of the solution of a stochastic 
differential equation.}

\KWE{stochastic differential equation; optimal control; Bellman function; linear 
differential equations of parabolic type; Kolmogorov equation; Feynman--Kac formula}



\DOI{10.14357/19922264190307} 

%\vspace*{-14pt}

\Ack
\noindent
This work was partially supported by the Russian Foundation for
Basic Research (grant  19-07-00187-A).


%\vspace*{-6pt}

  \begin{multicols}{2}

\renewcommand{\bibname}{\protect\rmfamily References}
%\renewcommand{\bibname}{\large\protect\rm References}

{\small\frenchspacing
 {%\baselineskip=10.8pt
 \addcontentsline{toc}{section}{References}
 \begin{thebibliography}{99}
\bibitem{1-bos-1}
\Aue{Bosov, A.\,V., and A.\,I.~Stefanovich.} 2018. Upravlenie vykhodom 
stokhasticheskoy differentsial'noy sistemy po kvadratichnomu kriteriyu 
[Stochastic differential system output control by the quadratic criterion. 
I.~Dynamic programming optimal solution]. \textit{Informatika i~ee 
Primeneniya~--- Inform. Appl.} 12(3):99--106.
doi: 
10.14357/19922264180314.
\bibitem{2-bos-1}
\Aue{Bosov, A.\,V., and A.\,I.~Stefanovich.} 2019. Upravlenie vykhodom 
stokhasticheskoy differentsial'noy sistemy po kvadratichnomu kriteriyu 
[Stochastic differential system output control by the quadratic criterion. 
II.~Dynamic programming equations numerical solution]. \textit{Informatika 
i~ee Primeneniya~--- Inform. Appl.} 13(1):9-15.
doi: 
10.14357/19922264190102.
\bibitem{3-bos-1}
\Aue{Gihman, I.\,I., and A.\,V.~Skorohod.} 2012. \textit{The theory of 
stochastic processes~III}. New York, NY: Springer-Verlag. 388~p.
\bibitem{4-bos-1}
\Aue{Bismut, J.-M.} 1973. Conjugate convex functions in optimal stochastic 
control. \textit{J.~Math. Anal. Appl.} 44(2):384--404.
doi: 
10.1016/0022-247X(73)90066-8.
\bibitem{5-bos-1}
\Aue{Pardoux, E., and S.\,G.~Peng}. 1990. Adapted solution of a~backward 
stochastic differential equation. \textit{Syst. Control Lett.} 14(1):55--61.
doi: 10.1016/0167-6911(90)90082-6.
\bibitem{6-bos-1}
\Aue{Delong, L.} 2013. \textit{Backward stochastic differential equations with 
jumps and their actuarial and financial applications.} New York, NY: 
Springer-Verlag. 288~p. doi: 10.1007/978-1-4471-5331-3.
\bibitem{7-bos-1}
\Aue{Touzi, N.} 2013. \textit{Optimal stochastic control, stochastic target 
problems, and backward SDEs.} New York, NY: Springer-Verlag. 214~p.
doi: 
10.1007/978-1-4614-4286-8.
\bibitem{8-bos-1}
\Aue{Fahim, A., N.~Touzi, and X.~Warin.} 2011. A~probabilistic numerical 
method for fully nonlinear parabolic pdes. \textit{Ann. Appl. Probab.} 
21(4):1322--1364. doi: 10.1214/10-AAP723.

\columnbreak


\bibitem{9-bos-1}
\Aue{{\ptb{\O}}ksendal, B.} 2003. \textit{Stochastic differential equations. An 
introduction with applications.} New York, NY: Springer-Verlag. 379~p.
doi: 10.1007/978-3-642-14394-6.
\bibitem{10-bos-1}
\Aue{Fleming, W.\,H., and R.\,W.~Rishel.} 1975. \textit{Deterministic and 
stochastic optimal control.} New York, NY: Springer-Verlag. 222~p.
\bibitem{11-bos-1}
\Aue{Bosov, A.\,V.} 2016. Discrete stochastic system linear output control 
with respect to a~quadratic criterion. \textit{J.~Comput. Syst. Sc. 
Int.} 55(3):349--364. 
\end{thebibliography}

 }
 }

\end{multicols}

%\vspace*{-7pt}

\hfill{\small\textit{Received February 21, 2019}}

%\pagebreak

%\vspace*{-22pt}
     

\Contr

\noindent
\textbf{Bosov Alexey V.} (b.\ 1969)~--- Doctor of Science in technology, principal 
scientist, Institute of Informatics Problems, Federal Research Center ``Computer 
Science and Control'' of the Russian Academy of Sciences, 44-2~Vavilov Str., 
Moscow 119333, Russian Federation; \mbox{ABosov@frccsc.ru}


\vspace*{3pt}

\noindent
\textbf{Stefanovich Alexey I.} (b.\ 1983)~--- junior scientist, Institute of 
Informatics Problems, Federal Research Center ``Computer Science and Control'' 
of the Russian Academy of Sciences, 44-2~Vavilov Str., Moscow 119333, Russian 
Federation; \mbox{AStefanovich@frccsc.ru}
\label{end\stat}

\renewcommand{\bibname}{\protect\rm Литература}  
      