\def\stat{zatsman}

\def\tit{ИНТЕРФЕЙСЫ ТРЕТЬЕГО ПОРЯДКА В~ИНФОРМАТИКЕ}

\def\titkol{Интерфейсы третьего порядка в~информатике}

\def\aut{И.\,М.~Зацман$^1$}

\def\autkol{И.\,М.~Зацман}

\titel{\tit}{\aut}{\autkol}{\titkol}

\index{Зацман И.\,М.}
\index{Zatsman I.\,M.}


%{\renewcommand{\thefootnote}{\fnsymbol{footnote}} \footnotetext[1]
%{Работа частично поддержана РФФИ (проект 18-07-00274).}}


\renewcommand{\thefootnote}{\arabic{footnote}}
\footnotetext[1]{Институт проблем информатики Федерального исследовательского центра 
<<Информатика и~управ\-ле\-ние>> Российской академии наук, 
\mbox{izatsman@yandex.ru}}

\vspace*{-6pt}


    
    
    
    
   \Abst{В Европейской стратегии <<Информатика для всех>>, объявленной в~Брюсселе 
в~марте 2018~г., различаются две ступени преподавания информатики в~системе среднего 
и~высшего профессионального образования. Вторая ступень, ориентированная на изучение 
информационных трансформаций в~искусственных, живых и~социальных системах, 
подразумевает сначала выбор для преподавания некоторой парадигмы информатики, а~затем и~ее развитие. Необходимость развития обусловлена двумя причинами: во-пер\-вых, 
существенным расширением сферы применения информационных технологий (ИТ), 
рассматриваемых в~учебном процессе, во-вто\-рых, интеграцией методов и~средств 
информатики в~учебные программы по другим областям знания, что расширяет спектр 
изучаемых информационных трансформаций. В~отсутствие доминирующей парадигмы 
информатики и~при наличии нескольких ее вариантов вопрос ее выбора как отправной точки 
развития является дискуссионным. В~обзоре ``Informatics Education in Europe'', 
опубликованном в~2017~г.\ и~предваряющем разработку Европейской стратегии 
<<Информатика для всех>>, перечислены три варианта парадигмы, включая 
позиционирование информатики как самостоятельной отрасли знания, предложенное 
Деннингом и~Розенблюмом в~2009~г. Подробное описание этого варианта под названием 
полиадического компьютинга дано Розенблюмом в~его монографии 2013~г. Цель статьи 
состоит в~определении нового понятия <<интерфейс 3-го порядка>> на основе деления на 
среды предметной области информатики как полиадического компьютинга. Актуальность 
введения этого понятия иллюстрируется на примере процессов управления 
роботизированной рукой сигналами головного мозга.}
   
   \KW{интерфейсы 3-го порядка; полиадический компьютинг; среды предметной области 
информатики; информационные трансформации}

\DOI{10.14357/19922264190312} 
  
\vspace*{-3pt}


\vskip 10pt plus 9pt minus 6pt

\thispagestyle{headings}

\begin{multicols}{2}

\label{st\stat}
   
\section{Введение}
    
  Апрельский номер журнала \textit{Communications of the ACM} за 2019~г.\ 
включает специальный раздел <<Информатика в~Европе>>, со\-сто\-ящий из 
статей, посвященных актуальным темам в~сфере информатики 
и~вычислительной техники~[1]. Одна из статей посвящена 
вопросам преподавания информатики~[2]. В~ней анализируется <<План 
действий в~об\-ласти циф\-ро\-во\-го образования>>, одобренный Еврокомиссией 
в~2018~г.~[3], а~так\-же отчет о~со\-сто\-янии преподавания дисциплин 
информатики в~Европе, вклю\-чая РФ~[4].
  
  Авторы отчета предложили ряд рекомендаций, которые послужили бы 
улучшению ситуации с~изуче\-ни\-ем дисциплин информатики в~Европейском 
регионе, на основе которых был подготовлен доклад ``Informatics for All: The 
strategy''. Основное внимание в~докладе уделяется преподаванию 
информатики в~системе среднего и~высшего профессионального образования, 
которая рассматривается как фундаментальная наука со своими собственными 
научными методами~[5].
  
  Во многих отношениях основные идеи этого доклада совпадают 
с~инициативой президента Обамы~[6]. Важнейшим элементом Европейского 
доклада ``Informatics for all: The strategy'', который отличает его от этой 
инициативы, является двухступенчатая стратегия преподавания информатики: 
(1)~информатика как самостоятельный предмет; (2)~интеграция методов 
и~средств информатики в~предметные области других наук~\cite{2-zat}.
  
  Для данной статьи наибольший интерес представляет вторая ступень. 
Реализация планов преподавания информатики на второй ступени, скорее 
всего, потребует существенного расширения ее предметной области, так как 
появляется необходимость рассматривать информационные трансформации не 
только в~искусственных, но и~в~живых и~социальных системах. Такой спектр 
информационных трансформаций будет охватывать объекты разных сред и, что 
важно отметить, разной природы. Сегодня с~точки зрения теоретического 
описания уже существующих ИТ необходимо 
различать как минимум пять сред разной природы: нейросреду (сигналы 
активности мозга в~нейрокоммуникационных технологиях); ментальную 
(концепты в~технологиях пред\-став\-ле\-ния знаний человека); 
социоинформационную (тексты на естественных языках, таблицы, рисунки, 
растровые и~векторные изображения и~т.\,д.\ в~технологиях их кодирования); 
ДНК (пред\-став\-ле\-ние информации и~данных с~по\-мощью синтезированных 
цепочек ДНК); циф\-ро\-вую (компьютерные коды). При этом тео\-ре\-ти\-че\-ское 
описание технологий и~интерфейсов будущих поколений может потребовать 
включения в~рас\-смот\-ре\-ние и~сред другой природы, отличной от пяти 
перечисленных.
  
  Основная цель статьи состоит в~описании границ между средами предметной 
области информатики как полиадического компьютинга, концепция которого 
предполагает расширение ее предметной области на информационные 
трансформации в~живых и~социальных системах. Предлагаемое описание сред 
и~границ дает возможность определить понятие <<интерфейс 3-го порядка>> 
и~ответить на вопрос: <<Зачем оно нужно в~информатике?>> Актуальность 
введения этого понятия иллюстрируется на примере процессов управ\-ле\-ния 
роботизированной рукой сигналами головного мозга.

\vspace*{-6pt}
  
\section{О предметной области информатики}

\vspace*{-4pt}
    
  На первых этапах становления информатики объектами ее исследований 
были в~основном компьютеры и~окружающие их явления. Например, в~1967~г.\ 
А.~Ньюэлл, А.\,Дж.~Перлис и~Х.\,А.~Саймон писали: <<Объектам и~явлениям 
соответствуют те науки, которые их изучают. Появились компьютеры. 
Следовательно, назначение информатики~--- это изучение 
компьютеров>>~\cite{7-zat}.
  
  В 1963~г.\ Сол Горн предлагал более широкий взгляд на предметную область 
новой развивающейся науки~\cite{8-zat}. Однако на первых этапах ее 
ста\-нов\-ле\-ния основными объектами предметной области были компьютерные 
коды, их генерация и~преобразования в~цифровой среде, а также явления, 
окружающие компьютеры.
  
  В конце прошлого столетия изучение информационных трансформаций 
в~живых системах стали также относить к~предметной области информатики 
вместе с~генерацией и~преобразованиями компьютерных кодов в~цифровой 
среде. Использование слова \textit{computing} вместо двух слов \textit{computer 
science} стало подчеркивать более широкую предметную область 
информатики. <<Первоначальное определение информатики~--- изучение 
явлений, окружающих компьютеры,~--- сейчас устарело. Она изучает 
и~естественные, и~искусственные информационные процессы. Информатика 
включает компьютерную науку, разработку программного обеспечения, 
ИТ, информационную науку и~информационные 
системы>>~\cite{9-zat}.
  
  Такое расширение предметной области потребовало пересмотра понятия 
\textit{computation}. <<Нам придется пересмотреть понятие \textit{computation}, которое 
должно быть в~состоянии учитывать также и~обработку информации, 
происходящую в~природе$\ldots$ На самом деле исследования 
информационных трансформаций в~живых системах уже привели к~пересмотру 
ряда парадигм, лежащих в~основе традиционного понятия 
\textit{computation}>>~\cite{10-zat}.
  
  Другое направление расширения предметной области информатики было 
связано с~исследованиями информационных процессов в~социальных системах. 
Проиллюстрируем это направление цитатой из статьи Марка Снира:
 % 
  <<Быстрое развитие информационных технологий стимулирует 
периодический пересмотр теоретического фундамента их разработки и~сейчас 
настало время провести очередной пересмотр$\ldots$ Раньше, когда 
с~компьютерами имел дело ограниченный круг лиц, можно было рассчитывать 
на то, что те, кто профессионально\linebreak
 общаются с~компьютерами, могут к~ним 
адаптироваться. В~наши дни, когда значительно больше\linebreak людей 
взаимодействуют с~цифровыми устройствами  
и~ин\-фор\-ма\-ци\-он\-но-ком\-пью\-тер\-ны\-ми системами, информационные 
технологии тесно вплетены во многие когнитивные и~социальные процессы. 
В~этих условиях нельзя игнорировать эти процессы при их создании. Кроме 
того, подобное вплетение является интересным объектом для 
исследований>>~\cite{11-zat}.

  \begin{figure*}[b] %fig1
  \vspace*{1pt}
    \begin{center}  
  \mbox{%
 \epsfxsize=136.145mm 
 \epsfbox{zac-1.eps}
 }
\end{center}
\vspace*{-9pt}
  \Caption{Пять сред разной природы в~предметной области тетрадического компьютинга}
  \end{figure*}
  
  
  Один из подходов к~пересмотру теоретического фундамента разработки 
ИТ в~2013~г.\ был предложен Полом 
Розенблюмом~\cite{12-zat}. В~рамках этого под\-хода он описал четыре этапа 
расширения предметной области информатики, взяв за основной\linebreak объект 
исследований информационные трансформации в~искусственных, живых 
и~социальных сис\-те\-мах. При этом он использовал деление всех научных 
знаний на самом верхнем уровне на четыре\linebreak отрас\-ли науки~\cite{13-zat}, 
обозначив их литерами: информатика~(C), физические науки~(P), науки 
о~жизни~(L) и~социальные науки~(S). Далее он рассматривает 
информационные трансформации, охватывающие две, три или четыре отрасли, 
вклю\-чая информатику, обозначая каж\-дый этап расширения предметной 
об\-ласти и~вербально, и~с использованием этих четырех литер, сле\-ду\-ющим 
образом: диадический компьютинг (C\;+\;P, C\;+\;L, C\;+\;S); триадический 
компьютинг (C\;+\;P\;+\;S, C\;+\;P\;+\;L, C\;+\;L\;+\;S); тетрадический 
компьютинг (P\;+\;S\;+\;L\;+\;C); полиадический компьютинг (объединение всех 
перечисленных выше расширений).
  
  В завершение этого раздела отметим в~истории науки аналогию между 
эволюционным расширением предметных областей геологии и~информатики. 
Развитие геологии как науки о Земле началось в~XVII~в.\ с~работы Николая 
Стенона~\cite{14-zat}. Отдельные модели развития Земли, которые были 
впервые разработаны в~геологии, затем использовались и~для описания 
эволюции планет и~других тел Солнечной системы. Появились новые научные 
дисциплины, такие как <<геология Луны>>, <<геология Марса>> и~<<геология 
Венеры>>, которые описывают модели эволюции этих тел Солнечной системы.
  
  Аналогичным образом могут создаваться теоретические модели для 
описания, например, взаимосвязей <<мозг--ком\-пью\-тер>> (brain--computer 
interface~--- BCI) или других видов интерфейсов, которые появятся в~ИТ 
будущих поколений вследствие расширения сферы их  
применения~\cite{15-zat}. Их описание может быть во многом выполнено по 
аналогии с~моделями че\-ло\-ве\-ко-ма\-шин\-ных интерфейсов (human--computer 
interface~--- HCI), разработанными и~уже широко используемыми в~системах 
и~средствах информатики. 

В~следующем разделе статьи будет дано описание 
взаимосвязей <<мозг--ком\-пью\-тер>> с~использованием сочетания 
традиционных взаимосвязей на границах двух сред, которые будем называть 
интерфейсами второго порядка, с~интерфейсами 3-го порядка на границах трех 
сред.

\vspace*{-6pt}
  
\section{Интерфейсы третьего порядка}

\vspace*{-4pt}
    
  Сначала приведем краткое описание вза\-имо\-связей <<мозг--ком\-пью\-тер>> 
из работы~\cite{12-zat}, которое будет использоваться далее при обосновании 
необходимости введения понятия <<интерфейс \mbox{3-го} порядка>>:
  %
  <<Если мы хотим представить все богатство вза\-имо\-свя\-зей между мозгом 
и~компьютером, нам необходимо рассмотреть информационные 
трансформации, относящиеся ко всей предметной об\-ласти тетрадического 
компьютинга (P\;+\;S\;+\;L\;+\;C). Эти интерфейсы включают не только 
трансформации сигналов моз\-га~(L) и~генерацию компьютерных кодов~(C), но 
также физические устройства~(P), которые обеспечивают преобразование 
сигналов мозга в~коды компьютера, плюс кон\-цеп\-ты знаний человека~(S), 
которые могут быть пред\-став\-ле\-ны сигналами, отражающими 
активность мозга, и~эти сигналы в~конечном итоге и~будут использоваться для управ\-ле\-ния 
компьютером или другими устройствами>>.
  
  До описания понятия <<интерфейс 3-го порядка>> на примере взаимосвязей 
между мозгом и~компьютером сначала рассмотрим пять ранее пе\-ре\-чис\-лен\-ных 
сред. Выделим их из пред\-мет\-ной об\-ласти тетрадического компьютинга, которая 
поз\-во\-ля\-ет охватить максимальный спектр информационных транс\-фор\-ма\-ций 
в~искусственных, живых и~социальных системах (рис.~1).

\begin{figure*} %fig2
\vspace*{1pt}
    \begin{center}  
  \mbox{%
 \epsfxsize=89.141mm 
 \epsfbox{zac-2.eps}
 }
\end{center}
\vspace*{-6pt}
\Caption{Четыре среды разной природы и~границы между ними}
\vspace*{3pt}
\end{figure*}
  

  Основное отличие выделения сред разной природы из предметной области 
тетрадического компьютинга от деления всей науки Деннингом и~Розенблюмом 
на четыре отрасли знания~\cite{13-zat} состоит в~том, что объекты одной и~той 
же среды могут изучать\-ся одновременно в~разных научных дис\-цип\-ли\-нах. 
Например, концепты встречаются в~задачах пред\-став\-ле\-ния знаний 
в~информатике~\cite{15-zat, 16-zat, 17-zat}, они являются объектами 
исследований в~науках о~жиз\-ни~\cite{18-zat} и~со\-цио\-гу\-ма\-ни\-тар\-ных 
науках~\cite{19-zat}, но относятся они только к~одной ментальной среде знаний 
человека. Отметим, что пять сред, перечисленных в~начале статьи, 
присутствуют в~предметных областях и~тетрадического компьютинга (слева на 
рис.~1), и~полиадического компьютинга, но они не описаны  
в~работе~\cite{12-zat}. В~работе~\cite{15-zat} было показано, что кроме пяти 
перечисленных могут быть выделены и~включены в~рассмотрение среды 
и~другой природы. Однако из пяти сред, условно обозначенных в~виде 
прямоугольников справа на рис.~1, четырех (кроме ДНК-сре\-ды) будет 
достаточно для описания интерфейсов 3-го порядка, рассматриваемых в~статье.
  
  Эти интерфейсы рассмотрим на примере процессов управления 
роботизированной рукой. 
%
В~этом примере расположим четыре среды по-дру\-го\-му, 
чтобы были видны границы между ними (рис.~2). На этом рисунке на 
границах между двумя средами кругами условно обозначены интерфейсы 2-го 
порядка шести разных видов (они пронумерованы от~1 до~6), а на границах 
между тремя средами обозначены интерфейсы 3-го порядка двух видов, 
которые по определению обеспечивают связи между объектами трех сред 
разной природы (7 и~8). В~этой статье другие виды интерфейсов \mbox{3-го} и~более 
высокого порядка не рассматриваются. Интерфейсы вида №\,7 связывают 
объекты нейросреды, ментальной и~цифровой сред, а вида №\,8~--- ментальной,  
со\-цио\-ин\-фор\-ма\-ци\-он\-ной и~цифровой сред.
  
  Если направление информационных трансформаций на границах сред не 
учитывать, то для четырех сред число видов интерфейсов 2-го порядка равно 
шести. Все они были описаны в~\cite{15-zat}. Число видов интерфейсов 3-го 
порядка равно четырем. Из шести видов интерфейсов 2-го порядка наиболее 
известными и~широкого используемыми являются: №\,2 на границе ментальной 
и~со\-цио\-ин\-фор\-ма\-ци\-он\-ной сред, описывающий взаимосвязи между 
концептами и~формами их пред\-став\-ле\-ния~\cite{19-zat}, и~№\,3, вклю\-ча\-ющий  
че\-ло\-ве\-ко-ма\-шин\-ные интерфейсы кодирования растровых и~векторных 
изображений, а~также символов, со\-став\-ля\-ющих
вербальные формы пред\-став\-ле\-ния знаний 
человека, например с~по\-мощью таб\-лиц Unicode.
  
  Интерфейсы вида №\,1 описывают связи потенциалов фокальных полей или 
других сигналов нейросреды как индикаторов активности головного мозга 
с~концептами знаний человека в~тех случаях, когда эти связи могут быть 
установлены. Интерфейсы вида №\,4 теоретически предназначены для 
описания связей между концептами ментальной среды и~компьютерными 
кодами цифровой среды. Однако интерфейсы этого вида, как будет показано 
ниже, практически невозможно вычленить из интерфейсов третьего порядка 
вида №\,7 между концептами, сигналами нейросреды и~компьютерными 
кодами. Интерфейсы вида №\,5 описывают связи сигналов нейросреды 
с~компьютерными кодами. Интерфейсы вида №\,6 описывают связи сигналов 
нейросреды с~формами представления знаний человека в~виде текстов, 
изображений и~других перцептивных форм.


  Для описания процессов управления роботизированной рукой понадобятся 
интерфейсы 2-го порядка трех видов (1, 4 и~5) и~один вид интерфейсов 3-го 
порядка~(7). 

Рисунок~3 содержит упрощенную схему сред и~границ 
с~интерфейсами <<мозг--ком\-пью\-тер>>, которую будем использовать, чтобы 
получить ответ на основной вопрос статьи: <<Зачем нужны интерфейсы 3-го 
порядка?>>

\begin{figure*} %fig3
\vspace*{1pt}
    \begin{center}  
  \mbox{%
 \epsfxsize=95.147mm 
 \epsfbox{zac-3.eps}
 }
\end{center}
\vspace*{-9pt}
\Caption{Три среды и~границы с~интерфейсами для управления роботизированной рукой}
\end{figure*}
  
  На схеме условно обозначены связи между концептами знаний человека 
(ментальная среда) и~соответствующими сигналами нейросреды (№\,1). 
С~нейрофизиологической точки зрения эти сигналы отражают активность 
разных участков головного мозга, но не концепты. Согласно Б.~Баарсу\linebreak 
и~Н.~Гейдж, <<чрезвычайно трудно обнаружить конкретные места коры мозга 
для определенных концептов>>~\cite{18-zat}. Однако в~некоторых случаях 
(определенные категории животных и~инструментов,\linebreak а~так\-же числа) связи 
между концептами и~локализованными сигналами уже  
установлены~\cite{20-zat, 21-zat}. При этом одному концепту может 
соответствовать сочетание сигналов.
  


  Процессы управления роботизированной рукой служат примером 
формирования и~последующего использования локализованных сигналов 
нейросреды на личностном уровне. Для конкретного человека связь между его 
имплицитными личностными концептами, релевантными его намерениям,\linebreak 
и~сигналами нейросреды, регистрируемыми с~по\-мощью интерфейсов  
<<мозг--ком\-пью\-тер>>, уста\-нав\-ли\-ва\-ет\-ся в~процессе обучения системы 
управления роботизированной рукой с~использованием\linebreak методов измерения 
мозговой активности, например электроэнцефалографии (ЭЭГ) или 
электрокортикографии\footnote{Исследования, проведенные за последние три 
десятилетия, показали, что можно использовать сигналы мозга для передачи 
информации о своих намерениях компьютеру, если применять интерфейсы 
<<мозг--ком\-пью\-тер>>. Системы, на них основанные, измеряют 
характеристики мозговой активности. Сначала чаще всего использовались 
методы ЭЭГ. В~первое десятилетие нашего века значительное число 
исследователей стало использовать метод электрокортикографии для 
регистрации активности непосредственно с~поверхности мозга~\cite{23-zat}. 
В~этой работе представлено описание проблемы сбора сигналов, протоколов 
и~результатов научных исследований, ориентированных на решение задач 
нейропротезирования.}. Проведенные эксперименты показали, что после 
обучения системы этот человек может управлять ею с~помощью сигналов 
нейросреды, регистрируемых с~помощью метода ЭЭГ~\cite{22-zat}.
  
  Интерфейсы второго порядка вида №\,4 теоретически предназначены для 
описания связей между концептами ментальной среды и~компьютерными 
кодами цифровой среды. Эти связи практически невозможно вычленить из 
интерфейса третьего порядка вида №\,7 между концептами, сигналами 
нейросреды и~компьютерными кодами. Невозможность вычленения является 
следствием того, что чрезвычайно трудно обнаружить конкретные участки 
мозга, зарегистрированные сигналы в~которых соответствуют определенному 
концепту или концептам, о чем писали Б.~Баарс и~Н.~Гейдж~\cite{18-zat}. 
Однако приведенный эксперимент показал, что использование компьютерной 
системы управления роботизированной рукой в~сочетании с~аппаратом для ЭЭГ 
дает возможность сформировать библиотеку кодов пар <<имплицитный  
кон\-цепт\,--\,сиг\-нал нейросреды>> (на рис.~3 и~далее слова 
<<имплицитный>> и~<<нейросреда>> в~парах опущены). Эта библиотека 
кодов описывает интерфейсы именно 3-го порядка вида №\,7 на границе между 
нейросредой, ментальной и~цифровой средами. 

Отсюда и~следует ответ на 
основной вопрос статьи: <<Зачем нужны интерфейсы 3-го порядка 
в~информатике?>>~--- <<Для описания взаимодействий объектов на границах 
трех сред, когда невозможна их декомпозиция на интерфейсы 2-го порядка>>. 
  %
  При этом отметим, что интерфейсы второго порядка вида №\,5 (в~отличие от 
№\,4) достаточно прос\-то вычленяются из интерфейсов вида №\,7 и~они могут 
использоваться независимо от интерфейсов 2-го порядка других видов или 
в~сочетании с~ними. Они применяются для установления связей между сигналами 
нейросреды и~компьютерными кодами. Однако интерфейсы 2-го порядка не 
предназначены для установления связей с~имплицитными личностными 
концептами. Поэтому в~процессе обучения системы для конкретного человека 
используются интерфейсы 3-го порядка вида №\,7 и~формируется библиотека 
компьютерных кодов пар <<кон\-цепт--сиг\-нал>> (на рис.~3 они обозначены 
как <<кон\-цепт$_1$--сиг\-нал>>, $\ldots$, <<кон\-цепт$_N$--сиг\-нал>>), 
соответствующих личностным концептам именно этого человека 
и~составляющих систему его имплицитных знаний о~движениях руки (см.\ рис.~3). 
Эта библиотека формируется в~процессе обучения до начала управ\-ле\-ния рукою. 
Когда библиотека сформирована, человек может начать управ\-ле\-ние 
роботизированной рукой с~по\-мощью сигналов, ре\-гист\-ри\-ру\-емых методом ЭЭГ, 
что получило экспериментальное подтверждение. Полученный код сигнала 
сравнивается с~кодами пар биб\-лио\-те\-ки. По их совпадению и~определяется 
именно то движение, которое намеревался выполнить человек, и~оно 
выполняется роботизированной рукой.

\vspace*{-6pt}
  
\section{Заключение}

\vspace*{-4pt}
    
  В статье дана краткая характеристика шести видов интерфейсов второго 
порядка на границах двух из четырех сред разной природы, входящих 
в~предметную область информатики: нейросреды, ментальной,  
со\-цио\-ин\-фор\-ма\-ци\-он\-ной и~цифровой сред. Предлагаемое выделение 
сред планируется использовать в~будущем для концептуально единого 
описания системы интерфейсов 2-го, 3-го и~более высоких порядков.
  
  Новое понятие <<интерфейс 3-го порядка>> вводится на примере 
взаимосвязей между имплицитными личностными концептами, сигналами 
нейросреды и~компьютерными кодами. При этом\linebreak
 были рассмотрены четыре 
среды из пяти упомянутых (кроме ДНК-сре\-ды, информационные 
трансформации c участием объектов которой планируется рассмотреть в~другой 
статье). Наблюдаемое и~возможное будущее увеличение числа сред предметной 
области информатики является во многом следствием расширения ее 
предметной области на информационные трансформации в~живых 
и~социальных системах.
  
  В статье актуальность введения понятия <<интерфейсы 3-го порядка>> 
иллюстрировалась на примере процессов управления роботизированной рукой 
сигналами нейросреды. При этом был упомянут и~еще один вид интерфейсов  
3-го порядка~--- №\,8 между объектами ментальной,  
со\-цио\-ин\-фор\-ма\-ци\-он\-ной и~цифровой сред. На его основе планируется 
определить сис\-те\-му одновременного кодирования концептов (ментальная 
среда) и~форм их пред\-став\-ле\-ния (со\-цио\-ин\-фор\-ма\-ци\-он\-ная среда) 
в~циф\-ро\-вой среде сис\-тем искусственного интеллекта, вклю\-чая личностные 
и~коллективные, а также темпоральные концепты знаний  
человека~\cite{16-zat, 17-zat}. Будет рас\-смот\-ре\-на еще одна причина 
необходимости использования интерфейсов 3-го порядка: в~общем случае 
информационные трансформации на границах №\,2 и~№\,3 (см.\ рис.~2), которые 
широко используются в~сис\-те\-мах и~средствах информатики, являются\linebreak 
асим\-мет\-рич\-ны\-ми. Многозначные слова и~синони\-мы могут служить примерами 
проявления асиммет\-рии на границе №\,2~\cite{24-zat}. Определение 
интерфейсов вида №\,8 служит основой разработки сис\-те\-мы кодирования каж\-дой 
из пар <<значение зна\-ка\,--\,фор\-ма знака>>~\cite{19-zat} одним кодом, что 
должно обеспечивать пред\-став\-ле\-ние разных значений и~каждого из синонимов 
уникальными кодами в~циф\-ро\-вой среде.

\vspace*{-6pt}
  
{\small\frenchspacing
 {%\baselineskip=10.8pt
 \addcontentsline{toc}{section}{References}
 \begin{thebibliography}{99}
 
 \vspace*{-3pt}
 
\bibitem{1-zat}
\Au{Fatourou P., Hankin~C.} Welcome to the Europe region special section~// 
Commun. ACM, 2019. Vol.~62. No.\,4. P.~30.
\bibitem{2-zat}
\Au{Caspersen M.\,E., Gal-Ezer~J., McGettrick~A., Nardelli~E.} Informatics as 
a~fundamental discipline for the 21st century~// Commun. ACM, 2019. 
Vol.~62. No.\,4. P.~58--63.
\bibitem{3-zat}
The European Commission. Digital Education Action Plan, 2018. 31~p. {\sf 
https://ec.europa.eu/education/ education-in-the-eu/digital-education-action-plan\_en}.
\bibitem{4-zat}
The Committee on European Computing Education. Informatics education in Europe: 
Are we all in the same boat?~--- New York, NY, USA: ACM, 
2017.  Technical Report. 251~p. {\sf https://doi.org/10.1145/3106077}.
\bibitem{5-zat}
\Au{Caspersen M.\,E., Gal-Ezer~J., McGettrick~A., Nardelli~E.} Informatics for all: 
The strategy.~--- New York, NY, USA: ACM, 2018. 16~p.
\bibitem{6-zat}
The White House. Office of the Press Secretary. FACT SHEET: President Obama 
Announces Computer Science for All Initiative, 2016. 12~p. {\sf 
https://\linebreak obamawhitehouse.archives.gov/the-press-office/2016/ 01/30/fact-sheet-president-obama-announces-computer-science-all-initiative-0}.
\bibitem{7-zat}
\Au{Newell A., Perlis~A., Simon~H.} Computer science~// Science, 1967. Vol.~157. 
No.\,3795. P.~1373--1374.
\bibitem{8-zat}
\Au{Gorn S.} The computer and information sciences: A~new basic discipline~// 
SIAM Rev., 1963. Vol.~5. No.\,2. P.~150--155.
\bibitem{9-zat}
\Au{Denning P.} Computing is a natural science~// Commun. ACM, 2007. 
Vol.~50. No.\,7. P.~13--18.
\bibitem{10-zat}
\Au{Rozenberg G.} Computer science, informatics, and natural computing~--- 
personal reflections~// New computational paradigms: Changing conceptions of what 
is computable~/ Eds.\ S.\,B.~Cooper, B.~L$\ddot{\mbox{o}}$we, A.~Sorbi.~--- 
New York, NY, USA: Springer, 2008. P.~373--379.
\bibitem{11-zat}
\Au{Snir M.} Computer and information science and engineering: One discipline, 
many specialties~// Commun. ACM, 2011. Vol.~54. No.\,3. P.~38--43.
\bibitem{12-zat}
\Au{Rosenbloom P.} On computing: The fourth great scientific domain.~--- 
Cambridge: MIT Press, 2013. 308~p.
\bibitem{13-zat}
\Au{Denning P., Rosenbloom~P.} Computing: The fourth great domain of science~// 
Commun.  ACM, 2009. Vol.~52. No.\,9. P.~27--29.
\bibitem{14-zat}
\Au{Стенон Н.} О~твердом, естественно содержащемся в~твердом~/
Пер. с~лат.~--- М.: Изд-во АН 
СССР, 1957. 152~с. (\Au{Stenonis~N.} De solido intra solidum naturaliter contento.~--- 
Florentiae: Ex typographia sub signo Stellae, 1669. {\sf 
http://www.biusante.parisdescartes.fr/histmed/ medica/cote?06916}.
\bibitem{15-zat}
\Au{Зацман И.\,М.} Таблица интерфейсов информатики как 
ин\-фор\-ма\-ци\-он\-но-компьютерной науки~// На\-уч\-но-тех\-ни\-че\-ская информация. Сер.~1: 
Организация и~методика информационной работы, 2014. №\,11. С.~1--15.
\bibitem{16-zat}
\Au{Зацман И.\,М., Косарик~В.\,В., Курчавова~О.\,А.} Задачи представления 
личностных и~коллективных концептов в~цифровой среде~// Информатика и~её 
применение, 2008. Т.~2. Вып.~3. С.~54--69.
\bibitem{17-zat}
\Au{Zatsman I.} Tracing emerging meanings by Computer: Semiotic framework~// 
13th European Conference on Knowledge Management Proceedings.~--- Reading: 
Academic Publishing International Ltd., 2012. Vol.~2. P.~1298--1307.
\bibitem{18-zat}
\Au{Baars B., Gage~N.} Cognition, brain, and consciousness: Introduction to 
cognitive neuroscience.~--- Amsterdam: Academic Press/Elsevier, 2010. 677~p.
\bibitem{19-zat}
\Au{Eco U.} A~theory of semiotics.~--- Bloomington: Indiana University Press, 
1976. 356~p.
\bibitem{20-zat}
\Au{Caramazza A., Mahon~B.\,Z.} The organization of conceptual knowledge: The 
evidence from category-specific semantic deficits~// Trends Cogn. Sci., 
2003. Vol.~7. No.\,8. P.~354--361.
\bibitem{21-zat}
\Au{Dehaene S., Molko~N., Cohen~L., Wilson~A.\,J.} Arithmetic and the brain~// 
Current Opin. Neurobiol., 2004. Vol.~14. No.\,2. P.~218--224.

\bibitem{23-zat}
\Au{Schalk~G., Leuthardt~E.\,C.} Brain--computer interfaces using 
electrocorticographic signals~// IEEE Reviews Biomedical Eng., 2011. 
Vol.~4. No.\,1. P.~140--154.

\bibitem{22-zat}
\Au{Sunny T.\,D., Aparna~T., Neethu~P., Venkateswaran~J., Vishnupriya~V., 
Vyas~P.\,S.} Robotic arm with brain--computer interfacing~// Proc. Tech., 
2016. Vol.~24. P.~1089--1096.

\bibitem{24-zat}
\Au{Зацман И.\,М.} Концептуальный поиск и~качество информации.~--- М.: 
Наука, 2003. 272~с.
 \end{thebibliography}

 }
 }

\end{multicols}

\vspace*{-9pt}

\hfill{\small\textit{Поступила в~редакцию 01.07.19}}

\vspace*{6pt}

%\pagebreak

%\newpage

%\vspace*{-28pt}

\hrule

\vspace*{2pt}

\hrule

%\vspace*{-2pt}

\def\tit{THIRD-ORDER INTERFACES IN~INFORMATICS}


\def\titkol{Third-order interfaces in~informatics}

\def\aut{I.\,M.~Zatsman}

\def\autkol{I.\,M.~Zatsman}

\titel{\tit}{\aut}{\autkol}{\titkol}

\vspace*{-11pt}


\noindent
Institute of Informatics Problems, Federal Research Center ``Computer Science and Control'' of the 
Russian Academy of Sciences, 44-2~Vavilov Str., Moscow 119333, Russian Federation 


\def\leftfootline{\small{\textbf{\thepage}
\hfill INFORMATIKA I EE PRIMENENIYA~--- INFORMATICS AND
APPLICATIONS\ \ \ 2019\ \ \ volume~13\ \ \ issue\ 3}
}%
 \def\rightfootline{\small{INFORMATIKA I EE PRIMENENIYA~---
INFORMATICS AND APPLICATIONS\ \ \ 2019\ \ \ volume~13\ \ \ issue\ 3
\hfill \textbf{\thepage}}}

\vspace*{3pt}   

   
    
    \Abste{The European Strategy ``Informatics for All,'' formally launched in 
Brussels in March 2018, distinguishes two tiers of teaching informatics in the system 
of secondary and higher education. The second tier, focused on the study of 
informational transformations in artificial, living, and social systems, involves 
choosing an informatics paradigm for teaching, and then its development. The need 
for development is due to two reasons: firstly, a~significant expansion of the scope of 
applications of information technologies considered in educational processes,  and
secondly, the integration of methods and tools of informatics into curricula in other 
areas of knowledge, which expands the range of information transformations. In the 
absence of a dominant informatics paradigm and the presence of several of its 
variants, the question of its choosing as the starting point of development is 
disputable. The ``Informatics education in Europe'' review, published in 2017 and 
preceding the development of the European strategy ``Informatics for All,'' lists three 
paradigm variants, including positioning informatics as the fourth great domain of 
science, proposed by Denning and Rosenbloom in 2009. A~detailed description of 
this variant under the name of polyadic computing was given by Rosenbloom in his 
book in 2013. The goal of the paper is to define a~new concept of 
``third-order 
interface'' based on the one-natured division of the domain of informatics as polyadic 
computing. The relevance of the concept is illustrated by the example of robotic arm 
control using brain--computer interfaces.}
    
    \KWE{third-order interface; polyadic computing; one-natured media of 
informatics domain; information transformations}
    
\DOI{10.14357/19922264190312} 

%\vspace*{-14pt}

%\Ack
%   \noindent



%\vspace*{-6pt}

  \begin{multicols}{2}

\renewcommand{\bibname}{\protect\rmfamily References}
%\renewcommand{\bibname}{\large\protect\rm References}

{\small\frenchspacing
 {%\baselineskip=10.8pt
 \addcontentsline{toc}{section}{References}
 \begin{thebibliography}{99}
\bibitem{1-zat-1}
\Aue{Fatourou, P., and C.~Hankin.} 2019. Welcome to the Europe region special 
section. \textit{Commun. ACM} 62(4):30.
\bibitem{2-zat-1}
\Aue{Caspersen, M.\,E., J.~Gal-Ezer, A.~McGettrick, and E.~Nardelli.} 2019. 
Informatics as a~fundamental discipline for the 21st century. \textit{Commun. 
ACM}  62(4):\linebreak 58--63. 
\bibitem{3-zat-1}
The European Commission. 2018. Digital Education Action Plan. Available at: {\sf 
https://ec.europa.eu/ education/education-in-the-eu/digital-education-action-plan\_en} 
(accessed April~2, 2019). 
\bibitem{4-zat-1}
The Committee on European Computing Education. 2017. Informatics education in 
Europe: Are we all in the same boat?  New York, NY: ACM.  Technical Report. 251~p.
Available at: {\sf  https://dl.acm.org/citation. cfm?id=3106077} (accessed April~2, 
2019).
\bibitem{5-zat-1}
\Aue{Caspersen, M.\,E., J.~Gal-Ezer, A.~McGettrick, and E.~Nardelli.} 2018. 
\textit{Informatics for all: The strategy}. New York, NY: ACM. 16~p.
\bibitem{6-zat-1}
The White House. Office of the Press Secretary. 2016. FACT SHEET: President 
Obama Announces Computer Science for All Initiative. 12~p. Available at: {\sf 
https://obamawhitehouse.archives.gov/the-press-\linebreak office/2016/01/30/fact-sheet-president-obama-announces-computer-science-all-initiative-0} 
(accessed April~2, 2019).
\bibitem{7-zat-1}
\Aue{Newell, A., A.~Perlis, and H.~Simon.} 1967. Computer science. 
\textit{Science} 157(3795):1373--1374.
\bibitem{8-zat-1}
\Aue{Gorn, S.} 1963. The computer and information sciences: A~new basic 
discipline. \textit{SIAM Rev.} 5(2):150--155.
\bibitem{9-zat-1}
\Aue{Denning, P.} 2007. Computing is a~natural science. \textit{Commun. 
ACM} 50(7):13--18.
\bibitem{10-zat-1}
\Aue{Rozenberg, G.} 2008. Computer science, informatics, and natural computing~--- 
personal reflections. \textit{New computational paradigms: Changing conceptions of 
what is computable}. Eds.\ S.\,B.~Cooper, B.~L$\ddot{\mbox{o}}$we, and A.~Sorbi.
New York, NY: Springer. 373--379.
\bibitem{11-zat-1}
\Aue{Snir, M.} 2011. Computer and information science and engineering: one 
discipline, many specialties. \textit{Commun.  ACM} 54(3):38--43.
\bibitem{12-zat-1}
\Aue{Rosenbloom, P.} 2013. \textit{On computing: The fourth great scientific 
domain.} Cambridge: MIT Press. 308~p.
\bibitem{13-zat-1}
\Aue{Denning, P., and P.~Rosenbloom.} 2009. Computing: The fourth great domain 
of science. \textit{Commun. ACM} 52(9):27--29.
\bibitem{14-zat-1}
\Aue{Stenonis, N.} 1669. \textit{De solido intra solidum naturaliter contento}. 
Florentiae: Ex typographia sub signo Stellae. Available at: {\sf 
http://www.biusante.parisdescartes.fr/histmed/\linebreak medica/cote?06916} (accessed 
April~2, 2019).
\bibitem{15-zat-1}
\Aue{Zatsman, I.} 2014. Table of interfaces of informatics as computer and 
information science. \textit{Scientific Technical Information Processing} 
41(4):233--246.
\bibitem{16-zat-1}
\Aue{Zatsman, I., V.~Kosarik, and O.~Kurchavova.} 2008. Zadachi predstavleniya 
lichnostnykh i~kollektivnykh kontseptov v~tsifrovoy srede [Problems of 
representation of personal and collective concepts in the digital medium]. 
\textit{Informatika i~ee Primeneniya~--- Inform. Appl.} 2(3):54--69.
\bibitem{17-zat-1}
\Aue{Zatsman, I.} 2012. Tracing emerging meanings by computer: Semiotic 
framework. \textit{13th European Conference on Knowledge Management 
Proceedings}. Reading: Academic Publishing International Ltd. 2:1298--1307.
\bibitem{18-zat-1}
\Aue{Baars, B., and N.~Gage.} 2010. \textit{Cognition, brain, and consciousness: 
Introduction to cognitive neuroscience.} Amsterdam: Academic Press/Elsevier. 
677~p.
\bibitem{19-zat-1}
\Aue{Eco, U.} 1976. \textit{A~theory of semiotics.} Bloomington: Indiana University 
Press. 356~p.
\bibitem{20-zat-1}
\Aue{Caramazza, A., and B.\,Z.~Mahon.} 2003. The organization of conceptual 
knowledge: The evidence from category-specific semantic deficits. \textit{Trends  
Cogn. Sci.} 7(8): 354--361.
\bibitem{21-zat-1}
\Aue{Dehaene, S., N.~Molko, L.~Cohen, and A.\,J.~Wilson.} 2004. Arithmetic and 
the brain. \textit{Current Opin. Neurobiol.} 14(2):218--224.

\bibitem{23-zat-1}
\Aue{Schalk, G., and E.\,C.~Leuthardt.} 2011. Brain--computer interfaces using 
electrocorticographic signals. \textit{IEEE Reviews Biomedical Eng.} 
4(1):140--154.
\bibitem{22-zat-1}
\Aue{Sunny, T.\,D., T.~Aparna, P.~Neethu, J.~Venkateswaran, V.~Vishnupriya, and 
P.\,S.~Vyas.} 2016. Robotic arm with brain--computer interfacing. \textit{Proc. 
Tech.} 24:1089--1096.

\bibitem{24-zat-1}
\Aue{Zatsman, I.} 2003. \textit{Kontseptual'nyy poisk i~kachestvo informatsii} 
[Conceptual retrieval and quality of information]. Moscow: Nauka. 272~p.
    
\end{thebibliography}

 }
 }

\end{multicols}

%\vspace*{-7pt}

\hfill{\small\textit{Received July 1, 2019}}

%\pagebreak

%\vspace*{-22pt}


\Contrl

    
    \noindent
    \textbf{Zatsman Igor M.} (b.\ 1952)~--- Doctor of Science in 
technology, Head of Department, Institute of Informatics Problems, Federal 
Research Center ``Computer Science and Control'' of the Russian Academy 
of Sciences, 44-2~Vavilov Str., Moscow 119333, Russian Federation; 
\mbox{izatsman@yandex.ru}

\label{end\stat}

\renewcommand{\bibname}{\protect\rm Литература}
    
    