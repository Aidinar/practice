\def\stat{pavlov}

\def\tit{ОБ АСИМПТОТИКЕ КЛАСТЕРНОГО КОЭФФИЦИЕНТА КОНФИГУРАЦИОННОГО ГРАФА 
С~НЕИЗВЕСТНЫМ РАСПРЕДЕЛЕНИЕМ СТЕПЕНЕЙ ВЕРШИН$^*$}

\def\titkol{Об асимптотике кластерного коэффициента конфигурационного графа 
с~неизвестным распределением степеней} % вершин}

\def\aut{Ю.\,Л.~Павлов$^1$}

\def\autkol{Ю.\,Л.~Павлов}

\titel{\tit}{\aut}{\autkol}{\titkol}

\index{Павлов Ю.\,Л.}
\index{Pavlov Yu.\,L.}


{\renewcommand{\thefootnote}{\fnsymbol{footnote}} \footnotetext[1]
{Финансовое обеспечение исследований осуществлялось из средств федерального бюджета 
на выполнение государственного задания Карельского научного центра 
Российской академии наук (Институт 
прикладных математических исследований КарНЦ РАН).}}


\renewcommand{\thefootnote}{\arabic{footnote}}
\footnotetext[1]{Институт прикладных математических исследований Федерального 
исследовательского центра <<Карельский научный центр 
Российской академии наук>>, \mbox{pavlov@krc.karelia.ru}}

%\vspace*{-2pt}



\Abst{Рассматриваются конфигурационные графы со случайными одинаково 
распределенными степенями вершин.
Степень каждой вершины равна числу выходящих из нее различимых полуребер. 
Граф строится путем попарного равновероятного соединения
полуребер друг с~другом для образования ребер. 
Такой граф допускает петли и~кратные ребра. Конфигурационные графы 
широко используются для моделирования сложных сетей коммуникаций, 
включая интернет, социальные, транспортные, телефонные сети. 
Распределение степеней вершин может быть неизвестным. Предполагается 
только, что оно имеет конечное математическое ожидание и~либо дисперсия 
тоже конечна, либо выполнены некоторые достаточно слабые ограничения 
на асимптотическое поведение хвоста распределения. Обсуждаются понятие 
кластерного коэффициента и~его свойства 
в~таких графах. При стремящемся к~бесконечности числе вершин доказана 
предельная теорема для кластерного коэффициента. Найдены условия, при 
которых этот коэффициент неограниченно возрастает.}


\KW{случайные графы; конфигурационные графы; кластерный коэффициент; предельные теоремы}

\DOI{10.14357/19922264190302} 
  
\vspace*{3pt}


\vskip 10pt plus 9pt minus 6pt

\thispagestyle{headings}

\begin{multicols}{2}

\label{st\stat}

\section{Введение} %\label{SC:1}

Случайные графы широко используются для моделирования сложных 
сетей коммуникаций, таких как интернет, системы мобильной связи,
социальные, транспортные, электрические сети и~др. 
Наиболее полный обзор таких моделей можно найти в~книге~\cite{Hof}.
Известно, что, хотя сети и~имеют различную природу, они обладают 
общими свойствами. Одним из важнейших свойств является эмпирически 
установленный факт, что степени вершин в~графах, моделирующих сети,
 можно считать независимыми одинаково распределенными случайными 
 величинами. При этом число вершин, степень которых не меньше~$k,$ 
 при достаточно больших~$k$ пропорционально~$k^{-\tau},$ где $\tau \hm>0.$ 
 Поэтому распределение случайной величины~$\xi,$
равной степени любой вершины, можно представить сле\-ду\-ющим образом:

\noindent
\begin{equation}
\label{EQ:1}
{\sf P}\{\xi \geqslant k\} = \fr{h(k)}{k^\tau},\enskip k=1,2,\ldots,
\end{equation}

\vspace*{-2pt}

\noindent
где $h(k)$~--- медленно меняющаяся функция. 
В~настоящее время существуют два основных класса случайных графов,
удовлетворяющих условию~(\ref{EQ:1}). Это графы предпочтительного присоединения, 
идея которых была изложена в~\cite{Bar}, и~конфигурационные графы, 
впервые предложенные в~\cite{Bol}.

Будем рассматривать конфигурационные графы с~$N$~вершинами, степени которых 
$\xi_1, \dots, \xi_N$ являются независимыми случайными величинами 
с~распределением~(\ref{EQ:1}). Степень каждой вершины равна числу выходящих из 
нее полуребер, т.\,е.\ ребер, для которых смежные вершины еще не
определены. Все полуребра занумерованы в~произвольном порядке. 
Сумма степеней вершин любого графа должна быть четной, поэтому в~случае
нечетной суммы в~граф добавляется дополнительная вершина единичной степени. 
В~статье \cite{RN1} отмечалось, что эта дополнительная вершина,
вместе с~инцидентным ей ребром, не влияет на асимптотические свойства графа 
при $N\hm\rightarrow \infty.$ Поэтому далее не будем учитывать
дополнительную вершину в~случае ее появления. 

Построение конфигурационного
 графа со случайными степенями вершин завершается путем попарного
равновероятного соединения полуребер друг с~другом для образования ребер. 
Поскольку такое соединение проводится без ограничений, в~графе
возможно появление петель и~кратных ре-\linebreak\vspace*{-12pt}

\pagebreak

\noindent
бер. Ранее был доказан ряд 
предельных тео\-рем о степенной структуре конфигурационных графов, 
описывающих асимптотическое поведение числа вершин заданной степени и~максимальной 
степени вершин (см.~\cite{Pav1, Pav2} и~библиографию в~них). Значительное 
внимание было уделено условным графам при условии, что число ребер известно.
 Такие случайные графы полезны при моделировании сетей, в~которых можно 
 оценить число связей. Но и~для сетей без ограничений разумно использовать 
 результаты об условных графах, усредняя их по известным распределениям числа ребер. 
 В~последние годы при изучении графов, степени которых имеют распределение 
 вида~(\ref{EQ:1}), внимание исследователей привлечено не только к~свойствам 
 степенной структуры, но и~к~другим числовым характеристикам, более подробно 
 описывающим особенности сетей (см., например,~\cite{New}). Одной из важнейших 
 таких характеристик служит кластерный коэффициент. 
 В~следующем разделе дается определение этого коэффициента и~обсуждаются 
 некоторые его свойства, а~в~последнем разделе доказывается тео\-ре\-ма о~предельном 
 поведении кластерного коэффициента при стремлении к~бесконечности числа вершин графа.

\section{Кластерный коэффициент} %\label{SC:2}

Для того чтобы дать определение кластерного коэффициента, воспользуемся 
терминологией и~обозначениями из~\cite{Hof}.
Рассмотрим граф $G\hm=G(V,E)$ с~$N$ вершинами, образующими множество~$V,$ 
и~с~множеством ребер~$E.$ Пусть $(ij)$ означает ребро из $E,$ соединяющее
вершины $i,j\hm\in V.$ Обозначим через~$i,j,t$ три разных вершины графа. 
Если $(ij),(jt),(ti)\hm\in E,$ то будем говорить, что вершины~$i,j,t$
образуют треугольник в~$G.$ Введем сумму
\begin{multline*}
W_G=\sum\limits_{1\leqslant i,j,t\leqslant N} 
I\{(ij),(jt)\in E\}={}\\
{}=2\sum\limits_{1\leqslant i<j<t\leqslant N} 
I\{(ij),(jt) \in E\},
\end{multline*}
где $I\{A\}$~--- индикатор события~$A.$ Пусть также
\begin{multline*}
\Delta_G=\sum\limits_{1\leqslant i,j,t\leqslant N} 
I\{(ij),(jt),(ti)\in E\}={}\\
{}=6\sum\limits_{1\leqslant i<j<t\leqslant N} 
I\{(ij),(jt),(ti)\in E\}.
\end{multline*}

Кластерным коэффициентом~$C_G$ графа~$G$ называется величина
\begin{equation}
\label{EQ:2}
C_G=\fr{\Delta_G}{W_G}\,.
\end{equation}
Таким образом, $C_G$ равен отношению утроенного числа различных треугольников
 графа к~числу различных пар смежных ребер. Поэтому коэффициент~(\ref{EQ:2})
можно интерпретировать как вероятность того, что если у произвольной вершины~$i$ 
имеются две смежные с~ней вершины~$j$ и~$t,$ то существует также и~реб\-ро~$(jt).$
Заметим, что реб\-ро~$(jt)$ может быть кратным, тогда вершины~$i,j,t$ 
при фиксированных ребрах~$(ij)$ и~$(it)$ образуют несколько треугольников, число
которых равно кратности ребра~$(jt).$

Обозначим
\begin{equation}
\label{EQ:3}
p_k={\sf P}\left\{\xi_i = k\right\}, \enskip k=1,2,\ldots, \enskip i=1,\ldots, N\,.
\end{equation}

Далее будем считать, что распределение степени любой вершины имеет 
конечное математическое ожидание
\begin{equation}
\label{EQ:4}
m=\sum\limits_{s=1}^{k_{\max}} sp_s,
\end{equation}
где $k_{\max}$~--- максимально возможная степень вершины.

Пусть произвольные вершины~$i$ и~$j$ $(i\hm\neq j)$ соединены ребром~$(ij)$, 
а~в~случае нескольких таких ребер рассмотрим одно из них.
Если при построении графа из вершины~$j$ выходит~$k\hm+1$ полуребро, 
то одно из них необходимо для формирования ребра~$(ij),$ а~остальные~$k$~полуребер
могут быть использованы произвольно. Обозначим~$q_k$ вероятность того, что 
степень вершины $j$ равна $k$ без учета ребра~$(ij).$ В~\cite{New} показано,
что
\begin{equation}
\label{EQ:5}
q_k=\fr{(k+1)p_{k+1}}{m}, \enskip k=0,1,\ldots,k_{\max} -1\,.
\end{equation}

Рассмотрим случай, когда кроме вершины~$j$ существует еще и~другая смежная с~$i$ 
вершина~$t.$ Очевидно, что если вершины~$j$ и~$t$ имеют степени $k_j\hm+1$ и~$k_t\hm+1$ 
соответственно, то число способов образования ребра~$(jt)$ равно~$k_jk_t.$ 
Поэтому в~\cite{New} доказано, что для рассматриваемых конфигурационных графов 
кластерный коэффициент~(\ref{EQ:2}) имеет вид:
\begin{equation}
\label{EQ:6}
C_G=\fr{\Big(\sum\nolimits_{k=1}^{k_{\max}-1} kq_k\Big)^2}{Nm}\,.
\end{equation}

В статье~\cite{RN1} утверждается, что медленно ме\-ня\-юща\-яся функция~$h(k)$ 
в~распределении~(\ref{EQ:1}) не влияет на асимптотические свойства графа при
$N\rightarrow\infty$, и~поэтому предложено использовать простейший случай $h(k)\hm=1.$ 
Тогда из~(\ref{EQ:1}) и~(\ref{EQ:3}) следует, что $k_{\max}\hm=\infty$ и~\begin{equation}
\label{EQ:7}
p_k=\fr{1}{k^\tau}-\fr{1}{(k+1)^\tau}\,, \enskip k=1,2,\ldots, \enskip \tau >0\,.
\end{equation}
Отсюда следует, что при $k\hm\rightarrow \infty$
\begin{equation}
\label{EQ:8}
p_k \sim \fr{\tau}{k^{\tau +1}}\,.
\end{equation}
Известно~\cite{Hof}, что для большинства реальных сетей $\tau \hm\in (1,2).$ 
Однако в~некоторых случаях, например при моделировании лесных пожаров~\cite{Ler}, 
оказалось, что $\tau \hm>2.$

В~\cite{Pav1} рассматривалось обобщение этой модели, в~котором 
предполагалось, что распределение степеней вершин конфигурационного графа
неизвестно и~имеется только следующее ограничение на поведение~$p_k$ 
при $k\hm\rightarrow \infty$:
\begin{equation}
\label{EQ:9}
p_k\sim \fr{d}{k^g(\ln k)^h}\,,
\end{equation}
где $d>0$, $g>1$, $h\hm\geqslant 0.$ Понятно, что распределение~(\ref{EQ:7}), 
в~силу~(\ref{EQ:8}), удовлетворяет условию~(\ref{EQ:9}) при $d\hm=\tau$,
$g\hm=\tau \hm+1$ и~$h\hm=0.$
В~\cite{Pav1} приведены также и~другие примеры распределений вида~(\ref{EQ:7}),
 приводящие к~(\ref{EQ:9}). К~числу таких примеров относятся и~модели
с~распределением~(\ref{EQ:1}) степеней вершин, в~котором параметр~$\tau$ 
является случайной величиной. Заметим еще, что если $g\hm>3$ или $g\hm=3$,
$h\hm>1,$ то из~(\ref{EQ:9}) следует, что распределение~$\xi$ имеет 
конечную дисперсию.

 В~следующем разделе будет доказана тео\-ре\-ма 
о~предельном поведении
кластерного коэффициента~(\ref{EQ:6}) в~случайном конфигурационном графе
 с~неизвестным распределением степеней вершин, име\-ющим конечную дисперсию
либо удовлетворяющим условию~(\ref{EQ:9}).

\section{Предельная теорема}
%\label{SC:3}

Обозначим $\xi_{(N)}$ максимальную степень вершины в~графе:
$$
\xi_{(N)}=\max\limits_{1\leqslant s \leqslant N} \xi_i.
$$

Целью данной статьи является получение следующего результата.

\smallskip

\noindent
\textbf{Теорема.} \textit{Пусть $N\hm\rightarrow \infty.$ Тогда 
с~вероятностью, сколь угодно близкой к~единице,
справедливы сле\-ду\-ющие утверждения}.
\begin{enumerate}
\item \textit{Если дисперсия $\xi$ конечна, то} 
$$
C_G=O\left(\fr{1}{N}\right).
$$
\item \textit{Если $g=3$, $h=1,$ то}
$$
C_G\sim \fr{(d\ln \ln N)^2}{{m^3N}}.
$$
\item \textit{Если $g=3$, $h<1,$ то}
$$
C_G\sim 
\left (\fr{d}{1-h}\left (\fr{\ln N}{2}\right)^{1-h}\right)^2 
\fr{1}{m^3N}.
$$
\item \textit{Если $1<g<3,$ то существует постоянная~$B$ такая, 
что $N^{{1}/({g-1})}/B \hm\leqslant \xi_{(N)}\hm \leqslant 
BN^{{1}/({g-1})}$ и~при условии $\xi_{(N)}\hm=
uN^{{1}/({g-1})}$, $0\hm<u\hm<\infty,$ имеет место соотношение}:
    $$
    C_G\sim \left(\fr{du^{3-g}}{3-g}\right)^2 \left( 
    \fr{g-1}{\ln N}\right)^{2h}\fr{N^{({7-3g})/({g-1})}}{m^3}\,.
    $$
\end{enumerate}

\noindent
\textbf{Замечание.} 
Нетрудно видеть, что при выполнении условия~4 теоремы, если $g\hm>7/3$ или 
$g\hm=7/3$, $h\hm>0,$ то $C_G\hm\rightarrow 0,$ если $g\hm=7/3$, $h\hm=0,$ то~$C_G$ 
стремится к~зависящей от~$u$ константе, а если $g\hm<7/3,$ то 
$C_G\hm\rightarrow \infty.$ Такое явление объясняется тем, что при малых значениях~$g$ 
пара смежных ребер~$(ij)$ и~$(it),$ как упоминалось выше, может образовывать 
значительное число треугольников при наличии кратных ребер вида~$(jt).$ 
Это свойство было описано в~\cite{New}, где рассматривались модели, 
в~которых $p_k\hm=1/(k^\tau \zeta (\tau))$, $k\hm=1,2,\ldots$,
$\tau\hm >1,$ а $\zeta (\tau)$~--- значение дзе\-та-функ\-ции Римана в~точке~$\tau.$

\noindent
Д\,о\,к\,а\,з\,а\,т\,е\,л\,ь\,с\,т\,в\,о\  теоремы. 
Пусть распределение~$\xi$ имеет конечную дисперсию. Тогда из~(\ref{EQ:5}) 
следует, что ряд $\Sigma_{k=1}^\infty kq_k$
сходится и~из~(\ref{EQ:6}) непосредственно вытекает первое утверждение теоремы.

Нетрудно видеть, что
\begin{equation}
\label{EQ:10}
{\sf P}\left\{\xi_{(N)}<x\right\} =(1-{\sf P}\left\{\xi\geqslant x\right\})^N\,.
\end{equation}


 Пусть $g=3$, $h\hm\leqslant 1.$ Тогда из~(\ref{EQ:9}) находим, что при 
 $x\hm\rightarrow \infty$
\begin{multline}
\label{EQ:11}
{\sf P}\left\{\xi\geqslant x\right\}=d(1+o(1))
\sum\limits_{k\geqslant x} 
\fr{1}{k^3(\ln k)^h}={}\\
{}=d(1+o(1))\int\limits_x^\infty \fr{dy}{y^3(\ln y)^h}\,.
\end{multline}
Из~(\ref{EQ:11}) следует, что для любого сколь угодно малого $\delta\hm >0$
\begin{equation}
\label{EQ:12}
d\int\limits_x^\infty \fr{dy}{y^{3+\delta}} < {\sf P}\left\{\xi\geqslant x\right\} 
<d\int\limits_x^\infty \fr{dy}{y^3}\,.
\end{equation}
Положим $x=\sqrt{N/z},$ где $0\hm<z\hm<\infty.$ Тогда для любого фиксированного 
$z\hm>0$ из~(\ref{EQ:12}) находим, что
\begin{equation}
\label{EQ:13}
{\sf P}\left\{\xi\geqslant \sqrt{\fr{N}{z}}\right\}<\fr{dz}{2N}\,.
\end{equation}
Если же $x=(zN)^{{1}/({2+\delta})},$ то
\begin{equation}
\label{EQ:14}
{\sf P}\left\{\xi\geqslant (zN)^{{1}/({2+\delta})}\right\}>\fr{d}{(2+\delta)zN}\,.
\end{equation}
Из (\ref{EQ:10}), (\ref{EQ:13}) и~(\ref{EQ:14}) следует, что вероятность
${\sf P}\left\{(\varepsilon N)^{{1}/({2+\delta})}<\xi_{(N)}\hm<\sqrt{{N}/{\varepsilon}}\right\}$
может быть сделана сколь угодно близкой к~единице 
выбором достаточно малых положительных~$\varepsilon$ и~$\delta.$ 
Поэтому достаточно рассмотреть множество графов, в~которых $\xi_{(N)}\hm=u\sqrt{N},$ 
где $(\varepsilon /N^{{\delta}/{2}})^{{1}/({2+\delta})}\hm<u\hm<1/\sqrt{\varepsilon}.$ 
Тогда из~(\ref{EQ:4}), (\ref{EQ:5}) и~(\ref{EQ:9}) 
получаем, что при выполнении условия~2 теоремы и~при достаточно больших 
натуральных~$D$
\begin{equation}
\label{EQ:15}
\sum\limits_{k=1}^{u\sqrt{N}} kq_k=C+\fr{d(1+o(1))}{m} 
\sum\limits_{k=D}^{u\sqrt{N}-1} \fr{1}{k\ln k}\,,
\end{equation}
где $C$~--- некоторая положительная постоянная. 
Заменяя суммирование интегрированием, нетрудно получить, что
$$
\sum\limits_{k=1}^{u\sqrt{N}-1}kq_k \sim \fr{d}{m}
\int\limits_D^{u\sqrt{N}} \fr{dy}{y\ln y} \sim \fr{d\ln \ln N}{m}\,.
$$
Отсюда и~из~(\ref{EQ:6}) следует второе утверждение тео\-ремы.

Если $g\hm=3$ и~$h\hm<1,$ то, рассуждая аналогично, приходим к~выводу, что
$$
\sum\limits_{k=1}^{u\sqrt{N}-1} kq_k=\fr{d}{m(1-h)} 
\left(\fr{\ln N}{2}\right)^{1-h}(1+o(1))\,.
$$
Третье утверждение теоремы вытекает отсюда и~из~(\ref{EQ:6}).

Пусть $1\hm<g\hm<3.$ Аналогично предыдущему случаю нетрудно показать, 
с~помощью~(\ref{EQ:10}), что вероятность
${\sf P}\left\{(\varepsilon N)^{{1}/({g+\delta -1})}\hm<
\xi_{(N)}<\left({N}/{\varepsilon}\right)^{{1}/({g-1})}\right\}
$
можно сделать сколь угодно близкой к~единице, выбрав достаточно малые~$\varepsilon,
\delta \hm>0.$ Если $\xi_{(N)}\hm=uN^{{1}/({g-1})},$
$(\varepsilon /N^{{\delta}/({g-1})})^{{1}/({g+\delta-1})}\hm <u\hm< 
1/\varepsilon^{{1}/({g-1})},$
то по аналогии с~(\ref{EQ:15}) видим, что
\begin{multline}
\label{EQ:16}
\sum\limits_{k=1}^{uN^{{1}/({g-1})}}kq_k={}\\
{}=C+\fr{d(1+o(1))}{m}
\sum\limits_D^{uN^{{1}/({g-1})}-1}\fr{1}{k^{g-2}(\ln k)^h}\,.
\end{multline}
Заметим, что при $N\hm\rightarrow \infty$
\begin{multline}
\label{EQ:17}
\sum\limits_D^{uN^{{1}/({g-1})}-1}\fr{1}{k^{g-2}(\ln k)^h}={}\\
{}=
(1+o(1))\int\limits_D^{uN^{{1}/({g-1})}} \fr{dy}{y^{g-2}(\ln y)^h}\,.
\end{multline}
Рассматривая последний интеграл как функцию верхнего предела и~используя 
правило Лопиталя, получаем, что при $x\hm\rightarrow \infty$
$$
\int\limits_D^x \fr{dy}{y^{g-2}(\ln y)^h} \sim \fr{x^{3-g}}{(3-g)(\ln x)^h}\,.
$$
Отсюда и~из~(\ref{EQ:16}), (\ref{EQ:17}) следует, что
\begin{multline*}
\sum\limits_{k=1}^{uN^{{1}/({g-1})}-1} kq_k={}\\
{}=
\fr{du^{3-g}N^{({3-g})/({g-1})}}{m(3-g)}\left(\fr{g-1}{\ln N}\right)^h(1+o(1)),
\end{multline*}
и из~(\ref{EQ:6}) получаем последнее утверждение теоремы.

{\small\frenchspacing
 {%\baselineskip=10.8pt
 \addcontentsline{toc}{section}{References}
 \begin{thebibliography}{9}
\bibitem{Hof}
\Au{Hofstad R.} Random graphs and complex networks.~--- Cambridge:
Cambridge University Press, 2017.  Vol.~1. 337~p.

\bibitem{Bar}
\Au{Barabasi A.-L., Albert~R.} Emergence of scaling in random networks~// 
Science, 1999. Vol.~286. P.~509--512.

\bibitem{Bol}
\Au{Bollobas B.\,A.} A~probabilistic proof of an asymptotic formula
for the number of labelled regular graphs~// Eur. J.~Combin., 1980. Vol.~1. P.~311--316.

\bibitem{RN1}
\Au{Reittu H., Norros~I.} On the power-law random graph model of massive
data networks~// Perform. Evaluation, 2004. Vol.~55. P.~3--23.


\bibitem{Pav1}
\Au{Павлов Ю.\,Л.} Условные конфигурационные графы со случайным параметром 
степенного распределения степеней~// Математический сборник, 2018. Т.~209. 
Вып.~2. С.~120--137.


\bibitem{Pav2}
\Au{Павлов Ю.\,Л., Чеплюкова~И.\,А.} 
Об асимптотике степенной структуры конфигурационных графов с~ограничениями 
на число ребер~// Дискретная математика, 2018. Т.~30. Вып.~1. С.~77--94.


\bibitem{New}
\Au{Newman M.\,E.\,J.} The structure and function of complex networks~// 
SIAM Review, 2003. Vol.~45. Iss.~3. P.~3--23.

\bibitem{Ler}
\Au{Leri M., Pavlov~Y.} Forest fire models on configuration random
graphs~// Fund. Inform., 2016. Vol.~145. Iss.~3. P.~313--322.
 \end{thebibliography}

 }
 }

\end{multicols}

\vspace*{-6pt}

\hfill{\small\textit{Поступила в~редакцию 09.01.19}}

%\vspace*{8pt}

%\pagebreak

\newpage

\vspace*{-28pt}

%\hrule

%\vspace*{2pt}

%\hrule

%\vspace*{-2pt}

\def\tit{ON THE ASYMPTOTICS OF~CLUSTERING COEFFICIENT IN~A~CONFIGURATION 
GRAPH WITH~UNKNOWN DISTRIBUTION~OF~VERTEX~DEGREES}


\def\titkol{On the asymptotics of~clustering coefficient in~a~configuration 
graph with~unknown distribution of~vertex degrees}

\def\aut{Yu.\,L.~Pavlov}

\def\autkol{Yu.\,L.~Pavlov}

\titel{\tit}{\aut}{\autkol}{\titkol}

\vspace*{-11pt}

\noindent
 Institute of Applied Mathematical Research, Karelian Research Centre
of the Russian Academy of Sciences, 11~Pushkinskaya Str., Petrozavodsk 185910,
Karelia, Russian Federation 


\def\leftfootline{\small{\textbf{\thepage}
\hfill INFORMATIKA I EE PRIMENENIYA~--- INFORMATICS AND
APPLICATIONS\ \ \ 2019\ \ \ volume~13\ \ \ issue\ 3}
}%
 \def\rightfootline{\small{INFORMATIKA I EE PRIMENENIYA~---
INFORMATICS AND APPLICATIONS\ \ \ 2019\ \ \ volume~13\ \ \ issue\ 3
\hfill \textbf{\thepage}}}

\vspace*{3pt}      




\Abste{The author considers configuration graphs with vertex degrees being
independent identically distributed random variables. 
The degree of each vertex equals to the number of incident 
half-edges that are numbered in an arbitrary order. The graph 
is constructed by joining each half-edge to another equiprobably 
to form edges. Configuration graphs are widely used for modeling 
of complex communication networks such as the Internet, social,
 transport, telephone networks. The distribution of vertex degrees 
 can be unknown. It is only assumed that this distribution either has 
 a~finite variance or that some sufficient weak constraints on the asymptotic
  behavior of the tail are satisfied. The notion of clustering coefficient 
  and its properties in such graphs are discussed. The author proves the 
  limit theorem for the clustering coefficient with the number of vertices 
  tending to infinity. The conditions under which this coefficient increases 
  indefinitely are found.}


\KWE{random graphs; configuration graphs; clustering coefficient; limit theorems}


\DOI{10.14357/19922264190302} 

%\vspace*{-14pt}

\Ack
\noindent
Financial support for research was carried out from the federal 
budget for the implementation of the state assignment of the Karelian 
Research Centre of the Russian Academy of Sciences (Institute of Applied 
Mathematical Research, KarRC RAS).



%\vspace*{-6pt}

  \begin{multicols}{2}

\renewcommand{\bibname}{\protect\rmfamily References}
%\renewcommand{\bibname}{\large\protect\rm References}

{\small\frenchspacing
 {%\baselineskip=10.8pt
 \addcontentsline{toc}{section}{References}
 \begin{thebibliography}{9}


\bibitem{1-pv}
\Aue{Hofstad, R.} 2017. \textit{Random graphs and complex networks.} Cambridge:
Cambridge University Press.  Vol.~1. 337~p.

\bibitem{2-pv}
\Aue{Barabasi, A.-L., and R.~Albert}. 
1999. Emergence of scaling in random networks. \textit{Science} 286:509--512.

\bibitem{3-pv}
\Aue{Bollobas, B.\,A.} 1980. A~probabilistic proof of an asymptotic formula for the number
of labelled regular graphs. \textit{Eur. J.~Combin.} 1:311--316.

\bibitem{4-pv}
\Aue{Reittu, H., and I.~Norros.} 2004. On the power-law random graph model of massive data
networks. \textit{Perform. Evaluation} 55:3--23.

\bibitem{5-pv}
\Aue{Pavlov, Yu.\,L.} 2018. Conditional configuration graphs
 with discrete power-law distribution of vertex degrees. 
 \textit{Sb. Math.} 209(2):258--275.

\bibitem{6-pv}
\Aue{Pavlov, Yu.\,L., and I.\,A.~Cheplyukova}. 
2018. On asymptotics of degree structure of configuration graphs 
with restrictions on number of edges. \textit{Discrete Math.} 28(2):58--79.

\bibitem{7-pv}
\Aue{Newman, M.\,E.\,J.} 
2003. The structure and function of complex networks. \textit{SIAM Rev.} 45(3):3--23.

\bibitem{8-pv}
\Aue{Leri, M., and Y.~Pavlov}. 2016. Forest fire models on configuration random
graphs. \textit{Fund. Inform.} 145(3):313--322.
\end{thebibliography}

 }
 }

\end{multicols}

%\vspace*{-7pt}

\hfill{\small\textit{Received January 9, 2019}}

%\pagebreak

%\vspace*{-22pt}


\Contrl

\noindent
\textbf{Pavlov Yuri L.} (b.\ 1949)~--- 
Doctor of Science in physics and mathematics, principal scientist,
Institute of Applied Mathematical Research, Karelian Research Centre
of the Russian Academy of Sciences, 11~Pushkinskaya Str., Petrozavodsk 185910,
Karelia, Russian Federation;  \mbox{pavlov@krc.karelia.ru}


\label{end\stat}

\renewcommand{\bibname}{\protect\rm Литература}  