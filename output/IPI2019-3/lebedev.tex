\def\stat{lebedev}

\def\tit{НЕТРАНЗИТИВНЫЕ ТРИПЛЕТЫ НЕПРЕРЫВНЫХ СЛУЧАЙНЫХ ВЕЛИЧИН И~ИХ~ПРИЛОЖЕНИЯ}

\def\titkol{Нетранзитивные триплеты непрерывных случайных величин и~их приложения}

\def\aut{А.\,В.~Лебедев$^1$}

\def\autkol{А.\,В.~Лебедев}

\titel{\tit}{\aut}{\autkol}{\titkol}

\index{А.\,В.~Лебедев$^1$}
\index{A.\,V.~Lebedev}


%{\renewcommand{\thefootnote}{\fnsymbol{footnote}} \footnotetext[1]
%{Работа частично поддержана РФФИ (проект 18-07-00274).}}


\renewcommand{\thefootnote}{\arabic{footnote}}
\footnotetext[1]{Московский государственный университет им.\ М.\,В.~Ломоносова,
ме\-ха\-ни\-ко-ма\-те\-ма\-ти\-че\-ский факультет, 
кафедра тео\-рии вероятностей, \mbox{avlebed@уandex.ru}}

%\vspace*{-2pt}



\Abst{Изучается явление нетранзитивности отношения стохастического предшествования
 для трех независимых случайных величин с~распределениями из некоторых классов непрерывных распределений. Первоначально этот вопрос был поставлен в~связи с~задачей из теории прочности. При парных сравнениях железных брусков с~трех заводов может сложиться парадоксальная ситуация, когда бруски с~первого завода <<хуже>> брусков со второго завода, бруски со второго <<хуже>> брусков с~третьего, a~бруски с~третьего <<хуже>> брусков с~первого. В дальнейшем тема нетранзитивности
стала популярной на примере так называемых нетранзитивных (игральных) костей, однако это привело
к ее сужению на дискретные случайные величины с~конечным множеством значений.
В работе показано, что для смесей нормальных и~показательных распределений нетранзитивность возможна в~широком диапазоне параметров. Указаны характерные особенности взаимного расположения
графиков функций распределения в~этих случаях.}

\KW{нетранзитивность; нетранзитивные кости; стохастическое предшествование;
непрерывные распределения; смеси распределений}

\DOI{10.14357/19922264190304} 
  
%\vspace*{1pt}


\vskip 10pt plus 9pt minus 6pt

\thispagestyle{headings}

\begin{multicols}{2}

\label{st\stat}

\section{Введение}

В теории и~практике различные отношения превосходства между объектами часто обладают
свойством транзитивности: если~$A$ превосходит~$B$ и~$B$ превосходит~$C$, то~$A$ 
превосходит~$C$.
Это имеет место, например, если каждый объект можно характеризовать действительным 
числом и~сравнение объектов сводится к~сравнению этих чисел.

Однако бывает и~по-другому. Например, в~классической игре <<камень, ножницы, бумага>> 
<<камень>> побеждает <<ножницы>>, <<ножницы>> побеждают <<бумагу>>, но при этом 
<<бумага>> побеждает <<камень>>, a~не наоборот.
Таким образом, отношение между стратегиями в~этой игре нетранзитивно и~не 
существует оптимальной чистой стратегии.

Различным аспектам и~многочисленным примерам нетранзитивности отношений 
превосходства в~природе, технике и~обществе посвящен ряд научных 
и~на\-уч\-но-по\-пу\-ляр\-ных статей А.\,Н.~Поддьякова, в~частности~[1--4].
%\cite{Podd1, Podd15, Podd2, Podd3}.
Отмечается, что нетранзитивность часто имеет контринтуитивный характер и~может 
затруднять принятие правильных решений.

Оказывается, даже в~шахматах существуют нетранзитивные наборы позиций. 
Это значит, что каждую
позицию нельзя описать ка\-ким-то одним числом (функцией от позиции), 
по которому она лучше или хуже других, и,~как следствие, основанные на 
этой идее шахматные программы имеют ограниченные возможности.

Далее будем изучать явление нетранзитивности в~ве\-ро\-ят\-но\-ст\-но-ста\-ти\-сти\-че\-ском аспекте.
Рас\-смот\-рим три независимые случайные величины (триплет)~$X$, $Y$ и~$Z$, такие что
\begin{equation}
\label{otnxy}
\mathbf{P}(X<Y)>\fr{1}{2}
\end{equation}
и
\begin{equation}\label{otnyz}
\mathbf{P}(Y<Z)>\fr{1}{2}\,.
\end{equation}
Можно было бы ожидать, что из~(\ref{otnxy}) и~(\ref{otnyz}) следует
$$
\mathbf{P}(X<Z)>\fr{1}{2}\,,
$$
однако существуют примеры, когда это не так, a~напротив,
\begin{equation*}
%\label{otnzx}
\mathbf{P}(Z<X)>\fr{1}{2}\,.
\end{equation*}
Таким образом, отношение~(\ref{otnxy}) между случайными величинами нетранзитивно 
и~может идти по кругу, как в~игре <<камень, ножницы, бумага>>.

Отношение (\ref{otnxy}) называют 
\textit{стохастическим предшествованием} (stochastic pre\-ce\-den\-ce).
В~работах~\cite{Arc, Bols} оно применялось в~задачах статистического анализа. 
В~работе \cite{Shah} оно
использовалось в~задачах ранжирования и~называлось доминированием по вероятности. 
Отметим также недавнюю работу~\cite{Lep}, где рассматривались различные стохастические порядки, в~том числе
стохастическое предшествование.

Далее для определенности под не\-тран\-зи\-тив\-ностью будем понимать ситуацию, когда
\begin{multline*}
p_{XYZ}={}\\
{}=\min\left\{\mathbf{P}(X<Y),\mathbf{P}(Y<Z),\mathbf{P}(Z<X)\right\}>\fr{1}{2}\,.
\end{multline*}

Первой попыткой изучения этого явления стало исследование С.~Тры\-бу\-лы, 
начатое совместно с~Г.~Штейнгаузом~\cite{TrybS, Tryb}. Было показано, что
$$
\max\limits_{X,Y,Z}p_{XYZ}=\fr{\sqrt{5}-1}{2}\approx 0{,}618
$$
и максимум достигается, например, на триплете
\begin{equation}
\label{tripl0}
\left.
\begin{array}{l}
X=\left\{\begin{array}{ll}
1 & \mbox{ с~вероятностью } p\,;\\[6pt]
4 & \mbox{ с~вероятностью } 1-p\,;
\end{array}\right.\\[12pt]
Y=2\,;\\[6pt]
Z=\left\{\begin{array}{ll}
0 & \mbox{ с~вероятностью } 1-p\,;\\[6pt]
3 & \mbox{ с~вероятностью } p\,,
\end{array}\right.
\end{array}
\right\}
\end{equation}
где

\noindent
$$
p=\fr{\sqrt{5}-1}{2}\,,
$$
тогда

\noindent
$$
\mathbf{P}(X<Y)=\mathbf{P}(Y<Z)=\mathbf{P}(Z<X)=\fr{\sqrt{5}-1}{2}\,.
$$

Далее в~работах З.~Усыскина~\cite{Usys} и~С.~Тры\-булы~\cite{Tryb2}
 был рассмотрен случай $n$ независимых случайных величин, получены некоторые оценки максимума вероятностей, однако явное выражение было
лишь недавно выведено И.\,И.~Богдановым \cite{Bogd}:

\noindent
\begin{multline*}
%\label{otnn}
\max\limits_{X_1,\dots, X_n}\min\left\{{\bf P}\left(X_1<X_2\right),\ldots\right.\\
\left.\ldots,\mathbf{P}
\left(X_{n-1}<X_n\right),\mathbf{P}\left(X_n<X_1\right)\right\}={}\\ 
{}=1-\left(4\cos^2\fr{\pi}{n+2}\right)^{-1},\enskip n\ge 3\,.
\end{multline*}

В качестве приложения в~\cite{Tryb} речь шла о прочности материалов. 
Пусть в~лаборатории сравнивают попарно на прочность железные бруски одинакового 
размера и~формы, помещая их в~одну рамку (тиски) и~прилагая к~ним одинаковую силу 
путем закручивания винта, пока один из брусков не сломается. Предположим, 
что бруски производятся на трех разных заводах (которые дают разное распределение
прочности) и~сравниваются бруски c первого и~второго, второго и~третьего, 
первого и~третьего заводов. Тогда теоретически может сложиться парадоксальная 
ситуация, что бруски с~первого завода <<хуже>> (т.\,е.\ чаще ломаются раньше) 
брусков со второго завода, бруски со второго <<хуже>> брусков с~третьего, 
а~бруски с~третьего <<хуже>> брусков с~первого.

Понятно, что конкретно в~приведенном примере можно поступить иначе: не 
сравнивать проч\-ности брусков друг с~другом, a~измерять их явно, затем выбирать 
лучший завод по ка\-кой-то важной на практике числовой характеристике распределения 
(по которой они достаточно заметно различаются). Однако в~приложениях бывают 
и~ситуации, когда ка\-кие-то величины просто не измеряются в~явном виде, 
и~нет другого способа выявить соотношения объектов или их совокупностей, 
кроме парных сравнений.

Например, в~биологии речь может идти о парных взаимодействиях животных в~борьбе 
за пищу, территорию, размножение или доминирование в~группе. Аналогичным образом 
при сравнительном анализе работы
алгоритмов на различных наборах данных или соревнованиях компьютерных программ 
могут быть
победители в~парах, но не быть лучшего в~группе из трех или более объектов. 
Для людей речь может идти как о~соревнованиях, так и~о~сравнительных оценках 
различных товаров и~услуг, голосовании и~др. Проблема нетранзитивности 
предпочтений здесь была известна с~XVIII~в.\ (\textit{парадокс Кондорсе}).

Разумеется, человеческий интеллект может \mbox{найти} ка\-кой-то выход из подобной
 ситуации, однако в~случае искусственного интеллекта (компьютерных программ) 
 проявления нетранзитивности могут привести машину в~тупик, если не будут 
 преду\-смот\-ре\-ны программистом, что необходимо учитывать в~разработках.

Тема нетранзитивности стохастического предшествования приобрела популярность 
на примере так называемых \textit{нетранзитивных костей} (nontransitive, 
intransitive dice).
Имеются в~виду наборы игральных костей, на грани которых нанесены чис\-ла таким образом,
чтобы создать нетранзитивные отношения соответствующих случайных величин.

Нетранзитивные кости были популяризованы М.~Гарднером~\cite{Gard1, Gard15},
им посвящена обширная литература, например работы~[16--19].
%\cite{Sav, Boz, Con, Buh}.
К~сожалению, при этом произошло, во-пер\-вых, сужение темы до дискретных
случайных величин с~конечным чис\-лом значений, во-вто\-рых, сложилось 
не слишком серьезное отношение
со стороны ученых, зачастую относящих эту тему к~игровой, развлекательной математике.

Однако изначально С.~Трыбулой была затронута вполне серьезная 
тео\-ре\-ти\-ко-ве\-ро\-ят\-ност\-ная проб\-ле\-ма с~возможными приложениями 
на практике. Прочность материалов, конечно, описывается непрерывным 
распределением, как и~многие другие характеристики объектов в~природе, 
технике и~обществе.
И при попарных сравнениях объектов из нескольких разных совокупностей 
(категорий) теоретически может
возникать нетранзитивность. Таким образом, вполне закономерно задаться 
вопросом, в~каких классах непрерывных распределений она бывает, а~в~каких нет.

В \cite[теорема 2]{Tryb} было показано, что если распределения случайных величин~$X$, 
$Y$ и~$Z$ относятся к~одному сдвиговому семейству, то не\-тран\-зи\-тив\-ности быть не может. 
Понятно,
что этот вывод также распространяется на масштабные семейства распределений 
на положительной
или отрицательной полупрямой (путем логарифмирования).

В частности, прочность материалов часто описывается распределением Вейбулла
\begin{equation*}
%\label{weib}
F(x)=1-\exp\left\{-\left(\fr{x-a}{b}\right)^\alpha\right\},\enskip
x\ge a\,,\ a,b,\alpha>0\,,
\end{equation*}
предложенным еще в~книге~\cite{Weib}. Таким образом, сразу можно сделать вывод, 
что если случайные величины (прочности железных брусков с~разных заводов) имеют распределение
Вейбулла с~постоянными~$a$ и~$\alpha$ и~разными~$b$ или с~постоянными~$b$ и~$\alpha$ 
и~разными~$a$, то нетранзитивность невозможна.

В работе автора~\cite{Leb-2019} выведены некоторые новые критерии, когда 
нетранзитивности быть не может. Показано, что в~случае распределения Вейбулла 
с~постоянными~$a$ и~$b$ и~разными~$\alpha$ нетранзитивность также невозможна. 
Рассмотрен класс распределений с~полиномиальной плотностью на единичном отрезке, 
где возможность нетранзитивности зависит от степени многочлена.

Далее будет приведена одна общая схема построения нетранзитивных
 триплетов и~содержательные
примеры в~классах конечных смесей нормальных и~показательных распределений.

Конечные смеси распределений возникают в~различных приложениях, проблемам 
их статистического
анализа посвящена обширная литература. Из недавних работ отметим~[22--27],
%\cite{Bat, Kriv, Ben, Gorsh, Kor, Gorsh2}
где смеси нормальных распределений рассматривались в~[23--27],
%\cite{Kriv, Ben, Gorsh, Kor, Gorsh2}
гам\-ма-рас\-пре\-де\-ле\-ний~--- в~\cite{Bat, Ben, Gorsh2}, показательных распределений~---
 в~\cite{Bat}.

\section{Когда нетранзитивность возможна}

Проведем обобщение базовой схемы~(\ref{tripl0}), заменив~0, 1, 2, 3 и~4 на независимые 
непрерывные
случайные величины~$\xi_0$, $\xi_1$, $\xi_2$, $\xi_3$ и~$\xi_4$, такие что
для вероятностей $p_{kl}\hm={\bf P}(\xi_k\hm<\xi_l)$ верно $p_{kl}\hm>1/2$ при 
$k\hm<l$ и~эти вероятности убывают по~$k$ и~возрастают по~$l$.

Положим
\begin{align*}
%\left.
%\begin{array}{l}
X&=\left\{\begin{array}{ll}
\xi_1 & \mbox{ с~вероятностью } r\,,\\
\xi_4 & \mbox{ с~вероятностью } 1-r\,,
\end{array}\right. 0\le r\le 1\,;\\
Y&=\xi_2;\\
Z&=\left\{\begin{array}{ll}
\xi_0 & \mbox{ с~вероятностью } 1-s\,,\\
\xi_3 & \mbox{ с~вероятностью } s\,,
\end{array}\right. 0\le s\le 1\,.
%\end{array}
%\right\}
%\label{tripl1}
\end{align*}

Предположим, что $p_{kl}$ фиксированы, а~$r$ и~$s$ можно варьировать. Тогда
\begin{align*}
p_{XY}(r)&={\bf P}(X<Y)=rp_{12}+(1-r)p_{42};\\ 
p_{YZ}(s)&={\bf P}(Y<Z)=(1-s)p_{20}+sp_{23};\\
p_{ZX}(r,s)&={\bf P}(Z<X)=r(1-s)p_{01}+{}\\
&{}+(1-r)(1-s)p_{04}+rsp_{31}+(1-r)sp_{34}.
\end{align*}

Обозначим также
\begin{align*}
p_{XYZ}(r,s)&=\min\left\{p_{XY}(r),p_{YZ}(s),p_{ZX}(r,s)\right\};\\
p_{XYZ}^{\max}&=\max\limits_{0\le r,s\le 1}p_{XYZ}(r,s)\,.
\end{align*}

Максимизация $p_{XYZ}(r,s)$ в~общем случае ведет к~множеству 
вариантов и~громоздким выражениям,
поэтому далее рассмотрим случай, когда~$p_{kl}$ зависят только от $l\hm-k$, 
обозначим $p_{l-k}\hm=p_{kl}$.
Тогда
\begin{align*}
p_{01}=p_{12}=p_{23}=p_{34}&=p_1;\\
p_{42}=p_{20}=p_{31}=p_{-2}&=1-p_2;\\
p_{04}&=p_4;\\
&\hspace*{-28mm}\fr{1}{2}<p_1<p_2<p_4\le 1,
\end{align*}
в силу симметрии можно положить $r\hm=s$ и~формулы упрощаются до
\begin{equation}
\left.
\begin{array}{rl}
p_{XY}(r)&=rp_1+(1-r)\left(1-p_2\right);\\[6pt]
p_{ZX}(r)&=2r(1-r)p_1+(1-r)^2p_4+{}\\[6pt]
&\hspace*{22mm}{}+r^2\left(1-p_2\right);\\[6pt]
p_{XYZ}(r)&=\min\left\{p_{XY}(r), p_{ZX}(r)\right\}.
\end{array}\!\!
\right\}\!\!
\label{pxyz}
\end{equation}

\begin{figure*}[b] %fig1
 \vspace*{1pt}
    \begin{center}  
  \mbox{%
 \epsfxsize=163mm 
 \epsfbox{leb-1.eps}
 }
\end{center}
\vspace*{-9pt}
\Caption{График $p_{XYZ}^{\max}$ к~примерам~1~~(\textit{а})  и~2~(\textit{б})}
%\label{Fig}
\end{figure*}


\noindent
\textbf{Замечание~1.} $p_{XY}(r)$ линейно возрастает от $1\hm-p_2\hm<1/2$ до 
$p_1\hm>1/2$, a~$p_{ZX}(r)$~---
квадратичная функция, принимающая значения между $p_4\hm>p_1\hm>1/2$ ($r\hm=0$) 
и~$1\hm-p_2\hm<1/2$ ($r\hm=1$),
а~значит, не имеющая локальных экстремумов на $(0,1)$ и~убывающая от 
$p_4\hm>1/2$ до $1\hm-p_2\hm<1/2$.

\smallskip

\noindent
\textbf{Теорема~1.}\ \textit{Нетранзитивность в~модели}~(\ref{pxyz}) 
\textit{возможна тогда и~только тогда, когда}
\begin{equation*}
%\label{teo2}
p_{ZX}\left(\fr{p_2-1/2}{p_1+p_2-1}\right)>\fr{1}{2}\,.
\end{equation*}

\begin{figure*} %fig2
 \vspace*{1pt}
    \begin{center}  
  \mbox{%
 \epsfxsize=163mm 
 \epsfbox{leb-3.eps}
 }
\end{center}
\vspace*{-9pt}
\Caption{Графики функций распределения к~примерам~1~(\textit{а})
и~2~(\textit{б})}
%\label{Fig}
\end{figure*}


\noindent
Д\,о\,к\,а\,з\,а\,т\,е\,л\,ь\,с\,т\,в\,о\,.\ \
 Функция~$p_{XY}(r)$ возрастает и~достигает значения~1/2 в~точке
$$
r^*=\fr{p_2-1/2}{p_1+p_2-1}\,.
$$
Необходимым и~достаточным условием наличия нетранзитивности является $p_{ZX}(r^*)\hm>1/2$.
Действительно, тогда в~некоторой правой окрест\-ности~$r^*$ верно $p_{XY}(r)\hm>1/2$ 
и~$p_{ZX}(r)\hm>1/2$. Если же $p_{ZX}(r^*)\hm\le 1/2$, то далее $p_{ZX}(r)\hm<1/2$, 
поскольку~$p_{ZX}(r)$ убывает.\hfill
$\square$

\smallskip

\noindent
\textbf{Теорема~2.}\ \textit{Максимум~$p_{XYZ}(r)$ достигается в~точке 
$0\hm<r_{\max}\hm<1$, являющейся
единственным на $(0,1)$ корнем уравнения}
$$
Ar^2+Br+C=0,
$$
\textit{где}
\begin{align*}
A&=2p_1+p_2-p_4-1\,;\\
B&=2p_4+p_2-p_1-1\,;\\ 
C&=1-p_2-p_4.
\end{align*}


\noindent
Д\,о\,к\,а\,з\,а\,т\,е\,л\,ь\,с\,т\,в\,о\,.\ \
В силу замечания~1
максимум минимума функций~$p_{XY}(r)$ и~$p_{ZX}(r)$ достигается в~единственной 
точке их равенства:
\begin{multline*}
rp_1+(1-r)(1-p_2)={}\\
{}=2r(1-r)p_1+(1-r)^2p_4+r^2(1-p_2),
\end{multline*}
откуда и~получаем уравнение.\hfill$\square$

\smallskip

\noindent
\textbf{Пример~1.} Пусть~$\xi_k$ имеют нормальное распределение со средними~$k$ 
и~дисперсией~$\sigma^2$, тогда
$$
p_m=\Phi\left(\fr{m}{\sigma\sqrt{2}}\right),\enskip m=1, 2, 4\,.
$$

На рис.~1,\,\textit{а} пред\-став\-лен график~$p_{XYZ}^{\max}$, построенный в~соответствии с~расчетами 
по теореме~2.
Фактически, нетранзитивность наблюдается при \textit{любых} $\sigma\hm>0$. 
При изменении~$\sigma$ от~$0$ до~$+\infty$ вероятность~$r_{\max}$ 
пробегает от~$(\sqrt{5}\hm-1)/2$ до~$2/3$ и~$p_{XYZ}^{\max}$ от~$(\sqrt{5}\hm-1)/2$
до~$1/2$.



В частности, при $\sigma\hm=1$ получаем $r_{\max}\hm\approx 0{,}638$,
 $p_{XYZ}^{\max}\hm\approx 0{,}514$. На рис.~2,\,\textit{а}
для этого случая пред\-став\-ле\-ны графики функций распределения~$X$, 
$Y$ и~$Z$.



\smallskip

\noindent
\textbf{Пример~2.}\ Пусть~$\xi_k$ имеют показательное распределение со 
средними~$\mu^k$, $\mu\hm>1$, тогда
$$
p_m=1-\fr{1}{1+\mu^m}\,,\enskip m=1, 2, 4\,.
$$

На рис.~1,\,\textit{б} пред\-став\-лен график~$p_{XYZ}^{\max}$.
Фактически, нетранзитивность наблюдается при \textit{любых} $\mu\hm>1$.
При изменении~$\mu$ от~$1$ до~$+\infty$ вероятность~$r_{\max}$ пробегает от~$2/3$ 
до~$(\sqrt{5}\hm-1)/2$
и~$p_{XYZ}^{\max}$ от~$1/2$ до~$(\sqrt{5}\hm-1)/2$.




В частности, при $\mu\hm=3$ получаем $r_{\max}\hm\approx 0{,}639$, 
$p_{XYZ}^{\max}\hm\approx 0{,}515$.
На рис.~2,\,\textit{б} для этого случая пред\-став\-ле\-ны графики функций 
распределения~$X$, $Y$ и~$Z$.

\smallskip

\noindent
\textbf{Замечание~2.}\ Обозначим $d_m\hm=p_m\hm-1/2$, $m\hm=1, 2, 4$. 
В~обоих примерах имеют место соотношения $d_4\hm\sim 4d_1$, $d_2\hm\sim 2d_1$ при
$d_1\hm\to 0$ (когда $\sigma\hm\to\infty$ в~примере~1 или $\mu\hm\to 1$ в~примере~2). 
По теореме~2 отсюда следует
$r_{\max}\hm\to 2/3$ и~$p_{XYZ}^{\max}\hm\to 1/2$.

\smallskip

\noindent
\textbf{Замечание~3.}\ 
В~обоих примерах на рис.~2 видно, что значения функций распределения одинаковым 
образом меняют свое взаимное расположение (четыре раза), графики~$F_X$ и~$F_Z$ 
имеют две точки
пересечения, a~график~$F_Y$ пересекается с~каждым из них по одному разу. 
Аналогичное явление
было отмечено и~в~\cite{Leb-2019}. Можно предположить, что такая картина 
характерна при нетранзитивности.

%\vspace*{-6pt}

\section{Заключение}

%\vspace*{-2pt}

В работе показано, что для смесей нормальных и~показательных распределений нетранзитивность
возможна в~широком диапазоне параметров. Учитывая инвариантность относительно  любого непрерывного строго возрастающего
преобразования (применяемого ко всем случайным величинам) этот вывод можно распространить на
смеси логнормальных распределений, распределений Вейбулла, Парето и~др.

Возвращаясь к~примеру С.~Трыбулы по проч\-ности брусков, если допустить, что хотя бы на
двух из трех заводов имеются не менее двух цехов или смен, выпускающих бруски с~заметно разным
распределением прочности (Вейбулла), нетранзитивность теоретически исключать нельзя.

Можно сделать вывод, что появлению нетранзитивности могут способствовать внутренние
неоднородности совокупностей, из которых производятся выборки, что и~проявляется статистически
как смешивание классических распределений, имеющих различные параметры.

%\vspace*{-6pt}


{\small\frenchspacing
 {%\baselineskip=10.8pt
 \addcontentsline{toc}{section}{References}
 \begin{thebibliography}{99}
 
 %\vspace*{-6pt}
 
\bibitem{Podd1} 
\Au{Поддьяков А.\,Н.} Непереходность (нетранзитивность) отношений
превосходства и~принятие решений~// Психология.~Ж. ВШЭ, 2006. №\,3. С.~88--111.

\bibitem{Podd15} 
\Au{Пермогорский М.\,С., Поддьяков~А.\,Н.} Отношение превосходства между объектами
и нетранзитивность их предпочтений человеком~// 
Вопросы психологии, 2014. №\,2. С.~3--14.
\bibitem{Podd2} 
\Au{Поддьяков А.\,Н.} Нетранзитивность~--- кладезь для изобретателей~//
Троицкий вариант,  21.11.2017. №\,242.
\bibitem{Podd3} 
\Au{Poddiakov~A.} Intransitive machines~// 
arXiv.org, 2018. ArXiv:1809.03869 [math.HO]. 11~p.

\bibitem{Arc}
\Au{Arcones M.\,A., Kvam P.\,H., Samaniego~F.\,J.} Nonparametric estimation of 
a~distribution subject to a~stochastic precedence constraint~// 
J.~Am. Stat. Assoc., 2002. Vol.~97. No.\,457. P.~170--182.
\bibitem{Bols}
\Au{Boland P.\,J., Singh H., Cukic~B.} 
The stochastic precedence ordering with applications
in sampling and testing~// J.~Appl. Probab., 2004. Vol.~41. No.\,1. P.~73--82.
\bibitem{Shah}
\Au{Шахнов И.~Ф.} Задачи ранжирования интервальных величин при многокритериальном
анализе сложных систем~// Изв. РАН. ТиСУ, 2008. №\,1. C.~37--44.
\bibitem{Lep}
\Au{Лепский А.\,Е.} 
Стохастическое и~нечеткое упорядочивание методом минимальных преобразований~// 
Автоматика и~телемеханика, 2017. №\,1. С.~59--79.
\bibitem{TrybS} 
\Au{Steinhaus H., Trybula~S.} 
On a~paradox in applied probabilities~// B.~Acad. Pol.
 Sci., 1959. Vol.~7. P.~67--69.
\bibitem{Tryb} %10
\Au{Trybula~S.} On the paradox of three random variables~// Zastos. Matem., 
1961. Vol.~5. No.\,4. P.~321--332.

\bibitem{Usys}  %11
\Au{Usyskin Z.} Max--min probabilities in the voting paradox~// 
Ann. Math. Stat., 1964. Vol.~35. No.\,2. P.~857--862.

\bibitem{Tryb2} %12
\Au{Trybula S.} 
On the paradox of $n$ random variables~// Zastos. Matem.,
1965. Vol.~8. No.\,2. P.~143--156.

\bibitem{Bogd} 
\Au{Богданов И.\,И.} Нетранзитивные рулетки~// Матем. просв., 2010. Сер.~3. Вып.~14. 
С.~240--255.
\bibitem{Gard1} 
\Au{Gardner M.} The paradox of the nontransitive dice and the
elusive principle of indifference~// Sci. Am., 1970. Vol.~223. No.\,6. P.~110--114.
\bibitem{Gard15} 
\Au{Gardner M.}  On the paradoxical situations that arise from 
nontransitive relations~//
Sci. Am., 1974. Vol.~231. No.\,6. P.~120--125.
\bibitem{Sav} 
\Au{Savage R.} The paradox of nontransitive dice~// Am. Math. Mon., 1994.
Vol.~101. No.\,5. P.~429--436.
\bibitem{Boz} 
\Au{Bozoki S.} Nontransitive dice sets releazing the Paley tournament for 
solving Sh$\ddot{\mbox{u}}$tte's
tournament problem~// Miskolc Math. Notes, 2014. Vol.~15. No.\,1. P.~39--50.
\bibitem{Con} 
\Au{Conrey B., Gabbard~J., Grant~K., Liu~A., Morrison~K.\,E.} 
Intransitive dice~// Math. Mag., 2016. Vol.~89. P.~133--143.
{\looseness=1

}
\bibitem{Buh} 
\Au{Buhler I., Graham~R., Hales~A.} Maximally nontransitive dice~// 
Am. Math. Mon., 2018. Vol.~125. No.\,5. P.~387--399.
\bibitem{Weib}
\Au{Weibull~W.} A~statistical theory of the strength of materials.~--- 
Stockholm: Generalstabens litografiska anstalts f$\ddot{\mbox{o}}$rlag, 1939. 45~p.
\bibitem{Leb-2019} 
\Au{Лебедев А.\,В.} Проблема нетранзитивности для трех независимых
случайных величин~// Автоматика и~телемеханика, 2019. №\,6. С.~91--103.
\bibitem{Bat} 
\Au{Батракова Д.\,А., Королев В.\,Ю., Шоргин С.\,Я.} 
Новый метод ве\-ро\-ят\-но\-ст\-но-ста\-ти\-сти\-че\-ско\-го 
анализа информационных потоков в~телекоммуникационных сетях~// Информатика и~её 
применения, 2007. Т.~1. Вып.~1. C.~40--53.
\bibitem{Kriv} 
\Au{Кривенко М.\,П.} 
Расщепление смеси вероятностных распределений на две составляющие~// Информатика 
и~её применения, 2008. Т.~2. Вып.~4. С.~48--56.
\bibitem{Ben} 
\Au{Бенинг~В.\,Е., Горшенин~А.\,К., Королев~В.\,Ю.}
 Асимптотически оптимальный критерий проверки гипотез о~чис\-ле 
 компонент смеси вероятностных распределений~// Информатика и~её 
применения, 2011. Т.~5. Вып.~3. С.~4--16.
\bibitem{Gorsh} 
\Au{Горшенин~А.\,К.} Об устойчивости сдвиговых смесей 
нормальных законов по отношению к~изменениям смешивающего распределения~// 
Информатика и~её 
применения, 2012. T.~6. Вып.~2. С.~22--28.
\bibitem{Kor} 
\Au{Королев~В.\,Ю., Горшенин~А.\,К., Гулев~С.\,К., Беляев~К.\,П.} 
Статистическое моделирование турбулентных потоков тепла между океаном 
и~атмосферой с~по\-мощью
метода скользящего разделения конечных нормальных смесей~// 
Информатика и~её применения, 2015. Т.~9. Вып.~4.
C.~3--13.
\bibitem{Gorsh2} 
\Au{Горшенин~А.\,K.} 
Зашумление данных конечными смесями нормальных и~гам\-ма-рас\-пре\-де\-ле\-ний 
с~применением к~задаче округления наблюдений~//
Информатика и~её применения, 2018. Т.~12. Вып.~3. С.~28--34.
 \end{thebibliography}

 }
 }

\end{multicols}

\vspace*{-6pt}

\hfill{\small\textit{Поступила в~редакцию 03.10.18}}

\vspace*{8pt}

%\pagebreak

%\newpage

%\vspace*{-28pt}

\hrule

\vspace*{2pt}

\hrule

%\vspace*{-2pt}

\def\tit{NONTRANSITIVE TRIPLETS OF~CONTINUOUS RANDOM VARIABLES AND~THEIR~APPLICATIONS}


\def\titkol{Nontransitive triplets of~continuous random variables and~their~applications}

\def\aut{A.\,V.~Lebedev}

\def\autkol{A.\,V.~Lebedev}

\titel{\tit}{\aut}{\autkol}{\titkol}

\vspace*{-11pt}


\noindent
Faculty of Mechanics and Mathematics,
M.\,V.~Lomonosov Moscow State University, Main Building, 1~Leninskiye Gory,
Moscow 119991, Russian Federation


\def\leftfootline{\small{\textbf{\thepage}
\hfill INFORMATIKA I EE PRIMENENIYA~--- INFORMATICS AND
APPLICATIONS\ \ \ 2019\ \ \ volume~13\ \ \ issue\ 3}
}%
 \def\rightfootline{\small{INFORMATIKA I EE PRIMENENIYA~---
INFORMATICS AND APPLICATIONS\ \ \ 2019\ \ \ volume~13\ \ \ issue\ 3
\hfill \textbf{\thepage}}}

\vspace*{6pt}   



\Abste{The phenomenon of nontransitivity of the stochastic precedence relation 
for three independent random variables with distributions from some classes 
of continuous distributions is studied. Initially, this question was posed 
in connection with the application in strength theory. With paired comparisons 
of iron bars from three factories, a paradoxical situation may arise when 
the bars from the first factory are ``worse'' than the bars from the second 
factory, the bars from the second factory are ``worse'' than the bars from the 
third factory, and the bars from the third factory are ``worse'' 
than the bars from the first factory. Further, the nontransitivity topic gained 
popularity for the example of the so-called nontransitive dice; however, this 
led to its narrowing down to discrete random variables with finite sets of 
values. The paper presents that for mixtures of normal and exponential 
distributions, nontransitivity is possible in a wide range of parameters. 
Specific features of the mutual arrangement of the graphs of the distribution 
functions in these cases are indicated.}

\KWE{nontransitivity; nontransitive dice; stochastic precedence; continuous distributions;
mixtures of distributions}

\DOI{10.14357/19922264190304} 

%\vspace*{-14pt}

%\Ack
%   \noindent



\vspace*{6pt}

  \begin{multicols}{2}

\renewcommand{\bibname}{\protect\rmfamily References}
%\renewcommand{\bibname}{\large\protect\rm References}

{\small\frenchspacing
 {%\baselineskip=10.8pt
 \addcontentsline{toc}{section}{References}
 \begin{thebibliography}{99}
 
 \vspace*{-2pt}
 
\bibitem{Podd1-1} 
\Aue{Poddiakov, A.\,N.} 
2006. Neperekhodnost' (netranzitivnost') otnosheniy prevoskhodstva
i~prinyatie re\-she\-niy [Intransitive character of superiority relations 
and decision-making].
\textit{Psikhologia. Zh. VShE} [Psychology. J.~HSE] 3(3):88--111.

\bibitem{Podd15-1} 
\Aue{Permogorskiy, M.\,S., and A.\,N.~Poddiakov.}
2014. Otnoshenie prevoskhodstva mezhdu
ob''ektami i~netranzitivnost' ikh predpochteniy chelovekom 
[The relation of superiority between objects and the nontransitivity of 
their preferences by human]. \textit{Voprosy psikhologii} [Psychology issues]
 2:3--14.
 
\bibitem{Podd2-1} 
\Aue{Poddiakov, A.} November~21, 2017. Netranzitivnost'~--- kladez' dlya izobretateley 
[Nontransitivity is a~treasure for inventors]. 
\textit{Troitskiy variant} [Troitsk Variant]  242.

\bibitem{Podd3-1} 
\Au{Poddiakov, A.} 2018. Intransitive machines. 
Available at: {\sf https://arxiv.org/abs/1809.03869}
(accessed September~8, 2018).


\bibitem{Arc-1} 
\Aue{Arcones, M.\,A., P.\,H.~Kvam, and F.\,J.~Samaniego}. 
2002. Nonparametric estimation of a
distribution subject to a~stochastic precedence constraint. 
\textit{J.~Am. Stat. Assoc.} 97(457):170--182.
\bibitem{Bols-1} 
\Aue{Boland, P.\,J., H.~Singh H., and B.~Cukic.} 2004. 
The stochastic precedence ordering with applications
in sampling and testing. \textit{J.~Appl. Probab.} 41(1):73--82.

\bibitem{Shah-1} 
\Aue{Shakhnov, I.\,F.} 2008. 
A~problem of ranking interval objects in a~multicriteria analysis 
of complex systems. \textit{J.~Comput. Sys. Sc. Int.} 47(1):33--39.

\bibitem{Lep-1} 
\Aue{Lepskiy, A.\,E.} 2017. 
Stochastic and fuzzy ordering with the method of minimal transformations.
\textit{Automat. Rem. Contr.} 78(1):50--66.

\bibitem{TrybS-1} 
\Aue{Steinhaus, H., and S.~Trybula.} 1959. 
On a~paradox in applied probabilities. \textit{B.~Acad. Pol.
Sci.} 7:67--69.
\bibitem{Tryb-1} %10
\Aue{Trybula, S.} 1961. On the paradox of three random variables. 
\textit{Zastos. Matem.} 5(4):321--332.


\bibitem{Usys-1} %11
\Aue{Usyskin, Z.} 1964. Max--min probabilities in the voting paradox. 
\textit{Ann. Math. Stat.} 35(2):857--862.

\bibitem{Tryb2-1} %12
\Aue{Trybula, S.} 1965. On the paradox of $n$ random variables. 
\textit{Zastos. Matem.} 8(2):143--156.

\bibitem{Bogd-1} 
\Aue{Bogdanov, I.\,I.} 2010. Netranzitivnye ruletki 
[Nontransitive roulettes]. \textit{Mat. Pros.}
[Math. Enlight.] 14:240--255.

\bibitem{Gard1-1} 
\Aue{Gardner, M.} 1970. The paradox of the nontransitive dice and the
elusive principle of indifference. \textit{Sci. Am.} 223(6):110--114.
\bibitem{Gard15-1} 
\Aue{Gardner, M.} 1974. On the paradoxical situations that arise from 
nontransitive relations. \textit{Sci. Am.} 231(6):120--125.
\bibitem{Sav-1} 
\Aue{Savage, R.} 1994. The paradox of nontransitive dice. \textit{Am. Math. Mon.}
101(5):429--436.
\bibitem{Boz-1} 
\Aue{Bozoki, S.} 2014. Nontransitive dice sets releazing the 
Paley tournament for solving Sh$\ddot{\mbox{u}}$tte's tournament problem. 
\textit{Miskolc Math. Notes} 15(1):39--50.
\bibitem{Con-1} 
\Aue{Conrey, B., J.~Gabbard, K.~Grant, A.~Liu, and K.\,E.~Morrison.}
2016. Intransitive dice.
\textit{Math. Mag.} 89:133--143.
\bibitem{Buh-1} 
\Aue{Buhler, I., R.~Graham, and A.~Hales.}
 2018. Maximally nontransitive dice. \textit{\it Am. Math. Mon.}
125(5):387--399.

\bibitem{Weib-1} 
\Aue{Weibull, W.} 1939. \textit{A~statistical theory of the strength of materials.} 
Stockholm: Generalstabens litografiska anstalts f$\ddot{\mbox{o}}$rlag. 45~p.

\bibitem{Leb-2019-1} 
\Aue{Levedev, A.~V.}
2019. Nontransitivity problem for three continuos random variables.
\textit{Automat. Rem. Contr.} 80(6): 1025--1036.

\bibitem{Bat-1} 
\Aue{Batrakova, D.\,A., V.\,Yu.~Korolev, and S.\,Ya.~Shorgin.}
 2007. Novyy metod veroyatnostno-statisticheskogo ana\-li\-za informatsyonnykh 
 potokov v~telekommunikatsionnykh setyakh
[A new method for the probabilistic and statistical analysis 
of information flows in telecommunication networks].
\textit{Informatika i~ee Primeneniya~--- Inform. Appl.} 1(1):40--53.

\bibitem{Kriv-1} 
\Aue{Krivenko, M.\,P.} 2008. 
Rasshcheplenie smesi veroyatnostnykh raspredeleniy na dve sostavlyayushchie
[Splitting of distribution mixture in two components] 
\textit{Informatika i~ee Primeneniya~--- Inform. Appl.} 2(4):48--56.
\bibitem{Ben-1} 
\Aue{Bening, B.\,E., A.\,K.~Gorshenin, and V.\,Yu.~Korolev.} 
2011. Asimptoticheski optimal'nyy kriteriy proverki gipotez 
o~chisle komponent smesi veroyatnostnykh raspredeleniy
 [An asymptotically optimal test for the number of components of 
 a~mixture of probability distributions]. 
 \textit{Informatika i~ee Primeneniya~--- Inform. Appl.} 5(3):4--16.
\bibitem{Gorsh-1} 
\Aue{Gorshenin, A.\,K.} 2012. 
Ob ustoychivosti sdvigovyhh smesey normal'nykh zakonov po otnosheniu
k~izmeneniyam smeshivayushchego raspredeleniya [On stability of normal 
location mixtures with respect to variations in mixing distribution]. 
\textit{Informatika i~ee Primeneniya~--- Inform. Appl.} 6(2):22--28.
\bibitem{Kor-1} 
\Aue{Korolev, V.\,Yu., A.\,K.~Gorshenin, S.\,K.~Gulev, and K.\,P.~Belyaev.} 
2015. Statisticheskoe
modelirovanie turbulentnykh potokov tepla mezhdu okeanom i~at\-mo\-sfe\-roy 
s~pomoshch'yu metoda skol'zyashchego
razdeleniya konechnykh normal'nykh smesey 
[Statistical modeling of air--sea turbulent heat fluxes by the method 
of moving separation of finite normal mixtures]. 
\textit{Informatika i~ee Primeneniya~--- Inform. Appl.} 9(4):3--13.
\bibitem{Gorsh2-1} 
\Aue{Gorshenin, A.\,K.} 2018. Zashumlenie dannykh konechnymi smesyami normal'nykh 
i~gamma-raspredeleniy s~primeneniem k~zadache okrugleniya nablyudeniy
[Data noising by finite normal and gamma mixtures with application 
to the problem of rounded observations].
\textit{Informatika i~ee Primeneniya~--- Inform. Appl.} 12(3):28--34.
\end{thebibliography}

 }
 }

\end{multicols}

%\vspace*{-7pt}

\hfill{\small\textit{Received October 3, 2018}}

%\pagebreak

%\vspace*{-22pt}

\Contrl

\noindent
\textbf{Lebedev Alexey V.} (b.\ 1971)~--- Doctor of Science in physics and mathematics,
associate professor, Department of Probability Theory, 
Faculty of Mechanics and Mathematics,
M.\,V.~Lomonosov Moscow State University, Main Building, 1~Leninskiye Gory,
Moscow 119991, Russian Federation; \mbox{avlebed@yandex.ru}
\label{end\stat}

\renewcommand{\bibname}{\protect\rm Литература}  