\def\stat{borodina}

\def\tit{ОБ  ОЦЕНИВАНИИ ЭФФЕКТИВНОЙ ПРОПУСКНОЙ СПОСОБНОСТИ
 СИСТЕМЫ С РЕГЕНЕРАТИВНЫМ\\ ВХОДНЫМ ПРОЦЕССОМ$^*$}

\def\titkol{Об  оценивании эффективной пропускной способности
 системы с регенеративным входным процессом}

\def\autkol{А.\,В.~Бородина,  Е.\,В.~Морозов}

\def\aut{А.\,В.~Бородина$^1$,  Е.\,В.~Морозов$^2$}

\titel{\tit}{\aut}{\autkol}{\titkol}

{\renewcommand{\thefootnote}{\fnsymbol{footnote}}\footnotetext[1]
{Работа выполнена при финансовой поддержке Программы
стратегического развития ПетрГУ   в рамках реализации комплекса
мероприятий  по развитию научно-исследовательской деятельности.}}

\renewcommand{\thefootnote}{\arabic{footnote}}
\footnotetext[1]{Институт прикладных
математических исследований Карельского научного центра Российской
академии наук; Петрозаводский государственный университет, borodina@krc.karelia.ru}
\footnotetext[2]{Институт прикладных математических исследований
Карельского научного центра Российской академии наук; Петрозаводский
государственный университет, emorozov@karelia.ru}


\Abst{Рассматривается  понятие эффективной
пропускной способности (ЭПС) коммуникационного узла, которая
гарантирует, что вероятность потери или превышения стационарной
нагрузкой некоторого уровня ограничена заданной (малой) величиной.
Показано, как вычисляется ЭПС в    жидкостной системе обслуживания в
случае входного процесса  с независимыми приращениями.  Далее
рассматривается жидкостная система с регенеративным входным
процессом. Для вычисления ЭПС ключевым является нахождение
предельной логарифмической экспоненциальной функции моментов
входного процесса. С использованием эвристических соображений
получена аппроксимация этого предела, которая  выражена в терминах
моментов длины цикла регенерации и величины работы, поступающей  в
систему в течение цикла.
Результаты численного моделирования ряда систем с
 регенеративным входным процессом показывают вполне удовлетворительную точность
оценивания вероятности потери в случае, когда  в системе
используется оценка ЭПС, получаемая на основе найденной
аппроксимации.}

\KW{система с постоянной скоростью обслуживания; эффективная пропускная
 способность; регенеративный процесс; регенеративная оценка; стационарный процесс нагрузки; вероятность потери}
 
 \vskip 14pt plus 9pt minus 6pt

      \thispagestyle{headings}

      \begin{multicols}{2}

            \label{st\stat}

\section{Введение}

В современных коммуникационных системах  одним из важнейших
показателей  качества  обслуживания (QoS) является вероятность
превышения стационарным процессом нагрузки~$W$ (незавершенной
работы) некоторого (большого) уровня~$b$. Для системы с конечным
буфером~$b$ указанная  вероятность является вероятностью потери.

Интерес к  системам с регенеративным входным процессом обусловлен
тем, что такие процессы сохраняют  свойство регенерации при
прохождении  через узлы коммуникационной сети~\cite{MorozovTechCyb87, MorozovOutput}. 
В~то же время пуассоновский
процесс   и даже общий процесс восстановления не обладают таким
свойством. (Исключением являются пуассоновские потоки в стационарной
сети Джексона без циклов, со\-сто\-ящей из узлов вида $M/M/1$.)
 Различные вопросы, связанные
с вычислением ЭПС в системе с регенеративным входным процессом,
рассматривались  в работах~[3--7], в
которых также предложено оценивать ЭПС на основе {\it
регенеративной} оценки, опирающейся на группировку данных по циклам
регенерации. Отметим  также близкую по тематике предшествующую
работу~\cite{Crosby}.   Важной работой в
 области исследования ЭПС является обзорная статья~\cite{Kelly}.

Если уровень $b$ задан, а  мощность   обслуживающего устройства~$C$
(величина работы, которую прибор может сделать за единицу времени)
можно изменять, то естественная задача QoS состоит в   выборе такого
значения~$C$, которое гарантирует, что  стационарная загрузка не
превысит уровня~$b$ с заданной (малой) вероятностью~$\Gamma$, т.\,е.\
\begin{equation}
\p(W>b)\le \Gamma\,.\label{1-bor}
\end{equation}
Минимальная  величина мощности~$C$, удовлетворяющая этому  условию,
и называется  \textit{эффективной пропускной способностью} сис\-темы.

Покажем, как решается  задача вычисления  ЭПС для системы с одним
обслуживающим устройством и неограниченным  буфером  для ожидающих
заявок. Удобно считать, что
  обслуженная работа поступает в систему и покидает ее
 в  (целочисленные) моменты~$t$   и что   величина $W(t)$
 равна  незавершенной работе в момент $t-1$ c учетом работы,
 поступившей в момент $t-1$,  и за вычетом  работы,
 покинувшей систему в момент $t-1,\,t=0,\,1,\ldots$~\cite{Lewis}.

 Предполагается, что объем уходящей работы в каждый  момент (дискретного) времени
равен~$C$, что согласуется с <<объемом жидкости>>, вытекшей из
системы в течение  интервала времени длины~1.
 Пусть $v_i$~--- величина работы,
поступившей в сис\-те\-му в момент~$i$, и тогда
величина $V(t):=\sum\limits_{i=0}^{t-1} v_i$ есть суммарная работа,
поступившая в систему в интервале  $[0,\,t-1]$. Обозначим через
 $W(t)$  незавершенную работу по
обслуживанию заявок, находящихся в системе в момент  времени
$t=0,1,\ldots$, полагая  $W(0)\hm=0$.
 Предположим, что $\{v_i,\,i\ge 0\}$~--- независимые, одинаково
распределенные случайные величины (н.о.р.с.в.), причем,  вообще
говоря, $\p(v_i=0)\hm>0$. Очевидно, имеет место такая рекурсия Линдли
(в дискретном времени):
\begin{equation}
W(t+1)=[W(t)+v_{t}-C]^+\,,\enskip t\ge0\,,
\label{2-bor}
\end{equation}
откуда следует, что процесс $\{W(t)\}$ образует марковскую цепь (с
общим пространством состояний). Обозначим $X_i\hm=v_i\hm-C$ и введем
случайное  блуждание
\begin{multline}
Z(t):=\sum\limits_{i=0}^{t-1} X_i=\sum\limits_{i=0}^{t-1}(v_{i}-C)={}\\{}=V(t)-C\,
t\,,\enskip t\ge1\,,\label{3-bor}
\end{multline}
где положено   $Z(0)\hm=0$. Обозначим  типичный шаг этого блуждания
через $X\hm=v\hm-C$. (Здесь и далее типичный элемент последовательности
н.о.р.с.в.\ обозначается без соответствующего индекса.) Так как
буфер для ожидания не ограничен, то предполагается  выполненным
следующее условие отрицательного сноса у случайного блуждания~$Z$:
\begin{equation}
\e X=\e v- C:=\lambda-C<0\,. \label{4-bor}
\end{equation}
Заметим, что~(\ref{4-bor}) является условием стационарности
процесса (незавершенной) нагрузки $\{W(t),\ t\hm\ge 0\}$.
Действительно, опираясь на рекурсию~(\ref{2-bor}), определим приращение
этого процесса $ \Delta(t)\hm=W(t+1)\hm-W(t)$
 между моментами  $t-1$ и $t$. Предположим, что марковская цепь
\begin{equation}
W(t)\stackrel{d}{\to} \infty\,,\enskip t\to   
\infty\,,
\label{5b-bor}
\end{equation} 
где $\stackrel{d}{\to}$  означает сходимость по вероятности.
 Поскольку $\Delta(t)\hm\le  v_t$ и $\e v\hm=\lambda\hm<\infty$, то  из~(\ref{2-bor}) 
 легко следует, что
$ \e \Delta (t)\hm\to \lambda\hm-C\hm<0$, $t\hm\to \infty. $ Этот результат,
как легко проверить, противоречит сходимости~(\ref{5b-bor}) и  поясняет
термин <<отрицательный снос>>. Для широкого класса цепей Маркова
последний результат, в свою очередь, влечет существование
стационарного процесса $W(t)\hm\Rightarrow W$~\cite{Asmus}. (Знак
$\Rightarrow $ обозначает сходимость по распределению.)

Данная статья является  продолжением работы~\cite{KRC}, в которой
основное внимание было уделено сравнению оценки по методу группового
среднего (batch mean) с регенеративной оценкой. Здесь представ\-ле\-на
более подробная мотивировка эвристических соображений, позволяющих
получить искомую аппроксимацию ЭПС   в жидкостной сис\-те\-ме  с
регенеративным входным процессом. Кроме того, представлены
результаты  численных экспериментов, которые подтверждают, что
полученная аппроксимация действительно может быть эффективно
использована для оценивания ЭПС в рассматриваемых сис\-те\=мах
обслуживания. В~разд.~2 показано как вычисляется  ЭПС,
удовлетворяющая условию~(\ref{1-bor}), в случае н.о.р.c.в.\
$\{v_i\}$.

\section{Вероятность большого уклонения и~эффективная пропускная способность}

Рассмотрим  асимптотику вероятности большого уклонения стационарной
нагрузки для рас\-смот\-рен\-ной выше системы с н.о.р.\ $\{v_i\}$ и с
{\it заданной скоростью обслуживания}~$C$. Напомним известный
результат~\cite{Asmus}:
\begin{multline}
W(n)=
 % V(t) -Ct-\inf_{u\le t}(V(u) -Cu)=
\sup\limits_{0\le t< n}[V(t)-C\,t]^+={}\\
{}=\sup\limits_{0\le t<n}Z(t)\,,\enskip
n=1,\ldots\,,
\label{4a-bor}
\end{multline}
связывающий  величину незавершенной работы с максимумом
 случайного блуждания~(\ref{3-bor}).  (Поскольку $Z(0)\hm=0$, то
$\sup[\cdot]=\sup[\cdot]^+$ в  соотношении~(\ref{4a-bor}), что
согласуется с~(\ref{2-bor}) и гарантирует $\inf_nW(n)\hm\ge 0$.)
 Отметим, что марковская цепь $\{W(t)\}$ с дискретным временем
регенерирует в последовательные моменты  опустошения системы, т.\,е.\ в моменты
$$
\beta_{n+1}=\min\{k>\beta_n: W(k)=0\}\,,\enskip n\ge 0\,,\ \beta_0:=0\,.
$$
Длина (типичного) цикла регенерации процесса $\alpha:=\beta_1$ есть
непериодическая с.в., поскольку ввиду~(\ref{4-bor}) выполнено условие:
\begin{multline}
\p(\alpha=1)=\p(W(t+1)=0|W(t)=0)={}\\{}=\p(v<C)>0\,. \label{6a-bor}
\end{multline}
Как хорошо известно, из~(\ref{4-bor}) и~(\ref{6a-bor}) следует, что $\e
\alpha\hm<\infty$, т.\,е.\ процесс $\{v_i\}$ является {\it положительно
возвратным} и, кроме того, стационарный процесс нагрузки~$W$
существует и является пределом  по распределению:
\begin{equation*}
W(t)\Rightarrow W=_{\mathrm{st}}\sup_{n\ge 0} Z(n)\,, %\label{11a-bor}
\end{equation*}
где $=_{\mathrm{st}}$  обозначает стохастическое равенство. % \cite{Asmus}.
 Предположим, что в некоторой положительной
 окрестности  $(0,\,\theta_0)$ параметра~$\theta$  выполнено условие:
\begin{equation}
\e e^{\theta v}<\infty\,, \label{7-bor}
\end{equation}
и приведем ряд  необходимых далее   понятий и результатов теории
 больших уклонений~\cite{Ganesh}.
Рассмотрим (нормированную)  {\it логарифмическую производящую
функцию моментов} случайного блуждания $Z(n)$
\begin{equation*}
\Lambda_n(\theta):=\fr{1}{n}\ln\e e^{\theta Z(n)}\,,\enskip n\ge1\,,
\end{equation*}
которая ввиду независимости
 слагаемых $\{X_i\}$, формирующих случайное блуждание $Z(n)$, может
 быть записана как
\begin{equation*}
\Lambda_n(\theta):=\Lambda(\theta)=\ln \e e^{\theta v}-\theta C\,,\enskip
n\ge 1\,. %\label{5a-bor}
\end{equation*}
Поскольку  функция $\psi(\theta):=\ln \e e^{\theta v}$  выпуклая,
$\psi(0)\hm=0$ и ввиду   условия~(\ref{4-bor})  $\psi^{\prime}(0)\hm=\e v\hm<C$, то
существует   единственный корень $\theta^*\hm>0$ уравнения
\begin{equation}
\ln \e e^{\theta v}=\theta C\,. \label{6-bor}
\end{equation}
   Будем предполагать, что с.в.~$v$ такова, что
 $\theta^*\hm\in (0,\,\theta_0)$.
Из работы~\cite{Glynn} следует, что при выполнении  условий~(\ref{4-bor}), (\ref{7-bor})
 стационарный процесс~$W$
 удовлетворяет   следующему асимптотическому соотношению~\cite{Ganesh, Glynn}:
\begin{equation}
\lim\limits_ {x\to \infty}\fr{1}{x}\ln \p(W> x)=-\theta^*\,. \label{10b-bor}
\end{equation}
Это  влечет такую экспоненциальную аппроксимацию  вероятности
большого уклонения стационарного процесса нагрузки:
%\begin{eqnarray}
 $\p(W> x)=e^{-\theta^*x+o(x)}$, $x\hm\to \infty.$
 Ввиду~(\ref{6-bor}) ЭПС равна
\begin{equation}
C=\fr{\ln\e e ^{\theta^* v}}{\theta^*} \label{12b-bor}
\end{equation}
и удовлетворяет условию~(\ref{4-bor}) (см.\ лемму~9.1.5 в работе~\cite{Chang}).

Отметим, что в  случае, когда с.в.\ $\{v_i\}$ являются зависимыми,
 асимптотика~(\ref{10b-bor}) также  может быть доказана в некоторых 
 случаях~[12--14]. В~частности, это верно, если
последовательность $\{v_i\}$~---  стационарная с перемешиванием~\cite{Lewis}.

Ключевыми для справедливости  асимптотики~(\ref{10b-bor}) являются
свойства   логарифмической производящей функции моментов {\it
входного процесса}
\begin{equation*}
%
\Lambda_V^{(n)}(\theta):=\fr{1}{n} \ln \e e ^{\theta V(n)}\,.
%\label{10-bor}
\end{equation*}
В первую очередь требуется   существование   {\it предела
Гарт\-не\-ра--Эл\-лиса}~\cite {Glynn, Chang}
\begin{equation*}
\lim\limits_{n\to \infty} \Lambda_V^{(n)}(\theta)=\Lambda _V
(\theta)\,,\enskip n\to \infty\,, %\label{10a-bor}
\end{equation*}
а также  условие  его конечности либо для всех $\theta\hm\in
(0,\,\theta_0)$~\cite{Ganesh, Glynn}, либо для всех $\theta\hm>0$~\cite{Chang}.

Когда скорость   $C$  задана,
 параметр $\theta^*$, найденный из условия~(\ref{6-bor}),
дает  скорость (экспоненциального) убывания хвоста вероятности
$\p(W>b)$. Рассмотрим теперь обратную задачу: скорость~$C$ (т.\,е.\
ЭПС) должна быть выбрана так, чтобы   обеспечить  требование~(\ref{1-bor}). 
ЭПС должна обеспечивать  условие стационарности~(\ref{4-bor})
и поэтому можно опираться на экспоненциальную асимптотику~(\ref{10b-bor}). 
Это дает такое (приближенное) уравнение:
 \begin{equation}
 \p(W>b)=\Gamma=e^{-\theta b}\,,\label{16a-bor}
 \end{equation}
решение которого имеет вид
\begin{equation}
 \theta^*=-\fr{\ln \Gamma}{b}>0\,. \label{12-bor}
\end{equation}
Таким образом, соотношения~(\ref{12b-bor}), (\ref{12-bor})  позволяют
(приближенно) определить ЭПС как
\begin{equation*}
C=\fr{\Lambda_V (\theta^*)}{\theta^*}=\fr{\Lambda_V(-\ln
\Gamma/b)b}{-\ln \Gamma}\,. %\label{12a-bor}
\end{equation*}
В случае н.о.р.\ $\{v_i\}$ получаем, см.~(\ref{6-bor}),
\begin{equation*}
C=-\fr{b}{\ln \Gamma}\ln \e e^{-{v\ln \Gamma
}/b}\,. %\label{13a-bor}
\end{equation*}
Это соотношение позволяет в ряде случаев получить явное  значение~$C$.

Приведенный  анализ
 верен также и для системы с конечным буфером
(большого) размера~$b$, т.\,е.\ для системы с потерями, поскольку
вероятность потери  $ \p(W> b)$  в такой системе ведет себя
асимптотически (при $b\hm\to \infty$) так же, как и в системе с
неограниченным буфером~\cite{Ganesh}.

\section{Вычисление  эффективной пропускной способности  для~регенеративного входного процесса}

Как было сказано выше,  интерес к  системам с регенеративным входным
потоком обусловлен сохранением свойства регенерации   при
прохождении потоков между узлами коммуникационной сети. Предлагаемый
ниже подход в значительной мере опирается на эвристические
соображения и в первую очередь мотивирован тем, что широко
используемая  оценка по методу группового среднего (batch mean) не
учитывает  зависимости между данными входного процесса. Как
отмечено, например, в работах~\cite{KRC, PPM09}, используемое в этой
оценке разбиение данных на блоки фиксированной длины, как правило,
приводит к недооценке доли потерь в системах  с (большим) конечным
буфером.  Поэтому использование такой оценки для вычисления ЭПС в
случае входного процесса с зависимыми данными может привести к
нарушению требуемых гарантий QoS, что неприемлемо в
системах высокой надежности.

 Рассмотрим вновь дискретную рекурсию Линдли~(\ref{2-bor}), и пусть теперь входная
 последовательность $\{v_n\}$ является регенерирующей с моментами
регенерации $\{\beta_k\}$ и периодами регенерации
$\alpha_k\hm=\beta_{k+1}\hm-\beta_k$. Таким образом,  значения~$v_i$
внутри каждого цикла регенерации могут быть зависимыми, но значения
$v_i$ и $v_j$, принадлежащие разным циклам, являются независимыми.
Поэтому  суммарная работа, поступающая на циклах регенерации
входного процесса,
\begin{equation}
   V_k:=\sum\limits_{i=\beta_k}^{\beta_{k+1}-1}v_i\,,\enskip k\ge
  0\,,\ \beta_0=0\,,
  \label{vor-eq9}
\end{equation}
образует последовательность н.о.р.\ {\it блоков} (c типичным
элементом~$V$). Будем предполагать, что для некоторого $\theta_0\hm>0$
при всех $\theta\hm\in (0,\,\theta_0)$
 выполнено  условие конечности экспоненциальных моментов величины блока
\begin{equation}
\ln \e e^{\theta V}<\infty\,.\label{5c-bor}
\end{equation}
Заметим, что поскольку величина блока~$V$ и длина цикла~$\alpha$
связаны очевидным образом
\begin{equation}
V=_{\mathrm{st}}\sum\limits_{i=0}^{\alpha-1}v_i\,, \label{24-bor}
\end{equation}
то их  моментные свойства также тесно связаны.  Скажем, в
специальном случае, когда
длина цик\-ла~$\alpha$ не зависит от последовательности н.о.р.с.в.\
$\{v_n\}$,
получаем
$
\e e^{\theta V}= \e [\e e^{\theta v}]^{\alpha}.$ 
В~дальнейшем потребуется лишь положительная возвратность
последовательности $\{v_n\}$, т.\,е.\ условие  $\e\alpha<\infty$.
Заметим, что если  длина цикла~$\alpha$ является моментом остановки
относительно последовательности н.о.р.с.в.\ $\{v_n\}$,  то по
неравенству Иенсена и тождеству Вальда $ \e e^{\theta V}\hm\ge \theta
\e V\hm=\theta\e \alpha \e v $ и положительная возвратность следует из
условия~(\ref{5c-bor}). В~этом случае положительную  возвратность можно
также получить  из разложения функции $\phi(\theta):=\e e^{\theta
V}\hm=1\hm+\theta \e V\hm+o(\theta)$ в ряд Тейлора в окрестности $\theta\hm=0$.

 Обозначим через $ k(n):=\max(k\ge0: \beta_k \le n)$ число циклов
регенерации  среди
заявок с номерами $0,\,1,\ldots,n$, так что $k(0)\hm=0$.  Заметим, что
$k(i)=0,\,i\le \beta_1-1$ и $k(\beta_i)=i,\,i\ge 1$. Напомним
обозначение: 
$$
V(n)\hm=\sum\limits_{i=0}^{n-1}v_i\,,\enskip V(0)=0\,.
$$
 В~теории
регенерации процесс $\{V(n)\}$ называется процессом накопления.
Заметим, что
\begin{multline*}
\ln \e e^{\theta \sum_{i=0}^{k(n)-1}V_i}\le \ln \e e ^{\theta
V(n)}\le {}\\
{}\le\ln \e e^{\theta \sum_{i=0}^{k(n)} V_i}\,,\enskip n\ge 0
\ \left(\sum\limits_\emptyset=0\right)\,. %\label{38a-bor}
\end{multline*}
Далее основное допущение состоит в   том, что {\it блоки
$V_i,\,i\le k(n),$ и величина $k(n)$ независимы при больших~$n$}.
(Вообще говоря, эти величины зависимы, поскольку  $k(n)$ зависит от
длин циклов, полученных к моменту~$n$.)
Обозначим $a:= \e e^{\theta V}$ и, используя свойство условного
математического ожидания, получим (при больших~$n$)
\begin{equation}
\fr{1}{n} \ln \e a ^{k(n)} \le \fr{1}{n} \ln \e e ^{\theta
V(n)}\le \fr{1}{n} \ln \e a ^{k(n)+1}\,.
  \label{27-bor}
\end{equation}
Следующее допущение состоит в том, что {\it нижняя и верхняя границы
в неравенстве~$(\ref{27-bor})$ асимптотически близки при больших~$n$}.
Приведем некоторые соображения в пользу   этого предположения.
Рассмотрим величину работы, поступающей в систему с момента~$n$ до
конца текущего  цикла регенерации, т.\,е.\
\begin{equation*}
U(n)=\sum\limits_{i=0}^{k(n)} V_i -V(n) =\sum\limits_{i=n}^{\beta_{k(n)}-1}v_i\,.
%\label{26a-bor}
\end{equation*}
Как показано в~\cite{Asmus}, при условии~(\ref{5c-bor}) для некоторого
$\varepsilon\hm>0$
$$
\e e^{\varepsilon U(n)}\to D\,,\enskip n\to \infty\,,
$$
где постоянная  $D<\infty$. Отметим также, что с.в.\
$Z(n):=V(n)\hm-\sum\limits_{i=0}^{k(n)-1} V_i$, равная величине  работы,
которая уже поступила в систему на текущем в момент $n$ цикле
регенерации, асимптотически (при $n\hm\to \infty$) распределена так же,
как и величина $U(n)$, и поэтому имеет (в пределе) такие же
моментные свойства. (С~учетом  сделанных  замечаний можно легко
доказать сближение нижней и верхней границы в~(\ref{27-bor})  в
предположении {\it независимости} величин $V(n)$, $U(n)$ и~$Z(n)$.)
Далее,  по неравенству Иенсена
\begin{equation*}
\fr{1}{n} \ln \e a ^{k(n)+1}\ge \fr{\e(k(n)+1)}{n}\ln a\,,
% \label{28-bor}
\end{equation*}
а по элементарной теореме восстановления $\e (k(n)\hm+1)/n\hm\to
1/\e\alpha$.  Приведенные выше  соображения позволяют заключить, что
при больших $n$ аппроксимация
\begin{equation}
\Lambda_V(\theta)= \fr{\ln \e e^{\theta V}}{\e \alpha}
  \label{29-bor}
\end{equation}
может дать   удовлетворительное  для практических целей значение
функции $\Lambda_V(\theta)$. Это, в свою очередь, является
основанием для аппроксимации искомой ЭПС с помощью формулы
\begin{equation}
C= \fr{\ln \e e^{\theta^* V}}{\theta^*\,\e \alpha}\,.
  \label{30-bor}
\end{equation}
Подчеркнем, что при получении приведенной выше аппроксимации ЭПС
используется  приближение~(\ref{16a-bor}), а также {\it предположение},
что в данной сис\-те\-ме верна экспоненциальная асимптотика~(\ref{10b-bor}).\linebreak
Кроме того, хотя это не оговаривалось ранее, в основе асимптотики~(\ref{10b-bor}) 
лежит также предположение о том, что входная
последовательность $\{v_n\}$ является {\it стационарной}.
(Конструкция стационарного регенерирующего процесса описана  в~\cite{Thorrison1}.)    
Наконец, отметим, что результат~(\ref{29-bor})
формально можно  получить из~(\ref{27-bor})
 заменой случайного чис\-ла цик\-лов  $k(n)$ его  математическим
ожиданием и переходом к переделу при $n\hm\to \infty$.

 Как и в работе~\cite{KRC}, ниже рассмотрены
 два следующих варианта получения  регенеративного
входного процесса.

В первом варианте   рассматривается  двухузловая  сеть, в которой на
вход  узла~1 в соответствии с процессом восстановления поступают
заявки c н.о.р.\ временами обслуживания $\{S_n^{(1)}\}$ и с
коэффициентом загрузки $\rho \hm<1$. В~этом случае выходной процесс из
узла~1 является положительно возвратным регенерирующим процессом со
средней длиной цикла $\e \alpha\hm<\infty$. Будем считать длину цикла~$\alpha$ 
равной числу заявок, поступивших в узел~1  на цикле
регенерации. (Это обычный подход при рассмотрении процессов,
вложенных в моменты прихода заявок. Разумеется, можно рассмотреть
длину цикла и в непрерывном времени.) Моментные свойства длины цикла $\alpha$ 
можно получать исходя из свойств  с.в.\ $S^{(1)}$.  В~этой
связи оказывается полезным следующий результат.
 Пусть~$\phi$ есть некоторая измеримая функция и $\rho\hm<1$.  Тогда (см.~\cite{Thorisson, Wolff})
\begin{equation}
\e \phi(S^{(1)})<\infty\ \mbox{влечет}\ 
\e\phi(\alpha)<\infty\,. \label{31-bor}
\end{equation}
 В частности, если $\e e^{\theta S^{(1)}}<\infty$, то $\e e^{\theta \alpha}<\infty$.  (Для
получения условий конечности моментов длины цикла регенерации узла~1
в непрерывном времени соответствующее условие надо наложить и на
длину интервала входного потока в узел~1, см.~\cite{Thorisson}.
Однако эта конструкция не используется в данной работе.) Далее
н.о.р.\ длины циклов регенерации $\{\alpha_n\}$ узла~1
 используются  в качестве  длин  циклов входного процесса
 в {\it жидкостной узел~2 с искомой скоростью~$C$},  где процесс
 загрузки
описывается  рекурсией~(\ref{2-bor}). Иными словами,  $\alpha$ равно
длине цикла регенерации (числу интервалов единичной длины) процесса
нагрузки $\{v_i\}$, поступающего в узел~2 и описываемого рекурсией~(\ref{2-bor}).
 Мотивировку такой модели, где  оба узла связаны
не напрямую, можно найти в работе~\cite{KRC}.

Во втором варианте входной регенеративный  процесс можно  считать
заданным и  наложить требуемые моментные условия на длины циклов  и
на объем поступающей на цикле работы.

Точность полученной  аппроксимации~(\ref{29-bor}) для обоих случаев
иллюстрируется в следующем разделе  с помощью имитационного
моделирования  ряда систем  с регенеративным входным процессом.

\medskip

\noindent
\textbf{Замечание~1.} С~учетом разложения функции
$\phi(\theta):=\Lambda_V(\theta)$ в ряд Тейлора в окрестности
$\theta\hm=0$ соотношение~(\ref{30-bor}) можно записать как
$$
C=\fr{1}{\e \alpha}\left(\e V +\fr{\e
(V^2)\theta^*}{2}+o(\theta^*) \right)\,,\enskip \theta^*\to 0\,,
$$
что в ряде случаев может быть использовано  для вычисления~$C$  на
основе лишь первых двух моментов  с.в.~$V$ и~$\alpha$, см.~(\ref{24-bor}).

\medskip

\noindent
\textbf{Замечание~2.}  Используя  неравенство Иенсена, из теории регенерации  получим
\begin{equation}
\fr{1}{n}\ln \e e^{\theta\, V(n)}\ge \theta\fr {\e V(n)}{n}\to
\theta \,\fr{\e V}{\e\alpha}\,,\ n\to \infty\,. \label{32a-bor}
\end{equation}
При этом величина  $ \e V/\e \alpha $ является ин\-тен\-сив\-ностью
входного регенерирующего процесса, и если существует стационарный
предел $v_n\hm\Rightarrow v$, то $\e v\hm=\e V/\e \alpha$~\cite{Asmus}.
 Если  использовать  нижнюю
границу~(\ref{32a-bor}) вместо  предела $\Lambda_V(\theta)$, то
 уравнение~(\ref{6-bor})
дает   $\e V/\e \alpha\hm=C$, что  влечет нестационарность сис\-те\-мы с
регенеративным входным процессом~\cite{Questa04}.


\medskip

\noindent
\textbf{Замечание~3.}  В~работах~[20--22]
приводится (в разных формах) сильный принцип инвариантности,
позволяющий аппроксимировать с вероятностью~1 процесс восстановления
$\{k(n)\}$, а также  исходный процесс накопления $\{V(n)\}$, с
помощью винеровского процесса и некоторой (случайной) {\it функции
уклонения}  $f(n)\hm=o( n)$. Однако  применение этих результатов для
получения асимптотики $\Lambda_V^{(n)}(\theta)$ опирается на
некоторые допущения о независимости и приводит в ряде случаев к
значению ЭПС, при котором   вероятность потери превышает величину~$\Gamma$.

\begin{table*}[b]\small
\vspace*{-12pt}
\begin{center}
\Caption{Регенеративная оценка ЭПС для двухузловой сети
}
\vspace*{2ex}


\tabcolsep=8pt
\begin{tabular}{|c|c|c|c|c|c|}
\hline
$\#$ & $\Gamma$ & $\theta^*$ & $\hat{C}(k)$ & $\hat\Gamma $  & $\Delta/\Gamma$ \\
\hline 
&&&&&\\[-9pt]
1 & $10^{-3}$ & 0,230259 & 0,264602 & $8{,}15\cdot 10^{-4}$ & 0,15\hphantom{9}\\
%\hline
%&&&&&\\[-9pt]
2 &  $10^{-4}$ & 0,307011 &  0,290134 & $2{,}05\cdot 10^{-5}$  & 0,75\hphantom{9}\\
%\hline
%&&&&&\\[-9pt]
3 &  $10^{-5}$ & 0,383764 & 0,348517 & $1{,}84\cdot 10^{-6}$ &  0,816\\
%\hline
%&&&&&\\[-9pt]
4 &  $10^{-6}$ & 0,460517 & 0,527721 &  $2{,}97\cdot 10^{-8}$ & 0,97\hphantom{9}\\
%\hline
%&&&&&\\[-9pt]
5 &  $10^{-7}$ & 0,53727\hphantom{9} & 0,661887  &  $0{,}45\cdot 10^{-8}$ &  0,955\\
%\hline
%&&&&&\\[-9pt]
6 &  $10^{-8}$ & 0,614023 & 0,986111  &  \hphantom{$^0$}$8{,}67\cdot 10^{-10}$ & 0,913\\
\hline 
\end{tabular}
\end{center}
\end{table*}

\section{Результаты численного моделирования}

В этом разделе рассматривается процесс незавершенной работы на
интервалах единичной длины  (слотах), удовлетворяющий рекурсии~(\ref{2-bor}), 
с положительно возвратной входной по\-сле\-до\-ва\-тель\-ностью
$\{v_i\}$, моментами регенерации $\{\beta_k\}$ и н.о.р.\ длинами
циклов $\{\alpha_k\}$. Ниже  исследуется точность оценивания
скорости обслуживания $C$ (искомой ЭПС)    на основе полученной в
предыдущем разделе аппроксимации
\begin{equation*}
C=\fr{\Lambda_V(\theta^*)}
%\ln \e e^{\theta^* V}}
{\theta^*},% \e \alpha}
% +\frac{\ln\e [\e
%e^{\theta^*V}]^{ N(0,C_1^2)}}{\theta^*},
%\label{37a-bor}
\end{equation*}
где
$$
\Lambda_V(\theta^*)=\fr{1}{\e \alpha} \ln \e e^{\theta^*
V}:=\Lambda_{\mathrm{REG}}(\theta^*)\,,
$$
а параметр $\theta^*\hm=-\ln\Gamma/b$. Поскольку $\beta_k/k\hm\to \e
\alpha$, то выборочная оценка
\begin{equation*}
\hat \Lambda_{\mathrm{REG}}^{(k)} (\theta^*):=\fr{\ln
(1/k)\sum\limits_{i=1}^k e^{\theta^* V_i}}{(1/k)\sum\limits_{i=1}^k
\alpha_i}=  \fr{k}{\beta_k}
  \ln\fr{1}{k} \sum\limits_{i=1}^{k} e^{\theta^*  V_i}
%\label{49-bor}
\end{equation*}
величины $\Lambda_{\mathrm{REG}}(\theta^*)$, полученная по $k$ регенеративным
блокам входного потока, является сильно состоятельной:
$
\hat \Lambda_{\mathrm{REG}}^{(k)} (\theta^*)\to
\Lambda_{\mathrm{REG}}(\theta^*)$, $k\hm\to \infty$ с вероятностью~1.
Поэтому ниже соответствующая регенеративная
оценка ЭПС вычисляется по формуле
\begin{equation}
\hat C(k):=\fr{\hat \Lambda_{\mathrm{REG}}^{(k)}(\theta^*)}{\theta^*}
\label{40-bor}
\end{equation}
для (достаточно большого) числа $k$ циклов регенерации, полученных в
процессе имитационного моделирования.

%\medskip

{\it Эксперимент~1.} Рассмотрена  двухузловая   сеть, где узел~1
есть система вида $M/M/1$ с показательным  временем обслуживания~$S$
с интенсивностью $\mu\hm= 1$ и с интенсивностью  пуассоновского
входного процесса $\lambda:=\rho\hm<1$. Поскольку время обслуживания
имеет конечные экспоненциальные моменты, $\e e^{\theta S}\hm<\infty$
для всех $\theta\hm<1$, то ввиду~(\ref{31-bor}) получаем также $\e
e^{\alpha \theta }\hm<\infty $ при $\theta\hm<1$. Подчеркнем, что значения
параметра $\theta\hm=\theta^*$ в табл.~1 выбраны с учетом этого
условия. (В~этой связи отметим, что в случае н.о.р.\ $\{v_i\}$ с
легким хвостом распределение хвоста суммы~$V$ в соотношении~(\ref{24-bor}) 
может иметь тяжелый хвост,  если индекс суммирования~$\alpha$ имеет тяжелый хвост~\cite{HT}.) 
Как было упомянуто выше,
длины цик\-лов регенерации  узла~1 используются  как длины циклов
регенерации в дискретном времени входного потока в узел~2, на основе
которых получаются регенеративные блоки~(14). В~узле~2
надо определить скорость~$C$, гарантирующую  непревышение заданного
уровня потерь~$\Gamma$.  В~данной модели коэффициент загрузки $\rho$
узла~1 существенно влияет на моментные свойства длины цикла~$\alpha$
и, кроме того,  значения с.в.\ $\{v_i\}$ на одном цикле являются
зависимыми. Точнее говоря, рассматриваются  независимые с.в.\
$\{\eta_k\}$, имеющие распределение Вейбулла (с легким хвостом):
\begin{equation*}
F_\eta (x) = 1 - e^{-\gamma x^c}\,, \ \gamma > 0\,, \ c \ge 1\,.
%\label{weibull}
\end{equation*}

\begin{table*}\small
\begin{center}
\Caption{Регенеративная оценка ЭПС в случае ограничения объема работы на цикле}
\vspace*{2ex}

\begin{tabular}{|c|c|c|c|c|c|c|c|c|}
\hline
$\#$ & $\Gamma$ & $\theta^*$ & $d$& $\hat\alpha$ &$\hat{C}(k)$ & $Var \hat{C}(k)$ & $\hat\Gamma$ & $\Delta/\Gamma$\\
\hline 
&&&&&&&&\\[-9pt]
1 &  $10^{-4}$ &  0,153506 & 50 & \hphantom{9}89,1 & 0,560441 & $5{,}23\cdot 10^{-6}$  & $0{,}3433 \cdot 10^{-5}$ & 0,6567\\
%&&&&&&&&\\[-9pt]
2 &  $10^{-5}$ & 0,191882 & 50 & \hphantom{9}89,2 & 0,560947 & $7{,}73\cdot 10^{-6}$ & $0{,}4153 \cdot 10^{-5}$ & 0,5847\\
%\hline
%&&&&&&&&\\[-9pt]
3 &  $10^{-6}$ & 0,230259 & 70 & 124,9& 0,561252 &  $2{,}64\cdot 10^{-6}$ & $0{,}8698 \cdot 10^{-6}$ & 0,1302\\
%\hline
%&&&&&&&&\\[-9pt]
4 &  $10^{-7}$ & 0,268635 & 70 & 124,8 & 0,562472 &  $4{,}23\cdot 10^{-6}$ & $0{,}8871 \cdot 10^{-7}$ & 0,1129\\
%\hline
%&&&&&&&&\\[-9pt]
5 &  $10^{-8}$ &  0,307011 & 70 & 124,5 & 0,563537  &  $6{,}98\cdot 10^{-6}$ & $0{,}2116 \cdot 10^{-8}$ & 0,7884\\
\hline
\end{tabular}
\end{center}
\vspace*{-6pt}
\end{table*}

\noindent
Тогда  величина работы, поступающей в узел~2 в момент~$j$ (любого)
текущего цикла регенерации входного процесса, задается соотношением:

\noindent
\begin{equation}
v_j = \fr{\sum\nolimits_{k=1}^{j}\eta_k}{j}\,,\enskip 1 \le j \le \alpha\,,
\label{dependence}
\end{equation}
где    $\alpha$  есть (типичная) длина цикла. Далее по формуле~(\ref{40-bor})  
вычисляется  значение  оценки $\hat{C}(k)$, которая
используется  в качестве скорости обслуживания~$C$ в узле~2.  Затем
(в предположении достижения
 стационарного режима в узле~2) с помощью выборочного среднего   оценивается
 вероятность $\p(W>b)$ в узле~2.  Заметим, что стандартный способ  получения стационарного
 значения~$W$  при имитационном моделировании состоит
  в пропуске начального, так называемого {\it  переходного},  периода.
 Кроме того, в рассматриваемой системе  пропуск  переходного  периода служит для   получения
 стационарного режима входного процесса.

 В табл.~1 представлены результаты оценивания
ЭПС, а также вероятности потери  при следующих условиях: параметры
распределения Вейбулла $\gamma \hm= 3$, $c\hm=4$; интенсивность трафика
$\rho\hm=0{,}4$ в узле~1; число циклов  $k\hm=40\,000$, уровень превышения
(буфер) $b\hm=30$. При этом разность  $\Delta: = \Gamma \hm-
\hat\Gamma\hm>0$, т.\,е.\ регенеративная оценка ЭПС~(\ref{40-bor})
обеспечивает требуемый  уровень надежности~$\Gamma$.

%\medskip

{\it Эксперимент~2.} Рассматривается  система с постоянной искомой
скоростью обслуживания $C$, а   цикл регенерации входного процесса
завершается, когда   величина работы, поступающей в систему на
цикле, достигает заданного объема~$d$. Как и выше,  в качестве
скорости $C$ использована ее оценка $\hat{C}(k)$. В табл.~2
представлены результаты моделирования при  $b\hm=60$, а зависимость
между с.в.\ $\{v_i\}$ внутри цик\-ла регенерации определяется
соотношением~(\ref{dependence}).




 Подчеркнем, что во всех
случаях $\Delta \hm> 0$, т.\,е.\ используемая оценка $\hat{C}(k)$
обеспечивает требуемую гарантию~$\Gamma$.
 В~этой связи важно отметить, что оценивание ЭПС по методу
группового среднего 
в ряде случаев приводит к значению  $\Delta \hm< 0$, т.\,е.\  к нарушению
гарантии~$\Gamma$ (см.~\cite{KRC}). В~то же время величина
$\Delta/\Gamma$ показывает, что при использовании оценки
$\hat{C}(k)$ уровень~$\Gamma$ обеспечивается с определенным запасом,
и это  связано с высокой чувствительностью доли потерь к изменению
скорости обслуживания. С~другой стороны, оценка ЭПС имеет очень
небольшую дисперсию и мало  чувствительна к изменению величины~$\Gamma$.  
Это особенно  хорошо заметно в табл.~2.
Например, изменение величины $\Gamma$ с  $10^{-4}$ до $10^{-5}$
влечет изменение $\hat{C}(k)$ с 0,5604 до 0,5609. Это несомненно
вызвано  тем, что  нагрузка на цикле ограничена постоянной величиной~$d$.

Отметим, что для вычисления оценки $\hat\Gamma$ вероятности
превышения стационарным процессом нагрузки (высокого) уровня $b\hm=60$
использовались ресурсы кластера КарНЦ РАН, а также метод ускоренного
регенеративного имитационного моделирования, разработанный  для
оценивания вероятностей редких событий~\cite{PPM09}.

\vspace*{-4pt}

\section{Заключение}

\vspace*{-2pt}

В статье обсуждается понятие ЭПС жидкостной сис\-те\-мы обслуживания в дискретном времени.
 Рас\-смат\-ри\-ва\-ет\-ся  входной процесс с независимыми приращениями, но  основное внимание
уделено  аппроксимации и  оцениванию  ЭПС  в случае входного
регенеративного процесса.
 Представлены  эвристические соображения, на основе
которых получена аппроксимация  ЭПС. Результаты численных
экспериментов   показывают, что оценка~(\ref{40-bor}), полученная на
основе аппроксимации~(\ref{30-bor}), гарантирует  уровень надежности
$\Gamma$. Таким образом,  аппроксимация~(\ref{30-bor}) может считаться
эффективной альтернативой оценке ЭПС по методу группового среднего
для системы с регенеративным входным процессом. Дальнейшие
исследования   в данном направлении  должны включать как
теоретическое обоснование предложенной аппроксимации, так и
расширение численных экспериментов по проверке ее точ\-ности.

\bigskip

Авторы благодарят М.\,А.~Лифшица за  ценные замечания.

{\small\frenchspacing
{%\baselineskip=10.8pt
\addcontentsline{toc}{section}{Литература}
\begin{thebibliography}{99}

\bibitem{MorozovTechCyb87} 
\Au{Морозов Е.} Критерий стационарности одного класса непуассоновских
сетей обслуживания~// Изв. АН СССР. Техн. кибернетика, 1988. №\,1. C.~129--133.

\bibitem{MorozovOutput}
\Au{Morozov E.} Stability of Jackson type network output~//
Queueing Syst., 2002. Vol.~40. P.~383--406.

\bibitem{Irina07} %3
\Au{Vorobieva I., Morozov E., Pagano~M., Procissi~G.} A~new
regenerative estimator for effective bandwidth prediction~// 
AMICT 2007 Proceedings.~--- Petrozavodsk, 2008. P.~175--187.

\bibitem{Rennes} %4
\Au{Morozov E., Dyudenko~I., Pagano~M.}  Regenerative estimator of the overflow probability in a
 tandem network~// 7th  Workshop (International)
 on Rare Event Simulation Proceedings.~---  Rennes, France, 2008. P.~283--287.

\bibitem{Minsk} %5
\Au{Dyudenko I., Morozov E., Pagano~M.}  Regenerative estimator for
effective bandwidth~// Mathematical  methods for analysis and
optimization of information telecommunication networks: Proceedings
of  the International Conference.~--- Minsk: Belarusian State
University, 2009. P.~58--60.

\bibitem{SP2009} %6
\Au{Dyudenko I., Morozov~E.,  Pagano~M., Sandmann~W.} Comparative
study of effective bandwidth estimators: Batch means and
regenerative cycles~// 6th St. Petersburg
Workshop on Simulation Proceedings.~--- St-Petersburg, 2009.
 Vol.~II. P.~1003--1007.
 
 \bibitem{KRC} %7
\Au{Бородина А.\,В., Морозов~Е.\,В.} Сравнение двух оценок
эффективной пропускной способности системы обслуживания~// Тр.
Карельского научного центра РАН, 2012. №\,5. C.~8--17.

\bibitem{Crosby} %8
\Au{Crosby S., Leslie I., Huggard~M., Lewis J.\,T., McGurk~B., Russel~R.} 
Predicting bandwidth requirements of ATM
 and Ethernet traffic~// IEE UK Teletraffic Symposium Proceedings.~--- Glasgow, U.K., 1996.

\bibitem{Kelly} %9
\Au{Kelly F.} Notes on   effective bandwiths~// Stochastic
networks: Theory and applications~/ Eds. F.\,P.~Kelly, S.~Zachary, 
I.~B. Ziedins.~--- Roy. Stat. Soc. Lecture Notes ser., 4.~---
Oxford University Press, 1996. P.~141--168.


\bibitem{Lewis}
\Au{Lewis J.\,T., Russell R.} An introduction to large deviation for
teletraffic engineers, 1997.
{\sf https://\linebreak engineering.purdue.edu/ece647/notes.html}.


\bibitem{Asmus}
\Au{Asmussen S.}  Applied probability and queues.~-- 2nd ed.~--- NY: Springer, 2003.

\bibitem{Ganesh}
\Au{Ganesh A., O'Connell N., Wischik~D.} Big queues.~--- Berlin:  Springer-Verlag, 2004.

\bibitem{Glynn} %13
\Au{Glynn P.\,W., Whitt W.} Logarithmic asymptotics for steady-state
tail probabilities in a single-server queue~// JAP, 1994. Vol.~31.
P.~131--156.

\bibitem{Chang} %14
\Au{Cheng-Shang Chang}.  Performance guarantees in communication
networks.~--- Springer, 2000.


\bibitem{PPM09} %15
\Au{Бородина А., Дюденко И., Морозов~Е.} Ускоренное оценивание
вероятности переполнения  регенеративных систем обслуживания~//
ОПиПМ, 2009. Т.~16. №\,4. С.~577--593.




\bibitem {Thorrison1} %16
\Au{Thorisson H.}  Coupling, stationarity, and regeneration.~--- NY: Springer, 2000.

\bibitem{Thorisson} %17
\Au{Thorisson H.} The queue $GI/GI/k$:  Finite moments of the cycle
variables and uniform rates of convergence~// Commun.
Stat.-Stochastic Models, 1985. Vol.~192. P.~221--238.

\bibitem {Wolff} %18
\Au{Wolff R.\,W.}   Stochastic modeling and the theory of queues.~---
Prentice-Hall, 1989.

\bibitem{Questa04} %19
\Au{Morozov E.} Weak regeneration in modeling of queueing processes~// 
Queueing  Syst., 2004.  Vol.~46. No.\,3--4. P.~295--315.

\bibitem{Csorgo} %20
\Au{Cs\"org\H{o} M., Horvath L., Steinebach~J.} Invariance
principles for renewal processes~//  Ann. Prob., 1987.
Vol.~15. No.\,4. P.~1441--1460.

\bibitem{Damerdji} %21
\Au{Damerdji H.} Strong consistency of the variance estimator in
steady-state simulation output analysis~// Math. Oper.
Res., 1994. Vol.~19. No.\,2. P.~494--512.

\bibitem{Sharma} %22
\Au{Sharma V.} Reliable estimation via simulation~// Queueing
Syst., 1995. Vol.~19. P.~169--192.


\label{end\stat}


\bibitem {HT} %23
 \Au{Robert C.\,Y.,  Segers~J.}  Tails of random sums of a
heavy-tailed number of light-tailed terms~// Insurance: Mathematics
and Economics, 2008. Vol.~43. P.~85--92.
\end{thebibliography}
}
}

\end{multicols}