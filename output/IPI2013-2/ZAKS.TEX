
\def\stat{zaks}

\def\tit{О СХОДИМОСТИ СЛУЧАЙНЫХ БЛУЖДАНИЙ, ПОРОЖДЕННЫХ ОБОБЩЕННЫМИ
ПРОЦЕССАМИ КОКСА, К ПРОЦЕССАМ ЛЕВИ$^*$}

\def\titkol{О сходимости случайных блужданий, порожденных обобщенными
процессами Кокса, к процессам Леви}

\def\autkol{В.\,Ю.~Королев, Л.\,М.~Закс, А.\,И.~Зейфман}

\def\aut{В.\,Ю.~Королев$^1$, Л.\,М.~Закс$^2$, А.\,И.~Зейфман$^3$}

\titel{\tit}{\aut}{\autkol}{\titkol}

{\renewcommand{\thefootnote}{\fnsymbol{footnote}}\footnotetext[1]
{Работа поддержана
Российским фондом фундаментальных исследований (проекты
11-01-00515а, 11-07-00112а, 12-07-00115а, 11-01-12026-офи-м).}}

\renewcommand{\thefootnote}{\arabic{footnote}}
\footnotetext[1]{Факультет вычислительной
математики и кибернетики Московского государственного университета
им.\ М.\,В.~Ломоносова; Институт проблем информатики Российской
академии наук, victoryukorolev@yandex.ru}
\footnotetext[2]{Альфа-банк, отдел моделирования и математической статистики; lily.zaks@gmail.com}
\footnotetext[3]{Вологодский государственный педагогический университет; Институт проблем
информатики Российской академии наук, a$\_$zeifman@mail.ru}


\Abst{Доказана функциональная предельная теорема о
сходимости случайных блужданий с непрерывным временем, порожденных
обобщенными дважды стохастическими пуассоновскими процессами
(обобщенными процессами Кокса), к процессам Леви в пространстве
Скорохода. В качестве следствий получены теоремы о сходимости
указанных случайных блужданий к процессам Леви со смешанными
нормальными распределениями, в частности к устойчивым процессам
Леви.}

\KW{устойчивое распределение; процесс Леви;
$\alpha$-устой\-чи\-вый процесс Леви; обобщенный дважды стохастический
пуассоновский процесс (обобщенный процесс Кокса); пространство
Скорохода; теорема переноса; подчиненный винеровский процесс}

\vskip 14pt plus 9pt minus 6pt

      \thispagestyle{headings}

      \begin{multicols}{2}

            \label{st\stat}



\section{Введение}

В финансовой математике принято моделировать эволюцию логарифмов цен
финансовых активов на временн$\acute{\mbox{ы}}$х микромасштабах с помощью
случай\-ных блужданий. Простейшим примером применения такого подхода
является модель Кок\-са--Рос\-са--Ру\-бин\-штей\-на (см., например,~\cite{Shiryaev1998}).
В~то же время наиболее успешными (адекватными)
моделями динамики логарифмов цен финансовых активов на временн$\acute{\mbox{ы}}$х
макромасштабах являются подчиненные (subordinated) винеровские
процессы (процессы броуновского движения со случайным временем),
примерами которых являются обобщенные гиперболические процессы, в
частности дисперсионные гам\-ма-про\-цес\-сы (variance gamma\linebreak processes),
нормальные$\backslash\!\backslash$обратные гауссовские\linebreak процессы
(NIG-processes) (см., например,~\cite{Korolev2011}). Естественным
теоретическим <<мостиком>> между случайными блужданиями и
подчиненными винеровскими процессами служат функциональные
предельные теоремы для случайных блужданий. Операция подчинения
(subordination) хорошо описывает наличие тяжелых хвостов у
наблюдаемых на практике конечномерных распределений цен финансовых
активов и объясняет их возникновение.

Благодаря работам Мандельброта~\cite{Mandelbrot1963}, одними из
первых успешно применяемых на практике моделей процессов с
<<тяжелохвостыми>> конечномерными распределениями стали устойчивые
процессы Леви. Согласно классическому подходу в функциональных
предельных теоремах для случайных блуж\-да\-ний устойчивые процессы Леви
могут возникнуть, только если дисперсии элементарных скачков
бесконечны. А~для этого, в свою очередь, необходимо, чтобы
вероятности произвольно больших по абсолютной величине значений
скачков были положительны. К~сожалению, последнее условие выглядит
весьма сомнительным с практической точки зрения. Поэтому в рамках
классического подхода теоретическое обоснование адекватности моделей
типа устойчивых процессов Леви при описании эволюции финансовых
индексов является как минимум небесспорным.

Относительно недавно в работах~\cite{Korolev1998, Korolev1999} было
показано, что устойчивые законы могут возникнуть в качестве
предельных распределений для сумм независимых одинаково
распределенных случайных величин с {\it конечными дисперсиями}, если
число слагаемых в суммах случайно и имеет асимптотически устойчивое
распределение. Схема случайного суммирования при изучении случайных
блужданий является естественным аналогом схемы подчинения случайных
процесов. В~книге~\cite{GnedenkoKorolev1996} и
статьях~\cite{Korolev1997, Korolev2000}\linebreak предложено моделировать эволюцию
неоднородных хаотических стохастических процессов, в част\-ности
динамику цен финансовых активов, с по\-мощью случайных блужданий,
порожденных обобщен\-ны\-ми дважды стохастическими пуассоновскими
процессами (обобщенными процессов Кокса). Этот подход, основанный на
универсальном принципе неубывания энтропии в замкнутых системах,
получил дополнительное обоснование и развитие в~[2, 9--11].
В~работах~\cite{Korolev2011, KorolevSkvortsova2006} этот подход
успешно применен к моделированию процессов плазменной
турбулентности. В~соответствии с указанным подходом поток
информативных событий, в результате каждого из которых появляется
очередное <<наблюденное>> значение рассматриваемой характеристики,
описывается с помощью точечного случайного процесса вида
$N_1(\Lambda(t))$, где $N_1(t)$, $t\hm\geqslant0$,~--- однородный
пуассоновский процесс с единичной интенсивностью, а $\Lambda(t)$,
$t\hm\geqslant0$,~--- независимый от $N_1(t)$ случайный процесс, обладающий
следующими свойствами: $\Lambda(0)\hm=0$, ${\sf
P}(\Lambda(t)\hm<\infty)\hm=1$ для любого $t\hm>0$, траектории $\Lambda(t)$
не убывают и непрерывны справа. Процесс $N_1(\Lambda(t))$, $t\hm\geqslant0$,
называется дважды стохастическим пуассоновским процессом (процессом
Кокса). В~част\-ности, если процесс $\Lambda(t)$ допускает
представление
$$
\Lambda(t)=\int\limits_{0}^{t}\lambda(\tau)\,d\tau\,,\ \ \ t\geqslant0\,,
$$
в котором $\lambda(t)$~--- положительный случайный процесс с
интегрируемыми траекториями, то $\lambda(t)$ можно интерпретировать
как мгновенную стохастическую интенсивность процесса Кокса.

В работах~\cite{Kashcheev2000, Kashcheev2001} доказаны
функциональные предельные теоремы для обобщенных процессов Кокса с
квадратично интегрируемым управ\-ля\-ющим процессом $\Lambda(t)$. Однако
класс процессов, предель\-ных для обобщенных процессов Кокса с \mbox{такими}
управ\-ля\-ющи\-ми процессами и скачками, имеющи\-ми конечные дисперсии,
недостаточно широк. В~частности, он не может содержать никаких
устойчивых процессов Леви кроме винеровского про\-цесса.

Цель данной работы~--- существенно обобщить упомянутые результаты и
восполнить указанный пробел. Здесь будет доказана функциональная
предельная теорема, описывающая сходимость обобщенных процессов
Кокса со скачками, имеющими конечные дисперсии, к процессам Леви из
очень широкого класса, содержащего, в частности, устойчивые процессы
Леви.

\section{Основные определения и~некоторые вспомогательные результаты}

Пусть $D=D[0,1]$~--- пространство вещественных функций, определенных
на $[0,1]$, непрерывных справа и имеющих конечные левосторонние
пре\-делы.

Пусть $\mathcal{F}$~--- класс строго возрастающих непрерывных
отображений отрезка $[0,1]$ на себя. Пусть $f$~--- неубывающая
функция на $[0,1]$,   $f(0)\hm=0$, $f(1)\hm=1$. Положим
$$
\|f\|=\sup\limits_{s\neq t}\left|\log\fr{f(t)-f(s)}{t-s}\right|\,.
$$
Если $\|f\|<\infty$, то функция $f$ непрерывна и строго возрастает
и, следовательно, принадлежит классу~$\mathcal{F}$.

Определим расстояние $d_0(x,y)$ в множестве $D[0,1]$ как нижнюю
грань таких положительных чисел~$\epsilon$, для которых
$\mathcal{F}$ содержит некоторую функцию~$f$ такую, что
$\|f\|\hm\leqslant\epsilon$ и
$$
\sup\limits_t|x(t)\hm-y(f(t))|\hm\leqslant\epsilon.
$$
Можно показать, что пространство $D[0,1]$ полно относительно метрики
$d_0$. Метрическое пространство $\mathcal{D}\hm=(D[0,1],d_0)$ принято
называть {\it пространством Скорохода}. Всюду в дальнейшем будем
рассматривать случайные процессы как случайные элементы в~$\mathcal{D}$.

Пусть $X, X_1,X_2,\ldots$~--- случайные элементы пространства
$\mathcal{D}$. Пусть $T_X$~--- такое множество точек отрезка $[0,1]$,
что $0\hm\in T_X$, $1\in T_X$ и если $0\hm<t\hm<1$, то $t\hm\in T_X$ тогда и
только тогда, когда ${\sf P}\left(X(t)\neq X(t-)\right)\hm=0$. Имеет
место следующая тео\-ре\-ма, устанавливающая достаточные условия слабой
сходимости случайных процессов в $\mathcal{D}$ (обозначаемой далее
символом~$\Longrightarrow$ и подразумеваемой при $n\hm\to\infty$).

\smallskip

\noindent
\textbf{Теорема A.} \textit{Пусть для любого натурального}~$k$
$$
\left(X_n(t_1),\ldots ,X_n(t_k)\right)\Longrightarrow
\left(X(t_1),\ldots ,X(t_k)\right)
$$

\noindent
\textit{для $t_1,\ldots ,t_k$, принадлежащих множеству $T_X$. Пусть ${\sf
P}\left(X(1)\hm\neq X(1-)\right)\hm=0$ и существует неубывающая непрерывная
функция~$F$ на $[0,1]$ такая, что для любого $\epsilon\hm>0$}
\begin{multline}
{\sf P}\left(|X_n(t)-X_n(t_1)|\geqslant\epsilon\,,\
|X_n(t_2)-X_n(t)|\geqslant\epsilon\right)
\leqslant{}\\
{}\leqslant\epsilon^{-2\nu}\left[F(t_2)-F(t_1)\right]^{2\gamma}
\label{e1-zaks}
\end{multline}
\textit{при $t_1\leqslant t\leqslant t_2$ и $n\hm\geqslant1$, где $\nu\hm\geqslant 0$, $\gamma\hm>1/2$.
Тогда} $X_n\hm\Longrightarrow X$.

\smallskip

\noindent
Д\,о\,к\,а\,з\,а\,т\,е\,л\,ь\,с\,т\,в\,о\ теоремы~A можно найти, например, в
книге~\cite{Billingsley}.

\smallskip

Условие~(1) может быть заменено более ограничительным моментным
условием
\begin{multline*}
{\sf E}\left[|X_n(t)-X_n(t_1)|^{\nu}|X_n(t_2)-X_n(t)|^{\nu}\right]
\leqslant{}\\
{}\leqslant\left[F(t_2)-F(t_1)\right]^{2\gamma}\,.
\end{multline*}

Под процессом Леви будем понимать случайный процесс $X(t)$, $t\hm\geqslant0$,
со следующими свойствами:
\begin{enumerate}
\item[1$^{\circ}$] $X(0)=0$ с вероятностью единица.
\item[2$^{\circ}$] $X(t)$~--- процесс с независимыми приращениями, т.\,е.\ для любых
$N\hm\geqslant1$ и $t_0,t_1,\dots ,t_N$ ($0\hm\leqslant t_0\hm\leqslant t_1\hm\leqslant\cdots \leqslant t_N$)
случайные величины $X(t_0)$, $X(t_1)-X(t_0)$, $\dots ,$
$X(t_N)\hm-X(t_{N-1})$ независимы в совокупности.
\item[3$^{\circ}$] $X(t)$~--- однородный процесс, т.\,е.\ для любых $s,t,h\hm>0$ случайные
величины $X(t+h)\hm-X(t)$ и $X(s+h)\hm-X(s)$ имеют одинаковое
распределение.
\item[4$^{\circ}$] Процесс $X(t)$ стохастически непрерывен, т.\,е.\ для любых $t\hm\geqslant0$ и
$\epsilon\hm>0$
$$
\lim\limits_{s\to t}{\sf P}(|X(t)-X(s)|>\epsilon)=0\,.
$$
\item[5$^{\circ}$] Траектории процесса $X(t)$ непрерывны справа и имеют конечные пределы
слева.
\end{enumerate}

\smallskip

Обозначим характеристическую функцию случайной величины $X(t)$
символом $\psi_t(s)$ ($\psi_t(s)\hm={\sf E}e^{isX(t)}$,
$s\hm\in\mathbb{R}$). Следующее утверждение описывает хорошо известное
свойство процессов Леви.

\smallskip

\noindent
\textbf{Лемма 1.} \textit{Пусть $X\hm=X(t)$, $t\hm\geqslant 0$,~--- процесс Леви. При
любом $t\hm>0$ характеристическая функция случайной величины $X(t)$
безгранично делима и имеет вид:
\begin{equation}
\psi_{t}(s) = \left[\psi_{1}(s)\right]^{t} = \left[{\sf
E}\,e^{isX(1)}\right]^t\,,\enskip  s\in\mathbb{R}\,.\label{e2-zaks}
\end{equation}
Обратно, пусть $Y$~--- произвольная  безгранично  делимая случайная
величина. Тогда семейство   безгранично   делимых распределений с
характеристическими функциями вида $\left[{\sf E}\,e^{isY}\right]^t$
полностью определяет конечномерные распределения процесса Леви
$X(t)$, $t\hm\geqslant 0$, причем} $X(1)\stackrel{d}{=}Y$.

\smallskip

Свойства процессов Леви подробно описаны в книгах~\cite{Bertoin1996, Sato1999}.
Книги~\cite{BarndorffNielsenMikoschResnick2001, Schoutens2003} и обзорная
статья~\cite{Geman2002} посвящены
использованию процессов Леви в задачах моделирования динамики финансовых индексов.

Функцию распределения строго устойчивого распределения с
характеристическим показателем~$\alpha$ и параметром~$\theta$,
задаваемого характеристической функцией:
$$
\mathfrak{g}_{\alpha,\theta}(s)=\exp
\left\{-|s|^{\alpha}\exp\left\{-\fr{i\pi\theta\alpha}{2}\,\mathrm{sign}\,s\right\}\right\},\
\enskip s\in\r\,,
$$
где $0<\alpha\leqslant2$,
$|\theta|\leqslant\theta_{\alpha}\hm=\min\{1,{2}/{\alpha}-1\}$, будем
обозначать $G_{\alpha,\theta}(x)$. Сим\-мет\-рич\-ным строго устойчивым
распределениям соответствует значение $\theta\hm=0$. Односторонним
устойчивым распределениям соответствуют значения $\theta\hm=1$ и
$0\hm<\alpha\hm\leqslant1$.

Если $\xi$~--- случайная величина с функцией распределения
$G_{\alpha,\theta}(x)$, $0\hm<\alpha\hm<2$, то ${\sf
E}|\xi|^{\delta}\hm<\infty$ для любого $\delta\hm\in(0,\alpha)$, но
моменты порядков, б$\acute{\mbox{о}}$льших или равных~$\alpha$, у случайной
величины~$\xi$ отсутствуют (см., например,~\cite{Zolotarev1983}).

Функцию распределения стандартного нормального закона ($\alpha\hm=2$,
$\theta\hm=0$) будем обозначать $\Phi(x)$:
$$
\Phi(x)=\int\limits_{-\infty}^x\phi(z)\,dz\,;\enskip
\phi(x)=\fr{1}{\sqrt{2\pi}}\,e^{-x^2/2}\,.
$$
Хорошо известно следующее представление сим\-мет\-рич\-ной строго
устойчивой функции распределения $G_{\alpha,0}(x)$ в виде масштабной
смеси нормальных законов:
\begin{equation}
G_{\alpha,0}(x)=\int\limits_{0}^{\infty}\Phi\left(\fr{x}{\sqrt{u}}\right)\,dG_{\alpha/2,1}(u)\,,\
\ \ x\in\mathbb{R}\label{e3-zaks}
\end{equation}
(см., например, теорему~3.3.1 в~\cite{Zolotarev1983} или задачу~9 на
с.~668 в~\cite{Feller1984}). Представлению~(\ref{e3-zaks}) соответствует
аналогичное представление в терминах характеристических функций:
\begin{equation*}
\mathfrak{g}_{\alpha,0}(s)=\int\limits_{0}^{\infty}\exp
\left\{-\fr{s^2u}{2}\right\}\,dG_{\alpha/2,1}(u)\,,\
\ \ s\in\mathbb{R}\,.
%\label{e4-zaks}
\end{equation*}

Процесс Леви $X(t)$, $t\geqslant0$, будем называть\linebreak {\it
$\alpha$-устой\-чи\-вым}, если
$$
{\sf P}\left(X(1)<x\right)=G_{\alpha,\theta}(x)\,,\enskip x\in\mathbb{R}\,.
$$
Можно показать, что если $X(t)$, $t\hm\geqslant0$,~--- процесс Леви такой, что
${\sf P}\left(X(0)\hm=0\right)\hm=1$, то $X(t)$ является $\alpha$-устой\-чи\-вым
тогда и только тогда, когда
\begin{equation}
X(t)\eqd t^{1/\alpha}X(1)\,,\enskip t\geqslant0
\label{e5-zaks}
\end{equation}
(см., например,~\cite{EmbrechtsMaejima2002}).

\section{Основные результаты}

В дальнейшем без существенного ограничения общности будем
рассматривать случайные процессы, определенные при $0\hm\leqslant t\hm\leqslant1$. Это
означает, что по сути рассматривается поведение случайных процессов
на конечном временн$\acute{\mbox{о}}$м горизонте. Добиться равенства правой
границы временн$\acute{\mbox{о}}$го горизонта единице можно, если соответствующим
образом задать единицу измерения времени. Другими словами, внимание
будет сосредоточено на изучении пространства Скорохода~$\mathcal{D}$.

Поскольку, как уже говорилось, в большинстве приложений нет никаких
оснований, чтобы отвергнуть предположение о конечности дисперсий
элементарных скачков обобщенных процессов Кокса, рассматриваем
только случай конечных дисперсий.

Пусть $Z_n(t)$~--- последовательность обобщенных процессов Кокса:
\begin{equation}
Z_n(t)=\sum\limits_{i=1}^{N^{(n)}_{1}(\Lambda_n(t))}X_{n,i}\,,\enskip
t\geqslant0\,,\label{e6-zaks}
\end{equation}
где
$\{N^{(n)}_{1}(t),\, t\hm\geqslant0\}_{n\geqslant1}$~--- последовательность
пуассоновских процессов с единичной интен\-сив\-ностью; при каждом
$n=1,2,\ldots$ случайные величины $X_{n,1},X_{n,2},\ldots$ одинаково
распределены, \mbox{причем} при каждом $n\hm\geqslant1$ величины
$X_{n,1},X_{n,2},\ldots$ и процесс $N^{(n)}_{1}(t)$, $t\hm\geqslant0$,
независимы; при каждом $n=1,2,\ldots$ $\Lambda_n(t)$, $t\hm\geqslant0$,~--- это
процесс Леви, независимый от процесса
\begin{equation}
X_n(t)=\sum\limits_{i=1}^{N^{(n)}_{1}(t)}X_{n,i}\,,\enskip t\geqslant 0\,,\label{e7-zaks}
\end{equation}
и такой, что $\Lambda_n(0)\hm=0$ и существуют $\delta\hm\in(0,1]$,
$\delta_1\hm\in(0,1]$ и не зависящие от $t$ числа $C_n\hm\in(0,\infty)$,
гарантирующие справедливость неравенства
\begin{equation}
{\sf E}\Lambda^{\delta}_n(t)\leqslant (C_nt)^{\delta_1}\label{e8-zaks}
\end{equation}
для всех $t\in(0,1]$. Здесь и далее для определенности полагаем
$\sum\limits_{i=1}^0\hm=0$.


Предположим, что ${\sf E}X_{n,1}\hm=0$ и
\begin{equation}
0<m_n^{\beta}\equiv{\sf E}|X_{n,1}|^{\beta}<\infty
\label{e9-zaks}
\end{equation}
для некоторого $\beta\hm\in[1,2]$.

Из~(\ref{e6-zaks}) и~(\ref{e7-zaks}) легко видеть, что $Z_n(t)\hm=X_n(\Lambda_n(t))$. Поскольку
при каждом $n\geqslant1$ $X_n(t)$ и $\Lambda_n(t)$~--- независимые процессы
Леви, то суперпозиция $Z_n(t)\hm=X_n(\Lambda_n(t))$ также является
процессом Леви (см., например, теорему~3.1.1 в~\cite{Kashcheev2001}).
Отсюда, в частности, вытекает

\smallskip

\noindent
\textbf{Лемма 2}. \textit{Для любых $0\hm\leqslant t_1\hm<t_2\hm<\infty$ и любого
$n\hm\geqslant1$}
$$
Z_n(t_2)-Z_n(t_1)\eqd Z_n(t_2-t_1)\,.
$$

\smallskip

Также справедлива

\smallskip

\noindent
\textbf{Лемма 3}. \textit{Пусть $Z_n(t)$~--- обобщенный процесс Кокса~$(\ref{e6-zaks})$,
удовлетворяющий условиям $(\ref{e8-zaks})$ и~$(\ref{e9-zaks})$. Тогда при каждом
$t\hm\in[0,1]$ для любого $\epsilon\hm>0$}
$$
{\sf P}\left(|Z_n(t)|\geqslant\epsilon\right)\leqslant
\fr{m_n^{\beta\delta}}{\epsilon^{\beta\delta}} \left(C_nt\right)^{\delta_1}\,.
$$

\smallskip

\noindent
Д\,о\,к\,а\,з\,а\,т\,е\,л\,ь\,с\,т\,в\,о\,.\ С~учетом того, что одномерные
распределения процесса Кокса~(\ref{e6-zaks}) являются смешанными пуассоновскими,
имеем:
\begin{multline}
{\sf P}\left(|Z_n(t)|\geqslant\epsilon\right)=
{\sf P}\left(\left|\sum\limits_{j=1}^{N^{(n)}_1(\Lambda_n(t))}X_{n,j}\right|\geqslant\epsilon\right)={}\\
{}=
\sum\limits_{k=0}^{\infty}{\sf P}\left(N^{(n)}_1(\Lambda_n(t))=k\right){\sf
P}\left(\left\vert\sum\limits_{j=1}^kX_{n,j}\right\vert \geqslant\epsilon\right)={}
\\
{}=\sum\limits_{k=0}^{\infty}\left(\int\limits_{0}^{\infty}
e^{-\lambda}\fr{\lambda^k}{k!}\,d {\sf P}\left(\Lambda_n(t)<\lambda\right)\right)\times{}\\
{}\times{\sf P}
\left(\left\vert \sum\limits_{j=1}^kX_{n,j}\right\vert\geqslant\epsilon\right)=
\int\limits_{0}^{\infty}\left[
\vphantom{\left(\left\vert\sum\limits_{j=1}^kX_{n,j}\right\vert
\geqslant\epsilon\right)}
\sum\limits_{k=0}^{\infty}e^{-\lambda}\fr{\lambda^k}{k!}
\times{}\right.\\
\left.{}\times {\sf P}\left(\left\vert\sum\limits_{j=1}^kX_{n,j}\right\vert
\geqslant\epsilon\right)
\right]\,d{\sf P}\left(\Lambda_n(t)<\lambda\right)\,.\label{e10-zaks}
\end{multline}
Изменение порядка суммирования и интегрирования возможно в силу
очевидной равномерной сходимости ряда. Цепочку~(\ref{e10-zaks}) продолжим,
последовательно применяя неравенства Маркова и Иенсена с
$\delta\hm\in(0,1]$, фигурирующим в условии~(\ref{e8-zaks}), и $\beta\hm\in[1,2]$,
фигурирующим в условии~(\ref{e9-zaks}), и получим:
\begin{multline}
{\sf P}\left(|Z_n(t)|\geqslant\epsilon\right)\leqslant
\fr{1}{\epsilon^{\beta\delta}}\times{}\\
{}\times\int\limits_{0}^{\infty}\left[
\sum\limits_{k=0}^{\infty}e^{-\lambda}\fr{\lambda^k}{k!}\,{\sf
E}\left\vert \sum\limits_{j=1}^kX_{n,j}\right\vert^{\beta\delta}\right]\,
d{\sf P}\left(\Lambda_n(t)<\lambda\right) \leqslant{}
\\
\leqslant
\fr{1}{\epsilon^{\beta\delta}}\int\limits_{0}^{\infty}\left[
\vphantom{\left({\sf E}\left|\sum\limits_{j=1}^kX_{n,j}\right|^{\beta}\right)^{\delta}}
\sum\limits_{k=0}^{\infty}e^{-\lambda}\fr{\lambda^k}{k!}\times{}\right.\\
\left.{}\times
\left({\sf E}\left|\sum\limits_{j=1}^kX_{n,j}\right|^{\beta}\right)^{\!\delta}\right]\,d{\sf P}\left(\Lambda_n(t)<\lambda\right)\,,
\label{e11-zaks}
\end{multline}
поскольку для $\delta\hm\in(0,1]$ функция $f(x)\hm=x^{\delta}$ является
вогнутой при $x\hm\geqslant0$. Легко видеть, что при $1\hm\leqslant\beta\hm\leqslant2$
$$
{\sf E}\left|\sum\limits_{j=1}^kX_{n,j}\right|^{\beta}\leqslant\sum\limits_{j=1}^k{\sf E}|X_{n,j}|^{\beta}=km_n^{\beta}\,.
$$
Поэтому, продолжив~(\ref{e11-zaks}) с учетом неравенства Иенсена для вогнутых
функций и условия~(\ref{e8-zaks}), получим:

\noindent
\begin{multline*}
{\sf P}\left(|Z_n(t)|\geqslant\epsilon\right)\leqslant{}\\
{}\leqslant
\fr{m_n^{\beta\delta}}{\epsilon^{\beta\delta}}\int\limits_{0}^{\infty}
\left(\sum\limits_{k=0}^{\infty}e^{-\lambda}\fr{k^{\delta}\lambda^k}{k!}\right)\,d{\sf P}\left(\Lambda_n(t)<\lambda\right)={}\\
{}=
\fr{m_n^{\beta\delta}}{\epsilon^{\beta\delta}}\int\limits_{0}^{\infty}{\sf E}\left[N^{(1)}_1(\lambda)\right]^{\delta}\,d{\sf P}
\left(\Lambda_n(t)<\lambda\right)\leqslant{}
\\
\leqslant\fr{m_n^{\beta\delta}}{\epsilon^{\beta\delta}}\int\limits_{0}^{\infty}\left[{\sf E}N^{(1)}_1(\lambda)\right]^{\delta}\,d{\sf P}
\left(\Lambda_n(t)<\lambda\right)={}\\
{}=\fr{m_n^{\beta\delta}}{\epsilon^{\beta\delta}}\int\limits_{0}^{\infty}\lambda^{\delta}\,d{\sf P}
\left(\Lambda_n(t)<\lambda\right)={}\\
{}=
\fr{m_n^{\beta\delta}}{\epsilon^{\beta\delta}}{\sf E}\Lambda_n^{\delta}(t)\leqslant
\fr{m_n^{\beta\delta}}{\epsilon^{\beta\delta}}\left(C_nt\right)^{\delta_1}\,.
\end{multline*}
Лемма доказана.

\smallskip

Чтобы установить слабую сходимость случайных процессов $Z_n(t)$ в
пространстве Скорохода~$\mathcal{D}$, сначала необходимо найти
предельное распределение случайных величин $Z_n(t)$ при каждом
$t\hm>0$.

Пусть $t=1$. Обозначим $N_n\hm=N^{(n)}_{1}(\Lambda_n(1))$. Предположим,
что для некоторых $k_n\hm\in\mathbb{N}$ при $n\hm\to\infty$ имеет место
сходимость
\begin{equation}
{\sf P}(X_{n,1}+\cdots +X_{n,k_n}<x)\longrightarrow \Phi(x)\,.\label{e12-zaks}
\end{equation}
Обозначим $\sigma_n^2={\sf D}X_{n,1}\ (=m_n^2)$. Из классической
теории предельных теорем хорошо известно, что сходимость~(\ref{e12-zaks}) имеет
место, если при $n\hm\to\infty$ выполнены условия:
\begin{align}
k_n\sigma_n^2&\longrightarrow 1
\label{e13-zaks}
\\
k_n{\sf E}X_{n,1}^2 \mathbb{I}(|X_{n,1}|\geqslant\epsilon)&\longrightarrow 0
\label{e14-zaks}
\end{align}
для любого $\epsilon\hm>0$ (условие Линдеберга).

Предположим, что
\begin{equation}
\fr{\Lambda_n(1)}{k_n}\Longrightarrow U\,,\label{e15-zaks}
\end{equation}
где $U$~--- некоторая неотрицательная случайная величи\-на,
распределение которой не является вы\-рож\-ден\-ным в нуле. Заметим, что
так как $\Lambda_n(t)$~--- процесс Леви, то случайная величина~$U$
безгранично делима как слабый предел безгранично делимых случайных
величин.

Как показано в~\cite{GnedenkoKorolev1996} (также см., например,~\cite{BeningKorolev2002}
или~\cite{KorolevBeningShorgin2011}),
сходимость~(\ref{e15-zaks}) эквивалентна тому, что
\begin{equation}
\fr{N_n}{k_n}\Longrightarrow U\,.\label{e16-zaks}
\end{equation}
По теореме переноса Гне\-ден\-ко--Фа\-хи\-ма~\cite{GnedenkoFahim1969} (см.\
теорему~2.9.1 в~\cite{KorolevBeningShorgin2011}) из условий~(\ref{e12-zaks}) и~(\ref{e16-zaks})
вытекает, что при $n\hm\to\infty$

\noindent
\begin{equation}
Z_n(1)=X_{n,1}+\cdots +X_{n,N_n}\Longrightarrow Z\,,\label{e17-zaks}
\end{equation}
где $Z$~--- случайная величина с характеристической функцией
$$
\mathfrak{f}(s)=\int\limits_{0}^{\infty}\exp\left\{-\fr{us^2}{2}\right\}\,d{\sf P}(U<u)\,.
$$
Несложно увидеть, что функция распределения $F(x)$ случайной
величины~$Z$ является масштабной смесью нормальных законов:
$$
F(x)=\int\limits_{0}^{\infty}\Phi\left(\fr{x}{\sqrt{u}}\right)\,d{\sf P}(U<u)\,,\enskip
x\in\mathbb{R}\,.
$$

Поскольку распределение случайной величины~$U$ безгранично делимо,
то и функция распределения $F(x)$ безгранично делима (см., например,~\cite{Feller1984}).
Поэтому можно определить процесс Леви $Z(t)$,
$t\hm\geqslant0$, такой, что $Z(1)\eqd Z$. Так как согласно~(\ref{e17-zaks})
$$
Z_n(1)=\sum\limits_{i=1}^{N_n}X_{n,i}\Longrightarrow Z(1)\,,
$$
а процессы $Z_n(t)$ и $Z(t)$ являются процессами Леви, то, используя
свойство характеристических функций процессов Леви~(\ref{e2-zaks}), можно
заключить, что для любого $t\hm>0$ имеет место сходимость
\begin{equation}
Z_n(t)=\sum\limits_{i=1}^{N_{n,1}(\Lambda_n(t))}X_{n,i}\Longrightarrow Z(t)
\enskip (n\to\infty)\,.\label{e18-zaks}
\end{equation}
Поскольку случайные процессы $Z_n(t)$, $0\hm\leqslant t\hm\leqslant T$, и $Z(t)$,
$0\hm\leqslant t\hm\leqslant T$, являются процессами Леви, почти все их траектории
принадлежат пространству Скорохода~${\cal D}$.

Рассмотрим вопрос о том, какие дополнительные условия нужны для
слабой сходимости обобщенного процесса Кокса $Z_n(t)$ к процессу
Леви $Z(t)$ в пространстве~${\cal D}$ при $n\hm\to\infty$.
Последовательно рассмотрим каждое из условий теоремы~A.

Во-первых, без ограничения общности положим $0\hm\leqslant t_1\hm<t_2<\cdots <t_k
\hm\leqslant 1$. Слабая сходимость
$$
\left(Z_n(t_1),\ldots ,Z_n(t_k)\right)\Longrightarrow
\left(Z(t_1),\ldots ,Z(t_k)\right)\  (n\to\infty)
$$
эквивалентна сходимости
\begin{multline}
\hspace*{-10pt}\left(Z_n(t_1),Z_n(t_2)-Z_n(t_1),\ldots ,Z_n(t_k)-Z_n(t_{k-1})\right)\!
\Longrightarrow{}\hspace*{-2.87125pt}
\\
{}\Longrightarrow
\left(Z(t_1),Z(t_2)-Z(t_1),\ldots ,Z(t_k)-Z(t_{k-1})\right)\\
(n\to\infty)\,,\label{e19-zaks}
\end{multline}
поскольку линейное преобразование
$(x_1,x_2,\ldots$\linebreak $\ldots ,x_{k-1},x_k)\longmapsto (x_1,x_2-x_1,\ldots ,x_k-x_{k-1})$
$\mathbb{R}^k$ в $\mathbb{R}^k$ взаимно однозначно и непрерывно в
обе стороны. Но сходимость~(\ref{e19-zaks}) вытекает из~(\ref{e18-zaks}) и того, что и
$Z_n(t)$, и $Z(t)$ являются процессами Леви.

Во-вторых, необходимо проверить условие ${\sf P}\left(Z(1)\hm\neq
Z(1-)\right)\hm=0$. Это условие выполнено тогда и только тогда, когда
$$
\lim\limits_{t\to1-}{\sf P}\left(|Z(1)-Z(t)|>\epsilon\right)=0
$$
для любого $\epsilon\hm>0$ (см.\ соотношение~(15.16) в~[15, с.~177]).
Рассмотрим величину ${\sf P}\left(|Z(1)\hm-Z(t)|\hm>\epsilon\right)$. Так
как $Z(t)$~--- процесс Леви, то по лемме~2 
$$
Z(1)\hm-Z(t)\eqd Z(1-t)\,.
$$
Поэтому
$$
{\sf P}\left(|Z(1)-Z(t)|>\epsilon\right)={\sf P}\left(|Z(1-t)|>\epsilon\right)\,.
$$
Для каждого $\epsilon>0$ и каждого $t\hm\in[0,1]$ найдется такое
$\epsilon_t\hm\in[\epsilon/2,\epsilon]$, что точки $\pm\epsilon_t$
являются точками непрерывности функции распределения случайной
величины $Z(1\hm-t)$. Так как $Z_n(t)\hm\Longrightarrow Z(t)$ при каж\-дом
$t\hm\in[0,1]$, то
$$
{\sf P}\left(|Z(1-t)|>\epsilon_t\right)=\lim_{n\to\infty}{\sf P}\left(|Z_n(1-t)|>\epsilon_t\right)\,.
$$
Таким образом, для каждого $\epsilon\hm>0$ и каждого $t\hm\in[0,1]$
\begin{multline}
{\sf P}\left(|Z(1-t)|>\epsilon\right)\leqslant{\sf P}\left(|Z(1-t)|>\epsilon_t\right)={}\\
{}=\lim\limits_{n\to\infty}{\sf P}
\left(|Z_n(1-t)|>\epsilon_t\right)\,.\label{e20-zaks}
\end{multline}
Продолжив~(\ref{e20-zaks}) с учетом условия~(\ref{e8-zaks}) и применяя лемму~3, для
$\delta\hm\in(0,1]$, фигурирующего в~(\ref{e8-zaks}), имеем
\begin{multline}
{\sf P}\left(|Z(1-t)|>\epsilon\right)\leqslant\sup\limits_n{\sf P}\left(|Z_n(1-t)|>\epsilon_t\right)\leqslant{}\\
{}\leqslant\sup\limits_n\left(\epsilon_t^{-\beta}m_n^{\beta}\right)^{\delta}\left(C_n|1-t|\right)^{\delta_1}\leqslant{}\\
{}\leqslant
\left(2^{\beta\delta}\epsilon^{-\beta\delta}|1-t|\right)^{\delta_1}\sup\limits_nm_n^{\beta\delta}C_n^{\delta_1}\,.
\label{e21-zaks}
\end{multline}
Поэтому если
\begin{equation}
Q\equiv\sup\limits_nC_n^{\delta_1/\delta}m_n^{\beta}<\infty\,,\label{e22-zaks}
\end{equation}
то из~(\ref{e21-zaks}) вытекает, что
\begin{multline*}
\lim\limits_{t\to1-}{\sf P}\left(|Z(1)-Z(t)|>\epsilon\right)\leqslant{}\\
{}\leqslant
4(Q\epsilon^{-\beta})^{\delta}\lim\limits_{t\to1-}|1-t|^{\delta_1}=0\,.
\end{multline*}

В-третьих, проверим условие~(\ref{e1-zaks}) в предположении, что выполнены
условия~(\ref{e8-zaks}) и~(\ref{e22-zaks}). Как уже отмечалось выше, $Z_n(t)$~--- процесс
Леви и, стало быть, имеет независимые приращения. Поэтому
\begin{multline}
{\sf P}\left(|Z_n(t)-Z_n(t_1)|\geqslant\epsilon,\,|Z_n(t_2)-Z_n(t)|\geqslant\epsilon\right)={}\\
{}={\sf P}\left(|Z_n(t)-Z_n(t_1)|\geqslant\epsilon\right)\times{}\\
{}\times{\sf P}\left(|Z_n(t_2)-Z_n(t)|\geqslant\epsilon\right)\,.\label{e23-zaks}
\end{multline}
Рассмотрим первый сомножитель в правой час\-ти~(\ref{e23-zaks}). По лемме~2
$Z_n(t)\hm-Z_n(t_1)\eqd Z_n(t-t_1)$. Учитывая условие~(\ref{e22-zaks}), по лемме~3
получим
\begin{multline}
{\sf P}\left(|Z_n(t)-Z_n(t_1)|\geqslant\epsilon\right)={\sf P}
\left(|Z_n(t-t_1)|\geqslant\epsilon\right)\leqslant{}\\
{}\leqslant \left(Q\epsilon^{-\beta}\right)^{\delta}|t-t_1|^{\delta_1}\,.\label{e24-zaks}
\end{multline}
Для второго сомножителя в правой части~(\ref{e23-zaks}) аналогично получим:
\begin{multline}
{\sf P}\left(|Z_n(t_2)-Z_n(t)|\geqslant\epsilon\right)={\sf P}\left(|Z_n(t_2-t)|\geqslant\epsilon\right)\leqslant{}\\
{}\leqslant
\left(Q\epsilon^{-\beta}\right)^{\delta}|t_2-t|^{\delta_1}\,.\label{e25-zaks}
\end{multline}
Таким образом, из~(\ref{e24-zaks}) и~(\ref{e25-zaks}) вытекает, что
\begin{multline}
{\sf P}\left(|Z_n(t)-Z_n(t_1)|\geqslant\epsilon,\,|Z_n(t_2)-Z_n(t)|\geqslant\epsilon\right)\leqslant{}\\
{}\leqslant
\left(Q\epsilon^{-\beta}\right)^{2\delta}\left[(t-t_1)(t_2-t)\right]^{\delta_1}\,.\label{e26-zaks}
\end{multline}
Несложно видеть, что для любых $t_1\hm\leqslant t\hm\leqslant t_2$
$$
(t-t_1)(t_2-t)\leqslant\fr{1}{4}(t_2-t_1)^2\,.
$$
Подставляя эту оценку в~(\ref{e26-zaks}), получаем
\begin{multline*}
{\sf P}\left(|Z_n(t)-Z_n(t_1)|\geqslant\epsilon,\,|Z_n(t_2)-Z_n(t)|\geqslant\epsilon\right)\leqslant{}\\
{}\leqslant
\epsilon^{-2\beta\delta}\left[\fr{Q(t_2-t_1)}{2}\right]^{2\delta_1}\,.
\end{multline*}
Следовательно, если выполнены условия~(\ref{e8-zaks}) и~(\ref{e22-zaks}), то условие~(\ref{e1-zaks})
выполнено с $F(t)\hm\equiv (1/2)Qt$, $\nu\hm=2\delta$ и
$\gamma\hm=\delta_1$.


Суммируя приведенные выше рассуждения, связанные с проверкой условий
теоремы~A, приходим к следующему утверждению.

\smallskip

\noindent
\textbf{Теорема 1.} \textit{Пусть обобщенные процессы Кокса $Z_n(t)$
$($см.~$(\ref{e6-zaks}))$ управляются процессами Леви $\Lambda_n(t)$,
удовле\-тво\-ря\-ющими условиям~$(\ref{e8-zaks})$ и~$(\ref{e15-zaks})$ с некоторыми
$\delta_1\hm\in(0,1]$ и $k_n\hm\in\mathbb{N}$. Предположим, что случайные
величины $\{X_{n,j}\}_{j\geqslant1}$, $n\hm=1,2,\ldots$,~--- скачки обобщенного
процесса Кокса $Z_n(t)$~--- удовлетворяют условиям~$(\ref{e13-zaks})$ и~$(\ref{e14-zaks})$ с
теми же самыми~$k_n$. Предположим также, что для некоторого
$\beta\hm\in[1,2]$ выполнено условие~$(\ref{e22-zaks})$. Тогда случайные блуждания,
порожденные указанными обобщенными процессами Кокса, слабо сходятся
в пространстве Скорохода~$\mathcal{D}$ к процессу Леви~$Z(t)$ с}
\begin{equation}
\hspace*{-1.5mm}{\sf P}\left(Z(1)<x\right)=
\int\limits_{0}^{\infty}\!\Phi\left(\fr{x}{\sqrt{u}}\right)\,d{\sf P}(U<u)\,,\enskip 
 x\in\mathbb{R}\,.\!\!
\label{e27-zaks}
\end{equation}

\smallskip

\noindent
\textbf{Пример 1.} Предположим, что $\Lambda_n(0)\hm=0$, $\Lambda_n(1)\eqd$\linebreak 
$\eqd\;k_nU^{(n)}_{\alpha,1}$, где $\{k_n\}_{n\geqslant1}$~--- неограниченно
воз\-рас\-та\-ющая последовательность натуральных чисел, а
$U^{(1)}_{\alpha,1},U^{(2)}_{\alpha,1},\ldots$~--- последовательность
одинаково распределенных почти наверное положительных случайных
величин, имеющих одностороннее устойчивое распределение с
параметрами $\alpha\hm\in(0,1]$ и $\theta\hm=1$. Из сказанного выше
вытекает, что
$$
{\sf E}\Lambda^{\delta}_n(1)<\infty
$$
для любого $\delta\hm<\alpha$ и с учетом~(\ref{e5-zaks})
\begin{multline}
\Lambda_n(t)\eqd t^{1/\alpha}\Lambda_n(1)\eqd
t^{1/\alpha}k_nU^{(n)}_{\alpha,1}\eqd{}\\
{}\eqd
t^{1/\alpha}k_nU^{(1)}_{\alpha,1}\,, \enskip t\geqslant0\,,\label{e28-zaks}
\end{multline}
так что
$$
{\sf E}\Lambda_n^{\delta}(t)=t^{\delta/\alpha}k_n^{\delta}{\sf
E}\left(U^{(1)}_{\alpha,1}\right)^{\delta}
$$
и условие~(\ref{e8-zaks}) выполняется с $\delta_1\hm=\delta/\alpha$ и
$C_n\hm=\left(k_n\left[{\sf E}\left(U_{\alpha,1}^{(1)}\right)^{\delta}\right]^{1/\delta}
\right)^{\alpha}$.
Предположим, что~(\ref{e9-zaks}) выполнено при $\beta\hm=2$, т.\,е.\
$$
0<\sigma_n^2\equiv m_n^2<\infty\,.
$$
Тогда $\delta_1/\delta\hm=\alpha$ и
$C_n^{\delta_1/\delta}\sigma_n^2\hm=C_n^{1/\alpha}\sigma_n^2\hm=\left[{\sf
E}\left(U_{\alpha,1}^{(1)}\right)^{\delta}\right]^{1/\delta}k_n\sigma_n^2$.
Поэтому если выполнено условие~(\ref{e13-zaks}), то условие~(\ref{e22-zaks}) выполняется
автоматически.

Более того, в силу~(\ref{e28-zaks}) очевидно, что
$$
\fr{\Lambda_n(1)}{k_n}\eqd \fr{k_nU^{(1)}_{\alpha,1}}{k_n}=U^{(1)}_{\alpha,1}\,.
$$
Поэтому условие~(\ref{e15-zaks}) также выполняется автоматически с
$U\hm=U^{(1)}_{\alpha,1}$. При этом в силу представления~(\ref{e3-zaks})
соотношение~(\ref{e27-zaks}) принимает вид:
$$
{\sf P}\left(Z(1)<x\right)=G_{2\alpha,0}(x)\,,\enskip x\in\mathbb{R}\,.
$$
Таким образом, для рассматриваемой ситуации из теоремы~1 вытекает

\smallskip

\noindent
\textbf{Следствие 1.} \textit{Пусть $\alpha\hm\in(0,1]$ и обобщенные процессы
Кокса $Z_n(t)$ $($см.~$(\ref{e6-zaks}))$ управляются процессами Леви\linebreak
$\Lambda_n(t)$ такими, что $\Lambda_n(1)\eqd k_nU_{\alpha,1}^{(n)}$,
$\{k_n\}_{n\geqslant1}$~--- неограниченно возрастающая последовательность
натуральных чисел, а $U^{(1)}_{\alpha,1},U^{(2)}_{\alpha,1},\ldots$~---
последовательность одинаково распределенных почти наверное
положитель\-ных случайных величин, имеющих одностороннее устойчивое
распределение с параметрами $\alpha$ и $\theta\hm=1$. Предположим, что
случайные величины $\{X_{n,j}\}_{j\geqslant1}$, $n\hm=1,2,\ldots$,~--- скачки
обобщенного процесса Кокса $Z_n(t)$~--- удовлетворяют условиям~$(\ref{e13-zaks})$
и~$(\ref{e14-zaks})$ с теми же самыми числами~$k_n$. Тогда случайные блуждания,
порожденные указанными обобщенными процессами Кокса, слабо сходятся
в пространстве Скорохода $\mathcal{D}$ к $2\alpha$-устой\-чи\-во\-му
процессу Леви~$Z(t)$ с ${\sf P}\left(Z(1)\hm<x\right)\hm=G_{2\alpha,0}(x)$.}

\smallskip

\noindent
\textbf{Пример~2.} Предположим, что $\Lambda_n(t)$~--- процесс Леви с
${\sf E}\Lambda_n(1)\hm<\infty$, $n\hm\in\mathbb{N}$. В~таком случае,
используя представление~(\ref{e2-zaks}) и известное соотношение между
математическими ожиданиями случайных величин и производными
соответствующих характеристических функций, легко убедиться, что
$$
{\sf E}\Lambda_n(t)=t{\sf E}\Lambda_n(1)\,, \enskip t\in[0,1]\,,\
n\in\mathbb{N}\,.
$$
Поэтому в рассматриваемой ситуации условие~(\ref{e8-zaks}) выполнено с
$\delta\hm=\delta_1\hm=1$ и $C_n\hm={\sf E}\Lambda_n(1)$, так что теореме~1
можно придать следующий вид.

\smallskip

\noindent
\textbf{Следствие~2.} \textit{Пусть обобщенные процессы Кокса $Z_n(t)$
$($см.~$(\ref{e6-zaks}))$ управляются процессами Леви $\Lambda_n(t)$,
удовлетворяющими условию~$(\ref{e15-zaks})$ с некоторыми $k_n\hm\in\mathbb{N}$.
Предположим дополнительно, что семейство случайных величин
$\{k_n^{-1}\Lambda_n(1)\}_{n\geqslant1}$ равномерно интегрируемо.
Предположим, что случайные величины $\{X_{n,j}\}_{j\geqslant1}$,
$n\hm=1,2,\ldots$,~--- скачки обобщенного процесса Кокса $Z_n(t)$~---
удовлетворяют условиям~$(\ref{e13-zaks})$ и~$(\ref{e14-zaks})$ с теми же самыми~$k_n$. Тогда
случайные блуж\-да\-ния, порожденные указанными обобщенными процессами
Кокса, слабо сходятся в пространстве Скорохода~$\mathcal{D}$ к
процессу Леви~$Z(t)$, удовлетворяющему соотношению~$(\ref{e27-zaks})$.}

\smallskip

Чтобы убедиться в справедливости следствия~2, достаточно заметить,
что условие~(\ref{e15-zaks}) в совокупности с равномерной интегрируемостью
семейства $\{k_n^{-1}\Lambda_n(1)\}_{n\geqslant1}$ означает, что ${\sf E}U\hm <\infty$ и
$$
\fr{{\sf E}\Lambda_n(1)}{k_n}\longrightarrow {\sf E}U
$$
(см., например,~[25, c.~196]). Поэтому существует конечное
положительное число~$A$ такое, что
\begin{equation}
{\sf E}\Lambda_n(1)\leqslant Ak_n
\label{e29-zaks}
\end{equation}
для всех $n$ начиная с некоторого. Поэтому условие~(\ref{e13-zaks}) и
соотношение~(\ref{e29-zaks}) автоматически влекут
$$
\sup\limits_n\sigma_n^2{\sf E}\Lambda_n(1)<\infty\,,
$$
т.\,е.\ условие~(\ref{e22-zaks}) выполнено с $\beta\hm=2$, $\delta\hm=\delta_1\hm=1$ и
$C_n\hm=\sigma_n^2$. Теперь требуемое утверждение вытекает из теоремы~1.

\smallskip

Следующее утверждение содержит немного более ограничительное, но,
возможно, более удобно проверяемое условие, нежели равномерная
интегрируемость семейства $\{k_n^{-1}\Lambda_n(1)\}_{n\geqslant1}$.

\smallskip

\noindent
\textbf{Следствие 3.} \textit{Пусть обобщенные процессы Кокса $Z_n(t)$
$($см.~$(\ref{e6-zaks}))$ управляются процессами Леви $\Lambda_n(t)$,
удовлетворяющими условию~$(\ref{e15-zaks})$ с некоторыми $k_n\hm\in\mathbb{N}$,
причем для некоторого $r\hm>1$
\begin{equation}
\sup_n{\sf E}\left[\fr{\Lambda_n(1)}{k_n}\right]^r<\infty\,.\label{e30-zaks}
\end{equation}
Предположим, что случайные величины $\{X_{n,j}\}_{j\geqslant1}$,
$n\hm=1,2,\ldots$,~--- скачки обобщенного процесса Кокса $Z_n(t)$~---
удовлетворяют условиям~$(\ref{e13-zaks})$ и~$(\ref{e14-zaks})$ с теми же самыми~$k_n$. Тогда
случайные блуждания, порожденные указанными обобщенными процессами
Кокса, слабо сходятся в пространстве Скорохода~$\mathcal{D}$ к
процессу Леви~$Z(t)$, удовлетворяющему соотношению~$(\ref{e27-zaks})$.}

\smallskip

Чтобы убедиться в справедливости следствия~3, достаточно заметить,
что из условия~(\ref{e30-zaks}) вытекает равномерная интегрируемость семейства
$\{k_n^{-1}\Lambda_n(1)\}_{n\geqslant1}$ (см., например,~[25, с.~198]), и
сослаться на следствие~2.

Частный случай следствия~3 для $r\hm=2$ доказан в~\cite{Kashcheev2000, Kashcheev2001} прямым методом.

{\small\frenchspacing
{%\baselineskip=10.8pt
\addcontentsline{toc}{section}{Литература}
\begin{thebibliography}{99}

\bibitem{Shiryaev1998} \Au{Ширяев А.\,Н.} Основы стохастической
финансовой математики. Т.~1: Факты. Модели.~--- М.: Фазис, 1998.

\bibitem{Korolev2011} \Au{Королев В.\,Ю.} Ве\-ро\-ят\-но\-ст\-но-ста\-ти\-сти\-че\-ские
методы декомпозиции волатильности хаотических процессов.~---
М.: Изд-во Московского университета, 2011.

\bibitem{Mandelbrot1963} \Au{Mandelbrot B.\,B.} The variation of certain
speculative prices~// J.~Business, 1963. Vol.~36. P.~394--419.

\bibitem{Korolev1998} \Au{Королев В.\,Ю.} О~сходимости pаспpеделений
случайных сумм независимых случайных величин к устойчивым законам~//
Теоpия веpоятностей и ее пpименения, 1997. Т.~42. Вып.~4. С. 818--820.

\bibitem{Korolev1999} \Au{Королев В.\,Ю.} О~сходимости pаспpеделений
обобщенных пpоцессов Кокса к устойчивым законам~// Теоpия
веpоятностей и ее пpименения, 1998. Т.~43. Вып.~4. С.~786--792.

\bibitem{GnedenkoKorolev1996} \Au{Gnedenko B.\,V., Korolev~V.\,Yu.}
Random summation: Limit theorems and applications.~--- Boca Raton: CRC Press, 1996.

\bibitem{Korolev1997} \Au{Королев В.\,Ю.} Постpоение моделей
pаспpеделений биpжевых цен пpи помощи методов асимптотической теоpии
случайного суммиpования~// Обозpение пpомышленной и пpикладной
математики, 1997. Т.~4. Вып.~1. С.~86--102.

\bibitem{Korolev2000} \Au{Королев В.\,Ю.} Асимптотические свойства
экстpемумов обобщенных пpоцессов Кокса и их пpименение к некотоpым
задачам финансовой математики~// Теоpия веpоятностей и ее
пpименения, 2000. Т.~45. Вып.~1. С.~182--194.

\bibitem{BeningKorolev2002} \Au{Bening V., Korolev~V.} Generalized
Poisson models and their applications in insurance and finance.~---
Utrecht: VSP, 2002.

\bibitem{KorolevSokolov2008} \Au{Королев В.\,Ю., Соколов И.\,А.}
Математические модели неоднородных потоков экстремальных событий.~---
М.: ТОРУС ПРЕСС, 2008.

\bibitem{KorolevBeningShorgin2011} \Au{Королев В.\,Ю., Бенинг~В.\,Е.,
Шоргин~С.\,Я.} Математические основы теории риска.~--- 2-е изд., перераб.
и доп.~--- М.: Физматлит, 2011.

\bibitem{KorolevSkvortsova2006}
Stochastic models of structural plasma turbulence~/
Eds. V.~Korolev, N.~Skvortsova~N.~--- Utrecht: VSP, 2006.

\bibitem{Kashcheev2000} \Au{Кащеев Д.\,Е.} Функциональные предельные
теоремы для сложных процессов Кокса~// Обозрение прикладной и
промышленной математики, 2000. Т.~7. Вып.~2. С.~494--495.

\bibitem{Kashcheev2001}
\Au{Кащеев Д.\,Е.} Моделирование динамики
финансовых временных рядов и оценивание производных ценных бумаг:
Дисс. \ldots канд. физ.-мат. наук.~--- Тверь: Тверской гос.
ун-т, 2001.

\bibitem{Billingsley} \Au{Биллингсли П.} Сходимость вероятностных мер.~--- М.: Наука, 1977.

\bibitem{Bertoin1996}
\Au{Bertoin J.} L$\acute{\mbox{e}}$vy processes~// Cambridge Tracts in
Mathematics. Vol.~121.~--- Cambridge: Cambridge University Press, 1996.

\bibitem{Sato1999} \Au{Sato K.} L$\acute{\mbox{e}}$vy processes and
infinitely divisible distributions.~--- Cambridge: Cambridge
University Press, 1999.

\bibitem{BarndorffNielsenMikoschResnick2001}
\Au{Barndorff-Nielsen O.\,E.,
Mikosch~T., Resnick~S.\,I.}
L$\acute{\mbox{e}}$vy processes: Theory and
applications.~--- Boston: Birkh$\ddot{\mbox{a}}$user, 2001.

\bibitem{Schoutens2003} \Au{Schoutens W.} L$\acute{\mbox{e}}$vy processes in
finance: Pricing financial derivatives.~--- New York: Wiley, 2003.

\bibitem{Geman2002} \Au{Geman~H.} Pure jump L$\acute{\mbox{e}}$vy processes for
asset price modelling~// J.~Banking Finance, 2002. Vol.~26. No.\,7. P.~1297--1316.

\bibitem{Zolotarev1983} \Au{Золотарев В.\,М.} Одномерные устойчивые
распределения.~--- М.: Наука, 1983.

\bibitem{Feller1984} \Au{Феллер~В.} Введение в теорию вероятностей и
ее приложения. Т.~2.~--- М.: Мир, 1984.

\bibitem{EmbrechtsMaejima2002} \Au{Embrechts P., Maejima~M.}
Selfsimilar processes.~--- Princeton: Princeton University Press, 2002.

\bibitem{GnedenkoFahim1969}
\Au{Гнеденко Б.\,В., Фахим~Х.} Об одной
теореме переноса~// Докл. АН СССР, 1969. Т.~187. Вып.~1. С.~15--17.


\label{end\stat}

\bibitem{Loeve}
\Au{Лоэв М.} Теория вероятностей.~--- М.: ИЛ, 1962.
\end{thebibliography}
}
}

\end{multicols}