\def\stat{bening}

\def\tit{АСИМПТОТИЧЕСКИЕ РАЗЛОЖЕНИЯ ДЛЯ ФУНКЦИЙ РАСПРЕДЕЛЕНИЯ
СТАТИСТИК, ПОСТРОЕННЫХ\\ ПО ВЫБОРКАМ СЛУЧАЙНОГО ОБЪЕМА$^*$}

\def\titkol{Асимптотические разложения для функций распределения
статистик, построенных по выборкам случайного объема}

\def\autkol{В.\,Е.~Бенинг, Н.\,К.~Галиева, В.\,Ю.~Королев}

\def\aut{В.\,Е.~Бенинг$^1$, Н.\,К.~Галиева$^2$, В.\,Ю.~Королев$^3$}

\titel{\tit}{\aut}{\autkol}{\titkol}

{\renewcommand{\thefootnote}{\fnsymbol{footnote}}\footnotetext[1]
{Работа
поддержана Российским фондом фундаментальных исследований (проекты
11-01-00515а, 11-07-00112а, 11-01-12026-офи-м), Министерством
образования и науки РФ (госконтракт 16.740.11.0133).}}

\renewcommand{\thefootnote}{\arabic{footnote}}
\footnotetext[1]{Факультет вычислительной
математики и кибернетики Московского государственного университета
им.\ М.\,В.~Ломоносова; Институт проблем информатики Российской
академии наук, bening@yandex.ru}
\footnotetext[2]{Казахстанский филиал Московского государственного
университета им.\ М.\,В.~Ломоносова, nurgul\_u@mail.ru}
\footnotetext[3]{Факультет вычислительной
математики и кибернетики Московского государственного университета
им.\ М.\,В.~Ломоносова; Институт проблем информатики Российской
академии наук, vkorolev@cs.msu.su}

\vspace*{4pt}

\Abst{Доказана общая теорема переноса,
позволяющая получать асимптотические разложения (а.р.)\ для функций
распределения (ф.р.)\ статистик, основанных на выборках случайного объема,
из а.р.\  для ф.р.\ случайного
объема выборки и а.р.\ для  ф.р.\ статистик, построенных по выборкам неслучайного
объема.}

\vspace*{2pt}

\KW{выборка случайного объема;
асимптотическое разложение; теорема переноса; смесь вероятностных
законов; распределение Лапласа; распределение Стьюдента}

\vspace*{4pt}

\vskip 14pt plus 9pt minus 6pt

      \thispagestyle{headings}

      \begin{multicols}{2}

            \label{st\stat}


\section{Введение}

В классических задачах математической статистики объем выборки,
доступной исследователю, традиционно считается детерминированным и в
асимптотических постановках играет роль (как правило, неограниченно
возрастающего) {\it известного} параметра. В~то же время на практике
час\-то возникают ситуации, когда размер выборки не является заранее
определенным и может рас\-смат\-ри\-вать\-ся как случайный. Такие ситуации,
как правило, связаны с тем, что статистические данные накапливаются
в течение фиксированного времени. Это \mbox{имеет} место, в частности, в
страховании, когда в течение разных отчетных периодов одинаковой
длины (скажем, месяцев) происходит разное число страховых событий
(страховых выплат и/или заключений страховых контрактов); в
медицине, когда число пациентов с тем или иным заболеванием
варьируется от года к году; в технике, когда при испытании на
надежность (скажем, при определении наработки на отказ) разных
партий приборов (изделий), чис\-ло отказавших приборов в разных
партиях оказывается разным. В~таких ситуациях заранее не известное
число наблюдений, которые будут доступны исследователю, разумно
считать случайной величиной (с.в.). Другими словами, в таких ситуациях
объем выборки является не (известным) параметром, а сам становится
{\it наблюдением}, т.\,е.\ статистикой. В~силу указанных
обстоятельств вполне естественным становится изуче\-ние
асимптотического поведения распределений статистик достаточно общего
вида, основанных на выборках случайного объема.

На естественность такого подхода, в частности, обратил внимание
Б.\,В.~Гнеденко в работе~\cite{2-ben}, в которой рассматривались
асимптотические свойства\linebreak распределений выборочных квантилей,
построенных по выборкам случайного объема, и было
продемонстрировано, что при замене неслучайного \mbox{объема} выборки
случайной величиной асимптотические свойства статистик могут
радикально измениться. К~примеру, если объем выборки является
геометрически распределенной с.в., то вместо
ожидаемого в соответствии с классической теорией нормального закона
в качестве асимптотического распределения выборочной медианы
возникает распределение Стьюдента с двумя степенями свободы, хвосты
которого столь тяжелы, что у него отсутствуют моменты порядков,
б$\acute{\mbox{о}}$льших второго. <<Тяжесть>> хвостов асимптотических распределений
имеет же  критически важное значение, в частности, в задачах проверки
гипотез.

Простейшей статистикой является сумма наблюдений. Для выборок
случайного объема число слагаемых в таких суммах само становится
случайным. Асимптотическим свойствам распределений сумм случайного
числа с.в.\ посвящено много работ (см., например,~[1--7]). 
Такого рода суммы находят широкое применение в страховании,
экономике, биологии и~т.\,п.~[2, 5, 7, 8]. В~классической статистике
суммирование наблюдений, как правило, возникает при определении
выборочных средних. При статистическом анализе, основанном на
моделях, в которых объем выборки считается неслучайным,
асимптотическое поведение статистик типа сумм и статистик типа
средних арифметических одинаково~--- эти статистики после нормировки,
обязательной для получения нетривиальных предельных распределений,
становятся неразличимыми. Однако, как уже говорилось, в реальной
практике очень часто объем выборки сам является статистикой и, как
недавно показано, например, в работе~\cite{24-ben}, асимптотическое
поведение статистик типа сумм и статистик типа средних
арифметических при их неслучайной нормировке оказывается различным.
Заметим, что, конечно же, формально допустима и случайная
нормировка, но для построения разумных асимптотических аппроксимаций
для распределений статистик (а именно это и является целью
асимптотической статистики) она не применима. Именно использованием
неслучайной нормировки и объясняется возникновение не <<чис\-то\-го>>
нормального закона, а (разных!) смешанных нормальных предельных
распределений у статистик типа сумм и типа средних арифметических.
При этом различие этих предельных законов может дать дополнительную
информацию о структуре исходных данных.

Более того, в математической статистике и ее приложениях часто
встречаются статистики, которые не являются суммами наблюдений.
Примерами могут служить ранговые статистики, $U$-ста\-ти\-сти\-ки,
линейные комбинации порядковых статистик\linebreak ($L$-ста\-ти\-сти\-ки) и~т.\,п.

В данной работе получены а.р.\ для
ф.р.\ статистик, построенных по выборкам
случайного объема. Эти а.р.\ непосредственно зависят от а.р.\ ф.р.\
случайного объема выборки и а.р.\ ф.р.\ статистики, основанной на
неслучайной выборке. Подобного рода утверждения принято называть
тео\-ре\-ма\-ми переноса. Таким образом, в данной работе доказаны теоремы
переноса для а.р.\ статистик, построенных по выборкам случайного
объема.

В работе приняты следующие обозначения: $\R$~--- множество
вещественных чисел; $\N$~--- множество натуральных чисел; $\Phi(x)$ и
$\varphi(x)$~--- соответственно ф.р.\ и плот\-ность стандартного
нормального закона.

В разд.~2 приведен эвристический вывод основного результата, в
разд.~3--5 содержатся строгая формулировка основной теоремы, ее
доказательство и примеры.

Рассмотрим с.в.\ $N_1, N_2, \ldots$ и  $X_1, X_2,\ldots$, 
заданные на одном и том же вероятностном пространстве
$(\Omega, {\cal A}, {\p})$. В~статистике с.в.\ $X_1, X_2, \ldots X_n$
имеют смысл наблюдений, $n$~--- неслучайный объем выборки, а с.в.\
$N_n$~--- случайный объем выборки, зависящий от натурального
параметра $n\hm\in \N$. Например, если с.в.~$N_n$ имеет геометрическое
распределение вида
$$
{\p}(N_n = k) =  \fr{1}{n} \left(1 - \fr{1}{n}\right)^{k-1}\,,\enskip
 k\in\N\,,
$$
то
$$
\e N_n = n\,,
$$
т.\,е.\ среднее значение случайного объема выборки равно~$n$.

Предположим, что при каждом $n\hm\geq1$ с.в.~$N_n$ принимают только
натуральные значения (т.\,е.\ $N_n\hm\in \N$) и независимы от
последовательности с.в.\ $X_1, X_2, \ldots$ Всюду далее считаем с.в.\
$X_1, X_2, \ldots$ независимыми и одинаково распределенными.

Обозначим через  $T_n=T_n(X_1,\ldots ,X_n)$ некоторую статистику, т.\,е.\
действительную измеримую функцию от наблюдений $X_1,\ldots ,X_n$.
Для каждого  $n\hm\geq1$ определим с.в.\ $T_{N_n}$, полагая
$$
T_{N_n}(\omega)\equiv
T_{N_n(\omega)}(X_1(\omega),\ldots ,X_{N_n(\omega)}(\omega))\,, \enskip
\omega\in\Omega\,.
$$
Можно сказать, что $T_{N_n}$~--- это статистика, построенная на
основе статистики $T_n$ по выборке случайного объема $N_n$.

Сформулируем условие, определяющее а.р.\ для ф.р.\ статистики $T_n$
при неслучайном объеме вы\-борки.

\smallskip

\noindent
\textbf{Условие 1.1.} \textit{Существуют константы  $l\hm\in\N$, $\mu\hm\in\R$,
$\sigma\hm>0$, $\alpha\hm>l/2$, $\gamma\hm\ge0$, $C_1\hm>0$, дифференцируемая
ф.р.\ $F(x)$ и дифференцируемые ограниченные функции $f_j(x)$,
$j=1,\ldots ,l$, такие что}
\begin{multline*}
\sup\limits_x \left|\vphantom{\sum\limits_{j=1}^l}
{\p}\left(\sigma n^\gamma(T_n - \mu) < x\right) -\ F(x)
-{}\right.\\
\left.{}- \sum\limits_{j=1}^l n^{-j/2} f_j(x) \right|  \leq \fr{C_1}{n^{\alpha}}\,,
  \ \ \ n\in\N\,.
\end{multline*}

\smallskip

Следующее условие определяет а.р.\ ф.р.\ нормированного случайного
индекса $N_n$.

\smallskip

\noindent
\textbf{Условие 1.2.} \textit{Существуют константы  $m\hm\in\N$, $\beta\hm>m/2$,
$C_2\hm>0$, функция $0\hm<g(n)\uparrow \infty$, $n\hm\to\infty$, ф.р.~$H(x),
H(0+)\hm=0$, и функции ограниченной вариации $h_i(x)$, $i\hm=1,\ldots ,m$, такие
что}
\begin{multline*}
\sup\limits_{x\ge0} \left|{\p}\left(\fr{N_n}{g(n)} < x\right) - H(x)\ -
\sum\limits_{i=1}^m n^{-i/2} h_i(x) \right|  \leq{}\\
{}\leq \fr{C_2}{n^{\beta}}\,, \ \
\  n\in\N\,.
\end{multline*}

\smallskip

В данной работе с помощью а.р.\ для ф.р.\ нормированной статистики
$T_{N_n}$, основанной на выборке случайного объема, получена
аппроксимация вида
\begin{equation*}
{\p}\left(\sigma g^\gamma(n)(T_{N_n}  -  \mu)  <  x\right)\ \approx
G_{n}(x)\,, \ \  n \to \infty\,,
%\label{e1.1-ben}
\end{equation*}
где функция $G_n(x)$ имеет вид (см.\ условия~1.1, 1.2):
\begin{multline}
G_n(x) = {}\\
{}=\!\!\int\limits_{1/g(n)}^\infty\!\!\!\! F(x y^\gamma)\, dH(y) +
\sum\limits_{i=1}^m n^{-i/2}\! \int\limits_{1/g(n)}^\infty \!\!F(xy^\gamma)\,
dh_i(y)\ +{}
\\
{}+ \sum\limits_{j=1}^l g^{-j/2}(n)\! \int\limits_{1/g(n)}^\infty \!
y^{-j/2}f_j(xy^\gamma)\, dH(y) +{}\\
\hspace*{-3mm}{}+
\sum\limits_{j=1}^l \sum\limits_{i=1}^m n^{-i/2}g^{-j/2}(n)
\!\!\!\int\limits_{1/g(n)}^\infty y^{-j/2}\!\!f_j(xy^\gamma)
\,dh_i(y).\!
\label{e1.2-ben}
\end{multline}
Для пояснения этой формулы, идеи доказательства и удобства
дальнейших ссылок приведем ее эвристический вывод.

\section{Эвристический вывод основного результата}

В идейном плане доказательство основного результата данной работы~---
теоремы~3.1~--- близко к доказательству теорем 6.6.1 и 6.7.3 для
случайных сумм из работы~\cite{6-ben} и оценкам скорости сходимости
распределений случайно индексированных последовательностей из работы~\cite{14-ben} 
($\S$~1.3, с.~63).

По формуле полной вероятности имеем:
\begin{multline}
{\p}\left(\sigma g^\gamma(n)(T_{N_n} - \mu) < x\right)  ={}\\
{}=
{\p}\left(\sigma N_n^\gamma(T_{N_n} - \mu) <
\left(\fr{N_n}{g(n)}\right)^\gamma x\right)  ={}\\
{}
= {\e} {\p}\left(\sigma N_n^\gamma(T_{N_n} - \mu) <
\left(\fr{N_n}{g(n)}\right)^\gamma x \Big| N_n\right)  ={}\\
\!{}=
\sum\limits_{k=1}^{\infty} {\p}\left(\!\sigma k^\gamma(T_{k}  -  \mu) <
\left(\fr{k}{g(n)}\right)^\gamma\!\! x\right){\p}(N_n=k).\!\!
\label{e2.1-ben}
\end{multline}
Используя условие~1.1, вероятность под знаком ряда аппроксимируем
следующим образом:
\begin{multline*}
\!\!\!\!{\p}\left(\sigma g^\gamma(n)(T_{N_n} - \mu) < x\right) \approx
\sum\limits_{k=1}^{\infty} \!\left(\!
\vphantom{\sum\limits_{j=1}^l\
k^{-j/2}f_j\left(x\left(\fr{k}{g(n)}\right)^\gamma\right)}
F\left(x
\left(\fr{k}{g(n)}\right)^\gamma\right) + {}\right.\hspace*{-1.8927pt}\\
\left.{}+\sum\limits_{j=1}^l\
k^{-j/2}f_j\left(x\left(\fr{k}{g(n)}\right)^\gamma\right)\right)
{\p}(N_n=k) ={}
\end{multline*}

\noindent
\begin{multline}
{}
= {\e} \left( \vphantom{\sum\limits_{j=1}^l\
k^{-j/2}f_j\left(x\left(\fr{k}{g(n)}\right)^\gamma\right)}
F\left(x \left(\fr{N_n}{g(n)}\right)^\gamma\right) +{}\right.\\
\left.{}+
\sum\limits_{j=1}^l 
N_n^{-j/2}f_j\left(x\left(\fr{N_n}{g(n)}\right)^\gamma\right)\right)
= \!\int\limits_{1/g(n)}^\infty \!\left(
\vphantom{\sum\limits_{j=1}^l}
F(x y^\gamma) +{}\right.\\
\left.{}+ \sum\limits_{j=1}^l
\left(yg(n)\right)^{-j/2}f_j(xy^\gamma)\right)\, d
{\p}\left(\fr{N_n}{g(n)} < y\right)\,.\label{e2.2}
\end{multline}
Теперь, аппроксимируя вероятность под знаком последнего интеграла с
помощью условия~1.2, получим формулу~(\ref{e1.2-ben}):
\begin{multline}
\!\!\!{\p}\left(\sigma g^\gamma(n)(T_{N_n} - \mu) < x\right) \approx G_n(x)
=\!\!\! \int\limits_{1/g(n)}^\infty\! \!\!\left(\!
\vphantom{\sum\limits_{j=1}^l}
F(x y^\gamma) + {}\right.\hspace*{-3.19522pt}\\
\left.{}+\sum\limits_{j=1}^l
\left(yg(n)\right)^{-j/2}f_j(xy^\gamma)\right)\, d \left(
\vphantom{\sum\limits_{i=1}^m}
H(y) +{}\right.\\
\left.{}+
\sum\limits_{i=1}^m n^{-i/2} h_i(y)\right) ={}\\
{}
 = \!\!\int\limits_{1/g(n)}^\infty\!\! F(x y^\gamma)\, dH(y) + 
 \sum\limits_{i=1}^m n^{-i/2} \!\!\int\limits_{1/g(n)}^\infty \!\!F(xy^\gamma) \,dh_i(y) +{}
\\
{}+ \sum\limits_{j=1}^l g^{-j/2}(n) \!\int\limits_{1/g(n)}^\infty\!
y^{-j/2}f_j(xy^\gamma) \,dH(y) +{}\\
\!\!\!{}+
\sum\limits_{j=1}^l \sum\limits_{i=1}^m n^{-i/2}g^{-j/2}(n)\!\!
\int\limits_{1/g(n)}^\infty \!\!y^{-j/2}f_j(xy^\gamma)
\,dh_i(y).\!\!
\label{e2.3-ben}
\end{multline}
Если в условии~1.1 статистика $T_n$ не нормирована, т.\,е.\
$\gamma\hm=0$, то полученное а.р.\ приобретает вид:
\begin{multline*}
G_n(x) =   F(x)\left(1 - H\left(\fr{1}{g(n)}\right)\right) +
 \sum\limits_{j=1}^l \left(g(n)\right)^{-j/2}\times{}\\
{}\times f_j(x)
\int\limits_{1/g(n)}^\infty y^{-j/2} d \left(H(y) + \sum\limits_{i=1}^m
n^{-i/2} h_i(y)\right)\,. %\label{e2.4-ben}
\end{multline*}
Таким образом, у ненормированной статистики $T_n$ исходное а.~р.
$$
{\p}\left(\sigma (T_n  -  \mu)  <  x\right) \approx  F(x) +
\sum\limits_{j=1}^l n^{-j/2} f_j(x)
$$
при переходе к выборке случайного объема $N_n$  заменяется на а.р.\
вида:

\noindent
\begin{multline*}
{\p}\left(\sigma (T_{N_n}  -  \mu)  <  x\right) \approx {}\\
{}\approx  F(x)\left(1 -
H\left(\fr{1}{g(n)}\right)\right) + \sum\limits_{j=1}^l c_{jn} f_j(x)\,,
\end{multline*}
где
\begin{multline*}
c_{jn} = {}\\
\hspace*{-4pt}{}=\left(g(n)\right)^{-j/2} \!\!\!\int\limits_{1/g(n)}^\infty\!\!\!
y^{-j/2} \,d \left(H(y) + \sum\limits_{i=1}^m n^{-i/2} h_i(y)\right).
\end{multline*}

\section {Формулировка основного результата}

\noindent
\textbf{Теорема~3.1.} \textit{Пусть  статистика
$T_n\hm=T_n(X_1,\ldots$\linebreak $\ldots ,X_n)$ удовлетворяет условию~$1.1,$ а случайный
объем выборки $N_n$~--- условию~$1.2$. Тогда  существует константа
$C_3\hm>0$ такая, что справедливо неравенство:
\begin{multline*}
\sup\limits_x \left| \p\left(\sigma g^\gamma(n)(T_{N_n}  -  \mu) < x\right)
-  G_n(x)\right| \leq {}\\
{}\leq C_1 \e N_n^{-\alpha} + \fr{C_3 + C_2
M_n}{n^\beta}\,,
\end{multline*}
где
\begin{multline*}
M_n = \sup_x \int\limits_{1/g(n)}^\infty
\Bigl|\fr{\partial}{\partial y} \bigl(F(xy^\gamma) +{}\\
{}+ \sum_{j=1}^l
(yg(n))^{-j/2}\ f_j(xy^\gamma)\bigr)\Bigr|\, dy
\end{multline*}
и а.р.\ $G_n(x)$ определено по формуле}~(\ref{e1.2-ben}).

\smallskip

\noindent
\textbf{Следствие 3.1.} \textit{Если моменты $\e (N_n/g(n))^{-\alpha}$
равномерно по $n$ ограничены, т.\,е.\
$$
\e \left(\fr{N_n}{g(n)}\right)^{-\alpha} \leq C_4\,,  \enskip C_4 >
0\,,\ \ n \in \N\,,
$$
то правая часть неравенства в формулировке теоремы~$3.1$ приобретает
вид:}
$
{C_1 C_4}/{g^\alpha(n)} \hm+  (C_3 \hm+ C_2 M_n)/{n^\beta}.
$


\smallskip

\noindent
\textbf{Следствие~3.2.} \textit{В силу неравенства Гёльдера при
$0\hm<\alpha\hm\leq1$ справедлива оценка
$$
\e N_n^{-\alpha} \leq \left(\e N_n^{-1}\right)^\alpha\,,
$$
которая может быть полезной при практическом применении теоремы. 
В~этом случае правая часть неравенства из формулировки теоремы может
быть записана в виде:}
$
C_1 \left(\e N_n^{-1}\right)^\alpha  +  ({C_3 + C_2
M_n})/{n^\beta}.
$


\smallskip

Для вычисления  $\e N_n^{-1}$ можно использовать следующую формулу
(см., например,~\cite{25-ben},  с.~93, задача~40,\,б). Если
неотрицательная целочисленная с.в.~$N$  имеет производящую функцию
$$
\Psi(s) =  \e s^{N}\,,   \enskip  |s| \le 1\,,
$$
то из теоремы Фубини непосредственно следует, что
\begin{equation}
\e N^{-1} = \int\limits_0^1 \fr{\Psi(s)}{s}\, ds\,. \label{e3.1-ben}
\end{equation}
Используя это соотношение, оценку из формулировки теоремы можно
представить  в виде:
\begin{equation}
C_1 \left(\int\limits_0^1 \fr{\Psi_n(s)}{s}\, ds\right)^\alpha  +
\fr{C_3 + C_2 M_n}{n^\beta}\,,\label{e3.2-ben}
\end{equation}
где $\Psi_n(s)$~--- производящая функция с.в.~$N_n$.

Приведем пример использования формулы~(\ref{e3.1-ben}). Пусть с.в.~$N_n$ имеет
геометрическое распределение:
$$
{\p}(N_n = k) =  \fr{1}{n} \left(1 - \fr{1}{n}\right)^{k-1}\,,\enskip
 k\in\N\,.
$$
В этом случае производящая функция имеет вид:
\begin{multline*}
\Psi_n(s) =  \e s^{N_n} =  \fr{s}{n}\left[1 -  s\left(1 -
\fr{1}{n}\right)\right]^{-1}={}\\
{}=\fr{s}{n-s(n-1)}\,, \enskip   |s| \le 1\,,
\end{multline*}
поэтому
\begin{equation}
\e N_n^{-1} = \int\limits_0^1 \fr{\Psi_n(s)}{s}\, ds = \fr{1}{n -
1} \,\log n\,,  \enskip  n > 1\,. \label{e3.3-ben}
\end{equation}
С учетом формулы~(\ref{e3.3-ben}) оценка~(\ref{e3.2-ben}) принимает вид:
$(C_1 \log^{\alpha} n)/(n - 1)^\alpha  +  (C_3 + C_2
M_n)/n^\beta$, $n \hm> 1$.

\smallskip

\textbf{Замечание~3.1.} Заметим, что из условия~1.2, в частности,
вытекает, что с.в.\ $N_n/g(n)$ слабо сходится к с.в.~$V$,  имеющей
ф.р.~$H(x)$. Из определения слабой сходимости с функцией
$x^{-\alpha}$, $ x \hm\ge 1$, в случае, если $N_n \hm\ge g(n)$, $n\hm\in\N$,
следует, что
$$
\e \left(\fr{N_n}{g(n)}\right)^{-\alpha} \longrightarrow   \e
\fr{1}{V^\alpha}\,,  \enskip  n\to\infty\,,
$$
т.\,е.\  моменты  $\e (N_n/g(n))^{-\alpha}$ равномерно ограничены по~$n$  
и справедливо утверждение из следствия~3.1.

\smallskip

Случай, когда $N_n\hm\ge g(n)$, возникает, например, если с.в.\ $N_n$
принимает значения $g(n), 2g(n), \ldots , kg(n)$  с равными
вероятностями    $1/k$ при любом фиксированном $k\hm\in\N$. В~этом
случае с.в.\ $N_n/g(n)$  вообще не зависит от~$n$ и, значит, слабо
сходится к с.в.~$V$,  которая принимает значения $1, 2, \ldots , k$ с
равными вероятностями~$1/k$.

\section{Доказательство теоремы~3.1}

Используя формулы~(\ref{e2.1-ben})--(\ref{e2.3-ben}),  получаем оценку:
\begin{multline}
\sup\limits_{x}\left| \p\left(\sigma g^\gamma(n)(T_{N_n}  -  \mu)  <
x\right) -  G_n(x)\right| \leq {}\\
{}\leq I_{1n} + I_{2n}\,, \label{e4.1-ben}
\end{multline}
где
\begin{multline}
I_{1n} =
 \sup\limits_{x}\left| \, \int\limits_{1/g(n)}^\infty \left(
 \vphantom{\sum\limits_{j=1}^l}
 F(x y^\gamma) +{}\right.\right.\\
\left. {}+
\sum\limits_{j=1}^l \left(yg(n)\right)^{-j/2}f_j(xy^\gamma)\right)  d
\left({\p}\left(\fr{N_n}{g(n)} < y\right) -{}\right.\\
\left.\left.{}- H(y) - \sum\limits_{i=1}^m
n^{-i/2} h_i(y)\right)
\vphantom{\int\limits_{1/g(n)}^\infty}\right|\,; \label{e4.2-ben}
\end{multline}

\vspace*{-12pt}

\noindent
\begin{multline}
I_{2n}  =  \sum\limits_{k=1}^{\infty} \sup\limits_{x}  \left|
\vphantom{\sum\limits_{j=1}^l}
 {\p}\left(\sigma
k^\gamma(T_{k}  -  \mu)  < x\left(\fr{k}{g(n)}\right)^\gamma\right)\right.
-{}
\\
- F\left(x \left(\fr{k}{g(n)}\right)^\gamma\right) - {}\\
\left.{}-\sum\limits_{j=1}^l
k^{-j/2}f_j\left(x\left(\fr{k}{g(n)}\right)^\gamma\right)\right|
{\p}(N_n=k)\,.    \label{e4.3-ben}
\end{multline}
Для оценки величины  $I_{1n}$ используем равенство~(\ref{e4.2-ben}),  
условие~1.2, формулу интегрирования по частям (см., например,~\cite{5-ben},
теорема~2.6.11, с.~222 или~\cite{15-ben}, теорема~18.4, с.~236) и
ограниченность функций $f_j(z)$, $j=1,\ldots ,l$, получим, что существует
константа $C_3\hm>0$ такая, что
\begin{multline*}
I_{1n} \le \fr{C_3}{n^\beta} + \sup\limits_{x}\left|\,
\int\limits_{1/g(n)}^\infty \left({\p}\left(\fr{N_n}{g(n)} <
y\right) -{}\right.\right.\\
\left.{}- H(y) - \sum\limits_{i=1}^m n^{-i/2} h_i(y)\right) 
d \left(\vphantom{\sum\limits_{j=1}^l}
F(x y^\gamma) + {} \right.\\
\left.\left.{}+\sum\limits_{j=1}^l
\left(yg(n)\right)^{-j/2}f_j(xy^\gamma)\right)
\vphantom{\int\limits_{1/g(n)}^\infty} \right| \le{}
\\
{}
\le \fr{C_3}{n^\beta} +   \sup\limits_{x}  \int\limits_{1/g(n)}^\infty
\left| \vphantom{\sum\limits_{j=1}^l}
{\p}\left(\fr{N_n}{g(n)} < y\right) -{}\right.\\
\left.{}- H(y) - \sum\limits_{i=1}^m
n^{-i/2} h_i(y)\right| \times{}
\end{multline*}

\noindent
\begin{multline}
{}\times  \left|\fr{\partial}{\partial y} \left(F(xy^\gamma) +
\sum\limits_{j=1}^l (yg(n))^{-j/2} f_j(xy^\gamma)\right)\right|\, dy \le{}\\
{}\le
\fr{C_3}{n^\beta} + \fr{C_1M_n}{n^\beta}\,.  \label{e4.4-ben}
\end{multline}
Ряд  в определении $I_{2n}$   (см.~(\ref{e4.3-ben})) оценим с помощью условия~1.1 и получим:
\begin{equation}
I_{2n}  \leq C_1  \sum\limits_{k=1}^{\infty} \fr{1}{k^\alpha}\, {\p}(N_n=k)
= C_1 \e N_n^{-\alpha}\,.    \label{e4.5-ben}
\end{equation}
Теперь утверждение теоремы следует из неравенств~(\ref{e4.1-ben}), (\ref{e4.4-ben}) и~(\ref{e4.5-ben}).
Теорема доказана.


\section{Примеры}

Приведем два примера применения теоремы~3.1 с вполне конкретными
предельными функциями распределения статистик, построенных по
выборкам случайного объема. Рассмотрим а.р.\
для ф.р.\ выборочных средних, построенных по выборкам случайного
объема. Аналогичные результаты могут быть получены для статистик,
допускающих а.р.\ типа Эдж\-вор\-та для ф.р.\ при
неслучайном объеме выборки. Например, используя результаты работ~[13--18], 
можно получить а.р.\ для  ф.р.\
ранговых статистик, $L$-ста\-ти\-стик и $U$-ста\-ти\-стик.

Пусть $X_1,X_2,\ldots$~--- независимые одинаково распределенные
случайные величины с ${\sf E}X_1 \hm= \mu$, $0\hm<{\sf D}X_1
\hm=\sigma^{-2}$, $\e |X_1|^{3+2\delta} \hm< \infty$,
$\delta\hm\in(0,1/2)$ и ${\sf E}(X_1 - \mu)^3 \hm= \mu_3$. Для
натурального~$n$ обозначим
\begin{equation}
T_n = \fr{X_1 + \cdots + X_n}{n}\,.\label{e5.1-ben}
\end{equation}
Предположим также, что случайная величина $X_1$ удовлетворяет
условию Крам$\acute{\mbox{е}}$ра ($C$)
$$
\limsup\limits_{|t|\to\infty} |\e \exp\{itX_1\}| < 1\,,
$$
тогда с учетом  теоремы~6.3.2 из~\cite{22-ben} получаем, что
\begin{multline}
\sup\limits_x \left|\vphantom{\fr{\mu_3 \sigma^3}{6 \sqrt n}}
{\p}\left(\sigma \sqrt n(T_n - \mu) < x\right) - \Phi(x)
- {}\right.\\
\left.{}-\fr{\mu_3 \sigma^3}{6 \sqrt n} \left(1 - x^2\right) \varphi(x) \right| \leq
\fr{C_1}{n^{1/2+\delta}}\,,  \\  C_1 > 0\,, \  \delta \in
\left(0,\fr{1}{2}\right)\,,
\   n \in \N\,. \label{e5.2-ben}
\end{multline}
Таким образом, статистика~(\ref{e5.1-ben}) удовлетворяет условию~1.1~с

\noindent
\begin{gather}
\gamma = \fr{1}{2}\,; \ \   \alpha = \fr{1}{2} +\delta\,; \ \ \   l = 1\,;
\label{e5.3-ben}
\\
F(x) = \Phi(x)\,;\ \    f_1(x) = \fr{\mu_3 \sigma^3}{6} (1 -\ x^2)
\varphi(x)\,. \label{e5.4-ben}
\end{gather}
Справедлива следующая лемма.

\smallskip

\noindent
\textbf{Лемма 5.1.} \textit{Пусть $l \hm= 1$,  $0 \hm< g(n) \uparrow \infty$,
$F(x) \hm= \Phi(x)$,
$$
f_1(x) = \fr{\mu_3 \sigma^3}{6} (1 - x^2) \varphi(x)\,.
$$
Тогда для величины $M_n$ $($см.\ теорему~$3.1)$ справедливо
неравенство
$$
M_n \le  2 + \widetilde C|\mu_3|\sigma^3\,,
$$
где}
\begin{multline*}
\widetilde C = \fr{1}{3}\,\sup\limits_{u\ge0}
\left\{\varphi(u)(u^4+2u^2+1)\right\}=\fr{16}{3\sqrt{2\pi
e^3}}\approx {}\\
{}\approx 0{,}474752293191785\ldots
\end{multline*}


\smallskip

\noindent
Д\,о\,к\,а\,з\,а\,т\,е\,л\,ь\,с\,т\,в\,о\,.\ \  С~учетом формул~(\ref{e5.1-ben})--(\ref{e5.4-ben}) 
имеем (см.\ теорему~3.1):
\begin{multline*}
M_n = \sup\limits_x \int\limits_{(g(n))^{-1}}^\infty
\left|
\fr{\partial}{\partial y} \left( \vphantom{\sum\limits_{j=1}^l}
F(xy^\gamma) + {}\right.\right.\\
\left.\left.{}+\sum\limits_{j=1}^l
(yg(n))^{-j/2} f_j(xy^\gamma)\right)\right|\, dy  ={}
\\
{}= \sup\limits_x \int\limits_{(g(n))^{-1}}^\infty
\left|\fr{\partial}{\partial y} \left(
\vphantom{\fr{\mu_3\sigma^3(1-x^2 y)\varphi(x\sqrt y)}{6
\sqrt{yg(n)}}}
\Phi(x\sqrt y) +{}\right.\right.\\
\left.\left.{}+
\fr{\mu_3\sigma^3(1-x^2 y)\varphi(x\sqrt y)}{6
\sqrt{yg(n)}}\right)\right|\, dy \le{}
\\
{}\le \sup\limits_{x\ge0} \int\limits_{x(g(n))^{-1/2}}^\infty
\left|\fr{\partial}{\partial u} \left(
\vphantom{\fr{x\mu_3\sigma^3(1-u^2)\varphi(u)}{6u \sqrt{g(n)}}}
\Phi(u) +{}\right.\right.\\
\left.\left.{}+\fr{x\mu_3\sigma^3(1-u^2)\varphi(u)}{6u \sqrt{g(n)}}\right)\right|
\,du +{}
\\
{}+ \sup\limits_{x<0} \int\limits_{-\infty}^{x(g(n))^{-1/2}}
\left|\fr{\partial}{\partial u} \left(
\vphantom{\fr{x\mu_3\sigma^3(1-u^2)\varphi(u)}{6u \sqrt{g(n)}}}
\Phi(u) +{}\right.\right.\\
\left.\left.{}+
\fr{x\mu_3\sigma^3(1-u^2)\varphi(u)}{6u \sqrt{g(n)}}\right)\right|
\,du ={}
\\
{}= \sup\limits_{x\ge0} \int\limits_{x(g(n))^{-1/2}}^\infty \!\!\!\!\!\!\!\!\!\!\varphi(u)
\left|1 + \fr{x\mu_3\sigma^3}{6
\sqrt{g(n)}}\fr{(u^4-2u^2-1)}{u^2}\right|\, du +{}\hspace*{-5.40286pt}
\end{multline*}

\noindent
\begin{multline}
{}+ \sup\limits_{x<0} \!\!\!\int\limits_{-\infty}^{x(g(n))^{-1/2}} \!\!\!\!\!\!\!\!\!\!\varphi(u)
\left|1 + \fr{x\mu_3\sigma^3}{6
\sqrt{g(n)}}\fr{(u^4-2u^2-1)}{u^2}\right|\, du \le{}\hspace*{-0.40298pt}
\\
\hspace*{-1.8mm}{}\le 2 + \fr{|\mu_3|\sigma^3}{3\sqrt{g(n)}} \sup\limits_{x\ge0} x
\!\!\!\int\limits_{x(g(n))^{-1/2}}^\infty\!\!\!\!\!\!\!\!\!\!\varphi(u)
\fr{u^4+2u^2+1}{u^2} \,du.\!\!\label{e5.5-ben}
\end{multline}
Далее заметим, что справедливо соотношение
$$
\sup\limits_{u\ge0}
\left\{\varphi(u)(u^4+2u^2+1)\right\}=\fr{1}{\sqrt{2\pi}}\,\sup\limits_{u\ge0}e^{-u}(2u+1)^2\,.
$$
Легко видеть, что $\left(e^{-u}(2u+1)^2\right)^\prime\hm=e^{-u}(2u\hm+1)(3\hm-2u)\hm=0$
при $u={3}/{2}$. Таким образом,
\begin{multline*}
\hspace*{-1.36589pt}\fr{1}{\sqrt{2\pi}}\,\sup\limits_{u\ge0}e^{-u}(2u+1)^2
=\fr{1}{\sqrt{2\pi}}\,e^{-u}(2u+1)^2\Big|_{u=3/2}={}\\
{}=\fr{16}{\sqrt{2\pi
e^3}}\approx 1{,}42425687951535\ldots ,
\end{multline*}
так что
\begin{multline}
\widetilde C=\fr{1}{3}\sup\limits_{u\ge0}
\left\{\varphi(u)(u^4+2u^2+1)\right\}=\fr{16}{3\sqrt{2\pi
e^3}}\approx{}\\
{}\approx 0{,}474752293191785\ldots 
\label{e5.6-ben}
\end{multline}
При этом из неравенства~(\ref{e5.5-ben}) следует, что
\begin{multline*}
M_n \le 2 + \widetilde C \fr{|\mu_3|\sigma^3}{\sqrt{g(n)}}
\sup\limits_{x\ge0} x \!\int\limits_{x(g(n))^{-1/2}}^\infty \!\!u^{-2}\, du
={}\\
{}= 2 + \widetilde C|\mu_3|\sigma^3\,.
\end{multline*}
Таким образом, справедливо неравенство:
\begin{equation}
M_n \le  2 + \widetilde C|\mu_3|\sigma^3\,,\label{e5.7-ben}
\end{equation}
где постоянная $\widetilde C$ определена в соотношении~(\ref{e5.6-ben}). Из
неравенства~(\ref{e5.7-ben}) следует утверждение леммы. Лемма доказана.

\subsection{Распределение Стьюдента}

В работе~\cite{3-ben} показано, что если случайный объем выборки $N_n$
имеет отрицательно биномиальное распределение с параметрами $p \hm=
1/n$ и $r \hm> 0$,  т.\,е.\
$$
{\p}(N_n = k) = \fr{(k+r-2)\cdots r}{(k-1)!}\, \fr{1}{n^r} \left(1
- \fr{1}{n}\right)^{k-1}\!\!, \   k\in\N
$$
(при $r=1$ имеем геометрическое распределение), то для
асимптотически нормальной статистики $T_n$ справедливо предельное
соотношение~(\cite{3-ben}, следствие~2.1)
\begin{equation}
{\p}(\sigma\sqrt{n} (T_{N_n} - \mu) < x)\longrightarrow G_{2r}(x
\sqrt r), \ \  n\to\infty,\!\! \label{e5.8-ben}
\end{equation}
где $G_{f}(x)$~--- функция распределения Стьюдента с параметром $f \hm= 2r$, соответствующая 
плотности вида

\noindent
$$
p_{f}(x) = \fr{\Gamma(f+1/2)}{\sqrt{\pi f}\, \Gamma(f/2)}\left(
1+\fr{x^2}{f}\right)^{-(\gamma+1)/2}\,,\enskip   x\in\R\,,
$$
где $\Gamma(\cdot)$~--- эйлерова гам\-ма-функ\-ция, а $f\hm>0$~--- параметр
формы (если параметр $f$ натурален, то он называется числом степеней
свободы). В~рас\-смат\-ри\-ва\-емой ситуации он может быть произвольно мал,
т.\,е.\ может иметь место типичное распределение с тяжелыми
хвостами. Если $f\hm=2$, т.\,е.\ $r\hm=1$, то ф.р.\ $G_2(x)$ выражается в
явном виде:

\noindent
$$
G_2(x) = \fr{1}{2}\left( 1+\fr{x}{\sqrt{2+x^2}} \right)\,, \enskip  x\in\R\,.
$$
При $r=1/2$ имеем распределение Коши.

В книге~\cite{22-ben} (формула~(6.112)) приведена следующая оценка
скорости сходимости:

\noindent
\begin{multline}
\sup_{x\ge0} \left|{\p}\left(\fr{N_n}{\e N_n} <  x\right) -
H_r(x)\right| \leq 
\begin{cases}
\displaystyle\fr{C_r}{n}\,, &  r \ge 1;\\
\displaystyle\fr{C_r}{n^r}\,, &  r \in (0,1),
\end{cases}\\  C_r > 0\,,\ \
  n\in\N\,,\label{e5.9-ben}
\end{multline}
где $H_r(x)$~--- функция гам\-ма-рас\-пре\-де\-ле\-ния с параметром $r \hm> 0$:

\noindent
\begin{equation}
H_r(x) = \fr{r^r}{\Gamma(r)} \int\limits_0^x e^{-ry} y^{r-1} \,dy\,, 
\enskip
 x\ge 0\,,\label{e5.10-ben}
\end{equation}
 При этом
\begin{equation}
\e N_n = r(n - 1) + 1\,. \label{e5.11-ben}
\end{equation}
Таким образом, из соотношений~(\ref{e5.9-ben})--(\ref{e5.11-ben}) следует, что случайный
индекс $N_n$ удовлетворяет условию~1.2 с
\begin{gather*}
g(n) = r(n - 1) + 1\,; \enskip  H(x) = H_r(x)\,; \enskip  m = 1\,; % \label{e5.12-ben}
\\
h_1(x) \equiv 0\,;  \enskip   C_2 = C_r > 0\,; %  \label{e5.13-ben}
\\
\beta = \begin{cases} 1\,, &  r \ge 1\,;\\ 
r\,, &  r \in (0,1)\,.
\end{cases}
%\label{e5.14-ben}
\end{gather*}
Далее, используя равенство

\noindent
\begin{multline*}
(1 + x)^\gamma = \sum\limits_{k=0}^\infty \fr{\gamma(\gamma -
1)\cdots(\gamma - k + 1 )}{k!}\,  x^k\,, \\
    |x| < 1\,,\enskip  \gamma
 \in \R\,,
\end{multline*}
нетрудно получить, что
\columnbreak


\noindent
\begin{multline}
\e N_n^{-1} = \fr{1}{(n - 1) \left(1 - r\right)}\left(\fr{1}{n^{r-1}} -
1\right) ={}\\
{}= {O}(n^{-r})\,, \ \ \   r > 0\,, \ \   r \ne 1\,, \ \  n \in \N\,.
\label{e5.15-ben}
\end{multline}
Для случая $r=1$, используя формулу~(\ref{e3.3-ben}), имеем:
\begin{equation}
\e N_n^{-1} = \fr{1}{n - 1} \log n\,, \enskip  n > 1\,. \label{e5.16-ben}
\end{equation}
Таким образом, учитывая теорему~3.1, следствие~3.2, формулы~(\ref{e5.2-ben})--(\ref{e5.4-ben}),
а также лемму~5.1, соотношения~(\ref{e5.15-ben}), (\ref{e5.16-ben}) и
равенства (справедливые равномерно по~$x$)
\begin{gather}
\int\limits_{(r(n-1)+1)^{-1}}^\infty \!\!\!\Phi(x\sqrt y)\, dH_r(y) =
 \int\limits_{0}^\infty \Phi(x\sqrt y)\, dH_r(y) +{}\notag\\
 \hspace*{15mm}{}+
{O}\left(\fr{1}{n}\right) =
 G_{2r}(x) + O\left(\fr{1}{n}\right)\,;\label{e5.17-ben}
\\
\int\limits_{(r(n-1)+1)^{-1}}^\infty \!\!\!\varphi(x\sqrt y)
\fr{1-x^2y}{\sqrt y}\, dH_r(y) ={}\notag\\
{}=
 \int\limits_{0}^\infty \varphi(x\sqrt y) \fr{1-x^2y}{\sqrt y}\,
dH_r(y) +  o(1) \equiv\notag\\
{}\hspace*{30mm}\equiv g_{r}(x) + {\it o}(1)\,,\label{e5.18-ben}
\end{gather}
получаем следующее утверждение.

\smallskip

\noindent
\textbf{Теорема 5.1.} \textit{Пусть статистика $T_n$ имеет вид $(\ref{e5.1-ben})$,
где  $X_1,X_2,\ldots$~--- независимые одинаково распределенные с.в.\ с
${\sf E}X_1 \hm= \mu$, $0\hm<{\sf D}X_1 \hm=\sigma^{-2}$, $\e
|X_1|^{3+2\delta} \hm< \infty$, $\delta\hm\in(0,1/2)$ и ${\sf E}(X_1 -
\mu)^3 \hm= \mu_3$, причем с.в.\ $X_1$ удовлетворяет условию
Крам$\acute{\mbox{е}}$ра $(C)$. Предположим, что при некотором $r\hm>0$
случайная величина $N_n$ имеет распределение вида:
\begin{multline*}
{\p}(N_n = k) = {}\\
{}=\fr{(k+r-2)\cdots r}{(k-1)!} \,\fr{1}{n^r} \left(1
- \fr{1}{n}\right)^{k-1}\,, \quad   k\in\N.
\end{multline*}
Тогда при $r > 1/(1+2\delta)$ для ф.р.\ нормированной статистики
$T_{N_n}$ при $n\to\infty$ справедливо а.р.\ вида
\begin{multline*}
\hspace*{-7.7pt}\sup\limits_x \left|{\p}\left(\sigma\sqrt {r(n-1)+1} (T_{N_n} - \mu) <
x\right) - G_{2r}(x) -{}\right.\\
\left.{}- \fr{\mu_3\sigma^3g_r(x)}{6\sqrt{r(n-1)+1}}
\right| ={}
\\
{}= \begin{cases} 
{O}\left(\left(\fr{\log
n}{n}\right)^{1/2+\delta}\right)\,, & \quad \hspace*{2pt} r = 1\,;\\[4pt]
{O}\left(\fr{1}{n^{\min(1, r(1/2+\delta))}}\right)\,,
&\quad \hspace*{2pt}  r > 1\,;\\[4pt]
{O}\left(\fr{1}{n^{r(1/2+\delta)}}\right)\,, &
\hspace*{-11mm}\fr{1}{1+2\delta} < r < 1\,,
\end{cases}
\end{multline*}
где функции $G_{2r}(x)$ и $g_r(x)$ определены в соотношениях
$(\ref{e5.17-ben})$ и $(\ref{e5.18-ben})$.}

\subsection{Распределение Лапласа}

Рассмотрим распределение Лапласа с ф.р.\ $\Lambda_\theta(x)$ и
плотностью
$$
\lambda_\theta(x)=\fr{1}{\theta\sqrt 2}\exp\left\{
-\fr{\sqrt{2}|x|}{\theta} \right\}\,, \ \ \   \theta > 0\,,\  x\in\R\,.
$$
В работе~\cite{9-ben} была построена последовательность с.в. $N_n(s)$,
зависящая от параметра $s \in \N$, сле\-ду\-юще\-го вида. Пусть $Y_1, Y_2,
\ldots$~--- независимые одинаково распределенные с.в., имеющие
непрерывную ф.р. Определим с.в.
$$
N(s) = \min\left\{ i\geq1: \max\limits_{1\leq j\leq s} Y_j < \max\limits_{s+1\leq k
\leq s+i} Y_k \right\}\,.
$$
Хорошо известно, что так определенные с.в.\ имеют распределение вида
\begin{equation}
{\p}(N(s) \geq k) = \fr{s}{s+k-1}\,, \enskip   k \geq \ 1\label{e5.19-ben}
\end{equation}
(см., например,~\cite{26-ben, 27-ben}). Пусть теперь  $N^{(1)}(s),
N^{(2)}(s),\ldots$~--- независимые одинаково распределенные с.в.,
имеющие распределение~(\ref{e5.19-ben}). Определим с.в.\
$$
N_n(s) = \max\limits_{1\leq j\leq n} N^{(j)}(s)\,,
$$
тогда, как показано в работе~\cite{9-ben},
\begin{equation}
\lim\limits_{n\to\infty} {\p}\left( \fr{N_n(s)}{n} < x \right) = e^{-s/x}\,,
\enskip  x>0\,,\label{e5.20-ben}
\end{equation}
и для асимптотически нормальной статистики $T_n$ справедливо соотношение:
\begin{multline*}
{\p}\left(\sigma\sqrt{n}(T_{N_n(s)} - \mu) < x\right) \longrightarrow{}\\
{}\longrightarrow
\Lambda_{1/s}(x)\,,  \quad n\to\infty\,,\enskip  x\in\R\,,
\end{multline*}
где $\Lambda_{1/s}(x)$~--- функция распределения Лапласа с параметром
$\theta\hm=1/s$.

В работе~\cite{11-ben} была получена следующая оценка скорости
сходимости в соотношении~(\ref{e5.20-ben}):
\begin{multline}
\sup\limits_{x\ge0} \left|\:{\p}\left(\fr{N_n(s)}{n} <  x\right) -
e^{-s/x} \right| \leq \fr{C_s}{n}\,, \\    C_s > 0\,, \quad
n\in\N\,.\label{e5.21-ben}
\end{multline}
Таким образом, из соотношения~(\ref{e5.21-ben}) следует, что случайный индекс
$N_n(s)$ удовлетворяет усло\-вию~1.2~с
\begin{equation}
g(n) = n\,; \ \  H(x) = e^{-s/x}\,; \ \  m = 1\,; \label{e5.22-ben}
\end{equation}
\begin{equation}
h_1(x) \equiv 0\,; \ \    C_2 = C_s > 0\,;  \ \  \beta = 1\,.\label{e5.23-ben}
\end{equation}
Рассмотрим более подробно величину $ \e N_n^{-1}(s)$. Из определения
с.в.\ $N_n(s)$ и равенства~(\ref{e5.19-ben}) имеем
\begin{multline*}
{\p}(N_n(s) = k) = \left( \fr{k}{s+k}\right)^n - \left( \fr{k -
1}{s+k-1}\right)^n ={}\\
{}=
 sn \int\limits_{k-1}^ k \fr{x^{n-1}}{(s + x)^{n+1}}\, dx\,,
\end{multline*}
поэтому
\begin{multline*}
\e N_n^{-1}(s) = \sum\limits_{k=1}^\infty \fr{1}{k} \,{\p}(N_n(s) = k) ={}\\
{}=
 sn \sum\limits_{k=1}^\infty \fr{1}{k} \int\limits_{k-1}^ k
\fr{x^{n-1}}{(s + x)^{n+1}}\, dx \leq{}
\\
{}\leq sn \sum\limits_{k=1}^\infty \int\limits_{k-1}^k \fr{x^{n-2}}{(s +
x)^{n+1}}\, dx =
 sn \int\limits_{0}^\infty \fr{x^{n-2}}{(s + x)^{n+1}} \,dx\,.\hspace*{-0.76227pt}
\end{multline*}
Для вычисления последнего интеграла используем формулу (см.~\cite{13-ben} формула~856.12, с.~184):
$$
\int\limits_{0}^\infty \fr{x^{s-1}}{(a + bx)^{s+n}} \,dx =
\fr{\Gamma(s)\Gamma(n)}{a^n b^s\Gamma(s+n)}\,,  \ \ \  a, b, s, n>0\,.
$$
Получим
\begin{equation*}
\e N_n^{-1}(s) \leq sn \fr{\Gamma(n-1)\Gamma(2)}{s^2\Gamma(n+1)} =
\fr{1}{s(n - 1)} = {O}(n^{-1})\,. \label{e5.24-ben}
\end{equation*}
Таким образом, учитывая теорему~3.1, следствие~3.2, формулы~(\ref{e5.2-ben})--(\ref{e5.4-ben}), 
а также лемму~5.1, соотношения~(\ref{e5.22-ben}), (\ref{e5.23-ben}) и
равенства (справедливые равномерно по~$x$)
\begin{gather}
\int\limits_{n^{-1}}^\infty \Phi(x\sqrt y)\, de^{-s/y} =
\int\limits_{0}^\infty \Phi(x\sqrt y) \,de^{-s/y} +
{O}\left(\fr{1}{n}\right) ={}\notag\\
\hspace*{30mm}{}= \Lambda_{1/s}(x) +
{O}\left(\fr{1}{n}\right)\,;\label{e5.25-ben}
\\
\hspace*{-20mm}\int\limits_{n^{-1}}^\infty \varphi(x\sqrt y) \fr{1-x^2y}{\sqrt
y}\, de^{-s/y} = {}\notag\\
{}=\int\limits_{0}^\infty \varphi(x\sqrt y)
\fr{1-x^2y}{\sqrt y}\, de^{-s/y} + {\it o}(1) \equiv{}\notag\\
\hspace*{40mm}{}\equiv l_{s}(x) +
{\it o}(1)\,,\label{e5.26-ben}
\end{gather}
непосредственно получаем следующую теорему.

\smallskip

\noindent
\textbf{Теорема 5.2.} \textit{Пусть статистика $T_n$ имеет вид $(\ref{e5.1-ben})$,
где $X_1,X_2,\ldots$~--- независимые одинаково распределенные с.в.\ с
${\sf E}X_1 \hm= \mu$, $0<{\sf D}X_1 \hm=\sigma^{-2}$, $\e
|X_1|^{3+2\delta} \hm< \infty$, $\delta\hm\in(0,1/2)$ и ${\sf E}(X_1 -
\mu)^3 \hm= \mu_3$, причем с.в.\ $X_1$ удовлетворяет условию
Крам$\acute{\mbox{е}}$ра $(C)$. Предположим, что при некотором $s\hm\in\N$ 
с.в.\ $N_n(s)$ имеет распределение вида:
$$
{\p}(N_n(s) = k) = \left( \fr{k}{s+k}\right)^n - \left( \fr{k -
1}{s+k-1}\right)^n\,, \ \ \   k\in\N\,.
$$
Тогда для ф.р.\ нормированной статистики $T_{N_n(s)}$ справедливо а.р.\ вида:
\begin{multline*}
\sup\limits_x \left| \vphantom{\fr{\mu_3\sigma^3l_s(x)}{6\sqrt{n}}}
{\p}\left(\sigma\sqrt {n} (T_{N_n(s)} - \mu) < x\right)
- \Lambda_{1/s}(x) - {}\right.\\
\left.{}-\fr{\mu_3\sigma^3l_s(x)}{6\sqrt{n}} \right| =
{O}\left(\fr{1}{n^{1/2+\delta}}\right)\,, \ \ n \to \infty\,,
\end{multline*}
где функции $\Lambda_{1/s}(x)$ и $l_s(x)$ определены соответственно
в соотношениях $(\ref{e5.25-ben})$ и~$(\ref{e5.26-ben})$.}

{\small\frenchspacing
{%\baselineskip=10.8pt
\addcontentsline{toc}{section}{Литература}
\begin{thebibliography}{99}

\bibitem{2-ben} %1
\Au{Гнеденко Б.\,В.} Об оценке неизвестных параметров
распределения при случайном числе независимых наблюдений~// Тр.
Тбилисского математического института, 1989. Т.~92. С.~146--150.

\bibitem{1-ben} %2
\Au{Гнеденко Б.\,В., Фахим Х.} Об одной теореме переноса~// Докл. АН
СССР, 1969. Т.~187. С.~15--17.

\bibitem{12-ben} %3
\Au{Von Ghossy R., Rappl G.} Some approximation methods for the
distri\-bution of random sums~// Insurance: Mathematics and
Economics, 1983. Vol.~2. P.~251--270.

\bibitem{6-ben}  %4
\Au{Круглов В.\,М., Королев~В.\,Ю.} Предельные теоремы для случайных
сумм.~--- М.: Изд-во Московского ун-та, 1990.

\bibitem{14-ben} %5
\Au{Королев В.\,Ю.} Предельные распределения для случайно
индексированных   последовательностей и их применения: Дисс. \ldots\
докт. физ.-мат. наук.~--- М.: МГУ, 1993.

\bibitem{7-ben} %6
\Au{Gnedenko B.\,V., Korolev V.\,Yu.} Random summation. Limit
theorems and applications.~--- Boca Raton: CRC Press, 1996.

\bibitem{8-ben}  %7
\Au{Bening V.\,E., Korolev V.\,Yu.} Generalized Poisson models and
their applications in insurance and finance.~--- Utrecht: VSP, 2002.

\bibitem{4-ben} %8
\Au{Гнеденко Б.\,В.} Курс теории вероятностей.~--- М.: Наука, 1988.

\bibitem{24-ben} %9
\Au{Королев В.\,Ю.} О~взаимосвязи обобщенного распределения
Стьюдента и дисперсионного гам\-ма-рас\-пре\-де\-ле\-ния при статистическом
анализе выборок случайного объема~// Докл. РАН, 2012. Т.~445.
Вып.~6. С.~622--627.

\bibitem{25-ben} %10
\Au{Климов Г.\,П.} Теория вероятностей и
математическая статистика.~--- М.: Изд-во Московского ун-та,
1983.

\bibitem{5-ben} 
\Au{Ширяев А.\,Н.} Вероятность.~--- М.: Наука, 1989.

\bibitem{15-ben} 
\Au{Billingsley P.} Probability and measure.~--- John Wiley \&
Sons, 1995.

\bibitem{16-ben} 
\Au{Bickel P.\,G.} Edgeworth expansions in nonparametric
statistics~// Ann. Stat., 1974. Vol.~2. P.~1--21.

\bibitem{17-ben} %14
\Au{Albers W.} Asymptotic  expansions and the deficiency
concept in statistics~// Mathematical Centre Tracts 58.~---
Amsterdam: Mathematisch Centrum, 1974.

\bibitem{19-ben} %15
\Au{Albers W., Bickel P.\,G., Van Zwet~W.\,R.} Asymptotic
expansions for the power of distribution free tests in the
one-sample problem~// Ann. Stat., 1976. Vol.~4. P.~108--156.

\bibitem{20-ben} %16
\Au{Bickel P.\,G., Van Zwet~W.\,R.} Asymptotic expansions for the
power of distribution free tests in the two-sample problem~// Ann.
Stat., 1978. Vol.~6. P.~947--1004.

\bibitem{18-ben} %17
\Au{Helmers R.} Edgeworth   expansions for linear combinations
of order statistics~// Mathematical Centre Tracts 105.~--- Amsterdam:
Mathematisch Centrum, 1984.

\bibitem{21-ben} %18
\Au{Bentkus V., Gotze F., Van Zwet~W.\,R.} An Edgeworth
expansions for symmetric statistics~// Ann. Stat., 1997. Vol.~25.
P.~851--896.

\bibitem{22-ben} 
\Au{Бенинг В.\,Е., Королев~В.\,Ю., Соколов~И.\,А., Шоргин~С.\,Я.}
Рандомизированные модели и методы теории надежности информационных и
технических систем.~--- М.: ТОРУС ПРЕСС, 2007.

\bibitem{3-ben} 
\Au{Бенинг В.\,Е., Королев В.\,Ю.} Об использовании распределения
Стьюдента в задачах теории вероятностей и математической статистики~// 
Теория вероятностей и ее применения, 2004. Т.~49. Вып.~3. С.~417--435.

\bibitem{9-ben} 
\Au{Бенинг В.\,Е., Королев В.\,Ю.} Некоторые статистические задачи,
связанные с распределением Лапласа~// Информатика и её применения,
2008. Т.~2. Вып.~2. С.~19--34.

\bibitem{26-ben} 
\Au{Wilks S.\,S.} Recurrence of extreme observations~// J.~Amer. Math. Soc., 
1959. Vol.~1. No.~1. P.~106--112.

\bibitem{27-ben} 
\Au{Невзоров В.\,Б.} Рекорды. Математическая теория.~--- М.: Фазис, 2000.

\bibitem{11-ben} 
\Au{Лямин О.\,О.} О~скорости сходимости распределений некоторых
статистик к распределению Лапласа и Стьюдента~// Вестник Московского
ун-та. Сер.~15: Вычислительная  математика и кибернетика,
2011. Вып.~1. С.~39--47.

\label{end\stat}


\bibitem{13-ben} 
\Au{Двайт Г.\,Б.} 
Таблицы интегралов и другие математические формулы.~--- М.: Наука, 1977.
\end{thebibliography}
}
}

\end{multicols}