\def\stat{razum}

\def\tit{СТАЦИОНАРНОЕ РАСПРЕДЕЛЕНИЕ ВРЕМЕНИ ОЖИДАНИЯ
В~СИСТЕМЕ ОБСЛУЖИВАНИЯ С~ОТРИЦАТЕЛЬНЫМИ ЗАЯВКАМИ,
БУНКЕРОМ ДЛЯ~ВЫТЕСНЕННЫХ ЗАЯВОК, РАЗЛИЧНЫМИ ИНТЕНСИВНОСТЯМИ
ОБСЛУЖИВАНИЯ\\ ПРИ ДИСЦИПЛИНЕ FIRST--FIFO--FIFO$^*$}

\def\titkol{Стационарное распределение времени ожидания в системе обслуживания с отрицательными
заявками} %, бункером для вытесненных заявок, различными интенсивностями обслуживания при дисциплине FIRST-FIFO-FIFO}

\def\autkol{Р.\,В.~Разумчик}

\def\aut{Р.\,В.~Разумчик$^1$}

\titel{\tit}{\aut}{\autkol}{\titkol}

{\renewcommand{\thefootnote}{\fnsymbol{footnote}}\footnotetext[1]
{Работа выполнена при финансовой поддержке Российского фонда
фундаментальных исследований (проекты \mbox{11-07-00112}, \mbox{13-07-00223}).}}

\renewcommand{\thefootnote}{\arabic{footnote}}
\footnotetext[1]{Институт проблем информатики
Российской академии наук, rrazumchik@ieee.org}


\Abst{Рассматривается система обслуживания с одним обслуживающим прибором
и пуассоновскими потоками обычных и отрицательных заявок. Обычная заявка,
поступающая в систему, занимает одно место в очереди в накопителе неограниченной
емкости. Отрицательная заявка при поступлении
выбивает заявку, находящуюся на первом месте в очереди в накопителе, перемещает ее в другой накопитель
неограниченной емкости (бункер) и сама покидает систему. Если при поступлении
 отрицательной заявки накопитель пуст, она покидает систему, не оказывая на нее никакого воздействия.
После окончания обслуживания очередной заявки на прибор поступает заявка, занимающая
первое место в очереди в накопителе или, если накопитель пуст, заявка с первого
места в очереди в бункере. Длительности обслуживания заявок
из накопителя и из бункера имеют экспоненциальные распределения с различными
параметрами.
Найдено стационарное распределение времени ожидания (обычной) заявкой начала
обслуживания в терминах преобразования Лап\-ла\-са--Стилть\-еса (ПЛС).}

\KW{система массового обслуживания; отрицательные заявки; бункер; различные интенсивности
обслуживания; время ожидания}

\vskip 14pt plus 9pt minus 6pt

      \thispagestyle{headings}

      \begin{multicols}{2}

            \label{st\stat}


\section{Введение и~описание системы}

Системы и сети с отрицательными заявками по-преж\-не\-му являются
актуальным предметом исследований, что обусловлено возможностью их
применения для моделирования различных аспектов телекоммуникационных
систем (таких как перебои в работе, потеря информации), различных
стадий процессов планирования, управления и контроля движения
материальных, информационных и финансовых ресурсов. Некоторое
представление о последних результатах можно найти, например, в~[1--7]. Настоящая работа посвящена нахождению стационарного
распределения времени ожидания начала обслуживания в системе массового
обслуживания (СМО) с одним обслуживающим прибором и специальным
видом отрицательных заявок, которые, в отличие от классического, не
<<убивают>> заявки, находящиеся в системе, а перемещают их в другую
очередь, откуда те обслуживаются с относительным приоритетом.

Рассматривается однолинейная СМО, в которую поступает пуассоновский поток заявок
интенсивности~$\lambda$. Далее заявки этого потока будем называть
положительными заявками. Для положительных заявок есть накопитель неограниченной емкости.
Дополнительно в сис\-те\-му поступает еще один пуассоновский поток отрицательных заявок
интенсивности $\lambda^-$. Отрицательная заявка, поступающая в
систему, вытесняет одну заявку из очереди в накопителе и
перемещает ее в накопитель для вытесненных заявок (далее~--- бун\-кер),
который также имеет неограниченную емкость. При этом сама отрицательная
заявка уходит из системы.
Если в момент поступления отрицательной заявки в очереди в накопителе нет
заявок, а на приборе обслуживается заявка, то
отрицательная заявка, не прерывая обслуживания на приборе,
покидает сис\-те\-му, не оказывая на нее никакого воздействия.
То же самое происходит и в случае, когда в момент поступления
отрицательной заявки накопитель и обслуживающий прибор пусты.

Выбор заявок на обслуживание производится следующим образом.
После окончания обслуживания очередной заявки на прибор
становится заявка из накопителя. Если же накопитель пуст, на прибор поступает заявка из бункера.
Обслуживание заявок не прерывается новыми как положительными, так
и отрицательными заявками.

Длительности обслуживания заявок из накопителя имеют экспоненциальное
распределение с параметром $\mu_1$, а из бункера~--- экспоненциальное
распределение с параметром $\mu_2$.

Будем считать, что поступающая
в систему отрицательная заявка <<убивает>> первую заявку в очереди в накопителе,
а в момент окончания обслуживания заявки на приборе на обслуживание выбирается
первая заявка из очереди в накопителе или, если накопитель пуст, первая
заявка из очереди в бункере. Такую дисциплину, по аналогии с~\cite{6}, будем
обозначать как FIRST--FIFO--FIFO.

Предполагается, что существует стационарный режим функционирования системы. Необходимое
и достаточное условие для этого приведено в~\cite{8}.

Основной результат работы состоит в нахождении в терминах
ПЛС стационарного распределения
времени ожидания начала обслуживания поступившей в сис\-те\-му (положительной)
заявки.

В следующем разделе приводятся некоторые вспомогательные результаты, которые
можно найти в~\cite{6, 8}. В~разд.~3 внимание уделяется
нахождению ПЛС времени ожидания начала обслуживания заявкой,
поступающей на прибор из накопителя, и заявкой, поступающей на прибор из бункера.


\section{Вспомогательные результаты}

В работе~\cite{8} получены следующие стационарные характеристики, связанные с
числом заявок в рассматриваемой системе:
\begin{itemize}
\item стационарная вероятность $p_0$ того, что система
находится в состоянии простоя;
\item
стационарные вероятности $\{p_{i,j,k},  \ i \hm\ge 0, \ \ j \hm\ge 0, \ \ k\hm=0,1 \}$ того, что в
накопителе находится $i$ заявок, в бункере ожидают $j$ заявок, вытесненных из накопителя,
и на приборе обслуживается
заявка либо из накопителя (при $k\hm=0$), либо из бункера (при $k\hm=1$),
в терминах двойных производящих функций
\begin{align*}
\label{(11)}
P(u,v) &= \sum\limits_{i=0}^{\infty}
\sum\limits_{j=0}^{\infty}
p_{i,j,0} u^i v^j\,;
\\
N(u,v)&= \sum\limits_{i=0}^{\infty}
\sum\limits_{j=0}^{\infty} p_{i,j,1} u^i v^j\,.
\end{align*}
\end{itemize}

Рассмотрим период занятости (ПЗ) системы $M/M/1/\infty$ с входящим потоком
интенсивности~$\lambda$ и интенсивностью обслуживания~$b$.
Обозначим через $G(x;b)$ функцию распределения (ФР) ПЗ этой системы,
а через $\gamma(s;b)\hm=\int\limits_0^{\infty} e^{-sx}\,dG(x;b)$~--- ПЛС ФР $G(x;b)$. Тогда
%%%%%%%%%%%%%%
\begin{equation*}
\gamma(s;b)= \fr{s+\lambda+b - \sqrt{(s+\lambda+b)^2 - 4 \lambda b}
}{2 \lambda}\,.
\end{equation*}
%%%%%%%%%%%%

Обозначим через $G_i(x)$ ФР ПЗ СМО $M/M/1/\infty$ с параметрами $\lambda$
и $\mu_1 + \lambda^-$, открываемого заявкой экспоненциальной длины с
параметром $\mu_i$ ($i \hm=1,2$). Тогда ПЛС ФР $G_i(x)$ имеет вид:
%%%%%%%%%%%%
\begin{equation*}
\gamma_i(s) = \fr{\mu_i }{\mu_i + s + \lambda\left[1 - \gamma(s;\mu_1+\lambda^-)\right]}\,.
\end{equation*}

Далее через $H_i(x;b)$ будем обозначать ФР~Эрланга с параметром $b$ и $i$
фазами обслуживания (а через $h_i(x;b)$ --- соответствующую плотность), т.\,е.
%%%%%%%%%%%%%
\begin{multline*}
H_i(x;b) = {}\\
{}=
\begin{cases}
u(x)\,,              &\! i=0, \\[3pt]
\displaystyle \int\limits_0^{x}
\fr{b^i t^{i-1}}{(i-1)!} \,e^{-b t} dt
= 1- \sum\limits_{k=0}^{i-1} \fr{b^k x^k }{k!}\, e^{-b t}\,, &\! i \ge 1.
\end{cases}\hspace*{-1.9182pt}
\end{multline*}
%%%%%%%%%%%%%
Напомним, что распределение Эрланга $H_i(x;b)$ имеет ПЛС
$(b/(s+b))^i$.

Наконец, обозначим через $D_k(t)$ вероятность того, что все заявки
из накопителя будут обслужены до момента времени~$t$
при условии, что в начальный момент на приборе начинает обслуживаться заявка из бункера,
а в накопителе имеется~$k$, $k \hm\ge 0$, других заявок.
В~работе~\cite{6} найдено выражение для ПЛС $\delta_k(s)\hm=\int\limits_{0}^{\infty} \,e^{-st}\,dD_k(t)$, $k\hm \ge 0$,
этой вероятности, которое можно представить в следующем виде:
\begin{multline*}
\delta_k(s) = \alpha (s) \left[\gamma(s+\mu_2;\lambda^-)\right]^k
+ {}\\
{}+\beta (s) \left[\gamma(s;\mu_1 + \lambda^-)\right]^k\,, \ k \ge 0\,,
\end{multline*}
где
\begin{multline*}
\alpha (s) = \fr{\mu^- }{\lambda + s+\mu_2 - \lambda \gamma(s+\mu_2;\lambda^-)}
\left [ \vphantom{\fr{s + \mu_2 + \lambda - \lambda \gamma(s; \mu_1+ \lambda^-) }
{s + \mu_1 + \lambda  - \lambda\gamma(s; \mu_1 +\lambda^-)}}
 1 -{}\right.\\
{}- { \mu}\Big / 
\left\{\left[ 
s+\mu_2 + \lambda^- + \lambda
- {}\right.\right.\\
\left.\left.{}-\lambda \gamma(s; \mu_1 + \lambda^-) \right] \gamma(s; \mu_1 + \lambda^-) - \lambda^- \right\}
\times{}
\\
\left.{}\times
\fr{s + \mu_2 + \lambda - \lambda \gamma(s; \mu_1+ \lambda^-) }
{s + \mu_1 + \lambda  - \lambda\gamma(s; \mu_1 +\lambda^-)}
\right ]\,,
\end{multline*}

\vspace*{-12pt}

\noindent
\begin{multline*}
\beta (s)=  \mu_1 \mu_2 /\left(\left\{
\left[ s+\mu_2 + \lambda^- + \lambda-{}\right.\right.\right.\\
\left.\left.\left.{}-
\lambda \gamma(s;\mu_1+\lambda^-)
\right] \gamma(s;\mu_1+\lambda^-) -
\lambda^- \right\}\times{}\right.
\\
\left.{}\times
 ( s+\mu_1  + \lambda - \lambda \gamma(s; \mu_1+\lambda^-))\right)\,.
\end{multline*}

После изложения всех необходимых вспомогательных результатов
перейдем к следующему раз\-делу.

\section{Распределение времени ожидания начала обслуживания}

Обозначим через $V_{\mathrm{nak}}(x)$ стационарную вероятность того, что
поступившая в систему положительная заявка будет ожидать начала обслуживания
в течение времени меньше~$x$ и до этого времени не попадет в бункер.
Для нахождения этой вероятности необходимо учитывать,
какая заявка (из накопителя или бункера) находится на приборе в момент
поступления выделенной заявки в систему.

Пусть $T_n(x)$, $n \hm\ge 1$,~--- вероятность того, что
поступившая в систему положительная заявка и заставшая в накопителе и на приборе
$n$ других заявок, а на приборе~--- заявку \textit{из накопителя},
поступит на прибор до момента времени~$x$.
В силу дисциплины обслуживания вероятность для выделенной заявки
поступить на прибор равна ${\mu_1/(\mu_1\hm+\lambda^-)}$.
Кроме того, также в силу дисциплины обслуживания, времена между соседними уходами заявок
из накопителя (на прибор или в бункер) независимы и имеют
экспоненциальное распределение с па\-ра\-мет\-ром $\mu_1\hm+\lambda^-$.
Положим $t_n(x)\hm=T'_n(x)$. Тогда имеет место
$$
t_n(x)= \fr{ \mu_1 }{\mu_1+\lambda^-}\, h_n\left(x;\mu_1+\lambda^-\right)\,.
$$

Пусть $R_n(x)$, $n \hm\ge 1$,~--- вероятность того, что поступившая в
систему положительная заявка и заставшая в накопителе и на приборе $n$ других
заявок, а на приборе~--- заявку \textit{из бункера}, поступит на
прибор до момента времени~$x$. Для нахождения этой вероятности
необходимо учитывать, сколько заявок было вытеснено из накопителя за
время~$x$ и успеет ли за время~$x$ обслужиться та заявка, которую
застала на приборе выделенная заявка при поступлении. Положим
$r_n(x)\hm=R'_n(x)$. В~случае $n\hm=1$ по\-сту\-па\-ющая заявка застает
накопитель пус\-тым, а на приборе обнаруживает заявку из бункера.
Поэтому $r_1(x)\hm= \mu_2 e^{-(\mu_2+\lambda^-) x}$. Рассмотрим случай
$n\ge2$. Выделенная заявка поступит на прибор, минуя бункер, до
момента времени~$x$, только если:
\begin{enumerate}[(1)]
\item за время $x$ будут <<убиты>> $n-1$
заявок, стоящих перед выделенной заявкой в накопителе,
а окончание обслуживания заявки на приборе произойдет
в промежутке $(x,x+dx)$;

\item за время $t, \ 0<t<x$, будут <<убиты>> $j\hm=\overline{0,n-2}$
заявок, стоящих перед выделенной заявкой в накопителе, в момент
$t$ произойдет окончание обслуживания заявки на приборе
и за оставшееся время будут обслужены или <<убиты>> $n-j$ заявок.
Заметим, что поскольку все из $n-j$ заявок обслуживаются
с одинаковой интенсивностью~$\mu_1$, то времена между соседними уходами заявок
из накопителя (на прибор или в бункер) независимы и имеют
экспоненциальное распределение с параметром $\mu_1\hm+\lambda^-$.
\end{enumerate}

Вообще говоря, возможны и другие случаи обслуживания и <<убийства>>
заявок (например, за время $x$ будут <<убиты>> больше чем $(n-1)$ заявок
в накопителе и за это время заявка на приборе не успеет обслужиться),
но во всех этих случаях вероятность поступления заявки на прибор,
минуя бункер, равна нулю. В~итоге при $n \hm\ge 2$ имеем:
\begin{multline*}
r_n(x)=\fr{\mu_2(\lambda^- x)^{n-1}}{(n-1)!}\, e^{-(\mu_2+\lambda^-) x}
+ \fr{ \mu_1 \mu_2}{\mu_1+\lambda^-}\times{}\\
{}\times
\sum\limits_{j=0}^{n-2} \int\limits_{0}^{x} \fr{(\lambda^- t)^{j}}{ j!}\, e^{-(\mu_2+\lambda^-) t}
h_{n-j-1}(x-t;\mu_1+\lambda^-) \,dt.
\end{multline*}

Учитывая, что, приходя в свободную систему, положительная заявка сразу же попадает
на прибор (время ожидания начала ее обслуживания равно нулю), и
воспользовавшись формулой полной вероятности, получаем следующее
выражение для плот\-ности $v_{\mathrm{nak}}(x)\hm=V'_{\mathrm{nak}}(x)$:
%%%%%%%%%%%%%%%%%
\begin{multline*}
v_{\mathrm{nak}}(x) = p_0 \delta(x)
+ \sum\limits_{i=0}^{\infty} \sum\limits_{j=0}^{\infty}
p_{i,j,0} t_{i+1}(x) +{}\\
{}+ \sum\limits_{j=0}^{\infty}
p_{0,j,1} r_1(x) + \sum\limits_{i=1}^{\infty}
\sum\limits_{j=0}^{\infty} p_{i,j,1} r_{i+1}(x)\,,
\end{multline*}
%%%%%%%%%%%%%
где $\delta(x)$~--- дель\-та-функ\-ция Дирака.

Производя традиционные преобразования,
можно показать, что в терминах ПЛС
$\varphi_{\mathrm{nak}}(s)\hm=
\int\limits_0^{\infty} e^{-sx} dV_{\mathrm{nak}}(x)$
вероятность $V_{\rm nak}(x)$ имеет вид:
%%%%%%%%%%%%%%%%%%
\begin{multline*}
\varphi_{\mathrm{nak}}(s)
= p_0 + \fr{ \mu_1}{\mu_1+\lambda^- + s}
\cdot {\sf P} \left (
\fr{ \mu_1 + \lambda^-}{ \mu_1+\lambda^- + s},1
\right ) +{}\\
{}+
\fr{\mu_2}{\mu_2+\lambda^- + s}
\cdot
N \left (
\fr{ \lambda^- }{\mu_2+\lambda^- + s},1\right )
+{}
\\
{}+
\fr{\mu_1 \mu_2 }{(\mu_1+\lambda^-) (\mu_2+\lambda^- + s) - \lambda^- (\mu_1+\lambda^- + s)}
\times{}
\\
{}\times
\left ( N\left ( \fr{ \mu_1 + \lambda^- }{ \mu_1+\lambda^- + s},
1 \right ) - N \left ( \fr{ \lambda^- }{ \mu_2+\lambda^- + s},
1 \right ) \right )\,.
\end{multline*}

Перейдем к вычислению стационарной вероятности $V_{\mathrm{bun}}(x)$ того,
что поступившая в систему положительная заявка будет ожидать начала обслуживания
в течение времени меньше~$x$ и до этого времени она попадет в бункер.

В силу того, что заявки из бункера обслуживаются в порядке поступления,
для нахождения вероятности $V_{\mathrm{bun}}(x)$ необходимо учитывать, какая заявка
(из накопителя или бункера) находится на приборе в момент поступления
выделенной заявки в систему, поскольку от этого зависит число
заявок, которые будут вытеснены в бункер до момента ухода туда выделенной заявки.

Пусть $D_n(t,i,j)$,  $n\ge 1$,  $i \hm\ge 0$,  $j\hm=\overline{0,n-1}$,~--- вероятность того, что поступающая в систему 
положительная заявка,
заставшая в накопителе и на приборе $n$ других заявок, а на приборе заявку \textit{из накопителя},
будет <<убита>> до момента~$t$, причем после ее ухода в бункер в накопителе останется $i$~заявок
и, кроме того, до ее ухода в бункер туда перейдут $j$
из $(n-1)$ заявок, находившихся в накопителе перед выделенной заявкой.
Положим $d_n(t,i,j) \hm= D'_n(t,i,j)$.
Тогда, обозначая через $C_{n}^m$ число сочетаний
из $n$ элементов по~$m$, запишем:
%%%%%%%%%%%%%%%
\begin{multline*}
d_n(t,i,j) = \left( \fr{\lambda^- }{\mu_1 + \lambda^-}\right)h_n(t,\mu_1+\lambda^-)
\fr{(\lambda t)^i }{i! }\times{}\\
{}\times e^{-\lambda t}
C_{n-1}^j \left(\fr{\lambda^-}{\mu_1 + \lambda^-}\right)^j
          \left(\fr{\mu_1 }{\mu_1 + \lambda^-}\right)^{n-1-j}\,.
\end{multline*}

Обозначим через
$\tilde d_n(t,z,j)\hm=\sum\limits_{i=0}^\infty z^i\, d_n(t,i,j)$
ПФ $d_n(t,i,j)$.
Тогда
\begin{multline*}
\!\!\!\!\!\tilde d_n(t,z,j) = \lambda^-
\fr{\left[(\mu_1+\lambda^-)t\right]^{n-1}}{ (n-1)!}\,
e^{-(\mu_1+\lambda^- + \lambda (1-z))t}\times{}\\
{}\times
C_{n-1}^j \left(\fr{\lambda^-}{\mu_1 + \lambda^-}\right)^j
          \left(\fr{\mu_1 }{\mu_1 + \lambda^-}\right)^{n-1-j}\,.
\end{multline*}
%%%%%%%%%%%%%%%%%


Теперь найдем $D_n(t,j)$, $n \hm\ge 1$, $j\hm=\overline{0,n-1}$,~--- вероятность того, что поступающая в систему 
положительная заявка,
заставшая в накопителе и на приборе $n$ других заявок, а на приборе заявку
\textit{из накопителя}, будет <<убита>>, до ее ухода в бункер туда перейдут $j$ из $(n-1)$ заявок,
находившихся в накопителе перед поступлением выделенной заявки, и
заявки из бункера начнут поступать на прибор до момента~$t$.
В~терминах ПЛС имеем:
\begin{multline*}
\hat d_n(s,j) = \int\limits_{0}^\infty
e^{-st}\, dD_n(t,j) ={}\\
{}= \gamma_1(s) \int\limits_{0}^{\infty}
\sum\limits_{i=0}^{\infty} e^{-st} \left[\gamma(s;\mu_1 + \lambda^-)\right]^i
d_n(t,i,j) \,dt
= {}\\
{}=
\fr{\lambda^- \gamma_1(s)}{ \mu_1 + \lambda^-}
C_{n-1}^j \left(\fr{\lambda^- }{ \mu_1 + \lambda^-}\right)^j
          \left(\fr{\mu_1 }{\mu_1 + \lambda^-}\right)^{n-1-j}\times{}\\
          {}\times
          \left(\fr{\mu_1+\lambda^-}{s + \mu_1 + \lambda^- + \lambda (1 - \gamma(s;\mu_1 + \lambda^-))}\right)^{n}\,.
\end{multline*}

Положим
\begin{multline*}
\psi_n(s,z)= \sum\limits_{j=0}^{n-1}
z^j \hat d_n(s,j) = \fr{\lambda^- \gamma_1(s)}{\mu_1 + \lambda^- z}\times{}\\
{}\times
          \left(\fr{\mu_1 + \lambda^- z }{
s + \mu_1 + \lambda^- + \lambda (1 - \gamma(s;\mu_1 + \lambda^-))}\right)^{n}\!\!.
\end{multline*}

Теперь перейдем к вычислению
вероятности $F_n(t,i,j)$, $n\hm\ge 1$, $i \hm\ge 0$, $j\hm=\overline{0,n-1}$, того, что поступающая в систему 
положительная заявка,
заставшая в накопителе и на приборе $n$ других заявок, а на приборе заявку \textit{из бункера},
будет <<убита>> до момента~$t$, причем после ее ухода в бункер в накопителе останется $i$~заявок
и, кроме того, до ее ухода в бункер туда перейдут $j$ из $n-1$ заявок, находившихся в накопителе перед поступлением выделенной
заявки.
Положим $f_n(t,i,j) \hm= F'_n(t,i,j)$.
Рассмотрим случай, когда $n\hm=1$. Тогда, так как $j$ может быть равно только~0, имеем:
$$ 
f_1(t,i,0) = \lambda^- \fr{(\lambda t)^i}{ i! }\, e^{-(\lambda+\mu_2 + \lambda^-) t}\,.
$$

Для случая $n \ge 2$ получаем, что при $j\hm=\overline{0,n-2}$
выражение для $f_n(t,i,j)$ имеет вид:
\begin{multline*}
f_n(t,i,j) = \mu_2 \sum\limits_{k=0}^{j} \int\limits_{0}^{t}
\fr{(\lambda^- y)^k (\lambda t)^i }{k! i!}\,
e^{-(\lambda+\mu_2+\lambda^-) y}\times{}\\
{}\times h_{n-1-k}(t-y,\mu_1+\lambda^-)\,dy
\times {}\\
{}\times C_{n-2-k}^{j-k} \left(\fr{\lambda^- }{\mu_1 + \lambda^-}\right)^{j-k+1}
          \left(\fr{\mu_1}{\mu_1 + \lambda^-}\right)^{n-2-j}\,,
\end{multline*}
а при $j=n-1$~---
$$
f_n(t,i,n-1) = \lambda^- \fr{(\lambda^- t)^{n-1} (\lambda t)^i }{(n-1)! i! }\, e^{-(\lambda+\mu_2+\lambda^-) t}\,.
$$


Обозначим через
$\tilde f_n(t,z,j)\hm=\sum\limits_{i=0}^\infty z^i f_n(t,i,j)$
ПФ $f_n(t,i,j)$.
Тогда в случае $n\hm=1$
$$
\tilde f_1(t,z,0) = \lambda^-  e^{-(\mu_2 + \lambda^- +\lambda (1-z)) t}\,,
$$
а в случае $n \ge 2$ при $j\hm=\overline{0,n-2}$ имеем
\begin{multline*}
\tilde f_n(t,z,j) = \mu_2 \sum\limits_{k=0}^{j}
\int\limits_{0}^{t} \fr{(\lambda^- y)^k}{k! }\, e^{-(\mu_2+\lambda^- + \lambda (1-z)) y}\times{}\\
{}\times
h_{n-1-k}(t-y,\mu_1+\lambda^-)\,dy
\times{}
\\
{}\times
C_{n-2-k}^{j-k} \left(\fr{\lambda^- }{\mu_1 + \lambda^-}\right)^{j-k+1}
          \left(\fr{\mu_1 }{\mu_1 + \lambda^-}\right)^{n-2-j}\,,
\end{multline*}
а при $j=n-1$~---
$$
\tilde f_n(t,z,n-1) = \lambda^- \fr{(\lambda^- t)^{n-1}}{(n-1)!}\, e^{-(\mu_2+\lambda^- + \lambda (1-z)) t}\,.
$$

Пусть теперь $F_n(t,j)$, $n\ge 1$, $j\hm=\overline{0,n-1}$,~--- вероятность того, что поступающая в систему 
положительная заявка,
заставшая в накопителе и на приборе $n$~других заявок, а на приборе заявку
\textit{из бункера}, будет <<убита>>, до ее ухода в бункер туда перейдут $j$ из $n-1$ заявок,
находившихся в накопителе перед поступлением выделенной заявки, и
заявки из бункера начнут поступать на прибор до момента~$t$.
В терминах ПЛС для случая $n\hm=1$ имеем:
\begin{multline*}
\hat f_1(s,0) = \int\limits_{0}^\infty
e^{-st}\, dF_1(t,0) = {}\\[2pt]
{}=\int\limits_{0}^{\infty}
\sum\limits_{i=0}^{\infty} e^{-st}
f_1(t,i,0) \delta_i(s)\ dt ={}
\\[2pt]
{}=
\fr{\lambda^- \alpha (s) }{s+\lambda+\mu_2 + \lambda^- - \lambda \gamma(s+\mu_2;\lambda^-)}
+{}\\[2pt]
{}+
\fr{\lambda^-\beta (s)}{s+\lambda+\mu_2 + \lambda^- - \lambda \gamma(s; \mu_1+ \lambda^-)}\,.
\end{multline*}

Для случая $n \ge 2$ при $j\hm=\overline{0,n-2}$ получаем:
\begin{multline*}
\hat f_n(s,j)=\int\limits_{0}^\infty e^{-st} dF_n(t,j)={}\\[2pt]
{}= \gamma_1(s) \int\limits_{0}^{\infty}
\sum\limits_{i=0}^{\infty} e^{-st}
\left[\gamma(s;\mu_1 + \lambda^-)\right]^i
f_n(t,i,j)\, dt = {}\\
{}=
\fr{\mu_2 \lambda^- \gamma_1(s) }{
(\mu_1+\lambda^-)(s + \mu_2 + \lambda^- + \lambda (1 - \gamma(s;\mu_1 + \lambda^-)))}\times{}\\[2pt]
{}\times
\left(\fr{\mu_1+\lambda^- }{s + \mu_1 + \lambda^- + \lambda (1 - \gamma(s;\mu_1 + \lambda^-))}\right)^{n-1}
\times{}
\\
{}\times
\sum\limits_{k=0}^{j} \left(
\left\{\lambda^- \left(s + \mu_1 + \lambda^- +{}\right.\right.\right.\\[2pt]
\left.\left.{}+ \lambda (1 - \gamma(s;\mu_1 + \lambda^-))\right) \right\}\Big / 
\left\{\left[s + \mu_2 + \lambda^- +{}\right.\right.\\[2pt] 
\left.\left.\left.{}+\lambda (1 - \gamma(s;\mu_1 + \lambda^-))\right]
(\mu_1+\lambda^-)\right\}\right)^{k}\times{}\\[2pt]
{}\times
C_{n-2-k}^{j-k} \left(\fr{\lambda^-}{\mu_1 + \lambda^-}\right)^{j-k}
          \left(\fr{\mu_1 }{\mu_1 + \lambda^-}\right)^{n-2-j}\,,
\end{multline*}
а при $j=n-1$ имеем:

\noindent
\begin{multline*}
\hat f_n(s,n-1)=\int\limits_{0}^\infty e^{-st} \,dF_n(t,n-1)={}
\\[2pt]
{}=
\int\limits_{0}^{\infty}\sum\limits_{i=0}^{\infty}
e^{-st} f_n(t,i,n-1) \delta_i(s)\, dt={}\\[2pt]
{}=
\alpha (s) \left (
\fr{\lambda^- }{s+\mu_2+\lambda^- + \lambda - \lambda \gamma(s+\mu_2;\lambda^-) }
\right )^{n}
+{}\\[2pt]
{}+\beta (s)\left (
\fr{\lambda^-}{s+\mu_2+\lambda^- + \lambda - \lambda \gamma(s; \mu_1 + \lambda^-)}
\right )^{n}\,.
\end{multline*}

Для удобства дальнейшей записи введем следующие обозначения:
\begin{gather*}
p=\fr{\lambda^-}{\mu_1 + \lambda^-}\,; \quad q = \fr{\mu_1}{\mu_1 + \lambda^-}\,;\\[2pt]
\theta = \fr{p \left[s + \mu_1 + \lambda^- + \lambda (1 - \gamma(s;\mu_1 + \lambda^-))\right] }{
\left[s + \mu_2 + \lambda^- + \lambda (1 - \gamma(s;\mu_1 + \lambda^-))\right]}\,.
\end{gather*}

Предположим, что рассматриваемая заявка попала в бункер
и перед ней в бункере оказалось $n$ других заявок.
Пусть $\tau$~--- момент времени, когда накопитель и прибор
окажутся пусты в первый раз.
Тогда нетрудно видеть, что время с момента~$\tau$, в течение которого рассматриваемая заявка будет
ожидать начала обслуживания по причине обслуживания впереди стоящих $n$ заявок и их потомков,
имеет ПЛС $(\delta_0(s))^n$.

Теперь можно выписать вид вероятности $V_{\mathrm{bun}}(x)$ в терминах ПЛС
$\varphi_{\mathrm{bun}}(s)\hm= \int\limits_0^{\infty} e^{-sx}\,dV_{\mathrm{bun}}(x)$:
\begin{multline*}
\varphi_{\mathrm{bun}}(s) = \sum\limits_{i=0}^\infty
\sum\limits_{j=0}^\infty \sum\limits_{k=0}^{i}
p_{i,j,0} \hat d_{i+1}(s,k) (\delta_0(s))^{j+k}
+ {}\\[2pt]
{}+\sum\limits_{i=0}^\infty \sum\limits_{j=0}^\infty
\sum\limits_{k=0}^{i} p_{i,j,1}
\hat f_{i+1}(s,k) (\delta_0(s))^{j+k}\,.
\end{multline*}
Данное выражение можно упростить,
подставляя найденные выше выражения для величин $\hat d_{i}(s,k)$,
$\hat f_{i}(s,k)$, $\delta_0(s)$ и производя
несложные, но утомительные преобразования.
Окончательное выражение для $\varphi_{\mathrm{bun}}(s)$ имеет вид:
\begin{multline*}
\varphi_{\mathrm{bun}}(s) = \fr{\lambda^- \gamma_1(s)}{s + \mu_1 + \lambda^- + \lambda (1 - \gamma(s;\mu_1 + \lambda^-))}
\times{}\\[2pt]
{}\times
P \left( \fr{\lambda^- \delta_0(s)  ) + \mu_1 }{
s + \mu_1 + \lambda^- + \lambda (1 - \gamma(s;\mu_1 + \lambda^-))}, \delta_0(s)
\right)
+{}
\end{multline*}

\noindent
\begin{multline*}
+
\left [
\fr{\lambda^- \alpha (s) }{s+\lambda+\mu_2 + \lambda^- - \lambda \gamma(s+\mu_2;\lambda^-)}
+{}
\right.\\[2pt]
\left.{}+\fr{\lambda^- \beta (s) }{s+\lambda+\mu_2 + \lambda^- - \lambda \gamma(s; \mu_1 + \lambda^-)}
\right]
\times{}
\\[2pt]
{}\times
N \left ( \fr{ \lambda^- \delta_0(s)  }{
s+\mu_2+\lambda^- + \lambda - \lambda \gamma(s; \mu_1+ \lambda^-)}, \delta_0(s)
\right )+{}
\\[2pt]
{}+
{ \mu_2 \lambda^- \gamma_1(s)}\Big /\left\{(\mu_1+\lambda^-)\left(s + \mu_2 + \lambda^- +{}\right.\right.\\
\left.\left.{}+ \lambda (1 - \gamma(s;\mu_1 + \lambda^-))(p \delta_0(s)  
+q- \theta \delta_0(s))\right)\right\}
\times{}
\\[2pt]
{}\times \left [
N \!\left ( \fr{ (\mu_1+\lambda^-)(p \delta_0(s)  +q)  }{
s + \mu_1 + \lambda^- + \lambda (1 - \gamma(s;\mu_1 + \lambda^-))}, \delta_0(s)
\!\right )
- {}\right.\\[2pt]
\!\!\left.{}-
N\! \left (
\fr{ \theta(\mu_1+\lambda^-)  \delta_0(s)   }{
s + \mu_1 + \lambda^- + \lambda (1 - \gamma(s;\mu_1 + \lambda^-))}, \delta_0(s)
\!\right )\! \right ]\,.\hspace*{-3.9619pt}
\end{multline*}

В итоге получаем, что стационарное распределение времени
ожидания произвольной положительной заявкой начала обслуживания имеет ПЛС
\begin{equation*}
\varphi(s) = \varphi_{\mathrm{nak}}(s) + \varphi_{\mathrm{bun}}(s)\,.
\end{equation*}

Заметим, что обратить ПЛС $\varphi(s)$ в явном виде едва ли
представляется возможным, но путем его дифференцирования можно получить
выражения для моментов любого порядка стационарного распределения времени ожидания начала
обслуживания заявки.

\section{Заключение}

В статье в терминах преобразования Лап\-ла\-са--Стилть\-еса
найдено распределение времени ожидания заявкой 
начала обслуживания в СМО
с отрицательными заявками и бункером для вытесненных заявок,
в которой заявки из накопителя и бункера обслуживаются с различными интенсивностями.
Предполагается, что отрицательная заявка при поступлении
выбивает заявку, находящуюся на первом месте в очереди в накопителе, а
после окончания обслуживания очередной заявки на прибор поступает заявка, занимающая
первое место в очереди в накопителе или, если накопитель пуст, заявка с первого
места в очереди в бункере.

Все полученные аналитические результаты были проверены путем сравнения с результатами
работы имитационной модели, разработанной с помощью программных средств GPSS (General Purpose Simulation System).


{\small\frenchspacing
{%\baselineskip=10.8pt
\addcontentsline{toc}{section}{Литература}
\begin{thebibliography}{9}

\bibitem{7}  %1
\Au{Jinting W., Yunbo H., Zhangmin D.} A discrete-time on-off source queueing system with negative customers~// 
Computers Ind. Eng., 2011. Vol.~61. No.\,4. P.~1226--1232.




\bibitem{3} %2
\Au{Klimenok V., Dudin A.} A~BMAP/PH/N queue with negative
customers and partial protection of service~//  Comm.
Statistics Simulation  Comput., 2012. Vol.~41. Iss.~7.
P.~1062--1082.

\bibitem{4} %3
\Au{Rakhee, Sharma G., Priya~K.} Analysis of G-queue with
unreliable server~// OPSEARCH, 2012. DOI~10.1007/s12597-012-0117-y.
P.~1--12.

\bibitem{5} %4
\Au{Pechinkin A., Razumchik~R.} A~method for calculating a
stationary queue distribution in a queuing system with flows of
ordinary and negative claims and a bunker for superseded claims~//
J.~Comm. Technol. Electronics, 2012. Vol.~57.
No.\,8. P.~882--891.

\bibitem{6}  %5
\Au{Pechinkin A., Razumchik R.} Stationary waiting time
distribution in queueing system with negative customers and bunker
for ousted customers under LAST--LIFO--LIFO service discipline~//
J.~Comm. Technol. Electronics, 2012. Vol.~57.
No.\,12. P.~1331--1339.

\bibitem{1} %6
\Au{Hannah Revathy P.,  Muthu Ganapathi Subramanian~A.} Two server
(s, S) inventory system with positive service time, positive lead
time, retrial customers and negative arrivals~// Int.
J.~Computer Appl., 2013. Vol.~62. No.\,10. P.~9--13.

\bibitem{2} %7
\Au{Bojarovich J., Marchenko L.} An open queueing network with
temporarily non-active customers and rounds modern probabilistic
methods for analysis of telecommunication networks~// Comm. Computer Information Sci., 2013. Vol.~356. P.~33--36.



\label{end\stat}

\bibitem{8}
\Au{Разумчик Р.\,В.} Система массового обслуживания с отрицательными
заявками, бункером для вытесненных заявок и различными
интенсивностями обслуживания~// Информатика и её применения, 2011.
Т.~5. Вып.~3. С.~39--43.
\end{thebibliography}
}
}

\end{multicols}