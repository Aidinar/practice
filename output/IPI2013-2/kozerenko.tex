\def\stat{kozerenko}

\def\tit{СТАТИСТИЧЕСКИЕ МЕХАНИЗМЫ ФОРМИРОВАНИЯ 
АССОЦИАТИВНЫХ ПОРТРЕТОВ ПРЕДМЕТНЫХ ОБЛАСТЕЙ 
НА~ОСНОВЕ ЕСТЕСТВЕННО-ЯЗЫКОВЫХ ТЕКСТОВ БОЛЬШИХ 
ОБЪЕМОВ ДЛЯ СИСТЕМ ИЗВЛЕЧЕНИЯ ЗНАНИЙ$^*$}

\def\titkol{Статистические механизмы формирования 
ассоциативных портретов предметных областей} 
%на~основе естественно-языковых текстов больших 
%объемов для систем извлечения знаний$^*$}

\def\autkol{М.\,М.~Шарнин, Н.\,В.~Сомин, И.\,П.~Кузнецов и др.} 
%Ю.\,И.~Морозова$^4$, И.\,В.~Галина$^5$, Е.\,Б.~Козеренко$^6$}

\def\aut{М.\,М.~Шарнин$^1$, Н.\,В.~Сомин$^2$, И.\,П.~Кузнецов$^3$, 
Ю.\,И.~Морозова$^4$, И.\,В.~Галина$^5$, Е.\,Б.~Козеренко$^6$}

\titel{\tit}{\aut}{\autkol}{\titkol}

{\renewcommand{\thefootnote}{\fnsymbol{footnote}}\footnotetext[1]
{Работа выполнена при частичной поддержке РФФИ (проект 13-07-00272-а).}}


\renewcommand{\thefootnote}{\arabic{footnote}}
\footnotetext[1]{Институт проблем информатики Российской академии наук, 1@keywen.com}
\footnotetext[2]{Институт проблем информатики Российской академии наук, somin@post.ru}
\footnotetext[3]{Институт проблем информатики Российской академии наук, igor-kuz@mtu-net.ru}
\footnotetext[4]{Институт проблем информатики Российской академии наук, judez@yandex.ru}
\footnotetext[5]{Институт проблем информатики Российской академии наук, irn\_gl@mail.ru}
\footnotetext[6]{Институт проблем информатики Российской академии наук, kozerenko@mail.ru}

 
  

      \Abst{Ассоциативные связи между терминами, понятиями и другими элементами 
естественного языка (ЕЯ) играют важную роль в решении широкого класса прикладных 
задач, среди которых интеллектуальная обработка текстов, извлечение знаний, их 
обработка с формированием баз знаний и организация различных видов поиска, в том 
числе семантических. Предложены методы автоматизированного выявления 
ассоциативных связей в текстах из Интернета и построения ассоциативных портретов 
различных предметных областей, ориентированных на решение перечисленных задач. 
Ассоциативный портрет предметной области (АППО) представляет собой словарь 
значимых терминов и словосочетаний, элементы которого связаны ассоциативными 
связями. Ассоциативный портрет предметной области создается автоматически на базе статистического 
анализа больших объемов текстов. Теоретическая значимость предлагаемого подхода 
заключается в использовании методов статистики, корпусной лингвистики и 
дистрибутивной семантики для обработки больших текстовых массивов на ЕЯ 
(постоянно пополняемых и обновляемых в Интернете) с целью построения модели 
предметной области в виде АППО.}
      
      \KW{автоматическая обработка корпусов текстов; статистические методы; 
      интел\-лек\-ту\-а\-ль\-ные ин\-тер\-нет-технологии; лек\-си\-ко-се\-ман\-ти\-че\-ский 
анализ; извлечение знаний из текстов; семантический поиск; семантические векторы; 
семантическое контекстное пространство}

\vskip 14pt plus 9pt minus 6pt

      \thispagestyle{headings}

      \begin{multicols}{2}

            \label{st\stat}
      
    \section{Постановка задачи}
     
     Для компьютерной лингвистики актуальной является 
фундаментальная научная проблема создания лингвистической и 
предметной базы для систем извлечения и обработки знаний. Создание 
вручную универсальных хорошо выверенных тезаурусов требует 
колоссальных трудовых и временн$\acute{\mbox{ы}}$х затрат, поэтому в последние годы 
ведется поиск новых путей создания лингвистической пред\-мет\-но-ори\-ен\-ти\-ро\-ван\-ной 
базы знаний. Одним из таких направлений является 
создание АППО, на 
основе которых можно с достаточной надежностью решать большинство 
важнейших проблем, связанных с информационным поиском, извлечением 
структур знаний и построением классификаций. 
     
     Ассоциативный портрет~--- это совокупность наиболее характерных 
предметных и лингвистических знаний, свойственных определенной 
предметной области. Под предметными знаниями понимаются присущие 
предметной области термины, понятия, связанные различного рода 
ассоциативными связями. Лингвистические знания~--- это варианты 
словарного выражения понятий. 

В~данной работе основное внимание 
обращено на статистические механизмы и обоснование методики 
автоматизированного выявления ассоциативных связей между значимыми 
словосочетаниями.
     
     Суть предлагаемого подхода состоит в обеспечении возможности 
создавать и использовать АППО автоматически, с минимальным 
вмешательством лингвиста. В~связи с этим трудоемкость разработки 
АППО будет на несколько порядков ниже разработки традиционных 
тезаурусов. Это позволит АППО стать реальным дополнением тезаурусов 
в новом поколении интеллектуальных ин\-тер\-нет-тех\-но\-ло\-гий, 
которые будут отличаться, как предполагается, большей полнотой охвата 
терминологии лишь при незначительном снижении точности. Тем самым 
преодолевается главная проблема~--- трудоемкости ручного труда 
лингвистов, не позволявшая проводить разработки с достаточной 
скоростью.
     
     Настоящая работа ставит своей целью описать методы построения 
АППО и их возможные применения для создания аналитических систем 
обработки текстов. 
     
     Решение проблемы связано с анализом больших массивов текстов 
русского языка на основе моделей векторных пространств и методов 
дистрибутивной семантики, ориентированных на автоматический анализ 
текстов (что отличает данный проект от широкомасштабных проектов типа 
WordNet и RusNet). 
     
     Отметим, что для русского языка подобные исследования проводятся 
впервые.
    
    \section{Существующие модели}
     
     В настоящее время известно значительное количество работ по 
автоматическому извлечению семантических связей из больших массивов 
текстов на ЕЯ~[1--15]. Наиболее успешные подходы используют метод 
дистрибутивной семантики и модели семантических векторных 
пространств (СВП). В~основе всех вариантов этого метода лежат 
количественные оценки, которые характеризуют совместную 
встречаемость языковых единиц текста в контекстах определенной 
величины. Основная гипотеза метода состоит в том, что слова, 
встречающиеся вместе в пределах некоторого текстового интервала, как-то 
связаны между собой. Для оценки связанности вводится коэффициент 
<<силы связи>>, который рассчитывается по некоторой формуле. Вне 
зависимости от вида формулы в ней обычно используются характеристики 
совместной встречаемости пар слов и одиночной встречаемости каждого 
из слов.
     
     Величина контекста, в рамках которого осуществляются подсчеты 
коэффициентов <<силы связи>>, как показывают результаты 
исследований, позволяет наиболее вероятно устанавливать:
\begin{itemize}
\item[(а)] при малых 
размерах контекста, ограниченного одним или двумя соседними 
словами,~--- контактные синтагматические связи словосочетаний;
\item[(б)] при  размере 
     5--10~слов~--- дистантные синтагматические связи и 
парадигматические отношения; 
\item[(в)] дальнейшее увеличение ширины 
контекста до 50--100~слов (размер предложения, сверхфразового 
единства, абзаца)~--- тематические связи между словами.
\end{itemize}

Тематические  связи могут оказаться доминирующими, если принять размер контекста 
величиной с сам текст~[1, с.~120]. 
     
     Модели векторных пространств находят все более широкое 
применение в исследованиях, связанных с семантическими моделями 
ЕЯ, и имеют разнообразный спектр потенциальных и 
действующих приложений~\cite{1-koz, 4-koz}. Данная область в настоящее 
время является одной из наиболее актуальных. Следует отметить модели 
Word-Space Model~\cite{16-koz} и Semantic Space Model~\cite{17-koz}. 
В~основе этих работ лежит пространство лексем и термов. Такое 
пространство базируется на распределении слов в корпусе текстов с целью 
представления их семантической связанности путем оценки 
пространственной близости. 
     
     Отметим, что концепция СВП впервые была реализована в информационно-поисковой системе 
SMART~[18]. SMART был пионером многих концепций, которые успешно 
используются современными поисковиками~[19]. Идея СВП состоит в 
представлении каждого документа из коллекции в виде точки в 
пространстве, т.\,е.\ вектора в векторном пространстве. Точки, 
расположенные ближе друг к другу в этом пространстве, считаются более 
близкими по смыслу. Пользовательский запрос рассматривается как 
псевдодокумент и тоже представляется как точка в этом же пространстве. 
Документы сортируются в порядке возрастания расстояния, т.\,е.\ в 
порядке уменьшения семантической близости от запроса, и в таком виде 
предоставляются пользователю. В~настоящее время большинство 
поисковиков используют СВП для измерения степени близости запроса и 
найденных документов~[19].
     
Baroni и Lenci~\cite{2-koz} предложили обобщенную модель, 
названную <<дистрибутивная память>>, которая является обобщением 
ранее известных моделей векторных пространств (vector spaces), 
семантических пространств (semantic spaces), пространств слов (word 
spaces), семантических моделей корпусной статистики (corpus-based 
semantic models) и дистрибутивных семантических моделй (distributional 
semantic models). 
     
     Успех СВП в решении ин\-фор\-ма\-ци\-он\-но-по\-иско\-вых задач 
направил исследователей на применение СВП для других семантических 
задач. Например, Rapp~\cite{3-koz} использовал контекстное 
векторное пространство для оценки семантической близости слов. Его 
система достигла результата 92,5\% на тес\-те по выбору наиболее 
подходящего синонима из стандартного тес\-та английского языка TOEFL, в 
то время как средний результат людей был 64,5\%.
     
     В настоящее время ведутся активные исследования по унификации 
модели СВП и выработке общего подхода к различным задачам 
выявления семантических связей из корпусов текстов~\cite{4-koz}.
     
     В предлагаемой работе используется более общий термин~--- 
семантическое контекстное пространство (СКП), где точки пространства 
соответствуют контекстным векторам не отдельных слов или термов, а 
значимых словосочетаний (см.\ разд.~3). Отметим, СКП является 
абстрактной моделью, не связанной с предметной областью. Приложение 
модели связано с выбором такой области и ее компонент: знаковых 
словосочетаний, ассоциативных связей и, соответственно, контекстов для 
их выделения. В~результате образуется упоминавшееся ранее АППО (см.\ 
разд.~1). 
     
     Более того, разработчики СВП отмечают, что основная проблема 
известных семантических пространств заключается в трудностях учета 
порядка слов, составляющих контексты. В~рамках данного проекта 
(СКП) эта проблема решается путем перехода от контекста слов к 
контексту значимых словосочетаний. Отметим также, что технология СВП 
развивалась для английского языка. Проект СКП предполагает работу как 
с русским, так и с английским языком. Возможно в дальнейшем включение 
других языков. Таким образом, модель СКП является расширением 
моделей СВП.
     
     Предлагаемые методы, разрабатываемые в рамках модели СКП, 
ориентированы на решение следующих задач:
     \begin{itemize}
     \item
 выявление синонимии и семантической близости слов и словосочетаний 
путем оценки их встречаемости в различных контекстах; 
\item автоматическая кластеризация слов по степени их близости в СКП;
\item автоматическая классификация слов путем использования 
лек\-си\-ко-се\-ман\-ти\-че\-ских форм;
\item автоматическая генерация тезаурусов методами статистической 
обработки терминов;
\item разрешение неоднозначности слов путем использования контекста;
\item расширение запросов за счет ассоциативных связей;
\item извлечение знаний из текстов с использованием статистических 
методов и лингвистических моделей;
\item оценка степени сходства лексических конструкций на основе их 
лек\-си\-ко-се\-ман\-ти\-че\-ско\-го анализа;
\item поиск отношений с помощью лексико-син\-так\-си\-че\-ских форм;
\item выявление близких по смыслу отношений и их классификация 
методами статистического анализа контекстных зависимостей.
    \end{itemize}
    
    \section{Значимые словосочетания и~определение 
семантического контекстного пространства}
     
     Для автоматического построения СКП предлагаются методики 
автоматического выявления значи\-мых словосочетаний (ЗС), которые в 
общем случае рассматриваются как лексические последовательности, 
имеющие самостоятельную значимость, и которые определяются по 
абсолютной час\-то\-те их встречаемости в текстах предметной 
     об\-ласти~\cite{5-koz, 6-koz}. Предполагается также выявление 
ассоциативных связей между ЗС в определенной предметной об\-ласти. При 
этом для выделения ЗС используются методы статистического 
ранжирования, а для расчета силы ассоциативной связи ЗС используется 
косинусная мера между контекстными векторами (компонентами вектора 
ЗС являются частоты совместной встречаемости данного ЗС с другими ЗС 
в одном и том же контексте). Такие векторы образуют СКП. Следует 
отметить, что посредством ЗС могут быть представлены как отдельные 
слова, словосочетания и термины, так и более сложные конструкции~--- 
объекты и именованные сущности.
     
     Для вычисления косинусной меры между контекстными векторами 
используется следующая формула:
     $$
     \fr{xy}{\vert x\vert \vert y \vert} =\fr{\sum\nolimits_{i=1}^n x_i y_i} 
{\sqrt{\sum\nolimits_{i=1}^n x 
_i^2} \,\sqrt{\sum\nolimits_{i=1}^n y_i^2}}\,.
$$
    
     В методе СКП существенное значение имеет выбор контекста. 
Вообще под контекстом в данном случае понимается некий алгоритм 
идентификации текстовых фрагментов, связанных с данным ЗС. 
Всевозможные контексты и являются координатами векторов СКП. 
В~зависимости от того, какие контексты считаются идентичными, 
различают типы контекстов. Классификация и даже перечисление типов 
контекстов~--- проблема, которая в силу своей новизны требует особого 
рассмотрения~\cite{15-koz}, поэтому ограничимся лишь примерами. Так, 
\mbox{статья} (абзац) может считаться контекстом данного ЗС (если данное ЗС не 
входит в эту статью, то можно говорить о нулевом контексте). В~этом 
случае координатами вектора СКП являются все статьи, относящиеся к 
определенной предметной области. Другой пример: все предложения, 
содержащие ЗС, считаются контекстом данного ЗС. 
     
     Именно на использовании такого контекста построен пример 
использования методики по\-стро\-ения СКП на основе следующего 
текстового фрагмента:
     
\textit{Искусственный интеллект~--- наука и технология создания 
интеллектуальных машин, особенно интеллектуальных компьютерных 
программ.}
 
\textit{Компьютерная лингвистика~--- направление искусственного 
интеллекта, которое ставит своей целью использование математических 
моделей для описания естественных языков. }

\textit{Дискретная математика~--- область математики, занимающаяся 
изучением дискретных структур, которые возникают как в пределах 
самой математики, так и в ее приложениях.}

\textit{Конструктивная математика~--- близкое к интуиционизму течение 
в математике, изучающее конструктивные построения.}

     Построим контекстные векторы для ЗС <<\textit{искусственный 
интеллект}>>, <<\textit{компьютерная лингвистика}>>, 
<<\textit{дискретная математика}>>, <<\textit{конструктивная 
математика}>> и слов, встречающихся в текстовом фрагменте более 
одного раза. 
     
     Применив формулу вычисления косинусной меры между 
контекстными векторами, получим следующие коэффициенты 
семантической бли\-зости между рассматриваемыми ЗС:

<<\textit{дискретная математика}>> и <<\textit{конструктивная 
математика}>>~--- 0,95;

<<\textit{искусственный интеллект}>> и <<\textit{компьютерная 
лингвистика}>>~--- 0,7;

<<\textit{компьютерная лингвистика}>> и <<\textit{дискретная 
математика}>>~--- 0,52;

<<\textit{компьютерная лингвистика}>> и <<\textit{конструктивная 
математика}>>~--- 0,4;

<<\textit{искусственный интеллект}>> и <<\textit{дискретная 
математика}>>~--- 0,36;

<<\textit{искусственный интеллект}>> и <<\textit{конструктивная 
математика}>>~--- 0,29.
     
     Отметим, что в реальных приложениях такого рода матрицы будут 
очень большими~--- порядка миллионов столбцов (строк). Однако 
благодаря тому, что подавляющее большинство элементов будет 
заполнено нулями, реально хранимая информация будет вполне 
обозримой. 
     
     Для нахождения ассоциативных связей, которые могут войти в 
ассоциативный портрет, необходимо выбрать из этой матрицы пары 
терминов с самыми большими коэффициентами семантической близости 
(косинусной меры). В~данном случае это будут пары 
(<<\textit{дискретная математика}>>, <<\textit{конструктивная 
математика}>>) и (<<\textit{искусственный интеллект}>>, 
<<\textit{компьютерная лингвистика}>>).

   \section{Ассоциативные портреты и~методы их получения}
    
     Ассоциативный портрет предметной области~--- это 
множество ассоциативных связей между значимыми терминами 
предметной об\-ласти. Формально АППО определяется как граф 
$G \hm= (V, E)$ с узлами $v$ из~$V$, представляющими значимые 
термины/словосочетания и дугами графа ($v_i$, $v_j$, Link, $w_{ij}$)\linebreak 
из~$E$, описывающие отношения/связи между словосочетаниями, где 
$w_{ij}$~--- это вес, выражающий силу связи, а 
     Link~--- тип связи, определяемый ти-\linebreak пом контекста. Тип контекста 
определяется параметрами алгоритма расчета контекстных векторов, 
такими как размер контекстного окна или тип лек\-си\-ко-син\-так\-си\-че\-ско\-го 
шаблона/конструкции, связывающей словосочетания. 
     
     Идеология АППО базируется на дистрибутивной гипотезе, 
утверждающей, что семантически близкие (или связанные) лексемы имеют 
похожий контекст и, наоборот, при похожем контексте лексемы 
семантически близки. В~предлагаемой модели используется расширенная 
гипотеза, пред\-по\-ла\-га\-ющая, что при сходстве контекстов близкими 
признаются не только отдельные лексемы, но и произвольные 
многолексемные фрагменты~--- значимые словосочетания.
     
     Получение АППО предполагает реализацию совокупности методов, 
включающих:
     \begin{itemize}
     \item методы выявления в Интернете текстов определенных 
предметных областей;
     \item методы выявления в текстах ЗС и их 
ранжирования;
     \item методы выявления и ранжирования ассоциативных связей 
между ЗС.\end{itemize}
     
     Методы основаны на прохождении предварительного обучения на 
текстах, в том числе взятых из различных интернет-ресурсов.
     
     Обработка больших массивов текстов, постоянно пополняемых в сети 
Интернет, позволяет собирать необходимые статистические данные для\linebreak 
формирования достаточно полной картины о предметной области, 
представленной в виде СКП. Возмож\-ность проводить машинное обучение 
на большом числе примеров придает системе определенную гибкость и 
улучшает результаты. 

\begin{figure*}[b] %fig1
\vspace*{9pt}
 \begin{center}
 \mbox{%
 \epsfxsize=155mm
 \epsfbox{koz-1.eps}
 }
 \end{center}
 \vspace*{-6pt}
\Caption{Фрагмент статьи Artificial Intelligence из Keywen}
%\vspace*{6pt}
\end{figure*}
     
     Отображение многомерных векторов на плоскость является удобным 
средством визуализации связей. В~результате образуются визуальные 
карты ЗС. На таких картах расстояние между ЗС 
тем меньше, чем больше сила их ассоциативной связи, что позволяет 
выделять сильно связанные и близкие по смыслу ЗС.
     
     Подход предусматривает исследование различных типов и 
источников контекста, а также различных методов выделения контекста и 
оценки силы ассоциативной связи по контекстным векторам. Важным 
представляется исследование таких типов контекста, как простая 
совместная встречаемость ЗС в тексте, а также совместная встречаемость 
ЗС в текстах в рамках заданных лек\-си\-ко-син\-так\-си\-че\-ских 
шаблонов.
     
     Предлагаемые в данном проекте методы и модели позволяют 
учитывать порядок слов в лексических последовательностях, что 
необходимо для более качественного решения вышеперечисленных задач, 
в том числе для выявления ассоциативных связей словосочетаний, 
объектов и именованных сущностей. Предлагается также разработать 
средства визуализации, которые позволят точно отоб\-ра\-зить координаты 
выявленных объектов в двумерном семантическом контекстном 
пространстве и использовать эвристические методы для уста\-нов\-ле\-ния 
степени близости объектов.
     
     Дистрибутивно-статический метод позволяет на\linebreak основе частотной 
информации о ЕЯ-еди\-ни\-цах полу\-чать по некоторой заданной формуле 
количественную характеристику их связанности. Философия данного 
метода состоит в том, <<что семантическую классификацию значимых 
элементов языка можно с б$\acute{\mbox{о}}$льшим основанием индуктивно извлечь из 
анализа текста, чем получить ее с некоторой точки зрения, внешней по 
отношению к структуре языка. Следует ожидать, что такая классификация 
даст более надежные ответы на проблемы синонимии и выражения смысла, 
чем существующие тезаурусы и списки синонимов, основанные, главным 
образом, на интуитивно ощущаемых сходствах без адекватной 
эмпирической проверки>> [1, с.~115, 116].
   
   \section{Энциклопедия ключевых понятий и~другие разработки 
авторов}
    
     Элементы предлагаемого подхода уже были час\-тич\-но апробированы 
в работах по созданию эн\-цик\-ло\-пе\-дии ключевых понятий KEYWEN 
(содержит 260\,000~статей и 5\,000\,000~клю\-че\-вых фраз, число которых 
постоянно растет) и в ряде работ других участников проекта.
     
     Система Keywen представляет собой средство построения больших 
энциклопедий по материалам Интернета и на их основе составления 
рефератов и аналитических статей. Имеется опыт построения корпуса 
английских текстов из Интернета размером более 1~ТБ, проведены 
эксперименты по по\-стро\-ению русских корпусов текстов для ряда 
предметных областей.
     
     На рис.~1 приведен фрагмент \mbox{статьи} Artificial Intelligence из Keywen 
({\sf http://keywen.com/en/\linebreak ARTIFICIAL\_INTELLIGENCE}) с автоматически 
построенным набором ключевых слов, выбранных из ассоциативных 
связей термина Artificial Intelligence. В~состав ключевых слов входят: MIT, 
MACHINE, COMPUTER SCIENCE, COGNITIVE SCIENCE, EXPERT 
SYSTEMS, TURING, MINSKY, REASONING.

  

    
     На рис.~2 представлен абзац Categories из той же статьи 
Artificial Intelligence, демонстрирующий автоматический выбор 
иерархических связей и до-\linebreak\vspace*{-12pt}

\pagebreak

\end{multicols}

\begin{figure} %fig2
\vspace*{1pt}
 \begin{center}
 \mbox{%
 \epsfxsize=155mm
 \epsfbox{koz-2.eps}
 }
 \end{center}
 \vspace*{-6pt}
\Caption{Категории и подкатегории статьи <<Искусственный интеллект>>}
%\end{figure}
%\begin{figure*} %fig3
\vspace*{15pt}
 \begin{center}
 \mbox{%
 \epsfxsize=155mm
 \epsfbox{koz-3.eps}
 }
 \end{center}
 \vspace*{-6pt}
\Caption{Фрагмент общего дерева категорий}
\end{figure}

\begin{multicols}{2}

\noindent
минирующей категории для статьи из набора 
ассоциативных связей, т.\,е.\ из набора ключевых слов статьи. 
    

     
     На рис.~2 также показаны доминирующие категории (Computer 
Science, Robotics, Science, Intelligence, Cognitive Science), которые система 
автома\-тически выбрала для статьи Artificial Intelligence. Для выбора 
доминирующей категории рассчитывались ассоциативные связи, 
найденные с использованием английских аналогов для следующих 
     лек\-си\-ко-син\-так\-си\-че\-ских шаблонов: <<\textit{относиться 
к}>>, <<\textit{включать в себя}>>, <<\textit{классифицировать}>>, 
<<\textit{различать}>>, <<\textit{подразделять}>>, 
<<\textit{разделяться на}>>, <<\textit{входить}>>, 
<<\textit{составлять}>>. Возможные шаблоны для русского и немецкого 
языка описаны в статье~\cite{15-koz}.
     
      Из доминирующих категорий строится дерево всех категорий ({\sf 
http://keywen.com/Category\_\linebreak Structure}), фрагмент которого представлен 
на рис.~3.
    

    
     На основе проведенных исследований оформлены следующие 
патентные заявки в США:
     \begin{enumerate}[1.]
\item System and method for generating hierarchical categories from 
collection of related terms. U.S.\ Provisional Patent Application 
No.\,61/096,255, filed 2008.12.22 (Charnine).
\item USPTO Applicaton \#20100161671, filed 2010.06.24 (Charnine).
\end{enumerate}

     В основу заявок положено изобретение, описанное в 
монографии~\cite{6-koz}, которое было использовано для построения 
одной из крупнейших иерархий категорий для электронной энциклопедии 
Keywen. В~монографии также опубликована информация о верхних 
уровнях иерархии категорий. Система Keywen не имеет аналогов как в 
России, так и за рубежом.
     
     Статистические механизмы формирования \mbox{АППО} в настоящее время встраиваются в технологии 
аналитической обработки текстов и извлечения информационных объектов 
(именованных сущностей) и их связей, в области которых авторский 
коллектив уже много лет ведет исследования и разработки: создан ряд 
прикладных систем, основанных на глубинном анализе текстов~[10--14]:
  \begin{description}
  \item[\,]   
     <<Криминал>> и <<Аналитик>>~--- системы, извлекающие 
информацию из милицейских сводок для ло\-ги\-ко-ана\-ли\-ти\-че\-ских решений, 
которые являются основой для проведения следственных действий;
      \item[\,] 
     <<Semantix>>~--- многоязычный лингвистический процессор, 
обеспечивающий выделение объектов (сущностей) и их связей из текстов на 
английском и русском языке;
       \item[\,]
     <<Антитеррор>>~--- система, извлекающая из текстов информацию о 
террористах и террористических акциях с автоматическим 
формированием базы знаний;
       \item[\,]
     <<Резюме>>~--- система семантической обработки резюме и их 
отображения в формат сайта крупной рекрутинговой компании;
       \item[\,]
     <<Памятники>>~--- обеспечивает извлечение информации из текстов 
описания памятников культуры и формирования ролевых функций 
относящихся к ним лиц;
       \item[\,]
     <<ДИЕС>>~--- диалоговая среда, обеспечивающая построение 
конкретных систем языкового взаимодействия с пользователем.
\end{description}
    
   \section{Направления дальнейших исследований}
   
     Авторской группой в первую очередь предполагается использование 
моделей СКП в двух областях приложений: поиске информации и 
построении классификаций. С~помощью СКП можно существенно 
повысить гибкость поиска, построив реальные механизмы 
переформулирования запроса, позволяющие заменять части запроса 
иными~--- близкими к ним по контексту. Автоматическое построение 
классификаций может быть достигнуто путем выявления ассоциативных 
связей между ЗС. 
     
     Авторами разработана программная среда ДЕКЛАР, включающая в 
себя систему представления знаний и язык программирования задач, 
связанных с обработкой знаний. В~рамках этой среды разработаны 
системы лек\-си\-ко-мор\-фо\-ло\-ги\-че\-ско\-го и 
     син\-так\-ти\-ко-се\-ман\-ти\-че\-ско\-го анализа русских и английских 
текстов, которые используются при извлечении знаний. Программная 
среда ДЕКЛАР будет также применяться для разработки новых 
прикладных систем извлечения и обработки знаний, интегрирующих 
АППО.

{\small\frenchspacing
{%\baselineskip=10.8pt
\addcontentsline{toc}{section}{Литература}
\begin{thebibliography}{99}
\bibitem{14-koz} %1
\Au{Москович В.\,А.} Информационные языки.~--- М.: Наука, 1971. 144~с.

\bibitem{3-koz} %2
\Au{Rapp R.} Word sense discovery based on sense descriptor dissimilarity~// 9th MT Summit 
Proceedings.~--- New Orleans, LA, 2003. P.~315--322.

\bibitem{7-koz} %3
\Au{Шарнин М.\,М., Кузнецов И.\,П.} Автоматическое формирование электронных 
энциклопедий и справочных пособий по информации из сети Интернет~// Системы и 
средства информатики.~--- М.: ИПИ РАН, 2004. Вып.~14. С.~210--223.

\bibitem{5-koz} %4
\Au{Charnine M.\,M., Kuznetsov I.\,P., Kozerenko~E.\,B.}
Semantic navigator for Internet search~// MLMTA'05:   Conference (International) on Machine 
Learning Proceedings.~--- Las Vegas: CSREA Press, 2005. P.~60--68.

\bibitem{8-koz} %5
\Au{Кузнецов И.\,П., Сомин Н.\,В.} Анг\-ло-рус\-ская система извлечения знаний из 
потоков информации в среде Интернет~// Системы и средства информатики.~--- М.: ИПИ 
РАН, 2007. Вып.~17. С.~236--254.
    
\bibitem{1-koz} %6
\Au{Lenci A.} Distributional semantics in linguistic and cognitive research~// Rivista di 
Linguistica, 2008. Vol.~1. Р.~1--30.


\bibitem{4-koz} %7
\Au{Turney P.} A~uniform approach to analogies, synonyms, antonyms and associations // 
Proceedings of COLING.~--- Manchester, 2008. P.~905--912.

\bibitem{6-koz} %8
\Au{Charnine M., Charnine V}. Keywen category structure.~--- Wordclay, USA, 2008. 60~p.

\bibitem{2-koz} %9
\Au{Baroni M., Lenci A.} Distributional memory: A~general framework for corpus-based 
semantics~// Comput. Linguistics, 2010. Vol.~36. Iss.~4. P.~673--721. 


\bibitem{9-koz} %10
\Au{Кузнецов И.\,П., Сомин Н.\,В.} Выявление имплицитной информации из текстов на 
естественном языке: проб\-ле\-мы и методы~// Информатика и её применения, 2012. Т.~6. 
Вып.~1. С.~48--57.
\bibitem{10-koz} %11
\Au{Kuznetsov I.\,P., Kozerenko E.\,B., Charnin~M.\,M.}
Technological peculiarity of knowledge extraction for logical-analytical systems~// 
WORLDCOMP'12:  ICAI'12 Proceedings.~---  Las Vegas: CSREA Press, USA, 2012. Vol.~II. 
P.~762--768.
\bibitem{11-koz} %12
\Au{Шарнин М.\,М., Кузнецов И.\,П.} Особенности семантического поиска 
информационных объектов на основе технологии баз знаний~// Информатика и её 
применения, 2012. Т.~6. Вып.~2. С.~47--56.
\bibitem{12-koz} %13
\Au{Kuznetsov I.\,P., Charnine M.\,M., Kozerenko~E.\,B., \textit{et al}.} Intelligent tools for the semantic Internet 
navigator design~// Электронные библиотеки: перспективные методы и технологии, 
электронные коллекции: Тр. XIV Всеросс. научн. конф. RCDL'2012.~--- 
Переславль-Залесский: Университет города Переславля, 2012. С.~274--283.
\bibitem{13-koz} %14
\Au{Кузнецов И.\,П., Шарнин М.\,М., Мацкевич~А.\,Г.}
Технология извлечения структур знаний с использованием аппарата расширенных 
семантических сетей~// Искусственный интеллект: Журнал НАН Украины, 2012. Т.~4. 
С.~190--203.

\bibitem{15-koz}
\Au{Schumann A.} Towards the automated enrichment of multilingual terminology databases 
with knowledge-rich contexts~// Conference (International)  Dialogue 2012 Proceedings. Vol.~1. 
P.~559--567. 
\bibitem{16-koz}
\Au{Sahlgren M.} The Word-Space Model: Using distributional analysis to represent syntagmatic 
and paradigmatic relations between words in high-dimensional vector spaces. Ph.D. 
Thesis.~--- Department of Linguistics, Stockholm University, 2006.
\bibitem{17-koz}
\Au{Sahlgren M.} Towards pertinent evaluation methodologies for word-space models~// LREC 
2006:  5th Conference (International) on Language Resources and Evaluation Proceedings.~--- 
Genoa, Italy, 2006.
\bibitem{18-koz}
The SMART retrieval system: Experiments in automatic document processing~/ Ed.
 G.\,M.~Salton.~--- Prentice-Hall, 1971.
 
 \label{end\stat}
 
\bibitem{19-koz}
\Au{ Manning C., Raghavan P., Sch$\ddot{\mbox{u}}$tze~H.}
 Introduction to information retrieval.~--- Cambridge: Cambridge University Press, 2008.
\end{thebibliography}

}
}


\end{multicols}