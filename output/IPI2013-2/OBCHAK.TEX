\def\stat{abstr}
{%\hrule\par
%\vskip 7pt % 7pt
\raggedleft\Large \bf%\baselineskip=3.2ex
A\,B\,S\,T\,R\,A\,C\,T\,S \vskip 17pt
    \hrule
    \par
\vskip 21pt plus 6pt minus 3pt }

\label{st\stat}

%\def\rightmark{\ }

%1
\def\tit{PARAMETRICAL STATISTICAL AND ANALYTICAL MODELING OF~DISTRIBUTIONS 
IN~NONLINEAR STOCHASTIC SYSTEMS ON MANIFOLDS}

\def\aut{I.\,N.~Sinitsyn}

\def\auf{IPI RAN, sinitsin@dol.ru\\}

\def\leftkol{\ } % ENGLISH ABSTRACTS}
\def\rightkol{\ } %ENGLISH ABSTRACTS}

\titele{\tit}{\aut}{\auf}{\leftkol}{\rightkol}

\vspace*{-2pt}

\noindent
Discrete parametrical statistical and analytical modeling methods in nonlinear Ito 
stochastic systems on manifolds with Wiener and Poisson noises have been developed. For 
one- and multidimensional densities parametrization, the coefficients in orthogonal 
expansions of different orders are taken. Special attention is paid to nonlinear 
correlational theory of statistical and analytical modeling.


\vspace*{-5pt}

\KWN{analytical modeling; method of normal approximation; nonlinear correlational theory;
nonlinear Ito stochastic system on manifold; orthogonal expansions method;
parametrization of one- and multidimensional densities;
statistical linearization method; statistical modeling}

%\thispagestyle{myheadings}



\vskip 14pt plus 6pt minus 3pt

%\pagebreak


%2
\def\tit{EVALUATION METHODS FOR EFFICIENCY AND~DIRECTIVE TERMS 
OF~PERFORMANCE OF~RESOURCE-INTENSIVE COMPUTING TASKS}

\def\aut{I.\,K.~Kupalov-Yaropolk$^1$, Yu.\,E.~Malashenko$^2$, 
I.\,A.~Nazarova$^3$, and~A.\,F.~Ronzhin$^4$}


\def\auf{$^1$Lebedev Institute of Precision Mechanics and Computer Engineering, 
 Russian Academy of Sciences,\\
 $\hphantom{^1}$kupyar@rambler.ru\\[1pt]
$^2$Dorodnicyn Computing Center, Russian Academy of Sciences, malash09@ccas.ru\\[2pt]
$^3$Dorodnicyn Computing Center, Russian Academy of Sciences, irina-nazar@yandex.ru\\[1pt]
$^4$Dorodnicyn Computing Center, Russian Academy of Sciences, raf-zao-zt@yandex.ru}


\def\leftkol{\ } % ENGLISH ABSTRACTS}

\def\rightkol{\ } %ENGLISH ABSTRACTS}

\titele{\tit}{\aut}{\auf}{\leftkol}{\rightkol}

\vspace*{-2pt}

\noindent
The problem of effective use of the heterogeneous computing system is considered at 
parallel processing of diverse tasks. In a case of date violation of works completion, the 
processor spent time belongs to production losses. Planning and optimization controls are 
exercised on the basis of the guaranteed estimates constructed for the worst case.

\vspace*{-5pt}

\KWN{parallel computing; multiprocessor systems; optimization; the principle of guaranteed result}


%\pagebreak

 \vskip 14pt plus 6pt minus 3pt

%\pagebreak

\def\leftkol{\ } % ENGLISH ABSTRACTS}
\def\rightkol{\ } %ENGLISH ABSTRACTS}

 %3
\def\tit{ON ESTIMATION OF THE EFFECTIVE BANDWIDTHS IN~A~SYSTEM WITH~REGENERATIVE INPUT}

\def\aut{A.\,V.~Borodina$^1$ and E.\,V.~Morozov$^2$}

\def\auf{$^1$Institute of Applied Mathematical Research, Karelian Research Center, Russian Academy of
Sciences,\\
$\hphantom{^1}$borodina@krc.karelia.ru\\[1pt]
$^2$Institute of Applied Mathematical Research, Karelian Research Center, Russian Academy of
Sciences,\\
$\hphantom{^1}$emorozov@karelia.ru}


\titele{\tit}{\aut}{\auf}{\leftkol}{\rightkol}

\vspace*{-2pt}
 
\noindent
The effective bandwidths (EB) of a communication system are considered. 
The EB guarantees that the  stationary workload overflow/loss   
probability to exceed a threshold is limited by a (small) quantity.  
It is shown how to calculate EB in a fluid queue  fed by an input  
with the independent increments. Then, a fluid system with regenerative input is considered. 
Using heuristic arguments, an approximation of the limiting   logarithmic exponential 
moment generating function of the input was  deduced.  Numerical simulations  show a  satisfactory accuracy 
of the  approximation applied to  a few systems with regenerative input. 


\vspace*{-5pt}

\KWN{effective bandwidths; fluid queue; workload process; overflow/loss probability;
effective bandwidths approximation; regenerative estimation}

 \vskip 14pt plus 6pt minus 3pt

%4
\def\tit{STATIONARY WAITING TIME DISTRIBUTION IN~QUEUEING SYSTEM WITH~NEGATIVE CUSTOMERS 
AND~BUNKER FOR~OUSTED CUSTOMERS UNDER FIRST--FIFO--FIFO SERVICE DISCIPLINE}

\def\aut{R.\,V.~Razumchik}

\def\auf{IPI RAN, rrazumchik@ieee.org}


%\def\leftkol{ENGLISH ABSTRACTS}
%\def\rightkol{ENGLISH ABSTRACTS}

\titele{\tit}{\aut}{\auf}{\leftkol}{\rightkol}

%\vspace*{-4pt}

\def\leftkol{ENGLISH ABSTRACTS}

\def\rightkol{ENGLISH ABSTRACTS}


\noindent
Consideration is given to the single server queueing system with 
Poisson input flows of ordinary and negative customers. An arriving 
ordinary customer occupies one place in an infinite buffer. Negative 
customer upon arrival pushes out one ordinary customer from the queue 
in the buffer to another queue (bunker) and leaves the system. Customers 
from bunker are served with relative priority. Service times of customers 
from buffer and bunker are both exponentially distributed but with different 
rates. It is assumed that negative customer always pushes out the first customer 
in the queue in the buffer and after service completion the first customer in the 
queue in the buffer enters server or, if buffer is empty, the first customer in the 
queue in the bunker. Stationary waiting time distribution of an arriving ordinary 
customer in terms of Laplace--Stieltjes transform is obtained.

%\vspace*{-6pt}

\KWN{queueing system; negative customers; waiting time}


 \vskip 14pt plus 6pt minus 3pt

%5
\def\tit{ON THE RATE OF CONVERGENCE TO~THE~NORMAL LAW OF~RISK ESTIMATE 
FOR~WAVELET COEFFICIENTS THRESHOLDING WHEN USING ROBUST VARIANCE ESTIMATES}

\def\aut{O.\,V.~Shestakov}

\def\auf{Department of Mathematical Statistics, Faculty of Computational Mathematics 
and Cybernetics, 
   M.\,V.~Lomonosov Moscow State University; IPI RAN, oshestakov@cs.msu.su}

%\def\leftkol{ENGLISH ABSTRACTS}
%\def\rightkol{ENGLISH ABSTRACTS}

\titele{\tit}{\aut}{\auf}{\leftkol}{\rightkol}

%\vspace*{-4pt}

\def\leftkol{ENGLISH ABSTRACTS}

\def\rightkol{ENGLISH ABSTRACTS}

\noindent
The asymptotic properties of risk estimate for thresholding 
wavelet coefficients of signal function are analyzed. 
Some estimates for rate of convergence to the normal law are obtained.


%\vspace*{-6pt}

\KWN{wavelets; thresholding; risk estimate; normal distribution; rate of convergence}

 \vskip 14pt plus 6pt minus 3pt

%6
\def\tit{STATISTICAL TESTING FOR THE NONEXECUTABILITY OF~FRAGMENTS OF~THE~CODE OF~A~LINEAR PROGRAM}


\def\aut{V.\,Yu.~Korolev$^1$, R.\,L.~Smelyansky$^2$, T.\,R.~Smelyansky$^3$, and~A.\,V.~Shalimov$^4$}

\def\auf{$^1$Faculty of Computational Mathematics and Cybernetics, 
   M.\,V.~Lomonosov Moscow State University;  IPI RAN,\\
$\hphantom{^1}$victoryukorolev@yandex.ru\\[1pt]
$^2$Faculty of Computational Mathematics and Cybernetics,
M.\,V.~Lomonosov Moscow State University,\\
 $\hphantom{^1}$smel@sc.msu.ru\\[1pt]
$^3$Faculty of Computational Mathematics and Cybernetics, 
   M.\,V.~Lomonosov Moscow State University,\\
   $\hphantom{^1}$smelyanskiy.t@bk.ru\\[1pt]
   $^4$Faculty of Computational Mathematics and Cybernetics, 
   M.\,V.~Lomonosov Moscow State University,\\
   $\hphantom{^1}$ashalimov@lvk.cs.msu.su}


%\def\leftkol{ENGLISH ABSTRACTS}
%\def\rightkol{ENGLISH ABSTRACTS}

\titele{\tit}{\aut}{\auf}{\leftkol}{\rightkol}

%\vspace*{-4pt}

\def\leftkol{ENGLISH ABSTRACTS}

\def\rightkol{ENGLISH ABSTRACTS}

\noindent
A problem of statistical testing for the nonexecutiveness of fragments of the code of a linear program is considered. Methods based on the
minimization of prior error probabilities are considered as well as those
based on the minimization of posterior error probabilities. 
 

%\vspace*{-6pt}

\KWN{testing statistical hypotheses; geometric distribution; probability of the error of the first kind;
probability of the error of the second kind; Neyman--Pearson lemma; posterior error probability}

%\pagebreak

 \vskip 14pt plus 6pt minus 3pt

%7
\def\tit{BAYESIAN RECURRENT MODEL OF RELIABILITY GROWTH: UNIFORM DISTRIBUTION OF~PARAMETERS}

\def\aut{A.\,A.~Kudriavtsev$^1$, I.\,A.~Sokolov$^2$, and S.\,Ya.~Shorgin$^3$} 


\def\auf{$^1$Faculty of Computational 
Mathematics and Cybernetics, M.\,V.~Lomonosov Moscow State University,\\
$\hphantom{^1}$nubigena@hotmail.com\\[1pt]
$^2$IPI RAN, isokolov@ipiran.ru\\[1pt]
$^3$IPI RAN, sshorgin@ipiran.ru}

%\def\leftkol{ENGLISH ABSTRACTS}
%\def\rightkol{ENGLISH ABSTRACTS}

\titele{\tit}{\aut}{\auf}{\leftkol}{\rightkol}

%\vspace*{-2pt}

\def\leftkol{ENGLISH ABSTRACTS}

\def\rightkol{ENGLISH ABSTRACTS}

\noindent
The paper is devoted to justification of expediency of Bayesian approach at the 
solution of the tasks connected with determination of reliability of complex modifiable systems. 
As illustration, average value of reliability of system is presented in a case where characteristics 
of ``defectiveness'' and ``efficiency'' of the means correcting imperfections of system are 
uniformly distributed.


%\vspace*{-2pt}

\KWN{modifiable information systems; reliability theory; Bayesian approach}


 \vskip 14pt plus 6pt minus 3pt


%8
\def\tit{THE ERROR-IN-VARIABLES MODEL IDENTIFICATION ON THE BASIS OF DEMING'S APPROACH}

\def\aut{V.\,S.~Timofeev$^1$, V.\,Yu.~Schekoldin$^2$, and~A.\,Yu.~Timofeeva$^3$} 

\def\auf{$^1$Novosibirsk State Technical University, netsc@rambler.ru\\[1pt]
$^2$Novosibirsk State Technical University, raix@ngs.ru\\[1pt]
$^3$Novosibirsk State Technical University, supernasty@mail.ru}

%\def\leftkol{ENGLISH ABSTRACTS}
%\def\rightkol{ENGLISH ABSTRACTS}

\titele{\tit}{\aut}{\auf}{\leftkol}{\rightkol}

%\vspace*{-2pt}

\def\leftkol{ENGLISH ABSTRACTS}

\def\rightkol{ENGLISH ABSTRACTS}

%\vspace*{-2pt}

\noindent
Some approaches to regression models constructing with stochasticity 
for both endogenous and exogenous variables are considered. The original 
geometric interpretation of particular cases of Deming regression for parameter estimation's   
functional is suggested. The proposition of mutual arrangement for straight, inverse, diagonal, 
and orthogonal regressions is proved. The bias and standard deviation for regression parameter 
estimators in dependence of weight's coefficients ratio in Deming model has been obtained.

%\vspace*{-2pt}

\KWN{least square estimation; Deming regression; geometric interpretation; dispersion ellipse}

\vskip 14pt plus 6pt minus 3pt



%9
\def\tit{ASYMPTOTIC NORMALITY OF THE ESTIMATION OF THE MULTIVARIATE LOGISTIC REGRESSION}

\def\aut{A.\,Yu.~Khaplanov}

\def\auf{M.\,V.~Lomonosov Moscow State University, Khaplanova@gmail.com}


\def\leftkol{ENGLISH ABSTRACTS}

\def\rightkol{ENGLISH ABSTRACTS}

\titele{\tit}{\aut}{\auf}{\leftkol}{\rightkol}

%\vspace*{-2pt}

\noindent
Estimation of characteristics of the multivariate logistic regression with a 
diverging number of covariates has been made. The convergence rate for estimate of 
characteristics of the multivariate logistic regression coefficients has been obtained. 
Asymptotic normality of its rejection has been proved.

%\vspace*{-2pt}

\KWN{logistic regression; convergence rate;  asymptotic normality; high-dimensional covariates}
%\pagebreak

\vskip 14pt plus 6pt minus 3pt

%10
\def\tit{ASYMPTOTIC EXPANSIONS FOR THE DISTRIBUTION FUNCTIONS OF STATISTICS CONSTRUCTED FROM SAMPLES WITH RANDOM SIZES}

\def\aut{V.\,E.~Bening$^1$, N.\,K.~Galieva$^2$, and~V.\,Yu.~Korolev$^3$}

\def\auf{$^1$Department of Mathematical Statistics, Faculty of Computational Mathematics 
and Cybernetics,\\
$\hphantom{^1}$M.\,V.~Lomonosov Moscow State University; IPI RAN, bening@yandex.ru\\[1pt]
$^2$Kazakhstan Branch of the M.\,V.~Lomonosov Moscow State University, nurgul\_u@mail.ru\\[1pt]
$^3$Faculty of Computational Mathematics and Cybernetics, 
   M.\,V.~Lomonosov Moscow State University;  IPI RAN,\\
   $\hphantom{^1}$victoryukorolev@yandex.ru}


\def\leftkol{ENGLISH ABSTRACTS}

\def\rightkol{ENGLISH ABSTRACTS}

\titele{\tit}{\aut}{\auf}{\leftkol}{\rightkol}

%\vspace*{-2pt}

\noindent
A general transfer tteorem is proved making it possible to construct asymptotic expansions 
for the distribution function of a statistic constructed from the sample with a random size 
from the asymptotic expansion for the distribution function of the random sample size and the 
asymptotic expansion for the distribution function of the same statistic constructed from the 
samples with a nonrandom size.

 
%\vspace*{-2pt}

\KWN{sample with a random size; asymptotic expansion; transfer theorem; 
mixture of probability distributions; 
Laplace distribution; Student distribution}

%\pagebreak

\vskip 14pt plus 6pt minus 3pt

% \vskip 12pt plus 6pt minus 3pt

%11
\def\tit{ON CONVERGENCE OF RANDOM WALKS GENERATED BY~COMPOUND COX PROCESSES
TO~LEVY PROCESSES}

\def\aut{V.\,Yu.~Korolev$^1$, L.\,M.~Zaks$^2$, and~A.\,I.~Zeifman$^3$}

\def\auf{$^1$Faculty of Computational Mathematics and Cybernetics, 
   M.\,V.~Lomonosov Moscow State University;  IPI RAN,\\
$\hphantom{^1}$victoryukorolev@yandex.ru\\[1pt]
$^2$Department of Modeling and Mathematical Statistics, Alpha-Bank, lily.zaks@gmail.com\\[1pt]
$^3$Vologda State Pedagogical University; IPI RAN, a$\_$zeifman@mail.ru}




\def\leftkol{ENGLISH ABSTRACTS}

\def\rightkol{ENGLISH ABSTRACTS}

\titele{\tit}{\aut}{\auf}{\leftkol}{\rightkol}

%\vspace*{-2pt}

\noindent 
A functional limit theorem is proved establishing weak convergence of random 
walks generated by compound doubly stochastic Poisson processes to L$\acute{\mbox{e}}$vy processes 
in the Skorokhod space. As corollaries, theorems on convergence of random 
walks with jumps having finite variances to L$\acute{\mbox{e}}$vy processes 
with mixed normal distributions, 
in particular, to stable L$\acute{\mbox{e}}$vy processes have been proved.


%\vspace*{-2pt}

\KWN{stable distribution; L$\acute{\mbox{e}}$vy process; stable L$\acute{\mbox{e}}$vy process; 
compound doubly stochastic Poisson process (compound Cox process); 
Skorokhod space; transfer theorem}

\vskip 14pt plus 6pt minus 3pt


%12
\def\tit{STATISTICAL MECHANISMS OF THE SUBJECT DOMAINS ASSOCIATIVE PORTRAITS FORMATION
 ON~THE~BASIS 
OF~BIG NATURAL LANGUAGE TEXTS FOR~THE~SYSTEMS OF~KNOWLEDGE EXTRACTION}

\def\aut{M.\,M.~Charnine$^1$, N.\,V.~Somin$^2$, 
I.\,P.~Kuznetsov$^3$, Yu.\,I.~Morozova$^4$, I.\,V.~Galina$^5$, and~E.\,B.~Kozerenko$^6$}

\def\auf{$^1$IPI RAN, 1@keywen.com\\[1pt]
$^2$IPI RAN, somin@post.ru\\[1pt]
$^3$IPI RAN, igor-kuz@mtu-net.ru\\[1pt]
$^4$IPI RAN, judez@yandex.ru\\[1pt]
$^5$IPI RAN, irn\_gl@mail.ru\\[1pt]
$^6$IPI RAN, kozerenko@mail.ru}
   
 \def\leftkol{ENGLISH ABSTRACTS}

\def\rightkol{ENGLISH ABSTRACTS}

\titele{\tit}{\aut}{\auf}{\leftkol}{\rightkol}

%\vspace*{-2pt}

\noindent 
Associative relations   between terms, concepts and other elements of natural language  
play an important role in decision of a wide variety of application tasks including intelligent 
texts processing, knowledge extraction, and management comprizing the formation of knowledge bases 
and semantic information retrieval.  The paper presents the methods of automatic establishment of 
the associative relations between terms and concepts in the texts from Internet and creation of 
subject domains associative portraits designed for the tasks of intelligent systems development.   
An associative portrait of a subject domain (APSD) is a dictionary of the meaningful support terms 
and word combinations interconnected by associative relations.  It is essential that the APSD are 
constructed automatically on the basis of statistical analysis of big volumes of texts.  
The theoretical impact of the proposed method consists in the use of statistics, corpus 
linguistics, and distributional semantics for processing big volumes of natural language 
texts which are dynamically updated and enriched in the Internet for constructing the model 
of a subject domain in the form of APSD.


%\vspace*{-2pt}

\KWN{automatic processing of text corpora; statistical methods; 
intelligent Internet technologies; lexical semantic analysis; 
knowledge extraction from texts; semantic retrieval; semantic vectors; semantic context space}


\vskip 14pt plus 6pt minus 3pt

%13
\def\tit{INFORMATION TECHNOLOGIES FOR CREATING THE DATABASE 
OF~EQUIVALENT VERBAL FORMS IN~THE~RUSSIAN-FRENCH MULTIVARIANT 
PARALLEL CORPUS}

\def\aut{S.~Loiseau$^1$, D.\,V.~Sitchinava$^2$, A.\,A.~Zalizniak$^3$, and~I.\,M.~Zatsman$^4$}

\def\auf{$^1$Universit$\acute{\mbox{e}}$ Paris 13, Sorbonne Paris Cit$\acute{\mbox{e}}$, Laboratoire LDI 
(Lexiques, dictionnaires, informatique), CNRS,\\ 
$\hphantom{^1}$UMR 7187, sylvain.loiseau@univ-paris13.fr\\[1pt]
$^2$Institute of the Russian Language of the Russian Academy of Sciences, mitrius@gmail.com\\[1pt]
$^3$Institute of Linguistics of the Russian Academy of Sciences; 
    IPI RAN,  anna.zalizniak@gmail.com\\[1pt]
$^4$IPI RAN, iz\_ipi@a170.ipi.ac.ru}


\def\leftkol{ENGLISH ABSTRACTS} % ENGLISH ABSTRACTS}

\def\rightkol{ENGLISH ABSTRACTS}

\titele{\tit}{\aut}{\auf}{\leftkol}{\rightkol}

%\vspace*{12pt}

\noindent
The Russian-French parallel corpus as a part of the Russian National Corpus is 
being transformed into a multivariant corpus with several translations corresponding to 
each original texts. Concurrently, a database of functionally equivalent 
lexicogrammatical verbal forms is being created using the multivariant corpus. The main 
purpose of database creation is to calculate the statistical estimates of the equivalences 
between Russian and French verbal forms. The paper discusses an information 
technology for creating the Russian-French multivariant parallel corpus and the 
database simultaneously.

%\vspace*{-5pt}
 \label{end\stat}


\KWN{parallel multivariant corpora; Russian National Corpus; information 
technologies; XML marking up Russian-French parallel texts; lexicogrammatical 
form; functional equivalence; statistical estimates of equivalences}


\newpage