\def\stat{malashenko}

\def\tit{МЕТОДЫ ОЦЕНКИ ЭФФЕКТИВНОСТИ И~ДИРЕКТИВНЫХ СРОКОВ ВЫПОЛНЕНИЯ РЕСУРСОЕМКИХ
 ВЫЧИСЛИТЕЛЬНЫХ ЗАДАНИЙ}

\def\titkol{Методы оценки эффективности и директивных сроков выполнения ресурсоемких вычислительных заданий}

\def\autkol{И.\,К.~Купалов-Ярополк, Ю.\,Е.~Малашенко, И.\,А.~Назарова, А.\,Ф.~Ронжин}

\def\aut{И.\,К.~Купалов-Ярополк$^1$, Ю.\,Е.~Малашенко$^2$, И.\,А.~Назарова$^3$, А.\,Ф.~Ронжин$^4$}

\titel{\tit}{\aut}{\autkol}{\titkol}

%{\renewcommand{\thefootnote}{\fnsymbol{footnote}}\footnotetext[1]
%{Работа
%поддержана Российским фондом фундаментальных исследований (проекты
%11-01-00515а, 11-07-00112а, 11-01-12026-офи-м), Министерством
%образования и науки РФ (госконтракт 16.740.11.0133).}}

\renewcommand{\thefootnote}{\arabic{footnote}}
\footnotetext[1]{Институт точной механики и вычислительной техники им.\ С.\,А.~Лебедева Российской академии наук,  
kupyar@rambler.ru}
\footnotetext[2]{Вычислительный центр им.\ А.\,А.~Дородницына Российской академии наук, malash09@ccas.ru}
\footnotetext[3]{Вычислительный центр им.\ А.\,А.~Дородницына Российской академии наук, irina-nazar@yandex.ru}
\footnotetext[4]{Вычислительный центр им.\ А.\,А.~Дородницына Российской академии наук, raf-zao-zt@yandex.ru}

\vspace*{4pt}

\Abst{Рассматривается проблема эффективного использования гетерогенной вычислительной 
системы при параллельной обработке разнородных заданий. В~случае нарушения сроков завершения 
работ затраченное процессорное время относится к производственным потерям. Планирование и 
оптимизация управления осуществляются на основе  гарантированных оценок, построенных для 
наихудшего случая.}

\vspace*{2pt}

\KW{параллельные вычисления; многопроцессорные системы;  оптимизация;  принцип гарантированного результата}

\vskip 14pt plus 9pt minus 6pt

      \thispagestyle{headings}

      \begin{multicols}{2}

            \label{st\stat}

\section{Введение}

В настоящей работе предлагается модель для анализа эффективности
и оперативного планирования процедуры обработки ресурсоемких
вы\-чис\-ли\-тель\-ных заданий, которые подробно описаны и определены в~\cite{Prep11} 
как citu-за\-да\-ния: computationally intensive task under
uncertainty.  Решение каж\-дой из citu-за\-дач  состоит в просмотре
большого массива исходных данных и выделении из него одного
уникального фрагмента с наперед заданными свойствами. В~ходе  поиска
реализуется один и тот же переборный алгоритм для различных
начальных данных, разбитых  на неделимые, содержательно значимые
фрагменты. Если при выполнении задания такой фрагмент найден, то
говорят, что задача решена, и обработка задания прекращается.  Если
же  просмотрены  все  предъявленные данные и  найти уникальный
фрагмент не удалось, то задание считается выполненным, но задача не
имеет решения.

Процесс просмотра происходит в условиях неопределенности, связанной
как с длительностью поиска, так и с потенциальной возможностью
получить решение. Все citu-за\-да\-ния выполняются в режиме реального
времени и  допускают распараллеливание по данным~\cite{Sour}.

В современной практике при выполнении citu-ра\-бот используются
гетерогенные высокопроизводительные специализированные
вычислительные сис\-те\-мы (СВС). Обычно СВС имеет несколько управ\-ля\-ющих
узлов, на каж\-дом из которых разворачивается ядро сис\-те\-мы управ\-ле\-ния,
и ряд вы\-чис\-ли\-тель\-ных узлов, непосредственно обрабатыва\-ющих задания.
Специализированные
вычислительные сис\-те\-мы оснащаются  специализированными устройствами, позволяющими
значительно повысить скорость исполнения отдельных вы\-чис\-ли\-тель\-ных
процедур по сравнению со стандартными реализациями. В качестве
элементной базы таких ускорителей, согласно~\cite{Kal}, могут
использоваться заказные сверхбольшие интегральные схемы (СБИС) (ASIC~--- application-specific
integrated circuits), базовые мат\-рич\-ные крис\-тал\-лы (БМК) (\mbox{eASIC}),
сис\-те\-мы-на-крис\-тал\-ле (SoC~--- systems on chip), программируемые логические интегральные схемы (ПЛИС) 
(FPGA~--- field-programmable gate array), графические ускорители
GPGPU (general-purpose graphics processing units).

Различные типы устройств выполняют одно и то же задание с разной
производительностью. Кроме того, некоторые ускорители могут
предназначаться  для работы только с определенными типами алгоритмов
и подходят для решения ограниченного класса задач. Для
специализированных устройств\linebreak в качестве ресурсной единицы
(единичного вы\-чис\-ли\-тель\-но\-го модуля) в СВС рассматривается отдельно
каж\-дая микросхема (ПЛИС, кристалл) как наимень\-ший возможный объект,
имеющий собственное множество состояний и допускающий независимое
управление.

При планировании вычислительных работ в\linebreak СВС возникает
на\-уч\-но-тех\-ни\-че\-ская проб\-ле\-ма, состоящая в отыскании способов
эффективного распределения разнотипных ресурсов между разнородными
работами, выполняющимися в заданных\linebreak  временн$\acute{\mbox{ы}}$х рамках.

В научной литературе понятие эффективности трактуется весьма широко.
В данной работе под эффективностью будем понимать отношение реально
произведенных в СВС специализированных вычислительных операций к
максимально возможному их числу, взятое на определенном интервале
времени.

Рассмотрим специализированную систему, состоящую из большого числа
вычислительных модулей различных типов. Пусть ее диспетчеру
взаимосвязанные пользователи в случайные моменты времени
предоставляют для решения разнородные  citu-за\-да\-чи. Все задания
вместе и каж\-дое в отдельности должны быть завершены  как можно
быстрее, но не позднее заранее определенных сроков. В~противном
случае полученное решение может потерять актуальность и не будет
представлять интереса, а вычислительные затраты будут отнесены к
производственным потерям (издержкам).

Массив исходных данных для каж\-дой из citu-за\-дач состоит из
одинаковых по размеру отдельных неделимых содержательно значимых
фрагментов и становится известен после предварительного анализа
задачи в  системе. Выполнение отдельной  citu-ра\-бо\-ты прекращается,
когда обнаружен фрагмент, удовлетворяющий наперед заданному
критерию,~--- уникальный фрагмент, или если просмотрен весь массив,
но ничего найти не удалось. Считается, что архитектура
вычислительной системы позволяет просматривать  фрагменты каж\-до\-го
citu-за\-да\-ния  в произвольном порядке и каж\-дая часть данных может
обрабатываться независимо, в том числе одновременно  с другими.

Администратор или опе\-ра\-тор-пла\-ни\-ров\-щик в режиме реального времени должен:
\begin{enumerate}[(1)]
\item эффективно использовать разнотипные вы\-чис\-ли\-тель\-ные модули при обработке разнородных вычислительных заданий;
\item завершать каж\-дое конкретное  задание  до наступления назначенного срока и тем самым минимизировать потери.
\end{enumerate}
При этом обработка происходит в условиях неопределенности, связанной
со случайным характером формирования текущего набора citu-за\-да\-ний,
длительностью процесса обработки каж\-до\-го из них и с потенциальной
возможностью получить решение.

В литературе при проектировании и анализе современных сис\-тем
реального времени  в основном рассматриваются <<жестко>> заданные
директивные сроки окончания, нарушение которых может иметь фатальные
последствия. В~ходе изучения таких систем вначале делается
предположение, что  директивный срок окончания может быть превышен,
а затем предлагаются меры, позволяющие этого избежать~\cite{RT2004}.
Настоящая работа посвящена изучению функционирования гетерогенной
вычислительной системы с более мягкими временн$\acute{\mbox{ы}}$ми ограничениями. 
В~рас\-смат\-ри\-ва\-емой модели считается, что при поступлении citu-за\-да\-чи в
систему  определяется\linebreak возможное время ее завершения с учетом
объек-\linebreak тивных показателей  загруженности последней и имеющихся
заданий. Если предварительная оценка~--- срок, к которому
citu-за\-да\-ние может быть\linebreak завершено,~--- устраивает пользователя, то
оно принима\-ется для обработки. При этом заданию на\-зна\-ча\-ет\-ся
директивный срок окончания (ДСО), совпада\-ющий с прогнозируемым. Таким
образом,\linebreak ДСО~--- это уста\-нов\-лен\-ный
диспетчером и согласованный с пользователем момент календарного\linebreak
времени, до наступления которого citu-за\-да\-ча должна быть решена.
Дальнейшее планирование осуществляется исходя из предположения, что
события будут развиваться по наихудшему сценарию. В~условиях
объективной неопределенности  ДСО  используется как ограничение при
формировании диспетчерских правил (политики) совместного выполнения
всех заданий, находящихся  в~СВС.

Заметим, что понятие ДСО, введенное выше, несколько отличается от
классического dead-line~\cite{Stan}, поскольку устанавливается
диспетчером, а не пользователем сис\-те\-мы, хотя и по договоренности с
последним. Однако принятое допущение больше соответствует реальной
ситуации. Действительно, в этом случае citu-ра\-бо\-та, которая не может
быть выполнена по объективным причинам, будет сразу отозвана, что
позволит решить другие задачи в срок. У~администратора появляется
возможность более гибко подходить к формированию пакета текущих
заданий и минимизировать возможные производственные потери.

Данная работа продолжает исследования, начатые в~[6--8], однако
отличается от них мето-\linebreak дами изуче\-ния и конечной целью. 
В~\cite{Gol11, Mal412} с\linebreak по\-мощью имитационного моделирования
анализировались временн$\acute{\mbox{ы}}$е показатели длительности обработки заданий.
В~настоящей статье показано, как, опираясь на  методы оптимизации~\cite{Suh} 
и принцип гарантированного результата~\cite{germ}, можно
оценить и минимизировать максимальную величину возможного превышения
установленных сроков завершения для любого набора исходных
citu-за\-да\-ний и повысить эффективность использования СВС. 

%
Перейдем к
описанию математической модели (М-мо\-де\-ли), которая может быть
использована при оперативном планировании работ и реализации
различных диспетчерских политик.


\section{Описание модели}

При описании М-модели верхний индекс любой переменной будет
всегда относиться к типу вычислительного модуля, а нижний (один или
два) в зависимости от рассматриваемого случая~--- отвечать номеру и
виду задачи соответственно. Для того чтобы избежать путаницы, второй
нижний идентификатор, соответствующий виду задачи,  помещен в
скобки. Кроме того, в зависимости от контекста  через~$t$ обозначим
либо текущий календарный момент времени,  либо контрольную точку
принятия решения. Таким образом,  $t$  не является переменной  в
традиционном смысле, это обозначение определенных моментов времени в
модельном описании процесса. Далее будем говорить об абстрактной
гетерогенной высокопроизводительной  СВС, в которой в условиях
неопределенности выполняются разнородные citu-ра\-боты.

В рамках М-модели считается, что  указанная СВС состоит из
центрального управляющего устройства (ЦУП-устрой\-ст\-ва) и набора
независимых исполняющих единичных вычислительных модулей
(ЕВ-мо\-ду\-лей) различных конструктивных типов. Введем обозначения:

$\textbf{H}$~--- ЦУП-устройство, в котором происходит анализ
поступающих заданий, определяются ДСО и порядок их выполнения, а
также осуществляется контроль процесса обработки;

$\textbf{E}$~ --- набор (множество)  ЕВ-мо\-ду\-лей нескольких типов.

Пусть
$M$~--- общее число различных типов ЕВ-мо\-ду\-лей, из которых состоит рассматриваемая СВС;
$e^m$~--- отдельный ЕВ-мо\-дуль (устройство) \mbox{$m$-го} типа;
$E^m$~ --- множество  ЕВ-мо\-ду\-лей $m$-го типа $e^m$, $m \hm= \overline{1, M}$,
следовательно,  множество $\textbf{E}$ всех ЕВ-мо\-ду\-лей в СВС можно
записать в виде объединения
$$
\textbf{E} = E^1 \bigcup E^2 \bigcup \cdots \bigcup E^M\,. 
$$

Под специализированной элементарной вы\-чис\-ли\-тельной операцией
(СЭВ-опе\-ра\-ция) будем \mbox{понимать} просмотр отдельного
неделимого содержательно значимого фрагмента данных определенного
вида  и проверку  его  уникальности. Считается, что любой ЕВ-модуль
может обрабатывать задания по крайней мере одного типа.
Производительность конкретного ЕВ-мо\-ду\-ля при выполнении citu-за\-да\-ний
разных видов  не одинакова и может меняться (перегрев оборудования,
изменение тактовой час\-то\-ты или работа с другим программным
обеспечением). Кроме того, будем считать, что ЕВ-мо\-ду\-ли с течением
времени  могут выходить из строя.

Пусть $p_k^m(t)$~--- производительность  ЕВ-мо\-ду\-ля $m$-го типа при
обработке задания $k$-го вида, т.\,е.\ модуль $e^m$ может выполнять в
единицу времени   $p_k^m(t)$  СЭВ-опе\-ра\-ций $k$-го вида, начиная с
момента $t$; $R^m(t)$~---  общее число работоспособных ЕВ-мо\-ду\-лей
$m$-го типа на момент времени~$t$.


Будем считать, что СВС может одновременно обрабатывать как одно, так
и несколько заданий на всех действующих в данный момент 
ЕВ-мо\-ду\-лях. Отдельное citu-за\-да\-ние может быть разделено на
подзадания, каж\-дое из которых, в свою очередь, выполняется
самостоятельно или в составе некоторого набора (пакета). Для решения
каж\-дой citu-за\-да\-чи достаточно найти хотя бы один уникальный
фрагмент.

Для формального описания  заданий будут использоваться следующие обозначения:

$z_{n}$~--- задание (задача) с собственным идентификационным номером~$n$;

$K$~--- общее число видов заданий, которые могут обрабатываться в
данной СВС;

$z_{n{(k)}}$~--- задание (задача) с собственным идентификационным
номером $n$, для которой явно указан ее вид~$k$, $k\hm=\overline {1,K}$.
Данное обозначение введено для удобства описания  процесса обработки
разнородных заданий;

$t_n^0$~--- календарный момент поступления задания $z_n$ в СВС;
$T_n(t)\hm=t \hm- t_n^0$~--- длительность  промежутка
времени,  в течение которого задание $z_n$ находится в СВС при
условии, что на момент~$t$ оно еще не выполнено до конца, т.\,е.\
$T_n(t)$~---  число единиц времени, которое прошло от поступления~$z_n$ 
в СВС  до текущего момента~$t$; $t_n^+$~--- момент завершения
задания $z_n$ и/или  его удаления из СВС; $T_n^+ \hm= t_n^+ \hm- t_n^0$~--- 
длительность  промежутка времени, в течение которого задание~$z_n$  обрабатывалось  СВС,\linebreak 
т.\,е.\ $T_n^+$~---  длительность пребывания~$z_n$ в сис\-теме.


Для множества $\mathcal{Z}(t)$ заданий $z_n$, находящихся в СВС в момент~$t$,
введем обозначения:

$\mathcal{N}(t)$~--- множество номеров  заданий $z_n$ из~$\mathcal{Z}(t)$;

$N(t)$~--- общее число заданий $z_n$ из  $\mathcal{Z}(t)$. Таким образом  
$|\mathcal{Z}(t)| \hm= |\mathcal{N}(t)| \hm= N(t)$;

$\mathcal{Z}_k(t)$~--- множество заданий $k$-го вида;

$\mathcal{N}_k(t)$~--- множество номеров  (индексов) заданий $k$-го вида;

$N_k(t)$~--- общее число заданий $k$-го вида.

Тогда
\begin{gather*}
\mathcal{Z}(t) = \bigcup\limits_{k = 1}^{K} \mathcal{Z}_k(t)\,; \quad 
\mathcal{N}(t) = \bigcup_{k = 1}^{K} \mathcal{N}_k(t)\,; \\
N(t) =\sum\limits_{k = 1}^{K} N_k(t)\,.
\end{gather*}

В М-модели предполагается, что для каж\-до\-го задания $z_n$ в момент
поступления $t_n^0$ становится известна нормативная величина
$\textbf{Z}_{n}$~---  общее  число СЭВ-опе\-ра\-ций, которые будет
необходимо выполнить для данного задания в случае, когда  задача~$z_{n}$  
не имеет решения, т.\,е.\ в исходной содержательной
постановке $\textbf{Z}_{n}$~--- общее число   неделимых фрагментов
данных задания~$z_{n}$, которые необходимо будет обработать, если
среди них нет уникального.

В СВС диспетчеризация процесса выполнения citu-за\-да\-ний
осуществляется с  помощью про\-грам\-мно\-го комплекса планирования и
анализа (ПК-план). Для каж\-дой поступившей  citu-за\-да\-чи в  ПК-пла\-не
определяются необходимые вычислительные  затраты и время решения  в
наихудшем случае с учетом общей загруженности СВС. После этого по
согласованию с пользователем назначается ДСО, т.\,е.\ календарный
момент времени, до наступления которого  $z_n$ должно быть
завершено. Обозначим  через $d_n$ ДСО для~$z_n$.

В СВС вся исходная и текущая информация о citu-за\-да\-ни\-ях, вновь
поступивших или находящихся в обработке, заносится  и далее хранится
в так называемой базе данных заданий (БД-за\-да\-ний). Про\-граммный комплекс планирования и
анализа
постоянно просматривает и анализирует БД-за\-да\-ний. Динамика обработки
(просмотров) массивов данных в ПК-пла\-не описывается сис\-те\-мой
неравенств и ко\-неч\-но-раз\-ност\-ных уравнений в дискретном времени. Шаг
по времени называется плановым периодом или операционным окном.

В контрольной точке принятия решения~$t$ по команде с
ЦУП-устрой\-ст\-ва ПК-план анализирует\linebreak данные о  выполнении всех
заданий, находящихся в СВС. Заметим, что  здесь $t$~--- начальный\linebreak
момент очередного планового периода. Далее диспетчер (специалист
и/или алгоритмическая процедура)  выбирает число единиц календарного
времени $\Delta(t)$~--- длительность планового периода.\linebreak  В~ПК-пла\-не
формируется пакет текущих работ\linebreak (ТР-па\-кет), состоящий из подмассивов
(поднаборов) неделимых фрагментов данных  подзаданий $z_{n}$, $n \hm\in
\mathcal{N}(t)$, которые будут  обрабатываться в СВС   в течение $\Delta(t)$.
Центральное управ\-ля\-ющее устрой\-ст\-во  помещает  ТР-пакет в специальный буфер текущих работ
(ТР-бу\-фер) и начиная с момента~$t$   СВС выполняет  его на всех
работоспособных ЕВ-мо\-дулях.

\section{Параметры планирования и~управления }

Обозначим через $z^-_{n(k)}(t)$ число СЭВ-опе\-ра\-ций $k$-го вида,
которые были завершены для задания~$z_n$ от его поступления в СВС до
момента времени~$t$, или в рамках М-мо\-де\-ли~--- чис\-ло неделимых
фрагментов, которые уже были просмотрены для задания~$z_n$ к моменту
времени~$t$;  соответственно $z^+_{n(k)}(t)$~--- число фрагментов,
которые  для~$z_n$ остались необработанными. Тогда
$$
z^+_{n(k)}(t)= \textbf{Z}_{n} - z_{n(k)}^-(t)\,, \ \ n \in \mathcal{N}(t)\,.      
$$

Как уже указывалось ранее, подзадания ТР-па\-ке\-та представляют собой
различающиеся по объему поднаборы фрагментов данных для заданий
$z_n$, $n \hm\in \mathcal{N}(t)$. Размер подзадания для  $z_{n{(k)}}$ обозначим
через $w_{n(k)}(\Delta(t))$. Таким образом, $w_{n(k)}(\Delta(t))$~--- 
чис\-ло неделимых фрагментов  в поднаборе, взятом из массива
начальных данных задания $z_{n(k)}$ в момент~$t$ для просмотра в
течение  $\Delta(t)$, или чис\-ло СЭВ-опе\-ра\-ций  $k$-го вида, которые
планируется выполнить в операционном окне $\Delta(t)$ для
$z_{n(k)}$. Размеры отобранных поднаборов данных (подзаданий)
$w_{n(k)}(\Delta(t))$  являются управ\-ле\-ни\-ями, или управляющими
параметрами, которые определяются в ПК-пла\-не с учетом текущей
диспетчерской политики соблюдения ДСО.


Поскольку в момент $t$ для $z_n$ нельзя взять в обработку больше
данных, чем осталось, управления $w_{n(k)}(\Delta(t))$ на интервале
$\Delta(t)$ должны удовлетворять ограничениям
\begin{equation}
0 \le w_{n}(\Delta(t)) \le  z^+_{n}(t)\,,  \ \ n \in \mathcal{N}(t)\,. 
\label{e1-mal}
\end{equation}

Обозначим через $W_{k} (\Delta(t))$ суммарное чис\-ло фрагментов
$k$-го вида, которые планируется включить в ТР-па\-кет. Тогда
$W (\Delta(t))$~--- общий объем ТР-па\-ке\-та  для просмотра в операционном окне $\Delta(t)$:
\begin{multline}
W (\Delta(t)) = \sum\limits_{k = 1}^K W_{k}(\Delta(t)) =  {}\\
{}=
\sum\limits_{k = 1}^K \sum\limits_{n \in \mathcal{N}_k(t)} w_{n(k)}(\Delta(t)) =  
\sum\limits_{n \in \mathcal{N}(t)} w_{n}(\Delta(t))\,.
\label{e2-mal}
\end{multline}


Введем переменную $r_{k}^m(t)$~--- общее число  ЕВ-мо\-ду\-лей  $m$-го
типа, которые начинают обрабатывать подзадания  $k$-го вида в момент
времени~$t$. $r_{k}^m(t)$ также являются управ\-ле\-ни\-ями  и
определяются исходя из   текущих установок диспетчерских правил.

Предположим, что ни один из поднаборов $w_{n(k)}(\Delta(t))$ не 
содержит уникального фрагмента, но все поднаборы планируется  
выполнить за  период $ \Delta(t)$.  Тогда в момент~$t$ для просмотра всех данных 
для подзаданий $k$-го вида необходимо выделить  $r_{k}^m(t)$  ЕВ-мо\-ду\-лей $m$-го типа,
\begin{equation}
W_{k} (\Delta(t)) = \sum\limits^M_{m = 1} p_{k}^m (t)  r_{k}^m (t)  \Delta(t)\,, 
\ \ k = \overline{1,K}\,, 
\label{e3-mal}
\end{equation}
которые за период $\Delta(t)$ должны  проделать $W_{k}(\Delta(t))$   СЭВ-опе\-ра\-ций $k$-го вида.
Поскольку число ЕВ-модулей, распределяемых для выполнения заданий, не может превышать 
общего числа работоспособных, то
значения $r_{k}^m(t)$ должны удовлетворять следующим ограничениям:
\begin{equation}
\left.  
\begin{array}{c}
\displaystyle\sum\limits_{k=1}^K r_{k}^m(t) \le R^m(t)\,, \ \ m = \overline{1,M}\,;\\[9pt]
r_{k}^m (t) \ge 0\,, \ \ k = \overline{1,K}\,, \ \ m = \overline{1,M}\,.
                    \end{array}
                    \right \} 
                    \label{e4-mal}
                    \end{equation}

\section{Интегральная производительность специализированной вычислительной системы}

При диспетчеризации СВС, где на разнотипном оборудовании с различной
скоростью выполняются разнородные СЭВ-опе\-ра\-ции, возникает проблема
эффективного использования имеющихся ресурсов.

Обозначим через $\rho(t)$ интегральную производительность СВС. Под
$\rho(t)$ будем понимать суммарную производительность СВС на
интервале $ \Delta(t)$, т.\,е.\ общее число СЭВ-операций,
которые выполняются на всех  действующих ЕВ-мо\-ду\-лях, начиная с
момента~$t$ на отрезке планирования $ \Delta(t)$:
$$
\rho(t) = \sum\limits_{m = 1}^{M} \sum\limits_{k = 1}^{K}  p_k^m(t) r_k^m(t)  \Delta(t)\,.
$$
В качестве максимальной интегральной производительности СВС $P^*(t)$  на интервале 
$\Delta(t)$ выберем скалярное решение следующей задачи линейного программирования: найти
$
P^*(t) \hm= \max\limits_{\rho, r} \rho(t) 
$
при ограничениях  на производительность СВС
\begin{equation}
\left.  
\begin{array}{rl}
\rho(t) &= \sum\limits_{m = 1}^{M} \sum\limits_{k = 1}^{K}  p_k^m(t) r_k^m(t)  \Delta(t)\,;\\[9pt]
\sum\limits_{k = 1}^{K} r_k^m (t)  &= R^m(t)\,, \  \ m = \overline{1, M}\,;\\[9pt]
r_k^m (t) &\ge 0\,, \ \ k = \overline {1, K}\,, \ \ m = \overline{1,M}\,.
                    \end{array}
                    \right \} 
                    \label{e5-mal}
\end{equation}

Значение $P^*(t) $ показывает, какое максимальное число СЭВ-операций
суммарно для всех видов заданий может быть выполнено СВС в
операционное окно $ \Delta(t)$ начиная с момента~$t$. Пусть на
интервале $ \Delta(t)$ предполагается произвести $W(\Delta(t))$
СЭВ-опе\-ра\-ций всех видов (см.~(\ref{e2-mal})). Тогда отношение
$$
\gamma(t) = \fr{ W(\Delta(t))}{P^*(t)}
$$
назовем плановой эффективностью использования СВС на интервале 
$\Delta(t)$ для данного набора заданий $\mathcal{Z}(t)$. Максимальное значение
$W(\Delta(t))$ и, соответственно, $\gamma (t)$  существенно зависит
от состава и объема заданий, находящихся на обработке в момент
времени $t$ (см.\ (\ref{e1-mal}) и~(\ref{e2-mal})). Сравнивая множества ограничений~(\ref{e1-mal})--(\ref{e4-mal}) 
и~(\ref{e5-mal}), можно утверждать, что для любого набора заданий $\mathcal{Z}(t)$
выполняются неравенства
$$
P^*(t) \ge  W(\Delta(t))\,, \quad 0 \le \gamma(t ) \le 1\,.
$$
Кроме того, при диспетчеризации следует учитывать другие ограничения
и требования пользователей, которым должны подчиняться управ\-ле\-ния
$r_k^m (t), w_{n(k)}(\Delta(t))$.

\section{Определение директивного срока окончания вновь поступившего задания}

Для управления  гетерогенными СВС в реальном времени необходимы и разрабатываются интерактивные 
сценарии диалога дис\-пет\-чер--поль\-зо\-ва\-тель в декларативной, час\-тич\-но вербальной, частично 
технологической, постановке: за\-да\-ние--тре\-бо\-ва\-ния--воз\-мож\-но\-сти--сро\-ки.
В~рамках М-мо\-де\-ли в ПК-пла\-не для каж\-до\-го вновь поступившего задания вычисляется 
гарантированная оценка времени завершения. Фактически решается задача быстродействия 
выполнения всех заданий в наихудшем случае.

Пусть в момент $t$ в СВС, которая обрабатывает пакет заданий $\mathcal{Z}(t)$,  поступает новая 
задача $\hat{z}$, например, $\hat{k}$-го вида. В ПК-пла\-не этой задаче присваивается порядковый 
номер $\hat n_{(\hat{k})}$. Кроме того,  становится известна величина  
$\mathbf{Z}_{\hat n{(\hat{k})}}$~--- общее чис\-ло фрагментов данных, которые необходимо 
будет просмотреть при условии, что среди них нет уникального.

Для каж\-до\-го $z_{n(k)}$, $n \hm\in \mathcal{N}_k(t)$, $k \hm= \overline{1,K}$, и каж\-до\-го~$m$ 
определим переменную $\nabla_{n(k)}^m (t )$~---  число единиц времени (единичных временн$\acute{\mbox{ы}}$х 
интервалов), в течение которых задание     $z_{n(k)}$  планируется  выполнять на всех 
работоспособных ЕВ-мо\-ду\-лях      $m$-го типа,    $m \hm= \overline{1, M}$, после  момента~$t$. 
При определении ДСО для вновь поступивших заданий рассматривается  наихудший случай: ни одна из 
вновь полученных и ни одна из находящихся в обработке задач не имеет решения. Другими словами, 
для окончания всех имеющихся заданий необходимо будет  просмотреть все имеющиеся фрагменты данных. 
Следовательно, для заданий, обработка которых уже начата, будет необходимо просмотреть
\begin{multline*}
\mathbf{Z}_{n(k)} - z^-_{n(k)}(t) =  \sum\limits_{ m = 1}^{M} p_k^m(t) R^m(t) \nabla _{n(k)}^m (t )\,,\\
n \in \mathcal{N}_k(t)\,,\ k = \overline{1, K}\,,
\end{multline*}
фрагментов, а для вновь поступившего задания~---
\begin{multline*}
\textbf{Z}_{\hat n(\hat{k})} =  \sum\limits_{ m = 1}^{M} p_k^m(t) R^m(t)  
\nabla_{\hat n{(\hat{k})}} ^m (t )\,,  
\\
\nabla _{n(k)}^m (t ) \ge 0\,,\ \nabla _{\hat n(\hat k)} ^m (t ) \ge 0\,,\ 
m = \overline {1, M}\,,\ n \in \mathcal{N}(t)\,. 
\end{multline*}

Введем переменную $\nabla ^m (t )$:
\begin{multline*}
\nabla ^m (t ) = \sum\limits_{k = 1}^K \ \sum_{n \in \mathcal{N}_k(t) } \nabla _{n(k)} ^m (t ) + 
 \nabla _{\hat n{(\hat{k})}}^m (t )  = {}\\
{}= \sum\limits_{n \in \mathcal{N}(t) } \nabla _{n(k)} ^m (t ) + \nabla _{\hat n{(\hat{k})}}^m (t )\,,
m = \overline {1, M}\,.
\end{multline*}

Значение $\nabla ^m (t )$ показывает, сколько единиц времени после момента~$t$ потребуется 
ЕВ-мо\-ду\-лям $m$-го типа для завершения всех задач, имеющихся в СВС, при условии, что ни 
одна из них не имеет решения и необходимо  просмотреть все данные всех заданий из $\mathcal{Z}(t)$.

В ПК-плане в качестве оценки ДСО для вновь поступившего задания  $z_{\hat n_{(\hat{k})}}$ 
используется решение  задачи \textit{быстродействия}:

\smallskip

\noindent
в момент $t$ для заданных $\mathbf{Z}_n$, $\mathbf{Z}_{\hat n{(\hat{k})}}$, 
$z^-_{n(k)}(t), \mathcal{N}(t)$,  $R^m(t)$, $d_j$ найти
\begin{equation*}
 \delta^* =\min\limits_{\delta, \nabla} \delta
%\label{e6=mal}
\end{equation*}
при условиях:
\begin{enumerate}[(1)]
\item ДСО для вновь поступившего задания будет  не меньше, чем у находящихся в обработке:
\begin{equation*}
\delta \ge d_j\,,\ j \in \mathcal{N}(t)\,;
%\label{e7-mal}
\end{equation*}
\item ДСО не меньше гарантированного времени завершения всех заданий на всех ЕВ-модулях:
\begin{equation*}
\delta \ge t + \sum\limits_{n \in \mathcal{N}(t) } \nabla _{n(k)}^m (t) +  \nabla _{\hat n{(\hat{k})}}^m 
(t)\,,\ m = \overline {1, M}\,; 
%\label{e8-mal}
\end{equation*}
\item для выполнения всех заданий необходимо будет просмотреть все имеющиеся в момент~$t$ данные
\begin{multline*}
\hspace*{-5mm}\mathbf{Z}_{n(k)} - z^-_{n(k)}(t) =  \displaystyle\sum\limits_{ m = 1}^{M} p_k^m(t) R^m(t) \nabla _{n(k)}^m (t)\,,\\
\!\nabla _{n(k)}^m (t ) \ge 0\,,  \ m = \overline {1, M}\,,\ n \in \mathcal{N}_k(t);\ 
k = \overline{1, K};
\end{multline*}

\vspace*{-18pt}

\begin{multline*}
\mathbf{Z}_{\hat n(\hat{k})} =  \sum\limits_{ m = 1}^{M} p_{\hat{k}}^m(t) R^m(t)  
\nabla_{\hat n{(\hat{k})}} ^m (t )\,, \\
\nabla_{\hat n(\hat k)}^m (t ) \ge 0\,,\   m = \overline {1, M}\,.
\end{multline*}
\end{enumerate}

Полученное $\delta ^*$ является гарантированной оценкой ДСО для вновь поступившей 
задачи. Администратор сообщает значение $\delta ^*$ пользователю, который предоставил 
данную задачу. Если пользователя не устраивают прогнозируемые сроки завершения задания, 
то он отказывается от обработки и снимает задачу со счета. В~противном случае  
задание $z_{\hat n(\hat{k})}$ помещается в общий пакет $\mathcal{Z}(t)$ с условием, 
что его обработка должна быть завершена до момента $d_{\hat n{(\hat{k})}} \hm= \delta ^*$.

\section{Гарантированные оценки потерь процессорного времени}

Рассмотрим проблему диспетчеризации citu-за\-да\-ний с фиксированными ДСО. 
В~рамках М-мо\-де\-ли предполагается, что в случае превышения ДСО произведенные 
вычислительные затраты записываются в производственные потери, которые администратору 
предписывается свести к минимуму.
На первый взгляд описанный выше способ назначения ДСО для всех заданий, принятых на обработку, 
гарантирует их окончание в срок. Однако в реальности в СВС могут происходить сбои, изменение 
приоритетов в обслуживании заявок и~др., что неизбежно приведет к нарушению ДСО.  
В~М-мо\-де\-ли в момент~$t$ анализируются достаточные условия завершения заданий  в 
срок и вычисляются гарантированные оценки возможных превышений ДСО в наихудшем случае.

В соответствии с принятыми ранее обозначениями $d_j$~ --- ДСО задачи~$z_j$. 
Каж\-до\-му   $d_j$, $j \hm\in \mathcal{N}(t)$, ставится  в соответствие
множество (список) номеров  $\mathcal{N}(t, d_j)$  всех заданий~$z_i$ из
$\mathcal{Z}(t)$, таких что каж\-дое~$z_i$ требуется завершить не позднее~$d_j$, т.\,е.
$$
\mathcal{N}(t, d_j) = \{ i \ \vert  \ d_i \le d_j\,, \  i \in  \mathcal{N}(t) \}\,,\ j \in \mathcal{N}(t)\,.   
$$
Для получения гарантированных оценок рас\-смот\-рим наихудший случай: пусть ни одна из 
задач  $z_n$,\linebreak\vspace*{-12pt}

\pagebreak

\noindent
 $n \hm\in \mathcal{N}(t)$,  не имеет решения. Для фиксированного момента времени 
$t + \Delta(t)$  введем дополнительный параметр управления
$\nabla _{n(k)}^m (t \hm+  \Delta(t))$~---  число единиц времени (единичных временных интервалов), 
в течение которых задание     $z_n$       может обрабатываться на всех работоспособных ЕВ-мо\-ду\-лях      
$m$-го типа,    $m \hm= \overline{1, M}$, начиная с  $t \hm+ \Delta(t)$.

В рассматриваемом наихудшем случае задача~$z_n$  не имеет решения, поэтому для завершения 
задания  необходимо будет перебрать все  данные
\begin{equation}
z^+_n(t + \Delta(t)) = \mathbf{Z}_{n(k)} - z_{n(k)}^-(t) - w_{n(k)}(\Delta(t))\,,
\label{e10-mal}
\end{equation}
 оставшиеся необработанными для  $z_n$ к моменту $t \hm+ \Delta(t)$.
Здесь управление  $w_{n(k)}(\Delta(t))$~--- число фрагментов данных для~$z_n$,  которые  
в момент~$t$ планируется просмотреть за время  $\Delta(t)$.

Таким образом, для оставшихся фрагментов  в момент $t \hm+ \Delta(t)$ должно выполняться равенство:
\begin{multline}
z^+_n(t + \Delta(t)) = \displaystyle\sum\limits_{m = 1}^M  p^m_k(t)  R^m(t)  \nabla _{n(k)}^m (t 
+ \Delta(t))\,,\\
 n \in \mathcal{N}_k(t)\,, k = \overline{1, K}\,,
 \label{e11-mal}
\end{multline}
и  условие неотрицательности:
$$
\nabla _{n(k)}^m (t + \Delta(t)) \ge 0\,,\ m = \overline{1, M}\,,\ n \in \mathcal{N}(t)\,.
$$

В момент $t + \Delta(t)$    рассмотрим дополнительную переменную
\begin{multline}
\nabla^m (t + \Delta(t), d_j) = {}\\
\vspace*{-3mm}{}=\sum\limits_{k = 1}^K  
\sum\limits_{n \in \mathcal{N}(t, d_j) \bigcap \mathcal{N}_k(t)} \nabla _{n(k)} ^m (t + \Delta(t)) = {}\\
{} = \!\!\!\!\sum\limits_{n \in \mathcal{N}(t, d_j)}\!\!\!\! \nabla _{n(k)}^m (t + \Delta(t))\,, \ \ m = \overline{1, M}\,, 
\ \  j \in \mathcal{N}(t).\!\!
\label{e12-mal}
\end{multline}
Значение $\nabla ^m (t + \Delta(t), d_j)$ показывает, сколько единиц времени после  
момента $t \hm+ \Delta(t)$  потребуется ЕВ-мо\-ду\-лям  $m$-го типа для выполнения всех 
заданий $z_n$, $n \hm\in \mathcal{N}(t, d_j)$, которые должны быть  завершены до наступления~$d_j$. 
В~случае, \mbox{когда}\linebreak ни одна из задач не имеет решения,  величина\linebreak   $\nabla ^m (t + \Delta(t), d_j)$~--- 
точная верхняя оценка временн$\acute{\mbox{ы}}$х затрат на обработку всех $z_n$, $n \hm\in \mathcal{N}(t, d_j)$, с помощью работоспособных  
ЕВ-мо\-ду\-лей $m$-го типа, $m \hm= \overline{1, M}$. 
В~момент  $t$ достаточные условия соблюдения всех ДСО в наихудшем случае можно записать в виде:
$$
 t + \Delta(t) + \nabla^{m}(t + \Delta(t), d_j) \le   d_j\,,  \   j \in \mathcal{N}(t)\,, \ m = \overline {1,M}\,.
 $$

Чтобы получить гарантированные оценки относительных величин возможного превышения  ДСО для $z_n$,   введем переменные:

\noindent
\begin{multline}
\omega^m(t + \Delta(t) , d_j) = {}\\
{} = \fr{t + \Delta(t) +  \nabla^{m}(t +  \Delta(t), d_j) - t^0_j}{d_j -  t_j^0 }\,, \\ 
j \in \mathcal{N}(t)\,, \ 
m = \overline {1,M}\,,
\label{e13-mal}
\end{multline}
где $ t_j^0$~--- время поступления $z_j$ в СВС.  Конкретное значение $\omega^m(t \hm+ \Delta(t), d_j)$  
характеризует возможность соблюдения ДСО следующим образом:
\begin{enumerate}[(1)]
\item если существуют управления $w_n( \Delta(t))$, $\nabla _{n(k)}^m (t \hm+ \Delta(t))$ и 
значение $\nabla ^m (t \hm+ \Delta(t), d_j)$, удовле\-тво\-ря\-ющие~(\ref{e10-mal})--(\ref{e13-mal}), 
при которых  $\omega^m (t \hm+ \Delta(t), d_j) \hm\le 1$ для всех 
$j \hm\in \mathcal{N} (t)$, $m\hm = \overline {1, M}$,
то все задания можно гарантированно завершить до наступления их ДСО даже в наихудшем случае;

\item если же при любых управлениях $w_n(\Delta(t))$, $\nabla _{n(k)}^m (t \hm+ \Delta(t))$ и значениях 
$\nabla^m (t \hm+ \Delta(t), d_j)$ в~(\ref{e10-mal})--(\ref{e13-mal})  хотя бы для одного сочетания~$m$  и~$j$ величина 
$\omega^m (t \hm+ \Delta(t), d_j) \hm> 1$, то   может реализоваться ситуация,  при которой хотя бы одно задание 
в наихудшем случае завершится после назначенного (предписанного) срока.
\end{enumerate}

На самом деле максимальная величина $\omega^m (t \hm+ \Delta(t), d_j)$ является своего рода индикатором, 
который указывает, располагает ли  СВС на момент  $t$  необходимой мощностью для того, чтобы обработать 
все имеющиеся разнородные задания до назначенных ДСО в наихудшем случае.
Для проверки достаточных условий выполнения ДСО в момент~$t$ и получения 
гарантированных оценок максимально  возможного их превышения в рамках М-модели введем параметр управления  $ \omega(t)$:
$$
\omega (t) \le 1 -  \omega^m(t + \Delta(t), d_j)\,, \ \ m = \overline {1, M}\,,\ j \in \mathcal{N}(t)\,. 
$$

Величина $\omega (t)$ характеризует максимальное относительное значение возможного превышения ДСО.

\section{Формирование  пакета  текущих работ}

Рассмотрим возможный способ формирования ТР-пакета, при котором в процессе выполнения citu-за\-да\-ний 
требуется добиться высокой эффективности использования СВС и минимизировать производственные потери, в 
данном случае за счет соблюдения ДСО.   В рамках М-модели для анализа эффективности и получения гарантированных 
оценок превышения ДСО в момент времени $t$ решается следующая задача оптимизации:

\smallskip

\noindent
для заданных $\mathbf{Z}_n$, $z^-_{n(k)}(t)$, $N(t)$,  $R^m(t)$, $d_j$, $\mathcal{N}(t)$, $P^*(\Delta(t))$ найти
\begin{equation}
\Phi^* = \max\limits_{\gamma, \omega, \nabla, w, r, W} (c_1\gamma(t) + c_2 \omega (t) + c_3 \omega ^{\Sigma} (t)) 
\label{e14-mal}
\end{equation}
при ограничениях:
\begin{enumerate}[(1)]
\item на эффективность использования  разнотипных ЕВ-модулей  в СВС
\begin{equation}
\hspace*{-10mm}0 \le \gamma (t) P^*(\Delta(t))  =  \sum\limits_{k = 1}^K \sum\limits_{m = 1}^M p^m_k(t)  r_k^m(t)  \Delta(t);
\label{e15-mal}
\end{equation}

\item на суммарную величину отклонения от заданных ДСО
\begin{equation}
\left.  
\begin{array}{rl}
\hspace*{-6mm}\displaystyle\omega^{\Sigma} (t ) &\displaystyle\le \sum\limits_{m = 1}^{M} \sum\limits_{j \in \mathcal{N}(t)} 
\fr{1-\omega^m(t+\Delta(t),d_j)}{d_j};\\[9pt] 
\hspace*{-6mm}\omega^{\Sigma} (t )& \le 0;
                    \end{array}
                    \right \}
                    \label{e16-mal}
                    \end{equation}

\item на максимальную величину отклонения от заданных ДСО
\begin{equation}
 \left.  \begin{array}{rl}
\omega (t) &\le 1 -  \omega^m(t + \Delta(t), d_j)\,,\\[9pt]
& \hspace*{10mm}j\in \mathcal{N}(t)\,, \ m = \overline{1,M}\,;\\[9pt]
\omega (t) &\le 0\,;
                    \end{array}
                    \right \} 
                    \label{e17-mal}
                    \end{equation}

\item на балансовые соотношения выполнения заданий после $t \hm+ \Delta(t)$
\begin{equation}
\left.  
\begin{array}{rl}
\displaystyle\mathbf{Z}_{n} - z^-_{n}(t) - w_{n} (\Delta(t)) &={}\\
&\hspace*{-33mm}{}= \displaystyle\sum\limits_{m = 1}^M  p^m_k (t)  R^m( t ) \nabla_{n(k)} ^m (t + \Delta(t))\,,\\[9pt]
 &\hspace*{-20mm}n \in \mathcal{N}_k(t)\,, \ \ k = \overline {1, K}\,;\\[9pt]
&\hspace*{-45mm}\nabla _{n(k)}^m (t + \Delta(t)) \ge 0\,, \ n\in \mathcal{N}(t)\,, \ m = \overline {1,M}\,;
                    \end{array}
                    \right \} \!\!
                    \label{e18-mal}
\end{equation}

\vspace*{-12pt}

\noindent
\begin{multline}
\hspace*{-8mm}\nabla^m (t + \Delta(t), d_j)  = \sum\limits_{n \in \mathcal{N}(t, d_j)} \nabla _{n(k)} ^m (t + \Delta(t))\,,\\[9pt]
m = \overline {1, M}\,,\  j \in \mathcal{N}(t)\,;
                    \label{e19-mal}
                    \end{multline}
                    
                    \vspace*{-12pt}
                    
                    \noindent
\begin{multline} 
\hspace*{-8mm}\omega^m (t + \Delta(t), d_j ) = {}\\[9pt]
{}=\displaystyle\fr{ t + \Delta(t) + \nabla^{m}(t + \Delta(t), d_j) - t^0_j}{d_j - t^0_j }\,,\\[9pt]
j \in \mathcal{N}(t)\,, \ m = \overline {1,M}\,; 
                    \label{e20-mal}
                    \end{multline}

\item на управления в операционном окне $\Delta(t)$:
\begin{itemize}
\item[(а)] суммарное число СЭВ-опе\-ра\-ций $k$-го вида, планируемых  для ТР-пакета в момент~$t$:
\begin{equation}   
 \left.  \begin{array}{c}
 \hspace*{-10mm}\displaystyle\sum\limits_{n \in \mathcal{N}_k(t)} w_n (\Delta(t)) = W_{k} (\Delta(t))\,,  \ k = \overline {1,K}\,;\\[9pt]
  \hspace*{-10mm}0 \le  w_{n} (\Delta(t)) \le \mathbf{Z}_n - z_{n}^- (t)\,, \ n\in \mathcal{N}(t)\,;
                    \end{array}
                    \right \}  \label{e21-mal}
                    \end{equation}

\item[(б)] на число ЕВ-мо\-ду\-лей, выделяемых  для обработки фрагментов для подзаданий $k$-го вида в момент $t$:
\begin{equation} 
\left.  \begin{array}{c}
 \hspace*{-13mm}\displaystyle W_{k} (\Delta(t)) \le \sum\limits_{m = 1}^{M} p_{k}^m (t) r_{k}^m(t)  \Delta(t)\,, \ k = \overline {1,K}\,;\\[9pt]
 \hspace*{-13mm}r_k^m (t) \ge 0\,, \ \ k = \overline {1, K}\,, \ \ m = \overline{1,M}\,; \\[9pt]
 \hspace*{-13mm}\displaystyle\sum\limits_{k = 1}^{K} r_{k}^m (t) \le R^m(t)\,, \ m = \overline{1,M}\,.
                    \end{array}
                    \right \} \!\! \label{e22-mal}
                    \end{equation}
                    \end{itemize}
                    \end{enumerate}

Задача~(\ref{e14-mal}) может быть решена стандартными методами линейного
программирования~\cite{Dan}. Поскольку переменные $\gamma (t)$ и
$\omega(t)$ являются относительными, безразмерными и изменяются в
ограниченном диапазоне, то с помощью выбора значений $c_1, c_2, c_3$
можно учитывать текущие приоритеты при выполнении заданий. Например,
если в момент времени~$t$ во главу угла ставится эффективность и при
этом необходимо обработать как можно больше заданий, следует выбрать
$c_1 \hm\gg c_2 \hm> c_3$. Если  в момент~$t$ в множестве $\mathcal{Z}(t)$ много
заданий, которые необходимо завершить до наступления их ДСО, то
следует выбрать $c_1 \ll c_2 \hm\simeq c_3$ и вновь решить~(\ref{e14-mal})--(\ref{e22-mal}).

\section*{Заключение}

Решение задачи~(\ref{e14-mal})--(\ref{e22-mal}) в условиях неопределенности помогает
оперативно планировать выполнение разнородных ресурсоемких заданий в
гетерогенной вычислительной системе. Очевидно, что при управлении
СВС можно рассмотреть и другие диспетчерские политики. Однако
описанная выше расчетная схема концептуально проста и позволяет
оценивать производственные потери и добиваться эффективного
использования СВС. Полученные на практике результаты показали, что
предлагаемый подход является хорошей основой для диалога
администратора и пользователей.

{\small\frenchspacing
{%\baselineskip=10.8pt
\addcontentsline{toc}{section}{Литература}
\begin{thebibliography}{99}

\bibitem{Prep11} 
\Au{Козлов М.\,В., Малашенко Ю.\,Е., Назарова~И.\,А. и~др.} 
Анализ режимов управления вычислительным комплексом в условиях неопределенности.~--- М.: ВЦ РАН, 2011.

\bibitem{Sour}  
{Sourcebook of parallel computing}.~--- San Francisco: Morgan Kaufmann Publs., 2003.

\bibitem{Kal} \Au{ Каляев И.\,А., Левин И.\,И. } 
Реконфигурируемые мультиконвейерные вычислительные структуры для решения потоковых 
задач обработки информации и управления~//  Суперкомпьютерные технологии: разработка, программирование, 
применение: Тр. Междунар. науч.-практич. конф.~-- Ростов-на-Дону: ЮФУ, 2010. Т.~1. С.~100--102.

\bibitem{RT2004} \Au{Sha L.,  Abdelzaher T.,  Arzen~K.-E., \textit{et. al}.} 
Real time scheduling theory: A~historical perspective~//  Real-Time Systems, 2004. Vol.~28.  No.\,2--3. P.~101--155.

\bibitem{Stan} \Au{Stankovic J.\,A., Spuri M., Ramamritham~K., \textit{et. al}.} 
Deadline scheduling for real-time systems, EDF and related algorithms.~--- Boston: Kluwer,  1998.

\bibitem{Kon} \Au{ Коновалов М.\,Г., Малашенко Ю.\,Е., Назарова~И.\,А. } 
Управление заданиями в гетерогенных вычислительных системах~// Изв. РАН. ТиСУ, 2011. №\,2. С.~72--90.

\bibitem{Gol11} \Au{Голосов П.\,Е., Козлов М.\,В., Малашенко~Ю.\,Е. и~др. } 
Анализ управления  специализированными вычислительными заданиями в условиях неопределенности~// 
Изв. РАН. ТиСУ, 2012. №\,1. С.~50--66.

\bibitem{Mal412} \Au{Малашенко Ю.\,Е., Назарова И.\,А. } 
Модель управления разнородными вычислительными заданиями на основе 
гарантированных оценок времени выполнения~// Изв. РАН. ТиСУ. 2012. №\,4. С.~29--38.


\bibitem{Suh} \Au{Сухарев А.\,Г., Тимохов А.\,В., Федоров~В.\,В. } Курс методов оптимизации.~--- М.: Наука, 1986.

\bibitem{germ} \Au{Гермейер Ю.\,Б.} Введение в теорию исследования операций.~--- М.: Наука, 1971.

%\bibitem{gned} \Au{Гнеденко Б.\,В., Коваленко И.\,Н.} Введение в теорию массового обслуживания.~--- М.: ЛКИ, 2011.

%\bibitem{Leung} Handbook of scheduling: Algorithms, models, and performance  analysis~/ Ed. J.\,Y.-T.~Leung.~--- 
%N.Y.: Chapman \& Hall/CRC, 2004.

\label{end\stat}

\bibitem{Dan} \Au{ Данциг Дж.\,Б. } Линейное программирование, его применения и обобщения.~--- М.: Прогресс, 1966.
\end{thebibliography}
}
}

\end{multicols}