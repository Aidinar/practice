\renewcommand{\bibname}{\protect\rmfamily References}

\def\stat{zatsman}

\def\tit{INFORMATION TECHNOLOGIES FOR CREATING THE DATABASE 
OF~EQUIVALENT VERBAL FORMS IN~THE~RUSSIAN-FRENCH MULTIVARIANT 
PARALLEL CORPUS$^*$}

\def\titkol{Information technologies for creating the database 
of~equivalent verbal forms in~a~Russian-French multivariant 
parallel corpus}

\def\autkol{S.~Loiseau, D.\,V.~Sitchinava, A.\,A.~Zalizniak, and~I.\,M.~Zatsman}

\def\aut{S.~Loiseau$^1$, D.\,V.~Sitchinava$^2$, A.\,A.~Zalizniak$^3$, and~I.\,M.~Zatsman$^4$}

\titel{\tit}{\aut}{\autkol}{\titkol}

{\renewcommand{\thefootnote}{\fnsymbol{footnote}}\footnotetext[1]
{The reported study was partially supported by RFBR, research project No. 12-06-33038 and project 
No.\,13-06-00403.}}


\renewcommand{\thefootnote}{\arabic{footnote}}
\footnotetext[1]{Universit$\acute{\mbox{e}}$ Paris 13, Sorbonne Paris Cit$\acute{\mbox{e}}$, Laboratoire LDI 
(Lexiques, dictionnaires, informatique), CNRS, UMR 7187, sylvain.loiseau@univ-paris13.fr}
\footnotetext[2]{Institute of the Russian Language of the Russian Academy of Sciences, mitrius@gmail.com}
\footnotetext[3]{Institute of Linguistics of the Russian Academy of Sciences; 
    Institute of Informatics Problems of the 
Russian Academy of Sciences, anna.zalizniak@gmail.com}
\footnotetext[4]{Institute of Informatics Problems of the Russian Academy of Sciences, iz\_ipi@a170.ipi.ac.ru}


\Abste{The Russian-French parallel corpus as a part of the Russian National Corpus (RNC) is 
being transformed into a multivariant corpus with several translations corresponding to 
each original texts. Concurrently, a Database of functionally equivalent lexicogrammatical verbal forms is being created using the multivariant corpus. The main 
purpose of database creation is to calculate the statistical estimates of the equivalences 
between Russian and French verbal forms. The paper discusses an information 
technology for creating the Russian-French multivariant parallel corpus and the 
database simultaneously.}

\KWE{parallel multivariant corpora; Russian National Corpus; information 
technologies; XML marking up Russian-French parallel texts; lexicogrammatical 
form; functional equivalence; statistical estimates of equivalences}


\vskip 14pt plus 9pt minus 6pt

      \thispagestyle{headings}

      \begin{multicols}{2}

            \label{st\stat}


\section{Introduction}

  \noindent
   Creating databases for cross-linguistic studies is a time-consuming process, but it is justified by 
the richness of linguistic data, including information on lexical units and grammatical structures 
they provide. Data can enter into such databases from off-line sources or digital resources such as 
parallel corpora~\cite{1-zat}. The use of parallel corpora to create databases for cross-linguistic and 
contrastive studies allows to save time for actual research~[2--5]. Contrastive study is very 
productive for linguistic representations design. Functional motivation of cross-linguistic models 
for machine translation and parallel texts alignment was presented in the\linebreak works~[6--9].
   
   The parallel corpora within the RNC have been developed since 
2005~\cite{2-zat, 3-zat}. At the present time, within the RNC, there are eight bilingual parallel 
corpora, with Russian as the source language and eight other languages (English, German, French, 
Spanish, Italian, Polish, Ukrainian, Belarusian) as target languages, and one multilingual parallel 
corpus. The RNC includes Russian-French and French-Russian parallel corpora with a single 
translation for each textual pair. In 2013, the Russian-French parallel corpus will be transformed 
into a multivariant corpus with one Russian original and its several French translations. 
The multivariant corpus of parallel texts will be formed on the basis of the preexisting monovariant 
parallel corpus. In this corpus the sentence-alignment association between an original text and 
translated texts should be maintained.
   
   The purpose of this paper is to outline the research questions related to the creation of the 
\textit{Database of functionally equivalent lexicogrammatical verbal forms} using the 
   Russian-French multivariant parallel corpus. One of the main objectives of the database creation 
is to obtain the statistical estimates of different types of equivalences between Russian and French 
verbal forms. Now, there is a need of an information technology for creating the Russian-French 
multivariant parallel corpus and the database simultaneously.
   
   The information technology for creating the corpus and the database should support the 
following basic functions:
   \begin{itemize}
\item alignment of the parallel texts with several variants of translation;
\item morphological annotation and lemmatization of the parallel texts\footnote[5]{For the 
solution of task~2 and 3 it is planned to use software of the Russian National Corpus.};
\item including the parallel texts with several variants of translation into the corpus;
\item constructing the database of the equivalent lexicogrammatical verbal forms 
(Russian-French);
\item calculating the statistical estimates of these equivalences between Russian and French 
verbal forms.
\end{itemize}

   The main research question is how to align and mark up the parallel texts in order to construct 
equivalences between Russian and French verbal forms. In this paper, the authors outline their experience 
with the development of information technologies for creating the Russian-French multivariant 
parallel corpus and the database simultaneously.

\section{Different Translations of~the~Same Text: Methodological Tasks}

\noindent
A framework for aligning several French translations of the same Russian text within 
the French-Russian parallel corpora is suggested. The Russian prose has been translated into French for more 
than 150~years, and the translations of different periods greatly vary as to their accuracy, choice of 
lexicon and grammar, and style. A~comparison of these translations can be a valid task per se. 
Alongside with this, lexical and grammatical studies performed on parallel corpora may use 
alternative options of translations, envisaging all possible correspondences. They can reflect 
objective variability in the target language and can be a valuable resource for contrastive 
grammatical description of both languages. For instance, one can expect to observe greater 
variability amongst the translations when there are structural differences between the two languages 
(a Russian ``signifi$\acute{\mbox{e}}$'' with no clear correspondence in the French lexicon is 
more likely to exhibit variation in its translations). Several fruitful avenues of research are opened:
\begin{itemize}
     \item  it may provide valuable insight into phenomena of short diachronical variation: some 
variation in the translation of the same source texts may be related to diachronic variation in the 
target language;
\item  when lexical and grammatical values differ between the two languages, the inventory of the 
choices made by the translators may help uncovering the values of the elements translated. For 
instance, how the Russian aspectual system is reflected into the French system? Which are the 
various options, and how are they conditioned by the context? This line of research will bring new 
elements for a contrastive grammar of Russian and French; and
     \item the variability of the translations may help discern the intricated parameters~--- cultural, 
stylistic, lexical, and grammatical~--- at work in the translation process, since local choice may be 
interpreted in the wider context of the choices made by the translator.
     \end{itemize}
     
   Since the corpus is built on texts for which several translations are available, the corpus will rely 
mainly on the classical Russian fiction writers of the last two centuries. It will be based on the 
larger Russian-French corpora previously built for the RNC, which also contains law texts, 
subtitles, and newspapers. In the literary classics, special attention will be given to some axes of 
variation, in particular, the ones of diachrony and genres (novel, short story, children's literature).
   
   Quantitative methodologies for contrastive grammatical description based on aligned corpora 
are being elaborated. The quantitative methods may be used for identifying the most frequent cases 
of variability. If there are some well known cases of discrepancies between the Russian and the 
French linguistic systems, a systematic outline of the importance of variation in the translation 
according to grammatical and lexical features is still to be done and may prove to be a valuable 
resource. This quantitative investigation will provide an empiric basis for building a typology of 
translation discrepancies.
   
   Quantitative analyses may also be used for analyzing the regular correspondences between the 
two linguistic systems. When a variability of translation is observed for a given lexical or 
grammatical feature, statistical analysis will allow for discovering which contextual features are 
correlated with each variants. These contextual features help describing the values of each variant. 
Factorial analysis, and, in particular, Correspondences analysis, is well suited for this task of 
identifying the stable groups of contextual features across numerous instances of variants. Statistical 
analysis is used as a mean for providing a summary of numerous translations under scrutiny. While 
each single context is of no use for drawing conclusions and while it is also impossible to figure 
out the big picture when dealing with numerous collected contexts, statistical analysis is useful for 
mapping variants.



\section{Equivalences Between Russian and French Verbal Forms}

 \noindent
   The entry of the database is an \textit{equivalence}, i.\,e., a pair of functionally equivalent 
Russian and French forms. Within the information technology for creating the Russian-French 
parallel corpus, two types of equivalences should be distinguished:
   \begin{itemize}
\item  ``translation models''--- the set of French translations $\{F_1, \ldots , F_n\}$ for a Russian 
verbal form~$R$;
\item ``translation stimuli''~--- the set of Russian ``stimuli'' $\{R_1, \ldots , R_n\}$, to which a 
French form~$F$ is a ``reaction.''
\end{itemize}
   
   The equivalences of the first type (including their statistical characteristics) provide relevant 
information on the structure of the Russian language, whereas the equivalences of the second type 
specify characteristic features of the French language. Since the main object of the  interest is 
Russian, on the first stage, only the equivalences of the first type, i.\,e., the 
``translation models,'' have been taken into consideration. 
 A~further expansion of the database is possible in the direction of including 
data from the parallel French-Russian corpus; they will answer the question: ``What are the French 
`stimuli' that provoke as `reaction' such or such a Russian form?''
   
   In the created database, the cluster of grammatical categories known in typological studies as 
TAM (Tense-Aspect-Mood, see~[10--14]) has been explored. The following restrictions have been
accepted: only 
finite verbal forms were included and the category of voice and other divergences due to the 
conceptualization modes were not consider. The set of TAM values for the languages in question is generally well 
established (although there is a number of controversial issues). The list was created on the basis of 
the existing Russian and French grammars and the researches on Russian grammar, including 
corpus-based ones~[15--22].
   
   The conceptual category of modality is generally considered to be composed of two parts, viz. 
objective and subjective modality. Objective modality is represented by the grammatical category 
`mood' and enters the TAM cluster. The set of values of the category of subjective modality 
constitutes another cluster (Mod): it was elaborated especially for the purposes of creating the 
database on the basis of grammatical descriptions~\cite{15-zat, 16-zat, 23-zat, 24-zat}.
   
   The notion ``lexical-grammatical form'' (LGF) was introduced; it is understood as a 
combination of grammatical features (TAM for Russian and Tense-Mood for 
French) or ascribed to a class of lexemes or even to an individual lexeme (cf. the term 
``construction'' in Construction Grammar~[25--27]). For an \textit{equivalence} as a database 
entry, the set of Russian LGFs constitutes the Source domain, the set of French LGFs~--- the Target 
domain. 
   
   A database entry is conceived as having a three-level structure. The upper level is designed to 
search for information objects that is called ``hyperequivalences,'' the medium level~--- search 
for ``polyequivalences,'' and the lower one~--- for ``monoequivalences.''


   
   \textit{Monoequivalence} is a binomial tuple (an ordered pair) of the form $\langle \mathrm{Rn}(i); 
\mathrm{Fm}(j)\rangle$, where the first position is occupied by the $i$th occurrence of an LGF Rn. Each 
occurrence $\mathrm{Rn}(i)$ is provided with its ``address'' in the Russian part of the database (all 
occurrences of each form, in both Russian and French parts of the database,
are indexed). The 
second position is occupied by the $j$th
occurrence of the French LGF~Fm, equivalent to 
$\mathrm{Rn}(i)$
\end{multicols}
\begin{table}[b]\small
\vspace*{-24pt}
\begin{center}
\Caption{Monoequivalence types and examples of Russian sentences and their French translations}
\vspace*{2ex}

\begin{tabular}{lp{56mm}p{50mm}}
\hline
\multicolumn{1}{c}{Monoequivalence type}&\multicolumn{1}{c}{Russian 
sentence}&\multicolumn{1}{c}{French translation}\\
\hline
$\langle$Fut-PF; Pr$\rangle$&Я {\bfseries\textit{вернусь}} через 10 минут;\newline
Что я ему {\bfseries\textit{скажу}}?&Je {\bfseries\textit{reviens}} dans 10 minutes;\newline 
Qu'est ce que je lui {\bfseries\textit{dis}}?\\
$\langle$Pres-PF; Pr$\rangle$&не {\bfseries\textit{пойму}}&Je n'y {\bfseries\textit{comprends}} 
rien\\
\hline
$\langle$Past-PF(= ``fut''); Pr$\rangle$&Ну я {\bfseries\textit{пошел}}&Eh bien, je 
{\bfseries\textit{pars}}\\
$\langle$Past-PF; Pr$\rangle$&Он только что {\bfseries\textit{вышел}} &Il {\bfseries\textit{sort}} 
$\grave{\mbox{a}}$ l'instant\\
\hline
$\langle$Imperat; Pr$\rangle$&А ты {\bfseries\textit{помолчи!}}&Toi, tu {\bfseries\textit{te 
tais}}!\\
$\langle$Fut-PF; Pr$\rangle$&Если я его {\bfseries\textit{увижу}}, я ему это скажу&Si je le 
{\bfseries\textit{vois}}, je le lui dis\\
$\langle$Fut-PF; Pr$\rangle$&Если я его увижу, я ему это {\bfseries\textit{скажу}}&Si je le vois, je le lui 
{\bfseries\textit{dis}}\\
$\langle$Pr; Imparf$\rangle$&Он понял, что она его {\bfseries\textit{любит}}&Il a compris 
qu'elle l'{\bfseries\textit{aimait}}\\
\hline
$\langle$Pst-PF: PqParf$\rangle$&Он понял, что она его {\bfseries\textit{любила}}&Il a compris 
qu'elle l'{\bfseries\textit{avait aimait}}\\
$\langle$Conj; Imparf$\rangle$&{\bfseries\textit{если бы я знал}}, я бы сказал&si 
{\bfseries\textit{je savais}}, je dirais\\
\hline
$\langle$Fut-PF; FutAnt$\rangle$&я {\bfseries\textit{закончу}} через 
час&{\bfseries\textit{J'aurai fini}} dans une heure \\
\hline
$\langle$Fut-PF; PasCom$\rangle$&Если в понедельник ничего от меня не 
{\bfseries\textit{получишь}}, возвращайся&Si lundi {\bfseries\textit{tu n'as}} rien 
{\bfseries\textit{\mbox{re\/{\normalsize\ptb{$\!\!\!\!\;$\/\c{c}}}u}}} de moi, 
reviens\\
\hline
$\langle$Vfin, \textit{что} Past-PF; Vfin\;+\;InfPas$\rangle$&сожалею, {\bfseries\textit{что не 
сказал}} &je regrette {\bfseries\textit{de n'avoir pas dit}}\\
\hline
$\langle$Vfin \textit{что/как}; Acc.c.Inf$\rangle$&Я {\bfseries\textit{видел, что}} он 
вернулся&je {\bfseries\textit{l'ai vu revenir}}\\
\hline
$\langle$Vfin\;+\;Inf; Vfin\;+\;Inf$\rangle$&Майор Ковалев {\bfseries\textit{имел 
обыкновение}} каж\-дый день {\bfseries\textit{прохаживаться}} по Невскому проспекту&Le 
major Kovaliov {\bfseries\textit{avait l'habitude de se promene}}r chaque jour dans l'avenue 
Nevski\\
\hline
$\langle$Vfin\;+\;Inf; Vfin$\rangle$ &Он {\bfseries\textit{начал щупать}} рукою, чтобы узнать: 
не спит ли он? &$\mbox{\ptb{\bfseries\textit{il t\^{a}ta}}}$ avec la main, se 
pin\!\!{\normalsize\ptb{\c{c}}}a 
pour se convaincre qu'il ne dormait pas\\
\hline
$\langle$Vfin\;+\;DET; Vfin\;+\;Inf$\rangle$&Послушай, голубушка, ~--- 
{\bfseries\textit{говорил}} он {\bfseries\textit{обыкновенно}}, встретивши на улице бабу, 
продававшую манишки&$\acute{\mbox{E}}$coute un peu, ma ch$\grave{\mbox{e}}$re, 
{\bfseries\textit{avait}}-il {\bfseries\textit{coutume de dire}} lorsqu'il rencontrait une de ces 
femmes qui vendent des chemises dans les rues\\
\hline
$\langle$Vfin\;+\;DET; Vfin\;+\;Inf$\rangle$&Бедный Ковалев {\bfseries\textit{чуть не сошел с 
ума}}&Le pauvre Kovaliov {\bfseries\textit{faillit devenir fou}}\\
\hline
\end{tabular}
\end{center}
\end{table}

\begin{table}[h]\small
\begin{center}
\Caption{Polyequivalence type and an example of Russian sentence and two French translations}
\vspace*{2ex}

\begin{tabular}{lp{48mm}p{48mm}}
\hline
\multicolumn{1}{c} {Polyequivalence type}&\multicolumn{1}{c} {Russian 
sentence}&\multicolumn{1}{c} {French translations}\\
\hline
$\langle$Conj;\{CondPas, VDisp(CondPas)\;+\;Inf\}$\rangle$
&То есть Иван Яковлевич {\bfseries\textit{хотел бы и того и другого}}, но знал, что было 
совершенно невозможно требовать двух вещей разом, ибо Прасковья Осиповна очень не 
любила таких прихотей.
&\textit{First translation}: \ldots Ivan Iakovl$\acute{\mbox{e}}$vitch {\bfseries\textit{se serait volontiers 
r$\acute{\mbox{e}}$gal$\acute{\mbox{e}}$}} de caf$\acute{\mbox{e}}$ et de pain frais 
[$\ldots$]\newline
\textit{Second translation}: $\ldots$Ivan {\bfseries\textit{aurait bien voulu 
go{\!\!\ptb{\^{u}}}ter}} de l'un et de 
l'autre [$\ldots$]\\
\hline
\end{tabular}
\end{center}
\vspace*{6pt}
\end{table}

\begin{multicols}{2}

\noindent 
  in one of the French translations. Table~1 contains monoequivalence types, 
and examples 
of Russian sentences and their French translations (some of them are taken from comparative 
Russian-French grammars~[28--30]).





   
   \textit{Polyequivalence} is a binomial tuple of the form $\langle \mathrm{Rn}(i); 
   \{\mathrm{Fm}(j), \mathrm{Fk}(r), 
\ldots\}\rangle$, which represents a  combination of several monoequivalences with the identical 
first position ($\langle \mathrm{Rn}(i); \mathrm{Fm}(j)\rangle$, 
$\langle \mathrm{Rn}(i); \mathrm{Fk}(r)\rangle$, etc.), i.\,e., the set of 
French translations of the given Russian form in the given Russian sentence. So, for the $i$th 
occurrence of the Russian LGF~Rn the French form~Fm appears as its functional equivalent in 
the first translation, the form~Fk appears in the second French translation, and so on.  The 
index~$j$ for the form~Fm indicates that the $i$th entry of the form~Rn corresponds to the $j$th 
occurrence of the form~Fm in the first French translation, index~$r$ denotes that to the $i$th 
occurrence of the Russian LGF, Rn corresponds to the $r$th occurrence of the French form Fk 
in the second translation, and so on; cf.\ a polyequivalence type and an example of Russian sentence 
and two French translations in Table~2.
{\looseness=1

}


   
   \textit{Hyperequivalence} is a binomial tuple of the form $\langle \mathrm{Rn}; \{F\}\rangle$, which 
comprises one Russian LGF and a multitude of its functionally equivalent French LGFs which enter 
into one or more monoequivalences with~Rn: $\langle \mathrm{Rn}(i); \mathrm{Fm}(j)\rangle$, 
$\langle \mathrm{Rn}(i); 
\mathrm{Fk}(r)\rangle$, etc. So, a hyperequivalence represents an aggregation of all the possible translation 
equivalents, i.\,e., the \textit{functionally equivalent fragments}\footnote{The notion ``functionally 
equivalent fragment~--- FEF'' was introduced in~\cite{2-zat} for resolving the tasks of computer linguistics 
within the domains of contrastive lexicology and bilingual lexicography.} of a given Russian form that 
exist in the present database. It is important to note that if a new monoequivalence type appears in the 
database (when a new translation or a new pair of parallel texts enter into the corpus), the content of 
the corresponding hyperequivalence is automatically updated. 



   
   Three-level structure of the database serves to provide the possibility of a cross-identification on 
all three levels, each of them being sources of relevant linguistic data. The database structure is 
designed in a way  to allow separate statistical estimates of mono- and polyequivalences.

\section{Information Technologies for~Database Creating}

\noindent
On the basis of the database architecture described above, a technology of developing the database 
using the parallel Russian-French texts was set up. The technology includes the following stages:
\begin{itemize}
\item selecting multivariant parallel texts (multiple French translations for each single Russian 
text);
\item optical character recognition (OCR) for the French texts with not available digitalized version;
\item aligning and XML (eXtensible Markup Language) marking (see section~5);
\item metatextual alignment;
\item morphological annotation of the texts and uploading them into the corpus (to provide an 
access to the corpus online via a search engine);
\item creating the \textit{Database of functionally equivalent lexicogrammatical verbal forms} 
using the multivariant parallel Russian-French corpus.
   \end{itemize}
   
   On the stage of selecting multivariant parallel texts the existing translations of Russian fiction 
into the French language are analyzed. For the multivariant corpus, only texts that exist in not less 
than two French versions are selected. Moreover, these translations should be created not earlier 
than in the middle of the XX century. The earlier French translations, and particularly those created in 
the XIX century, are obsolete linguistically and, which is even more important, include a great deal 
of mistakes and abridgements.
   
   Many Russian literary works (by Gogol, Goncharov, Leo Tolstoy, Dostoevsky and others) exist 
in different modern French versions translated by skilful experts. They exhibit both different 
approaches to translating fiction and different selection of linguistic means including those relevant 
to the grammar.
   
   While the texts are digitalized, the French translations are scanned, recognized, and checked (the 
OCR process). The Russian original texts are normally already available in a digital version (some 
of them are included in the monolingual RNC). The recent French translations have no public 
digital versions. To process pocket books, they are unstitched and then scanned on a workstation 
with simultaneous recognition by OCR software. A~recognized text is compared to its printed 
version.
   
   The alignment algorithm provides a separate alignment for each pair of a text and its translation. 
After that, the markup expert checks and corrects the results of alignment and moves, merges, or 
separates sentences in order to make the chunks of the translated text correspond the respective 
chunks of the original text.
   
   The same pattern is followed for each new translation, and afterwards, all the translation pairs 
are merged automatically into a multivariant array of three or more texts. The resulting texts are 
saved into an XML file. The technology of alignment and markup is described in the next section of 
this paper in more detail.
   
   At the stage of the metatextual markup of the corpus, the expert adds a piece of relevant 
metatextual information to each aligned text. This information consists of the following fields:
   \begin{itemize}
\item the name of the author in the original (source) language;
\item the title of the text in the source language;
\item the date of the author's birth;
\item the title of the text in any of the translations;
\item the name of the author in the target language of the translation (as specified in each 
translation);
\item the names of all the translators in the target language;
\item the dates of all the translations; and
\item the indication of the source and target languages.
\end{itemize}
   
   At the stage of the morphological markup the French and Russian texts are automatically 
marked with means of the morphological analyzers for these two languages. Each word is tagged by 
a set of grammatical features. The parallel texts with a grammatical markup are saved into an XML 
file and uploaded to the corpus.
   
   To grant an access to the corpus online, the aligned texts with metatextual and morphological 
markup are made searchable within the multivariant Russian-French parallel corpus. The parallel 
texts are made available via a search interface that supports the standard search functions of the 
RNC by lexical and/or grammatical features and their combinations. A~subcorpus can be also 
customized by metatextual attributes, the texts can be searched within a subcorpus; 
and the query result can be sorted and exported.
   
   As it has been already stated, the database is built on the basis of the bilingual multivariant 
aligned texts of the corpus. The main \textit{raison d'{\!\ptb{\!\^e}}tre} of this database is the 
possibility of a bilingual grammatical search for mono- and polyequivalences and finding statistical 
estimates for their types composed of Russian and French LGFs listed in Table~3.





   The methods used are based on a formalized description of each monoequivalence type 
$\langle \mathrm{Rn}; \mathrm{Fm}\rangle$ (see Table~1). This description consists in the 
ascribing to each LGF Rn 
and its Fm a corresponding combination of grammatical features, which appear during the 
morphological markup, when both Russian and French texts are automatically morphologically 
analyzed. The database is built in multiple iterations. During each iteration,
one search query is built 
in terms of Russian and French grammatical features of a monoequivalence type $\langle \mathrm{Rn}; 
\mathrm{Fm}\rangle$. Each search query is implemented in bilingual multivariant aligned texts of the corpus.
   
   The following monoequivalences from the corpus (\textit{Скучная 
история}\,/\,\textit{L'histoire banale} by Anton Chekhov) may be cited as an example:
   
   $\langle\mathbf{Pres}\mbox{-}\mathbf{Ipf;Pr}\rangle$ corresponds to the grammatical tags 
praes,ipf (present imperfective) in Russian and praes (pr$\acute{\mbox{e}}$sent) in French:
   
Все это и многое, что еще можно было бы сказать, \underline{составляет} 
(praes,ipf\;=\;Pres-Ipf) то, что 
\underline{называется} (praes,ipf\;=\;Pres-Ipf) моим именем.

Tout cela, et beaucoup d'autres choses que l'on pourrait ajouter, \underline{constitue}
(praes\;=\;Pr) ce qu'on 
\underline{appelle} (praes\;=\;Pr) ``mon nom.''

   $\langle\mathbf{Past}\mbox{-}\mathbf{Pf;PasCom}\rangle$ corresponds to the tags praet 
(past) and pf (perfective) in the Russian text, and the auxiliary verb \textit{avoir} or 
\textit{{\!\ptb{\!\^{e}}}tre} in present (praes) plus the passive participle (partcp, pf) in French (the 
compound tense \textit{pass$\acute{\mbox{e}}$ compos$\acute{\mbox{e}}$}):
   
   Наука, слава богу, \underline{отжила} (praet,pf\;=\;Past-Pf) свой век,~--- говорит Михаил 
Федорович с расстановкой.~--- Ее песня уже спета. Да-с.
   
   La science \underline{a fait} (avoir,praes\;+\;partcp,pf\linebreak =\;PasCom) son temps, Dieu merci, dit 
Fiodorovitch en d$\acute{\mbox{e}}$tachant les mots. N-i-ni, c'est fini! Oui!
   
   The retrieved fragments of the multivariant aligned texts are analyzed in order to identify and 
mark all monoequivalences of the type $\langle \mathrm{Rn}; \mathrm{Fm}\rangle$ in these fragments. After the search 
of all types of monoequivalences from Table~1 is completed, in the corpus, there remain unmarked the 
equivalences that are absent in this table. The search for those equivalences is performed using 
the queries including grammatical features only of the Russian LGFs listed in Table~3. On this 
stage, all types of monoequivalences, listed in Table~1 which contain the searched Russian 
LGFs are\linebreak\vspace*{-12pt}

\pagebreak

\end{multicols}

\begin{table}\small
\begin{center}
\Caption{Lists of LGFs in Russian and in French selected for database creation}
\vspace*{2ex}

\begin{tabular}{ll}
\hline
\multicolumn{1}{c}{Identifier}&\multicolumn{1}{c}{Verbal form, comments}\\
\hline 
\multicolumn{2}{c}{\textbf{Russian}}\\
\hline
Pres-IPF&Present\\
Pres-PF&Present Perfective (e.\,g., \textit{не припомню})\\
Past-IPF&Past Imperfective\\
Past-PF&Past Perfective\\
Fut-PF&Simple Future\\
Fut-IPF&Compound Future\\
Imperat-PF&Imperative Perfective\\
Imperat-IPF&Imperative Imperfective \\
Conj-PF&Subjunctive Perfective\\
Conj-IPF&Subjunctive Imperfective\\
Past\;+\;\textit{bylo}&Cancelled Result/Frame Past (e.\,g., \textit{хотел было, пошел было}), 
``Russian Pluperfect,'' cf. \cite{31-zat}\\
Vfin\;+\;Inf&Embedded infinitive construction\\
Vfin\;+\;DET&Conctruction with a modal determinant \\
Dupl(V-IPF)&Implicit conative (e.\,g., дед {\bfseries\textit{бил, бил}}~--- не разбил)
 cf.~\cite{32-zat}\\
\hline
\multicolumn{2}{c}{\textbf{French}}\\
\hline
Pr&Pr$\acute{\mbox{e}}$sent\\
PasCom&pass$\acute{\mbox{e}}$ compose\\
PasSim&pass$\acute{\mbox{e}}$ simple\\ 
Imparf&imparfait \\
PqParf&plus-que-parfait\\
PasAnt&pass$\acute{\mbox{e}}$ ant$\acute{\mbox{e}}$rieur\\
PasIm&pass$\acute{\mbox{e}}$ imm$\acute{\mbox{e}}$diat\\
Fut&Futur\\
FutAnt&futur ant$\acute{\mbox{e}}$rieur\\
FutIm&futur imm$\acute{\mbox{e}}$diat\\
Imperat&imp$\acute{\mbox{e}}$ratif\\
SubjPres&subjonctif pr$\acute{\mbox{e}}$sent\\
SubjPas&subjonctif pass$\acute{\mbox{e}}$\\
SubjImparf&subjonctif imparfait\\
SubjPqParf&subjonctif plus-que-parfait\\
CondPr&conditionnel pr$\acute{\mbox{e}}$sent\\
CondPas&conditionnel pass$\acute{\mbox{e}}$\\
Vfin\;+\;Inf&embedded infinitive conctruction\\
Vfin\;+\;InfPas&embedded past infinitive construction\\
Vfin\;+\;Acc.c.Inf&Accusativus cum infinitive (e.\,g., \textit{je la vois venir})\\
Vfin\;+\;Nom.c.Inf&Nominativus cum infinitivo (e.\,g., \textit{elle para{\!\!\ptb{\^{
\i}}}t comprendre})\\
\hline
\end{tabular}
\end{center}
\vspace*{-6pt}
\end{table}

\begin{multicols}{2}

\noindent
 excluded. If during the analysis of the found fragments of aligned texts new types of 
monoequivalences appear, they are added to Table~1. After completing
 the search, all marked 
monoequivalences are automatically grouped into polyequivalences according to their Russian 
LGFs $\mathrm{Rn}(i)$.

   
   After the mono- and polyequivalences' search is over, this database can be used for calculating 
type-frequencies. The frequency of monoequivalences of the type $\langle \mathrm{Rn}; \mathrm{Fm}\rangle$ are 
found by the formula:

\vspace*{2pt}

\noindent
$$
\mathrm{Freq}\left(t, n, m\right) = \fr{100 Q(t, n, m)}{Q(t)}
$$

\vspace*{-2pt}

\noindent
where $\mathrm{Freq}(t, n, m)$ is the percentage of type 
$\langle \mathrm{Rn}; \mathrm{Fm}\rangle$ 
 at the time moment~$t$; $Q(t, n,  m)$ is the number of monoequivalences of the type 
 $\langle \mathrm{Rn}; \mathrm{Fm}\rangle$ 
at the moment of~$t$; and $Q(t)$ is 
the overall number of monoequivalences of all the types in 
the database at the moment of~$t$.
   
   The frequency of polyequivalences of the type with two FEFs (functionally equivalent fragments)
$\langle \mathrm{Rn}; \{\mathrm{Fm}, \mathrm{Fk}\}\rangle$ which is 
called the second-order type is calculated according to the following formula: 
$$
\mathrm{Freq}(t, n, m, k) = \fr{100 Q(t, n, m, k)}{Q(t)}
$$
where $\mathrm{Freq}(t, n, m, k)$ is the percentage of second-order type 
$\langle \mathrm{Rn}; \{\mathrm{Fm}, \mathrm{Fk}\}\rangle$ at 
the moment of~$t$; $Q(t, n, m, k)$ is the number of polyequivalences of the type $\langle \mathrm{Rn}; 
\{\mathrm{Fm}, \mathrm{Fk}\}\rangle$ at the moment of~$t$; and $Q(t)$ is the overall number of polyequivalences of all 
the second-order types in the database at the moment of~$t$.

   The three- and $n$-order polyequivalences are also determined. Their frequencies are found by 
analogous formulae with the variable of~$t$, and there should be $n+1$ indices in the formula for 
the $n$th order of polyequivalences. All the type-frequencies depend on~$t$, as the corpus and the 
database are constantly updated. Consequently, the frequencies of all the mono- and polyequivalence 
types can fluctuate accordingly.
   
\section{Aligning and XML Marking}

\noindent
The technology of developing the database described above includes aligning and XML marking 
processes. The XML markup currently in use allows for building bi- or polylingual corpora with 
multiple translations of original texts (including the option of multiple translations into the same 
language); presently, a multilingual corpus featuring different other Slavic languages is available for 
search within the RNC.

   The parallel texts in the RNC are aligned sentence-by-sentence. 
   A lot of texts kindly offered for 
the use in the RNC by Adrian Barentsen and included into his  multilingual Amsterdam
Slavic Parallel Aligned Corpus (\mbox{ASPAC}) are 
already aligned paragraph-by-paragraph~\cite{33-zat}. This segmentation has been additionally 
refined by the present authors semiautomatically, introducing boundaries between the sentences 
within these predefined paragraphs. For the majority of the texts,
the alignment is made completely by the RNC 
team. The original text's sentence segmentation overrides the segmentation in the translation. So, 
each single element of parallel tagging corresponds to the original sentence, whilst its translated 
counterpart(s) may be either a part of a sentence starting with a space or a comma or more than 
one sentence. 
   
   Multiple alignment tools have been used for the RNC parallel corpora. It is evident that the 
procedure of alignment consists of two stages: introducing sentence boundaries into texts and the 
alignment in the narrow sense of the word. There exist programs that do not have an embedded 
sentence-splitting algorithm (HunAlign by Andras Farkas, {\sf 
http://mokk.bme.hu/resources/hunalign}; LeoBilingua by Leonid Brodsky, {\sf 
www.hot.ee/bclogic/}) and those who enable sentence-boundaries markup like TextAlign 
(http://www.englishelp.ru/soft/soft-for-translator/151-textalign.html) or \textit{Parallelnye Texty}, a 
program developed for the RNC by A.\,A.~Kretov's team in the Voronezh University and used for 
markup of some English-Russian and German-Russian texts.
   
   The algorithm of breaking the text into sentences is straightforward in programs of both types; it 
uses the punctuation marks like exclamation mark, quotation mark, and full stop without taking into 
consideration the initial letters, abbreviation, quotation, and parenthesis marks or the rules of direct 
speech (for problems in using TextAlign in an Ukrainian-Polish parallel corpus, see~\cite{34-zat}). 
The segmentation in programs of both types can be corrected manually, although the algorithm 
itself cannot be corrected, and some general mistakes are to be treated each time they occur. In the 
TextAlign program, additionally, the automatic sentence-breaking is obligatory, and one cannot 
escape it by creating a standalone program for this purpose. For LeoBilingua and HunAlign, this is 
the only option, and it is possible to elaborate rules of sentence-breaking and change them as the 
new texts bring new challenges.
   
   The alignment proper for all the four tools is automatic with a possible manual control. This is 
further divided, however, into two possible modes~--- step-by-step (with corrections possible in the 
middle of the consequent alignment) and total alignment with postcorrection. The first approach is 
embraced by LeoBilingua and Parallelnye teksty, while the other is chosen by TextAlign and 
HunAlign. The last two programs, therefore, call for a rereading of an already aligned text with 
correction of the wrongly aligned sentences. While in TextAlign a GUI interface is provided for this 
(however a single correction calls for realigning of the whole text), in HunAlign, only a manual 
editing of the output file is possible.
   
   The choice of parallel sentences may be additionally controlled for the point of view of sentence 
length and/or lexical contents; this feature is supported by LeoBilingua (one sentence should not be 
twice or more longer than the other) and HunAlign that uses a statistical mechanism evaluating the 
probability of a good alignment using the sentence length and, optionally, a bilingual dictionary. If 
the evaluation count in HunAlign is below zero, the alignment is usually mismatched; these places 
are to be corrected manually.
   
  So, LeoBilingua and HunAlign seem to be the best choice for the RNC and both are 
used in it currently, both allowing for user-defined sentence splitting and using statistical 
mechanisms of alignment. Both have their advantages. While LeoBilingua allows for a slow well 
controlled process, with a possibility to split sentences manually in all tricky places and correct 
possible text misprints in a GUI (graphical user interface), sending the results directly into a Unicode XML file, HunAlign 
aligns the whole text quickly with very few mismatches, marked and easily discerned. The latter is 
currently used by Ruprecht von Waldenfels in ParaSol, a project close to the tasks of the RNC 
parallel corpora (earlier aka Regensburg Parallel Corpus, {\sf http://www-korpus.uni-r.de/ParaSol/}, 
see~\cite{35-zat}). However, the material of Slavic corpora offers some challenges, including the 
dictionary problems, as the languages with rich inflection demand including most forms of the 
paradigm into the dictionaries used in alignment.
   
   In order to deal with some of the issues pointed above, another strategy is currently explored. It 
involves using a part-of-speech tagger for annotating the corpora in both languages before the 
alignment step. A~part-of-speech tagger such as TreeTagger~\cite{36-zat} is available for both 
Russian and French languages and provides, beside categorization, sentence-breaking as well as 
lemmatization. The sentence-breaking is based on linguistic rules and is more elaborated that 
\textit{ad-hoc} scripts mentioned above. However, it may be used only with alignment tools 
allowing for external sentence-breaking. The lemmatization helps reducing the morphological 
diversity: each original sentence (made of inflected forms) can be transformed into a sentence made 
of lemma. The task of aligning sentences of lemma is easier, since a bilingual dictionary of lemma 
is easier to produce. Once aligned, sentences of lemma can, of course, be replaced with original, 
inflected sentences.
   
   While the quality and accuracy of the alignments rely heavily on the quality and richness of the 
bilingual lexicon, it was found that a simple, word-by-word general lexicon was not available for 
Russian and French. Fortunately, the HunAlign program may produce, after an alignment task, the 
bilingual lexicon it built in memory while aligning the sentences. The acquisition of a bilingual 
lexicon was thus an incremental feedback process: the first lexicon produced by HunAlign has been 
corrected; in turn, it allows for a better alignment in a new run of HunAlign and, thus, new and 
better suggestions of word pairs were produced by HunAlign. These word pair suggestions were 
incorporated in the lexicon, entailing a new run of HunAlign.

\section{Format and~Morphological Tagging}

\noindent
     The parallel texts in the RNC are presented in an XML format where sentences are paired by 
the $\langle\mathrm{para}\rangle\langle/\mathrm{para}\rangle$ tag. Each sentence has an attribute 
indicating the language (this may be changed when bi- or polylingual texts are inaugurated). If a 
sentence is not translated, an omission is marked by three dashes.
     
   The texts are automatically annotated using the morphological analyzers designed by Yandex 
search engine. The lexical and grammatical annotation is included into 
$\langle\mathrm{ana}\rangle\langle/\mathrm{ana}\rangle$ tag. The tags are not currently 
disambiguated: however, some Russian texts selected for parallel corpora are already manually 
disambiguated for the monolingual corpus and may be later included into parallel corpora as well.
     
     There should be also a possibility for tagging translation discrepancies, i.\,e., adding of new text 
or omitting of the source fragments in the translated text. This kind of discrepancies is not a rare 
thing in translating fiction. They should be taken into consideration in every serious linguistic study 
based on parallel texts as the translation accuracy can never be taken for granted. These 
discrepancies are also a valid object of study in their own right, concerning the theory of translation 
as such.
     
     Let quote an example of aligned sentences in the Russian-French parallel corpora in XML. 
Note the tags for loose translation: ``add,'' ``omit,'' and ``change'' mark, respectively, meaningful 
change of the source as rendered into the target text.
     
     
     $\langle$para id\;=\;``292''$\rangle$
     
     \hspace*{2mm}$\langle$se lang\;=\;``ru''$\rangle$ Нужно знать, что одно значительное 
лицо недавно сделался значительным лицом, \textit{а до того времени он был 
незначительным лицом}.$\langle$/se$\rangle$
     
     \hspace*{2mm}$\langle$se lang\;=\;``fr'' loose\;=\;``omit''$\rangle$ Il y a lieu d'indiquer que 
ce personnage n'$\acute{\mbox{e}}$tait devenu important que depuis peu;$\langle$/se$\rangle$
     
     $\langle$/para$\rangle$
     
     $\langle$para id\;=\;``105''$\rangle$
     
     \hspace*{2mm}$\langle$se lang\;=\;``ru''$\rangle$ Рисунок: рожа Момуса. 
$\langle$/se$\rangle$
     
     \hspace*{2mm}$\langle$se lang\;=\;``fr'' loose\;=\;``add''$\rangle$ Un dessin 
\textit{repr$\acute{\mbox{e}}$sentant} la trogne \textit{hilare} du \textit{dieu} Momus, 
$\langle$/se$\rangle$
     
     $\langle$/para$\rangle$
     
     $\langle$para id\;=\;``338''$\rangle$
     
     \hspace*{2mm}$\langle$se lang\;=\;``ru''$\rangle$ Он позволял это себе потому, что 
чувствовал в себе силу всегда, когда ему понадобится, опять \textit{выделить одно служебное 
и откинуть человеческое}.$\langle$/se$\rangle$
     
     \hspace*{2mm}$\langle$se lang\;=\;``fr'' loose\;=\;``change''$\rangle$ Il le faisait seulement 
parce qu'il se sentait de force $\grave{\mbox{a}}$ \textit{r$\acute{\mbox{\textit{e}}}$tablir} 
$\grave{\mbox{a}}$ n'importe quel moment \textit{les barri{\!\ptb{\!\`{e}}}res fatidiques}.$\langle$/se$\rangle$
     
     $\langle$/para$\rangle$
       
   The parallel texts are made available for search online at the 
   {\sf www.ruscorpora.ru} website. Due to 
copyright reasons, no text is available for full view; the search results are always presented in the 
form of separate sentences with minimal context (so-called ``snippets''). The following parameters 
are searchable:
   \begin{itemize}
\item any combination of lemmata, exact word forms, and morphological tags within a 
10-word combination (e.\,g., ``avoir'' in the Imparfait tense\;+\;Past Participle yields the French 
Pluperfect); and
\item names of the author and the translator, language of the original text, language of the 
translation text.
\end{itemize}
   
   These parameters are available by selecting a subcorpus for further textual or grammatical 
search. New software is under preparation, providing for semiautomatic alignment of more than two 
translation texts. The result should be an XML document with multiple French phrases (marked as 
belonging to a given translation) corresponding to each Russian phrase of the source text. By the 
end of 2013, the first ``multivariant'' (as they are called by the team) aligned texts will be included 
into the Russian-French parallel corpus.
   
\section{Concluding Remarks}

\noindent
   A user working with a multivariant parallel corpus has as his/her typical task retrieving all the 
occurrences of an original LGF and its FEFs in all the translation variants present in the corpus. To 
solve such tasks, a technology is set up for creating a database of binomial tuples 
$\langle\mathrm{LGF}; \{\mathrm{FEF}\}\rangle$ based on the multivariant corpus. This database 
should offer an opportunity to search LGFs and their respective FEFs in parallel texts.
   
   The main difference between this database and the parallel corpus itself is that the former 
enables the search by LGFs, FEFs, and the types of mono- or polyequivalences (see Tables~1--3). 
The corpus can be searched only by the grammatical features of words. The query results 
for the corpus consist of the aligned text fragments including the requested grammatical features. 
The query results for the database consist of the LGFs, FEFs, mono-, polyequivalences and their 
types, as well as the fragments of the parallel texts including them. Moreover, the database allows 
for building hyperequivalences based on the retrieved monoequivalences.
   
   While creating the database, the expert has to select and to mark up the monoequivalences 
relevant for the monoequivalence type in question. This selection is personalized in the database by 
the expert's ID. The grammatical features used by him/her are also recorded in the database. 
Currently, the selection of the monoequivalences relevant for the type in question is done by one 
expert only. The present authors' plans provide that later multiple experts should work collaboratively, and the 
results of their collaboration should be processed using the methods and models of coordinating 
personal expert knowledge~[37--40].

{\small\frenchspacing
{%\baselineskip=10.8pt
\addcontentsline{toc}{section}{Литература}
\begin{thebibliography}{99}

\bibitem{1-zat}
The use of databases in cross-linguistic studies~/ Eds. M.~Everaert, S.~Musgrave, A.~Dimitriadis.~--- 
Belin--New York: Walter de Gruyter GmbH \& Co., 2009.
\bibitem{2-zat}
\Au{Dobrovolsky D.\,O., Kretov A.\,A., Sharoff~S.\,A.} Corpus of parallel texts~// Scientific and 
Technical Information. Ser.~2: Information Processes and Systems, 2005. No.\,6. P.~16--27.
\bibitem{3-zat}
\Au{Dobrovolsky D.\,O., Kretov A.\,A., Sharoff~S.\,A.} Corpus of parallel texts: Architecture and 
usage~// Russian National Corpus: 2003--2005.~--- Moscow: Indrik, 2005.  P.~263--296. [In Russian.]
\bibitem{4-zat}
\Au{Andreeva E.\,G., Kasevich V.\,B.} Grammar and lexicon in the English-Russian corpus of 
parallel texts~// Russian National Corpus: 2003--2005.~--- Moscow: Indrik, 2005.  P.~297--307.
[In Russian.]
\bibitem{5-zat}
\Au{Dobrovolsky D.\,O.} A~Corpus of parallel texts and studying culture-specific lexicon~// 
Russian National Corpus: 2006--2008. New results and prospects.~--- St.\ Petersburg: 
Nestor-Istoriya, 2009.  P.~383--401.
\bibitem{6-zat}
\Au{Kozerenko E.\,B.} Cognitive approach to language structure segmentation for machine 
translation algorithms~// Conference (International) on Machine Learning, Models, Technologies 
and Applications Proceedings.~---  Las Vegas, USA: CSREA Press, 2003. P.~49--55.
\bibitem{7-zat}
\Au{Kozerenko E.\,B.} Linguistic filters in statistical machine translation models~// Informatics 
and Applications, 2010. Vol.~4. No.\,2. P.~83--92.

\bibitem{9-zat} %8
\Au{Kozerenko E.\,B.} Syntactic transformations modelling for hybrid machine translation~// 
ICAI'11, WORLDCOMP'11 Proceedings.~--- Las Vegas, Nevada, USA:  CRSEA Press, 2011. P.~875--881.

\bibitem{8-zat} %9
\Au{Kozerenko E.\,B.} Parallel texts alignment strategies: The semantic aspects~// Informatics and 
Applications, 2013. Vol.~7. No.\,1.  P.~82--89.

\bibitem{10-zat} %10
\Au{Comrie B.} Aspect. An introduction to the study of verbal aspect and related problems.~---
Cambridge: Cambridge Univ. Press, 1976.
\bibitem{11-zat}
\Au{Comrie B.} Tense.~--- Cambridge: Cambridge Univ. Press, 1985.
\bibitem{12-zat}
\Au{Dahl $\ddot{\mbox{O}}$}. Tense and aspect systems.~--- Oxford: Blackwell, 1985.

\bibitem{14-zat} %13
\Au{Bybee J.\,L., Perkins~R., Pagliuca~W.} The evolution of grammar: Tense, aspect and modality 
in the languages of the world.~--- Chicago: University of Chicago Press, 1994.

\bibitem{13-zat} %14
Tense and aspect in the languages of Europe~/ Ed. $\ddot{\mbox{O}}$.~Dahl.~--- 
Belin\,--\,New York: Mouton de Gruyter, 2000.

\bibitem{15-zat}
Grammar of contemporary Russian.~--- 
In 2~vols.~--- Moscow, 1954. [In Russian.]
\bibitem{16-zat}
Russian grammar~/ Ed.\ N.\,Ju.~ Shvedova.~--- In 2~vols.~---Moscow, 
1980. [In Russian.]

\bibitem{18-zat} %17
\Au{Gak V.\,G.} Theoretical grammar of 
French.~--- Moscow, 2000. [In Russian.]

\bibitem{17-zat} %18
\Au{Vinogradov V.\,V.}
Russian language.~--- Moscow, 2001.  [In Russian.]

 
\bibitem{19-zat}
Russian National Corpus: 2003--2005.~--- Moscow: Indrik, 2005. 

\bibitem{20-zat}
Russian National Corpus: 2006--2008.~--- St.\ 
Petersbourg: Nestor-Istorija, 2009. [In Russian.]
 
\bibitem{21-zat}
Corpus studies on Russian grammar.~--- 
Moscow: Probel-2000, 2009. [In Russian.]
\bibitem{22-zat}
Construction linguistics~/ Ed. E.\,V.~Rakhilina.~--- Moscow: Azbukovnik, 
2010. [In Russian.]
\bibitem{23-zat}
\Au{Bondarko A.\,V., Beliaeva E.\,I., Biriulin~L.\,A., \textit{et al}.} 
The theory of functional 
grammar. Temporality. Modality.~--- Leningrad: Nauka, 1990. [In Russian.]
\bibitem{24-zat}
\Au{Paducheva E.\,V.} Modality.  {\sf http://rusgram.ru}.
\bibitem{25-zat}
\Au{Goldberg A.} Constructions: A construction grammar approach to argument structure.~--- 
Chicago: Univ. of Chicago Press, 1995.

\bibitem{27-zat} %26
\Au{Tomasello M.} Constructing a language. A~usage-based theory of language acquisition.~--- 
Cambridge--London: Harvard Univ. Press, 2003.
\bibitem{26-zat} %27
\Au{Goldberg A.} Constructions at work. The nature of generalization in grammar.~--- Oxford: 
Oxford Univ. Press, 2006.

\bibitem{28-zat}
\Au{Gak V.\,G.} Russian language compared to 
French.~--- Moscow: URSS, 2006. [In Russian.]



\bibitem{30-zat} %29
\Au{Kouznetsova I.\,N.} Grammaire contrastive du fran{\!\ptb{\c{c}}}ais et du russe.~--- M.: 
Nestor Academic Publs., 2009.

\bibitem{29-zat} %30
\Au{Gak V.\,G.} Comparative typology 
of French and Russian.~--- Moscow: URSS, 2010. [In Russian.]

\bibitem{31-zat}
\Au{Sitchinava D.\,V.} Seek to prevent at the root: Russian construction with \textit{bylo}: 
corpus-based analysis~// Corpus studies on 
Russian grammar.~--- Moscow: Probel-2000, 2009. P.~362--396.
\bibitem{32-zat}
\Au{Plungian V.\,A., Rakhilina~E.\,V.} Tushat-tushat~--- ne potushat: Grammar of one verb 
construction~// Construction linguistics~/ Ed.\ E.\,V.~Rakhilina.~--- 
Moscow: Azbukovnik, 2010. P.~83--94.
\bibitem{33-zat}
Information on ASPAC~--- Amsterdam Slavic Parallel Aligned Corpus. {\sf 
http://www.uva.nl/over-de-uva/\linebreak 
organisatie/medewerkers/content/b/a/a.a.barentsen/\linebreak  a.a.barentsen.html}.
\bibitem{34-zat}
\Au{Kotsyba N.} The current state of work on the Polish-Ukrainian Parallel Corpus (PolUKR)~// 
Problems of Slavic Lexicography:  Workshop (International) within MONDILEX Project 
Proceedings.~--- Kyiv, 2009. {
\sf http://www.domeczek.pl/$\sim$natko/papers/NKotsyba\_\linebreak Kyiv2009.pdf}.
\bibitem{35-zat}
\Au{Von Waldenfels R.} Compiling a parallel corpus of slavic languages. Text strategies, tools and 
the question of lemmatization in alignment~// Beitr$\ddot{\mbox{a}}$ge der 
Europ$\ddot{\mbox{a}}$ischen Slavistischen Linguistik (POLYSLAV)~/ Eds.\ B.~Brehmer,  
V.~Zdanova,  R.~Zimny.~--- M$\ddot{\mbox{u}}$nchen, 2006.  P.~123--138 (available at: 
{\sf 
http://www-nw.uni-\linebreak 
regensburg.de/\%7E.war05297.slavistik.sprachlit.uni-regensburg.de/pub/WaldenfelsParallelCorpora2006.\linebreak
pdf}).
\bibitem{36-zat}
\Au{Schmid H.} Probabilistic part-of-speech tagging using decision trees~// Conference 
(International) on New Methods in Language Processing Proceedings~/ Ed.\  D.~Jones.~--- 
Manchester, UK: UMIST, 1994. P.~44--49.
\bibitem{37-zat}
\Au{Zatsman I.} Time-dependent semiotic model of computer coding of concepts, information 
objects and denotata~// Informatics and Applications, 2009. Vol.~3. No.\,4. P.~87--101.
\bibitem{38-zat}
\Au{Zatsman~I.,  Durnovo~A.} Modelling of processes for creation of expert knowledge for 
monitoring of goal-oriented programme activities~// Informatics and Its Applications, 2011. Vol.~5. 
No.\,4. P.~84--98.
\bibitem{39-zat}
\Au{Zatsman I.} Denotatum-based models of knowledge creation for monitoring and evaluating 
R\&D program implementation~// 11th IEEE Conference (International) on Cognitive Informatics 
and Cognitive Computing Proceedings.~--- Los Alamitos, CA: IEEE Computer Society Press, 2012. 
P.~27--34.
\bibitem{40-zat}
\Au{Zatsman I.} Tracing emerging meanings by computer: Semiotic framework~// 13th European 
Conference on Knowledge Management Proceedings.~--- Reading: Academic Publishing 
Intern. Ltd., 2012. Vol.~2. P.~1298--1307.
\end{thebibliography}

}
}


\end{multicols}



\vspace*{3pt}

\hrule


\def\tit{ИНФОРМАЦИОННЫЕ ТЕХНОЛОГИИ СОЗДАНИЯ 
БАЗ ДАННЫХ ЭКВИВАЛЕНТНЫХ ГЛАГОЛЬНЫХ ФОРМ 
В~РУССКО-ФРАНЦУЗСКОМ ПОЛИВАРИАНТНОМ ПАРАЛЛЕЛЬНОМ КОРПУСЕ}

\def\aut{С.~Луазо$^1$, Д.\,В. Сичинава$^2$, А.\,A.~Зализняк$^3$, И.\,M.~Зацман$^4$}

\titelr{\tit}{\aut}

%\vspace*{2pt}

\noindent
$^1$Университет Париж-13, лаборатория лексики, словарей и информатики НЦНИ,\\
$\hphantom{^1}$sylvain.loiseau@univ-paris13.fr\\
\noindent
$^2$Институт русского языка РАН, mitrius@gmail.com\\
\noindent
$^3$Институт ИЯз РАН; ИПИ РАН, anna.zalizniak@gmail.com\\
\noindent
$^4$ИПИ РАН, iz\_ipi@a170.ipi.ac.ru\\

\vspace*{-6pt}


\Abst{Русско-французский параллельный корпус как часть Национального корпуса русского 
языка в настоящее время преобразовывается в поливариантный корпус с одним оригинальным и 
несколькими вариантами переводов этого текста. Одновременно создается база данных функционально 
эквивалентных лек\-си\-ко-грам\-ма\-ти\-че\-ских глагольных форм на основе текстов поливариантного корпуса. 
Главная цель создания базы данных состоит в том, чтобы вычислить статистические оценки соответствий 
между русскими и французскими глагольными формами. В статье рассматриваются вопросы разработки 
информационной технологии для одновременного создания русско-французского поливариантного 
параллельного корпуса и этой базы данных.}

\label{end\stat}


\KW{параллельные поливариантные корпуса; Национальный корпус русского языка; 
информационные технологии; XML разметка рус\-ско-фран\-цуз\-ских параллельных текстов; 
лек\-си\-ко-грам\-ма\-ти\-че\-ские формы; 
функциональные соответствия; статистические оценки соответствий}
 
 
 
\renewcommand{\bibname}{\protect\rmfamily Литература}