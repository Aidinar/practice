

\def\stat{sinits}

\def\tit{ПАРАМЕТРИЧЕСКОЕ СТАТИСТИЧЕСКОЕ И~АНАЛИТИЧЕСКОЕ МОДЕЛИРОВАНИЕ РАСПРЕДЕЛЕНИЙ В~НЕЛИНЕЙНЫХ 
СТОХАСТИЧЕСКИХ СИСТЕМАХ НА МНОГООБРАЗИЯХ$^*$}

\def\titkol{Параметрическое статистическое и~аналитическое моделирование распределений в~нелинейных СтС}
%стохастических системах на многообразиях}

\def\autkol{И.\,Н. Синицын}

\def\aut{И.\,Н. Синицын$^1$}

\titel{\tit}{\aut}{\autkol}{\titkol}

{\renewcommand{\thefootnote}{\fnsymbol{footnote}}\footnotetext[1]
{Работа выполнена при финансовой поддержке  программы <<Интеллектуальные информационные технологии, системный анализ 
и автоматизация>> (проект 1.7).}}

\renewcommand{\thefootnote}{\arabic{footnote}}
\footnotetext[1]{Институт проблем информатики Российской академии наук, sinitsin@dol.ru}

\vspace*{3pt}

\Abst{Рассматриваются дискретные методы статистического и аналитического моделирования в нелинейных 
системах на многообразиях описываемых дифференциальными стохастическими уравнениями Ито с винеровскими 
и пуассоновскими шумами. Предполагается, что в качестве параметров распределений взяты коэффициенты 
совместных ортогональных плотностей различных порядков. Особое внимание уделено нелинейной корреляционной 
теории статистического и аналитического моделирования.}


\vspace*{3pt}

\KW{аналитическое моделирование; метод нормальной аппроксимации; метод ортогональных разложений;
метод статистической линеаризации; нелинейная корреляционная теория;
нелинейная стохастическая система Ито на многообразии; параметризация одно- и многомерных распределений;
статистическое моделирование}

\vspace*{6pt}

\vskip 14pt plus 9pt minus 6pt

      \thispagestyle{headings}

      \begin{multicols}{2}

            \label{st\stat}

\section{Введение}

Известные методы статистического и аналитического моделирования процессов в стохастических системах (СтС), 
описываемых дифференциальными стохастическими уравнениями Ито с винеров\-скими и пуассоновскими шумами, основанные 
на параметризации распределений, подробно изложены в~[1--3].

Обобщение результатов [1--3] на случай многоканальных круговых и сферических СтС выполнено в~[4--12].

Статья посвящена развитию методов параметрического статистического и аналитического моделирования 
в СтС Ито на многообразиях. 

В~разд.~2 рассмотрены уравнения СтС на многообразиях (МСтС). 

В~разд.~3 и приложении представлены приближенные методы статистического моделирования (МСМ) различной точности. 

Раздел~4 посвящен как методам аналитического моделирования (МАМ), основанным на ортогональных разложениях, так 
 и совместным МСМ и МАМ. 
 
 Нелинейная корреляционная теория МСМ и  МАМ развита в разд.~5.

\section{Уравнения непрерывных стохастических систем на~многообразиях}

Как известно~[1, 2], для дифференциальных  СтС в
конечномерных пространствах используется дифференциальное стохастическое
уравнение Ито вида
\begin{multline}
dY= a (Y,\Theta,t) dt + b(Y,\Theta,t) d W_0 +{}\\{}+ \int\limits_{R_0^q} c(Y,\Theta,t,v)\, dP^0 (t, dv)\,.\label{e2.1-sin}
\end{multline}
Здесь $Y$~--- $p$-мер\-ный вектор состояния, $Y\hm\in \Delta^y$ ($\Delta^y$~--- многообразие состояний); 
$\Theta$~--- вектор случая случайных параметров размерности $p_\theta$;
$a\hm=a (y,\theta,t)$ и $b\hm= b(y,\theta,t)$~--- известные $(p\times 1)$-мер\-ная и
$(p\times r)$-мер\-ная функции вектора $Y$ и времени~$t$; $W_0\hm= W_0(t)$~---
$r$-мерный винеровский случайный процесс интенсивности $\nu_0\hm=
\nu_0(t)$; $c(y,\theta,t,v)$~--- $(p\times 1)$-мер\-ная функция $y$, $t$ и вспомогательного
$(q\times 1)$-мер\-но\-го параметра $v$; $\int\limits_{\Delta_t}\, d P^0 (t,A)$~--- центрированная пуассоновская мера:
\begin{equation*}
\int\limits_{\Delta_t}\, dP^0 (t,A)= \int\limits_{\Delta_t} \,dP (t,A)-
    \int\limits_{\Delta_t} \nu_P (t,A)\, dt\,, %\label{e2.2-sin}
    \end{equation*}
где $\int\limits_{\Delta_t}\, dP (t,A)$~--- число скачков пуассоновского
процесса в интервале времени $\Delta$; $\nu_P (t,A)$~--- интенсивность
пуассоновского процесса $P(t,A)$; $A$~--- некоторое борелевское
множество пространства $R^q_0$ с выколотым началом координат.
Интеграл~(\ref{e2.1-sin}) в общем случае распространяется на все пространство
$R_0^q$ с выколотым началом координат.
Начальное значение~$Y_0$ вектора~$Y$ представляет
собой случайную величину, не зависящую от приращений винеровского
процесса $W_0(t)$ и пуассоновского процесса $P(t,A)$ на интервалах
времени $\Delta_t \hm= (t_1, t_2]$, следующих за $t_0$, $t_0\hm\le t_1\hm\le t_2$,
для любого множества~$A$.

В случае, когда подынтегральная функция $c(y,\theta,t,v)$ в уравнении~(\ref{e2.1-sin})
допускает представление
\begin{equation}
c(y,\theta,t,u)= b(y,\theta,t) c'(v)\,,\label{e2.3-sin}
\end{equation}
уравнение (\ref{e2.1-sin}) приводится к  виду:
\begin{equation}
{\dot Y} = a (Y,\Theta,t) +b(Y,\Theta,t)V\,,\label{e2.4-sin}
\end{equation}
если принять

\noindent
    \begin{equation*}
    W(t)= W_0(t) +\int\limits_{R_0^q} c'(u) P^0 (t, dv)\,. %\label{e2.5-sin}
    \end{equation*}

В некоторых случаях  вводят расширенный вектор состояния $\bar Y \hm=\lk Y^{\mathrm{T}} \Theta^{\mathrm{T}}\rk^{\mathrm{T}}$ 
размерности $\bar p \hm= p\hm+p^\theta$. 
Тогда уравнения~(\ref{e2.1-sin}) и~(\ref{e2.4-sin}) совместно с уравнением формирующего фильтра для~$\Theta$
    \begin{multline*}
    d\Theta= {a}^\theta (\Theta,t) dt + {b}^\theta (\Theta, t)\, dW_0 +{}\\
    {}+ \iii_{R_0^q} {c}^\theta (\Theta,t,v)\, dP^0 (t, dv) %\label{e2.6-sin}
    \end{multline*}
примут вид:

\noindent
    \begin{multline}
    d \bar Y = \bar a(\bar Y,t) dt + \bar b (\bar Y,t)\, dW_0 +{}\\
    {}+ \iii_{R_0^q} \bar c (\bar Y, t, v)\, dP_0 (t, dv)\,,\label{e2.7-sin}
    \end{multline}
    
    \noindent
где

\noindent
   \begin{gather*}
    \bar a (\bar Y,t) = \begin{bmatrix}  
    a (\bar Y,\Theta,t)\\
    {a}^\theta (\Theta,t)\end{bmatrix}\,;\quad
    \bar b (\bar Y,t) =\begin{bmatrix}
    b (\bar Y,\Theta,t)\\
    {b}^\theta (\Theta,t)\end{bmatrix}\,;
\\
 \bar c'' (\bar Y,t,v) =\begin{bmatrix}
    c (\bar Y,\Theta,t,v)\\
    {c}^\theta (\Theta,t,v)\end{bmatrix}\,. %\label{e2.8-sin}
    \end{gather*}

Аналогично в условиях~(\ref{e2.3-sin}) имеем:
\begin{equation}
\dot{\bar Y} =\bar a (\bar Y, t) +\bar b  (\bar Y, t)\,.\label{e2.9-sin}
\end{equation}

\smallskip

\noindent
{\small \textbf{Замечание 2.1.}\ В дальнейшем будем пользоваться уравнениями~(\ref{e2.7-sin}) и~(\ref{e2.9-sin}), 
опуская черту над переменными количествами.}

\smallskip

Для вычисления вероятностей событий,
связанных со случайными функциями, в прикладных задачах достаточно
знания многомерных распре-\linebreak\vspace*{-12pt}

\columnbreak

\noindent
делений. Поэтому центральной
задачей теории непрерывных СтС является задача вероятностного анализа одно- и многомерных
распределений процессов, удовлетворяющих дифференциальным стохастическим
уравнениям Ито вида~(\ref{e2.7-sin}) или~(\ref{e2.9-sin}) при
соответствующих начальных условиях.

В теории непрерывных СтС различают два принципиально разных подхода
к вычислению распределений. Первый общий подход основан на
статистическом моделировании, т.\,е.\ на прямом\linebreak чис\-лен\-ном решении
дифференциальных  стохастических уравнений~(\ref{e2.7-sin}) или~(\ref{e2.9-sin}) с
последующей статис\-тической обработкой результатов. Второй \mbox{общий}
подход основан на теории непрерывных марковских процессов и
предполагает аналитическое моделирование, т.\,е.\ решение
детерминированных уравнений в функциональных пространствах
(уравнений Фок\-ке\-ра--План\-ка--Кол\-мо\-го\-ро\-ва, Фел\-ле\-ра--Кол\-мо\-го\-ро\-ва,
Пугачева и~др.)\ для одно- и многомерных распределений. В~практических задачах часто используют и комбинированные методы. При
этом будем предполагать, что существуют  одно- и многомерные
плотности процессов в МСтС~(\ref{e2.7-sin}) и~(\ref{e2.9-sin}). Достаточные условиях их
существования можно найти, например, в~\cite{13-sin}.

Будем полагать, что, во-пер\-вых, одно- и многомерные плотности распределений существуют и, 
во-вто\-рых, плотности можно параметризовать с помощью параметров $\{\Xi_n\}$: вероятностных моментов, 
квазимоментов, семиинварианитов, коэффициентов ортогональногого разложения плотностей, канонических разложений 
и~др. В~таких случаях наряду с прямыми методами численного решения уравнений~(\ref{e2.7-sin}) 
и статистической обработки данных и оценивания $\{\Xi_n\}$  используют следующий подход. Пользуясь обобщенной 
формулой~[1--3, 13], со\-став\-ля\-ют  в силу~(\ref{e2.7-sin}) 
дополнительно дифференциальные стохастические уравнения для параметров~$\Xi_n$:
\begin{multline*}
d\Xi_n = a^{\Xi_n} (\Xi_n,t) dt + b^{\Xi_n} (\Xi_n,t) dW_0+{}\\
{}+ \iii_{R_0^q} c^{\Xi_n} (\Xi_n,t,v)\, dP^0 (t,dv)\,. %\label{e2.10-sin}
\end{multline*}
При использовании такого подхода под расширенным вектором состояния $\bar Y$ будем рассматривать вектор 
$\tilde Y \hm= \lk Y^{\mathrm{T}} \Theta^{\mathrm{T}} \Xi_n^{\mathrm{T}}\rk^{\mathrm{T}}$.

\vspace*{-9pt}

\section{Параметрическое статистическое моделирование}

\vspace*{-2pt}

Численное интегрирование  дифференциальных стохастических
уравнений имеет некоторые особенности~[2, 14--16]. Дело в том, что все чис\-лен\-ные
методы интегрирования обыкновенных дифференциальных уравнений, кроме
простейшего метода Эйлера, основаны на вычислении приращений искомых
функций на каждом шаге путем применения интегральной теоремы о среднем
значении. В~соответствии с этим правые части уравнений (производные
искомых функций) берутся в средних точках интервалов.
Различные методы численного ин\-тег\-ри\-ро\-вания отличаются один от другого, по существу,
только способом приближенного нахождения средних значений правых частей уравнений.

Предположим, что требуется заменить уравнение~(\ref{e2.1-sin}) или~(\ref{e2.4-sin})  разностным
уравнением для значений процесса $Y(t)$ в заданном дискретном ряду
равноотстоящих точек $\{t_k\}$, $t_k\hm=kh$, где $h$~--- интервал между
соседними точками~$t_k$, $h\hm=t_{k+1}\hm-t_k$. Принципиально  задача
решается точно, так как значения марковского процесса $Y(t)$ в точках
$t_k$ образуют марковскую случайную последовательность $\{\bar Y_k\}$,
$\bar Y_k\hm=Y(kh)$, а всякая марковская последовательность определяется
некоторым стохастическим разностным уравнением. Однако составить это
точное разностное уравнение по данному дифференциальному уравнению
практически невозможно. Для его составления необходимо прежде всего
найти переходное распределение марковского процесса $Y(t)$,
определяемого уравнением~(\ref{e2.1-sin}) или~(\ref{e2.4-sin}), а потом по найденному переходному
распределению можно составить разностное уравнение для
последовательности $\{\bar Y_n\}$. Но точное определение переходного
распределения процесса $Y(t)$ возможно только в некоторых частных
случаях. В~общем же случае придется довольствоваться приближенным
определением переходного распределения процесса $Y(t)$. В~результате
по этому переходному распределению можно будет получить только
приближенное разностное уравнение. Последнее делает нецелесообразным
применение чрезвычайно сложного алгоритма вывода точного разностного
уравнения для последовательности $\{\bar Y_n\}$. Раз уж все равно
приходится довольствоваться лишь приближенным разностным уравнением\linebreak
даже при использовании данного алгоритма, то\linebreak целесообразно сделать это
более простыми спосо-\linebreak бами.

Прежде всего заменим интеграл по переменной~$v$ в уравнении~(\ref{e2.1-sin})
соответствующей интегральной суммой. В~результате~(\ref{e2.1-sin})
заменится уравнением
\begin{equation}
dY=a(Y,t)dt+b(Y,t)dW_0+\sumin c_i(Y,t)dP^0_i\,,\label{e3.1-sin}
\end{equation}
где $c_i(y,t)$~--- $p$-мер\-ные векторные функции, представляющие собой
значения функции $c(y,t,v)$ в некоторых средних точках $v_i$
соответствующих элементов~$A_i$ разбиения $q$-мер\-но\-го шара достаточно
большого радиуса, $v_i\hm\in A_i$ ($i=1,\ldots,N$), а $P^0_i(t)$~---
центрированные простые пуассоновские процессы:
\begin{equation*}
P^0_i(t)=P^0(\Delta_t,A_i)-\mu(\Delta_t,A_i)\,,\quad i=1,2,\ldots,N\,.
%    \label{e3.2-sin}
    \end{equation*}
Интенсивности этих процессов определяются через математическое
ожидание $\mu(\Delta_t, A)$  пуассоновской меры $P\da$ по формуле:
    \begin{equation*}
    \nu_i(t)=\fr{d\mu(\Delta_t,t),A_i)}{dt}\,. %\label{e3.3-sin}
    \end{equation*}
Простейший способ замены уравнения~(\ref{e3.1-sin}) разностным уравнением состоит в замене всех дифференциалов
элементами интегральных сумм:
\begin{multline*}
Y((n+1)h)-Y(nh)=a(Y(nh),nh)h+{}\\
{}+b(Y(nh,nh)) \left[W_0((n+1)h)-
W_0(nh)\right]+{}\\
{}+\sumin c_i(Y(nh),nh)\left[P_i^0((n+1)h)-P_i^0(nh)\right]\,.
\end{multline*}
Положим
\begin{align*}
\bar Y_n&=Y(nh)\,;\\ 
\varphi_n(\bar Y_n)&=Y(nh)+a(Y(nh),nh)\,;
   \\
    \psi_{1n}(\bar Y_n)&=b(Y(nh),nh)\,;\\
\psi_{in}(\bar     Y_n)&=c_{i-1}(Y(nh),nh)\,;\\
 V_{1n}&=W_0((n+1)h)-W_0(nh)\,;\\
V_{in}&=P_{i-1}^0((n+1)h)-P_{i-1}^0(nh)\,,\\
    &\hspace*{35mm}i=2,\ldots,N+1\,.
%\label{e3.4-sin}    
    \end{align*}
В результате получим разностное стохастическое уравнение:
    $$\bar Y_{n+1}=\varphi_n(\bar Y_n)+\sum\limits_{i=1}^{N+1}\psi_{in}(\bar Y_n)V_{in}\,.
    $$
Вводя блочную матрицу $p\times (r+N)$
\begin{equation*}
\psi_n(\bar Y_n)=\left[\psi_{1n}(\bar Y_n) \cdots \psi_{N+1,n}(\bar  Y_n)\right] %\label{e3.5-sin}
\end{equation*}
и $(r+N)$-мер\-ный случайный вектор
\begin{equation*}
V_n=\left[V_{1n}^{\mathrm{T}}\,\, V_{2n}\,\,\cdots \,\, V_{N+1,n}\right]^{\mathrm{T}}\,, %\label{e3.6-sin}
\end{equation*}
можем записать полученное разностное уравнение коротко в виде:
    \begin{equation}
    \bar Y_{n+1}=\w_n (\bar Y_n, V_n)=\varphi_n(\bar Y_n)+\psi_n(\bar Y_n)V_n\,. \label{e3.7-sin}
    \end{equation}

Так как винеровский и пуассоновский процессы являются процессами с
независимыми приращениями, то случайные векторы $V_n$ образуют
последовательность независимых случайных векторов $\{V_n\}$, причем
блоки $V_{1n}$ векторов $V_n$ имеют нормальное распределение ${\cal N}(0,\bar G_n)$, где
\begin{equation*}
\bar G_n =\inh \nu_0(\tau)\,d\tau \cong \nu_0(nh)h\,; %\label{e3.8-sin}
\end{equation*}
$\nu_0(t)$~--- интенсивность винеровского процесса $W_0(t)$;
скалярные блоки $V_{in}$ ($i\hm=2,\ldots,N+1$) имеют пуассоновские
распределения с параметрами
   \begin{equation*}
   \mu_{in}=\inh \nu_i(\tau)\,d\tau \cong \nu_i(nh)h\,. %\label{e3.9-sin}
   \end{equation*}

Эти распределения полностью определяют распределения случайных
векторов $V_n$. Ковариационная матрица $G_n$ вектора $V_n$
представляет собой блоч\-но-диа\-го\-наль\-ную матрицу
    \begin{equation*}
    G_n=\begin{bmatrix}
    \bar G_n  &0  &0  &\cdots  &0\\
    0  &\mu_{2n}  &0  &\cdots  &0\\
    \cdots  &\cdots  &\cdots  &\cdots  &\cdots\\
    0  &0  &0  &\cdots  &\mu_{N+1,n} \end{bmatrix}\,. %\label{e3.10-sin}
    \end{equation*}

Уравнение~(\ref{e3.7-sin}) определяет $\bar Y_{n+1}$ при данном $\bar Y_n$ с
точностью до~$h$ в детерминированном слагаемом $\varphi_n(\bar Y_n)$ и
с точностью до $h^{1/2}$ в случайном слагаемом $\psi_n(\bar Y_n)V_n$.

Изложенный метод замены стохастического дифференциального
уравнения разностными, по существу, не отличается от метода Эйлера
численного интегрирования обыкновенных дифференциальных уравнений.

Более точные уравнения (до $h^2$ и $h^3$, $h^{3/2}$ и $h^{5/2}$ соответственно в детерминированном 
и случайном слагаемом), лежащие в основе МСМ, приведены в приложении.

Таким образом, {\it с точностью до~$h$ в детерминированном слагаемом $\varphi_n$ и $h^{1/2}$ в случайном 
слагаемом~$\psi_n$ в основе совместного параметрического МСМ лежат разностные стохастические уравнения\/}~(\ref{e3.7-sin}). 
{\it Пользуясь уточненными уравнениями\/}~(\ref{ep34}) {\it приложения, получим соответствующие уравнения с точностью $h^2$ и $h^{3/2}$\/}.

\vspace*{-9pt}

\section{Совместное параметрическое аналитическое и~статистическое моделирование}

\vspace*{-2pt}

Как известно~[1--3], в задачах корреляционного аналитического моделирования процессов в СтС 
с аддитивными шумами широкое распространение получил метод статистической линеаризации (МСЛ). 
Для СтС с параметрическими шумами развит метод нормальной аппроксимации (МНА).

Обобщением МНА распределений являются различные
приближенные методы, основанные на параметризации распределений.
Аппроксимация одномерной характеристической функции $g_1 (\la;t)$
и соответствующей плотности $f_1 (y,t)$ известными функциями
$g_1^* (\la;\Xi)$, $f_1^* (y;\Xi)$, зависящими от
конечномерного векторного параметра $\Xi$,  сводит задачу
приближенного определения одномерного распределения к выводу из
уравнения для характеристических функций обыкновенных
дифференциальных уравнений, определяющих $\Xi$ как функцию
времени. Это относится и к остальным многомерным распределениям.
При аппроксимации многомерных распределений целесообразно выбирать
последовательности функций $\{g_n^* (\la_1\tr \la_n;\Xi_n)\}$ и
 $\{ f_n^* (y_1,\ldots,y_n;\Xi_n)\}$, каждая пара
которых находилась бы в такой  за\-ви\-си\-мости от векторного параметра
$\Xi_n$, чтобы при любом $n$ множество параметров, образующих
вектор $\Xi_n$, включало в качестве подмножества множество
параметров, образующих вектор $\Xi_{n-1}$. Тогда при
аппроксимации $n$-мер\-но\-го распределения придется определять только
те координаты вектора $\Xi_n$, которые не были определены ранее
при аппроксимации функций $g_1$, $f_1\tr g_{n-1}$, $f_{n-1}$.

В зависимости от того, что представляют собой параметры, от
которых зависят функции $f_n^* (y_1\tr y_n;\Xi_n)$ и $g_n^*
(\la_1\tr \la_n;\Xi_n)$, аппроксимирующие неизвестные
многомерные плотности $f_n (y_1,  \ldots,y_n; t_1 \tr t_n)$ и
характеристические функции $g_n (\la_1\tr \la_n; t_1,\ldots,t_n)$,
используются различные приближенные методы решения
 уравнений, определяющих многомерные
распределения вектора состояния системы $Y_t$, в частности методы
моментов, семиинвариантов, ортогональных разложений и~др.

Уравнения МАМ для МСтС~(\ref{e3.7-sin}), если использовать метод 
ортогональных разложений (МОР)~\cite{2-sin, 3-sin}, имеют следующий вид:

\vspace*{-8pt}

\noindent
    \begin{multline}
    \hspace*{-6pt}f_n^* (y_1 \tr y_n;\Xi_n) =w_n(y_1\tr y_n;\bar m_n ,\bar K_n) 
 \left\{\! \vphantom{\sss_{\rho=3}^N}
 1+{}\right.\\
\left. {}+\sss_{\rho=3}^N \sss_{\lv\nu_1\rv +\cdots+ \lv \nu_n\rv
    =\rho} \hspace*{-2.74464pt} c_{\nu_1\tr \nu_n}^{l_1\tr l_n} p_{\nu_1\tr \nu_n} (y_1\tr
    y_n)\right\}\\ 
    (n=1,2,\ldots)\,;\label{e4.1-sin}
    \end{multline}
    
    \vspace*{-12pt}
    
    \noindent
\begin{multline}
c_\nu^{l+1} = {\mathrm{M}}q_\nu (\w_l(Y_l,V_l))={}\\
{}=
    \lk q_\nu \left( \fr{\partial}{i\partial \la}\right)
    {\mathrm{M}}\exp \lf i\la^{\mathrm{T}} \w_l (Y_l, V_l)\rf\rk_{\la=0}\,;\label{e4.2-sin}
    \end{multline}
    
    \vspace*{-12pt}
    
    \noindent
    \begin{multline}
     c_{\nu_1\tr \nu_n}^{l_1\tr l_{n}+1} ={}\\
     {}= {\mathrm{M}}q_{\nu_1\tr \nu_n}
    \left(Y_{l_1}\tr Y_{l_{n-1}}, \w_{l_n} (Y_{l_n},
    V_{l_n})\right)\,.\label{e4.3-sin}
    \end{multline}
    
    
    \end{multicols}
    
%    \hrule
    
    \noindent
Здесь введены обозначения:
\begin{equation}
\bar m_n = \left[m_{l_1}^{\mathrm{T}} \ldots m_{l_n}^{\mathrm{T}}\right]^{\mathrm{T}}\,; \quad
%\label{nov}
    \bar K_n =\begin{bmatrix}
    K_{l_1}&K_{l_1l_2}&\cdots&K_{l_1l_n}\\
    K_{l_1l_2}^{\mathrm{T}}&K_{l_2}&\cdots&K_{l_2l_n}\\
    \cdots&\cdots&\cdots&\cdots\\
    K_{l_1l_n}^{\mathrm{T}}&K_{l_2l_n}^{\mathrm{T}}&\cdots&K_{l_n}\end{bmatrix}\,;\label{e4.4-sin}
    \end{equation}
            \begin{equation*}
    c_{\nu_1\tr \nu_n}^{l_1\tr l_n}=\lk q_{\nu_1\tr \nu_n} \left(
\fr{\partial}{i\partial \la_1} \cdots \fr{\partial}{i\partial
    \la_n}\right) g_{l_1\tr l_n}(\la_1\tr \la_n)\rk_{\la_1
    =\cdots=\la_n=0}\,; %\label{e4.5-sin}
    \end{equation*}
$\{ p_{\nu_1\tr \nu_n}, q_{\nu_1\tr \nu_n}\}$~--- биортогональные полиномы, причем
    \begin{equation*}
\iii_{-\infty}^\infty \cdots \iii_{-\infty}^\infty w (y_1 \tr y_n;
    \bar m,\bar K) p_{\nu_1\tr \nu_n} (y_1\tr y_n)
     q_{\mu_1\tr \mu_n} (y_1\tr y_n)
    \,dy_1 \cdots dy_n = \delta_{\nu_1\mu_1} \cdots \delta_{\nu_n\mu_n}\,, %\label{e4.6-sin}
    \end{equation*}
где $\nu_l =[\nu_{l_1} \cdots \nu_{l_p} ]^{\mathrm{T}}$,
$l\hm=1,2,\ldots$; $\nu_{l_1}\tr \nu_{l_n} \hm=0,1\tr N$;
$\lv \nu_1\rv,\ldots,\lv\nu_n\rv \hm= 1\tr N\hm-n\hm+1$; $\lv \nu_l \rv\hm=
 \lv \nu_{l_1} \rv+\cdots + \lv \nu_{l_p} \rv=n,
\ldots,N$; $w_n (y_1\tr y_n; \bar m_n, \bar K_n)$~---  эталонная плотность, как правило, нормальная:
\begin{equation*}
w_n (y_1\tr y_n; \bar m_n, \bar K_n)= \lk (2\pi)^n |\bar K_n|\rk^{-1/2}
 \exp 
\lf -\fr{1}{2} (u_n -\bar m_n)^{\mathrm{T}} \bar K_n^{-1} (u_n - \bar m_n)\rf\,, %\label{e4.7-sin}
\end{equation*}
где в дополнение к обозначениям~(\ref{e4.4-sin})  принято 
$u_n\hm= \lk y_1^{\mathrm{T}} y_2^{\mathrm{T}}\cdots y_n^{\mathrm{T}}\rk^{\mathrm{T}}$. 
Отсюда как частный случай вытекают уравнения методов моментов, квазимоментов, семиинвариантов и~др., 
а также параметров структурной параметризации~[1--3].

Таким образом, {\it с точностью до $h$ и $h^{1/2}$ соответственно при детерминированном и случайном 
слагаемом уравнениями совместного параметрического МСМ и МАМ служат уравнения\/}~(\ref{e3.7-sin})--(\ref{e4.3-sin}).

\vspace*{18pt}

\hrule

\vspace*{6pt}

\begin{multicols}{2}

\section{Корреляционные уравнения статистического и~аналитического моделирования}

Основываясь на результатах разд.~3 и~4, ограничимся уравнениями с точностью до~$h$ 
в детерминированном слагаемом $\varphi_n$ и с точностью до  $h^{1/2}$ в случайном слагаемом~$\psi_n$ 
и заменим нелинейные функции $ \varphi_n$, $\psi_{1n}$, $\psi_{in}$, $\psi_n$ в~(\ref{e3.7-sin}) 
статистически линеаризованными зависимостями:
    \begin{multline*}
    \varphi_n \approx \hat \varphi_n ={}\\
    {}=\bar Y_n + k_0^a (m_n, K_n) m_n + k_1^a (m_n, K_n) (\bar Y - m_n) ={}\\
{}=\alp_n (m_n, K_n)Y_n +\alp_{0n} (m_n, K_n)\,; %\label{e5.1-sin}
\end{multline*}
    \begin{equation*}
    \psi_{1n}\approx \hat \psi_{1n} = k_0^b (m_n, K_n) m_n + k_1^{b} (m_n, K_n) (\bar Y_n - m_n)\,; %\label{e5.2-sin}
    \end{equation*}
    
    \vspace*{-12pt}
    
\begin{multline*}
\psi_{in}\approx \hat \psi_{in} = {}\\
{}=k_0^{c_i-1} (m_n, K_n) m_n + k_1^{c_i-1} (m_n, K_n) (\bar Y_n - m_n)\,; %\label{e5.3-sin}
\end{multline*}
    
    
    \vspace*{-12pt}
    
    \begin{multline*}
    \psi_{n}\approx \hat \psi_{n} ={}\\
    {}= k_0^{\psi_n} (m_n, K_n) m_n + k_1^{\psi_n} (m_n, K_n) (\bar Y_n - m_n)\,, %\label{e5.4-sin}
    \end{multline*}
где $m_n= {\mathrm{M}} \bar Y_n \hm= {\mathrm{M}} Y(nh)$; 
$K_n \hm={\mathrm{M}}\bar Y_n \bar Y_n^{\mathrm{T}}\hm= K(nh)$; $k_0^\cdot (m_n, K_n)$ и $k_1^\cdot (m_n, K_n)$~--- 
коэффициенты нормальной статистической линеаризации соответствующих нелинейных функций. Тогда, введя обозначение
 \begin{equation*}
 \hat\psi_n (m_n, K_n) =\gamma_{0n} (m_n, K_n) +\sss_{\rho=1}^p \gamma_{\rho n} \bar Y_{\rho n}\,, %\label{e5.5-sin}
 \end{equation*}
представим искомое уравнение МСМ в виде:
\begin{multline}
\bar Y_{n+1} = \alpha_n (m_n, K_n) \bar Y_n +\alpha_{0n} (m_n, K_n)+{}\\
{}+\lk 
\gamma_{0n} (m_n, K_n)+\sss_{\rho=1}^p \gamma_{\rho n} (m_n, K_n)\bar Y_{\rho n}\rk V_n\,,\label{e5.6-sin}
\end{multline}
где $\bar Y_{\rho n}$~--- компонента $\bar Y_n$ с номером~$\rho$ $(\rho\hm=1\tr p)$.

\smallskip

\noindent
{\small \textbf{Замечание 5.1.}\ 
Уравнение~(\ref{e5.6-sin}) является разностным стохастическим уравнением Ито с параметрическими шумами. Оно подробно изучено в~[1--3].}

Применяя к~(\ref{e5.6-sin}) теорию дискретных стохастических систем с параметрическими шумами~[1--3], 
получим следующие детерминированные уравнения для $m_n, K_n$ и $K(j,l)$ вместе с соответствующими начальными 
условиями:
\begin{multline}
m_{n+1} = \alpha (m_n, K_n) m_n +\alpha_{0n} (m_n, K_n)\,,\\ m_1 =\mathrm{M} Y_1\,;\label{e5.7-sin}
\end{multline}

\vspace*{-12pt}

\noindent
\begin{multline}
K_{n+1} =\alpha_n (m_n, K_n)K_n \alpha_n(m_n, K_n)^{\mathrm{T}}+{}\\
{}+\gamma_{0n} (m_n, K_n)G_n \gamma_{0n} (m_n, K_n)^{\mathrm{T}} +{}\\
    {}+ \sss_{j=1}^p \left[  \gamma_{0n} (m_n, K_n)G_n \gamma_{jn}(m_n, K_n)^{\mathrm{T}}+ {}\right.\\
    \left.{}+\gamma_{jn} (m_n, K_n) G_n \gamma_{0n}(m_n, K_n)^{\mathrm{T}}\right]+{}\\
    {}+\ss2\limits_{j,l=1} \gamma_{jn}(m_n, K_n)G_n \gamma_{ln}(m_n, K_n) \left[ m_{nj} m_{nl} +{}\right.\\
    \left.{}+ K_{njl}\right]\,,\enskip K_1 ={\mathrm{M}} Y_1 Y_1^{\mathrm{T}}\,;
    \label{e5.8-sin}
    \end{multline}

\vspace*{-12pt}

\noindent
\begin{multline}
K(j, L+1) = K(j,l) \alpha_l (m_l, K_l)^{\mathrm{T}}\,,\\
K(j,j) = K_j\quad (l>j)\,,\\ 
K(j,l) = K(l,j)^{\mathrm{T}} \quad (l<j)\,.
\label{e5.9-sin}
\end{multline}


Таким образом, {\it совокупность стохастического разностного уравнения}~(\ref{e5.6-sin}) 
\textit{и детерминированных уравнений}~(\ref{e5.7-sin})--(\ref{e5.9-sin}) 
\textit{представляет собой систему искомых нелинейных корреляционных уравнений МСМ и МАМ}.

Точность совместного корреляционного статистического и аналитического моделирования, особенно 
для разрывных нелинейных функций $a$, $b$, $c$ в~(\ref{e3.7-sin}), можно повысить, если вместо~(\ref{e5.7-sin})--(\ref{e5.9-sin}) 
воспользоваться дискретной версией непрерывных уравнений МНА~[1--3].

\section{Заключение}


Для нелинейных дифференциальных СтС на многообразиях, понимаемых в смысле Ито, 
разработаны дискретные методы параметрического статистического и математического моделирования 
различной точности. Предполагается, что в качестве параметров одно- и многомерных распределений
выбраны коэффициенты совместных ортогональных разложений плотностей для различных моментов времени.
Выведены уравнения нелинейной корреляционной теории статистического и аналитического моделирования.

Полученные результаты положены в основу разрабатываемого в ИПИ РАН символьного инструментального 
программного обеспечения в среде MATLAB~\cite{17-sin}.

\end{multicols}





\setcounter{equation}{0}
{\small \section*{\raggedleft Приложение}

\renewcommand{\theequation}{П.\arabic{equation}}

\textbf{1.}\ Обобщая~\cite{2-sin, 3-sin}, для вывода более точных
разностных уравнений, чем~(\ref{e3.7-sin}), заменим~(\ref{e3.1-sin})
соответствующим интегральным уравнением:
    \begin{equation}
    \Delta Y_n=\inh a(Y_{\tau},\tau)\,d\tau +\inh
    b(Y_{\tau},\tau)\,dW_0(\tau)+\sumin \inh c_i(Y_{\tau},\tau)\,dP_i^0(d\tau)\,.\label{ep1}
    \end{equation}

С целью приближенного вычисления интегралов определим $Y_{\tau}$ путем
линейной интерполяции случайной функции $Y(t)$ на интервале $(nh,
(n+1)h)$. Тогда, полагая по-преж\-не\-му $\Delta Y_n\hm=Y((n+1)h)\hm-Y(nh)$,
будем иметь:
    \begin{gather*}
    a(Y_{\tau},\tau)\cong a\left(\bar Y_n+\ovth \Delta
    Y_n,\tau\right)\,;\quad
    b(Y_{\tau},\tau)\cong b\left(\bar Y_n+\ovth \Delta
    Y_n,\tau\right)\,;\\ 
c_i(Y_{\tau},\tau)\cong  c_i\left(\bar Y_n+\ovth \Delta Y_n,\tau\right),
    \quad i=1,\ldots,N.
    %    \label{ep2}
    \end{gather*}

Для вычисления правых частей в этих формулах применим обобщенную
формулу Ито~\cite{2-sin, 3-sin}:
\begin{multline}
\varphi(Y+dY, t+dt)=\varphi(Y,t)
    +\left\{\varphi_t(Y,t)+\varphi_y(Y,t)^{\mathrm{T}}a(Y,t)+\fr{1}{2}\,\varphi_{yy}(Y,t):\sigma(Y,t)+{}\right.\\
    {}+
    \sumin \left[ \vphantom{(Y)^{\mathrm{T}}}
    \varphi(Y+c_i(Y,t),t)-\varphi(Y,t)\right.
    \left.\left. -\varphi_y(Y,t)^{\mathrm{T}} c_i(Y,t)\right]\nu_i(t)
    \vphantom{\fr{1}{2}}\right\}\,dt +
    \varphi_y(Y,t)^{\mathrm{T}}b(X,t)\,dW_0+{}\\
    {}+\sumin \left[\varphi(Y+c_i(Y,t),t)-\varphi(Y,t)\right]\, dP_i^0\,,\label{ep3}
    \end{multline}
где $\varphi_t(y,t)$~--- частная производная функции $\varphi(y,t)$
(возможно, векторной) по времени~$t$; $\varphi_y(y,t)$~--- матрица,
строки которой представляют собой частные производные мат\-ри\-цы-стро\-ки
$\varphi(y,t)^{\mathrm{T}}$ по компонентам вектора~$y$;
$\varphi_{yy}(y,t):\sigma(y,t)$~--- вектор, компонентами которого
служат следы произведений матриц вторых производных соответствующих
компонент векторной функции $\varphi(y,t)$ по компонентам вектора~$y$
на мат\-ри\-цу~$\sigma(y,t)$:
    \begin{align}
    \left [\varphi_{yy}(y,t):\sigma(y,t)\right]_k&=\mbox{tr}
    \left[\varphi_{kyy}(y,t)\sigma(y,t)\right]\,;\notag %label{ep4}
\\
    \sigma(y,t)&=b(y,t)\nu_0(t)b(y,t)^{\mathrm{T}}\,.\label{ep5}
    \end{align}

Формулу~(\ref{ep3}) удобно преобразовать так, чтобы в нее входили
непосредственно пуассоновские процессы $P_i(t)$, а не центрированные
процессы $P_i^0(t)$. Имея в виду, что
\begin{equation}
P_i^0(t)=P_i(t)-\int\limits_0^t\nu_i(\tau)\,d\tau\,,\label{ep6}
\end{equation}
можем переписать формулу~(\ref{ep3}) в виде:
\begin{multline}
\varphi(Y+dY, t+dt)=\varphi(Y,t)+\left\{\varphi_t(Y,t)+\varphi_y(Y,t)^{\mathrm{T}}
    \left[a(Y,t)-
    \sumin c_i\bxt\nu_i(t)\right]+\fr{1}{2}\,\varphi_{yy}\bxt:\sigma\bxt\right\}\,dt
    +{}\\
    {}+\varphi_y{\bxt}^{\mathrm{T}}  b\bxt dW_0+\sumin \left[\varphi(Y+c_i\bxt,t)-
    \varphi\bxt\right]\,dP_i\,.\label{ep7}
    \end{multline}

Для распространения формул~(\ref{ep3}) и~(\ref{ep7}) на матричные функции~$\varphi$ необходимо видоизменить запись некоторых ее членов, чтобы
в рамках алгебры матриц выражения $\varphi_y\xt$ имели смысл и для
матричной функции~$\varphi$. Имея в виду, что в случае скалярной или
векторной функции~$\varphi$
    $$
    \varphi_y{\xt}^{\mathrm{T}}u=\sum\limits_{i=1}^n \fr{\partial\varphi\xt}
     {\partial y_i}\,u_i=u^{\mathrm{T}}\fr{\partial}{\partial y}\,\varphi\xt
     $$
для любого $p$-мер\-но\-го вектора $u$, можем заменить слагаемые
$\varphi_y{\bxt}^{\mathrm{T}} a\bxt$, $\varphi_y{\bxt}^{\mathrm{T}} b\bxt\, dW_0$,
$\varphi_y{\bxt}^{\mathrm{T}} c_i\bxt$
соответственно слагаемыми
$a{\bxt}^{\mathrm{T}} (\partial/\partial y)^{\mathrm{T}}\varphi\bxt$,
$dW_0^{\mathrm{T}}b\bxt(\partial/\partial y)\varphi\bxt$,
$c_i{\bxt}^{\mathrm{T}} (\partial/\partial y)\varphi\bxt.$
Поэтому формулы~(\ref{ep3}) и~(\ref{ep7}) будут справедливы и для
матричных функций~$\varphi$, так как $u^{\mathrm{T}}(\partial/\partial y)$ представляет собой скалярный дифференциальный оператор для любого
$p$-мер\-но\-го вектора~$u$ и его применение к векторной или матричной
функции~$\varphi$ означает его применение ко всем компонентам вектора~$\varphi$ или ко всем элементам матрицы~$\varphi$. Величина
$\varphi_{yy}\bxt:\sigma\bxt$ в случае матричной функции~$\varphi$
представляет собой матрицу, элементами которой служат следы
произведений на матрицу $\sigma(y,t)$ матриц вторых производных соответствующих элементов
матрицы $\varphi\xt$ по компонентам вектора~$y$ на матрицу~$\sigma\xt$:
    \begin{equation*}
    \left[\varphi_{yy}\xt : \sigma\xt\right]_{kl}=
    \mbox{tr}\left[\varphi_{klyy}\xt\sigma\xt\right]\,. %\label{ep8}
    \end{equation*}

Используя формулу~(\ref{ep7}) и ее модификацию, пригодную для матричной
функции~$\varphi$, а также учитывая~(\ref{ep5}) и то, что согласно~(\ref{e2.1-sin}) и~(\ref{ep6})
\begin{equation}    
\ovth \Delta Y_n=\ovth a(\bar Y_n,nh)h+\ovth b(\bar Y_n,nh)\Delta W_n+
\sumin \ovth c_i(\bar Y_n,nh)(\Delta P_{in}-\nu_{in}h)\,,\label{ep9}
\end{equation}
где, как и раньше, $\Delta W_n\hm=W_0((n+1)h)-W_0(nh)$, $\Delta
P_{in}\hm=P_i((n+1)h)\hm-P_i(nh)$, будем иметь с точностью до малых высшего
порядка относительно $h$ следующие равенства:
    \begin{multline}
     a\left(\bar Y_n+\ovth \Delta Y_n,\tau \right)=a
    +a_t(\tau-nh)+\ovth a_y^{\mathrm{T}}\left(a-\sumin c_i\nu_{in}\right)h+{}\\
    {}+
    \fr{1}{2}\left(\ovth\right)^2(a_{yy}:\sigma)h+\ovth a_y^{\mathrm{T}} b\Delta W_n+
    \sumin \left[a\left(\bar Y_n+\ovth c_i,nh\right)-
    a\right]\Delta P_{in}\,;\label{ep10}
    \end{multline}

\vspace*{-12pt}

\noindent
\begin{multline}
 b\left(\bar Y_n+\ovth \Delta Y_n,\tau \right)=
    b+b_{t}(\tau-nh)+\ovth  \left(a^{\mathrm{T}}-\sumin c_i^{\mathrm{T}}\nu_{in}\right)\fr{\partial}
    {\partial y}bh+{}\\
    {}+\fr{1}{2}\left(\ovth\right)^2(b_{yy}:\sigma)h+
    \ovth  \Delta W_n^{\mathrm{T}} b^{\mathrm{T}}\fr{\partial}{\partial y} b+
    \sumin \left[b\left(\bar Y_n+\ovth  c_i,nh\right)-b\right]\Delta P_{in}\,;\label{ep11}
    \end{multline}
    
    \vspace*{-12pt}
    
    \noindent
    \begin{multline}
     c_i\left(\bar Y_n+\ovth \Delta Y_n,\tau \right)=c_i+c_{it}(\tau-nh)+
    \ovth c_{iy}^{\mathrm{T}} \left(a-\sum\limits_{j=1}^N c_j\nu_{jn}\right)h+{}\\
    {}+\fr{1}{2}
    \left(\ovth\right)^2(c_{iyy}:\sigma)h+\ovth c_{iy}^{\mathrm{T}} b\Delta W_n+
    \sum\limits_{j=1}^N \left[c_i\left(\bar Y_n+\ovth c_j,nh\right)
    -c_i\right]\Delta P_{in}\,.\label{ep12}
    \end{multline}
Здесь аргументы $\bar Y_n$, $nh$ у всех функций, зависящих от них,
для краткости опущены.

\textbf{2.}\ Для повышения точности вычислений, в частности
ковариационной матрицы нормально распределенного случайного вектора
$\Delta W_n\hm=W_0((n+1)h)-W_0(nh)$ и параметров пуассоновских распределений
случайных величин $\Delta P_{in}\hm=P_i((n+1)h)\hm-P_i(nh)$, можно взять
значения интенсивностей $\nu_0(t)$ процесса $W_0(t)$ в формуле~(\ref{ep5}) и
 $\nu_i(t)$ пуассоновских потоков, порождающих
процессы $P_i(t)$, в средней точке $nh\hm+h/2$ интервала $(nh,(n+1)h)$:
    \begin{equation}
    \left.
    \begin{array}{rl}
    \sigma(\bar Y_n,nh) &\displaystyle=b(\bar Y_n,nh)\nu_0\left(nh+\fr{h}{2}\right)b(\bar Y_n,nh)^{\mathrm{T}}\,;\\[9pt]
    \nu_{in} &=\displaystyle\nu_i\left(nh+\fr{h}{2}\right)\,,\quad i=1,\ldots,N\,.
    \end{array}
    \right\}
    \label{ep13}
    \end{equation}

Для облегчения дальнейших вычислений найдем приращения функций~$a$,
$b$, $c_i$ в суммах уравнения~(\ref{ep9}) путем линейной интерполяции на малом
интервале $(nh,(n+1)h)$:
    \begin{equation}
    \left.
    \begin{array}{rl}
a\left(\bar Y_n+\ovth c_i,nh\right)-a&\cong \ovth \Delta_i a_n\,;\\[9pt]
    b\left(\bar Y_n+\ovth c_i,nh\right)-b &\cong \ovth \Delta_i b_{n}\,;\\[9pt]
    c_i\left(\bar Y_n+\ovth c_j,nh\right)-c_i&\cong \ovth \Delta_j c_{in}\,;
    \end{array}
    \right \}
\label{ep14}
\end{equation}
    \begin{equation*}
    \Delta_i a_n =a(\bar Y_n+c_i,nh)-a(\bar Y_n,nh)\,;\quad
    \Delta_i b_{n}=b(\bar Y_n+c_i,nh)-b(\bar Y_n,nh)\,;\quad
\Delta_j c_{in} =c_i(\bar Y_n+c_j,nh)-c_i(\bar Y_n,nh)\,.
%\label{ep15}
\end{equation*}

Используя формулы~(\ref{ep10})--(\ref{ep14}), найдем приближенные выражения
интегралов в~(\ref{ep1}):
    \begin{equation*}
    \int\limits_{nh}^{(n+1)h}a(Y_{\tau},\tau)\,d\tau=
    ah+\fr{1}{2}\left[a_t+a_y^{\mathrm{T}}\left( a-\sumin c_i\nu_{in}\right)+
    \fr{1}{3}\,a_{yy}:\sigma\right]h^2+
    \fr{1}{2}\left (a_y^{\mathrm{T}} b\Delta W_n +
    \sumin \Delta_i a_n \Delta P_{in}\right )h\,; %\label{ep16}
    \end{equation*}
    
    %\vspace*{-12pt}
    
    \noindent
\begin{multline*}
\int\limits_{nh}^{(n+1)h}b(Y_{\tau},\tau)\,dW_0(\tau)=b\Delta W_n+
    \left\{ \left[b_{t}+\left( a^{\mathrm{T}}-\sumin c_i^{\mathrm{T}}\nu_{in}\right)\fr{\partial}
{\partial y} b\right]h+{}\right.\\
\left.{}+\Delta W_n^{\mathrm{T}} b^{\mathrm{T}} \fr{\partial}{\partial y} b +
    \sumin \Delta_i b_{n} \Delta P_{in}\right \} \int\limits_{nh}^{(n+1)h}
    \ovth \,dW_0(\tau)+\fr{1}{2}\left(b_{yy}:\sigma\right) \int\limits_{nh}^{(n+1)h}
    \left(\ovth\right)^2 dW_0(\tau)\,, %\label{ep17}
    \end{multline*}
        \begin{multline*}
    \int\limits_{nh}^{(n+1)h}c_i(Y_{\tau},\tau)\,dP_i^0(\tau)=
    c_i\Delta P_{in}+\left\{ \left[c_{it}+c_{iy}^{\mathrm{T}}
    \left( a-\sum\limits_{j=1}^N c_j\nu_{jn}\right)\right]h+{}\right.\\
\left.    {}+
    c_{iy}^{\mathrm{T}} b\Delta W_n + \sum\limits_{j=1}^N \Delta_j c_{in}
    \Delta P_{jn}\right \} \int\limits_{nh}^{(n+1)h}\ovth\, dP_i^0(\tau)+\fr{1}{2}\left(c_{iyy}:\sigma\right) \int\limits_{nh}^{(n+1)h}
    \left(\ovth\right)^2 dP_i^0(\tau)\,;  %\label{ep18}
    \end{multline*}
\begin{multline*}
\int\limits_{nh}^{(n+1)h}c_i(Y_{\tau},\tau)\nu_i(\tau)\,d\tau ={}\\
{}=
    \left\{c_ih+\fr{1}{2}\left[c_{it}+c_{iy}^{\mathrm{T}}
    \left( a-\sum\limits_{j=1}^N c_j\nu_{jn}\right)+\fr{1}{3}\,c_{iyy}:\sigma\,\right]h^2+
\fr{1}{2}\left (c_{iy}^{\mathrm{T}} b\Delta W_n + \sum\limits_{j=1}^N
    \Delta_j c_{in} \Delta P_{jn}\right )h\right\}\nu_{in}\,. %\label{ep19}
    \end{multline*}

Далее, подставив полученные приближенные выражения интегралов в~(\ref{ep1}) и приняв во внимание, что $\Delta
Y_n\hm=Y((n+1)h)\hm-Y_i(nh)\hm=\bar Y_{n+1}\hm-\bar Y_ n$,
придем к разностному уравнению:
    \begin{equation}
    \bar Y_{n+1}=\varphi_n(\bar Y_n)+\sum\limits_{i=1}^{N+1}
    \left[\psi_{in}(\bar Y_n)V_{in}+
    \psi_{in}'(\bar Y_n,V_{in}^{(1)})V_{in}'+\psi_{in}''({\bar
    Y}_n)V_{in}''\right]\,.\label{ep20}
    \end{equation}
Здесь введены следующие обозначения:
    \begin{multline}
    \varphi_n(\bar Y_n)=\bar Y_n+\left[a(\bar Y_n,nh)-\sum\limits_{j=1}^N
    c_j(\bar Y_n,nh)\nu_{jn} \right]h+{}\\
    {}+\fr{1}{2}\left\{a_t(\bar Y_n,nh)-\sum\limits_{j=1}^N c_{jt}(\bar Y_n,nh)\nu_{jn}+
    \left[a_x(\bar Y_n,nh)^{\mathrm{T}}-
    \sum\limits_{j=1}^N c_{jy}(\bar Y_n,hn)^{\mathrm{T}} \nu_{jn}\right]
    \left[a_(\bar Y_n,nh)- \sum\limits_{j=1}^N c_j(\bar Y_n,hn) \nu_{jn}t\right]+{}\right.\\
    \left.{}+\fr{1}{3}\left[a_{yy}(\bar Y_n,nh)-\sum_{j=1}^N
    c_{jyy}(\bar Y_n,nh)\nu_{jn}\right]:\sigma(\bar Y_n,nh)\right\}h^2\,;
\label{ep21}
\end{multline}
    \begin{equation}
    \psi_{1n}(\bar Y_n)=b(\bar Y_n,nh)+\fr{1}{2}\left[a_y(\bar Y_{nh},nh)^{\mathrm{T}}-
    \sum_{j=1}^N c_j(\bar Y_n,hn)^{\mathrm{T}} \nu_{jn}\right] b(\bar
    Y_n,nh)\,;\label{ep22}
    \end{equation}
\begin{equation}
\psi_{in}(\bar Y_n)=c_{i-1}(\bar Y_n,nh)+\left[\Delta_{i-1}a_n-\sum_{j=1}^N
    \Delta_{i-1}c_{jn}\nu_{jn}\right]h\,;\label{ep23}
    \end{equation}
    
    \vspace*{-12pt}
    
    \noindent
    \begin{multline}
    \psi_{1n}'(Y_n,V_n^{(1)})=\left\{b(\bar Y_n,nh)+
    \left[a(\bar Y_n,nh)^{\mathrm{T}}-
    \sum\limits_{j=1}^N c_j(\bar Y_n,hn)^{\mathrm{T}} \nu_{jn}\right]\fr{\partial}
{\partial y} b(\bar Y_n,nh)+{}\right.\\
\left.{}+V_{1n}^{\mathrm{T}} b(\bar Y_n,nh)^{\mathrm{T}}\fr{\partial}{\partial y}\, b(\bar Y_n,nh)
    \right\}h+\sum\limits_{j=1}^N \Delta_j b_{n}V_{j+1,n}\,;\label{ep24}
    \end{multline}
    
    \vspace*{-12pt}
    
    \noindent
    \begin{multline}
    \psi_{in}'(\bar Y_n,V_n^{(1)})=\left\{c_{i-1,t}(\bar
    Y_n,nh)+c_{i-1,y}(\bar Y_n,nh)^{\mathrm{T}} \left[a(\bar Y_n,nh)-\sum\limits_{j=1}^N c_j(\bar Y_n,hn)\nu_{jn}
    \right]\right\}h+{}\\
    {}+c_{i-1,y}(\bar Y_n,nh)^{\mathrm{T}} b(\bar Y_n,nh)V_{1,n}+\sum\limits_{j=1}^N
    \Delta_j c_{j-1,n}V_{j+1,n}\,;\label{ep25}
    \end{multline}
\begin{equation}
\psi_{1n}''(\bar Y_n)=\fr{1}{2}\left[b_{yy}(\bar Y_n,nh):
    \sigma(\bar Y_n,nh)\right]\,;\label{ep26}
    \end{equation}
\begin{equation}
\psi_{in}''(\bar Y_n)=\fr{1}{2}\left[c_{i-1,yy}(\bar Y_n,nh):
    \sigma(\bar Y_n,nh)\right]\,;\label{ep27}
    \end{equation}
\begin{equation}
V_{1n}=\Delta W_n=W_0((n+1)h)-W_0(nh)\,;\label{ep28}
\end{equation}
    \begin{equation}
    V_{in}=\Delta P_{i-1,n}=P_{i-1}((n+1)h)-P_{i-1}(nh)\,,
    \quad i=2,\ldots,N\,; \label{ep29}
    \end{equation}
\begin{equation}
V_{1n}'=\inh \ovth \,dW_0(\tau)\,,\quad V_{1n}''=
    \inh \left(\ovth\right)^2 \,dW_0(\tau)\,;\label{ep30}
    \end{equation}
\begin{equation}
V_{in}'=\inh \ovth \,dP_{i-1}^0(\tau)\,;\label{ep31}
\end{equation}
\begin{equation}
V_{in}''=\inh \left(\ovth\right)^2 dP_{i-1}^0(\tau)\,,
    \quad i=2,\ldots,N+1\,.\label{ep32}
    \end{equation}
Обозначим
$V_n^{(1)}=\left[V_{1n}^{\mathrm{T}}\,\, V_{2n}\,\,\cdots\,\, V_{n+1}\right]^{\mathrm{T}}$
и введем блочную матрицу
    $$\psi_n(\bar Y_n,V_n^{(1)})=\left[\psi_{1n}(\bar Y_n)\,\,
    \psi_{1n}'(\bar Y_n,V_n^{(1)})\,\, \psi_{1n}''(\bar Y_n)\,\,\cdots\,\,
    \psi_{N+1,n}(\bar Y_n)\,\, \psi_{N+1,n}'(\bar Y_n,V_n^{(1)})\,\,
    \psi_{N+1,n}''(\bar Y_n) \right]$$
и блочный случайный вектор
    \begin{equation}
    V_n=\left[V_{1n}^{\mathrm{T}}\,\, V_{1n}^{'\mathrm{T}}\,\, V_{1n}^{''\mathrm{T}}\,\,
    V_{2n}^{\mathrm{T}}\,\, V_{2n}^{'\mathrm{T}}\,\, V_{2n}^{''\mathrm{T}}\,\,
    \cdots\,\,V_{N+1,n}\,\, V_{N+1,n}'\,\, V_{N+1,n}'' \right]^{\mathrm{T}}\,.\label{ep33}
    \end{equation}
Тогда можем коротко записать уравнение~(\ref{ep20}) в виде:
\begin{equation}
\bar Y_{n+1}=\varphi_n(\bar Y_n)+\psi_n(\bar
    Y_n,V_n^{(1)})V_n\,.\label{ep34}
    \end{equation}
Входящие в уравнения~(\ref{ep34}) величины определены формулами~(\ref{ep21})--(\ref{ep33}).

\textbf{3.}\ Найдем распределение случайного вектора $V_n$. Ясно, что математические
ожидания случайных величин $V_{1n}'$, $V_{1n}''$ равны нулю и что
случайный вектор $[V_{1n}^{\mathrm{T}}\,\, {V_{1n}'}^{\mathrm{T}}\,\, {V_{1n}''}^{\mathrm{T}}]$
имеет нормальное распределение, а скалярные случайные величины
$V_{2n},\ldots,V_{N+1,n}$~--- пуассоновские распределения с параметрами
\begin{equation*}
\mu_{in}=\inh \nu_{i-1}(\tau)\,d\tau=\nu_{i-1}
    \left(nh+\fr{1}{2}\,h\right)h\,,\quad i=2,\ldots,N+1\,. %\label{ep35}
    \end{equation*}
Очевидно также, что тройки случайных величин
$V_{1n}, V_{1n}'$, $V_{1n}'',\ldots,V_{N+1,n}$, $V_{N+1,n}'$, $V_{N+1,n}''$
независимы в силу независимости процессов $W_0(t)$,
$P_1(t),\ldots,P_N(t)$
и что при различных $n$ величины $V_n$ независимы. Однако при любых
данных $i$, $n$ величины $V_{in}$, $V_{in}'$, $V_{in}''$ зависимы.

Для полного определения распределения случайного вектора $V_n$ в~(\ref{ep34}) достаточно найти ковариационную матрицу нормально
распределенного случайного вектора $[V_{in}^{\mathrm{T}}\,\,{V_{in}'}^{\mathrm{T}}\,\,{V_{in}''}^{\mathrm{T}}]$. Пользуясь известными
формулами ковариационных и взаимных ковариационных матриц
стохастических интегралов, находим блоки ковариационной
матрицы $K_{1n}$ случайного вектора $[V_{in}^{\mathrm{T}}\,\,{V_{in}'}^{\mathrm{T}}\,\,{V_{in}''}^{\mathrm{T}}]$:
    $$K_{1n,11}=MV_{1n}V_{1n}^{\mathrm{T}}=\inh \nu_0(\tau)d\tau\cong \nu_0\left(nh+\fr{h}{2}\right)h\,;$$
    $$K_{1n,12}=MV_{1n}{V_{1n}'}^{\mathrm{T}}=\inh \ovth \nu_0(\tau)\,d\tau
    \cong \fr{1}{2}\,\nu_0\left(nh+\fr{h}{2}\right)h\,;$$
    $$K_{1n,13}=MV_{1n}{V_{1n}''}^{\mathrm{T}}=\inh \left(\ovth\right)^2 \nu_0(\tau)k\,d\tau
    \cong \fr{1}{3}\,\nu_0\left(nh+\fr{h}{2}\right)h\,;$$
    \begin{equation*}
     K_{1n,21}= K_{1n,12}\,,\quad K_{1n,31}= K_{1n,13}\,; %\label{ep36}
     \end{equation*}
    $$K_{1n,23}=MV_{1n}{V_{1n}''}^{\mathrm{T}}=\inh \left(\ovth\right)^3 \nu_0(\tau)\,d\tau
    \cong \fr{1}{4}\,\nu_0\left(nh+\fr{h}{2}\right)h\,;$$
    $$ K_{1n,31}= K_{1n,13}\,;\quad K_{1n,32}= K_{1n,23}\,;$$
    $$K_{1n,33}=MV_{1n}''{V_{1n}''}^{\mathrm{T}}=\inh \left(\ovth\right)^4
    \nu_0(\tau)d\tau\cong \fr{1}{5}\,\nu_0\left(nh+\fr{h}{2}\right)h\,.$$

Практически целесообразно аппроксимировать
стохастические интегралы от неслучайных
функций в~(\ref{ep31}), определяющие величины $V_{in}'$, $V_{in}^{\prime\prime}$
($i=2,\ldots,N+1$), с помощью аналога интегральной теоремы о среднем
для стохастических интегралов:
    \begin{equation*}
    V_{in}'=\inh \ovth\, dP_{i-1}^0(\tau)\cong \left[\fr{\inh \left(
    (\tau-nh)/h\right) \nu_{i-1}(\tau)\,d\tau}{\inh \nu_{i-1}(\tau)\,d\tau}\,
    \right]\Delta P_{i-1,n}\cong \fr{1}{2}\,\Delta P_{i-1,n}=
\fr{1}{2}\,V_{in}\,; %\label{ep37}
\end{equation*}

\vspace*{-12pt}

\noindent
\begin{multline*}
V_{in}''=\inh \left( \ovth\right)^2 dP_{i-1}^0(\tau)
    \cong \left[\fr{\inh \left((\tau-nh)/h\right)^2 \nu_{i-1}(\tau)\,d\tau}
    {\inh \nu_{i-1}(\tau)\,d\tau}\right]
    \Delta P_{i-1,n}\cong \fr{1}{3}\,\Delta P_{i-1,n}={}\\
    {}=
\fr{1}{3}\,V_{in}\,,\quad i=2,\ldots,N+1\,. %\label{ep38}
\end{multline*}

\textbf{4.}\ При статистическом  моделировании системы с помощью уравнения~(\ref{ep34}) также не
представляет трудностей моделировать случайные величины,
распределенные по нормальному и пуассоновским законам.
Легко видеть, что правая часть разностного уравнения~(\ref{ep34})
определена с точностью до $h^2$ в детерминированном (при данном $\bar Y_n$)
слагаемом $\varphi_n(\bar Y_n)$ и с точностью до
$h^{3/2}$ в случайном слагаемом $\psi_n(\bar Y_n,V_n^{(1)})V_n$.

При выводе уравнения~(\ref{ep34}) были допущены две небольшие неточности.
Во-пер\-вых, при замене $\bar Y_{\tau}$ величиной $\bar Y_n\hm+(\tau-nh)\Delta Y_n/h$ случайные функции $b(Y_{\tau},\tau)$ и
$c_i(Y_{\tau},\tau )$, независимые от $dW_0(\tau)$ и $dP_i^0(\tau)$
в силу конструкции интеграла Ито, были заменены неслучайными
функциями, зависящими от случайного параметра $\Delta Y_n$, который
зависит от значений $dW_0(\tau)$ и $dP_i^0(\tau)$ в интервале~$(nh,(n+1)h)$. Во-вто\-рых, 
если $c_i(y,t)\hm\ne 0$ хотя бы при одном~$i$, реализации случайного процесса $Y(t)$ имеют разрывы первого
рода в случайных точках, несмотря на его сред\-не\-квад\-ра\-ти\-че\-скую
непрерывность. Поэтому линейную интерполяцию данного процесса,
строго говоря, проводить нельзя. Первую из этих неточностей можно
устранить двумя способами. Первый состоит в замене интерполяции
процесса $Y(t)$ экстраполяцией, что равноценно замене $\Delta Y_n$ в
получаемом выражении для $Y_{\tau}$ величиной $\Delta Y_{n-1}$.
Однако это приведет к появлению в правой части разностного уравнения
величин $\bar Y_{n-1}$ и $V_{n-1}^{(1)}$, т.\,е.\ к замене уравнения
первого порядка разностным уравнением второго порядка. Второй способ
состоит в отказе от интерполяции процесса $Y(t)$ на интервале
$(nh,(n+1)h)$ и непосредственном выражении приращений функций
$a(Y_{\tau},\tau)$, $b(Y_{\tau},\tau)$, $c_i(Y_{\tau},\tau)$ на
малом интервале $(nh,(n+1)h)$ по обобщенной формуле Ито с заменой в
ней дифференциалов приращениями. При этом способе устраняется и
вторая допущенная неточность. Но полученное таким путем разностное
уравнение будет более сложным. В~него войдут случайные величины,
представляющие собой двойные интегралы по компонентам винеровского
процесса $W(t)\hm=W_0(t)$ и по пуассоновским процессам:
    $$
    \inh \int\limits_{nh}^{\tau}\,dW_j(\sigma)dW_j(\tau)\,;
    \enskip  \inh \int\limits_{nh}^{\tau}dP_i(\sigma)\,dP_j^0(\tau)\,;\enskip
    \inh \int\limits_{nh}^{\tau}\,dP_i^0(\sigma)\,dW_j(\tau)\,;
    \enskip  \inh \int\limits_{nh}^{\tau}\,dP_j^0(\sigma)\,dW_i(\tau)\,. %\label{ep39}
    $$
Распределения этих случайных величин найти чрезвычайно сложно, и
только первые два из них легко вычисляются при $j\hm=i$:
    $$
    \inh \int\limits_{nh}^{\tau}\,dW_i(\sigma)dW_i(\tau)=
\fr{\left[\Delta W_{in}\right]^2-\nu_{ii}(nh+h/2)h}{2}\,;\quad 
\inh \int\limits_{nh}^{\tau}\,dP_i^0(\sigma)dP_i^0(\tau)=
\fr{\left[\Delta P_{in}\right]^2-\Delta P_{in}}{2}\,. % \label{ep40}
$$
Что касается второй неточности, то она не может существенно повлиять
на результат, так как вероятность появления скачка пуассоновского
процесса в достаточно малом интервале $(nh,(n+1)h)$ ничтожно мала.

Точность аппроксимации дифференциального стохастического уравнения
разностным можно  повышать и дальше. В~частности, в одном из способов сначала
достаточно выразить $a(Y_{\tau},\tau)$,
$b(Y_{\tau},\tau)$, $c_i(Y_{\tau},\tau)$ на интервале $(nh,(n+1)h)$
интегральной формулой Ито, соответствующей дифференциальной формуле:
    \begin{multline}
    a(Y_{\tau},\tau)=a(\bar Y_n,nh)+\int\limits_{nh}^{\tau}\left( a_t(Y_s,s)+
    a_x(Y_s,s)^{\mathrm{T}}\left[\,a(Y_s,s)-\sumin c_i(Y_s,\nu_i(s))+\fr{1}{2}\,a_{yy}(Y_s,s):
    \sigma(Y_s,s)\right]\right)\,ds+{}\\
    {}+\int\limits_{nh}^{\tau}a_y(Y_s,s)^{\mathrm{T}}
 b(Y_s,s)\,dW_0(s)+\sumin \int\limits_{nh}^{\tau}
    \left[a(Y_s+c_i(Y_s,s),s)-a(Y_s,s)\right]\,dP_i^0(s)\,. \label{ep41}
    \end{multline}
При этом формулы для $b(Y_{\tau},\tau)$ и $c_i(Y_{\tau},\tau)$ аналогичны. Затем
применим к полученным интегралам тот же прием, который использовался
для вычисления интегралов в~(\ref{ep1}). В~результате получим правую
часть разностного уравнения с точностью до $h^3$ в детерминированном
(при данном $\bar Y_n$) слагаемом и $h^{5/2}$ в случайном слагаемом.
Процесс уточнения разностного уравнения, соответствующего данному
стохастическому дифференциальному уравнению, можно продолжать
неограниченно. Однако каждое новое уточнение требует существования
производных функций $a$, $b$, $c_i$ все более высоких порядков.

Для уточнения разностного уравнения можно применить следующий способ: выразить подынтегральные функции в~(\ref{ep41}) и
аналогичных формулах для $b(Y_{\tau},\tau)$ и $c_i(Y_{\tau},\tau)$ их
выражениями по обобщенной формуле Ито, заменив в ней дифференциалы
приращениями. При этом в разностное уравнение войдут тройные интегралы
по компонентам винеровского процесса $W_0(t)$ и пуассоновским процессам
$P_i(t)$. Для дальнейшего уточнения аппроксимации стохастического
дифференциального уравнения разностным в этом случае следует
подынтегральные функции в~(\ref{ep41}) и соответствующих формулах для
$b(Y_{\tau},\tau)$ и $c_i(Y_{\tau},\tau)$, в свою очередь,
представить интегральной формулой Ито, а затем уже применять
дифференциальную формулу Ито с заменой дифференциалов приращениями.
Данный процесс можно продолжать неограниченно, и в итоге он приведет к
представлению процесса $Y(t)$ на интервале $(nh,(n+1)h)$
стохастическими аналогами формулы Тейлора. При этом в разностное
уравнение войдут кратные стохастические интегралы по компонентам
винеровского процесса $W_0(t)$ и по пуассоновским процессам $P_i(t)$.
Нахождение распределения этих интегралов представляет практически
непреодолимые трудности. И лишь интегралы любой кратности по одной и
той же компоненте винеровского процесса $W_0(t)$ или по одному и тому же
пуассоновскому процессу $P_i(t)$ вычисляются очень просто.

Чтобы избежать вычисления производных
функций $a(Y_{\tau},\tau)$, $b(Y_{\tau},\tau)$, $c_i(Y_{\tau},\tau)$ при
применении двух изложенных способов аппроксимации дифференциального стохастического
уравнения разностным, можно рекомендовать заменить
их отношениями конечных приращений, например на интервале
$(nh,(n+1)h)$ по $t$ и на интервалах $(\bar Y_{nk}, \bar
Y_{nk}\hm+a_k(\bar Y_n,nh)h)$ по компонентам вектора~$y$.

\textbf{5.}\ Полученные разностные уравнения можно использовать как при
теоретических исследованиях, так и для чис\-лен\-но\-го интегрирования
дифференциальных стохастических уравнений. При этом нужно знать
распределение всех случайных величин, входящих в разностные уравнения.
В данном случае разностные уравнения будут представлять собой так называемую
 сильную аппроксимацию стохастических дифференциальных уравнений [1, 14--16].
При численном интегрировании такая аппроксимация нужна, когда требуется
получать реализации процесса $Y(t)$. Однако часто нет
нужды в получении реализаций процесса, а достаточно иметь лишь
оценки моментов или математических ожиданий каких-либо функций от значения
случайного процесса $Y(t)$ в определенный момент. В~таких случаях можно
отказаться от использования точных распределений входящих в разностные
уравнения случайных величин и заменить их ка\-ки\-ми-ни\-будь более
простыми распределениями с теми же моментными характеристиками.
Например, нормально распределенную скалярную величину с нулевым
математическим ожиданием и дисперсией $D$ можно заменить дискретной
случайной величиной, принимающей два значения $\pm\sqrt{D}$ с
вероятностями~1/2. При замене случайных величин на величины с
более простыми распределениями разностное уравнение будет представлять
собой слабую аппроксимацию дифференциального стохастического
уравнения [1, 14--16].

}



\renewcommand{\theequation}{\arabic{equation}}

\vspace*{4mm}


\begin{multicols}{2}

{\small\frenchspacing
{%\baselineskip=10.8pt
\addcontentsline{toc}{section}{Литература}
\begin{thebibliography}{99}


\bibitem{1-sin}
\Au{Пугачев В.\,С., Синицын И.\,Н.}
Стохастические дифференциальные системы. Анализ и фильтрация.~--- М.: Наука,  1990. 
[Англ. пер. Stochastic differential systems. Analysis and filtering.~--- Chichester, New York: Jonh Wiley, 1987.]

\bibitem{2-sin}
\Au{Пугачев В.\,С., Синицын И.\,Н. }
Теория стохастических систем.~--- М.: Логос, 2000; 2004. [Англ. пер. Stochastic systems. Theory and  applications.~--- Singapore: World
Scientific, 2001.]

\bibitem{3-sin}
\Au{Синицын И.\,Н.}
Канонические представления случайных функций и их применение в задачах компьютерной поддержки 
научных исследований.~--- М.: ТОРУС ПРЕСС, 2009.

\bibitem{4-sin}
\Au{Синицын И.\,Н. } Стохастические информационные технологии для исследования нелинейных круговых 
стохастических сис\-тем~// Информатика и её применения, 2011. Т.~5. Вып.~4. С.~78--89.

\bibitem{5-sin}
\Au{Sinitsyn I.\,N., Belousov V.\,V., Konashenkova~T.\,D.} 
Software tools for circular stochastic systems analysis~// 29th  Seminar (International)
on Stability Problems for Stochastic Models: Abstracts.~--- Svetlogorsk, Russia, 2011. Р.~86--87.

\bibitem{6-sin}
\Au{Синицын И.\,Н. } Математическое обеспечение для анализа нелинейных многоканальных 
круговых стохастических сис\-тем, основанное на параметризации распределений~// Информатика и её применения.~--- 
М.: ТОРУС ПРЕСС, 2012. Т.~6. Вып.~1. С.~12--18.

\bibitem{7-sin} 
\Au{Синицын И.\,Н., Корепанов Э.\,Р., Белоусов~В.\,В., Конашенкова~Т.\,Д.} 
Развитие математического обеспечения для анализа нелинейных многоканальных 
круговых стохастических сис\-тем~// Системы и средства информатики, 2012. Вып.~22. №\,1. С.~29--40.

\bibitem{8-sin} \Au{Sinitsyn I.\,N., Belousov V.\,V., Konashenkova~T.\,D.} 
Software tools for spherical stochastic systems analysis and filtering~//  
Прикладные задачи теории вероятности и математической статистики, связанные 
с моделированием информационных сис\-тем (АРТР+MS'2012): Сб. тезисов Международного семинара по проблемам 
устойчивости стохастических моделей (ISSPSM-2012) и  VI Международного рабочего семинара.~--- М.: ИПИ РАН, 2012. С.~91--93.

\bibitem{9-sin} 
\Au{Синицын И.\,Н., Синицын В.\,И., Корепанов~Э.\,Р., Белоусов~В.\,В., Сергеев~И.\,В., Басилашвили~Д.\,А.} 
Опыт моделирования эредитарных стохастических систем~// Кибернетика и высокие технологии XXI~века: Сб. докл.  
XIII Международного науч.-технич. конф.~--- Воронеж: Саквоее, 2012. Т.~2. C.~346--357.

\bibitem{10-sin} \Au{Синицын И.\,Н.} 
Развитие методов аналитического моделирования распределений с инвариантной мерой в стохастических сис\-те\-мах~// 
Современные проблемы прикладной математики, информатики, автоматизации, управления: Мат-лы Междунар. семинара.~--- 
Севастополь:  СевНТУ, 2012. С.~24--35.

\bibitem{11-sin} \Au{Синицын И.\,Н.} Аналитическое моделирование распределений с инвариантной мерой в стохастических 
сис\-те\-мах с автокоррелированными шумами~// Информатика и её применения, 2012. Т.~6. Вып.~4. С.~4--8.

\bibitem{12-sin} \Au{Синицын И.\,Н. }
Аналитическое моделирование распределений с инвариантной мерой в
стохастических системах с разрывными характеристиками~// Информатика
и её применения, 2013. Т.~7. Вып.~1.  С.~3--11.

\bibitem{13-sin} \Au{Ватанабэ С., Икэда Н.} Стохастические дифференциальные уравнения и диффузионные процессы.~--- М.: Наука, 1986.

\columnbreak

\bibitem{14-sin} \Au{Kloeden P., Platen E. } Numerical solution of stochastic differential equations.~--- 
Berlin\,--\,Heidelberg\,--\,New York: Springer, 1992.

\bibitem{15-sin} \Au{Артемьев С.\,С. } Численные методы решения задачи Коши для сис\-тем обыкновенных и 
стохастических дифференциальных уравнений.~--- Новосибирск: ВЦ СО РАН, 1993.

\bibitem{16-sin} \Au{Кузнецов Д.\,Ф. } Численное интегрирование стохастических дифференциальных уравнений.~--- СПб.: СПбГУ, 2001.

\label{end\stat}

\bibitem{17-sin} 
\Au{Sinitsyn I.\,N., Sinitsyn V.\,I., Korepanov~E.\,R., Belousov~V.\,V.}
Symbolic software tools for distributions parametrization in stochastic systems~// Международный
семинар по проблемам устойчивости стохастических моделей (ISSPSM-2013), 2013. С.~91--93.

\end{thebibliography}
}
}

\end{multicols}