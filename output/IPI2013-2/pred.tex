
   { %\Large  
   { %\baselineskip=16.6pt
   
   \vspace*{-24pt}
   \begin{center}\LARGE
   \textit{Предисловие}
   \end{center}
   
   %\vspace*{2.5mm}
   
   \vspace*{25mm}
   
   \thispagestyle{empty}
   
   { %\small 

      Вниманию читателей журнала <<Информатика и её применения>> предлагается 
традиционный тематический выпуск <<Вероятностно-статистические методы и задачи 
информатики и информационных технологий>>. Предыдущие тематические выпуски 
журнала по данному направлению выходили ежегодно в 2008--2012~гг. 
      
      Статьи, собранные в данном журнале, посвящены разработке и 
совершенствованию вероятностно-статистических и смежных с ними методов, 
ориентированных на применение к решению конкретных задач информатики и 
информационных технологий, а также~--- в ряде случаев~--- и других прикладных задач. 
Проблематика, охватываемая публикуемыми работами, в значительной степени 
развивается в рамках научного сотрудничества между Институтом проблем информатики 
Российской академии наук (ИПИ РАН) и факультетом вычислительной математики и 
кибернетики Московского государственного университета (МГУ) им.\ 
М.\,В.~Ломоносова в ходе работ над совместными научными проектами. Многие из 
авторов статей, включенных в данный номер журнала, являются активными участниками 
традиционного международного семинара по проблемам устойчивости стохастических 
моделей, руководимого В.\,М.~Золотаревым и В.\,Ю.~Королевым; регулярные сессии 
этого семинара проводятся под эгидой МГУ и ИПИ РАН. В~2012~г.\ указанный семинар 
проводился в сентябре в г.~Светлогорске Калининградской области РФ, в 2013~г.~--- в 
апреле в г. Москве. Среди статей, включённых в настоящий выпуск, некоторые являются 
развитием докладов, представленных на сессиях этого семинара; это~--- работы 
А.\,В.~Бородиной и Е.\,В.~Морозова; Р.\,В.~Разумчика; В.\,Е.~Бенинга, Н.\,К.~Галиевой и 
В.\,Ю.~Королева; В.\,Ю.~Королева, Л.\,М.~Закс и А.\,И.~Зейфмана.
      
      Наряду с представителями ИПИ РАН и МГУ им.\ М.\,В.~Ломоносова (факультет 
вычислительной математики и кибернетики и механико-математический факультет) в 
число авторов данного выпуска журнала входят ученые и специалисты из 
Вычислительного центра им.\ А.\,А.~Дородницына Российской академии наук (РАН), 
Института точной механики и вычислительной техники им.\ С.\,А.~Лебедева РАН, 
Института русского языка им.\ В.\,В.~Виноградова РАН, Института языкознания РАН, 
Института прикладных математических исследований Карельского научного центра 
РАН, Новосибирского государственного технического университета, Вологодского 
государственного педагогического университета, отдела моделирования и 
математической статистики Аль\-фа-бан\-ка, а также Казахстанского филиала МГУ им.\ 
М.\,В.~Ломоносова и Университета Париж-13 (Франция).
      
      Тематика статей данного выпуска включает вопросы математического 
моделирования и анализа реальных процессов и задач, в том числе моделирование 
распределений в нелинейных стохастических системах; построение и исследование 
моделей некоторых специальных систем и сетей передачи информации; задачи 
теоретической и прикладной математической статистики (включая развитие некоторых 
статистических методов анализа больших объемов текстов и создания параллельных 
поливариантных языковых корпусов). 
      
      Редакционная коллегия журнала выражает надежду, что данный тематический  
выпуск будет интересен специалистам в области теории вероятностей и математической 
статистики и их применения к решению задач информатики и информационных 
технологий.
      
      
      
      

     \vfill %vspace*{10mm}
     \noindent
     Заместитель главного редактора журнала <<Информатика и её 
применения>>,\\
     директор Института проблем информатики РАН, академик  \hfill
     \textit{И.\,А.~Соколов}\\
     
     \noindent
     Редактор-составитель тематического выпуска,\\
     профессор кафедры математической статистики\\
      факультета  вычислительной математики и кибернетики\\
 МГУ им.\  М.\,В.~Ломоносова,\\
ведущий научный сотрудник Института проблем информатики РАН,\\
     доктор физико-математических наук \hfill
      \textit{В.\,Ю.~Королев}
     
     } }
     }