\def\stat{kudr}

\def\tit{БАЙЕСОВСКАЯ РЕКУРРЕНТНАЯ МОДЕЛЬ РОСТА НАДЕЖНОСТИ: РАВНОМЕРНОЕ РАСПРЕДЕЛЕНИЕ ПАРАМЕТРОВ$^*$}

\def\titkol{Байесовская реккурентная модель роста надежности: равномерное распределение параметров}


\def\autkol{А.\,А.~Кудрявцев, И.\,А.~Соколов, С.\,Я.~Шоргин}

\def\aut{А.\,А.~Кудрявцев$^1$, И.\,А.~Соколов$^2$, С.\,Я.~Шоргин$^3$}

\titel{\tit}{\aut}{\autkol}{\titkol}

{\renewcommand{\thefootnote}{\fnsymbol{footnote}}\footnotetext[1]
{Работа выполнена при поддержке РФФИ (проекты 11-01-00515-а, 11-07-00112-а, 12-07-00109-а).}}

\renewcommand{\thefootnote}{\arabic{footnote}}
\footnotetext[1]{Факультет вычислительной
математики и кибернетики Московского государственного университета
им.\ М.\,В.~Ломоносова, nubigena@hotmail.com}
\footnotetext[2]{Институт проблем информатики Российской академии
наук, isokolov@ipiran.ru}
\footnotetext[3]{Институт проблем
информатики Российской академии наук, sshorgin@ipiran.ru}


%\input macros.tex

\Abst{Работа посвящена обоснованию целесообразности
байесовского подхода при решении задач, связанных с определением
надежности сложных модифицируемых систем. В качестве иллюстрации
приводится среднее значение надежности системы, в которой показатели
<<дефективности>> и <<эффективности>> средства, исправляющего
недостатки системы, имеют равномерное распределение.}


\KW{модифицируемые информационные системы; теория надежности; байесовский подход}

\vskip 14pt plus 9pt minus 6pt

      \thispagestyle{headings}

      \begin{multicols}{2}

            \label{st\stat}

\section{Введение}

Задача прогнозирования надежности сложных модифицируемых
информационных систем была сформулирована в~[1], а в дальнейшем
более подробно рассмотрена в~[2]. Изложение во введении строится, в
основном, на~[2].

Любой впервые созданный более или менее сложный агрегат,
предназначенный для переработки или передачи информационных потоков,
например новая программная система для компьютера, новая
информационно-вычислительная сеть или новая
ад\-ми\-ни\-стра\-тив\-но-ин\-фор\-ма\-ци\-он\-ная сис\-те\-ма, как правило, не обладает
требуемой надежностью. Для единства терминологии впредь будет
говориться о сложных информационных сис\-те\-мах. Такие сис\-те\-мы
подвергаются изменениям (модификациям) в ходе:
\begin{itemize}
\item[$\bullet$]  разработки;
\item[$\bullet$]  испытаний и опытной эксплуатации;
\item[$\bullet$]  штатного функционирования.
\end{itemize}

Модификации информационных систем на этапе разработки имеют своей
целью как уточнение задач, для решения которых предназначена
система, так и оптимальную адаптацию систем к решению этих задач.

Модификации информационных сис\-тем в ходе испытаний и опытной
эксплуатации имеют своей целью обнаружение и устранение имеющихся
дефектов, препятствующих правильному функционированию.

Модификации сложных информационных сис\-тем в ходе штатного
функционирования как результат усовершенствования их отдельных
подсистем или блоков, например замены отдельных морально устаревших
узлов более современными, имеют своей целью повышение эффективности
функционирования систем.

Целью таких модификаций является увеличение надежности
информационных систем. В~связи с этим возникает необходимость
формализации понятия надежности модифицируемых информационных систем
и разработки методов и алгоритмов оценивания и прогнозирования
различных надежностных характеристик.

Рассматриваемый круг вопросов имеет две отличительные особенности.
\begin{enumerate}[1.]
\item Из-за огромной сложности современных информационных сис\-тем
практически невозможно реализовать детерминированный подход к
тестированию системы на всех возможных вариан\-тах ее
функционирования. Поэтому, по-ви\-ди\-мо\-му, единственным возможным
подходом к исследованию таких задач является
ве\-ро\-ят\-но\-ст\-но-ста\-ти\-сти\-ческий.
\item
После каждой модификации свойства информационной системы
изменяются, и потому данные, используемые для
ве\-ро\-ят\-но\-ст\-но-ста\-ти\-сти\-че\-ско\-го анализа ее надежности, например \mbox{длины}
интервалов времени между отказами сис\-те\-мы (или между модификациями),
вообще говоря, не могут интерпретироваться как одинаково
распределенные случайные величины или, другими словами, не образуют
однородных выборок, традиционно изучаемых в классической теории
надежности. Однако методы теории вероятностей и математической
статистики позволяют найти приемлемое решение и в таких задачах.
\end{enumerate}

Первые работы, описывающие математические модели роста надежности
модифицируемых сис\-тем, появились в середине1950-х~гг., и
сейчас их чис\-ло огромно. Достаточно сказать, что по данным,
приведенным в докладе~[3], к концу 1980-х~гг.\ было
опубликовано более четырехсот научных работ только по надежности
программного обеспечения. В~то же время необходимо отметить, что
универсального метода решения задачи прогнозирования надежности
модифицируемых сис\-тем нет. Более того, до последнего времени не было
и унифицированного подхода к созданию математических моделей.
Возможно, именно поэтому модели роста надежности модифицируемых
систем обходились стороной в канонических учебниках и монографиях по
математической теории роста надежности. Такой подход был предложен в
книге~[1]. Именно на нем и основано изложение в книге~[2].

В указанной книге приводятся общие соображения, служащие основой для
построения математических моделей, описывающих изменение надежности
модифицируемых информационных систем, а затем обсуждаются конкретные
аналитические модели роста (изменения) надежности модифицируемых
информационных систем.

К числу таких моделей относятся рекуррентные модели роста
надежности. Они могут использоваться в случае, когда удобно иметь
дело непосредственно с параметром, интерпретируемым как надежность
системы. Рассмотрим произвольную систему, на вход которой подаются
некоторые сигналы (например, команды оператора или внешние
воздействия). Реакция системы на поданные сигналы может быть либо
правильной (корректной), либо неправильной (некорректной). В~каждый
момент времени~$t$ надежность системы можно характеризовать
параметром $p(t)$~--- вероятностью того, что на сигнал, поданный на
вход системы в момент~$t$, система отреагирует правильно. По смыслу
такая характеристика надежности ближе всего к традиционно
используемому коэффициенту готовности. В~случайные моменты времени
$0\hm=Y_0\hm\le Y_1 \hm\le Y_2 \le\ldots$ система подвергается (мгновенной)
модификации, в результате чего изменяется параметр $p(t)$. Следует
обратить внимание на то обстоятельство, что ниже рассматривается
непрерывное время, без привязки напрямую процесса модифицирования
системы к процессу ее тестирования. Предположим, что траектории
процесса $p(t)$ непрерывны справа и кусочно-постоянны, так что $p(t)
\hm= p(Y_j)$ при $Y_j \hm\le t\hm< Y_{j+1}$.

Задача прогнозирования поведения процесса $p(t)$ чрезвычайно важна.
Описанная выше очень общая схема может быть переформулирована в
терминах, традиционных для столь разных областей знания, как
медицина, программирование или менеджмент. Например, в
программировании параметр $p(t)$ можно рассматривать как надежность
программного  обеспечения, в которое по ходу отладки в моменты
$0\hm=Y_0\hm\le Y_1\hm \le Y_2 \le\ldots$ вносятся изменения для исправления
замеченных ошибок. Оценивание $p(t)$ и прогнозирование поведения
этого параметра здесь важно как для оценивания надежности всего
комплекса, составной частью которого является программное
обеспечение, так и для прогнозирования продолжительности отладки.
Более подробно об этом см.\ в книгах~[1, 4]. В~медицине параметр
$1\hm - p(t)$ (называемый индексом ле\-таль\-ности) характеризует
вероятность летального исхода в момент времени~$t$ для пациента,
организм которого в моменты $0\hm=Y_0\hm\le Y_1 \hm\le Y_2 \le\ldots$
подвергается какому-либо медицинскому вмешательству (операции,
инъекции, приему лекарств и т.\,п.). Здесь прогнозирование $p(t)$
чрезвычайно важно с точки зрения принятия решений о стратегии
лечения. Наконец, в менеджменте параметр $p(t)$ может
характеризовать надежность и дееспособность коллектива, организации
или предприятия, структура которых в моменты времени $0\hm=Y_0\hm\le Y_1
\hm\le Y_2 \le\ldots$ претерпевает изменения. Будем предполагать, что в
результате каждой модификации системы параметр $p(t)$ изменяется
случайным  образом.

Обозначим $p_j \hm= p(Y_j).$ Рассмотрим поведение $p_j$ в зависимости
от изменения~$j$. Другими словами, будем изучать изменение
надежности системы в зависимости от номера модификации.

В книге~[2] рассматривается, в частности, следующая рекуррентная
модель роста надежности. Пусть  $\{(\theta_j, \eta_j)\}$, $j\hm\ge1$,~--- 
последовательность независимых одинаково распределенных двумерных
случайных векторов таких, что $0 \hm< \eta_1 \hm< 1$; $0 \hm< \theta_1 \hm < 1$
почти наверное. Отметим, что независимость и совпадение
распределений $\theta_j$ и $\eta_j$ внутри каждой пары не
предполагается.

Задав начальную надежность $p_0$, рассмотрим модель, определяемую рекуррентным соотноше\-нием
\begin{equation}
p_{j+1} = \eta_{j+1}p_j + \theta_{j+1}(1-p_j)\,.\label{e1-kud}
\end{equation}
Эта модель названа дискретной экспоненциальной моделью. В~такой
модели случайные величины $\eta_j$ описывают возможное уменьшение
надежности из-за некачественных модификаций, в ходе которых вместо
исправления существующих дефектов в систему могут быть внесены
новые, в то время как величины $\theta_j$ описывают повышение
на\-деж\-ности за счет исправления дефектов. Част\-ные случаи модели~(1) с
двухточечными распределениями случайных величин $\eta_j$ и
$\theta_j$  рассматривались в книгах~[5, 6]. В~свою очередь, эти
частные случаи представляют собой переформулировку в терминах теории
надежности одной модели обучаемости, рассмотренной в книге~[7].

Обозначим $\lambda = 1 \hm- \e\theta_j$, $\mu \hm = \e\eta_j$.
В [2] доказано, что при условии $\lambda+\mu\neq1$
$$
p=\lim\limits_{j\to\infty}\e p_j = \fr{\mu}{\lambda+\mu}\,.
$$

\section{Постановка задачи}

Изучение предельного значения средней величины  $\e p_j$
представляет значительный интерес, посколь\-ку эта величина
характеризует асимптотическое значение надежности системы  в рамках
некоторой рекуррентной модели, задаваемой набором $\{(\theta_j,
\eta_j)\}$. Из результатов~[2] следует, что это асимптотическое
значение зависит только от средних значений величин $\{(\theta_j,
\eta_j)\}$, $j\hm\ge1$.

Каков <<физический смысл>> указанных случайных величин? Очевидно,
что величины $\{\eta_j\}$ являются количественной характеристикой
недостатков  в работе средств (устройств, людей и~т.\,п.), отвеча\-ющих
за исправление дефектов, а величины  $\{\theta_j\}$~---
количественной характеристикой успеш\-ности работы таких средств
(устройств, людей и~т.\,п.). Если говорят об одной модифицируемой
сис\-те\-ме или объекте, то можно считать  (в рамках рас\-смат\-ри\-ва\-емой
модели), что речь идет об одной и той же группе средств  и~т.\,п.
(будем условно говорить~--- об одной ремонтной бригаде (РБ)), которая
занимается исправлением дефектов данного сложного объекта в течение
достаточно длительного времени. В~этом случае естественно ставить
задачу вычисления средних параметров $\lambda$ и~$\mu$,
характеризующих данную РБ, и на основе этих
параметров вычислять величину~$p$.

Однако можно представить себе более сложную ситуацию, при которой
рассматривается целый набор однотипных сложных модифицируемых
объектов (МО), каждый из которых обслуживается собственной РБ. 
Исследователю хотелось бы определить усредненное
значение~$p$ по всем МО. Для решения этой задачи целесообразно
перейти к так называемой байесовской постановке.

Будем считать, что рассматривается целая группа однотипных МО и
группа им соответствующих однотипных РБ. Пусть $m\hm=1,2,\ldots$~---
номера этих объектов. Для каждого МО (вместе с его РБ) существует
собственный набор $\{(\theta_j^m, \eta_j^m)\}$ ($j\hm\ge1$, $m\hm\ge1$)
независимых одинаково распределенных при каждом фиксированном $j$
двумерных случайных векторов таких, что $0 \hm< \eta_1^m \hm< 1$; $0 \hm<
\theta_1^m  \hm< 1$ почти наверное. Но средние значения величин
$\theta_j^m$, $\eta_j^m$, $j\hm\ge1$, $m\hm\ge1$, не предполагаются
известными; более того, они не предполагаются даже одинаковыми.
Вводится предположение, что величины $\lambda \hm= 1 \hm- \e\theta_j^m$,
$\mu \hm= \e\eta_j^m$ сами по себе являются случайными, т.\,е.\ на
вероятностном пространстве, в которое в качестве элементарных
событий входят все рассматриваемые в рамках данной постановки МО
вместе с их РБ, заданы случайные величины $\lambda$ и~$\mu$ (которые
будем полагать независимыми), имеющие смысл $\lambda \hm= 1 \hm-
\e\theta_j^m$, $\mu \hm= \e\eta_j^m$, где $m$~--- случайный номер МО.
Принимаемые исследователем за основу распределения величин $\lambda$
и~$\mu$ будем называть априорными.

При этом подлежащие вычислению характеристики такой
<<рандомизированной>> группы МО, естественно, являются рандомизацией
аналогичных характеристик <<отдельно взятой>> МО с учетом априорного
распределения параметров $\lambda$ и~$\mu$, взятого исследователем
за основу. Наиболее естественной и удобной для изучения
характеристикой является усредненное по всем МО значение предельной
вероятности надежности, т.\,е.\
$$
p_{\mathrm{сред}} = \e p = \e\fr{\mu}{\lambda+\mu}\,,
$$
где усреднение ведется по совместному распределению случайных
величин $(\lambda,\mu)$.

Данный подход близок к байесовскому анализу систем массового
обслуживания, которому посвящены статьи~[8--12], но имеет следующее
существенное отличие. В~рамках  байесовской постанов\-ки в моделях
массового обслуживания возникали\linebreak задачи рандомизации <<обычных>>
характеристик сис\-тем массового обслуживания с учетом весьма
разнообразных априорных распределений входных параметров; скажем,
принималось предположение о показательном, эрланговском, равномерном
и многих других  распределениях одного или не-\linebreak скольких параметров
входящих потоков и/или обслуживания.  В~рассматриваемой сейчас
ситуации величины $\eta_1^m$ и $\theta_1^m$ удовлетворяют
ограничениям $0 \hm< \eta_1^m \hm< 1$, $0 \hm< \theta_1^m \hm < 1$. Значит, и
средние значения $\lambda$ и~$\mu$ величин $1 \hm- \e\theta_j$ и
$\e\eta_j$ соответственно также находятся на отрезке $[0,1]$.
Поэтому в качестве априорных распределений параметров $\lambda$ и~$\mu$ следует выбирать только распределения, сосредоточенные на
$[0,1]$.

В настоящей заметке будут рассмотрены независимые случайные
параметры $\lambda$ и~$\mu$, распределенные равномерно на некоторых
(вообще \mbox{говоря}, разных) отрезках, являющихся подмножествами отрезка
$[0,1]$.

\section{Предварительные выкладки}

Итак, пусть случайная величина $\lambda$ имеет равномерное
распределение на отрезке $[a_\lambda,\ b_\lambda]$, а случайная
величина $\mu$~--- равномерное распределение на отрезке $[a_\mu,\
b_\mu]$, причем $0\hm\le a_\lambda\hm\le b_\lambda\hm\le1$, $0\hm\le a_\mu\hm\le
b_\mu\hm\le1$.

Как уже отмечалось, надежность системы~$p$ имеет формально такой же
вид, как и коэффициент готовности в системе массового обслуживания
${M}|{M}|1|0$ с интенсивностью входящего потока~$\lambda$ и интенсивностью
обслуживания~$\mu$. Очевидно, что при байесовском подходе вычисление
вероятностных характеристик~$p$ удобно производить, базируясь на
известном распределении величины $\rho\hm=\lambda/\mu$, которая для
моделей массового обслуживания имеет смысл коэффициента загрузки
системы. При этом $p\hm=1/(1\hm+\rho)$.

В работе~[9] были получены формулы для плотности и функции
распределения коэффициента загрузки~$\rho$. Приведем их здесь для
удобства дальнейшего изложения для случая $a_\lambda/a_\mu\hm\le
b_\lambda/b_\mu$.

Обозначим
$K\hm=((b_\lambda\hm-a_\lambda)(b_\mu-a_\mu))^{-1}.$
Тогда (см.~[9]) 
\begin{multline*}
\p(\rho<x)={}\\
{}=
\begin{cases}
0 &\hspace*{-10mm}\mbox{при}\ x< \fr{a_\lambda}{b_\mu}\,;\\[3pt]
K\fr{(b_\mu x-a_\lambda)^2}{2x} &\hspace*{-10mm}\mbox{при}\ \fr{a_\lambda}{b_\mu}\le x\le \fr{a_\lambda}{a_\mu}\,;\\[3pt]
K\left(\fr{a_\mu+b_\mu}{2}\,x-a_\lambda\right)\left(b_\mu-a_\mu\right) &\\
&\hspace*{-10mm}\mbox{при}\ \fr{a_\lambda}{a_\mu}\le x\le \fr{b_\lambda}{b_\mu}\,;\\[3pt]
1-K\fr{(b_\lambda-a_\mu x)^2}{2x} &\hspace*{-10mm}\mbox{при}\ \fr{b_\lambda}{b_\mu}\le x\le \fr{b_\lambda}{a_\mu}\,;\\
1 &\hspace*{-10mm}\mbox{при}\ x>\fr{b_\lambda}{a_\mu}\,.
\end{cases}
\end{multline*}

Выпишем плотность случайной величины $\rho$:
$$
f_\rho(x)=
\begin{cases}
0 & \mbox{при}\ x< \fr{a_\lambda}{b_\mu}\ \mbox{и}\  x>\fr{b_\lambda}{a_\mu}\,;\\[3pt]
K\left(\fr{b_\mu^2}{2}-\fr{a_\lambda^2}{2x^2}\right) & \mbox{при}\ \fr{a_\lambda}{b_\mu}\le x\le \fr{a_\lambda}{a_\mu}\,;\\[3pt]
K\fr{b_\mu^2-a_\mu^2}{2} & \mbox{при}\ \fr{a_\lambda}{a_\mu}\le x\le \fr{b_\lambda}{b_\mu}\,;\\[3pt]
K\left(\fr{b_\lambda^2}{2x^2}-\fr{a_\mu^2}{2}\right) & \mbox{при}\ \fr{b_\lambda}{b_\mu}\le x\le \fr{b_\lambda}{a_\mu}\,.
\end{cases}
$$


\section{Основные результаты}


Основываясь на характеристиках распределения параметра~$\rho$ и учитывая, что $p\hm=1/(1\hm+\rho),$
имеем для функции распределения и плотности~$p$ следующие соотношения:
$$
F_p(x)=1-F_\rho\left(\fr{1-x}{x}\right)\,; \quad 
f_p(x)=\fr{1}{x^2}\,f_\rho\left(\fr{1-x}{x}\right)\,,
$$
а следовательно, для случая $a_\lambda/a_\mu\hm\le b_\lambda/b_\mu$ имеем:
$$
F_p(x)=
\begin{cases}
0 &  \hspace*{-53mm}\mbox{при}\ x<\fr{a_\mu}{a_\mu+b_\lambda}\,;\\[3pt]
1-K\fr{((a_\mu+b_\lambda)x-a_\mu)^2}{2x(1-x)} & \\
&\hspace*{-53mm}\mbox{при}\ \fr{a_\mu}{a_\mu+b_\lambda}\le x\le \fr{b_\mu}{b_\mu+b_\lambda}\,;\\[3pt]
1-K(b_\mu-a_\mu)\!\left(\!\fr{(a_\mu+b_\mu)(1-x)}{2x}-a_\lambda\!\right)\hspace*{-1.07954pt} &\hspace*{-1.07954pt} \\
&\hspace*{-53mm}\mbox{при}\ \fr{b_\mu}{b_\mu+b_\lambda}\le x\le 
\fr{a_\mu}{a_\mu\hm+a_\lambda}\,;\\[3pt]
1-K\fr{((b_\mu+a_\lambda)x-b_\mu)^2}{2x(1-x)} & \\
&\hspace*{-53mm}\mbox{при}\ \fr{a_\mu}{a_\mu+a_\lambda}\le x\le \fr{b_\mu}{b_\mu\hm+a_\lambda}\,;\\[3pt]
1 & \hspace*{-53mm}\mbox{при}\ x>\fr{b_\mu}{b_\mu+a_\lambda}\,.
\end{cases}
$$

Для плотности 
$$
f_p(x)=
\begin{cases}
0 & \hspace*{-30mm}\mbox{при}\ x\fr{<a_\mu}{a_\mu\hm+b_\lambda}\ \mbox{и}\ x>\fr{b_\mu}{b_\mu\hm+a_\lambda}\,;\\
K\left(\fr{b_\lambda^2}{2(1-x)^2}-\fr{a_\mu^2}{2x^2}\right) &\\
&\hspace*{-30mm}\mbox{при}\ \fr{a_\mu}{a_\mu+b_\lambda}\le x\le \fr{b_\mu}{b_\mu+b_\lambda}\,;\\
K\left(\fr{b_\mu^2}{2x^2}-\fr{a_\lambda^2}{2(1-x)^2}\right) &\\
&\hspace*{-30mm}\mbox{при}\ \fr{a_\mu}{a_\mu+a_\lambda}\le x\le \fr{b_\mu}{b_\mu+a_\lambda}\,.
\end{cases}
$$

Теперь можно легко найти интересующую величину $p_{\mathrm{сред}}=\e p$. Имеем:
\begin{multline}
p_{\mathrm{сред}}=\il{-\infty}{\infty}x f_p(x)\,dx={}\\
{}=\fr{1}{2}+\fr{Kb_\mu^2}{2}\ln\fr{b_\mu+b_\lambda}{b_\mu+a_\lambda}
-\fr{Ka_\mu^2}{2}\ln\fr{a_\mu+b_\lambda}{a_\mu+a_\lambda}+{}\\
{}+\fr{Kb_\lambda^2}{2}
\ln\fr{a_\mu+b_\lambda}{b_\mu+b_\lambda}-\fr{Ka_\lambda^2}{2}\ln\fr{a_\mu+a_\lambda}{b_\mu+a_\lambda}\,.\label{e2-kud}
\end{multline}

При соотношении параметров $a_\lambda/a_\mu\hm\ge b_\lambda/b_\mu$  аналогичные выкладки приводят к такому же результату.

Заметим, что полученный результат остается справедливым для случая
$a_\lambda\hm=b_\lambda$, $a_\mu\hm=b_\mu$, соответствующего вырожденному
распределению па\-ра\-мет\-ров $\lambda$ и~$\mu$. При этом, очевидно,
$p_{\mathrm{сред}}\hm=1/2$.\linebreak\vspace*{-12pt} 

\pagebreak

\end{multicols}

\begin{table}\small
\begin{center}
\begin{tabular}{|c|c|c|c|c|c|c|c|c|c|c|}
\multicolumn{11}{c}{Частные значения средней надежности}\\[6pt]
\hline
& \multicolumn{10}{c|}{$\mu$}\\
\cline{2-11}
\multicolumn{1}{|c|}{\raisebox{6pt}[0pt][0pt]{$\lambda$}}&$[0,1/4]$&$[0,1/2]$&$[0,3/4]$&$[0,1]$&$[1/4,1/2]$&$[1/4,3/4]$&$[1/4,1]$&$[1/2,3/4]$&$[1/2,1]$&$[3/4,1]$\\
\hline
$[0,1/4]$&0,50&0,63&0,70&0,74&0,76&0,80&0,83&0,84&0,86&0,88\\
%\hline
$[0,1/2]$&0,37&0,50&0,58&0,63&0,63&0,68&0,72&0,73&0,76&0,79\\
%\hline
$[0,3/4]$&0,30&0,42&0,50&0,56&0,55&0,60&0,64&0,66&0,69&0,72\\
%\hline
$[0,1]$&0,25&0,37&0,44&0,50&0,48&0,54&0,58&0,60&0,63&0,67\\
%\hline
$[1/4,1/2]$&0,24&0,37&0,45&0,52&0,50&0,56&0,61&0,63&0,66&0,70\\
%\hline
$[1/4,3/4]$&0,20&0,32&0,40&0,46&0,44&0,50&0,55&0,56&0,60&0,64\\
%\hline
$[1/4,1]$&0,17&0,28&0,36&0,42&0,39&0,46&0,50&0,51&0,55&0,60\\
%\hline
$[1/2,3/4]$&0,16&0,27&0,34&0,40&0,37&0,44&0,49&0,50&0,54&0,58\\
%\hline
$[1/2,1]$&0,14&0,24&0,31&0,37&0,34&0,40&0,45&0,46&0,50&0,54\\
%\hline
$[3/4,1]$&0,12&0,21&0,28&0,33&0,30&0,36&0,40&0,42&0,46&0,50\\
\hline
\end{tabular}
\end{center}
\end{table}

\begin{multicols}{2}

\noindent
Такое же значение величины
$p_{\mathrm{сред}}$ получаем в случае, когда оба па\-ра\-мет\-ра
$\lambda$ и~$\mu$ равномерно распределены на отрезке $[0,1]$.
Приведем еще несколько частных значений (с точ\-ностью до сотых)
средней надежности, основываясь на формуле~(2). В~таб\-ли\-це
строкам соответствуют отрезки, на которых сосредоточено
распределение $\lambda$, а столбцам~--- распределение~$\mu$.



\section{Заключение}

Полученные результаты могут применяться, например, для вычисления
других моментов и построения доверительных интервалов для
характеристики~$p$.

В дальнейшем предполагается рассмотреть более широкий класс
априорных распределений величин $\lambda$ и~$\mu$, сосредоточенных
на $[0,1]$, в частности бе\-та-рас\-пре\-де\-ление.

{\small\frenchspacing
{%\baselineskip=10.8pt
\addcontentsline{toc}{section}{Литература}
\begin{thebibliography}{99}


\bibitem{GK}
\Au{Gnedenko B.\,V., Korolev V.\,Yu.} Random summation: Limit
theorems and applications.~--- Boca Raton, FL: CRC Press, 1996.

\bibitem{KS}
\Au{Королев В.\,Ю., Соколов И.\,А.} Основы математической теории
надежности модифицируемых сис\-тем.~--- М.: ИПИ РАН, 2006.

\bibitem{VanРul1990)}
\Au{Van Pul M.\,C.} Asymptotic properties of statistical models
in software reliability~// 2nd Bernoulli Society World Congress:
Abstracts of communications.~--- Uppsala, 1990. P.~43--44.

\bibitem{K1997}
\Au{Королев В.\,Ю.} Прикладные задачи теории вероятностей:
модели роста надежности модифицируемых сис\-тем.~--- М.: Диалог-МГУ,
1997.

\bibitem{VolShi75}
\Au{Волков Л.\,И., Шишкевич А.\,М.} Надежность летательных
аппаратов.~--- М.: Высшая школа, 1975.

\bibitem{Vol81}
\Au{Волков Л.\,И.} Управление эксплуатацией летательных
комплексов.~--- М.: Высшая школа, 1981.

\bibitem{BuMo62}
\Au{Буш Р., Мостеллер Ф.} Стохастические модели обучаемости.~--- М.: ГИФМЛ, 1962.

\bibitem{Shorgin05}
\Au{Шоргин С.\,Я.}
О байесовских моделях массового обслуживания~//
II Научная сессия Института проблем информатики РАН: Тезисы докладов.~--- М.: ИПИ РАН, 2005. С.~120--121.

\bibitem{KuSh08}
\Au{Кудрявцев А.\,А., Шоргин С.\,Я.} Байесовский подход к
анализу систем массового обслуживания и показателей надежности~//
Информатика и её применения, 2007. Т.~1. Вып.~2. С.~76--82.

\bibitem{KuSh09a}
\Au{Кудрявцев А.\,А., Шоргин С.\,Я.} Байесовские модели
массового обслуживания и надежности: экспо\-нен\-ци\-аль\-но-эрлан\-гов\-ский
случай~// Информатика и её применения, 2009. Т.~3. Вып.~1. С.~44--48.

\bibitem{KuSh09b}
\Au{Кудрявцев А.\,А., Шоргин В.\,С., Шоргин~С.\,Я.} Байесовские
модели массового обслуживания и надежности: общий эрланговский
случай~// Информатика и её применения, 2009. Т.~3. Вып.~4. С.~30--34.

\label{end\stat}

\bibitem{KuSh10}
\Au{Кудрявцев А.\,А., Шоргин С.\,Я.} Байесовские модели
массового обслуживания и надежности: характеристики среднего числа
заявок в системе $M|M|1|\infty$~// Информатика и её применения, 2010.
Т.~4. Вып.~3. С.~16--21.
\end{thebibliography}
}
}

\end{multicols}