\def\stat{sharapov}

\def\tit{УНИВЕРСАЛЬНАЯ СИСТЕМА ПРОВЕРКИ ТЕКСТОВ НА~ПЛАГИАТ 
<<АВТОР.NET>>}

\def\titkol{Универсальная система проверки текстов на~плагиат 
<<Автор.NET>>}

\def\autkol{Е.\,В.~Шарапова, Р.\,В.~Шарапов}
\def\aut{Е.\,В.~Шарапова$^1$, Р.\,В.~Шарапов$^2$}

\titel{\tit}{\aut}{\autkol}{\titkol}

%{\renewcommand{\thefootnote}{\fnsymbol{footnote}}\footnotetext[1]
%{Работа поддерживается РГНФ, проект 11-02-12026-в.}}


\renewcommand{\thefootnote}{\arabic{footnote}}
\footnotetext[1]{Владимирский государственный университет имени Александра Григорьевича и Николая Григорьевича 
Столетовых,\linebreak mivlgu@mail.ru}
\footnotetext[2]{Владимирский государственный университет имени Александра Григорьевича и Николая Григорьевича 
Столетовых,\linebreak info@vanta.ru}

\vspace*{2pt}

\Abst{Обсуждается проблема обнаружения в текстах заимствований из других 
источников. Рассматриваются основные подходы к обнаружению заимствований, 
проводится обзор существующих на сегодняшний день программ. Дается обзор методов к 
сокрытию фактов заимствований. Дается описание разработанной сис\-те\-мы 
<<Автор.NET>>, способной проводить проверку заимствований по внутренним источникам 
и сети Интернет.}

\vspace*{2pt}

\KW{плагиат; обнаружение плагиата; заимствование}

%\vspace*{6pt}

\vskip 14pt plus 9pt minus 6pt

      \thispagestyle{headings}

      \begin{multicols}{2}

            \label{st\stat}


\section{Введение}
  
  Бурное развитие вычислительной техники привело к глубокому 
проникновению компьютеров в нашу жизнь. Компьютеры окружают нас 
везде~--- на работе, дома, в магазинах и общественных мес\-тах. Современное 
развитие информационных технологий и глобальной сети Интернет 
предоставило широким кругам пользователей доступ к огромным массивам 
информации. Появилось большое чис\-ло он\-лайн-биб\-лио\-тек, содержащих в 
электронном виде художественную и научно-техническую литературу. Стало 
возможным читать книги, новости и газеты непосредственно с экрана 
компьютера. 
  
  В сети Интернет стало доступно множество методических указаний, курсов 
лекций, учебников и~т.\,д. Кроме того, появились огромные коллекции 
рефератов, готовых лабораторных работ, курсовых и дипломных проектов и 
даже диссертаций. Использование компьютерной техники сильно облегчило 
задачу поиска и копирования подобной информации. Если раньше для 
написания реферата или контрольной работы информацию было нужно, по 
крайней мере, найти в книгах и переписать (вручную, перепечатать или ввести 
в компьютер с помощью сканера и программ распознавания текстов), то теперь 
достаточно ввести название темы в поисковую сис\-те\-му и скопировать 
найденные материалы. Стал распространяться метод написания работ, 
получивший название <<{Copy}\,\&\,{Paste}>>. Метод заключается в 
простом копировании информации из одного или нескольких источников с 
минимальным редактированием получающегося таким образом текста. 
  
  Аналогичная ситуация наблюдается с отчетными материалами внутри 
учебных заведений. В~связи с тем, что большое число пояснительных записок 
по курсовым и дипломным проектам выполняется с использованием 
компьютеров, происходит их распространение и повторное использование 
среди учащихся.
  
  В последнее время наблюдается бурный рост использования в учебном 
процессе подобной заимствованной информации. Ситуация усугубляется тем, 
что учащиеся иногда не знают (не читают) то, что написано в <<их>> работах. 
  
  Плагиат~--- умышленное присвоение авторства на чужое произведение 
литературы, науки, искусства, изобретение или рационализаторское 
предложение (полностью или частично)~[1].
  
  Как можно убедиться из определения, подобные заимствованные работы 
можно отнести к разряду плагиата. Задача обнаружения недобросовестного 
использования заимствованных текстов в учебных и ученых кругах (фактов 
плагиата) приобретает высокую актуальность.

\begin{table*}\small
\begin{center}
\Caption{Формы плагиата}
\vspace*{2ex}

\begin{tabular}{|l|c|}
\hline
\multicolumn{1}{|c|}{Форма плагиата}&Доля\\
\hline
Полное или частичное копирование текста из одного источника&36\%\\
Копирование и компоновка текста из нескольких источников&62\%\\
Копирование текста из другого источника и изменение порядка следования частей 
текста&\hphantom{9}2\%\\
\hline
\end{tabular}
\end{center}
%\end{table*}
%\begin{table*}\small
\begin{center}
\Caption{Частота использования подходов к сокрытию фактов плагиата}
\vspace*{2ex}

\tabcolsep=8pt
\begin{tabular}{|l|c|}
\hline
Подходы к сокрытию плагиата&Доля\\
\hline
Корректировка родов, чисел и времен, входящих в текст слов&32\%\\
Незначительное изменение текста&38\%\\
Сокращение заимствованного текста&44\%\\
Замена букв&\hphantom{9}4\%\\
Синонимизация текста&\hphantom{9}2\%\\
\hline
\end{tabular}
\end{center}
\end{table*}

\section{Формы заимствований текстов}

  Рассмотрим формы заимствований, встре\-ча\-ющи\-еся в практике учебных 
заведений и подлежащие выявлению. 
  \begin{enumerate}[1.]
  \item Полное или частичное копирование текста из одного источника. 
  \item Копирование и компоновка текста из нескольких источников.
  \item Копирование текста из другого источника и его частичное 
редактирование.
  \end{enumerate}
  
  Для того чтобы скрыть факт заимствований, могут применяться следующие 
подходы:
  \begin{enumerate}[1.]
  \item Корректировка родов, чисел и времен входящих в текст слов. 
Например, замена слова <<выполнил>> на <<выполнила>> или 
<<выполнили>>, использование местоимения <<я>> вместо <<мы>> в 
оригинальном тексте и~т.\,д.
  \item Незначительное изменение заимствованного текста.
\item  Сокращение заимствованного текста путем удаления слов, предложений, 
абзацев, рисунков, формул и~т.\,д.
  \item Обход сис\-тем проверки на плагиат путем замены русских букв на 
аналогичные по написанию английские и~т.\,д.
  \item Осуществление ручной или автоматической синонимизации текста.
  \end{enumerate}
  
  Все вышеописанное должно учитываться при создании и использовании 
  сис\-тем проверки на заимствования. О~правомочности того или иного 
заимствования решение выносит сам проверяющий.
  
  Для оценки частоты использования тех или иных форм плагиата мы провели 
следующий эксперимент. Студентам двух групп гуманитарных специальностей 
было предложено написать статьи на тему экологической ситуации в регионе 
(Владимирская область). Студенты были предупреждены о том, что статьи 
будут проверяться на наличие плагиата. Из полученного набора были 
исключены оригинальные статьи. Анализ статей, содержащих заимствованный 
контент, показал, что большинство из них скомпонованы из нескольких (реже 
одного) источников, чаще всего из учебников, статей из сети Интернет и 
публикаций региональной прессы (табл.~1). Тот факт, что доля статей, 
полностью или частично скопированных только из одного источника, 
составила всего 36\% (в реальных условиях она часто бывает больше), вероятно 
связан со знанием авторов о том, что работы будут проверяться. Доля работ, 
составленных путем копирования текста из другого источника и изменения 
порядка следования частей текста, оказалась незначительной (2\%).



  Анализ подходов, используемых студентами для сокрытия факта плагиата, 
показал, что в 32\% работ осуществлялась корректировка родов, чисел и времен 
слов (табл.~2). В~38\% работ (составленных как из одного, так и из 
нескольких источников) осуществлялось незначительное изменение 
заимствованного текста. Так, например, делались вставки слов и предложений в 
заимствованный текст, подвергались изменению названия населенных пунктов 
и рек (р.~Волга в оригинале заменялась на р.~Ока в статье). Надо заметить, что 
часть работ кроме заимствованных текстов содержала оригинальные блоки, 
чаще всего введение и заключение. Приведенная выше доля статей, 
подвергавшихся изменению, учитывает только заимствованные части таких 
текстов. Из работ, скопированных из одного источника, 44\% подвергались 
сокращению. В~данном случае под сокращением подразумевалось исключение 
части предложений, графиков, рисунков из заимствованных текстов, а также 
исключение начальных или конечных блоков текста, по смыслу составляющих 
единое целое с заимствованным фрагментом. Копирование законченного 
фрагмента из текста (например, раздела или главы) сокращением не считалось. 
Замена букв осуществлялась в 4\% работ. В~одной из работ замене подверглись 
практически все русские буквы, сходные по написанию с английскими 
буквами. В~остальных работах заменялись одна--две гласные буквы. Ручная 
синонимизация проводилась только в 2\% работ. Применения автоматической 
синонимизации в статьях замечено не было. Надо заметить, что около 40\% 
рассматриваемых работ вообще не подвергались каким-либо изменениям, 
призванным скрыть факты плагиата.



\section{Подходы к~обнаружению заимствований}

  Существует несколько подходов к обнаружению заимствований (или, как их 
еще называют, нечетких дублей текстов). Достаточно подробный обзор 
приведен в~[2].
  
  Наибольшую известность получил метод <<шинглов>>~[3]. Метод основан 
на представлении текстов в виде множества последовательностей 
фиксированной длины, состоящих из соседних слов. При значительном 
пересечении таких множеств документы будут похожи друг на друга. Одна из 
модификаций метода, получившая название <<супершинглов>>, используется 
для быстрого обнаружения подобных документов~[2].
  
  Существует ряд методов, использующих сигнатурную лексическую 
информацию документов. В~[4] для этих целей используется {I-Match} 
сигнатура, вычисляемая для слов со средним значением {IDF} (инверсной 
частоты слов в документах). Другим сигнатурным подходом, основанным на 
лексических принципах, является метод <<опорных>> слов~[5]. В~данном 
случае для документов со\-став\-ля\-ют\-ся по определенным правилам наборы 
опорных слов, для которых строятся сигнатуры документов. Совпадение 
сигнатур говорит о подобии самих документов. Эта группа методов, несмотря 
на большую сложность реализации, показывает более хорошие результаты в 
обнаружении похожих документов~[2].
  
  Для обнаружения заимствований иногда используются алгоритмы, 
построенные на классических принципах информационного поиска, таких как 
{TF}, {TF*IDF} и~т.\,д.~[6]. В~[7] предлагается использовать 
функцию схожести Джаккарда, применение которой позволяет добиться 
неплохих результатов даже в текстах с использованием синонимов и наличием 
орфографических ошибок.

\section{Обзор существующих систем}

  Рассмотрим практическое использование описанных подходов в задачах 
обнаружения плагиата. В~настоящее время существует достаточно большое 
количество сервисов и программ, позволяющих так или иначе выявить 
заимствованный контент. Большую известность получила сис\-те\-ма 
<<Антиплагиат>>, разработанная компанией <<Форексис>>~[8]. Сис\-те\-ма 
осуществляет поиск по большому количеству коллекций рефератов, 
контрольных работ и учебников, хранящихся в собственной базе сис\-те\-мы. 
Тем не менее сис\-те\-ма имеет ряд недостатков. Во-пер\-вых, она не 
осуществляет поиск по всем документам, доступным в сети Интернет. 
Особенно это касается тематических сайтов и новостных порталов: большое 
число заимствований осуществляется именно из таких источников. 
Соответственно, даже при полном дублировании подобной информации, 
  сис\-те\-ма <<Антиплагиат>> соответствий не обнаружит. Во-вто\-рых, 
присутствует ограничение размера проверяемого текста 3000 или 
5000~символами (доступно после регистрации). В-третьих, ограничен просмотр 
документов, частично соответствующих проверяемому тексту. Кроме того, 
сис\-те\-ма ограничивает возможность проверки по базе имеющихся работ.
  
  Программа {Advego Plagiatus} осуществляет проверку с применением 
поисковых сис\-тем~[9]. Использует разные поисковые сис\-те\-мы и проверяет 
их доступность. В~отличие от аналогичных сис\-тем, {Advego Plagiatus} 
не использует Яндекс.XML, а обращается напрямую к таким поисковым 
сис\-те\-мам, как Яндекс, {Google}, {Bing}, Рамблер, {Yahoo}, 
\mbox{Поиск@Mail.ru}, {Nigma}, {QIP}. Качество обнаружения 
плагиата достаточно высокое. Программа выдает процент совпадения текста и 
выводит найденные источники. Недостатком является отсутствие 
преобразования букв, отсутствие поддержки поиска по собственной базе. Из-за 
особенностей работы программы возникают ситуации, когда результаты 
проверки отличаются от раза к разу. 
  
  Сервис {\sf www.miratools.ru} позволяет осуществлять онлайн-проверку 
текста на плагиат~[10]. Сис\-те\-ма использует результаты выдачи поисковых 
сис\-тем. К~достоинствам можно отнести возможность замены английских букв 
на русские. Имеются возможности изменять длину и шаг шинглов 
(используемых для проверки). По результатам проверки выдается процент 
совпадений и найденные источники. Сис\-те\-ма не работает с собственной 
базой. Присутствует ограничение на длину текста в 3000~символов и на число 
проверок в течение суток. 

\begin{table*}\small %tabl3
\begin{center}
\Caption{Сравнение функциональности сервисов проверки текстов на плагиат}
\vspace*{2ex}

\begin{tabular}{|l|c|c|c|c|}
\hline
\multicolumn{1}{|c|}{Система}&
\tabcolsep=0pt\begin{tabular}{c}Поиск\\ в Интернете\end{tabular}&
\tabcolsep=0pt\begin{tabular}{c}Поиск\\ в локальной  базе\end{tabular}&
\tabcolsep=0pt\begin{tabular}{c}Обработка\\ замены букв\end{tabular}&
\tabcolsep=0pt\begin{tabular}{c}Подробный\\ отчет\end{tabular}\\
\hline
{Advego Plagiatus}&$+$&$-$&$-$&$+$\\
<<Антиплагиат>>&$-$&$+$&$-$&$-$\\
{Istio}&$+$&$-$&$-$&$-$\\
{Miratools}&$+$&$-$&$+$&$+$\\
{Plagiat-inform}&$+$&$+$&$-$&$+$\\
{Praide Unique Content Analyser~II}&$+$&$-$&$-$&$+$\\
\hline
\end{tabular}
\end{center}
\end{table*}
  
  Сервис {\sf www.istio.com} осуществляет проверку текста на наличие 
заимствованного контента с использованием поисковых сис\-тем~[11]. Для этих 
целей используют Яндекс.XML и Yahoo.com. Возможности 
сервиса несколько слабее по сравнению с Miratools. По результатам 
проверки выдается сообщение о том, является ли текст уникальным или нет, и 
выдается список подобных сайтов. Преобразование букв и поддержка поиска 
по собственной базе отсутствует. Сервис предоставляет дополнительные 
средства для анализа текстов, например проверку орфографии, анализ наиболее 
частотных слов и~т.\,д.
  
  Программа Praide Unique Content Analyser~II~[12] имеет широкие 
возможности по проверке текстов с использованием поисковых сис\-тем. 
Имеется возможность выбора используемых поисковых сис\-тем, содержит 
средства добавления новых поисковых сис\-тем. Проверка осуществляется 
пассажами и шинглами, длину которых можно изменять. Можно задавать 
количества слов перекрытия шинглов. Выводится подробный отчет по проверке 
в каждой поисковой сис\-те\-ме. К~недостаткам можно отнести отсутствие 
замены букв и обработки стоп-слов. Нет поддержки работы с собственной 
базой.
  
  Система {Plagiatinform}, по заверениям авторов, имеет наиболее 
широкий функционал~[13, 14]. Она умеет проверять документы на наличие 
заимствований как в локальной базе, так и в сети Интернет. Сис\-те\-ма умеет 
обрабатывать документы, скомпонованные из перемешанных кусков текста 
нескольких источников. Проверка может осуществляться с использованием 
быстрого или углубленного поиска. Результаты проверки выдаются в виде 
наглядного отчета. Авторы не предоставляют возможности свободного 
использования или тестирования сис\-те\-мы, и оценить качество ее работы 
невозможно.
  
  Результаты сравнения функциональности рассмотренных сервисов проверки 
на плагиат приведены в табл.~3. Несмотря на большое число существующих 
решений, ни одно из них не может служить универсальным средством проверки 
на плагиат. Основной недостаток большинства существующих сис\-тем~--- это 
направленность поиска либо на сеть Интернет, либо на собственную базу. 
Очевидно, что более точная и универсальная проверка будет обеспечена при 
использовании обоих видов источников. Кроме того, большинство сис\-тем не 
способны обрабатывать замену букв, чем часто пользуются недобросовестные 
авторы (чаще всего студенты).



  В большинстве рассмотренных сис\-тем используется метод <<шинглов>>. 
По исследованиям~[2] этот метод демонстрирует высокую точность 
обнаружения дублированных текстов. Тем не менее из-за особенностей 
реализации результаты проверки в каждой сис\-те\-ме сильно отличаются от 
других. Минусом метода является отсутствие возможности обработки 
синонимов~[7]. Это является значительным недостатком существующих 
  сис\-тем. 

\section{Практическая реализация}
  
  На базе Владимирского государственного университета авторами была 
разработана сис\-те\-ма проверки текстов на наличие заимствований из других 
источников (проверки на плагиат) <<Автор.NET>>. Сис\-те\-ма осуществляет 
проверку как по источникам, доступным в сети Интернет, так и по собственным 
источникам (базам статей, курсовых и контрольных работ, дипломных 
проектов и~т.\,д.). По результатам проверки формируется отчет с подсветкой 
найденных заимствований и воз\-мож\-ностью просмотра найденных источников.

Рассмотрим структуру сис\-те\-мы (рис.~1).

\begin{figure*}
\vspace*{1pt}
 \begin{center}
 \mbox{%
 \epsfxsize=104.248mm
 \epsfbox{sha-1.eps}
 }
 \end{center}
 \vspace*{-9pt}
\Caption{Структура сис\-те\-мы проверки текстов на заимствования}
\vspace*{6pt}
\end{figure*}

  Проверяемый исходный текст подвергается предварительной обработке, в 
которую входят:
  \begin{enumerate}[(1)]
  \item исключение из текста знаков препинания и спецсимволов;
  \item преобразование регистра;
  \item обработка замены символов (преобразование латинских букв в русских 
словах на аналогичные буквы русского алфавита для текстов на русском языке);
  \item удаление стоп-слов и знаков препинания (предлоги, наречия и~т.\,д.);
  \item фильтрация текста (удаление неинформативных слов);
  \item стемминг (обработка окончаний слов).
  \end{enumerate}
  
  Фильтрация текста заключается в удалении наиболее частотных слов, 
неинформативных слов и~т.\,д. Кроме того, фильтрации подвергаются цифры, 
спецсимволы и~т.\,д. Эта процедура позволяет существенно сократить объемы 
вычислений (длину проверяемого текста). 
  
  Стемминг заключается в обработке окончаний слов. В~описываемой 
  сис\-те\-ме они просто отбрасываются. Это позволяет исключить влияние 
таких модификаций текста, как изменение единственного и множественного 
числа, мужского и женского рода, настоящего и прошедшего времени и~т.\,д.
  
  Система проверки на плагиат <<Автор.NET>> состоит из двух модулей, 
которые функционируют независимо друг от друга.
  
  Первый модуль осуществляет проверку по внут\-рен\-ней базе источников. База 
источников включает в себя статьи, курсовые и контрольные работы, 
дипломные проекты, а также учебники и курсы лекций. Источники хранятся 
как в виде полных текстов, необходимых для оценки значимости 
заимствований (по результатам проверки), так и в виде специально 
организованного поискового индекса. Последний необходим для быстрой 
проверки на совпадения текста и базы источников. Нет необходимости при 
каждой проверке просматривать все имеющиеся тексты и производить их 
достаточно трудоемкую обработку. Вся необходимая для поиска информация 
уже включена в структурированный поисковый индекс, с которым и работает 
модуль. Поисковый индекс формируется из текстов, прошедших описанную 
выше предварительную обработку.
  
  Второй модуль осуществляет проверку по источникам сети Интернет. Для 
этих целей текст проверяемого документа разбивается на информативные куски 
(разбиение проводится по полному тексту документа без проведения 
фильтрации и стемминга). Число таких кусков зависит от размера документа. 
Далее с использованием поисковых сис\-тем проводится поиск источников, 
содержащих указанные информативные куски. Для осуществления поиска 
модуль использует Яндекс.XML, а также доступ к он\-лайн-поиску 
  сис\-тем {Google.ru}, {Rambler.ru}, {Aport.ru}, 
Поиск.Mail.ru, Nigma.ru и~т.\,д. Полученные таким образом 
источники проверяются затем на соответствие исходному документу. Для этого 
определяется формат источника (html-до\-ку\-мент, txt-файл, 
doc- или 
  rtf-до\-ку\-мент, pdf-файл). В~случае html-до\-ку\-мен\-та из 
источника удаляются теги разметки. Файлы *.doc, *.rtf и 
*.pdf преобразуются, если это возможно, в обычный текстовый формат 
без разметки. Далее источники проходят предварительную обработку, и затем 
проводится оценка их сходства с исходным документом (рис.~2). 

  
Для оценки сходства исходного документа и источ\-ни\-ков используется некая 
модификация ал\-горит\-ма <<шинглов>>. Модификация алгоритма заклю\-ча\-ет\-ся в 
том, что рассматривается не ори\-ги\-наль\-ный документ, а его обработанная и 
отфильтро\-ван\-ная копия с исключением неинформативных объектов. Основное 
требование к сис\-те\-ме~--- полнота и точность оценки совпадений. Авторы не 
ставили задачей сокращение времени проверки, проведение экс\-пресс-оцен\-ки 
на полные дубли и~т.\,д. 
{\looseness=1

}



\begin{figure*} %fig2
\vspace*{1pt}
 \begin{center}
 \mbox{%
 \epsfxsize=160mm
 \epsfbox{sha-2.eps}
 }
 \end{center}
 \vspace*{-9pt}
\Caption{Интерфейс программы <<Автор.NET>>}
\end{figure*}

  
  В настоящее время локальная база сис\-те\-мы содержит дипломные проекты, 
выполненные за последние 6~лет, и курсовые проекты, выполненные за 
последние 3~года студентами одной из специальностей. Также в базе 
содержится ряд контрольных работ, выполненных студентами заочной формы 
обучения. 


\section{Результаты исследования}

  Для проверки работоспособности сис\-те\-мы <<Автор.NET>> были 
составлены тесты трех видов:
  \begin{enumerate}[1.]
\item Заимствования с изменением в тексте времен и родов слов ($T_1$).
\item Заимствования из одного источника с измененным порядком следования 
предложений и добавлением оригинального текста между предложениями 
($T_2$).
\item Заимствования, взятые из нескольких источников, с измененным 
порядком следования предложений ($T_3$).
\end{enumerate}

  Все тесты имели приблизительно одинаковый размер в 2000~символов и 
содержали в среднем по 400~слов. В~качестве источника текстов для 
составления тестов использовалась коллекция рефератов, широко доступная в 
сети Интернет. Было составлено по 10~тестов каждого вида. 
  
  Для оценки качества обнаружения заимствований сравнивались результаты 
работы сис\-те\-мы с результатами сис\-тем <<Антиплагиат>>, {Advego 
Plagiatus} и {Miratools}. В~связи с тем, что каждая сис\-те\-ма имеет свои 
принципы подсчета оригинальности документа, в качестве метрики 
оригинальности использовалось процентное отношение оригинальных слов в 
документе к общему количеству слов.
  
  Для оценки качества обнаружения заимствований использовался показатель 
полноты (Recall), показывающий, какой процент заимствований был 
обнаружен (табл.~4). Точность обнаружения (Precision) во всех 
сис\-те\-мах была на высоком уровне и стремилась к~1~\cite{15-sha}.



  Как можно заметить, ни одна из трех рас\-смат\-ри\-ва\-емых сис\-тем не 
справилась с тестом на замену  окончаний ($T_1$). Показатель {Advego 
Plagiatus} объясняется наличием в измененном тексте цепочек из 5~слов, для 
которых окончания не менялись. Применение стемминга в сис\-те\-ме 
<<Автор.NET>> поз-\linebreak\vspace*{-12pt}

\noindent
\begin{center}  %tabl4
%\vspace*{-6pt}
{{\tablename~4}\ \ \small{Результаты тестирования}}
\vspace*{2ex}

{\small \begin{tabular}{|l|c|c|c|}
\hline
\multicolumn{1}{|c|}{Система}&$T_1$&$T_2$
&$T_3$\\
\hline
<<Антиплагиат>>&0\hphantom{,99}&1&0,97\\
{Miratools}&0\hphantom{,99}&\hphantom{,9}0.9&0,83\\
{Advego Plagiatus}&0,14&1&0,62\\
<<Автор.NET>>&0,99&1&0,98\\
\hline
\end{tabular}

}
%\vspace*{-9pt}
\end{center}


\pagebreak

%\vspace*{10pt}

\addtocounter{table}{1}


\noindent
волило ей справиться с указанной задачей и обнаружить 
заимствования.
  
  С задачей обнаружения изменения порядка следования предложений, взятых 
из одного источника ($T_2$), справились все сис\-те\-мы. Чуть худший 
результат {Miratools} (полнота~0,9) объясняется, видимо, особенностями 
реализации алгоритма сравнения в этой сис\-теме. 
  
  С задачей обнаружения предложений, взятых из разных источников с 
изменением порядка их следования ($T_3$), рассматриваемые сис\-те\-мы 
справились немного хуже. Сис\-те\-ма <<Антиплагиат>> показала хорошее 
значение полноты (0,97). Результаты сис\-те\-мы {Miratools} оказались 
более скромными (полнота 0,83). В~сис\-те\-ме {Advego Plagiatus} полнота 
иногда опускалась до 0,45 при среднем значении в~0,62. Сис\-те\-ма 
<<Автор.NET>> хорошо справилась с указанной задачей, продемонстрировав 
полноту в~0,98.
  
  Как можно заметить, сис\-те\-ма <<Автор.NET>> успешно справилась со 
всеми видами тестов и показала результаты, не уступающие, а иногда и 
превосходящие результаты работы существующих сис\-тем. 

\section{Выводы}

  Таким образом, разработанная сис\-те\-ма <<Автор.NET>> проверки текстов 
на плагиат показала достаточно хорошие результаты. Использование 
фильтрации текста, стемминга и преобразования символов позволило 
  сис\-те\-ме находить заимствованные тексты даже при их незначительной 
модификации. 
  
  Система позволяет работать не только с русскоязычными текстами, но с 
текстами на иных языках. 
  
  Особенностью системы является возможность проведения проверки как по 
внутренней базе источников, так и по источникам сети Интернет. Это делает ее 
достаточно универсальным средством проверки текстов и выгодно отличает от 
существующих сис\-тем. Выдаваемые сис\-те\-мой отчеты позволяют оценивать 
правомерность найденных заимствований текстов. 
  
  Система <<Автор.NET>> может использоваться для проверки уникальности 
студенческих работ (курсовых и дипломных проектов, рефератов и 
контрольных работ). Еще одной областью применения может служить 
использование сис\-те\-мы для проверки докладов, представляемых на 
студенческие и молодежные научные конференции.

{\small\frenchspacing
{%\baselineskip=10.8pt
\addcontentsline{toc}{section}{Литература}
\begin{thebibliography}{99}

  \bibitem{1-sha}
  Большой энциклопедический словарь.~--- М.: АСТ, Астрель, 2008. 1248~c.
  
  \bibitem{2-sha}
  \Au{Зеленков Ю.\,Г., Сегалович И.\,В.}
  Сравнительный анализ методов определения нечетких дубликатов для\linebreak 
  WEB-до\-ку\-мен\-тов~// Электронные библиотеки: перспективные методы и технологии, 
электронные коллекции (RCDL'2007): Труды IX Всеросс. научн. конф.~--- 
  Пе\-ре\-славль-За\-лес\-ский: Ун-т г.~Переславля, 2007. Т.~1. С.~166--174.
  
  \bibitem{3-sha}
  \Au{Broder A.} On the resemblance and containment of documents~// Compression and 
Complexity of Sequences (SEQUENCES'97).~--- IEEE Computer Society, 1998. P.~21--29.
  
  \bibitem{4-sha}
  \Au{Kolcz A., Chowdhury A., Alspector~J.} Improved robustness of signature-based 
  near-replica detection via lexicon randomization~//  KDD 2004 Proceedings.~--- Seattle, 2004.
  
  \bibitem{5-sha}
  \Au{Ilyinsky S., Kuzmin M., Melkov~A., Segalovich~I.}
  An efficient method to detect duplicates of Web documents with the use of inverted index~// 
WWW'2002: 11th World Wide Web Conference (International) Proceedings.~---  New York: ACM 
Press, 2002. 
  
  \bibitem{6-sha}
  \Au{Шарапов Р.\,В., Шарапова Е.\,В.}
  Пути расширения булевой модели поиска~// Информационные сис\-те\-мы и технологии. 
Известия ОрелГТУ.~--- Орел: ОрелГТУ, 2009. №\,6(56). С.~74--78.
  
  \bibitem{7-sha}
  \Au{Неелова Н.\,В., Сычугов А.\,А.}
  Сравнение результатов детектирования дублей методом шинглов и методом Джаккарда // 
Вестник РГРТУ, 2010. №\,4(34). С.~72--78.
  
  \bibitem{8-sha}
  Антиплагиат. {\sf http://www.antiplagiat.ru}.
  
  \bibitem{9-sha}
  Advego Plagiatus~--- проверка уникальности текста. {\sf http://advego.ru/plagiatus}.
  
  \bibitem{10-sha}
  Сервис проверки уникальности контента. {\sf http://\linebreak www.miratools.ru}.
  
  \bibitem{11-sha}
  Анализировать текст, поиск плагиата. {\sf 
http://\linebreak istio.com/rus/text/analyz}.
  
  \bibitem{12-sha}
  Проверка уникальности текста в Интернете~--- очень полезная программа для качественной 
раскрутки сайтов. {\sf http://www.nado.su/downloads.html}.
  
  \bibitem{13-sha}
  SearchInform Плагиат-Информ~--- сис\-те\-ма для определения плагиата в документах. {\sf 
http://www.\linebreak searchinform.ru/main/full-text-search-plagiarism-search-plagiatinform.html}.
  
  \bibitem{14-sha}
  \Au{Ширяев М.\,А., Мустакимов В.}
  Plagiatinform избавит от плагиата в научных работах~// Educational Technol. Soc., 
2009. №\,11(1). С.~367--374.

\label{end\stat}
  
  \bibitem{15-sha}
  \Au{Шарапов Р.\,В., Шарапова Е.\,В.} 
  Система проверки текстов на заимствования из других источников // Электронные 
библиотеки: перспективные методы и технологии, электронные коллекции (RCDL'2011):\linebreak 
Труды XIII Всеросс. научн. конф.~--- Воронеж: ВГУ, 2011. 
С.~233--238.

 \end{thebibliography}
}
}


\end{multicols}