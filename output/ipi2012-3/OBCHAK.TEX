\def\stat{abstr}
{%\hrule\par
%\vskip 7pt % 7pt
\raggedleft\Large \bf%\baselineskip=3.2ex
A\,B\,S\,T\,R\,A\,C\,T\,S \vskip 17pt
    \hrule
    \par
\vskip 21pt plus 6pt minus 3pt }

\label{st\stat}

%\def\rightmark{\ }

%1
\def\tit{RADIO SOURCES STUDY WITH~THE~VIRTUAL OBSERVATORY TOOLS}

\def\aut{O.\,P.~Zhelenkova}

\def\auf{Special Astrophysical Observatory, RAS, zhe@sao.ru}

\def\leftkol{\ } % ENGLISH ABSTRACTS}
\def\rightkol{\ } %ENGLISH ABSTRACTS}

\titele{\tit}{\aut}{\auf}{\leftkol}{\rightkol}

\vspace*{12pt}

\noindent
During a number of years (1980--1999), using different approaches on the basis of the Virtual 
Observatory means, the study of sources of the surveys was conducted on 
the largest Russian radio telescope RATAN-600. Their bulk identification 
is executed with the maximum use of data for different ranges of an electromagnetic spectrum 
available in open access. With the use of the Virtual Observatory program tools, the 
approach on automatic preparation and preliminary data processing is realized. The 
information system is developed for the resulting compilative catalogue 
preparation that was applied 
in the analysis of information on each source and making decision on its identification. 
Based on the experience gained while working with numerous heterogeneous resources, 
the author summarizes that the Virtual Observatory infrastructure provides convenient access to 
astronomical resources and significantly increases the efficiency of scientific research. 
However, there are still no advanced tools for further analysis, updating and publication 
of data collected by the researcher. A~number of projects is being developed to implement the 
most linked data based on already existing Web technologies and that will transfer the Virtual 
Observatory services to a new level of knowledge exchange through annotation of catalogue 
entries and the implementation of links between them.



%\vspace*{-5pt}

\KWN{virtual observatory; distributed information system; computer science technology 
in scientific research; integration of diverse information resources; databases}

%\thispagestyle{myheadings}


\vskip 12pt plus 6pt minus 3pt

%2
\def\tit{METHODOLOGICAL ASPECTS OF DEVELOPMENT OF~THE~ELECTRONIC 
RUSSIAN--ENGLISH THESAURUS ON~COMPUTATIONAL LINGUISTICS}

\def\aut{Yu.\,A.~Zagorulko$^1$, O.\,I.~Borovikova$^2$, I.\,S.~Kononenko$^3$, and~E.\,G.~Sokolova$^4$}

\def\auf{$^1$A.\,P.~Ershov Instute of Informatics Systems, SB RAS, zagor@iis.nsk.su\\[1pt]
$^2$A.\,P.~Ershov Instute of Informatics Systems, SB RAS, olesya@iis.nsk.su\\[1pt]
$^3$A.\,P.~Ershov Instute of Informatics Systems, SB RAS, irina\_k@cn.ru\\[1pt]
$^4$Russian State University for Humanities, minegot@rambler.ru}


\def\leftkol{\ } % ENGLISH ABSTRACTS}

\def\rightkol{\ } %ENGLISH ABSTRACTS}

\titele{\tit}{\aut}{\auf}{\leftkol}{\rightkol}

\vspace*{12pt}

\noindent
The paper discusses methodological aspects of development of the electronic Russian--English 
thesaurus on computational linguistics. Rationale for the creation of the thesaurus, and the 
design principles are substantiated. The general structure of the thesaurus, composition of 
the thesaurus entries, and set of relations between the thesaurus terms are described. The 
procedure of choice of terms for inclusion in the thesaurus is depicted and the problems 
with selection of the preferred term (descriptor) from a synonyms list as well as translation 
equivalents are discussed. Features of implementation of online digital version of the thesaurus 
are considered. The paper gives a particular attention to maintenance of a logical consistency 
of the thesaurus terminology system and to providing a convenient access to the thesaurus 
content.

%\vspace*{-5pt}

\KWN{multilingual thesaurus; computational linguistics; methodology for thesaurus development; 
ontology; conceptual model of the thesaurus; scientific knowledge portal technology}

%\vskip 12pt plus 6pt minus 3pt

\pagebreak


%3
\def\tit{CLASSIFICATION AND UTILIZATION OF SEMANTIC LINKAGES BETWEEN 
INFORMATION OBJECTS IN SCIENTIFIC DIGITAL LIBRARIES}

\def\aut{M.\,R.~Kogalovsky$^1$ and S.\,I.~Parinov$^2$}

\def\auf{$^1$Market Economy Institute, RAS, kogalov@gmail.com\\[1pt]
$^2$Central Economics and Mathematics Institute, RAS, sparinov@gmail.com}


\def\leftkol{\ } % ENGLISH ABSTRACTS}

\def\rightkol{\ } %ENGLISH ABSTRACTS}

\titele{\tit}{\aut}{\auf}{\leftkol}{\rightkol}

\vspace*{-2pt}

\noindent
An approach to increase the information value of scientific digital library by 
supporting the classified semantic linkages between its information objects
is discussed. An implementation of the proposed approach is based on the domestic 
Socionet system that integrates a large number of scientific digital libraries and 
which  is used \textit{de facto} as a research information space of the Social Science Division 
of the Russian Academy of Sciences.

\vspace*{-5pt}

\KWN{digital library; information object; collection of information resources; 
semantic linkage; linkage classifier; ontology; scientometrics}


%\pagebreak

 \vskip 12pt plus 6pt minus 3pt

%\pagebreak

\def\leftkol{\ } % ENGLISH ABSTRACTS}
\def\rightkol{\ } %ENGLISH ABSTRACTS}

 %4
\def\tit{INTEGRATION OF GEOGRAPHIC METADATA USING RETROSPECTIVE THESAURUS}

\def\aut{D.\,M.~Skachkov$^1$ and O.\,L.~Zhizhimov$^2$}

\def\auf{$^1$Institute of Computational Technologies, SB RAS, danil.skachkov@gmail.com\\[1pt]
$^2$Institute of Computational Technologies, SB RAS, zhizhim@sbras.ru}



\titele{\tit}{\aut}{\auf}{\leftkol}{\rightkol}

\vspace*{-2pt}
 
\noindent
The questions, related with the development of interoperable thesaurus of geographic names, 
with geometric and retrospective data included in, are discussed. The basic requirements to the 
thesaurus are defined. An overview of existing solutions is provided and the
main positions for thesaurus access profile have been formulated. The relational database schema 
for thesaurus data is described.


\vspace*{-5pt}

\KWN{geographical metadata; integration; retrospective geocoding; thesaurus}

 \vskip 12pt plus 6pt minus 3pt

%5
\def\tit{UNIVERSAL SYSTEM OF PLAGIARISM DETECTION ''AUTOR.NET''}

\def\aut{E.\,V. Sharapova$^1$ and R.\,V.~Sharapov$^2$}

\def\auf{$^1$Vladimir State University, mivlgu@mail.ru\\[1pt]
$^2$Vladimir State University, info@vanta.ru}

%\def\leftkol{ENGLISH ABSTRACTS}
%\def\rightkol{ENGLISH ABSTRACTS}

\titele{\tit}{\aut}{\auf}{\leftkol}{\rightkol}

%\vspace*{-2pt}

\def\leftkol{ENGLISH ABSTRACTS}

\def\rightkol{ENGLISH ABSTRACTS}

\vspace*{-2pt}

\noindent
The problem of duplicate texts detecting is discussed.  The basic approaches to detection 
of text duplicates are given. The existing programs of duplicate texts detecting
have been reviewed. An overview of methods to conceal the evidence of plagiarism is presented. 
A system ``Autor.NET'' which checks the text duplications founded on the internal base of 
the sources and on Internet has been created.


\vspace*{-5pt}

\KWN{plagiarism; detection of plagiarism; copying}


 \vskip 12pt plus 6pt minus 3pt

%6
\def\tit{FRACTIONAL LEVY MOTION WITH~DEPENDENT INCREMENTS
AND~ITS~APPLICATION TO~NETWORK TRAFFIC MODELING}

\def\aut{C.~De~Nikola$^1$,  Y.\,S.~Khokhlov$^2$, M.~Pagano$^3$, and~O.\,I.~Sidorova$^4$}

\def\auf{$^1$University of Salerno, denicola@diima.unisa.it\\[1pt]
$^2$People's Friendship University of Russia, yskhokhlov@yandex.ru\\[1pt]
$^3$University of Pisa, m.pagano@iet.unipi.it\\[1pt]
$^4$Tver State University, Oksana.I.Sidorova@yandex.ru}

%\def\leftkol{ENGLISH ABSTRACTS}
%\def\rightkol{ENGLISH ABSTRACTS}

\titele{\tit}{\aut}{\auf}{\leftkol}{\rightkol}

\vspace*{-2pt}

\def\leftkol{ENGLISH ABSTRACTS}

\def\rightkol{ENGLISH ABSTRACTS}

\noindent
Since the beginning of the 1990s, accurate traffic
measurements carried out in different network scenarios
highlighted that Internet traffic exhibits strong irregularities ({\it burstiness})
both in terms of extreme variability and long-term correlations.
These features, which cannot be
captured in a parsimonious way by traditional Markovian models, have a deep impact 
on the network performance and lead to the introduction  
of $\alpha$-stable distribution and self-similar processes into the network traffic modeling.
In this paper, a generalization of fractional Brownian motion, which is 
able to capture both above-mentioned features of the real traffic, is considered.

\vspace*{-5pt}

\KWN{fractional Brownian motion; $\alpha$-stable subordinator; self-similar processes; 
buffer overflow probability}

 \vskip 12pt plus 6pt minus 3pt

%7
\def\tit{ABOUT THE RATE OF~CONVERGENCE OF~ONE U-STATISTIC}


\def\aut{O.~Yanushkevichiene$^{1}$ and R.~Yanushkevichius$^{2}$}

\def\auf{$^1$Vilnius University, Institute of Mathematics and Informatics, Lithuania, olgjan@zebra.lt\\[1pt]
$^2$Lithuanian University of Educational Sciences, Vilnius, Lithuania}

%\def\leftkol{ENGLISH ABSTRACTS}
%\def\rightkol{ENGLISH ABSTRACTS}

\titele{\tit}{\aut}{\auf}{\leftkol}{\rightkol}

%\vspace*{-2pt}

\def\leftkol{ENGLISH ABSTRACTS}

\def\rightkol{ENGLISH ABSTRACTS}

\noindent
This paper pertains to the analysis of the asymptotic behavior of 
U-statistics, which are important in the construction and application of 
modern statistical methods for studying information systems.
 

%\vspace*{-2pt}

\KWN{rates of convergence; Berry--Esseen bound; quadratic form; second-degree U-statistics; 
Kolmogorov distance}

%\pagebreak

%8
\def\tit{ESTIMATES OF THE RATE OF~CONVERGENCE
OF~THE~DISTRIBUTIONS OF~RANDOM SUMS TO~VARIANCE-GAMMA DISTRIBUTIONS}

\def\aut{V.\,E.~Bening$^1$, L.\,M.~Zaks$^2$, and V.\,Yu.~Korolev$^3$} 


\def\auf{$^1$Department of Mathematical Statistics, Faculty of Computational Mathematics
and Cybernetics,\linebreak 
$\hphantom{^1}$M.\,V.~Lomonosov
Moscow State University; IPI RAN, bening@cs.msu.su\\[1pt]
$^2$Department of Modeling and Mathematical Statistics, Alpha-Bank,
lily.zaks@gmail.com\\[1pt]
$^3$Department of Mathematical Statistics, Faculty of Computational Mathematics
and Cybernetics, \linebreak
$\hphantom{^1}$M.\,V.~Lomonosov Moscow State University; IPI RAN,
vkorolev@cs.msu.su}

%\def\leftkol{ENGLISH ABSTRACTS}
%\def\rightkol{ENGLISH ABSTRACTS}

\titele{\tit}{\aut}{\auf}{\leftkol}{\rightkol}

%\vspace*{-2pt}

\def\leftkol{ENGLISH ABSTRACTS}

\def\rightkol{ENGLISH ABSTRACTS}

\noindent
The estimates are constructed for the accuracy of the approximation of the 
distributions of special random sums of independent random variables with nonzero means by 
variance-gamma distributions.


%\vspace*{-2pt}

\KWN{random sum; negative binomial distribution; gamma-distribution; 
convergence rate estimate; mixed Poisson distribution; variance-gamma distribution}


%9
\def\tit{CROSS-LAYER OPTIMIZATION PLANNING PROBLEMS IN WIRELESS NETWORKS}

\def\aut{Y.\,V.~Gaidamaka$^1$, T.\,V.~Efimushkina$^2$, A.\,K.~Samuylov$^3$, and~K.\,E.~Samouylov$^4$} 


\def\auf{$^1$Peoples' Friendship University of Russia, Telecommunication Systems Department, 
ygaidamaka@sci.pfu.edu.ru\\[1pt]
$^2$Peoples' Friendship University of Russia, Telecommunication Systems Department, tefimushkina@gmail.com\\[1pt]
$^3$Peoples' Friendship University of Russia, Telecommunication Systems Department, asam1988@gmail.com\\[1pt]
$^4$Peoples' Friendship University of Russia, Telecommunication Systems Department, ksam@sci.pfu.edu.ru}

%\def\leftkol{ENGLISH ABSTRACTS}
%\def\rightkol{ENGLISH ABSTRACTS}

\titele{\tit}{\aut}{\auf}{\leftkol}{\rightkol}

%\vspace*{-2pt}

\def\leftkol{ENGLISH ABSTRACTS}

\def\rightkol{ENGLISH ABSTRACTS}

\noindent
The review provides the state-of-the-art of the cross-layer optimization problems 
for the Orthogonal Frequency Division Multiplex  technology of 
Long-Term Evolution wireless networks. The notion of cross-layer optimization is considered, 
the detailed optimization problems and its constraints arising from the radio resources
 allocation in the downlink are investigated, and the 
 brief discussion of schedulers and its utilization functions are presented.

%\vspace*{-2pt}

\KWN{OFDM; cross-layer optimization; utility function; efficient frequency allocation}

% \vskip 12pt plus 6pt minus 3pt

\pagebreak
 


%10
\def\tit{ASYMPTOTICS OF THE MAXIMUM WORKLOAD FOR~A~CLASS OF~GAUSSIAN QUEUEING SYSTEMS}

\def\aut{О.\,V.~Lukashenko$^1$ and Е.\,V.~Morozov$^2$}

\def\auf{$^1$Institute of Applied Mathematical Research of Karelian Research Center, RAS;
Petrozavodsk State University,\linebreak
$\hphantom{^1}$lukashenko-oleg@mail.ru\\[1pt]
$^2$Institute of Applied Mathematical Research of Karelian Research Center, RAS;
Petrozavodsk State University,\linebreak
$\hphantom{^1}$emorozov@karelia.ru}


\def\leftkol{ENGLISH ABSTRACTS}

\def\rightkol{ENGLISH ABSTRACTS}

\titele{\tit}{\aut}{\auf}{\leftkol}{\rightkol}

%\vspace*{-2pt}

\noindent
The asymptotics of the maximum workload in a fluid queueing system fed by a 
process containing a random component are described by a centered Gaussian process. It is assumed 
that the variance of the process is a regularly varying at infinity function with index 
belonging to interval $(0,\,2)$. Such class of processes includes, in particular,  a sum 
of independent fractional Brownian motions.  It is shown that, under an appropriate scaling,  
the maximum workload  over interval  $[0,\,t]$  converges in probability to an explicitly given 
constant as  $t$ increases.
%\vspace*{-3pt}

\KWN{Gaussian queueing system; maximum workload; fractional Brownian motion; 
asymptotical analysis; regular variation}
%\pagebreak

%\vskip 12pt plus 6pt minus 3pt

%11
\def\tit{ON THE ESTIMATION OF THE OVERFLOW PROBABILITY IN~REGENERATIVE FINITE BUFFER 
QUEUEING SYSTEMS}

\def\aut{E.\,V.~Morozov$^1$ and R.\,S.~Nekrasova$^2$}

\def\auf{$^1$Institute of Applied Mathematical Research of Karelian Research Center, RAS;
Petrozavodsk State University,\linebreak
$\hphantom{^1}$emorozov@karelia.ru\\[1pt]
$^2$Institute of Applied Mathematical Research of Karelian Research Center,  RAS;
Petrozavodsk State University,\linebreak
$\hphantom{^1}$ruslana.nekrasova@mail.ru}


\def\leftkol{ENGLISH ABSTRACTS}

\def\rightkol{ENGLISH ABSTRACTS}

\titele{\tit}{\aut}{\auf}{\leftkol}{\rightkol}

%\vspace*{-2pt}

\noindent
A few aspects of the regenerative estimation of the stationary overflow probability in 
finite buffer system  are considered. A proof of a general relation connecting the loss 
probability and idle probability in stationary regime is given. An application of this 
result to a broad class of loss systems, in particular, to a retrial system with constant 
retrial rate, is shown.
An efficiency of this relation for  regenerative  estimation of the loss probability 
under different traffic regimes and also with use of $k$-regenerations,   
appearing in the analysis of non-Markovian systems,  is studied.   The 
numerical results of estimation are given.

 
%\vspace*{-2pt}

\KWN{finite buffer systems;  loss probability; idle probability;  regenerative estimation; 
$k$-regenerations; retrial queueing system}

%\pagebreak

\vskip 12pt plus 6pt minus 3pt

% \vskip 12pt plus 6pt minus 3pt

%12
\def\tit{STOCHASTIC MODELS OF MULTIPROCESSOR SYSTEMS: STABILITY AND~MOMENT PROPERTIES}

\def\aut{E.\,V.~Morozov and A.\,S.~Rumyantsev$^2$}

\def\auf{$^1$Institute of  Applied Mathematical Research of Karelian Research Center, RAS; 
Petrozavodsk State University,\linebreak
$\hphantom{^1}$emorozov@karelia.ru\\[1pt]
$^2$Institute of  Applied Mathematical Research of Karelian Research Center, RAS, ar0@krc.karelia.ru}


\def\leftkol{ENGLISH ABSTRACTS}

\def\rightkol{ENGLISH ABSTRACTS}

\titele{\tit}{\aut}{\auf}{\leftkol}{\rightkol}

%\vspace*{-2pt}

\noindent 
The basic models of multiprocessor  systems in which each job requires a random number 
of processors have been studied. A new model of  such system  is proposed and studied, 
where, for a given job, all processing times are identical. This assumption reflecting real 
service process  makes the analysis of the system more difficult. To investigate the model, 
some minorant and majorant (classical) systems are constructed, which allow, for a few important 
particular cases, to obtain both stability conditions and the moment properties of the original 
stationary workload process.


%\vspace*{-2pt}

\KWN{multiprocessor systems; random number of processors; identical processing times; 
stability conditions; moment properties; stationary workload;  high performance cluster}

%  \vskip 12pt plus 6pt minus 3pt

\pagebreak
  
  %13
\def\tit{A RESTRICTION ON THE TOTAL VOLUME OF DEMANDS IN~THE~DISCRETE-TIME SYSTEM Geo/$G/1/\infty$}

\def\aut{A.\,V.~Pechinkin$^1$, I.\,A.~Sokolov$^2$, and  S.\,Ya.~Shorgin$^3$}

\def\auf{$^1$IPI RAN, apechinkin@ipiran.ru\\[1pt]
$^2$IPI RAN, isokolov@ipiran.ru\\[1pt]
$^3$IPI RAN, sshorgin@ipiran.ru}

\def\leftkol{ENGLISH ABSTRACTS}

\def\rightkol{ENGLISH ABSTRACTS}

\titele{\tit}{\aut}{\auf}{\leftkol}{\rightkol}

%\vspace*{-2pt}

\noindent
Consideration is given to a discrete-time queueing system Geo$/G/1/\infty$ with inverse service 
discipline without service interruption, where each demand has random volume besides its length. 
The total volume of the demands in the queue is limited by a certain nonrandom value. 
The algorithms for the system main stationary characteristics evaluation are developed.


%\vspace*{-2pt}

\KWN{queueing system; discrete time; length and volume of a demand}

 \vskip 12pt plus 6pt minus 3pt
  
  %14
\def\tit{HEAVY-TRAFFIC ANALYSIS FOR THE QUEUEING SYSTEM WITH~HYPEREXPONENTIAL STREAM}

\def\aut{A.\,V.~Ushakov}

\def\auf{IPI RAN, ushakov@akado.ru}

\def\leftkol{ENGLISH ABSTRACTS}

\def\rightkol{ENGLISH ABSTRACTS}

\titele{\tit}{\aut}{\auf}{\leftkol}{\rightkol}

%\vspace*{-2pt}

\noindent
The heavy traffic limiting distributions of the virtual waiting time in a single server
queue with hyperexponential input stream and head-of-the-line priority discipline are
obtained.


 \label{end\stat}

%\vspace*{-2pt}

\KWN{virtual waiting time; head-of-the-line priority; hyperexponential input stream; heavy traffic}



\newpage