\def\stat{mor-luk}

\textit{\hfill Посвящается 100-летию  со дня рождения Б.\,В.~Гнеденко}

\def\tit{АСИМПТОТИКА МАКСИМУМА ПРОЦЕССА НАГРУЗКИ ДЛЯ~НЕКОТОРОГО
КЛАССА ГАУССОВСКИХ ОЧЕРЕДЕЙ$^*$}

\def\titkol{Асимптотика максимума процесса нагрузки для некоторого
класса гауссовских очередей}

\def\autkol{О.\,В.~Лукашенко, Е.\,В.~Морозов}
\def\aut{О.\,В.~Лукашенко$^1$, Е.\,В.~Морозов$^2$}

\titel{\tit}{\aut}{\autkol}{\titkol}

{\renewcommand{\thefootnote}{\fnsymbol{footnote}}\footnotetext[1]
{Работа поддержана РФФИ (проект 10-07-00017). Работа выполнена
при поддержке Программы стратегического развития на 2012--2016~гг.\
<<Университетский комплекс ПетрГУ в научно-образовательном пространстве
Европейского Севера: стратегия инновационного развития>>.}}


\renewcommand{\thefootnote}{\arabic{footnote}}
\footnotetext[1]{Институт прикладных математических исследований КарНЦ 
РАН, Петрозаводский государственный университет,\linebreak lukashenko-oleg@mail.ru}
\footnotetext[2]{Институт прикладных математических исследований КарНЦ РАН, 
Петрозаводский государственный университет,\linebreak emorozov@karelia.ru}

\vspace*{-12pt}

\Abst{Изучается  асимптотическое поведение максимума
 процесса нагрузки в жидкостной  системе обслуживания,  на вход которой  поступает
процесс, содержащий  случайную компоненту, описываемую
центрированным гауссовским процессом.  Предполагается, что дисперсия
этого процесса является регулярно меняющейся на бесконечности
функцией с показателем $V\hm\in (0,\,2)$. К такому классу процессов, в
частности, относится сумма независимых дробных броуновских движений (ДБД).
 Показано, что при соответствующей  нормировке
   максимум процесса нагрузки на интервале $[0,\,t]$
сходится по вероятности при  $t\to \infty$  к некоторой явно
выписанной константе.}

\vspace*{-2pt}

\KW{гауссовская система обслуживания; максимум
процесса нагрузки; дробное броуновское движение; асимптотический
анализ;  правильное изменение}

\vspace*{-8pt}

\vskip 14pt plus 9pt minus 6pt

      \thispagestyle{headings}

      \begin{multicols}{2}

            \label{st\stat}

\section{Введение}

В последнее время значительно возрос интерес исследователей к
анализу гауссовских жидкостных моделей  телекоммуникационных систем.
В~таких моделях входной поток, задающий величину поступившей в
сис\-те\-му работы, является гауссовским процессом (далее~--- гауссовский
входной процесс). Основная причина этого интереса состоит в том,
что, как было выяснено рядом исследователей, гауссовские процессы
позволяют учесть при моделировании современных телекоммуникационных
систем такие важные характеристики сетевого трафика, как самоподобие
(инвариантность по времени) и долговременную зависимость (долгую
память)~\cite{Leland, Willinger}. Наличие таких свойств существенно
затрудняет вероятностный анализ и, как правило, не позволяет
получить в явном виде ключевые характеристики системы, такие как
вероятность переполнения буфера, вероятность потери сообщения 
и~т.\,д. Эти характеристики   критически важны  для определения
качества обслуживания (QoS), обеспечиваемого данной системой. 
С~другой стороны, гауссовские процессы  достаточно хорошо изучены, и
это обстоятельство позволяет в ряде случаев осуществлять, по крайней
мере, асимптотический анализ систем с гауссовских входным процессом.

Наиболее важным (входным) гауссовским процессом в
телекоммуникационных системах  является ДБД, 
обла\-да\-ющее самоподобием и  долгой памятью. Важность
этого процесса обусловлена, в  частности, тем, что ДБД возникает при
суперпозиции большого числа независимых так называемых
on/off-источников с тяжелыми хвостами на больших масштабах времени.

Известно, что в  системе с  бесконечным буфером и входным процессом
ДБД  стационарный процесс загрузки~$Q^*$ (текущая
незавершенная работа в системе) распределен как  максимум
гауссовского процесса с отрицательным линейным сносом на
положительной полуоси~\cite{Reich}.

Отсутствие точных аналитических результатов для~$Q^*$ вызывает
необходимость исследования асимптотик соответствующих характеристик
системы. Такие результаты получены, например, для ве\-ро\-ят\-ности
$\mathbb{P}\left(Q^*>b \right)$ переполнения буфера размера~$b$~\cite{Narayan, Husler} 
либо для логарифма этой ве\-ро\-ят\-ности
(логарифмические асимптотики) при  $ b \hm\to \infty$~[6--8]. 
В~большинстве  упомянутых  работ в качестве входного процесса рассматривается  ДБД.

Наряду с вероятностью переполнения другой важной характеристикой
систем обслуживания является максимум процесса загрузки на конечном
интервале $[0,\,t]$. Для этой характеристики в работах~\cite{Zeevi, Husler1} 
найдены асимптотики (при   $t\hm\to \infty$)  в
случае единственного входного процесса ДБД. Эти результаты легли в основу представленного в данной статье асимптотического анализа
максимума процесса загрузки в более общей модели. Рас\-смат\-ри\-ва\-ет\-ся 
сис\-те\-ма с входным процессом, содержащим  гауссовский
процесс со стационарными приращениями, дисперсия которого
принадлежит к клас\-су {\it правильно меняющихся на бесконечности
функций}. Част\-ным случаем такого процесса является суперпозиция
независимых ДБД. Прежде всего дадим мотивировку такой постановки
задачи. Рас\-смот\-рим $N$ независимых on/off-ис\-точ\-ни\-ков, причем \mbox{$k$-й}
источник описывается процессом $\{I_k(t),\,\,t \hm \geq 0\}$,
$k=1,\ldots ,N$, где
\begin{equation*}
I_k(t)=\begin{cases}
 1\,, &\ t\in \mbox{ on-период}\,; \\[3pt]
 0\,, &\ t\in \mbox{ off-период\,.} \\
\end{cases}
%\label{Luk-l1}
\end{equation*}
Поясним, что on-период означает период непрерывной работы
источника, а следующий за ним (независимый) off-пе\-ри\-од  есть время
простоя. Таким образом, on/off-периоды каждого источника образуют
альтернирующий процесс восстановления. По условию, процессы для
разных источников независимы. Суммарная нагрузка (совокупный
агрегированный трафик), поступившая в систему от всех источников на
интервале $[0,t]$, равна
\begin{equation*}
A_N (t):=\int\limits_0^{t} {\left( {\sum\limits_{k=1}^N {I_{k}(u)} }
\right)\,du}\,,
\end{equation*}
т.\,е.\ это суммарное время работы (активности)  всех $N$ источников
на интервале $[0,t]$. Предположим, что имеется  $n$ типов
источников, среди которых  $N_i$ источников типа $i=1,\ldots ,n$.
Предположим также, что хвосты функций распределения
on/off-пе\-рио\-дов  источника $i$-го типа имеют следующую асимптотику при $x \hm\to \infty$:
\begin{equation}
\left.
\begin{array}{rl}
1-F_{\mathrm{on}}^i(x)& \sim  \ell_{\mathrm{on}}^i x^{-\alpha_{\mathrm{on}}^i}L_{\mathrm{on}}^i(x)\,,\\[9pt]
1-F^i_{\mathrm{off}}(x)& \sim  \ell_{\mathrm{off}}^i x^{-\alpha_{\mathrm{off}}^i}L_{\mathrm{off}}^i(x)\,,
\end{array}
\right\}
\label{e3-ml}
\end{equation}
где $\ell_{\mathrm{on}}^i,\ell_{\mathrm{off}}^i$~--- положительные константы,
показатели  $\alpha_{\mathrm{on}}^i,\alpha_{\mathrm{off}}^i\hm\in (1,\,2)$, а функции
$L_{\mathrm{on}}^i$, $L_{\mathrm{off}}^i$  медленно меняются  на бесконечности, т.\,е.\
$$
\lim_{x \to \infty} \fr{L^i(tx)}{L^i(x)}=1\,,\enskip i=1,\ldots,n\,,
$$
для любого фиксированного $t \hm>0$.  (Условия~(\ref{e3-ml}) означают, что
функции распределения $F_{\mathrm{on}}^i$ и $F_{\mathrm{off}}^i$ имеют {\it тяжелые
хвосты}.) Обозначим через $\mu_{\mathrm{on}}^i$ и $\mu_{\mathrm{off}}^i$
математическое ожидание on- и off-пе\-рио\-да соответственно для
источника~$i$. (Заметим, что эти величины конечны, поскольку
$\alpha_{\mathrm{on}}^i,\, \alpha_{\mathrm{off}}^i\hm>1$.) В~работе~\cite{Taqqu} доказана
функциональная предельная теорема, согласно которой  распределение
агрегированного трафика $\{A_{N}(tT)$, $t \geq 0 \}$ с  ростом
сначала величин~$N_i$, а затем па\-ра\-мет\-ра~$T$ (этот порядок важен)
сближается с распределением процесса
$ %\begin{multline}
T\left( \sum\limits_{i=1}^n N_i {\mu_{\mathrm{on}}^i}/({\mu_{\mathrm{on}}^i\hm+\mu_{\mathrm{off}}^i})
\right)t \hm+ \sum\limits_{i=1}^n T^{H_i} \sqrt{L_i(T)N_i}c_i B_{H_i}(t)$,
%\end{multline}
где $c_i$~--- положительные константы,
 $L_i$~--- медленно меняющиеся на бесконечности функции,
 выраженные через исходные параметры, а $B_{H_i}$~--- независимые
ДБД с па\-ра\-мет\-ра\-ми  Херста $H_i$, определяемыми  как
$$
H_i=\fr{3-\min(\alpha_{\mathrm{on}}^i,\alpha_{\mathrm{off}}^i)}{2}\in
\left(\fr{1}{2},\,1 \right)\,,\enskip i=1,\ldots, n\,.
$$
Таким образом,  суммарный трафик, порожденный  большим числом
(независимых) источников, у которых распределения on/off-пе\-рио\-дов
имеют тяжелые хвосты, приближенно  описывается процессом, включающим
сумму  независимых ДБД. Данный результат обосновывает
большой интерес  к  моделям телекоммуникационных систем, на вход
которых поступает один   или несколько  независимых ДБД.

\vspace*{-2pt}

\section{Гауссовские очереди}

Вначале дадим  описание гауссовской системы обслуживания в целом, а
затем описание новой модели, изучаемой  в данной статье. Пусть
$A(t)$~--- суммарная работа, поступившая в систему за время $[0,t]$.
При анализе  гауссовских очередей  входной процесс обычно
задается в следующем виде:
\begin{equation}
A(t)=mt+\sigma X(t)\,, 
\label{asymp-l1}
\end{equation}
где  $m$, $\sigma $~--- положительные константы, а  $X:=$\linebreak $:=\;\{X(t),\ t\hm\geq 0\}$~---  
центрированный гауссовский процесс со стационарными
приращениями,  $X(0)\hm=0$~\cite{Mandjes}.
 Подчеркнем, что такая форма входного процесса  является
 общепринятой при описании гауссовской системы обслуживания, однако более полное описание случайной компоненты  $X$
 зависит от специфики  модели.  Будем считать, что система имеет одно обслуживающее устройство с постоянной
скоростью обслуживания~$C$, причем $r:=C\hm-m\hm>0$. Обозначим также
$W(t)\hm=\sigma X(t)\hm-rt$ и пусть
 $Q(t)$~--- величина нагрузки (незавершенная работа в системе) в момент времени~$t$. 
 Если $Q(0)\hm=0$, то для $Q(t)$ справедливо выражение~\cite{Reich}:
 
 \noindent
\begin{multline}
Q(t)= \sup\limits_{0 \leq s \leq t}(A(t)-A(s)-C(t-s))={}\\
{}= \sup\limits_{0 \leq s \leq t}(\sigma(X(t)-X(s))-r(t-s))={}\\
{}=\sup\limits_{0 \leq s \leq t}(W(t)-W(s))\,.
\label{e6a-ml}
\end{multline}
Заметим, что $ \e A(1)\hm=m$, поэтому  условие  $r\hm>0$ обеспечивает
существование стационарного про-\linebreak\vspace*{-12pt}

\pagebreak

\noindent
цесса нагрузки,  который определяется
следующим образом~\cite{Mandjes}:
\begin{multline}
Q= \sup\limits_{t \geq 0} \left( A(t)-Ct \right)= \sup\limits_{t \geq 0} \left( \sigma X(t)-rt \right)={}\\
{}= \sup\limits_{t \geq 0} W(t)\,.
\label{e6-ml}
\end{multline}
Хорошо известно~\cite{Mandjes} (и легко установить), что
ковариационная функция $\Gamma(s,t)$ процесса~$X$ имеет вид
\begin{equation*}
\Gamma(s,t)=\e\left [X(s)\,X(t)\right]=\fr{1}{2}\left[
v(t)+v(s)-v(|t-s|) \right]\,,
\end{equation*}
где $v(t)$ есть дисперсия~$X(t)$.

Основное предположение, принятое в данной статье, состоит в том, что
функция~$v(t)$ {\it правильно меняется  на бесконечности c
индексом} $0\hm<V\hm<2$, т.\,е.\ для любого $y\hm>0$
$$
\lim\limits_{t \to \infty} \fr{v(yt) }{v(t)}=y^V\,.
$$
Известно,  что любая правильно меняющаяся на бесконечности функция
с индексом~$V$ может быть представлена в виде
\begin{equation*}
v(t)=t^V L(t)\,,
\end{equation*}
где $L(t)$~--- медленно меняющаяся на беско\-неч\-ности функция~\cite{Seneta}. 
Обозначим $\beta\hm={1}/(2-V)$, а также выберем и
зафиксируем любое $\varepsilon \hm\in (0,2-V)$. Будем далее считать,
что функция $L(t)$ является {\it дважды дифференцируемой} на~$\mathbb{R}_+$. 
(Вообще говоря, достаточно, чтобы это условие было
выполнено  на сколь угодно удаленном от начала координат луче
$[a,\infty)$, $a\hm>0$.) Кроме того, предположим, что также
выполнены следующие условия (при $t \hm\to \infty$):
\begin{align}
L(tL^\beta(t))& \sim L(t)\,,\label{e10-ml}\\
L''(t)&=o\left( \fr{1}{t^{V+\varepsilon}} \right)\,.\label{e11a-ml}
\end{align}
Из условия~(\ref{e11a-ml}) следует (с использованием правила Лопиталя),
что $L'(t)\hm=o\left(t^{-V-\varepsilon+1} \right)$. Отметим, что класс
медленно меняющихся функций, удовлетворяющих условиям~(\ref{e10-ml}) и~(\ref{e11a-ml}), 
достаточно обширен и, в частности, включает функции,
имеющие (на бесконечности) конечный ненулевой предел, а также
функции вида $(\ln t)^a$, $(\ln\ln t)^a$, $a\hm\ge 0$ и~т.\,д.

Из~(\ref{e6-ml}) следует, что  вероятность превышения процессом
стационарной нагрузки некоторого  уровня  $b\hm>0$ определяется
следующим образом:
\begin{equation*}
\mathbb{P}( Q>b) = \mathbb{P}\left(\sup\limits_{t \geq 0}W(t)>b\right)\,.
\end{equation*}
В~работе~\cite{Duffy} показано, что для центрированного
гауссовского процесса со стационарными приращениями и с дисперсией~$v(t)$, 
правильно меняющейся на бесконечности  с индексом $0\hm<V\hm<2$,
справедлива такая (логарифмическая) асимптотика:
\begin{equation}
\lim\limits_{b \to \infty} \fr{v(b)}{b^2} \ln \mathbb{P}(Q>b)=-\theta\,,
\label{asymp1-l13}
\end{equation}
где параметр $\theta>0$ имеет   вид
\begin{equation}
\theta=\fr{2}{\sigma^2(2-V)^{2-V}}\left( \fr{r}{V}
\right)^V\,.
\label{e11-ml}
\end{equation}
Кроме того, в~\cite{Konstantopoulos} показано, что на одном
веро\-ят\-ностном пространстве можно задать процесс $W(t)\hm=\sigma X(t)\hm-rt$ и
стационарный процесс $Q^*:=$\linebreak $:=\;\{Q^*(t),\, t\hm \geq 0\}$ таким образом,
что одновременно выполнены условия
\begin{align}
Q^*(t)&=_d Q \mbox{ для всех } t \geq 0\,,\label{e15-ml}\\
Q^*(t)&=W(t)+\max\{Q^*(0), L^*(t)\},\,\, t \geq 0\,,\label{e16-ml}
\end{align}
где $=_d$ означает равенство по распределению, а $L^*(t)\hm=-\min\limits_{0\le s\le t}\{W(s)\}$. 
Обозначим
\begin{equation*}
M(t)=\max\limits_{0 \leq s \leq t}Q(s),\;\; M^*(t)=\max\limits_{0 \leq s \leq
t}Q^*(s)\,. 
%\label{e13-ml}
\end{equation*}
Таким образом, $M^*(t)$  есть  максимум стационарного процесса
нагрузки $Q^*$ (удовлетворяющего условиям~(\ref{e15-ml}) и (\ref{e16-ml})), а
$M(t)$~--- максимум исходного (нестационарного) процесса нагрузки~(\ref{e6a-ml}) 
на интервале $[0,\,t]$. Далее будем изучать
асимптотическое (при $t\hm\to \infty$) поведение этих максимумов.

\section{Асимптотика  максимума процесса нагрузки}

В данном разделе сформулирован и доказан основной результат об
асимптотическом поведении  максимумов $M^*(t),\,M(t)$ при $t\hm\to \infty$. 
Этот результат   обобщает  работу~\cite{Zeevi}, где процесс
$X\hm=B_H$, т.\,е.\  является  ДБД  c параметром Херста  $H\hm\in (1/2,\,1)$. 
В~част\-ности, для модели из~\cite{Zeevi} соотношения~(\ref{asymp1-l8}) и~(\ref{asymp1-l9}) 
ниже  выполняются при $V\hm=2H$ (см.~(\ref{e11-ml})) и  $\gamma(t)\hm=\ln t.$
 Подчеркнем, что доказательство в данной статье в целом следует подходу,
использованному в~\cite{Zeevi}.

Для удобства обозначим далее
\begin{equation*}
\gamma(t)=L\left[\left(\ln t
\right)^\beta\right]  \ln t \,.
\end{equation*}

Справедлива следующая теорема.


\medskip

\noindent
\textbf{Теорема~3.1.}
\textit{Пусть дисперсия гауссовской компоненты~$X$ входного  процесса}~(\ref{asymp-l1}) 
\textit{удовлетворяет условиям}~(\ref{e10-ml}) и~(\ref{e11a-ml}), \textit{а
также $r\hm>0$. Тогда}
\begin{align}
\fr{M^*(t)}{\gamma^\beta(t)} & \Rightarrow
\left(\fr{1}{\theta}\right)^\beta\,,\enskip t \to \infty\,;
\label{asymp1-l8}\\
\fr{M(t)}{\gamma^\beta(t)} &\Rightarrow
\left(\fr{1}{\theta}\right)^\beta\,,\enskip t \to \infty\,,
\label{asymp1-l9}
\end{align}
где параметр $\theta$ удовлетворяет соотношению~(\ref{e11-ml}), а знак
$\Rightarrow$ означает сходимость по вероятности.

\medskip


\noindent
Д\,о\,к\,а\,з\,а\,т\,е\,л\,ь\,с\,т\,в\,о\,.\
Для доказательства~(\ref{asymp1-l8}) достаточно показать, что для
любого $\delta\hm>0$ выполняются следующие два соотношения:
\begin{align}
\mathbb{P} \left( \fr{M^*(t)}{\gamma^\beta(t)} \geq \left(
\fr{1-\delta}{\theta} \right)^\beta \right ) & \to 1\,,\enskip t \to
\infty \,; \label{asymp1-l1}\\
\mathbb{P} \left( \fr{M^*(t)}{\gamma^\beta(t)} \geq \left(
\fr{1+\delta}{\theta} \right)^\beta \right) &\to 0\,,\enskip t \to
\infty\,. \label{asymp1-l2}
\end{align}
Действительно,
$$
\fr{(1\pm \delta)^\beta}{{\theta}^\beta}=\fr{1}{{\theta}^\beta}\pm
\fr{\beta \delta}{{\theta}^\beta}+o(\delta)\,,\enskip  \delta \to 0\,.
$$
Поскольку $\epsilon :=\beta \delta\theta^{-\beta}\hm+o(\delta)\hm\to 0$   
при  $\delta\hm\to 0$, то~(\ref{asymp1-l1}) и~(\ref{asymp1-l2})
означают, что для любого $\epsilon\hm>0$ выполнены соответственно условия
\begin{align*}
\mathbb{P} \left(\fr{M^*(t)}{\gamma^\beta(t)}-\left(
\fr{1}{\theta}\right)^\beta > -\epsilon \right ) & \to 1\,,\enskip t \to \infty\,;\\
\mathbb{P} \left(\fr{M^*(t)}{\gamma^\beta(t)}-\left(
\fr{1}{\theta}\right)^\beta > \epsilon \right ) &\to 0\,,\enskip t \to \infty\,,
\end{align*}
которые вместе эквивалентны~(\ref{asymp1-l8}). Докажем вначале
соотношение~(\ref{asymp1-l1}). Для этого возьмем некоторое (пока
произвольное) $\Delta \hm\in (0,t)$. (Ниже величина $\Delta$ будет
выбрана специальным образом, зависящим от~$t$.) В~силу свойства~(\ref{e16-ml}) имеем
\begin{multline}
Q^*(t)=W(t)+\max\{Q^*(0), L^*(t)\}\geq{}\\
{}\geq W(t) - \inf_{0 \leq s \leq t} W(s)\geq W(t)-W(t-\Delta)\,.
\label{e18a-ml}
\end{multline}
Это дает  следующую цепочку неравенств:
\begin{multline*}
M^*(t)= \max\limits_{0 \leq s \leq t} Q^*(s)
\geq \max\limits_{k=1,\ldots,\lfloor t/\Delta \rfloor} Q^*(k \Delta)\geq{}\\
{}\geq \max\limits_{k=1,\ldots,\lfloor t/\Delta \rfloor}
[W(k\Delta)-W((k-1)\Delta)]:={}\\
{}:=\max\limits_{1\leq k \leq \lfloor t/\Delta \rfloor}Y_k^{(\Delta)}\,,
\end{multline*}
где $Y_k^{(\Delta)}:=W(k\Delta)\hm-W((k-1)\Delta)$, а $\lfloor x
\rfloor$ есть  наибольшее целое, не превосходящее~$x$. Таким
образом, справедливо неравенство
\begin{multline}
\mathbb{P} \left( \fr{M^*(t)}{\gamma^\beta(t)} \geq\left(
\fr{1-\delta}{\theta}\right)^\beta \right) \geq{}\\
{}\geq \mathbb{P} \left(
\vphantom{\left(\fr{1-\delta}{theta}\right)^\beta}
\max\limits_{i=1,\ldots ,\lfloor t/\Delta \rfloor}Y_i^{(\Delta)} \geq
\left(\fr{1-\delta}{\theta}\gamma (t)\right)^\beta\right)\,.
\label{asymp1-l4}
\end{multline}
Напомним, что  $W(t)=\sigma X(t)-rt$. Поэтому
\begin{multline*}
Y_k^{(\Delta)}= W(k\Delta)-W((k-1)\Delta)=_d{}\\
{}=_d \sigma X(\Delta)-r\Delta =_d \sigma \sqrt{v(\Delta)}\,\mathcal{N} (0,1)-r\Delta\,,
\end{multline*}
где $\mathcal{N} (0,1)$~--- нормальная случайная величина (с.\,в.). Поскольку
\begin{equation}
Z_i :=\fr{Y_i^{(\Delta)}+r\Delta }{\sigma \sqrt{v(\Delta)}}=_d
\mathcal{N}(0,1)\,,\label{e19-ml}
\end{equation} 
то $\{Z_i,\,i=1,\ldots,\lfloor t/\Delta \rfloor\}$  
является  стационарной последовательностью нормальных
с.\,в. Кроме того, для автоковариационной функции этой
последовательности получаем
\begin{multline}
\rho(k):=\mathbb{C}\mathrm{ov}(Z_1,Z_{1+k})
={}\\
{}=\fr{1}{\sigma^2 v(\Delta)}\mathbb{C}\mathrm{ov} \left( Y_1^{(\Delta)},Y_{1+k}^{(\Delta)} \right)={}\\
{}=\fr{1}{v(\Delta)}\mathbb{C}\mathrm{ov} \left(X(\Delta),\;X\left((k+1)\Delta\right)-X(k\Delta) \right)={}\\
{}=\fr{1}{v(\Delta)}\left[
\Gamma\left(\Delta,(k+1)\Delta\right)-\Gamma(\Delta,k\Delta) \right]=
\fr{1}{2v(\Delta)}\times{}\\
{}\times\left[ v\left( (k+1)\Delta
\right)-2v(k\Delta)+v\left( (k-1)\Delta\right) \right].\label{e21-ml}
\end{multline}
Из  формулы конечных приращений и~(\ref{e21-ml}) следует, что
$$
\rho(k)=\fr{\Delta}{2v(\Delta)}v''(u_3)(u_1-u_2)
$$
для некоторых $u_1\hm\in (k \Delta,\,(k+1)\Delta)$, $u_2\hm \in((k\hm-1)\Delta,\,k\Delta)$, 
$u_3\hm\in (u_2,\,u_1)$. (Вообще говоря,
величины~$u_i$  зависят от~$k$.) Легко проверить, что
\begin{multline}
v''(k)=k^VL''(k)+2Vk^{V-1}L'(k)+{}\\
{}+V(V-1)k^{V-2}L(k). 
\label{21'}
\end{multline}
Далее, с учетом  свойства~(\ref{e11a-ml})  легко показать, что при $k\hm\to \infty$
\begin{equation}
k^V \ln k \,L''(k)\to 0\,;\enskip  k^{V-1} \ln k\, L'(k)\to 0\,.
\label{asymp1-l14}
\end{equation}
Поскольку  $V<2$, то также
\begin{equation}
k^{V-2}\ln k \,L(k) \to 0\,. 
\label{asymp1-l16}
\end{equation}
Таким образом,  из (\ref{e21-ml})--(\ref{asymp1-l16})  следует, что
\begin{equation}
\rho(k)\ln k \to 0\,,\enskip k \to \infty\,. 
\label{asymp1-l3}
\end{equation}
Теперь воспользуемся леммой из~\cite{Leadbetter}, которая в
адап\-та\-ции к рассматриваемой ситуации примет следующий вид:

\smallskip

\noindent
\textbf{Лемма~1.} \textit{Пусть имеется стационарная последовательность
$\{Z_i\}_{i=1}^{m}$ стандартных нормальных с.\,в.\ с ковариационной
функцией, удовлетворяющей соотношению}~(\ref{asymp1-l3}). \textit{Тогда для
любой последовательности действительных чисел $u_m$ условие}

\vspace*{-2pt}

\noindent
\begin{eqnarray*}
\mathbb{P} \left( \max\limits_{i=1,\ldots,m} Z_i \geq u_m\right) \to 1\,,\enskip m \to \infty\,,
%\label{e24-ml}
\end{eqnarray*}
\textit{выполнено тогда и только тогда, когда}
\begin{equation}
\lim_{m\to \infty} m\,\mathbb{P}(Z>u_m)  = \infty \,. 
\label{e25-ml}
\end{equation}

\smallskip

Далее покажем, что можно выбрать  $m:=m(t)$ и $u_{m(t)}:=u(t)$
таким образом, что $m(t)\hm\to \infty$ при $t\hm\to \infty$ и условие~(\ref{e25-ml}) 
оказывается выполненным. Прежде всего положим
$\Delta:=\Delta(t)\hm=A\gamma^\beta(t)$, где  $A>0$~--- некоторая
постоянная.  (Таким образом, величина $\Delta(t)$ растет  вместе с~ $t$.)  
Далее  будет доказано, что  постоянную~$A$ можно выбрать так, чтобы обеспечить 
указанную выше сходимость.

Поскольку любая медленно меняющаяся функция растет медленнее
степенной, то существует   такое~$t_0$, что $\Delta(t)\hm<t$ при всех
$t\hm\ge t_0$. (Разумеется, $t_0$ зависит от параметров, определяющих
$\Delta(t)$.) Поэтому далее (где это требуется) предполагается, что
$t\hm\ge t_0$.  Фиксируем произвольное $\delta \hm\in (0,1)$ и введем обозначения

\vspace*{-2pt}

\noindent
\begin{align*}
\alpha(t)&=\left( \fr{1-\delta}{\theta}\,\,\gamma (t) \right)^\beta\,;\\
u(t)&= \fr{ \alpha(t)+r\Delta(t)}{\sigma\sqrt{v(\Delta(t))}}\,;\\
\tau(t)&= m(t)\mathbb{P}(Z_1>u(t))\,,
\end{align*}
где $m(t):=\lfloor t/\Delta(t)\rfloor$. Заметим, что ввиду~(\ref{asymp1-l4}) и~(\ref{e19-ml})

\vspace*{-2pt}

\noindent
\begin{multline}
\mathbb{P} \left( \max\limits_{i=1,\ldots, m(t)} Y_i^{(\Delta)} \geq \alpha(t)
\right) = {}\\
{}=\mathbb{P} \left( \max\limits_{i=1,\ldots ,m(t)} Z_i \geq u(t)
\right)\,.
\label{e27-ml}
\end{multline}
Для удобства дальнейшего анализа выпишем связь параметров
\begin{equation*}
\beta=\fr{1}{2-V}\,;\enskip \beta-\fr{\beta V}{2}=\fr{1}{2}\,;\enskip
\fr{\beta V +1}{2}=\beta\,,
\end{equation*}
а также введем обозначение  $(\ln t)^\beta\hm=\phi(t)$.
 Используя свойство~(\ref{e10-ml}), можно получить следующие соотношения:
 
 \noindent
\begin{multline*}
u(t) =  \fr{ \alpha(t)+r\Delta(t) }{\sigma\sqrt{v(\Delta(t))}}=\fr{1}{\sigma}\times{}\\
{}\times\fr{\left[((1-\delta)/\theta)^\beta+rA\right] \phi(t)\,
\left[L(  \phi(t))\right]^\beta}{\sqrt{ A^V \left[\phi(t)\right]^ V \left[L ( \phi(t)
)\right]^{\beta \,V}\,L\left( A \phi(t)\,\left[L(
\phi(t))\right]^\beta\right )}}={}\\
{}= \fr{\left[((1-\delta)/\theta)^\beta+rA\right]
\phi(t) \left[L(\phi(t))\right]^\beta} {\sigma A^{{V}/{2}}
\left[\phi(t)\right]^{{V}/{2}}\left[L(\phi(t))\right]^{(\beta V+1)/2}}\,
\left(1+o(1)\right)={}\\
{}=C(A)\,\sqrt{\ln t}\,\left(1+o(1)\right)\,,\enskip t \to \infty\,,
\end{multline*}
где использовано обозначение

\noindent
\begin{multline*}
C(A)=\fr{(1-\delta)^\beta}{\sigma\theta^\beta}\,
A^{-{V}/{2}}+\fr{r}{\sigma}\,A^{1-{V}/{2}}:={}\\
{}:=C_1\,A^{-{V}/{2}}+C_2 A^{1-{V}/{2}}\,,
%\label{asymp1-l10}
\end{multline*}
а также

\noindent
\begin{equation}
C_1=\fr{(1-\delta)^\beta}{\sigma{\theta}^\beta}\,;\quad C_2=\fr{r}{\sigma}\,.
\label{e29a-ml}
\end{equation}
Несложно проверить, что справедлива следующая оценка для вероятности
большого уклонения стандартной нормальной с.\,в.\ (см., например,~\cite{Lifshits}):

\noindent
\begin{multline*}
\mathbb{P} (Z_1 > u(t))=\fr{1}{\sqrt{2\pi}}\int\limits_{u(t)}^\infty
e^{-x^2/2}dx\geq{}\\
{}\geq \fr{1-u^{-2}(t)}{u(t)\sqrt{2\pi}}\,e^{-u^2(t)/2}\sim{}\\
{}\sim
\fr{1}{u(t)\sqrt{2\pi}}\,e^{-u^2(t)/2}\,,\enskip t \to \infty\,.
\end{multline*}
Далее с учетом того, что $u(t) \hm= C(A) \sqrt{\ln t}\,\left(1+o(1)\right)$,
при достаточно больших~$t$ получим:

\noindent
\begin{multline}
\tau(t)=m(t)\mathbb{P}(Z_1>u(t))\geq{}\\
{}\geq m(t)\fr{1-u^{-2}(t)}{u(t)\sqrt{2\pi}}\,e^{-u^2(t)/2}\sim{}\\
{}\sim t\left[\sqrt{2\pi}\,\Delta(t)\,u(t)\,e^{u^2(t)/2}\right]^{-1}\sim{}\\
{}\sim t \left[\sqrt{2\pi}\,A\,\gamma^\beta(t) \, C(A)\left(\ln t\right)^{1/2}\, t^{C^2(A)/2+o(1)}\right]^{-1}\sim{}\\
{}\sim t^{1-({C^2(A)})/{2}+o(1)}\left[L_1(t)\right]^{-1}:=g(t)\,,\label{e30-ml}
\end{multline}
где функция
$$
L_1(t):=\sqrt{2\pi}\,A\, C(A)(\ln t)^{\beta+1/2}\,[L(\phi(t))]^\beta
$$  
медленно меняется на бесконечности. Из~(\ref{e30-ml}) следует, что  если можно выбрать
$C(A)\hm<\sqrt{2}$, то будет иметь место  $g(t)\hm\to \infty$, $t\hm\to \infty$. 
Покажем, что та-\linebreak\vspace*{-12pt}

\pagebreak

\noindent
кой выбор  действительно возможен. Обозначим
$u\hm=V/2$, рассмотрим функцию
\begin{equation*}
f(x)=C_1 x^{-u}+C_2 x^{1-u} %\label{e32-ml}
\end{equation*}
и заметим, что $f(A)\hm=C(A)$. Легко убедиться, что функция $f(x)$
достигает минимума в точке
$$
x_*=\fr{C_1 u}{C_2(1-u)}\,,
$$
причем с учетом~(\ref{e29a-ml})
$$
f(x_*)=\fr{C_1^{1-u}C_2^u}{u^u\,(1-u)^{1-u}}=\sqrt{(1-\delta)2}<\sqrt{2}\,.
$$
Таким образом, выбор $A=x_*$ обеспечивает требуемую сходимость и
 ввиду~(\ref{e30-ml})
$$
\tau(t)\ge g(t) \to \infty\,,\enskip t \to \infty\,.
$$
Поэтому с учетом леммы~1 и соотношений~(\ref{asymp1-l4}), (\ref{e27-ml}) следует 
утверждение~(\ref{asymp1-l1}):
\begin{multline*}
\mathbb{P}\left(\fr{M^*(t)}{\gamma^\beta(t)} \geq
 \left(
\fr{1-\delta}{\theta} \right)^\beta \right)\geq{}\\
{}\geq \mathbb{P}\left(
\max\limits_{i=1,\ldots,m(t)}Z_i \geq u(t) \right) \to 1\,,\enskip t \to \infty\,.
\end{multline*}
Теперь обратимся к доказательству соотношения~(\ref{asymp1-l2}).
Положим
$$
Y_i=\sup\limits_{s \in [i-1,i)} Q^*(s)\,,\enskip i=1,\ldots,\lceil t \rceil\,,
$$
где $\lceil x \rceil$~--- наименьшее целое число, превы\-ша\-ющее~$x$.
Очевидно, что
$$
M^*(t) \leq \max\{Y_i:\,\,i=1,\ldots,\lceil t \rceil\}\,.
$$
Ввиду  стационарности  процесса $Q^*$ с.\,в.\ $\{Y_i\}$ одинаково
распределены (как некоторая с.\,в.~$Y$).  Далее зафиксируем
произвольное $\delta\hm>0$ и заметим, что
\begin{multline}
\mathbb{P} \left( M^*(t) \geq \left(\fr{1+\delta}{\theta}\right)^\beta
\gamma^\beta(t) \right) \leq{}\\
{}\leq \mathbb{P} \left( \max_{i=1,\ldots,\lceil t \rceil} Y_i \geq 
\left(\fr{1+\delta}{\theta}\right)^\beta
\gamma^\beta(t)  \right)\leq{}\\
{}\leq \lceil t \rceil \mathbb{P} \left(Y \geq \left(\fr{1+\delta}{\theta}\right)^\beta 
\gamma^\beta(t) \right)\,. 
\label{asymp1-l6}
\end{multline}
Для последующего анализа необходимо оценить хвост распределения
с.\,в.~$Y$. Для этого заметим, что верны следующие соотношения:
\begin{multline*}
Y =_d\max\limits_{s \in [0,1)} \left[ \vphantom{\max\limits_{s \in [0,1)}}
W(s)+{}\right.\\
\left.{}+\max\left\{Q^*(0),\,
-\min\limits_{0 \leq \tau \leq s}W(\tau)   \right\}\right]={}
\end{multline*}

\noindent
\begin{multline}
{}=\max\limits_{s \in [0,1)} \left[ \max\left\{  
\vphantom{\max\limits_{s \in [0,1)}}
W(s)+Q^*(0),\,\,W(s)-{}\right.\right.\\
\left.\left.{}-\min\limits_{0\leq \tau \leq s}W(\tau) \right\} \right]\leq{}\\
{}\leq \max\limits_{s \in [0,1]} \left[ Q^*(0)+W(s)-\min_{0 \leq \tau \leq s}W(\tau) \right]
\leq{}\\
{}\leq \max\limits_{s \in [0,1]} \left[ Q^*(0)+W(s)-\min_{0 \leq \tau
\leq 1}W(\tau) \right]={}\\
{}= Q^*(0)+\max\limits_{0 \leq s \leq 1}W(s)-\min_{0 \leq s \leq 1} W(s)\,.
\label{asymp1-l5}
\end{multline}
Для получения первого неравенства выше был использован тот факт,
что для любого~$a$ и любых~$b,\,c\hm\ge 0$ имеет место неравенство
$\max\{a+b,a+c\}\hm\leq a+b+c$.

Приведем вспомогательную лемму, которая понадобится при уточнении
оценки  хвоста распределения с.\,в.~$Y$. (Простое доказательство этой
леммы опущено.)

\medskip

\noindent
\textbf{Лемма~2.} \textit{Для двух неотрицательных с.\,в. $\xi$, $\eta$ и любых чисел $0\hm<a\hm<b$
справедливо неравенство}
$$
\mathbb{P} (\xi+\eta \geq b) \leq \mathbb{P} (\xi \geq b)+ \mathbb{P} (\eta \geq b-a)\,.
$$

\smallskip

Вернемся к доказательству теоремы. Запишем
$(1\hm+\delta)^\beta\hm=(1\hm+\delta/2)^\beta\hm+R(\delta)$ и обозначим
$R_1:=R(\delta)/(2{\theta}^\beta)\hm>0$. Введем также  обозначения
\begin{align*}
Q_1&:=\mathbb{P} \left(Q^*(0) \geq \left(\fr{1+\delta/2}{\theta}\right)^\beta \gamma^\beta(t)\right)\,;
\\
Q_2&:=\mathbb{P} \left(\max_{0 \leq s \leq 1}W(s) \geq R_1 \gamma^\beta(t) \right)\,;
\\
Q_3&:=\mathbb{P} \left(-\min_{0 \leq s \leq 1}W(s) \geq R_1 \gamma^\beta(t) \right)
\end{align*}
для хвостов функций распределения (случайных) слагаемых в формуле~(\ref{asymp1-l5}).
 Из неравенства~(\ref{asymp1-l5}) и леммы~2 следует цепочка неравенств
\begin{multline}
\mathbb{P} \left( Y \geq \left( \fr{1+\delta}{\theta}
\right)^\beta\gamma^\beta(t)\right)\leq{}\\
{}\leq \mathbb{P} \left( Q^*(0)+\max\limits_{0 \leq s \leq 1}W(s)-\min\limits_{0 \leq s
\leq 1}W(s)  \geq{}\right.\\
\left.{}\geq \left( \fr{1+\delta}{\theta}
\right)^\beta\gamma^\beta(t)\right)
\leq Q_1+Q_2+Q_3\,.\label{asymp1-l7}
\end{multline}
В свою очередь, из соотношений~(\ref{asymp1-l6}) и~(\ref{asymp1-l7})
следует неравенство
\begin{multline*}
0 \leq \mathbb{P}\left(M^*(t) \geq \left( \fr{1+\delta}{\theta}
\right)^\beta \gamma^\beta(t)  \right) \leq{}\\
{}\leq \lceil t \rceil
(Q_1+Q_2+Q_3)\,.
\end{multline*}
Чтобы получить неравенство~(\ref{asymp1-l2}), докажем, что
 $\lceil t \rceil Q_i \hm\to 0$, $t \hm\to \infty$,
$i\hm=1$, 2,~3.

\pagebreak

 Начнем с анализа $Q_2$, для чего заметим, что
\begin{equation}
\hspace*{-1mm}\max\limits_{0 \leq s \leq 1}W(s)=\max\limits_{0 \leq s \leq 1}(\sigma X(s)-rs)
\leq \max\limits_{0 \leq s \leq 1}\sigma X(s). \!\!
\label{asymp1-l11}
\end{equation}
Пусть $a=\e \max\limits_{0 \leq s \leq 1}X(s)<\infty$. Теперь
воспользуемся следствием из неравенства  Бо\-ре\-ля--Су\-да\-ко\-ва--Ци\-рель\-со\-на
для максимума на конечном интервале  центрированного гауссовского процесса со
стационарными приращениями~\cite{Adler}:
\begin{equation}
\hspace*{-1.5mm}\mathbb{P} \left( \max\limits_{0 \leq s \leq 1} X(s)> x \right) \leq
e^{-(1/(2v))(x-a)^2}, \ \ x>a,\!
\label{asymp1-l12}
\end{equation}
где дисперсия $v:=D X(1)$. Введем обозначение
$\lambda(t)\hm=R_1\,\gamma^\beta(t)\,\sigma^{-1}$. Из соотношений~(\ref{asymp1-l11})
и~(\ref{asymp1-l12}) следует, что при всех
достаточно больших~$t$
\begin{multline*}
\lceil t \rceil Q_2 \le  (t+1)\mathbb{P} \left( \max_{0 \leq s \leq 1}
X(s) \geq \lambda(t) \right)\leq{}\\
{}\leq (t+1)\,t^{-({\lambda^2(t)}/{\ln
t})\cdot (1+o(1))/(2v)} :=g_1(t)\,.
\end{multline*}
Известно~\cite{Seneta}, что   $t^\varepsilon L(t)\hm\to \infty $ при
$t\hm\to \infty$ для любого $\varepsilon\hm>0$, и поэтому
$L(t)\hm>t^{-\varepsilon}$ для всех  достаточно больших~$t$. Также
заметим, что $2\beta-1={V}/(2-V)\hm>0$ и выберем произвольное
$\varepsilon\hm\in \left(0,\,(2\beta-1)/(2\beta^2)\right)$. С~учетом этого
получаем, что при $t\hm\to \infty$
\begin{multline*}
\fr{\lambda^2(t)}{\ln t}=\left( \fr{R_1}{\sigma} \right)^2(\ln t)^{2\beta-1}
\left[L\left((\ln t)^\beta\right)\right] ^{2\beta}>{}\\
{}> \left( \fr{R_1}{\sigma} \right)^2 [\ln t]^{2\beta-1-2\beta^2
\varepsilon}\to \infty\,.
\end{multline*}
Тогда легко увидеть, что  $g_1(t) \hm\to  0$ и, значит, $ \lceil t
\rceil Q_2 \hm\to 0$, $t \hm\to \infty.$

 Теперь рассмотрим слагаемое~$Q_3$ и заметим, что
\begin{multline*}
-\min\limits_{0 \leq s \leq 1} W(s)=-\min\limits_{0 \leq s \leq 1} (\sigma
X(s)-rs)=_d{}\\
{}=_d\max\limits_{0 \leq s \leq 1} (rs+\sigma X(s))\leq
r+\sigma \max\limits_{0 \leq s \leq 1} X(s)\,,
\end{multline*}
поскольку для  центрированного гауссовского процесса $X(s)=_d -X(s)$.
Поэтому справедливо неравенство
$$
Q_3 \leq \mathbb{P} \left(\max\limits_{0 \leq s \leq 1}X(s) \geq
\fr{R_1}{\sigma}\, \gamma^\beta(t) - \fr{r}{\sigma}\right)\,.
$$
Расcуждая далее, как и при анализе $Q_2$, получим, что
$$
\lceil t \rceil Q_3 \to 0\,,\enskip t \to \infty\,.
$$

Осталось показать, что $\lceil t \rceil Q_1 \to 0$, $t \hm\to \infty$. 
Это эквивалентно тому, что
$$
a(t):=\ln \lceil t \rceil + \ln Q_1 \to -\infty\,.
$$
 Из асимптотики для вероятности переполнения~(\ref{asymp1-l13}) следует, что
\begin{equation}
b^{V-2}L(b)\ln \mathbb{P}(Q^*(0)>b)\to   -\theta\,,\enskip b\to \infty\,.
\label{asymp1-l17}
\end{equation}
В~рассматриваемом случае возьмем
$$
b=b(t)=\left( \fr{1+{\delta}/{2}}{\theta} \right)^\beta[\gamma(t)]^\beta\,.
$$
Заметим, что тогда
\begin{equation}
b^{V-2}=\fr{\theta}{1+{\delta}/{2}}\, \fr{1}{\gamma(t)}\,. 
\label{asymp1-l20}
\end{equation}
Используя условие~(\ref{e10-ml}), из~(\ref{asymp1-l17})
и~(\ref{asymp1-l20}) несложно получить, что
\begin{multline*}
\lim\limits_{t \to \infty} \fr{\ln \mathbb{P} \left( Q^*(0) \geq \left(
(1+\delta/2)/{\theta}\right)^\beta \gamma^\beta(t)
\right)}{\ln t}={}\\
{}=-\left(1+\fr{\delta}{2}\right)\,.
\end{multline*}
Следовательно, при $t \to \infty$
\begin{multline*}
a(t) \sim \ln t + \ln Q_1={}\\
{}= \ln t\! \left( 1+\fr{\ln \mathbb{P} \left( Q^*(0) \geq \left(\!\!
(1+\delta/2)/{\theta}\right)^\beta \gamma^\beta(t) \right)}{\ln t}\!\!\right)\to{}\\
{}\to -\infty \,.
\end{multline*}

Итак, доказательство соотношения~(\ref{asymp1-l8}) для $M^*(t)$ завершено.

\smallskip

Обратимся к   доказательству соотношения~(\ref{asymp1-l9}). С~учетом предыдущего анализа для этого достаточно сделать лишь
несколько пояснений. Во-пер\-вых, как и выше достаточно показать,
что
\begin{align}
\mathbb{P} \left( \fr{M(t)}{\gamma^\beta(t)} \geq \left(
\fr{1-\delta}{\theta} \right)^\beta \right ) &\to 1\,,\enskip t \to \infty \,; \label{asymp1-l18}
\\
\mathbb{P} \left( \fr{M(t)}{\gamma^\beta(t)} \geq \left(
\fr{1+\delta}{\theta} \right)^\beta \right) &\to 0\,,\enskip t \to \infty. \label{asymp1-l19}
\end{align}
 Для доказательства нижней границы~(\ref{asymp1-l18}) заметим, что
\begin{multline*}
Q(t)=\sup\limits_{0\leq s \leq t}[W(t)-W(s)]={}\\
{}=W(t)-\inf\limits_{0\leq s \leq t} W(s)\geq
 W(t)-W(t-\Delta)\,.
\end{multline*}
Далее все выкладки остаются без изменения с заменой лишь $M^*(t)$ на~$M(t)$.

Для доказательства верхней границы~(\ref{asymp1-l19}) заметим,
что в силу начального условия $Q(0)\hm=0$ имеем $Q(t)\hm \leq Q^*(t)$
и, следовательно, $M(t)\hm \leq M^*(t)$. В~этом случае справедливо неравенство
\begin{multline*}
\mathbb{P}\left( \fr{M(t)}{\gamma^\beta(t)}\geq
\left(\fr{1+\delta}{\theta}\right)^\beta\right) \leq{}\\
{}\leq \mathbb{P}\left(
\fr{M^*(t)}{\gamma^\beta(t)}\geq
\left(\fr{1+\delta}{\theta}\right)^\beta\right) \to 0\,,\enskip  t\to \infty\,.
\end{multline*}
Таким образом, соотношения~(\ref{asymp1-l18}) и~(\ref{asymp1-l19}),
а значит, и сходимость~(\ref{asymp1-l9})  доказаны.~\hfill~$\square$

\medskip


\noindent
\textbf{Замечание.}
Если $L(t) \to c\in (0,\,\infty)$, $t \hm\to \infty$,  то
$\gamma(t)\sim c \ln t$ и нормировка в утверждении теоремы~3.1
становится более простой, а~именно: формулы~(\ref{asymp1-l8})
и~(\ref{asymp1-l9}) принимают соответственно вид
$$
\fr{M^*(t)}{(c\ln t)^\beta} \Rightarrow
\left(\fr{1}{\theta}\right)^\beta\,,\enskip \fr{M(t)}{(c\ln
t)^\beta} \Rightarrow
\left(\fr{1}{\theta}\right)^\beta\,,\enskip t \to \infty\,.
$$

\smallskip

Доказанная теорема позволяет непосредственно  получить асимптотику
максимума процесса нагрузки для специального важного случая, когда
стохастическая компонента входного процесса является суммой
независимых ДБД, т.\,е.\
\begin{equation}
X(t)=\sum\limits_{i=1}^n B_{H_i}(t)\,,\enskip t\ge 0\,,
\label{e42-ml}
\end{equation}
где параметры Херста $H_i\hm\in (0,\,1)$.
 Без ограничения общности будем считать, что
$H_1\hm>\max\limits_{i>1}H_i$. Тогда  дисперсия $v(t)$  процесса $\{X(t)\}$  имеет вид
$$
v(t)=\sum\limits_{i=1}^n t^{2H_i}=t^{2H_1}L(t)\,,
$$
где медленно меняющаяся функция $ L(t)\hm=1\hm+\sum\limits_{i>1}
t^{2(H_i-H_1)}\hm\to 1$, $t \hm\to \infty. $ Таким образом,  дисперсия~$v(t)$ 
является правильно меняющейся на бесконечности функцией с
показателем $V\hm=2H_1\hm\in (0,\,2)$. Обозначим $ (2-2H_1)^{-1}\hm=\delta$.
Тогда в силу замечания выше имеет место такое

\smallskip

\noindent
\textbf{Следствие.}
\textit{Если компонента $X$  входного процесса имеет вид}~(\ref{e42-ml}), \textit{то}
$$
\fr{M^*(t)}{(\ln t)^\delta} \Rightarrow
\left(\fr{1}{\theta}\right)^{\delta}\,,\enskip \fr{M(t)}{(\ln
t)^\delta} \Rightarrow
\left(\fr{1}{\theta}\right)^\delta\,,\enskip t \to \infty\,.
$$

\smallskip

Данный результат говорит о том, что в асимптотическом анализе
максимума процесса нагрузки доминирующую роль играет ДБД с
наибольшим значением параметра Херста.



\section{Заключение}

В данной статье исследовано асимптотическое поведение максимума
 процесса нагрузки в  жидкостной  системе обслуживания с одним сервером. На вход системы  поступает
процесс, содержащий   линейную (детерминированную) компоненту и
случайную компоненту, описываемую центрированным гауссовским
процессом, у которого дисперсия является регулярно меняющейся
функцией  с показателем  $V\hm\in (0,\,2)$. К такому классу процессов,
в частности, относится сумма независимых  ДБД.
 Показано, что при соответствующей  нормировке максимум процесса нагрузки на интервале $[0,\,t]$
сходится по вероятности (при  $t\hm\to \infty$) к некоторой явно
выписанной  константе. Этот результат обобщает соответствующий
результат, полученный ранее в работе~\cite{Zeevi} для случая
единственного входного процесса ДБД.

{\small\frenchspacing
{%\baselineskip=10.8pt
\addcontentsline{toc}{section}{Литература}
\begin{thebibliography}{99}

\bibitem{Leland} %1
\Au{Leland~W.\,E., Taqqu~M.\,S., Willinger~W., Wilson~D.\,V.} On
the self-similar nature of ethernet traffic (extended version)~//
IEEE/ACM Transactions of Networking, 1994. Vol.~2. No.\,1. P.~1--15.

\bibitem{Willinger} %2
\Au{Willinger~W., Taqqu~M.\,S., Leland~W.\,E., Wilson~D.}
Self-similarity in high-speed packet traffic: Analysis and modeling
of Ethernet traffic measurements~// Statistical Sci., 1995.
Vol.~10. No.\,1. P.~67--85.

\bibitem{Reich} %3
\Au{Reich~E.} On the integrodifferential equation of Takacs~I~// Ann. Math. Stat., 1958. Vol.~29. P.~563--570.

\bibitem{Narayan} %4
\Au{Narayan~O.} Exact asymptotic queue length distribution for
fractional Brownian traffic~// Advances in Performance Analysis,
1998. Vol.~1. P.~39--63.

\bibitem{Husler} %5
\Au{H$\ddot{\mbox{u}}$sler~J., Piterbarg V.\,I.} Extremes of a certain class
of Gaussian processes~// Stochastic Processes and Their
Applications, 1999. Vol.~83. P.~257--271.


\bibitem{Duffield} %6
\Au{Duffield~N., O'Connell~N.} Large deviations and overflow
probabilities for the general single server queue, with applications~// 
Proc. Cambridge Philosophical Society, 1995.
Vol.~118. P.~363--374.

\bibitem{Debicki} %7
\Au{Debicki~K.} A~note on LDP for supremum of Gaussian processes
over infinite horizon~// Stat. Probab. Lett., 1999. Vol.~44.
P.~211--220.

\bibitem{Duffy} %8
\textit{Duffy~K., Lewis~J.\,T., Sullivan~W.\,G.} Logarithmic
asymptotics for the supremum of a stochastic process~// Ann. Appl. Probab., 2003. Vol.~13. No.\,2. P.~430--445.

\bibitem{Zeevi} %9
\Au{Zeevi~A., Glynn~P.} On the maximum workload in a queue fed
by fractional Brownian motion~// Ann. Appl. Probab., 2000. Vol.~10.
P.~1084--1099.

\bibitem{Husler1} %10
\Au{H$\ddot{\mbox{u}}$sler~J., Piterbarg V.\,I.} Limit theorem for maximum
of the storage process with fractional Brownian as input~// 
Stochastic Processes and their Applications, 2004. Vol.~114.
P.~231--250.

\bibitem{Taqqu}
\Au{Taqqu~M.\,S., Willinger~W., Sherman~R.} Proof of a
fundamental result in self-similar traffic modeling~// Computer
Communication Rev., 1997. Vol.~27. P.~5--23.

\bibitem{Mandjes}
\Au{Mandjes~M.} Large Deviations of Gaussian Queues.~---
Chichester: Wiley, 2007. 339~p.

\bibitem{Seneta}
\Au{Сенета~Е.} Правильно меняющиеся функции.~--- М.: Наука, 1985.
143~с.

\bibitem{Konstantopoulos}
\Au{Konstantopoulos~T., Zazanis~M., De~Veciana~G.}
Conservation laws and reflection mappings with application to
multiclass mean value analysis for stochastic fluid queues~// 
Stochastic Processes and their Applications, 1996. Vol.~65.
P.~139--146.

\bibitem{Leadbetter}
\Au{Лидбеттер~М., Линдгрен~Г., Ротсен~Х.} Экстремумы случайных
последовательностей и процессов.~--- М.: Мир, 1989. 392~с.

\bibitem{Lifshits}
\Au{Лифшиц~М.\,А.} Гауссовские случайные функции.~--- Киев: ТвиМС, 1995. 248~с.

\label{end\stat}

\bibitem{Adler}
\Au{Adler~R.\,J.} An introduction to continuity, extrema, and
related topics for general Gaussian processes.~--- Hayward, CA: Institute of
Mathematical Statistics, 1990. 170~p.
 \end{thebibliography}
}
}


\end{multicols}