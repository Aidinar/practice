\def\stat{mor-rum}

\def\tit{ВЕРОЯТНОСТНЫЕ МОДЕЛИ МНОГОПРОЦЕССОРНЫХ СИСТЕМ: СТАЦИОНАРНОСТЬ И~МОМЕНТНЫЕ СВОЙСТВА$^*$}

\def\titkol{Вероятностные модели многопроцессорных систем: стационарность и моментные свойства}

\def\autkol{Е.\,В.~Морозов,  А.\,С.~Румянцев}
\def\aut{Е.\,В.~Морозов$^1$,  А.\,С.~Румянцев$^2$}

\titel{\tit}{\aut}{\autkol}{\titkol}

{\renewcommand{\thefootnote}{\fnsymbol{footnote}}\footnotetext[1]
{Работа поддержана РФФИ (проект 10-07-00017). Работа выполнена
при поддержке Программы стратегического развития на 2012--2016~гг.\
<<Университетский комплекс ПетрГУ в научно-образовательном пространстве
Европейского Севера: стратегия инновационного развития>>.}}



\renewcommand{\thefootnote}{\arabic{footnote}}
\footnotetext[1]{Институт прикладных математических исследований КарНЦ 
РАН, Петрозаводский государственный университет,\linebreak emorozov@karelia.ru}
\footnotetext[2]{Институт прикладных математических исследований КарНЦ 
РАН, ar0@krc.karelia.ru}


\vspace*{-9pt}

\Abst{Дан анализ основных моделей многопроцессорных сис\-тем (МС), где
для обработки заявки требуется случайное число процессоров. Предложена
и исследована  новая модель таких  систем, в которой времена
обработки заданий данной заявки на  всех требуемых  процессорах
являются идентичными. Это предположение, отражающее реальный процесс
обработки,  существенно усложняет анализ. Для исследования данной
модели построены минорантная и мажорантная  (классические) модели, с
помощью которых для ряда важных частных случаев удалось получить как
условия стационарности исходной модели, так и моментные свойства
стационарного процесса нагрузки.}

\vspace*{-2pt}

\KW{многопроцессорные системы; групповое занятие
процессоров; идентичные времена обработки; условия стационарности;
моментные свойства; стационарный процесс нагрузки; вычислительный кластер}

\vspace*{-4pt}


\vskip 14pt plus 9pt minus 6pt

      \thispagestyle{headings}

      \begin{multicols}{2}

            \label{st\stat}
            
\section{Введение}

В настоящее время растет интерес к моделированию МС,  в первую очередь сис\-тем 
с массово-па-\linebreak раллельной
архитектурой, таких как, например,  вы\-чис\-ли\-тель\-ные клас\-те\-ры (ВК).
Для разработки эффектив\-ных алгоритмов управления очередями доступа к
МС нужна соответствующая (вероятностная) модель, которая позволяет
экспериментально определять качество обслуживания, обеспе\-чи\-ва\-емое
данной МС.  В~этой связи отметим работу~\cite{feit-coplot}, в
которой описаны особенности работы с МС и представлен хороший обзор
современных способов моделирования загрузки в таких системах.

Как известно, даже для классической системы $GI/G/m$  отсутствуют
явные формулы для   основных стационарных характеристик, а
асимптотические оценки, как правило, неточны~\cite{gupta10}. Еще
большие    трудности возникают  при моделировании современных МС.
Отметим ряд таких проблем. Достаточно часто  времена вычисления
задач в МС  адекватно описываются  распределениями с {\it тяжелыми
хвостами} (например, распределением Парето)~\cite{feit-coplot, gupta10}. 
Наличие тяжелых хвостов требует иного подхода, чем
традиционно используемый в теории очередей, и этот подход активно
развивается. Кроме того, как правило, некоторые характеристики
являются зависимыми (например, время вычисления задачи и ее
размер~\cite{feit-coplot,krampe10}). Также  во многих современных МС
условия конечности моментов незавершенной работы~\cite{scheller11},
распределение хвоста времени ожидания~\cite{foss06}, точность
асимптотических оценок~\cite{gupta10}, алгоритмы <<справедливого>>
распределения задач по процессорам~\cite{harchol01} существенно
зависят от коэффициента загрузки системы, в частности от наличия
так называемых <<резервных процессоров>>. Кроме того, важной
особенностью многих современных МС является свойство {\it долгой
памяти} входного процесса, т.\,е.\  расходимость ряда автокорреляций,
что существенно усложняет процедуру оценивания~\cite{feit-coplot, krampe10}. 
Поэтому анализ моделей, отражающих эти новые
аспекты современных  МС является весьма важным. В данной статье
проанализированы известные ранее модели МС с групповым занятием
процессоров приходящей заявкой. Кроме того, исследуется новый класс
таких моделей, где учитываются некоторые важные особенности
современных МС, а  также изучены моментные свойства процесса
загрузки в таких системах.

Статья организована следующим образом.  В~разд.~\ref{sec2} дан
 обзор  основных моделей МС, в которых каждой заявке требуется несколько
процессоров для обслуживания. Приведены    условия стационар\-ности
таких систем. Раздел~\ref{sec3} посвящен развитию модели ВК на
основе модифицированной рекурсии Ки\-фе\-ра--Воль\-фо\-ви\-ца, предложенной в
работах~[7--9]. В~част\-ности, на основе этой рекурсии строятся
классические системы, которые в определенном смысле являются
минорирующей и мажорирующей для исходной модели. Затем с помощью
этих систем получены
 условия  стационарности основной модели, а также найдены   моментные свойства
ее стационарного процесса  нагрузки. В~разд.~\ref{sec4}
рассматриваются  вопросы численного анализа данной  модели  на
основе реальных данных лог-фай\-ла кластера ЦКП КарНЦ РАН <<Центр
высокопроизводительной обработки данных>>.

\section{Многопроцессорные системы, в~которых для~обработки заявки 
требуется случайное число процессоров}\label{sec2}

Для дальнейшего анализа  важно  разделять сис\-те\-мы, в которых заявке
с номером $i$ одновременно требуется случайное число $N_i\hm\geqslant 1$
процессоров, на  системы с {\it независимым освобождением
процессоров} и системы с {\it одновременным освобождением
процессоров}. В~первом случае времена обслуживания на всех $N_i$
процессорах являются независимыми одинаково распределенными (н.\,о.\,р.),
а во втором случае  на всех $N_i$ процессорах используется одна и та
же реализация   времени обслуживания (т.\,е.\ времена обслуживания
{\it идентичны}). Системы второго  типа существенно более сложны для
анализа~\cite{green80-1}. Для таких систем   известны лишь
численные результаты~\cite{kim78},  а при отсутствии буфера~---
также некоторые аналитические результаты~[12--14].

Следуя работам~\cite{green80-1, green80, brill-green84}, рассмотрим важные
для дальнейшего анализа системы с независимым освобождением
процессоров. Рассмотрим систему
типа $M/M/m$ c интенсивностью входного (пуассоновского)  потока
$\lambda$, дисциплиной FIFO, где заявке $i$ требуется $N_i$
процессоров, на каждом из которых независимо реализуется
экспоненциальное  время обслуживания с параметром~$\mu$.
Предполагается, что  $\{N_i\}$ являются
н.\,о.\,р.\ случайными величинами
(с.\,в.)  с заданным распределением $p_k\hm=P(N=k)$, где  $N$~--- типичный
элемент последовательности. Для такой (изначально свободной) системы
в работе~\cite{green80} получено следующее {\it достаточное условие}
существования стационарного режима:
\begin{equation}
\label{green-stab3}
\lambda \sum\limits_{k=1}^m\sum_{j=0}^{k-1}
\fr{p_k}{\mu(m-j)}<1\,.
\end{equation}
В~\cite{brill-green84}  исследована  двухпроцессорная сис\-те\-ма с
двумя  независимыми пуассоновскими потоками с интенсивностями
$\lambda_i$, $i\hm=1,2$. Заявки первого класса обслуживаются на одном
процессоре, а заявки второго~--- на двух (с одинаковой
интенсивностью~$\mu$). Очевидно, эта модель является частным случаем
предыдущей. С помощью минорантной системы $M/M/2$ (c заявками только
1-го типа) и мажорантной системы типа $M/M/1$ (с заявками только
2-го типа) в работе~\cite{brill-green84} получено стационарное
распределение исходной системы, а также {\it критерий ее
стационарности}  в виде
\begin{equation}
\label{green-stab4}
 \fr{\lambda}{2\mu}(2-p_{1}^{2})<1\,.
\end{equation}
(Заметим, что~\eqref{green-stab3} влечет~\eqref{green-stab4}.) 
В~работе~\cite{green80-1} обсуждается неоптимальность  дисциплины FIFO
в рассматриваемых моделях, поскольку  некоторые процессоры могут
простаивать при наличии очереди. (Иными словами, используемая
дисциплина не является сохраняющей работу.)  В~этом состоит основная
трудность аналитического исследования таких моделей МС.

Модель с независимым освобождением процессоров может описывать
поведение узлов вычислительной грид-сис\-те\-мы (объединения
слабосвязанных вычислителей), так как в этом случае времена вычисления
заданий, запускаемые на отдельных узлах и связанные с данной
заявкой, являются  независимыми.

В работе~\cite{kaufman81} описана система хранения с
потерями, в которой на вход в блочное устройство хранения данных
поступают пуассоновские потоки заявок $k$ классов. Заявке~$i$ класса~$k$ 
требуется случайное чис\-ло $b_i$ единиц хранения на случайное время~$t_i$. 
(Эти величины имеют  заданные распределения, зависящие от~$k$.) 
Отказ в обслуживании (уход без возвращения) определяется при
помощи множества допустимых состояний $\Omega\hm=\{(n_{1},\ldots,
n_{k})\}$, описывающих распределение  числа заявок по классам. 
В~~\cite{kaufman81}  получено стационарное распределение состояний
сис\-те\-мы для произвольного множества~$\Omega$ и произвольных
распределений  времени хранения. Если~$\Omega$ допускает полное
разделение ресурсов (все ресурсы могут быть заняты любым классом
заявок), то в~\cite{kaufman81} получена эффективная с
вычислительной точки зрения рекурсия для приближенного расчета
вероятности отказа. В~работе~\cite{whitt85} рассматриваются вопросы
аппроксимации стационарного распределения для данной модели.
Описанная модель может быть полезна для моделирования ВК, в котором
пользователи  часто разделяются на группы с разными правами и
ограничениями на доступ к ресурсам. Вместо блочного устройства можно
рассматривать МС, в которой заявки занимают одновременно несколько
процессоров.

Современные ВК имеют сотни тысяч процессоров, но, как правило,
решаемые на них задачи не имеют столь же высокой степени
распараллеливания. Поэтому в качестве подходящей модели таких ВК
обосновано рассматривать системы с бесконечным числом процессоров и
групповым поступлением заявок. (Такие  модели  адекватно описывают
системы  с большим числом процессоров~\cite{Borovkov, Kovalenko}.)

В работе~\cite{liu-templeton} рассмотрена система с бесконечным
чис\-лом процессоров, в которую поступает поток групп заявок,
 представляющий собой процесс восстановления. Группа~$n$ характеризуется  {\it
 маркой}
$Y_{n}$, причем последовательность $\{Y_{n}\}$ образует цепь
Маркова. Группа~$n$ требует выполнения $N_{n}$ задач, причем
распределение~$N_{n}$ зависит  от пары $(Y_{n},\, Y_{n-1})$, а
времена вычисления  (в каждой группе) являются н.\,о.\,р. с.\,в. В~этой
модели процессоры освобождаются независимо, и поэтому она также может
быть использована  для описания вычислительной грид-сис\-те\-мы.
Важнейшей характеристикой является также
 число процессоров $\nu_{i}(n)$, занятых в момент прихода $n$-й заявки
класса $i\in[1,M]$, $n\hm\ge 1$. В~работе~\cite{liu-templeton} найдены
автокорреляционные функции числа занятых процессоров для
непрерывного времени и  в моменты прихода заявок каждого класса.

 В работе~\cite{tihonenko}  получено  преобразование Лап\-ла\-са--Стил\-ть\-еса
функции распределения стационарного числа $\nu$ занятых процессоров
в системе типа $M/G/\infty$ (c интенсивностью входного потока~$\lambda$  
и временем обслуживания~$S$), где каждой заявке
(единственного) класса требуется случайное  чис\-ло~$N$ процессоров (с
заданным распределением). Показано,  что  условие стационарности
имеет вид
\begin{equation}
\label{tihon1}
    \rho:=\lambda \e N\, \e S<\infty
\end{equation}
и что  $\e\nu =\rho$. (Последний  результат хорошо известен в случае
ординарного потока, т.\,е.\ при $\p(N=1)\hm=1$.) В~работе~\cite{eliazar}
изучаются моментные свойства,  свойство долгой
памяти  процесса величины очереди в системе $M/G/\infty$, а в
работе~\cite{daley-busy} исследуется  свойство долгой памяти
периодов занятости такой системы. В~\cite{brandt} получено
стационарное распределение  числа занятых процессоров в моменты
прихода в системе с групповым поступлением (размер группы~---
постоянная величина) и  независимым экспоненциальным временем
обслуживания каждой заявки внутри группы. Наконец, отметим
работу~\cite{krampe10}, которая содержит обзор име\-ющих\-ся в
литературе моделей мас\-со\-во-па\-рал\-лель\-ных МС, а также анализ некоторой новой
модели на основе марковских цепей.

\section{Модель вычислительного кластера}\label{sec3}

В данном разделе изучается модель МС, в которой новая заявка
занимает несколько процессоров на {\it идентичное} время, что
существенно усложняет  анализ по сравнению со случаем независимых
времен обработки. (Анализ этой модели был начат  в
работах~\cite{pavt11, rudn11}.)  В~данной модели в момент
освобождения нескольких процессоров на обслуживание может поступить
одновременно несколько заявок и это не позволяет  применить анализ
из работы~\cite{green80}.

Рассмотрим следующее обобщение классической  $m$-про\-цес\-сор\-ной
системы $GI/G/m$ с н.\,о.\,р.\ интервалами между заявками $\{T_n\}$ и
н.\,о.\,р.\ временами обслуживания $\{S_n\}$, в которой $i$-й приходящей
заявке требуется одновременно случайное число процессоров $N_{i}\hm\in[1,\, m]$.
Если число свободных процессоров меньше~$N_i$, то
она ожидает в буфере освобождения недостающего числа процессоров.
Соответствующая модификация рекурсии Ки\-фе\-ра--Воль\-фо\-ви\-ца для вектора
процесса нагрузки $W_i:=(W_{i}(1),\cdots, W_{i}(m))$~\cite{Kiefer}
принимает вид:
\begin{multline}
\label{our-rec}
 \hspace*{-6.13364pt}W_{i+1}=R\left(W_{i}(N_{i})+S_{i}-T_{i},\ldots,W_{i}(N_{i})
+S_{i}-T_{i},\right.\\ 
\left.W_{i}(N_{i}+1)-T_{i},\ldots,
W_{i}(m)-T_{i}\right)^{+},
\end{multline}
где оператор~$R$ упорядочивает компоненты в порядке возрастания,
$(\cdot)^{+}\hm=\max(0,\cdot)$ (для вектора операция применяется покомпонентно).
Заметим, что первые $N_{i}$ компонент вектора одинаковы, так как
заявка~$i$ занимает сразу  $N_{i}$ процессоров. (Если в момент
прихода $k\hm>N_{i}$ процессоров свободны, то заявка не ждет в
очереди.)  По условию, каждая заявка освобождает все занимаемые ею
процессоры одновременно. Обозначим через $D_{i}:=W_{i}(N_{i})$ время
ожидания заявки~$i$ в очереди.

Мы будем использовать так называемый кап\-линг-ме\-тод, позволяющий
сравнивать траектории случайных процессов~\cite{Shiryaev}. Нетрудно
доказать следующее  утверждение об операторе~$R$ из~\eqref{our-rec}.
(Доказательства можно также найти в~[26--28].)

\medskip

\noindent
\textbf{Лемма~3.1.}
%\begin{lem}\label{lem-sort}
\textit{Пусть векторы $X\leqslant Y$ (покомпонентно).
Тогда}
\begin{equation*}
%\label{lem-form}
R(X)\leqslant R(Y)\,.
\end{equation*}

Ниже доказано, что  для предложенной модели ВК минорантной  будет
система, в которой $i$-я заявка заменяется на группу из  $N_{i}$
заявок, причем каждая заявка  из данной группы  имеет одно и то же
время обслуживания~$S_{i}$. (Таким образом, очередная заявка из
группы немедленно занимает освободившийся процессор.) Обозначим эту
систему через $\Sigma^{(\mathrm{low})}$ и снабдим ее характеристики верхним
индексом low. Отметим, что в системе $\Sigma^{(\mathrm{low})}$   на один
процессор может распределяться  несколько заявок  из группы.

\medskip

\noindent
\textbf{Теорема 3.2}. 
\textit{Пусть $W_0^{(\mathrm{low})}\hm=W_0\hm=0$. Тогда}
\begin{equation}
W_{i+1}^{(\mathrm{low})}\leqslant W_{i+1}\,,\quad i\ge 0\,.\label{e9-mr}
\end{equation}

\medskip

\noindent
Д\,о\,к\,а\,з\,а\,т\,е\,л\,ь\,с\,т\,в\,о\,.\
Будем считать, что в момент прихода  заявки~$i$ процессоры
нумеруются в порядке возрастания нагрузки,  как и компоненты вектора~$W_i$ 
в результате применения оператора~$R$. Ввиду эквивалентности
процессоров такая процедура не меняет вероятностных свойств процесса
нагрузки. В~сис\-те\-ме  $\Sigma^{(\mathrm{low})}$ на процессор с номером $N_{i}$
из  $i$-й группы  может быть распределено не более одной заявки.
Действительно, предположим, что на процессор~$N_{i}$ распределено не
менее двух заявок (со временем обслуживания~$S_{i}$). Тогда на
некоторый процессор с номером $k\hm<N_{i}$ не поступает  заявка. Это
означает, что после  размещения заявки на процессоре~$N_{i}$ 
выполнено неравенство
$W_{i}^{(\mathrm{low})}(k)\hm>W_{i}^{(\mathrm{low})}(N_{i})\hm+S_{i}$. Но это противоречит
условию $W_{i}^{(\mathrm{low})}(k)\hm\leqslant W_{i}^{(\mathrm{low})}(N_{i})$. Далее,
если на некоторый процессор с номером $k\hm<N_{i}$ распределено $n\hm\in
(1,\, N_{i}]$ заявок, то
    \begin{equation}
    \label{minor-ntasks}
        W_{i}^{(\mathrm{low})}(k)+nS_{i}\leqslant W_{i}^{(\mathrm{low})}(N_{i})+S_{i}\,.
    \end{equation}
Действительно, в момент распределения заявки с номером $n\hm\leqslant
N_{i}$ из $i$-й группы на процессор~$k$ должно быть выполнено
неравенство $W_{i}^{(\mathrm{low})}(k)\hm+(n-1)S_{i}\hm\leqslant
W_{i}^{(\mathrm{low})}(N_{i})$.

Так как  обе системы изначально свободны, то
до применения оператора~$R$ первые $N_{i}$ компонент векторов
$W_{0}^{(\mathrm{low})}$ и~$W_{0}$ совпадают. Предположим (по индукции), что
$W_{i}^{(\mathrm{low})}\hm\leqslant W_{i}$ для некоторого~$i$.
  Поскольку процессор с номером~$N_{i}$ в
  сис\-те\-ме~$\Sigma^{(\mathrm{low})}$ получит не более одной заявки, то в силу
предположения индукции
\begin{multline*}
(W_{i}^{(\mathrm{low})}(N_{i})-T_{i})^{+}\leqslant
(W_{i}^{(\mathrm{low})}(N_{i})+S_{i}-T_{i})^{+} \leqslant{}\\
{}\leqslant
(W_{i}(N_{i})+S_{i}-T_{i})^{+}\,,
\end{multline*}
а для   компонент c номерами  $k\hm<N_i$ по
свойству~\eqref{minor-ntasks} и предположению индукции следует
\begin{multline*}
(W_{i}^{(\mathrm{low})}(k)+nS_{i}-T_{i})^{+} \leqslant
(W_{i}^{(\mathrm{low})}(N_{i})+S_{i}-T_{i})^{+}\leqslant{}\\
{}\leqslant
(W_{i}(N_{i})+S_{i}-T_{i})^{+}\,.
\end{multline*}
Теперь,  применив оператор~$R$,  в силу леммы~3.1
получаем~\eqref{e9-mr}.


\medskip

\noindent
\textbf{Замечание.}
Сис\-те\-ма~$\Sigma^{(\mathrm{low})}$ в действительности  может быть
использована как модель  вычислительной грид-сис\-те\-мы. На практике
поступающая  на грид заявка  часто является группой заданий,
требующих  перебора в пространстве параметров модели.
Задания в группе, как правило, независимы и поступают    на
свободные  вычислители, не дожидаясь одновременного освобождения
всех требуемых данной группе вычислителей. Полученный выше результат
показывает, что для подобных задач переборного типа целесообразнее
использовать архитектуру грид-сис\-те\-мы, чем ВК, так как время ожидания
заявки в такой системе в среднем оказывается меньше.

\smallskip

Поскольку в настоящее время  мощности ВК  исчисляются сотнями тысяч
процессоров, а  задач, масштабируемых на такое количество процессоров, немного, то
ограничение  $\p(N\leqslant N_{\max})\hm=1$ для некоторого $N_{\max}\hm\ll
m$ представляется вполне мотивированным. Обозначим
\begin{equation}
\left.
\begin{array}{rl}
    j&=\min\left\{ k\geqslant 1:
    \p\left(N\leqslant\left\lfloor\fr{m}{k}\right\rfloor\right)=1\right\}\,,\\[9pt]
    N_{\max}&=\left\lfloor\fr{m}{j}\right\rfloor\,.
    \end{array}
    \right\}
    \label{e8-mr}
\end{equation}
(Здесь $\lfloor x \rfloor$ означает наибольшее целое число, не превосходящее~$x$.)
Случай $j\hm=1$ (когда каждой заявке разрешено занимать все процессоры)
типичен для небольших ВК.

Как показано ниже, для исходной  системы мажорантной будет система
(в которой соответствующие величины снабжены  верхним индексом
up), где каждая заявка занимает ровно
$N_{\max}\hm=\lfloor{m}/{j}\rfloor$ процессоров. Для такой системы
рекурсия~\eqref{our-rec} примет вид
\begin{multline}
W^{(\mathrm{up})}_{i+1}= R\left(  W^{(\mathrm{up})}_{i}(N_{\max})+S_{i}-T_{i},\ldots\right.\\
\ldots ,W^{(\mathrm{up})}_{i}(N_{\max})+S_{i}-T_{i},
W^{(\mathrm{up})}_{i}(N_{\max}+1)-T_{i},\ldots\\
\left.\ldots,W^{(\mathrm{up})}_{i}(m)-T_{i}\right)^{+}\,.
\label{up-rec}
\end{multline}

\noindent

\textbf{Теорема~3.3.} 
\textit{Пусть $W_{0}=W^{(\mathrm{up})}_{0}=0$. Тогда}
\begin{equation}
W_{i+1}\leqslant W^{(\mathrm{up})}_{i+1}\,,\enskip i\geqslant 0\,.
\label{e11-mr}
\end{equation}


\noindent
Д\,о\,к\,а\,з\,а\,т\,е\,л\,ь\,с\,т\,в\,о\,.\
Очевидно, при нулевых начальных условиях неравенство $W_{1}\hm\leqslant
W^{(\mathrm{up})}_{1}$ выполнено. Предположим, что $W_{i}\hm\leqslant
W^{(\mathrm{up})}_{i}$ для некоторого $i\hm>1$.  Рассмотрим
рекурсии~\eqref{our-rec},~\eqref{up-rec} до применения оператора~$R$
и докажем индуктивный переход. Для компонент с номерами $1\hm\leqslant
k\hm\leqslant N_{i}$ имеем
\begin{equation*}
%\label{monotone-up1}
(W_{i}(N_{i})+S_{i}-T_{i})^{+}\leqslant (W^{(\mathrm{up})}_{i}(N_{\max})+S_{i}-T_{i})^{+}\,.
\end{equation*}
Для компонент с номерами $N_{i}+1\hm\leqslant k\hm\leqslant N_{\max}$
получаем неравенства
\begin{equation*}
%\label{monotone-up2}
(W_{i}(k)-T_{i})^{+}\leqslant
(W^{(\mathrm{up})}_{i}(N_{\max})+S_{i}-T_{i})^{+}\,,
\end{equation*}
а для компонент с номерами $k\hm> N_{\max}$ выполнены неравенства
\begin{equation*}
(W_{i}(k)-T_{i})^{+}\leqslant (W^{(\mathrm{up})}_{i}(k)-T_{i})^{+}\,.
\end{equation*}
В силу леммы~3.1 это влечет ~\eqref{e11-mr}.

\medskip

На самом деле каждую группу из~$N_{\max}$ процессоров можно считать
одним процессором, так как все они занимаются и освобождаются  заявкой
одновременно. Таким образом,  мажорирующая система эквивалентна
стандартной системе обслуживания $GI/G/j$ (с теми же интервалами
между приходами и  временами обслуживания). Хорошо известное
(достаточное) условие стационарности такой сис\-темы
\begin{equation}
\lambda \e S<j 
\label{e18-mr}
\end{equation}
является, таким образом, также  условием стационарности модели ВК.
(При $m\to\infty$ условие~\eqref{e18-mr} переходит в условие
стационарности $\lambda \e S\hm<\infty$ модели с бесконечным числом
процессоров для ординарного входного потока (см.\ \eqref{tihon1}).)

Получим теперь  достаточное условие неустойчивости (нестационарности)
 сис\-те\-мы $\Sigma^{(\mathrm{low})}$ (опус\-тив для простоты в обозначениях  индекс low).
 Обозначим через $A(t)$ число приходов в
систему $\Sigma^{(\mathrm{low})}$ в интервале $[0,t]$. (Это процесс
восстановления с интенсивностью $\lambda\hm=1/\e T$.) Очевидно, что
$M(t)\hm=\sum\limits_{i=1}^{A(t)}N_{i}$ есть суммарное число процессоров,
требуемых заявкам, поступившим  в интервале $[0,t]$. Пусть $D(t)$
обозначает число уходов (освобожденных процессоров) в сис\-те\-ме
$\Sigma^{(\mathrm{low})}$ в интервале $[0,t]$. Тогда $\nu(t)\hm=M(t)\hm-D(t)$  есть
чис\-ло процессоров, требуемых заявкам, находящимся в сис\-те\-ме
$\Sigma^{(\mathrm{low})}$ в момент~$t$. Обозначим $\rho\hm=\lambda \e N\e S$.

\medskip

\noindent
\textbf{Лемма 3.4.}
\textit{Если  $\rho\hm>m$, то $\nu(t)\to \infty$ c вероят\-ностью~1.}
\medskip

\noindent
Д\,о\,к\,а\,з\,а\,т\,е\,л\,ь\,с\,т\,в\,о\,.\
Обозначим через $\hat D(t)$ число уходов из системы $\Sigma^{(\mathrm{low})}$
в интервале $[0,t]$ в предположении, что {\it каждый  процессор
работает без простоев}.  Таким образом, $\{\hat D(t),\,t\hm\ge0\}$ есть
суперпозиция $m$ независимых процессов восстановления, каждый с
интенсивностью $\mu\hm=1/\e S$. Очевидно, $\hat D(t)\hm\ge D(t)$, $t\hm\ge0$,
и поэтому $\nu(t)\hm\geqslant M(t)\hm-\hat D(t)$.  Кроме того,
\begin{equation*}
\fr{M(t)}{t}=\fr{A(t)}{t}\, \fr{\sum_{i=1}^{A(t)} 
N_{i}}{A(t)} \to \lambda \e N\,,\quad t\to \infty\,.
\end{equation*}
Из теории восстановления следует, что c вероят\-ностью~1
\begin{equation*}
\fr{\hat D(t)}{t}\to \mu m\,,\quad t\to\infty\,.
\end{equation*}
Поэтому
\begin{equation*}
\liminf \fr{\nu(t)}{t}\geqslant  \mu(\rho-m)>0\,.
\end{equation*}

\medskip

Таким образом, необходимым условием устойчивости минорантной сис\-те\-мы
$\Sigma^{(\mathrm{low})}$, а следовательно, и исходной модели ВК является
условие $\rho\hm<m$.

\smallskip

\noindent
\textbf{Замечание}.  Полученный результат означает {\it сильную
неустойчивость}, в отличие от {\it слабой неустой\-чивости}, когда
неограниченный рост очереди \mbox{происходит} по вероятности. Различные
виды неустойчивости  процессов обслуживания рассматриваются,
например, в~[29--31]. В~част\-ности,
в~\cite{MorozovJMS} с использованием регенеративного
анализа показано, что минорирующая система слабо неустойчива при  $\rho\hm=m$.

По аналогии с работой~\cite{scheller-sigman97} построим рекурсию для
({\it скалярной}) компоненты $D_{i}\hm=W_{i}(N_{i})$, явля\-ющей\-ся
задержкой (временем ожидания в очереди)  заявки~$i$. Пусть для
сис\-те\-мы ВК~(\ref{our-rec})
\begin{equation}
\left.
\begin{array}{rl}
P_i&:=W_i(N_i+N_{i+1})-W_i(N_i)\,,\\[9pt]
Q_i&:=W_i(N_{i+1})-W_i(N_i)\,,\\[9pt]
(P_i&:=\infty\;\; \mbox{при}\;\;
N_i+N_{i+1}>m)\,,\\[9pt]
U_i&:=\max\left (Q_i,\, \min(P_i,\,
S_i)\right)={}\\[9pt]
&{}\hspace*{10mm}=\min\left (P_i,\, \max(Q_i,\, S_i)\right)\,.
\end{array}
\right\} 
\label{e17-mr}
\end{equation}

\medskip

\noindent
\textbf{Теорема~3.5.}  \textit{В~модели ВК величина задержки
удовлетворяет рекурсии}
\begin{equation}
\label{delay-onestep} 
D_{i+1}=(D_{i}+U_{i}-T_{i})^{+}\,,\enskip i\ge 0\,.
\end{equation}

\smallskip

\noindent
Д\,о\,к\,а\,з\,а\,т\,е\,л\,ь\,с\,т\,в\,о\,.\
В нижеследующем анализе  нумерация процессоров соответствует их
состоянию {\it перед приходом}  заявки. Нетрудно увидеть, что для
любого~$i$
\begin{equation}
\label{pq}
P_i\geqslant Q_i\,.
\end{equation}
Рассмотрим возможные случаи.
\begin{enumerate}[1.] 
\item Пусть
$N_{i}+N_{i+1}\hm\leqslant m$ и выполнено неравенство
$(W_{i}(N_{i}+N_{i+1})-T_{i})^{+}\hm\leqslant
(W_{i}(N_{i})+S_{i}-T_{i})^{+}$. Это означает, что в момент прихода
заявки $i+1$ процессоры с номерами $1,\ldots,N_{i}$ заняты заявкой~$i$ 
и для обслуживания заявки $i+1$ будут использованы процессоры с
номерами $N_{i}+1,\ldots, N_{i}+N_{i+1}$.  Поэтому
\begin{equation}
D_{i+1}=(W_{i}(N_{i}+N_{i+1})-T_{i})^{+}\,,\enskip i\ge 0\,.
\label{e19-mr}
\end{equation}
Поскольку в данном  случае $W_{i}(N_{i}+N_{i+1})\hm\leqslant
W_{i}(N_{i})+S_{i}$, то с учетом~\eqref{pq} выполнено неравенство
$Q_{i}\hm\leqslant P_{i}\hm\leqslant S_{i}$. Поэтому с учетом~(\ref{e17-mr})
$U_{i}\hm=P_{i}$
и, следовательно,~\eqref{e19-mr} влечет~\eqref{delay-onestep}.
\item
Пусть теперь  $N_{i+1}\hm>N_{i}$ и
$(W_{i}(N_{i+1})\hm-T_{i})^{+}\hm>(W_{i}(N_{i})\hm+S_{i}-T_{i})^{+}$. Это
означает, что в момент прихода заявки $i+1$ процессоры с номерами
$1,\ldots,N_{i}$ заняты заявкой~$i$. Однако   для заявки $i+1$ этих
процессоров недостаточно, и она ожидает  освобождения самого
загруженного (из требуемых ей) процессора с номером $N_{i+1}$. 
В~этом случае
\begin{equation}
D_{i+1}=(W_{i}(N_{i+1})-T_{i})^{+}\,,\enskip i\ge 0\,.
\label{e20-mr}
\end{equation}
Тогда с учетом~\eqref{pq} выполнено $P_{i}\hm\geqslant Q_{i}\hm\geqslant
S_{i}$. Следовательно, $U_{i}\hm=Q_{i}$ и~\eqref{e20-mr} снова
влечет~\eqref{delay-onestep}.

Полезно отметить, что эти два рассмотренных случая несовместны, т.\,е.\
\begin{multline*}
\left\{W_i(N_{i+1})>W_i(N_i)+S_i\right\}\cap\\
\cap
\left\{W_i(N_i+N_{i+1})<W_i(N_i)+S_i\right\}=\varnothing\,.
\end{multline*}
\item
Наконец, рассмотрим ситуацию, когда ни один из рассмотренных выше случаев
не имеет места. Тогда, как нетрудно понять, величина задержки
определяется нагрузкой на любом из процессоров с номерами
$1,\ldots,N_{i}$, т.\,е.\
\begin{multline}
D_{i+1}=(W_{i}(N_{i})+S_{i}-T_{i})^{+}={}\\
{}=(D_{i}+S_{i}-T_{i})^{+}\,,\enskip i\ge0\,.
\label{e21-mr}
\end{multline}
Легко проверить, что в данном случае  $U_{i}\hm=S_{i}$  и поэтому~\eqref{e21-mr} 
влечет~\eqref{delay-onestep}.
\end{enumerate}

\smallskip

Заметим, что в  классической системе GI/G/m $N_{i}\hm\equiv 1$ и
поэтому $Q_i\hm=0$, а условие $N_{i}+N_{i+1}\hm\leqslant m$ всегда верно
для $m\hm\geqslant 2$. Поэтому $P_{i}\hm=W_{i}(2)\hm-W_{i}(1)$,  $U_{i}\hm=\min
(P_{i},S_{i})$ и, как легко проверить,  рекурсия из~\cite{scheller-sigman97}  
оказывается частным случаем рекурсии~\eqref{delay-onestep}.  
А~поскольку в классической сис\-те\-ме $D_i\hm=W_{i}(1)$, то
\begin{multline*}
\hspace*{-9.687pt}D_{i+1}=\min\left(\left(W_{i}(1)+S_{i}-T_{i}\right)^{+},\,\left(W_{i}(2)-T_{i}\right)^{+}\right)={}\\
{}=
\left(D_{i}+U_{i}-T_{i}\right)^{+}\,.
\end{multline*}
Заметим, что  доказанная в теореме~3.3 монотонность
верна и для соответствующих моментов компонент вектора нагрузки в
модели ВК. Обозначим через $W\hm=(W_1,\ldots,W_m)$ стационарный вектор
нагрузки в модели ВК, т.\,е.\ предполагаем, что слабый предел
$W_n\hm\Rightarrow W$ существует.

 Напомним
обозначения~\eqref{e8-mr} и обозначим также $k(i)\hm=\lfloor
i/N_{\max}\rfloor$, $1\hm\leqslant i\hm\leqslant m$.
 Прямым следствием теоремы~3.3 являются следующие
моментные свойства стационарного вектора нагрузки для модели ВК,
полученные в работе~\cite{scheller11} для системы $GI/G/j$.

\medskip

\noindent
\textbf{Теорема~3.6.}
\textit{Пусть $\rho:=\e S/\e T<j$ и   $\alpha\geqslant 1$.
Тогда имеют место
    следующие импликации:}
\begin{enumerate}[1.]
\item \textit{Для компонент вектора~$W$ с индексами 
$1\hm\leqslant i\hm\leqslant \lceil\rho\rceil N_{\max}$}
        \begin{equation*}
%        \label{l5-mr}
        \e S^{1+{\alpha}/({j-\lfloor\rho\rfloor})}<\infty\  \Rightarrow
        \e W_i ^\alpha<\infty\,.
        \end{equation*}
\item \textit{Для компонент вектора~$W$ с индексами $\lceil\rho\rceil N_{\max}<i\hm\leqslant m$}
        \begin{equation*}
        \e S^{1+{\alpha}/({j-k(i)})}<\infty  \Rightarrow   \e W_i^\alpha<\infty\,.
        \end{equation*}
\end{enumerate}

\smallskip

В случае $j=1$, $N_{\max}\hm=m$ все компоненты  вектора загрузки имеют
одинаковые моментные свойства, а~именно: условие  $\e
S^{\,\alpha+1}\hm<\infty$ влечет $\e W_i^\alpha\hm<\infty$ для $1\hm\leqslant i\hm\leqslant m$. 
(Это классический результат из работы~\cite{Kiefer1}.) Следствием теоремы~3.6 являются
моментные свойства стационарной  задержки заявки в модели~ВК.


\medskip

\noindent
\textbf{Теорема~3.7.} \textit{Пусть выполнены условия теоремы~3.6.
    Тогда}
    \begin{equation*}
    \e S^{1+{\alpha}/({j-\lfloor \rho\rfloor})}<\infty\Rightarrow \e D^{\,\alpha}<\infty\,.
%    \label{e23-mr}
    \end{equation*}

\medskip


\noindent
Д\,о\,к\,а\,з\,а\,т\,е\,л\,ь\,с\,т\,в\,о\,.\
 Заметим, что  первые
$\lceil\rho\rceil N_{\max}${} компонент  вектора нагрузки имеют
одинаковые моментные свойства. Далее, поскольку $\lceil\rho\rceil
N_{\max}\hm\geqslant N_{\max}$ при любом $\rho\hm>0$, то условие
$P(N\hm\leqslant N_{\max})\hm=1$, следующее из~\eqref{e8-mr}, гарантирует, что
задержка $D_{i}:=W_{i}(N_{i})$ окажется среди первых
$\lceil\rho\rceil N_{\max}$ координат вектора нагрузки. В~то же
время условие $\e D^{\,\alpha}\hm<\infty$  для этих компонент  следует
из теоремы~3.6.


\section{Заключение}\label{sec4}

В  статье  дан краткий обзор основных  моделей МС, в которых  заявке
требуется  для обслуживания случайное число процессоров.   Кроме того,
 приведена новая модель ВК на основе модифицированной
рекурсии Ки\-фе\-ра--Воль\-фо\-ви\-ца. Эта модель исследована  с по\-мощью
классических многоканальных систем, одна из которых является
минорантной, а другая   мажорантной для процесса нагрузки в  модели
ВК. На основе этого подхода, в частности, получены   достаточные
условия стационарности, условие (сильной) неустойчивости, а также
моментные свойства компонент стационарного вектора нагрузки в модели~ВК.

На основе данных лог-фай\-ла запусков задач кластера ЦКП КарНЦ
РАН~\cite{cluster} была проведена апробация модели ВК. (Анализ
некоторых численных результатов, относящихся к данной модели,
представлен также в работах~\cite{krc11, aptpms11, hpc11}.)
Эксперименты проводились с использованием разработанного авторами
пакета расширения для вычислительной среды~R. Сформулируем  краткие
выводы из проведенных    исследований, которые   согласуются с
результатами авторитетных исследований в данной
области~\cite{feit-coplot}. В~част\-ности, обнаружено, что интервалы
между приходами заявок хорошо описываются с помощью лог-нормального
распределения ({\it с тяжелым хвостом}). Для моделирования времени
обслуживания заявок использовалось усеченное распределение Парето,
что соответствует принятой практике~\cite{gupta10}.  Обнаружена
зависимость времени обработки~$S_i$  от числа требуемых процессоров
$N_{i}$ (что, однако, не отражено в рассматриваемой модели ВК). Кроме
того, (визуально) обнаружено медленное убывание автокорреляционной
функции последовательности~$N_{i}$, что  может отражать наличие
тяжелого хвоста у  распределения с.\,в.~$N$.
 В~ходе исследования также обнаружено  доминирование значений
$N_i\hm=2^k$ для   $k\hm=0,\ldots,8$
 (около 87\%). (В~этой связи   укажем  работы~\cite{krampe10, downey99}.)
Наконец отметим, что почти 50\% всех задач были
однопроцессорными. В~целом модель показала хорошее согласие с
экспериментальными данными лог-фай\-ла, что говорит об определенном
потенциале ее практического применения для анализа существующих и
проектирования новых~МС.

{\small\frenchspacing
{%\baselineskip=10.8pt
\addcontentsline{toc}{section}{Литература}
\begin{thebibliography}{99}

\bibitem{feit-coplot}  
\Au{Talby D., Feitelson D., Raveh A.} 
A co-plot analysis of logs and models of parallel workloads~// 
ACM Transactions on Modeling and Computer Simulation (TOMACS), 2007. Vol.~17. No.\,3. Article~12.

\bibitem{gupta10} 
\Au{Gupta V., Harchol-Balter M., Dai J.\,G., Zwart~B.} 
On the inapproximability of M/G/K: Why two moments of job size distribution are not enough~// 
Queueing Syst., 2010. Vol.~64. P.~5--48.

\bibitem{krampe10} 
\Au{Krampe A., Lepping J., Sieben~W.} 
A~hybrid Markov chain modeling architecture for workload on parallel computers~// 
HPDC'10: 19th ACM  Symposium (International) on High Performance Distributed Computing
Proceedings.~--- New York: ACM, 2010. P.~589--596.

\bibitem{scheller11} 
\Au{Scheller-Wolf A., Vesilo~R.} 
Sink or swim together: Necessary and sufficient conditions for finite moments of workload components in FIFO 
multiserver queues~// Queueing Syst., 2011. Vol.~67. No.\,1. P.~47--61.

\bibitem{foss06} 
\Au{Foss S., Korshunov D.} 
Heavy tails in multi server queue~// Queueing Syst., 2006. Vol.~52. No.\,1. P.~31--48.

\bibitem{harchol01} 
\Au{Harchol-Balter M., Schroeder~B.} 
Evaluation of task assignment policies for supercomputing servers: The 
case for load unbalancing and fairness~// HPDC'00: 9th IEEE Symposium on High 
Performane Distributed Computing Proceedings.~--- New York: ACM, 2001.
P.~211--219.

\bibitem{pavt11} 
\Au{Морозов Е.\,В., Румянцев А.\,С.} 
Некоторые модели многопроцессорных систем обслуживания с тяжелыми хвостами~// 
Параллельные вычислительные технологии 2011: Сборник трудов междунар. научн. конф.~--- 
Челябинск: ЮУрГУ, 2011. С.~555--566.

\bibitem{rudn11} 
\Au{Румянцев А.\,С.} 
О~стохастическом моделировании вычислительного кластера~// 
Ин\-фор\-ма\-ци\-он\-но-те\-ле\-ком\-му\-ни\-ка\-ци\-он\-ные 
технологии и математическое моделирование высокотехнологичных систем: 
Тезисы докладов Всеросс. конф. с международным учас\-ти\-ем (18--22~апреля 2011).~--- 
М.: РУДН, 2011. С.~46--47.

\bibitem{krc11} 
\Au{Морозов Е.\,В., Румянцев А.\,С.} 
Модели многосерверных систем для анализа вычислительного кластера~// 
Труды Карельского научного центра Российской академии наук, 2011. №\,5. С.~75--86.

\bibitem{green80-1} 
\Au{Green L.} 
Comparing operating characteristics of queues in which customers require a random number of servers~// 
Management Sci., 1980. Vol.~27. No.\,1. P.~65--74.

\bibitem{kim78} %11
\Au{Kim S.} $M/M/s$ queueing system where customers demand multiple server use. 
Ph.D.\ Dissertation.~--- Southern Methodist University, 1979.

\bibitem{kaufman81}  %12
\Au{Kaufman J.} Blocking in a shared resource environment~// 
IEEE Transactions on Communications, 1981. Vol.~29. No.\,10. P.~1474--1481.

\bibitem{whitt85} %13
\Au{Whitt W.} Blocking when service is required from several facilities simultaneously~// 
AT\&T Techn.~J., 1985. Vol.~64. No.\,8. P.~1807--1856.


\bibitem{dijk88} %14
\Au{Van Dijk N., Smeitink E.} A non-exponential queueing system with batch servicing~// 
Researchmemorandum.~--- Amsterdam: Free University, 1988. No.\,13.

\bibitem{green80} %15
\Au{Green L.} A queueing system in which customers require a random number of servers~// 
Operations Res., 1980. Vol.~28. No.\,6. P.~1335--1346.

\bibitem{brill-green84}  %16
\Au{Brill P., Green L.} Queues in which customers receive simultaneous service from 
a random number of servers: A~system point approach~// Management Sci., 1984. Vol.~30. No.\,1. P.~51--68.

\bibitem{Borovkov} 
\Au{Боровков А.\,А.} Вероятностные процессы в теории массового обслуживания.~---  М.: Наука, 1972.

\bibitem{Kovalenko}  
\Au{Гнеденко Б.\,В., Коваленко И.\,Н.} Ведение в теорию массового осблуживания.~--- М.: Наука, 1987.

\bibitem{liu-templeton} 
\Au{Liu L., Templeton J.} Autocorrelations in infinite server batch arrival queues~// 
Queueing Syst., 1993. Vol.~14. P.~313--337.

\bibitem{tihonenko}  
\Au{Тихоненко О.\,М.} Модели массового обслуживания в системах обработки информации.~--- Минск: Университетское,
1990.

\bibitem{eliazar} 
\Au{Eliazar I.} The M/G/$\infty$ system revisited: Finiteness, summability, long range 
dependence, and reverse engineering~// Queueing Syst., 2007. Vol.~55. P.~71--82.

\bibitem{daley-busy} 
\Au{Daley D.} The busy period of the $M/GI/\infty$ queue~// Queueing Syst., 2001. Vol.~38. P.~195--204.

\bibitem{brandt} 
\Au{Brandt A., Sulanke H.} On the $GI/M/\infty$ queue with batch arrivals of constant size~// 
Queueing Syst., 1987. Vol.~2. P.~187--200.

\bibitem{Kiefer} 
\Au{Kiefer J., Wolfowitz J.} On the theory of queues with many servers~// 
Trans. Amer. Math. Soc., 1955. Vol.~78. P.~1--18.

\bibitem{Shiryaev} 
\Au{Ширяев А.\,Н.} Вероятность.~--- М.: Наука, 1989. 640~с.

\bibitem{Jacobs} %26
\Au{Jacobs D.\,R., Schach S.} Stochastic order relationships between $GI/G/k$ queues~// 
Ann. Math. Stat., 1972. Vol.~43.  P.~1623--1633.

\bibitem{scheller-further}  %27
\Au{Scheller-Wolf A.} Further delay moment results in FIFO multiserver queues~// 
Queueing Syst., 2000. Vol.~34. P.~387--400.

\bibitem{CRM} %28
\Au{Morozov E.\,V.}  Coupling and monotonicity of queues. Sci. Report. No.\,779.~--- 
Barcelona: CRM, 2008. P.~1--29.

\bibitem{Taha} \Au{El-Taha M.} Pathwise rate-stability for input-output processes~// 
Queueing Syst., 1996. Vol.~22. P.~47--63.

\bibitem{MorozovJMS} 
\Au{Morozov E.} Instability of nonhomogeneous queueing networks~// J.~Math. Sci., 2002. Vol.~112. No.\,2. P.~4155--4167.

\bibitem{Morozovoutput} 
\Au{Morozov E.} Stability of Jackson-type network output~// 
Queueing Syst., 2002. Vol.~40. P.~383--406.

\bibitem{scheller-sigman97} 
\Au{Scheller-Wolf A., Sigman K.} Delay moments for FIFO $GI/GI/s$ queues~// 
Queueing Syst., 1997. Vol.~25. P.~77--95.

\bibitem{Kiefer1} 
\Au{Kiefer J., Wolfowitz J.} On the characteristics of the general queueing process 
applications to random walks~// Ann.~Math.~Statist., 1956. Vol.~27. P.~147--161.

\bibitem{cluster} 
Центр высокопроизводительной обработки данных.~--- ЦКП КарНЦ РАН.
{\sf http://cluster.krc.karelia.ru}.

\bibitem{aptpms11} 
\Au{Morozov E.\,V., Rumyantsev A.\,S.} Stability analysis of a multiprocessor model 
describing a high performance cluster~// 
Applied problems in theory of probabilities and mathematical statistics 
related to modeling of information systems: Book of Abstracts of the 29th 
 Seminar (International) on Stability Problems for Stochastic Models and 
 5th Workshop (International).~--- Moscow: Institute of Informatics Problems, RAS, 2011. P.~82--83.

\bibitem{hpc11} 
\Au{Румянцев А.\,С.} Моделирование процесса нагрузки вычислительного кластера 
на примере кластера ЦКП\linebreak КарНЦ РАН <<Центр высокопроизводительной обработки данных>>~// 
Высокопроизводительные параллельные вычисления на кластерных системах:\linebreak Мат-лы 
XI~Всеросс. конф.~/ Под ред.~В.\,П.~Гергеля.~--- Нижний Новгород: Изд-во 
Нижегородского госуниверситета, 2011. С.~272--275.

\label{end\stat}

\bibitem{downey99} 
\Au{Downey A., Feitelson D.} The elusive goal of workload characterization~// 
Performance Evaluation Rev., 1999. Vol.~26. No.\,4. P.~14--29.
 \end{thebibliography}
}
}


\end{multicols}