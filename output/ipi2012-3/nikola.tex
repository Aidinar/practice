\renewcommand{\figurename}{\protect\bf Figure}
\renewcommand{\tablename}{\protect\bf Table}
\renewcommand{\bibname}{\protect\rmfamily References}

\def\stat{nikola}

{\begin{center}
{\Large
Статьи, являющиеся развитием докладов, %}\\[6pt]
%{\Large 
представленных }\\[6pt]
{\Large  на XXIX Международном семинаре}\\[6pt]
{\Large по проблемам устойчивости стохастических моделей}\\[9pt]
{\large (г.~Светлогорск Калининградской области России, 10--16~октября 2011~г.)}
\end{center}
}


\def\tit{FRACTIONAL LEVY MOTION WITH DEPENDENT INCREMENTS
AND~ITS~APPLICATION TO~NETWORK TRAFFIC MODELING}

\def\titkol{Fractional Levy motion with dependent increments
and~its~application to~network traffic modeling}

\def\autkol{C.~De~Nikola,  Y.\,S.~Khokhlov, M.~Pagano, and~O.\,I.~Sidorova}
\def\aut{C.~De~Nikola$^1$,  Y.\,S.~Khokhlov$^2$, M.~Pagano$^3$, and~O.\,I.~Sidorova$^4$}

\titel{\tit}{\aut}{\autkol}{\titkol}

%{\renewcommand{\thefootnote}{\fnsymbol{footnote}}\footnotetext[1]
%{Работа поддержана РФФИ (проект 10-07-00017). Работа выполнена
%при поддержке Программы стратегического развития на 2012--2016~гг.\
%<<Университетский комплекс ПетрГУ в научно-образовательном пространстве
%Европейского Севера: стратегия инновационного развития>>.}}


\renewcommand{\thefootnote}{\arabic{footnote}}
\footnotetext[1]{University of Salerno, denicola@diima.unisa.it}
\footnotetext[2]{People's Friendship University of Russia, yskhokhlov@yandex.ru}
\footnotetext[3]{University of Pisa, m.pagano@iet.unipi.it}
\footnotetext[4]{Tver State University, Oksana.I.Sidorova@yandex.ru}


\Abste{Since the beginning of the 1990s, accurate traffic
measurements carried out in different network scenarios
highlighted that Internet traffic exhibits strong irregularities ({\it burstiness})
both in terms of extreme variability and long-term correlations.
These features, which cannot be
captured in a parsimonious way by traditional Markovian models, have a deep impact 
on the network performance and lead to the introduction  
of $\alpha$-stable distribution and self-similar processes into the network traffic modeling.
In this paper, a generalization of fractional Brownian motion (fBm), which is 
able to capture both above-mentioned features of the real traffic, is considered.} 


\KWE{fractional Brownian motion; $\alpha$-stable subordinator; self-similar processes; 
buffer overflow probability}

\vskip 14pt plus 9pt minus 6pt

      \thispagestyle{headings}

      \begin{multicols}{2}

            \label{st\stat}

\section{Introduction}

\noindent
The application of probabilistic methods in the modeling and the analysis of telecommunication 
systems has a long history.
Namely, the first researches in this framework date back to the beginning of the last 
century when A.\,K.~Erlang (1878--1929), as a scientific collaborator and the head of 
the newly-established physico-technical laboratory of the Copenhagen Telephone 
Company, studied the issues related to loss and waiting time in automatic telephone 
exchanges.
In the 1930s, the interests for these topics grew from a practical 
as well as theoretical point of view. Indeed, Erlang's results were soon used by 
telephone companies in several countries and gave birth to a new branch  in the framework 
of probability theory, known as queueing theory, which attracted the interests of 
well-known probabilists such as Palm, Pollachek, Lindly, Khincine, Gnedenko, to name 
just a few.     

In the 1920--1930s, many empirical works showed that, in case of 
telephone traffic, a suitable model is represented by the Poisson process. 
At the same time, Poisson flows have many ``useful'' mathematical properties:
\begin{itemize}
\item the superposition of independent Poisson processes is still a Poisson process;    
\item it has independent and stationary increments; and
\item under some mild regularity conditions, the superposition of independent flows 
converges to a Poisson flow, if the number of flows grows, but the individual rates 
become infinitesimal so that the overall rate stays constant. 
\end{itemize}

Because of the last property, in many works it has been proposed that the amount of 
traffic in global telecommunication backbones can be modelled as a Poisson process. 
For several decades, such model has been used without any further experimental 
validation and applied to new network scenarios, such as packet-switching networks.

At the beginning of the 1990s, a lot of empirical studies have been conducted in 
order to better understand the statistical features of packet traffic in global 
networks, such as Internet, as well as in local area networks inside research institutes, university campuses, 
and corporates~[1--3]. Statistical studies of the collected data highlighted their radical 
differences with respect to the ubiquitous Poisson process and other traditional (typically 
Markovian, for the sake of analytical tractability) models.  For instance, it is enough to visually 
check the behavior of real traffic data under different level of aggregations~[4]. 
It is easy to see that at all the aggregation levels (in the range from milliseconds to hours) 
the data keep a random behavior, which appears to be almost the same 
at all the different scales (apart from a normalization factor, related to the length of the 
observation window).

More accurate mathematical analyses~\cite{1-nik} 
pointed out that real data presents {\bf fractal} properties, i.\,e., they can be interpreted as 
trajectories of so-called {\bf automodel} or {\bf self-similar processes}.
Moreover, it was showed that traffic flows, unlike the Poisson model, presents 
{\bf long range dependence}, which has a huge impact on queuing performance.
The third important characteristic of traffic data is that the distribution of many different 
traffic features (such as file length, duration of on and off periods of single sources) presents 
{\bf heavy tails}. 

These  properties of actual traffic flows pointed out the necessity of new traffic models, able to 
captures them in a parsimonious way. It is worth mentioning that similar models were already known 
in the field of probability theory since they have been successfully applied in different frameworks, 
such as turbulence modeling and statistical physics.  

The rationale behind the fractal nature of traffics and the links among the above-mentioned 
characteristics of measured traces have been widely investigated~\cite{5-nik}. 
In particular, it has been 
shown that if locally the traffic load presents heavy tails, then under a sufficiently high level 
of aggregation it converges to a self similar process (for a precise formulation of the problem and 
the related scaling conditions (see~[6--9]). According to the considered aggregation 
regime, two 
different models might arise: fBm and $\alpha$-stable Levy motion, 
which, as will be clarified in the following, present ``opposite'' features.  
In more detail, fBm presents long range dependence, but the tails of its marginal distribution decay 
fast (by definition, according to Gaussian law!). On the contrary,  $\alpha$-stable Levy motion is 
characterised by independent increments (i.\,e., no long memory at all!), but has heavy tailed 
distribution (i.\,e., its tails decay as a power law). 

The goal is to build a model, able to take into account both these features of real traffic. 
Moreover, using such model as input to a queeing system, it would 
be also interesting to determine relevant 
queueing parameters, such as the probability of buffer overflow, which gives an upper bound for 
the loss probability in finite buffer queues. 

From the historical point of view, 
the first attempt to apply the fractional concept to traffic modelling was to use 
fBm $B_H (t)$ instead of traditional Poisson-based models. 
Compared to standard Brownian motion (BM), fBm has one extra parameter, the Hurst parameter 
H, which quantifies the strength of the fractional scaling. It is said usually, that 
fBm is self-similar, or fractional, with Hurst parameter~$H$. In~\cite{10-nik},
Norros has proposed the following model for cumulative traffic 
$$
A(t) = m t + (\sigma m)^{1/2} B_H (t) 
$$
where $m>0$ is the mean input rate, $\sigma$ is the scale factor. This model has been widely studied 
and have been proposed asymptotic lower bounds~\cite{10-nik} as well as exact asymptotics 
in the case of large buffers~\cite{11-nik, 12-nik}.

It is important to point out that in this case, one has a
long-range correlation, but not heavy tails of marginal 
a distributions.

To deal with this issue, several papers extended Norros model by modelling the input traffic as 
$\alpha$-stable Levy motion~\cite{13-nik, 14-nik} or, to take into account also the long range 
correlations, fractional $\alpha$-stable Levy motion (see~\cite{16-nik, 15-nik}).

In the paper, a new variant of  fractional Levy motion is suggeated and, following the 
approach proposed in~\cite{10-nik},
an asymptotic lower bound for the overflow probability is determined. 


\section{Stable Distributions and~Processes}

\noindent
Levy processes have been popular in modeling the teletraffic. Below, some 
definitions are given and some properties of such processes are considered. 

\smallskip

\noindent
\textbf{Definition~1.} \textit{A stochastic process $Y = (Y(t), t\geq 0)$ is a Levy process if}
\begin{enumerate}[(1)]
\item $Y(0) = 0$ \textit{almost surely;} 
\item
 $Y$ \textit{has independent increments; and}  
\item
$Y$ \textit{has stationary increments}. 
\end{enumerate}

\smallskip

Usually, for the sake of regularity,  the following property is required: with 
probability one all trajectories of $Y$ are right-continuous and have finite limits 
from the left. 

The distributions of the process $Y$ is defined uniquely by the distribution of 
random variable $Y(1)$, which is infinitely divisible. 

The most familiar example of Levy process is the BM (Weiner process). 

\smallskip

\noindent
\textbf{Definition~2.} 
\textit{A Levy process $B = (B(t), t\geq 0)$ is called Brownian Motion if for any 
$t\geq 0, h>0$ the increment 
$B(t+h) -B(t)$ has Gaussian distribution with zero meaning and variance $\sigma^2 h$}. 

\smallskip

If $\sigma^2 =1$,  one has a standard BM. It is easily seen that 
$$
K(t,s) = \mbox{Cov} \left(Y(t), Y(s)\right) = \sigma^2 \min (t,s) \, . 
$$

By definition, BM has Gaussian distributions. Such distributions have been got for normalized 
sums of independent identically distributed random variables with finite variance. 
In the case of 
infinite variance,  the so-called stable distributions are considered. 

\smallskip

\noindent
\textbf{Definition~3.} 
\textit{A random variable $Y$ is said to have an $\alpha$-stable distribution if its 
characteristic function has the following form:} 

\noindent
\begin{multline*}
\varphi (\omega ) := E\left[ e^{j\omega X} \right] \\
{}=
\exp \left\{ j\mu\omega - \sigma |\omega |^{\alpha} [1 - j \beta\, \mbox{sgn}\left(\omega \right) 
\theta (\omega , \alpha )] \right\}  
\end{multline*}
\textit{where $0< \alpha \leq 2$, $\sigma\geq 0$, $-1 \leq \beta \leq 1$, $\mu\in R^1$, and} 
$$
\theta (\omega , \alpha ) = 
\begin{cases}
\tan \left( \fr{\alpha \pi }{2} \right)\,, &\ \alpha \not= 1\,; \\[6pt]
-\fr{2}{\pi} \ln |\omega |\,,  &\ \alpha =1\,. 
\end{cases}
$$


\smallskip

Parameter $\alpha$ is called {\it characteristic exponent} and specifies the level of 
burstiness in distribution, i.\,e., it specifies the weight of the tails of the distribution. 
$\sigma$ and $\mu$ are called {\it scale} and {\it location parameters}. 
$\beta$~is called {\it skewness parameter}. If $\beta =0$ then $X$ is symmetrically 
distributed around~$\mu$. If $0<\alpha <1$, $\mu =0$ and $\beta =1$ then~$X$ 
has positive 
values with probability~1. In what follows, a random variable $Y$ is said to have  standard 
$\alpha$-stable distribution if $\mu =0$ and $\sigma = 1$. 

The $\alpha$-stable distribution is infinitely divisible. So, it generates some Levy process. 

\smallskip

\noindent
\textbf{Definition~4.} 
\textit{A stochastic process $L_{\alpha} = (L_{\alpha} (t) , t\geq 0)$ is said to be an 
$\alpha$-stable Levy motion if it is a Levy process such that $L_{\alpha} (1)$ has a given 
$\alpha$-stable distribution}.


\smallskip

If the distribution of $L_{\alpha} (1)$ is totally positive skewed ($0<\alpha <1$, 
$\beta =1$), then all trajectories of the process $L_{\alpha}$ are nondecreasing and 
nonnegative. Such process is called {\it $\alpha$-stable subordinator}. 

If $\alpha =2$, $\mu =0$, one has again BM~$B$. 

There exists very interesting relation between\linebreak $\alpha$-stable Levy motions with 
different~$\alpha$.

\smallskip

\noindent
\textbf{Theorem~1.} 
\textit{If $(L_{\alpha_1 } (t), t\geq 0)$, $0< \alpha_1 \leq 2$, is a $\alpha_1$-stable 
Levy motion with symmetric distributions, 
$(L_{\alpha_2 } (t),$\linebreak $t\geq 0)$, $0<\alpha_2 <1$, is a $\alpha_2$-stable subordinator, 
then stochastic process $Y = (Y(t):= L_{\alpha_1 } (L_{\alpha_2} (t)), t\geq 0)$ 
is $\alpha_1\alpha_2$-stable Levy motion with symmetric distributions.}

\smallskip

This theorem is a corollary of the following result by Zolotarev~[17, theorem~3.3.1]. 

\smallskip

\noindent
\textbf{Theorem~2.} 
\textit{If $Y_1$ has symmetric $\alpha_1$-stable distribution, 
$0<\alpha_1 \leq 2$, $Y_2$ has one-sided 
$\alpha_2$-stable distribution, $0<\alpha_2 <1$, then random variable 
$Y = Y_1 Y_2^{1/\alpha_1}$ has symmetric $\alpha_1\alpha_2$-stable 
distribution.} 

\smallskip

In particular, for $\alpha_1 =2$ and $0<\alpha_2 =\alpha/2 <1$, one gets the following 


\smallskip

\noindent
\textbf{Theorem~3.} 
\textit{If $B= (B(t), t\geq 0)$ is the Brownian motion, 
$L_{\alpha/2} = (L_{\alpha/2} (t) , t\geq 0)$ is a $\alpha/2$-stable subordinator, 
then  $L_{\alpha} = (L_{\alpha} (t) := B(L_{\alpha/2} (t)), t\geq 0)$, $0<\alpha <2$, is 
an $\alpha$-stable Levy motion with  symmetric distributions. }


\section{Self-Similar Processes} 


\noindent
\textbf{Definition~5.} 
\textit{A process $Y = (Y(t), t\geq 0)$ is self-similar, with Hurst parameter $H\geq 0$, if it 
satisfies the condition 
$$
Y(t) \stackrel{d}{=} c^{-H} Y(ct)\,, \ \forall t\geq 0\,, \ \forall c>0\,,
$$
where the equality is the sense of finite-dimensional distributions.}

\smallskip

Two of the most popular examples of self-similar processes are 
fBm and $\alpha$-stable Levy motion. 

\smallskip

\noindent
\textbf{Definition~6.}
\textit{The fractional Brownian motion with Hurst parameter $H$ is a Gaussian process 
$(B_H (t), t\geq 0)$ with zero mean and correlation function}
$$
K_H (t,s) = \fr{1}{2} \left[ |t|^{2H} + |s|^{2H} - |t-s|^{2H} \right] \,.
$$ 

\smallskip

The definition of $\alpha$-stable Levy motion see above. 

More information about stable and self-similar processes can be found in~\cite{18-nik, 19-nik}. 

\section{New Variant of~Fractional Levy Motion}

\noindent
Above, it was shown  how to get symmetric $\alpha$-stable Levy motion using 
BM and $\alpha/2$-stable subordinator. Below, it is proposed to use the same 
construction to get fractional Levy motion from fBm $B_H$ and 
$\alpha/2$-stable subordinator~$L_{\alpha/2}$. 

Let $(B_H (t), t\geq 0)$ be the fBm with Hurst parameter~$H$,  
$(L_{\alpha}^1 (t), t\geq 0)$, $(L_{\alpha}^2 (t), t\geq 0)$  be standard $\alpha$-stable 
subordinators, $0<\alpha < 1$, and~$B_H$, $L_{\alpha}^1$ and 
$L_{\alpha}^2$ are independent. Consider the new process 
$$
X(t) := 
\begin{cases}
B_H (L_{\alpha}^1 (t))\,, &\ t\geq 0 \,; \\[6pt]
- B_H (L_{\alpha}^2 (t))\,, &\  t < 0 \,. 
\end{cases}
$$

\noindent
\textbf{Theorem~4.} \textit{The above process $X$ is self-similar with Hurst parameter 
$H_1 = H/\alpha$}. 

\smallskip

\noindent
P\,r\,o\,o\,f\,.\  The processes $(L_{\alpha}^k (t), t\geq 0)$, $k=1,2$, 
are $\alpha$-stable 
and self-similar with Hurst parameter $1/\alpha$. So, one has
$$
(L_{\alpha}^k (ct), t\geq 0) \stackrel{d}{=} 
(c^{1/\alpha} L_{\alpha}^k (t), t\geq 0) \,.
$$
Then, 
\begin{multline*}
(X(ct) , t\in R^1 ) = \pm B_H (L_{\alpha}^k (c|t|), t\in R^1 ) \\
{}\stackrel{d}{=} 
(\pm B_H (c^{1/\alpha} L_{\alpha}^k (|t|) , t\in R^1 ) \,.
\end{multline*}
Using self-similarity of $B_H$ for fixed 
$\tau = L_{\alpha}^k (|t|)$, for any $a>0$, one has
$$
(B_H (a\tau ), \tau\geq 0) \stackrel{d}{=} (a^H  B_H (\tau), \tau\geq 0) 
$$
or 
$$
(\pm B_H (c^{1/\alpha} \tau ), \tau\geq 0) \stackrel{d}{=} 
(\pm c^{H/\alpha} B_H (\tau), \tau\geq 0) \,. 
$$
Due to the complete probability formula, the result is obtained. 

\smallskip

\noindent
\textbf{Corollary~1.} \textit{For any $t>0$, 
$$
X(t) \stackrel{d}{=} (L_{\alpha}^1 (t))^H  Y
$$
where $Y$ has standard normal distribution and $L_{\alpha}^1 (t)$ and~$Y$ are 
independent.} 

\smallskip

\noindent
\textbf{Remark.} Hurst parameter $H_1$ for above process $X$ can be any positive 
number. But for traffic applications, it is more interesting the case where 
$1/2 <H_1 <1$. So, it is assumed in what follows. 

\smallskip

\noindent
\textbf{Theorem~5.} 
\textit{The above process $X$ has stationary increments.}

\smallskip

\noindent
P\,r\,o\,o\,f\,.\ \ Fractional Brownian motion $B_H$ has stationary increments. So for any 
$t_1 < t_2 $
$$
B_H (t_2 ) - B_H (t_1 ) \stackrel{d}{=} B_H (t_2 - t_1 ) \,. 
$$ 
Then, for any $t\geq 0$, $h>0$ and fixed $L_{\alpha}^k (t+h)\hm = t_2$, 
$L_{\alpha}^k (t) = t_1$, one has 
$$
B_H (L_{\alpha}^k (t+h)) - B_H (L_{\alpha}^k (t)) \stackrel{d}{=} 
B_H (L_{\alpha}^k (t+h) -  L_{\alpha}^k (t)) \,. 
$$ 
Due to the complete probability formula, one has the same for random moments of time. 
The process $L_{\alpha}^k (t)$ has stationary increments too. So, one gets  
$$
B_H (L_{\alpha}^k (t+h) -  L_{\alpha}^k (t))  
\stackrel{d}{=} 
B_H (L_{\alpha}^k (h) ) \,.
$$ 


\section{Application to~Traffic Modeling} 

\noindent
Define the cumulative traffic (or arrival) process $A(t)$, i.\,e.,
the total amount of load produced 
by a source in the time interval $[0,t]$, $t>0$, by 
$$
A(t) := mt + (\sigma m)^{1/\beta} X(t) \,, 
$$
where $m>0$ is the mean input rate, $\sigma$ is the scale factor, 
$\beta = \alpha/H = 1/H_1$, $X$ is the process defined above. 


Consider a single server queue with constant service rate $r>0$ and infinite buffer 
space, where input is the stable self-similar process defined above 
($r>m$ for stability). The buffer occupancy 
$Q(t,r)$ at time $t\in R^1$ (queue size or queue length) can be written as 
$$
Q(t,r) = \sup\limits_{s\leq t} (A(t) - A(s) - r (t-s)) \,.
$$

\smallskip

Due to theorem~2, the process $Q = (Q(t,r)$, $t$\linebreak $\in R^1),$ is stationary. 
So, the most 
interesting is the following probability of overflow: 
$$
\varepsilon (b) = P(Q(0, r) > b) = P\left(\sup\limits_{\tau \geq 0} 
\left(A(\tau ) - r \tau \right) >b \right) \,.
$$

Using the technique elaborated in papers~\cite{1-nik, 13-nik}, one can get the lower bound for 
the probability of buffer overflow for large~$b$. 

It is easily seen that 
\begin{multline*}
\varepsilon (b) \geq 
\sup\limits_{\tau\geq 0} P( (A(\tau ) - r\tau ) > b ) \\
{}=
\sup\limits_{\tau\geq 0} P( m\tau +(\sigma m )^{1\/\beta} X(\tau ) - r\tau  > b )\\
{}=
\sup\limits_{\tau\geq 0} P\left( X(\tau ) > 
\fr{b+(r-m)\tau}{(\sigma m)^{1/\beta}}\right ) \\
{}=
\sup\limits_{\tau\geq 0} P\left( \tau^{1/\beta} X(1) > 
\fr{b+(r-m)\tau}{(\sigma m)^{1/\beta}}\right ) \\
{}=
\sup\limits_{\tau\geq 0} P\left( X(1) > 
\fr{b+(r-m)\tau}{(\sigma m \tau )^{1/\beta}}\right ) \,. 
\end{multline*}
Last probability under supremum is a decreasing function of the value 
$$
f(\tau ) = \fr{b+(r-m)\tau}{(\sigma m \tau )^{1/\beta}}\,.
$$
Elementary calculations give us that the minimal value of this function is achieved at the 
point 
$$
\tau_0 = \fr{b}{\beta (1-1/\beta )(r-m)} = \fr{b H_1}{(1-H_1 ) (r-m)} \,. 
$$
It follows 
$$
\varepsilon (b) \geq P(X(1) > f(\tau_0 ) = b_1 ) 
$$ 
where 
$$
b_1 = \fr{(r-m)^{H_1} (1-H_1 )^{-(1-H_1 )} }{(\sigma m H_1 )^{H_1 } } \, b^{1-H_1} \,. 
$$
Using corollary 1, one gets 
\begin{multline*}
P(X(1) > b_1 ) = P((L_{\alpha}^1 (1))^H Y >b_1 ) \\
{}\geq  
P((L_{\alpha}^1 (1))^H Y >b_1 , Y>1) \\
{}\geq
P((L_{\alpha}^1 (1))^H >b_1 , Y>1)\\ = P(L_{\alpha}^1 (1) 
>(b_1 )^{1/H} ) P(Y>1) \,. 
\end{multline*}
For large $x>0$ (see~[20, theorem~2.4.1]), one has 
$$
 P(L_{\alpha}^1 (1) > x ) \sim C(\alpha ) x^{-\alpha} 
$$
where 
$$
C(\alpha ) = \fr{\sin (\pi\alpha )}{\pi} \Gamma (\alpha )\,.
$$
It follows for large~$b$ 
\begin{multline*}
\varepsilon (b) \geq C(\alpha ) (b_1 )^{-1/H_1} P(Y>1)\\
{}=
C_1 (\alpha , H_1 )\sigma \fr{m}{r-m} \, b^{-({1-H_1})/H_1} \,.
\end{multline*}
Finally, one has the following 

\pagebreak

\smallskip

\noindent
\textbf{Theorem~6.} \textit{An asymptotic lower bound for the overflow probability is given by}
$$
\varepsilon (b) \geq
C_1 (\alpha , H_1 )\sigma \fr{m}{r-m} \, b^{-({1-H_1})/{H_1} } \,, \quad
b\to\infty \,.
$$

\vspace*{-12pt}

{\small\frenchspacing
{%\baselineskip=10.8pt
\addcontentsline{toc}{section}{Литература}
\begin{thebibliography}{99}

\bibitem{1-nik}
\Au{Leland~W., Taqqu~M., Willinger~W.,  Wilson~D}. On the selfsimilar
nature of Ethernet traffic (extended version)~// IEEE/ACM
Trans. Networking, 1994. P.~1--15. 

\bibitem{3-nik} %2
\Au{Park K., Kim G., Crovella~M.}  
On the relationship between file sizes, transport protocols, and self-similar 
network traffic~//  Conference (International) on Network Protocols Proceedings, October 1996. 
P.~171--180. 

\bibitem{2-nik} %3
\Au{Crovella M.\,E., Bestavros~A.} Self-similarity in world wide web traffic: Evidence 
and possible  causes~// IEEE/ACM Transactions on Networking, December 1997. Vol.~5. No.\,6. 
P.~835--846. 


\bibitem{4-nik}
\Au{Willinger W., Paxson~V.} Where mathematics meets the Internet~// Notices of the AMS, 1998. 
Vol.~45. No.\,8. P.~961--970.  

\bibitem{5-nik}
\Au{Park~K., Willinger~W.} Self-similar network traffic and performance 
evaluation.~--- Wiley, 2000. 

\bibitem{6-nik}
\Au{Taqqu M.\,S., Levy J.\,B.} Using renewal processes to
generate long-range dependence and high variability~// 
Dependence in probability and statistics~/
Eds.\ E.~Eberlein and M.\,S.~Taqqu.~---  Boston: Birkhauser, 1986.  P.~73--89. 

\bibitem{7-nik}
\Au{Taqqu M.\,S., Willinger~W., Serman~R}. Proof of a
fundamental result in self-similar traffic modeling~// Computer Communications
Rev., 1997. Vol.~27. No.\,2. P.~5--23. 

\bibitem{8-nik}
\Au{Levy J.\,B., Taqqu M.\,S.} Renewal reward processes with
heavy-tailed inter-aarival times and heavy tailed rewards~// Bernoulli, 2000. Vol.~6. No.\,1. 
P.~23--44. 

\bibitem{9-nik}
\Au{Mikosch Th., Resnick S., Rootzen~H., Stegeman~A.} Is network
traffic approximated by stable Levy motion or frac-\linebreak\vspace*{-12pt}

\columnbreak

\noindent
tional Brownian motion?~//
Ann. Appl. Probab., 2002. Vol.~12. No.\,1. P.~23--68. 

\bibitem{10-nik}
\Au{Norros I.} A storage model with self-similar imput~// 
Queuing Syst., 1994. Vol.~16. P.~387--396. 

\bibitem{11-nik}
\Au{Narayan~O.} Exact asymptotic queue length distribution for fractional Brownian traffic~//
  Adv. Perf. Anal., 1998. Vol.~1. P.~39--63.

\bibitem{12-nik}
\Au{H$\ddot{\mbox{u}}$sler J., Piterbarg~V.} Extremes of a certain class of Gaussian processes~// 
Stoch. Proc. Appl., 1999. Vol.~83. P.~257--271.

\bibitem{13-nik}
\Au{Laskin N., Lambadaris I., Harmantzis~F.\,C., Devetsikiotis~M.}
Fractional Levy motion and its application to network traffic modeling~// 
Computer Networks, 2002. Vol.~ 40. P.~ 363-375. 

\bibitem{14-nik}
\Au{Garroppo~R.\,G., Giordano~S., Pagano~M., Procissi~G.}
Testing $\alpha$-stable processes in capturing the queuing behavior of
broadband teletraffic~// Signal Proc., 2002. Vol.~82.  P.~1861--1872.  

\bibitem{16-nik}
\Au{Gallardo J.\,R., Makrakis~D., Orozco-Barbosa~L.}  Use of $\alpha$-stable 
self-similar stochastic processes for modeling traffic~//  Performance Eval., 
2000. Vol.~40. P.~71--98.

\bibitem{15-nik}
\Au{Karasaridis~A.} Network heavy traffic modeling using\linebreak $\alpha$-stable
self-similar processes~// IEEE Transactions on  Communications, July 2001. Vol.~49. No.\,7. 
P.~1203--1214.


\bibitem{17-nik}
\Au{Zolotarev V.\,M.} One-dimensional stable distributions.~--- 
Translations of mathematical monographs.  AMS, 1986. Vol.~65.

\bibitem{18-nik}
\Au{Samorodnitsky~G.,  Taqqu~M.\,S.} Stable
non-Gaussian random processes.~---  Chapman \& Hall, 1994. 

\bibitem{19-nik}
\Au{Embrechts~P., Maejima~M.} Self-similar process.~---
Prinston University Press, 2002. 

\bibitem{20-nik}
\Au{Ibragimov I.\,A., Linnik~Yu.\,V.}  Independent and stationary sequences of 
random variables.~--- Gronengen: Wolters-Noordhoff, 1971. 

 \end{thebibliography}
}
}


\end{multicols}

\vspace*{3pt}

\hrule

%\vspace*{pt}


\def\tit{ДРОБНОЕ ДВИЖЕНИЕ ЛЕВИ С~ЗАВИСИМЫМИ ПРИРАЩЕНИЯМИ 
И~ЕГО~ПРИЛОЖЕНИЕ К~МОДЕЛИРОВАНИЮ СЕТЕВОГО  ТРАФИКА}

\def\aut{К.~Де~Никола$^1$, Ю.\,С.~Хохлов$^2$,  М.~Пагано$^3$,  О.\,И.~Сидорова$^4$}

\titelr{\tit}{\aut}

%\vspace*{2pt}

\noindent
$^1$Университет г.~Салерно, denicola@diima.unisa.it\\
\noindent
$^2$Российский университет дружбы народов, yskhokhlov@yandex.ru\\
\noindent
$^3$Университет г.~Пиза, m.pagano@iet.unipi.it\\
\noindent
$^4$Тверской государственный университет, Oksana.I.Sidorova@yandex.ru\\

\vspace*{-6pt}


\Abst{С начала 1990-х~гг.\ были проведены многочисленные высокоточные измерения для различных 
сетевых сценариев, которые показали, что трафик в Интернете проявляет сильную иррегулярность, выраженную 
в чрезвычайной вариабельности, а так\-же в наличии долговременной зависимости. Эти новые особенности, 
которые не удается описать экономным образом с помощью традиционных марковских моделей, имеют сильное 
влияние на поведение сети, и это привело к необходимости введения в моделирование сетевого трафика 
$\alpha$-устой\-чи\-вых распределений и самоподобных процессов.
В~настоящей работе рассматривается некоторое обобщение дробного броуновского движения, которое 
позволяет охватить одновременно обе отмеченные выше особенности реального трафика.}

\label{end\stat}


\KW{дробное броуновское движение; $\alpha$-устойчивый субординатор; самоподобные 
процессы; вероятность переполнения буфера}


\renewcommand{\figurename}{\protect\bf Рис.}
\renewcommand{\tablename}{\protect\bf Таблица}
\renewcommand{\bibname}{\protect\rmfamily Литература}