\documentclass[10pt]{book}
\usepackage[utf8]{inputenc}

\usepackage{latexsym,amssymb,amsfonts,amsmath,indentfirst,shapepar,%fleqn,%
picinpar,shadow,floatflt,enumerate,multicol,colortbl,ipi}

\usepackage{rotating}
\usepackage{mathrsfs}

\input{epsf}

%\nofiles

%\includeonly{avtor,avtor-eng} %+pdf
%\includeonly{avtor-eng}
%\includeonly{pred}      %+pdf
%\includeonly{podgot-1str}  %+
%\includeonly{ocherk} %+



%\includeonly{zhelenkova}     %1+pdf+авт
%\includeonly{zagorulko}      %2+pdf
%\includeonly{kogalovski}     %3+pdf+авт
%\includeonly{skachkov}       %4pdf
%\includeonly{sharapov}       %5 %+pdf+авт


%\includeonly{nikola}         %6+pdf+авт
%\includeonly{yanuskev}       %7+pdf+авт
%\includeonly{bening}         %8+pdf+авт
%\includeonly{gadamaka}       %9+pdf+авт
%\includeonly{mor-luk}        %10+pdf+авт
%\includeonly{mor-nekr}       %11+pdf+авт
%\includeonly{mor-rum}        %12+pdf+авт
%\includeonly{pechinkin-new}  %13+pdf
%\includeonly{ushakov}        %14+pdf


%\includeonly{toc-rus, toc-en}
%\includeonly{obchak,toc-en}

%\includeonly{obchak}
%\includeonly{reshal}  %pdf
%\includeonly{eng-index}
%\includeonly{cover3}

\usepackage{acad}
%\usepackage{courier}
\usepackage{decor}
\usepackage{newton}
\usepackage{pragmatica}
\usepackage{zapfchan}
\usepackage{petrotex}
\usepackage{bm}                     % полужирные греческие буквы
\usepackage{upgreek}                % прямые греческие буквы
\usepackage{eufrak}
%\usepackage{verbatim}

\renewcommand{\bottomfraction}{0.99}
\renewcommand{\topfraction}{0.99}
\renewcommand{\textfraction}{0.01}

\setcounter{secnumdepth}{1} %здесь - 3 + chapter = 4

\arraycolsep=1.5pt

%\usepackage[pdftex]{graphicx}

%\usepackage{oz}

%NEW COMMANDS


\renewcommand*{\hm}[1]{#1\nobreak\discretionary{}%
            {\hbox{$\mathsurround=0pt #1$}}{}} %% Дублирует знаки операций
                               %при переносе в формуле (перед знаком, который 
                               %надо продублировать ставится команда \hm)

%\renewcommand{\endproof}{\hfill$\Box$}
\renewcommand{\r}{\mathbb{R}}
\newcommand{\I}{{\rm I\hspace{-0.7mm}I}}
%\newcommand{\Ikl}{{\tt{1}}\hspace*{-1.44mm}\mathtt{1}}
\newcommand{\Ik}{\mbox{{\small \tt {1}}\hspace{-1.5mm}{\tt 1}}}
\newcommand{\argmin}{\mathop{\mathrm{arg}\,\mathrm{min}}}
%\def\argmin{\mathop{arg\,min}}

\def\vrp{\varphi}
\def\prt{\partial}
\def\mm{{\rm M}}

\newcommand{\il}[2]{\int\limits_{#1}^{#2}}%интеграл с пределами #1 и #2


\def\sss{\sum\limits}
\def\tr{\,,\,\ldots\,,\,}
\def\rk{\right]}
\def\lk{\left[}
\def\rf{\right\}}
\def\lf{\left\{}

\def\ee{{\cal E}}
\def\ww{{\cal W}}
\def\yy{{\cal Y}}
\def\vv{{\cal V}}

\newcommand{\R}{\mathbb R}
\newcommand{\N}{\mathbb N}

\newcommand{\h}{{\bf H}}
\newcommand{\p}{{\sf P}}  % вероятность
%\newcommand{\P}{\mathbb{P}}
\newcommand{\e}{{\sf E}}  % мат. ожидание
\newcommand{\D}{{\sf D}}  % дисперсия
\newcommand{\eps}{\varepsilon}
\newcommand{\vp}{{\mathbf p}}
\newcommand{\vz}{{\mathbf z}}
\newcommand{\vx}{{\mathbf x}}
\newcommand{\vf}{{\mathbf f}}
%\newcommand{\vp}{\mathrm{v.p.}}
\newcommand{\F}{{\mathcal F}}
\def\ap{{\mathrm{ЭР}}}

\newcommand{\abs}[1]{\left\vert#1\right\vert}
\def\w{\omega}
\def\W{\Omega}
\def\iii{\int\limits}
\def\iin{\int\limits_{-\infty}^\infty}

\DeclareMathOperator{\sign}{sign}

%\newcommand{\gr}{{\geqslant}}

\newcommand{\g}{\mbox{\textit{g}}}

\renewcommand{\la}{\lambda}
\newcommand{\si}{\sigma}
\newcommand{\alp}{\alpha}

%\newcommand{\pto}{\stackrel{P}{\longrightarrow}} % сходимость по веpоятности

%\newcommand{\eqd}{\stackrel{d}{=}} % равенство по pаспpеделению

%\newcommand{\kp}{\kappa}
%\def\Q{{\cal Q}} \def\H{{\cal H}}
%\newcommand{\bet}{\beta_{2+\delta}}


%\newtheorem{definition}{Определение}
%\renewcommand{\thedefinition}{\arabic{definition}.}
%END NEW COMMANDS

%\renewcommand{\baselinestretch}{1.2}

%\pagestyle{myheadings}

\setlength{\textwidth}{167mm}      % 122mm
\setlength{\textheight}{658pt}
%\setlength{\textheight}{635.6pt}
\setlength{\columnsep}{4.5mm}

\setcounter{secnumdepth}{4}

%\addtolength{\headheight}{2pt}
%\addtolength{\headsep}{-2mm}

%\addtolength{\topmargin}{-20mm}  % for printing


\hoffset=-30mm  % From Yap
%\hoffset=-20mm  % From Acrobat

%\voffset=0mm % From Yap
%\voffset=-15mm   % From Acrobat

\addtolength{\evensidemargin}{-9.5mm} % for printing
\addtolength{\oddsidemargin}{9.5mm}  % for printing

%\renewcommand{\thefootnote}{\fnsymbol{footnote}}
%\renewcommand{\thefootnote}{\arabic{footnote}}
\renewcommand{\figurename}{\protect\bf Рис.}
\renewcommand{\tablename}{\protect\bf Таблица}

\newcommand{\Caption}[1]{\caption{\protect\small %\baselineskip=2.5ex
#1}}

\renewcommand{\thefigure}{\arabic{figure}}
\renewcommand{\thetable}{\arabic{table}}
\renewcommand{\theequation}{\arabic{equation}}
\renewcommand{\thesection}{\arabic{section}}

\renewcommand{\contentsname}{СОДЕРЖАНИЕ}
\newcommand{\fr}[2]{\displaystyle\frac{\displaystyle #1\mathstrut}{\displaystyle #2\mathstrut}}

%\renewcommand{\thefootnote}{\fnsymbol{footnote}}
%\newcommand{\g}{\mbox{\textit{g}}}

%\newcommand{\Caption}[1]{\caption{\protect\small\baselineskip=2ex #1}}
\newcounter{razdel}
\setcounter{razdel}{0}


\newcommand{\titel}[4]{%
\

\vspace*{5pt}

\ifodd\therazdel {\raggedright\noindent\Large\textrm\textbf
 \lineskip .75em
  \baselineskip=3.2ex #1 \par}
\vskip 1em {\noindent\large\textrm\textbf #2 \par}
\addcontentsline{toc}{subsection}{{\textrm\textbf #3}\protect\newline #1}
\def\rightheadline{\underline{\noindent\hbox to \textwidth{\hfill\small\textrm{#4}
%\hfill \large\bf\thepage
}}}
\def\leftheadline{\underline{\noindent\parbox{\textwidth}{
%\raggedleft\large\bf\thepage \hfill
\small\textit{#3}\hfill}}}
\def\leftfootline{\small{\textbf{\thepage}
\hfill ИНФОРМАТИКА И ЕЁ ПРИМЕНЕНИЯ\ \ \ том~6\ \ \ выпуск 3\ \ \ 2012}
}%
 \def\rightfootline{\small{ИНФОРМАТИКА И ЕЁ ПРИМЕНЕНИЯ\ \ \ том~6\ \ \ выпуск~3\ \ \ 2012
\hfill \textbf{\thepage}}} 
\vskip 2em \setcounter{figure}{0}
\setcounter{table}{0} 
\setcounter{equation}{0} 
\setcounter{section}{0}
\setcounter{subsection}{0} 
\setcounter{subsubsection}{0}
\setcounter{footnote}{0} 
\setcounter{razdel}{0}
%\end{flushleft}
\else {
 \raggedright\noindent\Large\textrm\textbf
 \lineskip .75em
\baselineskip=3.2ex #1 \par} \vskip 1em
%\begin{flushleft}
{\noindent\large\textrm\textbf #2 \par}
\addcontentsline{toc}{subsection}{{\textrm\textbf #3}\protect\newline #1}
\def\rightheadline{\underline{\noindent\hbox to \textwidth{\hfill\small\textrm{#4}
%\hfill \large\bf\thepage
}}}
\def\leftheadline{\underline{\noindent\parbox{\textwidth}{%\raggedleft\large\bf\thepage \hfill
\small\textit{#3}\hfill}}}
\def\leftfootline{\small{\textbf{\thepage}
\hfill ИНФОРМАТИКА И ЕЁ ПРИМЕНЕНИЯ\ \ \ том~6\ \ \ выпуск~3\ \ \ 2012}
}%
 \def\rightfootline{\small{ИНФОРМАТИКА И ЕЁ ПРИМЕНЕНИЯ\ \ \ том~6\ \ \ выпуск~3\ \ \ 2012
\hfill \textbf{\thepage}}} \vskip 2em \setcounter{figure}{0}
\setcounter{table}{0} \setcounter{equation}{0} \setcounter{section}{0}
\setcounter{subsection}{0} \setcounter{subsubsection}{0}
\setcounter{footnote}{0}
%\end{flushleft}
\fi}

\newcommand{\titelr}[2]{%
\

\vspace*{5pt}

\ifodd\therazdel {\raggedright\noindent\large\textrm\textbf
 \lineskip .75em
  \baselineskip=3.2ex #1 \par}
\vskip 1em {\noindent\normalsize\textrm\textbf #2 \par}
\else {
 \raggedright\noindent\large\textrm\textbf
 \lineskip .75em
\baselineskip=3.2ex #1 \par} \vskip 1em
%\begin{flushleft}
{\noindent\normalsize\textrm\textbf #2 \par}
\fi}

\newcommand{\titele}[5]{%
\

%\vspace*{5pt}

\ifodd\therazdel {\raggedright\noindent%\large
\textrm\textbf
 \lineskip .75em
%  \baselineskip=3.2ex
#1 \par}
\vskip .5em {\noindent\large\textrm\textbf #2 \par}
\vskip .5em
 {\noindent\textrm #3 \par}
\addcontentsline{toc}{subsection}{{\textrm\textbf #1}\protect\newline #2}
\def\rightheadline{\underline{\noindent\hbox to \textwidth{\hfill\small\textrm{#4}
%\hfill \large\bf\thepage
}}}
\def\leftheadline{\underline{\noindent\parbox{\textwidth}{
%\raggedleft\large\bf\thepage \hfill
\small\textrm{#5}\hfill}}}
\def\leftfootline{\small{\textbf{\thepage}
\hfill ИНФОРМАТИКА И ЕЁ ПРИМЕНЕНИЯ\ \ \ том~6\ \ \ выпуск~3\ \ \ 2012}
}%
 \def\rightfootline{\small{ИНФОРМАТИКА И ЕЁ ПРИМЕНЕНИЯ\ \ \ том~6\ \ \ выпуск~3\ \ \ 2012
\hfill \textbf{\thepage}}} \vskip 1em \setcounter{figure}{0}
\setcounter{table}{0} \setcounter{equation}{0} \setcounter{section}{0}
\setcounter{subsection}{0} \setcounter{subsubsection}{0}
\setcounter{footnote}{0} \setcounter{razdel}{0}
%\end{flushleft}
\else {
 \raggedright\noindent%\large
 \textrm\textbf
 \lineskip .75em
%\baselineskip=3.2ex
#1 \par} \vskip .5em
%\begin{flushleft}
{\noindent\large\textrm\textbf #2 \par} \vskip .5em
 {\noindent\textrm #3 \par}
\addcontentsline{toc}{subsection}{{\textrm\textbf #1}\protect\newline #2}
\def\rightheadline{\underline{\noindent\hbox to \textwidth{\hfill\small\textrm{#4}
%\hfill \large\bf\thepage
}}}
\def\leftheadline{\underline{\noindent\parbox{\textwidth}{%\raggedleft\large\bf\thepage \hfill
\small\textrm{#5}\hfill}}}
\def\leftfootline{\small{\textbf{\thepage}
\hfill ИНФОРМАТИКА И ЕЁ ПРИМЕНЕНИЯ\ \ \ том~6\ \ \ выпуск~3\ \ \ 2012}
}%
 \def\rightfootline{\small{ИНФОРМАТИКА И ЕЁ ПРИМЕНЕНИЯ\ \ \ том~6\ \ \ выпуск~3\ \ \ 2012
\hfill \textbf{\thepage}}} \vskip 1em \setcounter{figure}{0}
\setcounter{table}{0} \setcounter{equation}{0} \setcounter{section}{0}
\setcounter{subsection}{0} \setcounter{subsubsection}{0}
\setcounter{footnote}{0}
%\end{flushleft}
\fi}

\def\Abst#1{
\begin{center}\small\nwt
\parbox{150mm}{%\baselineskip=2.5ex
\textbf{Аннотация:}\ \
%\hspace*{\parindent}
#1}
\end{center}}
\def\Abste#1{
\begin{center}\small\nwt
\parbox{150mm}{%\baselineskip=2.5ex
\textbf{Abstract:}\ \
%\hspace*{\parindent}
#1}
\end{center}}

\def\KW#1{
\begin{center}\small\nwt
\parbox{150mm}{%\baselineskip=2.5ex
\textbf{Ключевые слова:}\ \ #1}
\end{center}}

\def\KWE#1{
\begin{center}\small\nwt
\parbox{150mm}{%\baselineskip=2.5ex
\textbf{Keywords:}\ \ #1}
\end{center}}


\def\KWN#1{
%\begin{center}
%\small
%\parbox{150mm}\end{center}
}

\renewcommand{\thesubsection}{\thesection.\arabic{subsection}\hspace*{-5pt}}
\renewcommand{\thesubsubsection}{\thesubsection\hspace*{5pt}.\arabic{subsubsection}\hspace*{-3pt}}

\begin{document}
\Rus

\nwt
%\ptb

%\renewcommand{\contentsname}{\protect\Large\bf Содержание}

\setcounter{tocdepth}{2}

%\tableofcontents

\renewcommand{\bibname}{\protect\rmfamily Литература}
  \def\Au#1{{\it #1}}

%\newcommand{\No}{№}
  \newcommand{\tg}{\,\mathrm{tg}\,}
    \newcommand{\ctg}{\,\mathrm{ctg}\,}
  \newcommand{\arctg}{\,\mathrm{arctg}\,}
  
\def\forallb{\mathop{\forall}}
\def\cupb{\mathop{\cup}}
\def\existsb{\mathop{\exists}}

\setcounter{page}{1}

\newpage
\addtocounter{razdel}{1}
%\def\razd{РЕГУЛИРУЕМЫЙ ЭЛЕКТРОПРИВОД ДЛЯ ЭЛЕКТРОЭНЕРГЕТИКИ}
%\newpage
%\def\stat{zakh}
\def\tit{СРЕДСТВА ОБЕСПЕЧЕНИЯ ОТКАЗОУСТОЙЧИВОСТИ ПРИЛОЖЕНИЙ}
\def\titkol{Средства обеспечения отказоустойчивости приложений}

\def\aut{В.\,Н.~Захаров$^1$, В.\,А.~Козмидиади$^2$}
\titel{\razd}{\tit}{\aut}{\titkol}


\Abst{Рассмотрены проблемы построения отказоустойчивых серверов, возникающие в связи с недетерминированностью поведения приложений. Предложена формальная модель, описывающая поведение приложения, основными объектами которой являются ресурсы и события. Предложены алгоритмы протоколирования работы приложения на резервном узле кластера, а также восстановления и продолжения его работы при отказе основного узла. При этом для клиентов сбой остается незаметным, за исключением некоторого увеличения времени обслуживания.}

\KW{сервер приложений $\bullet$ прозрачная отказоустойчивость $\diamond$
 процесс $\diamond$ ресурс $\diamond$ событие $\diamond$ контрольная точка
$\bullet$ детерминированность}

\vskip 12pt plus 6pt minus 3pt

\begin{multicols}{2}

\section*{ВВЕДЕНИЕ}

Средства вычислительной техники стали использоваться в областях,
требующих безотказной работы систем в течение многих лет (24 часа
в сутки, 365 дней в году).

\label{st\stat}

\footnotetext{$^1$ФГУП Центральный институт авиационного моторостроения
им. П.И. Баранова, Москва, Россия}
\footnotetext{$^2$ФГУП Центральный институт авиационного моторостроения
им. П.И. Баранова, Москва, Россия}

К таким областям относятся, например, центры хранения и обработки данных  в сетях (системы резервирования билетов, биллинговые,  банковские и т.д.), массированные распределенные вычисления (GRID-вычисления) и другие.

\thispagestyle{headings}

Обычно в подобных системах применяются частные решения, ориентированные в основном на обеспечение надежного хранения данных (например, файловые серверы, использующие для хранения RAID-контроллеры) и корректного их состояния при отказах (серверы баз данных с транзакционным выполнением запросов). Однако большинство приложений не гарантируют, что не произойдет потери части данных при отказе системы. Обычно предполагается, что клиентские средства должны повторять запросы после восстановления серверов, для того, чтобы данные не были потеряны, или что можно сделать возврат по времени на некоторое время назад и повторить работу с этого места. Однако далеко не все клиентские средства и условия применения приложений допускают это.

Отказоустойчивые системы для критически важных приложений, корректно решающие проблемы восстановления после сбоев,   предлагаемые ведущими производителями, как правило, дороги. Кроме того, они включают специфические серверные и клиентские приложения, не совместимые со стандартными приложениями, не обеспечивающими отказоустойчивость. Примером такого подхода к решению проблемы отказоустойчивости  хранения данных являются системы NetApp FAS компании Network Appliance, работающие на базе специализированной операционной системы Data ONTAP [1].

Построение отказоустойчивых систем, использующих серверы со стандартными приложениями, в свете вышесказанного, является актуальной проблемой, вызывающей значительный интерес. Рассмотрение методов достижения прозрачной отказоустойчивости таких систем и является предметом статьи.
\begin{figure*} %fig1
\vspace*{1pt}
\begin{center}
\mbox{%
\epsfxsize=1.6in
\epsfxsize=100mm
\epsfbox{BbR-1.eps}
}
\end{center}
\vspace*{-9pt}
\Caption{Базовый вариант трубы с разными выходными устройствами
(цилиндрическое, расширяющееся и сужающееся сопло)
\label{f1bab}}
\vspace*{-3pt}
\end{figure*}


\section{ОСНОВНЫЕ ПОНЯТИЯ И ПОДХОДЫ}

Под сервером в данной работе понимается вычислительный центр
(отдельный компьютер или кластер) в сети, предоставляющий клиентам
(пользователям, клиентским компьютерам) определенные услуги, разделяя
между ними свои ресурсы. Подобные серверы названы серверами приложений.
Широко распространенным примером сервера такого типа является файловый сервер, обеспечивающий удаленный коллективный доступ к файловой системе. Часто используются вычислительные серверы, предоставляющие клиентам возможность выполнять на них свои программы (например, в центрах коллективного пользования).


Обычно приложение представляет собой программу или группу программ, работающих в операционной среде, создаваемой операционной системой (в другой терминологии - один или несколько взаимодействующих процессов или потоков (threads)), которые реализуют функциональность сервера. Для построения отказоустойчивых серверов приложений широко используется кластерная технология. Следуя [2], кластером, названа разновидность параллельной или распределенной системы, которая:
\begin{itemize}
\item состоит из нескольких компьютеров (узлов кластера), связанных как минимум необходимыми коммуникационными каналами;
\item используется как единый, унифицированный компьютерный ресурс.
\end{itemize}

Прозрачная отказоустойчивость (Transparent Fault Tolerance, TFT) сервера приложений - это такое его поведение при возникновении аппаратных или программных отказов либо отказов в сети, при котором:
\begin{itemize}
\item отказ не вызывает потери или искажения данных, находящихся в базе данных сервера;
\item сервер продолжает нормально функционировать, несмотря на имевшие место отказы.
\end{itemize}

Клиенты сервера "не замечают" произошедших отказов. Единственным\footnote{допустимым
отклонением сервера от нормального поведения с точки зрения клиента является
некоторое увеличение времени обслуживания} (на несколько секунд или десятков секунд).

Обычно приложения, работающие на серверах приложений, не ориентированы на прозрачную отказоустойчивость. Они "заботятся" лишь о собственной целостности (например, состояния файловой системы или базы данных). Восстановление работоспособности сервера приводит к разрыву соединений с клиентами и потере их запросов. Это замечают клиенты - запросы следует повторять, на что клиентские приложения далеко не всегда рассчитаны. В данной работе предполагается, что приложения (прикладные программные средства), выполняемые на сервере, являются стандартными, то есть не имеют специальных средств, обеспечивающих отказоустойчивость.
\begin{figure*}[b] %fig1
\vspace*{1pt}
\begin{center}
\mbox{%
\epsfxsize=1.6in
\epsfxsize=100mm
\epsfbox{BbR-1.eps}
}
\end{center}
\vspace*{-9pt}
\Caption{Базовый вариант трубы с разными выходными устройствами
(цилиндрическое, расширяющееся и сужающееся сопло)
\label{f1bab}}
\vspace*{-3pt}
\end{figure*}

Серьезные исследования в области обеспечения отказоустойчивости серверов были развернуты после создания вычислительных серверов, предназначенных для решения задач, требующих больших вычислительных ресурсов. Решение этих задач выполняется на суперкомпьютерах, обеспечивающих массово-параллельные вычисления и представляющих собой кластеры из сотен и тысяч узлов (процессоров). Однако даже на этих "монстрах" решение может требовать десятков или сотен часов, и одиночный сбой, если не предприняты специальные меры, может привести к необходимости начинать работу сначала. Обычно решение вычислительной задачи в таких случаях осуществляется в модели относительно редко взаимодействующих между собой процессов, выполняемых на разных узлах кластера. Эти взаимодействия нужны для координации работы процессов, в частности, для обмена данными и промежуточными результатами. Взаимодействия опираются на специальный протокол, называемый MPI (Message-Passing Interface) и представляющий собой стандарт "de facto" [3].

Для преодоления последствий сбоя достаточно давно была разработана и широко применяется технология, опирающаяся на механизм контрольных точек (checkpoints) [4-6]. По этой технологии система должна иметь стабильную память, которая не меняется при отказах. Соответствующие программные средства периодически сохраняют информацию о состоянии процессов приложения в стабильной памяти. Все процессы также имеют доступ к устройству стабильной памяти.  В случае отказа или сбоя, записанная в стабильную память информация используется для повторения вычисления с момента, когда была записана эта информация, то есть выполняется откат назад по времени. Данные, сохранение которых позволяет выполнить откат, называются контрольной точкой. В качестве устройства стабильной памяти может использоваться дисковый том, энергонезависимая оперативная память, память другого узла или узлов кластера. В последнем случае узел, которому требуется сохранить информацию, пересылает ее через быстрый канал связи на другой узел. Стабильная память после отказа одного из узлов должна быть доступной узлу, на котором делается повтор.

Однако решение, опирающееся только на контрольные точки, не является прозрачным, поскольку не скрывает от клиентов факт отказа системы и требует от них выполнения определенных действий. Так как при работе процессы обмениваются сообщениями, возможны два варианта решения проблемы. Первый - все процессы выполняют записи контрольных точек одновременно, что затруднительно. Второй вариант, при несоблюдении синхронности, - возврат в каждом процессе к такому скоординированному набору контрольных точек, при котором невозможна противоречивая ситуация. Такая ситуация возникает, когда один процесс вернулся к контрольной точке, после которой он должен получить сообщение от другого процесса, а этот другой процесс вернулся к точке, которая следует за выдачей этого сообщения. Однако при повторе ожидаемое первым процессом сообщение не поступит. В этом случае  возможен эффект домино, в результате процессы оказываются отброшены как угодно далеко назад.

В этом состоит первая проблема, которую необходимо преодолеть.

Если нужно, чтобы последствия отказа узла не были видны клиенту,  это означает:
\begin{itemize}
\item клиент не должен терять и потом восстанавливать соединения с сервером;
\item клиент не должен повторять свои запросы;
\item клиент не должен повторно получать сообщения, которые он уже получил.
\end{itemize}

Вторая проблема, которую надо решать, связана с недетерминированностью поведения сервера приложений. Приведем пример.  Пусть имеется система продажи билетов на самолеты. Два клиента одновременно обратились к системе с запросом билета на один и тот же рейс. Клиентам безразлично, какие места им зарезервирует система. Система выполняет запросы клиентов параллельно, поэтому в какой-то момент между процессами, обрабатывающими эти запросы, может возникнуть конкуренция за ресурс - в данном случае, скажем, рейс. Один из процессов захватывает ресурс первым, резервирует место и освобождает ресурс. Потом второй процесс проделывает то же самое.

Порядок, в котором в этом примере процессы захватили ресурс, зависит от многих факторов и, в конечном счете, случаен. Однако  это не мешает правильному функционированию системы, поскольку клиентам важно одно - получить билеты, причем на разные места. Однако отсутствие детерминизма в поведении приложения приводит к тому, что при повторном выполнении могут быть получены другие результаты: например, клиенту уже сообщено, что ему зарезервировано место №5, а при повторе может получиться, что зарезервировано место №6. Система должна устранить это несоответствие и сделать его невидимым для клиента.
\begin{figure*} %fig1
\vspace*{1pt}
\begin{center}
\mbox{%
\epsfxsize=1.6in
\epsfxsize=100mm
\epsfbox{BbR-1.eps}
}
\end{center}
\vspace*{-9pt}
\Caption{Базовый вариант трубы с разными выходными устройствами
(цилиндрическое, расширяющееся и сужающееся сопло)
\label{f1bab}}
\vspace*{-3pt}
\end{figure*}

Недетерминированность поведения системы это следствие, по крайней мере, двух обстоятельств. Во-первых, это присущая системам с разделением времени неопределенность в порядке выполнения процессов. Во-вторых, это конкуренция процессов за общие ресурсы. Перечислим некоторые причины недетерминированного поведения приложений:
\begin{itemize}
\item синхронизация процессов с помощью семафоров или атомарных операций над операндами в общей памяти процессов;
\item зависимость от порядка получения клиентских запросов;
\item время, затраченное процессом на обработку полученного запроса;
\item генераторы случайных чисел;
\item системное управление процессами и потоками;
\item локальные таймеры;
\item доступ к реальному времени.
\end{itemize}

По различным  причинам время, которое тратится на выполнение вычислительной задачи с одними и теми же исходными данными, не является константой, то есть повторное выполнение может дать другое время. Процессы используют общие ресурсы, обращение к которым требует организации очередности выполнения (сериализации) - первым пришел, первым захватил. И, наконец,  результат работы процесса может зависеть от состояния ресурса, а это состояние может изменить другой процесс, ранее захвативший ресурс. Все это создает значительные трудности при попытках воспроизведения поведения процессов с сохраненной контрольной точки.

Прозрачная отказоустойчивость серверов приложений обычно осуществляется переносом приложения на другой узел кластера, идентичный первому по конфигурации аппаратных средств и операционной среды. Это делается методом, называемым snapshot/restore. На основном узле (оригинале)  периодически фиксируется состояние приложения на этом узле кластера (так называемый снимок или snapshot). После отказа оригинала на резервном узле (копии) делается восстановление (restore), то есть восстанавливается последнее зафиксированное состояние приложения. Операционная среда при этом приводится в состояние, которое соответствует моменту изготовления снимка. После этого узел-копия продолжает работу с зафиксированного места. Сравнение метода  snapshot/restore с другими подходами приведено в [7].

Ниже рассматриваются информационные  технологии, позволяющие решить ряд принципиальных вопросов, связанных с реализацией прозрачной отказоустойчивости серверов приложений. Ими являются:
\begin{itemize}
\item виртуализация операционной среды, в которой работает серверное приложение;
\item отказоустойчивая реализация протокола TCP;
\item создание контрольных точек состояния приложения и файловой системы, которые делаются внешним по отношению к приложению образом;
\item восстановление серверного приложения на основании контрольной точки.
\end{itemize}
\begin{figure*} %fig1
\vspace*{1pt}
\begin{center}
\mbox{%
\epsfxsize=1.6in
\epsfxsize=100mm
\epsfbox{BbR-1.eps}
}
\end{center}
\vspace*{-9pt}
\Caption{Базовый вариант трубы с разными выходными устройствами
(цилиндрическое, расширяющееся и сужающееся сопло)
\label{f1bab}}
\vspace*{-3pt}
\end{figure*}

\section{МОДЕЛЬ ОПИСАНИЯ ПОВЕДЕНИЯ ПРИЛОЖЕНИЯ}

Предлагаемый подход опирается на построение модели вычислений, связанной с использованием понятия времени в многопроцессных приложениях. Впервые подобные проблемы были изучены в классической работе Л. Лампорта [8].

Многопроцессными приложения называются потому, что в них параллельно работают несколько процессов. Процесс ведет себя детерминированно, пока в предписанном кодом порядке выполняет процессорные инструкции. Конечно, его работа может быть прервана практически в любой момент и процессор передан другому процессу или ядру. Поэтому абсолютное время, которое затрачивает процесс на выполнение определенной работы, не  константа, а случайная  величина. То же  относится к относительному времени, то есть времени, которое процесс занимал процессор,  поскольку одни и те же обращения к операционной среде могут вызвать работы разной длительности, а значит потребовать разное время на свое выполнение.

Кэшированность инструкций и данных, а также длина хэш-списков влияют на действительное время пребывания в операционной среде. Утрачивает смысл понятие одновременность действий, поскольку  нельзя установить, выполнили ли два разных процесса какие-либо действия одновременно или одно из них предшествовало другому. Таким образом, с процессом можно связать только его локальное время, которое линейно упорядочивает события,  происходившие в этом процессе.  Глобальное время, линейно упорядочивающее действия во всех процессах, отсутствует. Расстояние (в этом качестве используется время) между действиями оказывается случайной величиной.

Эти соображения важны, поскольку процессы в интересующих нас приложениях взаимодействуют и используют общие ресурсы. Для взаимодействия они используют средства синхронизации, предоставляемые операционной средой - например, наборы семафоров SVR4 (System V Release 4), POSIX-семафоры, бинарные семафоры и другие примитивы взаимного исключения (POSIX- mutual exclusion locks) и т.д. Подобные средства операционной среды, которые позволяют процессам синхронизировать свою деятельность друг с другом или сериализовать обращения к совместно используемым объектам,  будут ниже  называться ресурсами.

С каждым ресурсом связано свое локальное время, линейно упорядочивающее события в жизни ресурса. Например, в случае двоичных семафоров это создание семафора, а также его захват и освобождение процессом. Заметим, что событие - это не намерение процесса (например, захватить бинарный семафор), а сам факт захвата семафора процессом (т.е. успешное выполнение намерения). От изъявления намерения до его осуществления может многое произойти. Например, семафор, который хочет захватить рассматриваемый процесс, принадлежал другому процессу, потом тот процесс его освободил, но семафор был сначала передан операционной средой третьему процессу, который также на него претендовал, и т.д. Поведение рассматриваемого процесса в это время нас не интересует - он ресурсом еще не овладел, а только его захват определяет его дальнейшее поведение. По причинам,  изложенным выше, расстояние между двумя событиями - случайная величина. Однако, события замечательны тем, что они одновременно присутствуют и в локальном времени процесса, и в локальном времени ресурса. Поэтому все, что произошло в истории процесса или/и ресурса до этого события, предшествует ему. Далее  будет считаться, что истории и ресурсов и процессов состоят только из событий, причем между двумя последовательными событиями в жизни процесса последний ведет себя детерминированно.

Это означает, что на  поведении процесса сказывается только его предыдущая история, то есть состояние ресурсов, с которыми он взаимодействовал. Это свойство процессов ниже будет называться локальной детерминированностью. Этим свойством не обладают ресурсы, поскольку - следующее событие в истории ресурса не определяется однозначно по его предыдущей истории. Утверждение, касающееся детерминированного поведения процессов, неявно опирается на предположение,  что учтены все ресурсы, которые могут привести к  недетерминированности процессов.

Таким образом, описанное нами очень неформально время в многопроцессном комплексе представляет собой отношение частичного порядка, введенное на множестве событий. Зная полное состояние комплекса в некоторый момент времени,  нельзя однозначно определить, какое событие в истории ресурса наступит следующим. Можно говорить только о вероятности наступления того или иного события. Недетерминированность поведения есть следствие двух обстоятельств. Во-первых, это неопределенность времени, которое тратит процесс на переход от одного события к другому. Во-вторых, конкуренция процессов за общие ресурсы.

Выполнение приложения, на множестве событий которого введена частичная упорядоченность, можно описать направленным ациклическим графом выполнения. Вершинами этого графа являются события, с каждым  из которых связаны две входящие в него дуги. Одна дуга начинается в событии, которое непосредственно предшествует данному событию в истории процесса, другая - в истории ресурса.

Построение средств обеспечения прозрачной отказоустойчивости приложений опирается на следующее утверждение: для восстановления работы приложения после отказа достаточно располагать:
\begin{itemize}
\item контрольной точкой, которая отражает на некоторый момент времени состояния процессов и других ресурсов, образующих приложение;
\item графом выполнения приложения, который описывает работу приложения, начинающуюся с контрольной точки и заканчивающуюся отказом. Данные, которые нужны для построения графа выполнения, далее называются протоколом.
\end{itemize}
\begin{figure*} %fig1
\vspace*{1pt}
\begin{center}
\mbox{%
\epsfxsize=1.6in
\epsfxsize=100mm
\epsfbox{BbR-1.eps}
}
\end{center}
\vspace*{-9pt}
\Caption{Базовый вариант трубы с разными выходными устройствами
(цилиндрическое, расширяющееся и сужающееся сопло)
\label{f1bab}}
\vspace*{-3pt}
\end{figure*}

Вся эта информация должна находиться в стабильной памяти, не разрушающейся при отказе.

Ниже неформально описан алгоритм восстановления работы приложения после отказа, который опирается на наличие контрольной точки и графа выполнения. Будем считать, что в распоряжении имеются средства, позволяющие остановить процесс в тот момент, когда он намерен совершить некоторую операцию над ресурсом. Заметим, что событие в графе выполнения соответствует не изъявлению намерения, а его удовлетворению, то есть завершению выполнения операции.

Предварительно сделаем следующее:
\begin{itemize}
\item используя контрольную точку, приведем приложение в состояние, соответствующее этой контрольной точке;
\item в графе выполнения пометим все вершины (события) как "не наступившие". У некоторых вершин графа отсутствуют им непосредственно предшествующие; соответствующие события наступили сразу же после создания контрольной точки. Для каждой такой вершины включим в граф дополнительную вершину, ей предшествующую в истории процесса, и отметим эту дополнительную вершину как "наступившую";
\item разрешим процессам приложения выполняться.
\end{itemize}

Пусть некоторый процесс проявляет намерение выполнить операцию над каким-либо ресурсом. Отыщем для этого процесса в его истории последнее наступившее событие. Следующее в его истории событие - это то, которое соответствует требуемой операции. Посмотрим, наступило ли событие в истории ресурса, которое ему предшествует. Если нет, переведем процесс в состояния ожидания, отметив в предшествующем событии, что данный процесс ожидает его наступления. Если да, разрешим процессу выполняться, то есть выполнить операцию над ресурсом.

Пусть некоторый процесс объявляет, что он выполнил операцию над каким-либо ресурсом (это соответствует моменту протоколирования при оригинальном выполнении). Отыщем для этого процесса в его истории последнее наступившее событие и перейдем к следующему событию в его истории. Это опять то событие, которое мы рассматриваем. Отметим его как "наступившее". Если наступления этого события ожидал какой-нибудь процесс, выведем этот процесс из состояния ожидания. Наконец, разрешим процессу, выполнившему операцию, продолжаться дальше.

Когда выясняется, что наступили все события графа выполнения, повторное выполнение считается законченным.

Естественным следствием из сказанного является следующее утверждение: для того, чтобы размер протокола не рос неограниченно, нужно периодически создавать контрольные точки, очищая при этом протокол.

\section{ФОРМАЛЬНОЕ ОПИСАНИЕ МОДЕЛИ ПОВЕДЕНИЯ МНОГОПРОЦЕССНОГО ПРИЛОЖЕНИЯ}
\begin{figure*} %fig1
\vspace*{1pt}
\begin{center}
\mbox{%
\epsfxsize=1.6in
\epsfxsize=100mm
\epsfbox{BbR-1.eps}
}
\end{center}
\vspace*{-9pt}
\Caption{Базовый вариант трубы с разными выходными устройствами
(цилиндрическое, расширяющееся и сужающееся сопло)
\label{f1bab}}
\vspace*{-3pt}
\end{figure*}

Опишем формально поведение приложения, неформальное описание которого было приведено выше. Рассматриваются два типа объектов:
\begin{itemize}
\item ресурсы (r), например, наборы семафоров (POSIX- или SVR4-семафоры), бинарные семафоры (POSIX-mutex's), таймер реального времени, сокеты (sockets), то есть двусторонние виртуальные соединения с внешним миром;
\item процессы (p), например, процессы или потоки (threads) пользователя.
\end{itemize}

\end{multicols}

\label{end\stat}

%\def\stat{batr}

\def\tit{НОВЫЙ МЕТОД ВЕРОЯТНОСТНО-СТАТИСТИЧЕСКОГО\newline
АНАЛИЗА ИНФОРМАЦИОННЫХ ПОТОКОВ
В~ТЕЛЕКОММУНИКАЦИОННЫХ СЕТЯХ$^*$}
\def\titkol{Новый метод вероятностно-статистического
анализа информационных потоков
в~телекоммуникационных сетях}
\def\autkol{Д.\,А.~Батракова, В.\,Ю.~Королев, С.\,Я.~Шоргин}
\def\aut{Д.\,А.~Батракова$^1$, В.\,Ю.~Королев$^2$, С.\,Я.~Шоргин$^3$}

\titel{\tit}{\aut}{\autkol}{\titkol}

{\renewcommand{\thefootnote}{\fnsymbol{footnote}}\footnotetext[1]{Работа 
выполнена при поддержке РФФИ, проекты №№\,04-01-00671, 05-07-90103.} 
\renewcommand{\thefootnote}{\arabic{footnote}}}
 \footnotetext[1]{ИПИ РАН, 
daria.batrakova@gmail.com} \footnotetext[2]{Факультет вычислительной математики 
и кибернетики МГУ им.~М.\,В.~Ломоносова, ИПИ РАН, vkorolev@comtv.ru} 
\footnotetext[3]{ИПИ РАН, sshorgin@ipiran.ru}



\Abst{В данной работе предлагается метод исследования стохастической структуры
хаотических информационных потоков в сложных телекоммуникационных
сетях. Предлагаемый метод основан на стохастической модели
телекоммуникационной сети, в рамках которой она представляется в виде
суперпозиции некоторых простых последовательно-параллельных структур.
Эта модель естественно порождает смеси гамма-распределений для времени
выполнения (обработки) запроса сетью. Параметры получаемой смеси
гамма-распределений характеризуют стохастическую структуру
информационных потоков в сети. Для решения задачи статистического
оценивания параметров смесей экспоненциальных и гамма-распределений
(задачи разделения смесей) используется ЕМ-алгоритм. Чтобы проследить
изменение стохастической структуры информационных потоков во времени,
ЕМ-алгоритм применяется в режиме скользящего окна. Описывается
программный инструментарий для применения полученного решения к
реальным статистическим данным. Приводится интерпретация результатов.}

\KW{телекоммуникационные сети; информационные потоки;
разделение смесей  распределений;
метод скользящего окна;  программа для разделения смесей}

\vskip 24pt plus 9pt minus 6pt

\thispagestyle{headings}

\begin{multicols}{2}


\label{st\stat}

\section{Введение}

Развитие телекоммуникационных сетей, их усложнение поставило перед
инженерами важную прикладную задачу исследования характеристик
информационных потоков, возникающих в этих сетях. Здесь под
информационным потоком мы будем понимать упорядоченное движение
любого вида информации по сети.

Если на заре эры телекоммуникаций, в эпоху первых телефонных линий и
телеграфа эта проблема не была столь насущной, то со временем, при
постепенном охвате мирового пространства сетями возникла необходимость в
построении и исследовании адекватных моделей сетей и процессов,
происходящих в них.

\thispagestyle{headings}


Современные сети~--- это \textit{конвергентные} сети, т.е.\ совокупность крайне
разнородных как по топологии, так и по физической архитектуре сетей, которые
предлагают конечному пользователю самые разнообразные сервисы. Это~--- огромное
виртуальное и физическое пространство, состоящее из миллионов процессоров,
операционных платформ, линий передачи данных и стыковочных узлов.
%
Существует множество классификаций телекоммуникационных сетей по различным
признакам:
\begin{itemize}
\item масштабу (локальные сети~--- LAN, масштаба города~---
MAN, широкого масштаба~--- WAN);
\item топологии, или логической организации (<<звезда>>,
<<кольцо>>, <<шина>>);
\item физической организации (оптические сети, радио);
\item предлагаемым услугам (сотовые сети, для доступа в
Интернет);
\item назначению (военные, гражданские) и~др.
\end{itemize}


Конвергентная сеть входит во все эти классы, причем, как правило,
обладает всеми этими признаками. Поэтому построение модели для ее анализа
является и очень важной, и очень сложной задачей.

Существуют достаточно многочисленные математические методы, ориентированные на
моделирование и анализ телекоммуникационных сетей. В~большинстве своем они
основываются на теории массового обслуживания, то есть разделе теории
вероятностей, посвященном описанию функционирования сложных систем обслуживания
(в том чис\-ле телекоммуникационных сетей и систем) с помощью стохастических
процессов особого вида и анализу таких процессов. Указанная теория является
весьма развитой и широко применяется на практике. Тем не менее, ее применимость
ограничена~--- во-первых, все возрастающей сложностью структур и дисциплин
обслуживания в рас\-смат\-ри\-ва\-емых реальных сетях. Эта сложность во многих
случаях принципиально не может найти адекватного отображения в моделях
массового обслуживания, даже несмотря на постоянно растущую сложность самих
этих моделей. В результате даже модели, допускающие точный математический
анализ, дают возможность расчета всего лишь приближенных значений характеристик
реальных сетей, ибо предположения, принимаемые при построении моделей, во
многих случаях не соответствуют практике. Во-вторых, для описания
телекоммуникационной сети в виде сети массового обслуживания исследователь
должен располагать детальным описанием структуры сети, что далеко не всегда
имеет мес\-то на практике. В-третьих, разработано крайне мало моделей массового
обслуживания, в которых используется в качестве входной информация о
наблюдаемых (статистических) показателях функционирования сети; в то же время,
такая информация очень часто доступна исследователю, и ее использование при
анализе сети весьма целесообразно.

В данной работе предлагается в определенной степени альтернативный подход к
моделированию сложных телекоммуникационных сетей. Строится и исследуется
вероятностная модель сложной телекоммуникационной сети как суперпозиции
достаточно простых структур. При этом практически никакая априорная информация
о структуре исследуемой сети не используется~--- наоборот, в результате
исследования модели исследователь получает приближенное представление об этой
структуре. Характеристики типовых простых структур, составляющих в совокупности
модель сети, и сети в целом при этом подходе описываются
гам\-ма-рас\-пре\-де\-ле\-ни\-я\-ми. Ставится задача оценки параметров модели
на основе статистических данных о функционировании сети, а также предлагается
математическое решение этой задачи. В статье описан также созданный на основе
разработанных математических методов программный инструментарий и приведены
результаты расчетов для реального трафика. {\looseness=-1

}

\section{Математическая модель и~постановка задачи}

\subsection{Логическая модель сети}
 %1.1

Рассмотрим абстрактную \textit{конвергентную телекоммуникационную
сеть}. Это может быть как крупномасштабная транспортная сеть (WAN), сеть
отдельного оператора масштаба города (MAN) с различными сервисами, так и
локальная сеть (LAN).

Любой из этих случаев можно описать как ($p,\,q$)-\textit{сеть}.

\medskip
\textbf{Определение 1.} В теории графов и сетей под ($p,\,q)$-сетью понимается
набор вида $S =$\linebreak $=(G,\,V^\prime ,\,V^{\prime\prime})$, где $G$~---
граф, а $V^\prime$ и $V^{\prime\prime}$~--- выборки из множества $V(G)$ (вершин
графа) длины~$p$ и $q$ соответственно. При этом выборка $V^\prime$
($V^{\prime\prime}$) считается \textit{входной} (\textit{выходной}) выборкой, а
ее $i$-я вершина называется $i$-\textit{м} \textit{входным} (\textit{выходным})
\textit{полюсом} или, иначе, $i$-\textit{м} \textit{входом} (\textit{выходом})
сети~$S$. Вершины, не участвующие во входной и выходной выборках сети,
считаются ее внутренними вершинами~\cite{1bat}.

Сеть $S$ (рис.~\ref{f1bat}) имеет $p$ точек входа~--- точек соединения
с внешней средой (это могут быть точки стыковки разнородных сетей, сетей
различных операторов, физические подключения к интерфейсам
маршрутизаторов и~т.п.). Под \textit{внешней средой} будем понимать другие
сети, которые передают данные в сеть~$S$. Отдельные <<единицы>> данных
(кадры, сообщения, датаграммы, пакеты) поступают на входы сети~$S$,
обрабатываются и подаются на каждый из $q$ выходов, которые могут быть
соединены как с конечными пользователями, так и с другими сетями.
\begin{figure*} %fig1
\vspace*{1pt}
\begin{center}
\mbox{%
\epsfxsize=139.7mm \epsfbox{bat-1.eps}
%\epsfxsize=139.698mm
%\epsfbox{bek-3.eps}
}
\end{center}
\vspace*{-9pt} \Caption{Абстрактная телекоммуникационная сеть \label{f1bat}}
\end{figure*}

Следует отметить, что структура сложных телекоммуникационных сетей обладает
свойством некоторого самоподобия, т.е.\ на каком бы уровне сетевой архитектуры
мы ни рассматривали поведение информационных потоков, мы можем выделить
некоторые базовые структуры, субпотоки, суперпозицией которых мы можем получить
модель конкретной сети, какой бы уровень <<детализации>> сегментов сети мы ни
взяли. Так, например, физические подключения к интерфейсам оптического
кросс-коннекта в этом смысле подобны <<виртуальным>> подключениям к портам TCP
на сервере приложений.

Итак, независимо от уровня сетевой архитектуры мы можем
рассматривать некоторую величину, характеризующую количество каких-либо
ресурсов сети~$S$, занимаемых в процессе передачи и обработки данных.  Это
могут быть ресурсы, относящиеся как к <<объему>> (памяти сетевого
устройства, количеству занятых линий, размеру пакета), так и ко <<времени>>
(времени обслуживания заявки, времени простоя). Эта величина случайна, т.к.\
мы не можем абсолютно точно сказать для сложной телекоммуникационной
сети, какое сообщение на какой из входов поступит и какого размера оно будет.
Таким образом, случайный характер данной величины определяется
случайностью поведения внешней среды.

Пусть $R$~--- это описанная выше случайная величина, $R>0$. Далее, не
ограничивая общности, будем подразумевать под ней время, необходимое для
какой-либо операции сети (обработки <<единицы>> данных), предполагая, что
время обработки прямо зависит от объема сообщения.

\subsection{Вероятностная модель сети} %1.2.

Даже не зная реальной топологии сети, мы можем описать
функционирование некоторых ее участков как процесс выполнения операций
(задач сети) в последовательном  порядке (например, если доступен только
один канал данных) или как процесс одновременного выполнения субопераций
(когда доступно более одного пути выполнения). Это значит, что мы можем
представить функционирование сложной телекоммуникационной сети как
\textit{суперпозицию} таких <<последовательных>> и <<параллельных>>
блоков.

Для построения вероятностной модели распределения~$R$ используется
комбинация асимптотического подхода, основанного на предельных теоремах
теории вероятностей, и принципа максимальной неопределенности (энтропии).

Рассмотрим следующую модель. Предположим, что мы можем разделить
сеть~$S$ на несколько сегментов $S_i$. Пусть $T$~--- случайная величина,
время выполнения операции в отдельно взятом блоке $S_i$ (сегменте сети).

Если операции выполняются \textit{параллельно}, то время, необходимое
для их выполнения~--- это максимальное время, затрачиваемое на какую-либо
субоперацию:
$$
T = \underset{i}{\max}\, T_i\,,
$$
где $T_i$~--- случайные величины для со\-от\-вет\-ст\-ву\-ющих субопераций.
Модель такого выполнения пред\-став\-ле\-на на рис.~\ref{f2bat}.

\begin{figure*} %fig2
\vspace*{1pt}
\begin{center}
\mbox{%
\epsfxsize=117.271mm
\epsfbox{bat-2.eps}
}
\end{center}
\vspace*{-9pt}
\Caption{Параллельное выполнение
\label{f2bat}}
\end{figure*}

Известно, что предельное распределение экстремальных значений для
выборок ~--- это экспоненциальное распределение с плотностью~\cite{2bat}
$$
f(x) =
\begin{cases}
\lambda e^{-\lambda x}\,, & x>0\,,\\
0\,, & x\leq 0\,,
\end{cases}
$$
где $\lambda >0$~--- параметр масштаба.

 Учитывая также энтропийный подход, естественно будет считать
распределение $T$ экспоненциальным, т.к.\ экспоненциальное распределение
обладает наибольшей энтропией среди всех распределений с $x>0$.

Если же операции сети выполняются \textit{последовательно}, то величина
$T$~--- это сумма времен $T_i$, необходимых для выполнения каждой
субоперации:
$$
T = \sum\limits_i T_i\,,
$$
где $T_i$~--- случайные величины для со\-от\-вет\-ст\-ву\-ющих субопераций.
%
Такая модель представлена на рис.~\ref{f3bat}.

\begin{figure*} %fig3
\vspace*{1pt}
\begin{center}
\mbox{%
\epsfxsize=139.592mm
\epsfbox{bat-3.eps}
}
\end{center}
\vspace*{-9pt}
\Caption{Последовательное  выполнение
\label{f3bat}}
\end{figure*}

Это значит, что распределение общей длительности $T$ выполнения
блока~--- это свертка распределений <<элементарных>> времен $T_i$
(экспоненциально распределенных).

Известно, что результатом свертки экспоненциальных распределений
является гамма-распределение, определяемое плотностью
$$
\g_{\lambda , \alpha} (x) =
\begin{cases}
\fr{\lambda_0^{\alpha_0}}{\Gamma (\alpha_0 )}\,x^{\alpha_0-1}
e^{\lambda_0 x}\,, & x>0\,,\\
0,\, & x\leq 0\,,
\end{cases}
$$
где $\alpha >0$~--- параметр формы,  $\lambda >0$  параметр масштаба, а
$\Gamma (z)$~--- гамма-функция Эйлера:
$$
\Gamma (z) = \int\limits_0^\infty x^{z-1} e^{-x}\,dx\,.
$$

\begin{figure*} %fig4
\vspace*{1pt}
\begin{center}
\mbox{%
\epsfxsize=120.831mm
\epsfbox{bat-4.eps}
}
\end{center}
\vspace*{-9pt}
\Caption{Модель пути  обработки сообщения сетью~$S$
\label{f4bat}}
\end{figure*}

Известно~\cite{2bat}, что класс гамма-распределений замкнут над операцией
свертки, поэтому ре\-зуль\-ти\-ру\-ющее распределение случайной величины~$R$
будет также гамма-распределением
$$
\g_{\lambda , \alpha} (x) =
\begin{cases}
\fr{\lambda^{\alpha}}{\Gamma (\alpha )}\,x^{\alpha -1} e^{-\lambda x}\,, &
x>0\,,\\
0,\, & x\leq 0\,.
\end{cases}
$$

В силу случайного характера ввода данных в сеть~$S$ из внешней среды маршрут
передачи данных становится случайным, что представлено на рис.~\ref{f4bat}. Это
означает, что параметры ре\-зуль\-ти\-ру\-юще\-го распределения~$R$ тоже
случайны. Отсюда имеем следующую модель \textit{смеси
гам\-ма-рас\-пре\-де\-ле\-ний}, определяемой плотностью

\begin{equation} %1
p(x) = \iint \g_{\lambda , \alpha}(x)\,dH (\lambda ,\,\alpha )\,,
\end{equation}
где $H(\lambda , \alpha )$~--- смешивающая функция, функция распределения
входных параметров.

Поясним понятие \textit{смеси распределений}.

\medskip
\textbf{Определение~2.} Пусть имеется двух\-па\-ра\-мет\-ри\-че\-ское
семейство $n$-мерных плотностей  распределения
\begin{equation}
F = \{ f_\omega (x;\, \theta (\omega ))\}\,,
\end{equation}
где одномерный (целочисленный или непрерывный) параметр $\omega$ в
качестве нижнего индекса функции $f$ определяет специфику общего вида
каж\-до\-го компонента~--- распределения смеси, а в качестве аргумента при
многомерном, вообще говоря, параметре $\theta$ определяет зависимость
значений хотя бы части компонентов этого параметра от того, в каком именно
составляющем распределении $f_\omega$ он присутствует. Кроме того, пусть
$P = \{P(\omega )\}$~--- \textit{семейство смешивающих функций}
распределения.

Функция плотности распределения
\begin{equation}
f(x) = \int f_\omega (x;\,\theta(\omega ))\,dP (\omega )
\end{equation}
называется $P$-\textit{смесью} (или просто \textit{смесью})
\textit{распределений} семейства~$F$,  интеграл в~(3) понимается в смысле
Лебега--Стильтьеса~\cite{3bat}.

\medskip
\textbf{Определение 3.} Семейство смесей~(3) называется
\textit{идентифицируемым} (\textit{различимым}), если из равенства
$$
\int f_\omega (x;\,\theta(\omega ))\,dP (\omega ) =\int f_\omega
(x,\,\theta(\omega )) dP^*(\omega )
$$
следует, что $P(\omega ) \equiv P^*(\omega )$ для всех $P \in P(\omega
)$~\cite{3bat}.

\subsection{Постановка задачи} %1.3.

Перед нами встает задача \textit{разделения} такой смеси. Вообще говоря,
задача разделения смесей распределений со смешивающими функциями
общего вида является \textit{некорректно поставленной}, т.к.\ она допускает
существование нескольких решений. Поэтому будем искать решение в классе
\textit{конечных идентифицируемых смесей распределений}, где смешивающая
функция дискретна.

Для этого сузим данное выше определение и будем рассматривать в дальнейшем лишь 
случай конечного числа $k$ возможных значений па\-ра\-мет\-ра~$\omega$, что 
соответствует конечному числу скачков смешивающих функций $P(\omega )$.  
Величины этих скачков как раз и будут играть роль \textit{удельных весов} 
(\textit{априорных вероятностей}) $p_j$ компонентов смеси. Более того, в нашем 
случае мы постулируем также однотипность компонентов распределений $f_j$, т.е.\ 
принадлежность всех $f_j$ к одному общему па\-ра\-мет\-ри\-че\-ско\-му 
семейству $\{ f(X;\,\theta )\}$, где $\theta$~--- многомерный, вообще говоря, 
параметр. Так что~(3) в этом случае может быть записано в виде
\begin{equation} %4
p(x) = \sum\limits^k_{j=1} p_j f_j (x;\,\theta_j )\,.
\end{equation}

Переформулируем понятие идентифицируемости (различимости) смесей
специально применительно к такому виду смесей.

\medskip
\textbf{Определение 4.} \textit{Конечная смесь}~(3) называется
\textit{идентифицируемой} (\textit{различимой}), если из равенства
$$
\sum\limits_{j=1}^k p_j f_j (x;\,\theta_j ) = \sum\limits_{l=1}^{k^*} p_l^* f_l
(x;\,\theta_l^* )
$$
следует, что $k=k^*$ и для любого $j$ ($1\leq j \leq k$) найдется такое $l$ 
($1\leq l \leq k^*$), что $p_j = p_l^*$ и $f_j (x;\,\theta_j ) = f_l 
(x;\,\theta_l^* )$~\cite{3bat}.

Решить эту задачу в выборочном варианте~--- значит по выборке
классифицируемых наблюдений
$X_1,\ldots , X_n, $ извлеченной из генеральной совокупности, яв\-ля\-ющей\-ся смесью~(3)
генеральных совокупностей типа~(2) (при заданном общем виде составляющих
смесь функций $f_j (x;\,\theta_j )$), построить статистические оценки для числа
компонентов смеси~$k$, их удельных весов $p_j$ и, главное, для каждого из
компонентов %f_j (x;\,\theta_j )$ анализируемой смеси. Далее будет считать, что
функции $f_j$ однозначно определяются своими параметрами $\theta_j$: $f_j
=f(x;\,\theta_j)$.

Однако не следует ставить знак тождества между задачей разделения смеси
и задачей статистического оценивания параметров в модели~(4) по выборке $
X_1,\ldots , X_n$, поскольку задача разделения сохраняет смысл и
применительно к генеральным совокупностям, т.е.\ в теоретическом
варианте~\cite{3bat}.

Итак, для статистического анализа на основе реальных данных мы
аппроксимируем нашу общую модель~(1) следующей:
$$
p(x) \approx \hat{p}(x) = \sum\limits_{j=1}^k p_j \g_{\lambda_j , \alpha_j}
(x)\,,
$$
где $p_j$~--- дискретные смешивающие параметры, $\g_{\lambda_j , \alpha_j}
(x)$~--- плотности гамма-распределений.

Такая аппроксимация не только позволяет решить поставленную статистическую
задачу, но и полу\-чить наглядную визуализацию результатов. Существуют
достаточно эффективные методики разделения смесей распределений, среди них~---
семейство так называемых \textit{ЕМ-алгоритмов}
(\textit{Expectation-Maximization Algorithms}).

Полученные результаты могут быть достаточно просто интерпретированы. Число
компонентов смеси символизирует число типичных параллельных или
последовательных структур. Значения параметров составляющих смесь
гам\-ма-рас\-пре\-де\-ле\-ний показывают <<степень параллелизма>>
соответствующей структуры: чем ближе параметр формы к~1, тем выше эта
<<степень>>. И наоборот, чем дальше значение параметра формы от~1, тем больше
последовательных операций выполняется в соответствующем блоке.

Веса компонентов характеризуют примерную долю использования
ресурсов для сообщений, соответствующих каждому распределению входных
данных.

Итак, предложенный подход позволяет получить представление о
стохастической структуре телекоммуникационной сети.

\section{ЕМ-алгоритм разделения смесей распределений}

\subsection{Описание алгоритма} %2.1.

Определяемый ниже итерационный алгоритм решения поставленной в
предыдущем разделе задачи относится к процедурам, базирующимся на
\textit{методе максимального правдоподобия}.

Этот алгоритм позволяет находить максимум логарифмической функции
правдоподобия по параметрам $p_1,\,p_2,\ldots ,\,p_k$, $\theta_1 ,\,\theta_2,\ldots ,\,
\theta_k$ при фиксированном $k$ по выборке $X_1, \ldots , X_n$, т.е.\ решение
оптимизационной задачи вида

\begin{equation} \sum\limits_{i=1}^n \ln \left ( \sum\limits_{j=1}^k p_j f_j
(X_i;\,\theta_j )\right ) \rightarrow \underset{p_j,\,\theta_j}{\max}\,.
\end{equation}

Конкретные алгоритмы, построенные по этой схеме, часто называют
\textit{алгоритмами типа ЕМ}, поскольку в каждом из них можно выделить два
этапа, находящихся по отношению друг к другу в последовательности
итерационного взаимодействия: \textit{оценивание} (\textit{Estimation}) и
\textit{максимизация} (\textit{Maximization})~\cite{4bat}.

Введем в рассмотрение так называемые апостериорные вероятности
$\g_{ij}$ принадлежности наблюдения $X_i$ к $j$-му классу:
\begin{equation} %6
\g_{ij} = \fr{p_j f(X_i;\,\theta_j )}{\sum\limits_{l=1}^k p_l f(X_i;\,\theta_l 
)} \ (i=1,\ldots , n;\ j=1,\ldots ,k)\,.\!\!\end{equation} 
Очевидно, что для 
всех $i=1,\ldots ,n$ и $j=1,\ldots ,k$
$$
\g_{ij} \geq 0,\quad \sum_{j=1}^k \g_{ij} =1\,.
$$


Далее обозначим $\Theta = (p_1,\ldots p_k,\,\theta_1,\ldots ,\theta_k )$ и
представим анализируемую логарифмическую функцию правдоподобия
$$
\ln L(\Theta ) = \sum\limits_{i=1}^n \ln \left (\sum\limits_{j=1}^k p_j f_j
(X_i;\,\theta_j )\right )
$$
в виде
\begin{multline}
\ln L (\Theta ) = \sum\limits_{j=1}^k\sum\limits_{i=1}^n \g_{ij} \ln p_j+{}\\
{}+\sum\limits_{j=1}^k\sum\limits_{i=1}^n \g_{ij} f(X_i;\,\theta_j)-
\sum\limits_{j=1}^k\sum\limits_{i=1}^n \g_{ij} \ln \g_{ij}\,.
\end{multline}

Справедливость этого тождества легко проверяется с учетом
$$
\sum\limits_{j=1}^k \g_{ij} =1\,.
$$

Далее идея построения итерационного алгоритма вычисления оценок
$\hat{\Theta} = (\hat{p}_1,\ldots , \hat{p}_k,\
\hat{\theta}_1,\ldots , \hat{\theta}_k)$
для параметров $\Theta = (p_1,\ldots , p_k,\ \theta_1,\ldots , \theta_k)$ состоит в
следующем:
\begin{enumerate}[1.]
\item Выбирается некоторое \textit{начальное приближение}~$\hat{\Theta}^0$.
\item \textbf{E-step:} вычисляются по формулам~(6) начальные приближения
$\g_{ij}^0$ для апостериорных вероятностей $\g_{ij}$~--- \textit{этап
оценивания}.
\item \textbf{M-step:} затем, возвращаясь к~(7), при вычисленных значениях
$\g^0_{ij}$ следует определить значения $\hat{\Theta}^1$ из условия
максимизации отдельно каждого из первых двух слагаемых правой
части~(7), поскольку первое слагаемое
$$
\sum_{j=1}^k \sum_{i=1}^n \g_{ij} \ln p_j
$$
зависит только от параметров $p_j$, а второе слагаемое
$$
\sum_{j=1}^k \sum_{i=1}^n \g_{ij} f(X_i;\,\theta_j )
$$
зависит только от параметров $\theta_j$~--- \textit{этап максимизации}.
\item Проверяется \textit{условие останова}:
$$
\parallel \Theta^{(t)} - \Theta^{t-1}\parallel <\varepsilon\,,
$$
где $t$~--- номер итерации, а
$\parallel\bullet\parallel$~--- евклидова норма, для некоторого $\varepsilon
>0$.
\end{enumerate}

Очевидно, решение оптимизационной задачи
$$
\sum\limits_{j=1}^k\sum\limits_{i=1}^n \g_{ij}^{(t)}\ln p_j \rightarrow
\underset{p_j}{\max}
$$
дается выражением (с учетом $\sum_{j=1}^k p_j =1$):
$$
p_{ij}^{(t+1)} =\fr{1}{n}\,\sum\limits_{i=1}^n \g_{ij}^{(t)}\,,
$$
где $t$~--- номер итерации, $t = 0$, 1, 2,\,\ldots

Решение оптимизационной задачи
$$
\sum\limits_{j=1}^k \sum\limits_{i=1}^n \g_{ij}^{(t)} f(X_i;\,\theta_j )
\rightarrow \underset{\theta_j}{\max}
$$
получить намного проще решения задачи~(5): выражение для $\theta_j$
записывается с учетом знания конкретного вида функций
$f(X,\,\theta)$~\cite{3bat}.

\subsection{О сходимости алгоритма} %2.2.

В работе М.\,И.~Шлезингера~\cite{5bat}, где эта схема (позднее названная
ЕМ-схемой) впервые предложена, установлены и основные свойства
реа\-ли\-зу\-ющих ее алгоритмов. В частности, было доказано, что при достаточно
широких предположениях \textit{предельные точки} всякой последовательности,
порожденной итерациями ЕМ-алгоритма, являются стационарными точками
оптимизируемой логарифмической функции правдоподобия $\ln L(\Theta )$ и что
найдется неподвижная точка алгоритма, к которой будет сходиться каждая из таких
последовательностей. Если дополнительно потребовать положительной
определенности информационной мат\-ри\-цы Фишера для $\ln L(\Theta )$ при
истинных зна\-че\-ни\-ях па\-ра\-мет\-ра $\Theta$, то можно показать, что
асимптотически по $n\rightarrow\infty$ (т.е.\ при больших выборках) существует
единственное сходящееся (по веро\-ят\-но\-сти) решение $\hat{\Theta}(n)$
уравнений метода максимального правдоподобия и, кроме того, существует в
пространстве параметров $\Theta$ норма, в которой последовательность
$\Theta^{(t)}(n)$, порожденная ЕМ-ал\-го\-рит\-мом, сходится к $\hat{\Theta}
(n)$, если только начальная аппроксимация $\hat{\Theta}^0$ не была слишком
далека от~$\hat{\Theta} (n)$. {%\looseness=1

}

Таким образом, результаты исследования свойств ЕМ-алгоритмов метода
максимального правдоподобия разделения смеси и их практическое
использование показали, что они являются достаточно работоспособными (при
известном чис\-ле компонентов смеси) даже при большом чис\-ле $k$ компонентов и
при высоких размерностях анализируемого признака~$X$~\cite{3bat}.

\subsection{Уравнения для смеси экспоненциальных распределений}
%2.3.

Применим описанный выше алгоритм к разделению смеси
экспоненциальных распределений:
$$
p(x) = \sum\limits_{j=1}^k p_j \lambda_j e^{-\lambda_j x}\,.
$$
Получаем следующие итерационные уравнения:
\begin{align*}
\g_{ij}^{(t+1)} & = \fr{p_j^{(t)}\lambda_j^{(t)}e^{-
\lambda_j^{(t)}X_i}}{\sum\limits_{l=1}^k p_l^{(t)}\lambda_l^{(t)}
e^{-\lambda_l^{(t)}X_i}}\,,\\
p_j^{(t+1)} & = \fr{1}{n}\,\sum\limits_{i=1}^n \g_{ij}^{(t)}\,.
\end{align*}

Чтобы найти  оценки $\lambda_j$, подсчитаем первую производную функции
$$\sum_{j=1}^k\sum_{i=1}^n \g_{ij}^{(t)} \ln (\lambda_j e^{-\lambda_j X_i}):$$
\vspace*{-8pt}
\begin{multline*}
\left ( \sum\limits_{j=1}^k \sum\limits_{i=1}^n
\g_{ij}^{(t)}\ln \left ( \lambda_j
e^{-\lambda_j X_i} \right ) \right )^\prime \lambda_j =\\[-3pt]
{}= \left (
\sum\limits_{j=1}^k\sum\limits_{i=1}^n \g_{ij}^{(t)}\ln (\lambda_j -\lambda_j X_i )
\right )^\prime \lambda_j =\\[-3pt]
{}= \sum\limits_{i=1}^n \g_{ij}^{(t)}\left (
\fr{1}{\lambda_j} - X_i \right )\,.
\end{multline*}

Разрешая уравнение
$$
\sum\limits_{i=1}^n \g_{ij}^{(t)}\left ( \fr{1}{\lambda_j} -X_i\right ) =0
$$
относительно $\lambda_j$, получаем следующее итерационное уравнение:
$$
\lambda_j^{(t+1)} = \fr{\sum\limits_{i=1}^n
\g_{ij}^{(t)}}{\sum\limits_{i=1}^n \g_{ij}^{(t)} X_i}\,.
$$

\subsection{Уравнения для смеси гамма-распределений } %2.4.

Применим теперь ЕМ-алгоритм к смеси гам\-ма-рас\-пре\-де\-ле\-ний вида
$$
p(x) = \sum\limits_{j=1}^k p_j \fr{\alpha_j^{\alpha_j} x^{\alpha_j -
1}}{\lambda_j^{\alpha_j} \Gamma (\alpha_j )}\,e^{-(\alpha_j / \lambda_j)x}\,.
$$

Такая параметризация удобна для нахождения
оценок~$\alpha_j$~\cite{6bat}.

Аналогичным способом выписываются итерационные уравнения:
\begin{align*}
\g_{ij}^{(t+1)} & =   \fr{p_j^{(t)}\fr{(\alpha_j^{\alpha_j} )^{(t)}
x^{\alpha_j - 1}}{(\lambda_j^{\alpha_j} )^{(t)}\Gamma (\alpha_j)}\,
e^{-(\alpha_j /\gamma_j)^{(t)}x}}{\sum\limits_{l=1}^k
p_l^{(t)}\fr{(\alpha_l^{\alpha_l})^{(t)} x^{\alpha_l -
1}}{(\lambda_l^{\alpha_l})^{(t)}\Gamma (\alpha_l )}\,
e^{-(\alpha_l /\lambda_l)^{(t)} x}}\,,\\
p_j^{(t+1)} & = \fr{1}{n}\,\sum\limits_{i=1}^n \g_{ij}^{(t)}\,.
\end{align*}

Далее найдем оценки $\lambda_j$ для данного случая, приравнивая
производную
\begin{equation} %8
\sum\limits_{j=1}^k \sum\limits_{i=1}^n \g_{ij}^{(t)} \ln \left (
\fr{\alpha_j^{\alpha_j} x^{\alpha_j -1}}{\lambda_j^{\alpha_j}\Gamma
(\alpha_j)}\,e^{-(\alpha_j /\lambda_j) x}\right )
\end{equation}
по $\lambda_j$ к нулю и разрешая относительно~$\lambda_j$ уравнение:
$$
\sum\limits_{i=1}^n \g_{ij}^{(t+1)}\left ( \fr{\alpha_j^{(t)}}{\lambda_j^{(t)}}
- \fr{\alpha_j^{(t)}X_i}{\left ( \lambda_j^{(t)}\right )^2}\right ) =0 \,.
$$
Получаем
$$
\lambda_j^{(t+1)} = \fr{\sum\limits_{i=1}^n \g_{ij}^{(t)}
X_i}{\sum\limits_{i=1}^n \g_{ij}^{(t)}}\,.
$$

Для того чтобы получить итерационные уравнения для $\alpha_j$, найдем
первую производную~(8):
\begin{multline*}
\left ( \sum\limits_{j=1}^k\sum\limits_{i=1}^n \g_{ij}^{(t)}\ln \left (
\fr{\alpha_j^{\alpha_j} x^{\alpha_j -1}}{\lambda_j^{\alpha_j}\Gamma (\alpha_j
)}\,e^{-(\alpha_j /\lambda_j ) x} \right ) \right )^\prime \alpha_j ={}\\[-3pt]
{}=\left ( \sum\limits_{j=1}^k\sum\limits_{i=1}^n \g_{ij}^{(t)}\ln \left (
\fr{\alpha_j^{\alpha_j}}{\lambda_j^{\alpha_j}}\right ) - \ln \Gamma (\alpha_j )+{} \right.\\[-3pt]
{}+\left.
(\alpha_j -1 )\ln X_i - \fr{\alpha_j}{\lambda_j}\,X_i \right )^\prime \alpha_j =\\[-3pt]
{}=\sum\limits_{i=1}^n \g_{ij}^{(t)} \left ( \ln \alpha_j +1-\ln \lambda_j -
\fr{\Gamma^\prime (\alpha_j )}{\Gamma (\alpha_j)}\right.+\\[-3pt]
{}+\left. \ln X_i - \fr{X_i}{\lambda_j}\right )\,;
\end{multline*}
\begin{multline*}
\sum\limits_{i=1}^n \g_{ij}^{(t)} \left(  \ln \alpha_j +1 -\ln \lambda_j -{}\right. \\[-3pt]
\left. {}-\fr{\Gamma^\prime (\alpha_j )}{\Gamma (\alpha_j )}+\ln X_i 
-\fr{X_i}{\lambda_j} \right) =0\,;
\end{multline*}
\begin{multline}
\fr{\Gamma^\prime (\alpha_j )}{\Gamma (\alpha_j )} ={}\\[-3pt]
{}= \fr{\sum\limits_{i=1}^n \g_{ij}^{(t)} \left ( \ln \alpha_j +1-\ln\lambda_j 
+\ln X_i -\fr{X_i}{\lambda_j} \right )}{\sum\limits_{i=1}^n \g_{ij}^{(t)}}\,.
\end{multline}
%
Здесь $\Gamma^\prime (\alpha_j ) / \Gamma (\alpha_j )$~--- это
\textit{логарифмическая производная гамма-функции}. Для нее существует так
называемое \textit{разложение Абрамовитца}--\textit{Стигана}~\cite{4bat}:
$$
\fr{\Gamma^\prime (\alpha ) }{ \Gamma (\alpha )} = \mathrm{log}\,\alpha -
\fr{1}{2\alpha }-\fr{1}{12\alpha^2 }+\ldots
$$

Подставим первые три члена разложения в~(9) и разрешим это уравнение
относительно~$\alpha_j$:
$$
\alpha_{ij}^{(t+1)} = \fr{\sum\limits_{i=1}^n
\g_{ij}^{(t+1)}}{2\sum\limits_{i=1}^n \g_{ij}^{(t +1)}\left ( \fr{X_i}{\lambda_j^{(t)}} -
\ln \fr{X_i}{\lambda_j^{(t)}} -1\right )}\,.
$$
В итоге получаем итерационные уравнения для ~$\alpha_j$.

\section{Описание программного обеспечения (программа~ЕМ)}

\subsection{Назначение программы} %3.1.

Разработанная авторами статьи программа ЕМ предназначена для решения задачи
разделения смесей экспоненциальных и гамма-распределений, поставленной в
разд.~2, с использованием ЕМ-ал\-го\-рит\-ма и формул, описанных в разд.~3.

\subsection{Инструменты разработки} %3.2.

Для создания программы была использована среда разработки Microsoft
Visual Studio .NET 2005 и объектно-ориентированный язык C\#. Для
визуализации результатов была использована свободно распространяемая
графическая библиотека ZedGraph~\cite{7bat}.


\subsection{Возможности  программы} %3.3.

\noindent
\begin{itemize}
\item Загрузка выборочных данных из текстового файла
\item Оценивание по выборке параметров смеси экспоненциальных
распределений
\item Оценивание по выборке параметров смеси гамма-распределений
\item Отслеживание изменений параметров смесей распределений во
времени в режиме <<скользящего окна>>
\item Построение гистограммы по выборке
\end{itemize}

\subsection{Входные и выходные данные. Функционирование
программы} %3.4.

В качестве \textit{входных данных} программа ЕМ получает:
\begin{itemize}
\item выборочные данные из текстового файла;
\item число компонентов смеси;
\item размер <<скользящего окна>>;
\item размер класса гистограммы.
\end{itemize}

На \textit{выходе} мы получаем:
\begin{itemize}
\item точечные оценки параметров смеси экспоненциальных
распределений;
\item точечные оценки параметров смеси гамма-распределений;
\item графическое изображение результирующей смеси распределения;
\item графическое изображение компонентов каж\-дой смеси;
\item графическое изображение того, как меняются параметры смесей
распределений с течением времени в режиме <<скользящего окна>>;
\item гистограмма, построенная по выборке;
\item значение статистического теста.
\end{itemize}

Выборочные данные загружаются из текстового файла в память программы и подаются
на вход двум независимо работающим реализациям ЕМ-алгоритма~--- для
идентификации смеси экспоненциальных распределений и для идентификации смеси
гамма-распределений. Результатом их работы являются наборы значений оцениваемых
параметров модели, предложенной в разд.~2. Кроме того, результирующие
распределения визуализируются в виде графиков. В программе можно запустить
режим <<скользящего окна>>, который для всех подвыборок заданного
размера с помощью ЕМ-алгоритма оценивает параметры смесей распределений этих
подвыборок. Все действия программы документируются в окне информации.

\section{Описание тестовых расчетов}

С использованием разработанной программы были проведены тестовые
расчеты на выборочных данных реального сетевого трафика.

На вход программы EM были поданы выборки трафика:
\begin{enumerate}[I]
\item Между лабораторией Lawrence Berkeley (Berkeley, California) и
внешним миром размера примерно 7000~\cite{8bat}~--- \textit{выборка~1}.
\item
Сети радиодоступа ЗАО <<Синтерра>> размера примерно 1000~\cite{9bat}~---
 \textit{выборка~2}.
\end{enumerate}

\subsection{Выборка 1 ``Berkeley''} %5.1.

При числе компонентов смеси~5 и случайном начальном приближении
были получены результаты, представленные в табл.~\ref{t1bat}.


Данные результаты иллюстрирует рис.~\ref{f5bat}.

Гистограмма  на рис.~\ref{f6bat} показывает статистическую значимость
полученных результатов.

Данная выборка обладает той особенностью, что она собиралась в течение
достаточно длительного времени и в ней агрегирован самый разнородный
трафик. Поэтому в ней присутствует не только большое количество
<<коротких>> сообщений (что обычно для выборок из телетрафика), но и
некоторый массив сообщений средней длины, а также определенный
<<выброс>> больших сообщений. Это свидетельствует о \textit{пиковости}
телетрафика на довольно больших промежутках времени.

Как мы видим, ЕМ-алгоритм удачно справился с задачей идентификации
смеси.

\subsection{Выборка~2 ``Synterra''} %5.2.

Результаты применения ЕМ-алгоритма к выборке ``Synterra''
представлены в табл.~\ref{t2bat}.
\begin{table*}\small
\begin{minipage}[t]{76mm}
\begin{center}
\Caption{Результаты применения ЕМ-алго\-рит\-ма к выборке~1 ``Berkeley'' 
\label{t1bat}} \vspace*{2ex}

\tabcolsep=8.7pt
\begin{tabular}{|c|c|c|}
\hline
№&Начальное приближение&Результат\\
\hline
\multicolumn{3}{|c|}{$P$}\\
\hline
0&0,2&0,1896\\
1&0,2&0,1858\\
2&0,2&0,1830\\
3&0,2&0,2259\\
4&0,2&0,2154\\
\hline
\multicolumn{3}{|c|}{$\alpha$}\\
\hline
0&2,7028&10,9783\hphantom{9}\\
1&3,6273&5,8621 \\
2&5,7598&2,7092\\
3&0,2315&1,0235\\
4&0,9110&0,4772\\
\hline
\multicolumn{3}{|c|}{$\lambda$}\\
\hline
0&85,2066&137,1714  \\
1&23,9592&136,7349\\
2&63,8425&132,6482\\
3&\hphantom{9}1,8026&116,7317\\
4&98,3882&102,5278\\
\hline
\end{tabular}
\end{center}
\end{minipage}\hfill
\begin{minipage}[t]{76mm}
%\end{table*}
%\begin{table*}\small
\begin{center}
\Caption{Результаты применения ЕМ-алго\-рит\-ма к выборке~2 ``Synterra'' 
\label{t2bat}} \vspace*{2ex}

\tabcolsep=8.7pt
\begin{tabular}{|c|c|c|}
\hline
№&Начальное приближение&Результат\\
\hline
\multicolumn{3}{|c|}{$P$}\\
\hline
0&0,2&$0{,}3815\hphantom{{}\cdot 10^{-9}}$\\
1&0,2&$0{,}3594\hphantom{{}\cdot 10^{-9}}$\\
2&0,2&$0{,}2589\hphantom{{}\cdot 10^{-9}}$\\
3&0,2&$0{,}4401\cdot 10^{-9}$\\
4&0,2&$0{,}0\hphantom{{}\cdot 10^{-9}999}$\\
\hline
\multicolumn{3}{|c|}{$\alpha$}\\
\hline
0&6,0804&1,5833\\
1&3,1838&0,8554\\
2&1,4886&0,4557\\
3&4,6407&0,2278\\
4&3,7843&0,1139\\
\hline
\multicolumn{3}{|c|}{$\lambda$}\\
\hline
0&17,3387&15,8682\\
1&47,8294&16,9150\\
2&54,1984&19,2866\\
3&\hphantom{1}8,6254&19,2866\\
4&\hphantom{1}5,7252&19,2866\\
\hline
\end{tabular}
\end{center}
\end{minipage}
\end{table*}


Данные результаты иллюстрируют рис.~\ref{f7bat}.


Эти результаты также отражают действительную картину, как показано на
рис.~\ref{f8bat}.


Этот трафик был снят с базовой станции <<Лукойл-Юго-Запад>> сети
широкополосного радиодоступа ЗАО <<Синтерра>>. Сеть радиодоступа
является реализацией так называемой <<последней мили>>, переносящей два
разных вида трафика: данные (Ethernet пакеты) и голос (IP-телефония, VoIP).
Поэтому здесь присутствуют в качестве основной массы короткие, но
интенсивные сообщения (пакеты SIP и голосовые фреймы), а также длинные
сообщения, содержащие данные.

Как мы видим, программная реализация ЕМ-ал\-го\-рит\-ма успешно справилась с
задачей разделения смесей распределений для этих двух выборок, что делает
данную программу удобным инструментом построения стохастической картины
конкретной сети. По полученным данным, используя метод интерпретации,
предложенный в разд.~2, можно получить представление о количестве
последовательных и параллельных структур вероятностной модели сети.

\subsection{Режим <<скользящего окна>>} %5.3.

Результаты для выборки
``Berkeley'' в режиме <<скользящего окна>>  представлены
на рис.~\ref{f9bat}.


Данные графики показывают изменение параметров распределений подвыборок выборки 
``Berkeley''. Видно, что параметры распределений подвыборок не остаются 
неизменными во времени, наоборот, они имеют внешне случайный характер. На 
рис.~\ref{f9bat},\,\textit{в} видна даже своеобразная пульсация первой 
компоненты.
%
На основании расчетов можно сделать вывод о том, что пиковость трафика
обусловливается как формой, так и интенсивностью сообщений.

\section{Заключение}

В данной работе исследована вероятностная модель  информационных потоков,
возникающих в сложных телекоммуникационных конвергентных сетях, построенная с
помощью асимптотического и энтропийного подходов. Эта модель предполагает, что
функционирование сложной телекоммуникационной сети можно представить в виде
суперпозиции довольно простых стохастических структур~--- последовательных и
параллельных, которые по\-рож\-да\-ют смеси гамма-распределений для случайной
величины времени обработки и передачи сообщений в сети. Предложена простая
интерпретация параметров данной модели.
\begin{figure*} %fig5
\vspace*{1pt}
\begin{center}
\mbox{%
\epsfxsize=130mm %145.109mm 
\epsfbox{bat-5.eps} }
\end{center}
\vspace*{-13pt} \Caption{Компоненты смеси начального приближения~(\textit{а}) и 
результата~(\textit{б}) для выборки~1 ``Berkeley'' \label{f5bat}}
%\end{figure*}
%\begin{figure*} %fig6
\vspace*{12pt}
\begin{center}
\mbox{%
\epsfxsize=130mm %148.256mm 
\epsfbox{bat-7.eps} }
\end{center}
\vspace*{-13pt} \Caption{График смеси распределений~(\textit{1}) и гистограмма 
для выборки~1 ``Berkeley''~(\textit{2}) \label{f6bat}}
\end{figure*}



\begin{figure*} %fig7
\vspace*{1pt}
\begin{center}
\mbox{%
\epsfxsize=130mm %144.283mm 
\epsfbox{bat-8.eps} }
\end{center}
\vspace*{-16pt} \Caption{Компоненты смеси начального приближения~(\textit{а}) и 
результата~(\textit{б}) для выборки~2 ``Synterra'' \label{f7bat}}
%\end{figure*}
%\begin{figure*} %fig8
\vspace*{12pt}
\begin{center}
\mbox{%
\epsfxsize=130mm %148.256mm 
\epsfbox{bat-10.eps} }
\end{center}
\vspace*{-11pt} \Caption{График смеси распределений~(\textit{1}) и гистограмма
для выборки~2 ``Synterra''~(\textit{2}) \label{f8bat}}
\end{figure*}

\begin{figure*} %fig9
\vspace*{1pt}
\begin{center}
\mbox{%
\epsfxsize=119.041mm
\epsfbox{bat-11.eps} }
\end{center}
\vspace*{-9pt} \Caption{Изменение  смешивающих параметров~(\textit{а}), 
параметров формы~(\textit{б}) и параметров масштаба~(\textit{в}) во времени для 
выборки~1 ``Berkeley'' \label{f9bat}}
\end{figure*}

Для решения вытекающей из модели задачи предложен итерационный алгоритм,
базирующийся на методе максимального правдоподобия~--- ЕМ-ал\-го\-ритм, для
которого получены формулы для конкретного вида смесей~--- экспоненциальных и
гамма-распределений.
%
Кроме того, разработан программный инструментарий для оценки параметров 
предложенной модели на выборках из реальных трафиковых данных. Проведены 
исследования, которые подтвердили предположения вероятностной модели. 


Получение информации о стохастической структуре
телекоммуникационных сетей и наличие программных инструментов для
выявления более или менее стабильных структур позволит понять причины
возникновения неожиданных больших нагрузок, предотвратить такие нагрузки,
а также поможет в будущем в проектировании надежных, оптимальных по
стоимости и уровню сервиса телекоммуникационных сетей нового поколения.

%\vspace*{-15pt} 
{\small\frenchspacing
{%\baselineskip=10.8pt
\addcontentsline{toc}{section}{Литература}
\begin{thebibliography}{9}
\bibitem{1bat}
Teletraffic Engeneering Handbook. International Telecommunication Union, 
Geneva, 2005 {\sf http://www.itu.int}. \vspace*{5pt} 
\bibitem{2bat}
\Au{Севастьянов~Б.\,А.} Курс теории вероятностей и математической статистики. 
М., 2004. \vspace*{5pt} 
\bibitem{3bat}
\Au{Айвазян~C.\,А., Бухштабер~В.\,М., Енюков~И.\,С, Мешалкин~Л.\,Д.} Прикладная 
статистика. Классификация и снижение размерности~// Финансы и статистика. М., 
1989. \vspace*{5pt} 
\bibitem{4bat}
\Au{Bilmes~J.\,A.} A gentle tutorial of the EM algorithm and its application to 
parameter estimation for Gaussian mixture and hidden Markov models. Berkeley, 
CA, USA: International Computer Science Institute,  1998. \vspace*{5pt} 
\bibitem{5bat}
\Au{Шлезингер~М.\,И.} О самопроизвольном различении образов~// Шлезингер~М.\,И. 
Читающие. автоматы. Киев: Наукова думка, 1965. С.~38--45. \vspace*{5pt} 
\bibitem{6bat}
\Au{Hsiao~I.-T., Rangarajan~A., Gindi~G.}. Joint-MAP 
reconstruction/segmentation for transmission tomography using mixture-models as 
priors. Yale University, 1998. \vspace*{5pt} 
\bibitem{7bat}
{\sf http://zedgraph.org}. \vspace*{4pt} 
\bibitem{8bat}
{\sf http://ita.ee.lbl.gov/html/contrib/LBL-PKT.html}. \vspace*{5pt} 
\bibitem{9bat}
{\sf http://www.synterra.ru}.
\end{thebibliography}

} } \label{end\stat}
\end{multicols}


%\addtocounter{razdel}{1}
%\def\razd{НЕРЕГУЛИРУЕМЫЙ ЭЛЕКТРОПРИВОД ДЛЯ ЭЛЕКТРОЭНЕРГЕТИКИ}

\setcounter{page}{3}

   { %\Large  
   { %\baselineskip=16.6pt
   
   \vspace*{-48pt}
   \begin{center}\LARGE
   \textit{Предисловие}
   \end{center}
   
   %\vspace*{2.5mm}
   
   \vspace*{25mm}
   
   \thispagestyle{empty}
   
   { %\small 

    
Вниманию читателей журнала <<Информатика и её применения>> предлагается 
очередной тематический выпуск <<Вероятностно-статистические методы и 
задачи информатики и информационных технологий>>. Предыдущие тематические 
выпуски журнала по данному направлению вышли в 2008~г.\ (т.~2, вып.~2), 
в 2009~г.\ (т.~3, вып.~3) и в 2010~г.\ (т.~4, вып.~2). 

Статьи, собранные в данном журнале, посвящены разработке новых вероятностно-статистических 
методов, ориентированных на применение к решению конкретных задач информатики и информационных 
технологий, а также~--- в ряде случаев~--- и других прикладных задач. Проблематика, охватываемая 
публикуемыми работами, развивается в рамках научного сотрудничества между Институтом проблем 
информатики Российской академии наук (ИПИ РАН) и Факультетом вычислительной математики и 
кибернетики Московского государственного университета им.\ М.\,В.~Ломоносова в ходе работ 
над совместными научными проектами (в том числе в рамках функционирования 
Научно-образовательного центра <<Вероятностно-статистические методы анализа рисков>>). 
Многие из авторов статей, включенных в данный номер журнала, являются активными участниками 
традиционного международного семинара по проблемам устойчивости стохастических моделей, 
руководимого В.\,М.~Золотаревым и В.\,Ю.~Королевым; регулярные сессии этого семинара 
проводятся под эгидой МГУ и ИПИ РАН (в 2011~г.\ указанный семинар проводится в октябре 
в Калининградской области РФ). 

Наряду с представителями ИПИ РАН и МГУ в число авторов данного выпуска журнала входят 
ученые из Научно-исследовательского института системных исследований РАН, Института 
проблем технологии микроэлектроники и особочистых материалов РАН, Института 
прикладных математических исследований Карельского НЦ РАН, Московского 
авиационного института, Вологодского государственного педагогического университета, 
НИИММ им.\ Н.\,Г.~Чеботарева, Казанского государственного университета, Дебреценского 
университета (Венгрия).

Несколько статей выпуска посвящено разработке и применению стохастических методов и 
информационных технологий для решения различных прикладных задач. В~работе В.\,Г.~Ушакова 
и О.\,В.~Шестакова рассмотрена задача определения вероятностных характеристик случайных 
функций по распределениям интегральных преобразований, возникающих в задачах эмиссионной 
томографии. В~статье Д.\,О.~Яковенко и М.\,А.~Целищева рассмотрены некоторые вопросы 
математической теории риска и предложен новый подход к диверсификации инвестиционных 
портфелей. Работа И.\,А.~Кудрявцевой и А.\,В.~Пантелеева посвящена построению и 
исследованию математической модели, описывающей динамику сильноионизованной плазмы. 
В~статье П.\,П.~Кольцова изучается качество работы ряда алгоритмов сегментации изображений. 
Статья А.\,Н.~Чупрунова и И.~Фазекаша посвящена вероятностному анализу числа без\-оши\-бочных 
блоков при помехоустойчивом кодировании; получены усиленные законы больших чисел для указанных 
величин.

В данном выпуске традиционно присутствует тематика, весьма активно разрабатываемая в течение 
многих лет специалистами ИПИ РАН и МГУ,~--- методы моделирования и управления для 
информационно-телекоммуникационных и вычислительных систем, в частности методы 
теории массового обслуживания. В~статье А.\,И.~Зейфмана с соавторами рассматриваются 
модели обслуживания, описываемые марковскими цепями с непрерывным временем в случае 
наличия катастроф. В~работе М.\,М.~Лери и И.\,А.~Чеплюковой рассматриваются случайные 
графы Интернет-типа, т.\,е.\ графы, степени вершин которых имеют степенные распределения; 
такие задачи находят применение при исследовании глобальных сетей передачи данных. 
Работа Р.\,В.~Разумчика посвящена исследованию систем массового обслуживания специального 
вида~--- с отрицательными заявками и хранением вытесненных заявок.

Ряд статей посвящен развитию перспективных теоретических 
вероятностно-статистических методов, которые находят широкое применение в различных 
задачах информатики и информационных технологий. В~работе В.\,Е.~Бенинга, А.\,К.~Горшенина 
и В.\,Ю.~Королева рассмотрена задача статистической проверки гипотез о числе компонент 
смеси вероятностных распределений, приводится конструкция асимптотически наиболее мощного 
критерия. Результаты этой работы найдут применение в ряде прикладных задач, использующих 
математическую модель смеси вероятностных распределений (в информатике, моделировании 
финансовых рынков, физике турбулентной плазмы и~т.\,д.). В~статье В.\,Ю.~Королева, 
И.\,Г.~Шевцовой и С.\,Я.~Шоргина строится новая, улучшенная оценка точности нормальной 
аппроксимации для пуассоновских случайных сумм; как известно, указанные случайные суммы 
широко используются в качестве моделей многих реальных объектов, в том числе в информатике, 
физике и других прикладных областях. Работа В.\,Г.~Ушакова и Н.\,Г.~Ушакова посвящена 
исследованию ядерной оценки плотности распределения; эти результаты могут применяться, 
в част\-ности, при анализе трафика в телекоммуникационных системах. Серьезные приложения 
в статистике могут получить результаты работы О.\,В.~Шестакова, в которой доказаны оценки 
скорости сходимости распределения выборочного абсолютного медианного отклонения к нормальному 
закону. 

\smallskip

Редакционная коллегия журнала выражает надежду, что данный тематический  выпуск 
будет интересен специалистам в области теории вероятностей и математической статистики 
и их применения к решению задач информатики и информационных технологий.
     
     %\vfill 
     \vspace*{20mm}
     \noindent
     Заместитель главного редактора журнала <<Информатика и её 
применения>>,\\
     директор ИПИ РАН, академик  \hfill
     \textit{И.\,А.~Соколов}\\
     
     \noindent
     Редактор-составитель тематического выпуска,\\
     профессор кафедры математической статистики факультета\\
      вычислительной математики и кибернетики МГУ им.\ М.\,В.~Ломоносова,\\
     ведущий научный сотрудник ИПИ РАН,\\ 
доктор физико-математических наук \hfill
      \textit{В.\,Ю.~Королев}
     
     } }
     }



\def\stat{zhelenkova}

{\begin{center}
{\Large
Статьи, являющиеся развитием докладов, %}\\[6pt]
%{\Large 
представленных }\\[6pt]
{\Large  на конференции <<Электронные библиотеки:}\\[6pt]
{\Large перспективные методы и технологии, %}\\[6pt]
%{\Large 
электронные коллекции>>}\\[9pt]
{\large (RCDL'2011, г.~Воронеж, 19--22~октября 2011~г.)}
\end{center}
}

\def\tit{ИССЛЕДОВАНИЕ РАДИОИСТОЧНИКОВ СРЕДСТВАМИ ВИРТУАЛЬНОЙ 
ОБСЕРВАТОРИИ$^*$}

\def\titkol{Исследование радиоисточников средствами виртуальной 
обсерватории}

\def\autkol{О.\,П.~Желенкова}
\def\aut{О.\,П.~Желенкова$^1$}

\titel{\tit}{\aut}{\autkol}{\titkol}

{\renewcommand{\thefootnote}{\fnsymbol{footnote}}\footnotetext[1]
{Работа поддержана грантом РФФИ №\,10-07-00412.}}


\renewcommand{\thefootnote}{\arabic{footnote}}
\footnotetext[1]{Специальная астрофизическая обсерватория РАН, zhe@sao.ru}

\vspace*{6pt}


\Abst{В течение ряда лет с использованием разных подходов на базе средств виртуальной обсерватории в САО 
РАН проводились исследования радиоисточников обзоров, выполненных на крупнейшем российском 
радиотелескопе РАТАН-600 в 1980--1999~гг. Проведено их массовое отождествление с максимальным 
использованием имеющихся в открытом доступе данных разных диапазонов электромагнитного спектра. 
С~применением программного инструментария виртуальной обсерватории реализован подход по автоматической 
подготовке и предварительной обработке данных. Для полученного компилятивного каталога разработана 
ин\-фор\-ма\-ци\-он\-но-поиско\-вая сис\-те\-ма, которая применялась при анализе информации о каждом источнике и 
принятии решения об отождествлении. Исходя из полученного опыта при работе c многочисленными 
разнородными ресурсами, можно подытожить, что программные средства виртуальной обсерватории 
обеспечивают удобный доступ к астрономическим данным и существенно повышают эффективность научных 
исследований. Однако все еще нет развитого инструментария для дальнейшего анализа, актуализации и 
публикации собранных исследователем данных. Рядом проектов ведутся разработки по реализации большей 
связности данных на базе уже существующих веб-технологий, что переведет сервисы виртуальной 
обсерватории на новый уровень, обеспечивающий обмен знаниями посредством аннотирования записей 
каталогов и реализацией связей между ними.}

\vspace*{2pt}

\KW{виртуальная обсерватория; распределенные информационные системы; информационные технологии в 
научных исследованиях; интеграция неоднородных информационных ресурсов; базы данных}

\vskip 14pt plus 9pt minus 6pt

      \thispagestyle{headings}

      \begin{multicols}{2}

            \label{st\stat}

\section{Введение}
        
      Сегодня виртуальная обсерватория~--- прежде всего средство удобного и эффективного 
доступа к разнообразным астрономическим данным. Первые шаги в сторону организации 
этой распределенной инфраструктуры были сделаны в 1990-х гг., когда в 
США была создана сеть центров данных для поддержки информации, полученной 
космическими миссиями НАСА. Достижения в области информационных технологий 
обеспечили основу, на которой распределенные коллекции данных стали рассматриваться 
как интегрированная информационная система. Виртуальная обсерватория открыла новые 
направления научных исследований, опирающиеся на статистический анализ, поиск новых 
закономерностей и объединение данных разных диапазонов.
      
      Обычной практикой в астрономии были отдельные и/или повторные наблюдения 
индивидуальных объектов, что хорошо работало при открытии фундаментальных законов. 
Но по мере того как понимание разных астрономических феноменов и закономерностей 
становилось более точным, чис\-ло вопросов, на которые можно ответить с по\-мощью одного 
наблюдения, становилось все меньшим. Методы наблюдений в настоящее время все больше 
смещаются в сторону накопления больших объемов данных, а подход к решению 
астрофизических задач~--- к статистическим методам. Большой объем данных может 
привести к обнаружению процессов, чьи наблюдательные проявления маскируются разными 
эффектами, которые трудно бывает учесть из-за недостаточного количества данных. 
Наблюдательное время самых мощных инструментов было, остается и будет оставаться 
весьма ограниченным, поэтому многие астрофизические вопросы, для решения которых 
требуется большое количество наблюдательных данных, час\-то прос\-то не могут 
рассматриваться. 
      
      Взрывообразный рост объема и сложности данных вызван прогрессом в получении 
цифровых изоб\-ра\-же\-ний (основной источник данных в астрономии), способов обработки, 
хранения и доступа к информации. В~астрономии происходит сдвиг в сторону науки, 
базирующейся на обзорах, которые становятся все более важным методом в исследовании 
Вселенной. Сейчас имеются средства для проведения обзоров практически во всем 
диапазоне электромагнитного спектра, пространственных масштабов и временных эпох. 
Каталоги, по\-лу\-ча\-емые из панхроматических обзоров, дают возможности для обнаружения 
новых явлений, которые могут фундаментально изменить наши представления о физике 
звезд и их эволюции, ближнем космосе и планетных сис\-те\-мах, формировании галактик и 
природе активных ядер галактик. Например, изоб\-ра\-же\-ние одного и того же участка неба в 
оптическом и радиодиапазоне привело к открытию квазаров в\linebreak 1960-е~гг., а 
данные в инфракрасном диапазоне позволили исследовать скрытые от наблюдателя \mbox{пылью} 
области звездообразования и активные ядра галактик, чего невозможно сделать по 
оптическим изображениям. 

Повторные наблюдения областей неба привели к открытию 
транзиентных событий~--- сверхновых и более редких явлений~--- микролинзирования. 
Панхроматические наборы данных позволяют сравнивать теоретические модели и реальные 
данные. \mbox{Такие} исследования предъявляют определенные требования как к постановке 
задачи, так и к методам их решения, которые все больше опираются на информационные 
технологии и в первую очередь на веб-сер\-ви\-сы, системы управления базами данных, грид 
и облачные вычисления. 
      
      Исследования радиогалактик важны для понимания механизмов излучения 
внегалактических объектов в радиодиапазоне и относятся к широкому кругу задач, 
связанному с изучением проявлений активности галактических ядер. Феномен мощного 
радиоизлучения ядра галактики является кратковременной по космологическим масштабам\linebreak 
(до $\sim10^8$~лет) эволюционной фазой самых массивных звездных систем. Хотя мощные 
радио\-га\-лактики~--- редкие объекты (пространственная плот\-ность оценивается в 
      $\sim10^{-6}$~Мпс$^{-3}$), их наблюдение в радиодиапазоне возможно практически на 
любых космологических расстояниях, что используется при изучении крупномасштабной 
структуры Вселенной, проверке гипотез формирования самых первых звездных систем и 
решении других космологических задач.
      
      Отождествление радиоисточников с объектами в других диапазонах~--- обязательная 
процедура при многочастотных исследованиях, и не такая прос\-тая, как это кажется на 
первый взгляд. Кросс-иден-\linebreak ти\-фи\-ка\-ция по координатам (около объекта в обла\-сти с 
заданным радиусом ищется объект другого каталога) оптических и радиокаталогов при 
разном угловом разрешении, предельной чув\-ст\-ви\-тель\-ности и координатной точности 
последних, а также морфологической структуре самих источников дает от 5\% до 30\% 
совпадений. Отметим, что только по оптическим данным можно оценить расстояние до 
родительских галактик радиоисточников, которые по большей час\-ти~--- весь\-ма слабые 
оптические объекты, особенно на больших красных смещениях. Их оптические наблюдения 
требуют больших затрат наблюдательного времени на крупнейших оптических телескопах 
при наилучших погодных условиях.
      
      В качестве примера приведем результаты исследований выборки источников с 
крутыми спектрами (SS, {Steep Spectra}) из каталога RC ({RATAN Cold}), 
полученного по материалам глубокого обзора полоски неба на радиотелескопе 
      РАТАН-600~[1--3]. Для кандидатов в выборку учитывались угловые размеры, 
морфологическая структура, а также яркость объекта в радиодиапазоне. Так из $\sim1000$ 
источников каталога~RC в выборку с крутыми спектрами вошли $\sim100$~объектов. 
Потребовалось 15~лет фотометрических и спектральных наблюдений на 6-м оптическом 
телескопе БТА (Большой телескоп азимутальный), чтобы отождествить и получить спектры 
для объектов выборки~[4]. Из этих объектов у четырех источников оказалось $1 \leq Z<2$, у 
трех $2\leq Z<3$, у одного радиоисточника $3\leq Z< 4$ и самый далекий объект выборки с 
$Z=4.51$. 
      
      Не только поиск далеких радиогалактик, но и статистические свойства 
радиоисточников в разных диапазонах электромагнитного спектра важны для понимания 
природы активных галактических ядер. Массовое исследование радиоисточников позволяет 
уточнять существующие и открывать новые селекционные критерии, которые можно 
использовать при классификации этих объектов. И такие исследования проводятся с 
привлечением современных цифровых обзоров в разных диапазонах. 
      
      На крупнейшем российском радиотелескопе РАТАН-600 в 1980--1999~гг.\ была 
проведена серия глубоких обзоров полосы неба шириной около 40~угловых минут. По 
данным этих обзоров получен каталог~RC, а затем RCR ({RATAN Cold 
Revised})~[5]. С~появлением глубоких цифровых обзоров в оптическом и инфракрасном 
диапазоне, таких как SDSS ({Sloan Digital Sky Survey})~[6] и UKIDSS ({United 
Kingdom Infra-red Deep Sky Survey})~[7], появилась возможность провести отождествление 
этих каталогов. Для выполнения этой задачи были максимально использованы все 
имеющиеся в открытом доступе данные~--- два оптических обзора: DSS2 ({Digital 
Sky Survey}) и SDSS (полосы $u$, $g$, $r$, $i$, $z$), включая каталоги GSC ({Guide 
Star Catalog})~[8] и USNO-B1~[9], обзоры ближнего инфракрасного диапазона 2MASS 
({Two Micron All Sky Survey})~[10] и UKIDSS (полосы~$J$, $H$, $K$), а также 
проведены многочастотные исследования радиоисточников по радиообзорам VLSS 
({VLA Low-frequency Sky Survey}, 74~МГц)~[11], TXS ({Texas Survey of radio 
sources}, 365~МГц)~[12], NVSS ({NRAO-VLA Sky Survey}, 1,4~ГГц)~[13], FIRST 
({Faint Images of the Radio Sky at Twenty centimeters}, 1,4~ГГц)~[14], GB6 
({Green Bank survey}, 4,85~ГГц)~[15]. Радиоисточники каталога RC~[16--18], а затем 
RCR~[19, 20] были отождествлены с данными этих обзоров. Радиоисточники, у которых не 
обнаруживались кандидаты в оптических/инфракрасных каталогах, дополнительно 
отождествлялись с суммарными изображениями обзора SDSS в трех фильтрах ($g$, $r$, 
$i$) и/или инфракрасного обзора UKIDSS в фильтрах ($J$, $H$, $K$) для достижения 
более глубокого предела кадров. 
      
      Для многочастотного исследования выборки источни\-ков каталога~RC, а затем 
каталога~RCR разработана методика детального отождествления радиоис\-точников, 
включающая подбор информационных ресурсов, автоматическую подготовку данных из 
выбранных ресурсов для каждого источника, морфологическую классификацию, визуальную 
инспекцию подготовленных данных для принятия решения об отождествлении~[21, 22]. 
Средствами интерактивного атласа неба Aladin~[23] (программного интерфейса Perl для 
командного интерфейса и макроконтроллера), а также с по\-мощью программного интерфейса 
Python к SAOImage DS9~[24] реализованы потоки работ по списку\linebreak радиоисточников для 
подготовки данных и визуализации результатов. Для полученного компилятивного каталога 
разработана ин\-фор\-ма\-ци\-он\-но-по\-иско\-вая сис\-те\-ма, которая использовалась при\linebreak 
отождествлении радиоисточников. 

\vspace*{-9pt}

\section{Программные средства виртуальной обсерватории}

\vspace*{-2pt}

      Сейчас активно развиваются программные средства (протоколы, метаданные и 
программы, функционирующие на их основе), учитывающие особенности информации, 
относящейся к определенной сфере человеческой деятельности, в частности к научным 
исследованиям. Такие разработки в астрономии объединены в виртуальную организацию, 
которая носит название виртуальной обсерватории и координируется международным 
альянсом IVOA ({International Virtual Observatory Alliance})~[25--27]. В~IVOA 
объединены виртуальные обсерватории разных стран, включая и Российскую виртуальную 
обсерваторию~[28]. Рабочими группами IVOA ведется разработка стандартов более чем по 
десятку направлений, среди которых: представление и формализация данных и знаний 
предметной области, разработка протоколов доступа к данным, стандарты программных 
сервисов для распределенных вычислений, протокол обмена сообщениями для программных 
клиентских приложений виртуальной обсерватории, описание и публикация ресурсов, 
формат обмена данными, язык запросов, поддержка сохранности данных и~пр. С~момента 
появления в 2003~г.\ альянса IVOA разработано около полусотни спецификаций 
протоколов, форматов и соглашений, используемых при создании программных продуктов 
виртуальной обсерватории. Подробный обзор стандартов виртуальной обсерватории и 
применяемых технологий можно найти в обзоре Брюхова и~др.~[28].

\vspace*{-2pt}

\subsection{Текущий статус}

      Инфраструктура виртуальной обсерватории является 
сер\-вис\-но-ориен\-ти\-ро\-ван\-ной. Веб-сер\-ви\-сы IVOA разделены на три класса: 
обнаружение и пуб\-ликация ресурсов, передача данных и организация\linebreak запросов, а также 
сервисы для распределенных вычислений. Обнаружение данных выполняется через регистры 
виртуальной обсерватории. Для сервисов, предоставляемых регистрами виртуальной 
обсерватории, и спецификаций описания ресурсов были рассмотрены несколько 
индустриальных стандартов, обеспечивающих механизмы обмена метаданными в Интернете, 
и был выбран протокол OAI-PMH ({Open Archives Initiative Protocol for 
Metadata Harvesting})~[29]. Для описания астрономических ресурсов (каталогов, цифровых 
обзоров, баз данных, архивов наблюдений, программных средств, функционирующих как 
      веб-сер\-ви\-сы) в регистрах используется стандарт описания сетевых ресурсов 
Dublin Core~[30]. Веб-сер\-ви\-сы ориентированы на то, чтобы операции над данными в сети 
выполнялись без участия человека. Повторное использование простых сервисов и 
комбинирование их для выполнения более сложных действий реализует поток работ. 
Потоковое выполнение использует принципы интероперабельности, когда компоненты 
потока работ взаимодействуют друг с другом посредством протоколов, опре\-де\-ля\-ющих 
правила запуска сервиса и структуру входных и выходных данных. Реализация таких 
протоколов опирается на модели данных. 
      
      Сервисы данных, кроме стандартных графических форматов (gif, jpeg), работают 
с двумя астрономическими форматами~--- FITS ({Flexible Image Transport 
System})~[31], который является с 1982~г.\ астрономическим стандартом для хранения и 
обмена данными, и VOTable~[32]. VOTable-фор\-мат используется в сервисах ВО для 
представления результатов запросов. Основой VOTable является индустриальный стандарт 
XML и опыт разработок астрономических форматов FITS и CDS\ Astrores. 
      
      Астрономы для обозначения одних и тех же физических величин и параметров 
используют \mbox{разные} названия. Чтобы избежать не\-од\-но\-знач\-ности при интерпретации 
величины, необходимо определить, что именно обозначают разные идентификаторы. 
В~VOTable-фор\-ма\-те используется семантический описатель UCD ({Uniform 
Content Descriptor})~[33], который устанавливает смысловую связь между обозначениями 
величин и астрономическими понятиями и/или физическими величинами. IVOA 
поддерживает и контролирует словарь дескрипторов. 
      
      Доступ к данным DAL ({Data Access Layer})~[34] включает стандарты, 
описывающие механизм доступа к распределенным астрономическим данным, и 
программные средства, обеспечивающие такой доступ. Для реализации запросов 
используется расширенное подмножество SQL~--- ADQL ({Astronomical Data 
Query Language})~[35]. Хотя SQL можно использовать для запросов к большинству 
современных астрономических баз данных, астрономическая специфика требует расширения 
возможностей языка. ADQL, кроме координатных запросов, поддерживает доступ по 
протоколам ВО к таблицам, изображениям и спектрам. 
      
      Так в общих чертах можно описать основы виртуальной обсерватории, которая за 
десять с небольшим лет превратилась в действующую инфраструктуру, и для астрономов 
сейчас нет проблем доступа к данным цифровых обзоров неба, архивам наблюдательных 
данных и каталогам. Есть удобные клиентские приложения и программные интерфейсы к 
ним, веб-интерфейсы к основным базам астрономических данных для поиска информации по 
одиночному объекту или списку объектов, запросов по списку объектов и визуализации 
полученной информации. 

\vspace*{-4pt}
  
\subsection{Новые задачи}

\vspace*{-1pt}

      На передний план выходит следующая цель виртуальной обсерватории~--- обеспечить 
профессиональных астрономов возможностью получать информацию о небесных объектах. 
Сейчас это еще в слабой степени решается средствами виртуальной обсерватории, так как 
данные в основном не структурированы и не являются связанной по смысловому 
содержимому информацией. По этой причине невозможно выполнить запрос типа <<найти 
все источники в каталогах, которые являются квазарами>> и~т.\,п. Это ограничивает 
пользователю эффективную работу с информацией. Нет прямого способа воспользоваться 
знаниями об объекте, полученными другими исследователями. 
      
      Вся имеющаяся у астрономического сообщества информация о небесных объектах 
собрана в каталогах и используется для статистических исследований объектов, подбора 
интересующих объектов, для поиска аномальных объектов или объектов с особой 
комбинацией свойств. Каталоги публикуются различными способами: от таблиц в журналах 
до публикации сервисами VizieR~[36]. Самые крупные каталоги доступны через 
специализированные веб-ин\-тер\-фей\-сы для архивов и центров данных, таких как 
WFCAM~[37], SDSS, IRSA~[38], SkyView, MAST~[39] и~др. Есть следующие 
проблемы при работе с каталогами: 
      \begin{itemize} %[1)]
  \item в каталогах содержатся измеренные величины, а результаты их анализа и 
интерпретации пуб\-ли\-ку\-ют\-ся обычно в журнальных статьях. Через гиперссылки, 
предоставляемые информационной системой ADS ({Astrophysics Data System}), в 
которой находится большая часть астрономических полнотекстовых статей, журнальная 
статья может указывать на каталог, используемый в ней, но в архивах данных не всегда 
реализованы аналогичные указатели на литературу;\\[-15pt]
  \item небесные объекты не имеют уникальных идентификаторов. Объединение 
информации в различных диапазонах электромагнитного спектра требует операций 
  кросс-идентификации, при которой строки из двух каталогов, содержащих различную, но 
при связывании вдвойне полезную информацию, объединяются на основе перекрытия 
координатных положений с допусками, учитывающими ошибки сравниваемых каталогов. 
Кросс-идентификация проводится пользователями многократно, поскольку эта важная связь 
между объектами каталогов не сохраняется;\\[-15pt] 
  \item каталоги являются статическими объектами. Ес\-ли создается новый каталог, 
полученный из одного или нескольких существующих каталогов, но с добавлением какой-то 
новой информации, то для него, как правило, не отслеживаются родительские каталоги;\\[-15pt]
  \item поиск данных в каталогах может быть трудоемким, а объединение данных из двух 
каталогов трудоемко и неудобно. Сложно работать с компилятивными каталогами, 
полученными на базе объединения нескольких каталогов разных диапазонов, даже если эта 
информация собирается по небольшому списку объектов;\\[-14pt]
  \item если появляются новые релизы каталогов или новые каталоги, то исследователю 
надо заново\linebreak выполнять одни и те же запросы для ин\-те\-ре\-су\-ющих его объектов. Нет средств 
для оповещения о появлении новой информации и об\-нов\-ле\-ния данных пользователя.
  \end{itemize}
  
\subsection{Новый качественный уровень виртуальной обсерватории}

      В последнее время появилось несколько проектов, направленных на дальнейшее 
развитие инфраструктуры виртуальной обсерватории и, в част\-ности, на решение проблем, 
которые возникают при работе с каталогами. Далее приведем сведения о тех из них, которые, 
вероятнее всего, могут помочь в работе с разнородными данными, полученными при 
массовом отождествлении радиоисточников. В~этих проектах ведутся разработки, связанные 
с внедрением инновационных информационных технологий (грид, облачные вычисления, 
Семантический Веб) в астрофизические исследования.
      
      Цель проекта AstroDAbis~[40]~--- создание независимого механизма публикации 
пояснений (комментариев, аннотаций). Пояснения могут создаваться пользователем для 
одиночного объекта (<<объект X есть квазар>>) или для нескольких объектов (<<объект с 
номером~123 в каталоге~$A$ есть то же самое, что объект с номером~456 в 
каталоге~$B$>>). Как полагают авторы AstroDAbis, этим решаются проблемы передачи 
знаний, создания компилятивных каталогов и реализации их связи с родительскими 
каталогами. Авторы статей, где представлена информация, полученная на основе анализа 
каталогов, с помощью аннотаций могут передать знания о небесном объекте в форме, 
которая может быть использована в последующих запросах к каталогу. Когда возникнет 
потребность объединить два каталога и создать компилятивный каталог (например, слияние 
оптических данных SDSS и инфракрасных данных тех же источников из UKIDSS), такая 
связь позволит обойтись без повторной кросс-иден\-ти\-фи\-ка\-ции ресурсов. С~по\-мощью 
аннотации такие каталоги сохранят связи с исходными каталогами, и связи будут однозначно 
зафиксированы.
      
      Результатом проекта AstroDAbis является прототип сервиса, который, в общем-то, 
является кросс-иден\-ти\-фи\-ка\-ци\-ей нескольких существующих каталогов. Однако он вносит 
новое качество в работу с имеющимися ресурсами. Целевая аудитория этого проекта~--- в 
первую очередь астрономы, которые являются довольно-таки небольшим по количеству 
сообществом, но взаимосвязанную информацию в базах данных можно предоставить и более 
широкой публике посредством API-сер\-ви\-сов, что облегчит будущим разработчикам 
создание удобной системы поиска информации об астрономических объектах для любой 
группы пользователей.
      
      Аналогичные разработки не являются новыми в науке (аннотирование данных в 
генетике~--- {Distributed Annotation System}, {\sf http://www.biodas.org}) или в 
Интернете~--- RDF ({Resource Description Framework})~[41] и LOD ({Linking 
Open Data})~[42]. Сис\-те\-ма AstroDAbis разработана так, чтобы естественным образом 
использовать TAP-factory~[43] на базе OGSA-DAI ({Open Grid Services 
Architecture Data Access Interface})~[44], где TAP ({Table Access Protocol}~[45])~--- 
протокол IVOA для работы с таб\-ли\-ца\-ми. Используя TAP-factory, можно создать 
сервис, который позволит выполнять запросы, об\-ра\-ща\-ющи\-еся к другим сервисам, 
совместимым по протоколу TAP. AstroDAbis также имеет LOD-ин\-тер\-фейс, который 
обеспечивает создание URI для аннотируемых объектов, что подготавливает платформу 
для будущих экспериментов с Семантическим Вебом в астрономии. 
      
      Чтобы найти и получить данные пользователь сам инициирует взаимодействие с 
инфраструктурой виртуальной обсерватории посредством клиентских приложений 
(TOPCAT~[46], ALADIN, DS9\ SAOImage и~др.)\ или веб-ин\-тер\-фей\-сов к базам данных. 
Всякий раз, когда пользователь хочет узнать о возможно уже появившихся обновлениях, ему 
надо повторить первоначальный запрос, сравнить полученный результат с существующим и 
скопировать, если это требуется, данные. Постоянно растущие объемы данных, включающие 
новые релизы существующих обзоров, и публикации новых каталогов требуют другого 
подхода при отслеживании новой информации о небесных объектах, интересных 
пользователю. Особенно это полезно при обновлении и актуализации компилятивных 
каталогов и баз данных. Решение этой задачи предлагается с по\-мощью веб-при\-ло\-же\-ния для 
поддержки данных пользователя \mbox{VOdka} ({VO Data Keeping-up Agent})~[47], который 
ретранслирует запросы пользователей в инфраструктуру виртуальной обсерватории и 
рассылает уведомления об обновлениях. При выбранном пользователем темпе опроса агент 
асинхронно посылает один и тот же запрос, сформулированный пользователем, и фиксирует 
результаты, отражающие временной срез информации, выполняет сравнение этих срезов и 
оповещает пользователя по электронной почте. У~пользователя есть возможность 
просматривать результаты запросов, сохраненные в $snapshot$-фай\-лах, журналы сравнения 
этих файлов, копировать снимки и новые появившиеся данные, а также инкрементальные 
файлы, включающие старые, новые и пропущенные данные.
      
      Во многих областях научных исследований имеется насущная потребность работы с 
большими по объему распределенными массивами данных и выполнения над ними 
разнообразных задач по извлечению знаний. Ита\-ло-аме\-ри\-кан\-ский проект DAME 
({DAta Mining} \& {Exploration})~[48] на\-правлен на создание междисциплинарной 
распределенной среды, специализированной под исследования больших массивов данных 
(MDS, {Massive Data Set}) с\linebreak
 помощью ма\-шин\-но-обуча\-емых алгоритмов и 
методов добычи данных, которая реализована на унифицированной технологической 
платформе. DAME включает несколько проектов по решению разных астрофизических 
задач и может предложить для\linebreak разных e-science сообществ широкий спектр 
вы\-чис\-ли\-тель\-ных мощ\-ностей для применения ма\-шин\-но-обуча\-емых и статистических 
алгоритмов к астрономическим данным. Эти проекты используют\linebreak единую технологическую 
платформу, ба\-зи\-ру\-ющу\-юся на архитектуре сер\-вис\-но-ориен\-ти\-ро\-ван\-ных приложений и 
совместимую со стандартами виртуальной обсерватории. 
  
  Обнаружение знаний в базах данных KDD ({Knowledge Discovery in Data Bases}) 
сейчас связывают с новым семейством научных дисциплин, называемым 
  X-Informatics. Оно считается четвертой парадигмой в науке после теории, 
эксперимента и моделирования. В~таком контексте проект DAME призван: 
  \begin{itemize}
\item обеспечить сообщество расширяемой интегрированной средой для добычи данных 
и исследований на базе технологий Web~2.0;
\item поддерживать стандарты и форматы виртуальной обсерватории для 
интероперабельности приложений;
\item обеспечить виртуальную обсерваторию общей вычислительной платформой, 
использующей современные технологии (грид, облачные вычисления и~т.\,п.). 
\end{itemize}

При происходящем в настоящее время росте сложности данных и необходимости 
проведения исследований с большими массивами данных альянсом IVOA было принято 
решение о создании группы по интересам, связанной с обнаружением знаний в базах данных 
(KDD-IG), которая должна согласовывать стандарты IVOA и потребности научных 
исследований с использованием баз данных.
      
      В этом проекте находятся в стадии разработки несколько научных сценариев, которые 
оформляются в виде веб-при\-ло\-же\-ний, базирующихся на архитектуре системы DAME. Из 
них для исследования интересующей автора выборки радиоисточников наиболее 
привлекательны разработки по оценке фотометрического красного смещения галактик и 
селекции квазаров на основе фотометрических данных, поскольку определение 
спектральных красных смещений требуют больших затрат наблюдательного времени в 
отличие от фотометрических оценок.
      
      Разработки проекта ADSASS ({The ADS All-Sky Survey})~[49] направлены на 
превращение системы NASA ADS ({Astrophysics Data System}), широко 
используемой среди астрономов в качестве полнотекстового библиографического ресурса, в 
карту неба. Система ADS не является источником наблю\-да\-тель\-ных данных, но является 
неявным хранилищем ценной астрономической информацией в форме изображений, таблиц 
и ссылок на небесные объекты, которые являются частью публикации. Необходимо сделать 
эту информацию доступной для запросов и просмотра. Рассматриваются три категории 
данных: 
      \begin{enumerate}[(1)]
\item ссылки на небесные объекты, которые предполагается собрать из внешних баз 
данных и добавить в виде аннотации (astrotag) связь со стать\-ями в ADS. Так же, 
как это сделано в geotags для объектов на земной поверхности, astrotags 
являются пространственными и временными аннотациями для небесных объектов;
\item оптические и изображения в других диапазонах, имеющиеся в стать\-ях, также 
получат связывающие ссылки (astroreference). Так это сделано для геоданных 
(georeferencing), которые ссылаются на карты, имеющие привязку к сис\-те\-ме 
земных координат, ссылки (astroreferencing) свяжут изображения, которые будут 
приведены к одной небесной координатной сис\-те\-ме с учетом ориентации, координатной 
привязки и масштаба пикселов каждого кадра; 
\item другого сорта данные, такие как текст или подписи под рисунками, будут 
привязаны к координатам или имени источника. 
\end{enumerate}

В результате выполнения проекта будет получена карта всего неба, которая будет 
активировать ссылки на статьи, показывая, какая часть неба в них описывается, а также слой 
исторических данных на базе хранилища astroreference-ссы\-лок и изоб\-ра\-же\-ний, 
извлеченных из статей, которые можно использовать для анализа. Для визуализации этой 
информации будут использоваться приложения, в которых можно отображать полностью все 
небо, а именно: WorldWide\ Telescope (Microsoft), ALADIN (CDS), Google\ Sky 
(Google) и~др. Сис\-те\-ма ADSASS будет опираться на постоянно об\-нов\-ля\-емую базу данных 
тегов, которая предназначена как для обнаружения новой информации о небесных объектах 
по любой тематике, так и для поиска событий переменного характера по данным 
исторического слоя. 

\section{Массовое отождествление радиоисточников}

\subsection{Научная мотивация}

  В противоположность начальной стадии своего возникновения Вселенная сегодня богата 
структурами~--- галактиками, скоплениями галактик, сверхскоплениями и пустотами~--- 
войдами. Все эти структуры эволюционируют с гравитационным расширением из небольших 
первоначальных неоднородностей плотности. В~иерархических космогониях первые 
  гра\-ви\-та\-ци\-он\-но-свя\-зан\-ные сис\-те\-мы могли быть звездами и/или небольшими 
звездоформирующими системами, при слиянии которых формируются галактики. 
Возникающие из конечных продуктов звездной эволюции и мерджинга (слияния) 
центральные черные дыры продолжают расти. В~любом случае аккреция, питающая 
массивные черные дыры, проявляет себя как феномен активного галактического ядра 
({Active Galaxy Nuclear}, AGN). Из-за своей экстремальной светимости AGN 
являются подходящими реперами для исследования Вселенной. Хотя почти все AGN 
име\-ют схожие источники энергии, их наблю\-да\-емые свойства сильно различаются. 
К~примеру, одни AGN имеют мощное радиоизлучение, а другие нет. Могут наблюдаться 
еще разные проявления активности ядра~--- широкие эмиссионные линии в оптике, высокая 
степень поляризации оптического излучения, переменность, рентгеновское или 
  гам\-ма-из\-лу\-че\-ние. В~моделях, объясняющих наблюдаемые свойства активных 
галактик, предполагается, что только несколько физических процессов обеспечивают 
наблюдаемый диапазон AGN-ха\-рак\-те\-ри\-стик. Полагают, что разнообразие типов 
AGN возникает из: (1)~отсутствия или наличия пыли вокруг ядра; (2)~направления 
релятивистского джета относительно наблюдателя; (3)~цикла активности; (4)~полной 
светимости галактики. Еще разделение по мощ\-ности радиоизлучения зависит от углового 
момента и массы центральной черной дыры. 
  
  Популяцию мощных радиогалактик с по\-мощью имеющихся радиотелескопов можно 
наблюдать практически на любом расстоянии. Это позволяет изучать их эволюцию в 
радиодиапазоне от момента образования до наших дней. От низких до умеренных красных 
смещений ($Z\sim1$) мощные радиоисточники связывают с гигантскими эл\-лип\-ти\-че\-ски\-ми 
галактиками, поэтому радиогалактики можно использовать для изучения формирования и 
эволюции самых массивных звездных систем, из истории звездообразования которых можно 
получить важные ограничения на модели формирования галактик и космологические 
параметры. Радиоисточники часто ассоциируются с центральными галактиками скоплений, 
поэтому далекие радиогалактики могут быть индикаторами первых протоскоплений. Так 
выглядят в общих чертах те задачи, которые решаются при исследовании радиогалактик.

\subsection{Предметный посредник для~поиска далеких радиогалактик}
 
Известны разные техники селекции объектов для поиска далеких галактик. К~ним относятся: 
глубокая спектроскопия пустых полей, узкополосные снимки, спектроскопия объектов 
вокруг радиогалактик с большим~$Z$, использование показателей цвета (разность звездных 
величин объекта в различных фотометрических фильтрах/полосах) для выбора кандидатов. 
Известно, что спектральное распределение энергии небесных объектов в зависимости от 
красного смещения сдвигается в красную область спектра. Из-за этого галактика может быть 
ярче или существенно слабее в каком-либо фильтре, быть видимой только в одном фильтре 
из-за того, что в эту полосу попадают особенности спектрального распределения объекта~--- 
излучение в водородной линии Лай\-ман-аль\-фа $\lambda=1216$~{\ptb\AA} 
или завал спектра 
на $\lambda=912$~{\ptb\AA}. Чтобы поймать это усиление или, наоборот, ослабление яркости, 
используют ограничения на цветовые индексы, например, следующие: $(u\hm+r)/2-g\hm>1$; 
$(g\hm+i)/2\hm-r \hm>1$ и~т.\,п., где $u$, $g$, $r$, $i$, $z$~--- звездные величины 
оптического объекта в полосах обзора SDSS. Это и есть так называемые 
dropout-тех\-ни\-ки, используемые при отборе кандидатов в далекие галактики. Затем 
для кандидатов проводят спектральные исследования, чтобы определить по смещению 
спектральных линий, действительно ли объект является далеким. Применение этих техник 
привело к обнаружению далеких объектов с $Z\sim6$--7. Однако радиогалактики 
продолжают оставаться интересными для изучения объектами, так как они являются самыми 
массивными звездными системами, во многих случаях указывают на скопления галактик, а 
на космологических расстояниях~--- на протоскопления. 
      
      С появлением оптического обзора неба SDSS и радиообзора FIRST, которые 
обладают надежной координатной привязкой, достаточной глубиной и угловым 
разрешением, а также программных средств виртуальной обсерватории исследования 
природы радиоисточников стало возможным проводить не только по небольшим по числу 
объектов выборкам, но и по любым каталогам/спискам. 
      
      Был предложен научный сценарий поиска далеких галактик по радиоисточникам 
каталога~RC, который использует обзоры FIRST, NVSS и SDSS~[50]. Для каждого 
RC-ис\-точ\-ни\-ка из обзора SDSS выбираются объекты, которые попадают в область, 
размерами равную боксу ошибок определения координат ($\pm 3\sigma$). При средней 
плотности объектов обзора SDSS ($\sim7$--8 объектов на кв.\ угл.\ минуту) в область 
поиска, размеры которой варьируются от 45$^{\prime\prime}$ до 2$^\prime$ в зависимости 
от положения RC-ис\-точ\-ни\-ка относительно центральной части диаграммы 
направленности телескопа, попадают сотни объектов. Поскольку низкая точность координат 
каталога~RC не позволяет выполнить отождествление по позиционному совпадению, 
можно провести дополнительную селекцию в оптике, а именно использовать ограничения 
для разности показателя цвета. И~если оптический объект, попадающий по координатам в 
область поиска, еще и удовлетворяет ограничениям по показателям цвета, то он является 
наиболее вероятным кандидатом для отождествления. 

Для сценария отождествления списка радиоисточников был создан предметный 
посредник~[51], архитектура которого была реализована как объединение системы 
AstroGrid~[52], которая разработана в Великобритании и совместима с протоколами 
IVOA, и средств поддержки предметных посредников, созданных в ИПИ РАН. Для 
прототипа гиб\-рид\-ной архитектуры было выполнено сопряжение исполнительных 
механизмов двух инфраструктур (AstroGrid и предметных посредников). Эта разработка 
выполнялась для решения задач, связанных с разработкой прототипа Российской 
виртуальной обсерватории (РВО)~[28].
      
      Сценарий автоматического отождествления выборки каталога~RC в области, 
пересекающейся с обзорами SDSS и FIRST, разделен на два этапа: подготовка данных и 
визуальная инспекция результатов для принятия решения об отождествлении 
радиоисточника с оптическим кандидатом. 
      
      Поток задач для подготовки данных состоит из следующей последовательности: 
      \begin{enumerate}[(1)]
\item выбор координат радиоисточника из каталога~RC;
\item извлечение списка объектов из области заданного размера из базы данных обзора 
SDSS; 
\item кросс-идентификация результатов запросов с использованием селекционного 
ограничения по цветовым индексам; 
\item извлечение радиоизображений из обзора FIRST;
\item извлечение оптических изображений SDSS;
\item суперпозиция изображений;
\item сохранение результатов запросов. 
\end{enumerate}

Чтобы можно было работать с каталогом~RC из потока задач, он оформлен как компонент 
AstroGrid для доступа к данным~--- DSA ({Data Set Access}). Запрос к базе данных, 
содержащей каталог~RC (шаг~1), выполняется CEA ({Common Execution 
Architecture}) приложением на узле AstroGrid, развернутом в ИПИ РАН. Результат запроса 
в формате VOTable автоматически запоминается в MySpace (виртуальная область памяти 
сис\-те\-мы AstroGrid). Запрос на языке ADQL выглядит следующим образом: 

%\end{multicols}

%\hrule

\noindent
{\small
\begin{verbatim}
SELECT crd.ra, crd.de, cat.name 
FROM RCCatalog as cat, CoordEQJ as crd
WHERE cat.coord_id = crd.coord_id
\end{verbatim}
}

%\hrule

%\begin{multicols}{2}
      
Запрос к каталогу SDSS (шаг~2) выполняется CEA-приложением с помощью веб-сер\-ви\-са, 
работающего на сервере SDSS 
({\sf http://voservices.net/ CasService/ws\_v1\_0/Cas} Service.asmx). Оно запрашивает данные и 
возвращает результат в файл в формате VOTable, сохраняемый в MySpace. Запрос к 
SDSS записывается следующим образом: 

%\end{multicols}

%\hrule

\noindent
{\small
\begin{verbatim}
SELECT ra=cast(ra as real),
dec=cast(dec as real), 
objid, u, g, i, r, z, 
colorIndexURG = (u+r)/2.0-g .GT. 1. 
FROM PhotoPrimary 
WHERE ra BETWEEN 225.0 AND 225.5 AND dec 
BETWEEN 4.0 AND 5.61 AND r BETWEEN 15.0 AND 23.0 
\end{verbatim}
}

%\hrule

%\begin{multicols}{2}

Кросс-идентификация (шаг~3) результатов двух предыдущих шагов выполняется 
веб-сер\-ви\-сом\linebreak
AstroGrid CrossMatchFull ({\sf ivo://org.astrogrid/\linebreak CrossMatcher}) в UK. 
Результатом служит таблица в VOTable формате, которая также помещается в MySpace. 

Извлечение и суперпозиция изображений (шаги~4--6) производится в ИПИ РАН. 
При\-ло\-же\-ние CEA вызывает ALADIN для каждого объекта из каталога RC, 
используя его координаты в качестве центра области. ALADIN извлекает изображения из 
DSS и FIRST, после чего контуры радиоизображения совмещаются с оптическим. 
Дополнительно извлекаются объекты из каталогов SDSS, 2MASS, FIRST, NVSS, 
попавшие в область поиска. Выполняющая эти запросы программа, написанная на языке 
команд ALADIN, показана ниже: 

%\end{multicols}

%\hrule

\noindent
{\small
\begin{verbatim}
get DSS.ESO(DSS1,14.1,14.1), address, 5'; sync; 
/* извлечь изображение из обзора DSS. 
/* Переменная address содержит координаты 
/* в формате `hh:mm:ss sdd:mm:ss'
get NVSS(0.2,15.0,Stokes I, Sine), address, 5'; 
/* извлечь изображение из радиообзора NVSS
sync; contour 4;                                
/* синхронизировать и построить контуры
get FIRST(10), address, 5';                     
/* извлечь изображение из радиообзора FIRST
sync; contour 4;                               
get SDSSDR3cat, address, 1';                    
/* извлечь данные из каталога SDSS
sync; 
get VizieR(2mass), address, 1';                 
/* извлечь данные из каталога 2MASS
sync;
backup st.aj                                    
/* сохранить данные
\end{verbatim}

}

%\hrule

%\begin{multicols}{2}
      
После выполнения программы данные запоминаются (шаг~7) в стеке ALADIN и 
сохраняются в MySpace. Подготовка данных на этом закончена. 

      На втором этапе для просмотра подготовленных данных запускается ALADIN и 
Workbench (клиентское приложение для работы с AstroGrid). Данные, сохраненные в 
MySpace AstroGrid, открываются в ALADIN, проводится визуальная инспекция 
подготовленных данных и принимается решение об отождествлении объекта. 
      
      Подробное описание реализации этого сценария на основе AstroGrid и средств 
поддержки предметных посредников приведено в~[51]. 

\subsection{Оптическое отождествление каталога RC} 

      Подходящих кандидатов в далекие галактики среди источников каталога RC не 
было обнаружено, но была разработана и опробована методика массового отождествления 
радиоисточников. Исследования были продолжены дальше, но уже с другой целью~--- 
отождествление всех источников каталога RC, попадающих в область обзоров SDSS и 
FIRST, и определение типов родительских галактик. 
      
      Сценарий массового отождествления радиоисточников включает следующие этапы: 
(1)~подготовка данных; (2)~предварительная обработка и визуальная инспекция; 
(3)~анализ~--- уточнение координат RC-ра\-дио\-ис\-точ\-ни\-ков; (4)~анализ~--- 
определение морфологических типов радиоисточников; (5)~анализ~--- оптическое 
отождествление. 

\begin{figure*}[b] %fig1
\vspace*{1pt}
 \begin{center}
 \mbox{%
 \epsfxsize=162.053mm
 \epsfbox{zhe-1.eps}
 }
 \end{center}
 \vspace*{-9pt}
\Caption{(\textit{a})~Рисунок, полученный средствами ALADIN. Полутоновое изображение~--- 
данные в полосе $r$ из оптического цифрового обзора неба SDSS; штриховые контуры~--- контурная 
радиокарта из обзора NVSS (угловое разрешение~--- 45$^{\prime\prime}$); сплошные контуры~--- 
радиокарта обзора FIRST с более высоким угловым разрешением (5$^{\prime\prime}$), которая 
позволяет определить детальную структуру радиоисточника; крестиками отмечены данные из 
каталога SDSS, ромбами~--- радиокаталоги. (\textit{б})~Морфологические типы 
радиоисточников: C ({core})~--- точечный; D и DC ({double})~--- двойной; CJ 
({core-jet})~--- ядро с выбросом; CL ({core-lobe})~--- ядро с компонентами; T 
(triple)~--- тройной}
\end{figure*}
      
\textbf{1. Подготовка данных.} Этот этап выполняется так же, как в сценарии поиска 
далеких радиоисточников, но список используемых каталогов существенно расширен. 
Подготовка данных производится автоматически с помощью {perl}-программы, 
которая использует средства программного интерфейса ALADIN. Для каждого источника 
каталога RC по координатам извлекаются изображения из оптических [DSS-II (сервер 
{Space Telescope Science Institute}) и SDSS (сервер {SkyView~--- the Internet's 
Virtual Telescope})] и радиообзоров неба [NVSS, FIRST (сервер {National Radio 
Astronomy Observatory})], данные из оптических (USNO-B1, SDSS) и 
инфракрасных (2MASS) каталогов, а также из радиокаталогов [VLSS, TXS, NVSS, 
FIRST, GB6 и RC (сервер \textit{Vizier})], выполняется суперпозиция изображений 
посредством наложения контуров радиоизображения на оптическое, извлеченные данные 
сохраняются для последующих фаз сценария. Пример программы приведен ниже:

%\end{multicols}

%\hrule

\noindent
{\small
\begin{verbatim}
#!/usr/bin/perl
die "Usage: $myname Catalog\n" unless (@ARGV);
$CATALOG = $ARGV[0];
open (READ_CAT, "<$CATALOG") 
or die ("Cannot open file"); 
\* открытие файла с координатами
open(ALADIN,"| java 
-Dhttp.proxyHost=192.168.2.33 -       
\* запуск ALADIN
Dhttp.proxyPort=8080 
-jar /Data/users/zhe/Aladin/Aladin.jar");
$rcN = 0;
while ($str = readline (*READ_CAT))
{
($pref,$name,$last) = split (/ /, $str, 3);               
\* преобразование координат
$rcname = join('',$pref,$name);
$raJ = substr($last,0,11);
$deJ = substr($last,12,11);
$obj = join (' ',$raJ,$deJ);
$stack = join('','s',$rcname,'.aj');
print ALADIN "reset; \n";                                 
\* передача команд ALADIN
print ALADIN "get Vizier(VIII/42/txs) 
$obj 5\';\n";       
\* извлечь данные из каталога TXS
print ALADIN "get aladin(DSS2,F) $obj 5\';\n";            
\* -"- изображение из обзора DSS
print ALADIN "get NVSS(0.2,15.0,Stokes I,Sine) 
$obj 5\';\n"; 
\* -"- -"-    из радиообзора  NVSS
print ALADIN "sync; contour 4; \n";                       
\* построить контуры радиоизображения
print ALADIN "get Vizier(VIII/65/nvss) $obj 5\'; 
sync;\n";
\* извлечь данные  из каталога  NVSS
print ALADIN "get Vizier(sdss) $obj 1\';\n";              
\* -"- -"-  из каталога  SDSS
print ALADIN "sync \n";
print ALADIN "get VizieR(2mass) $obj 1\';\n";             
\* -"- -"-  из каталога  2MASS
print ALADIN "sync \n";
print ALADIN "get VizieR(USNOB) $obj 1\';\n";             
\* -"- -"-  из каталога  USNO-B1
print ALADIN "sync \n";
print ALADIN "get FIRST(100) $obj 5\';\n";                
\* извлечь изображение из обзора FIRST
print ALADIN "sync; contour 4; \n";                       
\* построить контуры радиоизображения
print ALADIN "get Vizier(VIII/71/first) 
$obj 5\'; sync;\n"; 
\* извлечь данные  из каталога  FIRST
print ALADIN "get Vizier(J/A+AS/87/1/table1) 
$obj 5\'; sync;\n"; 
\*    -"- -"- из каталога  RC
print ALADIN "backup $stack\n";                             
\* сохранение подготовленных данных
};
\end{verbatim}

}

%\hrule

%\begin{multicols}{2}

\begin{figure*}[b] %fig2
\vspace*{6pt}
 \begin{center}
 \mbox{%
 \epsfxsize=165.339mm
 \epsfbox{zhe-2.eps}
 }
 \end{center}
 \vspace*{-9pt}
\Caption{(\textit{а})~Изображение радиоисточника каталога RCR из обзора FIRST. 
(\textit{б})~Составное изображение оп\-ти\-ка--ра\-дио. Контуры, построенные по изображению радиообзора 
FIRST, наложены на цветное RGB-изоб\-ра\-же\-ние, полученное из трех кадров обзора SDSS в 
фотометрических полосах $g$, $r$, $i$, где фильтр~$g$ соответствует~$B$, $r$~--- $G$ и $i$~--- $R$. 
Рисунок получен с помощью python-скрип\-та, использующего программный интерфейс к 
приложению для визуализации и доступа к данным виртуальной обсерватории DS9 SAOImage}
\end{figure*}




\textbf{2. Предварительная обработка и визуальная инспекция.} На этом этапе используется 
макроконтроллер ALADIN. Этот сервис производит интерпретацию двух файлов. Один 
файл содержит скрипт с командами ALADIN, второй~--- данные, которые являются 
параметрами команд. Предварительная обработка производится средствами графического 
интерфейса ALADIN (рис.~1,\,\textit{а}). Она состоит в подборе уровней контуров и диапазона 
отображаемых данных для изображений.

\textbf{3. Уточнение координат.} Поскольку координаты каталога RC грубы для 
оптического отождествления, то требуется уточнение координат по радиообзорам с более 
высокой координатной точностью. Отметим, эта задача легче решается, если у сравниваемых 
каталогов близкое угловое разрешение, при этом необходимо учитывать изменение 
плотности потока источника на разных частотах, а также предельную чувствительность 
каждого каталога.
      
      Идентификация RC-источников проводилась сначала с источниками обзора 
NVSS, поскольку каталог RC имеет близкое к NVSS угловое разрешение по прямому 
восхождению (45$^{\prime\prime}$), а координатная точность NVSS (1$^{\prime\prime}$) 
существенно лучше, чем у RC (15$^{\prime\prime}\times40^{\prime\prime}$).
      
      Перечислим условия в порядке убывания значимости, выполнение которых 
принималось во внимание при отождествлении источника каталога RC с источником 
обзора NVSS:
\begin{itemize}
\item координатное совпадение по прямому восхождению ($r < 3\sigma$, где $\sigma$~--- 
приведенная в каталоге RC ошибка координат по прямому восхождению); 
\item координатное совпадение по склонению; 
\item совпадение плотностей потока для RC-ис\-точ\-ни\-ка и NVSS-ис\-точ\-ни\-ка. Вызывают 
сомнения случаи, когда при координатном совпадении RC-ис\-точ\-ник не согласуется по 
плотности потока с NVSS (при пересчете плотностей потоков полагаем, что спектральный 
индекс источника $\alpha\sim 0.7$, $S(\nu) \sim \nu^{-\alpha}$); 
\item присутствие соседних источников. Если рядом с RC-ис\-точ\-ни\-ком есть не один, а два 
или несколько источников NVSS, которые попадают в диаграмму направленности 
РАТАН-600, то возникает неоднозначная ситуация при идентификации. В~этом случае 
принималось, что наибольший вклад дает самый яркий NVSS-ис\-точ\-ник, с которым и 
отождествлялся RC-источник. 
\item когда плотность потока источника на 3,9~ГГц оказывается больше, чем плотность 
потока на 1,4~ГГц, требуется дополнительная информация, подтверждающая рост 
плотности потока к более высоким частотам. В~этих случаях использовался как каталог, 
так и радиообзор GB6 на 4,85~ГГц. В каталог обычно включаются объекты с плотностью 
потока выше $5\sigma$ уровня отношения сигнал/шум. Источники с плот\-ностью потока на 
уровне $3\sigma\mbox{--}4\sigma$, отсутствующие в каталоге GB6, обнаруживаются при 
визуальной инспекции изображений обзора GB6. Эта дополнительная информация 
помогала при неоднозначных случаях отождествления.
  \end{itemize}
  

  
По такому алгоритму~[17] были отождествлены и уточнены координаты у 75\% источников 
каталога RC, для которых на следующем шаге проводилось оптическое отождествление.

%\vspace*{6pt} 

\textbf{4. Определение морфологического типа ра\-дио\-ис\-точ\-ни\-ка.} Корректность оптического 
отождествления радиоисточника зависит от правильного определения его морфологического 
типа, поскольку есть связь между типом и предполагаемым положением родительской 
галактики. Для этого использовался обзор FIRST, где из-за более высокого углового 
разрешения имеется более подробная информация о структуре источника. 
      
      Было использовано 5 морфологических типов радиоисточников (рис.~1,\,\textit{б}): 
точечные ({core}), двойные ({double}, {double-core}, {double-double}), 
тройные ({triple}), ядро с джетом ({core-jet}), ядро с компонентами 
      ({core-lobe}). Тип радиоисточника определялся по радиоизображениям (рис.~2,\,\textit{а}) 
      и дополнительно по картам с контурами интенсивности (рис.~2,\,\textit{б}), если 
структуру сложно классифицировать.


%\vspace*{6pt}

\textbf{5. Определение оптического кандидата.} После определения морфологического типа 
радиоисточника определялось предполагаемое положение родительской галактики. Именно 
эти координаты затем использовались при выборе оптического кандидата. Оптический 
объект считался надежным кандидатом на отождествление, если его положение (по каталогу 
SDSS) было не дальше $3\sigma$ от предполагаемого положения родительской галактики, 
где $\sigma$~--- ошибка координат. К~возможным отождествлениям отнесены следующие 
случаи:
\begin{itemize}
\item источник точечный или двойной с ядром, а оптический объект расположен дальше 
чем $3\sigma$ от центра радиоисточника; 
\item два оптических объекта рядом с предполагаемым положением оптического кандидата, 
и по имеющейся фотометрической и спектральной информации нельзя сделать уверенного 
выбора между объектами; 
\item источник двойной, положение ядра определяется неуверенно, оптический объект 
сдвинут в сторону от линии, соединяющей максимумы плотностей потока компонентов;
\item сложно сделать выводы о структуре радиоисточника по радиокарте FIRST.
\end{itemize}

Надежные оптические кандидаты были найде\-ны для 70\% радиоисточников, для 10\% 
радиоисточников имеются возможные кандидаты, а для 20\% не обнаружены оптические 
объекты, так как родительские галактики радиоисточников слабее предельной глубины 
обзора SDSS. Для 75\% оптических кандидатов было проведено разделение на галактики и 
звездные объекты~[18]. Результаты отождествления представлены на {\sf 
http://www.sao.ru/fetch/cgi-bin/SkyObj/rc.cgi}.

\subsection{Оптическое отождествление каталога RCR}
      
      Каталог RCR (RATAN Cold Refined)~[5] получен в результате обработки 
7~циклов наблюдений, проведенных на радиотелескопе РАТАН-600 с 1987 по 1999~гг., и 
повторной обработки данных 1980--1981~гг.\ для интервала прямых восхождений $7^h \leq 
\alpha_{2000}\leq 17^h$ с целью улучшения координат и уточнения плотностей потоков 
источников каталога RC. Отметим, что в результате отождествления каталога RC 
$\sim25\%$ из 432~объектов, попавших в область обзоров FIRST и SDSS, не 
отождествились с источниками NVSS. Собственно говоря, это и послужило толчком для 
подготовки следующего релиза каталога RC~--- каталога RCR, по которому работа по 
отождествлению радиоисточников обзоров <<Холод>> 1980--1999~гг.\ была продолжена. 
В~каталоге RCR 550 источников, что в 1,7~раза больше, чем количество RC-ис\-точ\-ни\-ков, 
для которых уже проводилось оптическое отождествление.
      
      Обычно радиоисточники отождествляются с доста\-точ\-но слабыми объектами в оптике. 
Чем слабее по потоку радиоисточники, тем более глубокие снимки в оптике необходимы для 
обнаружения объекта, ответственного за радиоизлучение. При предельной глубине обзора 
SDSS ~22.6 звездной величины по оценкам можно отождествить $\sim30\%$--50\% 
радиоисточников с плотностью потока ярче 1~мЯн на 1,4~ГГц. В~каталоге RCR источники 
ярче~--- слабые объекты имеют плотность потока на частоте 3,9~ГГц 10--15~мЯн, что при 
пересчете на 1,4~ГГц будет соответствовать 17--25~мЯн. Так по каталогу SDSS удалось 
отождествить 70\% RC-ис\-точ\-ни\-ков. Чтобы отождествить оставшиеся объекты, нужны 
были более глубокие снимки, которые можно получить, сложив изображения в трех 
фильтрах обзора SDSS~--- $g$, $r$ и $i$, а также используя обзоры инфракрасного 
диапазона. Чем больше диапазонов электромагнитного спектра привлекается для 
исследования радиоисточников, тем больше информации для определения типа 
родительской галактики, а также ее физических характеристик. Поэтому для исследования 
радиоисточников каталога RCR был добавлен глубокий обзор неба в ближнем 
инфракрасном диапазоне UKIDSS и в среднем инфракрасном диапазоне~--- WISE. 
      
      Сценарий для отождествления источников каталога RCR включает: (1)~подготовку 
данных; (2)~визуализацию и предварительную обработку; (3)~определение морфологического 
типа радиоисточника; (4)~оптическое отождествление. Эти этапы мало отличаются по 
методике от описанных в предыдущем подразделе шагов. 

      \begin{table*}[b]\small
      \begin{center}
      \Caption{Каталоги и обзоры, использовавшиеся для отождествления радиоисточников 
каталога RCR}
      \vspace*{2ex}
      
\tabcolsep=7pt
      \begin{tabular}{|l|c|c|c|c|}
      \hline
\multicolumn{1}{|c|}{Диапазон} &\tabcolsep=0pt\begin{tabular}{c}Каталоги,\\ обзоры\end{tabular}&
\tabcolsep=0pt\begin{tabular}{c}  Спектральный\\ диапазон\end{tabular}&
\tabcolsep=0pt\begin{tabular}{c}Разрешение/\\ ошибки\end{tabular}&
\tabcolsep=0pt\begin{tabular}{c}Предел \\ чувствительности\end{tabular}\\
\hline
Радио&VLSS&\hphantom{99}74 МГц&80$^{\prime\prime}$&500 мДжанки\\
&TXS&\hphantom{9}365 MГц&$\sim10^{\prime\prime}$\hphantom{99}&150 мДжанки\\
&NVSS&1400 MГц&45$^{\prime\prime}$&2.5 мДжанки\\
&FIRST&1400 MГц&\hphantom{999}5.4$^{\prime\prime}$&1 мДжанки\\
&GB6&4850 МГц&\hphantom{999}3.5$^{\prime\prime}$&28--37 мДжанки\\
\hline
Оптика&DSS-II&
\tabcolsep=0pt\begin{tabular}{c}blue, red, IR\\ ($J$, $F$, $N$)\end{tabular}&&
\tabcolsep=0pt\begin{tabular}{c}$\sim21^m$\\ ($\sum \mathrm{BRI} \sim 21.2^m {R}$)\end{tabular}\\
\cline{2-5}
&SDSS&
\tabcolsep=0pt\begin{tabular}{c}$u$, $g$, $r$, $i$, $z$\\ ($g + r + i$)\end{tabular}&
$\pm0.1^{\prime\prime}$&
\tabcolsep=0pt\begin{tabular}{c}22.0$^m$, 22.2$^m$, 22.2$^m$, 21.3$^m$,\\ 
20.5$^m$\\
($\sum \mathrm{gri} \sim 22.6^m$)\end{tabular}\\
\cline{2-5}
&&&&\\[-9pt]
&USNO-B1&
\tabcolsep=0pt\begin{tabular}{c}$B_1$, $R_1$, $B_2$, $R_2$, $I$\\ ($O$, $E$, $J$, $F$, $N$)\end{tabular}&
\tabcolsep=0pt\begin{tabular}{c}0.2$^{\prime\prime}$\\ 0.3$^m$\end{tabular}&$V =21^m$\\
\cline{2-5}
&&&&\\[-9pt]
&GSC 2.3.2&
\tabcolsep=0pt\begin{tabular}{c}$J$, $F$, $N$\\ ($B_J$, $R_F$, $I_N$)\end{tabular}&
\tabcolsep=0pt\begin{tabular}{c}0.2$^{\prime\prime}$--0.28$^{\prime\prime}$\\0.13$^m$--0.22$^m$\end{tabular}&
$R_F=20^m$\\
\hline
ИК&2MASS&$J$, $H$, $K$&
\tabcolsep=0pt\begin{tabular}{c}0.2$^{\prime\prime}$\\ 10\%\end{tabular}&
15.8$^m$, 15.1$^m$, 14.3$^m$\\
\cline{2-5}
&UKIDSS&
\tabcolsep=0pt\begin{tabular}{c}$Y$ (1.02~мкм), J\\ 
(1.25~мкм), $H$ (1.63~мкм),\\ $K$ (2.2~мкм)\\ ($H + K$)\end{tabular}&
$<0.1^{\prime\prime}$&
\tabcolsep=0pt\begin{tabular}{c}20.5$^m$, 20.0$^m$,\\ 18.8$^m$, 18.4$^m$ \\
($\sum HK \sim 22.8^m$)\end{tabular}\\
\cline{2-5}
&&&&\\[-9pt]
&WISE&
\tabcolsep=0pt\begin{tabular}{c}3.4~мкм, 4.6~мкм,\\12~мкм, 22~мкм\end{tabular}&
\tabcolsep=0pt\begin{tabular}{c}6.1$^{\prime\prime}$, 6.4$^{\prime\prime}$,\\
6.5$^{\prime\prime}$, 12$^{\prime\prime}$\end{tabular}&
\tabcolsep=0pt\begin{tabular}{c}16.5$^m$, 15.5$^m$,\\ 11.2$^m$, 7.9$^m$\end{tabular}\\
\hline
 \end{tabular}
\end{center}
\end{table*}
      
      Был расширен список каталогов и обзоров, а также использованы появившиеся в 6-й 
версии \mbox{ALADIN} возможности макроконтроллера по выполнению арифметических 
операций с изображениями. Ниже приведен пример скрипта для макроконтроллера с 
суммированием изображений:

%\end{multicols}

%\hrule

{\noindent
\small
      \begin{verbatim}
G = get Skyview(300,Default,"SDSS G",Tan,J2000)
$1 $2 
R = get Skyview(300,Default,"SDSS R",Tan,J2000) 
$1 $2 
\* извлечение изображений SDSS в фильтрах g, r, i 
I = get Skyview(300,Default,"SDSS I",Tan,J2000) 
$1 $2 
#
R_n = norm -cut R 
I_n = norm -cut I                                     
\* нормализация изображений
G_n = norm -cut G
sync 
RI = R_n + I_n 
GRI = RI + G_n                                        
\* суммирование изображений 
sync
\end{verbatim}

}

%\hrule

%\begin{multicols}{2}

С помощью программного интерфейса к приложению DS9\ SAOImage была реализована 
программа на языке Python для построения рисунков в формате postscript по списку 
радиоисточников (рис.~2,\,\textit{б}). Рисунки использовались для морфологической 
классификации радиоисточников.



В результате были получены надежные отождествления для 82\% радиоисточников, для 10\% 
обнаружены возможные кандидаты, а для 8\% так и не были найдены оптические и/или 
инфракрасные объекты.

\section{Информационно-поисковая система по~результатам отождествления 
радиоисточников каталога RCR}

      По результатам отождествления радиоисточников каталога RCR автором накоплен 
материал,\linebreak который предполагается использовать для даль\-нейших исследований, а именно: 
определения физических характеристик родительских галактик радиоисточников, их 
классификации, подготовки\linebreak
 выборок источников со схожими свойствами, поиска далеких 
объектов, поиска переменности в оптическом и радиодиапазоне. В~табл.~1 приведены 
используемые при отождествлении информационные ресурсы, указаны их основные 
характеристики и спектральные полосы. 


      
      Для работы с компилятивным каталогом разработана информационная система, 
включающая базу данных по радиоисточникам и их родительским галактикам и 
      веб-ин\-тер\-фейс для отображения разнообразной информации об объектах. Схема 
таб\-лиц ин\-фор\-ма\-ци\-он\-но-по\-иско\-вой сис\-те\-мы включает таблицы оригинальных 
каталогов (в опи\-сы\-ва\-емом случае это 12~каталогов), компилятивные таб\-ли\-цы, вклю\-ча\-ющие 
материал по результатам отож\-де\-ст\-вле\-ния радиоисточников. В~схему включены 
представления: 
      \begin{itemize}
\item v\_rcflux~--- блеск объекта в радио-, инфракрасном и оптическом диапазоне; 
\item v\_rcrparamr~--- параметры радиоисточника; 
\item v\_rcrparamo~--- оптические параметры родительской галактики.
\end{itemize}



Веб-ин\-тер\-фейс (рис.~3) позволяет отображать по имени радиоисточника весь материал, 
относящийся к объекту~--- изображения, данные каталогов и вычисленные па\-ра\-мет\-ры. 
Адрес ре-\linebreak\vspace*{-12pt}

\pagebreak

\end{multicols}

\begin{figure*} %fig3
\vspace*{1pt}
 \begin{center}
 \mbox{%
 \epsfxsize=161.754mm
 \epsfbox{zhe-3.eps}
 }
 \end{center}
 \vspace*{-9pt}
\Caption{Веб-интерфейс информационно-поисковой системы с результатами отождествления 
радиоисточников каталога~RCR}
\vspace*{6pt}
\end{figure*}

\begin{multicols}{2}

\noindent
сурса~--- {\sf http://www.sao.ru/fetch/cgi-bin/SkyObj/\linebreak rcrn.cgi}. Интерфейс реализован 
в архитектуре\linebreak <<клиент\,--\,сервер приложений\,--\,сервер СУБД>>. В~качестве клиента 
используется браузер, сервер приложений~--- \mbox{Apache}, сервер базы данных~--- 
\mbox{PostgreSQL}. При написании скриптов использовался Python со стандартными пакетами 
для поддерж\-ки CGI-интер\-фей\-са, графическая биб\-лио\-те\-ка PIL и модуль \mbox{PyGreSQL}
для связи с СУБД по DBD/DBI-интер\-фейсам.

%\pagebreak




%\begin{multicols}{2}



      Информация, которая представлена в веб-ин\-тер\-фей\-се, разделяется на следующие 
части: графическая статическая (подготовленные предварительно рисунки), графическая 
динамическая\linebreak
 (веб-сер\-ви\-сы извлекают на лету изображение из обзора и помещают рисунок 
на страничку) и параметры радиоисточника, хранящиеся в 
      ин\-фор\-ма\-ци\-он\-но-по\-иско\-вой сис\-теме. 
      
      Динамически выполняется построение спектрального распределения энергии 
радиоисточника (колонка <<Radio-IR-optics spectrum>>). Если источник 
отождествлен или есть возможный кандидат, то спектр строится по данным радио-, 
оптического и инфракрасного (ИК) диапазонов. Поскольку в радиодиапазоне плотность потока от 
объекта измеряется в янских на заданной частоте, а блеск в оптическом и ИК диа\-па\-зо\-нах 
измеряется в звездных величинах в полосе длин волн (ангстремы или нанометры), то все 
величины пересчитываются в звездные величины в фотометрической системе AB~[53]. На 
графике со спектральным распределением энергии объекта по оси абсцисс отложена 
величина десятичного логарифма частоты в герцах, а по оси ординат~--- звездная величина. 
Точки на графике обозначены разными цветами. Каждому цвету соответствуют данные 
определенного каталога. Поскольку ширина полос в оптическом и инфракрасном диапазоне 
обычно несколько сотен ангстрем, то звездная величина приписывается эффективной длине 
волны фильтра. 
      
      В радиопараметрах приводятся координаты центра источника, морфологический тип, 
угловые размеры источника в угловых секундах, число компонент в обзоре FIRST и 
спектральные индексы для радиодиапазона. В~оптических параметрах приведены 
координаты оптического кандидата, разница оптических и радиокоординат, тип оптического 
объекта и~др. В~третьей колонке приведены звездные величины в сис\-те\-ме AB и величины 
из каталогов. 
      
      Разработанная ин\-фор\-ма\-ци\-он\-но-по\-иско\-вая сис\-те\-ма упростила просмотр 
разнородных данных по радиоисточникам и использовалась при принятии решения об 
отождествлении.
  
\section{Заключение}

      Виртуальная обсерватория, объединяющая астрономические данные в 
распределенную инфраструктуру, за несколько лет своего существования обеспечила новый 
качественный уровень работы с цифровыми коллекциями. В~большинстве публикуемых 
статей по разным областям исследований в астрофизике присутствуют данные, которые 
получены с применением средств и ресурсов виртуальной обсерватории. 
      
      В течение ряда лет автором проводилось массовое изучение радиоисточников с 
применением разных приложений виртуальной обсерватории и накоплен как 
положительный, так и отрицательный опыт работы с многочастотными данными. 
Эффективность оптического отождествления радиоисточников несравненно выросла, как 
только появились глубокие цифровые обзоры неба и средства для работы с ними. То, на что 
раньше требовались годы, сейчас выполняется за обозримое время, однако идентификация 
списка объектов с каталогами остается трудоемкой. Когда появляются новые обзоры и 
релизы существующих обзоров, выборку информации из каталогов и обзоров приходится 
повторять, накапливая тем самым варианты компилятивных таблиц. Эффективность 
автоматической кросс-иден\-ти\-фи\-ка\-ции по координатам радиокаталогов с оптическими 
в худших случаях составляет всего несколько процентов, в лучших~--- около 30\%. Причем в 
одних случаях для установления связи между объектами каталогов и/или обзоров 
      кросс-иден\-ти\-фи\-ка\-ции достаточно, а в других случаях нужен более глубокий 
анализ с привлечением любых доступных данных, включая и информацию из 
опубликованных статей. 
      
      Установление связи между записями разных каталогов по смысловому содержанию 
является знанием, которое нужно сохранять и поддерживать. Проекты AstroDAbis и 
ADSASS нацелены на решение этой задачи, что послужит дальнейшему развитию 
виртуальной обсерватории как универсального инструмента исследований.
     
{\small\frenchspacing
{%\baselineskip=10.8pt
\addcontentsline{toc}{section}{Литература}
\begin{thebibliography}{99}

%\bibitem{1-zh}
%\Au{Miley G., De Breuck C.} Distant radio galaxies and their environments~// The Astronomy and 
%Astrophysics Review, 2008. Vol.~15. P.~67--144.
\bibitem{2-zh}
\Au{Parijskij Yu.\,N., Bursov~N.\,N., Lipovka~N.\,M., Soboleva~N.\,S., Temirova~A.\,V.} 
The  RATAN-600 7.6-cm catalog of radio sources from `Experiment Cold-80'~// Astronomy 
Astrophys. Supplement Ser., 1991. Vol.~87. P.~1--32.
\bibitem{3-zh}
\Au{Parijskij Yu.\,N., Soboleva~N.\,S., Goss~W.\,M., Kopylov~A.\,I., Verkhodanov~O.\,V., 
Temirova~A.\,V., Zhelenkova O.\,P.} The RATAN-600\,--\,VLA\,--\,6~m Russian telescope: 
Early Universe Project~// 
175th Symposium of the International Astronomical Union.~---  Bologna: Kluwer Acad. 
Publs., 1996. P.~591--602.
\bibitem{4-zh}
\Au{Verkhodanov O.\,V., Parijskij Yu.\,N., Soboleva~N.\,S., Temirova~A.\,V., 
Zhelenkova~O.\,P.} Color  redshifts and the age of the stellar population of distant RC 
radio galaxies~// Astronomy  Rep., 2002. Vol.~46. P.~531--542.
\bibitem{5-zh}
\Au{Parijskij Yu.\,N., Kopylov~A.\,I., Temirova~A.\,V., Soboleva~N.\,S., Zhelenkova~O.\,P., 
Verkhodanov~O.\,V., Goss~W.\,M., Fatkhullin~T.\,A.} 
Spectroscopy of ``Big Trio'' objects using the 
``Scorpio'' spectrograph of the 6-m telescope of the Special Astrophysical Observatory~// 
Astronomy Rep., 2010. Vol.~54. P.~675--695.
\bibitem{6-zh}
\Au{Soboleva N.\,S., Majorova~E.\,K., Zhelenkova~O.\,P., Temirova~A.\,V., Bursov~N.\,N.}
RATAN-600 7.6-cm deep sky strip surveys at the declination of the SS433 source during the 
1980--1999  period. Data reduction and the catalog of radio sources in the right-ascension 
interval  $7h < R.A. < 17h$~// Astrophys. Bull., 2010. Vol.~65. P.~42--59.
\bibitem{7-zh}
\Au{Abazajian K.\,N., Adelman-McCarthy~J.\,K., Ag$\ddot{\mbox{u}}$eros M.\,A.,
\textit{et al}.} The seventh data release 
of the sloan digital sky survey~// Astrophys. J. Suppl., 2009. Vol.~182. P.~543--558.
\bibitem{8-zh}
\Au{Dye S., Warren S.\,J., Hambly~N.\,C., %Cross~N.\,J.\,G., Hodgkin~S.\,T., Irwin~M.\,J., Lawrence~A., 
\textit{et al}.} 
The UKIRT infrared deep sky survey early data release~// Monthly Notices of the Royal 
Astronomical Society, 2006. Vol.~372. P.~1227--1252.
\bibitem{9-zh} 
\Au{Lasker B.\,M., Lattanzi~M.\,G., McLean~B.\,J., %Bucciarelli~B., 
\textit{et al}.}
The Second-Generation 
Guide Star Catalog: Description and properties~// Astronomical J., 2008. Vol.~136. P.~735--766.
\bibitem{10-zh}
\Au{Monet D.\,G., Levine S.\,E., Canzian~B., %Ables~H.\,D., Bird~A.\,R., 
\textit{et al}.} 
The USNO-B Catalog~// 
Astronomical J.~, 2003. Vol.~125. P.~984--993.
\bibitem{411zh}
\Au{Skrutskie M.\,F., Cutri~R.\,M., Stiening~R., %Weinberg~M.\,D., Schneider~S., Carpenter~J.\,M., 
\textit{et al}.} The two micron all sky survey (2MASS)~// 
Astronomical J., 2006. Vol.~131. P.~1163--1183. 
\bibitem{12-zh}
\Au{Cohen A.\,S., Lane W.\,M., Cotton~W.\,D., %Kassim~N.\,E., Lazio~T.\,J.\,W., 
\textit{et al}.}
The VLA low-frequency sky survey~// Astronomical J., 2007. Vol.~134. P.~1245--1262.
\bibitem{13-zh}  
\Au{Douglas J.\,N., Bash F.\,N., Bozyan~F.\,A., Torrence~G.\,W., Wolfe~C.}
The Texas survey of 
radio sources covering $-35.5 < \delta < 71.5$ at 365~MHz~// Astronomical J., 1996. Vol.~111. 
P.~1945--1963. 
\bibitem{14-zh}
\Au{Condon J.\,J., Cotton W.\,D., Greisen~E.\,W., Yin~Q.\,F., Perley~R.\,A., Taylor~G.\,B., 
Broderick~J.\,J.} The NRAO VLA sky survey~// Astronomical J., 1998. Vol.~115. P.~1693--1716. 
\bibitem{15-zh}  
\Au{Becker R.\,H., Helfand D.\,J., White~R.\,L., Gregg~M.\,D.,
 Laurent-Muehleisen~S.\,A.} The FIRST 
Survey Catalog, Version 2003Apr11~// Astrophysical J., 1997. Vol.~475. P.~479--493. 
\bibitem{16-zh} 
\Au{Gregory P.\,C., Scott W.\,K., Douglas~K., Condon~J.\,J.}
The GB6 catalog of radio sources~// 
Astrophys. J. Suppl., 1996. Vol.~103. P.~427--432. 
\bibitem{17-zh}
\Au{Zhelenkova O.\,P., Kopylov~A., Chernenkov~V.} The investigation of the RC catalog 
radiosources in the SDSS and FIRST crossing area with IVOA program tools~// \mbox{JENAM}. Our 
non-stable universe.~--- Yerevan: BAO, 2007. P.~84--85.
\bibitem{18-zh}
\Au{Zhelenkova O.\,P., Kopylov A.\,I.}
Analysis of the RC catalog sample in the region overlapping 
with the regions of the FIRST and SDSS surveys: I.~Identification of sources with the VLSS, TXS, 
NVSS, FIRST, and GB6 catalogs~// Astrophys. Bull., 2008. Vol.~63. P.~346--356.
\bibitem{19-zh}  
\Au{Zhelenkova O.\,P., Kopylov A.\,I.} Analysis of a sample of RC catalog objects in the region 
overlapping with the areas covered by FIRST and SDSS surveys. II:~Optical identification with the 
SDSS survey and USNO-B1 and 2MASS catalogs~// Astrophys. Bull., 2009. Vol.~64. P.~111--122. 

\bibitem{21-zh} 
\Au{Желенкова О.\,П., Майорова Е.\,К., Соболева~Н.\,С., Темирова~А.\,В.} Многочастотное 
исследование радиоисточников средствами виртуальной обсерватории~// Радиотелескопы, 
аппаратура и методы радиоастрономии: Мат-лы Всеросс. радиоастрономической 
конф. (ВРК-2011).~---  СПб.: ИПА РАН, 2011. С.~179--184. 

\bibitem{20-zh}
\Au{Zhelenkova O.\,P., Soboleva N.\,S., Majorova~E.\,K., Temirova~A.\,V.} Multiband study of 
radiosuorces of the RCR catalogue with the virtual observatory tools~// Baltic Astronomy, 2012. 
Vol.~21. P.~5--13. 

\bibitem{22-zh}
\Au{Желенкова О.\,П., Майорова Е.\,К., Соболева~Н.\,С., Темирова~А.\,В.}
Методы виртуальной 
обсерватории в задаче оптического отождествления радиоисточников~// Электронные 
библиотеки, 2010. Т.~13. Вып.~4. 
\bibitem{23-zh} 
\Au{Желенкова О.\,П.} Исследование радиоисточников средствами виртуальной 
обсерватории~// Электронные библиотеки: перспективные методы и технологии, 
электронные коллекции (RCDL'2011): Труды \mbox{XIII} Всеросс. науч. конф.~--- 
Воронеж: ВГУ, 2011. С.~326--333. 
\bibitem{24-zh}
\Au{Boch T., Oberto A., Fernique~P., Bonnarel~F.}
Aladin: An open source all-sky browser~// 
Astronomical Data Analysis Software and Systems XX.~--- Boston: ASP, 2011. Vol.~442. P.~683--691. 
\bibitem{25-zh}
\Au{Joye W.\,A.}
New features of SAOImage DS9~// Astronomical Data Analysis Software and 
Systems XV.~--- San Lorenzo de El Escorial: ASP, 2006. Vol.~351. P.~574--579. 
\bibitem{26-zh}
\Au{Brunner R.\,J., Djorgovski S.\,G., Lonsdale~C., Madore~B., Prince~T., Szalay~A.\,S.}
 Multi-wavelength cross-identification of the extragalactic sky: An NVO cornerstone~// Bull. 
Amer. Astronomical Soc., 2000. Vol.~32. P.~1601--1605.
\bibitem{27-zh} 
\Au{Quinn P.\,J., Benvenuti~P., Diamond~P.\,J., Genova~F., Lawrence~A., Mellier~Y.}
Astrophysical 
virtual observatory (AVO): A progress report~// SPIE Proceedings, 2002. Vol.~4846. P.~1--5.
\bibitem{28-zh}
\Au{Quinn P.\,J., Barnes D.\,G., Csabai~I., %Cui~Ch., Genova~F., Hanisch~R., Kembhavi~A.,
\textit{et al.}} 
The International Virtual Observatory Alliance: Recent technical developments and the road ahead~// 
SPIE Proceedings, 2004. Vol.~5493. P.~137--145.
\bibitem{29-zh}
\Au{Briukhov D.\,O., Kalinichenko L.\,A., Zakharov~V.\,N., Panchuk~V.\,E., Vitkovsky~V.\,V., 
Zhelenkova~O.\,P., Dluzhnevskaya~O.\,B., Malkov~O.\,Yu., Kovaleva~D.\,A.}
{Information infrastructure 
of the Russian Virtual Observatory (RVO)}.~--- 2nd ed.~--- M.: IPI RAS, 2005.
\bibitem{30-zh}
The open archives initiative protocol for metadata harvesting. 
{\sf http://www.openarchives.org/OAI/\linebreak openarchivesprotocol.htm}.
\bibitem{31-zh}
Dublin core metadata initiative. {\sf http://dublincore.org/ documents}.
\bibitem{32-zh}
\Au{Pence W.\,D., Chiappetti L., Page~C.\,G., Shaw~R.\,A., Stobie~E.}
Definition of the flexible 
image transport system (FITS), version 3.0~// Astronomy Astrophys., 2010. Vol.~524. P.~42--82.
\bibitem{33-zh} 
\Au{Ochsenbein F., Williams R., Davenhall~C., %Durand~D., Fernique~P., 
\textit{et al}.}
 IVOA  recommendation: VOTable format definition Version~1.2~// E-print, 2011. Arxiv:1110.0524. 
 P.~1--35.
\bibitem{34-zh}
\Au{Derriere S., Gray N., Mann~R., Martinez~A.\,P., McDowell~J., McGlynn~T., 
Ochsenbein~F., Osuna~P., Rixon~G., Williams~R.} An IVOA standard for unified content 
descriptors. Version~1.1. {\sf http://www.ivoa.net/ Documents/REC/UCD/UCD-20050812.pdf}.
\bibitem{35-zh}
IVOA data access layer. {\sf http://www.ivoa.net/cgi-bin/\linebreak twiki/bin/view/IVOA/IvoaDAL}.
\bibitem{36-zh} 
\Au{Ortiz I., Lusted J., Dowler~P., %Szalay~A., %Shirasaki~Y., Nieto-Santisteban~M.\,A.,
\textit{et al}.} IVOA 
recommendation: IVOA astronomical data query language Version~2.00~// E-print, 2011. 
Arxiv:1110.0503. P.~1--36.
\bibitem{37-zh}
\Au{Ochsenbein F., Bauer P., Marcout~J.}
The VizieR database of astronomical catalogues~// 
Astronomy Astrophys. Suppl., 2000. Vol.~143. P.~23--32.
\bibitem{38-zh}
\Au{Hambly N.\,C., Collins R.\,S., Cross~N.\,J.\,G., %Mann~R.\,G., Read~M.\,A.,
\textit{et al}.} The WFCAM 
science archive~// Monthly Notices Roy. Astronomical Soc., 2008. Vol.~384. P.~637--662.
\bibitem{39-zh}
\Au{Berriman G.\,B.} The NASA/IPAC Infrared Science Archive (IRSA) as a resource in 
supporting observatory operations~//  SPIE Proceedings, 2008. Vol.~7016. P.~701618--701618-9.
\bibitem{40-zh}
\Au{Kamp I., Thompson R., Conti~A., %Fraquelli~D., Kimball~T., Levay~K., Shiao~B.,
\textit{et al}.} MAST 
in the context of VO activities~// Astronomical Data Analysis Software and Systems XIV.~--- 
Pasadena: ASP, 2005. Vol.~347. P.~208--212.
\bibitem{41-zh}
\Au{Gray N., Mann R.\,G., Morris~D., Holliman~M., Noddle~K.}
AstroDAbis: Annotations and 
cross-matches for remote catalogues~// E-print, 2011. ArXiv:1111.6116. P.~1--4. 
\bibitem{42-zh}
Resource description framework. {\sf http://www.w3.org/\linebreak standards/techs/rdf}.
\bibitem{43-zh}
Linked data. {\sf http://www.w3.org/DesignIssues/\linebreak LinkedData.html}.
\bibitem{44-zh}
\Au{Holliman M., Alemu~T., Hume~A., van Hemert~J., Mann~R.\,G., Noddle~K., Valkonen~L.} 
Service infrastructure for cross-matching distributed datasets using OGSA-DAI and TAP~// 
Astronomical Data Analysis Software and Systems XX.~--- Boston: ASP, 2011. Vol.~442. P.~579--583.
\bibitem{45-zh} 
 OGSA-DAI. {\sf http://www.ogsadai.org.uk/about/\linebreak index.php}.
\bibitem{46-zh}
\Au{Dowler P., Rixon G., Tody~D.} Table access protocol (TAP, v1.0), IVOA recommendation~// 
ArXiv:1110.0497. {\sf http://www.ivoa.net/Documents/TAP}. 
\bibitem{47-zh}
\Au{Taylor M.\,B.}
TOPCAT \& STIL: Starlink Table/VOTable Processing Software~// 
Astronomical Data Analysis Software and Systems XIV.~--- Pasadena: ASP, 2005. Vol.~347. P.~29--32.
\bibitem{48-zh} 
\Au{Laurino O., Smareglia R.} 
VOdka: A~data keeping-up agent for the virtual observatory~// 
Astronomical Data Analysis Software and Systems XX.~--- Boston: ASP, 2011. Vol.~442. P.~571--574.
\bibitem{49-zh}
\Au{Brescia M., Longo G., Djorgovski~G.\,S., %Cavuoti~S., D'Abrusco~R., Donalek~C.,
\textit{et al}.} DAME: 
A~web oriented infrastructure for scientific data mining \& exploration~// E-print, 2011. 
Arxiv:1111.3983. P.~1--16.
\bibitem{50-zh}
\Au{Pepe A., Goodman A., Muench~A.} The ADS all-sky survey~// E-print, 2011. 
Arxiv:1111.6116. P.~1--4.
\bibitem{51-zh}
\Au{Zhelenkova O., Vitkovskij V.\,V., Briukhov~D., Kalinichenko~L.\,A.}
Search of distant radio 
galaxies as a subject mediator example~// Astronomical Data Analysis Software and Systems XV.~---
San Francisco: ASP, 2006. Vol.~351. P.~244--249. 
\bibitem{52-zh}
\Au{Брюхов Д.\,О., Вовченко А.\,Е., Захаров~В.\,Н., Желен\-кова~О.\,П., Калиниченко~Л.\,А., 
Мартынов~Д.\,О., Скворцов~Н.\,А., Ступников~С.\,А.} Архитектура промежуточного слоя 
предметных посредников для \mbox{решения} задач над множеством неоднородных распределенных 
информационных ресурсов в гибридной грид-инфра\-струк\-ту\-ре виртуальных обсерваторий~//\linebreak 
Информатика и её применения, 2008. Т.~2. Вып.~1. С.~2--34.
\bibitem{53-zh}
\Au{Walton N.\,A.}
The AstroGrid Consortium. The AstroGrid Virtual Observatory Service~// 
Astronomical Data Analysis Software and Systems XVII.~--- London: ASP, 2008. Vol.~394. P.~251--255.

\label{end\stat}

\bibitem{54-zh}
\Au{Oke J.\,B., Gunn J.\,E.} Secondary standard stars for absolute spectrophotometry~// 
Astrophys. J., 1983. Vol.~266. P.~713--717.
\end{thebibliography}
}
}


\end{multicols}  %1
\def\stat{zagor}

\def\tit{МЕТОДОЛОГИЧЕСКИЕ АСПЕКТЫ РАЗРАБОТКИ ЭЛЕКТРОННОГО 
РУССКО-АНГЛИЙСКОГО ТЕЗАУРУСА ПО~КОМПЬЮТЕРНОЙ 
ЛИНГВИСТИКЕ$^*$}

\def\titkol{Методологические аспекты разработки электронного 
русско-английского тезауруса по~компьютерной 
лингвистике}

\def\autkol{Ю.\,А.~Загорулько, О.\,И.~Боровикова, И.\,С.~Кононенко, 
Е.\,Г.~Соколова}
\def\aut{Ю.\,А.~Загорулько$^1$, О.\,И.~Боровикова$^2$, И.\,С.~Кононенко$^3$, 
Е.\,Г.~Соколова$^4$}

\titel{\tit}{\aut}{\autkol}{\titkol}

{\renewcommand{\thefootnote}{\fnsymbol{footnote}}\footnotetext[1]
{Работа выполнена при финансовой поддержке РГНФ (проект № 10-04-12108в).}}


\renewcommand{\thefootnote}{\arabic{footnote}}
\footnotetext[1]{Институт систем информатики имени А.\,П.~Ершова СО РАН, zagor@iis.nsk.su}
\footnotetext[2]{Институт систем информатики имени А.\,П.~Ершова СО РАН, olesya@iis.nsk.su}
\footnotetext[3]{Институт систем информатики имени А.\,П.~Ершова СО РАН, irina\_k@cn.ru}
\footnotetext[4]{Российский государственный гуманитарный университет, minegot@rambler.ru}


\Abst{Обсуждаются методологические аспекты разработки русско-английского 
электронного тезауруса по компьютерной лингвистике (КЛ). Обосновывается необходимость 
разработки такого тезауруса и принципы его построения. Описываются состав тезауруса, 
структура тезаурусной статьи и набор связей между терминами. Обсуждается методика 
выбора терминов для включения в тезаурус, а также проблемы выбора основного 
тер\-ми\-на-дескрип\-то\-ра из множества синонимичных терминов и подбора парных 
тер\-ми\-нов-экви\-ва\-лен\-тов. Рассматриваются особенности реализации электронной версии тезауруса, при 
этом особое внимание уделяется проблеме поддержания логической целостности 
терминологической системы тезауруса и обеспечению удобного доступа к его 
содержимому.}

\KW{многоязычный тезаурус; компьютерная лингвистика; методология разработки 
тезаурусов; онтология; концептуальная схема тезауруса; технология построения порталов 
научных знаний}

\vskip 14pt plus 9pt minus 6pt

      \thispagestyle{headings}

      \begin{multicols}{2}

            \label{st\stat}

\section{Введение}

  Обеспечить обработку и эффективное использование постоянно растущих объемов 
неструктурированной информации становится уже невозможно без привлечения методов 
КЛ. Чтобы успешно приме\-нять эти методы для решения задач 
индексирования и содержательного поиска документов,\linebreak извлече\-ния информации из текстов, 
машинного перевода и построения есте\-ст\-вен\-но-язы\-ко\-вых интерфейсов, требуется знание 
терминологии КЛ, причем не только русскоязычной, но и англоязычной, так как 
большинство учебников и научных публикаций по тематике КЛ представлено на английском 
языке. Однако на данный момент в КЛ не существует четкой и общепринятой системы 
научной терминологии, причем многие термины современной КЛ не представлены на 
русском языке ни в одном из лингвистических источников.
  
  Так, тезаурус по теоретической и прикладной лингвистике, созданный в 1978~г.\ 
С.\,Е.~Никитиной~[1], уже устарел. К~тому же он одноязычный и не содержит определений 
понятий. Англо-русский терминологический словарь В.\,З.~Демьянкова~[2] содержит 
толкования, но не отражает современную картину этого научного направления. 
  
  Собственно лингвистика представлена в нескольких фундаментальных источниках, в 
част\-ности в Лингвистическом энциклопедическом словаре (ЛЭС)~[3], словаре О.\,С.~Ахмановой~[4], а также интернет-энциклопедии 
<<Кругосвет>>~[5], содержащей статьи по новым для традиционной лингвистики понятиям. 
Разработанный в 2007~г.\ в ИНИОН РАН тезаурус по языкознанию~[6] содержит около 
3000~терминов, однако только около 4\% из них относятся к области КЛ.
  
  Определения терминов КЛ можно найти в толковом словаре по искусственному 
интеллекту~[7]. Однако он отражает терминологию на конец\linebreak 1980-х~гг.\ и содержит 
довольно мало терминов КЛ.
  
  Так как КЛ имеет междисциплинарный характер, то некоторые ее термины можно найти в 
общих энциклопедиях, например в Большом энциклопедическом словаре~[8]. Популярным источником знаний по КЛ сейчас 
является Википедия~[9], в которой можно найти объяснения, классификации и ссылки на 
источники по многим понятиям КЛ, однако эти сведения часто страдают односто\-рон\-ностью, 
неполнотой и эскизностью.
  
  Таким образом, на данный момент не существует источника, в котором вся терминология 
КЛ была бы приведена в единую систему. Это вызывает необходимость разработки 
двуязычного тезауруса, содержащего английские и русские термины КЛ и их толкования. 
Двуязычность тезауруса даст возможность отечественным ученым и специалистам быстрее и 
эффективнее ориентироваться в мировой ситуации в данной области. Составление такого 
тезауруса позволит выявлять различия и сходства между понятиями, используемыми в 
отечественной и зарубежной науке, а также вводить новые понятия и лингвистические 
термины, отсутствующие в русском языке~[10].
  
  В данной работе обсуждаются методологические аспекты разработки русско-английского 
тезауруса по компьютерной лингвистике. В~разд.~2 описываются принципы разработки и 
состав тезауруса, структура тезаурусной статьи и набор связей между терминами. В~разд.~3 
обсуждается методика выбора терминов для включения в тезаурус, а также проблемы выбора 
основного тер\-ми\-на-де\-скрип\-то\-ра из множества синонимичных терминов и подбора парных 
тер\-ми\-нов-экви\-ва\-лен\-тов. В~разд.~4 рас\-смат\-ри\-ва\-ют\-ся особенности реализации электронной 
версии тезауруса.

\section{Проектирование структуры тезауруса }

  Проектирование структуры русско-английского тезауруса по компьютерной лингвистике 
выполнялось в соответствии с существующими отечественными и международными 
стандартами~[11--15], регламентирующими построение информационно-поисковых 
тезаурусов (ИПТ), а также на основе анализа и обобщения накопленного к этому времени 
опыта разработки ряда отечественных тезаурусов ИНИОН~[16], РуТез~[17] и~др. 
  
  Упомянутые выше стандарты определяют основные единицы тезауруса и возможный 
набор отношений между ними, устанавливают общие правила сбора массива лексических 
единиц, формирования словника, построения словарных статей и оформления ИПТ.
  
  В зависимости от назначения ИПТ могут включать в свой состав либо только 
дескрипторы (предпочтительные термины), либо дескрипторы и аскрипторы (обычные 
термины). Во втором случае дескрипторы могут использоваться при индексировании 
документов и в поисковых запросах, а аскрипторы (как текстовые входы) подлежат замене 
одним или несколькими дескрипторами~\cite{17-zag}.
  
  Тезаурусы делятся на одноязычные и многоязычные. Многоязычный 
  ин\-фор\-ма\-ци\-он\-но-по\-иско\-вый тезаурус (МИПТ) содержит термины из нескольких естественных языков и 
представляет эквивалентные по смыслу понятия на каждом из них. 

\columnbreak
  
  Построение русско-английского тезауруса по КЛ выполнялось в соответствии с 
требованиями межгосударственного стандарта ГОСТ 7.24-2007~\cite{11-zag}, который 
разработан с учетом основных нормативных положений международного стандарта ISO 
5964-1985~\cite{12-zag} и устанавливает состав, структуру и основные требования к 
построению МИПТ. Тезаурус разрабатывался как набор одноязычных версий МИПТ, при 
этом выполнялось согласованное построение одновременно двух версий тезауруса~--- 
русскоязычной и англоязычной. Разработка одноязычных версий тезауруса выполнялась на 
основе международного стандарта ISO 2788-1986~\cite{13-zag}, межгосударственного 
стандарта ГОСТ 7.25-2001~\cite{14-zag} и американского стандарта 
  Z39.19-2005~\cite{15-zag}.

\subsection{Выбор структуры словарной статьи}

  Основными единицами разработанного тезауруса являются термины предметной области 
(ПрО), подразделяемые на дескрипторы и аскрипторы. В~тезаурус включаются следующие 
типы лексических единиц: одиночные слова (преимущественно\linebreak существительные), именные 
словосочетания, лексически значимые компоненты сложных слов, сокращения слов и 
словосочетаний. Близкие по\linebreak смыслу лексические единицы образуют класс эквивалентности, 
при этом одна из них выбирается в качестве представителя этого класса и получает статус 
дескриптора, остальные лексические единицы получают статус аскриптора. Статус 
аскриптора получают также и термины, представляемые аббревиатурами или иными 
вариантами написания (через дефис, с пробелом и~т.\,п.).
  
  В состав словарной статьи термина, вне зависимости от его статуса, входят следующие 
элементы:
\begin{itemize}
\item  \textit{название термина}, т.\,е.\ лексическая единица, представленная в нормальной 
форме (для одиночного существительного или опорного слова словосочетания это форма 
именительного падежа единственного числа);
  
  \item \textit{язык}, на котором дано название термина;
  \item
  \textit{комментарий}, включающий правила и рекомендации использования термина, а 
также замечания и пояснения автора словарной статьи;
  \item
  \textit{автор словарной статьи}, т.\,е.\ фамилия и имя разработчика словарной статьи 
(задается для контроля процесса коллективной разработки тезауруса).
  \end{itemize}
  
  Для описания терминов-дескрипторов, кроме перечисленных выше атрибутов, вводятся 
сле\-ду\-ющие дополнительные атрибуты:
  \begin{itemize}
\item  \textit{определение термина}, поясняющее на языке термина его смысл или значение. 
Наличие в тезаурусе определений терминов делает возможным его использование не только 
в качестве инструмента для ручного или автоматизированного индексирования, но и в 
качестве источника систематизированных знаний о данной ПрО;
  \item
  \textit{релятор}, представляющий собой помету, введенную для различения омонимичных 
терминов (омографов) в рамках описываемой ПрО. Он является частью термина и поясняет 
его значение, относя его к определенной понятийной категории или предметно-тематической 
области (в контексте данной статьи~--- подобласти КЛ или смежной с ней области/подобласти 
знаний). Например, для различения двух понятий, образованных на основе словосочетания 
РАЗМЕТКА ТЕКСТА, могут быть использованы реляторы ПРОЦЕСС и ОБЪЕКТ. 
В~результате получается два разных термина-дескриптора РАЗМЕТКА ТЕКСТА 
(ПРОЦЕСС) и РАЗМЕТКА ТЕКСТА (ОБЪЕКТ);
  \item
  \textit{область/подобласть знаний}, к которой относится данный термин-дескриптор;
  \item
  \textit{признак корневого термина}, указывающий на то, что дескриптор находится на 
самом верхнем уровне одной из представленных в тезаурусе иерархий понятий.
\end{itemize}
  
  Термины тезауруса связываются различными семантическими отношениями, 
отражающими место каждого термина в системе понятий выбранной ПрО.
  
  Для связи дескрипторов с аскрипторами используются отношения синонимии нескольких 
типов. Так, если дескриптор может однозначно во всех контекстах заменить какой-то 
аскриптор, то он связывается с ним отношением <<Синоним>>; при этом также 
устанавливается обратное отношение от аскриптора к дескриптору~--- <<Смотри>>. Для 
моделирования других соотношений между аскрипторами и дескрипторами в соответствии с 
ГОСТ 7.25-2001 в тезаурус вводятся отношения, позволяющие задавать связи между 
аскрипторами и альтернативными дескрипторами или представлять аскриптор комбинацией 
дескрипторов. 
  
  В тех случаях, когда нет однозначного соответствия между дескрипторами и 
аскрипторами, используются отношения <<Используй альтернативно>> или <<Используй 
комбинацию>>, задающие соответствие между аскриптором и заменяющими его 
дескрипторами; при этом вводятся обратные им отношения <<Сравни альтернативный 
выбор>> и <<Сравни комбинацию>>. 

Например, аскриптор ПАРТИЦИПАНТ может\linebreak быть 
связан отношением <<Используй альтерна-\linebreak тивно>> с дескрипторами СЕМАНТИЧЕСКАЯ 
ВА-\linebreak ЛЕНТНОСТЬ и УЧАСТНИК СИТУАЦИИ. В~то\linebreak же время аскриптор СИСТЕМА 
СТАТИСТИЧЕСКОГО МАШИННОГО ПЕРЕВОДА может быть\linebreak представлен с помощью 
связи <<Используй комбинацию>> как комбинация (сочетание) двух дескрипторов~--- 
СИСТЕМА МАШИННОГО ПЕРЕВОДА и СТАТИСТИЧЕСКИЙ МАШИННЫЙ\linebreak  ПЕРЕВОД.
{ %\looseness=1

}
  
  Для отражения семантических связей между\linebreak понятиями, выражаемыми дескрипторами, 
уста\-нав\-ли\-ва\-ют\-ся иерархические и ассоциативные отношения. (Следует заметить, что такого 
типа отношениями связываются только дескрипторы, входящие в одну и ту же одноязычную 
версию тезауруса.)
  
  В тезаурусе допускается использование таких иерархических отношений, как 
недифференцированная иерархическая связь <<Выше>>, направленная от нижестоящего 
дескриптора к вышестоящему; родовидовая связь <<Выше род>>, устанавливаемая между 
двумя дескрипторами, когда объем понятия нижестоящего дескриптора входит в объем 
понятия вышестоящего дескриптора; партонимическая связь <<Выше целое>>, задаваемая 
между двумя дескрипторами в том случае, когда нижестоящий дескриптор представляет 
компонент объекта, обозначаемого вышестоящим дескриптором. Вводятся также обратные 
им отношения: <<Ниже>>, <<Ниже вид>>, <<Ниже часть>>. 
  
  Для задания отношений между дескрипторами, представляющими класс понятий и 
экземпляр этого класса, были выбраны связи <<Выше класс>> и <<Экземпляр класса>>.
  
  При установлении иерархических отношений для некоторых дескрипторов можно указать 
признак <<Аспект деления иерархии>>. Так, например,\linebreak в иерархии, построенной по 
отношению <<\mbox{Ниже} вид>>, МАШИННЫЙ ПЕРЕВОД по признаку <<подход>> разделяется 
на СТАТИСТИЧЕСКИЙ МАШИННЫЙ ПЕРЕВОД, МАШИННЫЙ ПЕРЕВОД НА ОСНОВЕ 
ПРАВИЛ и МАШИННЫЙ ПЕРЕВОД, ОСНОВАННЫЙ НА ПРЕЦЕДЕНТАХ, а по признаку 
<<степень участия человека>>~--- на ПОЛНОСТЬЮ АВТОМАТИЧЕСКИЙ ПЕРЕВОД и 
ЧЕЛОВЕКО-МАШИННЫЙ ПЕРЕВОД.
  
  Таким образом, один и тот же дескриптор одновременно может входить в несколько 
иерархий понятий, построенных по различным отношениям (<<Выше>>, <<Выше род>>, 
<<Выше целое>>) и по различным аспектам деления иерархии. 
  
  Для задания произвольных ассоциативных связей между дескрипторами, например 
отношений, выражающих зависимости вида <<процесс--объект>>, <<причина--следствие>> 
и~др., вводится отношение <<Ассоциируется с>>.
  
  Для связывания эквивалентных по смыслу дескрипторов, входящих в разные 
одноязычные версии, служит отношение <<Эквивалент на другом \mbox{языке}>>. 

\subsection{Представление источников терминов}

  Для подтверждения актуальности введенных в тезаурус терминов и ознакомления 
пользователей тезауруса с практикой их употребления для каждого термина задаются его 
связи с источниками, т.\,е.\ текстовыми документами или коллекциями текстовых 
документов, в которых данный термин встречается или определяется.
  
   Этим целям служат отношения <<Встречается в>>, <<Встречается в части документа>> и 
<<Дается определение в>>.
  
  Отношение <<Встречается в>> служит для связывания любого термина с источником; при 
этом, если источник~--- коллекция текстов, то в качестве значения специального атрибута 
этого отношения можно указать частоту встречаемости термина в источнике.
  
  С помощью отношения <<Встречается в части документа>> можно сообщить, что данный 
термин встречается в предметном указателе или глоссарии источника, что указывает на 
важность термина и повышает степень доверия к нему. 
  
  С помощью отношения <<Дается определение в>> термины-дескрипторы, снабженные 
толкованиями-определениями, связываются с источниками определений.
  
  В тезаурусе источники описываются сле\-ду\-ющи\-ми параметрами: название, 
библиографическая ссылка, язык, тип (книга, монография, научная статья, документация, 
учебник, словарь, тезаурус, интернет-ресурс, коллекция текстов и~др.), краткое описание и 
адрес в сети Интернет. Для коллекции текстов дополнительно задается число текстов и 
словоупотреблений.

\section{Методика выбора терминов для~включения в~тезаурус}

  Важным моментом при построении тезауруса является методика подбора 
  терминов~--- кандидатов на включение в тезаурус,~--- выбор терминов-дескрипторов из 
множеств синонимичных терминов, а также подбор иноязычных эквивалентов. 
  
  Выбор терминов для включения в русско-анг\-лий\-ский тезаурус по КЛ сопряжен с 
трудностями, которые обусловлены особенностями самой КЛ как новейшей науки и 
состоянием ее развития в России. Здесь важно отметить следующие факторы, 
характеризующие КЛ в целом и русскоязычную КЛ (РКЛ) в частности:
  \begin{itemize}
\item  междисциплинарный характер КЛ;
  \item неоднородность ПрО <<Компьютерная лингвистика>>;
  \item неравномерность развития отдельных на\-прав\-ле\-ний КЛ;
  \item отличие русскоязычной КЛ от англоязычной (в частности, отставание отдельных 
направлений РКЛ).
  \end{itemize}
  
  Ранее КЛ рассматривалась как часть исследовательского направления <<искусственный 
интеллект>> (ИИ), терминология которого считается\linebreak зрелой: <<Специальная терминология 
по искусственному интеллекту и интеллектуальным системам начала формироваться в 
  60-е~годы ХХ~в. Первый этап формирования терминологии всегда\linebreak отличается наличием 
многих синонимических терминов, которые используют различные школы и группы 
специалистов. На этом этапе термины быстро возникают и часть из них так же быстро 
исчезает. К~середине 1970-х~гг.\ терминология в области искусственного интеллекта стала 
уста\-нав\-ли\-вать\-ся. Появились термины, которые признало подавляющее большинство 
специалистов. Все эти термины (за редким исключением) по происхождению англоязычные, 
так как именно в США проводились интенсивные исследования в этой области. 
Окончательно основная терминология закрепилась в первой половине 1980-х~гг.>>~[7]. 
  
  Искусственный интеллект~--- это методологическая область, методы которой применимы к разным ПрО, в 
част\-ности активно применяются в КЛ в последнее десятилетие. Терминология КЛ в 
отдельных разделах продолжает сохранять черты первого этапа (наличие большого числа 
синонимов, например в разделе семантических отношений). Искусственный интеллект тоже считается 
междисциплинарной областью, однако по этому параметру ИИ и\linebreak КЛ противоположны: ИИ 
междисциплинарна, потому что ее методы применяются в разных дис\-цип\-ли\-нах, КЛ~--- 
потому что она вбирает в себя разные дисциплины, такие как лингвистика (разделы, 
связанные с обработкой текстов и речи), психология, некоторые разделы ИИ.
   
   Следствием указанных выше факторов является отсутствие русскоязычных учебных и 
лексикографических источников, достаточно полно отражающих структуру современной 
КЛ, в отличие от англоязычных источников, где она представлена детально и отчетливо. До 
сих пор термины РКЛ входили лишь в состав словарей и глоссариев по лингвистике и 
смежным ей областям знаний. Так, имеются источники только по отдельным разделам\linebreak 
смежных областей и КЛ, например по искусственному интеллекту, информационному 
поиску, и\linebreak
 почти полностью отсутствуют русскоязычные термины по другим разделам КЛ, 
например по <<Оценке эффективности систем и методов>> (\textit{Evaluation}). Кроме того, 
один и тот же термин, например \textit{синтаксический анализ}, в таких смежных науках, как 
ИИ и КЛ, имеет разное толкование. 
  
  Учитывая вышеперечисленные особенности КЛ и связанный с ними недостаток 
современной справочной русскоязычной литературы по КЛ, при разработке тезауруса 
использовались источники <<живых>> терминов РКЛ и их толкований, и именно они 
фиксировались в словарных статьях тезауруса. 
  
  В качестве основного источника русскоязычных терминов была выбрана коллекция 
текстов докладов, представленных на международной конференции <<Диалог>>~[18] в 
2000--2010~гг., как <<зеркала>>, отражающего термины РКЛ в их реальном употреблении. 
Собранная коллекция имеет следующие характеристики: число документов~--- 1193, 
объем~--- 4\,610\,694 словоупотреблений, суммарный размер~--- 27,5~МБ.
  
  К данной коллекции была применена словарная технология~[19], с помощью которой на 
базе лингвистических моделей (морфологического и локального синтаксического анализа) и 
статистических показателей был создан список статистически значимых в данной ПрО слов 
и словосочетаний~--- кандидатов в термины ПрО. Затем этот список был обработан 
(отфильтрован) экспертами в области КЛ, которые существенно опирались не только на 
знания о предмете и направлениях КЛ, но и на общелингвистические представления о 
терминологичности и путях формирования терминологических словников. Таким образом, 
избранный авторами подход, учитывающий предварительное структурирование ПрО, 
согласуется с общей методикой формирования словников на базе классификационных схем 
предметных областей (см., например,~[20]).
  
  Для английской части словника, с учетом русско-английской направленности 
создаваемого тезауруса, выбирались переводные эквиваленты из доступных англоязычных 
источников по КЛ.
  
  С другой стороны, чтобы дополнить картину РКЛ в тех ее разделах, где имеются пробелы, 
при\linebreak сборе терминов по таким разделам пришлось опираться преимущественно на 
англоязычные источники. Так, учитывая скачок, совершенный в\linebreak течение последних 
нескольких лет в такой высокотехнологичной подобласти КЛ, как <<Речевые технологии>>, 
а также тот факт, что это направление слабо представлено в коллекции <<Диалог>>, при 
сборе терминов для этой подобласти была применена обратная методика, т.\,е.\ в качестве 
основных использовались англоязычные источники: предметные указатели нескольких 
современных и наиболее авторитетных англоязычных книжных источников 
  обзорно-учебного профиля и глоссарии, входящие в документацию известных звуковых 
анализаторов. На данной терминологической базе был составлен англо-русский словник 
параллельных терминов.
  
  Достаточно сложной оказалась и проблема выбора основного термина-дескриптора из 
множества синонимичных терминов. Прежде всего, эта проблема связана с появлением 
новых понятий и соответствующих им терминов. Так, появление систем \textit{translation 
memory} в сфере автоматизированного перевода привело к широкому использованию 
практиками-переводчиками термина \textit{память переводов}, который не был принят 
научным сообществом, противопоставившим ему термин \textit{переводческая память} 
(синонимический ряд терминов с частотными характеристиками из коллекции <<Диалог>>: 
\textit{переводческая память}~--- 8, \textit{память переводов}~--- 0, \textit{архив 
переводов}~--- 1, \textit{накопитель переводов}~--- 0, \textit{копилка переводов}~--- 0).
  
  Развитие некоторых направлений КЛ (например, таких как \textit{автоматический 
перевод в режиме онлайн}) приводит к столкновению вариантов старых терминов. Так, 
тезаурус ИНИОН~[6] и ЛЭС~[3] основным термином в паре \textit{автоматический 
перевод} и \textit{машинный перевод} считают \textit{автоматический перевод}, присвоив 
ему статус дескриптора. Однако показатели встречаемости в коллекции <<Диалог>> говорят 
в пользу термина \textit{машинный перевод}: 318 против~58. Интернет-энциклопедии 
<<Википедия>> и <<Кругосвет>>, а также учебники придерживаются этой же традиции. На 
сайте Европейской ассоциации машинного перевода~[21] также отмечается, что термин 
\textit{machine translation}, хоть и звучит архаично, но, тем не менее, сохраняется как 
основной общий термин для всей области. В~данном случае эксперты согласились с этой, 
соответствующей традиции, точкой зрения.
   
   Проблема выбора дескриптора возникает на фоне незрелости системы понятий КЛ, 
приводящей в некоторых случаях к очень широкой ва\-ри\-а\-тив\-ности терминов, с одной 
стороны, и к их мно\-го\-знач\-ности, с другой. Так, термин \textit{валентная структура} с
частотностью 20 имеет целый ряд вариантов: \textit{валентная рамка}~--- 14, \textit{рамка 
валентностей}~--- 64,\linebreak
\textit{валентностная структура}~--- 3, \textit{схема 
валентностей}~--- 3. В~то же время исследование реального упо\-треб\-ле\-ния термина 
\textit{валентная структура} показало, что он, как и термин \textit{модель управления}, 
имеет как узкое толкование (множество синтаксических валентностей предикатного слова), 
так и более широкое толкование (описание соответствия семантических валентностей слова 
их грамматическому оформлению, т.\,е.\ синтаксическим валентностям). В~этой ситуации в 
тезаурус вводится два одинаковых дескриптора, один из которых снабжается релятором.
    
    Серьезные трудности возникают при подборе парных терминов (англо-русских 
эквивалентов). В~качестве примера можно привести термин \textit{spoken language machine 
translation}. Задача автоматического перевода устной речи возникла на стыке <<Машинного 
перевода>> и <<Речевых технологий>>. \textit{Spoken language processing} обычно 
переводится как \textit{автоматическая обработка устного языка}, одной из задач которой 
является автоматический устный перевод (АУП) с его разновидностями, соответствующими 
АУП типа <<$\mbox{Речь}(L_1) \rightarrow \mbox{\ Текст}(L_2)$>> и АУП типа 
<<$\mbox{Речь}(L1) \rightarrow\ \mbox{Речь}(L_2)$>>. Вторая разновидность представлена 
английским термином \textit{speech-to-speech translation}. В~русскоязычной литературе 
такой традиции нет, как нет (или практически нет) и такого типа приложений. Поиск в 
Интернете дал в качестве эквивалента для \textit{spoken language machine translation} 
единично встретившийся вариант \textit{автоматический перевод устной речи}. Этот 
русский переводной эквивалент и был выбран в качестве парного русскоязычного термина-
дескриптора.
  
  Таким образом, при выборе терминов-де\-скрип\-то\-ров авторы опирались не только на 
статистику, но и на традиции словоупотребления, сложившиеся к настоящему времени в 
лингвистическом научном сообществе. Что же касается выбора парных терминов для 
новейших подобластей КЛ, не представленных в русскоязычной литературе, 
соответствующие дескрипторы предлагались как переводные эквиваленты, а в качестве 
основных критериев выбора перевода выступили знания и интуиция эксперта.

\section{Подход к реализации электронной версии тезауруса}

  Для реализации электронной версии тезауруса было решено использовать методологию и 
программные компоненты технологии построения\linebreak
порталов научных знаний~[22, 23], уже 
ранее применявшиеся при создании порталов знаний по археологии~[24] и компьютерной 
лингвистике~[25].\linebreak
Данная технология базируется на онтологии и предоставляет средства 
настройки на предметную область и управления контентом информационной системы, а 
также средства навигации и поиска. Средства настройки на предметную область и 
поддерживаемая ими методология достаточно хорошо подходят для разработки 
концептуальной схемы тезауруса, а остальные из перечисленных средств могут выполнять 
роль его основных программных компонентов, обеспечивающих создание, сопровождение и 
использование тезауруса.

\subsection{Разработка онтологии представления тезауруса}

  В используемой технологии в качестве информационной модели портала знаний 
(информа\-ци\-он\-ной системы) используется онтология, \mbox{которая}, обеспечивая формальное 
описание предметной области системы, не только определяет структуры для его 
информационного наполнения (контента), но и задает базис для организации 
содержательного доступа к знаниям и данным, содержащимся в нем.
  
  Для описания онтологии данная технология предоставляет формализм, который назовем 
онтологией представления знаний, и поддерживающий его редактор онтологии. С помощью 
этих средств была построена онтология представления тезауруса $O_{Th}$, задающая его 
концептуальную схему: 
  $$
  O_{\mathrm{Th}} =\big\langle C,R,T,D,At, P,Axt\big\rangle
  $$
  где $C=\{\mathrm{Tr}, S_T,S_K\}$~--- конечное непустое множество классов, представляющих 
основные сущности\linebreak тезауруса; здесь $\mathrm{Tr=Asc\,Tr=Asc\cup Des}$~--- класс терминов, 
представляющих понятия ПрО <<Компьютерная лингвистика>>, включающий два\linebreak 
подкласса~--- Asc (тер\-ми\-ны-аскрип\-то\-ры) и Des (тер\-ми\-ны-де\-скрип\-то\-ры); $S_T$~--- 
класс источников терминов; $S_K$~--- класс областей/подобластей знаний;
  $R\hm=R^{\mathrm{TT}}\cup R^{\mathrm{TST}}\cup R^{\mathrm{TSK}}$~--- конечное\linebreak
   множество отношений, где 
  $R^{\mathrm{TT}}\hm=\{R_a^{\mathrm{TT}},\ldots , R_m^{\mathrm{TT}}\}$, 
  $R_i^{\mathrm{TT}}\subseteq \mathrm{Tr}\times \mathrm{Tr}$~--- конечное 
множество бинарных отношений, заданных на терминах, 
  $R^{\mathrm{TST}}\hm=\{ R^{\mathrm{TSF}}, R^{\mathrm{TSP}}, R^{\mathrm{TSD}}\}$, 
  $R_i^{\mathrm{TST}}\subseteq \mathrm{Tr}\times S_T$~--- 
бинарные отношения, связывающие термины тезауруса с источниками, причем $R^{\mathrm{TSF}}$ 
связывает термин с источником, где он встречается, $R^{\mathrm{TSP}}$ связывает термин с 
источником, где он встречается в предметном указателе или глоссарии, а $R^{\mathrm{TSD}}$~--- 
связывает термин с источником, где дается его определение;
  $R^{\mathrm{TSK}}\hm=\{R^{\mathrm{SKT}},R^{\mathrm{SKS}}\}$~--- бинарные отношения, служащие для встраивания 
областей знаний в тезаурус, где $R^{\mathrm{SKT}}\subseteq \mathrm{Tr}\times S_K$ связывает термины 
тезауруса с областями знаний, а $R^{\mathrm{SKS}}\subseteq S_K\times S_K$ задает иерархию на 
подобластях знаний;
  $T$~--- множество стандартных типов;
  $D\hm=\{d_1,\ldots, d_n\}$~--- множество доменов $d_i\hm=\{s_1,\ldots, s_k\}$, где $s_i$~--- 
значение стандартного типа \textit{string}; 
  $at=\{at_1,\ldots, at_w\}$~--- конечное множество атрибутов, описывающих свойства 
основных сущностей тезауруса и отношений между ними; значения этих свойств определены 
на множестве $T\cup D$;
  $P=\{P_1,\ldots, P_n\}$~--- множество формальных свойств отношений~$R^{\mathrm{TT}}$;
  Axt~--- множество аксиом, задающих дополнительные ограничения на связи между 
терминами. 
  
  Таким образом, онтология представления тезауруса описывает классы, представляющие 
основные сущности тезауруса (термины тезауруса, их источники, области/подобласти 
знаний), отношения, связывающие объекты этих классов между собой, свойства понятий и 
отношений, а также аксиомы, определяющие их дополнительную семантику. Кроме того, в 
онтологии задается множество доменов, т.\,е.\ возможных значений атрибутов классов и 
отношений, что позволяет уменьшить число ошибок при создании/редактировании 
конкретного тезауруса.
  
  Для отношений в онтологии задаются математические свойства (симметричность, 
рефлексивность, транзитивность, асимметричность, антирефлексивность) и обратные 
отношения.
  
  Так, для введенных в подразд.~2.1 иерархических отношений (<<Выше>>, <<Выше род>>, 
<<Выше класс>>,\linebreak <<Выше целое>>) задаются математические  свойства <<транзитивность>> 
и <<асимметричность>> и соответствующие обратные отношения (<<Ниже>>, <<Ниже 
вид>>, <<Экземпляр класса>>, <<Ниже часть>>).\linebreak Отношения <<Эквивалент на другом 
языке>> и <<Ассоциируется с>> объявляются симметричными и антирефлексивными. Для 
отношений, выражающих синонимию терминов (<<Синоним>>, <<Используй\linebreak 
альтернативно>>, <<Используй комбинацию>>), задаются обратные отношения 
(соответственно <<Смот\-ри>>, <<Сравни альтернативный выбор>>, <<Сравни 
комбинацию>>).

\subsection{Организация управления контентом тезауруса}

  Для описания конкретных терминов, их источников, областей знаний, а также для 
установления связей между ними используется редактор данных, предоставляемый 
технологией построения порталов знаний и управляемый онтологией пред\-став\-ле\-ния 
тезауруса. Этот редактор реализован как веб-при\-ло\-же\-ние и доступен зарегистрированным 
пользователям через Интернет. (Заметим, что сразу после завершения ввода и/или 
редактирования описаний терминов, источников и связей между ними, новая информация 
становится доступной через пользовательский веб-интерфейс тезауруса.)
  
  С целью обеспечения распределенной коллективной разработки используемая технология 
поддерживает механизм делегирования прав экспертам разных уровней. В~соответствии с 
этим механизмом только эксперты самого высокого уровня могут редактировать структуры 
тезауруса (с помощью редактора онтологий), а эксперты более низких уровней~--- только его 
содержание (с помощью редактора данных).
  
  Кроме того, действует правило, по которому редактировать словарную статью может 
только ее автор. Если кто-то из экспертов захочет внести изменения в <<чужую>> статью, он 
должен согласовать такую возможность с ее автором, в частности, через специальный 
форум, на который имеется ссылка в электронном тезаурусе. 
  
  Для того чтобы тезаурус мог использоваться при индексировании и поиске текстовых 
документов, он должен представлять целостную и непротиворечивую систему понятий ПрО. 
Это обеспечивается встроенными в редактор механизмами вывода и поддержки логической 
целостности системы понятий тезауруса, работа которых базируется на описаниях свойств 
классов и отношений, заданных в онтологии представления тезауруса.
  
  В частности, на основе этих свойств происходит корректное установление связей между 
терминами тезауруса, при необходимости осуществляется их автоматическое добавление 
и/или удаление. Кроме того, контролируются ограничения на существование и число связей 
между терминами тезауруса в зависимости от принадлежности терминов к тем или иным 
классам.
  
  Например, если для рассмотренного выше отношения <<Смотри>> задано обратное 
отношение (<<Синоним>>) и ограничение на существование связей (<<только одна связь 
данного типа для каж\-до\-го тер\-ми\-на-аскрип\-то\-ра>>), то при связывании аскриптора ПАМЯТЬ 
ПЕРЕВОДОВ и дескриптора ПЕРЕВОДЧЕСКАЯ ПАМЯТЬ отношением 
\textit{Смот\-ри}(ПАМЯТЬ ПЕРЕВОДОВ, ПЕРЕВОДЧЕСКАЯ ПАМЯТЬ) произойдет создание 
обратной связи \textit{Си\-но\-ним}(ПЕРЕВОДЧЕСКАЯ ПАМЯТЬ, ПАМЯТЬ ПЕРЕВОДОВ) 
(если таковой еще не существует), а также для аскриптора ПАМЯТЬ ПЕРЕВОДОВ будет 
обеспечиваться запрет на создание связей <<Смот\-ри>> и <<Синоним>> с другими 
дескрипторами.

\subsection{Обеспечение доступа к~контенту тезауруса}

  Удобный доступ к терминам тезауруса обеспечивается пользовательским 
  веб-интерфейсом, также\linebreak предоставляемым технологией построения порталов научных 
знаний. В~этом интерфейсе содержимое тезауруса представляется пользователю в виде сети 
взаимосвязанных информационных\linebreak\vspace*{-12pt}

\pagebreak

\end{multicols}

\begin{figure}
 \vspace*{1pt}
 \begin{center}
 \mbox{%
 \epsfxsize=160mm
 \epsfbox{zag-1.eps}
 }
 \end{center}
 \vspace*{3pt}
\begin{center}
{\small Представление термина <<Переводческая память>>}
\end{center}
\vspace*{6pt}
  \end{figure}


\begin{multicols}{2}

\noindent
 объектов~--- элементов тезауруса: терминов 
(дескрипторов и аскрипторов) и описаний источников терминов и их определений. Набор 
атрибутов терминов и связей, установленных между ними, соответствует структуре 
тезауруса, описанной в подразд.~2.1.
  
  При навигации по тезаурусу обеспечивается возможность выбора необходимых 
пользователю терминов, детального просмотра их описаний (тезаурусных статей), а также 
описаний источников\linebreak (публикаций или коллекций текстов), в которых встречается термин 
и/или его определение.
  
  Пользователь может указать, какой тип информации его интересует~--- все термины, 
дескрипторы, аскрипторы, подобласти знаний или источники терминов. При этом ему 
выдается упорядоченный по алфавиту полный список имеющихся в тезаурусе объектов 
выбранного класса, который отображается в виде html-страницы, содержащей набор ссылок 
на эти объекты. 
  
  Информация о конкретном объекте и его связях также отображается в виде html-страницы 
(см.\ рисунок). При этом объекты, связанные с данным объектом, представляются на его 
странице в виде гиперссылок, по которым можно перейти к их детальному описанию.
  
  Дальнейшая навигация по тезаурусу представляет собой процесс перехода от одних 
объектов тезауруса к другим по заданным между ними связям, отражающим существующие 
между ними~--- тезаурусные (между терминами) или библиографические (между терминами 
и источниками)~--- отношения.


   
  Для обеспечения доступа к содержимому тезауруса из внешних систем разработан 
программный интерфейс, благодаря которому тезаурус может использоваться при решении 
задач индексирования и поиска текстовых документов по КЛ.
  
  Интерфейс поддерживает поиск терминов в тезаурусе по типу, наименованию или части 
наименования. Для каждого термина-дескриптора или аскриптора можно получить список 
связанных терминов по выбранному отношению (синонимия, эквивалентность, ассоциация 
и~т.\,п.). Дополнительно для каждого термина-дескриптора можно получить его перевод, 
список терминов-дескрипторов, связанных родовидовыми отношениями в соответствии с 
аспектом организации иерархии, список подобластей знаний, к которым относится данный 
термин, а также список источников (текстовых документов или коллекций текстов), в 
которых описан данный термин.

\section{Заключение}

  В статье рассмотрены методологические аспекты построения русско-английского 
электронного тезауруса по компьютерной лингвистике, разработанного в соответствии с 
международными и отечественными стандартами. Описана методика выбора терминов для 
включения в тезаурус, а также предложены подходы к выбору основного тер\-ми\-на-де\-скрип\-то\-ра 
из множества синонимичных терминов и подбору парных терминов-эквивалентов. Рассмотрены особенности реализации электронной версии тезауруса, 
обусловленные использованием в качестве инструмента разработки методологии и 
программных компонентов технологии построения порталов научных знаний~[22, 23].
  
  В настоящее время ведется активная разработка тезаурусных статей и заполнение ими 
контента электронного тезауруса, который на данный момент включает более 
1600~терминов, связанных примерно 8000~семантических отношений, а также описания 
более 180~источников терминов из 50~подобластей знаний. 
  
  Тезаурус ориентирован как на непосредственное использование людьми, желающими 
обратиться к системе понятий из области КЛ, так и для решения задач индексирования, 
тематического рубрицирования и информационного поиска (для этого он снабжен 
программным интерфейсом).
  
  Тезаурус может использоваться в учебном процессе~--- в тех вузах страны, где изучается 
компьютерная лингвистика и/или используются ее\linebreak
 результаты. Использование тезауруса в 
вузах повысит уровень профессиональной подготовки будущих специалистов в сфере КЛ и 
информационных технологий. По существу, это необходимый и профессионально 
выполненный инструмент и ресурс обучения, особенно ценный в виду междисциплинарной 
природы КЛ и полного отсутствия каких-либо русскоязычных учебников и даже 
методических пособий в этой новой и быстро развивающейся области знаний.


{\small\frenchspacing
{%\baselineskip=10.8pt
\addcontentsline{toc}{section}{Литература}
\begin{thebibliography}{99}
  
  \bibitem{1-zag}
  \Au{Никитина С.\,Е.}
  Тезаурус по теоретической и прикладной лингвистике.~--- М.: Наука, 1978.
  
  \bibitem{2-zag}
  \Au{Демьянков В.\,З.}
  Англо-русские термины по прикладной лингвистике и автоматической переработке 
текста. Вып.~2. Методы анализа текста~// Тетради новых терминов. №\,39.~--- М.: ВЦП, 
1982.
  
  \bibitem{3-zag}
  Лингвистический энциклопедический словарь~/ Под ред. В.\,Н.~Ярцевой.~--- М.: 
Советская энциклопедия, 1990. 

\bibitem{4-zag}
\Au{Ахманова О.\,С.}
Словарь лингвистических терминов.~--- 3-е изд., стер.~--- М.: КомКнига, 2005. 
  
  \bibitem{5-zag}
  Кругосвет: Онлайн-энциклопедия, 2001--2009. {\sf http://www.krugosvet.ru}.
  
  \bibitem{6-zag}
  Языкознание: Информационно-поисковый тезаурус ИНИОН.~--- М.: ИНИОН РАН, 2007.
  
  \bibitem{7-zag}
  \Au{Аверкин А.\,Н., Гаазе-Рапопорт~М.\,Г., Поспелов~Д.\,А.}
  Толковый словарь по искусственному интеллекту.~--- М.: Радио и связь, 1992.
  
  \bibitem{8-zag}
  Большой энциклопедический словарь (БЭС)~/ Гл. ред. А.\,М.~Прохоров.~---2-е изд., 
перераб. и доп.~--- СПб.: Норинт, 2004. 
  
  \bibitem{9-zag}
  Википедия: Свободная энциклопедия. {\sf http://ru. wikipedia.org}.
  
  \bibitem{10-zag}
  \Au{Соколова Е.\,Г., Семенова С.\,Ю., Кононенко~И.\,С., Загорулько~Ю.\,А., 
Кривнова~О.\,Ф., Захаров~В.\,П.}
  Особенности подготовки терминов для русско-английского тезауруса по компьютерной 
лингвистике~// Компьютерная лингвистика и интеллектуальные технологии: По мат-лам 
ежегодной междунар. конф. <<Диалог>> (Бекасово, 25--29~мая 2011).~--- М.: РГГУ, 2011. 
Вып.~10(17). С.~644--655.
  
 
  \bibitem{12-zag} %11
  ISO 5964-1985. Documentation~--- Guidelines for the establishment and development of 
multilingual thesauri, IDT. (Revised by: ISO/DIS 25964-1. Under development.)
  
  \bibitem{13-zag} %12
  ISO 2788-1986. Documentation~--- Guidelines for the establishment and development of 
monolingual thesauri. Ed.~2.
  
  \bibitem{14-zag} %13
  ГОСТ 7.25-2001. Система стандартов по информации, библиотечному и издательскому 
делу. Тезаурус ин\-фор\-мационно-поисковый одноязычный. Правила разработки, структура, 
состав и форма представления. Введен в действие с 1~июля 2002~г.
  
  \bibitem{15-zag} %14
  ANSI/NISO Z39.19-2005 Guidelines for the construction, format, and management of 
monolingual controlled vocabularies: Periodic review.

  \bibitem{11-zag} %15
  ГОСТ 7.24-2007. Система стандартов по информации, библиотечному и издательскому 
делу. Тезаурус информационно-поисковый многоязычный.\linebreak Состав, структура и основные 
требования к по\-стро\-ению. Введен в действие с 1~июля 2008~г.
  
  \bibitem{16-zag} %16
  \Au{Мдивани Р.\,Р.}
   О~разработке серии тезаурусов по социальным и гуманитарным наукам~// НТИ, 2004. 
Сер.~2. №\,7. С.~1--9.
  
  \bibitem{17-zag}
  \Au{Лукашевич Н.\,В.}
  Тезаурусы в задачах информационного поиска.~--- М.: Изд-во Московского ун-та, 2011.
  
  \bibitem{18-zag}
  Диалог: Сайт международной конференции. {\sf http:// www.dialog-21.ru}.
  
  \bibitem{19-zag}
  \Au{Сидорова Е.\,А.}
  Многоцелевая словарная подсистема извлечения предметной лексики~// Компьютерная 
лингвистика и интеллектуальные технологии: Труды междунар. конф. Диалог'2008.~--- М.: 
РГГУ, 2008. Вып.~7(14). С.~475--481.
  
  \bibitem{20-zag}
  \Au{Перерва В.\,М.}
  О~принципах и проблемах отбора терминов и составления словника терминологических 
словарей~// Проблематика определений терминов в словарях разных типов.~--- Л., 1976. 
С.~190--204.
  
  \bibitem{21-zag}
  EAMT (The European Association for Machine Translation ). {\sf http://www.eamt.org}.
  
  \bibitem{22-zag}
  \Au{Загорулько Ю.\,А., Боровикова~О.\,И.}
  Подход к построению порталов научных знаний~// Автометрия, 2008. Т.~44. №\,1. 
  С.~100--110.
  
  \bibitem{23-zag}
  \Au{Загорулько Ю.\,А.}
  Технология разработки порталов научных знаний~// Программные продукты и системы, 
2009. №\,4. С.~25--29.
  
  \bibitem{24-zag}
  \Au{Андреева О.\,А., Боровикова О.\,И., Булгаков~С.\,В., Загорулько~Ю.\,А., 
Сидорова~Е.\,А., Циркин~Б.\,Г., Холюшкин~Ю.\,П.}
  Археологический портал знаний: содержательный доступ к знаниям и информационным 
ресурсам по археологии~// КИИ-2006: Труды 10-й Национальной конф. по искусственному 
интеллекту с международным участием.~--- М.: Физматлит, 2006. Т.~3. С.~832--840.

\label{end\stat}
  
  \bibitem{25-zag}
  \Au{Боровикова О.\,И., Загорулько Ю.\,А., Загорулько~Г.\,Б., Кононенко~И.\,С., 
Соколова~Е.\,Г.}
  Разработка портала знаний по компьютерной лингвистике~// КИИ-2008: Труды 11-й 
Национальной конф. по искусственному интеллекту с международным участием.~--- М.: 
ЛЕНАНД, 2008. Т.~3. С.~380--388.
  
 \end{thebibliography}
}
}


\end{multicols}   %2
\def\stat{kogalovski}

\def\tit{КЛАССИФИКАЦИЯ И ИСПОЛЬЗОВАНИЕ СЕМАНТИЧЕСКИХ  
СВЯЗЕЙ МЕЖДУ ИНФОРМАЦИОННЫМИ ОБЪЕКТАМИ 
В~НАУЧНЫХ ЭЛЕКТРОННЫХ БИБЛИОТЕКАХ$^*$}

\def\titkol{Классификация и использование семантических  
связей между информационными объектами}
%в~научных электронных библиотеках}

\def\autkol{М.\,Р.~Когаловский, С.\,И. Паринов}
\def\aut{М.\,Р.~Когаловский$^1$, С.\,И. Паринов$^2$}

\titel{\tit}{\aut}{\autkol}{\titkol}

{\renewcommand{\thefootnote}{\fnsymbol{footnote}}\footnotetext[1]
{Работа поддерживается РГНФ, проект 11-02-12026-в.}}


\renewcommand{\thefootnote}{\arabic{footnote}}
\footnotetext[1]{Институт проблем рынка Российской академии наук, kogalov@cemi.rssi.ru}
\footnotetext[2]{Центральный экономико-математический институт Российской академии наук, 
sparinov@gmail.com}

\vspace*{-3pt}


\Abst{Обсуждается подход, обеспечивающий повышение информационной ценности 
контента научной электронной библиотеки благодаря поддержке классифицированных 
семантических связей между содержащимися в ней информационными объектами. 
Рассматривается реализация предлагаемого подхода на основе отечественной системы 
Соционет, объединяющей большое число научных электронных библиотек и являющейся 
де-факто институциональным исследовательским информационным пространством 
Отделения общественных наук Российской академии наук.}

\vspace*{-2pt}

\KW{электронная библиотека; информационный объект; коллекция информационных 
ресурсов; семантическая связь; классификатор связей; онтология; наукометрия}

\vspace*{-4pt}

\vskip 14pt plus 9pt minus 6pt

      \thispagestyle{headings}

      \begin{multicols}{2}

            \label{st\stat}


\section{Введение}

  Информационные объекты, содержащиеся в научных электронных 
библиотеках (статьи, книги, персональные профили авторов, профили 
организаций и~др.), имеют многие другие связи (отношения) друг с другом, 
помимо обычно поддерживаемых средствами управления информационными 
ресурсами и отображаемых программными интерфейсами электронных 
библиотек. Большое число видов связей из-за неразвитости технологий пока 
остается за рамками электронных библиотек, час\-то только в сознании 
исследователей. Такие связи являются ненаблюдаемыми и не фиксируются в 
цифровой форме. 
  
  К обычно отображаемым связям между информационными объектами 
электронных библиотек относятся, например, связи между научными 
пуб\-ли\-ка\-ци\-ями (статья, книга и~т.\,п.)\ и персональными профилями их авторов, 
а также с профилями организаций, в которых были получены соответствующие 
результаты исследований. Кроме этого, \mbox{статьи}, снабженные кодами 
тематических классификаторов, имеют связи с тематическими рубриками 
классификаторов научных дисциплин. Все чаще в электронных библиотеках 
встречаются связи между статьями и комментариями их читателей. 
В~некоторых крупных электронных библиотеках поддерживаются связи 
цитирования, профили авторов и связи публикаций с ними. Механизмы для 
этой цели имеются, в частности, в электронной биб\-лио\-те\-ке {ACM} 
({Association for Computing Machinery})~[1], научной библиотеке 
{eLibrary} РФФИ~[2], в Академии {Google} ({Google 
Scholar})~[3]. Однако в этих и в других аналогичных случаях связи 
цитирования не несут никакой информации, кроме самого факта цитирования, 
не характеризуют семантики отношения между цитирующим и цитируемым 
текстовым документом. Будем называть такие связи <<\textit{немыми}>>.
  
  Связи между информационными объектами в электронных библиотеках, в 
том числе и связи цитирования, обладают семантикой, и она может быть явно 
описана способом, доступным для пользователей и системных механизмов, и 
продуктивно использована. На ее основе может, в частности, формироваться 
более дифференцированная нау\-ко\-мет\-ри\-че\-ская статистика, учитывающая 
позитивное, негативное или иное отношение автора цитирующего документа к 
цитируемому. Связи с явно описанной семантикой будем называть далее 
\textit{семантическими связями}. 
  
  Как уже отмечалось, между содержащимися в научной электронной 
библиотеке информационными объектами могут поддерживаться не только 
семантические связи, такие как <<немые>> связи цитирования и другие, 
упоминаемые ранее. Например, семантическая связь может создаваться для 
выражения мнения автора одного из информационных объектов или экспертов 
о существовании некоторого отношения между контентом двух 
информационных объектов в ситуации, даже когда этот факт не отмечен явным 
образом в контенте рассматриваемых информационных объектов. 
  
  Явным образом описанные и поддерживаемые в библиотеке семантические 
связи могут быть \textit{классифицированы} на основе характера отношений 
между информационными объектами~--- участниками связей. Введение 
классификации приводит к образованию многослойной семантической 
структуры\linebreak контента электронной библиотеки, каждый слой которой 
соответствует некоторому классу семантических связей. Такая структура может 
служить источни\-ком информации для проведения качественно новых 
наукометрических измерений, для исследования структурных свойств корпуса 
знаний в различных областях науки, представительным образом отраженного в 
контенте электронной библиотеки. 
  
  Онлайновый режим функционирования научной электронной библиотеки 
позволяет реализовать ее систему управления таким образом, чтобы не только 
поддерживалась семантическая структура контента и обрабатывались 
пользовательские запросы, касающиеся ее характеристик, но и предостав\-лялась 
пользователям возможность самостоятельно в децентрализованном режиме 
описывать и создавать семантические связи. Могут быть также предусмотрены 
мониторинг состояния структуры связей и автоматическое оповещение авторов 
информационных объектов о том, что некоторый их объект стал участником 
вновь учрежденной связи или что ликвидирована существующая связь, в 
которой этот объект являлся участником. Благодаря этому автор 
информационного объекта, получивший указанное оповещение, стимулируется 
тем самым реагировать на эту ситуацию, если событие, о котором он 
информируется, противоречит его представлениям.
  
  Таким образом, в онлайновой электронной биб\-лио\-те\-ке, в которой 
поддерживаются классифицированные семантические связи, может быть 
обеспечен комплекс новых возможностей:
  \begin{itemize}
\item поддержка многослойной структуры семантических связей;
\item создание новых связей и аннулирование существующих связей не 
только администраторами информационных ресурсов, но и пользователями 
системы в децентрализованном режиме;
\item формирование дифференцированной по классам семантических связей 
статистики связей, в частности, касающейся связей цитирования;
\item оповещение авторов представленных в электронной библиотеке 
информационных объектов об их включении в новые связи или об 
аннулировании некоторых связей, в которых они были участниками. 
\end{itemize}

  Обладающая такими возможностями информационная среда обеспечивает 
качественно новые технологии для научной и 
  на\-уч\-но-ор\-га\-ни\-за\-ци\-он\-ной деятельности, открывает новые 
возможности для коммуникаций в научном сообществе. В~предлагаемой статье 
обсуждается подход авторов к созданию такой среды, реализуемый в научном 
информационном пространстве Соционет~[4].
  
  Остальная часть статьи организована сле\-ду\-ющим образом. В~разд.~2 
уточняется постановка рассматриваемой в статье проблемы и предлагаются 
пути ее решения. В~разд.~3 обсуждается вопрос о классификации связей и 
дается краткий обзор известных исследований в рассматриваемой области. 
В~разд.~4 обсуждаются принципы представления семантических связей в 
электронной библиотеке как самостоятельных информационных объектов. 
Свойства семантических связей рассматриваются в разд.~5. В~разд.~6 
обсуждается реализация предлагаемого в статье подхода в среде системы 
Соционет. В~заключении подводятся итоги обсуждения. 

\section{Уточнение постановки проблемы}

     Коллекции информационных ресурсов традиционных научных 
электронных библиотек состоят из множества объектов определенных типов: 
электронных версий публикаций, изданных типографским способом, научных 
отчетов, рабочих записок, рецензий, авторефератов диссертаций, полных 
текстов диссертационных работ, таблиц научных данных, карт звездного неба 
и~др. Коллекции могут содержать также сведения об авторах представленных в 
них публикаций, об организациях, в которых они работают, и информационные 
объекты других типов.
     
     В последние годы на основе библиографических ссылок, содержащихся в 
публикациях, которые выпускаются в авторитетных периодических из-\linebreak даниях, 
начали создаваться индексы цитирования, обеспечивающие формирование 
биб\-лио\-мет\-ри\-че\-ской статистики. Связи цитирования в текстовых публикациях 
обычно представляются неструктурированным образом в виде списка 
используемой литературы. Они не являются при этом носителями 
     ка\-кой-ли\-бо информации, кроме указания целевой публикации ссылки 
и существования самого факта ссылки. Однако с фактом цитирования связана 
еще и некоторая не отображаемая при этом семантика, выражающая отношение 
автора цитирующего документа к цитируемому источнику или ка\-кое-ли\-бо 
иное семантическое отношение между ци\-ти\-ру\-ющей и цитируемой 
публикацией. Как правило, связи цитирования аннотируются в тексте 
публикации, и в таких случаях семантика связи все-та\-ки описана, но в 
неструктурированном виде. Это создает значительные сложности для ее 
анализа, и в существующих системах такой анализ обычно не производится.
     
     Наряду со связями цитирования между информационными объектами 
научных электронных биб\-лио\-тек существуют разнообразные другие 
семантические связи. Например, связь может указывать, что ее целевой 
информационный объект содержит научные результаты, базирующиеся на 
результатах, описанных в исходном объекте связи, или что в исходном объекте 
связи опровергается результат, изложенный в ее целевом объекте. Связь может 
также указывать, что ее исходный информационный объект является новой 
редакцией целевого объекта или представляет собой его составную часть, 
например аннотацию. 
     
     Существует большое разнообразие семантических связей, которые можно 
при необходимости поддерживать между информационными объектами в 
библиотеке. Эти связи выявляются в результате участия представителей 
научного сообщества в процессах, реализующих их научную и на\-уч\-но-ор\-га\-ни\-за\-ци\-он\-ную 
деятельность. К~чис\-лу основных видов таких процессов 
можно отнести процессы систематизации, классификации и упорядочения 
корпуса научных знаний (например, при подготовке аналитических обзоров), 
процессы научной оценки опубликованных результатов (рецензирование 
работ), процессы продуцирования нового научного знания, процессы создания 
научных произведений, на\-уч\-но-ор\-га\-ни\-за\-ци\-он\-ные процессы. Именно 
на основе информации, рождающейся в результате учас\-тия в процессах 
перечисленных видов, пользователь электронной библиотеки может прийти к 
выводу о целесообразности создания тех или иных семантических связей 
между представленными в ней некоторыми информационными объектами.
     
     Определяемые явным и структурированным образом семантические связи 
могут быть представлены и могут динамически поддерживаться как 
самостоятельные информационные объекты электронной библиотеки. Такие 
объекты содержат идентификаторы участвующих в них информационных 
объектов и значения других атрибутов. Объек\-ты-свя\-зи могут быть 
классифицированы, и их свойства определяются значениями атрибутов, 
специфических для каждого класса. 
     
     В результате определения явным образом описанных 
классифицированных семантических связей, как уже отмечалось, порождается 
многослойная семантическая структура контента библиотеки. При этом 
каждому классу связей соответствует некоторый слой этой структуры, который 
наряду с полной структурой связей может служить для наукометрических 
измерений и анализа. В~частности, могут поддерживаться слои, отображающие 
структуру продуцирования научных результатов и другие содержательные 
отношения между научными пуб\-ли\-ка\-ци\-ями, например связи оценки 
пуб\-ли\-ка\-ций научными сотрудниками, связи между час\-тя\-ми научных 
пуб\-ли\-ка\-ций, связи на\-уч\-но-ор\-га\-ни\-за\-ци\-он\-но\-го характера (научное 
     учреж\-де\-ние\,--\,со\-труд\-ни\-ки\,--\,ав\-то\-ры пуб\-ли\-ка\-ций, 
     ав\-то\-ры--пуб\-ли\-ка\-ции) и~др. 
     
     Анализ структуры таких связей в научной электронной библиотеке 
позволяет решать также ряд\linebreak задач, связанных с поддержкой 
     на\-уч\-но-ор\-га\-ни\-за\-ци\-он\-ной деятельности, позволяет авторам 
публикаций более продуктивно использовать име\-ющиеся в электронной 
библиотеке научные информационные ресурсы, дает возможность извлекать из 
контента библиотеки ценную информацию, не содержащуюся в отдельных 
информационных объектах. Например, можно получать полезные 
наукометрические сведения, а также сведения, основанные на анализе 
топологии структуры связей, которые\linebreak достаточно сложно получить иным 
путем. Исследование топологии связей научных публикаций\linebreak позво\-ля\-ет, в 
частности, анализировать процесс формирования научных направлений и школ, 
влияние публикаций тех или иных исследователей на формирование научных 
направлений или теорий. Поддержка структуры семантических связей 
обеспечивает также дополнительные (навигационные) пути доступа 
пользователей к информационным объектам библиотеки. Другое направление, 
где необходима поддержка семантических связей между информационными 
объектами электронной биб\-лио\-те\-ки,~--- это технология <<живых>> 
публикаций, подробно рассмотренная в~[5, 6]. 
     
     Для эффективного использования новых возможностей, которые 
обеспечиваются благодаря под\-держ\-ке в онлайновой электронной \mbox{библиотеке} 
многослойной структуры семантических связей представленных в ней 
информационных объектов, необходимо, чтобы система управления 
электронной библиотекой удовлетворяла определенным требованиям. 
     В~част\-ности, она должна быть способна не только обрабатывать 
запросы относительно семантической структуры контента, но и располагать 
механизмами, позволяющими пользователям самостоятельно устанавливать, 
модифицировать или удалять семантические связи в рамках\linebreak  их полномочий, а 
также обеспечивать мониторинг измене\-ний состояния структуры 
семантических связей. Механизмы мониторинга позволяют автоматически 
оповещать авторов информационных объектов о том, что некоторый их 
информационный объект стал участником вновь учрежденной связи или что 
ликвидирована существующая связь, в которой он являлся участником, либо об 
изменениях значений ее атрибутов. 
     
     Семантическое структурирование контента научных электронных 
библиотек представляет значительно больший интерес, если оно 
поддерживается на представительном репозитории научных информационных 
объектов. Одним из популярных подходов к созданию крупных репозиториев 
научных публикаций, позволяющих интегрировать коллекции ряда научных и 
образовательных учреж\-де\-ний, является подход, основанный на технологии 
открытых архивов~[7]. Поддержка и исследование семантической структуры в 
создаваемом на ее основе крупном интегрированном контенте, формируемом 
на федеративных принципах рядом исследовательских организаций, дает 
возможность изучать структуру результатов научных исследований не только 
отдельных научных коллективов или школ, но и целых направлений науки и 
областей знаний. 
     
     Обеспечение возможностей поддержки в научных электронных 
библиотеках явно представленных классифицированных семантических связей 
между содержащимися в них информационными объектами в сочетании с 
методами мониторинга изменений структуры этих связей и основанными на 
такой структуре новыми функциональными возможностями является, по 
мнению авторов, весьма перспективным новым направлением развития 
научных электронных библиотек. Для эффективного практического 
использования обсуждаемых возможностей необходимо решить следующие 
задачи: 
     \begin{itemize}
\item разработать способы и конкретные форматы представления 
семантических связей между информационными объектами электронной 
библиотеки в виде самостоятельных информационных объектов 
специального типа;
\item создать классификатор семантических связей, которые 
целесообразно поддерживать в научных электронных библиотеках;
\item определить операционные возможности, которые должна 
обеспечивать система управления научной электронной библиотекой для 
того, чтобы извлекать в достаточно полной мере информацию, 
содержащуюся в структуре семантических связей представленных в ней 
информационных объектов. 
\end{itemize}

     В данной работе обсуждается предлагаемый авторами подход к решению 
этих задач и его реализация в среде крупного отечественного онлайнового 
     на\-уч\-но-об\-ра\-зо\-ва\-тель\-но\-го пространства, поддерживаемого 
системой Соционет~[4], основанного на технологии открытых архивов и 
содержащего боль-\linebreak шой объем информационных ресурсов по со\-ци\-ально-эко\-но\-ми\-че\-ской 
тематике. Соционет функ\-ци\-о\-нирует уже более десяти лет и 
приобрела в послед\-ние годы де-фак\-то институциональный\linebreak статус в Отделении 
общественных наук. Информационное пространство Соционет содержит 
также публикации ряда образовательных учреждений и \mbox{других} организаций. 
Соционет стала полигоном для проведения исследований в области 
перспективных технологий электронных библиотек. Постоянно проводятся 
работы по расширению разнообразия типов представляемых в этой системе 
информационных ресурсов и развитию функциональности механизмов 
управления библиотекой. Основные идеи данной работы сформировались на 
основе более ранних публикаций~[8--11] и были детально представлены на 
конференции RCDL-2011~[12]. 

\vspace*{-6pt}

\section{Классификация связей и~известные работы в~данной~области}

\vspace*{-2pt}

  Проблемы структуризации крупных коллекций информационных ресурсов 
электронных биб\-лио\-тек и классификации семантических связей в послед\-ние 
годы привлекают большое внимание исследователей. Известны попытки 
систематической классификации семантических связей между единицами 
информационных ресурсов и/или их компонентами, предпринятые для 
использования в электронных библиотеках, издательских системах, для 
представления знаний в среде Семантического Веба. Рассмотрим наиболее 
известные разработки в этой области.
  
  Прежде всего следует упомянуть работы по распознаванию и классификации 
используемых в научных статьях языковых конструкций (для английского 
языка и отдельных научных дисциплин),\linebreak проводимые средствами 
программного обеспечения компании {Xerox}. Они позволили 
эмпирическим путем выявить некоторые устойчивые виды семантических 
отношений, существующих как между разделами внутри научной статьи, так и 
между статьей и цитируемыми в ней материалами~[13, 14]. Эмпирическая 
классификация поводов цитирования в научных статьях предлагается также 
в~[15]. В~этой работе выделен ряд их типичных вариантов: <<слабость 
цитируемого подхода>>, <<автор использует цитируемую работу как основу 
или исходную точку>> и~др. Другой подход к развитию классификации 
семантических связей реализуется в исследованиях модульности научных 
документов~[16]. 
  
  К рассматриваемому направлению примыкает также рекомендация 
{SKOS} ({Simple Knowledge\linebreak Organization System})~[17] консорциума 
{W3С}. Эта спецификация предназначена для поддержки использования 
систем организации знаний, таких как тезаурусы, схемы классификации, 
таксономии и руб\-ри\-ка\-то\-ры ({Subject Heading Systems}) в среде 
Семантического Веба. Для этой цели определяется концептуальная схема (в 
спецификации она называется \textit{общей моделью данных}) для совместного 
использования и связывания систем организации знаний средствами Веба. 
Унификация концептуальной схемы, определяемой спецификацией 
{SKOS}, создает возможности для относительно нетрудоемкой интеграции 
существующих систем организации знаний в Семантический Веб. 
  
  Специалистами в области биомедицины из Оксфордского и Болонского 
университетов разработан модульный онтологический комплекс {SPAR} 
({the Semantic Publishing and Referencing Ontologies})~[18, 19]. Он состоит 
из восьми независимых повторно используемых детализированных онтологий. 
Фактически каждая из них представляет собой таксономию, описанную на 
языках {OWL2 DL} и {RDF} консорциума {W3C}. Первые 
четыре из них (FaBiO~--- FRBR-aligned Bibiographic Ontology,
где FRBR~--- Functional Requirements for Bibliographic Records);
CiTO~--- Citation Typing Ontology;
BiRO~--- Bibliographic Reference Ontology;  C4O~--- Citation Counting and Context
Characterization Ontology) 
полезны для описания библиографических объектов, библиографических 
записей и источников в списках литературы в пуб\-ли\-ка\-ци\-ях, связей 
цитирования, контекстов цитирования и их связей с релевантными разделами 
цитируемых пуб\-ли\-ка\-ций, а также для организации ссылок в библиографиях, в 
списках источников и в биб\-лио\-теч\-ных каталогах. Остальные онтологии 
(DoCO~--- Document Components Ontology; PRO~--- Publishing Roles Ontology;
PSO~--- Publishing Status Ontology; PWO~--- Publishing Workflow Ontology) служат для создания 
структурированных управляемых словарей классов компонентов документов, 
ролей пуб\-ли\-ка\-ций, со\-сто\-яний пуб\-ли\-ка\-ций и потоков работ в издательских 
процессах.
  
  В Главном госпитале Массачусетса и в Медицинской школе в Гарварде 
разработана онтология {SWAN} ({Semantic Web Applications in 
Neuromedicine})~[20]. Как и {SPAR}, эта онтология\linebreak состоит из набора 
он\-то\-ло\-гий-мо\-ду\-лей. Онтологии, входящие в состав {SWAN}, также 
описаны на языке описания онтологий {OWL DL}. Как указывается в 
спецификации {SWAN}, цель этой онтологии~--- обеспечение в рамках 
Семантического Веба комфортной среды, называемой авторами 
  \textit{со\-ци\-аль\-но-тех\-ни\-че\-ской экосистемой}, которая позволяет 
создавать и сохранять семантический контекст научных коммуникаций, 
обеспечивает доступ к нему, его интеграцию, а также обмен 
неструктурированной или слабоструктурированной цифровой научной 
информацией.
  
  Нужно отметить здесь важную тенденцию конструирования сложных 
онтологий, предназначенных для достаточно широкой сферы применения: они 
строятся по модульному принципу. Такой подход облегчает их повторное 
использование. Обыч\-но не требуется 
использовать полную онтологию и берется только нужный ее модуль. При этом 
модульность облегчает также интеграцию с другими онтологиями. Примером 
такой интеграции может служить комплекс {SPAR}, в котором 
использованы элементы {SWAN}. В~свою очередь, в {SWAN} 
используется {SKOS}. 
  
  Следует, наконец, упомянуть также имеющий отношение к обсуждаемому в 
этом разделе вопросу проект CERIF (Common European Research Information Format)~[21], 
который в 1980--1990-е~годы 
реализовывался при поддержке Европейской комиссии, а в 2000~г.\ был 
передан ею под опеку международной научной организации {euroCRIS}. 
Главная цель этого проекта фактически заключается в создании стандарта так 
называемой \textit{полной модели данных} ({Full Data Model}), которая 
рассматривается как единая основа создания информационных сис\-тем 
({Current Research Information Systems}, {CRIS}) для поддержки 
  на\-уч\-но-ор\-га\-ни\-за\-ци\-он\-ной деятельности в разных странах и 
научных организациях. Благодаря стандартизации модели данных 
обеспечивается интероперабельность таких систем. В~последнее время в 
проекте {CERIF} уделяется большое внимание семантическим аспектам 
созданной модели. Для этой цели разработаны онтологии 
{CERIF}~[22, 23].
  
  Рассмотренные результаты в области классификации возможных 
семантических связей между научными публикациями и/или другими 
продуктами научной деятельности могут использоваться в качестве основы для 
семантического структурирования контента научных электронных библиотек. 
В~разработке классификатора семантических связей в обсуждаемом в этой 
работе проекте авторы использовали фрагменты рассмотренных онтологий~--- 
{CiTO}, {DoCo}, {SWAN}, {SKOS} и {CERIF}. 
Наиболее существенную часть предлагаемого классификатора определяют 
фрагменты онтологий {CiTO} и {DoCo}. 
  
  Онтология {CiTO}~[24, 25] 
обеспечивает возможности для характеристики природы связей цитирования, 
как фактологических (например, <<цитирует как источник данных>> или 
<<цитирует как основополагающую>>), так и риторических (например, 
<<уточняет>> или <<опровергает>>). При этом учитываются как 
непосредственные и явные связи цитирования, так и косвенные и неявные. 
Онтология {DoCO}~[26] 
классифицирует составные части документов. Она предоставляет 
структурированный управляемый словарь классов их компонентов, например 
<<Введение>>, <<Обсуждение>>, <<Благодарности>>, <<Список 
использованных источников>>, <<Приложение>> и~т.\,д.
  
  Результаты рассмотренных исследований могут быть использованы для 
классификации некоторых видов связей на множестве не только текстовых 
научных информационных объектов. Это обстоятельство имеет в 
рассматриваемом случае существенное значение, поскольку, как отмечалось 
ранее, интерес представляют также связи, участниками которых являются 
профили организаций и их сотрудников~--- авторов и пользователей 
библиотеки, а также информационные объекты других типов, не являющиеся 
текстовыми документами. 
  
  Классификатор связей в системе Соционет предусмат\-ри\-ва\-ет разбиение 
множества семантических связей информационных объектов (текстовых 
информационных объектов, профилей пользователей и организаций и 
информационных объектов других типов) на категории (оценочные связи,
   на\-уч\-но-ор\-га\-ни\-за\-ци\-он\-ные связи и~др.). Каждой категории 
соответствует некоторый набор классов связей. Эти наборы представляются в 
виде словарей классов связей. При необходимости в процессе 
функционирования сис\-те\-мы может пополняться состав категорий и словари 
могут дополняться новыми классами связей. Более подробно принципы 
организации и содержание классификатора семантических связей, 
используемого в системе Соционет, обсуждается в~[12].
  
  Помимо рассмотренных выше работ, появляются также новые публикации, 
посвященные затронутой проблеме. Однако авторам не известны проекты, в 
которых реализован описанный выше комплекс возможностей использования 
классифицированных семантических связей между информационными 
объектами научных электронных биб\-лиотек.

\vspace*{-12pt}

\section{Семантические связи как информационные объекты 
библиотеки}

  В электронных библиотеках традиционно с помощью гиперссылок 
поддерживаются связи между каталогами и описываемыми в них 
информационными объектами. В~сис\-те\-ме Соционет таким же образом 
поддерживаются связи цитирования, связи с профилями авторов и организаций 
и некоторые другие. Для этого в Соционет имеются метаданные, 
описывающие информационные объекты, их авторов (профили авторов), 
коллекции информационных ресурсов, организации~--- мес\-та работы авторов 
(профили организаций) и~др. В~таком случае связи между информационными 
объектами представляются как атрибуты метаданных, описывающих эти 
информационные объекты. С~использованием связей такого вида можно 
анализировать структуру связей, осуществлять нау\-ко\-мет\-ри\-че\-ские измерения, 
визуализировать структуру связей. 
{\looseness=1

}
  
  Однако при таком традиционном способе представления связей явным 
образом не отображается семантика связей. Например, для связи цитирования 
одного информационного объекта с другим отсутствует информация, 
характеризующая цель цитирования, оценку цитируемой работы и другие 
характеристики. Предлагаемая далее модель связей между информационными 
объектами научной электронной библиотеки устраняет это ограничение.
  
  В общем случае связи могут представляться двумя способами. При 
использовании первого способа, представленного выше, данные, описывающие 
связи, содержатся в метаданных одного из связываемых объектов, например в 
метаданных исходного объекта связи. Однако поскольку в электронной 
библиотеке, построенной на федеративных принципах, изменять метаданные 
может только их автор или уполномоченный автором администратор 
информационных ресурсов, то при этом способе только они и могут создавать 
связи этого объекта с другими информационными объектами. 
  
  При втором способе создаваемые связи представляются как самостоятельные 
информационные объекты. Такой способ является более универсальным и 
предпочтительным, так как он охватывает все многообразие возможных 
ситуаций, обеспечивает более богатые возможности анализа структуры связей, 
которые значительно проще реализуются, и он позволяет создавать связи 
любому пользователю, поскольку при этом не затрагиваются недоступные ему 
метаданные связываемых объектов. 
  
  Описание связи в обоих представлениях должно включать уникальный 
идентификатор целевого объекта связи, а также может включать атрибуты, 
характеризующие семантику связи, различного рода комментарии и~пр. Если 
связь создается как самостоятельный информационный объект, то ее описание 
в дополнение к уже перечисленному должно включать: уникальный 
идентификатор данного объек\-та-свя\-зи в библиотеке; уникальный 
идентификатор пользователя, создающего данную %\linebreak 
связь; уникальный 
идентификатор исходного объек\-та связи (рассматриваются ориентированные 
бинарные связи), а также дата создания или изменения связи. Для описания 
семантики связи используется имя класса связи, выбираемое из 
поддерживаемых контролируемых словарей, а также значения свойств 
конкретного экземпляра связи, определяемые пользователем. Полномочия на 
создание связей между информационными объектами предоставляются только 
зарегистрированным в библиотеке пользователям, что обеспечивает 
автоматическую фиксацию идентификатора пользователя, создающего связи, 
при его входе в сис\-тему. 
{\looseness=1

}
  
  Рассмотрим процедуру создания связи между двумя информационными 
объектами в системе Соционет, в которой реализованы оба способа 
представления связей. При первом способе создание связи осуществляется 
автором исходного информационного объекта связи или его представителем. 
Рассмотрим процедуру второго способа. 
  
  Множество параметров, влияющих на создание связи, включает: тип 
исходного объекта связи; тип целевого объекта связи; множество категорий 
связей, учрежденных в системе для заданной пары типов связываемых 
объектов; множество словарей классов связей, предусмотренных в системе для 
связей заданной категории; множество классов связей в словаре, выбранном 
для создания связи между объектами заданных типов.
  
  Рассматриваемая процедура состоит из сле\-ду\-ющих шагов: 
  \begin{enumerate}[1.]
\item  Пользователь выбирает пару связываемых информационных объектов.
  \item Из множества категорий связей, предусмотренных в системе для 
выбранной пары типов объектов, выбирается конкретная категория. Если 
подходящей категории не существует, пользователь имеет возможность 
предложить новую категорию и предоставить соответствующий ей словарь 
классов связей для включения в систему. Это предложение вступит в силу 
только после одобрения администратором системы. 
  \item Если подходящая категория связей выбрана, то открывается 
соответствующий словарь классов связей. 
  \item Если в словаре имеется подходящий класс связей, характеризующий 
требуемое семантическое отношение между заданной парой объектов, то 
пользователь его выбирает. Если же такой класс отсутствует в данном словаре, 
пользователь может предложить подходящий класс связей для пополнения 
данного словаря. Предложение вступит в силу только после одобрения его 
администратором системы или соответствующего словаря. 
  \item По желанию пользовать может привести в описании связи 
комментарий, объясняющий мотивы ее создания.
  \item Сформированный информационный объект-связь сохраняется. При 
этом система запрашивает у пользователя, в какую его коллекцию следует 
поместить созданный объект, а также уникальный идентификатор этого 
объекта в соответствующей коллекции. 
  \end{enumerate}
  
  Рассмотренная процедура обеспечивает создание информационного объекта, 
представляющего требуемую связь среди других объектов библиотеки. При 
этом также осуществляется проверка непротиворечивости семантики новой 
связи с уже существующими связями между данными объектами, созданными 
тем же пользователем. 
  
  Хотя формирование семантических связей между информационными 
объектами требует определенных трудозатрат, в результате информативность 
контента научной электронной библиотеки существенно повышается. 
Создаются также дополнительные возможности для анализа семантической 
структуры контента. 
  
  Поддержка развитой структуры семантических связей в научной электронной 
библиотеке с достаточно представительным контентом позволяет в результате 
их анализа осуществлять наукометрические измерения, использовать 
технологии <<живых>> публикаций~[5, 6], а также получать качественно 
новую информацию о развитии научных знаний в конкретных областях 
исследований и о вкладе отдельных ученых.
  
  В описанной выше процедуре предполагается, что любой 
зарегистрированный пользователь научной электронной библиотеки может 
создавать связи между любыми ее информационными объектами. При 
определении семантики связей их создатель выражает свое субъективное 
мнение, которое в некоторых случаях может вызывать несогласие или протест 
как авторов объектов, участвующих в данных связях, так и других членов 
научного сообщества. Например, могут вызывать протесты случаи, когда 
устанавливаются семантические связи, несущие негативную оценку некоторого 
научного произведения (опровержение, высмеивание, обвинение в плагиате 
и~т.\,п.). 
  
  Как известно, научная истина устанавливается в борьбе мнений. Поэтому 
если научное сообщество начинает использовать подобные технические 
средства, то с учетом потенциального конфликта интересов научная среда 
должна предоставлять ученым равные права и одинаковый доступ к 
использованию этих средств, а также надежную фиксацию профессиональной и 
социально-этической ответственности ученого за характер использования им 
данных средств.
  
  Для выполнения данных принципов, по мнению авторов, крайне важно 
обеспечить модерирование всех создаваемых связей с точки зрения соблюдения 
авторами научной этики, а также наличия в создаваемых связях признаков 
добавленной научной <<стоимости>> или научного вклада (исключение связей 
с чисто эмоциональным или ненаучным содержанием). 
  
  В системе Соционет пользователи создают связи в своем личном (закрытом 
от свободного доступа) пространстве. Такое пространство с сервисами для его 
использования предусматривается для авторов или администраторов 
информационных ресурсов в системе и называется их Личной зоной. 
Создаваемые в Личной зоне объек\-ты-свя\-зи предлагаются далее для 
включения в общедоступные информационные ресурсы. Они становятся 
общедоступными только после одобрения модератором. 

\section{Свойства семантических связей}

  Обсуждаемая в данной работе структура семантических связей, формируемая 
и поддерживаемая над контентом электронной библиотеки, по\-рож\-да\-ет\-ся 
бинарными ориентированными семантическими связями между 
информационными объектами библиотеки, составляющими ее коллекции 
информационных ресурсов. 
  
  Как уже отмечалось, семантические связи, опре\-де\-ля\-емые в библиотеке 
явным образом в виде структурированных данных, представляются и могут 
динамически поддерживаться как самостоятельные информационные объекты. 


Информационные объек\-ты-свя\-зи категоризируются, как было описано выше, 
и в рамках каж\-дой категории классифицируются в соответствии с их 
семантикой. Таким образом, каждый экземпляр со\-зда\-ва\-емых в библиотеке 
связей относится к ка\-кой-ли\-бо категории, а в рамках категории~--- к 
  ка\-ко\-му-ли\-бо классу связей этой категории. Свойства экземпляров 
  объек\-тов-свя\-зей задаются значениями атрибутов, определенных для 
соответствующих классов связей. Между двумя информационными объектами 
библиотеки может быть определено несколько связей одной или нескольких 
категорий.
{ %\looseness=1

}
  
  Каждому экземпляру объекта-связи при его создании присваивается 
некоторое значение уникального идентификатора, а значения его атрибутов, 
наряду с другими возможными свойствами,\linebreak указывают категорию и класс 
представляемой им\linebreak связи, идентификатор пользователя, который создает этот 
объект-связь, идентификаторы исходного и целевого информационных 
объектов биб\-лио\-теки, участвующих в данной связи, дату ее\linebreak со\-здания.
{ %\looseness=1

}
  
  Структура семантических связей, под\-дер\-жи\-ва\-емых в электронной 
библиотеке, динамична. Могут создаваться новые, а также обновляться или 
ликвидироваться существующие связи~--- мнения авторов связей могут 
изменяться с течением времени. Динамичность структуры связей обусловлена и 
пополнением контента библиотеки новыми информационными объектами~--- 
потенциальными участниками связей. 
  
  В некоторых категориях связей существуют клас\-сы связей с противоречивой 
семантикой. Например, к категории оценочных связей относятся связи между 
информационными объектами, которые выражают одобрение или согласие 
исходного объекта с целевым, а также связи, выражающие опровержение 
результатов, представленных в целевом информационном объекте. 
Естественно, что между двумя информационными объектами не могут 
одновременно существовать связи этих\linebreak двух классов, созданные одним и тем 
же пользователем. Возникновение таких ситуаций долж\-ны предотвращать 
системные механизмы библиотеки. В~то же время вполне возможны 
семантически противоре\-чивые связи между двумя информационными 
объектами, созданные разными пользователями. Системные механизмы 
должны контролировать выполнение и некоторых других ограничений на 
создание, обновление и ликвидацию экземпляров связей. К~ним относятся, в 
частности, ограничения доступа~--- для выполнения таких операций 
пользователь должен обладать необходимыми полномочиями. 
  
  Каждая семантическая категория связей между информационными 
объектами библиотеки и каж\-дый относящийся к ней класс связей, как уже 
отмечалось, образуют некоторые слои в структуре связей. Таким образом, в 
электронной библиотеке, механизмы которой обладают рас\-смат\-ри\-ва\-емой\linebreak 
функциональностью, поддерживается многослойная структура семантических 
связей принадлежащих ей информационных объектов, которая при достижении 
достаточной ее представительности\linebreak становится весьма значимым полигоном 
для анализа и поддержки научной и на\-уч\-но-ор\-га\-ни\-за\-ци\-он\-ной 
деятельности. 

\section{Реализация предлагаемого подхода в~системе Соционет}

  Для формирования в электронной библиотеке и продуктивного 
использования многослойной структуры семантических связей 
информационных объектов ее контента необходимо, чтобы система управления 
библиотекой включала механизмы, предоставляющие необходимые 
операционные возможности. Кратко рассмотрим состав и функции таких 
механизмов, которые предусмотрены в системе Соционет.
  
  \smallskip
  
  \textbf{Механизмы формирования и поддержки словарей связей.} 
Классификатор семантических связей в сис\-те\-ме Соционет, как уже отмечалось, 
имеет модульную структуру и представлен в виде совокупности управляемых 
словарей. 
  
  Основу разработанных словарей составляют упоминавшийся выше комплекс 
онтологий {SPAR} (в частности, онтологии {CiTO} и {DoCo}), 
спецификация {SKOS} консорциума {W3C}, онтология проекта 
{SWAN}, а также один из разделов {CERIF}, опре\-де\-ля\-ющий 
семантику связей. 
  
  Каждый из словарей соответствует некоторой предусмотренной в 
классификаторе категории связей и содержит имена относящихся к ней классов 
связей. Рассматриваемые механизмы позволяют системному администратору 
формировать и модифицировать эти словари. Пользовательский интерфейс 
механизмов создания связей предоставляет доступ к словарям и справочной 
информации, необходимой для их корректного использования. 
  
  \smallskip
  
  \textbf{Механизмы управления связями.} Эти механизмы позволяют 
авторизованному пользователю создавать в модерируемом режиме связи между 
информационными объектами библиотеки. Как указывалось выше, связи 
создаются как информационные объекты специального типа. При создании 
новой связи используются управляемые словари классов связей. Новая связь 
создается только при условии, если ее создание не нарушает заданных 
ограничений (см.\ разд.~5). 
  
  В системе Соционет поддерживается множество типов информационных 
объектов~--- статьи, монографии, диссертации или авторефераты диссертаций, 
профили пользователей, профили организаций, рубрикаторы, научные 
артефакты, цитаты, информационные объек\-ты-свя\-зи и~т.\,п. Для каждой 
пары типов информационных объектов допустимы только определенные 
классы связей. При попытке создания конкретного экземпляра связи 
проверяется его допустимость. 
  
  Механизмы управления связями позволяют также ликвидировать 
существующие связи и обновлять значения их атрибутов. Можно, например, 
изменить текст комментария. В~рассматриваемой группе механизмов важное 
место занимает механизм мониторинга состояния структуры связей. При 
появлении новой связи, удалении связи или некоторых изменениях атрибутов 
связей этот механизм генерирует сообщения авторам информационных 
объектов~--- участников таких связей, стимулируя тем самым их реакцию на 
эти события. 
  
  Фактически предлагаемый подход предусматривает создание в сис\-те\-ме 
Соционет наряду с уже много лет функционирующим открытым крупным 
репозиторием метаданных научных статей, монографий, персональных 
профилей, профилей организаций и других информационных объектов также и 
открытого репозитория семантических связей, который является ценным 
информационным источником структурного анализа представленного в 
системе корпуса научных знаний. 
  
  \smallskip
  
  \textbf{Механизмы обработки запросов.} Эти механизмы выполняют 
довольно большой набор функций, позволяющих получать разнообразную 
информацию о структуре связей в библиотеке. Прежде всего, это 
статистическая информация. В~системе Соционет имеются развитые сервисы 
для нау\-ко\-мет\-ри\-че\-ских измерений. Они обсуждаются подробно 
  в~\cite{8-kog, 27-kog}. Измерения на основе структуры семантических связей 
существенно обогащают аналитические возможности системы. Можно, 
например, запросить количество связей заданного класса или некоторой 
категории, исходящих из данного информационного объекта библиотеки либо 
входящих в него. Например, можно узнать, сколько имеется положительных 
или негативных оценок данной работы. 
  
  Другая группа запросов позволяет получить перечень информационных 
объектов, связанных с заданным объектом как исходным или целевым в связях 
заданных классов или категорий. Запросы этого вида позволяют, например, 
выяснить, на результаты каких публикаций опирается некоторая конкретная 
работа или, наоборот, в каких публикациях получены результаты, основанные 
на данной работе. При этом можно учитывать как непосредственные, так и 
транзитивные связи. В~качестве критерия отбора связей или одного из термов 
критерия может также использоваться идентификатор автора связей. Таким 
образом может быть получена разнообразная аналитическая информация о 
структуре различных областей исследований, вкладе в их развитие конкретных 
ученых, о процессе эволюции интересующих областей знаний и~т.\,д. 
Исследования в этой области планируется развивать на основе системы 
Соционет. 
  
  Следует здесь упомянуть проект {SciVal} компании 
{Elsevier}~\cite{28-kog}. Функциональный модуль {SciVal Spotlight} 
созданного компанией программного продукта позволяет осуществлять анализ 
научной деятельности исследовательского учреждения или страны в целом, на 
основе которого может оцениваться эффективность исследований и могут 
приниматься стратегические решения. Принятый в этом интересном проекте 
подход основан на анализе структуры связей цитирования публикаций 
субъектов научной деятельности, поддерживаемых в индексе цитирования 
Scopus. Однако при этом используются традиционные <<немые>> связи~--- 
связи, не несущие семантики. В~этом смысле пред\-ла\-га\-емый авторами подход 
выгодно отличается, так как обеспечивает более дифференцированный анализ, 
результаты которого учитывают семантику связей. 
  
  \smallskip
  
  \textbf{Механизмы визуализации и анализа графа связей.} Важную группу 
запросов составляют запросы операций над полным графом связей. Здесь 
можно решать различные задачи, связанные как с анализом топологии графа и 
вычленением подграфов с заданными свойствами, так и с визуализацией 
подграфов полного графа. Например, можно вычленить и визуализировать из 
многослойной структуры связей слой, соответствующий связи некоторого 
класса, такой как связь, указывающая на использование одного 
информационного объекта как основополагающего для других объектов. 
Можно также запросить подграф, образованный связями, относящимися к 
категории развития научных результатов, и указать, что ему должна 
принадлежать некоторая имеющаяся в библиотеке общепризнанная 
основополагающая публикация в некоторой области исследований. 
Полученный подграф будет характеризовать логику развития данной области 
науки, если, конечно, в библиотеке будут достаточно основательно 
представлены публикации, относящиеся к этой области. Еще одним примером 
операций над полным графом связей библиотеки является операция 
вычленения из него подграфа связей, созданных данным пользователем, 
возможно, с указанием в запросе также категории или конкретного класса 
связей.
  
  Отметим, наконец, что визуализация графа связей или некоторого его 
подграфа может быть использована для навигации в структуре связей, а также 
просмотра свойств отдельных экземпляров связей и участвующих в них 
информационных объектов.

%\vspace*{-12pt}

\section{Заключение}

  В работе предложен подход, обеспечивающий повышение информационной 
ценности контента
 научной электронной библиотеки путем поддержки в ней 
классифицированных семантических связей между содержащимися в ее 
коллекциях информационными объектами, а также создания механизмов 
управления связями и обработки информации, носителями которой они 
являются.
  
  Реализующая этот подход технология позволяет более эффективно 
использовать существующий\linebreak корпус электронных знаний благодаря 
визуализации семантических связей между научными произведениями, 
навигации в такой многослойной семантической структуре, созданию основы 
для\linebreak получения качественно новых наукометрических измерений, а также для 
структурного исследования электронного корпуса научных знаний. 
  
  Предлагаемая технология обеспечивает также\linebreak естественный механизм 
мотивации научных коммуникаций в исследовательском сообществе в процессе 
создания и обсуждения новых научных результа\-тов. Она хорошо согласуется 
также с технологией <<живых>> публикаций, для поддержки которой 
применимы реализующие ее механизмы. 

{\small\frenchspacing
{%\baselineskip=10.8pt
\addcontentsline{toc}{section}{Литература}
\begin{thebibliography}{99}
  
\bibitem{1-kog}
ACM Digital Library. {\sf http://dl.acm.org}.

\bibitem{2-kog}
Научная электронная библиотека eLibrary.ru. {\sf http://elibrary.ru}.

\bibitem{3-kog}
Академия Google. {\sf http://scholar.google.com}.

\bibitem{4-kog}
\Au{Паринов С.\,И.}
СОЦИОНЕТ.РУ как модель информационного пространства 2-го поколения~// 
Информационное общество, 2001. Вып.~1. С.~43--45. 

\bibitem{5-kog}
\Au{Паринов С.\,И., Когаловский М.\,Р.}
Технология поддержки электронных научных публикаций как <<живых>> документов~// 
Электронные библиотеки: перспективные методы и технологии, электронные коллек\-ции 
(RCDL-2009): Труды XI Всеросс. науч. конф. (Петрозаводск, 17--21~сентября 2009).~--- 
Петрозаводск: КарНЦ РАН, 2009. С.~53--58.

\bibitem{6-kog}
\Au{Паринов С.\,И., Когаловский~М.\,Р.}
<<Живые>> документы в электронных библиотеках~// Прикладная информатика, 2009. №\,6. 
С.~123--131.

\bibitem{7-kog}
Open Archives Initiative. {\sf http://www.openarchives.org}.

\bibitem{8-kog}
\Au{Когаловский М.\,Р., Паринов~С.\,И.}
Метрики онлайновых информационных пространств~// Экономика и математические 
методы, 2008. Вып.~2. С.~108--120.

\bibitem{9-kog}
\Au{Когаловский М.\,Р., Паринов~С.\,И.}
Использование связей цитирования для наукометрических измерений в системе Соционет~// 
Депонировано в Соционет, 2009. 
{\sf http://socionet.ru/publication.xml?h=repec:\linebreak rus:rssalc:web-32}.

\bibitem{11-kog} %10
\Au{Parinov S.}
The electronic library: using technology to measure and support Open Science~// World Library 
and Information Congress: 76th IFLA General Conference and Assembly Proceedings. 
(10--15~August 2010, Gothenburg, Sweden). P.~1--13. 

\bibitem{10-kog} %11
\Au{Паринов С.\,И.}
Концепция виртуальной научной\linebreak среды <<Открытая Наука>>~// Научный сервис в сети 
Интернет: суперкомпьютерные центры и задачи: \mbox{Труды} междунар. суперкомпьютерной 
конф. (Новороссийск, 20--25~сентября 2010).~--- М.: МГУ, 2010. С.~473--481. 


\bibitem{12-kog}
\Au{Паринов С.\,И., Когаловский~М.\,Р.}
Технология семантического структурирования контента научных электронных библиотек~// 
Электронные библиотеки: перспективные методы и технологии, электронные коллекции 
(RCDL'2011): Труды XIII Всеросс. науч. конф. (Воронеж, 19--22~октября 2011).~--- Воронеж: 
ВГУ, 2011. С.~197--206.

\bibitem{13-kog}
\Au{\protect{\ptb\textit{\AA}}str$\ddot{\mbox{o}}$m~F., 
S$\acute{\mbox{a}}$ndor\ \protect{\ptb\textit{\'A}}}. 
Models of scholarly communication and citation analysis~// ISSI 2009: 12th  Conference 
(International)  of the International Society for Scientometrics and Informetrics Proceedings. 
Vol.~1. {\sf 
http://lup.lub.lu.se/luur/download?func=downloadFile\linebreak \&recordOId=1459018\&fileOId=1883080}.

\bibitem{14-kog}
\Au{S$\acute{\mbox{a}}$ndor\ \protect{\ptb\textit{\'A}}., Kaplan A., Rondeau~G.}
Discourse and citation 
analysis with concept-matching, Citeseer. {\sf 
http://\linebreak citeseerx.ist.psu.edu/viewdoc/download?doi=10.1.1.67.\linebreak
7518\&rep=rep1\&type=pdf}.

\bibitem{15-kog}
\Au{Teufel S., Siddharthan A., Tidhar~D.} 
Automatic classification of citation function~// 2006 Conference on Empirical Methods in Natural 
Language Processing Proceedings. {\sf http://portal.acm.org/citation.cfm?id=1610091}.

\bibitem{16-kog}
\Au{De Waard A., Kircz J.}
Modeling scientific research articles~--- shifting perspectives and persistent issues~// ELPUB 2008 
Conference on Electronic Publishing Proceedings.~--- Toronto, Canada, 2008. {\sf 
http://elpub.scix.net/\linebreak data/works/att/234\_elpub2008.content.pdf}.

\bibitem{17-kog}
SKOS~--- Simple Knowledge Organization System. {\sf http://www.w3.org/TR/skos-reference}.

\bibitem{18-kog}
\Au{Shotton D.}
Introducing the semantic publishing and referencing (SPAR) ontologies. October~14, 2010. {\sf 
http://\linebreak opencitations.wordpress.com/2010/10/14/introducing-\linebreak the-semantic-publishing-and-referencing-spar-ontologies}.

\bibitem{19-kog}
\Au{Shotton D., Peroni S.}
Semantic annotation of publication entities using the SPAR (Semantic Publishing and Referencing) 
ontologies~// Beyond the PDF Workshop. La Jolla. January 19, 2011. {\sf 
http://imageweb. zoo.ox.ac.uk/pub/2010/Publications/Shotton\&Peroni\_\linebreak semantic\_annotation\_of\_publication\_entities.pdf}.

\bibitem{20-kog}
Semantic Web Applications in Neuromedicine (SWAN) ontology. W3C Interest Group Note. 
October 20, 2009. {\sf http://www.w3.org/TR/2009/NOTE-hcls-swan-\linebreak 20091020}.

\bibitem{21-kog}
CERIF 2008~--- Final Release (1.2). {\sf 
http://www.\linebreak eurocris.org/Index.php?page=CERIF2008\&t=19}.

\bibitem{22-kog}
CERIF-2008-1.3 Ontology. {\sf http://spi-fm.uca.es/\linebreak neologism/cerif\#}.

\bibitem{23-kog}
CERIF-2008-1.3 Semantic Vocabulary. {\sf http://\linebreak spi-fm.uca.es/neologism/semcerif\#}.

\bibitem{24-kog}
\Au{Shotton D., Peroni~S.}
CiTO, the Citation Typing Ontology, v.~2.0. {\sf http://purl.org/spar/cito}.

\bibitem{25-kog}
\Au{Shotton D.}
CiTO, the Citation Typing Ontology~// J.~Biomedical Semantics, 2010. Vol.~1. Suppl.~1. P.~6. 
{\sf http://www.jbiomedsem.com/content/1/S1/S6}.

\bibitem{26-kog}
\Au{Shotton D., Peroni~S.}
DoCO, the Document Components Ontology. {\sf http://speroni.web.cs.unibo.it/cgi-bin/ lode/req.py?req=http:/purl.org/spar/doco}.

\bibitem{27-kog}
\Au{Когаловский М.\,Р., Паринов~С.\,И.}
Использование связей цитирования для наукометрических измерений в системе Соционет. 
Депонировано в Соционет, 2009. {\sf 
http://socionet.ru/publication.xml?h=repec:rus:rssalc: web-32}.

\label{end\stat}

\bibitem{28-kog}
SciVal.  {\sf 
http://www.elsevier.com/wps/find/electronic\linebreak productdescription.cws\_home/720941/description\linebreak \#description}.
 \end{thebibliography}
}
}


\end{multicols}  %3
\def\stat{skachkov}

\def\tit{ОБ ИНТЕГРАЦИИ ГЕОГРАФИЧЕСКИХ МЕТАДАННЫХ ПОСРЕДСТВОМ 
РЕТРОСПЕКТИВНОГО ТЕЗАУРУСА$^*$}

\def\titkol{Об интеграции географических метаданных посредством 
ретроспективного тезауруса}

\def\autkol{Д.\,М. Скачков, О.\,Л.~Жижимов}
\def\aut{Д.\,М. Скачков$^1$, О.\,Л.~Жижимов$^2$}

\titel{\tit}{\aut}{\autkol}{\titkol}

{\renewcommand{\thefootnote}{\fnsymbol{footnote}}\footnotetext[1]
{Работа выполнена при поддержке РФФИ, грант №\,10-07-00302-а.}}


\renewcommand{\thefootnote}{\arabic{footnote}}
\footnotetext[1]{Институт вычислительных технологий СО РАН, danil.skachkov@gmail.com}
\footnotetext[2]{Институт вычислительных технологий СО РАН, zhizhim@sbras.ru}
       
      
  \Abst{Обсуждаются вопросы, связанные с построением интероперабельного тезауруса 
географических наименований, включающего геометрические данные географических 
объектов, в том числе и ретроспективные. Определяются основные требования к подобному 
тезаурусу, производится обзор существующих решений исходя из описанных требований, 
формулируются основные позиции соответствующего профиля для организации доступа к 
тезаурусу, приводится реляционная схема, предназначенная для хранения данных 
тезауруса.}
  
  \KW{географические метаданные; интеграция; ретроспективное геокодирование; 
тезаурус}
  
  
  \vskip 14pt plus 9pt minus 6pt

      \thispagestyle{headings}

      \begin{multicols}{2}

            \label{st\stat}


  \section{Введение}
  
  В настоящее время в связи с возрастающей потребностью общества в информационном 
обеспечении, в том числе и связанном с географическим аспектом информации, все большую 
актуальность приобретают разработки, направленные на интеграцию <<негеографических>> 
информационных сис\-тем с информационными сис\-те\-ма\-ми, изначально ориентированными на 
обработку географической информации. Под <<негеографическими>> 
информационными системами здесь и в дальнейшем будем понимать информационные 
системы, для которых изначально не предполагалось использование пространственных 
данных. К~таким сис\-те\-мам относятся, например, электронные биб\-лио\-те\-ки. Добавление 
географического аспекта к информации, хранящейся в таких системах, позволило бы 
существенно повысить функциональность их навигационных, поисковых и 
визаулизационных сервисов. Подобная интеграция даст возможность, к примеру, 
производить поиск по заданному географическому региону~[1], отображать на карте 
материалы, относящиеся к соответствующим точкам на поверхности Земли (как это делается 
на {Google Maps}), повысить релевантность результатов поиска.
  
  Следует заметить, что существующие в настоящее время программные комплексы, явным\linebreak 
образом не связанные с географическими информационными сис\-те\-ма\-ми (ГИС), не содержат необходимой функциональности по хранению и 
обработке географических данных. Наделение же их требуемой функциональностью 
осложняется отсутствием единых стандартов на поиск и представление данных, связанных с 
географическим аспектом, а также отсутствием четкого описания технологии интеграции как 
таковой~[2].
  
  Таким образом, разработка технологии, обеспечивающей обработку географического 
аспекта информации в <<негеографических>> информационных системах общего 
назначения, является актуальной и перспективной.

\vspace*{-6pt}
  
  \section{Пути интеграции географических данных}
  
  \vspace*{-1pt}
  
  Прежде чем описывать варианты внедрения геогра\-фи\-че\-ской информации в объекты 
информационной системы, разграничим две важные со\-став\-ля\-ющие любого объекта, 
характерного для рас\-смат\-ри\-ва\-емых систем.
  
  Вся информация об объекте может быть разделена на две составляющие:
  \begin{enumerate}[(1)]
\item контент~--- информационное наполнение объекта;\\[-14pt]
\item контекст~--- среда, в которой существует объект.
\end{enumerate}

  В дальнейшем будем считать, что географический аспект информации может быть 
зафиксирован на уровне метаданных, описывающих контент и контекст. При этом 
<<географические>> метаданные объекта могут быть заданы двумя способами:
  \begin{enumerate}[(1)]
\item с помощью геометрического описания географического объекта на основе 
координат;\\[-14pt]
\item с помощью ссылки на элемент некоторого тезауруса, включающего географические 
назва-\linebreak\vspace*{-12pt}

\pagebreak

\noindent
ния соответствующих объектов. Так как термин <<тезаурус>> может употребляться в 
различных значениях, в данной работе под тезаурусом будем понимать 
ин\-фор\-ма\-ци\-он\-но-поиско\-вый тезаурус. Ин\-фор\-ма\-ци\-он\-но-поиско\-вый 
тезаурус~--- это нормативный словарь, явно указывающий отношения между терминами и 
предназначенный для описания содержания документов и поисковых запросов~[3].
  \end{enumerate}
  
  Оба варианта в применении к задаче интеграции имеют как положительные, так и 
отрицательные стороны.
  
  \textit{Первый} вариант исключает неоднозначные толкования, но в то же время он не 
очень удобен~по причине необходимости внесения существенных изменений в уже 
существующие информационные системы. \textit{Второй} вариант не является 
однозначным, но может быть реализован на базе существующих парадигм информационных 
систем при условии их небольшой модернизации, а также\linebreak облада\-ет большей гибкостью. 
Хотя реализация тезауруса географических названий сопряжена с большим объемом работ, 
но возможность его повторного использования оправдывает все затраты. Более того, при 
реализации первого варианта интеграции тезаурус географических наименований также 
необходим для определения координат географических объектов, имеющих отношение к 
записям информационной системы. Поэтому в данной работе речь пойдет именно о втором 
варианте.
  
  \section{Препятствия при интеграции посредством тезауруса}
  
  Существует множество тезаурусов географических наименований, но сложность их 
использования применительно к данной задаче заключается в том, что географический 
аспект объектов, хранящихся в негеографических информационных системах, зачастую 
относится не к текущему моменту, а к моментам времени прошедшим. Однако с течением 
времени могут изменяться как географические названия, так и границы географических 
объектов. Будем называть это изменение свойств с течением времени 
\textit{ретроспективным аспектом информации}. В~то же время большинство тезаурусов 
содержит сведения, относящиеся только к текущему моменту времени, т.\,е.\ не учитывает 
ретроспективный аспект информации. Данная особенность препятствует использованию 
существующих тезаурусов географических наименований в подобных системах. 
  
  Следует заметить, что любые изменения географических названий и геометрических 
объектов, ассоциированных с ними, как правило, связаны с каким-либо нормативным 
документом.
  
  Более того, в существующих тезаурусах координаты географического объекта чаще всего 
задаются в виде точки, в то время как реальные координаты объекта представляют собой 
далеко не точку, а, в общем случае, некоторую область. Что, конечно же, также уменьшает 
полезность таких тезаурусов при проведении поиска. Поэтому более предпочтительным 
будет тезаурус, где положение объектов задано с помощью координат границ области, 
занимаемой объектом.
  
  Для задач поиска полезными будут также данные о том, как географические объекты 
расположены относительно друг друга. Например, если производится поиск по некоему 
региону, целесообразно считать релевантными также и элементы, относящиеся к 
географическим объектам, лежащим в целевом регионе.
  
  Таким образом, для использования в информационных системах общего назначения 
географического аспекта в его любом виде необходим справочный аппарат (тезаурус), 
который включает в себя не только географический аспект информации, но и ее временной 
(ретроспективный) аспект.
  
  В данной работе сделана попытка сформулировать основные требования к подобному 
тезаурусу географических названий, который мог бы удовлетворить потребности 
существующих информационных систем по обработке географического и исторического 
аспекта информации. В~работе приводится обзор некоторых схем данных, а также 
существующих тезаурусов, анализ их сильных и слабых сторон (в контексте применения к 
задаче привязки географических метаданных к объектам информационных систем). 
Формулируются требования к тезаурусу географических наименований, подходящему для 
использования в рамках задачи интеграции географических метаданных. Также приводится 
вариант реляционной схемы данных тезауруса.
  
  \section{Основные требования к~тезаурусу}
  
  На основе всего вышеизложенного сформулируем список требований к тезаурусу, 
подходящему для использования в рассматриваемой задаче.
  
  Тезаурус должен:
  \begin{enumerate}[(1)]
\item обеспечивать прямое и обратное геокодирование;
\item обеспечивать ретроспективное прямое и обратное геокодирование;
\item позволять включать информацию в технологию поиска в существующих 
информационных массивах;
\item содержать внутренние связи: 
\begin{itemize}
\item[(a)] по географическим объектам,
\item [(б)] по временным характеристикам,
\item[(в)] по документам;
\end{itemize}
\item быть представлен в схеме, максимально приближенной к какой-либо стандартной;
\item однозначно отображаться на другие схемы тезаурусов, в частности 
необходимо однозначное соответствие профилю {Zthes}~\cite{4-sk}, быть\linebreak 
может расширенному, для интеграции с существующими информационными 
системами.
\end{enumerate}

  Для пояснения сформулированных требований рассмотрим основные сценарии 
использования тезауруса географических наименований. Работа с тезаурусом включает два 
основных сценария: запрос координат геометрического примитива объекта по имени этого 
объекта и запрос всех имен объектов по заданному ко\-ор\-ди\-нат\-но-при\-вя\-зан\-но\-му 
гео\-мет\-ри\-че\-ско\-му примитиву. Обычно это называется \textit{прямым} и \textit{обратным 
геокодированием}. Для целевых систем, которые потенциально могут содержать 
ретроспективную информацию, отличительной особенностью становится необходимость 
указания времени, для которого соответствующее геокодирование будет актуальным. При 
этом отсутствие задания момента времени может служить указанием на использование 
текущего момента времени в качестве параметра запроса.
  
  Существенным моментом прямого геокодирования является тот факт, что заданные в 
запросе имя и время могут быть взаимно противоречивы.\linebreak Например, запрос на координаты 
объекта (Новосибирск, 1920-05-20) должен возвращать ответ (Новониколаевск, {координаты 
геометрического\linebreak примитива}, 1920-05-20). Запросы обратного геокодирования в этом 
смысле более просты, так как задаваемые в запросе координаты не связаны с действующей 
топонимикой.
  
  \section{Обзор существующих решений}
  
  Рассмотрим имеющиеся на данный момент схемы представления данных и существующие 
тезаурусы, которые могут представлять интерес в рамках этой задачи.
  
  При рассмотрении будем обращать внимание на следующие свойства:
  \begin{enumerate}[(1)]
\item наличие ретроспективных данных. Возможность извлечь данные, относящиеся к 
прошлому;\\[-14pt]
\item наличие связей с нормативными документами. Возможность определить, 
согласно какому документу было изменено название или координаты объекта;
\\[-14pt]
\item описание координат географического объекта сообразно его форме. Представление 
географического объекта не только в виде точки, а также в виде замкнутого контура, линии, 
композиции примитивов;\\[-14pt]
\item наличие связей, отражающих относительное расположение географических 
объектов.
\end{enumerate}

   В первую очередь рассмотрим существующие схемы данных.
   
   \vspace*{-6pt}

  \subsection{ГОСТ Р 52573-2006}
  
  Национальный стандарт Российской Федерации <<Географическая информация. 
Метаданные>>. Стандарт предназначен для специалистов в области информационных 
технологий, разработчиков геоинформационных систем, баз и банков пространственных 
данных, а также прикладных информационных систем различного назначения. Стандарт 
разработан в соответствии с правилами создания профилей, указанными в стандарте 
{ISO}~19115 [5].
  
  Данный стандарт содержит рекомендацию к использованию ретроспективных данных 
(Сущность {EX\_Extent}). Координаты объекта задаются с по\-мощью одной из 
сущностей:
  \begin{enumerate}[(1)]
\item {EX\_BoundingPolygon}~--- многоугольник (задается множеством точек);\\[-14pt]
\item {EX\_GeographicBoundingBox}~--- прямоугольная область (задается координатами 
углов);\\[-14pt]
\item {EX\_GeographicDescription}~--- описание объекта с использованием 
географического идентификатора.
\end{enumerate}
  
  Есть сведения о документе-источнике, но они привязаны к объекту в целом, а не к данным 
о его координатах и наименовании.
  
  Сведения о связях между объектами отсутствуют в данной схеме.
  
     \vspace*{-6pt}

  \subsection{CIDOC Conceptual Reference Model}
  
{Committee on Documentation Conceptual Reference Model} (CRM) не является схемой 
тезауруса, но\linebreak\vspace*{-12pt}

\pagebreak

\noindent
 рассматривается в данной работе, так как пред\-став\-ля\-ет собой формальную 
онтологию, предназначенную для улучшения интеграции и обмена гетеро\-генной 
информацией по культурному наследию.\linebreak Более конкретно, CIDOC CRM определяет 
семантику схем баз данных и структур документов, используемых в культурном наследии и 
музейной документации, в терминах формальной онтологии. Модель не определяет 
терминологию, появляющуюся в конкретных структурах данных, но имеет характерные 
отношения для ее использования.
  
  Модель может служить как руководством для разработчиков информационных систем, 
так и общим языком для экспертов предметной области и специалистов по информационным 
технологиям. Она предназначена для покрытия контекстной информации исторического, 
географического и теоретического характера об отдельных экспонатах и музейных 
коллекциях в целом~\cite{6-sk}.
  
  В {CIDOC CRM} представляет интерес сущность {E53\_Place} (место), которая 
как раз описывает географические метаданные объекта. Данная сущность является 
экземпляром {E44\_Place\_Appellation}. {E44\_Place\_Appellation} содержит данные 
о координатах, адресе, географическом наименовании. Координаты могут задаваться в 
любом виде (не только географические). Имеется возможность задать ссылки на 
родительский элемент (иерархические связи).
  
  Но в онтологии {CIDOC CRM} не учтено изменение свойств географических 
объектов с течением времени, отсутствует связь географических метаданных с 
нормативными документами.

   \vspace*{-6pt}
  
  \subsection{{Getty Thesaurus of Geographic Names}}
  
  Тезаурус географических имен института {Getty}~--- англоязычный тезаурус, 
содержащий более чем миллион географических имен, информацию о континентах, 
физических объектах, административных сущностях и нациях современного политического 
мира, а также сведения об исторически значимых областях~\cite{7-sk}.
  
  Схеме тезауруса {Getty}, естественно, присущи как положительные, так и 
отрицательные черты.
  
  Из отрицательных моментов можно отметить отсутствие информации об изменении 
координат географических объектов с течением времени. Координаты объекта могут быть 
представлены либо точкой, либо прямоугольником, что недостаточно для полного описания 
области на земной поверхности.
  
  В то же время в схеме данного тезауруса учтено изменение названия объекта с течением 
времени ({Term\_Date}). Также учтены нормативные документы 
({Subject\_Sources}) для данного объекта и для его наименований 
({Term\_Source})~\cite{8-sk}. Записи содержат данные об иерархии.
  
  \subsection{Тезаурус Российской государственной библиотеки}
  
  Справочник географических названий Российской государственной библиотеки 
содержит наименования географических объектов (городов, рек, и~т.\,д.) на территории 
Российской Федерации~\cite{9-sk, 10-sk}.
  
  Тезаурус не содержит ретроспективных данных в записях. Невозможно получить ни 
данных о предыду\-щих названиях, ни данных о предыдущих координатах объектов.
  
  В записях присутствуют ссылки на нормативные документы, определяющие 
наименование объекта.
  
  Координаты географических объектов заданы в виде координат точек, что не совсем 
соответствует действительности.
  
  В записях тезауруса также есть данные о взаиморасположении объектов.
  
  \subsection{Служба геокодирования {API} сервиса Google Maps}
  
  Позволяет определить координаты объекта, а также найти адрес, наиболее близкий к 
указанным координатам~\cite{11-sk}. 
  
  В записях, предоставляемых данной службой, отсутствуют ретроспективные данные. 
Отсутствуют связи с нормативными документами. 
  
  Координаты объектов указаны в виде точек.
  
  В записях содержатся иерархические связи.
  
  В то же время стоит отметить, что тезаурус содержит данные не только о крупных 
географических объектах, но также и об адресах. Есть возможность произвести обратное 
геокодирование. Но использовать службу геокодирования можно только вместе с картами 
{Google}, что делает невозможным использование сервиса в данной задаче.
  
  \subsection{Служба геокодирования {API} сервиса Яндекс.Карты}
  
  Имеет функциональность, аналогичную геокодеру {Google}~\cite{12-sk}. Обладает 
практически теми же достоинствами и недостатками, но из дополнительных достоинств 
можно выделить более обширную базу российских наименований географических объектов. 
Сервис обладает аналогичными ограничениями по использованию, что также делает 
невозможным использование сервиса в данной задаче.
  
  \subsection{Сравнительная таблица}
  
  По итогам сравнения составим сравнительную таблицу рассмотренных схем (табл.~1). 
  
   \begin{table*}\small
   \begin{center}
   \Caption{Сравнительная таблица схем тезаурусов}
   \vspace*{2ex}
   
   \tabcolsep=5.4pt
   \begin{tabular}{|l|c|c|c|c|}
   \hline
\multicolumn{1}{|c|}{Схема/тезаурус}&
\tabcolsep=0pt\begin{tabular}{c}Содержит\\ ретроспективные\\ сведения\end{tabular}&
\tabcolsep=0pt\begin{tabular}{c}Содержит\\ ссылки на\\ документы-источники\end{tabular}&
\tabcolsep=0pt\begin{tabular}{c}Координаты\\ географических\\ объектов\\ заданы\\ соответственно\\
 их размерам\\ и форме\end{tabular}&
 \tabcolsep=0pt\begin{tabular}{c}Наличие \\ иерархических\\ связей\end{tabular}\\
\hline
ГОСТ Р 52573-2006&$+$&$+$&$\pm$&$-$\\
\hline
{CIDOC CRM}&$-$&$-$&$\pm$&$+$\\
\hline
{Getty}&$+$&$+$&$\pm$&$+$\\
\hline
\tabcolsep=0pt\begin{tabular}{l}Российская государственная\\ библиотека\end{tabular}&$-$&$+$&$-$&$+$\\
\hline
Геокодер сервиса {Google Maps}&$-$&$-$&$-$&$+$\\
\hline
Геокодер сервиса Яндекс.Карты&$-$&$-$&$-$&$+$\\
\hline
\end{tabular}
\end{center}
\end{table*}

  \begin{figure*}[b]
  \vspace*{1pt}
 \begin{center}
 \mbox{%
 \epsfxsize=115.191mm
 \epsfbox{ska-1.eps}
 }
 \end{center}
 \vspace*{-9pt}
   \Caption{Онтология тезауруса}
   \end{figure*}
  
  Таким образом, проанализировав сущест\-ву\-ющие решения, приходим к выводу, что схемы 
тезауруса с необходимой функциональностью нет. Но есть достаточно близкие схемы, 
которыми можно руководствоваться при разработке собственного тезауруса. 
  
  \section{Разработка схемы тезауруса}
  
  На основании приведенных выше данных была построена онтология тезауруса, 
отвечающего сформулированным ранее требованиям (рис.~1)~\cite{13-sk}.
  

  
  Заметим, что правильно организованный тезаурус географических названий может 
служить основой и для получения информации, отличной от результатов прямого и 
обратного геокодирования, в~част\-ности:
  \begin{itemize}
\item информации о документах, связанных с конкретным географическим объектом;
\item информации о времени актуальности названий объектов;
\item информации о времени актуальности координат объектов;
\item информации о временных характеристиках производных параметров.
\end{itemize}

  Следует также заметить, что любой тезаурус является лишь дополнительной базой 
данных, которая может быть задействована при обработке запросов к различным 
информационным массивам. Ретроспективный тезаурус географических названий может 
быть задействован при обработке запросов, включающих ретроспективные географические 
названия.
  
  Можно выделить три вида условий в запросе к тезаурусу:
  \begin{enumerate}[(1)]
\item по имени;
\item по координатам;
\item по времени.
\end{enumerate}

  Перечисленные условия могут комбинироваться друг с другом. 
  
  Для интеграции с существующими информационными системами и обеспечения 
интероперабельности необходимо зафиксировать профиль доступа к обсуждаемому 
тезаурусу ({RGeoThes}). Этот профиль, несомненно, должен являться расширением 
профиля {ZThes}~\cite{4-sk} для доступа к тезаурусам по протоколам Z39.50 и 
{SRW/SRU} и включать необходимые компоненты для временного и географического 
поиска. При этом профиль должен определять:
  \begin{itemize}
\item схему данных;
\item структуру записи и наборы элементов;
\item обязательные и дополнительные индексы (точки доступа);
\item синтаксис поисковых запросов и поисковые атрибуты;
\item форматы представления данных;
\item протоколы доступа к ресурсу.
\end{itemize}

  \subsection{Протоколы доступа}
  
  Для обеспечения интероперабельности доступ к \mbox{RGeoThes} должен обеспечиваться 
по протоколам:
  \begin{itemize}
\item Z39.50;
\item HTTP/XML/SOAP/SRW;
\item HTTP/SRU.
\end{itemize}

  Каждый из указанных способов доступа имеет свои специфические особенности, которые 
должны быть определены общим профилем.
  
  \subsection{Форматы представления данных}
  
  В качестве основного обязательного формата представления записи \mbox{RGeoThes} для 
всех способов доступа является формат {XML}. Дополнительным необязательным 
форматом является {HTML}. Для доступа по Z39.50 обязательным форматом 
также является GRS-1. В~качестве дополнительных (необязательных) форматов могут 
использоваться \mbox{RUSMARC}, {MARC}21 и~др.
  
  \subsection{Схема данных}
  
  Схема данных определяется в терминах {XML} (XSD) и должна соответствовать 
онтологии, схематично представленной на рис.~1.
  
  \subsection{Индексы и~точки доступа}
  
  Точками доступа записи \mbox{RGeoThes} должны быть элементы, представленные в 
табл.~2.
  
  \subsection{Синтаксис поисковых запросов и~поисковые атрибуты}
  
  Для доступа по Z39.59 обязательным синтаксисом запросов должен являться 
RPN-1, необязательным~--- {CQL}. Для доступа по \mbox{SRW/SRU} обязательным 
синтаксисом запросов должен являться {CQL}, необязательным~--- RPN-1 
  ({x-pquery}).
   
   \begin{table*}\small
   \begin{center}
   \Caption{Точки доступа записи {RGeoThes}}
   \vspace*{2ex}
   
   \begin{tabular}{|l|c|c|c|}
   \hline
\multicolumn{1}{|c|}{Точка доступа}&Набор&Тип&Значение\\
\hline
Локальный номер&{utility}&1&4\\
Название терма&{cross-domain}&1&1\\
Квалификатор терма&{zthes-1}&1&1\\
Тип терма&{zthes-1}&1&2\\
Статус терма&{zthes-1}&1&7\\
Категория терма&{zthes-1}&1&6\\
Язык названия&{utility}&1&3\\
Дата начала действия названия&{cip-1}&&\\
Дата окончания действия названия&{cip-1}&1&2073\\
&&2&14, 15, 16, 17, 18\\
Документ, фиксирующий название&{cross-domain}&1&6\\
Тип геометрического объекта&{cip-1}&4&201, 202\\
Координаты геометрического объекта&{cip-1}&1&2059, 2060\\
&&2&7, 8, 9, 10\\
Дата начала действия определения геометрии&{cip-1}&1&2072\\
&&2&14, 15, 16, 17, 18\\
Дата окончания действия определения геометрии&{cip-1}&1&2073\\
&&2&14, 15, 16, 17, 18\\
Документ, фиксирующий определения геометрии&{cross-domain}&1&6\\
Комментарий&{cross-domain}&1&4\\
Идентификатор связанного терма&{zthes-1}&1&4\\
\hline
  \end{tabular}
  \end{center}
  \end{table*}
  
  Поисковые атрибуты {RPN} для доступа по Z39.50 для обеспечения 
интероперабельности должны соответствовать поисковым атрибутам профиля \mbox{Z-Thes} 
из наборов {zthes-1}, {utility}, {cross-domain} (\mbox{xd-1}). Для поиска по 
времени и координатам должны использоваться атрибуты из набора {cip-1}. 
Аналогичное требование справедливо и для запросов~{CQL}.
  
  Соответствие поисковых атрибутов точкам доступа приведено в табл.~2~\cite{15-sk}.
  
  \section{Реляционная схема данных тезауруса}
  
  Рассмотрим вариант схемы для хранения записей тезауруса в случае использования 
реляционной СУБД {PostgreSQL} в качестве хранилища данных. Учитывая все 
вышеперечисленное, построим реляционную схему данных. Схема представлена на рис.~2.
  
  Основной в данной схеме является таблица <<\textbf{Запись тезауруса}>> (далее~--- 
главная таблица), в которой находится список квалификаторов записей тезауруса. Строка из 
данной таблицы может содержать ссылку на предыдущий вариант записи и на родительскую 
запись. Связи между записями тезауруса содержатся в таблице <<\textbf{Связь между 
записями}>>. Каждая связь содержит квалификаторы двух записей тезауруса, которые она 
связывает. Также связь характеризуется двумя документами (что представлено в виде 
внешних ключей). <<Начальный документ>> определяет документ, в котором 
зафиксировано появление связи. <<Конечный документ>>, который может быть не указан, 
определяет документ, в котором зафиксировано исчезновение связи.
  
  \begin{figure*} %fig2
    \vspace*{1pt}
 \begin{center}
 \mbox{%
 \epsfxsize=164.503mm
 \epsfbox{ska-2.eps}
 }
 \end{center}
 \vspace*{-9pt}
   \Caption{Реляционная схема данных тезауруса}
  \end{figure*}
  
  В таблице <<\textbf{Имя объекта}>> задаются наименования географических объектов, 
содержащихся в тезаурусе. Каждая из записей главной таблицы может быть связана с 
несколькими строками имен. В~свою очередь, имя может быть связано только с одной 
записью из главной таблицы. Каждое из имен характеризуется собственно именем, а также 
типом объекта и языком. Под типом объекта понимается, например, тип населенного пункта. 
Каждая запись таблицы <<Имя объекта>> содержит идентификаторы двух документов~--- 
начального и конечного. <<Начальный документ>> определяет документ, в котором 
зафиксировано присвоение данного имени географическому объекту. <<Конечный 
документ>> определяет документ, в котором зафиксировано окончание срока действия 
данного имени.
  
  Таблица <<\textbf{Местоположение объекта}>> содержит данные о координатах 
географических объектов тезауруса. Каж\-дая из записей главной таблицы может быть 
связана с несколькими строками данной таб\-ли\-цы. В~то же время запись таб\-ли\-цы 
<<Местоположение объекта>> может быть связана только с одной\linebreak строкой из главной 
таблицы. Каждая запись таб\-ли\-цы <<Местоположение объекта>> содержит идентификаторы 
двух документов~--- начального и конечного. <<Начальный документ>> определяет\linebreak 
документ, в котором зафиксировано присвоение данного мес\-то\-по\-ло\-же\-ния географическому 
объекту. <<Конечный документ>> определяет документ, в котором зафиксировано 
окончание срока действия местоположения применительно к данному объекту. Также 
каждая запись содержит поле <<тип местоположения>>, содержащее идентификатор типа 
местоположения объекта. Таким типом может быть точка, прямоугольник, многоугольник, 
линия, регион и прочие. Благодаря использованию отображения <<Одна ие\-рар\-хия\,--\,од\-на 
таблица>> для набора типов местоположений объекта становится возможным легко 
добавлять новые типы местоположения в уже работающую схему. Для хранения 
координатных данных используются поля <<Точка>>, <<Прямоугольник>>, 
<<Многоугольник>>, <<Линия>>, <<Регион>> с типами данных {point\_type}, 
{rectangle\_type}, {polygon\_type}, {line\_type}, {circle\_type} соответственно. 
Типы данных для этих полей являются композитными типами, содержащими всю 
координатную информацию, характерную для представления. Например, тип данных 
{rectangle\_type}, соответствующий прямоугольной области на поверхности Земли, 
содержит поле {rect} встроенного типа {box}. Данное разделение на типы для 
различных видов географических объектов сделано в целях повышения гибкости схемы.
  
  Таблица <<\textbf{Документ}>> содержит данные о документах, регистрирующих 
изменение характеристик объектов с течением времени. Каждый документ содержит 
описание, уникальный идентификатор ресурса ({URI}), дату создания и дату вступления 
в силу. Именно датой вступления в силу документов определяются временные рамки 
существования той или иной характеристики географического объекта.
  
  \section{Заключение}
  
  В данной работе была показана необходимость реализации информационно-поискового 
тезауруса географических наименований в рамках задачи интеграции географических 
метаданных в информационные системы общего назначения.
  
  Изложенные выше основные положения процесса интеграции, а также организации 
ретроспективного геокодирования и соответствующего тезауруса географических названий 
будут в дальнейшем использованы для построения модели информационной системы с 
возможностями геометрического и ретроспективного поиска информации. Поиск 
предполагается организовать на основе картографических интерфейсов в соответствии с 
описанным выше профилем.
  
  Приведен вариант реляционной схемы для хранения данных тезауруса. Схема в настоящее 
время используется для хранения данных, собираемых в тезаурус в рамках данной работы. 
В~дальнейшем планируется реализация доступа к тезаурусу на основе протоколов 
SRW/SRU, Z39.50, {HTTP}, а также экспериментальная интеграция 
метаданных в работающую систему.

{\small\frenchspacing
{%\baselineskip=10.8pt
\addcontentsline{toc}{section}{Литература}
\begin{thebibliography}{99}
  
\bibitem{1-sk}
\Au{Жижимов О.\,Л., Мазов Н.\,А.}
География и стандарты метаданных для электронных библиотек: содержание, 
применение, проблемы~// Электронные библиотеки, 2009. Т.~12. №\,1. {\sf 
http://www.elbib.ru/\linebreak index.phtml?page=elbib/rus/journal/2009/part1/\linebreak ZM}.

\bibitem{2-sk}
\Au{Жижимов О.\,, Мазов Л.\,А.}
Проблемы географической привязки цифровых объектов в электронных библиотеках~// 
Электронные библиотеки: перспективные методы и технологии, электронные коллекции 
(RCDL'2010): Труды XII Всеросс. научн. конф.~--- Казань: КГУ, 2010. С.~207--214.

\bibitem{3-sk}
\Au{Лукашевич Н.\,В.}
Тезаурусы в задачах информационного поиска.~--- М.: Изд-во Московского ун-та, 2011.

\bibitem{4-sk}
The Zthes specifications for thesaurus representation, access and navigation. {\sf 
http://zthes.z3950.org}.

\bibitem{5-sk}
ГОСТ Р 7.24-2007. Тезаурус Ин\-фор\-ма\-ци\-он\-но-по\-иско\-вый многоязычный. Состав, 
структура и основные требования к построению.~--- М.: Стандартинформ, 2006.

\bibitem{6-sk}
Онтология в области документации в сфере культурного наследия: CIDOC CRM. {\sf 
http://\linebreak www.intuit.ru/department/expert/ontoth/5}.

\bibitem{7-sk}
Getty Thesaurus of Geographic Names$^\registered$ Online. {\sf 
http://\linebreak www.getty.edu/research/tools/vocabularies/tgn/index.\linebreak html}.

\bibitem{8-sk}
Contribute to the Getty Vocabularies. {\sf 
http://www.\linebreak getty.edu/research/tools/vocabularies/contribute.html}.

\bibitem{9-sk}
Тезаурус РГБ. {\sf http://aleph.rsl.ru/F/?func=file\&file\_\linebreak name=find-b\&local\_base=tst11}.

\bibitem{10-sk}
\Au{Лаврёнова О.\,А.}
Многоязычный доступ к данным на основе тезауруса географических названий~// 
Электронные библиотеки: перспективные методы и технологии, электронные коллекции 
(RCDL'2007):\linebreak
 Труды IX Всеросс. научн. конф.~--- Пе\-ре\-славль-За\-лес\-ский: Ун-тет 
г.~Переславля, 2007. С.~57--62.

\bibitem{11-sk}
Геокодирование~--- Службы API Карт Google. {\sf 
http:// code.google.com/intl/ru/apis/maps/documentation/\linebreak geocoding}.

\bibitem{12-sk}
Поиск по карте~--- Яндекс.Карты. {\sf http://api.\linebreak yandex.ru/maps/geocoder}.

\bibitem{13-sk}
\Au{Соловьев В.\,Д., Добров Б.\,В., Иванов~В.\,В., Лукашевич~Н.\,В.}
Онтологии и тезаурусы: Учебное пособие.~--- Казань, М., 2006.

%\bibitem{14-sk}
%The Zthes specifications for thesaurus representation, access and navigation. {\sf 
%http://zthes.z3950.org}.

\label{end\stat}

\bibitem{15-sk}
Catalogue Interoperability Protocol (CIP) Specification~--- Release~B~// 
CEOS/WGISS/ICS/CIP-B. April 2005. Issue~2.4.75.

  
 \end{thebibliography}
}
}


\end{multicols}    %4
\def\stat{sharapov}

\def\tit{УНИВЕРСАЛЬНАЯ СИСТЕМА ПРОВЕРКИ ТЕКСТОВ НА~ПЛАГИАТ 
<<АВТОР.NET>>}

\def\titkol{Универсальная система проверки текстов на~плагиат 
<<Автор.NET>>}

\def\autkol{Е.\,В.~Шарапова, Р.\,В.~Шарапов}
\def\aut{Е.\,В.~Шарапова$^1$, Р.\,В.~Шарапов$^2$}

\titel{\tit}{\aut}{\autkol}{\titkol}

%{\renewcommand{\thefootnote}{\fnsymbol{footnote}}\footnotetext[1]
%{Работа поддерживается РГНФ, проект 11-02-12026-в.}}


\renewcommand{\thefootnote}{\arabic{footnote}}
\footnotetext[1]{Владимирский государственный университет имени Александра Григорьевича и Николая Григорьевича 
Столетовых,\linebreak mivlgu@mail.ru}
\footnotetext[2]{Владимирский государственный университет имени Александра Григорьевича и Николая Григорьевича 
Столетовых,\linebreak info@vanta.ru}

\vspace*{2pt}

\Abst{Обсуждается проблема обнаружения в текстах заимствований из других 
источников. Рассматриваются основные подходы к обнаружению заимствований, 
проводится обзор существующих на сегодняшний день программ. Дается обзор методов к 
сокрытию фактов заимствований. Дается описание разработанной сис\-те\-мы 
<<Автор.NET>>, способной проводить проверку заимствований по внутренним источникам 
и сети Интернет.}

\vspace*{2pt}

\KW{плагиат; обнаружение плагиата; заимствование}

%\vspace*{6pt}

\vskip 14pt plus 9pt minus 6pt

      \thispagestyle{headings}

      \begin{multicols}{2}

            \label{st\stat}


\section{Введение}
  
  Бурное развитие вычислительной техники привело к глубокому 
проникновению компьютеров в нашу жизнь. Компьютеры окружают нас 
везде~--- на работе, дома, в магазинах и общественных мес\-тах. Современное 
развитие информационных технологий и глобальной сети Интернет 
предоставило широким кругам пользователей доступ к огромным массивам 
информации. Появилось большое чис\-ло он\-лайн-биб\-лио\-тек, содержащих в 
электронном виде художественную и научно-техническую литературу. Стало 
возможным читать книги, новости и газеты непосредственно с экрана 
компьютера. 
  
  В сети Интернет стало доступно множество методических указаний, курсов 
лекций, учебников и~т.\,д. Кроме того, появились огромные коллекции 
рефератов, готовых лабораторных работ, курсовых и дипломных проектов и 
даже диссертаций. Использование компьютерной техники сильно облегчило 
задачу поиска и копирования подобной информации. Если раньше для 
написания реферата или контрольной работы информацию было нужно, по 
крайней мере, найти в книгах и переписать (вручную, перепечатать или ввести 
в компьютер с помощью сканера и программ распознавания текстов), то теперь 
достаточно ввести название темы в поисковую сис\-те\-му и скопировать 
найденные материалы. Стал распространяться метод написания работ, 
получивший название <<{Copy}\,\&\,{Paste}>>. Метод заключается в 
простом копировании информации из одного или нескольких источников с 
минимальным редактированием получающегося таким образом текста. 
  
  Аналогичная ситуация наблюдается с отчетными материалами внутри 
учебных заведений. В~связи с тем, что большое число пояснительных записок 
по курсовым и дипломным проектам выполняется с использованием 
компьютеров, происходит их распространение и повторное использование 
среди учащихся.
  
  В последнее время наблюдается бурный рост использования в учебном 
процессе подобной заимствованной информации. Ситуация усугубляется тем, 
что учащиеся иногда не знают (не читают) то, что написано в <<их>> работах. 
  
  Плагиат~--- умышленное присвоение авторства на чужое произведение 
литературы, науки, искусства, изобретение или рационализаторское 
предложение (полностью или частично)~[1].
  
  Как можно убедиться из определения, подобные заимствованные работы 
можно отнести к разряду плагиата. Задача обнаружения недобросовестного 
использования заимствованных текстов в учебных и ученых кругах (фактов 
плагиата) приобретает высокую актуальность.

\begin{table*}\small
\begin{center}
\Caption{Формы плагиата}
\vspace*{2ex}

\begin{tabular}{|l|c|}
\hline
\multicolumn{1}{|c|}{Форма плагиата}&Доля\\
\hline
Полное или частичное копирование текста из одного источника&36\%\\
Копирование и компоновка текста из нескольких источников&62\%\\
Копирование текста из другого источника и изменение порядка следования частей 
текста&\hphantom{9}2\%\\
\hline
\end{tabular}
\end{center}
%\end{table*}
%\begin{table*}\small
\begin{center}
\Caption{Частота использования подходов к сокрытию фактов плагиата}
\vspace*{2ex}

\tabcolsep=8pt
\begin{tabular}{|l|c|}
\hline
Подходы к сокрытию плагиата&Доля\\
\hline
Корректировка родов, чисел и времен, входящих в текст слов&32\%\\
Незначительное изменение текста&38\%\\
Сокращение заимствованного текста&44\%\\
Замена букв&\hphantom{9}4\%\\
Синонимизация текста&\hphantom{9}2\%\\
\hline
\end{tabular}
\end{center}
\end{table*}

\section{Формы заимствований текстов}

  Рассмотрим формы заимствований, встре\-ча\-ющи\-еся в практике учебных 
заведений и подлежащие выявлению. 
  \begin{enumerate}[1.]
  \item Полное или частичное копирование текста из одного источника. 
  \item Копирование и компоновка текста из нескольких источников.
  \item Копирование текста из другого источника и его частичное 
редактирование.
  \end{enumerate}
  
  Для того чтобы скрыть факт заимствований, могут применяться следующие 
подходы:
  \begin{enumerate}[1.]
  \item Корректировка родов, чисел и времен входящих в текст слов. 
Например, замена слова <<выполнил>> на <<выполнила>> или 
<<выполнили>>, использование местоимения <<я>> вместо <<мы>> в 
оригинальном тексте и~т.\,д.
  \item Незначительное изменение заимствованного текста.
\item  Сокращение заимствованного текста путем удаления слов, предложений, 
абзацев, рисунков, формул и~т.\,д.
  \item Обход сис\-тем проверки на плагиат путем замены русских букв на 
аналогичные по написанию английские и~т.\,д.
  \item Осуществление ручной или автоматической синонимизации текста.
  \end{enumerate}
  
  Все вышеописанное должно учитываться при создании и использовании 
  сис\-тем проверки на заимствования. О~правомочности того или иного 
заимствования решение выносит сам проверяющий.
  
  Для оценки частоты использования тех или иных форм плагиата мы провели 
следующий эксперимент. Студентам двух групп гуманитарных специальностей 
было предложено написать статьи на тему экологической ситуации в регионе 
(Владимирская область). Студенты были предупреждены о том, что статьи 
будут проверяться на наличие плагиата. Из полученного набора были 
исключены оригинальные статьи. Анализ статей, содержащих заимствованный 
контент, показал, что большинство из них скомпонованы из нескольких (реже 
одного) источников, чаще всего из учебников, статей из сети Интернет и 
публикаций региональной прессы (табл.~1). Тот факт, что доля статей, 
полностью или частично скопированных только из одного источника, 
составила всего 36\% (в реальных условиях она часто бывает больше), вероятно 
связан со знанием авторов о том, что работы будут проверяться. Доля работ, 
составленных путем копирования текста из другого источника и изменения 
порядка следования частей текста, оказалась незначительной (2\%).



  Анализ подходов, используемых студентами для сокрытия факта плагиата, 
показал, что в 32\% работ осуществлялась корректировка родов, чисел и времен 
слов (табл.~2). В~38\% работ (составленных как из одного, так и из 
нескольких источников) осуществлялось незначительное изменение 
заимствованного текста. Так, например, делались вставки слов и предложений в 
заимствованный текст, подвергались изменению названия населенных пунктов 
и рек (р.~Волга в оригинале заменялась на р.~Ока в статье). Надо заметить, что 
часть работ кроме заимствованных текстов содержала оригинальные блоки, 
чаще всего введение и заключение. Приведенная выше доля статей, 
подвергавшихся изменению, учитывает только заимствованные части таких 
текстов. Из работ, скопированных из одного источника, 44\% подвергались 
сокращению. В~данном случае под сокращением подразумевалось исключение 
части предложений, графиков, рисунков из заимствованных текстов, а также 
исключение начальных или конечных блоков текста, по смыслу составляющих 
единое целое с заимствованным фрагментом. Копирование законченного 
фрагмента из текста (например, раздела или главы) сокращением не считалось. 
Замена букв осуществлялась в 4\% работ. В~одной из работ замене подверглись 
практически все русские буквы, сходные по написанию с английскими 
буквами. В~остальных работах заменялись одна--две гласные буквы. Ручная 
синонимизация проводилась только в 2\% работ. Применения автоматической 
синонимизации в статьях замечено не было. Надо заметить, что около 40\% 
рассматриваемых работ вообще не подвергались каким-либо изменениям, 
призванным скрыть факты плагиата.



\section{Подходы к~обнаружению заимствований}

  Существует несколько подходов к обнаружению заимствований (или, как их 
еще называют, нечетких дублей текстов). Достаточно подробный обзор 
приведен в~[2].
  
  Наибольшую известность получил метод <<шинглов>>~[3]. Метод основан 
на представлении текстов в виде множества последовательностей 
фиксированной длины, состоящих из соседних слов. При значительном 
пересечении таких множеств документы будут похожи друг на друга. Одна из 
модификаций метода, получившая название <<супершинглов>>, используется 
для быстрого обнаружения подобных документов~[2].
  
  Существует ряд методов, использующих сигнатурную лексическую 
информацию документов. В~[4] для этих целей используется {I-Match} 
сигнатура, вычисляемая для слов со средним значением {IDF} (инверсной 
частоты слов в документах). Другим сигнатурным подходом, основанным на 
лексических принципах, является метод <<опорных>> слов~[5]. В~данном 
случае для документов со\-став\-ля\-ют\-ся по определенным правилам наборы 
опорных слов, для которых строятся сигнатуры документов. Совпадение 
сигнатур говорит о подобии самих документов. Эта группа методов, несмотря 
на большую сложность реализации, показывает более хорошие результаты в 
обнаружении похожих документов~[2].
  
  Для обнаружения заимствований иногда используются алгоритмы, 
построенные на классических принципах информационного поиска, таких как 
{TF}, {TF*IDF} и~т.\,д.~[6]. В~[7] предлагается использовать 
функцию схожести Джаккарда, применение которой позволяет добиться 
неплохих результатов даже в текстах с использованием синонимов и наличием 
орфографических ошибок.

\section{Обзор существующих систем}

  Рассмотрим практическое использование описанных подходов в задачах 
обнаружения плагиата. В~настоящее время существует достаточно большое 
количество сервисов и программ, позволяющих так или иначе выявить 
заимствованный контент. Большую известность получила сис\-те\-ма 
<<Антиплагиат>>, разработанная компанией <<Форексис>>~[8]. Сис\-те\-ма 
осуществляет поиск по большому количеству коллекций рефератов, 
контрольных работ и учебников, хранящихся в собственной базе сис\-те\-мы. 
Тем не менее сис\-те\-ма имеет ряд недостатков. Во-пер\-вых, она не 
осуществляет поиск по всем документам, доступным в сети Интернет. 
Особенно это касается тематических сайтов и новостных порталов: большое 
число заимствований осуществляется именно из таких источников. 
Соответственно, даже при полном дублировании подобной информации, 
  сис\-те\-ма <<Антиплагиат>> соответствий не обнаружит. Во-вто\-рых, 
присутствует ограничение размера проверяемого текста 3000 или 
5000~символами (доступно после регистрации). В-третьих, ограничен просмотр 
документов, частично соответствующих проверяемому тексту. Кроме того, 
сис\-те\-ма ограничивает возможность проверки по базе имеющихся работ.
  
  Программа {Advego Plagiatus} осуществляет проверку с применением 
поисковых сис\-тем~[9]. Использует разные поисковые сис\-те\-мы и проверяет 
их доступность. В~отличие от аналогичных сис\-тем, {Advego Plagiatus} 
не использует Яндекс.XML, а обращается напрямую к таким поисковым 
сис\-те\-мам, как Яндекс, {Google}, {Bing}, Рамблер, {Yahoo}, 
\mbox{Поиск@Mail.ru}, {Nigma}, {QIP}. Качество обнаружения 
плагиата достаточно высокое. Программа выдает процент совпадения текста и 
выводит найденные источники. Недостатком является отсутствие 
преобразования букв, отсутствие поддержки поиска по собственной базе. Из-за 
особенностей работы программы возникают ситуации, когда результаты 
проверки отличаются от раза к разу. 
  
  Сервис {\sf www.miratools.ru} позволяет осуществлять онлайн-проверку 
текста на плагиат~[10]. Сис\-те\-ма использует результаты выдачи поисковых 
сис\-тем. К~достоинствам можно отнести возможность замены английских букв 
на русские. Имеются возможности изменять длину и шаг шинглов 
(используемых для проверки). По результатам проверки выдается процент 
совпадений и найденные источники. Сис\-те\-ма не работает с собственной 
базой. Присутствует ограничение на длину текста в 3000~символов и на число 
проверок в течение суток. 

\begin{table*}\small %tabl3
\begin{center}
\Caption{Сравнение функциональности сервисов проверки текстов на плагиат}
\vspace*{2ex}

\begin{tabular}{|l|c|c|c|c|}
\hline
\multicolumn{1}{|c|}{Система}&
\tabcolsep=0pt\begin{tabular}{c}Поиск\\ в Интернете\end{tabular}&
\tabcolsep=0pt\begin{tabular}{c}Поиск\\ в локальной  базе\end{tabular}&
\tabcolsep=0pt\begin{tabular}{c}Обработка\\ замены букв\end{tabular}&
\tabcolsep=0pt\begin{tabular}{c}Подробный\\ отчет\end{tabular}\\
\hline
{Advego Plagiatus}&$+$&$-$&$-$&$+$\\
<<Антиплагиат>>&$-$&$+$&$-$&$-$\\
{Istio}&$+$&$-$&$-$&$-$\\
{Miratools}&$+$&$-$&$+$&$+$\\
{Plagiat-inform}&$+$&$+$&$-$&$+$\\
{Praide Unique Content Analyser~II}&$+$&$-$&$-$&$+$\\
\hline
\end{tabular}
\end{center}
\end{table*}
  
  Сервис {\sf www.istio.com} осуществляет проверку текста на наличие 
заимствованного контента с использованием поисковых сис\-тем~[11]. Для этих 
целей используют Яндекс.XML и Yahoo.com. Возможности 
сервиса несколько слабее по сравнению с Miratools. По результатам 
проверки выдается сообщение о том, является ли текст уникальным или нет, и 
выдается список подобных сайтов. Преобразование букв и поддержка поиска 
по собственной базе отсутствует. Сервис предоставляет дополнительные 
средства для анализа текстов, например проверку орфографии, анализ наиболее 
частотных слов и~т.\,д.
  
  Программа Praide Unique Content Analyser~II~[12] имеет широкие 
возможности по проверке текстов с использованием поисковых сис\-тем. 
Имеется возможность выбора используемых поисковых сис\-тем, содержит 
средства добавления новых поисковых сис\-тем. Проверка осуществляется 
пассажами и шинглами, длину которых можно изменять. Можно задавать 
количества слов перекрытия шинглов. Выводится подробный отчет по проверке 
в каждой поисковой сис\-те\-ме. К~недостаткам можно отнести отсутствие 
замены букв и обработки стоп-слов. Нет поддержки работы с собственной 
базой.
  
  Система {Plagiatinform}, по заверениям авторов, имеет наиболее 
широкий функционал~[13, 14]. Она умеет проверять документы на наличие 
заимствований как в локальной базе, так и в сети Интернет. Сис\-те\-ма умеет 
обрабатывать документы, скомпонованные из перемешанных кусков текста 
нескольких источников. Проверка может осуществляться с использованием 
быстрого или углубленного поиска. Результаты проверки выдаются в виде 
наглядного отчета. Авторы не предоставляют возможности свободного 
использования или тестирования сис\-те\-мы, и оценить качество ее работы 
невозможно.
  
  Результаты сравнения функциональности рассмотренных сервисов проверки 
на плагиат приведены в табл.~3. Несмотря на большое число существующих 
решений, ни одно из них не может служить универсальным средством проверки 
на плагиат. Основной недостаток большинства существующих сис\-тем~--- это 
направленность поиска либо на сеть Интернет, либо на собственную базу. 
Очевидно, что более точная и универсальная проверка будет обеспечена при 
использовании обоих видов источников. Кроме того, большинство сис\-тем не 
способны обрабатывать замену букв, чем часто пользуются недобросовестные 
авторы (чаще всего студенты).



  В большинстве рассмотренных сис\-тем используется метод <<шинглов>>. 
По исследованиям~[2] этот метод демонстрирует высокую точность 
обнаружения дублированных текстов. Тем не менее из-за особенностей 
реализации результаты проверки в каждой сис\-те\-ме сильно отличаются от 
других. Минусом метода является отсутствие возможности обработки 
синонимов~[7]. Это является значительным недостатком существующих 
  сис\-тем. 

\section{Практическая реализация}
  
  На базе Владимирского государственного университета авторами была 
разработана сис\-те\-ма проверки текстов на наличие заимствований из других 
источников (проверки на плагиат) <<Автор.NET>>. Сис\-те\-ма осуществляет 
проверку как по источникам, доступным в сети Интернет, так и по собственным 
источникам (базам статей, курсовых и контрольных работ, дипломных 
проектов и~т.\,д.). По результатам проверки формируется отчет с подсветкой 
найденных заимствований и воз\-мож\-ностью просмотра найденных источников.

Рассмотрим структуру сис\-те\-мы (рис.~1).

\begin{figure*}
\vspace*{1pt}
 \begin{center}
 \mbox{%
 \epsfxsize=104.248mm
 \epsfbox{sha-1.eps}
 }
 \end{center}
 \vspace*{-9pt}
\Caption{Структура сис\-те\-мы проверки текстов на заимствования}
\vspace*{6pt}
\end{figure*}

  Проверяемый исходный текст подвергается предварительной обработке, в 
которую входят:
  \begin{enumerate}[(1)]
  \item исключение из текста знаков препинания и спецсимволов;
  \item преобразование регистра;
  \item обработка замены символов (преобразование латинских букв в русских 
словах на аналогичные буквы русского алфавита для текстов на русском языке);
  \item удаление стоп-слов и знаков препинания (предлоги, наречия и~т.\,д.);
  \item фильтрация текста (удаление неинформативных слов);
  \item стемминг (обработка окончаний слов).
  \end{enumerate}
  
  Фильтрация текста заключается в удалении наиболее частотных слов, 
неинформативных слов и~т.\,д. Кроме того, фильтрации подвергаются цифры, 
спецсимволы и~т.\,д. Эта процедура позволяет существенно сократить объемы 
вычислений (длину проверяемого текста). 
  
  Стемминг заключается в обработке окончаний слов. В~описываемой 
  сис\-те\-ме они просто отбрасываются. Это позволяет исключить влияние 
таких модификаций текста, как изменение единственного и множественного 
числа, мужского и женского рода, настоящего и прошедшего времени и~т.\,д.
  
  Система проверки на плагиат <<Автор.NET>> состоит из двух модулей, 
которые функционируют независимо друг от друга.
  
  Первый модуль осуществляет проверку по внут\-рен\-ней базе источников. База 
источников включает в себя статьи, курсовые и контрольные работы, 
дипломные проекты, а также учебники и курсы лекций. Источники хранятся 
как в виде полных текстов, необходимых для оценки значимости 
заимствований (по результатам проверки), так и в виде специально 
организованного поискового индекса. Последний необходим для быстрой 
проверки на совпадения текста и базы источников. Нет необходимости при 
каждой проверке просматривать все имеющиеся тексты и производить их 
достаточно трудоемкую обработку. Вся необходимая для поиска информация 
уже включена в структурированный поисковый индекс, с которым и работает 
модуль. Поисковый индекс формируется из текстов, прошедших описанную 
выше предварительную обработку.
  
  Второй модуль осуществляет проверку по источникам сети Интернет. Для 
этих целей текст проверяемого документа разбивается на информативные куски 
(разбиение проводится по полному тексту документа без проведения 
фильтрации и стемминга). Число таких кусков зависит от размера документа. 
Далее с использованием поисковых сис\-тем проводится поиск источников, 
содержащих указанные информативные куски. Для осуществления поиска 
модуль использует Яндекс.XML, а также доступ к он\-лайн-поиску 
  сис\-тем {Google.ru}, {Rambler.ru}, {Aport.ru}, 
Поиск.Mail.ru, Nigma.ru и~т.\,д. Полученные таким образом 
источники проверяются затем на соответствие исходному документу. Для этого 
определяется формат источника (html-до\-ку\-мент, txt-файл, 
doc- или 
  rtf-до\-ку\-мент, pdf-файл). В~случае html-до\-ку\-мен\-та из 
источника удаляются теги разметки. Файлы *.doc, *.rtf и 
*.pdf преобразуются, если это возможно, в обычный текстовый формат 
без разметки. Далее источники проходят предварительную обработку, и затем 
проводится оценка их сходства с исходным документом (рис.~2). 

  
Для оценки сходства исходного документа и источ\-ни\-ков используется некая 
модификация ал\-горит\-ма <<шинглов>>. Модификация алгоритма заклю\-ча\-ет\-ся в 
том, что рассматривается не ори\-ги\-наль\-ный документ, а его обработанная и 
отфильтро\-ван\-ная копия с исключением неинформативных объектов. Основное 
требование к сис\-те\-ме~--- полнота и точность оценки совпадений. Авторы не 
ставили задачей сокращение времени проверки, проведение экс\-пресс-оцен\-ки 
на полные дубли и~т.\,д. 
{\looseness=1

}



\begin{figure*} %fig2
\vspace*{1pt}
 \begin{center}
 \mbox{%
 \epsfxsize=160mm
 \epsfbox{sha-2.eps}
 }
 \end{center}
 \vspace*{-9pt}
\Caption{Интерфейс программы <<Автор.NET>>}
\end{figure*}

  
  В настоящее время локальная база сис\-те\-мы содержит дипломные проекты, 
выполненные за последние 6~лет, и курсовые проекты, выполненные за 
последние 3~года студентами одной из специальностей. Также в базе 
содержится ряд контрольных работ, выполненных студентами заочной формы 
обучения. 


\section{Результаты исследования}

  Для проверки работоспособности сис\-те\-мы <<Автор.NET>> были 
составлены тесты трех видов:
  \begin{enumerate}[1.]
\item Заимствования с изменением в тексте времен и родов слов ($T_1$).
\item Заимствования из одного источника с измененным порядком следования 
предложений и добавлением оригинального текста между предложениями 
($T_2$).
\item Заимствования, взятые из нескольких источников, с измененным 
порядком следования предложений ($T_3$).
\end{enumerate}

  Все тесты имели приблизительно одинаковый размер в 2000~символов и 
содержали в среднем по 400~слов. В~качестве источника текстов для 
составления тестов использовалась коллекция рефератов, широко доступная в 
сети Интернет. Было составлено по 10~тестов каждого вида. 
  
  Для оценки качества обнаружения заимствований сравнивались результаты 
работы сис\-те\-мы с результатами сис\-тем <<Антиплагиат>>, {Advego 
Plagiatus} и {Miratools}. В~связи с тем, что каждая сис\-те\-ма имеет свои 
принципы подсчета оригинальности документа, в качестве метрики 
оригинальности использовалось процентное отношение оригинальных слов в 
документе к общему количеству слов.
  
  Для оценки качества обнаружения заимствований использовался показатель 
полноты (Recall), показывающий, какой процент заимствований был 
обнаружен (табл.~4). Точность обнаружения (Precision) во всех 
сис\-те\-мах была на высоком уровне и стремилась к~1~\cite{15-sha}.



  Как можно заметить, ни одна из трех рас\-смат\-ри\-ва\-емых сис\-тем не 
справилась с тестом на замену  окончаний ($T_1$). Показатель {Advego 
Plagiatus} объясняется наличием в измененном тексте цепочек из 5~слов, для 
которых окончания не менялись. Применение стемминга в сис\-те\-ме 
<<Автор.NET>> поз-\linebreak\vspace*{-12pt}

\noindent
\begin{center}  %tabl4
%\vspace*{-6pt}
{{\tablename~4}\ \ \small{Результаты тестирования}}
\vspace*{2ex}

{\small \begin{tabular}{|l|c|c|c|}
\hline
\multicolumn{1}{|c|}{Система}&$T_1$&$T_2$
&$T_3$\\
\hline
<<Антиплагиат>>&0\hphantom{,99}&1&0,97\\
{Miratools}&0\hphantom{,99}&\hphantom{,9}0.9&0,83\\
{Advego Plagiatus}&0,14&1&0,62\\
<<Автор.NET>>&0,99&1&0,98\\
\hline
\end{tabular}

}
%\vspace*{-9pt}
\end{center}


\pagebreak

%\vspace*{10pt}

\addtocounter{table}{1}


\noindent
волило ей справиться с указанной задачей и обнаружить 
заимствования.
  
  С задачей обнаружения изменения порядка следования предложений, взятых 
из одного источника ($T_2$), справились все сис\-те\-мы. Чуть худший 
результат {Miratools} (полнота~0,9) объясняется, видимо, особенностями 
реализации алгоритма сравнения в этой сис\-теме. 
  
  С задачей обнаружения предложений, взятых из разных источников с 
изменением порядка их следования ($T_3$), рассматриваемые сис\-те\-мы 
справились немного хуже. Сис\-те\-ма <<Антиплагиат>> показала хорошее 
значение полноты (0,97). Результаты сис\-те\-мы {Miratools} оказались 
более скромными (полнота 0,83). В~сис\-те\-ме {Advego Plagiatus} полнота 
иногда опускалась до 0,45 при среднем значении в~0,62. Сис\-те\-ма 
<<Автор.NET>> хорошо справилась с указанной задачей, продемонстрировав 
полноту в~0,98.
  
  Как можно заметить, сис\-те\-ма <<Автор.NET>> успешно справилась со 
всеми видами тестов и показала результаты, не уступающие, а иногда и 
превосходящие результаты работы существующих сис\-тем. 

\section{Выводы}

  Таким образом, разработанная сис\-те\-ма <<Автор.NET>> проверки текстов 
на плагиат показала достаточно хорошие результаты. Использование 
фильтрации текста, стемминга и преобразования символов позволило 
  сис\-те\-ме находить заимствованные тексты даже при их незначительной 
модификации. 
  
  Система позволяет работать не только с русскоязычными текстами, но с 
текстами на иных языках. 
  
  Особенностью системы является возможность проведения проверки как по 
внутренней базе источников, так и по источникам сети Интернет. Это делает ее 
достаточно универсальным средством проверки текстов и выгодно отличает от 
существующих сис\-тем. Выдаваемые сис\-те\-мой отчеты позволяют оценивать 
правомерность найденных заимствований текстов. 
  
  Система <<Автор.NET>> может использоваться для проверки уникальности 
студенческих работ (курсовых и дипломных проектов, рефератов и 
контрольных работ). Еще одной областью применения может служить 
использование сис\-те\-мы для проверки докладов, представляемых на 
студенческие и молодежные научные конференции.

{\small\frenchspacing
{%\baselineskip=10.8pt
\addcontentsline{toc}{section}{Литература}
\begin{thebibliography}{99}

  \bibitem{1-sha}
  Большой энциклопедический словарь.~--- М.: АСТ, Астрель, 2008. 1248~c.
  
  \bibitem{2-sha}
  \Au{Зеленков Ю.\,Г., Сегалович И.\,В.}
  Сравнительный анализ методов определения нечетких дубликатов для\linebreak 
  WEB-до\-ку\-мен\-тов~// Электронные библиотеки: перспективные методы и технологии, 
электронные коллекции (RCDL'2007): Труды IX Всеросс. научн. конф.~--- 
  Пе\-ре\-славль-За\-лес\-ский: Ун-т г.~Переславля, 2007. Т.~1. С.~166--174.
  
  \bibitem{3-sha}
  \Au{Broder A.} On the resemblance and containment of documents~// Compression and 
Complexity of Sequences (SEQUENCES'97).~--- IEEE Computer Society, 1998. P.~21--29.
  
  \bibitem{4-sha}
  \Au{Kolcz A., Chowdhury A., Alspector~J.} Improved robustness of signature-based 
  near-replica detection via lexicon randomization~//  KDD 2004 Proceedings.~--- Seattle, 2004.
  
  \bibitem{5-sha}
  \Au{Ilyinsky S., Kuzmin M., Melkov~A., Segalovich~I.}
  An efficient method to detect duplicates of Web documents with the use of inverted index~// 
WWW'2002: 11th World Wide Web Conference (International) Proceedings.~---  New York: ACM 
Press, 2002. 
  
  \bibitem{6-sha}
  \Au{Шарапов Р.\,В., Шарапова Е.\,В.}
  Пути расширения булевой модели поиска~// Информационные сис\-те\-мы и технологии. 
Известия ОрелГТУ.~--- Орел: ОрелГТУ, 2009. №\,6(56). С.~74--78.
  
  \bibitem{7-sha}
  \Au{Неелова Н.\,В., Сычугов А.\,А.}
  Сравнение результатов детектирования дублей методом шинглов и методом Джаккарда // 
Вестник РГРТУ, 2010. №\,4(34). С.~72--78.
  
  \bibitem{8-sha}
  Антиплагиат. {\sf http://www.antiplagiat.ru}.
  
  \bibitem{9-sha}
  Advego Plagiatus~--- проверка уникальности текста. {\sf http://advego.ru/plagiatus}.
  
  \bibitem{10-sha}
  Сервис проверки уникальности контента. {\sf http://\linebreak www.miratools.ru}.
  
  \bibitem{11-sha}
  Анализировать текст, поиск плагиата. {\sf 
http://\linebreak istio.com/rus/text/analyz}.
  
  \bibitem{12-sha}
  Проверка уникальности текста в Интернете~--- очень полезная программа для качественной 
раскрутки сайтов. {\sf http://www.nado.su/downloads.html}.
  
  \bibitem{13-sha}
  SearchInform Плагиат-Информ~--- сис\-те\-ма для определения плагиата в документах. {\sf 
http://www.\linebreak searchinform.ru/main/full-text-search-plagiarism-search-plagiatinform.html}.
  
  \bibitem{14-sha}
  \Au{Ширяев М.\,А., Мустакимов В.}
  Plagiatinform избавит от плагиата в научных работах~// Educational Technol. Soc., 
2009. №\,11(1). С.~367--374.

\label{end\stat}
  
  \bibitem{15-sha}
  \Au{Шарапов Р.\,В., Шарапова Е.\,В.} 
  Система проверки текстов на заимствования из других источников // Электронные 
библиотеки: перспективные методы и технологии, электронные коллекции (RCDL'2011):\linebreak 
Труды XIII Всеросс. научн. конф.~--- Воронеж: ВГУ, 2011. 
С.~233--238.

 \end{thebibliography}
}
}


\end{multicols}    %5


\renewcommand{\figurename}{\protect\bf Figure}
\renewcommand{\tablename}{\protect\bf Table}
\renewcommand{\bibname}{\protect\rmfamily References}

\def\stat{nikola}

{\begin{center}
{\Large
Статьи, являющиеся развитием докладов, %}\\[6pt]
%{\Large 
представленных }\\[6pt]
{\Large  на XXIX Международном семинаре}\\[6pt]
{\Large по проблемам устойчивости стохастических моделей}\\[9pt]
{\large (г.~Светлогорск Калининградской области России, 10--16~октября 2011~г.)}
\end{center}
}


\def\tit{FRACTIONAL LEVY MOTION WITH DEPENDENT INCREMENTS
AND~ITS~APPLICATION TO~NETWORK TRAFFIC MODELING}

\def\titkol{Fractional Levy motion with dependent increments
and~its~application to~network traffic modeling}

\def\autkol{C.~De~Nikola,  Y.\,S.~Khokhlov, M.~Pagano, and~O.\,I.~Sidorova}
\def\aut{C.~De~Nikola$^1$,  Y.\,S.~Khokhlov$^2$, M.~Pagano$^3$, and~O.\,I.~Sidorova$^4$}

\titel{\tit}{\aut}{\autkol}{\titkol}

%{\renewcommand{\thefootnote}{\fnsymbol{footnote}}\footnotetext[1]
%{Работа поддержана РФФИ (проект 10-07-00017). Работа выполнена
%при поддержке Программы стратегического развития на 2012--2016~гг.\
%<<Университетский комплекс ПетрГУ в научно-образовательном пространстве
%Европейского Севера: стратегия инновационного развития>>.}}


\renewcommand{\thefootnote}{\arabic{footnote}}
\footnotetext[1]{University of Salerno, denicola@diima.unisa.it}
\footnotetext[2]{People's Friendship University of Russia, yskhokhlov@yandex.ru}
\footnotetext[3]{University of Pisa, m.pagano@iet.unipi.it}
\footnotetext[4]{Tver State University, Oksana.I.Sidorova@yandex.ru}


\Abste{Since the beginning of the 1990s, accurate traffic
measurements carried out in different network scenarios
highlighted that Internet traffic exhibits strong irregularities ({\it burstiness})
both in terms of extreme variability and long-term correlations.
These features, which cannot be
captured in a parsimonious way by traditional Markovian models, have a deep impact 
on the network performance and lead to the introduction  
of $\alpha$-stable distribution and self-similar processes into the network traffic modeling.
In this paper, a generalization of fractional Brownian motion (fBm), which is 
able to capture both above-mentioned features of the real traffic, is considered.} 


\KWE{fractional Brownian motion; $\alpha$-stable subordinator; self-similar processes; 
buffer overflow probability}

\vskip 14pt plus 9pt minus 6pt

      \thispagestyle{headings}

      \begin{multicols}{2}

            \label{st\stat}

\section{Introduction}

\noindent
The application of probabilistic methods in the modeling and the analysis of telecommunication 
systems has a long history.
Namely, the first researches in this framework date back to the beginning of the last 
century when A.\,K.~Erlang (1878--1929), as a scientific collaborator and the head of 
the newly-established physico-technical laboratory of the Copenhagen Telephone 
Company, studied the issues related to loss and waiting time in automatic telephone 
exchanges.
In the 1930s, the interests for these topics grew from a practical 
as well as theoretical point of view. Indeed, Erlang's results were soon used by 
telephone companies in several countries and gave birth to a new branch  in the framework 
of probability theory, known as queueing theory, which attracted the interests of 
well-known probabilists such as Palm, Pollachek, Lindly, Khincine, Gnedenko, to name 
just a few.     

In the 1920--1930s, many empirical works showed that, in case of 
telephone traffic, a suitable model is represented by the Poisson process. 
At the same time, Poisson flows have many ``useful'' mathematical properties:
\begin{itemize}
\item the superposition of independent Poisson processes is still a Poisson process;    
\item it has independent and stationary increments; and
\item under some mild regularity conditions, the superposition of independent flows 
converges to a Poisson flow, if the number of flows grows, but the individual rates 
become infinitesimal so that the overall rate stays constant. 
\end{itemize}

Because of the last property, in many works it has been proposed that the amount of 
traffic in global telecommunication backbones can be modelled as a Poisson process. 
For several decades, such model has been used without any further experimental 
validation and applied to new network scenarios, such as packet-switching networks.

At the beginning of the 1990s, a lot of empirical studies have been conducted in 
order to better understand the statistical features of packet traffic in global 
networks, such as Internet, as well as in local area networks inside research institutes, university campuses, 
and corporates~[1--3]. Statistical studies of the collected data highlighted their radical 
differences with respect to the ubiquitous Poisson process and other traditional (typically 
Markovian, for the sake of analytical tractability) models.  For instance, it is enough to visually 
check the behavior of real traffic data under different level of aggregations~[4]. 
It is easy to see that at all the aggregation levels (in the range from milliseconds to hours) 
the data keep a random behavior, which appears to be almost the same 
at all the different scales (apart from a normalization factor, related to the length of the 
observation window).

More accurate mathematical analyses~\cite{1-nik} 
pointed out that real data presents {\bf fractal} properties, i.\,e., they can be interpreted as 
trajectories of so-called {\bf automodel} or {\bf self-similar processes}.
Moreover, it was showed that traffic flows, unlike the Poisson model, presents 
{\bf long range dependence}, which has a huge impact on queuing performance.
The third important characteristic of traffic data is that the distribution of many different 
traffic features (such as file length, duration of on and off periods of single sources) presents 
{\bf heavy tails}. 

These  properties of actual traffic flows pointed out the necessity of new traffic models, able to 
captures them in a parsimonious way. It is worth mentioning that similar models were already known 
in the field of probability theory since they have been successfully applied in different frameworks, 
such as turbulence modeling and statistical physics.  

The rationale behind the fractal nature of traffics and the links among the above-mentioned 
characteristics of measured traces have been widely investigated~\cite{5-nik}. 
In particular, it has been 
shown that if locally the traffic load presents heavy tails, then under a sufficiently high level 
of aggregation it converges to a self similar process (for a precise formulation of the problem and 
the related scaling conditions (see~[6--9]). According to the considered aggregation 
regime, two 
different models might arise: fBm and $\alpha$-stable Levy motion, 
which, as will be clarified in the following, present ``opposite'' features.  
In more detail, fBm presents long range dependence, but the tails of its marginal distribution decay 
fast (by definition, according to Gaussian law!). On the contrary,  $\alpha$-stable Levy motion is 
characterised by independent increments (i.\,e., no long memory at all!), but has heavy tailed 
distribution (i.\,e., its tails decay as a power law). 

The goal is to build a model, able to take into account both these features of real traffic. 
Moreover, using such model as input to a queeing system, it would 
be also interesting to determine relevant 
queueing parameters, such as the probability of buffer overflow, which gives an upper bound for 
the loss probability in finite buffer queues. 

From the historical point of view, 
the first attempt to apply the fractional concept to traffic modelling was to use 
fBm $B_H (t)$ instead of traditional Poisson-based models. 
Compared to standard Brownian motion (BM), fBm has one extra parameter, the Hurst parameter 
H, which quantifies the strength of the fractional scaling. It is said usually, that 
fBm is self-similar, or fractional, with Hurst parameter~$H$. In~\cite{10-nik},
Norros has proposed the following model for cumulative traffic 
$$
A(t) = m t + (\sigma m)^{1/2} B_H (t) 
$$
where $m>0$ is the mean input rate, $\sigma$ is the scale factor. This model has been widely studied 
and have been proposed asymptotic lower bounds~\cite{10-nik} as well as exact asymptotics 
in the case of large buffers~\cite{11-nik, 12-nik}.

It is important to point out that in this case, one has a
long-range correlation, but not heavy tails of marginal 
a distributions.

To deal with this issue, several papers extended Norros model by modelling the input traffic as 
$\alpha$-stable Levy motion~\cite{13-nik, 14-nik} or, to take into account also the long range 
correlations, fractional $\alpha$-stable Levy motion (see~\cite{16-nik, 15-nik}).

In the paper, a new variant of  fractional Levy motion is suggeated and, following the 
approach proposed in~\cite{10-nik},
an asymptotic lower bound for the overflow probability is determined. 


\section{Stable Distributions and~Processes}

\noindent
Levy processes have been popular in modeling the teletraffic. Below, some 
definitions are given and some properties of such processes are considered. 

\smallskip

\noindent
\textbf{Definition~1.} \textit{A stochastic process $Y = (Y(t), t\geq 0)$ is a Levy process if}
\begin{enumerate}[(1)]
\item $Y(0) = 0$ \textit{almost surely;} 
\item
 $Y$ \textit{has independent increments; and}  
\item
$Y$ \textit{has stationary increments}. 
\end{enumerate}

\smallskip

Usually, for the sake of regularity,  the following property is required: with 
probability one all trajectories of $Y$ are right-continuous and have finite limits 
from the left. 

The distributions of the process $Y$ is defined uniquely by the distribution of 
random variable $Y(1)$, which is infinitely divisible. 

The most familiar example of Levy process is the BM (Weiner process). 

\smallskip

\noindent
\textbf{Definition~2.} 
\textit{A Levy process $B = (B(t), t\geq 0)$ is called Brownian Motion if for any 
$t\geq 0, h>0$ the increment 
$B(t+h) -B(t)$ has Gaussian distribution with zero meaning and variance $\sigma^2 h$}. 

\smallskip

If $\sigma^2 =1$,  one has a standard BM. It is easily seen that 
$$
K(t,s) = \mbox{Cov} \left(Y(t), Y(s)\right) = \sigma^2 \min (t,s) \, . 
$$

By definition, BM has Gaussian distributions. Such distributions have been got for normalized 
sums of independent identically distributed random variables with finite variance. 
In the case of 
infinite variance,  the so-called stable distributions are considered. 

\smallskip

\noindent
\textbf{Definition~3.} 
\textit{A random variable $Y$ is said to have an $\alpha$-stable distribution if its 
characteristic function has the following form:} 

\noindent
\begin{multline*}
\varphi (\omega ) := E\left[ e^{j\omega X} \right] \\
{}=
\exp \left\{ j\mu\omega - \sigma |\omega |^{\alpha} [1 - j \beta\, \mbox{sgn}\left(\omega \right) 
\theta (\omega , \alpha )] \right\}  
\end{multline*}
\textit{where $0< \alpha \leq 2$, $\sigma\geq 0$, $-1 \leq \beta \leq 1$, $\mu\in R^1$, and} 
$$
\theta (\omega , \alpha ) = 
\begin{cases}
\tan \left( \fr{\alpha \pi }{2} \right)\,, &\ \alpha \not= 1\,; \\[6pt]
-\fr{2}{\pi} \ln |\omega |\,,  &\ \alpha =1\,. 
\end{cases}
$$


\smallskip

Parameter $\alpha$ is called {\it characteristic exponent} and specifies the level of 
burstiness in distribution, i.\,e., it specifies the weight of the tails of the distribution. 
$\sigma$ and $\mu$ are called {\it scale} and {\it location parameters}. 
$\beta$~is called {\it skewness parameter}. If $\beta =0$ then $X$ is symmetrically 
distributed around~$\mu$. If $0<\alpha <1$, $\mu =0$ and $\beta =1$ then~$X$ 
has positive 
values with probability~1. In what follows, a random variable $Y$ is said to have  standard 
$\alpha$-stable distribution if $\mu =0$ and $\sigma = 1$. 

The $\alpha$-stable distribution is infinitely divisible. So, it generates some Levy process. 

\smallskip

\noindent
\textbf{Definition~4.} 
\textit{A stochastic process $L_{\alpha} = (L_{\alpha} (t) , t\geq 0)$ is said to be an 
$\alpha$-stable Levy motion if it is a Levy process such that $L_{\alpha} (1)$ has a given 
$\alpha$-stable distribution}.


\smallskip

If the distribution of $L_{\alpha} (1)$ is totally positive skewed ($0<\alpha <1$, 
$\beta =1$), then all trajectories of the process $L_{\alpha}$ are nondecreasing and 
nonnegative. Such process is called {\it $\alpha$-stable subordinator}. 

If $\alpha =2$, $\mu =0$, one has again BM~$B$. 

There exists very interesting relation between\linebreak $\alpha$-stable Levy motions with 
different~$\alpha$.

\smallskip

\noindent
\textbf{Theorem~1.} 
\textit{If $(L_{\alpha_1 } (t), t\geq 0)$, $0< \alpha_1 \leq 2$, is a $\alpha_1$-stable 
Levy motion with symmetric distributions, 
$(L_{\alpha_2 } (t),$\linebreak $t\geq 0)$, $0<\alpha_2 <1$, is a $\alpha_2$-stable subordinator, 
then stochastic process $Y = (Y(t):= L_{\alpha_1 } (L_{\alpha_2} (t)), t\geq 0)$ 
is $\alpha_1\alpha_2$-stable Levy motion with symmetric distributions.}

\smallskip

This theorem is a corollary of the following result by Zolotarev~[17, theorem~3.3.1]. 

\smallskip

\noindent
\textbf{Theorem~2.} 
\textit{If $Y_1$ has symmetric $\alpha_1$-stable distribution, 
$0<\alpha_1 \leq 2$, $Y_2$ has one-sided 
$\alpha_2$-stable distribution, $0<\alpha_2 <1$, then random variable 
$Y = Y_1 Y_2^{1/\alpha_1}$ has symmetric $\alpha_1\alpha_2$-stable 
distribution.} 

\smallskip

In particular, for $\alpha_1 =2$ and $0<\alpha_2 =\alpha/2 <1$, one gets the following 


\smallskip

\noindent
\textbf{Theorem~3.} 
\textit{If $B= (B(t), t\geq 0)$ is the Brownian motion, 
$L_{\alpha/2} = (L_{\alpha/2} (t) , t\geq 0)$ is a $\alpha/2$-stable subordinator, 
then  $L_{\alpha} = (L_{\alpha} (t) := B(L_{\alpha/2} (t)), t\geq 0)$, $0<\alpha <2$, is 
an $\alpha$-stable Levy motion with  symmetric distributions. }


\section{Self-Similar Processes} 


\noindent
\textbf{Definition~5.} 
\textit{A process $Y = (Y(t), t\geq 0)$ is self-similar, with Hurst parameter $H\geq 0$, if it 
satisfies the condition 
$$
Y(t) \stackrel{d}{=} c^{-H} Y(ct)\,, \ \forall t\geq 0\,, \ \forall c>0\,,
$$
where the equality is the sense of finite-dimensional distributions.}

\smallskip

Two of the most popular examples of self-similar processes are 
fBm and $\alpha$-stable Levy motion. 

\smallskip

\noindent
\textbf{Definition~6.}
\textit{The fractional Brownian motion with Hurst parameter $H$ is a Gaussian process 
$(B_H (t), t\geq 0)$ with zero mean and correlation function}
$$
K_H (t,s) = \fr{1}{2} \left[ |t|^{2H} + |s|^{2H} - |t-s|^{2H} \right] \,.
$$ 

\smallskip

The definition of $\alpha$-stable Levy motion see above. 

More information about stable and self-similar processes can be found in~\cite{18-nik, 19-nik}. 

\section{New Variant of~Fractional Levy Motion}

\noindent
Above, it was shown  how to get symmetric $\alpha$-stable Levy motion using 
BM and $\alpha/2$-stable subordinator. Below, it is proposed to use the same 
construction to get fractional Levy motion from fBm $B_H$ and 
$\alpha/2$-stable subordinator~$L_{\alpha/2}$. 

Let $(B_H (t), t\geq 0)$ be the fBm with Hurst parameter~$H$,  
$(L_{\alpha}^1 (t), t\geq 0)$, $(L_{\alpha}^2 (t), t\geq 0)$  be standard $\alpha$-stable 
subordinators, $0<\alpha < 1$, and~$B_H$, $L_{\alpha}^1$ and 
$L_{\alpha}^2$ are independent. Consider the new process 
$$
X(t) := 
\begin{cases}
B_H (L_{\alpha}^1 (t))\,, &\ t\geq 0 \,; \\[6pt]
- B_H (L_{\alpha}^2 (t))\,, &\  t < 0 \,. 
\end{cases}
$$

\noindent
\textbf{Theorem~4.} \textit{The above process $X$ is self-similar with Hurst parameter 
$H_1 = H/\alpha$}. 

\smallskip

\noindent
P\,r\,o\,o\,f\,.\  The processes $(L_{\alpha}^k (t), t\geq 0)$, $k=1,2$, 
are $\alpha$-stable 
and self-similar with Hurst parameter $1/\alpha$. So, one has
$$
(L_{\alpha}^k (ct), t\geq 0) \stackrel{d}{=} 
(c^{1/\alpha} L_{\alpha}^k (t), t\geq 0) \,.
$$
Then, 
\begin{multline*}
(X(ct) , t\in R^1 ) = \pm B_H (L_{\alpha}^k (c|t|), t\in R^1 ) \\
{}\stackrel{d}{=} 
(\pm B_H (c^{1/\alpha} L_{\alpha}^k (|t|) , t\in R^1 ) \,.
\end{multline*}
Using self-similarity of $B_H$ for fixed 
$\tau = L_{\alpha}^k (|t|)$, for any $a>0$, one has
$$
(B_H (a\tau ), \tau\geq 0) \stackrel{d}{=} (a^H  B_H (\tau), \tau\geq 0) 
$$
or 
$$
(\pm B_H (c^{1/\alpha} \tau ), \tau\geq 0) \stackrel{d}{=} 
(\pm c^{H/\alpha} B_H (\tau), \tau\geq 0) \,. 
$$
Due to the complete probability formula, the result is obtained. 

\smallskip

\noindent
\textbf{Corollary~1.} \textit{For any $t>0$, 
$$
X(t) \stackrel{d}{=} (L_{\alpha}^1 (t))^H  Y
$$
where $Y$ has standard normal distribution and $L_{\alpha}^1 (t)$ and~$Y$ are 
independent.} 

\smallskip

\noindent
\textbf{Remark.} Hurst parameter $H_1$ for above process $X$ can be any positive 
number. But for traffic applications, it is more interesting the case where 
$1/2 <H_1 <1$. So, it is assumed in what follows. 

\smallskip

\noindent
\textbf{Theorem~5.} 
\textit{The above process $X$ has stationary increments.}

\smallskip

\noindent
P\,r\,o\,o\,f\,.\ \ Fractional Brownian motion $B_H$ has stationary increments. So for any 
$t_1 < t_2 $
$$
B_H (t_2 ) - B_H (t_1 ) \stackrel{d}{=} B_H (t_2 - t_1 ) \,. 
$$ 
Then, for any $t\geq 0$, $h>0$ and fixed $L_{\alpha}^k (t+h)\hm = t_2$, 
$L_{\alpha}^k (t) = t_1$, one has 
$$
B_H (L_{\alpha}^k (t+h)) - B_H (L_{\alpha}^k (t)) \stackrel{d}{=} 
B_H (L_{\alpha}^k (t+h) -  L_{\alpha}^k (t)) \,. 
$$ 
Due to the complete probability formula, one has the same for random moments of time. 
The process $L_{\alpha}^k (t)$ has stationary increments too. So, one gets  
$$
B_H (L_{\alpha}^k (t+h) -  L_{\alpha}^k (t))  
\stackrel{d}{=} 
B_H (L_{\alpha}^k (h) ) \,.
$$ 


\section{Application to~Traffic Modeling} 

\noindent
Define the cumulative traffic (or arrival) process $A(t)$, i.\,e.,
the total amount of load produced 
by a source in the time interval $[0,t]$, $t>0$, by 
$$
A(t) := mt + (\sigma m)^{1/\beta} X(t) \,, 
$$
where $m>0$ is the mean input rate, $\sigma$ is the scale factor, 
$\beta = \alpha/H = 1/H_1$, $X$ is the process defined above. 


Consider a single server queue with constant service rate $r>0$ and infinite buffer 
space, where input is the stable self-similar process defined above 
($r>m$ for stability). The buffer occupancy 
$Q(t,r)$ at time $t\in R^1$ (queue size or queue length) can be written as 
$$
Q(t,r) = \sup\limits_{s\leq t} (A(t) - A(s) - r (t-s)) \,.
$$

\smallskip

Due to theorem~2, the process $Q = (Q(t,r)$, $t$\linebreak $\in R^1),$ is stationary. 
So, the most 
interesting is the following probability of overflow: 
$$
\varepsilon (b) = P(Q(0, r) > b) = P\left(\sup\limits_{\tau \geq 0} 
\left(A(\tau ) - r \tau \right) >b \right) \,.
$$

Using the technique elaborated in papers~\cite{1-nik, 13-nik}, one can get the lower bound for 
the probability of buffer overflow for large~$b$. 

It is easily seen that 
\begin{multline*}
\varepsilon (b) \geq 
\sup\limits_{\tau\geq 0} P( (A(\tau ) - r\tau ) > b ) \\
{}=
\sup\limits_{\tau\geq 0} P( m\tau +(\sigma m )^{1\/\beta} X(\tau ) - r\tau  > b )\\
{}=
\sup\limits_{\tau\geq 0} P\left( X(\tau ) > 
\fr{b+(r-m)\tau}{(\sigma m)^{1/\beta}}\right ) \\
{}=
\sup\limits_{\tau\geq 0} P\left( \tau^{1/\beta} X(1) > 
\fr{b+(r-m)\tau}{(\sigma m)^{1/\beta}}\right ) \\
{}=
\sup\limits_{\tau\geq 0} P\left( X(1) > 
\fr{b+(r-m)\tau}{(\sigma m \tau )^{1/\beta}}\right ) \,. 
\end{multline*}
Last probability under supremum is a decreasing function of the value 
$$
f(\tau ) = \fr{b+(r-m)\tau}{(\sigma m \tau )^{1/\beta}}\,.
$$
Elementary calculations give us that the minimal value of this function is achieved at the 
point 
$$
\tau_0 = \fr{b}{\beta (1-1/\beta )(r-m)} = \fr{b H_1}{(1-H_1 ) (r-m)} \,. 
$$
It follows 
$$
\varepsilon (b) \geq P(X(1) > f(\tau_0 ) = b_1 ) 
$$ 
where 
$$
b_1 = \fr{(r-m)^{H_1} (1-H_1 )^{-(1-H_1 )} }{(\sigma m H_1 )^{H_1 } } \, b^{1-H_1} \,. 
$$
Using corollary 1, one gets 
\begin{multline*}
P(X(1) > b_1 ) = P((L_{\alpha}^1 (1))^H Y >b_1 ) \\
{}\geq  
P((L_{\alpha}^1 (1))^H Y >b_1 , Y>1) \\
{}\geq
P((L_{\alpha}^1 (1))^H >b_1 , Y>1)\\ = P(L_{\alpha}^1 (1) 
>(b_1 )^{1/H} ) P(Y>1) \,. 
\end{multline*}
For large $x>0$ (see~[20, theorem~2.4.1]), one has 
$$
 P(L_{\alpha}^1 (1) > x ) \sim C(\alpha ) x^{-\alpha} 
$$
where 
$$
C(\alpha ) = \fr{\sin (\pi\alpha )}{\pi} \Gamma (\alpha )\,.
$$
It follows for large~$b$ 
\begin{multline*}
\varepsilon (b) \geq C(\alpha ) (b_1 )^{-1/H_1} P(Y>1)\\
{}=
C_1 (\alpha , H_1 )\sigma \fr{m}{r-m} \, b^{-({1-H_1})/H_1} \,.
\end{multline*}
Finally, one has the following 

\pagebreak

\smallskip

\noindent
\textbf{Theorem~6.} \textit{An asymptotic lower bound for the overflow probability is given by}
$$
\varepsilon (b) \geq
C_1 (\alpha , H_1 )\sigma \fr{m}{r-m} \, b^{-({1-H_1})/{H_1} } \,, \quad
b\to\infty \,.
$$

\vspace*{-12pt}

{\small\frenchspacing
{%\baselineskip=10.8pt
\addcontentsline{toc}{section}{Литература}
\begin{thebibliography}{99}

\bibitem{1-nik}
\Au{Leland~W., Taqqu~M., Willinger~W.,  Wilson~D}. On the selfsimilar
nature of Ethernet traffic (extended version)~// IEEE/ACM
Trans. Networking, 1994. P.~1--15. 

\bibitem{3-nik} %2
\Au{Park K., Kim G., Crovella~M.}  
On the relationship between file sizes, transport protocols, and self-similar 
network traffic~//  Conference (International) on Network Protocols Proceedings, October 1996. 
P.~171--180. 

\bibitem{2-nik} %3
\Au{Crovella M.\,E., Bestavros~A.} Self-similarity in world wide web traffic: Evidence 
and possible  causes~// IEEE/ACM Transactions on Networking, December 1997. Vol.~5. No.\,6. 
P.~835--846. 


\bibitem{4-nik}
\Au{Willinger W., Paxson~V.} Where mathematics meets the Internet~// Notices of the AMS, 1998. 
Vol.~45. No.\,8. P.~961--970.  

\bibitem{5-nik}
\Au{Park~K., Willinger~W.} Self-similar network traffic and performance 
evaluation.~--- Wiley, 2000. 

\bibitem{6-nik}
\Au{Taqqu M.\,S., Levy J.\,B.} Using renewal processes to
generate long-range dependence and high variability~// 
Dependence in probability and statistics~/
Eds.\ E.~Eberlein and M.\,S.~Taqqu.~---  Boston: Birkhauser, 1986.  P.~73--89. 

\bibitem{7-nik}
\Au{Taqqu M.\,S., Willinger~W., Serman~R}. Proof of a
fundamental result in self-similar traffic modeling~// Computer Communications
Rev., 1997. Vol.~27. No.\,2. P.~5--23. 

\bibitem{8-nik}
\Au{Levy J.\,B., Taqqu M.\,S.} Renewal reward processes with
heavy-tailed inter-aarival times and heavy tailed rewards~// Bernoulli, 2000. Vol.~6. No.\,1. 
P.~23--44. 

\bibitem{9-nik}
\Au{Mikosch Th., Resnick S., Rootzen~H., Stegeman~A.} Is network
traffic approximated by stable Levy motion or frac-\linebreak\vspace*{-12pt}

\columnbreak

\noindent
tional Brownian motion?~//
Ann. Appl. Probab., 2002. Vol.~12. No.\,1. P.~23--68. 

\bibitem{10-nik}
\Au{Norros I.} A storage model with self-similar imput~// 
Queuing Syst., 1994. Vol.~16. P.~387--396. 

\bibitem{11-nik}
\Au{Narayan~O.} Exact asymptotic queue length distribution for fractional Brownian traffic~//
  Adv. Perf. Anal., 1998. Vol.~1. P.~39--63.

\bibitem{12-nik}
\Au{H$\ddot{\mbox{u}}$sler J., Piterbarg~V.} Extremes of a certain class of Gaussian processes~// 
Stoch. Proc. Appl., 1999. Vol.~83. P.~257--271.

\bibitem{13-nik}
\Au{Laskin N., Lambadaris I., Harmantzis~F.\,C., Devetsikiotis~M.}
Fractional Levy motion and its application to network traffic modeling~// 
Computer Networks, 2002. Vol.~ 40. P.~ 363-375. 

\bibitem{14-nik}
\Au{Garroppo~R.\,G., Giordano~S., Pagano~M., Procissi~G.}
Testing $\alpha$-stable processes in capturing the queuing behavior of
broadband teletraffic~// Signal Proc., 2002. Vol.~82.  P.~1861--1872.  

\bibitem{16-nik}
\Au{Gallardo J.\,R., Makrakis~D., Orozco-Barbosa~L.}  Use of $\alpha$-stable 
self-similar stochastic processes for modeling traffic~//  Performance Eval., 
2000. Vol.~40. P.~71--98.

\bibitem{15-nik}
\Au{Karasaridis~A.} Network heavy traffic modeling using\linebreak $\alpha$-stable
self-similar processes~// IEEE Transactions on  Communications, July 2001. Vol.~49. No.\,7. 
P.~1203--1214.


\bibitem{17-nik}
\Au{Zolotarev V.\,M.} One-dimensional stable distributions.~--- 
Translations of mathematical monographs.  AMS, 1986. Vol.~65.

\bibitem{18-nik}
\Au{Samorodnitsky~G.,  Taqqu~M.\,S.} Stable
non-Gaussian random processes.~---  Chapman \& Hall, 1994. 

\bibitem{19-nik}
\Au{Embrechts~P., Maejima~M.} Self-similar process.~---
Prinston University Press, 2002. 

\bibitem{20-nik}
\Au{Ibragimov I.\,A., Linnik~Yu.\,V.}  Independent and stationary sequences of 
random variables.~--- Gronengen: Wolters-Noordhoff, 1971. 

 \end{thebibliography}
}
}


\end{multicols}

\vspace*{3pt}

\hrule

%\vspace*{pt}


\def\tit{ДРОБНОЕ ДВИЖЕНИЕ ЛЕВИ С~ЗАВИСИМЫМИ ПРИРАЩЕНИЯМИ 
И~ЕГО~ПРИЛОЖЕНИЕ К~МОДЕЛИРОВАНИЮ СЕТЕВОГО  ТРАФИКА}

\def\aut{К.~Де~Никола$^1$, Ю.\,С.~Хохлов$^2$,  М.~Пагано$^3$,  О.\,И.~Сидорова$^4$}

\titelr{\tit}{\aut}

%\vspace*{2pt}

\noindent
$^1$Университет г.~Салерно, denicola@diima.unisa.it\\
\noindent
$^2$Российский университет дружбы народов, yskhokhlov@yandex.ru\\
\noindent
$^3$Университет г.~Пиза, m.pagano@iet.unipi.it\\
\noindent
$^4$Тверской государственный университет, Oksana.I.Sidorova@yandex.ru\\

\vspace*{-6pt}


\Abst{С начала 1990-х~гг.\ были проведены многочисленные высокоточные измерения для различных 
сетевых сценариев, которые показали, что трафик в Интернете проявляет сильную иррегулярность, выраженную 
в чрезвычайной вариабельности, а так\-же в наличии долговременной зависимости. Эти новые особенности, 
которые не удается описать экономным образом с помощью традиционных марковских моделей, имеют сильное 
влияние на поведение сети, и это привело к необходимости введения в моделирование сетевого трафика 
$\alpha$-устой\-чи\-вых распределений и самоподобных процессов.
В~настоящей работе рассматривается некоторое обобщение дробного броуновского движения, которое 
позволяет охватить одновременно обе отмеченные выше особенности реального трафика.}

\label{end\stat}


\KW{дробное броуновское движение; $\alpha$-устойчивый субординатор; самоподобные 
процессы; вероятность переполнения буфера}


\renewcommand{\figurename}{\protect\bf Рис.}
\renewcommand{\tablename}{\protect\bf Таблица}
\renewcommand{\bibname}{\protect\rmfamily Литература}      %6

 \renewcommand{\figurename}{\protect\bf Figure}
\renewcommand{\tablename}{\protect\bf Table}
\renewcommand{\bibname}{\protect\rmfamily References}

\def\stat{yan}

\def\tit{ABOUT THE RATE OF~CONVERGENCE OF~ONE U-STATISTIC}

\def\titkol{About the rate of~convergence of~one U-statistic}

\def\autkol{O.~Yanushkevichiene and R.~Yanushkevichius}
\def\aut{O.~Yanushkevichiene$^{1}$ and R.~Yanushkevichius$^{2}$}

\titel{\tit}{\aut}{\autkol}{\titkol}

%{\renewcommand{\thefootnote}{\fnsymbol{footnote}}\footnotetext[1]
%{Received by the editors November, 2011. 1991 \textit{Mathematics Subject Classification}.
%Primary 62E20; Secondary 60F05.}}


\renewcommand{\thefootnote}{\arabic{footnote}}
\footnotetext[1]{Vilnius University, Institute of Mathematics and Informatics, Lithuania, olgjan@zebra.lt}
\footnotetext[2]{Lithuanian University of Educational Sciences, Vilnius, Lithuania}

\vspace*{-10pt}

\Abste{This paper pertains to the analysis of the asymptotic behavior of 
U-statistics, which are important in the construction and application of 
modern statistical methods for studying information systems.}

\vspace*{-1pt}

\KWE{rates of convergence; Berry--Esseen bound; quadratic form; second-degree U-statistics; 
Kolmogorov distance}

\vskip 14pt plus 9pt minus 6pt

      \thispagestyle{headings}

      \begin{multicols}{2}

            \label{st\stat}


\section{Introduction and~Formulation of~the~Result}

\noindent
Analysis of modern information systems is impossible without the use of statistical 
methods. Function exploring the characteristics of the traffic is usually characterized 
by a symmetry property with respect to the arguments. This makes it especially important 
to study the behavior of U-statistics. This paper pertains to the investigation of the 
limit distribution of the second-order U-statistic.

Let $X,X_1,\ldots ,X_n$ be independent identically distributed (i.i.d.)\
random variables taking values in a measurable space  $(\Theta, \
\mathcal{R})$. Let $h: \ \Theta^{2}\rightarrow\mathbf{R}$ be real-valued
measurable functions. Let $h$ be symmetric, that is,
$h(x,y)=h(y,x)$ for all $x,y\in\Theta$. Consider the \mbox{U-statistic}

\vspace*{2pt}

\noindent
\begin{equation}
\label{e1-y}
  Z=Z(X_1,\ldots ,X_n)=n^{-1} \sum\limits_{1\leq i<k\leq n}h(X_i,X_k)
\end{equation}
assuming that $ \beta_{2}=\mathbf{E}|h(X,X_1)|^{2}>0$, 
$\beta_{3}$\linebreak $=\mathbf{E}|h(X,X_1)|^{3} <\infty$, and 
$\mathbf{E}h(x,X)=0$ for all  $x \in\Theta$.

    The condition $\beta_{2}>0$ ensures
    that the quadratic part of the statistic is not asymptotically negligible and,
    therefore,~$Z$ is not asymptotically normal. To be more precise, the asymptotic distribution
    of $Z$ is non-Gaussian and is given by the distribution of the random variable
    
    \vspace*{2pt}
    
    \noindent
  \begin{equation*}
%  \label{5-y}
    Z_0=\fr{1}{2}\sum\limits_{i\geq 1}q_i(\eta_i^{2}-1)\,. 
    \end{equation*}
    Here, $\eta_i$ is the sequence of i.i.d.\ standard normal random variables and $q_1,q_2,\ldots$
    denote the eigenvalues of
    the Hilbert--Schmidt operator, say~$Q$, associated with the kernel~$h$ (see~[1] for
    detailed definitions). Without loss
    of generality, let assume throughout that $|q_1|\geq |q_2|\geq\ldots$

    Let
    
    \noindent
    \begin{equation*}
    \label{o}
    F(x)=\mathbf{P}\{Z\leq x\}\,; \enskip F_0(x)=\mathbf{P}\left\{Z_0\leq x\right\}\,.
\end{equation*}
     Let write
     
     \noindent
     \begin{equation*}
%     \label{o1}
    \Delta_{n}=\rho(Z,Z_0)=\sup\limits_{x}|F(x)-F_0(x)|\,.
    \end{equation*}

    Korolyuk and Borovskikh proved~\cite{3-y} 
    that $\Delta_{n}$\linebreak $ =o(n^{-1/2})$ if $|q_i| >0$ for all~$i$.
    Under the condition $|q_9| >0$, an optimal bound $\Delta_{n} =O(n^{-1})$ was established
    by Bentkus and G$\ddot{\mbox{o}}$tze~\cite{1-y}. From a construction in~\cite{1-y}, it is
    clear that the bounds for $\Delta_{n}$ are related to the estimates of
    the convergence rate in Hilbert spaces. Therefore, if $|q_k| >0$, a result of
    Senatov~\cite{6-y} implies a lower bound for $\Delta_{n}$, namely,
    $\liminf\limits_{n\rightarrow\infty}\Delta_{n}n^{k/12}(q_1\cdots q_k)^{1/2} >0,$ then $k \leq6$. 
    In the considered case, $k =1$ and the lower bound is $n^{k/12}$. The rate of convergence of such 
    order was got in~\cite{9-y} for the case, when the expression~(\ref{e1-y}) 
    has not only the quadratic part, but also the linear part. Now, the question is: 
    what order of the rate of convergence will be, if the linear part does not exist? 
    The theorem bellow answers this question.

    By $c$, positive absolute constants which may
    differ from line to line or from formula to formula are denoted. The 
    following theorem is the authors' result.
    
\smallskip

\noindent
\textbf{Theorem~1.} \textit{If $q_1>0$, one gets}

\noindent
   \begin{equation}\label{6-y}
    \Delta_{n}\leq \fr{c\beta_3^{1/6}}{\sqrt{q_1}n^{1/14}}+\fr{c}{n^{1/4}}
    \left(\sum\limits_{i\geq 1}\left\vert\fr{q_i}{q_1}\right\vert\right)^{1/2}+
    \fr{c}{\sqrt{q_1}n}\,.
    \end{equation}

\smallskip


\noindent
\textbf{Remark~1.} If on the right-hand side of~(\ref{6-y}) the sum $\sum\limits_{i \geq
    1}|q_i|$ is divergent, i.\,e.,
    
    \noindent
    \begin{equation}
    \label{r}
    \sum\limits_{i\geq 1}|q_i|=\infty\,,  
    \end{equation}
    then $\Delta_{n}=O(n^{-1}).$

\smallskip

    Indeed, in principle, the series on the right-hand side of~(\ref{6-y}) 
    can diverge. Then, the estimate~(\ref{6-y}) is naturally
    true, but useless. However, if~(\ref{r}) holds, then $|q_9| >0$,
    since
    $|q_1| \geq |q_2| \geq\cdots\geq |q_9| \geq\cdots\geq |q_n| \geq0$. Then,
let use Theorem~1.1 from~\cite{1-y}. The theorem states that
    if $|q_9| >0$, then
     $\Delta_{n} =O(n^{-1})$. That is, if the series mentioned is
    divergent,then estimate~(\ref{6-y}) can be replaced by the
    improved estimate from~\cite{1-y}.

%\pagebreak

    In the one-dimensional case, that is, in the case
    $q_2 =q_3=\cdots =0$,
    the bound~(\ref{6-y}) $\Delta_{n} =O(n^{-1/14})$ improves to $\Delta_{n} =O(n^{-1/4})$, and
    the rate $O(n^{-1/4})$ is best possible~\cite{8-y}.
    The result of Senatov~\cite{6-y} shows that the optimal order of bound~(\ref{6-y}) 
    is $O(n^{-1/12})$.

    The authors' results answer the question of V.~Bentkus, who
    also suggested an idea of the proof.
    
    \vspace*{-6pt}


    \section{Proof}

    \vspace*{-2pt}

    \subsection{Lemma}

\noindent
    To prove the theorem, one needs the following lemma.

\smallskip

\noindent
\textbf{Lemma~1.} \textit{Let $\eta$ be a standard Gaussian random variable and $q_1>0$.
    Then the distribution function
    $H(x)=\textbf{P}\{q_1\eta^2<x\}$ satisfies the Lipschitz condition
       \begin{equation}
       \label{6l}
   \sup\limits_{x}|H(x+\varepsilon)-H(x)|\leq c\sqrt{\fr{\varepsilon}{q_1}}
    \end{equation}
    where} $\varepsilon >0$.

\smallskip

\noindent
P\,r\,o\,o\,f\ \ of Lemma~1.
    The distribution of
    $\eta^2$ has the density
    $$ 
    f(x)=
    \begin{cases}
    \fr{1}{\sqrt{2}\Gamma(1/2)\sqrt{x}}e^{-x/2} &\ \mbox{for} \ x>0\,;\\
    0 &\ \mbox{for} \ x\leq 0\,.
    \end{cases}
    $$
    The function $f(u)$ is decreasing for $u >0$.
    One can write:
    $$\sup\limits_{x}|H(x+\varepsilon)-H(x)|\leq
    c\int\limits^{\varepsilon/q_1}_{0}\fr{e^{-u}}{\sqrt{u}}\,du\leq c\sqrt{
    \fr{\varepsilon}{q_1}}\,.
    $$
    The lemma is proved.

\smallskip

\noindent
\textbf{Lemma~2.} \textit{Let $\eta$ be a standard Gaussian random variable and $q_1 >0$.
    Then the the following inequality is true:}
      \begin{equation*} %multline*}
%      \label{le}
   \sup\limits_{x}\left\vert\mathbf{P}(q_1\eta_1^{2}- \fr{1}{n}q_1\eta_1^{2}\leq x)-
   \mathbf{P}(q_1\eta_1^{2}\leq x)\right\vert %\\
\leq
\fr{c}{\sqrt{q_1}n}\,.
    \end{equation*} %multline*}
   
\smallskip

\noindent
P\,r\,o\,o\,f\ \ of Lemma~2.
    It is easy to see that
\begin{multline*}
\left\vert\mathbf{P}\left(q_1\left(1-\fr{1}{n}\right)\eta_1^{2}\leq x\right)-
\mathbf{P}(q_1\eta_1^{2}\leq
    x)\right\vert\\
    {}\leq c\int\limits_{0}^{2/\sqrt{q_1}n}e^{-t^{2}/2}\,dt\leq \fr{c}{\sqrt{q_1}\,n}\,.
    \end{multline*}
    The lemma is proved.

\columnbreak

It is necessary to prove the authors' theorem, i.\,e., the bound~(\ref{6-y}).
    Let $G$, $G_i$, $1 \leq i \leq n$, be i.i.d.\ Gaussian random vectors
    $G_i =(G_{1,i},G_{2,i},\ldots)$ with
    values in $\mathbb{R}^{\infty}$ where $G_{1,i},G_{2,i},\ldots$ denote i.i.d.\ standard normal
    random variables. Let assume throughout that
\begin{equation}
\left.
\begin{array}{rl}
     \mathbf{E}h(x,G)&=0\,;\\[9pt]
     \mathbf{E}h(x,G)h(y,G)&=\mathbf{E}h(x,X)h(y,X)
    \end{array}
 \right\}
 \label{m}
    \end{equation}
for~all $x\in\Theta$.
    Note that the possibility of selecting Gaussian random
    variables so that Eqs.~(\ref{m}) are valid is substantiated
    in~[1, p.~461].

     According to the triangle inequality, one has
     \begin{multline}
     \label{7-y}
    \rho (Z,Z_0)\leq \rho(Z(X_1,\ldots,X_n),Z(G_1,\ldots,G_n))\\
    {}+\rho(Z(G_1,\ldots,G_n),Z_0)\,.
     \end{multline}
     In view of~(\ref{7-y}), to prove~(\ref{6-y}),  it is sufficient to establish that
     \begin{multline}
   \rho(Z(X_1,\ldots,X_n),Z(G_1,\ldots,G_n))\\
   {}\leq cq_1^{-1/2}\beta_3^{1/6}n^{-1/14}
        \label{a-y}
     \end{multline}
     and
    \begin{multline}
   \rho(Z(G_1,\ldots,G_n),Z_0)\\
   {}\leq \fr{c}{n^{1/4}}
    \left(\sum\limits_{i\geq 1}\left\vert\fr{q_i}{q_1}\right\vert\right)^{1/2}
+
    \fr{c}{\sqrt{q_1}n}\,,
        \label{b}
     \end{multline}

     \subsection{Proof of inequality~(\ref{a-y})}

\noindent
Let estimate $\rho(Z(X_1,\ldots,X_n),Z(G_1,\ldots,G_n))$.
    It is to verify using lemma~1 that the distribution function of
    $Z(G_1,\ldots,G_n)$ satisfies the Lipschitz condition with the exponent~$1/2$.

    Let prove that, for any $\varepsilon  >0$, one has
         \begin{equation}
         \label{c4}
\rho(Z(X_1,\ldots,X_n),Z(G_1,\ldots,G_n))\leq \fr{c}{\sqrt{q_1}}\sqrt{\varepsilon}+\Delta\!
    \end{equation}
where
    $$
    \Delta=\max\limits_\varphi |\mathbf{E}\varphi(Z(X_1,\ldots,X_n))-
    \mathbf{E}\varphi(Z(G_1,\ldots,G_n))|\,,  
    $$
    and maximum is taken over all infinite differentiable $\varphi$ such
    that $0 \leq\varphi(u) \leq1$ and
      $|\varphi^{(k)}(u)| \leq
    c_1/\varepsilon^{k}$, $k =1,2,3$:
    \begin{equation}
    \label{c2}
\varphi(u)=
    \begin{cases}
    1\,, &\mbox{if}\  u\leq x-\epsilon\,; \\[6pt]
    0\,, &\mbox{if}\ u\geq x\,,
    \end{cases}
\end{equation}
    or
    
    \noindent
    \begin{equation}
    \label{c3}
    \varphi(u)=
    \begin{cases}
    1\,, &\mbox{if}\ u\leq x\,; \\[6pt]
    0\,, &\mbox{if}\ u\geq x+\epsilon\,. 
    \end{cases}
    \end{equation}
Let write $\delta^{*} =P\{Z(X_1,\ldots,X_n) \leq x\}$\linebreak $ -\;P\{Z(G_1,\ldots,G_n) \leq x\}$.
    Let prove~(\ref{c4}) in the case
    $\delta^{*} \geq 0.$  Let 
    take a function $\varphi$ such that~(\ref{c2}) holds. Then
    
    \noindent
    \begin{multline*}
\delta^{*}  = EI\{Z(X_1,\ldots,X_n)\leq x\}\\
{}-P\{Z(G_1,\ldots,G_n)\leq x\} \\
 \leq 
    |E\varphi(Z(X_1,\ldots,X_n))-E\varphi(Z(G_1,\ldots,G_n))|\\
    {}+|E\varphi(Z(G_1,\ldots,G_n))-
    P\{Z(G_1,\ldots,G_n)\leq
    x\}|  \\
     \leq 
    \Delta+P\{x\leq Z(G_1,\ldots,G_n)\leq x+\varepsilon\}\,. 
    \end{multline*}
    Using the Lipschitz condition, one gets~(\ref{c4}).

    The proof is similar if $\delta^{*} <0$, taking a function~$\varphi$ satisfying~(\ref{c3})
    instead of~(\ref{c2}).

    Let estimate
    
    \vspace*{1pt}
    
    \noindent
    $$
    \Delta^{*}(\varphi)=|\mathbf{E}\varphi(Z(X_1,\ldots,X_n))-
    \mathbf{E}\varphi(Z(G_1,\ldots,G_n))|\,.
    $$
    It is easy to see that
    
    \noindent
    \begin{multline}
    \Delta^{*}(\varphi)\leq |\mathbf{E}\varphi(Z(X_1,\ldots,X_n))\\
    {}-
    \mathbf{E}\varphi(Z(X_1,\ldots,X_{n-1},G_n))|\\
{}+
    |\mathbf{E}\varphi(Z(X_1,\ldots,X_{n-1},G_n))\\
    {}-
    \mathbf{E}\varphi(Z(X_1,\ldots,X_{n-2},G_{n-1},G_n))|+\ldots{}
\\
\ldots{} +
     |\mathbf{E}\varphi(Z(X_1,G_{2},\ldots,G_n))\\
     {}-
     \mathbf{E}\varphi(Z(G_1,\ldots,G_n))|=\Delta^{*}_{1,n}+\ldots+\Delta^{*}_{n,n}\,.
     \label{e8-y}
     \end{multline}
One has

\noindent
    \begin{multline*}
    Z(X_1,\ldots,X_n)=\fr{1}{n}\{h(X_1,X_2)+\cdots{}\\
\cdots +h(X_1,X_{n-1})+h(X_1,X_n)+\cdots\\
\cdots+h(X_{n-2},X_{n-1})+h(X_{n-2},X_n)+{}\\
{}+h(X_{n-1},X_n)\}\,.
\end{multline*}
    one designates:
    
    \noindent
    \begin{multline*}
    w=\fr{1}{n}\{h(X_1,X_2)+\cdots+h(X_1,X_{n-1})+
    \cdots{}\\
    {}\cdots +h(X_{n-2},X_{n-1})\}
    \end{multline*}
    and
    \begin{multline*}
    l_n=\fr{1}{n}\left\{h(X_1,X_n)+\cdots+h(X_{n-2},X_n)\right.\\
\left.    {}+h(X_{n-1},X_n)\right\}\,.
    \end{multline*}
    Replacing $X_n$ by $G_n$, one gets:
    
    \noindent
    \begin{multline*}
    Z(X_1,\ldots,X_{n-1},G_n)=w+\fr{1}{n}\{h(X_1,G_n)+\cdots{}\\
{}    \cdots+h(X_{n-2},G_n)+
    h(X_{n-1},G_n)\}
    {}= w+l_n^{*}\,.
    \end{multline*}
    Let expand into the Taylor series:
    \begin{multline*}
    \varphi(x+y)=\varphi(x)+\varphi^{'}(x)y+\fr{1}{2}\varphi^{''}(x)y^{2}\\
    {}+\fr{1}{2}
    \mathbf{E}\varphi^{'''}(x+\tau y)(1-\tau)^{2}y^3\,. 
    \end{multline*}
    Here, $\tau$ is the random
    variable uniformly distributed in $[0,1]$ and independent of
    all the other random variables. Let replace $x$ by $w$ and $y$ by $l_n$, respectively.
Let write

\noindent
     \begin{multline*}
    \Delta^{*}_{1,n}\\
    {}=\left\vert\mathbf{E}\varphi(Z(X_1,...,X_n))-
    \mathbf{E}\varphi(Z(X_1,\ldots,X_{n-1},G_n))\right\vert\\
    {}=
    \left\vert\vphantom{\fr{1}{2}}
    \mathbf{E}\varphi(w)-\mathbf{E}\varphi(w)+\mathbf{E}\varphi^{'}(w)l_n-
    \mathbf{E}\varphi^{'}(w)l_n^{*}\right.\\
    {}+
    \fr{1}{2}\,\mathbf{E}\varphi^{''}(w)l_n^2-
    \fr{1}{2}\,\mathbf{E}\varphi^{''}(w)(l_n^{*})^2\\
{}+
    \fr{1}{2}\mathbf{E}\varphi^{'''}(w+l_n\tau )l_n^3(1-\tau)^{2}\\
\left.    {}-
    \fr{1}{2}\mathbf{E}\varphi^{'''}(w+l_n^{*}\tau)(l_n^{*})^3(1-\tau)^{2}\right\vert\,.
    \end{multline*}
    Now, let condition $X_1,\ldots,X_{n-1},\tau$:
    
    \noindent
    \begin{multline*}
    \Delta^{*}_{1,n}=
    \Big|\mathbf{E}_{X_1,\ldots,X_{n-1},\tau} \mathbf{E}_{X_{n}}\varphi^{'}(w)l_n\\
    {}-
    \mathbf{E}_{X_1,\ldots,X_{n-1},\tau}
    \mathbf{E}_{X_{n}}\varphi^{'}(w)l_n^{*}\\
{}+\fr{1}{2}\,
    \mathbf{E}_{X_1,\ldots,X_{n-1},\tau} \mathbf{E}_{X_{n}}\varphi^{''}(w)l_n^2\\
    {}-
    \fr{1}{2}\,\mathbf{E}_{X_1,\ldots,X_{n-1},\tau} \mathbf{E}_{X_{n}}\varphi^{''}(w)(l_n^{*})^2\\
{}+
    \fr{1}{2}\,\mathbf{E}\varphi^{'''}(w+l_n\tau )l_n^3(1-\tau)^{2}\\
    {}-
    \fr{1}{2}\,\mathbf{E}\varphi^{'''}(w+l_n^{*}\tau)(l_n^{*})^3(1-\tau)^{2}\Big|\,.
    \end{multline*}
    Using~(\ref{m}), one gets:
    
    \noindent
     \begin{multline}
     \label{e11-y}
    \Delta^{*}_{1,n}=\left\vert\fr{1}{2}\,\mathbf{E}\varphi^{'''}(w+l_n\tau)l_n^3(1-\tau)^{2}\right.\\
\left.    {}-
    \fr{1}{2}\,\mathbf{E}\varphi^{'''}(w+l_n^{*}\tau)(l_n^{*})^3(1-\tau)^{2}\right\vert\,.
    \end{multline}
    It is easy to see that
    
    \noindent
    \begin{multline*}
    |\mathbf{E}\varphi^{'''}(w+l_n\tau)l_n^3(1-\tau)^{2}|\leq 
    \fr{c}{\varepsilon^{3}}\mathbf{E} \,    \mathbf{I}(x\leq w+\tau l_n\\
    {}\leq x
    +\varepsilon)|l_{n}|^{3}\leq \fr{c}{\varepsilon^{3}}\,\mathbf{E} 
    \left\vert l_{n}\right\vert^{3}\,.
    \end{multline*}

    From theorem~20 in~[5, p.~89], one gets:
    
    \noindent
     \begin{multline}
     \label{11d}
     \mathbf{E}|l_{n}|^{3}=\left\vert\mathbf{E}\left(h(X_1,X_n)+
     \cdots+h(X_{n-2},X_n)\right.\right.\\
\left.\left.     {}+h(X_{n-1},X_n)\right)\right\vert^{3}
    \leq c(n-1)^{3/2}\beta_3\,.
    \end{multline}
    

   Combining~(\ref{c4}), (\ref{e8-y})--(\ref{11d}), one has
   
   \noindent
   \begin{multline*}
   \rho(Z(X_1,\ldots,X_n),Z(G_1,\ldots,G_n))\\
   {}\leq 
   \fr{c}{\sqrt{q_1}}\sqrt{\varepsilon}+\fr{c}{\varepsilon^{3}}(n-1)^{3/2}\beta_3\,.
   \end{multline*}
   
   \pagebreak


\noindent
   Let $\varepsilon= n^{-1/7}\beta_3^{1/3}$, then one gets:
  \begin{multline*}
   \rho(Z(X_1,\ldots,X_n),Z(G_1,\ldots,G_n))\\
   {}\leq
   c_2q_1^{-1/2}\beta_3^{1/6}n^{-1/14}\,.
   \end{multline*}
   
    \subsection{Proof of inequality~(\ref{b})}

\noindent
   Now, let estimate $\rho(Z(G_1,\ldots,G_n),Z_0).$

   It has been shown in~\cite{1-y} that it is possible to represent the statistic
   $Z(G_1,\ldots,G_n)$ in the form:
   $$
   Z(G_1,\ldots,G_n)=n^{-1} \sum\limits_{1\leq i<k\leq n}\langle Q G_i,G_k\rangle
$$
    where $a=(a_i)_{i\geq 1}$ is some constant. Let $G_{i,j}$,  $i \geq 1$ be the components
    of vector $G_{j}$. Then, one can rewrite this expression in the form:
    $$
    Z(G_1,\ldots,G_n)= n^{-1}\sum\limits_{i\geq 1}q_i
    \sum\limits_{1\leq j<k\leq n} G_{i,j}G_{i,k}
    $$
    where $G_{i,j}$ are the normally distributed random variables with the moments $(0,1)$.
        It is easy to see that
        $$
    2n^{-1}q_i\sum\limits_{1\leq j<k\leq n} G_{i,j}G_{i,k}=(n-1)q_i\overline{G}_i^{2}-
    q_iS_i^{2}
        $$
    where
    $$
    \overline{G}_i=n^{-1}\sum\limits_{1\leq j\leq n}G_{i,j} \
    \mbox{and} \ S_i^{2}=n^{-1}\sum\limits_{1\leq j\leq n}G_{i,j}^{2}-\overline{G}_i^{2}\,.
    $$
    It is well known~\cite{2-y} that random variables $\overline{G}_i$ and $S_i$
    are independent. The random variables $\overline{G}_i$ can be written in the form
    $\overline{G}_i =\eta_i/\sqrt{n}$; so, one can write:
    $$
    n^{-1}q_i\sum\limits_{1\leq j<k\leq n} G_{i,j}G_{i,k}=
    \fr{n-1}{n}\,q_i\eta_i^{2}-q_iS_i^{2}\,.
    $$
    Then
    \begin{multline*}
    Z(G_1,\ldots,G_n)= \sum\limits_{i\geq 1}\left(
    \fr{n-1}{2n}q_i\eta_i^{2}-\fr{q_i}{2}\,S_i^{2}\right)\\
{}=\sum\limits_{i\geq 1}\left(\fr{1}{2}\,q_i(\eta_i^{2}-1)-
    \fr{1}{2n}\,q_i\eta_i^{2}-\fr{q_i}{2}\left(S_i^{2}-1\right)\right)\,.
    \end{multline*}
    Assume
    \begin{multline*}
    \rho(Z(G_1,\ldots,G_n),Z_0)
=\sup\limits_{x}
    \left\vert\mathbf{P}\left(\sum\limits_{i\geq 1}\left(\fr{1}{2}q_i
    (\eta_i^{2}-1)\right.\right.\right.\\
%    \label{12-y}
\left.\left.    {}-
    \fr{1}{2n}\,q_i\eta_i^{2}-\fr{q_i}{2}\left(S_i^{2}-1\right) \right)\leq x
    \vphantom{\sum\limits_{i\leq1}}\right)\hspace*{3mm}
        \end{multline*}
    
\noindent
        \begin{multline}
\label{12-y}
\hspace*{5mm}\left.    {}-
    \mathbf{P}\left(\fr{1}{2}\sum\limits_{i\geq 1}q_i(\eta_i^{2}-1)\leq
    x\right)\right\vert \\
{}\leq\sup\limits_{x}\left\vert\mathbf{P}\left(\sum\limits_{i\geq 1}\left(\fr{1}{2}\,q_i
    (\eta_i^{2}-1)\right.\right.\right.\\
\left.\left.    {}-
    \fr{1}{2n}\,q_i\eta_i^{2}-\fr{q_i}{2}\left(S_i^{2}-1\right)
    \right)\leq x     
    \vphantom{\sum\limits_{i\leq1}}
    \right)\\
\left.    {}-
    \mathbf{P}\left(\sum\limits_{i\geq 1}\left(\fr{1}{2}\,q_i(\eta_i^{2}-1)-
    \fr{1}{2n}\,q_i\eta_i^{2}\right)\leq x\right)\right\vert
\\
  {}+\sup\limits_{x}\left\vert\mathbf{P}\left(\sum\limits_{i\geq 1}\left(\fr{1}{2}\,q_i
  \left(\eta_i^{2}-1\right)-
    \fr{1}{2n}\,q_i\eta_i^{2}\right)\leq x\right)\right.\\
\left.    {}-
    \mathbf{P}\left(\fr{1}{2}\sum\limits_{1\leq i}q_i(\eta_i^{2}-1)\leq
    x\right)\right\vert=  v_1+v_2\,.
    \end{multline}
    Let estimate $v_1.$ Using the independence of $S_i$,  $\eta_i$,  and $\eta_j$, $i \neq j$
    and inequality~(\ref{6l}), one obtains:
    $$
    v_1\leq c\left(|q_1|^{-1}\sum\limits_{i\geq 1}|q_i|\mathbf{E}|S_1^{2}-1|\right)^{{1}/{2}}\,.
    $$
    It follows that
    \begin{multline*}
\hspace*{-11.82176pt}\left(\mathbf{E}|S_1^{2}-1|\right)^{{1}/{2}}=\left(\mathbf{E}|
    n^{-1}\sum_{1\leq j\leq n}(G_{1,j}^{2}-1)-\overline{G}_1^{2}|\right)^{{1}/{2}}\\
    {}\leq
    c\left(\mathbf{E}|
    n^{-1}\sum\limits_{1\leq j\leq
    n}(G_{1,j}^{2}-1)|\right)^{{1}/{2}}+\left(\mathbf{E}|\overline{G}_1^{2}|\right)^{{1}/{2}}\,.
    \end{multline*}
    Let estimate the summands. One has:
    $$
    \left(\mathbf{E}|  \overline{G}_1^{2}|\right)^{{1}/{2}}=n^{-1/4}\left(\mathbf{E}|
    \eta_i^{2}|\right)^{{1}/{2}}=n^{-1/4}\,.
    $$
    Also, it is easy to see that
    \begin{multline*}
    \left(\mathbf{E}|n^{-1}\sum\limits_{1\leq j\leq
    n}(G_{1,j}^{2}-1)|\right)^{{1}/{2}}\\
    {}=
    n^{-1/4}\left(\mathbf{E}\left\vert\sum\limits_{1\leq j\leq n}
    \fr{(G_{1,j}^{2}-1)}{\sqrt{n}}\right\vert\right)^{{1}/{2}}\\
   {}\leq        n^{-1/4}\left(\mathbf{E}\left(\sum\limits_{1\leq j\leq
    n}\fr{(G_{i,j}^{2}-1)}{\sqrt{n}}\right)^{2}\right)^{{1}/{4}}\leq cn^{-1/4}\,.
    \end{multline*}
    Consequently,
    $$
    v_1\leq c\left(\sum\limits_{i\geq 1}\left\vert\fr{q_i}{q_1}\right\vert\right)^{1/2}n^{-1/4}\,.
    $$
    Let estimate the second summand in~(\ref{12-y}):
    \begin{multline*}
    v_2=\sup\limits_{x}\left\vert\mathbf{P}\left(\sum\limits_{i\geq 1}\left(q_i(\eta_i^{2}-1)-
    \fr{1}{n}\,q_i\eta_i^{2}\right)\leq x\right)\right.\\
\left.    {}-
    \mathbf{P}\left(q_i(\eta_i^{2}-1)\leq
    x\right)
    \vphantom{\fr{1}{n}\sum\limits_{i\geq 1}}\right\vert\,.
    \end{multline*}
    Denote
    $$
    \theta=-\fr{1}{n}\,q_1\eta_1^{2}\,; \
    T=\sum\limits_{i\geq 2}q_i(\eta_i^{2}-1)\,; \ 
    Y=-  \fr{1}{n}\sum\limits_{i\geq 2}q_i\eta_i^{2}\,.
    $$
    Let $F_{1}(u)=\mathbf{P}(q_1(\eta_1^{2}-1)+
    \theta\leq x)$, $F(u)=\mathbf{P}(q_1(\eta_1^{2}$\linebreak $-\;1)\leq x)$,
    then
    $$
    v_2=\sup\limits_{x}|\mathbf{E} \, F_{1}(x-T-Y)-\mathbf{E} \,
    F(x-T)|\,.
    $$
    Using triangle inequality, one finds:
\begin{multline}
\label{do}
v_2\leq \sup\limits_{x}\left\vert\mathbf{E} \, F_{1}(x-T-Y)-\mathbf{E} \,
    F_{1}(x-T)\right\vert\\
    {}+\sup\limits_{x}\left\vert\mathbf{E} \, F_{1}(x-T)-\mathbf{E} \,
    F(x-T)\right\vert\,.
    \end{multline}
    One has:
    \begin{multline*}
    \hspace*{-4pt}\sup\limits_{x}|\mathbf{E} \, F_{1}(x-T-Y)-\mathbf{E} \,
    F_{1}(x-T)|\leq c \left\vert q_1\right\vert^{-1/2}\sqrt{\mathbf{E} \, Y}\\
    {}\leq c\left\vert q_1\right\vert^{-1/2}
    (\mathbf{E} \, Y^{2})^{1/4}\,.
    \end{multline*}
    It is easy to see that
    $$
    \mathbf{E} \ (Y^{2})^{1/4}\leq  cn^{-1/2}\left(\sum_{i\geq 2}q_i^{2}\right)^{1/4}\,.
    $$
    For the second summand in~(\ref{do}), let apply Lemma~2:
    $$
    \sup\limits_{x}|\mathbf{E} \, F_{1}(x-T)-\mathbf{E} \,
    F(x-T)| \leq \fr{c}{\sqrt{q_1}n}\,.
    $$
    Finally, one has:
    $$
    v_2\leq \fr{c}{\sqrt{q_1}n}+
    cn^{-1/2}\left(\sum\limits_{i\geq 2}\left(\fr{q_i}{q_1}\right)^{2}\right)^{1/4}\,.
    $$
    Consequently,
    $$
    \rho(Z(G_1,\ldots,G_n),Z_0)\leq \fr{c}{n^{1/4}}
    \left(\sum\limits_{i\geq 1}\left\vert\fr{q_i}{q_1}\right\vert\right)^{1/2}+\fr{c}{\sqrt{q_1}n}\,.
    $$
     The theorem is proved.

  {\small\frenchspacing
{%\baselineskip=10.8pt
\addcontentsline{toc}{section}{Литература}
\begin{thebibliography}{9}
  
    \bibitem{1-y}
    \Au{Bentkus~V., G$\ddot{\mbox{o}}$tze F.}
    Optimal bounds in non-Gaussian limit
      theorems for U-statistics~// Ann. Prob., 1999.
      Vol.~27. No.\,1. P.~454--521.


\bibitem{3-y} %2
\Au{Korolyuk V.\,S., Borovskikh~Yu.\,V.}
Rate of convergence of degenerate von Mises functionals~// Theory Prob. Appl., 
1988. Vol.~33. No.\,1. P.~125--135.

%\bibitem{4-y}
%\Au{Nagaev S.\,V., Chebotarev V.\,I.}
%Estimates of the rate of
%convergence in the central limit theorem in space $l_{2}$~//
%Math. Anal. Smezhn. voprpsy mat.~--- Novosibirsk: Nauka, 1978.
%P.~153--182. [In Russian.]



\bibitem{6-y} %3
\Au{Senatov V.\,V.}
Qualitative effects in the estimates of the
convergence rate in the central limit theorem in
multidimensional spaces~// Proceedings of the Steklov Inst. of
      Math., 1996. Vol.~215. No.\,4. 

%\bibitem{7-y}
%\Au{Yanushkevichiene O.}
%On the rate of convergence of second-degree
%random polynomials~// J. Math. Sci., 1998. Vol.~92. No.\,3.
%P.~3955--3959.

\bibitem{9-y} %4
\Au{Yanushkevichiene O.}
Asymptotic rate of convergence in the degenerate U-statistics of second order~//
Banach Center Publs., 2010. Vol.~90.  P.~275--284.

%\bibitem{5-y}
%\Au{Petrov V.\,V.}
%Limit theorems for the summs of random variables.~--- Moscow: Nauka, 1987.
%[In Russian.]

\bibitem{8-y} %5
\Au{Yanushkevichiene O.}
Optimal rates of convergence of second degree
polynomials in several metrics~// J.~Math. Sci., 2006.
Vol.~138, No.\,1. P.~5472--5479.

\bibitem{2-y} %6
\Au{Cramer H.} 
Mathematical methods of statisics.~--- Stockholm, 1946.


\end{thebibliography}
}
}


\end{multicols}

\vspace*{6pt}

\hrule

\vspace*{6pt}


\def\tit{О СКОРОСТИ СХОДИМОСТИ НЕКОТОРОЙ U-СТАТИСТИКИ}

\def\aut{О.\,Л.~Янушкявичене$^1$, Р.~Янушкявичюс$^2$}

\titelr{\tit}{\aut}

\vspace*{6pt}

\noindent
$^1$Институт математики и информатики Вильнюсского университета, Литва, olgjan@zebra.lt\\
\noindent
$^2$Литовский  университет эдукологии, Вильнюс, Литва 


\Abst{Анализируется асимптотическое поведение U-ста\-ти\-стик, которые важ\-ны для
конструирования статистических методов, применяемых для изучения информационных
систем.}


\label{end\stat}

\KW{скорость сходимости; неравенство Берри--Эссеена; квадратичная форма, U-статистика второго
порядка; мет\-ри\-ка Колмогорова}


\renewcommand{\figurename}{\protect\bf Рис.}
\renewcommand{\tablename}{\protect\bf Таблица}
\renewcommand{\bibname}{\protect\rmfamily Литература}    %7
\def\stat{bening}


\def\tit{АСИМПТОТИЧЕСКОЕ
РАЗЛОЖЕНИЕ ДЛЯ МОЩНОСТИ КРИТЕРИЯ, ОСНОВАННОГО НА ВЫБОРОЧНОЙ
МЕДИАНЕ, В~СЛУЧАЕ РАСПРЕДЕЛЕНИЯ ЛАПЛАСА$^*$}
\def\titkol{Асимптотическое
разложение для мощности критерия, основанного на выборочной
медиане} %, в случае распределения Лапласа}

\def\autkol{В.\,Е.~Бенинг, А.\,В.~Сипина}
\def\aut{В.\,Е.~Бенинг$^1$, А.\,В.~Сипина$^2$}

\titel{\tit}{\aut}{\autkol}{\titkol}

{\renewcommand{\thefootnote}{\fnsymbol{footnote}}\footnotetext[1]
{Работа выполнена
при финансовой поддержке РФФИ, проекты 08-01-00567 и
08-07-00152.}}

\renewcommand{\thefootnote}{\arabic{footnote}}
\footnotetext[1]{Московский государственный университет им.\
М.\,В.~Ломоносова, факультет вычислительной математики и
кибернетики; Институт проблем информатики Российской академии наук, bening@yandex.ru}
\footnotetext[2]{Московский государственный университет им.\
М.\,В.~Ломоносова, факультет вычислительной математики и
кибернетики, anna@sipin.ru}



\Abst{В работе прямыми методами, использующими асимптотические разложения,
получена формула для предельного отклонения мощности критерия, 
основанного на выборочной медиане, от мощности наилучшего критерия в случае распределения Лапласа.}

\KW{выборочная медиана; асимптотичсекое разложение; функция мощности; распределение Лапласа}

      \vskip 18pt plus 9pt minus 6pt

      \thispagestyle{headings}

      \begin{multicols}{2}

      \label{st\stat}


\section{Введение}

Следуя работе~\cite{3ben}, рассмотрим задачу проверки гипотезы
\begin{equation*}
{\sf H}_0: \theta = 0     
%\label{e1.1b}
\end{equation*}
против последовательности сложных близких альтернатив вида
\begin{equation*}
{\sf H}_{n,1}: \theta = \fr{t}{\sqrt{n}}\,,\quad 0<t<C\,,\quad
 C > 0
% \label{e1.2b}
\end{equation*}
на основе выборки $(X_1, \ldots , X_n)$~--- независимых одинаково распределенных наблюдений, имеющих распределение Лапласа 
с плотностью
\begin{equation}
p(x, \theta) = \fr{1}{2}e^{-|x-\theta|}\,, \quad x,\:
\theta \in{\sf R}^1\,. 
\label{e1.3b}
\end{equation}
Распределение Лапласа широко применяется в прикладной статистике, например
в задачах вы\-де\-ле\-ния полезного сигнала на фоне помех~[2--4].
Естественность возникновения этого распределения обоснована в
работе~\cite{6ben}.

Для каждого фиксированного $t\in (0,C]$
обозначим через~$\beta_n^*(t)$ мощность наилучшего критерия размера
$\alpha\in (0,1)$. По лемме Неймана--Пирсона %\linebreak 
[6, с.~94]
такой критерий всегда существует и  основан на логарифме отношения правдоподобия
\begin{equation}
\Lambda_n(t) = 
\sum_{i=1}^{n}\left( \left|X_i\right|-\left|X_i-tn^{-1/2}\right|\right)\,.
 \label{e1.4b}
\end{equation}
В работах~\cite{3ben, 2ben} рассмотрен критерий, основанный на знаковой статистике,
и получена формула для предельного отклонения мощности данного
критерия от мощности наилучшего критерия, основанного на~$\Lambda_n(t)$.
Поскольку у плотности~$p(x,\theta)$ не существует производной по~$\theta$ в 
точке $\theta = 0$, то это семейство не является регулярным.
Это выражается в нарушении естественного порядка~$n^{-1}$ разности мощностей
этих критериев и приводит к порядку~$n^{-1/2}$.

В  работе рассматривается статистика
\begin{equation*}
T_n = \sqrt{2k}\, \zeta_n\,,\quad k=\left[\fr{n}{2}\right]\,, 
%\label{e1.5b}
\end{equation*}
где $\zeta_n$~--- выборочная медиана:
\begin{equation*}
\zeta_n= 
\begin{cases}
X_{(k+1)}\,, & n=2k+1\,; \\
\fr{X_{(k)}+X_{(k+1)}}{2}\,, &  n=2k\,.
\end{cases}
%\label{e1.6b}
\end{equation*}
Заметим, что в случае распределения Лапласа выборочная медиана
совпадает с оценкой максимального правдоподобия (см.~\cite{1ben}).

Обозначим через~$\beta_n(t)$ мощность критерия размера $\alpha\in (0,1)$,
основанного на статистике~$T_n$. В работе получено асимптотическое
разложение для~$\beta_n(t)$ и вычислен предел разности мощностей~$\beta_n^*(t)$ и~$\beta_n(t)$
$$
r(t)\equiv\lim_{n\to\infty}\sqrt n\left(\beta_n^*(t)-\beta_n(t)\right)
$$
критериев (см.~(\ref{e2.14b})),
основанных соответственно на статистиках~$\Lambda_n$ и~$T_n$.

В работе также приведено полное доказательство  (см.~\cite{5ben})
представления выборочной медианы в виде случайной суммы
независимых экспоненциально распределенных  случайных величин.


\section{Асимптотическое разложение для мощности критерия,
основанного на выборочной медиане}

В этом разделе будет построено  асимптотическое разложение  для мощности~$\beta_n(t)$.
Основой для его получения служит  работа~\cite{1ben} (см.\ теорему~2.1),
в которой получено разложение для функции распределения выборочной медианы.
Члены порядка~$n^{-1/2}$ в разложении для функции распределения выборочной медианы
без доказательства приведены  также в работе~\cite{9ben}.

\medskip
\noindent
\textbf{Теорема 1.} {\it Для мощности~$\beta_n(t)$ равномерно по
$t\in(0,C]$, $C>0$,
справедливо следующее асимптотическое разложение:
\begin{equation*}
\beta_n(t)=
\begin{cases}
\Phi(t-u_\alpha)-\fr{t(2u_\alpha-t)}{2\sqrt{n}}\,\varphi(u_\alpha-t)+{} \\
\hspace*{8mm}{}+o\left(n^{-1/2}\right)\,,  \quad t \le u_\alpha\,,\enskip  \alpha <\fr{1}{2}\,;\\
\Phi(t-u_\alpha)-\fr{2u_\alpha^2+t^2-2u_\alpha t}{2\sqrt{n}}\,\varphi(u_\alpha -t)+{}\\
\hspace*{8mm}{}+o\left(n^{-1/2}\right)\,, \quad t>u_\alpha\,, \enskip \alpha <\fr{1}{2}\,;\\
\Phi(t-u_\alpha)+\fr{t(2u_\alpha-t)}{2\sqrt{n}}\,\varphi(u_\alpha -t)+{}\\
\hspace*{22mm}{}+{} o\left(n^{-1/2}\right)\,, \quad 
\alpha \ge \fr{1}{2}\,,
\end{cases}\hspace*{-6pt}
%\label{e2.1b}
\end{equation*}
где  $\Phi(x)$  и~ $\varphi(x)$~---  функция распределения и
плотность стандартного нормального закона и $\Phi(u_\alpha)=1-\alpha$.}

\medskip

\noindent
Д\,о\,к\,а\,з\,а\,т\,е\,л\,ь\,с\,т\,в\,о\,.\
Для доказательства теоремы воспользуемся асимптотическим разложением
для функции распределения выборочной медианы в случае
распределения Лапласа из работы~\cite{1ben} (см.\ формулу~(1.3)):
\begin{multline}
\p_{n,\theta} \left( \sqrt{2k}(\zeta_n - \theta) < x \right) = 
\Phi(x)-\fr{x|x|}{2\sqrt{2k}}\,\varphi(x)+{}\\
{}+
\fr{x(18+10x^2-3x^4)}{48k}\,\varphi(x)+ o(n^{-1})\,.
\label{e2.2b}
\end{multline}
Подберем критическое значение~$d_n$, исходя из условия
\begin{equation*}
\p_{n,0}(T_n>d_n)=\alpha+ o(n^{-1})\,.
%\label{e2.3b}
\end{equation*}
Будем искать $d_n$ в виде
\begin{equation*}
d_n = u_\alpha +\fr{a}{\sqrt{2k}}+\fr{b}{2k}\,.
%\label{e2.4b}
\end{equation*}
Из формулы~(\ref{e2.2b}) следует, что

\noindent
\begin{multline}
\p_{n, 0} \left( T_n> d_n \right) = 1 -
\Phi(d_n)+\fr{d_n|d_n|}{2\sqrt{2k}}\varphi(d_n)-{}\\
{}-
\fr{d_n(18+10d_n^2-3d_n^4)}{48k}\,\varphi(d_n)+ o(n^{-1})\,.
\label{e2.5b}
\end{multline}
Чтобы раскрыть модуль в выражении~(\ref{e2.5b}),  рас\-смот\-рим два случая:
$\alpha<1/2$ и $\alpha \ge 1/2$.

Рассмотрим случай $\alpha < 1/2$. Это означает, что при достаточно
больших $n$ справедливо неравенство $d_n > 0$.
Подставляя выражение для~$d_n$ в формулу~(\ref{e2.5b}) и применяя следующие разложения:
\begin{multline*}
\Phi(d_n)=\Phi\left(u_\alpha+\fr{a}{\sqrt{2k}}+\fr{b}{2k}\right)=
\Phi(u_\alpha)+{}\\
{}+
\left(\fr{a}{\sqrt{2k}}+\fr{b}{2k}\right)\varphi(u_\alpha)-
\fr{u_\alpha a^2}{4k}\varphi(u_\alpha)+ o(n^{-1})\,;
\end{multline*}
\vspace*{-12pt}

\noindent
\begin{multline*}
\varphi(d_n)=\varphi\left(u_\alpha+\fr{a}{\sqrt{2k}}
+\fr{b}{2k}\right)= {}\\
{}=
\varphi(u_\alpha)-\left(\fr{a}{\sqrt{2k}}+\fr{b}{2k}\right)u_\alpha
\varphi(u_\alpha)+ o(n^{-1})\,,
\end{multline*}
получаем
\begin{multline*}
1-\Phi(u_\alpha)-\left(\fr{a}{\sqrt{2k}}+
\fr{b}{2k}\right)\varphi(u_\alpha)+\fr{u_\alpha a^2}{4k}\,\phi(u_\alpha)
+{}\\
{}+\fr{(u_\alpha+(a/\sqrt{2k})+b/(2k))^2}
{2\sqrt{2k}}\times{}\\
{}\times \left(\varphi(u_\alpha) - \fr{a}{\sqrt{2k}}\,u_\alpha
\varphi(u_\alpha)\right)-{}\\
{}-
\fr{u_\alpha(18+10u_\alpha^2-3u_\alpha^4)}{48k}\,\varphi(u_\alpha)=
\alpha + o(n^{-1})\,.
\end{multline*}
Приравнивая коэффициенты при~$1/\sqrt{2k}$ и~$1/(2k)$ к нулю,
находим выражения для~$a$ и~$b$:
\begin{gather*}
a=\fr{u_\alpha^2}{2}\,;
\\
b=-\fr{3}{4}\,u_\alpha+\fr{1}{12}\,u_\alpha^3\,;
\\
d_n = u_\alpha+\fr{u_\alpha^2}{2\sqrt{2k}}-\fr{3}{8k}\,
u_\alpha+\fr{1}{24k}\,u_\alpha^3\,.
\end{gather*}
Теперь для получения асимптотического разложения мощности критерия используем
разложение
\begin{multline*}
\p_{n,tn^{-1/2}}(T_n<x)= \Phi\left(x-t\sqrt{2k}n^{-1/2}\right) -{}\\
{}-
\fr{\left(x-t\sqrt{2k}n^{-1/2}\right)\left| x\:-\:t\sqrt{2k}\,n^{-1/2}\right|}{2\sqrt{2k}}\,
{}\times{}\\
{}\times\varphi(x-t\sqrt{2k}\,n^{-1/2})+ {}
\end{multline*}
\begin{multline*}
{}+
\fr{ x-t\sqrt{2k}\,n^{-1/2}}{48k}
\left(18+10(x-
t\sqrt{2k}\,n^{-1/2})^2-{}\right.\\
\left.{}-3(x-t\sqrt{2k}\,n^{-1/2})^4\right)\times{}
\\
{}\times\varphi\left(x-t\sqrt{2k}\,n^{-1/2}\right)+ o\left(n^{-1}\right)\,,
%\label{e2.6b}
\end{multline*}
которое  получается при подстановке $\theta=tn^{-1/2}$ в
формулу~(\ref{e2.2b}).

Имеем
\begin{multline*}
\beta_n(t)=\p_{n,tn^{-1/2}}\left(T_n>d_n\right) ={}\\
{}=
1-\Phi\left(d_n-t\right) +
\fr{\left(d_n-t\right)\left|d_n-t\right|}{2\sqrt{2k}}\,\varphi\left(d_n-t\right)-{}
\\\!
{}-\fr{d_n-t}{48k}\left(18+10\left(d_n-t\right)^2
-3(d_n-t)^4\right)\, \varphi\left(d_n-t\right)+{}\\
{}+ o\left(n^{-1}\right)\,.
%\label{e2.7b}
\end{multline*}
Аналогично предыдущему, рассмотрим  два случая: $t\le u_\alpha$ и
$t>u_\alpha$.

Пусть сначала $t \le u_\alpha$.
Используя разложения
\begin{multline*}
\Phi\left(d_n-t\right)={}\\
{}=\Phi\left(u_\alpha-t+
\fr{u_\alpha^2}{2\sqrt{2k}}-\fr{3}{8k}\,u_\alpha+
\fr{1}{24k}\,u_\alpha^3\right)={}\\
{}=\Phi\left(u_\alpha-t\right)+
\left(\fr{u_\alpha^2}{2\sqrt{2k}}-\fr{3}{8k}\,u_\alpha+
\fr{1}{24k}\,u_\alpha^3\right)\times{}\\
{}\times\varphi\left(u_\alpha-t\sqrt{2k}\,n^{-1/2}\right)-{}
\\
{}-
\fr{\left(u_\alpha-t\sqrt{2k}\,n^{-1/2}\right)\varphi\left(u_\alpha-
t\sqrt{2k}\,n^{-1/2}\right)u_\alpha^4}{16k}+{}\\
{}+ o\left(n^{-1}\right)\,; 
%\label{e2.8b}
\end{multline*}

\vspace*{-12pt}

\noindent
\begin{multline*}
\varphi\left(d_n-t\right)={}\\
{}= \varphi\left(u_\alpha-t+
\fr{u_\alpha^2}{2\sqrt{2k}}-\fr{3}{8k}\,u_\alpha+
\fr{1}{24k}\,u_\alpha^3\right)={}\\
{}=
\varphi\left(u_\alpha-t\right)-\left(u_\alpha-t\right)
\varphi\left(u_\alpha-t\right)\fr{u_\alpha^2}{2\sqrt{2k}}+{}\\
{}+
o\left(n^{-1/2}\right)\,,
%\label{e2.9}
\end{multline*}
получаем, что
\begin{multline*}
\beta_n(t)=1-\Phi\left(u_\alpha-t\right)-
\fr{u_\alpha^2}{2\sqrt{2k}}\,\varphi\left(u_\alpha-t\right)+{}\\
{}+\fr{u_\alpha^2}{2\sqrt{2k}}\,\varphi(u_\alpha-t)-
\fr{2u_\alpha t - t^2}{2\sqrt{2k}}\,\varphi(u_\alpha-t)+{}\\
{}+
o\left(n^{-1/2}\right)=
\Phi\left(t-u_\alpha\right)-\fr{t\left(2u_\alpha - t\right)}{2\sqrt{2k}}\,
\varphi\left(u_\alpha - t\right)+{}\\
{}+ o\left(n^{-1/2}\right)\,.
%\label{e2.10b}
\end{multline*}
Во втором случае при $t > u_\alpha$  выражение
для мощности приобретает вид:

\noindent
\begin{multline*}
\beta_n(t)=\Phi\left(t-u_\alpha\right)-{}\\
{}-
\fr{t\left(2u_\alpha^2+t^2 -2u_\alpha t\right)}{2\sqrt{n}}\,
\varphi\left(u_\alpha-t\right)+ o\left(n^{-1/2}\right)\,.
%\label{e2.11b}
\end{multline*}
При $\alpha \ge 1/2$  аналогичным образом имеем
\begin{multline*}
\beta_n(t)={}\\
{}=
 \Phi\left(t-u_\alpha\right)+
\fr{t\left(2u_\alpha - t\right)}{2\sqrt{n}}\,\varphi\left(u_\alpha - t\right)+
o\left(n^{-1/2}\right)\,.
%\label{e2.12b}
\end{multline*}
Из этих формул следует утверждение теоремы.~$\Box$

\medskip

В работе~\cite{2ben} было показано, что для мощ\-ности~$\beta_n^*(t)$ 
критерия размера $\alpha\in (0,1)$, осно\-ван\-но\-го на
логарифме отношения прав\-до\-подобия~$\Lambda_n(t)$~(\ref{e1.4b}),
справедливо  асимптотическое\linebreak разложение
\begin{equation*}
\beta_n^*(t)=\Phi(t-u_\alpha) - \fr{t^2}{6\sqrt{n}}\,
\varphi(t-u_\alpha)+ o(n^{-1/2})\,.
%\label{e2.13b}
\end{equation*}
Используя это разложение и теорему~1, получаем формулу
для предельного отклонения нормированной разности мощностей
рассматриваемых критериев:
\begin{multline}
r(t)= \lim_{n \to \infty}\sqrt{n}(\beta_n^*(t)-\beta_n(t))
={}\\
{}=
\begin{cases}
\left(t u_\alpha-\fr{2t^2}{3}\right)
\varphi(u_\alpha-t)\,,\\
\hspace*{30mm} t \le u_\alpha\,,\enskip \alpha < \fr{1}{2}\,; \\
\left(u_\alpha^2+\fr{t^2}{3}-u_\alpha t \right)
\varphi(u_\alpha - t)\,,\\
\hspace*{30mm}  t>u_\alpha\,,\enskip \alpha<\fr{1}{2}\,; \\
\left(\fr{t^2}{3}-t u_\alpha\right)\varphi(u_\alpha-t)\,, \quad\quad\ \  \alpha \ge \fr{1}{2}\,. 
\end{cases}
\label{e2.14b}
\end{multline}

\section{Представление выборочной медианы в~виде случайной суммы}

В этом разделе докажем лемму о представлении выборочной медианы
в случае распределения Лапласа в виде суммы случайного числа
независимых экспоненциально распределенных случайных величин.
Формулы для представления порядковых статистик в случае распределения
Лапласа в виде подобной суммы приведены в работе~[4, с.~63],
но без строгого доказательства.

\bigskip

\noindent
\textbf{Лемма 1.}
{\it В случае распределения Лапласа выборочную медиану
можно представить в следующем виде (здесь равенства по распределению):
\begin{align}
\zeta_{2k+1} &\stackrel{d}{=}\delta_{2k+1}
\sum\limits_{j=k+1}^{K_{2k+1}}{\fr{W_j}{j}}\,;
\label{e3.1b}\\
\zeta_{2k}&\stackrel{d}{=}\fr{W_1-W_2}{2k}\,\mathbf{1}(B_{2k+1}=k)+{}\notag\\[1pt]
&\!\!\!\!\!\!\!\!\!\!\!\!\!\!{}+
\left(\delta_{2k}\sum\limits_{j=k+1}^{K_{2k+1}}\fr{W_j}{j}+
\delta_{2k}\fr{W_k}{2k}\right)\mathbf{1}\left(B_{2k+1} \ne k\right)\,,
\label{e3.2b}
\end{align}
где
$$
\delta_n=\mathrm{sign}\left(B_n-k-\fr{1}{2}\right)\,,
$$
$W_j$~--- независимые экспоненциально (с параметром~1) распределенные
случайные величины; $B_n$~--- бернуллиевские случайные величины с параметрами
$p=1/2$ и~$n$, независимые от~$W_j$;
\begin{equation*}
K_n = \max\left(B_n, \bar{B_n}\right)\,,\quad
\bar{B_n}= n - B_n\,.
\end{equation*}
}

\smallskip

\noindent
Д\,о\,к\,а\,з\,а\,т\,е\,л\,ь\,с\,т\,в\,о\,.

Вначале докажем две вспомогательные формулы, справедливые для любого
действительного чис\-ла~$s$
и любых натуральных чисел~$a$ и~ $b$:
\begin{gather}
\prod\limits_{j=a}^{a+b}{\fr{1}{j+is}}=
\sum\limits_{j=0}^b \fr{(-1)^j}{(a+j+is)(b-j)!j!}\,;
\label{e3.3b}
\\[3pt]
\!\!\!\!\!\!\!\!\sum\limits_{l=0}^k\fr{k!}{l!} \prod\limits_{j=a}^{a+k-l}\fr{1}{j+is}=
\sum\limits_{l=0}^k \begin{pmatrix}
k\\ l\end{pmatrix}
\fr{(-1)^l 2^{k-l}}{a+l+is}\,.
\label{e3.4b}
\end{gather}
Формулу~(\ref{e3.3b}) докажем методом математической индукции.

При $b=1$ формула верна. Предполагая ее верной при $b\ge1$,
докажем что она  верна и  при~$b+1$:
\begin{multline*}
\prod\limits_{j=a}^{a+b+1}\fr{1}{j+is}=\fr{1}{a+b+1+is}\prod\limits_{j=a}^{a+b}
\fr{1}{j+is}={}\\[2pt]
{}=
\fr{1}{a+b+1+is}\left(\sum\limits_{l=0}^k 
\begin{pmatrix}
k\\ l
\end{pmatrix}
\fr{(-1)^l 2^{k-l}}
{a+l+is}\right)={}\\[2pt]
{}=
\sum\limits_{j=0}^{b}\fr{(-1)^j}{(b-j)!j!} \left(\fr{1}{(b+1-j)(a+j+is)}
- {}\right.\\[2pt]
\left.{}-\fr{1}{(b+1-j)(a+b+1+is)} \right)={}
\end{multline*}
\begin{multline*}
{}=
\sum\limits_{j=0}^{b}\fr{(-1)^j}{(a+j+is)(b+1-j)!j!}-{}\\
{}-
\fr{1}{a+b+1+is}\sum\limits_{j=0}^{b}\fr{(-1)^j}{(b-j+1)!j!}\,.
\end{multline*}
Заметим, что
\begin{multline*}
\!\!\sum\limits_{j=0}^b\fr{(-1)^j}{(b-j+1)!j!}=
\sum\limits_{j=0}^{b+1}\fr{(-1)^j}{(b-j+1)!j!}
-\fr{(-1)^{b+1}}{(b+1)!}={}\\
{}=
\fr{1}{(b+1)!}(1-1)^{b+1}-\fr{(-1)^{b+1}}{(b+1)!}=
-\fr{(-1)^{b+1}}{(b+1)!}\,.
\end{multline*}
И следовательно, формула~(\ref{e3.3b}) доказана.
Формула~(\ref{e3.4b}) следует  из доказанной формулы~(\ref{e3.3b}), по\-скольку
\begin{multline*}
\sum_{l=0}^k{\fr{k!}{l!}}\prod\limits_{j=a}^{a+k-l}\fr{1}{j+is}={}\\
{}=
\sum\limits_{l=0}^{k}\fr{k!}{l!}\sum\limits_{j=0}^{k-l}
\fr{(-1)^j}{(a+j+is)(k-l-j)! j!}={}\\
{}
=\sum\limits_{j=0}^{k}\fr{(-1)^j}{a+j+is}\sum\limits_{l=0}^{k-j}
\begin{pmatrix}
k\\ j
\end{pmatrix}
\begin{pmatrix}
k-j\\  l
\end{pmatrix}={}\\
{}=
\sum\limits_{j=0}^k\fr{(-1)^j}{a+j+is}
\begin{pmatrix}
k\\ j
\end{pmatrix}
2^{k-j}\,.
\end{multline*}
Теперь приступим к доказательству основного утверждения леммы.
Рассмотрим сначала случай $n=2k+1$.
Плотность $(k+1)$-й порядковой статистики, как известно,
выражается формулой (см.~\cite{4ben})
\begin{equation*}
p_{2k+1}(x) = (2k+1)
\begin{pmatrix}
2k\\  k\end{pmatrix}
f(x)(F(x)(1-F(x))^k\,,
%\label{3.5b}
\end{equation*}
где $f(x)$ и  $F(x)$~--- соответственно плотность и
функция распределения исходных случайных величин.

Найдем характеристическую функцию~$\phi_{2k+1}(s)$ выборочной
медианы~$\zeta_{2k+1}$:
\begin{multline*}
\phi_{2k+1}(s)=\e e^{is\zeta_{2k+1}}=
\int\limits_{-\infty}^{\infty}e^{isx}f(x)\,dx={}\\
{}=
(2k+1)
\begin{pmatrix}
2k\\  k\end{pmatrix}
2^{-(k+1)}\times{}\\
{}\times
\sum\limits_{j=0}^k (-1)^j 2^{-j}
\begin{pmatrix}
k\\ e j\end{pmatrix}
\fr{2(k+1+j)}{(k+1+j)^2+s^2}\,.
%\label{e3.6b}
\end{multline*}
Теперь найдем характеристическую функцию~$f_{2k+1}(s)$ случайной величины, определенной\linebreak\vspace*{-12pt}\pagebreak

\noindent
в правой части  формулы~(\ref{e3.1b}).
С учетом того, что
 характеристическая функция стандартной экспоненциальной
случайной величины равна $1/(1-is)$, имеем
\begin{multline*}
f_{2k+1}(s)={}\\
{}=
\sum\limits_{l=0}^{2k+1}\e \exp \left(is\delta_{2k+1}
\sum\limits_{j=k+1}^{K_{2k+1}}\fr{W_j}{j}\right)\mathbf{1}(B_{2k+1}=l)={}
\\
=2^{-(2k+1)}\left(\sum\limits_{l=0}^k \begin{pmatrix}
2k+1\\  l\end{pmatrix}
\prod\limits_{j=k+1}^{2k+1-l}\fr{j}{j+is}+{}\right.\\
\left.{}+
\sum\limits_{l=k+1}^{2k+1}
\begin{pmatrix}
2k+1\\ l\end{pmatrix}
\prod\limits_{j=k+1}^{l}\fr{j}{j-is}
\right)={}\\
{}
=2^{-(2k+1)}(2k+1)
\begin{pmatrix}
2k\\ k\end{pmatrix}
\sum\limits_{l=0}^k\fr{k!}{l!}
\left(\prod\limits_{j=k+1}^{2k+1-l}\fr{1}{j+is} +{}\right.\\
\left.{}+
\prod\limits_{j=k+1}^{2k+1-l}\fr{1}{j-is}\right)\,.
%\label{e3.7b}
\end{multline*}
Применяя формулу~(\ref{e3.4b}), получаем
\begin{multline*}
f_{2k+1}(s)=(2k+1)
\begin{pmatrix}
2k\\  k
\end{pmatrix}
2^{-(k+1)}\times{}\\
{}\times
\sum\limits_{j=0}^k(-1)^j 2^{-j}
\begin{pmatrix}
k\\  j\end{pmatrix}
\fr{2(k+1+j)}{(k+1+j)^2+s^2}\,.
%\label{e3.8b}
\end{multline*}
Значит, $f_{2k+1}(s)\equiv\phi_{2k+1}(s)$ и представление~(\ref{e3.1b}) доказано.
\medskip

Перейдем теперь к рассмотрению случая четного $n=2k$.
Совместная плотность двух порядковых статистик~$X_{(k)}$ и~$X_{(k+1)}$
определяется формулой (см.~\cite{4ben})
\begin{equation*}
p(x,y)=\fr{(2k)!}{((k-1)!)^2}\,(F(x)(1-F(y)))^{k-1}f(x)f(y)\,.
%\label{e3.9b}
\end{equation*}
Из этой формулы нетрудно получить, что плотность случайной величины
$$
\zeta_{2k}=\fr{X_{(k)}+X_{(k+1)}}{2}
$$
равна
\begin{multline*}
p_{2k}(x) = \fr{(2k)!}{2^k ((k-1)!)^2}\times{}\\
{}\times
\left(\sum_{j=0}^{k-2}\fr{(-1)^j
\begin{pmatrix}
k-1\\ j
\end{pmatrix}
2^{-j}}{k-1-j}
e^{-(k+1+j)|x|}\times{}\right.
\end{multline*}
\begin{multline}
\left.{}\times \left(1-e^{-(k-1-j)|x|}\right)- \right.{}\\
{}\left.
- \fr{(-1)^k}{2^{k-1}}|x|e^{-2k|x|} + \fr{1}{k2^k}e^{-2k|x|}
\vphantom{\fr{(-1)^j
\begin{pmatrix}
k-1\\ j
\end{pmatrix}
2^{-j}}{k-1-j}}\right)\,.
\label{e3.10b}
\end{multline}
Подробный вывод этой формулы приведен в работе~\cite{8ben}.
Исходя их формулы~(\ref{e3.10b}), найдем характеристическую функцию~$\phi_{2k}(s)$
выборочной медианы~$\zeta_{2k}$:
\begin{multline*}
\!\phi_{2k}(s)=
\fr{(2k)!}{2^k ((k-1)!)^2}
\left( \sum\limits_{j=0}^{k-2}
\fr{(-1)^j
\begin{pmatrix}
k-1\\ j
\end{pmatrix}
2^{-j}}{k-1-j}\times{}\right.
\\
\left.{}\times
\left(
\fr{2(k+1+j)}{(k+1+j)^2+s^2}  -
 \fr{4k}{4k^2+s^2} \right)-{}\right.\\
\left. {}- 
 \fr{(-1)^k}{2^{k-2}(4k^2+s^2)} + \fr{1}{2^{k-2}(4k^2+s^2)}
 \vphantom{\sum\limits_{j=0}^{k-2}
\fr{(-1)^j
\begin{pmatrix}
k-1\\ j
\end{pmatrix}
2^{-j}}{k-1-j}}
\right)\,.
%\label{e3.11b}
\end{multline*}
Найдем теперь характеристическую функ-\linebreak цию~$f_{2k}(s)$ случайной величины,
определенной
 в правой части формулы~(\ref{e3.2b}). Учитывая формулу~(\ref{e3.4b}), получим
\begin{multline*}
f_{2k}(s)=\sum\limits_{l=0}^{k-1}{\p(B_{2k}=l)
\fr{2k}{2k+is}\prod\limits_{j=k+1}^{2k-l}{\fr{j}{j+is}}}+{}\\
{}+
\sum\limits_{j=k+1}^{2k}{\p(B_{2k}=l)\fr{2k}{2k-is}\prod\limits_{j=k+1}^{2k-l}\fr{j}{j-is}}+{}\\
{}+
\p(B_{2k}=k)\fr{4k^2}{4k^2+s^2}={}\\
{}=
\fr{(2k)!}{2^k ((k-1)!)^2} \left( \fr{1}{2^{k-2}(4k^2+s^2)}
+{}\right.\\
\left.{}+2^{1-k}\sum\limits_{l=0}^{k-1}(-1)^l 2^{k-l-1}
\begin{pmatrix}
k-1\\ l\end{pmatrix}\times\right.{}\\
{}\times
\left( \fr{1}{(2k+is)(k+1+l-is)}+{}\right.\\
\left.\left.{}+ 
\fr{1}{(2k-is)(k+1+l-is)}\right) \right)\,.
\end{multline*}
Применяя при $l \ne k-1$ следующее соотношение:
\begin{multline*}
\fr{1}{(2k+is)(k+1+l+is)}={}\\
{}=
\fr{1}{k-1-l}\left( \fr{1}{k+1+l+is} - \fr{1}{2k+is}\right)\,,
\end{multline*}
получаем равенство

\noindent
\begin{multline*}
f_{2k}(s)=
\fr{(2k)!}{2^k ((k-1)!)^2}
\left( \sum\limits_{j=0}^{k-2}
\fr{(-1)^j 
\begin{pmatrix}
k-1\\ j
\end{pmatrix}
2^{-j}}{k-1-j}\times{}\right.\\
\left.{}\times
\left(
\fr{2\left(k+1+j\right)}{(k+1+j)^2+s^2} 
-  \fr{4k}{4k^2+s^2} \right)
-{}\right. \\
\left.{}- \fr{\left(-1\right)^k}{2^{k-2}\left(4k^2+s^2\right)} + \fr{1}{2^{k-2}(4k^2+s^2)}
\vphantom{\sum_{j=0}^{k-2}
\fr{(-1)^j 
\begin{pmatrix}
k-1\\ j
\end{pmatrix}
2^{-j}}{k-1-j}}
\right)\,.
%\label{e3.12b}
\end{multline*}
Таким образом,  $\phi_{2k}(s)\equiv f_{2k}(s)$ и утверждение~(\ref{e3.2b})
доказано.~$\Box$

{\small\frenchspacing
{%\baselineskip=10.8pt
\addcontentsline{toc}{section}{Литература}
\begin{thebibliography}{9}

\bibitem{3ben} %1
\Au{Королев Р.\,А., Тестова  А.\,В., Бенинг~В.\,Е.} 
О мощ\-ности асимптотически оптимального критерия в случае 
распределения Лапласа~// Вестник Тверского Государственного Университета, 
серия Прикладная математика, 2008. Вып.~8. №\,4(64). С.~5--23.

\bibitem{9ben} %2
\Au{Takeuchi K.} 
Asymptotic theory of statistical estimation.~---  Tokyo, 1974. (In Japanese.)

\bibitem{1ben} %3
\Au{Бурнашев М.\,В.} 
Асимптотические разложения для 
медианной оценки параметра~// Теор. вероятн. и ее
прим., 1996. Т.~41. Вып.~4. С.~738--753.

\bibitem{5ben}  %4
\Au{Kotz S., Kozubowski~T.\,J., Podgorski~K.}
The Laplace distribution and generalizations: 
A revisit with applications to communications, economics, engineering, 
and finance.~--- Birkhauser, 2001.  P.~349.

\bibitem{6ben}  %5
\Au{Бенинг В.\,Е., Королев В.\,Ю.}
Некоторые статистические  задачи, связанные с распределением Лапласа~// 
Информатика и её применения, 2008. Т.~2.  Вып.~2. С.~19--34.

\bibitem{7ben}  %6
\Au{Леман Э.} 
Проверка статистических гипотез.~--- М.: Наука, 1964. 498~с.

\bibitem{2ben} %7
\Au{Королев Р.\,А., Бенинг В.\,Е.}
Асимптотические 
разложения для мощностей критериев в случае распределения Лапласа~//
Вестник Тверского Государственного Университета, серия 
Прикладная математика, 2008. Вып.~3(10). №\,26(86). С.~97--107.

\bibitem{4ben} %8
\Au{David H.\,A., Nagaraja H.\,N.}
Order Statistics.  3rd ed.~--- New Jersey: Wiley, 2003.  P.~458.

\label{end\stat}

\bibitem{8ben} %9
\Au{Asrabadi B.\,R.} 
The exact confidence interval for 
the scale parameter and the MVUE of the Laplace distribution~// 
Communications in statistics. Theory and methods, 1985. Vol.~14. No.\,3. 
P.~713--733.

 \end{thebibliography}
}
}
\end{multicols}      %8
\def\stat{gaidamaka}

\def\tit{ЗАДАЧИ ОПТИМАЛЬНОГО ПЛАНИРОВАНИЯ МЕЖУРОВНЕВОГО 
ИНТЕРФЕЙСА В БЕСПРОВОДНЫХ СЕТЯХ$^*$}

\def\titkol{Задачи оптимального планирования межуровневого 
интерфейса в беспроводных сетях}

\def\autkol{Ю.\,В.~Гайдамака, Т.\,В.~Ефимушкина, А.\,К.~Самуйлов, 
К.\,Е.~Самуйлов}
\def\aut{Ю.\,В.~Гайдамака$^1$, Т.\,В.~Ефимушкина$^2$, А.\,К.~Самуйлов$^3$, 
К.\,Е.~Самуйлов$^4$}

\titel{\tit}{\aut}{\autkol}{\titkol}

{\renewcommand{\thefootnote}{\fnsymbol{footnote}}\footnotetext[1]
{Работа выполнена при частичной поддержке РФФИ (гранты 10-07-00487-a и 12-07-00108)
и Рособразования 
(проект 020619-1-174).}}

\renewcommand{\thefootnote}{\arabic{footnote}}
\footnotetext[1]{Российский университет дружбы народов, кафедра систем телекоммуникаций, ygaidamaka@sci.pfu.edu.ru}
\footnotetext[2]{Российский университет дружбы народов, кафедра систем телекоммуникаций, tefimushkina@gmail.com}
\footnotetext[3]{Российский университет дружбы народов, кафедра систем телекоммуникаций, asam1988@gmail.com}
\footnotetext[4]{Российский университет дружбы народов, кафедра систем телекоммуникаций, ksam@sci.pfu.edu.ru}
 
 
   \Abst{В данном обзоре проведено исследование современного состояния задач оптимального 
планирования межуровневого интерфейса на базе механизма мультиплексирования с 
ортогональным частотным разделением (OFDM, Orthogonal Frequency Division Multiplexing) для 
нисходящего канала в сетевой технологии LTE (Long-Term Evolution). При этом рассматривается 
понятие межуровневой оптимизации, подробно описаны оптимизационные задачи и ограничения, 
возникающие при разделении радиоресурсов в нисходящем канале, дан краткий обзор 
планировщиков и соответствующих им функций полезности, определяющих уровень 
удовлетворенности пользователей схемой распределения радиоресурсов при заданных 
ограничениях.}

%\vspace*{2pt}
   
   \KW{технология OFDM; межуровневая оптимизация; функция полезности; планировщик; 
эффективное распределение частот}

%\vspace*{6pt}

\vskip 14pt plus 9pt minus 6pt

      \thispagestyle{headings}

      \begin{multicols}{2}

            \label{st\stat}
   
\section{Введение}
  
  К основным задачам в беспроводных сетях относится оптимизация распределения 
ограниченного числа радиоресурсов между пользователями. Различные типы пакетного 
трафика, передаваемого по сети, предполагают динамическое выделение ресурсов 
пользователям. Решением задач планирования ресурсов, назначения приоритетов доступа 
в\linebreak зависимости от типа трафика с заданными требованиями к качеству обслуживания (QoS, 
Quality of\linebreak Service) занимаются модули управления радиоресурсами, называемые 
планировщиками (англ.\ \textit{schedulers}).
  
  Динамичное изменение загруженности канала\linebreak в беспроводной сети определяет 
требования к планиров\-щику, одним из которых является меж\-уров\-не\-вый (англ.\ 
\textit{crosslayer}) подход к решению\linebreak задачи оптимального распределения ресурсов. 
Основным принципом межуровневой оптимизации является комплексное решение задачи 
эффективного использования ограниченного числа радиоресурсов, учитывающее ряд 
первостепенных факторов: повышение пропускной способности; обеспечение 
равнодоступности~--- справедливого (англ.~\textit{fair}) разделения ресурсов между 
пользователями; достижение требуемого или, по крайней мере, наилучшего возможного 
качества обслуживания~[1].
  
  В обзоре в общем виде сформулированы основные задачи оптимизации, возникающие 
при планировании ресурсов в беспроводных сетях в целях повышения эффективности 
работы сети с большим числом несущих, сети, построенной на базе механизма OFDM, 
характерного для нисходящего канала в технологии LTE~[2, 3]. Вводится понятие 
межуровневой оптимизации, подробно описаны оптимизационные задачи и ограничения, 
возникающие при разделении ресурсов в нисходящем канале. Исследованы два алгоритма 
межуровневой оптимизации, предназначенных для максимизации функции полезности в 
различных условиях~--- алгоритм динамического назначения поднесущих DSA (Dynamic 
Subcarrier Assignment) и алгоритм адап\-тив\-но\-го распределения мощности APA (Adaptive 
Power Allocation). Для рассматриваемых алгоритмов сформулированы задачи 
максимизации функции полезности и получены их решения.

\section{Виды межуровневой оптимизации}

  С точки зрения терминологии межуровневая оптимизация заключается в объединении 
нескольких уровней модели взаимодействия открытых сис\-тем (OSI, Open Systems 
Interconnection) для полу\-чения более качественных решений и \mbox{эффективных} алгоритмов 
без лишних межуровневых обменов информа\-ции. Межуровневый подход к решению 
задачи оптимального распределения ресурсов позволяет в динамическом режиме учесть 
изменения типов трафика в беспроводной сети, потребности в услугах, значений 
параметров в канале связи и мобильность абонентов.
  
  Выделим три основных вида межуровневой оптимизации. Главной задачей 
  ка\-наль\-но-ориен\-ти\-ро\-ван\-но\-го вида межуровневой оптимизации является 
эффективное использование ограниченного числа изменяющихся во времени 
радиоресурсов с целью обеспечения высокой пропускной способности, заданных 
требований к качеству и равнодоступности. С~точки зрения модели OSI данная задача 
рассматривается между физическим и канальным уровнями.
  
  Второй вид, ориентированный на качество пред\-остав\-ле\-ния услуг абоненту <<из конца 
в конец>>, решает задачи адаптации протоколов верхних уровней к нестабильным, 
изменяющимся во времени канальным ресурсам для достижения заданных требований к 
QoS параметрам, например к производительности и задержкам. Заметим, что 
эффективность функционирования протокола TCP в беспроводных сетях связи~--- одна из 
типичных задач данного вида оптимизации~[4--6].
  
  Выбор наилучшего маршрута определяет третий вид межуровневой оптимизации, 
рассматриваемый в~[7--9]. При этом поиск наиболее эффективного маршрута происходит 
с учетом сетевого и физического или канального уровней. Далее в статье рассматриваются 
задачи оптимизации только с точки зрения ка\-наль\-но-ориен\-ти\-ро\-ван\-но\-го 
межуровневого подхода.

\section{Задачи оптимизации}

\subsection{Задача минимизации мощности}
  
  В~[10] сформулирована задача минимизации общей выделяемой пользователям сети 
мощ\-ности,\linebreak учитывающая распределение поднесущих с определением числа бит и уровня 
выделяемой мощ\-ности для каждой из поднесущих на основе мгновенных 
(\textit{instantaneous}) характеристик состояния\linebreak канала, измеренных для каждого из 
пользователей сети. В~рамках данной задачи предложен и реализован итерационный 
алгоритм распределения поднесущих между пользователями, а также обобщенный 
алгоритм задания числа бит и уровня мощ\-ности для поднесущих, назначенных 
пользователям. В~[10] рассматривается сеть связи с $K$ пользователями, в которой 
  $k$-поль\-зо\-ва\-тель имеет скорость передачи, равную $R_k$ бит на OFDM-сим\-вол 
(далее бит/символ), $k\hm=\overline{1,K}$. На передатчике реализован алгоритм 
назначения $n$-под\-не\-су\-щей $k$-поль\-зо\-ва\-те\-лю, после применения которого\linebreak на 
основе характеристик состояния канала для\linebreak $k$-поль\-зо\-ва\-те\-ля применяется 
обобщенный алгоритм задания $c_{k,n}$ чис\-ла бит/символ для $n$-под\-не\-су\-щей 
(здесь и далее $n\hm=\overline{1,N}$).
  
  В зависимости от числа назначенных $c_{k,n}$ бит из множества 
$\mathcal{D}=\{0,1,2,\ldots ,M\}$, где $M$~--- максимально возможное для передачи 
число бит/символ, адаптивный модулятор выбирает соответствующую схему модуляции, 
при этом уровень выделяемой мощности адаптируется согласно обобщенному алгоритму. 
Заметим, что $n$-поднесущая предоставляется только одному пользователю, т.\,е.\ при 
$c_{k^\prime,n}\not=0$, $c_{k,n}=0$ для всех $k\not=k^\prime$.
  
  В частотно-селективном канале с замираниями $n$-поднесущая характеризуется 
уровнем мощ\-ности сигнала $a_{k,n}$ по отношению к $k$-пользователю. При этом 
дисперсия уровня спектральной плот\-ности шума $\sigma_{k,n}^2$ принята равной 
единице для всех поднесущих. Для поддержания требуемого качества услуги на 
приемнике выделяемая мощность для передачи $k$-пользователю на $n$-поднесущей 
рассчитывается по формуле 
$$
P_{k,n}=\fr{f_k(c_{k,n})}{a_{k,n}^2}\,,
$$ 
где $f_k(c_{k,n})$~--- 
требуемая мощность для приема данных и их последующей демодуляции. Таким образом, 
задача минимизации мощности представляется в виде 
$$
P^*=\min\limits_{c_{k,n}\in\mathcal{D}}\sum\limits_{n=1}^N \sum\limits_{k=1}^K 
\fr{f_k(c_{k,n})}{a_{k,n}^2}
$$ 
c ограничением для $k$-поль\-зо\-ва\-те\-ля по числу бит для передачи 
$R_k=\sum\limits_{n=1}^N c_{k,n}$.
  
  Исследованный в~[10--12] алгоритм задания чис\-ла бит и уровня мощности в сети с 
одним пользователем служит основой для решения задачи минимиза\-ции мощ\-ности для 
случая многопользовательской сети. Этот алгоритм относится к так называемым 
<<жад\-ным>> алгоритмам и назначает бит под\-не\-су\-щей, требующей выделения 
наименьшей дополнительной мощности. Процесс назначения происходит по одному биту 
за один раз до тех пор, пока $R$ бит не будут распределены между $N$ поднесущими.
  
  Решение оптимизационной задачи для случая многопользовательской сети 
предусматривает использование действительных значений для числа бит/символ 
$c_{k,n}\in \mathbb{R}[0,\,M]$, а также введение функции назначения $k$-поль\-зо\-ва\-те\-лю 
  $n$-под\-не\-су\-щей, $\rho_{k,n}\in \mathbb{R}[0,\,1]$. Тогда задача минимизации 
мощности принимает вид:
$$
p^*=\min\limits_{c_{k,n},\rho_{k,n}} \sum\limits_{n=1}^N 
\sum\limits_{k=1}^K \fr{\rho_{k,n} f_k(c_{k,n})}{a_{k,n}^2}
$$
c ограничениями $\sum\limits_{n=1}^N \rho_{k,n}c_{k,n}=R_k$ и $\sum\limits_{k=1}^K 
\rho_{k,n}=1$. Данное предположение позволяет решить задачу назначения поднесущей 
  $k$-пользователю методом множителей Лагранжа по алгоритму, приведенному в~[10]. 
В~рамках решения задачи оптимизации полученные величины определяют нижнюю 
границу искомого минимального значения выделяемой мощности. Однако из-за 
принятого ранее предположения о величинах $c^*_{k,n}\not\in \mathbb{Z}$ и 
$\rho_{k,n}^*\in \mathbb{R}[0,\,1]$ предполагается разделение поднесущей между 
несколькими пользователями. Решение данной проблемы методом квантования 
полученных величин может не удовлетворять требованию $k$-пользователя к скорости 
передачи~$R_k$. В~предположении, что $\rho_{k^\prime,n}^*=1$, 
$k^\prime\hm=\mathrm{arg}\,\max\limits_k \rho_{k,n}^*$ и $\rho^*_{k,n}=0$, 
$k\not=k^\prime$, алгоритм назначения $n$-поднесущей в~[10] дополняется алгоритмом 
задания числа бит и уровня мощности в сети с одним пользователем. В~результате в~[10] 
предложен многопользовательский адаптивный алгоритм (MAO, Multiuser Adaptive 
OFDM).
  
  Данный поход к решению задачи оптимизации, согласно~\cite{13-gai}, относят к 
методу релаксаций. Использование нецелого числа бит и разделения поднесущей между 
пользователями позволяет эффективно решать задачу оптимизации, однако требует 
применения дополнительных процедур для получения целых величин, являющихся 
целесообразными с точки зрения функционирования сети. Двумя другими методами, 
предложенными в~\cite{13-gai}, являются разбиение задачи на несколько более прос\-тых и 
эвристический алгоритм. Первый метод предполагает определение числа поднесущих для 
$k$-поль\-зо\-ва\-те\-ля с учетом требований к скорости передачи $R_k$ и далее назначение 
конкретных, выбранных по некоторому алгоритму, поднесущих. Эвристический подход 
основывается на методе сортировки и представляет собой реализацию двухэтапного 
аналитического метода, описанного выше. Решение задачи оптимизации с помощью 
эвристического метода представлено также в~\cite{14-gai}.

\subsection{Задача максимизации скорости передачи}

  В~[15] нелинейная оптимизационная задача преобразована в линейную задачу 
максимизации скорости передачи путем равномерного разделения общей мощности 
$p_{tot}$ между пользователями в сети для каждой из поднесущих: 

\noindent
$$
p_{k,n}=\fr{p_{tot}}{N}\,.
$$ 
Задача максимизации общей пропускной способности сети представлена в виде:

\noindent
$$
R^*=\max\limits_{c_{k,n}\in\mathcal{D}} \sum\limits_{n=1}^N \sum\limits_{k=1}^K 
c_{k,n} \rho_{k,n}\,,
$$ 
принимая во внимание требование $r_k$, предъявляемое $k$-поль\-зо\-ва\-те\-лем 
к минимальной ско\-рости передачи чис\-ла бит на один OFDM-сим\-вол, 

\noindent
$$
\sum\limits_{n=1}^N c_{k,n}\rho_{k,n}\geq r_k\,.
$$
  
  Фиксируя уровень выделяемой мощности $p_{k,n}$ и допуская, что значения 
заданного для $k$-поль\-зо\-ва\-те\-ля коэффициента ошибок BER (Bit Error Rate) и состояния 
канала $a_{k,n}$ известны на базовой станции для всех пользователей, находятся 
значения числа бит/символ 

\noindent
$$
c_{k,n}=f(\mathrm{BER}, p_{k,n}, a_{k,n})\,.
$$

Данный подход позволяет 
эффективно решить линейную задачу оптимизации методом це\-ло\-чис\-лен\-но\-го линейного 
программирования, однако предусматривает экспоненциальный рост уровня сложности с 
увеличением числа поднесущих и пользователей в сети.
  
  В~\cite{15-gai} предложен алгоритм понижения слож\-ности, состоящий из двух этапов: 
назначение под-\linebreak несущих пользователям с наибольшим воз\-мож-\linebreak ным числом бит для 
передачи и перераспределение поднесу\-щих для соблюдения требований~$r_k$. 
%
Поднесущие на первом этапе распределяются между пользователями с целью достижения 
максимальной пропускной способности без учета требований~$r_k$ к минимальной 
скорости передачи. 
%
Для перераспределения пользователей на втором этапе требуется 
соблюдение следующих условий:
  \begin{enumerate}[(1)]
\item выделенная на первом этапе $k_n^*$-поль\-зо\-ва\-те\-лю $n$-поднесущая не может 
быть переназначена другим пользователям, если переназначение повлечет возможное 
нарушение требования~$r_k$ к минимальной скорости передачи 
$k_n^*$-поль\-зо\-ва\-те\-ля, $R_{k_n^*,n}\hm-c_{k_n^*,n}\hm<r_{k_n^*}$;
\item каждое переназначение поднесущих должно минимально сокращать общую 
пропускную способность сис\-темы;
\item число переназначений должно быть наименьшим.
\end{enumerate}

  Для выполнения последних двух условий вводится функция $e_{k,n}= (c_{k_n^*,n}-
c_{k,n})/c_{k,n}$ оцен-\linebreak ки переназначения $n$-поднесущей $k$-пользователю.\linebreak
 Согласно 
алгоритму~[15] перераспределение происходит поочередно для всех пользователей в сети, 
которым назначены поднесущие после первого этапа, не удовлетворяющие требованиям 
по скорости передачи. При этом для $k$-поль\-зо\-ва\-те\-ля выбирается поднесущая с 
наименьшей функцией оценки переназначения. Перераспределение $k$-поль\-зо\-ва\-те\-ля на 
$n^\prime$-под\-не\-су\-щую происходит, если $R_{k^*_{n^\prime},n^\prime}-
c_{k^*_{n^\prime},n}\geq r_{k^*_{n^\prime}}$. В~противном случае выбирается другая 
поднесущая с минимальной функцией перераспределения.

\subsection{Задача обеспечения равнодоступности}

  В~[16] приведены три из наиболее известных схем распределения поднесущих между 
пользователями. 

Согласно первой из них обеспечение максимальной пропускной 
способности (maxBR, maximum bit-rate) достигается за счет предоставления\linebreak 
  $n$-поднесущей $k$-пользователю, находящемуся в лучших канальных условиях, т.\,е.\ 
обладающему наибольшим частотным откликом (англ.\ \textit{frequency response}) 
$H_{k,n}$. Следует отметить, что данный метод не решает задачу обеспечения 
равнодоступности. Однако, рассматривая величину частотного отклика в качестве 
единственного параметра при распределении поднесущих, метод maxBR определяет 
верхнюю границу возможной скорости передачи данных.
  
  Вторая схема распределения канальных ресурсов, изложенная в~[17], предполагает 
решение задачи обеспечения равнодоступности путем предо\-став\-ле\-ния одинаковой 
скорости передачи всем пользователям. Помимо ограничения по мощности передачи 
$\sum\limits_{k=1}^K \sum\limits_{n=1}^N p_{k,n}\leq p_{\mathrm{tot}}$ в~[17] также вводится 
пропорциональное ограничение: 
$$
N_1:N_2:\ldots:N_K=\phi_1:\phi_2:\ldots\phi_K\,,
$$ 
где 
$N_k\hm=\phi_kN$~--- чис\-ло поднесущих, назначенных $k$-поль\-зо\-ва\-те\-лю, и $\phi_k$~--- 
нормированная пропорциональная постоянная скорости передачи данных для 
  $k$-поль\-зо\-ва\-те\-ля. При этом $\tilde{n}=N/K$ определяет максимально возможное чис\-ло 
выделяемых пользователю поднесущих.
  
  Третья схема назначения поднесущих, предложенная в~[18], заключается в выборе 
$\tilde{n}$ поднесущих с наибольшими значениями частотных откликов для пользователя. 
Данная процедура повторяется для всех пользователей в сети.
  
  Введем величину отношения сигнал-шум (SNR, Signal-to-Noise Ratio), используемую 
для постановки оптимизационной задачи:
$$
\mathrm{SNR}_{k,n}=\fr{\vert H_{k,n}\vert^2 
p_{k,n}}{\sigma_{k,n}^2}\,.
$$ 
Задача максимизации общей пропускной способности всех 
пользователей в сети формулируется в виде:
$$
F^*=\max\limits_{\mathrm{SNR}_{k,n}} 
\sum\limits_{k=1}^K \sum\limits_{n=1}^N f(\mathrm{SNR}_{k,m})\,.
$$
Следует отметить, что подобные 
задачи не учитывают улучшения пропускной способности отдельных пользователей. 
В~[16] приводится решение данной оптимизационной задачи с учетом обеспечения 
рав\-но\-до\-ступ\-ности путем назначения равного чис\-ла поднесущих по одному из двух 
алгоритмов, кратко охарактеризованных ниже, а далее путем использования обобщенного 
алгоритма задания чис\-ла бит/символ для каждой поднесущей.
  
  Первый алгоритм заключается в сравнении и выборе поднесущей с наибольшим 
частотным откликом, а также ее назначении пользователю. В~ходе данного назначения 
пользователь, получивший оптимальное число поднесущих, удаляется из 
рассматриваемого множества. Данная процедура продолжается для всех оставшихся 
пользователей и поднесущих. 

Второй алгоритм заключается в поиске для первого 
пользователя максимального частотного отклика и назначении ему соответствующей 
поднесущей. После выделения по одной поднесущей каждому из $K$ пользователей 
данный алгоритм повторяется в противоположном порядке: от $K$-го до 1-го 
пользователя.

\subsection{Задача максимизации полезности}

  Следует отметить, что все рассмотренные выше задачи подразумевают оптимизацию 
некоторой функции полезности, описывающей тот или иной уровень удовлетворенности 
пользователей для определенной схемы распределения радиоресурсов при некоторых 
ограничениях. Тем не менее, как будет показано ниже, оптимизация полезности может 
оказаться задачей, представляющей самостоятельный интерес.
  
  В~[19, 20] предложены два алгоритма межуровневой оптимизации, предназначенные для 
максимизации функции полезности (Utility Function) в различных условиях~--- алгоритм 
динамического назначения поднесущих DSA (Dynamic Subcarrier Assignment) и 
алгоритм адаптивного распределения мощности APA, а 
также комбинация этих алгоритмов. Эффективность алгоритмов оценивалась с помощью 
имитационного моделирования, при котором они сравнивались с алгоритмом 
фиксированного назначения поднесущих FSA (Fixed Subcarrier Assignment). Ниже для 
алгоритмов DSA и APA сформулированы две задачи нелинейного целочисленного 
программирования, для которых получены условия оптимальности. Вводятся следующие 
обозначения: $\mathcal{N}$~--- множество поднесущих $\{1,\ldots ,N\}$; 
$\mathcal{K}$~--- множество пользователей $\{1,\ldots ,K\}$; $\beta$~--- коэффициент 
побитовой ошибки (BER); $\mathbf{p}=(p[1],\ldots ,p[N])$~--- вектор мощностей\linebreak 
поднесущих; $\rho$~--- состояние поднесущей (отношение сиг\-нал--шум); $\Delta f$~--- 
ширина полосы пропускания поднесущей; $c_k^p[n]$~--- достижимая эффективность 
передачи (бит/символ); $r_k$~--- скорость \mbox{передачи} $k$-пользователя, $k\in\mathcal{K}$.
  
  Предполагается, что нисходящий канал базовой станции ячейки сети OFDM
используется всеми пользователями, причем базовой станции известно со\-сто\-яние 
назначенной пользователю поднесущей. Достижимая скорость передачи данных зависит 
от отношения сиг\-нал--шум и мощности передачи: 

\vspace*{4pt}

\noindent
$$
c_k^p[n]= f(\log_2(1+\beta 
p[n]\rho_k[n])\,,
$$
где функция $f(\cdot)$ зависит от выбранной схемы адап\-та\-ции скорости. 
Например, если применять непрерывную адаптацию скорости, то $f(x)=x$ и 
$c_k^p[n]=\log_2(1+\beta p[n]\rho_k[n])$.
  
  Введем $x_{kn}\in \{0,\,1\}$ состояние $n$-поднесущей так, что $x_{kn}\hm=1$, если 
  $n$-под\-не\-су\-щая назначена $k$-поль\-зо\-ва\-те\-лю, и $x_{kn}\hm=0$ в противном случае. Тогда 
$x_k\hm=(x_{kn})_{n\in\mathcal{N}}$~--- вектор состояния поднесущих для 
  $k$-пользователя, причем условие $\sum\limits_{k\in\mathcal{K}} x_{kn}=1$ означает, 
что $n$-поднесущая может быть назначена только одному пользователю. Множество 
$\mathrm{D}_k(\mathbf{x}_k) =\{n:\ x_{kn}=1\}$ включает все поднесущие, назначенные 
$k$-пользователю в состоянии $\mathbf{x}_k$, а набор множеств 
$\mathrm{D}(\mathbf{x})=(\mathrm{D}_k (\mathbf{x}_k))_{k\in \mathcal{K}}$ определяет 
распределение поднесущих по пользователям, когда система находится в состоянии 
$\mathbf{x}=(\mathbf{x}_k)_{k\in\mathcal{K}}$. Тогда множество состояний системы 
можно определить в виде:

\vspace*{-4pt}

\noindent
  \begin{multline*}
  \mathrm{X}=\left\{ \vphantom{\mathop{\bigcup}\limits_{k\in \mathcal{K}}}
  \mathbf{x} =(\mathbf{x}_k)_{k\in\mathcal{K}}:\ 
\mathrm{D}_i(\mathbf{x}_i) \cap \mathrm{D}_j(\mathbf{x}_j) =\varnothing\,,\right.\\[-3pt]
  \left. i\not=j\in\mathcal{K}\,,\enskip \mathop{\bigcup}\limits_{k\in \mathcal{K}} 
\mathrm{D}_k(\mathbf{x}_k) \subseteq \mathcal{N}\right\}\,.
  \end{multline*}
  
%  \pagebreak
  
     Введем множество всех возможных наборов поднесущих  
$\mathcal{D}\hm=\{D(\mathbf{x}):\ \mathbf{x}\hm\in \mathcal{X}\}$ и множество возможных 
вариантов распределения мощностей 
$$
\mathcal{P}=\{\mathbf{p}:\ 0\leq p(n)\leq P, \ 
\sum\limits_{n\in N} p(n)=P\,, \  n\in \mathcal{N}\}\,.
$$

%\columnbreak

Скорость передачи данных $r_k$ [бит/с] 
для\linebreak $k$-поль\-зо\-вателя в состоянии $\mathbf{x}_k$ представима в виде:
     \begin{multline*}
     r_k {:=} r_k (\mathbf{x}_k,\mathbf{p}) =\sum\limits_{n\in\mathcal{N}} 
c_k^{\mathbf{p}}(n) \Delta f x_{kn}={}\\
     {}=\sum\limits_{n\in \mathcal{D}_k(\mathbf{x}_k)}  c_k^{\mathbf{p}}(n)\Delta f =r_k 
(\mathcal{D}_k (\mathbf{x}_k),\mathbf{p}) {=:} r_k(\mathcal{D}_k, \mathbf{p})\,,\\ k\in 
\mathcal{K}\,.
     \end{multline*}
     
     Пусть $U_k(\cdot)$~--- функция полезности для $k$-поль\-зо\-ва\-те\-ля, $k\hm\in 
\mathcal{K}$. Будем рассматривать в качестве основного блага для пользователя величину 
скорости передачи данных $r_k$ и определим функцию полезностив следующем виде:
     $$
     U(r(d,\mathbf{p})) {:=} \sum\limits_{k\in \mathcal{K}} 
U_k\left(r_k\left(D_k,\mathbf{p}\right)\right)\,.
     $$
     
     Таким образом, задача межуровневой оптимизации в общем случае может быть 
сформулирована как максимизация функции полезности для ячейки сети OFDM в виде:
     $$
     \max\limits_{d,\mathbf{p}}\sum\limits_{k\in \mathcal{K}} U_k(r(d,\mathbf{p}))
     $$
с ограничениями $d\in \mathcal{D}$ и $\mathbf{p}\in \mathcal{P}$.
     
     Пусть вектор \textbf{p} распределения мощностей фиксирован, т.\,е.\ 
$\mathbf{p}=\tilde{\mathbf{p}}$. Тогда функцию полезности для алгоритма 
динамического назначения поднесущих DSA можно определить в виде:
    $$
     U(r(d)){:=} U\left(r\left(d,\tilde{\mathbf{p}}\right)\right)\,.
     $$
     
     В случае алгоритма адаптивного распределения мощностей APA фиксированным 
является набор $d$ множеств поднесущих $d=\tilde{d}$, и, следовательно, функция 
полезности имеет вид:
     $$
     U(r(\mathbf{p})) {:=}  U\left( r\left( \tilde{d},\mathbf{p}\right)\right)\,.
     $$
  
  Отметим, что данная задача относится к классу задач целочисленного нелинейного 
программирования. В~[19] для решения этой задачи используется метод релаксаций, идея 
которого заключается в том, что при разработке метода решения задач отбрасывается 
требование к целочисленности переменных. Предполагается, что функция полезности 
$U_k(r_k)$ для $k$-пользователя является неубывающей выпуклой и существует ее 
производная $U^\prime_k(r_k)$.

\bigskip

\noindent
\textit{Алгоритм динамического назначения поднесущих} DSA
    
    
    \vspace*{2pt}
     
     С учетом введенных обозначений задача максимизации полезности для алгоритма 
DSA записывается в виде $\max\limits_{d\in \mathcal{D}} U(r(d))$.
     
     Используя метод математической индукции можно доказать, что максимум функции 
по\-лез\-ности $U(r(d))$ достигается в сформулированных ниже достаточных условиях 
оптимальности.


\medskip

\noindent
\textbf{Утверждение 1.} Если для набора $d^*=(D^*_k)_{k\in \mathcal{K}}$ выполняется 
условие:
$$
U^\prime_k (r^*_k) c_k^{\tilde{\mathbf{p}}}(n)\geq U^\prime_j (r^*_j) 
c_j^{\tilde{\mathbf{p}}}(n)\,,\enskip  k\not=j\in \mathcal{K}\,,\ n\in D_k^*\,,
$$
где $r_k^*=\sum\limits_{n\in D_k^*} c_n^{\tilde{\mathbf{p}}}(n)\Delta f$, тогда функция 
полезности $U(r(d))$ достигает глобального максимума на наборе $d=d^*\in 
\mathcal{D}$.
     
     \medskip
     
     Из полученных условий оптимальности получаем правило назначения поднесущей 
пользователю. Для заданного вектора $\mathbf{p}=\tilde{\mathbf{p}}$ распределения 
мощности номер пользователя, которому назначается $n$-поднесущая, определяется 
формулой:
     $$
     k(n) =\mathrm{arg}\max\limits_{k\in\mathcal{K}} \left\{ U^\prime_k (r_k^*) 
c_k^{\tilde{p}}(n)\right\}\,.
     $$

\medskip

\noindent
\textit{Алгоритм адаптивного распределения мощности} APA

\smallskip

     Алгоритм адаптивного распределения мощ\-ности APA состоит в назначении 
каждой поднесущей $n\in \mathcal{N}$ определенной мощности передачи при условии 
фиксированного набора $d=\tilde{d}$ распределения поднесущих между пользователями.
     
  Задача межуровневой оптимизации для алгоритма APA может быть сформулирована 
в виде $\max\limits_{p\in \mathcal{P}} U(r(\mathbf{p}))$. Ниже сформулированы 
необходимые условия достижения максимума функцией полезности $U(r(\mathbf{p}))$.
  
  \medskip
  
  \noindent
\textbf{Утверждение 2.} Если $p^*(n)$  является решением задачи 
$\max\limits_{\mathbf{p}\in \mathcal{P}} U(r(\mathbf{p}))$, тогда
$$
p^*(n)=\left[ \fr{U^\prime_k(r^*_k)}{\lambda}-\fr{1}{\beta \rho_k(n)}\right]^*\,,\enskip k\in 
\mathcal{K}\,,\ n\in \tilde{D}_k\,,
$$
где $\lambda>0$~--- нормирующая константа оптимального распределения мощностей.
     
     Из полученных условий оптимальности функции $U(r(\mathbf{p}))$ очевидным 
образом следует правило назначения мощностей поднесущих для алгоритма APA.

\section{Заключение}
  
  Планирование межуровневого интерфейса является наиболее эффективным подходом к 
согласованию возможностей современных беспроводных технологий и возрастающих 
требований по обслуживанию больших объемов трафика пользователей с заданным 
качеством. 
  
  В обзоре технические и алгоритмические проблемы создания планировщиков 
межуровневого интерфейса иллюстрированы постановками оптимизационных задач, 
возникающих при распределении ресурсов в сетях с большим числом несущих 
радиочастот. Рассмотрены наиболее известные задачи и ограничения, характерные для 
технологии LTE, даны краткие комментарии по их решению, алгоритмам поиска 
оптимального решения и условиям оптимальности. Приведен типичный пример задачи 
оптимизации функции полезности как наиболее общей задачи оптимального 
планирования межуровневого интерфейса. Исследовано и сформулировано достаточное 
условие нахождения глобального максимума функции полезности для задачи DSA, а 
также необходимое условие для задачи APA.

{\small\frenchspacing
{%\baselineskip=10.8pt
\addcontentsline{toc}{section}{Литература}
\begin{thebibliography}{99}

\bibitem{1-gai}
\Au{Shariat M., Quddus A.\,U., Ghorashi~S.\,A., Tafazolli~R.}
 Scheduling as an important cross-layer operation for emerging broadband wireless systems~// 
IEEE Commun. Surveys Tuts., 2009. Vol.~11. No.\,2. P.~74--86.
\bibitem{2-gai}
\Au{Вишневский В.\,М., Ляхов А.\,И., Портной~С.\,Л., Шахнович~И.\,В.}
Широкополосные беспроводные сети передачи информации.~--- М.: Техносфера, 2005. 
597~c.
\bibitem{3-gai}
\Au{Тихвинский В.\,О., Терентьев С.\,В., Юрчук~А.\,Б.} Сети мобильной связи LTE: 
технология и архитектура.~--- М.: Эко-Трендз, 2010. 284~с.
\bibitem{4-gai}
\Au{Wu G., Bai~Y., Lai~J., Ogielski~A.} Interaction between TCP and RLP in wireless 
Internet~// IEEE Global Communication Conference Proceedings, 1999. Vol.~1b. P.~661--666.
\bibitem{5-gai}
\Au{Kim B.\,J.} A~network service providing wireless channel information for adaptive mobile 
applications: Part~I: Proposals~// IEEE  Conference (International) on Communications 
Proceedings, 2001. Vol.~5. P.~1345--1351.
\bibitem{6-gai}
\Au{Sudame P., Badrinath~B.\,R.}
On providing support for protocol adaptation in mobile networks~// Mobile Networks 
Applications, 2001. Vol.~6. No.\,1. P.~43--55.
\bibitem{7-gai}
\Au{Chiang M.} To layer or not to layer: Balancing transport and physical layers in wireless 
multihop networks~// IEEE J.~Selected Areas  Commun., 2005. Vol.~23. No.\,1. 
P.~104--116. 
\bibitem{8-gai}
\Au{Kawadia V., Kumar P.\,R.}
A~cautionary perspective on cross-layer design~// IEEE Wireless Commun., 2005. Vol.~12. 
No.\,1. P.~3--11.
\bibitem{9-gai}
\Au{Iannone L., Fdida S.} Evaluating a cross-layer approach for routing in wireless mesh 
networks~// Telecommunication Systems J. (Springer) Special issue: Next Generation 
Networks~--- Architectures, Protocols, Performance, 2006. Vol.~31. No.\,2--3. P.~173--193.
\bibitem{10-gai}
\Au{Wong C.\,Y., Cheng R.\,S., Letaief~K.\,B.} Multiuser OFDM with adaptive subcarrier, bit, 
and power allocation~// IEEE J. Selected Areas  Commun., 1999. Vol.~17. No.\,10. 
P.~1747--1757.
\bibitem{11-gai}
\Au{Hughes-Hartogs D.} Ensemble modem structure for imperfect transmission media. U.S.\ 
Patents No.\,4679227, July 1987; No.\,4731816, March 1988; No.\,4833796, May 1989.
\bibitem{12-gai}
\Au{Lai S.\,K., Cheng R.\,S., Letaief K.\,B., Murch~R.\,D.} Adaptive trellis coded MQAM and 
power optimization for OFDM transmission~// IEEE Trans. Commun., 1999. Vol.~47. 
P.~538--545.
\bibitem{13-gai} %12
\Au{Bohge M., Gross J., Wolisz~A., Meyer~M.} Dynamic resource allocation in OFDM Systems: 
an overview of cross-layer optimization principles and techniques~// IEEE Networks, 2007. 
Vol.~21. No.\,1. P.~53--59.
\bibitem{14-gai}
\Au{Kivanc D., Li G., Liu~H.}
Computationally efficient bandwidth allocation and power control for OFDMA~// IEEE 
Trans. Wireless Commun., 2003. Vol.~2. No.\,6. P.~1150--1158.
\bibitem{15-gai}
\Au{Zhang Y.\,J., Letaief K.\,B.} Multiuser adaptive subcarrier and bit allocation with adaptive 
cell selection for OFDM systems~// IEEE Trans. Wireless Commun., 2004. 
Vol.~3. No.\,5. P.~1566--1575.
\bibitem{16-gai}
\Au{Otani Y., Ohno S., Teo K., Teo~D., Hinamoto~T.}
Subcarrier allocation for multi-user OFDM system~// Asia-Pacific Communication Conference 
Proceedings, 2005. P.~1073--1077.
\bibitem{17-gai}
\Au{Wong C., Shen Z., Evans L., Andrews~J.\,G.} A~low complexity algorithm for proportional 
resource allocation in OFDMA systems~// IEEE Workshop on Signal Processing Systems 
Proceedings.~--- Texas, USA, 2004. P.~1--6.
\bibitem{18-gai}
\Au{Fu J., Karasawa Y.} Fundamental analysis on throughput characteristics of orthogonal 
frequency division multiple access OFDMA in multipath propagation environments~// IEICE  
Trans., 2002. Vol.~J85-B. No.\,11. P.~1884--1894.

\label{end\stat}
\bibitem{19-gai}
\Au{Song G., Li~Ye.}
Cross-layer optimization for OFDM wireless networks~--- Part~I: Theoretical framework~// 
IEEE Trans. Wireless Commun., 2005. Vol.~4. No.\,2. P.~614--624.


\bibitem{20-gai}
\Au{Song G., Li Y.} Cross-layer optimization for OFDM wireless networks~--- Part~II: 
Algorithm development~// IEEE Trans. Wireless Commun., 2005. Vol.~4. 
No.\,2. P.~625--634.
%\bibitem{21-gai}
%\Au{Глебов Н.\,И., Кочетов Ю.\,А., Плясунов~А.\,В.}
%Методы оптимизации: Учебное пособие.~--- Новосибирск: НГУ, 2000. 105~с.
 \end{thebibliography}
}
}


\end{multicols}    %9
\def\stat{mor-luk}

\textit{\hfill Посвящается 100-летию  со дня рождения Б.\,В.~Гнеденко}

\def\tit{АСИМПТОТИКА МАКСИМУМА ПРОЦЕССА НАГРУЗКИ ДЛЯ~НЕКОТОРОГО
КЛАССА ГАУССОВСКИХ ОЧЕРЕДЕЙ$^*$}

\def\titkol{Асимптотика максимума процесса нагрузки для некоторого
класса гауссовских очередей}

\def\autkol{О.\,В.~Лукашенко, Е.\,В.~Морозов}
\def\aut{О.\,В.~Лукашенко$^1$, Е.\,В.~Морозов$^2$}

\titel{\tit}{\aut}{\autkol}{\titkol}

{\renewcommand{\thefootnote}{\fnsymbol{footnote}}\footnotetext[1]
{Работа поддержана РФФИ (проект 10-07-00017). Работа выполнена
при поддержке Программы стратегического развития на 2012--2016~гг.\
<<Университетский комплекс ПетрГУ в научно-образовательном пространстве
Европейского Севера: стратегия инновационного развития>>.}}


\renewcommand{\thefootnote}{\arabic{footnote}}
\footnotetext[1]{Институт прикладных математических исследований КарНЦ 
РАН, Петрозаводский государственный университет,\linebreak lukashenko-oleg@mail.ru}
\footnotetext[2]{Институт прикладных математических исследований КарНЦ РАН, 
Петрозаводский государственный университет,\linebreak emorozov@karelia.ru}

\vspace*{-12pt}

\Abst{Изучается  асимптотическое поведение максимума
 процесса нагрузки в жидкостной  системе обслуживания,  на вход которой  поступает
процесс, содержащий  случайную компоненту, описываемую
центрированным гауссовским процессом.  Предполагается, что дисперсия
этого процесса является регулярно меняющейся на бесконечности
функцией с показателем $V\hm\in (0,\,2)$. К такому классу процессов, в
частности, относится сумма независимых дробных броуновских движений (ДБД).
 Показано, что при соответствующей  нормировке
   максимум процесса нагрузки на интервале $[0,\,t]$
сходится по вероятности при  $t\to \infty$  к некоторой явно
выписанной константе.}

\vspace*{-2pt}

\KW{гауссовская система обслуживания; максимум
процесса нагрузки; дробное броуновское движение; асимптотический
анализ;  правильное изменение}

\vspace*{-8pt}

\vskip 14pt plus 9pt minus 6pt

      \thispagestyle{headings}

      \begin{multicols}{2}

            \label{st\stat}

\section{Введение}

В последнее время значительно возрос интерес исследователей к
анализу гауссовских жидкостных моделей  телекоммуникационных систем.
В~таких моделях входной поток, задающий величину поступившей в
сис\-те\-му работы, является гауссовским процессом (далее~--- гауссовский
входной процесс). Основная причина этого интереса состоит в том,
что, как было выяснено рядом исследователей, гауссовские процессы
позволяют учесть при моделировании современных телекоммуникационных
систем такие важные характеристики сетевого трафика, как самоподобие
(инвариантность по времени) и долговременную зависимость (долгую
память)~\cite{Leland, Willinger}. Наличие таких свойств существенно
затрудняет вероятностный анализ и, как правило, не позволяет
получить в явном виде ключевые характеристики системы, такие как
вероятность переполнения буфера, вероятность потери сообщения 
и~т.\,д. Эти характеристики   критически важны  для определения
качества обслуживания (QoS), обеспечиваемого данной системой. 
С~другой стороны, гауссовские процессы  достаточно хорошо изучены, и
это обстоятельство позволяет в ряде случаев осуществлять, по крайней
мере, асимптотический анализ систем с гауссовских входным процессом.

Наиболее важным (входным) гауссовским процессом в
телекоммуникационных системах  является ДБД, 
обла\-да\-ющее самоподобием и  долгой памятью. Важность
этого процесса обусловлена, в  частности, тем, что ДБД возникает при
суперпозиции большого числа независимых так называемых
on/off-источников с тяжелыми хвостами на больших масштабах времени.

Известно, что в  системе с  бесконечным буфером и входным процессом
ДБД  стационарный процесс загрузки~$Q^*$ (текущая
незавершенная работа в системе) распределен как  максимум
гауссовского процесса с отрицательным линейным сносом на
положительной полуоси~\cite{Reich}.

Отсутствие точных аналитических результатов для~$Q^*$ вызывает
необходимость исследования асимптотик соответствующих характеристик
системы. Такие результаты получены, например, для ве\-ро\-ят\-ности
$\mathbb{P}\left(Q^*>b \right)$ переполнения буфера размера~$b$~\cite{Narayan, Husler} 
либо для логарифма этой ве\-ро\-ят\-ности
(логарифмические асимптотики) при  $ b \hm\to \infty$~[6--8]. 
В~большинстве  упомянутых  работ в качестве входного процесса рассматривается  ДБД.

Наряду с вероятностью переполнения другой важной характеристикой
систем обслуживания является максимум процесса загрузки на конечном
интервале $[0,\,t]$. Для этой характеристики в работах~\cite{Zeevi, Husler1} 
найдены асимптотики (при   $t\hm\to \infty$)  в
случае единственного входного процесса ДБД. Эти результаты легли в основу представленного в данной статье асимптотического анализа
максимума процесса загрузки в более общей модели. Рас\-смат\-ри\-ва\-ет\-ся 
сис\-те\-ма с входным процессом, содержащим  гауссовский
процесс со стационарными приращениями, дисперсия которого
принадлежит к клас\-су {\it правильно меняющихся на бесконечности
функций}. Част\-ным случаем такого процесса является суперпозиция
независимых ДБД. Прежде всего дадим мотивировку такой постановки
задачи. Рас\-смот\-рим $N$ независимых on/off-ис\-точ\-ни\-ков, причем \mbox{$k$-й}
источник описывается процессом $\{I_k(t),\,\,t \hm \geq 0\}$,
$k=1,\ldots ,N$, где
\begin{equation*}
I_k(t)=\begin{cases}
 1\,, &\ t\in \mbox{ on-период}\,; \\[3pt]
 0\,, &\ t\in \mbox{ off-период\,.} \\
\end{cases}
%\label{Luk-l1}
\end{equation*}
Поясним, что on-период означает период непрерывной работы
источника, а следующий за ним (независимый) off-пе\-ри\-од  есть время
простоя. Таким образом, on/off-периоды каждого источника образуют
альтернирующий процесс восстановления. По условию, процессы для
разных источников независимы. Суммарная нагрузка (совокупный
агрегированный трафик), поступившая в систему от всех источников на
интервале $[0,t]$, равна
\begin{equation*}
A_N (t):=\int\limits_0^{t} {\left( {\sum\limits_{k=1}^N {I_{k}(u)} }
\right)\,du}\,,
\end{equation*}
т.\,е.\ это суммарное время работы (активности)  всех $N$ источников
на интервале $[0,t]$. Предположим, что имеется  $n$ типов
источников, среди которых  $N_i$ источников типа $i=1,\ldots ,n$.
Предположим также, что хвосты функций распределения
on/off-пе\-рио\-дов  источника $i$-го типа имеют следующую асимптотику при $x \hm\to \infty$:
\begin{equation}
\left.
\begin{array}{rl}
1-F_{\mathrm{on}}^i(x)& \sim  \ell_{\mathrm{on}}^i x^{-\alpha_{\mathrm{on}}^i}L_{\mathrm{on}}^i(x)\,,\\[9pt]
1-F^i_{\mathrm{off}}(x)& \sim  \ell_{\mathrm{off}}^i x^{-\alpha_{\mathrm{off}}^i}L_{\mathrm{off}}^i(x)\,,
\end{array}
\right\}
\label{e3-ml}
\end{equation}
где $\ell_{\mathrm{on}}^i,\ell_{\mathrm{off}}^i$~--- положительные константы,
показатели  $\alpha_{\mathrm{on}}^i,\alpha_{\mathrm{off}}^i\hm\in (1,\,2)$, а функции
$L_{\mathrm{on}}^i$, $L_{\mathrm{off}}^i$  медленно меняются  на бесконечности, т.\,е.\
$$
\lim_{x \to \infty} \fr{L^i(tx)}{L^i(x)}=1\,,\enskip i=1,\ldots,n\,,
$$
для любого фиксированного $t \hm>0$.  (Условия~(\ref{e3-ml}) означают, что
функции распределения $F_{\mathrm{on}}^i$ и $F_{\mathrm{off}}^i$ имеют {\it тяжелые
хвосты}.) Обозначим через $\mu_{\mathrm{on}}^i$ и $\mu_{\mathrm{off}}^i$
математическое ожидание on- и off-пе\-рио\-да соответственно для
источника~$i$. (Заметим, что эти величины конечны, поскольку
$\alpha_{\mathrm{on}}^i,\, \alpha_{\mathrm{off}}^i\hm>1$.) В~работе~\cite{Taqqu} доказана
функциональная предельная теорема, согласно которой  распределение
агрегированного трафика $\{A_{N}(tT)$, $t \geq 0 \}$ с  ростом
сначала величин~$N_i$, а затем па\-ра\-мет\-ра~$T$ (этот порядок важен)
сближается с распределением процесса
$ %\begin{multline}
T\left( \sum\limits_{i=1}^n N_i {\mu_{\mathrm{on}}^i}/({\mu_{\mathrm{on}}^i\hm+\mu_{\mathrm{off}}^i})
\right)t \hm+ \sum\limits_{i=1}^n T^{H_i} \sqrt{L_i(T)N_i}c_i B_{H_i}(t)$,
%\end{multline}
где $c_i$~--- положительные константы,
 $L_i$~--- медленно меняющиеся на бесконечности функции,
 выраженные через исходные параметры, а $B_{H_i}$~--- независимые
ДБД с па\-ра\-мет\-ра\-ми  Херста $H_i$, определяемыми  как
$$
H_i=\fr{3-\min(\alpha_{\mathrm{on}}^i,\alpha_{\mathrm{off}}^i)}{2}\in
\left(\fr{1}{2},\,1 \right)\,,\enskip i=1,\ldots, n\,.
$$
Таким образом,  суммарный трафик, порожденный  большим числом
(независимых) источников, у которых распределения on/off-пе\-рио\-дов
имеют тяжелые хвосты, приближенно  описывается процессом, включающим
сумму  независимых ДБД. Данный результат обосновывает
большой интерес  к  моделям телекоммуникационных систем, на вход
которых поступает один   или несколько  независимых ДБД.

\vspace*{-2pt}

\section{Гауссовские очереди}

Вначале дадим  описание гауссовской системы обслуживания в целом, а
затем описание новой модели, изучаемой  в данной статье. Пусть
$A(t)$~--- суммарная работа, поступившая в систему за время $[0,t]$.
При анализе  гауссовских очередей  входной процесс обычно
задается в следующем виде:
\begin{equation}
A(t)=mt+\sigma X(t)\,, 
\label{asymp-l1}
\end{equation}
где  $m$, $\sigma $~--- положительные константы, а  $X:=$\linebreak $:=\;\{X(t),\ t\hm\geq 0\}$~---  
центрированный гауссовский процесс со стационарными
приращениями,  $X(0)\hm=0$~\cite{Mandjes}.
 Подчеркнем, что такая форма входного процесса  является
 общепринятой при описании гауссовской системы обслуживания, однако более полное описание случайной компоненты  $X$
 зависит от специфики  модели.  Будем считать, что система имеет одно обслуживающее устройство с постоянной
скоростью обслуживания~$C$, причем $r:=C\hm-m\hm>0$. Обозначим также
$W(t)\hm=\sigma X(t)\hm-rt$ и пусть
 $Q(t)$~--- величина нагрузки (незавершенная работа в системе) в момент времени~$t$. 
 Если $Q(0)\hm=0$, то для $Q(t)$ справедливо выражение~\cite{Reich}:
 
 \noindent
\begin{multline}
Q(t)= \sup\limits_{0 \leq s \leq t}(A(t)-A(s)-C(t-s))={}\\
{}= \sup\limits_{0 \leq s \leq t}(\sigma(X(t)-X(s))-r(t-s))={}\\
{}=\sup\limits_{0 \leq s \leq t}(W(t)-W(s))\,.
\label{e6a-ml}
\end{multline}
Заметим, что $ \e A(1)\hm=m$, поэтому  условие  $r\hm>0$ обеспечивает
существование стационарного про-\linebreak\vspace*{-12pt}

\pagebreak

\noindent
цесса нагрузки,  который определяется
следующим образом~\cite{Mandjes}:
\begin{multline}
Q= \sup\limits_{t \geq 0} \left( A(t)-Ct \right)= \sup\limits_{t \geq 0} \left( \sigma X(t)-rt \right)={}\\
{}= \sup\limits_{t \geq 0} W(t)\,.
\label{e6-ml}
\end{multline}
Хорошо известно~\cite{Mandjes} (и легко установить), что
ковариационная функция $\Gamma(s,t)$ процесса~$X$ имеет вид
\begin{equation*}
\Gamma(s,t)=\e\left [X(s)\,X(t)\right]=\fr{1}{2}\left[
v(t)+v(s)-v(|t-s|) \right]\,,
\end{equation*}
где $v(t)$ есть дисперсия~$X(t)$.

Основное предположение, принятое в данной статье, состоит в том, что
функция~$v(t)$ {\it правильно меняется  на бесконечности c
индексом} $0\hm<V\hm<2$, т.\,е.\ для любого $y\hm>0$
$$
\lim\limits_{t \to \infty} \fr{v(yt) }{v(t)}=y^V\,.
$$
Известно,  что любая правильно меняющаяся на бесконечности функция
с индексом~$V$ может быть представлена в виде
\begin{equation*}
v(t)=t^V L(t)\,,
\end{equation*}
где $L(t)$~--- медленно меняющаяся на беско\-неч\-ности функция~\cite{Seneta}. 
Обозначим $\beta\hm={1}/(2-V)$, а также выберем и
зафиксируем любое $\varepsilon \hm\in (0,2-V)$. Будем далее считать,
что функция $L(t)$ является {\it дважды дифференцируемой} на~$\mathbb{R}_+$. 
(Вообще говоря, достаточно, чтобы это условие было
выполнено  на сколь угодно удаленном от начала координат луче
$[a,\infty)$, $a\hm>0$.) Кроме того, предположим, что также
выполнены следующие условия (при $t \hm\to \infty$):
\begin{align}
L(tL^\beta(t))& \sim L(t)\,,\label{e10-ml}\\
L''(t)&=o\left( \fr{1}{t^{V+\varepsilon}} \right)\,.\label{e11a-ml}
\end{align}
Из условия~(\ref{e11a-ml}) следует (с использованием правила Лопиталя),
что $L'(t)\hm=o\left(t^{-V-\varepsilon+1} \right)$. Отметим, что класс
медленно меняющихся функций, удовлетворяющих условиям~(\ref{e10-ml}) и~(\ref{e11a-ml}), 
достаточно обширен и, в частности, включает функции,
имеющие (на бесконечности) конечный ненулевой предел, а также
функции вида $(\ln t)^a$, $(\ln\ln t)^a$, $a\hm\ge 0$ и~т.\,д.

Из~(\ref{e6-ml}) следует, что  вероятность превышения процессом
стационарной нагрузки некоторого  уровня  $b\hm>0$ определяется
следующим образом:
\begin{equation*}
\mathbb{P}( Q>b) = \mathbb{P}\left(\sup\limits_{t \geq 0}W(t)>b\right)\,.
\end{equation*}
В~работе~\cite{Duffy} показано, что для центрированного
гауссовского процесса со стационарными приращениями и с дисперсией~$v(t)$, 
правильно меняющейся на бесконечности  с индексом $0\hm<V\hm<2$,
справедлива такая (логарифмическая) асимптотика:
\begin{equation}
\lim\limits_{b \to \infty} \fr{v(b)}{b^2} \ln \mathbb{P}(Q>b)=-\theta\,,
\label{asymp1-l13}
\end{equation}
где параметр $\theta>0$ имеет   вид
\begin{equation}
\theta=\fr{2}{\sigma^2(2-V)^{2-V}}\left( \fr{r}{V}
\right)^V\,.
\label{e11-ml}
\end{equation}
Кроме того, в~\cite{Konstantopoulos} показано, что на одном
веро\-ят\-ностном пространстве можно задать процесс $W(t)\hm=\sigma X(t)\hm-rt$ и
стационарный процесс $Q^*:=$\linebreak $:=\;\{Q^*(t),\, t\hm \geq 0\}$ таким образом,
что одновременно выполнены условия
\begin{align}
Q^*(t)&=_d Q \mbox{ для всех } t \geq 0\,,\label{e15-ml}\\
Q^*(t)&=W(t)+\max\{Q^*(0), L^*(t)\},\,\, t \geq 0\,,\label{e16-ml}
\end{align}
где $=_d$ означает равенство по распределению, а $L^*(t)\hm=-\min\limits_{0\le s\le t}\{W(s)\}$. 
Обозначим
\begin{equation*}
M(t)=\max\limits_{0 \leq s \leq t}Q(s),\;\; M^*(t)=\max\limits_{0 \leq s \leq
t}Q^*(s)\,. 
%\label{e13-ml}
\end{equation*}
Таким образом, $M^*(t)$  есть  максимум стационарного процесса
нагрузки $Q^*$ (удовлетворяющего условиям~(\ref{e15-ml}) и (\ref{e16-ml})), а
$M(t)$~--- максимум исходного (нестационарного) процесса нагрузки~(\ref{e6a-ml}) 
на интервале $[0,\,t]$. Далее будем изучать
асимптотическое (при $t\hm\to \infty$) поведение этих максимумов.

\section{Асимптотика  максимума процесса нагрузки}

В данном разделе сформулирован и доказан основной результат об
асимптотическом поведении  максимумов $M^*(t),\,M(t)$ при $t\hm\to \infty$. 
Этот результат   обобщает  работу~\cite{Zeevi}, где процесс
$X\hm=B_H$, т.\,е.\  является  ДБД  c параметром Херста  $H\hm\in (1/2,\,1)$. 
В~част\-ности, для модели из~\cite{Zeevi} соотношения~(\ref{asymp1-l8}) и~(\ref{asymp1-l9}) 
ниже  выполняются при $V\hm=2H$ (см.~(\ref{e11-ml})) и  $\gamma(t)\hm=\ln t.$
 Подчеркнем, что доказательство в данной статье в целом следует подходу,
использованному в~\cite{Zeevi}.

Для удобства обозначим далее
\begin{equation*}
\gamma(t)=L\left[\left(\ln t
\right)^\beta\right]  \ln t \,.
\end{equation*}

Справедлива следующая теорема.


\medskip

\noindent
\textbf{Теорема~3.1.}
\textit{Пусть дисперсия гауссовской компоненты~$X$ входного  процесса}~(\ref{asymp-l1}) 
\textit{удовлетворяет условиям}~(\ref{e10-ml}) и~(\ref{e11a-ml}), \textit{а
также $r\hm>0$. Тогда}
\begin{align}
\fr{M^*(t)}{\gamma^\beta(t)} & \Rightarrow
\left(\fr{1}{\theta}\right)^\beta\,,\enskip t \to \infty\,;
\label{asymp1-l8}\\
\fr{M(t)}{\gamma^\beta(t)} &\Rightarrow
\left(\fr{1}{\theta}\right)^\beta\,,\enskip t \to \infty\,,
\label{asymp1-l9}
\end{align}
где параметр $\theta$ удовлетворяет соотношению~(\ref{e11-ml}), а знак
$\Rightarrow$ означает сходимость по вероятности.

\medskip


\noindent
Д\,о\,к\,а\,з\,а\,т\,е\,л\,ь\,с\,т\,в\,о\,.\
Для доказательства~(\ref{asymp1-l8}) достаточно показать, что для
любого $\delta\hm>0$ выполняются следующие два соотношения:
\begin{align}
\mathbb{P} \left( \fr{M^*(t)}{\gamma^\beta(t)} \geq \left(
\fr{1-\delta}{\theta} \right)^\beta \right ) & \to 1\,,\enskip t \to
\infty \,; \label{asymp1-l1}\\
\mathbb{P} \left( \fr{M^*(t)}{\gamma^\beta(t)} \geq \left(
\fr{1+\delta}{\theta} \right)^\beta \right) &\to 0\,,\enskip t \to
\infty\,. \label{asymp1-l2}
\end{align}
Действительно,
$$
\fr{(1\pm \delta)^\beta}{{\theta}^\beta}=\fr{1}{{\theta}^\beta}\pm
\fr{\beta \delta}{{\theta}^\beta}+o(\delta)\,,\enskip  \delta \to 0\,.
$$
Поскольку $\epsilon :=\beta \delta\theta^{-\beta}\hm+o(\delta)\hm\to 0$   
при  $\delta\hm\to 0$, то~(\ref{asymp1-l1}) и~(\ref{asymp1-l2})
означают, что для любого $\epsilon\hm>0$ выполнены соответственно условия
\begin{align*}
\mathbb{P} \left(\fr{M^*(t)}{\gamma^\beta(t)}-\left(
\fr{1}{\theta}\right)^\beta > -\epsilon \right ) & \to 1\,,\enskip t \to \infty\,;\\
\mathbb{P} \left(\fr{M^*(t)}{\gamma^\beta(t)}-\left(
\fr{1}{\theta}\right)^\beta > \epsilon \right ) &\to 0\,,\enskip t \to \infty\,,
\end{align*}
которые вместе эквивалентны~(\ref{asymp1-l8}). Докажем вначале
соотношение~(\ref{asymp1-l1}). Для этого возьмем некоторое (пока
произвольное) $\Delta \hm\in (0,t)$. (Ниже величина $\Delta$ будет
выбрана специальным образом, зависящим от~$t$.) В~силу свойства~(\ref{e16-ml}) имеем
\begin{multline}
Q^*(t)=W(t)+\max\{Q^*(0), L^*(t)\}\geq{}\\
{}\geq W(t) - \inf_{0 \leq s \leq t} W(s)\geq W(t)-W(t-\Delta)\,.
\label{e18a-ml}
\end{multline}
Это дает  следующую цепочку неравенств:
\begin{multline*}
M^*(t)= \max\limits_{0 \leq s \leq t} Q^*(s)
\geq \max\limits_{k=1,\ldots,\lfloor t/\Delta \rfloor} Q^*(k \Delta)\geq{}\\
{}\geq \max\limits_{k=1,\ldots,\lfloor t/\Delta \rfloor}
[W(k\Delta)-W((k-1)\Delta)]:={}\\
{}:=\max\limits_{1\leq k \leq \lfloor t/\Delta \rfloor}Y_k^{(\Delta)}\,,
\end{multline*}
где $Y_k^{(\Delta)}:=W(k\Delta)\hm-W((k-1)\Delta)$, а $\lfloor x
\rfloor$ есть  наибольшее целое, не превосходящее~$x$. Таким
образом, справедливо неравенство
\begin{multline}
\mathbb{P} \left( \fr{M^*(t)}{\gamma^\beta(t)} \geq\left(
\fr{1-\delta}{\theta}\right)^\beta \right) \geq{}\\
{}\geq \mathbb{P} \left(
\vphantom{\left(\fr{1-\delta}{theta}\right)^\beta}
\max\limits_{i=1,\ldots ,\lfloor t/\Delta \rfloor}Y_i^{(\Delta)} \geq
\left(\fr{1-\delta}{\theta}\gamma (t)\right)^\beta\right)\,.
\label{asymp1-l4}
\end{multline}
Напомним, что  $W(t)=\sigma X(t)-rt$. Поэтому
\begin{multline*}
Y_k^{(\Delta)}= W(k\Delta)-W((k-1)\Delta)=_d{}\\
{}=_d \sigma X(\Delta)-r\Delta =_d \sigma \sqrt{v(\Delta)}\,\mathcal{N} (0,1)-r\Delta\,,
\end{multline*}
где $\mathcal{N} (0,1)$~--- нормальная случайная величина (с.\,в.). Поскольку
\begin{equation}
Z_i :=\fr{Y_i^{(\Delta)}+r\Delta }{\sigma \sqrt{v(\Delta)}}=_d
\mathcal{N}(0,1)\,,\label{e19-ml}
\end{equation} 
то $\{Z_i,\,i=1,\ldots,\lfloor t/\Delta \rfloor\}$  
является  стационарной последовательностью нормальных
с.\,в. Кроме того, для автоковариационной функции этой
последовательности получаем
\begin{multline}
\rho(k):=\mathbb{C}\mathrm{ov}(Z_1,Z_{1+k})
={}\\
{}=\fr{1}{\sigma^2 v(\Delta)}\mathbb{C}\mathrm{ov} \left( Y_1^{(\Delta)},Y_{1+k}^{(\Delta)} \right)={}\\
{}=\fr{1}{v(\Delta)}\mathbb{C}\mathrm{ov} \left(X(\Delta),\;X\left((k+1)\Delta\right)-X(k\Delta) \right)={}\\
{}=\fr{1}{v(\Delta)}\left[
\Gamma\left(\Delta,(k+1)\Delta\right)-\Gamma(\Delta,k\Delta) \right]=
\fr{1}{2v(\Delta)}\times{}\\
{}\times\left[ v\left( (k+1)\Delta
\right)-2v(k\Delta)+v\left( (k-1)\Delta\right) \right].\label{e21-ml}
\end{multline}
Из  формулы конечных приращений и~(\ref{e21-ml}) следует, что
$$
\rho(k)=\fr{\Delta}{2v(\Delta)}v''(u_3)(u_1-u_2)
$$
для некоторых $u_1\hm\in (k \Delta,\,(k+1)\Delta)$, $u_2\hm \in((k\hm-1)\Delta,\,k\Delta)$, 
$u_3\hm\in (u_2,\,u_1)$. (Вообще говоря,
величины~$u_i$  зависят от~$k$.) Легко проверить, что
\begin{multline}
v''(k)=k^VL''(k)+2Vk^{V-1}L'(k)+{}\\
{}+V(V-1)k^{V-2}L(k). 
\label{21'}
\end{multline}
Далее, с учетом  свойства~(\ref{e11a-ml})  легко показать, что при $k\hm\to \infty$
\begin{equation}
k^V \ln k \,L''(k)\to 0\,;\enskip  k^{V-1} \ln k\, L'(k)\to 0\,.
\label{asymp1-l14}
\end{equation}
Поскольку  $V<2$, то также
\begin{equation}
k^{V-2}\ln k \,L(k) \to 0\,. 
\label{asymp1-l16}
\end{equation}
Таким образом,  из (\ref{e21-ml})--(\ref{asymp1-l16})  следует, что
\begin{equation}
\rho(k)\ln k \to 0\,,\enskip k \to \infty\,. 
\label{asymp1-l3}
\end{equation}
Теперь воспользуемся леммой из~\cite{Leadbetter}, которая в
адап\-та\-ции к рассматриваемой ситуации примет следующий вид:

\smallskip

\noindent
\textbf{Лемма~1.} \textit{Пусть имеется стационарная последовательность
$\{Z_i\}_{i=1}^{m}$ стандартных нормальных с.\,в.\ с ковариационной
функцией, удовлетворяющей соотношению}~(\ref{asymp1-l3}). \textit{Тогда для
любой последовательности действительных чисел $u_m$ условие}

\vspace*{-2pt}

\noindent
\begin{eqnarray*}
\mathbb{P} \left( \max\limits_{i=1,\ldots,m} Z_i \geq u_m\right) \to 1\,,\enskip m \to \infty\,,
%\label{e24-ml}
\end{eqnarray*}
\textit{выполнено тогда и только тогда, когда}
\begin{equation}
\lim_{m\to \infty} m\,\mathbb{P}(Z>u_m)  = \infty \,. 
\label{e25-ml}
\end{equation}

\smallskip

Далее покажем, что можно выбрать  $m:=m(t)$ и $u_{m(t)}:=u(t)$
таким образом, что $m(t)\hm\to \infty$ при $t\hm\to \infty$ и условие~(\ref{e25-ml}) 
оказывается выполненным. Прежде всего положим
$\Delta:=\Delta(t)\hm=A\gamma^\beta(t)$, где  $A>0$~--- некоторая
постоянная.  (Таким образом, величина $\Delta(t)$ растет  вместе с~ $t$.)  
Далее  будет доказано, что  постоянную~$A$ можно выбрать так, чтобы обеспечить 
указанную выше сходимость.

Поскольку любая медленно меняющаяся функция растет медленнее
степенной, то существует   такое~$t_0$, что $\Delta(t)\hm<t$ при всех
$t\hm\ge t_0$. (Разумеется, $t_0$ зависит от параметров, определяющих
$\Delta(t)$.) Поэтому далее (где это требуется) предполагается, что
$t\hm\ge t_0$.  Фиксируем произвольное $\delta \hm\in (0,1)$ и введем обозначения

\vspace*{-2pt}

\noindent
\begin{align*}
\alpha(t)&=\left( \fr{1-\delta}{\theta}\,\,\gamma (t) \right)^\beta\,;\\
u(t)&= \fr{ \alpha(t)+r\Delta(t)}{\sigma\sqrt{v(\Delta(t))}}\,;\\
\tau(t)&= m(t)\mathbb{P}(Z_1>u(t))\,,
\end{align*}
где $m(t):=\lfloor t/\Delta(t)\rfloor$. Заметим, что ввиду~(\ref{asymp1-l4}) и~(\ref{e19-ml})

\vspace*{-2pt}

\noindent
\begin{multline}
\mathbb{P} \left( \max\limits_{i=1,\ldots, m(t)} Y_i^{(\Delta)} \geq \alpha(t)
\right) = {}\\
{}=\mathbb{P} \left( \max\limits_{i=1,\ldots ,m(t)} Z_i \geq u(t)
\right)\,.
\label{e27-ml}
\end{multline}
Для удобства дальнейшего анализа выпишем связь параметров
\begin{equation*}
\beta=\fr{1}{2-V}\,;\enskip \beta-\fr{\beta V}{2}=\fr{1}{2}\,;\enskip
\fr{\beta V +1}{2}=\beta\,,
\end{equation*}
а также введем обозначение  $(\ln t)^\beta\hm=\phi(t)$.
 Используя свойство~(\ref{e10-ml}), можно получить следующие соотношения:
 
 \noindent
\begin{multline*}
u(t) =  \fr{ \alpha(t)+r\Delta(t) }{\sigma\sqrt{v(\Delta(t))}}=\fr{1}{\sigma}\times{}\\
{}\times\fr{\left[((1-\delta)/\theta)^\beta+rA\right] \phi(t)\,
\left[L(  \phi(t))\right]^\beta}{\sqrt{ A^V \left[\phi(t)\right]^ V \left[L ( \phi(t)
)\right]^{\beta \,V}\,L\left( A \phi(t)\,\left[L(
\phi(t))\right]^\beta\right )}}={}\\
{}= \fr{\left[((1-\delta)/\theta)^\beta+rA\right]
\phi(t) \left[L(\phi(t))\right]^\beta} {\sigma A^{{V}/{2}}
\left[\phi(t)\right]^{{V}/{2}}\left[L(\phi(t))\right]^{(\beta V+1)/2}}\,
\left(1+o(1)\right)={}\\
{}=C(A)\,\sqrt{\ln t}\,\left(1+o(1)\right)\,,\enskip t \to \infty\,,
\end{multline*}
где использовано обозначение

\noindent
\begin{multline*}
C(A)=\fr{(1-\delta)^\beta}{\sigma\theta^\beta}\,
A^{-{V}/{2}}+\fr{r}{\sigma}\,A^{1-{V}/{2}}:={}\\
{}:=C_1\,A^{-{V}/{2}}+C_2 A^{1-{V}/{2}}\,,
%\label{asymp1-l10}
\end{multline*}
а также

\noindent
\begin{equation}
C_1=\fr{(1-\delta)^\beta}{\sigma{\theta}^\beta}\,;\quad C_2=\fr{r}{\sigma}\,.
\label{e29a-ml}
\end{equation}
Несложно проверить, что справедлива следующая оценка для вероятности
большого уклонения стандартной нормальной с.\,в.\ (см., например,~\cite{Lifshits}):

\noindent
\begin{multline*}
\mathbb{P} (Z_1 > u(t))=\fr{1}{\sqrt{2\pi}}\int\limits_{u(t)}^\infty
e^{-x^2/2}dx\geq{}\\
{}\geq \fr{1-u^{-2}(t)}{u(t)\sqrt{2\pi}}\,e^{-u^2(t)/2}\sim{}\\
{}\sim
\fr{1}{u(t)\sqrt{2\pi}}\,e^{-u^2(t)/2}\,,\enskip t \to \infty\,.
\end{multline*}
Далее с учетом того, что $u(t) \hm= C(A) \sqrt{\ln t}\,\left(1+o(1)\right)$,
при достаточно больших~$t$ получим:

\noindent
\begin{multline}
\tau(t)=m(t)\mathbb{P}(Z_1>u(t))\geq{}\\
{}\geq m(t)\fr{1-u^{-2}(t)}{u(t)\sqrt{2\pi}}\,e^{-u^2(t)/2}\sim{}\\
{}\sim t\left[\sqrt{2\pi}\,\Delta(t)\,u(t)\,e^{u^2(t)/2}\right]^{-1}\sim{}\\
{}\sim t \left[\sqrt{2\pi}\,A\,\gamma^\beta(t) \, C(A)\left(\ln t\right)^{1/2}\, t^{C^2(A)/2+o(1)}\right]^{-1}\sim{}\\
{}\sim t^{1-({C^2(A)})/{2}+o(1)}\left[L_1(t)\right]^{-1}:=g(t)\,,\label{e30-ml}
\end{multline}
где функция
$$
L_1(t):=\sqrt{2\pi}\,A\, C(A)(\ln t)^{\beta+1/2}\,[L(\phi(t))]^\beta
$$  
медленно меняется на бесконечности. Из~(\ref{e30-ml}) следует, что  если можно выбрать
$C(A)\hm<\sqrt{2}$, то будет иметь место  $g(t)\hm\to \infty$, $t\hm\to \infty$. 
Покажем, что та-\linebreak\vspace*{-12pt}

\pagebreak

\noindent
кой выбор  действительно возможен. Обозначим
$u\hm=V/2$, рассмотрим функцию
\begin{equation*}
f(x)=C_1 x^{-u}+C_2 x^{1-u} %\label{e32-ml}
\end{equation*}
и заметим, что $f(A)\hm=C(A)$. Легко убедиться, что функция $f(x)$
достигает минимума в точке
$$
x_*=\fr{C_1 u}{C_2(1-u)}\,,
$$
причем с учетом~(\ref{e29a-ml})
$$
f(x_*)=\fr{C_1^{1-u}C_2^u}{u^u\,(1-u)^{1-u}}=\sqrt{(1-\delta)2}<\sqrt{2}\,.
$$
Таким образом, выбор $A=x_*$ обеспечивает требуемую сходимость и
 ввиду~(\ref{e30-ml})
$$
\tau(t)\ge g(t) \to \infty\,,\enskip t \to \infty\,.
$$
Поэтому с учетом леммы~1 и соотношений~(\ref{asymp1-l4}), (\ref{e27-ml}) следует 
утверждение~(\ref{asymp1-l1}):
\begin{multline*}
\mathbb{P}\left(\fr{M^*(t)}{\gamma^\beta(t)} \geq
 \left(
\fr{1-\delta}{\theta} \right)^\beta \right)\geq{}\\
{}\geq \mathbb{P}\left(
\max\limits_{i=1,\ldots,m(t)}Z_i \geq u(t) \right) \to 1\,,\enskip t \to \infty\,.
\end{multline*}
Теперь обратимся к доказательству соотношения~(\ref{asymp1-l2}).
Положим
$$
Y_i=\sup\limits_{s \in [i-1,i)} Q^*(s)\,,\enskip i=1,\ldots,\lceil t \rceil\,,
$$
где $\lceil x \rceil$~--- наименьшее целое число, превы\-ша\-ющее~$x$.
Очевидно, что
$$
M^*(t) \leq \max\{Y_i:\,\,i=1,\ldots,\lceil t \rceil\}\,.
$$
Ввиду  стационарности  процесса $Q^*$ с.\,в.\ $\{Y_i\}$ одинаково
распределены (как некоторая с.\,в.~$Y$).  Далее зафиксируем
произвольное $\delta\hm>0$ и заметим, что
\begin{multline}
\mathbb{P} \left( M^*(t) \geq \left(\fr{1+\delta}{\theta}\right)^\beta
\gamma^\beta(t) \right) \leq{}\\
{}\leq \mathbb{P} \left( \max_{i=1,\ldots,\lceil t \rceil} Y_i \geq 
\left(\fr{1+\delta}{\theta}\right)^\beta
\gamma^\beta(t)  \right)\leq{}\\
{}\leq \lceil t \rceil \mathbb{P} \left(Y \geq \left(\fr{1+\delta}{\theta}\right)^\beta 
\gamma^\beta(t) \right)\,. 
\label{asymp1-l6}
\end{multline}
Для последующего анализа необходимо оценить хвост распределения
с.\,в.~$Y$. Для этого заметим, что верны следующие соотношения:
\begin{multline*}
Y =_d\max\limits_{s \in [0,1)} \left[ \vphantom{\max\limits_{s \in [0,1)}}
W(s)+{}\right.\\
\left.{}+\max\left\{Q^*(0),\,
-\min\limits_{0 \leq \tau \leq s}W(\tau)   \right\}\right]={}
\end{multline*}

\noindent
\begin{multline}
{}=\max\limits_{s \in [0,1)} \left[ \max\left\{  
\vphantom{\max\limits_{s \in [0,1)}}
W(s)+Q^*(0),\,\,W(s)-{}\right.\right.\\
\left.\left.{}-\min\limits_{0\leq \tau \leq s}W(\tau) \right\} \right]\leq{}\\
{}\leq \max\limits_{s \in [0,1]} \left[ Q^*(0)+W(s)-\min_{0 \leq \tau \leq s}W(\tau) \right]
\leq{}\\
{}\leq \max\limits_{s \in [0,1]} \left[ Q^*(0)+W(s)-\min_{0 \leq \tau
\leq 1}W(\tau) \right]={}\\
{}= Q^*(0)+\max\limits_{0 \leq s \leq 1}W(s)-\min_{0 \leq s \leq 1} W(s)\,.
\label{asymp1-l5}
\end{multline}
Для получения первого неравенства выше был использован тот факт,
что для любого~$a$ и любых~$b,\,c\hm\ge 0$ имеет место неравенство
$\max\{a+b,a+c\}\hm\leq a+b+c$.

Приведем вспомогательную лемму, которая понадобится при уточнении
оценки  хвоста распределения с.\,в.~$Y$. (Простое доказательство этой
леммы опущено.)

\medskip

\noindent
\textbf{Лемма~2.} \textit{Для двух неотрицательных с.\,в. $\xi$, $\eta$ и любых чисел $0\hm<a\hm<b$
справедливо неравенство}
$$
\mathbb{P} (\xi+\eta \geq b) \leq \mathbb{P} (\xi \geq b)+ \mathbb{P} (\eta \geq b-a)\,.
$$

\smallskip

Вернемся к доказательству теоремы. Запишем
$(1\hm+\delta)^\beta\hm=(1\hm+\delta/2)^\beta\hm+R(\delta)$ и обозначим
$R_1:=R(\delta)/(2{\theta}^\beta)\hm>0$. Введем также  обозначения
\begin{align*}
Q_1&:=\mathbb{P} \left(Q^*(0) \geq \left(\fr{1+\delta/2}{\theta}\right)^\beta \gamma^\beta(t)\right)\,;
\\
Q_2&:=\mathbb{P} \left(\max_{0 \leq s \leq 1}W(s) \geq R_1 \gamma^\beta(t) \right)\,;
\\
Q_3&:=\mathbb{P} \left(-\min_{0 \leq s \leq 1}W(s) \geq R_1 \gamma^\beta(t) \right)
\end{align*}
для хвостов функций распределения (случайных) слагаемых в формуле~(\ref{asymp1-l5}).
 Из неравенства~(\ref{asymp1-l5}) и леммы~2 следует цепочка неравенств
\begin{multline}
\mathbb{P} \left( Y \geq \left( \fr{1+\delta}{\theta}
\right)^\beta\gamma^\beta(t)\right)\leq{}\\
{}\leq \mathbb{P} \left( Q^*(0)+\max\limits_{0 \leq s \leq 1}W(s)-\min\limits_{0 \leq s
\leq 1}W(s)  \geq{}\right.\\
\left.{}\geq \left( \fr{1+\delta}{\theta}
\right)^\beta\gamma^\beta(t)\right)
\leq Q_1+Q_2+Q_3\,.\label{asymp1-l7}
\end{multline}
В свою очередь, из соотношений~(\ref{asymp1-l6}) и~(\ref{asymp1-l7})
следует неравенство
\begin{multline*}
0 \leq \mathbb{P}\left(M^*(t) \geq \left( \fr{1+\delta}{\theta}
\right)^\beta \gamma^\beta(t)  \right) \leq{}\\
{}\leq \lceil t \rceil
(Q_1+Q_2+Q_3)\,.
\end{multline*}
Чтобы получить неравенство~(\ref{asymp1-l2}), докажем, что
 $\lceil t \rceil Q_i \hm\to 0$, $t \hm\to \infty$,
$i\hm=1$, 2,~3.

\pagebreak

 Начнем с анализа $Q_2$, для чего заметим, что
\begin{equation}
\hspace*{-1mm}\max\limits_{0 \leq s \leq 1}W(s)=\max\limits_{0 \leq s \leq 1}(\sigma X(s)-rs)
\leq \max\limits_{0 \leq s \leq 1}\sigma X(s). \!\!
\label{asymp1-l11}
\end{equation}
Пусть $a=\e \max\limits_{0 \leq s \leq 1}X(s)<\infty$. Теперь
воспользуемся следствием из неравенства  Бо\-ре\-ля--Су\-да\-ко\-ва--Ци\-рель\-со\-на
для максимума на конечном интервале  центрированного гауссовского процесса со
стационарными приращениями~\cite{Adler}:
\begin{equation}
\hspace*{-1.5mm}\mathbb{P} \left( \max\limits_{0 \leq s \leq 1} X(s)> x \right) \leq
e^{-(1/(2v))(x-a)^2}, \ \ x>a,\!
\label{asymp1-l12}
\end{equation}
где дисперсия $v:=D X(1)$. Введем обозначение
$\lambda(t)\hm=R_1\,\gamma^\beta(t)\,\sigma^{-1}$. Из соотношений~(\ref{asymp1-l11})
и~(\ref{asymp1-l12}) следует, что при всех
достаточно больших~$t$
\begin{multline*}
\lceil t \rceil Q_2 \le  (t+1)\mathbb{P} \left( \max_{0 \leq s \leq 1}
X(s) \geq \lambda(t) \right)\leq{}\\
{}\leq (t+1)\,t^{-({\lambda^2(t)}/{\ln
t})\cdot (1+o(1))/(2v)} :=g_1(t)\,.
\end{multline*}
Известно~\cite{Seneta}, что   $t^\varepsilon L(t)\hm\to \infty $ при
$t\hm\to \infty$ для любого $\varepsilon\hm>0$, и поэтому
$L(t)\hm>t^{-\varepsilon}$ для всех  достаточно больших~$t$. Также
заметим, что $2\beta-1={V}/(2-V)\hm>0$ и выберем произвольное
$\varepsilon\hm\in \left(0,\,(2\beta-1)/(2\beta^2)\right)$. С~учетом этого
получаем, что при $t\hm\to \infty$
\begin{multline*}
\fr{\lambda^2(t)}{\ln t}=\left( \fr{R_1}{\sigma} \right)^2(\ln t)^{2\beta-1}
\left[L\left((\ln t)^\beta\right)\right] ^{2\beta}>{}\\
{}> \left( \fr{R_1}{\sigma} \right)^2 [\ln t]^{2\beta-1-2\beta^2
\varepsilon}\to \infty\,.
\end{multline*}
Тогда легко увидеть, что  $g_1(t) \hm\to  0$ и, значит, $ \lceil t
\rceil Q_2 \hm\to 0$, $t \hm\to \infty.$

 Теперь рассмотрим слагаемое~$Q_3$ и заметим, что
\begin{multline*}
-\min\limits_{0 \leq s \leq 1} W(s)=-\min\limits_{0 \leq s \leq 1} (\sigma
X(s)-rs)=_d{}\\
{}=_d\max\limits_{0 \leq s \leq 1} (rs+\sigma X(s))\leq
r+\sigma \max\limits_{0 \leq s \leq 1} X(s)\,,
\end{multline*}
поскольку для  центрированного гауссовского процесса $X(s)=_d -X(s)$.
Поэтому справедливо неравенство
$$
Q_3 \leq \mathbb{P} \left(\max\limits_{0 \leq s \leq 1}X(s) \geq
\fr{R_1}{\sigma}\, \gamma^\beta(t) - \fr{r}{\sigma}\right)\,.
$$
Расcуждая далее, как и при анализе $Q_2$, получим, что
$$
\lceil t \rceil Q_3 \to 0\,,\enskip t \to \infty\,.
$$

Осталось показать, что $\lceil t \rceil Q_1 \to 0$, $t \hm\to \infty$. 
Это эквивалентно тому, что
$$
a(t):=\ln \lceil t \rceil + \ln Q_1 \to -\infty\,.
$$
 Из асимптотики для вероятности переполнения~(\ref{asymp1-l13}) следует, что
\begin{equation}
b^{V-2}L(b)\ln \mathbb{P}(Q^*(0)>b)\to   -\theta\,,\enskip b\to \infty\,.
\label{asymp1-l17}
\end{equation}
В~рассматриваемом случае возьмем
$$
b=b(t)=\left( \fr{1+{\delta}/{2}}{\theta} \right)^\beta[\gamma(t)]^\beta\,.
$$
Заметим, что тогда
\begin{equation}
b^{V-2}=\fr{\theta}{1+{\delta}/{2}}\, \fr{1}{\gamma(t)}\,. 
\label{asymp1-l20}
\end{equation}
Используя условие~(\ref{e10-ml}), из~(\ref{asymp1-l17})
и~(\ref{asymp1-l20}) несложно получить, что
\begin{multline*}
\lim\limits_{t \to \infty} \fr{\ln \mathbb{P} \left( Q^*(0) \geq \left(
(1+\delta/2)/{\theta}\right)^\beta \gamma^\beta(t)
\right)}{\ln t}={}\\
{}=-\left(1+\fr{\delta}{2}\right)\,.
\end{multline*}
Следовательно, при $t \to \infty$
\begin{multline*}
a(t) \sim \ln t + \ln Q_1={}\\
{}= \ln t\! \left( 1+\fr{\ln \mathbb{P} \left( Q^*(0) \geq \left(\!\!
(1+\delta/2)/{\theta}\right)^\beta \gamma^\beta(t) \right)}{\ln t}\!\!\right)\to{}\\
{}\to -\infty \,.
\end{multline*}

Итак, доказательство соотношения~(\ref{asymp1-l8}) для $M^*(t)$ завершено.

\smallskip

Обратимся к   доказательству соотношения~(\ref{asymp1-l9}). С~учетом предыдущего анализа для этого достаточно сделать лишь
несколько пояснений. Во-пер\-вых, как и выше достаточно показать,
что
\begin{align}
\mathbb{P} \left( \fr{M(t)}{\gamma^\beta(t)} \geq \left(
\fr{1-\delta}{\theta} \right)^\beta \right ) &\to 1\,,\enskip t \to \infty \,; \label{asymp1-l18}
\\
\mathbb{P} \left( \fr{M(t)}{\gamma^\beta(t)} \geq \left(
\fr{1+\delta}{\theta} \right)^\beta \right) &\to 0\,,\enskip t \to \infty. \label{asymp1-l19}
\end{align}
 Для доказательства нижней границы~(\ref{asymp1-l18}) заметим, что
\begin{multline*}
Q(t)=\sup\limits_{0\leq s \leq t}[W(t)-W(s)]={}\\
{}=W(t)-\inf\limits_{0\leq s \leq t} W(s)\geq
 W(t)-W(t-\Delta)\,.
\end{multline*}
Далее все выкладки остаются без изменения с заменой лишь $M^*(t)$ на~$M(t)$.

Для доказательства верхней границы~(\ref{asymp1-l19}) заметим,
что в силу начального условия $Q(0)\hm=0$ имеем $Q(t)\hm \leq Q^*(t)$
и, следовательно, $M(t)\hm \leq M^*(t)$. В~этом случае справедливо неравенство
\begin{multline*}
\mathbb{P}\left( \fr{M(t)}{\gamma^\beta(t)}\geq
\left(\fr{1+\delta}{\theta}\right)^\beta\right) \leq{}\\
{}\leq \mathbb{P}\left(
\fr{M^*(t)}{\gamma^\beta(t)}\geq
\left(\fr{1+\delta}{\theta}\right)^\beta\right) \to 0\,,\enskip  t\to \infty\,.
\end{multline*}
Таким образом, соотношения~(\ref{asymp1-l18}) и~(\ref{asymp1-l19}),
а значит, и сходимость~(\ref{asymp1-l9})  доказаны.~\hfill~$\square$

\medskip


\noindent
\textbf{Замечание.}
Если $L(t) \to c\in (0,\,\infty)$, $t \hm\to \infty$,  то
$\gamma(t)\sim c \ln t$ и нормировка в утверждении теоремы~3.1
становится более простой, а~именно: формулы~(\ref{asymp1-l8})
и~(\ref{asymp1-l9}) принимают соответственно вид
$$
\fr{M^*(t)}{(c\ln t)^\beta} \Rightarrow
\left(\fr{1}{\theta}\right)^\beta\,,\enskip \fr{M(t)}{(c\ln
t)^\beta} \Rightarrow
\left(\fr{1}{\theta}\right)^\beta\,,\enskip t \to \infty\,.
$$

\smallskip

Доказанная теорема позволяет непосредственно  получить асимптотику
максимума процесса нагрузки для специального важного случая, когда
стохастическая компонента входного процесса является суммой
независимых ДБД, т.\,е.\
\begin{equation}
X(t)=\sum\limits_{i=1}^n B_{H_i}(t)\,,\enskip t\ge 0\,,
\label{e42-ml}
\end{equation}
где параметры Херста $H_i\hm\in (0,\,1)$.
 Без ограничения общности будем считать, что
$H_1\hm>\max\limits_{i>1}H_i$. Тогда  дисперсия $v(t)$  процесса $\{X(t)\}$  имеет вид
$$
v(t)=\sum\limits_{i=1}^n t^{2H_i}=t^{2H_1}L(t)\,,
$$
где медленно меняющаяся функция $ L(t)\hm=1\hm+\sum\limits_{i>1}
t^{2(H_i-H_1)}\hm\to 1$, $t \hm\to \infty. $ Таким образом,  дисперсия~$v(t)$ 
является правильно меняющейся на бесконечности функцией с
показателем $V\hm=2H_1\hm\in (0,\,2)$. Обозначим $ (2-2H_1)^{-1}\hm=\delta$.
Тогда в силу замечания выше имеет место такое

\smallskip

\noindent
\textbf{Следствие.}
\textit{Если компонента $X$  входного процесса имеет вид}~(\ref{e42-ml}), \textit{то}
$$
\fr{M^*(t)}{(\ln t)^\delta} \Rightarrow
\left(\fr{1}{\theta}\right)^{\delta}\,,\enskip \fr{M(t)}{(\ln
t)^\delta} \Rightarrow
\left(\fr{1}{\theta}\right)^\delta\,,\enskip t \to \infty\,.
$$

\smallskip

Данный результат говорит о том, что в асимптотическом анализе
максимума процесса нагрузки доминирующую роль играет ДБД с
наибольшим значением параметра Херста.



\section{Заключение}

В данной статье исследовано асимптотическое поведение максимума
 процесса нагрузки в  жидкостной  системе обслуживания с одним сервером. На вход системы  поступает
процесс, содержащий   линейную (детерминированную) компоненту и
случайную компоненту, описываемую центрированным гауссовским
процессом, у которого дисперсия является регулярно меняющейся
функцией  с показателем  $V\hm\in (0,\,2)$. К такому классу процессов,
в частности, относится сумма независимых  ДБД.
 Показано, что при соответствующей  нормировке максимум процесса нагрузки на интервале $[0,\,t]$
сходится по вероятности (при  $t\hm\to \infty$) к некоторой явно
выписанной  константе. Этот результат обобщает соответствующий
результат, полученный ранее в работе~\cite{Zeevi} для случая
единственного входного процесса ДБД.

{\small\frenchspacing
{%\baselineskip=10.8pt
\addcontentsline{toc}{section}{Литература}
\begin{thebibliography}{99}

\bibitem{Leland} %1
\Au{Leland~W.\,E., Taqqu~M.\,S., Willinger~W., Wilson~D.\,V.} On
the self-similar nature of ethernet traffic (extended version)~//
IEEE/ACM Transactions of Networking, 1994. Vol.~2. No.\,1. P.~1--15.

\bibitem{Willinger} %2
\Au{Willinger~W., Taqqu~M.\,S., Leland~W.\,E., Wilson~D.}
Self-similarity in high-speed packet traffic: Analysis and modeling
of Ethernet traffic measurements~// Statistical Sci., 1995.
Vol.~10. No.\,1. P.~67--85.

\bibitem{Reich} %3
\Au{Reich~E.} On the integrodifferential equation of Takacs~I~// Ann. Math. Stat., 1958. Vol.~29. P.~563--570.

\bibitem{Narayan} %4
\Au{Narayan~O.} Exact asymptotic queue length distribution for
fractional Brownian traffic~// Advances in Performance Analysis,
1998. Vol.~1. P.~39--63.

\bibitem{Husler} %5
\Au{H$\ddot{\mbox{u}}$sler~J., Piterbarg V.\,I.} Extremes of a certain class
of Gaussian processes~// Stochastic Processes and Their
Applications, 1999. Vol.~83. P.~257--271.


\bibitem{Duffield} %6
\Au{Duffield~N., O'Connell~N.} Large deviations and overflow
probabilities for the general single server queue, with applications~// 
Proc. Cambridge Philosophical Society, 1995.
Vol.~118. P.~363--374.

\bibitem{Debicki} %7
\Au{Debicki~K.} A~note on LDP for supremum of Gaussian processes
over infinite horizon~// Stat. Probab. Lett., 1999. Vol.~44.
P.~211--220.

\bibitem{Duffy} %8
\textit{Duffy~K., Lewis~J.\,T., Sullivan~W.\,G.} Logarithmic
asymptotics for the supremum of a stochastic process~// Ann. Appl. Probab., 2003. Vol.~13. No.\,2. P.~430--445.

\bibitem{Zeevi} %9
\Au{Zeevi~A., Glynn~P.} On the maximum workload in a queue fed
by fractional Brownian motion~// Ann. Appl. Probab., 2000. Vol.~10.
P.~1084--1099.

\bibitem{Husler1} %10
\Au{H$\ddot{\mbox{u}}$sler~J., Piterbarg V.\,I.} Limit theorem for maximum
of the storage process with fractional Brownian as input~// 
Stochastic Processes and their Applications, 2004. Vol.~114.
P.~231--250.

\bibitem{Taqqu}
\Au{Taqqu~M.\,S., Willinger~W., Sherman~R.} Proof of a
fundamental result in self-similar traffic modeling~// Computer
Communication Rev., 1997. Vol.~27. P.~5--23.

\bibitem{Mandjes}
\Au{Mandjes~M.} Large Deviations of Gaussian Queues.~---
Chichester: Wiley, 2007. 339~p.

\bibitem{Seneta}
\Au{Сенета~Е.} Правильно меняющиеся функции.~--- М.: Наука, 1985.
143~с.

\bibitem{Konstantopoulos}
\Au{Konstantopoulos~T., Zazanis~M., De~Veciana~G.}
Conservation laws and reflection mappings with application to
multiclass mean value analysis for stochastic fluid queues~// 
Stochastic Processes and their Applications, 1996. Vol.~65.
P.~139--146.

\bibitem{Leadbetter}
\Au{Лидбеттер~М., Линдгрен~Г., Ротсен~Х.} Экстремумы случайных
последовательностей и процессов.~--- М.: Мир, 1989. 392~с.

\bibitem{Lifshits}
\Au{Лифшиц~М.\,А.} Гауссовские случайные функции.~--- Киев: ТвиМС, 1995. 248~с.

\label{end\stat}

\bibitem{Adler}
\Au{Adler~R.\,J.} An introduction to continuity, extrema, and
related topics for general Gaussian processes.~--- Hayward, CA: Institute of
Mathematical Statistics, 1990. 170~p.
 \end{thebibliography}
}
}


\end{multicols}     %10
\def\stat{mor-nekr}

\textit{\hfill  Посвящается 70-летию со дня рождения В.\,В.~Калашникова
(1942--2001),}

\textit{\hfill внесшего большой вклад в развитие регенеративного
метода}

\def\tit{ОБ ОЦЕНИВАНИИ ВЕРОЯТНОСТИ ПЕРЕПОЛНЕНИЯ КОНЕЧНОГО БУФЕРА В~РЕГЕНЕРАТИВНЫХ СИСТЕМАХ
ОБСЛУЖИВАНИЯ$^*$}

\def\titkol{Об оценивании вероятности переполнения конечного буфера в регенеративных системах
обслуживания}

\def\autkol{Е.\,В.~Морозов,  Р.\,С.~Некрасова}
\def\aut{Е.\,В.~Морозов$^1$,  Р.\,С.~Некрасова$^2$}

\titel{\tit}{\aut}{\autkol}{\titkol}


{\renewcommand{\thefootnote}{\fnsymbol{footnote}}\footnotetext[1]
{Работа поддержана РФФИ (проект 10-07-00017). Работа выполнена
при поддержке Программы стратегического развития на 2012--2016~гг.\
<<Университетский комплекс ПетрГУ в научно-образовательном пространстве
Европейского Севера: стратегия инновационного развития>>.}}



\renewcommand{\thefootnote}{\arabic{footnote}}
\footnotetext[1]{Институт прикладных математических исследований КарНЦ 
РАН, Петрозаводский государственный университет,\linebreak emorozov@karelia.ru}
\footnotetext[2]{Институт прикладных математических исследований КарНЦ 
РАН, Петрозаводский государственный университет,\linebreak ruslana.nekrasova@mail.ru}

\vspace*{-12pt}


\Abst{Рассмотрены  вопросы оценивания
стационарной вероятности переполнения конечного буфера на основе
регенеративного моделирования. Приведен
 вывод общего соотношения, связывающего в стационарном режиме вероятность  потери с
вероятностью простоя обслуживающего канала.  Показано его применение для
широкого класса систем с потерями, а также  для системы с повторными вызовами и
постоянной скоростью возвращения заявок с орбиты на обслуживание. Исследована
эффективность этого соотношения при регенеративном оценивании вероятности
потери при различных режимах загрузки сис\-те\-мы, а также при использовании
$k$-ре\-ге\-не\-ра\-ций, возникающих при анализе немарковских сис\-тем с потерями.
Приведены результаты численного моделирования.}

\vspace*{-2pt}

\KW{системы с конечным буфером; вероятность потери;
вероятность простоя; регенеративный метод оценивания; $k$-ре\-ге\-не\-ра\-ции; сис\-те\-ма
с повторными вызовами}

\vspace*{-7pt}

\vskip 14pt plus 9pt minus 6pt

      \thispagestyle{headings}

      \begin{multicols}{2}

            \label{st\stat}


\section{Введение} 


Модели с ограничениями, в частности с конечным буфером, играют
важную роль в анализе современных     телекоммуникационных систем. 
В~таких системах поток, образованный потерянными заявками, часто
является входным потоком для другого узла коммуникационной системы,
а вероятность потери является ключевым показателем качества
обслуживания.


В данной работе  представлено  доказательство весьма общей формулы, связывающей
стационарную вероятность потери $\p_{\mathrm{loss}}$ со ста\-цио\-нарной\linebreak
вероятностью
простоя канала~$\p_0$.  Этот результат\linebreak
 позволяет сравнить эффективность оценки
$\p_{\mathrm{loss}}$, полученной с помощью оценки~$\p_0$, и  стандартной оценки, равной
доле потерянных заявок.  Формула через $\p_0$ применяется также
 к системе обслуживания с повторными вызовами и
с  постоянной скоростью возвращения заявок с орбиты~\cite{Avr}.   \mbox{В~статье}
также приведено условие стационарности такой модели в случае
двух серверов. Исследования по регенеративному оцениванию
вероятности блокировки в такой сис\-те\-ме  с повторными вызовами  (как  модели телефонных
протоколов  множественного доступа) были начаты   в работе~\cite{Minsk}.
 В  данной работе оценивание  также  опирается на регенеративный  метод с
использованием так называемых {\it  $k$-ре\-ге\-не\-ра\-ций} (возникающих в сис\-те\-мах с
входным пуассоновским потоком и/или экспоненциальным временем обслуживания),
когда в моменты прихода (или ухода) заявок в сис\-те\-ме находится $k$ других
заявок. В~статье исследуется эффективность оценивания при использовании
разных типов $k$-ре\-ге\-не\-ра\-ций (в том числе классической $0$-ре\-ге\-не\-ра\-ции). Как
известно,  доверительные интервалы, использующие разные последовательности
точек регенерации, асимптотически (с ростом числа наблюдений) эквивалентны~\cite{Shedler}.  
Однако выбор по\-сле\-до\-ва\-тель\-ности может существенно повлиять на
{\it скорость получения оценки}, и этот вопрос также рассматривается в статье.

Статья организована следующим образом. В~разд.~2  описана регенеративная
структура процессов в сис\-те\-мах с потерями и входным процессом восстановления.
Рассмотрены как классические, так и $k$-ре\-ге\-не\-ра\-ции, возникающие в немарковских
сис\-те\-мах $M/G/1/n$ и  $GI/M/m/n$. Приведена процедура построения
доверительного интервала при регенеративном оценивании. Также в разд.~2
затрагивается вопрос асимптотической эквивалентности длин доверительных
интервалов, использующих различные последовательности  регенераций. В~разд.~3
 выведено основное соотношение между  вероятностями   $\p_{\mathrm{loss}}$ и~$\p_0$.
 В~разд.~4 рассматривается двухсерверная система с конечным буфером,
 повторными вызовами и постоянной скоростью возвращения заявок с орбиты на обслуживание.
Для такой системы в явном виде получено условие стационарности, включающее
вероятность~$\p_{\mathrm{loss}}$ в некоторой тесно связанной с ней системе с потерями. 
В~заключительном разделе~5 приведены некоторые результаты численного оценивания
ве\-ро\-ят\-ности потери $\p_{\mathrm{loss}}$ с использованием $k$-ре\-ге\-не\-ра\-ций, а также
формулы, связывающей $\p_{\mathrm{loss}}$ и~$\p_0$.

\section{Регенеративная структура систем с~потерями}

При отсутствии точной формулы для вычисления вероятности $\p_{\mathrm{loss}}$
возникает необходимость в ее надежном  оценивании. В~данной работе с
этой целью  в сис\-те\-ме вида $GI/G/m/n$ применяется регенеративный
метод. Обозначим через $\{t_n\}$ моменты прихода заявок в сис\-те\-му, и
пусть $\{\tau_i:=t_{i+1}\hm-t_i\}$~--- независимые одинаково
распределенные (н.\,о.\,р.)\ интервалы входного потока, а $\{S_i\}$~---
н.\,о.\,р.\ интервалы обслуживания (с типичными элементами $\tau,S$
соответственно). Пусть $\nu_n$~--- число заявок в сис\-те\-ме в момент
прихода заявки $n\hm\ge 1$. Пусть $\{\nu(t)t\hm\ge 0\}$ есть
(непрерывный справа) процесс числа заявок в сис\-те\-ме, т.\,е.\
$\nu_n\hm=\nu(t_n^-)$. Тогда классические регенерации процесса $\{\nu(t)\}$ 
(и других процессов в непрерывном времени в данной сис\-те\-ме),
по\-рож\-да\-емые приходом заявок в пустую сис\-те\-му, рекурсивно
определяются хорошо известным образом: 
\begin{equation}
T_{n+1}=\min\limits_k\{t_k>T_n: \nu_k=0\}\,, \ n \ge 0\,,\ T_0:=0\,.
\label{e1-mn} 
\end{equation} 
Заметим, что регенерации~(\ref{e1-mn}) не являются событиями в потоке отказов и 
поэтому являются \emph{скрытыми} по отношению к этому потоку. Регенерирующий процесс
$\{\nu(t)\}$ называется {\it положительно возвратным}, если $\e T\hm<\infty$, где через~$T$ 
обозначена типичная длина цикла регенерации~\cite{Wolff, Morozov2004}.
 Положительная возвратность является необходимым условием
применимости регенеративного метода оценивания~\cite{Asmus}. Отметим, что более
общая конструкция так называемой однозависимой регенерации возникает в цепях
Маркова, возвратных по Харрису. Анализ такой регенерации, в том числе в
сис\-те\-мах с потерями и в  сетях обслуживания, содержится, например, в~\cite{Sig2, Sig3}.

Для  более специальных систем существуют альтернативные (классические)
регенерации. Со\-хра\-няя введенные обозначения, в сис\-те\-ме $GI/M/m/n$ зафиксируем
любое целое $k\hm\in [0,\, m+n]$ и определим рекурсивно моменты  {\it $k$-ре\-ге\-не\-ра\-ции} 
в {\it дискретном времени}  как номера тех заявок, которые
находят  в системе $k$ других заявок, т.\,е.\ 
 \begin{equation}
\beta^{(k)}_{n+1}=\inf_l\{l>\beta^{(k)}_{n}: \nu_l=k\}\,, \enskip
\beta^{(k)}_0:=0\,.
\label{e2-mn}
\end{equation}
Моменты  (\ref{e2-mn}) связаны с~(\ref{e1-mn}), как
$T_n\hm=t_{\beta^{(0)}_n}$, $n\hm\ge0$. Процесс $\{\nu_n\}$ образует апериодическую,
неприводимую цепь Маркова с  состояниями $\{0,\ldots,n+m\}$, когда на 
событии $\{\nu_l\hm=k \}$ в момент прихода   заявки
времена обслуживания  в каналах  разыгрываются заново. 

Заметим, что
{\it мера регенерации}~--- распределение процесса в момент
регенерации~--- является невырожденной при  $k\hm>0$. Для пояснения
рассмотрим событие $\{\beta^{(k)}_{n}\hm=r\}$, на котором заявка~$r$
встречает $k$ других заявок, т.\,е.\ $\nu_r\hm=k$ (и это $n$-я подобная
заявка среди первых $r$ заявок). Рассмотрим $(m+1)$-мер\-ный процесс
$W(l):=(Q_l,\, W_l^1, \ldots, W_l^m)$, $l\hm\ge1$, где $Q_l$~--- число
заявок  в буфере, а $W_l^i$~--- $i$-е в порядке возрастания
остаточное время обслуживания заявки  в момент~$t_l$ прихода заявки~$l$
($i=1,\dots,m$). Отметим очевидное соотношение
$\nu_l\hm=Q_l\hm+\sum\limits_{i=1}^mI(W_l^i>0)$, где $I$~--- индикатор. На событии
$\{\beta^{(k)}_{n}\hm=r\}$, если $k\hm< m$, то
$W(r)\hm{=}_{\!\mathrm{st}}\;(0,\,0,\ldots,\phi_{k+1}, \ldots, \phi_m)$, а если $k\hm\ge m$, 
то $W(r)\hm{=}_{\!\mathrm{st}}\;(k-m,\,\phi_1, \ldots, \phi_m$), где $\{\phi_i\}$
есть н.\,о.\,р.\   случайные величины (с.\,в.), а знак ${=}_{\mathrm{st}}$ означает 
стохастическое равенство. (С.\,в.~$\{\phi_i\}$  имеют показательное распределение    времени
обслуживания.) Поэтому распределение $W(r)$ зависит лишь от типа регенерации~$k$, но не от момента~$r$.

Аналогично, если $\nu_n^*$ есть число заявок, которое оставляет $n$-я
(уходящая)  заявка в системе  $M/G/1/n$,  то для фиксированного $k\hm\in[0,\,n]$
 \begin{equation*}
\alpha^{(k)}_{n+1}=\inf\limits_l\{l>\alpha^{(k)}_{n}: \nu_l^*=k\}\,,
\enskip\alpha^{(k)}_{0}:=0
 \end{equation*}
 есть  моменты $k$-регенерации для вложенной  цепи Маркова $\{\nu_n^*\}$, 
 мера регенерации которой   является невырожденной при $k\hm>0$.
 В~сис\-те\-мах с конечным буфером   незавершенная работа  (стохастически) ограничена.
Кроме того, в системе $GI/M/m/n$  длина $k$-цик\-ла непериодическая, поскольку
\begin{multline*}
\p(A_k=1)=\p(\nu^*_{l+1}=k|\nu^*_l=k)={}\\
{}=\p(\tau_1+\tau_2>S>\tau_1)>0\,.
\end{multline*}
Аналогично непериодичность длины $k$-цик\-ла в сис\-те\-ме $M/G/1/n$
следует из условия $\p(S_1+S_2>\tau\hm>S_1)\hm>0$.  Это обеспечивает положительную
 возвратность  $k$-ре\-ге\-не\-ра\-ций~\cite{MorozovDelgado}.
 
 \pagebreak

Дадим теперь строгое определение веро\-ят\-ности $\p_{\mathrm{loss}}$.
 Пусть $R(t)$~--- число потерянных заявок, а $A(t)$~--- число приходов в систему в
интервале $[0,\,t]$.  Поскольку поведение системы на цикле $k$-ре\-ге\-не\-ра\-ции (далее~--- 
{\it $k$-цикл}), зависит от~$k$, то  будем  обозначать  через $A_k$, $R_k$, $T_k$ число
приходов, число потерь на $k$-цик\-ле и  длину $k$-цик\-ла в непрерывном времени соответственно.
  Из теории регенерации следует, что в системе $GI/G/m/n$ с условием
$\p(\tau>S)\hm>0$  существуют и равны следующие   пределы:
\begin{equation}
 \lim\limits_{t \to \infty}\fr{R(t)}{A(t)}= \fr{\e R_k}{\e
A_k}:=\p_{\mathrm{loss}}=\lim\limits_{n\to\infty}\p(I_n=1)\,,
\label{e4b-mn}
\end{equation}
где $I_n=1$, если $n$-я  заявка получает отказ ($I_n\hm=0$, иначе).
Заметим, что условие $\p(\tau>S)>0$  влечет непериодичность
дискретной длины цикла занятости $A_0$ и существование слабого предела $\lim \p(I_n=1)$.


Обозначим $ Z_k:= R_k-\p_{\mathrm{loss}}A_k$.
      Регенеративный метод позволяет построить  доверительный интервал для
 стационарной   вероятности потери    $\p_{\mathrm{loss}}$,
если дисперсия $\D Z_k:=\D (R_k-\p_{\mathrm{loss}}A_k)\hm<\infty$.

Рассмотрим систему $M/G/1/n$ в предположении $\e S^2\hm<\infty$, что
влечет   $\D T_0\hm<\infty$~\cite{Wolff}. Пусть $I_k$ есть индикатор
события $\{\tau_1+\cdots+\tau_{k+1}>S_1\hm\ge \tau_1+\cdots+\tau_k\}$,
вероятность которого равна 
$$ 
\p_k =\int\limits_{0^-}^\infty e^{-\lambda x}\fr{(\lambda x)^k}{k!} \,dB(x)>0\,, 
$$ 
где $B$~--- распределение времени обслуживания. Заметим, что при $I_k\hm=1$  за время
обслуживания одной заявки, начинающей 0-цикл, в системе появится
ровно $k$  других заявок, т.\,е.\ начнется $k$-цикл. Следовательно,  
$T_0\hm\ge I_k T_k$ и  неравенство $ \e T_k^2\hm\le \e T_0^2/\p_k\hm<\infty $
влечет $\D T_k\hm<\infty.$ Поскольку
$T_k\;{=}_{\!\mathrm{st}}$\linebreak ${=}_{\!\mathrm{st}}\;\tau_1+\cdots+\tau_{A_k}$, то 
\begin{equation*} 
\D T_k=\e A_k \D \tau + \e \tau^2\D A_k \ge \e \tau^2\D
A_k\,,\enskip k=1,\ldots,n\,. 
\end{equation*} 
Это дает  $\D A_k\hm<\infty$, а
также (поскольку $R_k\hm\le{\!_{\mathrm{st}}} A_k$) и $cov(R_k,\,A_k)\hm\le \e A_k^2$.
Поэтому условие $\e S^2\hm<\infty$ влечет $\D Z_k\hm<\infty$ и позволяет
применить в сис\-те\-ме $M/G/1/n$ оценивание вероятности $\p_{\mathrm{loss}}$ на
основе $k$-ре\-ге\-не\-раций.

Предположим, что в рассматриваемой сис\-те\-ме $\D Z_k\hm\in(0,\,\infty)$, и обозначим
через~$p_k$ чис\-ло  $k$-цик\-лов, полученных в процессе моделирования сис\-те\-мы.
Тогда стандартным образом можно получить $(1-\gamma)\%$ доверительный интервал
для $\p_{\mathrm{loss}}$ в следующей форме: 
\begin{equation}
\hspace*{-1pt}\left[\p_{\mathrm{loss}}(p_k)-\fr{z_{\gamma}\sigma(p_k)}{\hat{A}(p_k)\sqrt{p_k}},
\,\p_{\mathrm{loss}}(p_k)+\fr{z_{\gamma}\sigma(p_k)}{\hat{A}(p_k)\sqrt{p_k}}\right],\!\!
\label{e9-mn}
\end{equation}
 где $\p_{\mathrm{loss}}(p_k)$~--- оценка $\p_{\mathrm{loss}}$, $\hat{A}(p_k)$~---
выборочная длина $k$-цик\-ла, $\sigma^2(p_k)$~--- выборочная оценка дисперсии
$\D Z_k$, а квантиль $z_{\gamma}$ находится из условия $z_{\gamma}\hm=\phi ^{-1}((1-\gamma)/2)$ 
($\phi(x)$~--- функция Лапласа). При этом $\hat{A}(p_k)\hm\to \e A_k,\,\sigma^2(p_k)\to \D Z_k$ 
при $p_k\to \infty$ c в.~1.

Рассмотрим подробнее  построение доверительного интервала по
случайному числу $p_k(t)$ $k$-цик\-лов, завершенных на периоде
моделирования $[0,t]$. Введем последовательность н.\,о.\,р.\ с.\,в.\
$\{Z_k^{(i)}\hm=R_k^{(i)}\hm-\p_{\mathrm{loss}}A_k^{(i)}$, $i\hm\ge1 \}$ с типичным
элементом $Z_k\hm=R_k\hm-\p_{\mathrm{loss}}A_k$ и  заметим, что
\begin{equation} 
R(t)-\p_{\mathrm{loss}}A(t)=\sum\limits_{i=1}^{p_k(t)}Z_k^{(i)}+o(t)\,,\enskip t\to
\infty\,,
\label{e6-mn}
\end{equation} 
где величина  $o(t)$  описывает число потерь на
оставшейся части цикла, {\it накрывающего}  момент~$t$~\cite{Asmus, Smith}.
 С~учетом $\e Z_k=0$ получаем  такую центральную предельную теорему~\cite{Asmus}:
\begin{equation}
\fr{R(t)-\p_{\mathrm{loss}} A(t)}{\sqrt{A(t)}}\Rightarrow N
\left(0, \fr{\D Z_k}{\e A_k}\right)\,,\enskip
 t \to \infty\,, 
 \label{e7-mn}
\end{equation}
где $N$ обозначает нормальную с.\,в.
  Таким образом,   отношение $\D Z_k/\e A_k$ является константой, не зависящей от~$k$ 
  (см.\ также~\cite{Shedler}). Следующий результат также известен~\cite{Shedler},
однако представляется  полезным  дать краткое пояснение к его
выводу. Пусть $p_i(t),\,p_j(t)$~--- число $i$-цик\-лов и $j$-цик\-лов
соответственно, полученных в интервале $[0,\,t]$.  Отметим, что в
рассматриваемом случае доверительный интервал строится как и~(\ref{e9-mn}),
 но с использованием случайного (а не детерминированного) чис\-ла  циклов.
 Обозначим через
$|I_i(t)|$ длину доверительного интервала, построенного по $i$-цик\-лам   и
пусть $\hat{A}_i(t)$~--- выборочная длина $i$-цик\-ла. Обозначая $\D Z_i\hm=\sigma_i^2$, получаем
\begin{multline}
\fr{|I_i(t)|}{|I_j(t)|}=
\fr{ \sigma_i\,\,\;\;\hat{A}_j(t)\sqrt{p_j(t)}}{\sigma_j\;\;\hat{A}_i(t)\,\,\sqrt{p_i(t)}}
={}\\
{}=\sqrt{\fr{\sigma^2_i\,\;\;\hat{A}_j(t)}{\hat{A}_i(t)\;
\sigma^2_j}}\,\,\sqrt{ \fr{\hat{A}_j(t)\;p_j(t)} {\hat{A}_i(t)\;p_i(t)}} \to
1 \mbox{ при } t \to \infty\,, 
\label{ints}
\end{multline} 
где использовано постоянство отношения $\D Z_k/\e A_k$ и  усиленный закон больших чисел
\begin{equation}
\left.
\begin{array}{c}
\hat{A}_i(t)\to \e A_i\,,\quad   \hat{A}_j(t)\to \e A_j\,;\\[9pt]
 \hat{A}_j(t)p_j(t)\sim \hat{A}_i(t)p_i(t)\sim t\,,\quad t\to \infty\,,
 \end{array}
 \right\}
\label{eqints}
\end{equation} 
($a\sim b$ означает $a/b\hm\to 1$). Таким образом, интервалы, построенные по
различным последовательностям $k$-ре\-ге\-не\-ра\-ций, асимптотически эквивалентны.
 Однако существенное различие в числе цик\-лов,
полученных при ограниченном времени моделирования, может  дать преимущество
одной последовательности перед другими.  Этот вопрос рассматривается в секции~5. 
Подробное описание построения доверительных интервалов при регенеративном
оценивании можно   найти в~[3, 11--14]. % \cite{Shedler, Crane}.
 (Доказательство сходимости~(\ref{e7-mn}) для случайных сумм вида~(\ref{e6-mn}) содержится также  
 в~\cite{Asmus, Billingsley}.)

\section{Соотношение между вероятностью потери и~вероятностью простоя}

В данном разделе для широкого класса регенеративных систем с потерями
доказано соотношение, связывающее вероятность $\p_{\mathrm{loss}}$ со стационарной
вероятностью простоя канала~$\p_0$. При этом используются лишь  0-ре\-ге\-не\-ра\-ции.
 Сохраняя  прежние обозначения,  рассмотрим сис\-те\-му $GI/G/m/n$ с коэффициентом  загрузки
$\rho\hm=\e S/E\tau$. Заметим, что в формулировке теоремы~1 предполагается, что
выбор свободного канала для новой заявки происходит равновероятно, если таких
каналов несколько.

\medskip

\noindent
\textbf{Теорема 1.}  {\it В системе $GI/G/m/n$  при условии
$\p(\tau>S)\hm>0$ стационарная вероятность потери $\p_{\mathrm{loss}}$
определяется формулой~(\ref{e4b-mn}) и
  связана со стационарной  вероятностью простоя (любого) канала~$\p_0$
следующим образом:} 
\begin{equation} 
%\label{viaPb}
\p_{\mathrm{loss}}=1-\fr{m}{\rho}(1-\p_0)\,.
 \label{e4-mn}
\end{equation}

\smallskip

\noindent
Д\,о\,к\,а\,з\,а\,т\,е\,л\,ь\,с\,т\,в\,о\,.\
Определим процесс накопленной работы в сис\-те\-ме
$W(t)\hm=\sum\limits_{i=1}^mW_i(t)$, где $W_i(t)$ есть не завершенная  в момент~$t$ 
работа, предназначенная для   канала~$i$. Поскольку буфер
конечен, то процесс $\{W(t)\}$  является стохастически ограниченным, 
а из условия $\p(\tau>S)\hm>0$ следует  положительная возвратность
 $\e T_0\hm<\infty$.  (Подробный анализ стационарности регенеративных сис\-тем, 
 в том числе с конечным буфером, содержится в~\cite{Morozov2004, MorozovDelgado}.) 
 Пусть  $V(t)$~--- суммарная нагрузка, поступившая в сис\-те\-му, 
 $B(t)$~--- обслуженная нагрузка, а $L(t)$~--- потерянная нагрузка (время, которое требовалось для обслуживания
потерянных заявок), все в интервале $[0,\,t]$. Получаем следующее уравнение баланса:
\begin{equation}
 V(t)=W(t)+B(t)+L(t)\,,
 \label{bal}
\end{equation} 
где, возможно, $V(0)=W(0)\not =0$. Заметим, что $B(t)\hm=\sum\limits_{i=1}^mB_i(t)$, а 

\noindent
$$
B_i(t)=\int\limits_0^tI(W_i(u)>0)\,du 
$$ есть время занятости канала~$i$ в интервале $[0,\,t]$, $i=1,\ldots,m$. Очевидно,  что
\begin{equation}
V(t)=\sum_{k=1}^{A(t)}S_k,\;\; L(t)=\sum\limits_{k=1}^{R(t)}S_k\,.
\label{e11-mn}
\end{equation}
Из  усиленного закона больших чисел при $t\hm\to\infty$  легко следует
\begin{equation}
\fr{V(t)}{t}\to \rho\,,\enskip \fr{L(t)}{V(t)} \to
\fr{\e R}{\e A}= \p_{\mathrm{loss}}\,.
\label{e14-mn}
 \end{equation} 
 Поскольку
$W(t)\le_{\!\mathrm{st}}\sum\limits_{i=1}^nS_i+\sum\limits_{i=1}^mS_i(t)$, 
где не завершенное в канале~$i$ в момент~$t$ время обслуживания 
$S_i(t)\hm=o(t)$~\cite{Smith}, то $W(t)\hm=o(t)$, $t\hm\to \infty$. (Последний результат
можно также вывести непосредственно из положительной возвратности
$0$-ре\-ге\-не\-ра\-ций.) Так как каналы идентичны, то получаем также
\begin{equation}
\fr{B(t)}{t}=\fr{\sum\limits_{i=1}^mB_i(t)}{t}\to m(1-\p_0)\,,\enskip t\to \infty\,,
\label{e16-mn}
 \end{equation}
  что  вместе с~(\ref{e14-mn}) дает
 \begin{equation*}
\fr{B(t)}{V(t)}\to \fr{m(1-\p_0)}{\rho}\,.
\end{equation*} 
Таким образом, предел $(t-B_i(t))/t\hm\to \p_0$ является стационарной вероятностью простоя любого
канала. Поделив обе части~(\ref{bal}) на $V(t)$ и перейдя к пределу при $t\hm\to\infty$, 
получаем~(\ref{e4-mn}). Наконец, как было отмечено (ввиду  условия
$\p(\tau>S)\hm>0$),  $\p_{\mathrm{loss}}$ также  является стационарной вероятностью потери приходящей 
заявки.~\hfill$\square$

\smallskip

В случае одного сервера  статистический аналог
формулы~(\ref{e4-mn}) используется   в работах~\cite{Whitt91, Whitt90} при
построении так называемой непрямой оценки вероятности $\p_{\mathrm{loss}}$ (см.\ также~\cite{GM}). 
Однако в случае нескольких серверов представление
 суммарного времени занятости в виде $B(t)\hm=\sum\limits_{i=1}^mB_i(t)$ там не используется.
 Формула~(9) для системы с потерями вида $M/G/1/n$ так\-же получена другим методом 
 в~[19, с.~333].

 Как показано  в разд.~5,  формула~(\ref{e4-mn})  может быть полезна для  оценки
 вероятности $\p_{\mathrm{loss}}$ через оценку вероятности простоя (или занятости)
 канала в случае большой нагрузки, когда прямая оценка $\p_{\mathrm{loss}}$ с помощью метода Мон\-те-Кар\-ло
оказывается неэффективной.

В заключение этого раздела отметим, что  соотношение~(\ref{e4-mn}) можно
непосредственно применить к системам с буфером {\it случайного размера},
который  регенерирует на периодах занятости сис\-те\-мы. Такие сис\-те\-мы могут
представлять интерес для моделирования  узлов связи в некоторых современных
коммуникационных сетях. Кроме того, соотношение~(\ref{e4-mn}) верно  для предложенной 
в~\cite{Tih} системы со {\it случайным объемом поступающих заявок} и ограничением
на  суммарный объем. В~такой  системе  $\p_{\mathrm{loss}}$   равна предельной доле
потерянного объема. Также соотношение~(\ref{e4-mn}) верно для систем с ограниченным
ожиданием/пребыванием заявки.

Нетрудно видеть, что  для {\it жидкостной системы} со скоростью обслуживания~$C$  
соотношение~(\ref{e4-mn}) принимает вид 
\begin{equation} 
\p_{\mathrm{loss}}=1-\fr{Cm (1-\p_0)}{\rho}\,.
\label{e22-mn} 
\end{equation} 
Заметим, что в классических моделях $C\hm=1$, а  в жидкостных моделях поступающая 
нагрузка~$V(t)$ не разделяется на отдельные заявки, а следует, например,  
процессу Леви (с независимыми стационарными приращениями) с {\it заданной интенсивностью}~$\rho$. 
В~этом случае величина $\rho/C$  является {\it коэффициентом загрузки} и, таким
образом, формулы~(\ref{e22-mn}) и~(\ref{e4-mn}) эквивалентны. 

Отметим, что если рассматривать $\p_{\mathrm{loss}},\,\p_0$ лишь как пределы в среднем,
то соотношение~(\ref{e4-mn}) имеет место   для гораздо более широкого класса
сис\-тем, чем регенеративные.

\section{Система с~повторными вызовами и~постоянной скоростью возвращения
заявок~с~орбиты}

В данном разделе рассматривается  система с повторными вызовами типа $M/G/2/0$
без буфера  с входным (пуассоновским) потоком первичных заявок с интенсивностью~$\lambda$
 и произвольным  временем  обслуживания  с интенсивностью~ $\mu$.  
 Эта сис\-те\-ма, далее обозначенная через~$\Sigma_O$,  успешно
применена для моделировании занятости телефонных линий в мобильных сис\-те\-мах~\cite{F86}, 
протоколов множественного доступа ALOHA~\cite{C93}, а также
 протокола  TCP с короткими сообщениями~\cite{AY08}. Когда серверы заняты, первичные заявки
 уходят  на орбиту  бесконечного объема, а затем вновь пытаются попасть  на серверы.
  Поток повторных попыток (при непустой орбите) является пуассоновским с  па\-ра\-мет\-ром~$\mu_0$ 
  и  {\it не зависит от величины орбиты}, в отличие от классических  сис\-тем с повторными вызовами.
Как показано в работах~\cite{Avr, GM, MorNek},  для анализа стационарности сис\-те\-мы~$\Sigma_O$
можно использовать  следующую сис\-те\-му с потерями~$\Sigma_L$ (без орбиты).
 Сис\-те\-ма~$\Sigma_L$ имеет  тот же входной   поток первичных заявок, то же распределение
  времени обслуживания, что и сис\-те\-ма~$\Sigma_O$, но
  имеет еще  один (независимый) пуассоновский входной поток заявок с параметром~$\mu_0$.
 Если в момент прихода  сис\-те\-ма~$\Sigma_L$ занята, то заявка   теряется.
 Обозначим через $\p_{\mathrm{loss}}$ стационарную вероятность потери в сис\-те\-ме~$ \Sigma_L $. 
 В~работе~\cite {Avr} доказано,  что условие 
 \begin{equation}
(\lambda+\mu_0)\p_{\mathrm{loss}}<\mu_0 
\label{st_cond}
\end{equation}
является достаточным (а в марковском случае также и необходимым) условием стационарности
сис\-те\-мы~$\Sigma_O$. Очевидно, что вероятность $\p_{\mathrm{loss}}$ удовлетворяет
основному соотношению~(\ref{e4-mn}). Сис\-те\-ма с потерями~$ \Sigma_L $, однако, важна
не только для определения зоны стационарности~$\Sigma_O$. Как показано в~\cite{Minsk, GM}, 
в {\it зоне нестационарности} системы $\Sigma_O$ оценка
вероятности $\p_{\mathrm{loss}}$ сходится c в.~1 к вероятности блокировки вызова
$\p_{\mathrm{orb}}$ в сис\-те\-ме~$\Sigma_O$. Более того, в работе~\cite{MorNek} доказано,
что в односерверной сис\-те\-ме~$\Sigma_O$ в нестационарном режиме 
\begin{equation}
\label{ret-loss} 
\p_{\mathrm{orb}}=1-\fr{P_b}{\tilde{\rho}}\,, 
\end{equation} 
где $\tilde{\rho}:=(\lambda\hm+ \mu_0)/\mu $, а $\p_b$ есть стационарная веро\-ятность
занятости системы~$\Sigma_L $. Очевидно, (\ref{ret-loss}) является  аналогом~(\ref{e4-mn}), 
поскольку $\p_{\mathrm{orb}}\hm=\p_{\mathrm{loss}}$. Утверждение~(\ref{ret-loss})  поз\-во\-ля\-ет
провести анализ стационарности в {\it частично нестационарной} сис\-те\-ме и
достоверно оценить вероятность блокировки при неограниченно
растущей орбите. Такая тесная связь двух систем мотивирует анализ условий
стационарности системы~$\Sigma_O$, представленный ниже.

 Напомним формулу Эрланга для вероятности потери  в
 системе $M/G/m/0$~ \cite{Gnedenko}
\begin{equation} 
\label{MMP_L}
 \p_{\mathrm{loss}}=\fr{\rho^{m}/m!}{\sum\limits_{k=0}^m \rho^k/k!}\,,\enskip\rho:=\lambda/\mu\,.
\end{equation}
 В работе~\cite{MorNek} для системы $\Sigma_O$ типа $M/G/1/0 $  использование~(\ref{MMP_L})
позволило получить условие~(\ref{st_cond}) при $\lambda\hm=1$ в  форме
$1/\mu_0\hm+1\hm<\mu$. Ниже приведено в явном виде условие стационарности
{\it двухсерверной} сис\-те\-мы~$\Sigma_O$. 

\medskip

\noindent
\textbf{Теорема 2.}
 {\it Cиcтема с повторными вызовами  типа $M/G/2/0$} (\textit{с
интенсивностью первичных заявок $\lambda\hm=1$}) \textit{стационарна, если $\mu>1/2$  и
если} 
\begin{equation*} 
%\label{cond2}
 \mu_0>\fr{\mu^2+\mu-1-\mu \sqrt{\mu^2+2\mu-1}}{1-2 \mu}:=\mu_{01}\,. 
\end{equation*} 

\smallskip

\noindent
Д\,о\,к\,а\,з\,а\,т\,е\,л\,ь\,с\,т\,в\,о\,.\ Используя
формулу~(\ref{MMP_L}) для системы Эрланга с коэффициентом загрузки
 $\rho\hm= (1\hm+\mu_0)/\mu$ и $m\hm=2$, получим условие стационарности~(\ref{st_cond}) в
 форме  неравенства 
\begin{equation} 
\label{ret2-1}
-\mu^2\mu_0-\mu(\mu_0+\mu_0^2)+\fr{1}{2}+\mu_0+\fr{\mu_0^2}{2}<0\,.
\end{equation}
 Разложение на множители при условии $\mu\hm\ge \sqrt{2}-1$ имеет вид
\begin{equation} 
\label{mnoj}
\left(\fr{1}{2}-\mu\right)\left(\mu_0-\mu_{01}\right)\left(\mu_0-\mu_{02}\right)<0\,. 
\end{equation}
При $\mu < \sqrt{2}-1$ неравенство~(\ref{ret2-1}) не выполняется и
сис\-те\-ма нестационарна. Таким образом, стационарность может иметь
место только при $\mu\hm\ge \sqrt{2}-1$, когда справедливо разложение в
левой части~(\ref{mnoj}). Анализ значения  параметров
$(\mu,\,\mu_0)$ в области $ (\sqrt{2}-1, \infty)\times (0, \infty)$,
при которых выполняется неравенство~(\ref{ret2-1}), легко приводит к
утверждению  теоремы.\hfill~$\square$

Отметим, что  условие~(\ref{st_cond}) совпадает с критерием
стационарности   системы с повторными вызовами вида $M/M/2/0$,
полученным в~\cite{A96} на основе традиционной для марковских моделей техники   (см.\ также~\cite {AGN01}).
 В~принятых обозначениях  (и  при $\lambda\hm=1$) оба эти условия могут
быть записаны, например,  в форме
\begin{equation*}
\left(1+\mu_0\right)^2<2 \mu \mu_0\left(1+\mu+\mu_0\right)\,.
\end{equation*}

Область значений параметров $\mu$ и~$\mu_0$,  в которой сис\-те\-ма с
повторными вызовами типа  $M/M/2/0$ обладает стационарностью, представлена на рис.~1.


Рассмотрим  более общую  $m$-сер\-вер\-ную сис\-те\-му с повторными вызовами
(с постоянной ско\-ростью возвращения с орбиты) с конечным буфером  и
с входным потоком восстановления с интенсив\-ностью~$\lambda$,
изученную в~\cite{Avr}.  Предполагается, что
  серверы стохастически
эквивалентны (и\linebreak $S$~--- ти\-пич\-ное  время обслуживания), так что, в
частности, вновь поступающие заявки распределяются  равновероятно по
свободным серверам (если их несколько).  Пусть $V(t)$~--- поступившая,
 а  $B(t)$~--- обслуженная нагрузка  в интервале $[0,\,t]$, причем
$B(t)\hm=\sum\limits_{i=1}^mB_i(t)$, где $B_i(t)$ есть время занятости сервера~$i$ 
в интервале $[0,\,t]$, $i=1,\ldots,m$. \linebreak\vspace*{-12pt}
%\noindent
\vspace*{2pt}
\begin{center}  %fig1
 \mbox{%
 \epsfxsize=77.686mm
 \epsfbox{mo1-1.eps}
 }
 \end{center}
% \vspace*{6pt}
{{\figurename~1}\ \ \small{Область  стационарности  системы с повторными
вызовами вида  $M/M/2/0$}}



%\pagebreak

\vspace*{12pt}

\addtocounter{figure}{1}


\begin{table*}[b]\small
\vspace*{-12pt}
\begin{center} 
\Caption{Системы $M/M/m/n$, $\lambda=4$
\label{tab:pars-mn}}
\vspace*{2ex}

\begin{tabular}{|c|c|c|c|c|c|c|c|c|c|} 
\hline
&&&&&&&&&\\[-9pt]
 $\rho$ & $m$   &$n$   &  $\p_{\mathrm{loss}}$ & $\hat{\p}_{\mathrm{loss}}$ &  $i_0$  &
 $\hat{\p}_l$  &  $i_0$& $VR$& $\varepsilon$ \\
 \hline
1& 2& 0& 0.200& 0.203& 1242& 0.228& 4439& 6.269& 0.05\\ 
2& 2& 0& 0.400& 0.419&1790& 0.391& 1791& 1.007& 0.05\\ 
4& 4& 4& 0.139 & 0.144& 3810& 0.041& 5914&11.318& 0.05\\
8& 2& 0& 0.781& 0.783& 2193& 0.779& 1332& 0.034& 0.05\\
4& 1& 3& 0.751& 0.734& 1840& 0.692& 8531& 0.059& 0.05\\
 \hline 
 \end{tabular} 
 \end{center} 
 \end{table*}

\noindent
Пусть    $W(t)$
 есть не завершенная  в момент~$t$ работа в системе (в буфере, в серверах  и на
 орбите). Для данной  сис\-те\-мы имеет место такой аналог уравнения баланса~(\ref{bal}), 
 в котором отсутствует слагаемое, связанное с потерями:
\begin{equation}
V(t)=W(t)+B(t)=W(t)+\sum\limits_{i=1}^mB_i(t),\enskip t\ge0\,.
\label{e26-mn}
\end{equation}
Предположим теперь, что система стационарна, т.\,е.\ вложенный процесс
регенераций положительно возвратен. Обозначим $\rho\hm=\lambda\e S$.
Тогда, в част\-ности, $W(t)\hm=o(t)$, $t\hm\to \infty$, и из~(\ref{e26-mn})
следует  предельное соотношение (см.~(\ref{e11-mn})--(\ref{e16-mn}))
\begin{equation}
\rho= m \p_b\,,
\label{e27-mn}
\end{equation}
где $\p_b$ есть стационарная вероятность занятости любого сервера. В~част\-ности, $\p_b\hm=\rho$ 
при $m\hm =1$. Более того, если входной процесс
пуассоновский, то (по свойству PASTA)  $\rho$ есть  также
стационарная вероятность  блокировки (ухода на орбиту) вновь
поступающей первичной заявки (при $m\hm=1$).  Интересно отметить, что
соотношение~(\ref{e27-mn}) верно также и для общей сис\-те\-мы $GI/G/m$  (с
бесконечным буфером и без орбиты). (Здесь важно лишь отсутствие
потерь.) Однако критерий  стационарности $\rho/m\hm<1$ этой сис\-те\-мы,
разумеется, отличается от (достаточного) условия стационарности~(\ref{st_cond}) сис\-те\-мы 
с повторными вызовами.  Например, для
системы с повторными вызовами вида $M/G/1/K$ условие  стационарности
имеет вид
\begin{equation}
\lambda \rho<\mu_0\left(1-\rho\right)\,. 
\label{e27c-mn}
\end{equation}
С учетом свойства PASTA~\cite{Asmus} условие~(\ref{e27c-mn}) имеет
ясную физическую интерпретацию: левая часть неравенства есть
интенсивность ухода заявок на орбиту, а правая часть есть
максимальная интенсивность (успешного) ухода заявок с орбиты на
сервер, поскольку множитель $1-\rho$ есть стационарная вероятность
незанятости сервера (когда  успешный уход с орбиты возможен).





%\vspace*{-12pt}

\section{Результаты численного  моделирования} 

В данном разделе представлены
некоторые чис\-лен\-ные результаты оценивания вероятности  $\p_{\mathrm{loss}}$ по методу
Мон\-те-Кар\-ло с использованием (в очевидных обозначениях) традиционной оценки
$\hat{\p}_{\mathrm{loss}}(i_0)\hm=\hat{R}(i_0)/\hat{A}(i_0)$ и оценки на основе формулы~(\ref{e4-mn})
 $$
  \hat{\p}_l:=\hat{\p}_l(i_0)=1-\fr{m}{\rho}\,\hat{\p}_b(i_0)
$$
 соответственно в зависимости от   чис\-ла 0-ре\-ге\-не\-ра\-ций~$i_0$, где
$\hat{P_b}(i_0)$~--- оценка стационарной вероятности занятости. (Обозначение 
$\hat{\p}_l(i_0)$ и $\hat{\p}_b(i_0)$ 
 подчеркивает, что используется {\it шкала циклов, а не отдельных
наблюдений}.) В~табл.~1 приведены результаты оценивания  для  сис\-те\-мы
$M/M/m/n$  в зависимости от коэффициента загрузки~$\rho$, а также чис\-ла
серверов~$m$ и величины буфера~$n$. Величина
$VR\hm=D[\hat{\p}_l(i_0)]/D[\hat{\p}_{\mathrm{loss}}(i_0)]$ равна отношению выборочных
дисперсий оценок. Наблюдения проводились до получения заданной точности
$\varepsilon\hm>0$. Значения оценок также сравниваются  с точным значением,
вычисленным по формуле~\cite{Gnedenko}:
\begin{equation*} 
%\label{MMP_L1}
 \p_{\mathrm{loss}}=\fr{\rho^{m}}{m!}\theta ^{\,n}\p_0\,,
\end{equation*} 
где $\theta=\rho/m$, а 
\begin{equation*} 
%\label{MMP_0}
\p_0=\left(\sum\limits_{k=0}^m\fr{\rho^k}{k!}+\fr{\rho^m}{m!}\sum\limits_{k=1}^n
\theta^{\,k}\right)^{-1}\,. 
\end{equation*} 
Критерием эффективности является необходимое число 0-цик\-лов для получения заданной точности,
а также  дисперсия  оценки.
Как видно из табл.~1,
в случае \textit{малой нагрузки} ($\rho/m\hm<1$) классическая оценка
$\hat{\p}_{\mathrm{loss}}$ более  эффективна, чем
 $\hat{\p}_l$,  как по скорости построения, так и по величине  дисперсии.
При \textit{средней нагрузке} ($\rho/m\hm=1$) и  буфере $n\hm=0$
эффективность оценок близка, однако с ростом величины   буфера
оценка $\hat{\p}_{\mathrm{loss}}$ становится предпочтительнее. При
\textit{большой нагрузке} ($\rho/m\hm=4$) и $n\hm=0$ оценка $\hat{\p}_l$
оказывается эффективней как по времени оценивания, так и по
величине дисперсии.  Это согласуется с результатами работ~\cite{Whitt90, MorNek}, 
где  показана эффективность {\it непрямой}
оценки $\hat{\p}_l$ при большой нагрузке (а также и ее отрицательная
корреляция с оценкой $\hat{\p}_{\mathrm{loss}}$). При увеличении величины
буфера оценка $\hat{\p}_{\mathrm{loss}}$ строится быстрее, однако имеет
б$\acute{\mbox{о}}$льшую дисперсию, чем $\hat{\p}_l$ и поэтому нельзя
сделать однозначный вывод об эффективности оценок.



Аналогичные численные результаты получены  для системы Pareto$/M/m/n$ (с
распределением Парето входного потока).



Исследовалась также эффективность применения различных $k$-ре\-ге\-не\-ра\-ций.
Результаты приведены на  рис.~2 и в табл.~2, где оценка $\hat{\p}_{\mathrm{loss}}(i)\hm=R(i)/A(i)$ 
строилась по {\it числу наблюдений}~$i$. Число
наблюдений, как правило, пропорционально времени моделирования и позволяет
оценить реальную\linebreak скорость получения оценки, в отличие от чис\-ла цик\-лов,
 различие в длинах которых может быть очень
значительным. На рис.~2 представлена зависимость от \textit{числа наблюдений }
границ доверительного интервала  для  вероятности потери в системе $M/M/2/4$,
где  рассматривались $k$-ре\-ге\-не\-ра\-ции при  $k\hm=0,3, 6$. 
С~ростом числа   наблюдений доверительные
интервалы для всех  типов регенераций сближаются между собой, что
соответствует~(\ref{eqints}). Однако при малых~ $k$ оценивание
происходит быстрее, поэтому (при малой нагрузке)  такие регенерации
(в том чис\-ле 0-ре\-ге\-не\-ра\-ции) более эффективны. 
Отметим, что скорость оценивания возрастает с ростом~$k$, однако при $k\hm\ge 7$ она растет
незначительно. При максимальном $k\hm=m\hm+n\hm=10$ оценка имеет наименьшую
дисперсию, и поэтому (при большой нагрузке) такие регенерации в
моменты потерь наиболее эффективны.  Причина состоит в том, что
классические регенерации в данном случае редки  и их использование  для оценки
вероятности $\p_{\mathrm{loss}}$ оказывается неэффективным.
В табл.~3.  пред\-став\-ле\-ны результаты
анализа системы Pareto$/M/4/6$ при $\rho/m\hm=4$ и нескольких типах
$k$-ре\-ге\-не\-раций.  %-\linebreak\vspace*{-12pt}

\pagebreak

\noindent
%\noindent
\vspace*{-9pt}  %fig2
 \begin{center}
 \mbox{%
 \epsfxsize=78.578mm
 \epsfbox{mo1-2.eps}
 }
 \end{center}
% \vspace*{6pt}
{{\figurename~2}\ \ \small{Доверительное оценивание  $\p_{\mathrm{loss}}$ на основе\protect\linebreak
$k$-ре\-ге\-нераций, $k=0$~(\textit{1}), 3~(\textit{2}), и~6~(\textit{3}),  
в сис\-те\-ме $M/M/2/4$ при $\rho/m=0.5$}}



%\pagebreak

\vspace*{6pt}

\addtocounter{figure}{1}

\noindent
\begin{center}  %tabl2
%\vspace*{-6pt}
{{\tablename~2}\ \ \small{Система $M/M/2/4$  при $\rho/m=0.5$}}
\vspace*{2ex}

{\small 
\tabcolsep=6.4pt
\begin{tabular}{|c|c|c|c|c|} 
\hline
&&&&\\[-9pt]
 $k$ & $i$   &$\hat{\p}_{\mathrm{loss}}(i)$ &  $\mathrm{Var}[\hat{\p}_{\mathrm{loss}}(i)]$ & $\varepsilon$ \\
 \hline
0& \hphantom{9,}78\hphantom{9}& 0,013& 0,001& 0,05\\ 
3& 291& 0,028& 0,005& 0,05\\ 
6& 1979\hphantom{9}& 0,005& 0,228&0,05\\
 \hline
\end{tabular} 
}
%\vspace*{-9pt}
\end{center}



%\pagebreak

%\vspace*{10pt}

\addtocounter{table}{1}


\noindent
\begin{center}  %tabl3
%\vspace*{-6pt}
{{\tablename~3}\ \ \small{Система
Pareto$/M/4/6$  при $\rho/m=4$}}
\vspace*{2ex}

{\small 
\tabcolsep=7pt
\begin{tabular}{|c|c|c|c|c|}
\hline
&&&&\\[-9pt]
 $k$ & $i$   &$\hat{\p}_{\mathrm{loss}}(i)$ &  $\mathrm{Var}[\hat{\p}_{\mathrm{loss}}(i)]$ & $\varepsilon$ \\
 \hline
4& 14489\hphantom{9}& 0,756& 0,131& 0,05\\ 
6& 2933& 0,738& 0,120& 0,05\\ 
7& 1927& 0,753& 0,135& 0,05\\
 \hline
\end{tabular} 
}
%\vspace*{-9pt}
\end{center}



%\pagebreak

%\vspace*{10pt}

\addtocounter{table}{1}


\section{Заключение} 

В статье рассмотрен ряд   вопросов
 регенеративного оценивания  стационарной вероятности потери в системах
обслуживания с  конечным буфером. Доказано
 общее соотношение,  связывающее в стационарном режиме вероятность потери с
ве\-ро\-ят\-ностью простоя обслуживающего канала  для широкого класса систем, где
потери могут быть вызваны  различными причинами.  Исследована эффективность этого
соотношения при регенеративном оценивании вероятности потери при использовании
$k$-регенераций, возникающих при анализе немарковских систем на основе
вложенных цепей Маркова. Получены в явном виде условия стационарности для
двухсерверной системы с повторными вызовами и постоянной скоростью возвращения
заявок с орбиты на обслуживание. Приведены некоторые  результаты численного
моделирования.

{\small\frenchspacing
{%\baselineskip=10.8pt
\addcontentsline{toc}{section}{Литература}
\begin{thebibliography}{99}

\bibitem{Avr} 
\Au{Avrachenkov~K., Morozov ~E.\,V.}  Stability  analysis of $GI/G/c/K$ retrial queue
with constant retrial rate. \mbox{INRIA} (Sophia Antipolis): Research
Report,  2010. No.\,7335.

\bibitem{Minsk}   %2
\Au{Avrachenkov~K., Goricheva~R.\,S., Morozov~E.\,V.}
Verification of stability region of a retrial queuing system by
    regenerative method~// Modern Probabilistic Methods for Analysis 
    and Optimization of Information and Telecommunication Networks: 
    Intenational Conference Proceedings.~--- Minsk, 2011. P.~22--28.

   
    \bibitem{Shedler} %3
\Au{Shedler~G.\,S.} Regeneration and networks of queues.~--- New-York: Springer-Verlag,  1987.

\bibitem{Wolff} %4
\Au{Wolff~R.\,W.} Stochastic modeling and the theory of
    queues.~--- Englewood Cliffs, NJ: Prentice Hall, 1989.
    
    \bibitem{Morozov2004}  %5
\Au{Morozov~E.\,V.} Weak regeneration in modeling of
    queueing processes~// Queueing Syst., 2004. Vol.~46.   P.~295--315.

\bibitem{Asmus} %6
\Au{Asmussen~S.} Applied probability and queues.~--- 2nd ed. -- New York: Springer-Verlag, 2003.

\bibitem{Sig2} %7
\Au{Sigman~K.} Queues as Harris recurrent Markov chains~//
Queueing Syst., 1988. No.\,3.  P.~179--198.

\bibitem{Sig3} %8 
\Au{Sigman~K.} One-dependent regenerative processes and
    queues in continuous time~// Math. Oper. Res.,  1990. No.\,15. P.~175--189.

\bibitem{MorozovDelgado}  %9
\Au{Морозов~Е., Делгадо~Р.} Анализ
    стационарности регенеративных систем обслуживания~//  Автоматика и
    телемеханика, 2009. No.\,12.  С.~42--58.

\bibitem{Smith}  %10
\Au{Smith~W.\,L.} Regenerative stochastic processes~//
    Proc. Roy. Soc.  Ser.~A, 1955. Vol.~232. P.~6--31.


\bibitem{Crane} %11
\Au{Крэйн~М., Лемуан~О.} Введение в регенеративный
    метод анализа моделей.~--- М.: Наука, 1982. 104~с.

\bibitem{Iglehart} %12
\Au{Иглехарт~Д.\,Л., Шедлер~Д.\,С.} Регенеративное
    моделирование сетей массового обслуживания.~--- М.: Радио и связь, 1984.
    136~с.
    
\bibitem{Glynn} %13
\Au{Glynn~P.\,W., Iglehart D.\,L.} Conditions for the
    applicability of the regenerative method~// Management Sci., 1993. Vol.~39. P.~1108--1111.

\bibitem{Glynn1} %14
\Au{Glynn~P.\,W.} Some topics in regenerative
    steady-state simulation~// Acta Appl. Math.,   1994. No.\,34. P.~225--236.

\bibitem{Billingsley}  %15
\Au{Биллингсли~П.} Сходимость вероятностных мер.~--- М.: Наука, 1977.  352~с.

\bibitem{Whitt91} %16
\Au{Whitt~W.} A~review of $L=\lambda W$ and extensions~//
    Queueing Syst.,  1991. Vol.~9. P.~235--268.
    
    \bibitem{Whitt90}  %17
\Au{Srikant~R., Whitt~W.} Variance reduction in
    simulations of loss models~// Oper. Res., 1999. Vol.~47. No.\,4.     P.~509--523.
    
    \bibitem{GM} %18
\Au{Горичева~Р.\,С., Морозов~Е.\,В.} Регенеративное
    моделирование вероятности потери в системах обслуживания с конечным
    буфером~// Труды Карельского научного центра РАН, 2010. №\,3. С.~20--29.
    
    \bibitem{19-nn}
    \Au{Бочаров П.\,П., Печинкин А.\,В.}
    Теория массового обслуживания.~--- М.: РУДН, 1955. 529~с.

\bibitem{Tih} %19+i
\Au{Тихоненко~О.\,М.} Модели массового обслуживания в
    системах обработки информации.~--- Минск: Университетское, 1990. 191~с.

\bibitem{F86} %20
\Au{Fayolle~G.} A~simple telephone exchange with delayed
    feedback~// Teletraffic Anal. Comp. Performance Evaluation,  1986. Vol.~7. P.~245--253.

\bibitem{C93} %21
\Au{Choi~B.\,D.,  Rhee~K.\,H., Park~K.\,K.} The $M/G/1$
    retrial queue with retrial rate control policy~// Prob. 
    Engng. Informational Sci.,  1993.  Vol.~7. P.~29--46.

\bibitem{AY08} %22
\Au{Avrachenkov~K., Yechiali~U.} Retrial networks with
    finite buffers and their application to Internet data traffic~// Prob.
Engng. Informational Sci., 2008. Vol.~22. P.~519--536.

\bibitem{MorNek}  %23
\Au{Морозов~Е.\,В., Некрасова~Р.\,С.} Оценивание
    вероят\-ности блокировки в системе с повторными вызовами и постоянной
    скоростью возвращения заявок с орбиты~// Труды Карельского научного центра
    РАН, 2011.  №\,5. С.~63--74.
    
    \bibitem{Gnedenko} %24
\Au{Гнеденко~Б.\,В., Коваленко~И.\,Н.} Введение  в
    теорию массового обслуживания.~--- М.: Наука, 1987. 336~с.

\bibitem{A96} %25
\Au{Artalejo~J.\,R.} Stationary analysis of the characteristics
of the $M/M/2$ queue with constant repeated attempts~// Opsearch,
1996. Vol.~33. P.~83--95.

\label{end\stat}
    
    \bibitem{AGN01} %26
\Au{Artalejo~J.\,R.,  G$\acute{\mbox{o}}$mez-Corral~A.,   Neuts~M.\,F.}
Analysis of multiserver queues with constant retrial rate~//
Eur. J.~Oper. Res., 2001. Vol.~135. P.~569--581.
 \end{thebibliography}
}
}


\end{multicols}    %11
\def\stat{mor-rum}

\def\tit{ВЕРОЯТНОСТНЫЕ МОДЕЛИ МНОГОПРОЦЕССОРНЫХ СИСТЕМ: СТАЦИОНАРНОСТЬ И~МОМЕНТНЫЕ СВОЙСТВА$^*$}

\def\titkol{Вероятностные модели многопроцессорных систем: стационарность и моментные свойства}

\def\autkol{Е.\,В.~Морозов,  А.\,С.~Румянцев}
\def\aut{Е.\,В.~Морозов$^1$,  А.\,С.~Румянцев$^2$}

\titel{\tit}{\aut}{\autkol}{\titkol}

{\renewcommand{\thefootnote}{\fnsymbol{footnote}}\footnotetext[1]
{Работа поддержана РФФИ (проект 10-07-00017). Работа выполнена
при поддержке Программы стратегического развития на 2012--2016~гг.\
<<Университетский комплекс ПетрГУ в научно-образовательном пространстве
Европейского Севера: стратегия инновационного развития>>.}}



\renewcommand{\thefootnote}{\arabic{footnote}}
\footnotetext[1]{Институт прикладных математических исследований КарНЦ 
РАН, Петрозаводский государственный университет,\linebreak emorozov@karelia.ru}
\footnotetext[2]{Институт прикладных математических исследований КарНЦ 
РАН, ar0@krc.karelia.ru}


\vspace*{-9pt}

\Abst{Дан анализ основных моделей многопроцессорных сис\-тем (МС), где
для обработки заявки требуется случайное число процессоров. Предложена
и исследована  новая модель таких  систем, в которой времена
обработки заданий данной заявки на  всех требуемых  процессорах
являются идентичными. Это предположение, отражающее реальный процесс
обработки,  существенно усложняет анализ. Для исследования данной
модели построены минорантная и мажорантная  (классические) модели, с
помощью которых для ряда важных частных случаев удалось получить как
условия стационарности исходной модели, так и моментные свойства
стационарного процесса нагрузки.}

\vspace*{-2pt}

\KW{многопроцессорные системы; групповое занятие
процессоров; идентичные времена обработки; условия стационарности;
моментные свойства; стационарный процесс нагрузки; вычислительный кластер}

\vspace*{-4pt}


\vskip 14pt plus 9pt minus 6pt

      \thispagestyle{headings}

      \begin{multicols}{2}

            \label{st\stat}
            
\section{Введение}

В настоящее время растет интерес к моделированию МС,  в первую очередь сис\-тем 
с массово-па-\linebreak раллельной
архитектурой, таких как, например,  вы\-чис\-ли\-тель\-ные клас\-те\-ры (ВК).
Для разработки эффектив\-ных алгоритмов управления очередями доступа к
МС нужна соответствующая (вероятностная) модель, которая позволяет
экспериментально определять качество обслуживания, обеспе\-чи\-ва\-емое
данной МС.  В~этой связи отметим работу~\cite{feit-coplot}, в
которой описаны особенности работы с МС и представлен хороший обзор
современных способов моделирования загрузки в таких системах.

Как известно, даже для классической системы $GI/G/m$  отсутствуют
явные формулы для   основных стационарных характеристик, а
асимптотические оценки, как правило, неточны~\cite{gupta10}. Еще
большие    трудности возникают  при моделировании современных МС.
Отметим ряд таких проблем. Достаточно часто  времена вычисления
задач в МС  адекватно описываются  распределениями с {\it тяжелыми
хвостами} (например, распределением Парето)~\cite{feit-coplot, gupta10}. 
Наличие тяжелых хвостов требует иного подхода, чем
традиционно используемый в теории очередей, и этот подход активно
развивается. Кроме того, как правило, некоторые характеристики
являются зависимыми (например, время вычисления задачи и ее
размер~\cite{feit-coplot,krampe10}). Также  во многих современных МС
условия конечности моментов незавершенной работы~\cite{scheller11},
распределение хвоста времени ожидания~\cite{foss06}, точность
асимптотических оценок~\cite{gupta10}, алгоритмы <<справедливого>>
распределения задач по процессорам~\cite{harchol01} существенно
зависят от коэффициента загрузки системы, в частности от наличия
так называемых <<резервных процессоров>>. Кроме того, важной
особенностью многих современных МС является свойство {\it долгой
памяти} входного процесса, т.\,е.\  расходимость ряда автокорреляций,
что существенно усложняет процедуру оценивания~\cite{feit-coplot, krampe10}. 
Поэтому анализ моделей, отражающих эти новые
аспекты современных  МС является весьма важным. В данной статье
проанализированы известные ранее модели МС с групповым занятием
процессоров приходящей заявкой. Кроме того, исследуется новый класс
таких моделей, где учитываются некоторые важные особенности
современных МС, а  также изучены моментные свойства процесса
загрузки в таких системах.

Статья организована следующим образом.  В~разд.~\ref{sec2} дан
 обзор  основных моделей МС, в которых каждой заявке требуется несколько
процессоров для обслуживания. Приведены    условия стационар\-ности
таких систем. Раздел~\ref{sec3} посвящен развитию модели ВК на
основе модифицированной рекурсии Ки\-фе\-ра--Воль\-фо\-ви\-ца, предложенной в
работах~[7--9]. В~част\-ности, на основе этой рекурсии строятся
классические системы, которые в определенном смысле являются
минорирующей и мажорирующей для исходной модели. Затем с помощью
этих систем получены
 условия  стационарности основной модели, а также найдены   моментные свойства
ее стационарного процесса  нагрузки. В~разд.~\ref{sec4}
рассматриваются  вопросы численного анализа данной  модели  на
основе реальных данных лог-фай\-ла кластера ЦКП КарНЦ РАН <<Центр
высокопроизводительной обработки данных>>.

\section{Многопроцессорные системы, в~которых для~обработки заявки 
требуется случайное число процессоров}\label{sec2}

Для дальнейшего анализа  важно  разделять сис\-те\-мы, в которых заявке
с номером $i$ одновременно требуется случайное число $N_i\hm\geqslant 1$
процессоров, на  системы с {\it независимым освобождением
процессоров} и системы с {\it одновременным освобождением
процессоров}. В~первом случае времена обслуживания на всех $N_i$
процессорах являются независимыми одинаково распределенными (н.\,о.\,р.),
а во втором случае  на всех $N_i$ процессорах используется одна и та
же реализация   времени обслуживания (т.\,е.\ времена обслуживания
{\it идентичны}). Системы второго  типа существенно более сложны для
анализа~\cite{green80-1}. Для таких систем   известны лишь
численные результаты~\cite{kim78},  а при отсутствии буфера~---
также некоторые аналитические результаты~[12--14].

Следуя работам~\cite{green80-1, green80, brill-green84}, рассмотрим важные
для дальнейшего анализа системы с независимым освобождением
процессоров. Рассмотрим систему
типа $M/M/m$ c интенсивностью входного (пуассоновского)  потока
$\lambda$, дисциплиной FIFO, где заявке $i$ требуется $N_i$
процессоров, на каждом из которых независимо реализуется
экспоненциальное  время обслуживания с параметром~$\mu$.
Предполагается, что  $\{N_i\}$ являются
н.\,о.\,р.\ случайными величинами
(с.\,в.)  с заданным распределением $p_k\hm=P(N=k)$, где  $N$~--- типичный
элемент последовательности. Для такой (изначально свободной) системы
в работе~\cite{green80} получено следующее {\it достаточное условие}
существования стационарного режима:
\begin{equation}
\label{green-stab3}
\lambda \sum\limits_{k=1}^m\sum_{j=0}^{k-1}
\fr{p_k}{\mu(m-j)}<1\,.
\end{equation}
В~\cite{brill-green84}  исследована  двухпроцессорная сис\-те\-ма с
двумя  независимыми пуассоновскими потоками с интенсивностями
$\lambda_i$, $i\hm=1,2$. Заявки первого класса обслуживаются на одном
процессоре, а заявки второго~--- на двух (с одинаковой
интенсивностью~$\mu$). Очевидно, эта модель является частным случаем
предыдущей. С помощью минорантной системы $M/M/2$ (c заявками только
1-го типа) и мажорантной системы типа $M/M/1$ (с заявками только
2-го типа) в работе~\cite{brill-green84} получено стационарное
распределение исходной системы, а также {\it критерий ее
стационарности}  в виде
\begin{equation}
\label{green-stab4}
 \fr{\lambda}{2\mu}(2-p_{1}^{2})<1\,.
\end{equation}
(Заметим, что~\eqref{green-stab3} влечет~\eqref{green-stab4}.) 
В~работе~\cite{green80-1} обсуждается неоптимальность  дисциплины FIFO
в рассматриваемых моделях, поскольку  некоторые процессоры могут
простаивать при наличии очереди. (Иными словами, используемая
дисциплина не является сохраняющей работу.)  В~этом состоит основная
трудность аналитического исследования таких моделей МС.

Модель с независимым освобождением процессоров может описывать
поведение узлов вычислительной грид-сис\-те\-мы (объединения
слабосвязанных вычислителей), так как в этом случае времена вычисления
заданий, запускаемые на отдельных узлах и связанные с данной
заявкой, являются  независимыми.

В работе~\cite{kaufman81} описана система хранения с
потерями, в которой на вход в блочное устройство хранения данных
поступают пуассоновские потоки заявок $k$ классов. Заявке~$i$ класса~$k$ 
требуется случайное чис\-ло $b_i$ единиц хранения на случайное время~$t_i$. 
(Эти величины имеют  заданные распределения, зависящие от~$k$.) 
Отказ в обслуживании (уход без возвращения) определяется при
помощи множества допустимых состояний $\Omega\hm=\{(n_{1},\ldots,
n_{k})\}$, описывающих распределение  числа заявок по классам. 
В~~\cite{kaufman81}  получено стационарное распределение состояний
сис\-те\-мы для произвольного множества~$\Omega$ и произвольных
распределений  времени хранения. Если~$\Omega$ допускает полное
разделение ресурсов (все ресурсы могут быть заняты любым классом
заявок), то в~\cite{kaufman81} получена эффективная с
вычислительной точки зрения рекурсия для приближенного расчета
вероятности отказа. В~работе~\cite{whitt85} рассматриваются вопросы
аппроксимации стационарного распределения для данной модели.
Описанная модель может быть полезна для моделирования ВК, в котором
пользователи  часто разделяются на группы с разными правами и
ограничениями на доступ к ресурсам. Вместо блочного устройства можно
рассматривать МС, в которой заявки занимают одновременно несколько
процессоров.

Современные ВК имеют сотни тысяч процессоров, но, как правило,
решаемые на них задачи не имеют столь же высокой степени
распараллеливания. Поэтому в качестве подходящей модели таких ВК
обосновано рассматривать системы с бесконечным числом процессоров и
групповым поступлением заявок. (Такие  модели  адекватно описывают
системы  с большим числом процессоров~\cite{Borovkov, Kovalenko}.)

В работе~\cite{liu-templeton} рассмотрена система с бесконечным
чис\-лом процессоров, в которую поступает поток групп заявок,
 представляющий собой процесс восстановления. Группа~$n$ характеризуется  {\it
 маркой}
$Y_{n}$, причем последовательность $\{Y_{n}\}$ образует цепь
Маркова. Группа~$n$ требует выполнения $N_{n}$ задач, причем
распределение~$N_{n}$ зависит  от пары $(Y_{n},\, Y_{n-1})$, а
времена вычисления  (в каждой группе) являются н.\,о.\,р. с.\,в. В~этой
модели процессоры освобождаются независимо, и поэтому она также может
быть использована  для описания вычислительной грид-сис\-те\-мы.
Важнейшей характеристикой является также
 число процессоров $\nu_{i}(n)$, занятых в момент прихода $n$-й заявки
класса $i\in[1,M]$, $n\hm\ge 1$. В~работе~\cite{liu-templeton} найдены
автокорреляционные функции числа занятых процессоров для
непрерывного времени и  в моменты прихода заявок каждого класса.

 В работе~\cite{tihonenko}  получено  преобразование Лап\-ла\-са--Стил\-ть\-еса
функции распределения стационарного числа $\nu$ занятых процессоров
в системе типа $M/G/\infty$ (c интенсивностью входного потока~$\lambda$  
и временем обслуживания~$S$), где каждой заявке
(единственного) класса требуется случайное  чис\-ло~$N$ процессоров (с
заданным распределением). Показано,  что  условие стационарности
имеет вид
\begin{equation}
\label{tihon1}
    \rho:=\lambda \e N\, \e S<\infty
\end{equation}
и что  $\e\nu =\rho$. (Последний  результат хорошо известен в случае
ординарного потока, т.\,е.\ при $\p(N=1)\hm=1$.) В~работе~\cite{eliazar}
изучаются моментные свойства,  свойство долгой
памяти  процесса величины очереди в системе $M/G/\infty$, а в
работе~\cite{daley-busy} исследуется  свойство долгой памяти
периодов занятости такой системы. В~\cite{brandt} получено
стационарное распределение  числа занятых процессоров в моменты
прихода в системе с групповым поступлением (размер группы~---
постоянная величина) и  независимым экспоненциальным временем
обслуживания каждой заявки внутри группы. Наконец, отметим
работу~\cite{krampe10}, которая содержит обзор име\-ющих\-ся в
литературе моделей мас\-со\-во-па\-рал\-лель\-ных МС, а также анализ некоторой новой
модели на основе марковских цепей.

\section{Модель вычислительного кластера}\label{sec3}

В данном разделе изучается модель МС, в которой новая заявка
занимает несколько процессоров на {\it идентичное} время, что
существенно усложняет  анализ по сравнению со случаем независимых
времен обработки. (Анализ этой модели был начат  в
работах~\cite{pavt11, rudn11}.)  В~данной модели в момент
освобождения нескольких процессоров на обслуживание может поступить
одновременно несколько заявок и это не позволяет  применить анализ
из работы~\cite{green80}.

Рассмотрим следующее обобщение классической  $m$-про\-цес\-сор\-ной
системы $GI/G/m$ с н.\,о.\,р.\ интервалами между заявками $\{T_n\}$ и
н.\,о.\,р.\ временами обслуживания $\{S_n\}$, в которой $i$-й приходящей
заявке требуется одновременно случайное число процессоров $N_{i}\hm\in[1,\, m]$.
Если число свободных процессоров меньше~$N_i$, то
она ожидает в буфере освобождения недостающего числа процессоров.
Соответствующая модификация рекурсии Ки\-фе\-ра--Воль\-фо\-ви\-ца для вектора
процесса нагрузки $W_i:=(W_{i}(1),\cdots, W_{i}(m))$~\cite{Kiefer}
принимает вид:
\begin{multline}
\label{our-rec}
 \hspace*{-6.13364pt}W_{i+1}=R\left(W_{i}(N_{i})+S_{i}-T_{i},\ldots,W_{i}(N_{i})
+S_{i}-T_{i},\right.\\ 
\left.W_{i}(N_{i}+1)-T_{i},\ldots,
W_{i}(m)-T_{i}\right)^{+},
\end{multline}
где оператор~$R$ упорядочивает компоненты в порядке возрастания,
$(\cdot)^{+}\hm=\max(0,\cdot)$ (для вектора операция применяется покомпонентно).
Заметим, что первые $N_{i}$ компонент вектора одинаковы, так как
заявка~$i$ занимает сразу  $N_{i}$ процессоров. (Если в момент
прихода $k\hm>N_{i}$ процессоров свободны, то заявка не ждет в
очереди.)  По условию, каждая заявка освобождает все занимаемые ею
процессоры одновременно. Обозначим через $D_{i}:=W_{i}(N_{i})$ время
ожидания заявки~$i$ в очереди.

Мы будем использовать так называемый кап\-линг-ме\-тод, позволяющий
сравнивать траектории случайных процессов~\cite{Shiryaev}. Нетрудно
доказать следующее  утверждение об операторе~$R$ из~\eqref{our-rec}.
(Доказательства можно также найти в~[26--28].)

\medskip

\noindent
\textbf{Лемма~3.1.}
%\begin{lem}\label{lem-sort}
\textit{Пусть векторы $X\leqslant Y$ (покомпонентно).
Тогда}
\begin{equation*}
%\label{lem-form}
R(X)\leqslant R(Y)\,.
\end{equation*}

Ниже доказано, что  для предложенной модели ВК минорантной  будет
система, в которой $i$-я заявка заменяется на группу из  $N_{i}$
заявок, причем каждая заявка  из данной группы  имеет одно и то же
время обслуживания~$S_{i}$. (Таким образом, очередная заявка из
группы немедленно занимает освободившийся процессор.) Обозначим эту
систему через $\Sigma^{(\mathrm{low})}$ и снабдим ее характеристики верхним
индексом low. Отметим, что в системе $\Sigma^{(\mathrm{low})}$   на один
процессор может распределяться  несколько заявок  из группы.

\medskip

\noindent
\textbf{Теорема 3.2}. 
\textit{Пусть $W_0^{(\mathrm{low})}\hm=W_0\hm=0$. Тогда}
\begin{equation}
W_{i+1}^{(\mathrm{low})}\leqslant W_{i+1}\,,\quad i\ge 0\,.\label{e9-mr}
\end{equation}

\medskip

\noindent
Д\,о\,к\,а\,з\,а\,т\,е\,л\,ь\,с\,т\,в\,о\,.\
Будем считать, что в момент прихода  заявки~$i$ процессоры
нумеруются в порядке возрастания нагрузки,  как и компоненты вектора~$W_i$ 
в результате применения оператора~$R$. Ввиду эквивалентности
процессоров такая процедура не меняет вероятностных свойств процесса
нагрузки. В~сис\-те\-ме  $\Sigma^{(\mathrm{low})}$ на процессор с номером $N_{i}$
из  $i$-й группы  может быть распределено не более одной заявки.
Действительно, предположим, что на процессор~$N_{i}$ распределено не
менее двух заявок (со временем обслуживания~$S_{i}$). Тогда на
некоторый процессор с номером $k\hm<N_{i}$ не поступает  заявка. Это
означает, что после  размещения заявки на процессоре~$N_{i}$ 
выполнено неравенство
$W_{i}^{(\mathrm{low})}(k)\hm>W_{i}^{(\mathrm{low})}(N_{i})\hm+S_{i}$. Но это противоречит
условию $W_{i}^{(\mathrm{low})}(k)\hm\leqslant W_{i}^{(\mathrm{low})}(N_{i})$. Далее,
если на некоторый процессор с номером $k\hm<N_{i}$ распределено $n\hm\in
(1,\, N_{i}]$ заявок, то
    \begin{equation}
    \label{minor-ntasks}
        W_{i}^{(\mathrm{low})}(k)+nS_{i}\leqslant W_{i}^{(\mathrm{low})}(N_{i})+S_{i}\,.
    \end{equation}
Действительно, в момент распределения заявки с номером $n\hm\leqslant
N_{i}$ из $i$-й группы на процессор~$k$ должно быть выполнено
неравенство $W_{i}^{(\mathrm{low})}(k)\hm+(n-1)S_{i}\hm\leqslant
W_{i}^{(\mathrm{low})}(N_{i})$.

Так как  обе системы изначально свободны, то
до применения оператора~$R$ первые $N_{i}$ компонент векторов
$W_{0}^{(\mathrm{low})}$ и~$W_{0}$ совпадают. Предположим (по индукции), что
$W_{i}^{(\mathrm{low})}\hm\leqslant W_{i}$ для некоторого~$i$.
  Поскольку процессор с номером~$N_{i}$ в
  сис\-те\-ме~$\Sigma^{(\mathrm{low})}$ получит не более одной заявки, то в силу
предположения индукции
\begin{multline*}
(W_{i}^{(\mathrm{low})}(N_{i})-T_{i})^{+}\leqslant
(W_{i}^{(\mathrm{low})}(N_{i})+S_{i}-T_{i})^{+} \leqslant{}\\
{}\leqslant
(W_{i}(N_{i})+S_{i}-T_{i})^{+}\,,
\end{multline*}
а для   компонент c номерами  $k\hm<N_i$ по
свойству~\eqref{minor-ntasks} и предположению индукции следует
\begin{multline*}
(W_{i}^{(\mathrm{low})}(k)+nS_{i}-T_{i})^{+} \leqslant
(W_{i}^{(\mathrm{low})}(N_{i})+S_{i}-T_{i})^{+}\leqslant{}\\
{}\leqslant
(W_{i}(N_{i})+S_{i}-T_{i})^{+}\,.
\end{multline*}
Теперь,  применив оператор~$R$,  в силу леммы~3.1
получаем~\eqref{e9-mr}.


\medskip

\noindent
\textbf{Замечание.}
Сис\-те\-ма~$\Sigma^{(\mathrm{low})}$ в действительности  может быть
использована как модель  вычислительной грид-сис\-те\-мы. На практике
поступающая  на грид заявка  часто является группой заданий,
требующих  перебора в пространстве параметров модели.
Задания в группе, как правило, независимы и поступают    на
свободные  вычислители, не дожидаясь одновременного освобождения
всех требуемых данной группе вычислителей. Полученный выше результат
показывает, что для подобных задач переборного типа целесообразнее
использовать архитектуру грид-сис\-те\-мы, чем ВК, так как время ожидания
заявки в такой системе в среднем оказывается меньше.

\smallskip

Поскольку в настоящее время  мощности ВК  исчисляются сотнями тысяч
процессоров, а  задач, масштабируемых на такое количество процессоров, немного, то
ограничение  $\p(N\leqslant N_{\max})\hm=1$ для некоторого $N_{\max}\hm\ll
m$ представляется вполне мотивированным. Обозначим
\begin{equation}
\left.
\begin{array}{rl}
    j&=\min\left\{ k\geqslant 1:
    \p\left(N\leqslant\left\lfloor\fr{m}{k}\right\rfloor\right)=1\right\}\,,\\[9pt]
    N_{\max}&=\left\lfloor\fr{m}{j}\right\rfloor\,.
    \end{array}
    \right\}
    \label{e8-mr}
\end{equation}
(Здесь $\lfloor x \rfloor$ означает наибольшее целое число, не превосходящее~$x$.)
Случай $j\hm=1$ (когда каждой заявке разрешено занимать все процессоры)
типичен для небольших ВК.

Как показано ниже, для исходной  системы мажорантной будет система
(в которой соответствующие величины снабжены  верхним индексом
up), где каждая заявка занимает ровно
$N_{\max}\hm=\lfloor{m}/{j}\rfloor$ процессоров. Для такой системы
рекурсия~\eqref{our-rec} примет вид
\begin{multline}
W^{(\mathrm{up})}_{i+1}= R\left(  W^{(\mathrm{up})}_{i}(N_{\max})+S_{i}-T_{i},\ldots\right.\\
\ldots ,W^{(\mathrm{up})}_{i}(N_{\max})+S_{i}-T_{i},
W^{(\mathrm{up})}_{i}(N_{\max}+1)-T_{i},\ldots\\
\left.\ldots,W^{(\mathrm{up})}_{i}(m)-T_{i}\right)^{+}\,.
\label{up-rec}
\end{multline}

\noindent

\textbf{Теорема~3.3.} 
\textit{Пусть $W_{0}=W^{(\mathrm{up})}_{0}=0$. Тогда}
\begin{equation}
W_{i+1}\leqslant W^{(\mathrm{up})}_{i+1}\,,\enskip i\geqslant 0\,.
\label{e11-mr}
\end{equation}


\noindent
Д\,о\,к\,а\,з\,а\,т\,е\,л\,ь\,с\,т\,в\,о\,.\
Очевидно, при нулевых начальных условиях неравенство $W_{1}\hm\leqslant
W^{(\mathrm{up})}_{1}$ выполнено. Предположим, что $W_{i}\hm\leqslant
W^{(\mathrm{up})}_{i}$ для некоторого $i\hm>1$.  Рассмотрим
рекурсии~\eqref{our-rec},~\eqref{up-rec} до применения оператора~$R$
и докажем индуктивный переход. Для компонент с номерами $1\hm\leqslant
k\hm\leqslant N_{i}$ имеем
\begin{equation*}
%\label{monotone-up1}
(W_{i}(N_{i})+S_{i}-T_{i})^{+}\leqslant (W^{(\mathrm{up})}_{i}(N_{\max})+S_{i}-T_{i})^{+}\,.
\end{equation*}
Для компонент с номерами $N_{i}+1\hm\leqslant k\hm\leqslant N_{\max}$
получаем неравенства
\begin{equation*}
%\label{monotone-up2}
(W_{i}(k)-T_{i})^{+}\leqslant
(W^{(\mathrm{up})}_{i}(N_{\max})+S_{i}-T_{i})^{+}\,,
\end{equation*}
а для компонент с номерами $k\hm> N_{\max}$ выполнены неравенства
\begin{equation*}
(W_{i}(k)-T_{i})^{+}\leqslant (W^{(\mathrm{up})}_{i}(k)-T_{i})^{+}\,.
\end{equation*}
В силу леммы~3.1 это влечет ~\eqref{e11-mr}.

\medskip

На самом деле каждую группу из~$N_{\max}$ процессоров можно считать
одним процессором, так как все они занимаются и освобождаются  заявкой
одновременно. Таким образом,  мажорирующая система эквивалентна
стандартной системе обслуживания $GI/G/j$ (с теми же интервалами
между приходами и  временами обслуживания). Хорошо известное
(достаточное) условие стационарности такой сис\-темы
\begin{equation}
\lambda \e S<j 
\label{e18-mr}
\end{equation}
является, таким образом, также  условием стационарности модели ВК.
(При $m\to\infty$ условие~\eqref{e18-mr} переходит в условие
стационарности $\lambda \e S\hm<\infty$ модели с бесконечным числом
процессоров для ординарного входного потока (см.\ \eqref{tihon1}).)

Получим теперь  достаточное условие неустойчивости (нестационарности)
 сис\-те\-мы $\Sigma^{(\mathrm{low})}$ (опус\-тив для простоты в обозначениях  индекс low).
 Обозначим через $A(t)$ число приходов в
систему $\Sigma^{(\mathrm{low})}$ в интервале $[0,t]$. (Это процесс
восстановления с интенсивностью $\lambda\hm=1/\e T$.) Очевидно, что
$M(t)\hm=\sum\limits_{i=1}^{A(t)}N_{i}$ есть суммарное число процессоров,
требуемых заявкам, поступившим  в интервале $[0,t]$. Пусть $D(t)$
обозначает число уходов (освобожденных процессоров) в сис\-те\-ме
$\Sigma^{(\mathrm{low})}$ в интервале $[0,t]$. Тогда $\nu(t)\hm=M(t)\hm-D(t)$  есть
чис\-ло процессоров, требуемых заявкам, находящимся в сис\-те\-ме
$\Sigma^{(\mathrm{low})}$ в момент~$t$. Обозначим $\rho\hm=\lambda \e N\e S$.

\medskip

\noindent
\textbf{Лемма 3.4.}
\textit{Если  $\rho\hm>m$, то $\nu(t)\to \infty$ c вероят\-ностью~1.}
\medskip

\noindent
Д\,о\,к\,а\,з\,а\,т\,е\,л\,ь\,с\,т\,в\,о\,.\
Обозначим через $\hat D(t)$ число уходов из системы $\Sigma^{(\mathrm{low})}$
в интервале $[0,t]$ в предположении, что {\it каждый  процессор
работает без простоев}.  Таким образом, $\{\hat D(t),\,t\hm\ge0\}$ есть
суперпозиция $m$ независимых процессов восстановления, каждый с
интенсивностью $\mu\hm=1/\e S$. Очевидно, $\hat D(t)\hm\ge D(t)$, $t\hm\ge0$,
и поэтому $\nu(t)\hm\geqslant M(t)\hm-\hat D(t)$.  Кроме того,
\begin{equation*}
\fr{M(t)}{t}=\fr{A(t)}{t}\, \fr{\sum_{i=1}^{A(t)} 
N_{i}}{A(t)} \to \lambda \e N\,,\quad t\to \infty\,.
\end{equation*}
Из теории восстановления следует, что c вероят\-ностью~1
\begin{equation*}
\fr{\hat D(t)}{t}\to \mu m\,,\quad t\to\infty\,.
\end{equation*}
Поэтому
\begin{equation*}
\liminf \fr{\nu(t)}{t}\geqslant  \mu(\rho-m)>0\,.
\end{equation*}

\medskip

Таким образом, необходимым условием устойчивости минорантной сис\-те\-мы
$\Sigma^{(\mathrm{low})}$, а следовательно, и исходной модели ВК является
условие $\rho\hm<m$.

\smallskip

\noindent
\textbf{Замечание}.  Полученный результат означает {\it сильную
неустойчивость}, в отличие от {\it слабой неустой\-чивости}, когда
неограниченный рост очереди \mbox{происходит} по вероятности. Различные
виды неустойчивости  процессов обслуживания рассматриваются,
например, в~[29--31]. В~част\-ности,
в~\cite{MorozovJMS} с использованием регенеративного
анализа показано, что минорирующая система слабо неустойчива при  $\rho\hm=m$.

По аналогии с работой~\cite{scheller-sigman97} построим рекурсию для
({\it скалярной}) компоненты $D_{i}\hm=W_{i}(N_{i})$, явля\-ющей\-ся
задержкой (временем ожидания в очереди)  заявки~$i$. Пусть для
сис\-те\-мы ВК~(\ref{our-rec})
\begin{equation}
\left.
\begin{array}{rl}
P_i&:=W_i(N_i+N_{i+1})-W_i(N_i)\,,\\[9pt]
Q_i&:=W_i(N_{i+1})-W_i(N_i)\,,\\[9pt]
(P_i&:=\infty\;\; \mbox{при}\;\;
N_i+N_{i+1}>m)\,,\\[9pt]
U_i&:=\max\left (Q_i,\, \min(P_i,\,
S_i)\right)={}\\[9pt]
&{}\hspace*{10mm}=\min\left (P_i,\, \max(Q_i,\, S_i)\right)\,.
\end{array}
\right\} 
\label{e17-mr}
\end{equation}

\medskip

\noindent
\textbf{Теорема~3.5.}  \textit{В~модели ВК величина задержки
удовлетворяет рекурсии}
\begin{equation}
\label{delay-onestep} 
D_{i+1}=(D_{i}+U_{i}-T_{i})^{+}\,,\enskip i\ge 0\,.
\end{equation}

\smallskip

\noindent
Д\,о\,к\,а\,з\,а\,т\,е\,л\,ь\,с\,т\,в\,о\,.\
В нижеследующем анализе  нумерация процессоров соответствует их
состоянию {\it перед приходом}  заявки. Нетрудно увидеть, что для
любого~$i$
\begin{equation}
\label{pq}
P_i\geqslant Q_i\,.
\end{equation}
Рассмотрим возможные случаи.
\begin{enumerate}[1.] 
\item Пусть
$N_{i}+N_{i+1}\hm\leqslant m$ и выполнено неравенство
$(W_{i}(N_{i}+N_{i+1})-T_{i})^{+}\hm\leqslant
(W_{i}(N_{i})+S_{i}-T_{i})^{+}$. Это означает, что в момент прихода
заявки $i+1$ процессоры с номерами $1,\ldots,N_{i}$ заняты заявкой~$i$ 
и для обслуживания заявки $i+1$ будут использованы процессоры с
номерами $N_{i}+1,\ldots, N_{i}+N_{i+1}$.  Поэтому
\begin{equation}
D_{i+1}=(W_{i}(N_{i}+N_{i+1})-T_{i})^{+}\,,\enskip i\ge 0\,.
\label{e19-mr}
\end{equation}
Поскольку в данном  случае $W_{i}(N_{i}+N_{i+1})\hm\leqslant
W_{i}(N_{i})+S_{i}$, то с учетом~\eqref{pq} выполнено неравенство
$Q_{i}\hm\leqslant P_{i}\hm\leqslant S_{i}$. Поэтому с учетом~(\ref{e17-mr})
$U_{i}\hm=P_{i}$
и, следовательно,~\eqref{e19-mr} влечет~\eqref{delay-onestep}.
\item
Пусть теперь  $N_{i+1}\hm>N_{i}$ и
$(W_{i}(N_{i+1})\hm-T_{i})^{+}\hm>(W_{i}(N_{i})\hm+S_{i}-T_{i})^{+}$. Это
означает, что в момент прихода заявки $i+1$ процессоры с номерами
$1,\ldots,N_{i}$ заняты заявкой~$i$. Однако   для заявки $i+1$ этих
процессоров недостаточно, и она ожидает  освобождения самого
загруженного (из требуемых ей) процессора с номером $N_{i+1}$. 
В~этом случае
\begin{equation}
D_{i+1}=(W_{i}(N_{i+1})-T_{i})^{+}\,,\enskip i\ge 0\,.
\label{e20-mr}
\end{equation}
Тогда с учетом~\eqref{pq} выполнено $P_{i}\hm\geqslant Q_{i}\hm\geqslant
S_{i}$. Следовательно, $U_{i}\hm=Q_{i}$ и~\eqref{e20-mr} снова
влечет~\eqref{delay-onestep}.

Полезно отметить, что эти два рассмотренных случая несовместны, т.\,е.\
\begin{multline*}
\left\{W_i(N_{i+1})>W_i(N_i)+S_i\right\}\cap\\
\cap
\left\{W_i(N_i+N_{i+1})<W_i(N_i)+S_i\right\}=\varnothing\,.
\end{multline*}
\item
Наконец, рассмотрим ситуацию, когда ни один из рассмотренных выше случаев
не имеет места. Тогда, как нетрудно понять, величина задержки
определяется нагрузкой на любом из процессоров с номерами
$1,\ldots,N_{i}$, т.\,е.\
\begin{multline}
D_{i+1}=(W_{i}(N_{i})+S_{i}-T_{i})^{+}={}\\
{}=(D_{i}+S_{i}-T_{i})^{+}\,,\enskip i\ge0\,.
\label{e21-mr}
\end{multline}
Легко проверить, что в данном случае  $U_{i}\hm=S_{i}$  и поэтому~\eqref{e21-mr} 
влечет~\eqref{delay-onestep}.
\end{enumerate}

\smallskip

Заметим, что в  классической системе GI/G/m $N_{i}\hm\equiv 1$ и
поэтому $Q_i\hm=0$, а условие $N_{i}+N_{i+1}\hm\leqslant m$ всегда верно
для $m\hm\geqslant 2$. Поэтому $P_{i}\hm=W_{i}(2)\hm-W_{i}(1)$,  $U_{i}\hm=\min
(P_{i},S_{i})$ и, как легко проверить,  рекурсия из~\cite{scheller-sigman97}  
оказывается частным случаем рекурсии~\eqref{delay-onestep}.  
А~поскольку в классической сис\-те\-ме $D_i\hm=W_{i}(1)$, то
\begin{multline*}
\hspace*{-9.687pt}D_{i+1}=\min\left(\left(W_{i}(1)+S_{i}-T_{i}\right)^{+},\,\left(W_{i}(2)-T_{i}\right)^{+}\right)={}\\
{}=
\left(D_{i}+U_{i}-T_{i}\right)^{+}\,.
\end{multline*}
Заметим, что  доказанная в теореме~3.3 монотонность
верна и для соответствующих моментов компонент вектора нагрузки в
модели ВК. Обозначим через $W\hm=(W_1,\ldots,W_m)$ стационарный вектор
нагрузки в модели ВК, т.\,е.\ предполагаем, что слабый предел
$W_n\hm\Rightarrow W$ существует.

 Напомним
обозначения~\eqref{e8-mr} и обозначим также $k(i)\hm=\lfloor
i/N_{\max}\rfloor$, $1\hm\leqslant i\hm\leqslant m$.
 Прямым следствием теоремы~3.3 являются следующие
моментные свойства стационарного вектора нагрузки для модели ВК,
полученные в работе~\cite{scheller11} для системы $GI/G/j$.

\medskip

\noindent
\textbf{Теорема~3.6.}
\textit{Пусть $\rho:=\e S/\e T<j$ и   $\alpha\geqslant 1$.
Тогда имеют место
    следующие импликации:}
\begin{enumerate}[1.]
\item \textit{Для компонент вектора~$W$ с индексами 
$1\hm\leqslant i\hm\leqslant \lceil\rho\rceil N_{\max}$}
        \begin{equation*}
%        \label{l5-mr}
        \e S^{1+{\alpha}/({j-\lfloor\rho\rfloor})}<\infty\  \Rightarrow
        \e W_i ^\alpha<\infty\,.
        \end{equation*}
\item \textit{Для компонент вектора~$W$ с индексами $\lceil\rho\rceil N_{\max}<i\hm\leqslant m$}
        \begin{equation*}
        \e S^{1+{\alpha}/({j-k(i)})}<\infty  \Rightarrow   \e W_i^\alpha<\infty\,.
        \end{equation*}
\end{enumerate}

\smallskip

В случае $j=1$, $N_{\max}\hm=m$ все компоненты  вектора загрузки имеют
одинаковые моментные свойства, а~именно: условие  $\e
S^{\,\alpha+1}\hm<\infty$ влечет $\e W_i^\alpha\hm<\infty$ для $1\hm\leqslant i\hm\leqslant m$. 
(Это классический результат из работы~\cite{Kiefer1}.) Следствием теоремы~3.6 являются
моментные свойства стационарной  задержки заявки в модели~ВК.


\medskip

\noindent
\textbf{Теорема~3.7.} \textit{Пусть выполнены условия теоремы~3.6.
    Тогда}
    \begin{equation*}
    \e S^{1+{\alpha}/({j-\lfloor \rho\rfloor})}<\infty\Rightarrow \e D^{\,\alpha}<\infty\,.
%    \label{e23-mr}
    \end{equation*}

\medskip


\noindent
Д\,о\,к\,а\,з\,а\,т\,е\,л\,ь\,с\,т\,в\,о\,.\
 Заметим, что  первые
$\lceil\rho\rceil N_{\max}${} компонент  вектора нагрузки имеют
одинаковые моментные свойства. Далее, поскольку $\lceil\rho\rceil
N_{\max}\hm\geqslant N_{\max}$ при любом $\rho\hm>0$, то условие
$P(N\hm\leqslant N_{\max})\hm=1$, следующее из~\eqref{e8-mr}, гарантирует, что
задержка $D_{i}:=W_{i}(N_{i})$ окажется среди первых
$\lceil\rho\rceil N_{\max}$ координат вектора нагрузки. В~то же
время условие $\e D^{\,\alpha}\hm<\infty$  для этих компонент  следует
из теоремы~3.6.


\section{Заключение}\label{sec4}

В  статье  дан краткий обзор основных  моделей МС, в которых  заявке
требуется  для обслуживания случайное число процессоров.   Кроме того,
 приведена новая модель ВК на основе модифицированной
рекурсии Ки\-фе\-ра--Воль\-фо\-ви\-ца. Эта модель исследована  с по\-мощью
классических многоканальных систем, одна из которых является
минорантной, а другая   мажорантной для процесса нагрузки в  модели
ВК. На основе этого подхода, в частности, получены   достаточные
условия стационарности, условие (сильной) неустойчивости, а также
моментные свойства компонент стационарного вектора нагрузки в модели~ВК.

На основе данных лог-фай\-ла запусков задач кластера ЦКП КарНЦ
РАН~\cite{cluster} была проведена апробация модели ВК. (Анализ
некоторых численных результатов, относящихся к данной модели,
представлен также в работах~\cite{krc11, aptpms11, hpc11}.)
Эксперименты проводились с использованием разработанного авторами
пакета расширения для вычислительной среды~R. Сформулируем  краткие
выводы из проведенных    исследований, которые   согласуются с
результатами авторитетных исследований в данной
области~\cite{feit-coplot}. В~част\-ности, обнаружено, что интервалы
между приходами заявок хорошо описываются с помощью лог-нормального
распределения ({\it с тяжелым хвостом}). Для моделирования времени
обслуживания заявок использовалось усеченное распределение Парето,
что соответствует принятой практике~\cite{gupta10}.  Обнаружена
зависимость времени обработки~$S_i$  от числа требуемых процессоров
$N_{i}$ (что, однако, не отражено в рассматриваемой модели ВК). Кроме
того, (визуально) обнаружено медленное убывание автокорреляционной
функции последовательности~$N_{i}$, что  может отражать наличие
тяжелого хвоста у  распределения с.\,в.~$N$.
 В~ходе исследования также обнаружено  доминирование значений
$N_i\hm=2^k$ для   $k\hm=0,\ldots,8$
 (около 87\%). (В~этой связи   укажем  работы~\cite{krampe10, downey99}.)
Наконец отметим, что почти 50\% всех задач были
однопроцессорными. В~целом модель показала хорошее согласие с
экспериментальными данными лог-фай\-ла, что говорит об определенном
потенциале ее практического применения для анализа существующих и
проектирования новых~МС.

{\small\frenchspacing
{%\baselineskip=10.8pt
\addcontentsline{toc}{section}{Литература}
\begin{thebibliography}{99}

\bibitem{feit-coplot}  
\Au{Talby D., Feitelson D., Raveh A.} 
A co-plot analysis of logs and models of parallel workloads~// 
ACM Transactions on Modeling and Computer Simulation (TOMACS), 2007. Vol.~17. No.\,3. Article~12.

\bibitem{gupta10} 
\Au{Gupta V., Harchol-Balter M., Dai J.\,G., Zwart~B.} 
On the inapproximability of M/G/K: Why two moments of job size distribution are not enough~// 
Queueing Syst., 2010. Vol.~64. P.~5--48.

\bibitem{krampe10} 
\Au{Krampe A., Lepping J., Sieben~W.} 
A~hybrid Markov chain modeling architecture for workload on parallel computers~// 
HPDC'10: 19th ACM  Symposium (International) on High Performance Distributed Computing
Proceedings.~--- New York: ACM, 2010. P.~589--596.

\bibitem{scheller11} 
\Au{Scheller-Wolf A., Vesilo~R.} 
Sink or swim together: Necessary and sufficient conditions for finite moments of workload components in FIFO 
multiserver queues~// Queueing Syst., 2011. Vol.~67. No.\,1. P.~47--61.

\bibitem{foss06} 
\Au{Foss S., Korshunov D.} 
Heavy tails in multi server queue~// Queueing Syst., 2006. Vol.~52. No.\,1. P.~31--48.

\bibitem{harchol01} 
\Au{Harchol-Balter M., Schroeder~B.} 
Evaluation of task assignment policies for supercomputing servers: The 
case for load unbalancing and fairness~// HPDC'00: 9th IEEE Symposium on High 
Performane Distributed Computing Proceedings.~--- New York: ACM, 2001.
P.~211--219.

\bibitem{pavt11} 
\Au{Морозов Е.\,В., Румянцев А.\,С.} 
Некоторые модели многопроцессорных систем обслуживания с тяжелыми хвостами~// 
Параллельные вычислительные технологии 2011: Сборник трудов междунар. научн. конф.~--- 
Челябинск: ЮУрГУ, 2011. С.~555--566.

\bibitem{rudn11} 
\Au{Румянцев А.\,С.} 
О~стохастическом моделировании вычислительного кластера~// 
Ин\-фор\-ма\-ци\-он\-но-те\-ле\-ком\-му\-ни\-ка\-ци\-он\-ные 
технологии и математическое моделирование высокотехнологичных систем: 
Тезисы докладов Всеросс. конф. с международным учас\-ти\-ем (18--22~апреля 2011).~--- 
М.: РУДН, 2011. С.~46--47.

\bibitem{krc11} 
\Au{Морозов Е.\,В., Румянцев А.\,С.} 
Модели многосерверных систем для анализа вычислительного кластера~// 
Труды Карельского научного центра Российской академии наук, 2011. №\,5. С.~75--86.

\bibitem{green80-1} 
\Au{Green L.} 
Comparing operating characteristics of queues in which customers require a random number of servers~// 
Management Sci., 1980. Vol.~27. No.\,1. P.~65--74.

\bibitem{kim78} %11
\Au{Kim S.} $M/M/s$ queueing system where customers demand multiple server use. 
Ph.D.\ Dissertation.~--- Southern Methodist University, 1979.

\bibitem{kaufman81}  %12
\Au{Kaufman J.} Blocking in a shared resource environment~// 
IEEE Transactions on Communications, 1981. Vol.~29. No.\,10. P.~1474--1481.

\bibitem{whitt85} %13
\Au{Whitt W.} Blocking when service is required from several facilities simultaneously~// 
AT\&T Techn.~J., 1985. Vol.~64. No.\,8. P.~1807--1856.


\bibitem{dijk88} %14
\Au{Van Dijk N., Smeitink E.} A non-exponential queueing system with batch servicing~// 
Researchmemorandum.~--- Amsterdam: Free University, 1988. No.\,13.

\bibitem{green80} %15
\Au{Green L.} A queueing system in which customers require a random number of servers~// 
Operations Res., 1980. Vol.~28. No.\,6. P.~1335--1346.

\bibitem{brill-green84}  %16
\Au{Brill P., Green L.} Queues in which customers receive simultaneous service from 
a random number of servers: A~system point approach~// Management Sci., 1984. Vol.~30. No.\,1. P.~51--68.

\bibitem{Borovkov} 
\Au{Боровков А.\,А.} Вероятностные процессы в теории массового обслуживания.~---  М.: Наука, 1972.

\bibitem{Kovalenko}  
\Au{Гнеденко Б.\,В., Коваленко И.\,Н.} Ведение в теорию массового осблуживания.~--- М.: Наука, 1987.

\bibitem{liu-templeton} 
\Au{Liu L., Templeton J.} Autocorrelations in infinite server batch arrival queues~// 
Queueing Syst., 1993. Vol.~14. P.~313--337.

\bibitem{tihonenko}  
\Au{Тихоненко О.\,М.} Модели массового обслуживания в системах обработки информации.~--- Минск: Университетское,
1990.

\bibitem{eliazar} 
\Au{Eliazar I.} The M/G/$\infty$ system revisited: Finiteness, summability, long range 
dependence, and reverse engineering~// Queueing Syst., 2007. Vol.~55. P.~71--82.

\bibitem{daley-busy} 
\Au{Daley D.} The busy period of the $M/GI/\infty$ queue~// Queueing Syst., 2001. Vol.~38. P.~195--204.

\bibitem{brandt} 
\Au{Brandt A., Sulanke H.} On the $GI/M/\infty$ queue with batch arrivals of constant size~// 
Queueing Syst., 1987. Vol.~2. P.~187--200.

\bibitem{Kiefer} 
\Au{Kiefer J., Wolfowitz J.} On the theory of queues with many servers~// 
Trans. Amer. Math. Soc., 1955. Vol.~78. P.~1--18.

\bibitem{Shiryaev} 
\Au{Ширяев А.\,Н.} Вероятность.~--- М.: Наука, 1989. 640~с.

\bibitem{Jacobs} %26
\Au{Jacobs D.\,R., Schach S.} Stochastic order relationships between $GI/G/k$ queues~// 
Ann. Math. Stat., 1972. Vol.~43.  P.~1623--1633.

\bibitem{scheller-further}  %27
\Au{Scheller-Wolf A.} Further delay moment results in FIFO multiserver queues~// 
Queueing Syst., 2000. Vol.~34. P.~387--400.

\bibitem{CRM} %28
\Au{Morozov E.\,V.}  Coupling and monotonicity of queues. Sci. Report. No.\,779.~--- 
Barcelona: CRM, 2008. P.~1--29.

\bibitem{Taha} \Au{El-Taha M.} Pathwise rate-stability for input-output processes~// 
Queueing Syst., 1996. Vol.~22. P.~47--63.

\bibitem{MorozovJMS} 
\Au{Morozov E.} Instability of nonhomogeneous queueing networks~// J.~Math. Sci., 2002. Vol.~112. No.\,2. P.~4155--4167.

\bibitem{Morozovoutput} 
\Au{Morozov E.} Stability of Jackson-type network output~// 
Queueing Syst., 2002. Vol.~40. P.~383--406.

\bibitem{scheller-sigman97} 
\Au{Scheller-Wolf A., Sigman K.} Delay moments for FIFO $GI/GI/s$ queues~// 
Queueing Syst., 1997. Vol.~25. P.~77--95.

\bibitem{Kiefer1} 
\Au{Kiefer J., Wolfowitz J.} On the characteristics of the general queueing process 
applications to random walks~// Ann.~Math.~Statist., 1956. Vol.~27. P.~147--161.

\bibitem{cluster} 
Центр высокопроизводительной обработки данных.~--- ЦКП КарНЦ РАН.
{\sf http://cluster.krc.karelia.ru}.

\bibitem{aptpms11} 
\Au{Morozov E.\,V., Rumyantsev A.\,S.} Stability analysis of a multiprocessor model 
describing a high performance cluster~// 
Applied problems in theory of probabilities and mathematical statistics 
related to modeling of information systems: Book of Abstracts of the 29th 
 Seminar (International) on Stability Problems for Stochastic Models and 
 5th Workshop (International).~--- Moscow: Institute of Informatics Problems, RAS, 2011. P.~82--83.

\bibitem{hpc11} 
\Au{Румянцев А.\,С.} Моделирование процесса нагрузки вычислительного кластера 
на примере кластера ЦКП\linebreak КарНЦ РАН <<Центр высокопроизводительной обработки данных>>~// 
Высокопроизводительные параллельные вычисления на кластерных системах:\linebreak Мат-лы 
XI~Всеросс. конф.~/ Под ред.~В.\,П.~Гергеля.~--- Нижний Новгород: Изд-во 
Нижегородского госуниверситета, 2011. С.~272--275.

\label{end\stat}

\bibitem{downey99} 
\Au{Downey A., Feitelson D.} The elusive goal of workload characterization~// 
Performance Evaluation Rev., 1999. Vol.~26. No.\,4. P.~14--29.
 \end{thebibliography}
}
}


\end{multicols}     %12

%\def\v{\varphi}
%\def\g{\gamma}
%\def\w{\omega}
%\def\a{\overline a}
%\def\b{\beta}




\def\stat{pechinkin}

\def\tit{ОГРАНИЧЕНИЕ НА СУММАРНЫЙ ОБЪЕМ ЗАЯВОК В~ДИСКРЕТНОЙ СИСТЕМЕ Geo$/G/1/\infty$$^*$}

\def\titkol{Ограничение на суммарный объем заявок в~дискретной системе Geo/$G/1/\infty$}

\def\autkol{А.\,В.~Печинкин,  И.\,А.~Соколов, С.\,Я.~Шоргин}
\def\aut{А.\,В.~Печинкин$^1$,  И.\,А.~Соколов$^2$, С.\,Я.~Шоргин$^3$}

\titel{\tit}{\aut}{\autkol}{\titkol}

{\renewcommand{\thefootnote}{\fnsymbol{footnote}}\footnotetext[1]
{Работа выполнена при поддержке РФФИ
(гранты 11-07-00112, 12-07-00108 и 11-01-12026-офи-м).}}


\renewcommand{\thefootnote}{\arabic{footnote}}
\footnotetext[1]{Институт проблем информатики Российской академии наук; apechinkin@ipiran.ru}
\footnotetext[2]{Институт проблем информатики Российской академии наук, isokolov@ipiran.ru} 
\footnotetext[3]{Институт проблем информатики Российской академии наук, sshorgin@ipiran.ru}

\vspace*{-6pt}

\Abst{Рассматривается функционирующая в дискретном времени
однолинейная сис\-те\-ма массового обслуживания
Geo$/G/1$ с инверсионным порядком обслуживания без
прерывания обслуживания, в которой каждая заявка
наряду с (дискретной) случайной длиной имеет также (дискретный)
случайный объем. Суммарный объем находящихся в сис\-те\-ме заявок ограничен
некоторым (неслучайным) числом. Получены алгоритмы, позволяющие вычислять основные
стационарные показатели функционирования этой сис\-темы.}

\KW{система массового обслуживания; дискретное время;
длина и объем заявки}


\vskip 12pt plus 9pt minus 6pt

      \thispagestyle{headings}

      \begin{multicols}{2}

            \label{st\stat}

\section{Введение. Описание системы}

Задача исследования систем массового обслуживания (СМО),
в которых каждая поступающая в систему заявка наряду со
случайной длиной имеет случайный объем, причем
суммарный объем всех находящихся в системе заявок
ограничен, как было замечено еще в работах~[1--3],
играет важную роль при моделировании работы
самых разнообразных технических устройств, в част\-ности
современных ин\-фор\-ма\-ци\-он\-но-вы\-чис\-ли\-тель\-ных сис\-тем.
Однако аналитических решений этой задачи при дисциплине
выбора заявок из очереди на обслуживание в порядке
поступления (FIFO) до сих пор не найдено, поскольку
для корректного построения соответствующего
марковского процесса, описывающего функционирование СМО,
необходимо учитывать объемы всех заявок в сис\-те\-ме.
Фактически приходится сталкиваться с теми же самыми трудностями, что и при исследовании
многолинейных СМО, для которых также не найдено удовлетворительных аналитических решений.

В работах [4--11]
были исследованы СМО с ограничением на суммарный объем
заявок, но при инверсионном порядке обслуживания (дисциплина LIFO).
Оказалось, что в этом случае можно получить алгоритмы,
пригодные для численных расчетов стационарных характеристик.

Тем не менее во всех этих работах, в том числе и в~[9--11],
где рассматривались СМО в дискретном времени, распределение
объема заявки предполагалось непрерывным, а тогда алгоритмы расчета 
опира\-лись на решения довольно
сложных интегральных уравнений, что снижало практическую
значимость полученных результатов. 
В~рас\-смат\-ри\-ва\-емой в настоящей статье СМО Geo$_m/G/1$
с ограничением на суммарный объем находящихся
в ней заявок, в отличие от цитированных выше работ,
объем каждой заявки является дискретной случайной
величиной. Это позволяет получить более прос\-тые и эффективные
алгоритмы рас\-че\-та основных стационарных показателей функционирования.

Рассмотрим функционирующую в дискретном времени
однолинейную СМО Geo$_m/G/1$, в которую
поступает геометрический поток заявок с ве\-ро\-ят\-ностью~$a$ поступления заявки на такте.

Каждая поступающая в систему заявка наряду с длиной
имеет случайный целочисленный (не\-от\-ри\-ца\-тель\-ный) объем.
Совместное распределение длины и объема заявки задается
вероятностью $b_{k,l}$,  $k,l\ge 0$, того, что ее длина
(число тактов обслуживания) равна~$k$, а объем равен~$l$.
Будем предполагать выполненным естественное условие, что
длина заявки и ее объем не могут равняться нулю, т.\,е.\
$b_{k,0}\hm=b_{0,l}\hm=0$ для любых $k$ и~$l$.

Общий объем находящихся в системе заявок ограничен
(неслучайным) числом~$L$,\ \ $0\hm<L\hm<\infty$.
Если объем поступающей в систему заявки в
сумме с объемами остальных находящихся в сис\-те\-ме
заявок больше~$L$, то она теряется. Будем предполагать, что если в момент поступления
новой заявки систему покидает обслуженная заявка, 
то ее объем при определении суммарного объема не учитывается.

В системе реализован инверсионный порядок обслуживания
без прерывания обслуживания, при котором принятая в
сис\-те\-му заявка становится на первое место в очереди.
Будем считать для определенности, что если в момент
поступления новой заявки сис\-те\-му покидает обслуженная
заявка, то на прибор становится новая заявка.

Введем также обозначения:

\noindent
$b(l)=\sum\limits_{k=1}^\infty b_{k,l}$,\  $l\ge 1$~---
вероятность того, что объем заявки равен $l$;

\noindent
$B(l)=\sum\limits_{j=1}^l b(j) =
\sum\limits_{j=1}^l \sum\limits_{k=1}^\infty b_{k,j}
$,\  $l\ge 1$~--- вероятность того, что объем заявки
не более $l$;

\noindent
$b(k\,|\,l)=b_{k,l}/b(l)$,\ $k,l\ge 1$~--- условная
ве\-ро\-ят\-ность того, что длина заявки равна~$k$, при условии, что
ее объем равен $l$;

\noindent
$B(k\,|\,l)=
\sum\limits_{i=k}^\infty b(i\,|\,l)$,\ $k,l\ge 1$~---
условная вероятность того, что длина заявки не
менее~$k$, при условии, что ее объем равен~$l$;

\noindent
$\beta(z\,|\,l)=\sum\limits_{k=1}^\infty z^k b(k\,|\,l)$,\ 
$l\ge 1$~---
производящая функ\-ция (ПФ) длины заявки при условии,
что ее объем равен~$l$;

\noindent
$\beta^*(z\,|\,l)=
\sum\limits_{k=1}^\infty z^{k-1} B(k+1\,|\,l)
=[z-\beta(z\,|\,l)]/[z(1\hm-z)]$,\  $l\ge 1$;

\noindent
$\overline m=\sum\limits_{k=0}^\infty
\sum\limits_{l=0}^\infty k b_{k,l}$~---
математическое ожидание длины заявки.

Далее будем предполагать, что выполнено условие
$\overline m \hm< \infty$.
Это условие является необходимым и достаточным для
существования стационарного режима функционирования
сис\-те\-мы. Кроме того, чтобы избежать непринципиальных
трудностей в изложении, будем считать, что объем заявки,
во-пер\-вых, не превосходит $L$, а во-вто\-рых, с ненулевой
вероятностью может принимать значение единица.

\section{Стационарные вероятности состояний}

Обозначим через $p_0$ стационарную вероятность
отсутствия заявок в сис\-те\-ме, а через
$p_{k,i}(l_1,\ldots,l_i)$,\ 
$i \hm\ge 1$,\  $k\hm\ge 1$,\  $l_1,\ldots,l_i \hm\ge 1$,~---
стационарную вероятность того, что в сис\-те\-ме находится
$i$ заявок, причем (обслуженная) длина и объем заявки
на приборе равны~$k$ и~$l_1$, а объемы остальных
находящихся в сис\-те\-ме заявок равны (в порядке очереди)
$l_2,\ldots,l_i$.

Заметим, что поскольку объем каждой заявки~---
целое положительное чис\-ло, то
суммарный объем $l_1 +\ldots+ l_i$ заявок в системе
не может быть меньше числа $i$ заявок, т.\,е.\
неравенство $p_{k,i}(l_1,\ldots,l_i) \hm> 0$
может выполняться только при
\begin{equation}
\label{6.2.0}
i\le l_1 +\ldots+ l_i \le L
\,,\enskip i \ge 1\,.
\end{equation}
Поэтому всюду далее, не оговаривая этого особо, будем
предполагать, что условие~\eqref{6.2.0} выполнено.

Используя метод исключения состояний~[12, с.~22],
получаем сис\-те\-му урав\-нений:
\begin{multline}
\label{6.2.1}
p_{k,1}(l)
=
[1 - a B(L-l)]
\fr{B(k+1\,|\,l)}{B(k\,|\,l)}
\,p_{k-1,1}(l)\,,\\
\enskip k\ge 1\,,
\end{multline}
%%%%%%%%%%%%%%%%%%%%%

\vspace*{-9pt}

\noindent
\begin{multline}
p_{k,i}(l_1,\ldots,l_i)
=
[1 - a B(L-l_1-\cdots\\
\cdots -l_i)]
\fr{B(k+1\,|\,l_1)}{B(k\,|\,l_1)}
\,p_{k-1,i}(l_1,\ldots,l_i)
 +{}\\
{}
+\
a b(l_2)
\fr{B(k+1\,|\,l_1)}{ B(k\,|\,l_1)}
\,p_{k-1,i-1}(l_1,l_3,\ldots,l_i)\,,
\\
i \ge 2\,,\ \  k\ge 1\,,
\label{6.2.2}
\end{multline}
%%%%%%%%%%%%%%%%%%%%%%%%%%%%%%%%%%%%%%
с начальными условиями
\begin{multline}
\label{6.2.3}
p_{0,1}(l)
={}\\
{}=p_0 a b(l)
+a b(l)\sum\limits_{s=1}^{L-l}
\sum\limits_{k=1}^\infty
\fr{B(k+1\,|\,s) }{B(k\,|\,s)}\, p_{k-1,1}(s)
+{}\\
{}+
a b(l)
\sum\limits_{s=1}^{L}
\sum\limits_{k=1}^\infty
\fr{b(k\,|\,s)}{ B(k\,|\,s)}\, p_{k-1,1}(s)\,;
\end{multline}

\vspace*{-9pt}

\noindent
\begin{multline}
p_{0,i}(l_1,\ldots,l_i)
= a b(l_1)\times{}\\
{}\times 
\sum\limits_{l=1}^{L-l_1-\cdots-l_i}
\sum\limits_{k=1}^\infty
\fr{B(k+1\,|\,l) }{ B(k\,|\,l)}\,
p_{k-1,i}(l,l_2,\ldots,l_i)
 +{}
\\
{}+
a b(l_1) \sum\limits_{l=1}^{L-l_2-\cdots-l_i}
\sum\limits_{k=1}^\infty
\fr{b(k\,|\,l) }{B(k\,|\,l)}\,
p_{k-1,i}(l,l_2,\ldots,l_i)\,,
\\
i \ge 2\,.
\label{6.2.4}
\end{multline}

Из уравнений~(\ref{6.2.1}) и~(\ref{6.2.3}) получаем:
\begin{multline}
\label{6.2.5}
p_{k,1}(l)=
[1 - a B(L-l)]^k B(k+1\,|\,l)
\,p_{0,1}(l)\,;
\\ 
k\ge 1\,;
\end{multline}

\vspace*{-9pt}

\noindent
\begin{multline}
\label{6.2.6}
p_{0,1}(l)= p_0 a b(l)+{}\\
{}+
a b(l)\sum\limits_{s=1}^{L-l}
\beta^*([1 - a B(L-s)]\,|\,s)
p_{0,1}(s)
 +{}
\\
{}+
a b(l)
\sum\limits_{s=1}^{L}
\beta([1 - a B(L-s)]\,|\,s)
p_{0,1}(s)\,.
\end{multline}
%%%%%%%%%%%%%%%%%%%%%%%%%%%%%

Из уравнения~(\ref{6.2.2}) имеем:
%%%%%%%%%%%%%%%%%%%%%%%%%%%%%
\begin{multline}
p_{k,i}(l_1,\ldots,l_i)
={}\\
{}=[1 - a B(L-l_1-\cdots-l_i)]^k
B(k+1\,|\,l_1) p_{0,i}(l_1,\ldots,l_i)
+{}
\\
{}+r_{k,i}(l_1,\ldots,l_i)\,,
\enskip
i \ge 2\,,\ \ k\ge 1\,,
\label{6.2.7}
\end{multline}
%%%%%%%%%%%%%%%%%%%%%%
где
\begin{equation}
\label{6.2.8}
r_{0,i}(l_1,\ldots,l_i)
= 0\,,\quad
i \ge 2,
\end{equation}

\vspace*{-9pt}

\noindent
\begin{multline}
r_{k,i}(l_1,\ldots,l_i)
=[1 - a B(L-l_1-\cdots\\
\cdots -l_i)]
\fr{B(k+1\,|\,l_1)}{ B(k\,|\,l_1)}
\,r_{k-1,i}(l_1,\ldots,l_i)
 +{}
\\
{}+
a b(l_2)
\fr{B(k+1\,|\,l_1)}{ B(k\,|\,l_1)}
\,p_{k-1,i-1}(l_1,l_3,\ldots,l_i),
\\
 i \ge 2\,,\ \ k\ge 1\,,
\label{6.2.9}
\end{multline}
а из уравнения~(\ref{6.2.4}) находим:
%%%%%%%%%%%%%%%%%%%%%%%%%
\begin{multline}
p_{0,i}(l_1,\cdots,l_i)
=a b(l_1)
\sum\limits_{l=1}^{L-l_1-\cdots-l_i}
\beta^*\times{}\\
{}\times([1 - a B(L-l-l_2-\cdots-l_i)]\,|\,l)
p_{0,i}(l,l_2,\ldots,l_i)
 +{}
\\
+a b(l_1)
\sum\limits_{l=1}^{L-l_2-\cdots-l_i}
\beta([1 - a B(L-l-l_2-\cdots-l_i)]\,|\,l)\times{}\\
{}\times
p_{0,i}(l,l_2,\ldots,l_i)
 +a b(l_1)\times{}
\\
{}\times
\sum\limits_{l=1}^{L-l_1-\cdots-l_i}
\sum\limits_{k=1}^\infty
\fr{B(k+1\,|\,l)}{ B(k\,|\,l)}\,
r_{k-1,i}(l,l_2,\ldots,l_i)
 +{}\\
{}+ a b(l_1)
\sum\limits_{l=1}^{L-l_2-\cdots-l_i}
\sum\limits_{k=1}^\infty
\fr{b(k\,|\,l)}{B(k\,|\,l)}\,
r_{k-1,i}(l,l_2,\ldots,l_i)
\,,
\\
  i \ge 2\,.
\label{6.2.10}
\end{multline}

Соотношения (\ref{6.2.5})--(\ref{6.2.10}) позволяют
последовательно, начиная с $i\hm=1$ и кончая $i\hm=L$,
с точностью до~$p_0$
определять вероятности $p_{k,i}(l,l_1,\ldots,l_{i-1})$.

Вероятность $p_0$, как обычно, определяется из условия
нормировки.

Однако нахождение $p_{k,i}(l,l_1,\ldots,l_{i-1})$
из-за большой размерности вычисляемых вероятностей
невозможно уже при совсем небольших значениях~$i$ даже для современной вычислительной техники.
Поэтому в следующем разделе будет определено более простое маргинальное распределение и приведены
формулы для его расчета, а в разд.~4 описан удобный алгоритм вычислений.

\section{Маргинальное распределение стационарных
вероятностей состояний}

С точки зрения практики вполне достаточно знать не
совместное распределение объемов всех находящихся в
системе заявок, а только лишь совместное распределение
объема обслуживаемой на приборе заявки и суммарного
объема остальных находящихся в сис\-те\-ме заявок.
Более того, в сис\-те\-ме без ограничения на чис\-ло
находящихся в ней заявок при инверсионном порядке
обслуживания чис\-ло
заявок в сис\-те\-ме также не представляет особого интереса.
Поэтому обозначим через
\begin{multline*}
p_{k}(l,m)
= \sum\limits_{i=1}^\infty
\sum\limits_{l_1+\cdots+l_{i-1}=m}
p_{k,i}(l,l_1,\ldots,l_{i-1})\,,
\\
 k\ge 0\,,\ \ m\ge 0\,,\ \ l=\overline{L-m}\,,
\end{multline*}
стационарную вероятность того, что (обслуженная) длина
и объем заявки на приборе равны $k$ и~$l$, а суммарный
объем остальных находящихся в сис\-те\-ме заявок равен $m$
(напомним, что в силу условия~\eqref{6.2.0} обязательно
должно выполняться двойное неравенство
$i-1\hm\le m \hm\le L-l$).
Примем соглашение, что значение $m\hm=0$ соответствует
отсутствию заявок в накопителе ($i\hm=1$).

Из соотношений (\ref{6.2.5})--(\ref{6.2.9}) имеем:
%%%%%%%%%%%%%%%%%%%%%%%%%%%%%
\begin{multline}
\label{6.2*.11}
p_{k}(l,m)
=[1 - a B(L-l-m)]^k
B(k+1\,\vert\,l) p_{0}(l,m)
+{}\\
{}+ r_{k}(l,m)\,,
\enskip
k \ge 1\,,\ \ m\ge 0\,,
\end{multline}
где
\begin{equation}
\label{6.2*.12}
r_{k}(l,0) = 0\,, \quad k \ge 1\,;
\end{equation}
%%%%%%%%%%%%%%%%%%%%%%

\noindent
\begin{equation}
\label{6.2*.13}
r_{0}(l,m) = 0\,,
\quad m \ge 1\,;
\end{equation}

\vspace*{-9pt}

\noindent
\begin{multline}
r_{k}(l,m) ={}\\
{}=
[1 - a B(L-l-m)]
\fr{B(k+1\,|\,l)}{ B(k\,|\,l)}
\,r_{k-1}(l,m) +{}\\
{}+ a \fr{B(k+1\,|\,l) }{ B(k\,|\,l)}
\sum\limits_{s=1}^{m-1}
b(s) p_{k-1}(l,m-s)\,,
\\
 k\ge 1\,,
\ \ m \ge 1\,,
\label{6.2*.14}
\end{multline}
%%%%%%%%%%%%%%%%%%%%%%%%%%%%%%%%%%%%%
а из соотношений (\ref{6.2.6}) и (\ref{6.2.10}) находим:
\begin{multline}
p_{0}(l,m) =q(l,m) + {}\\
{}+ a b(l)
\sum\limits_{s=1}^{L-l-m}
\beta^*([1 - a B(L-s-m)]\,|\,s)
p_{0}(s,m)
 +{}
\\
{}+ a b(l) \sum\limits_{s=1}^{L-m}
\beta([1 - a B(L-s-m)]\,|\,s)
p_{0}(s,m)\,,
\\
m \ge 0\,,
\label{6.2*.15}
\end{multline}
где
\begin{equation}
\label{6.2*.16}
q(l,0) = p_0 a b(l) \,;
\end{equation}

\vspace*{-9pt}

\noindent
\begin{multline}
q(l,m) = a b(l)
\sum\limits_{s=1}^{L-l-m}
\sum\limits_{k=1}^\infty
\fr{B(k+1\,|\,s)}{ B(k\,|\,s)}\,
r_{k-1}(s,m)  +{}\\
{}+ a b(l) \sum\limits_{s=1}^{L-m}
\sum\limits_{k=1}^\infty
\fr{b(k\,|\,s)}{ B(k\,|\,s)}\,
r_{k-1}(s,m)\,,
\enskip m \ge 1\,.
\label{6.2*.17}
\end{multline}
%%%%%%%%%%%%%%%%%%%%%%

\section{Алгоритм решения системы уравнений}

В этом разделе приведем простой алгоритм чис\-лен\-но\-го
решения сис\-тем линейных алгебраических урав\-не\-ний~(\ref{6.2*.11})--(\ref{6.2*.17}),
который со\-сто\-ит в последовательном по $m$ от $m=0$
до $m\hm=L\hm-1$ вычислении стационарных вероятностей
$p_{k}(l,m)$.

Начнем с определения $p_{k}(l,0)\hm=p_{k,1}(l)$.

Для сокращения записи введем обозначения:
\begin{alignat*}{2}
x_l &= p_{0}(l,0)\,, &\enskip l&=\overline{1,L}\,;
\\
b_l &= q(l,0) = a b(l)\,, &\enskip l&=\overline{1,L}\,;
\\
\beta^*_l &= \beta^*([1 - a B(L-l)]\,|\,l)\,, &\enskip l&=\overline{1,L}\,;
\\
\beta_l &= \beta([1 - a B(L-l)]\,|\,l)\,, &\enskip  l&=\overline{1,L}\,;
\\
y_l &= \sum\limits_{s=1}^{L-l} \beta^*([1 - a B(L-s)]\,|\,s) p_{0}(s,0)\,, &\enskip l&=\overline{1,L}\,;
\\
y &= \sum\limits_{s=1}^{L} \beta([1 - a B(L-s)]\,|\,s) p_{0}(s,0) \,.&&
\end{alignat*}
Тогда систему~(\ref{6.2*.15}) можно записать следующим
образом:
%%%%%%%%%%%%%%%%%%%%%
\begin{gather}
\label{6.3*.1}
x_{l}= b_l p_0 + b_l y_l + b_l y \,,\enskip l=\overline{1,L}\,;
\\
\label{6.3*.2}
y_l = \sum\limits_{s=1}^{L-l} \beta^*_s x_s \,,\enskip l=\overline{1,L-1}\,;
\ y_{L} = 0\,;
\\
\label{6.3*.3}
y= \sum\limits_{s=1}^{L} \beta_s x_s \,.
\end{gather}
Заметим, что при $l=\overline{2,L-1}$ каждое $y_l$
представимо в виде:
\begin{equation}
\label{6.3*.4}
y_l = y_{l-1} - \beta^*_{L-l+1} x_{L-l+1}\,,
\end{equation}
а при $l=\overline{1,L-2}$~--- в виде:
\begin{equation}
\label{6.3*.5}
y_l = y_{l+1} + \beta^*_{L-l} x_{L-l}\,.
\end{equation}

Сначала выразим $x_l$, $l\hm=\overline{1,L}$, и
$y_l$,  $l\hm=\overline{2,L-1}$, через $y_1$ и $y$ по формулам:
%%%%%%%%%%%%%%%%%%%%%
\begin{gather*}
x_{l} = c_l + d_l y_1 + e_l y \,,\enskip l=\overline{1,L}\,;
\\
y_{l} = f_l + g_l y_1 + h_l y \,,\enskip l=\overline{2,L-1}\,.
\end{gather*}
%%%%%%%%%%%%%%%%%%%%%%%%%%%%%%%%%%%%%%%%
Подставляя в (\ref{6.3*.1}) $l=L$, имеем:
%%%%%%%%%%%%%%%%%%%%%
\begin{equation*}
x_{L} = b_L p_0 + b_L y \,,
\end{equation*}
т.\ е.
\begin{equation*}
c_{L} = b_L p_0\,;
\enskip d_{L} = 0\,;
\enskip
e_L= b_L \,.
\end{equation*}
%%%%%%%%%%%%%%%%%%%%%%%%%%%%%%%%%%%%%%%
При $l=1$ из (\ref{6.3*.1}) и~(\ref{6.3*.2}) находим:
\begin{gather*}
x_{1} = b_{1} p_0 + b_{1} y_1 + b_{1} y\,;
\enskip 
y_{L-1}= b^*_{1} x_{1} \,;
\\
c_{1} = b_{1} p_0\,; \enskip d_{1}=b_{1}\,;
\enskip 
e_{1} =b_{1}\,;\enskip
f_{L-1}=b^*_{1} c_{1} \,;
\\
g_{L-1}=b^*_{1} d_{1}\,;
\enskip
h_{L-1}= b^*_{1} e_{1}\,.
\end{gather*}
При $l=L-1$ из (\ref{6.3*.1}) и (\ref{6.3*.4}) получаем:

\noindent
\begin{gather*}
x_{L-1} = b_{L-1} p_0 + b_{L-1} y_{L-1} + b_{L-1} y\,;
\\
y_{2}= y_{1} - \beta^*_{L-1} x_{L-1}\,;
\\
c_{L-1} = b_{L-1} p_0 + b_{L-1} f_{L-1} \,;
\enskip
d_{L-1} = b_{L-1} g_{L-1}\,;
\\
e_{L-1}=b_{L-1} h_{L-1} + b_{L-1}\,;
\\
f_{2} = - b^*_{L-1} c_{L-1}\,;
\enskip
g_{2}=1 - b^*_{L-1} d_{L-1}\,;
\\
h_{2}=- b^*_{L-1} e_{L-1}\,.
\end{gather*}
При $l=2$ из (\ref{6.3*.1}) и~(\ref{6.3*.5}) имеем:

\noindent
%%%%%%%%%%%%%%%%%%%%%
\begin{gather*}
x_{2} = b_{2} p_0 + b_{2} y_2 + b_{2} y\,;
\
y_{L-2} = y_{L-1} + \beta^*_{2} x_{2} \,;
\\
c_{2} =b_{2} p_0 + b_{2} f_{2} \,;
\ 
d_{2} = b_{2} g_{2}\,;
\
e_{2} = b_{2} h_{2} + b_{2}\,;
\\
f_{L-2} = f_{L-1} - b^*_{2} c_{2}\,;
\ 
g_{L-2} = g_{L-1} - b^*_{2} d_{2}\,;
\\
h_{L-2} = h_{L-1} - b^*_{2} e_{2} \,.
\end{gather*}

Продолжая эту процедуру, получаем, что $x_l$
при четном $l$ в пределах от $l\hm=L/2\hm+1$ до $L-2$ и
при нечетном $l$ в пределах от $l\hm=(L+1)/2$ до $L-2$
вычисляются по формулам:

\noindent
\begin{gather*}
x_{L-s} =о b_{L-s} p_0 + b_{L-s} y_{L-s} + b_{L-s} y\,;
\\
y_{s+1} = y_{s} - \beta^*_{L-s} x_{L-s} \,;
\\
c_{L-s} = b_{L-s} p_0 + b_{L-s} f_{L-s} \,;
\enskip 
d_{L-s} = b_{L-s} g_{L-s}\,;
\\ 
e_{L-s} = b_{L-s} h_{L-s} + b_{L-s}\,;
\\
f_{s+1} = f_{s} - b^*_{L-s} c_{L-s}\,;
\enskip
g_{s+1} = g_{s} - b^*_{L-s} d_{L-s}\,;
\\
h_{s+1} = h_{s} - b^*_{L-s} e_{L-s}\,,
\end{gather*}
а при четном $l$ в пределах от $l\hm=3$ до $L/2$ и при
нечетном~$l$ в пределах от $l\hm=1$ до $(L-1)/2$~---
по формулам:

\noindent
%%%%%%%%%%%%%%%%%%%%%%%%%%%%%%%%%
\begin{gather*}
%\label{6.2.6}
x_{s}=b_{s} p_0+b_{s} y_s+b_{s} y\,;
\enskip 
y_{L-s}=y_{L-s+1} + \beta^*_{s} x_{s}\,;
\\
c_{s} = b_{s} p_0 + b_{s} f_{s} \,;
\enskip 
d_{s} = b_{s} g_{s}\,;
\enskip
e_{s} = b_{s} h_{s} + b_{s}\,;
\\
f_{L-s} =f_{L-s+1} - b^*_{s} c_{s}\,;
\enskip 
g_{L-s} = g_{L-s+1} - b^*_{s} d_{s}\,;
\\ 
h_{L-s} = h_{L-s+1} - b^*_{s} e_{s}\,.
\end{gather*}

Подставляя найденные значения $x_l$ в равенство~(\ref{6.3*.2}) при $l=1$ и в 
равенство~(\ref{6.3*.3}), приходим к сис\-те\-ме из двух линейных алгебраических
уравнений относительно $y_1$ и $y$, решая которую,
находим эти величины, затем стационарные
ве\-ро\-ят\-ности $x_l=p_{0}(l,0)$,\ \ $l=\overline{1,L}$,
и далее с по\-мощью формул~(\ref{6.2*.11}) и~(\ref{6.2*.12}) стационарные вероятности
$p_{k}(l,0)$, $l\hm=\overline{1,L}$, $k\hm\ge 1$.

Предположим теперь, что вероятности $p_{k}(l,j)$,
$k\hm\ge0$, $l\hm=\overline{1,L-j}$,
$j\hm=\overline{j-1,L}$,
уже найдены для всех $j\hm=\overline{0,m-1}$,
$m\hm=\overline{0,L-1}$.
Найдем эти вероятности для~$m$.

Как и прежде, для сокращения записи введем обозначения:

\noindent
\begin{alignat*}{2}
x_l &= p_{0}(l,m)\,,&\enskip l&=\overline{1,L-m}\,;
\\
b_l&=a b(l)\,,&\enskip l&=\overline{1,L-m}\,;
\end{alignat*}


\vspace*{-3pt}

\pagebreak

\noindent
\begin{alignat*}{2}
\beta^*_l&=\beta^*([1 - a B(L-l-m)]\,|\,l)\,,&\enskip l&=\overline{1,L-m}\,;
\\[6pt]
\beta_l&=\beta([1 - a B(L-l-m)]\,|\,l)\,,&\enskip l&=\overline{1,L-m}\,;
\end{alignat*}

\vspace*{-12pt}

\noindent
\begin{align*}
y_l&=\sum\limits_{s=1}^{L-l-m}\beta^*([1 - a B(L-s-m)]\,|\,s)
p_{0}(s,m)\,,\\[1pt]
&\hspace*{50mm}l=\overline{1,L-m}\,;\\
y&=\sum\limits_{s=1}^{L-m}\beta([1 - a B(L-s-m)]\,|\,s)p_{0}(s,m)\,;
\end{align*}

\vspace*{-12pt}

\noindent
\begin{multline*}
%\label{6.6.15}
\tilde b_l=q(l,m)={}\\[2pt]
{}=a b(l)\left[\sum\limits_{s=1}^{L-l-m}
\sum\limits_{k=1}^\infty\fr{B(k+1\,|\,s) }{B(k\,|\,s)}\,
r_{k-1}(s,m) +
{}\right.\\[2pt]
\left.{}+
\sum\limits_{s=1}^{L-m} \sum\limits_{k=1}^\infty
\fr{b(k\,|\,s)}{ B(k\,|\,s)}\,
r_{k-1}(s,m)
\right]\,,\enskip l=\overline{1,L-m}\,.
\end{multline*}
При этом, как видно из~(\ref{6.2*.13}), (\ref{6.2*.14}),
(\ref{6.2*.16}) и~(\ref{6.2*.17}), $\tilde b_l$
выражается через уже известные величины.

С учетом введенных обозначений
сис\-те\-ма~(\ref{6.2*.15}) записывается сле\-ду\-ющим об\-разом:
\begin{gather}
\label{6.3*.8}
x_{l}= \tilde b_l + b_l y_l + b_l y \,,\enskip l=\overline{1,L-m}\,;
\\
\!y_l = \sum\limits_{s=1}^{L-l-m} \beta^*_s x_s\,,\
l=\overline{1,L-1-m}\,;\  y_{L-m} = 0\,;\!\!
\label{6.3*.9}
\\
\label{6.3*.10}
y= \sum\limits_{s=1}^{L-m} \beta_s x_s \,.
\end{gather}

Нетрудно видеть, что при каждом фиксированном~$m$
алгоритм решения сис\-те\-мы~(\ref{6.3*.8})--(\ref{6.3*.10})
полностью совпадает с алгоритмом решения сис\-те\-мы~(\ref{6.3*.1})--(\ref{6.3*.3}).

Оставшиеся неизвестными вероятности $p_{k}(l,m)$,
$k\hm\ge1$, вычисляются по формуле~(\ref{6.2*.11}).

Приведенный здесь алгоритм позволяет вы\-чис\-лить
стационарные вероятности $p_{k,i}(l,m)$ с точ\-ностью
до вероятности~$p_0$, которая, как уже говорилось,
определяется из условия нормировки
$$
p_0+ \sum\limits_{k=0}^{\infty} \sum\limits_{m=0}^{L-1}
\sum\limits_{l=1}^{L-m} p_{k}(l,m)
= 1 \,.
$$

\section{Некоторые стационарные показатели,
связанные с~числом заявок в~системе}

Выпишем выражения для некоторых стационарных
характеристик, связанных со стационарными
вероятностями со\-сто\-яний.

Стационарная вероятность $p(m)$, $m\hm= \overline{1,L}$,
того, что суммарный объем находящихся в сис\-те\-ме заявок
равен~$m$, задается формулой
\begin{equation*}
p(m)= \sum\limits_{k=0}^{\infty} \sum\limits_{l=1}^{m-1}
p_{k}(l,m-l)\,,
\enskip m= \overline{1,L}\,.
\end{equation*}

Стационарная вероятность $p^*_{0}$ того, что в момент
поступления новой заявки система будет свободна (в том
числе на приборе закончится обслуживание единственной
находящейся в сис\-те\-ме заявки), задается формулой
\begin{equation*}
p^*_{0} = p_0 + \sum\limits_{k=1}^\infty \sum\limits_{l=1}^{L}
\fr{b(k\,|\,l)}{ B(k\,|\,l)} \, p_{k-1}(l,0) \,.
\end{equation*}

Стационарная вероятность
$p^*_{k}(l,m)$, $m\hm= \overline{0,L-1}$,
$l\hm= \overline{1,L-m}$, $k\hm\ge 1$,
того, что поступающая (не обязательно принятая в сис\-те\-му)
заявка застанет в сис\-те\-ме, по крайней мере, одну заявку,
причем длина и объем заявки на приборе равны $k$ и $l$
и она продолжит обслуживаться, а суммарный
объем остальных находящихся в сис\-те\-ме заявок равен~$m$,
определяется выражением:
\begin{multline*}
p^*_{k}(l,m)= \fr{B(k+1\,|\,l)}{B(k\,|\,l)}
\, p_{k-1}(l,m)\,,\\
m= \overline{0,L-l}\,,
\enskip
 l= \overline{1,L-m}\,,
\enskip  k\ge 1\,.
\end{multline*}

Стационарная вероятность
$p^*(m)$\,, $m\hm= \overline{1,L-1}$,
того, что в момент поступления новой заявки на приборе
закончится обслуживание заявки, а суммарный объем
оставшихся заявок будет равен~$m$, задается формулой
%%%%%%%%%%%%%
\begin{equation*}
p^*(m)= \sum\limits_{k=1}^\infty \sum\limits_{l=1}^{L-m}
\fr{b(k\,|\,l)}{ B(k\,|\,l)} \, p_{k-1}(l,m)\,,
\ \ m= \overline{1,L-1}\,.
\end{equation*}

Наконец, стационарная вероятность $\pi(l)$ того, что поступающая
заявка объема~$l$ будет принята в сис\-те\-му, и стационарная
вероятность~$\pi$ того, что по\-сту\-па\-ющая заявка
произвольной длины будет принята в сис\-те\-му, имеют вид:

\noindent
\begin{align*}
\pi(l)&= p^*_0 + \sum\limits_{k=1}^\infty \sum\limits_{m=1}^{L-l}
\sum\limits_{j=1}^{m} p^*_{k}(j,m-j) +{}\\
&\hspace*{15mm}{}+\sum\limits_{m=1}^{L-l} p^*(m)\,,\  l= \overline{1,L}\,;
\\
\pi &= \sum\limits_{l=1}^{L} b(l) p^*_0(l) \,.
\end{align*}

\section{Стационарное распределение времени
пребывания заявки в~системе}


Будем называть $M$-системой систему, аналогичную
исходной, но с ограничением~$M$, ${1\hm\le M\hm\le L}$,
на суммарный объем заявок и вероятностью~$a$
поступления заявки на такте. Нетрудно видеть, что $M$-сис\-те\-ма представляет собой
исходную СМО, но при условии, что в ней постоянно
находятся заявки суммарного объема $L-M$.


Обозначим через $g_i(k,l;M)$, $i\hm\ge 1$,
$k\hm\ge 0$, $l\hm=\overline{1,M}$,
вероятность того, что период занятости (ПЗ) $M$-сис\-те\-мы,
открываемый заявкой (обслуженной) длины~$k$ и объема~$l$,
продлится $i$ тактов. Для $g_i(k,l;M)$ справедливы следующие соотношения:
\begin{multline}
\label{4.1}
g_1(k,l;M) = [1 - a B(M)]\fr{b(k+1\,|\,l)}{B(k+1\,|\,l)}\,,\\
l= \overline{1,M}\,,\enskip  k\ge 0\,;
\end{multline}

\vspace*{-9pt}

\noindent
\begin{multline}
\label{4.2}
g_2(k,l;M)={}\\
{}= [1 - a B(M-l)]\fr{B(k+2\,|\,l) }{B(k+1\,|\,l)}
\, g_{1}(k+1,l;M)+{}\\
{}+
a \fr{b(k+1\,|\,l) }{ B(k+1\,|\,l)} \sum\limits_{m=1}^{M}
b(m) g_{1}(0,m;M)\,,\\  l= \overline{1,M}\,,\enskip  k\ge 0\,;
\end{multline}

\vspace*{-12pt}

\noindent
\begin{multline}
\label{4.3}
g_i(k,l;M) ={}\\
{}= [1 - a B(M-l)] \fr{B(k+2\,|\,l) }{ B(k+1\,|\,l)}
\, g_{i-1}(k+1,l;M)  +{}
\\
{}+
a \fr{B(k+2\,|\,l) }{ B(k+1\,|\,l)} \sum\limits_{m=1}^{M-l}
b(m) \times{}\\
{}\times
\sum\limits_{j=1}^{i-2} g_{j}(k+1,l;M-m)
g_{i-1-j}(0,m;M)  +{}\\
{}+
a \fr{b(k+1\,|\,l) }{ B(k+1\,|\,l)} \sum\limits_{m=1}^M
b(m) g_{i-1}(0,m;M)\,,\\
\  l= \overline{1,M}\,,\enskip  i\ge 3\,,\enskip  k\ge 0\,.
\end{multline}
%%%%%%%%%%%%%%%%%%%%
В последнем соотношении принято соглашение, что
$\sum\limits_{j=1}^{0} (\cdot) \hm= 0$.

Система уравнений~(\ref{4.1})--(\ref{4.3}) задает
рекуррентную по $M$ от $M\hm=1$ до $M\hm=L$ процедуру
определения вероятностей $g_i(k,l;M)$, которые из этих уравнений
вычисляются при каждом~$M$ последовательно по~$i$ от
$i\hm=1$ для всех возможных значений $k$ и~$l$.

Обозначим через
$w_k(l)$, $l\hm= \overline{1,L}$, $k\hm\ge 0$,
стационарную вероятность того, что заявка объема~$l$
будет принята в систему и будет ожидать начала
обслуживания $k$ тактов,
а через $v_k(l)$, $l\hm= \overline{1,L}$, $k\hm\ge 1$,~---
стационарную вероятность того, что заявка объема~$l$
будет принята в систему и будет находиться в сис\-те\-ме
$k$ тактов. Тогда
\begin{align*}
w_0(l) &= p^*_0 + \sum\limits_{m=1}^{L-l} p^*(m)\,,\quad\ \ l= \overline{1,L}\,,
\\
w_k(l) &= \sum_{i=1}^{\infty} \sum\limits_{j=1}^{L-l} \sum\limits_{m=0}^{L-l-j} p^*_{i}(j,m) 
g_{k}(i,j;L-m-l)\,,\\
&\hspace*{36mm}  l= \overline{1,L}\,,\  k\ge 1\,,
\\
v_k(l) &= \sum\limits_{i=0}^{k-l} w_i(l) b(k-i\,|\,l) \,,\  l= \overline{1,L}\,,\  k\ge 1\,.
\end{align*}
%%%%%%%%%%%%%%%%%%%%%%%%%%%%%%%%%

Наконец, обозначая через $w_k$, $k\hm\ge 0$ условную
стационарную вероятность того, что заявка объема~$l$,
принятая в сис\-те\-му, будет ожидать начала обслуживания $k$ тактов
и через $v_k$, $k\hm\ge 1$,~--- стационарную вероятность
того, что эта заявка будет находиться в сис\-те\-ме $k$~тактов, имеем:
\begin{align*}
w_k&= \fr{1}{\pi} \sum\limits_{l=1}^{L} b(l) w_{k}(l) \,,\ \ k\ge 0\,,
\\
v_k &= \fr{1}{\pi} \sum\limits_{l=1}^{L} b(l) v_k(l) \,,\ \ k\ge 1\,.
\end{align*}

\section{Заключение}

В настоящей статье получены математические
соотношения, позволяющие вычислять основные стационарные
характеристики функционирующей в дискретном времени СМО,
в которой каждая заявка наряду с длиной (временем
обслуживания) имеет (дискретный) случайный объем и суммарный объем находящихся в
системе заявок ограничен. Приведены просто реализуемые алгоритмы для численных
расчетов по этим соотношениям.

{\small\frenchspacing
{%\baselineskip=10.8pt
\addcontentsline{toc}{section}{Литература}
\begin{thebibliography}{99}


\bibitem{romm}
\Au{Ромм Э.\,Л., Скитович В.\,В.}
Об одном обобщении задачи Эрланга~//
Автоматика и телемеханика, 1971. №\,6. С.~164--167.

% 2.
\bibitem{alex}
\Au{Александров А.\,М., Кац Б.\,А.}
Обслуживание потоков неоднородных требований~//
Изв.\ АН СССР. Технич.\ кибернетика, 1973. №\,2. С.~47--53.

% 3.
\bibitem{tich}
\Au{Тихоненко О.\,М.}
Модели массового обслуживания в системах обработки
информации.~--- Минск: Университетское, 1990.

% 4.
\bibitem{pech2} 
\Au{Печинкин А.\,В., Печинкина О.\,А.}
Система $M_k/G/1/n$ с дисциплиной LIFO с прерыванием и
ограничением на суммарный объем требований~//
Вестник Российского ун-та дружбы народов.
Сер.\ Прикладная математика и информатика, 1996. №\,1.
С.~86--93.

% 5.
\bibitem{pech3} 
\Au{Печинкин А.\,В.}
Система обслуживания с дисциплиной LIFO и ограничением
на суммарный объем требований~//
Вестник Российского ун-та дружбы народов.
Сер.\ Прикладная математика и информатика, 1996. №\,2.
С.~85--99.

% 6.
\bibitem{pech1} 
\Au{Печинкин А.\,В.}
Система $M_l/G/1/n$ с дисциплиной LIFO и ограничением на
суммарный объем требований~//
Автоматика и телемеханика, 1998. №\,4. С.~106--116.

% 7.
\bibitem{pech4} 
\Au{Абрамушкина Т.\,В., Апарина С.\,В.,
Кузнецова Е.\,Н., Печинкин А.\,В.}
Численные методы расчета стационарных вероятностей
состояний системы $M/G/1/n$ с дисциплиной LIFO\ PR\/
и ограничением на суммарный объем требований~//
Вестник Российского ун-та дружбы народов. Сер.\ Прикладная математика и
информатика, 1998. №\,1. С.~40--47.



% 8 - new
\bibitem{new} 
\Au{Manzo R., Cascone A., Razumchik R.\,V.} 
Exponential queuing system with negative customers and 
bunker for ousted customers~// 
Automation Remote Control, 2008.
Vol.\,69. No.\,9. P.~1542--1551.




% 8.
\bibitem{cmps} 
\Au{Cascone A., Manzo R., Pechinkin A.\,V., Shorgin~S.\,Ya.} 
A~Geo$_m/G/1/n$ queueing system with LIFO discipline, service
interruptions and resumption, and restrictions on the total volume
of demands~// World Congress on Engineering 2010 Proceedings.
Vol.~III. WCE 2010.~--- London, U.K., 2010. P.~1765--1769. 
ISBN (Vol.~III):  978-988-18210-8-9
ISSN: 2078-0958 (Print)
ISSN: 2078-0966 (Online).


% 9.
\bibitem{cmps-2} 
\Au{Pechinkin A., Shorgin~S.}
A $Geo_m/G/1/n$ queueing system with LIFO discipline, service
interruptions and repeat again service, and restrictions on the
total volume of demands~// Multiple Access Communication (MACOM
2010): Proceedings of the 3rd  Workshop (International).~---
Barcelona, Spain, 2010. P.~98--106.


% 10.
%\bibitem{kmps}
%\Au{Касконе А., Мандзо~Р., Печинкин~А.\,В.,
%Шоргин~С.\,Я.}
%Система $Geo_m/G/1/n$ с дисциплиной LIFO без прерывания
%обслуживания и ограничением на суммарный объем заявок~//
%Автоматика и телемеханика, 2011. №\,1. С.~107--120.
%%%%%
\bibitem{kmps}
\Au{Cascone A., Manzo R., Pechinkin A.\,V., Shorgin S.\,Ya.}  
Geo$_{m}/G/1/n$ system with LIFO discipline without 
interrupts and constrained total amount of customers~// 
Automation Remote Control, 2011.
Vol.\,72, No.\,1. P.~99--110.




\label{end\stat}

% 11.
\bibitem{BDPS}
\Au{Bocharov P.\,P., D'Apice~C., Pechinkin~A.\,V.,
Salerno~S.}
Queueing theory. Modern probability and
statistics ser.~--- Utrecht, Boston: VSP Publ., 2004.
 \end{thebibliography}
}
}


\end{multicols} %13
\def\stat{kondranin+ushakov}

\def\tit{СИСТЕМА ОБСЛУЖИВАНИЯ С~ОТНОСИТЕЛЬНЫМ ПРИОРИТЕТОМ  И~ПРОФИЛАКТИКАМИ ПРИБОРА$^*$}

\def\titkol{Система обслуживания с~относительным приоритетом  и~профилактиками прибора}

\def\aut{Е.\,С.~Кондранин$^1$,  В.\,Г.~Ушаков$^2$}

\def\autkol{Е.\,С.~Кондранин,  В.\,Г.~Ушаков}

\titel{\tit}{\aut}{\autkol}{\titkol}

\index{Кондранин Е.\,С.}
\index{Ушаков В.\,Г.}
\index{Kondranin E.\,S.}
\index{Ushakov V.\,G.}




{\renewcommand{\thefootnote}{\fnsymbol{footnote}} \footnotetext[1]
{Работа выполнена при финансовой поддержке РФФИ (проект 18-07-00678).}}


\renewcommand{\thefootnote}{\arabic{footnote}}
\footnotetext[1]{Факультет вычислительной математики и~кибернетики Московского государственного 
университета им.\ М.\,В.~Ломоносова, \mbox{ekondranin@yandex.ru}}
\footnotetext[2]{Факультет вычислительной математики и~кибернетики
Московского государственного университета им.\ М.\,В.~Ломоносова;
Институт проб\-лем информатики Федерального исследовательского
центра <<Информатика и~управ\-ле\-ние>> Российской академии наук,
\mbox{vgushakov@mail.ru}}

\vspace*{-10pt}




\Abst{Изучена одноканальная система
массового обслуживания с~двумя типами требований, бесконечным
числом мест для ожидания, гиперэкспоненциальным входящим потоком 
и~профилактиками обслуживающего прибора при освобождении системы.
Тип  требования определяется случайно с~заданными вероятностями 
в~момент его поступления в~систему обслуживания. Требования первого
типа имеют относительный приоритет перед требованиями второго
типа. Найдено нестационарное совместное распределение числа
требований каждого типа в~системе. Профилактики прибора
заключаются в~том, что в~момент освобождения системы от требований
прибор на случайное время с~заданным распределением становится
недоступным для обслуживания. Если за время профилактики поступает
хотя бы одно требование, то начинается нормальное функционирование
системы. Если требования не поступают, то прибор отправляется на
новую профилактику. Такие системы хорошо описывают
функционирование большого числа реальных вычислительных и~информационных систем.}

\KW{гиперэкспоненциальный поток; профилактики
обслуживающего прибора; одноканальная система; относительный
приоритет; длина очереди}

\DOI{10.14357/19922264180405}
  
%\vspace*{4pt}


\vskip 10pt plus 9pt minus 6pt

\thispagestyle{headings}

\begin{multicols}{2}

\label{st\stat}

\section{Введение}

В классической системе массового обслуживания ожидание требований
в очереди связано только с~занятостью обслуживающего прибора. В~то
же время в~реальных системах сам  прибор может пребывать как 
в~активном, так и~в~неактивном состоянии. Такое неактивное
состояние прибора (в~литературе на английском языке используется
термин vacation, а~на русском~--- профилактика или прогулка) может
быть связано со многими причинами. В~част\-ности, сис\-те\-мы
обслуживания с~профилактиками прибора хорошо описывают
функционирование  реальных вычислительных и~информационных систем,
в которых наряду с~основными требованиями имеются второстепенные.
Второстепенные требования всегда присутствуют в~сис\-те\-ме, а~их
обслуживание может проводиться только тогда, когда нет основных,
т.\,е.\ в~фоновом режиме.

С точки зрения самого процесса профилактики прибора существует
несколько ее разновидностей. Во-пер\-вых, могут быть разными
правила, задающие условия начала профилактики: прибор может брать
перерыв только при  полном исчерпании требований в~очереди
(exhaustive service) либо при наличии определенного их числа
(nonexhaustive service). Во-вто\-рых, могут быть разными правила
возвращения прибора в~работу. С~этой точки зрения различают случаи
однократного (single vacation) и~многократного (multiple vacation)
перерыва в~работе. В~первом случае ушедший на профилактику прибор
после ее окончания находится в~рабочем состоянии независимо от
наличия требований в~системе. Во втором случае прибор, не
обнаружив новых требований в~очереди, уходит на новую
профилактику.


В работах~[1--4] можно найти обзор известных результатов, большое
число постановок задач, описание различных приложений и~обширную
библиографию по анализу систем с~профилактиками обслуживающего
прибора.


В настоящей работе исследуется совместное распределение длин
очередей в~нестационарном режиме в~однолинейной системе 
с~ожиданием, гиперэкспоненциальным входящим потоком, двумя типами
требований и~относительным приоритетом. Аналогичная неприоритетная
система обслуживания исследована в~[5].

\vspace*{-6pt}

\section{Описание модели}

Рассматривается однолинейная система массового обслуживания 
с~двумя приоритетными классами требований. Входящий поток~---
гиперэкспоненциальный с~функцией распределения интервалов между
поступлениями требований вида:
\begin{multline*}
A(t)=\sum\limits_{i=1}^kc_i\left(1-e^{-a_it}\right),\enskip t>0,\enskip
a_i>0,\enskip c_i>0,\\
a_i\ne a_j\,,\enskip i\ne j\,,\enskip  \sum\limits_{i=1}^k c_i=1\,.
\end{multline*}

Каждое поступившее требование направляется в~первый класс 
с~вероятностью~$p$ и~во второй класс с~вероятностью $1\hm-p$
независимо от остальных требований. Требования первого класса
обладают относительным приоритетом перед требованиями второго
класса. Длительности обслуживания требований $i$-го приоритетного
класса~--- независимые в~совокупности и~не зависящие от входящего
потока случайные величины с~функцией распределения~$B_i(x)$,
$i\hm=1,2.$
 Если в~некоторый момент времени система освободилась от требований, 
 то обслуживающий прибор
 отправляется на профилактику, которая длится случайное время с~функцией 
 распределения~$C(x).$
 Не ограничивая общности, будем считать, что $B_i(x)\hm<1$
 и~$C(x)\hm<1$  для любого~$x$ 
 и~существуют плотности
 распределения~$b_i(x)$ и~$c(x).$
  Обозначим:
$$
 \beta_i(s)=\int\limits_0^{\infty}e^{-sx}b_i(x)\,dx\,;\enskip 
  \gamma(s)=\int\limits_0^{\infty}e^{-sx}c(x)\,dx\,.
$$
Пока прибор находится на профилактике, он не доступен для
обслуживания. Если за время профилактики поступают требования,
после ее завершения начинается их обслуживание. Если ни одно
требование не поступает, то прибор отправляется на новую
профилактику. Длительности различных профилактик являются
независимыми случайными величинами 
и~не зависят от входящего потока и~времен обслуживания.

\section{Вспомогательные результаты}

  Рассмотрим многочлен по $\mu$ степени $k$ вида:
\begin{multline}
\label{1}
\prod\limits_{i=1}^k\left(\mu+a_i\right)-{}\\
{}-
\left(pz_1+(1-p)z_2\right)\sum\limits_{j=1}^kc_ja_j\prod\limits_{i\ne
j}\left(\mu+a_i\right)\,.
\end{multline}
Занумеруем его корни $\mu_1(z_1,z_2),\ldots,\mu_k(z_1,z_2)$ таким образом,
чтобы они были непрерывными функциями и~$\mu_1(1,1)\hm=0.$ Тогда
$\mathrm{Re}\, \mu_j\left(z_1,z_2\right)\hm<0$, $|z_1|\hm<1$, 
$|z_2|\hm<1,$ $\mu_i(z_1,z_2)\hm\ne \mu_j(z_1,z_2),$ $ i\hm\ne j$,
$j\hm=1,\ldots,k.$ Обозначим:
$$
\alpha_m(z_1,z_2)=\prod\limits_{j\ne m}\left(\mu_m\left(z_1,z_2\right)-
\mu_j\left(z_1,z_2\right)\right)\,.
$$
Справедливы следующие леммы.

\smallskip

\noindent
\textbf{Лемма~1.}\
\textit{Для любого $l=1,\ldots,\:k$ система уравнений}
$$
z_j=\beta_j(s-\mu_l(z_1,z_2)),\ \ j=1,2,
$$
\textit{имеет единственное решение $z_i=z_{il}(s)$ такое, 
что $|z_{il}(s)|\hm<1$ при $l\hm=2,\ldots, k,$ $\mathrm{Re}\, s\hm\geqslant 0,$ 
а~$z_{i1}(0)\hm=1$, $|z_{i1}(s)|\hm<1$ при} $\mathrm{Re}\, s\hm> 0$, $i\hm=1,2.$

\smallskip

\noindent
\textbf{Лемма~2.}\
\textit{При каждом $l\hm=1,\ldots,k$ уравнение}
$$
z_1=\beta_1\left(s-\mu_l(z_1,z_2)\right)
$$
\textit{имеет единственное решение $z_1\hm=z_{1l}(z_2,s),$ 
аналитическое в~области $\mathrm{Re}\, s\hm>0$, $|z_2|\hm<1.$
}

\smallskip

Положим
$$
\lambda_l(s)=\mu_l\left(z_{1l}(s),z_{2l}(s)\right)\,.
$$




\section{Распределение длины очереди}

  Гиперэкспоненциальный поток можно рас\-смат\-ри\-вать как
пуассоновский поток со случайной интен\-сив\-ностью~$a,$ которая
принимает $k$ различных значений $a_1,\ldots,a_k$  с~вероятностями
$c_1,\ldots,c_k.$ Текущее значение~$a$ разыгрывается в~момент
поступления требования и~не меняется между двумя соседними
поступлениями. Введем случайный процесс~$j(t)$ такой, что если
$a\hm=a_j$ в~момент времени $t,$ то $j(t)\hm=j.$

Целью работы является нахождение распределения случайного процесса
$\left(L_1(t),L_2(t)\right),$ где $L_i(t)$~--- число требований из
$i$-го приоритетного класса, находящихся в~системе в~момент
времени~$t.$

При сделанных предположениях относительно параметров изучаемой
системы обслуживания\linebreak процесс $\left(L_1(t),L_2(t)\right)$ не
является, вообще говоря, марковским. Пусть $i(t)=i$, $i\hm=1,2,$ если
в~момент времени~$t$ обслуживается требование из $i$-го
приоритетного класса, и~$i(t)\hm=0,$ если в~момент времени~$t$ прибор
находится на профилактике. Случайный процесс~$x(t)$ определим
следующим образом. Если $i(t)\hm\ne 0,$ то $x(t)$ есть
время, прошедшее с~начала обслуживания требования, находящегося на
приборе, до момента~$t.$ Если $i(t)\hm=0,$ то $x(t)$ есть время,
прошедшее с~начала профилактики прибора до момента~$t.$ Случайный
процесс $\left(L_1(t),L_2(t),i(t),j(t),x(t)\right)$ является
однородным марковским процессом. Положим
\begin{multline*}
P_{ij}(n_1,n_2,x,t)=\fr{\partial}{\partial x}
\mathbf{P}\left(L_1(t)=n_1,L_2(t)=n_2,\right.\\
\left. i(t)=i,j(t)=j,x(t)<x
\vphantom{L_1}\right)\,,\enskip 
 x\geqslant 0,\\ 
 j=1,\ldots,k,\enskip i=0,1,2;
\end{multline*}
\begin{gather*}
\eta_i(x)=\fr{b_i(x)}{1-B_i(x)},\ i=1,2;\enskip 
\eta_0(x)=\fr{c(x)}{1-C(x)}\,;\\
\delta_{i,j}=\begin{cases}
1,&\ i=j;\\ 
0,&\ i\ne j\,.
\end{cases}
\end{gather*}
Функции $P_{ij}(n_1,n_2,x,t)$  удовлетворяют при $x\hm>0$
системам дифференциальных уравнений:
\begin{multline}
\label{3}
\fr{\partial P_{ij}(n_1,n_2,x,t)}{\partial t}+\fr{\partial
P_{ij}(n_1,n_2,x,t)}{\partial
x}={}\\
{}=-(a_j+\eta_i(x))P_{ij}(n_1,n_2,x,t)+ {}\\
{}+
c_j\sum\limits_{l=1}^ka_l\left(p\:P_{il}(n_1-1,n_2,x,t)+{}\right.\\
\left.{}+
(1-p)P_{il}(n_1,n_2-1,x,t)\right)
\end{multline}
и краевым условиям при $x\hm=0$:
\begin{multline}
\label{5}
P_{0j}(n_1,n_2,0,t)=0,\ n_1+n_2>0;\\
P_{0j}(0,0,0,t)=\int\limits_0^{\infty}P_{0j}(0,0,x,t)\eta_0(x)\,dx+{}\\
 {}+\int\limits_0^{\infty}P_{1j}(1,0,x,t)\eta_1(x)dx+{}\\
 {}+
\int\limits_0^{\infty}P_{2j}(0,1,x,t)\eta_2(x)\,dx\,;
\end{multline}

\vspace*{-12pt}

\noindent
\begin{multline}
\label{6}
P_{1j}(n_1,n_2,0,t)+P_{2j}(n_1,n_2,0,t)={}\\
{}=\int\limits_0^{\infty}P_{1j}(n_1+1,n_2,x,t)\eta_1(x)\,dx+{}\\
{}+
\int\limits_0^{\infty}P_{2j}(n_1,n_2+1,x,t)\eta_2(x)\,dx+{}\\
{}+\int\limits_0^{\infty}P_{0j}(n_1,n_2,0,t)\eta_0(x)\,dx\,.
\end{multline}

Будем предполагать, что в~начальный момент времени $t\hm=0$ система
свободна от требований, а~с~начала профилактики прибора прошло
случайное время с~заданным распределением с~плотностью $d(x).$
Таким образом,
\begin{align*}
P_{ij}\left(n_1,n_2,x,0\right)&=0,\ i=1,2;
\\
P_{0j}\left(n_1,n_2,x,0\right)&=c_jd(x)\delta_{n_1+n_2,0},\ \
j=1,\ldots,k\,.
\end{align*}
Положим
\begin{multline*}
p_{ij}\left(z_1,z_2,x,s\right)={}\\
{}=\sum\limits_{n_1=0}^{\infty}
\sum\limits_{n_2=0}^{\infty}z_1^{n_1}z_2^{n_2}\!
\int\limits_0^{\infty}e^{-st}P_{ij}(n_1,n_2,x,t)\,dt\,;
\end{multline*}
$$
  \psi(s)=\int\limits_0^{\infty}e^{-sx}\,dx
  \int\limits_0^{\infty}\fr{c(u+x)d(u)}{1-C(u)}\,du\,.
$$
Тогда, учитывая начальные условия,  из \eqref{3}
получаем:
\begin{multline}
\label{7} 
\fr{\partial p_{ij}(z_1,z_2,x,s)}{\partial x}={}\\
{}=-\left(s+a_j+\eta_i(x)\right)p_{ij}
\left(z_1,z_2,x,s\right)+{}\\
{}+c_j\left(pz_1+(1-p)z_2\right)
\sum\limits_{l=1}^ka_lp_{il}\left(z_1,z_2,x,s\right),\\ 
i=1,2;
\end{multline}

\vspace*{-12pt}

\noindent
\begin{multline}
\label{8} 
\fr{\partial p_{0j}(z_1,z_2,x,s)}{\partial x}={}\\
{}=-\left(s+a_j+\eta_0(x)\right)p_{0j}\left(z_1,z_2,x,s\right)+{}\\
{}+c_j\left(pz_1+(1-p)z_2\right)\sum\limits_{l=1}^ka_lp_{0l}\left(z_1,z_2,x,s\right)+{}\\
{}+ c_jd(x).
\end{multline}
Решения \eqref{7} и~\eqref{8} имеют вид:
\begin{multline}
\label{9}
p_{ij}\left(z_1,z_2,x,s\right)=\left(1-B_i(x)\right)c_j\times{}\\
{}\times \sum\limits_{m=1}^k\fr{\gamma_i^{(m)}(z_1,z_2,s)}{\mu_m(z_1,z_2)+a_j}\,
e^{-(s-\mu_m(z_1,z_2))x}\,,\\
 i=1,2\,,
\end{multline}
\vspace*{-12pt}

\noindent
\begin{multline}
\label{10}
p_{0j}\left(z_1,z_2,x,s\right)={}\\
{}=\left(1-C(x)\right)
c_j\!\!\sum\limits_{m=1}^k\!\! e^{-(s-\mu_m(z_1,z_2))x}\!
\!\left(\!
\vphantom{\int\limits_{l=1}^k}
\delta^{(m)}\left(z_1,z_2,s\right)+{}\right.\\
%\left.
{}+\alpha_m^{-1}\left(z_1,z_2\right)
\prod\limits_{l=1}^k
\left(\mu_m\left(z_1,z_2\right)+a_l\right)\times{}\\
\left.{}\times \int\limits_0^x\!
e^{(s-\mu_m(z_1,z_2))u}
\fr{d(u)}{1-C(u)}\,du
\right)
\!\Bigg/ \!\left(\mu_m\left(z_1,z_2\right)+{}\right.\\
\left.{}+a_j\right)\,,
\end{multline}
где функции $\gamma_i^{(m)}(z_1,z_2,s)$  и~$\delta^{(m)}(z_1,z_2,s)$ являются
произвольными функциями указанных переменных и~определяются из
краевых условий. Из~\eqref{5} и~\eqref{6} получаем:
\begin{multline}
\label{11}
p_{1j}\left(z_1,z_2,0,s\right)+p_{2j}\left(z_1,z_2,0,s\right)={}\\
{}=z_1^{-1}\int\limits_0^{\infty}p_{1j}\left(z_1,z_2,x,s\right)\eta_1(x)\,dx+{}
\\
+z_2^{-1}\int\limits_0^{\infty}p_{2j}\left(z_1,z_2,x,s\right)\eta_2(x)\,dx+{}\\
{}+
\int\limits_0^{\infty}p_{0j}\left(z_1,z_2,x,s\right)\eta_0(x)\,dx
-p_{0j}\left(z_1,z_2,0,s\right)\,.
\end{multline}
Заметим, что $p_{0j}(z_1,z_2,0,s)$ не зависит от $z_1$ и~$z_2,$ т.\,е.\
$p_{0j}(z_1,z_2,0,s)\hm=q_j(s).$ 
Подставляя~\eqref{9} и~\eqref{10} в~\eqref{11}, получаем:
\begin{multline}
\label{12}
\gamma_1^{(m)}\left(z_1,z_2,s\right)\left(1-z_1^{-1}\beta_1(s-\mu_m(z_1,z_2))\right)+{}\\
{}+
\gamma_2^{(m)}(z_1,z_2,s)\left(1-z_2^{-1}\beta_2(s-\mu_m(z_1,z_2))\right)={}\\
{} =
\delta^{(m)}\left(z_1,z_2,s\right)\left(\gamma\left(s-\mu_m\left(z_1,z_2\right)\right)-1\right)+{}\\
{}+
\alpha_m^{-1}\left(z_1,z_2\right)\prod\limits_{l=1}^k
\left(\mu_m\left(z_1,z_2\right)+a_l\right)\psi\left(s-\mu_m(z_1,z_2)\right),\\
j=1,\ldots,k.
\end{multline}
В силу леммы~1 левая часть~\eqref{12} обращается в~0 при
$z_1\hm=z_{1m}(s)$ и~$z_2\hm=z_{2m}(s)$, $m\hm=1,\ldots,k.$ Следовательно,
\begin{multline}
\label{13}
\delta^{(m)}\left(z_{1m}(s),z_{2m}(s),s\right)={}\\
{}=\fr{\psi(s-\lambda_m(s))}{\alpha_m(z_{1m}(s),z_{2m}(s))
(1-\gamma(s-\lambda_m(s)))}\times{}\\
{}\times \prod\limits_{l=1}^k\left(\lambda_m(s)+a_l\right).
\end{multline}
Из \eqref{10} следует, что
$$
q_j(s)=c_j\sum\limits_{m=1}^k\fr{\delta^{(m)}(z_1,z_2,s)}{\mu_m(z_1,z_2)+a_j},\
j=1,\ldots,k .
$$
Решая эту систему уравнений относительно
$\delta^{(m)}(z_1,z_2,s),$ получаем:
\begin{multline}
\label{n1}
\delta^{(m)}(z_1,z_2,s)=\left(pz_1+(1-p)z_2\right)\times{}\\
{}\times
\fr{\prod\nolimits_{j=1}^k(\mu_m(z_1,z_2)+a_j)}
{\alpha_m(z_1,z_2)}\sum\limits_{l=1}^k\frac{a_lq_l(s)}{\mu_m(z_1,z_2)+a_l}.
\end{multline}
Подставляя в~\eqref{n1} $z_1\hm=z_{1m}(s)$ и~$z_2\hm=z_{2m}(s),$ имеем:
\begin{multline}
\label{14}
\delta^{(m)}\left(z_{1m}(s),z_{1m}(s),s\right)={}\\
{}=
\left(pz_{1m}(s)+(1-p)z_{2m}(s)\right)\times{}\\
{}\times
\fr{\prod\nolimits_{j=1}^k
(\lambda_m(s)+a_j)}{\alpha_m(z_{1m}(s),z_{1m}(s))}
\sum\limits_{l=1}^k\fr{a_lq_l(s)}{\lambda_m(s)+a_l}\,.
\end{multline}
Сравнивая два представления~\eqref{13} в~\eqref{14} для
$\delta^{(m)}(z_m(s),s),$ получаем систему уравнений для~$q_l(s)$:
\begin{multline*}
\sum\limits_{l=1}^k\fr{a_lq_l(s)}{\lambda_m(s)+a_l}={}\\
{}=\fr{\psi(s-\lambda_m(s))}{(pz_{1m}(s)+(1-p)z_{2m}(s))
(1-\gamma(s-\lambda_m(s)))},\\
m=1,\ldots,k\,,
\end{multline*}
из которой находим
\begin{multline}
\hspace*{-3pt}q_l(s)=c_l\prod\limits_{j=1}^k
\left(\lambda_l(s)+a_j\right) 
\sum\limits_{m=1}^k
%\fr
\psi(s-\lambda_m(s))\!\Bigg/ \!
\Bigg(\left(1-{}\right.\\
\left.
{}-\gamma\left(s-\lambda_m(s)\right)\right)(\lambda_m(s)+a_l)\times{}\\
{}\times \prod\limits_{n\ne m}(\lambda_m(s)-\lambda_n(s))\!\Bigg).
\label{15}
\end{multline}
Подставляя \eqref{15} в~\eqref{n1} и~учитывая~\eqref{1}, получаем:
\begin{multline*}
\delta^{(m)}(z_1,z_2,s)=\fr{(pz_1+(1-p)z_2)}{\alpha_m(z_1,z_2)}\times
\\
\times\sum\limits_{j=1}^k
\fr{\psi(s-\lambda_j(s))\prod\nolimits_{l=1}^k(\lambda_j(s)+a_l)}
{(pz_{1j}(s)+(1-p)z_{2j}(s))(1-\gamma(s-\lambda_j(s)))}\times{}\\
{}\times\prod\limits_{\nu\ne j}
\fr{\mu_m(z_1,z_2)-\lambda_{\nu}(s)}{\lambda_j(s)-\lambda_{\nu}(s)}\,.
\end{multline*}
Положим
$$
\lambda_m(z_2,s)=\mu_m\left(z_{1m}(z_2,s),z_2\right),\enskip m=1,\ldots,k\,.
$$
Подставляя в~\eqref{12} $z_1\hm=z_{1m}(z_2,s)$, имеем:
\begin{multline}
\label{1q}
\gamma_2^{(m)}\left(z_{1m}(z_2,s),z_2,s\right)={}\\
{}=\fr{\delta^{(m)}(z_{1m}(z_2,s),z_2,s)(\gamma_m(s-\lambda_m(z_2,s))-1)}
{1-z_2^{-1}\beta_2(s-\lambda_m(z_2,s))}+{}
\\
{}+\alpha_m^{-1}(z_{1m}(z_2,s),z_2)\psi(s-\lambda_m(z_2,s))
\prod\limits_{l=1}^k\left(\lambda_m(z_2,s)+{}\right.\\
\left.{}+a_l\right)\!\Bigg/\!
\left(
1-z_2^{-1}\beta_2(s-\lambda_m(z_2,s))\right).
\end{multline}
Далее, из~\eqref{9} следует:
$$
p_{2j}(z_1,z_2,0,s)=c_j\sum\limits_{m=1}^k
\fr{\gamma_2^{(m)}(z_1,z_2,s)}{\mu_m(z_1,z_2)+a_j}\,.
$$
Отсюда
\begin{multline}
\label{2q}
\gamma_2^{(m)}(z_1,z_2,s)=\fr{pz_1+(1-p)z_2}{\alpha_m(z_1,z_2)}\times{}\\
{}\times
\prod\limits_{j=1}^k(\mu_m(z_1,z_2)+a_j)
\sum\limits_{l=1}^k\fr{a_lp_{2l}(z_1,z_2,0,s)}{\mu_m(z_1,z_2)+a_l}\,.
\end{multline}
Так как $p_{2j}(z_1,z_2,0,s)$ не зависит от $z_1$, то
\begin{multline}
\label{3q}
p_{2j}\left(z_1,z_2,0,s\right)={}\\
{}=c_j
\sum\limits_{m=1}^k\fr{\gamma_2^{(m)}\left(z_{1m}(z_2,s),z_2,s\right)}{\lambda_m(z_2,s)+a_j}\,.
\end{multline}
Таким образом, соотношения~\eqref{1q}--\eqref{3q} полностью
определяют $\gamma_2^{(m)}(z_1,z_2,s)$ при любых $z_1$ и~$z_2$.
Теперь из~\eqref{12} можно найти $\gamma_2^{(m)}(z_1,z_2,s)$.

Все функции, необходимые для вычисления $p_{ij}(z_1,z_2,x,s)$,
$i\hm=0,1,2$, $j\hm=1,\ldots,k,$ найде-\linebreak\vspace*{-12pt}

\columnbreak

\noindent
ны. Искомая производящая функция
процесса $(L_1(t),L_2(t))$ равна:

\noindent
\begin{multline*}
\int\limits_0^{\infty}e^{-st}\mathbf{E}
z_1^{L_1(t)} z_2^{L_2(t)}\,dt={}\\
{}=
\sum\limits_{i=0}^2\sum\limits_{j=1}^k\int\limits_0^{\infty}p_{ij}
\left(z_1,z_2,x,s\right)\,dx\,.
\end{multline*}

\vspace*{-18pt}

{\small\frenchspacing
 {%\baselineskip=10.8pt
 \addcontentsline{toc}{section}{References}
 \begin{thebibliography}{9}
\bibitem{1-u}
\Au{Doshi B.\,T.} Queueing systems with vacations~--- a~survey~// 
Queueing Syst., 1986. Vol.~1.  P.~29--66.
\bibitem{2-u}
\Au{Takagi H.} Time-dependent analysis of $M\vert G\vert 1$ vacation models 
with exhaustive service~// Queueing Syst.,
1990. Vol.~6.  P.~369--390.
\bibitem{3-u}
\Au{Li J., Tian N., Zhang~Z.\,G. , Luh~H.\,P.} 
Analysis of the $M\vert G\vert 1$ queue with exponentially working vacations~--- 
a~matrix analytic approach~// Queueing Syst., 2009. Vol.~61.
P.~139--166.
\bibitem{4-u}
\Au{Bouman N., Borst S.\,C., Boxma~O.\,J., Leeuwaarden~J.\,S.\,H.} 
Queues with random back-offs~// Queueing Syst.,
2014. Vol.~77. P.~33--74.
\bibitem{5-u}
\Au{Ушаков~В.\,Г.} Система обслуживания с~гиперэкспоненциальным входящим потоком 
и~профилактиками прибора~// Информатика и~её применения, 2016. Т.~10. 
Вып.~2. С.~93--98.
 \end{thebibliography}

 }
 }

\end{multicols}

\vspace*{-9pt}

\hfill{\small\textit{Поступила в~редакцию 11.05.18}}

\vspace*{6pt}

%\pagebreak

%\newpage

%\vspace*{-28pt}

\hrule

\vspace*{2pt}

\hrule

%\vspace*{-2pt}

\def\tit{A~HEAD OF~THE~LINE PRIORITY QUEUE\\ WITH~WORKING VACATIONS}

\def\titkol{A head of the line priority queue with working vacations}

\def\aut{E.\,S.~Kondranin$^1$ and~V.\,G.~Ushakov$^{1,2}$}

\def\autkol{E.\,S.~Kondranin and~V.\,G.~Ushakov}

\titel{\tit}{\aut}{\autkol}{\titkol}

\vspace*{-11pt}


\noindent
$^1$Department of 
Mathematical Statistics, Faculty of Computational Mathematics and Cybernetics, 
M.\,V.~Lo\-mo-\linebreak
$\hphantom{^1}$no\-sov Moscow State University, 1-52~Leninskiye Gory, 
Moscow 119991, GSP-1, Russian Federation

\noindent
$^2$Institute of Informatics Problems, Federal Research Center 
``Computer Science and Control'' of the Russian\linebreak
$\hphantom{^1}$Academy of Sciences,  44-2~Vavilov Str., Moscow 119333, Russian Federation

\def\leftfootline{\small{\textbf{\thepage}
\hfill INFORMATIKA I EE PRIMENENIYA~--- INFORMATICS AND
APPLICATIONS\ \ \ 2018\ \ \ volume~12\ \ \ issue\ 4}
}%
 \def\rightfootline{\small{INFORMATIKA I EE PRIMENENIYA~---
INFORMATICS AND APPLICATIONS\ \ \ 2018\ \ \ volume~12\ \ \ issue\ 4
\hfill \textbf{\thepage}}}

\vspace*{3pt}



\Abste{The authors analyze the single-server queueing system with 
two types of customers, head of the line priority, hyperexponential 
input stream, and working vacations. The authors obtain the Laplace 
transform (with respect to an arbitrary point in time) of the joint 
distribution of server state, queue size, and elapsed time in that state. 
The authors restrict themselves to a~system with exhaustive service (the 
queue must be empty when the server starts a vacation) and multiple vacations. 
The queueing systems with vacations have been well studied because of their 
applications in modeling computer networks, communication, and manufacturing 
systems. For example, in many digital systems, the processor is multiplexed 
among a~number of jobs and, hence, is not available all the time to handle one job type. 
Besides such an application, theoretical interest in vacation models 
has been aroused with respect to their relationship with polling models.}

\KWE{hyperexponential input stream; working vacations; single server; 
head of the line priority; queue length}



\DOI{10.14357/19922264180405}

\vspace*{-14pt}

\Ack
\noindent
This work was supported by the Russian Foundation for Basic Research 
(project 18-07-00678).


%\vspace*{6pt}

  \begin{multicols}{2}

\renewcommand{\bibname}{\protect\rmfamily References}
%\renewcommand{\bibname}{\large\protect\rm References}

{\small\frenchspacing
 {%\baselineskip=10.8pt
 \addcontentsline{toc}{section}{References}
 \begin{thebibliography}{9}
\bibitem{1-u-1}
\Aue{Doshi, B.\,T.} 1986. Queueing systems with vacations~--- a~survey. 
\textit{Queueing Syst.} 1:29--66.
\bibitem{2-u-1}
\Aue{Takagi, H.} 1990. Time-dependent analysis of $M\vert G\vert M\vert 1$ 
vacation models with exhaustive service. \textit{Queueing Syst.} 6:369--390.
\bibitem{3-u-1}
\Aue{Li, J., N. Tian, Z.\,G.~Zhang,  and H.\,P.~Luh.} 2009. Analysis of the 
$M\vert G\vert 1$ queue with exponentially working vacations~--- 
a~matrix analytic approach. \textit{Queueing Syst.} 61:139--166.
{\looseness=1

}
\bibitem{4-u-1}
\Aue{Bouman, N., S.\,C.~Borst, O.\,J.~Boxma, and J.\,S.\,H.~Leeuwaarden.} 
2014. Queues with random back-offs. \textit{Queueing Syst.} 77:33--74.
\bibitem{5-u-1}
\Aue{Ushakov, V.\,G.} 2016. Sistema obsluzhivaniya s~gipereksponentsialnym 
vkhodyashchim potokom i~profilaktikami\linebreak pribora [Queueing system with working 
vacations and hyperexponential input stream]. 
\textit{Informatika i~ee Primeneniya~--- Inform. Appl.} 10(2):93--98.
\end{thebibliography}

 }
 }

\end{multicols}

\vspace*{-6pt}

\hfill{\small\textit{Received May 11, 2018}}

%\pagebreak

%\vspace*{-18pt}

\Contr

\noindent
\textbf{Kondranin Egor S.} (b.\ 1995)~---  MSc student, Department of 
Mathematical Statistics, Faculty of Computational Mathematics and Cybernetics, 
M.\,V.~Lomonosov Moscow State University, 1-52~Leninskiye Gory, 
Moscow 119991, GSP-1, Russian Federation; \mbox{ekondranin@yandex.ru}

\vspace*{6pt}

\noindent
\textbf{Ushakov Vladimir G.} (b.\ 1952)~--- 
Doctor of Science in physics and mathematics, professor, Department of Mathematical 
Statistics, Faculty of Computational Mathematics and Cybernetics, 
M.\,V.~Lomonosov Moscow State University, 1-52~Leninskiye Gory, Moscow 119991, 
GSP-1, Russian Federation; 
senior scientist, Institute of Informatics Problems, Federal Research Center 
``Computer Science and Control'' of the Russian Academy of Sciences, 
44-2~Vavilov Str., Moscow 119333, Russian Federation; \mbox{vgushakov@mail.ru}
\label{end\stat}

\renewcommand{\bibname}{\protect\rm Литература}            %14




%\end{document}

%   { %\Large  
   { %\baselineskip=16.6pt
   
   \vspace*{-48pt}
   \begin{center}\LARGE
   \textit{Предисловие}
   \end{center}
   
   %\vspace*{2.5mm}
   
   \vspace*{25mm}
   
   \thispagestyle{empty}
   
   { %\small 

    
Вниманию читателей журнала <<Информатика и её применения>> предлагается 
очередной тематический выпуск <<Вероятностно-статистические методы и 
задачи информатики и информационных технологий>>. Предыдущие тематические 
выпуски журнала по данному направлению вышли в 2008~г.\ (т.~2, вып.~2), 
в 2009~г.\ (т.~3, вып.~3) и в 2010~г.\ (т.~4, вып.~2). 

Статьи, собранные в данном журнале, посвящены разработке новых вероятностно-статистических 
методов, ориентированных на применение к решению конкретных задач информатики и информационных 
технологий, а также~--- в ряде случаев~--- и других прикладных задач. Проблематика, охватываемая 
публикуемыми работами, развивается в рамках научного сотрудничества между Институтом проблем 
информатики Российской академии наук (ИПИ РАН) и Факультетом вычислительной математики и 
кибернетики Московского государственного университета им.\ М.\,В.~Ломоносова в ходе работ 
над совместными научными проектами (в том числе в рамках функционирования 
Научно-образовательного центра <<Вероятностно-статистические методы анализа рисков>>). 
Многие из авторов статей, включенных в данный номер журнала, являются активными участниками 
традиционного международного семинара по проблемам устойчивости стохастических моделей, 
руководимого В.\,М.~Золотаревым и В.\,Ю.~Королевым; регулярные сессии этого семинара 
проводятся под эгидой МГУ и ИПИ РАН (в 2011~г.\ указанный семинар проводится в октябре 
в Калининградской области РФ). 

Наряду с представителями ИПИ РАН и МГУ в число авторов данного выпуска журнала входят 
ученые из Научно-исследовательского института системных исследований РАН, Института 
проблем технологии микроэлектроники и особочистых материалов РАН, Института 
прикладных математических исследований Карельского НЦ РАН, Московского 
авиационного института, Вологодского государственного педагогического университета, 
НИИММ им.\ Н.\,Г.~Чеботарева, Казанского государственного университета, Дебреценского 
университета (Венгрия).

Несколько статей выпуска посвящено разработке и применению стохастических методов и 
информационных технологий для решения различных прикладных задач. В~работе В.\,Г.~Ушакова 
и О.\,В.~Шестакова рассмотрена задача определения вероятностных характеристик случайных 
функций по распределениям интегральных преобразований, возникающих в задачах эмиссионной 
томографии. В~статье Д.\,О.~Яковенко и М.\,А.~Целищева рассмотрены некоторые вопросы 
математической теории риска и предложен новый подход к диверсификации инвестиционных 
портфелей. Работа И.\,А.~Кудрявцевой и А.\,В.~Пантелеева посвящена построению и 
исследованию математической модели, описывающей динамику сильноионизованной плазмы. 
В~статье П.\,П.~Кольцова изучается качество работы ряда алгоритмов сегментации изображений. 
Статья А.\,Н.~Чупрунова и И.~Фазекаша посвящена вероятностному анализу числа без\-оши\-бочных 
блоков при помехоустойчивом кодировании; получены усиленные законы больших чисел для указанных 
величин.

В данном выпуске традиционно присутствует тематика, весьма активно разрабатываемая в течение 
многих лет специалистами ИПИ РАН и МГУ,~--- методы моделирования и управления для 
информационно-телекоммуникационных и вычислительных систем, в частности методы 
теории массового обслуживания. В~статье А.\,И.~Зейфмана с соавторами рассматриваются 
модели обслуживания, описываемые марковскими цепями с непрерывным временем в случае 
наличия катастроф. В~работе М.\,М.~Лери и И.\,А.~Чеплюковой рассматриваются случайные 
графы Интернет-типа, т.\,е.\ графы, степени вершин которых имеют степенные распределения; 
такие задачи находят применение при исследовании глобальных сетей передачи данных. 
Работа Р.\,В.~Разумчика посвящена исследованию систем массового обслуживания специального 
вида~--- с отрицательными заявками и хранением вытесненных заявок.

Ряд статей посвящен развитию перспективных теоретических 
вероятностно-статистических методов, которые находят широкое применение в различных 
задачах информатики и информационных технологий. В~работе В.\,Е.~Бенинга, А.\,К.~Горшенина 
и В.\,Ю.~Королева рассмотрена задача статистической проверки гипотез о числе компонент 
смеси вероятностных распределений, приводится конструкция асимптотически наиболее мощного 
критерия. Результаты этой работы найдут применение в ряде прикладных задач, использующих 
математическую модель смеси вероятностных распределений (в информатике, моделировании 
финансовых рынков, физике турбулентной плазмы и~т.\,д.). В~статье В.\,Ю.~Королева, 
И.\,Г.~Шевцовой и С.\,Я.~Шоргина строится новая, улучшенная оценка точности нормальной 
аппроксимации для пуассоновских случайных сумм; как известно, указанные случайные суммы 
широко используются в качестве моделей многих реальных объектов, в том числе в информатике, 
физике и других прикладных областях. Работа В.\,Г.~Ушакова и Н.\,Г.~Ушакова посвящена 
исследованию ядерной оценки плотности распределения; эти результаты могут применяться, 
в част\-ности, при анализе трафика в телекоммуникационных системах. Серьезные приложения 
в статистике могут получить результаты работы О.\,В.~Шестакова, в которой доказаны оценки 
скорости сходимости распределения выборочного абсолютного медианного отклонения к нормальному 
закону. 

\smallskip

Редакционная коллегия журнала выражает надежду, что данный тематический  выпуск 
будет интересен специалистам в области теории вероятностей и математической статистики 
и их применения к решению задач информатики и информационных технологий.
     
     %\vfill 
     \vspace*{20mm}
     \noindent
     Заместитель главного редактора журнала <<Информатика и её 
применения>>,\\
     директор ИПИ РАН, академик  \hfill
     \textit{И.\,А.~Соколов}\\
     
     \noindent
     Редактор-составитель тематического выпуска,\\
     профессор кафедры математической статистики факультета\\
      вычислительной математики и кибернетики МГУ им.\ М.\,В.~Ломоносова,\\
     ведущий научный сотрудник ИПИ РАН,\\ 
доктор физико-математических наук \hfill
      \textit{В.\,Ю.~Королев}
     
     } }
     }

%%%%%%%%%%%%%%%%%%%%%%%%%%%%%%%%%%%%%%%%%%%%%%%


                       


\def\stat{rez}
{%\hrule\par
%\vskip 7pt % 7pt
\raggedleft\Large \bf%\baselineskip=3.2ex
Р\,Е\,Ц\,Е\,Н\,З\,И\,И \vskip 17pt
    \hrule
    \par
\vskip 6pt plus 6pt minus 3pt }

%\thispagestyle{headings} %с верхним колонтитулом
%\thispagestyle{myheadings} %с нижним колонтитулом, но в верхнем РЕЦЕНЗИИ

\def\tit{НОВАЯ КНИГА И.\,Н.~СИНИЦЫНА, А.\,С.~ШАЛАМОВА <<ЛЕКЦИИ ПО ТЕОРИИ 
ИНТЕГРИРОВАННОЙ ЛОГИСТИЧЕСКОЙ ПОДДЕРЖКИ>> (М.: ТОРУС ПРЕСС, 2012. 624~с.)}

%1
\def\aut{Д.ф.-м.н., профессор С.\,Я.~Шоргин}

\def\auf{\ }

\def\leftkol{\ % РЕЦЕНЗИИ
}

\def\rightkol{ \ } 

%\def\leftkol{\ } % ENGLISH ABSTRACTS}

%\def\rightkol{\ } %ENGLISH ABSTRACTS}

%\def\leftkol{РЕЦЕНЗИИ}

%\def\rightkol{РЕЦЕНЗИИ}

\titele{\tit}{\aut}{\auf}{\leftkol}{\rightkol}
\vspace*{-18pt}


     \label{st\stat}

     \begin{multicols}{2}
     {\small
     {\baselineskip=10.1pt
     

      В книге представлено системное изложение теоретических основ одного из новейших 
направлений в \mbox{об\-ласти} экономики послепродажного обслуживания изделий наукоемкой 
продукции (ИНП) длительного пользования~--- интегрированной логистической поддержки
(ИЛП). 
{\looseness=1

}

Приведены также результаты новых работ, выполненных в Институте проблем информатики 
Российской академии наук в рамках научного направления <<Информационные технологии и 
анализ сложных сис\-тем>>.
 {%\looseness=1

}
     
      Излагаемые в книге научные подходы позво\-ляют карди\-наль\-но реформировать 
существующие системы производства и эксплуатации ИНП путем создания и внед\-ре\-ния 
методов рационального и оптимального управ\-ле\-ния процессами расходования 
вре\-мен\-н$\acute{\mbox{ы}}$х, 
мате\-ри\-аль\-ных, трудовых и других ресурсов на всех стадиях жизненного цикла изделий (ЖЦИ) по 
критериям экономической целесообразности и эф\-фек\-тив\-ности.
  {\looseness=1

}
    
      В книге приведен краткий обзор причин возник\-новения и
      развития CALS-методологии как основы 
современных международных стандартов по созданию и функционированию глобальных 
ин\-фор\-ма\-ци\-он\-но-ком\-му\-ни\-ка\-ци\-он\-ных систем, ее ключевых возможностей и эффективности 
результатов ее использования. 
Авторы %\linebreak 
предлагают ряд научных обоснований для разработки 
единой теории проектирования и управления систем ИЛП для полноценного использования 
преимуществ %\linebreak
 суще\-ст\-ву\-ющей методологии, определяют \mbox{общую} структурную схему 
комплексной системы <<ИНП-СППО>> и необходимость разработки для ее описания 
гибридных стохастических моделей.
{%\looseness=1

}

%\columnbreak
      
      Книга состоит из пяти частей, где последовательно излагается материал по каждой из 
следующих тем: <<Интегрированная логистическая поддержка>>, <<Теория гибридных 
стохастических систем и компьютерная поддержка исследований и разработок>>, <<Основы 
математического моделирования, анализа и синтеза систем послепродажного обслуживания>>, 
<<Определение и анализ показателей экспортного потенциала ИНП при проектировании>>, 
<<Задачи управления поддержкой послепродажного обслуживания>>, а также 
<<Моделирование инвестиционных процессов ИЛП в условиях неравновесных финансовых 
рынков>>. 
   
      В конце каждой главы приведены выводы и даны вопросы и задания для 
самоконтроля. В~приложениях содержатся основные определения по программам работ по 
анализу ИЛП, логистическим базам данных и компьютерным решениям, эквивалентной статистической 
линеаризации нелинейных преобразований ИЛП, справочный материал, а также развернутые 
уравнения для вероятностных характеристик.


      \def\leftkol{РЕЦЕНЗИИ}

\def\rightkol{РЕЦЕНЗИИ} 

      
      Книга заинтересует широкий круг специалистов и может быть использована научными 
проектными организациями в сфере промышленного производства ИНП. Большое количество 
иллюстраций, примеров и вопросов, обращенных к читателю, позволяет использовать книгу 
также в качестве учебного пособия для студентов и аспирантов машиностроительных, 
транспортных и~других специальностей, а также для самостоятельного изучения. 
{%\looseness=-1

}

Книга 
представляет несомненный интерес для специалистов и студентов в области прикладной 
математики и информатики.
    

}

}
\end{multicols}

%\newpage

%\end{document}

\include{obchak}

%\end{document}



\def\stat{authorsrus}
{%\hrule\par
%\vskip 7pt % 7pt
\raggedleft\Large \bf%\baselineskip=3.2ex
О\,Б\ \ А\,В\,Т\,О\,Р\,А\,Х \vskip 17pt
    \hrule
    \par
\vskip 21pt plus 8pt minus 4pt }


\def\tit{\ }

\def\aut{\ }

\def\auf{\ }

\def\leftkol{\ } % ENGLISH ABSTRACTS}

\def\rightkol{ОБ АВТОРАХ} %ENGLISH ABSTRACTS}

\titele{\tit}{\aut}{\auf}{\leftkol}{\rightkol}
      
            \label{st\stat}



\vspace*{24pt}

\begin{multicols}{2}




\noindent
\textbf{Архипов Олег Петрович} (р.\ 1948)~---
кандидат технических наук, директор Орловского филиала Института проб\-лем информатики
Российской академии наук
%302025, г.Орел, Московское шоссе, д.137

\vspace*{3pt}

\noindent
\textbf{Бирюкова Татьяна Константиновна} (р.\ 1968)~---
кандидат фи\-зи\-ко-ма\-те\-ма\-ти\-че\-ских наук, старший научный сотрудник Института проб\-лем информатики
Российской академии наук

\vspace*{3pt}

\noindent 
\textbf{Бобков  Сергей Геннадьевич} (р.\ 1955)~---
доктор технических наук,  заведующий отделением На\-уч\-но-ис\-сле\-до\-ва\-тель\-ско\-го 
института системных исследований Российской академии наук
%117218, Москва, Нахимовский просп., 36, к.1 

\vspace*{3pt}

\noindent \textbf{Васильев Николай Семенович} (р.\ 1952)~--- доктор 
фи\-зи\-ко-ма\-те\-ма\-ти\-че\-ских наук, профессор, 
МГТУ им.\ Н.\,Э.~Баумана 
%, Москва 105005, 2-я Бауманская ул., д.~5,

\vspace*{3pt}

\noindent
\textbf{Гершкович Максим Михайлович} (р.\ 1968)~---
старший научный сотрудник Института проб\-лем информатики
Российской академии наук

\vspace*{3pt}

\noindent 
\textbf{Дьяченко Юрий Георгиевич} (р.\ 1958)~--- кандидат технических наук, 
старший научный сотрудник Института проб\-лем информатики
Российской академии наук

\vspace*{3pt}

\noindent 
\textbf{Ерошенко Александр Андреевич} (р.\ 1989)~--- аспирант кафедры 
математической статистики факультета вычисли\-тельной математики и кибернетики 
Московского государственного университета им.\ М.\,В.~Ломоносова
%119991, Москва ГСП-1, Ленинские горы, д.\ 1, стр. 52

\vspace*{3pt}
 
\noindent 
\textbf{Захаров Виктор Николаевич} (р.\ 1948)~--- 
доктор технических наук, доцент, ученый секретарь Института проб\-лем информатики
Российской академии наук

\vspace*{3pt}

\noindent
\textbf{Зейфман Александр Израилевич} (р.\ 1954)~---
доктор фи\-зи\-ко-ма\-те\-ма\-ти\-че\-ских наук, профессор, 
заведующий кафедрой Вологодского государственного университета; 
старший научный сотрудник Института проб\-лем информатики
Российской академии наук; главный научный сотрудник ИСЭРТ Российской академии наук

\vspace*{3pt}

\noindent
\textbf{Зыкин Сергей Владимирович} (р.\ 1959)~--- 
доктор технических наук, профессор, заведующий лабораторией Института математики 
им.\ С.\,Л.~Соболева Сибирского отделения Российской академии наук, Новосибирск 
%630090, пр.\ ак.\ Коптюга, 4 

\vspace*{4pt}

\noindent
\textbf{Киреев Владимир Иванович} (р.\ 1938)~---
доктор фи\-зи\-ко-ма\-те\-ма\-ти\-че\-ских наук, профессор Московского 
государственного горного университета
%Адрес: Россия, 119991, г. Москва, Ленинский проспект, д. 6

%\columnbreak

\vspace*{4pt}

\noindent
\textbf{Козеренко Елена Борисовна} (р.\ 1959)~---
кандидат филологических наук, заведующая лабораторией Института проб\-лем информатики
Российской академии наук

\vspace*{4pt}

\noindent
\textbf{Королев Виктор Юрьевич} (р.\ 1954)~--- доктор
фи\-зи\-ко-ма\-те\-ма\-ти\-че\-ских наук, профессор кафедры математической 
статистики факультета вычисли\-тельной математики и кибернетики 
Московского государственного университета; 
ведущий научный сотрудник Института проб\-лем информатики
Российской академии наук

\vspace*{4pt}

\noindent
\textbf{Коротышева Анна Владимировна} (р.\ 1988)~---
старший преподаватель Вологодского государственного университета

\vspace*{4pt}

\noindent 
\textbf{Кун Де Турк} (р.\ 1981)~--- научный сотрудник 
исследовательской группы SMACS факультета телекоммуникаций и обработки информации
Университета Гента, Бельгия
%В-9000 Гент, Бельгия

\vspace*{4pt}

\noindent
\textbf{Лупенцов Олег Сергеевич} (р.\ 1986)~---
аспирант Омского государственного института сервиса
%Омск 644043, ул.\ Певцова 13

\vspace*{4pt}

\noindent
\textbf{Лучко Олег Николаевич} (р.\ 1961)~---
кандидат педагогических наук, профессор, заведующий кафедрой 
Омского государственного института сервиса
%Омск 644043, ул.\ Певцова 13

\vspace*{4pt}

\noindent
\textbf{Малашенко Юрий Евгеньевич} (р.\ 1946)~---
доктор фи\-зи\-ко-ма\-те\-ма\-ти\-че\-ских наук, заведующий сектором 
Вычислительного центра им.\ А.\,А.~Дородницына Российской академии наук
%Адрес: 119333, Москва, ул. Вавилова, 40,

\vspace*{4pt}

\noindent
\textbf{Маньяков Юрий Анатольевич} (р.\ 1984)~---
кандидат технических наук, научный сотрудник Орловского филиала Института проб\-лем информатики
Российской академии наук
%302025, г.Орел, Московское шоссе, д.137

\vspace*{4pt}

\noindent
\textbf{Маренко Валентина Афанасьевна} (р.\ 1951)~---
кандидат технических наук, доцент, старший научный сотрудник 
Института математики им.\ С.\,Л.~Соболева Сибирского отделения Российской академии наук
%Новосибирск 630090, пр. ак. Коптюга, 4 

\vspace*{3pt}

\noindent 
\textbf{Морозов Евсей Викторович} (р.\ 1947)~--- доктор 
фи\-зи\-ко-ма\-те\-ма\-ти\-че\-ских, профессор, ведущий научный сотрудник 
Института прикладных математических исследований Карельского научного центра Российской
академии наук; 
%%185910 Россия, Республика Карелия, г.\ Петрозаводск, ул.\ Пушкинская, 11
профессор Петрозаводского государственного университета, Петрозаводск
%185910 Россия, Республика Карелия, г.\ Петрозаводск, пр.\ Ленина, 33

%\pagebreak

\vspace*{3pt}

\noindent
\textbf{Назарова Ирина Александровна} (р.\ 1966)~---
кандидат фи\-зи\-ко-ма\-те\-ма\-ти\-че\-ских наук, 
научный сотрудник Вычислительного центра им.\ А.\,А.~Дородницына Российской академии наук 
%Адрес: 119333, Москва, ул. Вавилова, 40

\vspace*{3pt}

\noindent
\textbf{Павлов Игорь Валерианович} (р.\ 1945)~--- 
доктор фи\-зи\-ко-ма\-те\-ма\-ти\-че\-ских наук, профессор МГТУ им.\ Н.\,Э.~Баумана 
%Москва 105005, 2-я Бауманская ул., д.~5 

%\pagebreak

\vspace*{3pt}

\noindent 
\textbf{Потахина Любовь Викторовна} (р.\ 1989)~--- аспирантка
Института прикладных математических исследований Карельского научного центра
Российской академии наук; 
%%185910 Россия, Республика Карелия, г.\ Петрозаводск, ул.\ Пушкинская, 11
инженер Петрозаводского государственного университета, Петрозаводск
%185910 Россия, Республика Карелия, г.\ Петрозаводск, пр.\ Ленина, 33

\vspace*{3pt}

\noindent 
\textbf{Рождественский Юрий Владимирович} (р.\ 1952)~--- 
кандидат технических наук, заведующий сектором Института проб\-лем информатики
Российской академии наук

\vspace*{3pt}

\noindent 
\textbf{Синицын Игорь Николаевич} (р.\ 1940)~--- доктор технических наук,
профессор, заслуженный деятель\linebreak\vspace*{-12pt}

\columnbreak

\noindent
 науки РФ, заведующий отделом Института проб\-лем информатики
Российской академии наук

\vspace*{7pt}


\noindent
\textbf{Сиротинин Денис Олегович} (р.\ 1984)~---
кандидат технических наук, научный сотрудник Орловского филиала Института проб\-лем информатики
Российской академии наук
%302025, г.Орел, Московское шоссе, д.137

\vspace*{7pt}

%\columnbreak

\noindent 
\textbf{Соколов  Игорь Анатольевич} (р.\ 1954)~--- академик (действительный член) Российской 
академии наук, доктор технических наук, директор Института проб\-лем информатики
Российской академии наук

\vspace*{7pt}

\noindent
\textbf{Степченков Юрий Афанасьевич} (р.\ 1951)~---
кандидат технических наук, заведующий отделом Института проб\-лем информатики
Российской академии наук

\vspace*{7pt}

\noindent
\textbf{Сурков Алексей Викторович} (р.\ 1978)~--- 
старший научный сотрудник На\-уч\-но-ис\-сле\-до\-ва\-тель\-ско\-го 
института системных исследований Российской академии наук
%117218, Москва, Нахимовский просп., 36, к.1 

\vspace*{7pt}

\noindent 
\textbf{Шестаков Олег Владимирович} (р.\ 1976)~--- доктор 
фи\-зи\-ко-ма\-те\-ма\-ти\-че\-ских, доцент кафедры математической статистики 
факультета вычисли\-тельной математики и кибернетики Московского 
государственного университета им.\ М.\,В.~Ломоносова; 
%119991, Москва ГСП-1, Ленинские горы, д.\ 1, стр. 52
старший научный сотрудник Института проб\-лем информатики
Российской академии наук
%, Москва 119333, ул. Вавилова, д.~44, корп.~2

\vspace*{7pt}

\noindent 
\textbf{Шоргин Сергей Яковлевич} (р.\ 1952.)~--- доктор
фи\-зи\-ко-ма\-те\-ма\-ти\-че\-ских наук, профессор, заместитель директора Института 
проб\-лем информатики Российской академии наук





%%%%%%%%%%%%%%%%%%%%%%%%%%%%%%%%%%%%%%%%%%%%%%%%%%%%%%%%%%%%%%%%%%%%%%%%%%%%%%%




%\def\rightkol{ОБ АВТОРАХ}
%\def\leftkol{ОБ АВТОРАХ}

 \label{end\stat}





%\def\leftfootline{\small{\textbf{\thepage}
%\hfill ИНФОРМАТИКА И ЕЁ ПРИМЕНЕНИЯ\ \ \ том~7\ \ \ выпуск~1\ \ \ 2013}
%}%
% \def\rightfootline{\small{ИНФОРМАТИКА И ЕЁ ПРИМЕНЕНИЯ\ \ \ том~7\ \ \ выпуск~1\ \ \ 2013
%\hfill \textbf{\thepage}}}


%\thispagestyle{myheadings}



\end{multicols}

\newpage


%\vspace*{-48pt}
\begin{center}\LARGE
\textit{About Authors}
\end{center}

\thispagestyle{empty}
\def\tit{\ }

\def\aut{\ }

\def\auf{\ }


\def\leftkol{ABOUT AUTHORS}

\def\rightkol{ABOUT AUTHORS}

\vspace*{-18pt}

\titele{\tit}{\aut}{\auf}{\leftkol}{\rightkol}

%\vspace*{36pt}

\def\rightmark{{\noindent\hbox to \textwidth{\hfill\small ABOUT AUTHORS
%\hfill \large\bf\thepage
}}}
\def\leftmark{{\noindent\parbox{\textwidth}{
%\raggedleft\large\bf\thepage \hfill
\small\textrm{ABOUT AUTHORS}\hfill}}}


\def\leftfootline{\small{\textbf{\thepage}
\hfill ИНФОРМАТИКА И ЕЁ ПРИМЕНЕНИЯ\ \ \ том~6\ \ \ выпуск~2\ \ \ 2012}
}%
 \def\rightfootline{\small{ИНФОРМАТИКА И ЕЁ ПРИМЕНЕНИЯ\ \ \ том~6\ \ \ выпуск~2\ \ \ 2012
\hfill \textbf{\thepage}}}


\begin{multicols}{2}

\noindent
\textbf{Agalarov Yaver M.} (b.\ 1952)~--- Candidate of Science (PhD)
in technology, 
leading scientist, Institute of Informatics Problems, Russian Academy of Sciences

\vspace*{5pt}


  \noindent
\textbf{Bosov Alexey V.} (b.\ 1969)~--- Doctor of Science in technology, Head of
Laboratory, Institute of Informatics Problems, Russian Academy of Sciences

\vspace*{5pt}


\noindent
\textbf{Dulin Sergey K.} (b.\ 1950)~--- Doctor of Science in technology, 
professor, senior scientist, Institute of Informatics Problems, Russian Academy of Sciences

\vspace*{5pt}

\noindent
\textbf{Gorshenin Andrey K.}~--- (b.\ 1986)~--- Candidate of Science (PhD)
in physics and mathematics,
senior scientist, Institute of Informatics Problems, Russian Academy of Sciences

\vspace*{5pt}

\noindent
\textbf{Kalenov Nikolay E.}  (b.\ 1945)~--- Doctor of Science in technology,
professor, Director, Library for Natural Sciences,  Russian Academy of Sciences 

\vspace*{5pt}

\noindent
\textbf{Kalinichenko Leonid A.} (b.\ 1937)~--- Doctor of Science in physics and mathematics, 
professor, Honored scientist of RF, 
Head of Laboratory, Institute of Informatics Problems, Russian Academy of Sciences 

\vspace*{5pt}

\noindent
\textbf{Karpov Alexey A.} (b.\ 1978)~--- Candidate of Science (PhD) in technology, 
senior scientist, St.\ Petersburg Institute for
Informatics and Automation,  Russian Academy of Sciences

\vspace*{5pt}

\noindent
\textbf{Kuznetsov Igor P.} (b.\ 1938)~--- Doctor of Science in technology, 
professor, principal scientist, Institute of Informatics Problems, Russian Academy of Sciences

\vspace*{5pt}


\noindent
\textbf{Markova Natalia A.} (b.\ 1950)~--- Candidate of Science (PhD) in
physics and mathematics, leading scientist,  
Institute of Informatics Problems, Russian Academy of Sciences

\vspace*{5pt}

\noindent
\textbf{Nikolaev Andrey V.} (b.\ 1985)~--- Candidate of Science (PhD) in technology, 
senior lecturer, Tchaikovsky Technological Institute, Branch of the Izhevsk State Technical 
University

\vspace*{6pt}

\noindent
\textbf{Pavlov Igor V.} (b.\ 1945)~---  Doctor of Science in physics and mathematics,
professor, Bauman Moscow State Technical University

\vspace*{6pt}

%\columnbreak

\noindent
\textbf{Rozenberg Igor N.} (b.\ 1965)~--- Doctor of Science in technology, 
First Deputy Director General, Research \& Design Institute for Information 
Technology, Signalling and Telecommunications on Railway Transport (JSC NIIAS)

\vspace*{6pt}


\noindent
\textbf{Semenov Konstantin K.} (b.\ 1986)~--- MPhil, 
associate professor, St.\ Petersburg State Polytechnical University

\vspace*{6pt}

\noindent
\textbf{Sharnin Mikhail M.} (b.\ 1959)~--- Candidate of Science (PhD) 
in technology, senior scientist, Institute of Informatics Problems, Russian Academy of Sciences

\vspace*{6pt}

\noindent 
\textbf{Shestakov Oleg V.} (b.\ 1976)~--- Candidate of Science (PhD) in physics and mathematics,
associate professor, Department of Mathematical Statistics, Faculty of Computational Mathematics and Cybernetics,
M.\,V.~Lomonosov Moscow State University; senior scientist, Institute of Informatics Problems, 
Russian Academy of Sciences

\vspace*{6pt}

\noindent
\textbf{Stupnikov Sergey A.} (b.\ 1978)~--- Candidate of Science (PhD) in technology, 
senior scientist, Institute of Informatics Problems, Russian Academy of Sciences 

\vspace*{6pt}

\noindent
\textbf{Umansky Vladimir I.} (b.\ 1954)~--- Candidate of Science (PhD) in technology, 
Director General, ``IntechGeoTrans'' Closed Joint Stock Company

\vspace*{6pt}

\noindent
\textbf{Zhevnerchuk Dmitry V.} (b.\ 1978)~--- Candidate of Science (PhD) in technology, 
associate professor, Tchaikovsky Technological Institute, Branch of the Izhevsk State 
Technical University

%\vspace*{6pt}

\def\leftfootline{\small{\textbf{\thepage}
\hfill ИНФОРМАТИКА И ЕЁ ПРИМЕНЕНИЯ\ \ \ том~6\ \ \ выпуск~2\ \ \ 2012}
}%
 \def\rightfootline{\small{ИНФОРМАТИКА И ЕЁ ПРИМЕНЕНИЯ\ \ \ том~6\ \ \ выпуск~2\ \ \ 2012
\hfill \textbf{\thepage}}}



%\thispagestyle{myheadings}

\end{multicols}
\newpage

%   \vspace*{-48pt}

\begin{center}
\vspace*{6pt}
\mbox{%
\epsfxsize=53.502mm
\epsfbox{foto-1.eps}
}
\end{center}

\vspace*{6pt} %Академик


   \begin{center}
\fbox{\Large\textbf{Профессор Игорь Алексеевич Ушаков}}\\[12pt]
\textbf{\large 22.01.1935--27.02.2015}
   \end{center}


   %\vspace*{2.5mm}

   \vspace*{5mm}

   \thispagestyle{empty}

%\

%\vspace*{-12pt}


Редакционный совет и редакционная коллегия журнала <<Информатика и~её применения>> с~глубоким прискорбием извещают, что 27~февраля 2015~г.\ после тяжелой
и~продолжительной болезни скончался Игорь Алексеевич Ушаков~--- доктор технических наук, профессор, член редколлегии журнала <<Информатика и ее применения>>.

Игорь Алексеевич Ушаков окончил Московский авиационный институт, в~1963~г.\ защитил кандидатскую, а~в~1968~г.~--- докторскую диссертацию. С~1958 по 1989~гг.\ работал в~ряде научно-исследовательских организаций СССР, в~том числе руководил отделами в~НИИ АА и~ВЦ АН СССР; с 1969 по 1989 гг. преподавал в~МФТИ (был профессором, а~затем заведующим кафедрой) и~в~МЭИ. С~1989~г.~---- в~США: являлся профессором университета Дж.\ Вашингтона, университета Дж.\ Мэйсона и~Калифорнийского университета, сотрудником компаний MCI, Qualcomm и Hughes.

И.\,А.~Ушаков с момента основания журнала <<Надежность и~контроль качества>> был заместителем ответственного редактора, а~затем на протяжении многих лет членом редколлегии. В~2006~г.\ основал электронный международный журнал ``Reliability: Theory \& Application'', главным редактором которого оставался до конца жизни.

Учебниками и справочниками по теории надежности, написанными И.\,А.~Ушаковым, пользовались и~пользуются несколько поколений ученых и~специалистов в~разных странах мира.

Игорь Алексеевич всегда уделял огромное внимание работе с~молодежью; более~50 его учеников защитили докторские и~кандидатские диссертации.

И.\,А.~Ушаков вел активную научно-про\-све\-ти\-тель\-скую деятельность. В~частности, он был одним из организаторов и~руководителей Московского кабинета качества и~надежности при Политехническом музее (целью этого Кабинета было оказание консультаций работникам промышленных предприятий и~чтение курсов лекций для инженеров, занимающихся проблемой надежности). Находясь в~США, И.\,А.~Ушаков создал международный ин\-тер\-нет-фо\-рум им.\ Б.\,В.~Гнеденко, объединивший около~400~видных специалистов по приложениям теории вероятностей и~математической статистики, преимущественно в~об\-ласти теории надежности и~анализа риска, из десятков стран мира; коллективным членов этого Форума является и~наш журнал. Цели Форума~--- содействие контактам между специалистами из разных стран, организация обмена профессиональными 
новостями и~информацией (новые публикации, предстоящие события и~др.). Также необходимо отметить большое число на\-уч\-но-по\-пу\-ляр\-ных работ, опубликованных И.\,А.~Ушаковым.

И.\,А.~Ушаков обладал большим личным обаянием, имел широкий круг интересов. Все знавшие И.\,А.~Ушакова всегда будут помнить его как замечательного ученого и~прекрасного человека.

\bigskip

Редакционный совет и редакционная коллегия журнала <<Информатика и~её применения>> 
выражают глубокие соболезнования родным и близким покойного, всем, кто его знал и~работал с~ним.


\vspace*{-60pt} {\small
{\baselineskip=9.1pt
\section*{Правила подготовки рукописей статей для публикации в журнале
<<Информатика и её применения>>}

\thispagestyle{empty}

 Журнал <<Информатика и её применения>> публикует
теоретические, обзорные и дискуссионные статьи, посвященные научным
исследованиям и разработкам в области информатики и ее приложений. Журнал
издается на русском языке. По специальному решению редколлегии отдельные статьи,
в виде исключения, могут печататься на английском языке.
Тематика журнала охватывает следующие направления:
\begin{itemize}
\item теоретические основы информатики; %\\[-13.5pt]
\item математические методы исследования сложных систем и процессов; %\\[-13.5pt]
\item информационные системы и сети; %\\[-13.5pt]
\item информационные технологии; %\\[-13.5pt]
\item архитектура и программное
обеспечение вычислительных комплексов и сетей.
\end{itemize}
\begin{enumerate}
\item В журнале печатаются результаты, ранее не
опубликованные и не предназначенные к одновременной публикации в других
изданиях. Публикация не должна нарушать закон об авторских правах. Направляя
свою рукопись в редакцию, авторы автоматически передают учредителям и
редколлегии неисключительные права на издание данной статьи на русском языке и
на ее распространение в России и за рубежом. При этом за авторами сохраняются
все права как собственников данной рукописи. В связи с этим авторами должно
быть представлено в редакцию письмо в следующей форме:
Соглашение о передаче права на публикацию:

\textit{<<Мы, нижеподписавшиеся, авторы рукописи <<$\qquad\qquad$>>, передаем
учредителям и редколлегии журнала <<Информатика и её применения>>
неисключительное право опубликовать данную рукопись статьи на русском языке как
в печатной, так и в электронной версиях журнала. Мы подтверждаем, что данная
публикация не нарушает авторского права других лиц или организаций. Подписи
авторов: (ф.\,и.\,о., дата, адрес)>>.}

Указанное соглашение может быть представлено 
как в бумажном виде, так и в виде отсканированной копии (с подписями авторов).


Редколлегия вправе запросить у авторов экспертное заключение о возможности
опубликования представленной статьи в открытой печати. %\\[-13.5pt]
\item Статья
подписывается всеми авторами. На отдельном листе представляются данные автора
(или всех авторов): фамилия, полные имя и отчество, телефон, факс, e-mail,
почтовый адрес. Если работа выполнена несколькими авторами, указывается фамилия
одного из них, ответственного за переписку с редакцией. %\\[-13.5pt]
\item Редакция журнала
осуществляет самостоятельную экспертизу присланных статей. Возвращение рукописи
на доработку не означает, что статья уже принята к печати. Доработанный вариант
с ответом на замечания рецензента необходимо прислать в редакцию. %\\[-13.5pt]
\item Решение
редакционной коллегии о принятии статьи к печати или ее отклонении сообщается
авторам. Редколлегия не обязуется направлять рецензию авторам отклоненной
статьи. %\\[-13.5pt]
\item Корректура статей высылается авторам для просмотра. Редакция
просит авторов присылать свои замечания в кратчайшие сроки. %\\[-13.5pt]
\item При
подготовке рукописи в MS Word рекомендуется использовать следующие настройки.
Параметры страницы: формат~--- А4; ориентация~--- книжная; поля (см): внутри~---
2,5, снаружи~--- 1,5, сверху~--- 2, снизу~--- 2, от края до нижнего
колонтитула~--- 1,3. Основной текст: стиль~--- <<Обычный>>: шрифт Times New
Roman, размер 14~пунктов, абзацный отступ~--- 0,5~см, 1,5 интервала,
выравнивание~--- по ширине. Рекомендуемый объем рукописи~--- не свыше
25~страниц указанного формата. Ознакомиться с шаблонами, содержащими примеры
оформления, можно по адресу в Интернете:
\textsf{http://www.ipiran.ru/journal/template.doc}.
\item К рукописи, предоставляемой в 2-х
экземплярах, обязательно прилагается электронная версия статьи (как правило, в
форматах MS WORD (.doc) или \LaTeX\ (.tex), а также~--- дополнительно~--- в
формате .pdf) на дискете, лазерном диске или по электронной почте. Сокращения
слов, кроме стандартных, не применяются. Все страницы рукописи должны быть
пронумерованы. %\\[-13.5pt]
\item Статья должна содержать следующую информацию на русском и
английском языках: название, Ф.И.О. авторов, места работы авторов и их
электронные адреса, подробные сведения об авторах, оформленные в соответствии с форматом, 
определяемым файлами {\sf http://www.ipiran.ru/journal/issues/2011\_05\_01/authors.asp} и 
{\sf http://www.ipiran.ru/journal/issues/2011\_01\_eng/authors.asp},
аннотация (не более 100~слов), ключевые слова. Ссылки на
литературу в тексте статьи нумеруются (в квадратных скобках) и располагаются в
порядке их первого упоминания. В~списке литературы не должно быть позиций, на которые нет ссылки в тексте статьи.
Все фамилии авторов, заглавия статей, названия
книг, конференций и~т.\,п.\ даются на языке оригинала, если этот язык
использует кириллический или латинский алфавит. %\\[-13.5pt]
\item Присланные в редакцию материалы авторам не возвращаются.
\item При отправке файлов по электронной
почте просим придерживаться следующих правил:
\begin{itemize}
\item указывать в поле subject (тема) название журнала и фамилию автора; %\\[-13.5pt]
\item использовать attach (присоединение); %\\[-13.5pt]
\item в случае больших объемов информации возможно
использование общеизвестных архиваторов (ZIP, RAR); %\\[-13.5pt]
\item в состав электронной версии статьи должны входить: файл, содержащий текст статьи, и файл(ы),
содержащий(е) иллюстрации. %\\[-13.5pt]
\end{itemize}
\item Журнал <<Информатика и её применения>> является некоммерческим изданием. 
Плата за публикацию с авторов не взимается, гонорар авторам не выплачивается.
\end{enumerate}
\thispagestyle{empty}
\textbf{Адрес редакции:} Москва 119333,
ул.~Вавилова, д.~44, корп.~2, ИПИ РАН\\
\hphantom{\textbf{Адрес редакции:} }Тел.: +7 (499) 135-86-92\ \
Факс:  +7 (495) 930-45-05\ \  E-mail:   rust@ipiran.ru }
}


\end{document}

%\include{IPPM-25}

\def\stat{cont}
{%\hrule\par
%\vskip 7pt % 7pt
\raggedleft\Large \bf%\baselineskip=3.2ex
А\,В\,Т\,О\,Р\,С\,К\,И\,Й\ \ У\,К\,А\,З\,А\,Т\,Е\,Л\,Ь\ \ З\,А\ \ 2\,0\,1\,0 г. \vskip 17pt
    \hrule
    \par
\vskip 21pt plus 6pt minus 3pt }

\label{st\stat}

\def\tit{\ }

\def\aut{\ }
\def\auf{\ }

\def\leftkol{\ } % ENGLISH ABSTRACTS}

\def\rightkol{\ } %АВТОРСКИЙ УКАЗАТЕЛЬ ЗА 2010 г.} %ENGLISH ABSTRACTS}

\titele{\tit}{\aut}{\auf}{\leftkol}{\rightkol}

\vspace*{-12pt}

{\tabcolsep=3pt
\begin{tabular}{p{388pt}rr}
&\textbf{Выпуск} & \textbf{Стр.}\\[6pt]
\hangindent=23pt\noindent\textbf{Арутюнян~А.\,Р.} Моделирование влияния деформаций отпечатков пальцев на 
точность\linebreak
\vspace*{-12pt}\\
\hspace*{23pt}дактилоскопической идентификации$\dotfill$&1&51\\
\hangindent=23pt\noindent\textbf{Архипов~О.\,П., Зыкова~З.\,П.} Интеграция гетерогенной информации о цветных 
пикселях\linebreak
\vspace*{-12pt}\\
\hspace*{23pt}и их цветовосприятии$\dotfill$&4&15\\
\hangindent=23pt\noindent\textbf{Баранов~С.\,И., Френкель~С.\,Л., Захаров~В.\,Н.} Полуформальная верификация 
цифрового устройства с конвейером, основанная на использовании алгоритмических машин\linebreak
\vspace*{-12pt}\\
\hspace*{23pt}состояния$\dotfill$&4&49\\
\textbf{Бекетова~И.\,В.} см.~Каратеев~С.\,Л.&&\\
\textbf{Белоусов~В.\,В.} см.~Синицын~И.\,Н.&&\\
\hangindent=23pt\noindent\textbf{Бенинг~В.\,Е., Королев~Р.\,А.} О предельном поведении мощностей критериев в 
случае\linebreak
\vspace*{-12pt}\\
\hspace*{23pt}распределения Лапласа$\dotfill$&2&63\\
\hangindent=23pt\noindent\textbf{Бенинг~В.\,Е., Сипина~А.\,В.} Асимптотическое разложение для мощности 
критерия,\linebreak
\vspace*{-12pt}\\
\hspace*{23pt}основанного на выборочной медиане, в случае распределения Лапласа$\dotfill$&1&18\\
\textbf{Бондаренко~А.\,В.} см.~Каратеев~С.\,Л.&&\\
\hangindent=23pt\noindent\textbf{Бородина~А.\,В., Морозов~Е.\,В.} Об оценивании асимптотики вероятности 
большого\linebreak
\vspace*{-12pt}\\
\hspace*{23pt}уклонения стационарной регенеративной очереди с одним прибором$\dotfill$&3&29\\
\hangindent=23pt\noindent\textbf{Бунтман~Н.\,В., Минель~Ж.-Л., Ле~Пезан~Д., Зацман~И.\,М.} Типология и 
компьютерное\linebreak
\vspace*{-12pt}\\
\hspace*{23pt}моделирование трудностей перевода$\dotfill$&3&77\\
\textbf{Визильтер~Ю.\,В.} см.~Каратеев~С.\,Л.&&\\
\hangindent=23pt\noindent\textbf{Гавриленко~С.\,В.} Оценки скорости сходимости распределений случайных сумм с 
безгранично делимыми индексами к нормальному закону$\dotfill$&4&81\\
\hangindent=23pt\noindent\textbf{Григорьева~М.\,Е., Шевцова~И.\,Г.} Уточнение неравенства 
Каца--Берри--Эссеена$\dotfill$&2&75\\
\hangindent=23pt\noindent\textbf{Грушо~А.\,А., Грушо~Н.\,А., Тимонина~Е.\,Е.} Поиск конфликтов в политиках 
безопасности: модель случайных графов$\dotfill$&3&38\\
\textbf{Грушо~Н.\,А.} см.~Грушо~А.\,А.&&\\
\hangindent=23pt\noindent\textbf{Гудков~В.\,Ю.} Математические модели изображения отпечатка пальца на основе 
описания линий$\dotfill$&1&58\\
\textbf{Гуртов~А.\,В.} см.~Лукьяненко~А.\,С.&&\\
\textbf{Желтов~С.\,Ю.} см.~Каратеев~С.\,Л.&&\\
\hangindent=23pt\noindent\textbf{Захаров~А.\,А., Серебряков~В.\,А.} Система управления электронной библиотекой 
LibMeta$\dotfill$&4&2\\
\textbf{Захаров~В.\,Н.} см.~Баранов~С.\,И.&&\\
\textbf{Захарова~Т.\,В.} см.~Матвеева~С.\,С.&&\\
\hangindent=23pt\noindent\textbf{Зацаринный~А.\,А., Чупраков~К.\,Г.} Некоторые аспекты выбора технологии для 
постро-\linebreak
\vspace*{-12pt}\\
\hspace*{23pt}ения систем отображения информации ситуационного центра$\dotfill$&3&59\\
\textbf{Зацман~И.\,М.} см.~Бунтман~Н.\,В.&&\\
\hangindent=23pt\noindent\textbf{Зейфман~А.\,И., Коротышева~А.\,В., Сатин~Я.\,А., Шоргин~С.\,Я.} Об 
устойчивости нестаци-\linebreak
\vspace*{-12pt}\\
\hspace*{23pt}онарных систем обслуживания с катастрофами$\dotfill$&3&9\\
\textbf{Зыкова~З.\,П.} см.~Архипов~О.\,П.&&\\
\hangindent=23pt\noindent\textbf{Илюшин~Г.\,Я., Соколов~И.\,А.} Организация управляемого доступа пользователей 
к\linebreak
\vspace*{-12pt}\\
\hspace*{23pt}разнородным ведомственным информационным ресурсам$\dotfill$&1&24\\
\hangindent=23pt\noindent\textbf{Кавагучи~Ю., Ульянов~В.\,В., Фуджикоши~Я.} Приближения для статистик, 
описывающих\linebreak
\vspace*{-12pt}\\
\hspace*{23pt}геометрические свойства данных большой размерности, с оценками 
ошибок$\dotfill$&1&12\\
\hangindent=23pt\noindent\textbf{Каратеев~С.\,Л., Бекетова~И.\,В., Ососков~М.\,В., Князь~В.\,А., 
Визильтер~Ю.\,В., Бондаренко~А.\,В., Желтов~С.\,Ю.} Автоматизированный контроль 
качества цифровых\linebreak
\vspace*{-12pt}\\
\hspace*{23pt}изображений для персональных документов$\dotfill$&1&65\\
\end{tabular}
}

\pagebreak

\def\leftkol{АВТОРСКИЙ УКАЗАТЕЛЬ ЗА 2010 г.} % ENGLISH ABSTRACTS}

\def\rightkol{АВТОРСКИЙ УКАЗАТЕЛЬ ЗА 2010 г.} %ENGLISH ABSTRACTS}

{\tabcolsep=3pt
\begin{tabular}{p{388pt}rr}
&\textbf{Выпуск} & \textbf{Стр.}\\[3pt]
\hangindent=23pt\noindent\textbf{Козеренко~Е.\,Б.} Лингвистические фильтры в статистических моделях машинного\linebreak
\vspace*{-12pt}\\
\hspace*{23pt}перевода$\dotfill$&2&83\\
\hangindent=23pt\noindent\textbf{Козеренко~Е.\,Б., Кузнецов~И.\,П.} Когнитивно-лингвистические представления в 
систе-\linebreak
\vspace*{-12pt}\\
\hspace*{23pt}мах обработки текстов$\dotfill$&3&69\\
\textbf{Князь~В.\,А.} см.~Каратеев~С.\,Л.&&\\
\hangindent=23pt\noindent\textbf{Колесников~А.\,В., Солдатов~С.\,А.} Алгоритм координации для гибридной 
интеллектуальной системы решения сложной задачи оперативно-производственного\linebreak
\vspace*{-12pt}\\
\hspace*{23pt}планирования$\dotfill$&4&61\\
\hangindent=23pt\noindent\textbf{Коновалов~М.\,Г.} О планировании потоков в системах вычислительных 
ресурсов$\dotfill$&2&3\\
\textbf{Конушин~А.\,С.} см.~Конушин~В.\,С.&&\\
\hangindent=23pt\noindent\textbf{Конушин~В.\,С., Кривовязь~Г.\,Р., Конушин~А.\,С.} Алгоритм распознавания людей 
в видео-\linebreak
\vspace*{-12pt}\\
\hspace*{23pt}последовательности по одежде$\dotfill$&1&74\\
\textbf{Корепанов~Э.\, Р.} см.~Синицын~И.\,Н.&&\\
\textbf{Королев~В.\,Ю.} см.~Соколов~И.\,А.&&\\
\textbf{Королев~Р.\,А.} см.~Бенинг~В.\,Е.&&\\
\textbf{Коротышева~А.\,В.} см.~Зейфман~А.\,И.&&\\
\hangindent=23pt\noindent\textbf{Кривенко~М.\,П.} Непараметрическое оценивание элементов байесовского 
клас\-си-\linebreak
\vspace*{-12pt}\\
\hspace*{23pt}фикатора$\dotfill$&2&13\\
\textbf{Кривовязь~Г.\,Р.} см.~Конушин~В.\,С.&&\\
\textbf{Крылов~А.\,С.} см.~Павельева~Е.\,А.&&\\
\hangindent=23pt\noindent\textbf{Крылов~В.\,А.} Моделирование и классификация многоканальных дистанционных\linebreak
\vspace*{-12pt}\\
\hspace*{23pt}изображений с использованием копул$\dotfill$&4&34\\
\hangindent=23pt\noindent\textbf{Крючин~О.\,В.} Разработка параллельных эвристических алгоритмов подбора 
весовых\linebreak
\vspace*{-12pt}\\
\hspace*{23pt}коэффициентов искусственной нейтронной сети$\dotfill$&2&53\\
\hangindent=23pt\noindent\textbf{Кудрявцев~А.\,А., Шоргин~С.\,Я.} Байесовские модели массового обслуживания и 
надеж-\linebreak
\vspace*{-12pt}\\
\hspace*{23pt}ности: характеристики среднего числа заявок в системе $M\vert M \vert 1\vert 
\infty$$\dotfill$&3&16\\
\hangindent=23pt\noindent\textbf{Кузнецов~А.\,А.} Связь между временными и структурно-топологическими 
характери-\linebreak
\vspace*{-12pt}\\
\hspace*{23pt}стиками диаграмм ритма сердца здоровых людей$\dotfill$&4&39\\
\textbf{Кузнецов~И.\,П.} см.~Козеренко~Е.\,Б.&&\\
\textbf{Ле~Пезан~Д.} см.~Бунтман~Н.\,В.&&\\
\hangindent=23pt\noindent\textbf{Лукьяненко~А.\,С., Морозов~Е.\,В., Гуртов~А.\,В.} Анализ сетевого протокола с общей 
функ-\linebreak
\vspace*{-12pt}\\
\hspace*{23pt}цией расширения окна передачи сообщения при конфликтах$\dotfill$&2&46\\
\hangindent=23pt\noindent\textbf{Лямин~О.\,О.} О предельном поведении мощностей критериев в случае обобщенного\linebreak
\vspace*{-12pt}\\
\hspace*{23pt}распределения Лапласа$\dotfill$&3&47\\
\hangindent=23pt\noindent\textbf{Маркин~А.\,В., Шестаков~О.\,В.} Асимптотики оценки риска при пороговой 
обработке\linebreak
\vspace*{-12pt}\\
\hspace*{23pt}вейвлет-вейглет коэффициентов в задаче томографии$\dotfill$&2&36\\
\hangindent=23pt\noindent\textbf{Матвеева~С.\,С., Захарова~Т.\,В.} Сети массового обслуживания с наименьшей 
длиной\linebreak
\vspace*{-12pt}\\
\hspace*{23pt}очереди$\dotfill$&3&22\\
\hangindent=23pt\noindent\textbf{Матюшенко~С.\,И.} Стационарные характеристики двухканальной системы 
обслужива-\linebreak
\vspace*{-12pt}\\
\hspace*{23pt}ния с переупорядочиванием заявок и распределениями фазового типа$\dotfill$&4&68\\
\textbf{Минель~Ж.-Л.} см.~Бунтман~Н.\,В.&&\\
\textbf{Морозов~Е.\,В.} см.~Бородина~А.\,В.&&\\
\textbf{Морозов~Е.\,В.} см.~Лукьяненко~А.\,С.&&\\
\textbf{Ососков~М.\,В.} см.~Каратеев~С.\,Л.&&\\
\hangindent=23pt\noindent\textbf{Павельева~Е.\,А., Крылов~А.\,С.} Поиск и анализ ключевых точек радужной 
оболочки\linebreak
\vspace*{-12pt}\\
\hspace*{23pt}глаза методом преобразования Эрмита$\dotfill$&1&79\\
\textbf{Печинкин~А.\,В.} см.~Френкель~С.\,Л.,&&\\
\hangindent=23pt\noindent\textbf{Протасов~В.\,И.} Составление субъективного портрета с использованием 
эволюционно-\linebreak
\vspace*{-12pt}\\
\hspace*{23pt}го морфинга и квалиметрия метода$\dotfill$&1&83\\
\hangindent=23pt\noindent\textbf{Рудаков~К.\,В., Торшин~И.\,Ю.} Вопросы разрешимости задачи распознавания 
вторичной\linebreak
\vspace*{-12pt}\\
\hspace*{23pt}структуры белка$\dotfill$&2&25\\
\textbf{Сатин~Я.\,А.} см.~Зейфман~А.\,И.&&\\
\hangindent=23pt\noindent\textbf{Сейфуль-Мулюков~Р.\,Б.} Нефть как носитель информации о своем 
происхождении,\linebreak
\vspace*{-12pt}\\
\hspace*{23pt}структуре и эволюции$\dotfill$&1&41\\
\end{tabular}
}

{\tabcolsep=3pt
\begin{tabular}{p{388pt}rr}
&\textbf{Выпуск} & \textbf{Стр.}\\[6pt]
\textbf{Семендяев~Н.\,Н.} см.~Синицын~И.\,Н.&&\\
\textbf{Серебряков~В.\,А.} см.~Захаров~А.\,А.&&\\
\textbf{Синицын~В.\,И.} см.~Синицын~И.\,Н.&&\\
\hangindent=23pt\noindent\textbf{Синицын~И.\,Н., Синицын~В.\,И., Корепанов~Э.\, Р., Белоусов~В.\,В., 
Семендяев~Н.\,Н.} Оперативное построение информационных моделей движения полюса 
Земли\linebreak
\vspace*{-12pt}\\
\hspace*{23pt}методами линейных и линеаризованных фильтров$\dotfill$&1&2\\
\textbf{Сипина~А.\,В.} см.~Бенинг~В.\,Е.&&\\
\hangindent=23pt\noindent\textbf{Соколов~И.\,А.} О работах заслуженного деятеля науки Российской Федерации 
И.\,Н.~Синицына в области информационных технологий и автоматизации (к 70-летию\linebreak
\vspace*{-12pt}\\
\hspace*{23pt}со дня рождения)$\dotfill$&3&84\\
\textbf{Соколов~И.\,А.} см.~Илюшин~Г.\,Я.&&\\
\hangindent=23pt\noindent\textbf{Соколов~И.\,А., Королев~В.\,Ю.} Предисловие$\dotfill$&2&2\\
\textbf{Солдатов~С.\,А.} см.~Колесников~А.\,В.&&\\
\hangindent=23pt\noindent\textbf{Степанов~С.\,Ю.} Использование координатного метода фрагментации 
коммутаторной\linebreak
\vspace*{-12pt}\\
\hspace*{23pt}нейронной сети для сокращения трафика$\dotfill$&2&57\\
\textbf{Тимонина~Е.\,Е.} см.~Грушо~А.\,А.&&\\
\textbf{Торшин~И.\,Ю.} см.~Рудаков~К.\,В.&&\\
\textbf{Ульянов~В.\,В.} см.~Кавагучи~Ю.&&\\
\textbf{Фазекаш~И.} см.~Чупрунов~А.\,Н.&&\\
\textbf{Френкель~С.\,Л.} см.~Баранов~С.\,И.&&\\
\hangindent=23pt\noindent\textbf{Френкель~С.\,Л., Печинкин~А.\,В.} Оценка времени самовосстановления в 
цифровых\linebreak
\vspace*{-12pt}\\
\hspace*{23pt}системах после сбоев, вызываемых переходными помехами$\dotfill$&3&2\\
\textbf{Фуджикоши~Я.} см.~Кавагучи~Ю.&&\\
\hangindent=23pt\noindent\textbf{Цискаридзе~А.\,К.} Математическая модель и метод восстановления позы человека 
по\linebreak
\vspace*{-12pt}\\
\hspace*{23pt}стереопаре силуэтных изображений$\dotfill$&4&27\\
\hangindent=23pt\noindent\textbf{Чупраков~К.\,Г.} К вопросу о размещении коллективных средств отображения в 
ситуа-\linebreak
\vspace*{-12pt}\\
\hspace*{23pt}ционном зале с заданными параметрами$\dotfill$&4&89\\
\textbf{Чупраков~К.\,Г.} см.~Зацаринный~А.\,А.&&\\
\hangindent=23pt\noindent\textbf{Чупрунов~А.\,Н., Фазекаш~И.} Законы повторного логарифма для числа 
безошибочных\linebreak
\vspace*{-12pt}\\
\hspace*{23pt}блоков при помехоустойчивом кодировании$\dotfill$&3&42\\
\textbf{Шевцова~И.\,Г.} см.~Григорьева~М.\,Е.&&\\
\hangindent=23pt\noindent\textbf{Шестаков~О.\,В.} Аппроксимация распределения оценки риска пороговой 
обработки вейвлет-коэффициентов нормальным распределением при использовании 
выбо-\linebreak
\vspace*{-12pt}\\
\hspace*{23pt}рочной дисперсии$\dotfill$&4&73\\
\textbf{Шестаков~О.\,В.} см.~Маркин~А.\,В.&&\\
\textbf{Шоргин~С.\,Я.} см.~Зейфман~А.\,И.&&\\
\textbf{Шоргин~С.\,Я.} см.~Кудрявцев~А.\,А.&&\\
\end{tabular}
}

%\thispagestyle{myheadings}
\def\leftfootline{\small{\textbf{\thepage}
\hfill ИНФОРМАТИКА И ЕЁ ПРИМЕНЕНИЯ\ \ \ том~4\ \ \ выпуск~4\ \ \ 2010}
}%
 \def\rightfootline{\small{ИНФОРМАТИКА И ЕЁ ПРИМЕНЕНИЯ\ \ \ том~4\ \ \ выпуск~4\ \ \ 2010
 \hfill \textbf{\thepage}}}
 \label{end\stat}


%Том 10 Выпуск 1-4 Год 2016

\def\stat{cont-e}
{%\hrule\par
%\vskip 7pt % 7pt
\raggedleft\Large \bf%\baselineskip=3.2ex
2\,0\,1\,6\ \ A\,U\,T\,H\,O\,R\ \ I\,N\,D\,E\,X \vskip 17pt
 \hrule
 \par
\vskip 21pt plus 6pt minus 3pt }

\label{st\stat}

\def\tit{\ }

\def\aut{\ }
\def\auf{\ }

\def\leftkol{\ } %2016 AUTHOR INDEX} % ENGLISH ABSTRACTS}

\def\rightkol{\ } %2016 AUTHOR INDEX} %ENGLISH ABSTRACTS}

\titele{\tit}{\aut}{\auf}{\leftkol}{\rightkol}

\def\leftfootline{\small{\textbf{\thepage}
\hfill INFORMATIKA I EE PRIMENENIYA~--- INFORMATICS AND APPLICATIONS\ \ \ 2016\
\ \ volume~10\ \ \ issue\ 4}
}%
 \def\rightfootline{\small{INFORMATIKA I EE PRIMENENIYA~--- INFORMATICS AND APPLICATIONS\ \ \ 2016\ \ \ volume~10\ \ \ issue\ 4
\hfill \textbf{\thepage}}}

\vspace*{-12pt}
\vspace*{-18pt}

{\tabcolsep=2.8pt
\begin{tabular}{p{382pt}cc}
&\textbf{Issue} & \textbf{Page}\\[6pt]
\Avtors{Agalarov~M.\,Ya.} see~Agalarov~Ya.\,M.&&\\
\Avtors{Agalarov~Ya.\,M., Agalarov~M.\,Ya., and
Shorgin~V.\,S.} About the optimal threshold of queue\linebreak
\\[-12pt]
\hspace*{23pt}length in a~particular problem of profit maximization
in the $M/G/1$ queuing system&2&70--79\\
\Avtors{Alexeyevsky~D.\,A.} BioNLP ontology extraction from 
a~restricted language corpus with\linebreak
\\[-12pt]
\hspace*{23pt}context-free grammars&1&119--128\\
\Avtors{Andreev~S.\,D.} see~Gaidamaka~Yu.\,V.&&\\
\Avtors{Andreev~S.\,D.} see~Ometov~A.\,Ya.&&\\
\Avtors{Arkhipov~O.\,P., Arkhipov~P.\,O., and Sidorkin~I.\,I.} The
option to create a~local coordinate\linebreak
\\[-12pt]
\hspace*{23pt}system for synchronization of selected images&3&91--97\\
\Avtors{Arkhipov~P.\,O.} see~Arkhipov~O.\,P.&&\\
\Avtors{Belousov~V.\,V.} see~Shnurkov~P.\,V.&&\\
\Avtors{Belousov~V.\,V.} see~Shnurkov~P.\,V.&&\\
\Avtors{Bening~V.\,E.} Calculation of~the~asymptotic deficiency
of~some statistical procedures based\linebreak
\\[-12pt]
\hspace*{23pt}on~samples with~random sizes&4&34--45\\
\Avtors{Borisov~A.\,V., Bosov~A.\,V., and Miller~G.\,B.} Modeling and
monitoring of VoIP connection&2&\hphantom{1}2--13\\
\Avtors{Bosov~A.\,V.} see~Borisov~A.\,V.&&\\
\Avtors{Briukhov~D.\,O.} see~Stupnikov~S.\,A.&&\\
\Avtors{Callaos~N.\,K.\ and Seyful-Mulyukov~R.\,B.} Complexity and
its information content&1&129--139\\
\Avtors{Chertok~A.\,V., Kadaner~A.\,I., Khazeeva~G.\,T., and
Sokolov~I.\,A.} Regime switching detection\linebreak
\\[-12pt]
\hspace*{23pt}for~the~Levy driven
Ornstein--Uhlenbeck process using CUSUM methods&4&46--56\\
\Avtors{Chichagov~V.\,V.} Asymptotic expansions of mean absolute
error of uniformly minimum variance unbiased and maximum likelihood
estimators on the one-parameter exponential\linebreak
\\[-12pt]
\hspace*{23pt}family model of lattice distributions&3&66--76\\
\Avtors{Danishevsky~V.\,I.} see~Kolesnikov A.\,V.&&\\
\Avtors{Fazliev~A.\,Z.} see~Kalinichenko~L.\,A.&&\\
\Avtors{Fedoseev~A.\,A.} What is behind the concept of ``knowledge in
small packages''&3&105--110\\
\Avtors{Gaidamaka~Yu.\,V., Andreev~S.\,D., Sopin~E.\,S.,
Samouylov~K.\,E., and Shorgin~S.\,Ya.} Interference analysis
of~the~device-to-device communications model with~regard to~a~signal\linebreak
\\[-12pt]
\hspace*{23pt}propagation environment&4&\hphantom{1}2--10\\
\Avtors{Gasilov~A.\,V.} see~Yakovlev~O.\,A.&&\\
\Avtors{Goncharov~A.\,V.\ and Strijov~V.\,V.} Metric time series
classification using weighted dynamic\linebreak
\\[-12pt]
\hspace*{23pt}warping relative to centroids of classes&2&36--47\\
\Avtors{Gordov~E.\,P.} see~Kalinichenko~L.\,A.&&\\
\Avtors{Gorshenin~A.\,K.} Concept of online service for stochastic
modeling of real processes&1&72--81\\
\Avtors{Gorshenin~A.\,K.} see~Shnurkov~P.\,V.&&\\
\Avtors{Gorshenin~A.\,K.} see~Shnurkov~P.\,V.&&\\
\Avtors{Grusho~A.\,A., Grusho~N.\,A., Zabezhailo~M.\,I., and
Timonina~E.\,E.} Integration of statistical and\linebreak
\\[-12pt]
\hspace*{23pt}deterministic methods for
analysis of information security&3&2--8\\
\Avtors{Grusho~A.\,A., Zabezhailo~M.\,I., and Zatsarinny~A.\,A.} On
the advanced procedure to reduce\linebreak
\\[-12pt]
\hspace*{23pt}calculation of Galois closures&4&\hphantom{1}96--104\\
\Avtors{Grusho~N.\,A.} see~Grusho~A.\,A.&&\\
\Avtors{Havanskov~V.\,A.} see~Minin~V.\,A.&&\\
\Avtors{Inkova~O.\,Yu.} see~Zatsman~I.\,M.&&\\
\Avtors{Isachenko~R.\,V.\ and Strijov~V.\,V.} Metric learning in
multiclass time series classification\linebreak
\\[-12pt]
\hspace*{23pt}problem&2&48--57\\
\end{tabular}
}
\pagebreak

\def\leftfootline{\small{\textbf{\thepage}
\hfill INFORMATIKA I EE PRIMENENIYA~--- INFORMATICS AND APPLICATIONS\ \ \ 2016\
\ \ volume~10\ \ \ issue\ 4}
}%
 \def\rightfootline{\small{INFORMATIKA I EE PRIMENENIYA~---
INFORMATICS AND APPLICATIONS\ \ \ 2016\ \ \ volume~10\ \ \ issue\ 4
\hfill \textbf{\thepage}}}

\def\leftkol{2016 AUTHOR INDEX} % ENGLISH ABSTRACTS}

\def\rightkol{2016 AUTHOR INDEX} %ENGLISH ABSTRACTS}


{\tabcolsep=2.83pt
\begin{tabular}{p{382pt}cc}
&\textbf{Issue} & \textbf{Page}\\[6pt]
\Avtors{Kadaner~A.\,I.} see~Chertok~A.\,V.&&\\[.255pt]
\Avtors{Kalinichenko~L.\,A., Volnova~A.\,A., Gordov~E.\,P.,
Kiselyova~N.\,N., Kovaleva~D.\,A., Malkov~O.\,Yu., Okladnikov~I.\,G.,
Podkolodnyy~N.\,L., Pozanenko~A.\,S., Ponomareva~N.\,V.,
Stupnikov~S.\,A.,} \textbf{and Fazliev~A.\,Z.} Data access challenges for data
intensive\linebreak
\\[-12pt]
\hspace*{23pt}research in Russia&1& 2--22\\[.255pt]
\Avtors{Karasikov~M.\,E.\ and Strijov~V.\,V.} Feature-based
time-series classification&4&121--131\\[.255pt]
\Avtors{Khazeeva~G.\,T.} see~Chertok~A.\,V.&&\\[.255pt]
\Avtors{Khokhlov~Yu.\,S.} Multivariate fractional Levy motion and its
applications&2&\hphantom{1}98--106\\[.255pt]
\Avtors{Kirikov~I.\,A., Kolesnikov~A.\,V., Listopad~S.\,V., and
Rumovskaya~S.\,B.} Fine-grained hybrid\linebreak
\\[-12pt]
\hspace*{23pt}intelligent systems. Part 2:
Bidirectional hybridization&1&\hphantom{1}96--105\\[.255pt]
\Avtors{Kirikov~I.\,A., Kolesnikov~A.\,V., Listopad~S.\,V., and
Rumovskaya~S.\,B.} ``Virtual council''~---\linebreak
\\[-12pt]
\hspace*{23pt}source environment
supporting complex diagnostic decision making&3&81--90\\[.255pt]
\Avtors{Kiselyova~N.\,N.} see~Kalinichenko~L.\,A.&&\\[.255pt]
\Avtors{Kolesnikov A.\,V., Listopad~S.\,V., Rumovskaya~S.\,B., and
Danishevsky~V.\,I.} Informal axiomatic\linebreak
\\[-12pt]
\hspace*{23pt}theory of~the~role visual models&4&114--120\\[.255pt]
\Avtors{Kolesnikov~A.\,V.} see~Kirikov~I.\,A.&&\\[.255pt]
\Avtors{Kolesnikov~A.\,V.} see~Kirikov~I.\,A.&&\\[.255pt]
\Avtors{Kolin~K.\,K.} Humanitarian aspects of information
security&3&111--121\\[.255pt]
\Avtors{Konovalov~M.\,G.\ and Razumchik~R.\,V.} Dispatching
to~two parallel nonobservable queues using\linebreak
\\[-12pt]
\hspace*{23pt}only static
information&4&57--67\\[.255pt]
\Avtors{Korchagin~A.\,Yu.} see~Korolev~V.\,Yu.&&\\[.255pt]
\Avtors{Korchagin~A.\,Yu.} see~Korolev~V.\,Yu.&&\\[.255pt]
\Avtors{Korepanov~E.\,R.} see~Sinitsyn~I.\,N.&&\\[.255pt]
\Avtors{Korepanov~E.\,R.} see~Sinitsyn~I.\,N.&&\\[.255pt]
\Avtors{Korolev~V.\,Yu., Korchagin~A.\,Yu., and Zeifman~A.\,I.} The
Poisson theorem for Bernoulli trials\linebreak
\\[-12pt]
\hspace*{23pt}with~a~random probability
of~success and~a~discrete analog of~the~Weibull distribution&4&11--20\\[.255pt]
\Avtors{Korolev~V.\,Yu., Zeifman~A.\,I., and Korchagin~A.\,Yu.}
Asymmetric Linnik distributions as~limit\linebreak
\\[-12pt]
\hspace*{23pt}laws for~random sums
of~independent random variables with~finite variances&4&21--33\\[.255pt]
\Avtors{Koucheryavy~E.\,A.} see~Ometov~A.\,Ya.&&\\[.255pt]
\Avtors{Kovaleva~D.\,A.} see~Kalinichenko~L.\,A.&&\\[.255pt]
\Avtors{Kovalyov~S.\,P.} Metaprogramming to increase
manufacturability of large-scale software-\linebreak
\\[-12pt]
\hspace*{23pt}intensive systems&1&56--66\\[.255pt]
\Avtors{Krivenko~M.\,P.} Significance tests of feature selection for
classification&3&32--40\\[.255pt]
\Avtors{Kruzhkov~M.\,G.} see~Zalizniak~Anna~A.&&\\[.255pt]
\Avtors{Kruzhkov~M.\,G.} see~Zatsman~I.\,M.&&\\[.255pt]
\Avtors{Kudryavtsev~A.\,A.} Bayesian queueing and reliability models:
\textit{A~priori} distributions with\linebreak
\\[-12pt]
\hspace*{23pt}compact support&1&67--71\\[.255pt]
\Avtors{Kudryavtsev~A.\,A.} Characteristics dependent on the balance
coefficient in Bayesian models\linebreak
\\[-12pt]
\hspace*{23pt}with compact support of \textit{a priori}
distributions&3&77--80\\[.255pt]
\Avtors{Kudryavtsev~A.\,A.\ and Palionnaia~S.\,I.} Bayesian recurrent
model of reliability growth:\linebreak
\\[-12pt]
\hspace*{23pt}Parabolic distribution of parameters&2&80--83\\[.255pt]
\Avtors{Kudryavtsev~A.\,A.\ and Titova~A.\,I.} Bayesian queuing
and~reliability models: Degenerate-\linebreak
\\[-12pt]
\hspace*{23pt}Weibull case&4&68--71\\[.255pt]
\Avtors{Leontyev~N.\,D.\ and Ushakov~V.\,G.} Analysis of a queueing
system with autoregressive arrivals\linebreak
\\[-12pt]
\hspace*{23pt}and nonpreemptive priority&3&15--22\\[.255pt]
\Avtors{Listopad~S.\,V.} see~Kirikov~I.\,A.&&\\[.255pt]
\Avtors{Listopad~S.\,V.} see~Kirikov~I.\,A.&&\\[.255pt]
\Avtors{Listopad~S.\,V.} see~Kolesnikov A.\,V.&&\\[.255pt]
\Avtors{Malkov~O.\,Yu.} see~Kalinichenko~L.\,A.&&\\[.255pt]
\Avtors{Markov~A.\,S., Monakhov~M.\,M., and
Ulyanov~V.\,V.} Generalized Cornish--Fisher expansions\linebreak
\\[-12pt]
\hspace*{23pt}for distributions of statistics based on samples
of random size&2&84--91\\[.255pt]
\Avtors{Melnikov~A.\,K.\ and Ronzhin~A.\,F.} Generalized statistical
method of~text analysis based\linebreak
\\[-12pt]
\hspace*{23pt}on~calculation of~probability distributions
of~statistical values&4&89--95\\
\end{tabular}
}
\pagebreak

\def\leftfootline{\small{\textbf{\thepage}
\hfill INFORMATIKA I EE PRIMENENIYA~--- INFORMATICS AND APPLICATIONS\ \ \ 2016\
\ \ volume~10\ \ \ issue\ 4}
}%
 \def\rightfootline{\small{INFORMATIKA I EE PRIMENENIYA~---
INFORMATICS AND APPLICATIONS\ \ \ 2016\ \ \ volume~10\ \ \ issue\ 4
\hfill \textbf{\thepage}}}

\def\leftkol{2016 AUTHOR INDEX} % ENGLISH ABSTRACTS}

\def\rightkol{2016 AUTHOR INDEX} %ENGLISH ABSTRACTS}


{\tabcolsep=3pt
\begin{tabular}{p{381pt}cc}
&\textbf{Issue} & \textbf{Page}\\[6pt]
\Avtors{Meykhanadzhyan~L.\,A.} Stationary characteristics of the finite
capacity queueing system with\linebreak
\\[-12pt]
\hspace*{23pt}inverse service order and generalized
probabilistic priority&2&123--131\\[.23pt]
\Avtors{Miller~G.\,B.} see~Borisov~A.\,V.&&\\[.23pt]
\Avtors{Minin~V.\,A., Zatsman~I.\,M., Havanskov~V.\,A., and
Shubnikov~S.\,K.} Intensity of citation of scientific publications in
inventions on information and computer technologies patented\linebreak
\\[-12pt]
\hspace*{23pt}in Russia by domestic and foreign applicants&2&107--122\\[.23pt]
\Avtors{Monakhov~M.\,M.} see~Markov~A.\,S.&&\\[.23pt]
\Avtors{Naumov~V.\,A.\ and Samouylov~K.\,E.} On relationship
between queuing systems with resources\linebreak
\\[-12pt]
\hspace*{23pt}and Erlang networks&3&\hphantom{1}9--14\\[.23pt]
\Avtors{Okladnikov~I.\,G.} see~Kalinichenko~L.\,A.&&\\[.23pt]
\Avtors{Ometov~A.\,Ya., Andreev~S.\,D., Turlikov~A.\,M., and
Koucheryavy~E.\,A.} Performance analysis of\linebreak
\\[-12pt]
\hspace*{23pt}a wireless data
aggregation system with contention for contemporary sensor
networks&3&23--31\\[.23pt]
\Avtors{Palionnaia~S.\,I.} see~Kudryavtsev~A.\,A.&&\\[.23pt]
\Avtors{Podkolodnyy~N.\,L.} see~Kalinichenko~L.\,A.&&\\[.23pt]
\Avtors{Ponomareva~N.\,V.} see~Kalinichenko~L.\,A.&&\\[.23pt]
\Avtors{Popkova~N.\,A.} see~Zatsman~I.\,M.&&\\[.23pt]
\Avtors{Pozanenko~A.\,S.} see~Kalinichenko~L.\,A.&&\\[.23pt]
\Avtors{Razumchik~R.\,V.} see~Konovalov~M.\,G.&&\\[.23pt]
\Avtors{Ronzhin~A.\,F.} see~Melnikov~A.\,K.&&\\[.23pt]
\Avtors{Rumovskaya~S.\,B.} see~Kirikov~I.\,A.&&\\[.23pt]
\Avtors{Rumovskaya~S.\,B.} see~Kirikov~I.\,A.&&\\[.23pt]
\Avtors{Rumovskaya~S.\,B.} see~Kolesnikov A.\,V.&&\\[.23pt]
\Avtors{Samouylov~K.\,E.} see~Gaidamaka~Yu.\,V.&&\\[.23pt]
\Avtors{Samouylov~K.\,E.} see~Naumov~V.\,A.&&\\[.23pt]
\Avtors{Serebryanskii~S.\,M.} see~Tyrsin~A.\,N.&&\\[.23pt]
\Avtors{Seyful-Mulyukov~R.\,B.} see~Callaos~N.\,K.&&\\[.23pt]
\Avtors{Shestakov~O.\,V.} Statistical properties of the denoising method
based on the stabilized hard\linebreak
\\[-12pt]
\hspace*{23pt}thresholding&2&65--69\\[.23pt]
\Avtors{Shestakov~O.\,V.} The strong law of large numbers for the risk
estimate in the problem of\linebreak
\\[-12pt]
\hspace*{23pt}tomographic image reconstruction from
projections with a correlated noise&3&41--45\\[.23pt]
\Avtors{Shestakov~O.\,V.} see~Zakharova~T.\,V.&&\\[.23pt]
\Avtors{Shnurkov~P.\,V., Gorshenin~A.\,K., and Belousov~V.\,V.}
Analytical solution of~the~optimal control\linebreak
\\[-12pt]
\hspace*{23pt}task of~a~semi-Markov
process with~finite set of~states&4&72--88\\[.23pt]
\Avtors{Shnurkov~P.\,V., Zasypko~V.\,V., Belousov~V.\,V., and
Gorshenin~A.\,K.} Development of the algorithm of numerical solution
of the optimal investment control problem\linebreak
\\[-12pt]
\hspace*{23pt}in the closed dynamical model of three-sector economy&1&82--95\\[.23pt]
\Avtors{Shorgin~S.\,Ya.} see~Gaidamaka~Yu.\,V.&&\\[.23pt]
\Avtors{Shorgin~V.\,S.} see~Agalarov~Ya.\,M.&&\\[.23pt]
\Avtors{Shubnikov~S.\,K.} see~Minin~V.\,A.&&\\[.23pt]
\Avtors{Sidorkin~I.\,I.} see~Arkhipov~O.\,P.&&\\[.23pt]
\Avtors{Sinitsyn~I.\,N.} Analytical modeling of processes in stochastic
systems with complex fractional\linebreak
\\[-12pt]
\hspace*{23pt}order Bessel nonlinearities&3&55--65\\[.23pt]
\Avtors{Sinitsyn~I.\,N.} Orthogonal supoptimal filters for nonlinear
stochastic systems on manifolds&1&34--44\\[.23pt]
\Avtors{Sinitsyn~I.\,N.\ and Korepanov~E.\,R.} Normal Pugachev
conditionally-optimal filters and extra-\linebreak
\\[-12pt]
\hspace*{23pt}polators for state linear stochastic systems&2&14--23\\[.23pt]
\Avtors{Sinitsyn~I.\,N.\ and Sinitsyn~V.\,I.} Analytical modeling of
distributions in stochastic systems on\linebreak
\\[-12pt]
\hspace*{23pt}manifolds based on ellipsoidal approximation&1&45--55\\[.23pt]
\Avtors{Sinitsyn~I.\,N., Sinitsyn~V.\,I., and
Korepanov~E.\,R.} Ellipsoidal suboptimal filters for nonlinear\linebreak
\\[-12pt]
\hspace*{23pt}stochastic systems on manifolds&2&24--35\\[.23pt]
\Avtors{Sinitsyn~V.\,I.} see~Sinitsyn~I.\,N.&&\\[.23pt]
\Avtors{Sinitsyn~V.\,I.} see~Sinitsyn~I.\,N.&&\\[.23pt]
\Avtors{Skvortsov~N.\,A.} see~Stupnikov~S.\,A.&&\\[.23pt]
\Avtors{Sokolov~I.\,A.} see~Chertok~A.\,V.&&\\
\end{tabular}
}
\pagebreak

\def\leftfootline{\small{\textbf{\thepage}
\hfill INFORMATIKA I EE PRIMENENIYA~--- INFORMATICS AND APPLICATIONS\ \ \ 2016\
\ \ volume~10\ \ \ issue\ 4}
}%
 \def\rightfootline{\small{INFORMATIKA I EE PRIMENENIYA~---
INFORMATICS AND APPLICATIONS\ \ \ 2016\ \ \ volume~10\ \ \ issue\ 4
\hfill \textbf{\thepage}}}

\def\leftkol{2016 AUTHOR INDEX} % ENGLISH ABSTRACTS}

\def\rightkol{2016 AUTHOR INDEX} %ENGLISH ABSTRACTS}


{\tabcolsep=3pt
\begin{tabular}{p{382pt}cc}
&\textbf{Issue} & \textbf{Page}\\[6pt]
\Avtors{Sopin~E.\,S.} see~Gaidamaka~Yu.\,V.&&\\
\Avtors{Strijov~V.\,V.} see~Goncharov~A.\,V.&&\\
\Avtors{Strijov~V.\,V.} see~Isachenko~R.\,V.&&\\
\Avtors{Strijov~V.\,V.} see~Karasikov~M.\,E.&&\\
\Avtors{Stupnikov~S.\,A., Briukhov~D.\,O., and Skvortsov~N.\,A.}
Co-lending systemic risk analysis over\linebreak
\\[-12pt]
\hspace*{23pt}heterogeneous data collections&1&23--33\\
\Avtors{Stupnikov~S.\,A.} see~Kalinichenko~L.\,A.&&\\
\Avtors{Suchkov~A.\,P.} see~Zatsarinny~A.\,A.&&\\
\Avtors{Timonina~E.\,E.} see~Grusho~A.\,A.&&\\
\Avtors{Titova~A.\,I.} see~Kudryavtsev~A.\,A.&&\\
\Avtors{Turlikov~A.\,M.} see~Ometov~A.\,Ya.&&\\
\Avtors{Tyrsin~A.\,N.\ and Serebryanskii~S.\,M.} Recognition of
dependences on the basis of inverse\linebreak
\\[-12pt]
\hspace*{23pt}mapping&2&58--64\\
\Avtors{Ulyanov~V.\,V.} see~Markov~A.\,S.&&\\
\Avtors{Ushakov~V.\,G.} Queueing system with working vacations and
hyperexponential input stream&2&92--97\\
\Avtors{Ushakov~V.\,G.} see~Leontyev~N.\,D.&&\\
\Avtors{Volnova~A.\,A.} see~Kalinichenko~L.\,A.&&\\
\Avtors{Yakovlev~O.\,A.\ and Gasilov~A.\,V.} Speeded-up stereo
matching using geodesic support weights&3&\hphantom{1}98--104\\
\Avtors{Zabezhailo~M.\,I.} see~Grusho~A.\,A.&&\\
\Avtors{Zabezhailo~M.\,I.} see~Grusho~A.\,A.&&\\
\Avtors{Zakharova~T.\,V.\ and Shestakov~O.\,V.} Precision analysis of
wavelet processing of aerodynamic\linebreak
\\[-12pt]
\hspace*{23pt}flow patterns&3&46--54\\
\Avtors{Zalizniak~Anna~A.\ and Kruzhkov~M.\,G.} Database
of~Russian impersonal verbal constructions&4&132--141\\
\Avtors{Zasypko~V.\,V.} see~Shnurkov~P.\,V.&&\\
\Avtors{Zatsarinny~A.\,A.\ and Suchkov~A.\,P.} Systems engineering
approaches to~the~establishment of\linebreak
\\[-12pt]
\hspace*{23pt}a~system for~decision support based
on~situational analysis&4&105--113\\
\Avtors{Zatsarinny~A.\,A.} see~Grusho~A.\,A.&&\\
\Avtors{Zatsman~I.\,M., Inkova~O.\,Yu., Kruzhkov~M.\,G., and
Popkova~N.\,A.} Representation of cross-\linebreak
\\[-12pt]
\hspace*{23pt}lingual knowledge about
connectors in supracorpora databases&1&106--118\\
\Avtors{Zatsman~I.\,M.} see~Minin~V.\,A.&&\\
\Avtors{Zeifman~A.\,I.} see~Korolev~V.\,Yu.&&\\
\Avtors{Zeifman~A.\,I.} see~Korolev~V.\,Yu.&&\\
\end{tabular}
}

%\thispagestyle{myheadings}
\def\leftfootline{\small{\textbf{\thepage}
\hfill INFORMATIKA I EE PRIMENENIYA~--- INFORMATICS AND APPLICATIONS\ \ \ 2016\
\ \ volume~10\ \ \ issue\ 4}
}%
 \def\rightfootline{\small{INFORMATIKA I EE PRIMENENIYA~---
INFORMATICS AND APPLICATIONS\ \ \ 2016\ \ \ volume~10\ \ \ issue\ 4
\hfill \textbf{\thepage}}}

 \label{end\stat}

\newpage


%\vspace*{-60pt} {\small
{\baselineskip=9.1pt
\section*{Правила подготовки рукописей статей для публикации в журнале
<<Информатика и её применения>>}

\thispagestyle{empty}

 Журнал <<Информатика и её применения>> публикует
теоретические, обзорные и дискуссионные статьи, посвященные научным
исследованиям и разработкам в области информатики и ее приложений. Журнал
издается на русском языке. По специальному решению редколлегии отдельные статьи,
в виде исключения, могут печататься на английском языке.
Тематика журнала охватывает следующие направления:
\begin{itemize}
\item теоретические основы информатики; %\\[-13.5pt]
\item математические методы исследования сложных систем и процессов; %\\[-13.5pt]
\item информационные системы и сети; %\\[-13.5pt]
\item информационные технологии; %\\[-13.5pt]
\item архитектура и программное
обеспечение вычислительных комплексов и сетей.
\end{itemize}
\begin{enumerate}
\item В журнале печатаются результаты, ранее не
опубликованные и не предназначенные к одновременной публикации в других
изданиях. Публикация не должна нарушать закон об авторских правах. Направляя
свою рукопись в редакцию, авторы автоматически передают учредителям и
редколлегии неисключительные права на издание данной статьи на русском языке и
на ее распространение в России и за рубежом. При этом за авторами сохраняются
все права как собственников данной рукописи. В связи с этим авторами должно
быть представлено в редакцию письмо в следующей форме:
Соглашение о передаче права на публикацию:

\textit{<<Мы, нижеподписавшиеся, авторы рукописи <<$\qquad\qquad$>>, передаем
учредителям и редколлегии журнала <<Информатика и её применения>>
неисключительное право опубликовать данную рукопись статьи на русском языке как
в печатной, так и в электронной версиях журнала. Мы подтверждаем, что данная
публикация не нарушает авторского права других лиц или организаций. Подписи
авторов: (ф.\,и.\,о., дата, адрес)>>.}

Указанное соглашение может быть представлено 
как в бумажном виде, так и в виде отсканированной копии (с подписями авторов).


Редколлегия вправе запросить у авторов экспертное заключение о возможности
опубликования представленной статьи в открытой печати. %\\[-13.5pt]
\item Статья
подписывается всеми авторами. На отдельном листе представляются данные автора
(или всех авторов): фамилия, полные имя и отчество, телефон, факс, e-mail,
почтовый адрес. Если работа выполнена несколькими авторами, указывается фамилия
одного из них, ответственного за переписку с редакцией. %\\[-13.5pt]
\item Редакция журнала
осуществляет самостоятельную экспертизу присланных статей. Возвращение рукописи
на доработку не означает, что статья уже принята к печати. Доработанный вариант
с ответом на замечания рецензента необходимо прислать в редакцию. %\\[-13.5pt]
\item Решение
редакционной коллегии о принятии статьи к печати или ее отклонении сообщается
авторам. Редколлегия не обязуется направлять рецензию авторам отклоненной
статьи. %\\[-13.5pt]
\item Корректура статей высылается авторам для просмотра. Редакция
просит авторов присылать свои замечания в кратчайшие сроки. %\\[-13.5pt]
\item При
подготовке рукописи в MS Word рекомендуется использовать следующие настройки.
Параметры страницы: формат~--- А4; ориентация~--- книжная; поля (см): внутри~---
2,5, снаружи~--- 1,5, сверху~--- 2, снизу~--- 2, от края до нижнего
колонтитула~--- 1,3. Основной текст: стиль~--- <<Обычный>>: шрифт Times New
Roman, размер 14~пунктов, абзацный отступ~--- 0,5~см, 1,5 интервала,
выравнивание~--- по ширине. Рекомендуемый объем рукописи~--- не свыше
25~страниц указанного формата. Ознакомиться с шаблонами, содержащими примеры
оформления, можно по адресу в Интернете:
\textsf{http://www.ipiran.ru/journal/template.doc}.
\item К рукописи, предоставляемой в 2-х
экземплярах, обязательно прилагается электронная версия статьи (как правило, в
форматах MS WORD (.doc) или \LaTeX\ (.tex), а также~--- дополнительно~--- в
формате .pdf) на дискете, лазерном диске или по электронной почте. Сокращения
слов, кроме стандартных, не применяются. Все страницы рукописи должны быть
пронумерованы. %\\[-13.5pt]
\item Статья должна содержать следующую информацию на русском и
английском языках: название, Ф.И.О. авторов, места работы авторов и их
электронные адреса, подробные сведения об авторах, оформленные в соответствии с форматом, 
определяемым файлами {\sf http://www.ipiran.ru/journal/issues/2011\_05\_01/authors.asp} и 
{\sf http://www.ipiran.ru/journal/issues/2011\_01\_eng/authors.asp},
аннотация (не более 100~слов), ключевые слова. Ссылки на
литературу в тексте статьи нумеруются (в квадратных скобках) и располагаются в
порядке их первого упоминания. В~списке литературы не должно быть позиций, на которые нет ссылки в тексте статьи.
Все фамилии авторов, заглавия статей, названия
книг, конференций и~т.\,п.\ даются на языке оригинала, если этот язык
использует кириллический или латинский алфавит. %\\[-13.5pt]
\item Присланные в редакцию материалы авторам не возвращаются.
\item При отправке файлов по электронной
почте просим придерживаться следующих правил:
\begin{itemize}
\item указывать в поле subject (тема) название журнала и фамилию автора; %\\[-13.5pt]
\item использовать attach (присоединение); %\\[-13.5pt]
\item в случае больших объемов информации возможно
использование общеизвестных архиваторов (ZIP, RAR); %\\[-13.5pt]
\item в состав электронной версии статьи должны входить: файл, содержащий текст статьи, и файл(ы),
содержащий(е) иллюстрации. %\\[-13.5pt]
\end{itemize}
\item Журнал <<Информатика и её применения>> является некоммерческим изданием. 
Плата за публикацию с авторов не взимается, гонорар авторам не выплачивается.
\end{enumerate}
\thispagestyle{empty}
\textbf{Адрес редакции:} Москва 119333,
ул.~Вавилова, д.~44, корп.~2, ИПИ РАН\\
\hphantom{\textbf{Адрес редакции:} }Тел.: +7 (499) 135-86-92\ \
Факс:  +7 (495) 930-45-05\ \  E-mail:   rust@ipiran.ru }
}

\end{document}


%\tableofcontents

%\end{document}





%\def\stat{cont}
{%\hrule\par
%\vskip 7pt % 7pt
\raggedleft\Large \bf%\baselineskip=3.2ex
А\,В\,Т\,О\,Р\,С\,К\,И\,Й\ \ У\,К\,А\,З\,А\,Т\,Е\,Л\,Ь\ \ З\,А\ \ 2\,0\,0\,7 г. \vskip 17pt
    \hrule
    \par
\vskip 21pt plus 6pt minus 3pt }

\label{st\stat}

\def\tit{\ }

\def\aut{\ }
\def\auf{\ }

\def\leftkol{\ } % ENGLISH ABSTRACTS}

\def\rightkol{\ } %ENGLISH ABSTRACTS}

\titele{\tit}{\aut}{\auf}{\leftkol}{\rightkol}


\contentsline {chapter}{\ }{Выпуск \quad Стр.} 
\contentsline {section}{\textbf{Батракова Д.\,А., Королев В.\,Ю., Шоргин С.\,Я.}\ \ Новый метод вероятностно-ста\-ти\-сти\-че\-ско\-го анализа информационных потоков в\nobreakspace {}телекоммуникационных сетях}{\qquad 1 \qquad 40} 
\contentsline {section}{\textbf{Борисов А.\,В.}\ \ Байесовское оценивание в системах наблюдения с\nobreakspace {}марковскими скачкообразными процессами: игровой подход}{\qquad 2 \qquad 65}
\contentsline {section}{\textbf{Босов А.\,В., Иванов А.\,В.}\ \ Программная инфраструктура информационного Web-пор\-тала}{\qquad 2 \qquad 50}
\contentsline {section}{\textbf{Захаров В.\,Н., Калиниченко Л.\,А., Соколов И.\,А., Ступников С.\,А.}\ \ Конструирование канонических информационных моделей для интегрированных информационных систем}{\qquad 2 \qquad 15}
\contentsline {section}{\textbf{Захаров В.\,Н., Козмидиади В.\,А.}\ \ Средства обеспечения отказоустойчивости при\-ло\-жений}{\qquad 1 \qquad 14} 
\contentsline {section}{\textbf{Иванов А.\,В.}\ \ см. Босов А.\,В.\hfill\hfill\hfill\hfill\hfill\hfill\hfill\hfill\hfill\hfill\hfill\hfill\hfill\hfill\hfill\hfill\hfill\hfill\hfill\hfill\hfill\hfill\hfill\hfill\hfill\hfill\hfill\hfill\hfill\hfill\hfill\hfill\hfill\hfill\hfill}{\ }
\contentsline {section}{\textbf{Ильин В.\,Д., Соколов И.\,А.}\ \ Символьная модель системы знаний информатики в\nobreakspace {}че\-ло\-ве\-ко-автоматной среде}{\qquad 1 \qquad 66} 
\contentsline {section}{\textbf{Калиниченко Л.\,А.}\ \ см. Захаров В.\,Н.\hfill\hfill\hfill\hfill\hfill\hfill\hfill\hfill\hfill\hfill\hfill\hfill\hfill\hfill\hfill\hfill\hfill\hfill\hfill\hfill\hfill\hfill\hfill\hfill\hfill\hfill\hfill\hfill\hfill\hfill\hfill\hfill\hfill\hfill\hfill}{\ }
\contentsline {section}{\textbf{Козеренко Е.\,Б.}\ \ Лингвистическое моделирование для систем машинного перевода и обработки знаний}{\qquad 1 \qquad 54} 
\contentsline {section}{\textbf{Козмидиади В.\,А.}\ \ см. Захаров В.\,Н.\hfill\hfill\hfill\hfill\hfill\hfill\hfill\hfill\hfill\hfill\hfill\hfill\hfill\hfill\hfill\hfill\hfill\hfill\hfill\hfill\hfill\hfill\hfill\hfill\hfill\hfill\hfill\hfill\hfill\hfill\hfill\hfill\hfill\hfill\hfill }{\ } 
\contentsline {section}{\textbf{Королев В.\,Ю.}\ \ см. Батракова Д.\,А.\hfill\hfill\hfill\hfill\hfill\hfill\hfill\hfill\hfill\hfill\hfill\hfill\hfill\hfill\hfill\hfill\hfill\hfill\hfill\hfill\hfill\hfill\hfill\hfill\hfill\hfill\hfill\hfill\hfill\hfill\hfill\hfill\hfill\hfill\hfill}{\ } 
\contentsline {section}{\textbf{Кудрявцев А.\,А., Шоргин С.\,Я.}\ \ Байесовский подход к\nobreakspace {}анализу систем массового обслуживания и\nobreakspace {}показателей надежности}{\qquad 2 \qquad 76}
\contentsline {section}{\textbf{Печинкин А.\,В., Соколов И.\,А., Чаплыгин В.\,В.}\ \ Многолинейная система массового обслуживания с конечным накопителем и ненадежными приборами}{\qquad 1 \qquad 27} 
\contentsline {section}{\textbf{Печинкин А.\,В., Соколов И.\,А., Чаплыгин В.\,В.}\ \ Стационарные характеристики многолинейной\nobreakspace {}системы массового обслуживания с\nobreakspace {}одновременными отказами приборов}{\qquad 2 \qquad 39}
\contentsline {section}{\textbf{Синицын И.\,Н.}\ \ Корреляционные методы построения аналитических информационных моделей флуктуаций полюса Земли по априорным данным}{\qquad 2 \qquad \hphantom{9}2}
\contentsline {section}{\textbf{Синицын И.\,Н.}\ \ Развитие теории фильтров Пугачева для оперативной обработки информации в стохастических системах}{{\qquad 1 \qquad \hphantom{9}3}} 
\contentsline {section}{\textbf{Соколов И.\,А.}\ \ см. Захаров В.\,Н.\hfill\hfill\hfill\hfill\hfill\hfill\hfill\hfill\hfill\hfill\hfill\hfill\hfill\hfill\hfill\hfill\hfill\hfill\hfill\hfill\hfill\hfill\hfill\hfill\hfill\hfill\hfill\hfill\hfill\hfill\hfill\hfill\hfill\hfill\hfill}{\ }
\contentsline {section}{\textbf{Соколов И.\,А.}\ \ см. Ильин В.\,Д.\hfill\hfill\hfill\hfill\hfill\hfill\hfill\hfill\hfill\hfill\hfill\hfill\hfill\hfill\hfill\hfill\hfill\hfill\hfill\hfill\hfill\hfill\hfill\hfill\hfill\hfill\hfill\hfill\hfill\hfill\hfill\hfill\hfill\hfill\hfill}{\ } 
\contentsline {section}{\textbf{Соколов И.\,А.}\ \ см. Печинкин А.\,В.\hfill\hfill\hfill\hfill\hfill\hfill\hfill\hfill\hfill\hfill\hfill\hfill\hfill\hfill\hfill\hfill\hfill\hfill\hfill\hfill\hfill\hfill\hfill\hfill\hfill\hfill\hfill\hfill\hfill\hfill\hfill\hfill\hfill\hfill\hfill}{\ } 
\contentsline {section}{\textbf{Соколов И.\,А.}\ \ см. Печинкин А.\,В.\hfill\hfill\hfill\hfill\hfill\hfill\hfill\hfill\hfill\hfill\hfill\hfill\hfill\hfill\hfill\hfill\hfill\hfill\hfill\hfill\hfill\hfill\hfill\hfill\hfill\hfill\hfill\hfill\hfill\hfill\hfill\hfill\hfill\hfill\hfill}{\ }
\contentsline {section}{\textbf{Ступников С.\,А.}\ \ см. Захаров В.\,Н.\hfill\hfill\hfill\hfill\hfill\hfill\hfill\hfill\hfill\hfill\hfill\hfill\hfill\hfill\hfill\hfill\hfill\hfill\hfill\hfill\hfill\hfill\hfill\hfill\hfill\hfill\hfill\hfill\hfill\hfill\hfill\hfill\hfill\hfill\hfill}{\ }
\contentsline {section}{\textbf{Чаплыгин В.\,В.}\ \ см. Печинкин А.\,В.\hfill\hfill\hfill\hfill\hfill\hfill\hfill\hfill\hfill\hfill\hfill\hfill\hfill\hfill\hfill\hfill\hfill\hfill\hfill\hfill\hfill\hfill\hfill\hfill\hfill\hfill\hfill\hfill\hfill\hfill\hfill\hfill\hfill\hfill\hfill}{\ } 
\contentsline {section}{\textbf{Чаплыгин В.\,В.}\ \ см. Печинкин А.\,В.\hfill\hfill\hfill\hfill\hfill\hfill\hfill\hfill\hfill\hfill\hfill\hfill\hfill\hfill\hfill\hfill\hfill\hfill\hfill\hfill\hfill\hfill\hfill\hfill\hfill\hfill\hfill\hfill\hfill\hfill\hfill\hfill\hfill\hfill\hfill}{\ }
\contentsline {section}{\textbf{Шоргин С.\,Я.}\ \ см. Батракова Д.\,А.\hfill\hfill\hfill\hfill\hfill\hfill\hfill\hfill\hfill\hfill\hfill\hfill\hfill\hfill\hfill\hfill\hfill\hfill\hfill\hfill\hfill\hfill\hfill\hfill\hfill\hfill\hfill\hfill\hfill\hfill\hfill\hfill\hfill\hfill\hfill}{\ } 
\contentsline {section}{\textbf{Шоргин С.\,Я.}\ \ см. Кудрявцев А.\,А.\hfill\hfill\hfill\hfill\hfill\hfill\hfill\hfill\hfill\hfill\hfill\hfill\hfill\hfill\hfill\hfill\hfill\hfill\hfill\hfill\hfill\hfill\hfill\hfill\hfill\hfill\hfill\hfill\hfill\hfill\hfill\hfill\hfill\hfill\hfill}{\ }
%\thispagestyle{myheadings}
\def\leftfootline{\small{\textbf{\thepage}
\hfill ИНФОРМАТИКА И ЕЁ ПРИМЕНЕНИЯ\ \ \ том~1\ \ \ выпуск~2\ \ \ 2007}
}%
 \def\rightfootline{\small{ИНФОРМАТИКА И ЕЁ ПРИМЕНЕНИЯ\ \ \ том~1\ \ \ выпуск~2\ \ \ 2007
 \hfill \textbf{\thepage}}}
 \label{end\stat}

%\def\stat{cont-e}
{%\hrule\par
%\vskip 7pt % 7pt
\raggedleft\Large \bf%\baselineskip=3.2ex
2\,0\,0\,7\ \ A\,U\,T\,H\,O\,R\ \ I\,N\,D\,E\,X \vskip 17pt
    \hrule
    \par
\vskip 21pt plus 6pt minus 3pt }

\label{st\stat}

\def\tit{\ }

\def\aut{\ }
\def\auf{\ }

\def\leftkol{\ } % ENGLISH ABSTRACTS}

\def\rightkol{\ } %ENGLISH ABSTRACTS}

\titele{\tit}{\aut}{\auf}{\leftkol}{\rightkol}


\contentsline {chapter}{\ }{Issue \quad Page} 
\contentsline {subsection}{\textbf{Batrakova D.\,A., Korolev V.\,Yu., Shorgin S.\,Ya.}\ \ A New Method for the Probabilistic and Statistical Analysis of Information Flows in Telecommunication Networks}{\qquad 1 \qquad 40} 
\contentsline {subsection}{\textbf{Borisov A.\,V.}\ \ Bayesian Estimation in\nobreakspace {}Observation Systems with\nobreakspace {}Markov Jump Processes: Game-Theoretic Approach}{\qquad 2 \qquad 65} 
\contentsline {subsection}{\textbf{Bosov A.\,V., Ivanov A.\,V.}\ \ Linguistic Simulation for Machine Translation and Knowledge Management Systems}{\qquad 2 \qquad 50} 
\contentsline {subsection}{\textbf{Chaplygin V.\,V.} see Pechinkin A.\,V.\hfill\hfill\hfill\hfill\hfill\hfill\hfill\hfill\hfill\hfill\hfill\hfill\hfill\hfill\hfill\hfill\hfill\hfill\hfill\hfill\hfill\hfill\hfill\hfill\hfill\hfill\hfill\hfill\hfill\hfill\hfill\hfill\hfill\hfill\hfill}{\ }
\contentsline {subsection}{\textbf{Chaplygin V.\,V.} see Pechinkin A.\,V.\hfill\hfill\hfill\hfill\hfill\hfill\hfill\hfill\hfill\hfill\hfill\hfill\hfill\hfill\hfill\hfill\hfill\hfill\hfill\hfill\hfill\hfill\hfill\hfill\hfill\hfill\hfill\hfill\hfill\hfill\hfill\hfill\hfill\hfill\hfill}{\ }
\contentsline {subsection}{\textbf{Ilyin V.\,D., Sokolov I.\,A.}\ \ The Symbol Model of Informatics Knowledge System in Human-Automaton Environment}{\qquad 1 \qquad 66} 
\contentsline {subsection}{\textbf{Ivanov A.\,V.} see Bosov A.\,V.\hfill\hfill\hfill\hfill\hfill\hfill\hfill\hfill\hfill\hfill\hfill\hfill\hfill\hfill\hfill\hfill\hfill\hfill\hfill\hfill\hfill\hfill\hfill\hfill\hfill\hfill\hfill\hfill\hfill\hfill\hfill\hfill\hfill\hfill\hfill}{\ }
\contentsline {subsection}{\textbf{Kalinichenko L.\,A.} see Zakharov V.\,N.\hfill\hfill\hfill\hfill\hfill\hfill\hfill\hfill\hfill\hfill\hfill\hfill\hfill\hfill\hfill\hfill\hfill\hfill\hfill\hfill\hfill\hfill\hfill\hfill\hfill\hfill\hfill\hfill\hfill\hfill\hfill\hfill\hfill\hfill\hfill}{\ }
\contentsline {subsection}{\textbf{Korolev V.\,Yu.} see Batrakova D.\,A.\hfill\hfill\hfill\hfill\hfill\hfill\hfill\hfill\hfill\hfill\hfill\hfill\hfill\hfill\hfill\hfill\hfill\hfill\hfill\hfill\hfill\hfill\hfill\hfill\hfill\hfill\hfill\hfill\hfill\hfill\hfill\hfill\hfill\hfill\hfill}{\ }
\contentsline {subsection}{\textbf{Kozerenko E.\,B.}\ \ Linguistic Simulation for Machine Translation and Knowledge Management Systems}{\qquad 1 \qquad 54} 
\contentsline {subsection}{\textbf{Kozmidiady V.\,A.} see Zakharov V.\,N.\hfill\hfill\hfill\hfill\hfill\hfill\hfill\hfill\hfill\hfill\hfill\hfill\hfill\hfill\hfill\hfill\hfill\hfill\hfill\hfill\hfill\hfill\hfill\hfill\hfill\hfill\hfill\hfill\hfill\hfill\hfill\hfill\hfill\hfill\hfill}{\ }
\contentsline {subsection}{\textbf{Kudryavtsev A.\,A., Shorgin S.\,Ya.}\ \ Bayesian Approach to Queueing Systems and Reliability Characteristics}{\qquad 2 \qquad 76} 
\contentsline {subsection}{\textbf{Pechinkin A.\,V., Sokolov I.\,A., Chaplygin V.\,V.}\ \ Multichannel Queuing System with Finite Buffer and Unreliable Servers}{\qquad 1 \qquad 27} 
\contentsline {subsection}{\textbf{Pechinkin A.\,V., Sokolov I.\,A., Chaplygin V.\,V.}\ \ Stationary Characteristics of a Multichannel Queueing System with\nobreakspace {}Simultaneous Refusals of Servers}{\qquad 2 \qquad 39} 
\contentsline {subsection}{\textbf{Shorgin S.\,Ya.} see Batrakova D.\,A.\hfill\hfill\hfill\hfill\hfill\hfill\hfill\hfill\hfill\hfill\hfill\hfill\hfill\hfill\hfill\hfill\hfill\hfill\hfill\hfill\hfill\hfill\hfill\hfill\hfill\hfill\hfill\hfill\hfill\hfill\hfill\hfill\hfill\hfill\hfill}{\ }
\contentsline {subsection}{\textbf{Shorgin S.\,Ya.} see Kudryavtsev A.\,A.\hfill\hfill\hfill\hfill\hfill\hfill\hfill\hfill\hfill\hfill\hfill\hfill\hfill\hfill\hfill\hfill\hfill\hfill\hfill\hfill\hfill\hfill\hfill\hfill\hfill\hfill\hfill\hfill\hfill\hfill\hfill\hfill\hfill\hfill\hfill}{\ }
\contentsline {subsection}{\textbf{Sinitsyn I.\,N.}\ \ Correlational Methods for Analytical Informational Models of the Earth Pole Fluctuations Design Based on a priori Data}{\qquad 2 \qquad \hphantom{9}2}
\contentsline {subsection}{\textbf{Sinitsyn I.\,N.}\ \ Development of Pugachev Filtering for Stochastic Systems}{\qquad 1 \qquad \hphantom{9}3}
\contentsline {subsection}{\textbf{Sokolov I.\,A.} see Ilyin V.\,D.\hfill\hfill\hfill\hfill\hfill\hfill\hfill\hfill\hfill\hfill\hfill\hfill\hfill\hfill\hfill\hfill\hfill\hfill\hfill\hfill\hfill\hfill\hfill\hfill\hfill\hfill\hfill\hfill\hfill\hfill\hfill\hfill\hfill\hfill\hfill}{\ }
\contentsline {subsection}{\textbf{Sokolov I.\,A.} see Pechinkin A.\,V.\hfill\hfill\hfill\hfill\hfill\hfill\hfill\hfill\hfill\hfill\hfill\hfill\hfill\hfill\hfill\hfill\hfill\hfill\hfill\hfill\hfill\hfill\hfill\hfill\hfill\hfill\hfill\hfill\hfill\hfill\hfill\hfill\hfill\hfill\hfill}{\ }
\contentsline {subsection}{\textbf{Sokolov I.\,A.} see Pechinkin A.\,V.\hfill\hfill\hfill\hfill\hfill\hfill\hfill\hfill\hfill\hfill\hfill\hfill\hfill\hfill\hfill\hfill\hfill\hfill\hfill\hfill\hfill\hfill\hfill\hfill\hfill\hfill\hfill\hfill\hfill\hfill\hfill\hfill\hfill\hfill\hfill}{\ }
\contentsline {subsection}{\textbf{Sokolov I.\,A.} see Zakharov V.\,N.\hfill\hfill\hfill\hfill\hfill\hfill\hfill\hfill\hfill\hfill\hfill\hfill\hfill\hfill\hfill\hfill\hfill\hfill\hfill\hfill\hfill\hfill\hfill\hfill\hfill\hfill\hfill\hfill\hfill\hfill\hfill\hfill\hfill\hfill\hfill}{\ }
\contentsline {subsection}{\textbf{Stupnikov S.\,A.} see Zakharov V.\,N.\hfill\hfill\hfill\hfill\hfill\hfill\hfill\hfill\hfill\hfill\hfill\hfill\hfill\hfill\hfill\hfill\hfill\hfill\hfill\hfill\hfill\hfill\hfill\hfill\hfill\hfill\hfill\hfill\hfill\hfill\hfill\hfill\hfill\hfill\hfill}{\ }
\contentsline {subsection}{\textbf{Zakharov V.\,N., Kalinichenko L.\,A., Sokolov I.\,A., Stupnikov S.\,A.}\ \ Development of Canonical Information Models for Integrated Information Systems}{\qquad 2 \qquad 15} 
\contentsline {subsection}{\textbf{Zakharov V.\,N., Kozmidiady V.\,A.}\ \ Means Providing Applications Fault Tolerance}{\qquad 1 \qquad 14} 
\def\leftfootline{\small{\textbf{\thepage}
\hfill ИНФОРМАТИКА И ЕЁ ПРИМЕНЕНИЯ\ \ \ том~1\ \ \ выпуск~2\ \ \ 2007}
}%
 \def\rightfootline{\small{ИНФОРМАТИКА И ЕЁ ПРИМЕНЕНИЯ\ \ \ том~1\ \ \ выпуск~2\ \ \ 2007
 \hfill \textbf{\thepage}}}
 \label{end\stat}


%\tableofcontents


\end{document}