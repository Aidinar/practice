%\renewcommand{\P}{{\sf P}}
%\renewcommand{\endproof}{\hfill$\Box$}
%\renewcommand{\phi}{\varphi}
%\newcommand{\la}{\lambda}
%\newcommand{\si}{{\rm Si}\:}
%\renewcommand{\Re}{{\rm Re}\:}
%\newcommand{\eqd}{\stackrel{d}{=}}

\def\stat{bening}

\def\tit{ОЦЕНКИ СКОРОСТИ СХОДИМОСТИ РАСПРЕДЕЛЕНИЙ СЛУЧАЙНЫХ
СУММ К ДИСПЕРСИОННЫМ ГАММА-РАСПРЕДЕЛЕНИЯМ$^*$}

\def\titkol{Оценки скорости сходимости распределений случайных
сумм к дисперсионным гамма-распределениям}

\def\autkol{В.\,Е.~Бенинг, Л.\,М.~Закс,  В.\,Ю.~Королев}
\def\aut{В.\,Е.~Бенинг$^1$, Л.\,М.~Закс$^2$,  В.\,Ю.~Королев$^3$}

\titel{\tit}{\aut}{\autkol}{\titkol}

{\renewcommand{\thefootnote}{\fnsymbol{footnote}}\footnotetext[1]
{Работа поддержана РФФИ, гранты 12-07-00109а, 12-07-00115а.}}


\renewcommand{\thefootnote}{\arabic{footnote}}
\footnotetext[1]{Факультет вычислительной
математики и кибернетики Московского государственного университета
им.\ М.\,В.~Ломоносова; Институт проблем информатики РАН,
bening@cs.msu.su}
\footnotetext[2]{Альфа-банк, отдел моделирования и математической статистики, lily.zaks@gmail.com}
\footnotetext[3]{Факультет вычислительной математики и
кибернетики Московского государственного университета 
им.~М.\,В.~Ломоносова; Институт проблем информатики РАН,
vkorolev@cs.msu.su}

\Abst{Строятся оценки точности приближения
распределений отрицательных биномиальных случайных сумм независимых
случайных величин с ненулевыми средними несимметричными
дисперсионными гамма-рас\-пре\-де\-ле\-ниями.}

\KW{случайная сумма; отрицательное биномиальное
распределение; смешанное пуассоновское распределение;
гамма-распределение; оценка скорости сходимости; дисперсионное
гамма-распределение}

\vskip 14pt plus 9pt minus 6pt

      \thispagestyle{headings}

      \begin{multicols}{2}

            \label{st\stat}

\section{Введение}

Дисперсионные гамма-распределения являются традиционными
математическими моделями статистических закономерностей,
используемыми для описания эффекта наличия тяжелых или полутяжелых
хвостов. Такие модели очень важны для адекватного описания
статистических закономерностей поведения различных характеристик
открытых сложных систем, эволюция которых в значительной мере
зависит от информационных потоков, к примеру телекоммуникационных
сетей или финансовых рынков. В~частности, в финансовой математике
хорошо известны так называемые {\it дис\-пер\-си\-он\-ные гам\-ма-про\-цес\-сы}
(variance gamma processes или VG-processes). В~развитие идей
теоретического моделирования таких эффектов с помощью предельных
теорем для обобщенных дважды стохастических пуассоновских процессов~--- 
в определенном смысле наилучших моделей хаотических процессов со
случайной интенсивностью,~--- изложенных в работах~[1--3], в данной
статье приводятся оценки скорости сходимости распределений
специальных смешанных пуассоновских случайных сумм независимых
случайных величин с ненулевыми средними к несимметричным
дисперсионным гам\-ма-рас\-пре\-де\-ле\-ни\-ям (variance gamma-distributions).
Как показано в работе~\cite{KorolevSokolov2012}, такие распределения
возникают как асимптотические аппроксимации для аддитивных
характеристик информационных потоков в том случае, когда случайная
интенсивность соответствующего потока информативных событий имеет
гамма-распределение. Таким образом, приводимые в данной статье
оценки могут быть полезны при определении адекватности смешанных
вероятностных моделей статистических закономерностей, в которых
смешивающим является гам\-ма-рас\-пре\-де\-ление.

Дисперсионные гамма-про\-цес\-сы, предложенные в работах~\cite{MadanSeneta1990, CarrMadanChang1998},~--- 
это процессы Леви
(слу\-чай-\linebreak ные процессы с независимыми стационарными приращениями),
одномерные распределения которых являются дисперсионными
гам\-ма-рас\-пре\-де\-ле\-ни\-ями. Плот\-ность дисперсионного гам\-ма-рас\-пре\-де\-ле\-ния
имеет вид:
\begin{multline}
f_{\mathrm{VG}}(x;\alpha,\sigma,\mu,\lambda)={}\\
\hspace*{-2mm}{}=\fr{1}{\sqrt{2\pi}\sigma}
\int\limits_{0}^{\infty}\exp\left\{-\fr{1}{2}\left(\fr{x-\alpha
u}{\sigma\sqrt{u}}\right)^2\right\}\fr{g(u;\mu,\lambda)}{\sqrt{u}}\,du,\!\!\!
\label{e1-ben}
\end{multline}
где $\alpha\in\r$, $\sigma\hm>0$, $\mu\hm>0$, $\lambda\hm>0$, a
$g(x;\mu,\lambda)$~--- плот\-ность гам\-ма-рас\-пре\-де\-ле\-ния с параметрами
$\mu$ и~$\lambda$,
\begin{equation}
g(x;\mu,\lambda)=\fr{\lambda^{\mu}}{\Gamma(\mu)}\,x^{\mu-1}e^{-\lambda
x}\,,\enskip x\geqslant0\,.\label{e2-ben}
\end{equation}
Здесь и далее $\Gamma(\,\cdot\,)$~--- эйлерова гам\-ма-функ\-ция,
$$
\Gamma(z) = \int\limits_{0}^{\infty}e^{-y}y^{z-1}\,dy\,,\enskip z>0\,.
$$
В дальнейшем функцию гамма-распределения, соответствующую плот\-ности
$g(x;\mu,\lambda)$ (см.~(\ref{e2-ben})), будем обозначать $G(x;\mu,\lambda)$, а
функцию дисперсионного гам\-ма-рас\-пре\-де\-ле\-ния, соответствующую
плотности $f_{\mathrm{VG}}(x;\alpha,\sigma,\mu,\lambda)$ (см.~(\ref{e1-ben})), будем
обозначать $F_{\mathrm{VG}}(x;\alpha,\sigma,\mu,\lambda)$.

Как отмечено в упомянутых работах, подобные модели также
демонстрируют высокую адекватность при описании динамики цен
финансовых активов.

Вместе с тем в прикладной теории вероятностей хорошо известен
принцип, согласно которому та или иная модель может считаться в
достаточной мере обоснованной только тогда, когда она является {\it
асимптотической аппроксимацией}, т.\,е.\ ког\-да существует довольно
простая предельная теорема, в которой рассматриваемая модель
выступает в качестве предельного распределения~\cite{GnedenkoKolmogorov1949}. 
В~книге~\cite{GnedenkoKorolev1996}
прослежена глубокая связь этого принципа с универсальным принципом
неубывания энтропии в замкнутых сис\-те\-мах. Обе рассматриваемые в
данной статье модели имеют вид сдвиг-мас\-штаб\-ных смесей нормальных
законов. Как известно, нормальное распределение обладает
максимальной энтропией среди всех распределений, носителем которых
является вся числовая прямая, и имеющих конечный второй момент. Если
бы моделируемая сложная сис\-те\-ма была информационно изолирована от
окружающей среды, то в соответствии с принципом неубывания энтропии,
который в теории вероятностей проявляется в виде предельных 
тео\-рем~\cite{GnedenkoKorolev1996}, наблюдаемые статистические распределения
ее характеристик были бы неотличимы от нормального. Но поскольку
любая математическая модель по своему определению не может учесть
все факторы, влияющие на состояние или эволюцию моделируемой
сис\-те\-мы, то параметры этого нормального закона изменяются в
зависимости от состояния среды, внешней по отношению к моделируемой
системе. Другими словами, эти параметры являются случайными и
изменяются под влиянием информационных потоков между сис\-те\-мой и
внешней средой. Таким образом, во многих ситуациях разумные модели
статистических закономерностей изменения параметров сложных сис\-тем
должны иметь вид сдвиг-масштабных смесей нормальных законов, частным
случаем которых является~(\ref{e1-ben}).

В первоисточниках упомянутые выше модели вводились чисто
умозрительно как распределения процесса броуновского движения со
случайным временем, в каждый момент имеющим гам\-ма-рас\-пре\-де\-ле\-ние.
<<Асимптотическое>> обоснование моделей типа дисперсионных
гам\-ма-рас\-пре\-де\-ле\-ний было дано лишь недавно в работе~\cite{KorolevSokolov2012}, 
где показано, что такие распределения
могут выступать в качестве предельных в довольно простых предельных
теоремах для регулярных статистик, построенных по выборкам
случайного объема, в част\-ности, в схеме случайного суммирования
случайных величин и, следовательно, могут считаться {\it
естественными} асимптотическими аппроксимациями для распределений
многих процессов, например, сходных с неоднородными случайными
блужданиями.

Изучению условий и скорости сходимости распределений специальных
случайных сумм к част\-ному случаю дисперсионных гам\-ма-рас\-пре\-де\-ле\-ний~--- 
несимметричному распределению Лапласа\linebreak посвя\-щен раздел~12.7.7 в
книге~\cite{KorolevBeningShorgin2011}. Свойства самог$\acute{\mbox{о}}$
несимметричного распределения Лапласа и его практическое применение
описаны в работе~\cite{KotzKozubowskiPodgorski2001}.

Пусть $\{X_{n,j}\}_{j\geqslant1}$, $n\hm=1,2,\ldots$~--- последовательность
серий независимых и одинаково в каждой серии распределенных
случайных величин, а $N_n$, $n\hm=1,2,\ldots$~--- положительные
целочисленные случайные величины такие, что при каждом~$n$ случайная
величина~$N_n$ независима от по\-сле\-до\-ва\-тель\-ности
$\{X_{n,j}\}_{j\geqslant1}$. Для натуральных~$k$ обозначим
$$
S_{n,k}=X_{n,1}+\ldots+X_{n,k}\,.
$$
Для определенности будем считать, что $\sum\limits_{j=1}^0\hm=0$ и все функции
распределения, о которых пойдет речь ниже, непрерывны справа. Символ
$\Longrightarrow$ будет обозначать сходимость по распределению.
Пусть $\Phi(x)$~--- стандартная нормальная функция рас\-пре\-де\-ления:
$$
\Phi(x)=\fr{1}{\sqrt{2\pi}}\int\limits_{-\infty}^xe^{-z^2/2}\,dz\,.
$$


В работе~\cite{KorolevSokolov2012} доказана следующая теорема.

\smallskip

\noindent
\textbf{Теорема A}. {\it Предположим, что существуют числа $\alpha \hm \in
\mathbb{R}$, $\sigma ^{2} \hm\in (0, \infty )$, $\mu\hm \in(0,\infty)$, и
по\-сле\-до\-ва\-тель\-ность натуральных чисел $\left\{m_n \right\}_{n
\hm\geqslant 1}$ такие, что при $n\to\infty$}
\begin{align}
{\sf P}(S_{n, m_n}<x) &\longrightarrow
\Phi\left(\fr{x-\alpha}{\sigma}\right)\,;\label{e3-ben}
\\
{\sf P}(N_{n}<m_nx) &\longrightarrow G(x;\mu,\lambda)\,.
\label{e4-ben}
\end{align}
\textit{Тогда
$$
{\sf P}(S_{n, N_{n}}<x)\longrightarrow
F_{\mathrm{VG}}(x;\alpha,\sigma,\mu,\lambda)\,,
$$
причем предельная функция дисперсионного гамма-распределения может
быть представлена как свертка функций распределения
$G(x;\mu,\lambda_1)$ и $1\hm-G(-x;\mu,\lambda_2)$, где}
$$
\lambda_1 = \left(\fr{1}{2}\sqrt{\fr{\alpha
^{2}}{\lambda^{2}}+\fr{2\sigma^{2}}{\lambda}} -\fr{ \alpha}
{2\lambda}\right)^{-1}\,;
$$
$$
\lambda_2 =
\left(\fr{1}{2}\sqrt{\fr{\alpha^{2}}{\lambda^{2}}+\fr{2\sigma^{2}}{\lambda}} +\fr{ \alpha}
{2\lambda}\right)^{-1}\,.
$$


\smallskip

В данной статье будет рассмотрена скорость сходимости в теореме~А
для одного довольно ил\-люст\-ра\-тив\-но\-го частного случая,
демонстрирующего один из возможных механизмов формирования
несимметричных предельных законов для случайных блужданий.

\section{Основной результат}

Пусть $\xi_1,\xi_2,\ldots$~--- независимые одинаково распределенные
случайные величины с ${\sf E}\xi_1\hm=0$, $0\hm<{\sf
D}\xi_1\hm=\sigma^2<\infty$, $\beta^3\hm={\sf E}|\xi_1|^3\hm<\infty$,
$a\hm\in\r$, $n$~--- натуральное число. Положим
$$
X_{n,j}=\fr{\xi_j}{\sqrt{n}}+\fr {\alpha}{n}\,.
$$
В терминах случайных блужданий рассматриваемая конструкция слагаемых
предполагает {\it одинаковый порядок малости} элементарных трендов и
{\it дис\-пер\-сий}, что характерно, например, для приращений
винеровского процесса со сносом. Обозначим
$$
S_n=\sum\limits_{j=1}^n X_{n,j}\
\left(=\fr{1}{\sqrt{n}}\sum\limits_{j=1}^n\xi_j+\alpha\right)\,.
$$
В силу классической центральной предельной теоремы имеем
$$
\lim_{n\to\infty}\sup\limits_{x\in\r}\left\vert{\sf
P}(S_n<x)-\Phi\left(\fr{x-\alpha}{\sigma}\right)\right\vert=0\,,
$$
т.\,е.\ так определенные случайные величины $X_{n,j}$ удовлетворяют
условию~(\ref{e3-ben}) с $m_n\hm=n$.

Пусть $V_{\mu,\lambda}$~--- случайная величина, имеющая
гам\-ма-рас\-пре\-де\-ле\-ние $G(x;\mu,\lambda)$, независимая от стандартного
пуассоновского процесса~$M(t)$, $t\hm\geqslant0$. Для каждого натурального~$n$ положим
$$
N_n=M(nV_{\mu,\lambda})\,.
$$
Очевидно, что так определенная случайная величина $N_n$ имеет
смешанное пуассоновское распределение:
\begin{multline*}
{\sf P}(N_n=k)={\sf P}\left(M(nV_{\mu,\lambda})=k\right)=\\
{}=
\int\limits_0^{\infty}e^{-nz}\fr{(nz)^k}{k!}\,g(z;\mu,\lambda)\,dz\,,\enskip
k=0,1,\ldots
\end{multline*}
Более того, несложно проверить, что это распределение является
отрицательным биномиальным (распределением Паскаля) с параметрами
$\mu$ и $p_n={\lambda}/(n+\lambda)$:
\begin{multline}
{\sf P}\left(M(nV_{\mu,\lambda})=k\right)={}\\
{}=\int\limits_0^{\infty}e^{-nz}
\fr{(nz)^k}{k!}\,\fr{\lambda^{\mu}}{\Gamma(\mu)}\,e^{-\lambda z}z^{\mu-1}\,dz={}\\
{}=
\fr{n^k\lambda^{\mu}\Gamma(k+\mu)}{k!\Gamma(\mu)(\lambda+n)^{k+\mu}}={}\\
{}
=\fr{\Gamma(k+\mu)}{k!\Gamma(\mu)}\left(\fr{\lambda}{\lambda+n}\right)^{\mu}
\left(1-\fr{\lambda}{\lambda+n}\right)^k\!,\\ 
k=0,1,\ldots\!\!
\label{e5-ben}
\end{multline}
Подобные случайные величины хорошо описывают число несчастных
случаев в некоторой большой группе, отдельные индивидуумы которой в
разной степени подвержены риску~\cite{GreenwoodYule1920}, число
страховых выплат по неоднородным страховым портфелям~\cite{KorolevBeningShorgin2011}, 
чис\-ло сделок на больших биржах, на
которых интенсивность торгов неоднородна в силу разных интересов
разных групп игроков в зависимости от времени~\cite{Korolev2011}.

Обозначим $A_n(z)=A_n(z;\mu,\lambda)\hm={\sf P}(N_n<nz)$, $z\hm\geqslant0$
($A_n(z)\hm=0$ при $z\hm<0$). Несложно видеть, что
$$
A_n(x)\Longrightarrow G(x;\mu,\lambda)\enskip (n\to\infty)\,.
$$
Действительно, как известно, если $\Pi(x;\ell)$~--- функция
распределения Пуассона с параметром $\ell>0$ и $E(x;c)$~--- функция
распределения с единственным единичным скачком в точке $c\hm\in\r$, то
$$
\Pi(\ell x;\ell)\Longrightarrow E(x;1)\enskip (\ell\to\infty)\,.
$$
Так как для $x\in\r$
$$
A_n(x)=\int\limits_{0}^{\infty}\Pi(n x; n z)\,dG(z;\mu,\lambda)\,,
$$
то по теореме Лебега о мажорируемой сходимости при $n\hm\to\infty$
\begin{multline*}
A_n(x)\Longrightarrow\int\limits_{0}^{\infty}E(x/z;1)\,dG(z;\mu,\lambda)={}\\
{}=
\int\limits_{0}^{x}\,dG(z;\mu,\lambda)=G(x;\mu,\lambda)\,,
\end{multline*}
т.\,е.\ так определенные случайные величины $N_n$ удов\-ле\-тво\-ря\-ют
условию~(\ref{e4-ben}) с $m_n\hm=n$. Впредь будем считать, что при каждом $n\hm\geqslant1$
случайная величина $N_n$ независима от последовательности
$\{\xi_j\}_{j\geqslant1}$, что гарантирует независимость случайных величин
$N_n,X_{n,1},X_{n,2},\ldots$

Таким образом, в силу непрерывности функции распределения
$F_{\mathrm{VG}}(x;\alpha,\sigma,\mu,\lambda)$ из теоремы~A следует, что
\begin{multline*}
D_n\equiv{}\\
{}\equiv \sup\limits_{x\in\r}\left\vert \,{\sf P}
\left(\sum\limits_{j=1}^{N_n}X_{n,j}<x\right)-F_{\mathrm{VG}}(x;\alpha,\sigma,\mu,\lambda)
\right\vert\longrightarrow{}\\
{}\longrightarrow 
0\enskip (n\to\infty)\,.
\end{multline*}
Скорость стремления $D_n$ к нулю описывается следующим утверждением.

\smallskip

\noindent
\textbf{Теорема~1.} \textit{Пусть $\mu>1/2$. Для любого $n\hm\geqslant1$
справедлива оценка}
$$
D_n\leqslant\fr{0{,}4532}{\sqrt{n}}\,\fr{\beta^3}{\sigma^3}\,
\fr{\sqrt{\lambda}\Gamma\left(\mu-{1}/{2}\right)}{\Gamma(\mu)}
+\fr{0{,}1210}{n}\,\fr{\alpha^2}{\sigma^2}\,.
$$

\smallskip

\noindent
Д\,о\,к\,а\,з\,а\,т\,е\,л\,ь\,с\,т\,в\,о\,.\ Как уже было показано, распределение
случайной величины $N_n$ является смешанным пуассоновским (см.~(\ref{e5-ben})).
Следовательно, по теореме Фубини
\begin{multline*}
{\sf P}\left(\sum\limits_{j=1}^{N_n}X_{n,j}<x\right)=
{\sf P}\left(\sum\limits_{j=1}^{M(nU_{\mu,\lambda})}X_{n,j}<x\right)={}\\
{}=
\int\limits_0^{\infty}{\sf P}\left(\sum\limits_{j=1}^{M(nz)}X_{n,j}<x\right)g(z;\mu,\lambda)\,dz\,.
\end{multline*}
При этом
\begin{gather*}
{\sf E}X_{n,j}=\fr{\alpha}{n}\,;\quad
{\sf D}X_{n,j}=\fr{\sigma^2}{n}\,;\\ 
{\sf E}|X_{n,j}-{\sf E}X_{n,j}|^3=\fr{\beta^3}{n^{3/2}}.
\end{gather*}
Таким образом, при каждом $z\hm\in(0,\infty)$
\begin{align*}
{\sf E}\sum\limits_{j=1}^{M(nz)}X_{n,j}&=\alpha z\,;
\\
{\sf D}\sum\limits_{j=1}^{M(nz)}X_{n,j}&=nz\left(\fr{\alpha^2}{n^2}+\fr{\sigma^2}{n}\right)=
z\sigma^2\left(1+\fr{\alpha^2}{n\sigma^2}\right)\,.
\end{align*}
Согласно определению дисперсионного гам\-ма-рас\-пре\-де\-ления,
$$
F_{\mathrm{VG}}(x;\alpha,\sigma,\mu,\lambda)=\int\limits_{0}^{\infty}
\Phi\left(\fr{x-az}{\sigma\sqrt{z}}\right)g(z;\mu,\lambda)\,dz\,.
$$
Поэтому
\begin{multline}
D_n=\sup_x\Bigg|\int\limits_{0}^{\infty}g(z;\mu,\lambda)\Bigg[{\sf P}
\Bigg(\sum\limits_{j=1}^{M(nz)}X_{n,j}<x\Bigg)-{}\\
{}-\Phi\Bigg(\fr{x-\alpha z}
{\sigma\sqrt{z(1+{\alpha^2}/({n\sigma^2}))}}\Bigg)+{}\\
{}+\Phi\Bigg(\fr{x-\alpha z}{\sigma\sqrt{z(1+{\alpha^2}/({n\sigma^2}))}}\Bigg)-
\Phi\Bigg(\fr{x-\alpha z}{\sigma\sqrt{z}}\Bigg)\Bigg]\,dz\Bigg| \leqslant{}\\
{}\leqslant I_1+I_2\,,
\label{e6-ben}
\end{multline}
где
\begin{align*}
I_1&=\int\limits_{0}^{\infty}g(z;\mu,\lambda)\sup_x\Bigg|\,{\sf P}
\left(\sum\limits_{j=1}^{M(nz)}X_{n,j}<x\right)-{}\\
&\hspace*{15mm}{}-\Phi\Bigg(\fr{x-\alpha z}
{\sigma\sqrt{z\left(1+{\alpha^2}/(n\sigma^2)\right)}}\Bigg)\Bigg|\,dz\,;
\\
I_2&=\int\limits_{0}^{\infty}g(z;\mu,\lambda)
\sup\limits_x\Bigg|\,\Phi\Bigg(\fr{x-\alpha
z}{\sigma\sqrt{z\left(1+{\alpha^2}/(n\sigma^2)\right)}}\Bigg)-{}\\
&\hspace*{30mm}{}-
\Phi\left(\fr{x-az}{\sigma\sqrt{z}}\right)\Bigg|\,dz\,.
\end{align*}
В дальнейшем понадобится следующее утверждение.

\smallskip

\noindent
\textbf{Лемма~1.} \textit{Пусть случайные величины $X_1,X_2,\ldots$
одинаково распределены. Пусть $N_{\lambda}$~--- пуассоновская
случайная величина с параметром $\lambda\hm>0$. Предположим, что
случайные величины $N_{\lambda},X_1,X_2,\ldots$ независимы в
совокупности. Обозначим
$$
S_{\lambda}=X_1+\cdots+X_{N_{\lambda}}\,.
$$
Тогда}
\begin{multline*}
\sup\limits_x\bigg|\,{\sf P}(S_{\lambda}<x)-\Phi\left(\fr{x-{\sf
E}S_{\lambda}}{\sqrt{{\sf D}S_{\lambda}}}\right)\bigg|
\leqslant{}\\
{}\leqslant
\fr{0{,}4532}{\sqrt{\lambda}}\,
\fr{{\sf E}|X_1-{\sf E}X_1|^3}{({\sf D}X_1)^{3/2}}\,.
\end{multline*}

\smallskip

\noindent
Д\,о\,к\,а\,з\,а\,т\,е\,л\,ь\,с\,т\,в\,о\ леммы~1 приведено в 
работе~\cite{KorolevShevtsovaShorgin2011}
(см.\ также~[8, с.~144]).

\smallskip

Продолжим доказательство теоремы~1. Рас\-смот\-рим $I_1$. Применяя лемму~1 с учетом~(\ref{e5-ben}),
получаем:
\begin{multline}
I_1\leqslant\fr{0{,}4532}{\sqrt{n}}\,
\fr{\beta^3}{\sigma^3}\,
\fr{\lambda^{\mu}}{\Gamma(\mu)}\int\limits_{0}^{\infty}\,
e^{-\lambda z}z^{\mu-3/2}\,dz={}\\
{}=
\fr{0{,}4532}{\sqrt{n}}\,
\fr{\beta^3}{\sigma^3}\,
\fr{\lambda^{\mu}\Gamma\left(\mu-{1}/{2}\right)}{\lambda^{\mu-1/2}\Gamma(\mu)}={}\\
{}=
\fr{0{,}4532}{\sqrt{n}}\,\fr{\beta^3}{\sigma^3}\,
\fr{\sqrt{\lambda}\Gamma\left(\mu-{1}/{2}\right)}{\Gamma(\mu)}\,.
\label{e7-ben}
\end{multline}

Рассмотрим $I_2$. В~дальнейшем понадобится еще одно вспомогательное
утверждение.

\smallskip

\noindent
\textbf{Лемма~2.} \textit{Пусть $b\hm\in\r$, $0\hm<c\hm<\infty$, $0\hm<d\hm<\infty$.
Тогда}
\begin{gather}
\sup\limits_y|\Phi(y)-\Phi(cy)|\leqslant\fr{1}{\sqrt{2\pi e}}
\Big|\max\left\{c,\,\fr{1}{c}\right\}-1\Big|\,;\label{e8-ben}
\\
\sqrt{1+d}-1\leqslant\fr{d}{2}\,.\label{e9-ben}
\end{gather}

\smallskip

\noindent
Элементарное д\,о\,к\,а\,з\,а\,т\,е\,л\,ь\,с\,т\,в\,о\ этого утверждения основано
на применении формулы Лагранжа.

\smallskip

Продолжим доказательство теоремы~1. В~лемме~2 положим
$$
y=\fr{x-\alpha z}{\sigma\sqrt{z}\sqrt{1+{\alpha^2}/({n\sigma^2})}}\,;\enskip
c=\sqrt{1+\fr{\alpha^2}{n\sigma^2}}\,.
$$
Тогда $c\geqslant1$ и в силу утверждения~(\ref{e8-ben}) леммы~2 имеем:
$$
I_2\leqslant\fr{1}{\sqrt{2\pi e}}\left(\sqrt{1+\fr{\alpha^2}{n\sigma^2}}-1\right)\,.
$$
При этом в силу утверждения~(\ref{e9-ben}) леммы~2
$$
\sqrt{1+\fr{\alpha^2}{n\sigma^2}}-1\leqslant\fr{\alpha^2}{2n\sigma^2}\,.
$$
Окончательно получаем:
\begin{equation}
I_2\leqslant\fr{\alpha^2}{2\sqrt{2\pi e}n\sigma^2}\,.\label{e10-ben}
\end{equation}
Подставляя~(\ref{e7-ben}) и~(\ref{e10-ben}) в~(\ref{e6-ben}), получаем утверждение теоремы. 
Теорема доказана.

\smallskip

Если $\alpha\hm=0$, то оценку, представленную в теореме~1, можно
уточнить не только за счет того, что в таком случае второе слагаемое
в правой части обнуляется, но и за счет уменьшения коэффициента при
первом слагаемом. Аналогом леммы~1 в таком случае может служить
следующее утверждение.

\smallskip

\noindent
\textbf{Лемма 3.} \textit{Пусть случайные величины $X_1,X_2,\ldots$
одинаково распределены. Пусть $N_{\ell}$~--- пуассоновская случайная
величина с параметром $\ell\hm>0$. Предположим, что случайные величины
$N_{\ell},X_1,X_2,\ldots$ независимы в совокупности. Обозначим
$$
S_{\ell}=X_1+\cdots+X_{N_{\ell}}\,.
$$
Тогда}
$$
\sup_x\left\vert\,{\sf P}(S_{\ell}<x)-\Phi\left(\!\fr{x-{\sf E}S_{\ell}}{\sqrt{{\sf D}S_{\ell}}}
\!\right)\right\vert\leqslant\fr{0{,}3041}{\sqrt{\ell}}\,
\fr{{\sf E}|X_1|^3}{({\sf E}X_1^2)^{3/2}}.
$$

\smallskip

\noindent
Д\,о\,к\,а\,з\,а\,т\,е\,л\,ь\,с\,т\,в\,о\ этого утверждения приведено 
в~\cite{KorolevShevtsova2010}, также см.~\cite{KorolevBeningShorgin2011}, теорема~2.4.3.

\smallskip

Если в случае $\alpha\hm=0$ в доказательстве теоремы~1 вместо леммы~1
воспользоваться леммой~3, то в результате получится следующее
утверждение.

\smallskip

\noindent
\textbf{Следствие 1.} \textit{Пусть в дополнение к условиям теоремы~$1$
$\alpha\hm=0$. Тогда для любого $n\geqslant1$ справедлива оценка}
$$
D_n\leqslant\fr{0{,}3041}{\sqrt{n}}\,
\fr{\beta^3}{\sigma^3}\,\fr{\sqrt{\lambda}\Gamma\left(\mu-{1}/{2}\right)}{\Gamma(\mu)}\,.
$$

\smallskip

Это утверждение обобщает следствие~12.2.2 из~\cite{KorolevBeningShorgin2011} 
на случай произвольных параметров~$\mu$ и~$\lambda$.

{\small\frenchspacing
{%\baselineskip=10.8pt
\addcontentsline{toc}{section}{Литература}
\begin{thebibliography}{99}

\bibitem{KorolevSokolov2008} 
\Au{Королев В.\,Ю., Соколов И.\,А.} Математические модели
неоднородных потоков экстремальных событий.~--- М.: Торус Пресс, 2008.

\bibitem{KorolevShorgin2011} 
\Au{Королев В.\,Ю., Шоргин С.\,Я.} Математические
методы анализа стохастической структуры информационных потоков.~---
М.: ИПИ РАН, 2011.

\bibitem{KorolevSokolov2012} 
\Au{Королев В.\,Ю., Соколов И.\,А.} Скошенные распределения Стьюдента,
дисперсионные гам\-ма-рас\-пре\-де\-ле\-ния и их обобщения как асимптотические
аппроксимации~// Информатика и её применения, 2012. Т.~6. Вып.~1.
С.~2--10.

\bibitem{MadanSeneta1990} 
\Au{Madan D.\,B., Seneta~E.} The variance gamma (V.G.) model for
share market return~// J.~Business, 1990. Vol.~63.
P.~511--524.

\bibitem{CarrMadanChang1998} 
\Au{Carr P.\,P., Madan D.\,B., Chang~E.\,C.} The variance gamma process and option pricing~//
Eur. Finance Rev., 1998. Vol.~2. P.~79--105.

\bibitem{GnedenkoKolmogorov1949} 
\Au{Гнеденко Б.\,В., Колмогоpов А.\,Н.} Пpедельные
pаспpеделения для сумм независимых случайных величин.~--- М.--Л.:
ГИТТЛ, 1949.

\bibitem{GnedenkoKorolev1996} 
\Au{Gnedenko B.\,V., Korolev V.\,Yu.} Random summation:
Limit theorems and applications.~--- Boca Raton: CRC Press, 1996.

\bibitem{KorolevBeningShorgin2011} 
\Au{Королев В.\,Ю., Бенинг В.\,Е., Шоргин~С.\,Я.} Математические
основы теории риска.~--- 2-е изд., перераб. и доп.~--- М.: Физматлит,
2011. 620~с.

\bibitem{KotzKozubowskiPodgorski2001} 
\Au{Kotz S., Kozubowski T.\,J.,
Podgorski~K.} The Laplace distribution and generalizations: A~revisit with applications to 
communications, economics, engineering and finance.~--- Boston: Birkhauser, 2001.

\bibitem{GreenwoodYule1920} 
\Au{Greenwood M., Yule G.\,U.} An
inquiry into the nature of frequency-distributions of multiple
happenings, etc.~// J.~Royal Statist. Soc., 1920. Vol.~83. P.~255--279.

\bibitem{Korolev2011} 
\Au{Королев В.\,Ю.} Ве\-ро\-ят\-но\-ст\-но-ста\-ти\-сти\-че\-ские методы
декомпозиции волатильности хаотических процессов.~--- М.: Изд-во Московского ун-та, 2011. 510~с.

\bibitem{KorolevShevtsovaShorgin2011} 
\Au{Королев В.\,Ю., Шевцова И.\,Г., Шоргин~С.\,Я.} О неравенствах типа Бер\-ри--Эс\-се\-ена для
пуассоновских случайных сумм~// Информатика и её применения, 2011.
Т.~5. Вып.~3. С.~64--66.

\label{end\stat}

\bibitem{KorolevShevtsova2010} 
\Au{Korolev V., Shevtsova~I.} An improvement of the Berry--Esseen
inequality with applications to Poisson and mixed Poisson random
sums~// Scandinavian Actuarial J., June 04, 2010.
DOI:10.1080/03461238.2010.485370.
 \end{thebibliography}
}
}


\end{multicols}