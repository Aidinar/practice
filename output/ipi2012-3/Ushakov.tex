

%\renewcommand{\le}{\leqslant}
%\renewcommand{\ge}{\geqslant}
%\renewcommand{\P}{\mathbf P}
%\newcommand{\E}{\mathbf E\,}
%\newcommand{\D}{\mathbf D\,}
%\renewcommand{\Re}{\mathrm{Re}\,}
%\renewcommand{\Im}{\mathrm{Im}\,}

\def\stat{ushakov}

\def\tit{АНАЛИЗ СИСТЕМЫ ОБСЛУЖИВАНИЯ С~ГИПЕРЭКСПОНЕНЦИАЛЬНЫМ
ВХОДЯЩИМ ПОТОКОМ В~УСЛОВИЯХ КРИТИЧЕСКОЙ ЗАГРУЗКИ$^*$}

\def\titkol{Анализ системы обслуживания с гиперэкспоненциальным
входящим потоком в условиях критической загрузки}

\def\autkol{А.\,В.~Ушаков}
\def\aut{А.\,В.~Ушаков$^1$}

\titel{\tit}{\aut}{\autkol}{\titkol}

{\renewcommand{\thefootnote}{\fnsymbol{footnote}}\footnotetext[1]
{Работа выполнена при финансовой поддержке РФФИ, гранты 11-07-00112а, 12-07-00109а.}}


\renewcommand{\thefootnote}{\arabic{footnote}}
\footnotetext[1]{Институт проблем информатики Российской академии наук; ushakov@akado.ru}

\vspace*{3pt}

\Abst{Найдены предельные распределения виртуального времени ожидания 
при критической загрузке в одноканальной
системе обслуживания с относительным приоритетом и рекуррентным входящим потоком 
с гиперэкспоненциальным распределением
интервалов между поступлениями требований.}

\vspace*{1pt}

\KW{виртуальное время ожидания; относительный приоритет; гиперэкспоненциальный поток; 
критическая загрузка}

\vspace*{4pt}

\vskip 14pt plus 9pt minus 6pt

      \thispagestyle{headings}

      \begin{multicols}{2}

            \label{st\stat}


\section{Введение}

В данной статье изучено предельное поведение виртуального времени ожидания в одноканальной 
системе с гиперэкспоненциальным
входящим потоком и относительным приоритетом в условиях критической загрузки. Рассмотрен 
случай, когда не существует момент второго порядка длительностей обслуживания требований, 
но существует момент порядка $1\hm<\gamma\hm<2.$
Показано, что в зависимости
от соотношения скоростей стремления времени к бесконечности и загрузки к единице возможны 
три различных предельных распределения.
Доказательство предельной теоремы основано на результатах работы~[1], в которой найдено 
нестационарное распределение времени ожидания,
работ~[2--5], в которых подробно изучены свойства решений ряда функциональных 
уравнений, а также асимптотических разложениях этих решений, полученных в данной работе.

\section{Описание системы. Основные обозначения и~предположения}

Рассматривается последовательность одноканальных сис\-тем обслуживания. Каждое 
поступившее в систему требование направляется в один из $r$, $r\hm\geqslant 1$, 
приоритетных классов. В~$n$-й сис\-те\-ме (а)~длительности обслуживания~---
независимые в совокупности и не зависящие от входящего потока случайные 
величины с функцией распределения $B_i^{(n)}(x)$ для требований $i$-го класса;
(б)~входящий поток требований~--- рекуррентный, определяемый плотностью распределения 
интервалов между поступлениями требований вида
$$
a^{(n)}(x)=
\begin{cases}
\sum\limits_{j=1}^{N}c_j^{(n)}a_j^{(n)}\exp\left(-a_j^{(n)}x\right)\,,& x\geqslant 0\,;\\[3pt]
0\,,& x<0\,,
\end{cases}
$$
где $a_i^{(n)}\not= a_j^{(n)}$ при $i\not= j,\ c_j^{(n)}>0$, $\sum\limits_{i=1}^{N}c_i^{(n)}=1$;
(в)~поступившее требование направляется в $i$-й приоритетный класс с вероятностью $p_i^{(n)}$,
$i=1,\ldots,r,$ независимо от остальных требований.

Будем предполагать, что системы обслуживания в рассматриваемой последовательности 
функционируют независимо друг от друга, требования из класса с меньшим номером имеют 
относительный приоритет перед требованиями из класса с большим номером.  Требования 
из одного приоритетного класса обслуживаются в порядке их поступления в сис\-те\-му 
(дисциплина FIFO~--- first in, first out). Пусть, кроме того, в начальный момент $t\hm=0$ 
сис\-те\-мы свободны  от требований.

В дальнейшем изложении для сокращения записи будем опускать индекс~$n$ 
(номер в серии). При этом $\lim$ будет означать
$\lim\limits_{n\rightarrow \infty}.$ Предельные значения параметров
входящего потока и длительностей обслуживания (функции
распределения, моменты и~т.\,п.) будем обозначать теми же символами,
что и допредельные, но с дополнительным индексом~<<$*$>>. Пусть
$b_i(x)$, $\beta_i(s)$, $\beta_{ij}$~--- соответственно плот\-ность
распределения, преобразование Лап\-ла\-са--Стилтье\-са и $j$-й момент
случайной величины с функцией распределения $B_i(x),$

\noindent
\begin{alignat*}{2}
a&=\left(\sum\limits_{j=1}^N c_ja_j^{-1}\right)^{-1}\,;\ &\quad
\rho_{k1}&=a\cdot\sum\limits_{i=1}^k p_i\beta_{i1}\,;\\
\rho_{k}&=1-\rho_{k1}\,;\ &\quad \rho&=\rho_{r}\,.
\end{alignat*}

Введем случайный процесс
$w(t)$~--- виртуальное время ожидания для требований $r$-го приоритетного 
класса в момент времени~$t$.

Положим
\begin{align*}
W_{j}(s,t)&=\int\limits_0^{\infty}e^{-sy}d_y\mathbf{P}(w(t)<y,j(t)=j)\,,\\
\omega_{j}(s,v)&=\int\limits_0^{\infty}e^{-vt}
W_{j}(s,t)\,dt\,.
\end{align*}

Сделаем следующие предположения:
\begin{enumerate}[(1)]
\item существует  момент порядка $1\hm<\gamma\hm<2$ длительностей обслуживания 
требований всех приоритетов, причем справедливы разложения
\begin{equation}
\label{2}
\beta_i(s)=1-\beta_{i1}s+\hat{\beta}_{i\gamma}s^{\gamma}+o_n(s^{\gamma})\,,
\end{equation}
где $({o_n(s^{\gamma})}/s^{\gamma})\rightarrow 0$ при $s\hm\rightarrow 0$ равномерно по~$n;$
\item для любого $n\geqslant 1$
$\rho_{r1}<1;$
\item
существуют пределы $\lim c_j\hm=c_j^*,$ $\lim a_j\hm=a_j^*$, $j\hm=1,\ldots,N,$
$\lim\beta_{i1}\hm=\beta_{i1}^*$, $\lim\hat{\beta}_{i\gamma}\hm=\hat{\beta}_{i\gamma}^*$, 
$i\hm=1,\ldots,r,$
$\lim p_i\hm=p_i^*$, $i=1,\ldots,r,$ $\lim\rho_{r-11}\hm<1,$ $\lim\rho_{r1}\hm=1.$
\end{enumerate}

\section{Предварительные результаты}

Из результатов работы~[1] вытекают следующие соотношения для нахождения распределения 
виртуального времени ожидания:
\begin{multline}
\label{3}
\omega_{j}(s,v)=
\sum\limits_{\nu=1}^N\sum\limits_{k=1}^N
\prod\limits_{l\ne \nu}\fr{\mu_{k,r-1}^{(k)}(s)+a_l}{a_l}\times\\
{}\times
\prod\limits_{p\ne k}\fr{\mu_{p,r-1}^{(p)}(s)}{\mu_{p,r-1}^{(p)}(s)-\mu_{k,r-1}^{(k)}(s)}
c_{\nu}a_j\!\left(\sum\limits_{m=1}^{r-1} p_mz_{r-1,m}^{(k)}(s)+{}\right.\\
\left.{}+ p_r
\vphantom{\sum\limits_{m=1}^{r-1}}
\right)
\sum\limits_{c=1}^N
\left(\prod\limits_{q\ne j}(\mu_{c,r-1}^{(k)}(s)+a_q)\Bigg /
\left(\left(\mu_{c,r-1}^{(k)}(s)+{}\right.\right.\right.\\
\left.\left.\left.{}+a_{\nu}\vphantom{\mu_{c,r-1}^{(k)}}\right)
\alpha_c\left(z_{r-1,1}^{(k)}(s),\ldots,z_{r-1,r-1}^{(k)}(s),1\right) \right)
\vphantom{\prod\limits_{q\ne j} \sum\limits_{c=1}^N}
\right)\times{}\\
{}\times 
\omega_{rj}\left(s-\mu_{c,r-1}^{(k)}(s),v\right)\,,
\end{multline}
где
\begin{multline}
\omega_{rj}(s,v)=\fr{c_j}{v-s+a_j}-\fr{s}{v-s+a_j}\, p_{0j}(v)+{}\\
{}+\fr{c_j}{v-s+a_j}
\sum\limits_{k=1}^N a_k\,\omega_{rk}(s,v)\sum\limits_{m=1}^rp_m\beta_m(s)\,;
\label{4}
\end{multline}

\vspace*{-12pt}

\noindent
\begin{multline}
\hspace*{-12pt}\left(\!1-\sum\limits_{j=1}^N\fr{c_ja_j}{v-s+a_j}\sum\limits_{m=1}^rp_m
\beta_m(s)\right)\!
\sum\limits_{k=1}^N a_k \omega_{rk}(s,v)={}\\
{}=
\sum\limits_{j=1}^N\fr{c_ja_j}{v-s+a_j}-s\sum\limits_{j=1}^N\fr{a_jp_{0j}(v)}{v-s+a_j}\,;
\label{5}
\end{multline}
функции $p_{0k}(v)$, $k=1,\ldots,N,$ определяются по формулам
\begin{multline*}
a_k p_{0k}(v)=\sum\limits_{l=1}^N\sum\limits_{\nu=1}^N\fr{c_{\nu}a_{\nu}}
{(v+\psi_{lr}(v))(a_{\nu}-\psi_{lr}(v))}\times{}\\
{}\times
\fr{\prod\limits_{j=1}^N((a_k-\psi_{jr}(v))(a_j-\psi_{lr}(v)))}{(a_k-\psi_{lr}(v))
\prod\limits_{n\ne l}(\psi_{nr}(v)-\psi_{lr}(v))
\prod\limits_{i\ne k}(a_k-a_i)}\,,
\end{multline*}
а функции $\psi_{li}(v)$, $l=1,\ldots,N$, $i=0,\ldots,r,$ являются решениями 
(относительно~$s$) уравнения
\begin{equation*}
\sum\limits_{j=1}^N\fr{c_ja_j}{a_j-s}\left(\sum\limits_{m=1}^ip_m\beta_m(s+v)+
\sum\limits_{m=i+1}^rp_m\right)=1\,.
\end{equation*}
Пусть далее
$\alpha_k(z)=\prod\limits_{i\ne k}(\mu_k(z)-\mu_i(z)),$ а $\mu_1(z),\ldots,\mu_N(z)$~--- 
корни многочлена
$
\prod\limits_{i=1}^N(\mu+a_i)\hm-(p,z) \sum\limits_{j=1}^Nc_ja_j\prod\limits_{i\ne j}(\mu+a_i)$,
\begin{align*}
z_{ij}^{(k)}(s)&=\beta_j(s+\psi_{ki}(s))\,;\\
\mu_{ci}^{(k)}(s)&=\mu_c(z_{i1}^{(k)}(s),\ldots,z_{ii}^{(k)}(s),1,\ldots,1)\,;\\
(p,z)&=\sum\limits_{m=1}^rp_m z_m\,.
\end{align*}

\smallskip

\noindent
\textbf{Замечание 1.} 
Легко показать, что только одна из функций $\psi_{li}(v)$, $l=1,\ldots,N,$ 
обращается в нуль при $v\hm=0$ и только одна
из функций $\mu_k(z),k\hm=1,\ldots,N,$ обращается в нуль при $z\hm=1.$ Не ограничивая
общности, будем считать, что $\psi_{1i}(0)\hm=0$ и $\mu_1(1)\hm=0.$

\section{Основные результаты}

Для доказательства основной предельной теоремы понадобится
асимтотика при $n \hm\rightarrow \infty$ функций, определенных в
предыдущем параграфе.

\smallskip

\noindent
\textbf{Лемма 1.} \textit{Справедливы следующие асимптотические разложения функции 
$\psi_{1r}(v):$
\begin{multline}
\label{8}
\psi_{1r}(v\rho^{\alpha})
={}\\
{}=\begin{cases}
\left(\fr{v}{u}\right)^{{1}/{\gamma}}\rho^{{\alpha}/{\gamma}}+o\left(\rho^{{\alpha}/{\gamma}}\right)\,,&\alpha<\fr{\gamma}{\gamma-1}\,;
\\[6pt]
\psi(v)\rho^{{1}/({\gamma-1})}+o\left(\rho^{{1}/({\gamma-1})}\right),&\alpha=\fr{\gamma}{\gamma-1}\,;\\[6pt]
v\rho^{\alpha-1}+o(\rho^{\alpha-1})\,,&\alpha>\fr{\gamma}{\gamma-1}\,,
\end{cases}
\end{multline}
где $u=a\sum\limits_{m=1}^rp_m\hat{\beta}_{m\gamma},$ а $\psi(v)$~--- решение 
уравнения $u\psi^{\gamma}(v)\hm+\psi(v)\hm-v\hm=0,$ удовлетворяющее
условию} $\psi(0)=0.$

\smallskip

\noindent
Д\,о\,к\,а\,з\,а\,т\,е\,л\,ь\,с\,т\,в\,о\,.\
Так как
$$
\sum\limits_{j=1}^N\fr{c_ja_j}{a_j-\psi_{1r}(v\rho^{\alpha})}
\!\left(\sum\limits_{m=1}^rp_m\beta_m(v\rho^{\alpha}+\psi_{1r}(v\rho^{\alpha}))\!\right)=1,\hspace*{-2.73624pt}
$$
то, используя~(\ref{2}), имеем:
\begin{multline*}
\left(1+a^{-1}\psi_{1r}\left(v\rho^{\alpha}\right)+
\sum\limits_{j=1}^Nc_ja_j^{-2}\psi_{1r}\left(v\rho^{\alpha}\right)^2
+{}\right.\\
\left.{}+o\left(\psi_{1r}\left(v\rho^{\alpha}\right)^2\right)
\vphantom{\sum\limits_{j=1}^N}
\!\right)\!
\left(1-a^{-1}\rho_{r1}\left(v\rho^{\alpha}+\psi_{1r}\left(v\rho^{\alpha}\right)\right)+{}\right.\\
\left.{}+
a^{-1}u\left(v\rho^{\alpha}+\psi_{1r}\left(v\rho^{\alpha}\right)\right)^{\gamma}+{}\right.\\
{}+\left.
o\left(\left(v\rho^{\alpha}+\psi_{1r}\left(v\rho^{\alpha}\right)\right)^{\gamma}\right)\right)
=1\,.
\end{multline*}
Отсюда
\begin{multline}
-v\rho^{\alpha}+\rho\psi_{1r}\left(v\rho^{\alpha}\right)+u\left(v\rho^{\alpha}+\psi_{1r}\left(v\rho^{\alpha}\right)\right)^{\gamma}
={}\\
{}=o\left(\left(v\rho^{\alpha}+\psi_{1r}\left(v\rho^{\alpha}\right)\right)^{\gamma}\right)\,.
\label{9}
\end{multline}
Положим $\psi_{1r}\left(v\rho^{\alpha}\right)\hm=\psi(v)\rho^{\varphi}.$ Рассмотрим три случая.
\begin{enumerate}[1.]
\item Пуcть $\alpha<{\gamma}/({\gamma-1}).$ Тогда из~(\ref{9}) имеем 
$\varphi={\alpha}/{\gamma}$ и $u\psi^{\gamma}(v)=v.$
\item Пусть $\alpha={\gamma}/({\gamma-1}).$ Тогда  $\varphi={1}/({\gamma-1}),$ 
а $\psi(v)$ является решением уравнения $u\psi^{\gamma}(v)\hm+\psi(v)\hm-v\hm=0.$
\item Пусть $\alpha>{\gamma}/({\gamma-1}).$ Тогда $\varphi\hm=\alpha\hm-1$ и $\psi(v)\hm=v.$
\end{enumerate}

Отсюда вытекают разложения~(\ref{8}).

\medskip

\noindent
\textbf{Лемма 2.} \textit{Справедливо следующее асимптотическое разложение 
функции} $\psi_{1,r-1}(s)$:
\begin{equation*}
\psi_{1,r-1}(s\rho^{\delta})=\fr{\rho_{r-1,1}}{\rho_{r-1}}s\rho^{\delta}+o(\rho^{\delta})\,.
\end{equation*}

\noindent
Д\,о\,к\,а\,з\,а\,т\,е\,л\,ь\,с\,т\,в\,о\,.\
Из определения функции $\psi_{1,r-1}(s)$ имеем:
\begin{multline*}
\sum\limits_{j=1}^N\fr{c_ja_j}{a_j-\psi_{1,r-1}(s\rho^{\delta})}\times{}\\
{}\times
\left(\sum\limits_{m=1}^{r-1}p_m\beta_m(s\rho^{\delta}+\psi_{1,r-1}(s\rho^{\delta}))+
p_r\right)=1\,.
\end{multline*}
Отсюда и из~(\ref{2}) получаем:
\begin{multline*}
\left(1+a^{-1}\psi_{1,r-1}\left(s\rho^{\delta}\right)+
o\left(\psi_{1,r-1}\left(s\rho^{\delta}\right)\right)\right)\times{}\\
{}\times\left(1-a^{-1}\rho_{r-1,1}\left(s\rho^{\delta}+\psi_{1,r-1}\left(s\rho^{\delta}\right)\right)
+{}\right.\\
\left.{}+o\left(s\rho^{\delta}+\psi_{1,r-1}\left(s\rho^{\delta}\right)\right)\right)=1\,.
\end{multline*}

Отсюда следует утверждение леммы.

В дальнейшем
$$
\delta=\begin{cases}
\fr{\alpha}{\gamma},&\alpha\leqslant \fr{\gamma}{\gamma-1}\,;\\[6pt]
\fr{1}{\gamma-1}\,,&\alpha> \fr{\gamma}{\gamma-1}\,.
\end{cases}
$$

\noindent
\textbf{Лемма 3.} \textit{Справедливы следующие асимптотические разложения}:
\begin{multline*}
1-\sum\limits_{j=1}^N\fr{c_ja_j}{v\rho^{\alpha}+
a_j-\left(s\rho^{\delta}+\psi_{1,r-1}\left(s\rho^{\delta}\right)\right)}\times{}\\
{}\times
\sum\limits_{m=1}^rp_m\beta_m\left(s\rho^{\delta}+\psi_{1,r-1}\left(s\rho^{\delta}\right)\right)={}\\
{}=\begin{cases}
a^{-1}\rho^{\alpha}\left(v-\fr{us^{\gamma}}{\rho_{r-1}^{\gamma}}\right)+
o\left(\rho^{\alpha}\right),&\hspace*{-30mm}\alpha<\fr{\gamma}{\gamma-1}\,;\\[6pt]
a^{-1}\rho^{{\gamma}/({\gamma-1})}\!\left(\!v-\fr{s}{\rho_{r-1}}-
\fr{us^{\gamma}}{\rho_{r-1}^{\gamma}}\right)+
o\left(\rho^{{\gamma}/({\gamma-1})}\right),&\hspace*{-18.81908pt}\\[6pt]
&\hspace*{-30mm}\alpha=\fr{\gamma}{\gamma-1}\,;\\[6pt]
-a^{-1}\fr{s}{\rho_{r-1}}\rho^{{\gamma}/({\gamma-1})}\left(\!1+
\fr{s^{\gamma-1}u}{\rho_{r-1}^{\gamma-1}}\!\right)+{}\\
&\hspace*{-55mm}{}+o\left(\rho^{{\gamma}/({\gamma-1})}\right),\ \alpha>\fr{\gamma}{\gamma-1}.
\end{cases}
\end{multline*}


\noindent
Д\,о\,к\,а\,з\,а\,т\,е\,л\,ь\,с\,т\,в\,о\,.\
Используя разложения~(\ref{2})  и результаты леммы~2, имеем:

\noindent
\begin{multline*}
1-\sum\limits_{j=1}^N\fr{c_ja_j}{v\rho^{\alpha}+a_j-\left(s\rho^{\delta}+\psi_{1,r-1}\left(s\rho^{\delta}\right)\right)}
\times{}\\
{}\times
\sum\limits_{m=1}^rp_m\beta_m\left(s\rho^{\delta}+\psi_{1,r-1}\left(s\rho^{\delta}\right)\right)={}\\
{}=1-\left(
\vphantom{\sum\limits_{j=1}^N}
1+a^{-1}\left(s\rho^{\delta}+\psi_{1,r-1}\left(s\rho^{\delta}\right)-v\rho^{\alpha}\right)\right.+{}\\
\hspace*{-3.59697pt}\left.{}+\sum\limits_{j=1}^N
c_ja_j^{-2}\left(s\rho^{\delta}+\psi_{1,r-1}\left(s\rho^{\delta}\right)-v\rho^{\alpha}\right)^2+o\left(\rho^{2\delta}\right)
\!\right)\times{}\\
{}\times
\left(1-\sum\limits_{m=1}^rp_m\beta_{m1}
\left(s\rho^{\delta}+\psi_{1,r-1}\left(s\rho^{\delta}\right)\right)
+{}\right.\\
\left.{}+\sum\limits_{m=1}^rp_m\hat{\beta}_{m\gamma}\left(s\rho^{\delta}+\psi_{1,r-1}\left(s\rho^{\delta}\right)\right)^{\gamma}
+o\left(\rho^{\delta\gamma}\right)\right)
={}\\
{}=a^{-1}v\rho^{\alpha}-a^{-1}\rho\left(s\rho^{\delta}+\psi_{1,r-1}\left(
s\rho^{\delta}\right)\right)-{}\\
{}-a^{-1} u
\left(s\rho^{\delta}+\psi_{1,r-1}\left(s\rho^{\delta}\right)\right)^{\gamma}+
o\left(\rho^{\delta\gamma}\right)
={}\\
{}=
a^{-1}\left(v\rho^{\alpha}-\fr{s\rho^{\delta+1}}{\rho_{r-1}}-\fr{s^{\gamma}u}
{\rho_{r-1}^{\gamma}}\rho^{\delta\gamma}\right)+
o\left(\rho^{\delta\gamma}\right)\,.
\end{multline*}
Отсюда вытекает утверждение леммы.

\medskip


\noindent
\textbf{Лемма 4.} \textit{Для функций $p_{0k}(v)$ справедливы следующие асимптотические 
разложения:}
\begin{multline*}
p_{0k}(v\rho^{\alpha})
={}\\
\hspace*{-4pt}{}=\begin{cases}
\left(\fr{u}{v}\right)^{{1}/{\gamma}}\ f_k\: \rho^{-{\alpha}/{\gamma}}+
o\left(\rho^{-{\alpha}/{\gamma}}\right),&\hspace*{-6pt}\alpha<\fr{\gamma}{\gamma-1};
\\[6pt]
\psi^{-1}(v)\ f_k\rho^{-{1}/({\gamma-1})}+o\left(\rho^{-{1}/({\gamma-1})}\right),&
\hspace*{-6pt}\alpha=\fr{\gamma}{\gamma-1};\\[6pt]
\fr{f_k}{v} \rho^{1-\alpha}+o(\rho^{1-\alpha})\,,&\hspace*{-6pt}\alpha>\fr{\gamma}{\gamma-1},
\end{cases}\hspace*{-7.47086pt}
\end{multline*}
\textit{где}
$$
f_k=\prod\limits_{i\ne k}\fr{a_i}{a_k-a_i}
 \prod\limits_{j=2}^{N}\fr{a_k-\psi_{jr}(0)}{\psi_{jr}(0)}\,.
$$

\noindent
Д\,о\,к\,а\,з\,а\,т\,е\,л\,ь\,с\,т\,в\,о\
непосредственно вытекает из определения функций $p_{0k}(v)$ и результатов леммы~1.

\medskip

\noindent
\textbf{Лемма 5.} 
\begin{multline*}
\lim\rho^{\alpha} \sum\limits_{k=1}^N a_k\omega_{rk}\left(s\rho^{\delta}-\mu_{1,r-1}^{(1)}\left(s\rho^{\delta}\right),v
\rho^{\alpha}\right)={}\\
{}=
\begin{cases}
\fr{a^{*}\left(1-{s}/({\rho^*_{r-1}})\left({u^*}/{v}\right)^{{1}/{\gamma}}\right)}
{v\left(1-{u^*}/{v}\left({s}/({\rho^*_{r-1}})\right)^{\gamma}\right)},&\alpha<\fr{\gamma}{\gamma-1}\,;\\[9pt]
\fr{a^{*}\left(1-{s}/({\rho^*_{r-1}\psi(v)})\right)}{v-{s}/({\rho_{r-1}^*})-
{u^*s^{\gamma}}/{(\rho^{*}_{r-1})^{\gamma}}}\,,& \alpha=\fr{\gamma}{\gamma-1}\,;\\[9pt]
\fr{a^{*}}{v\left(1+{s^{\gamma-1}u^*}/{(\rho_{r-1}^*)^{\gamma-1}}\right)}\,,& \alpha>\fr{\gamma}{\gamma-1}\,.
\end{cases}
\end{multline*}


\noindent
Д\,о\,к\,а\,з\,а\,т\,е\,л\,ь\,с\,т\,в\,о\,.\
Прежде всего заметим,что $\sum\limits_{k=1}^Nf_k\hm=1.$
Далее из~(\ref{5}) имеем:
\begin{multline*}
\rho^{\alpha} \sum\limits_{k=1}^N a_k\omega_{rk}\left(s\rho^{\delta}-\mu_{1,r-1}^{(1)}\left(
s\rho^{\delta}\right),v\rho^{\alpha}\right)={}\\
{}=\left(\sum\limits_{j=1}^N\fr{c_ja_j}{v\rho^{\alpha}-
s\rho^{\delta}+\mu_{1,r-1}^{(1)}\left(
s\rho^{\delta}\right)+a_j}-{}\right.\\
{}-
\left(s\rho^{\delta}-\mu_{1,r-1}^{(1)}\left(s\rho^{\delta}\right)\right)\times{}\\
{}\times
\left.\sum\limits_{j=1}^N\fr{a_jp_{0j}\left(v\rho^{\alpha}\right)}
{v\rho^{\alpha}-s\rho^{\delta}+\mu_{1,r-1}^{(1)}\left(s\rho^{\delta}\right)+a_j}\right)\times{}\\
\!{}\times
{\rho^{\alpha}}\!\!\Bigg / \!\!
\left(1-\sum\limits_{j=1}^N\fr{c_ja_j}{v\rho^{\alpha}-s\rho^{\delta}+
\mu_{1,r-1}^{(1)}\left(s\rho^{\delta}\right)+a_j}\times{}\right.\\
\left.{}\times
\sum\limits_{m=1}^rp_m\beta_m\left(s\rho^{\delta}-\mu_{1,r-1}^{(1)}\left(
s\rho^{\delta}\right)\right)
\vphantom{\sum\limits_{j=1}^N\fr{c_ja_j}{v\rho^{\alpha}-s\rho^{\delta}+
\mu_{1,r-1}^{(1)}\left(s\rho^{\delta}\right)+a_j}}
\right)\,.
\end{multline*}

Рассмотрим отдельно три случая.
\begin{enumerate}[1.]
\item $\alpha<{\gamma}/({\gamma-1}).$ В силу лемм~2--4
\begin{multline*}
\rho^{\alpha}\ \sum\limits_{k=1}^N a_k\omega_{rk}\left(s\rho^{\delta}-\mu_{1,r-1}^{(1)}\left(
s\rho^{\delta}\right),v\rho^{\alpha}\right)={}\\
{}=a\left(v-\fr{us^{\gamma}}{\rho_{r-1}^{\gamma}}+
o(1)\right)^{-1}\times{}\\
{}\times
\left(1-\fr{s\rho^{{\alpha}/{\gamma}}}{\rho_{r-1}}\left(\fr{u}{v}\right)^{{1}/{\gamma}}
\rho^{-{\alpha}/{\gamma}}+o(1)\right)={}\\
{}=\fr{a\left(1-({s}/({\rho_{r-1}}))
\left({u}/{v}\right)^{{1}/{\gamma}}\right)}
{v\left(1-({u}/{v})\left({s}/({\rho_{r-1}})\right)^{\gamma}\right)}
+o(1)\,.
\end{multline*}

\item $\alpha={\gamma}/({\gamma-1}).$ В~этом случае
\begin{multline*}
\rho^{\alpha}\ \sum\limits_{k=1}^N a_k\omega_{rk}\left(s\rho^{\delta}-\mu_{1,r-1}^{(1)}
\left(s\rho^{\delta}\right),v\rho^{\alpha}\right)={}\\
{}=
\left(1-\fr{s}{\rho_{r-1}}\,\psi^{-1}(v)+o(1)\right)\times{}
\\
{}\times a\left(v-\fr{s}{\rho_{r-1}}-\fr{us^{\gamma}}{\rho_{r-1}^{\gamma}}+o(1)\right)^{-1}={}\\
{}=
\fr{a\left(1-{s}/({\rho_{r-1}\psi(v)})\right)}{v-{s}/({\rho_{r-1}})-
{us^{\gamma}}/({\rho_{r-1}^{\gamma}})}+o(1)\,.
\end{multline*}

\item $\alpha>{\gamma}/({\gamma-1}).$ В~этом случае
\begin{multline*}
\rho^{\alpha}\ \sum\limits_{k=1}^N a_k\omega_{rk}\left(s\rho^{\delta}-\mu_{1,r-1}^{(1)}\left(s
\rho^{\delta}\right),v\rho^{\alpha}\right)={}
\end{multline*}

\noindent
\begin{multline*}
{}=
\fr{-a\rho^{\alpha-{\gamma}/({\gamma-1})}+o\left(\rho^{\alpha-{\gamma}/({\gamma-1})}\right)}
{({s}/{\rho_{r-1}})
\left(1+{s^{\gamma-1}u}/{\rho_{r-1}^{\gamma-1}}\right)}
\times{}\\
{}\times\left(1-\fr{s\rho^{{1}/({\gamma-1})}}{\rho_{r-1}}\,\fr{\rho^{1-\alpha}}{v}+
o\left(\rho^{{1}/({\gamma-1})+1-\alpha}\right)\right)={}\\
{}=
\fr{a}{v\left(1+{s^{\gamma-1}u}/{\rho_{r-1}^{\gamma-1}}\right)}+o(1)\,.
\end{multline*}
\end{enumerate}
Переходя в полученных соотношениях к пределу при $n\hm\rightarrow+\infty,$ 
получаем утверждение леммы.

\medskip

\noindent
\textbf{Теорема.} \textit{При $n\rightarrow +\infty$ существует предел}
$$
\lim{\sf P}\left(\rho^{\delta}w\left(t\rho^{-\alpha}\right)<x\right)=G(x,t)\,,\enskip x>0\,,\enskip t>0\,,
$$
\textit{где функция $G(x,t)$ определяется своими преобразованиями Лапласа и Лап\-ла\-са--Стилть\-еса}
\begin{multline*}
\int\limits_{0}^{\infty}\int\limits_{0}^{\infty}e^{-vt-sx}\,d_xG(x,t)={}\\
{}=
\begin{cases}
\fr{\left(1-({s}/{\rho^*_{r-1}})\left({u^*}/{v}\right)^{{1}/{\gamma}}\right)}
{v\left(1-({u^*}/{v})\left({s}/{\rho^*_{r-1}}\right)^{\gamma}\right)}\,,&\alpha<\fr{\gamma}{\gamma-1}\,;\\[9pt]
\fr{\left(1-{s}/({\rho^*_{r-1}\psi(v)})\right)}{v-{s}/{\rho_{r-1}^*}-
{u^*s^{\gamma}}/{(\rho^{*}_{r-1})^{\gamma}}}\,,& \alpha=\fr{\gamma}{\gamma-1}\,;\\[9pt]
\fr{1}{v\left(1+{s^{\gamma-1}u^*}/{(\rho_{r-1}^*)^{\gamma-1}}\right)}\,,& \alpha>\fr{\gamma}{\gamma-1}\,.
\end{cases}
\end{multline*}

\noindent
Д\,о\,к\,а\,з\,а\,т\,е\,л\,ь\,с\,т\,в\,о\,.\
Заметим, что
\begin{multline*}
\int\limits_{0}^{+\infty}e^{-vt}{\sf E}\left(\exp\left(-s\rho^{\delta}
w(t\rho^{-\alpha}\right)\right)\,dt={}\\
{}=
\sum\limits_{j=1}^N\rho^{\alpha}\omega_{j}(s\rho^{\delta},v\rho^{\alpha})\,.
\end{multline*}
Поэтому достаточно найти 
$\lim \sum\limits_{j=1}^N\rho^{\alpha}\omega_{j}(s\rho^{\delta},v\rho^{\alpha}).$

Из~(\ref{3}) и свойств функций $\mu_{c,r-1}^{(p)}(s)$ и 
$\alpha_c\left(z_{r-1,1}^{(k)}(s),\ldots,z_{r-1,1}^{(k)}(s),1\right)$ следует, что

\noindent
\begin{multline*}
\lim \sum\limits_{j=1}^N\rho^{\alpha}\omega_{j}(s\rho^{\delta},v\rho^{\alpha})=
\lim\sum\limits_{\nu=1}^Nc_{\nu}a_{\nu}^{-1}\times{}\\
{}\times\fr{\prod\limits_{q=1}^Na_q}{\alpha_1\left(z_{r-1,1}^{(k)}(0),\ldots,z_{r-1,1}^{(k)}(0),1\right)}\,
\rho^{\alpha}\times{}\\
{}\times 
\sum\limits_{j=1}^N\omega_{rj}\left(s\rho^{\delta}-\mu_{1,r-1}^{(1)}(s\rho^{\delta}),v\rho^{\alpha}\right)\,.
\end{multline*}
Но $\alpha_1\left(z_{r-1,1}^{(k)}(0),\ldots,z_{r-1,1}^{(k)}(0),1\right)\hm=a^{-1}
\prod\limits_{q=1}^Na_q.$ Следовательно,
\begin{multline*}
\lim \sum\limits_{j=1}^N\rho^{\alpha}\omega_{j}(s\rho^{\delta},v\rho^{\alpha})={}\\
{}=
\lim\rho^{\alpha}\sum\limits_{j=1}^N\omega_{rj}\left(s\rho^{\delta}-\mu_{1,r-1}^{(1)}(s\rho^{\delta}),v\rho^{\alpha}\right)\,.
\end{multline*}
Теперь из~(\ref{4}) и из лемм~4 и~5 получаем утверждение теоремы.

{\small\frenchspacing
{%\baselineskip=10.8pt
\addcontentsline{toc}{section}{Литература}
\begin{thebibliography}{9}

\bibitem{1-ush}
\Au{Ушаков А.\,В.} О~виртуальном времени ожидания в сис\-те\-ме с
относительным приоритетом и гиперэкспоненциальным входящим потоком~// 
Информатика и её применения, 2012. Т.~6. Вып.~1. С.~2--6.

\bibitem{2-ush}
\Au{Ушаков В.\,Г.} Система обслуживания с эрланговским входящим
потоком и относительным приоритетом~// Теория вероятн. и ее примен.,
1977. Т.~22. С.~860--866.

\bibitem{3-ush}
\Au{Матвеев В.\,Ф., Ушаков В.\,Г.} Системы массового обслуживания.~--- М.: МГУ, 1984.

\bibitem{4-ush}
\Au{Ушаков В.\,Г.} Аналитические методы анализа системы массового
обслуживания $GI|G_r|1|\infty$ с относительным приоритетом~// Вестн.
Моск. ун-та. Сер.~15. Вычисл. матем. и киберн., 1993. №\,4.
С.~57--69.


\label{end\stat}

\bibitem{5-ush}
\Au{Ушаков А.\,В., Ушаков В.\,Г.}  О длине очереди в системе с
абсолютным приоритетом и гиперэкспоненциальным входящим потоком~//
Вестн. Моск. ун-та. Сер.~15. Вычисл. матем. и киберн., 2012. №\,1.
С.~27--34.
 \end{thebibliography}
}
}


\end{multicols}