\def\stat{skachkov}

\def\tit{ОБ ИНТЕГРАЦИИ ГЕОГРАФИЧЕСКИХ МЕТАДАННЫХ ПОСРЕДСТВОМ 
РЕТРОСПЕКТИВНОГО ТЕЗАУРУСА$^*$}

\def\titkol{Об интеграции географических метаданных посредством 
ретроспективного тезауруса}

\def\autkol{Д.\,М. Скачков, О.\,Л.~Жижимов}
\def\aut{Д.\,М. Скачков$^1$, О.\,Л.~Жижимов$^2$}

\titel{\tit}{\aut}{\autkol}{\titkol}

{\renewcommand{\thefootnote}{\fnsymbol{footnote}}\footnotetext[1]
{Работа выполнена при поддержке РФФИ, грант №\,10-07-00302-а.}}


\renewcommand{\thefootnote}{\arabic{footnote}}
\footnotetext[1]{Институт вычислительных технологий СО РАН, danil.skachkov@gmail.com}
\footnotetext[2]{Институт вычислительных технологий СО РАН, zhizhim@sbras.ru}
       
      
  \Abst{Обсуждаются вопросы, связанные с построением интероперабельного тезауруса 
географических наименований, включающего геометрические данные географических 
объектов, в том числе и ретроспективные. Определяются основные требования к подобному 
тезаурусу, производится обзор существующих решений исходя из описанных требований, 
формулируются основные позиции соответствующего профиля для организации доступа к 
тезаурусу, приводится реляционная схема, предназначенная для хранения данных 
тезауруса.}
  
  \KW{географические метаданные; интеграция; ретроспективное геокодирование; 
тезаурус}
  
  
  \vskip 14pt plus 9pt minus 6pt

      \thispagestyle{headings}

      \begin{multicols}{2}

            \label{st\stat}


  \section{Введение}
  
  В настоящее время в связи с возрастающей потребностью общества в информационном 
обеспечении, в том числе и связанном с географическим аспектом информации, все большую 
актуальность приобретают разработки, направленные на интеграцию <<негеографических>> 
информационных сис\-тем с информационными сис\-те\-ма\-ми, изначально ориентированными на 
обработку географической информации. Под <<негеографическими>> 
информационными системами здесь и в дальнейшем будем понимать информационные 
системы, для которых изначально не предполагалось использование пространственных 
данных. К~таким сис\-те\-мам относятся, например, электронные биб\-лио\-те\-ки. Добавление 
географического аспекта к информации, хранящейся в таких системах, позволило бы 
существенно повысить функциональность их навигационных, поисковых и 
визаулизационных сервисов. Подобная интеграция даст возможность, к примеру, 
производить поиск по заданному географическому региону~[1], отображать на карте 
материалы, относящиеся к соответствующим точкам на поверхности Земли (как это делается 
на {Google Maps}), повысить релевантность результатов поиска.
  
  Следует заметить, что существующие в настоящее время программные комплексы, явным\linebreak 
образом не связанные с географическими информационными сис\-те\-ма\-ми (ГИС), не содержат необходимой функциональности по хранению и 
обработке географических данных. Наделение же их требуемой функциональностью 
осложняется отсутствием единых стандартов на поиск и представление данных, связанных с 
географическим аспектом, а также отсутствием четкого описания технологии интеграции как 
таковой~[2].
  
  Таким образом, разработка технологии, обеспечивающей обработку географического 
аспекта информации в <<негеографических>> информационных системах общего 
назначения, является актуальной и перспективной.

\vspace*{-6pt}
  
  \section{Пути интеграции географических данных}
  
  \vspace*{-1pt}
  
  Прежде чем описывать варианты внедрения геогра\-фи\-че\-ской информации в объекты 
информационной системы, разграничим две важные со\-став\-ля\-ющие любого объекта, 
характерного для рас\-смат\-ри\-ва\-емых систем.
  
  Вся информация об объекте может быть разделена на две составляющие:
  \begin{enumerate}[(1)]
\item контент~--- информационное наполнение объекта;\\[-14pt]
\item контекст~--- среда, в которой существует объект.
\end{enumerate}

  В дальнейшем будем считать, что географический аспект информации может быть 
зафиксирован на уровне метаданных, описывающих контент и контекст. При этом 
<<географические>> метаданные объекта могут быть заданы двумя способами:
  \begin{enumerate}[(1)]
\item с помощью геометрического описания географического объекта на основе 
координат;\\[-14pt]
\item с помощью ссылки на элемент некоторого тезауруса, включающего географические 
назва-\linebreak\vspace*{-12pt}

\pagebreak

\noindent
ния соответствующих объектов. Так как термин <<тезаурус>> может употребляться в 
различных значениях, в данной работе под тезаурусом будем понимать 
ин\-фор\-ма\-ци\-он\-но-поиско\-вый тезаурус. Ин\-фор\-ма\-ци\-он\-но-поиско\-вый 
тезаурус~--- это нормативный словарь, явно указывающий отношения между терминами и 
предназначенный для описания содержания документов и поисковых запросов~[3].
  \end{enumerate}
  
  Оба варианта в применении к задаче интеграции имеют как положительные, так и 
отрицательные стороны.
  
  \textit{Первый} вариант исключает неоднозначные толкования, но в то же время он не 
очень удобен~по причине необходимости внесения существенных изменений в уже 
существующие информационные системы. \textit{Второй} вариант не является 
однозначным, но может быть реализован на базе существующих парадигм информационных 
систем при условии их небольшой модернизации, а также\linebreak облада\-ет большей гибкостью. 
Хотя реализация тезауруса географических названий сопряжена с большим объемом работ, 
но возможность его повторного использования оправдывает все затраты. Более того, при 
реализации первого варианта интеграции тезаурус географических наименований также 
необходим для определения координат географических объектов, имеющих отношение к 
записям информационной системы. Поэтому в данной работе речь пойдет именно о втором 
варианте.
  
  \section{Препятствия при интеграции посредством тезауруса}
  
  Существует множество тезаурусов географических наименований, но сложность их 
использования применительно к данной задаче заключается в том, что географический 
аспект объектов, хранящихся в негеографических информационных системах, зачастую 
относится не к текущему моменту, а к моментам времени прошедшим. Однако с течением 
времени могут изменяться как географические названия, так и границы географических 
объектов. Будем называть это изменение свойств с течением времени 
\textit{ретроспективным аспектом информации}. В~то же время большинство тезаурусов 
содержит сведения, относящиеся только к текущему моменту времени, т.\,е.\ не учитывает 
ретроспективный аспект информации. Данная особенность препятствует использованию 
существующих тезаурусов географических наименований в подобных системах. 
  
  Следует заметить, что любые изменения географических названий и геометрических 
объектов, ассоциированных с ними, как правило, связаны с каким-либо нормативным 
документом.
  
  Более того, в существующих тезаурусах координаты географического объекта чаще всего 
задаются в виде точки, в то время как реальные координаты объекта представляют собой 
далеко не точку, а, в общем случае, некоторую область. Что, конечно же, также уменьшает 
полезность таких тезаурусов при проведении поиска. Поэтому более предпочтительным 
будет тезаурус, где положение объектов задано с помощью координат границ области, 
занимаемой объектом.
  
  Для задач поиска полезными будут также данные о том, как географические объекты 
расположены относительно друг друга. Например, если производится поиск по некоему 
региону, целесообразно считать релевантными также и элементы, относящиеся к 
географическим объектам, лежащим в целевом регионе.
  
  Таким образом, для использования в информационных системах общего назначения 
географического аспекта в его любом виде необходим справочный аппарат (тезаурус), 
который включает в себя не только географический аспект информации, но и ее временной 
(ретроспективный) аспект.
  
  В данной работе сделана попытка сформулировать основные требования к подобному 
тезаурусу географических названий, который мог бы удовлетворить потребности 
существующих информационных систем по обработке географического и исторического 
аспекта информации. В~работе приводится обзор некоторых схем данных, а также 
существующих тезаурусов, анализ их сильных и слабых сторон (в контексте применения к 
задаче привязки географических метаданных к объектам информационных систем). 
Формулируются требования к тезаурусу географических наименований, подходящему для 
использования в рамках задачи интеграции географических метаданных. Также приводится 
вариант реляционной схемы данных тезауруса.
  
  \section{Основные требования к~тезаурусу}
  
  На основе всего вышеизложенного сформулируем список требований к тезаурусу, 
подходящему для использования в рассматриваемой задаче.
  
  Тезаурус должен:
  \begin{enumerate}[(1)]
\item обеспечивать прямое и обратное геокодирование;
\item обеспечивать ретроспективное прямое и обратное геокодирование;
\item позволять включать информацию в технологию поиска в существующих 
информационных массивах;
\item содержать внутренние связи: 
\begin{itemize}
\item[(a)] по географическим объектам,
\item [(б)] по временным характеристикам,
\item[(в)] по документам;
\end{itemize}
\item быть представлен в схеме, максимально приближенной к какой-либо стандартной;
\item однозначно отображаться на другие схемы тезаурусов, в частности 
необходимо однозначное соответствие профилю {Zthes}~\cite{4-sk}, быть\linebreak 
может расширенному, для интеграции с существующими информационными 
системами.
\end{enumerate}

  Для пояснения сформулированных требований рассмотрим основные сценарии 
использования тезауруса географических наименований. Работа с тезаурусом включает два 
основных сценария: запрос координат геометрического примитива объекта по имени этого 
объекта и запрос всех имен объектов по заданному ко\-ор\-ди\-нат\-но-при\-вя\-зан\-но\-му 
гео\-мет\-ри\-че\-ско\-му примитиву. Обычно это называется \textit{прямым} и \textit{обратным 
геокодированием}. Для целевых систем, которые потенциально могут содержать 
ретроспективную информацию, отличительной особенностью становится необходимость 
указания времени, для которого соответствующее геокодирование будет актуальным. При 
этом отсутствие задания момента времени может служить указанием на использование 
текущего момента времени в качестве параметра запроса.
  
  Существенным моментом прямого геокодирования является тот факт, что заданные в 
запросе имя и время могут быть взаимно противоречивы.\linebreak Например, запрос на координаты 
объекта (Новосибирск, 1920-05-20) должен возвращать ответ (Новониколаевск, {координаты 
геометрического\linebreak примитива}, 1920-05-20). Запросы обратного геокодирования в этом 
смысле более просты, так как задаваемые в запросе координаты не связаны с действующей 
топонимикой.
  
  \section{Обзор существующих решений}
  
  Рассмотрим имеющиеся на данный момент схемы представления данных и существующие 
тезаурусы, которые могут представлять интерес в рамках этой задачи.
  
  При рассмотрении будем обращать внимание на следующие свойства:
  \begin{enumerate}[(1)]
\item наличие ретроспективных данных. Возможность извлечь данные, относящиеся к 
прошлому;\\[-14pt]
\item наличие связей с нормативными документами. Возможность определить, 
согласно какому документу было изменено название или координаты объекта;
\\[-14pt]
\item описание координат географического объекта сообразно его форме. Представление 
географического объекта не только в виде точки, а также в виде замкнутого контура, линии, 
композиции примитивов;\\[-14pt]
\item наличие связей, отражающих относительное расположение географических 
объектов.
\end{enumerate}

   В первую очередь рассмотрим существующие схемы данных.
   
   \vspace*{-6pt}

  \subsection{ГОСТ Р 52573-2006}
  
  Национальный стандарт Российской Федерации <<Географическая информация. 
Метаданные>>. Стандарт предназначен для специалистов в области информационных 
технологий, разработчиков геоинформационных систем, баз и банков пространственных 
данных, а также прикладных информационных систем различного назначения. Стандарт 
разработан в соответствии с правилами создания профилей, указанными в стандарте 
{ISO}~19115 [5].
  
  Данный стандарт содержит рекомендацию к использованию ретроспективных данных 
(Сущность {EX\_Extent}). Координаты объекта задаются с по\-мощью одной из 
сущностей:
  \begin{enumerate}[(1)]
\item {EX\_BoundingPolygon}~--- многоугольник (задается множеством точек);\\[-14pt]
\item {EX\_GeographicBoundingBox}~--- прямоугольная область (задается координатами 
углов);\\[-14pt]
\item {EX\_GeographicDescription}~--- описание объекта с использованием 
географического идентификатора.
\end{enumerate}
  
  Есть сведения о документе-источнике, но они привязаны к объекту в целом, а не к данным 
о его координатах и наименовании.
  
  Сведения о связях между объектами отсутствуют в данной схеме.
  
     \vspace*{-6pt}

  \subsection{CIDOC Conceptual Reference Model}
  
{Committee on Documentation Conceptual Reference Model} (CRM) не является схемой 
тезауруса, но\linebreak\vspace*{-12pt}

\pagebreak

\noindent
 рассматривается в данной работе, так как пред\-став\-ля\-ет собой формальную 
онтологию, предназначенную для улучшения интеграции и обмена гетеро\-генной 
информацией по культурному наследию.\linebreak Более конкретно, CIDOC CRM определяет 
семантику схем баз данных и структур документов, используемых в культурном наследии и 
музейной документации, в терминах формальной онтологии. Модель не определяет 
терминологию, появляющуюся в конкретных структурах данных, но имеет характерные 
отношения для ее использования.
  
  Модель может служить как руководством для разработчиков информационных систем, 
так и общим языком для экспертов предметной области и специалистов по информационным 
технологиям. Она предназначена для покрытия контекстной информации исторического, 
географического и теоретического характера об отдельных экспонатах и музейных 
коллекциях в целом~\cite{6-sk}.
  
  В {CIDOC CRM} представляет интерес сущность {E53\_Place} (место), которая 
как раз описывает географические метаданные объекта. Данная сущность является 
экземпляром {E44\_Place\_Appellation}. {E44\_Place\_Appellation} содержит данные 
о координатах, адресе, географическом наименовании. Координаты могут задаваться в 
любом виде (не только географические). Имеется возможность задать ссылки на 
родительский элемент (иерархические связи).
  
  Но в онтологии {CIDOC CRM} не учтено изменение свойств географических 
объектов с течением времени, отсутствует связь географических метаданных с 
нормативными документами.

   \vspace*{-6pt}
  
  \subsection{{Getty Thesaurus of Geographic Names}}
  
  Тезаурус географических имен института {Getty}~--- англоязычный тезаурус, 
содержащий более чем миллион географических имен, информацию о континентах, 
физических объектах, административных сущностях и нациях современного политического 
мира, а также сведения об исторически значимых областях~\cite{7-sk}.
  
  Схеме тезауруса {Getty}, естественно, присущи как положительные, так и 
отрицательные черты.
  
  Из отрицательных моментов можно отметить отсутствие информации об изменении 
координат географических объектов с течением времени. Координаты объекта могут быть 
представлены либо точкой, либо прямоугольником, что недостаточно для полного описания 
области на земной поверхности.
  
  В то же время в схеме данного тезауруса учтено изменение названия объекта с течением 
времени ({Term\_Date}). Также учтены нормативные документы 
({Subject\_Sources}) для данного объекта и для его наименований 
({Term\_Source})~\cite{8-sk}. Записи содержат данные об иерархии.
  
  \subsection{Тезаурус Российской государственной библиотеки}
  
  Справочник географических названий Российской государственной библиотеки 
содержит наименования географических объектов (городов, рек, и~т.\,д.) на территории 
Российской Федерации~\cite{9-sk, 10-sk}.
  
  Тезаурус не содержит ретроспективных данных в записях. Невозможно получить ни 
данных о предыду\-щих названиях, ни данных о предыдущих координатах объектов.
  
  В записях присутствуют ссылки на нормативные документы, определяющие 
наименование объекта.
  
  Координаты географических объектов заданы в виде координат точек, что не совсем 
соответствует действительности.
  
  В записях тезауруса также есть данные о взаиморасположении объектов.
  
  \subsection{Служба геокодирования {API} сервиса Google Maps}
  
  Позволяет определить координаты объекта, а также найти адрес, наиболее близкий к 
указанным координатам~\cite{11-sk}. 
  
  В записях, предоставляемых данной службой, отсутствуют ретроспективные данные. 
Отсутствуют связи с нормативными документами. 
  
  Координаты объектов указаны в виде точек.
  
  В записях содержатся иерархические связи.
  
  В то же время стоит отметить, что тезаурус содержит данные не только о крупных 
географических объектах, но также и об адресах. Есть возможность произвести обратное 
геокодирование. Но использовать службу геокодирования можно только вместе с картами 
{Google}, что делает невозможным использование сервиса в данной задаче.
  
  \subsection{Служба геокодирования {API} сервиса Яндекс.Карты}
  
  Имеет функциональность, аналогичную геокодеру {Google}~\cite{12-sk}. Обладает 
практически теми же достоинствами и недостатками, но из дополнительных достоинств 
можно выделить более обширную базу российских наименований географических объектов. 
Сервис обладает аналогичными ограничениями по использованию, что также делает 
невозможным использование сервиса в данной задаче.
  
  \subsection{Сравнительная таблица}
  
  По итогам сравнения составим сравнительную таблицу рассмотренных схем (табл.~1). 
  
   \begin{table*}\small
   \begin{center}
   \Caption{Сравнительная таблица схем тезаурусов}
   \vspace*{2ex}
   
   \tabcolsep=5.4pt
   \begin{tabular}{|l|c|c|c|c|}
   \hline
\multicolumn{1}{|c|}{Схема/тезаурус}&
\tabcolsep=0pt\begin{tabular}{c}Содержит\\ ретроспективные\\ сведения\end{tabular}&
\tabcolsep=0pt\begin{tabular}{c}Содержит\\ ссылки на\\ документы-источники\end{tabular}&
\tabcolsep=0pt\begin{tabular}{c}Координаты\\ географических\\ объектов\\ заданы\\ соответственно\\
 их размерам\\ и форме\end{tabular}&
 \tabcolsep=0pt\begin{tabular}{c}Наличие \\ иерархических\\ связей\end{tabular}\\
\hline
ГОСТ Р 52573-2006&$+$&$+$&$\pm$&$-$\\
\hline
{CIDOC CRM}&$-$&$-$&$\pm$&$+$\\
\hline
{Getty}&$+$&$+$&$\pm$&$+$\\
\hline
\tabcolsep=0pt\begin{tabular}{l}Российская государственная\\ библиотека\end{tabular}&$-$&$+$&$-$&$+$\\
\hline
Геокодер сервиса {Google Maps}&$-$&$-$&$-$&$+$\\
\hline
Геокодер сервиса Яндекс.Карты&$-$&$-$&$-$&$+$\\
\hline
\end{tabular}
\end{center}
\end{table*}

  \begin{figure*}[b]
  \vspace*{1pt}
 \begin{center}
 \mbox{%
 \epsfxsize=115.191mm
 \epsfbox{ska-1.eps}
 }
 \end{center}
 \vspace*{-9pt}
   \Caption{Онтология тезауруса}
   \end{figure*}
  
  Таким образом, проанализировав сущест\-ву\-ющие решения, приходим к выводу, что схемы 
тезауруса с необходимой функциональностью нет. Но есть достаточно близкие схемы, 
которыми можно руководствоваться при разработке собственного тезауруса. 
  
  \section{Разработка схемы тезауруса}
  
  На основании приведенных выше данных была построена онтология тезауруса, 
отвечающего сформулированным ранее требованиям (рис.~1)~\cite{13-sk}.
  

  
  Заметим, что правильно организованный тезаурус географических названий может 
служить основой и для получения информации, отличной от результатов прямого и 
обратного геокодирования, в~част\-ности:
  \begin{itemize}
\item информации о документах, связанных с конкретным географическим объектом;
\item информации о времени актуальности названий объектов;
\item информации о времени актуальности координат объектов;
\item информации о временных характеристиках производных параметров.
\end{itemize}

  Следует также заметить, что любой тезаурус является лишь дополнительной базой 
данных, которая может быть задействована при обработке запросов к различным 
информационным массивам. Ретроспективный тезаурус географических названий может 
быть задействован при обработке запросов, включающих ретроспективные географические 
названия.
  
  Можно выделить три вида условий в запросе к тезаурусу:
  \begin{enumerate}[(1)]
\item по имени;
\item по координатам;
\item по времени.
\end{enumerate}

  Перечисленные условия могут комбинироваться друг с другом. 
  
  Для интеграции с существующими информационными системами и обеспечения 
интероперабельности необходимо зафиксировать профиль доступа к обсуждаемому 
тезаурусу ({RGeoThes}). Этот профиль, несомненно, должен являться расширением 
профиля {ZThes}~\cite{4-sk} для доступа к тезаурусам по протоколам Z39.50 и 
{SRW/SRU} и включать необходимые компоненты для временного и географического 
поиска. При этом профиль должен определять:
  \begin{itemize}
\item схему данных;
\item структуру записи и наборы элементов;
\item обязательные и дополнительные индексы (точки доступа);
\item синтаксис поисковых запросов и поисковые атрибуты;
\item форматы представления данных;
\item протоколы доступа к ресурсу.
\end{itemize}

  \subsection{Протоколы доступа}
  
  Для обеспечения интероперабельности доступ к \mbox{RGeoThes} должен обеспечиваться 
по протоколам:
  \begin{itemize}
\item Z39.50;
\item HTTP/XML/SOAP/SRW;
\item HTTP/SRU.
\end{itemize}

  Каждый из указанных способов доступа имеет свои специфические особенности, которые 
должны быть определены общим профилем.
  
  \subsection{Форматы представления данных}
  
  В качестве основного обязательного формата представления записи \mbox{RGeoThes} для 
всех способов доступа является формат {XML}. Дополнительным необязательным 
форматом является {HTML}. Для доступа по Z39.50 обязательным форматом 
также является GRS-1. В~качестве дополнительных (необязательных) форматов могут 
использоваться \mbox{RUSMARC}, {MARC}21 и~др.
  
  \subsection{Схема данных}
  
  Схема данных определяется в терминах {XML} (XSD) и должна соответствовать 
онтологии, схематично представленной на рис.~1.
  
  \subsection{Индексы и~точки доступа}
  
  Точками доступа записи \mbox{RGeoThes} должны быть элементы, представленные в 
табл.~2.
  
  \subsection{Синтаксис поисковых запросов и~поисковые атрибуты}
  
  Для доступа по Z39.59 обязательным синтаксисом запросов должен являться 
RPN-1, необязательным~--- {CQL}. Для доступа по \mbox{SRW/SRU} обязательным 
синтаксисом запросов должен являться {CQL}, необязательным~--- RPN-1 
  ({x-pquery}).
   
   \begin{table*}\small
   \begin{center}
   \Caption{Точки доступа записи {RGeoThes}}
   \vspace*{2ex}
   
   \begin{tabular}{|l|c|c|c|}
   \hline
\multicolumn{1}{|c|}{Точка доступа}&Набор&Тип&Значение\\
\hline
Локальный номер&{utility}&1&4\\
Название терма&{cross-domain}&1&1\\
Квалификатор терма&{zthes-1}&1&1\\
Тип терма&{zthes-1}&1&2\\
Статус терма&{zthes-1}&1&7\\
Категория терма&{zthes-1}&1&6\\
Язык названия&{utility}&1&3\\
Дата начала действия названия&{cip-1}&&\\
Дата окончания действия названия&{cip-1}&1&2073\\
&&2&14, 15, 16, 17, 18\\
Документ, фиксирующий название&{cross-domain}&1&6\\
Тип геометрического объекта&{cip-1}&4&201, 202\\
Координаты геометрического объекта&{cip-1}&1&2059, 2060\\
&&2&7, 8, 9, 10\\
Дата начала действия определения геометрии&{cip-1}&1&2072\\
&&2&14, 15, 16, 17, 18\\
Дата окончания действия определения геометрии&{cip-1}&1&2073\\
&&2&14, 15, 16, 17, 18\\
Документ, фиксирующий определения геометрии&{cross-domain}&1&6\\
Комментарий&{cross-domain}&1&4\\
Идентификатор связанного терма&{zthes-1}&1&4\\
\hline
  \end{tabular}
  \end{center}
  \end{table*}
  
  Поисковые атрибуты {RPN} для доступа по Z39.50 для обеспечения 
интероперабельности должны соответствовать поисковым атрибутам профиля \mbox{Z-Thes} 
из наборов {zthes-1}, {utility}, {cross-domain} (\mbox{xd-1}). Для поиска по 
времени и координатам должны использоваться атрибуты из набора {cip-1}. 
Аналогичное требование справедливо и для запросов~{CQL}.
  
  Соответствие поисковых атрибутов точкам доступа приведено в табл.~2~\cite{15-sk}.
  
  \section{Реляционная схема данных тезауруса}
  
  Рассмотрим вариант схемы для хранения записей тезауруса в случае использования 
реляционной СУБД {PostgreSQL} в качестве хранилища данных. Учитывая все 
вышеперечисленное, построим реляционную схему данных. Схема представлена на рис.~2.
  
  Основной в данной схеме является таблица <<\textbf{Запись тезауруса}>> (далее~--- 
главная таблица), в которой находится список квалификаторов записей тезауруса. Строка из 
данной таблицы может содержать ссылку на предыдущий вариант записи и на родительскую 
запись. Связи между записями тезауруса содержатся в таблице <<\textbf{Связь между 
записями}>>. Каждая связь содержит квалификаторы двух записей тезауруса, которые она 
связывает. Также связь характеризуется двумя документами (что представлено в виде 
внешних ключей). <<Начальный документ>> определяет документ, в котором 
зафиксировано появление связи. <<Конечный документ>>, который может быть не указан, 
определяет документ, в котором зафиксировано исчезновение связи.
  
  \begin{figure*} %fig2
    \vspace*{1pt}
 \begin{center}
 \mbox{%
 \epsfxsize=164.503mm
 \epsfbox{ska-2.eps}
 }
 \end{center}
 \vspace*{-9pt}
   \Caption{Реляционная схема данных тезауруса}
  \end{figure*}
  
  В таблице <<\textbf{Имя объекта}>> задаются наименования географических объектов, 
содержащихся в тезаурусе. Каждая из записей главной таблицы может быть связана с 
несколькими строками имен. В~свою очередь, имя может быть связано только с одной 
записью из главной таблицы. Каждое из имен характеризуется собственно именем, а также 
типом объекта и языком. Под типом объекта понимается, например, тип населенного пункта. 
Каждая запись таблицы <<Имя объекта>> содержит идентификаторы двух документов~--- 
начального и конечного. <<Начальный документ>> определяет документ, в котором 
зафиксировано присвоение данного имени географическому объекту. <<Конечный 
документ>> определяет документ, в котором зафиксировано окончание срока действия 
данного имени.
  
  Таблица <<\textbf{Местоположение объекта}>> содержит данные о координатах 
географических объектов тезауруса. Каж\-дая из записей главной таблицы может быть 
связана с несколькими строками данной таб\-ли\-цы. В~то же время запись таб\-ли\-цы 
<<Местоположение объекта>> может быть связана только с одной\linebreak строкой из главной 
таблицы. Каждая запись таб\-ли\-цы <<Местоположение объекта>> содержит идентификаторы 
двух документов~--- начального и конечного. <<Начальный документ>> определяет\linebreak 
документ, в котором зафиксировано присвоение данного мес\-то\-по\-ло\-же\-ния географическому 
объекту. <<Конечный документ>> определяет документ, в котором зафиксировано 
окончание срока действия местоположения применительно к данному объекту. Также 
каждая запись содержит поле <<тип местоположения>>, содержащее идентификатор типа 
местоположения объекта. Таким типом может быть точка, прямоугольник, многоугольник, 
линия, регион и прочие. Благодаря использованию отображения <<Одна ие\-рар\-хия\,--\,од\-на 
таблица>> для набора типов местоположений объекта становится возможным легко 
добавлять новые типы местоположения в уже работающую схему. Для хранения 
координатных данных используются поля <<Точка>>, <<Прямоугольник>>, 
<<Многоугольник>>, <<Линия>>, <<Регион>> с типами данных {point\_type}, 
{rectangle\_type}, {polygon\_type}, {line\_type}, {circle\_type} соответственно. 
Типы данных для этих полей являются композитными типами, содержащими всю 
координатную информацию, характерную для представления. Например, тип данных 
{rectangle\_type}, соответствующий прямоугольной области на поверхности Земли, 
содержит поле {rect} встроенного типа {box}. Данное разделение на типы для 
различных видов географических объектов сделано в целях повышения гибкости схемы.
  
  Таблица <<\textbf{Документ}>> содержит данные о документах, регистрирующих 
изменение характеристик объектов с течением времени. Каждый документ содержит 
описание, уникальный идентификатор ресурса ({URI}), дату создания и дату вступления 
в силу. Именно датой вступления в силу документов определяются временные рамки 
существования той или иной характеристики географического объекта.
  
  \section{Заключение}
  
  В данной работе была показана необходимость реализации информационно-поискового 
тезауруса географических наименований в рамках задачи интеграции географических 
метаданных в информационные системы общего назначения.
  
  Изложенные выше основные положения процесса интеграции, а также организации 
ретроспективного геокодирования и соответствующего тезауруса географических названий 
будут в дальнейшем использованы для построения модели информационной системы с 
возможностями геометрического и ретроспективного поиска информации. Поиск 
предполагается организовать на основе картографических интерфейсов в соответствии с 
описанным выше профилем.
  
  Приведен вариант реляционной схемы для хранения данных тезауруса. Схема в настоящее 
время используется для хранения данных, собираемых в тезаурус в рамках данной работы. 
В~дальнейшем планируется реализация доступа к тезаурусу на основе протоколов 
SRW/SRU, Z39.50, {HTTP}, а также экспериментальная интеграция 
метаданных в работающую систему.

{\small\frenchspacing
{%\baselineskip=10.8pt
\addcontentsline{toc}{section}{Литература}
\begin{thebibliography}{99}
  
\bibitem{1-sk}
\Au{Жижимов О.\,Л., Мазов Н.\,А.}
География и стандарты метаданных для электронных библиотек: содержание, 
применение, проблемы~// Электронные библиотеки, 2009. Т.~12. №\,1. {\sf 
http://www.elbib.ru/\linebreak index.phtml?page=elbib/rus/journal/2009/part1/\linebreak ZM}.

\bibitem{2-sk}
\Au{Жижимов О.\,, Мазов Л.\,А.}
Проблемы географической привязки цифровых объектов в электронных библиотеках~// 
Электронные библиотеки: перспективные методы и технологии, электронные коллекции 
(RCDL'2010): Труды XII Всеросс. научн. конф.~--- Казань: КГУ, 2010. С.~207--214.

\bibitem{3-sk}
\Au{Лукашевич Н.\,В.}
Тезаурусы в задачах информационного поиска.~--- М.: Изд-во Московского ун-та, 2011.

\bibitem{4-sk}
The Zthes specifications for thesaurus representation, access and navigation. {\sf 
http://zthes.z3950.org}.

\bibitem{5-sk}
ГОСТ Р 7.24-2007. Тезаурус Ин\-фор\-ма\-ци\-он\-но-по\-иско\-вый многоязычный. Состав, 
структура и основные требования к построению.~--- М.: Стандартинформ, 2006.

\bibitem{6-sk}
Онтология в области документации в сфере культурного наследия: CIDOC CRM. {\sf 
http://\linebreak www.intuit.ru/department/expert/ontoth/5}.

\bibitem{7-sk}
Getty Thesaurus of Geographic Names$^\registered$ Online. {\sf 
http://\linebreak www.getty.edu/research/tools/vocabularies/tgn/index.\linebreak html}.

\bibitem{8-sk}
Contribute to the Getty Vocabularies. {\sf 
http://www.\linebreak getty.edu/research/tools/vocabularies/contribute.html}.

\bibitem{9-sk}
Тезаурус РГБ. {\sf http://aleph.rsl.ru/F/?func=file\&file\_\linebreak name=find-b\&local\_base=tst11}.

\bibitem{10-sk}
\Au{Лаврёнова О.\,А.}
Многоязычный доступ к данным на основе тезауруса географических названий~// 
Электронные библиотеки: перспективные методы и технологии, электронные коллекции 
(RCDL'2007):\linebreak
 Труды IX Всеросс. научн. конф.~--- Пе\-ре\-славль-За\-лес\-ский: Ун-тет 
г.~Переславля, 2007. С.~57--62.

\bibitem{11-sk}
Геокодирование~--- Службы API Карт Google. {\sf 
http:// code.google.com/intl/ru/apis/maps/documentation/\linebreak geocoding}.

\bibitem{12-sk}
Поиск по карте~--- Яндекс.Карты. {\sf http://api.\linebreak yandex.ru/maps/geocoder}.

\bibitem{13-sk}
\Au{Соловьев В.\,Д., Добров Б.\,В., Иванов~В.\,В., Лукашевич~Н.\,В.}
Онтологии и тезаурусы: Учебное пособие.~--- Казань, М., 2006.

%\bibitem{14-sk}
%The Zthes specifications for thesaurus representation, access and navigation. {\sf 
%http://zthes.z3950.org}.

\label{end\stat}

\bibitem{15-sk}
Catalogue Interoperability Protocol (CIP) Specification~--- Release~B~// 
CEOS/WGISS/ICS/CIP-B. April 2005. Issue~2.4.75.

  
 \end{thebibliography}
}
}


\end{multicols}