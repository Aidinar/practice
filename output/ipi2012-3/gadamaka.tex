\def\stat{gaidamaka}

\def\tit{ЗАДАЧИ ОПТИМАЛЬНОГО ПЛАНИРОВАНИЯ МЕЖУРОВНЕВОГО 
ИНТЕРФЕЙСА В БЕСПРОВОДНЫХ СЕТЯХ$^*$}

\def\titkol{Задачи оптимального планирования межуровневого 
интерфейса в беспроводных сетях}

\def\autkol{Ю.\,В.~Гайдамака, Т.\,В.~Ефимушкина, А.\,К.~Самуйлов, 
К.\,Е.~Самуйлов}
\def\aut{Ю.\,В.~Гайдамака$^1$, Т.\,В.~Ефимушкина$^2$, А.\,К.~Самуйлов$^3$, 
К.\,Е.~Самуйлов$^4$}

\titel{\tit}{\aut}{\autkol}{\titkol}

{\renewcommand{\thefootnote}{\fnsymbol{footnote}}\footnotetext[1]
{Работа выполнена при частичной поддержке РФФИ (гранты 10-07-00487-a и 12-07-00108)
и Рособразования 
(проект 020619-1-174).}}

\renewcommand{\thefootnote}{\arabic{footnote}}
\footnotetext[1]{Российский университет дружбы народов, кафедра систем телекоммуникаций, ygaidamaka@sci.pfu.edu.ru}
\footnotetext[2]{Российский университет дружбы народов, кафедра систем телекоммуникаций, tefimushkina@gmail.com}
\footnotetext[3]{Российский университет дружбы народов, кафедра систем телекоммуникаций, asam1988@gmail.com}
\footnotetext[4]{Российский университет дружбы народов, кафедра систем телекоммуникаций, ksam@sci.pfu.edu.ru}
 
 
   \Abst{В данном обзоре проведено исследование современного состояния задач оптимального 
планирования межуровневого интерфейса на базе механизма мультиплексирования с 
ортогональным частотным разделением (OFDM, Orthogonal Frequency Division Multiplexing) для 
нисходящего канала в сетевой технологии LTE (Long-Term Evolution). При этом рассматривается 
понятие межуровневой оптимизации, подробно описаны оптимизационные задачи и ограничения, 
возникающие при разделении радиоресурсов в нисходящем канале, дан краткий обзор 
планировщиков и соответствующих им функций полезности, определяющих уровень 
удовлетворенности пользователей схемой распределения радиоресурсов при заданных 
ограничениях.}

%\vspace*{2pt}
   
   \KW{технология OFDM; межуровневая оптимизация; функция полезности; планировщик; 
эффективное распределение частот}

%\vspace*{6pt}

\vskip 14pt plus 9pt minus 6pt

      \thispagestyle{headings}

      \begin{multicols}{2}

            \label{st\stat}
   
\section{Введение}
  
  К основным задачам в беспроводных сетях относится оптимизация распределения 
ограниченного числа радиоресурсов между пользователями. Различные типы пакетного 
трафика, передаваемого по сети, предполагают динамическое выделение ресурсов 
пользователям. Решением задач планирования ресурсов, назначения приоритетов доступа 
в\linebreak зависимости от типа трафика с заданными требованиями к качеству обслуживания (QoS, 
Quality of\linebreak Service) занимаются модули управления радиоресурсами, называемые 
планировщиками (англ.\ \textit{schedulers}).
  
  Динамичное изменение загруженности канала\linebreak в беспроводной сети определяет 
требования к планиров\-щику, одним из которых является меж\-уров\-не\-вый (англ.\ 
\textit{crosslayer}) подход к решению\linebreak задачи оптимального распределения ресурсов. 
Основным принципом межуровневой оптимизации является комплексное решение задачи 
эффективного использования ограниченного числа радиоресурсов, учитывающее ряд 
первостепенных факторов: повышение пропускной способности; обеспечение 
равнодоступности~--- справедливого (англ.~\textit{fair}) разделения ресурсов между 
пользователями; достижение требуемого или, по крайней мере, наилучшего возможного 
качества обслуживания~[1].
  
  В обзоре в общем виде сформулированы основные задачи оптимизации, возникающие 
при планировании ресурсов в беспроводных сетях в целях повышения эффективности 
работы сети с большим числом несущих, сети, построенной на базе механизма OFDM, 
характерного для нисходящего канала в технологии LTE~[2, 3]. Вводится понятие 
межуровневой оптимизации, подробно описаны оптимизационные задачи и ограничения, 
возникающие при разделении ресурсов в нисходящем канале. Исследованы два алгоритма 
межуровневой оптимизации, предназначенных для максимизации функции полезности в 
различных условиях~--- алгоритм динамического назначения поднесущих DSA (Dynamic 
Subcarrier Assignment) и алгоритм адап\-тив\-но\-го распределения мощности APA (Adaptive 
Power Allocation). Для рассматриваемых алгоритмов сформулированы задачи 
максимизации функции полезности и получены их решения.

\section{Виды межуровневой оптимизации}

  С точки зрения терминологии межуровневая оптимизация заключается в объединении 
нескольких уровней модели взаимодействия открытых сис\-тем (OSI, Open Systems 
Interconnection) для полу\-чения более качественных решений и \mbox{эффективных} алгоритмов 
без лишних межуровневых обменов информа\-ции. Межуровневый подход к решению 
задачи оптимального распределения ресурсов позволяет в динамическом режиме учесть 
изменения типов трафика в беспроводной сети, потребности в услугах, значений 
параметров в канале связи и мобильность абонентов.
  
  Выделим три основных вида межуровневой оптимизации. Главной задачей 
  ка\-наль\-но-ориен\-ти\-ро\-ван\-но\-го вида межуровневой оптимизации является 
эффективное использование ограниченного числа изменяющихся во времени 
радиоресурсов с целью обеспечения высокой пропускной способности, заданных 
требований к качеству и равнодоступности. С~точки зрения модели OSI данная задача 
рассматривается между физическим и канальным уровнями.
  
  Второй вид, ориентированный на качество пред\-остав\-ле\-ния услуг абоненту <<из конца 
в конец>>, решает задачи адаптации протоколов верхних уровней к нестабильным, 
изменяющимся во времени канальным ресурсам для достижения заданных требований к 
QoS параметрам, например к производительности и задержкам. Заметим, что 
эффективность функционирования протокола TCP в беспроводных сетях связи~--- одна из 
типичных задач данного вида оптимизации~[4--6].
  
  Выбор наилучшего маршрута определяет третий вид межуровневой оптимизации, 
рассматриваемый в~[7--9]. При этом поиск наиболее эффективного маршрута происходит 
с учетом сетевого и физического или канального уровней. Далее в статье рассматриваются 
задачи оптимизации только с точки зрения ка\-наль\-но-ориен\-ти\-ро\-ван\-но\-го 
межуровневого подхода.

\section{Задачи оптимизации}

\subsection{Задача минимизации мощности}
  
  В~[10] сформулирована задача минимизации общей выделяемой пользователям сети 
мощ\-ности,\linebreak учитывающая распределение поднесущих с определением числа бит и уровня 
выделяемой мощ\-ности для каждой из поднесущих на основе мгновенных 
(\textit{instantaneous}) характеристик состояния\linebreak канала, измеренных для каждого из 
пользователей сети. В~рамках данной задачи предложен и реализован итерационный 
алгоритм распределения поднесущих между пользователями, а также обобщенный 
алгоритм задания числа бит и уровня мощ\-ности для поднесущих, назначенных 
пользователям. В~[10] рассматривается сеть связи с $K$ пользователями, в которой 
  $k$-поль\-зо\-ва\-тель имеет скорость передачи, равную $R_k$ бит на OFDM-сим\-вол 
(далее бит/символ), $k\hm=\overline{1,K}$. На передатчике реализован алгоритм 
назначения $n$-под\-не\-су\-щей $k$-поль\-зо\-ва\-те\-лю, после применения которого\linebreak на 
основе характеристик состояния канала для\linebreak $k$-поль\-зо\-ва\-те\-ля применяется 
обобщенный алгоритм задания $c_{k,n}$ чис\-ла бит/символ для $n$-под\-не\-су\-щей 
(здесь и далее $n\hm=\overline{1,N}$).
  
  В зависимости от числа назначенных $c_{k,n}$ бит из множества 
$\mathcal{D}=\{0,1,2,\ldots ,M\}$, где $M$~--- максимально возможное для передачи 
число бит/символ, адаптивный модулятор выбирает соответствующую схему модуляции, 
при этом уровень выделяемой мощности адаптируется согласно обобщенному алгоритму. 
Заметим, что $n$-поднесущая предоставляется только одному пользователю, т.\,е.\ при 
$c_{k^\prime,n}\not=0$, $c_{k,n}=0$ для всех $k\not=k^\prime$.
  
  В частотно-селективном канале с замираниями $n$-поднесущая характеризуется 
уровнем мощ\-ности сигнала $a_{k,n}$ по отношению к $k$-пользователю. При этом 
дисперсия уровня спектральной плот\-ности шума $\sigma_{k,n}^2$ принята равной 
единице для всех поднесущих. Для поддержания требуемого качества услуги на 
приемнике выделяемая мощность для передачи $k$-пользователю на $n$-поднесущей 
рассчитывается по формуле 
$$
P_{k,n}=\fr{f_k(c_{k,n})}{a_{k,n}^2}\,,
$$ 
где $f_k(c_{k,n})$~--- 
требуемая мощность для приема данных и их последующей демодуляции. Таким образом, 
задача минимизации мощности представляется в виде 
$$
P^*=\min\limits_{c_{k,n}\in\mathcal{D}}\sum\limits_{n=1}^N \sum\limits_{k=1}^K 
\fr{f_k(c_{k,n})}{a_{k,n}^2}
$$ 
c ограничением для $k$-поль\-зо\-ва\-те\-ля по числу бит для передачи 
$R_k=\sum\limits_{n=1}^N c_{k,n}$.
  
  Исследованный в~[10--12] алгоритм задания чис\-ла бит и уровня мощности в сети с 
одним пользователем служит основой для решения задачи минимиза\-ции мощ\-ности для 
случая многопользовательской сети. Этот алгоритм относится к так называемым 
<<жад\-ным>> алгоритмам и назначает бит под\-не\-су\-щей, требующей выделения 
наименьшей дополнительной мощности. Процесс назначения происходит по одному биту 
за один раз до тех пор, пока $R$ бит не будут распределены между $N$ поднесущими.
  
  Решение оптимизационной задачи для случая многопользовательской сети 
предусматривает использование действительных значений для числа бит/символ 
$c_{k,n}\in \mathbb{R}[0,\,M]$, а также введение функции назначения $k$-поль\-зо\-ва\-те\-лю 
  $n$-под\-не\-су\-щей, $\rho_{k,n}\in \mathbb{R}[0,\,1]$. Тогда задача минимизации 
мощности принимает вид:
$$
p^*=\min\limits_{c_{k,n},\rho_{k,n}} \sum\limits_{n=1}^N 
\sum\limits_{k=1}^K \fr{\rho_{k,n} f_k(c_{k,n})}{a_{k,n}^2}
$$
c ограничениями $\sum\limits_{n=1}^N \rho_{k,n}c_{k,n}=R_k$ и $\sum\limits_{k=1}^K 
\rho_{k,n}=1$. Данное предположение позволяет решить задачу назначения поднесущей 
  $k$-пользователю методом множителей Лагранжа по алгоритму, приведенному в~[10]. 
В~рамках решения задачи оптимизации полученные величины определяют нижнюю 
границу искомого минимального значения выделяемой мощности. Однако из-за 
принятого ранее предположения о величинах $c^*_{k,n}\not\in \mathbb{Z}$ и 
$\rho_{k,n}^*\in \mathbb{R}[0,\,1]$ предполагается разделение поднесущей между 
несколькими пользователями. Решение данной проблемы методом квантования 
полученных величин может не удовлетворять требованию $k$-пользователя к скорости 
передачи~$R_k$. В~предположении, что $\rho_{k^\prime,n}^*=1$, 
$k^\prime\hm=\mathrm{arg}\,\max\limits_k \rho_{k,n}^*$ и $\rho^*_{k,n}=0$, 
$k\not=k^\prime$, алгоритм назначения $n$-поднесущей в~[10] дополняется алгоритмом 
задания числа бит и уровня мощности в сети с одним пользователем. В~результате в~[10] 
предложен многопользовательский адаптивный алгоритм (MAO, Multiuser Adaptive 
OFDM).
  
  Данный поход к решению задачи оптимизации, согласно~\cite{13-gai}, относят к 
методу релаксаций. Использование нецелого числа бит и разделения поднесущей между 
пользователями позволяет эффективно решать задачу оптимизации, однако требует 
применения дополнительных процедур для получения целых величин, являющихся 
целесообразными с точки зрения функционирования сети. Двумя другими методами, 
предложенными в~\cite{13-gai}, являются разбиение задачи на несколько более прос\-тых и 
эвристический алгоритм. Первый метод предполагает определение числа поднесущих для 
$k$-поль\-зо\-ва\-те\-ля с учетом требований к скорости передачи $R_k$ и далее назначение 
конкретных, выбранных по некоторому алгоритму, поднесущих. Эвристический подход 
основывается на методе сортировки и представляет собой реализацию двухэтапного 
аналитического метода, описанного выше. Решение задачи оптимизации с помощью 
эвристического метода представлено также в~\cite{14-gai}.

\subsection{Задача максимизации скорости передачи}

  В~[15] нелинейная оптимизационная задача преобразована в линейную задачу 
максимизации скорости передачи путем равномерного разделения общей мощности 
$p_{tot}$ между пользователями в сети для каждой из поднесущих: 

\noindent
$$
p_{k,n}=\fr{p_{tot}}{N}\,.
$$ 
Задача максимизации общей пропускной способности сети представлена в виде:

\noindent
$$
R^*=\max\limits_{c_{k,n}\in\mathcal{D}} \sum\limits_{n=1}^N \sum\limits_{k=1}^K 
c_{k,n} \rho_{k,n}\,,
$$ 
принимая во внимание требование $r_k$, предъявляемое $k$-поль\-зо\-ва\-те\-лем 
к минимальной ско\-рости передачи чис\-ла бит на один OFDM-сим\-вол, 

\noindent
$$
\sum\limits_{n=1}^N c_{k,n}\rho_{k,n}\geq r_k\,.
$$
  
  Фиксируя уровень выделяемой мощности $p_{k,n}$ и допуская, что значения 
заданного для $k$-поль\-зо\-ва\-те\-ля коэффициента ошибок BER (Bit Error Rate) и состояния 
канала $a_{k,n}$ известны на базовой станции для всех пользователей, находятся 
значения числа бит/символ 

\noindent
$$
c_{k,n}=f(\mathrm{BER}, p_{k,n}, a_{k,n})\,.
$$

Данный подход позволяет 
эффективно решить линейную задачу оптимизации методом це\-ло\-чис\-лен\-но\-го линейного 
программирования, однако предусматривает экспоненциальный рост уровня сложности с 
увеличением числа поднесущих и пользователей в сети.
  
  В~\cite{15-gai} предложен алгоритм понижения слож\-ности, состоящий из двух этапов: 
назначение под-\linebreak несущих пользователям с наибольшим воз\-мож-\linebreak ным числом бит для 
передачи и перераспределение поднесу\-щих для соблюдения требований~$r_k$. 
%
Поднесущие на первом этапе распределяются между пользователями с целью достижения 
максимальной пропускной способности без учета требований~$r_k$ к минимальной 
скорости передачи. 
%
Для перераспределения пользователей на втором этапе требуется 
соблюдение следующих условий:
  \begin{enumerate}[(1)]
\item выделенная на первом этапе $k_n^*$-поль\-зо\-ва\-те\-лю $n$-поднесущая не может 
быть переназначена другим пользователям, если переназначение повлечет возможное 
нарушение требования~$r_k$ к минимальной скорости передачи 
$k_n^*$-поль\-зо\-ва\-те\-ля, $R_{k_n^*,n}\hm-c_{k_n^*,n}\hm<r_{k_n^*}$;
\item каждое переназначение поднесущих должно минимально сокращать общую 
пропускную способность сис\-темы;
\item число переназначений должно быть наименьшим.
\end{enumerate}

  Для выполнения последних двух условий вводится функция $e_{k,n}= (c_{k_n^*,n}-
c_{k,n})/c_{k,n}$ оцен-\linebreak ки переназначения $n$-поднесущей $k$-пользователю.\linebreak
 Согласно 
алгоритму~[15] перераспределение происходит поочередно для всех пользователей в сети, 
которым назначены поднесущие после первого этапа, не удовлетворяющие требованиям 
по скорости передачи. При этом для $k$-поль\-зо\-ва\-те\-ля выбирается поднесущая с 
наименьшей функцией оценки переназначения. Перераспределение $k$-поль\-зо\-ва\-те\-ля на 
$n^\prime$-под\-не\-су\-щую происходит, если $R_{k^*_{n^\prime},n^\prime}-
c_{k^*_{n^\prime},n}\geq r_{k^*_{n^\prime}}$. В~противном случае выбирается другая 
поднесущая с минимальной функцией перераспределения.

\subsection{Задача обеспечения равнодоступности}

  В~[16] приведены три из наиболее известных схем распределения поднесущих между 
пользователями. 

Согласно первой из них обеспечение максимальной пропускной 
способности (maxBR, maximum bit-rate) достигается за счет предоставления\linebreak 
  $n$-поднесущей $k$-пользователю, находящемуся в лучших канальных условиях, т.\,е.\ 
обладающему наибольшим частотным откликом (англ.\ \textit{frequency response}) 
$H_{k,n}$. Следует отметить, что данный метод не решает задачу обеспечения 
равнодоступности. Однако, рассматривая величину частотного отклика в качестве 
единственного параметра при распределении поднесущих, метод maxBR определяет 
верхнюю границу возможной скорости передачи данных.
  
  Вторая схема распределения канальных ресурсов, изложенная в~[17], предполагает 
решение задачи обеспечения равнодоступности путем предо\-став\-ле\-ния одинаковой 
скорости передачи всем пользователям. Помимо ограничения по мощности передачи 
$\sum\limits_{k=1}^K \sum\limits_{n=1}^N p_{k,n}\leq p_{\mathrm{tot}}$ в~[17] также вводится 
пропорциональное ограничение: 
$$
N_1:N_2:\ldots:N_K=\phi_1:\phi_2:\ldots\phi_K\,,
$$ 
где 
$N_k\hm=\phi_kN$~--- чис\-ло поднесущих, назначенных $k$-поль\-зо\-ва\-те\-лю, и $\phi_k$~--- 
нормированная пропорциональная постоянная скорости передачи данных для 
  $k$-поль\-зо\-ва\-те\-ля. При этом $\tilde{n}=N/K$ определяет максимально возможное чис\-ло 
выделяемых пользователю поднесущих.
  
  Третья схема назначения поднесущих, предложенная в~[18], заключается в выборе 
$\tilde{n}$ поднесущих с наибольшими значениями частотных откликов для пользователя. 
Данная процедура повторяется для всех пользователей в сети.
  
  Введем величину отношения сигнал-шум (SNR, Signal-to-Noise Ratio), используемую 
для постановки оптимизационной задачи:
$$
\mathrm{SNR}_{k,n}=\fr{\vert H_{k,n}\vert^2 
p_{k,n}}{\sigma_{k,n}^2}\,.
$$ 
Задача максимизации общей пропускной способности всех 
пользователей в сети формулируется в виде:
$$
F^*=\max\limits_{\mathrm{SNR}_{k,n}} 
\sum\limits_{k=1}^K \sum\limits_{n=1}^N f(\mathrm{SNR}_{k,m})\,.
$$
Следует отметить, что подобные 
задачи не учитывают улучшения пропускной способности отдельных пользователей. 
В~[16] приводится решение данной оптимизационной задачи с учетом обеспечения 
рав\-но\-до\-ступ\-ности путем назначения равного чис\-ла поднесущих по одному из двух 
алгоритмов, кратко охарактеризованных ниже, а далее путем использования обобщенного 
алгоритма задания чис\-ла бит/символ для каждой поднесущей.
  
  Первый алгоритм заключается в сравнении и выборе поднесущей с наибольшим 
частотным откликом, а также ее назначении пользователю. В~ходе данного назначения 
пользователь, получивший оптимальное число поднесущих, удаляется из 
рассматриваемого множества. Данная процедура продолжается для всех оставшихся 
пользователей и поднесущих. 

Второй алгоритм заключается в поиске для первого 
пользователя максимального частотного отклика и назначении ему соответствующей 
поднесущей. После выделения по одной поднесущей каждому из $K$ пользователей 
данный алгоритм повторяется в противоположном порядке: от $K$-го до 1-го 
пользователя.

\subsection{Задача максимизации полезности}

  Следует отметить, что все рассмотренные выше задачи подразумевают оптимизацию 
некоторой функции полезности, описывающей тот или иной уровень удовлетворенности 
пользователей для определенной схемы распределения радиоресурсов при некоторых 
ограничениях. Тем не менее, как будет показано ниже, оптимизация полезности может 
оказаться задачей, представляющей самостоятельный интерес.
  
  В~[19, 20] предложены два алгоритма межуровневой оптимизации, предназначенные для 
максимизации функции полезности (Utility Function) в различных условиях~--- алгоритм 
динамического назначения поднесущих DSA (Dynamic Subcarrier Assignment) и 
алгоритм адаптивного распределения мощности APA, а 
также комбинация этих алгоритмов. Эффективность алгоритмов оценивалась с помощью 
имитационного моделирования, при котором они сравнивались с алгоритмом 
фиксированного назначения поднесущих FSA (Fixed Subcarrier Assignment). Ниже для 
алгоритмов DSA и APA сформулированы две задачи нелинейного целочисленного 
программирования, для которых получены условия оптимальности. Вводятся следующие 
обозначения: $\mathcal{N}$~--- множество поднесущих $\{1,\ldots ,N\}$; 
$\mathcal{K}$~--- множество пользователей $\{1,\ldots ,K\}$; $\beta$~--- коэффициент 
побитовой ошибки (BER); $\mathbf{p}=(p[1],\ldots ,p[N])$~--- вектор мощностей\linebreak 
поднесущих; $\rho$~--- состояние поднесущей (отношение сиг\-нал--шум); $\Delta f$~--- 
ширина полосы пропускания поднесущей; $c_k^p[n]$~--- достижимая эффективность 
передачи (бит/символ); $r_k$~--- скорость \mbox{передачи} $k$-пользователя, $k\in\mathcal{K}$.
  
  Предполагается, что нисходящий канал базовой станции ячейки сети OFDM
используется всеми пользователями, причем базовой станции известно со\-сто\-яние 
назначенной пользователю поднесущей. Достижимая скорость передачи данных зависит 
от отношения сиг\-нал--шум и мощности передачи: 

\vspace*{4pt}

\noindent
$$
c_k^p[n]= f(\log_2(1+\beta 
p[n]\rho_k[n])\,,
$$
где функция $f(\cdot)$ зависит от выбранной схемы адап\-та\-ции скорости. 
Например, если применять непрерывную адаптацию скорости, то $f(x)=x$ и 
$c_k^p[n]=\log_2(1+\beta p[n]\rho_k[n])$.
  
  Введем $x_{kn}\in \{0,\,1\}$ состояние $n$-поднесущей так, что $x_{kn}\hm=1$, если 
  $n$-под\-не\-су\-щая назначена $k$-поль\-зо\-ва\-те\-лю, и $x_{kn}\hm=0$ в противном случае. Тогда 
$x_k\hm=(x_{kn})_{n\in\mathcal{N}}$~--- вектор состояния поднесущих для 
  $k$-пользователя, причем условие $\sum\limits_{k\in\mathcal{K}} x_{kn}=1$ означает, 
что $n$-поднесущая может быть назначена только одному пользователю. Множество 
$\mathrm{D}_k(\mathbf{x}_k) =\{n:\ x_{kn}=1\}$ включает все поднесущие, назначенные 
$k$-пользователю в состоянии $\mathbf{x}_k$, а набор множеств 
$\mathrm{D}(\mathbf{x})=(\mathrm{D}_k (\mathbf{x}_k))_{k\in \mathcal{K}}$ определяет 
распределение поднесущих по пользователям, когда система находится в состоянии 
$\mathbf{x}=(\mathbf{x}_k)_{k\in\mathcal{K}}$. Тогда множество состояний системы 
можно определить в виде:

\vspace*{-4pt}

\noindent
  \begin{multline*}
  \mathrm{X}=\left\{ \vphantom{\mathop{\bigcup}\limits_{k\in \mathcal{K}}}
  \mathbf{x} =(\mathbf{x}_k)_{k\in\mathcal{K}}:\ 
\mathrm{D}_i(\mathbf{x}_i) \cap \mathrm{D}_j(\mathbf{x}_j) =\varnothing\,,\right.\\[-3pt]
  \left. i\not=j\in\mathcal{K}\,,\enskip \mathop{\bigcup}\limits_{k\in \mathcal{K}} 
\mathrm{D}_k(\mathbf{x}_k) \subseteq \mathcal{N}\right\}\,.
  \end{multline*}
  
%  \pagebreak
  
     Введем множество всех возможных наборов поднесущих  
$\mathcal{D}\hm=\{D(\mathbf{x}):\ \mathbf{x}\hm\in \mathcal{X}\}$ и множество возможных 
вариантов распределения мощностей 
$$
\mathcal{P}=\{\mathbf{p}:\ 0\leq p(n)\leq P, \ 
\sum\limits_{n\in N} p(n)=P\,, \  n\in \mathcal{N}\}\,.
$$

%\columnbreak

Скорость передачи данных $r_k$ [бит/с] 
для\linebreak $k$-поль\-зо\-вателя в состоянии $\mathbf{x}_k$ представима в виде:
     \begin{multline*}
     r_k {:=} r_k (\mathbf{x}_k,\mathbf{p}) =\sum\limits_{n\in\mathcal{N}} 
c_k^{\mathbf{p}}(n) \Delta f x_{kn}={}\\
     {}=\sum\limits_{n\in \mathcal{D}_k(\mathbf{x}_k)}  c_k^{\mathbf{p}}(n)\Delta f =r_k 
(\mathcal{D}_k (\mathbf{x}_k),\mathbf{p}) {=:} r_k(\mathcal{D}_k, \mathbf{p})\,,\\ k\in 
\mathcal{K}\,.
     \end{multline*}
     
     Пусть $U_k(\cdot)$~--- функция полезности для $k$-поль\-зо\-ва\-те\-ля, $k\hm\in 
\mathcal{K}$. Будем рассматривать в качестве основного блага для пользователя величину 
скорости передачи данных $r_k$ и определим функцию полезностив следующем виде:
     $$
     U(r(d,\mathbf{p})) {:=} \sum\limits_{k\in \mathcal{K}} 
U_k\left(r_k\left(D_k,\mathbf{p}\right)\right)\,.
     $$
     
     Таким образом, задача межуровневой оптимизации в общем случае может быть 
сформулирована как максимизация функции полезности для ячейки сети OFDM в виде:
     $$
     \max\limits_{d,\mathbf{p}}\sum\limits_{k\in \mathcal{K}} U_k(r(d,\mathbf{p}))
     $$
с ограничениями $d\in \mathcal{D}$ и $\mathbf{p}\in \mathcal{P}$.
     
     Пусть вектор \textbf{p} распределения мощностей фиксирован, т.\,е.\ 
$\mathbf{p}=\tilde{\mathbf{p}}$. Тогда функцию полезности для алгоритма 
динамического назначения поднесущих DSA можно определить в виде:
    $$
     U(r(d)){:=} U\left(r\left(d,\tilde{\mathbf{p}}\right)\right)\,.
     $$
     
     В случае алгоритма адаптивного распределения мощностей APA фиксированным 
является набор $d$ множеств поднесущих $d=\tilde{d}$, и, следовательно, функция 
полезности имеет вид:
     $$
     U(r(\mathbf{p})) {:=}  U\left( r\left( \tilde{d},\mathbf{p}\right)\right)\,.
     $$
  
  Отметим, что данная задача относится к классу задач целочисленного нелинейного 
программирования. В~[19] для решения этой задачи используется метод релаксаций, идея 
которого заключается в том, что при разработке метода решения задач отбрасывается 
требование к целочисленности переменных. Предполагается, что функция полезности 
$U_k(r_k)$ для $k$-пользователя является неубывающей выпуклой и существует ее 
производная $U^\prime_k(r_k)$.

\bigskip

\noindent
\textit{Алгоритм динамического назначения поднесущих} DSA
    
    
    \vspace*{2pt}
     
     С учетом введенных обозначений задача максимизации полезности для алгоритма 
DSA записывается в виде $\max\limits_{d\in \mathcal{D}} U(r(d))$.
     
     Используя метод математической индукции можно доказать, что максимум функции 
по\-лез\-ности $U(r(d))$ достигается в сформулированных ниже достаточных условиях 
оптимальности.


\medskip

\noindent
\textbf{Утверждение 1.} Если для набора $d^*=(D^*_k)_{k\in \mathcal{K}}$ выполняется 
условие:
$$
U^\prime_k (r^*_k) c_k^{\tilde{\mathbf{p}}}(n)\geq U^\prime_j (r^*_j) 
c_j^{\tilde{\mathbf{p}}}(n)\,,\enskip  k\not=j\in \mathcal{K}\,,\ n\in D_k^*\,,
$$
где $r_k^*=\sum\limits_{n\in D_k^*} c_n^{\tilde{\mathbf{p}}}(n)\Delta f$, тогда функция 
полезности $U(r(d))$ достигает глобального максимума на наборе $d=d^*\in 
\mathcal{D}$.
     
     \medskip
     
     Из полученных условий оптимальности получаем правило назначения поднесущей 
пользователю. Для заданного вектора $\mathbf{p}=\tilde{\mathbf{p}}$ распределения 
мощности номер пользователя, которому назначается $n$-поднесущая, определяется 
формулой:
     $$
     k(n) =\mathrm{arg}\max\limits_{k\in\mathcal{K}} \left\{ U^\prime_k (r_k^*) 
c_k^{\tilde{p}}(n)\right\}\,.
     $$

\medskip

\noindent
\textit{Алгоритм адаптивного распределения мощности} APA

\smallskip

     Алгоритм адаптивного распределения мощ\-ности APA состоит в назначении 
каждой поднесущей $n\in \mathcal{N}$ определенной мощности передачи при условии 
фиксированного набора $d=\tilde{d}$ распределения поднесущих между пользователями.
     
  Задача межуровневой оптимизации для алгоритма APA может быть сформулирована 
в виде $\max\limits_{p\in \mathcal{P}} U(r(\mathbf{p}))$. Ниже сформулированы 
необходимые условия достижения максимума функцией полезности $U(r(\mathbf{p}))$.
  
  \medskip
  
  \noindent
\textbf{Утверждение 2.} Если $p^*(n)$  является решением задачи 
$\max\limits_{\mathbf{p}\in \mathcal{P}} U(r(\mathbf{p}))$, тогда
$$
p^*(n)=\left[ \fr{U^\prime_k(r^*_k)}{\lambda}-\fr{1}{\beta \rho_k(n)}\right]^*\,,\enskip k\in 
\mathcal{K}\,,\ n\in \tilde{D}_k\,,
$$
где $\lambda>0$~--- нормирующая константа оптимального распределения мощностей.
     
     Из полученных условий оптимальности функции $U(r(\mathbf{p}))$ очевидным 
образом следует правило назначения мощностей поднесущих для алгоритма APA.

\section{Заключение}
  
  Планирование межуровневого интерфейса является наиболее эффективным подходом к 
согласованию возможностей современных беспроводных технологий и возрастающих 
требований по обслуживанию больших объемов трафика пользователей с заданным 
качеством. 
  
  В обзоре технические и алгоритмические проблемы создания планировщиков 
межуровневого интерфейса иллюстрированы постановками оптимизационных задач, 
возникающих при распределении ресурсов в сетях с большим числом несущих 
радиочастот. Рассмотрены наиболее известные задачи и ограничения, характерные для 
технологии LTE, даны краткие комментарии по их решению, алгоритмам поиска 
оптимального решения и условиям оптимальности. Приведен типичный пример задачи 
оптимизации функции полезности как наиболее общей задачи оптимального 
планирования межуровневого интерфейса. Исследовано и сформулировано достаточное 
условие нахождения глобального максимума функции полезности для задачи DSA, а 
также необходимое условие для задачи APA.

{\small\frenchspacing
{%\baselineskip=10.8pt
\addcontentsline{toc}{section}{Литература}
\begin{thebibliography}{99}

\bibitem{1-gai}
\Au{Shariat M., Quddus A.\,U., Ghorashi~S.\,A., Tafazolli~R.}
 Scheduling as an important cross-layer operation for emerging broadband wireless systems~// 
IEEE Commun. Surveys Tuts., 2009. Vol.~11. No.\,2. P.~74--86.
\bibitem{2-gai}
\Au{Вишневский В.\,М., Ляхов А.\,И., Портной~С.\,Л., Шахнович~И.\,В.}
Широкополосные беспроводные сети передачи информации.~--- М.: Техносфера, 2005. 
597~c.
\bibitem{3-gai}
\Au{Тихвинский В.\,О., Терентьев С.\,В., Юрчук~А.\,Б.} Сети мобильной связи LTE: 
технология и архитектура.~--- М.: Эко-Трендз, 2010. 284~с.
\bibitem{4-gai}
\Au{Wu G., Bai~Y., Lai~J., Ogielski~A.} Interaction between TCP and RLP in wireless 
Internet~// IEEE Global Communication Conference Proceedings, 1999. Vol.~1b. P.~661--666.
\bibitem{5-gai}
\Au{Kim B.\,J.} A~network service providing wireless channel information for adaptive mobile 
applications: Part~I: Proposals~// IEEE  Conference (International) on Communications 
Proceedings, 2001. Vol.~5. P.~1345--1351.
\bibitem{6-gai}
\Au{Sudame P., Badrinath~B.\,R.}
On providing support for protocol adaptation in mobile networks~// Mobile Networks 
Applications, 2001. Vol.~6. No.\,1. P.~43--55.
\bibitem{7-gai}
\Au{Chiang M.} To layer or not to layer: Balancing transport and physical layers in wireless 
multihop networks~// IEEE J.~Selected Areas  Commun., 2005. Vol.~23. No.\,1. 
P.~104--116. 
\bibitem{8-gai}
\Au{Kawadia V., Kumar P.\,R.}
A~cautionary perspective on cross-layer design~// IEEE Wireless Commun., 2005. Vol.~12. 
No.\,1. P.~3--11.
\bibitem{9-gai}
\Au{Iannone L., Fdida S.} Evaluating a cross-layer approach for routing in wireless mesh 
networks~// Telecommunication Systems J. (Springer) Special issue: Next Generation 
Networks~--- Architectures, Protocols, Performance, 2006. Vol.~31. No.\,2--3. P.~173--193.
\bibitem{10-gai}
\Au{Wong C.\,Y., Cheng R.\,S., Letaief~K.\,B.} Multiuser OFDM with adaptive subcarrier, bit, 
and power allocation~// IEEE J. Selected Areas  Commun., 1999. Vol.~17. No.\,10. 
P.~1747--1757.
\bibitem{11-gai}
\Au{Hughes-Hartogs D.} Ensemble modem structure for imperfect transmission media. U.S.\ 
Patents No.\,4679227, July 1987; No.\,4731816, March 1988; No.\,4833796, May 1989.
\bibitem{12-gai}
\Au{Lai S.\,K., Cheng R.\,S., Letaief K.\,B., Murch~R.\,D.} Adaptive trellis coded MQAM and 
power optimization for OFDM transmission~// IEEE Trans. Commun., 1999. Vol.~47. 
P.~538--545.
\bibitem{13-gai} %12
\Au{Bohge M., Gross J., Wolisz~A., Meyer~M.} Dynamic resource allocation in OFDM Systems: 
an overview of cross-layer optimization principles and techniques~// IEEE Networks, 2007. 
Vol.~21. No.\,1. P.~53--59.
\bibitem{14-gai}
\Au{Kivanc D., Li G., Liu~H.}
Computationally efficient bandwidth allocation and power control for OFDMA~// IEEE 
Trans. Wireless Commun., 2003. Vol.~2. No.\,6. P.~1150--1158.
\bibitem{15-gai}
\Au{Zhang Y.\,J., Letaief K.\,B.} Multiuser adaptive subcarrier and bit allocation with adaptive 
cell selection for OFDM systems~// IEEE Trans. Wireless Commun., 2004. 
Vol.~3. No.\,5. P.~1566--1575.
\bibitem{16-gai}
\Au{Otani Y., Ohno S., Teo K., Teo~D., Hinamoto~T.}
Subcarrier allocation for multi-user OFDM system~// Asia-Pacific Communication Conference 
Proceedings, 2005. P.~1073--1077.
\bibitem{17-gai}
\Au{Wong C., Shen Z., Evans L., Andrews~J.\,G.} A~low complexity algorithm for proportional 
resource allocation in OFDMA systems~// IEEE Workshop on Signal Processing Systems 
Proceedings.~--- Texas, USA, 2004. P.~1--6.
\bibitem{18-gai}
\Au{Fu J., Karasawa Y.} Fundamental analysis on throughput characteristics of orthogonal 
frequency division multiple access OFDMA in multipath propagation environments~// IEICE  
Trans., 2002. Vol.~J85-B. No.\,11. P.~1884--1894.

\label{end\stat}
\bibitem{19-gai}
\Au{Song G., Li~Ye.}
Cross-layer optimization for OFDM wireless networks~--- Part~I: Theoretical framework~// 
IEEE Trans. Wireless Commun., 2005. Vol.~4. No.\,2. P.~614--624.


\bibitem{20-gai}
\Au{Song G., Li Y.} Cross-layer optimization for OFDM wireless networks~--- Part~II: 
Algorithm development~// IEEE Trans. Wireless Commun., 2005. Vol.~4. 
No.\,2. P.~625--634.
%\bibitem{21-gai}
%\Au{Глебов Н.\,И., Кочетов Ю.\,А., Плясунов~А.\,В.}
%Методы оптимизации: Учебное пособие.~--- Новосибирск: НГУ, 2000. 105~с.
 \end{thebibliography}
}
}


\end{multicols}