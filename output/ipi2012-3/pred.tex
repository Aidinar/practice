   { %\Large  
   { %\baselineskip=16.6pt
   
   \vspace*{-48pt}
   \begin{center}\LARGE
   \textit{Предисловие}
   \end{center}
   
   %\vspace*{2.5mm}
   
   \vspace*{25mm}
   
   \thispagestyle{empty}
   
   { %\small 


     
     Вниманию читателей предлагается очередной выпуск журнала <<Информатика и её 
применения>>, полностью посвященный публикации расширенных версий докладов, 
представленных на ряде научных мероприятий, которые были проведены в 2011~г.\ с 
активным организационным и научным участием Российской академии наук (и прежде 
всего Института проблем информатики Российской академии наук). 
     
     \smallskip
     
     Серия Всероссийских конференций с международным участием 
<<{\bfseries\textit{Электронные библиотеки: перспективные методы и технологии, 
электронные коллекции}}>> (Russian Conference on Digital Libraries~--- RCDL) направлена 
на формирование сообщества специалистов России, ведущих исследования и разработки в 
области электронных библиотек и близких областях. Для такого сообщества конференции 
RCDL пред\-остав\-ля\-ют возможность обмена накопленным опытом, идеями и полученными 
результатами, а также установления контактов для более тесного сотрудничества. 
Конференции RCDL проводятся начиная с 1999~г. В~конференциях RCDL приняли 
участие многочисленные российские и зарубежные специалисты из различных научных 
областей~--- не только представители университетов и исследовательских центров, но 
также разработчики и пользователи электронных библиотек, студенты и аспиранты. 
Конференция RCDL активно сотрудничает с близкими по тематике зарубежными конференциями 
ECDL (European Conference on Digital Libraries)
и JCDL (Joint Conference on Digital Libraries), что способствует 
развитию международного сотрудничества в области 
электронных библиотек. Конференция активно поддерживается Российским фондом фундаментальных
исследований (РФФИ) и Московской  секцией ACM SIGMOD (Special Interest Group on
Management Of Data).
     
     В 2011~г.\ конференция RCDL проводилась в г. Воронеже с 19 по 22~октября. 
Среди организаторов конференции~--- Российская академия наук, РФФИ, Воронежский государственный университет, Институт 
проблем информатики РАН, Московская секция АСМ SIGMOD. 
     
     Программный комитет конференции RCDL-2011 рекомендовал к публикации в 
журнале <<Информатика и её применения>> пять статей, являющихся развернутыми 
вариантами докладов, представленных на конференции:
     \begin{itemize}
\item О.\,П.~Желенкова <<Исследование радиоисточников средствами виртуальной 
обсерватории>>;
\item Ю.\,А.~Загорулько, О.\,И.~Боровикова, И.\,С.~Кононенко, Е.\,Г.~Соколова 
<<Методологические аспекты разработки электронного русско-английского 
тезауруса по компьютерной лингвистике>>;
\item М.\,Р.~Когаловский, С.\,И.~Паринов <<Классификация и использование 
семантических связей между информационными объектами в научных 
электронных библиотеках>>;
\item Е.\,В.~Шарапова, Р.\,В.~Шарапов <<Универсальная система проверки текстов 
на плагиат <<АВТОР.NET>>;
\item Д.\,М.~Скачков, О.\,Л.~Жижимов <<Об интеграции географических 
метаданных посредством ретроспективного тезауруса>>.
\end{itemize}

     В число авторов этих статей входят ученые из организаций Российской академии 
наук (Специальная астрофизическая обсерватория РАН, Институт систем информатики 
им.~А.\,П.~Ершова СО РАН, Институт проблем рынка РАН, Центральный 
     экономико-математический институт РАН, Институт вычислительных технологий 
СО РАН), а также из Российского государственного гуманитарного университета и 
Владимирского государственного университета имени Александра Григорьевича и 
Николая Григорьевича Столетовых.
     
     {\bfseries\textit{Международный семинар по проблемам устойчивости 
стохастических моделей}} имеет давние традиции. Он был основан профессором 
В.\,М.~Золотарёвым в 1970~г.
     
     В семинарах принимали и принимают участие ведущие специалисты в области 
теории вероятностей и ее приложений, представляющие разные страны мира. В~XXI~в.\ 
семинары проходили в Венгрии, Болгарии, Испании, Латвии, Италии, Израиле, Румынии 
и Польше.
     
     В 2011~г.\ XXIX Международный семинар по проблемам устойчивости 
стохастических моделей проводился в г.~Светлогорске Калининградской области России 
с 10 по 16~октября. Организаторы семинара: Московский государственный университет 
им.\ М.\,В.~Ломоносова и Институт проблем информатики РАН.
     
     Программный комитет XXIX Международного семинара по проблемам 
устойчивости стохастических моделей рекомендовал к публикации в журнале 
<<Информатика и её применения>> девять статей, являющихся развернутыми вариантами 
или дальнейшим развитием докладов, представленных на конференции:
     \begin{itemize}
\item C.\,De~Nikola, Y.\,S.~Khokhlovy, M.~Pagano, and O.\,I.~Sidorova ``Fractional 
Levy motion with dependent increments and its application to network traffic modeling'';
\item O.~Yanushkevichiene and R.~Yanushkevichius ``About the rate of convergence of one 
U-statistic'';
\item В.\,Е.~Бенинг, Л.\,М.~Закс, В.\,Ю.~Королев <<Оценки скорости сходимости 
распределений случайных сумм к дисперсионным гамма-распределениям>>;
\item Ю.\,В.~Гайдамака, Т.\,В.~Ефимушкина, А.\,К.~Самуйлов, К.\,Е.~Самуйлов 
<<Задачи оптимального планирования межуровневого интерфейса в беспроводных 
сетях>>$^*$;
\item О.\,В.~Лукашенко, Е.\,В.~Морозов <<Асимптотика максимума процесса 
нагрузки для некоторого класса>>$^*$;
\item Е.\,В.~Морозов, Р.\,С.~Некрасова <<Об оценивании вероятности 
переполнения конечного буфера в регенеративных системах обслуживания>>$^*$;
\item Е.\,В.~Морозов, А.\,С.~Румянцев <<Вероятностные модели 
многопроцессорных систем: стационарность и моментные свойства>>$^*$;
\item А.\,В.~Печинкин, И.\,А.~Соколов <<Ограничение на суммарный объем заявок 
в дискретной системе Geo$/G/1/\infty$>>$^*$;
\item А.\,В.~Ушаков <<Анализ системы обслуживания с гиперэкспоненциальным 
входящим потоком в условиях критической загрузки>>$^*$.
\end{itemize}

     В число авторов этих статей входят исследователи из организаций Российской 
академии наук (Институт проблем информатики РАН, Институт проблем рынка РАН, 
Центральный экономико-математический институт РАН, Институт вычислительных 
технологий СО РАН), представители факультета вычислительной математики и 
кибернетики Московского государственного университета им.\ М.\,В.~Ломоносова, 
Российского университета дружбы народов, Тверского государственного университета, 
отдела моделирования и математической статистики Альфа-банка (все~--- Россия), 
университетов гг.~Салерно и Пиза (оба~--- Италия), Института математики и 
информатики Вильнюсского университета и Литовского университета эдукологии (оба~--- 
Литва).
     
     Статьи, отмеченные звездочкой ($^*$), являются расширенными вариантами или 
научным развитием докладов, представленных на осенней сессии V Международного 
семинара <<Прикладные задачи теории вероятностей и математической статистики, 
связанные с моделированием информационных систем>>, проведенной в качестве 
сопутствующего мероприятия в рамках XXIX Международного семинара по проб\-ле\-мам 
устойчивости стохастических моделей. 

Организатором Международного семинара 
<<Прикладные задачи теории вероятностей и математической статистики, связанные с 
моделированием информационных систем>>, проводящегося с 2006~г., является Институт 
проблем информатики РАН.
     } }
     }