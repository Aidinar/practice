
%\def\v{\varphi}
%\def\g{\gamma}
%\def\w{\omega}
%\def\a{\overline a}
%\def\b{\beta}




\def\stat{pechinkin}

\def\tit{ОГРАНИЧЕНИЕ НА СУММАРНЫЙ ОБЪЕМ ЗАЯВОК В~ДИСКРЕТНОЙ СИСТЕМЕ Geo$/G/1/\infty$$^*$}

\def\titkol{Ограничение на суммарный объем заявок в~дискретной системе Geo/$G/1/\infty$}

\def\autkol{А.\,В.~Печинкин,  И.\,А.~Соколов, С.\,Я.~Шоргин}
\def\aut{А.\,В.~Печинкин$^1$,  И.\,А.~Соколов$^2$, С.\,Я.~Шоргин$^3$}

\titel{\tit}{\aut}{\autkol}{\titkol}

{\renewcommand{\thefootnote}{\fnsymbol{footnote}}\footnotetext[1]
{Работа выполнена при поддержке РФФИ
(гранты 11-07-00112, 12-07-00108 и 11-01-12026-офи-м).}}


\renewcommand{\thefootnote}{\arabic{footnote}}
\footnotetext[1]{Институт проблем информатики Российской академии наук; apechinkin@ipiran.ru}
\footnotetext[2]{Институт проблем информатики Российской академии наук, isokolov@ipiran.ru} 
\footnotetext[3]{Институт проблем информатики Российской академии наук, sshorgin@ipiran.ru}

\vspace*{-6pt}

\Abst{Рассматривается функционирующая в дискретном времени
однолинейная сис\-те\-ма массового обслуживания
Geo$/G/1$ с инверсионным порядком обслуживания без
прерывания обслуживания, в которой каждая заявка
наряду с (дискретной) случайной длиной имеет также (дискретный)
случайный объем. Суммарный объем находящихся в сис\-те\-ме заявок ограничен
некоторым (неслучайным) числом. Получены алгоритмы, позволяющие вычислять основные
стационарные показатели функционирования этой сис\-темы.}

\KW{система массового обслуживания; дискретное время;
длина и объем заявки}


\vskip 12pt plus 9pt minus 6pt

      \thispagestyle{headings}

      \begin{multicols}{2}

            \label{st\stat}

\section{Введение. Описание системы}

Задача исследования систем массового обслуживания (СМО),
в которых каждая поступающая в систему заявка наряду со
случайной длиной имеет случайный объем, причем
суммарный объем всех находящихся в системе заявок
ограничен, как было замечено еще в работах~[1--3],
играет важную роль при моделировании работы
самых разнообразных технических устройств, в част\-ности
современных ин\-фор\-ма\-ци\-он\-но-вы\-чис\-ли\-тель\-ных сис\-тем.
Однако аналитических решений этой задачи при дисциплине
выбора заявок из очереди на обслуживание в порядке
поступления (FIFO) до сих пор не найдено, поскольку
для корректного построения соответствующего
марковского процесса, описывающего функционирование СМО,
необходимо учитывать объемы всех заявок в сис\-те\-ме.
Фактически приходится сталкиваться с теми же самыми трудностями, что и при исследовании
многолинейных СМО, для которых также не найдено удовлетворительных аналитических решений.

В работах [4--11]
были исследованы СМО с ограничением на суммарный объем
заявок, но при инверсионном порядке обслуживания (дисциплина LIFO).
Оказалось, что в этом случае можно получить алгоритмы,
пригодные для численных расчетов стационарных характеристик.

Тем не менее во всех этих работах, в том числе и в~[9--11],
где рассматривались СМО в дискретном времени, распределение
объема заявки предполагалось непрерывным, а тогда алгоритмы расчета 
опира\-лись на решения довольно
сложных интегральных уравнений, что снижало практическую
значимость полученных результатов. 
В~рас\-смат\-ри\-ва\-емой в настоящей статье СМО Geo$_m/G/1$
с ограничением на суммарный объем находящихся
в ней заявок, в отличие от цитированных выше работ,
объем каждой заявки является дискретной случайной
величиной. Это позволяет получить более прос\-тые и эффективные
алгоритмы рас\-че\-та основных стационарных показателей функционирования.

Рассмотрим функционирующую в дискретном времени
однолинейную СМО Geo$_m/G/1$, в которую
поступает геометрический поток заявок с ве\-ро\-ят\-ностью~$a$ поступления заявки на такте.

Каждая поступающая в систему заявка наряду с длиной
имеет случайный целочисленный (не\-от\-ри\-ца\-тель\-ный) объем.
Совместное распределение длины и объема заявки задается
вероятностью $b_{k,l}$,  $k,l\ge 0$, того, что ее длина
(число тактов обслуживания) равна~$k$, а объем равен~$l$.
Будем предполагать выполненным естественное условие, что
длина заявки и ее объем не могут равняться нулю, т.\,е.\
$b_{k,0}\hm=b_{0,l}\hm=0$ для любых $k$ и~$l$.

Общий объем находящихся в системе заявок ограничен
(неслучайным) числом~$L$,\ \ $0\hm<L\hm<\infty$.
Если объем поступающей в систему заявки в
сумме с объемами остальных находящихся в сис\-те\-ме
заявок больше~$L$, то она теряется. Будем предполагать, что если в момент поступления
новой заявки систему покидает обслуженная заявка, 
то ее объем при определении суммарного объема не учитывается.

В системе реализован инверсионный порядок обслуживания
без прерывания обслуживания, при котором принятая в
сис\-те\-му заявка становится на первое место в очереди.
Будем считать для определенности, что если в момент
поступления новой заявки сис\-те\-му покидает обслуженная
заявка, то на прибор становится новая заявка.

Введем также обозначения:

\noindent
$b(l)=\sum\limits_{k=1}^\infty b_{k,l}$,\  $l\ge 1$~---
вероятность того, что объем заявки равен $l$;

\noindent
$B(l)=\sum\limits_{j=1}^l b(j) =
\sum\limits_{j=1}^l \sum\limits_{k=1}^\infty b_{k,j}
$,\  $l\ge 1$~--- вероятность того, что объем заявки
не более $l$;

\noindent
$b(k\,|\,l)=b_{k,l}/b(l)$,\ $k,l\ge 1$~--- условная
ве\-ро\-ят\-ность того, что длина заявки равна~$k$, при условии, что
ее объем равен $l$;

\noindent
$B(k\,|\,l)=
\sum\limits_{i=k}^\infty b(i\,|\,l)$,\ $k,l\ge 1$~---
условная вероятность того, что длина заявки не
менее~$k$, при условии, что ее объем равен~$l$;

\noindent
$\beta(z\,|\,l)=\sum\limits_{k=1}^\infty z^k b(k\,|\,l)$,\ 
$l\ge 1$~---
производящая функ\-ция (ПФ) длины заявки при условии,
что ее объем равен~$l$;

\noindent
$\beta^*(z\,|\,l)=
\sum\limits_{k=1}^\infty z^{k-1} B(k+1\,|\,l)
=[z-\beta(z\,|\,l)]/[z(1\hm-z)]$,\  $l\ge 1$;

\noindent
$\overline m=\sum\limits_{k=0}^\infty
\sum\limits_{l=0}^\infty k b_{k,l}$~---
математическое ожидание длины заявки.

Далее будем предполагать, что выполнено условие
$\overline m \hm< \infty$.
Это условие является необходимым и достаточным для
существования стационарного режима функционирования
сис\-те\-мы. Кроме того, чтобы избежать непринципиальных
трудностей в изложении, будем считать, что объем заявки,
во-пер\-вых, не превосходит $L$, а во-вто\-рых, с ненулевой
вероятностью может принимать значение единица.

\section{Стационарные вероятности состояний}

Обозначим через $p_0$ стационарную вероятность
отсутствия заявок в сис\-те\-ме, а через
$p_{k,i}(l_1,\ldots,l_i)$,\ 
$i \hm\ge 1$,\  $k\hm\ge 1$,\  $l_1,\ldots,l_i \hm\ge 1$,~---
стационарную вероятность того, что в сис\-те\-ме находится
$i$ заявок, причем (обслуженная) длина и объем заявки
на приборе равны~$k$ и~$l_1$, а объемы остальных
находящихся в сис\-те\-ме заявок равны (в порядке очереди)
$l_2,\ldots,l_i$.

Заметим, что поскольку объем каждой заявки~---
целое положительное чис\-ло, то
суммарный объем $l_1 +\ldots+ l_i$ заявок в системе
не может быть меньше числа $i$ заявок, т.\,е.\
неравенство $p_{k,i}(l_1,\ldots,l_i) \hm> 0$
может выполняться только при
\begin{equation}
\label{6.2.0}
i\le l_1 +\ldots+ l_i \le L
\,,\enskip i \ge 1\,.
\end{equation}
Поэтому всюду далее, не оговаривая этого особо, будем
предполагать, что условие~\eqref{6.2.0} выполнено.

Используя метод исключения состояний~[12, с.~22],
получаем сис\-те\-му урав\-нений:
\begin{multline}
\label{6.2.1}
p_{k,1}(l)
=
[1 - a B(L-l)]
\fr{B(k+1\,|\,l)}{B(k\,|\,l)}
\,p_{k-1,1}(l)\,,\\
\enskip k\ge 1\,,
\end{multline}
%%%%%%%%%%%%%%%%%%%%%

\vspace*{-9pt}

\noindent
\begin{multline}
p_{k,i}(l_1,\ldots,l_i)
=
[1 - a B(L-l_1-\cdots\\
\cdots -l_i)]
\fr{B(k+1\,|\,l_1)}{B(k\,|\,l_1)}
\,p_{k-1,i}(l_1,\ldots,l_i)
 +{}\\
{}
+\
a b(l_2)
\fr{B(k+1\,|\,l_1)}{ B(k\,|\,l_1)}
\,p_{k-1,i-1}(l_1,l_3,\ldots,l_i)\,,
\\
i \ge 2\,,\ \  k\ge 1\,,
\label{6.2.2}
\end{multline}
%%%%%%%%%%%%%%%%%%%%%%%%%%%%%%%%%%%%%%
с начальными условиями
\begin{multline}
\label{6.2.3}
p_{0,1}(l)
={}\\
{}=p_0 a b(l)
+a b(l)\sum\limits_{s=1}^{L-l}
\sum\limits_{k=1}^\infty
\fr{B(k+1\,|\,s) }{B(k\,|\,s)}\, p_{k-1,1}(s)
+{}\\
{}+
a b(l)
\sum\limits_{s=1}^{L}
\sum\limits_{k=1}^\infty
\fr{b(k\,|\,s)}{ B(k\,|\,s)}\, p_{k-1,1}(s)\,;
\end{multline}

\vspace*{-9pt}

\noindent
\begin{multline}
p_{0,i}(l_1,\ldots,l_i)
= a b(l_1)\times{}\\
{}\times 
\sum\limits_{l=1}^{L-l_1-\cdots-l_i}
\sum\limits_{k=1}^\infty
\fr{B(k+1\,|\,l) }{ B(k\,|\,l)}\,
p_{k-1,i}(l,l_2,\ldots,l_i)
 +{}
\\
{}+
a b(l_1) \sum\limits_{l=1}^{L-l_2-\cdots-l_i}
\sum\limits_{k=1}^\infty
\fr{b(k\,|\,l) }{B(k\,|\,l)}\,
p_{k-1,i}(l,l_2,\ldots,l_i)\,,
\\
i \ge 2\,.
\label{6.2.4}
\end{multline}

Из уравнений~(\ref{6.2.1}) и~(\ref{6.2.3}) получаем:
\begin{multline}
\label{6.2.5}
p_{k,1}(l)=
[1 - a B(L-l)]^k B(k+1\,|\,l)
\,p_{0,1}(l)\,;
\\ 
k\ge 1\,;
\end{multline}

\vspace*{-9pt}

\noindent
\begin{multline}
\label{6.2.6}
p_{0,1}(l)= p_0 a b(l)+{}\\
{}+
a b(l)\sum\limits_{s=1}^{L-l}
\beta^*([1 - a B(L-s)]\,|\,s)
p_{0,1}(s)
 +{}
\\
{}+
a b(l)
\sum\limits_{s=1}^{L}
\beta([1 - a B(L-s)]\,|\,s)
p_{0,1}(s)\,.
\end{multline}
%%%%%%%%%%%%%%%%%%%%%%%%%%%%%

Из уравнения~(\ref{6.2.2}) имеем:
%%%%%%%%%%%%%%%%%%%%%%%%%%%%%
\begin{multline}
p_{k,i}(l_1,\ldots,l_i)
={}\\
{}=[1 - a B(L-l_1-\cdots-l_i)]^k
B(k+1\,|\,l_1) p_{0,i}(l_1,\ldots,l_i)
+{}
\\
{}+r_{k,i}(l_1,\ldots,l_i)\,,
\enskip
i \ge 2\,,\ \ k\ge 1\,,
\label{6.2.7}
\end{multline}
%%%%%%%%%%%%%%%%%%%%%%
где
\begin{equation}
\label{6.2.8}
r_{0,i}(l_1,\ldots,l_i)
= 0\,,\quad
i \ge 2,
\end{equation}

\vspace*{-9pt}

\noindent
\begin{multline}
r_{k,i}(l_1,\ldots,l_i)
=[1 - a B(L-l_1-\cdots\\
\cdots -l_i)]
\fr{B(k+1\,|\,l_1)}{ B(k\,|\,l_1)}
\,r_{k-1,i}(l_1,\ldots,l_i)
 +{}
\\
{}+
a b(l_2)
\fr{B(k+1\,|\,l_1)}{ B(k\,|\,l_1)}
\,p_{k-1,i-1}(l_1,l_3,\ldots,l_i),
\\
 i \ge 2\,,\ \ k\ge 1\,,
\label{6.2.9}
\end{multline}
а из уравнения~(\ref{6.2.4}) находим:
%%%%%%%%%%%%%%%%%%%%%%%%%
\begin{multline}
p_{0,i}(l_1,\cdots,l_i)
=a b(l_1)
\sum\limits_{l=1}^{L-l_1-\cdots-l_i}
\beta^*\times{}\\
{}\times([1 - a B(L-l-l_2-\cdots-l_i)]\,|\,l)
p_{0,i}(l,l_2,\ldots,l_i)
 +{}
\\
+a b(l_1)
\sum\limits_{l=1}^{L-l_2-\cdots-l_i}
\beta([1 - a B(L-l-l_2-\cdots-l_i)]\,|\,l)\times{}\\
{}\times
p_{0,i}(l,l_2,\ldots,l_i)
 +a b(l_1)\times{}
\\
{}\times
\sum\limits_{l=1}^{L-l_1-\cdots-l_i}
\sum\limits_{k=1}^\infty
\fr{B(k+1\,|\,l)}{ B(k\,|\,l)}\,
r_{k-1,i}(l,l_2,\ldots,l_i)
 +{}\\
{}+ a b(l_1)
\sum\limits_{l=1}^{L-l_2-\cdots-l_i}
\sum\limits_{k=1}^\infty
\fr{b(k\,|\,l)}{B(k\,|\,l)}\,
r_{k-1,i}(l,l_2,\ldots,l_i)
\,,
\\
  i \ge 2\,.
\label{6.2.10}
\end{multline}

Соотношения (\ref{6.2.5})--(\ref{6.2.10}) позволяют
последовательно, начиная с $i\hm=1$ и кончая $i\hm=L$,
с точностью до~$p_0$
определять вероятности $p_{k,i}(l,l_1,\ldots,l_{i-1})$.

Вероятность $p_0$, как обычно, определяется из условия
нормировки.

Однако нахождение $p_{k,i}(l,l_1,\ldots,l_{i-1})$
из-за большой размерности вычисляемых вероятностей
невозможно уже при совсем небольших значениях~$i$ даже для современной вычислительной техники.
Поэтому в следующем разделе будет определено более простое маргинальное распределение и приведены
формулы для его расчета, а в разд.~4 описан удобный алгоритм вычислений.

\section{Маргинальное распределение стационарных
вероятностей состояний}

С точки зрения практики вполне достаточно знать не
совместное распределение объемов всех находящихся в
системе заявок, а только лишь совместное распределение
объема обслуживаемой на приборе заявки и суммарного
объема остальных находящихся в сис\-те\-ме заявок.
Более того, в сис\-те\-ме без ограничения на чис\-ло
находящихся в ней заявок при инверсионном порядке
обслуживания чис\-ло
заявок в сис\-те\-ме также не представляет особого интереса.
Поэтому обозначим через
\begin{multline*}
p_{k}(l,m)
= \sum\limits_{i=1}^\infty
\sum\limits_{l_1+\cdots+l_{i-1}=m}
p_{k,i}(l,l_1,\ldots,l_{i-1})\,,
\\
 k\ge 0\,,\ \ m\ge 0\,,\ \ l=\overline{L-m}\,,
\end{multline*}
стационарную вероятность того, что (обслуженная) длина
и объем заявки на приборе равны $k$ и~$l$, а суммарный
объем остальных находящихся в сис\-те\-ме заявок равен $m$
(напомним, что в силу условия~\eqref{6.2.0} обязательно
должно выполняться двойное неравенство
$i-1\hm\le m \hm\le L-l$).
Примем соглашение, что значение $m\hm=0$ соответствует
отсутствию заявок в накопителе ($i\hm=1$).

Из соотношений (\ref{6.2.5})--(\ref{6.2.9}) имеем:
%%%%%%%%%%%%%%%%%%%%%%%%%%%%%
\begin{multline}
\label{6.2*.11}
p_{k}(l,m)
=[1 - a B(L-l-m)]^k
B(k+1\,\vert\,l) p_{0}(l,m)
+{}\\
{}+ r_{k}(l,m)\,,
\enskip
k \ge 1\,,\ \ m\ge 0\,,
\end{multline}
где
\begin{equation}
\label{6.2*.12}
r_{k}(l,0) = 0\,, \quad k \ge 1\,;
\end{equation}
%%%%%%%%%%%%%%%%%%%%%%

\noindent
\begin{equation}
\label{6.2*.13}
r_{0}(l,m) = 0\,,
\quad m \ge 1\,;
\end{equation}

\vspace*{-9pt}

\noindent
\begin{multline}
r_{k}(l,m) ={}\\
{}=
[1 - a B(L-l-m)]
\fr{B(k+1\,|\,l)}{ B(k\,|\,l)}
\,r_{k-1}(l,m) +{}\\
{}+ a \fr{B(k+1\,|\,l) }{ B(k\,|\,l)}
\sum\limits_{s=1}^{m-1}
b(s) p_{k-1}(l,m-s)\,,
\\
 k\ge 1\,,
\ \ m \ge 1\,,
\label{6.2*.14}
\end{multline}
%%%%%%%%%%%%%%%%%%%%%%%%%%%%%%%%%%%%%
а из соотношений (\ref{6.2.6}) и (\ref{6.2.10}) находим:
\begin{multline}
p_{0}(l,m) =q(l,m) + {}\\
{}+ a b(l)
\sum\limits_{s=1}^{L-l-m}
\beta^*([1 - a B(L-s-m)]\,|\,s)
p_{0}(s,m)
 +{}
\\
{}+ a b(l) \sum\limits_{s=1}^{L-m}
\beta([1 - a B(L-s-m)]\,|\,s)
p_{0}(s,m)\,,
\\
m \ge 0\,,
\label{6.2*.15}
\end{multline}
где
\begin{equation}
\label{6.2*.16}
q(l,0) = p_0 a b(l) \,;
\end{equation}

\vspace*{-9pt}

\noindent
\begin{multline}
q(l,m) = a b(l)
\sum\limits_{s=1}^{L-l-m}
\sum\limits_{k=1}^\infty
\fr{B(k+1\,|\,s)}{ B(k\,|\,s)}\,
r_{k-1}(s,m)  +{}\\
{}+ a b(l) \sum\limits_{s=1}^{L-m}
\sum\limits_{k=1}^\infty
\fr{b(k\,|\,s)}{ B(k\,|\,s)}\,
r_{k-1}(s,m)\,,
\enskip m \ge 1\,.
\label{6.2*.17}
\end{multline}
%%%%%%%%%%%%%%%%%%%%%%

\section{Алгоритм решения системы уравнений}

В этом разделе приведем простой алгоритм чис\-лен\-но\-го
решения сис\-тем линейных алгебраических урав\-не\-ний~(\ref{6.2*.11})--(\ref{6.2*.17}),
который со\-сто\-ит в последовательном по $m$ от $m=0$
до $m\hm=L\hm-1$ вычислении стационарных вероятностей
$p_{k}(l,m)$.

Начнем с определения $p_{k}(l,0)\hm=p_{k,1}(l)$.

Для сокращения записи введем обозначения:
\begin{alignat*}{2}
x_l &= p_{0}(l,0)\,, &\enskip l&=\overline{1,L}\,;
\\
b_l &= q(l,0) = a b(l)\,, &\enskip l&=\overline{1,L}\,;
\\
\beta^*_l &= \beta^*([1 - a B(L-l)]\,|\,l)\,, &\enskip l&=\overline{1,L}\,;
\\
\beta_l &= \beta([1 - a B(L-l)]\,|\,l)\,, &\enskip  l&=\overline{1,L}\,;
\\
y_l &= \sum\limits_{s=1}^{L-l} \beta^*([1 - a B(L-s)]\,|\,s) p_{0}(s,0)\,, &\enskip l&=\overline{1,L}\,;
\\
y &= \sum\limits_{s=1}^{L} \beta([1 - a B(L-s)]\,|\,s) p_{0}(s,0) \,.&&
\end{alignat*}
Тогда систему~(\ref{6.2*.15}) можно записать следующим
образом:
%%%%%%%%%%%%%%%%%%%%%
\begin{gather}
\label{6.3*.1}
x_{l}= b_l p_0 + b_l y_l + b_l y \,,\enskip l=\overline{1,L}\,;
\\
\label{6.3*.2}
y_l = \sum\limits_{s=1}^{L-l} \beta^*_s x_s \,,\enskip l=\overline{1,L-1}\,;
\ y_{L} = 0\,;
\\
\label{6.3*.3}
y= \sum\limits_{s=1}^{L} \beta_s x_s \,.
\end{gather}
Заметим, что при $l=\overline{2,L-1}$ каждое $y_l$
представимо в виде:
\begin{equation}
\label{6.3*.4}
y_l = y_{l-1} - \beta^*_{L-l+1} x_{L-l+1}\,,
\end{equation}
а при $l=\overline{1,L-2}$~--- в виде:
\begin{equation}
\label{6.3*.5}
y_l = y_{l+1} + \beta^*_{L-l} x_{L-l}\,.
\end{equation}

Сначала выразим $x_l$, $l\hm=\overline{1,L}$, и
$y_l$,  $l\hm=\overline{2,L-1}$, через $y_1$ и $y$ по формулам:
%%%%%%%%%%%%%%%%%%%%%
\begin{gather*}
x_{l} = c_l + d_l y_1 + e_l y \,,\enskip l=\overline{1,L}\,;
\\
y_{l} = f_l + g_l y_1 + h_l y \,,\enskip l=\overline{2,L-1}\,.
\end{gather*}
%%%%%%%%%%%%%%%%%%%%%%%%%%%%%%%%%%%%%%%%
Подставляя в (\ref{6.3*.1}) $l=L$, имеем:
%%%%%%%%%%%%%%%%%%%%%
\begin{equation*}
x_{L} = b_L p_0 + b_L y \,,
\end{equation*}
т.\ е.
\begin{equation*}
c_{L} = b_L p_0\,;
\enskip d_{L} = 0\,;
\enskip
e_L= b_L \,.
\end{equation*}
%%%%%%%%%%%%%%%%%%%%%%%%%%%%%%%%%%%%%%%
При $l=1$ из (\ref{6.3*.1}) и~(\ref{6.3*.2}) находим:
\begin{gather*}
x_{1} = b_{1} p_0 + b_{1} y_1 + b_{1} y\,;
\enskip 
y_{L-1}= b^*_{1} x_{1} \,;
\\
c_{1} = b_{1} p_0\,; \enskip d_{1}=b_{1}\,;
\enskip 
e_{1} =b_{1}\,;\enskip
f_{L-1}=b^*_{1} c_{1} \,;
\\
g_{L-1}=b^*_{1} d_{1}\,;
\enskip
h_{L-1}= b^*_{1} e_{1}\,.
\end{gather*}
При $l=L-1$ из (\ref{6.3*.1}) и (\ref{6.3*.4}) получаем:

\noindent
\begin{gather*}
x_{L-1} = b_{L-1} p_0 + b_{L-1} y_{L-1} + b_{L-1} y\,;
\\
y_{2}= y_{1} - \beta^*_{L-1} x_{L-1}\,;
\\
c_{L-1} = b_{L-1} p_0 + b_{L-1} f_{L-1} \,;
\enskip
d_{L-1} = b_{L-1} g_{L-1}\,;
\\
e_{L-1}=b_{L-1} h_{L-1} + b_{L-1}\,;
\\
f_{2} = - b^*_{L-1} c_{L-1}\,;
\enskip
g_{2}=1 - b^*_{L-1} d_{L-1}\,;
\\
h_{2}=- b^*_{L-1} e_{L-1}\,.
\end{gather*}
При $l=2$ из (\ref{6.3*.1}) и~(\ref{6.3*.5}) имеем:

\noindent
%%%%%%%%%%%%%%%%%%%%%
\begin{gather*}
x_{2} = b_{2} p_0 + b_{2} y_2 + b_{2} y\,;
\
y_{L-2} = y_{L-1} + \beta^*_{2} x_{2} \,;
\\
c_{2} =b_{2} p_0 + b_{2} f_{2} \,;
\ 
d_{2} = b_{2} g_{2}\,;
\
e_{2} = b_{2} h_{2} + b_{2}\,;
\\
f_{L-2} = f_{L-1} - b^*_{2} c_{2}\,;
\ 
g_{L-2} = g_{L-1} - b^*_{2} d_{2}\,;
\\
h_{L-2} = h_{L-1} - b^*_{2} e_{2} \,.
\end{gather*}

Продолжая эту процедуру, получаем, что $x_l$
при четном $l$ в пределах от $l\hm=L/2\hm+1$ до $L-2$ и
при нечетном $l$ в пределах от $l\hm=(L+1)/2$ до $L-2$
вычисляются по формулам:

\noindent
\begin{gather*}
x_{L-s} =о b_{L-s} p_0 + b_{L-s} y_{L-s} + b_{L-s} y\,;
\\
y_{s+1} = y_{s} - \beta^*_{L-s} x_{L-s} \,;
\\
c_{L-s} = b_{L-s} p_0 + b_{L-s} f_{L-s} \,;
\enskip 
d_{L-s} = b_{L-s} g_{L-s}\,;
\\ 
e_{L-s} = b_{L-s} h_{L-s} + b_{L-s}\,;
\\
f_{s+1} = f_{s} - b^*_{L-s} c_{L-s}\,;
\enskip
g_{s+1} = g_{s} - b^*_{L-s} d_{L-s}\,;
\\
h_{s+1} = h_{s} - b^*_{L-s} e_{L-s}\,,
\end{gather*}
а при четном $l$ в пределах от $l\hm=3$ до $L/2$ и при
нечетном~$l$ в пределах от $l\hm=1$ до $(L-1)/2$~---
по формулам:

\noindent
%%%%%%%%%%%%%%%%%%%%%%%%%%%%%%%%%
\begin{gather*}
%\label{6.2.6}
x_{s}=b_{s} p_0+b_{s} y_s+b_{s} y\,;
\enskip 
y_{L-s}=y_{L-s+1} + \beta^*_{s} x_{s}\,;
\\
c_{s} = b_{s} p_0 + b_{s} f_{s} \,;
\enskip 
d_{s} = b_{s} g_{s}\,;
\enskip
e_{s} = b_{s} h_{s} + b_{s}\,;
\\
f_{L-s} =f_{L-s+1} - b^*_{s} c_{s}\,;
\enskip 
g_{L-s} = g_{L-s+1} - b^*_{s} d_{s}\,;
\\ 
h_{L-s} = h_{L-s+1} - b^*_{s} e_{s}\,.
\end{gather*}

Подставляя найденные значения $x_l$ в равенство~(\ref{6.3*.2}) при $l=1$ и в 
равенство~(\ref{6.3*.3}), приходим к сис\-те\-ме из двух линейных алгебраических
уравнений относительно $y_1$ и $y$, решая которую,
находим эти величины, затем стационарные
ве\-ро\-ят\-ности $x_l=p_{0}(l,0)$,\ \ $l=\overline{1,L}$,
и далее с по\-мощью формул~(\ref{6.2*.11}) и~(\ref{6.2*.12}) стационарные вероятности
$p_{k}(l,0)$, $l\hm=\overline{1,L}$, $k\hm\ge 1$.

Предположим теперь, что вероятности $p_{k}(l,j)$,
$k\hm\ge0$, $l\hm=\overline{1,L-j}$,
$j\hm=\overline{j-1,L}$,
уже найдены для всех $j\hm=\overline{0,m-1}$,
$m\hm=\overline{0,L-1}$.
Найдем эти вероятности для~$m$.

Как и прежде, для сокращения записи введем обозначения:

\noindent
\begin{alignat*}{2}
x_l &= p_{0}(l,m)\,,&\enskip l&=\overline{1,L-m}\,;
\\
b_l&=a b(l)\,,&\enskip l&=\overline{1,L-m}\,;
\end{alignat*}


\vspace*{-3pt}

\pagebreak

\noindent
\begin{alignat*}{2}
\beta^*_l&=\beta^*([1 - a B(L-l-m)]\,|\,l)\,,&\enskip l&=\overline{1,L-m}\,;
\\[6pt]
\beta_l&=\beta([1 - a B(L-l-m)]\,|\,l)\,,&\enskip l&=\overline{1,L-m}\,;
\end{alignat*}

\vspace*{-12pt}

\noindent
\begin{align*}
y_l&=\sum\limits_{s=1}^{L-l-m}\beta^*([1 - a B(L-s-m)]\,|\,s)
p_{0}(s,m)\,,\\[1pt]
&\hspace*{50mm}l=\overline{1,L-m}\,;\\
y&=\sum\limits_{s=1}^{L-m}\beta([1 - a B(L-s-m)]\,|\,s)p_{0}(s,m)\,;
\end{align*}

\vspace*{-12pt}

\noindent
\begin{multline*}
%\label{6.6.15}
\tilde b_l=q(l,m)={}\\[2pt]
{}=a b(l)\left[\sum\limits_{s=1}^{L-l-m}
\sum\limits_{k=1}^\infty\fr{B(k+1\,|\,s) }{B(k\,|\,s)}\,
r_{k-1}(s,m) +
{}\right.\\[2pt]
\left.{}+
\sum\limits_{s=1}^{L-m} \sum\limits_{k=1}^\infty
\fr{b(k\,|\,s)}{ B(k\,|\,s)}\,
r_{k-1}(s,m)
\right]\,,\enskip l=\overline{1,L-m}\,.
\end{multline*}
При этом, как видно из~(\ref{6.2*.13}), (\ref{6.2*.14}),
(\ref{6.2*.16}) и~(\ref{6.2*.17}), $\tilde b_l$
выражается через уже известные величины.

С учетом введенных обозначений
сис\-те\-ма~(\ref{6.2*.15}) записывается сле\-ду\-ющим об\-разом:
\begin{gather}
\label{6.3*.8}
x_{l}= \tilde b_l + b_l y_l + b_l y \,,\enskip l=\overline{1,L-m}\,;
\\
\!y_l = \sum\limits_{s=1}^{L-l-m} \beta^*_s x_s\,,\
l=\overline{1,L-1-m}\,;\  y_{L-m} = 0\,;\!\!
\label{6.3*.9}
\\
\label{6.3*.10}
y= \sum\limits_{s=1}^{L-m} \beta_s x_s \,.
\end{gather}

Нетрудно видеть, что при каждом фиксированном~$m$
алгоритм решения сис\-те\-мы~(\ref{6.3*.8})--(\ref{6.3*.10})
полностью совпадает с алгоритмом решения сис\-те\-мы~(\ref{6.3*.1})--(\ref{6.3*.3}).

Оставшиеся неизвестными вероятности $p_{k}(l,m)$,
$k\hm\ge1$, вычисляются по формуле~(\ref{6.2*.11}).

Приведенный здесь алгоритм позволяет вы\-чис\-лить
стационарные вероятности $p_{k,i}(l,m)$ с точ\-ностью
до вероятности~$p_0$, которая, как уже говорилось,
определяется из условия нормировки
$$
p_0+ \sum\limits_{k=0}^{\infty} \sum\limits_{m=0}^{L-1}
\sum\limits_{l=1}^{L-m} p_{k}(l,m)
= 1 \,.
$$

\section{Некоторые стационарные показатели,
связанные с~числом заявок в~системе}

Выпишем выражения для некоторых стационарных
характеристик, связанных со стационарными
вероятностями со\-сто\-яний.

Стационарная вероятность $p(m)$, $m\hm= \overline{1,L}$,
того, что суммарный объем находящихся в сис\-те\-ме заявок
равен~$m$, задается формулой
\begin{equation*}
p(m)= \sum\limits_{k=0}^{\infty} \sum\limits_{l=1}^{m-1}
p_{k}(l,m-l)\,,
\enskip m= \overline{1,L}\,.
\end{equation*}

Стационарная вероятность $p^*_{0}$ того, что в момент
поступления новой заявки система будет свободна (в том
числе на приборе закончится обслуживание единственной
находящейся в сис\-те\-ме заявки), задается формулой
\begin{equation*}
p^*_{0} = p_0 + \sum\limits_{k=1}^\infty \sum\limits_{l=1}^{L}
\fr{b(k\,|\,l)}{ B(k\,|\,l)} \, p_{k-1}(l,0) \,.
\end{equation*}

Стационарная вероятность
$p^*_{k}(l,m)$, $m\hm= \overline{0,L-1}$,
$l\hm= \overline{1,L-m}$, $k\hm\ge 1$,
того, что поступающая (не обязательно принятая в сис\-те\-му)
заявка застанет в сис\-те\-ме, по крайней мере, одну заявку,
причем длина и объем заявки на приборе равны $k$ и $l$
и она продолжит обслуживаться, а суммарный
объем остальных находящихся в сис\-те\-ме заявок равен~$m$,
определяется выражением:
\begin{multline*}
p^*_{k}(l,m)= \fr{B(k+1\,|\,l)}{B(k\,|\,l)}
\, p_{k-1}(l,m)\,,\\
m= \overline{0,L-l}\,,
\enskip
 l= \overline{1,L-m}\,,
\enskip  k\ge 1\,.
\end{multline*}

Стационарная вероятность
$p^*(m)$\,, $m\hm= \overline{1,L-1}$,
того, что в момент поступления новой заявки на приборе
закончится обслуживание заявки, а суммарный объем
оставшихся заявок будет равен~$m$, задается формулой
%%%%%%%%%%%%%
\begin{equation*}
p^*(m)= \sum\limits_{k=1}^\infty \sum\limits_{l=1}^{L-m}
\fr{b(k\,|\,l)}{ B(k\,|\,l)} \, p_{k-1}(l,m)\,,
\ \ m= \overline{1,L-1}\,.
\end{equation*}

Наконец, стационарная вероятность $\pi(l)$ того, что поступающая
заявка объема~$l$ будет принята в сис\-те\-му, и стационарная
вероятность~$\pi$ того, что по\-сту\-па\-ющая заявка
произвольной длины будет принята в сис\-те\-му, имеют вид:

\noindent
\begin{align*}
\pi(l)&= p^*_0 + \sum\limits_{k=1}^\infty \sum\limits_{m=1}^{L-l}
\sum\limits_{j=1}^{m} p^*_{k}(j,m-j) +{}\\
&\hspace*{15mm}{}+\sum\limits_{m=1}^{L-l} p^*(m)\,,\  l= \overline{1,L}\,;
\\
\pi &= \sum\limits_{l=1}^{L} b(l) p^*_0(l) \,.
\end{align*}

\section{Стационарное распределение времени
пребывания заявки в~системе}


Будем называть $M$-системой систему, аналогичную
исходной, но с ограничением~$M$, ${1\hm\le M\hm\le L}$,
на суммарный объем заявок и вероятностью~$a$
поступления заявки на такте. Нетрудно видеть, что $M$-сис\-те\-ма представляет собой
исходную СМО, но при условии, что в ней постоянно
находятся заявки суммарного объема $L-M$.


Обозначим через $g_i(k,l;M)$, $i\hm\ge 1$,
$k\hm\ge 0$, $l\hm=\overline{1,M}$,
вероятность того, что период занятости (ПЗ) $M$-сис\-те\-мы,
открываемый заявкой (обслуженной) длины~$k$ и объема~$l$,
продлится $i$ тактов. Для $g_i(k,l;M)$ справедливы следующие соотношения:
\begin{multline}
\label{4.1}
g_1(k,l;M) = [1 - a B(M)]\fr{b(k+1\,|\,l)}{B(k+1\,|\,l)}\,,\\
l= \overline{1,M}\,,\enskip  k\ge 0\,;
\end{multline}

\vspace*{-9pt}

\noindent
\begin{multline}
\label{4.2}
g_2(k,l;M)={}\\
{}= [1 - a B(M-l)]\fr{B(k+2\,|\,l) }{B(k+1\,|\,l)}
\, g_{1}(k+1,l;M)+{}\\
{}+
a \fr{b(k+1\,|\,l) }{ B(k+1\,|\,l)} \sum\limits_{m=1}^{M}
b(m) g_{1}(0,m;M)\,,\\  l= \overline{1,M}\,,\enskip  k\ge 0\,;
\end{multline}

\vspace*{-12pt}

\noindent
\begin{multline}
\label{4.3}
g_i(k,l;M) ={}\\
{}= [1 - a B(M-l)] \fr{B(k+2\,|\,l) }{ B(k+1\,|\,l)}
\, g_{i-1}(k+1,l;M)  +{}
\\
{}+
a \fr{B(k+2\,|\,l) }{ B(k+1\,|\,l)} \sum\limits_{m=1}^{M-l}
b(m) \times{}\\
{}\times
\sum\limits_{j=1}^{i-2} g_{j}(k+1,l;M-m)
g_{i-1-j}(0,m;M)  +{}\\
{}+
a \fr{b(k+1\,|\,l) }{ B(k+1\,|\,l)} \sum\limits_{m=1}^M
b(m) g_{i-1}(0,m;M)\,,\\
\  l= \overline{1,M}\,,\enskip  i\ge 3\,,\enskip  k\ge 0\,.
\end{multline}
%%%%%%%%%%%%%%%%%%%%
В последнем соотношении принято соглашение, что
$\sum\limits_{j=1}^{0} (\cdot) \hm= 0$.

Система уравнений~(\ref{4.1})--(\ref{4.3}) задает
рекуррентную по $M$ от $M\hm=1$ до $M\hm=L$ процедуру
определения вероятностей $g_i(k,l;M)$, которые из этих уравнений
вычисляются при каждом~$M$ последовательно по~$i$ от
$i\hm=1$ для всех возможных значений $k$ и~$l$.

Обозначим через
$w_k(l)$, $l\hm= \overline{1,L}$, $k\hm\ge 0$,
стационарную вероятность того, что заявка объема~$l$
будет принята в систему и будет ожидать начала
обслуживания $k$ тактов,
а через $v_k(l)$, $l\hm= \overline{1,L}$, $k\hm\ge 1$,~---
стационарную вероятность того, что заявка объема~$l$
будет принята в систему и будет находиться в сис\-те\-ме
$k$ тактов. Тогда
\begin{align*}
w_0(l) &= p^*_0 + \sum\limits_{m=1}^{L-l} p^*(m)\,,\quad\ \ l= \overline{1,L}\,,
\\
w_k(l) &= \sum_{i=1}^{\infty} \sum\limits_{j=1}^{L-l} \sum\limits_{m=0}^{L-l-j} p^*_{i}(j,m) 
g_{k}(i,j;L-m-l)\,,\\
&\hspace*{36mm}  l= \overline{1,L}\,,\  k\ge 1\,,
\\
v_k(l) &= \sum\limits_{i=0}^{k-l} w_i(l) b(k-i\,|\,l) \,,\  l= \overline{1,L}\,,\  k\ge 1\,.
\end{align*}
%%%%%%%%%%%%%%%%%%%%%%%%%%%%%%%%%

Наконец, обозначая через $w_k$, $k\hm\ge 0$ условную
стационарную вероятность того, что заявка объема~$l$,
принятая в сис\-те\-му, будет ожидать начала обслуживания $k$ тактов
и через $v_k$, $k\hm\ge 1$,~--- стационарную вероятность
того, что эта заявка будет находиться в сис\-те\-ме $k$~тактов, имеем:
\begin{align*}
w_k&= \fr{1}{\pi} \sum\limits_{l=1}^{L} b(l) w_{k}(l) \,,\ \ k\ge 0\,,
\\
v_k &= \fr{1}{\pi} \sum\limits_{l=1}^{L} b(l) v_k(l) \,,\ \ k\ge 1\,.
\end{align*}

\section{Заключение}

В настоящей статье получены математические
соотношения, позволяющие вычислять основные стационарные
характеристики функционирующей в дискретном времени СМО,
в которой каждая заявка наряду с длиной (временем
обслуживания) имеет (дискретный) случайный объем и суммарный объем находящихся в
системе заявок ограничен. Приведены просто реализуемые алгоритмы для численных
расчетов по этим соотношениям.

{\small\frenchspacing
{%\baselineskip=10.8pt
\addcontentsline{toc}{section}{Литература}
\begin{thebibliography}{99}


\bibitem{romm}
\Au{Ромм Э.\,Л., Скитович В.\,В.}
Об одном обобщении задачи Эрланга~//
Автоматика и телемеханика, 1971. №\,6. С.~164--167.

% 2.
\bibitem{alex}
\Au{Александров А.\,М., Кац Б.\,А.}
Обслуживание потоков неоднородных требований~//
Изв.\ АН СССР. Технич.\ кибернетика, 1973. №\,2. С.~47--53.

% 3.
\bibitem{tich}
\Au{Тихоненко О.\,М.}
Модели массового обслуживания в системах обработки
информации.~--- Минск: Университетское, 1990.

% 4.
\bibitem{pech2} 
\Au{Печинкин А.\,В., Печинкина О.\,А.}
Система $M_k/G/1/n$ с дисциплиной LIFO с прерыванием и
ограничением на суммарный объем требований~//
Вестник Российского ун-та дружбы народов.
Сер.\ Прикладная математика и информатика, 1996. №\,1.
С.~86--93.

% 5.
\bibitem{pech3} 
\Au{Печинкин А.\,В.}
Система обслуживания с дисциплиной LIFO и ограничением
на суммарный объем требований~//
Вестник Российского ун-та дружбы народов.
Сер.\ Прикладная математика и информатика, 1996. №\,2.
С.~85--99.

% 6.
\bibitem{pech1} 
\Au{Печинкин А.\,В.}
Система $M_l/G/1/n$ с дисциплиной LIFO и ограничением на
суммарный объем требований~//
Автоматика и телемеханика, 1998. №\,4. С.~106--116.

% 7.
\bibitem{pech4} 
\Au{Абрамушкина Т.\,В., Апарина С.\,В.,
Кузнецова Е.\,Н., Печинкин А.\,В.}
Численные методы расчета стационарных вероятностей
состояний системы $M/G/1/n$ с дисциплиной LIFO\ PR\/
и ограничением на суммарный объем требований~//
Вестник Российского ун-та дружбы народов. Сер.\ Прикладная математика и
информатика, 1998. №\,1. С.~40--47.



% 8 - new
\bibitem{new} 
\Au{Manzo R., Cascone A., Razumchik R.\,V.} 
Exponential queuing system with negative customers and 
bunker for ousted customers~// 
Automation Remote Control, 2008.
Vol.\,69. No.\,9. P.~1542--1551.




% 8.
\bibitem{cmps} 
\Au{Cascone A., Manzo R., Pechinkin A.\,V., Shorgin~S.\,Ya.} 
A~Geo$_m/G/1/n$ queueing system with LIFO discipline, service
interruptions and resumption, and restrictions on the total volume
of demands~// World Congress on Engineering 2010 Proceedings.
Vol.~III. WCE 2010.~--- London, U.K., 2010. P.~1765--1769. 
ISBN (Vol.~III):  978-988-18210-8-9
ISSN: 2078-0958 (Print)
ISSN: 2078-0966 (Online).


% 9.
\bibitem{cmps-2} 
\Au{Pechinkin A., Shorgin~S.}
A $Geo_m/G/1/n$ queueing system with LIFO discipline, service
interruptions and repeat again service, and restrictions on the
total volume of demands~// Multiple Access Communication (MACOM
2010): Proceedings of the 3rd  Workshop (International).~---
Barcelona, Spain, 2010. P.~98--106.


% 10.
%\bibitem{kmps}
%\Au{Касконе А., Мандзо~Р., Печинкин~А.\,В.,
%Шоргин~С.\,Я.}
%Система $Geo_m/G/1/n$ с дисциплиной LIFO без прерывания
%обслуживания и ограничением на суммарный объем заявок~//
%Автоматика и телемеханика, 2011. №\,1. С.~107--120.
%%%%%
\bibitem{kmps}
\Au{Cascone A., Manzo R., Pechinkin A.\,V., Shorgin S.\,Ya.}  
Geo$_{m}/G/1/n$ system with LIFO discipline without 
interrupts and constrained total amount of customers~// 
Automation Remote Control, 2011.
Vol.\,72, No.\,1. P.~99--110.




\label{end\stat}

% 11.
\bibitem{BDPS}
\Au{Bocharov P.\,P., D'Apice~C., Pechinkin~A.\,V.,
Salerno~S.}
Queueing theory. Modern probability and
statistics ser.~--- Utrecht, Boston: VSP Publ., 2004.
 \end{thebibliography}
}
}


\end{multicols}