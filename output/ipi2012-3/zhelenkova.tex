\def\stat{zhelenkova}

{\begin{center}
{\Large
Статьи, являющиеся развитием докладов, %}\\[6pt]
%{\Large 
представленных }\\[6pt]
{\Large  на конференции <<Электронные библиотеки:}\\[6pt]
{\Large перспективные методы и технологии, %}\\[6pt]
%{\Large 
электронные коллекции>>}\\[9pt]
{\large (RCDL'2011, г.~Воронеж, 19--22~октября 2011~г.)}
\end{center}
}

\def\tit{ИССЛЕДОВАНИЕ РАДИОИСТОЧНИКОВ СРЕДСТВАМИ ВИРТУАЛЬНОЙ 
ОБСЕРВАТОРИИ$^*$}

\def\titkol{Исследование радиоисточников средствами виртуальной 
обсерватории}

\def\autkol{О.\,П.~Желенкова}
\def\aut{О.\,П.~Желенкова$^1$}

\titel{\tit}{\aut}{\autkol}{\titkol}

{\renewcommand{\thefootnote}{\fnsymbol{footnote}}\footnotetext[1]
{Работа поддержана грантом РФФИ №\,10-07-00412.}}


\renewcommand{\thefootnote}{\arabic{footnote}}
\footnotetext[1]{Специальная астрофизическая обсерватория РАН, zhe@sao.ru}

\vspace*{6pt}


\Abst{В течение ряда лет с использованием разных подходов на базе средств виртуальной обсерватории в САО 
РАН проводились исследования радиоисточников обзоров, выполненных на крупнейшем российском 
радиотелескопе РАТАН-600 в 1980--1999~гг. Проведено их массовое отождествление с максимальным 
использованием имеющихся в открытом доступе данных разных диапазонов электромагнитного спектра. 
С~применением программного инструментария виртуальной обсерватории реализован подход по автоматической 
подготовке и предварительной обработке данных. Для полученного компилятивного каталога разработана 
ин\-фор\-ма\-ци\-он\-но-поиско\-вая сис\-те\-ма, которая применялась при анализе информации о каждом источнике и 
принятии решения об отождествлении. Исходя из полученного опыта при работе c многочисленными 
разнородными ресурсами, можно подытожить, что программные средства виртуальной обсерватории 
обеспечивают удобный доступ к астрономическим данным и существенно повышают эффективность научных 
исследований. Однако все еще нет развитого инструментария для дальнейшего анализа, актуализации и 
публикации собранных исследователем данных. Рядом проектов ведутся разработки по реализации большей 
связности данных на базе уже существующих веб-технологий, что переведет сервисы виртуальной 
обсерватории на новый уровень, обеспечивающий обмен знаниями посредством аннотирования записей 
каталогов и реализацией связей между ними.}

\vspace*{2pt}

\KW{виртуальная обсерватория; распределенные информационные системы; информационные технологии в 
научных исследованиях; интеграция неоднородных информационных ресурсов; базы данных}

\vskip 14pt plus 9pt minus 6pt

      \thispagestyle{headings}

      \begin{multicols}{2}

            \label{st\stat}

\section{Введение}
        
      Сегодня виртуальная обсерватория~--- прежде всего средство удобного и эффективного 
доступа к разнообразным астрономическим данным. Первые шаги в сторону организации 
этой распределенной инфраструктуры были сделаны в 1990-х гг., когда в 
США была создана сеть центров данных для поддержки информации, полученной 
космическими миссиями НАСА. Достижения в области информационных технологий 
обеспечили основу, на которой распределенные коллекции данных стали рассматриваться 
как интегрированная информационная система. Виртуальная обсерватория открыла новые 
направления научных исследований, опирающиеся на статистический анализ, поиск новых 
закономерностей и объединение данных разных диапазонов.
      
      Обычной практикой в астрономии были отдельные и/или повторные наблюдения 
индивидуальных объектов, что хорошо работало при открытии фундаментальных законов. 
Но по мере того как понимание разных астрономических феноменов и закономерностей 
становилось более точным, чис\-ло вопросов, на которые можно ответить с по\-мощью одного 
наблюдения, становилось все меньшим. Методы наблюдений в настоящее время все больше 
смещаются в сторону накопления больших объемов данных, а подход к решению 
астрофизических задач~--- к статистическим методам. Большой объем данных может 
привести к обнаружению процессов, чьи наблюдательные проявления маскируются разными 
эффектами, которые трудно бывает учесть из-за недостаточного количества данных. 
Наблюдательное время самых мощных инструментов было, остается и будет оставаться 
весьма ограниченным, поэтому многие астрофизические вопросы, для решения которых 
требуется большое количество наблюдательных данных, час\-то прос\-то не могут 
рассматриваться. 
      
      Взрывообразный рост объема и сложности данных вызван прогрессом в получении 
цифровых изоб\-ра\-же\-ний (основной источник данных в астрономии), способов обработки, 
хранения и доступа к информации. В~астрономии происходит сдвиг в сторону науки, 
базирующейся на обзорах, которые становятся все более важным методом в исследовании 
Вселенной. Сейчас имеются средства для проведения обзоров практически во всем 
диапазоне электромагнитного спектра, пространственных масштабов и временных эпох. 
Каталоги, по\-лу\-ча\-емые из панхроматических обзоров, дают возможности для обнаружения 
новых явлений, которые могут фундаментально изменить наши представления о физике 
звезд и их эволюции, ближнем космосе и планетных сис\-те\-мах, формировании галактик и 
природе активных ядер галактик. Например, изоб\-ра\-же\-ние одного и того же участка неба в 
оптическом и радиодиапазоне привело к открытию квазаров в\linebreak 1960-е~гг., а 
данные в инфракрасном диапазоне позволили исследовать скрытые от наблюдателя \mbox{пылью} 
области звездообразования и активные ядра галактик, чего невозможно сделать по 
оптическим изображениям. 

Повторные наблюдения областей неба привели к открытию 
транзиентных событий~--- сверхновых и более редких явлений~--- микролинзирования. 
Панхроматические наборы данных позволяют сравнивать теоретические модели и реальные 
данные. \mbox{Такие} исследования предъявляют определенные требования как к постановке 
задачи, так и к методам их решения, которые все больше опираются на информационные 
технологии и в первую очередь на веб-сер\-ви\-сы, системы управления базами данных, грид 
и облачные вычисления. 
      
      Исследования радиогалактик важны для понимания механизмов излучения 
внегалактических объектов в радиодиапазоне и относятся к широкому кругу задач, 
связанному с изучением проявлений активности галактических ядер. Феномен мощного 
радиоизлучения ядра галактики является кратковременной по космологическим масштабам\linebreak 
(до $\sim10^8$~лет) эволюционной фазой самых массивных звездных систем. Хотя мощные 
радио\-га\-лактики~--- редкие объекты (пространственная плот\-ность оценивается в 
      $\sim10^{-6}$~Мпс$^{-3}$), их наблюдение в радиодиапазоне возможно практически на 
любых космологических расстояниях, что используется при изучении крупномасштабной 
структуры Вселенной, проверке гипотез формирования самых первых звездных систем и 
решении других космологических задач.
      
      Отождествление радиоисточников с объектами в других диапазонах~--- обязательная 
процедура при многочастотных исследованиях, и не такая прос\-тая, как это кажется на 
первый взгляд. Кросс-иден-\linebreak ти\-фи\-ка\-ция по координатам (около объекта в обла\-сти с 
заданным радиусом ищется объект другого каталога) оптических и радиокаталогов при 
разном угловом разрешении, предельной чув\-ст\-ви\-тель\-ности и координатной точности 
последних, а также морфологической структуре самих источников дает от 5\% до 30\% 
совпадений. Отметим, что только по оптическим данным можно оценить расстояние до 
родительских галактик радиоисточников, которые по большей час\-ти~--- весь\-ма слабые 
оптические объекты, особенно на больших красных смещениях. Их оптические наблюдения 
требуют больших затрат наблюдательного времени на крупнейших оптических телескопах 
при наилучших погодных условиях.
      
      В качестве примера приведем результаты исследований выборки источников с 
крутыми спектрами (SS, {Steep Spectra}) из каталога RC ({RATAN Cold}), 
полученного по материалам глубокого обзора полоски неба на радиотелескопе 
      РАТАН-600~[1--3]. Для кандидатов в выборку учитывались угловые размеры, 
морфологическая структура, а также яркость объекта в радиодиапазоне. Так из $\sim1000$ 
источников каталога~RC в выборку с крутыми спектрами вошли $\sim100$~объектов. 
Потребовалось 15~лет фотометрических и спектральных наблюдений на 6-м оптическом 
телескопе БТА (Большой телескоп азимутальный), чтобы отождествить и получить спектры 
для объектов выборки~[4]. Из этих объектов у четырех источников оказалось $1 \leq Z<2$, у 
трех $2\leq Z<3$, у одного радиоисточника $3\leq Z< 4$ и самый далекий объект выборки с 
$Z=4.51$. 
      
      Не только поиск далеких радиогалактик, но и статистические свойства 
радиоисточников в разных диапазонах электромагнитного спектра важны для понимания 
природы активных галактических ядер. Массовое исследование радиоисточников позволяет 
уточнять существующие и открывать новые селекционные критерии, которые можно 
использовать при классификации этих объектов. И такие исследования проводятся с 
привлечением современных цифровых обзоров в разных диапазонах. 
      
      На крупнейшем российском радиотелескопе РАТАН-600 в 1980--1999~гг.\ была 
проведена серия глубоких обзоров полосы неба шириной около 40~угловых минут. По 
данным этих обзоров получен каталог~RC, а затем RCR ({RATAN Cold 
Revised})~[5]. С~появлением глубоких цифровых обзоров в оптическом и инфракрасном 
диапазоне, таких как SDSS ({Sloan Digital Sky Survey})~[6] и UKIDSS ({United 
Kingdom Infra-red Deep Sky Survey})~[7], появилась возможность провести отождествление 
этих каталогов. Для выполнения этой задачи были максимально использованы все 
имеющиеся в открытом доступе данные~--- два оптических обзора: DSS2 ({Digital 
Sky Survey}) и SDSS (полосы $u$, $g$, $r$, $i$, $z$), включая каталоги GSC ({Guide 
Star Catalog})~[8] и USNO-B1~[9], обзоры ближнего инфракрасного диапазона 2MASS 
({Two Micron All Sky Survey})~[10] и UKIDSS (полосы~$J$, $H$, $K$), а также 
проведены многочастотные исследования радиоисточников по радиообзорам VLSS 
({VLA Low-frequency Sky Survey}, 74~МГц)~[11], TXS ({Texas Survey of radio 
sources}, 365~МГц)~[12], NVSS ({NRAO-VLA Sky Survey}, 1,4~ГГц)~[13], FIRST 
({Faint Images of the Radio Sky at Twenty centimeters}, 1,4~ГГц)~[14], GB6 
({Green Bank survey}, 4,85~ГГц)~[15]. Радиоисточники каталога RC~[16--18], а затем 
RCR~[19, 20] были отождествлены с данными этих обзоров. Радиоисточники, у которых не 
обнаруживались кандидаты в оптических/инфракрасных каталогах, дополнительно 
отождествлялись с суммарными изображениями обзора SDSS в трех фильтрах ($g$, $r$, 
$i$) и/или инфракрасного обзора UKIDSS в фильтрах ($J$, $H$, $K$) для достижения 
более глубокого предела кадров. 
      
      Для многочастотного исследования выборки источни\-ков каталога~RC, а затем 
каталога~RCR разработана методика детального отождествления радиоис\-точников, 
включающая подбор информационных ресурсов, автоматическую подготовку данных из 
выбранных ресурсов для каждого источника, морфологическую классификацию, визуальную 
инспекцию подготовленных данных для принятия решения об отождествлении~[21, 22]. 
Средствами интерактивного атласа неба Aladin~[23] (программного интерфейса Perl для 
командного интерфейса и макроконтроллера), а также с по\-мощью программного интерфейса 
Python к SAOImage DS9~[24] реализованы потоки работ по списку\linebreak радиоисточников для 
подготовки данных и визуализации результатов. Для полученного компилятивного каталога 
разработана ин\-фор\-ма\-ци\-он\-но-по\-иско\-вая сис\-те\-ма, которая использовалась при\linebreak 
отождествлении радиоисточников. 

\vspace*{-9pt}

\section{Программные средства виртуальной обсерватории}

\vspace*{-2pt}

      Сейчас активно развиваются программные средства (протоколы, метаданные и 
программы, функционирующие на их основе), учитывающие особенности информации, 
относящейся к определенной сфере человеческой деятельности, в частности к научным 
исследованиям. Такие разработки в астрономии объединены в виртуальную организацию, 
которая носит название виртуальной обсерватории и координируется международным 
альянсом IVOA ({International Virtual Observatory Alliance})~[25--27]. В~IVOA 
объединены виртуальные обсерватории разных стран, включая и Российскую виртуальную 
обсерваторию~[28]. Рабочими группами IVOA ведется разработка стандартов более чем по 
десятку направлений, среди которых: представление и формализация данных и знаний 
предметной области, разработка протоколов доступа к данным, стандарты программных 
сервисов для распределенных вычислений, протокол обмена сообщениями для программных 
клиентских приложений виртуальной обсерватории, описание и публикация ресурсов, 
формат обмена данными, язык запросов, поддержка сохранности данных и~пр. С~момента 
появления в 2003~г.\ альянса IVOA разработано около полусотни спецификаций 
протоколов, форматов и соглашений, используемых при создании программных продуктов 
виртуальной обсерватории. Подробный обзор стандартов виртуальной обсерватории и 
применяемых технологий можно найти в обзоре Брюхова и~др.~[28].

\vspace*{-2pt}

\subsection{Текущий статус}

      Инфраструктура виртуальной обсерватории является 
сер\-вис\-но-ориен\-ти\-ро\-ван\-ной. Веб-сер\-ви\-сы IVOA разделены на три класса: 
обнаружение и пуб\-ликация ресурсов, передача данных и организация\linebreak запросов, а также 
сервисы для распределенных вычислений. Обнаружение данных выполняется через регистры 
виртуальной обсерватории. Для сервисов, предоставляемых регистрами виртуальной 
обсерватории, и спецификаций описания ресурсов были рассмотрены несколько 
индустриальных стандартов, обеспечивающих механизмы обмена метаданными в Интернете, 
и был выбран протокол OAI-PMH ({Open Archives Initiative Protocol for 
Metadata Harvesting})~[29]. Для описания астрономических ресурсов (каталогов, цифровых 
обзоров, баз данных, архивов наблюдений, программных средств, функционирующих как 
      веб-сер\-ви\-сы) в регистрах используется стандарт описания сетевых ресурсов 
Dublin Core~[30]. Веб-сер\-ви\-сы ориентированы на то, чтобы операции над данными в сети 
выполнялись без участия человека. Повторное использование простых сервисов и 
комбинирование их для выполнения более сложных действий реализует поток работ. 
Потоковое выполнение использует принципы интероперабельности, когда компоненты 
потока работ взаимодействуют друг с другом посредством протоколов, опре\-де\-ля\-ющих 
правила запуска сервиса и структуру входных и выходных данных. Реализация таких 
протоколов опирается на модели данных. 
      
      Сервисы данных, кроме стандартных графических форматов (gif, jpeg), работают 
с двумя астрономическими форматами~--- FITS ({Flexible Image Transport 
System})~[31], который является с 1982~г.\ астрономическим стандартом для хранения и 
обмена данными, и VOTable~[32]. VOTable-фор\-мат используется в сервисах ВО для 
представления результатов запросов. Основой VOTable является индустриальный стандарт 
XML и опыт разработок астрономических форматов FITS и CDS\ Astrores. 
      
      Астрономы для обозначения одних и тех же физических величин и параметров 
используют \mbox{разные} названия. Чтобы избежать не\-од\-но\-знач\-ности при интерпретации 
величины, необходимо определить, что именно обозначают разные идентификаторы. 
В~VOTable-фор\-ма\-те используется семантический описатель UCD ({Uniform 
Content Descriptor})~[33], который устанавливает смысловую связь между обозначениями 
величин и астрономическими понятиями и/или физическими величинами. IVOA 
поддерживает и контролирует словарь дескрипторов. 
      
      Доступ к данным DAL ({Data Access Layer})~[34] включает стандарты, 
описывающие механизм доступа к распределенным астрономическим данным, и 
программные средства, обеспечивающие такой доступ. Для реализации запросов 
используется расширенное подмножество SQL~--- ADQL ({Astronomical Data 
Query Language})~[35]. Хотя SQL можно использовать для запросов к большинству 
современных астрономических баз данных, астрономическая специфика требует расширения 
возможностей языка. ADQL, кроме координатных запросов, поддерживает доступ по 
протоколам ВО к таблицам, изображениям и спектрам. 
      
      Так в общих чертах можно описать основы виртуальной обсерватории, которая за 
десять с небольшим лет превратилась в действующую инфраструктуру, и для астрономов 
сейчас нет проблем доступа к данным цифровых обзоров неба, архивам наблюдательных 
данных и каталогам. Есть удобные клиентские приложения и программные интерфейсы к 
ним, веб-интерфейсы к основным базам астрономических данных для поиска информации по 
одиночному объекту или списку объектов, запросов по списку объектов и визуализации 
полученной информации. 

\vspace*{-4pt}
  
\subsection{Новые задачи}

\vspace*{-1pt}

      На передний план выходит следующая цель виртуальной обсерватории~--- обеспечить 
профессиональных астрономов возможностью получать информацию о небесных объектах. 
Сейчас это еще в слабой степени решается средствами виртуальной обсерватории, так как 
данные в основном не структурированы и не являются связанной по смысловому 
содержимому информацией. По этой причине невозможно выполнить запрос типа <<найти 
все источники в каталогах, которые являются квазарами>> и~т.\,п. Это ограничивает 
пользователю эффективную работу с информацией. Нет прямого способа воспользоваться 
знаниями об объекте, полученными другими исследователями. 
      
      Вся имеющаяся у астрономического сообщества информация о небесных объектах 
собрана в каталогах и используется для статистических исследований объектов, подбора 
интересующих объектов, для поиска аномальных объектов или объектов с особой 
комбинацией свойств. Каталоги публикуются различными способами: от таблиц в журналах 
до публикации сервисами VizieR~[36]. Самые крупные каталоги доступны через 
специализированные веб-ин\-тер\-фей\-сы для архивов и центров данных, таких как 
WFCAM~[37], SDSS, IRSA~[38], SkyView, MAST~[39] и~др. Есть следующие 
проблемы при работе с каталогами: 
      \begin{itemize} %[1)]
  \item в каталогах содержатся измеренные величины, а результаты их анализа и 
интерпретации пуб\-ли\-ку\-ют\-ся обычно в журнальных статьях. Через гиперссылки, 
предоставляемые информационной системой ADS ({Astrophysics Data System}), в 
которой находится большая часть астрономических полнотекстовых статей, журнальная 
статья может указывать на каталог, используемый в ней, но в архивах данных не всегда 
реализованы аналогичные указатели на литературу;\\[-15pt]
  \item небесные объекты не имеют уникальных идентификаторов. Объединение 
информации в различных диапазонах электромагнитного спектра требует операций 
  кросс-идентификации, при которой строки из двух каталогов, содержащих различную, но 
при связывании вдвойне полезную информацию, объединяются на основе перекрытия 
координатных положений с допусками, учитывающими ошибки сравниваемых каталогов. 
Кросс-идентификация проводится пользователями многократно, поскольку эта важная связь 
между объектами каталогов не сохраняется;\\[-15pt] 
  \item каталоги являются статическими объектами. Ес\-ли создается новый каталог, 
полученный из одного или нескольких существующих каталогов, но с добавлением какой-то 
новой информации, то для него, как правило, не отслеживаются родительские каталоги;\\[-15pt]
  \item поиск данных в каталогах может быть трудоемким, а объединение данных из двух 
каталогов трудоемко и неудобно. Сложно работать с компилятивными каталогами, 
полученными на базе объединения нескольких каталогов разных диапазонов, даже если эта 
информация собирается по небольшому списку объектов;\\[-14pt]
  \item если появляются новые релизы каталогов или новые каталоги, то исследователю 
надо заново\linebreak выполнять одни и те же запросы для ин\-те\-ре\-су\-ющих его объектов. Нет средств 
для оповещения о появлении новой информации и об\-нов\-ле\-ния данных пользователя.
  \end{itemize}
  
\subsection{Новый качественный уровень виртуальной обсерватории}

      В последнее время появилось несколько проектов, направленных на дальнейшее 
развитие инфраструктуры виртуальной обсерватории и, в част\-ности, на решение проблем, 
которые возникают при работе с каталогами. Далее приведем сведения о тех из них, которые, 
вероятнее всего, могут помочь в работе с разнородными данными, полученными при 
массовом отождествлении радиоисточников. В~этих проектах ведутся разработки, связанные 
с внедрением инновационных информационных технологий (грид, облачные вычисления, 
Семантический Веб) в астрофизические исследования.
      
      Цель проекта AstroDAbis~[40]~--- создание независимого механизма публикации 
пояснений (комментариев, аннотаций). Пояснения могут создаваться пользователем для 
одиночного объекта (<<объект X есть квазар>>) или для нескольких объектов (<<объект с 
номером~123 в каталоге~$A$ есть то же самое, что объект с номером~456 в 
каталоге~$B$>>). Как полагают авторы AstroDAbis, этим решаются проблемы передачи 
знаний, создания компилятивных каталогов и реализации их связи с родительскими 
каталогами. Авторы статей, где представлена информация, полученная на основе анализа 
каталогов, с помощью аннотаций могут передать знания о небесном объекте в форме, 
которая может быть использована в последующих запросах к каталогу. Когда возникнет 
потребность объединить два каталога и создать компилятивный каталог (например, слияние 
оптических данных SDSS и инфракрасных данных тех же источников из UKIDSS), такая 
связь позволит обойтись без повторной кросс-иден\-ти\-фи\-ка\-ции ресурсов. С~по\-мощью 
аннотации такие каталоги сохранят связи с исходными каталогами, и связи будут однозначно 
зафиксированы.
      
      Результатом проекта AstroDAbis является прототип сервиса, который, в общем-то, 
является кросс-иден\-ти\-фи\-ка\-ци\-ей нескольких существующих каталогов. Однако он вносит 
новое качество в работу с имеющимися ресурсами. Целевая аудитория этого проекта~--- в 
первую очередь астрономы, которые являются довольно-таки небольшим по количеству 
сообществом, но взаимосвязанную информацию в базах данных можно предоставить и более 
широкой публике посредством API-сер\-ви\-сов, что облегчит будущим разработчикам 
создание удобной системы поиска информации об астрономических объектах для любой 
группы пользователей.
      
      Аналогичные разработки не являются новыми в науке (аннотирование данных в 
генетике~--- {Distributed Annotation System}, {\sf http://www.biodas.org}) или в 
Интернете~--- RDF ({Resource Description Framework})~[41] и LOD ({Linking 
Open Data})~[42]. Сис\-те\-ма AstroDAbis разработана так, чтобы естественным образом 
использовать TAP-factory~[43] на базе OGSA-DAI ({Open Grid Services 
Architecture Data Access Interface})~[44], где TAP ({Table Access Protocol}~[45])~--- 
протокол IVOA для работы с таб\-ли\-ца\-ми. Используя TAP-factory, можно создать 
сервис, который позволит выполнять запросы, об\-ра\-ща\-ющи\-еся к другим сервисам, 
совместимым по протоколу TAP. AstroDAbis также имеет LOD-ин\-тер\-фейс, который 
обеспечивает создание URI для аннотируемых объектов, что подготавливает платформу 
для будущих экспериментов с Семантическим Вебом в астрономии. 
      
      Чтобы найти и получить данные пользователь сам инициирует взаимодействие с 
инфраструктурой виртуальной обсерватории посредством клиентских приложений 
(TOPCAT~[46], ALADIN, DS9\ SAOImage и~др.)\ или веб-ин\-тер\-фей\-сов к базам данных. 
Всякий раз, когда пользователь хочет узнать о возможно уже появившихся обновлениях, ему 
надо повторить первоначальный запрос, сравнить полученный результат с существующим и 
скопировать, если это требуется, данные. Постоянно растущие объемы данных, включающие 
новые релизы существующих обзоров, и публикации новых каталогов требуют другого 
подхода при отслеживании новой информации о небесных объектах, интересных 
пользователю. Особенно это полезно при обновлении и актуализации компилятивных 
каталогов и баз данных. Решение этой задачи предлагается с по\-мощью веб-при\-ло\-же\-ния для 
поддержки данных пользователя \mbox{VOdka} ({VO Data Keeping-up Agent})~[47], который 
ретранслирует запросы пользователей в инфраструктуру виртуальной обсерватории и 
рассылает уведомления об обновлениях. При выбранном пользователем темпе опроса агент 
асинхронно посылает один и тот же запрос, сформулированный пользователем, и фиксирует 
результаты, отражающие временной срез информации, выполняет сравнение этих срезов и 
оповещает пользователя по электронной почте. У~пользователя есть возможность 
просматривать результаты запросов, сохраненные в $snapshot$-фай\-лах, журналы сравнения 
этих файлов, копировать снимки и новые появившиеся данные, а также инкрементальные 
файлы, включающие старые, новые и пропущенные данные.
      
      Во многих областях научных исследований имеется насущная потребность работы с 
большими по объему распределенными массивами данных и выполнения над ними 
разнообразных задач по извлечению знаний. Ита\-ло-аме\-ри\-кан\-ский проект DAME 
({DAta Mining} \& {Exploration})~[48] на\-правлен на создание междисциплинарной 
распределенной среды, специализированной под исследования больших массивов данных 
(MDS, {Massive Data Set}) с\linebreak
 помощью ма\-шин\-но-обуча\-емых алгоритмов и 
методов добычи данных, которая реализована на унифицированной технологической 
платформе. DAME включает несколько проектов по решению разных астрофизических 
задач и может предложить для\linebreak разных e-science сообществ широкий спектр 
вы\-чис\-ли\-тель\-ных мощ\-ностей для применения ма\-шин\-но-обуча\-емых и статистических 
алгоритмов к астрономическим данным. Эти проекты используют\linebreak единую технологическую 
платформу, ба\-зи\-ру\-ющу\-юся на архитектуре сер\-вис\-но-ориен\-ти\-ро\-ван\-ных приложений и 
совместимую со стандартами виртуальной обсерватории. 
  
  Обнаружение знаний в базах данных KDD ({Knowledge Discovery in Data Bases}) 
сейчас связывают с новым семейством научных дисциплин, называемым 
  X-Informatics. Оно считается четвертой парадигмой в науке после теории, 
эксперимента и моделирования. В~таком контексте проект DAME призван: 
  \begin{itemize}
\item обеспечить сообщество расширяемой интегрированной средой для добычи данных 
и исследований на базе технологий Web~2.0;
\item поддерживать стандарты и форматы виртуальной обсерватории для 
интероперабельности приложений;
\item обеспечить виртуальную обсерваторию общей вычислительной платформой, 
использующей современные технологии (грид, облачные вычисления и~т.\,п.). 
\end{itemize}

При происходящем в настоящее время росте сложности данных и необходимости 
проведения исследований с большими массивами данных альянсом IVOA было принято 
решение о создании группы по интересам, связанной с обнаружением знаний в базах данных 
(KDD-IG), которая должна согласовывать стандарты IVOA и потребности научных 
исследований с использованием баз данных.
      
      В этом проекте находятся в стадии разработки несколько научных сценариев, которые 
оформляются в виде веб-при\-ло\-же\-ний, базирующихся на архитектуре системы DAME. Из 
них для исследования интересующей автора выборки радиоисточников наиболее 
привлекательны разработки по оценке фотометрического красного смещения галактик и 
селекции квазаров на основе фотометрических данных, поскольку определение 
спектральных красных смещений требуют больших затрат наблюдательного времени в 
отличие от фотометрических оценок.
      
      Разработки проекта ADSASS ({The ADS All-Sky Survey})~[49] направлены на 
превращение системы NASA ADS ({Astrophysics Data System}), широко 
используемой среди астрономов в качестве полнотекстового библиографического ресурса, в 
карту неба. Система ADS не является источником наблю\-да\-тель\-ных данных, но является 
неявным хранилищем ценной астрономической информацией в форме изображений, таблиц 
и ссылок на небесные объекты, которые являются частью публикации. Необходимо сделать 
эту информацию доступной для запросов и просмотра. Рассматриваются три категории 
данных: 
      \begin{enumerate}[(1)]
\item ссылки на небесные объекты, которые предполагается собрать из внешних баз 
данных и добавить в виде аннотации (astrotag) связь со стать\-ями в ADS. Так же, 
как это сделано в geotags для объектов на земной поверхности, astrotags 
являются пространственными и временными аннотациями для небесных объектов;
\item оптические и изображения в других диапазонах, имеющиеся в стать\-ях, также 
получат связывающие ссылки (astroreference). Так это сделано для геоданных 
(georeferencing), которые ссылаются на карты, имеющие привязку к сис\-те\-ме 
земных координат, ссылки (astroreferencing) свяжут изображения, которые будут 
приведены к одной небесной координатной сис\-те\-ме с учетом ориентации, координатной 
привязки и масштаба пикселов каждого кадра; 
\item другого сорта данные, такие как текст или подписи под рисунками, будут 
привязаны к координатам или имени источника. 
\end{enumerate}

В результате выполнения проекта будет получена карта всего неба, которая будет 
активировать ссылки на статьи, показывая, какая часть неба в них описывается, а также слой 
исторических данных на базе хранилища astroreference-ссы\-лок и изоб\-ра\-же\-ний, 
извлеченных из статей, которые можно использовать для анализа. Для визуализации этой 
информации будут использоваться приложения, в которых можно отображать полностью все 
небо, а именно: WorldWide\ Telescope (Microsoft), ALADIN (CDS), Google\ Sky 
(Google) и~др. Сис\-те\-ма ADSASS будет опираться на постоянно об\-нов\-ля\-емую базу данных 
тегов, которая предназначена как для обнаружения новой информации о небесных объектах 
по любой тематике, так и для поиска событий переменного характера по данным 
исторического слоя. 

\section{Массовое отождествление радиоисточников}

\subsection{Научная мотивация}

  В противоположность начальной стадии своего возникновения Вселенная сегодня богата 
структурами~--- галактиками, скоплениями галактик, сверхскоплениями и пустотами~--- 
войдами. Все эти структуры эволюционируют с гравитационным расширением из небольших 
первоначальных неоднородностей плотности. В~иерархических космогониях первые 
  гра\-ви\-та\-ци\-он\-но-свя\-зан\-ные сис\-те\-мы могли быть звездами и/или небольшими 
звездоформирующими системами, при слиянии которых формируются галактики. 
Возникающие из конечных продуктов звездной эволюции и мерджинга (слияния) 
центральные черные дыры продолжают расти. В~любом случае аккреция, питающая 
массивные черные дыры, проявляет себя как феномен активного галактического ядра 
({Active Galaxy Nuclear}, AGN). Из-за своей экстремальной светимости AGN 
являются подходящими реперами для исследования Вселенной. Хотя почти все AGN 
име\-ют схожие источники энергии, их наблю\-да\-емые свойства сильно различаются. 
К~примеру, одни AGN имеют мощное радиоизлучение, а другие нет. Могут наблюдаться 
еще разные проявления активности ядра~--- широкие эмиссионные линии в оптике, высокая 
степень поляризации оптического излучения, переменность, рентгеновское или 
  гам\-ма-из\-лу\-че\-ние. В~моделях, объясняющих наблюдаемые свойства активных 
галактик, предполагается, что только несколько физических процессов обеспечивают 
наблюдаемый диапазон AGN-ха\-рак\-те\-ри\-стик. Полагают, что разнообразие типов 
AGN возникает из: (1)~отсутствия или наличия пыли вокруг ядра; (2)~направления 
релятивистского джета относительно наблюдателя; (3)~цикла активности; (4)~полной 
светимости галактики. Еще разделение по мощ\-ности радиоизлучения зависит от углового 
момента и массы центральной черной дыры. 
  
  Популяцию мощных радиогалактик с по\-мощью имеющихся радиотелескопов можно 
наблюдать практически на любом расстоянии. Это позволяет изучать их эволюцию в 
радиодиапазоне от момента образования до наших дней. От низких до умеренных красных 
смещений ($Z\sim1$) мощные радиоисточники связывают с гигантскими эл\-лип\-ти\-че\-ски\-ми 
галактиками, поэтому радиогалактики можно использовать для изучения формирования и 
эволюции самых массивных звездных систем, из истории звездообразования которых можно 
получить важные ограничения на модели формирования галактик и космологические 
параметры. Радиоисточники часто ассоциируются с центральными галактиками скоплений, 
поэтому далекие радиогалактики могут быть индикаторами первых протоскоплений. Так 
выглядят в общих чертах те задачи, которые решаются при исследовании радиогалактик.

\subsection{Предметный посредник для~поиска далеких радиогалактик}
 
Известны разные техники селекции объектов для поиска далеких галактик. К~ним относятся: 
глубокая спектроскопия пустых полей, узкополосные снимки, спектроскопия объектов 
вокруг радиогалактик с большим~$Z$, использование показателей цвета (разность звездных 
величин объекта в различных фотометрических фильтрах/полосах) для выбора кандидатов. 
Известно, что спектральное распределение энергии небесных объектов в зависимости от 
красного смещения сдвигается в красную область спектра. Из-за этого галактика может быть 
ярче или существенно слабее в каком-либо фильтре, быть видимой только в одном фильтре 
из-за того, что в эту полосу попадают особенности спектрального распределения объекта~--- 
излучение в водородной линии Лай\-ман-аль\-фа $\lambda=1216$~{\ptb\AA} 
или завал спектра 
на $\lambda=912$~{\ptb\AA}. Чтобы поймать это усиление или, наоборот, ослабление яркости, 
используют ограничения на цветовые индексы, например, следующие: $(u\hm+r)/2-g\hm>1$; 
$(g\hm+i)/2\hm-r \hm>1$ и~т.\,п., где $u$, $g$, $r$, $i$, $z$~--- звездные величины 
оптического объекта в полосах обзора SDSS. Это и есть так называемые 
dropout-тех\-ни\-ки, используемые при отборе кандидатов в далекие галактики. Затем 
для кандидатов проводят спектральные исследования, чтобы определить по смещению 
спектральных линий, действительно ли объект является далеким. Применение этих техник 
привело к обнаружению далеких объектов с $Z\sim6$--7. Однако радиогалактики 
продолжают оставаться интересными для изучения объектами, так как они являются самыми 
массивными звездными системами, во многих случаях указывают на скопления галактик, а 
на космологических расстояниях~--- на протоскопления. 
      
      С появлением оптического обзора неба SDSS и радиообзора FIRST, которые 
обладают надежной координатной привязкой, достаточной глубиной и угловым 
разрешением, а также программных средств виртуальной обсерватории исследования 
природы радиоисточников стало возможным проводить не только по небольшим по числу 
объектов выборкам, но и по любым каталогам/спискам. 
      
      Был предложен научный сценарий поиска далеких галактик по радиоисточникам 
каталога~RC, который использует обзоры FIRST, NVSS и SDSS~[50]. Для каждого 
RC-ис\-точ\-ни\-ка из обзора SDSS выбираются объекты, которые попадают в область, 
размерами равную боксу ошибок определения координат ($\pm 3\sigma$). При средней 
плотности объектов обзора SDSS ($\sim7$--8 объектов на кв.\ угл.\ минуту) в область 
поиска, размеры которой варьируются от 45$^{\prime\prime}$ до 2$^\prime$ в зависимости 
от положения RC-ис\-точ\-ни\-ка относительно центральной части диаграммы 
направленности телескопа, попадают сотни объектов. Поскольку низкая точность координат 
каталога~RC не позволяет выполнить отождествление по позиционному совпадению, 
можно провести дополнительную селекцию в оптике, а именно использовать ограничения 
для разности показателя цвета. И~если оптический объект, попадающий по координатам в 
область поиска, еще и удовлетворяет ограничениям по показателям цвета, то он является 
наиболее вероятным кандидатом для отождествления. 

Для сценария отождествления списка радиоисточников был создан предметный 
посредник~[51], архитектура которого была реализована как объединение системы 
AstroGrid~[52], которая разработана в Великобритании и совместима с протоколами 
IVOA, и средств поддержки предметных посредников, созданных в ИПИ РАН. Для 
прототипа гиб\-рид\-ной архитектуры было выполнено сопряжение исполнительных 
механизмов двух инфраструктур (AstroGrid и предметных посредников). Эта разработка 
выполнялась для решения задач, связанных с разработкой прототипа Российской 
виртуальной обсерватории (РВО)~[28].
      
      Сценарий автоматического отождествления выборки каталога~RC в области, 
пересекающейся с обзорами SDSS и FIRST, разделен на два этапа: подготовка данных и 
визуальная инспекция результатов для принятия решения об отождествлении 
радиоисточника с оптическим кандидатом. 
      
      Поток задач для подготовки данных состоит из следующей последовательности: 
      \begin{enumerate}[(1)]
\item выбор координат радиоисточника из каталога~RC;
\item извлечение списка объектов из области заданного размера из базы данных обзора 
SDSS; 
\item кросс-идентификация результатов запросов с использованием селекционного 
ограничения по цветовым индексам; 
\item извлечение радиоизображений из обзора FIRST;
\item извлечение оптических изображений SDSS;
\item суперпозиция изображений;
\item сохранение результатов запросов. 
\end{enumerate}

Чтобы можно было работать с каталогом~RC из потока задач, он оформлен как компонент 
AstroGrid для доступа к данным~--- DSA ({Data Set Access}). Запрос к базе данных, 
содержащей каталог~RC (шаг~1), выполняется CEA ({Common Execution 
Architecture}) приложением на узле AstroGrid, развернутом в ИПИ РАН. Результат запроса 
в формате VOTable автоматически запоминается в MySpace (виртуальная область памяти 
сис\-те\-мы AstroGrid). Запрос на языке ADQL выглядит следующим образом: 

%\end{multicols}

%\hrule

\noindent
{\small
\begin{verbatim}
SELECT crd.ra, crd.de, cat.name 
FROM RCCatalog as cat, CoordEQJ as crd
WHERE cat.coord_id = crd.coord_id
\end{verbatim}
}

%\hrule

%\begin{multicols}{2}
      
Запрос к каталогу SDSS (шаг~2) выполняется CEA-приложением с помощью веб-сер\-ви\-са, 
работающего на сервере SDSS 
({\sf http://voservices.net/ CasService/ws\_v1\_0/Cas} Service.asmx). Оно запрашивает данные и 
возвращает результат в файл в формате VOTable, сохраняемый в MySpace. Запрос к 
SDSS записывается следующим образом: 

%\end{multicols}

%\hrule

\noindent
{\small
\begin{verbatim}
SELECT ra=cast(ra as real),
dec=cast(dec as real), 
objid, u, g, i, r, z, 
colorIndexURG = (u+r)/2.0-g .GT. 1. 
FROM PhotoPrimary 
WHERE ra BETWEEN 225.0 AND 225.5 AND dec 
BETWEEN 4.0 AND 5.61 AND r BETWEEN 15.0 AND 23.0 
\end{verbatim}
}

%\hrule

%\begin{multicols}{2}

Кросс-идентификация (шаг~3) результатов двух предыдущих шагов выполняется 
веб-сер\-ви\-сом\linebreak
AstroGrid CrossMatchFull ({\sf ivo://org.astrogrid/\linebreak CrossMatcher}) в UK. 
Результатом служит таблица в VOTable формате, которая также помещается в MySpace. 

Извлечение и суперпозиция изображений (шаги~4--6) производится в ИПИ РАН. 
При\-ло\-же\-ние CEA вызывает ALADIN для каждого объекта из каталога RC, 
используя его координаты в качестве центра области. ALADIN извлекает изображения из 
DSS и FIRST, после чего контуры радиоизображения совмещаются с оптическим. 
Дополнительно извлекаются объекты из каталогов SDSS, 2MASS, FIRST, NVSS, 
попавшие в область поиска. Выполняющая эти запросы программа, написанная на языке 
команд ALADIN, показана ниже: 

%\end{multicols}

%\hrule

\noindent
{\small
\begin{verbatim}
get DSS.ESO(DSS1,14.1,14.1), address, 5'; sync; 
/* извлечь изображение из обзора DSS. 
/* Переменная address содержит координаты 
/* в формате `hh:mm:ss sdd:mm:ss'
get NVSS(0.2,15.0,Stokes I, Sine), address, 5'; 
/* извлечь изображение из радиообзора NVSS
sync; contour 4;                                
/* синхронизировать и построить контуры
get FIRST(10), address, 5';                     
/* извлечь изображение из радиообзора FIRST
sync; contour 4;                               
get SDSSDR3cat, address, 1';                    
/* извлечь данные из каталога SDSS
sync; 
get VizieR(2mass), address, 1';                 
/* извлечь данные из каталога 2MASS
sync;
backup st.aj                                    
/* сохранить данные
\end{verbatim}

}

%\hrule

%\begin{multicols}{2}
      
После выполнения программы данные запоминаются (шаг~7) в стеке ALADIN и 
сохраняются в MySpace. Подготовка данных на этом закончена. 

      На втором этапе для просмотра подготовленных данных запускается ALADIN и 
Workbench (клиентское приложение для работы с AstroGrid). Данные, сохраненные в 
MySpace AstroGrid, открываются в ALADIN, проводится визуальная инспекция 
подготовленных данных и принимается решение об отождествлении объекта. 
      
      Подробное описание реализации этого сценария на основе AstroGrid и средств 
поддержки предметных посредников приведено в~[51]. 

\subsection{Оптическое отождествление каталога RC} 

      Подходящих кандидатов в далекие галактики среди источников каталога RC не 
было обнаружено, но была разработана и опробована методика массового отождествления 
радиоисточников. Исследования были продолжены дальше, но уже с другой целью~--- 
отождествление всех источников каталога RC, попадающих в область обзоров SDSS и 
FIRST, и определение типов родительских галактик. 
      
      Сценарий массового отождествления радиоисточников включает следующие этапы: 
(1)~подготовка данных; (2)~предварительная обработка и визуальная инспекция; 
(3)~анализ~--- уточнение координат RC-ра\-дио\-ис\-точ\-ни\-ков; (4)~анализ~--- 
определение морфологических типов радиоисточников; (5)~анализ~--- оптическое 
отождествление. 

\begin{figure*}[b] %fig1
\vspace*{1pt}
 \begin{center}
 \mbox{%
 \epsfxsize=162.053mm
 \epsfbox{zhe-1.eps}
 }
 \end{center}
 \vspace*{-9pt}
\Caption{(\textit{a})~Рисунок, полученный средствами ALADIN. Полутоновое изображение~--- 
данные в полосе $r$ из оптического цифрового обзора неба SDSS; штриховые контуры~--- контурная 
радиокарта из обзора NVSS (угловое разрешение~--- 45$^{\prime\prime}$); сплошные контуры~--- 
радиокарта обзора FIRST с более высоким угловым разрешением (5$^{\prime\prime}$), которая 
позволяет определить детальную структуру радиоисточника; крестиками отмечены данные из 
каталога SDSS, ромбами~--- радиокаталоги. (\textit{б})~Морфологические типы 
радиоисточников: C ({core})~--- точечный; D и DC ({double})~--- двойной; CJ 
({core-jet})~--- ядро с выбросом; CL ({core-lobe})~--- ядро с компонентами; T 
(triple)~--- тройной}
\end{figure*}
      
\textbf{1. Подготовка данных.} Этот этап выполняется так же, как в сценарии поиска 
далеких радиоисточников, но список используемых каталогов существенно расширен. 
Подготовка данных производится автоматически с помощью {perl}-программы, 
которая использует средства программного интерфейса ALADIN. Для каждого источника 
каталога RC по координатам извлекаются изображения из оптических [DSS-II (сервер 
{Space Telescope Science Institute}) и SDSS (сервер {SkyView~--- the Internet's 
Virtual Telescope})] и радиообзоров неба [NVSS, FIRST (сервер {National Radio 
Astronomy Observatory})], данные из оптических (USNO-B1, SDSS) и 
инфракрасных (2MASS) каталогов, а также из радиокаталогов [VLSS, TXS, NVSS, 
FIRST, GB6 и RC (сервер \textit{Vizier})], выполняется суперпозиция изображений 
посредством наложения контуров радиоизображения на оптическое, извлеченные данные 
сохраняются для последующих фаз сценария. Пример программы приведен ниже:

%\end{multicols}

%\hrule

\noindent
{\small
\begin{verbatim}
#!/usr/bin/perl
die "Usage: $myname Catalog\n" unless (@ARGV);
$CATALOG = $ARGV[0];
open (READ_CAT, "<$CATALOG") 
or die ("Cannot open file"); 
\* открытие файла с координатами
open(ALADIN,"| java 
-Dhttp.proxyHost=192.168.2.33 -       
\* запуск ALADIN
Dhttp.proxyPort=8080 
-jar /Data/users/zhe/Aladin/Aladin.jar");
$rcN = 0;
while ($str = readline (*READ_CAT))
{
($pref,$name,$last) = split (/ /, $str, 3);               
\* преобразование координат
$rcname = join('',$pref,$name);
$raJ = substr($last,0,11);
$deJ = substr($last,12,11);
$obj = join (' ',$raJ,$deJ);
$stack = join('','s',$rcname,'.aj');
print ALADIN "reset; \n";                                 
\* передача команд ALADIN
print ALADIN "get Vizier(VIII/42/txs) 
$obj 5\';\n";       
\* извлечь данные из каталога TXS
print ALADIN "get aladin(DSS2,F) $obj 5\';\n";            
\* -"- изображение из обзора DSS
print ALADIN "get NVSS(0.2,15.0,Stokes I,Sine) 
$obj 5\';\n"; 
\* -"- -"-    из радиообзора  NVSS
print ALADIN "sync; contour 4; \n";                       
\* построить контуры радиоизображения
print ALADIN "get Vizier(VIII/65/nvss) $obj 5\'; 
sync;\n";
\* извлечь данные  из каталога  NVSS
print ALADIN "get Vizier(sdss) $obj 1\';\n";              
\* -"- -"-  из каталога  SDSS
print ALADIN "sync \n";
print ALADIN "get VizieR(2mass) $obj 1\';\n";             
\* -"- -"-  из каталога  2MASS
print ALADIN "sync \n";
print ALADIN "get VizieR(USNOB) $obj 1\';\n";             
\* -"- -"-  из каталога  USNO-B1
print ALADIN "sync \n";
print ALADIN "get FIRST(100) $obj 5\';\n";                
\* извлечь изображение из обзора FIRST
print ALADIN "sync; contour 4; \n";                       
\* построить контуры радиоизображения
print ALADIN "get Vizier(VIII/71/first) 
$obj 5\'; sync;\n"; 
\* извлечь данные  из каталога  FIRST
print ALADIN "get Vizier(J/A+AS/87/1/table1) 
$obj 5\'; sync;\n"; 
\*    -"- -"- из каталога  RC
print ALADIN "backup $stack\n";                             
\* сохранение подготовленных данных
};
\end{verbatim}

}

%\hrule

%\begin{multicols}{2}

\begin{figure*}[b] %fig2
\vspace*{6pt}
 \begin{center}
 \mbox{%
 \epsfxsize=165.339mm
 \epsfbox{zhe-2.eps}
 }
 \end{center}
 \vspace*{-9pt}
\Caption{(\textit{а})~Изображение радиоисточника каталога RCR из обзора FIRST. 
(\textit{б})~Составное изображение оп\-ти\-ка--ра\-дио. Контуры, построенные по изображению радиообзора 
FIRST, наложены на цветное RGB-изоб\-ра\-же\-ние, полученное из трех кадров обзора SDSS в 
фотометрических полосах $g$, $r$, $i$, где фильтр~$g$ соответствует~$B$, $r$~--- $G$ и $i$~--- $R$. 
Рисунок получен с помощью python-скрип\-та, использующего программный интерфейс к 
приложению для визуализации и доступа к данным виртуальной обсерватории DS9 SAOImage}
\end{figure*}




\textbf{2. Предварительная обработка и визуальная инспекция.} На этом этапе используется 
макроконтроллер ALADIN. Этот сервис производит интерпретацию двух файлов. Один 
файл содержит скрипт с командами ALADIN, второй~--- данные, которые являются 
параметрами команд. Предварительная обработка производится средствами графического 
интерфейса ALADIN (рис.~1,\,\textit{а}). Она состоит в подборе уровней контуров и диапазона 
отображаемых данных для изображений.

\textbf{3. Уточнение координат.} Поскольку координаты каталога RC грубы для 
оптического отождествления, то требуется уточнение координат по радиообзорам с более 
высокой координатной точностью. Отметим, эта задача легче решается, если у сравниваемых 
каталогов близкое угловое разрешение, при этом необходимо учитывать изменение 
плотности потока источника на разных частотах, а также предельную чувствительность 
каждого каталога.
      
      Идентификация RC-источников проводилась сначала с источниками обзора 
NVSS, поскольку каталог RC имеет близкое к NVSS угловое разрешение по прямому 
восхождению (45$^{\prime\prime}$), а координатная точность NVSS (1$^{\prime\prime}$) 
существенно лучше, чем у RC (15$^{\prime\prime}\times40^{\prime\prime}$).
      
      Перечислим условия в порядке убывания значимости, выполнение которых 
принималось во внимание при отождествлении источника каталога RC с источником 
обзора NVSS:
\begin{itemize}
\item координатное совпадение по прямому восхождению ($r < 3\sigma$, где $\sigma$~--- 
приведенная в каталоге RC ошибка координат по прямому восхождению); 
\item координатное совпадение по склонению; 
\item совпадение плотностей потока для RC-ис\-точ\-ни\-ка и NVSS-ис\-точ\-ни\-ка. Вызывают 
сомнения случаи, когда при координатном совпадении RC-ис\-точ\-ник не согласуется по 
плотности потока с NVSS (при пересчете плотностей потоков полагаем, что спектральный 
индекс источника $\alpha\sim 0.7$, $S(\nu) \sim \nu^{-\alpha}$); 
\item присутствие соседних источников. Если рядом с RC-ис\-точ\-ни\-ком есть не один, а два 
или несколько источников NVSS, которые попадают в диаграмму направленности 
РАТАН-600, то возникает неоднозначная ситуация при идентификации. В~этом случае 
принималось, что наибольший вклад дает самый яркий NVSS-ис\-точ\-ник, с которым и 
отождествлялся RC-источник. 
\item когда плотность потока источника на 3,9~ГГц оказывается больше, чем плотность 
потока на 1,4~ГГц, требуется дополнительная информация, подтверждающая рост 
плотности потока к более высоким частотам. В~этих случаях использовался как каталог, 
так и радиообзор GB6 на 4,85~ГГц. В каталог обычно включаются объекты с плотностью 
потока выше $5\sigma$ уровня отношения сигнал/шум. Источники с плот\-ностью потока на 
уровне $3\sigma\mbox{--}4\sigma$, отсутствующие в каталоге GB6, обнаруживаются при 
визуальной инспекции изображений обзора GB6. Эта дополнительная информация 
помогала при неоднозначных случаях отождествления.
  \end{itemize}
  

  
По такому алгоритму~[17] были отождествлены и уточнены координаты у 75\% источников 
каталога RC, для которых на следующем шаге проводилось оптическое отождествление.

%\vspace*{6pt} 

\textbf{4. Определение морфологического типа ра\-дио\-ис\-точ\-ни\-ка.} Корректность оптического 
отождествления радиоисточника зависит от правильного определения его морфологического 
типа, поскольку есть связь между типом и предполагаемым положением родительской 
галактики. Для этого использовался обзор FIRST, где из-за более высокого углового 
разрешения имеется более подробная информация о структуре источника. 
      
      Было использовано 5 морфологических типов радиоисточников (рис.~1,\,\textit{б}): 
точечные ({core}), двойные ({double}, {double-core}, {double-double}), 
тройные ({triple}), ядро с джетом ({core-jet}), ядро с компонентами 
      ({core-lobe}). Тип радиоисточника определялся по радиоизображениям (рис.~2,\,\textit{а}) 
      и дополнительно по картам с контурами интенсивности (рис.~2,\,\textit{б}), если 
структуру сложно классифицировать.


%\vspace*{6pt}

\textbf{5. Определение оптического кандидата.} После определения морфологического типа 
радиоисточника определялось предполагаемое положение родительской галактики. Именно 
эти координаты затем использовались при выборе оптического кандидата. Оптический 
объект считался надежным кандидатом на отождествление, если его положение (по каталогу 
SDSS) было не дальше $3\sigma$ от предполагаемого положения родительской галактики, 
где $\sigma$~--- ошибка координат. К~возможным отождествлениям отнесены следующие 
случаи:
\begin{itemize}
\item источник точечный или двойной с ядром, а оптический объект расположен дальше 
чем $3\sigma$ от центра радиоисточника; 
\item два оптических объекта рядом с предполагаемым положением оптического кандидата, 
и по имеющейся фотометрической и спектральной информации нельзя сделать уверенного 
выбора между объектами; 
\item источник двойной, положение ядра определяется неуверенно, оптический объект 
сдвинут в сторону от линии, соединяющей максимумы плотностей потока компонентов;
\item сложно сделать выводы о структуре радиоисточника по радиокарте FIRST.
\end{itemize}

Надежные оптические кандидаты были найде\-ны для 70\% радиоисточников, для 10\% 
радиоисточников имеются возможные кандидаты, а для 20\% не обнаружены оптические 
объекты, так как родительские галактики радиоисточников слабее предельной глубины 
обзора SDSS. Для 75\% оптических кандидатов было проведено разделение на галактики и 
звездные объекты~[18]. Результаты отождествления представлены на {\sf 
http://www.sao.ru/fetch/cgi-bin/SkyObj/rc.cgi}.

\subsection{Оптическое отождествление каталога RCR}
      
      Каталог RCR (RATAN Cold Refined)~[5] получен в результате обработки 
7~циклов наблюдений, проведенных на радиотелескопе РАТАН-600 с 1987 по 1999~гг., и 
повторной обработки данных 1980--1981~гг.\ для интервала прямых восхождений $7^h \leq 
\alpha_{2000}\leq 17^h$ с целью улучшения координат и уточнения плотностей потоков 
источников каталога RC. Отметим, что в результате отождествления каталога RC 
$\sim25\%$ из 432~объектов, попавших в область обзоров FIRST и SDSS, не 
отождествились с источниками NVSS. Собственно говоря, это и послужило толчком для 
подготовки следующего релиза каталога RC~--- каталога RCR, по которому работа по 
отождествлению радиоисточников обзоров <<Холод>> 1980--1999~гг.\ была продолжена. 
В~каталоге RCR 550 источников, что в 1,7~раза больше, чем количество RC-ис\-точ\-ни\-ков, 
для которых уже проводилось оптическое отождествление.
      
      Обычно радиоисточники отождествляются с доста\-точ\-но слабыми объектами в оптике. 
Чем слабее по потоку радиоисточники, тем более глубокие снимки в оптике необходимы для 
обнаружения объекта, ответственного за радиоизлучение. При предельной глубине обзора 
SDSS ~22.6 звездной величины по оценкам можно отождествить $\sim30\%$--50\% 
радиоисточников с плотностью потока ярче 1~мЯн на 1,4~ГГц. В~каталоге RCR источники 
ярче~--- слабые объекты имеют плотность потока на частоте 3,9~ГГц 10--15~мЯн, что при 
пересчете на 1,4~ГГц будет соответствовать 17--25~мЯн. Так по каталогу SDSS удалось 
отождествить 70\% RC-ис\-точ\-ни\-ков. Чтобы отождествить оставшиеся объекты, нужны 
были более глубокие снимки, которые можно получить, сложив изображения в трех 
фильтрах обзора SDSS~--- $g$, $r$ и $i$, а также используя обзоры инфракрасного 
диапазона. Чем больше диапазонов электромагнитного спектра привлекается для 
исследования радиоисточников, тем больше информации для определения типа 
родительской галактики, а также ее физических характеристик. Поэтому для исследования 
радиоисточников каталога RCR был добавлен глубокий обзор неба в ближнем 
инфракрасном диапазоне UKIDSS и в среднем инфракрасном диапазоне~--- WISE. 
      
      Сценарий для отождествления источников каталога RCR включает: (1)~подготовку 
данных; (2)~визуализацию и предварительную обработку; (3)~определение морфологического 
типа радиоисточника; (4)~оптическое отождествление. Эти этапы мало отличаются по 
методике от описанных в предыдущем подразделе шагов. 

      \begin{table*}[b]\small
      \begin{center}
      \Caption{Каталоги и обзоры, использовавшиеся для отождествления радиоисточников 
каталога RCR}
      \vspace*{2ex}
      
\tabcolsep=7pt
      \begin{tabular}{|l|c|c|c|c|}
      \hline
\multicolumn{1}{|c|}{Диапазон} &\tabcolsep=0pt\begin{tabular}{c}Каталоги,\\ обзоры\end{tabular}&
\tabcolsep=0pt\begin{tabular}{c}  Спектральный\\ диапазон\end{tabular}&
\tabcolsep=0pt\begin{tabular}{c}Разрешение/\\ ошибки\end{tabular}&
\tabcolsep=0pt\begin{tabular}{c}Предел \\ чувствительности\end{tabular}\\
\hline
Радио&VLSS&\hphantom{99}74 МГц&80$^{\prime\prime}$&500 мДжанки\\
&TXS&\hphantom{9}365 MГц&$\sim10^{\prime\prime}$\hphantom{99}&150 мДжанки\\
&NVSS&1400 MГц&45$^{\prime\prime}$&2.5 мДжанки\\
&FIRST&1400 MГц&\hphantom{999}5.4$^{\prime\prime}$&1 мДжанки\\
&GB6&4850 МГц&\hphantom{999}3.5$^{\prime\prime}$&28--37 мДжанки\\
\hline
Оптика&DSS-II&
\tabcolsep=0pt\begin{tabular}{c}blue, red, IR\\ ($J$, $F$, $N$)\end{tabular}&&
\tabcolsep=0pt\begin{tabular}{c}$\sim21^m$\\ ($\sum \mathrm{BRI} \sim 21.2^m {R}$)\end{tabular}\\
\cline{2-5}
&SDSS&
\tabcolsep=0pt\begin{tabular}{c}$u$, $g$, $r$, $i$, $z$\\ ($g + r + i$)\end{tabular}&
$\pm0.1^{\prime\prime}$&
\tabcolsep=0pt\begin{tabular}{c}22.0$^m$, 22.2$^m$, 22.2$^m$, 21.3$^m$,\\ 
20.5$^m$\\
($\sum \mathrm{gri} \sim 22.6^m$)\end{tabular}\\
\cline{2-5}
&&&&\\[-9pt]
&USNO-B1&
\tabcolsep=0pt\begin{tabular}{c}$B_1$, $R_1$, $B_2$, $R_2$, $I$\\ ($O$, $E$, $J$, $F$, $N$)\end{tabular}&
\tabcolsep=0pt\begin{tabular}{c}0.2$^{\prime\prime}$\\ 0.3$^m$\end{tabular}&$V =21^m$\\
\cline{2-5}
&&&&\\[-9pt]
&GSC 2.3.2&
\tabcolsep=0pt\begin{tabular}{c}$J$, $F$, $N$\\ ($B_J$, $R_F$, $I_N$)\end{tabular}&
\tabcolsep=0pt\begin{tabular}{c}0.2$^{\prime\prime}$--0.28$^{\prime\prime}$\\0.13$^m$--0.22$^m$\end{tabular}&
$R_F=20^m$\\
\hline
ИК&2MASS&$J$, $H$, $K$&
\tabcolsep=0pt\begin{tabular}{c}0.2$^{\prime\prime}$\\ 10\%\end{tabular}&
15.8$^m$, 15.1$^m$, 14.3$^m$\\
\cline{2-5}
&UKIDSS&
\tabcolsep=0pt\begin{tabular}{c}$Y$ (1.02~мкм), J\\ 
(1.25~мкм), $H$ (1.63~мкм),\\ $K$ (2.2~мкм)\\ ($H + K$)\end{tabular}&
$<0.1^{\prime\prime}$&
\tabcolsep=0pt\begin{tabular}{c}20.5$^m$, 20.0$^m$,\\ 18.8$^m$, 18.4$^m$ \\
($\sum HK \sim 22.8^m$)\end{tabular}\\
\cline{2-5}
&&&&\\[-9pt]
&WISE&
\tabcolsep=0pt\begin{tabular}{c}3.4~мкм, 4.6~мкм,\\12~мкм, 22~мкм\end{tabular}&
\tabcolsep=0pt\begin{tabular}{c}6.1$^{\prime\prime}$, 6.4$^{\prime\prime}$,\\
6.5$^{\prime\prime}$, 12$^{\prime\prime}$\end{tabular}&
\tabcolsep=0pt\begin{tabular}{c}16.5$^m$, 15.5$^m$,\\ 11.2$^m$, 7.9$^m$\end{tabular}\\
\hline
 \end{tabular}
\end{center}
\end{table*}
      
      Был расширен список каталогов и обзоров, а также использованы появившиеся в 6-й 
версии \mbox{ALADIN} возможности макроконтроллера по выполнению арифметических 
операций с изображениями. Ниже приведен пример скрипта для макроконтроллера с 
суммированием изображений:

%\end{multicols}

%\hrule

{\noindent
\small
      \begin{verbatim}
G = get Skyview(300,Default,"SDSS G",Tan,J2000)
$1 $2 
R = get Skyview(300,Default,"SDSS R",Tan,J2000) 
$1 $2 
\* извлечение изображений SDSS в фильтрах g, r, i 
I = get Skyview(300,Default,"SDSS I",Tan,J2000) 
$1 $2 
#
R_n = norm -cut R 
I_n = norm -cut I                                     
\* нормализация изображений
G_n = norm -cut G
sync 
RI = R_n + I_n 
GRI = RI + G_n                                        
\* суммирование изображений 
sync
\end{verbatim}

}

%\hrule

%\begin{multicols}{2}

С помощью программного интерфейса к приложению DS9\ SAOImage была реализована 
программа на языке Python для построения рисунков в формате postscript по списку 
радиоисточников (рис.~2,\,\textit{б}). Рисунки использовались для морфологической 
классификации радиоисточников.



В результате были получены надежные отождествления для 82\% радиоисточников, для 10\% 
обнаружены возможные кандидаты, а для 8\% так и не были найдены оптические и/или 
инфракрасные объекты.

\section{Информационно-поисковая система по~результатам отождествления 
радиоисточников каталога RCR}

      По результатам отождествления радиоисточников каталога RCR автором накоплен 
материал,\linebreak который предполагается использовать для даль\-нейших исследований, а именно: 
определения физических характеристик родительских галактик радиоисточников, их 
классификации, подготовки\linebreak
 выборок источников со схожими свойствами, поиска далеких 
объектов, поиска переменности в оптическом и радиодиапазоне. В~табл.~1 приведены 
используемые при отождествлении информационные ресурсы, указаны их основные 
характеристики и спектральные полосы. 


      
      Для работы с компилятивным каталогом разработана информационная система, 
включающая базу данных по радиоисточникам и их родительским галактикам и 
      веб-ин\-тер\-фейс для отображения разнообразной информации об объектах. Схема 
таб\-лиц ин\-фор\-ма\-ци\-он\-но-по\-иско\-вой сис\-те\-мы включает таблицы оригинальных 
каталогов (в опи\-сы\-ва\-емом случае это 12~каталогов), компилятивные таб\-ли\-цы, вклю\-ча\-ющие 
материал по результатам отож\-де\-ст\-вле\-ния радиоисточников. В~схему включены 
представления: 
      \begin{itemize}
\item v\_rcflux~--- блеск объекта в радио-, инфракрасном и оптическом диапазоне; 
\item v\_rcrparamr~--- параметры радиоисточника; 
\item v\_rcrparamo~--- оптические параметры родительской галактики.
\end{itemize}



Веб-ин\-тер\-фейс (рис.~3) позволяет отображать по имени радиоисточника весь материал, 
относящийся к объекту~--- изображения, данные каталогов и вычисленные па\-ра\-мет\-ры. 
Адрес ре-\linebreak\vspace*{-12pt}

\pagebreak

\end{multicols}

\begin{figure*} %fig3
\vspace*{1pt}
 \begin{center}
 \mbox{%
 \epsfxsize=161.754mm
 \epsfbox{zhe-3.eps}
 }
 \end{center}
 \vspace*{-9pt}
\Caption{Веб-интерфейс информационно-поисковой системы с результатами отождествления 
радиоисточников каталога~RCR}
\vspace*{6pt}
\end{figure*}

\begin{multicols}{2}

\noindent
сурса~--- {\sf http://www.sao.ru/fetch/cgi-bin/SkyObj/\linebreak rcrn.cgi}. Интерфейс реализован 
в архитектуре\linebreak <<клиент\,--\,сервер приложений\,--\,сервер СУБД>>. В~качестве клиента 
используется браузер, сервер приложений~--- \mbox{Apache}, сервер базы данных~--- 
\mbox{PostgreSQL}. При написании скриптов использовался Python со стандартными пакетами 
для поддерж\-ки CGI-интер\-фей\-са, графическая биб\-лио\-те\-ка PIL и модуль \mbox{PyGreSQL}
для связи с СУБД по DBD/DBI-интер\-фейсам.

%\pagebreak




%\begin{multicols}{2}



      Информация, которая представлена в веб-ин\-тер\-фей\-се, разделяется на следующие 
части: графическая статическая (подготовленные предварительно рисунки), графическая 
динамическая\linebreak
 (веб-сер\-ви\-сы извлекают на лету изображение из обзора и помещают рисунок 
на страничку) и параметры радиоисточника, хранящиеся в 
      ин\-фор\-ма\-ци\-он\-но-по\-иско\-вой сис\-теме. 
      
      Динамически выполняется построение спектрального распределения энергии 
радиоисточника (колонка <<Radio-IR-optics spectrum>>). Если источник 
отождествлен или есть возможный кандидат, то спектр строится по данным радио-, 
оптического и инфракрасного (ИК) диапазонов. Поскольку в радиодиапазоне плотность потока от 
объекта измеряется в янских на заданной частоте, а блеск в оптическом и ИК диа\-па\-зо\-нах 
измеряется в звездных величинах в полосе длин волн (ангстремы или нанометры), то все 
величины пересчитываются в звездные величины в фотометрической системе AB~[53]. На 
графике со спектральным распределением энергии объекта по оси абсцисс отложена 
величина десятичного логарифма частоты в герцах, а по оси ординат~--- звездная величина. 
Точки на графике обозначены разными цветами. Каждому цвету соответствуют данные 
определенного каталога. Поскольку ширина полос в оптическом и инфракрасном диапазоне 
обычно несколько сотен ангстрем, то звездная величина приписывается эффективной длине 
волны фильтра. 
      
      В радиопараметрах приводятся координаты центра источника, морфологический тип, 
угловые размеры источника в угловых секундах, число компонент в обзоре FIRST и 
спектральные индексы для радиодиапазона. В~оптических параметрах приведены 
координаты оптического кандидата, разница оптических и радиокоординат, тип оптического 
объекта и~др. В~третьей колонке приведены звездные величины в сис\-те\-ме AB и величины 
из каталогов. 
      
      Разработанная ин\-фор\-ма\-ци\-он\-но-по\-иско\-вая сис\-те\-ма упростила просмотр 
разнородных данных по радиоисточникам и использовалась при принятии решения об 
отождествлении.
  
\section{Заключение}

      Виртуальная обсерватория, объединяющая астрономические данные в 
распределенную инфраструктуру, за несколько лет своего существования обеспечила новый 
качественный уровень работы с цифровыми коллекциями. В~большинстве публикуемых 
статей по разным областям исследований в астрофизике присутствуют данные, которые 
получены с применением средств и ресурсов виртуальной обсерватории. 
      
      В течение ряда лет автором проводилось массовое изучение радиоисточников с 
применением разных приложений виртуальной обсерватории и накоплен как 
положительный, так и отрицательный опыт работы с многочастотными данными. 
Эффективность оптического отождествления радиоисточников несравненно выросла, как 
только появились глубокие цифровые обзоры неба и средства для работы с ними. То, на что 
раньше требовались годы, сейчас выполняется за обозримое время, однако идентификация 
списка объектов с каталогами остается трудоемкой. Когда появляются новые обзоры и 
релизы существующих обзоров, выборку информации из каталогов и обзоров приходится 
повторять, накапливая тем самым варианты компилятивных таблиц. Эффективность 
автоматической кросс-иден\-ти\-фи\-ка\-ции по координатам радиокаталогов с оптическими 
в худших случаях составляет всего несколько процентов, в лучших~--- около 30\%. Причем в 
одних случаях для установления связи между объектами каталогов и/или обзоров 
      кросс-иден\-ти\-фи\-ка\-ции достаточно, а в других случаях нужен более глубокий 
анализ с привлечением любых доступных данных, включая и информацию из 
опубликованных статей. 
      
      Установление связи между записями разных каталогов по смысловому содержанию 
является знанием, которое нужно сохранять и поддерживать. Проекты AstroDAbis и 
ADSASS нацелены на решение этой задачи, что послужит дальнейшему развитию 
виртуальной обсерватории как универсального инструмента исследований.
     
{\small\frenchspacing
{%\baselineskip=10.8pt
\addcontentsline{toc}{section}{Литература}
\begin{thebibliography}{99}

%\bibitem{1-zh}
%\Au{Miley G., De Breuck C.} Distant radio galaxies and their environments~// The Astronomy and 
%Astrophysics Review, 2008. Vol.~15. P.~67--144.
\bibitem{2-zh}
\Au{Parijskij Yu.\,N., Bursov~N.\,N., Lipovka~N.\,M., Soboleva~N.\,S., Temirova~A.\,V.} 
The  RATAN-600 7.6-cm catalog of radio sources from `Experiment Cold-80'~// Astronomy 
Astrophys. Supplement Ser., 1991. Vol.~87. P.~1--32.
\bibitem{3-zh}
\Au{Parijskij Yu.\,N., Soboleva~N.\,S., Goss~W.\,M., Kopylov~A.\,I., Verkhodanov~O.\,V., 
Temirova~A.\,V., Zhelenkova O.\,P.} The RATAN-600\,--\,VLA\,--\,6~m Russian telescope: 
Early Universe Project~// 
175th Symposium of the International Astronomical Union.~---  Bologna: Kluwer Acad. 
Publs., 1996. P.~591--602.
\bibitem{4-zh}
\Au{Verkhodanov O.\,V., Parijskij Yu.\,N., Soboleva~N.\,S., Temirova~A.\,V., 
Zhelenkova~O.\,P.} Color  redshifts and the age of the stellar population of distant RC 
radio galaxies~// Astronomy  Rep., 2002. Vol.~46. P.~531--542.
\bibitem{5-zh}
\Au{Parijskij Yu.\,N., Kopylov~A.\,I., Temirova~A.\,V., Soboleva~N.\,S., Zhelenkova~O.\,P., 
Verkhodanov~O.\,V., Goss~W.\,M., Fatkhullin~T.\,A.} 
Spectroscopy of ``Big Trio'' objects using the 
``Scorpio'' spectrograph of the 6-m telescope of the Special Astrophysical Observatory~// 
Astronomy Rep., 2010. Vol.~54. P.~675--695.
\bibitem{6-zh}
\Au{Soboleva N.\,S., Majorova~E.\,K., Zhelenkova~O.\,P., Temirova~A.\,V., Bursov~N.\,N.}
RATAN-600 7.6-cm deep sky strip surveys at the declination of the SS433 source during the 
1980--1999  period. Data reduction and the catalog of radio sources in the right-ascension 
interval  $7h < R.A. < 17h$~// Astrophys. Bull., 2010. Vol.~65. P.~42--59.
\bibitem{7-zh}
\Au{Abazajian K.\,N., Adelman-McCarthy~J.\,K., Ag$\ddot{\mbox{u}}$eros M.\,A.,
\textit{et al}.} The seventh data release 
of the sloan digital sky survey~// Astrophys. J. Suppl., 2009. Vol.~182. P.~543--558.
\bibitem{8-zh}
\Au{Dye S., Warren S.\,J., Hambly~N.\,C., %Cross~N.\,J.\,G., Hodgkin~S.\,T., Irwin~M.\,J., Lawrence~A., 
\textit{et al}.} 
The UKIRT infrared deep sky survey early data release~// Monthly Notices of the Royal 
Astronomical Society, 2006. Vol.~372. P.~1227--1252.
\bibitem{9-zh} 
\Au{Lasker B.\,M., Lattanzi~M.\,G., McLean~B.\,J., %Bucciarelli~B., 
\textit{et al}.}
The Second-Generation 
Guide Star Catalog: Description and properties~// Astronomical J., 2008. Vol.~136. P.~735--766.
\bibitem{10-zh}
\Au{Monet D.\,G., Levine S.\,E., Canzian~B., %Ables~H.\,D., Bird~A.\,R., 
\textit{et al}.} 
The USNO-B Catalog~// 
Astronomical J.~, 2003. Vol.~125. P.~984--993.
\bibitem{411zh}
\Au{Skrutskie M.\,F., Cutri~R.\,M., Stiening~R., %Weinberg~M.\,D., Schneider~S., Carpenter~J.\,M., 
\textit{et al}.} The two micron all sky survey (2MASS)~// 
Astronomical J., 2006. Vol.~131. P.~1163--1183. 
\bibitem{12-zh}
\Au{Cohen A.\,S., Lane W.\,M., Cotton~W.\,D., %Kassim~N.\,E., Lazio~T.\,J.\,W., 
\textit{et al}.}
The VLA low-frequency sky survey~// Astronomical J., 2007. Vol.~134. P.~1245--1262.
\bibitem{13-zh}  
\Au{Douglas J.\,N., Bash F.\,N., Bozyan~F.\,A., Torrence~G.\,W., Wolfe~C.}
The Texas survey of 
radio sources covering $-35.5 < \delta < 71.5$ at 365~MHz~// Astronomical J., 1996. Vol.~111. 
P.~1945--1963. 
\bibitem{14-zh}
\Au{Condon J.\,J., Cotton W.\,D., Greisen~E.\,W., Yin~Q.\,F., Perley~R.\,A., Taylor~G.\,B., 
Broderick~J.\,J.} The NRAO VLA sky survey~// Astronomical J., 1998. Vol.~115. P.~1693--1716. 
\bibitem{15-zh}  
\Au{Becker R.\,H., Helfand D.\,J., White~R.\,L., Gregg~M.\,D.,
 Laurent-Muehleisen~S.\,A.} The FIRST 
Survey Catalog, Version 2003Apr11~// Astrophysical J., 1997. Vol.~475. P.~479--493. 
\bibitem{16-zh} 
\Au{Gregory P.\,C., Scott W.\,K., Douglas~K., Condon~J.\,J.}
The GB6 catalog of radio sources~// 
Astrophys. J. Suppl., 1996. Vol.~103. P.~427--432. 
\bibitem{17-zh}
\Au{Zhelenkova O.\,P., Kopylov~A., Chernenkov~V.} The investigation of the RC catalog 
radiosources in the SDSS and FIRST crossing area with IVOA program tools~// \mbox{JENAM}. Our 
non-stable universe.~--- Yerevan: BAO, 2007. P.~84--85.
\bibitem{18-zh}
\Au{Zhelenkova O.\,P., Kopylov A.\,I.}
Analysis of the RC catalog sample in the region overlapping 
with the regions of the FIRST and SDSS surveys: I.~Identification of sources with the VLSS, TXS, 
NVSS, FIRST, and GB6 catalogs~// Astrophys. Bull., 2008. Vol.~63. P.~346--356.
\bibitem{19-zh}  
\Au{Zhelenkova O.\,P., Kopylov A.\,I.} Analysis of a sample of RC catalog objects in the region 
overlapping with the areas covered by FIRST and SDSS surveys. II:~Optical identification with the 
SDSS survey and USNO-B1 and 2MASS catalogs~// Astrophys. Bull., 2009. Vol.~64. P.~111--122. 

\bibitem{21-zh} 
\Au{Желенкова О.\,П., Майорова Е.\,К., Соболева~Н.\,С., Темирова~А.\,В.} Многочастотное 
исследование радиоисточников средствами виртуальной обсерватории~// Радиотелескопы, 
аппаратура и методы радиоастрономии: Мат-лы Всеросс. радиоастрономической 
конф. (ВРК-2011).~---  СПб.: ИПА РАН, 2011. С.~179--184. 

\bibitem{20-zh}
\Au{Zhelenkova O.\,P., Soboleva N.\,S., Majorova~E.\,K., Temirova~A.\,V.} Multiband study of 
radiosuorces of the RCR catalogue with the virtual observatory tools~// Baltic Astronomy, 2012. 
Vol.~21. P.~5--13. 

\bibitem{22-zh}
\Au{Желенкова О.\,П., Майорова Е.\,К., Соболева~Н.\,С., Темирова~А.\,В.}
Методы виртуальной 
обсерватории в задаче оптического отождествления радиоисточников~// Электронные 
библиотеки, 2010. Т.~13. Вып.~4. 
\bibitem{23-zh} 
\Au{Желенкова О.\,П.} Исследование радиоисточников средствами виртуальной 
обсерватории~// Электронные библиотеки: перспективные методы и технологии, 
электронные коллекции (RCDL'2011): Труды \mbox{XIII} Всеросс. науч. конф.~--- 
Воронеж: ВГУ, 2011. С.~326--333. 
\bibitem{24-zh}
\Au{Boch T., Oberto A., Fernique~P., Bonnarel~F.}
Aladin: An open source all-sky browser~// 
Astronomical Data Analysis Software and Systems XX.~--- Boston: ASP, 2011. Vol.~442. P.~683--691. 
\bibitem{25-zh}
\Au{Joye W.\,A.}
New features of SAOImage DS9~// Astronomical Data Analysis Software and 
Systems XV.~--- San Lorenzo de El Escorial: ASP, 2006. Vol.~351. P.~574--579. 
\bibitem{26-zh}
\Au{Brunner R.\,J., Djorgovski S.\,G., Lonsdale~C., Madore~B., Prince~T., Szalay~A.\,S.}
 Multi-wavelength cross-identification of the extragalactic sky: An NVO cornerstone~// Bull. 
Amer. Astronomical Soc., 2000. Vol.~32. P.~1601--1605.
\bibitem{27-zh} 
\Au{Quinn P.\,J., Benvenuti~P., Diamond~P.\,J., Genova~F., Lawrence~A., Mellier~Y.}
Astrophysical 
virtual observatory (AVO): A progress report~// SPIE Proceedings, 2002. Vol.~4846. P.~1--5.
\bibitem{28-zh}
\Au{Quinn P.\,J., Barnes D.\,G., Csabai~I., %Cui~Ch., Genova~F., Hanisch~R., Kembhavi~A.,
\textit{et al.}} 
The International Virtual Observatory Alliance: Recent technical developments and the road ahead~// 
SPIE Proceedings, 2004. Vol.~5493. P.~137--145.
\bibitem{29-zh}
\Au{Briukhov D.\,O., Kalinichenko L.\,A., Zakharov~V.\,N., Panchuk~V.\,E., Vitkovsky~V.\,V., 
Zhelenkova~O.\,P., Dluzhnevskaya~O.\,B., Malkov~O.\,Yu., Kovaleva~D.\,A.}
{Information infrastructure 
of the Russian Virtual Observatory (RVO)}.~--- 2nd ed.~--- M.: IPI RAS, 2005.
\bibitem{30-zh}
The open archives initiative protocol for metadata harvesting. 
{\sf http://www.openarchives.org/OAI/\linebreak openarchivesprotocol.htm}.
\bibitem{31-zh}
Dublin core metadata initiative. {\sf http://dublincore.org/ documents}.
\bibitem{32-zh}
\Au{Pence W.\,D., Chiappetti L., Page~C.\,G., Shaw~R.\,A., Stobie~E.}
Definition of the flexible 
image transport system (FITS), version 3.0~// Astronomy Astrophys., 2010. Vol.~524. P.~42--82.
\bibitem{33-zh} 
\Au{Ochsenbein F., Williams R., Davenhall~C., %Durand~D., Fernique~P., 
\textit{et al}.}
 IVOA  recommendation: VOTable format definition Version~1.2~// E-print, 2011. Arxiv:1110.0524. 
 P.~1--35.
\bibitem{34-zh}
\Au{Derriere S., Gray N., Mann~R., Martinez~A.\,P., McDowell~J., McGlynn~T., 
Ochsenbein~F., Osuna~P., Rixon~G., Williams~R.} An IVOA standard for unified content 
descriptors. Version~1.1. {\sf http://www.ivoa.net/ Documents/REC/UCD/UCD-20050812.pdf}.
\bibitem{35-zh}
IVOA data access layer. {\sf http://www.ivoa.net/cgi-bin/\linebreak twiki/bin/view/IVOA/IvoaDAL}.
\bibitem{36-zh} 
\Au{Ortiz I., Lusted J., Dowler~P., %Szalay~A., %Shirasaki~Y., Nieto-Santisteban~M.\,A.,
\textit{et al}.} IVOA 
recommendation: IVOA astronomical data query language Version~2.00~// E-print, 2011. 
Arxiv:1110.0503. P.~1--36.
\bibitem{37-zh}
\Au{Ochsenbein F., Bauer P., Marcout~J.}
The VizieR database of astronomical catalogues~// 
Astronomy Astrophys. Suppl., 2000. Vol.~143. P.~23--32.
\bibitem{38-zh}
\Au{Hambly N.\,C., Collins R.\,S., Cross~N.\,J.\,G., %Mann~R.\,G., Read~M.\,A.,
\textit{et al}.} The WFCAM 
science archive~// Monthly Notices Roy. Astronomical Soc., 2008. Vol.~384. P.~637--662.
\bibitem{39-zh}
\Au{Berriman G.\,B.} The NASA/IPAC Infrared Science Archive (IRSA) as a resource in 
supporting observatory operations~//  SPIE Proceedings, 2008. Vol.~7016. P.~701618--701618-9.
\bibitem{40-zh}
\Au{Kamp I., Thompson R., Conti~A., %Fraquelli~D., Kimball~T., Levay~K., Shiao~B.,
\textit{et al}.} MAST 
in the context of VO activities~// Astronomical Data Analysis Software and Systems XIV.~--- 
Pasadena: ASP, 2005. Vol.~347. P.~208--212.
\bibitem{41-zh}
\Au{Gray N., Mann R.\,G., Morris~D., Holliman~M., Noddle~K.}
AstroDAbis: Annotations and 
cross-matches for remote catalogues~// E-print, 2011. ArXiv:1111.6116. P.~1--4. 
\bibitem{42-zh}
Resource description framework. {\sf http://www.w3.org/\linebreak standards/techs/rdf}.
\bibitem{43-zh}
Linked data. {\sf http://www.w3.org/DesignIssues/\linebreak LinkedData.html}.
\bibitem{44-zh}
\Au{Holliman M., Alemu~T., Hume~A., van Hemert~J., Mann~R.\,G., Noddle~K., Valkonen~L.} 
Service infrastructure for cross-matching distributed datasets using OGSA-DAI and TAP~// 
Astronomical Data Analysis Software and Systems XX.~--- Boston: ASP, 2011. Vol.~442. P.~579--583.
\bibitem{45-zh} 
 OGSA-DAI. {\sf http://www.ogsadai.org.uk/about/\linebreak index.php}.
\bibitem{46-zh}
\Au{Dowler P., Rixon G., Tody~D.} Table access protocol (TAP, v1.0), IVOA recommendation~// 
ArXiv:1110.0497. {\sf http://www.ivoa.net/Documents/TAP}. 
\bibitem{47-zh}
\Au{Taylor M.\,B.}
TOPCAT \& STIL: Starlink Table/VOTable Processing Software~// 
Astronomical Data Analysis Software and Systems XIV.~--- Pasadena: ASP, 2005. Vol.~347. P.~29--32.
\bibitem{48-zh} 
\Au{Laurino O., Smareglia R.} 
VOdka: A~data keeping-up agent for the virtual observatory~// 
Astronomical Data Analysis Software and Systems XX.~--- Boston: ASP, 2011. Vol.~442. P.~571--574.
\bibitem{49-zh}
\Au{Brescia M., Longo G., Djorgovski~G.\,S., %Cavuoti~S., D'Abrusco~R., Donalek~C.,
\textit{et al}.} DAME: 
A~web oriented infrastructure for scientific data mining \& exploration~// E-print, 2011. 
Arxiv:1111.3983. P.~1--16.
\bibitem{50-zh}
\Au{Pepe A., Goodman A., Muench~A.} The ADS all-sky survey~// E-print, 2011. 
Arxiv:1111.6116. P.~1--4.
\bibitem{51-zh}
\Au{Zhelenkova O., Vitkovskij V.\,V., Briukhov~D., Kalinichenko~L.\,A.}
Search of distant radio 
galaxies as a subject mediator example~// Astronomical Data Analysis Software and Systems XV.~---
San Francisco: ASP, 2006. Vol.~351. P.~244--249. 
\bibitem{52-zh}
\Au{Брюхов Д.\,О., Вовченко А.\,Е., Захаров~В.\,Н., Желен\-кова~О.\,П., Калиниченко~Л.\,А., 
Мартынов~Д.\,О., Скворцов~Н.\,А., Ступников~С.\,А.} Архитектура промежуточного слоя 
предметных посредников для \mbox{решения} задач над множеством неоднородных распределенных 
информационных ресурсов в гибридной грид-инфра\-струк\-ту\-ре виртуальных обсерваторий~//\linebreak 
Информатика и её применения, 2008. Т.~2. Вып.~1. С.~2--34.
\bibitem{53-zh}
\Au{Walton N.\,A.}
The AstroGrid Consortium. The AstroGrid Virtual Observatory Service~// 
Astronomical Data Analysis Software and Systems XVII.~--- London: ASP, 2008. Vol.~394. P.~251--255.

\label{end\stat}

\bibitem{54-zh}
\Au{Oke J.\,B., Gunn J.\,E.} Secondary standard stars for absolute spectrophotometry~// 
Astrophys. J., 1983. Vol.~266. P.~713--717.
\end{thebibliography}
}
}


\end{multicols}