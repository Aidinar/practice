\def\stat{zagor}

\def\tit{МЕТОДОЛОГИЧЕСКИЕ АСПЕКТЫ РАЗРАБОТКИ ЭЛЕКТРОННОГО 
РУССКО-АНГЛИЙСКОГО ТЕЗАУРУСА ПО~КОМПЬЮТЕРНОЙ 
ЛИНГВИСТИКЕ$^*$}

\def\titkol{Методологические аспекты разработки электронного 
русско-английского тезауруса по~компьютерной 
лингвистике}

\def\autkol{Ю.\,А.~Загорулько, О.\,И.~Боровикова, И.\,С.~Кононенко, 
Е.\,Г.~Соколова}
\def\aut{Ю.\,А.~Загорулько$^1$, О.\,И.~Боровикова$^2$, И.\,С.~Кононенко$^3$, 
Е.\,Г.~Соколова$^4$}

\titel{\tit}{\aut}{\autkol}{\titkol}

{\renewcommand{\thefootnote}{\fnsymbol{footnote}}\footnotetext[1]
{Работа выполнена при финансовой поддержке РГНФ (проект № 10-04-12108в).}}


\renewcommand{\thefootnote}{\arabic{footnote}}
\footnotetext[1]{Институт систем информатики имени А.\,П.~Ершова СО РАН, zagor@iis.nsk.su}
\footnotetext[2]{Институт систем информатики имени А.\,П.~Ершова СО РАН, olesya@iis.nsk.su}
\footnotetext[3]{Институт систем информатики имени А.\,П.~Ершова СО РАН, irina\_k@cn.ru}
\footnotetext[4]{Российский государственный гуманитарный университет, minegot@rambler.ru}


\Abst{Обсуждаются методологические аспекты разработки русско-английского 
электронного тезауруса по компьютерной лингвистике (КЛ). Обосновывается необходимость 
разработки такого тезауруса и принципы его построения. Описываются состав тезауруса, 
структура тезаурусной статьи и набор связей между терминами. Обсуждается методика 
выбора терминов для включения в тезаурус, а также проблемы выбора основного 
тер\-ми\-на-дескрип\-то\-ра из множества синонимичных терминов и подбора парных 
тер\-ми\-нов-экви\-ва\-лен\-тов. Рассматриваются особенности реализации электронной версии тезауруса, при 
этом особое внимание уделяется проблеме поддержания логической целостности 
терминологической системы тезауруса и обеспечению удобного доступа к его 
содержимому.}

\KW{многоязычный тезаурус; компьютерная лингвистика; методология разработки 
тезаурусов; онтология; концептуальная схема тезауруса; технология построения порталов 
научных знаний}

\vskip 14pt plus 9pt minus 6pt

      \thispagestyle{headings}

      \begin{multicols}{2}

            \label{st\stat}

\section{Введение}

  Обеспечить обработку и эффективное использование постоянно растущих объемов 
неструктурированной информации становится уже невозможно без привлечения методов 
КЛ. Чтобы успешно приме\-нять эти методы для решения задач 
индексирования и содержательного поиска документов,\linebreak извлече\-ния информации из текстов, 
машинного перевода и построения есте\-ст\-вен\-но-язы\-ко\-вых интерфейсов, требуется знание 
терминологии КЛ, причем не только русскоязычной, но и англоязычной, так как 
большинство учебников и научных публикаций по тематике КЛ представлено на английском 
языке. Однако на данный момент в КЛ не существует четкой и общепринятой системы 
научной терминологии, причем многие термины современной КЛ не представлены на 
русском языке ни в одном из лингвистических источников.
  
  Так, тезаурус по теоретической и прикладной лингвистике, созданный в 1978~г.\ 
С.\,Е.~Никитиной~[1], уже устарел. К~тому же он одноязычный и не содержит определений 
понятий. Англо-русский терминологический словарь В.\,З.~Демьянкова~[2] содержит 
толкования, но не отражает современную картину этого научного направления. 
  
  Собственно лингвистика представлена в нескольких фундаментальных источниках, в 
част\-ности в Лингвистическом энциклопедическом словаре (ЛЭС)~[3], словаре О.\,С.~Ахмановой~[4], а также интернет-энциклопедии 
<<Кругосвет>>~[5], содержащей статьи по новым для традиционной лингвистики понятиям. 
Разработанный в 2007~г.\ в ИНИОН РАН тезаурус по языкознанию~[6] содержит около 
3000~терминов, однако только около 4\% из них относятся к области КЛ.
  
  Определения терминов КЛ можно найти в толковом словаре по искусственному 
интеллекту~[7]. Однако он отражает терминологию на конец\linebreak 1980-х~гг.\ и содержит 
довольно мало терминов КЛ.
  
  Так как КЛ имеет междисциплинарный характер, то некоторые ее термины можно найти в 
общих энциклопедиях, например в Большом энциклопедическом словаре~[8]. Популярным источником знаний по КЛ сейчас 
является Википедия~[9], в которой можно найти объяснения, классификации и ссылки на 
источники по многим понятиям КЛ, однако эти сведения часто страдают односто\-рон\-ностью, 
неполнотой и эскизностью.
  
  Таким образом, на данный момент не существует источника, в котором вся терминология 
КЛ была бы приведена в единую систему. Это вызывает необходимость разработки 
двуязычного тезауруса, содержащего английские и русские термины КЛ и их толкования. 
Двуязычность тезауруса даст возможность отечественным ученым и специалистам быстрее и 
эффективнее ориентироваться в мировой ситуации в данной области. Составление такого 
тезауруса позволит выявлять различия и сходства между понятиями, используемыми в 
отечественной и зарубежной науке, а также вводить новые понятия и лингвистические 
термины, отсутствующие в русском языке~[10].
  
  В данной работе обсуждаются методологические аспекты разработки русско-английского 
тезауруса по компьютерной лингвистике. В~разд.~2 описываются принципы разработки и 
состав тезауруса, структура тезаурусной статьи и набор связей между терминами. В~разд.~3 
обсуждается методика выбора терминов для включения в тезаурус, а также проблемы выбора 
основного тер\-ми\-на-де\-скрип\-то\-ра из множества синонимичных терминов и подбора парных 
тер\-ми\-нов-экви\-ва\-лен\-тов. В~разд.~4 рас\-смат\-ри\-ва\-ют\-ся особенности реализации электронной 
версии тезауруса.

\section{Проектирование структуры тезауруса }

  Проектирование структуры русско-английского тезауруса по компьютерной лингвистике 
выполнялось в соответствии с существующими отечественными и международными 
стандартами~[11--15], регламентирующими построение информационно-поисковых 
тезаурусов (ИПТ), а также на основе анализа и обобщения накопленного к этому времени 
опыта разработки ряда отечественных тезаурусов ИНИОН~[16], РуТез~[17] и~др. 
  
  Упомянутые выше стандарты определяют основные единицы тезауруса и возможный 
набор отношений между ними, устанавливают общие правила сбора массива лексических 
единиц, формирования словника, построения словарных статей и оформления ИПТ.
  
  В зависимости от назначения ИПТ могут включать в свой состав либо только 
дескрипторы (предпочтительные термины), либо дескрипторы и аскрипторы (обычные 
термины). Во втором случае дескрипторы могут использоваться при индексировании 
документов и в поисковых запросах, а аскрипторы (как текстовые входы) подлежат замене 
одним или несколькими дескрипторами~\cite{17-zag}.
  
  Тезаурусы делятся на одноязычные и многоязычные. Многоязычный 
  ин\-фор\-ма\-ци\-он\-но-по\-иско\-вый тезаурус (МИПТ) содержит термины из нескольких естественных языков и 
представляет эквивалентные по смыслу понятия на каждом из них. 

\columnbreak
  
  Построение русско-английского тезауруса по КЛ выполнялось в соответствии с 
требованиями межгосударственного стандарта ГОСТ 7.24-2007~\cite{11-zag}, который 
разработан с учетом основных нормативных положений международного стандарта ISO 
5964-1985~\cite{12-zag} и устанавливает состав, структуру и основные требования к 
построению МИПТ. Тезаурус разрабатывался как набор одноязычных версий МИПТ, при 
этом выполнялось согласованное построение одновременно двух версий тезауруса~--- 
русскоязычной и англоязычной. Разработка одноязычных версий тезауруса выполнялась на 
основе международного стандарта ISO 2788-1986~\cite{13-zag}, межгосударственного 
стандарта ГОСТ 7.25-2001~\cite{14-zag} и американского стандарта 
  Z39.19-2005~\cite{15-zag}.

\subsection{Выбор структуры словарной статьи}

  Основными единицами разработанного тезауруса являются термины предметной области 
(ПрО), подразделяемые на дескрипторы и аскрипторы. В~тезаурус включаются следующие 
типы лексических единиц: одиночные слова (преимущественно\linebreak существительные), именные 
словосочетания, лексически значимые компоненты сложных слов, сокращения слов и 
словосочетаний. Близкие по\linebreak смыслу лексические единицы образуют класс эквивалентности, 
при этом одна из них выбирается в качестве представителя этого класса и получает статус 
дескриптора, остальные лексические единицы получают статус аскриптора. Статус 
аскриптора получают также и термины, представляемые аббревиатурами или иными 
вариантами написания (через дефис, с пробелом и~т.\,п.).
  
  В состав словарной статьи термина, вне зависимости от его статуса, входят следующие 
элементы:
\begin{itemize}
\item  \textit{название термина}, т.\,е.\ лексическая единица, представленная в нормальной 
форме (для одиночного существительного или опорного слова словосочетания это форма 
именительного падежа единственного числа);
  
  \item \textit{язык}, на котором дано название термина;
  \item
  \textit{комментарий}, включающий правила и рекомендации использования термина, а 
также замечания и пояснения автора словарной статьи;
  \item
  \textit{автор словарной статьи}, т.\,е.\ фамилия и имя разработчика словарной статьи 
(задается для контроля процесса коллективной разработки тезауруса).
  \end{itemize}
  
  Для описания терминов-дескрипторов, кроме перечисленных выше атрибутов, вводятся 
сле\-ду\-ющие дополнительные атрибуты:
  \begin{itemize}
\item  \textit{определение термина}, поясняющее на языке термина его смысл или значение. 
Наличие в тезаурусе определений терминов делает возможным его использование не только 
в качестве инструмента для ручного или автоматизированного индексирования, но и в 
качестве источника систематизированных знаний о данной ПрО;
  \item
  \textit{релятор}, представляющий собой помету, введенную для различения омонимичных 
терминов (омографов) в рамках описываемой ПрО. Он является частью термина и поясняет 
его значение, относя его к определенной понятийной категории или предметно-тематической 
области (в контексте данной статьи~--- подобласти КЛ или смежной с ней области/подобласти 
знаний). Например, для различения двух понятий, образованных на основе словосочетания 
РАЗМЕТКА ТЕКСТА, могут быть использованы реляторы ПРОЦЕСС и ОБЪЕКТ. 
В~результате получается два разных термина-дескриптора РАЗМЕТКА ТЕКСТА 
(ПРОЦЕСС) и РАЗМЕТКА ТЕКСТА (ОБЪЕКТ);
  \item
  \textit{область/подобласть знаний}, к которой относится данный термин-дескриптор;
  \item
  \textit{признак корневого термина}, указывающий на то, что дескриптор находится на 
самом верхнем уровне одной из представленных в тезаурусе иерархий понятий.
\end{itemize}
  
  Термины тезауруса связываются различными семантическими отношениями, 
отражающими место каждого термина в системе понятий выбранной ПрО.
  
  Для связи дескрипторов с аскрипторами используются отношения синонимии нескольких 
типов. Так, если дескриптор может однозначно во всех контекстах заменить какой-то 
аскриптор, то он связывается с ним отношением <<Синоним>>; при этом также 
устанавливается обратное отношение от аскриптора к дескриптору~--- <<Смотри>>. Для 
моделирования других соотношений между аскрипторами и дескрипторами в соответствии с 
ГОСТ 7.25-2001 в тезаурус вводятся отношения, позволяющие задавать связи между 
аскрипторами и альтернативными дескрипторами или представлять аскриптор комбинацией 
дескрипторов. 
  
  В тех случаях, когда нет однозначного соответствия между дескрипторами и 
аскрипторами, используются отношения <<Используй альтернативно>> или <<Используй 
комбинацию>>, задающие соответствие между аскриптором и заменяющими его 
дескрипторами; при этом вводятся обратные им отношения <<Сравни альтернативный 
выбор>> и <<Сравни комбинацию>>. 

Например, аскриптор ПАРТИЦИПАНТ может\linebreak быть 
связан отношением <<Используй альтерна-\linebreak тивно>> с дескрипторами СЕМАНТИЧЕСКАЯ 
ВА-\linebreak ЛЕНТНОСТЬ и УЧАСТНИК СИТУАЦИИ. В~то\linebreak же время аскриптор СИСТЕМА 
СТАТИСТИЧЕСКОГО МАШИННОГО ПЕРЕВОДА может быть\linebreak представлен с помощью 
связи <<Используй комбинацию>> как комбинация (сочетание) двух дескрипторов~--- 
СИСТЕМА МАШИННОГО ПЕРЕВОДА и СТАТИСТИЧЕСКИЙ МАШИННЫЙ\linebreak  ПЕРЕВОД.
{ %\looseness=1

}
  
  Для отражения семантических связей между\linebreak понятиями, выражаемыми дескрипторами, 
уста\-нав\-ли\-ва\-ют\-ся иерархические и ассоциативные отношения. (Следует заметить, что такого 
типа отношениями связываются только дескрипторы, входящие в одну и ту же одноязычную 
версию тезауруса.)
  
  В тезаурусе допускается использование таких иерархических отношений, как 
недифференцированная иерархическая связь <<Выше>>, направленная от нижестоящего 
дескриптора к вышестоящему; родовидовая связь <<Выше род>>, устанавливаемая между 
двумя дескрипторами, когда объем понятия нижестоящего дескриптора входит в объем 
понятия вышестоящего дескриптора; партонимическая связь <<Выше целое>>, задаваемая 
между двумя дескрипторами в том случае, когда нижестоящий дескриптор представляет 
компонент объекта, обозначаемого вышестоящим дескриптором. Вводятся также обратные 
им отношения: <<Ниже>>, <<Ниже вид>>, <<Ниже часть>>. 
  
  Для задания отношений между дескрипторами, представляющими класс понятий и 
экземпляр этого класса, были выбраны связи <<Выше класс>> и <<Экземпляр класса>>.
  
  При установлении иерархических отношений для некоторых дескрипторов можно указать 
признак <<Аспект деления иерархии>>. Так, например,\linebreak в иерархии, построенной по 
отношению <<\mbox{Ниже} вид>>, МАШИННЫЙ ПЕРЕВОД по признаку <<подход>> разделяется 
на СТАТИСТИЧЕСКИЙ МАШИННЫЙ ПЕРЕВОД, МАШИННЫЙ ПЕРЕВОД НА ОСНОВЕ 
ПРАВИЛ и МАШИННЫЙ ПЕРЕВОД, ОСНОВАННЫЙ НА ПРЕЦЕДЕНТАХ, а по признаку 
<<степень участия человека>>~--- на ПОЛНОСТЬЮ АВТОМАТИЧЕСКИЙ ПЕРЕВОД и 
ЧЕЛОВЕКО-МАШИННЫЙ ПЕРЕВОД.
  
  Таким образом, один и тот же дескриптор одновременно может входить в несколько 
иерархий понятий, построенных по различным отношениям (<<Выше>>, <<Выше род>>, 
<<Выше целое>>) и по различным аспектам деления иерархии. 
  
  Для задания произвольных ассоциативных связей между дескрипторами, например 
отношений, выражающих зависимости вида <<процесс--объект>>, <<причина--следствие>> 
и~др., вводится отношение <<Ассоциируется с>>.
  
  Для связывания эквивалентных по смыслу дескрипторов, входящих в разные 
одноязычные версии, служит отношение <<Эквивалент на другом \mbox{языке}>>. 

\subsection{Представление источников терминов}

  Для подтверждения актуальности введенных в тезаурус терминов и ознакомления 
пользователей тезауруса с практикой их употребления для каждого термина задаются его 
связи с источниками, т.\,е.\ текстовыми документами или коллекциями текстовых 
документов, в которых данный термин встречается или определяется.
  
   Этим целям служат отношения <<Встречается в>>, <<Встречается в части документа>> и 
<<Дается определение в>>.
  
  Отношение <<Встречается в>> служит для связывания любого термина с источником; при 
этом, если источник~--- коллекция текстов, то в качестве значения специального атрибута 
этого отношения можно указать частоту встречаемости термина в источнике.
  
  С помощью отношения <<Встречается в части документа>> можно сообщить, что данный 
термин встречается в предметном указателе или глоссарии источника, что указывает на 
важность термина и повышает степень доверия к нему. 
  
  С помощью отношения <<Дается определение в>> термины-дескрипторы, снабженные 
толкованиями-определениями, связываются с источниками определений.
  
  В тезаурусе источники описываются сле\-ду\-ющи\-ми параметрами: название, 
библиографическая ссылка, язык, тип (книга, монография, научная статья, документация, 
учебник, словарь, тезаурус, интернет-ресурс, коллекция текстов и~др.), краткое описание и 
адрес в сети Интернет. Для коллекции текстов дополнительно задается число текстов и 
словоупотреблений.

\section{Методика выбора терминов для~включения в~тезаурус}

  Важным моментом при построении тезауруса является методика подбора 
  терминов~--- кандидатов на включение в тезаурус,~--- выбор терминов-дескрипторов из 
множеств синонимичных терминов, а также подбор иноязычных эквивалентов. 
  
  Выбор терминов для включения в русско-анг\-лий\-ский тезаурус по КЛ сопряжен с 
трудностями, которые обусловлены особенностями самой КЛ как новейшей науки и 
состоянием ее развития в России. Здесь важно отметить следующие факторы, 
характеризующие КЛ в целом и русскоязычную КЛ (РКЛ) в частности:
  \begin{itemize}
\item  междисциплинарный характер КЛ;
  \item неоднородность ПрО <<Компьютерная лингвистика>>;
  \item неравномерность развития отдельных на\-прав\-ле\-ний КЛ;
  \item отличие русскоязычной КЛ от англоязычной (в частности, отставание отдельных 
направлений РКЛ).
  \end{itemize}
  
  Ранее КЛ рассматривалась как часть исследовательского направления <<искусственный 
интеллект>> (ИИ), терминология которого считается\linebreak зрелой: <<Специальная терминология 
по искусственному интеллекту и интеллектуальным системам начала формироваться в 
  60-е~годы ХХ~в. Первый этап формирования терминологии всегда\linebreak отличается наличием 
многих синонимических терминов, которые используют различные школы и группы 
специалистов. На этом этапе термины быстро возникают и часть из них так же быстро 
исчезает. К~середине 1970-х~гг.\ терминология в области искусственного интеллекта стала 
уста\-нав\-ли\-вать\-ся. Появились термины, которые признало подавляющее большинство 
специалистов. Все эти термины (за редким исключением) по происхождению англоязычные, 
так как именно в США проводились интенсивные исследования в этой области. 
Окончательно основная терминология закрепилась в первой половине 1980-х~гг.>>~[7]. 
  
  Искусственный интеллект~--- это методологическая область, методы которой применимы к разным ПрО, в 
част\-ности активно применяются в КЛ в последнее десятилетие. Терминология КЛ в 
отдельных разделах продолжает сохранять черты первого этапа (наличие большого числа 
синонимов, например в разделе семантических отношений). Искусственный интеллект тоже считается 
междисциплинарной областью, однако по этому параметру ИИ и\linebreak КЛ противоположны: ИИ 
междисциплинарна, потому что ее методы применяются в разных дис\-цип\-ли\-нах, КЛ~--- 
потому что она вбирает в себя разные дисциплины, такие как лингвистика (разделы, 
связанные с обработкой текстов и речи), психология, некоторые разделы ИИ.
   
   Следствием указанных выше факторов является отсутствие русскоязычных учебных и 
лексикографических источников, достаточно полно отражающих структуру современной 
КЛ, в отличие от англоязычных источников, где она представлена детально и отчетливо. До 
сих пор термины РКЛ входили лишь в состав словарей и глоссариев по лингвистике и 
смежным ей областям знаний. Так, имеются источники только по отдельным разделам\linebreak 
смежных областей и КЛ, например по искусственному интеллекту, информационному 
поиску, и\linebreak
 почти полностью отсутствуют русскоязычные термины по другим разделам КЛ, 
например по <<Оценке эффективности систем и методов>> (\textit{Evaluation}). Кроме того, 
один и тот же термин, например \textit{синтаксический анализ}, в таких смежных науках, как 
ИИ и КЛ, имеет разное толкование. 
  
  Учитывая вышеперечисленные особенности КЛ и связанный с ними недостаток 
современной справочной русскоязычной литературы по КЛ, при разработке тезауруса 
использовались источники <<живых>> терминов РКЛ и их толкований, и именно они 
фиксировались в словарных статьях тезауруса. 
  
  В качестве основного источника русскоязычных терминов была выбрана коллекция 
текстов докладов, представленных на международной конференции <<Диалог>>~[18] в 
2000--2010~гг., как <<зеркала>>, отражающего термины РКЛ в их реальном употреблении. 
Собранная коллекция имеет следующие характеристики: число документов~--- 1193, 
объем~--- 4\,610\,694 словоупотреблений, суммарный размер~--- 27,5~МБ.
  
  К данной коллекции была применена словарная технология~[19], с помощью которой на 
базе лингвистических моделей (морфологического и локального синтаксического анализа) и 
статистических показателей был создан список статистически значимых в данной ПрО слов 
и словосочетаний~--- кандидатов в термины ПрО. Затем этот список был обработан 
(отфильтрован) экспертами в области КЛ, которые существенно опирались не только на 
знания о предмете и направлениях КЛ, но и на общелингвистические представления о 
терминологичности и путях формирования терминологических словников. Таким образом, 
избранный авторами подход, учитывающий предварительное структурирование ПрО, 
согласуется с общей методикой формирования словников на базе классификационных схем 
предметных областей (см., например,~[20]).
  
  Для английской части словника, с учетом русско-английской направленности 
создаваемого тезауруса, выбирались переводные эквиваленты из доступных англоязычных 
источников по КЛ.
  
  С другой стороны, чтобы дополнить картину РКЛ в тех ее разделах, где имеются пробелы, 
при\linebreak сборе терминов по таким разделам пришлось опираться преимущественно на 
англоязычные источники. Так, учитывая скачок, совершенный в\linebreak течение последних 
нескольких лет в такой высокотехнологичной подобласти КЛ, как <<Речевые технологии>>, 
а также тот факт, что это направление слабо представлено в коллекции <<Диалог>>, при 
сборе терминов для этой подобласти была применена обратная методика, т.\,е.\ в качестве 
основных использовались англоязычные источники: предметные указатели нескольких 
современных и наиболее авторитетных англоязычных книжных источников 
  обзорно-учебного профиля и глоссарии, входящие в документацию известных звуковых 
анализаторов. На данной терминологической базе был составлен англо-русский словник 
параллельных терминов.
  
  Достаточно сложной оказалась и проблема выбора основного термина-дескриптора из 
множества синонимичных терминов. Прежде всего, эта проблема связана с появлением 
новых понятий и соответствующих им терминов. Так, появление систем \textit{translation 
memory} в сфере автоматизированного перевода привело к широкому использованию 
практиками-переводчиками термина \textit{память переводов}, который не был принят 
научным сообществом, противопоставившим ему термин \textit{переводческая память} 
(синонимический ряд терминов с частотными характеристиками из коллекции <<Диалог>>: 
\textit{переводческая память}~--- 8, \textit{память переводов}~--- 0, \textit{архив 
переводов}~--- 1, \textit{накопитель переводов}~--- 0, \textit{копилка переводов}~--- 0).
  
  Развитие некоторых направлений КЛ (например, таких как \textit{автоматический 
перевод в режиме онлайн}) приводит к столкновению вариантов старых терминов. Так, 
тезаурус ИНИОН~[6] и ЛЭС~[3] основным термином в паре \textit{автоматический 
перевод} и \textit{машинный перевод} считают \textit{автоматический перевод}, присвоив 
ему статус дескриптора. Однако показатели встречаемости в коллекции <<Диалог>> говорят 
в пользу термина \textit{машинный перевод}: 318 против~58. Интернет-энциклопедии 
<<Википедия>> и <<Кругосвет>>, а также учебники придерживаются этой же традиции. На 
сайте Европейской ассоциации машинного перевода~[21] также отмечается, что термин 
\textit{machine translation}, хоть и звучит архаично, но, тем не менее, сохраняется как 
основной общий термин для всей области. В~данном случае эксперты согласились с этой, 
соответствующей традиции, точкой зрения.
   
   Проблема выбора дескриптора возникает на фоне незрелости системы понятий КЛ, 
приводящей в некоторых случаях к очень широкой ва\-ри\-а\-тив\-ности терминов, с одной 
стороны, и к их мно\-го\-знач\-ности, с другой. Так, термин \textit{валентная структура} с
частотностью 20 имеет целый ряд вариантов: \textit{валентная рамка}~--- 14, \textit{рамка 
валентностей}~--- 64,\linebreak
\textit{валентностная структура}~--- 3, \textit{схема 
валентностей}~--- 3. В~то же время исследование реального упо\-треб\-ле\-ния термина 
\textit{валентная структура} показало, что он, как и термин \textit{модель управления}, 
имеет как узкое толкование (множество синтаксических валентностей предикатного слова), 
так и более широкое толкование (описание соответствия семантических валентностей слова 
их грамматическому оформлению, т.\,е.\ синтаксическим валентностям). В~этой ситуации в 
тезаурус вводится два одинаковых дескриптора, один из которых снабжается релятором.
    
    Серьезные трудности возникают при подборе парных терминов (англо-русских 
эквивалентов). В~качестве примера можно привести термин \textit{spoken language machine 
translation}. Задача автоматического перевода устной речи возникла на стыке <<Машинного 
перевода>> и <<Речевых технологий>>. \textit{Spoken language processing} обычно 
переводится как \textit{автоматическая обработка устного языка}, одной из задач которой 
является автоматический устный перевод (АУП) с его разновидностями, соответствующими 
АУП типа <<$\mbox{Речь}(L_1) \rightarrow \mbox{\ Текст}(L_2)$>> и АУП типа 
<<$\mbox{Речь}(L1) \rightarrow\ \mbox{Речь}(L_2)$>>. Вторая разновидность представлена 
английским термином \textit{speech-to-speech translation}. В~русскоязычной литературе 
такой традиции нет, как нет (или практически нет) и такого типа приложений. Поиск в 
Интернете дал в качестве эквивалента для \textit{spoken language machine translation} 
единично встретившийся вариант \textit{автоматический перевод устной речи}. Этот 
русский переводной эквивалент и был выбран в качестве парного русскоязычного термина-
дескриптора.
  
  Таким образом, при выборе терминов-де\-скрип\-то\-ров авторы опирались не только на 
статистику, но и на традиции словоупотребления, сложившиеся к настоящему времени в 
лингвистическом научном сообществе. Что же касается выбора парных терминов для 
новейших подобластей КЛ, не представленных в русскоязычной литературе, 
соответствующие дескрипторы предлагались как переводные эквиваленты, а в качестве 
основных критериев выбора перевода выступили знания и интуиция эксперта.

\section{Подход к реализации электронной версии тезауруса}

  Для реализации электронной версии тезауруса было решено использовать методологию и 
программные компоненты технологии построения\linebreak
порталов научных знаний~[22, 23], уже 
ранее применявшиеся при создании порталов знаний по археологии~[24] и компьютерной 
лингвистике~[25].\linebreak
Данная технология базируется на онтологии и предоставляет средства 
настройки на предметную область и управления контентом информационной системы, а 
также средства навигации и поиска. Средства настройки на предметную область и 
поддерживаемая ими методология достаточно хорошо подходят для разработки 
концептуальной схемы тезауруса, а остальные из перечисленных средств могут выполнять 
роль его основных программных компонентов, обеспечивающих создание, сопровождение и 
использование тезауруса.

\subsection{Разработка онтологии представления тезауруса}

  В используемой технологии в качестве информационной модели портала знаний 
(информа\-ци\-он\-ной системы) используется онтология, \mbox{которая}, обеспечивая формальное 
описание предметной области системы, не только определяет структуры для его 
информационного наполнения (контента), но и задает базис для организации 
содержательного доступа к знаниям и данным, содержащимся в нем.
  
  Для описания онтологии данная технология предоставляет формализм, который назовем 
онтологией представления знаний, и поддерживающий его редактор онтологии. С помощью 
этих средств была построена онтология представления тезауруса $O_{Th}$, задающая его 
концептуальную схему: 
  $$
  O_{\mathrm{Th}} =\big\langle C,R,T,D,At, P,Axt\big\rangle
  $$
  где $C=\{\mathrm{Tr}, S_T,S_K\}$~--- конечное непустое множество классов, представляющих 
основные сущности\linebreak тезауруса; здесь $\mathrm{Tr=Asc\,Tr=Asc\cup Des}$~--- класс терминов, 
представляющих понятия ПрО <<Компьютерная лингвистика>>, включающий два\linebreak 
подкласса~--- Asc (тер\-ми\-ны-аскрип\-то\-ры) и Des (тер\-ми\-ны-де\-скрип\-то\-ры); $S_T$~--- 
класс источников терминов; $S_K$~--- класс областей/подобластей знаний;
  $R\hm=R^{\mathrm{TT}}\cup R^{\mathrm{TST}}\cup R^{\mathrm{TSK}}$~--- конечное\linebreak
   множество отношений, где 
  $R^{\mathrm{TT}}\hm=\{R_a^{\mathrm{TT}},\ldots , R_m^{\mathrm{TT}}\}$, 
  $R_i^{\mathrm{TT}}\subseteq \mathrm{Tr}\times \mathrm{Tr}$~--- конечное 
множество бинарных отношений, заданных на терминах, 
  $R^{\mathrm{TST}}\hm=\{ R^{\mathrm{TSF}}, R^{\mathrm{TSP}}, R^{\mathrm{TSD}}\}$, 
  $R_i^{\mathrm{TST}}\subseteq \mathrm{Tr}\times S_T$~--- 
бинарные отношения, связывающие термины тезауруса с источниками, причем $R^{\mathrm{TSF}}$ 
связывает термин с источником, где он встречается, $R^{\mathrm{TSP}}$ связывает термин с 
источником, где он встречается в предметном указателе или глоссарии, а $R^{\mathrm{TSD}}$~--- 
связывает термин с источником, где дается его определение;
  $R^{\mathrm{TSK}}\hm=\{R^{\mathrm{SKT}},R^{\mathrm{SKS}}\}$~--- бинарные отношения, служащие для встраивания 
областей знаний в тезаурус, где $R^{\mathrm{SKT}}\subseteq \mathrm{Tr}\times S_K$ связывает термины 
тезауруса с областями знаний, а $R^{\mathrm{SKS}}\subseteq S_K\times S_K$ задает иерархию на 
подобластях знаний;
  $T$~--- множество стандартных типов;
  $D\hm=\{d_1,\ldots, d_n\}$~--- множество доменов $d_i\hm=\{s_1,\ldots, s_k\}$, где $s_i$~--- 
значение стандартного типа \textit{string}; 
  $at=\{at_1,\ldots, at_w\}$~--- конечное множество атрибутов, описывающих свойства 
основных сущностей тезауруса и отношений между ними; значения этих свойств определены 
на множестве $T\cup D$;
  $P=\{P_1,\ldots, P_n\}$~--- множество формальных свойств отношений~$R^{\mathrm{TT}}$;
  Axt~--- множество аксиом, задающих дополнительные ограничения на связи между 
терминами. 
  
  Таким образом, онтология представления тезауруса описывает классы, представляющие 
основные сущности тезауруса (термины тезауруса, их источники, области/подобласти 
знаний), отношения, связывающие объекты этих классов между собой, свойства понятий и 
отношений, а также аксиомы, определяющие их дополнительную семантику. Кроме того, в 
онтологии задается множество доменов, т.\,е.\ возможных значений атрибутов классов и 
отношений, что позволяет уменьшить число ошибок при создании/редактировании 
конкретного тезауруса.
  
  Для отношений в онтологии задаются математические свойства (симметричность, 
рефлексивность, транзитивность, асимметричность, антирефлексивность) и обратные 
отношения.
  
  Так, для введенных в подразд.~2.1 иерархических отношений (<<Выше>>, <<Выше род>>, 
<<Выше класс>>,\linebreak <<Выше целое>>) задаются математические  свойства <<транзитивность>> 
и <<асимметричность>> и соответствующие обратные отношения (<<Ниже>>, <<Ниже 
вид>>, <<Экземпляр класса>>, <<Ниже часть>>).\linebreak Отношения <<Эквивалент на другом 
языке>> и <<Ассоциируется с>> объявляются симметричными и антирефлексивными. Для 
отношений, выражающих синонимию терминов (<<Синоним>>, <<Используй\linebreak 
альтернативно>>, <<Используй комбинацию>>), задаются обратные отношения 
(соответственно <<Смот\-ри>>, <<Сравни альтернативный выбор>>, <<Сравни 
комбинацию>>).

\subsection{Организация управления контентом тезауруса}

  Для описания конкретных терминов, их источников, областей знаний, а также для 
установления связей между ними используется редактор данных, предоставляемый 
технологией построения порталов знаний и управляемый онтологией пред\-став\-ле\-ния 
тезауруса. Этот редактор реализован как веб-при\-ло\-же\-ние и доступен зарегистрированным 
пользователям через Интернет. (Заметим, что сразу после завершения ввода и/или 
редактирования описаний терминов, источников и связей между ними, новая информация 
становится доступной через пользовательский веб-интерфейс тезауруса.)
  
  С целью обеспечения распределенной коллективной разработки используемая технология 
поддерживает механизм делегирования прав экспертам разных уровней. В~соответствии с 
этим механизмом только эксперты самого высокого уровня могут редактировать структуры 
тезауруса (с помощью редактора онтологий), а эксперты более низких уровней~--- только его 
содержание (с помощью редактора данных).
  
  Кроме того, действует правило, по которому редактировать словарную статью может 
только ее автор. Если кто-то из экспертов захочет внести изменения в <<чужую>> статью, он 
должен согласовать такую возможность с ее автором, в частности, через специальный 
форум, на который имеется ссылка в электронном тезаурусе. 
  
  Для того чтобы тезаурус мог использоваться при индексировании и поиске текстовых 
документов, он должен представлять целостную и непротиворечивую систему понятий ПрО. 
Это обеспечивается встроенными в редактор механизмами вывода и поддержки логической 
целостности системы понятий тезауруса, работа которых базируется на описаниях свойств 
классов и отношений, заданных в онтологии представления тезауруса.
  
  В частности, на основе этих свойств происходит корректное установление связей между 
терминами тезауруса, при необходимости осуществляется их автоматическое добавление 
и/или удаление. Кроме того, контролируются ограничения на существование и число связей 
между терминами тезауруса в зависимости от принадлежности терминов к тем или иным 
классам.
  
  Например, если для рассмотренного выше отношения <<Смотри>> задано обратное 
отношение (<<Синоним>>) и ограничение на существование связей (<<только одна связь 
данного типа для каж\-до\-го тер\-ми\-на-аскрип\-то\-ра>>), то при связывании аскриптора ПАМЯТЬ 
ПЕРЕВОДОВ и дескриптора ПЕРЕВОДЧЕСКАЯ ПАМЯТЬ отношением 
\textit{Смот\-ри}(ПАМЯТЬ ПЕРЕВОДОВ, ПЕРЕВОДЧЕСКАЯ ПАМЯТЬ) произойдет создание 
обратной связи \textit{Си\-но\-ним}(ПЕРЕВОДЧЕСКАЯ ПАМЯТЬ, ПАМЯТЬ ПЕРЕВОДОВ) 
(если таковой еще не существует), а также для аскриптора ПАМЯТЬ ПЕРЕВОДОВ будет 
обеспечиваться запрет на создание связей <<Смот\-ри>> и <<Синоним>> с другими 
дескрипторами.

\subsection{Обеспечение доступа к~контенту тезауруса}

  Удобный доступ к терминам тезауруса обеспечивается пользовательским 
  веб-интерфейсом, также\linebreak предоставляемым технологией построения порталов научных 
знаний. В~этом интерфейсе содержимое тезауруса представляется пользователю в виде сети 
взаимосвязанных информационных\linebreak\vspace*{-12pt}

\pagebreak

\end{multicols}

\begin{figure}
 \vspace*{1pt}
 \begin{center}
 \mbox{%
 \epsfxsize=160mm
 \epsfbox{zag-1.eps}
 }
 \end{center}
 \vspace*{3pt}
\begin{center}
{\small Представление термина <<Переводческая память>>}
\end{center}
\vspace*{6pt}
  \end{figure}


\begin{multicols}{2}

\noindent
 объектов~--- элементов тезауруса: терминов 
(дескрипторов и аскрипторов) и описаний источников терминов и их определений. Набор 
атрибутов терминов и связей, установленных между ними, соответствует структуре 
тезауруса, описанной в подразд.~2.1.
  
  При навигации по тезаурусу обеспечивается возможность выбора необходимых 
пользователю терминов, детального просмотра их описаний (тезаурусных статей), а также 
описаний источников\linebreak (публикаций или коллекций текстов), в которых встречается термин 
и/или его определение.
  
  Пользователь может указать, какой тип информации его интересует~--- все термины, 
дескрипторы, аскрипторы, подобласти знаний или источники терминов. При этом ему 
выдается упорядоченный по алфавиту полный список имеющихся в тезаурусе объектов 
выбранного класса, который отображается в виде html-страницы, содержащей набор ссылок 
на эти объекты. 
  
  Информация о конкретном объекте и его связях также отображается в виде html-страницы 
(см.\ рисунок). При этом объекты, связанные с данным объектом, представляются на его 
странице в виде гиперссылок, по которым можно перейти к их детальному описанию.
  
  Дальнейшая навигация по тезаурусу представляет собой процесс перехода от одних 
объектов тезауруса к другим по заданным между ними связям, отражающим существующие 
между ними~--- тезаурусные (между терминами) или библиографические (между терминами 
и источниками)~--- отношения.


   
  Для обеспечения доступа к содержимому тезауруса из внешних систем разработан 
программный интерфейс, благодаря которому тезаурус может использоваться при решении 
задач индексирования и поиска текстовых документов по КЛ.
  
  Интерфейс поддерживает поиск терминов в тезаурусе по типу, наименованию или части 
наименования. Для каждого термина-дескриптора или аскриптора можно получить список 
связанных терминов по выбранному отношению (синонимия, эквивалентность, ассоциация 
и~т.\,п.). Дополнительно для каждого термина-дескриптора можно получить его перевод, 
список терминов-дескрипторов, связанных родовидовыми отношениями в соответствии с 
аспектом организации иерархии, список подобластей знаний, к которым относится данный 
термин, а также список источников (текстовых документов или коллекций текстов), в 
которых описан данный термин.

\section{Заключение}

  В статье рассмотрены методологические аспекты построения русско-английского 
электронного тезауруса по компьютерной лингвистике, разработанного в соответствии с 
международными и отечественными стандартами. Описана методика выбора терминов для 
включения в тезаурус, а также предложены подходы к выбору основного тер\-ми\-на-де\-скрип\-то\-ра 
из множества синонимичных терминов и подбору парных терминов-эквивалентов. Рассмотрены особенности реализации электронной версии тезауруса, 
обусловленные использованием в качестве инструмента разработки методологии и 
программных компонентов технологии построения порталов научных знаний~[22, 23].
  
  В настоящее время ведется активная разработка тезаурусных статей и заполнение ими 
контента электронного тезауруса, который на данный момент включает более 
1600~терминов, связанных примерно 8000~семантических отношений, а также описания 
более 180~источников терминов из 50~подобластей знаний. 
  
  Тезаурус ориентирован как на непосредственное использование людьми, желающими 
обратиться к системе понятий из области КЛ, так и для решения задач индексирования, 
тематического рубрицирования и информационного поиска (для этого он снабжен 
программным интерфейсом).
  
  Тезаурус может использоваться в учебном процессе~--- в тех вузах страны, где изучается 
компьютерная лингвистика и/или используются ее\linebreak
 результаты. Использование тезауруса в 
вузах повысит уровень профессиональной подготовки будущих специалистов в сфере КЛ и 
информационных технологий. По существу, это необходимый и профессионально 
выполненный инструмент и ресурс обучения, особенно ценный в виду междисциплинарной 
природы КЛ и полного отсутствия каких-либо русскоязычных учебников и даже 
методических пособий в этой новой и быстро развивающейся области знаний.


{\small\frenchspacing
{%\baselineskip=10.8pt
\addcontentsline{toc}{section}{Литература}
\begin{thebibliography}{99}
  
  \bibitem{1-zag}
  \Au{Никитина С.\,Е.}
  Тезаурус по теоретической и прикладной лингвистике.~--- М.: Наука, 1978.
  
  \bibitem{2-zag}
  \Au{Демьянков В.\,З.}
  Англо-русские термины по прикладной лингвистике и автоматической переработке 
текста. Вып.~2. Методы анализа текста~// Тетради новых терминов. №\,39.~--- М.: ВЦП, 
1982.
  
  \bibitem{3-zag}
  Лингвистический энциклопедический словарь~/ Под ред. В.\,Н.~Ярцевой.~--- М.: 
Советская энциклопедия, 1990. 

\bibitem{4-zag}
\Au{Ахманова О.\,С.}
Словарь лингвистических терминов.~--- 3-е изд., стер.~--- М.: КомКнига, 2005. 
  
  \bibitem{5-zag}
  Кругосвет: Онлайн-энциклопедия, 2001--2009. {\sf http://www.krugosvet.ru}.
  
  \bibitem{6-zag}
  Языкознание: Информационно-поисковый тезаурус ИНИОН.~--- М.: ИНИОН РАН, 2007.
  
  \bibitem{7-zag}
  \Au{Аверкин А.\,Н., Гаазе-Рапопорт~М.\,Г., Поспелов~Д.\,А.}
  Толковый словарь по искусственному интеллекту.~--- М.: Радио и связь, 1992.
  
  \bibitem{8-zag}
  Большой энциклопедический словарь (БЭС)~/ Гл. ред. А.\,М.~Прохоров.~---2-е изд., 
перераб. и доп.~--- СПб.: Норинт, 2004. 
  
  \bibitem{9-zag}
  Википедия: Свободная энциклопедия. {\sf http://ru. wikipedia.org}.
  
  \bibitem{10-zag}
  \Au{Соколова Е.\,Г., Семенова С.\,Ю., Кононенко~И.\,С., Загорулько~Ю.\,А., 
Кривнова~О.\,Ф., Захаров~В.\,П.}
  Особенности подготовки терминов для русско-английского тезауруса по компьютерной 
лингвистике~// Компьютерная лингвистика и интеллектуальные технологии: По мат-лам 
ежегодной междунар. конф. <<Диалог>> (Бекасово, 25--29~мая 2011).~--- М.: РГГУ, 2011. 
Вып.~10(17). С.~644--655.
  
 
  \bibitem{12-zag} %11
  ISO 5964-1985. Documentation~--- Guidelines for the establishment and development of 
multilingual thesauri, IDT. (Revised by: ISO/DIS 25964-1. Under development.)
  
  \bibitem{13-zag} %12
  ISO 2788-1986. Documentation~--- Guidelines for the establishment and development of 
monolingual thesauri. Ed.~2.
  
  \bibitem{14-zag} %13
  ГОСТ 7.25-2001. Система стандартов по информации, библиотечному и издательскому 
делу. Тезаурус ин\-фор\-мационно-поисковый одноязычный. Правила разработки, структура, 
состав и форма представления. Введен в действие с 1~июля 2002~г.
  
  \bibitem{15-zag} %14
  ANSI/NISO Z39.19-2005 Guidelines for the construction, format, and management of 
monolingual controlled vocabularies: Periodic review.

  \bibitem{11-zag} %15
  ГОСТ 7.24-2007. Система стандартов по информации, библиотечному и издательскому 
делу. Тезаурус информационно-поисковый многоязычный.\linebreak Состав, структура и основные 
требования к по\-стро\-ению. Введен в действие с 1~июля 2008~г.
  
  \bibitem{16-zag} %16
  \Au{Мдивани Р.\,Р.}
   О~разработке серии тезаурусов по социальным и гуманитарным наукам~// НТИ, 2004. 
Сер.~2. №\,7. С.~1--9.
  
  \bibitem{17-zag}
  \Au{Лукашевич Н.\,В.}
  Тезаурусы в задачах информационного поиска.~--- М.: Изд-во Московского ун-та, 2011.
  
  \bibitem{18-zag}
  Диалог: Сайт международной конференции. {\sf http:// www.dialog-21.ru}.
  
  \bibitem{19-zag}
  \Au{Сидорова Е.\,А.}
  Многоцелевая словарная подсистема извлечения предметной лексики~// Компьютерная 
лингвистика и интеллектуальные технологии: Труды междунар. конф. Диалог'2008.~--- М.: 
РГГУ, 2008. Вып.~7(14). С.~475--481.
  
  \bibitem{20-zag}
  \Au{Перерва В.\,М.}
  О~принципах и проблемах отбора терминов и составления словника терминологических 
словарей~// Проблематика определений терминов в словарях разных типов.~--- Л., 1976. 
С.~190--204.
  
  \bibitem{21-zag}
  EAMT (The European Association for Machine Translation ). {\sf http://www.eamt.org}.
  
  \bibitem{22-zag}
  \Au{Загорулько Ю.\,А., Боровикова~О.\,И.}
  Подход к построению порталов научных знаний~// Автометрия, 2008. Т.~44. №\,1. 
  С.~100--110.
  
  \bibitem{23-zag}
  \Au{Загорулько Ю.\,А.}
  Технология разработки порталов научных знаний~// Программные продукты и системы, 
2009. №\,4. С.~25--29.
  
  \bibitem{24-zag}
  \Au{Андреева О.\,А., Боровикова О.\,И., Булгаков~С.\,В., Загорулько~Ю.\,А., 
Сидорова~Е.\,А., Циркин~Б.\,Г., Холюшкин~Ю.\,П.}
  Археологический портал знаний: содержательный доступ к знаниям и информационным 
ресурсам по археологии~// КИИ-2006: Труды 10-й Национальной конф. по искусственному 
интеллекту с международным участием.~--- М.: Физматлит, 2006. Т.~3. С.~832--840.

\label{end\stat}
  
  \bibitem{25-zag}
  \Au{Боровикова О.\,И., Загорулько Ю.\,А., Загорулько~Г.\,Б., Кононенко~И.\,С., 
Соколова~Е.\,Г.}
  Разработка портала знаний по компьютерной лингвистике~// КИИ-2008: Труды 11-й 
Национальной конф. по искусственному интеллекту с международным участием.~--- М.: 
ЛЕНАНД, 2008. Т.~3. С.~380--388.
  
 \end{thebibliography}
}
}


\end{multicols}