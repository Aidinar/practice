\def\stat{mor-nekr}

\textit{\hfill  Посвящается 70-летию со дня рождения В.\,В.~Калашникова
(1942--2001),}

\textit{\hfill внесшего большой вклад в развитие регенеративного
метода}

\def\tit{ОБ ОЦЕНИВАНИИ ВЕРОЯТНОСТИ ПЕРЕПОЛНЕНИЯ КОНЕЧНОГО БУФЕРА В~РЕГЕНЕРАТИВНЫХ СИСТЕМАХ
ОБСЛУЖИВАНИЯ$^*$}

\def\titkol{Об оценивании вероятности переполнения конечного буфера в регенеративных системах
обслуживания}

\def\autkol{Е.\,В.~Морозов,  Р.\,С.~Некрасова}
\def\aut{Е.\,В.~Морозов$^1$,  Р.\,С.~Некрасова$^2$}

\titel{\tit}{\aut}{\autkol}{\titkol}


{\renewcommand{\thefootnote}{\fnsymbol{footnote}}\footnotetext[1]
{Работа поддержана РФФИ (проект 10-07-00017). Работа выполнена
при поддержке Программы стратегического развития на 2012--2016~гг.\
<<Университетский комплекс ПетрГУ в научно-образовательном пространстве
Европейского Севера: стратегия инновационного развития>>.}}



\renewcommand{\thefootnote}{\arabic{footnote}}
\footnotetext[1]{Институт прикладных математических исследований КарНЦ 
РАН, Петрозаводский государственный университет,\linebreak emorozov@karelia.ru}
\footnotetext[2]{Институт прикладных математических исследований КарНЦ 
РАН, Петрозаводский государственный университет,\linebreak ruslana.nekrasova@mail.ru}

\vspace*{-12pt}


\Abst{Рассмотрены  вопросы оценивания
стационарной вероятности переполнения конечного буфера на основе
регенеративного моделирования. Приведен
 вывод общего соотношения, связывающего в стационарном режиме вероятность  потери с
вероятностью простоя обслуживающего канала.  Показано его применение для
широкого класса систем с потерями, а также  для системы с повторными вызовами и
постоянной скоростью возвращения заявок с орбиты на обслуживание. Исследована
эффективность этого соотношения при регенеративном оценивании вероятности
потери при различных режимах загрузки сис\-те\-мы, а также при использовании
$k$-ре\-ге\-не\-ра\-ций, возникающих при анализе немарковских сис\-тем с потерями.
Приведены результаты численного моделирования.}

\vspace*{-2pt}

\KW{системы с конечным буфером; вероятность потери;
вероятность простоя; регенеративный метод оценивания; $k$-ре\-ге\-не\-ра\-ции; сис\-те\-ма
с повторными вызовами}

\vspace*{-7pt}

\vskip 14pt plus 9pt minus 6pt

      \thispagestyle{headings}

      \begin{multicols}{2}

            \label{st\stat}


\section{Введение} 


Модели с ограничениями, в частности с конечным буфером, играют
важную роль в анализе современных     телекоммуникационных систем. 
В~таких системах поток, образованный потерянными заявками, часто
является входным потоком для другого узла коммуникационной системы,
а вероятность потери является ключевым показателем качества
обслуживания.


В данной работе  представлено  доказательство весьма общей формулы, связывающей
стационарную вероятность потери $\p_{\mathrm{loss}}$ со ста\-цио\-нарной\linebreak
вероятностью
простоя канала~$\p_0$.  Этот результат\linebreak
 позволяет сравнить эффективность оценки
$\p_{\mathrm{loss}}$, полученной с помощью оценки~$\p_0$, и  стандартной оценки, равной
доле потерянных заявок.  Формула через $\p_0$ применяется также
 к системе обслуживания с повторными вызовами и
с  постоянной скоростью возвращения заявок с орбиты~\cite{Avr}.   \mbox{В~статье}
также приведено условие стационарности такой модели в случае
двух серверов. Исследования по регенеративному оцениванию
вероятности блокировки в такой сис\-те\-ме  с повторными вызовами  (как  модели телефонных
протоколов  множественного доступа) были начаты   в работе~\cite{Minsk}.
 В  данной работе оценивание  также  опирается на регенеративный  метод с
использованием так называемых {\it  $k$-ре\-ге\-не\-ра\-ций} (возникающих в сис\-те\-мах с
входным пуассоновским потоком и/или экспоненциальным временем обслуживания),
когда в моменты прихода (или ухода) заявок в сис\-те\-ме находится $k$ других
заявок. В~статье исследуется эффективность оценивания при использовании
разных типов $k$-ре\-ге\-не\-ра\-ций (в том числе классической $0$-ре\-ге\-не\-ра\-ции). Как
известно,  доверительные интервалы, использующие разные последовательности
точек регенерации, асимптотически (с ростом числа наблюдений) эквивалентны~\cite{Shedler}.  
Однако выбор по\-сле\-до\-ва\-тель\-ности может существенно повлиять на
{\it скорость получения оценки}, и этот вопрос также рассматривается в статье.

Статья организована следующим образом. В~разд.~2  описана регенеративная
структура процессов в сис\-те\-мах с потерями и входным процессом восстановления.
Рассмотрены как классические, так и $k$-ре\-ге\-не\-ра\-ции, возникающие в немарковских
сис\-те\-мах $M/G/1/n$ и  $GI/M/m/n$. Приведена процедура построения
доверительного интервала при регенеративном оценивании. Также в разд.~2
затрагивается вопрос асимптотической эквивалентности длин доверительных
интервалов, использующих различные последовательности  регенераций. В~разд.~3
 выведено основное соотношение между  вероятностями   $\p_{\mathrm{loss}}$ и~$\p_0$.
 В~разд.~4 рассматривается двухсерверная система с конечным буфером,
 повторными вызовами и постоянной скоростью возвращения заявок с орбиты на обслуживание.
Для такой системы в явном виде получено условие стационарности, включающее
вероятность~$\p_{\mathrm{loss}}$ в некоторой тесно связанной с ней системе с потерями. 
В~заключительном разделе~5 приведены некоторые результаты численного оценивания
ве\-ро\-ят\-ности потери $\p_{\mathrm{loss}}$ с использованием $k$-ре\-ге\-не\-ра\-ций, а также
формулы, связывающей $\p_{\mathrm{loss}}$ и~$\p_0$.

\section{Регенеративная структура систем с~потерями}

При отсутствии точной формулы для вычисления вероятности $\p_{\mathrm{loss}}$
возникает необходимость в ее надежном  оценивании. В~данной работе с
этой целью  в сис\-те\-ме вида $GI/G/m/n$ применяется регенеративный
метод. Обозначим через $\{t_n\}$ моменты прихода заявок в сис\-те\-му, и
пусть $\{\tau_i:=t_{i+1}\hm-t_i\}$~--- независимые одинаково
распределенные (н.\,о.\,р.)\ интервалы входного потока, а $\{S_i\}$~---
н.\,о.\,р.\ интервалы обслуживания (с типичными элементами $\tau,S$
соответственно). Пусть $\nu_n$~--- число заявок в сис\-те\-ме в момент
прихода заявки $n\hm\ge 1$. Пусть $\{\nu(t)t\hm\ge 0\}$ есть
(непрерывный справа) процесс числа заявок в сис\-те\-ме, т.\,е.\
$\nu_n\hm=\nu(t_n^-)$. Тогда классические регенерации процесса $\{\nu(t)\}$ 
(и других процессов в непрерывном времени в данной сис\-те\-ме),
по\-рож\-да\-емые приходом заявок в пустую сис\-те\-му, рекурсивно
определяются хорошо известным образом: 
\begin{equation}
T_{n+1}=\min\limits_k\{t_k>T_n: \nu_k=0\}\,, \ n \ge 0\,,\ T_0:=0\,.
\label{e1-mn} 
\end{equation} 
Заметим, что регенерации~(\ref{e1-mn}) не являются событиями в потоке отказов и 
поэтому являются \emph{скрытыми} по отношению к этому потоку. Регенерирующий процесс
$\{\nu(t)\}$ называется {\it положительно возвратным}, если $\e T\hm<\infty$, где через~$T$ 
обозначена типичная длина цикла регенерации~\cite{Wolff, Morozov2004}.
 Положительная возвратность является необходимым условием
применимости регенеративного метода оценивания~\cite{Asmus}. Отметим, что более
общая конструкция так называемой однозависимой регенерации возникает в цепях
Маркова, возвратных по Харрису. Анализ такой регенерации, в том числе в
сис\-те\-мах с потерями и в  сетях обслуживания, содержится, например, в~\cite{Sig2, Sig3}.

Для  более специальных систем существуют альтернативные (классические)
регенерации. Со\-хра\-няя введенные обозначения, в сис\-те\-ме $GI/M/m/n$ зафиксируем
любое целое $k\hm\in [0,\, m+n]$ и определим рекурсивно моменты  {\it $k$-ре\-ге\-не\-ра\-ции} 
в {\it дискретном времени}  как номера тех заявок, которые
находят  в системе $k$ других заявок, т.\,е.\ 
 \begin{equation}
\beta^{(k)}_{n+1}=\inf_l\{l>\beta^{(k)}_{n}: \nu_l=k\}\,, \enskip
\beta^{(k)}_0:=0\,.
\label{e2-mn}
\end{equation}
Моменты  (\ref{e2-mn}) связаны с~(\ref{e1-mn}), как
$T_n\hm=t_{\beta^{(0)}_n}$, $n\hm\ge0$. Процесс $\{\nu_n\}$ образует апериодическую,
неприводимую цепь Маркова с  состояниями $\{0,\ldots,n+m\}$, когда на 
событии $\{\nu_l\hm=k \}$ в момент прихода   заявки
времена обслуживания  в каналах  разыгрываются заново. 

Заметим, что
{\it мера регенерации}~--- распределение процесса в момент
регенерации~--- является невырожденной при  $k\hm>0$. Для пояснения
рассмотрим событие $\{\beta^{(k)}_{n}\hm=r\}$, на котором заявка~$r$
встречает $k$ других заявок, т.\,е.\ $\nu_r\hm=k$ (и это $n$-я подобная
заявка среди первых $r$ заявок). Рассмотрим $(m+1)$-мер\-ный процесс
$W(l):=(Q_l,\, W_l^1, \ldots, W_l^m)$, $l\hm\ge1$, где $Q_l$~--- число
заявок  в буфере, а $W_l^i$~--- $i$-е в порядке возрастания
остаточное время обслуживания заявки  в момент~$t_l$ прихода заявки~$l$
($i=1,\dots,m$). Отметим очевидное соотношение
$\nu_l\hm=Q_l\hm+\sum\limits_{i=1}^mI(W_l^i>0)$, где $I$~--- индикатор. На событии
$\{\beta^{(k)}_{n}\hm=r\}$, если $k\hm< m$, то
$W(r)\hm{=}_{\!\mathrm{st}}\;(0,\,0,\ldots,\phi_{k+1}, \ldots, \phi_m)$, а если $k\hm\ge m$, 
то $W(r)\hm{=}_{\!\mathrm{st}}\;(k-m,\,\phi_1, \ldots, \phi_m$), где $\{\phi_i\}$
есть н.\,о.\,р.\   случайные величины (с.\,в.), а знак ${=}_{\mathrm{st}}$ означает 
стохастическое равенство. (С.\,в.~$\{\phi_i\}$  имеют показательное распределение    времени
обслуживания.) Поэтому распределение $W(r)$ зависит лишь от типа регенерации~$k$, но не от момента~$r$.

Аналогично, если $\nu_n^*$ есть число заявок, которое оставляет $n$-я
(уходящая)  заявка в системе  $M/G/1/n$,  то для фиксированного $k\hm\in[0,\,n]$
 \begin{equation*}
\alpha^{(k)}_{n+1}=\inf\limits_l\{l>\alpha^{(k)}_{n}: \nu_l^*=k\}\,,
\enskip\alpha^{(k)}_{0}:=0
 \end{equation*}
 есть  моменты $k$-регенерации для вложенной  цепи Маркова $\{\nu_n^*\}$, 
 мера регенерации которой   является невырожденной при $k\hm>0$.
 В~сис\-те\-мах с конечным буфером   незавершенная работа  (стохастически) ограничена.
Кроме того, в системе $GI/M/m/n$  длина $k$-цик\-ла непериодическая, поскольку
\begin{multline*}
\p(A_k=1)=\p(\nu^*_{l+1}=k|\nu^*_l=k)={}\\
{}=\p(\tau_1+\tau_2>S>\tau_1)>0\,.
\end{multline*}
Аналогично непериодичность длины $k$-цик\-ла в сис\-те\-ме $M/G/1/n$
следует из условия $\p(S_1+S_2>\tau\hm>S_1)\hm>0$.  Это обеспечивает положительную
 возвратность  $k$-ре\-ге\-не\-ра\-ций~\cite{MorozovDelgado}.
 
 \pagebreak

Дадим теперь строгое определение веро\-ят\-ности $\p_{\mathrm{loss}}$.
 Пусть $R(t)$~--- число потерянных заявок, а $A(t)$~--- число приходов в систему в
интервале $[0,\,t]$.  Поскольку поведение системы на цикле $k$-ре\-ге\-не\-ра\-ции (далее~--- 
{\it $k$-цикл}), зависит от~$k$, то  будем  обозначать  через $A_k$, $R_k$, $T_k$ число
приходов, число потерь на $k$-цик\-ле и  длину $k$-цик\-ла в непрерывном времени соответственно.
  Из теории регенерации следует, что в системе $GI/G/m/n$ с условием
$\p(\tau>S)\hm>0$  существуют и равны следующие   пределы:
\begin{equation}
 \lim\limits_{t \to \infty}\fr{R(t)}{A(t)}= \fr{\e R_k}{\e
A_k}:=\p_{\mathrm{loss}}=\lim\limits_{n\to\infty}\p(I_n=1)\,,
\label{e4b-mn}
\end{equation}
где $I_n=1$, если $n$-я  заявка получает отказ ($I_n\hm=0$, иначе).
Заметим, что условие $\p(\tau>S)>0$  влечет непериодичность
дискретной длины цикла занятости $A_0$ и существование слабого предела $\lim \p(I_n=1)$.


Обозначим $ Z_k:= R_k-\p_{\mathrm{loss}}A_k$.
      Регенеративный метод позволяет построить  доверительный интервал для
 стационарной   вероятности потери    $\p_{\mathrm{loss}}$,
если дисперсия $\D Z_k:=\D (R_k-\p_{\mathrm{loss}}A_k)\hm<\infty$.

Рассмотрим систему $M/G/1/n$ в предположении $\e S^2\hm<\infty$, что
влечет   $\D T_0\hm<\infty$~\cite{Wolff}. Пусть $I_k$ есть индикатор
события $\{\tau_1+\cdots+\tau_{k+1}>S_1\hm\ge \tau_1+\cdots+\tau_k\}$,
вероятность которого равна 
$$ 
\p_k =\int\limits_{0^-}^\infty e^{-\lambda x}\fr{(\lambda x)^k}{k!} \,dB(x)>0\,, 
$$ 
где $B$~--- распределение времени обслуживания. Заметим, что при $I_k\hm=1$  за время
обслуживания одной заявки, начинающей 0-цикл, в системе появится
ровно $k$  других заявок, т.\,е.\ начнется $k$-цикл. Следовательно,  
$T_0\hm\ge I_k T_k$ и  неравенство $ \e T_k^2\hm\le \e T_0^2/\p_k\hm<\infty $
влечет $\D T_k\hm<\infty.$ Поскольку
$T_k\;{=}_{\!\mathrm{st}}$\linebreak ${=}_{\!\mathrm{st}}\;\tau_1+\cdots+\tau_{A_k}$, то 
\begin{equation*} 
\D T_k=\e A_k \D \tau + \e \tau^2\D A_k \ge \e \tau^2\D
A_k\,,\enskip k=1,\ldots,n\,. 
\end{equation*} 
Это дает  $\D A_k\hm<\infty$, а
также (поскольку $R_k\hm\le{\!_{\mathrm{st}}} A_k$) и $cov(R_k,\,A_k)\hm\le \e A_k^2$.
Поэтому условие $\e S^2\hm<\infty$ влечет $\D Z_k\hm<\infty$ и позволяет
применить в сис\-те\-ме $M/G/1/n$ оценивание вероятности $\p_{\mathrm{loss}}$ на
основе $k$-ре\-ге\-не\-раций.

Предположим, что в рассматриваемой сис\-те\-ме $\D Z_k\hm\in(0,\,\infty)$, и обозначим
через~$p_k$ чис\-ло  $k$-цик\-лов, полученных в процессе моделирования сис\-те\-мы.
Тогда стандартным образом можно получить $(1-\gamma)\%$ доверительный интервал
для $\p_{\mathrm{loss}}$ в следующей форме: 
\begin{equation}
\hspace*{-1pt}\left[\p_{\mathrm{loss}}(p_k)-\fr{z_{\gamma}\sigma(p_k)}{\hat{A}(p_k)\sqrt{p_k}},
\,\p_{\mathrm{loss}}(p_k)+\fr{z_{\gamma}\sigma(p_k)}{\hat{A}(p_k)\sqrt{p_k}}\right],\!\!
\label{e9-mn}
\end{equation}
 где $\p_{\mathrm{loss}}(p_k)$~--- оценка $\p_{\mathrm{loss}}$, $\hat{A}(p_k)$~---
выборочная длина $k$-цик\-ла, $\sigma^2(p_k)$~--- выборочная оценка дисперсии
$\D Z_k$, а квантиль $z_{\gamma}$ находится из условия $z_{\gamma}\hm=\phi ^{-1}((1-\gamma)/2)$ 
($\phi(x)$~--- функция Лапласа). При этом $\hat{A}(p_k)\hm\to \e A_k,\,\sigma^2(p_k)\to \D Z_k$ 
при $p_k\to \infty$ c в.~1.

Рассмотрим подробнее  построение доверительного интервала по
случайному числу $p_k(t)$ $k$-цик\-лов, завершенных на периоде
моделирования $[0,t]$. Введем последовательность н.\,о.\,р.\ с.\,в.\
$\{Z_k^{(i)}\hm=R_k^{(i)}\hm-\p_{\mathrm{loss}}A_k^{(i)}$, $i\hm\ge1 \}$ с типичным
элементом $Z_k\hm=R_k\hm-\p_{\mathrm{loss}}A_k$ и  заметим, что
\begin{equation} 
R(t)-\p_{\mathrm{loss}}A(t)=\sum\limits_{i=1}^{p_k(t)}Z_k^{(i)}+o(t)\,,\enskip t\to
\infty\,,
\label{e6-mn}
\end{equation} 
где величина  $o(t)$  описывает число потерь на
оставшейся части цикла, {\it накрывающего}  момент~$t$~\cite{Asmus, Smith}.
 С~учетом $\e Z_k=0$ получаем  такую центральную предельную теорему~\cite{Asmus}:
\begin{equation}
\fr{R(t)-\p_{\mathrm{loss}} A(t)}{\sqrt{A(t)}}\Rightarrow N
\left(0, \fr{\D Z_k}{\e A_k}\right)\,,\enskip
 t \to \infty\,, 
 \label{e7-mn}
\end{equation}
где $N$ обозначает нормальную с.\,в.
  Таким образом,   отношение $\D Z_k/\e A_k$ является константой, не зависящей от~$k$ 
  (см.\ также~\cite{Shedler}). Следующий результат также известен~\cite{Shedler},
однако представляется  полезным  дать краткое пояснение к его
выводу. Пусть $p_i(t),\,p_j(t)$~--- число $i$-цик\-лов и $j$-цик\-лов
соответственно, полученных в интервале $[0,\,t]$.  Отметим, что в
рассматриваемом случае доверительный интервал строится как и~(\ref{e9-mn}),
 но с использованием случайного (а не детерминированного) чис\-ла  циклов.
 Обозначим через
$|I_i(t)|$ длину доверительного интервала, построенного по $i$-цик\-лам   и
пусть $\hat{A}_i(t)$~--- выборочная длина $i$-цик\-ла. Обозначая $\D Z_i\hm=\sigma_i^2$, получаем
\begin{multline}
\fr{|I_i(t)|}{|I_j(t)|}=
\fr{ \sigma_i\,\,\;\;\hat{A}_j(t)\sqrt{p_j(t)}}{\sigma_j\;\;\hat{A}_i(t)\,\,\sqrt{p_i(t)}}
={}\\
{}=\sqrt{\fr{\sigma^2_i\,\;\;\hat{A}_j(t)}{\hat{A}_i(t)\;
\sigma^2_j}}\,\,\sqrt{ \fr{\hat{A}_j(t)\;p_j(t)} {\hat{A}_i(t)\;p_i(t)}} \to
1 \mbox{ при } t \to \infty\,, 
\label{ints}
\end{multline} 
где использовано постоянство отношения $\D Z_k/\e A_k$ и  усиленный закон больших чисел
\begin{equation}
\left.
\begin{array}{c}
\hat{A}_i(t)\to \e A_i\,,\quad   \hat{A}_j(t)\to \e A_j\,;\\[9pt]
 \hat{A}_j(t)p_j(t)\sim \hat{A}_i(t)p_i(t)\sim t\,,\quad t\to \infty\,,
 \end{array}
 \right\}
\label{eqints}
\end{equation} 
($a\sim b$ означает $a/b\hm\to 1$). Таким образом, интервалы, построенные по
различным последовательностям $k$-ре\-ге\-не\-ра\-ций, асимптотически эквивалентны.
 Однако существенное различие в числе цик\-лов,
полученных при ограниченном времени моделирования, может  дать преимущество
одной последовательности перед другими.  Этот вопрос рассматривается в секции~5. 
Подробное описание построения доверительных интервалов при регенеративном
оценивании можно   найти в~[3, 11--14]. % \cite{Shedler, Crane}.
 (Доказательство сходимости~(\ref{e7-mn}) для случайных сумм вида~(\ref{e6-mn}) содержится также  
 в~\cite{Asmus, Billingsley}.)

\section{Соотношение между вероятностью потери и~вероятностью простоя}

В данном разделе для широкого класса регенеративных систем с потерями
доказано соотношение, связывающее вероятность $\p_{\mathrm{loss}}$ со стационарной
вероятностью простоя канала~$\p_0$. При этом используются лишь  0-ре\-ге\-не\-ра\-ции.
 Сохраняя  прежние обозначения,  рассмотрим сис\-те\-му $GI/G/m/n$ с коэффициентом  загрузки
$\rho\hm=\e S/E\tau$. Заметим, что в формулировке теоремы~1 предполагается, что
выбор свободного канала для новой заявки происходит равновероятно, если таких
каналов несколько.

\medskip

\noindent
\textbf{Теорема 1.}  {\it В системе $GI/G/m/n$  при условии
$\p(\tau>S)\hm>0$ стационарная вероятность потери $\p_{\mathrm{loss}}$
определяется формулой~(\ref{e4b-mn}) и
  связана со стационарной  вероятностью простоя (любого) канала~$\p_0$
следующим образом:} 
\begin{equation} 
%\label{viaPb}
\p_{\mathrm{loss}}=1-\fr{m}{\rho}(1-\p_0)\,.
 \label{e4-mn}
\end{equation}

\smallskip

\noindent
Д\,о\,к\,а\,з\,а\,т\,е\,л\,ь\,с\,т\,в\,о\,.\
Определим процесс накопленной работы в сис\-те\-ме
$W(t)\hm=\sum\limits_{i=1}^mW_i(t)$, где $W_i(t)$ есть не завершенная  в момент~$t$ 
работа, предназначенная для   канала~$i$. Поскольку буфер
конечен, то процесс $\{W(t)\}$  является стохастически ограниченным, 
а из условия $\p(\tau>S)\hm>0$ следует  положительная возвратность
 $\e T_0\hm<\infty$.  (Подробный анализ стационарности регенеративных сис\-тем, 
 в том числе с конечным буфером, содержится в~\cite{Morozov2004, MorozovDelgado}.) 
 Пусть  $V(t)$~--- суммарная нагрузка, поступившая в сис\-те\-му, 
 $B(t)$~--- обслуженная нагрузка, а $L(t)$~--- потерянная нагрузка (время, которое требовалось для обслуживания
потерянных заявок), все в интервале $[0,\,t]$. Получаем следующее уравнение баланса:
\begin{equation}
 V(t)=W(t)+B(t)+L(t)\,,
 \label{bal}
\end{equation} 
где, возможно, $V(0)=W(0)\not =0$. Заметим, что $B(t)\hm=\sum\limits_{i=1}^mB_i(t)$, а 

\noindent
$$
B_i(t)=\int\limits_0^tI(W_i(u)>0)\,du 
$$ есть время занятости канала~$i$ в интервале $[0,\,t]$, $i=1,\ldots,m$. Очевидно,  что
\begin{equation}
V(t)=\sum_{k=1}^{A(t)}S_k,\;\; L(t)=\sum\limits_{k=1}^{R(t)}S_k\,.
\label{e11-mn}
\end{equation}
Из  усиленного закона больших чисел при $t\hm\to\infty$  легко следует
\begin{equation}
\fr{V(t)}{t}\to \rho\,,\enskip \fr{L(t)}{V(t)} \to
\fr{\e R}{\e A}= \p_{\mathrm{loss}}\,.
\label{e14-mn}
 \end{equation} 
 Поскольку
$W(t)\le_{\!\mathrm{st}}\sum\limits_{i=1}^nS_i+\sum\limits_{i=1}^mS_i(t)$, 
где не завершенное в канале~$i$ в момент~$t$ время обслуживания 
$S_i(t)\hm=o(t)$~\cite{Smith}, то $W(t)\hm=o(t)$, $t\hm\to \infty$. (Последний результат
можно также вывести непосредственно из положительной возвратности
$0$-ре\-ге\-не\-ра\-ций.) Так как каналы идентичны, то получаем также
\begin{equation}
\fr{B(t)}{t}=\fr{\sum\limits_{i=1}^mB_i(t)}{t}\to m(1-\p_0)\,,\enskip t\to \infty\,,
\label{e16-mn}
 \end{equation}
  что  вместе с~(\ref{e14-mn}) дает
 \begin{equation*}
\fr{B(t)}{V(t)}\to \fr{m(1-\p_0)}{\rho}\,.
\end{equation*} 
Таким образом, предел $(t-B_i(t))/t\hm\to \p_0$ является стационарной вероятностью простоя любого
канала. Поделив обе части~(\ref{bal}) на $V(t)$ и перейдя к пределу при $t\hm\to\infty$, 
получаем~(\ref{e4-mn}). Наконец, как было отмечено (ввиду  условия
$\p(\tau>S)\hm>0$),  $\p_{\mathrm{loss}}$ также  является стационарной вероятностью потери приходящей 
заявки.~\hfill$\square$

\smallskip

В случае одного сервера  статистический аналог
формулы~(\ref{e4-mn}) используется   в работах~\cite{Whitt91, Whitt90} при
построении так называемой непрямой оценки вероятности $\p_{\mathrm{loss}}$ (см.\ также~\cite{GM}). 
Однако в случае нескольких серверов представление
 суммарного времени занятости в виде $B(t)\hm=\sum\limits_{i=1}^mB_i(t)$ там не используется.
 Формула~(9) для системы с потерями вида $M/G/1/n$ так\-же получена другим методом 
 в~[19, с.~333].

 Как показано  в разд.~5,  формула~(\ref{e4-mn})  может быть полезна для  оценки
 вероятности $\p_{\mathrm{loss}}$ через оценку вероятности простоя (или занятости)
 канала в случае большой нагрузки, когда прямая оценка $\p_{\mathrm{loss}}$ с помощью метода Мон\-те-Кар\-ло
оказывается неэффективной.

В заключение этого раздела отметим, что  соотношение~(\ref{e4-mn}) можно
непосредственно применить к системам с буфером {\it случайного размера},
который  регенерирует на периодах занятости сис\-те\-мы. Такие сис\-те\-мы могут
представлять интерес для моделирования  узлов связи в некоторых современных
коммуникационных сетях. Кроме того, соотношение~(\ref{e4-mn}) верно  для предложенной 
в~\cite{Tih} системы со {\it случайным объемом поступающих заявок} и ограничением
на  суммарный объем. В~такой  системе  $\p_{\mathrm{loss}}$   равна предельной доле
потерянного объема. Также соотношение~(\ref{e4-mn}) верно для систем с ограниченным
ожиданием/пребыванием заявки.

Нетрудно видеть, что  для {\it жидкостной системы} со скоростью обслуживания~$C$  
соотношение~(\ref{e4-mn}) принимает вид 
\begin{equation} 
\p_{\mathrm{loss}}=1-\fr{Cm (1-\p_0)}{\rho}\,.
\label{e22-mn} 
\end{equation} 
Заметим, что в классических моделях $C\hm=1$, а  в жидкостных моделях поступающая 
нагрузка~$V(t)$ не разделяется на отдельные заявки, а следует, например,  
процессу Леви (с независимыми стационарными приращениями) с {\it заданной интенсивностью}~$\rho$. 
В~этом случае величина $\rho/C$  является {\it коэффициентом загрузки} и, таким
образом, формулы~(\ref{e22-mn}) и~(\ref{e4-mn}) эквивалентны. 

Отметим, что если рассматривать $\p_{\mathrm{loss}},\,\p_0$ лишь как пределы в среднем,
то соотношение~(\ref{e4-mn}) имеет место   для гораздо более широкого класса
сис\-тем, чем регенеративные.

\section{Система с~повторными вызовами и~постоянной скоростью возвращения
заявок~с~орбиты}

В данном разделе рассматривается  система с повторными вызовами типа $M/G/2/0$
без буфера  с входным (пуассоновским) потоком первичных заявок с интенсивностью~$\lambda$
 и произвольным  временем  обслуживания  с интенсивностью~ $\mu$.  
 Эта сис\-те\-ма, далее обозначенная через~$\Sigma_O$,  успешно
применена для моделировании занятости телефонных линий в мобильных сис\-те\-мах~\cite{F86}, 
протоколов множественного доступа ALOHA~\cite{C93}, а также
 протокола  TCP с короткими сообщениями~\cite{AY08}. Когда серверы заняты, первичные заявки
 уходят  на орбиту  бесконечного объема, а затем вновь пытаются попасть  на серверы.
  Поток повторных попыток (при непустой орбите) является пуассоновским с  па\-ра\-мет\-ром~$\mu_0$ 
  и  {\it не зависит от величины орбиты}, в отличие от классических  сис\-тем с повторными вызовами.
Как показано в работах~\cite{Avr, GM, MorNek},  для анализа стационарности сис\-те\-мы~$\Sigma_O$
можно использовать  следующую сис\-те\-му с потерями~$\Sigma_L$ (без орбиты).
 Сис\-те\-ма~$\Sigma_L$ имеет  тот же входной   поток первичных заявок, то же распределение
  времени обслуживания, что и сис\-те\-ма~$\Sigma_O$, но
  имеет еще  один (независимый) пуассоновский входной поток заявок с параметром~$\mu_0$.
 Если в момент прихода  сис\-те\-ма~$\Sigma_L$ занята, то заявка   теряется.
 Обозначим через $\p_{\mathrm{loss}}$ стационарную вероятность потери в сис\-те\-ме~$ \Sigma_L $. 
 В~работе~\cite {Avr} доказано,  что условие 
 \begin{equation}
(\lambda+\mu_0)\p_{\mathrm{loss}}<\mu_0 
\label{st_cond}
\end{equation}
является достаточным (а в марковском случае также и необходимым) условием стационарности
сис\-те\-мы~$\Sigma_O$. Очевидно, что вероятность $\p_{\mathrm{loss}}$ удовлетворяет
основному соотношению~(\ref{e4-mn}). Сис\-те\-ма с потерями~$ \Sigma_L $, однако, важна
не только для определения зоны стационарности~$\Sigma_O$. Как показано в~\cite{Minsk, GM}, 
в {\it зоне нестационарности} системы $\Sigma_O$ оценка
вероятности $\p_{\mathrm{loss}}$ сходится c в.~1 к вероятности блокировки вызова
$\p_{\mathrm{orb}}$ в сис\-те\-ме~$\Sigma_O$. Более того, в работе~\cite{MorNek} доказано,
что в односерверной сис\-те\-ме~$\Sigma_O$ в нестационарном режиме 
\begin{equation}
\label{ret-loss} 
\p_{\mathrm{orb}}=1-\fr{P_b}{\tilde{\rho}}\,, 
\end{equation} 
где $\tilde{\rho}:=(\lambda\hm+ \mu_0)/\mu $, а $\p_b$ есть стационарная веро\-ятность
занятости системы~$\Sigma_L $. Очевидно, (\ref{ret-loss}) является  аналогом~(\ref{e4-mn}), 
поскольку $\p_{\mathrm{orb}}\hm=\p_{\mathrm{loss}}$. Утверждение~(\ref{ret-loss})  поз\-во\-ля\-ет
провести анализ стационарности в {\it частично нестационарной} сис\-те\-ме и
достоверно оценить вероятность блокировки при неограниченно
растущей орбите. Такая тесная связь двух систем мотивирует анализ условий
стационарности системы~$\Sigma_O$, представленный ниже.

 Напомним формулу Эрланга для вероятности потери  в
 системе $M/G/m/0$~ \cite{Gnedenko}
\begin{equation} 
\label{MMP_L}
 \p_{\mathrm{loss}}=\fr{\rho^{m}/m!}{\sum\limits_{k=0}^m \rho^k/k!}\,,\enskip\rho:=\lambda/\mu\,.
\end{equation}
 В работе~\cite{MorNek} для системы $\Sigma_O$ типа $M/G/1/0 $  использование~(\ref{MMP_L})
позволило получить условие~(\ref{st_cond}) при $\lambda\hm=1$ в  форме
$1/\mu_0\hm+1\hm<\mu$. Ниже приведено в явном виде условие стационарности
{\it двухсерверной} сис\-те\-мы~$\Sigma_O$. 

\medskip

\noindent
\textbf{Теорема 2.}
 {\it Cиcтема с повторными вызовами  типа $M/G/2/0$} (\textit{с
интенсивностью первичных заявок $\lambda\hm=1$}) \textit{стационарна, если $\mu>1/2$  и
если} 
\begin{equation*} 
%\label{cond2}
 \mu_0>\fr{\mu^2+\mu-1-\mu \sqrt{\mu^2+2\mu-1}}{1-2 \mu}:=\mu_{01}\,. 
\end{equation*} 

\smallskip

\noindent
Д\,о\,к\,а\,з\,а\,т\,е\,л\,ь\,с\,т\,в\,о\,.\ Используя
формулу~(\ref{MMP_L}) для системы Эрланга с коэффициентом загрузки
 $\rho\hm= (1\hm+\mu_0)/\mu$ и $m\hm=2$, получим условие стационарности~(\ref{st_cond}) в
 форме  неравенства 
\begin{equation} 
\label{ret2-1}
-\mu^2\mu_0-\mu(\mu_0+\mu_0^2)+\fr{1}{2}+\mu_0+\fr{\mu_0^2}{2}<0\,.
\end{equation}
 Разложение на множители при условии $\mu\hm\ge \sqrt{2}-1$ имеет вид
\begin{equation} 
\label{mnoj}
\left(\fr{1}{2}-\mu\right)\left(\mu_0-\mu_{01}\right)\left(\mu_0-\mu_{02}\right)<0\,. 
\end{equation}
При $\mu < \sqrt{2}-1$ неравенство~(\ref{ret2-1}) не выполняется и
сис\-те\-ма нестационарна. Таким образом, стационарность может иметь
место только при $\mu\hm\ge \sqrt{2}-1$, когда справедливо разложение в
левой части~(\ref{mnoj}). Анализ значения  параметров
$(\mu,\,\mu_0)$ в области $ (\sqrt{2}-1, \infty)\times (0, \infty)$,
при которых выполняется неравенство~(\ref{ret2-1}), легко приводит к
утверждению  теоремы.\hfill~$\square$

Отметим, что  условие~(\ref{st_cond}) совпадает с критерием
стационарности   системы с повторными вызовами вида $M/M/2/0$,
полученным в~\cite{A96} на основе традиционной для марковских моделей техники   (см.\ также~\cite {AGN01}).
 В~принятых обозначениях  (и  при $\lambda\hm=1$) оба эти условия могут
быть записаны, например,  в форме
\begin{equation*}
\left(1+\mu_0\right)^2<2 \mu \mu_0\left(1+\mu+\mu_0\right)\,.
\end{equation*}

Область значений параметров $\mu$ и~$\mu_0$,  в которой сис\-те\-ма с
повторными вызовами типа  $M/M/2/0$ обладает стационарностью, представлена на рис.~1.


Рассмотрим  более общую  $m$-сер\-вер\-ную сис\-те\-му с повторными вызовами
(с постоянной ско\-ростью возвращения с орбиты) с конечным буфером  и
с входным потоком восстановления с интенсив\-ностью~$\lambda$,
изученную в~\cite{Avr}.  Предполагается, что
  серверы стохастически
эквивалентны (и\linebreak $S$~--- ти\-пич\-ное  время обслуживания), так что, в
частности, вновь поступающие заявки распределяются  равновероятно по
свободным серверам (если их несколько).  Пусть $V(t)$~--- поступившая,
 а  $B(t)$~--- обслуженная нагрузка  в интервале $[0,\,t]$, причем
$B(t)\hm=\sum\limits_{i=1}^mB_i(t)$, где $B_i(t)$ есть время занятости сервера~$i$ 
в интервале $[0,\,t]$, $i=1,\ldots,m$. \linebreak\vspace*{-12pt}
%\noindent
\vspace*{2pt}
\begin{center}  %fig1
 \mbox{%
 \epsfxsize=77.686mm
 \epsfbox{mo1-1.eps}
 }
 \end{center}
% \vspace*{6pt}
{{\figurename~1}\ \ \small{Область  стационарности  системы с повторными
вызовами вида  $M/M/2/0$}}



%\pagebreak

\vspace*{12pt}

\addtocounter{figure}{1}


\begin{table*}[b]\small
\vspace*{-12pt}
\begin{center} 
\Caption{Системы $M/M/m/n$, $\lambda=4$
\label{tab:pars-mn}}
\vspace*{2ex}

\begin{tabular}{|c|c|c|c|c|c|c|c|c|c|} 
\hline
&&&&&&&&&\\[-9pt]
 $\rho$ & $m$   &$n$   &  $\p_{\mathrm{loss}}$ & $\hat{\p}_{\mathrm{loss}}$ &  $i_0$  &
 $\hat{\p}_l$  &  $i_0$& $VR$& $\varepsilon$ \\
 \hline
1& 2& 0& 0.200& 0.203& 1242& 0.228& 4439& 6.269& 0.05\\ 
2& 2& 0& 0.400& 0.419&1790& 0.391& 1791& 1.007& 0.05\\ 
4& 4& 4& 0.139 & 0.144& 3810& 0.041& 5914&11.318& 0.05\\
8& 2& 0& 0.781& 0.783& 2193& 0.779& 1332& 0.034& 0.05\\
4& 1& 3& 0.751& 0.734& 1840& 0.692& 8531& 0.059& 0.05\\
 \hline 
 \end{tabular} 
 \end{center} 
 \end{table*}

\noindent
Пусть    $W(t)$
 есть не завершенная  в момент~$t$ работа в системе (в буфере, в серверах  и на
 орбите). Для данной  сис\-те\-мы имеет место такой аналог уравнения баланса~(\ref{bal}), 
 в котором отсутствует слагаемое, связанное с потерями:
\begin{equation}
V(t)=W(t)+B(t)=W(t)+\sum\limits_{i=1}^mB_i(t),\enskip t\ge0\,.
\label{e26-mn}
\end{equation}
Предположим теперь, что система стационарна, т.\,е.\ вложенный процесс
регенераций положительно возвратен. Обозначим $\rho\hm=\lambda\e S$.
Тогда, в част\-ности, $W(t)\hm=o(t)$, $t\hm\to \infty$, и из~(\ref{e26-mn})
следует  предельное соотношение (см.~(\ref{e11-mn})--(\ref{e16-mn}))
\begin{equation}
\rho= m \p_b\,,
\label{e27-mn}
\end{equation}
где $\p_b$ есть стационарная вероятность занятости любого сервера. В~част\-ности, $\p_b\hm=\rho$ 
при $m\hm =1$. Более того, если входной процесс
пуассоновский, то (по свойству PASTA)  $\rho$ есть  также
стационарная вероятность  блокировки (ухода на орбиту) вновь
поступающей первичной заявки (при $m\hm=1$).  Интересно отметить, что
соотношение~(\ref{e27-mn}) верно также и для общей сис\-те\-мы $GI/G/m$  (с
бесконечным буфером и без орбиты). (Здесь важно лишь отсутствие
потерь.) Однако критерий  стационарности $\rho/m\hm<1$ этой сис\-те\-мы,
разумеется, отличается от (достаточного) условия стационарности~(\ref{st_cond}) сис\-те\-мы 
с повторными вызовами.  Например, для
системы с повторными вызовами вида $M/G/1/K$ условие  стационарности
имеет вид
\begin{equation}
\lambda \rho<\mu_0\left(1-\rho\right)\,. 
\label{e27c-mn}
\end{equation}
С учетом свойства PASTA~\cite{Asmus} условие~(\ref{e27c-mn}) имеет
ясную физическую интерпретацию: левая часть неравенства есть
интенсивность ухода заявок на орбиту, а правая часть есть
максимальная интенсивность (успешного) ухода заявок с орбиты на
сервер, поскольку множитель $1-\rho$ есть стационарная вероятность
незанятости сервера (когда  успешный уход с орбиты возможен).





%\vspace*{-12pt}

\section{Результаты численного  моделирования} 

В данном разделе представлены
некоторые чис\-лен\-ные результаты оценивания вероятности  $\p_{\mathrm{loss}}$ по методу
Мон\-те-Кар\-ло с использованием (в очевидных обозначениях) традиционной оценки
$\hat{\p}_{\mathrm{loss}}(i_0)\hm=\hat{R}(i_0)/\hat{A}(i_0)$ и оценки на основе формулы~(\ref{e4-mn})
 $$
  \hat{\p}_l:=\hat{\p}_l(i_0)=1-\fr{m}{\rho}\,\hat{\p}_b(i_0)
$$
 соответственно в зависимости от   чис\-ла 0-ре\-ге\-не\-ра\-ций~$i_0$, где
$\hat{P_b}(i_0)$~--- оценка стационарной вероятности занятости. (Обозначение 
$\hat{\p}_l(i_0)$ и $\hat{\p}_b(i_0)$ 
 подчеркивает, что используется {\it шкала циклов, а не отдельных
наблюдений}.) В~табл.~1 приведены результаты оценивания  для  сис\-те\-мы
$M/M/m/n$  в зависимости от коэффициента загрузки~$\rho$, а также чис\-ла
серверов~$m$ и величины буфера~$n$. Величина
$VR\hm=D[\hat{\p}_l(i_0)]/D[\hat{\p}_{\mathrm{loss}}(i_0)]$ равна отношению выборочных
дисперсий оценок. Наблюдения проводились до получения заданной точности
$\varepsilon\hm>0$. Значения оценок также сравниваются  с точным значением,
вычисленным по формуле~\cite{Gnedenko}:
\begin{equation*} 
%\label{MMP_L1}
 \p_{\mathrm{loss}}=\fr{\rho^{m}}{m!}\theta ^{\,n}\p_0\,,
\end{equation*} 
где $\theta=\rho/m$, а 
\begin{equation*} 
%\label{MMP_0}
\p_0=\left(\sum\limits_{k=0}^m\fr{\rho^k}{k!}+\fr{\rho^m}{m!}\sum\limits_{k=1}^n
\theta^{\,k}\right)^{-1}\,. 
\end{equation*} 
Критерием эффективности является необходимое число 0-цик\-лов для получения заданной точности,
а также  дисперсия  оценки.
Как видно из табл.~1,
в случае \textit{малой нагрузки} ($\rho/m\hm<1$) классическая оценка
$\hat{\p}_{\mathrm{loss}}$ более  эффективна, чем
 $\hat{\p}_l$,  как по скорости построения, так и по величине  дисперсии.
При \textit{средней нагрузке} ($\rho/m\hm=1$) и  буфере $n\hm=0$
эффективность оценок близка, однако с ростом величины   буфера
оценка $\hat{\p}_{\mathrm{loss}}$ становится предпочтительнее. При
\textit{большой нагрузке} ($\rho/m\hm=4$) и $n\hm=0$ оценка $\hat{\p}_l$
оказывается эффективней как по времени оценивания, так и по
величине дисперсии.  Это согласуется с результатами работ~\cite{Whitt90, MorNek}, 
где  показана эффективность {\it непрямой}
оценки $\hat{\p}_l$ при большой нагрузке (а также и ее отрицательная
корреляция с оценкой $\hat{\p}_{\mathrm{loss}}$). При увеличении величины
буфера оценка $\hat{\p}_{\mathrm{loss}}$ строится быстрее, однако имеет
б$\acute{\mbox{о}}$льшую дисперсию, чем $\hat{\p}_l$ и поэтому нельзя
сделать однозначный вывод об эффективности оценок.



Аналогичные численные результаты получены  для системы Pareto$/M/m/n$ (с
распределением Парето входного потока).



Исследовалась также эффективность применения различных $k$-ре\-ге\-не\-ра\-ций.
Результаты приведены на  рис.~2 и в табл.~2, где оценка $\hat{\p}_{\mathrm{loss}}(i)\hm=R(i)/A(i)$ 
строилась по {\it числу наблюдений}~$i$. Число
наблюдений, как правило, пропорционально времени моделирования и позволяет
оценить реальную\linebreak скорость получения оценки, в отличие от чис\-ла цик\-лов,
 различие в длинах которых может быть очень
значительным. На рис.~2 представлена зависимость от \textit{числа наблюдений }
границ доверительного интервала  для  вероятности потери в системе $M/M/2/4$,
где  рассматривались $k$-ре\-ге\-не\-ра\-ции при  $k\hm=0,3, 6$. 
С~ростом числа   наблюдений доверительные
интервалы для всех  типов регенераций сближаются между собой, что
соответствует~(\ref{eqints}). Однако при малых~ $k$ оценивание
происходит быстрее, поэтому (при малой нагрузке)  такие регенерации
(в том чис\-ле 0-ре\-ге\-не\-ра\-ции) более эффективны. 
Отметим, что скорость оценивания возрастает с ростом~$k$, однако при $k\hm\ge 7$ она растет
незначительно. При максимальном $k\hm=m\hm+n\hm=10$ оценка имеет наименьшую
дисперсию, и поэтому (при большой нагрузке) такие регенерации в
моменты потерь наиболее эффективны.  Причина состоит в том, что
классические регенерации в данном случае редки  и их использование  для оценки
вероятности $\p_{\mathrm{loss}}$ оказывается неэффективным.
В табл.~3.  пред\-став\-ле\-ны результаты
анализа системы Pareto$/M/4/6$ при $\rho/m\hm=4$ и нескольких типах
$k$-ре\-ге\-не\-раций.  %-\linebreak\vspace*{-12pt}

\pagebreak

\noindent
%\noindent
\vspace*{-9pt}  %fig2
 \begin{center}
 \mbox{%
 \epsfxsize=78.578mm
 \epsfbox{mo1-2.eps}
 }
 \end{center}
% \vspace*{6pt}
{{\figurename~2}\ \ \small{Доверительное оценивание  $\p_{\mathrm{loss}}$ на основе\protect\linebreak
$k$-ре\-ге\-нераций, $k=0$~(\textit{1}), 3~(\textit{2}), и~6~(\textit{3}),  
в сис\-те\-ме $M/M/2/4$ при $\rho/m=0.5$}}



%\pagebreak

\vspace*{6pt}

\addtocounter{figure}{1}

\noindent
\begin{center}  %tabl2
%\vspace*{-6pt}
{{\tablename~2}\ \ \small{Система $M/M/2/4$  при $\rho/m=0.5$}}
\vspace*{2ex}

{\small 
\tabcolsep=6.4pt
\begin{tabular}{|c|c|c|c|c|} 
\hline
&&&&\\[-9pt]
 $k$ & $i$   &$\hat{\p}_{\mathrm{loss}}(i)$ &  $\mathrm{Var}[\hat{\p}_{\mathrm{loss}}(i)]$ & $\varepsilon$ \\
 \hline
0& \hphantom{9,}78\hphantom{9}& 0,013& 0,001& 0,05\\ 
3& 291& 0,028& 0,005& 0,05\\ 
6& 1979\hphantom{9}& 0,005& 0,228&0,05\\
 \hline
\end{tabular} 
}
%\vspace*{-9pt}
\end{center}



%\pagebreak

%\vspace*{10pt}

\addtocounter{table}{1}


\noindent
\begin{center}  %tabl3
%\vspace*{-6pt}
{{\tablename~3}\ \ \small{Система
Pareto$/M/4/6$  при $\rho/m=4$}}
\vspace*{2ex}

{\small 
\tabcolsep=7pt
\begin{tabular}{|c|c|c|c|c|}
\hline
&&&&\\[-9pt]
 $k$ & $i$   &$\hat{\p}_{\mathrm{loss}}(i)$ &  $\mathrm{Var}[\hat{\p}_{\mathrm{loss}}(i)]$ & $\varepsilon$ \\
 \hline
4& 14489\hphantom{9}& 0,756& 0,131& 0,05\\ 
6& 2933& 0,738& 0,120& 0,05\\ 
7& 1927& 0,753& 0,135& 0,05\\
 \hline
\end{tabular} 
}
%\vspace*{-9pt}
\end{center}



%\pagebreak

%\vspace*{10pt}

\addtocounter{table}{1}


\section{Заключение} 

В статье рассмотрен ряд   вопросов
 регенеративного оценивания  стационарной вероятности потери в системах
обслуживания с  конечным буфером. Доказано
 общее соотношение,  связывающее в стационарном режиме вероятность потери с
ве\-ро\-ят\-ностью простоя обслуживающего канала  для широкого класса систем, где
потери могут быть вызваны  различными причинами.  Исследована эффективность этого
соотношения при регенеративном оценивании вероятности потери при использовании
$k$-регенераций, возникающих при анализе немарковских систем на основе
вложенных цепей Маркова. Получены в явном виде условия стационарности для
двухсерверной системы с повторными вызовами и постоянной скоростью возвращения
заявок с орбиты на обслуживание. Приведены некоторые  результаты численного
моделирования.

{\small\frenchspacing
{%\baselineskip=10.8pt
\addcontentsline{toc}{section}{Литература}
\begin{thebibliography}{99}

\bibitem{Avr} 
\Au{Avrachenkov~K., Morozov ~E.\,V.}  Stability  analysis of $GI/G/c/K$ retrial queue
with constant retrial rate. \mbox{INRIA} (Sophia Antipolis): Research
Report,  2010. No.\,7335.

\bibitem{Minsk}   %2
\Au{Avrachenkov~K., Goricheva~R.\,S., Morozov~E.\,V.}
Verification of stability region of a retrial queuing system by
    regenerative method~// Modern Probabilistic Methods for Analysis 
    and Optimization of Information and Telecommunication Networks: 
    Intenational Conference Proceedings.~--- Minsk, 2011. P.~22--28.

   
    \bibitem{Shedler} %3
\Au{Shedler~G.\,S.} Regeneration and networks of queues.~--- New-York: Springer-Verlag,  1987.

\bibitem{Wolff} %4
\Au{Wolff~R.\,W.} Stochastic modeling and the theory of
    queues.~--- Englewood Cliffs, NJ: Prentice Hall, 1989.
    
    \bibitem{Morozov2004}  %5
\Au{Morozov~E.\,V.} Weak regeneration in modeling of
    queueing processes~// Queueing Syst., 2004. Vol.~46.   P.~295--315.

\bibitem{Asmus} %6
\Au{Asmussen~S.} Applied probability and queues.~--- 2nd ed. -- New York: Springer-Verlag, 2003.

\bibitem{Sig2} %7
\Au{Sigman~K.} Queues as Harris recurrent Markov chains~//
Queueing Syst., 1988. No.\,3.  P.~179--198.

\bibitem{Sig3} %8 
\Au{Sigman~K.} One-dependent regenerative processes and
    queues in continuous time~// Math. Oper. Res.,  1990. No.\,15. P.~175--189.

\bibitem{MorozovDelgado}  %9
\Au{Морозов~Е., Делгадо~Р.} Анализ
    стационарности регенеративных систем обслуживания~//  Автоматика и
    телемеханика, 2009. No.\,12.  С.~42--58.

\bibitem{Smith}  %10
\Au{Smith~W.\,L.} Regenerative stochastic processes~//
    Proc. Roy. Soc.  Ser.~A, 1955. Vol.~232. P.~6--31.


\bibitem{Crane} %11
\Au{Крэйн~М., Лемуан~О.} Введение в регенеративный
    метод анализа моделей.~--- М.: Наука, 1982. 104~с.

\bibitem{Iglehart} %12
\Au{Иглехарт~Д.\,Л., Шедлер~Д.\,С.} Регенеративное
    моделирование сетей массового обслуживания.~--- М.: Радио и связь, 1984.
    136~с.
    
\bibitem{Glynn} %13
\Au{Glynn~P.\,W., Iglehart D.\,L.} Conditions for the
    applicability of the regenerative method~// Management Sci., 1993. Vol.~39. P.~1108--1111.

\bibitem{Glynn1} %14
\Au{Glynn~P.\,W.} Some topics in regenerative
    steady-state simulation~// Acta Appl. Math.,   1994. No.\,34. P.~225--236.

\bibitem{Billingsley}  %15
\Au{Биллингсли~П.} Сходимость вероятностных мер.~--- М.: Наука, 1977.  352~с.

\bibitem{Whitt91} %16
\Au{Whitt~W.} A~review of $L=\lambda W$ and extensions~//
    Queueing Syst.,  1991. Vol.~9. P.~235--268.
    
    \bibitem{Whitt90}  %17
\Au{Srikant~R., Whitt~W.} Variance reduction in
    simulations of loss models~// Oper. Res., 1999. Vol.~47. No.\,4.     P.~509--523.
    
    \bibitem{GM} %18
\Au{Горичева~Р.\,С., Морозов~Е.\,В.} Регенеративное
    моделирование вероятности потери в системах обслуживания с конечным
    буфером~// Труды Карельского научного центра РАН, 2010. №\,3. С.~20--29.
    
    \bibitem{19-nn}
    \Au{Бочаров П.\,П., Печинкин А.\,В.}
    Теория массового обслуживания.~--- М.: РУДН, 1955. 529~с.

\bibitem{Tih} %19+i
\Au{Тихоненко~О.\,М.} Модели массового обслуживания в
    системах обработки информации.~--- Минск: Университетское, 1990. 191~с.

\bibitem{F86} %20
\Au{Fayolle~G.} A~simple telephone exchange with delayed
    feedback~// Teletraffic Anal. Comp. Performance Evaluation,  1986. Vol.~7. P.~245--253.

\bibitem{C93} %21
\Au{Choi~B.\,D.,  Rhee~K.\,H., Park~K.\,K.} The $M/G/1$
    retrial queue with retrial rate control policy~// Prob. 
    Engng. Informational Sci.,  1993.  Vol.~7. P.~29--46.

\bibitem{AY08} %22
\Au{Avrachenkov~K., Yechiali~U.} Retrial networks with
    finite buffers and their application to Internet data traffic~// Prob.
Engng. Informational Sci., 2008. Vol.~22. P.~519--536.

\bibitem{MorNek}  %23
\Au{Морозов~Е.\,В., Некрасова~Р.\,С.} Оценивание
    вероят\-ности блокировки в системе с повторными вызовами и постоянной
    скоростью возвращения заявок с орбиты~// Труды Карельского научного центра
    РАН, 2011.  №\,5. С.~63--74.
    
    \bibitem{Gnedenko} %24
\Au{Гнеденко~Б.\,В., Коваленко~И.\,Н.} Введение  в
    теорию массового обслуживания.~--- М.: Наука, 1987. 336~с.

\bibitem{A96} %25
\Au{Artalejo~J.\,R.} Stationary analysis of the characteristics
of the $M/M/2$ queue with constant repeated attempts~// Opsearch,
1996. Vol.~33. P.~83--95.

\label{end\stat}
    
    \bibitem{AGN01} %26
\Au{Artalejo~J.\,R.,  G$\acute{\mbox{o}}$mez-Corral~A.,   Neuts~M.\,F.}
Analysis of multiserver queues with constant retrial rate~//
Eur. J.~Oper. Res., 2001. Vol.~135. P.~569--581.
 \end{thebibliography}
}
}


\end{multicols}