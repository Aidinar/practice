\def\stat{abstr}
{%\hrule\par
%\vskip 7pt % 7pt
\raggedleft\Large \bf%\baselineskip=3.2ex
A\,B\,S\,T\,R\,A\,C\,T\,S \vskip 17pt
    \hrule
    \par
\vskip 21pt plus 6pt minus 3pt }

\def\tit{DEVELOPMENT OF PUGACHEV FILTERING FOR
STOCHASTIC SYSTEMS}


\def\aut{I.\,N.~Sinitsyn}
\def\auf{IPI RAN, sinitsin@dol.ru}

\def\leftkol{\ } % ENGLISH ABSTRACTS}

\def\rightkol{\ } %ENGLISH ABSTRACTS}

\titele{\tit}{\aut}{\auf}{\leftkol}{\rightkol}


\noindent Statistical methods of information processing (filtering,
extrapolation, interpolation etc) in stochastic systems are the basis of modern
statistical informatics. Analytical survey and main tendencies of nonlinear
conditionally optimal Pugachev filters for stochastic systems are considered.
Pugachev filters for regular and nonregular, continuous and discrete, Gaussian
and non-Gaussian stochastic systems are discussed. Interconnection between
Pugachev and Kalman filters including various statistical criteria is given.
Problems of joint filtering and recognition and Pugachev filters trends are
considered. \label{st\stat}

 \KWN{stochastic system, conditionally optimal filtering, extrapolation, interpolation, Pugachev filter,
 online information processing, autocorrelated noise}

% \thispagestyle{headings}

%\vskip 18pt plus 6pt minus 3pt

\vfill

\def\tit{MEANS PROVIDING APPLICATIONS FAULT TOLERANCE}
\def\aut{V.~Zakharov$^1$ and V.~Kozmidiady$^2$}

\def\auf{$^1$IPI RAN, vzakharov@ipiran.ru\\[1pt]
$^2$IPI RAN, kozmidiady\_v@tochka.ru}


%\def\leftkol{ENGLISH ABSTRACTS}

%\def\rightkol{ENGLISH ABSTRACTS}

\titele{\tit}{\aut}{\auf}{\leftkol}{\rightkol}

\noindent The problems of fault tolerant servers creation, caused by
nondeterminate applications behavior, are considered. A formal model based on
resources and events and describing an application behavior is described. The
algorithms of an application execution logging on the reserved cluster node, of
restoring and continuation of running after main node fault are proposed. In
the proposed approach, both fault and recovery are hidden from the client who
experiences slight service quality degradation at worst.

\KWN{applications server;   transparent fault tolerance;
process;  resource; event;  check point;  determinate}

\vfill
% \vskip 18pt plus 6pt minus 3pt
% \vskip 24pt plus 9pt minus 6pt

\def\tit{MULTICHANNEL QUEUING SYSTEM WITH FINITE BUFFER AND UNRELIABLE SERVERS}

\def\aut{A.\,Pechinkin$^1$, I.~Sokolov$^2$, and V.~Chaplygin$^3$}

\def\auf{$^1$IPI RAN, apechinkin@ipiran.ru\\[1pt]
$^2$IPI RAN, isokolov@ipiran.ru\\[1pt]
$^3$IPI RAN, vchaplygin@ipiran.ru}

%\def\leftkol{ENGLISH ABSTRACTS}

%\def\rightkol{ENGLISH ABSTRACTS}

\titele{\tit}{\aut}{\auf}{\leftkol}{\rightkol}

\noindent Multichannel queuing system with a semimarkovian input flow, a phase
type distribution of the servicing, a finite buffer and unreliable servers
refusing independently one from other and regargless of the whole process of
refusals and restorations is considered. Mathematical relations allowing
calculation of the main stationary characteristics of system functioning are
found under some different variants of the process of refusals and
restorations.

\KWN{queuing system; unreliable servers}

\pagebreak
% \vskip 24pt plus 9pt minus 6pt
%\vskip 18pt plus 6pt minus 3pt
\ \vspace*{-24pt}

\def\tit{A NEW METHOD FOR THE PROBABILISTIC AND
STATISTICAL ANALYSIS\newline OF INFORMATION FLOWS IN~TELECOMMUNICATION
NETWORKS}
\def\aut{D.~Batrakova$^1$, V.~Korolev$^2$, and~S.Shorgin$^3$}
\def\auf{$^1$IPI RAN, daria.batrakova@gmail.com\\[1pt]
$^2$Faculty of Computational Mathematics \& Cybernetics of Moscow State
University; IPI RAN, vkorolev@comtv.ru\\[1pt]
$^3$IPI RAN, sshorgin@ipiran.ru}

%\def\leftkol{ENGLISH ABSTRACTS}

%\def\rightkol{ENGLISH ABSTRACTS}

\titele{\tit}{\aut}{\auf}{\leftkol}{\rightkol}

\noindent
A new method is proposed for the analysis of the stochastic structure
of chaotic information flows in convergent telecommunication networks. The
proposed method is based on a stochastic model of a telecommunication network
which has the form of a superposition of some simple sequential-parallel
structures. This model quite naturally generates mixtures of
gamma-distributions for the network operation execution time. The parameters of
the mixture of gamma-distributions characterize the stochastic structure of
chaotic information flows in the network. The problem of statistical estimation
of the parameters of gamma-distributions is solved by the EM-algorithm. To
trace the variation of the stochastic structure of information flows in time,
the EM-algorithm is applied for the moving window. The software for the
division of distribution mixtures is developed and documented. The real input
data is used. The interpretation of results is given.

\KWN{telecommunication networks; information flows; division of mixes of
distributions; method of a moving window; software for division of distribution
mixtures}

%\vfil
% \vskip 24pt plus 9pt minus 6pt
\vskip 6pt plus 3pt minus 3pt
%\vspace*{12pt}

\def\tit{LINGUISTIC SIMULATION FOR MACHINE
TRANSLATION AND KNOWLEDGE MANAGEMENT SYSTEMS}

\def\aut{E.\,B.~Kozerenko}
\def\auf{IPI RAN, kozerenko@mail.ru}

\def\leftkol{ENGLISH ABSTRACTS}

\def\rightkol{ENGLISH ABSTRACTS}

\titele{\tit}{\aut}{\auf}{\leftkol}{\rightkol}

\noindent This paper is dedicated to the vital problems of creating
semantic-syntactic presentations for the systems of machine translation and
extraction of knowledge from natural language texts. The purpose of our studies
is the construction of an integral linguistic model on the basis of a
synergetic approach, which uses linguistic knowledge, statistical methods, and
mechanisms of machine learning for the extraction of new grammar rules from
text corpora and disambiguation of language structures. To formalize linguistic
knowledge, we have developed a new Cognitive Transfer Grammar which is a
semantically motivated version of a generative unification grammar. For the
preparation of system training components and obtaining statistical data about
language structures, a multilingual resource is being created, comprising a
Treebank and a corpus of semantically aligned parallel texts in Russian,
English, and a number of other European languages.

\KWN{machine translation; grammar formalisms; linguistic model;
parallel texts alignment; semantics; syntax; phrase structure}

% \vspace*{12pt}
%\pagebreak
% \vskip 24pt plus 9pt minus 6pt
\vskip 6pt plus 3pt minus 3pt


\def\tit{THE SYMBOL MODEL OF INFORMATICS KNOWLEDGE SYSTEM\newline IN
HUMAN-AUTOMATON ENVIRONMENT}

\def\aut{V.\,D.~Ilyin$^1$ and I.\,A.~Sokolov$^2$}

\def\auf{$^1$IPI RAN, vdilyin@ipiran.ru.\\[1pt]
$^2$IPI RAN, isokolov@ipiran.ru}

\def\leftkol{ENGLISH ABSTRACTS}

\def\rightkol{ENGLISH ABSTRACTS}

\titele{\tit}{\aut}{\auf}{\leftkol}{\rightkol}

\noindent The paper describes the authors approach to the construction of
informatics knowledge system in human-automaton environment. This system is
investigated as means of formalized representation of scientific results. The
bases of conception of the construction and application methodology of this
system are described. This methodology is studied as scientific foundation for
automation of research, design, and education.

\KWN{informatics; symbol; symbol model; human-automaton environment;
knowledge system; symbol modeling in human-automaton environment}

\label{end\stat} \pagebreak