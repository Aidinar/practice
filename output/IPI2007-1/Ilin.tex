\def\stat{iliin}
\def\tit{
СИМВОЛЬНАЯ МОДЕЛЬ СИСТЕМЫ ЗНАНИЙ ИНФОРМАТИКИ
В~ЧЕЛОВЕКО-АВТОМАТНОЙ СРЕДЕ}
\def\titkol{
Символьная модель системы знаний информатики
в~человеко-автоматной среде}
\def\autkol{В.\,Д.~Ильин, И.\,А.~Соколов}
\def\aut{В.\,Д.~Ильин$^1$, И.\,А.~Соколов$^2$}

\titel{\tit}{\aut}{\autkol}{\titkol}

\footnotetext[1]{ИПИ РАН, vdilyin@ipiran.ru} \footnotetext[2]{ИПИ РАН, 
isokolov@ipiran.ru}


\Abst{Предложен подход к построению системы знаний информатики
в человеко-автоматной среде как средству формализованного представления
научных результатов. Изложены основы концепции методологии построения и
применения этой системы. Методология изучается как научное основание
технологий автоматизации научных исследований, проектирования и
образовательных процессов.}

\KW{информатика; символ; символьная модель;
человеко-автоматная среда; система знаний; символьное моделирование в
человеко-автоматной среде}

%\vskip 35pt plus 9pt minus 6pt
\vskip 24pt plus 9pt minus 6pt

\begin{multicols}{2}


\label{st\stat} 

\thispagestyle{headings}

\section{Введение}

\paragraph*{Символьные модели в познании.}Совершенствование инструментов
познания связано с изобретением и применением символов и символьных моделей
изучаемых объектов. При этом \textbf{символ} рассматривается как заменитель 
некоторого объекта, включая другие символы. В~наши дни наиболее употребляемыми 
\textbf{типами символов} являются аудио, текстовые, графические и видео 
(например, в гипермедийных документах, выложенных на веб-сайтах). Моделируемые 
объекты могут иметь любую физическую сущность, размещение, происхождение и 
назначение. Это могут быть объекты, наблюдаемые человеком непосредственно или с 
помощью технических устройств, порожденные его разумом и построенные с помощью 
машин. Другими словами, объектами символьного моделирования могут быть 
\textbf{произвольные объекты}: проектирование машин, программирование или 
что-то еще.

Результаты символьного моделирования размещаются в памяти человека и
во внешней среде (в частности, в памяти машин). Затраты на построение,
копирование, передачу, сохранение и накопление символьных моделей, как
правило, значительно меньше, чем аналогичные затраты, связанные с
несимвольными моделями (например, макетами судов, зданий и~др.).

Основное достоинство символьного моделирования заключается в том, что оно не 
только повышает продуктивность познания, но, что не менее важно, расширяет его 
пространство. Достаточно вспомнить, какое значение на заре символьного 
моделирования имело изобретение \textbf{графических моделей языков сообщений} 
(основания письменности). Это изобретение сыграло выдающуюся роль в 
формировании инструментария научного познания. 
{\looseness=1

}

Символьные модели систем понятий, построенных на их основе систем
знаний и других объектов, методы их построения, сохранения и передачи~---
все это вызывает неуклонно растущий интерес с тех давних пор, как люди
убедились в преимуществах, которые дают способности видеть актуальные
задачи, формулировать их и находить методы решения. Этот интерес привел к
идее применения машин, помогающих решать задачи символьного
моделирования. На пути от рукописных текстов, рисунков и схем к
книгопечатанию и графическим моделям в проектировании, от звукозаписи,
фотографии и радио к кино и телевидению, от компьютеров и локальных сетей
к глобальной сети, виртуальным лабораториям и дистанционному образованию
постоянно растет роль символьных моделей, которые человек создает с
помощью машин. Сначала решаемые с помощью машин задачи были
математическими и имели вычислительный характер (отсюда и
название~--- вычислительная машина (computer)). За <<деревьями>> проблем
построения и применения машин для решения задач не сразу удалось
разглядеть их принадлежность <<лесу>> проблем символьного моделирования
произвольных объектов в человеко-автоматной среде.

В наши дни трудно найти область деятельности, где бы не применялись
результаты символьного моделирования в человеко-автоматной среде.
Примеров удачной реализации символьных моделей в человеко-автоматной
среде так много, что из них непросто выбрать. Это, конечно же, Интернет,
электронная почта, Веб, САПРы и множество других, включая системы для
игры в шахматы, лучшие из которых на равных соперничают с чемпионами
мира.

\paragraph*{Современный этап.}На современном этапе в центре
внимания~--- цифровое символьное моделирование. <<Цифровое>>~--- потому, что 
символы любого типа и построенные из них модели представлены в виде цифровых 
кодов, рассчитанных на манипулирование посредством электронных машин. В~таких 
машинах основу аппаратной со\-став\-ля\-ющей представляют микроэлектронные 
схемы, а цифровые коды реализуются такими схемами в виде двоичных кодов. Отсюда 
и слова <<электронный>> и <<циф\-ро\-вой>> в названиях понятий (электронный 
документ, электронная почта, цифровая технология и~др.). {\looseness=1

}

\subsection{Цель исследований} %1.1

Для науки особое значение имеют методологии символьного моделирования
сис\-тем понятий и систем знаний~\cite{1il}. Результаты изучения символьного
моделирования в человеко-автоматной среде целесообразно представлять в
виде символьных моделей систем понятий и систем знаний, удовлетворяющих
требованиям реализации в человеко-автоматной среде.

В Институте проблем информатики РАН создается методология построения и 
применения \textbf{символьной модели системы знаний информатики}. Эта модель, 
получившая имя Синф (в формульных час\-тях текста~--- Sinf), рассматривается 
как средство формализованного представления научных результатов, рассчитанных 
на применение \textbf{символьного моделирования произвольных объектов в 
человеко-автоматной среде}, причем не только исследователями 
(информатиками\footnote{Информатика~-- информатик (так же, как математика~-- 
математик).}). Методология построения и применения Синф изучается как основа 
технологий автоматизации научных исследований, образовательных процессов, 
проектирования и~др. Результаты исследований, полученные при создании 
Синф-методологии, используются для научно-методологической поддержки создания 
Большой Российской Энциклопедии (части, отнесенной к разделу <<Информатика>>) и 
в вузовском образовательном процессе (МИРЭА, базовая кафедра проблем 
информатики ИПИ РАН).

В статье представлены основы концепции создания Синф как гипермедийной системы 
знаний, имеющей сервис-ориентированную архитектуру. 
{\looseness=1

}

\subsection{Обозначения }

Применяемые в статье обозначения введены для компактного представления
часто по\-вто\-ря\-ющих\-ся устойчивых словосочетаний и выделения частей текста, к
которым авторы хотят привлечь внимание читателя.
\begin{enumerate}[1.]
\item \textit{Префикс s-}  в сокращениях имен понятий введен как
\textit{обозначение принадлежности сис\-те\-ме понятий символьного
моделирования в че\-ло\-ве\-ко-ав\-то\-мат\-ной среде}. Он может размещаться 
перед именем одного понятия (например, \textit{s-мо\-де\-ли\-ро\-ва\-ние}) или 
списком имен понятий, заключенных в скобки Например, s-(пред\-став\-ле\-ние, 
конструирование, интерпретация). В~определениях наиболее важных понятий 
сокращения не применяются.
\item К наиболее употребляемым сокращениям относятся:

s-моделирование~--- символьное моделирование произвольных объектов
в человеко-автоматной среде;

s-модель~--- символьная модель произвольного объекта в
человеко-автоматной среде;

s-машина~--- машина для построения и применения s-моделей;

s-среда~--- человеко-автоматная среда s-мо\-де\-ли\-ро\-ва\-ния;

Dn~--- обозначения;

$\approx$~--- заменяет.
\item Для выделения определений, замечаний, примеров, имен понятий и
отдельных частей текста используются следующие средства:

$\triangleright~<$текст$>~\triangleleft$~--- часть текста с фиксированными в ее
пределах обозначениями, до\-пол\-ня\-ющи\-ми приведенные здесь;

$\square\ <$текст$>~\square$~--- определение;

$\diamond~<$текст$>~\diamond$~--- замечание;

$\oplus~<$текст$>~\oplus$~--- пример;

$\{\mbox{S} <$$\textit{текст}$$>$$<$$\mbox{список}$$>\mbox{S}\}$~--- здесь
$<$\textit{текст}$>$~$\approx$ набранный \textit{курсивом} текст (может быть
пус\-тым), который следует интерпретировать как расширенный префикс
s-$<$\textit{текст}$>$ для выделенных \textit{курсивом} элементов списка;

$\oplus\ \{\mbox{S}$~\textit{модель}$<\mbox{список}>\mbox{S}\}$~--- здесь
расширенный префикс~--- \textit{s-модель}; $\{\mbox{S}\ <\mbox{список}>\
\mbox{S}\}$~--- здесь префикс \textit{s-}.~$\oplus$
\end{enumerate}

Р\,а\,з\,р\,е\,ж\,е\,н\,н\,ы\,й\,\ ш\,р\,и\,ф\,т,\,\ \textit{курсив},\,\
\textit{р\,а\,з\,\-р\,е\,\-ж\,е\,н\,\-н\,ы\,й\,\,\ к\,у\,р\,с\,и\,в\,}\,\ 
применяются для названий понятий и частей текста, значение которых авторы 
вы\-де\-ляют. 

\section{Cимвольное моделирование\newline в человеко-автоматной среде:
основные понятия} %2

Изложение подхода к решению комплекса задач построения и применения
системы знаний информатики в человеко-автоматной среде включает
характеристику s-моделирования как объекта исследований, определение
основных понятий, описание классов основных задач. Значительное место
отведено мотивировке выбранного подхода и пояснениям предложенных
определений.

С развитием техники, видимо, не могла не родиться идея решения задач с помощью 
автоматов. С тех пор, как эта идея стала постоянным стимулом для изобретателей 
машин, помогающих решать задачи, было предложено и построено немало автоматов 
такого назначения. Особое место среди них принадлежит \textit{программируемым 
машинам со сменяемыми программами}. По мере роста производительности s-машин 
этого типа возрастал интерес к их применению. Настало время, когда в математике 
сформировался и стал интенсивно развиваться раздел вычислительной (машинной) 
математики. Изобретатели s-машин различных архитектур и программисты начального 
этапа s-моделирования, физики и инженеры, создававшие элементную базу, 
проектировавшие ком\-п\-лек\-ту\-ющие изделия и периферийные устройства для 
первых s-машин~--- все они результатами доказали целесообразность серийного 
производства s-машин.

\subsection{S-моделирование: новый тип символьного моделирования }
%2.1

Как инструмент познания этот тип символьного моделирования привлек
внимание исследователей прежде всего своей высокой продуктивностью, не
зависящей от природы моделируемых объектов. На определенном этапе стало
ясно, что новая реальность, созданная людьми, заслуживает внимания
организационно оформленных коллективов исследователей. Так, \textit{изучение
свойств и закономерностей символьного моделирования произвольных
объектов в человеко-автоматной среде, со\-сто\-ящей из людей и управляемых
ими машин для построения и применения символьных моделей}, привело к
созданию новой науки.

Для краткости такое моделирование авторы назвали 
\textit{s-мо\-де\-ли\-ро\-ва\-ни\-ем}, среду его реализии~--- 
\textit{s-сре\-дой}, получаемые модели~--- 
\textit{s-мо\-де\-ля\-ми}~\cite{1il}, а машину для построения и применения 
символьных моделей~--- \textit{s-ма\-ши\-ной}.

Главной особенностью s-моделирования является то, что оно осуществляется с 
помощью s-ма\-шин, помогающих людям в построении и применении символьных 
моделей. Заметим, что и до того машины (печатные станки, типографские машины 
и~др.) использовались людьми в символьном моделировании. Но это не была помощь 
в изобре\-те\-нии и конструировании символьных моделей. Такая помощь стала 
реальностью, когда появились программируемые машины со сменяемыми программами.
{\looseness=1

}



\noindent
$\oplus$~Исследования в области автоматизации программирования~---
пример изучения одного из типов s-моделирования, которым является
программирование.~$\oplus$

Продуктивность изучения свойств и закономерностей s-моделирования на каждом 
этапе развития информатики зависит от точности не только формулировки предмета 
исследований, но и основных классов задач, основы научного метода, содержания и 
формы представления научной продукции.

Какими свойствами обладает s-моделирование? Какие закономерности ему
присущи? Как его типизировать? Что представляют собой s-модели
произвольных объектов в s-среде, каковы их свойства, как строить такие
модели и манипулировать ими, используя s-машины? От ответов на такие
вопросы зависят ответы на более конкретные вопросы.

Какими должны быть
правила конструирования s-моделей с помощью s-машин? Какой должна быть
s-среда? Из поиска ответов на эти и связанные с ними вопросы состоит процесс
исследования s-моделирования.

Какие задачи s-моделирования актуальны на современном этапе и как их
решить? Какой должна стать s-среда, чтобы соответствовать требованиям к
реализации задач s-моделирования ближайшего будущего? На разных этапах
развития информатики ответы на поставленные вопросы будут отличаться.

\paragraph*{Формализация.}Особое место в развитии символьного моделирования принадлежит идее
его формализации, заключающейся в том, чтобы строить символьные модели
по определенным правилам из заранее определенных элементов. Эта идея
реализована в математических методах символьного моделирования. Однако
метод формализации~[2],
применяемый в математике для получения
формальных систем~[3],
нельзя перенести на s-моделирование, так как
s-модели не являются формальными системами. Объясним подробнее это
важное замечание.

Задача в s-мо\-де\-ли\-ро\-ва\-нии имеет более широкий смысл, чем в математике:
рассматриваемые в разд.~2.3 задачи s-(пред\-став\-ле\-ния, распознавания,
преобразования, конструирования, интерпретации, обмена, сохранения, накопления, 
поиска, защиты) не являются математическими, хотя при их решении применяются и 
математические методы. Но математический арсенал недостаточен для того, чтобы 
каждую из указанных задач можно было сформулировать и решить как математическую 
задачу. Дело здесь не только в том, что в классической математике царствует 
формальное доказательство (существования, единственности решения), а 
в\linebreak
 s-мо\-де\-ли\-ро\-ва\-нии~--- конструктивное доказательство 
(существования s-мо\-де\-ли; а о единственности вообще речь не идет). Важно 
другое: неформальность s-мо\-де\-лей~--- это их полезное отличие, связанное с 
возможностью привлечения неформализованного знания че\-ло\-ве\-ка-экс\-пер\-та 
для управления ходом решения (примером может служить разработанная в ИПИ РАН 
методология интерактивного преобразования ресурсов~[4]).

\noindent
$\diamond$~\textit{S-моделирование предполагает представление символов и
построенных из них s-моделей в двух формах, одна из которых рассчитана на
интерпретацию человеком, другая (в форме кодов символов
и s-моделей)~--- на
интерпретацию программой, выполняемой s-машиной}. Множество символов,
применимых для построения s-моделей~--- это множество элементарных
конструктивных объектов, каждый из которых наделен набором атрибутов и
совокупностью допустимых операций. Построение конструкций из элементов
этого множества определено системой правил конструирования
s-моделей.~$\diamond$

\subsection{Символы и коды в s-среде} %2.2

Исходим из того, что произвольному объекту можно поставить в
соответствие символ или символьную модель. Из этого следует, что и символу
можно поставить в соответствие другой символ того же или другого типа.
Символы, рассчитанные на s-машину, называют кодами, процесс
преобразования символов в коды~--- кодированием, а обратный ему
процесс~--- декодированием.

\noindent
$\square$~S-символ~--- это заменитель некоторого объекта,
удовлетворяющий требованиям представления в s-среде.~$\square$

Не накладывается никаких ограничений на типы заменяемых объектов. В частности, 
одни символы могут служить заменителями других. Единственным ограничением, 
определяющим существование того или иного типа\footnote{Под типом понимается 
множество значений.} символов, служит реализуемость этого типа в s-среде.

\noindent
$\oplus$~Примеры s-символов различных
типов: буквы текста при работе с текстовым редактором (текстовый);
гиперссылки веб-страницы (гипертекстовый); неподвижные изображения,
полученные с помощью цифровой фотокамеры (графический); звуковые
сигналы вызова мобильного телефона (аудио); видеоклипы (видео); вибровызов
мобильного телефона (механический).~$\oplus$

\paragraph*{Базовые типы символов.}Для представления
сим\-воль\-ных моделей используются следующие\,\ 
\textit{б\,а\,\-з\,о\,\-в\,ы\,е\,\ т\,и\,п\,ы\,\ с\,и\,м\,в\,о\,л\,о\,в}:
\begin{itemize}
\item \textit{а\,у\,д\,и\,о\,} (для представления звукового сообщения);
\item \textit{г\,р\,а\,ф\,и\,ч\,е\,с\,к\,и\,й\,} (для представления
сообщения в форме неподвижного изображения);
\item \textit{т\,е\,к\,с\,т\,о\,в\,ы\,й\,} (специализация типа
\textit{графический});
\item \textit{ч\,и\,с\,л\,о\,в\,о\,й\,} (специализация типа
\textit{текстовый});
\item \textit{в\,и\,д\,е\,о\,} (для представления сообщения в форме
движущихся изображений);
\item \textit{г\,и\,п\,е\,р\,т\,е\,к\,с\,т\,о\,в\,ы\,й\,} (для представления сообщения в форме
гипертекста~[5];
\item \textit{м\,е\,х\,а\,н\,и\,ч\,е\,с\,к\,и\,й\,} (для представления сообщения в форме
механических воздействий\footnote{Пример: вибровызов мобильного телефона.}).
\end{itemize}

\paragraph*{Композиции из базовых типов символов.}Композицией
из базовых типов символов будем называть любое сочетание из
числа возможных сочетаний базовых типов. Вряд ли нуждается в пояснениях
стремление использовать в сообщениях все типы базовых символов.
Достаточно вспомнить, какую роль сыграла мультимедийная форма
представления документов, которая с добавлением видео затем стала
гипермедийной.

\subsection{S-моделирование: основные составляющие и классы задач}
%2.3

Изучение свойств и закономерностей s-мо\-де\-ли\-ро\-ва\-ния позволяет 
определить задачи s-мо\-де\-ли\-ро\-ва\-ния и заняться поиском методов их 
решения. Говоря о задачах, имеем в виду задачи исследований. Поэтому названия 
задач далее представлены как названия тематических подпространств пространства 
s-мо\-де\-ли\-ро\-ва\-ния.

Результат наших исследований \textit{s-моделирования} как \textit{технологии} 
(\textit{комплекса методов}) \textit{изготовления s-моделей произвольных 
объектов и манипулирования ими} представлен далее развернутым определением 
составляющих s-моделирования и соответствующих им классов задач. Форма 
определения может служить примером текстовой модели сообщения\footnote[1]{Эта 
форма применяется также в приложении.}, рассчитанной на представление на 
бумажном носителе.

\noindent 
$\triangleright$~[Dn: PrC\;$\approx$ <<класс задач>>; $\rightarrow 
\approx$\;<<связанный с предыдущим>>]

\noindent 
$\square \{$S

\begin{enumerate}[1.]
\item \textit{П\,р\,е\,д\,с\,т\,а\,в\,л\,е\,н\,и\,е\,\ моделей} на \textit{языках
сообщений}, рассчитанных на восприятие человеком и \textit{машинами}:


PrC языки как средство представления \textit{моделей сообщений}: базовые типы 
\textit{символов} и \textit{кодов}\;$\rightarrow$\;системы \textit{символов} и 
\textit{кодов};~\textit{модели} че\-ловеко- и 
ма\-шин\-но-ори\-ен\-ти\-ро\-ван\-ных \textit{языков сооб\-ще\-ний} 
(\textit{языки} спецификации, программирования, запросов, системы команд 
машин);~\textit{модели} представления: данных~[6], документов; представление 
\textit{моделей} систем понятий, на которых интерпретируются 
\textit{сообщения}, составленные на языках~$\rightarrow$\;\textit{представление 
моделей} систем знаний; $\rightarrow$~\textit{модели} технологий 
\textit{представления};
\item \textit{П\,р\,е\,о\,б\,р\,а\,з\,о\,в\,а\,н\,и\,е\,\ типов и форм
представления моделей}.

PrC \textit{преобразование} типов пред\-став\-ле\-ния \textit{моделей} 
($\oplus$~аудио в текст и обратно~$\oplus$);

\textit{преобразование} форм \textit{представления моделей} (аналоговой в 
цифровую и обратно; несжатой в сжатую и обратно; незашифрованной в 
за\-шиф\-ро\-ван\-ную и обратно; несжатой и незашифрованной в сжатую и 
за\-шиф\-ро\-ван\-ную и обратно) ($\oplus$~одной формы представления документа 
в другую:~*.doc в 
*.pdf $\oplus$)\;$\rightarrow$~\textit{модели} технологий 
\textit{преобразования}.
\item \textit{Р\,а\,с\,п\,о\,з\,н\,а\,в\,а\,н\,и\,е\,\ моделей сообщений}:

PrC сопоставление с \textit{моделью-образцом}, со\-по\-став\-ле\-ние свойств
распознаваемой \textit{модели} со свойствами \textit{модели-образца}; 
\textit{распознавание содержимого документов} $\rightarrow$~\textit{модели} 
технологий \textit{распознавания};
\item \textit{К\,о\,н\,с\,т\,р\,у\,и\,р\,о\,в\,а\,н\,и\,е\,\ моделей}:

PrC \textit{конструирование моделей}: s-мо\-де\-ли\-ро\-ва\-ния и его 
составляющих; s-сре\-ды; \textit{языков}, систем понятий и систем знаний, 
\textit{интерпретаторов сообщений} на \textit{моделях} сис\-тем понятий, систем 
\textit{символов}, систем \textit{кодов}; \textit{моделей} специфицирования, 
программирования, взаимодействия в s-сре\-де; \textit{моделей}: 
задач~\cite{7il}, алгоритмов, программ; моделей: архитектур машин, архитектур 
сетей, сер\-вис-ори\-ен\-ти\-ро\-ван\-ных архитектур. [$\oplus$~В~марте 
2007~г.\ группа\footnote[2]{BEA, CA, Cisco, EMC (Documentum), HP, IBM, Intel, 
Microsoft and Sun Microsystems.} участников консорциума W3C представила~[8] 
спецификации языка SML моделирования сервисов (Service Modeling Language~[9]) и 
формата обмена SML моделями (SML Interchange Format~[10] Version 
1.0).~$\oplus$]; \textit{моделей}: \textit{сообщений} и средств их построения; 
документов и документооборота~$\rightarrow$~модели технологий 
\textit{конструирования}.
\item \textit{И\,н\,т\,е\,р\,п\,р\,е\,т\,а\,ц\,и\,я\,\ моделей:}

PrC \textit{модели} интерпретации \textit{сообщений на моделях} систем понятий;
\textit{модели} трансляции (компиляции, интерпретации, ассемблирования)
$\rightarrow$~\textit{модели} технологий \textit{интерпретации};
\item \textit{О\,б\,м\,е\,н\,\ моделями}:

PrC \textit{модели} взаимодействия в s-среде: че\-ло\-век--\textit{ма\-ши\-на};
\textit{ма\-ши\-на}--\textit{ма\-ши\-на}; \textit{модели}: типов отправителей и
получателей; средств отправки, пере\-да\-чи и получения сообщений; сред 
передачи \textit{сооб\-ще\-ний}; \textit{модели}: архитектур сетей, 
сервис-ори\-ен\-ти\-ро\-ван\-ных архитектур; \textit{модели} сис\-тем правил 
обмена сообщениями (коммуникационных протоколов); \textit{модели} 
документооборота; $\rightarrow$~\textit{модели} технологий \textit{обмена};
\item \textit{С\,о\,х\,р\,а\,н\,е\,н\,и\,е,\,\ н\,а\,к\,о\,п\,л\,е\,н\,и\,е\,\ 
и\,\ п\,о\,и\,с\,к\,\ моделей:}

PrC \textit{модели} процессов сохранения, накопления и поиска; типов памяти и
накопителей; управ\-ле\-ния памятью и накопителями; форм сохранения и
накопления; типов носителей; средств сохранения, накопления и поиска;
\textit{модели} баз данных, библиотек программ и~др.; \textit{модели}
спецификации предмета поиска; \textit{модели} поиска по образцу, по
признакам, по описанию свойств; \textit{модели} поисковых машин;
$\rightarrow$~\textit{модели} технологий сохранения, накопления и поиска.
\item \textit{З\,а\,щ\,и\,т\,а\,\ моделей} от несанкционированного доступа и
применения:

PrC \textit{модели} уязвимостей, контроля доступа, защиты от вторжений, 
вредоносных программ, перехвата сообщений, несанкционированного применения 
$\rightarrow$~\textit{модели} технологий защиты.
\end{enumerate}
\noindent S$\}\square\triangleleft$


Далее определения приведены с использованием названий основных классов задач  
s-мо\-де\-ли\-ро\-ва\-ния.

\noindent
$\square$~\textit{S-модель}~--- символьный конструктивный объект,
s-представление которого удовлетворяет требованиям s-распознавания
человеком и s-машиной.~$\square$

\noindent $\oplus$~Веб-страница на экране монитора~--- это гипермедийная 
\textit{s-модель} сообщения, где используются аудио-, текстовый, графический и 
видеотипы; сканируемая таблица~--- графическая s-модель; содержимое ячеек 
таблицы табличного процессора, для которых задан числовой формат (числовой 
тип); мелодия вызова мобильного телефона~--- аудио\linebreak
 s-мо\-дель; 
схематический видеоклип из электронной энциклопедии, сопровождаемый 
пояснениями~--- s-модель, построенная с использованием аудио- и 
видеотипов.~$\oplus$

\noindent $\square$~\textit{S-код}~--- символьный конструктивный объект, 
являющийся s-пред\-став\-ле\-ни\-ем s-мо\-де\-ли, рассчитанным на s-ма\-ши\-ну. 
Удовлетворяет требованиям s-(рас\-по\-зна\-ва\-ния, преобразования, 
конструирования, интерпретации, обмена, сохранения, накопления, поиска, 
защиты).~$\square$

\noindent
$\oplus$~\textit{S-коды}: \textit{код} исполняемой программы, \textit{код}
цифровой аудиозаписи.~$\oplus$

\noindent
$\diamond$~Разработка эффективных систем \textit{символов} и
\textit{кодов}, методов \textit{кодирования} и \textit{декодирования}~--- важные
составляющие проблемы построения s-среды.~$\diamond$

\paragraph*{Цифровое представление символов.}Символу
лю\-бо\-го типа можно поставить в соответствие некоторое число. Постановку 
числа в соответствие символу называют цифровым кодированием, а его 
результат~--- цифровым представлением символа. Для\linebreak чисел, заменяющих 
символы, выбирают целесообразную систему счис\-ле\-ния. Этот выбор 
на\-прав\-ля\-ет\-ся стремлением обеспечить наиболее эффективное 
манипулирование кодами символов и s-мо\-де\-лей, которое выполняет цифровая 
s-ма\-ши\-на при решении различных задач. Выбор ограничен рядом условий, среди 
которых фи\-зи\-ко-тех\-ни\-че\-ская реализуемость используемых для построения 
s-ма\-шин элементов с количеством устойчивых состояний, равным основанию 
выбранной системы счисления.

\noindent
$\oplus$~Программы и данные в современных
цифровых s-машинах (компьютерах, смартфонах, цифровых фото- и
видеокамерах и~др.) представлены в виде двоичных кодов.~$\oplus$

Преобразование
(АЦП) из аналоговой формы представления символов в цифровую при вводе и
обратное преобразование (ЦАП) при выводе связывают цифровые s-машины с
их аналоговым окружением в s-среде, к которому относятся и управляющие
ими люди.

\subsection{S-моделирование: общий метод\newline и условие реализации} %2.4

\noindent $\square$~\textit{Общий метод s-моделирования}~--- конструктивное 
доказательство существования s-мо\-де\-ли, представимой в двух формах, одна из 
которых рассчитана на интерпретацию человеком, а другая~--- 
s-ма\-ши\-ной.~$\square$

\textit{Необходимое условие реализации s-моделирования} предполагает
существование удов\-ле\-тво\-ря\-ющих требованиям s-(представления,
распознавания, преобразования, конструирования, интерпретации, обмена,
сохранения, накопления, поиска и защиты):
\begin{enumerate}[(1)]
\item языка описания s-моделей, рассчитанного на человека;
\item языка команд s-машины;
\item программ s-преобразования s-моделей на языке для человека в описания
на языке команд s-машины.
\end{enumerate}

\noindent
$\diamond$~Формальное символьное моделирование в математике не
стеснено требованиями~(1)--(3). Конечно, языку математического моделирования
можно поставить в соответствие язык описания s-моделей. [$\oplus$~Пролог
(логика предикатов первого порядка), Лисп ($\lambda$-исчисление).~$\oplus$]

%\noindent
Развитие языков s-моделирования, рассчитанных на человека, направляется
стремлением использовать полную композицию базовых символов для
построения символьных систем языков, биб\-лио\-те\-ки и средства
конструирования спецификаций задач, а по полученным спецификациям~---
программ решения задач.~$\diamond$

В Приложении приведены этапные результаты, полученные в процессе становления и
развития символьного моделирования.

\section{S-моделирование системы знаний информатики: основания } %3

Системы понятий и системы знаний являются наиболее важными для науки объектами 
символьного моделирования. Рассмотрение задачи построения s-мо\-де\-ли системы 
\textit{S-ics} понятий и системы \textit{Sinf} знаний информатики начнем с 
формулирования требований, которым должны удовлетворять определения систем 
понятий информатики, чтобы быть применимыми в \textit{Sinf}.

В создаваемой \textit{Синф}-методологии система \textit{\mbox{S-ics}} понятий 
рассматривается как пара: множество \textit{set[S-ics]} систем понятий и 
семейство \textit{rel[set[S-ics]]} связей, заданных на множестве 
\textit{set[S-ics]} (где \textit{[\mbox{S-ics}]} и \textit{ [set[S-ics]]}~--- 
пометы)\footnote{<<Одноэтажная>> форма записи та же, что и разработанная 
в~\cite{7il} для языка спецификации задачных конструктивных объектов. 
В~\textit{Синф} эта форма применяется в языках спецификации s-моделей систем 
понятий и систем знаний.}. Каждая система понятий из \textit{set[S-ics}] 
представлена аналогично. Не устанавливается никаких ограничений на добавление 
новых систем понятий, детализацию и коррекцию существующих. При описании связей 
между s-моделями понятийных систем за основу взят подход, разработанный в ИПИ 
РАН в рамках исследований, посвященных методологии порождения 
программ~\cite{7il}.

\noindent
$\oplus$~В частности, две системы понятий считаются
связанными, если их пересечение по памяти непусто (память каждой системы пред\-став\-ле\-на
элементами множества входящих в нее понятий).~$\oplus$

Язык спецификации s-моделей
систем понятий и систем знаний, разрабатываемый применительно к \textit{Синф}, является
развитием языка спецификации задачных конструктивных объектов, разработанного
в~\cite{7il}.

Естественно, рамки статьи не позволяют рас\-суж\-дать о всей системе понятий
информатики. Поэтому изложение сосредоточено на описании отдельных
составляющих \textit{S-ics}, среди которых \textit{символьная модель системы
понятий, символьная модель системы знаний, сообщение, данные,
информация}.

\subsection{Конструктивность определений} %3.1

Конструктивность~--- это свойство объектов, заключающееся в том, что они могут 
использоваться для построения конструкций по определенной сис\-те\-ме правил. 
При этом для каждого типа объектов должны быть заданы набор атрибутов и 
совокупность допустимых операций.

Определение системы понятий конструктивно, если оно может быть использовано для 
построения определений других систем понятий. Реализация конструирования 
определений систем понятий в \linebreak
s-сре\-де накладывает ограничение, 
связанное с представлением определений в форме s-мо\-де\-лей.

\noindent $\square$~Чтобы быть конструктивным, определение системы понятий 
информатики должно удовлетворять следующим необходимым требованиям:
\begin{enumerate}[1.]
\item Определение системы понятий должно быть представлено в виде пары
$<$\textit{описание области применимости}$>$, $<$\textit{s-модель системы
понятий}$>$.
\item В систему понятий, считающуюся определенной, не должны входить
понятия, не име\-ющие определений (и при этом не относящиеся к
понятиям-аксиомам).
\item Область применимости определяемой системы должна принадлежать
информатике.~$\square$
\end{enumerate}

Требование~1 связано с тем, что информатика имеет дело с моделями,
которые рассчитаны на реализацию в s-среде (то есть с s-моделями).

\noindent
$\square$~Описание области применимости (точки зрения~\cite{7il})~--- это
описание типов:
\begin{itemize}
\item \textit{корреспондента} (кому адресовано определение);
\item \textit{цели}, в процессе достижения которой оно имеет смысл (классы
задач, при изучении которых определение может быть полезно);
\item \textit{стадии}, на которой целесообразно использовать определение
(концепция исследования проблемы, методология решения проблемы
(постановки задач и методы их решения), проектирование (разработка
программно-аппаратных средств, сервис-ориентированных архитектур),
поддержка применения сервисов, применение сервисов и~т.д.).~$\square$
\end{itemize}

Формулировка определения в виде пары $<$\textit{описание области применимости}$>$,
$<$\textit{сим\-воль\-ная модель системы понятий}$>$ позволяет уменьшить неопределенность
истолкования\footnote{Большая часть недоразумений с употреблением определений не по
назначению происходит из-за отсутствия описаний области применимости.}. Когда по контексту
ясно, что \textit{описание области применимости} не изменилось, будем ограничиваться ссылкой на него
(указанием идентификатора). Идентификатор имеет вид \textit{as~id}, где \textit{a}~--- постоянная
составляющая идентификатора (сокр. англ. applicability specification), а \textit{id}~--- переменная, которая
может быть представлена любой совокупностью символов.

\noindent
$\diamond$~Немало дискуссий, разворачивающихся вокруг определений,
посвящены выяснению области их применимости. Конечно, тому, кто
предложил определение, не всегда удается увидеть сразу все классы задач, где
определение может работать. Это объяснимо. По мере накопления знаний
первоначально указанные классы задач могут быть изменены.~$\diamond$

Предлагаемые в статье определения имеют следующее описание области
применимости:
\begin{itemize}\sf
\item[\ ] $\{$\textsl{as Sinf}$\}$ 
\item[\ ] Корреспондент~--- информатик.
\item[\ ] Цель~--- создание \textsl{Синф}-методологии.
\item[\ ] Стадия~--- концепция \textsl{Синф}; постановка задач и разработка методов 
решения. 
\end{itemize}

Условимся в определениях, имеющих это описание области применимости, не 
воспроизводить его каждый раз, а только сопровождать указателем 
$\{$\textit{as~Sinf}$\}$.

Если определение системы понятий удовлетворяет требованиям 1--3, то ее s-модель 
рассчитана на применение при изучении задач, класс (или несколько классов) 
которых указан в строке \textit{цель} описания области применимости.

В основание подхода к построению \textit{Синф} положены следующие
утверждения.

А1.~Любые системы знаний представлены моделями (символьными и
несимвольными).

А2.~Интеллектуальная деятельность связана с построением символьных
моделей и манипулированием ими.

А3.~$\square$~\textit{S-моделирование в реализации}~--- это
автоматизированное конструирование символьных моделей произвольных
объектов в s-среде, среди которых~--- s-модели систем понятий и систем
знаний.~$\square$

Из утверждений А1--А3 следует, что в s-сре\-де научную продукцию информатики 
целесообразно представлять в форме s-мо\-де\-лей систем знаний о свойствах и 
закономерностях s-мо\-де\-ли\-ро\-ва\-ния. {\looseness=1

}
\subsection{Методологии s-моделирования: научная продукция
информатики } %3.2

$\square$~\textit{Н\,а\,у\,ч\,н\,а\,я\,\ продукция информатики}~--- методологии 
s-моделирования, представленные в виде s-моделей систем понятий и систем 
знаний.

\textit{Методологии s-моделирования}~--- это комплексы методов
s-(представления, распознавания, преобразования, конструирования,
интерпретации, обмена, сохранения, накопления, поиска, защиты). Они
предназначены для автоматизированного конструирования и применения
s-моделей произвольных объектов в s-среде (включая модель этой
среды).~$\square$

Среди множества моделируемых объектов особое внимание уделяется процессу 
s-мо\-де\-лирования, системам понятий, системам знаний, s-ма\-ши\-нам и 
s-сре\-де. Создаваемые методологии служат научным основанием для разработки 
информационных технологий, программных и аппаратных средств.

Информатика создает методологическое обеспечение для построения s-сре\-ды, 
повышающей продуктивность деятельности людей в различных областях. Особое 
значение имеет повышение продуктивности интеллектуальной деятельности. 
Применение s-мо\-де\-ли\-ро\-ва\-ния вместо физического моделирования приносит 
не только экономию средств. Оно позволяет существенно быстрее, многостороннее и 
глубже изучить и проверить на состоятельность результаты, полученные при 
разработке программ, аппаратных средств, информационных технологий. То же можно 
сказать и относительно результатов в других областях: физике, биологии и~др. 

Итак, методы s-моделирования служат общим для всех наук арсеналом. Методы 
организации коллективного применения этого арсенала~--- также одна из важных 
задач информатики.

\paragraph*{Применение методологий s-моделирования.}Системы
понятий и системы знаний~--- целевые объекты применения методологий
s-моделирования.

\noindent
$\diamond$~Задача s-моделирования становится относящейся к информатике,
если она соответствует определениям ее предмета и основы научного метода:
\begin{itemize}
\item символы и построенные из них s-модели изучаются как элементарные и
составные конструктивные объекты, представленные кодами, рассчитанными
на манипулирование s-машинами;
\item конструктивно доказать существование s-мо\-де\-ли~--- значит построить ее
из конструктивных объектов, реализуемых в s-среде.
\end{itemize}

Другими словами, \textit{задачи s-мо\-де\-ли\-ро\-ва\-ния, изуча\-е\-мые 
информатикой, являются задачами автоматизированного конструирования 
s-мо\-де\-лей в\linebreak s-сре\-де, где люди и s-ма\-ши\-ны взаимодействуют 
между собой}.~$\diamond$

\noindent $\oplus$~На практике: гипермедийные документы, получаемые от 
веб-сер\-ве\-ра и интерпретируемые браузером,~--- это гипермедийные 
s-мо\-де\-ли сообщений; текст, напечатанный на машинке, после сканирования, 
распознавания и интерпретации одним из текстовых процессоров может стать 
текстовой\linebreak 
s-мо\-делью сообщения (так как приобретет свойства 
составного символьного конструктивного объекта, удовлетворяющего требованиям 
манипулирования с помощью программы s-ма\-ши\-ны).~$\oplus$

\subsection{S-модель системы понятий} %3.3

Начнем с пояснения, почему говорим о моделировании системы понятий, а
не одного понятия. Дело в том, что любое определяемое понятие, которое
будем называть \textit{ядром системы понятий}, необходимо связать с другими
ранее определенными понятиями.

$\square$~$\{$\textit{as~Sinf}$\}$ \textit{S}-модель \textit{s-cs} системы
\textit{Cs} понятий~--- это пара~$<$\textit{set[Cs]}, \textit{rul[set[Cs]]}$>$, где
\textit{set[Cs]}~--- множество понятий, а \textit{rul[set[Cs]]}~--- сис\-те\-ма связей,
заданных на \textit{set[Cs]}.~$\square$

\noindent
$\oplus$~В модели \textit{произвольный треугольник} множество понятий
включает стороны \textit{a}, \textit{b}, \textit{c}, углы $\alpha$, $\beta$,
$\gamma$, площадь~$q$, периметр $p$ и~др. Связи: $\alpha +\beta+\gamma
=\pi$; $p=a+b+c$ и~др.~$\oplus$

\paragraph*{S-модель системы метапонятий.}Система понятий может
рассматриваться как некоторое метапонятие. Чтобы определить систему
метапонятий, также необходимо представить ее s-мо\-дель и описание области
применимости. Подобное движение вверх логически не ограничено. То же можно 
заметить и относительно движения вниз: не существует никаких ограничений на 
детализирующие определения понятийных систем путем построения их s-мо\-де\-лей.

\noindent
$\diamond$~S-модель \textit{S-ics} системы понятий
информатики может служить примером s-модели системы
метапонятий.~$\diamond$

\noindent
$\square$~\textit{S-модель предметной области}\;$\approx$\;s-модель
системы метапонятий.~$\square$

\noindent $\diamond$~\textit{S-ics~--- это s-мо\-дель предметной области 
информатики.}~$\diamond$

\subsection{S-модель системы знаний} %3.4

\noindent
$\square$~$\{$\textit{as Sinf}$\}$~\textit{S}-модель \textit{s-kn} системы
знаний \textit{Kn}~--- это триада $<$\textit{s-ca},\ \textit{set[s-l]},\
\textit{set[s-i]}$>$, где \textit{s-ca}~--- модель системы \textit{Sa} метапонятий,
,~ \textit{set[s-l]}~--- множество языков сообщений, а \textit{set[s-i]}~---
множество интерпретаторов на модели \textit{s-ca} сообщений, составленных
на языках из \textit{set[s-l]}.~$\square$

\noindent
$\square$~\textit{Интерпретация сообщения} на модели \textit{s-ca} системы
\textit{Sa} метапонятий~--- это построение выходного сообщения по заданному
входному сообщению.~$\square$

\noindent
$\oplus$~Пример применения s-модели системы знаний дорожного движения
был рассмотрен в~\cite{1il}. В~\cite{7il} приведены примеры применения
s-моделей систем знаний, связанные с автоматизацией
программирования.~$\oplus$

\noindent
$\diamond$~S-модель системы знаний произвольного назначения
рассматривается как \textit{s-развертка} (расширение \textit{p-конкретизации}
задачных конструктивных объектов~\cite{7il}) определенной здесь s-модели
сис\-те\-мы знаний.~$\diamond$

\paragraph*{S-язык сообщений: условия построения и реализации.}

\noindent $\square$~\textit{Необходимым условием построения s-мо\-де\-ли языка 
сообщений} является существование s-мо\-де\-лей системы метапонятий, на которой 
предполагается интерпретировать сообщения, составленные на языке, и базовых 
типов символов, композиции которых предполагается использовать для построения 
системы символов языка. Эти модели играют роль исходных для построения языка. 
\textit{Построение s-мо\-де\-ли языка сообщений}~--- это разработка моделей:
\begin{enumerate}[(1)]
\item композиции базовых типов (разд.~2.1.) символов;
\item системы символов языка, построенной на основе модели композиции
базовых типов символов;
\item системы правил конструирования сообщений с использованием модели
системы символов.~$\square$
\end{enumerate}

\noindent
$\oplus$~Примером относительно легко s-формализуемого языка служит язык
шахматной игры, поскольку эта игра основана на однозначно определенной
системе понятий, в которой семейство связей между понятиями задано
правилами игры.~$\oplus$

\noindent
$\oplus$~Сложность формализации произвольного языка общения связана с
построением s-модели системы метапонятий, на
которой должны интерпретироваться сообщения, являющиеся предложениями
этого языка. Пункты~(1)--(3) выполнить существенно легче.~$\oplus$

Реализация s-языка предполагает существование реализованного в s-среде
интерпретатора сообщений, построенных на этом языке.

\paragraph*{S-интерпретатор сообщений: условие построения.}

\noindent $\square$~Необходимым условием построения s-ин\-тер\-пре\-та\-то\-ра 
сообщений является существование s-мо\-де\-лей входного и выходного языков, а 
также системы метапонятий, на которой должны интерпретироваться сообщения, 
составленные на входном языке. Построение s-мо\-де\-ли интерпретатора~--- это 
разработка моделей:
\begin{enumerate}[(1)]
\item s-распознавания сообщений на принадлежность входному языку;
\item s-интерпретации распознанных сообщений на модели системы
метапонятий;
\item s-представления результата интерпретации в виде сообщения на
выходном языке.~$\square$
\end{enumerate}

\subsection{Пример s-моделирования систем понятий} %3.5

Построение s-мо\-де\-лей систем понятий, пред\-став\-ля\-ющих результаты 
изучения s-мо\-де\-ли\-ро\-ва\-ния, ограничено сформулированными в разд.~2.1 
требованиями конструктивности. В~рассматриваемом примере приведены определения 
понятий \textit{сообщение}, \textit{данные} и \textit{информация}, являющихся 
ядрами соответствующих понятийных систем, име\-ющих не\-пус\-тое пересечение по 
памяти~\cite{7il}.

\noindent
$\square$~$\{$\textit{as~Sinf}$\}$~\textit{Сообщение}~--- это s-модель
произвольного объекта, представленная в форме, удовлетворяющей системе
правил взаимодействия источника с получателем.~$\square$

В множество понятий этой системы кроме \textit{сообщения} (\textit{ядра}
этой системы) входят \textit{источник [сообщения]}, \textit{получатель
[сообщения]}, \textit{передача [сообщения]} и~др., каждое из которых является
ядром в своей сис\-те\-ме. Например, \textit{передача}~--- ядро в сис\-те\-ме,
включающей \textit{метод [передачи]}, \textit{среда [передачи]} и~др. Таким
образом, система понятий с ядром \textit{сообщение} является системой
метапонятий.

\noindent
$\oplus$~Все, что передается в s-среде~--- это сообщения, каждое из которых
имеет отправителя и получателя: веб-страница, текст электронной книги,
исходный текст программы, программа в исполняемом формате и~др.~$\oplus$

\noindent
$\square$~$\{$\textit{as~Sinf}$\}$~\textit{Данные}~--- это s-модель сообщения, получателем которого
является решатель задач\footnote{Задача в s-моделировании имеет более широкий смысл, чем в
математике (см. разд.~2.1).} (человек или программа s-машины).~$\square$

S-модель системы понятий с ядром \textit{данные} связана \textit{отношением 
специализации}~\cite{7il} с s-мо\-делью сис\-те\-мы понятий, где ядром является 
\textit{сообщение}. То есть \textit{данные}~--- это \textit{специализация 
сообщения} по типу \textit{получателя}, которым здесь является \textit{решатель 
задач}.

\noindent
$\square$~$\{$\textit{as~Sinf}$\}$~\textit{Информация}~--- это s-модель
результата интерпретации сообщения на s-модели выбранной системы
понятий.~$\square$

Для извлечения информации необходимо иметь: принятое сообщение;
хранящиеся в памяти s-мо\-де\-ли систем понятий, среди которых~--- необходимая
для интерпретации принятого сообщения; механизмы поиска необходимой
s-модели, интерпретации сообщения, представления результата интерпретации
в виде s-модели и записи этой s-модели в память. S-модель системы понятий с
ядром \textit{информация} имеет непустое пересечение по памяти~\cite{7il} с
s-моделями систем, где ядрами являются \textit{сообщение}, \textit{s-модель
системы понятий}, \textit{интерпретация на s-модели системы понятий}
и~др.

\noindent
$\diamond$~Если следовать предложенным определениям, то:
\begin{itemize}
\item более корректно говорить о передаче сообщений, а не информации или
данных;
\item не понять полученное сообщение~--- то же, что не суметь его
интерпретировать;
\item неправильная интерпретация принятого сообщения (из-за неправильного
выбора системы понятий, на которой необходимо интерпретировать
сообщение, или неправильной работы механизма интерпретации) приведет к
получению некоторой информации. Но она будет ошибочной.~$\diamond$
\end{itemize}

\noindent $\oplus$~Запрос веб-кли\-ен\-та~--- сообщение, интерпретируемое 
веб-сер\-ве\-ром. Веб-стра\-ни\-ца, сформированная для отправки 
веб-кли\-ен\-ту~--- информация, полученная в результате интерпретации на 
s-мо\-де\-ли. Отправленная веб-сер\-ве\-ром веб-стра\-ни\-ца~--- отправленное 
сообщение. Она же принятая веб-кли\-ен\-том~--- принятое сообщение. Результат 
интерпретации принятого веб-кли\-ен\-том сообщения~--- экранное представление 
веб-стра\-ни\-цы, рассчитанное на восприятие человеком.~$\oplus$

Мы не затрагиваем здесь проблемы истинности извлеченной из сообщения
информации, правильности s-модели, механизма интерпретации и~др. Это
отдельные важные задачи информатики.

\paragraph*{Об определениях без указания области применимости.}В
работах К.~Шеннона~\cite{11il, 12il} и А.\,Н.~Колмогорова~\cite{13il}
<<сообщение>> и <<информация>> рассматриваются как составляющие
систем понятий, имеющих другие (по сравнению с рассматриваемой в этой
статье) области применимости. Явно эти области не указаны.

Употребление этих понятий связано там с задачами оценки объема кода
некоторого сообщения или изменения предсказуемости исхода опыта. Так, для
оценки изменения предсказуемости исхода опыта~$b$ в зависимости от исхода
опыта~$a$ применяется разность энтропий
$$
I(a,b)=H(b)-Ha(b)\,,
$$
где $H(b)$ и $Ha(b)$~--- энтропия исхода опыта~$b$ при неизвестном и известном исходе
опыта $a$ соответственно. При этом $I(a,b)$ рассматривается как приращение
предсказуемости исхода опыта~$b$, если известен исход опыта~$a$. Заметим,
что содержание опытов $a$ и~$b$ и типы возможных исходов предполагаются
заранее известными. Предполагается также, что знание исхода опыта~$a$
поможет в предсказании исхода опыта~$b$. Другими словами, все известно,
кроме исхода опыта.

Шеннон в~\cite{11il} определил основную задачу связи как <<точное или
приближенное воспроизведение в некотором месте сообщения, которое было выбрано 
из некоторого множества возможных сообщений и отправлено из другого места>>. Он 
рассматривал эту работу именно как математическую теорию связи. В~предложенной 
им коммуникационной модели определены основные элементы, присущие любой 
коммуникационной системе. Теория связи К.~Шеннона представляет собой 
методологическое обеспечение технологий кодирования, передачи, декодирования и 
приема сообщений.

Шеннон разделяет задачи передачи сообщений и определения их
смыслового значения: <<семантические аспекты связи не имеют отношения к
технической стороне вопроса>>: <<$\ldots$часто сообщения имеют значение,
т.е.\ находятся в соответствии с некоторой системой с определенной
физической или умозрительной сущностью>>.

\noindent
$\diamond$~Обратим внимание на следующее утверждение К.~Шеннона:
<<Если множество возможных сообщений конечно, то число сообщений или
любую монотонную функцию от этого числа можно рассматривать как меру
информации, создаваемой выбором сообщения из этого множества, в
предположении, что все сообщения равновероятны>>.

Однако сам по себе выбор сообщения не создает информацию. Нетрудно
представить, что одно и то же сообщение, может иметь различающиеся
результаты интерпретации. То есть одно и то же сообщение может порождать
различную информацию (точнее, различные экземпляры информации).
Полученные экземпляры зависят от того, какие будут применены модели
систем понятий и методы интерпретации на выбранных моделях.~$\diamond$

В работах К.~Шеннона~\cite{11il} и А.\,Н.~Колмогорова~\cite{13il} говорится о 
<<количестве информации>> и рассматриваются задачи, связанные с этим понятием. 
Понятие <<информация>> там не определено. В~этих и других работах этих авторов 
задача извлечения информации путем интерпретации сообщений на моделях систем 
понятий не изучалась\footnote{Насколько нам известно, эта задача не изучалась и 
в других работах, связанных с понятием <<информация>>.}.

\section{Реализация и применение: общая характеристика}

При том, что успех реализации в основном определяется тем, как построена
s-модель системы метапонятий информатики, он не может быть достигнут, если 
неудачно построены и реализованы языки и интерпретаторы системы. Гипермедийные 
s-модели систем символов для построения входных и выходных языков системы 
\textit{Синф}, включающие тип <<интерактивное видео>>, обладают рядом важных 
для реализации \textit{Синф} достоинств. Для науки и образования особого 
внимания заслуживают интерактивные видеоклипы, содержащие схематически 
изображенные движущиеся изображения, сопровождаемые аудио-, текстовыми и 
графическими пояснениями. Поэтому наиболее целесообразной выглядит реализация 
\textit{Синф} как распределенной интерактивной гипермедийной системы знаний.

\subsection{Применение в научной деятельности } %4.1

Существование \textit{Синф}, ее обновление и применение имеет смысл
связать с деятельностью виртуальной лаборатории информатики. Деятельность
по созданию, обновлению и применению \textit{Синф} могла бы содействовать
существенным изменениям в технологиях представления и апробации научных
результатов.

Полагаем, что по правилам виртуальной лаборатории информатики научные
результаты сначала могли бы выкладываться для обсуждения на семинарах,
каждый из которых объединяет специалистов в определенном разделе
информатики. Результаты, получившие признание на семинарах, могли бы
быть оформлены в виде проектов обновлений \textit{Синф}. Такие проекты
имело бы смысл представлять для обсуждения на конференциях, каждая из
которых объединяет специалистов нескольких смежных разделов. Результату,
одобренному конференцией, целесообразно было бы присваивать статус
рекомендуемого (для реализации) обновления \textit{Синф}.

Таким образом, процесс апробации результатов был бы связан с построением
\textit{Синф}. Научный результат, изменивший состояние \textit{Синф}, целесообразно
считать \textit{Синф-сертифицированным}, а справку о его авторе (или
авторах) заносить в соответствую базу авторов \textit{Синф}. Легко
представить, как это изменило бы научный процесс.

\subsection{Применение в проектировании } %4.2

Уже сегодня, говоря о проектировании, прежде всего имеют в виду
автоматизированное проектирование, представленное САПРами различного
назначения, уровня сложности, совершенства и доступности. По существу
мозговым центром любой САПР является система знаний, реализованная в той
или иной форме. От того, насколько удачно построена s-модель системы
знаний, положенная в основу реализации этой системы, зависит
продуктивность проектирования. Пример \textit{Синф} и здесь был бы
небесполезен.

\noindent
$\oplus$~В частности, предложенный подход к построению
s-моделей системы метапонятий, языков и интерпретаторов целесообразно
использовать в проектировании систем машинного перевода, \textit{особое
внимание уделяя s-модели системы метапонятий}.~$\oplus$

\paragraph*{Информатика и инфотехника.}$\square$~$\{$\textit{as~Sinf}$\}$~\textit{Информационная технология}~---
это комплекс методов, предназначенный для решения некоторого класса задач
s-моделирования.~$\square$

Каждая информационная технология может иметь
различные реализации. Разработка и исследование информационных
технологий относятся к информатике, а все, что связано с их реализацией~--- к
\textit{инфотехнике}.

\noindent
$\square$~$\{$\textit{as~Sinf}$\}$~\textit{Предметом инфотехники} является
построение s-среды как основы для поддержки деятельности в различных
областях.~$\square$

Научным основанием для решения задач инфотехники
служат результаты информатики.

\subsection{Применение в образовании} %4.3

Вполне естественно, что \textit{Синф} могла бы стать профессионально изготовленным образцом для
энциклопедий информатики разного уровня сложности\footnote{Это могло бы способствовать
школьному и вузовскому образованию.} (от <<для начинающих>> до <<для
информатиков>>\footnote{Поскольку знания, не относящиеся к своему разделу, конечно же,
сначала лучше получить в адаптированной форме.}) и систем знаний дистанционного образования.

\section{Заключение}

\noindent
\begin{enumerate}[1.]
\item Символьная модель системы знаний информатики, основы концепции
построения и применения которой изложены в статье, рас\-смат\-ри\-ва\-ет\-ся 
как средство формирования по\-ня\-тий\-но\-го аппарата информатики 
коллективными\linebreak
 усилиями исследователей. Методология со\-зда\-ния этой 
системы изучается как составля\-ющая методологического обеспечения 
раз\-рабо\-ток технологий автоматизации научных исследова\-ний, образовательных 
процессов и проектирования. {\looseness=1

}
\item Построение системы знаний информатики, ее обновление и применение
целесообразно связать с деятельностью виртуальной лаборатории
информатики, что важно как для формирования понятийного аппарата
информатики, так и для реализации полезных изменений в технологиях
представления и апробации научных результатов.
\item Результаты исследований, полученные при создании методологии, используются в
научно-исследовательском и образовательном процессах\footnote{Студентам МИРЭА, 
обучающимся на базовой кафедре проблем информатики ИПИ РАН, с 2008~г.\ будет 
читаться профильный курс <<Символьное моделирование в информатике>>, в основу 
которого положены обсуждаемые в этой статье результаты.}, включая процесс 
на\-уч\-но-ме\-то\-до\-ло\-ги\-че\-ской поддержки создания Большой Российской 
Энциклопедии (в части, отнесенной к разделу <<Информатика>>)\footnote{Эту 
работу ИПИ РАН выполняет совместно с редакцией <<Техника>> БРЭ (формирование 
словника раздела <<Информатика>>, научное консультирование, написание и 
рецензирование статей).}.
\end{enumerate}

{\small {\baselineskip=10.59pt 
\subsection*{Приложение}

Схематически\footnote{Без учета хронологической последовательности их
появления.} представленные здесь результаты сгруппированы по типам символов, 
играющих роли основных в своих группах. Этот, конечно же, неисчерпывающий 
перечень не содержит комментариев и является своеобразной схемой 
очерка\footnote{В обычной форме такой очерк потребовал бы как минимум отдельной 
статьи.}, посвященного становлению и развитию символьного моделирования, в 
котором все более значительную роль играет символьное моделирование в 
че\-ло\-ве\-ко-ав\-то\-мат\-ной среде, названное нами 
\textit{s-мо\-де\-ли\-ро\-ва\-ни\-ем}.

\noindent
$\triangleright$[Dn:\;$\rightarrow \approx$\;<<связанный с предыдущим>>]

\noindent
 $\{S$~\textit{модели}

\medskip
{\bfseries\textit{Аудио:}}
\smallskip

\begin{itemize}
\item $\mbox{звуки}\rightarrow\mbox{речь}$;
\item языки звуковых сообщений: языки общения,
про\-фес\-сио\-наль\-но-ори\-ен\-ти\-ро\-ван\-ные расширения языков общения;
\item символьные модели звуковых сообщений: анало\-го\-вое
кодирование\;$\rightarrow$\;запись и воспроизведение, сохранение и накопление 
звуковых сообщений (грамзапись, запись на магнитных 
носителях\;$\rightarrow$\;фонотеки); удаленная передача и прием 
(те\-ле\-фо\-ния, радиосвязь, радиовещание); 
\item \textit{звуковых сообщений}: цифровое 
ко\-ди\-ро\-ва\-ние\;$\rightarrow$\;циф\-ро\-вые технологии записи (на 
магнитных и оптических носителях), редактирования и воспроизведения, удаленной 
передачи и приема, синтеза и распознавания, сохранения и накопления звуковых 
со\-об\-ще\-ний\;$\rightarrow$\;циф\-ро\-вые технологии мобильной и 
стационарной связи, спутниковой радиосвязи, Интернет-телефонии, радиовещания. 
{%\looseness=1

}
\end{itemize}

\medskip
{\bfseries\textit{Графический}}\;
$\rightarrow$\;{\bfseries\textit{текстовый}}\;$\rightarrow$\;{\bfseries\textit{числовой}}:
\smallskip
\begin{itemize}
\item $\mbox{рисунки}\rightarrow\mbox{схемы}$;
\item графические символьные модели звуковых сообщений: буквы,
иероглифы, ноты (как элементарные графические символы для по\-стро\-ения
текстов и нотных записей)\;$\rightarrow$ т\,е\,к\,с\,т\,о\,в\,ы\,е\,\ 
символьные модели языков сообщений\footnote{Включая графический язык записи 
музыкальных композиций (нотной записи).} $\rightarrow$ 
о\,с\,н\,о\,в\,а\,н\,и\,е\,\ п\,и\,с\,ь\,м\,е\,н\,н\,о\,с\,т\,и;
\item символьные модели системы понятий, включающей
ч\,и\,с\,л\,о\;$\rightarrow$ позиционные системы счисления;
\item символьные модели систем понятий, включающих з\,а\,д\,а\,ч\,у\,\ и\,\,\
а\,л\,г\,о\,р\,и\,т\,м:\,\ формулировка задачи\;$\rightarrow$ существование
решения, единственность~$\rightarrow$ методы решения~$\rightarrow$
алгоритмы~$\rightarrow$ оценки трудоемкости реализации алгоритмов~$\rightarrow$ 
конструктивные доказательства существования алгоритмов;

\textit{цифрового кодирования символов любого типа}\;$\rightarrow$
предпосылка изобретения автоматов, манипулирующих кодами символов и
построенных из них s-моделей\;$\rightarrow$ цифровые автоматы s-среды;
\item \textit{алгоритма}\;$\rightarrow$ программа;
\item \textit{s-машины}\;$\rightarrow$ компьютеры и компьютерные
устройства (смартфоны, цифровые фотокамеры и~др.) \;$\rightarrow$
формирование s-среды;
\item \textit{аппаратно реализуемых конструктивных элементов
(микропроцессора, памяти и~др.)\ и систем правил конструирования s-машин}: 
элементные базы\;$\rightarrow$ комплектующие\;$\rightarrow$ автоматизированное 
конструирование аппаратных средств;
\item \textit{архитектур s-машин}\;$\rightarrow$ персональные компьютеры,
суперкомпьютеры и~др.;
\item \textit{языков сообщений в s-среде}: языки программирования, запросов,
спецификации;
\item \textit{процессов s-моделирования} различного типа (программирования,
специфицирования задач и~др.)\;$\rightarrow$ средства автоматизации
программирования: языки, трансляторы (ассемблеры, компиляторы,
интерпретаторы), библиотеки программ, инструментальные системы
программирования\;$\rightarrow$ s-автоматизация проектирования;
\item \textit{систем понятий, включающих д\,а\,н\,н\,ы\,е}\;$\rightarrow$
базы данных, системы управления базами данных;
\item \textit{процессов взаимодействия человека с окружением и решения им
задач различного назначения в s-среде}\;$\rightarrow$ искусственный
интеллект;
\item \textit{компьютерных сетей}: сетевые архитектуры, локальные и
региональные сети, Интернет\;$\rightarrow$ службы, работающие на базе
Интернета (электронная почта, Веб, поиск и~др.);
\item \textit{методов решения задач объединениями s-машин
s-сре\-ды}\;$\rightarrow$ Грид;
\item \textit{сервис-ориентированных архитектур~(СОА)} $\rightarrow$
конструирование служб (сервисов)~$\diamond$~н\,о\,в\,ы\,й\,\ э\,т\,а\,п\,\ в
р\,а\,з\,в\,и\,т\,и\,\ и s-среды~$\diamond$;
\item \textit{конструирования изображений в s-среде (где пиксель~---
элементарный конструктивный объ\-ект)}\;$\rightarrow$ цифровая фотография
и редактирование изображений;
\item \textit{конструирования документов в s-среде}\;$\rightarrow$
элек\-т\-рон\-ный документооборот;
\item \textit{механизма перемещения между документами и их
составляющими}\;$\rightarrow$ гипертекст;
\item \textit{автоматизированного проектирования в элек\-т\-рон\-ной
промышленности, машиностроении и~др. областях}\;$\rightarrow$ САПР как
инструмент проектирования аппаратных средств s-среды, изделий
электроники, машиностроения и~др.;
\item \textit{распознавания изображений в s-среде}: распознавание текстов,
биометрическая идентификация.
\end{itemize}

\medskip
{\bfseries\textit{Видео}:}
\smallskip

\begin{itemize}
\item символьные модели процессов записи и воспроизведения движущихся
изображений (видеосъемка, кино, телевидение);
\item \textit{циф\-ро\-во\-го кодирования движущихся изображений, записи и
воспроизведения цифрового видео}\;$\rightarrow$ циф\-ро\-вая видеосъемка и
монтаж, цифровые технологии видеоконференций, видеомобильной связи,
телевидения, интерактивного видео и телевидения.
\end{itemize}

\medskip
\textit{Механический:}
\smallskip

\begin{itemize}
\item жесты\;$\rightarrow$ языки жестовых сообщений;
\item \textit{составляющей гипермедиа}\;$\rightarrow$ вибросигнализация в
мобильной связи, игровые приложения.
\end{itemize}

\medskip
{\bfseries\textit{Композиции базовых типов символов}:}
\smallskip

\noindent \textit{различных объектов, для представления которых используются 
текстовые, гипертекстовые, графические, аудио-, видео- и механический типы 
символов}: мультимедиа\;$\rightarrow$ гипермедиа\;$\rightarrow$ конструирование 
веб-сер\-ви\-сов\;$\rightarrow$ СОА, гипермедийные системы знаний, имеющие 
сер\-вис-ори\-ен\-ти\-ро\-ван\-ную ар\-хи\-тек\-туру. {\looseness=1

}
%\end{itemize}
\noindent
 S$\}\triangleleft$

} }

{\small\frenchspacing {\baselineskip=11pt 
\addcontentsline{toc}{section}{Литература}
\begin{thebibliography}{99}
\bibitem{1il}
\Au{Ильин В.\,Д., Соколов~И.\,А.}
Информация как результат интерпретации
сообщений на символьных моделях систем понятий~// Информационные
технологии и вычислительные системы, 2006. №\,4. С.~74--82.

\bibitem{2il}
\Au{Артемов~С.\,Н.}
Формализации метод~// Математическая
энциклопедия, 1985. Т.~5. С.~635.

\bibitem{3il}
\Au{Гришин~В.\,Н.}
Формальная система~// Математическая
энциклопедия, 1985. Т.~5. С~ 639.

\bibitem{4il}
\Au{Ильин~А.\,В., Ильин~В.\,Д.} Интерактивный преобразователь ресурсов с 
изменяемыми правилами поведения~// Информационные технологии и вычислительные 
системы, 2004. №\,2, С.~67--77.

\bibitem{5il}
\Au{Ильин~В.\,Д.}
Гипертекст. Большая Российская Энциклопедия, 2007.
Т.~7.

\bibitem{6il}
\Au{Соколов~И.\,А.}
Данные в информатике. Большая Российская
Энциклопедия, 2007. Т.~7.

\bibitem{7il}
\Au{Ильин~В.\,Д.}
Система порождения программ. М.: Наука, 1989.

\bibitem{8il}
{\sf http://www.w3.org/Submission/2007/01/}

\bibitem{9il}
{\sf http://www.w3.org/Submission/2007/SUBM-sml-20070321/.}

\bibitem{10il}
{\sf http://www.w3.org/Submission/2007/SUBM-sml-if-20070321/}.

\bibitem{11il}
\Au{Shannon~C.\,E.}
A mathematical theory of communication~// Bell System
Technical J., 1948. July and October, Vol.~27. P.~379--423 and 623--656.
{\sf http://cm.bell-labs.com/cm/ms/what/shannonday/shannon1948.pdf}

\bibitem{12il}
\Au{Шеннон~К.}
Работы по теории информации и кибернетике. Пер. с англ.
под ред. Р.\,Л.~Добрушина и О.\,Б.~Лупанова. С предисловием А.\,Н.~Колмогорова.
М.: Иностранная литература, 1963.

\bibitem{13il}
\Au{Колмогоров~А.\,Н.}
Три подхода к определению понятия <<Количество
информации>>~// Проблемы передачи информации, 1965.
Т.~I. Вып.~1. С.~3--11.
\end{thebibliography}

} } \label{end\stat}
\end{multicols}