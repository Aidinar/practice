\def\Bar{\hat}
\def\stat{pechin}
\def\tit{МНОГОЛИНЕЙНАЯ СИСТЕМА МАССОВОГО
ОБСЛУЖИВАНИЯ С~КОНЕЧНЫМ НАКОПИТЕЛЕМ
 И НЕНАДЕЖНЫМИ ПРИБОРАМИ$^*$}
\def\titkol{Многолинейная система массового
обслуживания с конечным накопителем
 и ненадежными приборами}
\def\autkol{А.\,В.~Печинкин, И.\,А.~Соколов, В.\,В.~Чаплыгин}
\def\aut{А.\,В.~Печинкин$^1$, И.\,А.~Соколов$^2$, В.\,В.~Чаплыгин$^3$}

\titel{\tit}{\aut}{\autkol}{\titkol}

{\renewcommand{\thefootnote}{\fnsymbol{footnote}} \footnotetext[1]{Работа 
выполнена при поддержке РФФИ, гранты 05-07-90103 и 06-07-89056.} 
\renewcommand{\thefootnote}{\arabic{footnote}}}
 \footnotetext[1]{Институт 
проблем информатики РАН, apechinkin@ipiran.ru} \footnotetext[2]{Институт 
проблем информатики РАН, isokolov@ipiran.ru} \footnotetext[3]{Институт проблем 
информатики РАН, vchaplygin@ipiran.ru}

\Abst{Рассматривается многолинейная система массового обслуживания
с полумарковским входящим потоком,
обслуживанием фазового типа,
накопителем конечной емкости и
ненадежными приборами, отказывающими независимо друг от друга и
от всего процесса функционирования системы.
%
Получены математические соотношения для расчета основных
стационарных показателей функционирования сис\-те\-мы при некоторых
вариантах процесса отказов--восстановлений приборов.}

\KW{система массового обслуживания;
ненадежные приборы}

\vskip 36pt plus 9pt minus 6pt

\begin{multicols}{2}


\label{st\stat}


\section{Описание системы}

В последние несколько лет стремительное развитие и активное внедрение особого 
класса инфотелекоммуникационных систем, главной отличительной чертой которых 
является обработка потоков заданий путем их разделения и передачи на выполнение 
нескольким исполнителям или группам исполнителей, привлекли внимание к 
математическим моделям, позволяющим получать характеристики подобных систем и 
их отдельных узлов, учитывая особенности их функционирования. Для систем этого 
класса, включающего, например, системы распределенных вычислений, разнообразные 
Грид-системы и~др., необходимо учитывать различного рода перерывы в работе, 
сбои и неисправности, возникающие как на этапе передачи, так и на этапе 
выполнения заданий отдельным устройством или кластером устройств. Поэтому 
важным направлением в современной теории массового обслуживания является 
изучение систем массового обслуживания (СМО) с ненадежными приборами. 

\thispagestyle{headings}

 Несмотря на то, что и ранее публиковались работы, 
посвященные СМО с различными видами сбоев и задержек при обслуживании заявок 
(см., например, [1--8]), появление новых эффективных методов расчета СМО с 
немарковскими входящими потоками позволяет рассчитывать характеристики этих 
систем, дополнительно рассматривая факторы, препятствующие их безотказному 
функционированию. Такая возможность обусловлена спецификой появившихся методов, 
поскольку учет в модели негативных воздействий на систему (например с помощью 
введения отрицательных заявок или дополнительных состояний прибора) хотя и 
приводит к усложнению математических построений, но не меняет конструкции 
самого метода, позволяя оставаться в русле идей, в нем заложенных. В частности, 
в~\cite{PSCh06} на основе результатов, полученных в работе~\cite{PCh03}, 
найдены соотношения, позволяющие вычислять основные стационарные показатели 
функционирования многолинейной СМО $SM/MSP/n/\infty$ с полумарковским входящим 
потоком, марковским процессом обслуживания заявок, ненадежными приборами и 
накопителем бесконечной емкости при различных вариантах процессов отказов и 
восстановлений приборов.

В настоящей работе рассматривается СМО $SM/MSP/n/r$, аналогичная
исследованной в~\cite{PSCh06}, но с накопителем
конечной емкости. Применяемые здесь методы также основываются на результатах
работы~\cite{PCh03}.



Рассмотрим многолинейную СМО с полумарковским входящим потоком заявок, распределением времени
обслуживания фазового типа, накопителем конечной емкости $r$ и
работающими независимо друг от друга ненадежными приборами и
опишем общие принципы ее функционирования.
Поскольку для систем с емкостью накопителя, равной нулю или одной
заявке, расчетные формулы несколько отличаются от общего случая,
далее будем предполагать, что $r\geq 2$.

В системе имеется $n$ идентичных приборов, которые обслуживают
поступающие на них однотипные заявки.
Будем называть прибор занятым, если на нем находится заявка, и
свободным в противном случае.
Каждый из $n$ приборов может находиться либо в исправном, либо в
неисправном состоянии.

Состояние прибора будем считать исправным, если на приборе находится заявка и 
прибор занят ее обслуживанием или если прибор свободен, готов принять заявку и 
немедленно начать ее обслуживание. 

Состояние прибора будем считать неисправным, если на приборе находится заявка, 
но прибор ее не обслуживает, или если прибор свободен, но не может немедленно 
начать обслуживание заявки, если таковая на него поступит. Если при отказе 
прибора (переходе прибора из исправного состояния в неисправное) на нем 
находится заявка, то она остается на приборе до момента восстановления 
(перехода прибора из неисправного состояния в исправное) и затем, в зависимости 
от варианта функционирования системы, либо дообслуживается, либо обслуживается 
заново.

Если в некоторый момент времени на обслуживание поступает очередная заявка, но 
все приборы заняты, то эта заявка попадает в накопитель, становясь в очередь на 
обслуживание. Заявки из очереди на обслуживание выбираются в порядке их 
поступления в накопитель. Если поступающая заявка при поступлении застает $r$ 
заявок в накопителе, то она сразу же, не обслуживаясь, покидает систему 
(теряется). Далее для сокращения записи положим $R=n+r$.

Далее всюду, за исключением раздела~7, будет рассматриваться экспоненциальная
модель процессов отказов--восстановлений приборов, при которой отказ
прибора, где обслуживается заявка, происходит с интенсивностью
$\alpha$, а окончание ремонта этого прибора --- с интенсивностью $\beta$.
В некоторых вариантах функционирования системы с интенсивностями $\alpha^*$
и $\beta^*$ может также отказывать и восстанавливаться свободный прибор.

Полумарковский входящий поток заявок определяется полумарковским процессом с 
конечным множеством состояний $\{1,2,\ldots,I\}$, $1\le I<\infty$. В каждый 
момент изменения состояния полумарковского процесса в систему поступает заявка. 
Вероятность того, что полумарковский процесс за время меньше $x$ перейдет из 
состояни $i$ сразу в состояние $j$, $i,j=\overline{1,I}$, равна $A_{i,j}(x)$. 
Обозначим через $A(x)$ матрицу из элементов $A_{i,j}(x)$, через 
$A=A(\infty)$~--- матрицу переходных вероятностей вложенной цепи Маркова 
полумарковского процесса и через $\vec\pi_a$~--- вектор-строку стационарных 
вероятностей вложенной цепи Маркова. Вектор $\vec\pi_a$ можно определить из 
системы уравнений равновесия (СУР)

\noindent
$$
\vec\pi_a A = \vec\pi_a
$$
с условием нормировки
$$
\vec\pi_a \vec1 = 1
$$
(здесь и далее через $\vec1$ будем обозначать вектор-столбец из единиц,
а через $E$ --- единичную матрицу, размерность и порядок которых
определяются из контекста).
Среднее время между поступлениями заявок в стационарном режиме
функционирования системы можно записать в виде
$$
\overline a = \vec\pi_a \int\limits_0^\infty x\, dA(x)\, \vec1\,.
$$
Будем предполагать, что вложенная цепь Маркова полумарковского процесса
является неприводимой и непериодической,
а $\overline a < \infty$.
Кроме того, там, где речь пойдет о стационарном распределении
по времени, будем считать, что времена генерации заявок не могут
принимать только значения $jt$, где $t$~--- положительное число,
а $j=0,1,\ldots\,.$


Распределение фазового типа (PH-рас\-пре\-де\-ле\-ние) времени обслуживания
заявки можно трактовать следующим образом. Исправный прибор, обслуживающий 
заявку, может находиться на одной из $J$ фаз обслуживания, $1\le J <\infty$. 
Поступающая на исправный свободный прибор заявка с вероятностью $h_i$, 
$i=\overline{1,J}$, начинает обслуживаться с фазы~$i$. Если в некоторый момент 
времени прибор обслуживает заявку на фазе $i$, $i=\overline{1,J}$, то за 
<<малое>>\ время $\Delta$ с вероятностью $h_{i,j}\Delta+o(\Delta)$, 
$j=\overline{1,J}$, $j\ne i$, фаза обслуживания меняется на $j$-ю и с 
вероятностью $h_i^* \Delta+o(\Delta)$, где $$ h_i^*=- \sum\limits_{j=1}^J 
h_{i,j}\,,
$$
обслуживание заявки заканчивается и она покидает систему.
Вектор-строку с координатами $h_i$ будем обозначать через
$\vec h$, а матрицу с элементами $h_{i,j}$ --- через $H$.
Тогда функцию распределения фазового типа времени обслуживания
заявки можно записать в виде
$$
H(x) = 1-\vec h\, e^{Hx} \vec1\,.
$$


Цель настоящей работы заключается в на\-хож\-де\-нии программно-реализуемых
математических соотношений для расчета основных стационарных показателей
функционирования описанной выше СМО с ненадежными приборами.
Не оговаривая это особо, всюду в дальнейшем будем считать, что отказы
приборов происходят {\it независимо друг от друга}.

Будут рассмотрены следующие варианты функционирования системы при
экспоненциальном процессе отказов--восстановлений приборов:
\begin{itemize}
\item  обслуживание заявки заново, свободные приборы находятся только в
исправном состоянии;
\item дообслуживание заявки, свободные приборы находятся только в
исправном состоянии;
\item обслуживание заявки заново, свободные приборы могут отказывать,
заявки поступают на все приборы, вне зависимости от того, в исправном
или неисправном состоянии они находятся;
\item  обслуживание заявки заново, свободные приборы могут отказывать,
заявки поступают только на исправные приборы.
\end{itemize}

Кроме того, в разделе~7 рассмотрен вариант сис\-те\-мы с отказами только
работающих приборов и марковским процессом отказов--восстановлений
приборов.

\section{Общая модель}

В этом разделе мы рассмотрим общую базовую модель СМО с накопителем
конечной емкости, на которой будут основываться наши дальнейшие
выкладки.

Общая модель представляет собой многолинейную СМО с накопителем конечной
емкости, надежными приборами, полумарковским входящим потоком
(процессом генерации) заявок, описанным в предыдущем разделе, и марковским
процессом обслуживания заявок, определяемым следующим образом.

Если в системе находится $k$  заявок (далее
будем говорить также, что процесс обслуживания находится на
слое $k$), $k=\overline{0,R}$,
то процесс обслуживания может находиться в одном
из $l_k$  состояний (фаз обслуживания), $l_k<\infty$.
Далее, если в некоторый момент в системе находится $k$
 заявок, $k = \overline{1,R}$, и фаза обслуживания равна $i$,
$i=\overline{1,l_k}$, то
за <<малое>> время $\Delta$ с вероятностью
$\lambda^{(k)}_{i,j}\Delta + o(\Delta)$ фаза изменится на $j$,
$j=\overline{1,l_k}$, $j\ne i$, и все заявки будут продолжать
обслуживаться,
а с вероятностью $n^{(k)}_{i,j}\Delta + o(\Delta)$ фаза изменится
на $j$, $j=\overline{1,l_{k-1}}$, но обслуживание одной из
заявок закончится и она покинет систему. Матрицы из элементов
$\lambda^{(k)}_{i,j}$ и $n^{(k)}_{i,j}$ будем обозначать через
$\Lambda_k$ и $N_k$, $k =\overline{1,R}$.
Если же в системе отсутствуют заявки, то за <<малое>> время $\Delta$
с вероятностью $\lambda^{(0)}_{i,j}\Delta + o(\Delta)$ фаза изменится
с $i$ на $j$, $i,j=\overline{1,l_0}$, $j\ne i$, естественно,
без окончания обслуживания заявки.
Матрицу из элементов $\lambda^{(0)}_{i,j}$ будем обозначать через
$\Lambda_0$.

Кроме того, будем предполагать, что $l_k=l$ при
$k =\overline{n,R}$, матрицы $\Lambda_k = \Lambda$ совпадают при
$k =\overline{n,R}$, а матрицы $N_k = N$ совпадают при
$k =\overline{n+1,R}$.

Предположим, что матрица $\Lambda^* = \Lambda+N$ является неразложимой,
а матрица $N$~--- ненулевой.
Более того, будем предполагать, что при исходных
параметрах рассматриваемой СМО введенная далее вложенная цепь
Маркова будет неприводимой.

Наконец, при $k=\overline{0,n-1}$ будем предполагать, что если в
момент поступления очередной заявки в системе имеется $k$ других
заявок и фаза обслуживания равна $i$, $i=\overline{1,l_k}$, то
после поступления новой заявки фаза обслуживания с вероятностью
$\omega^{(k)}_{i,j}$ изменится на $j$, $j=\overline{1,l_{k+1}}$.
Соответственно, матрицу из элементов $\omega^{(k)}_{i,j}$ будем
обозначать через~$\Omega_k$.

Поскольку описанная выше модель отличается от модели, рассмотренной
в~\cite{PSCh06}, только емкостью накопителя,
то здесь мы остановимся лишь на тех изменениях, которые нужно
внести в результаты из~\cite{PSCh06}.

Инфинитезимальная матрица $L$ и матрица $B(t)$ из вероятностей
переходов за время $t$ марковского процесса обслуживания заявок
отличаются от аналогичных матриц из~\cite{PSCh06}
только тем, что они получены <<урезанием>> последних на $R$-м уровне:
\end{multicols}
\noindent
$$
L=
\begin{pmatrix}
\Lambda_0 &   0       &  0        & \ldots & 0             & 0       &  0
    & \ldots & 0       & 0 \\
N_1       & \Lambda_1 &  0        & \ldots & 0             & 0       &  0
    & \ldots & 0       & 0 \\
0         & N_2       & \Lambda_2 & \ldots & 0             & 0       &  0
    & \ldots & 0       & 0 \\
\vdots    & \vdots    & \vdots    & \ddots & \vdots        & \vdots  &
\vdots
    & \ldots & 0       & 0 \\
0         &  0        &  0        & \ldots & \Lambda_{n-1} & 0       &  0
    & \ldots & 0       & 0 \\
0         &  0        &  0        & \ldots & N_n           & \Lambda &  0
    & \ldots & 0       & 0 \\
0         &  0        &  0        & \ldots & 0             & N       &
\Lambda
    & \ldots & 0       & 0 \\
0         &  0        &  0        & \ldots & 0             & 0       & N
    & \ldots & 0       & 0 \\
\vdots    & \vdots    & \vdots    & \ddots & \vdots        & \vdots  &
\vdots
    & \ddots & 0       & 0 \\
0         & 0         & 0         & \ldots & 0             & 0       & 0
    & \ldots & \Lambda & 0 \\
0         & 0         & 0         & \ldots & 0             & 0       & 0
    & \ldots & N       & \Lambda \\
  \end{pmatrix},
$$
$$
B(t)=
  \begin{pmatrix}
B_{0,0}(t)   &  0           &  0          & \ldots &  0            &  0
     &  0        & \ldots    & 0   \\
B_{1,0}(t)   & B_{1,1}(t)   &  0          & \ldots &  0            &  0
     &  0        & \ldots    & 0   \\
B_{2,0}(t)   & B_{2,1}(t)   &  B_{2,2}(t) & \ldots &  0            &  0
     &  0        & \ldots    & 0   \\
\vdots      & \vdots      & \vdots      & \ddots & \vdots        & \vdots
     & \vdots    & \ldots    & 0   \\
B_{n-1,0}(t)& B_{n-1,1}(t)& B_{n-1,2}(t)& \ldots & B_{n-1,n-1}(t)&  0
     &  0        & \ldots    & 0   \\
B_{n,0}(t)   & B_{n,1}(t)   & B_{n,2}(t)   & \ldots & B_{n,n-1}(t)  & B_0(t)
     & 0         & \ldots    & 0   \\
B_{n+1,0}(t)& B_{n+1,1}(t)& B_{n+1,2}(t)& \ldots & B_{n+1,n-1}(t)& B_1(t)
     & B_0(t)    & \ldots    & 0   \\
B_{n+2,0}(t)& B_{n+2,1}(t)& B_{n+2,2}(t)& \ldots & B_{n+2,n-1}(t)& B_2(t)
     & B_1(t)    & \ldots & 0   \\
\vdots      & \vdots      & \vdots      & \vdots & \vdots        & \vdots
     & \vdots    & \ddots & 0 \\
B_{R,0}(t)& B_{R,1}(t)& B_{R,2}(t)& \ldots & B_{R,n-1}(t)& B_r(t)
     & B_{r-1}(t)& \ldots & B_0(t)  \\
  \end{pmatrix}.
$$
\smallskip
\begin{multicols}{2}

Матрица переходных вероятностей вложенной цепи Маркова, порожденной количеством 
заявок и фазами процессов генерации и обслуживания заявок непосредственно после 
моментов поступления заявок в систему, будет иметь вид: 
{\looseness=1

}
\end{multicols}

\smallskip
\noindent
$$
B=
  \begin{pmatrix}
B_{1,1}      & B_{1,2}      &  0          & \ldots &  0         &  0
     &  0      & \ldots & 0 & 0    \\
B_{2,1}      & B_{2,2}      & B_{2,3}      & \ldots &  0         &  0
     &  0      & \ldots & 0 & 0    \\
\vdots      & \vdots      & \vdots      & \ddots & \vdots     & \vdots
     & \vdots  & \ldots & 0 & 0 \\
B_{n-1,1}   & B_{n-1,2}   & B_{n-1,3}   & \ldots & B_{n-1,n}  &  0
     &  0      & \ldots & 0 & 0    \\
B_{n,1}      & B_{n,2}      & B_{n,3}      & \ldots & B_{n,n}     & B_0
     &  0      & \ldots & 0 & 0  \\
B_{n+1,1}   & B_{n+1,2}   & B_{n+1,3}   & \ldots & B_{n+1,n}  & B_1
     & B_0     & \ldots & 0 & 0  \\
B_{n+2,1}   & B_{n+2,2}   & B_{n+2,3}   & \ldots & B_{n+2,n}  & B_2
     & B_1     & \ldots & 0 & 0 \\
\vdots      & \vdots      & \vdots      & \vdots & \vdots      & \vdots
     & \vdots  & \ddots & \vdots &\vdots  \\
B_{R-1,1} & B_{R-1,2} & B_{R-1,3} & \ldots & B_{R-1,n} & B_{r-1}
     & B_{r-2} & \ldots & B_{1}  & B_{0} \\
B_{R,1}   & B_{R,2}   & B_{R,3}   & \ldots & B_{R,n}   & B_r
     & B_{r-1} & \ldots & B_{2}  & B_{1}+B_{0} \\
\end{pmatrix}.
$$
\smallskip
\begin{multicols}{2}

В приведенных выше формулах матрицы $B_{k,s}(t)$, $B_k(t)$,
$B_k$ и $B_{k,s}$ определены теми же самыми формулами, что
и в~\cite{PSCh06}.

Обозначим через $\vec p_k^{\,*}$, $k=\overline{1,R}$, вектор-строку,
координатами которой
$p_{k,m}^{\,*}$, где $m=\overline{1,I l_k}$ при $k=\overline{1,n-1}$ и
$m=\overline{1,Il}$ при $k=\overline{n,R}$, являются стационарные
вероятности
по вложенной
цепи Маркова того, что в системе находится $k$ заявок, а фазы процессов
генерации и обслуживания заявок равны $i$ и $j$ соответственно.
Здесь положено
$m=(\vec e^{\,\prime}_i\otimes \vec e^{\,\prime\prime}_j) \vec u_k$, где
$\otimes$~--- символ кронекерова произведения матриц,
$\vec u_k=(1,2,\ldots,Il_k)^T$~--- вектор-столбец размерности $Il_k$ или
$Il$, а $\vec e^{\,\prime}_i$ и $\vec e^{\,\prime\prime}_j$~---
вектор-строки
размерностей $I$ и $l_k$ или $l$, в которых $i$-я и $j$-я координаты
соответственно равны единице, а остальные нулю (это позволит нам в
дальнейшем
использовать кронекерово произведение матриц).

Векторы $\vec p^{\,*}_{k}$, $k=\overline{1,n}$, находятся из СУР:
\begin{align}
\label{ES_1}
\vec p_1^{\,*} & =
\sum\limits_{i=1}^{R}
\vec p_i^{\,*} B_{i,1}\,,\\
\vec p_k^{\,*} &= \sum\limits_{i=k-1}^{R} \vec p_i^{\,*} B_{i,k}\,, \quad 
k=\overline{2,n}\,,
\end{align}
\begin{align}
\vec p_k^{\,*} & = \sum\limits_{i=k-1}^{R} \vec p_i^{\,*} B_{i-k+1}\,, \quad 
k=\overline{n+1,R-1}\,,\\
\vec p_{R}^{\,*} & = \vec p_{R-1}^{\,*} B_0 + \vec p_{R}^{\,*} (B_0+B_1)
\end{align}
с условием нормировки
\begin{equation}
\sum\limits_{k=1}^{R} \vec p^{\,*}_k \vec1 = 1\,.
\end{equation}

Для решения СУР (1)--(4) с условием нормировки~(5) можно
воспользоваться методом, изложенным в~\cite{BDPS04}, с.~22.
Суть этого метода заключается в последовательном исключении
состояний цепи Маркова. Как показывают многочисленные расчеты,
метод позволяет проводить расчеты стационарных вероятностей
состояний с очень высокой точностью.

Приведем формулы для вычисления вектор-строки $\vec p_k$,
$k=\overline{0,R}$, координатами которой $p_{k,m}$, где
$m=\overline{1,Il_k}$
при $k=\overline{0,n-1}$ и $m=\overline{1,Il}$ при $k\geq n$, являются
стационарные
вероятности по времени того, что в системе находится $k$ заявок, а фазы
процессов генерации и обслуживания заявок равны $i$ и $j$ (здесь, как и прежде,
$m=(\vec e^{\,\prime}_i\otimes \vec e^{\,\prime\prime}_j)\vec u_k$).
Эти формулы имеют следующий вид:

\noindent
\begin{align}
\vec p_{0}  & = \frac{1}{\overline a} \sum\limits_{t=1}^{R} \vec p_t^{\,*} 
\int\limits_0^\infty
\left ( E-A^{(d)}(x)\right ) \otimes B_{t,0}(x)\, dx\,,\\
\vec p_{k} & = \frac{1}{\overline a} \sum\limits_{t=k}^{R} \vec p_t^{\,*} 
\int\limits_0^\infty
\left ( E-A^{(d)}(x)\right ) \otimes B_{t,k}(x)\, dx\,,\notag\\
&\ \ \ \ \ \ \ \ \ \ \ \ \ \ \ \ \ \ \ \ \ \ \ \ \ \ \ \ \ \ \ \ \ \ \ \ \ \  k =\overline{1,n-1}\,,\\
\label{p_3} \vec p_{k} & = \frac{1}{\overline a} \sum\limits_{t=k}^{R} \vec 
p_t^{\,*} \int\limits_0^\infty
\left ( E-A^{(d)}(x)\right ) \otimes B_{t-k}(x)\, dx\,,\notag\\
&\ \ \ \ \ \ \ \ \ \ \ \ \ \ \ \ \ \ \ \ \ \ \ \ \ \ \ \ \ \ \ \ \ \ \ \ \ \ k= \overline{n,R}\,,
\end{align}
где $A^{(d)}(x)$ --- диагональная матрица с элементами
$$
A_{i,i}^{(d)}(x) = \sum\limits_{j=1}^I A_{i,j}(x)
$$
на главной диагонали.

Обозначим через $\vec p_k^{\,-}$, $k=\overline{0,R}$, вектор-строку,
координатами которой
$p_{k,m}^{-}$, где $m=\overline{1,I l_k}$ при $k=\overline{0,n-1}$ и
$m=\overline{1,Il}$ при $k=\overline{n,R}$, являются стационарные
вероятности того,
что поступающая заявка застанет в системе $k$ других заявок, а фазы
процессов генерации и обслуживания заявок равны $i$ и $j$.
Здесь, как и ранее,
$m=(\vec e^{\,\prime}_i\otimes \vec e^{\,\prime\prime}_j) \vec u_k$.
Векторы $\vec p_k^{\,-}$, $k\ge 0$, находятся из соотношений
\begin{align}
\label{p-_1}
\vec p_0^{\,-} & =
\sum\limits_{t=1}^{R}
\vec p_t^{\,*} \Bar B_{t,0}\,, &&\\
\vec p_k^{\,-} & =
\sum\limits_{t=k}^{R}
\vec p_t^{\,*} \Bar B_{t,k},
& k & =\overline{1,n-1}\,,\\
\vec p_k^{\,-} &=\vec p_{k+1}^{\,*}, & k & =\overline{n,R-2}\,,\\
\label{p-_4} \vec p^{\,-}_k & =
\sum\limits_{j=k}^{R} \vec p_{j}^{\,*}
B_{j-k}, & k & = R-1,R\,,
\end{align}
где матрица $\Bar B_{t,k}$ определена в~\cite{PSCh06}.

Стационарное распределение $W(x)$ времени ожидания начала обслуживания
заявки может быть записано в виде
\begin{multline}
\label{w_x}
\!\!\!W(x) =1-{}\\
\!\!{}-\frac{1}{1-\pi} \sum\limits_{j=0}^{r-1} \left( \sum\limits_{i=j+1}^{r} 
\vec p^{\,-}_{i+n-1} \left ( \vec 1_m \otimes I_l\right ) \right) B_j (x){\vec 
1}\,, \!\!
\end{multline}
где $\pi =\vec p^{\,-}_{R} {\vec 1}$ --- стационарная вероятность
потери заявки.

В дальнейшем, наряду с линейной нумерацией состояний процесса
обслуживания, будем использовать также мультииндексную
нумерацию, при которой номер состояния определяется некоторым
мультииндексом или объединением мультииндексов.
В частности, рациональный способ перехода от мультииндексной нумерации
со\-сто\-яний процесса обслуживания
к линейной нумерации, необходимой для программирования, для сис\-те\-мы
$SM/PH/n/r$
предложен в~\cite{PCh03}.
Там же показано, как в этом случае формируются элементы матрицы $L$.
Теперь для вычисления стационарных показателей функционирования
СМО $SM/PH/n/r$ осталось воспользоваться приведенными выше
формулами~(\ref{ES_1})--(\ref{w_x}).

Для СМО $SM/PH/n/r$ нетрудно найти также стационарное
распределение $V(x)$ времени пребывания заявки в системе,
определяемое формулой свертки:
\begin{equation}
\label{v_x}
V(x) =
\int\limits_0^x
W(x-y)\, dH(y)\,,
\end{equation}
где $H(x)$ --- распределение времени пребывания заявки на приборе.

Более подробный анализ алгоритмов расчета
основных и вспомогательных стационарных характеристик СМО $SM/PH/n/r$
можно найти в~\cite{PCh03}.

\section{Отказы работающих приборов с~обслуживанием заявки заново}

Наиболее просто к общей модели из разд.~2
приводится $n$-линейная СМО $SM/PH/n/r$ с отказами
только тех приборов, на которых обслуживаются заявки.
При отказе прибора заявка ждет окончания ремонта этого прибора и затем
обслуживается заново.
Прибор, на котором заявка отсутствует, всегда находится в исправном
состоянии.

Пусть прибор, на котором находится заявка, с интенсивностью $\alpha$
может отказать и перейти в неисправное состояние, прекратив обслуживание
этой заявки, и с интенсивностью $\beta$ может вновь перейти в исправное
состояние и продолжить обслуживание заявки.

Время обслуживания заявки прибором, находящимся в исправном состоянии,
имеет распределение фазового типа с параметрами матрицей $H$ порядка $J$
и вектор-строкой $\vec h$ размерности $J$.

Для каждого из $n$ приборов добавим к $J$ фазам обслуживания еще
одну фазу, со\-от\-вет\-ст\-ву\-ющую нахождению прибора в неисправном состоянии.
Тогда рассматриваемая система с точки зрения таких характеристик, как
распределения длины очереди и времени пребывания в системе, будет
эквивалентна СМО $SM/PH/n/r$ без отказов приборов с
распределением времени обслуживания фазового типа с матрицей $G$ порядка
$S=J+1$ и вектор-строкой $\vec g$ размерности $S=J+1$, определяемыми
соотношениями
$$
G=
\begin{pmatrix}
H-\alpha E       &  \alpha \vec1  \\
\beta \vec h     &  - \beta        \\
\end{pmatrix}\,,
\qquad
\vec g =
(h_1,\ldots, h_J,0)\,.
$$

Теперь для нахождения основных стационарных показателей функционирования
СМО $SM/PH/n/r$ с описанным типом отказов приборов осталось
воспользоваться результатами, изложенными в предыдущем разделе для
системы $SM/PH/n/r$ без отказов.

\section{Отказы работающих приборов с~дообслуживанием заявки}

Так же просто к общей модели из разд.~2 приводится $n$-линейная СМО,
отличающаяся от описанной в предыдущем разделе только тем, что заявка
не начинает обслуживаться заново на восстановленном приборе, а
дообслуживается.


Определяя квадратную матрицу $G$ порядка $S=$ $=2J$ и вектор-строку $\vec g$ 
размерности $S=2J$ сле\-ду\-ющим образом:
$$
G=
\begin{pmatrix}
H-\alpha E     &    \alpha E   \\
\beta E        &    -\beta E   \\
\end{pmatrix},\ \
\vec g =
(h_1,\ldots,h_J,0,\ldots,0)
$$
и рассматривая СМО $SM/PH/n/r$ без отказов приборов с такими
матрицей $G$ и вектор-строкой $\vec g$, мы снова приходим к
формулам расчета основных стационарных
характеристик~(\ref{ES_1})--(\ref{v_x}),
полученным для общей модели.

\section{Отказы всех приборов с~обслуживанием заявки заново;
заявки поступают\newline на все приборы}

Следующая СМО отличается от СМО из разд.~3 лишь тем, что отказывать могут все 
приборы, а не только те, на которых обслуживаются заявки. Однако ее 
исследование уже несколько сложнее. Положим, что заявки могут поступать как на 
исправный, так и на неисправный прибор.

Будем предполагать, что вероятность отказа не занятого обслуживанием
заявки исправного прибора за <<малое>> время $\Delta$ равна
$\alpha^* \Delta + o(\Delta)$,
а вероятность восстановления не занятого обслуживанием заявки
неисправного прибора равна $\beta^* \Delta + o(\Delta)$,
где $\alpha^*$ и $\beta^*$ --- соответственно интенсивности отказов и
восстановлений
приборов, не занятых обслуживанием заявок.

Если в системе имеется свободный исправный прибор, то
поступающая в систему заявка идет на этот прибор; если же свободных
исправных приборов нет, но есть свободный восстанавливаемый прибор, то
заявка идет на него. Интенсивность восстановления такого прибора
становится равной $\beta$ (вместо $\beta^*$).

Как и прежде, время обслуживания заявки прибором, находящимся в исправном
состоянии, имеет распределение фазового типа с параметрами мат\-ри\-цей $H$
порядка $J$ и вектор-строкой $\vec h$ размерности~$J$.

Назовем слоем $k$, $k=\overline{0,R}$, множество
всех состояний процесса обслуживания, в которых общее число заявок
в системе равно $k$.

Слой $k$ при $k=\overline{n,R}$ будет иметь вид
$$
\{(i_1,\ldots,i_n)\}\,,
$$
где
$i_1,\ldots,i_n=\overline{1,J+1}$.
Состояние $(i_1,\ldots,i_n)$ означает, что первый прибор
обслуживает заявку на фазе $i_1,\ldots,$ $n$-й прибор --- на фазе $i_n$ и
еще $k-n$ заявок находятся в очереди.


Слой $k$ при $k=\overline{0,n-1}$ представляет собой множество
$$
\{(i_1,\ldots,i_k;m)\}\,,
\quad m=\overline{0,n-k}\,,
$$
где состояние $(i_1,\ldots,i_k;m)$ означает, что первый прибор обслуживает
заявку на фазе $i_1,\ldots,$ $k$-й прибор~--- на фазе $i_k$ и
$m$ приборов свободны от заявок и восстанавливаются.

Матрицы $\Lambda_k$, $k=\overline{0,n-1}$, $N_k$,
$k=\overline{0,n}$,
$\Lambda$ и $N$, порождающие инфинитезимальную матрицу $L$ общей
модели разд.~2, и матрицы $\Omega_s$, $s=\overline{0,n-1}$,
формируются точно так же, как и для аналогичной системы с бесконечным
накопителем~\cite{PSCh06}.
Полученные матрицы $L$ и $\Omega_s$, $s=\overline{0,n-1}$, позволяют
по формулам~(\ref{ES_1})--(\ref{p_3}) из разд.~2 определить стационарные
вероятности состояний по вложенной цепи Маркова и по времени,
а по формуле (\ref{w_x})~--- стационарную функцию распределения времени
ожидания начала обслуживания заявки.

Обратимся теперь к вычислению стационарного распределения времени
пребывания заявки в системе.

Обозначим через $G_1(x)$ и $G_2(x)$ распределения фазового типа с
одинаковой матрицей
$$
G=
\begin{pmatrix}
H-\alpha E       &  \alpha \vec1  \\
\beta \vec h     &  - \beta       \\
\end{pmatrix}
$$
порядка $J+1$, но с разными векторами ${\vec g}_1$ и ${\vec g}_2$
размерности $J+1$:
$$
{\vec g}_1=(0,\ldots,0,1)\,,\qquad
{\vec g}_2 = (h_1,\ldots,h_J,0)\,.
$$

Если заявка поступает в систему, в которой находится
$k$  других заявок, $k=\overline{0,n-1}$,
и все свободные $n-k$ приборов восстанавливаются,
то время ее обслуживания будет иметь функцию распределения
$G_1(x)$; в противном случае время обслуживания заявки будет иметь
функцию распределения $G_2(x)$.

Обозначим через $p_k^{(1)}$, $k=\overline{0,n-1}$, вероятность того, что 
по\-ступа\-ющая в систему заявка застанет в ней $k$ других заявок и среди 
свободных приборов будут отсутствовать исправные, а через $p_k^{(2)}$, 
$k=\overline{0,n-1}$,~--- вероятность того, что по\-сту\-па\-ющая заявка 
застанет в системе $k$ других заявок и среди свободных приборов будут 
исправные. Ясно, что $p_k^{(1)}$ представляет собой сумму координат вектора 
$\vec p_k^{\,-}$ по всем возможным фазам генерации заявок и всем возможным 
индексам $i_1,\ldots,i_{k}$, отвечающим состояниям процесса обслуживания 
$(i_1,\ldots,i_{k};n-k)$, а $p_k^{(2)}$~--- сумма остальных координат вектора 
$\vec p_k^{\,-}$. Напомним, что вектора $\vec p_k^{\,-}$ стационарных 
вероятностей того, что поступающая заявка застанет в системе $k$ других заявок, 
$k=\overline{0,R}$, определяются из соотношений~(\ref{p-_1})--(\ref{p-_4}).

Положим
$$
p^{(1)}=\sum\limits_{k=0}^{n-1} p_k^{(1)}\,,\qquad
p^{(2)}=\sum\limits_{k=0}^{n-1} p_k^{(2)}\,.
$$


С учетом введенных обозначений стационарное распределение $V(x)$ времени
пребывания заявки в системе можно записать в виде %\noindent
\begin{multline*}
V(x) = \frac{1}{1-\pi} \left( \vphantom{\int\limits_0^x}
p^{(1)}G_1(x) + p^{(2)}G_2(x) \right. + {}\\
{}+\sum\limits_{i=n}^{R-1}
\vec p_{i}^{\,-} G_2(x) {\vec 1} -{}
\\
\left. -
\sum\limits_{j=0}^{r-1}
\left(
\sum\limits_{i=n+j}^{R-1} \vec p_{i}^{\,-}
\right)
\int\limits_0^x
B_{j}(x-y)\,
dG_2(y)
{\vec 1}\,
\right).
\end{multline*}

\section{Отказы всех приборов с~обслуживанием заявки заново; заявки
поступают только на исправные приборы}

Как и в системе из предыдущего раздела, в рассматриваемой здесь СМО отказывать 
могут все приборы, а не только те, на которых обслуживаются заявки. 
По-прежнему, если в системе имеется свободный исправный прибор, то поступающая 
в систему заявка идет на этот прибор; однако если свободных исправных приборов 
нет, то заявка становится в очередь. Остальные предположения и обозначения 
предыдущего раздела остаются в %\linebreak 
силе.

В отличие от случая бесконечного накопителя~\cite{PSCh06}, эта система
уже не приводится к общей сис\-те\-ме, разобранной в разд.~2.

Как и прежде, назовем слоем $k$, $k=\overline{0,R}$, множество
всех состояний процесса обслуживания, в которых общее число заявок
в системе равно $k$.

Рассмотрим сначала случай $k<r$.
Тогда слой~$k$ при $k \geq n$ имеет вид
$$
\{(i_1,\ldots,i_n) \cup
(i_1,\ldots,i_{n-1})
\cup \ldots \cup (0)\}\,,
$$
где $i_1,\ldots,i_n=\overline{1,J+1}$.
Состояние $(i_1,\ldots,i_n)$ означает, что первый прибор обслуживает
заявку на фазе $i_1,\ldots,$ $n$-й прибор~--- на фазе $i_n$ и еще
$k-n$ заявок находятся в очереди;
состояние $(i_1,\ldots,i_{n-1})$~--- первый прибор обслуживает
заявку на фазе $i_1,\ldots,$ $(n{-}1)$-й прибор~--- на фазе $i_{n-1}$,
один прибор восстанавливается (без заявки) и $k-n+1$ заявок находятся в
очереди;
$\ldots;$ состояние $(0)$~--- все приборы восстанавливаются (без заявок) и
$k$ заявок находятся в очереди.
При этом число фаз обслуживания заявки равно $J+1$.
Последняя, $(J{+}1)$-я, фаза соответствует восстановлению неисправного
прибора (на котором находится заявка).
Слой $k$ при $k=\overline{0,n-1}$ представляет собой множество
\begin{multline*}
\!\!\{(i_1,\ldots,i_k;m) \cup
(i_1,\ldots,i_k) \cup\\
\ \cup (i_1,\ldots,i_{k-1})
\cup \ldots \cup
(0)\}\,,\quad
  m=\overline{0,n-k-1}\,,
\end{multline*}
где
состояние $(i_1,\ldots,i_k;m)$ означает, что первый прибор обслуживает
заявку на фазе $i_1,\ldots,$ $k$-й прибор~--- на фазе $i_k$, а
$m$ приборов свободны от заявок и восстанавливаются;
состояние $(i_1,\ldots,i_k)$ означает, что первый прибор обслуживает
заявку на фазе $i_1,\ldots,$ $k$-й прибор~--- на фазе $i_k$, а
остальные $n-k$ приборов свободны от заявок и восстанавливаются;
состояние $(i_1,\ldots,i_{k-1})$~--- первый прибор обслуживает заявку
на фазе $i_1,\ldots,$ $(k{-}1)$-й прибор~--- на фазе $i_{k-1}$,
$n-k+1$ приборов свободны от заявок и восстанавливаются и имеется одна
заявка в очереди;
$\ldots;$
состояние $(0)$~--- $n$ приборов восстанавливаются (без заявок) и $k$
заявок в очереди.

Перейдем к случаю $k\geq r$.
Теперь слой $k$ при $k \geq n$ имеет вид
$$
\{(i_1,\ldots,i_n) \cup
(i_1,\ldots,i_{n-1})
\cup \ldots \cup
(i_1,\ldots,i_{k-r})\}\,.
$$
Здесь состояния имеют тот же смысл, что и преж\-де, но последнее
состояние $(i_1,\ldots,i_{k-r})$ означает, что первый прибор обслуживает
заявку на фазе $i_1,\ldots,$ $(k-r)$-й прибор~--- на фазе $i_{k-r}$,
$R-k$ приборов восстанавливаются (без заявок) и $r$ заявок находятся
в очереди (накопитель полон~--- все мес\-та ожидания заняты).
%%%%%%%%%%%%%%%%%%%%
Слой $k$ при $k=\overline{0,n-1}$ представляет собой множество
\begin{multline*}
\!\!\{(i_1,\ldots,i_k;m) \cup
(i_1,\ldots,i_k) \cup
(i_1,\ldots,i_{k-1})
\cup \ldots \cup\\
\ \cup
(i_1,\ldots,i_{k-r})\}\,,
\quad m=\overline{0,n-k-1}\,,
\end{multline*}
причем и теперь отличие от случая $k< r$ состоит в уменьшении числа
состояний за счет ограничения емкости накопителя.

Обозначим через $D$ матрицу переходных вероятностей вложенной цепи Маркова, 
порожденной фазами полумарковского процесса поступления заявок и фазами 
обслуживания сразу же после моментов поступления заявок в систему. Очевидно, 
эта матрица имеет вид:

\end{multicols}
\smallskip
\noindent
$$
D=
  \begin{pmatrix}
D_{1,1}      & D_{1,2}      &  0          & \ldots &  0         &  0
               & \ldots & 0 & 0    \\
D_{2,1}      & D_{2,2}      & D_{2,3}      & \ldots &  0         &  0
               & \ldots & 0 & 0    \\
\vdots      & \vdots      & \vdots      & \ddots & \vdots     & \vdots
               & \ldots & 0 & 0 \\
D_{n-1,1}   & D_{n-1,2}   & D_{n-1,3}   & \ldots & D_{n-1,n}  &  0
               & \ldots & 0 & 0    \\
D_{n,1}      & D_{n,2}      & D_{n,3}      & \ldots & D_{n,n}     &
D_{n,n+1}
               & \ldots & 0 & 0  \\
D_{n+1,1}   & D_{n+1,2}   & D_{n+1,3}   & \ldots & D_{n+1,n}  & D_{n+1,n+1}
               & \ldots & 0 & 0  \\
\vdots      & \vdots      & \vdots      & \vdots & \vdots      & \vdots
               & \ddots & \vdots &\vdots  \\
D_{R-1,1} & D_{R-1,2} & D_{R-1,3} & \ldots & D_{R-1,n} & D_{R-1,n+1}
               & \ldots & D_{R-1,R-1}  & D_{R-1,R} \\
D_{R,1}   & D_{R,2}   & D_{R,3}   & \ldots & D_{R,n}   & D_{R,n+1}
               & \ldots & D_{R,R-1}  & D_{R,R} \\
\end{pmatrix}.
$$
\smallskip
\begin{multicols}{2}

Для нахождения элементов $D_{k,s}$ матрицы $D$ обратимся к аналогичной СМО с 
бесконечным накопителем. Матрица $B$ переходных вероятностей вложенной цепи 
Маркова системы с бесконечным накопителем, где алгоритм построения матриц 
$B_{k,s}$ получен в~\cite{PSCh06}, имеет вид: 
\end{multicols}
\smallskip
\noindent
$$
B=
\begin{pmatrix}
B_{1,1} & B_{1,2} &  0     & \ldots &  0        &  0  &  0  &  0  & \ldots
  \\
B_{2,1} & B_{2,2} & B_{2,3} & \ldots &  0        &  0  &  0  &  0  & \ldots
   \\
\vdots & \vdots & \vdots & \ddots & \vdots & \vdots & \vdots & \vdots &
\ldots \\
B_{n-1,1} & B_{n-1,2} & B_{n-1,3} & \ldots & B_{n-1,n} &  0  &  0  &  0  &
\ldots    \\
B_{n,1} & B_{n,2} & B_{n,3} & \ldots & B_{n,n} & B_0 &  0  &  0  & \ldots
\\
B_{n+1,1} & B_{n+1,2} & B_{n+1,3} & \ldots & B_{n+1,n} & B_1 & B_0 &  0  &
\ldots  \\
B_{n+2,1} & B_{n+2,2} & B_{n+2,3} & \ldots & B_{n+2,n} & B_2 & B_1 & B_0 &
\ldots  \\
\vdots & \vdots & \vdots & \vdots & \vdots & \vdots & \vdots & \vdots &
\ddots    \\
  \end{pmatrix},
$$
\smallskip
%\begin{multicols}{2}
%\noindent
%где алгоритм построения матриц $B_{k,s}$ получен в~\cite{PSCh06}.
Для матрицы $B$ используем также обозначение:
%\end{multicols}
\smallskip
%\noindent
$$
B=
  \begin{pmatrix}
B_{1,1} & B_{1,2} &  0     & \ldots &  0        &  0  &  0  &  0  & \ldots
  \\
B_{2,1} & B_{2,2} & B_{2,3} & \ldots &  0        &  0  &  0  &  0  & \ldots
   \\
\vdots & \vdots & \vdots & \ddots & \vdots & \vdots & \vdots & \vdots &
\ldots \\
B_{n-1,1} & B_{n-1,2} & B_{n-1,3} & \ldots & B_{n-1,n} &  0  &  0  &  0  &
\ldots    \\
B_{n,1} & B_{n,2} & B_{n,3} & \ldots & B_{n,n} & B_{n,n+1} &  0  &  0  &
\ldots  \\
B_{n+1,1} & B_{n+1,2} & B_{n+1,3} & \ldots & B_{n+1,n} & B_{n+1,n+1}
                  & B_{n+1,n+2} & 0 & \ldots    \\
B_{n+2,1} & B_{n+2,2} & B_{n+2,3} & \ldots & B_{n+2,n} & B_{n+2,n+1}
                  & B_{n+2,n+2} & B_{n+2,n+3} & \ldots   \\
\vdots & \vdots & \vdots & \vdots & \vdots & \vdots & \vdots & \vdots &
\ddots    \\
  \end{pmatrix}.
$$
\smallskip
\begin{multicols}{2}

Заметим теперь, что $D_{k,s}=B_{k,s}$ при $s<k$, поскольку в этом случае как
в системе с конечным, так и в системе с бесконечным накопителями
потери заявки при поступлении не происходит.
Кроме того, $D_{k,s}=B_{k,s}$ и при $k<r$, так как тогда при любом
числе отказавших приборов имеются места ожидания и заявка в обеих
системах не теряется.

Обратимся к случаю $k\geq r$.
Тогда если на предыду\-щем шаге вложенная цепь Маркова находилась в
каком-либо состоянии, соответствующему состоянию
$(i_1,\ldots,i_{k-r})$ слоя $k$ процесса обслуживания заявок,
то в системе с бесконечным накопителем она переходит в
некоторое состояние, соответствующее состоянию
$(i_1,\ldots,i_{k-r})$, но слоя $k+1$ процесса обслуживания заявок,
в то время как в сис\-те\-ме с конечным накопителем переходит в
состояние, соответствующее состоянию
$(i_1,\ldots,i_{k-r})$ того же слоя $k$ процесса обслуживания заявок,
причем с той же самой фазой полумарковского входящего процесса.
Поэтому получаем при $k\geq r$, что
$$
D_{k,k}=B_{k,k}+B_{k,k+1} E_k\,,
$$
где $E_k$~--- матрица того же размера,
что и $B_{k+1,k}$, все элементы которой равны 0, кроме равных единице
элементов,
опре\-де\-ля\-ющих переходы вложенной цепи Маркова из состояний, соответствующих
всевозможным состояниям
$(i_1,\ldots,i_{k-r})$ слоя $k+1$
процесса обслуживания заявок (для системы с бесконечным накопителем),
в состояния
$(i_1,\ldots,i_{k-r})$ слоя $k$
процесса обслуживания заявок (для системы с конечным накопителем),
причем каждая пара состояний вложенной цепи  Маркова имеет одну
и ту же фазу полумарковского входящего процесса.
Соответственно, матрица $D_{k,k+1}$ получается из матрицы $B_{k,k+1}$
удалением столбцов, соответствующих всем состояниям вложенной цепи
Маркова (системы с бесконечным накопителем), при нахождении в которых
процесс обслуживания заявок пребывает в каком-либо состоянии
$(i_1,\ldots,i_{k-r})$ слоя $k+1$.

Обозначая, как и в разд.~2, через $\vec p_k^{\,*}$, $k=\overline{1,R}$,
вектор-строку, координатами которой $p_{km}^{\,*}$, где
$m=\overline{1,I l_k}$ при
$k=\overline{1,n-1}$ и $m=\overline{1,Il}$ при $k=\overline{n,R}$,
являются стационарные вероятности по вложенной цепи Маркова того, что
в системе находится $k$ заявок, а фазы процессов
генерации и обслуживания заявок равны $i$ и $j$ соответственно,
получаем СУР
\begin{align*}
%\label{EQS_1}
\vec p_1^{\,*} &=
\sum\limits_{i=1}^{R}
\vec p_i^{\,*} B_{i,1}\,,\\
\vec p_k^{\,*} &=
\sum\limits_{i=k-1}^{R}
\vec p_i^{\,*} B_{i,k}\,,
\quad k=\overline{2,R}
\end{align*}
с условием нормировки
\begin{equation*}
\sum\limits_{k=1}^{R} \vec p^{\,*}_k \vec1 = 1\,,
\end{equation*}
решение которой можно получить методом, упомянутым в разд.~2.

Дальнейшее исследование рассматриваемой СМО, в том числе вычисление
стационарных распределений, связанных с временем пребывания заявки
в системе, также проводится по алгоритму разд.~2.

\section{Отказы работающих приборов;
марковский процесс отказов--восстановлений приборов}

В заключение рассмотрим вариант СМО с марковским процессом
от\-ка\-зов--вос\-ста\-нов\-ле\-ний.
При этом, так же как и в~\cite{PSCh06}, ограничимся рассмотрением СМО,
в которой отказывают только работающие приборы, причем отказы
приборов проявляются в изменении скорости обслуживания заявок.
Сохраняются все предположения разд.~4 относительно параметров
функционирования системы, за исключением экспоненциальности времен,
проводимых системой в исправном и неисправном состояниях.

Марковский процесс отказов--восстановлений приборов определим следующим
образом.
Имеются инфинитезимальная матрица $F=(f_{i,j})$ порядка $M$, вектор
$\vec f=(f_1,\ldots,f_M)$ размерности $M$ и диагональная матрица $D$
порядка $M$ с неотрицательными элементами $d_{i,i}$ на главной диагонали.
Число $M$ назовем числом фаз процесса отказов--восстановлений.
Элемент $d_{i,i}$ будем называть скоростью обслуживания заявки при
фазе $i$ процесса отказов--восстановлений приборов.

В момент поступления заявки на прибор с вероятностью $f_i$ выбирается
фаза $i$ процесса отказов--восстановлений прибора и далее с вероятностью
$h_k$ выбирается фаза обслуживания $k$ фазового распределения
процесса обслуживания заявки.
Затем начинается фазовое обслуживание заявки с матрицей $d_{i,i}H$.
Обслуживание при фазе $i$ процесса отказов--восстановлений прибора
происходит либо до момента окончания обслуживания, либо до момента,
когда в соответствии с интенсивностью $f_{i,j}$ фаза
процесса отказов--восстановлений не изменится на $j$-ю,
после чего обслуживание продолжится, но уже с матрицей $d_{j,j}H$,
и так далее до момента окончания обслуживания.


Для того чтобы привести систему к общей модели, необходимо положить
$$
G =
H \otimes D + E \otimes F\,,\qquad
\vec g =
\vec h \otimes \vec f\,.
$$
Теперь осталось воспользоваться результатами разд.~2, позволяющими
получить соотношения для вычисления основных стационарных характеристик
системы.

В частности, для системы из разд.~4 с экспоненциальными временами
пребывания
прибора в исправном и неисправном состояниях

\noindent
$$
F =
\begin{pmatrix}
-\alpha   &   \alpha \\
\beta     &   -\beta \\
\end{pmatrix}\,,\quad
\vec f = (1,0)\,,\quad D =
\begin{pmatrix}
1   &   0 \\
0   &   0 \\
\end{pmatrix}\,.
$$

Если же времена пребывания прибора в исправном и неисправном состояниях (для 
прибора, обслуживающего заявку) имеют фазовые распределения с матрицами 
$F_1=(f_{1,i,j})$ порядка $M_1$ и $F_2=(f_{2,i,j})$ порядка $M_2$ и векторами 
$\vec f_1=(f_{1,1},\ldots,f_{1,M_1})$ размерности $M_1$ и $\vec f_2=$ 
$=(f_{2,1},\ldots,f_{2,M_2})$ размерности $M_2$, то нужно положить
\begin{align*} %$$
M &=M_1+M_2\,,\quad
\vec f = (f_{1,1},\ldots,f_{1,M_1},0,\ldots, 0)\,,\\
F &=
\begin{pmatrix}
F_1                             &   \vec f_2 \otimes (F_1 \vec1)     \\
\vec f_1 \otimes (F_2 \vec1)  &     F_2                              \\
\end{pmatrix}\,,\quad
D =
\begin{pmatrix}
E   &   0 \\
0   &   0 \\
\end{pmatrix}\,.
\end{align*} %$$

Введение марковского процесса отказов--восстановлений приборов, как
правило, существенно увеличивает размерность процесса, описы\-ва\-юще\-го
функционирование СМО.

\section{Примеры численного расчета
характеристик СМО с ненадежными приборами}

На основе математических соотношений для общей модели {\it SM/MSP/n/r}
с надежными приборами был разработан программный комплекс,
который, в частности, позволяет рассчитывать стационарные показатели
функционирования для некоторых вариантов СМО с ненадежными приборами.
Приведем несколько примеров таких расчетов.

Расчеты производились для двух вариантов СМО с ненадежными приборами,
в которых могут происходить отказы только исправных приборов,
причем в первом случае заявка обслуживается заново (см.\ разд.~3),
а во втором --- дообслуживается (см.\ разд.~4).
%Каждая группа примеров содержит расчет характеристик СМО
%с надежными приборами (без отказов) и расчет характеристик этой же СМО, но с ненадежными
%приборами для двух пар значений интенсивностей
%отказов и восстановлений. Причем пара значений интенсивностей отказов и пара
%значений интенсивностей восстановлений выбираются таким образом, чтобы
%загрузка системы с большими значениями из пар интенсивностей отказов и восстановлений
%совпадала с загрузкой системы с меньшими значениями из этих пар.

{\bfseries\textit{Пример 1а.}}
Рассмотрим СМО $SM_2/PH_2/8/5$ c надежными приборами.
В системе имеется 8~идентичных приборов и 5~мест ожидания.
Полумарковский процесс генерации заявок имеет следующую матрицу
переходных вероятностей вложенной цепи:
$$
A=
\begin{pmatrix}
0{,}8 & 0{,}2\\
0{,}6 & 0{,}4\\
\end{pmatrix}\,.
$$
Условные функции распределения времени пребывания на фазе полумарковского
процесса генерации заявок определим следующим образом.
Пусть $d_{11}=0{,}25$ и $d_{22}=0{,}025$~--- времена между соответствующими
сменами фаз полумарковского процесса генерации заявок
для детерминированных условных функций распределения,
а $\lambda_{12}=4$ и $\lambda_{21}=2{,}5$~--- интенсивности
соответствующих смен фаз полумарковского процесса
генерации заявок для экспоненциальных условных функций распределения.
%В этом случае вектор стационарных вероятностей вложенной цепи Маркова
%полумарковского процесса
%генерации заявок
%$
%\vec\pi_a=
%\begin{pmatrix}
%0.75\\
%0.25\\
%\end{pmatrix}
%$,
%среднее время между поступлениями заявок $\overline a =0{,}25$,
%интенсивность поступления заявок $\lambda=4$.
Время обслуживания заявки каждым прибором имеет распределение
фазового типа с мат\-ри\-цей
$$
H=
\begin{pmatrix}
-2 & 0\\
2 & -2\\
\end{pmatrix}
$$
и вектором
$$
\vec h =
\begin{pmatrix}
0\\
1\\
\end{pmatrix}\,.
$$



{\bfseries\textit{Пример 1б.}}
Рассматривается СМО из примера~\textit{1а}, но теперь прибор, на котором находится
заявка, может отказывать с интенсивностью $\alpha = 2{,}5$
и восстанавливаться с интенсивностью $\beta= 12$.
Заявки после восстановления обслуживаются заново.
Свободные приборы находятся только в исправном состоянии.

{\bfseries\textit{Пример 1в.}}
Рассматривается СМО из примера~\textit{1б}, но с интенсивностями отказа
$\alpha = 2{,}5$ и восстановления $\beta= 32{,}5$.

{\bfseries\textit{Пример 1г.}}
Рассматривается СМО из примера~\textit{1б}, но с интенсивностями отказа
$\alpha = 2$ и восстановления $\beta= 12$.

\begin{table*}\small  %tabl1
\begin{center}
\Caption{Распределение числа заявок в системе по времени
\label{t1pich}}
\vspace*{2ex}

\begin{tabular}{|l|c|c|c|c|c|}
\hline
\tabcolsep=0pt
\begin{tabular}{c}Распределение\\ числа\\ заявок\end{tabular}
 & \textit{1а}      & \textit{1б}
 & \textit{1в} & \textit{1г}  & \textit{1д}    \\
\hline
\hspace*{10mm}$p_0$         & 0,01487 & 0,00031 & 0,00077 & 0,00077 & 0,00148\\
\hline
\hspace*{10mm}$p_1$         & 0,05038 & 0,00179 & 0,00406 & 0,00406 & 0,00725\\
\hline
\hspace*{10mm}$p_2$         & 0,12344 & 0,00629 & 0,01322 & 0,01321 & 0,02232\\
\hline
\hspace*{10mm}$p_3$         & 0,21045 & 0,01674 & 0,03291 & 0,03288 & 0,05257\\
\hline
\hspace*{10mm}$p_4$         & 0,23628 & 0,03613 & 0,06551 & 0,06545 & 0,09730\\
\hline
\hspace*{10mm}$p_5$         & 0,18028 & 0,06405 & 0,10466 & 0,10461 & 0,14123\\
\hline
\hspace*{10mm}$p_6$         & 0,10173 & 0,09354 & 0,13466 & 0,13466 & 0,16190\\
\hline
\hspace*{10mm}$p_7$         & 0,04735 & 0,11367 & 0,14174 & 0,14180 & 0,14993\\
\hline
\hspace*{10mm}$p_8$         & 0,02020 & 0,11861 & 0,12685 & 0,12694 & 0,11741\\
\hline
\hspace*{10mm}$p_9$         & 0,00864 & 0,11667 & 0,10634 & 0,10644 & 0,08596\\
\hline
\hspace*{10mm}$p_{10}$      & 0,00378 & 0,11470 & 0,08900 & 0,08905 & 0,06288\\
\hline
\hspace*{10mm}$p_{11}$      & 0,00164 & 0,11301 & 0,07440 & 0,07441 & 0,04588\\
\hline
\hspace*{10mm}$p_{12}$      & 0,00069 & 0,10917 & 0,06079 & 0,06074 & 0,03266\\
\hline
\hspace*{10mm}$p_{13}$      & 0,00028 & 0,09530 & 0,04509 & 0,04499 & 0,02124\\
\hline
\end{tabular}
\end{center}
%\end{table*}
%\begin{table*}\small %tabl2
\begin{center}
\Caption{Агрегированные характеристики
\label{t2pich}}
\vspace*{2ex}

\begin{tabular}{|l|c|c|c|c|c|}
\hline \multicolumn{1}{|c|}{Характеристики}   & \textit{1а} & \textit{1б}   & 
\textit{1в}   & \textit{1г}
 & \textit{1д}   \\
\hline
Интенсивность входящего потока              & 4,00000 & 4,00000 & 4,00000 & 4,00000 & 4,00000 \\
\hline
Интенсивность обслуживания                  & 8,00000 & 4,07427 & 4,57143 & 4,57143 & 5,02415 \\
\hline
Загрузка                                    & 0,50000 & 0,98177 & 0,87500 & 0,87500 & 0,79615 \\
\hline
Вероятность потери заявки                   & 0,00031 & 0,07996 & 0,03793 & 0,03785 & 0,01808 \\
\hline
Среднее число занятых приборов              & 3,99877 & 7,22613 & 6,73451 & 6,73508 & 6,25405 \\
\hline
Средняя длина очереди                       & 0,02529 & 1,59832 & 0,97615 & 0,97571 & 0,58620 \\
\hline
Среднее число заявок в системе              & 4,02406 & 8,82445 & 7,71066 & 7,71078 & 6,84025 \\
\hline
Среднее время ожидания & & & & & \\
начала обслуживания                         & 0,00632 & 0,43431 & 0,25366 & 0,25352 & 0,14925 \\
\hline
Среднее время обслуживания && & & & \\
заявки                                      & 1,00000 & 1,96354 & 1,75000 & 1,75000 & 1,59231 \\
\hline
Среднее время пребывания & & && & \\
заявки в системе                            & 1,00632 & 2,39785 & 2,00366 & 2,00352 & 1,74156 \\
\hline
\end{tabular}
\end{center}
\end{table*}
{\bfseries\textit{Пример 1д.}} Рассматривается СМО из примера~\textit{1б}, но с 
интенсивностями отказа $\alpha = 2$ и восстановления $\beta= 32{,}5$.

Сравнительные результаты расчетов для этих примеров приведены в табл.~\ref{t1pich},
\ref{t2pich} и на рис.~\ref{f1pich}.


{\bfseries\textit{Пример~2а.}} Рассматривается СМО $SM_2/PH_2/$ $4/5$ c 
4~надежными приборами и 5~мес\-та\-ми ожидания. Полумарковский процесс 
генерации заявок того же типа, что в примере~\textit{1а}, но с параметрами 
$d_{11}=0{,}5$, $d_{22}=0{,}05$, $\lambda_{12}=2$ и $\lambda_{21}=1{,}25$. 
Обслуживания фазового типа те же, что и в примере~\textit{1а}.

{\bfseries\textit{Пример~2б.}}
Рассматривается СМО из примера~\textit{2а}, но с ненадежными приборами и дообслуживанием заявок.
Свободные приборы находятся только в исправном состоянии.
Прибор, на котором находится заявка, может отказывать с интенсивностью $\alpha = 3$
и восстанавливаться с интенсивностью $\beta= 3{,}75$.

{\bfseries\textit{Пример~2в.}}
Рассматривается СМО из примера~\textit{2б} с интенсивностями отказов $\alpha = 3$
и восстановлений $\beta= 5$.

{\bfseries\textit{Пример~2г.}}
Рассматривается СМО из примера~\textit{2б} с интенсивностями отказов $\alpha = 2{,}25$
и восстановлений $\beta= 3{,}75$.

{\bfseries\textit{Пример~2д.}}
Рассматривается СМО из примера~\textit{2б} с интенсивностями отказов $\alpha = 2{,}25$
и восстановлений $\beta= 5$.

\begin{figure*} %fig1
\vspace*{1pt}
\begin{center}
\mbox{%
\epsfxsize=162.45mm
\epsfbox{pech-1.eps}
}
\end{center}
\vspace*{-9pt}
\Caption{Стационарные распределения времени ожидания начала
обслуживания $W(x)$~(\textit{а}) и времени пребывания заявки в системе
$V(x)$~(\textit{б}) для примеров~\textit{1а}--\textit{1д}
\label{f1pich}}
\end{figure*}
Результаты расчетов для примеров~\textit{2а}--\textit{2д}
приведены в табл.~\ref{t3pich},~\ref{t4pich}
и на рис.~\ref{f2pich}.

\begin{table*}\small
\begin{center}
\Caption{Распределение числа заявок в системе по времени
\label{t3pich}}
\vspace*{2ex}

\begin{tabular}{|l|c|c|c|c|c|}
\hline
\tabcolsep=0pt
\begin{tabular}{c}Распределение\\ числа\\ заявок\end{tabular}
& \textit{2а} & \textit{2б} & \textit{2в} & \textit{2г} & \textit{2д}   \\
\hline
\hspace*{10mm}$p_0$         & 0,09418 & 0,01386 & 0,02385 & 0,02389 & 0,03471 \\
\hline
\hspace*{10mm}$p_1$         & 0,25648 & 0,04807 & 0,07823 & 0,07913 & 0,10989\\
\hline
\hspace*{10mm}$p_2$         & 0,32943 & 0,10798 & 0,16334 & 0,16385 & 0,21190\\
\hline
\hspace*{10mm}$p_3$         & 0,19683 & 0,15921 & 0,21000 & 0,20879 & 0,23709\\
\hline
\hspace*{10mm}$p_4$         & 0,07473 & 0,16184 & 0,17797 & 0,17619 & 0,16977\\
\hline
\hspace*{10mm}$p_5$         & 0,02835 & 0,14153 & 0,12716 & 0,12626 & 0,10259\\
\hline
\hspace*{10mm}$p_6$         & 0,01200 & 0,12183 & 0,08993 & 0,08994 & 0,06249\\
\hline
\hspace*{10mm}$p_7$         & 0,00510 & 0,10363 & 0,06302 & 0,06357 & 0,03798\\
\hline
\hspace*{10mm}$p_8$         & 0,00210 & 0,08388 & 0,04209 & 0,04299 & 0,02214\\
\hline
\hspace*{10mm}$p_9$         & 0,00082 & 0,05818 & 0,02440 & 0,02540 & 0,01142\\
\hline
\end{tabular}
\end{center}
\end{table*}

\begin{table*}\small
\begin{center}
\Caption{Агрегированные характеристики
\label{t4pich}}
\vspace*{2ex}

\begin{tabular}{|l|c|c|c|c|c|}
\hline
\multicolumn{1}{|c|}{Характеристики}    &  \textit{2а}     & \textit{2б} & \textit{2в} & \textit{2г}     & \textit{2д}\\
\hline
Интенсивность входящего потока              & 2,00000 & 2,00000 &  2,00000 & 2,00000 & 2,00000\\
\hline
Интенсивность обслуживания                  & 4,00000 & 2,22222 &  2,50000 & 2,50000 & 2,75862\\
\hline
Загрузка                                    & 0,50000 & 0,90000 &  0,80000 & 0,80000 & 0,72500\\
\hline
Вероятность потери заявки                   & 0,00092 & 0,04855 &  0,02087 & 0,02169 & 0,01014\\
\hline
Среднее число
занятых приборов                            & 1,99816 & 3,42522 &  3,13323 & 3,13060 & 2,87059\\
\hline
Средняя длина очереди                       & 0,08010 & 1,32251 &  0,78646 & 0,79578 & 0,48721\\
\hline
Среднее число заявок в системе              & 2,07826 & 4,74773 &  3,91968 & 3,92638 & 3,35780\\
\hline
Среднее время ожидания  & & & & & \\
начала обслуживания                         & 0,04009 & 0,69500 &  0,40161 & 0,40671 & 0,24610\\
\hline
Среднее время обслуживания & & & & & \\
заявки                                      & 1,00000 & 1,80000 &  1,60000 & 1,60000 & 1,45000\\
\hline
Среднее время пребывания & & & & &  \\
заявки в системе                            & 1,04009 & 2,49500 &  2,00161 & 2,00671 & 1,69610\\
\hline
\end{tabular}
\end{center}
\end{table*}

\begin{figure*} %fig2
\vspace*{1pt}
\begin{center}
\mbox{%
\epsfxsize=162.45mm
\epsfbox{pech-2.eps}
}
\end{center}
\vspace*{-9pt}
\Caption{Стационарные распределения времени ожидания начала
обслуживания W(x)~(\textit{а}) и времени пребывания заявки в
системе $V(x)$~(\textit{б}) для примеров~\textit{2а}--\textit{2д}
\label{f2pich}}
\end{figure*}

\section{Заключение}

В настоящей работе получены математические соотношения для расчета
стационарных характеристик СМО с полумарковским входящим потоком,
обслуживанием фазового типа, конечным накопителем и ненадежными приборами.
Рассмотрены различные варианты функционирования СМО с независимо отказывающими приборами
при экспоненциальном процессе отказов--восстановлений.
Кроме этого, рассмотрен вариант системы с отказами только работающих приборов
и марковским процессом отказов--восстановлений.
Для нескольких СМО с ненадежными приборами на основе полученных в настоящей работе результатов
проведены численные расчеты следующих стационарных показателей функционирования:
распределения очереди по моментам поступления заявок и по времени,
среднего числа заявок в системе, среднего числа занятых приборов,
среднего числа заявок в очереди, распределений времени ожидания начала обслуживания
и времени пребывания заявки в системе и их средних значений.

{\small\frenchspacing
{%\baselineskip=10.8pt
\addcontentsline{toc}{section}{Литература}
\begin{thebibliography}{99}

\bibitem{Kabak_1968}
\Au{Kabak~I.\,V.}
Blocking and delays in $M^{\,(n)}/M/c$ queuing systems~//
Operations Res., 1968. Vol.~16. P.~830--840.

\bibitem{Mytrany-Avi-Itzhak_1968}
\Au{Mytrany~I.\,L., Avi-Itzhak~B.}
A many-server queue with service interruptions~//
Operations Res., 1968. Vol.~16. P.~628--638.

\bibitem{Neuts-Lucantoni_1979}
\Au{Neuts~M.\,F., Lucantoni~D.\,M.}
A Markovian queue with $N$ servers subject to breakdowns and repairs.
Mgmt. Sci., 1979. Vol.~25. P.~849--861.

\bibitem{Babitsky-Dudin-Klimenok_1996}
\Au{Бабицкий А.\,В., Дудин~А.\,Н., Клименок~В.\,И.}
К расчету характеристик ненадежной системы массового обслуживания с конечным
источником~// Автоматика и телемеханика, 1996. №\,1. С.~92--103.

\bibitem{Dudin_2002}
\Au{Дудин А.\,Н.}
Оптимальное гистерезисное управление ненадежной системой
ВМAP/SM/1 с двумя режимами работы~// Автоматика и телемеханика, 2002. №\,10.
С.~58--72.

\bibitem{Dudin-Kazimirsky-Klimenok_2004}
\Au{Dudin A.\,N., Kazimirsky~A.\,V., Klimenok~V.\,I.}
$BMAP/G/1$ system unreliable in an idle state~//
Bulletin of Kerala Mathematics Association, 2004. No.\,2. P.~1--19.

\bibitem{Mikadze-Khocholava-Khurodze_2004}
\Au{Микадзе И.\,С., Хочолава~В.\,В., Хуродзе~Р.\,А.}
Виртуальное время ожидания в однолинейной СМО с ненадежным прибором~//
Автоматика и телемеханика, 2004. №\,12. С.~119--128.

\bibitem{Mikadze-Khocholava_2005}
\Au{Mikadze~I.\,S., Khocholava~V.\,V.}
Studing the queue length in a single server queuing system with unreliable
server~//
Automation and Remote Control, 2005. Vol.~6. No.\,1. P.~65--73.

\bibitem{PSCh06}
\Au{Печинкин~А.\,В., Соколов~И.\,А., Чаплыгин~В.\,В.}
Многолинейные системы массового обслуживания с независимыми
отказами и восстановлениями приборов~// Системы и средства информатики:
спец. выпуск <<Математическое и алгоритмическое обеспечение
информационно-телекоммуникационных систем>>.
М.: Изд-во Института проблем информатики РАН, 2006. С.~99--123.

\bibitem{PCh03}
\Au{Печинкин~А.\,В., Чаплыгин~В.\,В.}
Стационарные характеристики системы массового обслуживания $SM/MSP/n/r$~//
Автоматика и телемеханика, 2004. №\,9. С.~85--100.

\bibitem{BDPS04}
\Au{Bocharov~P.\,P., D'Apice~C., Pechinkin~A.\,V., Salerno~S.}
Queueing Theory. Utrecht, Boston: VSP, 2004.

\end{thebibliography}

}
}


\end{multicols}

\label{end\stat}