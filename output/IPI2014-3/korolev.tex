%\renewcommand{\r}{\mathbb R}
%\renewcommand{\P}{{\sf P}}
%\renewcommand{\le}{\leqslant}
%\renewcommand{\ge}{\geqslant}
%\newcommand{\betm}{{\beta_{m+1+\delta}}}
%\newcommand{\bet}{\beta_{2+\delta}}
%\newcommand{\si}{{\rm Si}\:}
\def\stat{korolev}

\def\tit{ОБ УСЛОВИЯХ СХОДИМОСТИ РАСПРЕДЕЛЕНИЙ ЭКСТРЕМАЛЬНЫХ ПОРЯДКОВЫХ
СТАТИСТИК К~РАСПРЕДЕЛЕНИЮ ВЕЙБУЛЛА$^*$}

\def\titkol{Об условиях сходимости распределений экстремальных порядковых
статистик к распределению Вейбулла}

\def\aut{В.\,Ю.~Королев$^1$, И.\,А.~Соколов$^2$}

\def\autkol{В.\,Ю.~Королев, И.\,А.~Соколов}

\titel{\tit}{\aut}{\autkol}{\titkol}

{\renewcommand{\thefootnote}{\fnsymbol{footnote}}
\footnotetext[1]{Работа частично поддержана
Российским фондом фундаментальных исследований (проекты
12-07-00115a, 12-07-00109a, 14-07-00041а).}}

\renewcommand{\thefootnote}{\arabic{footnote}}
\footnotetext[1]{Факультет
вычислительной математики и кибернетики Московского государственного
университета им.\ М.\,В.~Ломоносова; Институт проблем информатики
Российской академии наук, victoryukorolev@yandex.ru}
\footnotetext[2]{Институт проблем
информатики Российской академии наук, ISokolov@ipiran.ru}

\vspace*{-8pt}


\Abst{Получены факторизационные представления
для случайных величин, имеющих распределения Вейбулла, через
случайные величины с устойчивым распределением. Эти результаты
использованы для описания условий сходимости распределений линейно
преобразованных минимальных порядковых статистик в выборках
случайного объема к распределению Вейбулла. Приведенные результаты
расширяют традиционные представления об условиях сходимости
распределений экстремальных порядковых статистик к распределению
Вейбулла и дают дополнительное теоретическое обоснование высокой
адекватности распределения Вейбулла при анализе данных типа времени
жизни, в частности в теории надежности.}

\vspace*{-4pt}

\KW{распределение Вейбулла; показательное
распределение; распределение Рэлея; строго устойчивое распределение;
выборка случайного объема}

\DOI{10.14357/19922264140301}

%\vspace*{-2pt}

\vskip 10pt plus 9pt minus 6pt

      \thispagestyle{headings}

      \begin{multicols}{2}

            \label{st\stat}

\section{Введение. Распределение Вейбулла}


В теории вероятностей и математической статистике распределением
Вейбулла принято называть специальное абсолютно непрерывное
распределение, сосредоточенное на неотрицательной полуоси, хвост
которого убывает экс\-по\-нен\-ци\-аль\-но-сте\-пен\-н$\acute{\mbox{ы}}$м образом.
Оно названо в
честь шведского ученого Валодди Вейбулла (Waloddi Weibull,
1887--1979), который в 1939~г.\ предложил использовать это
распределение при анализе прочности материалов~\cite{Weibull1939a, Weibull1939b}
и детально описал и исследовал его в 1951~г.~\cite{Weibull1951},
продемонстрировав широкие возможности этого
распределения при описании многих статистических закономерностей.

Пусть $\gamma>0$. Распределением Вейбулла с параметром формы~$\gamma$
называется распределение случайной величины~$W_{\gamma}$:

\noindent
\begin{equation}
 {\sf P}\left(W_{\gamma}<x\right)=
 \left[1-e^{-x^{\gamma}}\right]\mathbf{1}(x\ge 0)\,,\ \ \ x\in\mathbb{R}\,.
 \label{e1-kor}
\end{equation}
Здесь и далее  $\mathbf{1}(C)$ обозначает индикатор множества~$C$.

Однако Вейбулл не был первым, кто предложил это распределение,
названное впоследствии его именем. Впервые это распределение было
описано в 1927~г.\ в работе Мориса Фреше (Maurice
Fr$\acute{\mbox{e}}$chet)~\cite{Frechet1927},
посвященной изучению предельного\linebreak\vspace*{-12pt}

\columnbreak

\noindent
 поведения
крайних членов вариационного ряда. Иногда распределение Вейбулла
называют распределением Розина--Раммлера в честь Пауля Розина (Paul
Rosin) и Эриха Раммлера (Erich Rammler), немецких ученых, впервые
применивших это распределение для описания статистических
закономерностей в размерах частиц в 1933~г.~\cite{RosinRammler1933}.
Однако и это название не в полной мере соответствует исторической
истине.
%
В~своей работе~\cite{Stoyan2013} Дитрих Штойян прямо пишет,
что это распределение было открыто Розином, Раммлером, Шперлингом
(Karl Sperling)~\cite{RosinRammler1933, RosinRammlerSperling1933} и
Беннеттом (John Godolphin Bennett)~\cite{Bennett1936} в контексте
моделирования размеров частиц. 

Хорошо известно, что распределение
Вейбулла обладает свойством замкнутости относительно операции взятия
минимума независимых случайных величин. Как было показано в
работах~\cite{Frechet1927, FisherTippett1928}, благодаря этому свойству оно
является одним из возможных предельных распределений для
экстремальных порядковых статистик. 

Б.\,В.~Гнеденко нашел необходимые
и достаточные условия сходимости распределений линейно нормированных
экстремальных порядковых статистик к распределению Вейбулла~\cite{Gnedenko1943}.
Поэтому это распределение иногда также называют
(особенно в русскоязычной литературе) распределением
Вей\-бул\-ла--Гне\-ден\-ко~\cite{JohnsonKotzBalakrishnan1994}.

В уже упоминавшейся работе~\cite{Stoyan2013} Д.~Штойян замечает, что
для этого распределения больше бы \mbox{подошло} нейт\-раль\-но-тех\-ни\-че\-ское
название {\it экс\-по\-нен\-ци\-а\-ль\-но-сте\-пен\-н$\acute{\mbox{о}}$е
распределение}. Однако
последний термин традиционно используется для другого абсолютно
непрерывного распределения со сходным поведением
хвостов~\cite{BoxTiao1973, GrigoryevaKorolev2013}, у которого, в отличие от
распределения~(1), вид экс\-по\-нен\-ци\-а\-ль\-но-сте\-пен\-н$\acute{\mbox{о}}$й
функции имеет {\it плотность}
$\ell_{\gamma}(x)=((\gamma/{2})e^{-|x|^{\gamma}})/\Gamma(1/\gamma)$,
$x\hm\in\mathbb{R}$, где $\gamma\hm>0$, тогда как у распределения~(1)
экс\-по\-нен\-ци\-а\-ль\-но-сте\-пен\-н$\acute{\mbox{о}}$й
функцией является {\it функция распределения}, т.\,е.\ интеграл плотности.
%
В данной работе для распределения~(1) будет использоваться
традиционный термин {\it распределение Вейбулла}.

Распределение Вейбулла широко используется в медицине как модель
распределения вероятностей выживания (survival analysis)~\cite{JohnsonJohnson1999},
в страховании жизни как модель
распределения вероятностей дожития, в рисковом страховании как
модель распределения размера страховых требований~\cite{HoggKlugman1983},
в~экономике и финансовой математике как
модель распределения доходностей ценных бумаг~[16--18] и доходов
фирм и отдельных лиц~\cite{Bartels1977, Bordleyetal1996}, в тео\-рии
надежности как модель распределения времени безотказной
работы~\cite{Lawless1982, Abernethy2004}, в промышленной\linebreak технологии как
модель распределения времени этапов производства (например,
изготовления или постав\-ки каких-либо деталей) или как модель
распределения времени между появлениями новых технологий (так
называемая модель Ша\-ри\-фа--Ис\-ла\-ма~\cite{SharifIslam1980}), в угольной
промышленности для описания распределения размеров частиц угля при
дроб\-ле\-нии (здесь распределение Вейбулла используется под именем
распределения Ро\-зи\-на--Рам\-мле\-ра~\cite{RosinRammler1933}), в
радиотехнике и радиолокации, в метеорологии, гидрологии и многих
других областях (см., например,~\cite{JohnsonKotzBalakrishnan1994,
Lawless1982, Abernethy2004, JohnsonKotz1970, KotzNadaraja2000}).

Огромная роль, которую играет распределение Вейбулла в теории
надежности, и его популярность в инженерных приложениях не в
последнюю очередь обусловлены его простотой. Еще одним
обстоятельством, объясняющим популярность распределения Вейбулла
среди инженеров, по-ви\-ди\-мо\-му, является то, что основополагающая
статья~\cite{Weibull1951} была опубликована в инженерном журнале
{\it The ASME Journal of Applied Mechanics~--- Transactions of the
American Society of Mechanical Engineers} после того, как она была
отвергнута редколлегией статистического журнала {\it The Journal of
the American Statistical Association} как не представляющая интереса
(по другим данным, статья Вейбулла была изначально отвергнута одним
из известных английских научных журналов~\cite{Saunders1975}).
Классическая статистика того времени базировалась на стереотипе
нормальности распределения анализируемых данных, и не-нормальность
данных воспринималась как нетипичный бесполезный курьез.

В прикладной теории вероятностей принято считать, что та или иная
модель может быть в достаточной мере обоснованной (адекватной)
только тогда, когда она является {\it асимптотической
аппроксимацией}, т.\,е.\ когда существует довольно простая
предельная схема (например, схема максимума или схема суммирования)
и соответствующая предельная теорема, в которой рассматриваемая
модель выступает в качестве предельного распределения. Наличие такой
формальной асимптотической схемы может дать дополнительную
информацию о реальных механизмах, формирующих те или иные
наблюдаемые статистические закономерности.

В классических работах~\cite{Frechet1927,
FisherTippett1928, Gnedenko1943} было показано, что распределение
Вейбулла является одним из возможных предельных распределений в
схеме минимума независимых случайных величин (также см.,
например,~\cite{Gumbel1965, Galambos1984}). В~данной статье будет показано, что
на самом деле теоретических предпосылок высокой адекватности
распределения Вейбулла при анализе данных типа времени жизни намного
больше, поскольку оно также может выступать в качестве предельного и
для линейно преобразованных минимальных порядковых статистик в
выборках случайного объема при нетривиальных асимптотически
невырожденных распределениях случайного объема выборки. Приводимые
результаты расширяют традиционные представления об условиях
сходимости распределений экстремальных порядковых статистик к
распределению Вейбулла.

Необходимо также отметить, что в работе~\cite{GrigoryevaKorolevSokolov2013}
была рассмотрена версия закона
больших чисел для случайных сумм и показано, что распределение
Вейбулла с произвольным параметром формы может быть предельным для
геометрических случайных сумм независимых неодинаково распределенных
случайных величин. В~работе~\cite{Korolev_et_al2014} указанный
результат был применен к исследованию рисков, связанных с
наводнениями в Санкт-Пе\-тер\-бурге.

\vspace*{-6pt}

\section{Факторизационные представления для~случайных величин
с~распределением Вейбулла через случайные величины с~устойчивым
распределением}

\vspace*{-2pt}

В этом разделе будет показано, что распределение Вейбулла с
произвольным параметром формы $\gamma\hm>0$ может быть представлено в
виде <<масштабной>> смеси распределения Вейбулла с любым параметром
формы $\delta\hm>\gamma$, в которой в качестве смешивающего
распределения выступает односторонний строго устойчивый закон. По
своей сути приводимые далее результаты касаются специальных
представлений вероятностных распределений. Однако без какого бы то
ни было ограничения общности и для большей наглядности и
компактности результаты будут формулироваться в терминах
соответствующих случайных величин в предположении, что все случайные
величины, появляющиеся в дальнейшем изложении, заданы на одном
вероятностном пространстве $(\Omega,\,\mathfrak{A},\,{\sf P})$.

Очевидно, что $W_1$~--- это случайная величина со стандартной
показательной функцией распределения: ${\sf
P}(W_1<x)\hm=\left[1-e^{-x}\right]{\bf 1}$ $(x\hm\ge0)$. При $\gamma\hm=2$
распределение Вейбулла называется распределением Рэлея
${\sf P}(W_2<x)=\left[1-e^{-x^2}\right]{\bf 1}$ $(x\hm\ge0)$ в честь Джона Уильяма
Стрэтта лорда Рэлея (John William Strutt, Lord Rayleigh), который
ввел это распределение в 1880~г.\ в связи с задачей сложения
гармонических колебаний со случайными фазами~\cite{Rayleigh1880}.

Пусть $\Phi(x)$~--- стандартная нормальная функция распределения:
$$
\Phi(x)=\fr{1}{\sqrt{2\pi}}\int\limits_{-\infty}^{x}e^{-z^2/2}\,dz\,,\enskip
x\in\mathbb{R}\,.
$$
Функцию распределения и плотность строго устойчивого распределения с
характеристическим показателем~$\alpha$ и параметром~$\theta$,
задаваемого характеристической функцией
\begin{multline}
\mathfrak{f}_{\alpha,\theta}(t)=
\exp\left\{-|t|^{\alpha}\exp\left\{-\fr{1}{2}\,i\pi\theta\alpha\,\mathrm{sign}\,t\right\}\right\}\,,
\\ t\in\r\,,\
0<\alpha\le2\,,\  |\theta|\le\min\left\{1,\fr{2}{\alpha}-1\right\}\,,
\label{e2-kor}
\end{multline}
 будем
обозначать соответственно $G_{\alpha,\theta}(x)$ и
$g_{\alpha,\theta}(x)$ (см., например,~\cite{Zolotarev1983}). Любую
случайную величину с функцией распределения $G_{\alpha,\theta}(x)$
будем обозначать $Z_{\alpha,\theta}$.

Из~(2) вытекает, что характеристическая функция симметричного
($\theta=0$) строго устойчивого распределения имеет вид:
\begin{equation}
\mathfrak{f}_{\alpha,0}(t)=e^{-|t|^{\alpha}},\enskip t\in\r\,.
\label{e3-kor}
\end{equation}

\smallskip

\noindent

\textbf{Лемма 1.} \textit{Cимметричное строго устойчивое распределение с
характеристическим показателем~$\alpha$ является масштабной смесью
нормальных законов, в которой смешивающим распределением является
односторонний устойчивый закон $(\theta\hm=1)$ с характеристическим
показателем $\alpha/2$}:
\begin{equation}
G_{\alpha,0}(x)=\int\limits_{0}^{\infty}\Phi\left(\fr{x}{\sqrt{z}}\right)\,
dG_{\alpha/2,1}(z)\,,\enskip  x\in\r\,.\label{e4-kor}
\end{equation}

\smallskip

\noindent
Д\,о\,к\,а\,з\,а\,т\,е\,л\,ь\,с\,т\,в\,о\,.\ См., например, теорему~3.3.1
в~\cite{Zolotarev1983}.

\smallskip

Чтобы доказать, что каждое распределение Вейбулла с параметром
$\gamma\hm>0$ является масштабной смесью распределения Вейбулла с
параметром $\delta\hm>\gamma$, прежде всего убедимся, что каждое
распределение Вейбулла с параметром $\gamma\hm\in(0,2]$ является
масштабной смесью распределений Рэлея.

\smallskip

\noindent
\textbf{Лемма 2.} \textit{Для любого $\gamma\hm\in(0,2]$ справедлива
факторизация}
\begin{equation}
W_{\gamma}\eqd W_2 \sqrt{V_{\gamma/2}}\,,\label{e5-kor}
\end{equation}
\textit{где} $V_{\gamma/2}\hm=2Z_{\gamma/2,1}^{-1}$, {\it а}
$Z_{\gamma/2,1}$~--- \textit{случайная величина с односторонней строго
устойчивой плотностью $g_{\gamma/2,1}(x)$, причем случайные величины
в правой час\-ти~$(\ref{e5-kor})$ независимы.}

\smallskip

\noindent
Д\,о\,к\,а\,з\,а\,т\,е\,л\,ь\,с\,т\,в\,о\,.\ Запишем соотношение~(\ref{e4-kor}) в терминах
характеристических функций с учетом~(\ref{e3-kor}):
\begin{equation}
e^{-|t|^{\alpha}}=\int\limits_{0}^{\infty}\exp
\left\{-\fr{1}{2}\,t^2z\right\}g_{\alpha/2,1}(z)\,dz\,,
\enskip t\in\mathbb{R}\,.\label{e6-kor}
\end{equation}
Формально полагая в~(\ref{e6-kor}) $|t|\hm=x$, где $x\hm\ge0$~--- {\it произвольное
неотрицательное} число, имеем:
\begin{multline}
{\sf
P}(W_{\gamma}>x)={}\\
{}=e^{-x^{\gamma}}=\int\limits_{0}^{\infty}
\exp\left\{-\fr{1}{2}\,x^2z\right\}g_{\gamma/2,1}(z)\,dz\,.\label{e7-kor}
\end{multline}
В то же время, очевидно, что если~$W_2$ и $Z_{\gamma/2,1}$
независимы, то
\begin{multline}
{\sf P}\left(W_2 \sqrt{V_{\gamma/2}}>x\right)=
{\sf P}\left(W_2>x\sqrt{\fr{1}{2}Z_{\gamma/2,1}}\right)={}\\
{}=
\int\limits_{0}^{\infty}\exp\left\{-\fr{1}{2}\,x^2z\right\}g_{\gamma/2,1}(z)\,dz\,.
\label{e8-kor}
\end{multline}
Поскольку правые части~(\ref{e7-kor}) и~(\ref{e8-kor}) тождественно (по $x\hm\ge0$)
совпадают, тождественно совпадают и левые части этих соотношений.
Лемма доказана.

\smallskip

\noindent
\textbf{Лемма 3.} \textit{Для любого $\gamma\hm\in(0,1]$ распределение
Вейбулла с параметром~$\gamma$ является смешанным показательным
распределением}:
\begin{equation}
W_{\gamma}\eqd W_1 V_{\gamma}\,,\label{e9-kor}
\end{equation}
\textit{где} $V_{\gamma}\hm=2Z_{\gamma,1}^{-1}$, \textit{а}
$Z_{\gamma,1}$~---
\textit{случайная величина с односторонней строго устойчивой плотностью
$g_{\gamma,1}(x)$, причем случайные величины в правой час\-ти~$(\ref{e9-kor})$
независимы.}

\smallskip

\noindent
Д\,о\,к\,а\,з\,а\,т\,е\,л\,ь\,с\,т\,в\,о\,.\ \ Несложно видеть, что
${\sf P}(W_1^{1/\gamma}\hm\ge x)\hm={\sf P}(W_1\hm\ge
x^{\gamma})\hm=e^{-x^{\gamma}}\hm={\sf P}(W_{\gamma}\hm\ge x)$, $x\hm\ge 0$, т.\,е.\
\begin{equation}
W_{\gamma}\eqd W_1^{1/\gamma}\,.\label{e10-kor}
\end{equation}
Из~(\ref{e10-kor}) вытекает, что $W_2\eqd\sqrt{W_1}$. Поэтому из леммы~2
вытекает, что для $\gamma\hm\in(0,2]$
$$
W_{\gamma}\eqd W_2 \sqrt{V_{\gamma/2}}\eqd\sqrt{W_1 V_{\gamma/2}}
$$
или, с учетом~(\ref{e10-kor}),
$$
W_{\gamma/2}\eqd W_{\gamma}^2\eqd W_1 V_{\gamma/2}\,.
$$
Переобозначив $\gamma/2\longmapsto \gamma\in(0,1]$, получим
тре\-бу\-емое утверждение.

\smallskip

\noindent
\textbf{Замечание~1.} Случай малых значений параметра $\gamma\hm\in(0,1]$
представляет особый интерес, поскольку распределения Вейбулла с
такими параметрами за\-ни\-ма\-ют промежуточное место между
распределениями с экспоненциальным убыванием хвостов (показательное
распределение, гам\-ма-рас\-пре\-де\-ле\-ние) и <<тяжелохвостыми>>
распределениями со степенным убыванием хвостов типа Цип\-фа--Па\-ре\-то.

\smallskip

Из соотношения~(\ref{e9-kor}) вытекает основной результат данного раздела,
обобщающий леммы~2 и~3 и устанавливающий, что распределение Вейбулла
с произвольным положительным параметром формы~$\gamma$ является
масштабной смесью распределений Вейбулла с произвольным
положительным па\-ра\-мет\-ром формы $\delta>\gamma$.

\smallskip

\noindent
\textbf{Теорема~1.} \textit{Пусть $\delta\hm>\gamma\hm>0$~--- произвольные числа.
Тогда
$$
W_{\gamma}\eqd W_{\delta} V_{\alpha}^{1/\delta}\,,
$$
где $\alpha\hm=\gamma/\delta\in(0,1)$, а случайные величины в правой
части независимы}.

\smallskip

\noindent
Д\,о\,к\,а\,з\,а\,т\,е\,л\,ь\,с\,т\,в\,о\,.\ \
При доказательстве теоремы~2 было установлено,
что распределение Вейбулла с параметром $\alpha\hm\in(0,1]$ является
смешанным показательным распределением. Действительно, из~(\ref{e9-kor})
вытекает, что
\begin{multline*}
e^{-x^{\alpha}}={\sf P}(W_{\alpha}>x)={\sf P}
\left(W_1>\fr{1}{2}\,Z_{\alpha,1}x\right)={}\\
{}= \int\limits_{0}^{\infty}
e^{-(1/2)zx}g_{\alpha,1}(z)\,dz\,,\enskip x\ge0\,.
\end{multline*}
Поэтому для любых $\delta\hm>\gamma\hm>0$, обозначив
$\alpha\hm=\gamma/\delta$ (при этом, очевидно, $\alpha\hm\in(0,1)$), для
любого $x\hm\in\mathbb{R}$ получим:
\begin{multline*}
{\sf P}(W_{\gamma}>x)=e^{-x^{\gamma}}=e^{-x^{\delta\alpha}}={}\\
{}=
{\sf P}\left(W_{\alpha}>x^{\delta}\right)=
{\sf P}\left(W_1>\fr{1}{2}\,Z_{\alpha,1}x^{\delta}\right)={}\\
{}
=\int\limits_{0}^{\infty}e^{-(1/2) zx^{\delta}}
g_{\alpha,1}(z)\,dz={}\\
{}=\int\limits_{\!\!0}^{\infty}
{\sf P}\left(W_{\delta}>x\left(\fr{1}{2}\,z\right)^{1/\delta}\right)
g_{\alpha,1}(z)\,dz={}\\
{}=
{\sf P}(W_{\delta} V_{\alpha}^{1/\delta}>x)\,,
\end{multline*}
что и требовалось доказать. Теорема доказана.

\smallskip

\noindent
\textbf{Замечание 2}. Если $0\hm<\gamma\hm<\delta\hm<2$, то результат теоремы~1
непосредственно вытекает из теоремы~3.3.1 в~\cite{Zolotarev1983} в
силу формального совпадения характеристической функции строго
устойчивого закона и дополнительной функции распределения Вейбулла
(см.\ доказательство леммы~2).

\smallskip

Теорема~1 будет использована ниже с целью расширения представлений
об условиях сходимости распределений экстремальных порядковых
статистик к распределению Вейбулла.

\section{Об условиях сходимости распределений экстремальных порядковых
статистик в~выборках случайного объема к~распределению Вейбулла}

В классических задачах математической статистики объем выборки,
доступной исследователю, традиционно считается детерминированным и в
асимптотических постановках играет роль (неограниченно
возрастающего) {\it известного} па\-ра\-мет\-ра. В~то же время на практике
часто возникают ситуации, когда размер выборки не является заранее
определенным и может рассматриваться как случайный. Эти ситуации,
как правило, связаны с тем, что статистические данные накапливаются
в течение фиксированного времени. Это имеет место, в частности, в
страховании, когда в течение разных отчетных периодов одинаковой
длины (скажем, месяцев) происходит разное число страховых событий
(страховых выплат и/или заключений страховых контрактов); в
медицине, когда число пациентов с тем или иным заболеванием
варьируется от года к году; в технике, когда при испытании на
надежность (скажем, при определении наработки на отказ) разных
партий приборов (изделий), число отказавших приборов в разных
партиях будет разным; в информатике при разработке методов оценки
<<своевременности>> завершения программ, включая методы решения задач
предсказания времени безотказного функционирования или времени
выполнения прикладных программ в случайных вычислительных средах.
В~таких ситуациях заранее не известное число наблюдений, которые будут
доступны исследователю, разумно считать случайной величиной. Другими
словами, в таких ситуациях объем выборки не является (известным)
па\-ра\-мет\-ром, а сам становится {\it наблюдением}, т.\,е.\ статистикой.
В~силу указанных обстоятельств вполне естественным становится
изучение асимптотического поведения распределений статистик
достаточно общего вида, основанных на выборках случайного объема.

На естественность такого подхода, в частности, обратил внимание
Б.\,В.~Гнеденко в работе~\cite{Gnedenko1989}, в которой
рассматривались асимптотические свойства распределений выборочных
квантилей, построенных по выборкам случайного объема, и было
продемонстрировано, что при замене неслучайного объема выборки
случайной величиной асимптотические свойства статистик могут
радикально измениться. К~примеру, если объем выборки является
геометрически распределенной случайной величиной, то вместо
ожидаемого в соответствии с классической теорией нормального закона
в качестве асимптотического распределения выборочной медианы
возникает распределение Стьюдента с двумя степенями свободы, хвосты
которого столь тяжелы, что у него отсутствуют моменты порядков,
б$\acute{\mbox{о}}$льших второго.

Как уже говорилось, еще с работ М.~Фреше хорошо известно, что
распределение Вейбулла может быть предельным для линейно
преобразованных минимальных порядковых статистик в выборках
неслучайного объема. Несложно убедиться, что при тех же условиях на
распределение генеральной совокупности распределение Вейбулла может
быть предельным для линейно преобразованных минимальных порядковых
статистик в выборках {\it случайного} объема, если случайный объем
выборки имеет асимптотически вырожденное распределение (см.,
например,~\cite{Galambos1984} или~\cite{KorolevSokolov2008}). Однако
эта тривиальная ситуация отнюдь не исчерпывает все возможные
условия, при которых распределения линейно преобразованных
минимальных порядковых статистик в выборках случайного объема
сходятся к распределению Вейбулла.

Убедимся в этом на примере, в котором объем выборки формируется в
соответствии с дважды стохастическим пуассоновским процессом, иначе
называемым процессом Кокса. В~книгах~[34--38] показано, что
такие процессы являются удобными, реалистичными и адекватными
моделями неоднородных во времени хаотических потоков событий. В
соответствии с описанным в указанных книгах подходом поток
информативных событий, в результате каждого из которых появляется
очередное <<наблюденное>> значение рассматриваемой характеристики,
описывается с помощью точечного случайного процесса вида
$M(\Lambda(t))$, где $M(t)$, $t\hm\geq0$,~--- однородный пуассоновский
процесс с единичной интенсивностью, а $\Lambda(t)$, $t\hm\geq0$,~---
независимый от $M(t)$ случайный процесс, обладающий следующими
свойствами: $\Lambda(0)\hm=0$, ${\sf P}(\Lambda(t)\hm<\infty)\hm=1$ для
любого $t\hm>0$, траектории $\Lambda(t)$ не убывают и непрерывны
справа. Процесс $M(\Lambda(t))$, $t\hm\geq0$, называется дважды
стохастическим пуассоновским процессом (процессом Кокса).

В соответствии с такой моделью в каждый момент времени~$t$
распределение случайной величины $M(\Lambda(t))$ является смешанным
пуассоновским. Для большей наглядности рассмотрим случай, когда в
рассматриваемой модели <<бесконечно большой>> параметр~$t$
дискретен: $\Lambda(t)\hm=\Lambda_k$,  $k\hm\in\mathbb{N}$,
а~$\{\Lambda_k\}_{k\ge1}$~--- неограниченно возрастающая
последовательность случайных величин,
$\Lambda_{k+1}(\omega)\hm\ge\Lambda_k(\omega)$ для каждого
$\omega\hm\in\Omega$, $k\hm\ge1$. Соответственно, положим
$N_k\hm=M(\Lambda_k)$, $k\hm\ge1$. При этом асимптотика $k\hm\to\infty$ может
интерпретироваться как то, что (случайная) интенсивность потока
информативных событий считается очень большой.

Пусть $Y_1,Y_2,\ldots$~--- независимые случайные величины с общей
функцией распределения $F(x)\hm={\sf P}(Y_i<x)$, $x\hm\in\mathbb{R}$,
$i\hm\ge1$. Обозначим $\mathrm{lext}\left(F\right)\hm=\inf\{x:\,F(x)>0\}$.
Предположим, что при \mbox{каж\-дом} $k\hm\in\mathbb{N}$ случайная величина
$N_k$ независима от последовательности $Y_1,Y_2,\ldots$
В~книге~\cite{KorolevSokolov2008} доказан следующий результат, который в
данном контексте играет вспомогательную роль и потому оформлен в
виде леммы.

\bigskip

\noindent
\textbf{Лемма 4.} \textit{Предположим, что существуют неограниченно
возрастающая после\-до\-ва\-тель\-ность положительных чисел
$\{d_k\}_{k\ge1}$ и неотрицательная случайная величина~$\Lambda$
такие, что имеет место сходимость}
$$
d_k^{-1}\Lambda_k\Longrightarrow \Lambda\enskip (k\to\infty)\,.
$$
\textit{Предположим также, что $\mathrm{lext}\left(F\right)\hm>-\infty$ и функция
распределения $W_F(x)\hm=F\left( \mathrm{lext}\,(F)-x^{-1}\right)$
удовлетворяет условию$:$ существует положительное число~$\delta$
такое, что для любого $x\hm>0$}
\begin{equation}
\lim\limits_{y\to\-\infty}\fr{W_F(yx)}{W_F(y)}=x^{-\delta}\,.\label{e11-kor}
\end{equation}
\textit{Тогда существуют числовые последовательности $a_k$ и $b_k$ такие,
что}
\begin{multline*}
{\sf P}\left(\min_{1\le j\le N_k}Y_j-a_k<b_kx\right) \Longrightarrow{}\\
{}\Longrightarrow \left[
 1-\int\limits_{0}^{\infty}e^{-\lambda x^{\delta}}d{\sf P}
(\Lambda<\lambda)\right]\mathbf{1}(x\ge 0),\  \ \ k\to\infty.
\end{multline*}
\textit{При этом числа $a_k$ и~$b_k$ могут быть определены как}
\begin{equation}
\hspace*{-3mm}\left.
\begin{array}{rl}
a_k&=\mathrm{lext}\,(F)\,;\\[9pt]
b_k&=\sup\left\{x:\ F(x)\le
d_k^{-1}\right\}-\mathrm{lext}\,(F),\enskip k\ge1\,.
\end{array}\!
\right\}\!\!
\label{e12-kor}
\end{equation}

\smallskip

Из леммы~4 и теоремы~1 вытекает следующий результат.

\smallskip

\noindent
\textbf{Теорема 2.} \textit{Предположим, что функция распределения~$F$
принадлежит к области $\min$-при\-тя\-же\-ния распределения Вейбулла с
некоторым параметром формы $\delta\hm>0$, т.\,е.\ $\mathrm{lext}\,(F)\hm>-\infty$
и выполнено условие~$(\ref{e11-kor})$}.

\noindent{\bf I.}
\textit{Предположим, что существуют неограниченно
возрастающая после\-до\-ва\-тель\-ность положительных чисел
$\{d_k\}_{k\ge1}$ и число $\alpha\hm\in(0,1)$ такие, что имеет место
сходимость}
\begin{equation}
d_k^{-1}\Lambda_k\Longrightarrow 1\enskip (k\to\infty)\,.\label{e13-kor}
\end{equation}
\textit{Тогда существуют числовые последовательности $a_k$ и~$b_k$
такие, что}
$$
\fr{1}{b_k}\left(\min_{1\le j\le N_k}Y_j-a_k\right)\Longrightarrow
W_{\delta}\enskip (k\to\infty)\,,
$$
\textit{где числа $a_k$ и $b_k$ могут быть определены в соответствии с}~(\ref{e12-kor}).

\noindent{\bf II.} \textit{Предположим, что существуют неограниченно
возрастающая после\-до\-ва\-тель\-ность положительных чисел
$\{d_k\}_{k\ge1}$ и число $\alpha\hm\in(0,1)$ такие, что имеет место
сходимость}
\begin{equation}
d_k^{-1}\Lambda_k\Longrightarrow \fr{1}{2}\,Z_{\alpha,1}\enskip
(k\to\infty)\,. \label{e14-kor}
\end{equation}
\textit{Тогда существуют числовые последовательности $a_k$ и~$b_k$
такие, что}
$$
\fr{1}{b_k}\left(\min_{1\le j\le N_k}Y_j-a_k\right)\Longrightarrow
W_{\gamma}\enskip (k\to\infty)\,,
$$
\textit{где $\gamma=\delta\alpha$, а числа $a_k$ и~$b_k$ могут быть
определены в соответствии с}~(\ref{e12-kor}).

\smallskip

Таким образом, существенная случайность объема выборки может заметно
<<утяжелить>> хвост предельного вейбулловского закона по сравнению с
тривиальной ситуацией~(\ref{e13-kor}) с асимптотически вырожденным
(асимптотически неслучайным) индексом. Например, если $\delta\hm=1$ (т.\,е.\
 <<изначально>> предельным распределением минимальной порядковой
статистики должно быть экспоненциальное), но объем выборки имеет вид
$N_k\hm=M(\Lambda_k)$, причем при некоторых~$d_k$ выполнено условие~(\ref{e14-kor})
с $\alpha\hm=1/2$, то реальное предельное распределение
минимальной порядковой статистики~--- это распределение Вейбулла с
параметром $1/2$: $[1-e^{-\sqrt{x}}]\mathbf{1}(x\ge0)$,
$x\hm\in\mathbb{R}$. При этом, в част\-ности, квантиль порядка~0,99 этого
распределения примерно равна~21,208, что почти в 5~раз больше
соответствующей квантили показательного распределения, примерно
равной~4,605.

{\small\frenchspacing
 {%\baselineskip=10.8pt
 \addcontentsline{toc}{section}{References}
 \begin{thebibliography}{99}

\bibitem{Weibull1939a}
\Au{Weibull W.} A~statistical theory of the strength of
materials~// Ingeni$\ddot{\mbox{o}}$rs Vetenskaps Akademien Handlingar, 1939.
Nr.\,151. P.~1--45.

\bibitem{Weibull1939b}
\Au{Weibull W.} The phenomenon of rupture in
solids~// Ingeni$\ddot{\mbox{o}}$rs Vetenskaps Akademien Handlingar, 1939.
Nr.\,153. P.~16--53.

\bibitem{Weibull1951}
\Au{Weibull W.} A~statistical distribution function
of wide applicability~//  J.~Appl. Mech.
Trans. ASME, 1951.
Vol.~18. No.\,3. P.~293--297.

\bibitem{Frechet1927}
\Au{Fr$\acute{\mbox{e}}$chet M.} Sur la loi de
probabilit$\acute{\mbox{e}}$ de l'$\acute{\mbox{e}}$cart maximum~// Annales de la
Soci$\acute{\mbox{e}}$t$\acute{\mbox{e}}$ polonaise de Mathematique (Cracovie), 1927. Vol.~6.
P.~93--116.

\bibitem{RosinRammler1933}
\Au{Rosin P., Rammler~E.} The laws governing the fineness
of powdered coal~// J.~Inst. Fuel, 1933. Vol.~7.
P.~29--36.

\bibitem{Stoyan2013}
\Au{Stoyan D.} Weibull, RRSB or extreme-value
theorists?~// Metrika, 2013. Vol.~76. P.~153-159. doi: 10.1007/s00184-011-0380-6.

\bibitem{RosinRammlerSperling1933}
\Au{Rosin P., Rammler~E., Sperling~K.}
Korngr$\acute{\mbox{o}}${\hspace*{-0.6mm}\ptb{\ss}}enprobleme des Kohlenstaubes und ihre Bedeutung
f$\ddot{\mbox{u}}$r die Vermahlung. Bericht C 52 des Reichskohlenrates.~---
Berlin: VDI-Verlag, 1933. 25~p.

\bibitem{Bennett1936}
\Au{Bennett J.\,G.} Broken coal~// J.~Inst. Fuel, 1936. Vol.~10. P.~22--39.

\bibitem{FisherTippett1928}
\Au{Fisher R.\,A., Tippett~L.\,H.\,C.}
Limiting forms of the frequency distribution of the largest or
smallest member of a sample~// Proc. Camb.
Philos. Soc., 1928. Vol.~24. P.~180--190.

\bibitem{Gnedenko1943}
\Au{Gnedenko B.\,V.} Sur la distribution limite
du terme maximum d'une serie al$\acute{\mbox{e}}$atoire~// Ann. Math.,
1943. Vol.~44. No.\,3. P.~423--453.

\bibitem{JohnsonKotzBalakrishnan1994}
\Au{Johnson N.\,L., Kotz~S., Balakrishnan~N.} Continuous
univariate distributions.~--- 2nd ed.~--- New York: John Wiley \& Sons,
1994. 756~p.

\bibitem{BoxTiao1973}
\Au{Box G., Tiao~G.} Bayesian inference in statistical analysis.~---
Reading: Addison--Wesley, 1973. 578~p.

\bibitem{GrigoryevaKorolev2013}
\Au{Григорьева М.\,Е., Королев В.\,Ю.}
О~сходимости распределений случайных сумм к скошенным
экс\-по\-нен\-ци\-а\-ль\-но-сте\-пен\-ным законам~// Информатика и её применения,
2013. Т.~7. Вып.~4. С.~66--74.

\bibitem{JohnsonJohnson1999}
\Au{Elandt-Johnson R., Johnson N.} Survival models
and data analysis.~--- New York: John Wiley \& Sons, 1999. 457~p.

\bibitem{HoggKlugman1983}
\Au{Hogg R.\,V., Klugman S.\,A.} Loss distributions.~--- New York: John
Wiley \& Sons, 1984. 235~p.

\bibitem{DAddario1974}
\Au{D'Addario R.} Intorno ad una funzione di distribuzione~//
Giorn. Econ. Ann. Econ., 1974. Vol.~33. P.~205--214.

\bibitem{MittnikRachev1989}
\Au{Mittnik S., Rachev S.\,T.} Stable distributions for
asset returns~// Appl. Math. Lett., 1989. Vol.~2. No.\,3. P.~301--304.

\bibitem{MittnikRachev1993}
\Au{Mittnik S., Rachev S.\,T.} Modeling asset returns with alternative
stable distributions~// Economet. Rev., 1993. Vol.~12. P.~261--330.

\bibitem{Bartels1977} %19
\Au{Bartels C.\,P.\,A.} Economic aspects of regional welfare.~--- Leiden: Martinus Nijhoff, 1977.
261~p.

\bibitem{Bordleyetal1996} %20
\Au{Bordley R.\,F., McDonald J.\,B., Mantrala~A.}
Something new, something old: Parametric models for the size
distribution of income~// J.~Income Distribution, 1996. Vol.~6. P.~91--103.

\bibitem{Lawless1982} %21
\Au{Lawless J.\,F.} Statistical models and methods for lifetime data.~--- New
York: John Wiley \& Sons, 1982. 580~p.

\bibitem{Abernethy2004} %22
\Au{Abernethy R.\,B.} The new Weibull handbook. Reliability an
statistical analysis for predicting life, safety, survivability,
risk, cost and warranty claims.~--- 5th ed.~--- 536 Oyster Road, North
Palm Beach, FL: Robert B.~Abernethy, 2004. 310~p.


\bibitem{SharifIslam1980} %23
\Au{Nawaz Sharif~M., Nazrul Islam~M.} The Weibull distribution as a general
model for forecasting technological change~// Technol.
Forecast. Soc. Change, 1980. Vol.~18. No.\,3. P.~247--256.

\bibitem{JohnsonKotz1970} %24
\Au{Johnson N.\,L., Kotz~S.} Continuous univariate distributions.~--- Boston: Houghton Mifflin Company, 1970.
761~p.

\bibitem{KotzNadaraja2000} %25
\Au{Kotz S., Nadarajah S.} Extreme value distributions. Theory
and applications.~--- London: Imperial College Press, 2000. 186~p.

\bibitem{Saunders1975} %26
\Au{Saunders S.\,C.} Birnbaum's contributions
to reliability~//
Reliability and fault tree analysis, theoretical and applied aspects
of system reliability and safety assessment~/ Eds. R.\,E.~Barlow,
J.\,B.~Fussell, N.\,D.~Singpurwalla.~--- Philadelphia: Society for
Industrial and Applied Mathematics, 1975. XV--XXXIX.

\bibitem{Gumbel1965} \Au{Гумбель Э.} 
Статистика экстремальных значений~/ Пер. с англ.~--- М.: Мир, 1965.
452~с. (\Au{Gumbel~E.\,J.} Statistics of extremes.~--- New York: Columbia
University Press, 1958.)

\bibitem{Galambos1984} \Au{Галамбош Я.} Асимптотическая теория
экстремальных порядковых статистик~/
Пер. с англ. В.\,А.~Егорова, В.\,Б.~Невзорова.~--- М.: Наука, 1984.
303~с. (\Au{Galambos~J.} The asymptotic
theory of extreme order statistics.~--- New York: John Wiley, 1978. 366~p.)
\bibitem{GrigoryevaKorolevSokolov2013}
\Au{Григорьева М.\,Е., Королев В.\,Ю., Соколов~И.\,А.}\linebreak
Предельная теорема для геометрических сумм независимых
неодинаково распределенных случайных величин и ее
применение к прогнозированию ве\-ро\-ят\-ности катастроф в неоднородных
потоках экстремальных событий~// Информатика и её применения, 2013.
Т.~7. Вып.~4. С.~11--19.

\bibitem{Korolev_et_al2014} \Au{Королев В.\,Ю., Григорьева~М.\,Е., Нефедова~Ю.\,С.,\linebreak
Лазовский~Р.} Метод оценивания вероятностей катастроф в неоднородных
потоках экстремальных событий и его применение к прогнозированию
наводнений в Санкт-Пе\-тер\-бур\-ге~// Актуарий, 2014. Т.~5. Вып.~1. С.~53--58.

\bibitem{Rayleigh1880} \Au{Rayleigh J.\,W.\,S.} On the resultant of
a large number of vibrations of the same pitch and of arbitrary
phase~// Philos. Mag. 5th ser., 1880. Vol.~10. P.~73--78.

\bibitem{Zolotarev1983} %32
\Au{Золотарев В. М.} Одномерные устойчивые
распределения.~--- М.: Наука, 1983. 304~с.



\bibitem{Gnedenko1989} %33
\Au{Гнеденко Б.\,В.} Об оценке неизвестных параметров
распределения при случайном числе независимых наблюдений~// Труды
Тбилисского математического института, 1989. Т.~92. С.~146--150.

\bibitem{KorolevSokolov2008} %34
\Au{Королев В.\,Ю., Соколов И.\,А.} Математические модели
неоднородных потоков экстремальных событий.~--- М.: ТОРУС ПРЕСС, 2008.
192~с.

\bibitem{GnedenkoKorolev1996} %35
 \Au{Gnedenko B.\,V., Korolev~V.\,Yu.}
Random summation: Limit theorems and applications.~--- Boca Raton: CRC Press, 1996.
267~p.

\bibitem{BeningKorolev2002} %36
\Au{Bening V., Korolev V.} Generalized Poisson models and their
applications in insurance and finance.~--- Utrecht: VSP, 2002. 434~p.

\bibitem{KorolevBeningShorgin2011} %37
\Au{Коpолев В.\,Ю., Бенинг В.\,Е., Шоргин~С.\,Я.}
Математические основы теории риска.~--- 2-е изд., перераб. и дополн.~---
М.: Физматлит, 2011. 620~с.

\bibitem{Korolev2011} %38
\Au{Королев В.\,Ю.} Ве\-ро\-ят\-но\-ст\-но-ста\-ти\-сти\-че\-ские методы декомпозиции
волатильности хаотических процессов.~--- М.: Изд-во Московского ун-та, 2011.
510~с.

\end{thebibliography}
} }

\end{multicols}

\vspace*{-6pt}

\hfill{\small\textit{Поступила в редакцию 31.07.14}}

%\newpage


%\vspace*{12pt}

%\hrule

%\vspace*{2pt}

%\hrule

\newpage


\def\tit{ON CONDITIONS OF CONVERGENCE OF~THE~DISTRIBUTIONS
OF~EXTREMAL ORDER STATISTICS TO~THE~WEIBULL DISTRIBUTION}

\def\titkol{On conditions of convergence of the distributions of extremal order statistics to the
Weibull distribution}

\def\aut{V.\,Yu.~Korolev$^{1,2}$ and~I.\,A.~Sokolov$^2$}
\def\autkol{V.\,Yu. Korolev, I.\,A. Sokolov}


\titel{\tit}{\aut}{\autkol}{\titkol}

\vspace*{-9pt}

\noindent
$^1$Department of
Mathematical Statistics, Faculty of Computational Mathematics and Cybernetics,\linebreak
$\hphantom{^1}$M.\,V.~Lomonosov Moscow State University,
 1-52 Leninskiye Gory,  Moscow 119991, GSP-1, Russian\linebreak
 $\hphantom{^1}$Federation

\noindent
$^2$Institute of Informatics Problems, Russian Academy of Sciences,
44-2 Vavilov Str., Moscow 119333, Russian\\
$\hphantom{^1}$Federation



\def\leftfootline{\small{\textbf{\thepage}
\hfill INFORMATIKA I EE PRIMENENIYA~--- INFORMATICS AND APPLICATIONS\ \ \ 2014\ \ \ volume~8\ \ \ issue\ 2}
}%
 \def\rightfootline{\small{INFORMATIKA I EE PRIMENENIYA~--- INFORMATICS AND APPLICATIONS\ \ \ 2014\ \ \ volume~8\ \ \ issue\ 3
\hfill \textbf{\thepage}}}

\vspace*{6pt}

\Abste{Some product representations are obtained
for random variables with the Weibull distribution by stable
random variables. These results are used to describe the conditions
providing convergence of the distributions of linearly transformed
minimum order statistics in samples with random sizes to the Weibull
distribution. The presented results broaden traditional conceptions
concerning conditions of convergence of extremal order statistics to
the Weibull distribution and give additional theoretical explanation
for high adequacy of the Weibull distribution in lifetime data analysis,
in particular, in reliability theory.}

\KWE{Weibull distribution; exponential distribution;
Rayleigh distribution; strictly stable distribution;
sample with random size}


\DOI{10.14357/19922264140301}

\Ack
\noindent
The research was partially supported by the Russian Foundation for
Basic Research (projects Nos.\,12-07-00115a, 12-07-00109a, and 14-07-00041a).

\vspace*{9pt}

  \begin{multicols}{2}

\renewcommand{\bibname}{\protect\rmfamily References}
%\renewcommand{\bibname}{\large\protect\rm References}



{\small\frenchspacing
{%\baselineskip=10.8pt
\addcontentsline{toc}{section}{References}
\begin{thebibliography}{99}

\bibitem{1-k-1}
\Aue{Weibull, W.} 1939.
A~statistical theory of the strength of materials.
\textit{Ingeni$\ddot{\mbox{o}}$rs Vetenskaps Akademien Handlingar} Nr.\ 151.
Stockholm: Generalstabens Litografiska Anstalts Forlag. 45~p.

\bibitem{2-k-1}
\Aue{Weibull, W.} 1939. The phenomenon of rupture in solids.
\textit{Ingeni$\ddot{\mbox{o}}$rs Vetenskaps Akademien Handlingar}
 Nr.\ 153. Stockholm: Generalstabens Litografiska Anstalts Forlag. 55~p.

\bibitem{3-k-1}
\Aue{Weibull, W.} 1951. A~statistical distribution function of wide applicability.
\textit{J.~Appl. Mech. Trans. ASME}
18(3):293--297.

\bibitem{4-k-1}
\Aue{Fr$\acute{\mbox{e}}$chet, M.} 1927. Sur la loi de probabilit$\acute{\mbox{e}}$
de l'$\acute{\mbox{e}}$cart maximum.
\textit{Annales de la Soci$\acute{\mbox{e}}$t$\acute{\mbox{e}}$
polonaise de Mathematique}. Cracovie. 6:93--116.

\bibitem{5-k-1}
\Aue{Rosin, P., and E. Rammler.} 1933. The laws
governing the fineness of powdered coal. \textit{J.~Inst. Fuel}
7:29--36.

\bibitem{6-k-1}
\Aue{Stoyan, D.} 2013. Weibull, RRSB or extreme-value theorists?
\textit{Metrika} 76:153--159. doi: 10.1007/s00184-011-0380-6.

\bibitem{7-k-1}
\Aue{Rosin, P., E. Rammler, and K.~Sperling}. 1933.\linebreak
Korngr$\acute{\mbox{o}}${\ptb{\hspace*{-0.8mm}\ss}}enprobleme des Kohlenstaubes und
ihre Bede\-utung f$\ddot{\mbox{u}}$r die Vermahlung.
Bericht C 52 des Reichskohlenrates. Berlin: VDI-Verlag. 25~p.

\bibitem{8-k-1}
\Aue{Bennett, J.\,G.} 1936. Broken coal. \textit{J.~Inst. Fuel} 10:22--39.

\bibitem{9-k-1}
\Aue{Fisher, R.\,A., and L.\,H.\,C.~Tippett}. 1928.
Limiting forms of the frequency distribution of the largest or smallest
member of a sample. \textit{Proc. Camb. Philos. Soc.} 24:180--190.

\bibitem{10-k-1}
\Aue{Gnedenko, B.\,V.} 1943. Sur la distribution limite du terme maximum d'une
serie al$\acute{\mbox{e}}$atoire. \textit{Ann. Math.} 44(3):423--453.

\bibitem{11-k-1}
\Aue{Johnson, N.\,L., S. Kotz, and N.~Balakrishnan.}
1994. \textit{Continuous univariate distributions}. 2nd ed.
New York: John Wiley \& Sons. 756~p.

\bibitem{12-k-1}
\Aue{Box, G., and G. Tiao.} 1973. \textit{Bayesian inference in statisticalal analysis}.
Reading: Addison-Wesley. 578~p.

\bibitem{13-k-1}
\Aue{Grigor'eva, M.\,E., and V.\,Yu.~Korolev}. 2013.
O~skhodimosti raspredeleniy sluchaynykh summ k skoshennym
eks\-po\-nen\-tsi\-al'no-ste\-pen\-nym zakonam
[On convergence of the distributions of random sums to skew exponential power laws].
\textit{Informatika i ee Primeneniya}~--- \textit{Inform. Appl.} 7(4):66--74.
doi: 10.14357/19922264130407.

\bibitem{14-k-1}
\Aue{Elandt-Johnson, R., and N.~Johnson}. 1999.
\textit{Survival models and data analysis}. New York: John Wiley \& Sons. 457~p.

\bibitem{15-k-1}
\Aue{Hogg, R.\,V., and S.\,A.~Klugman}. 1984.
\textit{Loss distributions}. New York: John Wiley \& Sons. 235~p.

\bibitem{16-k-1}
\Aue{D'Addario, R.} 1974. Intorno ad una funzione di distribuzione.
\textit{Giorn. Econ. Ann. Econ.} 33:205--214.

\bibitem{17-k-1}
\Aue{Mittnik, S., and S.\,T.~Rachev}. 1989. Stable distributions
for asset returns. \textit{Appl. Math. Lett.} 2(3):301--304.

\bibitem{18-k-1}
\Aue{Mittnik, S., and S.\,T.~Rachev}. 1993.
Modeling asset returns with alternative stable distributions.
\textit{Economet. Rev.} 12:261--330.

\bibitem{19-k-1}
\Aue{Bartels, C.\,P.\,A.} 1977. \textit{Economic aspects of regional welfare}.
Leiden: Martinus Nijhoff. 261~p.

\bibitem{20-k-1}
\Aue{Bordley, R.\,F., J.\,B.~McDonald, and A.~Mantrala}. 1996.
Something new, something old: Parametric models for the size distribution of
income. \textit{J.~Income Distribution} 6:91--103.



\bibitem{22-k-1} %21
\Aue{Lawless, J.\,F.} 1982.
\textit{Statistical models and methods for lifetime data}.
New York: John Wiley \& Sons. 580~p.

\bibitem{21-k-1} %22
\Aue{Abernethy, R.\,B.} 2004.
\textit{The new Weibull handbook. Reliability
and statistical analysis for predicting life, safety, survivability, risk, cost and
warranty claims}. 5th ed. North Palm Beach, FL: Robert B.~Abernethy. 310~p.

\bibitem{23-k-1} %23
\Aue{Nawaz Sharif, M., and M.~Nazrul Islam}. 1980.
The Weibull distribution as a general model for forecasting technological change.
\textit{Technol. Forecast. Soc. Change} 18(3):247--256.

\bibitem{25-k-1} %24
\Au{Johnson, N.\,L., and S.~Kotz.} 1970.
\textit{Continuous univariate distributions}.  Boston: Houghton Mifflin Company. 761~p.


\bibitem{24-k-1} %25
\Aue{Kotz, S., and S.~Nadarajah}. 2000. \textit{Extreme value distributions.
Theory and applications}. London: Imperial College Press.  186~p.


\bibitem{26-k-1}
\Aue{Saunders, S.\,C.} 1975. Birnbaum's contributions to reliability.
\textit{Reliability and fault tree analysis, theoretical and applied aspects of
system reliability and safety assessment}.
Eds.\ R.\,E.~Barlow, J.\,B.~Fussell, and N.\,D.~Singpurwalla.
Philadelphia: Society for Industrial and Applied Mathematics. XV--XXXIX.

\bibitem{27-k-1}
\Aue{Gumbel, E.\,J.} 2004. \textit{Statistics of extremes}.
Mineola, New York: Courier Publications. 375~p.

\bibitem{28-k-1}
\Aue{Galambos, J.} 1987. \textit{Asymptotic theory of extreme order statistics}.
2nd ed. Malabar: Krieger Publ. Co. 430~p.

\bibitem{29-k-1}
\Au{Grigor'eva, M.\,E., V.\,Yu.~Korolev, and I.\,A.~Sokolov}.
2013. Predel'naya teorema dlya geometricheskikh sum nezavisimykh neodinakovo
raspredelennykh sluchaynykh velichin i ee primenenie k prognozirovaniyu
ve\-ro\-yat\-nosti katastrof v neodnorodnykh potokakh ekstremal'nykh sobytiy
[A~limit theorem for geometric sums of independent nonidentically distributed
random variables and its application to the prediction of the probabilities
of catastrophes in nonhomogeneous flows of extremal events].
\textit{Informatika i ee Primeneniya}~--- \textit{Inform. Appl}. 7(4):11--19.
doi: 10.14357/19922264130402.

\bibitem{30-k-1}
\Aue{Korolev, V.\,Yu., M.\,E.~Grigor'eva, Yu.\,S.~Nefedova, and R.~Lazovskiy}.
2014. Metod otsenivaniya veroyatnostey katastrof v neodnorodnykh potokakh
ekstremal'nykh sobytiy i ego primenenie k prognozirovaniyu navodneniy v
Sankt-Pe\-ter\-burge [A~method of estimation of probabilities of catastrophes
in nonstationary flows of extremal events and its application to prediction of
floods in Saint-Petersburg].  \textit{Aktuariy} [Actuary] 5(1):53--58.

\bibitem{31-k-1}
\Aue{Rayleigh, J.\,W.\,S.} 1880. On the resultant of a large number of
vibrations of the same pitch and of arbitrary phase.
\textit{Philos. Mag.} 5th ser. 10:73--78.

\bibitem{32-k-1}
\Aue{Zolotarev, V.\,M.} 1986. One-dimensional stable distributions.
\textit{Translations of mathematical monographs}. Vol.~65.
Providence:  American Mathematical Society. 284~p.


\bibitem{34-k-1} %33
\Aue{Gnedenko, B.\,V.} 1989. Ob otsenke neizvestnykh pa\-ra\-met\-rov raspredeleniya
pri sluchaynom chisle ne\-za\-vi\-si\-mykh nablyudeniy [On estimation of unknown
parameters of distributions from a random number of independent observations].
\textit{Trudy Tbilisskogo Matematicheskogo Instituta}
[Proceedings of Tbilisi Mathematical Institute]. 92:146--150.

\bibitem{37-k-1} %34
\Aue{Korolev, V.\,Yu., and I.\,A.~Sokolov}. 2008.
\textit{Matema\-ti\-che\-skie modeli neodnorodnykh potokov ekstremal'nykh sobytiy}
[Mathematical models of nonstationary flows of extremal events]. Moscow: TORUS PRESS. 192~p.

\bibitem{35-k-1} %35
\Aue{Gnedenko, B.\,V., and V.\,Yu.~Korolev}. 1996. \textit{Random summation:
Limit theorems and applications}. Boca Raton: CRC Press. 267~p.

\bibitem{36-k-1} %36
\Aue{Bening, V., and V.~Korolev}. 2002.
\textit{Generalized Poisson models and their applications in insurance and finance}.
Utrecht: VSP. 434~p.

\bibitem{33-k-1} %37
\Aue{Kopolev, V.\,Yu., V.\,E.~Bening, and S.\,Ya.~Shorgin}. 2011.
\textit{Matematicheskie osnovy teorii riska}
[Mathematical foundations of risk theory]. 2nd ed. Moscow: Fizmatlit. 620~p.

\bibitem{38-k-1}
\Aue{Korolev, V.\,Yu.} 2011. \textit{Veroyatnostno-statisticheskie metody dekompozitsii
volatil'nosti khaoticheskikh protsessov} [Probabilistic and statistical methods
for the decomposition of volatility of chaotic processes]. Moscow:
Moscow University Press. 510~p.

\end{thebibliography}
} }


\end{multicols}

\vspace*{-6pt}

\hfill{\small\textit{Received July 31, 2014}}

\vspace*{-12pt}

\Contr

\noindent
\textbf{Korolev Victor Yu.} (b.\ 1954)~---
Doctor of Science in physics and mathematics, professor, Department of
Mathematical Statistics, Faculty of Computational Mathematics and Cybernetics,
M.\,V.~Lomonosov Moscow State University,
 1-52 Leninskiye Gory,  Moscow 119991, GSP-1, Russian Federation;
leading scientist, Institute of Informatics Problems,
Russian Academy of Sciences,
44-2 Vavilov Str., Moscow 119333, Russian Federation;
victoryukorolev@yandex.ru

\vspace*{3pt}

\noindent
\textbf{Sokolov Igor A.} (b.\ 1954)~---
Academician of the Russian Academy of Sciences,
Doctor of Science in technology, Director,  Institute of
Informatics Problems, Russian Academy of Sciences,
44-2 Vavilov Str., Moscow 119333, Russian Federation;
ISokolov@ipiran.ru
 \label{end\stat}

\renewcommand{\bibname}{\protect\rm Литература}