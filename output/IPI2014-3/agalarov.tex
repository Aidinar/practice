\def\stat{agalarov}

\def\tit{МОДЕЛИ ДЛЯ СРАВНИТЕЛЬНОГО АНАЛИЗА МЕТОДОВ КЛАССИФИКАЦИИ В НЕКОТОРЫХ
РАСПРЕДЕЛЕННЫХ СИСТЕМАХ РАСПОЗНАВАНИЯ ОБРАЗОВ}

\def\titkol{Модели для сравнительного анализа методов классификации в некоторых
распределенных системах распознавания} % образов}

\def\aut{Я.\,М. Агаларов$^1$}

\def\autkol{Я.\,М. Агаларов}

\titel{\tit}{\aut}{\autkol}{\titkol}

\renewcommand{\thefootnote}{\arabic{footnote}}
\footnotetext[1]{Институт проблем информатики Российской академии наук, agglar@yandex.ru}

\vspace*{-12pt}


\Abst{В работе рассматриваются системы распознавания, в которых классы
заданы прямым указанием образцов, размещенных в распределенной базе
данных (БД), и критерием распознавания (идентификации) является совпадение
предъявленного образа хотя бы с одним из образцов. Ставится задача
сравнительного анализа последовательного и параллельного методов
классификации с точки зрения среднего времени ответа на запрос
распознавания и необходимой мощности вычислительных ресурсов.
Приведены результаты вычислительных экспериментов, полученных с
использованием аналитических и компьютерных моделей сетей массового
обслуживания (СеМО) с очередями на примерах распределенных
мультибиометрических систем идентификации. }

\KW{распределенные системы распознавания; последовательные и
параллельные методы классификации; распределение вычислительных
ресурсов; сеть массового обслуживания}

\DOI{10.14357/19922264140306}


\vskip 10pt plus 9pt minus 6pt

\thispagestyle{headings}

\begin{multicols}{2}

\label{st\stat}

\section{Введение}

Появление распределенных вычислительных систем и методов распределенных
вычислений дает возможность более быстрой обработки больших объемов
информации и выполнения трудоемких вычислений и значительно расширяет
круг реализуемых на практике ресурсоемких вычислительных задач.
Целесообразность применения распределенных вычислений в той или иной
области определяется эффективностью распределенных вычислений при
решении соответствующих задач, которую, в свою очередь, определяют
степенью возможного повышения пропускной способности оборудования и
ускорения процесса вычислений~[1].

Эффективность методов организации распределенных вычислений зависит как
от аппаратных и программных особенностей вычислительной сис\-те\-мы, так и от
того, насколько сложна решаемая задача в смысле организации параллельных
вы\-чис\-ле\-ний и алгоритм ее решения соответствует архитектуре распределенной
вычислительной системы. Особенно эффективны распределенные вычисления
в системах, в которых поступающие ресурсоемкие задания можно разбить на
независимые (в смысле процесса выполнения) блоки любой длины (в смыс\-ле
времени выполнения) и выполнять на различных вычислительных ресурсах
одновременно. К~таким сис\-те\-мам относятся некоторые сис\-те\-мы
автоматического распознавания образов~[2], среди которых отметим
компьютерные сис\-те\-мы криптоанализа~[3, 4], распределенные
автоматизированные сис\-те\-мы оперативного розыска личностей~[5--7],
распределенные электронные архивы документов с автоматической
подсистемой массового распознавания изображений~[8] и~др.

Используемые в современных искусственных системах автоматического
распознавания образов методы (процедуры распознавания) классификации
делят, как правило, на два типа:
\begin{enumerate}[(1)]
\item параллельные~--- принадлежность
предъявленного объекта (образа) проверяется одновременно по всем признакам
объекта;
\item последовательные~--- принадлежность объекта (образа) проверяется
поочередно для каждого признака в зависимости от результатов уже
проведенных для этого же образа проверок~[9].
\end{enumerate}
Каждый из этих методов имеет
свои недостатки и достоинства, и их применимость в конкретной области
зависит от предъявляемых к системе требований, от целесообразности их
применения, от архитектуры вычислительной системы и~др.

В работе рассматриваются распределенные сис\-те\-мы распознавания, в которых
классы заданы \mbox{прямым} указанием составляющих их образцов объектов, а
критерием принадлежности классу предъявленного образа объекта служит
наличие в классе образца, совпадающего с предъявленным образом. При этом
образ (образец) объекта задается значением группы признаков. К~таким
системам относятся упомянутые выше в качестве примеров сис-\linebreak\vspace*{-12pt}


\pagebreak

\begin{center}  %fig1
\vspace*{2pt}
\mbox{%
\epsfxsize=76mm
\epsfbox{aga-1.eps}
}
  \vspace*{2pt}

{{\figurename~1}\ \ \small{Функциональная схема системы}}
  \end{center}

\vspace*{3pt}


\addtocounter{figure}{1}

\noindent
те\-мы
распознавания. Как правило, такие системы
 представляют собой
распределенные системы БД, в которых каждая БД содержит в
виде записей цифровые образцы некоторой части объектов по одному или
нескольким признакам. Условно упрощенную функциональную схему
подобных систем можно представить в виде, приведенном на рис.~1. Блок~1
системы формирует модели объектов и генерирует задания в виде запросов на
распознавание объектов. Блок~2 выполняет обработку заданий (преобразует
модели объектов в образы необходимого формата, транслирует запросы,
получаемые от блока~1, на соответствующие подзапросы, направляемые на
серверы БД), управ\-ля\-ет процессами передачи заданий на серверы БД и приема
от них значений целевых (решающих) функций, формирует ответы на запросы
блока~1. В~блоке~3 решается основная задача системы~--- проверяется
принадлежность предъявленного образа к классам: каждый сервер БД в
соответствии с запросом и образом объекта, полученными от блока~2,
производит в БД поиск образца, совпадающего с предъявленным образом, и
выдает со\-от\-вет\-ст\-ву\-ющее значение целевой функции в блок~2.


     Поступающие в блок~2 запросы могут содержать различные требования:
найти в указанных БД хотя бы один образец, совпадающий с предъявленным
образом (именно для таких запросов ниже проводится анализ характеристик
системы); найти в указанных БД все образцы, совпадающие с предъявленным
образом; найти образец, содержащийся во всех указанных БД и совпадающий с
предъявленным образом и~т.\,д.

Важнейшей характеристикой любой системы распознавания является
достоверность при\-ни\-ма\-емых ею решений, которая в основном определяется
информативностью образа (образца). Чем больше он включает в себя
информативных признаков, тем больше размер образа, но тем ниже скорость и
эффективность поиска совпадающего с ним образца~\cite{2-aga, 9-aga}. Однако
в системах оперативного принятия решений наряду с высокой достоверностью
решений требуется и высокая производительность\linebreak подсистемы распознавания.
Так как мощность вычислительных ресурсов системы ограничена, то
необходимо выбрать компромиссные значения показателей
производительности и достоверности, отвеча\-ющие противоречивым
требованиям к сис\-те\-ме. Отсюда можно сделать вывод, что в больших
распределенных сис\-те\-мах распознавания, к которым предъявляются высокие
требования по оперативности и достоверности, могут возникать очереди
заданий к вычислительным ресурсам.

Хотя проведено множество исследований в направлении повышения
производительности и достоверности распределенных систем распознавания
рассматриваемого типа, все они, как правило, исходят из упрощенных моделей,
в частности без учета очередей к вычислительным ресурсам~[10].

Ниже приведены примеры использования компьютерной модели системы с
очередями для оценки характеристик блока~3 (см.\ рис.~1) системы
распознавания описанного выше типа и сравнительного анализа
последовательного и параллельного методов классификации.

\vspace*{-9pt}

\section{Модель системы}

\vspace*{-2pt}

В качестве модели системы распознавания образов рассматривается
СеМО~[11], в которой узлы представляют сервер
управления (обработки запросов и управления) и серверы БД (см.\ рис.~1),
линии~--- каналы связи (сеть связи). Предполагается, что узлы пронумерованы
числами $0, \ldots , M$ и допускают неограниченные очереди, каналы связи
имеют достаточную пропускную способность (очередями к ним можно
пренебречь). В~узел~0, представляющий сервер управления, извне поступают
$L$ пронумерованных числами $1, \ldots , L$ потоков заданий, требующих
выполнения на соответствующих множествах узлов $A_i$, $i\hm= 1,\ldots , L$,
представляющих серверы БД. Задания, заставшие узел в занятом состоянии,
становятся в очередь к узлу в порядке поступления и выполняются узлом в
порядке поступления. Каж\-дое задание во время его выполнения занимает узел
полностью. Время пребывания задания в узле есть сумма времени ожидания в
очереди и времени выполнения. Время пребывания задания в узле~0 считается
близким к нулю, и ниже при расчете времени пребывания задания в системе
узел~0 не рассматривается.

Процесс выполнения задания на узле заключается в следующем: в узле
предъявленный образ поочеред\-но сравнивается с хранящимися в нем
образцами, пока не произойдет совпадение. Предполагается, что априорная
вероятность совпадения образа с любым образцом одинаковая. Если произошло
совпадение, то выполнение задания в этом узле завершается (произошло
успешное завершение), иначе выполнение задания в узле завершается после
сравнения со всеми образцами (завершение неуспешное). После завершения
выполнения задание освобождает ресурсы узла (очередь и вычислительные
ресурсы) и покидает его. Считается, что для каждого узла известна априорная
вероятность успешного завершения для каждого типа задания.

Рассматриваются два типа процедур выполнения заданий:
\begin{enumerate}[(1)]
\item % (1)~
параллельная
процедура~--- %\linebreak
$i$-за\-да\-ние (задание типа с номером~$i$) из узла~0
на\-прав\-ля\-ет\-ся одновременно на все узлы со\-от\-вет\-ст\-ву\-юще\-го множества~$A_i$ и
выполняется на них независимо;
\item %(2)~
последовательная процедура~---
$i$-за\-да\-ние из узла~0 направляется на узлы из множества $A_i$,
выбираемые в узле~0 по одному в определенном порядке, причем задание
направляется в последующий узел только после неуспешного выполнения на
предыдущем.
\end{enumerate}

В~случае параллельной процедуры после успешного завершения
выполнения $i$-за\-да\-ния в одном из узлов задание покидает все узлы из
множества $A_i$, освободив занятые им ресурсы.

Введем обозначения:
\begin{itemize}
\item $t_j$~--- время сравнения образа с образцом в $j$-м узле;
\item
$n_j$~--- число образцов в $j$-м узле;
\item
ts$_j$~--- случайная величина времени сравнения в $j$-м узле при
несовпадении образца, $\overline{\mbox{ts}}_j$~--- ее среднее значение;
\item
tu$_j$~--- случайная величина времени сравнения в $j$-м узле при совпадении
образца, $\overline{\mbox{tu}}_j$~--- ее среднее значение;
\item
$\tau_{ji}$~--- случайная величина времени выполнения в $j$-м узле
$i$-за\-да\-ния, $\overline{\tau}_{ji}$~--- ее среднее значение;
\item
$\overline{W}_{ji}$~--- среднее значение времени ожидания в $j$-м узле
$i$-за\-да\-ния;
\item
$\overline{\mbox{Tzad}}_i$~--- среднее значение времени пребывания $i$-за\-да\-ния на
вычислительных устройствах (в блоке~3);
\item
$\overline{T}_j$~--- среднее время пребывания произвольного задания в $j$-м
узле;
\item
$\overline{T}$~--- среднее время пребывания произвольного задания в системе;

\columnbreak

\item
$\overline{N}_j$~--- среднее число заданий в $j$-м узле;
\item
$\overline{N}$~--- среднее число заданий в системе;
\item
$\alpha_{ji}$~--- априорная вероятность нахождения в $j$-м узле образца,
совпадающего с образом $i$-за\-да\-ния;
\item
$p_{ji}$~--- вероятность того, что $i$-за\-да\-ние поступит для выполнения в
$j$-й узел;
\item
$\lambda_i$~--- интенсивность потока $i$-за\-да\-ний, поступающих извне;
\item
$\Lambda_j$~--- интенсивность суммарного потока заданий, поступающих в
$j$-й узел.
\end{itemize}

Приведем выражения для некоторых
ве\-ро\-ят\-но\-ст\-но-вре\-мен\-н$\acute{\mbox{ы}}$х характеристик
рассматриваемой системы. Всюду ниже предполагаем, что система работает в
установившемся режиме (в узлах системы конечные средние длины очередей).

Заметим, что среднее время ожидания в каждом узле системы для всех типов
заданий будет одинаковой величиной, поэтому вместо величин
$\overline{W}_{\!ji}$ можно рассмотреть величины $\overline{W}_{\!j}$, $j\hm= 1,
\ldots ,M$, соответственно. Заметим также, что случайная величина времени
выполнения в $j$-м узле $i$-за\-да\-ния при условии, что в узле произойдет
совпадение образца с образом, равна одной из случайных величин $\mbox{ts}_j, \mbox{tu}_j
\hm+\mbox{ts}_j, \ldots , (n_j-1)\mbox{tu}_j\hm+\mbox{ts}_j$ с одинаковой вероятностью $1/n_j$. Тогда
для среднего времени выполнения в $j$-м узле $i$-за\-да\-ния при условии, что
в узле произойдет совпадение образца с образом, справедлива формула $(n_j-
1)\overline{\mbox{tu}}_j/2+\overline{\mbox{ts}}_j$.
Для среднего времени выполнения в $j$-м
узле $i$-за\-да\-ния справедлива формула:
\begin{multline}
\overline{\tau}_{ji} =\alpha_{ji} \fr{(n_j-1)\overline{\mbox{tu}}_j}{2}+ \alpha_{ji}
\overline{\mbox{ts}}_j + \left( 1-\alpha_{ji}\right) n_j \overline{\mbox{tu}}_j ={}\\
{}= \fr{\overline{\mbox{tu}}_j [
n_j(2-\alpha_{ji})-\alpha_{ji}]}{2}+\alpha_{ji} \overline{\mbox{ts}}_j\,.
\label{e1-aga}
\end{multline}

Второй момент случайной величины~$\tau_{ji}$ равен
\begin{multline}
\overline{\tau^2_{ji}}= \alpha_{ji} \fr{\overline{\mbox{tu}^2_j}}{n_j}
\sum\limits_{k=1}^{n_j-1} k^2 +\alpha_{ji} \overline{\mbox{ts}^2_j} +\left( 1-
\alpha_{ji}\right) n^2_j \overline{\mbox{tu}^2_j}={}\\
{}= \overline{\mbox{tu}^2_j} \left[ n^2_j -\fr{\alpha_{ji}}{6}
\left(  4n^2_j +3n_j-
1\right)\right] +\alpha_{ji} \overline{\mbox{ts}^2_j}\,.
\label{e2-aga}
\end{multline}

Легко заметить, что при большом значении параметра $n_j$ и
детерминированных величинах $\mbox{tu}_j$ и $\mbox{ts}_j$ функцию распределения времени
выполнения задания в узле~$j$ можно аппроксимировать равномерным
распределением, заданным на отрезке $[0, n_j \mbox{tu}_j]$, при этом погрешность
первого момента случайной величины времени выполнения $i$-за\-да\-ния
будет равна $\alpha_{ji}(2\mbox{ts}_j-\mbox{tu}_j)/(2n_j)$,
а погрешность второго момента~---
$\alpha_{ji}(6\mbox{ts}_j^2+\mbox{tu}_j^2)/(6n_j^2) \hm- \alpha_{ji} \mbox{tu}_j^2/(2n_j)$.

Пусть в последовательной процедуре для узлов $i_k\hm\in A_i$, $k\hm= 1, \ldots
, \vert A_i\vert$, индекс~$k$ показывает порядок выполнения $i$-за\-да\-ний на
множестве узлов $A_i$, где $i_1$~--- номер первого узла, на котором
выполняется $i$-за\-да\-ние; $i_r$~--- номер узла, на который поступает
$i$-за\-да\-ние после неуспешного завершения выполнения в узле с номером
$i_{r-1}$, $r\hm= 2, \ldots , \vert A_i\vert$. Для среднего времени пребывания
\begin{multline*}
\overline{\mbox{Tzad}}_i = \alpha_{i_1i} \left[ \overline{W}_{i_1}
+\fr{\left(n_{i_1}+1\right)t_{i_1}}{2}\right] +{}\\
{}+ (1-\alpha_{i_1i}) \alpha_{i_2i} \left[ \overline{W}_{i_1} +\overline{W}_{i_2}
+t_{i_1} +\fr{\left(n_{i_2}+1\right)t_{i_2}}{2}\right] +\cdots\\
\cdots + \prod\limits_{k=1}^{\vert A_i\vert -1} \left(1-\alpha_{i_ki}\right)
\left[
\sum\limits_{k=1}^{\vert A_i\vert -1} \left(
\overline{W}_{i_k} +t_{i_k}\right) +
\overline{W}_{i_{\vert A_i\vert }} +{}
\right.
\\
\left.{}+\alpha_{i_{\vert A_i\vert }i}
\fr{\left(n_{i_{\vert A_i\vert }}+1\right)t_{i_{\vert A_i\vert }}}{2}+
\left(1-\alpha_{i_{\vert A_i\vert }i}\right) t_{i_{\vert A_i\vert }}
\vphantom{\sum\limits_{k=1}^{\vert A_i\vert -1}}
\right] = {}\\
{}=\overline{W}_{i_1}+
\overline{\tau}_{i_1i} +\left(1-\alpha_{i_1i}\right) \left(\overline{W}_{i_2}
+\overline{\tau}_{i_2i}\right) +\cdots\\
\cdots + \prod\limits_{k=1}^{\vert A_i\vert -1}
\left(1-\alpha_{i_ki}\right)\left(\overline{W}_{i_{\vert A_i\vert }}
+\overline{\tau}_{i_{\vert A_i\vert }i}\right)={}\\
{}= \sum\limits_{k=1}^{\vert A_i\vert } p_{i_ki}
\left( \overline{W}_{i_k}+t_{i_k}\right) =
\sum\limits_{k=1}^{\vert A_i\vert } p_{i_ki}\overline{T}_{i_k}\,.
%\label{e3-aga}
\end{multline*}
Для среднего времени пребывания произвольного задания
в системе справедливо соотношение:
\begin{multline*}
\overline{T} = \sum\limits_{i=1}^L \fr{\lambda_i}{\lambda}\,\overline{\mbox{Tzad}}_i =
\fr{1}{\lambda}\sum\limits_{i=1}^L \lambda_i \sum\limits_{k=1}^{\vert A_i\vert }
p_{i_k i} \overline{T}_{i_k}={}\\
{}= \fr{1}{\lambda} \sum\limits_{j=1}^M \overline{T}_j \sum\limits_{i:\ j\in A_i}
\lambda_i p_{ji} =\fr{1}{\lambda} \sum\limits_{j=1}^M \overline{T}_j\Lambda_j\,,
\end{multline*}
где $\lambda= \sum\limits_{i=1}^L \lambda_i$.

Использовав формулу Литтла~\cite{11-aga}, получим
\begin{equation}
\overline{T} =\fr{1}{\lambda} \sum\limits^M_{j=1} \overline{N}_j
=\fr{\overline{N}}{\lambda}\,.
\label{e4-aga}
\end{equation}

Пусть $R_j$~--- мощность вычислительных ресурсов $j$-го узла, а суммарная
мощность вычислительных ресурсов ограничена величиной~$R$, т.\,е.\
$\sum\limits_{j=1}^M R_j\hm\leq R$. Рассмотрим задачу выбора оптимальных
значений величин~$R_j$, $j\hm= 1, \ldots , M$, при заданных значениях
остальных параметров системы при последовательной процедуре. Пусть tu$_j$ и
ts$_j$, $j\hm= 1,\ldots , M$,~--- величины, заданные для вычислительных ресурсов
единичной мощности. Будем считать, что время выполнения $i$-за\-да\-ния в
$j$-м узле при мощности ресурсов $R_j$ равно $\tau_{ji}/R_j$ (на практике это
достигается путем фрагментации БД), а величина~$R$ такова, что
существуют $R_j$, $j\hm= 1, \ldots , M$, при которых в узлах в установившемся
режиме работы системы не могут возникать бесконечные очереди, т.\,е.\ $\rho_j
= \Lambda_j (\overline{\tau}_{ji}/R_j) \hm <1$, $j\hm= 1, \ldots , M$.
Предположим, что потоки заданий, поступающие на узлы системы, являются
пуассоновскими. Тогда для среднего времени ожидания задания в узле
справедлива следующая формула (вытекает из формулы
Пол\-ля\-че\-ка--Хин\-чи\-на~\cite{11-aga} и из~(\ref{e1-aga}), (\ref{e2-aga})):
\begin{multline}
\overline{W}_j = \fr{\lambda_j}{2}\,\fr{\sum\limits_{i:\ j\in A_i}
\left(\lambda_{ji}/\Lambda_j\right) \left(\overline{\tau^2_{ji}}/R^2_j\right)}{1-\Lambda_j
\sum\limits_{i:\ j\in A_i}\left(\lambda_{ji}/\Lambda_j\right)
\left(\overline{\tau_{ji}}/R_j\right)}={}\\
{}=
\fr{\sum\limits_{i:\ j\in A_i} \lambda_{ji} \overline{\tau^2_{ji}}}{2R_j\left( R_j -
\sum\limits_{i:\ j\in A_i} \lambda_{ji} \overline{\tau_{ji}}\right)}\,,
\label{e5-aga}
\end{multline}
где $\lambda_{ji} = \sum\limits_{i:\ j\in A_i} \lambda_i p_{ji}$; $\Lambda_j\hm=
\sum\limits^L_{i:\ j\in A_i} \lambda_{ji}$; $\sum\limits_{i:\ j\in A_i}
(\lambda_{ji}/\Lambda_j)(\overline{\tau_{ji}}/R_j)$~--- среднее время
выполнения произвольного задания в $j$-м узле.

Среднее время пребывания вычисляется по формуле (следует из~(\ref{e5-aga})):
\begin{equation}
\overline{T}_j = \fr{\sum\limits_{i:\ j\in A_i} \lambda_{ji}
\overline{\tau^2_{ji}}}{2R_j \left(R_j-
\sum\limits_{i:\ j\in A_i} \lambda_{ji} \overline{\tau_{ji}}\right)}+\sum\limits_{i:\ j\in A_i}
\fr{\lambda_{ji}}{\Lambda_j}\,\fr{\overline{\tau}_{ji}}{R_j}\,.
\label{e6-aga}
\end{equation}

Среднее число заданий в узле вычисляется по формуле (следует из формулы
Литтла и из~(\ref{e6-aga})):
\begin{multline}
\overline{N}_j = \fr{\Lambda_j\sum\limits_{i:\ j\in A_i} \lambda_{ji}
\overline{\tau^2_{ji}}}{2R_j \left(R_j-\sum\limits_{i:\ j\in A_i} \lambda_{ji}
\overline{\tau}_{ji}\right)}+{}\\
{}+\fr{1}{R_j} \sum\limits_{i:\ j\in A_i} \lambda_{ji} \overline{\tau}_{ji}\,.
\label{e7-aga}
\end{multline}

Обозначим
$$
a_j= \fr{\Lambda_j \sum\limits_{i:\ j\in A_i} \lambda_{ji}
\overline{\tau^2_{ji}}}{2}\,;\quad b_j=\sum\limits_{i:\ j\in A_i} \lambda_{ji}
\overline{\tau_{ji}}\,.
$$
Тогда~(\ref{e7-aga}) принимает вид:
$$
\overline{N}_j = \fr{a_j}{R_j(R_j-b_j)} +\fr{b_j}{R_j}\,.
$$
Предположив, что $R_j$, $j\hm= 1, \ldots , M$,~--- непрерывные величины,
из~(\ref{e7-aga}) для производной функции $\overline{N}_j$ по $R_j$ получим:
\begin{equation}
\fr{\partial \overline{N}_j}{\partial R_j} = - \fr{a_j (2R_j-b_j)}
{R^2_j (R_j-b_j)^2}
- \fr{b_j}{R_j^2}\,.
\label{e8-aga}
\end{equation}
Для второй производной получим:
\begin{equation}
\fr{\partial^2 \overline{N}_j}{\partial^2 R_j} = \fr{a_j[6(R_j-b_j)+2b_j]}{R^3_j
(R_j-b_j)^3}+ \fr{2b_j}{R_j^3}\,.
\label{e9-aga}
\end{equation}
Так как $R_j>b_j$ (следует из введенного выше предположения $\rho_j\hm<1$,
$j\hm= 1,\ldots , M$), то из~(\ref{e8-aga}) и~(\ref{e9-aga}) следуют неравенства
$$
\fr{\partial \overline{N}_j}{\partial R_j}<0;\quad
\fr{\partial^2\overline{N}_j}{\partial^2 R_j}>0
$$
(т.\,е.\ среднее число заданий в узле является вы\-пук\-лой функцией по
переменной~$R_j$) и $\partial^2\overline{N}_j/\partial^2 R_j$~--- непрерывные
по $R_j\hm>b_j$, $j\hm=1,\ldots , M$, функции. Следовательно, существует
единственный набор значений $\{R_j^*,\ j\hm= 1,\ldots , M\}$, такой что
$\sum\limits_{j=1}^M R_j^*\hm=R$, $R^*_j\hm>b_j$, $j\hm= 1,\ldots , M$, и при
этих значениях среднее число заданий в системе~$\overline{N}$~достигает
глобального минимума~\cite{12-aga}, причем единственного. Так как
$\Lambda_j\hm>0$ не зависит от~$R_i$, $j,i\hm= 1,\ldots , M$, и
$\overline{N}_j\hm= \Lambda_j \overline{T}_j$, то функция~$\overline{T}$
также достигает при наборе значений $\{R_j^*,\ j\hm= 1,\ldots , M\}$
единственного глобального минимума (так как верна формула~(\ref{e4-aga})).

Необходимые и достаточные условия существования минимума для
функции~$\overline{T}$ при непрерывных величинах~$R_j$, $j\hm= 1,\ldots ,
M$, имеют вид~\cite{12-aga}:
\begin{equation}
\fr{\partial \overline{N}_j}{\partial R_j}= \fr{\partial \overline{N}_k }{\partial
R_k}\,,\enskip j,k=1,\ldots , M\,.
\label{e10-aga}
\end{equation}
Таким образом, как следует из~(\ref{e8-aga})--(\ref{e10-aga}), набор $\{R_j^*,\
j\hm= 1,\ldots ,M\}$ является решением системы уравнений:
\begin{equation}
\left.
\begin{array}{rl}
\hspace*{-5mm}\fr{a_1(2R_1-b_1)}{R^2_1(R_1-b_1)^2}+\fr{b_1}{R_1^2} &=\\
&\hspace*{-30mm}{}= \fr{a_j(2R_j-b_j)}
{R^2_j(R_j-b_j)^2} + \fr{b_j}{R_j^2}\,,\enskip j=2,\ldots M\,;\\[9pt]
\sum\limits_{j=1}^M R_j&=R\,.
\end{array}
\right\}
\label{e11-aga}
\end{equation}

Как легко видеть из~(\ref{e11-aga}), распределение ресурсов $R_j=k\sqrt{b_j}$,
$j\hm= 1,\ldots , M$, где $k= R\Big /\sum\limits^M_{j=1}\sqrt{b_j} \hm >1$, является
хорошим приближением решения сис\-те\-мы уравнений~(\ref{e11-aga}), если
$R_j\hm\gg b_j$, $R_j\hm\gg a_j$ (т.\,е.\ время ожидания близко к нулю). Иными
словами, распределение ресурсов, балансирующее нагрузку в системе, дает
хорошее приближение к оптимальному, если времена ожидания в узлах при
этом распределении ресурсов близки к нулю, иначе следует искать более
точное приближение решения сис\-темы~(\ref{e11-aga}).

\vspace*{-6pt}

\section{Вычислительные эксперименты}

\vspace*{-2pt}

Ниже приведены два примера системы распознавания, для которых с помощью
компьютерных моделей проведены расчеты среднего времени\linebreak выполнения
произвольного задания и среднего чис\-ла заданий в системе при использовании
параллельной и последовательной процедур распо-\linebreak знавания.
{ %\looseness=1

}

В первом примере рассмотрена мультимодальная система идентификации по
изображению лица и отпечаткам пальцев. Система состоит из трех узлов,
соединенных коммуникационной средой: центра обработки запросов и
управления системой (узел~0), узла идентификации по изображению лица
(узел~1), узла идентификации по отпечаткам пальцев (узел~2). Основные
параметры системы имеют следующие значения: размер базы образцов в
каждом из узлов~1 и~2 равен 20\,000\,000 записей, скорость сравнения записи
предъявленного образа с одной записью в БД на одном процессоре в первом
узле 5000~сравнений/с, на одном процессоре во втором узле
250\,000~сравнений/с, суточный поток запросов в узел~0 равен 60\,000,
вероятность обнаружения лица лежит в интервале 0{,}9--0{,}99,
вероятность распознавания по отпечаткам пальцев равна~1,0~\cite{5-aga}.
Предполагается, что в системе выполняются следующие условия:
\begin{figure*} %fig2
\vspace*{1pt}
\begin{center}
\mbox{%
\epsfxsize=144mm
\epsfbox{aga-2.eps}
}
\end{center}
\vspace*{-9pt}
\Caption{Зависимость суммарного среднего числа заданий~(\textit{а})
и среднего времени пребывания заданий~(\textit{б}) в узлах~1 и~2 от общего объема
процессоров}
\end{figure*}
\begin{table*}\small
\vspace*{-3pt}
\begin{center}
\begin{tabular}{|c|c|c|c|c|c|}
\multicolumn{6}{c}{Распределение процессоров между узлами при последовательной
процедуре}\\[6pt]
\hline
\tabcolsep=0pt\begin{tabular}{c}Вероятность\\ обнаружения\\
 \end{tabular}&
 \tabcolsep=0pt\begin{tabular}{c}Среднее число\\ заданий\end{tabular}&
 \tabcolsep=0pt\begin{tabular}{c}Среднее время\\ пребывания задания\end{tabular}&
 \multicolumn{3}{c|}{\tabcolsep=0pt\begin{tabular}{c}Число процессоров\\ в узлах\end{tabular}}\\
\cline{4-6}
($\alpha_{j1}$, $j=2,4,6,7$)& в системе& в системе&\ \ \ \,1\,\ \ \ &\ \ \ \,2\,\ \ \ &3\\
\hline
0,4&\hphantom{9999}5,835192&\hphantom{9999}4,167994&28&14&8\\
0,5&4,38&3,13&28&13&9\\
0,6&3,23&2,3\hphantom{9}&29&13&8\\
0,7&2,4\hphantom{9}&1,77&29&13&8\\
0,8&2,02&1,44&30&12&8\\
0,9&1,62&1,16&30&11&9\\
\hline
\end{tabular}
\vspace*{-6pt}
\end{center}
\end{table*}
\begin{figure*}[b] %fig3
\vspace*{-6pt}
\begin{center}
\mbox{%
\epsfxsize=144mm
\epsfbox{aga-4.eps}
}
\end{center}
\vspace*{-9pt}
\Caption{Зависимость среднего числа заданий~(\textit{а})
и среднего времени пребывания задания~(\textit{б}) в сис\-те\-ме от вероятности обнаружения
$\alpha_{ji}$, $j\hm= 1,\ldots , 7$, $i\hm= 1,2,3$ (\textit{1}~---
последовательная процедура, \textit{2}~--- параллельная процедура)}
%\vspace*{-6pt}
\end{figure*}
\begin{itemize}
\item поиск в БД записи, совпадающей с записью предъявленного образа,
осуществляется методом последовательного поиска;
\item указанный в предыдущем пункте процесс поиска завершается либо по
совпадению, либо после сравнения всех записей БД (т.\,е.\ осуществляется
поиск по совпадению);
\item в одном узле не может одновременно выполняться несколько заданий;
\item задания, заставшие вычислительные ресурсы узла занятыми,
становятся в очередь на выполнение в порядке поступления;
\item задания в системе не теряются;
\item времена пребывания в узле~0 и передачи заданий по
коммуникационной среде не влияют на
 стационарные характеристики узлов
(среднее\linebreak\vspace*{-12pt}

\pagebreak

\noindent
время пребывания заданий или среднее число заданий в узлах) и
считаются равными нулю.
  \end{itemize}

  Для сокращения расчетов функция распределения вероятностей времени
сравнения образцов в узле при условии, что в нем произойдет совпадение
образа с одним из образцов, в примерах аппроксимирована равномерным
распределением.

На рис.~2 приведены графики зависимости среднего числа
заданий и среднего времени пребывания задания в системе от общего числа
однородных процессоров в узлах~1 и~2 для параллельной (\textit{1}
и~\textit{3}) и последовательной (\textit{2} и~\textit{4}) процедур при
вероятностях обнаружения лица 0,9 (\textit{1} и~\textit{2}) и 0,99
(\textit{3} и~\textit{4}). При этом процессоры распределены между
узлами~1 и~2 так, чтобы величины нагрузок на эти узлы были примерно
одинаковыми, а именно: объемы процессоров были выбраны в соотношении $1:8$
(общее число однородных процессоров в указанных узлах меняется в пределах
192--432).



	Во втором примере рассматривается система из четырех узлов (с
номерами~0, 1, 2 и 3), где в трех узлах (с номерами~1, 2 и 3) размещены базы
образцов (например, распределенные базы отпечатков пальцев), четвертый узел
(с номером~0) выполняет функции центрального узла управления (как и в
первом примере).

Считается, что в системе выполняются все условия, которые
перечислены для системы в первом примере.

Также считается, что общее число
однородных процессоров в узлах~1, 2 и~3 равно 50 и распределены они между
узлами в таких соотношениях, чтобы среднее время пребывания произвольного
задания в системе при последовательной процедуре было минимальным (см.\
таблицу).



     На рис.~3 приведены графики зависимости
суммарного среднего числа заданий и среднего времени пребывания задания в
системе.	Параметры системы имеют следующие значения: $M\hm=7$;
$A_1=\{1\}$; $A_2\hm= \{1,2\}$; $A_3\hm= \{2\}$; $A_4\hm= \{2,3\}$;
$A_5\hm= \{3\}$; $A_6\hm= \{1,3\}$; $A_7\hm= \{1,2.3\}$;
$\lambda_1\hm =0{,}2$; $\lambda_2\hm= 0{,}4$; $\lambda_3\hm= 0{,}8$;
$\lambda_4\hm= 0{,}4$; $\lambda_5\hm= 1{,}6$; $\lambda_6\hm= 0{,}4$;
$\lambda_7\hm= 2{,}6$; $\alpha_{j1}\hm=1$ для $j\hm= 1,3,5$ и
$\alpha_{j1}\hm= \{0{,}4, 0{,}5, 0{,}6, 0{,}7, 0{,}8, 0{,}9\}$, $\alpha_{j2}\hm=
\alpha_{j1}/4$, $\alpha_{j2}\hm= a_{j1}/16$  для $j\hm= 2,4,6,7$;
$n_1\overline{\mbox{tu}}_1\hm= 8$; $n_2\overline{\mbox{tu}}_2\hm= 4$;
$n_3\overline{\mbox{tu}}_3\hm=2$; время поиска образца в узлах~1, 2, 3 при условии,
что там найдется образец, совпадающий с образом, имеет равномерное
распределение на отрезках [0, 8], [0, 4], [0, 2] соответственно.



\section{Заключение}

Сравнительный анализ результатов, полученных в данной работе при
исследовании последовательной и параллельной процедур распознавания,
показывает, что в системах распознавания образов рассмотренного выше типа
при оптимальном (для последовательной процедуры) распределении
вычислительных ресурсов:
\begin{itemize}
\item последовательная процедура предпочтительней, чем параллельная, в
смысле среднего чис\-ла заданий, находящихся одновременно в сис\-теме;
\item параллельная процедура предпочтительней, чем последовательная, в
смысле среднего времени пребывания произвольного задания в сис\-теме;
\item при малых нагрузках на вычислительные ресурсы (при
$\overline{W}_{ji}\sim 0$) распределение вы\-чис\-лительных ресурсов, которое в
системе с по-\linebreak следовательной процедурой распознавания \mbox{об\-ра\-зов} балансирует
нагрузку, обеспечивает среднее время пребывания в системе, близкое к
оптимальному (в смысле выполнения условий~(\ref{e11-aga})).
\end{itemize}

Рассмотренные в работе модели могут найти применение на этапах
проектирования и эксплуатации распределенных систем распознавания
рассмотренного выше типа для оценки оперативности системы и необходимой
мощности вычислительных ресурсов в узлах системы при параллельной и
последовательной процедурах распознавания.

\vspace*{-6pt}

{\small\frenchspacing
 {%\baselineskip=10.8pt
 \addcontentsline{toc}{section}{References}
 \begin{thebibliography}{99}
\bibitem{1-aga}
\Au{Тель Ж.} Введение в распределенные алгоритмы.~--- М.: МЦНМО, 2009.
616~с.
\bibitem{2-aga}
\Au{Горелик А.\,Л., Скрипкин В.\,А.} Методы распознавания.~--- М.: Высшая
школа, 2004. 261~с.
\bibitem{3-aga}
\Au{Хрулев А.} Системы распознавания лиц. Состояние рынка. Перспективы
развития~// Сис\-те\-мы безопасности, 2012. №\,1. С.~70--72.
\bibitem{4-aga}
\Au{Бабенко Л.\,К., Ищукова Е.\,А., Сидоров~И.\,Д.} Параллельные вычисления
в криптоанализе~// Известия ЮФУ. Технические науки, 2012. Т.~137.
№\,12(137). С.~148--157.

\bibitem{6-aga} %5
\Au{Мурынин А.\,Б., Ковков Д.\,В., Лобанцов~В.\,В., Ма\-ковкин~К.\,А.,
Матвеев~И.\,А., Десятников~Д.\,Д., Чучупал~В.\,Я.} Мультимодальная
биометрия~--- перспективное решение. Объединение алгоритмов для
\mbox{повышения} надежности распознавания человека~// Сис\-те\-мы безопасности,
2006. №\,6. С.~156--160.
\bibitem{7-aga} %6
\Au{Ушмаев О.\,С., Босов А.\,В.} Реализация концепции многофакторной
биометрической идентификации в интегрированных аналитических
сис\-те\-мах~// Бизнес и безопасность в России, 2008. №\,49. С.~104--105.

\bibitem{5-aga} %7
\Au{Ушмаев О.\,С.} Проблемы распараллеливания био\-мет\-ри\-ческих вычислений
в крупномасштабных идентификационных сис\-те\-мах~// Информатика и её
применения, 2009. Т.~3. Вып.~1. С.~8--18.

\bibitem{8-aga}
\Au{Смирнов С.\,В.} Подсистема массового распознавания изображений
архивных документов~// Тр. \mbox{СПИИРАН}, 2012. Вып.~3(22). С.~234--246.
\bibitem{9-aga}
\Au{Хант Э.} Искусственный интеллект.~--- М.: Мир, 1978. 558~с.
\bibitem{10-aga}
\Au{Tamer Ezsu M., Valduriez P.} Распределенные и параллельные системы баз
данных~// Сис\-те\-мы управ\-ле\-ния базами данных, 1996. №\,4. {\sf
http://www.osp.ru/dbms/1996/04/13031501}.
\bibitem{11-aga}
\Au{Клейнрок Л.} Вычислительные системы с очередями.~--- М.: Мир, 1979.
600~с.
\bibitem{12-aga}
\Au{Корн Г., Корн Т.} Справочник по математике.~--- М.: Наука, 1974. 831~с.
 \end{thebibliography}

 }
 }

\end{multicols}

\vspace*{-12pt}

\hfill{\small\textit{Поступила в редакцию 19.06.14}}

\newpage

%\vspace*{12pt}

%\hrule

%\vspace*{2pt}

%\hrule

%\vspace*{12pt}

\def\tit{MODELS FOR COMPARATIVE ANALYSIS OF~CLASSIFICATION METHODS IN~DISTRIBUTED OBJECT RECOGNITION SYSTEMS}

\def\titkol{Models for comparative analysis of classification methods in distributed object recognition systems}

\def\aut{Ya.\,M.~Agalarov}

\def\autkol{Ya.\,M.~Agalarov}

\titel{\tit}{\aut}{\autkol}{\titkol}

\vspace*{-9pt}

\noindent
Institute of Informatics Problems, Russian Academy of Sciences,
44-2 Vavilov Str., Moscow 119333, Russian Federation


\def\leftfootline{\small{\textbf{\thepage}
\hfill INFORMATIKA I EE PRIMENENIYA~--- INFORMATICS AND
APPLICATIONS\ \ \ 2014\ \ \ volume~8\ \ \ issue\ 3}
}%
 \def\rightfootline{\small{INFORMATIKA I EE PRIMENENIYA~---
INFORMATICS AND APPLICATIONS\ \ \ 2014\ \ \ volume~8\ \ \ issue\ 3
\hfill \textbf{\thepage}}}

\vspace*{3pt}

\Abste{The paper considers recognition systems where classes are
defined by appropriate patterns located in distributed data base.
Recognition criterion is full coincidence of the presented sample with at least
one of the patterns. Parallel and sequential classification methods are
compared in terms of mean response time to recognition request and performance
requirements. The results of numerical experiments which were carried out for
multibiometric recognition systems using analytical and simulation models of
queueing networks are presented.}

\KWE{distributed recognition system; parallel and sequential classification methods;
resource allocation; queueing network}

\DOI{10.14357/19922264140306}

%\vspace*{3pt}

  \begin{multicols}{2}

\renewcommand{\bibname}{\protect\rmfamily References}
%\renewcommand{\bibname}{\large\protect\rm References}



{\small\frenchspacing
 {%\baselineskip=10.8pt
 \addcontentsline{toc}{section}{References}
 \begin{thebibliography}{99}

\bibitem{1-aga-1}
\Aue{Tel, G.} 1995. \textit{Introduction to distributed algorithms}.
Cambridge: Cambridge University Press. 546~p.
\bibitem{2-aga-1}
\Aue{Gorelik, A.\,L., and V.\,A.~Skripkin}.
2004. \textit{Metody raspoznavaniya} [Methods of recognition].
Moscow: Vysshaya Shkola. 261~p.
\bibitem{3-aga-1}
\Aue{Khrulev, A.} 2012. Sistemy raspoznavaniya lits.
Sostoyanie rynka. Perspektivy razvitiya [Face recognition systems.
State of the market. Prospects of development].
\textit{Sistemy Bezopasnosti} [Security Systems] 1:70--72.
\bibitem{4-aga-1}
\Aue{Babenko, L.\,K., E.\,A. Ishchukova, and I.\,D.~Sidorov}.
2012. Parallel'nye vychisleniya v kriptoanalize
[Parallel computation in cryptanalysis].
\textit{Izvestiya YuFU. Tekhnicheskie Nauki}
[Gerald of South Federal University. Engineering Sciences] 12(137):148--157.


\bibitem{6-aga-1}
\Aue{Murynin, A.\,B., D.\,V. Kovkov, V.\,V.~Lobantsov, K.\,A.~Makovkin,
I.\,A.~Matveev, D.\,D.~Desyatnikov, and V.\,Ya.~Chuchupal}. 2006.
Mul'timodal'naya biometriya~--- perspektivnoe reshenie. Ob"edinenie
algoritmov dlya povysheniya nadezhnosti raspoznavaniya cheloveka
[Multimodal biometrics is a promising solution. Merging algorithms
to improve the reliability of recognition of a person].
\textit{Sistemy Bezopasnosti} [Security Systems] 6:156--160.
\bibitem{7-aga-1}
\Aue{Ushmaev, O.\,S., and A.\,V.~Bosov}. 2008. Realizatsiya
kontseptsii mnogofaktornoy biometricheskoy iden\-ti\-fi\-ka\-tsii v
integrirovannykh analiticheskikh sistemakh [Implementation of multimodal
biometrics for integrated analytics].
\textit{Biznes i Bezopasnost' v Rossii} [Business and Security in Russia]
49:104--105.

\bibitem{5-aga-1} %7
\Aue{Ushmaev, O.\,S.} 2009. Problemy rasparallelivaniya biometricheskikh
vychisleniy v krupnomasshtabnykh identifikatsionnykh sistemakh
[Parallel computing in large-scale multimodal biometric systems].
\textit{Informatika i ee Primeneniya}~--- \textit{Inform. Appl}. 3(1):8--18.

\bibitem{8-aga-1}
\Aue{Smirnov, S.\,V.} 2012. Podsistema massovogo raspoznavaniya izobrazheniy
arkhivnykh dokumentov [Subsystem of mass image recognition of archival documents].
\textit{Tr. SPIIRAN} [SPIIRAS Proceedings] 3(22):234--246.
\bibitem{9-aga-1}
\Aue{Hunt, E.\,B.} 1975. \textit{Artificial intelligence}.
New York: Academic Press. 468~p.
\bibitem{10-aga-1}
\Aue{Tamer Ezsu, M., and P.~Valduriez}. 1996.
Raspredelennye i parallel'nye sistemy baz dannykh
[Distributed and parallel database systems].
\textit{Sistemy Upravleniya Bazami Dannykh} [Databases Management Systems] 4.
Available at: {\sf http://www.osp.ru/dbms/1996/04/13031501}
(accessed July 8, 2014).
\bibitem{11-aga-1}
\Aue{Kleinrock, L.} 1976. \textit{Queueing systems: Volume~II~---
Computer Applications}. New York: Wiley Interscience. 576~p.
\bibitem{12-aga-1}
\Aue{Korn, G.\,A., and T.\,M.~Korn}. 1961. \textit{Mathematical
handbook for scientists and engineers.} New York: McGraw-Hill Book Co. 943~p.


\end{thebibliography}

 }
 }

\end{multicols}

\vspace*{-6pt}

\hfill{\small\textit{Received June 19, 2014}}

\vspace*{-18pt}

\Contrl

  \noindent
  \textbf{Agalarov Yaver M.} (b.\ 1952)~--- Candidate of Science (PhD) in
technology, associate professor; leading scientist,
Institute of Informatics Problems, Russian
Academy of Sciences, 44-2 Vavilov Str., Moscow 119333, Russian Federation;  agglar@yandex.ru


\label{end\stat}

\renewcommand{\bibname}{\protect\rm Литература}