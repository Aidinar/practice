


 \newcommand{\pp}[1]{\mathbf{P}\left\{ #1 \right\}}
\newcommand {\ppp}{{\mathcal P}}
\newcommand {\ff}{{\mathcal F}}
\newcommand {\ebd}{\triangleq}
\newcommand {\G}{{\mathcal G}}
\newcommand{\me}[2]{\mathbf{E}_{ #1 }\left\{ \mathop{#2} \right\} }
\newcommand {\pc}{{\mathbf P}}
\newcommand {\diag}{\mathop{\mathrm{diag}}}
\newcommand{\mmme}[3]{\mathsf{M}_{ #1 }^{ #2 }\left\{ \mathop{#3} \right\} }
\newcommand {\col}{\mathop{\mathrm{col}}}
\newcommand{\bi}{{\mathbf I}}

\def\stat{borisov}



\def\tit{ПРИМЕНЕНИЕ АЛГОРИТМОВ ОПТИМАЛЬНОЙ ФИЛЬТРАЦИИ ДЛЯ~РЕШЕНИЯ ЗАДАЧИ
 МОНИТОРИНГА ДОСТУПНОСТИ УДАЛЕННОГО СЕРВЕРА$^*$}



\def\titkol{Применение алгоритмов оптимальной фильтрации для решения задачи
 мониторинга доступности удаленного сервера}

\def\aut{А.\,В.~Борисов$^1$}

\def\autkol{А.\,В.~Борисов}

\titel{\tit}{\aut}{\autkol}{\titkol}

{\renewcommand{\thefootnote}{\fnsymbol{footnote}} \footnotetext[1]
{Работа поддержана грантами РФФИ № 13-01-00406 и № 13-07-00408.}}


\renewcommand{\thefootnote}{\arabic{footnote}}
\footnotetext[1]{Институт проблем информатики Российской академии наук;
факультет прикладной математики и физики Московского авиационного института
(национального исследовательского университета); ABorisov@ipiran.ru}


\Abst{Задача оперативного оценивания доступности удаленного сервера
баз данных по протоколу {\sf http} сформулирована в терминах оптимальной
фильтрации состояния марковского процесса с конечным множеством состояний
по наблюдениям конечномерного мультивариантного точечного процесса (МТП).
Особенностью системы наблюдения является то, что случайная интенсивность
наблюдаемого процесса является линейной функцией  ненаблюдаемого марковского
состояния. Оценка оптимальной фильтрации определяется с помощью решения некоторой
замкнутой конечной системы дискретных рекуррентных соотношений и обыкновенных
линейных дифференциальных уравнений со случайной правой частью. Применимость
полученных теоретических результатов проиллюстрирована на примере мониторинга
состояния пары <<телекоммуникационный ка\-нал\,--\,сер\-вер баз данных>> с тремя
допустимыми состояниями по наблюдениям ответов на запросы и сообщений об ошибках.
Предложены возможные перспективы дальнейших научных исследований.}

\KW{марковские модели; случайные скачкообразные процессы; оптимальная фильтрация;
условное распределение вероятностей; системы массового обслуживания}

\DOI{10.14357/19922264140307}


\vskip 12pt plus 9pt minus 6pt

\thispagestyle{headings}

\begin{multicols}{2}

\label{st\stat}

 \section{Введение}

 Задачи определения состояний {\it сис\-тем массового обслуживания} (СМО)
 и их сетей в реальном масштабе времени по косвенным данным встречаются на
 практике в различных ин\-фор\-ма\-ци\-он\-но-те\-ле\-ком\-му\-ни\-ка\-ци\-он\-ных
 сис\-те\-мах. Оценки
 состояний нужны для характеризации СМО и идентификации их параметров~\cite{IvKK_82, KR_88}.
 Оценивание состояний СМО также является промежуточной процедурой при решении задачи
 оптимального управления потоком передачи данных~[3--5]
 относительно различных критериев. Наконец, оперативное оценивание состояния
 отдельных элементов и систем обслуживания в целом может потребоваться при
 решении разнообразных задач балансировки нагрузки в децентрализованных
 информационных сетях и СМО с разделяемыми ресурсами~\cite{AKU_05,AAP_11}
 при наличии только косвенных данных об их состоянии. Соответствующие
 алгоритмы должны служить основанием для создания эффективных программных
 продуктов.

 При разработке специального программного обеспечения информационных веб-пор\-та\-лов~\cite{Bosov_09, Bosov_11}
 необходимо рационально распределять пользовательские запросы, поступающие на портал, между различными
 информационными серверами, содержащими идентичный контент. Сделать это можно, %\linebreak
  зная ресурсоемкость и
 алгоритмы обработки\linebreak индивидуаль\-ных запросов, текущую загрузку и мощ\-ности серверов,
 а также характеристики и топологию телекоммуникационных каналов. На практике
 подобная детерминированная постановка задачи оптимизации не реализуема из-за
 отсутствия большинства перечисленных данных. Доступная информация подразделяется
 на две составляющие: априорные данные и статистическую информацию.

 Прежде всего, СМО <<телекоммуникационный ка\-нал\,--\,сер\-вер>> является
 разделяемым ресурсом: наряду с текущим запросом, она обслуживала и обслуживает
 также и множество других. Марковская модель смены состояний этой СМО описывает
 ее с достаточной степенью адекватности. Очевидно, что одним из состояний будет
 <<отсутствие связи>>, а остальные соответствуют наличию связи и разной степени
 загруженности СМО. Интенсивности перехода СМО в различные со\-сто\-яния могут быть
 оценены по предварительно полученной статистической информации~\cite{IvKK_82, KR_88}.

 Каждый индивидуальный запрос к исследуемой СМО обслуживается случайное время, по
 истечении которого пользователю выдается либо сообщение об ошибке, либо предметный
 ответ на запрос.  Сообщения об ошибке бывают различными и обычно связаны с
 истечением ка\-ко\-го-ли\-бо времени ожидания~--- тайм-аутом. Распределение времени
 получения ответа на запрос зависит от особенностей функционирования
 ин\-тер\-нет-сер\-ве\-ра~\cite {ONA_04}, семантики самого запроса~\cite{OV_11}
 и объема базы данных. Вероятности появления сообщений об ошибках и распределение
 времени получения ответа зависят от состояния СМО, в котором она пребывает в
 процессе всего времени обработки запроса. Все эти распределения можно
 предварительно идентифицировать по статистическим данным. При этом
 максимальное время ожидания ответа или сообщения об ошибке всегда
 ограничено известным тайм-аутом.

 Запросы к серверу образуют процесс вос\-ста\-нов\-ле\-ния общего вида, не
 являющийся ни пуассо\-новским, ни даже процессом Кокса. Процесс исходящих
 предметных ответов или сообщений об ошибках порождается входным процессом
 {\it разнородных} запросов и также является обобщенным процессом восстановления.

 Прикладная задача заключается в определении оценок состояния СМО
 <<телекоммуникационный ка\-нал\,--\,сер\-вер>> по наблюдениям входного
 потока запросов и выходного потока предметных ответов и сообщений об ошибках.

Основной целью данной работы является формализация и решение поставленной
практической задачи оценивания. В~разд.~2 она сформулирована в терминах
оценивания состояний динамических систем наблюдения, описываемых стохастическими
дифференциальными уравнениями с мартингалами в правой части. Ненаблюдаемый процесс,
определяющий состояние системы, предполагается марковским с конечным множеством\linebreak
со\-сто\-яний. Процесс наблюдений является мультивариантным и также принимает
значения из\linebreak конечного множества. Основной особенностью предложенной модели
является представление наблюдений в форме мультивариантного точечного процесса,
случайная интенсивность которого является функционалом от предыдущих состояний и\linebreak
наблюдений. При этом зависимость интенсив\-ности от не\-наблю\-да\-емо\-го со\-сто\-яния
предполагается линейной. В~разд.~3 проведен анализ пред\-ложенной сис\-те\-мы
наблюдения и представлены \mbox{базовые} выводы о распределении процесса наблюдений.
Основной результат~--- уравнения оптимальной фильтрации~--- доказан в разд.~4.
Раздел~5 содержит иллюстративный численный пример. В~нем дано решение задачи
оценивания доступности удаленного сервера баз данных по протоколу {\sf http}
по результатам запросов к нему. В~задаче предполагается, что СМО
<<телекоммуникационный ка\-нал\,--\,сер\-вер>> находится в одном из трех состояний:
 <<связи нет>>, <<сервер не загружен>> и <<сервер загружен>>,~---
 а сообщения могут быть двух видов: <<предметный ответ>> и <<сообщение об ошибке>>.
 Заключительные замечания и возможные перспективы развития предложенных исследований
 представлены в разд.~6.

 \section{Формальная постановка задачи}

 На полном вероятностном пространстве с фильтрацией
 $(\Omega,\ff,\ppp,\{\ff_t\}_{t \in \mathbb{R}_+})$ задан ненаблюдаемый
 \textit{марковский скачкообразный процесс} (МСП) $\{X(t)\}_{t \in \mathbb{R}_+}$~\cite{ElliottAM_94}
 \begin{equation}
 X(t) = X_0 + \int\limits_0^t \Lambda^{\top}(s)X(s)\,ds+M^X(t)\,,
 \label{eq:state}
 \end{equation}
 где
% \begin{itemize}
 %\item[$\bullet$]
 $X(t) \in \mathbb{S}_N$~---
 со\-сто\-яние процесса ($\mathbb{S}_N \ebd$\linebreak $\ebd\;\{e_1, \ldots , e_N\} \subset
 \mathbb{R}^{N}$~--- множество единичных век\-тор-столб\-цов евклидова пространства~$\mathbb{R}^{N}$);
 %\item[$\bullet$]
 $\{\Lambda(t)\}_{t \in \mathbb{R}_+}$~--- матрица
 интенсивностей МСП ($\Lambda(\cdot):\quad \mathbb{R}_+ \to
 \mathbb{R}^{N \times N}$);
% \item[$\bullet$]
$\{M^X(t)\}_{t \in \mathbb{R}_+}$~---
 $\ff_t$-со\-гла\-со\-ван\-ный мартингал, $M^X(t)\hm \in \mathbb{R}^N$.
% \end{itemize}
 Здесь и далее для обозначения множества, по которому
 выполняется интегрирование, используется обозначение
 $\int\limits_a^b \cdots \ebd \int\limits_{(a,b]}\cdots$, в противном случае
 множество будет указываться явным образом.

 В качестве наблюдений выступает \textit{мультивариантный точечный
 процесс}~\cite{LS_86} $(\tau_n,Y_n)_{n \in \mathbb{N}}$:
 \begin{itemize}
  \item $\{\tau_n\} $~--- возрастающая последовательность
  марковских моментов;
  \item $Y_n \in \mathbb{S}_M$~--- значение наблюдения в
  момент $\tau_n$ ($\mathbb{S}_M \ebd\{f_1, \ldots , f_M\}
  \subset\mathbb{R}^{M}$~--- множество единичных век\-тор-столб\-цов
  евклидова пространства~$\mathbb{R}^{M}$).
 \end{itemize}

 С наблюдениями $\{(\tau_n,Y_n)\}_{n \in \mathbb{N}}$ связана неубывающая
 последовательность $\sigma$-под\-алгебр $\{\G(n)\}_{n \in \mathbb{N}}$:
 $\G(n) \ebd \sigma\{(\tau_k,Y_k), \; 1 \hm\leqslant k \hm\leqslant n\}$;
 помимо этого полагается, что $\G(0) \ebd \{\emptyset, \Omega\}$.

 Наблюдениям в форме МТП однозначно соответствует случайный процесс $Y(t)$ в
 непрерывном времени со считающими компонентами
 \begin{equation*}
 Y(t) \ebd \sum\limits_{n \in \mathbb{N}}Y_n\mathbf{I}_{[\tau_{n},+\infty)}(t):
% \label{eq:obs_1}
 \end{equation*}
 $j$-я компонента $Y^j(t)$ ($j\hm=\overline{1,M}$) равна числу наблюдений с
 исходом $j$-го типа ($Y_k\hm=f_j, \; \tau_k \hm\leqslant t$), появившихся
 на отрезке времени $[0,t]$.

  По непрерывному процессу $Y(t)$ можно построить естественный поток
  $\sigma$-под\-алгебр $\G_t \ebd$\linebreak $\ebd\;\sigma\{Y(s), \; s \hm\leqslant t\}$,
  при этом $\forall \; n \in \mathbb{N}$ верно тож\-де\-ст\-во $\G_{\tau_n}
  \equiv \G(n)$~\cite{JSh_94}.

 Для процесса $Y(t)$ существует мартингальное представление~\cite{El_86}.
 По условию рассматриваемой задачи оно имеет следующий вид (далее в изложении
 полагается $\tau_0\hm \ebd 0$):
 \begin{multline}
 Y(t) = \sum\limits_{n \in \mathbb{N}}\int\limits_{\tau_{n-1} \wedge t}^{\tau_{n}
 \wedge t}  \left(
 \phi^0_{n}(s)X(\tau_{n-1})+{}\right.\\
\left. {}+\phi^1_{n}(s)X(s-)
 \right)\Phi_{n}(ds)
 +\sum\limits_{n \in \mathbb{N}}M^Y_n(t)\,.
 \label{eq:obs_2}
 \end{multline}
 Здесь
% \begin{itemize}
 % \item[$\bullet$]
 $\phi^i_{n}(t)\hm=\phi^i_{n}(\omega,t)$ ($i\hm=\overline{0,1}$, $n \hm
  \in \mathbb{N}$), $\phi^i_{n}(\cdot): \quad \Omega \times \mathbb{R}_+
  \to \mathbb{R}^{M \times N}$~--- $\G(n-1)$-из\-ме\-ри\-мые векторные процессы;
%  \item[$\bullet$]
$\Phi_{n}(dt)\hm=\Phi_{n}(\omega,dt)$~--- $\G(n-1)$-из\-ме\-ри\-мые
  не\-от\-ри\-ца\-тель\-ные стохастические $\sigma$-ко\-неч\-ные меры;
%  \item[$\bullet$]
$M^Y_n(t)$~--- $\ff_t$-со\-гла\-со\-ван\-ные
  векторные мартингалы, $M^Y_n(t) \hm\in \mathbb{R}^{M}$.
% \end{itemize}

\textit{Задача оптимальной фильтрации} состояния~$X$ по наблюдениям
МТП $\{(\tau_n,Y_n)\}$ заключается в вычислении \textit{условного математического
ожидания} (УМО):
 $$
 \widehat{X}(t) \ebd \me{}{X(t)|\G_t}\,.
 $$


 Благодаря предложенной модели наблюдений~(\ref{eq:obs_2}) данная сис\-те\-ма
 имеет ряд существенных особенностей.

 Во-пер\-вых, в отличие от традиционного
 вероятностного определения наблюдаемого МТП с \mbox{помощью} семейства условных
 конечномерных распределений~\cite{LS_86}, в данной работе модель наблюдений
 задана в тех же терминах, что и динамика состояния, т.\,е.\ в форме мартингального
 пред\-став\-ле\-ния. Помимо единообразия в описании данные уравнения представляют
 <<кинематику>> наблюдений более наглядно. Подобное представление
 также иллюстративно для практиков. Действительно, $j$-я компонента
 подынтегрального выражения~(\ref{eq:obs_2}) определяет вероятность
 немедленного появления в момент времени~$s$ $n$-го наблюдения $Y_n$ с $j$-м
 исходом при условии, что до момента~$s$ наблюдение~$Y_n$ не появилось~\cite{El_86},
 т.\,е.\
\begin{multline*}
 f^{\top}_j\left[
 \phi^0_{n}(s)X(\tau_{n-1})+\phi^1_{n}(s)X(s-)
 \right]\Phi_{n}(ds) = {}\\
 {}=\pc\left\{\tau_{n} \in [s,s+ds),Y_n=f_j | \G(n-1)\wedge
 \left\{\tau_{n}\geqslant s\right\}\right\}\,.
 \end{multline*}

 Во-вторых, модель~(\ref{eq:obs_2}) является достаточно общей.
 Она включает классический случай нестационарного пуассоновского процесса,
 интенсивность которого зависит от текущего состояния систе\-мы~$X$~\cite{WongH_85}.
 Она также описывает ситуацию, когда интенсивность появления наблюдений и их значения
 зависят от состояния системы только в момент предыдущего наблюдения~\cite{bms_14}.

  В-третьих, первое слагаемое в (\ref{eq:obs_2})~--- компенсатор~---
  является случайной разрывной функцией. Это свойство компенсатора не
  позволяет преобразовать наблюдения $Y(t)$ в процесс с независимыми
  приращениями ни с помощью гирсановской замены вероятностной меры, ни
  с по\-мощью замены времени~\cite{JSh_94}.
Это ведет к тому, что поставленная задача оценивания может быть решена
только с помощью общей формулы фильтрации семимартингала по наблюдениям
семимартингала~\cite{LS_86}.

 \section{Определение вероятностных характеристик наблюдений}

 Прежде чем приступать к решению задачи оптимальной фильтрации, необходимо
 формализовать свойства системы наблюдения. Далее в работе предполагается,
 что для~(\ref{eq:state}), (\ref{eq:obs_2}) выполнены следующие условия:
  \begin{itemize}
 \item[(a)] все траектории процессов $X(t)$ и $Y(t)$  непрерывны справа;
 \item[(б)] общий поток $\sigma$-алгебр $\{\ff_t\}_{t\in \mathbb{R}_+}$
 порожден только процессами $X(t)$ и $Y(t)$, т.\,е.\
 $\ff_t\hm=\sigma\{X(s), \; Y(s):\; 0 \hm\leqslant s\hm \leqslant t\}$;
 помимо этого поток $\{\ff_t\}_{t\in \mathbb{R}_+}$ непрерывен справа, т.\,е.\
 $\bigcap\limits_{s: \; s>t}\ff_s \hm= \ff_t$ $\forall \; t\hm>0$;
 \item[(в)] для МСП $X(t)$ вектор начального распределения $p(0) \ebd \me{}{X_0}$
 и матрица интенсивностей переходов $\Lambda(t)$ известны; при этом все компоненты
 $\Lambda(t)$ являются кусочно-непрерывными функциями времени~$t$;
 \item[(г)] моменты поступления наблюдений $\{\tau_n\}_{n \geqslant 0}$
 удовлетворяют условиям
\begin{equation*}
   \hspace*{-3mm}0\ebd \tau_0 <\tau_1<\tau_2<\cdots\ \mbox{и}\
    \lim_{k\to\infty}\tau_n=+\infty\quad\mbox{$\pc$--п.~н.;}
\end{equation*}
при этом определено подмножество множества натуральных чисел
$\mathbb{N}_1 \subset \mathbb{N}$, обладающее следующим свойством:
$\forall \; n \hm\in \mathbb{N}_1$ определена такая $\G(n-1)$-из\-ме\-ри\-мая
случайная величина $T_n\hm=T_n(\tau_1,\ldots,\tau_{n-1})$, что
$$
\hspace*{-0.5mm}\mathbf{P}\left\{ \tau_{n-1} < \tau_{n} \leqslant T_n < +\infty\vert \G(n-1)\right\}=1
\ \ \forall\  n\in \mathbb{N}_1\hspace*{-1.9903pt}
$$
или
$$
\mathbf{P}\left\{ \tau_n>v\vert \G(n-1)\right\} =1\ \ \forall\ n\in \mathbb{N}\backslash \mathbb{N}_1\
\mbox{и}\ v>0\,;
$$
 \item[(д)] мартингалы в уравнении состояния и наблюдениях сильно ортогональны,
 т.\,е.
 $$
  [M^X,M^{Y}_{n}]_t \ebd \sum\limits_{s \leqslant t}
  \Delta M^X(s) \left(\Delta M^{Y}_{n}(s)\right)^{\top} \equiv 0
  $$
 $\pc$--п.~н. $\forall \; t \in \mathbb{R}_+$ и $n\hm \in \mathbb{N}$;
 \item[(е)]
 для меры $\Phi_n(\cdot)$ верно разложение
 $$
 \Phi_n(\cdot) = \Phi^c_n(\cdot) + \Phi^d_n(\cdot)\,,
 $$
 где $\Phi^c_n(\cdot)$~--- абсолютно непрерывная по мере Лебега,
 а $\Phi^d_n(\cdot)$~--- дискретная (считающая) составляющие, а также
 $\Phi_n([0,\tau_{n-1}]) \hm\equiv 0$ $\pc$--п.~н.; $\forall \; n \hm\in \mathbb{N}$
 меры $\Phi_n(\cdot)$ содержат лишь конечное множество скачков
 $\mathbb{U}_n \ebd \{u_{\ell}^n\}_{{\ell}=\overline{1,K}_n}$~---
 $\G(n-1)$-из\-ме\-ри\-мых моментов.

 Для функций $\phi_n^i(\cdot)$ $\pc$--п.~н. выполняются условия:
 \begin{equation}
 \left.
 \begin{array}{l}
 \left(\phi^i_n(s)\right)_{kj} \geqslant 0\,,\\[6pt]
 \hspace*{2mm}\forall \; i=\overline{0,1}, \; j=\overline{1,N}, \; k=\overline{1,M},\; s \in \mathbb{R}_+\,;\\[9pt]
\displaystyle \mathbf{1}_M^{\top}\left(
\phi^0_n(s)X(\tau_{n-1}) + \phi^1_n(s)X(s-) \right) > 0\,,\\[6pt]
\hspace*{2mm}\Phi_n-\mbox{п. в. на \;}
  (\tau_{n-1}, T_n] \; \forall \; n \in \mathbb{N}_1;\\[9pt]
  \displaystyle
  \mathbf{1}_M^{\top}\left(\phi^0_n(s)X(\tau_{n-1}) + \phi^1_n(s)X(s-) \right) > 0\,,\\[6pt]
  \hspace*{2mm}\Phi_n-\mbox{п. в. на \;}
  (\tau_{n-1}, + \infty) \; \forall \; n \in \mathbb{N} \backslash
  \mathbb{N}_1;\\[9pt]
 \displaystyle
  \mathbf{1}_M^{\top}\left(\phi^0_n(u_{\ell}^n)X(\tau_{n-1}) +{}\right.\\[6pt]
\hspace*{10mm}\left.{}+  \phi^1_n(u_{\ell}^n)X(u_{\ell}^n-) \right) \Phi_n^d(\{u_{\ell}^n\}) \leqslant 1;
  \\[9pt]
  \displaystyle
  \exp \Bigg\{ %\left\{
 -\!\!\!\int\limits_{\tau_{n-1}}^{T_n} \hspace*{-1mm}\mathbf{1}_M^{\top}\left(
 \phi^0_n(s)e_{k_1}+{}\right. %\right.
 \\
\hspace*{5mm}\left. {}+\phi^1_n(s)e_{k_2}
 \right)\Phi^c_n(ds)
% \vphantom{\int\limits_{\tau_{n-1}}^{T_n}}
 \Bigg\} %\right\}
 \times \\[9pt]
 \displaystyle
 \hspace*{8mm}\times \!\! \prod\limits_{\tau_{n-1} < u_n^{\ell} \leqslant T_n}\!\!\!\!\left[
 1- \mathbf{1}_M^{\top}\left(
 \phi^0_n(u_n^{\ell})e_{k_3}+{}\right.\right.\\[6pt]
\hspace*{12mm}\left.\left. {}+\phi^1_n(u_n^{\ell})e_{k_4}
 \right)\Phi^d_n(\{u_n^{\ell}\})
 \right]=0\,, \\[6pt]
 \hspace*{24mm}\forall \; k_i=\overline{1,N}, \; n \in \mathbb{N}_1\,;\\[9pt]
 \displaystyle
  \exp \Bigg\{
 -\!\!\int\limits_{(\tau_{n-1},+\infty)} \hspace*{-3mm}\mathbf{1}_M^{\top}\left(
\phi^0_n(s)e_{k_1}+{}\right.\\[6pt]
\left. {}+\phi^1_n(s)e_{k_2}
 \right)\Phi^c_n(ds)
 \vphantom{\int\limits_{(\tau_{n-1},+\infty)}}
 \Bigg\} \times \\[6pt]
 \hspace*{12mm}\displaystyle
 \times \prod\limits_{\tau_{n-1} < u_n^{\ell}}\left[ 1-
 \mathbf{1}_M^{\top}\left(
 \phi^0_n(u_n^{\ell})e_{k_3}+{}\right.\right.\\[6pt]
\hspace*{14mm}\left.\left. {}+\phi^1_n(u_n^{\ell})e_{k_4}
 \right)\Phi^d_n(\{u_n^{\ell}\})
 \right]=0\,, \\[6pt]
\hspace*{18mm} \forall \; k_i=\overline{1,N}, \; n \in \mathbb{N} \backslash \mathbb{N}_1\,,
 \end{array}\!
 \right\}\!\!
 \label{eq:cond_1}
 \end{equation}
 где $\mathbf{1}_L$~--- $L$-мер\-ный век\-тор-стол\-бец, составленный из единиц.
\end{itemize}

  Далее, необходимо построить совместное распределение
  наблюдений $\{(\tau_n,Y_n)\}$ и состояния~$X$. Распределение процесса~$X$
  известно~\cite{y_77}. Пусть $\{\sigma_n\}_{n \in \mathbb{N}}$~---
  последовательность моментов скачков~$X$, тогда $\forall \; t \hm\in \mathbb{R}_+$
  и $j \hm \neq i$

  \noindent
 \begin{multline*}
 \pc\{\sigma_{n+1} \leqslant t, X(\sigma_{n+1})=e_j|\ff_{\sigma_{n}}\}=
\mathbf{I}_{[\sigma_{n},+ \infty)}(t)\times{}\\
{}\times
 \sum\limits_{i=1}^N e_i^{\top}X(\sigma_n)
 \int\limits_{\sigma_n}^t\lambda_{ij}(s)\exp\left[\,
 \int\limits_{\sigma_n}^s\lambda_{ii}(u)\,du\right]ds\,.
% \label{eq:state_distr}
 \end{multline*}
 Осталось лишь построить условное распределение $\{(\tau_n,Y_n)\}$ относительно~$X$.

 Прежде всего, с каждым МТП $\{(\tau_n,Y_n)\}$ взаимно однозначным образом связана
 стохастическая мера~$\mu$:
 \begin{equation*}
 \mu(\omega, dt, dy) \ebd \sum\limits_{n \in \mathbb{N}}
 \delta_{(\tau_n(\omega),Y_n(\omega))}(dt,dy)\,.
% \label{eq:st_mes}
 \end{equation*}
 Для вывода уравнений фильтрации необходимо определить компенсатор
 $\nu(\omega, dt, dy)$ меры $\mu(\omega, dt, dy)$, т.\,е.\
 такую $\G_t$-пред\-ска\-зу\-емую стохастическую меру, для которой равенство
 \begin{multline*}
  \me{}{\displaystyle \int\limits_{\mathbb{R}_+\times \mathbb{R}^{M}}\psi(\omega,t,y)
  \mu(\omega, dt, dy)}={}\\
  {}=
  \me{}{\displaystyle \int\limits_{\mathbb{R}_+
  \times \mathbb{R}^{M}}\psi(\omega,t,y)\nu(\omega, dt, dy)}
\end{multline*}
выполняется для любого неотрицательного ограниченного $\G_t$-пред\-ска\-зу\-емо\-го
случайного процесса $\psi(\omega,t,y)$.
 В дальнейшем изложении в обозначении процессов $\psi(\omega,t,y)$ и стохастических
 мер $\mu(\omega, dt, dy)$ для краткости зависимость от случайного события~$\omega$
 будет опускаться, т.\,е.\ $\psi(t,y) \ebd$\linebreak $\ebd\;\psi(\omega,t,y)$ и
 $\mu(dt, dy) \ebd \mu(\omega, dt, dy)$. Будет также использоваться
 обозначение $\widehat{X}(t,u) \ebd\me{}{X(t)|\G_u}$ ($u>t$) для
 оценки оптимального сглаживания состояния в фиксированной точке~$t$
 по наблюдениям на отрезке $[0,u]$.


 Как и в случае классического фильтра Кал\-ма\-на--Бью\-си~\cite{KalmanB_60}
 и обобщения фильтра Вонэма~[16], оценку оптимальной фильтрации в данной задаче удается
 представить в конечномерном виде: в каждый момент времени~$t$ искомая оценка
 $\widehat{X}(t)$ выражается с помощью решения некоторой конечной системы
 дифференциальных и алгебраических уравнений. Это означает, что условное
 распределение $X(t)$ относительно $\G_t$ характеризуется некоторым конечным
 набором достаточных статистик. В~данной системе наблюдений такими статистиками
 являются матричнозначные функции
 $\overline{Z}_n(t): \; [\tau_{n-1},\tau_n) \hm\to \mathbb{R}^{N \times N}$,
 $n \hm\in \mathbb{N}$:
 \begin{equation*}
 \overline{Z}_n(t) \ebd \me{}{X(\tau_{n-1})\widetilde{I}_n(t)X^{\top}(t)|\G(n-1)}\,,
% \label{eq:Z_bar}
 \end{equation*}
 где
 \begin{equation*}
 \widetilde{I}_n(t) \ebd \me{}{\mathbf{I}_{[\tau_{n-1},\tau_{n})}(t)|\ff^X \vee \G(n-1)}\,.
% \label{eq:I_tilde}
 \end{equation*}

 Следующее вспомогательное утверждение определяет искомое условное
 распределение $\{(\tau_n,Y_n)\}$ относительно его предыстории и МСП~$X$,
 уравнение для процессов $\{\overline{Z}_n(\cdot)\}_{n \in \mathbb{N}}$, а
 также компенсатор~$\nu$ меры~$\mu$.

\medskip

\noindent
\textbf{Лемма 1.}
 \begin{enumerate}
 \item
\textit{ Условная функция распределения момента скачка $\tau_n$ относительно
$\ff^X  \bigvee \G (n-1)$ определяется формулой} ($t \hm> \tau_{n-1}$):
 \begin{multline}
 \pp{\tau_n \leqslant t | \ff^X  \vee \G (n-1)} ={}\\
 {}=
 \pp{\tau_n \leqslant t | \ff_{t}^X  \vee \G (n-1)} ={} \\
{} = 1-\exp \left\{
 -\int\limits_{\tau_{n-1}}^t \hspace*{-1mm}\mathbf{1}_M^{\top}\left(
 \phi^0_n(s)X(\tau_{n-1})+{}\right.\right.\\
\left.\left. {}+\phi^1_n(s)X(s-)
 \right)\Phi^c_n(ds)
 \vphantom{\int\limits_{\tau_{n-1}}^t}
 \right\} \times {}\\
{} \times \prod\limits_{\tau_{n-1} < s \leqslant t}\left[
 1-
 \mathbf{1}_M^{\top}\left(
 \phi^0_n(s)X(\tau_{n-1})+{}\right.\right.\\
\left.\left. {}+\phi^1_n(s)X(s-)
 \right)\Phi^d_n(\{s\})
 \right]\,.
 \label{eq:tau_dist}
 \end{multline}
 \item \textit{Совместное условное распределение момента
 скачка $\tau_n$ и значения наблюдения $Y_n$
 относительно $\ff^X  \bigvee \G (n-1)$ определяется формулой}
 ($t\hm>\tau_{n-1}$):
 \begin{multline*}
 \pp{\tau_n \leqslant t, \; Y_n = f_i | \ff^X  \vee \G (n-1)} = \\ =\pp{\tau_n \leqslant t, \; Y_n = f_i | \ff_{t}^X  \vee \G (n-1)} = \\
 =
 \int\limits_{\tau_{n-1}}^{t}f_i^{\top}\left(\phi^0_{n}(s)X(\tau_{n-1})+\phi^1_{n}(s)X(s-)
 \right) \times {}\\
 {}\times
 \exp \left\{
 -\int\limits_{\tau_{n-1}}^s \mathbf{1}_M^{\top}\left(
 \phi^0_n(u)X(\tau_{n-1})+{}\right.\right.\\
\left.\left. {}+\phi^1_n(u)X(u-)
 \right)\Phi^c_n(du)
 \vphantom{\int\limits_{\tau_{n-1}}^s}
 \right\} \times {}\\
{} \times \prod\limits_{\tau_{n-1} < u < s}\left[
 1-
 \mathbf{1}_M^{\top}\left(
 \phi^0_n(u)X(\tau_{n-1})+{}\right.\right.\\
\left.\left. {}+\phi^1_n(u)X(u-)
 \right)\Phi^d_n(\{u\})
 \right]
 \Phi_{n}(ds)\,.
% \label{eq:tau_Y_dist}
 \end{multline*}
 \item
\textit{ На каждом полуинтервале $[\tau_{n-1},\tau_{n})$
статистика $\{\overline{Z}_n(\cdot)\}_{n \in \mathbb{N}}$ является единственным
решением системы обыкновенных линейных дифференциальных уравнений с разрывной
случайной правой \mbox{частью}:}

\noindent
 \begin{multline}
  \!\!\overline{Z}_n(t)=\diag(\widehat{X}(\tau_{n-1}))
 + \int\limits_{\tau_{n-1}}^t \overline{Z}_n (s-)\Lambda(s)ds
 -{}\\
 {} -\int\limits_{\tau_{n-1}}^t
 \left[\diag(\mathbf{1}^{\top}_M\phi^{0}_n(s))
 \overline{Z}_n (s-)+{}\right.\\
\left. {}+\overline{Z}_n (s-)\diag(\mathbf{1}^{\top}_M\phi^{1}_n(s))
 \right]\Phi_n(ds)\,.
 \label{eq:Z_sys}
 \end{multline}
 \item
\textit{ Компенсатор $\nu$ меры $\mu$ равен}
 \begin{multline}
 \nu(dt,dy) = \\
 =\sum\limits_{n \in \mathbb{N}} \sum\limits_{j=1}^M
 \mathbf{I}_{(\tau_{n-1},\tau_n]}(t) y^{\top}\left(\phi^0_{n}(t)
 \overline{X}_n(\tau_{n-1},t-)+{}\right.\\
\left. {}+\phi^1_{n}(t)\overline{X}_n(t-)
 \right)\delta_{\{f_j\}}(dy)\Phi_n(dt)\,,
 \label{eq:nu}
 \end{multline}
 \textit{где процессы}
 \begin{multline}
  \overline{X}_n(\tau_{n-1},t) \ebd \fr{1}{
\me{}{\widetilde{I}_n(t)|\G(n-1)}}
\times{}\\
{}\times\me{}{X(\tau_{n-1})\widetilde{I}_n(t)|\G(n-1)}
\label{eq:X_0_bar_0}
 \end{multline}
 \textit{и}
 \begin{multline}
 \overline{X}_n(t) \ebd \frac{1}{
\me{}{\widetilde{I}_n(t)|\G(n-1)}
}\times{}\\
{}\times \me{}{X(t)\widetilde{I}_n(t)|\G(n-1)}
\label{eq:Xbar_0}
 \end{multline}
\textit{выражаются через статистики} $\overline{Z}_n(t)$:
 \begin{equation}
 \left.
 \begin{array}{rl}
 \overline{X}_n(\tau_{n-1},t) &=
\fr{1}{\mathbf{1}_N^{\top}\overline{Z}_n(t)\mathbf{1}}_N\overline{Z}_n(t)\mathbf{1}_N\,;
\\[6pt]
 \overline{X}_n(t) &=
\fr{1}{\mathbf{1}_N^{\top}\overline{Z}_n(t)\mathbf{1}_N}\overline{Z}^{\top}_n(t)\mathbf{1}_N\,.
\end{array}
\right\}
\label{eq:Xbar}
 \end{equation}
 \end{enumerate}

\medskip

\noindent
 Д\,о\,к\,а\,з\,а\,т\,е\,л\,ь\,с\,т\,в\,о\ \ леммы~1 приведено в приложении.

\smallskip
 Лемма~1 позволяет получить важное следствие, необходимое в дальнейшем для
 обеспечения того, чтобы искомая оценка фильтрации являлась регулярной
 версией условного распределения. Для этого рассмотрим множества $\mathcal{O}_n^i$
 ($n \hm\in \mathbb{N}$, $i\hm=\overline{1,M}$)
 \begin{multline*}
 \mathcal{O}_n^i \ebd \left\{ \omega \in \Omega: \; \tau_n
 \geqslant \tau_{n-1}, \right.\\
\left. Y_n= f_i, \;
 f_i^{\top}\left(\phi^0_{n}(\tau_{n})X(\tau_{n-1})+\phi^1_{n}(\tau_{n})X(\tau_{n}-)
 \right)=0,
 \right.\hspace*{-0.79958pt}
 \\
 \left.
 \mathbf{1}_M^{\top}\left(\phi^0_{n}(\tau_{n})X(\tau_{n-1})+\phi^1_{n}(\tau_{n})X(\tau_{n}-)
 \right)>0\right \}
 \end{multline*}
 и вычислим условные вероятности $\mathbf{P}\{\mathcal{O}_n^i|\ff^X \;\vee$\linebreak
 $\vee\; \G(n-1)\}$. С использованием утверждения~2 леммы~1
 это происходит следующим образом:

 \noindent
 \begin{multline*}
 \pp{\mathcal{O}_n^i|\ff^X \vee \G(n-1)} = {}\\
 {}=
 \me{}{I_{\mathcal{O}_n^i}(\omega)|\ff^X \vee \G(n-1)}={} \\
 \hspace*{-5mm}{}=
\int\limits_{\left\{
 \substack{{s \geqslant \tau_{n-1}: \;\;
  f_i^{\top}\left(\phi^0_{n}(s)X(\tau_{n-1})+\phi^1_{n}(s)X(s-)
 \right)=0;}\\
{ \mathbf{1}_M^{\top}\left(\phi^0_{n}(s)X(\tau_{n-1})+\phi^1_{n}(s)X(s-)
 \right)>0}}
 \right\} }\hspace*{-30mm}\mathbf{P}\left\{
 \vphantom{\ff^X}
 \tau_n \in ds,\right.\\
 \left. Y_n=f_i|\ff^X \vee \G(n-1)\right\}={}\\
{}= \int\limits_{\left\{
\substack{{s \geqslant \tau_{n-1}: \;\;
 f_i^{\top}\left(\phi^0_{n}(s)X(\tau_{n-1})+\phi^1_{n}(s)X(s-)
 \right)=0,} \\
{ \mathbf{1}_M^{\top}\left(\phi^0_{n}(s)X(\tau_{n-1})+\phi^1_{n}(s)X(s-)
 \right)>0}}\right\}} \hspace*{-30mm} f_i^{\top}\left(
 \phi^0_{n}(s)X(\tau_{n-1})+{}\right.\\
\left. {}+\phi^1_{n}(s)X(s-)
 \right)  \exp \left\{
 -\int\limits_{\tau_{n-1}}^s \mathbf{1}_M^{\top}\left(
 \phi^0_n(u)X(\tau_{n-1})+{}\right.\right.\\
\left.\left. {}+\phi^1_n(u)X(u-)
 \right)\Phi^c_n(du)
 \vphantom{\int\limits_{\tau_{n-1}}^s}
 \right\} \times {}\\
{} \times \prod\limits_{\tau_{n-1} < u < s}\left[
 1-
 \mathbf{1}_M^{\top}\left(
 \phi^0_n(u)X(\tau_{n-1})+{}\right.\right.\\
\left.\left. {}+\phi^1_n(u)X(u-)
 \right)\Phi^d_n(\{u\})
 \right]
 \Phi_{n}(ds) =0\,.
  \end{multline*}
 Таким образом, доказано

 \smallskip

 \noindent
 \textbf{Следствие 1.}\
 $\forall \; n \in \mathbb{N}$ и $i=\overline{1,M}$ $\pc-$п.~н.\ верно равенство
$$
 \pp{\mathcal{O}_n^i|\G(n-1)\vee \ff^X}=0\,.
$$

\smallskip

 С практической точки зрения смысл следствия вполне прозрачен: если в
 момент времени~$t$ появилось наблюдение $Y_n$ и интенсивность появления
 исхода наблюдения~$f_i$ равна~0, т.\,е.\
 $f_i^{\top}\left(\phi^0_{n}(t)X(\tau_{n-1})\hm+\phi^1_{n}(t)X(t-)
 \right)\hm=0$, то вероятность того, что $Y_n\hm=f_i$, равна~0.

 Утверждения леммы~1 также позволяют
 сделать важные выводы как относительно вида компенсатора~$\nu$, так и
 относительно искомой оценки фильт\-рации $\widehat{X}(t)$.

\smallskip

\noindent
\textbf{Теорема 1.}\
 \begin{enumerate}
 \item
 \textit{Оценка $\widehat{X}(t)$ оптимальной фильтрации состояния $X(t)$ имеет вид}:
 \begin{multline}
 \widehat{X}(t) = \sum\limits_{n \in \mathbb{N}}\mathbf{I}_{[\tau_{n-1},\tau_n)}(t)\times{}\\
{}\times \mathbf{E}\left\{X(t)|\G(n-1)
\vee \{\tau_n > t\} \right\}\,,
 \label{eq:Xhat}
 \end{multline}
\textit{ где}
 \begin{multline}
 \mathbf{I}_{[\tau_{n-1},\tau_n)}(t)\me{}{X(t)|\G(n-1) \vee
 \{\tau_n > t\}} ={}\\
 {}=\mathbf{I}_{[\tau_{n-1},\tau_n)}(t) \overline{X}_n(t) \quad
 \pc-{\mbox{п.~н.,}}
  \label{eq:Xhat_2}
 \end{multline}
 \textit{а $\overline{X}_n(t)$, в свою очередь, определяется
 формулами}~(\ref{eq:Xbar}) \textit{и}~(\ref{eq:Z_sys}).
 \item
\textit{ Компенсатор $\nu$ меры~$\mu$ равен}
 \begin{multline}
 \nu(dt,dy) = {}\\
{} =\sum\limits_{n \in \mathbb{N}} \sum\limits_{j=1}^M \mathbf{I}_{(\tau_{n-1},\tau_n]}(t) y^{\top}
\left(\phi^0_{n}(t)\widehat{X}(\tau_{n-1},t-)+{}\right.\\
\left.{}+\phi^1_{n}(t)\widehat{X}(t-)
 \right)\delta_{\{f_j\}}(dy)\Phi_n(dt)\,.
 \label{eq:nu_fin}
 \end{multline}
 \end{enumerate}


\noindent
 Д\,о\,к\,а\,з\,а\,т\,е\,л\,ь\,с\,т\,в\,о\ \ теоремы~1 приведено в приложении.

\smallskip

 Утверждение~1 теоремы дает весьма важное пред\-ставление о разнице между
 <<дискретными>> \mbox{$\sigma$-ал}\-геб\-ра\-ми $\G(n)$ и их <<непрерывными>> аналогами
 $\G_t$, о которой упоминалось в~\cite{CLR_06}. Если до момента времени~$t$
 произошло $n-1$ скачков процесса $Y(t)$, то $\G_t$~---
 минимальная $\sigma$-ал\-геб\-ра, содержащая $\G(n-1)$, а также событие
 $\{\omega:\; \tau_n(\omega)\hm>t\}$.

 Из определений (\ref{eq:X_0_bar_0}) и~(\ref{eq:Xbar_0}),
 уравнения~(\ref{eq:Z_sys}) для вычисления $\overline{Z}_n(t)$ и
 вида~(\ref{eq:Xhat}) для УМО $\widehat{X}(t)$ следует, что
 компенсатор~(\ref{eq:nu_fin}) может быть записан более компактно. Для этого
на промежутках $[\tau_{n-1},\tau_n)$ введем в рассмотрение вспомогательные
матричнозначные процессы с траекториями, $\pc-$п.~н. непрерывными слева:
 \begin{multline}
 \hspace*{-9pt}\beta_n(t) \ebd \phi^0_n(t)\mathbf{E}\{X(\tau_{n-1})
 \widetilde{I}_{n}(t-)X^{\top}(t-)|\G(n-1)\}+{}\\
{} +\phi^1_n(t)\me{}{X(t-)\widetilde{I}_{n}(t-)X^{\top}(t-)|\G(n-1)}={} \\
{}=
 \phi^0_n(t)\overline{Z}_n(t-)+\phi^1_n(t)\diag(\mathbf{1}_N^{\top}\overline{Z}_n(t-))\,;
 \label{eq:beta}
 \end{multline}

\vspace*{-12pt}

\noindent
 \begin{multline}
 \widehat{\beta}_n(t) \ebd \phi^0_n(t)\me{}{X(\tau_{n-1})X^{\top}(t-)|\G_{t-}}+{}\\
 {}+
 \phi^1_n(t)\me{}{X(t-)X^{\top}(t-)|\G_{t-}} ={}\\
 {}=\fr{1}{\mathbf{1}_N^{\top} \overline{Z}_n(t-)\mathbf{1}_N} \beta_n(t)\,.
 \label{eq:hatbeta}
 \end{multline}
  При этом выполняются равенства:
\begin{multline*}
\phi^0_n(t)\me{}{X(\tau_{n-1})\widetilde{I}_{n}(t-)|\G(n-1)}+{}\\
{}+
\phi^1_n(t)\me{}{X(t-)\widetilde{I}_{n}(t-)|\G(n-1)}=
\beta_n(t)\mathbf{1}_N\,;
\end{multline*}
  $$
\phi^0_n(t)\widehat{X}(\tau_{n-1},t-)+\phi^1_n(t)\widehat{X}(t-)=
\widehat{\beta}_n(t)\mathbf{1}_N\,.
$$
 Таким образом, компенсатор~$\nu$ также может быть записан в виде:
 \begin{multline*}
 \nu(dt,dy) = {}\\
 {}=\sum\limits_{n \in \mathbb{N}}
 \sum\limits_{j=1}^M \mathbf{I}_{(\tau_{n-1},\tau_n]}(t)
 \mathbf{1}_N^{\top}\widehat{\beta}^{\top}_n(t)y \delta_{\{f_j\}}(dy)\Phi_n(dt)\,.
% \label{eq:nu_finfin}
 \end{multline*}

 \section{Решение задачи фильтрации}

 Следующее утверждение представляет основной результат~---
 формулы для вычисления искомой оценки оптимальной фильтрации.

 \smallskip

 \noindent
 \textbf{Теорема 2.}
\textit{Пусть для системы наблюдений}~(\ref{eq:state}), (\ref{eq:obs_2})
\textit{выполнены условия} (а)--(е).
\textit{Тогда регулярная версия условного распределения $\widehat{X}(t)$
вычисляется с помощью решения следующих рекуррентно связанных дифференциальных и
алгебраических уравнений}:
  \begin{enumerate}[(1)]
  \item \textit{при} $t=0$
  \begin{equation*}
  \widehat{X}(0)=p_0\,;
%  \label{eq:Xbar_init}
  \end{equation*}
  \item
\textit{при} $t \hm\in (\tau_{n-1},\tau_n)$
  \begin{equation*}
  \widehat{X}(t)=\fr{1}{\mathbf{1}^{\top}_N\overline{Z}_n(t)\mathbf{1}_N}\,
  \overline{Z}^{\top}_n(t)\mathbf{1}_N\,,
%  \label{eq:Xbar_t}
  \end{equation*}
\textit{где $\overline{Z}_n(t)$ является решением уравнения}~(\ref{eq:Z_sys});
  \item
\textit{при} $t =\tau_n $
\begin{multline}
\widehat{X}(\tau_n) ={}\\
\hspace*{-5mm}{}=
\left\{
\begin{array}{l}
\widehat{X}(\tau_n-), \quad \mbox{если}\quad \mathbf{1}_N^{\top}\widehat{\beta}^{\top}_n(\tau_n)Y_n = 0;\\[9pt]
\displaystyle
\fr{1}{\mathbf{1}_N^{\top}\widehat{\beta}^{\top}_n(\tau_n)Y_n}
\widehat{\beta}^{\top}_n(\tau_n)Y_n,   \\[9pt]
\hspace*{20mm}\mbox{если } \mathbf{1}_N^{\top}\widehat{\beta}^{\top}_n(\tau_n)Y_n >0\\[9pt]
\hspace*{10mm}\mbox{и }
\displaystyle\mathbf{1}_N^{\top}\widehat{\beta}^{\top}_n(\tau_n)
\mathbf{1}_M\Phi_n^d(\{\tau_n\})<1;\\[12pt]
\displaystyle
\fr{1}{\mathbf{1}_N^{\top}\widehat{\beta}^{\top}_n(\tau_n)Y_n}
\widehat{\beta}^{\top}_n(\tau_n)Y_n -{}\\[9pt]
{}-
\left[
\widehat{\beta}^{\top}_n(\tau_n)\mathbf{1}_M \Phi_n^d(\{\tau_n\}) -
\widehat{X}(\tau_n-) \right], \\[9pt]
\hspace*{20mm}\mbox{если } \mathbf{1}_N^{\top}\widehat{\beta}^{\top}_n(\tau_n)Y_n >0 \\[9pt]
\hspace*{10mm}\mbox{и }
\displaystyle\mathbf{1}_N^{\top}\widehat{\beta}^{\top}_n(\tau_n)\mathbf{1}_M\Phi_n^d(\{\tau_n\})=
1\,,
\end{array}
\right.
\label{eq:Xhat_4}
\end{multline}
\textit{где $\widehat{\beta}_n(t)$ определяется формулами}~(\ref{eq:beta}) \textit{и}~(\ref{eq:hatbeta}).
\end{enumerate}

\smallskip

\noindent
 Д\,о\,к\,а\,з\,а\,т\,е\,л\,ь\,с\,т\,в\,о\ \ теоремы~2 приведено в приложении.

 \section{Численный пример}

 В качестве иллюстративного примера рас\-смат\-ри\-ва\-ется задача оперативного
 оценивания со\-сто\-яния удаленного сервера по результатам одного запроса к нему.
 Предполагается, что состояние\linebreak \mbox{пары} <<телекоммуникационный
 ка\-нал\,--\,сер\-вер>> $X(t)$ недоступно прямому наблюдению и
 описывается однородным МСП с тремя возможными состоя\-ниями:
 \begin{enumerate}[(1)]
 \item
 $X(t)=e_1$~--- с сервером отсутствует связь;
 \item
 $X(t)=e_2$~--- с сервером есть связь, сервер не загружен;
 \item
 $X(t)=e_3$~--- с сервером есть связь, сервер загружен.
 \end{enumerate}
 Матрица интенсивности~$\Lambda$ переходов МСП $X(t)$ известна, в начальный момент
 времени $t\hm=0$ распределение $X(0)$ совпадает со стационарным.

 О состоянии удаленного сервера имеется косвенная информация~---
 результат отправленного в начальный момент времени запроса к данному
 серверу. Возможными исходами запроса являются:
  \begin{itemize}
 \item
 $Y_1=f_1$~--- <<предметный>> ответ сервера;
 \item
 $Y_1=f_2$~--- общее сообщение об ошибке.
 \end{itemize}
 Результат запроса к серверу появляется в случайный момент времени~$\tau_1$,
 распределение которого зависит от состояния~$X$. О~модели результата запроса
 к серверу доступна следующая априорная информация. Сообщение об ошибке может
 появиться только в один из дискретных моментов истечения времен ожидания
 (тайм-аутов):
 \begin{itemize}
 \item
 $t_1$ --- тайм-аут, связанный с временем ожидания установления сетевого
 соединения;
 \item
 $t_2$~--- предельное время выполнения запроса сис\-те\-мой
 управления базой данных на удаленном сервере;
 \item
 $t_3$~--- предельное время формирования результатов запроса на ин\-тер\-нет-сер\-ве\-ре;
 \item
 $t_4$~--- тайм-аут, связанный с временем передачи по сети результатов
 выполнения запроса.
 \end{itemize}
 Если в начальный момент времени связь с сервером отсутствует, то сервер выдает
 сообщение об ошибке. Первое слагаемое интенсивности наблюдений имеет вид:
 \begin{multline*}
 \int\limits_0^t \phi^0(s)X(0)\Phi(ds)={}\\
 {}=
  \displaystyle  \left[
  \begin{array}{ccc}
  \displaystyle \int\limits_0^t\fr{\displaystyle \sum\limits_{j=1}^{4}p_{1j}\delta_{t_j}(ds)}
  {\displaystyle \int\limits_0^s \sum\limits_{k=1}^{4}p_{1k}\delta_{t_k}(du)} &
  0 & 0\\[9pt]
  0 & 0 & 0
  \end{array}
  \right]
  X(0)\,.
 \end{multline*}
 Если удаленный сервер пребывает в состоянии~$e_i$ ($i\hm=2,3$),
 то <<предметный>> ответ поступает через случайное время, имеющее
 плот\-ность распределения $\psi_i(t)$. В~этих же состояниях сервер
 также может выдавать общее сообщение об ошибках: это связано с превышением
 времени выполнения запросов, подготовки их отправки и~пр.
 Второе слагаемое интенсивности имеет вид:
\begin{multline*}
 \int\limits_0^t \phi^1(s)X(0)\Phi(ds)={}\\
 \hspace*{-3pt}{}=
  \displaystyle \!\int\limits_0^t \!\left[\!
  \begin{array}{ccc}
  0 & \displaystyle\fr{\displaystyle \pi_{21}\psi_2(s)\,ds}
  {\displaystyle \Psi_2(s)} &
  \displaystyle\fr{\displaystyle \pi_{31}\psi_3(s)\,ds}{\displaystyle \Psi_3(s)}
  \\[9pt]
  0 & \displaystyle\fr{\displaystyle \pi_{22}\sum\limits_{j=1}^{4}p_{2j}
  \delta_{t_j}(ds)}{\displaystyle \Psi_2(s)} &
  \displaystyle\fr{\displaystyle \pi_{32}\sum\limits_{j=1}^{4}p_{3j}
  \delta_{t_j}(ds)}{\displaystyle \Psi_3(s)}
  \end{array}\!
  \right]\!
  X(s-),\hspace*{-7.48343pt}
  \end{multline*}
  где
  \begin{multline*}
 \Psi_i(s) \ebd \int\limits_{[s,+\infty)} \left(
 \pi_{i1}\psi_2(u)du+
 \pi_{i2}\sum\limits_{j=1}^{4}p_{2j}\delta_{t_j}(du)
 \right), \\ i=2,3,
  \end{multline*}
  есть условные <<функции дожития>>, соответст\-ву\-ющие распределению момента~$\tau_1$
  при условии $X(t)\hm\equiv e_i$.

  Численный эксперимент проводился для следующих вероятностных параметров и
  плотностей распределения:
  \begin{equation*}
  \Lambda = \left[
  \begin{array}{ccc}
-0{,}01 & 0{,}01 & 0 \\
0{,}00002 & -0{,}00022 & 0{,}0002 \\
0{,}00003 & 0{,}0008 & -0{,}00083
  \end{array}
  \right]\,;
  \end{equation*}
\begin{equation*}
  \pi_{21} = 0{,}99; \quad \pi_{22} = 0{,}01;
  \quad \pi_{31} = 0{,}85; \quad \pi_{32} = 0{,}15;
\end{equation*}
$\psi_i(t),\; i=2,3,$~--- усеченные плотности распределения Парето:
\begin{multline*}
\psi_i(t) \ebd{}\\
{}\ebd C(k_i,\sigma_i,\theta_i,T_i)\mathbf{I}_{[\sigma_i,T_i]}(t)
\left(1+k_i\fr{t-\theta_i}{\sigma_i}
\right)^{-1-{1}/{k_i}},
\end{multline*}
где $C(\cdot)$~--- нормировочные константы:
\begin{gather*}
k_2= 0{,}5; \quad \sigma_2= 50; \quad \theta_2= 1; \quad T_2= 1000; \\
k_3= 5; \quad \sigma_3= 300; \quad \theta_3= 100; \quad T_3= 1000.
\end{gather*}
Временн$\acute{\mbox{ы}}$е и вероятностные характеристики дискретной части
распределений приведены в таб-\linebreak лице.

\vspace*{6pt}

\noindent
\begin{center}
{\small{Характеристики дискретной части распределений}}
\end{center}

%\vspace*{2pt}

\noindent
{\small\begin{center}
\tabcolsep=15.5pt
\begin{tabular}{|c|c|c|c|c|}
%\multicolumn{5}{c}{Вероятностные характеристики дискретной части распределений}\\[6pt]
\hline
$i$ & $t_i$ & $p_{1i}$& $p_{2i}$ & $p_{3i}$ \\
\hline
1 &  200 & 0,6& 0,01& 0,01\\
2 &  600 & 0,4 & 0,79& 0,09\\
3 &  900 & 0\hphantom{,9} & 0,1\hphantom{9}& 0,7\hphantom{9}\\
4  & 1000\hphantom{9} & 0\hphantom{,9} & 0,1\hphantom{9}& 0,2\hphantom{9}\\
\hline
\end{tabular}
\vspace*{3pt}
\end{center}}


На рис.~1,\,\textit{а} для сравнения пред\-став\-лены значения априорных вероятностей
$\pp{X(t)\hm=e_1}\hm=e_1^{\top}p(t)$ и условных вероятностей
$\pp{X(t)\hm=e_1|\G(t)}\hm=e_1^{\top}\widehat{X}(t)$. Для этих графиков
предполагается, что в качестве результата обращения к серверу выступает
<<предметный>> ответ $Y_1\hm=f_1$. Каждый график представляет собой эволюцию
во времени априорной или условной вероятности, а разные графики соответствуют
разным моментам~$\tau_1$.

\addtocounter{table}{1}


Рисунки~1,\,\textit{б}  и~1,\,\textit{в}
содержат соответствующие графики для со\-сто\-яний~$e_2$ и~$e_3$.


Рисунки~2\textit{а}\,--\,2\textit{в}
аналогичны рисункам~1,\,\textit{а}\,--\,1,\,\textit{в},
но соответствуют случаю, когда
в качестве результата запроса к серверу выступает сообщение об ошибке, т.\,е.\
$Y_1\hm=f_2$. Каждый график представляет собой эволюцию во времени априорной
или условной вероятности, а разные графики соответствуют разным моментам~$\tau_1$.

\begin{figure*} %fig1
\vspace*{1pt}
\begin{center}
\mbox{%
\epsfxsize=105mm
\epsfbox{bor-1.eps}
}
\end{center}
\vspace*{-9pt}
\Caption{Априорные (\textit{0}) и условные вероятности событий $X(t)\hm=e_1$~(\textit{а}),
$e_2$~(\textit{б}) и $e_3$~(\textit{в}) при $Y_1\hm=f_1$
и различных~$\tau_1$: \textit{1}~--- 50; \textit{2}~--- 100;
\textit{3}~--- 150; \textit{4}~--- 250; \textit{5}~--- 300; \textit{6}~--- 350;
\textit{7}~--- 400; \textit{8}~--- 450; \textit{9}~--- 500;
\textit{10}~--- 550; \textit{11}~--- 650; \textit{12}~--- 700;
\textit{13}~--- 750; \textit{14}~--- 800; \textit{15}~--- 850; \textit{16}~--- 900}
\label{pic1}
\end{figure*}
\begin{figure*} %fig2
\vspace*{1pt}
\begin{center}
\mbox{%
\epsfxsize=104mm
\epsfbox{bor-4.eps}
}
\end{center}
\vspace*{-9pt}
\Caption{Априорные (\textit{0}) и условные вероятности событий $X(t)\hm=e_1$~(\textit{а}),
$e_2$~(\textit{б}) и $e_3$~(\textit{в})
при $Y_1\hm=f_2$ и различных~$\tau_1$: \textit{1}~--- 200; \textit{2}~--- 600;
\textit{3}~--- 900; \textit{4}~--- 1000}
 \label{pic4}
\end{figure*}


Полученные результаты демонстрируют вариативность условного распределения
по отношению к соответствующему априорному в зависимости от момента появления
наблюдения~$\tau_1$ и его исхода~$Y_1$. Например, априорная вероятность того,
что сервер свободен, равна $\pp{X(t)\hm=e_1} \hm\equiv 0{,}801$. В~то же время
соответствующая условная вероятность колеблется в пределах от 0,5 до~1
в случае $Y_1\hm=f_1$ и от 0,17 до~0,82 в случае $Y_1\hm=f_2$. Этот факт
означает, что имеющиеся косвенные неполные измерения, тем не менее, несут
серьезную информацию о состоянии удаленного сервера. Их учет в алгоритмах
управления распределением нагрузки между удаленными серверами может существенно
увеличить эффективность их использования.

  \section{Заключение}

  Предложенная в работе концепция иссле\-до\-вания системы наблюдений с марковским
  со\-сто\-янием и наблюдением в форме МТП с компенсатором~--- линейной функцией
  со\-сто\-яния, представляет\-ся достаточно перспективной с точки зрения дальнейших
  теоретических и практических исследований. Прежде всего необходимо рас\-смот\-реть
  управ\-ля\-емый вариант предложенной в работе сис\-те\-мы наблюдений и исследовать для
  нее мартингальную проб\-ле\-му. Во-вто\-рых, важным ре\-зультатом являлось бы построение
  некоторого <<универсального>> вероятностного пространства, канонического для
  всего рас\-смат\-ри\-ва\-емо\-го семейства управ\-ля\-емых марковских процессов. Интересным
  \mbox{также} представляется определение в этом пространстве условий, обеспечивающих
  непрерывную зависимость траекторий управ\-ля\-емых процессов от применяемого
  управления. Эти результаты могли бы позволить корректно сформулировать и
  решить задачу оптимального управ\-ле\-ния марковским процессом по наблюдениям~МТП.

  Одновременно с этим для практики важной является разработка на основании
  полученных тео\-ре\-тических результатов эффективных численных субоптимальных
  и робастных алгоритмов балансировки нагрузки в распределенных
  инфотелекоммуникационных сис\-те\-мах, а также решение для этих управляемых
  сис\-тем наблюдения задачи комплексной идентификации параметров.

Все перечисленные выше вопросы являются предметом дальнейших исследований.

%\setcounter{equation}{0}

 \section*{\raggedleft Приложение}

{\small

\noindent
 Д\,о\,к\,а\,з\,а\,т\,е\,л\,ь\,с\,т\,в\,о\ \ леммы~1.
 Определим на $[\tau_{n-1},+\infty)$ процесс
 с одним скачком, описывающий появление $n$-го наблюдения:

 \vspace*{2pt}

 \noindent
 \begin{equation*}
 Y_n(t) = \begin{cases}
 0\,, &\ t \in [\tau_{n-1},\tau_{n})\,;\\
 Y_n\,, &\ t \geqslant \tau_{n}\,.
\end{cases}
 \end{equation*}
Согласно~(\ref{eq:obs_2}) данный процесс является специальным
семимартингалом и допускает  мартингальное представ\-ление

 \vspace*{-2pt}

\noindent
 \begin{multline}
 Y_n(t) = \int\limits_{\tau_{n-1}\wedge t}^{\tau_{n} \wedge t} \left(
 \phi^0_{n}(s)X(\tau_{n-1})+{}\right.\\
\left. {}+\phi^1_{n}(s)X(s-)
 \right)\Phi_{n}(ds) + M^Y_n(t)\,.
 \label{eq:Yn}
 \end{multline}
 Сначала определим условное распределение~$\tau_{n}$ относительно
 $\ff_t^X \bigvee \G(n-1)$, для чего на $[\tau_{n-1},+\infty)$
 рассмотрим случайный процесс $I_n(t) \ebd \mathbf{I}_{\{0\}}
 (\mathbf{1}^{\top}Y_n(t))\hm=\mathbf{I}_{[\tau_{n-1},\tau_{n})}(t)$.
 В~соответствии с этим определением, формулой~(\ref{eq:Yn}) и свойствами
 меры $\Phi_n(\cdot)$ процесс $I_n(t)$ представим в виде:

 \noindent
 \begin{multline}
 I_n(t) = 1 - \int\limits_{\tau_{n-1}}^{t}I_n(s-)\mathbf{1}_M^{\top}\,dY_n(s) = {}\\
{} = 1 - \int\limits_{\tau_{n-1}}^{t}I_n(s-)\mathbf{1}_M^{\top}\left(
\phi^0_{n}(s)X(\tau_{n-1})+{}\right.\\
\!\!\!\left.{}+\phi^1_{n}(s)X(s-)
 \right) \Phi_{n}(ds) -
 \int\limits_{\tau_{n-1}}^{t}I_n(s-)\mathbf{1}_M^{\top} \,dM^Y_n(s).
\!\! \label{eq:In}
 \end{multline}
 Тогда $\widetilde{I}_n(t) \ebd \me{}{I_n(t)|\ff_t^X \bigvee \G(n-1)} \hm=
 \mathbf{I}_{[\tau_{n-1},+\infty)}(t)\pp{\tau_{n}>t|\ff_t^X \bigvee \G(n-1)}$
 представляет собой так называемую <<функцию дожития>>, зная которую, легко получить
 условную функцию распределения: $\widetilde{F}_n(t) \ebd \pp{\tau_{n}\hm\leqslant t|\ff_t^X \bigvee \G(n\hm-1)} \hm=
 1\hm- \widetilde{I}_n(t)$.

 Для нахождения $\widetilde{I}_n(t)$ воспользуемся методом вывода уравнений
 оптимальной нелинейной фильтрации, предложенным в~\cite{WongH_85}. Беря УМО
 от обеих частей~(\ref{eq:In}) относительно $\ff_t^X \bigvee \G(n-1)$, легко
 проверить, что

 \columnbreak

 \noindent
 \begin{multline}
 \widetilde{I}_n(t) = 1 - \int\limits_{\tau_{n-1}}^{t}\widetilde{I}_n(s-)
 \mathbf{1}_M^{\top}\left(\phi^0_{n}(s)X(\tau_{n-1})+{}\right.\\
\left. {}+\phi^1_{n}(s)X(s-)
 \right) \Phi_{n}(ds) + \widetilde{M}^I_n(t)\,,
 \label{eq:wtIn}
 \end{multline}
 где $\widetilde{M}^I_n(t)$~--- некоторый $\ff_t^X \bigvee \G(n-1)$-со\-гла\-со\-ван\-ный
 мартингал. Он представим в виде:
 $$
 \widetilde{M}^I_n(t)= \int\limits_{\tau_{n-1}}^{t}\gamma(s)\,dM^X(s)\,,
 $$
 где $\gamma(s) \ebd \gamma(\omega,s): \ \Omega \times \mathbb{R}_+
 \to \mathbb{R}^{1 \times N}\ \mbox{--}\ \ff_s^X \bigvee \G(n-1)$-пред\-ска\-зу\-емый процесс, подлежащий определению:



По обобщенному правилу Ито

\noindent
\begin{multline}
 X(t)\widetilde{I}_n(t)=X(\tau_{n-1})+ \int\limits_{\tau_{n-1}}^{t}X(s-)\,
 d\widetilde{I}_n(s)+{} \\
 {}+
 \int\limits_{\tau_{n-1}}^{t}dX(s)\widetilde{I}_n(s-) +
 \sum\limits_{\tau_{n-1} < s \leqslant t} \Delta X(s)\Delta \widetilde{I}_n(s) ={}\\
{} =  X(\tau_{n-1}) - \int\limits_{\tau_{n-1}}^{t}
   X(s-)\widetilde{I}_n(s-)\mathbf{1}^{\top}\left(\phi^0_{n}(s)X(\tau_{n-1})+{}\right.\\
\left.   {}+
   \phi^1_{n}(s)X(s-)  \right) \Phi_{n}(ds)+ {}\\
   {}+ \int\limits_{\tau_{n-1}}^{t}\left( \Lambda^{\top}(s)X(s-)
   \widetilde{I}_n(s-) +{}\right.\\
\left.   {}+
 \diag(\gamma(s))\Lambda^{\top}(s)X(s-)
 \right)\,ds+M_1(t)\,,
 \label{eq:XwtIn_1}
 \end{multline}
 где $M_1(t)$~--- $\ff_t^X \bigvee \G(n-1)$-со\-гла\-со\-ван\-ный мартингал.

 Из условия~(д) следует, что $[X,Y]_t \hm\equiv 0$ $\pc$--п.~н., а
 значит, и $[X,I_n]_t \hm\equiv 0$, поэтому по правилу Ито
 \begin{multline}
 X(t)I_n(t)=X(\tau_{n-1})+{}\\
 {}+ \int\limits_{\tau_{n-1}}^{t}X(s-)\,dI_n(s)+
 \int\limits_{\tau_{n-1}}^{t}dX(s)I_n(s-)  ={}\\
{} =  X(\tau_{n-1}) - \int\limits_{\tau_{n-1}}^{t}
   X(s-)I_n(s-)\mathbf{1}_M^{\top}\left(\phi^0_{n}(s)X(\tau_{n-1})+{}\right.\\
\left.   {}+
   \phi^1_{n}(s)X(s-)  \right) \Phi_{n}(ds)+ {}\\
   {} + \int\limits_{\tau_{n-1}}^{t} \Lambda^{\top}(s)X(s-)I_n(s-)\, ds+M_2(t)\,,
 \label{eq:XIn}
 \end{multline}
 где $M_2(t)$~--- $\ff_t$-со\-гла\-со\-ван\-ный мартингал. Вычисляя УМО
 от обеих частей~(\ref{eq:XIn}) относительно $\ff_t^X \bigvee \G(n-1)$,
 получаем выражение

 \noindent
 \begin{multline*}
 X(t)\widetilde{I}_n(t)=
 X(\tau_{n-1}) -{} \\
 {}-\int\limits_{\tau_{n-1}}^{t}
   X(s-)\widetilde{I}_n(s-)\mathbf{1}_M^{\top}\left(\phi^0_{n}(s)
   X(\tau_{n-1})+{}\right.
   \end{multline*}

   \noindent
\begin{multline}
\left.   {}+\phi^1_{n}(s)X(s-)
 \right) \Phi_{n}(ds)+{} \\
 {}+ \int\limits_{\tau_{n-1}}^{t} \Lambda^{\top}(s)X(s-)\widetilde{I}_n(s-)\, ds+M_3(t)\,,
 \label{eq:XwtIn_2}
 \end{multline}
 где $M_3(t)$~--- $\ff_t^X \bigvee \G(n-1)$-со\-гла\-со\-ван\-ный
 мартингал. Формулы~(\ref{eq:XwtIn_1}) и~(\ref{eq:XwtIn_2}) представляют
 собой две записи мартингального разложения одного и того же специального
 семимартингала $X(t)\widetilde{I}_n(t)$, и в силу единственности этого
 разложения верно равенство
 $$
 \diag \gamma(s) \Lambda^{\top}(s)X(s-) \equiv 0
 $$
 $\pc$--п.~н. почти для всех $s \hm\in [\tau_{n-1},+\infty)$.
 Решение последнего уравнения не единственно, но версии полученного
 процесса $\widetilde{I}_n(t)$~(\ref{eq:wtIn}) при выборе разных~$\gamma$
 будут неотличимыми, поэтому можно выбрать частный случай решения~---
 тождественный ноль, т.\,е.\ положить $\gamma(s)\hm \equiv 0$, $s \hm
 \in [\tau_{n-1},+\infty)$. Таким образом, функция дожития
 $\widetilde{I}_n(t)$ удовлетворяет обыкновенному линейному
 дифференциальному уравнению с разрывной случайной правой частью

 \noindent
 \begin{multline}
 \widetilde{I}_n(t) = 1 - \int\limits_{\tau_{n-1}}^{t}\mathbf{1}_M^{\top}
 \left(\phi^0_{n}(s)X(\tau_{n-1})+{}\right.\\
\left. {}+\phi^1_{n}(s)X(s-)
 \right) \widetilde{I}_n(s-) \Phi_{n}(ds)\,.
 \label{eq:wtIn_2}
 \end{multline}
 Решение этого уравнения известно:
 \begin{multline}
 \widetilde{I}_n(t) = \exp \left\{
 -\int\limits_{\tau_{n-1}}^t \mathbf{1}_M^{\top}\left(
 \phi^0_n(s)X(\tau_{n-1})+{}\right.\right.\\
\left.\left. {}+\phi^1_n(s)X(s-)
 \right)\Phi^c_n(ds)
 \vphantom{\int\limits_{\tau_{n-1}}^t}
 \right\} \times{} \\
{} \times \prod\limits_{\tau_{n-1} < s \leqslant t}\left[
 1-
 \mathbf{1}_M^{\top}\left(
 \phi^0_n(s)X(\tau_{n-1})+{}\right.\right.\\
\left.\left. {}+\phi^1_n(s)X(s-)
 \right)\Phi^d_n(\{s\})
 \right].
 \label{eq:wtIn_3}
 \end{multline}

 Рассмотрим теперь процесс $\widetilde{I}_n(t,u) \ebd$\linebreak $\ebd\;\me{}{I_n(t)|
 \ff_u^X \bigvee \G (n-1)}$ как функцию времени~$u$ ($t \hm\leqslant u$).
 Можно показать, что
 $\widetilde{I}_n(t,u)$ представим в виде:
 \begin{equation}
 \widetilde{I}_n(t,u) = \widetilde{I}_n(t) +\int\limits_{t}^{u}\epsilon(s)\,dM^X(s)\,,
 \label{eq:wtIn_4}
 \end{equation}
 где $\epsilon(s) \ebd \epsilon(\omega,s): \
 \Omega \times \mathbb{R}_+ \hm\to \mathbb{R}^{1 \times N}$~---
 $\ff_s^X \bigvee \G(n-1)$-пред\-ска\-зу\-емый процесс, вновь подлежащий
 определению. Действуя аналогичным образом как при выводе коэффициента
 $\gamma(s)$, можно показать, что $\epsilon(s) \hm\equiv 0$, $s \hm\in
 (t,+\infty)$, т.\,е.\ $\widetilde{I}_n(t,u) \hm\equiv \widetilde{I}_n(t)$ при
 $u \hm> t$. Этот факт совместно с формулой~(\ref{eq:wtIn_4}) доказывает
 истинность утверждения~1 леммы~1. При этом выполнение условий~(\ref{eq:cond_1})
 гарантирует, что функция, определяемая формулой~(\ref{eq:wtIn_3}),
 действительно $\pc$--п.~н.\ обладает свойствами функции распределения.

 Далее, процесс $Y_n(t)$ может быть представлен следующим образом:

 \noindent
 \begin{multline*}
 Y_n(t) = \int\limits_{\tau_{n-1}}^{t}dY_n(s)I_n(s-)={} \\
 {}=
 \int\limits_{\tau_{n-1}}^{t}\left(\phi^0_{n}(s)X(\tau_{n-1})+\phi^1_{n}(s)X(s-)
 \right) I_n(s-)\Phi_{n}(ds)+{}\\
 {}+\int\limits_{\tau_{n-1}}^{t}dM^Y_n(s)I_n(s-)\,.
% \label{eq:Yn_2}
 \end{multline*}
 Обозначим $\widetilde{Y}_n(t) \ebd \me{}{Y_n(t)|\ff_t^X \bigvee \G(n-1)}$.
 Действуя аналогично доказательству утверждения~1 леммы~1
 и используя~(\ref{eq:wtIn_3}) для $\widetilde{I}_n(t)$, легко показать, что

 \noindent
 \begin{multline}
 \widetilde{Y}_n(t) =\me{}{Y_n(t)\left|\ff^X \vee \G(n-1)\right.}
 ={}\\
 {}= \!\int\limits_{\tau_{n-1}}^{t}\!\left(\phi^0_{n}(s)X(\tau_{n-1})+\phi^1_{n}(s)X(s-)
 \right) \widetilde{I}_n(s-)\Phi_{n}(ds) = {}\\
 {}=
 \int\limits_{\tau_{n-1}}^{t}\left(\phi^0_{n}(s)X(\tau_{n-1})+\phi^1_{n}(s)X(s-)
 \right) \times{} \\
 {}\times
 \exp \left\{
 -\int\limits_{\tau_{n-1}}^s \mathbf{1}_M^{\top}\left(
 \phi^0_n(u)X(\tau_{n-1})+{}\right.\right.\\
\left.\left. {}+\phi^1_n(u)X(u-)
 \right)\Phi^c_n(du)
 \vphantom{\int\limits_{\tau_{n-1}}^s}
 \right\} \times{} \\
{} \times \prod\limits_{\tau_{n-1} < u < s}\left[
 1-  \mathbf{1}_M^{\top}\left(
 \phi^0_n(u)X(\tau_{n-1})+{}\right.\right.\\
\left.\left. {}+\phi^1_n(u)X(u-)
 \right)\Phi^d_n(\{u\})
 \right]
 \Phi_{n}(ds)\,.
 \label{eq:Yn_3}
 \end{multline}
 Формула (\ref{eq:tau_dist})~--- утверждение~2 леммы~1~---
 непосредственно следует из~(\ref{eq:Yn_3}) благодаря тому факту,
 что $\pp{\tau_n \leqslant t, \; Y_n \hm= f_i | \ff_{t}^X  \bigvee \G (n-1)}
 \hm= f_i^{\top}\widetilde{Y}_n(t)$.

 Рассмотрим на промежутке $[\tau_{n-1},\tau_n)$ процесс
 $Z_n(t) \ebd X(\tau_{n-1}) \widetilde{I}_{n}(t)X^{\top}(t)$.
 Используя правило Ито, формулы~(\ref{eq:obs_2}), (\ref{eq:wtIn_2}) и
 свойства матриц, можно получить стохастическое дифференциальное уравнение
 для $Z_n(t)$:
 \begin{multline}
 Z_n(t)=\diag(X(\tau_{n-1})) +
 \int\limits_{\tau_{n-1}}^t Z_n(s-)\Lambda(s)\,ds -{}\\
 {}-\int\limits_{\tau_{n-1}}^t
 \left[\diag(\phi^{0\top}_n(s)\mathbf{1}_M)
 Z_n(s-)+{}\right.
\\
\left. {}+Z_n(s-)\diag(\phi^{1\top}_n(s)\mathbf{1}_M)
 \right]\Phi_n(ds)+ {}\\
 {}+ \int\limits_{\tau_{n-1}}^t X(\tau_{n-1})\widetilde{I}_n(s-)\,d\left(M^{X}
 \right)^{\top}(s)\,,
 \label{eq:Z_2}
 \end{multline}
 где $M^{X}(t)$~--- мартингал из разложения~(\ref{eq:state}).

 Беря УМО от левой и правой частей~(\ref{eq:Z_2}) относительно $\G(n-1)$,
 получаем искомую систему~(\ref{eq:Z_sys}) для $\overline{Z}_n(t)$, т.\,е.\
 доказываем истинность утверждения~3 леммы~1.

 Согласно~\cite{JSh_94} компенсатор~$\nu$ определяется формулой:
 \begin{equation}
 \nu(dt,dy) = \sum\limits_{n \in \mathbb{N}} \mathbf{I}_{(\tau_{n-1},\tau_n]}(t)
 \fr{\pc_n(dt,dy)}{\pc_n([t,+\infty)\times \mathbb{R}^M)}\,,
 \label{eq:nu_2}
 \end{equation}
 где $\pc_n(dt,dy)$~--- условное распределение пары $(\tau_n,Y_n)$
 относительно $\G(n-1)$.
 Формула~(\ref{eq:nu_2}) определена корректно, так как
 на множестве $\{\omega: \tau_{n-1}(\omega) \hm< t \hm\leqslant \tau_{n}(\omega) \}$
 случайная величина
 $\pc_n([t,+\infty)\times \mathbb{R}^M)\hm > 0$ $\pc$--п.~н.
 Формула~(\ref{eq:Yn_3}) задает распределение относительно более широкой
 $\sigma$-ал\-геб\-ры $\ff^X \vee \G(n-1)$. Используя свойства УМО,
 можно получить выражение для знаменателя в~(\ref{eq:nu_2}):
 \begin{multline}
 \pc_n([t,+\infty)\times \mathbb{R}^M) =
 \pp{\tau_n \geqslant t|\G(n-1)}={}\\
 {}= \me{}{\pc \{\tau_n \geqslant t|\ff^X \vee \G(n-1)\}|\G(n-1)}={}\\
 {}=
 \me{}{\widetilde{I}_n(t-)|\G(n-1)}
 \label{eq:Pn}
 \end{multline}
  и для числителя
  \begin{multline}
\pc_n(dt,dy) ={}\\
{}= \pp{\tau_n \in [t,t+dt), Y_n \in [y,y+dy)|\G(n-1)}={} \\
 {}=  \mathbf{E}\left\{\pc\left\{\tau_n \in [t,t+dt), Y_n \in [y,y+dy)|\ff^X \vee\right.\right.{}\\
\left. \left.{}\vee
 \G(n-1)  \vphantom{\ff^X}
 \right\}|\G(n-1)\right\} ={}\\
 {}=  \mathbf{E}\left\{  \sum\limits_{i=1}^M y^{\top} \left(
 \phi_n^0(t)X(\tau_{n-1})+{}\right.\right.\\
\left.\left. {}+\phi_n^1(t)X(t-)
 \right)\widetilde{I}_n(t-) \delta_{\{f_i\}}(dy)\Phi_n(dt)
 |\G(n-1)
 \vphantom{\sum\limits_{i=1}^M}\right\} ={} \\
 {}=  \sum\limits_{i=1}^M y^{\top}
 \left[
 \phi_n^0(t)\mathbf{E}\!\left\{ X(\tau_{n-1})\widetilde{I}_n(t-)|\G(n-1)\right\}
 +{} \right. \\
\hspace*{-2.5mm}\left.{} +
 \phi_n^1(t)\mathbf{E}\!\left\{ X(t-)\widetilde{I}_n(t-)|\G(n-1)\right\}
 \right]\!\delta_{\{f_i\}}(dy)\Phi_n(dt).\!\!\!
 \label{eq:dPn}
 \end{multline}

 Подставляя (\ref{eq:Pn}) и~(\ref{eq:dPn}) в~(\ref{eq:nu_2}), получаем
 требуемую формулу~(\ref{eq:nu}). Все утверждения леммы доказаны.

\smallskip

\noindent
Д\,о\,к\,а\,з\,а\,т\,е\,л\,ь\,с\,т\,в\,о\ \ теоремы~1.
 По определению
$$
\G (n-1) \vee \{\tau_n > t\} \ebd \sigma
\{ \{\omega:\,\tau_n(\omega) > t\} , A: \, A \in \G (n-1) \}
$$
есть минимальная $\sigma$-ал\-геб\-ра,
  содержащая $\G (n-1)$ и $\{\omega \hm\in \Omega:\;\tau_n \hm> t\}$.

 Определим процессы
\begin{align*}
 J_n(t) &\ebd \mathbf{I}_{[\tau_n, + \infty)}(t)\,;
\\
\widetilde{J}_n(t) &\ebd \me{}{J_n(t)|\ff^X\bigvee \G (n-1)}\,;
\\
 \breve{X}_n(t) &\ebd
 \mathbf{I}_{[0,\tau_{n-1})}(t)\me{}{X(t)|\G (n-1)}+{}\\
& {}+\mathbf{I}_{[\tau_{n-1},\tau_n)}
 (t)\overline{X}_n(t)+  \mathbf{I}_{[\tau_n, + \infty)}(t)\times{}\\
& {}\times \me{}{\widetilde{J}_n(t)|\G (n-1)}^+
\me{}{X(t)\widetilde{J}_n(t)|\G (n-1)}\,.
% \label{eq:bbrX}
 \end{align*}
Выберем произвольное $A \hm\in \G (n-1)$ и положим
$B \ebd$\linebreak $\ebd\;A \bigcap \{\tau_n > t\}$, тогда из свойств УМО следует цепочка равенств:
 \begin{multline*}
 \me{}{\mathbf{I}_B(\omega)(X(t)-\breve{X}_n(t))}=
  \mathbf{E}\left\{
  \vphantom{\breve{X}_n(t)}
  \mathbf{I}_A(\omega)\left(\mathbf{I}_{[0,\tau_{n-1})}(t)+{}\right.\right.\\[2pt]
\left.\left.  {}+\mathbf{I}_{[\tau_{n-1},\tau_n)}(t)\right)
 \left(X(t)-\breve{X}_n(t)\right)\right\}={}\\[2pt]
 {}=
 \mathbf{E}\left\{\mathrm{E}\left\{\mathbf{I}_A(\omega)
 \mathbf{I}_{[0,\tau_{n-1})}(t)\left(
 X(t)-{}\right.\right.\right.\\[2pt]
\left.\left.\left. {}-\mathbf{E}\left\{X(t)|\G(n-1)\right\}\right)\left|\G(n-1)\right.\right\}\right\}+{}\\[2pt]
 {}+
 \mathbf{E}\left\{\mathbf{I}_A(\omega)\mathbf{I}_{[\tau_{n-1},\tau_n)}(t)
 \left(\vphantom{\fr{1}{\mathbf{E}\left\{\widetilde{I}_n(t)|\G (n-1)\right\}}}
 X(t)-{}\right.\right.\\[2pt]
\left.\left. {}-\fr{1}{\mathbf{E}\left\{\widetilde{I}_n(t)|\G (n-1)\right\}}\,
\mathbf{E}\left\{X(t)\widetilde{I}_n(t)|\G (n-1)\right\}\right)\right\}={}\\[2pt]
{} =\mathbf{E}\left\{\mathbf{E}\left\{
\mathbf{I}_A(\omega)\mathbf{I}_{[\tau_{n-1},\tau_n)}(t)\left(X(t)-
\vphantom{\overline{X}_n\Bigl|}{}\right.\right.\right.\\[2pt]
\left.\left.\left.{}-\overline{X}_n(t)\right)\Bigl|\G (n-1)\right\}\right\}=0.
 \end{multline*}

 Доказательство равенств $\me{}{\mathbf{I}_B(\omega)(X(t)\hm-\breve{X}_n(t))}\hm=0$
 для $B \hm= A \bigcup \{\tau_n > t\}$, $B \hm= A \bigcap
 \{\tau_n \hm\leqslant t\}$ и $B \hm= A \bigcup \{\tau_n \hm\leqslant t\}$
 проводится аналогично. Таким образом, доказано, что $\breve{X}_n(t) \hm=
 \me{}{X(t)|\G (n-1) \bigvee \{\tau_n > t\}}$ $\pc$--п.~н., из чего
 непосредственно следует истинность формулы~(\ref{eq:Xhat_2}).

 Естественный поток $\sigma$-алгебр $\G_t$,
 порожденный процессом наблюдений в непрерывном времени $Y(t)$, совпадает с
 естественным потоком, порожденным стохастической мерой~$\mu$, т.\,е.\
 \begin{equation*}
 \G_t = \sigma\{\mu(A \times B):\; A \in \mathcal{B}([0,t])\,,
 B \in \mathbb{R}^M\}\,.
 \end{equation*}
 Поэтому для доказательства того, что оценка $\widehat{X}(t)$,
 вычисленная по формуле~(\ref{eq:Xhat}), действительно является УМО
 относительно $\G_t$, достаточно показать, что
 $\me{}{(X(t) \hm- \widehat{X}(t))\mu([a,b) \times\{f_j\})}\hm\equiv 0$
 для любых $0 \hm\leqslant a \hm< b \hm\leqslant t$ и $j\hm=\overline{1,M}$.

 Свойства УМО позволяют получить следующую цепочку равенств:
 \begin{multline*}
 \me{}{(X(t) - \widehat{X}(t))\mu([a,b) \times\{f_j\})} ={} \\
 {}=\!\!\!\!
 \sum\limits_{m \leqslant n-1}\!\!\!\!\!\mathbf{E}\!\left\{\mathbf{I}_{[a,b)}(\tau_m)
 \mathbf{I}_{[\tau_{n-1},\tau_n)}(t)\left(X(t)-
 \overline{X}_n(t)\right)Y_m^{\top}\right\}\!f_j={}\hspace*{-0.13318pt}\\
 {}=  \sum\limits_{m \leqslant n-1}
 \mathbf{E}\left\{\mathbf{E}\left\{
 \mathbf{I}_{[a,b)}(\tau_m)\mathbf{I}_{[\tau_{n-1},\tau_n)}(t)
 \left(X(t)-\vphantom{\overline{X_n}}
  {}\right.\right.\right.\\
\left.\left.\left. {}- \overline{X}_n(t)\right)Y_m^{\top}|\ff^X\vee\G(n-1)\right\}\right\}f_j={}\\
{} =  \sum\limits_{m \leqslant n-1}\me{}{\mathbf{I}_{[a,b)}(\tau_m)
 \widetilde{I}_n(t)\left(X(t)-
 \overline{X}_n(t)\right)Y_m^{\top}  }f_j={}\\
 {}=  \sum\limits_{m \leqslant n-1}
 \mathbf{E}\left\{\mathbf{E}\left\{\mathbf{I}_{[a,b)}(\tau_m)
 \widetilde{I}_n(t)\left(\vphantom{\overline{X}_n}
 X(t)- {}\right.\right.\right.\\
 \left.\left.\left. {}- \overline{X}_n(t)\right)Y_m^{\top}
 \Big\vert\G(n-1)\right\}\right\}f_j={}
 \end{multline*}

 \noindent
 \begin{multline*}
 {}=  \sum\limits_{m \leqslant n-1}\mathbf{E}\left\{\mathbf{I}_{[a,b)}(\tau_m)
 \mathbf{E}\left\{\widetilde{I}_n(t)\left(\vphantom{\overline{X}_n}
 X(t)- {}\right.\right.\right.\\
\left.\left.\left. {}- \overline{X}_n(t)\right)
 \Big \vert\G(n-1)\right\}Y_m^{\top}\right\}f_j=0\,.
 \end{multline*}
 Истинность представления~(\ref{eq:Xhat}) для $\me{}{X(t)|\G_t}$ доказана.
 Формула~(\ref{eq:nu_fin}) получается непосредственно из~(\ref{eq:nu}) и
 соотношения $\mathbf{I}_{[\tau_{n-1},\tau_n)}(t)\overline{X}_n(t) \hm=
 \mathbf{I}_{[\tau_{n-1},\tau_n)}(t)\widehat{X}(t)$. Теорема~1 доказана.

\smallskip

\noindent
Д\,о\,к\,а\,з\,а\,т\,е\,л\,ь\,с\,т\,в\,о\ теоремы~2.
В~силу включения ${\G_t\subseteq\ff_t}$ из уравнения~\eqref{eq:state}
следует, что найдется такой $\G_t$-со\-гла\-со\-ван\-ный
мартингал $\widehat{M}^{X}(t)$, что
\begin{equation}
    \widehat{X}(t)=p_0+\int\limits_0^t\Lambda^{\top}(s)\widehat{X}(s-)
\,    ds+\widehat{M}^{X}(t).
\label{eq:martrepr}
\end{equation}
Общий вид $\widehat{M}^{X}(t)$ представлен в~\cite{LS_86} (теорема~4.10.1),
и с учетом специфических условий рассматриваемой задачи оценивания этот
мартингал представим в виде интеграла по случайной мере
\begin{equation}
    \widehat{M}^{X}(t)=\int\limits_{[0,t]\times\mathbb{R}^M}H(s,y)
    \left(\mu(ds,dy)-\nu(ds,dy)\right)\,,
\label{eq:mhat}
\end{equation}
где
\begin{equation}
\left.
\begin{array}{rl}
\displaystyle
U(s,y) &\ebd \mmme{\mu}{P}{X|\widetilde{\ppp}(\mathcal{G})}(s,y)- \widehat{X}(s-)\,;
\\[6pt]
\displaystyle
H(s,y) &\ebd U(s,y)+{}\\
&\hspace*{-4mm}{}+ \fr{\displaystyle \bi_{\{0 < \alpha(s) <1\}}(s)}
{\displaystyle 1-\alpha(s)}
\int\limits_{\mathbb{R}^M}U(s,z)\,\nu(\{s\}, dz)\,;
\\[6pt]
\displaystyle
\alpha(s)& \ebd \nu\left(\{s\}\times \mathbb{R}^M\right)\,.
\end{array}
\right\}
\label{eq:not}
\end{equation}
 Ключевым в выводе уравнения оптимальной фильт\-ра\-ции является выражение
 случайного процесса $\mmme{\mu}{P}{X|\widetilde{\ppp}(\mathcal{G})}(s,y)$~---
 \textit{УМО относительно $\sigma$-ал\-геб\-ры $\widetilde{\mathcal{P}}(\G)$}.
 Это такая предсказуемая случайная функция, что
\begin{multline}
    \mathbf{E}\left\{\int\limits_{\mathbb{R}_{+}\times \mathbb{R}^M}\!\!\!\psi(s,y)X(s)\,\mu(ds,dy)\right\}={}\\
    {}=
    \mathbf{E}\left\{\int\limits_{\mathbb{R}_{+}\times \mathbb{R}^M}\!\!\!\psi(s,y)\,
    \mmme{\mu}{\mathbf{P}}{X|\widetilde{\mathcal{P}}(\G)}\!(s,y)\,\nu(ds,dy)\right\}\!
    \label{eq:cmemu}
\end{multline}
для любой ограниченной неотрицательной пред\-ска\-зу\-емой случайной
функции ${\psi(s,y)}$. Можно проверить, что такая функция~$\psi$ может быть
представлена в~виде:
\begin{equation*}
 \psi(s,y)=\sum\limits_{n \in \mathbb{N}}
 \overline{h}^{\top}\!y\,v_n(s)\mathbf{I}_{(\tau_{n-1},\tau_n]}(s)\,,
% \label{eq:psi}
\end{equation*}
где $\overline{h}\ebd\col(h^1,\ldots,h^M)$~--- детерминированный вектор,
$\{v_n(s)\}_{n \in \mathbb{N}}$~--- последовательность
$\mathcal{G}(n-1)$-из\-ме\-ри\-мых функций.
Из определения интеграла по~$\mu$ следует, что

\noindent
\begin{multline*}
 \me{}{\int\limits_{\mathbb{R}_{+}\times \mathbb{R}^M} \psi(s,y)X(s)\mu(ds,dy)} ={}\\
 {}=
 \me{}{\sum\limits_{n \in \mathbb{N}} \overline{h}^{\top}\!Y_n v_n(\tau_n)X(\tau_n)}\,.
\end{multline*}
Обозначим $\mathcal{L}_n \ebd \me{}{\overline{h}^{\top}\!Y_n v_n(\tau_n)X(\tau_n)}$
и выразим его в виде интеграла по мере~$\nu$ подобно~(\ref{eq:cmemu}):
\begin{multline*}
\mathcal{L}_n = \me{}{\me{}{\overline{h}^{\top}\!Y_n v_n(\tau_n)X(\tau_n)|
\ff^X\vee \G(n-1)}}={}\\
{}=
\sum\limits_{j=1}^M \overline{h}^{\top}f_j
\mathbf{E}\left\{\int\limits_{\tau_{n-1}}^{\infty}v_n(s)X(s)f^{\top}_j\left(
\phi^0_n(s)X(\tau_{n-1})+{}\right.\right.\\
\left.\left.{}+\phi^1_n(s)X(s-)\right)
\widetilde{I}_{n}(s-)\Phi_n(ds)
\vphantom{\int\limits_{\tau_{n-1}}^{\infty}}
\right\}={}\\
{}= \sum\limits_{j=1}^M \overline{h}^{\top}f_j
\me{}{\int\limits_{\tau_{n-1}}^{\infty}\!\!v_n(s)
\beta_n^{\top}(s)f_j\Phi_n(ds)}=
\sum\limits_{j=1}^M \overline{h}^{\top}f_j\times{}\\
{}\times
\me{}{\int\limits_{\tau_{n-1}}^{\infty}\!\!v_n(s)\widehat{\beta}^{\top}_n(s)
f_j\me{}{\widetilde{I}_{n}(s-)|\G(n-1)}
\Phi_n(ds)}\,.
%\label{eq:cmemu_2}
\end{multline*}
 Из уравнения~(\ref{eq:wtIn_2}) для $\widetilde{I}_{n}(t)$ следует,
 что $\widetilde{I}_{n}(\tau_{n-1})\hm=1$, а из условия~({е})~---
$\lim\limits_{t \to \infty}\widetilde{I}_{n}(t)\hm = 0$,
 поэтому выполняются равенства:
\begin{multline*}
  \int\limits_{(\tau_{n-1},+\infty)}\mathbf{1}_M^{\top}\left(
  \phi^0_n(s)X(\tau_{n-1})+{}\right.\\
\left.  {}+\phi^1_n(s)X(s-)\right)\widetilde{I}_{n}(s-)\Phi_n(ds) =1\,;
\end{multline*}

\vspace*{-8pt}

\noindent
\begin{multline*}
 \widetilde{I}_{n}(t)=\int\limits_{(t,+\infty)}\mathbf{1}_M^{\top}\left(
 \phi^0_n(s)X(\tau_{n-1})+{}\right.\\
\left. {}+\phi^1_n(s)X(s-)\right)\widetilde{I}_{n}(s-)\Phi_n(ds)\,;
\end{multline*}

\vspace*{-8pt}

\noindent
\begin{multline*}
 \widetilde{I}_{n}(t-)=\int\limits_{[t,+\infty)}\mathbf{1}_M^{\top}\left(
 \phi^0_n(s)X(\tau_{n-1})+{}\right.\\
\left. {}+\phi^1_n(s)X(s-)\right)\widetilde{I}_{n}(s-)\Phi_n(ds)\,.
\end{multline*}
 Используя введенные процессы~$\beta_n$ и~$\widehat{\beta}_n$, свойства УМО
 и теорему Фубини, преобразования интеграла~$\mathcal{L}_n$ можно продолжить:
 \begin{multline*}
\mathcal{L}_n =
\sum\limits_{j=1}^M \overline{h}^{\top}f_j
\mathbf{E}\left\{\int\limits_{\tau_{n-1}}^{\infty}v_n(s)\widehat{\beta}^{\top}_n(s)
f_j \times{}\right.\\
\left.{}\times\left[\,\int\limits_{[s,+\infty)}\mathbf{1}_M^{\top}\beta_n(u)\mathbf{1}_N
\Phi_n(du)\right] \Phi_n(ds)\right\}={}\\
{}=
\sum\limits_{j=1}^M \overline{h}^{\top}f_j
\mathbf{E}\left\{\int\limits_{\tau_{n-1}}^{\infty} \left[
\int\limits_{\tau_{n-1}}^u v_n(s)\widehat{\beta}^{\top}_n(s)
f_j\Phi_n(ds) \right]\times{}\right.
\end{multline*}

\noindent
\begin{multline*}
\left.{}\times
\mathbf{1}_M^{\top}\beta_n(u)\mathbf{1}_N\Phi_n(du)
\vphantom{\int\limits_{\tau_{n-1}}^{\infty}}
\right\}= \sum\limits_{j=1}^M \overline{h}^{\top}f_j\times{}\\
{}\times
\me{}{\me{}{\int\limits_{\tau_{n-1}}^{\tau_{n}}
v_n(s)\widehat{\beta}^{\top}_n(s) f_j\Phi_n(ds)|\G(n-1) }}={}\\
{}= \me{}{ \int\limits_{(\tau_{n-1},\tau_n] \times \mathbb{R}^M}\hspace*{-7mm}
\overline{h}^{\top}y v_n(s)\widehat{\beta}^{\top}_n(s)y (\mathbf{1}_N^{\top}
\widehat{\beta}^{\top}_n(s)y)^+\nu(ds,dy) }.\hspace*{-7.91641pt}
%\label{eq:cmemu_3}
\end{multline*}
Из последнего равенства следует, что
\begin{multline}
\mmme{\mu}{P}{X|\widetilde{\ppp}(\mathcal{G})}(t,y)={}\\
{}=
\sum\limits_{n \in \mathbb{N}}\mathbf{I}_{(\tau_{n-1},\tau_n]}(t)
\widehat{\beta}^{\top}_n(t)y (\mathbf{1}_N^{\top}\widehat{\beta}^{\top}_n(t)y)^+\,.
\label{eq:cmemu_4}
\end{multline}
Подставим (\ref{eq:cmemu_4}) в~(\ref{eq:not}):
\begin{equation}
\hspace*{-3mm}\left.
\begin{array}{l}
\displaystyle
U(s,y) =\sum\limits_{n \in \mathbb{N}}\mathbf{I}_{(\tau_{n-1},\tau_n]}(s)\times{}\\[12pt]
\hspace*{15mm}{}\times\left[
\fr{1}{\mathbf{1}_N^{\top}\widehat{\beta}^{\top}_n(s)y}
\widehat{\beta}^{\top}_n(s)y-\widehat{X}(s-)
\right]\,;\\[12pt]
\displaystyle
\alpha(s) = \sum\limits_{n \in \mathbb{N}}\mathbf{I}_{(\tau_{n-1},\tau_n]}(s)\mathbf{1}_N^{\top}\widehat{\beta}^{\top}_n(s)\mathbf{1}_M
\Phi_n^d(\{s\})\,;\\[14pt]
\displaystyle
H(s,y) =\sum\limits_{n \in \mathbb{N}}\mathbf{I}_{(\tau_{n-1},\tau_n]}(s)\times{}\\[12pt]
{}\times
\left(\left[
\fr{1}{\mathbf{1}_N^{\top}\widehat{\beta}^{\top}_n(s)y}
\widehat{\beta}^{\top}_n(s)y-\widehat{X}(s-)
\right] + {}\right. \\[12pt]
\hspace*{12mm}\left.{}+
\left( 1 - \mathbf{1}_N^{\top}\widehat{\beta}^{\top}_n(s)
\mathbf{1}_M\Phi_n^d(\{s\})\right)^+{}\times\right.\\[9pt]
\hspace*{1mm}\left.{}\times
\left[
\widehat{\beta}^{\top}_n(s)\mathbf{1}_M -
\widehat{X}(s-) \mathbf{1}_N^{\top}\widehat{\beta}^{\top}_n(s)\mathbf{1}_M
\right]\Phi_n^d(\{s\})
\vphantom{\fr{1}{\mathbf{1}_N^{\top}\widehat{\beta}^{\top}_n(s)y}}
\right)\,.
\end{array}\!\!
\right\}
\label{eq:not_2}
\end{equation}

Далее, комбинируя (\ref{eq:not_2}), (\ref{eq:mhat}) и~(\ref{eq:martrepr}),
можно получить дифференциальную систему, описывающую эволюцию $\widehat{X}(t)$
на полуинтервалах $[\tau_{n-1},\tau_n)$ между наблюде-\linebreak ниями:
\begin{multline}
\widehat{X}(t) = \widehat{X}(\tau_{n-1})+ \int\limits_{\tau_{n-1}}^{t} \Lambda^{\top}(s)\widehat{X}(s-)\,ds-{}\\
{}- \int\limits_{\tau_{n-1}}^{t}\left[
\widehat{\beta}^{\top}_n(s)\mathbf{1}_M -
\widehat{X}(s-) \mathbf{1}_N^{\top}\widehat{\beta}^{\top}_n(s)\mathbf{1}_M
\right]\Phi_n(ds)\,.
\label{eq:Xhat_3}
\end{multline}
Уравнение (\ref{eq:Xhat_3}) не замкнуто относительно неизвестных: помимо
искомой оценки фильтрации $\widehat{X}(s)$ в правую часть~(\ref{eq:Xhat_3})
через матричнозначную функцию $\widehat{\beta}_n(s)$ также входит оценка
$\widehat{X}(\tau_{n-1},s)$ сглаживания в фиксированной точке $\tau_{n-1}$.
В~то же время имеется уравнение~(\ref{eq:Z_sys}), замкнутое относительно
$\overline{Z}_n(\cdot)$. Из~(\ref{eq:Z_sys}) уравнение~(\ref{eq:Xhat_3})
может быть легко получено подстановкой в него~(\ref{eq:hatbeta}).
Агрегируя~(\ref{eq:martrepr}), (\ref{eq:mhat}) и~(\ref{eq:not_2}),
можно также получить алгебраические соотношения, определяющие оценки
фильтрации в моменты~$\tau_n$ получения наблюдений:

\noindent
\begin{equation}
\widehat{X}(\tau_n) =
\begin{cases}
&\displaystyle
\!\!(\mathbf{1}_N^{\top}\widehat{\beta}^{\top}_n(\tau_n)Y_n)^+
\widehat{\beta}^{\top}_n(\tau_n)Y_n\,,\\[3pt]
& \hspace*{2mm}\mbox{если }
\mathbf{1}_N^{\top}\widehat{\beta}^{\top}_n(\tau_n)\mathbf{1}_M\Phi_n^d
(\{\tau_n\})<1\,;\\[6pt]
&\displaystyle
\!\!(\mathbf{1}_N^{\top}\widehat{\beta}^{\top}_n(\tau_n)Y_n)^+
\widehat{\beta}^{\top}_n(\tau_n)Y_n - {}\\[3pt]
&\displaystyle{}-\left[
\widehat{\beta}^{\top}_n(\tau_n)\mathbf{1}_M \Phi_n^d(\{\tau_n\}) -
\widehat{X}(\tau_n-) \right]\,,\\[3pt]
&\hspace*{2mm}\mbox{если }
\mathbf{1}_N^{\top}\widehat{\beta}^{\top}_n(\tau_n)
\mathbf{1}_M\Phi_n^d(\{\tau_n\})=1\,.
\end{cases}
\label{eq:est_in_obs}
\end{equation}
Оценка $\widehat{X}(\tau_n)$~(\ref{eq:est_in_obs}) не является регулярной
версией УМО в случае, если $\mathbf{1}_N^{\top}\widehat{\beta}^{\top}_n(\tau_n)Y_n\hm
=0$ из-за нарушения условия нормировки. Тем не менее в силу следствия~1
$\pp{\mathbf{1}_N^{\top}\widehat{\beta}^{\top}_n(\tau_n)Y_n=0}\hm=0$, и
на пренебрежимом множестве $\{\omega: \;\mathbf{1}_N^{\top}
\widehat{\beta}^{\top}_n(\tau_n)Y_n\hm=0 \}$ искомая оценка может быть
переопределена так, чтобы выполнялось условие регулярности:
$\widehat{X}(\tau_n) \ebd \widehat{X}(\tau_n-)$. Тем самым показано,
что в моменты~$\tau_n$ появления наблюдений оценка фильтрации действительно
определяется формулой~(\ref{eq:Xhat_4}). Теорема~2 доказана.

}

{\small\frenchspacing
 {%\baselineskip=10.8pt
 \addcontentsline{toc}{section}{References}
 \begin{thebibliography}{99}
  \bibitem{IvKK_82}
\Au{Gnedenko~B.\,V., Kovalenko~I.\,N.}
Introduction to queueing theory.~--- Boston: Birkhauser, 1989. 314~p.

   \bibitem{KR_88}
\Au{Kalashnikov~V.\,V.} Mathematical methods for construction of queueing models.~--- N.Y.: Springer-Verlag, 1990.~431~p.

 \bibitem{B_79}
\Au{Bremaud P.} Optimal thinning of a point process~//
SIAM~J.~Contr.~Optim., 1979. Vol.~17. No.\,2. P.~222--230.
doi: 10.1137/0317017.

\bibitem{masm_05}
\Au{Миллер~Б.\,М., Авраченков~К.\,Е., Степанян~К.\,В., Миллер~Г.\,Б.}
Задача оптимального стохастического управ\-ле\-ния потоком данных по
неполной информации~//
Проблемы~передачи~информации, 2005. Т.~41. №\,2. C.~89--110.
doi: 10.1007/s11122-005-0020-8.
\bibitem{SWS_13}
\Au{Shen~B., Wang~Z., Shu~H.} Nonlinear stochastic systems with incomplete
information: Filtering and control.~--- N.Y.: Springer~Verlag, 2013. 248~p.

 \bibitem{AKU_05}
\Au{Anagnostopoulos~A., Kirsch~A., Upfal~E.}
Load balancing in arbitrary network topologies
with stochastic adversarial input~// SIAM~J.~Comput., 2005. Vol.~17. No.\,3. P.~616--639.
doi: 10.1137/S0097539703437831.

 \bibitem{AAP_11}
\Au{Altman~E., Ayesta~U., Prabhu~B.} Load balancing in processor sharing systems~//
Telecommun. Syst., 2011. Vol.~47. No.\,1-2. P.~35--48.
doi: 10.4108/ICST.VALUETOOLS2008.4462.

\bibitem{Bosov_09}
\Au{Босов~А.\,В.} Моделирование и оптимизация процессов функционирования
Информационного web-портала~// Программирование, 2009. №\,6. С.~53--66.
doi: 10.1134/S0361768809060048.

\bibitem{Bosov_11}
\Au{Босов~А.\,В.} Задачи анализа и оптимизации для модели пользовательской
активности. Часть~1. Анализ и прогнозирование~// Информатика и её применения, 2011.
Т.~5. Вып.~4. С.~40--52.

\bibitem{ONA_04}
\Au{Olshefski~D., Nieh~J., Agrawal~D.} Using CERTES to infer client
response time at the web server~// ACM Trans. Comput. Syst., 2004.
Vol.~22. No.\,1. P.~49--93.
doi: 10.1145/511334.511355.

%\bibitem{Al_86}
%{\sl Alonso~R.} Query Optimization in Distributed Databases Through Load Balancing.~//Berkley. 1986.

\bibitem{OV_11}
\Au{Ozsu~M.\,T., Valduriez~P.} Principles of distributed database systems.~---
N.Y.: Springer-Verlag, 2011.~845~p.

\bibitem{ElliottAM_94}
\Au{Elliott~R.\,J., Aggoun~L., Moore~J.\,B.} Hidden Markov models:
Estimation and control.~--- N.Y.: Springer-Verlag, 1994.~382~p.

 \bibitem{LS_86}
\Au{Liptser~R.\,Sh., Shiryayev~A.\,N.} Theory of martingales.~---
N.Y.: Springer-Verlag, 1989.~812~p.

\bibitem{JSh_94}
\Au{Jacod~J., Shiryayev~A.\,N.} Limit theorems for stochastic processes.~---
N.Y.: Springer-Verlag, 2003.~664~p.

\bibitem{El_86}
\Au{Elliott~R.\,J.} Stochastic calculus and applications.~---
N.Y.: Springer-Verlag, 1982.~302~p.

\bibitem{WongH_85}
\Au{Wong~E., Hajek~B.} Stochastic processes in engineering
systems.~--- N.Y.: Springer-Verlag, 1985.~361~p.

\bibitem{bms_14}
\Au{Борисов~А.\,В., Миллер~Б.\,М., Семенихин~К.\,В.}
Фильт\-ра\-ция марковского скачкообразного процесса по
наблюдениям мультивариантного точечного процесса~//
Автоматика и телемеханика, 2014 (в печати).

\medskip

\bibitem{y_77}
\Au{Юшкевич~А.\,А.} Управляемые марковские модели со счетным множеством состояний
и непрерывным временем~// Теория вероятностей и ее применения, 1977. Т.~22. №\,2. C.~222--241.
doi: 10.1137/1122029.

\medskip

\bibitem{KalmanB_60}
\Au{Kalman~R.\,E., Bucy~R.\,S.} New results in linear filtering and
prediction theory~// Trans. ASME. J.~Basic Eng., 1960. Vol.~83D. No.\,1. P.~95--108.

\medskip

%\bibitem{Won_65}
%\Au{Wonham~W.\,W.} Some applications of stochastic differential equations
%to optimal nonlinear filtering~// SIAM J.~Control., 1964. Vol.~2. No.\,3. P.~347--369.

\bibitem{CLR_06}
\Au{Cvitanic~J., Liptser~R., Rozovskii~B.} A~filtering approach to
tracking volatility from prices observed at random times~//
Ann. Appl. Probab., 2006. Vol.~16. No.\,3. P.~1633--1652.
doi: 10.1214/105051606000000222.
 \end{thebibliography}

 }
 }

\end{multicols}

%\vspace*{-9pt}

\hfill{\small\textit{Поступила в редакцию 27.06.14}}

%\newpage

\vspace*{10pt}

\hrule

\vspace*{2pt}

\hrule

%\vspace*{12pt}

\def\tit{MONITORING REMOTE SERVER ACCESSIBILITY:\\ THE~OPTIMAL FILTERING APPROACH}

\def\titkol{Monitoring remote server accessibility: The optimal filtering approach}

\def\aut{A.\,V.~Borisov$^{1,2}$}

\def\autkol{A.\,V.~Borisov}

\titel{\tit}{\aut}{\autkol}{\titkol}

\vspace*{-9pt}

\noindent
$^1$Institute of Informatics Problems, Russian Academy of Sciences,
44-2 Vavilov Str., Moscow 119333, Russian\\
$\hphantom{^1}$Federation

\noindent
$^2$Department of Probability Theory, School of Applied Mathematics and Physics,
Moscow Aviation Institute,\\
$\hphantom{^1}$4~Volokolamskoe Shosse, GSP-3, A-80, Moscow 125993,
Russian Federation


\def\leftfootline{\small{\textbf{\thepage}
\hfill INFORMATIKA I EE PRIMENENIYA~--- INFORMATICS AND
APPLICATIONS\ \ \ 2014\ \ \ volume~8\ \ \ issue\ 3}
}%
 \def\rightfootline{\small{INFORMATIKA I EE PRIMENENIYA~---
INFORMATICS AND APPLICATIONS\ \ \ 2014\ \ \ volume~8\ \ \ issue\ 3
\hfill \textbf{\thepage}}}

\vspace*{3pt}


\Abste{The online monitoring problem of a remote server, accessible via the
 {\sf http} protocol, is formulated in the terms of optimal filtering.
 The unobservable server state is treated as a finite-dimensional Markov
 jump process, meanwhile the observation is supposed to be a multivariate
 point process with a finite set of possible values.
The key point of the investigated observation system is that the random
intensity of observations is a linear function of the unobservable Markov state.
It is proved that the optimal filtering estimate is a solution to some closed finite
system of recursive formulae and ordinary linear differential equations with a
random right-hand side. The applicability of the obtained theoretical results
is illustrated by an example of monitoring accessibility of the queueing system
``communication channel\,--\,database server.''
The unobservable state of this system consists of three possible values
(no connection, low workload, high workload), meanwhile the possible
observations belong to the set of two possible values (answer to the query,
error message). The conclusion of the paper contains possible prospectives
for the further research.}

\KWE{Markov models; optimal filtering; stochastic jump processes; conditional probability distribution;
queueing theory}

\DOI{10.14357/19922264140307}

\vspace*{-9pt}

\Ack
\noindent
The research is supported by the Russian Foundation for Basic Research
(grants Nos.~13-01-00406 and 13-07-00408).

%\vspace*{3pt}

  \begin{multicols}{2}

\renewcommand{\bibname}{\protect\rmfamily References}
%\renewcommand{\bibname}{\large\protect\rm References}

{\small\frenchspacing
 {%\baselineskip=10.8pt
 \addcontentsline{toc}{section}{References}
 \begin{thebibliography}{99}
\bibitem{1-b}
\Aue{Gnedenko,~B.V., and I.\,N.~Kovalenko}. 1989.
\textit{Introduction to queueing theory}. Boston: Birkhauser. 314~p.

\bibitem{2-b}
\Aue{Kalashnikov,~V.\,V.}
1990. \textit{Mathematical methods for construction of queueing models}.
N.Y.: Springer-Verlag.~431~p.

\bibitem{3-b}
\Aue{Bremaud,~P.} 1979. Optimal thinning of a point process.
\textit{SIAM J.~Contr. Optim.} 17(2):222--230. doi: 10.1137/0317017.

\bibitem{4-b}
\Aue{Miller,~B.\,M., K.\,E.~Avrachenkov, K.\,V.~Stepanyan, and G.\,B.~Miller}.
2005. Flow control as stochastic optimal control problem with incomplete
information. \textit{Problems Information Transmission}
41(2):150--170. doi: 10.1007/s11122-005-0020-8.

\bibitem{5-b}
\Aue{Shen,~B., Z.~Wang, and H.~Shu}. 2013.
\textit{Nonlinear stochastic systems with incomplete information: Filtering and
control}. N.Y.: Springer Verlag. 248~p.

\bibitem{6-b}
\Aue{Anagnostopoulos,~A., A.~Kirsch, and E.~Upfal}. 2005.
Load balancing in arbitrary network topologies
with stochastic adversarial input. \textit{SIAM J.~Comput.}
17(3)616--639. doi: 10.1137/S0097539703437831.

\bibitem{7-b}
\Aue{Altman,~E., U.~Ayesta, and B.~Prabhu}. 2011.
Load balancing in processor sharing systems. \textit{Telecommun. Syst.}
47(1-2):35--48. doi: 10.4108/ICST.VALUETOOLS2008.4462.

\bibitem{8-b}
\Aue{Bosov,~A.\,V.} 2009. Modeling and
optimization of functioning of the Information Web Portal.
\textit{Programming Computer Software} 35(6):340--350
doi: 10.1134/S0361768809060048.

\bibitem{9-b}
\Aue{Bosov,~A.\,V.} 2012. Zadachi analiza i optimizatsii dlya modeli
pol'zovatel'skoy aktivnosti. Chast'~1. Analiz i prognozirovanie
[Analysis and optimization problems
for some users activity model. Part~1. Analysis and prediction].
\textit{Informatika i ee Primeneniya}~--- \textit{Inform.
Appl.} 5(4):40--52.

\bibitem{10-b}
\Aue{Olshefski,~D., J.~Nieh, and D.~Agrawal}. 2004.
Using CERTES to infer client response time at the web server.
\textit{ACM Trans. Comput. Syst.} 22(1):49--93.
doi: 10.1145/511334.511355.

\bibitem{11-b}
\Aue{Ozsu,~M.\,T., and P.~Valduriez}. 2011.
Principles of distributed database systems. N.Y.: Springer-Verlag. 845~p.

\bibitem{12-b}
\Aue{Elliott,~R.\,J., L.~Aggoun, and J.\,B.~Moore}. 1994.
\textit{Hidden Markov models: Estimation and control}.
N.Y.: Springer-Verlag. 382~p.

\bibitem{13-b}
\Aue{Liptser,~R.\,Sh., and A.\,N.~Shiryayev}. 1989.
\textit{Theory of martingales}. N.Y.: Springer-Verlag. 812~p.

\bibitem{14-b}
\Aue{Jacod,~J., and A.~Shiryayev}. 2003.
\textit{Limit theorems for stochastic processes}. N.Y.: Springer-Verlag. 664~p.

\bibitem{15-b}
\Aue{Elliott,~R.\,J.} 1982.
\textit{Stochastic calculus and applications}. N.Y.: Springer-Verlag. 302~p.

\bibitem{16-b}
\Aue{Wong,~E., and B.~Hajek}. 1985. \textit{Stochastic processes in engineering
systems}. N.Y.: Springer-Verlag. 361~p.

\bibitem{17-b}
\Aue{Borisov,~A.\,V., B.\,M.~Miller, and K.\,V.~Semenikhin}. 2014 (in press).
Filtering of Markov jump process given the observations the observations of
multivariate point process. \textit{Autom. Rem. Contr.}

\bibitem{18-b}
\Aue{Yushkevich,~A.\,A.}
1978. Controlled Markov models with countable state space and continuous time.
\textit{Theory Probabl. Appl.} 22(2):215--235. doi: 10.1137/1122029.

\bibitem{19-b}
\Aue{Kalman,~R.\,E., and R.\,S.~Bucy}. 1960.
New results in linear filtering and prediction theory.
\textit{Trans. ASME. J.~Basic Eng.} 83D(1):95--108.

%\bibitem{20-b}
%\Aue{Wonham,~W.\,W.} 1965. Some applications of stochastic differential equations
%to optimal nonlinear filtering. \textit{SIAM J.~Control.} 2(3):347--369.

\bibitem{21-b}
\Aue{Cvitanic,~J., R.~Liptser, and B.~Rozovskii}. 2006.
A~filtering approach to tracking volatility from prices observed at random times.
\textit{Ann. Appl. Probab.} 16:1633--1652.
doi: 10.1214/105051606000000222.


\end{thebibliography}

 }
 }

\end{multicols}

\vspace*{-6pt}

\hfill{\small\textit{Received June 27, 2014}}

\vspace*{-18pt}

\Contrl

  \noindent
  \textbf{Borisov Andrey V.} (b.\ 1965)~--- Doctor of Science in
physics and mathematics, leading scientist,
Institute of Informatics Problems, Russian
Academy of Sciences, 44-2 Vavilov Str., Moscow 119333, Russian Federation;
professor, Department of Probability Theory, School of Applied Mathematics and
Physics, Moscow Aviation Institute, 4~Volokolamskoe Shosse, GSP-3, A-80, Moscow
125993, Russian Federation; ABorisov@ipiran.ru


\label{end\stat}

\renewcommand{\bibname}{\protect\rm Литература}