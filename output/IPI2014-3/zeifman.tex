
\newcommand{\A}{{\mathbf A}}
\newcommand{\B}{{\mathbf B}}

\def\stat{zeifman}



\def\tit{ОБ ОЦЕНКАХ СКОРОСТИ СХОДИМОСТИ И УСТОЙЧИВОСТИ ДЛЯ НЕКОТОРЫХ МОДЕЛЕЙ
МАССОВОГО ОБСЛУЖИВАНИЯ$^*$}



\def\titkol{Об оценках скорости сходимости и устойчивости для некоторых моделей
массового обслуживания}

\def\aut{А.\,И.~Зейфман$^1$,
  А.\,В.~Коротышева$^2$, К.\,М.~Киселева$^3$,
  В.\,Ю.~Королев$^4$, С.\,Я.~Шоргин$^5$}

\def\autkol{А.\,И.~Зейфман,   А.\,В.~Коротышева, К.\,М.~Киселева и др.}

\titel{\tit}{\aut}{\autkol}{\titkol}

{\renewcommand{\thefootnote}{\fnsymbol{footnote}} \footnotetext[1]
{Исследование выполнено за счет гранта Российского научного фонда (проект
№\,14-11-00397).}}


\renewcommand{\thefootnote}{\arabic{footnote}}
\footnotetext[1]{Институт проблем информатики Российской академии наук; Вологодский государственный университет;
Институт социально-экономического развития территорий Российской
академии наук, a\_zeifman@mail.ru}
\footnotetext[2]{Вологодский государственный университет,
a\_korotysheva@mail.ru}
\footnotetext[3]{Вологодский
государственный университет, a\_zeifman@mail.ru}
\footnotetext[4]{Институт проблем информатики Российской академии
наук; Московский государственный университет им. М.~В. Ломоносова,
 vkorolev@cs.msu.su}
 \footnotetext[5]{Институт проблем информатики Российской академии наук, SShorgin@ipiran.ru}

 \vspace*{-6pt}

\Abst{Рассматривается некоторое обобщение известной
модели Эрланга с потерями, а именно: изуча\-ется класс марковских
моделей систем обслуживания, в которых допускается одновременное
поступление группы требований (ограниченное максимальным общим
количеством требований) и предусмотрено групповое обслуживание.
Установлен критерий слабой эргодичности для процесса, описывающего
число требований в такой системе, получены оценки скорости
сходимости и устойчивости. Исследование опирается на общий подход,
разработанный в предыдущих работах авторов для неоднородных
марковских  систем с групповым поступлением и обслуживанием
требований. Рассмотрены также конкретные модели с периодическими
интенсивностями при разном максимальном размере группы поступающих
требований, строятся основные предельные характеристики этих моделей
и выясняется влияние максимально допустимого размера группы
одновременно поступающих требований на предельное среднее для числа
требований в системе и предельную вероятность отсутствия требований
в системе.}

\KW{нестационарная марковская система
обслуживания; модель Эрланга; групповое поступление и обслуживание
требований; эргодичность; устойчивость; оценки}

\DOI{10.14357/19922264140303}

%\vspace*{9pt}


\vskip 14pt plus 9pt minus 6pt

\thispagestyle{headings}

\begin{multicols}{2}

\label{st\stat}

\section{Введение}

Стационарная и нестационарная модели Эрланга для системы с потерями
изучались во многих работах (см., например,~[1--10]).
По-видимому, в большой степени это связано со сравнительной
простотой исследования и удобством применения этой модели.
В~\cite{z13a} была рассмотрена аналогичная модель, в которой
допускается групповое обслуживание требований, получен критерий
слабой эргодичности для такой модели и оценки ско\-рости сходимости, а
в~\cite{z14c} получены первые оценки устойчивости этой модели по
отношению к малым возмущениям интенсивностей поступления и
обслуживания требований.

В настоящей заметке рассматривается некоторый более общий класс
моделей систем об\-служивания, в которых допускается и групповое
поступление требований, установлен критерий слабой эргодич\-ности для
них, получены оценки скорости сходимости и устойчивости.
Исследование опирается на общий подход, разработанный в предыдущих
работах авторов для некоторых классов неоднородных марковских систем
с групповым поступлением и обслуживанием требований~[12--14].
{\looseness=1

}


Рассмотрим систему обслуживания <<с потерями>>, в которой общее
количество требований не превосходит~$S$,  максимальный размер
группы поступающих требований равен $N \hm\le S$, интенсивность
поступления группы $k \hm\le N$ требований есть
$\lambda_{k}(t)\hm={\lambda (t)}/{k}$. Интенсивность обслуживания
группы $k \le S$ имеющихся в системе требований есть
$\mu_{k}(t)\hm={\mu (t)}/{k}$. Пусть $X\hm=X(t)$, $t\hm\geq 0$,~-- число
требований в системе обслуживания, это неоднородная марковская цепь
с непрерывным временем и пространством  состояний $E\hm=\{0,1, \ldots,
S \}$. <<Базовые>> интенсивности поступления и обслуживания
$\lambda(t)$ и $\mu(t)$  предполагаются локально интегри\-ру\-емы\-ми на
$[0,\infty)$ функциями времени~$t$.


Тогда для описания вероятностной динамики процесса получаем прямую систему
Колмогорова в виде:
\begin{equation*}
%\label{ur_1}
\fr{d\mathbf{p}}{dt}=A(t)\mathbf{p}(t)\,,
\end{equation*}
где
\begin{multline*}
A(t)={}\\
\!{}=\left(
\begin{array}{ccccccc}
a_{00}(t) & \mu_1(t)  & \mu_2(t)   & \mu_3(t)  & \cdots & \mu_S(t)  \\
\la_1(t)   & a_{11}(t)  & \mu_1(t)  & \mu_2(t)    & \cdots & \mu_{S-1}(t)   \\
\la_2(t)  & \la_1(t)    & a_{22}(t)& \mu_1(t)    &  \cdots & \mu_{S-2}(t)  \\
\cdots \\
\la_S(t)  & \la_{S-1}(t)  & \la_{S-2}(t) & \la_{S-3}(t) & \cdots   & a_{SS}(t)  \\
\end{array}
\right),\hspace{-0.44418pt}
\end{multline*}
\noindent
причем
$$
a_{ii}(t)=-\sum\limits_{k=1}^{i}\mu_k(t) -
\sum\limits_{k=1}^{S-i} \la_{k}(t)\,;\quad
\lambda_{k}(t)= 0
$$
 при k > N\,.


Обозначим через
\begin{multline*}
p_{ij}(s,t)=\mathrm{Pr}\left\{ X(t)=j\left| X(s)=i\right.
\right\}\,,\\ i,j \ge 0\,,\enskip 0\leq s\leq t\,,
\end{multline*}
переходные вероятности
процесса $X=X(t)$, а через
$$
p_i(t)=\mathrm{Pr}\left\{ X(t) =i \right\}
$$
его вероятности состояний.


Будем обозначать через $\|\bullet\|$   $l_1$-нор\-му вектора и
матрицы, т.\,е.\  $\|{\bf x}\|\hm=\sum|x_i|$ и $\|B\| \hm= \max_j \sum\limits_i
|b_{ij}|$ при $B \hm= (b_{ij})_{i,j=0}^{S}$, а через  $\Omega$~---
множество всех стохастических векторов, т.\,е.\ множество векторов с
неотрицательными координатами и единичной $l_1$-нор\-мой.

Через
$$
E(t,k) = E\left\{X(t)\left|X(0)=k\right.\right\}
$$
будем далее обозначать математическое ожидание процесса (среднее число
требований) в момент~$t$ при условии, что в нулевой момент времени
он находится в состоянии~$k$.

Напомним, что марковская цепь $X(t)$ называется слабо эргодичной,
если $\| {\bf p^*}(t) \hm- {\bf p^{**}}(t) \| \hm\to 0 $ при $t \hm\to
\infty$ для любых начальных условий  ${\bf p^*}(s)$, ${\bf p^{**}}(s)$
и любом $ s \hm\ge 0$. Марковская цепь $X(t)$ имеет предельное среднее
$\phi (t)$, если $E(t,k)\hm - \phi (t)\hm \to 0$ при $t \hm\to \infty$ и
любом~$k$.

\section{Слабая эргодичность и~скорость сходимости}

\noindent
\textbf{Теорема~1.} \textit{Процесс $X(t)$,
описывающий число требований в рассматриваемой сис\-те\-ме,
слабо эргодичен}\linebreak\vspace*{-12pt}
\columnbreak

\noindent
\textit{при любом $N$ ($1 \hm\le N \hm\le S$)
тогда и только тогда, когда выполнено условие}:
\begin{equation}
\int\limits_0^\infty \left(\lambda(t) + \mu(t) \right)\, dt = + \infty\,.
\label{021}
\end{equation}


\noindent
Д\,о\,к\,а\,з\,а\,т\,е\,л\,ь\,с\,т\,в\,о\,.\ \
Отметим, прежде всего, что если~(\ref{021}) не выполнено, то
\begin{multline*}
\|A(t)\| \hm= 2 \max |a_{ii}(t)| \le {}\\
{}\le 2\left(
\lambda(t)+\mu(t)\right)\sum\limits_{k=1}^S\fr{1}{k} \le
2\left(1+\ln S\right)\left(\lambda(t)+ \mu(t)\right)
\end{multline*}
 и, значит,
$$
\int\limits_0^\infty  \|A(t)\|\, dt \hm< + \infty\,.
$$
А~тогда
в~соответствии с теоремой~3.3 из~\cite{z94}
 $X(t)$ слабо эргодичным быть не может.

Пусть~(\ref{021}) выполняется. Пользуясь стандартным приемом и
полагая $p_0\hm=1\hm-\sum\limits_{1\le i \le S}{p_i}$, получаем:
\begin{equation*}
%\label{ur_10}
\fr{d{\bf z}}{dt}=B(t){\bf z}(t)+{\bf f}(t)\,,
\end{equation*}
где $\vf(t)=\left(\la_1, \la_2,\cdots,\la_S \right)^{\mathrm{T}}$;
\begin{multline*}
\hspace*{-3mm}B(t)=\left(
\begin{array}{ccccc}
a_{11}- \la_1   & \mu_1 - \la_1   & \mu_2 - \la_1   & \mu_3 -\la_1   & \cdots\\
\la_1 -\la_2    & a_{22} -\la_2  & \mu_1-\la_2   & \mu_2 -\la_2      & \cdots\\
\la_2 -\la_3    & \la_1 -\la_3   & a_{33} -\la_3  & \mu_1-\la_2      & \cdots\\
\cdots &\cdots&\cdots&\cdots                                         &\cdots \\
\la_{S-1} -\la_S  &\la_{S-2} -\la_S & \cdots & \cdots                &\cdots
\end{array}
\right.\hspace*{-4.8643pt}\\
\left.
\begin{array}{cccc}
\cdots & \cdots &\cdots& \mu_{S-1}-\lambda_1\\
\cdots&\cdots&  \cdots & \mu_{S-2} -\la_2 \\
\cdots &\cdots&\cdots& \mu_{S-3}-\lambda_3\\
\cdots&\cdots&\cdots&\cdots\\
\cdots& \la_2-\la_S&\la_1-\la_S& a_{SS}-\la_S
\end{array}
\right)\,.
\end{multline*}

Рассмотрим треугольную матрицу
\begin{equation*}
D=\begin{pmatrix}
d_1   & d_1 & d_1 & \cdots & d_1 \\
0   & d_2  & d_2  &   \cdots & d_2 \\
\cdots &\cdots & \cdots & \cdots&\cdots\\
0  & 0 & \cdots & 0 &  d_S
\end{pmatrix}
\end{equation*}
с положительными элементами~$d_i$ и соответст\-ву\-ющую норму
 $\|{\bf z}\|_{D}\hm =\|D {\bf z}\|_1$.
Тогда получим равенство:

\pagebreak

\end{multicols}

{ %\scriptsize
\begin{equation*}
DB(t)D^{-1}= %{}\\
%{}=
\begin{pmatrix}
a_{11}-\la_S    &  (\mu_1-\mu_2) \fr{d_1}{d_2}
 & (\mu_2-\mu_3)\fr{d_1}{d_3}
 & \cdots &  (\mu_{S-1}-\mu_S)\fr{d_1}{d_S} \\
(\la_1-\la_S) \fr{d_2}{d_1}     &  a_{22}-\la_{S-1}
 &(\mu_1-\mu_3)\fr{d_2}{d_3}
 & \cdots &  (\mu_{S-2}-\mu_S)\fr{d_2}{d_S} \\
(\la_2-\la_S) \fr{d_3}{d_1}     &  (\la_1-\la_{S-1})\fr{d_3}{d_2}
 &a_{33}-\la_{S-2}
& \cdots &  (\mu_{S-3}-\mu_S)\fr{d_3}{d_S}  \\
\cdots \\
(\la_{S-1} -\la_S) \fr{d_S}{d_1} & (\la_{S-2} -\la_{S-1}) \fr{d_S}{d_2}
& (\la_{S-3} -\la_{S-2}) \fr{d_S}{d_3}
    & \cdots & a_{SS}-\la_1 \\
\end{pmatrix}\,,
\end{equation*}
%\end{multline}}

\hrule

\begin{multicols}{2}

\noindent
дающее соответствующее выражение для логарифмической нормы $B(t)$
(см.\ подробное обсуждение в~\cite{gz04,z95,z06}):
\begin{multline}
\gamma \left(B(t)\right)_{1D} = \gamma \left(DB(t)D^{-1}\right) = {} \\
{}=\max\limits_{1 \le i \le S} \left(a_{ii}(t) -\lambda_{S+1-i}(t) +{}\right.\\
\left.{}+
\sum\limits_{k=1}^{i-1}(\mu_{i-k}(t)-\mu_i(t))\fr{d_k}{d_i} +{} \right. \\
{}+\left.\sum\limits_{k=1}^{S-i}(\lambda_k (t)-\lambda_{S+1-i})\fr{d_{k+i}}{d_i}\right).
\label{031}
\end{multline}

\medskip

Пусть вначале
\begin{equation}
\label{030}
\int\limits_0^{+\infty} \mu(t)\,dt = +\infty\,.
\end{equation}

Положим все $d_i\hm=1$. Тогда справедлива следу\-ющая оценка:
\begin{multline*}
%\label{032}
\gamma \left(B(t)\right)_{1D} \le {}\\
{}\le
\max_{1 \le i \le S} \left(-
\sum\limits_{k=1}^{i} \mu_{i}(t) +
\sum\limits_{k=1}^{i-1}(\mu_{i-k}(t)-\mu_i(t)) \right) = {}\\
{}= -\min\limits_{1 \le i \le S} \left( k\mu_k(t)\right) = -  \mu (t)\,.
\end{multline*}
Имеем  при этом
$$
\|D\|=\sum\limits_{i=1}^S d_i=S;\
 \|D^{-1}\|=2\max\limits_{1\le k \le S}\left(\fr{1}{d_k}\right)=2,
 $$
 а значит,
\begin{multline*}
\|\mathbf{p}^*(t)-\mathbf{p}^{**}(t)\| \le
2\|\vz^*(t)-\vz^{**}(t)\| ={}\\
{}=  2\|D \left(\vz^*(t)-\vz^{**}(t)\right) D^{-1}\| \le  {}\\
{}\le 4 \|\vz^*(t)-\vz^{**}(t)\|_{1D}
 \le{}\\
 {}\le 4e^{-\int\limits_s^t {\mu(u)\,du}}\|\vz^*(s)-\vz^{**}(s)\|_{1D} \le{}\\
 {}\le 4S e^{-\int\limits_s^t {\mu(u)\,du}}\|\vz^*(s)-\vz^{**}(s)\| \le{} \\
{}\le 4Se^{-\int\limits_s^t {\mu(u)\,du}}\|\mathbf {p}^*(s)-\mathbf {p}^{**}(s)\| \le
8Se^{-\int\limits_s^t {\mu(u)\,du}}
%\label{033}
\end{multline*}
для любых начальных условий ${\bf p^*}(s), {\bf p^{**}}(s)$ и
любых $s,t$, $0\hm\le s\hm\le t$. Отсюда вытекает слабая эргодичность $X(t)$.

\medskip

Пусть теперь
\begin{equation}
\int\limits_0^\infty  \lambda(t)\, dt = + \infty\,.
\label{040}
\end{equation}

Положим $d_k={1}/{k}$. Тогда из~(\ref{031})  получаем такую
оценку логарифмической нормы  $B(t)$:
\begin{equation*}
\gamma \left(B(t)\right)_{1D} \le -
\min\limits_{1 \le i \le S} \left( \mu_i(t) + \fr{1}{i}\,\lambda_1(t)\right)
\le -\fr{\lambda(t)}{S}\,.
%\label{041}
\end{equation*}

Теперь
\begin{align*}
\|D\| &= \sum\limits_{i=1}^S d_i \le 1 + \ln S\,;\\
\|D^{-1}\| &= 2\max \left( \fr{1}{d_k}\right) = 2S\,.
\end{align*}
Значит, получаем слабую эргодичность процесса и следующую оценку:
\begin{multline*}
\|{\bf p^*}(t) - {\bf p^{**}}(t)\| \le  2\|D^{-1}D\left({\bf z^*}(t) - {\bf z^{**}}(t)\right)\| \le {} \\
{}\le  8S\left(1+\ln S\right)  e^{-(1/S)\int\limits_s^t\lambda
(\tau)\,d\tau}
%\label{043}
\end{multline*}
при любых начальных условиях ${\bf p^*}(s)$, ${\bf p^{**}}(s)$ и
любых $s,t$, $0 \hm\le s \hm\le t$.

\medskip

\noindent
\textbf{Следствие~1.}
Если~(\ref{021}) справедливо, то процесс $X(t)$ слабо эргодичен,
имеет предельное  среднее $\phi(t)$ и справедливы следующие оценки скорости
сходимости:
\begin{equation*}
\|\mathbf {p}^*(t)-\mathbf {p}^{**}(t)\| \le 8Se^{-\int\limits_0^t {\mu(u)\,du}}
%\label{051}
\end{equation*}
при любых начальных условиях,
\begin{equation*}
|E(t,k)-\phi(t)|\le 8S^2e^{-\int\limits_0^t {\mu(u)\,du}}
%\label{052}
\end{equation*}
при любом $k$,  если выполнено~(\ref{030}), а при выполнении~(\ref{040})
соответственно:

\begin{figure*}[b] %fig1
\vspace*{1pt}
\begin{center}
\mbox{%
\epsfxsize=161.688mm
\epsfbox{zei-1.eps}
}
\end{center}
\vspace*{-15pt}
%\center{\includegraphics[width=10cm]{phi100.eps}}
\Caption{Приближенные значения предельного среднего $\phi^*_{100}(t)$~(\textit{а}) и
предельной <<пустой очереди>> $p^*_{0,100}(t)$~(\textit{б}) }
\end{figure*}


\pagebreak

\noindent
\begin{equation*}
\|{\bf p^*}(t) - {\bf p^{**}}(t)\| \le
8S\left(1+\ln S\right)  e^{-(1/S)\int\limits_0^t\lambda (\tau)\,d\tau};
\label{053*}
\end{equation*}
\begin{equation*}
|E(t,k)-\phi(t)|\le  8S^2\left(1+\ln S\right)
e^{-(1/S)\int\limits_0^t\lambda (\tau)\,d\tau}.
%\label{054}
\end{equation*}



\smallskip

Используя результаты теоремы~1 и следствия~1, а также общий подход,
описанный в работах~\cite{z14a,z14b,z14d}, можно получить
соответст\-ву\-ющие оценки устойчивости процесса, описывающего чис\-ло
требований в системе. Ограничимся здесь одной из возможных
формулировок. Пусть $\bar{X}\hm=\bar{X}(t)$~--- число требований для
<<возмущенной>> системы обслуживания. Соответствующие его
характеристики будем обозначать теми же буквами с чертой сверху.
Предположим, что его инфинитезимальная матрица имеет произвольную
структуру (т.\,е.\ интенсивности поступления и обслуживания
требований произвольны) и при этом для всех $t \hm\ge 0$ выполнено
условие малости $\| \A(t)\hm-\bar{\A}(t)\| \hm\le \varepsilon$. Следующее
утверждение вытекает из теоремы~1 и оценок~(32), (33) из~\cite{z14a}.

\smallskip

\noindent
\textbf{Следствие~2.} Пусть выполнено~(\ref{030}) и вдобавок
при некоторых положительных $M,\,\alpha$ и всех $0 \hm\le s \hm\le t$
справедливо неравенство:
\begin{equation}
e^{-\int\limits_s^t{\mu(u)\,du}} \le M e^{-\alpha (t-s)}\,.
\label{061}
\end{equation}
Тогда при любых начальных условиях $\vp(0)$ и $\bar{\vp}(0)$
соответственно справедливы неравенства:
\begin{align*}
%\label{062}
\limsup_{t \to \infty}  \|{\bf p}(t)- \bar{\bf p}(t)\| &\le
\frac{\varepsilon(1+\ln{4SM})}{\alpha}\,;
\\
%\label{063}
\limsup_{t \to \infty}   |E_{\bf p}(t)- \bar{E}_{\bar{\bf p}(t)}|&\le
\fr{\varepsilon S(1+\ln{4SM})}{\alpha}\,.
\end{align*}

\smallskip

Отметим, что условие~(\ref{061}) заведомо выполнено, если
интенсивность обслуживания 1-пе\-рио\-дична, при этом

\noindent
\begin{align*}
\alpha &= \int\limits_0^1 \mu(t)\, dt\,,\\
M &= e^{\max_{|t-s|\le 1}\int\limits_s^t\mu(u)\,du}\,.
\end{align*}

\section{Примеры}

Рассмотрим теперь несколько моделей описываемой системы обслуживания
с периодическими инфинитезимальными характеристиками и для простоты
вычислений, чтобы не было необходимости рассматривать усеченные
процессы, как это делается обычно (см.~\cite{z14a, z06}), будем
полагать $S\hm=10^2$.

{\bf 1.} Пусть $N\hm=S$, а интенсивности поступления и обслуживания
требований определяются функциями $\lambda^*(t)\hm= 3\hm+\sin{2\pi t}$ и
$\mu(t)\hm=3\hm+\cos{2 \pi t}$ соответственно. Тогда выполнено условие~(\ref{030}),
существует предельный $1$-пе\-рио\-ди\-че\-ский режим, скажем
$\vp^*(t)$, и соответствующее предельное среднее $\phi^*(t)$ и
справедливы оценки скорости сходимости к ним:

\noindent
\begin{equation*}
\|\vp^*(t)-\vp^{**}(t)\| \le 10^3e^{-3t}\,, \
|E(t,s)-\phi(t)|\le 10^5e^{-3t},
%\label{101}
\end{equation*}
откуда вытекает, что при $t \hm\ge 8$ предельные характеристики
получаются с точностью выше $10^{-4}$. Их можно найти, решая прямую
систему Колмогорова с начальным условием ${\bf e_0}$ на отрезке
$[0;9]$ и взяв затем полученные функции на отрезке $[8;9]$. Отметим,
что  как в этом, так и в следующих примерах имеем $\alpha \hm= 3$,
$M \hm\le 2$ и соответствующие оценки устойчивости, основанные на
следствии~2, выглядят следующим образом:
%\pagebreak

\noindent
\begin{align*}
%\label{102}
\limsup\limits_{t \to \infty}  \|{\bf p}(t)- \bar{\bf p}(t)\| &\le
\fr{\varepsilon(1+\ln{800})}{3} \le 3\varepsilon\,;
\\
%\label{103}
\limsup_{t \to \infty}   |E_{\bf p}(t)- \bar{E}_{\bar{\bf p}(t)}|&\le
\fr{\varepsilon 10^2(1+\ln{800})}{3} \le 300\varepsilon\,.
\end{align*}

\begin{figure*}[b] %fig2
\vspace*{9pt}
\begin{center}
\mbox{%
\epsfxsize=161.688mm
\epsfbox{zei-2.eps}
}
\end{center}
\vspace*{-9pt}
%\center{\includegraphics[width=10cm]{phi50.eps}}
\Caption{Приближенные значения предельного среднего $\phi^*_{50}(t)$~(\textit{а}) и
предельной <<пустой очереди>> $p^*_{0,50}(t)$~(\textit{б})}
%\end{figure*}
%\begin{figure*} %fig3
\vspace*{18pt}
\begin{center}
\mbox{%
\epsfxsize=161.688mm
\epsfbox{zei-3.eps}
}
\end{center}
\vspace*{-9pt}
%\center{\includegraphics[width=10cm]{phi40.eps}}
\Caption{Приближенные значения предельного среднего $\phi^*_{40}(t)$~(\textit{а}) и
 предельной <<пустой очереди>> $p^*_{0,40}(t)$~(\textit{б})}
\end{figure*}



На рис.~1 приведены приближенные графики двух предельных
характеристик: предельного  среднего (математического ожидания)
$\phi^*_{100}(t)$ и предельной вероятности отсутствия требований в
сис\-те\-ме $p^*_{0,100}(t)\hm=\mathrm{Pr}\left(X(t)\hm=0\right)$ при $N\hm=S\hm=100$.

\smallskip


{\bf 2.} Пусть теперь $N\hm=S/2\hm=50$, интенсивности обслуживания
требований предполагаются такими же, а интенсивности поступления
требо\-ва\-ний подберем так, чтобы неизменной осталась <<средняя
интенсивность поступления требований>>, которая получается как
величина $\lambda_1(t)\hm+2\lambda_2(t)+ \dots + N\lambda_N(t)\hm=
N\lambda(t)$, т.\,е.\ должно\linebreak выполнять\-ся равенство $N\lambda(t)\hm =
S\left(3\hm+\sin{2\pi t}\right)$, откуда в этой ситуации
$\lambda(t)\hm=2\left(3\hm+\sin{2\pi t}\right)$. Поскольку полученные
оценки не зависят от функции, описывающей интенсивность поступления
требований, справедливы те же оценки. На рис.~2 приведены
приближенные графики двух предельных характеристик: предельного
среднего (математического ожидания) $\phi^*_{50}(t)$ и предельной
вероятности отсутствия требований в системе
$p^*_{0,50}(t)\hm=\mathrm{Pr}\left(X(t)=0\right)$ при $N\hm=S/2\hm=50$.

\smallskip

{\bf 3.} Возьмем теперь убывающие значения $N=40$, 20, 10 и~1 с
интенсивностями поступления требований
$\lambda(t)\hm=2,5\left(3\hm+\sin{2\pi t}\right)$,
$\lambda(t)\hm=5\left(3\hm+\sin{2\pi t}\right)$,
$\lambda(t)\hm=10\left(3\hm+\sin{2\pi t}\right)$ и
$\lambda(t)\hm=10^2\left(3\hm+\sin{2\pi t}\right)$  соответственно.
Поскольку полученные оценки не\linebreak зависят от функции, описывающей
интенсив-\linebreak ность поступления требований, справедливы те же оценки. На
рис.~3--6 для каждого из этих случаев приведены приближенные
графики двух \mbox{предельных} характеристик: предельного среднего
(математического ожидания) $\phi^*_{N}(t)$ и предельной вероятности
отсутствия требований в системе $p^*_{0,N}(t)\hm=\mathrm{Pr}\left(X(t)=0\right)$ при
$N=40$, 20, 10 и~1 соответственно.



\begin{figure*} %fig4
\vspace*{1pt}
\begin{center}
\mbox{%
\epsfxsize=161.688mm
\epsfbox{zei-4.eps}
}
\end{center}
\vspace*{-9pt}
%\center{\includegraphics[width=10cm]{phi20.eps}}
\Caption{Приближенные значения предельного среднего $\phi^*_{20}(t)$ ~(\textit{а}) и
предельной <<пустой очереди>> $p^*_{0,20}(t)$~(\textit{б})}
\end{figure*}


\begin{figure*} %fig5
\vspace*{1pt}
\begin{center}
\mbox{%
\epsfxsize=161mm
\epsfbox{zei-5.eps}
}
\end{center}
\vspace*{-9pt}
%\center{\includegraphics[width=10cm]{phi10.eps}}
\Caption{Приближенные значения предельного среднего $\phi^*_{10}(t)$~(\textit{а}) и
 предельной <<пустой очереди>> $p^*_{0,10}(t)$~(\textit{б})}
\end{figure*}


\begin{figure*} %fig6
\vspace*{1pt}
\begin{center}
\mbox{%
\epsfxsize=161.688mm
\epsfbox{zei-6.eps}
}
\end{center}
\vspace*{-9pt}
%\center{\includegraphics[width=10cm]{phi1.eps}}
\Caption{Приближенные значения предельного среднего $\phi^*_{1}(t)$~(\textit{а}) и
 предельной <<пустой очереди>> $p^*_{0,1}(t)$~(\textit{б})}
\end{figure*}


\medskip

\noindent
\textbf{Замечание.} Полученные результаты показывают, что
увеличение одномоментно допустимого поступающего количества
требований при той же <<средней интенсивности поступления
требований>> приводит к сокращению предельной средней длины очереди
и увеличению вероятности отсутствия требований в системе.


{\small\frenchspacing
 {%\baselineskip=10.8pt
 \addcontentsline{toc}{section}{References}

 \begin{thebibliography}{99}
\bibitem{E17} %1
\Au{Erlang A.\,K.} L{\hspace*{-1mm}\ptb{\o}}\hspace*{1pt}sning af nogle Problemer fra Sandsynlighedsregningen af Betydning for de automatiske
Telefoncentraler~// Elektroteknikeren, 1917. Vol.~13. P.~5--13.

\bibitem{gm} %2
\Au{Гнеденко Б.\,В., Макаров~И.\,П.}
Свойства решений задачи с потерями в случае периодических интенсивностей~//
Дифф. уравнения, 1971.  Т.~7. С.~1696--1698.

\bibitem{z89} %3
\Au{Зейфман А.\,И.}
Некоторые свойства системы с потерями в случае переменных интенсивностей~//
Автоматика и телемеханика, 1989. №\,1. С.~107--113.

\bibitem{ki90}  %4
\Au{Kijima M.}
On the largest negative eigenvalue of the infinitesimal generator associated with $M/M/n/n$
queues~// Oper. Res. Lett., 1990. Vol.~9. P.~59--64.

\bibitem{m94}   %5
\Au{Massey W.\,A., Whitt~W.}
On analysis of the modified offered-load approximation for the nonstationary Erlang
loss model~// Ann. Appl. Probab., 1994. Vol.~4. P.~1145--1160.

\bibitem{FRT} %6
\Au{ Fricker C., Robert P., Tibi~D.}
On the rate of convergence of Erlang's model~//
J.~Appl. Probab., 1999. Vol.~36. P.~1167--1184.

%\bibitem{z11c} Satin, Ya. A., Zeifman, A. I., Korotysheva, A. V., Shorgin, S. Ya.: On a class of Markovian queues. Informatics and its applications. 5. No. 4, 6--12 (2011, in Russian)

\bibitem{voit}   %7
\Au{Voit M.}  A~note of the rate of convergence to equilibrium
 for Erlang's model in the subcritical case~//
 J.~Appl. Probab., 2000. Vol.~37. P.~918--923.

 \bibitem{gz04} %8
\Au{Granovsky B., Zeifman~A.} Nonstationary queues:
Estimation of the rate of convergence~// Queueing Syst., 2004.
Vol.~46. P.~363--388.

\bibitem{dz} %9
\Au{Van Doorn E.\,A., Zeifman~A.\,I.}
On the speed of convergence to stationarity of the Erlang loss system~//
Queueing Syst., 2009. Vol.~63. P.~241--252.

%\bibitem{z85}  Zeifman, A. I.: Stability for contionuous-time nonhomogeneous Markov chains. Lect. Notes Math. 1155, 401--414 (1985)
%\bibitem{zbs} Zeifman, A. I., Bening, V. E.,  Sokolov, I. A.: Markov Chains and Models in Continuous Time. Elex-KM, Moscow (2008) (In Russian)

\bibitem{zAT} %10
\Au{Зейфман А.\,И.} О~нестационарной модели Эрланга~//
Автоматика и телемеханика, 2009. №\,12. С.~71--80.

\bibitem{z13a} %11
\Au{Zeifman A.\,I., Korotysheva~A., Satin~Ya., Shilova~G.,
Panfilova~T.} On a queueing model with group services~//
Modern probabilistic methods for
analysis of telecommunication network~/
Eds. A.~Dudin, V.~Klimenok, G.~Tsarenkov, and S.~Dudin.
Communications in computer and information science ser.
Vol.~356. P.~198--205.


\bibitem{z14c} %12
\Au{Zeifman A.\,I., Korotysheva~A., Shilova~G., Korolev~V.}
On perturbation bounds for a queueing model with group services. Preprint, 2014.

\bibitem{z14a} %13
\Au{Zeifman A.\,I., Korotysheva~A., Satin~Ya., Korolev~V., Bening~V.}
Perturbation bounds and truncations for a class of Markovian queues~//
Queueing Syst., 2014. Vol.~76. P.~205--221.

\bibitem{z14b} %14
\Au{Zeifman A.\,I., Korolev~V.\,Y.}
On perturbation bounds for continuous-time Markov chains~//
Stat. Probab. Lett., 2014. Vol.~88. P.~66--72.

\bibitem{z94}  %15
\Au{Zeifman A.\,I., Isaacson~D.}
On strong ergodicity for nonhomogeneous continuous-time Markov chains~//
Stoch. Proc. Appl., 1994. Vol.~50. P.~263--273.

\bibitem{z95} %16
\Au{Zeifman A.\,I.}
Upper and lower bounds on the rate of convergence for nonhomogeneous birth and death processes~//
Stoch. Proc. Appl., 1995. Vol.~59. P.~157--173.

\bibitem{z06} %17
\Au{Zeifman A., Leorato~S., Orsingher~E., Satin~Ya., Shilova~G.}
Some universal limits for nonhomogeneous birth and death
processes~// Queueing Syst., 2006.  Vol.~52. P.~139--151.

\bibitem{z14d} %18
\Au{Зейфман А.\,И., Королев~В.\,Ю., Коротышева~А.\,В., Шоргин~С.\,Я.}
Общие оценки устойчивости для нестационарных марковских цепей с непрерывным временем~//
Информатика и её применения, 2014. Т.~8. Вып.~1. С.~106--117.

 \end{thebibliography}

 }
 }

\end{multicols}

\vspace*{-9pt}

\hfill{\small\textit{Поступила в редакцию 12.08.14}}

%\newpage

\vspace*{8pt}

\hrule

\vspace*{2pt}

\hrule

%\vspace*{12pt}

\def\tit{ON THE BOUNDS OF THE RATE OF~CONVERGENCE AND~STABILITY FOR~SOME QUEUEING MODELS}

\def\titkol{On the bounds of the rate of convergence and stability for some queueing models}

\def\aut{A.\,I.~Zeifman$^{1,2,3}$, A.\,V.~Korotysheva$^2$, K.\,M.~Kiseleva$^2$,
V.\,Yu.~Korolev$^{1,4}$,
and~S.\,Ya.~Shorgin$^1$}

\def\autkol{A.\,I.~Zeifman, A.\,V.~Korotysheva, K.\,M.~Kiseleva et al.}

\titel{\tit}{\aut}{\autkol}{\titkol}

\vspace*{-12pt}

\noindent
$^1$Institute of Informatics Problems, Russian Academy of Sciences,
44-2 Vavilov Str., Moscow 119333, Russian\\
$\hphantom{^1}$Federation

\noindent
$^2$Vologda State University, 15~Lenin Str., Vologda 160000, Russian Federation

\noindent
$^3$Institute of Socio-Economic Development of Territories,
Russian Academy of Sciences,
56A~Gorkogo Str.,\linebreak
$\hphantom{^1}$Vologda 160014, Russian Federation


\noindent
$^4$Department of Mathematical Statistics, Faculty of Computational
 Mathematics and Cybernetics,\linebreak
$\hphantom{^1}$M.\,V.~Lomonosov Moscow State University,
1-52 Leninskiye Gory, GSP-1, Moscow 119991, Russian\linebreak
$\hphantom{^1}$Federation

\def\leftfootline{\small{\textbf{\thepage}
\hfill INFORMATIKA I EE PRIMENENIYA~--- INFORMATICS AND
APPLICATIONS\ \ \ 2014\ \ \ volume~8\ \ \ issue\ 3}
}%
 \def\rightfootline{\small{INFORMATIKA I EE PRIMENENIYA~---
INFORMATICS AND APPLICATIONS\ \ \ 2014\ \ \ volume~8\ \ \ issue\ 3
\hfill \textbf{\thepage}}}

\vspace*{3pt}

\Abste{A~generalization of the famous Erlang loss system has been considered,
namely, a class of Markovian queueing systems with possible simultaneous
arrivals and group services has been studied. Necessary and sufficient
conditions of
weak ergodicity have been obtained for the respective queue-length process
and explicit
bounds on\linebreak}

\Abstend{the rate of convergence and stability have been found.
The research is based on
the general approach developed in the authors' previous studies for nonhomogeneous
Markov systems with batch arrival and service requirements. Also,
specific models with periodic intensities and different maximum
size of number of  arrival customers are discussed. The main limiting characteristics
of these models have been computed and
the effect of the maximum size of the group of arrival
customers on the limiting characteristics of the queue has been studied.}

\KWE{nonstationary Markovian queue; Erlang model; batch arrivals and group services;
ergodicity; stability; bounds}


\DOI{10.14357/19922264140303}

\Ack
\noindent
This work was financially  supported by the Russian Science Foundation
(grant No.\,14-11-00397).

%\vspace*{3pt}

  \begin{multicols}{2}

\renewcommand{\bibname}{\protect\rmfamily References}
%\renewcommand{\bibname}{\large\protect\rm References}



{\small\frenchspacing
 {%\baselineskip=10.8pt
 \addcontentsline{toc}{section}{References}
 \begin{thebibliography}{99}



\bibitem{2zz} %1
\Aue{Erlang, A.\,K.} 1917. \mbox{L{\hspace*{-0.9mm}\ptb{\o}}sning} af nogle
Problemer fra Sandsynlighedsregningen af Betydning for de automatiske
Telefoncentraler. \textit{Elektroteknikeren} 13:5--13.



\bibitem{4zz} %2
\Aue{Gnedenko, B.\,V., and I.\,P.~Makarov}. 1971. Svoystva re\-she\-niy
zadachi s poteryami v sluchae periodicheskikh intensivnostey
[Properties of a problem with losses in the case of periodic intensities].
\textit{Diff. Uravneniya} [Differential Equations] 7:1696--1698.

\bibitem{9zz} %3
\Aue{Zeifman, A.\,I.} 1989. Properties of a system with losses in the
case of variable rates. \textit{Autom. Rem. Contr.} 50:82--87.

\bibitem{6zz} %4
\Aue{Kijima, M.} 1990. On the largest negative eigenvalue of the infinitesimal
generator associated with $M/M/n/n$ queues. \textit{Oper. Res. Lett.}
9:59--64. doi: 10.1016/0167-6377(90)90041-3.
\bibitem{7zz} %5
\Aue{Massey, W.\,A., and W.~Whitt}. 1994.
On analysis of the modified offered-load approximation for the nonstationary
Erlang loss model. \textit{Ann. Appl. Probab.} 4:1145--1160.
doi:10.1214/aoap/1177004908.

 \bibitem{3zz} %6
\Aue{Fricker, C., P.~Robert, and  D.~Tibi}. 1999. On the rate of
convergence of Erlang's model. \textit{J.~Appl. Probab.} 36:1167--1184.
doi: 10.1239/jap/1032374763.


\bibitem{8zz} %7
\Aue{Voit, M.} 2000. A~note of the rate of convergence to equilibrium for
Erlang's model in the subcritical case. \textit{J.~Appl. Probab.} 37:918--923.
doi: 10.1239/jap/1014842847.

\bibitem{5zz} %8
\Aue{Granovsky, B.\,L., and A.\,I.~Zeifman}. 2004.
Nonstationary queues: Estimation of the rate of convergence.
\textit{Queueing Syst.} 46:363--388.
doi: 46: 363-388.0.1023/ B:QUES.0000027991.19758.b4.

\bibitem{1zz} %9
\Aue{Van Doorn, E.\,A., and A.\,I.~Zeifman}.
2009. On the speed of convergence to stationarity of the Erlang loss system.
\textit{Queueing Syst.} 63:241--252. doi: 10.1007/s11134-009-9134-9.


\bibitem{13zz} %10
\Aue{Zeifman, A.\,I.} 2009. On the nonstationary Erlang loss model.
\textit{Autom. Rem. Contr.} 70:2003--2012. doi: 10.1134/S000511790912008X.
\bibitem{14zz} %11
\Aue{Zeifman, A., A.~Korotysheva, Ya.~Satin, G.~Shilova, and T.~Panfilova}.
2013. On a queueing model with group services.
\textit{Modern probabilistic methods for
analysis of telecommunication network}.
Eds. A.~Dudin, V.~Klimenok, G.~Tsarenkov, and S.~Dudin.
{Communications in computer and information science ser.}
356:198--205. doi: 10.1007/978-3-642-35980-4\_22.

\bibitem{17zz} %12
\Aue{Zeifman, A., A.~Korotysheva, G.~Shilova, and V.~Korolev}.
2014. On perturbation bounds for a queueing model with group services. Preprint.

\bibitem{15zz} %13
\Aue{Zeifman, A., A.~Korotysheva, Y.~Satin, V.~Korolev, and V.~Bening}.
2014. Perturbation bounds and truncations for a class of Markovian queues.
\textit{Queueing Syst.} 76:205--221.
doi: 10.1007/s11134-013-9388-0.
\bibitem{16zz} %14
\Aue{Zeifman, A.\,I., and V.\,Yu.~Korolev}. 2014. On perturbation
bounds for continuous-time Markov chains.
\textit{Stat. Probab. Lett.} 88:66--72. doi: 10.1016/j.spl.2014.01.031.

\bibitem{10zz} %15
\Aue{Zeifman, A.\,I., and D.~Isaacson}. 1994. On strong ergodicity for
nonhomogeneous continuous-time Markov chains. \textit{Stoch. Proc. Appl.} 50:263--273.
doi: 10.1016/0304-4149(94)90123-6.
\bibitem{11zz} %16
\Aue{Zeifman, A.\,I.} 1995. Upper and lower bounds on the rate
of convergence for nonhomogeneous birth and death processes.
\textit{Stoch. Proc. Appl.} 59:157--173.
doi: 10.1016/0304-4149(95)00028-6.
\bibitem{12zz} %17
\Aue{Zeifman, A.\,I., S.~Leorato, E.~Orsingher, Y.~Satin, and G.~Shilova}. 2006.
Some universal limits for nonhomogeneous birth and death processes.
\textit{Queueing Syst.} 52:139--151.
doi: 10.1007/s11134-006-4353-9.


\bibitem{18zz} %18
\Aue{Zeifman, A.\,I., V.\,Yu.~Korolev, A.\,V.~Korotysheva, and S.\,Ya.~Shorgin}.
2014. Obshchie otsenki ustoychivosti dlya nestatsionarnykh markovskikh tsepey s
nepreryvnym vremenem [General bounds for nonstationary continuous-time Markov].
\textit{Informatika i ee Primeneniya}~--- \textit{Inform. Appl.} 8(1):106--117.
doi: 10.14357/ 19922264140111.

\end{thebibliography}

 }
 }

\end{multicols}

\vspace*{-6pt}

\hfill{\small\textit{Received August 12, 2014}}

\vspace*{-18pt}

\Contr

\noindent
\textbf{Zeifman Alexander I.}\ (b.\ 1954)~--- Doctor of Science in
physics and mathematics,
senior scientist, Institute of Informatics Problems,
Russian Academy of Sciences, 44-2 Vavilov Str., Moscow 119333,
Russian Federation;  professor, Head of Department,
Vologda State University, 15 Lenin Str., Vologda 160000, Russian Federation;
principal scientist, Institute of Socio-Economic Development of Territories,
Russian Academy of Sciences,
56A Gorkogo Str., Vologda 160014, Russian Federation; a\_zeifman@mail.ru


\vspace*{3pt}


\noindent
\textbf{Korotysheva Anna V.} (b.\ 1988)~--- senior lecturer, Vologda State  University,
15 Lenin Str., Vologda 160000, Russian Federation;
a\_korotysheva@mail.ru


\vspace*{3pt}


\noindent
\textbf{Kiseleva Ksenia M.} (b.\ 1992)~--- PhD student, Vologda State University,
15 Lenin Str., Vologda 160000, Russian Federation;  a\_zeifman@mail.ru

\vspace*{3pt}


\noindent
\textbf{Korolev Victor Yu.}\ (b.\ 1954)~---
Doctor of Science in physics and mathematics,  leading scientist,
Institute of Informatics Problems, Russian Academy of Sciences,
44-2 Vavilov Str.,
Moscow 119333, Russian Federation;
professor,
Department of Mathematical Statistics, Faculty of Computational Mathematics and Cybernetics,
M.\,V.~Lomonosov Moscow State University,
1-52 Leninskiye Gory, GSP-1, Moscow 119991, Russian Federation;
vkorolev@cs.msu.su

\vspace*{3pt}

\noindent
\textbf{Shorgin Sergey Ya.}\ (b.\ 1952)~--- Doctor of Science in physics and
mathematics, professor, Deputy Director, Institute of Informatics Problems,
Russian Academy of Sciences, 44-2 Vavilov Str.,
Moscow 119333, Russian Federation; SShorgin@ipiran.ru



\label{end\stat}

\renewcommand{\bibname}{\protect\rm Литература}