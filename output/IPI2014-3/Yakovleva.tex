\def\stat{yakovleva}

\def\tit{МЕТОДЫ МАТЕМАТИЧЕСКОЙ СТАТИСТИКИ КАК ИНСТРУМЕНТ
ДВУХПАРАМЕТРИЧЕСКОГО АНАЛИЗА МАГНИТНО-РЕЗОНАНСНОГО
ИЗОБРАЖЕНИЯ}

\def\titkol{Методы математической статистики как инструмент
%двухпараметрического
анализа магнитно-резонансного изображения}

\def\aut{Т.\,В. Яковлева$^1$, Н.\,С. Кульберг$^2$}

\def\autkol{Т.\,В. Яковлева, Н.\,С. Кульберг}

\titel{\tit}{\aut}{\autkol}{\titkol}

\renewcommand{\thefootnote}{\arabic{footnote}}
\footnotetext[1]{Вычислительный центр имени А.\,А. Дородницына Российской академии наук, yakovleva@ccas.ru}
\footnotetext[2]{Вычислительный центр имени А.\,А. Дородницына Российской академии наук,  kulberg@yandex.ru}

\Abst{Рассмотрены методы анализа магнитно-резонансного изображения,
основанные на решении так называемой двухпараметрической задачи. Разработанные
методы обеспечивают совместное вычисление обоих статистических параметров~---
математического ожидания анализируемой случайной величины и ее дисперсии, т.\,е.\
одновременный расчет как полезного сигнала, так и шума. В~рассматриваемых вариантах
решения задачи используются методы математической статистики: метод максимума
правдоподобия и варианты метода моментов. Существенное преимущество развитого
двухпараметрического подхода состоит в том, что он обеспечивает эффективное решение
нелинейных задач, к каковым относятся задачи шумоподавления в системах
маг\-нит\-но-ре\-зо\-нанс\-ной визуализации. Оценивание искомых параметров основано исключительно на
данных выборочных измерений и не ограничено какими-либо априорными
предположениями. В~работе проводится сопоставительный анализ вариантов
рассматриваемой методологии, представлены результаты компьютерного моделирования,
в ходе которого получены статистические характеристики смещения и разброса
оцениваемых параметров при решении задачи различными методами. Представленные
методы двухпараметрического анализа райсовского сигнала могут использоваться в
составе новых информационных технологий на этапе обработки стохастических величин.}

\KW{распределение Райса; метод максимума правдоподобия; метод моментов;
двух\-па\-ра\-мет\-ри\-че\-ский анализ; отношение сигнала к шуму}

\DOI{10.14357/19922264140309}


\vskip 12pt plus 9pt minus 6pt

\thispagestyle{headings}

\begin{multicols}{2}

\label{st\stat}

\section{Введение}

      Методы фильтрации данных, традиционно используемые в составе
информационных технологий при формировании, анализе и обработке изображений в
системах маг\-нит\-но-ре\-зо\-нансной визуализации, можно объединить в следующие
основные группы:
      \begin{itemize}
\item  методы анизотропной фильтрации магнитно-ре\-зо\-нансных изображений, основанные
на процессе диффузии~[1, 2];
      \item методы фильтрации шума в магнитно-ре\-зо\-нансных изображениях,
основанные на вейв\-лет-пре\-об\-ра\-зо\-ва\-ни\-ях~[3, 4], а также на так называемых
преобразованиях ridgelet и curvelet~[5, 6], которые объединены единой концепцией
разложения функций по определенным базисным функциям;
      \item методы фильтрации, основанные на принципах математической статистики,
главным \mbox{образом}~--- на методе максимального правдоподобия~[7--13]. Эти методы, как
правило, направле\-ны на оценивание параметра средней величины или математического
ожидания анализируемого сигнала маг\-нит\-но-ре\-зо\-нансно\-го изоб\-ра\-же\-ния. При
решении задач анализа магнитно-ре\-зо\-нансно\-го изображения традиционно
используются так на\-зы\-ва\-емые од\-но\-па\-ра\-мет\-ри\-че\-ские методы, основанные на
предположении, что один из статистических параметров задачи~--- дисперсия шума~---
является известным априори. При этом задача решается в рамках так на\-зы\-ва\-емой
однопараметрической модели, состоящей в предположении о том, что неизвестным
является лишь один параметр задачи~--- параметр средней величины сигнала, в~то время
как второй значимый статистический параметр~--- дисперсия шума~--- предполагается
априорно известной величиной. Эта однопараметрическая модель, используемая многими
авторами при решении задачи, никогда не реализуется на практике, и поэтому ее
применение является серьезным ограничением традиционно используемого
однопараметрического подхода к обработке райсовских сигналов~[8--11, 13]. Методы,
рассматриваемые в настоящей работе, основаны на использовании так на\-зы\-ва\-емой
двухпараметрической модели для решения задачи, т.\,е.\ модели, предполагающей
наличие двух неизвестных параметров анализируемого райсовского сигнала: как его
средней величины, так и дисперсии шума.
\end{itemize}

      Задачи анализа и обработки маг\-нит\-но-ре\-зо\-нансных изображений, как
известно, относятся к кругу задач, в которых выходной сигнал пред\-став\-ля\-ет собой сумму
искомого исходного сигнала и случайного шума, образованного многими независимыми
нор\-маль\-но-рас\-пре\-де\-лен\-ны\-ми сла\-га\-емыми с нулевым средним значением, а
измеряемой и анализируемой величиной является амплитуда суммарного сигнала.
Амплитуда, или огибающая, такого суммарного сигнала подчиняется распределению
Райса~\cite{14-ya}. Таким образом, задачи анализа и обработки
      маг\-нит\-но-ре\-зо\-нансно\-го изображения адекватно описываются
статистической моделью Райса~\cite{15-ya}.

      Двухпараметрические методы анализа рай\-совской случайной величины,
представленные в настоящей работе, обладают следующими существенными
преимуществами по сравнению с упомянутыми выше традиционными методами:
      \begin{itemize}
      \item возможность применения к существенно нелинейным задачам, каковыми
являются задачи шумоподавления в условиях райсовского распределения, что отличает
эту технологию от первых двух упомянутых выше групп методов;
      \item отсутствие ограничений, характерных для однопараметрических методов,
представленных третьей группой и связанных с априорными предположениями
относительно величины \mbox{шума}.
     \end{itemize}

     Задачу анализа магнит\-но-ре\-зо\-нансно\-го изображения можно обобщенно
отнести к кругу задач анализа состояний стохастической системы, когда начальное
состояние системы характеризуется неким исходным <<незашумленным>> значением
анализируемой величины, т.\,е.\ значением величины сигнала в отсутствие шума, а
измеряются и анализируются данные, полученные в результате <<зашумления>>
исходного сигнала гауссовским шумом. Ставшие предметом настоящей работы новые
методы анализа и обработки данных на основе двухпараметрического анализа райсовской
случайной величины можно рассматривать как варианты методологии, которая
потенциально может использоваться в составе новых информационных технологий для
обработки информации, характеризующей состояния вышеуказанной стохастической
системы. Эта сис\-те\-ма изучается на основе измерения данных в <<зашумленном>>
состоянии и восстановления ее исходного, не искаженного шумом значения на основе
этих данных путем решения двухпараметрической задачи.

     В работах~\cite{15-ya, 16-ya} рассматривается решение двухпараметрической
задачи методом максимума правдоподобия, а в работах~\cite{17-ya, 18-ya}~---
вариантами метода моментов. Задача, поставленная и решаемая в настоящей работе,
состоит в проведении со\-по\-став\-ле\-ния этих трех способов двухпараметрического анализа
данных. В~разд.~2 кратко изложены теоретические основы трех сопоставляемых
методов, а в разд.~3 представлены результаты их сравнительного анализа, а именно:
приведены данные, полученные в ходе численного моделирования срав\-ни\-ва\-емых
методов, причем для выявления обес\-пе\-чи\-ва\-емых ими статистических характеристик
смещения и разброса рассчитанных искомых параметров задачи при моделировании
численного эксперимента проводится усреднение по очень большому числу (порядка
10$^5$) выборок измерений. Рассмотрены преимущества и ограничения указанных
вариантов решения двухпараметрической задачи с точки зрения обеспечиваемой ими
точности вычисления искомых параметров анализируемого сигнала. Тео\-рия
представленных в работе методов двух\-па\-ра\-мет\-ри\-че\-ско\-го анализа райсовских данных
развита в работах~[15--18]. В~результате
проведенного теоретического исследования разработаны алгоритмы расчета
неизвестных параметров, соответствующие трем рассмотренным методам
математической статистики. Данные алгоритмы легли в основу созданных
Н.\,С.~Кульбергом программ расчета искомых параметров задачи с целью
сопоставления пред\-став\-лен\-ных в работе методов~\cite{16-ya, 17-ya}.

\section{Теоретические основы двухпараметрического анализа~данных }

     В данном разделе кратко изложены базовые положения теории
двухпараметрического анализа райсовского сигнала на основе методов математической
статистики, при этом опущены объемные выкладки и доказательства, представленные в
работах~[15--18].

\subsection{Метод максимума правдоподобия}

    Ниже для краткости будем обозначать метод максимума правдоподобия как МП.

     Как известно, при анализе данных маг\-нит\-но-ре\-зо\-нанс\-ного изображения измеряемой
величиной является амплитуда $x=\sqrt{x^2_{\mathrm{Re}}+x^2_{\mathrm{Im}}}$
комплексной величины, действительная $x_{\mathrm{Re}}$ и мнимая $x_{\mathrm{Im}}$
час\-ти которой, характеризуемые величиной математического ожидания~$\nu$,
искажаются нормально распределенным гауссовским шумом с дисперсией~$\sigma^2$.
При этом амплитуда~$x$ имеет распределение Райса с плотностью вероятности
     \begin{equation*}
     P\left( x\vert \nu, \sigma^2\right) = \fr{x}{\sigma^2}\exp \left( -
\fr{x^2+\nu^2}{2\sigma^2}\right) I_0 \left( \fr{x\nu}{\sigma^2}\right)\,,
%     \label{e1-ya}
     \end{equation*}
где $I_\alpha(z)$~--- модифицированная функция Бесселя первого рода порядка~$\alpha$;
$x_i$~--- величина сигнала \mbox{$i$-й} выборки;
$n$~--- количество элементов в выборке (длина
выборки). Поставленная задача состоит в том, чтобы определить неизвестные
параметры~$\nu$ и $\sigma^2$ на основе данных выборок измерений. Используя
логарифмическую функцию правдоподобия и дифференцируя ее по~$\nu$ и~$\sigma^2$ с
целью выявления ее максимума, после ряда преобразований получаем систему уравнений
максимума правдоподобия~\cite{15-ya, 16-ya}:
\begin{equation}
\left.
\begin{array}{rl}
\nu &= \fr{1}{n} \sum\limits_{i=1}^n x_i \tilde{I} \left( \fr{2x_i\nu}{
\left\langle x^2\right\rangle -
\nu^2}\right)\,;\\[9pt]
\sigma^2 &= \fr{1}{2}\left( \left\langle x^2\right\rangle -\nu^2\right)\,,
\end{array}
\right\}
\label{e2-ya}
\end{equation}
где
$$
\tilde{I}(z)=\fr{I_1(z)}{I_0(z)}\,; \quad
\langle x^2\rangle \hm = \fr{1}{n} \sum\limits_{i=1}^n x_i^2\,.
$$

В~\cite{15-ya} показано, что функция $\tilde{I}(t)$ представляет собой ограниченную
монотонно возрастающую вы\-пук\-лую вверх функцию на полуоси $(0,+\infty)$. Первое из
уравнений системы~(\ref{e2-ya})~--- уравнение для одной неизвестной~$\nu$, решив
которое, определяем вторую неизвестную на основании второго уравнения
из~(\ref{e2-ya}). Таким образом, решение системы двух уравнений для двух неизвестных
сведено к решению одного уравнения для одной неизвестной, что существенно облегчает
численное решение задачи. Вводя переменную $g(\nu) = 2\langle x \rangle \nu / (\langle
x^2\rangle -\nu^2)$, явля\-ющу\-юся однозначной функцией~$\nu$, первое из уравнений
системы~(\ref{e2-ya}) можно представить в виде~\cite{16-ya}:
\begin{equation}
\xi(g) = \fr{1}{n}\sum\limits_{i=1}^n x_i \tilde{I} \left( \fr{x_i}{\langle x\rangle } \, g\right)\,,
\label{e3-ya}
\end{equation}
где левая часть представляет собой величину~$\nu$ как обратную функцию
аргумента~$g$: $\nu \hm= \xi(g(\nu))$. Доказано существование и единственность
нетривиального решения уравнения~(\ref{e3-ya}) и, следовательно, существование и
единственность решения системы~(\ref{e2-ya})~\cite{15-ya, 16-ya}. Тем самым доказано
существование и единственность нетривиального решения рассматриваемой
двухпараметрической задачи методом МП.

\subsection{Метод, основанный на измерениях второг и~четвертого
моментов анализируемой величины}

    Ниже для краткости будем обозначать этот метод как ММ24.

    Для 2-го и 4-го начальных моментов случайной величины~$x$, подчиняющейся
распределению Райса, известны следующие формулы~\cite{19-ya}:
    \begin{equation}
    \left.
    \begin{array}{rl}
    \overline{x^2} &= 2\sigma^2+\nu^2\,;\\[9pt]
    \overline{x^4} &= 8\sigma^4+ 8\sigma^2\nu +\nu^4\,.
    \end{array}
    \right\}
    \label{e4-ya}
    \end{equation}
	
    Принимая во внимание, что левые части уравнений~(\ref{e4-ya}), т.\,е.\ величины
моментов анализируемого сигнала, представляют собой результаты выборочных
измерений, формулы~(\ref{e4-ya}) можно рассматривать как простую систему двух
уравнений для двух неизвестных~$\nu$ и~$\sigma^2$. В~решении данной системы и
состоит метод ММ24.

    Из формул~(\ref{e4-ya}) для определения параметров~$\nu$ и~$\sigma^2$ нетрудно
получить:
    \begin{align}
    \nu &= \sqrt[4]{2\left( \overline{x^2}\right)^2-\overline{x^4}}\,;\label{e5-ya}\\
    \sigma^2 &= \fr{1}{2}\left[ \overline{x^2} -\sqrt{2\left( \overline{x^2}\right)^2 -
\overline{x^4}}\right]\,.\label{e6-ya}
    \end{align}

    Важно отметить, что содержащиеся в формулах~(\ref{e5-ya}) и~(\ref{e6-ya})
величины моментов измеряемого сигна\-ла изображения определяются в выборках
измере\-ний тем точнее, чем больше длина выборки. {С~помощью} несложных выкладок
(см.\ также~\cite{15-ya, 17-ya, 18-ya}), можно сделать вывод о неотрицательности
подкоренных выражений в формулах~(\ref{e5-ya}) и~(\ref{e6-ya}) при достаточно
большой длине выборки измерений анализируемого сигнала и, следовательно, о
существовании решения системы~(\ref{e4-ya}) для~$\nu$ и~$\sigma^2$ метода ММ24.
Единственность этого решения следует из неотрицательности значений искомых
па\-ра\-мет\-ров, в силу чего в решениях~(\ref{e5-ya}) и~(\ref{e6-ya}) остается лишь один из
двух возможных знаков перед корнем.

    Производя несложные преобразования формул~(\ref{e5-ya}) и~(\ref{e6-ya}),
выражения для квадратов искомых параметров~$\nu^2$ и~$\sigma^2$ можно представить
в виде:
    \begin{equation}
    \left.
    \begin{array}{rl}
    \nu^2 &= \overline{x^2}\sqrt{1-t}\,;\\[9pt]
    \sigma^2 &= \fr{\overline{x^2}}{2}\left( 1-\sqrt{1-t}\right)\,.
    \end{array}
    \right\}
    \label{e7-ya}
    \end{equation}
	
     В формулах~(\ref{e7-ya}) введено обозначение
$t\hm=\overline{x^4}/(\overline{x^2})^2 \hm-1$. Как нетрудно видеть, введенный
параметр~$t$ удовлетворяет соотношению: $0\hm<t\hm\leq 1$. В~частном случае
распределения Рэлея, когда полезный сигнал отсутствует ($\nu\hm=0$), имеем $t\hm=1$.
При этом второе из соотношений~(\ref{e7-ya}) дает очевидную для распределения Рэлея
формулу для дисперсии: $\sigma^2 \hm= \overline{x^2}/2$.

    Таким образом, вариант метода моментов ММ24 позволяет рассчитать величины
искомых па\-ра\-мет\-ров математического ожидания~$\nu$ и дисперсии~$\sigma^2$ сигнала
по формулам~(\ref{e5-ya})--(\ref{e7-ya}) на основе выборок измерений без затрат времени
на численное решение задачи. Этот оригинальный метод является простым в
практической реализации и позволяет существенно ускорить процесс обработки
анализируемого сигнала.

\subsection{Метод, основанный на измерениях первого и~второго моментов анализируемой
величины}

     Ниже для краткости будем обозначать этот метод как ММ12. Согласно данному
методу искомые параметры~$\nu$ и~$\sigma^2$ рассчитываются на основе использования
измеренных данных для 1-го и 2-го моментов райсовской величины, т.\,е.\ на основе
метода ММ12.

     Формула для 1-го момента $\overline{x}=\lim\limits_{n\to\infty}
\left((1/n)\sum\limits_{i=1}^n x_i\right)$ райсовской величины~$x$ имеет вид~\cite{19-ya}
     \begin{equation}
     \overline{x} =\sigma \sqrt{\fr{\pi}{2}}\, L_{1/2} \left( - \fr{\nu^2}{2\sigma^2}\right)\,,
     \label{e8-ya}
     \end{equation}
где $L_{1/2}$~--- полином Лагерра. Используя соотношение между $L_{1/2}$ и
модифицированными функциями Бесселя $I_0(z)$ и $I_1(z)$~\cite{20-ya},
вместо~(\ref{e8-ya}) получаем:
\begin{multline}
\overline{x} =\sigma \sqrt{\fr{\pi}{2}}\, e^{-\nu^2/(4\sigma^2)} \left[ \left(
1+\fr{\nu^2}{2\sigma^2 } \right) I_0 \left( \fr{\nu^2}{4\sigma^2} \right)
+{}\right.\\
\left.{}+\fr{\nu^2}{2\sigma^2}\, I_1 \left( \fr{\nu^2}{4\sigma^2}\right) \right]\,.
\label{e9-ya}
\end{multline}

    Формула~(\ref{e9-ya}) определяет аналитическую зависимость первого момента
случайной величины~$x$ от искомых параметров задачи~$\nu$ и~$\sigma^2$ и может
использоваться в качестве одного из исходных уравнений для определения~$\nu$
и~$\sigma^2$. В~качестве второго уравнения в методе ММ12 используется выражение для
второго момента (первая из формул~(\ref{e4-ya})).

    Вводя переменную $r=\nu^2/(2\sigma^2)$, характеризующую отношение сигнала к
шуму, и преобразуя уравнения для~$\nu$ и~$\sigma^2$, получаем следующие уравнения
для~$r$ и~$\sigma^2$~\cite{18-ya}:
    \begin{multline}
    \sqrt{\fr{\pi}{4}\,\fr{\left\langle x^2\right\rangle}{1+r}}\, e^{-r/2} \left[ (1+r) I_0 \left( \fr{r}{2}\right)
+rI_1\left( \fr{r}{2}\right) \right] ={}\\
{}=\langle x\rangle\,;\label{e10-ya}
\end{multline}
$$
\sigma^2 = \fr{\langle x^2\rangle}{2(1+r)}\,, %\label{e11-ya}
$$
в которых была произведена замена моментов~$\overline{x}$ и~$\overline{x^2}$ на
соответствующие средние значения по выборке измерений $\langle x\rangle \hm= (1/n)
\sum\limits_{i=1}^n x_i$ и $\langle x^2\rangle \hm= (1/n) \sum\limits_{i=1}^n x_i^2$ в
предположении, что длина выборки достаточно велика.

    Уравнение~(\ref{e10-ya}) представляет собой уравнение с одной неизвестной~$r$.
Вопросы, связанные с существованием и свойствами его решения, рас\-смот\-ре\-ны
в~\cite{18-ya}. Не останавливаясь на деталях вычислений, отметим следующий важный
результат: решение системы двух уравнений для двух неизвестных сведено к решению
одного уравнения для одной неизвестной, что обеспечивает существенное сокращение
объема вычислений.

    В результате несложных преобразований из~(\ref{e10-ya}) получаем уравнение:
    \begin{multline}
   \hspace*{-4.6pt}\sqrt{\fr{\pi}{4}\,\langle x^2\rangle}\,\sqrt{1+r}\, e^{-r/2} I_0 \left( \fr{r}{2}\right) \left[
1+\fr{r}{1+r}\,\tilde{I}\left( \fr{r}{2}\right)\right] ={}\\
{}=\langle x\rangle\,,
    \label{e12-ya}
    \end{multline}
где используется введенное выше обозначение  $\tilde{I} (z) \hm= I_1(z)/I_0(z)$.

    Правомерность использования метода ММ12 для расчета параметров~$\nu$
и~$\sigma^2$ основывается на доказательстве существования решения
уравнения~(\ref{e12-ya}) и, следовательно, существования решения исходной системы
уравнений для~$\nu$ и~$\sigma^2$. На основе найден\-но\-го решения для~$r$ рассчитываются
искомые~$\nu$ и~$\sigma^2$ анализируемого изображения:
    \begin{equation}
    \left.
    \begin{array}{rl}
    \nu &= \sqrt{\fr{r}{1+r}}\,\sqrt{\langle x^2\rangle}\,;\\[9pt]
    \sigma^2 &= \fr{\langle x^2\rangle}{2(1+r)}\,.
    \end{array}
    \right\}
    \label{e13-ya}
    \end{equation}

     Таким образом, определение искомых па\-ра\-мет\-ров анализируемого
     магнитно-ре\-зо\-нансного изоб\-ра\-же\-ния методом ММ12, как и в случае метода МП, сводится к
решению одного уравнения для одной неизвестной, что представляет собой одно из
важных преимуществ анализируемых двухпараметрических методов и обусловливает
возможное эффективное использование данных методов в составе новых
информационных технологий для целей маг\-нит\-но-ре\-зо\-нанс\-ной визуализации.

\section{Сопоставление результатов численного решения
двухпараметрической задачи различными методами }

    С целью сопоставления точности трех рас\-смотренных выше методов решения
двух\-па\-ра\-мет\-ри\-че\-ской задачи, которые могут использоваться в \mbox{составе} новых
информационных технологий маг\-нит\-но-ре\-зо\-нансной визуализации, были проведены
чис\-лен\-ные эксперименты, в которых по \mbox{одним} и тем же данным выборок измерений
вы\-чис\-ля\-лись искомые статистические параметры тремя алгоритмами, соответствующими
методам МП, ММ24 и ММ12. Проведенное ниже сопоставление результатов чис\-лен\-ных
экспериментов позволило выявить особенности каждого из рассматриваемых методов с
точки зрения обеспечиваемой точности расчета параметров~$\nu$ и~$\sigma^2$,
характеризующих величину исходного сигнала и дисперсию шума в каж\-дой
анализируемой точке изучаемого магнитно-ре\-зо\-нанс\-ного изображения.

    Численные эксперименты проводились следующим образом. Для каждого из
параметров~$\nu$ и~$\sigma$ задавались массивы исходных значений, и в каждой точке
такой двумерной сетки с помощью датчика случайных чисел, подчиняющихся
распределению Райса, генерировались данные выборок случайного\linebreak
 сигнала~$x$: $x_i$
$(i\hm= 1,\ldots ,n)$, причем параметры распределе\-ния Райса для данных выборки,
генерируемых в каждой точке, соответствовали исходно заданным значениям средней
величины~$\nu$ и стандартного отклонения~$\sigma$ сигнала изображения в \mbox{данной}
точке. Для каждой точки двумерной сетки в плоскости $(\nu, \sigma)$, используя
сгенерированные датчиком случайных чисел выборки значений сигнала~$x$,
вычислялись значения вышеуказанных параметров~$\nu$ и~$\sigma$ на основе трех
алгоритмов, соответствующих сопоставляемым методам решения задачи:
    \begin{enumerate}[(1)]
    \item для метода МП: решение первого из уравнений~(\ref{e2-ya}) для
параметра~$\nu$ и последующий расчет параметра~$\sigma$ по второму из
уравнений~(\ref{e2-ya});
    \item для метода ММ24: расчет параметров~$\nu$ и~$\sigma$ по
    формулам~(\ref{e7-ya}) на основе рассчитанной по измеренным выборочным данным
величины  $t\hm= \overline{x^4}/(\overline{x^2})^2-1$;
    \item для метода ММ12: решение уравнения~(\ref{e12-ya}) для переменной $r\hm=
\nu^2/(2\sigma^2)$ и последующий расчет параметров~$\nu$ и~$\sigma$ по
формулам~(\ref{e13-ya}).
    \end{enumerate}

    Результатом численного решения задачи каж\-дым из методов являются две
поверхности, образуемые рассчитанными значениями для каждого из параметров~$\nu$
и~$\sigma$, как функции исходно заданных значений этих параметров, соответствующих
узлам двумерной сетки. Графические данные, пред\-став\-лен\-ные на рис.~1 и~2, являются
сечениями этих поверхностей, при этом графики для~$\nu$ пред\-став\-ля\-ют собой сечения
поверхности, образованной расчетными значениями для~$\nu$, плоскостями $\sigma
\hm=const$, а~графики, отображающие результаты расчетов параметра~$\sigma$, являются
сечениями по\-верх\-ности, образованной расчетными значениями для~$\sigma$,
плоскостями $\nu\hm=const$.



    Представленные на рис.~1 и~2 графики соответствуют следующим значениям
параметров: исходно заданные значения~$\nu$ и~$\sigma$, определяющие узлы
двумерной сетки, на которой производились вычисления, изменялись в диапазоне от 0,2
до~3,0 с шагом~0,2; длина $n$ выборки составляла 64; усреднение результатов
проводилось по 50~выборкам.

    На рис.~1 представлены графики, отобра\-жа\-ющие расчетные значения
параметра~$\nu$ в сечениях $\sigma\hm=const$.
На рис.~1,\,\textit{а}\,--\,1,\,\textit{в}
показаны зависимости отклонений расчетных значений~$\nu$  от реальной, исходно
заданной величины этого параметра (сплошные черные прямые)
при различных исходных значениях
параметра~$\sigma$. Значения по оси абсцисс на\linebreak
графиках соответствуют точкам отсчета
исходно \mbox{заданных} значений параметра~$\nu$, а~по оси ординат~--- расчетным значениям
данного параметра. %\linebreak
На графике
по оси абсцисс отложены цифры, соответствующие точкам отсчета для исходно заданного
параметра~$\nu$. Отклонения ломаной кривой от прямой на графиках характеризуют точность
расчетов при следующих значениях параметра стандартного отклонения~$\sigma$:
1,6 (см.\ рис.~1,\,\textit{а}), 0,8 (см.\ рис.~1,\,\textit{б})
и 0,4 (см.\ рис.~1,\,\textit{в}).

    Из проведенного численного эксперимента, графические результаты которого
представлены на рис.~1, можно сделать следующие выводы:
    \begin{itemize} %[1)]
    \item  точность вычисления искомого параметра~$\nu$ при решении
двухпараметрической задачи анализа маг\-нит\-но-ре\-зо\-нансно\-го изображения всеми
тремя рассматриваемыми методами заметно повышается с увеличением величины
отношения сигнала к шуму, т.\,е.\ при перемещении слева направо вдоль расчетных
кривых, а также по мере уменьшения значений <<шумового>> параметра~$\sigma$;
\end{itemize}


\end{multicols}

\begin{figure} %fig1
\vspace*{1pt}
\begin{minipage}[t]{79.5mm}
\begin{center}
\mbox{%
\epsfxsize=78.5mm
\epsfbox{yak-1.eps}
}
\end{center}
\vspace*{-11pt}
\Caption{Результаты численного расчета параметра~$\nu$ при решении двухпараметрической задачи
методами МП~(\textit{1}), ММ24~(\textit{2}) и ММ12~(\textit{3}):
(\textit{а})~$\sigma\hm= 1{,}6$; (\textit{б})~0{,}8; (\textit{в})~$\sigma\hm=0{,}4$}
%\end{figure*}
\end{minipage}
\hfill
%\begin{figure*}  %fig2
\vspace*{1pt}
\begin{minipage}[t]{79.5mm}
\begin{center}
\mbox{%
\epsfxsize=78.5mm
\epsfbox{yak-2.eps}
}
\end{center}
\vspace*{-11pt}
\Caption{Результаты численного расчета параметра $\sigma$ при решении
двухпараметрической задачи методами МП (\textit{1}), ММ24~(\textit{2}) и
ММ12~(\textit{3}): (\textit{а})~$\nu=1{,}0$; (\textit{б})~$\nu\hm= 2{,}0$;
(\textit{в})~$\nu\hm=3{,}0$ }
\end{minipage}
\vspace*{-7pt}
\end{figure}


\begin{multicols}{2}

\noindent
\begin{itemize}
    \item оба варианта метода моментов ММ24 и ММ12, будучи примерно одинаковыми
по обеспечиваемой точности вычислений, заметно превосходят метод МП;\\[-15pt]
    \item все три рассматриваемых метода характеризуются существенным повышением
точности расчетов параметра~$\nu$ и обеспечением поддержания ее на высоком уровне
при достижении величиной $\nu/\sigma$ некоторого граничного значения:
для метода МП
это граничное значение близко к~2, т.\,е.\ удовлетворительная точность для
параметра~$\nu$ достигается при $\nu\hm>2\sigma$. Что касается обоих вариантов метода
моментов, то в данном случае аналогичное граничное значение для отношения
$\nu/\sigma$, начиная с которого обеспечивается существенное повышение точности
расчетов параметра~$\nu$, составляет величину $\nu/\sigma\sim 1{,}5$--2. Другими
словами, оба варианта метода моментов обеспечивают высокую точность расчетов в
практически таком же диапазоне значений параметра~$\nu$, характеризующего полезную,
незашумленную составляющую данных, формирующих маг\-нит\-но-ре\-зо\-нансное
изображение, как и при использовании метода МП, в то время как вне этого диапазона
точность вариантов метода моментов выше, чем точность, обеспечиваемая методом~МП.
\end{itemize}

     На рис.~2 представлены зависимости рассчитанной каждым из методов величины
параметра стандартного отклонения~$\sigma$, характеризующего уровень шума на
изображении, показанные в сечениях $\nu\hm=const$. Графики на
     рис.~2,\,\textit{а}\,--\,2,\,\textit{в} иллюстрируют зависимости отклонений расчетных
значений~$\sigma$ от реальной величины этого параметра
(сплошные черные прямые) при различных~$\nu$. На графике по оси абсцисс отложены
цифры, соответствующие точкам отсчета для исходно заданного параметра~$\sigma$.
Отклонение ломаной
линии от прямой на графиках характеризует точность расчетов. Графики построены для
следующих значений~$\nu$: 1,0 (см.\ рис.~2,\,\textit{а}), 2,0
(см.\ рис.~2,\,\textit{б}) и 3,0 (см.\ рис.~2,\,\textit{в}).

    Анализируя графические результаты численного эксперимента, представленные на
рис.~2 и характеризующие особенности расчета параметра стандартного
отклонения~$\sigma$, можно сделать следующие основные выводы:
    \begin{itemize} %[1)]
\item точность вычисления параметра~$\sigma$, характеризующего уровень шума в
анализируемом маг\-нит\-но-ре\-зо\-нансном изображении, при решении
двухпараметрической задачи всеми тремя рассматриваемыми методами заметно падает с
уменьшением величины отношения сигнала к шуму (т.\,е.\ при перемещении слева
направо вдоль расчетных кривых), так как при этом увеличивается величина~$\sigma$,
являющаяся показателем стохастичности процесса. С~ростом параметра~$\nu$ отношение
сигнала к шуму увеличивается и точность расчетов параметра~$\sigma$ возрастает, что
демонстрируется сравнением графиков на рис.~2,\,\textit{а}\,--\,2,\,\textit{в},
соответствующих различным~$\nu$;
\item варианты метода моментов ММ24 и ММ12 характеризуются примерно одинаковой
точ\-ностью расчета~$\sigma$ при $\nu\hm=const$, но уступают методу МП;
\item все три рассматриваемых метода характеризуются существенным повышением
точности расчетов параметра~$\sigma$, характеризующего степень зашумленности
маг\-нит\-но-ре\-зо\-нансно\-го изображения, при достижении величиной $\nu/\sigma$
некоторого граничного значения. В~отличие от проанализированных выше результатов
расчета параметра~$\nu$ в сечениях $\sigma\hm =const$ (см.\ рис.~1), графики на рис.~2,
иллюстрирующие расчеты~$\sigma$ в сечениях $\nu \hm=const$, показывают, что
граничное значение отношения $\nu/\sigma$, обеспечивающее существенное улучшение
точности расчетов, примерно одинаково для всех трех методов и составляет величину
$\nu/\sigma \sim 1{,}5$. Другими словами, удовлетворительная точность для
расчета~$\sigma$ достигается при $\sigma \hm< 0{,}7\nu$ и этот диапазон высокой
точности расчетов параметра~$\sigma$ примерно одинаков для всех трех методов.
\end{itemize}

    Для выявления статистических характеристик развитых методов решения
двухпараметрической задачи были проведены численные эксперименты, в которых
усреднение полученных данных производилось по большому количеству ($\sim 10^5$)
выборок измерений, что позволило выявить и сопоставить характерные статистические
зависимости смещения и разброса расчетных данных для параметров~$\nu$ и~$\sigma^2$
от условий экспериментов для методов МП, ММ24 и ММ12. Расчеты проводились для
сле\-ду\-ющих значений длины выборки~$n$: 16 (рис.~3,\,\textit{а}
и~4,\,\textit{а}) и 64 (рис.~3,\,\textit{б} и~4,\,\textit{б}).




     На рис.~3 представлены графики, характеризующие точность расчетов
параметра~$\nu$ методами ММ12~(\textit{1}), ММ24~(\textit{2}) и
МП~(\textit{3}). По оси абсцисс отложена исходно заданная величина
параметра~$\nu$ в диапазоне от 0 до~3, в то время как величина параметра~$\sigma$ была
принята равной~1, так что точки оси абсцисс фактически соответствуют величине
отношения сигнала к шуму $\mbox{SNR}\hm=\nu/\sigma$. В~левом столбце рис.~3 приведены
графики рассчитанных значений~$\nu$, их отклонение от прямой линии характеризует
точность сопоставляемых методов. В~правом столбце приведены графики
среднеквадратичных отклонений (СКО) значений~$\nu$ в зависимости от~SNR.


     На рис.~4 приведены графики, иллюстрирующие точность расчетов
параметра~$\sigma$ методами ММ12~(\textit{1}), ММ24~(\textit{2})
и МП~(\textit{3}) при тех же условиях эксперимента, которые указаны в
комментариях к рис.~3. Горизонтальная линия соответствует исходно заданному
значению $\sigma\hm=1$. В~левом столбце рис.~4 приведены графики рассчитанных
значений~$\sigma$, а в правом столбце~--- графики СКО
значений~$\sigma$.



     Таким образом, данные в левых столбцах рис.~3 и~4 характеризуют смещения
величин искомых па-\linebreak\vspace*{-12pt}

\pagebreak

\end{multicols}

     \begin{figure} %fig3
     \vspace*{1pt}
\begin{center}
\mbox{%
\epsfxsize=147mm
\epsfbox{yak-3.eps}
}
\end{center}
\vspace*{-9pt}
     \Caption{Графики рассчитанных значений параметра~$\nu$ (левый столбец) и его
среднеквадратичного отклонения (правый столбец) в зависимости от отношения сигнала к
шуму при двухпараметрическом анализе райсовского сигнала методами ММ12~(\textit{1}),
ММ24~(\textit{2}) и МП~(\textit{3}) при длинах выборки $n\hm= 16$~(\textit{а}) и 64~(\textit{б}) }
     \end{figure}

\begin{multicols}{2}

\noindent
раметров, а данные в правых столбцах характеризуют разброс
результатов расчетов рассматриваемыми методами.


     Графические результаты численных экспериментов, представленные на рис.~3 и~4,
иллюстрируют следующие ожидаемые выводы:
     \begin{itemize} %[1)]
\item как смещение, так и разброс данных при расчете искомых параметров
маг\-нит\-но-ре\-зо\-нансно\-го изображения на основе рассматриваемых методов, т.\,е.\
посредством двухпараметрического анализа данных изображения, заметно уменьшаются с
ростом отношения сигнала к шуму, а также с увеличением длины выборки~$n$;
\item точность вычисления как~$\nu$, так и~$\sigma$, методом ММ12 выше, чем другими
методами при малом SNR и небольшой длине выборки, а с ростом длины выборки или
увеличением SNR точность рассматриваемых методов выравнивается.
     \end{itemize}

     Представленные данные иллюстрируют возможность вычисления обоих искомых
статистических параметров~$\nu$ и~$\sigma^2$ в условиях распределения Райса с
высокой точностью в достаточно широком диапазоне значений отношения сигнала к
шуму.




\section{Заключение}

    В работе представлены теоретические основы, проведено численное исследование и
со\-по\-став\-ление методов решения задачи анализа и обработки маг\-нит\-но-резонансного
изображения на \mbox{основе} нового подхода, состоящего в решении двух\-па\-ра\-мет\-ри\-че\-ской
задачи в условиях статистического распределения Райса. Рассматриваемые методы
разработаны на основе использования так называемой двух\-па\-ра\-мет\-ри\-че\-ской модели, в
рамках которой анализируемые данные характеризуются не одним (как в традиционном
однопараметрическом приближении), а двумя неизвестными параметрами: средней
величиной сигнала и дисперсией шума. Рассматриваемые методы объединены концепцией
двухпараметрического подхода к решению задачи анализа и обработки
магнитно-резонансного изоб\-ра\-же\-ния. Суть данной концепции заключается в обоснованной
возможности математического расчета сразу обоих априорно неизвестных статистических
параметров сигнала и шума, формирующих анализируемое изображение. Знание этих
параметров позволяет, в свою очередь, восстановить исходное, не искаженное шумом
изображе-\linebreak\vspace*{-12pt}

\pagebreak

\end{multicols}

     \begin{figure} %fig4
     \vspace*{1pt}
\begin{center}
\mbox{%
\epsfxsize=147mm
\epsfbox{yak-4.eps}
}
\end{center}
\vspace*{-9pt}
     \Caption{Графики рассчитанных значений параметра $\sigma$ (левый столбец) и его
среднеквадратичного отклонения (правый столбец) в зависимости от отношения сигнала к
шуму при двухпараметрическом анализе райсовского сигнала методами ММ12~(\textit{1}),
ММ24~(\textit{2}) и МП~(\textit{3})
при длинах выборки $n= 16$~(\textit{а}) и~64~(\textit{б})}
     \end{figure}

\begin{multicols}{2}

\noindent
ние в сис\-те\-мах маг\-нит\-но-ре\-зо\-нанс\-ной визуализации. Разработаны различные
алгоритмы расчета искомых величин сигнала и шума в зависимости от используемого
статистического подхода к решению двухпараметрической задачи: метод максимума
правдоподобия, метод моментов, основанный на измерении второго и четвертого
моментов сигнала, метод моментов, основанный на измерении первого и второго
моментов.

    В работе проведен сравнительный анализ статистических характеристик
рассматриваемых методов решения двухпараметрической задачи. В~результате
численного моделирования получены и со\-по\-став\-ле\-ны данные для смещения и разброса
искомых параметров, обеспечиваемые методами МП, ММ24 и ММ12.

    Одним из важных преимуществ развитых методов является их применимость к
решению существенно нелинейных задач, каковыми, в част\-ности,\linebreak являются задачи
шумоподавления в маг\-нит\-но-ре\-зо\-нансной визуализации.

    Результаты численного эксперимента демонстрируют возможность эффективного
восстановления исходного, не искаженного шумом сигнала магнитно-резонансного
изображения посредством развитых методов, основанных на решении
двухпараметрической задачи в условиях статистического распределения Райса.

{\small\frenchspacing
 {%\baselineskip=10.8pt
 \addcontentsline{toc}{section}{References}
 \begin{thebibliography}{99}
\bibitem{1-ya}
\Au{Perona P., Malik J.} Scale-space and edge detection using anisotropic diffusion~// IEEE
Trans. Pattern Anal. Machine Intelligence, 1990. Vol.~12. No.\,7.
P.~629--639.
\bibitem{2-ya}
\Au{Gerig G., Kubler O., Kikinis~R., Jolesz~F.\,A.} Nonlinear anisotropic filtering of MRI
data~// IEEE Trans. Med. Imaging, 1992. Vol.~11. No.\,6. P.~221--232.
\bibitem{3-ya}
\Au{Wood J.\,C., Johnson K.\,M.} Wavelet packet denoising of magnetic resonance images:
Importance of Rician noise at low SNR~// Magnet. Reson. Med., 1999. Vol.~41. No.\,3.
P.~631--635.
\bibitem{4-ya}
\Au{Delakis I., Hammad O., Kitney~R.\,I.}
Wavelet based denoising algorithm for images acquired with
parallel magnetic resonance imaging (MRI)~// Phys. Med. Biol., 2007. Vol.~52. No.\,13.
P.~3741--3751.
\bibitem{5-ya}
\Au{Starck J.-L., Cand{\ptb{\!\!\`{e}}}s E.\,J., Donoho~D.\,L.} The curvelet transform for image
denoising~// IEEE Trans. Image Process., 2002. Vol.~11. No.\,6. P.~670--684.
\bibitem{6-ya} %6
\Au{Ma J., Plonka G.} The curvelet transform~// IEEE Signal Proc. Mag., 2010.
Vol.~27. No.\,2. P.~118--133.

\bibitem{12-ya} %7
\Au{Benedict T.\,R., Soong T.\,T.} The joint estimation of signal and noise from the sum
envelope~// IEEE Trans. Inform. Theory, 1967. Vol.~IT-13. No.\,3. P.~447--454.

\bibitem{9-ya} %8
\Au{Henkelman R.\,M.} Measurement of signal intensities in the presence of noise in MR
images~// Med. Phys., 1985. Vol.~12. No.\,2. P.~232--233.

\bibitem{8-ya} %9
\Au{Wang T., Lei T.} Statistical analysis of MR imaging and its application in image
modeling~// IEEE  Conference (International) Image Processing and Neural Networks
Proceedings, 1994. Vol.~I. P.~866--870.


\bibitem{10-ya} %10
\Au{Gudbjartsson H., Patz S.} The Rician distribution of noisy MRI data~// Magnet. Reson.
Med., 1995. Vol.~34. P.~910--914.
\bibitem{11-ya} %11
\Au{Sijbers J., den Dekker A.\,J., Scheunders~P., van Dyck~D.} Maximum-likelihood
estimation of Rician distribution parameters~// IEEE Trans. Med. Imaging, 1998.
Vol.~17. No.\,3. P.~357--361.

\bibitem{13-ya} %12
\Au{Carobbi C.\,F.\,M., Cati M.} The absolute maximum of the likelihood function of the
Rice distribution: Existence and uniqueness~// IEEE Trans. Instrum.
Meas., 2008. Vol.~57. No.\,4. P.~682--689.
\bibitem{7-ya} %13
\Au{Sheil W.\,C.} Magnetic resonance imaging (MRI Scan).
MedicineNet.com. Retrieved April 27, 2012.

\bibitem{14-ya}
\Au{Rice S.\,O.} Mathematical analysis of random noise~// Bell Syst. Tech.~J.,
1944. Vol.~23. P.~282--322.
\bibitem{15-ya}
\Au{Яковлева T.\,В.} Условия применимости статистической модели Райса и расчет
параметров райсовского сигнала методом максимума правдоподобия~//
Компьютерные исследования и моделирование, 2014. Т.~6. №\,1. С.~13--25.
\bibitem{16-ya}
\Au{Яковлева Т.\,В., Кульберг Н.\,С.} Особенности функции правдоподобия
статистического распределения Райса~// Докл. РАН, 2014. Т.~457. №\,4.
С.~394--397.
\bibitem{17-ya}
\Au{Yakovleva T.\,V., Kulberg N.\,S.} Noise and signal estimation in MRI: Two-parametric
analysis of Rice-distributed data by means of the maximum likelihood approach~// Am.
J.~Theor. Appl. Stat., 2013. Vol.~2. No.\,3. P.~67--79.
\bibitem{18-ya}
\Au{Яковлева T.\,В.} Обзор методов обработки маг\-нит\-но-ре\-зо\-нансных
изображений и развитие нового двухпараметрического метода моментов~//
Компьютерные исследования и моделирование, 2014. Т.~6. №\,2. С.~231--244.
\bibitem{19-ya}
\Au{Park, Jr.,  J.\,H.} Moments of generalized Rayleigh distribution~// Q.~Appl. Math., 1961.
Vol.~19. No.\,1. P.~45--49.
\bibitem{20-ya}
\Au{Абрамовиц М., Стиган И.} Справочник по специальным функциям.~--- М.: Наука,
1979. 832~с.
 \end{thebibliography}

 }
 }

\end{multicols}

\vspace*{-6pt}

\hfill{\small\textit{Поступила в редакцию 09.06.14}}

%\newpage

\vspace*{12pt}

\hrule

\vspace*{2pt}

\hrule

%\vspace*{12pt}

\def\tit{MATHEMATICAL STATISTICS METHODS AS~A~TOOL
OF~TWO-PARAMETRIC MAGNETIC-RESONANCE IMAGE  ANALYSIS}

\def\titkol{Mathematical statistics methods as~a~tool
of~two-parametric magnetic-resonance image analysis}

\def\aut{T.\,V.~Yakovleva and N.\,S.~Kulberg}

\def\autkol{T.\,V.~Yakovleva and N.\,S.~Kulberg}

\titel{\tit}{\aut}{\autkol}{\titkol}

\vspace*{-9pt}

\noindent
Dorodnicyn Computing Center of the Russian
Academy of Sciences, 40~Vavilov Str., Moscow 119333, Russian Federation


\def\leftfootline{\small{\textbf{\thepage}
\hfill INFORMATIKA I EE PRIMENENIYA~--- INFORMATICS AND
APPLICATIONS\ \ \ 2014\ \ \ volume~8\ \ \ issue\ 3}
}%
 \def\rightfootline{\small{INFORMATIKA I EE PRIMENENIYA~---
INFORMATICS AND APPLICATIONS\ \ \ 2014\ \ \ volume~8\ \ \ issue\ 3
\hfill \textbf{\thepage}}}

\vspace*{3pt}

\Abste{The paper considers the methods of the magnetic-resonance image analysis, based on
the solution of the so-called two-parametric task. The elaborated methods provide
joint
calculation of both statistical parameters~--- the mathematical expectation of the random value
being analyzed and its dispersion, i.\,e., simultaneous estimation of both the useful signal and
the noise. The considered variants of the task solution employ the methods of mathematical
statistics: the maximum likelihood method and variants of the method of moments.
A~significant advantage of the elaborated two-parametric approach consists in the fact that it
provides an
efficient solution of nonlinear tasks including the tasks of noise suppression in
the systems of magnetic-resonance visualization. Estimation of the sought-for parameters
is based upon measured samples' data only and is not limited by any \textit{a priori}
suppositions. The paper provides the comparative analysis of the considered
methodology's
variants and presents the results of the computer simulation providing the statistical
characteristics of the estimated parameters' shift and scatter while
solving the task by various
methods. The presented methods of the Rician signal's two-parametric analysis can be used
within new information technologies at the stage of the stochastic values' processing.}

\KWE{Rice distribution; maximum likelihood method; method of moments; two-parametric
analysis; signal-to-noise ratio}

\DOI{10.14357/19922264140309}

%\vspace*{3pt}

  \begin{multicols}{2}

\renewcommand{\bibname}{\protect\rmfamily References}
%\renewcommand{\bibname}{\large\protect\rm References}

{\small\frenchspacing
 {%\baselineskip=10.8pt
 \addcontentsline{toc}{section}{References}
 \begin{thebibliography}{99}
\bibitem{1-ya-1}
\Aue{Perona, P., and J.~Malik}. 1990.
Scale-space and edge detection using anisotropic
diffusion. \textit{IEEE Trans. Pattern Anal. Machine Intelligence}
12(7):629--639.
\columnbreak

\bibitem{2-ya-1}
\Aue{Gerig, G., O.~Kubler, R.~Kikinis, and F.\,A.~Jolesz}. 1992.
Nonlinear anisotropic filtering of
MRI data. \textit{IEEE Trans. Med. Imaging} 11:221--232.

\vspace*{-2pt}

\bibitem{3-ya-1}
\Aue{Wood, J.\,C., and K.\,M.~Johnson}. 1999.
Wavelet packet denoising of magnetic resonance
images: Importance of Rician noise at low SNR. \textit{Magnet. Reson. Med.} 41(3):631--635.
\bibitem{4-ya-1}
\Aue{Delakis,~I., O.~Hammad, and R.\,I.~Kitney.}
2007. Wavelet based denoising algorithm for images acquired with parallel
magnetic resonance imaging (MRI). \textit{Phys. Med. Biol.} 52(13):3741--3751.
\bibitem{5-ya-1}
\Aue{Starck, J.\,L., E.\,J.~Cand{\ptb{\!\!\`{e}}}s, and D.\,L.~Donoho}. 2002.
The curvelet transform for image denoising.
\textit{IEEE Trans. Image Process.} 11(6):670--684.
\bibitem{6-ya-1}
\Aue{Jianwei, M., and G.~Plonka}. 2010.
The Curvelet transform. \textit{IEE Signal Proc. Mag.} 27(2):118--133.


\bibitem{12-ya-1} %7
\Aue{Benedict, T.\,R., and T.\,T.~Soong}. 1967.
The joint estimation of signal and noise from the sum
envelope. \textit{IEEE Trans. Inform. Theory} IT-13(3):447--454.


\bibitem{9-ya-1} %8
\Aue{Henkelman, R.\,M.} 1985.
Measurement of signal intensities in the presence of noise in MR
images. \textit{Med. Phys.} 12(2):232--233.

\bibitem{8-ya-1} %9
\Aue{Wang T., and T.~Lei}. 1994.
Statistical analysis of MR imaging and its application in image
modeling. \textit{IEEE Conference (International) Image Processing
and Neural Networks Proceedings}. I:866--870.

\bibitem{10-ya-1} %10
\Aue{Gudbjartsson, H., and S.~Patz}. 1995.
The Rician distribution of noisy MRI data. \textit{Magnet.
Reson. Med.} 34:910--914.
\bibitem{11-ya-1} %11
\Aue{Sijbers, J., A.\,J.~den Dekker, P.~Scheunders, and D.~Van Dyck}. 1998.
Maximum-likelihood
estimation of Rician distribution parameters. \textit{IEEE Trans.
Med. Imaging} 17(3):357--361.


\bibitem{13-ya-1} %12
\Aue{Carobbi, C.\,F.\,M., and M.~Cati}. 2008.
The absolute maximum of the likelihood function
of the Rice distribution: Existence and uniqueness. \textit{IEEE Trans. Instrum.
Meas.} 57(4):682--689.

\bibitem{7-ya-1} %13
\Aue{Sheil, W.\,C.} 2012. Magnetic resonance imaging (MRI Scan). MedicineNet.com.
Retrieved April~27, 2012.

\bibitem{14-ya-1}  %14
\Aue{Rice, S.\,O.} 1944. Mathematical analysis of random noise. \textit{Bell Syst. Tech.~J.}
23:282--322.
\bibitem{15-ya-1}
\Aue{Yakovleva, T.\,V.} 2014. Usloviya primenimosti sta\-ti\-sti\-che\-skoy modeli
Raysa i raschet parametrov raysovskogo signala metodom maksimuma
pravdopodobiya [Conditions of
Rice statistical model applicability and estimation of the Rician signal's
parameters by maximum likelihood technique].
\textit{Komp'yuternye issledovaniya i modelirovanie}  [Computer
Research and Simulation] 6(1):13--25.
\bibitem{16-ya-1}
\Aue{Yakovleva, T.\,V., and N.\,S.~Kulberg}.
2014. Oso\-ben\-no\-sti funktsii pravdopodobiya
statisticheskogo raspredeleniya Raysa [The likelihood function's peculiarities for Rice
statistical dictribution]. \textit{Dokl. RAS} 457(4):394--397.
\bibitem{17-ya-1}
\Aue{Yakovleva, T.\,V., and N.\,S.~Kulberg}. 2013.
Noise and signal estimation in MRI: Two-parametric
analysis of Rice-distributed data by means of the maximum likelihood
approach. \textit{Am. J.~Theor. Appl. Stat.} 2(3):67--79.
\bibitem{18-ya-1}
\Aue{Yakovleva, T.\,V.} 2014. Obzor metodov obrabotki magnitno-rezonansnykh
izobrazheniy i razvitie novogo dvukhparametricheskogo metoda momentov
[Review of MRI processing techniques and elaboration of a new two-parametric
method of moments]. \textit{Komp'yuternye
Issledovaniya i Modelirovanie} [Computer Research and Simulation] 6(2):231--244.
\bibitem{19-ya-1}
\Aue{Park, Jr., J.\,H.} 1961. Moments of generalized Rayleigh distribution.
\textit{Q. Appl. Math.} 19(1):45--49.
\bibitem{20-ya-1}
\Aue{Abramovits, М., and I.~Stigan}. 1979. \textit{Spravochnik
po spe\-tsi\-al'\-nym funktsiyam} [Reference book on special functions]. Moscow: Nauka.
832~p.
\end{thebibliography}

 }
 }

\end{multicols}

\vspace*{-6pt}

\hfill{\small\textit{Received June 09, 2014}}

\vspace*{-18pt}

\Contr


\noindent
\textbf{Yakovleva Tatiana V.} (b.\ 1956)~---
 Candidate of Science (PhD) in physics and mathematics, senior scientist,
 Dorodnicyn Computing Center, Russian Academy of Sciences,
 40~Vavilov Str., Moscow 119333, Russian Federation;
 yakovleva@ccas.ru

 \vspace*{3pt}

\noindent
\textbf{Kulberg Nickolas S.} (b.\ 1970)~--- Candidate of Science (PhD) in physics and
mathematics, senior scientist, Dorodnicyn Computing Center, Russian Academy of
Sciences, 40~Vavilov Str., Moscow 119333, Russian Federation;
kulberg@yandex.ru

\label{end\stat}

\renewcommand{\bibname}{\protect\rm Литература}