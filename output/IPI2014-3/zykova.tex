\def\stat{zykova}

\def\tit{ПРИМЕНЕНИЕ ПОЛУТОНОВЫХ ПРЕДСТАВЛЕНИЙ ПРИ~АНАЛИЗЕ
ИЗМЕНЕНИЙ ЦВЕТНЫХ ИЗОБРАЖЕНИЙ}

\def\titkol{Применение полутоновых представлений при~анализе
изменений цветных изображений}

\def\aut{О.\,П. Архипов, З.\,П. Зыкова}

\def\autkol{О.\,П. Архипов, З.\,П. Зыкова}

\titel{\tit}{\aut}{\autkol}{\titkol}

\renewcommand{\thefootnote}{\arabic{footnote}}
\footnotetext[1]{Орловский филиал Института проблем информатики Российской академии наук, arkhipov12@yandex.ru}
\footnotetext[2]{Орловский филиал Института проблем информатики Российской академии наук, zykzoya@yandex.ru}


\Abst{Чтобы сократить вычислительные и временн$\acute{\mbox{ы}}$е затраты при анализе цветных
изображений, на практике часто используют их упрощенные представления.
Например, применение полутоновых пред\-став\-ле\-ний цветных изображений в
видеоаналитике позволяет уменьшить объем затрат в 2--3~раза. Часть
информации об исходном изображении при этом теряется, что снижает точность
анализа. Погрешности результатов существенно зависят от выбора функций
преобразования цветных изображений в полутоновые. Предлагается алгоритм
совместного использования различных полутоновых пред\-став\-ле\-ний цветных
изображений, обеспечивающий повышение точности результатов. Указаны области
практического применения, в которых использование предлагаемого алгоритма не
влечет дополнительных временн$\acute{\mbox{ы}}$х затрат на соответствующие процедуры.
Приведены иллюстрирующие примеры.}

\KW{цветные изображения; полутоновые представления; погрешность анализа;
видеоаналитика}

\DOI{10.14357/19922264140310}

\vskip 14pt plus 9pt minus 6pt

\thispagestyle{headings}

\begin{multicols}{2}

\label{st\stat}

\section{Полутоновые представления изображений}

  При анализе цветных изображений в большинстве случаев применяют
методы, основанные на использо\-вании полутонового представления
изоб\-ра\-же\-ний. Для получения полутонового представления цветного
изображения пикселы трехмерного RGB-про\-стран\-ст\-ва преобразуются в
пикселы одномерного пространства, тем или иным способом
характеризующего яркость RGB-пик\-се\-лов.

  Традиционно используемые полутоновые представления описываются
функциями вида:
  $$
  f_0:\ f_0(r, g, b) = (c_0, c_0, c_0)\,,
  $$
  где $c_0 = 0{,}2126r + 0{,}7152g + 0{,}0722b$ (функция соответствует
американскому стандарту аналогового телевидения NTSC~--- National Television
Standards Committee);
  $$
  f_1:\ f_1(r, g, b) = (c_1, c_1, c_1)\,,
  $$
   где $c_1 = 0{,}21r + 0{,}72g + 0{,}07b$ (коэффициенты получены при
округлении коэффициентов формулы значений функции~$f_0$);
  $$
  f_2:\ f_2(r, g, b) = (c_2, c_2, c_2)\,,
  $$
  где $c_2 = 0{,}299r + 0{,}587g + 0{,}114b$ (функция соответствует
американскому стандарту цифрового телевидения ATSC~--- Advanced Television
Systems Committee);
  $$
  f_3:\ f_3(r, g, b) = (c_3, c_3, c_3)\,,
  $$
где  $c_3 = 0{,}3r + 0{,}59g + 0{,}11b$ (коэффициенты получены при округлении
коэффициентов формулы значений функции~$f_2$);
  $$
  f_4:\ f_4(r, g, b) = (c_4, c_4, c_4)\,,$$
  где
  $$c_4 = \fr{r+g+b}{3}\,;$$
  $$
  f_5:\ f_5(r, g, b) = (c_5, c_5, c_5)\,,$$
  где
  $$c_5=\fr{\max (r,g,b)+\min(r,g,b)}{2}\,;
  $$
  $$
  f_6:\ f_6(r, g, b) = (c_6, c_6, c_6)\,,
  $$
где
$$
c_6= \fr{255 L(r,g,b)}{\rho(\max L(r,g,b)+\min L(r,g,b))}\,,
  $$
  $L$~--- координата из Lab-про\-стран\-ст\-ва, соответствующая пикселу
$(r, g, b)$;
  $$
  f_7:\ f_7(r, g, b) = (c_7, c_7, c_7)\,,
  $$
где  $c_7$  определяется алгоритмом PhotoShop.

  Заметим, что вклад значений координат $r$, $g$ и~$b$ в значения функций
полутонового преобразования $f_4$ и~$f_5$ одинаков, а для других функций,
учи\-ты\-ва\-ющих цветовосприятие наблюдателя, различен.

\setcounter{figure}{1}
\begin{figure*}[b] %fig2
%\vspace*{1pt}
\begin{center}
\mbox{%
\epsfxsize=164.261mm
\epsfbox{arh-2.eps}
}
\end{center}
\vspace*{-9pt}
\Caption{Изображения $f_0(I_1)$~(\textit{а}),$f_1(I_1)$~(\textit{б}), $f_2(I_1)$~(\textit{в}),
$f_3(I_1)$~(\textit{г}), $f_4(I_1)$~(\textit{д}), $f_5(I_1)$~(\textit{е}), $f_6(I_1)$~(\textit{ж}) и
$f_7(I_1)$~(\textit{з})}
%\vspace*{6pt}
%\end{figure*}
\renewcommand{\figurename}{\protect\bf Таблица}
%\renewcommand{\tablename}{\protect\bf Таблица}
\setcounter{figure}{0}
%\begin{table*}
{\small
\begin{center}
\Caption{Несовпадение пикселов изображений, приведенных на рис.~2}
\vspace*{2ex}

\tabcolsep=10pt
\begin{tabular}{|l|c|c|c|c|c|c|c|}
\hline
\multicolumn{1}{|c|}{Рисунок}&2,\,\textit{а}&2,\,\textit{б}&2,\,\textit{в}&2,\,\textit{г}&2,\,\textit{д}&2,\,\textit{е}&
2,\,\textit{ж}\\
\hline
\ \ \ \ 2,\,\textit{б}&41\%&&&&&&\\
\ \ \ \ 2,\,\textit{в}&97\%&97\%&&&&&\\
\ \ \ \ 2,\,\textit{г}&97\%&97\%&35\%&&&&\\
\ \ \ \ 2,\,\textit{д}&99\%&99\%&98\%&98\%&&&\\
\ \ \ \ 2,\,\textit{е}&99\%&99\%&99\%&99\%&98\%&&\\
\ \ \ \ 2,\,\textit{ж}&100\%\hphantom{9}&100\%\hphantom{9}&100\%\hphantom{9}&
100\%\hphantom{9}&99\%&99\%&\\
\ \ \ \ 2,\,\textit{з}&98\%&98\%&97\%&97\%&99\%&99\%&100\%\\
\hline
\end{tabular}
\end{center}}
%\end{table*}
\end{figure*}



\pagebreak

\begin{center}  %fig1
\vspace*{2pt}
\mbox{%
\epsfxsize=78mm
\epsfbox{arh-1.eps}
}
  \vspace*{2pt}

{{\figurename~1}\ \ \small{Изображение $I_1$}}
  \end{center}

\vspace*{3pt}

\renewcommand{\figurename}{\protect\bf Рис.}
\renewcommand{\tablename}{\protect\bf Таблица}
\setcounter{figure}{2}
\setcounter{table}{1}





  В соответствии с потребностями телевидения в каждой из функций~$f_0$--$f_3$
  при получении серых пикселов применяются разные весовые
коэффициенты при координатах~$r$, $g$ и~$b$.
%
В~указанных формулах
учитывается различная значимость координат~$r$, $g$ и~$b$ в формировании
визуального восприятия человека. Так, больший вес зеленого цвета обусловлен
тем, что зеленый цвет воспринимается более светлым, чем красный и синий.
Меньший вес синего цвета используется, поскольку синий цвет
воспринимается более темным, чем зеленый и красный.

  Значения функций полутонового преобразо\-вания~$f_6$ и~$f_7$ нелинейно
зависят от значений ко\-ор\-динат~$r$, $g$ и~$b$, поскольку различный вклад
этих координат определяется правилами по\-стро\-ения равноконтраст\-ных
цветовых пространств, которые ориентированы на цветовосприятие
стандартного наблюдателя.

  Часть информации об исходном изображении в любом их полутоновом
представлении теряется, поскольку происходит слияние цветов (одному серому
пикселу соответствует множество RGB-пик\-селов).

  Рассмотрим, например, изображение~$I_1$, со\-сто\-ящее из двенадцати
разноцветных квадратов (рис.~1).

\begin{figure*}[b] %fig3	
\vspace*{1pt}
\begin{center}
\mbox{%
\epsfxsize=164mm
\epsfbox{arh-3.eps}
}
\end{center}
\vspace*{-9pt}
\Caption{Выборочные изображения полей градиентов для $f_0(I_1)$~(\textit{а}) и
$f_7(I_1)$~(\textit{б})}
\end{figure*}


  Полутоновые представления изображения~$I_1$ имеют вид в соответствии с
рис.~2 и различаются даже для близких по определению функций ($f_0(I_1)$ и
$f_1(I_1)$, $f_2(I_1)$ и $f_3(I_1)$).


  В табл.~1 приведены результаты подсчета несовпадений пикселов
изображений, приведенных на рис.~2. Доля несовпадений представлена в
процентах при округлении до целого.


  Как видно из табл.~1, между значительным чис\-лом пикселов полутоновых
представлений тестового изображения имеются существенные различия с
количественной точки зрения. При этом, как и следовало ожидать, визуально
сходные изображения ($f_0(I_1)$ и $f_1(I_1)$, $f_2(I_1)$ и $f_3(I_1)$) имеют
меньшие различия по сравнению с общим случаем.

\section{Характеристики полутоновых изображений}

  При автоматическом анализе используются различ\-ные количественные
характеристики полутоновых изображений, которые могут значительно
различаться даже для визуально неразличимых изоб\-ра\-жений.

  Например, для обнаружения объектов на цветных изображениях и их
особенностей часто применяют методы, основанные на поиске локальных
экстремумов градиента (методы Робертса, Превитта, Собела и др.) в
полутоновых представлениях изображений~[1].

  Например, градиенты $\nabla p = (G_x, G_y)$ во внут\-рен\-них пикселах $p$
полутоновых изображений, вычисленные с помощью оператора Собела,
являются результатом линейной обработки фрагментов полутоновых
изображений масками $3\times 3$. Перед дальнейшим использованием
вычисленные значения градиентов квантуются (минимум по восьми ячейкам:
0$^\circ$, 45$^\circ$, 90$^\circ$, 135$^\circ$, 180$^\circ$, $-45$$^\circ$, $-
90$$^\circ$,\linebreak $-135$$^\circ$).

  Поля градиентов полутоновых представлений существенно зависят от вида
функции полутонового преобразования. Несмотря на грубое квантование и
малую выборку результатов, можно \mbox{видеть} существенное различие полей
градиентов, например для полутоновых представлений $f_0(I_1)$ и $f_7(I_1)$,
 представленных на рис.~3 стрелками для пикселов, координаты которых
по вертикали и горизонтали кратны десяти.

  Для многих практически важных проектов анализа изображений необходимо
и применяется квантование по большому числу ячеек. Например, в~[2] описан
автоматизированный программный метод анализа изображений
аэрокосмических фотопланов, при реализации которого применяется\linebreak
180~ячеек. Очевидно, что увеличение числа ячеек квантования
приведет к еще большему разбросу результатов.

  Аналогичные выводы можно сделать и при расширении базы сравнения
результатов (рис.~4\linebreak и~5).

  Различным направлениям на рис.~4 соответствуют векторы и соответственно
пикселы, имеющие следующие цвета: 0$^\circ$~--- красный; 45$^\circ$~---
пурпурный; 90$^\circ$~--- синий; 135$^\circ$~--- циан; 180$^\circ$~---
  тем\-но-зе\-ле\-ный; $-45^\circ$~--- оранжевый; $-90^\circ$~--- желтый; $-
135^\circ$~--- свет\-ло-зе\-леный.

  Вследствие того, что для графического пред\-став\-ле\-ния соотношения модулей
градиентов использованы серые пикселы с координатами вида

  \begin{table*}[b]\small %tabl2
  \begin{center}
  \Caption{Доля несовпадений пикселов полей направлений градиентов}
  \vspace*{2ex}

  \begin{tabular}{|c|c|c|c|c|c|c|c|}
  \hline
\multicolumn{1}{|c|}{Функция}&$f_0(I_1)$&$f_1(I_1)$&$f_2(I_1)$&$f_3(I_1)$&$f_4(I_1)$&$f_5(I_1)$&$f_6(I_1)$\\
\hline
$f_1(I_1)$&17\%&&&&&&\\
$f_2(I_1)$&47\%&48\%&&&&&\\
$f_3(I_1)$&47\%&48\%&15\%&&&&\\
$f_4(I_1)$&80\%&80\%&52\%&51\%&&&\\
$f_5(I_1)$&73\%&74\%&70\%&70\%&73\%&&\\
$f_6(I_1)$&38\%&38\%&50\%&50\%&72\%&72\%&\\
$f_7(I_1)$&44\%&44\%&48\%&48\%&72\%&71\%&24\%\\
\hline
\end{tabular}
\end{center}
\end{table*}

  \begin{table*}\small %tabl3
  \begin{center}
  \Caption{Доля несовпадений пикселов полей модулей градиентов}
  \vspace*{2ex}

  \begin{tabular}{|c|c|c|c|c|c|c|c|}
  \hline
\multicolumn{1}{|c|}{Функция}&$f_0(I_1)$&$f_1(I_1)$&$f_2(I_1)$&$f_3(I_1)$&$f_4(I_1)$&$f_5(I_1)$&$f_6(I_1)$\\
\hline
$f_1(I_1)$&48\%&&&&&&\\
$f_2(I_1)$&81\%&83\%&&&&&\\
$f_3(I_1)$&81\%&82\%&40\%&&&&\\
$f_4(I_1)$&95\%&96\%&79\%&79\%&&&\\
$f_5(I_1)$&89\%&90\%&85\%&85\%&64\%&&\\
$f_6(I_1)$&83\%&83\%&82\%&82\%&83\%&86\%&\\
$f_7(I_1)$&85\%&85\%&86\%&86\%&90\%&90\%&81\%\\
\hline
\end{tabular}
\end{center}
\end{table*}

\end{multicols}

\begin{figure} %fig4
\vspace*{1pt}
\begin{center}
\mbox{%
\epsfxsize=164mm
\epsfbox{arh-4.eps}
}
\end{center}
\vspace*{-9pt}
\Caption{Изображения полей градиентов для $f_0(I_1)$~(\textit{а}) и $f_7(I_1)$~(\textit{б})}
%\end{figure*}
%\begin{figure*} %fig5
\vspace*{6pt}
\begin{center}
\mbox{%
\epsfxsize=164.55mm
\epsfbox{arh-5.eps}
}
\end{center}
\vspace*{-9pt}
\Caption{Изображения полей модулей градиентов для $f_0(I_1)$~(\textit{а}) и
$f_7(I_1)$~(\textit{б})}
\end{figure}

\begin{multicols}{2}

\noindent
  \begin{equation}
  m = \mbox{round} \left( 255 \left( 1-\sqrt { \fr{G_x^2+G_y^2}{2(4\cdot
255)^2}}\right)\right)\,,
  \label{e-zyk}
  \end{equation}
меньшим значениям модуля градиента на рис.~5 соответствуют более светлые
серые пикселы. Так, при наименьшем значении модуля градиента
($G_x\hm=G_y\hm=0$) $m\hm=255$, а при максимальном
($G_x\hm=G_y\hm=4\cdot255$)~--- $m\hm=0$.
\setcounter{figure}{6}

\begin{figure*}[b] %fig7
\vspace*{1pt}
\begin{center}
\mbox{%
\epsfxsize=164mm
\epsfbox{arh-7.eps}
}
\end{center}
\vspace*{-9pt}
\Caption{Изображения фрагментов контуров для $f_0(I_1)$~(\textit{а}), $f_2(I_1)$~(\textit{б})
и $f_4(I_1)$~(\textit{в}) }
\end{figure*}



  В табл.~2 и~3 приведены результаты подсчета несовпадений пикселов
полутоновых представлений тестового изображения~$I_1$. Доля несовпадений
представлена в процентах при округлении до це\-лого.

  Как видно из табл.~2 и~3, направления и модули локальных градиентов
различаются для значительного числа пикселов полутоновых представлений
тестового изображения. При этом, как и следовало ожидать, для визуально
сходных изображений ($f_0(I_1)$ и $f_1(I_1)$, $f_2(I_1)$ и $f_3(I_1)$) поля
направлений и модулей локальных градиентов имеют меньшие различия по
сравнению с общим случаем.





\section{Анализ полутоновых изображений}

  Направления и модули локальных градиентов применяются для вычисления
различных особенностей цветных изображений. Например, на основе значений
модуля градиентов могут быть вычислены контуры содержащихся в
изображении объектов. Визуально в тестовом изображении~$I_1$ выделяются
контуры, представляющие равномерную сетку (рис.~6).



  Чтобы хотя бы отчасти компенсировать погрешности, возникающие при
полутоновом преобра\-зовании цветных изображений, для совместного
использова\-ния при определении контуров тестового изображения будет
использован один из многих возможных наборов функций преобразования,
в~котором для любой пары RGB-пик\-се\-лов найдется функция, принимающая от
этих аргументов разные значения: $f_0(I_1)$, $f_2(I_1)$, $f_4(I_1)$.




  После применения пороговой обработки вида
  $$
  n = \begin{cases} 255 &\ \mbox{при } m\geq 180\,;\\
  0 &\ \mbox{при } m< 180\,,
  \end{cases}
  $$
где $m$~--- нормализованные значения модуля градиентов из~(1), к
соответствующим полутоновым\linebreak\vspace*{-12pt}
\begin{center}  %fig6
\vspace*{8pt}
\mbox{%
\epsfxsize=78mm
\epsfbox{arh-6.eps}
}
  \vspace*{2pt}

{{\figurename~6}\ \ \small{Изображение визуально различимых контуров~$I_1$}}
  \end{center}

%\vspace*{6pt}

%\addtocounter{figure}{1}

\noindent изображениям вычисленные совокупности
фрагментов контуров тестового изображения имеют вид в соответствии с
рис.~7.

  В зависимости от используемой функции полутонового преобразования
обнаруживаются разные совокупности фрагментов контуров тестового
изображения. Ни одна из применяемых функций не позволила получить для
тестового изображения точные результаты.



\section{Совместный анализ полутоновых изображений}

  Точность результатов возрастает при совместном использовании нескольких
полутоновых представлений цветного изображения для анализа цветных
изображений~[3, 4].

  В рассматриваемом тестовом примере объединению данных (совокупностей
фрагментов контуров, полученных при раздельном исследовании полутоновых
изображений (см.\ рис.~7)), соответствует совокупность фрагментов контуров,
изображенных на рис.~8.



  Погрешность определения контуров значительно уменьшается, хотя и не
устраняется полностью (два небольших фрагмента контуров не
идентифицированы).

  В рассмотренном примере причиной ненулевой погрешности определения
контуров (см.\ рис.~8) при объединении отдельных результатов стало
преувеличение значимости координаты~$g$ в двух из трех использованных
функций по сравнению с другими координатами.




  В связи с этим имеет смысл рассмотреть функции, которые в совокупности
обеспечивают равнозначный вклад координат $r$, $g$ и~$b$ в результат\linebreak\vspace*{-12pt}
\begin{center}  %fig8
\vspace*{9pt}
\mbox{%
\epsfxsize=78mm
\epsfbox{arh-8.eps}
}
\end{center}
  \vspace*{2pt}

\noindent
{{\figurename~8}\ \ \small{Изображение объединенной совокупности фрагментов контуров}}

%\vspace*{3pt}

\addtocounter{figure}{1}


\noindent
анализа. Этому требованию удовлетворяет набор функций вида
  \begin{equation}
  \left.
  \begin{array}{rl}
  f_8 (rgb) &=(r,r,r)\,;\\[6pt]
  f_9(r,g,b) &= (g,g,g)\,;\\[6pt]
  f_{10}(r,g,b) &= (b,b,b)\,,
  \end{array}
  \right\}
  \label{e2-zyk}
  \end{equation}
а также, например,
\noindent
\begin{equation}
\left.
\begin{array}{rl}
f_{11}(r,g,b) &=(c_{11}, c_{11}, c_{11})\,; \\[6pt]
f_{12}(r,g,b) &=(c_{12}, c_{12}, c_{12})\,; \\[6pt]
f_{13}(r,g,b) &=(c_{13}, c_{13}, c_{13})\,,
\end{array}
\right\}
\label{e3-zyk}
\end{equation}
где

\noindent
\begin{align*}
c_{11}&=0{,}2r+0{,}6g+0{,}2b\,;\\
c_{12}&=0{,}2r+0{,}2g+0{,}6b\,;\\
c_{13}&=0{,}6r+0{,}2g+0{,}2b\,.
\end{align*}

Применение этих функций позволяет получить при объединении фрагментов
точное описание контуров тестового изображения (рис.~9 и~10).

  Затраты на проведение процедуры анализа складываются из суммы затрат на
исследование каждого полутонового представления, на объединение данных и
анализ интегрированных данных. При этом объем затрат увеличивается в
несколько раз (по сравнению с затратами на реализацию аналитических
процедур для единственного полутонового представления).

  Это допустимо в исследовательских, но считается неприемлемым в
инженерных проектах~[5]. В~связи с этим актуально выделение таких областей
применения, которые допускают разработку более экономичных алгоритмов
анализа на основе совместного исследования нескольких полутоновых
представлений цветных изображений и учета особенностей их практического
применения. В~рамках данной работы рассматривается одна из таких
областей~--- обнаружение изменений в последовательностях
видеоизображений.

\begin{figure*} %fig9
\vspace*{1pt}
\begin{center}
\mbox{%
\epsfxsize=128.18mm
\epsfbox{arh-9.eps}
}
\end{center}
\vspace*{-9pt}
\Caption{Изображения фрагментов контуров для $f_8(I_1)$~(\textit{а}), $f_9(I_1)$~(\textit{б}),
$f_{10}(I_1)$~(\textit{в}) и их объединение~(\textit{г})}
%\end{figure*}
			%\begin{figure*} %fig10
\vspace*{1pt}
\begin{center}
\mbox{%
\epsfxsize=128.18mm
\epsfbox{arh-10.eps}
}
\end{center}
\vspace*{-9pt}
\Caption{Изображения фрагментов контуров для $f_{11}(I_1)$~(\textit{а}),
$f_{12}(I_1)$~(\textit{б}), $f_{13}(I_1)$~(\textit{в}) и их объединение~(\textit{г})}
\end{figure*}

\vspace*{-6pt}

\section{Совместное использование функций полутонового
преобразования для~обнаружения изменений в~последовательностях
видеоизображений}

\vspace*{-2pt}

\begin{figure*} %fig11
\vspace*{6pt}
\begin{center}
\mbox{%
\epsfxsize=164.55mm
\epsfbox{arh-11.eps}
}
\end{center}
\vspace*{-9pt}
\Caption{Изображения $I_2$~(\textit{а}) и $I_3$~(\textit{б})}
%\end{figure*}
%\begin{figure*} %fig12
\vspace*{18pt}
\begin{center}
\mbox{%
\epsfxsize=154mm
\epsfbox{arh-12.eps}
}
\end{center}
\vspace*{-9pt}
\Caption{Изображения $\varphi_1(V_1)$~(\textit{а}), $\varphi_2(V_2)$~(\textit{б}) и
$\varphi_3(V_3)$~(\textit{в})
при использовании набора функций полутонового преобразования~(2)
}
\end{figure*}

  Совместное использование функций полутонового преобразования
необходимо для уменьшения вероятности того, что возникновение новых или
исчезновение имеющихся объектов цветного изоб\-ра\-же\-ния останется
незамеченным при автомати-\linebreak\vspace*{-12pt}

\pagebreak

\noindent
ческом анализе последовательности полутоновых
представлений видеоизображений.

  Введем обозначения:
\begin{itemize}
\item  $V_i$, $i = 0, 1, \ldots,$~--- последовательность цветных видеоизображений;
\item
  $\varphi_j$, $j = 1, 2, \ldots , J$,~--- набор функций полутонового
преобразования;
\item
  $F$~--- функция анализа полутоновых изображений;
\item
  $\rho$~--- мера близости значений функции~$F$.
  \end{itemize}

  Два изображения $V_{i^\prime}$ и~$V_{i^{\prime\prime}}$ считаются
различными, если для некоторой константы~$K$ выполнено соотношение:
  $$
  \rho (F (\varphi_j(V_{i^\prime})), F(\varphi_j(V_{i^{\prime\prime}})))>K\,.
  $$

  На начальном этапе предлагаемого алгоритма вычисляется $J$ эталонных
значений функции~$F$:
  $$
F(\varphi_j(V_0))\,,\enskip  j = 1, 2, \ldots , J\,.
$$

  Затем при исследовании изображений из видеопоследовательности
поочередно используются функции полутонового преобразования:
  \begin{equation}
  \rho ( F( \varphi_j(V_0)), F(\varphi_j(V_i)))> K\,,\enskip i=1,2,\ldots ,
  \label{e4-zyk}
  \end{equation}
причем значение~$j$ определяется как остаток от деления~$i$ на $(J\hm+1)$.

\columnbreak

  Таким образом, изменения, не обнаруженные при использовании одной
функции полутонового преобразования, могут быть обнаружены при
использовании ка\-кой-ли\-бо другой. Затраты при этом не увеличиваются,
поскольку для каждого видеоизображения используется только одно
полутоновое представление.

  Рассмотрим изображения~$I_2$ и~$I_3$ (рис.~11).

  Пусть
  $$
V_0 = I_2\,;\enskip  V_i = I_3\,,\enskip  i = 1, 2, 3\,.
$$



  При использовании набора функций полутонового преобразования~(2),
когда
  \begin{gather*}
\varphi_1 = f_8\,;\enskip \varphi_2 = f_9\,;\enskip \varphi_3 = f_{10}\,;\\
\varphi_1(V_0) = \varphi_2(V_0) = \varphi_3(V_0) = V_0\,,
\end{gather*}
последовательно идентифицируются все объекты, возникшие на изображениях
видеопоследовательности (рис.~12).



  При использовании набора функций полутонового преобразования~(3),
когда
  \begin{gather*}
\varphi_1 = f_{11}\,;\enskip  \varphi_2 = f_{12}\,;\enskip  \varphi_3 = f_{13}\,;\\
\varphi_1(V_0) = \varphi_2(V_0) = \varphi_3(V_0) = V_0\,,
\end{gather*}
также последовательно идентифицируются все объекты, возникшие на
изображениях видеопоследовательности (рис.~13).

\begin{figure*} %fig13
\vspace*{1pt}
\begin{center}
\mbox{%
\epsfxsize=164mm
\epsfbox{arh-13.eps}
}
\end{center}
\vspace*{-9pt}
\Caption{Изображения $\varphi_1(V_1)$~(\textit{а}), $\varphi_2(V_2)$~(\textit{б}) и
$\varphi_3(V_3)$~(\textit{в})
при использовании набора функций полутонового преобразования~(3)}
\vspace*{-6pt}
\end{figure*}

  Обнаружение изменений в рассмотренной тес\-то\-вой последовательности
изображений может быть установлено автоматически после вычисления нормы
разности матриц, соответствующих полутоновым представлениям
изображений тестовой последовательности и проверки выполнения
критерия~(4).

\vspace*{-10pt}

\section{Заключение}

\vspace*{-2pt}

  Предложен алгоритм совместного использования полутоновых
представлений при анализе цветных изображений, повышающий точность
обнаружения изменений в видеопоследовательностях без увеличения затрат на
выполнение соответствующих процедур.

\vspace*{-10pt}

{\small\frenchspacing
 {%\baselineskip=10.8pt
 \addcontentsline{toc}{section}{References}
 \begin{thebibliography}{9}

 \vspace*{-2pt}

\bibitem{1-zyk}
Методы компьютерной обработки изображений~/ Под ред. В.\,А.~Сойфера.~--- 2-е
изд.~--- М.: Физматлит, 2003. 784~с.

\columnbreak


%\vspace*{12pt}

\bibitem{2-zyk}
\Au{Щепин М.\,В.} Автоматизированный программный \mbox{метод} анализа
изображений аэрокосмических фотопланов. Векторизация~--- анализ
ландшафтных и тек\-то\-ни\-ческих структур~// Современные проб\-ле\-мы
дистан\-ционного зондирования Земли из космо\-са (Физические основы, методы
и технологии мониторинга окружающей среды, потенциально опасных
явлений и объектов): Сб. докладов 2-й Открытой Всеросс. конф.
Т.~2. С.~209--215. {\sf
www.iki.rssi.ru/earth/articles/sec7\_08.pdf}.

\vspace*{9pt}

\bibitem{3-zyk}
Recog.ru~--- Распознавание образов для программистов. {\sf
http://recog.ru/blog/opencv}.

\vspace*{9pt}

\bibitem{4-zyk}
\Au{Van de Sande K.\,E.\,A., Gevers~T., Shoek~C.\,G.\,M.} Evaluating color
descriptors for object and scene recognition~// IEEE Trans. Pattern Analysis
Machine Intelligence, 2010. Vol.~32. No.\,9. P.~1582--1596.

\vspace*{9pt}

\bibitem{5-zyk}
\Au{Конушин А.\,C.} Введение в компьютерное зрение. Лекция~5.~---
Лекториум, 20.02.12. {\sf www.lektorium.tv/\linebreak lecture/13541?id=13541}.
 \end{thebibliography}

 }
 }

\end{multicols}

\vspace*{-9pt}

\hfill{\small\textit{Поступила в редакцию 24.06.14}}

%\newpage

\vspace*{8pt}

\hrule

\vspace*{2pt}

\hrule

%\vspace*{12pt}

\def\tit{APPLICATION OF GRAY-SCALE PRESENTATIONS IN~THE~CASE~OF~TREND
ANALYSIS OF COLOR IMAGES}

\def\titkol{Application of gray-scale presentations in the case of trend
analysis of color images}

\def\aut{O.\,P.~Arkhipov and Z.\,P.~Zykova}

\def\autkol{O.\,P.~Arkhipov and Z.\,P.~Zykova}

\titel{\tit}{\aut}{\autkol}{\titkol}

\vspace*{-10pt}

\noindent
Orel Branch of Institute of Informatics Problems, Russian
Academy of Sciences, 137 Moskovskoe Highway, Orel 302025, Russian
Federation

\def\leftfootline{\small{\textbf{\thepage}
\hfill INFORMATIKA I EE PRIMENENIYA~--- INFORMATICS AND
APPLICATIONS\ \ \ 2014\ \ \ volume~8\ \ \ issue\ 3}
}%
 \def\rightfootline{\small{INFORMATIKA I EE PRIMENENIYA~---
INFORMATICS AND APPLICATIONS\ \ \ 2014\ \ \ volume~8\ \ \ issue\ 3
\hfill \textbf{\thepage}}}

\vspace*{5pt}


\Abste{In order to cut computing and time expenses during analysis of color
images, in practice often the simplified models are used. For example,
application of
gray-scale presentations of color images in video analysis allows reducing
expenses by 2--3 times. At the same time, a
part of information about the original image is
lost which reduces the analysis accuracy. The error of the
result depends essentially on choice
of functions used for gray scaling. The algorithm of shared usage
of multivariate gray-scale presentations of
color images providing increase of accuracy of result is proposed. The regions of
practical application are indicated, in which using the proposed algorithm
does not result in
additional temporary costs on corresponding procedures. Illustrative examples are given.}

\KWE{color images; gray-scale presentations; analysis error; video analysis}

\DOI{10.14357/19922264140310}

  \begin{multicols}{2}

\renewcommand{\bibname}{\protect\rmfamily References}
%\renewcommand{\bibname}{\large\protect\rm References}

{\small\frenchspacing
 {%\baselineskip=10.8pt
 \addcontentsline{toc}{section}{References}
 \begin{thebibliography}{9}

 \bibitem{1-zyk-1}
  Sojfer, V.\,A., ed. 2003.
  \textit{Metody komp'yuternoy obrabotki izobrazheniy} [Methods of computer
image processing]. 2nd ed. Moscow: Fizmatlit. 784~p.
  \bibitem{2-zyk-1}
  \Aue{Shchepin, M.\,V.} 2004. \textit{Avtomatizirovannyy pro\-gram\-mnyy metod
analiza izobrazheniy aerokosmicheskikh fotopla\-nov. Vektorizatsiya~--- analiz
landshaftnykh i tektoni\-cheskikh struktur} [Automated image analysis \mbox{software}
method aerospace photoplans. Vectorization~--- landscape analysis and tectonic
structures]. Available at: {\sf www.iki.rssi.ru/earth/articles/sec7\_08.pdf} (accessed
June~27, 2014).
  \bibitem{3-zyk-1}
  Recog.ru~--- Raspoznavanie obrazov dlya pro\-gram\-mi\-stov [Pattern recognition for
programmers]. Available at: {\sf http://recog.ru/blog/opencv/} (accessed June~27, 2014).
  \bibitem{4-zyk-1}
  \Aue{Van de Sande, K.\,E.\,A., T.~Gevers, and C.\,G.\,M.~Shoek.} 2010.
Evaluating color descriptors for object and scene recognation. \textit{IEEE Trans.
Pattern Analysis Machine Intellegence} 32(9):1582--1596.
  \bibitem{5-zyk-1}
  \Aue{Konushin, A.\,C.} 2012. \textit{Vvedenie v komp'yuternoe zrenie}
[Introduction to computer vision]. Lektsiya~5 [Lecture~5]. Moscow: Lektorium.
Available at: {\sf www.lektorium. tv/lecture/13541?id= 13541} (accessed
June~27, 2014).

\end{thebibliography}

 }
 }

\end{multicols}

\vspace*{-6pt}

\hfill{\small\textit{Received June 24, 2014}}

\vspace*{-18pt}

\Contr

\noindent
  \textbf{Arkhipov Oleg P.} (b.\ 1948)~--- Candidate of Science (PhD) in
technology, Director, Orel Branch of Institute of Informatics Problems, Russian
Academy of Sciences, 137 Moskovskoe Highway, Orel 302025, Russian
Federation; arkhipov12@yandex.ru

\vspace*{3pt}

\noindent
\textbf{Zykova Zoya P.} (b.\ 1953)~--- Candidate of Science (PhD) in physics
and mathematics, Head of Laboratory, Oryol Branch of the
Institute of
Information Problems, Russian Academy of Sciences,
137 Moskovskoe Highway, Orel 302025, Russian Federation; zykzoya@yandex.ru




\label{end\stat}

\renewcommand{\bibname}{\protect\rm Литература}