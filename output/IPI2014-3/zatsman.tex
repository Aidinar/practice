\def\stat{minin}

\def\tit{ИНДИКАТОРЫ ТЕМАТИЧЕСКИХ ВЗАИМОСВЯЗЕЙ НАУКИ И~ТЕХНОЛОГИЙ: ОТ~ТЕКСТА К~ЧИСЛАМ$^*$}

\def\titkol{Индикаторы тематических взаимосвязей науки и~технологий: от~текста к~числам}

\def\aut{В.\,А.~Минин$^1$, И.\,М.~Зацман$^2$, В.\,А.~Хавансков$^3$, С.\,К.~Шубников$^4$}

\def\autkol{В.\,А.~Минин, И.\,М.~Зацман, В.\,А.~Хавансков, С.\,К.~Шубников}

\titel{\tit}{\aut}{\autkol}{\titkol}

{\renewcommand{\thefootnote}{\fnsymbol{footnote}}
\footnotetext[1]{Работа выполнена в ИПИ РАН при частичной поддержке
РГНФ (грант №\,12-02-12019в).}}

\renewcommand{\thefootnote}{\arabic{footnote}}
\footnotetext[1]{Российский фонд фундаментальных исследований, minin@rfbr.ru}
\footnotetext[2]{Институт проблем
информатики Российской академии наук, iz\_ipi@a170.ipi.ac.ru}
\footnotetext[3]{Институт проблем
информатики Российской академии наук, havanskov@a170.ipi.ac.ru}
\footnotetext[4]{Институт проблем
информатики Российской академии наук, sergeysh50@yandex.ru}


      \Abst{Дано описание основных этапов вычисления индикаторов тематических взаимосвязей науки и технологий. Исходные данные
для проведения расчетов представляют собой результаты обработки полных текстов описаний изобретений на естественном языке. Целью
обработки является извлечение из полных текстов информации о научных публикациях, цитируемых в описаниях изобретений.
Со\-по\-став\-ле\-ние тематики этой информации с индексами Международной патентной
классификации (МПК) дает возможность экспертам выявлять
взаимосвязи результатов направлений научных исследований (ННИ)
с развитием технологий и оценивать их с помощью количественных
индикаторов. В~статье исследуются те технологические стадии определения индикаторов тематических взаимосвязей, на которых
осуществляется переход от обработки текста к численным расчетам. Дано описание разработанных методов извлечения информации о
научных публикациях из полных текстов изобретений и расчета значений индикаторов. Их применение дало возможность определить
индикаторы взаимосвязей ННИ с информационными технологиями, запатентованными в РФ за период с 2000
по 2012~гг. Для отечественной на\-уч\-но-тех\-ни\-че\-ской сферы значения этих индикаторов вычислены впервые.}

      \KW{взаимосвязи науки и технологий; методология определения индикаторов взаимосвязей; информационные технологии;
обработка текста; расчет значений индикаторов}

\DOI{10.14357/19922264140313}

\vskip 12pt plus 9pt minus 6pt

      \thispagestyle{headings}

      \begin{multicols}{2}

            \label{st\stat}

\section{Введение}

     В 2012--2014~гг.\ в Институте проблем информатики РАН был создан действующий макет
     Ана\-ли\-ти\-ко-ин\-фор\-ма\-ци\-он\-ной сис\-те\-мы (АИС), позволяющий
     выявлять тематические взаимосвязи ННИ с заданным видом патентуемых технологий
     и оце\-нивать их с  использованием количественных
ин\-дикаторов. Основная цель статьи заключается в описании результатов,
полученных с использованием действующего макета
АИС, и тех технологических стадий вычисления индикаторов, с по\-мощью которых осуществляется переход от
обработки текс\-то\-вых описаний запатентованных изобретений к численным расчетам значений индикаторов.

     Методология вычисления индикаторов тематических взаимосвязей науки и технологий изложена в работе~[1], в
которой приводится обзор зарубежных работ по исследованию взаимосвязей науки и технологий~[2--11]. На основе этой
методологии были разработаны архитектурные решения АИС для вычисления индикаторов. Предложенные решения
стали основой создания макета АИС, которая не имеет аналогов в российской на\-уч\-но-тех\-ни\-че\-ской сфере.
Создание таких систем необходимо для мониторинга и оценивания программ научных исследований и принятия
решений на всех этапах программной
     на\-уч\-но-тех\-ни\-че\-ской деятельности.

     В процессе создания макета был разработан метод извлечения из полных текстов изобретений ссылок на
цитируемые научные публикации, которые и дают возможность определять тематические взаимосвязи науки и
технологий. Разработанный метод извлечения учитывает тот факт, что ссылки могут размещаться не только в списке в
конце описания изобретения, но и внутри текстовых абзацев описания на естественном языке~\cite{12-zat}.

     С помощью созданного макета был проведен эксперимент, в рамках которого было обработано более
     6000~пол\-ных текстов изобретений по информационным технологиям, запатентованных по индексам, входящим в
     \label{snoska}
класс\footnote[5]{Все индексы МПК
образуют иерархическую структуру и делятся на 5~категорий: разделы (верхний уровень иерархии),
классы, подклассы, группы и подгруппы (нижний уровень иерархии).} G06 (Обработка данных; вычисление; счет)
МПК, и опубликованных Федеральной службой по интеллектуальной
соб\-ствен\-ности (Роспатент) за период с 2000 по 2012~гг.

     В текстовых описаниях отобранных изобретений с помощью макета были найдены ссылки на научные
публикации, библиографические данные которых позволили соотнести их по тематике с
теми или иными рубриками
ННИ. В~статье приводятся результаты вычисления двух индикаторов. Были вычислены интенсивности тематических
взаимосвязей патентуемых информационных технологий с рубриками ННИ Государственного рубрикатора
на\-уч\-но-тех\-ни\-че\-ской информации (ГРНТИ) и распределение периода времени между цитируемыми научными публикациями и выдачей
патентов на те изобретения, в которых они цитируются.

\section{Операции обработки текстов изобретений}

     Архитектура АИС подробно изложена в работе~\cite{13-zat}, в которой перечислены ее входные и выходные
информационные ресурсы, а также структурированные наборы данных, объединенные в
базу данных (БД) АИС. В~этой же работе
приведена структурная схема АИС и входящие в нее подсистемы, которые взаимодействуют между собой через БД
АИС.

     Технология обработки текстов изобретений, поступающих на вход АИС, определяется с помощью сценария, в
котором задаются:
     \begin{itemize}
\item вид технологий, взаимосвязи которого с ННИ исследуются с помощью этого сценария, например информационные технологии;
\item перечень индексов МПК, которые относятся к заданному виду технологий, например все индексы класса G06, относящиеся к
информационным технологиям;
\item временной интервал, на котором исследуются взаимосвязи заданного вида технологий с ННИ, например 2000--2012~гг.;
\item поисковый запрос, с помощью которого с оптических носителей Роспатента будут
отобраны\linebreak изобретения заданного в сценарии
вида тех\-нологий за выбранный период времени и сфор\-мирован массив рефератов
изобретений как \mbox{входных} информационных
ресурсов АИС с адресами полных описаний этих изобретений на сайте Роспатента;
\item перечень тех индикаторов, значения которых должны быть вычислены для заданного вида технологий;
\item автор(ы) сценария;
\item имя сценария.
\end{itemize}

     Приведенный список свидетельствует о том, что сценарий задает состав информационных ресурсов, обрабатываемых на
технологических операциях, и перечень вычисляемых индикаторов. Основные технологические операции,
реализованные в макете АИС, представлены на рис.~1.


     Опишем их, обратив основное внимание на те стадии вычисления индикаторов тематических взаимосвязей, где
осуществляется переход от обработки текста изобретений к численным расчетам значений индикаторов.

     Как следует из приведенного выше списка, сценарий включает перечень тех индикаторов, значения которых
должны быть вычислены в рамках сформированного сценария. Индикаторы выбираются из списка, содержание
которого определяется спектром доступных ресурсов.

     Формирование массива данных, необходимого для проведения расчетов,
     осуществляется под\-сис\-те\-мой
<<Импорт>>, исходя из заданного сценария исследования. Сначала с оптических дисков Роспатента копируются
библиографические описания изобретений, отобранные на основе критериев, которые заложены в сценарий. Для этого
используется поисковый запрос из сценария.

     Для отбора патентов по этому запросу используется
     ин\-фор\-ма\-ци\-он\-но-по\-иско\-вая система (ИПС) MIMOSA, разработанная в Федеральном институте
промышленной собственности (ФИПС). Ин\-фор\-ма\-ци\-он\-но-поиско\-вая
система MIMOSA поставляется вместе с библиографическими описаниями
изобретений на оптических ком\-пакт-дис\-ках, которые распространяются Роспатентом. Для поиска документов с
помощью ИПС MIMOSA существует специально разработанный язык запросов, на котором написан поисковый запрос в
сценарии. Ниже приводится запрос, который использовался в процессе проведения эксперимента:

     \noindent
     {\sf (ICA=G06* OR ICAA=G06* OR ICAI=G06*) AND DP=2000* AND \ldots AND DP=2012*,}

\noindent
где символ <<*>> служит знакозаменителем, а многоточие обозначает пропущенные для краткости годы от 2001 до 2011
и операторы AND.

     В этом запросе отбираются все изобретения, содержащие как основные, так и дополнительные индексы МПК,
входящие в класс G06 <<Обработка данных; вычисление; счет>> МПК. При записи в БД АИС сохраняется
информация о том, какой индекс помечен в изобретении как основной.

     Обращение к ИПС MIMOSA происходит при выполнении каждого сценария. На первом его шаге определяется
список номеров патентов, удовле\-тво\-ря\-ющих поисковому запросу, и формируется структурированный текстовый файл
библиографических описаний изобретений, который пополня-\linebreak\vspace*{-12pt}

\pagebreak

\end{multicols}

\begin{figure*} %fig1
\vspace*{1pt}
\begin{center}
\mbox{%
\epsfxsize=110.708mm
\epsfbox{zac-1.eps}
}
\end{center}
\vspace*{-9pt}
\Caption{Основные технологические операции: от текста до значений индикаторов}
\end{figure*}

\begin{multicols}{2}

\noindent
ет БД АИС (рис.~2). В~правой час\-ти рис.~2 номер патента RU 02438579
является ссылкой на полное описание изобретения на сайте Роспатента.


     В процессе проведения эксперимента по этому запросу было отобрано 6666~изобретений, распределение которых
по годам показано на диаграмме (рис.~3). Из них 5243 имели индексы класса G06 в качестве основного.

     Из диаграммы видно, что за период 2000--2012~гг.\ число выданных в РФ патентов по информационным
технологиям выросло более чем в три раза: с~286 до~903.

     Аналитико-информационная сис\-те\-ма, используя ссылки из всех отобранных биб\-ли\-о\-гра\-фи\-че\-ских описаний, копирует в свою БД полные описания
изобретений. После завершения их копирования, выполняются две технологические операции (см.\ рис.~1):
     \begin{enumerate}[(1)]
\item извлечение списка индексов МПК в каждом библиографическом описании изобретения заданного вида технологий;
\item извлечение списка рубрик ННИ журналов, \mbox{статьи} которых цитируются в полнотекстовом описании изобретения заданного вида
технологий.
\end{enumerate}
     Эти операции не зависят друг от друга и могут выполняться в любой последовательности.



     Первая операция выполняется программно с использованием алгоритма, разработанного согласно правилам
написания индексов МПК на патентных документах и их записи в отдельном
структури-\linebreak\vspace*{-12pt}

\pagebreak

\end{multicols}

\noindent
\begin{figure} %fig2
\vspace*{1pt}
\begin{center}
\mbox{%
\epsfxsize=160mm
\epsfbox{zac-2.eps}
}
\end{center}
\vspace*{-9pt}
      \Caption{Список патентов, полученных в результате выполнения запроса}
%      \end{figure}
%      \begin{figure} %fig3
\vspace*{9pt}
\begin{center}
\mbox{%
\epsfxsize=120.51mm
\epsfbox{zac-3.eps}
}
\end{center}
\vspace*{-9pt}
\Caption{Распределение отобранных изобретений по годам}
\vspace*{-18pt}
      \end{figure}

\begin{multicols}{2}

\noindent
рованном
     поле~\cite{15-zat, 14-zat}. Таблица~1 показывает в третьем столбце распределение по 11~подклассам класса G06
отобранных по основному индексу 5243~изобретений.

\begin{table*}\small
\begin{center}
\Caption{Распределение изобретений и цитируемых статей по основному индексу МПК}
      \vspace*{2ex}

      \begin{tabular}{|l|p{63mm}|c|c|c|}
      \hline
\multicolumn{1}{|c|}{\tabcolsep=0pt\begin{tabular}{c}Код\\ подкласса\end{tabular}}&
\multicolumn{1}{c|}{Название подкласса}&
\tabcolsep=0pt\begin{tabular}{c}Число\\ изобретений\end{tabular}&
\tabcolsep=0pt\begin{tabular}{c}Число\\ статей\end{tabular}&
\tabcolsep=0pt\begin{tabular}{c}Статей\\ на 1 изобретение\end{tabular}\\
\hline
\multicolumn{1}{|l|}{\raisebox{-6pt}[0pt][0pt]{\ \ \ \ \ \ \ \ \ \ \ G06C}}&
Механические цифровые вычислительные машины&
\multicolumn{1}{c|}{\raisebox{-6pt}[0pt][0pt]{\hphantom{99}7}}&
\multicolumn{1}{c|}{\raisebox{-6pt}[0pt][0pt]{0}}&
\multicolumn{1}{c|}{\raisebox{-6pt}[0pt][0pt]{0\hphantom{,00}}}\\
\hline
\multicolumn{1}{|l|}{\raisebox{-6pt}[0pt][0pt]{\ \ \ \ \ \ \ \ \ \ \ G06D}}&Гидравлические и пневматические цифровые вычислительные
устройства&\multicolumn{1}{c|}{\raisebox{-6pt}[0pt][0pt]{\hphantom{99}1}}&
\multicolumn{1}{c|}{\raisebox{-6pt}[0pt][0pt]{0}}&\multicolumn{1}{c|}{\raisebox{-6pt}[0pt][0pt]{0\hphantom{,00}}}\\
\hline
\multicolumn{1}{|l|}{\raisebox{-0pt}[0pt][0pt]{\ \ \ \ \ \ \ \ \ \ \ G06E}}&Оптические вычислительные устройства&\hphantom{9}52&8&0,15\\
\hline
\multicolumn{1}{|l|}{\raisebox{-6pt}[0pt][0pt]{\ \ \ \ \ \ \ \ \ \ \ G06F}}&
Обработка цифровых данных с помощью электрических устройств&
\multicolumn{1}{c|}{\raisebox{-6pt}[0pt][0pt]{3415\hphantom{9}}}&
\multicolumn{1}{c|}{\raisebox{-6pt}[0pt][0pt]{107\hphantom{99}}}&
\multicolumn{1}{c|}{\raisebox{-6pt}[0pt][0pt]{0,03}}\\
\hline
\multicolumn{1}{|l|}{\ \ \ \ \ \ \ \ \ \ \ G06G}&Аналоговые вычислительные машины $\ldots$ &228&14\hphantom{9}&0,06\\
\hline
\multicolumn{1}{|l|}{\ \ \ \ \ \ \ \ \ \ \ G06J}&Гибридные вычислительные устройства&\hphantom{99}3&0&0\hphantom{,00}\\
\hline
\multicolumn{1}{|l|}{\raisebox{-12pt}[0pt][0pt]{\ \ \ \ \ \ \ \ \ \ \ G06K}}&
Распознавание, представление и воспроизведение данных;
манипулирование носителями информации; носители информации&
\multicolumn{1}{c|}{\raisebox{-12pt}[0pt][0pt]{681}}&
\multicolumn{1}{c|}{\raisebox{-12pt}[0pt][0pt]{64\hphantom{9}}}&
\multicolumn{1}{c|}{\raisebox{-12pt}[0pt][0pt]{0,09}}\\
\hline
\multicolumn{1}{|l|}{\raisebox{-12pt}[0pt][0pt]{\ \ \ \ \ \ \ \ \ \ \ G06M}}&
Счетчики; способы и устройства для подсчета предметов, не отнесенные
к другим подклассам&\multicolumn{1}{c|}{\raisebox{-12pt}[0pt][0pt]{\hphantom{9}12}}&
\multicolumn{1}{c|}{\raisebox{-12pt}[0pt][0pt]{0}}&
\multicolumn{1}{c|}{\raisebox{-12pt}[0pt][0pt]{0\hphantom{,00}}}\\
\hline
\multicolumn{1}{|l|}{\raisebox{-6pt}[0pt][0pt]{\ \ \ \ \ \ \ \ \ \ \ G06N}}&
Компьютерные системы, основанные на специфических вычислительных моделях&
\multicolumn{1}{c|}{\raisebox{-6pt}[0pt][0pt]{107}}&
\multicolumn{1}{c|}{\raisebox{-6pt}[0pt][0pt]{8}}&
\multicolumn{1}{c|}{\raisebox{-6pt}[0pt][0pt]{0,07}}\\
\hline
\multicolumn{1}{|l|}{\raisebox{-12pt}[0pt][0pt]{\ \ \ \ \ \ \ \ \ \ \ G06Q}}&Системы обработки данных или способы,
специально предназначенные для административных, коммерческих $<\ldots>$ целей&
\multicolumn{1}{c|}{\raisebox{-12pt}[0pt][0pt]{417}}&
\multicolumn{1}{c|}{\raisebox{-12pt}[0pt][0pt]{5}}&
\multicolumn{1}{c|}{\raisebox{-12pt}[0pt][0pt]{0,01}}\\
\hline
\multicolumn{1}{|l|}{\raisebox{-6pt}[0pt][0pt]{\ \ \ \ \ \ \ \ \ \ \ G06T}}&
Обработка или генерация данных изображения&
\multicolumn{1}{c|}{\raisebox{-6pt}[0pt][0pt]{320}}&
\multicolumn{1}{c|}{\raisebox{-6pt}[0pt][0pt]{43\hphantom{9}}}&
\multicolumn{1}{c|}{\raisebox{-6pt}[0pt][0pt]{0,13}}\\
\hline
Всего по классу G06&&5243\hphantom{9}&249\hphantom{9}&0,05\\
     \hline
     \end{tabular}
     \end{center}
     \end{table*}

     В процессе выполнения второй операции в текстах изобретений производится поиск и структурирование ссылок
на научные публикации, цити\-ру\-емые в изобретениях. По окончании обеих операций завершается текстовая обработка и
происходит переход к численным расчетам.

     Задача поиска ссылок была подробно описана в работе~\cite{16-zat}, в которой было показано, что сложность ее
решения является следствием несоблюдения требований стандартов Всемирной организации интеллектуальной
собственности (\mbox{ВОИС}) к оформлению ссылок на цитируемые публикации~\cite{17-zat, 18-zat}.

     В отличие от принятой в научной сфере и нормативно закрепленной практики размещения списка используемой
литературы в конце научной статьи или ее страниц, в описаниях изобретений ссылки на цитируемую публикацию могут
встретиться в любом месте описания, включая любой его абзац. Поэтому известные методы извлечение
биб\-лио\-гра\-фи\-че\-ских ссылок из текстов статей~\cite{19-zat, 20-zat}, по оценке авторов, применимы только к
     10\%--15\% изобретений.

     Для поиска ссылок на научные публикации в текстах изобретений и их структурирования был разработан метод,
подробно описанный в
     работе~\cite{12-zat}. Этот метод основан на следующей формализованной схеме описания ссылки как объекта ее
поиска в тексте изобретения:
\begin{multline*}
[\mbox{\textit{автор}}\{S_1\}] [\mbox{\textit{название публикации}}] \\
[\{S_2\}\mbox{\textit{название источника}}]\\
\{S_3\}\mbox{\textit{атрибуты публикации}}\,.
\end{multline*}

     Наличие квадратных скобок говорит о необязательности присутствия данного
     элемента схемы в реальной ссылке;
$\{S_i\}$ обозначает множество возможных зна\-ков-раз\-де\-ли\-те\-лей, которые стоят между элементами схемы.
В~отличие от биб\-ли\-о\-гра\-фи\-че\-ских элементов [\textit{название публикации}] и [\{$S_2$\}\textit{название источника}],
которые могут рас\-смат\-ри\-вать\-ся как простые множества слов и знаков пунктуации, элементы [\textit{автор}\{$S_1$\}] и
\textit{атрибуты публикации} имеют свои собственные структурные особенности~\cite{12-zat}.

     Разработанный метод был использован при поиске ссылок в массиве отобранных изобретений по классу G06 за
период 2000--2012~гг. В~результате из этого массива было выделено 2758~патентов, в которых найдены фрагменты
текста, соответствующие шаблонам поиска, построенным на основе приведенной выше схемы описания ссылок.

     После анализа выделенных фрагментов текста обнаружилось около 30\% случаев, когда используемые шаблоны
выделили фрагменты текста описаний, не являющиеся ссылками на цитируемую публикацию, но отвечающие
критериям поиска. Данное обстоятельство не влияет на конечный результат, так как при дальнейшей структуризации
такие фрагменты текста исключаются из рас\-смот\-рения.

     В то же время среди полнотекстовых описаний, в которых программой не обнаружено ссылок на цитируемые
публикации, при визуальном просмотре были найдены пропущенные программой ссылки на цитируемые публикации,
что говорит о необходимости дальнейшего развития системы шаблонов, используемых для поиска ссылок. Процедура
добавления и редактирования шаблонов уже разработана и используется в макете АИС.

     Из общего массива найденных ссылок на цитируемые публикации в описываемом эксперименте были отобраны
только те, которые относятся к \mbox{статьям} в журналах или в трудах конференций. Само рубрицирование выполняется в
несколько этапов:
     \begin{enumerate}[(1)]
\item выделение в ссылке на статью названия журнала или трудов конференции;
\item поиск названия в нормализованной БД;
\item присвоение ссылке на цитируемую статью кодов рубрик ННИ соответствующего журнала или трудов
конференции.
     \end{enumerate}

     Именно после выделения названия проходит граница между обработкой неструктурированных текстов
изобретений и операциями подготовки данных для расчетов значений индикаторов.

\section{От анализа текста изобретений к~численным расчетам}

     На предыдущих этапах, предшествующих поиску названий журналов или трудов конференций в нормализованной
БД, использовались методы анализа текстов изобретений. Эти тексты описывают технические решения
изобретений и могут включать ссылки на цитируемые статьи. После этапа поиска названий в процессе вычисления
значений индикаторов в макете АИС и в аналогичных зарубежных системах~\cite{6-zat, 11-zat, 21-zat}
используются не тексты ссылок, а коды рубрик ННИ, присвоенных тому изданию, в котором опубликована цитируемая статья. Использование
этих кодов является необходимым условием проведения расчетов значений индикаторов. Отметим, что
полнофункциональный вариант АИС будет отличаться от разработанного макета тем, что будут использоваться не все
коды рубрик ННИ, присвоенные изданию, а только те, которые соответствуют тематике цитируемой статьи.

     Это отличие является принципиальным. В~макете и в аналогичных зарубежных системах для вычисления
индикаторов используются все коды руб\-рик ННИ издания, среди которых могут оказаться те, которые не соответствуют
тематике цитируемой статьи, что может повлиять на точность вычисления значений индикаторов тематических
взаимосвязей заданного вида технологий с ННИ. Степень этого влияния можно будет оценить только после создания
полнофункционального варианта АИС.

     Для выделения названия издания, в котором опубликована статья, используются шаблоны описания разделителей
между библиографическими элементами структуры описания ссылки на цитируемую публикацию, а именно $\{S_2\}$
(разделитель~--- издание) и $\{S_3\}$ (разделитель~--- атрибуты). Если использовать действующий государственный
стандарт <<Библиографическая запись. Библиографическое описание>>~\cite{22-zat}, то разделителем перед названием
издания является двойной слеш <<//>>. Как показал эксперимент, двойной слеш используется только в 60\% ссылок.
Поэтому в список разделителей $\{S_2\}$ были добавлены еще несколько признаков кроме двойного слеша.

     После выделения издания осуществляется его поиск в нормализованной базе названий журналов и трудов
конференций. В~ней кроме полного официального названия журнала или трудов
конференции указываются и его
варианты, в частности сокращенные (так называемые псевдонимы). Именно по псевдонимам, как правило, ведется поиск
издания после выделения его названия на предыдущем этапе, так как официальные названия в ссылках на цитируемые
статьи встречаются редко. Если издание найдено, то его коды ННИ присваиваются цитируемой статье.

     В случае отсутствия искомого названия в БД в работу включается оператор, задачей которого является
привязка искомого названия к существующему периодическому изданию и до\-бав\-ле\-ние его в список псевдонимов. Если
это издание отсутствует, то оператор вводит в БД новое официальное название издания вместе с перечнем его
кодов ННИ, которые автоматически присваиваются цитируемой статье.

     В результате обработки массива полнотекстовых описаний изобретений были автоматически идентифицированы
512~выделенных фрагментов текста как ссылки на статьи в журналах или трудах конференций. Отметим, что в
нормализованном списке названий журналов и трудов макета АИС не все они имеют коды ННИ. Поэтому из
512~выделенных ссылок было зарубрицировано 249~цитируемых статей. Таблица~1 показывает в четвертом столбце
распределение зарубрицированных статей по 11~подклассам класса G06.



     В пятом столбце табл.~1 приведены относительные данные распределения зарубрицированных статей. Первые три
позиции по интенсивности цитирования научных статей занимают <<Оптические вычислительные устройства>>,
<<Обработка или генерация данных изображения>>, <<Распознавание, представление и воспроизведение данных>>.
Аналогичные зарубежные данные по интенсивности цитирования приведены в работах~\cite{1-zat, 4-zat}.

\section{Расчет значений индикаторов}

     В проведенном эксперименте вычислялись значения двух индикаторов:
     \begin{enumerate}[(1)]
\item матрицы корреляций между индексами МПК и рубриками ННИ по ГРНТИ;
\item распределения времени отклика на статью (от момента ее публикации до момента публикации патента на
изобретение, где она цитируется).
\end{enumerate}

     Вычисление первого индикатора как матрицы коэффициентов корреляции сводится к подсчету числа связей
между индексами МПК и рубриками ГРНТИ, т.\,е.\ частоты встречаемости в массиве изобретений пар <<индекс МПК
изобре\-те\-ния\,--\,руб\-ри\-ка ГРНТИ цитируемой статьи>>. Эта частотность может быть вычислена с помощью разных
алгоритмов вычисления частоты, каждому из которых соответствует свой индикатор.

     Для описания используемого в проведенном эксперименте индикатора потребуется дополнительная информация
о (1)~уровне и (2)~значении классификации, а также о (3)~позициях индексов МПК~\cite{15-zat}, которая кодируется в
отдельных полях для каждого индекса латинскими литерами следующим образом.

     Во-первых, изобретения могут классифицироваться индексами по трем уровням: или только по подклассу, или
только по основным группам, или по полному тексту МПК, иерархические структурные
уровни которой перечислены в сноске~5
на c.~\pageref{snoska},
что помечается литерой~$S$, $C$ или~$A$ соответственно.
В~эксперименте не проводились различия между этими
уровнями. Во-вто\-рых, индексами могут классифицироваться сами изобретения или дополнительная информация к ним
(это помечается литерой~$I$ или~$N$ соответственно), что также
не различалось при расчете значений индикаторов.
Кодирование с помощью этих двух литер в патентном деле называется определением <<значения классификации>>.
     В-третьих, первый по расположению в изобретении индекс считается основным. Поэтому разные индикаторы
соответствуют алгоритмам расчета их значений по основному и дополнительным (неосновным) индексам, что
помечается литерой~$F$ и~$L$ соответственно. Возможен также расчет значений по их совокупности, который и
использовался в проведенном эксперименте. Однако выше (см.\ табл.~1) для расчета распределения изобретений и
цитируемых статей использовался только основной индекс.

     Таким образом, из всего спектра возможных индикаторов был выбран только один,
     алгоритм которого не учитывал литеры уровня и значения классификации. Позиции индексов МПК фиксировались, но не учитывались
при вычислении значений этого индикатора. Иначе говоря, используемый в эксперименте индикатор характеризует
связи между всеми индексами МПК и рубриками ГРНТИ, т.\,е.\ если $n$~--- число индексов некоторого изобретения с
литерами~$S$, $C$, $A$, $I$, $N$, $F$ и~$L$, а~$m_j$~--- число рубрик ГРНТИ,
присвоенных $j$-й \mbox{статье}, цитируемой
в этом изобретении, то число его связей~$N$, образуемых этими индексами и рубриками,\linebreak
равно
     $$
     N=n\sum\limits_j m_j\,.
     $$

     Частотности связей между индексами МПК и рубриками ГРНТИ, вычисленные для
     всего массива изобретений по
информационным технологиям, запатентованных в РФ за 2000--2012~гг.,
показаны в ячейках мат\-ри\-цы (табл.~2).
\begin{table*}\small
\begin{center}
\Caption{Частотности связей между индексами МПК и рубриками ГРНТИ (\%)}
      \vspace*{2ex}

      \begin{tabular}{|c|l|l|c|l|c|l|l|l|l|r|}
      \hline
\tabcolsep=0pt\begin{tabular}{c}Код\\ рубрики\\ ГРНТИ\end{tabular}&
\multicolumn{1}{c|}{Название рубрики}&
\multicolumn{1}{c|}{G06E}&G06F&
\multicolumn{1}{c|}{G06G}&G06K&
\multicolumn{1}{c|}{G06M}&
\multicolumn{1}{c|}{G06N}&
\multicolumn{1}{c|}{G06Q}&
\multicolumn{1}{c|}{G06T}&
\multicolumn{1}{c|}{G06}\\
\hline
50.00.00&\tabcolsep=0pt\begin{tabular}{l}АВТОМАТИКА.\\ ВЫЧИСЛИТЕЛЬНАЯ\\ ТЕХНИКА\end{tabular}&0&9,58&0,27&4,85&0&0,62&1,01&2,05&18,38\\
\hline
28.00.00&КИБЕРНЕТИКА&0&8,40&0,11&4,10&0&0,59&1,05&1,35&15,60\\
\hline
47.00.00&\tabcolsep=0pt\begin{tabular}{l}ЭЛЕКТРОНИКА.\\ РАДИОТЕХНИКА\end{tabular}&0,10&5,16&0,41&3,86&0&0,17&0&0,50&10,20\\
\hline
45.00.00&ЭЛЕКТРОТЕХНИКА&0&4,88&0,26&3,49&0&0,19&0&0,35&9,17\\
\hline
20.00.00&ИНФОРМАТИКА&0&4,41&0,02&3,25&0&0,12&0&0,05&7,85\\
\hline
30.00.00&МЕХАНИКА&0&4,19&0,00&3,23&0&0,11&0&0,04&7,57\\
\hline
29.00.00&ФИЗИКА&0&3,79&0,02&3,17&0&0,01&0&0&6,99\\
\hline
84.00.00&СТАНДАРТИЗАЦИЯ&0&3,50&0&3,11&0&0&0,01&0&6,62\\
\hline
27.00.00&МАТЕМАТИКА&0&2,08&0&0,71&0&0,52&1,01&1,30&5,62\\
\hline
&\tabcolsep=0pt\begin{tabular}{l}Остальные рубрики\\ ГРНТИ\end{tabular}&0,00&8,69&0,23&1,39&0,01&1,16&0,15&0,37&12,00\\
\hline
\end{tabular}
\end{center}
\end{table*}
     В ее столбцах перечислены коды класса G06 и 8 его подклассов, названия которых содержит табл.~1.
     В~строках
мат\-ри\-цы приведены первые 9~ННИ в классификации ГРНТИ в порядке убывания числа связей с индексами по всему
классу G06 (см.\ последний столбец табл.~2). Данные табл.~2 на первый взгляд выглядят парадоксально. С~одной
стороны, в описаниях технических решений широко используются математические формулы и методы, с другой
стороны, рубрика <<МАТЕМАТИКА>> оказалась на 9-м месте в табл.~2. Этот парадокс был отмечен более 10~лет
назад. Эксперты, которые изучали тексты изобретений и исследовали этот парадокс, отметили широко
распространенную практику использования математических формул и методов без цитирования в изобретениях
математических пуб\-ли\-ка\-ций~\cite{10-zat}.

     Есть еще одна причина, объясняющая этот парадокс. Согласно п.~5 ст.~1350 ГК РФ (часть четвертая)
изобретениями не являются научные теории и математические методы~\cite{17-zat}. Если бы такое ограничение
отсутствовало, то в ряде классов МПК появились бы <<математические>> изобретения со ссылками на статьи, что
привело бы к увеличению значений индикаторов интенсивностей
<<математических>> взаимосвязей по этим классам.

\begin{figure*}[b] %fig4
\vspace*{1pt}
\begin{center}
\mbox{%
\epsfxsize=120.51mm
\epsfbox{zac-4.eps}
}
\end{center}
\vspace*{-9pt}
\Caption{Распределение времени между публикацией статьи и выдачей патента
по классу G06: \textit{1}~--- по полным описаниям изобретений; \textit{2}~--- по спискам
документов, цитируемых в отчетах об информационном поиске}
      \end{figure*}


     Данные о частотности взаимосвязей между индексами МПК и рубриками ГРНТИ были
     вы\-чис\-ле\-ны для всех
статей, цитируемых в описаниях изобретений, независимо от того, кем цитируется статья: экспертами, что обозначается
в описании изобретения меткой~56, или авторами изобретений. Поэтому кроме вычисленного индикатора существуют
еще два, которые учитывают авторство ссылки на статью. Эти индикаторы в статье не рас\-смат\-ри\-ва\-ются.

В~рамках проведенного эксперимента авторство ссылок
учитывалось при вычислении значений другого индикатора~--- распределения времени отклика на статьи (рис.~4).
В~процессе вычисления этого индикатора для каждой пары <<индекс МПК\,--\,руб\-ри\-ка ГРНТИ>> было определено
время отклика как разность между годом публикации патента и годом публикации статьи. Отдельно отмечались статьи,
цитируемые экспертами в отчетах о патентном поиске. Затем было построено распределение времени отклика с учетом
авторства ссылок на статьи.

\vspace*{-6pt}

\section{Заключение}

     Разработанный макет АИС и технология его применения впервые в
     отечественной практике \mbox{дают} возможность
выявлять количественные взаимосвязи ННИ с заданным видом технологий. С~по\-мощью макета были вычислены
значения индикатора тематических взаимосвязей информационных технологий, относящихся к классу G06 МПК, с ННИ
в виде рубрик ГРНТИ.


Вычисленные значения показывают, что наиболее часто в изобретениях по информационным технологиям
цитируются статьи по автоматике, вычислительной технике, кибернетике,
электронике, радиотехнике, электротехнике и информатике (см.\ табл.~2). Таким
образом, использование ННИ в виде рубрик ГРНТИ дает в первую очередь
прикладной разрез тематических взаимосвязей. Поэтому для получения более
полной картины взаимосвязей научных дисциплин с технологиями необходимо
также использовать рубрики фундаментальных наук, например классификатор
РФФИ.

     В макете АИС предусмотрена возможность выбора между разными
рубрикаторами ННИ, что является его ключевым отличием от имеющихся
зарубежных аналогов, в которых используется только один рубрикатор без
возможности его замены~\cite{11-zat}. Однако при использовании любого
рубрикатора ННИ в процессе принятия решений полученные значения могут
использоваться только после их экспертной проверки, так как любые известные в
настоящее время индикаторы тематических взаимосвязей науки и технологий
являются косвенными. Отметим, что проведение экспертной верификации
выходит за рамки выполненного проекта по гранту РГНФ. Его главная цель
заключалась в создании действующего макета АИС и работоспособной
технологии.

     Кроме косвенного индикатора были определены значения одного прямого
индикатора~--- распределение времени отклика на статьи. Рисунок~4 содержит
экспериментальные данные, которые позволяют утверждать, что в период
     2000--2012~гг.\ эксперты в отчетах о поиске и авторы изобретений по
информационным технологиям наиболее часто цитировали статьи,
опубликованные за 10, 20 и 30~лет до выдачи патентов на эти изобретения.
{\looseness=1

}

     Разработанная технология включает операции обработки полных текстов
описаний, на которых осуществляется переход от текстов к численным расчетам
значений индикаторов. В~процессе перехода одновременно использовались
индексы МПК и рубрики ННИ. При использовании разных классификаторов ННИ
результаты численных расчетов значений индикаторов для одного и того же
массива изобретений будут отличаться. Это позволяет выявлять тематические
взаимосвязи как в прикладном, так и в фундаментальном разрезе. Однако если на
основе значений индикаторов (после их экспертной верификации) планируется
формировать и реализовывать стратегию в сфере науки, в том числе распределять
финансовые средства по ориентированным научным направлениям, то должен
использоваться тот вариант классификатора ННИ, который применяется для
стратегического планирования.
{ %\looseness=1

}

{\small\frenchspacing
 {%\baselineskip=10.8pt
 \addcontentsline{toc}{section}{References}
 \begin{thebibliography}{99}
\bibitem{1-zat}
\Au{Минин В.\,А., Зацман И.\,М., Кружков~М.\,Г., Но\-ре\-кян~Т.\,П.}
Методологические основы создания информационных сис\-тем для вычисления
индикаторов тематиче\-ских взаимосвязей науки и технологий~// Информатика и её
применения, 2013. Т.~7. Вып.~1. С.~70--81.

\bibitem{5-zat} %2
\Au{Narin F., Noma E.} Is technology becoming science?~// Scientometrics, 1985.
Vol.~7. No.\,3-6. P.~369--381.

\bibitem{7-zat} %3
\Au{Mansfield E.} Academic research and innovation~// Res. Policy, 1991.
Vol.~20. No.\,1. P.~1--12.

\bibitem{2-zat} %4
\Au{Schmoch U.} Tracing the knowledge transfer from science to technology as
reflected in patent indicators~// Scientometrics, 1993. Vol.~26. No.\,1. P.~193--211.

\bibitem{8-zat}  %5
\Au{Mansfield E.} Academic research underlying industrial innovations: Sources,
characteristics and financing~// Rev. Econ. Stat., 1995. Vol.~77.
No.\,1. P.~55--62.


\bibitem{6-zat} %6
\Au{Narin F., Olivastro D.} Linkage between patents and papers: An interim EPO/US
comparison~// Scientometrics, 1998. Vol.~41. No.\,1-2. P.~51--59.


\bibitem{9-zat} %7
\Au{Mansfield E.} Academic research and industrial innovation: An update of empirical
findings~// Res. Policy, 1998. Vol.~26. No.\,7-8. P.~773--776.

\bibitem{3-zat} %8
\Aue{Tijssen R.\,J.\,W., Buter R.\,K., Van~Leeuwen~Th.\,N.} Technological relevance of
science: An assessment of citation linkages between patents and research papers~//
Scientometrics, 2000. Vol.~47. No.\,2. P.~389--412.


\bibitem{11-zat} %9
\Au{Verbeek А., Debackere~K., Luwel~M., Andries~P., Zimmermann~E., Deleus~D.}
Linking science to technology: Using bibliographic references in patents to build
linkage schemes~// Scientometrics, 2002. Vol.~54. No.\,3. P.~399--420.

\bibitem{4-zat} %10
\Au{Van Looy B., Zimmermann~E., Veugelers~R., Verbeek~A., Mello~J.,
Debackere~K.} Do science--technology interactions pay on when developing
technology? An exploratory investigation of 10 science-intensive technology
domains~// Scientometrics, 2003. Vol.~57. No.\,3. P.~355--367.

\bibitem{10-zat} %11
European Commission. Third European Report on Science \& Technology
Indicators.~--- Luxembourg: Office for Official Publications of the European
Communities, 2003. 451~p.



\bibitem{12-zat}
\Au{Зацман И.\,М., Хавансков~В.\,А., Шубников~С.\,К.} Метод извлечения
библиографической информации из полнотекстовых описаний изобретений~//
Информатика и ее применения, 2013. Т.~7. Вып.~4. С.~52--65.
\bibitem{13-zat}
\Au{Минин В.\,А., Зацман И.\,М., Хавансков~В.\,А., Шубников~С.\,К.}
Архитектурные решения для сис\-тем вы\-чис\-ле\-ния индикаторов тематических
взаимосвязей науки и технологий~// Сис\-те\-мы и средства информатики, 2013.
Т.~23. №\,2. C.~260--283.



\bibitem{15-zat} %14
Стандарт ВОИС ST.8~// Справочник по информации и документации в области
промышленной собственности.~--- М.: Роспатент, 2011. С.~3.8.2--3.8.4. {\sf
http://www.rupto.ru/rupto/nfile/2381c8b7-1033-11e1-a520-9c8e9921fb2c/03\_08\_01.pdf}.

\bibitem{14-zat} %15
Написание классификационных индексов и индексов кодирования на патентных
документах.~--- М.: ФИПС, 2014. {\sf
http://www1.fips.ru/wps/wcm/\linebreak
connect/ content\_ru/ru/inform\_resources/international\_\linebreak classification/inventions/mpk\_begin/article\_12}.

\bibitem{16-zat}
\Au{Зацман И.\,М., Шубников С.\,К.} Принципы обработки
информационных ресурсов для оценки инновационного потенциала
направлений научных исследований~// Электронные библиотеки:
перспективные методы и технологии, электронные коллекции~---
RCDL'2007: Тр. IX Всеросс. научн. конф.~--- Переславль: Университет
города Переславля, 2007. С.~35--44.
\bibitem{17-zat}
Административный регламент исполнения Роспатентом приема заявок на
изобретение, их рас\-смот\-ре\-ния и экспертизы.~--- М.: ФИПС, 2008. {\sf
http://www1. fips.ru/wps/wcm/connect/content\_ru/ru/documents/ russian\_laws/order\_minobr/administrative\_regulations/ test\_8}.
\bibitem{18-zat}
Стандарт ВОИС ST.14~// Справочник по информации и документации в области
промышленной собственности.~--- М.: Роспатент, 2008.\linebreak {\sf
http://www.rupto.ru/rupto/nfile/52b8dfc1-1049-11e1-a520-9c8e9921fb2c/03\_14\_01.pdf}.
\bibitem{19-zat}
\Au{Васильев А., Козлов Д., Самусев~С., Шамина~О.} Извлечение
метаинформации и библиографических ссылок из текстов русскоязычных
научных статей~// Электронные библиотеки: перспективные методы и
технологии, электронные коллекции~--- RCDL'2007: Тр. IX Всеросс. научн.
конф.~--- Переславль: Университет города Переславля, 2007. С.~175--184.
\bibitem{20-zat}
\Au{Васильев А., Козлов Д., Самусев~С., Шамина~О.} Создание электронной
библиотеки русскоязычных научных статей~// Ин\-тер\-нет-ма\-те\-ма\-ти\-ка
2007: Сб. работ участников конкурса научных проектов по
информационному поиску.~--- Екатеринбург: Изд-во Уральского университета,
2007. С.~37--45.
\bibitem{21-zat}
\Au{Van Looy B., Hansen~W., Hollanders~H., Tijssen~R.} Using concordance tables to
disentangle performance dynamics of HT manufacturing industries: An empirical
assessment of national innovation systems~// 10th Conference (International) on
Science and Technology Indicators (STI'2008): Book of abstracts.~---
Vienna: ARC GmbH, 2008. P.~196--200.
\bibitem{22-zat}
ГОСТ 7.1-2003 Библиографическая запись. Библиографическое описание. Общие
требования и правила составления.~--- М.: Изд-во стандартов, 2004. 48~c.



\end{thebibliography}
} }

\end{multicols}

\vspace*{-6pt}

\hfill{\small\textit{Поступила в редакцию 17.06.14}}

%\newpage


\vspace*{12pt}

\hrule

\vspace*{2pt}

\hrule


\def\tit{INDICATORS OF THEMATIC SCIENCE--TECHNOLOGY
LINKAGES: FROM TEXT TO~NUMBERS}

\def\titkol{Indicators for thematic science--technology
linkages: From text to~numbers}

\def\aut{V.\,A.~Minin$^1$, I.\,M.~Zatsman$^2$, V.\,A.~Havanskov$^2$,
and~S.\,K.~Shubnikov$^2$}
\def\autkol{V.\,A.~Minin, I.\,M.~Zatsman, V.\,A.~Havanskov,
and~S.\,K.~Shubnikov}




\titel{\tit}{\aut}{\autkol}{\titkol}

\vspace*{-9pt}

\noindent
$^1$RFFI, 32A Leninsky Prosp., Moscow 119991, Russian Federation

\noindent
$^2$Institute of Informatics Problems, Russian Academy of Sciences,
44-2 Vavilov Str., Moscow 119333, Russian\\
$\hphantom{^1}$Federation



\def\leftfootline{\small{\textbf{\thepage}
\hfill INFORMATIKA I EE PRIMENENIYA~--- INFORMATICS AND APPLICATIONS\ \ \ 2014\ \ \ volume~8\ \ \ issue\ 2}
}%
 \def\rightfootline{\small{INFORMATIKA I EE PRIMENENIYA~--- INFORMATICS AND APPLICATIONS\ \ \ 2014\ \ \ volume~8\ \ \ issue\ 3
\hfill \textbf{\thepage}}}

\vspace*{6pt}



      \Abste{The article describes the principal stages of calculation of
      indicators for thematic science--technology linkages. Source data for carrying out calculations are results of processing of
full-text descriptions of inventions in natural language. The purpose of processing is extracting
information about scientific publications cited in full-text descriptions of inventions. Comparing this
information with indexes of the International Patent Classification helps experts to reveal
science--technology linkages and to estimate them by means of quantitative indicators. In the article,
the technological stages of calculating indicators for linkage assessment on which transition from text
processing to numerical calculations is carried out are investigated.
The article describes the techniques developed for extracting information about scientific publications from 
full-text descriptions of
inventions and calculating indicator values. These techniques helped to define indicators of
science--technology linkages for information technology research fields,
 which were patented in the Russian Federation in
2000--2012. For the domestic scientific and technical sphere, these indicators were calculated for the first
time.}

      \KWE{science--technology linkages; methodology for calculation of
      linkage indicators;
information technology; text processing; calculation of indicator values}

\DOI{10.14357/19922264140313}

\Ack
\noindent
The work was performed in the Institute of Informatics Proplems of
the Russian Academy of Sciences and was partially supported by the
Russian Foundation for Humanities (grant No.\,12-02-12019v).


\vspace*{9pt}

  \begin{multicols}{2}

\renewcommand{\bibname}{\protect\rmfamily References}
%\renewcommand{\bibname}{\large\protect\rm References}



{\small\frenchspacing
{%\baselineskip=10.8pt
\addcontentsline{toc}{section}{References}
\begin{thebibliography}{99}

\bibitem{1-zat-1}
\Aue{Minin, V.\,A., I.\,M. Zatsman, M.\,G.~Kruzhkov, and T.\,P.~Norekjan}. 2013.
Metodologicheskie osnovy so\-zda\-niya informatsionnykh sistem dlya vychisleniya
indikatorov tematicheskikh vzaimosvyazey nauki i tekhnologiy [Methodological basis
for the creation of information systems for the calculation of indicators of thematic
linkages between science and technologies]. \textit{Informatika i ee Primeneniya}~---
\textit{Inform. Appl.} 7(1):70--81.

\bibitem{5-zat-1} %2
\Aue{Narin, F., and E. Noma.} 1985. Is technology becoming science?
\textit{Scientometrics} 7(3-6):369--381.

\bibitem{7-zat-1} %3
\Aue{Mansfield, E.} 1991. Academic research and innovation. \textit{Res. Policy}
20(1):1--12.

\bibitem{2-zat-1} %4
\Aue{Schmoch, U.} 1993. Tracing the knowledge transfer from science to technology
as reflected in patent indicators. \textit{Scientometrics} 26(1):193--211.

\bibitem{8-zat-1} %5
\Aue{Mansfield, E.} 1995. Academic research underlying industrial innovations:
Sources, characteristics and financing. \textit{Rev. Econ. Stat.}
77(1):55--62.


\bibitem{6-zat-1}
\Aue{Narin, F., and D.~Olivastro}. 1998. Linkage between patents and papers: An
interim EPO/US comparison. \textit{Scientometrics} 41(1-2):51--59.

\bibitem{9-zat-1} %7
\Aue{Mansfield, E.} 1998. Academic research and industrial innovation: An update of
empirical findings. \textit{Res. Policy} 26(7-8):773--776.

\bibitem{3-zat-1} %8
\Aue{Tijssen, R.\,J.\,W., R.\,K. Buter, and Th.\,N.~Van Leeuwen}. 2000. Technological
relevance of science: An assessment of citation linkages between patents and research
papers. \textit{Scientometrics} 47(2):389--412.

\bibitem{11-zat-1} %9
\Aue{Verbeek, А., K.~Debackere, M.~Luwel, P.~Andries, E.~Zimmermann, and
D.~Deleus}. 2002. Linking science to technology: Using bibliographic references
in patents to build linkage schemes. \textit{Scientometrics} 54(3):399--420.


\bibitem{4-zat-1} %10
\Aue{Van Looy, B., E.~Zimmermann, R.~Veugelers, A.~Verbeek, J.~Mello, and
K.~Debackere}. 2003. Do science--technology interactions pay on when developing
technology? An exploratory investigation of 10 science-intensive technology domains.
\textit{Scientometrics} 57(3):355--367.
\bibitem{10-zat-1} %11
European Commission. 2003. Third European Report on Science \& Technology
Indicators. Luxembourg: Office for Official Publications of the European Communities.
451~p.

\bibitem{12-zat-1} %12
\Aue{Zatsman, I.\,M., V.\,A.~Havanskov, and S.\,K.~Shubnikov}. 2013. Metod
izvlecheniya bibliograficheskoy informatsii iz polnotekstovykh opisaniy izobreteniy
[Extraction method of bibliographic information for full-text descriptions of inventions].
\textit{Informatika i ee primeneniya}~--- \textit{Inform. Appl.} 7(4):52--65.
\bibitem{13-zat-1} %13
\Aue{Minin, V.\,A., I.\,M.~Zatsman, V.\,A.~Havanskov, and S.\,K.~Shubnikov}. 2013.
Arkhitekturnye resheniya dlya sis\-tem vychisleniya indikatorov tematicheskikh
vzaimo\-svya\-zey nauki i tekhnologiy [Information system conceptual decisions for
assessment of linkages between science and technologies]. \textit{Sistemy i Sredstva
Informatiki}~--- \textit{Systems and Means of Informatics} 23(2):260--283.



\bibitem{15-zat-1} %14
Standart VOIS ST.8 [WIPO standard ST.8]. Available at: {\sf
http://www.rupto.ru/rupto/nfile/2381c8b7-1033-11e1-a520-9c8e9921fb2c/03\_08\_01.pdf} (accessed June~23, 2014).

\bibitem{14-zat-1} %15
Napisanie klassifikatsionnykh indeksov i indeksov kodirovaniya na patentnykh
dokumentakh [Writing of classification indexes and coding indexes on patent
documents]. Available at: {\sf
http://www1.fips.ru/wps/ wcm/connect/content\_ru/ru/inform\_resources/\linebreak international\_classification/inventions/mpk\_begin/\linebreak article\_12} (accessed June~23, 2014).

\bibitem{16-zat-1}
\Aue{Zatsman, I.\,M., and S.\,K.~Shubnikov}. 2007. Printsipy obrabotki
informatsionnykh resursov dlya otsenki innova\-tsi\-on\-no\-go potentsiala napravleniy
nauchnykh issledovaniy [Processing principles of information resources for an
assessment of innovation potential of the scientific domains].
\textit{Tr. IX Vseross. Nauchn. Konf. ``Elektronnye Biblioteki''} [9th All-Russian
Scientific Conference on Digital Libraries Proceedings]. Pereslavl'. 35--44.
\bibitem{17-zat-1}
Administrativnyy reglament ispolneniya Rospatentom priema zayavok na izobretenie,
ikh rassmotreniya i ekspertizy [Rospatent administrative regulations for filing invention
applications, their considerations, and examination]. Available at: {\sf
http://www1.fips.ru/wps/ wcm/connect/content\_ru/ru/documents/russian\_laws/ order\_minobr/administrative\_regulations/test\_8/} (accessed June~23, 2014).
\bibitem{18-zat01}
Standart VOIS ST.14 [WIPO standard ST.14]. Available at: {\sf
http://www.rupto.ru/rupto/nfile/52b8dfc1-1049-11e1-a520-9c8e9921fb2c/03\_14\_01.pdf} (accessed June~23, 2014).
\bibitem{19-zat-1}
\Aue{Vasil'ev, A., D. Kozlov, S.~Samusev, and O.~Shamina}. 2007. Izvlechenie
metainformatsii i bibliograficheskikh ssylok iz tekstov russkoyazychnykh nauchnykh
statey [Extraction of metainformation and bibliographic references from texts of
Russian scientific articles]. \textit{Tr. IX Vseross. Nauchn. Konf. ``Elektronnye
Biblioteki''} [9th All-Russian Scientific Conference on Digital
Libraries Proceedings]. Pereslavl'. 175--184.
\bibitem{20-zat-1}
\Aue{Vasil'ev, A., D. Kozlov, S.~Samusev, and O.~Shamina}. 2007. Sozdanie
elektronnoy biblioteki russkoyazychnykh nauchnykh statey [Creation of digital library
of Russian scientific articles]. \textit{Sb. rabot stipendiatov granta
``Internet-matematika 2007'}' [Paper collection of scholars of a grant
``Internet-mathematics 2007'']. Ekaterinburg: Publishing house of Urals' University.
37--45.
\bibitem{21-zat-1}
\Aue{Van Looy, B., W.~Hansen, H.~Hollanders, and R.~Tijssen}. 2008. Using
concordance tables to disentangle perfor-\linebreak\vspace*{-12pt}

\pagebreak

\noindent
mance dynamics of HT manufacturing
industries: An
 empirical assessment of national innovation systems. \textit{10th
Conference (International)  on Science and Technology Indicators (STI'2008): Book of abstracts}. Vienna: ARC GmbH. 196--200.
\bibitem{22-zat-1}
GOST 7.1-2003. 2004. Bibliograficheskaya zapis'. Bibliograficheskoe opisanie.
Obshchie trebovaniya i pravila sostavleniya [Bibliographic record. Bibliographic
description. General requirements and drawing up rules]. Moscow: Standardinform
Publs. 48~p.

\end{thebibliography}
} }


\end{multicols}

\vspace*{-6pt}

\hfill{\small\textit{Received June 17, 2014}}

\vspace*{-18pt}

\Contr

\noindent
\textbf{Minin Vladimir A.} (b.\ 1941)~---
Doctor of Science in physics and mathematics, adviser,
Russian Foundation for Basic Research,  32A Leninsky Prosp., Moscow 119991,
Russian Federation; minin@rfbr.ru

\vspace*{3pt}

\noindent
\textbf{Zatsman Igor M.} (b.\ 1952)~---
Doctor of Science in technology, Head of Department, Institute of Informatics Problems,
Russian Academy of Sciences, 44-2 Vavilov Str., Moscow 119333, Russian
Federation; iz\_ipi@a170.ipi.ac.ru

\vspace*{3pt}

\noindent
\textbf{Havanskov Valerij A.} (b.\ 1950)~---
scientist, Institute of Informatics Problems,
Russian Academy of Sciences, 44-2 Vavilov Str., Moscow 119333, Russian
Federation; havanskov@a170.ipi.ac.ru

\vspace*{3pt}

\noindent
\textbf{Shubnikov Sergej K.} (b.\ 1955)~---
senior scientist, Institute of Informatics Problems, Russian Academy of Sciences,
44-2 Vavilov Str., Moscow 119333, Russian Federation;
 sergeysh50@yandex.ru




 \label{end\stat}

\renewcommand{\bibname}{\protect\rm Литература}