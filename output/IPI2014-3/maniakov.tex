\def\stat{maniakov}

\def\tit{ТЕКСТУРИРОВАНИЕ ВОКСЕЛЬНЫХ МОДЕЛЕЙ НА ОСНОВЕ
ЦВЕТОВОЙ ИНФОРМАЦИИ ОБ ОПОРНЫХ ТОЧКАХ}

\def\titkol{Текстурирование воксельных моделей на основе
цветовой информации об опорных точках}

\def\aut{О.\,П.~Архипов$^1$, Ю.\,А.~Маньяков$^2$}

\def\autkol{О.\,П.~Архипов, Ю.\,А.~Маньяков}

\titel{\tit}{\aut}{\autkol}{\titkol}

\renewcommand{\thefootnote}{\arabic{footnote}}
\footnotetext[1]{Орловский филиал Института проблем информатики Российской академии наук, arkhipov12@yandex.ru}
\footnotetext[2]{Орловский филиал Института проблем информатики Российской академии наук,
maniakov\_yuri@mail.ru}

\vspace*{3pt}


\Abst{Описана технология текстурирования воксельных моделей на основе
имеющейся цветовой информации об опорных точках (ОТ). Опорные точки представляют
собой особые точки в трехмерном пространстве, описывающие определенную часть
поверхности объекта и содержащие обобщенную информацию о цвете соответствующей
им области.
  Цвет воксела в области, ограниченной ОТ, определяется цветом каждой
из ОТ, ограничивающих эту область. Вес цветовой со\-став\-ля\-ющей каждой
ОТ при расчете цвета воксела обратно пропорционален евклидову расстоянию
от воксела до нее. Пред\-став\-лен механизм расчета весовых коэффициентов
цветовых компонент ОТ для вычисления цветов вокселов, зависящих от
рас\-сто\-яния до каждой ОТ, ограничивающей область. Приведены описание и
результаты экспериментов, в рамках которых осуществлялось текстурирование
синтетических моделей и моделей натурных объектов.}

\KW{трехмерная реконструкция; текстурирование; воксельная модель}

\DOI{10.14357/19922264140311}

\vspace*{16pt}

\vskip 14pt plus 9pt minus 6pt

\thispagestyle{headings}

\begin{multicols}{2}

\label{st\stat}



    В современных условиях роста популярности медиатехнологий и
востребованности компьютерной графики решение проблемы быстрого
создания трехмерных моделей натурных объектов и их анимации является
одной из наиболее актуальных задач. Трехмерная реконструкция является
одним из перспективных направлений в области компьютерной графики с
точки зрения повышения скорости и уменьшения сложности процесса
трехмерного моделирования.

   Результатом применения трехмерной реконструкции являются
трехмерные модели объектов, которые могут быть представлены в
различных форматах.
%
Одним из наиболее перспективных способов
представления трехмерных моделей является воксельное представление.
Воксельные модели, в частности, позволяют легко отображать деформации и
разрушения объектов, а также их внутреннюю структуру, что может быть
весьма полезно при создании анимации различных взаимодействий объектов.

  Одним из этапов трехмерной реконструкции является текстурирование.
В~случае с воксельными моделями этот термин подразумевает присвоение
цветов вокселам трехмерной модели. На сегодняшний день большинство
технологий текстурирования подразумевают достаточно большое количество
творческой, сложно автоматизируемой работы.

  Перспективной является технология, основанная на понятии
<<мегатекстура>>, которая позволяет работать с воксельными объектами и
используется в игровом графическом движке id~Tech~4 и id~Tech~5~[1].
Данная технология относится к методике распределения текстур, когда вся
поверхность трехмерной сцены покрывается одной большой текстурой с
заданным уровнем детализации вместо множества мелких. Однако такой
подход используется в основном для текстурирования достаточно прос\-тых
поверхностей, например ландшафтов, и не подходит для объектов со
сложной геометрией.

  Целью настоящей работы является разработка технологии
текстурирования воксельной модели, позволяющей на основе цветовой
информации, содержащейся в ОТ, производить перенос цветовой
информации с изображений натурного объекта на его воксельную модель.

  Опорную точку можно представить в виде кортежа
  $  \langle x, y, z, r, g, b\rangle,$
где ($x, y, z$)~--- координаты ОТ в трехмерном пространстве, $(r, g, b)$~---
цветовые координаты ОТ в пространстве RGB (red, green, blue).
{\looseness=1

}

  Опорные точки генерируются на основе смещенных изображений,
получаемых в результате круговой съемки объекта, что позволяет создать
замкнутое описание поверхности модели объекта в трехмерном
пространстве~[2]. При этом цветом ОТ является цвет соответствующего ей
сегмента, полученного в результате аппроксимации на основе
сгенерированной палитры различимых цветов.

  Технология построения воксельной модели~[3] подразумевает, что на
основе известных координат ОТ в трехмерном пространстве производится
интерполяция координат вокселов, заключенных между ОТ. Таким образом,
единичной областью интерполяции как декартовых координат вокселов
модели, так и их цветовых координат в пространстве RGB является область,
ограниченная пространственным треугольником, образованным тремя
взаимно ближайшими (ограничива\-ющи\-ми)~ОТ.

  Цвет воксела в области, ограниченной ОТ, определяется цветом каждой из
ОТ, ограничивающих эту область. Вес цветовой составляющей каждой ОТ
при расчете цвета воксела обратно пропорционален евклидову расстоянию от
воксела до нее. Таким образом, необходимо вычислить коэффициенты
пропорциональности цветовых значений каждого воксела расстоянию до
  $i$-й ОТ, ограничивающей область:
  $$ k_i=\fr{1}{dd_i}\,,\quad d=\sum\limits_{i=1}^3 \fr{1}{d_i}\,,
  $$
где $d_i$~--- евклидово расстояние от рассматриваемого воксела до $i$-й ОТ.

  Сумма произведений соответствующих коэффициентов и цветовых
составляющих каждой ограничивающей ОТ образует цвет каждого воксела.

  Таким образом, зависимость, описывающая способ вычисления цвета
воксела в области, ограниченной ОТ, может быть представлена в виде:
  $$
  f(c)=\sum\limits_{i=1}^3 c_i k_i\,,
  $$
где $c_i$~--- значение цвета $i$-й ограничивающей ОТ:
$$
c=\begin{bmatrix}
r\\[-2pt]
g\\
b
\end{bmatrix}\,.
$$

  В результате получим текстурированную воксельную модель, которую
можно формально представить в виде совокупности вокселов:
  $$
  V_T= \mathop{\bigcup}\limits_{i=1}^m v_i\,,
  $$
где $m$~--- количество вокселов модели;
$v$~--- воксел модели; представляющий собой структуру вида
$v = \langle x,y,z,r,g,b\rangle,$
в которой ($x, y, z$)~--- координаты воксела в трехмерном пространстве,
$(r, g, b)$~--- координаты цвета воксела в пространстве RGB.

{\small \begin{center}
\begin{tabular}{|c|r|r|c|r|r|r|}
\multicolumn{7}{c}{Координаты ОТ контрольной модели}\\[6pt]
\hline
№ ОТ&$X$&$Y$&$Z$&$R$&$G$&$B$\\
\hline
1&10&$-39$&0&255&0&0\\
2&$-10$&17&0&0&255&0\\
3&14&17&0&0&0&255\\
\hline
\end{tabular}
\end{center}
}
%\end{table*}

\vspace*{9pt}

  На основе представленной модели был создан алгоритм и реализующий
его программный\linebreak модуль, а также были проведены экспериментальные
исследования. Экспериментальные исследования про\-водились на
совокупности тестовых воксель\-ных моделей. В~ходе исследования
осуществлялось текстурирование данных моделей с использованием
информации об ОТ, на основе которых они были построены, в
соответствии с технологией пофрагментного анализа и представления
натурного объекта и изменения его пространственного положения с
последующей интеграцией полученных данных в трехмерные
изображения~[4].

  На начальном этапе с целью повышения точности проверки корректности
работы алгоритма использовалась контрольная воксельная модель,
представляющая собой пространственный треугольник с расположенными в
вершинах ОТ разных цветов, координаты которых представлены в таблице.


%\begin{table*}\small


  Результат текстурирования модели, пред\-став\-лен\-ный на рис.~1, показывает,
что цвета составля-\linebreak

\vspace*{3pt}

\noindent
\begin{center}  %fig1
\mbox{%
\epsfxsize=60mm %78mm
\epsfbox{man-1.eps}
}
  \end{center}

%  \vspace*{3pt}

\noindent
{{\figurename~1}\ \ \small{Визуализация текстурированной контрольной модели}}


%\vspace*{8pt}

\addtocounter{figure}{1}

\end{multicols}

\begin{figure} %fig2
\vspace*{1pt}
\begin{center}
\mbox{%
\epsfxsize=160mm
\epsfbox{man-2.eps}
}
\end{center}
\vspace*{-6pt}
\Caption{Тестовые воксельные модели}
%\end{figure*}
%\begin{figure*} %fig3
\vspace*{24pt}
\begin{center}
\mbox{%
\epsfxsize=160mm
\epsfbox{man-3.eps}
}
\end{center}
\vspace*{-9pt}
\Caption{Визуализация текстурированных тестовых воксельных моделей:
(\textit{а})~текстурирование синтетической воксельной модели;
(\textit{б})~текстурирование модели натурного объекта}
\end{figure}

\begin{multicols}{2}


\noindent
ющих ее вокселов образуют плавный градиентный
переход между цветами ОТ, расположенными в вершинах
треугольника.

  В связи с тем, что алгоритм текстурирования работает с совокупностью
пространственных треугольников, аппроксимирующих трехмерную \mbox{модель},
учитывая корректность текстурирования контрольной модели, можно
сделать вывод о корректности работы алгоритма и модуля текстурирования
воксельных моделей. Это подтверждают исследования с использованием
других тестовых моделей, представленных на рис.~2.



  В результате текстурирования рассмотренных моделей были получены
результаты, представленные на рис.~3.
  При этом стоит уточнить, что источником исходной информации для
генерации ОТ для тестовой модели натурного объекта (рис.~3,\,\textit{б}) в
соответствии с технологией представления натурного объекта и изменения
его пространственного положения~[4] являлись фотоснимки натурного
объекта, один из которых представлен на рис.~4.

	Таким образом, можно заключить, что рассмотренная технология
текстурирования позволяет с  необходимой точностью получить отображение\linebreak

\noindent
\begin{center}  %fig1
\mbox{%
\epsfxsize=78mm
\epsfbox{man-4.eps}
}
  \end{center}

%  \vspace*{3pt}

\noindent
{{\figurename~4}\ \ \small{Изображение исходного натурного объекта для создания тестовой модели
(см.\ рис.~3,\,\textit{б})}}


\vspace*{8pt}

%\addtocounter{figure}{1}


\noindent
цветов поверхности натурного объекта на воксельной модели при условии
ограниченного объема информации о цветах ОТ. Необходимая точность
текстурирования достигается путем варьирования уровня детализации
модели, который, в свою очередь, определяется количеством ОТ.

{\small\frenchspacing
 {%\baselineskip=10.8pt
 \addcontentsline{toc}{section}{References}
 \begin{thebibliography}{9}
\bibitem{1-man}
Использование мегатекстур (megatexture, clipmaps)~// Gamedev.ru. 9~ноября
2007. {\sf http://www.gamedev.ru/ code/articles/Megatexture}.
\bibitem{2-man}
\Au{Архипов О.\,П., Маньяков Ю.\,А., Сиротинин~Д.\,О.} Метод генерации
виртуальной сетки опорных точек на цветных изображениях~//
Информационные технологии в науке, образовании и производстве
(ИТНОП-2012): Мат-лы V Междунар. науч.-технич. конф.~--- Орел:
Госуниверситет-УНПК, 2012. CD-ROM.
\bibitem{3-man}
\Au{Маньяков Ю.\,А.} Технология регистрации поведения объектов в
трехмерном пространстве~// Информационные технологии в науке,
образовании и производстве (ИТНОП-2010): Мат-лы IV Междунар.
науч.-технич. конф.~--- Орел: ОрелГТУ, 2010. Т.~3. С.~182--186.
\bibitem{4-man}
\Au{Архипов О.\,П., Маньяков Ю.\,А., Сиротинин~Д.\,О.} Информационная
модель технологии представления натурного объекта и изменения его
пространственного положения~// Информатика и её применения, 2014. Т.~8.
Вып.~1. С.~71--76.


 \end{thebibliography}

 }
 }

\end{multicols}

\vspace*{-9pt}

\hfill{\small\textit{Поступила в редакцию 03.06.14}}

%\newpage

\vspace*{12pt}

\hrule

\vspace*{2pt}

\hrule

%\vspace*{12pt}

\def\tit{VOXEL MODELS TEXTURING BASED ON~REFERENCE POINTS'
COLOR INFORMATION}

\def\titkol{Voxel models texturing based on~reference points'
color information}

\def\aut{O.\,P.~Arkhipov and Y.\,A.~Maniakov}

\def\autkol{O.\,P.~Arkhipov and Y.\,A.~Maniakov}

\titel{\tit}{\aut}{\autkol}{\titkol}

\vspace*{-9pt}

\noindent
Orel Branch of Institute of Informatics Problems, Russian Academy of Sciences,
137 Moskovskoe Highway, Orel 302025, Russian Federation


\def\leftfootline{\small{\textbf{\thepage}
\hfill INFORMATIKA I EE PRIMENENIYA~--- INFORMATICS AND
APPLICATIONS\ \ \ 2014\ \ \ volume~8\ \ \ issue\ 3}
}%
 \def\rightfootline{\small{INFORMATIKA I EE PRIMENENIYA~---
INFORMATICS AND APPLICATIONS\ \ \ 2014\ \ \ volume~8\ \ \ issue\ 3
\hfill \textbf{\thepage}}}

\vspace*{6pt}

\Abste{The article describes a technology for texturing voxel models based on color information
contained in reference points. Reference points are
special points in three-dimensional space which describe a definite part of an object
surface and contain generalized color information about a corresponding area.
  The color of each voxel located in the area bounded by reference points is
defined by each reference point bounding this area. The color components weights
of each reference point are proportional to Euclidean distance from each voxel to
these reference points. The article presents a method for calculating
reference points color components weights
based on dependence on the distance from reference points to voxels.
This method is used for calculating colors of voxels. The
description and the results of experiments are also provided. The
experiments consisted in texturing generated models and real objects models.}

\KWE{three-dimensional reconstruction; texture mapping; voxel model}

\DOI{10.14357/19922264140311}

\vspace*{3pt}

  \begin{multicols}{2}

\renewcommand{\bibname}{\protect\rmfamily References}
%\renewcommand{\bibname}{\large\protect\rm References}

{\small\frenchspacing
 {%\baselineskip=10.8pt
 \addcontentsline{toc}{section}{References}
 \begin{thebibliography}{9}


\bibitem{1-man-1}
Ispol'zovanie megatekstur (megatexture, clipmaps) [Using megatextures
(megatexture, clipmaps)]. Available at:\linebreak\vspace*{-12pt}

\columnbreak


\noindent {\sf
http://www.gamedev.ru/code/articles/Megatexture} (accessed March~25,
2014).
\bibitem{2-man-1}
\Aue{Arhipov, O.\,P., Yu.\,A.~Maniakov, and D.\,O.~Sirotinin}. 2012. Metod
generatsii virtual'noy setki opornykh tochek \linebreak\vspace*{-12pt}

\pagebreak

\noindent
na tsvetnykh izobrazheniyakh
[Method of virtual markers grid generation on color images]. \textit{Mat-ly VI
Mezhdunar. Nauch.-Tekhnich. Konf. Informatsionnye Tekhnologii v Nauke,
Obrazovanii i Proizvodstve ITNOP-2012} [Information Technologies in
Science, Education, and Production Science and Technology Conference
(International) Proceedings]. Orel. CD-ROM.
\bibitem{3-man-1}
\Aue{Maniakov, Y.\,A.} 2010. Tekhnologiya registratsii povedeniya ob''ektov v
trekhmernom prostranstve [Three-dimensional space object behavior tracking technology].
\textit{Informatsionnye Tekhnologii v Nauke, Obrazovanii i Proizvodstve
(ITNOP). Mater. Mezhdunar. Nauch.-Tehnich. Konf.} [Information
Technologies in Science, Education, and Production Science and Technology
Conference (International) Proceedings]. Orel. 3:182--186.
\bibitem{4-man-1}
\Aue{Arhipov, O.\,P., Yu.\,A.~Maniakov, and D.\,O.~Sirotinin}. 2014.
Informatsionnaya model tekhnologii predstavleniya naturnogo ob''ekta i
izmeneniya ego prostranstvennogo polozheniya [Information model of the
full-scale object and its attitude changes representation technology].
\textit{Informatika i ee Primeneniya}~--- \textit{Inform. Appl.} 1:71--76.

\end{thebibliography}

 }
 }

\end{multicols}

\vspace*{-6pt}

\hfill{\small\textit{Received June 03, 2014}}

\vspace*{-18pt}

\Contr

  \noindent
  \textbf{Arkhipov Oleg P.} (b.\ 1948)~--- Candidate of Science (PhD) in
technology, Director, Orel Branch of Institute of Informatics Problems, Russian
Academy of Sciences, 137 Moskovskoe Highway, Orel 302025, Russian
Federation; arkhipov12@yandex.ru

  \vspace*{3pt}

  \noindent
  \textbf{Maniakov Yury A.} (b.\ 1984)~---
  Candidate of Science (PhD) in technology, scientist, Orel Branch of Institute of
Informatics Problems, Russian Academy of Sciences, 137 Moskovskoe Highway,
Orel 302025, Russian Federation; maniakov\_yuri@mail.ru

\label{end\stat}

\renewcommand{\bibname}{\protect\rm Литература}


