\def\l{\lambda}
\def\tl{\tilde\lambda}
\def\tB{\widetilde B}
\def\tb{\tilde b}


\def\stat{razum}

\def\tit{СТАЦИОНАРНЫЕ ВЕРОЯТНОСТИ СОСТОЯНИЙ В~СИСТЕМЕ ОБСЛУЖИВАНИЯ С~ИНВЕРСИОННЫМ ПОРЯДКОМ
ОБСЛУЖИВАНИЯ И~ОБОБЩЕННЫМ ВЕРОЯТНОСТНЫМ
ПРИОРИТЕТОМ$^*$}

\def\titkol{Стационарные вероятности состояний в~системе обслуживания} % с~инверсионным порядком обслуживания и~обобщенным вероятностным приоритетом}

\def\aut{Л.\,А.~Мейханаджян$^1$,  Т.\,А.~Милованова$^2$,
А.\ В.\ Печинкин$^3$,  Р.\,В.~Разумчик$^4$}

\def\autkol{Л.\,А.~Мейханаджян,  Т.\,А.~Милованова,
А.\ В.\ Печинкин,  Р.\,В.~Разумчик}

\titel{\tit}{\aut}{\autkol}{\titkol}

{\renewcommand{\thefootnote}{\fnsymbol{footnote}} \footnotetext[1]
{Работа выполнена при частичной поддержке РФФИ (проект 13-07-00223).}}

\renewcommand{\thefootnote}{\arabic{footnote}}
\footnotetext[1]{Российский университет дружбы народов, lameykhanadzhyan@gmail.com}
\footnotetext[2]{Российский университет дружбы народов, tmilovanova77@mail.ru}
\footnotetext[3]{Институт проблем информатики Российской академии наук,
apechinkin@ipiran.ru}
\footnotetext[4]{Институт проблем информатики Российской академии наук;
Российский университет дружбы народов, rrazumchik@ieee.org}

\Abst{Рассматривается однолинейная система массового
обслуживания (СМО), в которую поступает поток заявок,
называемый здесь потоком пуассоновского типа.
Отличие этого потока от пуассоновского
заключается в том, что интенсивность поступления
заявок равна~$\lambda$, если на приборе имеется
заявка, и $\tl$, если сис\-те\-ма пуста.
Если заявка поступает в сис\-те\-му, в которой
на приборе имеется заявка, то исходное
распределение времени обслуживания поступающей
заявки является произвольным с функцией
распределения (ФР) $B(x)$, в противном случае~---
произвольным с ФР $\tB(x)$.
В~сис\-те\-ме реализован инверсионный порядок
обслуживания с обобщенным вероятностным
приоритетом, заключающийся в сле\-ду\-ющем.
Предполагается, что в любой момент времени
известна остаточная длина каждой заявки в сис\-те\-ме.
В~момент поступления в сис\-те\-му новой заявки ее
исходная длина сравнивается с остаточной
длиной заявки на приборе и в зависимости
от результатов сравнения одна из них становится
на прибор, а другая~--- на первое место в
очередь, причем каждая заявка приобретает новую
(случайную) длину и даже может покинуть систему.
В~статье предложены математические соотношения
для вычисления основных показателей
функционирования системы, связанных со
стационарным распределением числа заявок в ней.}


\KW{система массового обслуживания; специальные
дисциплины; инверсионный порядок
обслуживания; вероятностный приоритет}

\DOI{10.14357/19922264140304}


\vskip 12pt plus 9pt minus 6pt

\thispagestyle{headings}

\begin{multicols}{2}

\label{st\stat}


\section{Введение}

В текущих условиях активного развития
инфотелекоммуникационной отрасли необходимы
аналитические средства создания и анализа
объектов и их совокупностей, используемых для
передачи, хранения и обработки информации.
Традиционно применяемые для описания
функционирования инфотелекоммуникационных
систем (ИТС) СМО
и их комбинации позволяют получать точные
и/или приближенные оценки для ключевых
показателей производительности ИТС, хотя анализ
зачастую связан с определенными
вычислительными трудностями, что может приводить
к бессодержательным результатам.
Другой подход к анализу показателей производительности
ИТС заключается в моделировании реальной системы
с помощью простой модели СМО и применении в ней
сложных дисциплин обслуживания.
Например, предоставление приоритета обслуживания
заявкам с наименьшей остаточной длиной (дисциплина
{SRPT}~--- shortest remaining processing time)~\cite{shrage} дает оптимальную стратегию
с точки зрения минимизации числа заявок в системе.
Но при расчетах показателей функционирования СМО
с дисциплиной {SRPT} необходимо знание времени
обслуживания (длины) каждой поступающей в систему
заявки и использование информации о всех остаточных
длинах.
Подобные ограничения свойственны многим специальным
дисциплинам.

На практике потоки, циркулирующие в ИТС, являются
неоднородными, и требования к обслуживанию у
сообщений различных классов могут быть различными.
Кроме того, могут происходить сбои систем ввиду
флуктуации нагрузки
или появления дестабилизирующих факторов.
Вместо введения в рассмотрение большого числа потоков
можно воспользоваться следующим приемом:
предположить, что длины заявок, а также эффекты
(события), которые связаны с поступлением заявок
той или иной длины, известны лишь с
некоторой вероятностью.
Поэтому представляет интерес рассмотренный в~\cite{aaa1} инверсионный порядок обслуживания с
вероятностным приоритетом (дис\-цип\-ли\-на {LCFS PP}),
при котором при поступлении в систему новой заявки
место на приборе и первое место в очереди
разыг\-ры\-ва\-ют между собой вновь поступившая заявка
и заявка, находившаяся ранее на приборе, причем
с вероятностью, являющейся произвольной функцией
от длины поступившей заявки и остаточной длины
заявки на приборе.


По данной проблематике опубликовано и продолжает
появляться значительное число работ как
теоретического, так и прикладного характера
(см., например,~[3--8]).
%\cite{aaa2,aaa3,aaa4,av1,av2,av3}).
В~частности, в~\cite{aaa2} на примере одноканальной
СМО с пуассоновским входящим потоком была рассмотрена
дисциплина, предо\-став\-ля\-ющая преимущество заявкам с
меньшими остаточными длинами.
Эта дисциплина является в определенном смысле
промежуточной между обычной дисциплиной без
прерывания обслуживания и дисциплиной {SRPT}.
Считаются известными
остаточные времена обслуживания (длины) всех
находящихся в системе заявок.
При поступлении новой заявки ее длина сравнивается
с длиной заявки на приборе, и та из них, длина
которой минимальна, становится на
прибор, оставляя за второй первое место в очереди.


В настоящей статье на основе развития идей
вышеприведенных работ вводится новая дисциплина
(инверсионный порядок обслуживания с обобщенным
вероятностным приоритетом, или LCFS GPP),
согласно которой при поступлении новой заявки система
принимает решение не только о том, что с ней делать
(поставить в очередь, поменять местами с заявкой
на приборе, удалить из системы и~т.\,д.), но и
как долго обслуживать оставшиеся в системе заявки.
Решение принимается путем задания вероятностей
соответствующих событий.
Исследуется СМО с такой дисциплиной, входящим потоком
пуассоновского типа и произвольным распределением
времени обслуживания заявки.
На основе метода исключения состояний получены
в терминах вычислительных алгоритмов и
производящих функций выражения для стационарного
распределения числа заявок в системе.
Как будет показано, вычисление стационарного
распределения (а также его моментов) связано с
решением уравнения Фредгольма 2-го рода.

В следующем разделе дается подробное опи\-сание
системы,
затем выводятся формулы для нахожде\-ния стационарных
вероятностных характеристик, и в заключении
приводятся некоторые примеры численных расчетов,
выполненных по полученным соотношениям.

\section{Описание системы}

Рассмотрим СМО с входящим потоком заявок,
который для простоты будем называть здесь
потоком пуассоновского типа.
Отличие этого потока от пуассоновского
заключается в следующем:
интенсивность поступления заявки равна~$\lambda$,
если на приборе имеется заявка, и~$\tl$, если
система пуста.
Если в момент поступления заявки в систему
на приборе имеется заявка, то исходное
распределение времени обслуживания поступающей
заявки является произвольным с ФР $B(x)$.
Если же заявка поступает в систему в тот
момент, когда система пуста, то исходное
распределение времени обслуживания поступающей
заявки является произвольным с ФР $\tB(x)$.

Далее для простоты изложения будем считать,
что ФР $B(x)$ и $\tB(x)$ имеют непрерывные
ограниченные плотности
распределения~$b(x)\hm=B'(x)$ и $\tb(x)\hm=\tB'(x)$.

Обобщенный инверсионный порядок обслуживания
с вероятностным приоритетом ({LCFS GPP})
заключается в следующем.
Предполагается, что в любой момент времени
известна остаточная длина (далее будем говорить
просто длина) каждой заявки в системе.
В момент поступления в систему новой заявки ее
исходная длина $u$ сравнивается с (остаточной)
длиной $v$ заявки на приборе.
С вероятностью~$D(x,y|u,v)$,
зависящей только от $u$ и~$v$, обслуживавшаяся
ранее заявка продолжает обслуживаться, причем
ее длина становится меньше $y$, а вновь
поступившая становится на первое место в очереди
и ее длина становится меньше~$x$.
Кроме того, с вероятностью~$D^*(x,y|u,v)$,
зависящей только от $u$ и~$v$, вновь поступившая
заявка занимает прибор, вытесняя обслуживавшуюся
ранее на первое место в очереди, причем длина
заявки, бывшей ранее на приборе, становится
меньше~$y$, а вновь поступившей~--- меньше~$x$.

Если на приборе находится заявка остаточной
длины~$v$ и в систему поступает заявка
длины~$u$, то с вероятностью $D_0(x|u,v)$
заявка, находящаяся на приборе, покидает
систему, а поступившая заявка становится на
прибор, причем ее длина становится меньше~$x$.
Кроме того, с вероятностью
$D_0^*(y|u,v)$ поступившая заявка сразу же
покидает систему, а~заявка, находящаяся на
приборе, продолжает обслуживаться, причем ее
длина становится меньше~$y$.
Введем также обозначение:
%$$
\begin{equation*}
%\label{(2.1)}
D(x|u,v) = D_0(x|u,v) + D_0^*(x|u,v)\,.
\end{equation*}
%$$
Здесь $D(x|u,v)$~--- вероятность того, что одна
из двух заявок покинет систему, а вторая встанет
на прибор и примет длину меньше~$x$.

Наконец, предполагается, что с
вероят\-ностью~$d_0(u,v)$ обе заявки покидают
систему, а на прибор становится первая заявка
из очереди.

Будем считать для удобства изложения, что все
ФР $D(x,y|u,v)$, $D^*(x,y|u,v)$, $D_0(x|u,v)$,
$D_0^*(y|u,v)$, $D(y|u,v)$ и $D_0(u,v)$
имеют непрерывные ограниченные плотности
$d(x,y|u,v)\hm=\partial^2 D(x,y|u,v)/
(\partial x\, \partial y)$,\ \
$d^*(x,y|u,v)\hm=\partial^2 D^*(x,y|u,v)/
(\partial x\, \partial y)$,\ \
$d_0(x|u,v)\hm= \partial D_0(x|u,v)/\partial x$,\ \
$d_0^*(y|u,v)\hm=\partial D_0^*(y|u,v)/\partial y$
и
$d(x|u,v)\hm=\partial D(x|u,v)/\partial x$.


Естественно, для любых $u$ и~$v$ выполнено условие:
\begin{multline}
\label{(2.1)}
\int\limits_0^\infty
\int\limits_0^\infty
[d(x,y|u,v) + d^*(x,y|u,v)]
\,dx\,dy
+ {}\\
{}+\int\limits_0^\infty d(x|u,v) \,dx + d_0(u,v) = {}\\
{}= D(\infty,\infty|u,v)+D^*(\infty,\infty|u,v)
+{}\\
 {}+D(\infty|u,v) + d_0(u,v) = 1\,.
\end{multline}



Если длина заявки на приборе становится
равной нулю, то она мгновенно покидает
систему и на прибор переходит первая
заявка из очереди.
Остальная очередь сдвигается на единицу.


Далее будем предполагать, что система
функционирует в стационарном режиме.
К~сожалению, для рассматриваемой СМО не
удается выписать общее необходимое и
достаточное условие существования стационарного
режима функционирования.
Это условие зависит от конкретных параметров
сис\-те\-мы и в каждом отдельном случае нуждается
в специальном исследовании.
Здесь приведем прос\-тое достаточ\-ное условие,
вытекающее из сравнения суммарной имеющейся
работы в описанной СМО и~суммарной работы в
стандартной СМО $M/G/1/\infty$.

Условие состоит из выполнения следующих
соотношений:
\begin{enumerate}[1.]
\item  $\tb =
\int\limits_0^\infty x\,\tb(x)\,dx < \infty$.

\item $\rho = \lambda b =
\lambda \int\limits_0^\infty x\, b(x)\,dx < 1$.

\item $d(x,y|u,v) = 0$ при всех $u,v$ и
$y\hm>v$ или $x\hm>u$.

\item $d^*(x,y|u,v) = 0$ при всех $u,v$ и
$y\hm>v$ или $x\hm>u$.

\item $d(x|u,v) = 0$ при всех $u,v$ и
$x\hm>u$.

\item $d^*(y|u,v) = 0$ при всех $u,v$ и
$y\hm>v$.
\end{enumerate}

Соотношения~{3}--{6} соответствуют
тому факту, что после поступления новой заявки
измененные длины заявок не превышают те длины,
которые были до поступления.
Отметим, что здесь параметр $\rho \hm= \lambda b$ не
является загрузкой в традиционном смысле и может
существенно от нее отличаться.

\section{Система уравнений равновесия}

Обозначим через $\nu(t)$ число заявок в системе
в момент $t$, а через $\vec\xi(t)\hm=
(\xi_{1}(t),\ldots,\xi_{\nu(t)}(t))$~---
вектор, координатой $\xi_{1}(t)$ которой
является (остаточное) время обслуживания
заявки, находящейся в этот момент на приборе,
$\xi_{2}(t)$~--- первой заявки в
очереди$,\ldots,$ $\xi_{\nu(t)-1}(t)$~---
последней, \mbox{$(\nu(t)-1)$-й} заявки в очереди.
При $\nu(t)\hm=0$ вектор $\vec\xi(t)$
не определяется.
Тогда $\eta(t)\hm=(\nu(t),\vec\xi(t))$ представляет
собой марковский процесс, описывающий
поведение числа заявок в рассматриваемой сис\-теме.

Положим
\begin{align*}
p_{0}(t) &= {\bf P}\{\nu(t)=0\} \,;
\\
P_{n}(t;x_1,\ldots,x_{n})
&= {}\\
&\hspace*{-30mm}{}={\bf P}\{\nu(t)=n,\, \xi_{1}(t)<x_{1},\ldots,\xi_{n}(t)<x_{n}\}
\,,\ \ n\ge 1\,.
\end{align*}
Обозначим через
\begin{align*}
p_{0} &= \lim\limits_{t\to\infty} p_{0}(t) \,;
\\
P_{n}(x_1,\ldots,x_{n}) &=
\lim\limits_{t\to\infty} P_{n}(t;x_1,\ldots,x_{n})\,,\ \ n\ge 1\,,
\end{align*}
стационарное распределение процесса $\eta(t)$.
В~силу сделанных в предыдущем разделе
предположений относительно параметров системы
существуют (см., например,~\cite{ppav})
непрерывные ограниченные плотности
\begin{equation*}
p_n(x_1,\ldots,x_{n}) =
\fr{\partial^n}{\partial x_1\cdots \partial x_n}
P_n(x_1,\ldots,x_{n}) \,,\ \ n\ge 1\,.
\end{equation*}

Выпишем систему интегродифференциальных
уравнений, которой удовлетворяют стационарные
плотности $p_n(x_1,\ldots,x_{n})$ и которую
для краткости по аналогии с простейшими СМО
будем называть системой уравнений равновесия (СУР).
Для этого рассмотрим вспомогательную систему
с $(n-1)$ мес\-та\-ми ожидания, отличающуюся от исходной
сис\-те\-мы только тем, что если в очереди
находится $(n-1)$\linebreak заявок, заявка на приборе имеет
остаточную длину~$v$ и поступает новая заявка
длины~$u$, то с вероятностью $d(x,y|u,v)$ на
приборе остается вновь\linebreak поступившая заявка,
длина которой становится равной~$x$, а
обслуживавшаяся ранее заявка покидает
систему и, наоборот, с вероятностью
$d^*(y,x|u,v)$ систему покидает вновь
поступившая заявка, а находившаяся ранее на
приборе заявка продолжает обслуживаться, но
ее длина становится равной~$x$.

В силу метода исключения состояний~\cite{ppav}
стационарные вероятности состояний в исходной
и вспомогательной системах отличаются лишь на
постоянный множитель.
Это дает возможность при составлении
уравнений для $p_n(x_1,\ldots,x_{n})$, $n\hm\ge 1$,
воспользоваться вспомогательной системой и
получить следующие соотношения:
\begin{multline}
\label{(3.1)}
-p'_1(x) =
\tl \tb(x) p_0 - \lambda p_1(x) +{}\\
{}+ \lambda \left(
\int\limits_0^\infty \int\limits_0^\infty
d(x|u,v) b(u) p_1(v) \,du\,dv
+ {}\right.\\
{}+
\int\limits_0^\infty \int\limits_0^\infty \int\limits_0^\infty
[d(x,y|u,v) b(u) p_1(v) +{}\\
\left.{}+ d^*(y,x|u,v) b(u) p_1(v)]
\,dy\,du\,dv \vphantom{\int\limits_0^\infty}
\right)\,;
\end{multline}

\vspace*{-12pt}

\noindent
\begin{multline}
\label{(3.2)}
-p'_{n}(x_1,\ldots,x_n) = {}\\
{}= \lambda \left(
\int\limits_0^\infty \int\limits_0^\infty
\left[d(x_2,x_1|u,v) b(u) p_{n-1}(v,x_3\ldots,x_n)
+ {}\right.\right.\\
\left.\left.{}+
d^*(x_1,x_2|u,v) b(u) p_{n-1}(v,x_3,\ldots,x_n)\right] \,du\,dv
\vphantom{\int\limits_0^\infty}
\right)
- {}\\
{}- \lambda p_{n}(x_1,\ldots,x_n) +{}\\
{}+ \lambda
\left(
\int\limits_0^\infty
\int\limits_0^\infty
d(x_1|u,v) b(u) p_{n}(v,x_2,\ldots,x_n)
\,du\,dv
+ {}\right.\\
{}+ \int\limits_0^\infty
\int\limits_0^\infty
\int\limits_0^\infty
\left[d(x_1,y|u,v) b(u) p_{n}(v,x_2,\ldots,x_n)
+ {}\right.\\
\left.\left.{}+
d^*(y,x_1|u,v) b(u) p_{n}(v,x_2,\ldots,x_n)\right]
\,dy\,du\,dv \vphantom{\int\limits_0^\infty}\right)\,,
\\ n\ge 2\,.
\end{multline}

Остановимся подробнее на выводе уравнения~\eqref{(3.2)}.
Рассмотрим моменты времени~$t$ и $(t\hm+\Delta)$.
Тогда для того чтобы в момент времени
$(t\hm+\Delta)$ в системе
находилось~$n$, $n\hm\ge 2$, заявок, причем
на приборе заявка длины~$x_1$, а в очереди
заявки длин $x_2,\ldots,x_n$, нужно, чтобы
произошло одно из следующих событий:
\begin{itemize}
\item
в момент $t$ в системе находилось $(n-1)$
заявок, причем заявка на приборе имела
длину~$v$, первая заявка в очереди имела
длину $x_3,\ldots,$ последняя заявка в очереди
\mbox{имела} длину~$x_n$ (с плотностью
вероятностей $p_{n-1}(t;v,x_3,\ldots,x_n)$),
и за время~$\Delta$ поступила заявка
(с вероятностью $\lambda\Delta$) длины~$u$
(с плотностью вероятностей $b(u)$).
Заявка на приборе продолжает обслуживаться,
но ее длина становится равной~$x_1$, а вновь
поступившая заявка занимает первое место в
очереди и ее длина становится равной~$x_2$
(с плотностью вероятностей $d(x_2,x_1|u,v)$);
\item
в момент $t$ в системе находилось $(n\hm-1)$
заявок, причем заявка на приборе имела
длину~$v$, первая заявка в очереди имела
длину $x_3,\ldots,$ последняя заявка в
очереди \mbox{имела} длину~$x_n$ (с плот\-ностью
вероятностей $p_{n-1}(t;v,x_3,\ldots,x_n)$),
и за время~$\Delta$ поступила заявка
(с~вероятностью $\lambda\Delta$) длины~$u$
(с~плот\-ностью вероятностей $b(u)$).
Поступившая заявка занимает прибор, и ее длина
становится равной~$x_1$, а заявка,
обслуживавшаяся до поступления новой заявки,
занимает первое мес\-то в очереди, и ее длина
становится равной~$x_2$ (с~плот\-ностью
вероятностей $d^*(x_1,x_2|u,v)$);
\item
в момент $t$ в сис\-те\-ме находилось $n$ заявок,
причем заявка на приборе имела
длину $x_1\hm+\Delta$, первая заявка в очереди
имела длину $x_2,\ldots,$ последняя заявка в
очереди имела длину~$x_n$ (с плотностью
вероятностей
$p_n(t;x_1\hm+\Delta,x_2,\ldots,x_n)$), и за
время~$\Delta$ не поступили заявки (с
вероятностью $(1\hm-\lambda\Delta$));
\item
в момент $t$ в сис\-те\-ме находилось $n$ заявок,
причем заявка на приборе имела длину~$v$,
первая заявка в очереди имела длину
$x_2,\ldots,$ последняя заявка в очереди
имела длину~$x_n$ (с плот\-ностью
вероятностей $p_n(t;v,\ldots,x_n)$), и за время~$\Delta$ поступила заявка (с вероятностью
$\lambda\Delta$), име\-ющая длину~$u$
(с~плот\-ностью вероятностей $b(u)$).
Заявка, находившаяся на приборе, покидает
сис\-те\-му, а поступившая заявка становится на
прибор, причем ее длина становится
равной~$x_1$, или, наоборот, поступившая
заявка сразу же покидает сис\-те\-му, а заявка,
находившаяся на приборе, продолжает
обслуживаться, причем ее длина становится
равной~$x_1$ (с плотностью
вероятностей $d(x_1|u,v)$);
\item
в момент $t$ в сис\-те\-ме находилось $n$ заявок,
причем заявка на приборе имела длину~$v$,
первая заявка в очереди имела длину
$x_2,\ldots,$ последняя заявка в очереди
имела длину~$x_n$ (с плотностью вероятностей
$p_n(t;v,x_2,\ldots,x_n)$), и за время~$\Delta$
поступила заявка (с вероятностью $\lambda\Delta$) длины~$u$
(с плотностью вероятностей $b(u)$).
Заявка, находившаяся на приборе, покинула
сис\-те\-му, а~на прибор встала поступившая
заявка, длина которой стала~$x_1$ (с
плотностью вероятностей $d(x_1,y|u,v)$);
\item
в момент $t$ в сис\-те\-ме находилось $n$ заявок,
причем заявка на приборе имела длину~$v$,
первая заявка в очереди имела длину
$x_2,\ldots,$ последняя заявка в очереди
имела длину~$x_n$ (с плотностью вероятностей
$p_n(t;v,x_2,\ldots,x_n)$), и за время~$\Delta$ поступила заявка (с вероятностью
$\lambda\Delta$) длины~$u$ (с плотностью
вероятностей $b(u)$).
Вновь поступившая заявка покидает сис\-те\-му,
на приборе продолжает обслуживаться заявка,
на\-хо\-див\-шая\-ся на приборе до поступления новой
заявки, но длина ее становится равной~$x_1$
(с плотностью вероятностей $d^*(y,x_1|u,v)$).
\end{itemize}
Вероятности других событий равны $o(\Delta)$.

Применяя формулу полной вероятности, имеем
\begin{multline*}
p_{n}(t+\Delta;x_1,\ldots,x_n)
={}\\
{}= \lambda\Delta \left(
\int\limits_0^\infty \!\int\limits_0^\infty
\left[d(x_2,x_1|u,v) b(u) p_{n-1}(t;v,x_3\ldots,x_n)
+{}\right.\right.\hspace*{-8pt}\\
\left.\left.{}+
d^*(x_1,x_2|u,v) b(u) p_{n-1}(t;v,x_3,\ldots,x_n)\right]
\,du\,dv \vphantom{\int\limits_0^\infty}\right)
+ {}\\
{}+
(1-\lambda\Delta) p_{n}(t;x_1+\Delta,x_2,\ldots,x_n)
+ {}\\
{}+\lambda\Delta \left(
\int\limits_0^\infty \int\limits_0^\infty
d(x_1|u,v) b(u) p_{n}(t;v,x_2,\ldots,x_n)
\,du\,dv + {}\right.\hspace*{-1.28279pt}\\
{}+
\int\limits_0^\infty \int\limits_0^\infty
\int\limits_0^\infty \left[d(x_1,y|u,v) b(u) p_{n}(t;v,x_2,\ldots,x_n)\right.
+{}\\
{}+
\left.\left.d^*(y,x_1|u,v) b(u) p_{n}(t;v,x_2,\ldots,x_n)\right]
\,dy\,du\,dv \vphantom{\int\limits_0^\infty}
\right)+{}\\
{}+o(\Delta)\,, \quad n\ge 2\,,
\end{multline*}
откуда, перенося слагаемое
$p_n(t;x_1+\Delta,x_2,\ldots,x_{n})$ в
левую часть равенства, деля на~$\Delta$,
устремляя~$\Delta$ к нулю и учитывая
стационарный режим функционирования сис\-те\-мы,
получаем уравнение~\eqref{(3.2)}.

Уравнение~\eqref{(3.1)} получается аналогично.

К системе уравнений~\eqref{(3.1)},
\eqref{(3.2)} нужно добавить начальные условия,
которые удобно записать в виде:
\begin{equation}
\label{(3.3)}
p_{1}(\infty) = \lim\limits_{X\to \infty}
p_{1}(X) = 0\,;                         %      \eqno(3.3)
\end{equation}

\vspace*{-12pt}

\noindent
\begin{multline}
p_{n}(\infty,x_2,\ldots,x_n) = \lim\limits_{X\to \infty}
p_{n}(X,x_2,\ldots,x_n) = 0\,,\\  n\ge 2\,.
\label{(3.4)}
\end{multline}

Покажем, как получаются соотношения~\eqref{(3.3)}, \eqref{(3.4)}
на примере~\eqref{(3.3)}.
Интегрируя равенство~\eqref{(3.1)} в пределах
от~0 до~$X$, приходим к формуле
\begin{multline*}
p_1(0) - p_1(X) = \tl \tB(X) p_0 - \lambda P_1(X)
+ {}\\
{}+\lambda \int\limits_0^X \left(
\int\limits_0^\infty
\int\limits_0^\infty
d(x|u,v) b(u) p_1(v)
\,du\,dv
+{}\right.
\end{multline*}

\noindent
\begin{multline*}
{}+
\int\limits_0^\infty
\int\limits_0^\infty
\int\limits_0^\infty
\left[d(x,y|u,v) b(u) p_1(v)
+{}\right.\\
\left.\left.{}+
d^*(y,x|u,v) b(u) p_1(v)\right]
\,dy\,du\,dv \vphantom{\int\limits_0^\infty}
\right) dx\,,
\end{multline*}
которую перепишем в виде:
\begin{multline}
\label{(3.5)}
p_1(X) = p_1(0) - \tl \tB(X) p_0
+ \lambda P_1(X) - {}\\
{}-\lambda \int\limits_0^\infty
\int\limits_0^\infty b(u) p_1(v)
\,du\,dv \int\limits_0^X
\left( \vphantom{\int\limits_0^\infty}
d(x|u,v)
+{}\right.\\
\left.{}+
\int\limits_0^\infty
[d(x,y|u,v) + d^*(y,x|u,v) ]\,dy
\right)\,dx.                         %   \eqno(3.5)
\end{multline}
Правая часть~\eqref{(3.5)} имеет предел
(и даже с учетом~\eqref{(2.1)} конечный)
при $X\hm\to\infty$.
Поэтому $p_1(X)$ также стремится при
$X\hm\to\infty$ к пределу, который для плотности
вероятностей не может быть ничем иным, кроме
нуля, что доказывает справедливость~\eqref{(3.3)}.

Оставшаяся неизвестной стационарная
вероятность~$p_0$ отсутствия заявок в системе
находится, как обычно, из условия нормировки:
\begin{equation}
\label{(3.6)}
\sum\limits_{n=0}^\infty p_n = 1\,,
\end{equation}
где
$p_n=P_n(\infty,\ldots,\infty)$, $n\hm\ge1$,~---
стационарная вероятность наличия в системе~$n$~заявок.

Полученные соотношения~\eqref{(3.1)}--\eqref{(3.6)} позволяют
(теоретически) последовательно по~$n$
находить стационарные плотности вероятностей
$p_n(x_1,\ldots,x_{n})$ (один из методов
решения СУР~\eqref{(3.1)}--\eqref{(3.6)}
будет рассмотрен в следующем разделе на примере маргинальных плотностей).
Но практическое применение такого подхода
невозможно, поскольку при $n\hm\to\infty$ число
аргументов~$x_i$ стационарных плотностей
вероятностей $p_n(x_1,\ldots,x_{n})$ стремится к бесконечности.

Однако в большинстве практических случаев
достаточно знать только маргинальные плотности
\begin{equation*}
p_{n}(x) = \mathop{\int\!\cdots\!\int}\limits_{x_2,\ldots,x_n>0}
\!p_{n}(x,x_2\ldots,x_n) \,dx_2\cdots dx_n\,,
\ n\ge 2\,.
\end{equation*}

Интегрируя~\eqref{(3.2)} по~$x_2,\ldots ,x_n$ в пределах от нуля до
бесконечности и вспоминая равенство~\eqref{(3.1)}, получаем следующее
интегродифференциальное уравнение
для $p_{n}(x)$, $n\hm\ge 1$:
\begin{multline}
\label{(3.7)}
-p'_{n}(x) = a_n(x) - \lambda p_{n}(x)
+ \int\limits_0^\infty K_n(x,v) p_{n}(v)\,dv \,,\\ n\ge 1\,,
\end{multline}
где
\begin{equation*}
a_1(x) = \tl \tb(x) p_0 \,;
\end{equation*}

\vspace*{-12pt}

\noindent
\begin{multline*}
K_1(x,v) = \lambda \int\limits_0^\infty
b(u)\,du \left(\vphantom{\int\limits_0^\infty}
d(x|u,v)
+{}\right.\\
\left.{}+ \int\limits_0^\infty
[d(x,y|u,v) + d^*(y,x|u,v)]\,dy
\right)\,;
\end{multline*}

\vspace*{-12pt}

\noindent
\begin{multline*}
a_{n}(x) = {}\\
{}=\lambda \left(
\int\limits_0^\infty
p_{n-1}(v)\,dv
\int\limits_0^\infty
b(u)\,du
\int\limits_0^\infty
\left[d(y,x|u,v) +{}\right.\right.\\
\left.\left.{}+ d^*(x,y|u,v)\right] \,dy
\vphantom{\int\limits_0^\infty}
\right)\,,\enskip  n\ge 2\,;
\end{multline*}
\vspace*{-12pt}

\noindent
\begin{multline*}
K_n(x,v) = \l \int\limits_0^\infty
b(u)\,du \left( \vphantom{\int\limits_0^\infty}
d(x|u,v)
+ {}\right.\\
\left.{}+\int\limits_0^\infty
[d(x,y|u,v) + d^*(y,x|u,v)]
\,dy \right)\,, \ \ n\ge 2\,.
\end{multline*}
Начальное условие для уравнения~\eqref{(3.7)}
по аналогии с~\eqref{(3.3)} запишем в виде:
\begin{equation}
\label{(3.8)}
p_{n}(\infty) = \lim\limits_{X\to \infty}
p_{n}(X) = 0\,,\ \ n\ge 1\,.
\end{equation}

\section{Метод численного решения системы
уравнений равновесия}

Приведем один из возможных методов решения
интегродифференциального уравнения~\eqref{(3.7)}
с начальным условием~\eqref{(3.8)}.

Решение будем искать в виде:
\begin{equation}
\label{(4.1)}
p_n(x) = e^{\l x} q_n(x) \,.
\end{equation}
Подставляя в~\eqref{(3.7)} вместо $p_n(x)$
ее выражение по формуле~\eqref{(4.1)}, получаем новое
интегродифференциальное уравнение:
\begin{equation*}
- q'_n(x) = e^{-\l x} a_n(x) +
\int\limits_0^\infty e^{\l v} e^{-\l x} K_n(x,v) q_n(v)\, dv\,.
\end{equation*}
Интегрируя последнее равенство по~$x$ в
пределах от~$y$ до~$\infty$ и учитывая
начальное условие \eqref{(3.8)}, получаем
интегральное уравнение Фредгольма 2-го рода:
\begin{equation*}
q_n(y) = b_n(y) + \int\limits_0^\infty
G_n(y,v) q_n(v)\, dv \,,
\end{equation*}
где
\begin{equation*}
b_n(y) = \int\limits_y^\infty e^{-\l x} a_n(x)\, dx\,;
\end{equation*}
\begin{equation}
\label{(4.2)}
G_n(y,v) = \int\limits_y^\infty e^{\l (v-x)} K_n(x,v)\, dx\,.
\end{equation}

Отметим, что свободный член $b_n(y)$ и ядро
$G_n(y,v)$ интегрального уравнения~\eqref{(4.2)}
являются неотрицательными функциями.

Численные методы решения интегральных
уравнений Фредгольма 2-го рода хорошо известны
(см., например,~[10--12]).
Так, в данном случае хороший результат дает
итерационный метод
\begin{equation*}
q^{(s)}_n(y) = b_n(y) + \int\limits_0^\infty
G_n(y,v) q^{(s-1)}_n(v)\, dv \,,
\end{equation*}
причем в качестве начальной итерации
необходимо взять нулевое приближение
${q^{(0)}_n(v)\hm\equiv 0}$.
Тогда итерации будут возрастающими, что
позволит контролировать скорость сходимости к точному решению.

\section{Применение производящей функции}

Введем производящую функцию (ПФ)
$$
\pi(z,x)= \sum\limits_{n=1}^\infty p_n(x)z^n\,.
$$

Умножая уравнение~\eqref{(3.7)} на~$z^n$
и суммируя по всем значениям $n\hm\ge1$, получаем:
\begin{multline}
\label{(5.1)}
-\fr{\partial \pi(z,x)}{\partial x}
= z \tl \tb(x) p_0 - \l \pi(z,x)
+ {}\\
{}+\lambda \int\limits_0^\infty \pi(z,v)\,dv
\int\limits_0^\infty
b(u)\,du \left(\vphantom{\int\limits_0^\infty}
d(x|u,v) +{}\right.
\\
{}+ \int\limits_0^\infty
\left[d(x,y|u,v) + d^*(y,x|u,v)
+ z d(y,x|u,v) +{}\right.\\
\left.\left.{}+ z d^*(x,y|u,v)\right]
\,dy \vphantom{\int\limits_0^\infty}\right)\,.
\end{multline}
Граничное условие для интегродифференциального
уравнения~\eqref{(5.1)}, как и прежде, имеет вид:
\begin{equation*}
\pi(z,\infty) = \lim\limits_{x\to\infty} \pi(z,x)
= 0\,.
\end{equation*}

Трактуя $z$ как параметр, для решения
уравнения~\eqref{(5.1)} можно применить метод,
описанный в предыдущем разделе.


Однако ПФ $\pi(z,x)$, как правило, мало
пригодна для вычисления стационарного
распределения числа заявок в системе.
Но с ее помощью можно найти моменты этого
распределения.


Покажем, как это делается, на примере
математического ожидания ${\sf E}\,\nu$
стационарного распределения числа заявок в
системе.
При этом, не вдаваясь в математические тонкости,
будем считать, что ${\sf E}\,\nu$ существует и
операции дифференцирования, которые будут
применены далее, законны.



Сначала, решая уравнение~\eqref{(5.1)} при $z\hm=1$,
\mbox{найдем} значение $\pi(1,x)$.


Затем продифференцируем уравнение \eqref{(5.1)}
по $z$ в точке $z=1$.
При этом для сокращения записи применим
обозначение:
$$
\fr{\partial \pi(z,x) }{\partial z}\Big|_{z=1} =
\pi'(1,x) \,.
$$
Имеем:
\begin{multline}
-\fr{\partial \pi'(1,x)}{\partial x} = \tl \tb(x) p_0 - \l \pi'(1,x)
+{}\\
{}+ \lambda \int\limits_0^\infty \pi'(1,v)\,dv
\int\limits_0^\infty b(u)\,du
\Bigg( d(x|u,v)
+ {}\\
{}+
\int\limits_0^\infty
\left[d(x,y|u,v) + d^*(y,x|u,v)
+{}\right.\\
\left.{}+ z d(y,x|u,v) + z d^*(x,y|u,v)\right] \,dy
\Bigg)
+ {}\\
{}+
\l \int\limits_0^\infty \pi(1,v)\,dv
\int\limits_0^\infty b(u)\,du
\Bigg( \int\limits_0^\infty
\left[d(y,x|u,v) +{}\right.\\
\left.{}+ d^*(x,y|u,v)\right]
\,dy \Bigg) \,.
\label{(5.3)}
\end{multline}
Граничное условие для интегродифференциального
уравнения~\eqref{(5.3)} определяется прежним равенством:
\begin{equation*}
\pi'(1,\infty) = \lim\limits_{x\to\infty} \pi'(1,x) = 0\,.
\end{equation*}
Уравнение~\eqref{(5.3)} имеет
точно такой же вид, что и уравнение~\eqref{(5.1)},
и может быть решено теми же самыми методами.

Осталось заметить, что
$$
{\sf E}\,\nu = \sum\limits_{n=1}^\infty n
\int\limits_0^\infty p_n(x)\, dx
= \int\limits_0^\infty \pi'(1,x)\, dx \,.
$$

\section{Примеры расчетов}

На основе полученных результатов с помощью
программных средств MATLAB была написана
программа, позволяющая вычислять совместное
стационарное распределение числа заявок в
системе и остаточного времени обслуживания
заявки на \mbox{приборе} и другие связанные с
ним характеристики, а также исследовать
поведение рассматриваемой СМО
в зависимости от значений определяющих ее
исходных параметров.
{Разработанная} программа позволяет
проводить исследование в широкой области
изменения исходных параметров
и при любых заданных аналитически функциях
$d(x,y|u,v)$, $d^*(x,y|u,v)$, $d_0(x|u,v)$,
$d_0^*(x|u,v)$ и $d_0(u,v)$,
удовле\-тво\-ря\-ющих условию существования стационар-\linebreak ного режима.
Однако вид этих функций существенным образом
влияет на эффективность работы\linebreak про\-граммы.

Приведем некоторые результаты расчетов, проведенных с помощью разработанной программы.
При этом будем предполагать, что $\tB(x)\hm=B(x)$,
и среднее время обслуживания~$b$ заявки на приборе положим равным~1.
Параметр~$\l$ далее будет использоваться как аргумент при
построении графиков, поскольку он совпадает с
параметром $\rho\hm=\l b\hm=\l$.

\begin{figure*} %fig1
\begin{minipage}[t]{80mm}
\vspace*{1pt}
\begin{center}
\mbox{%
\epsfxsize=73.96mm
\epsfbox{raz-1.eps}
}
\end{center}
\vspace*{-9pt}
%\label{ris:image2}
\Caption{Пример~1: cреднее число заявок в системе~(\textit{1}) и
СКО числа заявок~(\textit{2})}
\end{minipage}
%\end{figure*}
\hfill
%\begin{figure*} %fig2
\begin{minipage}[t]{81.5mm}
\vspace*{1pt}
\begin{center}
\mbox{%
\epsfxsize=71.986mm
\epsfbox{raz-2.eps}
}
\end{center}
\vspace*{-9pt}
%\label{ris:image2}
\Caption{Пример~2: cреднее число заявок в системе~(\textit{1}) и
СКО числа заявок~(\textit{2})}
\end{minipage}
\end{figure*}

\subsection{Пример~1}

Предположим, что $\l\hm=\tl$, функции $d^*(x,y|u,v)$,
$d(x|u,v)$ и $d_0(u,v)$ тождественно равны нулю
при всех~$x$, $y$, $u$ и~$v$, а функция $d(x,y|u,v)$
при всех~$u$ и~$v$ имеет вид $d(x,y|u,v)\hm=b(x)b(y)$.
Стационарные вероятности состояний этой СМО
совпадают со стационарными вероятностями состояний
СМО $M|G|1$ при следующей дисциплине обслуживания:
поступающая заявка становится на первое место в
очереди, а обслуживавшаяся ранее заявка остается на
приборе, но ее длина разыгрывается заново с той же
функцией распределения $B(x)$.

Если же $d(x,y|u,v)$, $d(x|u,v)$ и $d_0(u,v)$
тождественно равны нулю, а $d^*(x,y|u,v)\hm=b(x)b(y)$,
то с точки зрения стационарных вероятностей
со\-сто\-яний сис\-те\-ма эквивалентна СМО $M|G|1$ с
дис\-цип\-ли\-ной, при которой поступающая заявка
становится на прибор, а обслуживавшаяся ранее заявка
переходит на первое место в очереди и ее длина
разыгрывается заново с функцией распределения $B(x)$.

Для численных расчетов в примере~1 будем считать, что
время обслуживания заявки распределено по
экспоненциальному закону, т.\,е.\ $B(x)\hm=1\hm-e^{-x}$.
Тогда при обоих вариантах выбора $d(x,y|u,v)$
и $d^*(x,y|u,v)$ стационарные вероятности
состояний будут совпадать со стационарными
вероятностями состояний обычной СМО $M|M|1$.
Графики среднего и среднего квадратичного отклонения
(СКО) числа заявок в системе
в зависимости от интенсивности входящего потока~$\lambda$ представлены на
рис.~1.
Как видно, эти данные полностью совпадают
с хорошо известными для СМО $M|M|1$ результатами.

\subsection{Пример 2}

 В следующем примере предположим, что функция
$d(x|u,v)$ тождественно равна нулю при всех~$x$, $u$
и $v$, а функция $d_0(u,v)>0$ при всех $u$ и $v$.
 Это фактически означает, что с вероятностью
$d_0(u,v)$ поступает отрицательная заявка,
которая уничтожает находящуюся на приборе
заявку и вместе с ней покидает систему.
 В частности, если интенсивность поступления
в систему отрицательных заявок равна~$\gamma$,
то $d_0(u,v)= \gamma/\lambda$
при всех $u$, $v$.
 Теперь, если положить $\tl\hm=\lambda - \gamma$,
функцию $d^*(x,y|u,v)\hm=0$ при
всех $x$, $y$, $u$ и $v$, функцию
$d(x,y|u,v)=(\lambda-\gamma) b(x)b(y)/\l$
при всех~$u$, $v$, $x$ и $y$, то получим первый
вариант СМО $M|G|1$ из примера 1, в которую,
наряду с потоком интенсивности $\tl$ обычных
заявок, поступает поток отрицательных заявок
интенсивности~$\gamma$.
Второй вариант СМО из примера~1 с дополнительным
потоком отрицательных заявок получается,
если, наоборот, положить
$d^*(x,y|u,v)\hm=(\lambda\hm-\gamma) b(x)b(y) / \l$
и $d(x,y|u,v)\hm=0$ при тех же самых значениях остальных
параметров.


Положим теперь $B(x)\hm=1-e^{-x}$.  При этом пусть $\gamma\hm=1$.
 Тогда приходим к СМО $M|M|1$ с инверсионным порядком обслуживания
с (без) прерыванием обслуживания и отрицательными заявками.
 Графики среднего и СКО числа заявок в системе в зависимости от интенсивности входящего
потока~$\lambda$ представлены на рис.~2.
 Нетрудно видеть, что данные результаты
полностью совпадают с результатами, которые
легко получить по аналитическим формулам для данной СМО.


\subsection{Пример 3}

 Пусть задана функция распределения $R(x)$
не\-от\-ри\-ца\-тельной случайной величины,
определенная на интервале $(0,v)$, которая имеет ограниченную плотность
распределения~$r(x)\hm=R'(x)$.  Предположим, что функции $d^*(x,y|u,v)$,
$d_0(x|u,v)$ и $d_0(u,v)$ тождественно равны нулю
при всех~$x$, $y$, $u$, $v$.
 При этом функция $d_0^*(y|u,v)\hm=\gamma r(y)/\lambda$
при всех~$u$, $v$ и $y\hm<v$ и $d_0^*(y|u,v)\hm=0$
иначе; функция $d(x,y|u,v)\hm=(\lambda - \gamma) b(x)b(y)/\lambda$
при всех~$u$, $v$ и $x\hm>0$ и $d(x,y|u,v)\hm=0$ иначе.
 Тогда при $\tl\hm=\lambda \hm- \gamma$ систему
можно трактовать как СМО $M|G|1$ с инверсионным порядком обслуживания, в которую
с вероятностью $\gamma/\lambda$ поступают заявки,
приводящие к обслуживанию заявок на приборе
заново с распределением $R(x)$.  С~вероятностью $(\lambda \hm- \gamma)/\lambda$
в систему поступают обычные заявки.

В этом примере положим $B(x)\hm=1\hm-e^{-x}$
и $R(x)\hm=0{,}05 (1\hm- e^{- x})/(1\hm-e^{- v})$.
В~результате получаем СМО $M|M|1$ с инверсионным порядком обслуживания,
в которую с вероятностью 0,05 поступают заявки,
приводящие к обслуживанию заявок на приборе заново.
Графики среднего и СКО числа заявок в системе в зависимости
от интенсивности входящего потока~$\lambda$ представлены на рис.~3.

\begin{figure*} %fig3
\begin{minipage}[t]{80mm}
\vspace*{1pt}
\begin{center}
\mbox{%
\epsfxsize=73.119mm
\epsfbox{raz-3.eps}
}
\end{center}
\vspace*{-9pt}
\Caption{Пример~3: cреднее число заявок в системе~(\textit{1}) и
СКО числа заявок~(\textit{2})}
%\label{ris:image2}
\end{minipage}
%\end{figure*}
\hfill
%\begin{figure*} %fig4
\begin{minipage}[t]{81.5mm}
\vspace*{1pt}
\begin{center}
\mbox{%
\epsfxsize=73.126mm
\epsfbox{raz-4.eps}
}
\end{center}
\vspace*{-9pt}
\Caption{Пример~4: cреднее число заявок в системе~(\textit{1}) и
СКО числа заявок~(\textit{2})}
\label{ris:image2}
\end{minipage}
\end{figure*}




\subsection{Пример 4}


 Предположим, что функция $d^*(x,y|u,v)\hm=0$
при всех $u$, $v$, $y$, $x$, а функции
$d(x,y|u,v)$, $d_0(x|u,v)$, $d_0^*(x|u,v)$
и $d_0(u,v)$ заданы следующим образом:
$$
d(x,y|u,v)=0{,}85 e^{-(x+y)}
(e^{- v}-1)^{-1}(e^{- u}-1)^{-1}
$$
при всех $u$, $v$ и $y\hm<v$, $x\hm<u$
и $d(x,y|u,v)=0$ иначе;
$$
d_0(x|u,v)=0{,}05\fr{1}{u+1}\, \fr{ e^{-(1/(u+1))x}}{
1-e^{-(1/(u+1))u}}
$$
при всех $u$, $v$ и $x<u$ и $d_0(x|u,v)\hm=0$
иначе;
$$
d_0^*(y|u,v)=0{,}05\fr{1}{v+1 }\,
\fr{e^{-(1/(v+1))y} }{1-e^{{-(1/(v+1))}v}}
$$
при всех $u$, $v$ и $y\hm<v$ и $d_0^*(x|u,v)\hm=0$
иначе;
$d_0(u,v)=0{,}05$ при всех $u$,~$v$.

В этом случае имеет место СМО $M|G|1$ с
инверсионным порядком обслуживания, причем
допускается уход из системы вновь
поступившей или недообслуженной заявки (с равными вероятностями $0{,}05$),
а также <<выбивание>>  заявки с прибора
(также с вероятностью $0{,}05$). При данном определении функций $d_0(x|u,v)$
($d_0^*(y|u,v)$) чем больше была длина вновь
поступившей заявки (обслуживаемой заявки),
тем меньше будет средняя длина заявки, которая
останется обслуживаться на приборе.
Отметим, что определенные выше функции
подобраны таким образом, чтобы
удовлетворять достаточному условию существования
стационарного режима.

Для примера~4 графики среднего и СКО числа заявок
в системе в зависимости от интенсивности входящего потока~$\lambda$ представлены на
рис.~4.

\section{Заключение}

 В настоящей работе получены интегродифференциальные уравнения
для стационарных плотностей вероятностей СМО $M/G/1/\infty$ с входящим потоком
пуассоновского типа и инверсионным порядком обслуживания с обобщенным вероятностным
приоритетом ({LCFS GPP}) и предложен метод их решения.
 По полученным соотношениям была разработана программа, которая позволяет проводить исследование
показателей про\-из\-во\-ди\-тель\-ности рассматриваемой
СМО в широкой области изменения исходных параметров.
 Приводятся результаты численных расчетов как
для частных случаев, так и для общего случая рассматриваемой системы.

В дальнейших исследованиях предполагается
обратиться к решению задачи вычисления характеристик,
связанных с временем пребывания заявки в системе.

{\small\frenchspacing
 {%\baselineskip=10.8pt
 \addcontentsline{toc}{section}{References}
 \begin{thebibliography}{99}
\bibitem{shrage} %1
\Au{Schrage L.} A~proof of the
optimality of the shortest remaining processing
time discipline~//
Oper.\ Res., 1968. Vol.~16. P.~687--690.



\bibitem{aaa1} %2
\Au{Нагоненко В.\,А.}
О~характеристиках одной нестандартной системы
массового обслуживания. I, II~//
Изв.\ АН СССР. Технич.\ кибернет., 1981.
№\,1. С.~187--195; №\,3. С.~91--99.

\bibitem{aaa3} %3
\Au{Нагоненко В.\,А., Печинкин~А.\,В.}
О~большой загрузке в системе с инверсионным
обслуживанием и вероятностным приоритетом~//
Изв.\ АН СССР. Технич.\ кибернет., 1982. №\,1. С.~86--94.

\bibitem{aaa2} %4
\Au{Печинкин А.\,В.} Об одной
инвариантной системе массового обслуживания~//
Math.\ Operationsforsch.\ und Statist.
Ser.\ Optimization, 1983. Vol.~14. №\,3. S.~433--444.



\bibitem{aaa4} %5
\Au{Нагоненко~В.\,А., Печинкин~А.\,В.}
О~малой загрузке в системе с инверсионным порядком
обслуживания и вероятностным приоритетом~//
Изв.\ АН СССР. Технич.\ кибернет., 1984. №\,6. С.~82--89.

\bibitem{av1} %6
\Au{Печинкин А.\,В., Стальченко И.\,В.}
Система MAP$/G/1/\infty$ с инверсионным порядком
обслуживания и вероятностным приоритетом,
функционирующая в дискретном времени~//
Вестник Российского ун-та дружбы народов.
Сер.\ Математика. Информатика. Физика, 2010.
№\,2. С.~26--36.

\bibitem{av2} %7
\Au{Касконе А., Манзо~Р., Печинкин~А.\,В., Салерно~С.}
Система MAP$/G/1/\infty$ в дискретном
времени с инверсионной вероятностной дисциплиной
обслуживания~//
Автоматика и телемеханика, 2010. №\,12. С.~57--69.

\bibitem{av3} %8
\Au{Милованова Т.\,А., Печинкин А.\,В.}
Стационарные характеристики системы обслуживания с
инверсионным порядком обслуживания, вероятностным
приоритетом и гистерезисной политикой~//
Информатика и её применения, 2013. Т.~7. Вып.~1. С.~22--36.

\bibitem{ppav} %9
\Au{Бочаров  П.\,П., Печинкин~А.\,В.}
Теория массового обслуживания.~--- М.: РУДН, 1995.
529~с.

\bibitem{jerri} %10
\Au{Jerri A.}
Introduction to integral equations with
applications.~--- N.Y.: John Wiley \& Sons, 1999.
433~p.

\vspace*{-1pt}

\bibitem{wh} %11
\Au{Press W.\,H., Teukolsky~S.\,A.,
Vetterling~W.\,T., Flannery~B.\,P.}
Numerical recipes:
The art of scientific computing.  3rd ed. New York: Cambridge University
Press, 2007. 1235~p.

\vspace*{-1pt}

\bibitem{adav} %12
\Au{Полянин А.\,Д., Манжиров~А.\,В.}
Справочник по интегральным уравнениям. ---
М.: Физматлит, 2003. 608~с.
%Chapman \& Hall, CRC Press, 2008.




 \end{thebibliography}

 }
 }

\end{multicols}

\vspace*{-9pt}

\hfill{\small\textit{Поступила в редакцию 17.06.14}}

%\newpage

\vspace*{10pt}

\hrule

\vspace*{2pt}

\hrule

%\vspace*{12pt}

\def\tit{STATIONARY DISTRIBUTION IN~A~QUEUEING SYSTEM
WITH~INVERSE SERVICE ORDER AND~GENERALIZED
PROBABILISTIC PRIORITY}

\def\titkol{Stationary distribution in a queueing system
with inverse service order and generalized
probabilistic priority}

\def\aut{L.\,A.~Meykhanadzhyan$^1$, T.\,A.~Milovanova$^1$, A.\,V.~Pechinkin$^2$,
and~R.\,V.~Razumchik$^{1,2}$}

\def\autkol{L.\,A.~Meykhanadzhyan, T.\,A.~Milovanova, A.\,V.~Pechinkin,
and~R.\,V.~Razumchik}

\titel{\tit}{\aut}{\autkol}{\titkol}

\vspace*{-9pt}

\noindent
$^1$Peoples' Friendship University of Russia,
6~Miklukho-Maklaya Str., Moscow 117198, Russian Federation

\noindent
$^2$Institute of Informatics Problems, Russian Academy of Sciences,
44-2 Vavilov Str., Moscow 119333, Russian\\
$\hphantom{^1}$Federation


\def\leftfootline{\small{\textbf{\thepage}
\hfill INFORMATIKA I EE PRIMENENIYA~--- INFORMATICS AND
APPLICATIONS\ \ \ 2014\ \ \ volume~8\ \ \ issue\ 3}
}%
 \def\rightfootline{\small{INFORMATIKA I EE PRIMENENIYA~---
INFORMATICS AND APPLICATIONS\ \ \ 2014\ \ \ volume~8\ \ \ issue\ 3
\hfill \textbf{\thepage}}}

\vspace*{3pt}


\Abste{Consideration is given to $M|G|1$ type queueing system.
Inverse service order with generalized probabilistic
priority is implemented in the system.
It is assumed that at any instant, the remaining service
time of each customer residing in the system is known.
Upon arrival of a new customer, the system finds out its service time
and compares it with the remaining service
time of the currently served customer. The result of this comparison
leads to one of the cases: one of them
enters the server and another occupies the first place in the queue;
one of them leaves the system and another enters the server; or
both leave the system.
In each case when customer remains in the system,
its remaining service time may be updated.
An analytical method that allows computing
stationary performance characteristics related to the number
of customers in the system is presented.
Numerical examples based on the developed mathematical relations are provided.}

\KWE{queueing system; special discipline; LIFO; probabilistic priority; general service time}

\DOI{10.14357/19922264140304}

\vspace*{-9pt}

\Ack
\noindent
The research was partially supported by the Russian Foundation for
Basic Research (grant No.\,13-07-00223).

%\vspace*{3pt}

  \begin{multicols}{2}

\renewcommand{\bibname}{\protect\rmfamily References}
%\renewcommand{\bibname}{\large\protect\rm References}

{\small\frenchspacing
 {%\baselineskip=10.8pt
 \addcontentsline{toc}{section}{References}
 \begin{thebibliography}{99}


\bibitem{shrage-1} %1
\Aue{Schrage, L.} 1968.
A~proof of the optimality of
the shortest remaining processing time discipline.
\textit{Oper.\ Res.} 16:687--690.

\bibitem{aaa1-1} %2
\Aue{Nagonenko, V.\,A.} 1981.
O~kharakteristikakh odnoy nestandartnoy sistemy
massovogo obsluzhivaniya
[On the characteristics of one nonstandard queuing
system].~I, II.
\textit{Izv.\ AN SSSR. Tekhnich.\ kibernet.}
[Technical Cybernetics] 1:187--195; 3:91--99.


\bibitem{aaa3-1} %3
\Aue{Nagonenko, V.\,A., and A.\,V.~Pechinkin}. 1982.
O~bol'shoy zagruzke v sisteme s inversionnym
obsluzhivaniem i ve\-ro\-yat\-nostnym prioritetom
[On high load in the system with an inversion
procedure service and probabilistic priority].
\textit{Izv.\ AN SSSR. Tekhnich.\ kibernet.}
[Technical Cybernetics] (1):86--94.

\bibitem{aaa2-1} %4
\Aue{Pechinkin, A.\,V.} 1983.
Ob odnoy invariantnoy sisteme massovogo
obsluzhivaniya [On an invariant queuing system].
\textit{Math.\ Operationsforsch.\ und Statist.
Ser.\ Optimization} 14(3):433--444.


\bibitem{aaa4-1} %5
\Aue{Nagonenko, V.\,A., and A.\,V.~Pechinkin}. 1984.
O~ma\-loy zagruzke v sisteme s inversionnym poryadkom
obsluzhivaniya i veroyatnostnym prioritetom
[On low load in the system with an inversion
procedure service and probabilistic priority].
\textit{Izv.\ AN SSSR. Tekhnich.\ kibernet}
[{Technical Cybernetics}] (6):82--89.

\bibitem{av1-1} %6
\Aue{Pechinkin, A.\,V., and I.\,V.~Stalchenko}. 2010.
Sistema MAP$/G/1/\infty$ s inversionnym poryadkom
obsluzhivaniya i veroyatnostnym prioritetom,
funktsioniruyushchaya v diskretnom vremeni
[The MAP$/G/1/\infty$ discrete-time queueing
system with inversive service order and probabilistic
priority].
\textit{Vestnik Rossiyskogo Un-ta druzh\-by
na-}\linebreak\vspace*{-12pt}

\pagebreak

\noindent
\textit{ro\-dov. Ser.\ Matematika. Informatika. Fizika.}
[Bulletin of Peoples' Friendship University
of Russia. Ser. Mathematics. Information Sciences.
Physics] (2):26--36.

\bibitem{av2-1} %7
\Aue{Cascone, A., R.~Manzo, A.\,V.~Pechinkin,
and S.~Salerno}. 2010.
Sistema MAP$/G/1/\infty$ v diskretnom vremeni s
inversionnoy veroyatnostnoy distsiplinoy
obsluzhivaniya
[Discrete-time MAP$/G/1/\infty$ system with inversive
probabilistic servicing discipline].
\textit{Avtomat.\ i Telemekh.}
[{Automation Remote Control}] (12):57--69.

\bibitem{av3-1} %8
\Aue{Milovanova, T.\,A., and A.\,V.~Pechinkin}. 2013.
Sta\-tsi\-o\-nar\-nye kharakteristiki sistemy obsluzhivaniya
s inversionnym poryadkom obsluzhivaniya,
veroyatnostnym prioritetom i gisterezisnoy politikoy
[Stationary characteristics of queueing system with
an inversion procedure service probabilistic priority
and hysteresis policy]
\textit{Informatika i ee Primeneniya}~---
\textit{Inform. Appl}. 7(1):22--35.

\bibitem{ppav-1} %9
\Aue{Bocharov,  P.\,P., and A.\,V.~Pechinkin}. 1995.
\textit{Teoriya massovogo obsluzhivaniya} [Queueing theory].
Moscow: RUDN. 529~p.


\bibitem{jerri-1} %10
\Aue{Jerri, A.} 1999.
\textit{Introduction to integral equations with applications.}
{John Wiley} \& {Sons}.
433~p.

\bibitem{wh-1} %11
\Aue{Press, W.\,H., S.\,A.~Teukolsky, W.\,T.~Vetterling, and
B.\,P.~Flannery}. 2007.
\textit{Numerical recipes:
The art of scientific computing}.  3rd ed. New York: Cambridge University Press. 1235~p.

\bibitem{adav-1} %12
\Aue{Polyanin, A.\,D., and A.\,V.~Manzhirov}. 2008.
\textit{Handbook of integral equations.} 2nd ed.
Boca Raton\,--\,London: Chapman \& Hall\,/\,CRC Press. 1144~p.



\end{thebibliography}

 }
 }

\end{multicols}

\vspace*{-6pt}

\hfill{\small\textit{Received June 17, 2014}}

\vspace*{-18pt}

\Contr



\noindent
\textbf{Meykhanadzhyan Lusine A.} (b.\ 1990)~---
postgraduate, Peoples' Friendship University of Russia,
6~Miklukho-Maklaya Str., Moscow 117198, Russian Federation;
lameykhanadzhyan@gmail.com

\vspace*{3pt}



\noindent
\textbf{Milovanova Tatiana A.} (b.\ 1977)~---
Candidate of Science (PhD) in physics and
mathematics, senior lecturer,
Peoples' Friendship University of Russia,
6~Miklukho-Maklaya Str., Moscow 117198, Russian Federation;
tmilovanova77@mail.ru


\vspace*{3pt}

\noindent
\textbf{Pechinkin Alexander V.} (b.\ 1946)~--- Doctor
of Science in physics and mathematics; principal
scientist, Institute of Informatics Problems of
the Russian Academy of Sciences, 44-2 Vavilov Str.,
Moscow 119333, Russian Federation; apechinkin@ipiran.ru


\vspace*{3pt}

\noindent
\textbf{Razumchik Rostislav V.} (b.\ 1984)~--- Candidate
of Science (PhD) in physics and mathematics,
senior research fellow, Institute of Informatics
Problems of the Russian Academy of Sciences, 44-2 Vavilov Str.,
Moscow 119333, Russian Federation;
assistant professor,
Peoples' Friendship University of Russia,
6~Miklukho-Maklaya Str., Moscow 117198, Russian Federation;
rrazumchik@ieee.org


\label{end\stat}

\renewcommand{\bibname}{\protect\rm Литература}