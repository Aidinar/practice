\def\stat{plehanov}



\def\tit{ПРОЕКТИРОВАНИЕ САМОСИНХРОННЫХ СХЕМ: СТРУКТУРНЫЕ МЕТОДЫ В ИЕРАРХИЧЕСКОМ
АНАЛИЗЕ$^*$}



\def\titkol{Проектирование самосинхронных схем:
  структурные методы в иерархическом
анализе}

\def\aut{Л.\,П. Плеханов$^1$}

\def\autkol{Л.\,П. Плеханов}

\titel{\tit}{\aut}{\autkol}{\titkol}

{\renewcommand{\thefootnote}{\fnsymbol{footnote}} \footnotetext[1]
{Работа выполнена при частичной финансовой поддержке по Программе фундаментальных исследований
Президиума РАН №\,1 за 2014~г.\ (проект №\,43П) и РФФИ (проекты 13-07-12062 офи\_м и 13-07-12068 офи\_м).}}


\renewcommand{\thefootnote}{\arabic{footnote}}
\footnotetext[1]{Институт проблем информатики Российской академии наук, LPlekhanov@inbox.ru}

   \Abst{Самосинхронные схемы (СС-схеиы)
   имеют уникальные свойства независимости от задержек и
от\-ка\-зо\-безопас\-ности. Рассмотрена одна из главных проблем проектирования таких схем~---
анализ самосинхронности больших схем. В~традиционном подходе схемы анализируются
событийными методами, по переключениям элементов. Сложность вычислений в таком
подходе экспоненциально растет от размера и/или других параметров схем, что не позволяет
анализировать большинство практически значимых схем. Решение проблемы предлагается в
функциональном подходе~--- без использования переключений~--- и~иерархическом
описании схем. В~иерархическом анализе самосинхронности наряду с анализом логических
функций предлагается использовать структурные методы~--- исследование взаимосвязей
элементов и фрагментов. Такой способ позволяет резко уменьшить трудоемкость
вычислений и в итоге решить одну из главных проблем проектирования
СС-схем~--- анализ схем любого размера. Эффективность предложенных методов подтверждена
с помощью экспериментальных программных средств.}

   \KW{самосинхронные схемы; асинхронные схемы; проектирование схем; анализ
самосинхронности}

\DOI{10.14357/19922264140312}

\vspace*{9pt}


\vskip 12pt plus 9pt minus 6pt

\thispagestyle{headings}

\begin{multicols}{2}

\label{st\stat}


\section{Введение}

  Самосинхронные схемы относятся к классу асинхронных схем, то
есть схем, не име\-ющих тактовых генераторов. Русскоязычный термин
<<самосинхронная схема>> предложен в книге~[1] как аналог термина
  \textit{speed-independent}~[2]~--- <<схема, правильность функционирования
которой не зависит от величин задержек элементов>>. С практической точки
зрения использование такого термина неудобно, так как он определен для
замкнутых схем с одним начальным состоянием. (Другие близкие термины
подробнее обсуждены в~[3].) Поэтому для практических целей проектирования
СС-схем было предложено более широкое определение~[4, 5].

  \textit{Самосинхронная схема~--- это разомкнутая или замкнутая схема, при
всех реальных начальных состояниях и переходах между ними имеющая два
свойства: отсутствие состязаний при любых конечных задержках элементов и
отказобезопасность по отношению к константным залипаниям на~$0$ и~$1$
выходов элементов.}

  Так определенная схема удовлетворяет также и критериям~[1, 2] (в случае
разомкнутой схемы~--- при правильном замыкании). При реализации на чипе
самосинхронность может зависеть от задержек в трассах. Поэтому на
функционально-логическом уровне разработки свойства, указанные в
определении, становятся необходимыми условиями и должны быть обеспечены
схемотехническими решениями. Но и в случае возможного нарушения
самосинхронности от задержек в трассах есть реальные способы
(схемотехнические и топологические) обеспечения самосинхронности~[5].

  Уникальные свойства СС-схем, присущие им по определению, имеют и
уникальные следствия. Они обеспечивают максимально широкий диапазон
правильного функционирования, опреде\-ля\-емый только физическими (не
схемотехническими) возможностями переключений элементов, что
недостижимо для других типов схем. Устойчивость работы СС-схем при малых
напряжениях питания позволяет создавать схемы с малым энер\-го\-по\-треб\-ле\-ни\-ем.
А~свойство отказобезопасности дает возможность получения высоконадежных
схем, в том числе с самопроверкой и саморемонтом.

  Разработка практических СС-схем началась в СССР группой
В.\,И.~Варшавского в 70-х гг.\ прошлого века~[6] и с запозданием на 20~лет
за рубежом~[7].\linebreak
%
  Последняя ссылка указывает на единственную зарубежную методологию
построения СС-схем \textit{Null Convention Logic} (NCL). Эта методология была
\mbox{создана} чисто интуитивно, по принципу <<запрос--от\-вет>>. Методология имеет
ряд существенных недостат\-ков: ограниченный базис реализации
(27~элементов), жесткая конвейерная структура по\-стро\-ения схем, не
допускающая альтернатив, и~др. В~результате схемы, получаемые по
методологии NCL, имеют большие затраты в транзисторах, что и отмечено
бывшими сотрудниками В.\,И.~Варшавского~[8].

  Методология разработки СС-схем в ИПИ РАН опирается на математические
методы Маллера и Варшавского и развивает методы группы
В.\,И.~Варшавского. Одним из результатов разработок ИПИ РАН являются
22~патента РФ и 2 США в области СС-схемотехники.

  Методология ИПИ РАН позволяет создавать СС-схе\-мы в тысячи элементов
как на базовых мат\-рич\-ных кристаллах (БМК), так и в заказном исполнении.
Методология дает возможность сравнения множества вариантов и выбора
оптимального. Подробный сравнительный анализ NCL-ме\-то\-до\-ло\-гии,
выполненный в ИПИ РАН~[9], показывает, что схемы NCL уступают
идентичным по функционированию схемам ИПИ РАН по всем показателям:
затратам в транзисторах, быстродействию и энергопотреблению.

  В настоящее время институтом в содружестве с другими учреждениями (НТЦ
МИЭТ, \mbox{НИИСИ} РАН) разработаны, изготовлены и испытаны ряд
изделий на БМК и в заказном формате~[10--12]. Работы по
развитию СС-схем продолжаются~[9, 13--15].

  Одна из главных проблем проектирования\linebreak СС-схем заключается
в необходимости применения специальных математических и других методов,
требующих больших, а подчас огромных вычислитель\-ных ресурсов.
Существующие классические методы ограничивают возможности анализа и
синтеза СС-схем размером до нескольких десятков элементов.
%
  Поэтому актуальной является задача разработки и реализации новых методов,
альтернативных существующим, с целью довести возможности проектирования
до реальных потребностей, т.\,е.\ до схем любого размера. Одним из путей
решения этой задачи могут служить методы анализа и синтеза, основанные на
функциональном подходе~\cite{5-p, 14-p}.

  Данная статья продолжает развитие методов, приведенных в докладе на
  МЭС-2012~\cite{16-p}.

\section{Проблемы анализа в~функциональном подходе}

  Как показано в~\cite{5-p, 16-p}, проблему полного анализа СС-схем можно
решить только иерархическим способом. В~этом способе на нижнем уровне
используются описания фрагментов в логических уравнениях, а на верхних
уровнях~--- взаимосвязи фрагментов и результаты анализа более низких
уровней.

  В классическом подходе, основанном на событийных описаниях замкнутых
схем (событие~--- изменение состояния схемы, в частности изменение одного
сигнала), иерархического анализа пока не предложено.

  В рамках функционального подхода (разомкнутое представление схем)
анализ проводится иерархически, с разделением схем на фрагменты. Для
каждого фрагмента должны проверяться оба требования определения СС-схем:
отказобезопасность и отсутствие состязаний. При этом анализ фрагментов
нижнего уровня и анализ на верхних уровнях делаются по-раз\-ному.

  Основные проблемы анализа здесь возникают на нижнем уровне. На нем с
неизбежностью необходимо использовать полные описания фрагментов в
уравнениях. Уравнения должны учитывать, помимо функционирования, все
возможные реальные состояния и переходы между ними.

  Для учета всех состояний уравнения элементов записываются в зависимости
от параметров анализа~\cite{15-p, 16-p}: независимых переменных
информационных входов и переменных памяти. Для проверки
отказобезопасности используется индицирование сигналов~[1], для определения
состязаний предложен способ проверочных функций для каждого
  элемента~\cite{5-p, 16-p}.

  Созданная по этому методу программа \mbox{ФАЗАН}~\cite{17-p} показала
правильное выполнение анализа, а также выявила ряд трудностей. Выяснилось,
что нахождение индицируемости сигналов выполняется достаточно быстро
даже при большом чис\-ле параметров анализа и не представляет реальной
проб\-ле\-мы.
%
  Основная трудность заключается в определении состязаний. Проверочные
функции и часть уравнений в общем случае имеют увеличенное число
аргументов, поскольку добавляются изопеременные, необходимые для
выявления состязаний. В~результате определение мо\-но\-тон\-ности проверочных
функций становится громоздким и ограничивает размер анализируемого
фрагмента нижнего \mbox{уровня}.

  Простым решением этой проблемы может быть построение нижнего уровня
схем из фрагментов небольшого размера, что вполне реализуемо. Однако это не
всегда удобно и ограничивает маневр разработчика.

  Задачу радикального уменьшения трудоемкости анализа СС-схем в
функциональном подходе можно решить привлечением структурных методов.

\section{Структурные методы в~иерархическом анализе}

  Идея структурных методов состоит в том, что при анализе состязаний можно
отказаться от вычисления логических функций, а вместо этого выявить
элементарные ячейки памяти и анализировать взаимосвязи этих ячеек и
подключенных к ним сигналов.

  Необходимым требованием отсутствия состязаний станет блокировка (запрет
записи) запоминающих ячеек на время изменения их входов. Двухфазный
характер работы СС-схем (чередование \mbox{рабочей} фазы и вспомогательной~---
спейсера) обеспечивает полную возможность для проверки этого требования.

  Поскольку в СС-схе\-мах допустимы любые конечные задержки элементов, то
в пределах одной фазы нельзя допустить и блокировку, и изменение входов
ячеек. Поэтому необходимое требование выполняется с разделением по фазам: в
фазе, в которой входы запоминающих ячеек меняются, делается блокировка, в
другой фазе, когда эти входы не изменяются, разрешается перезапись. Такой
порядок и делает возможным анализ состязаний структурным методом.

\subsection{Анализ на верхних уровнях}

  Описание схемы на любом верхнем уровне содержит только фрагменты,
которые были успешно проверены ранее.

  Иерархический анализ на верхних уровнях подробно изложен
  в~\cite{5-p, 16-p}.

  Предложенный метод был реализован и проверен с помощью
экспериментальной программы ЛИМАН. Испытания программы на схемах
средней сложности (8-бит\-ный микропроцессор <<Мик\-ро\-ядро>>) показали
высокую эффективность метода. Анализ на любом из верхних уровней иерархии
с полнотой по всем состояниям и переходам занимает сотые или десятые доли
секунды.

  Результаты анализа программой ЛИМАН в сравнении с событийным методом
приведены в разд.~4.

\subsection{Анализ на~нижнем уровне}

  В отличие от верхних уровней, фрагмент на нижнем уровне описания
содержит логические уравнения элементов и, возможно, другие фрагменты,
ранее прошедшие анализ. Совокупность всех логических уравнений фрагмента
будем называть его \textit{логической частью}.

  Предлагается описание логической части представить в виде
взаимосвязанных фрагментов\linebreak для иерархического анализа, полностью исключив
при этом трудоемкий анализ состязаний нижне-\linebreak го уровня. Тем самым логическая
часть может быть проанализирована как фрагмент верхнего\linebreak уровня.
{ %\looseness=1

}

  На первом шаге определяются необходимые для дальнейшего атрибуты
интерфейса~\cite{5-p, 16-p} логической части по ее взаимосвязям: СС-ти\-пы ее
входов и выходов (фазовые, нефазовые и~др.), значения входных спейсеров.

  На следующем шаге функциональным методом~\cite{3-p} проверяется
индицирование всех внут\-рен\-них сигналов логической части на ее фазовых
\mbox{выходах}. Как упоминалось выше, это действие не представляет больших
вычислительных трудностей. Попутно на этом шаге вычисляются и параметры
интерфейса, необходимые для дальнейшего: значения спейсеров фазовых
выходов, списки индицируемости.

  Далее в логической части следует выделить комбинационную часть (КЧ) и
отдельно все бистабильные ячейки (БСЯ). (Довольно редко используемые
многостабильные ячейки для простоты не рас\-смат\-ри\-ва\-ют\-ся.) Выделение
производится по взаимосвязям элементов, т.\,е.\ уже структурным способом.

  Одну БСЯ будут составлять два элемента с перекрестными связями~--- с
выхода одного элемента на один из входов другого. Элементы, не попавшие в
БСЯ, будут относиться к~КЧ.

  И КЧ, и все БСЯ затем оформляются как фрагменты для иерархического
анализа. Все требуемые при этом атрибуты интерфейсов новых фрагментов
определяются либо непосредственно, либо на основании предыдущих шагов.

  В соответствии с общим правилом анализа самосинхронности~\cite{5-p, 16-p}
каждый новый фрагмент должен быть проверен на отказобезопасность и
отсутствие состязаний внутренних сигналов.

  Отказобезопасность фрагмента обеспечивается индицированием всех его
внутренних сигналов на его внешних фазовых выходах. Ранее на одном из
шагов была проверена индицируемость внутренних сигналов логической части.
По свойству транзитивности индикации, если внутренний сигнал фрагмента
индицируется на выходах логической части, то он с необходимостью
индицируется и на выходах фрагмента, так как другого пути для его индикации
нет.

  Таким образом, после проверки индикации логической части все новые ее
фрагменты также становятся проверенными на отказобезопасность.

  \begin{figure*} %fig1
  \vspace*{1pt}
\begin{center}
\mbox{%
\epsfxsize=125.496mm
\epsfbox{ple-1.eps}
}
\end{center}
\vspace*{-9pt}
  \Caption{Ячейка сдвигового регистра}
   \end{figure*}

  Отсутствие состязаний внутренних сигналов в БСЯ априорно следует из их
структуры: внутренних сигналов всего два~--- это сигналы, свя\-зы\-ва\-ющие
последовательно два элемента в кольцо. Такие сигналы не могут вызвать
состязаний. Состязания могут появиться из-за внешних сигналов БСЯ, но их
проверка осуществляется уже на более высоком уровне иерархии.

  Для КЧ отсутствие состязаний следует из серии доказанных в~[1] теорем для
разомкнутых схем, заканчивающихся следующим:

  <<\textbf{Утверждение~4.4.} Двухфазная комбинационная схема является
апериодической тогда и только тогда, когда она индицируема>>.

  В цитируемой книге термин <<апериодическая>> является синонимом
термина <<самосинхронная>>, а под индицируемой понимается схема, все
внутренние элементы которой индицируются на внешних выходах. Поэтому КЧ
не нуждается в отдельной проверке на отсутствие состязаний.

  В результате описанной процедуры логическая часть будет представлена как
соединение фрагментов, уже прошедших анализ.

  Последним шагом предлагаемого метода будет иерархический анализ сначала
логической части, затем целиком рассматриваемого фрагмента нижнего уровня.

  Можно заметить, что вычислительная сложность этого метода линейно
зависит от количества элементов и их взаимосвязей. Такая зависимость не
порождает громоздких вычислений на ЭВМ рассматриваемой части задачи.

  Рассмотрим пример схемы ячейки сдвигового регистра~\cite{18-p} (рис.~1).
  Здесь и далее знак <<$\wedge$>> означает отрицание, знак <<$\vee$>>~---
операцию ИЛИ, отсутствие знака~--- операцию~И.

  Схема задается уравнениями:
  \begin{align}
  I_2&= \wedge C\,;\label{e1-p}\\
  Q_1 &= \wedge ((D_1 \vee I_2) Q_2)\,;
  \label{e2-p}\\
  Q_2 &= \wedge ((D_2 \vee I_2) Q_1)\,;
  \label{e3-p}\\
  Y_1 & = \wedge (Q_1 E I_2 \vee Y_2)\,;
  \label{e4-p}\\
  Y_2& = \wedge (Q_2 E I_2 \vee Y_1)\,;
  \label{e5-p}\\
  I_1& = \notag\\
&\hspace*{-7.5mm}=\!\wedge ((Y_1 \!\vee\! Q_1) (D_1\!\vee\!Q_2\!\vee\!I_2)
(D_2\!\vee\!Q_1\!\vee\!I_2) (Y_2\!\vee\!Q_2)).\!\!\!
  \label{e6-p}
  \end{align}
  Здесь $C$~--- управляющий сигнал; $E$~--- сигнал дополнительной
блокировки.

  В данном случае вся схема составляет логическую часть. Анализ индикации
показывает, что все ее элементы индицируются: (2)--(5)~--- на индикаторном
выходе~$I_1$, инвертор~(1)~--- на выходе~$I_2$.

  Схема разбивается на три фрагмента: БСЯ-1~--- элементы~(2) и~(3),
  БСЯ-2~--- элементы~(4) и~(5) и КЧ, в которую входят инвертор~(1) и
индикатор~(6). Все фрагменты являются самосинхронными, так как прошли
общую проверку на индицирование.

  Проверка самосинхронности всей схемы будет состоять в анализе
межсоединений фрагментов иерархическим методом~\cite{15-p, 16-p}, для чего
требуется вычислить параметры блокировки нефазовых сигналов $Q_1$, $Q_2$,
$Y_1$, $Y_2$. Здесь входы и выходы БСЯ-1 и нефазовые входы индикатора
блокируются в фазе спейсера сигналом~$C$ с задержкой. Входы и выходы
  БСЯ-2 блокируются в рабочей фазе сигналом~$C$ также с задержкой и
сигналом~$E$ без задержки.

  Опуская подробности, можно сделать вывод, что все соединения фрагментов
корректны по отсутствию состязаний. Сигнал~$E$, вследствие задержки
блокировки БСЯ-2, следует подключать к сигналу~$I_2$ последующей ячейки
регистра, если таковая предусмотрена.

\section{Сравнение результатов иерархического и событийного
анализа}

  Для анализа была выбрана СС-схема 8-бит\-но\-го
микропроцессора <<Микроядро>>, ранее подробно проанализированная
событийными методами~\cite{19-p}. Верхний уровень схемы показан на рис.~2
(на рисунке не показан управляющий блок, задающий режимные константы и не
подлежащий анализу).

  \begin{figure*} %fig2
  \vspace*{1pt}
\begin{center}
\mbox{%
\epsfxsize=130.668mm
\epsfbox{ple-2.eps}
}
\end{center}
\vspace*{-9pt}
  \Caption{Верхний уровень микропроцессора <<Микроядро>>}
  \end{figure*}

  Иерархическая структура схемы по уровням и именам блоков выглядит
следующим образом (в скобках даны размеры блоков в вентилях).

  \noindent
  MicroCore (892)

  \noindent
  1~--- ROM (81)

     2~--- GI2 (3), ROM\_REG (28), PC\_S (32),

  \hspace*{4mm}ROM\_DC (18)

\noindent
1~--- DCF (11)

   2~--- INV (1), GI2 (3), R0RE10N (6), INV (1)

  \noindent
  1~--- CLK\_MC (123)

  \noindent
  1~--- MUL (448)

2~--- GI2 (3), INV (1), INV (1), GI2 (3), INV (1),

\hspace*{4mm}GI2 (3), GI3 (4), SSMULT (337)

  \hspace*{2mm}3~--- SSSM (12), MINUSA (33), SSSM (12),

  \hspace*{4mm}SSSM (12), SSDC (48), SUM20 (23),

  \hspace*{4mm}SSSM1 (6), SUM21 (19), SUM22 (22),

  \hspace*{4mm}MULIND (51), PARTP0 (17), PARTP1 (34),

  \hspace*{4mm}PARTP2 (48)

     2~--- R1R10 (64)

  \hspace*{2mm}3~--- R1RE11 (6), R1RE11 (6), INV (1), GI4 (5),

  \hspace*{4mm}INV2 (1), GI2 (3), GI3 (4), INV2 (1), INV (1),

  \hspace*{4mm}R1RE11 (6), R1RE11 (6), R1RE11 (6),

  \hspace*{4mm}R1RE11 (6), R1RE11 (6), R1RE11 (6)

     2~--- MX22 (25)

  \noindent
  1~--- ROT (152)

  2~--- R010 (4), R010 (4), R010 (4), R010 (4),

  \hspace*{4mm}GI2 (3), GI2 (3), GI4 (5), MX21R (17),

  \hspace*{4mm}SHFT43\_N (110)

  \hspace*{2mm}3~--- GI3M (3), GI2 (3), GI2 (3), INV (1), GI3 (4),

  \hspace*{4mm}INV3 (2), MX21N (25), PPTR (13), PPTR (13),

  \hspace*{4mm}PPTR (13), PPTR (13), MX310N (17)

  \noindent
  1~--- MX311 (29)

  Суммарное время анализа всех блоков схемы программой ЛИМАН (без учета
файловых операций) составило 0,84~с.

  Иерархический анализ в функциональном подходе, как указывалось ранее,
обеспечивает исчерпывающую полноту анализа, т.\,е.\ учитываются все
реальные начальные состояния и все возможные реальные сочетания значений
входов блоков и переходов между ними.

  Однократный анализ событийным методом этого же микропроцессора с
одним начальным со\-сто\-яни\-ем и одной из комбинаций входных
значений~\cite{12-p} занимает 7~с.

  Для обеспечения необходимой полноты требуется выполнить десятки тысяч
таких сеансов анализа.

  Приведенные данные показывают, какие практические трудности возникают
при проектировании СС-схем классическими (событийными) методами.

\vspace*{-4pt}

\section{Заключение}

\vspace*{-2pt}

  Структурные методы в иерархическом анализе\linebreak самосинхронности схем
состоят в том, что исследуются не уравнения элементов, а их взаимосвязи.
Применение этих методов оказалось возможным в рамках функционального
подхода, когда рассматрива\-ются разомкнутые схемы, работающие по принципу
чередования двух фаз.

  Эффективность структурных методов следует из того, что их вычислительная
сложность линейно зависит от количества элементов/фрагментов и их
взаимосвязей.

  На всех уровнях иерархического анализа выше нижнего проверки
осуществляются исключительно структурным методом. Так, анализ
  8-бит\-но\-го ядра микропроцессора <<Микроядро>> занимает десятые доли
секунды.

  На нижнем уровне одна из двух проверок~--- отсутствие состязаний~---
сводится к такому же структурному методу, что и анализ на верхних уровнях.
Вторая проверка~--- индицирование~--- заключается в последовательном
вычислении логических функций. На практике она обычно не вызывает
затруднений, особенно с учетом того, что размер фрагмента нижнего уровня
может быть выбран самим пользователем.

  Таким образом, применение структурных методов в анализе СС-схем
позволяет радикально уменьшить сложность вычислений и решить проб\-ле\-му
анализа схем любого размера (не решенную в классическом событийном
подходе).

\vspace*{-4pt}

{\small\frenchspacing
 {%\baselineskip=10.8pt
 \addcontentsline{toc}{section}{References}

 \vspace*{-2pt}
 \begin{thebibliography}{99}
 \bibitem{1-p}
 Автоматное управление асинхронными процессами в ЭВМ и дискретных системах~/ Под
ред. В.\,И.~Варшавского.~--- М.: Наука, 1986. 398~с.
 \bibitem{2-p}
 \Au{Muller D.\,E., Bartky W.\,C.} A~theory of asynchronous circuits~// Symposium (International)
on the Theory of Switching.~--- Harvard University Press, 1959. Part~1. P.~204--243.
 \bibitem{3-p}
 \Au{Плеханов Л.\,П.} Проектирование самосинхронных схем: функциональный подход~//
Проблемы разработки перспективных микро- и наноэлектронных сис\-тем: IV Всеросс.
 научно-технич. конф. (МЭС-2010): Сб. науч. тр.~--- М.: ИППМ РАН, 2010.
С.~424--429.
 \bibitem{4-p}
 \Au{Плеханов Л.\,П.} О~свойстве самосинхронности цифровых электронных схем~//
Системы и средства информатики, 2011. Вып.~21. №\,1. С.~84--91.
 \bibitem{5-p}
 \Au{Плеханов Л.\,П.} Основы самосинхронных электронных схем.~--- М.: Бином.
Лаборатория знаний, 2013. 208~с.
 \bibitem{6-p}
 Апериодические автоматы~/ Под ред. В.\,И.~Варшавского.~--- М.: Наука, 1976. 423~с.
 \bibitem{7-p}
 \Au{Fant K.\,M., Brandt S.\,A.} NULL convention logic~// Theseus Research: Technical Papers.
P.~1--26. {\sf http://www.theseusresearch.com/NCLPaper01.htm}.
 \bibitem{8-p}
 \Au{Taubin A., Cortadella J., Lavagno~L., Kondratyev~A., Peeters~A.} Design automation of
 real-life asynchronous devices and systems~// Foundations and Trends in Electronic Design
Automation, 2007. Vol.~2. No.\,1. P.~1--133.
 \bibitem{9-p}
 \Au{Соколов И.\,А., Степченков Ю.\,А., Бобков~С.\,Г. и~др.} Базис реализации супер-ЭВМ
эксафлопного класса~// Информатика и её применения, 2014. Т.~8. Вып.~1. С.~45--70.
 \bibitem{10-p}
 \Au{Степченков Ю.\,А., Петрухин В.\,С., Дьяченко~Ю.\,Г.} Опыт разработки
самосинхронного ядра на базовом матричном кристалле~// Проблемы разработки
перспективных микро- и наноэлектронных систем: I~Всеросс. научно-технич. конф.
 (МЭС-2005): Сб. научных трудов.~--- М.: ИППМ РАН, 2005. С.~235--242.
 \bibitem{11-p}
 \Au{Степченков Ю.\,А., Дьяченко Ю.\,Г., Бобков~С.\,Г.} Квазисамосинхронный
вычислитель: методологические и алгоритмические аспекты~// Проблемы разработки
перспективных микро- и наноэлектронных систем: III Всеросс. науч.-технич. конф.
 (МЭС-2008): Сб. науч. тр.~--- М.: ИППМ РАН, 2008. С.~441--446.
 \bibitem{12-p}
 \Au{Степченков Ю.\,А., Дьяченко Ю.\,Г., Рождественский~Ю.\,В., Морозов~Н.\,В.,
Степченков~Д.\,Ю.} Самосинхронный вычислитель для высоконадежных применений~//
Проблемы разработки перспективных микро- и наноэлектронных систем: IV Всеросс.
 науч.-технич. конф. (МЭС-2010): Сб. науч. тр.~--- М.: ИППМ РАН, 2010.
 С.~418--423.
 \bibitem{13-p}
 \Au{Степченков Ю.\,А., Денисов А.\,Н., Дьяченко~Ю.\,Г., Гринфельд~Ф.\,И.,
Филимонов~О.\,П., Морозов~Н.\,В., Степченков~Д.\,Ю.} Библиотека элементов для
проектирования самосинхронных полузаказных БМК микросхем серий 5503/5507 и
5508/5509.~--- М.: ИПИ РАН, 2013. 391~с.
 \bibitem{14-p}
 \Au{Бобков С.\,Г., Горбунов М.\,С., Дьяченко~Ю.\,Г., Рож\-дественский~Ю.\,В.,
Степченков~Ю.\,А., Сурков~А.\,В.} Использова\-ние самосинхронной логики для снижения
потребляемой мощности и повышения надежности микропроцессоров~// Проблемы
разработки перспективных микро- и наноэлектронных систем: VI~Всеросс.
 науч.-технич. конф. (МЭС-2014): Сб. науч. тр.~--- М.: ИППМ РАН, 2014.
 С.~254--257.
 \bibitem{15-p}
 \Au{Степченков Ю.\,А., Рождественский~Ю.\,В., Дьяченко~Ю.\,Г., Морозов~Н.\,В.,
Степченков~Д.\,Ю.,\linebreak Сурков~А.\,В.} Самосинхронное устройство умно\-же\-ния-сло\-же\-ния
гигафлопсного класса: варианты реализации~// Проблемы разработки перспективных микро- и
наноэлектронных систем: VI~Всеросс. науч.-технич. конф. (МЭС-2014): Сб. науч.
тр.~--- М.: ИППМ РАН, 2014. С.~258--263.
 \bibitem{16-p}
 \Au{Плеханов Л.\,П.} Функциональный метод анализа самосинхронных схем любого
размера~// Проблемы разработки перспективных микро- и наноэлектронных систем:
V~Всеросс. науч.-технич. конф. (МЭС-2012): Сб. науч. тр.~--- М.: ИППМ РАН, 2012.
С.~107--112.
 \bibitem{17-p}
 \Au{Плеханов Л.\,П.} Программа анализа самосинхронных схем функциональным методом
(ФАЗАН). Свидетельство о государственной регистрации программы для ЭВМ
№\,2011611102 от 01.02.2011.
 \bibitem{18-p}
 \Au{Степченков Ю.\,А., Дьяченко Ю.\,Г., Рождественский~Ю.\,В.,
Рождественскене~А.\,В.} Разряд самосинхронного регистра сдвига. Патент на изобретение
2319232 (РФ). Приоритет от 10.03.2008.
 \bibitem{19-p}
 \Au{Степченков Ю.\,А., Дьяченко Ю.\,Г., Рождественский~Ю.\,В., Морозов~Н.\,В.,
Степченков~Д.\,Ю.} Разработка вычислителя, не зависящего от задержек элементов~//
Системы и средства информатики, 2010. Вып.~20. №\,1. С.~5--23.
 \end{thebibliography}

 }
 }

\end{multicols}

\vspace*{-9pt}

\hfill{\small\textit{Поступила в редакцию 10.07.14}}

%\newpage

\vspace*{10pt}

\hrule

\vspace*{2pt}

\hrule

%\vspace*{12pt}

\def\tit{DESIGN OF SELF-TIMED CIRCUITS: STRUCTURAL METHODS IN~HIERARCHICAL ANALYSIS}

\def\titkol{Design of self-timed circuits: Structural methods in~hierarchical analysis}

\def\aut{L.~Plekhanov}

\def\autkol{L.~Plekhanov}

\titel{\tit}{\aut}{\autkol}{\titkol}

\vspace*{-12pt}

\noindent
Institute of Informatics Problems, Russian Academy of Sciences,
44-2 Vavilov Str., Moscow 119333, Russian Federation

\def\leftfootline{\small{\textbf{\thepage}
\hfill INFORMATIKA I EE PRIMENENIYA~--- INFORMATICS AND
APPLICATIONS\ \ \ 2014\ \ \ volume~8\ \ \ issue\ 3}
}%
 \def\rightfootline{\small{INFORMATIKA I EE PRIMENENIYA~---
INFORMATICS AND APPLICATIONS\ \ \ 2014\ \ \ volume~8\ \ \ issue\ 3
\hfill \textbf{\thepage}}}

\vspace*{3pt}


\Abste{Self-timed circuits have unique properties of the delay-independence and
fail-safe. One of the major problems of circuits design, self-timed analysis of
large circuits, is considered. In the traditional approach,
circuits\linebreak}

\Abstend{are analyzed by event
methods with elements switches. Computational complexity in this approach
increase
exponentially with the size and/or other circuit parameters, which does not
allow analyzing the most practically important circuits. The solution is proposed in
the functional approach, without using switches, and in the hierarchical description
of circuits. In the hierarchical analysis along with the analysis of logical functions,
the author
proposes to use structural methods, i.\,e., to study the interaction of elements and
fragments. This method allows reducing the complexity of
calculations dramatically and thus solves one of the major problems of self-timed circuits design~---
analysis of circuits of any size. Efficiency of the suggested methods is confirmed
using the experimental software.}

  \KWE{self-timed circuits; asynchronous circuits; circuit design;
  self-timed analysis}


\DOI{10.14357/19922264140312}

\Ack
\noindent
  The work was performed with partial financial support of the Program of Basic
Research of the Presidium of the Russian Academy of Sciences No.\,1 for
2014 (project No.\,43P) and Russian Foundation for Basic Research
(projects 13-07-12062 and 13-07-12068).

%\vspace*{3pt}

  \begin{multicols}{2}

\renewcommand{\bibname}{\protect\rmfamily References}
%\renewcommand{\bibname}{\large\protect\rm References}

{\small\frenchspacing
 {%\baselineskip=10.8pt
 \addcontentsline{toc}{section}{References}
 \begin{thebibliography}{99}
\bibitem{1-p1}
Varshavskiy, V.\,I., ed. 1986.
\textit{Avtomatnoe upravlenie asinkhronnymi protsessami v EVM i
diskretnykh sistemakh} [Automata control of asynchronous processes
in computers and discrete
systems]. Moscow: Nauka. 398~p.
\bibitem{2-p1}
\Aue{Muller, D.\,E., and W.\,C.~Bartky}.
1959. A~theory of asynchronous circuits. \textit{Symposium (International) on the Theory
of Switching Proceedings}.  Harvard University Press. 1:204--243.
\bibitem{3-p1}
\Aue{Plekhanov, L.\,P.} 2010.
Proektirovanie samosinkhronnykh skhem: Funktsional'nyy podkhod [The
design of self-timed circuits: Functional approach].
\textit{Tr. IV Vseross. nauch.-tekhnich.
konf. ``Problemy razrabotki perspektivnykh mikro- i nanoelektronnykh sistem''
(MES-2010)} [4th Russian Scientific and Technical Conference
``Problems of the Perspective Micro- and
Nanoelectronic Systems Development'' (MES-2010) Proceedings].
Moscow. 424--429.
\bibitem{4-p1}
\Aue{Plekhanov, L.\,P.} 2011. O~svoystve samo\-sinkh\-ron\-nosti tsifrovykh elektronnykh skhem [About the
self-timed property of digital electronic circuits].
\textit{Sistemy i Sredstva Informatiki}~--- \textit{Systems and Means
of Informatics} 21(1):84--91.
\bibitem{5-p1}
\Aue{Plekhanov, L.\,P.} 2013. \textit{Osnovy samosinkhronnykh elektronnykh skhem}
[Basics of self-timed
electronic circuits].
Moscow: Binom. Laboratoriya znaniy [Binom. Laboratory of knowledge]. 208~p.
\bibitem{6-p1}
Varshavskiy, V.\,I., ed. 1976. \textit{Aperiodicheskie avtomaty}
[Aperiodic automata]. Moscow: Nauka. 423~p.
\bibitem{7-p1}
\Aue{Fant, K.\,M., and S.\,A.~Brandt}.
NULL convention logic. \textit{Theseus Research: Technical Papers}. 1--26.
Available at: {\sf http://www.theseusresearch.com/NCLPaper01.htm} (accessed July~17,
2014).
\bibitem{8-p1}
\Aue{Taubin, A., J.~Cortadella, L.~Lavagno, A.~Kondratyev, and
A.~Peeters}. 2007. Design automation of real-life asynchronous devices and systems.
\textit{Foundations and Trends in Electronic Design
Automation}.  2(1):1--133.
\bibitem{9-p1}
\Aue{Sokolov, I.\,A., Yu.\,A.~Stepchenkov, S.\,G.~Bobkov, \textit{et al.}}
2014. Bazis realizatsii super-EVM
eksaflopnogo klassa [The basis for the implementation of the super
computer of exaflops class].
\textit{Informatika i ee Primeneniya}~---
\textit{Inform. Appl.} 25(1):5--34.
\bibitem{10-p1}
\Aue{Stepchenkov, Yu.\,A., V.\,S.~Petrukhin, and Yu.\,G.~D'ya\-chen\-ko}.
2005. Opyt razrabotki samosinkhronnogo yadra na bazovom matrichnom
kristalle [The experience of developing self-timed kernel on basic matrix crystal].
\textit{Tr. I Vseross. nauchno-tekhnich. konf.
``Problemy Razrabotki Perspektivnykh Mikro- i Nanoelektronnykh Sistem'' (MES-2005)}
[1st Russian Scientific and Technical Conference ``Problems of the Perspective
micro- and nanoelectronic systems development'' (MES-2005) Proceedings].
Moscow. 235--242.
\bibitem{11-p1}
\Aue{Stepchenkov, Yu.\,A., Yu.\,G.~D'yachenko, and S.\,G.~Bob\-kov}.
2008. Kvazisamosinkhronnyy vychislitel': Metodologicheskie i
algoritmicheskie aspekty [Quasi-self-timed calculator:
Methodological and algorithmic aspects].
Tr. III Vseross. nauch.-tekhnich.
konf. ``Problemy Razrabotki Perspektivnykh Mikro- i Nanoelektronnykh Sistem''
(MES-2008)
[3rd Russian Scientific and Technical Conference ``Problems of the
Perspective Micro- and Nanoelectronic Systems Development'' (MES-2008) Proceedings].
Moscow. 441--446.
\bibitem{12-p1}
\Aue{Stepchenkov, Yu.\,A., Yu.\,G.~D'yachenko, Yu.\,V.~Rozh\-dest\-ven\-skiy,
N.\,V.~Morozov, and D.\,Yu.~Step\-chen\-kov}. 2010. Samosinkhronnyy
vychislitel' dlya vysokonadezhnykh primeneniy [Self-timed
calculator fof high reliable applications].
\textit{Tr. IV Vseross. nauch.-tekhnich.
konf. ``Problemy Razrabotki Perspektivnykh Mikro- i Nanoelektronnykh
Sistem" (MES-2010)}
[4th Russian Scientific and Technical Conference
``Problems of the Perspective Micro- and Nanoelectronic Systems Development''
(MES-2010) Proceedings]. Moscow. 418--423.
\bibitem{13-p1}
\Aue{Stepchenkov, Yu.\,A., A.\,N. Denisov, Yu.\,G.~D'yachenko,
F.\,I.~Grinfel'd, O.\,P.~Filimonov, N.\,V.~Morozov, and
D.\,Yu.~Stepchenkov}. 2013. \textit{Biblioteka elementov dlya proektirovaniya
samosinkhronnykh poluzakaznykh BMK mikroskhem seriy 5503/5507 i 5508/5509}
[Library of
elements for designing self-timed semicustom VLSI 5503/5507 and 5508/5509]. Moscow: IPI
RAN. 391~p.
\bibitem{14-p1}
\Aue{Bobkov, S.\,G., M.\,S.~Gorbunov, Yu.\,G.~D'yachenko, Yu.\,V.~Rozhdestvenskiy,
Yu.\,A.~Stepchenkov, and A.\,V.~Surkov}. 2014. Ispol'zovanie
samosinkhronnoy logiki dlya snizheniya
potreblyaemoy moshchnosti i povysheniya nadezhnosti mikroprotsessorov
[The use of self-timed
logic to reduce power consumption and increase reliability of microprocessors].
\textit{Tr. VI~Vseross. nauch.-tekhnich. konf.
``Problemy Razrabotki Perspektivnykh Mikro- i
Nanoelektronnykh Sistem" (MES-2014)} [6th
Russian Scientific and Technical Conference ``Problems
of the Perspective Micro- and Nanoelectronic Systems Development'' (MES-2014)
Proceedings]. Moscow. 254--257.
\bibitem{15-p1}
\Aue{Stepchenkov, Yu.\,A., Yu.\,V.~Rozhdestvenskiy, Yu.\,G.~D'ya\-chen\-ko,
N.\,V.~Morozov, D.\,Yu.~Stepchenkov, and A.\,V.~Sur\-kov}. 2014.
Samosinkhronnoe ustroystvo umnozheniya-slozheniya gigaflopsnogo klassa:
Varianty realizatsii [Self-timed device for multiplication-addition of
gigaflops class]. \textit{Tr. VI~Vseross. nauch.-tekhnich. konf.
``Problemy Razrabotki
Perspektivnykh Mikro- i Nanoelektronnykh Sistem'' (MES-2014)}
[6th~Russian Scientific and
Technical Conference ``Problems of the Perspective Micro- and Nanoelectronic
Systems Development'' (MES-2014) Proceedings]. Moscow. 258--263.
\bibitem{16-p1}
\Aue{Plekhanov, L.\,P.} 2012. Funktsional'nyy metod analiza samosinkhronnykh skhem lyubogo
razmera [The functional method of analysis of self-timed circuits of any size].
\textit{Tr. V~Vseross. nauch.-tekhnich. konf. ``Problemy Razrabotki
Perspektivnykh Mikro- i Nanoelektronnykh Sistem" (MES-2012)}
[5th~Russian Scientific and Technical Conference ``Problems
of the Perspective Micro- and Nanoelectronic Systems Development''
(MES-2012) Proceedings].
Moscow. 107--112.
\bibitem{17-p1}
\Aue{Plekhanov, L.\,P.} 2011. Programma analiza samosinkhronnykh skhem
funktsional'nym
metodom (FAZAN) [The program of the analysis of self-timed circuits with the functional method
(FAZAN)].
Svidetel'stvo o gosudarstvennoy registratsii programmy dlya EVM [Sertificate of
the State Registration of the Computer Program] No.\,2011611102.
\bibitem{18-p1}
\Aue{Stepchenkov, Yu.\,A., Yu.\,G.~D'yachenko, Yu.\,V.~Rozh\-dest\-ven\-skiy,
and A.\,V.~Rozhdestvenskene}.
2008. Razryad samosinkhronnogo registra sdviga [Binary digit of self-timed shift register]. Patent
RF No.\,2319232.
\bibitem{19-p1}
\Aue{Stepchenkov, Yu.\,A., Yu.\,G.~D'yachenko, Yu.\,V.~Rozh\-dest\-ven\-skiy,
N.\,V.~Morozov, and D.\,Yu.~Step\-chen\-kov}. 2010.
Razrabotka vychislitelya, ne zavisyashchego ot zaderzhek elementov [Design
of speed-independent calculator]. \textit{Sistemy i Sredstva Informatiki}~---
\textit{Systems and Means of Informatics} 20(1):5--23.
\end{thebibliography}

 }
 }

\end{multicols}

\vspace*{-6pt}

\hfill{\small\textit{Received July 10, 2014}}

\vspace*{-18pt}

\Contrl

  \noindent
  \textbf{Plekhanov Leonid P.} (b.\ 1943)~--- Candidate of Science (PhD) in
technology, senior scientist,
Institute of Informatics Problems, Russian
Academy of Sciences, 44-2 Vavilov Str., Moscow 119333, Russian Federation;
LPlekhanov@inbox.ru


\label{end\stat}

\renewcommand{\bibname}{\protect\rm Литература}

