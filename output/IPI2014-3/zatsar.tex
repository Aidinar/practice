\def\stat{zatsar}

\def\tit{АНАЛИТИЧЕСКИЕ АСПЕКТЫ ОЦЕНКИ ЭФФЕКТИВНОСТИ В~ТЕХНОЛОГИИ
ПОДДЕРЖКИ ДЕЯТЕЛЬНОСТИ ОРГАНИЗАЦИОННОЙ СИСТЕМЫ}

\def\titkol{Аналитические аспекты оценки эффективности в технологии
поддержки деятельности организационной системы}

\def\aut{А.\,А. Зацаринный$^1$, А. П. Шабанов$^2$}

\def\autkol{А.\,А. Зацаринный, А. П. Шабанов}

\titel{\tit}{\aut}{\autkol}{\titkol}

\renewcommand{\thefootnote}{\arabic{footnote}}
\footnotetext[1]{Институт проблем
информатики Российской академии наук, AZatsarinny@ipiran.ru}
\footnotetext[2]{Институт проблем
информатики Российской академии наук, AShabanov@ibs.ru}


     \Abst{Рассматривается система нормированных и фактических показателей,
используемая в технологии поддержки деятельности организационной системы~---
ведомства, предприятия. Приводятся критерии оценки, которые обеспечивают
лицам, принимающим решения в отношении различных ситуаций, возникающих в
среде деятельности организационной системы, классификационный подход к выбору
сценария принятия решения. Классификация проводится по иерархии деятельности,
осуществляемой в организационной системе: виды деятельности в подразделениях
организационной системы, виды деятельности и деятельность организационной
системы в целом. Основной акцент представленного методического подхода к
оценке эффективности деятельности носит практический характер, учитывающий
степень угрозы для деятельности организационной системы. Авторы выделяют три
степени состояния деятельности: деятельность осуществляется в диапазоне
заданных нормированных показателей, априорно существует угроза для
продолжения деятельности и деятельность уже находится под воздействием
проявлений угрозы.}

     \KW{организационная система; технология поддержки деятельности; система
показателей; информация; принятие решений; эффективность; объекты наблюдения}

\DOI{10.14357/19922264140314}

\vspace*{9pt}

\vskip 16pt plus 9pt minus 6pt

\thispagestyle{headings}

\begin{multicols}{2}

\label{st\stat}


\section{Введение }

     В настоящее время существует \textit{проблема} сокращения времени
предоставления информации лицам, принимающим решения по предотвращению
угроз, а в случаях их реализации~--- по быстрейшей ликвидации и минимизации
потерь. Серьезность проблемы обусловлена возрастанием угроз, вызванных
техногенными, природными и человеческими факторами. Другой проблемой
является постоянный рост объемов информации,\linebreak \mbox{которую} необходимо собрать,
обработать и пред\-оста\-вить уполномоченным лицам для принятия решений по
предотвращению проявления или по ликвидации угроз~[1--3]. Существование
указанных проблем обусловило необходимость разработки научно-обоснованной
технологии поддержки деятельности организационных сис\-тем~--- ведомств и
предприятий различных форм собственности и отраслей хозяйствования. Важным
компонентом этой технологии является система показателей, предназначенная для
оценки эффективности деятельности в организационной системе. В~статье
рассматривается система нормированных и фактических показателей, используемая
в технологии поддержки деятельности организационной системы~[4].

\section{Показатели эффективности деятельности}

     Технология информационной поддержки деятельности организационной
системы~[4] характеризуется тем, что содержит следующие этапы:
     \begin{enumerate}[1.]
     \item Формирование информации:
     \begin{itemize}
  \item о нормированных показателях и состояниях объектов наблюдения с учетом
их влияния на деятельность организационной системы;
  \item о критических и допустимых показателях эффективности деятельности
организационной системы;
  \item о сценариях принятия решений в их привязке к различным ситуациям,
которые априорно могут произойти в среде деятельности организационной
системы.
  \end{itemize}
  \begin{table*}[b]\small
\begin{center}
\Caption{Показатели эффективности деятельности}
      \vspace*{2ex}

     \begin{tabular}{|l|p{130mm}|}
     \hline
\multicolumn{1}{|c|}{Показатель}&\multicolumn{1}{c|}{Описание}\\
\hline
$N$&Число видов деятельности организационной системы\\
\hline
$n = 1, 2, \ldots , N$&Номер вида деятельности организационной системы\\
\hline
$M$ &Число подразделений организационной системы\\
\hline
$m = 1, 2, \ldots , M$&Номер подразделения организационной системы\\
\hline
\multicolumn{1}{|l|}{\raisebox{-6pt}[0pt][0pt]{$D$ и $D^*$}}&Нормированный и фактический показатели состояния деятельности
организационной сис\-те\-мы в целом соответственно\\
\hline
\multicolumn{1}{|l|}{\raisebox{-6pt}[0pt][0pt]{$D_n$ и $D_n^*$}}&Нормированный и фактический показатели состояния $n$-го вида
деятельности организационной системы соответственно\\
\hline
\multicolumn{1}{|l|}{\raisebox{-6pt}[0pt][0pt]{$S_{mn}$ и $S_{nm}^*$}}&Нормированный и фактический показатели состояния $n$-го
вида деятельности в $m$-м подразделении соответственно\\
\hline
\multicolumn{1}{|l|}{\raisebox{-11pt}[0pt][0pt]{$\alpha_n$ и $\beta_{nm}$ }}&Приоритет $n$-го вида деятельности организационной
системы и приоритет $n$-го вида дея\-тельности, осуществляемой в $m$-м подразделении
организационной системы, соответст\-венно\\
\hline
\multicolumn{1}{|l|}{\raisebox{-11pt}[0pt][0pt]{$\Delta D^*$, $\Delta D_n^*$ и
$\Delta S^*_{nm}$}}&Фактические показатели эффективности
деятельности организационной системы в целом, $n$-го вида деятельности
организационной системы и $n$-го вида деятельности организационной системы в $m$-м
подразделении соответственно\\
\hline
\multicolumn{1}{|l|}{\raisebox{-16pt}[0pt][0pt]{$\Delta D_{\mathrm{крит}}$}}&Критический показатель эффективности деятельности организационной
системы в целом, снижение по сравнению с которым фактического показателя $\Delta D^*$
эффективности означает существование угрозы для деятельности организационной системы в целом и
необходимости принятия действий по ее ликвидации\\
\hline
\multicolumn{1}{|l|}{\raisebox{-16pt}[0pt][0pt]{$\Delta D_{\mathrm{доп}}$}}&Допустимый показатель эффективности деятельности организационной
системы в целом, снижение по сравнению с которым фактического показателя
$\Delta D^*$ эффективности означает возможность появления угрозы для деятельности организационной системы в
целом и необходимости принятия действий по недопущению ее появления\\
\hline
\multicolumn{1}{|l|}{\raisebox{-11pt}[0pt][0pt]{$\Delta D_{n\mbox{-}\mathrm{крит}}$}}
& Критический показатель эффективности $n$-го вида деятельности, снижение по
сравнению с которым фактического показателя $\Delta D_n^*$ эффективности означает существование
угрозы для этого вида деятельности и необходимости принятия действий по ее ликвидации\\
\hline
\multicolumn{1}{|l|}{\raisebox{-16pt}[0pt][0pt]{$\Delta D_{n\mbox{-}\mathrm{доп}}$}}&Допустимый показатель эффективности $n$-го вида деятельности,
снижение по сравнению с которым фактического показателя $\Delta D_n^*$ эффективности означает
возможность появления угрозы для этого вида деятельности и необходимости принятия действий по
недопущению ее появления\\
\hline
\multicolumn{1}{|l|}{\raisebox{-16pt}[0pt][0pt]{$\Delta S_{nm\mbox{-}\mathrm{крит}}$}}&Критический показатель эффективности $n$-го вида деятельности в
$m$-м подразделении, снижение по сравнению с которым фактического показателя $\Delta S^*_{nm}$
эффективности означает существование угрозы для этого вида деятельности в данном подразделении и
необходимости принятия действий по ее ликвидации\\
\hline
\multicolumn{1}{|l|}{\raisebox{-16pt}[0pt][0pt]{$\Delta S_{nm\mbox{-}\mathrm{доп}}$}}&Допустимый показатель эффективности $n$-го вида деятельности в $m$-м
подразделении, снижение по сравнению с которым фактического показателя
$\Delta S^*_{nm}$
эффективности означает возможность появления угрозы для этого вида деятельности в данном
подразделении и необходимости принятия действий по недопущению ее появления\\
\hline
\end{tabular}
\end{center}
\end{table*}
     \item Установление объектов наблюдения в нормированные состояния.
\item Определение фактических показателей объектов наблюдения.
     \item Формирование информации о фактических показателях и состояниях
объектов наблюдения.
     \item Проведение оценки эффективности деятель\-ности организационной
системы.
     \item Определение сценария принятия решения в зависимости от результата
оценки эффектив\-ности и от ситуации, сложившейся в среде деятельности
организационной системы.
     \item Применение сценария~--- установление объектов наблюдения,
оказывающих влияние на деятельность организационной системы, в состояния,
соответствующие командам, содержащимся в сценарии.
     \end{enumerate}

          \begin{table*}\small
     \begin{center}
     \Caption{Показатели объектов наблюдения и показатели их состояния}
     \vspace*{2ex}

     \begin{tabular}{|l|p{100mm}|}
     \hline
\multicolumn{1}{|c|}{Показатель}&\multicolumn{1}{c|}{Описание}\\
\hline
\multicolumn{1}{|l|}{\raisebox{-6pt}[0pt][0pt]{$K$}} &Число объектов наблюдения в зоне ответственности организационной системы\\
\hline
$k = 1, 2, \ldots , K$&Номер объекта наблюдения\\
\hline
$L_k$ &Число показателей $k$-го объекта наблюдения\\
\hline
$l = 1, 2, \ldots , L_k$&Номер показателя $k$-го объекта наблюдения\\
\hline
\multicolumn{1}{|l|}{\raisebox{-24pt}[0pt][0pt]{$V_k^l$ и
$V^{*l}_k$, $V^l_{nmk}$ и $V^{*l}_{nmk}$, $V_{nmk}$ и $V^*_{nmk}$}}&
Нормированный и
фактический $l$-й показатели $k$-го объекта наблюдения, нормированный и фактический
$l$-й показатели $k$-го объекта наблюдения с учетом его влияния на $n$-й вид деятельности в $m$-м
подразделении, нормированный и фактический показатели состояния \mbox{$k$-го} объекта наблюдения с
учетом его влияния на $n$-й вид де\-ятельности в $m$-м подразделении соответственно\\
\hline
\multicolumn{1}{|l|}{\raisebox{-16pt}[0pt][0pt]{$\Delta V^{*l}_{nmk}$}}&Отклонение фактического $l$-го показателя от нормированного $l$-го показателя
$k$-го объекта наблюдения, который оказывает влияние на $n$-й вид деятельности, осуществляемой в
$m$-м подразделении организационной системы\\
\hline
\multicolumn{1}{|l|}{\raisebox{-11pt}[0pt][0pt]{$\mu^l_k$ и $\gamma_{nmk}$}} &Приоритет $l$-го показателя $k$-го объекта наблюдения и
приоритет $k$-го объекта наблюдения с учетом его влияния на $n$-й вид деятельности в $m$-м
подразделении организационной сис\-те\-мы соответственно\\
\hline
\end{tabular}
\end{center}
\end{table*}

     Объектами наблюдения являются материальные и нематериальные объекты
организационной сис\-те\-мы, внешней среды и субъекты, которые влияют на
состояние деятельности организационной сис\-те\-мы или отражают результаты этой
деятельности.

Объекты наблюдения размещаются в контролируемом пространстве
и/или с возможностью удаленного управления и наблюдения над ними. В~состав
информации об объектах наблюдения входят сведения о физических, логических,
информационных, территориальных, конструктивных, организационных и других
типах связи.

     Оценка фактической эффективности видов деятель\-ности в подразделениях
организационной сис\-те\-мы, видов деятельности и деятельности организационной
системы в целом (этапы~1--5 технологии) производится с применением
нормированных, фактических, критических и допустимых показателей,
приведенных в табл.~1.



     Исходной информацией для определения по\-казателей эффективности
деятельности организационной системы являются фактические и нормированные
показатели объектов наблюдения. Нормированные показатели объектов наблюдения
определяются при проектировании ситуационных центров, уточняются в ходе
испытаний и при их вводе в промышленную эксплуатацию. Фактические показатели
объектов наблюдения опре\-де\-ляются с помощью компонентов мониторинга,
в~част\-ности центра мониторинга устойчивости информационных систем~[5].
В~табл.~2 приведены показатели объектов наблюдения и показатели их состояния.



\section{Расчетные соотношения }

     Для определения показателей эффективности деятельности организационной
системы в рас\-смат\-ри\-ва\-емой технологии используются следующие соотношения:
     \begin{itemize}
\item при оценке эффективности деятельности организационной системы в целом:
\begin{align*}
\Delta D^* &= \fr{D^*}{ D}\,;\\ %\label{e1-zts}\\
D^*& = \alpha_1D^*_1 + \alpha_2D^*_2, +\cdots + \alpha_ND^*_N\,;\\ %\label{e2-zts}\\;
      D &= \alpha_1D_1 + \alpha_2D_2, +\cdots+ \alpha_ND_N\,;
%      \label{e3-zts}
      \end{align*}
      \item при оценке эффективности видов деятельности
организационной системы:
      \begin{align*}
      \Delta D^*_n&= \fr{D^*_n}{D_n}\,;\\ %\label{e4-zts}\\
D^*_n &= \beta_{n1}S^*_{n1} +\beta_{n2}S^*_{n2} +\cdots+ \beta_{nM}S^*_{nM}\,;
%\label{e5-zts}
\\
      D_n &= \beta_{n1}S_{n1} + \beta_{n2}S_{n2} +\cdots+
\beta_{nM}S_{nM}\,;
%      \label{e6-zts}
      \end{align*}
      \item при оценке эффективности видов деятельности в
подразделениях организационной системы:
\begin{align*}
    \hspace*{-5mm}  \Delta S^*_{nm} & = \fr{S^*_{nm}}{ S_{nm}}\,;\\
%      \label{e7-zts}\\
    \hspace*{-5mm} S^*_{nm}& ={}\\
&   \hspace*{-5mm}  {}= \gamma_{nm1}V^*_{nm1} + \gamma_{nm2}V^*_{nm2}
+\cdots + \gamma_{nmK}V^*_{nmK}\,;
%     \label{e8-zts}\\
\\
\hspace*{-5mm}S_{nm} & = {}\\
&\hspace*{-5mm}{}=\gamma_{nm1}V_{nm1} + \gamma_{nm2}V_{nm2}
+\cdots+ \gamma_{nmK}V_{nmK}\,.
%\label{e9-zts}
\end{align*}
\end{itemize}
Для оценки показателей фактического состояния $k$-го объекта наблюдения,
который оказывает влияние на $n$-й вид деятельности, осуществляемой в $m$-м
подразделении организационной системы, используются следующие соотношения:

\end{multicols}

\begin{table}\small
\begin{center}
\Caption{Классификация сценариев принятия решений}
      \vspace*{2ex}

     \begin{tabular}{|l|p{100mm}|}
     \hline
\multicolumn{1}{|c|}{Классификатор сценария}&\multicolumn{1}{c|}{Описание содержания классификатора}\\
\hline
\multicolumn{1}{|l|}{\raisebox{-16pt}[0pt][0pt]{$W_{nm\mbox{-}\mathrm{норм}}$}}&Множество сценариев принятия решений, предназначенных для установления
нормированного состояния $n$-го вида деятельности в \mbox{$m$-м} подразделении организационной системы;
$n \hm= 1, 2, \ldots, N$; $m \hm= 1, 2, \ldots, M$\\
\hline
\multicolumn{1}{|l|}{\raisebox{-11pt}[0pt][0pt]{$W^{y_0}_{nm\mbox{}\mathrm{-норм}}$}} &Множество команд управления, предназначенных
для установления
нормированного состояния $n$-го вида деятельности в $m$-м подразделении
организационной системы;
$y_0 \hm= 1, \ldots, Y_{nm\mbox{-}\mathrm{норм}}$\\
\hline
\multicolumn{1}{|l|}{\raisebox{-16pt}[0pt][0pt]{$W_{\mathrm{крит}}$,
$W_{\mathrm{пред}}$ и $W_{\mathrm{план}}$}}&Множества соответственно критических,
предупреждающих и плановых сценариев принятия решений, предназначенных соответственно для
лик\-ви\-да\-ции угрозы, для предотвращения угрозы и для повышения эффективности деятельности
организационной системы в целом\\
\hline
\multicolumn{1}{|l|}{\raisebox{-21pt}[0pt][0pt]{$W^{q_1}_{\mathrm{крит}}$,
$W^{q_2}_{\mathrm{пред}}$ и $W^{q_3}_{\mathrm{план}}$}}&Множества соответственно
критических, предупреждающих и плановых команд управления, предназначенных
соответственно для
лик\-ви\-да\-ции угрозы, для предотвращения угрозы и для повышения эффективности организационной
системы в целом; $q_1 \hm= 1, \ldots, Q_{\mathrm{крит}}$; $q_2 \hm= 1, \ldots, Q_{\mathrm{пред}}$; $q_3 \hm= 1,
\ldots , Q_{\mathrm{план}}$\\
\hline
\multicolumn{1}{|l|}{\raisebox{-21pt}[0pt][0pt]{$W_{n\mbox{-}\mathrm{крит}}$, $W_{n\mbox{-}\mathrm{пред}}$ и
$W_{n\mbox{-}\mathrm{план}}$}}&Множества соответственно критических, предупреждающих и плановых
сценариев принятия решений, предназначенных соответственно для ликвидации угрозы, для
предотвращения угрозы и для повышения эффективности $n$-го вида деятельности организационной
системы; $n \hm= 1, 2, \ldots , N$\\
\hline
\multicolumn{1}{|l|}{\raisebox{-21pt}[0pt][0pt]{$W^{u_1}_{n\mbox{-}\mathrm{крит}}$,
$W^{u_2}_{n\mbox{-}\mathrm{пред}}$ и $W^{u_3}_{n\mbox{-}\mathrm{план}}$}}&Множества
соответственно критических, предупреждающих и плановых команд управления, предназначенных
соответственно для лик\-ви\-да\-ции угрозы, для предотвращения угрозы и для повышения эффективности
$n$-го вида деятельности организационной системы;\linebreak $u_1 \hm= 1, \ldots, U_{n\mbox{-}\mathrm{крит}}$; $u_2 \hm= 1,
\ldots , U_{n\mbox{-}\mathrm{пред}}$; $u_3 \hm= 1, \ldots , U_{n\mbox{-}\mathrm{план}}$\\
\hline
\multicolumn{1}{|l|}{\raisebox{-21pt}[0pt][0pt]{$W_{nm\mbox{-}\mathrm{крит}}$,
$W_{nm\mbox{-}\mathrm{пред}}$ и $W_{nm\mbox{-}\mathrm{план}}$}}&Множества соответственно
критических, предупреждающих и плановых сценариев принятия решений, предназначенных
соответственно для ликвидации угрозы, для предотвращения угрозы и для повышения эффективности
$n$-го вида деятельности в $m$-м подразделении организационной системы; $n \hm= 1, 2, \ldots, N$; $m
\hm= 1, 2, \ldots , M$\\
\hline
\multicolumn{1}{|l|}{\raisebox{-27pt}[0pt][0pt]{$W^{y_1}_{nm\mbox{-}\mathrm{крит}}$,
 $W^{y_2}_{nm\mbox{-}\mathrm{пред}}$ и $W^{y_3}_{nm\mbox{-}\mathrm{план}}$}}&Множества
соответственно критических, предупреждающих и плановых команд управления,
предназначенных
соответственно для лик\-ви\-да\-ции угрозы, для предотвращения угрозы и
для повышения эффектив\-ности
$n$-го вида деятельности в $m$-м подразделении организационной системы; $y_1\hm = 1, \ldots,
Y_{nm\mbox{-}\mathrm{крит}}$;
$y_2 \hm= 1, \ldots , Y_{nm\mbox{-}\mathrm{пред}}$; $y_3 \hm= 1, \ldots ,
Y_{nm\mbox{-}\mathrm{план}}$\\
\hline
\end{tabular}
\end{center}
\vspace*{-6pt}
\end{table}

\begin{multicols}{2}

\noindent
\begin{align*}
V^*_{nmk} &= \mu^1_k\left(V^1_{nmk} -\Delta V^{*1}_{nmk}\right)+{}\\
&\hspace*{-30.11438pt}{}+ \mu^2_k\left(V^2_{nmk} -\Delta
V^{*2}_{nmk}\right)+ \cdots+ \mu^{L_k}_k\left (V^{*L_k}_{nmk} -V^{*L_k}_{nmk}\right)\,;
%\label{e10-zts}
\\
     \Delta V^{*l}_{nmk} &= \left\vert V^l_{nmk} -
V^{*l}_{nmk}\right\vert\,; %\label{e11-zts}
    \\
     V_{nmk} & = \mu^1_kV^1_{nmk} + \mu^2_k V^2_{nmk} +\cdots+
\mu^{L_k}_kV^{L_k}_{nmk}\,.
%       \label{e12-zts}
     \end{align*}

     На основе результатов оценки эффективности деятельности организационной
системы:
     \begin{itemize}
\item  определяются сценарии принятия решений;
\item определяются, передаются на исполнение и исполняются команды,
предназначенные для управления объектами, которые влияют на деятельность
организационной системы (этапы~6 и~7 технологии).
     \end{itemize}

     В табл.~3 показана классификация сценариев принятия решений и команд
управления.

\vspace*{-6pt}

\section{Правила определения сценариев принятия решения}

     На рис.~1--3 приведены диаграммы, по\-яс\-ня\-ющие подход в рассматриваемой
технологии к определению состояния деятельности организа\-ционной системы и в
дальнейшем к определению сценари\-ев принятия решения, соответству\-ющих
данному состоянию и сложившейся ситуации, описываемой показателями объектов
наблюдения.


     Как видно из приведенных диаграмм, в основе применения
рассматриваемой системы показате-\linebreak

\begin{center}  %fig1
\vspace*{2pt}
\mbox{%
\epsfxsize=77.63mm
\epsfbox{zar-1.eps}
}
\end{center}
  \vspace*{2pt}

\noindent
{{\figurename~1}\ \ \small{Соотношения сценариев в зависимости от критических и допустимых показателей
эффек\-тив\-ности (пример~1). Выбор одного из множеств сценариев ($W_{\mathrm{крит}}$,
$W_{\mathrm{пред}}$, $W_{\mathrm{план}}$), из числа которых определяется сценарий принятия
решения, производится в за\-ви\-си\-мости от об\-ласти размещения показателя
эффективности\protect\linebreak
(0--$\Delta D_{\mathrm{крит}}$;
$\Delta D_{\mathrm{крит}}$--$\Delta D_{\mathrm{доп}}$;
$\Delta D_{\mathrm{доп}}$--1)}}


\vspace*{12pt}

\begin{center}  %fig2
\vspace*{2pt}
\mbox{%
\epsfxsize=77.526mm
\epsfbox{zar-2.eps}
}
\end{center}

  \vspace*{2pt}

\noindent
{{\figurename~2}\ \ \small{Соотношения сценариев в зависимости от критических и допустимых показателей
эф\-фек\-тив\-ности (пример~2). Выбор одного из множеств сценариев ($W_{n\mbox{-}\mathrm{крит}}$,
$W_{n\mbox{-}\mathrm{пред}}$, $W_{n\mbox{-}\mathrm{план}}$), из числа которых определяется сценарий принятия
решения, производится в за\-ви\-си\-мости от об\-ласти размещения показателя эффективности\linebreak
(0--$\Delta D_{n\mbox{-}\mathrm{крит}}$;
$\Delta D_{n\mbox{-}\mathrm{крит}}$--$\Delta D_{n\mbox{-}\mathrm{доп}}$;
$\Delta D_{n\mbox{-}\mathrm{доп}}$--1)}}


\vspace*{12pt}


\begin{center}  %fig3
\vspace*{2pt}
\mbox{%
\epsfxsize=77.526mm
\epsfbox{zar-3.eps}
}
\end{center}

  \vspace*{2pt}

\noindent
{{\figurename~3}\ \ \small{Соотношения сценариев в зависимости от\protect\linebreak критических и допустимых показателей
эф\-фек\-тив\-ности (пример~3). Выбор одного из множеств сценариев ($W_{nm\mbox{-}\mathrm{крит}}$,
$W_{nm\mbox{-}\mathrm{пред}}$, $W_{nm\mbox{-}\mathrm{план}}$), из числа которых определяется сценарий принятия
решения, производится в зависимости от об\-ласти размещения показателя эф\-фек\-тив\-ности
(0--$\Delta S_{n\mbox{-}\mathrm{крит}}$;
$\Delta S_{nm\mbox{-}\mathrm{крит}}$--$\Delta S_{nm\mbox{-}\mathrm{доп}}$;
$\Delta S_{nm\mbox{-}\mathrm{доп}}$--1)}}


\vspace*{12pt}



\addtocounter{figure}{3}


\noindent
лей лежат критерии оценки эффективности
деятельности организационной системы в трехуровневом измерении:
     \begin{enumerate}[(1)]
\item  оценка видов деятельности в подразделениях организационной системы;
\item  оценка видов деятельности организационной системы;
\item  оценка деятельности организационной системы в целом.
\end{enumerate}

     При выборе сценария принятия решения и соответствующих этому сценарию
команд управления используются следующие правила:
\begin{enumerate}[1.]
\item В исходном состоянии из множества $W_{nm\mbox{-}\mathrm{норм}}$ нормированных
сценариев принятия решений определяют и исполняют команды
$W^{y_0}_{nm\mbox{-}\mathrm{норм}}$, предназначенные для установления нормированного
состояния $n$-го вида деятельности в \mbox{$m$-м} подразделении организационной системы:
\begin{multline}
      W_{nm\mbox{-}\mathrm{норм}} = {}\\
      \hspace*{-2mm}{}=\left\{W^1_{nm\mbox{-}\mathrm{норм}};
      W^2_{nm\mbox{-}\mathrm{норм}}; \ldots  ;
      W^{Y_{nm\mbox{-}\mathrm{норм}}}_{nm\mbox{-}\mathrm{норм}}\right\}\,,
      \label{e13-zts}
      \end{multline}
      где $y_0 = 1, 2, \ldots, Y_{nm\mbox{-}\mathrm{норм}}$.

Принимают, что при выполнении этих команд показатели состояния видов
деятельности и показатели состояния деятельности организационной системы в
целом также будут нормированными.

\item В ходе деятельности организационной сис\-те\-мы, если фактический показатель
$\Delta D^*$ эффективности деятельности организационной сис\-те\-мы в целом меньше
соответствующего критиче\-ского показателя $\Delta D_{\mathrm{крит}}$:
\begin{equation*}
      0 \leq \Delta D^* < \Delta D_{\mathrm{крит}}\,,
%	\label{e14-zts}
     \end{equation*}
то из множества $W_{\mathrm{крит}}$ критических сценариев принятия решений
определяют и исполняют команды $W^{q_1}_{\mathrm{крит}}$, предназначенные для
ликвидации последствий реализации угроз для де\-ятель\-ности организационной
системы в целом:
\begin{equation*}
      W_{\mathrm{крит}} = \left\{W^1_{\mathrm{крит}}; W^2_{\mathrm{крит}}; \ldots;
W^{Q_{\mathrm{крит}}}_{\mathrm{крит}}\right\}\,,
%     \label{e15-zts}
     \end{equation*}
     где $q_1 = 1, 2, \ldots , Q_{\mathrm{крит}}$.


\item В ходе деятельности организационной сис\-те\-мы, если фактический показатель
$\Delta D^*$ эффективности деятельности организационной сис\-те\-мы в целом не
меньше соответствующего\linebreak критиче\-ского показателя $\Delta D_{\mathrm{крит}}$ и
меньше соответствующего допустимого показате\-ля~$\Delta D_{\mathrm{доп}}$:
\begin{equation*}
      \Delta D_{\mathrm{крит}}\leq \Delta D^* < \Delta
D_{\mathrm{доп}}\,,
%      \label{e16-zts}
      \end{equation*}
      то из множества $W_{\mathrm{пред}}$ предупреждающих
сценариев определяют и исполняют команды $W^{q_2}_{\mathrm{пред}}$,
предназначенные для предотвращения угроз деятель\-ности
организационной системы в целом:
      \begin{equation*}
     W_{\mathrm{пред}} = \left\{W^1_{\mathrm{пред}}; W^2_{\mathrm{пред}}; \ldots ;
W^{Q_{\mathrm{пред}}}_{\mathrm{пред}}\right\}\,,
%     \label{e17-zts}
     \end{equation*}
     где $q_2 = 1, 2, \ldots , Q_{\mathrm{пред}}$.

\item В ходе деятельности организационной сис\-те\-мы, если фактический показатель
$\Delta D^*$ эффективности деятельности организационной сис\-те\-мы в целом не
меньше соответствующего\linebreak допустимого показателя $\Delta D_{\mathrm{доп}}$ и меньше
единицы:
\begin{equation*}
      \Delta D_{\mathrm{доп}}\leq \Delta D^* < 1\,, 		
%      \label{e18-zts}
      \end{equation*}
      то из множества $W_{\mathrm{план}}$ плановых сценариев
определяют и исполняют команды $W^{q_3}_{\mathrm{план}}$,
предназначенные для повышения эффективности деятельности
организационной системы в целом:
      \begin{equation*}
     W_{\mathrm{план}} = \left\{W^1_{\mathrm{план}}; W^2_{\mathrm{план}}; \ldots ;
W^{Q_{\mathrm{план}}}_{\mathrm{план}}\right\}\,,
%     \label{e19-zts}
     \end{equation*}
     где $q_3 = 1, 2, \ldots , Q_{\mathrm{план}}$.

\item В ходе деятельности организационной сис\-те\-мы, если фактический показатель
$\Delta D_n^*$ эффективности $n$-го вида деятельности организационной системы
меньше соответствующего\linebreak критического показателя $\Delta D_{n\mbox{-}\mathrm{крит}}$:
\begin{equation*}
     0 \leq \Delta D_n^* < \Delta D_{n\mbox{-}\mathrm{крит}}\,,
%     \label{e20-zts}
     \end{equation*}
     то из множества $W_{n\mbox{-}\mathrm{крит}}$ критических
сценариев определяют и исполняют команды
     $W^{u_1}_{n\mbox{-}\mathrm{крит}}$, предназначенные для
ликвидации последствий реализованных угроз $n$-му виду
деятельности организационной системы:
     \begin{equation*}
     W_{n\mbox{-}\mathrm{крит}} = \left\{W^1_{n\mbox{-}\mathrm{крит}};
     W^2_{n\mbox{-}\mathrm{крит}}; \ldots  ;
     W^{U_{n\mbox{-}\mathrm{крит}}}_{n\mbox{-}\mathrm{крит}}\right\}\,,	
%     \label{e21-zts}
     \end{equation*}
     где $u_1 = 1, 2, \ldots, U_{n\mbox{-}\mathrm{крит}}$.

\item В ходе деятельности организационной сис\-темы, если фактический показатель
$\Delta D^*_n$ эффектив\-ности $n$-го вида деятельности организационной сис\-те\-мы не
меньше соответствующего критического показателя $\Delta D_{n\mbox{-}\mathrm{крит}}$ и
меньше соответствующего допустимого показателя $\Delta D_{n\mbox{-}\mathrm{доп}}$:
\begin{equation*}
      \Delta D_{n\mbox{-}\mathrm{крит}}\leq  \Delta D_n^* < \Delta
D_{n\mbox{-}\mathrm{доп}}\,,
%      \label{e22-zts}
      \end{equation*}
      то из множества $W_{n\mbox{-}\mathrm{пред}}$ предупреждающих
сценариев определяют и исполняют команды
      $W^{u_2}_{n\mbox{-}\mathrm{пред}}$, предназначенные для
предотвращения угроз $n$-му виду деятельности организационной
системы:
      \begin{equation*}
     W_{n\mbox{-}\mathrm{пред}} = \left\{W^1_{n\mbox{-}\mathrm{пред}};
     W^2_{n\mbox{-}\mathrm{пред}}; \ldots ;
     W^{U_{n\mbox{-}\mathrm{пред}}}_{n\mbox{-}\mathrm{пред}}\right\}\,,
%     \label{e23-zts}
     \end{equation*}
     где $u_2 = 1, 2, \ldots , U_{n\mbox{-}\mathrm{пред}}$.

\item В ходе деятельности организационной сис\-те\-мы, если фактический показатель
$\Delta D^*_n$ эффективности $n$-го вида деятельности организационной системы не
меньше соответствующего допустимого показателя $\Delta D_{n\mbox{-}\mathrm{доп}}$ и
меньше единицы:
\begin{equation*}
      \Delta D_{n\mbox{-}\mathrm{доп}}\leq  \Delta D_n^* < 1\,,
%      \label{e24-zts}
      \end{equation*}
      то из множества $W_{n\mbox{-}\mathrm{план}}$ плановых
сценариев определяют и исполняют команды
      $W^{u_3}_{n\mbox{-}\mathrm{план}}$, предназначенные для
повышения эф\-фек\-тив\-ности $n$-го вида деятельности
организационной сис\-темы:
      \begin{equation*}
     W_{n\mbox{-}\mathrm{план}} = \left\{ W_{n\mbox{-}\mathrm{план}}^1;
     W_{n\mbox{-}\mathrm{план}}^2; \ldots  ;
     W^{U_{n\mbox{-}\mathrm{план}}}_{n\mbox{-}\mathrm{план}}\right\}\,,
%     \label{e25-zts}
     \end{equation*}
     где $u_3 = 1, 2, \ldots , U_{n\mbox{-}\mathrm{план}}$.\\[-15pt]

\item В ходе деятельности организационной сис\-те\-мы, если фактический показатель
$\Delta S^*_{nm}$ эффективности $n$-го вида деятельности организационной системы в
$m$-м подразделении \mbox{меньше} соответствующего критического показателя $\Delta
S_{nm\mbox{-}\mathrm{крит}}$:
\begin{equation*}
     0 \leq \Delta S^*_{nm}< \Delta S_{nm\mbox{-}\mathrm{крит}}\,,
%     \label{e26-zts}
     \end{equation*}
     то из множества $W_{nm\mbox{-}\mathrm{крит}}$ критических
сценариев определяют и исполняют команды
     $W^{y_1}_{nm\mbox{-}\mathrm{крит}}$, предназначенные для
ликвидации последствий реализованных угроз $n$-му виду
деятельности организационной системы в \mbox{$m$-м}
подразделении:
     \begin{multline*}
     W_{nm\mbox{-}\mathrm{крит}} ={}\\
     {}= \left\{W^1_{nm\mbox{-}\mathrm{крит}};
     W^2_{nm\mbox{-}\mathrm{крит}}; \ldots ;
     W^{Y_{nm\mbox{-}\mathrm{крит}}}_{nm\mbox{-}\mathrm{крит}}\right\}\,,
%     \label{e27-zts}
     \end{multline*}
     где $y_1 = 1, 2, \ldots , Y_{nm\mbox{-}\mathrm{крит}}$.\\[-15pt]

\item В ходе деятельности организационной сис\-те\-мы, если фактический показатель
$\Delta S^*_{nm}$ эффективности $n$-го вида деятельности организационной системы в
$m$-м подразделении не меньше соответствующего критического показателя
$\Delta S_{nm\mbox{-}\mathrm{крит}}$ и меньше соответствующего допустимого показателя
$\Delta S_{nm\mbox{-}\mathrm{доп}}$:
\begin{equation*}
      \Delta S_{nm\mbox{-}\mathrm{крит}}\leq \Delta S^*_{nm}< \Delta
      S_{nm\mbox{-}\mathrm{доп}}\,,
%      \label{e28-zts}
      \end{equation*}
      то из множества $W_{nm\mbox{-}\mathrm{пред}}$
предупреждающих сценариев определяют и исполняют команды
$W^{y_2}_{nm\mbox{-}\mathrm{пред}}$, предназначенные для
предотвращения угроз $n$-му виду деятельности организационной
сис\-те\-мы в $m$-м подразделении:
      \begin{multline*}
     W_{nm\mbox{-}\mathrm{пред}} ={}\\
     {}= \left\{W^1_{nm\mbox{-}\mathrm{пред}};
     W^2_{nm\mbox{-}\mathrm{пред}}; \ldots ;
     W^{Y_{nm\mbox{-}\mathrm{пред}}}_{nm\mbox{-}\mathrm{пред}} \right\}\,, 	
%     \label{e29-zts}
     \end{multline*}
где $y_2 = 1, 2, \ldots , Y_{nm\mbox{-}\mathrm{пред}}$.\\[-15pt]

\item В ходе деятельности организационной сис\-те\-мы, если фактический показатель
$\Delta S^*_{nm}$ эф-\linebreak\vspace*{-12pt}

\pagebreak

\noindent
фективности $n$-го вида деятельности организационной системы в
$m$-м подразделении не меньше соответствующего допустимого показателя $\Delta
S_{nm\mbox{-}\mathrm{доп}}$ и меньше единицы:
\begin{equation*}
      \Delta S_{nm\mbox{-}\mathrm{доп}}\leq  \Delta S^*_{nm}< 1\,, 	
%      \label{e30-zts}
      \end{equation*}
      то из множества $W_{nm\mbox{-}\mathrm{план}}$ плановых
сценариев определяют и исполняют команды
      $W^{y_3}_{nm\mbox{-}\mathrm{план}}$, предназначенные для
повышения эффек\-тив\-ности \mbox{$n$-го} вида деятельности
организационной системы в \mbox{$m$-м} подразделении:
      \begin{multline*}
     W_{nm\mbox{-}\mathrm{план}} = {}\\
     {}=\left\{ W^1_{nm\mbox{-}\mathrm{план}};
     W^2_{nm\mbox{-}\mathrm{план}}; \ldots ;
     W^{Y_{nm\mbox{-}\mathrm{план}}}_{nm\mbox{-}\mathrm{план}}\right\}\,,
%     \label{e31-zts}
     \end{multline*}
     где $y_3 = 1, 2, \ldots , Y_{nm\mbox{-}\mathrm{план}}$.
     \end{enumerate}


\section*{Заключение}

\vspace*{-4pt}

     \noindent
     \begin{enumerate}[1.]
     \item  Представленная в статье система показателей обеспечивает выполнение
оценки эффективности технологии поддержки деятельности организационной
системы на трех уровнях:\\[-15pt]
     \begin{enumerate}[(1)]
\item  при управлении деятельностью организационной системы в целом;\\[-15pt]
\item при управлении видами деятельности организационной системы;\\[-15pt]
\item при управлении деятельностью подразделений организационной системы.\\[-15pt]
\end{enumerate}
     \item Аналитическое обеспечение системы показателей эффективности
деятельности учитывает влияние, оказываемое на деятельность организационной
системы различных объектов, входящих как в состав организационной системы, так
и в состав других систем.\\[-15pt]
\item  Предложена логика взаимосвязи нормированных и фактических показателей
эффективности с критериями оценки, которая предоставляет возможность лицам,
принимающим решения в отношении ситуаций, возникающих в среде деятельности
организационной системы и во внешней среде, осуществлять обоснованный выбор
сценария принятия решения (в диапазоне критических, предупреждающих или
плановых сценариев).
\end{enumerate}


{\small\frenchspacing
 {%\baselineskip=10.8pt
 \addcontentsline{toc}{section}{References}
 \begin{thebibliography}{9}


\bibitem{2-zts}
Информатика: состояние, проблемы, перспективы~/ Под ред. И.\,А.~ Соколова.~---
М.: ИПИ РАН, 2009. 46~с.
\bibitem{3-zts}
\Au{Зацаринный А.\,А., Шабанов А.\,П.} Системные аспекты эффективности
ситуационных центров~// Вестник Моск. ун-та им.\ С.\,Ю.~Витте. Сер.~1:
Экономика и управление, 2013. Вып.~2. С.~110--123.

\bibitem{1-zts} %3
\Au{Цымблер М.} Big Data: несколько прос\-тых вопросов о сложном явлении~//
Суперкомпьютеры, 2014. №\,1(17). С.~8--11.
{\sf http://mzym.susu.ru/papers/\linebreak Zymbler\_Supercomputers-14a.pdf}.
\bibitem{4-zts}
\Au{Зацаринный А.\,А., Сучков~А.\,П., Шабанов~А.\,П.} Технология поддержки
деятельности организационной сис\-те\-мы. Заявка на изобретение RU2012148411A.,
Опубл. 20.05.2014. Бюл. №\,14.
\bibitem{5-zts}
\Au{Голяндин А.\,Н., Шабанов~А.\,П.} Центр мониторинга устойчивости
информационных систем. Полезная модель. Патент RU130109U1, G06F21/50,
01.2013. Опубл. 10.07.2013. Бюл. №\,19.


 \end{thebibliography}

 }
 }

\end{multicols}

%\vspace*{-6pt}

\hfill{\small\textit{Поступила в редакцию 31.07.14}}

%\newpage

\vspace*{8pt}

\hrule

\vspace*{2pt}

\hrule

\vspace*{6pt}

\def\tit{ANALYTICAL ASPECTS OF~EVALUATION OF~EFFECTIVENESS
OF~TECHNOLOGICAL SUPPORT OF~AN~ORGANIZATIONAL SYSTEM}

\def\titkol{Analytical aspects of~evaluation of~effectiveness
of~technological support of an~organizational system}

\def\aut{A.\,A. Zatsarinnyy and A.\,P.~Shabanov}

\def\autkol{A.\,A. Zatsarinnyy and A.\,P.~Shabanov}

\titel{\tit}{\aut}{\autkol}{\titkol}

\vspace*{-9pt}

\noindent
Institute of Informatics Problems, Russian Academy of Sciences,
44-2 Vavilov Str., Moscow 119333, Russian Federation


\def\leftfootline{\small{\textbf{\thepage}
\hfill INFORMATIKA I EE PRIMENENIYA~--- INFORMATICS AND
APPLICATIONS\ \ \ 2014\ \ \ volume~8\ \ \ issue\ 3}
}%
 \def\rightfootline{\small{INFORMATIKA I EE PRIMENENIYA~---
INFORMATICS AND APPLICATIONS\ \ \ 2014\ \ \ volume~8\ \ \ issue\ 3
\hfill \textbf{\thepage}}}

\vspace*{9pt}

\Abste{The article discusses the system consisting of
normalized and actual indicators, which is used in technological
support of activities of an organizational system.
The article presents the evaluation criteria that are provided by the classification
approach to the choice of scenario for decision for responsible persons for different situations arising in
an environment of activities of an organizational system. Classification is
performed according to the hierarchy of activities
of an organizational system: (a)~activities in the divisions of
the organizational system; (b)~activities in the
organizational system; and (c)~all activities of the
organizational system in general. The main emphasis of the
methodical approach presented in the article is practical. This allows
one to take the degree\linebreak}

\Abstend{ of threat to
activities of an organizational system. The authors distinguish three degrees of
state activities: ($i$)~activities are
carried out in a range of specified normalized indicators;
($ii$)~\textit{a priori}, there is a threat to continuation of
activities of an organizational system; and ($iii$)~activities of
an organizational system are under threat.}

\KWE{organizational system; technological support of
activities of an organizational system; evaluation of the effectiveness;
normalized and actual indicators; information;
decision making; monitoring objects}

\DOI{10.14357/19922264140314}

  \begin{multicols}{2}

\renewcommand{\bibname}{\protect\rmfamily References}
%\renewcommand{\bibname}{\large\protect\rm References}

{\small\frenchspacing
 {%\baselineskip=10.8pt
 \addcontentsline{toc}{section}{References}
 \begin{thebibliography}{9}




\bibitem{2-zts-1}
Sokolov, I.\,A., ed. 2009.
\textit{Informatika: Sostoyanie, problemy, perspektivy}
[Informatics: State, problems, prospects].
Moscow: IPI RAN. 46~p.


\bibitem{3-zts-1}
\Aue{Zatsarinnyy, A.\,A., and A.\,P.~Shabanov}.
2013. Sistemnye aspekty effektivnosti situatsionnykh tsentrov
[System aspects of effectiveness of situation centers].
\textit{Vestnik Mosk. un-ta im.\ S.\,Yu.~Vitte. Ser.~1:
Ekonomika i upravlenie}
[Gerald of Moscow University named after S.\,J.~Witte. Ser.~1: Economy and Management]
2:110--123.

 \bibitem{1-zts-1}
 \Aue{Tsymbler, M.} 2014. Big Data:
 neskol'ko prostykh voprosov o slozhnom yavlenii
 [Big Data: A~few simple questions about complex phenomenon].
 \textit{Superkomp'yutery} [Supercomputers] 1(17):8--11.
 Available at: {\sf http://\linebreak mzym.susu.ru/papers/Zymbler\_Supercomputers-14a.pdf}
 (accessed August~4, 2014).

\bibitem{4-zts-1}
\Aue{Zatsarinnyy, A.\,A., A.\,P.~Suchkov, and A.\,P.~Shabanov}.
2014. Tekhnologiya podderzhki deyatel'nosti organizatsionnoy sistemy
[Technology of support of organizational system]. Zayavka na izobretenie
[Application for invention] RU2012148411A. Published 20.05.2014.
\bibitem{5-zts-1}
\Aue{Golyandin, A.\,N., and A.\,P.~Shabanov}.
January 2013. Tsentr monitoringa ustoychivosti informatsionnykh sistem
[Center for monitoring the sustainability of information systems].
Poleznaya model' [Utility model]. Patent RU130109U1, G06F21/50.
Published 10.07.2013.
\end{thebibliography}

 }
 }

\end{multicols}

\vspace*{-6pt}

\hfill{\small\textit{Received July 31, 2014}}

\vspace*{-18pt}

\Contr

\noindent
\textbf{Zatsarinnyy Alexander A.} (b.\ 1951)~---
Doctor of Science in technology, professor, Deputy Director, Institute of Informatics Problems, Russian Academy of Sciences,
44-2 Vavilov Str., Moscow 119333, Russian Federation; AZatsarinny@ipiran.ru

\vspace*{3pt}

\noindent
\textbf{Shabanov Alexander P.} (b.\ 1949)~---
 Doctor of Science in technology, leading scientist, Institute of Informatics Problems, Russian Academy of Sciences,
 44-2 Vavilov Str., Moscow 119333, Russian Federation;
 AShabanov@ibs.ru


\label{end\stat}

\renewcommand{\bibname}{\protect\rm Литература}
