\def\stat{sinits}

\def\tit{АНАЛИТИЧЕСКОЕ МОДЕЛИРОВАНИЕ
НОРМАЛЬНЫХ ПРОЦЕССОВ В~СТОХАСТИЧЕСКИХ СИСТЕМАХ СО~СЛОЖНЫМИ~НЕЛИНЕЙНОСТЯМИ}

\def\titkol{Аналитическое моделирование
нормальных процессов в~стохастических системах со~сложными нелинейностями}

\def\aut{И.\,Н.~Синицын$^1$, В.\,И.~Синицын$^2$}

\def\autkol{И.\,Н.~Синицын, В.\,И.~Синицын}

\titel{\tit}{\aut}{\autkol}{\titkol}

\renewcommand{\thefootnote}{\arabic{footnote}}
\footnotetext[1]{Институт проблем
информатики Российской академии наук, sinitsin@dol.ru}
\footnotetext[2]{Институт проблем
информатики Российской академии наук, vsinitsin@ipiran.ru}


\Abst{Рассматриваются конечномерные дифференциальные стохастические системы
(ДСтС) и эредитарные (интегродифференциальные) стохастические системы  (ЭСтС)
с винеровскими и пуассоновскими шумами, приводимые к ДСтС со сложными конечными,
дифференциальными и интегральными нелинейностями. Такие модели функционирования
описывают поведение многих современных нано- и кван\-то\-во-оп\-ти\-че\-ских
технических средств информатики. Приводятся уравнения методов нормальной
аппроксимации (МНА) и статистической линеаризации (МСЛ) для аналитического
моделирования нестационарных и стационарных нормальных (гауссовских) процессов
в нелинейных ДСтС и  нелинейных ЭСтС путем аппроксимации эредитарных ядер
линейными операторными уравнениями для дифференцируемых нелинейностей и
сингулярными ядрами для недифференцируемых нелинейностей. Рассматриваются
методы вычисления типовых интегралов МНА (МСЛ) для сложных (многомерных и
векторного аргумента) конечных и дифференциальных нелинейностей. Особое
внимание уделяется иррациональным и дробно-рациональным нелинейностям
скалярного аргумента. Приводятся примеры вычисления интегралов. Подробно
рассматриваются вопросы вычисления типовых интегралов МНА (МСЛ) для сложных
интегральных нелинейностей.}

\KW{аналитическое моделирование;
дифференциальные стохастические системы с винеровскими и пуассоновскими шумами (ДСтС);
метод нормальной аппроксимации (МНА);
метод статистической линеаризации (МСЛ);
сложные иррациональные нелинейности;
сложные конечные, дифференциальные и интегральные нелинейности;
эредитарные стохастические системы (ЭСтС), приводимые к дифференциальным}

\DOI{10.14357/19922264140302}

\vspace*{9pt}

\vskip 16pt plus 9pt minus 6pt

\thispagestyle{headings}

\begin{multicols}{2}

\label{st\stat}


\section{Введение}


Моделями функционирования многих современных технических сис\-тем информатики
служат стохастические системы (СтС), описываемые дифференциальными, интегральными
и интегродифференциальными уравнениями со сложными дроб\-но-ра\-ци\-о\-наль\-ны\-ми,
иррациональными и интегральными нелинейностями. В~[1] дано систематическое
изложение МНА и МСЛ для ДСтС и ЭСтС, приводимых к дифференциальным.

Обобщая~[2--7], рассмотрим развитие МНА и МСЛ для аналитического моделирования
нормальных стохастических процессов (СтП) на случай СтС со сложными конечными,
дифференциальными и интегральными нелинейностями.

Как показано в~\cite{4-sin}, альтернативным подходом к аналитическому моделированию
СтП в ДСтС и ЭСтС служит подход, основанный на дискретизации стохастических
дифференциальных уравнений на основе использования обобщенной формы Ито и
кратных стохастических интегралов от винеровских и пуассоновских СтП с
последующим применением дискретных версий МНА (МСЛ).

Статья состоит из введения, пяти разделов и заключения.

В~разд.~2 и~3
приводятся уравнения МНА и МСЛ для аналитического моделирования одно- и
двумерных распределений стационарных и нестационарных СтП в ДСтС и ЭСтС,
приводимых к ДСтС.

Типовые интегралы МНА и МСЛ рассматриваются в разд.~4.

Особенности аналитического моделирования в ДСтС со сложными конечными и
дифференциальными нелинейностями обсуждаются в разд.~5.

Раздел~6
посвящен аналитическому моделированию СтП в ДСтС со сложными интегральными
нелинейностями.

Приводятся примеры.


\section{Уравнения методов нормальной~аппроксимации и~статистической
линеаризации для~дифференциальных стохастических систем}

Как известно~\cite{2-sin, 3-sin},  уравнения конечномерных непрерывных нелинейных сис\-тем
со стохастическими возмущениями путем расширения вектора состояния ДСтС
могут быть записаны в виде следующего векторного стохастического
дифференциального уравнения Ито:
    \begin{multline}
    dY_t = a(Y_t, t)\, dt + b (Y_t, t) \,dW_0+{}\\
    {}+ \iii_{R_0} c (Y_t, t, v) P^0
    (dt, dv)\,,\enskip Y(t_0) = Y_0\,.\label{e2.1-sin}
    \end{multline}
Здесь $a=a(Y_t, t)$ и $b\hm=b(y_t, t)$~--- известные
$(p\times 1)$-мер\-ная и  $(p\times m)$-мер\-ная функции~$Y_t$ и~$t$;
$W_0\hm= W_0(t)$~--- $r$-мер\-ный винеровский СтП интенсивности
$\nu_0 \hm= \nu_0(t)$; $c(Y_t, t, v)$~--- $(p\times 1)$-мер\-ная функция  $Y_t, t$
и вспомогательного $(q\times 1)$-мер\-но\-го па\-ра\-мет\-ра~$v$;
$\iii_{\Delta} dP^0 (t, A)$~--- центрированная пуассоновская мера,
определяемая
\begin{equation*}
\iii_{\Delta} dP^0 (t, A) = \iii_{\Delta} dP (t,A) =
\iii_{\Delta} \nu_P (t,A)\, dt\,. %\label{e2.2-sin}
\end{equation*}
В~(\ref{e2.1-sin}) принято: $\iii_{\Delta}$~-- число скачков пуассоновского
СтП в интервале времени  $\Delta \hm= (t_1, t_2]$; $\nu_P (t, A)$~---
интенсивность пуассоновского СтП  $P(t,A)$; $A$~--- некоторое борелевское
множество пространства  $R_0^q$ с выколотым началом.
Начальное значение~$Y_0$ представляет собой случайную величину, не зависящую
от приращений СтП  $W_0(t)$ и $P(t,A)$ на интервалах времени, следующих
за~$t_0$, $t_0 \hm\le t_1\hm\le t_2$ для любого множества~$A$.

В случае аддитивных нормальных (гауссовских) и обобщенных
пуассоновских возмущений уравнение~(\ref{e2.1-sin}) имеет вид:
\begin{equation}
\dot Y_t = a(Y_t,t)+ b_0 (t) V\,, \enskip
V = \dot W\,,\enskip Y(t_0) = Y_0\,.\label{e2.3-sin}
\end{equation}
Здесь $W$~--- СтП с независимыми приращениями, представляющий собой
смесь нормального и обобщенного пуассоновского СтП.

Если предположить существование конечных вероятностных
моментов второго порядка для моментов времени~$t_1$ и~$t_2$, то уравнения
МНА примут следующий вид~\cite{2-sin, 3-sin}:
\begin{itemize}
\item  для характеристических функций
    \begin{equation}
    g_1^N (\la;t) =\exp \lk i\la^{\mathrm{T}} m_t - \fr{1}{2}\, \la^{\mathrm{T}} K_t \la\rk\,;\label{e2.4-sin}
    \end{equation}
\begin{equation}
\hspace*{-7.5mm}g_{t_1, t_2}^N (\la_1, \la_2;t_1, t_2 ) =\exp \lk i\bar \la^{\mathrm{T}} \bar m_2 -
\fr{1}{2}\, \bar \la^{\mathrm{T}} \bar K_2 \la\rk\,,\!\!\label{e2.5-sin}
\end{equation}
где
    \begin{gather*}
    \bar \la =\lk \la_1^{\mathrm{T}}\la_2^{\mathrm{T}}\rk^{\mathrm{T}}\,; \quad
        \bar m_2 = \lk m_{t_1}^{\mathrm{T}} m_{t_2}^{\mathrm{T}}\rk^{\mathrm{T}}\,;\\
        \bar K_2= \begin{bmatrix}
    K(t_1, t_1)& K(t_1, t_2)\\
    K(t_2, t_1)& K(t_2, t_2)
    \end{bmatrix}\,;
    \end{gather*}

\item для математических ожиданий  $m_t$, ковариационной матрицы~$K_t$ и
матрицы ковариационных функций $K(t_1, t_2)$:
    \begin{equation}
    \dot m_t = a_1 (m_t, K_t, t)\,,\enskip m_0 = m(t_0)\,;\label{e2.6-sin}
    \end{equation}
\begin{equation}
\dot K_t = a_2 (m_t, K_t, t)\,,\enskip K_0 = K(t_0)\,;\label{e2.7-sin}
\end{equation}

\vspace*{-12pt}

\noindent
\begin{multline}
\fr{\prt K(t_1, t_2)}{\prt t_2 }= K(t_1, t_2) a_{21} (m_{t_2}, K_{t_2}, t_2)^{\mathrm{T}}\,;\\
K(t_1, t_1) = K_{t_1}\,.
\label{e2.8-sin}
\end{multline}
    \end{itemize}
Здесь приняты следующие обозначения:
\begin{equation}
a_1 = a_1 (m_t, K_t, t) = M_N a (Y_t, t)\,;\label{e2.9-sin}
\end{equation}

\vspace*{-12pt}

\noindent
\begin{multline}
a_2 = a_2 (m_t, K_t, t) = a_{21} (m_t, K_t, t)+{}\\
{}+ a_{21} (m_t, K_t, t)^{\mathrm{T}} +
a_{22}(m_t, K_t, t)\,;\label{e2.10-sin}
\end{multline}

\vspace*{-12pt}

\noindent

\begin{equation}
a_{21} = a_{21}(m_t, K_t, t)=  M_N a(Y_t, t) Y_{t}^{0\mathrm{T}}\,;\label{e2.11-sin}
\end{equation}
\begin{equation*}
a_{22} = a_{22}(m_t, K_t, t)= M_N \sigma (Y_t, t)\,;
%\label{e2.12-sin}
\end{equation*}

\vspace*{-12pt}

\noindent
\begin{multline*}
\sigma (Y_t, t) = b(Y_t, t) \nu_0(t) b(Y_t, t)^{\mathrm{T}} +{}\\
{}+
\iii_{R_0^q} c (Y_t, t, v) c(Y_t, t,v)^{\mathrm{T}}
\nu_P (t, dv)\,; %\label{e2.13-sin}
\end{multline*}

\vspace*{-12pt}

\begin{gather*}
m_t = MY_t\,,\quad Y_t^0 = Y_t - m_t\,,\\
K_t = M_N Y_0^0 Y_t^{0\mathrm{T}}\,,\quad K(t_1, t_2) =
M_N Y_{t_1}^0 Y_{t_2}^0\,; %\label{e2.14-sin}
\end{gather*}
$M_N$~--- символ вычисления математического ожидания для нормальных
распределений~(\ref{e2.4-sin}) и~(\ref{e2.5-sin}).

Для стационарных ДСтС нормальные стационарные СтП~--- если они существуют,
то  $m_t \hm=\bar m$, $ K_t \hm=\bar K$, $K(t_1, t_2) \hm= k(\tau)$
$(\tau \hm= t_1\hm-t_2)$,~--- определяются уравнениями~\cite{2-sin, 3-sin}:
   \begin{equation}
   a_1 (\bar m, \bar K) =0\,;\enskip a_2 (\bar m, \bar K)=0\,;\label{e2.15-sin}
   \end{equation}
   \begin{equation}
   \left.
   \hspace*{-2.8mm}\begin{array}{l}
  \dot k_\tau (\tau) = a_{21} (\bar m, \bar K)\bar K^{-1} k(\tau)\,;\\[9pt]
  k(0) =\bar K \enskip (\forall \tau >0)\,, \
  k(\tau) = k(-\tau)^{\mathrm{T}} \enskip
  (\forall\tau <0)\,.
  \end{array}\!\!
  \right\}\!\!
  \label{e2.16-sin}
  \end{equation}
При этом необходимо, чтобы матрица  $a_{21} (\bar m, \bar K)\hm=\bar a_{21}$
была бы асимптотически устойчивой.

Для ДСтС~(\ref{e2.3-sin}) уравнения МНА переходят в уравнения МСЛ
Казакова~\cite{2-sin, 3-sin}, если принять
\begin{equation}
a(Y_t,t) = a_1 (m_t, K_t) + k_1^a (m_t, K_t) Y_t^0\,;\label{e2.17-sin}
\end{equation}
\begin{equation}\left.
\begin{array}{rl}
b(Y_t,t) &= b_0 (t)\,;\\[9pt]
    \si(Y_t, t)&= b_0(t) \nu(t) b_0(t)^{\mathrm{T}} = \si_0(t)\,,
    \end{array}
    \right\}\label{e2.18-sin}
    \end{equation}
    \begin{equation}
k_1^a (m_t, K_t, t) =\lk \left(\fr{\prt}{\prt m_t} \right)
    a_0 (m_t, K_t, t)^{\mathrm{T}}\rk^{\mathrm{T}}\,;\label{e2.19-sin}
    \end{equation}
    \begin{equation}
\dot m_t = a_1 (m_t, K_t, t) \,,\enskip m_0 = m(t_0)\,,\label{e2.20-sin}
\end{equation}

\vspace*{-12pt}

\noindent
\begin{multline}
\dot K_t = k_1^a (m_t, K_t, t) K_t + K_t k_1^a (m_t, K_t, t)^{\mathrm{T}}
    +\si_0(t)\,;\\
    K_0 = K(t_0)\,;
    \label{e2.21-sin}
    \end{multline}

    \vspace*{-12pt}

    \noindent
\begin{multline}
\fr{\prt K(t_1, t_2)}{\prt t_2} =
    K(t_1, t_2) k_{t_2} k_1^a (m_{t_2}, K_{t_2}, t_2)^{\mathrm{T}}\,;\\
    K(t_1, t_2) = K_{t_1}\,.
    \label{e2.22-sin}
\end{multline}

Для стационарных ДСтС~(\ref{e2.3-sin})
при условии асимптотической устойчивости матрицы $k_1^a (\bar m, \bar K)$
в основе МСЛ лежат уравнения~(\ref{e2.15-sin}), записанные в виде:
    \begin{gather}
    a_1 (\bar m, \bar K) =0\,; \label{e2.23-sin}\\
k_1^a (\bar m, \bar K) \bar K + \bar K k_1^a
(\bar m, \bar K)^{\mathrm{T}} +\bar \si_0 =0\,;\label{e2.24-sin}
\end{gather}

\vspace*{-12pt}

\noindent
\begin{multline}
k_\tau (\tau) = k_1^a (\bar m, \bar K)k(\tau)\,,\enskip
k(0) =\bar K \enskip (\forall \tau >0)\,,\\
k(\tau) = k (-\tau)^{\mathrm{T}} \enskip (\forall \tau <0)\,.
\label{e2.25-sin}
\end{multline}

Уравнения~(\ref{e2.4-sin})--(\ref{e2.8-sin})
лежат в основе МНА для ДСтС~(\ref{e2.1-sin}), а уравнения~(\ref{e2.17-sin})--(\ref{e2.22-sin})~---
в основе МСЛ для ДСтС~(\ref{e2.3-sin}). Для определения стационарных СтП
согласно МНА служат соотношения~(\ref{e2.15-sin}) и~(\ref{e2.16-sin}),
а МСЛ~--- (\ref{e2.17-sin})--(\ref{e2.25-sin}).

\section{Уравнения методов нормальной~аппроксимации и~статистической линеаризации
для~эредитарных стохастических систем, приводимых к~дифференциальным}

Рассмотрим ЭСтС, описываемую интегродифференциальным уравнением Ито
следующего вида~\cite{7-sin}:

\noindent
\begin{multline}
dX_t = \lk a(X_t, t) +\iii_{t_0}^t a_1 (X(\tau) ,\tau, t)\,d\tau\rk dt+{}\\
{}+\lk b(X_t, t) +\iii_{t_0}^t b_1 (X(\tau) ,\tau, t)\,d\tau\rk dW_0+{}\\
\hspace*{-1.5mm}{}+\!\!\iii_{R_0^q}\!\!\lk c(X_t, t,v) +\!\iii_{t_0}^t\! c_1 (X(\tau) ,\tau, t,v)\,d\tau\!\rk\! dP^0 (t, dv)
\!\!\!\!\label{e3.1-sin}
\end{multline}
с начальным условием  $X(t_0) = X_0$. В~(\ref{e3.1-sin})
сохранены обозначения разд.~2.

Функции $a=a(X_t, t)$, $a_1\hm = a_1(X (\tau),\tau, t)$,
$b\hm=b(X_t, t)$, $b_1\hm = b_1(X (\tau),\tau, t)$,
$c\hm=c(X_t,t,v)$ и $c_1\hm = c_1(X (\tau),\tau, t,v)$ имеют
соответственно размерности $p\times 1$, $p\times 1$, $p\times r$,
$p\times r$, $p\times 1$ и $p\times 1$ и допускают представления следующего вида:
\begin{equation}
\left.
\begin{array}{rl}
a_1&=A(t,\tau) \vrp (X(\tau) , \tau)\,;\\[9pt]
b_1&=B(t,\tau) \psi (X(\tau) ,  \tau)\,;\\[9pt]
c_1&=C(t,\tau) \chi (X(\tau) ,  \tau, v)\,.
\end{array}
\right\}
\label{e3.2-sin}
\end{equation}
Здесь эредитарные ядра $A\hm=A(t,\tau)\hm=\lk A_{ij}(t,\tau)\rk$
$(i,j\hm=\overline{1,p})$,
$B\hm=B(t,\tau)=\lk B_{i l}(t,\tau)\rk$ $(i\hm=\overline{1,p}$;
$l\hm=\overline{1,r})$ и $C\hm=C(t,\tau)=\lk C_{ij}(t,\tau)\rk$
$(i,j\hm=\overline{1,p})$ имеют соответственно размерности
$p\times p$, $p\times r$ и $p\times p$. Они удовлетворяют следующим условиям
физической реализуемости и асимптотического затухания:
\begin{multline}
A_{ij}(t,\tau)=0;\enskip B_{i l}(t,\tau)=0;\\[1pt]
C_{ij}(t,\tau)=0\enskip \forall \tau >t;\label{e3.3-sin}
\end{multline}

\vspace*{-12pt}

\begin{equation}
\left.
\hspace*{-3mm}\begin{array}{c}
\displaystyle\iin\! \lv A_{ij} (t,\tau) \rv d\tau <\infty\,;\
\displaystyle\iin\! \lv B_{i l} (t,\tau) \rv d\tau <\infty \,;\\[9pt]
\displaystyle\iin \!\lv C_{ij} (t,\tau) \rv d\tau <\infty\,.
\end{array}\!
\right\}\!
\label{e3.4-sin}
\end{equation}

В дальнейшем ограничимся случаем, когда эредитарные ядра удовлетворяют
линейным операторным уравнениям~\cite{6-sin, 5-sin, 7-sin}.

Нелинейные в общем случае функции $\vrp\hm=\vrp(X(\tau),\tau)$,
$\psi \hm=\psi(X(\tau), \tau)$, $\chi \hm=\chi (X(\tau),  \tau, v)$
отражают нелинейные свойства ЭСтС, зависят от  $X(\tau)$ и имеют размерности
$p\times 1$, $p\times p$, $p\times 1$ соответственно.

Важный класс  эредитарных ядер представляют собой
сингулярные (вырожденные) ядра, когда имеют место представления:
\begin{equation}
\left.
\hspace*{-3mm}\begin{array}{rl}
A_{ij} (t,\tau) &= A_{ij}^+(t) A_{ij}^-(\tau)\,;\\[9pt]
B_{i l} (t,\tau)& = B_{il}^+(t) B_{il}^-(\tau)\,;\\[9pt]
C_{ij} (t,\tau) &= C_{ij}^+ ( t) C_{ij}^- (\tau)\
(i,l= \overline{1,p}, j=\overline{1,r}).
\end{array}\!
\right\}\!\!
\label{e3.5-sin}
\end{equation}

В~\cite{6-sin, 5-sin, 7-sin} показано, что для дифференцируемых нелинейных
функций~$\vrp$, $\psi$, $\chi$ путем расширения вектора состояния за счет
инструментальных переменных, аппроксимируемых линейными операторными уравнениями,
определяющими эредитарные ядра в ЭСтС, (\ref{e3.1-sin})--(\ref{e3.4-sin})
приводятся к ДСтС вида~(\ref{e2.1-sin}) или~(\ref{e2.3-sin}).
В~случае недифференцируемых нелинейных функций~$\vrp$, $\psi$, $\chi$
ЭСтС~(\ref{e3.1-sin})--(\ref{e3.4-sin}) приводятся к~(\ref{e2.1-sin}) или~(\ref{e2.3-sin})
на основе аппроксимации вырожденными (сингулярными) ядрами~\cite{6-sin, 5-sin, 7-sin}.

Таким образом, после приведения ЭСтС~(\ref{e3.1-sin}) к ДСтС~(\ref{e2.1-sin})
или~(\ref{e2.3-sin}) можно воспользоваться уравнениями МНА и МСЛ разд.~2.

\section{Типовые интегралы методов нормальной аппроксимации и~статистической
линеаризации}

Как следует из уравнений~(\ref{e2.9-sin})--(\ref{e2.11-sin}),
для МНА необходимо уметь вычислять следующие интегралы:
\begin{multline}
I_0^a = I_0^a (m_t, K_t, t) = a_1 (m_t, K_t, t)={}\\
{}= M_N a(Y_t, t)\,;
\label{e4.1-sin}
\end{multline}

\vspace*{-12pt}

\noindent
\begin{multline}
I_1^a = I_1^a (m_t, K_t, t)= a_{21}(m_t, K_t, t)= {}\\
{}=M_N a(Y_t , t) Y_t^{0\mathrm{T}}\,;\label{e4.2-sin}
\end{multline}

\vspace*{-12pt}

\noindent
\begin{multline}
I_0^{\bar \si} = I_0^{\bar \si} (m_t, K_t, t) = a_{22}(m_t, K_t, t) ={}\\
{}= M_N \bar \si (Y_t, t)\,.\label{e4.3-sin}
\end{multline}
Для МСЛ достаточно вычислить интеграл~(\ref{e4.1-sin}),
причем интеграл~$I_1^a$ вычисляется по формуле~\cite{2-sin, 3-sin, 4-sin}:
\begin{equation*}
k_1^a = k_1^a (m_t, K_t, t)=\lk \left( \fr{\prt}{\prt m_t}\right)
I_0^a (m_t, K_t, t)^{\mathrm{T}}\rk^{\mathrm{T}}. %\label{e4.4-sin}
\end{equation*}

\medskip

\noindent
\textbf{Пример 1.} В~[1] для типовых степенных, тригоно\-мет\-ри\-че\-ских,
показательных и ку\-соч\-но-по\-сто\-ян\-ных нелинейностей $Z_t \hm=\vrp (Y_t, t)$
скалярного и векторного аргумента приведены формулы для интегралов
$I_0^\vrp \hm= I_0^\vrp (m_t^y, K_t^y, t)$, а также
$k_1^\vrp \hm= k_1^\vrp (m_t^y, K_t^y, t)$.

\medskip

\noindent
\textbf{Замечание.}
 Важно иметь в виду, что уравнения МНА (МСЛ) содержат интегралы
 $I_0^a$, $I_1^a$, $I_0^\si$ в виде соответствующих коэффициентов.
 Поэтому процедура вычисления интегралов должна быть согласована с
 методом численного решения обыкновенных дифференциальных уравнений для
 $m_t$, $K_t$ и $K(t_1, t_2)$. Эти коэффициенты допускают дифференцирование
 по~$m_t$ и~$K_t$, так как под интегралом стоит сглаживающая нормальная плотность.

\section{Сложные конечные и~дифференциальные нелинейности}

Важный класс сложных конечных нелинейностей (многомерных и векторного аргумента)
представляют собой сложные функции вида:
    \begin{equation*}
    \xi =\vrp (X_t, Y_t, t)\,,\enskip X_t =\psi (Y_t, t)\,. %\label{e5.1-sin}
    \end{equation*}
В~этом случае вычисление интегралов (см.\ разд.~4) проводится по совокупности
переменных  $\lk X_t^{\mathrm{T}} Y_t^{\mathrm{T}}\rk^{\mathrm{T}}$.
К таким нелинейностям, например, относятся дроб\-но-ра\-ци\-о\-наль\-ные,
иррациональные  нелинейности, выражаемые специальными функциями, многозначные
нелинейности, зависящие от СтП~$X_t$ и его производных~$\dot X_t$,  $\ddot X_t$
и~др.

\medskip

\noindent
\textbf{Пример 2.}
Рассмотрим вычисление интегралов~(\ref{e4.1-sin}) и~(\ref{e4.2-sin})
для сложных одномерных иррациональных нелинейностей скалярного аргумента
\begin{equation}
\vrp (Y_t, t) =\lv Y_t\rrv^{\alpha-1}\, \mathrm{sgn}\, Y_t
\label{e5.2-sin}
\end{equation}
($\alpha$~--- нецелый показатель).

Пользуясь~(\ref{e2.16-sin}) и~(\ref{e2.19-sin}), представим~(\ref{e5.2-sin}) в виде
\begin{equation*}
\vrp(Y_t, t) = \vrp_0 (m_t, D_t, t) + k_1^\vrp(m_t, D_t, t) Y_t^0. %\label{e5.3-sin}
\end{equation*}
Здесь введены следующие обозначения:
\begin{gather*}
\vrp_0(m_t, D_t, t) =\Gamma(\alpha) D_t^{1/2} e^{-\xi^2/4} D_{-\alpha} (\xi)\,;%\label{e5.4-sin}
\\
k_1^a (m_t, D_t, t) =\fr {\prt \vrp_0(m_t, D_t, t)}{\prt m_t}\,,%\label{e5.5-sin}
\end{gather*}
где  $\Gamma(\alpha)$~--- гамма-функция,  $\xi \hm= m_t/\sqrt{D_t}$~---
отношение <<сиг\-нал--шум>>; $D_{-\alpha} (\xi)$~---
функция параболического цилиндра~\cite{9-sin}.
При вычислении были учтены следующие соотношения~\cite{9-sin, 8-sin}:
\begin{multline}
\iii_0^\infty x^{\alpha-1} e^{-\beta x^2 - \gamma x} \,dx ={}\\
{}=
(2\beta)^{-\alpha/2} \Gamma(\alpha) \exp \left(\fr{\gamma^2}{8\beta}\right)
D_{-\alpha} \left(\fr{\gamma}{\sqrt{2\beta}}\right)\,;\label{e5.6-sin}
\end{multline}

\vspace*{-12pt}

\noindent
\begin{multline}
\fr{dD_\rho(\xi)}{d\xi} =
   -\fr{\xi}{2}\, D_\rho (\xi) -\rho D_{\rho-1} (\xi) =
   \fr{\xi}{2}\, D_\rho (\xi) -{}\\
   {}- D_{\rho+1} (\xi) \enskip
   (\mathrm{Re}\, \beta>0\,,\enskip \mathrm{Re}\,\alpha>0\,,\enskip
   \rho=-\alpha)\,.\label{e5.7-sin}
   \end{multline}

Соотношения~(\ref{e5.6-sin}) и~(\ref{e5.7-sin})
могут быть использованы также для вычисления интегралов~(\ref{e4.3-sin}).

\medskip

\noindent
\textbf{Замечание.}
Для вычисления интегралов $I_0^a$, $I_1^a$ и $I_0^{\bar \si}$
применительно к типовым иррациональным нелинейностям вида
    $\lv Y_t\rrv^{\alp-1} e^{\delta Y_t}$, $\lv Y_t\rrv^{\alp-1}  \cos \w Y_t$,
    $\lv Y_t\rrv^{\alp-1}  \sin \w Y_t$
и более общим нелинейностям \mbox{вида}
    \begin{equation*}
    \vrp (Y_t, t) =\Phi^\vrp \left( \lv Y_t\rrv^{\alpha-1}, t\right) %\label{e5.8-sin}
    \end{equation*}
можно рекомендовать известные численные методы вычисления функций на ЭВМ~\cite{8-sin}.

\smallskip

\noindent
\textbf{Пример 3.}
Для нелинейной дроб\-но-ра\-ци\-о\-наль\-ной функции

\noindent
\begin{equation*}
\vrp (Y_t, t) = \fr{a}{(b+Y_t)^2} %\label{e5.9-sin}
\end{equation*}
имеем

\vspace*{-3pt}

\noindent
\begin{gather*}
\vrp_0 (m_t, D_t, t) =a b^{-2} \lk 1+ \chi (m_t, D_t, t)\rk\,; %\label{e5.10-sin}
\\
k_1^\vrp (m_t, D_t, t) =  a b^{-2}\fr{\prt \chi (m_t, D_t, t)}{\prt m_t}\,. %\label{e5.11-sin}
\end{gather*}
Здесь

\vspace*{-3pt}

\noindent
\begin{multline*}
\chi (m_t, D_t, t) ={}\\
{}=\sss_{n=1}^\infty \sss_{l=0}^{E(n/2)}
\fr{(-1)^n (n+1) n!}{(n-2l)! (2l)!}\, b^{-n} m_t^n \left( \fr{D_t}{ 2 m_t^2}
\right)^l, %\label{e5.12-sin}
\end{multline*}
где  $E(n/2)$~--- целая часть~$n/2$; $a\hm=a(t)$; $b\hm= b(t)$.

\vspace*{-6pt}

\section{Сложные интегральные нелинейности}

\vspace*{-2pt}

Пусть сначала векторно-матричная нелинейность имеет эредитарный характер, т.\,е.\
\begin{equation}
\underline{\vrp} (Y_t, t) =\iii_{t_0}^t A(t,\tau) \vrp (Y(\tau), \tau) \,d\tau\,.
\label{e6.1-sin}
\end{equation}
Тогда, как показано в~\cite{6-sin, 5-sin, 7-sin}, следует соответст\-ву\-ющие
интегродифференциальные соотношения путем введения  инструментальных
переменных привести к дифференциальным соотношениям.  Для
дифференцируемых функций~$\vrp$ и асимптотически устойчивых ядер
$A(t,\tau)$ зависимость~(\ref{e3.5-sin}) имеет следующий дифференциальный вид:
\begin{equation*}
F^A (t, D) \underline{\vrp} (Y_t, t) = H^A (t, D) \vrp (Y_t, t)\,. %\label{e6.2-sin}
\end{equation*}
Здесь $F^A (t, D)$ и  $H^A (t, D)$~--- линейные дифференциальные операторы $(D\hm= d/dt)$.

Для недифференцируемых функций~$\vrp$ и асимптотически устойчивых
сингулярных ядер~(\ref{e3.5-sin}) используются соотношения:
\begin{equation*}
\underline{\vrp} (Y_t, t) = A^+ Z\,,\enskip
\dot Z = A^- \vrp\,,\enskip
Z(t_0)=0\,. %\label{e6.3-sin}
\end{equation*}

Многочисленные примеры аналитического моделирования ЭСтС можно найти
в~[1--3, 5, 7, 10, 11].

Как отмечалось в~\cite{3-sin}, часто наряду с интегральными
нелинейностями~(\ref{e6.1-sin}) рассматривают нелинейности вида:

\columnbreak

\noindent
\begin{equation*}
Z_s =\sss_{\rho=1}^R \mathcal{A}_\rho \vrp_\rho (Y_{t_1}\tr Y_{t_r})\,, %\label{e6.2-sin}
\end{equation*}
где $\mathcal{A}_1 \tr \mathcal{A}_R$~--- произвольные линейные операторы,
действующие над функциями~$r$ переменных  $t_1\tr t_r$; $\vrp_\rho
\hm=\vrp_\rho (Y_{t_1} \tr Y_{t_r})$~--- линейные функции отмеченных
переменных. Такие нелинейности называются приводимыми к линейным.
Важным частным случаем~(\ref{e6.1-sin}) являются интегральные нелинейности вида:

\noindent
\begin{gather}
Z_s =\iii_T \vrp (Y_t, t, s)\, dt\,; \notag%\label{e6.3-sin}
\\
Z_s =\!\iii_T \!\cdots\!\iii_T\! \vrp (Y_{t_1}\tr Y_{t_r}; t_1\tr t_r, s)\,dt_1
\ldots dt_r,\notag %\label{e6.4-sin}
\end{gather}
В этом случае используется МСЛ по совокупности переменных  $Y_{t_1} \tr Y_{t_r}$.

\vspace*{-9pt}

\section{Заключение}

\vspace*{-2pt}

Разработаны методы и алгоритмы МНА и МСЛ для ДСтС и ЭСтС,
приводимых к ДСтС со сложными конечными, дроб\-но-ра\-ци\-о\-наль\-ны\-ми,
иррациональными, а также дифференциальными и интегральными нелинейностями.
Приведены примеры.

Результаты допускают обобщение на случай ДСтС и ЭСтС со
стохастическими нелинейностями, заданными каноническими разложениями и
интегральными каноническими  представлениями~\cite{1-sin, 3-sin, 11-sin}.

\vspace*{-9pt}

{\small\frenchspacing
 {%\baselineskip=10.8pt
 \addcontentsline{toc}{section}{References}
 \begin{thebibliography}{99}

 \vspace*{-2pt}

\bibitem{1-sin}
\Au{Синицын И.\,Н.,  Синицын~В.\,И.}
Лекции по нормальной и эллипсоидальной аппроксимации распределений в
стохастических сис\-те\-мах.~--- М.: ТОРУС ПРЕСС, 2013. 488~с.

\bibitem{6-sin} %2
\Au{Синицын И.\,Н. }
Stochastic hereditary control systems~// Проблемы управления и
теории информации, 1986. Т.~15. №\,4. С.~287--298.

\bibitem{2-sin} %3
\Au{Пугачев В.\,С., Синицын~И.\,Н.}
Стохастические дифференциальные сис\-те\-мы. Анализ и фильтрация.~--- М.:
Наука,  1990.  632~с. [Англ. пер.
 Stochastic differential systems.
Analysis and filtering.~--- Chichester, New York: Jonh Wiley, 1987.
549~p.].

\bibitem{5-sin} %4
\Au{Синицын И.\,Н. }
Конечномерные распределения процессов в стохастических интегральных
и интегродифференциальных системах~// Preprints of the 2nd IFAC
Symposium on Stochastic Control.~--- Vilnius: Pergamon Press,
1987.  Vol.~1. P.~144--153.

\bibitem{3-sin} %5
\Au{Пугачев В.\,С., Синицын~И.\,Н.}
Теория стохастических систем.~--- М.: Логос, 2000; 2004. 1000~с.
[Англ. пер.\linebreak\vspace*{-12pt}

\pagebreak

\noindent Stochastic systems. Theory and  applications.~---
Singapore: World Scientific, 2001. 908~p.].

\bibitem{4-sin} %6
\Au{Синицын И.\,Н.}
Параметрическое статистическое и аналитическое моделирование распределений
в нелинейных стохастических сис\-те\-мах на многообразиях~//
Информатика и её применения, 2013. Т.~7. Вып.~2. С.~4--16.

\bibitem{7-sin} %7
\Au{Синицын И.\,Н. }
Анализ и моделирование распределений в эредитарных стохастических
сис\-те\-мах~// Информатика и её применения, 2014. Т.~8. Вып.~1.\linebreak
С.~2--11.



\bibitem{9-sin} %8
\Au{Градштейн И.\,С., Рыжик~И.\,М.}
Таблицы интегралов, сумм, рядов и произведений.~--- М.: ГИФМЛ, 1963. 1100~с.

\bibitem{8-sin} %9
\Au{Попов Б.\,А., Теслер~Г.\,С. }
Вычисление функций на ЭВМ: Справочник.~--- Киев: Наукова Думка, 1984. 599~с.


\bibitem{11-sin} %10
\Au{Синицын И.\,Н.}
Канонические представления случайных функций и их применение в
задачах компьютерной поддержки научных исследований.~--- М.: ТОРУС
ПРЕСС, 2009. 768~с.

\bibitem{10-sin} %11
\Au{Синицын И.\,Н., Синицын~В.\,И., Корепанов~Э.\,Р., Белоусов~В.\,В.,
Сергеев~И.\,В., Басилашвили~Д.\,А.}
Опыт моделирования эредитарных стохастических сис\-тем~//
Кибернетика и высокие технологии XXI века: Сб. докл.  XIII Междунар.
науч.-технич. конф.~--- Воронеж: Саквоее, 2012. Т.~2. C.~346--357.

 \end{thebibliography}

 }
 }

\end{multicols}

\vspace*{-9pt}

\hfill{\small\textit{Поступила в редакцию 05.05.14}}

%\newpage

\vspace*{12pt}

\hrule

\vspace*{2pt}

\hrule

\vspace*{12pt}

\def\tit{ANALYTICAL MODELING OF NORMAL PROCESSES
 IN~STOCHASTIC SYSTEMS WITH~COMPLEX NONLINEARITIES}

\def\titkol{Analytical modeling of normal processes
 in~stochastic systems with~complex nonlinearities}

\def\aut{I.\,N.~Sinitsyn and V.\,I.~Sinitsyn}

\def\autkol{I.\,N.~Sinitsyn and V.\,I.~Sinitsyn}

\titel{\tit}{\aut}{\autkol}{\titkol}

\vspace*{-9pt}

\noindent
Institute of Informatics Problems, Russian Academy of Sciences,
44-2 Vavilov Str., Moscow 119333, Russian Federation


\def\leftfootline{\small{\textbf{\thepage}
\hfill INFORMATIKA I EE PRIMENENIYA~--- INFORMATICS AND
APPLICATIONS\ \ \ 2014\ \ \ volume~8\ \ \ issue\ 3}
}%
 \def\rightfootline{\small{INFORMATIKA I EE PRIMENENIYA~---
INFORMATICS AND APPLICATIONS\ \ \ 2014\ \ \ volume~8\ \ \ issue\ 3
\hfill \textbf{\thepage}}}

\vspace*{6pt}

\Abste{Differential stochastic systems (DStS) with Wiener and Poisson
noises and complex finite, differential, and  integral nonlinearities and
hereditary StS reducible to DStS are considered. Equations and algorithms
of analytical modeling based on the normal approximation method (NAM) and the
statistical linearization method (SLM) are given. The case of complex
continuous and discontinuous nonlinearities of scalar and vector arguments
is considered. Special attention is paid to NAM (SLM) typical integrals
for finite rational and irrational nonlinear and integral scalar and vector
nonlinear functions. The general case of integral nonlinearities reducible to
linear is considered. Test examples are given.}

\KWE{analytical modeling;
complex finite differential and integral nonlinearities;
complex irrational nonlinerarites
differential stochastic system with Wiener and Poisson noises;
method of normal approximation;
method of statistical linearization;
hereditary stochastic systems reducible to differential}

\DOI{10.14357/19922264140302}

  \begin{multicols}{2}

\renewcommand{\bibname}{\protect\rmfamily References}
%\renewcommand{\bibname}{\large\protect\rm References}

{\small\frenchspacing
 {%\baselineskip=10.8pt
 \addcontentsline{toc}{section}{References}
 \begin{thebibliography}{99}



\bibitem{1-sin-1}
\Aue{Sinitsyn, I.\,N., and  V.\,I.~Sinitsyn}.  2013.
Lektsii po normal'noy i ellipsoidal'noy approksimatsii raspredeleniy
v stokhasticheskikh sistemakh [Lectures on normal and ellipsoidal
approximation of distributions in stochastic systems].
Moscow: TORUS PRESS. 488~p.

\bibitem{6-sin-1} %2
\Aue{Sinitsyn, I.\,N.}  1986.
{Stochastic hereditary control systems}.
\textit{Problems Control Inform. Theory} 15(4):287--298.

\bibitem{2-sin-1} %3
\Aue{Pugachev, V.\,S., and  I.\,N.~Sinitsyn}.  1987.
\textit{Stochastic differential systems. Analysis and filtering.}
Chichester, New York: Jonh Wiley. 549~p.

\bibitem{5-sin-1} %4
\Aue{Sinitsyn, I.\,N.}  1987.
Konechnomernye raspredeleniya protsessov v stokhasticheskikh integral'nykh
i in\-teg\-ro\-dif\-fe\-ren\-tsial'nykh sistemakh [Finite dimensional distributions
of processes in stochastic integral and integrodifferential systems].
\textit{2nd  Symposium (International) IFAC on Stochastic Control
Preprints}. Vilnius: Pergamon Press. 1:144--153.

\bibitem{3-sin-1} %5
\Aue{Pugachev, V.\,S., and I.\,N.~Sinitsyn}. 2001.
\textit{Stochastic systems. Theory and  applications}.
Singapore: World Scientific. 908~p.

\bibitem{4-sin-1} %6
\Aue{Sinitsyn, I.\,N.}  2013.
Parametricheskoe statisticheskoe i analiticheskoe modelirovanie
raspredeleniy v nelineynykh stokhasticheskikh sistemakh na mnogoobraziyakh
[Parametric statistical and analytical modeling of distributions in
stochastic systems on manifolds].
\textit{Informatika i ee Primeneniya}~--- \textit{Inform. Appl.} 7(2):4--16.


\bibitem{7-sin-1} %7
\Aue{Sinitsyn, I.\,N.}  2014.
Analiz i modelirovanie raspredeleniy v ereditarnykh stokhasticheskikh sistemakh
[Analysis and modeling of distributions in hereditary stochastic systems].
\textit{Informatika i ee Primeneniya}~--- \textit{Inform. Appl.} 8(1):2--11.

\bibitem{9-sin-1} %8
\Aue{Gradshteyn, I.\,S., and I.\,M.~Ryzhik}.  1963.
\textit{Tablitsy integralov, summ, ryadov i proizvedeniy}
[Tables of integrals, sums, series, and products]. Moscow:  GIFML.   1100~p.

\pagebreak

\bibitem{8-sin-1} %9
\Aue{Popov, B.\,A., and G.\,S.~Tesler}.  1984.
\textit{Vychislenie funktsiy na EVM}. Spravochnik [Computing of functions].
Kiev: Naukova Dumka.  599~p.


\bibitem{11-sin-1} %10
\Au{Sinitsyn, I.\,N.} 2009.
\textit{Kanonicheskie predstavleniya sluchaynykh funktsiy i ikh primenenie v
zadachakh komp'yuternoy podderzhki nauchnykh issledovaniy}
[Canonical expansions of random functions and its application to
scientific computer-aided support]. Moscow: TORUS PRESS. 768~p.

\bibitem{10-sin-1} %11
\Aue{Sinitsyn, I.\,N., V.\,I.~Sinitsyn, E.\,R.~Korepanov,
V.\,V.~Belousov, I.\,V.~Sergeev, and D.\,A.~Basilashvili}.
2012. Opyt modelirovaniya ereditarnykh stokhasticheskikh sistem
[Experience of modeling in hereditary stochastic systems].
\textit{Kibernetika i Vysokie Tekhnologii XXI~Veka:
Sbornik dokladov  XIII Mezhdunar. nauch.-tekhnich. konf.}
[Cybernatics ans High Technologies of the XXI Century: Materials of
XIII  Scientific and Technological Conference (International)].
Voronezh: Sakvoee. 2:346--357.

\end{thebibliography}

 }
 }

\end{multicols}

\vspace*{-6pt}

\hfill{\small\textit{Received May 05, 2014}}

\vspace*{-18pt}

\Contr

\noindent
\textbf{Sinitsyn Igor N.} (b.\ 1940)~---
Doctor of Science in technology, professor, Honored scientist of RF, Head of Department, Institute of
Informatics Problems, Russian Academy of Sciences,
44-2 Vavilov Str., Moscow 119333, Russian
Federation; sinitsin@dol.ru

\vspace*{3pt}

\noindent
\textbf{Sinitsyn Vladimir I.} (b.\ 1968)~--- Doctor of Science in physics
and mathematics, associate professor, Head of Department, Institute of
Information Problems, Russian Academy of Sciences,
44-2 Vavilov Str., Moscow 119333, Russian Federation; VSinitsin@ipiran.ru




\label{end\stat}

\renewcommand{\bibname}{\protect\rm Литература}