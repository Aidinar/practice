
%\renewcommand{\le}{\leqslant}
%\renewcommand{\ge}{\geqslant}
%\renewcommand{\P}{\mathbf P}
%\newcommand{\E}{\mathbf E\,}
%\newcommand{\D}{\mathbf D\,}
%\renewcommand{\Im}{\mathrm{Im}\,}

\def\stat{ushakov}

\def\tit{АНАЛИЗ СИСТЕМЫ ОБСЛУЖИВАНИЯ С ВХОДЯЩИМ ПОТОКОМ АВТОРЕГРЕССИОННОГО ТИПА$^*$}

\def\titkol{Анализ системы обслуживания с входящим потоком авторегрессионного типа}

\def\aut{Н.\,Д.~Леонтьев$^1$, В.\,Г.~Ушаков$^2$}

\def\autkol{Н.\,Д.~Леонтьев, В.\,Г.~Ушаков}

\titel{\tit}{\aut}{\autkol}{\titkol}

{\renewcommand{\thefootnote}{\fnsymbol{footnote}} \footnotetext[1]
{Работа выполнена при частичной
финансовой поддержке РФФИ (проекты 12-07-00109а, 13-07-00223a).}}

\renewcommand{\thefootnote}{\arabic{footnote}}
\footnotetext[1]{Факультет вычислительной математики и кибернетики Московского государственного
университета им.\ М.\,В.~Ломоносова; ndleontyev@gmail.com}
\footnotetext[2]{Факультет вычислительной математики и кибернетики Московского государственного университета
им.\ М.\,В.~Ломоносова; Институт проблем информатики Российской академии наук;
vgushakov@mail.ru}


\Abst{Рассматривается одноканальная система
массового обслуживания с неограниченным числом мест для ожидания, в
которую поступает пуассоновский поток групп требований. Особенностью
системы является авторегрессионная зависимость размеров групп
поступающих требований: размер $n$-й поступившей в систему группы
требований либо с некоторой фиксированной вероятностью равен размеру
$(n-1)$-й поступившей в систему группы требований, либо с
дополнительной вероятностью является независимой от него случайной
величиной. Длительности обслуживания требований являются
независимыми случайными величинами с произвольным распределением.
Основным объектом изучения является длина очереди в произвольный
момент времени. Получены соотношения, позволяющие найти
преобразование Лапласа по времени производящей функции числа
требований в системе в нестационарном режиме, а также ряд
вспомогательных характеристик: время дообслуживания требования,
находящегося на приборе в момент~$t$, распределение размера
последней группы требований, поступившей в систему до момента~$t$.}

\KW{теория массового обслуживания; нестационарный режим; системы с групповым поступлением требований}

\DOI{10.14357/19922264140305}


\vskip 14pt plus 9pt minus 6pt

\thispagestyle{headings}

\begin{multicols}{2}

\label{st\stat}

\section{Введение}

При разработке сложных ин\-фор\-ма\-ци\-он\-но-вы\-чис\-ли\-тель\-ных сис\-тем важную
роль играют методы оценки интенсивности потоков запросов с целью
выявления недостаточных или избыточных вы\-чис\-ли\-тель\-ных ресурсов на
базе соответствующих моделей. В~данной работе рассматривается
система массового обслуживания типа $M|G|1$ с групповым поступлением
требований. При этом интенсивность потока запросов определяется
регрессионной зависимостью между размерами групп тре\-бо\-ваний.
{\looseness=1

}

Исследованию систем обслуживания с па\-ра\-мет\-ра\-ми регрессионного типа
посвящен целый ряд работ, опубликованных в специализированной
литературе в последние годы. Отметим статьи~[1,~2],\linebreak где изучается
система обслуживания с дискретным временем, в которой размеры
пакетов по\-сту\-па\-ющих требований связаны регрессионной за\-ви\-си\-мостью.

В основной теореме настоящей статьи приводятся формулы
для распределения длины очереди в произвольный момент времени.

\section{Описание системы}

Рассмотрим систему обслуживания, состоящую из одного обслуживающего
устройства, на которое поступает простейший поток групп требований с
интенсивностью~$a$. Будем считать, что число мест для ожидания не
ограничено, а длительность обслуживания требований имеет
распределение с плотностью $b(x)$ и преобразованием Лапласа
$\beta(s)$. Поступающие в систему группы требований могут иметь
размер $1,\ldots,M$ с вероятностями соответственно $h_1,\ldots,h_M$.
При этом размер $n$-й поступившей в систему группы требований с
вероятностью $0\hm\leq p\hm<1$ равен размеру $(n-1)$-й, либо с
дополнительной вероятностью $1\hm-p$ является независимой от него
случайной величиной.

Определим следующие случайные процессы:
\begin{itemize}
\item $L(t)$~--- число требований в системе в момент~$t$;
\item
$X(t)$~--- время, прошедшее с начала обслуживания требования,
находящегося на обслуживании в момент~$t$ (в случае, когда
система свободна, можно для определенности положить $X(t)\hm=0$);
\item
$N(t)$~--- размер последней поступившей в систему до момента~$t$
группы требований.
\end{itemize}

Введем следующие обозначения:

\noindent
\begin{align*}
P(n,k,x,t)&=\fr{\partial}{\partial x}\mathbf{P}(L(t)=n,N(t)=k,X(t)<x)\,;\\
P(n,k,t)&=\mathbf{P}(L(t)=n,N(t)=k)\,.
\end{align*}

Обозначим

\noindent
\begin{align*}
\pi(z,k,x,s)&=\sum\limits_{n=1}^{\infty}z^n\int\limits_0^{\infty}e^{-st}
P(n,k,x,t)\,dt\,;\\
\pi_0(k,s)&=\int\limits_0^{\infty}e^{-st} P(0,k,t)\,dt
\end{align*}
при $|z|\leq1$, $\Re(s)\hm>0$.

\vspace*{-4pt}

\section{Основные результаты}

Формула для определения $\pi(z,k,x,s)$ и ее вывод содержатся в
следующей основной теореме.

\smallskip

\noindent
\textbf{Теорема.}
\textit{Функция $\pi(z,k,x,s)$ при $|z|\hm<1$, $k\hm=1,\ldots,M$, $x\hm\geq 0$,
$\Re(s)\hm>0$ определяется по формуле}:

\noindent
\begin{multline*}
\pi(z,k,x,s)=(1-B(x))\times{}\\
{}\times \sum\limits_{n=1}^{M}C_n(z,s)
\fr{(1-p)h_k az^k}{\widetilde{\lambda}_n(z)-paz^k}
\times{}\\
{}\times \exp\left(-\left(
s+a-\widetilde{\lambda}_n(z)\right)x\right),
\end{multline*}
\textit{где}

\noindent
\begin{multline*}
C_n(z,s)=\fr{1}{1-z^{-1}\beta\left(s+a-\widetilde{\lambda}_n(z)\right)}\times{}\\
{}\times
\fr{\prod\limits_{i=1}^M\left(\widetilde{\lambda}_n(z)-paz^i\right)}
{\prod\limits_{\substack{j=1\\j\neq n}}^M
\left(\widetilde{\lambda}_n(z)-\widetilde{\lambda}_j(z)\right)}
\sum\limits_{m=1}^M
\fr{b_m(z,s)}{\widetilde{\lambda}_n(z)-paz^m}\,;
\end{multline*}

\vspace*{-14pt}

\noindent
\begin{multline*}
b_m (z,s)=-(s+a)\pi_0(m,s)+h_m+{}\\
{}+\left[p\pi_0(m,s)az^m+(1-p)
\sum\limits_{n=1}^{M}\pi_0(n,s)h_m az^m\right]\,;
\end{multline*}
$\widetilde{\lambda}_1(z),\ldots,\widetilde{\lambda}_M(z)$ определяются из уравнения
\begin{equation*}
\prod\limits_{i=1}^M\left(paz^i-\widetilde{\lambda}\right)+
\sum_{i=1}^M(1-p)h_i az^i\prod_{\substack{j=1\\j\neq i}}^M(paz^j-\widetilde{\lambda})=0,
\end{equation*}
а $\pi_0(1,s),\ldots,\pi_0(M,s)$~--- из системы


\noindent
\begin{multline*}
\hspace*{-6.1518pt}\sum\limits_{m=1}^M\prod\limits_{\substack{j=1\\j\neq m}}^M(\widetilde{\lambda}_n(z_n)-
paz_n^j)\left(
\vphantom{\widetilde{\lambda}_n}
-(s+a)\pi_0(m,s)+h_m+{}\right.\\[-6pt]
\left.{}+\widetilde{\lambda}_n(z_n)\pi_0(m,s)\right)=0,
\end{multline*}
%\begin{multline*}
%\sum\limits_{m=1}^M\prod\limits_{\substack{j=1\\j\neq m}}^M(\widetilde{\lambda}_n-
%paz_n^j)\left(
%\vphantom{\widetilde{\lambda}_n}
%-(s+a)\pi_0(m,s)+h_m+{}\right.\\
%\left.{}+\widetilde{\lambda}_n \pi_0(m,s)\right)=0\,.
%\end{multline*}
в которой $z_n=z_n(s)$~--- решение функционального уравнения

\vspace*{-2pt}

\noindent
$$
z_n=\beta(s+a-\widetilde{\lambda}_n(z_n))\,.
$$

\vspace*{-2pt}


\noindent
Д\,о\,к\,а\,з\,а\,т\,е\,л\,ь\,с\,т\,в\,о\,.\ \
Функционирование системы описывается следующими уравнениями:

\vspace*{-1pt}

\noindent
\begin{multline*}
P(n,k,x+\Delta,t+\Delta)=P(n,k,x,t)[1-(a+\eta(x))\Delta]+{}\\
{}+\mathbf{1}_{\{n>k\}}\left[
\vphantom{\sum\limits_{m=1}^{M}}
pP(n-k,k,x,t)a\Delta+{}\right.\\
\left.{}+(1-p)\sum\limits_{m=1}^{M}
P(n-k,m,x,t)h_k a\Delta\right]\,; %\label{before1}
\end{multline*}

\vspace*{-14pt}

\noindent
\begin{multline*}
P(0,k,t+\Delta)=P(0,k,t)[1-a\Delta]+{}\\
{}+\int\limits_0^{\infty}
P(1,k,x,t)\eta(x)dx\Delta\,; %\label{before2}
\end{multline*}

\vspace*{-16pt}

\noindent
\begin{multline*}
\int\limits_0^\Delta P(n,k,u,t+\Delta)\,du={}\\[-0.5pt]
{}=
\int\limits_0^{\infty}P(n+1,k,x,t)\eta(x)\,dx\Delta+
\delta_{n,k}\left[
\vphantom{\sum\limits_{m=1}^{M}}
pP(0,k,t)a\Delta+{}\right.\\[-0.5pt]
\left.{}+(1-p)\sum\limits_{m=1}^{M}
P(0,m,t)h_k a\Delta\right]\,,
%\label{before3}
\end{multline*}
где $\eta(x)\hm=b(x)/(1-B(x))$.

Переходя к пределу при $\Delta\hm\rightarrow0$, получим:

\noindent
\begin{multline}
\fr{\partial P(n,k,x,t)}{\partial t}+\fr{\partial P(n,k,x,t)}{\partial x}={}\\[-0.5pt]
{}=
-(a+\eta(x))P(n,k,x,t)+{}\\[-0.5pt]
{}+\mathbf{1}_{\{n>k\}}\left[
\vphantom{\sum\limits_{m=1}^{M}}
pP(n-k,k,x,t)a+{}\right.\\[-0.5pt]
\left.{}+(1-p)
\sum\limits_{m=1}^{M}P(n-k,m,x,t)h_k a\right]\,;\label{after1}
\end{multline}

\vspace*{-14pt}

\noindent
\begin{multline}
\fr{\partial P(0,k,t)}{\partial t}={}\\[-2pt]
{}=-aP(0,k,t)+
\int\limits_0^{\infty}P(1,k,x,t)\eta(x)\,dx\,;\label{after2}
\end{multline}

\vspace*{-14pt}

\noindent
\begin{multline}
P(n,k,0,t)=\int\limits_0^{\infty}P(n+1,k,x,t)\eta(x)\,dx+{}\\[-0.5pt]
{}+\delta_{n,k}\!\left[pP(0,k,t)a+(1-p)\!\sum\limits_{m=1}^{M}
\!\!P(0,m,t)h_k a\right].\!\label{after3}
\end{multline}

Переходя в уравнениях~(\ref{after1})--(\ref{after3}) к производящим
функциям и преобразованиям Лапласа по~$t$, получим:
\begin{multline*}
\fr{\partial \pi(z,k,x,s)}{\partial x}=-(s+a+\eta(x))\pi(z,k,x,s)+{}\\
{}+\left[p\pi(z,k,x,s)az^k+(1-p)\!\sum\limits_{m=1}^{M}\!\!
\pi(z,m,x,s)h_k az^k\right];\hspace*{-6.45119pt} %\label{first}
\end{multline*}

\vspace*{-12pt}

\noindent
\begin{multline}
(s+a)\pi_0(k,s)-h_k={}\\
{}=\int\limits_0^{\infty}\int\limits_0^{\infty}
P(1,k,x,t)\eta(x)\,dx\, e^{-st}\,dt\,;\label{second}
\end{multline}

\vspace*{-12pt}

\noindent
\begin{multline}
\pi(z,k,0,s)=z^{-1}\int\limits_0^{\infty}\pi(z,k,x,s)
\eta(x)\,dx-{}\\
{}-\int\limits_0^{\infty}\int\limits_0^{\infty}
P(1,k,x,t)\eta(x)\,dx e^{-st}\,dt+{}\\
{}+\left[p\pi_0(k,s)az^k+(1-p)\sum\limits_{m=1}^{M}\pi_0(m,s)h_k az^k\right]\,.
\label{third}
\end{multline}

Подставляя~(\ref{second}) в~(\ref{third}), получим:
\begin{multline}
\fr{\partial \pi(z,k,x,s)}{\partial x}={}\\
{}=-(s+a+\eta(x))\pi(z,k,x,s)+
\left[\vphantom{\sum\limits_{m=1}^{M}}
p\pi(z,k,x,s)az^k+{}\right.\\
\left.{}+(1-p)\sum\limits_{m=1}^{M}\pi(z,m,x,s)h_k az^k\right]\,;
\label{newfirst}
\end{multline}

\vspace*{-12pt}

\noindent
\begin{multline}
\pi(z,k,0,s)=z^{-1}\int\limits_0^{\infty}\pi(z,k,x,s)\eta(x)\,dx-{}\\
{}-
(s+a)\pi_0(k,s)+h_k+{}\\
{}+\left[p\pi_0(k,s)az^k+(1-p)\sum\limits_{m=1}^{M}\pi_0(m,s)h_k az^k\right]\,.
\label{newsecond}
\end{multline}

Обозначим
$$
\pi(z,k,x,s)=(1-B(x))\widetilde{\pi}(z,k,x,s)\,.
$$
В новых обозначениях~(\ref{newfirst}) примет вид:
\begin{multline}
\fr{\partial\widetilde{\pi}(z,k,x,s)}{\partial x}={}\\
{}=
-(s+a)\widetilde{\pi}(z,k,x,s)+\left[
\vphantom{\sum\limits_{m=1}^{M}}
p\widetilde{\pi}(z,k,x,s)az^k+{}\right.\\
\left.{}+(1-p)
\sum\limits_{m=1}^{M}\widetilde{\pi}(z,m,x,s)h_k az^k\right]\,.\label{system}
\end{multline}
Это линейная система дифференциальных уравнений первого порядка с постоянными
коэффициентами. Принимая во внимание результат леммы, приведенной ниже, решение
сис\-те\-мы можно записать в виде:
\begin{multline}
\widetilde{\pi}(z,k,x,s)={}\\
{}=\sum\limits_{n=1}^{M}C_n(z,s)u_{kn}(z)
\exp(-(s+a-\widetilde{\lambda}_n(z))x)\,,
\label{solution}
\end{multline}
где $\widetilde{\lambda}_k (z)\hm=\lambda_k (z,s)+(s+a)$,
$k\hm=1,\ldots ,M$;\linebreak $\lambda_1(z,s),\ldots,\lambda_M(z,s)$~---
собственные значения матрицы системы, а
$u_1(z)\hm=(u_{11}(z),\ldots$\linebreak $\ldots ,u_{M1}(z))^{\mathrm{T}},\ldots,
u_M (z)$=$(u_{1M}(z),\ldots,u_{MM}(z))^{\mathrm{T}}$~--- соответствующие собственные векторы.

Функции $\widetilde{\lambda}_1(z),\ldots,\widetilde{\lambda}_M(z)$
являются решениями характеристического уравнения
\begin{equation}
\prod\limits_{i=1}^M(paz^i-\widetilde{\lambda})+
\sum\limits_{i=1}^M(1-p)h_i az^i
\prod\limits_{\substack{j=1\\j\neq i}}^M(paz^j-\widetilde{\lambda})=0\,.
\label{determinant}
\end{equation}

Установим следующее вспомогательное утверж\-де\-ние.

\smallskip

\noindent
\textbf{Лемма.}\
\textit{Функции $\widetilde{\lambda}_1(z),\ldots,\widetilde{\lambda}_M(z)$
обладают следу\-ющи\-ми свойствами:}
\begin{enumerate}[(1)]
\item \textit{$\Re(\widetilde{\lambda}_m(z))\leq a$ для всех $m$ и $|z|\leq 1$};
\item \textit{функции $\widetilde{\lambda}_i(z)$ и $\widetilde{\lambda}_j(z)$ при
$i\hm\neq j$ могут совпадать лишь на конечном множестве точек}~$z$.
\end{enumerate}

\noindent
Д\,о\,к\,а\,з\,а\,т\,е\,л\,ь\,с\,т\,в\,о\ \ леммы
полностью аналогично доказательству леммы~1 из~[3] (с.~181).

\smallskip

Вернемся к доказательству теоремы. Под\-став\-ляя~(\ref{solution})
в~(\ref{system}), находим:
\begin{equation}
u_{mn}(z)=\fr{(1-p)h_m az^m}{\widetilde{\lambda}_n(z)-paz^m}\,.\label{eigenvectors}
\end{equation}

Подставив~(\ref{solution}) в~(\ref{newsecond}), получим:
\begin{multline*}
\sum\limits_{n=1}^{M}C_n(z,s)u_{mn}(z)={}\\
{}=
z^{-1}\sum\limits_{n=1}^{M}C_n(z,s)u_{mn}(z)
\beta(s+a-\widetilde{\lambda}_n(z))-{}\\
{}-(s+a)\pi_0(m,s)+h_m+{}\\
{}+\left[p\pi_0(m,s)az^m+(1-p)\sum\limits_{n=1}^{M}\pi_0(n,s)h_m az^m\right]\,.
\end{multline*}
Перепишем это уравнение в виде:
\begin{multline}
\hspace*{-2.96277pt}\sum\limits_{n=1}^{M}C_n(z,s)\left(1-z^{-1}\beta\left(s+a-\widetilde{\lambda}_n(z)\right)
\right)u_{mn}(z)={}\\
{}=b_m(z,s),\label{linear}
\end{multline}
где
\begin{multline*}
b_m (z,s)=-(s+a)\pi_0(m,s)+h_m+{}\\
{}+\left[p\pi_0(m,s)az^m+(1-p)
\sum\limits_{n=1}^{M}\pi_0(n,s)h_m az^m\right]\,.
\end{multline*}

Подставим (\ref{eigenvectors}) в~(\ref{linear}) и поделим обе части полученного уравнения на $(1-p)h_m az^m$:
\begin{multline*}
\sum\limits_{n=1}^{M}C_n(z,s)\left(1-z^{-1}\beta
\left(s+a-\widetilde{\lambda}_n(z)\right)\right)\times{}\\
{}\times\fr{1}{\widetilde{\lambda}_n(z)-paz^m}=\fr{b_m(z,s)}{(1-p)h_m az^m}\,.
\end{multline*}
Это система линейных алгебраических уравнений с матрицей Коши. Ее решение записывается в виде\footnote{Про обращение матриц Коши см.~[4].}:
\begin{multline*}
C_n(z,s)\left(1-z^{-1}\beta\left(s+a-\widetilde{\lambda}_n(z)\right)\right)={}\\
{}=
\fr{\prod\limits_{i=1}^M\left(\widetilde{\lambda}_n(z)-paz^i\right)}
{\prod\limits_{\substack{j=1\\j\neq n}}^M\left(
\widetilde{\lambda}_n(z)-\widetilde{\lambda}_j(z)\right)}\times{}\\
{}\times
\sum\limits_{m=1}^M\fr{\prod\limits_{\substack{i=1\\i\neq n}}^M
\left(paz^m-\widetilde{\lambda}_i(z)\right)}
{\prod\limits_{\substack{j=1\\j\neq m}}^M\left(paz^m-paz^j\right)}\,
\fr{b_m(z,s)}{(1-p)h_m az^m}\,.
\end{multline*}
Далее, поскольку функции $\widetilde{\lambda}_m(s)$, $m=1,\ldots,M$, являются решениями уравнения (\ref{determinant}), можно записать:
\begin{multline}
\prod_{i=1}^M(paz^i-\widetilde{\lambda})+\sum_{i=1}^M(1-p)h_i az^i
\prod\limits_{\substack{j=1\\j\neq i}}^M\left(paz^j-\widetilde{\lambda}\right)={}\\
{}=
\prod_{j=1}^M\left(\widetilde{\lambda}_j(z)-\widetilde{\lambda}\right)\,.\label{polynom}
\end{multline}
Подставляя в (\ref{polynom}) $\widetilde{\lambda}=paz^m$, получим:
\begin{multline*}
(1-p)h_m az^m\prod\limits_{\substack{j=1\\j\neq m}}^M\left(paz^j-paz^m\right)={}\\
{}=\prod_{j=1}^M\left(\widetilde{\lambda}_j(z)-paz^m\right)\,.
\end{multline*}
Отсюда
\begin{multline}
C_n(z,s)=\fr{1}{1-z^{-1}\beta\left(s+a-\widetilde{\lambda}_n(z)\right)}\times{}\\
{}\times
\fr{\prod\limits_{i=1}^M\left(\widetilde{\lambda}_n(z)-paz^i\right)}
{\prod\limits_{\substack{j=1\\j\neq n}}^M
\left(\widetilde{\lambda}_n(z)-\widetilde{\lambda}_j(z)\right)}
\sum\limits_{m=1}^M\fr{b_m(z,s)}{\widetilde{\lambda}_n(z)-paz^m}\,. \label{C}
\end{multline}

Остается найти $\pi_0(m,s)$, $m\hm=1,\ldots,M$. Домножим
уравнение~(\ref{C}) на
$1\hm-z^{-1}\beta(s\hm+a\hm-\widetilde{\lambda}_n(z))$:
\begin{multline}
C_n(z,s)\left(1-z^{-1}\beta\left(s+a-\widetilde{\lambda}_n(z)\right)\right)={}\\
{}=
\fr{\prod\limits_{i=1}^M\left(\widetilde{\lambda}_n(z)-paz^i\right)}
{\prod\limits_{\substack{j=1\\j\neq n}}^M
\left(\widetilde{\lambda}_n(z)-
\widetilde{\lambda}_j(z)\right)}
\sum\limits_{m=1}^M\fr{b_m(z,s)}{\widetilde{\lambda}_n(z)-paz^m}\,.
\label{findpi1}
\end{multline}
Рассмотрим уравнение
\begin{equation}
z=\beta(s+a-\widetilde{\lambda}_n(z))\,.\label{functional}
\end{equation}
Обе части уравнения являются аналитическими в области $|z|\hm\leq 1$ функциями.
Имеем, с учетом доказанной леммы,
\begin{multline*}
\left\vert \beta\left(s+a-\widetilde{\lambda}_n(z)\right)\right\vert
\leq{}\\
{}\leq \beta\left(\Re\left(s+a-\widetilde{\lambda}_n(z)\right)\right)
\leq\beta(\Re(s))<1=|z|
\end{multline*}
при $|z|=1$. В~силу теоремы Руше отсюда следует, что функциональное
уравнение~(\ref{functional}) имеет единственное решение $z\hm=z_n(s)$,
причем функция $z_n(s)$ является аналитической в области $\Re(s)\hm>0$.

Подставляя $z\hm=z_n(s)$ в уравнение~(\ref{findpi1}), приходим после ряда
преобразований к уравнению\footnote{В дальнейшем будем для краткости писать
$z_n$ вместо $z_n(s)$ и $\widetilde{\lambda}_n$ вместо
$\widetilde{\lambda}_n(z_n)$.}:
\begin{equation}
\sum\limits_{m=1}^M\prod\limits_{\substack{j=1\\j\neq m}}^M
\left(\widetilde{\lambda}_n-
paz_n^j\right) b_m(z_n,s)=0\,.\label{findpi2}
\end{equation}
Вспомним, что $b_m(z,s)\hm=-(s\hm+a)\pi_0(m,s)\hm+h_m+
\left[p\pi_0(m,s)az^m+(1-p)\sum\limits_{k=1}^{M}\pi_0(k,s)h_m az^m\right]$.
С~учетом уравнения~(\ref{determinant}) будем иметь:
\begin{multline*}
\sum\limits_{m=1}^M\prod\limits_{\substack{j=1\\j\neq m}}^M
\left(\widetilde{\lambda}_n- paz_n^j\right)\left[
\vphantom{\sum\limits_{k=1}^{M}}
p\pi_0(m,s)az_n^m+{}\right.\\[-6pt]
\left.{}+(1-p)\sum\limits_{k=1}^{M}\pi_0(k,s)h_m az_n^m\right]={}\\[-4pt]
{}=\sum\limits_{m=1}^M\prod\limits_{\substack{j=1\\j\neq m}}^M
\left(\widetilde{\lambda}_n- paz_n^j\right)\left[
\vphantom{\sum\limits_{k=1}^{M}}
(paz_n^m-\widetilde{\lambda}_n)\pi_0(m,s)+{}\right.\\[-4pt]
\left.{}+
\widetilde{\lambda}_n \pi_0(m,s)+(1-p)\sum\limits_{k=1}^{M}\pi_0(k,s)h_m az_n^m\right]={}\\[-4pt]
{}=-\prod\limits_{j=1}^M
\left(\widetilde{\lambda}_n-paz_n^j\right)\sum\limits_{m=1}^M \pi_0(m,s)+{}\\[-4pt]
{}+
\sum\limits_{m=1}^M\prod\limits_{\substack{j=1\\j\neq m}}^M
\left(\widetilde{\lambda}_n-paz_n^j\right)\widetilde{\lambda}_n \pi_0(m,s)+{}\\[-4pt]
{}+\prod\limits_{j=1}^M
\left(\widetilde{\lambda}_n-paz_n^j\right)\sum\limits_{m=1}^M \pi_0(m,s)={}\\[-4pt]
{}=\sum\limits_{m=1}^M\prod\limits_{\substack{j=1\\j\neq m}}^M
\left(\widetilde{\lambda}_n-paz_n^j\right)\widetilde{\lambda}_n \pi_0(m,s)\,.
\end{multline*}
Возвращаясь к уравнению~(\ref{findpi2}), получим:
\begin{multline*}
\sum\limits_{m=1}^M\prod\limits_{\substack{j=1\\j\neq m}}^M
\left(\widetilde{\lambda}_n-paz_n^j\right)\left(
\vphantom{\widetilde{\lambda}_n}
-(s+a)\pi_0(m,s)+h_m+{}\right.\\
\left.{}+\widetilde{\lambda}_n \pi_0(m,s)\right)=0\,.
\end{multline*}

Там самым доказательство теоремы завершено.

{\small\frenchspacing
 {%\baselineskip=10.8pt
 \addcontentsline{toc}{section}{References}
 \begin{thebibliography}{9}
\bibitem{2-ush}
\Au{Hwang G.\,U., Choi~B.\,D., Kim~J.-K.}
The waiting time analysis of a discrete-time queue with arrivals as an
autoregressive process of order~1~//
J.~Appl. Probab., 2002. Vol.~39. No.\,3. P.~619--629.

\bibitem{1-ush}
\Au{Hwang G.\,U., Sohraby~K.} On the exact analysis of a discrete-time
queueing system with autoregressive inputs~// Queueing Syst., 2003.
Vol.~43. No.\,1--2. P.~29--41.

\bibitem{3-ush}
\Au{Матвеев В.\,Ф., Ушаков В.\,Г.} Системы массового обслуживания.~---
М.: МГУ, 1984. 242~с.
\bibitem{4-ush}
\Au{Schechter S.} On the inversion of certain matrices~//
Math. Tab. Aids Comput., 1959. Vol.~13. No.\,66. P.~73--77.

 \end{thebibliography}

 }
 }

\end{multicols}

\vspace*{-6pt}

\hfill{\small\textit{Поступила в редакцию 14.07.14}}

%\newpage

\vspace*{8pt}

\hrule

\vspace*{2pt}

\hrule

%\vspace*{8pt}

\def\tit{ANALYSIS OF A QUEUEING SYSTEM WITH~AUTOREGRESSIVE ARRIVALS}

\def\titkol{Analysis of a queueing system with autoregressive arrivals}

\def\aut{N.\,D.~Leontyev$^1$ and V.\,G.~Ushakov$^{1,2}$}

\def\autkol{N.\,D.~Leontyev and V.\,G.~Ushakov}

\titel{\tit}{\aut}{\autkol}{\titkol}

\vspace*{-9pt}

\noindent
$^1$Faculty of Computational Mathematics and Cybernetics,
M.\,V.~Lomonosov Moscow State University, 1-52\linebreak
$\hphantom{^1}$Leninskiye Gory,
Moscow 119991, GSP-1, Russian Federation

\noindent
$^2$Institute of Informatics Problems, Russian Academy of Sciences,
44-2 Vavilov Str., Moscow 119333, Russian\linebreak
$\hphantom{^1}$Federation


\def\leftfootline{\small{\textbf{\thepage}
\hfill INFORMATIKA I EE PRIMENENIYA~--- INFORMATICS AND
APPLICATIONS\ \ \ 2014\ \ \ volume~8\ \ \ issue\ 3}
}%
 \def\rightfootline{\small{INFORMATIKA I EE PRIMENENIYA~---
INFORMATICS AND APPLICATIONS\ \ \ 2014\ \ \ volume~8\ \ \ issue\ 3
\hfill \textbf{\thepage}}}

\vspace*{6pt}

\Abste{The paper studies a single server queueing system with infinite
capacity and with the Poisson batch arrival process. A feature of the system
under study is autoregressive dependence of the arriving batch sizes:
the size of the $n$th batch is equal to the size of the $(n-1)$st batch
with a fixed probability and is an independent random variable with
complementary probability. Service times are supposed to be independent
random variables with a specified distribution. The main object of the study
is the queue length at an arbitrary moment. The relations derived make it
possible to find Laplace transorm in time of the probability generating function of
the transient queue length, and also a number of additional characteristics
such as the residual service time and the distribution of the size of the
last batch that arrived before time~$t$.}

\KWE{queueing theory; transient behavior; batch arrivals}

\DOI{10.14357/19922264140305}

\vspace*{-3pt}

\Ack
The research was partially supported by the Russian Foundation for Basic Research
(Grants Nos.\,12-07-00109a and 13-07-00223a).

 \begin{multicols}{2}

\renewcommand{\bibname}{\protect\rmfamily References}
%\renewcommand{\bibname}{\large\protect\rm References}

{\small\frenchspacing
 {%\baselineskip=10.8pt
 \addcontentsline{toc}{section}{References}
 \begin{thebibliography}{9}


\bibitem{2-ush-1}
\Aue{Hwang, G.\,U., B.\,D.~Choi, and J.-K.~Kim}. 2002.
The waiting time analysis of a discrete-time queue with arrivals as an
autoregressive process of order~1. \textit{J.~Appl. Probab.} 39(3):619--629.

\bibitem{1-ush-1}
\Aue{Hwang, G.\,U., and K.~Sohraby}. 2003.
On the exact analysis of a discrete-time queueing system with autoregressive
inputs. \textit{Queueing Syst.} 43(1-2):29--41.

\vspace*{-1pt}

\bibitem{3-ush-1}
\Aue{Matveev, V.\,F., and V.\,G.~Ushakov}. 1984.
\textit{Sistemy massovogo obsluzhivaniya} [{Queueing systems}]. Moscow: MSU. 242~p.

\vspace*{-1pt}

\bibitem{4-ush-1}
\Aue{Schechter,~S.} 1959. On the inversion of certain matrices.
\textit{Math. Tab. Aids Comput.} 13(66):73--77.

\end{thebibliography}

 }
 }

\end{multicols}

\vspace*{-6pt}

\hfill{\small\textit{Received July 14, 2014}}

\vspace*{-12pt}

\Contr

\noindent
\textbf{Leontyev Nikolay D.} (b.\ 1988)~---
PhD student, Department of Mathematical Statistics,
Faculty of Computational Mathematics and Cybernetics,
M.\,V.~Lomonosov Moscow State University,
1-52 Leninskiye Gory,
Moscow 119991, GSP-1, Russian Federation; ndleontyev@gmail.com

\vspace*{3pt}

\noindent
\textbf{Ushakov Vladimir G.} (b.\ 1952)~---
Doctor of Science in physics and mathematics, professor,
Department of Mathematical Statistics, Faculty of Computational
Mathematics and Cybernetics, M.\,V.~Lomonosov Moscow State University,
1-52 Leninskiye Gory,
Moscow 119991, GSP-1, Russian Federation;
senior scientist, Institute of Informatics Problems,
Russian Academy of Sciences, 44-2 Vavilov Str., Moscow 119333, Russian
Federation; vgushakov@mail.ru


\label{end\stat}

\renewcommand{\bibname}{\protect\rm Литература}