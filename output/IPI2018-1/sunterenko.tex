\def\stat{suntarenko}

\def\tit{ФИНАНСИРОВАНИЕ ФУНДАМЕНТАЛЬНЫХ ИССЛЕДОВАНИЙ: КОНЦЕПТУАЛЬНЫЙ 
ОБЛИК СИСТЕМЫ ПОДДЕРЖКИ ПРИНЯТИЯ РЕШЕНИЙ С~ИСПОЛЬЗОВАНИЕМ 
МЕТОДОВ НАУКОМЕТРИИ И~АНАЛИЗА ДАННЫХ$^*$}

\def\titkol{Финансирование фундаментальных исследований: концептуальный 
облик СППР %системы поддержки принятия решений 
с~использованием 
%методов 
наукометрии}  % и~анализа данных}

\def\aut{О.\,В. Сюнтюренко$^1$}

\def\autkol{О.\,В. Сюнтюренко}

\titel{\tit}{\aut}{\autkol}{\titkol}

\index{Сюнтюренко О.\,В.}
\index{Syuntyurenko O.\,V.}




{\renewcommand{\thefootnote}{\fnsymbol{footnote}} \footnotetext[1]
{Работа поддержана РФФИ (проект 17-07-00256).}}


\renewcommand{\thefootnote}{\arabic{footnote}}
\footnotetext[1]{Всероссийский институт научной и~технической информации Российской академии наук, 
\mbox{olegasu@mail.ru}}

\vspace*{-6pt}

    
      
   
     
     \Abst{Статья посвящена разработке концептуальной модели информационной системы, 
ориентированной на решение задач управления финансированием фундаментальных 
исследований с~использованием методов наукометрии и~анализа данных. Сформулированы 
базовые принципы и~предложена методология создания и~функционирования системы 
поддержки принятия решений (СППР) при распределении фонда госзаказа по тематическим 
научным направлениям. Представлена функциональная структура СППР. Рассматриваются 
перспективы использования нового многоцелевого информационного ресурса 
и~аналитической постобработки информации. Показана взаимосвязь задач управления 
наукой с~задачами инновационного развития отраслей экономики.}
     
     \KW{наукометрия; анализ данных; сис\-те\-мы управ\-ле\-ния; научные приоритеты; 
мониторинг; на\-уч\-но-тех\-ни\-че\-ский потенциал; постобработка данных}

\DOI{10.14357/19922264180115} 
  
\vspace*{-6pt}


\vskip 10pt plus 9pt minus 6pt

\thispagestyle{headings}

\begin{multicols}{2}

\label{st\stat}
    
    \section{Введение}
    
    Эффективность управления и~объемы фи\-нансирования фундаментальных 
исследований в~наиболь\-шей степени (сравнительно с~другими факторами) 
влияют на их результативность~[1]. В~настоящее время для оценивания 
результативности научной деятельности совместно с~экс\-пертными 
заключениями все чаще используют и~наукометрические показатели~[2]. Эти 
показатели\linebreak основаны на числе публикаций автора и~на количестве ссылок на 
его работы. Возросший интерес\linebreak
 к~наукометричеcким показателям вызван 
в~первую очередь возможностью автоматизации процесса оценивания 
с~использованием программных средств баз данных (БД)  Web of Science, Scopus, 
РИНЦ (Российский индекс научного цитирования). 
Кроме того, можно использовать бесплатные программы, например 
Publish or Perish, ра\-бо\-тающие на данных поисковой системы научных\linebreak 
пуб\-ли\-ка\-ций Google Scholar. Доступность таких показателей связана 
с~развитием электронных биб\-лио\-гра\-фи\-че\-ских БД и,~как следствие, 
возможностей автоматического расчета соответствующих индексов. 
Потенциально эти показатели могут использоваться на всех этапах процесса 
управления на\-уч\-но-ис\-сле\-до\-ва\-тель\-ской деятельностью, и~преж\-де 
всего

\columnbreak

\noindent
\begin{itemize}
\item при формировании приоритетных направлений развития и~при 
создании новых исследовательских центров и~групп;\\[-14pt] 
\item при распределении 
финансирования между исследовательскими программами и~проектами.
\end{itemize}

 Ниже 
рассмотрена модель и~концептуальный облик СППР 
при управ\-ле\-нии бюджетным финансированием тематических 
направлений с~использованием критериев  
и~ин\-фор\-ма\-ци\-он\-но-ана\-ли\-ти\-че\-ских методов наукометрии и~анализа 
данных.\linebreak
 Нетривиальность задачи заключается в~ее значительной размерности: 
распределение (и~мониторинг) выделяемых бюджетных средств по 
$\sim8000$~научных направлений, исследования по которым ведут\-ся в~$\sim 
400$~научных организациях Российской академии наук (РАН). 

\vspace*{-12pt}
    
    \section{Концептуальные основы информационной системы 
управления финансированием фундаментальных исследований }
    
    Планирование развития науки в~целом или ка\-кой-ли\-бо ее отрасли, 
перераспределение средств и~капитальных вложений требуют учета тенден-\linebreak\vspace*{-12pt}

\noindent
ций 
развития науки, прогнозирования появления и~отмирания ее различных 
направлений. С~развитием компьютерных технологий наиболее 
перспективным для решения задач управления является использование 
информационных методов нау\-ко\-мет\-рии и~анализа данных~[3--7]. 
Результативность\linebreak исследований зависит от уровня соответствия 
исследовательских задач и~на\-уч\-но-тех\-ни\-че\-ско\-го потенциала (НТП)
научной 
организации. Струк\-тур\-но-ин\-фо\-ло\-ги\-че\-скую модель  
НТП $j$-й научной\linebreak организации можно 
представить как 
    \begin{equation*}
    E^j(t)=\Phi\left( P^j_{\mathrm{in}}, P^j_{\mathrm{if}}, P^j_m, P^j_{\mathrm{fn}}, S^j, t\right)\,,
   % \label{e1-sun}
    \end{equation*}
где $P^j_{\mathrm{in}}$, $P^j_{\mathrm{if}}$, $P^j_m$ и~$P^j_{\mathrm{fn}}$~--- 
интеллектуальные, 
информационные, материальные и~финансовые ресурсы научной организации 
соответственно; $S^j$~--- уровень организации и~управления ресурсами. Таким 
образом, $E^j(t)$~--- это обобщенная характеристика, определяющая 
совокупность основных видов ресурсов и~факторов в~период времени~$t$ 
и~отражающая способность научной организации решать стоящие перед ней 
задачи. Результативность исследований является функцией максимизации 
соответствия научных задач и~НТП
$[Z^j: E^j(t)]\hm= R_{RQ}\Rightarrow \max$ научной организации. Уровень 
соответствия можно оценить с~по\-мощью  
проб\-лем\-но-ори\-ен\-ти\-ро\-ван\-ных оценок НТП. 
Результаты количественной оценки потенциала могут быть 
поставлены в~зависимость от тех или иных сформулированных (выбранных) 
целей из множества~$\{Z\}$~\cite{4-sun}. 

     Оптимальная стратегия управления научными исследованиями~--- 
серьезная методологическая проблема. Следует отметить, что новые подходы 
к~решению проб\-лем современного на\-уч\-но-тех\-ни\-че\-ско\-го развития 
с~учетом их возрастающей масштабности, сложности, ресурсоемкости 
реализуются на основе: 
\begin{itemize}
\item научного обоснования приоритетного ряда научных 
и~на\-уч\-но-тех\-ни\-че\-ских направлений; 
\item использования 
преимущественно информационных и~экономических методов управления\linebreak 
научными исследованиями и~разработками (в~том чис\-ле минимизации 
факторов неэффективного использования имеющихся и/или доступных 
ресурсов).
\end{itemize}

 Наиболее перспективным и~экономичным решением является 
использование проб\-лем\-но-ори\-ен\-ти\-ро\-ван\-ной СППР 
при управлении финанси\-рованием 
фундаментальных исследований.  \mbox{Интерактивная} компьютерная 
система СППР путем сбора и~анализа большого количества информации 
может влиять на процесс принятия решений (и~моделировать решения) при 
распределении фонда госзаказа по тематическим фундаментальным научным 
направлениям. В~наборе показателей СППР доминируют агрегированные 
и~производные показатели, такие как индексы и~индикаторы. Сис\-те\-мы 
такого класса предназначены для поддержки многокритериальных решений 
в~слож\-ной информационной среде. При этом под многокритериальностью 
понимается тот факт, что результаты принимаемых решений оцениваются не по 
одному, а~по совокупности многих показателей (критериев), рассматриваемых 
одновременно. Информационная сложность определяется не\-об\-хо\-ди\-мостью 
учета большого объема данных, обработка которых без помощи современной 
вычислительной техники практически невыполнима. С~по\-мощью СППР 
может производиться выбор решений неструктурированных 
и~слабоструктурированных задач, в~том чис\-ле и~многокритериальных.  
Сис\-те\-ма поддержки принятия решений позволяет решить две основные задачи: 
\begin{enumerate}[(1)]
\item выбор 
наилучшего решения из множества возможных (оптимизация); 
\item упорядочение возможных решений по предпочтительности (ранжирование).
\end{enumerate}

    С~системных позиций рассмотрим \textbf{концептуальные основы 
разработки (и функционирования) сис\-те\-мы управ\-ле\-ния бюджетным 
финансированием} тематических фундаментальных научных на\-прав\-ле\-ний: 
\begin{description}
\item[A.] Использование неэкспертных квантитативных 
(количественных) наукометрических методов (и~критериев) оценок уровня: 
научного тематического на\-прав\-ле\-ния; научной организации.
    
 \item[B.] Неэкспертное формирование комплексной оценки уровня 
научного направлении на осно-\linebreak ве многомерного анализа данных и~методов 
наукометрии. Распределение бюджетных средств осуществляется 
пропорционально вы\-чис\-ля\-емо\-му рейтингу научных на\-прав\-ле\-ний, который 
синтезируется на основе критериев: со\-во\-куп\-ности наукометрических 
показателей научной организации (по данному направлению), ее  
НТП, уровня приоритетности 
на\-прав\-ле\-ния.
    
     \item[C.] Автоформализация совокупной оценки наукометрических 
показателей ученых $i$-го темати\-че\-ско\-го направления $j$-й научной 
организации\linebreak (соискателя бюджетного финансирования). Автоформализация 
интегральной оценки уровня научной организации с~учетом ее  
НТП и~приоритетности тематических 
на\-прав\-ле\-ний исследований.
    
    \item[D.] Использование экспертов и~экспертных методов на этапе 
формирования критериев и~сис\-те\-мы показателей~--- на начальном этапе 
разработки сис\-те\-мы. При сопоставительном анализе и~моделировании 
принципиальным является сопоставимость параметров, прежде всего 
наукометрических.
    
     \item[E.] Автоматическое (автоматизированное) формирование 
исходных массивов данных тема\-тических научных на\-прав\-ле\-ний и~научных 
\mbox{организаций} (соискателей бюджетного финансирования). Постоянный 
мониторинг показателей и~актуализация банка данных.
    
    \item[F.] Автоматическое формирование (расчет) сопоставительного 
уровня (рейтинга) тематических научных на\-прав\-ле\-ний и~распределение 
объемов бюджетного финансирования.
    
     \item[G.] Поддержание статуса открытости системы 
и~мультипликативного использования фор\-ми\-ру\-емо\-го электронного 
информационного ресурса. Соответствие требованиям масштабируемости, 
многомерного и~многовариантного представления данных, гиб\-кости 
и~адап\-ти\-ру\-емости к~внешним изменениям, сетевая интеграция (прежде всего 
веб).
\end{description}
    
    Распределение финансирования по тематическим научным на\-прав\-ле\-ни\-ям 
из фонда госзаказа должно осуществляться с~учетом следующих показателей:
    \begin{itemize}
    \item наукометрической оценки научного уровня разрабатываемого 
тематического направления с~использованием таких показателей, как чис\-ло 
публикаций, индекс цитируемости, индекс Хирша, им\-пакт-фак\-тор, 
ожидаемый отклик, индекс Прайса, ПРНД (показатель результативности 
научной деятельности, принятый в~РАН) и~др.~\cite{2-sun, 8-sun};
    \item  оценки НТП научной организации~--- соискателя финансирования;
\item  принадлежности тематического на\-прав\-ле\-ния к~Перечню приоритетных 
направлений развития на\-уч\-но-тех\-ни\-че\-ско\-го комплекса России 
на~2014--2020~гг.~\cite{9-sun}. 
\end{itemize}

    Основная прагматическая задача~--- разработка сводной (весовой) 
эмпирической формулы количественной оценки уровня тематического 
направления (с~учетом НТП научной организации и~приоритет\-ности 
направления). В~определенной степени прототипом предлагаемой технологии 
является сис\-те\-ма оценки и~отбора проектов Российского фонда 
фундаментальных исследований (РФФИ), при которой субъект~--- соискатель 
грантового финансирования (ученый, группа ученых или научная организация) 
сам информационно обосновывает актуальность, важ\-ность, научный задел 
и~уровень работ по выбранной тематике. Однако решения принимаются на 
основе экспертных заключений (в~РФФИ $> 3000$~экспертов). Следует 
отметить, что РФФИ ежегодно поддерживает более~70~тыс.\ ученых. 
    
    \textbf{Уточним исходную постановку задачи.} Общее число 
выделенных финансируемых фундаментальных научных на\-прав\-ле\-ний $\leq 
8000$. Количество научных организаций РАН $\leq 500$. Научные 
организации~--- соискатели финансирования по распоряжению (приказу) 
Федерального агентства научных организаций
(ФАНО) РФ или Президиума РАН заблаговременно ($\sim6$~мес.)\ 
предоставляют в~\mbox{ВИНИТИ} РАН в~электронном виде формализованные данные, 
характеризующие:
\begin{enumerate}[(1)]
\item научное на\-прав\-ле\-ние, на развитие которого 
запрашивается бюджетное финансирование, с~авторизованной 
наукометрической оценкой (по РИНЦ), включая данные по бюджетному 
финансированию направления за последние~3--5~лет; 
\item формализованные 
данные по НТП и~эффективности научной 
организации. 
\end{enumerate}

Показатели эффективности: общее число пуб\-ли\-ка\-ций 
сотрудников научной организации (за фиксированный период времени), 
отнесенное к~чис\-лен\-ности научных сотрудников, в~том числе в~зарубежных 
на\-уч\-но-тех\-ни\-че\-ских изданиях (Scopus, Web of Science), в~отечественных изданиях, 
включенных в~перечень ВАК. 
    
    На основе представленных в~электронном виде заявок в~определенном 
формате практически в~автоматическом режиме (в ВИНИТИ) формируются 
БД Сис\-те\-мы. Логические структуры БД и~ан\-ке\-ты-фор\-мы 
разрабатываются заранее и~утверждаются в~ФАНО. Логическая структура БД 
по тематическому на\-прав\-ле\-нию должна также включать в~себя текстовое поле 
с~расширенной аннотацией (объемом $\leq 3500$~знаков). ВИНИТИ может 
дополнять БД экс\-пресс-ана\-ли\-ти\-че\-ски\-ми обзорами по актуальным 
темам, входящим в~Перечень приоритетных направлений или критических 
технологий. Общая кар\-ти\-на-ха\-рак\-те\-ри\-сти\-ка  
\textit{НТП и~эффективности научной 
организации} определяется со\-во\-куп\-ностью таких показателей, как 
    \begin{itemize}
\item число публикаций по всем научным на\-прав\-ле\-ни\-ям, по годам (временной 
лаг $\sim5$~лет), дифференцированно по отечественным и~зарубежным 
журналам;
\item численность и~структура кадрового состава организации (с~учетом 
возрастного сдвига и~по годам);
\item информационно-ана\-ли\-ти\-че\-ские оценки име\-ющих\-ся  
на\-уч\-но-тех\-ни\-че\-ских результатов сопоставительно с~российским 
и~мировым уровнем;
\item количество (и их финансовый объем) по\-лу\-ча\-емых грантов (по 
на\-прав\-ле\-ни\-ям) от российских фондов, зарубежных фондов и~программ, 
спонсоров;
\item состав (и~объем) выполненных внебюджетных конкурсных (и~заказных) 
исследований, проектов, разработок (временной лаг $\sim4$~года) и~др. 
    \end{itemize}
    
    Функциональная структура информационной сис\-те\-мы распределения 
бюджетного фонда госзаказа по тематическим научным на\-прав\-ле\-ни\-ям 
включает в~себя: 
    \begin{itemize}
\item  банк данных СППР, включающий в~себя:
\begin{itemize}
\item[(а)] БД с~формализованными 
характеристиками тематических направлений (включая приоритеты); 
\item[(б)] БД 
с~унифицированными характеристиками, отражающими научный 
(и~технический) потенциал организаций РАН; 
\item[(в)] БД по исследованиям 
и~инновационным разработкам, пред\-став\-ля\-ющим интерес для дальнейшей 
коммерциализации;
\end{itemize}
    \item  подсистемы: 
    \begin{itemize}
    \item[(а)] сбора и~регистрации данных по тематическим 
на\-прав\-ле\-ни\-ям и~научным организациям РАН; 
\item[(б)] моделирования, расчета 
показателей и~распределения фонда госзаказа по тематическим на\-прав\-ле\-ни\-ям; 
\item[(в)] взаимодействия с~реферативным банком данных \mbox{ВИНИТИ}; 
\item[(г)] мониторинга, 
неэкспертного информационного анализа данных по промежуточным 
результатам, контроля расходования средств по целевому финансированию;
\end{itemize}
\item  блоки: 
\begin{itemize}
\item[(а)] сопоставительного анализа тематических на\-прав\-ле\-ний на 
предмет выявления дублируемых исследований;
\item[(б)] информационного, 
математического и~нор\-ма\-тив\-но-ме\-то\-ди\-че\-ско\-го (регламентного) 
обеспечения; 
\item[(в)] управ\-ле\-ния, интерактивного взаимодействия, визуализации 
и~генерации отчетов. 
\end{itemize}
\end{itemize}
    
    Финальным результатом обработки данных научных организаций является 
автоматическое (автоматизированное) распределение фонда госзаказа по 
выделенным тематическим на\-прав\-ле\-ни\-ям при выполнении общего условия 
сохранения баланса затрат.
    
    Рассмотрим \textbf{алгоритм базовой модели распределения} 
бюджетного финансирования (получаемого в~рамках госзаказа) по выделенным 
фундаментальным научным на\-прав\-ле\-ни\-ям с~прос\-той  
ло\-ги\-ко-вы\-чис\-ли\-тель\-ной структурой. 
    
    Общее число финансируемых научных на\-прав\-ле\-ний обозначим как~$M$ 
($\sim8000$). С~учетом множества $\{N\}$, где $N$~--- общее число научных 
организаций, получающих финансирование из фонда госзаказа (для РАН 
$\sim400$), приведенный оценочный уровень каждого из тематических 
на\-прав\-ле\-ний определяется эмпирическим выражением:
    \begin{equation*}
    m^\prime_{ij}=m_{ij} \left[ \sum\limits_{i=1}^M \sum\limits_{j=1}^N m_{ij} \left( 
100\%\right)^{-1}\right]^{-1}\,,
    %\label{e2-sun}
    \end{equation*}
где $m_{ij}$~--- наукометрическая оценка уровня $i$-го тематического 
направления (по пуб\-ли\-ка\-ци\-ям) в~$j$-й научной организации, где $m_{ij}\hm> 
0$ (с~некоторой долей упрощения $m_{ij}$ может интерпретироваться как 
уровень научного задела по $i$-му тематическому на\-прав\-ле\-нию); 
$m^\prime_{ij}$~--- <<удельный вес>> тематического на\-прав\-ле\-ния (или 
относительный уровень $i$-го направления в~$j$-й научной организации) 
в~общей совокупности выделенных для финансирования научных на\-прав\-ле\-ний 
РАН (в~\%). 
    
    Априори полагаем, что успешность выполнения исследований по каждому 
тематическому на\-прав\-ле\-нию в~общем случае зависит от уровня  
НТП организации, где эти исследования 
проводятся. Эмпирическая формула уровня 
НТП $j$-й организации~$W_j$ определяется как
    \begin{equation}
    W_j=k_j\left(a_j B_j\right)\,.
    \label{e3-sun}
    \end{equation}
Здесь $k_j$~--- коэффициент, учи\-ты\-ва\-ющий технический потенциал $j$-й 
организации, в~первую очередь\linebreak со\-сто\-яние инфраструктуры (возможность 
ши\-ро\-кополосного телекоммуникационного доступа\linebreak
 к~отечественным 
и~зарубежным БД с~на\-уч\-но-тех\-ни\-че\-ской информацией, парк 
измерительной аппаратуры, компьютерное оснащение, наличие супер\-ЭВМ, 
уникального оборудования коллективного доступа, состояние 
специализированных служб и~сервисов и~т.\,п.), условно шкала $k\hm= 0\mbox{--}V$ 
($V\leq 100$);
$B_j$~--- эффективность научной деятельности организации~--- уровень 
пуб\-ли\-ка\-ци\-он\-ной активности научной организации (см.\ РИНЦ), опре\-де\-ля\-емый 
как

\noindent
\begin{equation}
B_j=L_j(t)\left( G_j\right)^{-1}\,,
\label{e4-sun}
\end{equation}
где $L_j$~--- общее чис\-ло пуб\-ли\-ка\-ций за период~$t$, $G_j$~--- чис\-ло 
исследователей; $a_j$~--- нивелирующий коэффициент пуб\-ли\-ка\-ци\-он\-ной 
активности для научных организаций с~превалирующей долей 
экспериментальных исследований:
\begin{multline*}
\hspace*{-6pt}a_j=\begin{cases}
1\,, &\!\mbox{ординарное~направление};\\
1+\delta\,, &\!\mbox{экспериментальное\ направление},\\
& 1<\delta\leq 10\,.
\end{cases}\hspace*{-2.5pt}
\end{multline*}


Тогда $m_{ij}^0$, с~учетом~(\ref{e3-sun}) и~(\ref{e4-sun}),~--- приведенный 
оценочный уровень $i$-го тематического на\-прав\-ле\-ния с~учетом НТП $j$-й 
научной организации: 
\begin{equation*}
m_{ij}^0=W_j m^1_{ij}\,.
%\label{e5-sun}
\end{equation*}
    
    Далее, пусть $Z_0$~--- общий бюджетный фонд госзаказа (для РАН $\leq 
100$~млрд руб.). В~общем случае для повышения операционной гиб\-кости 
системы может пред\-усмат\-ри\-вать\-ся функциональный бюджетный резерв 
(руководства РАН)~--- $\Delta Z_0$ ($\Delta Z_0\hm\leq 0{,}5\%\mbox{--}3\%$); тогда 
далее оперируем величиной $Z_0^\prime\hm= Z_0\hm- \Delta Z_0$. 
    
    Из условия сохранения баланса 
    \begin{equation*}
    Z^\prime_0= \sum\limits_{i=1}^M \sum\limits_{j=1}^N Z_{ij}= \sum\limits_{i=1}^M 
\sum\limits_{j=1}^N \left( x  m^0_{ij}\right)
    %\label{e6-sun}
    \end{equation*}
вычисляем базисную величину~$x$, на основе которой определяются вложения 
средств по тематическим направлениям в~соответствии с~принципом 
пропорциональности оценочного (рейтингового) уровня научного на\-прав\-ле\-ния: 
$$
x= Z^\prime_0 \left(S^{-1}\right).
$$
Здесь $S$~--- суммарный уровень всех 
тематических на\-прав\-ле\-ний:
$$
S= \sum\limits_{i=1}^M 
\sum\limits_{j=1}^N \left(d_{ij} m^0_{ij}\right)\,,
$$  
где $d_i$~--- коэффициент, учи\-ты\-ва\-ющий 
вхождение $i$-го на\-прав\-ле\-ния $j$-й организации в~Перечень приоритетных 
на\-прав\-ле\-ний:
$$
d_i=\begin{cases}
1\,,&  \mbox{неприоритетное~на\-прав\-ле\-ние};\\
1+\Delta\,, & \mbox{приоритетное на\-уч\-но-тех\-ни\-че\-ское}\\
&\mbox{на\-прав\-ле\-ние},\enskip  
\Delta=const.
\end{cases}
$$
 
 В~общем случае может иметь место ситуация, когда $\Delta 
\hm = var$, однако это, при больших размерностях~$M$ и~$N$, повлечет 
существенное услож\-не\-ние рас\-че\-тов по распределению бюджетных средств. 
Таким образом, итоговая расчетная формула базисной величины~$x$ имеет вид: 
\begin{equation}
x=Z_0^\prime \left(\sum\limits_{i=1}^M \sum\limits_{j=1}^N(d_{ij} 
m_{ij}^0)\right)^{-1} =Z_0^\prime\cdot 100^{-1}\,.
\label{e7-sun}
\end{equation}
    
    С учетом формулы~(\ref{e7-sun}) объем финансовых вложений из фонда 
госзаказа в~каждое $i$-е тематическое на\-прав\-ле\-ние определяется итоговым 
выражением:

\vspace*{3pt}

\noindent
    \begin{equation}
    Z_{ij}= x \left( d_{ij} m^0_{ij}\right)\,.
    \label{e8-sun}
    \end{equation}
    
    Используя равенство~(\ref{e8-sun}), выполняется последний шаг~--- 
заключительная проверка сохранения баланса суммарных затрат:

\vspace*{3pt}

\noindent
    \begin{equation*}
    \sum\limits_{i=1}^M \sum\limits_{j=1}^N Z_{ij}\leq Z_0^\prime\,.
   % \label{e9-sun*}
    \end{equation*}
    
    \vspace*{-2pt}
    
    \noindent
     Здесь следует отметить два детерминирующих фактора. 
\begin{enumerate}[1.]
\item Задача объективной оценки 
НТП еще весьма далека от своего завершения, к~оценке НТП до сих пор 
нет единого подхода~\cite{10-sun}. Однако в~последнее десятилетие при оценке 
НТП научных организаций помимо прочих все шире используется 
наукометрический подход (на основе учета числа пуб\-ли\-ка\-ций и~индекса
цитирования~$B_j$). Принципиально важно исходить из того, что оценка всегда есть 
функция цели, а~следовательно, оценка НТП в~разных случаях может 
осуществляться на основе разных наборов показателей. При наличии 
количественных значений каж\-до\-го из показателей, характеризующих НТП, 
возможен подход, при котором производится <<свертка>> част\-ных показателей 
в~интегральный (агрегированный) показатель,\linebreak
 принимаемый за численную 
оценку потенциала (индикатор потенциала). Выбор оптимальной операции 
<<свертки>>, позволяющий получать интегральный показатель, достаточно\linebreak 
объективно ха\-рак\-те\-ри\-зу\-ющий оценку НТП, является непростой теоретической 
и~практической задачей, включающей в~себя как выбор вида функции 
агрегирования, так и~<<взвешивание>> отдельных входящих в~нее 
показателей.\\[-14pt] 
    
    \item Выбор приоритетов лежит в~слабо формализуемой сфере 
целеобразования~\cite{8-sun, 11-sun, 12-sun}. Не\linebreak существует универсальной 
модели процесса формирования приоритетов. Приоритетные на\-прав\-ле\-ния 
развития науки и~техники России\linebreak
 в~последнее десятилетие связаны 
с~\textit{фундаментальными исследованиями}, информационными\linebreak 
технологиями и~электроникой, производственными технологиями, новыми 
материалами и~химическими продуктами, технологиями\linebreak живых сис\-тем, 
транспортом, топливом и~энергетикой, экологией и~рациональным 
природопользованием. Данные приоритетные на\-прав\-ле\-ния конкретизируются 
в~подпрограммах\linebreak и~подкреплены разработкой критических\linebreak технологий. 
Институциональное обеспечение\linebreak
 разработки на\-уч\-но-тех\-ни\-че\-ской 
политики и~определения соответствующих национальных приоритетов 
осуществляется различными структурами на разных уровнях управ\-ле\-ния 
экономикой. Поскольку ресурсы всегда ограничены, а~круг разнородных задач 
в~сфере на\-уч\-но-тех\-но\-ло\-ги\-че\-ско\-го развития, тре\-бу\-ющих решения, 
чрезвычайно широк, проб\-ле\-ма выбора национальных приоритетов  
на\-уч\-но-тех\-но\-ло\-ги\-че\-ско\-го развития приобретает первостепенную 
зна\-чи\-мость. Бюджетные процедуры очень важны для реализации  
на\-уч\-но-тех\-ни\-че\-ских приоритетов (для любой страны) и~являются 
одной из функциональных задач управ\-ле\-ния наукой.
\end{enumerate}
    
    \textbf{Перечень задач, на решение которых ориентирована СППР, 
включает:}
    \begin{itemize} 
\item  автоматизированный расчет и~распределение бюджетных средств по 
выделенным научным на\-прав\-ле\-ни\-ем (основная задача);
\item выявление с~помощью методов наукометрического анализа стагнирующих 
научных на\-прав\-ле\-ний (анализ на временн$\acute{\mbox{о}}$м интервале $\sim5$~лет);
\item  выявление <<точек роста>>, анализ и~оценка трендов развивающихся 
научных на\-прав\-ле\-ний;
    \item информационно-ана\-ли\-ти\-че\-ская поддержка разработки 
прогнозов на\-уч\-но-тех\-но\-ло\-ги\-че\-ско\-го развития (в~РФ и~за рубежом);
    \item  информационно-ана\-ли\-ти\-че\-ская поддержка формирования 
национального перечня приоритетных на\-прав\-ле\-ний  
на\-уч\-но-тех\-но\-ло\-ги\-че\-ско\-го развития и~критических технологий;
    \item информационная поддержка подготовки предложений по 
формированию государственных (и~региональных) на\-уч\-но-тех\-ни\-че\-ских 
программ;
    \item текущий мониторинг выполнения работ и~расходования бюджетных 
средств по научным на\-прав\-ле\-ниям. 
    \end{itemize}
    
    Пользователи Системы: Президиум РАН, 
\mbox{ФАНО}, Министерство образования и~науки РФ и~другие федеральные 
ведомства, научные фонды, экспертное сообщество РАН, научные организации 
РФ и~СНГ, информационные центры, промышленные корпорации. 
    
    \section{ Потенциальные приложения интегрированной 
постобработки информации базы данных СППР и~банка~данных ВИНИТИ} 
    
    Использование наукометрических методов и~методов анализа данных при 
совместной пост\-об\-ра\-бот\-ке актуальной информации по вы\-пол\-ня\-ющим\-ся в~РАН 
исследованиям и~сис\-те\-ма\-ти\-зи\-ро\-ван\-ных информационных ресурсов ВИНИТИ 
является весьма перспективным для решения целого ряда задач~\cite{13-sun, 
14-sun}, таких как
    \begin{itemize}
    \item анализ структуры (и уровня) отечественной и~мировой науки;
    \item определение тенденций и~процессов, происходящих в~мировой 
и~региональной науке;
    \item выявление (на ранней стадии) наиболее актуальных или, 
напротив, теряющих свою актуальность научных на\-прав\-ле\-ний;
    \item отслеживание генезиса конкретных научных идей (или 
на\-прав\-ле\-ний) и~истории их развития;
    \item определение продуктивности научных организаций и~работы 
отдельных исследователей (научных групп) в~конкретной научной об\-ласти 
и~эффективности материальных и~иных затрат в~этой об\-ласти;
    \item анализ трендов развития инновационной деятельности в~рамках 
отдельных научных организаций, на\-прав\-ле\-ний (или отделений РАН);
    \item анализ структуры научного сообщества и~изуче\-ние науки как 
социального организма.
    \end{itemize}
    
    Постобработка больших массивов на\-уч\-но-тех\-ни\-че\-ской  
(и~тех\-ни\-ко-эко\-но\-ми\-че\-ской) информации с~использованием методов 
наукометрии и~анализа данных (в~том чис\-ле статистических методов) 
\mbox{априори} позволяет выявлять закономерности, выражающие зависимости между 
распределениями различных па\-ра\-мет\-ров исследуемых сис\-тем и~процессов, 
и~характер изменения распределений во времени. Совместная постобработка 
информации БД Сис\-те\-мы, ВИНИТИ РАН и~данных Росстата\linebreak
 (РФФИ, 
БД eLIBRARY и~др.), таких как величина валового внут\-рен\-не\-го продукта
(ВВП), про\-из\-веденной энергии, 
среднего годового дохода\linebreak
 на душу населения, величины произведенного 
продукта, приходящейся на высокие технологии,\linebreak и~ряда других,~--- это 
перспективное множество пред\-став\-ля\-ющих практический интерес (для 
управ\-ле\-ния и~прогнозирования) показателей и~распределений. 
    
    Рассмотрим несколько гипотетических при\-ме\-ров-ва\-ри\-ан\-тов аналитической 
постобработки информации. Определенный интерес могут пред\-став\-лять:
    \begin{itemize}
    \item данные и~визуализация распределений и~оценка корреляции 
изменений бюджета фундаментальных исследований (совокупный госзаказ) 
и~роста числа пуб\-ли\-ка\-ций по годам (с использованием данных Росстата); 
    \item  графики распределений и~оценки зависимости числа публикаций от 
роста ВВП (структурно) по на\-прав\-ле\-ни\-ям: энергетика, транспорт, цветная 
металлургия и~т.\,д.,~--- по годам (желательно для сопоставления использовать 
данные бюджета и~ВВП предыдущего года);
    \item  анализ распределений и~корреляции роста индекса производства 
(по~10~основным отраслям про\-мыш\-лен\-ности) в~процентах к~предыдущему 
периоду и~роста расходов на исследования и~разработки (или, например, числа 
пуб\-ли\-ка\-ций российских авторов) за тот же период;
    \item анализ сравнительного роста: 
    \begin{itemize}
    \item[(а)] ВВП; 
    \item[(б)] расходов на образование;
        \item[(в)] расходов на исследования и~разработки; 
    \item[(г)] объема пуб\-ли\-ка\-ций российских авторов;
    \end{itemize}
    \item  анализ изменений объемов (и~структуры) ВВП и~распределения 
затрат по на\-уч\-но-тех\-ни\-че\-ским на\-прав\-ле\-ниям;
    \item  анализ зависимости роста инвестиций в~от\-расли экономики и~рост 
объемов публикаций (то же по отраслям народного хозяйства).
    \end{itemize}
    
    В~связи с~быстрым ростом цифрового пространства и~сетевой среды 
целенаправленное ис\-поль\-зование методов и~средств аналитической 
пост\-обработки информационных ресурсов могло бы трансформи\-ро\-вать\-ся 
в~новое направление информатики~--- сетевой анализ и~сетевую наукометрию. 
    
    \section{Выводы}
    
    \noindent
    \begin{enumerate}[1.]
    \item  Синтезирован концептуальный облик СППР 
    при управ\-ле\-нии бюджетным финансированием\linebreak тематических 
направлений с~использованием критериев и~методов наукометрии и~анализа 
данных, а~также с~учетом приоритетности направлений  
и~НТП научных организаций. Основные 
достоинства проекта: 
\begin{itemize}
\item[(а)] простота решения и~относительно невысокая 
трудоемкость (и ресурсоемкость) реализации Системы; 
\item[(б)] отсутствие  
ор\-га\-ни\-за\-ци\-он\-но-фи\-нан\-со\-вых проблем по корпусу экспертов 
и~организации экспертизы; 
\item[(в)] использование современных  
ин\-фор\-ма\-ци\-он\-но-ана\-ли\-ти\-че\-ских и~наукометрических методов для 
моделирования и~получения оценок (унифицированных и~сопоставимых) 
и~финишных результатов по распределению бюджетных средств госзаказа 
между $\sim8000$~научных направлений.
\end{itemize}
    \item  Важным и~перспективным является формирование нового 
многоцелевого информационного ресурса РАН для целей управления, анализа 
развития отечественной и~мировой науки, на\-уч\-но-тех\-ни\-че\-ско\-го 
прогнозирования, развития экспертной деятельности, оптимизации процессов 
финансирования исследований и~разработок, мониторинга текущего состояния.
    \item  Системы управления фундаментальными исследованиями могут 
быть разработаны на основе иных подходов, отличных от предложенного. 
Однако следует отметить, что по базовым положениям теории (и~практики) 
управления не может быть эффективной и~устойчивой система, если центр 
целеполагания и~компетенции находится в~одном месте, а центр управления 
и~распределения финансовых ресурсов~--- в~другом. Разделение управления 
ресурсами (\mbox{ФАНО} РФ) и~исследованиями (РАН) существенно снижает 
эффективность науки, так как они имеют разные целевые критерии своей 
деятельности.
    \end{enumerate}
    
    \section{Заключение}
    
    Несколько выходя за рамки рассматриваемой проблематики, хотелось бы 
сделать два замечания более общего характера.
    \begin{enumerate}[1.]
    \item  Основная задача науки состоит в~научном обеспечении  
со\-ци\-аль\-но-эко\-но\-ми\-че\-ско\-го развития страны, поэтому наличие 
в~структуре Системы базы совокупных данных научных организаций РАН по 
инновационным разработкам позволит в~перспективе более объективно 
подойти к~оценкам эффективности фундаментальной науки. В~мировой 
практике оценка результатов фундаментальной науки, как правило, проводится 
на основе показателей публикационной\linebreak активности и~цитирования, а~также 
экспертных оценок научного сообщества. В~настоящее\linebreak время российским 
научным и~экономическим сообществом не выработаны единые подходы\linebreak 
к~оценке эффективности фундаментальной\linebreak науки. Очевидно, что уровень 
развития отече\-ственной промышленности, особенно ее %\linebreak
 высокотехнологичного 
сектора, т.\,е.\ ее <<платеже\-способный спрос>> на результаты исследований\linebreak 
и~разработок, будет существенно влиять на оценку эффективности 
фундаментальной науки. (По статистике РФФИ каждый десятый завершенный 
проект имеет прикладной инновационный потенциал и~коммерческую, 
рыночную перспективу.) 
    \item  Задача управления финансированием фундаментальных 
исследований является частью более общей задачи управления наукой как 
особым сегментом национальной экономики. Актуаль\-ная задача 
академического сообщества, и~в~первую очередь руководства РАН,~---
разработка пакета рабочих предложений для формирования научно 
обоснованной стратегии со\-ци\-аль\-но-эко\-но\-ми\-че\-ско\-го развития 
России. Это далеко не тривиальная задача, так как в~российской экономике есть 
два существенных, если не сказать важнейших, фак\-то\-ра-де\-тер\-ми\-нан\-та. 
Они взаимосвязаны и~взаимозависимы. 
    
    \textit{Во-первых}, это явная струк\-тур\-но-функ\-цио\-наль\-ная 
недостаточность существующего <<промежуточного слоя>> между 
фундаментальной наукой и~промышленностью, необходимого для создания 
инновационных продуктов и~трансфера технологий. До постсоветского периода 
<<промежуточный слой>> состоял из отраслевых прикладных на\-уч\-но-ис\-сле\-до\-ва\-тель\-ских 
и~проектных организаций. В~постсоветский период этот <<промежуточный 
слой>> практически деградировал, по отдельным направлениям необратимо 
деформирован и~фактически утратил имевшийся 
НТП. Сейчас в~разных отраслях экономики с~разным уровнем 
эффективности функции <<промежуточного слоя>> выполняют технопарки, 
внедренческие центры, венчурные фонды, инжиниринговые центры,  
биз\-нес-ин\-ку\-ба\-то\-ры, кластеры и~отдельные сохранившиеся 
и~приспособившиеся к~новым условиям на\-уч\-но-ис\-сле\-до\-ва\-тель\-ские институты
и~конструкторские бюро (в~основном 
    в~на\-уч\-но-про\-из\-вод\-ст\-вен\-ных объединениях).
    
    \textit{Во-вторых}, несоответствие возможностей существующей 
национальной информационной инфраструктуры современным требованиям 
новой российской экономической институциональной среды. Это важно, так 
как во многом основой успешного инновационного развития\linebreak
 отраслей 
промышленности является использование информационных технологий. Новая\linebreak 
парадиг\-ма экономического развития предполагает в~качестве важнейшего 
фактора конкурентоспособности максимально широкое <<вплетение>> 
цифровых информационных технологий в~ткань любых производственных, 
технологических и~управленческих процессов.
    
    Таким образом, вторая важная (и безальтернативная) задача~--- разработка 
эффективных механизмов (экономических, правовых) вовлечения, в~той или 
иной форме, научных организаций РАН в~модернизацию существующих 
отраслей российской экономики. 
    \end{enumerate}
    
   {\small\frenchspacing
 {%\baselineskip=10.8pt
 \addcontentsline{toc}{section}{References}
 \begin{thebibliography}{99}
\bibitem{1-sun}
\Au{Авдулов А.\,Н. Кулькин~А.\,М.} Финансирование науки в~развитых странах 
мира.~--- М.: ИНИОН РАН, 2007. 114~с.
\bibitem{2-sun}
\Au{Калачихин П.\,А.} Принципы построения государственной 
наукометрической системы~// На\-уч\-но-тех\-ни\-че\-ская информация. Сер.~2, 
2016. №\,7. С.~11--23.

\bibitem{7-sun} %3
\Au{Сюнтюренко О.\,В., Черепанов~Е.\,В.} Информатика: анализ данных 
и~эконометрия~// Средства связи, 1986. №\,4. С.~39--44. 
\bibitem{3-sun} %4
\Au{Когаловский М.\,Р., Паринов~С.\,И.} Новый источник данных для 
наукометрических исследований~// Тр. XV Всеросс.  
научной конф. <<Электронные библиотеки: перспективные 
методы и~технологии, электронные коллекции>>.~--- Ярославль: 
ЯрГУ им.\ П.\,Г.~Демидова, 2013. 
С.~107--117.


\bibitem{6-sun} %5
\Au{Антопольский А.\,Б.} О~целесообразности российского национального 
вебометрического индекса~// На\-уч\-но-тех\-ни\-че\-ская информация. Сер.~1, 
2014. №\,2. С.~14--18.

\bibitem{5-sun} %6
\Au{Месропян В.\,Р., Овсянников~М.\,В.} Перспективы использования 
наукометрических методов в~прогнозировании~// На\-уч\-но-тех\-ни\-че\-ская 
информация. Сер.~1, 2014. №\,2. С.~19--27. 

\bibitem{4-sun} %7
\Au{Сюнтюренко О.\,В., Гиляревский~Р.\,С.} Использование методов 
наукометрии и~сопоставительного анализа данных для управления научными 
исследованиями по тематическим направлениям~// На\-уч\-но-тех\-ни\-че\-ская 
информация. Сер.~2, 2016. №\,12. С.~1--12. 

\bibitem{8-sun}
\Au{Борисова Л.\,Ф., Сюнтюренко~О.\,В.} Реферативный банк данных 
ВИНИТИ РАН: перспективы постобработки информации с~использованием 
методов анализа данных~// На\-уч\-но-тех\-ни\-че\-ская информация. Сер.~1, 
2007. №\,11. С.~6--11.
\bibitem{9-sun}
О~внесении изменений в~Федеральную целевую программу <<Исследования 
и~разработки по приори\-тетным направлениям развития  
на\-уч\-но-тех\-но\-ло\-гическо\-го комплекса России на~2014--2020~гг.: 
Постанов\-ле\-ние Правительства РФ от 25~сентября 2017~г. №\,1156.
\bibitem{10-sun}
\Au{Ладный А.\,О.} Анализ данных в~задачах управ\-ле\-ния  
на\-уч\-но-тех\-ни\-че\-ским потенциалом. 2012. {\sf  
http://\linebreak it-claim.ru/Library/Books/ITS/wwwbook/ist6/ladni/ ladni.htm}. 
\bibitem{11-sun}
\Au{Петровский А.\,Б., Бойченко~В.\,С., Стернин~М.\,Ю., Шепелев~Г.\,И.} 
Выбор приоритетов на\-уч\-но-тех\-ни\-че\-ско\-го развития: опыт зарубежных 
стран~// Тр. Института системного анализа РАН, 2015. Т.~65. №\,3. С.~13--26.
\bibitem{12-sun}
Обоснование выбора приоритетов на\-уч\-но-тех\-но\-ло\-ги\-че\-ско\-го 
развития. Федеральный портал PROTOWN.RU,  2015. {\sf 
http://www.protown.ru/\linebreak information/hide/4500.html}. 
\bibitem{13-sun}
\Au{Биктимиров М.\,Р., Гиляревский~Р.\,С., Сюнтюренко~О.\,В.} Новая 
концептуальная основа развития информационной деятельности ВИНИТИ 
РАН~// На\-уч\-но-тех\-ни\-че\-ская информация. Сер.~1, 2016. №\,1. С.~1--8. 
\bibitem{14-sun}
\Au{Сюнтюренко О.\,В.} Производство  
ин\-фор\-ма\-ци\-он\-но-ана\-ли\-ти\-че\-ских продуктов и~услуг 
с~использованием методов наукометрии и~анализа данных~// Информация 
в~современном мире: Мат-лы Междунар. конф. к~65-ле\-тию ВИНИТИ  
РАН.~--- М.: ВИНИТИ, 2017. С.~317--321.
  \end{thebibliography}

 }
 }

\end{multicols}

\vspace*{-6pt}

\hfill{\small\textit{Поступила в~редакцию 26.12.17}}

\vspace*{8pt}

%\newpage

%\vspace*{-24pt}

\hrule

\vspace*{2pt}

\hrule

%\vspace*{8pt}


\def\tit{FINANCING OF BASIC RESEARCH: CONCEPTUAL SHAPE OF~A~SYSTEM OF~SUPPORT 
OF~DECISION-MAKING WITH USE OF~METHODS OF~SCIENTOMETRICS AND~ANALYSIS OF~DATA}

\def\titkol{Financing of basic research: Conceptual shape of~a~system of~support 
of~decision-making with use of~methods of~scientometrics} % and~analysis of~data}

\def\aut{O.\,V.~Syuntyurenko}

\def\autkol{O.\,V.~Syuntyurenko}

\titel{\tit}{\aut}{\autkol}{\titkol}

\vspace*{-9pt}


\noindent
All Russian Institute of Scientific and Technical Information of the 
Russian Academy of Sciences, 20a~Usievich Str., Moscow 125190, Russian Federation 



\def\leftfootline{\small{\textbf{\thepage}
\hfill INFORMATIKA I EE PRIMENENIYA~--- INFORMATICS AND
APPLICATIONS\ \ \ 2018\ \ \ volume~12\ \ \ issue\ 1}
}%
 \def\rightfootline{\small{INFORMATIKA I EE PRIMENENIYA~---
INFORMATICS AND APPLICATIONS\ \ \ 2018\ \ \ volume~12\ \ \ issue\ 1
\hfill \textbf{\thepage}}}

\vspace*{3pt}   


     

\Abste{The article is devoted to development of a conceptual model of an 
information system focused on solution of the task of management of financing of 
basic research with use of methods of scientometrics and analysis of data. The basic 
principles and the methodology of creation are formulated. Functioning of the system 
of support of decision-making (SSDM) in the process of distribution of fund of the 
state order in thematic scientific directions is sugessted. The functional structure of 
SPPR is presented. The prospects of use of a~new multipurpose information 
resource and analytical postinformation processing are considered. The interrelation 
of tasks of management of science with problems of innovative development of 
branches of economy is shown.}

    \KWE{scientometrics; analysis of data; control systems; scientific 
priorities; monitoring; scientific and technical potential; postdata processing}
    
    
     
  \DOI{10.14357/19922264180115} 

%\vspace*{-12pt}

\Ack
    \noindent
     The work was supported by the Russian Foundation for Basic Research  
(projects 17-07-00256).



\vspace*{6pt}

  \begin{multicols}{2}

\renewcommand{\bibname}{\protect\rmfamily References}
%\renewcommand{\bibname}{\large\protect\rm References}

{\small\frenchspacing
 {%\baselineskip=10.8pt
 \addcontentsline{toc}{section}{References}
 \begin{thebibliography}{99} 
    \bibitem{1-sun-1}
    \Aue{Avdulov, A.\,N., and A.\,M.~Kul'kin}. 2007. \textit{Finansirovanie nauki 
v~razvitykh stranakh mira} [Science funding in the developed countries of the 
world]. Moscow: INION RAN. 114~p.
    \bibitem{2-sun-1}
    \Aue{Kalachikhin, P.\,A.} 2016. Printsipy postroeniya gosudarstvennoy 
naukometricheskoy sistemy [Principles of creation of the state scientometric system]. 
\textit{Nauchno-tekhnicheskaya informatsiya} [Scientific and Technical 
Information]. Ser.~2. 7:11--23.

 \bibitem{7-sun-1} %3
    \Aue{Syuntyurenko, O.\,V., and E.\,V.~Cherepanov}. 1986. Informatika: analiz 
dannykh i~ekonometriya  [Informatics: Data  analysis  and econometrics]. 
\textit{Sredstva  svyazi}  [Means of Communications] 4:39--44.
    \bibitem{3-sun-1} %4
    \Aue{Kogalovskiy, M.\,R., and S.\,I.~Parinov}. 2013. Novyy istochnik dannykh 
dlya naukometricheskikh issledovaniy [New data source for scientometric 
researches]. \textit{Tr. 15-y Vseross. konf. ``Elektronnye biblioteki: Perspektivnye 
metody i~tekhnologii, elektronnye kollektsii''}  
[All-Russian Conference ``Electronic Libraries and Collections'']. Yaroslavl': YarGU 
im.\ P.\,G.~Demidova.  
107--117.

 \bibitem{6-sun-1} %5
    \Aue{Antopol'skiy, A.\,B.} 2014. O tselesoobraznosti rossiyskogo 
natsional'nogo vebometricheskogo indeksa [About expediency of the Russian 
national vebometrical index]. \textit{Nauchno-tekhnicheskaya informatsiya} 
[Scientific and Technical Information]. Ser.~1. 2:14--18.

\bibitem{5-sun-1} %6
    \Aue{Mesropyan, V.\,R., and M.\,V.~Ovsyannikov}. 2014. Perspektivy 
ispol'zovaniya naukometricheskikh  metodov v prognozirovanii [Perspectives of use 
of scientometric methods in forecasting]. \textit{Nauchno-tekhnicheskaya 
informatsiya} [Scientific and Technical Information]. Ser.~1. 2:19--27. 
    \bibitem{4-sun-1} %7
    \Aue{Syuntyurenko, O.\,V., and R.\,S.~Gilyarevskiy}. 2016. Ispol'zovanie 
metodov naukometrii i~sopostavitel'nogo analiza dannykh dlya upravleniya 
nauchnymi issledovaniyami po tematicheskim napravleniyam [Use of scientometric 
methods and comparative data analysis for management of scientific research in the 
thematic directions] \textit{Nauchno-tekhnicheskaya informatsiya} 
[Scientific and Technical Information]. Ser.~2. 12:1--12. 
    
   
   
    \bibitem{8-sun-1} %8
    \Aue{Borisova, L.\,F., and O.\,V.~Syuntyurenko}. 2007. Referativnyy bank 
dannykh VINITI RAN: perspektivy postobrabotki informatsii s~ispol'zovaniem 
metodov analiza dannykh  [VINITI Abstract's databank: Perspectives of  
postinformation processing and use of data analysis].  
\textit{Nauchno-tekhnicheskaya informatsiya} [Scientific and Technical 
Information]. Ser.~1. 11:6--11.
    \bibitem{9-sun-1}
    Postanovlenie Pravitel'stva RF ot 25~sentyabrya 2017~g. No.\,1156 
``O~vnesenii izmeneniy v~Federal'nuyu tselevuyu programmu ``Issledovaniya 
i~razrabotki po prioritetnym napravleniyam razvitiya nauchno-tekhnologicheskogo 
kompleksa Rossii na 2014-2020~gg.'' [About modification of the Federal target 
program ``Research and development in the priority directions of development of 
a~scientific and technological complex of Russia for 2014--2020:''  Resolution of the 
Government of the Russian Federation  No.\,1156 of September~25, 2017].
    \bibitem{10-sun-1}
    \Aue{Ladnyy, A.\,O.} 2012. Analiz dannykh v~zadachakh upravleniya 
nauchno-tekhnicheskim potentsialom [Data analysis  in tasks of management of 
scientific and technical potential]. Available at: {\sf  
http://it-claim.ru/Library/ Books/ITS/wwwbook/ist6/ladni/ladni.htm} (accessed 
February~5, 2018).
    \bibitem{11-sun-1}
    \Aue{Boychenko,V.\,S., A.\,B.~Petrovsky, M.\,Yu.~Sternin, and 
G.\,I.~Shepelev}. 2015. Vybor prioritetov nauchno-tekhnicheskogo razvitiya: opyt 
zarubezhnykh stran  [Setting the priorities of scientific and technological development: 
The experience of foreign countries]. \textit{Proceedings of the Institute for
System Analysis of RAS} 
65:13--26.
    \bibitem{12-sun-1}
    Obosnovanie vybora prioritetov nauchno-tekh\-no\-lo\-gi\-che\-sko\-go razvitiya  
[Choice justification of priorities of scientific and technical development]. 2015. 
Available at: {\sf http://www.protown.ru/information/hide/ 4500.html} (accessed 
February~5, 2018).
    \bibitem{13-sun-1}
    \Aue{Biktimirov, M.\,R., R.\,S.~Gilyarevskiy, and O.\,V.~Syuntyurenko}. 
2016. Novaya kontseptual'naya osnova razvitiya informatsionnoy deyatel'nosti 
VINITI RAN [New conceptual basis of the development of the activity of \mbox{VINITI}]. 
\textit{Nauchno-tekhnicheskaya informatsiya} [Scientific and Technical 
Information]. Ser.~1. 1:1--8. 
    \bibitem{14-sun-1}
    \Aue{Syuntyurenko, O.\,V.} 2017. Proizvodstvo informatsionno-analiticheskikh 
produktov i~uslug s~ispol'zovaniem metodov naukometrii i~analiza dannykh 
[Production of information and analytical products and services with use of 
scientometric methods and data  analysis].  \textit{Information in the 
modern world:  \mbox{VINITI} 65 Anniversary  Conference (International) Proceedings}. 
Moscow. 317--321.
 \end{thebibliography}

 }
 }

\end{multicols}

\vspace*{-6pt}

\hfill{\small\textit{Received December 26, 2017}}

%\vspace*{-10pt}   
    
    \Contrl
    
      \noindent
      \textbf{Syuntyurenko Oleg V.} (b.\ 1946)~--- Doctor of Science in 
technology, professor, leading scientist, All Russian Institute of Scientific 
and Technical Information of the Russian Academy of Sciences, 20a~Usievich Str., 
Moscow 125190, Russian Federation; \mbox{olegasu@mail.ru}
   \label{end\stat}


\renewcommand{\bibname}{\protect\rm Литература} 
     