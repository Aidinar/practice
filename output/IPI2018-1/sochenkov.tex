\def\stat{sochenkov}

\def\tit{ЭКСПЛОРАТИВНЫЙ ПАТЕНТНЫЙ ПОИСК$^*$}

\def\titkol{Эксплоративный патентный поиск}

\def\aut{И.\,В.~Соченков$^1$, Д.\,В.~Зубарев$^2$, И.\,А.~Тихомиров$^3$}

\def\autkol{И.\,В.~Соченков, Д.\,В.~Зубарев, И.\,А.~Тихомиров}

\titel{\tit}{\aut}{\autkol}{\titkol}

\index{Соченков И.\,В.}
\index{Зубарев Д.\,В.}
\index{Тихомиров И.\,А.}
\index{Sochenkov I.}
\index{Zubarev D.}
\index{Tikhomirov I.}




{\renewcommand{\thefootnote}{\fnsymbol{footnote}} \footnotetext[1]
{Работа выполнена при поддержке РФФИ (проект 16-29-12929).}}


\renewcommand{\thefootnote}{\arabic{footnote}}
\footnotetext[1]{Институт системного анализа Федерального исследовательского центра 
  <<Информатика и~управ\-ле\-ние>> 
Российской академии наук; Сколковский институт науки и~технологий, 
\mbox{I.Sochenkov@skoltech.ru}}
\footnotetext[2]{Институт системного анализа Федерального исследовательского центра 
<<Информатика и~управ\-ле\-ние>> 
Российской академии наук; Российский университет дружбы народов, \mbox{zubarev@isa.ru}}
\footnotetext[3]{Институт системного анализа Федерального исследовательского центра 
<<Информатика и~управ\-ле\-ние>> Российской академии наук, \mbox{tih@isa.ru}}

\vspace*{-6pt}

\Abst{Предложен эффективный метод поиска тематически похожих документов. Показана 
его применимость для решения задач эксплоративного патентного поиска. Метод позволяет 
снизить трудоемкость и~повысить качество решения аналитических задач, связанных 
с~патентным поиском и~анализом. В~качестве признаков для представления текстовых 
документов используются как отдельные лексемы, так и~словосочетания, выделяемые 
синтаксически. Это позволяет решить проблему экспоненциального роста размерности 
признакового пространства и~дает возможность эффективной индексации больших массивов 
текстовой информации. Представлены результаты экспериментов по решению 
одной из задач экспертизы заявок на изобретения и~полезные модели. Сделаны выводы 
о~перспективности использования метода при решении других задач.}

\KW{эксплоративный поиск; патентный поиск; тематическое моделирование; поиск 
тематически похожих документов; по\-иско\-во-ана\-ли\-ти\-че\-ские сис\-те\-мы}

\DOI{10.14357/19922264180111} 
  
\vspace*{-6pt}


\vskip 10pt plus 9pt minus 6pt

\thispagestyle{headings}

\begin{multicols}{2}

\label{st\stat}

\section{Введение}

  Активное развитие мировой науки и~технологий ведет к~увеличению 
накопленных объемов знаний, представленных, в~том числе, в~виде патентной 
информации. Практически любое прикладное научное исследование 
начинается с~патентного поиска, это позволяет избежать повторного открытия 
уже известных технических решений и~обеспечивает правильный выбор пути 
в~достижении по\-став\-лен\-ных целей и~задач исследования.
  
  Вид патентного поиска, направленный на получение наиболее актуальной 
  и~полной информации по заданной теме, называется эксплоративным (exploratory 
search)~[1, 2]. Его особенность заключается в~том, что аналитик ищет не 
конкретный факт или документ, а~собирает сведения о~заданной проб\-ле\-ме 
в~целом, делая акцент на полноту и~актуальность. Зачастую аналитик может не 
вполне ясно пред\-став\-лять себе предмет поиска, внося уточнения по ходу 
изучения собираемой информации. 
  
  Проведение качественного эксплоративного патентного поиска является 
весьма трудоемкой зада\-чей, которая не может быть решена без использования 
современных патентных поисково-ана\-ли\-ти\-че\-ских сис\-тем. Традиционно 
сложились три основных подхода к~патентному поиску~[3]:
  \begin{enumerate}[(1)]
\item   поиск по классам МПК (международная патентная классификация);
\item   поиск по ключевым словам;
\item   тематический поиск по заданному до\-ку\-мен\-ту-об\-разцу.
\end{enumerate}

  Первый подход требует от аналитика априорного знания структуры 
применяемого рубрикатора для формирования области поиска в~соответствии 
с~его информационной потребностью. Это требование зачастую оказывается 
излишне жестким, например редакция МПК-2016.01 содержит более~73~тыс.\ 
подгрупп~--- классов нижнего уровня, что затрудняет подбор правильных 
кодов МПК для поиска и~превращает этот процесс в~долгую рутинную 
процедуру.
  
  Второй подход позволяет эксперту сформулировать информационную 
потребность в~виде ключевых слов, которые должны встречаться в~текстах 
описаний патентов или их метаданных: в~заголовке, аннотации, списке авторов и~т.\,д. 
Составление списка ключевых слов, исчерпывающе опи\-сы\-ва\-ющих объект 
поиска, является весьма трудоемкой задачей, требующей привлечения эксперта в~предметной области. Таким образом, поиск по ключевым словам, с~одной 
стороны, повышает полноту поиска, с~другой~--- снижает его точность.
  
  Третий подход является альтернативой вышеприведенным и~заключается 
в~автоматическом поиске документов, тематически похожих на за\-данный 
эталонный документ, который содержит описа\-ние на\-уч\-но-тех\-ни\-че\-ской 
задачи или изобретения по исследуемой теме. Такой подход лишен недостатков 
первых двух, однако требует качественной и~эффективной реализации.
  
  В настоящей работе представлены разработанный авторами метод поиска 
тематически похожих документов в~приложении к~задаче эксплоративного 
патентного поиска и~результаты его экспериментального исследования.

\vspace*{-9pt}
  
\section{Метод поиска тематически похожих документов}

\vspace*{-3pt}
     
  Задача поиска тематически похожих документов по заданному эталону 
в~коллекциях полнотекстовых документов решается методами, которые 
условно делятся на две группы:
  \begin{enumerate}[(1)]
  \item  вероятностное тематическое моделирование~[4];
    \item  построение и~сопоставление ключевых дескрипторов документов. 
    \end{enumerate}
    
  Для обеих групп обычно применяется векторное представление документа. 
В~качестве признаков используются  отдельные лексемы. В~последнее время 
также наблюдается тенденция к~использованию биграмм и~триграмм слов (см., 
например, работы~[5, 6]). Для эффективной реализации применяют 
предварительную индексацию информации~[7, 8].
  
  Разработанный авторами метод поиска тематически похожих документов 
относится ко второй группе и~является развитием метода, описанного 
в~работах~[9, 10]. Отличительными особенностями этого метода являются: 
  \begin{itemize}
\item учет двусловных сочетаний (выделяемых синтаксическими методами);
\item способ оценки значимости признаков, позволяющий эффективно 
выделять наиболее важные из них;
\item эффективная реализация с~помощью инвертированных индексов, 
позволяющая решать задачу в~коллекциях, содержащих десятки и~сотни 
миллионов документов.
\end{itemize}
  
  Учет словосочетаний, выделяемых синтаксически, позволяет избежать 
проблемы роста раз\-мер\-ности признакового пространства (в~отличие от 
мето\-дов, учитывающих коллокации в~качестве признаков). В~результате такого 
подхода достигаются высокие показатели качества результатов поиска 
тематически похожих документов. Рас\-смот\-рим метод более подробно.
  
  Пусть $C=\{d_i\}^N_{i=1}$~--- коллекция текстовых документов, $e$~--- 
эталонный документ из~$C$: $e\hm=d_k$, $1\hm\leq k\leq N$. Задача поиска 
тематически похожих документов заключается в~формировании списка 
документов, упорядоченного по убыванию значений некоторой функции 
оценок $S(e,d_i)\hm\geq \mathrm{minSim}$, $i\hm=1,\ldots, N$, где $\mathrm{minSim}$~--- пороговое 
значение сходства.
  
  Каждому документу $d_i$, $1\hm\leq i\hm\leq N$, сопоставим его текст~--- 
множество словоупотреблений:
$$
T\left(d_i\right)= \left\{x_j^i\right\}_{j=1}^{M_i}\,.
$$

Лексическим дескриптором (ЛД) будем называть:
  \begin{itemize}
\item отдельную лексему;\\[-14pt]
\item словосочетание, состоящее из нескольких грамматически связанных 
лексем.
\end{itemize}

  Положим $L=\{w_m\}^Z_{m=1}$~--- конечный лексикон (словарь) 
естественных языков. Пусть $2^{T(d_i)}$~--- булеан множества~$T(d_i)$; 
$y\hm\in 2^{T(d_i)}$~--- некоторая совокупность словоупотреблений. 
Отображение $D(y,d_i):2^{T(d_i)}\hm\to L$ представляет <<нормализацию>> 
словоупотреблений текста~$d_i$ (приведение к~канонической, т.\,е.\ словарной, 
форме). Множество $G(d_i)\hm= \mathop{\cup}_{y\in 2^{T(d_i)}} D(y,d_i)$ 
представляет лексикон документа $\left(G \left(d_i\right)\hm\subseteq L\right)$. Обозначим:
$$
D^{-1}\left(w_m,d_i\right)=\left\{ y\in 2^{T(d_i)}\vert D\left(y,d_i\right)=w_m\right\}\,.
$$

 Тогда 
словоупотребления текста~$T(d_i)$, пред\-став\-ля\-ющие в~нем все употребления 
заданного ЛД~$w_m$, описываются следующим образом:
  $$
  G(w_m,d_i)=\mathop{\bigcup}\limits_{y\in D^{-1}(w_m,d_i)} y\,.
  $$
    В этом случае вес ЛД~$w_m$ в~тексте~$T(d_i)$ документа~$d_i$ 
определяется как мощность соответствующего множества словоупотреблений: 
$\vert G(w_m, d_i)\vert$.
  
  Совокупный вес лексикона документа:
   $$
  g(d_i)=\sum\limits_{w\in G(d_i)} \vert G(w_m,d_i)\vert\,.
  $$
  
  Аналогом частоты встречаемости ЛД в~тексте является величина:
     $$
   \mathrm{lTF} 
\left(w_m,d_i\right)= \log_{1+g(d_i)} \left( 1+ \vert G(w_m,d_i)\vert \right)\,. 
$$
  
  Для оценки значимости ЛД в~коллекции применяется величина:
  
  \noindent
  $$
  \mathrm{IDF}\left(w_m,C\right)=\log_{\vert C\vert} \fr{\vert C\vert}{\vert \{d_i\in C\vert w_m\in 
G(d_i)\}\vert}\,.
  $$
  
  Таким образом, итоговая оценка значимости ЛД в~тексте документа является 
модификацией классической схемы <<взвешивания>> лексики TF-IDF\,: 
$$
R\left(w_m, d_i, C\right)= \mathrm{lTF}\left(w_m, d_i\right)
\mathrm{IDF}\left(w_m, C\right)\,.
$$
  
  Редкие ЛД, для которых $\vert \{d_i\in C\vert w_m\in G(d_i) \}\vert \hm\leq 2$, 
считаются незначимыми: для них полагаем $\mathrm{IDF}\left(w_m, C\right)\hm=0$ по 
определению. Это позволяет исключить сло\-ва-опе\-чат\-ки, редкие 
словосочетания, записи адресов электронной почты, веб-ад\-ре\-сов и~т.\,п.
  
  Рассмотрим функции оценок сходства, эффективно вычисляемые 
  с~применением инвертированных индексов. 
  
  Пусть $\mathrm{wID}\left(w_m\right): L\hm\to 
{\mathbb{N}}$~--- функционал, взаимно однозначно сопоставляющий каж\-до\-му 
ЛД его числовой идентификатор. Документ~$d_i$, $1\hm\leq i\hm\leq N$ 
описывается с~помощью упорядоченного (по возрастанию $\mathrm{wID}\left(w_m\right)$) 
множества пар  
$\mathrm{DI}\left(d_i\right)\hm= \langle \langle \mathrm{wID}\left(d_i\right), 
\mathrm{lTF}\left(w_m, d_i\right)\rangle \vert 1\hm\leq 
m\hm\leq Z, \mathrm{lTF}\left(w_m, d_i\right)\hm> 0\rangle$~--- 
прямого индекса документа.

  Пусть $\mathrm{dID}\left(d_i\right): C\hm\to \mathbb{N}$~--- функционал, взаимно однозначно 
сопоставляющий каждому документу его числовой идентификатор. Для поиска 
тематически похожих документов применяется инвертированный индекс, 
сопоставляющий каждому ЛД~$w_m$, $1\hm\leq m\hm\leq Z$, упорядоченное 
по возрастанию идентификатора документа $\mathrm{dID}\left(d_i\right)$ 
конечное множество пар 
$\mathrm{RI}\left(w_m, C\right)\hm= \langle \langle \mathrm{dID}\left(d_i\right), 
\mathrm{lTF}\left(w_m, d_i\right)\rangle\vert 1\hm\leq 
i\hm\leq N, \mathrm{lTF}\left(w_m, d_i\right)\hm>0\rangle$. 
  
  Инвертированный индекс коллекции строится на основе прямых индексов 
всех документов.
  
  Для поиска тематически похожих документов с~помощью инвертированного 
индекса производится вычисление оценки сходства $S(e,d_i)$, $i\hm=1,\ldots ,
N$, следующим образом. В~качестве запроса выступает наиболее значимая 
часть лексикона эталонного документа
  \begin{multline*}
  G_1(e)\subseteq G(e):\ \forall\ w\in G_1(e)\\
  \forall\ w^\prime \in G(e)\backslash G_1(e)\gamma(w,e,C)\geq 
\gamma\left(w^\prime,e, C\right)\,,
  \end{multline*}
где в~качестве функции~$\gamma$, определяющей значимость ЛД в~эталонном 
документе, может быть взята функция~$R$ или~$\mathrm{IDF}$. Информация, 
необходимая для расчета~$\mathrm{lTF}$ ЛД берется из прямого индекса эталонного 
документа. Величина~$\mathrm{IDF}$ рассчитывается на основе частот встречаемости 
ЛД в~коллекции, которая хранится в~счетчиках, инкрементально обновляемых 
на этапе индексации коллекции. В~результате выборки данных из 
инвертированного индекса формируется множество 
$$
\mathrm{SR}\,(e,C)= 
\left\{\mathrm{RI}\,(w,C)\vert w\in G_1(e)\right\}\,.
$$

 Далее каждому~$\mathrm{RI}\,(w,C)$ сопоставляется 
множество троек (дескрипторов) $\beta_0(\mathrm{RI}\,(w,C))\hm= \{\langle 
\mathrm{wID}\,(w), \mathrm{dID}\,(d_i), \mathrm{lTF}\,(w,d_i)\rangle \vert 
1\hm\leq i\hm\leq N,\linebreak \mathrm{lTF}\,(w,d_i)\hm>0\}$, 
упорядоченное, как и~исходное множество, по возрастанию $\mathrm{dID}\,(d_i)$. Полагая 
далее $\beta_1(\mathrm{SR}\,(e,C))\hm= \{\beta(\mathrm{RI}\,(w,C))\vert w\hm\in 
G_1(e)\}$,  
с~по\-мощью однопроходной операции слияния упорядоченных множеств 
получим множество троек:
$$
U(e,C)= \mathop{\bigcup}\limits_{u\in \beta_1(\mathrm{SR}\,(e,C))} u\,.
$$ 
При слиянии тройки упорядочиваются естественным образом по воз\-рас\-та\-нию 
$\mathrm{dID}\left(d_i\right)$, а~при их совпадении~--- по воз\-рас\-та\-нию 
$\mathrm{wID}\left(w_m\right)$. Таким 
образом, в~$U(e,C)$ последовательно друг за другом содержатся дескрипторы, 
характеризующие употребления ЛД эталонного документа в~документах 
коллекции. Это дает возможность вычислить следующие оценки тематического 
сходства за один проход последовательности~$U(e,C)$:
\begin{align*}
S_{\mathrm{Ham}}(e,d_i)&=1-\hspace*{-3mm}\sum\limits_{w\in G_1(e)\cap G(d_i)}\hspace*{-5mm}
\left\vert \mathrm{lTF}\,(w,e)-{}\right.\\
&\left.\hspace*{-6mm}{}-
\mathrm{lTF}\,(w,d_i)\right\vert \mathrm{IDF}\,(w,C)\Bigg/\hspace*{-1mm}
\sum\limits_{w\in G_1(e)}\hspace*{-3mm} R(w,e,C)\,;
\\
S_{\mathrm{cos}}(e,d_i) &=\hspace*{-5mm}\sum\limits_{w\in G_1(e)\cap G(d_i)}\hspace*{-6mm}
 R(w,e,C) R(w,d_i, 
C)^2\Bigg/ \\
&\hspace*{-34pt}\Bigg/
\left(\sum\limits_{w\in G_1(e)} \hspace*{-2mm}R(w,e,C) \sum\limits_{w\in G_1(e)\cap 
G(d_i)}\hspace*{-2mm} R(w,d_i,C)\right)\,.
\end{align*}
  
  Обе величины нормированы на~1. Оценка тематического сходства для 
неполных дубликатов документов дает значения, близкие к~1. Для оценки 
тематического сходства документов, которые не являются полными 
дубликатами, важно выбирать такой размер множества $G_1(e)$, чтобы в~него 
входили наиболее значимые ЛД. В~противном случае учитывается 
и~нетематическая лексика, что приводит к~попаданию в~результаты поиска 
также документов по другим темам. Заметим, что при реализации описанного 
подхода на практике, как правило, не требуется, чтобы эталонный документ 
был предварительно проиндексирован в~коллекцию, т.\,е.\ до\-пус\-ти\-мо 
$e\hm\not\in C$.

\begin{table*}\small %tabl1
\begin{center}
\Caption{Исходные данные~--- коллекция патентов}
\vspace*{2ex}

\begin{tabular}{|l|c|c|}
\hline
\multicolumn{1}{|c|}{Вид документа}&Число документов&Ретроспектива\\
\hline
Патент на изобретение&580 тыс.&1994--2015 гг.\\
Патент на полезную модель&120 тыс.&2004--2015 гг.\\
\hline
ВСЕГО&700 тыс.&\\
\hline
\end{tabular}
\end{center}
%\end{table*}
%\begin{table*}\small %tabl2
\begin{center}
\Caption{Исходные данные~--- коллекция отклоненных заявок}
\vspace*{2ex}

\begin{tabular}{|c|c|c|}
\hline
Вид документа&Число документов&\tabcolsep=0pt\begin{tabular}{c}Общее число ссылок\\ 
на предшествующие аналоги\end{tabular}\\
\hline
Заявка на выдачу патента&158&237\\
\hline
\end{tabular}
\end{center}
%\end{table*}
%\begin{table*}\small %tabl3
\begin{center}
\Caption{Результаты оценки качества обнаружения до\-ку\-мен\-тов-ана\-ло\-гов}
\vspace*{2ex}

\tabcolsep=10pt
\begin{tabular}{|l|c|c|c|c|c|}
\hline
\multicolumn{1}{|c|}{Метод}&{\sf nDCG}&Полнота&$D_{10}$&$D_{20}$&Медиана\\
\hline
Базовый$^*$&0,141&0,414&0,198&0,262&200\\
Предложенный &0,411&0,751&0,451&0,502&\hphantom{9}20\\
\hline
  \multicolumn{6}{p{110mm}}{\footnotesize \hspace*{2mm}$^*$Значения оценок вычислены 
на основе используемого в~ФИПС метода по состоянию на август 2016~г.}\\
\end{tabular}
\end{center}
  \end{table*}
  
\section{Экспериментальное исследование метода на~примере 
экспертизы патентных заявок} 
  
  В процессе экспертизы патентных заявок основанием для отказа в~выдаче 
патента могут быть аналогичные изобретения или полезные модели, 
запатентованные ранее. Авторами статьи выдвинута следующая гипотеза: чем 
выше тематическое сходство заявки и~некоторого ранее запатентованного 
изобретения, тем больше оснований у эксперта отказать в~выдаче патента из-за 
отсутствия новизны или изобретательского уровня. Для проверки данной 
гипотезы проведены эксперименты, суть которых представлена далее.
  
  В качестве набора исходных данных использована коллекция полных текстов 
описаний российских патентов на изобретения и~полезные модели (табл.~1) 
и~коллекция заявок на выдачу патента, по которым был получен отказ из-за 
отсутствия новизны или изобретательского уровня с~указанием ссылки на ранее 
запатентованный аналог (пред\-шест\-ву\-ющие аналоги) из первой коллекции 
(табл.~2). 


    
  Задача поиска тематически похожих документов в~эксперименте заключается в~обнаружении для заданной заявки всех предшествующих аналогов 
и~упорядочении поисковой выдачи по убыванию оценки тематического 
сходства.
  
  Тестовый набор данных разделялся случайным образом на обучающую 
(126~заявок) и~контрольную (32~заявки) выборки для применения трехкратной 
кросс-ва\-ли\-да\-ции. На обучающей выборке по\-до\-бра\-ны значения параметров 
алгоритма поиска тематически похожих документов, обеспечивающие на этой 
выборке наибольшее среднее значение мет\-ри\-ки ${\sf nDCG}$. Наиболее 
значимая часть лексикона эталонного документа $G_1(e)$ строилась на основе 
оценки~$R$ с~ограничением $\vert G_1(e)\vert \hm= a\vert G(e)\vert$, $0\hm< 
a\hm\leq 1$. Простым перебором параметра~$a$ в~диапазоне $[0{,}15; 0{,}5]$ 
с~шагом~0,05 было получено значение $a\hm=0{,}3$ для всех трех блоков 
валидации с~проверкой по контрольной выборке. При этом использовалась 
оценка тематического сходства~$S_{\mathrm{Ham}}$ (результаты для оценки $S_{\mathrm{cos}}$ 
были на несколько процентов ниже).
  
  После этого выполнена оценка качества поиска на всем наборе тестовых 
данных. Использованы следующие оценки: {\sf nDCG}; полнота; $D_b$~--- 
доля предшествующих аналогов, обнаруженных\linebreak
 среди~$b$~первых документов 
выдачи ($D_{10}$, $D_{20}$); медиана~--- медианное значение позиции 
первого аналога в~выдаче. В~табл.~3 представлены результаты оценки 
разработанного метода и~приведено его сравнение с~базовым методом, который 
применяется в~Федеральном институте промышленной соб\-ст\-вен\-ности
(ФИПС) для решения аналогичной задачи.


  
  По сравнению с~базовым методом предложенный метод позволяет 
в~3/4~случаев обнаружить документы, которые являются прямыми аналогами 
поданных заявок (против менее половины случаев у базового метода). При 
каждом втором поиске предшествующий аналог находится среди 
первых~20~документов выдачи (для базового метода это происходит только 
при одном поиске из четырех). По~$D_{10}$ предложенный метод 
демонстрирует двукратное преимущество.
  
\section{Заключение}
     
  В статье предложен новый эффективный метод поиска тематически похожих 
документов по заданному эталону. Метод позволяет реализовать поддержку 
эксперта при эксплоративном патентном поиске. Проведенные эксперименты 
показали работоспособность и~качество предложенного метода. Его внедрение в~процессы патентной экспертизы будет способствовать снижению 
трудоемкости анализа патентных заявок по существу.
  
  Возможны и~другие приложения разработанного метода. Например, его 
можно использовать для подбора экспертов на научные проекты, публикации 
или на\-уч\-но-тех\-ни\-че\-ские отчеты, применять в~качестве предварительного 
фильтра при поиске текстовых заимствований или подбора кодов МПК при 
формировании новой патентной заявки. 
  
  Поиск тематически похожих документов лежит в~основе метода выделения 
научных направлений с~использованием метрической тематической 
клас\-те\-ри\-за\-ции~[11]. Последний, в~свою очередь, служит основой для 
выявления тенденций развития и~обнаружения <<точек роста>> в~областях 
знаний.
  
  Кроме того, предложенный метод был успешно использован для 
обнаружения до\-ку\-мен\-тов-кан\-ди\-да\-тов, которые потенциально являются 
источниками текстовых заимствований~[12].
  
{\small\frenchspacing
 {%\baselineskip=10.8pt
 \addcontentsline{toc}{section}{References}
 \begin{thebibliography}{99}
\bibitem{1-soc}
\Au{Marchionini G.} Exploratory search: From finding to understanding~// Commun. 
ACM, 2006. Vol.~49. No.\,4. P.~41--46.
\bibitem{2-soc}
\Au{White R.\,W., Roth~R.\,A.} Exploratory search: Beyond the query-response paradigm.~--- 
Synthesis lectures on information concepts, retrieval, and services
ser.~--- Morgan \& Claypool Publs., 2009. Vol.~1. No.\,1.  
98~p.
\bibitem{3-soc}
\Au{Osipov G., Smirnov~I., Tikhomirov~I., Sochenkov~I., Shelmanov~A., Shvets~A.} Information 
retrieval for R\&D support~// Professional search in the modern world~/ Eds.  G.~Paltoglou, 
F.~Loizides, P.~Hansen.~--- Lecture notes in computer science ser.~---  
Springer, 2014. Vol.~8830. P.~45--69.
\bibitem{4-soc}
\Au{Воронцов К.\,В.} 
Аддитивная регуляризация тематических моделей коллекций текстовых документов~//
Докл. РАН, 2014. Т.~456. №\,3. С.~268--271.


\bibitem{5-soc}
\Au{Moloshnikov I., Sboev~A., Gudovskikh~D., Rybka~R.} An algorithm of finding thematically 
similar documents with creating context-semantic graph based on probabilistic-entropy approach~// 
Procedia Comput. Sci., 2015. Vol.~66. P.~297--306.
\bibitem{6-soc}
\Au{Nokel M., Loukachevitch~N.} Accounting ngrams and multi-word terms can improve topic 
models~// 54th Annual Meeting of the Association for Computational Linguistics. Proceedings of 
12th Workshop on Multiword Expressions.~--- Stroudsburg, PA, USA: ACL, 2016. 
P.~44--49.

\bibitem{8-soc} %7
\Au{Glauner P.\,O., Iwaszkiewicz~J., Meur~J.\,Y.\,L., Simko~T.} Use of Solr and Xapian in the 
Invenio document repository software~// ArXiv, 2013. {\sf https://arxiv.org/ pdf/1310.0250.pdf}. 

\bibitem{7-soc} %8
\Au{Grainger T., Potter~T.} Solr in action.~--- New York, NY, USA: Manning 
Publications, 2014. 664~p.

\bibitem{9-soc}
\Au{Ilyinsky S., Kuzmin~M., Melkov~A., Segalovich~I.} An efficient method to detect duplicates of 
Web documents with the use of inverted index~// 11th World Wide Web Conference (International) 
Proceedings.~--- New York, NY, USA: ACM, 2002. 4~p.
\bibitem{10-soc}
\Au{Агеев М.\,С., Добров~Б.\,В.} Метод эффективного расчета матрицы ближайших соседей 
для полнотекстовых документов~// Вестник СПб ун-та. Сер.~10: 
Прикладная математика. Информатика, 2011. №\,3. С.~72--84.
\bibitem{11-soc}
\Au{Shvets A., Devyatkin~D., Sochenkov~I., Tikhomirov~I., Popov~K., Yarygin~K.} Detection of 
current research directions based on full-text clustering~// Science and Information Conference.~--- 
London: IEEE, 2015. P.~483--488.
\bibitem{12-soc}
\Au{Zubarev D., Sochenkov~I.} Using sentence similarity measure for plagiarism source retrieval~// 
CEUR Workshop Proceedings, 2014. Vol.~1180: Working Notes for CLEF 2014 Conference. 
P.~1027--1034. {\sf http://ceur-ws.org/Vol-1180/CLEF2014wn-Pan-ZubarevEt2014.pdf}.
 \end{thebibliography}

 }
 }

\end{multicols}

\vspace*{-6pt}

\hfill{\small\textit{Поступила в~редакцию 04.04.17}}

\vspace*{8pt}

%\newpage

%\vspace*{-24pt}

\hrule

\vspace*{2pt}

\hrule

%\vspace*{8pt}


\def\tit{EXPLORATORY PATENT SEARCH}

\def\titkol{Exploratory patent search}

\def\aut{I.~Sochenkov$^{1,2}$, D.~Zubarev$^{1,3}$, and~I.~Tikhomirov$^1$}

\def\autkol{I.~Sochenkov, D.~Zubarev, and~I.~Tikhomirov}

\titel{\tit}{\aut}{\autkol}{\titkol}

\vspace*{-9pt}


\noindent
$^1$Institute for Systems Analysis, Federal Research 
Center ``Computer Science and Control'' of the Russian Academy\linebreak
$\hphantom{^1}$of Sciences, 
\mbox{44-2}~Vavilov Str., Moscow 119333, Russian Federation

\noindent
$^2$Skolkovo Institute of Science and Technology,
3~Nobelya Str., Moscow 121205, Russian Federation

\noindent
$^3$Peoples' Friendship 
University of Russia (RUDN University), 6~Miklukho-Maklaya Str., Moscow 117198, 
Russian\linebreak
$\hphantom{^1}$Federation


\def\leftfootline{\small{\textbf{\thepage}
\hfill INFORMATIKA I EE PRIMENENIYA~--- INFORMATICS AND
APPLICATIONS\ \ \ 2018\ \ \ volume~12\ \ \ issue\ 1}
}%
 \def\rightfootline{\small{INFORMATIKA I EE PRIMENENIYA~---
INFORMATICS AND APPLICATIONS\ \ \ 2018\ \ \ volume~12\ \ \ issue\ 1
\hfill \textbf{\thepage}}}

\vspace*{3pt}


    



  
  \Abste{The paper presents an effective method for topically similar document 
retrieval. The exploratory patent search based on this method is proposed. The 
developed method reduces complexity and time of patent expertise providing the 
computer assistance of patent search and analysis. The phrases extracted by the 
parser as well as single lexemes are used as descriptors for a document. This 
approach prevents exponential growth of the feature space\linebreak\vspace*{-12pt}}

\Abstend{and provides effective 
indexing even for large text collections. The results of experiments show that the 
proposed method significantly outperforms the basic keyword-based approach. 
Conclusions are made about the prospects of using the method for solving other 
problems such as source retrieval for plagiarism detection and full-text clustering.}
  
  \KWE{exploratory search; patent search; topic modeling; topically similar 
document retrieval; search and analytical engines}
\DOI{10.14357/19922264180111} 

%\vspace*{-12pt}

\Ack
\noindent
The work was supported  by the Russian Foundation for
Basic Research (project 16-29-12929).



%\vspace*{3pt}

  \begin{multicols}{2}

\renewcommand{\bibname}{\protect\rmfamily References}
%\renewcommand{\bibname}{\large\protect\rm References}

{\small\frenchspacing
 {%\baselineskip=10.8pt
 \addcontentsline{toc}{section}{References}
 \begin{thebibliography}{99} 


\bibitem{1-soc-1}
\Aue{Marchionini, G.} 2006. Exploratory search: From finding to 
understanding. \textit{Commun. ACM} 49(4):41--46.
\bibitem{2-soc-1}
\Aue{White, R.\,W., and R.\,A.~Roth}. 2009. \textit{Exploratory search: Beyond the 
query-response paradigm}. {Synthesis lectures on information concepts, 
retrieval, and services ser.} Morgan \& Claypool Publs. 1(1). 98~p.
\bibitem{3-soc-1}
\Aue{Osipov, G., I.~Smirnov, I.~Tikhomirov, I.~Sochenkov, A.~Shelmanov, 
and A.~Shvets}. 2014. Information retrieval for R\&D support. 
\textit{Professional search in the modern world.}
Eds. G.~Paltoglou, F.~Loizides, and P.~Hansen. Lecture notes in computer 
science ser. Springer. 8830:45--69.
\bibitem{4-soc-1}
\Aue{Vorontsov, K.\,V.} 2014. Additive regularization for topic models of text 
collections. \textit{Dokl. Math.} 89(3):301--304.
\bibitem{5-soc-1}
\Aue{Moloshnikov, I., A.~Sboev, D.~Gudovskikh, and R.~Rybka}. 2015. An 
algorithm of finding thematically similar documents with creating  
context-semantic graph based on probabilistic-entropy approach. 
\textit{Procedia Comput. Sci.} 66:297--306.
\bibitem{6-soc-1}
\Aue{Nokel, M., and N.~Loukachevitch}. 2016. Accounting ngrams and 
multi-word terms can improve topic models. \textit{54th Annual Meeting of the 
Association for Computational Linguistics. Proceedings of 12th Workshop on 
Multiword Expressions}. Stroudsburg, PA: ACL. 44--49.

\bibitem{8-soc-1}
\Aue{Glauner, P.\,O., J.~Iwaszkiewicz, J.\,Y.\,L.~Meur, and T.~Simko}. 2013 
Use of Solr and Xapian in the Invenio document repository software. 
\textit{ArXiv preprint}. Available at: {\sf https://arxiv.org/pdf/1310.0250.pdf} 
(accessed November~17, 2017).

\bibitem{7-soc-1}
\Aue{Grainger, T., and T.~Potter}. 2014. \textit{Solr in action}. 
New York, NY: Manning Publications. 664~p.

\bibitem{9-soc-1}
\Aue{Ilyinsky, S., M.~Kuzmin, A.~Melkov, and I.~Segalovich}. 2002. An 
efficient method to detect duplicates of Web documents with the use of inverted 
index. \textit{11th  World Wide Web Conference (International) Proceedings}. 
New York, NY: ACM. 4~p.
\bibitem{10-soc-1}
\Aue{Ageev, M.\,S., and B.\,V.~Dobrov}. 2011. Metod effektivnogo rascheta 
matritsy blizhayshikh sosedey dlya polnotekstovykh dokumentov [An efficient 
nearest neighbors search algorithm for full-text documents]. \textit{Vestnik 
SPb un-ta. Ser.~10. Prikladnaya matematika. Informatika} [Vestnik 
of Saint Petersburg University. Applied mathematics. Computer science. 
Control processes] 3:72--84.
\bibitem{11-soc-1}
\Aue{Shvets, A., D.~Devyatkin, I.~Sochenkov, I.~Tikhomirov, K.~Popov, and 
K.~Yarygin}. 2015. Detection of current research directions based on full-text 
clustering. \textit{Science and Information Conference}. London: IEEE. 483--488.
\bibitem{12-soc-1}
\Aue{Zubarev, D., and I.~Sochenkov}. 2014. Using sentence similarity 
measure for plagiarism source retrieval. \textit{CEUR Workshop Proceedings: 
CLEF 2014 (Working Notes)}. 1027--1034. Available at: {\sf  
http://ceur-ws.org/Vol-1180/CLEF2014wn-Pan-ZubarevEt2014.pdf} (accessed 
November~17, 2017).

\end{thebibliography}

 }
 }

\end{multicols}

\vspace*{-6pt}

\hfill{\small\textit{Received April 4, 2017}}

%\vspace*{-10pt}

\Contr

\noindent
\textbf{Sochenkov Ilya V.} (b.\ 1985)~--- Candidate of Science (PhD) in physics 
and mathematics, senior scientist, Skolkovo Institute of Science and Technology,
3~Nobelya Str., Moscow 121205, Russian Federation;
lead scientist, Institute for Systems Analysis, Federal Research Center 
``Computer Science and Control''of the Russian Academy of Sciences, 44-2~Vavilov 
Str., Moscow 119333, Russian Federation; \mbox{I.Sochenkov@skoltech.ru}

\vspace*{3pt}


\noindent
\textbf{Zubarev Denis  V.} (b.\ 1989)~--- PhD student, Peoples' Friendship 
University of Russia (RUDN University), 6~Miklukho-Maklaya Str., Moscow 117198, 
Russian Federation; researcher, Institute for Systems Analysis, Federal Research 
Center ``Computer Science and Control'' of the Russian Academy of Sciences, 
44-2~Vavilov Str., Moscow 119333, Russian Federation; \mbox{zubarev@isa.ru}

\vspace*{3pt}

\noindent
\textbf{Tikhomirov Ilya A.} (b.\ 1980)~--- Candidate of Science (PhD) in 
technology, Head of Laboratory, Institute for Systems Analysis, Federal Research 
Center ``Computer Science and Control'' of the Russian Academy of Sciences, 
44-2~Vavilov Str., Moscow 119333, Russian Federation; \mbox{tih@isa.ru}

     
\label{end\stat}


\renewcommand{\bibname}{\protect\rm Литература} 