\def\stat{aleshin}

\def\tit{О~ФОРМАЛЬНОЙ ПОСТАНОВКЕ ЗАДАЧ ПОИСКА СГУЩЕНИЙ 
В~РАЗРЕЖЕННЫХ БУЛЕВЫХ  МАТРИЦАХ$^*$}

\def\titkol{О~формальной постановке задач поиска сгущений в~разреженных булевых 
матрицах}

\def\aut{И.\,С.~Алешин$^1$}

\def\autkol{И.\,С.~Алешин}

\titel{\tit}{\aut}{\autkol}{\titkol}

\index{Алешин И.\,С.}
\index{Aleshin I.\,S.}




{\renewcommand{\thefootnote}{\fnsymbol{footnote}} \footnotetext[1]
{Работа выполнена при поддержке РФФИ (проект 17-20-02200).}}


\renewcommand{\thefootnote}{\arabic{footnote}}
\footnotetext[1]{Факультет вычислительной математики и~кибернетики 
Московского госудаственного университета им. М.\,В.\,Ломоносова, 
\mbox{ilyaaln@yandex.ru}}

\vspace*{-6pt}
    


\Abst{В~огромном числе прикладных задач интеллектуального 
анализа данных, таких как исследование генной экспрессии и~тканей,  текстовой 
и~веб-ин\-фор\-ма\-ции, рыночных корзин, клиентских сред, входная информация 
естественным образом представляется в~виде двумерной матрицы <<субъ\-ек\-ты--объ\-ек\-ты>> 
(<<кли\-ен\-ты--сер\-ви\-сы>>). Основной целью в~рамках указанных областей  
является так называемая бикластеризация данных, т.\,е.\ выделение групп 
в~определенном смысле схожих строк и~столбцов. Немалая часть таких задач 
характеризуется сильной разреженностью соответствующих матриц. Важным аспектом 
бикластеризации является поиск в~некотором смысле плотных подматриц в~булевых 
матрицах, что является основной целью данного исследования. В~работе 
производится формализация предметной области в~рамках алгебраического подхода, 
описаны системы универсальных и~локальных ограничений, предложены и~доказаны 
соответствующие критерии разрешимости рассматриваемых задач.}

\KW{разреженные матрицы; плотные подматрицы; 
алгебраический подход; тео\-ре\-ти\-ко-мно\-же\-ст\-вен\-ные ограничения; бикластеризация}

\DOI{10.14357/19922264180105}
  
\vspace*{-6pt}


\vskip 10pt plus 9pt minus 6pt

\thispagestyle{headings}

\begin{multicols}{2}

\label{st\stat}

\section{Введение}

Данная работа выполнена в~рамках алгебраического подхода, развиваемого 
академиком РАН Ю.\,И.~Журавлевым~\cite{Jur1,Jur2,Jur3} и~его научной школой.
Будем рассматривать булевы разреженные матрицы. Разреженные матрицы определяются 
как мат\-ри\-цы с~преимущественно нулевыми элементами. Возникают ситуации, когда 
в~исходной матрице ненулевые элементы распределены неравномерно, например 
существуют в~некотором смысле плотные подмножества строк и~столбцов. Понятия 
разреженности и~плотности являются плохо формализованными, что делает 
целесообразным применение методов алгебраического подхода для поиска таких 
подматриц.

Выделение плотных подматриц играет важную роль во многих прикладных задачах 
машинного обучения и~анализа данных, например анализе данных генной экспрессии 
и~тканей в~молекулярной биологии, тематического моделирования и~построения 
рекомендательных систем. Во многих источниках используется термин 
<<бикластеризация>> (автором которого считается Миркин \cite{Mirkin}),  
подразумевающий выделение групп объектов, обладающих схожими подмножествами 
признаков~\cite{Hartigan, cheng_church,tanay}. Значительная часть прикладных 
задач характеризуется сильной разреженностью данных.

Однако возможно, что данные матрицы могут иметь в~некотором смысле <<плохую>> 
структуру, например ненулевые элементы могут быть распределены равномерно или 
входные данные могут быть сильно зашумлены, что затрудняет получение ответа на 
вопрос о наличии плотных подматриц. Тем не менее при решении многих прикладных 
задач предполагается, что положения точек соответствуют некоторым ярко 
выраженным, возможно зашумленным, подмножествам.

Данная работа устроена следующим образом.
В~разд.~2 проводится формализация предметной области, в~подразд.~3.1 предложены 
возможные универсальные и~локальные ограничения, в~подразд.~3.2 описаны понятия 
разрешимости задач для случая одного сгущения, а~также приведены и~доказаны 
соответствующие критерии.

\section{Постановка задачи}

Рассматривается задача синтеза алгоритмов~$A$, реализующих отображения из 
пространства возможных начальных информаций~$\mathfrak{J_i}$ в~пространство 
возможных финальных информаций~$\mathfrak{J_f}$.
Множество всех отображений из~$\mathfrak{J_i}$ в~$\mathfrak{J_f}$ обозначим~$\mathfrak{M_*}$:
$$
\mathfrak{M_*} = \{A|A:\mathfrak{J_i} \rightarrow \mathfrak{J_f}\}\,.
$$

В данном случае множество начальных информаций представляет собой множество 
всевозможных $\{0,1\}$-мат\-риц размера $m \times n$:
$$
\mathfrak{J_i} =  \mathfrak{C}_{mn}\{0,1\}\,,$$
где $\mathfrak{C}_{mn}\{0,1\}$ 
обозначает множество матриц порядка $m\times n$ с~элементами из 
множества~$\{0,1\}.$

Далее будем рассматривать два случая возможных пространств финальных информаций.
 Наиболее тривиальной является ситуация, когда в~мат\-ри\-це требуется найти не 
более одной подматрицы.
 В~данном случае пространство финальных информаций (обозначим~$\mathfrak{J_f^1}$) 
 выглядит следующим образом:
$$
\mathfrak{J_f^1} = \left\{(u, v), u \in \{0,1\}^m, v \in \{0,1\}^n\right\}\,,
$$
 т.\,е.\ 
пространство финальных информаций представляет собой множество упорядоченных пар 
двоичных векторов длины~$m$ и~$n$.
Данные векторы характеризуют принадлежность строк и~столбцов подматрице, т.\,е.\ 
строка с~номером~$i$ (столбец с~номером~$j$) принадлежит подматрице некоторой 
мат\-ри\-цы $U \hm\in \mathfrak{J_i}$, если $u_i \hm=1$ ($v_j\hm= 1$), и~не принадлежит 
в~противном случае. Заметим, что пара $(u, v)$, для которой 
$$
u= 
(\underbrace{0,\dots, 0}_m)$$
 либо 
 $$
 v= (\underbrace{0,\dots, 0}_n),$$ 
соответствует пус\-той подматрице, что рассматриваться не будет. Финальную 
информацию, со\-от\-вет\-ст\-ву\-ющую случаю, когда искомая подматрица не найдена, 
обозначим~$\triangle$. Также отметим, что заведомо не будем рассматривать 
нулевую матрицу, поскольку она не содержит ненулевых подматриц.

Рассмотрим произвольные $U \hm\in \mathfrak{C}_{mn}\{0,1\}$,  $u\hm \in \{0,1\}^m$, 
$v\hm \in \{0,1\}^n$. Обозначим символом $U(u,v)$ подматрицу матрицы~$U$, 
образованную строками с~номерами $\{i|u_i\hm=1\}$ и~столбцами с~номерами 
$\{j|v_j\hm=1\}$.

Обозначим через $||u|| \hm=\sum\nolimits_{i=1}^m u_i$ число строк, входящих 
в~подматрицу~$U(u,v)$, а~через   $||v||\hm=\sum\nolimits_{j=1}^m v_j$ 
соответствующее чис\-ло столбцов.

\smallskip

\noindent
\textbf{Определение~2.1.}\ \textbf{Площадью} 
подматрицы $U(u,v)$ мат\-ри\-цы~$U$ будем называть величину $\mathfrak{s}(U(u,v))$, 
равную~$||u||*||v||$.

\smallskip

\noindent
\textbf{Определение~2.2.}\ \textbf{Плотностью}
подматрицы будем называть отношение числа ее ненулевых элементов к~ее размеру. 
Обозначим плот\-ность~$\rho(U(u,v))$.

\smallskip

Для общего случая некоторого числа под\-мат\-риц определим пространство финальных 
информаций~$\mathfrak{J_f}$ как множество векторов нефиксированного размера~$K$, 
элементы которых представляют собой попарно различные упорядоченные пары булевых 
векторов принадлежностей строк и~столбцов подматрицам:

\noindent
\begin{multline*}
\mathfrak{J_f} = \{(\bar u, \bar v) =\left(\left(u^1, v^1\right), \ldots,\left(u^K, v^K\right)\right), \\
u^i \in \{0,1\}^m,\enskip v^i \in \{0,1\}^n\},\ \mbox{при } 
\left(u^i, v^i\right) \ne \left(u^j, v^j\right),\\ 
u^k \ne \left(\underbrace{0,\dots, 0}_m\right),\ 
v^r \ne \left(\underbrace{0,\dots, 0}_n\right)\\
 \mbox{для~всех }
i,j,k,r \in \{1,\ldots,K\}\,,
\end{multline*}
где $K$~--- число рассматриваемых подматриц 
некоторой матрицы $U$. Стоит заметить, что $K \hm\in \left\{1,\ldots,(2^m-1)*(2^n- 
1)\right\}$.

При данном представлении финальной информации может возникнуть неоднозначность. 
Рассмотрим следующий пример матрицы $5\times 5$:
$$
\begin{pmatrix} 
1 & 1 & 1 & 0 & 0 \\ 1 & 1 & 1 & 0 & 0 \\ 0 & 0 & 0 & 1 & 1 \\ 
0 & 0 & 0 & 1 & 1 \\ 0 & 0 & 0 & 1 & 1
 \end{pmatrix}\,.
$$

Предположим, необходимо найти подматрицы, характеризующиеся максимальной 
плотностью и~нерасширяемостью по размеру с~неуменьшением плот\-ности. Как можно 
легко видеть, исходная\linebreak матрица содержит две такие подматрицы.
Однако в~данном случае существуют два со\-от\-вет\-ст\-ву\-ющих представления, 
отличающихся порядком подматриц: 
\begin{multline*}
\left(\left( (1,1,0,0,0),(1,1,1,0,0)\right),\right.\\
\left.\left((0,0,1,1,1), (0,0,0,1,1)\right)\right)
\end{multline*}
 и
 \begin{multline*}
\left(\left((0,0,1,1,1),  (0,0,0,1,1)\right),\right.\\ 
\left.\left((1,1,0,0,0),(1,1,1,0,0)\right)\right)\,.
\end{multline*}
Для устранения данной неоднозначности будем считать, что все множества пар 
булевых векторов, соответствующих финальным информациям, упорядочены по размеру 
подматриц:
$$
||u^1||*||v^1|| \geq ||u^2||*||v^2|| \geq \cdots \geq~||u^k||*||v^k||\,.
$$
 При 
совпадении площадей подматриц упорядочим их таким образом, чтобы $\rho(u^i, 
v^i)\hm \geq \rho(u^j,v^j)$. При совпадении плотностей будем считать, что первой 
идет подматрица, имеющая строку (при полном совпадении строк~--- столбец) 
с~наименьшим номером.

Также стоит отметить, что при указанном выше упорядочении в~рамках каждого 
элемента множества финальной информации 
\begin{multline*}
|\mathfrak{J_f}|=  \sum\limits_{k = 1}^{(2^m-1)(2^n- 1)}
\begin{pmatrix}
{(2^m-1)(2^n- 1)}\\
{k}\end{pmatrix}
={}\\
{}=2^{(2^m-1)(2^n- 1)} - 1\,.
\end{multline*}


Перейдем к~рассмотрению понятия алгоритма выделения сгущений.

\smallskip

\noindent
\textbf{Определение~2.3.}\ \textbf{Алгоритмом 
выделения} подматриц будем называть всякое отображение $A: \mathfrak{J_i} 
\hm\rightarrow \mathfrak{J_f}$ (или  $\mathfrak{J_i} \hm\rightarrow 
\mathfrak{J^1_f}$).

\smallskip

Согласно теме данной работы интерес пред\-став\-ля\-ют в~некотором смысле плотные 
подматрицы. Плотные подматрицы также будем именовать сгущениями. Из интуитивных 
соображений следует, что далеко не любые подматрицы можно считать плотными.
Например, в~сгущения не должны входить <<малоплотные>> строки (столбцы) при 
наличии ненулевых строк (столбцов), не входящих в~сгущение, с~большей 
плотностью. Множество разумных требований к~сгущениям накладывает определенные 
ограничения на алгоритмы.
Таким образом, задачи определяются структурной информацией~$I_s$, выделяющей из 
множества всех отображений~$\mathfrak{M_*}$ подмножества допустимых отображений, 
обозначаемых~$\mathfrak{M}[I_s]$. Алгоритмы, реализующие данные отображения, 
будем называть корректными. Требования к~алгоритмам будут подробно рассмотрены 
в~следующем разделе.

\section{Универсальные и~локальные ограничения. Разрешимые  задачи}

В алгебраическом подходе синтез алгоритмов осуществляется на базе структурной 
информации, т.\,е.\ информации о требованиях к~искомым 
алгоритмам~\cite{Rudakov1,Rudakov2,Rudakov3,Rudakov4,Rudakov5}. Характерной частью 
структурной информации являются так называемые локальные ограничения, 
представляющие собой набор прецедентов.

\smallskip

\noindent
\textbf{Определение~3.1.}\ \textbf{Набором прецедентов}
называется произвольная совокупность $(S_1,\ldots,S_q)$, где $S_i \hm= (U^i , 
(\bar u^i, \bar v^i))$, $U^i \hm\in \mathfrak{J_i}$, $(\bar u^i, \bar v^i) \hm\in 
\mathfrak{J_f}$ (или $(u^i, v^i) \hm\in \mathfrak{J_f^1}$)   для некоторого $q \hm\in 
\mathbb{N}$.
Иными словами, множество прецедентов представляет собой набор матриц, для 
которых известны расположения подматриц, интерпретируемых как сгущения.

\smallskip

Однако во многих задачах явно недостаточно рассматривать только прецедентную 
информацию для построения корректного алгоритма.
В~рамках алгебраического подхода при недостаточно формализованной постановке 
задачи легко получить формально верные, но абсолютно бесполезные результаты.
Например, в~рамках задачи выделения плотных подматриц, если потребовать, чтобы 
для некоторого алгоритма~$A$ выполнялась в~определенном смысле <<точность на 
прецедентах>>, то формально правильным окажется алгоритм, такой что $A(U^i) \hm= 
(\bar u^i, \bar v^i$) $\forall\ U^i$,
 принадлежащих прецедентной информации, и~принимающий произвольное константное 
значение для всех остальных~$U^i$. В~данном случае возникает так называемое 
явление переобучения. Ясно, что данные алгоритмы являются <<несодержательными>> в~текущей постановке.

Таким образом, в~постановке задачи необходимо\linebreak учитывать дополнительные 
ограничения на пары\linebreak (<<начальная информация>>, <<финальная информа\-ция>>), 
выделяющие из множества всех отоб\-ра\-же\-ний так называемые допустимые отоб\-ра\-же\-ния.
Данные требования получили название универсальных ограничений, которые будем 
обозначать~$I_s^u$.

\smallskip

\noindent
\textbf{Определение~3.2.} \textbf{Универсальными 
ограничениями {\boldmath{$I_s^u$}}} будем называть систему требований, представляющую 
собой набор предикатов $I_s^u \hm= \{I_s^{u_1}, \ldots, I_s^{u_N}, \ldots, \}$, 
где 
$$
{I_s^{u_i}}: (\mathfrak{J_i}\times\mathfrak{J_f})\rightarrow \{0,1\}\, 
\left(\mbox{ или~}\mathfrak{J_i}\times\mathfrak{J_f^1} \rightarrow \{0,1\}\right)\,.
$$

 Описание конкретных примеров систем универсальных ограничений и~будет одной из 
главных целей текущего раздела.

\subsection{Описание универсальных ограничений для~случая одного~сгущения}

Приведем примеры некоторых <<разумных>> требований. Для начала рассмотрим 
ситуацию, когда матрица содержит не более одного сгущения.
\begin{description}
\item[\,] \textbf{Требование максимальности плот\-ности/пло\-щади}

Обозначим данное ограничение $I_s^{u_{\max\_\rho s}}$. Пусть $U$~--- произвольная 
матрица из $\mathfrak{C}_{mn}\{0,1\}$, $(u,v)\hm\in  \mathfrak{J_f^1}$~---
соответствующая ей финальная информация. Будем говорить, что пара $(U, (u,v))$ 
удовлетворяет требованию максимальности плотности/площадии $I_s^{u_{\max\_\rho 
s}}$ (т.\,е.\ $I_s^{u_{\max\_\rho s}}(U,(u,v))\hm=1$), если не существует финальной 
информации  $(u',v') \hm\in  \mathfrak{J_f^1}$, такой что
\begin{multline*}
 \hspace*{-1mm}\left ( \left(\rho(u',v') > \rho(u,v)\right)  \&  
\left(\mathfrak{s}(u',v') \geq  \mathfrak{s}(u,v)\right)\right) \vee {}
\\
{} \vee 
\left(\left(\rho(u',v')\geq \rho(u,v)\right) \&  \left(\mathfrak{s}(u',v') > 
\mathfrak{s}(u,v)\right)\right)\,.
\end{multline*}


Таким образом,
не должно существовать подматрицы, более плотной при данной площади или 
подматрицы большей плот\-ности при данной площади. Если рассмотреть двумерную 
область, точки которой представляют собой подматрицы, характеризующиеся
плот\-ностью\linebreak\vspace*{-12pt}

 { \begin{center}  %fig1
 \vspace*{-1pt}
 \mbox{%
 \epsfxsize=74mm %.233mm 
 \epsfbox{ale-1.eps}
 }


\end{center}


\noindent
{\small{Точки, характеризующие требование мак\-си\-маль\-ности 
плотности/площади}}
}

%\pagebreak

\vspace*{12pt}


\noindent
 и~площадью $(\rho(u,v), (||u||*||v||)/(mn))$, то\linebreak данное 
требование со\-от\-вет\-ст\-ву\-ет множеству\linebreak
 то\-чек~$(\rho(u,v), (||u||*||v||)/(mn))$, 
которое  определяет некоторую монотонно убывающую кривую (см.\ рисунок).





Стоит отметить тот факт, что для некоторых матриц из~$J_i$ ни одна подматрица 
некоторой площади $m'*n'$, $m'\hm \le m$, $n' \hm\le n$, не может считаться плотной при 
данном ограничении. Рассмотрим следующий пример матрицы $3\times 3$:
$$
\begin{pmatrix} 
1 & 1 & 0\\ 1 & 0 & 0 \\  0 & 0 & 0\\
\end{pmatrix}\,.
$$

В данной матрице не существует подматриц площади~1 и~3, удовлетворяющих 
требованию максимальности.

\item[\,] \textbf{Требование максимальности плотности/размеров}

Обозначим данное ограничение $I_s^{u_{\max\_\rho rc}}$. Данное требование 
определяется аналогично ограниче\-нию $I_s^{u_{\max\_\rho s}}$, только вместо 
площади используются параметры числа строк и~столбцов, т.\,е.\ для $U$~--- 
произвольной матрицы из $\mathfrak{C}_{mn}\{0,1\}$, $(u,v) \hm\in  
\mathfrak{J_f^1}$ не существует финальной информации  $(u',v') \hm\in  
\mathfrak{J_f^1}$, такой что
\begin{multline*}
\left (\left(\rho(u',v') > \rho(u,v)\right)   \& \right. \\  
\left.\left(||u'||~\geq~||u||\right)  \&  \left(||v'||\geq~||v||\right) \right) 
\vee{} \\ 
{}\vee \left( \left(\rho(u',v') \geq \rho(u,v)\right)  \& \right. \\ 
\left.\&   \left(||u'|| > ||u||\right)  \&  \left(||v'||\geq ||v||\right) \right) 
\vee{} \\ 
{}\vee \left( \left(\rho(u',v') \geq \rho(u,v)\right) \& \right.\\ 
\left.\&  \left(||u'|| \ge ||u||\bigr) \, \& \, \bigl(||v'|| > ||v||\right) \right)\,.
\end{multline*}
\end{description}



Для описания остальных ограничений определим следующую величину: 
$$
\forall\ x^1, 
x^2  \in \{0,1\}^l,\enskip r\left(x^1, x^2\right) = \sum\limits_{i=1}^l 
\left[ x^1_i\, \&\, x^2_i\right]\,,
$$ 
где $[I]$~--- индикатор 
события~$I$.

\begin{description}
\item[\,] \textbf{Требование сильной связности}

Обозначим данное требование $I_s^{u_{con_s}}$.

Интуитивно данное требование можно определить следующим образом. Если 
рассматривать матрицу $U(u,v)$ как матрицу смежности некоторого графа, то 
требование сильной связности соответствует наличию пути длины~2 между любыми 
двумя вершинами этого графа.

Рассмотрим пример матрицы $U^{6\times 6}$:
$$
\begin{pmatrix} 
1 & 1 & 1 & 0 & 0 & 0 \\ 
1 & 1 & 1 & 0 & 0 & 0 \\
 1 & 1 & 1 & 0 & 0 & 0\\
0 & 0 & 0 & 1 & 1 & 0 \\ 
0 & 0 & 0 & 1 & 1 & 0 \\ 
0 & 0 & 0 & 0 & 0 & 
0 \end{pmatrix}\,.
$$

Данное ограничение запрещает считать плотной подматрицу
$$
(u,v)=((1,1,1,1,1,0), (1,1,1,1,1,0))\,,
$$
 поскольку, например,
$$
r\left(U\left(e^{6[3]}, v\right), U\left(e^{6[4]}, v\right)\right) = 0\,.
$$

Формализуем данное требование. Пусть $U \hm\in \mathfrak{C_{mn}}\{0,1\}$, $(u,v) 
\hm\in  \mathfrak{J_f^1}$~--- соответствующая ей финальная информация. 
Обозначим $I \hm= \{i:u(i)=1\}$, $J \hm= \{j:v(j)=1\}$.

Обозначим $e^{m[i]}$ вектор длины~$m$, у~которого $i$-я компонента равна~1, все 
остальные~--- 0. Тогда не существует $i_1, i_2\hm \in I$, что
 $r\left(U(e^{m[i_1]}, v), U(e^{m[i_2]}, v)\right) \hm= 0$. 
 Не существует $j_1, j_2 
\hm\in J$, что 
$$
r\left(U\left(u, e^{n[j_1]}\right), U\left(u, e^{n[j_2]}\right)\right) = 0\,.
$$

\item[\,] \textbf{Требование связности}

Обозначим данное требование $I_s^{u_{con_l}}$.

Если рассматривать матрицу $U(u,v)$ как матрицу смежности некоторого графа, то 
требование связности соответствует наличию пути между любыми двумя вершинами 
этого графа.

 Формализуем данное требование. Пусть $U\hm \in \mathfrak{C_{mn}}\{0,1\}$, $(u,v) 
\hm\in  \mathfrak{J_f^1}$~--- соответствующая ей финальная информация.
Обозначим $U' \hm= U(u,v)$. Пусть $U' \hm\in {C_{m'n'}}\{0,1\}$. Обозначим $I' \hm= 
\{1,\ldots,m'\}$, $J'\hm = \{1,\ldots,n'\}$.

Тогда существуют $I_1, \ldots,I_l:$
\begin{enumerate}[(1)]
\item $\bigcup\nolimits_{k=1}^l I_k = I'$, $I_i \bigcap I_j \hm= \emptyset$ 
при $i \hm\ne j$, что для 
любого~$p$, для любых $i_1, i_2 \hm\in I_p$
    $$
    r\left(U\left(e^{m'[i_1]}, \{1\}^{n'}\right), U\left(e^{m'[i_2]}, 
\{1\}^{n'}\right)\right) \ne 0\,;
$$


\item $\forall\ I_i, I_j$ $\exists\ i_1, i_2, i_1 \hm\in I_i, i_2 \hm\in I_j$ 
$$
r\left(U\left(e^{m'[i_1]}, \{1\}^{n'}\right), U\left(e^{m'[i_2]}, 
\{1\}^{n'}\right)\right) \ne 0\,.
$$
\end{enumerate}
Требование существования разбиения для столбцов определяется аналогично.

Рассмотренный выше пример также не удовле\-тво\-ря\-ет требованию связности. 
Рассмотрим сле\-ду\-ющий пример матрицы $U^{6\times 6}$:
$$
\begin{pmatrix} 
1 & 1 & 1 & 0 & 0 & 0 \\ 
1 & 1 & 1 & 0 & 0 & 0 \\ 
1 & 1 & 1 & 0 
& 0 & 0\\
 0 & 0 & 1 & 1 & 1 & 0 \\ 
 0 & 0 & 0 & 1 & 1 & 0 \\ 
 0 & 0 & 0 & 0 & 0 & 0 
 \end{pmatrix}\,.
 $$

Данное ограничение не запрещает считать плотной подматрицу 
$(u,v)\hm= ((1,1,1,1,1,0),$\linebreak $ (1,1,1,1,1,0))$, однако ограничение сильной связности 
запрещает считать данную подматрицу плотной, поскольку  $r\left(U\left(e^{6[3]}, v\right), 
U\left(e^{6[5]}, v\right)\right) \hm= 0$.
\end{description}

Рассмотрим некоторые свойства предложенных ограничений.
    
\smallskip

\noindent
\textbf{Утверждение~1.}\ \textit{Пусть $(U,(u,v))$~--- произвольная пара 
(начальная информация, финальная информация). Тогда из выполнения условия 
$I_s^{u_{\mathrm{con}\_s}}(U,(u,v)) \hm= 1$ следует выполнение условия} 
$I_s^{u_{\mathrm{con}\_l}}(U,(u,v)) \hm= 1$.

\smallskip

\noindent
\textbf{Определение~3.3.} Система ограничений~$I_s^{u}$ называется 
покрывающей, если для любой матрицы $U \hm\in \mathfrak{J_i}$ существует пара 
$(u,v)\hm \in \mathfrak{J_{f}^1}$, такая что $I_s^{u}((U,(u,v)))\hm=1$.

\smallskip

\noindent
\textbf{Утверждение~2.} \textit{Системы ограничений $I_s^{u_{\max\_\rho rc}}$ 
и~$I_s^{u_{\max\_\rho s}}$~--- покрывающие}.

\smallskip

\noindent
Д\,о\,к\,а\,з\,а\,т\,е\,л\,ь\,с\,т\,в\,о\ \ данного утверждения очевидно, поскольку для любой матрицы $ U \hm\in 
\mathfrak{J_i}$ выполнены условия: 
\begin{align*}
I_s^{u_{\max\_\rho rc}}\left(\left(U, \left(\{1\}^m, 
\{1\}^n\right)\right)\right)&=1\,;\\
I_s^{u_{\max\_\rho s}}\left(\left(U, \left(\{1\}^m, \{1\}^n\right)\right)\right)&=1\,.
\end{align*}


\noindent
\textbf{Утверждение~3.}\ \textit{Системы ограничений $I_s^{u_{\mathrm{con}_s}}$
и~$I_s^{u_{\mathrm{con}_l}}$~--- покрывающие}.

\smallskip

\noindent
Для д\,о\,к\,а\,з\,а\,т\,е\,л\,ь\,с\,т\,в\,а\ данного утверждения для любой матрицы $U\hm\ne O^{m\times n}$ 
можно выбрать $(u,v) \hm\in \mathfrak{J_{f}^1}$, соответствующие подматрице из 
одной строки и~одного столбца, содержащих ненулевой элемент. Поскольку случай 
нулевой матрицы $O^{m\times n}$ не рассматривается в~данной работе, утверждение 
доказано.


\smallskip

\noindent
\textbf{Утверждение~4.}\ \textit{Системы ограничений $\{I_s^{u_{\mathrm{con}_s}}, 
I_s^{u_{\max\_\rho s}}\}$, $\{I_s^{u_{\mathrm{con}_l}}$, $I_s^{u_{\max\_\rho s}}\}$,
$\{I_s^{u_{\mathrm{con}_s}}$, $I_s^{u_{\max\_\rho rc}}\}$ 
и~$\{I_s^{u_{\mathrm{con}_l}}$, $I_s^{u_{\max\_\rho rc}}\}$~--- покрывающие}.

\smallskip

\noindent
Для д\,о\,к\,а\,з\,а\,т\,е\,л\,ь\,с\,т\,в\,а\ данного утверждения для любой 
матрицы $U\hm\ne O^{m\times n}$ 
можно выбрать $(u,v) \hm\in \mathfrak{J_{f}^1}$, соответствующие подматрице из 
единиц максимального размера.


\subsection{Разрешимые  задачи для~случая~одного~сгущения}

Итак, задача $Z$ определяется множеством начальных информаций~$\mathfrak{J_i}$, 
множеством финальных информаций $\mathfrak{J_{f}^1}$, а~также совокупностью 
универсальных ($I_s^u$) и~локальных ($I_s^l$) ограничений.
Локальные ограничения представляются в~виде прецедентной информации  
$(S_1,\ldots,S_q)$, где $S_i \hm= (U^i , (u^i, v^i))$, $U^i \hm\in \mathfrak{J_i}$, 
$(u^i, v^i) \hm\in \mathfrak{J_{f}^1}$.

Таким образом, $Z \hm= (S$, $\mathfrak{J_i}$, $\mathfrak{J_f^1}$, $I_s^u$).

В качестве системы универсальных ограничений~$I_s^u$ рассмотрим систему 
$\{I_s^{u_{\mathrm{con}_s}}, I_s^{u_{\max\_\rho s}}\}$ (аналогично рассматриваются системы 
$\{I_s^{u_{\mathrm{con}_l}}, I_s^{u_{\max\_\rho s}}\}$, $\{I_s^{u_{\mathrm{con}_s}}, 
I_s^{u_{\max\_\rho rc}}\}$, 
$\{I_s^{u_{\mathrm{con}_l}}, I_s^{u_{\max\_\rho rc}}\}$,
$\{I_s^{u_{\max\_\rho rc}}\}$ и~$\{I_s^{u_{\max\_\rho s}}\}$).

\smallskip

\noindent
\textbf{Определение~3.4.}  Финальная 
информация $(u,v) \hm\in \mathfrak{J_{f}^1}$ называется допустимой для начальной 
информации $U \hm\in \mathfrak{C}_{mn}\{0,1\}$ в~рамках системы~$I_s^u$, если 
$I_s^u((U, (u,v)))\hm=1$.

\smallskip

\noindent
\textbf{Определение~3.5.}\ Алгоритм~$A: 
\mathfrak{J_i} \hm\rightarrow \mathfrak{J_f^1}$ будем называть подходящим в~рамках 
системы~$I_s^u$, если для любой начальной информации~$A(U)$~--- допустимая 
финальная информация.

\smallskip

\noindent
\textbf{Определение~3.6.}\ Алгоритм~$A: 
\mathfrak{J_i} \hm\rightarrow \mathfrak{J_f^1}$ будем называть корректным на 
прецедентах $(S_1,\ldots,S_q)$, где $S_i \hm= (U^i , (u^i, v^i))$, если для любого~$i$ 
из множества $\{1,\ldots,q\}$ верно, что либо $\rho(U^i(A(U^i)) \hm= 
\rho(U^i(u^i, v^i))$ и~$||u^i||~*~||v^i||\hm = \mathfrak{s}(U^i(A(U^i)))$, 
либо $A(U^i)\hm=(u^i,  v^i)\hm=\triangle$.

\smallskip

\noindent
\textbf{Определение~3.7.}\
Задача~$Z$, определяемая множеством начальных информаций~$\mathfrak{J_i}$, 
множеством финальных информаций~$\mathfrak{J_{f}^1}$, универсальными 
ограничениями ($I_s^u$) и~набором прецедентов  $(S_1,\ldots,S_q)$, называется 
разрешимой, если для нее существует подходящий корректный алгоритм.

\subsubsection{Разрешимые  задачи для~коммутирующих алгоритмов}

Далее определим некоторые классы эквивалентности прецедентной информации.

Рассмотрим $\mathfrak{S_m}$ и~$\mathfrak{S_n}$~--- симметрические группы на 
множествах $\{1,\ldots,m\}$ и~$\{1,\ldots,n\}$, со\-от\-вет\-ст\-ву\-ющие перестановкам номеров 
строк (столбцов). Рассмотрим произвольные $s^i\hm\in \mathfrak{S_m}$  
и~$s^j\hm\in \mathfrak{S_n}$. Для произвольной матрицы $U \hm\in 
\mathfrak{C}_{mn}\{0,1\}$ определим матрицу~$s^i(s^j(U))$ как матрицу, 
полученную перестановкой строк (столбцов) матрицы~$U$ местами согласно 
подстановкам $s^i \hm\in \mathfrak{S_m}$ на множестве номеров строк,  $s^j \hm\in 
\mathfrak{S_n}$ на множестве номеров столбцов.

\smallskip

\noindent
\textbf{Определение~3.8.}\
Будем называть две матрицы $U^1, U^2 \hm\in \mathfrak{J_i}$ эквивалентными, если 
существуют $ s^i \hm\in \mathfrak{S_m}$,  $s^j \hm\in \mathfrak{S_n}$ такие, что 
$s^i(s^j(U^1)) \hm= U^2$.

\smallskip

\noindent
\textbf{Определение~3.9.}\
Будем говорить, что алгоритм~$A: \mathfrak{J_i} \hm\rightarrow \mathfrak{J_f^1}$ 
коммутирует с~$\mathfrak{S_m}$ ($\mathfrak{S_n}$), если для любой матрицы $U \hm\in 
\mathfrak{J_i}$, для любой $s \hm\in \mathfrak{S_m}$ ($\mathfrak{S_n}$) либо 
$\rho(A(U))\hm=\rho(A(s(U)))$ и~$\mathfrak{s}(A(U)) \hm= \mathfrak{s}(A(s(U)))$, 
либо $A(U)\hm=A(s(U))\hm=\triangle$.

\smallskip

Необходимо подчеркнуть, что часто требование коммутации определяется как 
$A(s(U))~=~s(A(U))$, однако в~рамках рассматриваемой работы такое определение 
может быть некорректным. Рас\-смот\-рим пример матрицы~$U$ размера $3\times 3$:
$$
\begin{pmatrix}
 1 & 1 & 0\\ 
 1 & 0 & 0
  \\  0 & 1 & 0
  \end{pmatrix}\,.
  $$

Данная матрица имеет две плотные подматрицы размера~4: $((1,1,0),\, (1,1,0))$ 
и~$((1,0,1),\, (1,1,0))$. Так как требуется найти лишь одно сгущение, выберем, 
например, подматрицу, имеющую строку с~наименьшим номером, т.\,е.\ 
$A(U)\hm=((1,1,0),\, (1,1,0))$. Рассмотрим подстановки на множестве номеров строк 
$s^i \hm\in \mathfrak{S_3}$, в~циклической записи имеющую вид $(23)(1)$, 
и~множестве номеров столбцов $s^j \hm= (12)(3)$. Обозначим через $s$ последовательную 
перестановку столбцов матрицы~$U$, соответствующую~$s^j$, и~строк, 
соответствующую~$s^i$. Подействуем $s$ на матрицу~$U$. В~результате получим, что 
$s(U)\equiv U$, $A(s(U))\hm=  ((1,1,0),\, (1,1,0))\hm\ne ((1,0,1),\, 
(1,1,0)) \hm= s(A(U))$.

Далее будем рассматривать алгоритмы, коммутирующие с~$\mathfrak{S_m}$ 
и~$\mathfrak{S_n}$. Стоит отметить, что условие коммутации может быть задано 
в~виде универсального ограничения на алгоритмы.

Рассмотрим набор прецедентов $(S_1,\ldots,S_q) \hm= ((U^1, (u^1,v^1)),\ldots,
(U^q, (u^q,v^q)))$, где $U^1, \ldots, U^q$~--- эквивалентные 
матрицы.

\smallskip

\noindent
\textbf{Определение~3.10.}\
Набор прецедентов $((U^1, (u^1,v^1)),\ldots,(U^q, (u^q,v^q)))$ будем называть 
противоречивым, если $\exists$\ $i1, i2$, что матрицы $U^1, U^2$ эквивалентны, 
однако
либо $(u^{i1},v^{i1})\hm=\triangle$, $(u^{i2},v^{i2})\hm\ne\triangle$,  либо 
$\rho(U^{i1}(u^{i1},v^{i1}))\hm\ne \rho(U^{i2}(u^{i2},v^{i2}))$, либо 
$\mathfrak{s}(U^{i1}(u^{i1},v^{i1})) \hm\ne \mathfrak{s}(U^{i2}(u^{i2},v^{i2}))$. 
В~противном случае набор пар $((U^1, (u^1,v^1)),\ldots,(U^q, (u^q,v^q)))$ будем 
называть непротиворечивым.


\smallskip

\noindent
\textbf{Теорема~1 (критерий разрешимости)}.
\textit{Задача $Z$, определяемая 
множеством начальных информаций $\mathfrak{J_i}$, множеством финальных 
информаций~$\mathfrak{J_{f}^1}$, универсальными ограничениями~($I_s^u$) 
и~набором прецедентов  $(S_1,\ldots,S_q)$, является разрешимой, если и~только если 
набор прецедентов непротиворечив}.

\smallskip


\noindent
\textbf{Необходимость.}\ Пусть задача $Z$, определяемая множеством 
начальных информаций~$\mathfrak{J_i}$, множеством финальных 
информаций~$\mathfrak{J_{f}^1}$, универсальными ограничениями ($I_s^u$) и~набором 
прецедентов  $(S_1,\ldots,S_q)$, является разрешимой, т.\,е.\ существует корректный 
подходящий алгоритм $A:~\mathfrak{J_i} \hm\rightarrow \mathfrak{J_f^1}$, 
коммутирующий с~$\mathfrak{S_m}$ и~$\mathfrak{S_n}$. Пусть прецедентная 
информация противоречива, т.\,е.\ существует пара прецедентов $S_i \hm= (U^i, 
(u^i,v^i))$ и~$S_j\hm=(U^j, (u^j,v^j))$, таких что матрицы~$U^i$ и~$U^j$ 
эквивалентны, но либо $(u^{i},v^{i})\hm = \triangle$, $(u^{j},v^{j})\hm\ne 
\triangle$, либо $\rho(U^{i}(u^{i},v^{i}))\hm\ne \rho(U^{j}(u^{j},v^{j}))$, либо
$\mathfrak{s}(U^{i}(u^{i},v^{i}))\hm\ne \mathfrak{s}(U^{j}(u^{j},v^{j}))$. 
Поскольку алгоритм~$A$ корректный, то либо

\noindent
\begin{align*}
 \rho\left(U^i\left(A\left(U^i\right)\right)\right)&=\rho\left(U^i\left(u^i, v^i\right)\right)\,; \\
    \rho\left(U^j\left(A\left(U^j\right)\right)\right)&=\rho\left(U^j\left(u^j, v^j\right)\right)\,;\\ 
    \mathfrak{s}\left(U^i\left(A\left(U^i\right)\right)\right)&=  \mathfrak{s}
    \left(U^i\left(u^i, v^i\right)\right)\,;\\ 
    \mathfrak{s}\left(U^j\left(A\left(U^j\right)\right)\right)&=\mathfrak{s}\left(U^j\left(u^j, v^j\right)\right),
\end{align*}
 либо
 
\noindent
\begin{align*}
A\left(U^i\right)= \left(u^{i},v^{i}\right)&=\triangle\,;\\
\rho\left(U^j\left(A\left(U^j\right)\right)\right)&=\rho\left(U^j\left(u^j, v^j\right)\right)\,;\\ 
\mathfrak{s}\left(U^j\left(A\left(U^j\right)\right)\right)&=\mathfrak{s}\left(U^j\left(u^j, v^j\right)\right).
\end{align*}
 Однако возникает противоречие с~требованием коммутации на алгоритм~$A$.


\smallskip

\noindent
\textbf{Достаточность.}\ Пусть прецедентная информация непротиворечива.

Определим алгоритм~$A$, коммутирующий с~$\mathfrak{S_m}$ и~$\mathfrak{S_n}$, для 
произвольной матрицы $U \hm\in \mathfrak{J_i}$ сле\-ду\-ющим образом:

\noindent
$$
 A(U)=\begin{cases}
 &\!\!\!\!\!\!\left(u^i,v^i\right), \mbox{ если } \exists\ i,\ \mbox{что~для }\\
&\!\!\!\!\!\!\hspace*{3mm} S_i =\left(U^i,(u^i,v^i)\right) \mbox{ выполнено } U^i=U\,;\\ 
&\!\!\!\!\!\!(u,v)\!:  (U,(u,v)) \mbox{ ---~допустимая~пара} \\ 
&\!\!\!\!\!\!\hspace*{20.5mm}\mbox{в~рамках~системы } I_s^u \mbox{ иначе}. 
\end{cases}
$$

Алгоритм~$A$ является корректным и~подходящим
по построению; следовательно, задача~$Z$ является разрешимой.

\subsubsection{Разрешимые задачи для~монотонных прецедентов}

Рассмотрим задачу~$Z$, которая определяется множеством начальных 
информаций~$\mathfrak{J_i}$, множеством финальных информаций~$\mathfrak{J_{f}^1}$, а также 
совокупностью универсальных ($I_s^u$) и~локальных~($I_s^l$) ограничений.

Пусть для $Z$ существует корректный подходящий алгоритм~$A$, коммутирующий 
с~$\mathfrak{S_m}$ и~$\mathfrak{S_n}$.

Рассмотрим две произвольные допустимые пары $(U^1, (u^1, v^1))$
и~$(U^2, (u^2, v^2))$. Пусть $(u^1, v^1)\hm \ne \triangle$ и~$(u^2, 
v^2)\hm \ne\triangle$.

\smallskip

\noindent
\textbf{Определение~3.11.}\ Будем 
говорить, что $(U^2, (u^2, v^2))$ мажорирует $(U^1, (u^1, v^1))$ (обозначим: 
$(U^2, (u^2, v^2))\hm>(U^1, (u^1, v^1))$), если $\rho (U^2, (u^2, v^2))) \hm\ge \rho 
(U^1, (u^1, v^1)))$ и~$||u^2||~*~||v^2||\hm\ge ||u^1||~*~||v^1||$, причем хотя бы 
одно из этих неравенств строгое.

\smallskip

Стоит отметить, что данное определение приведено для системы, в~которую входит 
ограничение $I_s^{u_{\max\_\rho s}}$. Если же в~нее входит 
ограничение$I_s^{u_{\max\_\rho rc}}$, то условие 
$||u^2||~*~||v^2||\hm\ge ||u^1||~*~||v^1||$ следует заменить на пару условий 
$(||u^1||\hm\ge ||u^2||) \bigwedge (||v^1||\hm\ge ||v^2||)$.

Рассмотрим произвольную допустимую пару $(U^1,(u^1,v^1))$ и~произвольную 
матрицу~$U^2$.

\smallskip

\noindent
\textbf{Определение~3.12.}\ Будем 
говорить, что матрица~$U^2$ мажорирует пару $(U^1,(u^1,v^1))$, если
существует допустимая в~рамках~$I_s^{u}$ финальная информация $(u^2, v^2)$, 
такая что $(U^2, (u^2, v^2))\hm>(U^1, (u^1, v^1)).$

\smallskip

Приведем следующий пример. Рассмотрим две пары $(U^1, (u^1,v^1))$ и~$(U^2, 
(u^2,v^2))$:
$$
U^1 = \begin{pmatrix}  
1 & 1 & 1 & 0 & 0 & 0 \\ 
1 & 1 & 1 & 0 & 0 & 0 \\ 
1 & 1 & 1 & 0 & 0 & 0\\ 
1 & 1 & 1 & 0 & 0 & 0 \\
 1 & 1 & 0 & 0 & 0 & 0 
 \\ 0 & 0 & 0 & 0 & 0 & 0 
 \end{pmatrix}\,;
 \enskip
U^2 = \begin{pmatrix} 
 1 & 1 & 1 & 0 & 0 & 0 
 \\ 1 & 1 & 1 & 0 & 1 & 0
  \\ 1 & 1 & 1 & 0 & 0 & 0\\
   1 & 1 & 1 & 0 & 0 & 0 \\ 
   1 & 0 & 0 & 0 & 0 & 0 \\
    0 & 0 & 0 & 0 & 0 & 0 
    \end{pmatrix}\,;
    $$

$$\left(u^1, v^1\right) =\left( (1,1,1,1,0,0), (1,1,1,0,0,0)\right)\,;
$$
$$\left(u^2, v^2\right) =\left( 
(1,1,1,1,1,0), (1,1,1,0,0,0)\right)\,.
$$ 
Как можно видеть, $U^1 \hm> (U^2, (u^2,v^2))$, 
однако $(U^1, (u^1,v^1)) \hm< (U^2, (u^2,v^2))$. Из интуитивных соображений может 
показаться,  что указанные пары (<<начальная информация>>, <<финальная 
информация>>) противоречат друг другу.

Рассмотрим произвольную допустимую пару $(U^1,(u^1,v^1))$ и~мажорирующую  
матрицу~$U^2$.

\smallskip

\noindent
\textbf{Определение~3.13.}\ Будем 
говорить, что пары $(U^1,(u^1,v^1))$ и~$(U^2,(u^2,v^2))$, где~$U^2$ мажорирует 
$(U^1,(u^1,v^1))$, несогласованны (монотонно противоречивы), если не выполнено 
условие
$(U^2,(u^2,v^2)) \hm> (U^1,(u^1,v^1))$. В~противном случае пары $(U^1,(u^1,v^1))$
и~$(U^2,(u^2,v^2))$ будем называть согласованными (монотонно непротиворечивыми)

\smallskip

\noindent
\textbf{Определение~3.14.}\ Будем 
говорить, что алгоритм~$A:~\mathfrak{J_i} \hm\rightarrow \mathfrak{J_f^1}$  
удовлетворяет требованию согласованности (монотонной непротиворечивости), если 
для любых $ U^1, U^2 \hm\in \mathfrak{J_i}$ пары $(U^1,A(U^1))$ и~$(U^2,A(U^2))$ 
согласованные (монотонно непротиворечивые).

\smallskip

Далее рассмотрим согласованные (монотонно непротиворечивые) алгоритмы 
$A:\mathfrak{J_i} \hm\rightarrow \mathfrak{J_f^1}$.

\smallskip

\noindent
\textbf{Определение~3.15.}\
Набор прецедентов $(S_1,\ldots,S_q)$ будем называть согласованным (монотонно 
непротиворечивым), если любые две пары $S_i, S_j$, $i,j\hm=1,\ldots,q,$ являются 
согласованными. В~противном случае набор прецедентов $(S_1,\ldots,S_q)$ будем 
называть несогласованным.

\smallskip

\noindent
\textbf{Теорема~1*\ (критерий разрешимости)} \textit{Задача~$Z$, определяемая 
множеством начальных информаций $\mathfrak{J_i}$, множеством финальных 
информаций~$\mathfrak{J_{f}^1}$, универсальными ограничениями~($I_s^u$) 
и~набором прецедентов  $(S_1,\ldots,S_q)$, является разрешимой, если и~только если 
набор прецедентов является согласованным $($монотонно непротиворечивым$)$}.

\smallskip

\noindent
\textbf{Необходимость.}\ Пусть задача~$Z$, определяемая множеством 
начальных информаций~$\mathfrak{J_i}$, множеством финальных 
информаций~$\mathfrak{J_{f}^1}$, универсальными ограничениями ($I_s^u$) и~набором 
прецедентов  $(S_1,\ldots,S_q)$, является разрешимой, т.\,е.\ существует корректный 
подходящий, монотонно непротиворечивый алгоритм $A: \mathfrak{J_i} \hm\rightarrow 
\mathfrak{J_f^1}$, коммутирующий с~$\mathfrak{S_m}$ и~$\mathfrak{S_n}$. Пусть 
прецедентная информация монотонно противоречива, т.\,е.\ существует пара 
прецедентов $S_i\hm = (U^i, (u^i,v^i))$ и~$S_j\hm=(U^j, (u^j,v^j))$, таких что $U^i\hm > 
(U^j, (u^j,v^j))$, однако $(U^i, (u^i,v^i))\hm< (U^j, (u^j,v^j))$. Поскольку 
алгоритм~$A$ корректный, то 
\begin{align*}
\rho\left(U^i\left(A\left(U^i\right)\right)\right)&= \rho\left(U^i\left(u^i, v^i\right)\right)\,;\\ 
\rho\left(U^j\left(A(U^j)\right)\right)&= \rho\left(U^j\left(u^j, v^j\right)\right)\,;\\ 
\mathfrak{s}\left(U^i\left(A\left(U^i\right)\right)\right)&=
\mathfrak{s}\left(U^i\left(u^i, v^i\right)\right)\,;\\ 
\mathfrak{s}\left(U^j\left(A\left(U^j\right)\right)\right)&= 
\mathfrak{s}\left(U^j\left(u^j, v^j\right)\right)\,.
\end{align*}
 Однако возникает противоречие с~требованием 
монотонной непротиворечивости на алгоритм~$A$.

\smallskip

\noindent
\textbf{Достаточность}.\ Пусть прецедентная информация монотонно 
непротиворечива.

Определим алгоритм~$A$ для произвольной мат\-ри\-цы $U \hm\in \mathfrak{J_i}$ следующим 
образом:

\noindent
$$
A(U)=\begin{cases}
&\!\!\!\!\!\!\left(u^i,v^i\right),  \mbox{ если } \exists\ i,\ \mbox{что~для}\\
&\!\!\!\!\!\!\hspace*{3mm}S_i  =\left(U^i,(u^i,v^i)\right) \mbox{ выполнено } U^i=U\,;\\ 
&\!\!\!\!\!\!(u,v)\!:  (U,(u,v)) \mbox{ --- допустимая~пара} \\ 
&\!\!\!\!\!\!\hspace*{20mm}\mbox{в~рамках~системы } I_s^u \mbox{ иначе}, 
\end{cases}
$$
а также чтобы алгоритм $A$ удовлетворял условию монотонной непротиворечивости 
для любых пар $(U,A(U))$, $(U',A(U')))$. Алгоритм~$A$ является корректным, 
подходящим и~монотонно непротиворечивым
по построению; следовательно, задача~$Z$ является разрешимой.

\section{Заключение}

В данной работе описана формальная постановка задачи поиска сгущений 
в~разреженных булевых матрицах, а~также доказаны критерии раз\-ре\-ши\-мости для двух 
важных типов прецедентной информации. В~рамках дальнейшей работы планируется 
исследование понятия регулярности, а~также полноты семейств алгоритмических 
операторов, решающих правил и~корректирующих операций.

\bigskip

Автор выражает благодарность своему научному руководителю академику РАН 
Константину Владимировичу Рудакову за постоянное внимание, советы и~неоценимую 
помощь при выполнении данной работы.

{\small\frenchspacing
 {%\baselineskip=10.8pt
 \addcontentsline{toc}{section}{References}
 \begin{thebibliography}{99}
\bibitem{Jur1}
\Au{Журавлев Ю.\,И.} Корректные алгебры над множествами 
некорректных (эвристических) алгоритмов.  Часть~I~// Кибернетика, 1977. №\,4. 
С.~5--17.

\bibitem{Jur2}
\Au{Журавлев Ю.\,И.} Корректные алгебры над множествами 
некорректных (эвристических) алгоритмов.  Часть~II~// Кибернетика, 1977. №\,6. 
С.~21--27.

\bibitem{Jur3}
\Au{Журавлев Ю.\,И.} Корректные алгебры над множествами 
некорректных (эвристических) алгоритмов.  Часть~III~// Кибернетика, 1978. №\,2. 
С.~35--43.

\bibitem{Mirkin} 
\Au{Mirkin B.} Mathematical classification and clustering. ~--- 
Kluwer Academic Publs., 1996. 439~p.


\bibitem{Hartigan}
\Au{Hartigan J.\,A.} Direct clustering of a data matrix~// 
J.~Am. Stat. Assoc., 1972. Vol.~67. No.\,337. P.~123--129.

\bibitem{cheng_church} 
\Au{Cheng Y., Church~G.\.M.}
Biclustering of expression data~//  Conference
(International) on Intelligent Systems for 
Molecular Biology.~--- AAAI Press, 2000. P.~93--103.

\bibitem{tanay} 
\Au{Tanay A., Sharan~R., Shamir~R.} Discovering 
statistically significant biclusters in gene expression data~// Bioinformatics, 
2002. Vol.~18 (Suppl.~1). P.~136--144.



\bibitem{Rudakov2}  %8
\Au{Рудаков К.\,В.} О~некоторых универсальных 
ограничениях для алгоритмов классификации~// ЖВМ МФ, 1986. Т.~26. №\,11. 
С.~1719--1730.

\bibitem{Rudakov1} %9
\Au{Рудаков К.\,В.} 
Универсальные и~локальные ограничения в~проблеме коррекции эвристических 
алгоритмов~// Кибернетика, 1987. №\,2. C.~30--35.

\bibitem{Rudakov3} 
\Au{Рудаков К.\,В.} О~применении универсальных 
ограничений при
исследовании алгоритмов классификации~// Кибернетика, 1988. №\,1. C.~1--5.

\bibitem{Rudakov4}
\Au{Рудаков~К.\,В.} Об алгебраической теории универсальных 
и~локальных ограничений для задач классификации~// Распознавание, классификация, 
прогноз.~--- М.: Наука, 1989. С.~176--201.

\bibitem{Rudakov5}
\Au{Рудаков К.\,В., Чехович~Ю.\,В.} О~проблеме синтеза 
обуча\-емых алгоритмов выделения трендов (алгебраический подход)~// Прикладная 
математика и~информатика, 2001. №\,8. С.~97--113.

 \end{thebibliography}

 }
 }

\end{multicols}

%\vspace*{-6pt}

\hfill{\small\textit{Поступила в~редакцию 08.06.17}}

\vspace*{10pt}

%\newpage

%\vspace*{-24pt}

\hrule

\vspace*{2pt}

\hrule

\vspace*{8pt}


\def\tit{ON THE FORMALIZATION OF~TASKS SEARCHING DENSE SUBMATRICES 
IN~BOOLEAN  SPARSE MATRICES}

\def\titkol{On the formalization of tasks searching dense 
submatrices in~boolean  sparse matrices}

\def\aut{I.\,S.~Aleshin}

\def\autkol{I.\,S.~Aleshin}

\titel{\tit}{\aut}{\autkol}{\titkol}

\vspace*{-9pt}


\noindent
Faculty of Computational Mathematics and Cybernetics, 
M.\,V.~Lomonosov Moscow State University, GSP-1, Leninskie Gory, 
Moscow 119991, Russian Federation 



\def\leftfootline{\small{\textbf{\thepage}
\hfill INFORMATIKA I EE PRIMENENIYA~--- INFORMATICS AND
APPLICATIONS\ \ \ 2018\ \ \ volume~12\ \ \ issue\ 1}
}%
 \def\rightfootline{\small{INFORMATIKA I EE PRIMENENIYA~---
INFORMATICS AND APPLICATIONS\ \ \ 2018\ \ \ volume~12\ \ \ issue\ 1
\hfill \textbf{\thepage}}}

\vspace*{6pt}



\Abste{In a significant part of data mining applications such as microbiology, 
gene expression data, text and web information, market baskets, 
customer environments, input information is represented as a~two-dimensional matrix 
``subjects--objects'' (``clients--services''). The main goal of such problems 
is  biclustering of data, based on the selection of groups in a~certain 
sense of similar rows and columns. A~lot of such problems is characterized 
by strong sparseness of the corresponding matrices. An important aspect 
of biclustering is the search in some sense of dense submatrices in
 boolean matrices, which is the main purpose of this research.  
The author formalizes subject\linebreak\vspace*{-12pt}}

\pagebreak

\Abstend{area within the framework 
 of algebraic approach, describes the systems of universal and local 
 constraints, proposes and proves the corresponding criteria for 
 solvability of the problems under consideration.}

\KWE{sparse matrices; dense submatrices; algebraic approach; 
set-theoretic constraints; biclustering}




\DOI{10.14357/19922264180105}

%\vspace*{-12pt}

\Ack
\noindent
The investigation was supported by the Russian Foundation for Basic Research 
(project No.\,17-20-02200).



%\vspace*{3pt}

  \begin{multicols}{2}

\renewcommand{\bibname}{\protect\rmfamily References}
%\renewcommand{\bibname}{\large\protect\rm References}

{\small\frenchspacing
 {%\baselineskip=10.8pt
 \addcontentsline{toc}{section}{References}
 \begin{thebibliography}{99}
\bibitem{1-al}
\Aue{Zhuravlev, Yu.\,I.} 1977. Korrektnye  algebry nad mnozhestvami nekorrektnykh 
(evristicheskikh) algoritmov. Chast'~I 
[Correct algebras over sets of incorrect (heuristic) algorithms. Part~I]. 
\textit{Kibernetika} [Cybernetics] 4:5--17.

\bibitem{2-al}
\Aue{Zhuravlev, Yu.\,I.} 1977. Korrektnye  algebry nad mno\-zhe\-st\-va\-mi nekorrektnykh 
(evristicheskikh) algoritmov. Chast'~II 
[Correct algebras over sets of incorrect (heuristic) algorithms. Part~II]. 
\textit{Kibernetika} [Cybernetics] 6:21--27.

\bibitem{3-al}
\Aue{Zhuravlev, Yu.\,I.} 1978. Korrektnye  algebry nad mno\-zhe\-st\-va\-mi nekorrektnykh 
(evristicheskikh) algoritmov. Chast'~III 
[Correct algebras over sets of incorrect (heuristic) algorithms. Part~III]. 
\textit{Kibernetika} [Cybernetics] 2:35--43.
\bibitem{4-al}
\Aue{Mirkin, B.} 1996. 
\textit{Mathematical classification and clustering.} Kluwer Academic Publs. 439~p.
\bibitem{5-al}
\Aue{Hartigan, J.\,A.} 1972. Direct clustering of a~data matrix. 
\textit{J.~Am. Stat. Assoc.} 67(337):123--129.
\bibitem{6-al}
\Aue{Cheng, Y., and G.\,M.~Church.} 2000. 
Biclustering of expression data. 
\textit{Conference (International) on Intelligent Systems for Molecular Biology}.
 AAAI Press. P.~93--103.
\bibitem{7-al}
\Aue{Tanay, Y., R.~Sharan, and R.~Shamir.} 2002. 
Discovering statistically significant biclusters in gene expression data. 
\textit{Bioinformatics} 18(Suppl.~1):136--144.

\bibitem{9-al}
\Aue{Rudakov, K.\,V.} 1986. O~nekotorykh universal'nykh ogranicheniyakh dlya 
algoritmov klassifikatsii [Some universal constraints for classification algorithms]. 
\textit{ZhVM MF} [Comp. Math. Math. Phys.] 26(11):1719--1730.

\bibitem{8-al}
\Aue{Rudakov, K.\,V.} 1987. Universal'nye i~lokal'nye ogranicheniya 
v~probleme korrektsii evristicheskikh algoritmov 
[Universal and local constraints in the problem of correction of 
heuristic algorithms]. \textit{Kibernetika} [Cybernetics] 2:30--35.

\bibitem{10-al}
\Aue{Rudakov, K.\,V.} 1988. O~primenenii universal'nykh ogranicheniy pri issledovanii 
algoritmov klassifikatsii [About the application of universal constraints in the 
research of classification algorithms]. \textit{Kibernetika} [Cybernetics] 1:1--5.
\bibitem{11-al}
\Aue{Rudakov, K.\,V.} 1989. Ob algebraicheskoy teorii universal'nykh 
i~lokal'nykh ogranicheniy dlya zadach klassifikatsii [About the algebraic 
theory of universal and local constraints for classification problems].
\textit{Raspoznavanie, klassifikatsiya, prognoz} 
[Recognition, classification, forecast]. Moscow: Nauka. 176--201.
\bibitem{12-al}
\Aue{Rudakov, K.\,V., and Yu.\,V.~Chekhovich.} 
2002. Design of trend-identification algorithms with learning (algebraic approach).
 \textit{Comput. Math. Modeling} 13(3):281--293.

\end{thebibliography}

 }
 }

\end{multicols}

\vspace*{-6pt}

\hfill{\small\textit{Received June 8, 2017}}

%\vspace*{-10pt}

\Contrl

\noindent
\textbf{Aleshin Ilya  S.} (b.\ 1993)~--- 
PhD Student, Faculty of Computational Mathematics and Cybernetics, 
M.\,V.~Lomonosov Moscow State University, GSP-1, Leninskie Gory, 
Moscow 119991, Russian Federation; \mbox{ilyaaln@yandex.ru}

\label{end\stat}


\renewcommand{\bibname}{\protect\rm Литература} 