%\newcommand{\e}{{\sf E}}

\def\stat{kudr}

\def\tit{БАЙЕСОВСКИЕ МОДЕЛИ ТЕСТИРОВАНИЯ БОЛЬШИХ ГРУПП ОБСЛУЖИВАЮЩИХ ПРИБОРОВ$^*$}

\def\titkol{Байесовские модели тестирования больших групп обслуживающих приборов}

\def\aut{А.\,А.~Кудрявцев$^1$, О.\,В.~Шестаков$^2$}

\def\autkol{А.\,А.~Кудрявцев, О.\,В.~Шестаков}

\titel{\tit}{\aut}{\autkol}{\titkol}

\index{Кудрявцев А.\,А.}
\index{Шестаков О.\,В.}
\index{Kudryavtsev A.\,A.}
\index{Shestakov O.\,V.}




{\renewcommand{\thefootnote}{\fnsymbol{footnote}} \footnotetext[1]
{Работа выполнена при частичной финансовой поддержке РФФИ (проект 17--07--00577).}}


\renewcommand{\thefootnote}{\arabic{footnote}}
\footnotetext[1]{Московский государственный университет им.~М.\,В.~Ломоносова, 
факультет вычислительной математики и~кибернетики, \mbox{nubigena@mail.ru}}

\footnotetext[2]{Московский государственный университет им.~М.\,В.~Ломоносова, факультет вычислительной
 математики и~кибернетики; Институт проблем информатики Федерального 
 исследовательского центра <<Информатика и~управление>> Российской академии наук, 
 \mbox{oshestakov@cs.msu.su}}

%\vspace*{-6pt}



\Abst{Рассматривается новый подход к~задачам верификации заявленных показателей 
функционирования больших групп сложных обслуживающих приборов в~условиях 
невозможности (или высокой стоимости) тотальной проверки. Применяемый метод 
основан на изучении асимптотики статистик в~рамках байесовского подхода к~задачам 
теории массового обслуживания. На примере коэффициента готовности в~модели $M/M/1$ 
приводится утверждение, относящееся к~классу теорем переноса, дающее возможность 
сравнивать реальные данные при стендовых испытаниях сложных агрегатов с~теоретическими 
предсказаниями. Рассматривается специфика применения метода с~точки зрения 
производителя приборов и~заказчика большой партии однотипных приборов. 
Описываемый метод и~результаты могут найти применение не только в~области 
байесовских задач массового обслуживания, но и~в~других областях, использующих 
рандомизацию базовых параметров модели.}

\KW{байесовский подход; системы массового обслуживания; смешанные
распределения; теорема переноса; статистические методы}

\DOI{10.14357/19922264180113} 
  
\vspace*{-6pt}


\vskip 10pt plus 9pt minus 6pt

\thispagestyle{headings}

\begin{multicols}{2}

\label{st\stat}


\section{Введение}

В~настоящее время все более остро встает проб\-ле\-ма анализа 
адекватности функционирования сис\-тем и~сетей массового обслуживания.
 Это связано прежде всего с~высокой слож\-ностью со\-временных 
 сис\-тем и~агрегатов, которая не дает воз\-мож\-ности проверить функ\-цио\-наль\-ность 
 на всех допустимых входных па\-ра\-мет\-рах. Кроме того, при не\-об\-хо\-ди\-мости 
 верификации больших партий об\-слу\-жи\-ва\-ющих приборов возникают естественные 
 труд\-ности, связанные с~временн$\acute{\mbox{ы}}$ми и~финансовыми 
 затратами на проверку 
 воз\-мож\-ности допуска всех элементов партии к~эксплуатации. По этим\linebreak
  причинам 
 единственным реальным средством анали\-за пригодности больших групп слож\-ных 
 приборов, по всей видимости, является ве\-ро\-ят\-ност\-но-ста\-ти\-сти\-че\-ский. 
 Если дополнительно учесть различные вариации, вызванные технологическими 
 особенностями при производстве приборов и~изменчивостью среды, в~которой 
 происходит функционирование, также разумно дополнить статистическую
  со\-став\-ля\-ющую тестирования байесовским подходом~\cite{CSMIJ}, пред\-по\-ла\-га\-ющим 
  рандомизацию базовых па\-ра\-мет\-ров~\cite{KuSh2015}.

Рассмотрим следующую модель. Заказчику требуется большая партия обслуживающих 
приборов, соответствующая некоторому техническому заданию. Заказ поступает 
производителю. Требуется произвести две проверки адекватности приборов техническому 
заданию: одну на стороне производителя перед отправкой опытного образца в~серию, 
другую на стороне заказчика при получении изготовленной партии. Поскольку 
партия мыслится большой, а~приборы сложными, проводится выборочное 
тестирование на стенде, имитирующем условия реального функционирования. 
При этом у~производителя и~заказчика, естественно, существуют особенности и~отличия 
методики проверки.

В данной работе рассматриваются теоретические статистические результаты 
на примере парамет\-ра загрузки~$\rho$ системы $M/M/1$, равного отношению 
интенсивности входящего потока~$\lambda$ к~интенсивности обслуживания~$\mu$, 
в~предположении, что параметры модели независимы и~имеют известные априорные 
распределения.

\section{Стендовые испытания производителя}

Рассматривается задача верификации удовлетворения тестового образца обслуживающего 
прибора требованиям технического задания заказчика для последующего запуска 
образца в~серию. Для этого на испытательном стенде формируется набор 
интенсивностей входящего потока требований, являющийся конкретной 
реализацией выборки $X_1,\ldots,X_n$ параметра потока требований~$\lambda$ из
априорного распределения, предварительно изучен\-но\-го заказчиком и~описанного в~тексте 
технического задания. Конкретный параметр обслуживания исследуемого прибора является 
реализацией случайной величины~$\mu$, априорное распределение которой обусловлено 
техническим регламен\-том на производстве. Результаты стендовых испытаний прибора 
сравниваются с~теоретическими статистическими прогнозами, что дает возможность 
сделать вывод о~целесообразности запуска тестового образца в~серию.

В данном случае теоретические статистические результаты, касающиеся параметра 
загрузки~$\rho$, основываются на выборке $X_1/\mu,\ldots,X_n/\mu$.

\section{Стендовые испытания заказчика}

Рассматривается задача стендовых испытаний большой партии обслуживающих приборов, 
поступивших для верификации удовлетворения требованиям заказчика. Поскольку партия 
представляется достаточно объемной, производить тотальную проверку приборов 
нецелесообразно ввиду высокой стоимости и~больших временн$\acute{\mbox{ы}}$х затрат. В~связи 
с~этим на стенде фиксируется некоторая интенсивность входящего потока 
требований~$\lambda$ и~формируется конкретная реализация выборки приборов из партии, 
которая подлежит исследованию. Каждый прибор обладает своим параметром 
обслуживания $Y_1,\ldots,Y_n$, являющимся реализацией параметра~$\mu$ 
с~известным распределением. 

Предполагается, что заказчик предварительно 
исследовал условия, в~которых должны функционировать приборы, и,~таким образом, 
знает априорное распределение интенсивности входящего\linebreak потока требований~$\lambda$. 
Априорное распределение параметров обслуживания приборов задается техническим 
регламентом на производстве. Результаты работы каждого прибора при заданной 
интенсивности требований проходят статистический анализ, результатом 
которого является вывод о~допустимости принятия всей партии в~экс\-плу\-а\-та\-цию.
{\looseness=1

}

В данном случае теоретические статистические результаты, касающиеся 
параметра загрузки~$\rho$, основываются на выборке
 $\lambda/Y_1,\ldots,\lambda/Y_n$.

\section{Основные результаты}


Из постановки задачи следует, что параметр входящего потока~$\lambda$ и~параметр 
обслуживания~$\mu$~--- не\-от\-ри\-ца\-тель\-ные собственные случайные величины, не имеющие 
атома в~нуле. Требование ${\sf P}(\lambda=0)\hm={\sf P}(\mu=0)\hm=0$ 
означает, что прибор не может с~положительной вероятностью предназначаться 
для вечного хранения на складе и~не должен игнорировать поступающие требования.

Предположим, что параметр входящего потока~$\lambda$ и~одинаково распределенные 
параметры обслуживания $Y_1,\ldots,Y_n$ независимы. Обозначим 
$$
A(u)={\sf P}(\lambda<u)\,;
\enskip 
Z_n=\fr{1}{n}\sum\limits_{i=1}^{n}Y_i^{-1}\,.
$$

\noindent
\textbf{Теорема.} \textit{Пусть $a\hm={\sf E} Y_i^{-1}$, $0\hm<\sigma^2\hm={\sf D} 
Y_i^{-1}\hm<\infty$. Тогда при $n\hm\to\infty$}
\begin{equation}
\label{lim_1}
\sup\limits_{x\in \mathsf{R}}\abs{{\sf P}\left(\sqrt{n}\frac{\lambda(Z_n- a)}{\sigma}<x\right)-F(x)}\to 0,
\end{equation}

\noindent
\textit{где $F(x)$~--- функция распределения, со\-от\-вет\-ст\-ву\-ющая 
характеристической функции}
$$
f(t)=\int\limits_{0}^{\infty}e^{-u^2t^2/2}\,dA(u)\,.
$$

\noindent
Д\,о\,к\,а\,з\,а\,т\,е\,л\,ь\,с\,т\,в\,о\,.\ \ 
Обозначим через~$h_n(t)$ характеристическую функцию случайной величины 
$\sqrt{n}{(Z_n- a)}/{\sigma}$, а~через~$f_n(t)$~--- характеристическую функцию 
случайной величины $\sqrt{n}{\lambda(Z_n- a)}/{\sigma}$. Для
 произвольного $t\hm\in\mathsf{R}$ и~$M\hm>0$
\begin{multline*}
\abs{f_n(t)-f(t)}={}\\
{}=\abs{\int\limits_{0}^{\infty}h_n(ut)\,dA(u)-
\int\limits_{0}^{\infty}e^{-u^2t^2/2}\,dA(u)}\leq{}\\
{}\leq\int\limits_{0}^{\infty} \abs{h_n(ut)-e^{-u^2t^2/2}}\,dA(u)={}\\
{}=\int\limits_{0}^{M} \abs{h_n(ut)-e^{-u^2t^2/2}}\,dA(u)+{}\\
{}+
\int\limits_{M}^{\infty} \abs{h_n(ut)-e^{-u^2t^2/2}}\,dA(u)\equiv I_1+I_2\,.
\end{multline*}
Далее
$$
I_2\leq2\int\limits_{M}^{\infty}\,dA(u)\,,
$$
и для произвольного $\varepsilon\hm>0$ можно выбрать такое $M\hm=M(\varepsilon)$, 
что $I_2\hm<\varepsilon/2$. Поскольку в~силу цент\-раль\-ной предельной 
теоремы $h_n(v)\hm\to e^{-v^2/2}$ равномерно на любом конечном отрезке, найдется 
$N\hm=N(\varepsilon)$ такое, что $\abs{h_n(ut)-e^{-u^2t^2/2}}\hm<\varepsilon/2$ 
для всех $n\hm>N$ и~всех $u\hm\in[0,M]$. Следовательно, $I_1\hm<\varepsilon/2$ при 
$n\hm>N$ и,~таким образом, $\abs{f_n(t)-f(t)}\hm<\varepsilon$. Так как функция 
распределения $F(x)$ непрерывна, выполнено~(\ref{lim_1}).

Теорема доказана.

\smallskip

\noindent
\textbf{Замечание~1.}\ В~теореме $\lambda\sqrt{n} Z_n$ центрируется 
случайной величиной $a\lambda\sqrt{n}$. 
Для построения доверительных интервалов было бы желательно центрировать 
$\sqrt{n}\lambda Z_n$ некоторой неслучайной возрастающей последовательностью
 чисел~$a_n$. Однако сделать этого нельзя.

\smallskip

Действительно, предположим, что такая последовательность существует. Тогда
$$
\sqrt{n}\,\lambda Z_n-a_n=\sqrt{n}\,\lambda\left(Z_n- a\right)+
\left(\sqrt{n}\,\lambda a-a_n\right)\,.
$$
По доказанной теореме первое слагаемое имеет некоторое 
предельное распределение. Рассмотрим второе слагаемое. 
Если $a_n\hm=o(\sqrt{n})$ или наоборот $\sqrt{n}\hm=o(a_n)$, то 
очевидно, что ${\sf P}(\abs{\lambda a\sqrt{n}- a_n}\hm>A)\hm\to1$ при 
$n\hm\to\infty$ для любого $A\hm>0$. Если же $a_n\hm=O(\sqrt{n})$, т.\,е.\
 существуют такие положительные числа~$c$ и~$C$, что $c\sqrt{n}\hm<a_n\hm<C\sqrt{n}$ 
 для всех $n\hm>1$, то $\sqrt{n}\left(\lambda a-C\right)\hm<\sqrt{n}\,\lambda a
 \hm-a_n\hm<\sqrt{n}\left(\lambda a\hm-c\right)$ п.в.\ 
 и~${\sf P}(\abs{\sqrt{n}(\lambda a-C)}\hm>A)\hm\to1$, 
 ${\sf P}(\abs{\sqrt{n}(\lambda a-c)}\hm>A)\hm\to1$ при $n\hm\to\infty$, т.\,е.\
 снова 
${\sf P}(\abs{\lambda a\sqrt{n}- a_n}\hm>A)\hm\to1$. Таким образом, 
предельного распределения у~последовательности $\sqrt{n}\,\lambda Z_n\hm-a_n$ 
не существует.

\smallskip

\noindent
\textbf{Замечание 2.} Теорема дает возможность исследовать асимптотическое 
поведение статистики, основанной на выборке $\lambda/Y_1,\ldots,\lambda/Y_n$, 
в~стендовых испытаниях заказчика. Очевидно, что теорема останется верна при 
замене случайной величины~$\lambda$ на~$\mu^{-1}$, а~случайных величин~$Y_i^{-1}$ 
на~$X_i$, что соответствует стендовым испытаниям производителя.

\smallskip

\noindent
\textbf{Замечание 3.} Как видно, предельное распределение для испытаний 
заказчика однозначно определяется априорным распределением параметра 
входящего потока~$\lambda$ ($\mu^{-1}$ для испытаний производителя), 
что дает возможность применять результаты теоремы для широкого класса 
распределений с~конечными первыми моментами случайных величин~$Y_i^{-1}$ 
($X_i$~соответственно).

\smallskip

Для получения теоретических прогнозов можно использовать исследованные 
ранее вероятностные характеристики для некоторых априорных распределений~$Y_i$, 
например вейбулловского~\cite{KuTi16}. Однако
математическое ожидание~$Y_i^{-1}$ может не существовать~\cite{KuSh07}, например 
для показательного распределения параметра обслуживания.

{\small\frenchspacing
 {%\baselineskip=10.8pt
 \addcontentsline{toc}{section}{References}
 \begin{thebibliography}{9}



\bibitem{CSMIJ}
\Au{Sutton Ch., Jordan M.\,I.}
Bayesian inference for queueing networks and modeling of internet services~// 
Ann. Appl. Stat., 2011. Vol.~5. No.\,1. P.~254--282.

\bibitem{KuSh2015}
\Au{Кудрявцев А.\,А., Шоргин~С.\,Я.\/}
Байесовские модели в~тео\-рии массового обслуживания и~надежности.~--- 
М.: ФИЦ ИУ РАН, 2015. 76~с.

\bibitem{KuTi16}
\Au{Кудрявцев А.\,А., Титова~А.\,И.}
Байесовские модели массового обслуживания и~надежности: 
вы\-рож\-ден\-но-вей\-бул\-лов\-ский случай~// 
Информатика и~её применения, 2016. Т.~10. Вып.~4. С.~68--71.

\bibitem{KuSh07}
\Au{Кудрявцев А.\,А., Шоргин~С.\,Я.}
Байесовский подход к~анализу сис\-тем массового обслуживания и~показателей надежности~// 
Информатика и~её применения, 2007. Т.~1. Вып.~2. С.~76--82.
 \end{thebibliography}

 }
 }

\end{multicols}

\vspace*{-6pt}

\hfill{\small\textit{Поступила в~редакцию 12.06.17}}

\vspace*{8pt}

%\newpage

%\vspace*{-24pt}

\hrule

\vspace*{2pt}

\hrule

%\vspace*{8pt}


\def\tit{BAYESIAN MODELS FOR TESTING LARGE GROUPS OF~SERVICE~DEVICES}

\def\titkol{Bayesian models for~testing large groups of~service devices}

\def\aut{A.\,A.~Kudryavtsev$^1$ and O.\,V.~Shestakov$^{1,2}$}

\def\autkol{A.\,A.~Kudryavtsev and O.\,V.~Shestakov}

\titel{\tit}{\aut}{\autkol}{\titkol}

\vspace*{-9pt}

\noindent
$^1$Department of Mathematical Statistics, Faculty of Computational 
Mathematics and Cybernetics,\linebreak
$\hphantom{^1}$M.\,V.~Lomonosov Moscow State University, 
1-52~Leninskiye Gory, GSP-1, Moscow 119991, Russian\linebreak
$\hphantom{^1}$Federation

\noindent
$^2$Institute of Informatics Problems, Federal Research Center 
``Computer Science and Control'' of the Russian\linebreak
$\hphantom{^1}$Academy of Sciences, 
44-2~Vavilov Str., Moscow 119333, Russian Federation



\def\leftfootline{\small{\textbf{\thepage}
\hfill INFORMATIKA I EE PRIMENENIYA~--- INFORMATICS AND
APPLICATIONS\ \ \ 2018\ \ \ volume~12\ \ \ issue\ 1}
}%
 \def\rightfootline{\small{INFORMATIKA I EE PRIMENENIYA~---
INFORMATICS AND APPLICATIONS\ \ \ 2018\ \ \ volume~12\ \ \ issue\ 1
\hfill \textbf{\thepage}}}

\vspace*{3pt}



\Abste{The paper considers a new approach to the verification tasks for 
the declared performance indicators of large groups of complex servicing 
devices when the total verification is impossible or expensive. 
The method used is\linebreak\vspace*{-12pt}}

\Abstend{based on the study of the asymptotics of 
statistics in the framework of the Bayesian approach to the problems 
of the queuing theory. By the example of the readiness coefficient 
in the $M|M|1$ model, the authors prove the proposition relating to the class 
of transfer theorems which makes it possible to compare real data in bench 
tests of complex aggregates with theoretical predictions. The specifics 
of the application of the method from the point of view of a~device 
manufacturer and a~customer of a large batch of similar devices are considered. 
The described method and results can find application not only in the field of 
Bayesian queuing problems but also in other areas that use randomization 
of the basic parameters of the model.}

\KWE{Bayesian approach; mass service systems; mixed distributions; 
transfer theorem; statistical methods}



\DOI{10.14357/19922264180113} 

%\vspace*{-12pt}

\Ack
\noindent
The work was partly supported by the Russian Foundation for Basic 
Research (project 17-07-00577).



%\vspace*{3pt}

  \begin{multicols}{2}

\renewcommand{\bibname}{\protect\rmfamily References}
%\renewcommand{\bibname}{\large\protect\rm References}

{\small\frenchspacing
 {%\baselineskip=10.8pt
 \addcontentsline{toc}{section}{References}
 \begin{thebibliography}{9}


\bibitem{1-kud-1}
\Aue{Sutton, Ch., and M.\,I.~Jordan.} 
2011. Bayesian inference for queueing networks and modeling of internet services. 
\textit{Ann. Appl. Stat.} 5(1):254--282.

\bibitem{2-kud-1}
\Aue{Kudryavtsev, A.\,A., and S.\,Ya.~Shorgin.} 2015. \textit{Bayesovskie modeli 
v~teorii massovogo obsluzhivaniya i~nadezhnosti} 
[Bayesian models in mass service and reliability theories]. Moscow: FIC IU RAN. 76~p.

\bibitem{3-kud-1}
\Aue{Kudryavtsev, A.\,A., and A.\,I.~Titova.} 2016. Bayesovskie modeli massovogo 
obsluzhivaniya i~nadezhnosti: vyrozhdenno-veybullovskiy sluchay  
[Bayesian queuing and reliability models: Degenerate-Weibull case]. 
\textit{Informatika i~ee Primeneniya~--- Inform. Appl.} 10(4):68--71.

\bibitem{4-kud-1}
\Aue{Kudryavtsev, A.\,A., and S.\,Ya.~Shorgin.} 2007. Bayesovskiy podkhod 
k~analizu sistem massovogo obsluzhivaniya i~pokazateley nadezhnosti 
[Bayesian approach to queueing systems and reliability characteristics]. 
\textit{Informatika i~ee Primeneniya~--- Inform. Appl.} 1(2):76--82.
\end{thebibliography}

 }
 }

\end{multicols}

\vspace*{-6pt}

\hfill{\small\textit{Received June 12, 2017}}

%\vspace*{-10pt}


\Contr

\noindent
\textbf{Kudryavtsev Alexey A.} (b.\ 1978)~--- 
Candidate of Science (PhD) in physics and mathematics, associate professor, 
Department of Mathematical Statistics, Faculty of Computational Mathematics 
and Cybernetics, M.\,V.~Lomonosov Moscow State University, 1-52~Leninskiye Gory, 
GSP-1, Moscow 119991, Russian Federation; \mbox{nubigena@mail.ru}

\vspace*{3pt}

\noindent
\textbf{Shestakov Oleg V.} (b.\ 1976)~--- 
Doctor of Science in physics and mathematics, associate professor, 
Department of Mathematical Statistics, Faculty of Computational Mathematics 
and Cybernetics, M.\,V.~Lomonosov Moscow State University, 1-52~Leninskiye Gory, 
GSP-1, Moscow 119991, Russian Federation; senior scientist, 
Institute of Informatics Problems, Federal Research Center ``Computer Science 
and Control'' of the Russian Academy of Sciences, 44-2~Vavilov Str., Moscow 119333, 
Russian Federation; \mbox{oshestakov@cs.msu.su}
\label{end\stat}


\renewcommand{\bibname}{\protect\rm Литература} 