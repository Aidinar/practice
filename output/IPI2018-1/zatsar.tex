\def\stat{zatsar}

\def\tit{СИСТЕМА СИТУАЦИОННОГО УПРАВЛЕНИЯ КАК~МУЛЬТИСЕРВИСНАЯ 
ТЕХНОЛОГИЯ В~ОБЛАЧНОЙ СРЕДЕ}

\def\titkol{Система ситуационного управления как мультисервисная 
технология в~облачной среде}

\def\aut{А.\,А.~Зацаринный$^1$, А.\,П.~Сучков$^2$}

\def\autkol{А.\,А.~Зацаринный, А.\,П.~Сучков}

\titel{\tit}{\aut}{\autkol}{\titkol}

\index{Зацаринный А.\,А.}
\index{Сучков А.\,П.}
\index{Zatsarinny A.\,A.}
\index{Suchkov A.\,P.}




%{\renewcommand{\thefootnote}{\fnsymbol{footnote}} \footnotetext[1]
%{Работа выполнена при финансовой поддержке РФФИ (проект 17-01-00816).}}


\renewcommand{\thefootnote}{\arabic{footnote}}
\footnotetext[1]{Институт проблем информатики Федерального исследовательского центра <<Информатика 
и~управ\-ле\-ние>> Российской академии наук, \mbox{AZatsarinny@ipiran.ru}}
\footnotetext[2]{Институт проблем информатики Федерального исследовательского центра <<Информатика 
и~управ\-ле\-ние>> Российской академии наук, \mbox{ASuchkov@ipiran.ru}}

\vspace*{6pt}
    

      \Abst{Рассматриваются системотехнические подходы к~созданию 
унифицированной системы управ\-ле\-ния в~среде облачных технологий в~виде совокупности 
сервисов. Предложенный подход на основе процессной пятистадийной модели позволяет 
определить общий вид (прототип) унифицированной  системы ситуационного управления с~учетом всех видов внешних и~внутренних информационных взаимодействий в~составе 
иерархической системы управления. Показано, что унифицированная система должна 
обладать средствами настройки (локализации) на конкретную об\-ласть применения. 
Предложен перечень сервисов локализации и~сервисов обеспечения основных функций 
сис\-те\-мы управ\-ле\-ния. Ожидается, что такие сервисы будут  востребованы в~широком круге 
организационных сис\-тем, что должно привести  к~существенному уменьшению затрат на 
разработку и~внедрение, обеспечить возможность реализации единой информационной, 
программной и~технической политики в~об\-ласти межведомственного 
и~внутриведомственного электронного взаимодействия.}
     
     \KW{унифицированная система ситуационного управления; облачный сервис; 
процессы локализации}

\DOI{10.14357/19922264180110} 
  
\vspace*{6pt}


\vskip 10pt plus 9pt minus 6pt

\thispagestyle{headings}

\begin{multicols}{2}

\label{st\stat}


\section{Введение}

    В настоящее время подавляющее большинство организационных сис\-тем 
(ведомства, корпорации, предприятия в~государственном секторе, 
биз\-нес-ком\-па\-нии, банковский и~торговый секторы экономики и~т.\,п.)\ при реализации 
стратегических целей связывают свое развитие с~созданием и~эффективным 
применением информационных систем и~технологий.  При этом требования 
к~ним сегодня неизмеримо возросли: речь идет не о~компьютеризации\linebreak 
отдельных служб или компании в~целом; на первый план выходят системы 
с~технологиями, реа\-ли\-зу\-ющи\-ми функции поддержки принятия 
управ\-ля\-ющих решений как в~рамках текущих задач организаци\-он\-ной 
системы, так и~задач среднесрочного и~перспективного планирования на 
основе прогнозирования. Более того, возросла актуальность задач 
информационного взаимодействия различных организационных сис\-тем 
между собой на основе единых регламентов взаимодействия. Такие задачи 
поставлены в~рамках  создания Сис\-те\-мы распределенных ситуационных 
центров~[1].  
    
    Это определяет тенденцию возрастания спроса на разработку 
и~внедрение автоматизированных систем управления для широкого спектра 
организаций. Очевидно, что затраты на разработку существенно снижаются 
при использовании доступных готовых типовых решений 
(инфраструктурных и~программных), настраиваемых на конкретные 
приложения.

\begin{figure*}[b] %fig1
\vspace*{11pt}
 \begin{center}
 \mbox{%
 \epsfxsize=156.428mm 
 \epsfbox{zac-1.eps}
 }
 \end{center}
\vspace*{-9pt}
\Caption{Функциональная модель ситуационной системы управления}
\end{figure*}
    
    Предпосылками для создании системы управ\-ле\-ния как общедоступной 
мультисервисной технологии в~облачной среде являются существующие 
модели обслуживания IaaS (Infrastructure as a~Service~---
инфраструктура как услуга)
и~DaaS (Desktop as 
a~Service), позволяющие реализовать  определенную 
инфраструктуру (серверы, рабочие места, хранилища, базы данных (БД)) 
и~специализированные рабочие места с~прикладным программным 
обеспечением, реализующие необходимый функционал. 

Услуга IaaS пред\-остав\-ля\-ет 
возможность конструировать ин\-фор\-ма\-ци\-он\-но-тех\-но\-ло\-ги\-че\-скую 
ин\-фра\-струк\-ту\-ру и~управ\-лять ею. 
Самые известные IaaS-решения: Amazon CloudFormation~[2], Google Compute 
Engine~[3], Windows Azure~[4]. 

При предоставлении услуги DaaS 
клиенты получают полностью готовые к~работе (под ключ) 
специализированные виртуальные рабочие места, которые каждый 
пользователь имеет возможность дополнительно настраивать под свои задачи 
и~получать доступ не к~отдельной программе, а~к~необходимому для 
полноценной работы программному комплексу. Такой подход используется 
в~Amazon WorkSpaces~[5] и~Oracle Virtual Desktop Infrastructure~[6].
    
    Реализацию мультисервисной технологии можно осуществлять 
в~гибридном облаке, со\-че\-та\-ющем частные (для организаций) 
и~общественные (для иерархической сис\-те\-мы управ\-ле\-ния) реализации.
{ %\looseness=1

}
    
    Специализированный облачный сервис в~виде системы управления 
можно назвать CSaaS (Control System as a~Service). Некоторый 
положительный опыт успешного применения облачных сис\-тем управ\-ле\-ния 
уже накоплен в~об\-ласти розничной, оптовой  
и~ин\-тер\-нет-тор\-гов\-ли, гостиничного бизнеса, медицинских услуг 
(см., например,~[7]).
    
    В статье рассматриваются системотехнические подходы к~созданию 
унифицированной системы управления в~среде облачных технологий в~виде\linebreak 
совокупности сервисов. Ожидается, что такие сервисы будут  востребованы 
в~широком круге организационных систем, что должно привести  
к~существенному уменьшению затрат на разработку и~внедрение, обеспечить 
возможность реализации единой информационной, программной 
и~технической политики в~об\-ласти межведомственного 
и~внутриведомственного электронного взаимодействия.
{\looseness=1

}



    
\section{Процессная модель унифицированной системы 
управления}

\vspace*{-8pt}

    В~[8] обоснована процессная пятистадийная модель обобщенной 
системы управления, которая апробирована в~ряде приложений~[9]. Суть 
данной модели со\-сто\-ит в~представлении сис\-те\-мы управ\-ле\-ния в~виде пяти 
групп процессов, вза\-и\-мо\-дей\-ст\-ву\-ющих с~внешней средой и~между собой: 
целеполагание, мониторинг, анализ, решение и~действие. При этом, как 
показано на рис.~1,  выход каждого процесса является входом другого, что 
обеспечивает непрерывность управ\-ле\-ния. 
    
    Первая группа процессов обслуживает такое свойство системы 
управления, как целесообразность, и~позволяет поддерживать 
формализованное представление целевых показателей (ЦП), устанавливать 
количественные и~временн$\acute{\mbox{ы}}$е критерии их достижения (целеполагание 
и~планирование). Сформированная сис\-те\-ма целей определяет со\-во\-куп\-ность 
па\-ра\-мет\-ров объектов мониторинга, подлежащих постоянному наблюдению, 
а~также формы отчетности и~других служебных документов. 
    
    Взаимодействие системы управления с~внешней средой обслуживает ряд 
процессов сбора информации о~со\-сто\-янии контролируемых объектов, 
включая со\-сто\-яние ЦП, взаимодействие с~сопряженными сис\-те\-ма\-ми, а~также 
сбор и~анализ общедоступных данных по тематике предметной об\-ласти. 
Состав данных мониторинга может быть достаточно разнообразен~--- 
сигналы сенсоров,\linebreak отчеты и~донесения нижестоящих и~вза\-имо\-действующих 
сис\-тем управ\-ле\-ния, инструкции вышесто\-ящих органов, разнородные данные 
общедоступных источников. В~процессе мониторинга фиксируются 
события~--- изменения со\-сто\-яния контролируемых объектов внешней среды.
    

    
    Третья группа процессов, получая на входе событийную информацию, 
осуществляет анализ и~распознавание скла\-ды\-ва\-ющих\-ся 
про\-стран\-ст\-вен\-но-вре\-мен\-н$\acute{\mbox{ы}}$х конфигураций событий (ситуации), имеется в~виду пространство 
взаимосвязанных объектов мониторинга, изменяющихся во времени. При 
этом выявляются значимые ситуации, тре\-бу\-ющие быст\-ро\-го реагирования, 
а~также определяется тип ситуации (штатная, нештатная), анализируется 
су\-ще\-ст\-ву\-ющий опыт нормализации ситуаций и~вырабатываются 
альтернативы решений по управ\-ле\-нию. Важнейшим компонентом третьей 
группы является анализ  текущего со\-сто\-яния  ЦП  сис\-те\-мы управ\-ле\-ния 
(планов).


    На следующей стадии осуществляется поддержка процессов принятия 
решений, в~общем случае это процесс многокритериального выбора (МКВ) среди 
альтернативных вариантов нормализации ситуации с~учетом различных 
ресурсных ограничений и~результатов сценарного прогнозирования 
по\-след\-ст\-вий возможных управ\-ля\-ющих воздействий. В~случае штатной 
ситуации осуществляется адап\-та\-ция типовых решений к~скла\-ды\-ва\-ющей\-ся 
обстановке.
    
    Пятая стадия реализует процессы выполнения выработанного решения 
путем применения управ\-ля\-ющих воздействий, исходя из име\-ющих\-ся 
ресурсов, а~также контроль исполнения (путем установления динамических 
целей для осуществления контроля исполнения решения).
    

\section{Общий вид (прототип) унифицированной  системы 
ситуационного управления}

    Предлагаемая функциональная модель системы ситуационного 
управления может быть основой для описания унифицированной 
мультисервисной технологии высокой доступности. Для формирования 
полного облика такой технологии необходимо также учесть процессы 
взаимодействия в~иерархической многоуровневой сис\-те\-ме управ\-ле\-ния, так 
как обычно любая сис\-те\-ма управ\-ле\-ния может обладать как внутренними 
под\-сис\-те\-ма\-ми управ\-ле\-ния, например подразделениями, так и~внешними 
под\-сис\-те\-ма\-ми, например территориально распределенными сис\-те\-ма\-ми 
управления.
    
     Информационные взаимодействия, происходящие в~иерархической 
многоуровневой системе\linebreak ситуационного управления, подразделяются на 
внешние и~внутренние (рис.~2). Внешние взаимодействия с~окружающей 
средой происходят с~по\-мощью сенсорных под\-сис\-тем, которые могут 
осуществлять прием сигналов, донесений, различных видов 
структурированных и~неструктурированных данных, а~также инструкций от 
вышестоящих органов, осу\-ще\-ст\-вля\-ющих целеполагание и~другое 
регулирование деятельности сис\-те\-мы управ\-ле\-ния на основе  
нор\-ма\-тив\-но-пра\-во\-вых документов.
    
    В структуре информационных взаимодействий внутри иерархической 
сис\-те\-мы управ\-ле\-ния можно выделить три вида данных: команды, отчеты 
и~сведения.
    
    Команды представляют собой управляющие воздействия в~виде, как 
правило, формализованных\linebreak сообщений (сигналов, целеуказаний, 
распоряжений, инструкций),  передаваемых по иерархии подчиненности 
<<сверху вниз>>~---  от вышестоящих\linebreak к~нижестоящим под\-сис\-те\-мам. 
С~по\-мощью команд\linebreak
 выполняются действия по исполнению  принятых 
решений, включая уста\-нов\-ле\-ние новых целей, определение сроков, 
ответственных исполнителей и~выделение требуемых ресурсов. Такой 
функционал обычно реа\-ли\-зу\-ет\-ся средствами формализованного 
электронного документооборота (ФЭД).


    Отчеты передаются от нижестоящих к~вышестоящим подсистемам 
    в~виде формализованных донесений о~реализации поставленных целей, 
которые находятся на контроле вышестоящего органа управления. Сюда 
могут включаться доклады о~готовности и~работоспособности, а~также 
результаты деятельности по реализации решений.
    
    Сведения представляют собой формализованные и~неформализованные 
сообщения, пе\-ре\-да\-ва\-емые с~целью информирования и~координации 
деятельности взаимодействующих под\-сис\-тем, а~также обмена данными 
о~результатах  аналитики (прогнозирование, идентификация ситуаций, 
выявленные тренды, аномалии, артефакты). В~[10] рас\-смот\-ре\-ны  вопросы 
организации обмена аналитическими данными различных видов 
и~обосно\-ва\-ны способы их формализации.
    
    Таким образом, предложенный подход на основе процессной 
пятистадийной модели позволяет определить общий вид (прототип) 
унифицированной  сис\-те\-мы ситуационного управ\-ле\-ния с~учетом\linebreak\vspace*{-12pt}

\pagebreak

\end{multicols}

\begin{figure*} %fig2
\vspace*{1pt}
 \begin{center}
 \mbox{%
 \epsfxsize=161.434mm 
 \epsfbox{zac-2.eps}
 }
 \end{center}
\vspace*{-8pt}
\Caption{Взаимодействие подсистем иерархической трехуровневой 
сис\-те\-мы управ\-ле\-ния}
\end{figure*}
    

\begin{multicols}{2}

\noindent
 всех видов 
внешних и~внут\-рен\-них информационных взаимодействий в~со\-ста\-ве 
иерархической сис\-те\-мы управ\-ле\-ния (см.\ рис.~2).
    
\section{Прикладные вопросы локализации прототипа}

    Унифицированная сис\-те\-ма ситуационного управ\-ле\-ния высокой 
доступности в~облачной среде должна обладать средствами настройки 
(локализации) на конкретную об\-ласть применения. Следовательно, в~ее 
структуре должны быть универсальные (неизменные для любого приложения 
или автоматически настраиваемые) компоненты и~специфические 
(на\-стра\-и\-ва\-емые) компоненты. 
    
    К настраиваемым компонентам относятся: структура БД, 
инструментальные программные средства анализа неструктурированных 
и~структурированных данных, программные средства адаптеров входных 
данных и~прочие средства специального программного обеспечения (СПО), 
ФЭД. 



Все эти 
компоненты опираются на проб\-лем\-но-ори\-ен\-ти\-ро\-ван\-ную модель данных, 
включающую в~себя описание сущностей предметной об\-ласти и~их 
взаимосвязей, ин\-фор\-ма\-ци\-он\-но-линг\-ви\-сти\-че\-ское обеспечение
(ИЛО)  сис\-те\-мы, 
формы служебных документов. Информационная модель формируется 
в~процессе первоначальной локализации сис\-те\-мы в~основном по данным 
стадии целеполагания, на которой определяются ЦП сис\-те\-мы управ\-ле\-ния 
и~связанные с~ними объекты мони-\linebreak\vspace*{-12pt}

\pagebreak

\end{multicols}

\begin{figure*} %fig3
\vspace*{1pt}
 \begin{center}
 \mbox{%
 \epsfxsize=165.77mm 
 \epsfbox{zac-3.eps}
 }
 \end{center}
\vspace*{-6pt}
\Caption{Процессы настройки унифицированной сис\-те\-мы ситуационного управ\-ле\-ния}
\vspace*{9pt}
\end{figure*}

\begin{multicols}{2}

\noindent
торинга, формы отчетности и~другие 
служебные до\-ку\-менты.
 
    
    Настройка унифицированной сис\-те\-мы, а~также ее модификация должны 
осуществляться с~по\-мощью ряда процессов, позволяющих сформировать 
информационную модель предметной об\-ласти и~на основе адаптированной 
модели данных осуществить локализацию сис\-те\-мы в~облачной среде (рис.~3).
    
    В~\cite{8-zac} систематизированы \textbf{основные} процессы 
управления: целеполагание, мониторинг, анализ данных мониторинга, 
поддержка процессов принятия решений в~сис\-те\-ме управ\-ле\-ния, реализация 
принятых решений, \textbf{вспомогательные} внут\-рен\-ние процессы 
управ\-ле\-ния, а~также \textbf{обеспечивающие} процессы технологического 
управ\-ле\-ния, реа\-ли\-зу\-ющие функции поддержания инфраструктуры: 
управ\-ле\-ние информационными ресурсами, организация сетевого 
взаимодействия, обеспечение информационной без\-опас\-ности и~управ\-ле\-ние 
доступом к~ресурсам. 

Все эти процессы подлежат определенной настройке, 
в~статье будут рас\-смот\-ре\-ны процессы локализации основных 
и~вспомогательных процессов управ\-ле\-ния.
  \begin{enumerate}[1.]  
\item 
Процесс автоматизированной настройки модели данных в~формате 
XML или JSON. 
    
    К сожалению, в~настоящее время большинство разработок в~об\-ласти 
при\-клад\-ной информатики в~нашей стране сопровождается со\-зда\-ни\-ем все 
новых и~новых информационных мо\-де\-лей. 
{\looseness=1

}

Общегосударственной сис\-те\-мы 
обмена дан\-ны\-ми не существует, такие проекты, как, на\-пример, СМЭВ
(Система межведомственного электронного взаимодействия)~[11], 
не подкрепленные едиными моделями данных и~документов и~сис\-те\-ма\-ми их 
ведения, столкнулись с~большими трудностями по объемам 
перепрограммирования. Вместе с~тем  имеется положительный мировой опыт 
создания подобных моделей, например NIEM  (National Information Exchange 
Model)~--- Национальная информационная XML-мо\-дель обмена данными 
в~государственных органах США~\cite{12-zac}, в~рамках которой разработаны 
и~ведутся XML-мо\-де\-ли многих на\-прав\-ле\-ний государственной 
дея\-тель\-ности, организованные в~виде доменов, например 
правоохранительная и~социальная сфера, транспорт, без\-опас\-ность и~др.
    

    
    Используя такую модель, можно было бы собирать конкретные 
прикладные модели, как из <<кубиков>>. В~данном же случае необходима 
кропотливая работа прикладных аналитиков и~использование име\-ющей\-ся 
мозаики конкретных разработок и~час\-тич\-но сформированного 
ИЛО государственных, 
ведомственных и~корпоративных классификаторов. 

С~другой стороны, 
осуществление таких работ по единым технологиям может позволить 
по\-сле\-до\-ва\-тель\-но осуществлять сборку единой модели. С~этой целью 
необходимо предусмот\-реть сервисы формирования и~ведения такой модели.
    
    Основными разделами такой модели, в~общем случае пред\-став\-ля\-ющей 
собой семантическую сеть, могут являться:
    \begin{itemize}
\item классы сущностей, позволяющие формировать иерархические 
структуры (обес\-пе\-чи\-ва\-ющие наследование свойств в~иерархии);\\[-10pt]
\item сущности:\\[-10pt]
\begin{itemize}
\item внешние объекты мониторинга (объекты, задаваемые целями 
управ\-ле\-ния);\\[-10pt]
\item внутренние объекты мониторинга (свои силы, материальные, 
технические, финансовые и~другие средства);\\[-10pt]
\item противодействующие ресурсы (силы, средства);\\[-10pt]
\item элементы окружающей среды (природные, техногенные, социальные, 
политические и~экономические факторы контролируемого пространства):\\[-10pt]
\begin{itemize}
\item ИЛО (классификаторы, справочники, словари терминов);\\[-10pt]
\item система служебной документации.\\[-10pt]
\end{itemize}
\end{itemize}
\end{itemize}

   \item Автоматическая настройка БД~--- формирование таблиц данных, 
входных и~выходных форм при первичной локализации может 
осуществляться в~автоматическом режиме с~использованием специальных 
программных процедур и~стандартных средств сис\-тем управ\-ле\-ния БД. Сложнее обстоит дело 
с~корректировкой су\-ще\-ст\-ву\-ющей структуры данных, где уже требуется 
применение нестандартных информационных технологий и~за\-час\-тую 
вмешательство оператора. Это один из центральных вопросов поддержания 
жизнеспособности сис\-те\-мы и,~в~свою очередь, один из слож\-ней\-ших 
прикладных вопросов информатики.

\columnbreak
    
    \item Автоматизированная настройка баз знаний и~СПО, реа\-ли\-зу\-юще\-го 
методы анализа неструктурированных данных, извлечение фактов и~знаний.\\[-8pt]

    
    Основу процесса извлечения структурированной информации из 
неструктурированного текс\-та со\-став\-ля\-ет определение и~идентификация 
сущностей из текста на естественном языке\linebreak и~вы\-яв\-ле\-ние связей между этими 
сущностями. Данная проб\-ле\-ма актуальна на стадии мо-\linebreak ниторинга данных из 
различных источников (СМИ, 
новостные ленты, социальные сети)\linebreak  
и~позволяет решить ряд прикладных задач в~интересах сис\-те\-мы управ\-ле\-ния: 
выявление новых сведений об объектах мониторинга, оценка социальных, 
политических и~других последствий при\-ни\-ма\-емых решений, выявление 
тенденций и~аномалий в~развитии обстановки. Обычно такие методы 
анализа опираются на формализованные в~том или ином виде знания 
о~пред\-мет\-ной об\-ласти и~специализированные методы обработки 
информации, реализованные в~виде наукоемкого про\-грам\-мно\-го 
обеспечения~\cite{13-zac, 14-zac}. Как правило, такие про\-грам\-мные 
продукты обладают инструментальными средствами настройки на 
конкретное при\-ло\-жение.
{\looseness=1

}

\vspace*{6pt}
    
  
  \item Автоматизированная настройка СПО, реа\-ли\-зу\-юще\-го методы 
статистического, дискретного и~чис\-лен\-но\-го анализа.\\[-8pt]
    
    Целью анализа складывающейся обстановки является выявление 
штатных и~нештатных ситуаций, декомпозиция нештатных ситуаций 
(упрощение до воз\-мож\-ности применения типовых решений), оценка 
воз\-мож\-ности применения типовых решений, выработка альтернатив 
решений.
    
    Так как информационная модель контроли\-руемого пространства 
формируется в~виде семантической сети и~отражает со\-сто\-яние 
конт\-ро\-ли\-ру\-емых объектов и~их взаимосвязей, для анализа данных 
применяются методы дискретной математики, связанные с~тео\-ри\-ей графов, 
математической логикой и~лингвистическим анализом:
    \begin{itemize}
\item идентификация и~регистрация объектов, сли\-яние подсетей;\\[-10pt]
\item поиск подобных про\-стран\-ст\-вен\-но-вре\-мен\-н$\acute{\mbox{ы}}$х конфигураций 
методами теории графов (изоморфизм и~изоморфное вложение 
графов), декомпозиция конфигураций;\\[-10pt]
\item логические выводы (поиск решения) на семантической сети;\\[-10pt]
\item поиск прямых и~ассоциативных связей (путей на графе);\\[-10pt]
\item расчет интегральных показателей и~ЦП на графах.
\end{itemize}

    Декомпозиция ситуаций на более простые составляющие может 
позволить сформировать множество альтернативных решений. Например, 
если имеется несколько ЦП, то задача максимизации одного ЦП при 
фиксированных остальных может быть отнесена к~типовой и~ре\-ша\-емой. Как 
правило, таких альтернатив может быть бесконечное чис\-ло (образуя так 
на\-зы\-ва\-емое множество Парето) и~стоит вопрос о~выборе одного решения, 
который должен быть решен на сле\-ду\-ющей стадии процесса управ\-ле\-ния.
    
    Специальное программное обеспечение, реализующее указанные методы анализа, настраивается на 
сформированные БД и~структуры данных, за\-да\-ва\-емые информационной 
моделью предметной об\-ласти.
   
   \item Автоматизированная настройка СПО, реа\-ли\-зу\-юще\-го методы 
МКВ и~сценарного прогнозирования.
    
    Как правило, для выбора альтернативы из множества вариантов решений 
в~условиях ограничения на ресурсы применяются технологии, 
обеспечивающие снижение раз\-мер\-ности признакового про\-стран\-ст\-ва 
и~по\-стро\-ение интегрального показателя качества. При этом агрегируются 
исходные признаки с~использованием знаний эксперта и/или пред\-по\-чте\-ний 
лица, при\-ни\-ма\-юще\-го решение (ЛПР)~[15]. Например, можно использовать 
алгоритм сужения множества Парето на основе нечеткой информации об 
отношениях предпочтения ЛПР~[16].
{\looseness=1

}
    
    Оценка качества выбранных альтернатив решений может 
осуществляться с~использова\-нием метода сценарного прогнозирования,\linebreak 
который позволяет учесть управляющие воздействия при прогнозировании 
развития обстановки. Для анализа обстановки и~поддержки процессов 
принятия решений применяются математические методы статистического 
анализа:
{ %\looseness=1

}

\noindent
    \begin{itemize}
\item анализ временн$\acute{\mbox{ы}}$х рядов, характеризующих изменения количественных 
и~качественных атрибутов узлов и~связей (анализ трендов, сезонных 
колебаний, тенденций и~аномалий);
\item прогнозирование изменения параметров с~учетом выявленных трендов 
и~анализируемых сценариев развития обстановки;
\item динамическое моделирование ситуаций;
\item статистическая оценка количественных и~качественных характеристик 
потоков событий.
\end{itemize}

    После принятия решения формируется динамическая цель, которая 
должна быть конкретной, количественно измеримой, достижимой,  
обеспеченной ресурсами и~привязанной к~точ\-ке/ин\-тер\-ва\-лу времени.
    
    Настройка СПО, реализующего указанные технологии, осуществляется 
также в~соответствии со структурой данных, за\-да\-ва\-емой информационной 
моделью предметной об\-ласти.
    
  \item  Настройка адаптеров. Администрирование ФЭД.
    
    Система управления должна обладать возможностями настройки на 
необходимые источники данных (технические средства, информационные 
сис\-те\-мы, СМИ, новостные сайты и~др.)\ с~по\-мощью сис\-те\-мы 
специализированных адап\-те\-ров, поз\-во\-ля\-ющих преобразовывать по\-лу\-ча\-емую 
информацию в~соответствии с~информационной моделью сис\-те\-мы.  
Получение\linebreak информации может осуществляться по регламенту, по запросу 
и~по событию, поэтому техно\-логии нормализации (преобразования 
в~тре\-бу-\linebreak емые форматы) данных и~запросов должны обладать воз\-мож\-ностью 
настройки пользовательской среды поддержки процесса интеграции, 
реализованной, например, в~виде средств создания персональных 
семантических цифровых биб\-лио\-тек как инструмента обработки 
и~интеграции данных~[17]. При отсутствии автоматизированного 
взаимодействия с~источником данных адаптеры должны обеспечивать режим 
ручного ввода информации.

\begin{figure*}[b] %fig4
%\vspace*{12pt}
 \begin{center}
 \mbox{%
 \epsfxsize=134.972mm 
 \epsfbox{zac-4.eps}
 }
 \end{center}
\vspace*{-9pt}
\Caption{Структура мультисервисной системы управ\-ле\-ния
(НСИ~--- нор\-ма\-тив\-но-спра\-воч\-ная информация)}
\end{figure*}
    
    Существуют методы виртуальной интеграции разнородных источников, 
основанные на XML-мо\-де\-ли данных, когда с~использованием 
декларативных языков локальные источники единообразно отоб\-ра\-жа\-ют\-ся на 
глобальную схему в~терминах XML (информационную модель). При этом 
в~сис\-те\-ме виртуальной интеграции решаются вопросы отображения схем, 
оптимизации, декомпозиции и~выполнения запросов~[18]. 
    
    Формализованный электронный документооборот предназначен для 
обмена структурированными данными внутри сис\-те\-мы управ\-ле\-ния 
(команды, доклады, сообщения, данные аналитики). Администрирование 
ФЭД подразумевает сис\-тем\-ные настройки комплекса, дизайн отчетов 
и~других служебных документов, со\-зда\-ние хранилищ на сервере БД, 
управ\-ле\-ние учетными записями и~правами доступа, настройки 
протоколирования и~мониторинга.
\end{enumerate}


\vspace*{-6pt}

\section{Унифицированная система ситуационного управления как 
мультисервисная технология}
    
    Совокупность процессов локализации и~функционирования 
унифицированной сис\-те\-мы ситуационного управ\-ле\-ния реа\-ли\-зу\-ет\-ся 
со\-во\-куп\-ностью взаимосвязанных облачных сервисов (рис.~4), 
поз\-во\-ля\-ющих как интерактивную, так и~их событийную активацию (рис.~5).


    
    Сервисы, поддерживающие единую XML-мо\-дель, целесообразно 
реализовать в~виде глобальной портальной технологии, поз\-во\-ля\-ющей 
формировать ин\-тер\-нет-со\-об\-щест\-ву рас\-смат\-ри\-ва\-емой пред\-мет\-ной 
об\-ласти единую информационную модель по образу NIEM.
    

\vspace*{-6pt}


\section{Использование унифицированной мультисервисной 
технологии для~создания предметно-ориентированной системы 
ситуационного управления (например, ситуационного центра)}

    Реализация мультисервисной технологии предполагает наличие 
у~провайдера типовой инфраструктуры сис\-те\-мы управ\-ле\-ния, 
пред\-став\-ля\-емой\linebreak\vspace*{-12pt}

\pagebreak

\end{multicols}

\begin{figure*} %fig5
\vspace*{1pt}
 \begin{center}
 \mbox{%
 \epsfxsize=126.858mm 
 \epsfbox{zac-5.eps}
 }
 \end{center}
\vspace*{-9pt}
\Caption{Интерактивная и~событийная активация сервисов:
\textit{1}~--- изменение состава целей;
\textit{2}~--- изменение состава объектов мониторинга;
\textit{3}~--- возникновение ситуации;
\textit{4}~--- формирование альтернатив решений;
\textit{5}~--- принятие решения}
\end{figure*}

\begin{multicols}{2}

  \noindent
   пользователю в~модели обслуживания IaaS, 
инструментального и~прикладного программного обес\-пе\-че\-ния, раз\-ме\-ща\-емо\-го 
на специализированных рабочих местах, доступ пользователя к~которым 
осущест\-вляется в~режиме <<тонкого клиента>>.  

    С точки зрения реализации функционала облачной сис\-те\-мы управ\-ле\-ния 
    в~рамках услуги DaaS целесообразно создание специализированных рабочих 
мест в~соответствии с~организационной структурой ситуационного 
центра~[19]:
    \begin{itemize}
\item сегмент руководства~--- ЛПР, реа\-ли\-зу\-ющий 
группу сервисов целеполагания и~решения; 
\item сегмент мониторинга состояния контролируемых объектов 
и~окружающей среды (оперативная дежурная служба), реа\-ли\-зу\-ющий группу 
сервисов мониторинга; 
\item сегмент ситуационного анализа и~систематизации информации, 
реа\-ли\-зу\-ющий группу сервисов анализа и~действия;
\item сегмент администрирования, реа\-ли\-зу\-ющий группу сервисов 
локализации.
\end{itemize}

    После создания и~оснащения инфраструктуры запускаются 
интерактивные процессы локализации: осуществляется настройка 
информационной модели предметной об\-ласти и~компонентов сис\-те\-мы под 
нужды заказчика (адаптеры, документооборот, аналитические приложения). 
Активация сис\-те\-мы осуществляется с~запуском сервисов мониторинга 
конт\-ро\-ли\-ру\-емо\-го пространства сис\-те\-мы управ\-ле\-ния.

%\vspace*{-12pt}
    
{\small\frenchspacing
 {\baselineskip=10.8pt
 \addcontentsline{toc}{section}{References}
 \begin{thebibliography}{99}

\bibitem{1-zac}
\Au{Зацаринный А.\,А.} О~повышении эффективности ин\-фор\-ма\-ци\-он\-но-ана\-ли\-ти\-че\-ской 
поддержки принятия стратегических решений в~органах государственной власти~// 
Межотраслевая информационная служба, 2015. №\,1(170). С.~11--22.
\bibitem{2-zac}
AWS CloudFormation. {\sf https://aws.amazon.com/ru/ cloudformation}.
\bibitem{3-zac}
Compute Engine: Scalable, High-Performance Virtual Machines. {\sf 
https://cloud.google.com/compute}. 
\bibitem{4-zac}
Глобальность. Надежность. Гибридность: Облако на ваших условиях. {\sf 
https://azure.microsoft.com/ru-ru}.
\bibitem{5-zac}
Amazon WorkSpaces: Полностью управляемые и~защищенные виртуальные облачные 
рабочие столы на AWS. {\sf https://aws.amazon.com/ru/workspaces.}
\bibitem{6-zac}
Oracle Virtual Desktop Infrastructure. {\sf  
http://www.\linebreak oracle.com/technetwork/server-storage/virtualdesktop/ overview/index.html.} 
\bibitem{7-zac}
Облачная система управления бизнесом. {\sf http:// cloud-automation.ru/management\_automation.}
\bibitem{8-zac}
\Au{Сучков А.\,П.} Полнофункциональный процессный подход к~реализации сис\-тем 
ситуационного управ\-ле\-ния~// Сис\-те\-мы и~средства информатики, 2017. Т.~27. №\,1.  
С.~85--99.
\bibitem{9-zac}
Федеральный исследовательский центр <<Информатика и~управ\-ле\-ние>> 
Российской академии наук: Основные на\-прав\-ле\-ния исследований. {\sf 
http://www. frccsc.ru/expertise.}
\bibitem{10-zac}
\Au{Сучков А.\,П.} Некоторые подходы к~интеграции аналитических данных 
су\-ще\-ст\-ву\-ющих и~перспективных сис\-тем поддержки принятия решений~// Системы 
и~средства информатики, 2015. Т.~25. №\,3. С.~201--211.
\bibitem{11-zac}
Электронное правительство: Госуслуги. {\sf http://smev. gosuslugi.ru/portal}.
\bibitem{12-zac}
About NIEM (National Information Exchange Model). {\sf https://www.niem.gov/about-niem}.
\bibitem{13-zac}
\Au{Осипов Г.\,С.} Приобретение знаний интеллектуальными сис\-те\-ма\-ми.~--- М.: Наука. 
1997. 112~с.
\bibitem{14-zac}
\Au{Быстров И.\,И., Тарасов~Б.\,В., Хорошилов~А.\,А., Радоманов~С.\,И.} Основы 
применения онтологии и~компьютерной лингвистики при проектировании перспективных 
автоматизированных информационных сис\-тем~// Сис\-те\-мы и~средства информатики, 
2015. Т.~25. №\,4. С.~128--149.
\bibitem{15-zac}
\Au{Петровский А.\,Б., Лобанов~В.\,Н.} Многокритериальный выбор в~про\-стран\-ст\-ве 
признаков большой раз\-мер\-ности: мультиметодная технология ПАКС-М~// 
Искусственный интеллект и~принятие решений, 2014. №\,3. С.~92--104.
\bibitem{16-zac}
\Au{Басков О.\,В.} Сужение множества Парето на основе нечеткой информации об 
отношении предпочтения ЛПР~// Искусственный интеллект и~принятие решений, 2014. 
№\,1. С.~57--65.
\bibitem{17-zac}
\Au{Атаева~О.\,М., Серебряков~В.\,А.} Основные понятия формальной модели 
семантических биб\-лио\-тек и~формализация процессов интеграции в~ней~// 
Программные продукты и~системы, 2015. №\,4(112). С.~180--187.
\bibitem{18-zac}
\Au{Антипин К.\,В., Фомичев~А.\,В., Гринев~М.\,Н. и~др.} Оперативная интеграция данных на основе 
XML: сис\-тем\-ная архитектура BizQuery~// Труды Института сис\-тем\-но\-го
программирования РАН,  2004. Т.~5. С.~157--174.
\bibitem{19-zac}
\Au{Зацаринный А.\,А., Сучков~А.\,П., Козлов~С.\,В.} Особенности проектирования 
и~функционирования ситуационных центров~// Системы высокой до\-ступ\-ности, 2012. 
Т.~8. №\,1. С.~12--22.
 \end{thebibliography}

 }
 }

\end{multicols}

\vspace*{-9pt}

\hfill{\small\textit{Поступила в~редакцию 17.04.17}}

\vspace*{7pt}

%\newpage

%\vspace*{-24pt}

\hrule

\vspace*{2pt}

\hrule

\vspace*{-6pt}


\def\tit{THE SITUATIONAL MANAGEMENT SYSTEM AS~A~MULTISERVICE TECHNOLOGY 
IN~THE~CLOUD}

\def\titkol{The situational management system as~a~multiservice technology 
in~the~cloud}

\def\aut{A.\,A.~Zatsarinny and A.\,P.~Suchkov}

\def\autkol{A.\,A.~Zatsarinny and A.\,P.~Suchkov}

\titel{\tit}{\aut}{\autkol}{\titkol}

\vspace*{-9pt}


\noindent
Institute of Informatics Problems, Federal Research Center ``Computer Science and Control'' of 
the Russian Academy of Sciences, 44-2~Vavilov Str., Moscow 119333, Russian Federation 



\def\leftfootline{\small{\textbf{\thepage}
\hfill INFORMATIKA I EE PRIMENENIYA~--- INFORMATICS AND
APPLICATIONS\ \ \ 2018\ \ \ volume~12\ \ \ issue\ 1}
}%
 \def\rightfootline{\small{INFORMATIKA I EE PRIMENENIYA~---
INFORMATICS AND APPLICATIONS\ \ \ 2018\ \ \ volume~12\ \ \ issue\ 1
\hfill \textbf{\thepage}}}

\vspace*{3pt}




\Abste{The article discusses the approaches to creation of a~unified management 
system as a~set of services in cloud computing. The proposed approach is based on the 
five-stage processing model. It allows defining the general form (prototype) 
of a~unified system of situational management, taking into account all types of internal 
and external information interactions in the structure of a~hierarchical control 
system.  It is shown 
that a~unified system must have a~means of configuration (localization) on a~specific 
application field. The proposed list of services and localization services 
provides the main functions of the control system. It is expected that such services 
will be  in demand in a~wide range of organizational systems, which should lead 
to a~significant reduction in the cost of development and implementation and 
ensure the possibility of  implementation of uniform information, program, 
and technical policy in the field of 
interagency and interdepartmental electronic interaction.}

\KWE{unified system of situational management; cloud service; process of localization}

\DOI{10.14357/19922264180110} 

\pagebreak

%\vspace*{-12pt}

%\Ack
%\noindent



%\vspace*{6pt}

  \begin{multicols}{2}

\renewcommand{\bibname}{\protect\rmfamily References}
%\renewcommand{\bibname}{\large\protect\rm References}

{\small\frenchspacing
 {%\baselineskip=10.8pt
 \addcontentsline{toc}{section}{References}
 \begin{thebibliography}{99} 

\bibitem{1-zac-1}
\Aue{Zatsarinny, A.\,A.} 2015. O~povyshenii ef\-fek\-tiv\-nosti informatsionno-analiticheskoy 
podderzhki prinyatiya%\linebreak\vspace*{-12pt}

%\columnbreak

\noindent
 strategicheskikh resheniy v~organakh gosudarstvennoy vlasti [Some 
problems of information-analytical support for strategic decision-making in 
government departments]. \textit{Mezhotraslevaya informatsionnaya sluzhba} [Intersectoral Information Service]  
1(170):11--22.
\bibitem{2-zac-1}
AWS CloudFormation. Available at: {\sf https://aws.amazon. com/ru/cloudformation} (accessed 
October~25, 2017).
\bibitem{3-zac-1}
Compute Engine: Scalable, High-Performance Virtual Machines. Available at:  {\sf 
https://cloud.google.com/\linebreak compute} (accessed October~25, 2017). 
\bibitem{4-zac-1}
Global'nost'. Nadezhnost'. Gibridnost': Oblako na vashikh usloviyakh
[The Global. Reliability. Hybridity: The cloud on your terms]. Available at:  {\sf 
https://azure. microsoft.com/ru-ru} (accessed October~25, 2017).
\bibitem{5-zac-1}
Amazon WorkSpaces: Fully managed, secure virtual cloud  desktops 
running on AWS. Available at:  
{\sf https://aws.amazon.com/workspaces} (accessed October~25, 2017).
\bibitem{6-zac-1}
Oracle Virtual Desktop Infrastructure. Available at: {\sf  
http://www.oracle.com/technetwork/server-storage/ virtualdesktop/overview/index.html} (accessed 
October~25, 2017). 
\bibitem{7-zac-1}
Oblachnaya sistema upravleniya biznesom [Cloud-based business management system]. 
Available at: {\sf http://cloud-automation.ru/management\_automation} (accessed October~25, 2017).
\bibitem{8-zac-1}
\Aue{Suchkov, A.\,P.} 2017. Polnofunktsional'nyy protsessnyy podkhod k~realizatsii sistem 
situatsionnogo upravleniya [A~fully functional process-based approach to the implementation of 
systems of situational management]. \textit{Sistemy i~Sredstva Informatiki~--- Systems and 
Means of Informatics} 27(1):85--99.
\bibitem{9-zac-1}
Federal'nyy issledovatel'skiy tsentr ``Informatika i~upravlenie'' Rossiyskoy akademii nauk: 
Osnovnye napravleniya issledovaniy [Federal Research Center ``Computer Science and Control'' 
of the Russian Academy of Sciences: Main research areas]. Available at: {\sf 
http:// www.frccsc.ru/expertise} (accessed October~25, 2017).
\bibitem{10-zac-1}
\Aue{Suchkov, A.\,P.} 2015. Nekotorye podkhody k~integratsii analiticheskikh dannykh 
sushchestvuyushchikh i~perspektivnykh sistem podderzhki prinyatiya resheniy [Some 
approaches to the analytical data integration of the existing and future decision support systems]. 
\textit{Sistemy i~Sredstva Informatiki~--- Systems and Means of Informatics} 25(3):201--211.

%\columnbreak

\bibitem{11-zac-1}
Elektronnoe pravitel'stvo: Gosuslugi [E-government: Public services] Available at: {\sf 
http://smev.gosuslugi.ru/ portal/} (accessed October~25, 2017).
\bibitem{12-zac-1}
About NIEM (National Information Exchange Model).  Available at: {\sf 
https//www.niem.gov/about-niem} (accessed October~25, 2017).
\bibitem{13-zac-1}
\Aue{Osipov, G.\,S.} 1997. \textit{Priobretenie znaniy intellektual'nymi sistemami} [Knowledge 
acquisition intelligent systems]. Moscow: Nauka. 112~p.
\bibitem{14-zac-1}
\Aue{Bystrov, I.\,I., B.\,V.~Tarasov, A.\,A.~Khoroshilov, and S.\,I.~Ra\-do\-ma\-nov.} 2015. Osnovy 
primeneniya ontologii i~komp'yuternoy lingvistiki pri proektirovanii perspektivnykh 
avtomatizirovannykh informatsionnykh sistem [The application basis 
of ontology and 
computational linguistics in the design of advanced automated information systems]. 
\textit{Sistemy i~Sredstva Informatiki~--- Systems and Means of Informatics} 25(4):128--149.
\bibitem{15-zac-1}
\Aue{Petrovsky, A.\,B., and V.\,N.~Lobanov.} 2015. Multi-criteria choice 
in the attribute space of large dimension: Multi-method technology PAKS-M. 
\textit{Scientific Technical Information Processing} 
42(5):76--86.
\bibitem{16-zac-1}
\Aue{Baskov, O.\,V.} 2015. An algorithm for Pareto set reduction using 
fuzzy information on decision-maker's preference relation. 
\textit{Scientific Technical Information Processing} 42(5):382--387.
doi: 10.3103/S0147688215050020.

\bibitem{17-zac-1}
\Aue{Ataeva, O.\,M., and V.\,A.~Serebryakov}. 2015. Osnovnye ponyatiya formal'noy modeli 
semanticheskikh bibliotek i~formalizatsiya protsessov integratsii v~ney 
[The basic concepts of a~semantic libraries
formal model  and its integration process formalization]. 
\textit{Programmnye produkty i~sistemy} [Software Systems] 4(112):180--187.
\bibitem{18-zac-1}
\Aue{Antipin, K.\,V., A.\,V.~Fomichev, M.\,N.~Grinev, \textit{et al.}}. 2004. Operativnaya integratsiya dannykh na 
osnove XML: Sistemnaya arkhitektura BizQuery [Operational data integration based on XML: 
The system architecture of BizQuery]. \textit{Proceedings of the Institute for System Programming of the RAS} 
5:157--174.
\bibitem{19-zac-1}
\Aue{Zatsarinny, A.\,A., A.\,P.~Suchkov, and S.\,V.~Kozlov.} 2012. Osobennosti 
proektirovaniya i~funktsionirovaniya si\-tu\-a\-tsi\-on\-nykh tsentrov [The features of design and 
operation of situational centers]. \textit{Sistemy vysokoy dostupnosti} 
[Highly Available Systems] 8(1):12--22.
\end{thebibliography}

 }
 }

\end{multicols}

\vspace*{-6pt}

\hfill{\small\textit{Received April 17, 2017}}

%\vspace*{-10pt}

\Contr

\noindent
\textbf{Zatsarinny Alexander A.} (b.\ 1951)~--- Doctor of Science in technology, professor, 
Deputy Director, Federal Research Center ``Computer Science and Control'' of the Russian 
Academy of Sciences, 44-2~Vavilov Str., Moscow 119333, Russian Federation; 
\mbox{AZatsarinny@ipiran.ru}

\vspace*{3pt}

\noindent
\textbf{Suchkov Alexander P.} (b.\ 1954)~--- Doctor of Science in technology, leading 
scientist, Institute of Informatics Problems, Federal Research Center ``Computer Science and 
Control'' of the Russian Academy of Sciences, 44-2~Vavilov Str., Moscow 119333, Russian 
Federation; \mbox{ASuchkov@frccsc.ru}

\label{end\stat}


\renewcommand{\bibname}{\protect\rm Литература} 