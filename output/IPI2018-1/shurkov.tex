\def\stat{shnurkov}

\def\tit{РАЗРАБОТКА И ПРЕДВАРИТЕЛЬНОЕ ИССЛЕДОВАНИЕ СТОХАСТИЧЕСКОЙ 
ПОЛУМАРКОВСКОЙ МОДЕЛИ УПРАВЛЕНИЯ ЗАПАСОМ НЕПРЕРЫВНОГО ПРОДУКТА 
ПРИ~ПОСТОЯННО ПРОИСХОДЯЩЕМ ПОТРЕБЛЕНИИ}

\def\titkol{Разработка и~предварительное исследование стохастической 
полумарковской модели управления запасом %непрерывного 
продукта}
%при~постоянно происходящем потреблении}

\def\aut{П.\,В.~Шнурков$^1$, А.\,Ю.~Егоров$^2$}

\def\autkol{П.\,В.~Шнурков, А.\,Ю.~Егоров}

\titel{\tit}{\aut}{\autkol}{\titkol}

\index{Шнурков П.\,В.}
\index{Егоров А.\,Ю.}
\index{Shnurkov P.\,V.}
\index{Egorov A.\,Y.}




%{\renewcommand{\thefootnote}{\fnsymbol{footnote}} \footnotetext[1]
%{Работа выполнена при финансовой поддержке РФФИ (проект 17-01-00816).}}


\renewcommand{\thefootnote}{\arabic{footnote}}
\footnotetext[1]{Национальный исследовательский университет <<Высшая школа экономики>>, 
\mbox{pshnurkov@hse.ru}}
\footnotetext[2]{Национальный исследовательский университет <<Высшая школа экономики>>, 
\mbox{ayuegorov@hse.ru}}

%\vspace*{-6pt}

 
   
\Abst{Проводится построение и~предварительное исследование дискретной стохастической 
полумарковской модели, описывающей функционирование некоторой системы управления 
запасом непрерывного продукта при постоянно происходящем потреблении. Модель 
представляет собой пару случайных процессов ($x(t), \zeta(t)$), где основной процесс~$x(t)$ 
описывает объем запаса в~системе в~момент времени~$t$, а~сопровождающий 
полумарковский случайный процесс~$\zeta(t)$ определяется по состояниям основного 
непосредственно после пополнений. Задача оптимального управления ставится по 
отношению к~стационарному показателю, имеющему характер средней удельной прибыли, 
полученной при эволюции исходной системы управления запасом.}
 
\KW{управление запасом; полумарковский случайный процесс; стационарный стоимостный 
функционал; оптимальное управление стохастическими системами}
  
  \DOI{10.14357/19922264180114} 
  
%\vspace*{6pt}


\vskip 10pt plus 9pt minus 6pt

\thispagestyle{headings}

\begin{multicols}{2}

\label{st\stat}

\section{Введение}

  Настоящая работа посвящена созданию и~предварительному исследованию 
новой стохастической полумарковской модели управления запасом 
непрерывного продукта при постоянно происходящем потреблении. Несмотря 
на обилие работ, посвященных общим проблемам управления запасом, 
исследований стохастических полумарковских моделей как в~отечественной, 
так и~в~зарубежной научной литературе немного. Краткая характеристика 
нескольких современных исследований, в~которых рассматриваются проблемы 
управления запасом в~различных стохастических полумарковских моделях, 
приведена в~приложении~[1]. 
  
  Заметим, что необходимые сведения по общей теории полумарковских 
процессов можно найти в~монографии~[2]. Из современных изданий можно 
упомянуть работу~[3]. Классические результаты по теории управления 
полумарковскими процессами изложены в~работах~[4, 5].
  
  Целый ряд моделей управления запасом непрерывного продукта в~форме 
регенерирующих и~полумарковских случайных процессов был разработан 
и~исследован в~работах П.\,В.~Шнуркова и~его соавторов. Библиография этих 
работ также приведена в~приложении~[1]. Стохастическая полумарковская 
модель, наиболее близкая к~исследуемой в~настоящей статье, рассмотрена 
П.\,В.~Шнурковым и~А.\,В.~Ивановым в~[6]. Отметим, что кроме 
непосредственного исследования полумарковской модели управ\-ле\-ния запасом 
в~этой работе был доказан общий теоретический результат о~пред\-став\-ле\-нии 
стационарного стоимостного показателя качества управ\-ле\-ния в~виде  
дроб\-но-ли\-ней\-но\-го интегрального функционала от вероятностных 
распределений, задающих стратегию управления полумарковским процессом. 
Этот результат играет важную роль в~решении задачи управления 
в~рассматриваемой полумарковской модели.
  
  Настоящая работа продолжает упомянутый цикл исследований 
регенерирующих и~полумарковских моделей управления запасами. При этом 
полумарковская модель управления запасом, исследуемая в~ней, является более 
глубоким и~сложным вариантом модели, рассмотренной в~\cite{6-sh}. 
Основные отличия данной модели от предшествующей заключаются 
в~следующем:
  \begin{enumerate}[(1)]
  \item  в~новой модели предполагается, что потребление продукта 
продолжается постоянно в~течение всего времени жизни системы;
  \item интенсивность потребления принимает два разных значения на 
различных периодах функционирования рассматриваемой системы.
  \end{enumerate}
  
  Указанные особенности приводят к~тому, что вероятностный анализ новой 
версии модели оказывается существенно более сложным. Отметим также, что 
структура рассматриваемой модели управ\-ле\-ния запасом имеет общие черты со 
стохастической моделью с~плановыми переключениями, которая была 
пред\-ло\-же\-на П.\,В.~Шнурковым в~работе~\cite{7-sh}.
  
  Основная цель первой части проведенного исследования, изложенной 
в~настоящей работе, заключается в~нахождении явного аналитического 
представления для стационарного стоимостного показателя эффективности 
управления, который по своему экономическому содержанию пред\-став\-ля\-ет 
собой среднюю удельную прибыль, обра\-зу\-ющу\-юся при управ\-ле\-нии данной 
системой. 

\section{Общее описание и~некоторые особенности 
функционирования рассматриваемой системы управления запасом 
непрерывного продукта}
  
  Рассмотрим систему, предназначенную для хранения и~поставки 
потребителю запаса определенного вида непрерывного продукта. 
Предположим, что объем запаса продукта в~данной системе в~момент 
времени~$t$, который будет обозначаться $x(t)$, представляет собой 
некоторый случайный процесс, формальное описание которого будет 
приведено в~дальнейшем. Случайный процесс $x(t)$ принимает значения 
в~множестве $(-\infty,\tau]$, где $\tau\hm>0$~--- фиксированная максимальная 
вместимость хранилища. Как принято в~классической теории управления 
запасами, отрицательная величина запаса означает отложенный спрос.
  
  \smallskip
  
  \noindent
  \textbf{Определение~1.} Периодом эволюции системы будем называть 
интервал времени [$t_n;t_{n+1}$) между последовательными моментами 
пополнения запаса. 
  
  \smallskip
  
  Каждый период эволюции естественным образом допускает разбиение на две 
части: период времени от момента последнего пополнения запаса до момента 
заказа на очередную поставку (период чис\-то\-го потребления) и~период времени 
от момента заказа до следующего пополнения (период задержки поставки). 
В~течение периода задержки поставки потребление продолжается. 
Длительность задержки поставки является известной детерминированной 
величиной, обозначаемой через~$h$.
  
  Пусть потребление происходит непрерывно с~постоянной скоростью 
$\alpha_c\hm>0$ в~течение периода\linebreak\vspace*{-12pt}

 { \begin{center}  %fig1
 \vspace*{-1pt}
  \mbox{%
 \epsfxsize=77.535mm 
 \epsfbox{shn-1.eps}
 }


\end{center}


\noindent
{{\figurename~1}\ \ \small{Возможная траектория процесса $x(t)$, характеризующая функционирование 
исследуемой системы}}
}

\vspace*{9pt}


 
\noindent
 чистого потребления. В~момент заказа 
скорость потребления меняется и~становится равной величине 
$\alpha_w\hm=const$, $\alpha_w\hm\geq 0$, причем $\alpha_w\hm\leq \alpha_c$. 
В~частном случае может быть $\alpha_w\hm=\alpha_c$, что соответствует 
непрерывному потреблению с~одной фиксированной скоростью на всем 
периоде жизни системы. Предполагается, что непосредственное пополнение 
запаса происходит мгновенно в~момент окончания очередного периода 
эволюции системы~$t_n$, $n\hm=1,2,\ldots$ Возможная траектория процесса 
$x(t)$ изображена на рис.~1.
  

  
  Для удобства дальнейших рассуждений и~аналитических выводов введем 
следующие предположения: $\exists\ L\hm>0$ такое, что заданные 
величины~$\tau$ и~($\alpha_w h$) делятся нацело на~$L$. При построении 
модели используем дискретизацию: разобьем множество значений объема 
запаса на конечное число подмножеств~--- полузамкнутых интервалов, длины 
которых, кроме одного бесконечного интервала, одинаковы и~равны~$L$ 
(содержание данных предположений подробно прокомментировано 
в~замечании~1 приложения~\cite{1-sh}).
  
  Будем использовать следующие обозначения для вводимых множеств 
возможных значений процесса~$x(t)$:
  \begin{align*}
E_0^{(-)} &= \left(\tau_1^{(-)},\tau_0^{(-)}\right)= (-L,0)\,,\ldots\\
E_k^{(-)} &= \left( \tau_{k+1}^{(-)}, \tau_k^{(-)} \right] = 
\left( (-k+1)L, -kL\right]\,,\\
& \hspace*{40mm}k\in \{1,\ldots, N_1-1\}\,;\\
E_{N_1}^{(-)} &= \left( \tau^{(-)}_{N_1+1},\tau_{N_1}^{(-)}\right] =\left( -
\infty, -N_1L\right]\,;\\
E_k^{(+)} &= \left[ \tau_k^{(+)},\tau^{(+)}_{k+1}\right) = \left[ kL, 
(k+1)L\right)\,,\\
&\hspace*{40mm} k\in \{0,\ldots ,N_0-1\}\,;\\
E^{(+)}_{N_0} &= \left[ \tau^{(+)}_{N_0}, \tau^{(+)}_{N_0+1}\right] = \left[ N_0 
L, \tau\right]\,.
\end{align*}
  
  Зададим в~каждом множестве~$E_i^{(+)}$ фиксированную точку 
$\hat{\tau}_i^{(+)}\hm\in E_i^{(+)}=\left[ \tau_i^{(+)}, \tau^{(+)}_{i+1}\right)$, 
которую назовем $i$-м уровнем планирования объема запаса, $i\hm\in \{0,\ldots 
, N_0\}$.
  
  Для описания функционирования системы и~построения ее 
математической модели необходимо задать исходные стоимостные 
характеристики. Предположим, что заданы следующие функции:
  \begin{description}
  \item[\,]
  $c_1(x)$~--- затраты, связанные с~хранением продукта объема $x$   в~системе 
в~единицу времени; $c_1(x)\hm=0$, $x\hm\leq 0$;
  \item[\,] $c_2(x)$~--- штрафы, связанные с~дефицитом продукта 
объема~$x$ в~системе в~единицу времени; $c_2(x)\hm=0$, $x\hm\geq 0$;
  \item[\,]
  $d(x)$~--- доход, связанный с~поставкой одной единицы продукта 
потребителю при условии, что объем имеющегося продукта равен~$x$; при 
этом представления данной функции при $x\hm\geq 0$ (реальный запас) 
и~$x\hm\leq 0$ (дефицит) определяются выражениями~$d_1(x)$ и~$d_2(x)$ 
соответственно.
  \end{description}
  
\section{Формальное построение полумарковской стохастической 
модели, описывающей функционирование исследуемой системы}

  Зададим исходное вероятностное пространство $(\Omega,\mathcal{F}, P)$), на 
котором будут определены все необходимые для построения модели 
стохастические объекты. Отметим прежде всего, что основной стохастический 
объект~$x(t)$, характеризующий уровень запаса в~системе в~момент 
времени~$t$, представляет собой случайный процесс, заданный на данном 
вероятностном пространстве и~множестве значений параметра времени: 
  $$
  x(t)=x(\omega,t),\enskip \omega\in \Omega,\enskip t\in T= [0,\infty).
  $$ 
  
  Введем также две последовательности случайных величин: $\{t_n\hm= 
t_n(\omega),\ \omega\hm\in \Omega\}^\infty_{n=0}$~--- последовательность 
случайных моментов времени,\linebreak в~которые производится пополнение запаса, 
и~$\{t_n^\prime\hm= t^{\prime}_n (\omega),\ \omega\hm\in \Omega\}^\infty_{n=0}$~--- 
последовательность случайных моментов времени, в~которые производится 
заказ на пополнение запаса, $t_n\hm\leq t_n^\prime\hm< t_{n+1}$, $n\hm= 
0,1,2,\ldots$; $t_0\hm=0$. 

Предположим, что в~момент очередного пополнения 
запаса $t_n\hm= t_n(\omega)$, $\omega\hm\in \Omega$, состояние системы 
(уровень запаса) принимает фиксированное произвольное значение $x(t_n)\hm= 
x\hm\in E_i^{(+)}$. Дальнейшее описание модели будем проводить при 
условии, что реализуется указанное событие ($x(t_n)\hm= x\hm\in E_i^{(+)}$). 
Предполагается, что траектория процесса~$x(t)$ в~моменты пополнения 
запаса~$t_n$, $n\hm=0,1,2,\ldots$, непрерывна справа, т.\,е.\ $\lim\nolimits_{t\to 
t_n+0} x(t)\hm= x(t_n)$. 
  
  Зафиксируем номер $i\hm\in \{0,1,\ldots ,N_0\}$ и~введем на пространстве 
$(\Omega, \mathcal{F}, P)$) последовательность независимых одинаково 
распределенных случайных величин $\xi_{i,n} \hm= \xi_{i,n}(\omega)$, 
$n\hm=0,1,2,\ldots$, $\omega\hm\in \Omega$, принимающих значения 
в~множестве $\mathbf{U}\hm = [0, \infty)$. Обозначим через $G_i(t)\hm= 
{\sf P}(\xi_{i,n}<t)$ функции распределения случайных величин 
$\{ \xi_{i,n}\}^\infty_{n=0}$. Кроме того, дополнительно обозначим $G_i(0)\hm= 
g_{i,0}$, $0\hm\leq g_{i,0}\hm<1$, где $g_{i,0}$~--- фиксированный параметр, 
вводимый для удобства дальнейших рассуждений.
  
  Будем интерпретировать случайную величину~$\xi_{i,n}$ как время от 
момента пополнения~$t_n$ до момента следующего заказа на пополнение. 
Тогда момент заказа на пополнение есть $t^\prime_n \hm= t_n\hm+\xi_{i,n}$, 
и~с~единичной вероятностью выполняется соотношение $t_n\hm\leq t_n^\prime 
\hm< t_{n+1}$. Случайная величина~$\xi_{i,n}$ служит параметром управ\-ле\-ния 
в~рассматриваемой стохастической модели, или решением, при\-ни\-ма\-емым 
в~момент~$t_n$, при условии, что $x(t_n)\hm= x\hm\in E_i^{(+)}$.
  
  Зафиксируем реализацию случайной величины  $\xi_{i,n}: \xi_{i,n}\hm=u_i$. При этом 
условии траектория процесса $x(t)$ на интервале времени $[t_n;t_n^\prime)$ 
определяется равенством:
  \begin{equation*}
  x(t)=x-\alpha_c(t-t_n)\,,\enskip 
  t_n\leq t<t_n^\prime =t_n+u_i\,.
  %\label{e1-sh}
  \end{equation*}
  
  Предположим, что 
  $$
  x\left(t_n^\prime\right)= x\left(t_n\hm+u_i\right)= x^\prime\hm\in 
E_k^{(+)}\,,
$$
 где $k\hm= 0,1,\ldots, i$, т.\,е.\ в~момент заказа~$t_n^\prime$ объем 
продукта принимает значение $x^\prime\in E_k^{(+)}$. Описание модели для 
случая, когда $x^\prime\hm\in E_k^{(-)}$, производится совершенно аналогично.
  
  В момент~$t_n^\prime$ начинается так называемый период задержки 
поставки. В~течение данного периода поставщик готовит новую партию 
продукта для пополнения. Непосредственное пополнение запаса происходит 
мгновенно в~конечный момент периода задержки.
  
    В период задержки потребление не прекращается, а продолжается 
с~заданной ско\-ростью~$\alpha_w$, $\alpha_w\hm\leq \alpha_c$. Поскольку 
длительность периода задержки не случайна и~из\-вест\-на, объем потребления 
продукта в~течение этого периода также из\-вес\-тен и~равен~$\alpha_w h$. Таким 
образом, управляющему рассматриваемой системы и~поставщику известен 
объем запаса в~момент окончания периода задержки, т.\,е.\ непосредственно 
перед пополнением. Действительно, если выполняются принятые выше условия 
$x(t_n)\hm= x\hm\in E_i^{(+)}$, $\xi_{i,n}\hm= t_n^\prime \hm- t_n\hm= u_i$, то 
траектория процесса~$x(t)$ в~течение периода задержки $[t_n^\prime, 
t_{n+1})$ определяется равенством:
    \begin{multline}
    x(t)=x-\alpha_c u_i -\alpha_w (t-t^\prime_n)\,,\\
    t_n^\prime \leq t< t^\prime_n+h=t_{n+1}\,.
    \label{e2-sh}
    \end{multline}
        Тогда значение процесса~$x(t)$ непосредственно перед пополнением 
формально определяется как предел слева функции~$x(t)$, задаваемой 
формулой~(\ref{e2-sh}), в~точке~$t_{n+1}$, а именно:
    $$
    x\left(t_{n+1}-0\right) =x-\alpha_c u_i -\alpha_w h\,.
    $$
    
    Таким образом, в~рассматриваемой модели величину объема заказа на 
пополнение можно планировать непосредственно в~момент заказа. 
  
  Перейдем к~описанию процедуры пополнения запаса. При фиксированном 
условии $x(t_n^\prime)\hm= x^\prime\hm\in E_k^{(+)}$ рассмотрим систему 
случайных событий $\{B^{(+)}_{kj}\}^{N_0}_{j=0}$, имеющих заданные 
вероятности
  \begin{multline*}
  {\sf P}\left( B^{(+)}_{kj}\left\vert x\left(t_n^\prime\right) \right. =x^\prime\in 
E_k^{(+)}\right) =\beta^{(+)}_{kj}\,,\\ 
  j=\{0,1,\ldots ,N_0\}\,,\enskip
   \sum\limits_j \beta^{(+)}_{kj}=1\,.
  \end{multline*}
  
  \noindent
  \textbf{Замечание~1.} Если $x^\prime\hm-\alpha_w h\hm\in E_m^{(+)}$, где $m 
\hm= 0, 1, \ldots, k$, то предполагается, что $\beta_{kj}^{(+)}\hm=0$ при $j 
\hm= \{0, 1, \ldots, m\}$. Это условие означает, что при пополнении запаса 
объективно не может планироваться уменьшение его объема. 
  
  \smallskip
  
  Событие $B_{kj}^{(+)}$ связано с~траекторией процесса $x(t)$ следующим 
образом. Если выполняется условие, что $x(t_n^\prime)\hm= x^\prime\hm\in 
E_k^{(+)}$, то реализация случайного события $B_{kj}^{(+)}$ означает, что 
выбирается значение~$\hat{\tau}_j^{(+)}\hm\in E_j^{(+)}$ из возможных 
значений уровней планирования. С~точки зрения содержания модели 
в~момент~$t_n^\prime$ происходит планирование объема заказа. План имеет 
случайный характер: в~соответствии с~этим планом объем запаса 
непосредственно после пополнения должен достигнуть 
уровня~$\hat{\tau}_j^{(+)}$ с~вероятностью~$\beta_{kj}^{(+)}$. Как уже 
отмечалось, если $x^\prime\hm- \alpha_w h\hm\in E_m^{(+)}$, $m\hm = 0, 1, 
\ldots, k$, то $\sum\nolimits_{j=m}^{N_0} \beta_{kj}^{(+)}\hm=1$. Обозначим
также через $\Delta \hat{\gamma}_j^{(+)}$ величину 
планируемого заказа, которая при указанном условии $x(t_n^\prime)\hm= x^\prime\hm\in 
E_k^{(+)}$ равна $\hat{\tau}_j^{(+)}\hm- \left( x^\prime\hm- \alpha_w 
h\right).$
  
  \smallskip
  
  \noindent
  \textbf{Замечание~2.} Если состояние процесса $x(t)$ в~момент заказа 
соответствует дефициту продукта, т.\,е.\ $x(t_n^\prime)\hm= 
x^\prime\hm\in E_k^{(-)}$, то событие~$B_{kj}^{(-)}$ также заключается 
в~выборе уровня планирования~$\tau_j^{(+)}$. Данное условие отражает 
особенность модели, за\-клю\-ча-\linebreak\vspace*{-12pt}
  
{ \begin{center}  %fig2
 \vspace*{-2pt}
  \mbox{%
 \epsfxsize=64mm 
 \epsfbox{shn-2.eps}
 }


\end{center}

\vspace*{-5pt}


\noindent
{{\figurename~2}\ \ \small{Процедура пополнения запаса: планирование уровня пополнения}}
}

\vspace*{10pt}


\noindent
ющу\-юся в~том, что при каждом пополнении 
запаса отложенный спрос полностью удовлетворяется и~объем запаса 
становится положительным.

  Иллюстрация к~описанной процедуре пополнения запаса приведена на 
рис.~2. Для удобства начальный момент данного периода эволюции 
предполагается равным нулю: $t_n\hm=0$. 
  
  Продолжим описание процедуры пополнения запаса. Предположим, что 
реализовалось событие~$B_{kj}^{(+)}$ или $B_{kj}^{(-)}$ и~зафиксирован 
планируемый уровень пополнения~$\hat{\tau}_j^{(+)}\hm\in E_j^{(+)}$. 
Введем еще одну случайную величину~$\hat{\Delta}_j^{(+)}$, которая 
характеризует отклонение от планируемого уровня 
запаса~$\hat{\tau}_j^{(+)}$, возникающее при выполнении заказа. Данное 
отклонение обусловлено объективными случайными факторами, 
действующими на рынке продукта, которые влияют на выполнение заказа 
поставщиком. 
  
  Для описания процедуры пополнения с~учетом случайных отклонений от 
уровней планирования зададим набор распределений~$B_j(x)$, 
сосредоточенных на соответствующих интервалах~$E_j^{(+)}$, причем 
%\noindent
$$
B_j(x)=
\begin{cases}
0\,, &\ x< jL\,;\\
1\,, &\ x\geq (j+1)L\,,
\end{cases}
$$ 

%\vspace*{-3pt}

\noindent
$j\hm=0,1,\ldots, N_0$. Распределение~$B_j(x)$ описывает уровень запаса 
после пополнения с~учетом отклонения от планируемого уровня~$\hat{\tau}_j^{(+)}$, 
а~именно: значение уровня запаса после пополнения пред\-став\-ля\-ет 
собой величину $\hat{\gamma}_j^{(+)} \hm= \hat{\tau}_j^{(+)} \hm+ 
\hat{\Delta}_j^{(+)}$, имеющую распределение 
${\sf P}\left( 
\hat{\gamma}_j^{(+)}\hm< x\right)\hm= B_j(x).$
 При этом функция 
распределения величины~$\hat{\Delta}_j^{(+)}$ определяется равенством: 

\noindent
  \begin{multline*}
  \hat{B}_j(x) ={\sf P}\left( \hat{\Delta}_j^{(+)} <x\right) ={\sf P}\left( 
\hat{\gamma}_j^{(+)} -\hat{\tau}_j^{(+)} <x\right) ={}\\
{}= B_j \left( \hat{\tau}_j^{(+)} 
+x\right)\,,\enskip
  j=0,1,\ldots , N_0\,.
  \end{multline*}
  
  Более полное формализованное описание траектории процесса~$x(t)$ 
и,~в~частности, процедуры пополнения содержится в~приложении~[1].
  
  \section{Дополнительные стохастические объекты 
в~полумарковской модели}
  
  В предыдущем разделе проведено формальное построение случайного 
процесса $x(t)\hm= x(\omega, t)$, $\omega\hm\in \Omega$, $t\hm\in [0,\infty)$, 
описывающего уровень запаса в~исследуемой системе в~произвольный момент 
времени~$t$. Теперь введем дополнительные стохастические объекты на 
вероятностном пространстве $(\Omega, \mathcal{F}, P)$, связанные со 
случайным процессом~$x(t)$. Определим последовательность случайных 
величин $\zeta_n\hm= \zeta_n(\omega)$, $\omega\hm\in \Omega$, 
$n\hm=0,1,2\ldots$, следующим образом. Рассмотрим функцию $\mathrm{ind}\,(x)$, 
заданную на множестве~$[0,\tau]$, принимающую значения из конечного 
множества $\{0,1,\ldots ,N_0\}$ и~определяемую равенством:
  $$
  \mathrm{ind}\,(x) =i\,,\ \mbox{если } x\in \left[ \tau_i^{(+)}; 
  \tau_{i+1}^{(+)}\right)\,,\ i=0,1,\ldots, 
N_0\,.
  $$
    Тогда случайная величина~$\zeta_n(\omega)$ определяется соотношением:
  $$
  \zeta_n(\omega) =\zeta_n=\mathrm{ind}\left( x\left( t_n\right)\right)\,,\ n=0,1,2,\ldots
  $$
  
  Иначе говоря, если $x(t_n)\hm= x(t_n+0)\hm= x\hm\in \left[ \tau_i^{(+)}; 
\tau_{i+1}^{(+)}\right)$, то $\zeta_n\hm=i$. Начальное значение 
последовательности~$\{\zeta_n\}$ определяется начальным значением процесса 
$x(t)$. Конкретно, если $x(0)\hm= x_0\hm\in \left[ \tau^{(+)}_{i_0}; 
\tau^{(+)}_{i_0+1}\right)$, то полагаем $\zeta_0\hm=i_0$. Поскольку исходный 
случайный процесс~$x(t)$ обладает марковским свойством в~моменты 
пополнения запаса, случайная последовательность $\{\zeta_n,\ 
n\hm=0,1,2,\ldots\}$ образует цепь Маркова с~конечным множеством состояний 
$\{0,1,\ldots, N_0\}$. 

Определим случайный процесс~$\zeta(t)$ следующим 
образом:
  $$
  \zeta(t)=\zeta_n\ \mbox{при } t_n\leq t < t_{n+1}\,,\ n=0,1,2\ldots
  $$
    Случайный процесс~$\zeta(t)$ является полумарковским с~множеством 
состояний $\{0,1,\ldots ,N_0\}$. Данный\linebreak процесс будем называть 
сопровождающим по\-лу\-мар\-ковским процессом. Случайная последо\-ва\-тельность 
$\{\zeta_n,\ n\hm=0,1,2,\ldots\}$ является цепью\linebreak Маркова, вложенной в~данный 
полумарковский процесс. Обозначим через 
$p_{ij}\hm= {\sf P}\left( 
\zeta_{n+1}=j\vert \zeta_n=i\right)$, $i,j\hm\in \{0,1,\ldots, N_0\}$,
переходные 
вероятности этой марковской цепи.
  
  Теперь введем аддитивный стоимостный функционал $V(t)\hm= V(\omega, 
t)$, $\omega\hm\in \Omega$, связанный с~сопровождающим полумарковским 
процессом~$\zeta(t)$. Этот функционал будет характеризовать прибыль, 
получаемую при функционировании системы. Общая схема задания такого 
функционала описана в~классических работах~\cite{4-sh, 5-sh}. Отметим, что 
в~рассматриваемой стохастической модели приращение этого функционала на 
периоде эволюции будет определяться исходными стоимостными 
характеристиками~$c_1(x)$, $c_2(x)$, $d_1(x)$ и~$d_2(x)$. Обозначим 
через~$V(t)$ случайное значение этого функционала в~момент времени~$t$. 
  
  Пусть $\Delta V_n\hm= V(t_{n+1})\hm- V(t_n)$~--- приращение функционала 
на периоде времени между моментами изменения состояний полумарковского 
процесса~$\zeta(t)$. 
  
   Для постановки и~решения задачи оптимального управления 
в~рассматриваемой модели потребуются также следующие вероятностные 
ха\-рак\-те\-ри\-стики:
  \begin{description}
  \item[\,]
  $d_i=E\left[ \Delta V_n\vert \zeta_n=i\right]$~--- условное математическое 
ожидание прибыли на периоде между последовательными моментами 
изменения состояний процесса~$\zeta(t)$ при условии, что на этом периоде 
процесс находился в~состоянии~$i$; 
  \item[\,] $T_i=E\left[ t_{n+1}\hm- t_n\vert \zeta_n=i\right]$~--- условное 
математическое ожидание длительности периода между последовательными 
моментами изменения состояний процесса~$\zeta(t)$ при условии, что на этом 
периоде процесс находился в~состоянии~$i$;
  \item[\,]
  $\Pi=\left\{ \pi_0, \ldots , \pi_{N_0}\right\}$~--- стационарное распределение 
цепи Маркова $\{\zeta_n\}$, вложенной в~данный полумарковский 
процесс~$\zeta(t)$. 
\end{description}
  
  Известно, что при достаточно общих условиях, основным из которых 
является существование стационарного распределения вложенной цепи 
Маркова $\{\zeta_n\}$, имеет место следующее утверждение (обычно 
называемое эргодической теоремой) о~поведении стоимостного аддитивного 
функционала~$V(t)$: 
  \begin{equation}
  I=\lim\limits_{t\to\infty} \fr{EV(t)}{t}=\fr{\sum\nolimits_{i=0}^{N_0} d_i \pi_i 
}{\sum\nolimits^{N_0}_{i=0} T_i \pi_i}\,.
  \label{e3-sh}
  \end{equation}
  
  Доказательство соотношения~(\ref{e3-sh}) для полумарковского процесса с~конечным множеством состояний приведено в~работе~\cite{5-sh}. 
В~настоящее время известно, что аналогичное утверждение имеет место для 
весьма общих полумарковских моделей с~произвольными пространствами 
состояний и~управлений~\cite{8-sh}.
  
  По своему прикладному содержанию величина~$I$ представляет собой 
среднюю удельную прибыль, возникающую при длительной эволюции 
полумарковской модели. В~дальнейшем величина~$I$ будет 
рас\-смат\-ри\-вать\-ся как показатель эф\-фек\-тив\-ности управления в~данной модели. 
  
  Теперь приведем формулировки утверждений о явных аналитических 
представлениях вспомогательных характеристик, входящих в~выражение для 
функционала~(\ref{e3-sh}). 

\section{Основные вероятностные характеристики модели}

  Зафиксируем состояние сопровождающего процесса $\zeta(t)$ в~момент 
очередного пополнения запаса, т.\,е.\ условие $\zeta(t_n)\hm=\zeta_n\hm=i$. 
Введем в~множестве~$\Omega$ систему несовместных событий, связанных 
с~этим условием:
  \begin{align*}
  A_k^{(+)} &= \left( x\left( t_{n+1}-0\right)\in E_k^{(+)}\right)\,,\enskip 
k=0,1,\ldots ,i\,;\\
  A_k^{(-)} &= \left( x\left( t_{n+1}-0\right)\in E_k^{(-)}\right)\,,\enskip 
k=0,1,\ldots , N_1\,.
  \end{align*}
  
  Заметим, что система несовместных событий $\left\{ A_k^{(+)},\ 
k\hm=0,1,\ldots ,i;\ A_k^{(-)},\ k=0,1,\ldots , N_1\right\}$ образует разбиение 
множества элементарных исходов $(\zeta_n\hm=i)\hm= \left( \omega\hm\in 
\Omega:\ \zeta_n(\omega)\hm=i\right)$, т.\,е.\ множества, соответствующего 
принятому условию. Рассмотрим условные вероятности 
  \begin{equation}
  \left.
  \begin{array}{rl}
  p_{ik}^{(+)} &= {\sf P}\left( A_k^{(+)}\vert \zeta_n=i\right) 
  ={}\\[6pt]
  &{}=P\left( x(t_{n+1}-
0) \in E_k^{(+)}\vert \zeta_n=i\right)\,,\\
&  \hspace*{35mm}k=0,1,\ldots , i\,;%\label{e4-sh}
\\[6pt]
  p_{ik}^{(-)} &= {\sf P}\left( A_k^{(-)}\vert \zeta_n=i\right) ={}\\[6pt]
  &{}=
  {\sf P}\left( x(t_{n+1}-0) 
\in E_k^{(-)}\vert \zeta_n=i\right)\,,\\[6pt]
&
  \hspace*{35mm}k=0,1,\ldots , N_1\,;
  \end{array}
  \right\}
  \label{e5-sh}
  \end{equation}
  
  Как уже отмечалось в~разд.~3, в~принятой модели значения процесса~$x(t)$ 
в~момент заказа~$x(t_n^\prime)$ и~непосредственно перед пополнением 
$x(t_{n+1}\hm-0)$ связаны однозначно: 
  $$
  x\left(t_{n+1}-0\right)= x\left(t_n^\prime\right)- \alpha_w h\,.
  $$
     Отсюда следует, что 
можно описать переход процесса из произвольного подмножества 
состояний~$E_k^{(+)}$, $k\hm=0,1,\ldots ,i$,  или $E_k^{(-)}$, $k\hm=0,1,\ldots , 
N_1$, на планируемый уровень $\hat{\tau}_j^{(+)}$ при помощи вероятностей 
$\beta_{kj}^{(+)}$, $k\hm=0,1,\ldots ,i$,  или~$\beta_{kj}^{(-)}$, $k\hm=0,1,\ldots 
, N_1$, соответственно. Учитывая, что в~момент заказа~$t_n^\prime$ процесс 
$x(t)$ обладает марковским свойством, и~используя указанные свойства 
системы событий $\{A_k^{(+)},\ k\hm= 0,1,\ldots, i;\ A_k^{(-)}, k\hm= 0,1,\ldots , 
N_1\}$, можно применить формулу полной вероятности. По\-лу\-чаем:
  \begin{multline}
  p_{ij}=\sum\limits^i_{k=0} p_{ik}^{(+)} \beta^{(+)}_{(k+r)j} 
+\sum\limits^r_{k=0} p_{ik}^{(-)} \beta^{(+)}_{(r-k)j} 
+{}\\[3pt]
{}+\sum\limits^{N_1}_{k=r+1} p_{ik}^{(-)} \beta^{(-)}_{(k-r)j}
+ \sum\limits_{k=N_1+1}^{N_1+r} p_{iN_1}^{(-)} \beta_{(k-r)j}^{(-)}\,,
  \label{e6-sh}
  \end{multline}
где $r=\alpha_w h/L$.

  Формула~(\ref{e6-sh}) позволяет выразить вероятности перехода вложенной 
цепи Маркова~$\{\zeta_n\}$ через вспомогательные вероятностные 
характеристики~(\ref{e5-sh}) и~заданные вероятности  
$\{ \beta_{kj}^{(+)}\}$ и~$\{\beta_{kj}^{(-)}\}$. В~дальнейшем будут найдены 
явные представления для указанных характеристик~(\ref{e5-sh}). 
{ %\looseness=1

}
  
  Обозначим:
  \begin{align}
  d_{ik}^{(+)} &={}\notag\\
&\hspace*{-10mm} {}= E\left[ V_{n+1} -V_n;\ x\left( t_{n+1}-0\right)\in E_k^{(+)}\vert 
\zeta_n=i\right]\,;\label{e7-sh}\\[3pt]
  d_{ik}^{(-)} &={}\notag\\
  &\hspace*{-10mm}{}= E\left[ V_{n+1} -V_n;\ x\left( t_{n+1}-0\right)\in E_k^{(-)}\vert 
\zeta_n=i\right]\,.\label{e8-sh}
  \end{align}
  
  Аналогичным образом используем свойство сис\-те\-мы событий  
$\{ A_k^{(+)}, k\hm= 0,1,\ldots , i;\ A_k^{(-)},\ k\hm=0,1,\ldots , N_1\}$ 
и~формулу полного математического ожидания. Получим:
  \begin{equation}
  d_i=\sum\limits^i_{k=0} d_{ik}^{(+)} +\sum\limits_{k=0}^{N_1} 
  d_{ik}^{(-)}\,.
  \label{e9-sh}
  \end{equation}
  
  Представление для условного математического ожидания длительности 
периода пребывания полумарковского процесса~$\zeta(t)$ в~состоянии~$i$ 
может быть получено непосредственно (см.\ ниже теорему~2). 
{\looseness=1

}         
    
  Теперь приведем формулировки утверждений о явных аналитических 
представлениях вспомогательных вероятностных характеристик, входящих 
в~приведенные выше формулы~(\ref{e6-sh}) и~(\ref{e9-sh}). 
  
  \medskip
  
  \noindent
  \textbf{Теорема~1.}\ \textit{Предположим, что в~рассматриваемой 
стохастической модели реализуется событие $x(t_n)\hm= x\hm\in E_i^{(+)}$, 
где $i$~--- некоторое фиксированное целое число, $i\hm=0,1,\ldots , N_0$. Тогда 
аналитические представления условных вероятностей $p_{ik}^{(+)}$  
и~$p_{ik}^{(-)}$ имеют следующий вид}:

\noindent
  \begin{align*}
  p_{ik}^{(+)} &= \int\limits_{iL}^{(i+1)L} \left[ 
\int\limits_{\sigma_{k+1}^{(+)}(x)}^{\sigma_k^{(+)}(x)} dG_i(u)\right] \,dB_i(x)\\ 
&\hspace*{25mm}\mbox{при } k=0,\ldots , \left( k_i^{(+)} -1\right)\,;\\
  p^{(+)}_{ik_i^{(+)}} &= \int\limits_{iL}^{(i+1)L} \left[ 
\int\limits_0^{\sigma^{(+)}_{k_i^{(+)}}(x)}\,dG_i(u)\right] dB_i(x)\\
& \hspace*{45mm}\mbox{при } 
k=k_i^{(+)}\,;\\
 p_{ik}^{(+)} &=0 \ \mbox{при } k=\left( k_i^{(+)}+1\right),\ldots , i\,;\\
 p_{ik}^{(-)} &=0\ \mbox{при } k=0,\ldots , \left( k_i^{(-)}-1\right)\,;\\
  p^{(-)}_{ik_i^{(-)}} &= \int\limits_{iL}^{(i+1)L} \left[
   \int\limits_0^{\sigma^{(-)}_{k_i^{(-)}+1}(x)} dG_i(u)\right] dB_i(x)\\
   & \hspace*{45mm}\mbox{при } k=k_i^{(-)}\,;\\
  p_{ik}^{(-)} &=\int\limits_{iL}^{(i+1)L} \left[ 
  \int\limits_{\sigma_k^{(-)}(x)}^{\sigma^{(-)}_{k+1}(x)}dG_i(u)\right] dB_i(x)\\
  & \hspace*{15mm}\mbox{при } k=\left( 
k_i^{(-)}+1\right),\ldots , (N_1-1)\,;\\
  p_{iN_1}^{(-)} &= \int\limits_{iL}^{(i+1)L} \left[ 1-G_i\left( 
  \sigma^{(-)}_{N_1}(x)\right)\right]\,dB_i(x)\\
  & \hspace*{20mm}\mbox{при } k=N_1\,,\ (i+N_1)L> 
\alpha_w h\,;\\
  p^{(-)}_{iN_1} &= \int\limits_{iL}^{(i+1)L} \left[ 1-g_{i,0}\right]\,dB_i(x)\\ 
&\hspace*{20mm}\mbox{при } k=N_1\,,\ (i+N_1)L\leq \alpha_w h\,,
\end{align*}
\textit{где $k_i^{(\pm)}$~--- граничные значения, которые зависят от известных 
фиксированных параметров системы~$\alpha_w$, $h$ и~$L$ и~конкретно определены 
в~приложении}~[1];
\begin{align*}
\sigma_k^{(\pm)}(x) &= \fr{x-\tau_k^{(\pm)}}{\alpha_c}-
\fr{\alpha_w}{\alpha_c}\,h\,;\\
\tau^{(+)}_{N_0+1} &=\tau\,;\\[3pt]
\tau_k^{(+)}&=kL\,,\enskip k=0,1,\ldots , N_0\,;\\[3pt]
   \tau_{N_1+1}^{(-)} &= -\infty\,;\\[3pt]
    \tau_k^{(-)}& =-kL\,,\enskip k=0,1,\ldots ,N_1\,.\\[-9pt]
   \end{align*}
  
  \noindent
  \textbf{Теорема~2.}\ \textit{Предположим, что выполнено вероятностное 
условие, сформулированное в~теореме~$1$}: $x(t_n)\hm= x\hm\in E_i^{(+)}$, 
$i\hm=0,1,\ldots , N_0$. \textit{Тогда математическое ожидание времени пребывания 
полумарковского случайного процесса~$\zeta(t)$ в~состоянии~$i$ представимо 
в~виде}:
  $$
  T_i=\int\limits^{(i+1)L}_{iL} \left[ \int\limits_0^\infty u \,d 
G_i(u)\right]\,dB_i(x)+h\,.
  $$
  
  \noindent
  \textbf{Теорема~3.}\ \textit{Предположим, что выполнено вероятностное 
условие, сформулированное в~теореме~$1$: $x(t_n)\hm= x\hm\in E_i^{(+)}$, 
$i\hm= 0,1,\ldots , N_0$. Тогда условные математические ожидания 
приращений функционала прибыли в~рассматриваемой стохастической 
полумарковской модели, определяемые соотношениями}~(\ref{e7-sh})  
\textit{и}~(\ref{e8-sh}), \textit{допускают следующие аналитические представления}:
  \begin{align*}
  d^{(+)}_{ik} &= \int\limits_{iL}^{(i+1)L} \left[ 
\int\limits_{\max\left(0,\sigma_{k+1}^{(+)}(x)\right)}^{\sigma_k^{(+)}(x)} 
\hspace*{-6mm}D_{ik}^{(+)}(x,u)\,dG_i(u)\right]\,dB_i(x)\\
  &\hspace*{41mm}\mbox{при } k=0,\ldots, k_i^{(+)}\,;\\
  d_{ik}^{(-)} &= \int\limits_{iL}^{(i+1)L} \left[ \int\limits_{\max\left(0,\sigma_k^{(-
)}(x)\right)}^{\sigma_{k+1}^{(-)}(x)} 
\hspace*{-6mm}D_{ik}^{(c-)} (x,u)\,dG_i(u)\right]\, dB_i(x)\\
  &\hspace*{13mm}\mbox{при } k=k_i^{(-)},\ldots , N_1\,,\ \  \alpha_w h-kL<0\,;\\
  d_{ik}^{(-)} &= \int\limits_{iL}^{(i+1)L} \left[ \int\limits_{\max \left(0, \sigma_k^{(-
)}(x)\right)}^{x/\alpha_c} \hspace*{-6mm}D_{ik}^{(w-)} (x,u) \,dG_i(u) +{}\right.\\
&\left.{}+
  \int\limits_{x/\alpha_c}^{\sigma^{(-)}_{k+1}(x)} \hspace*{-3mm}D_{ik}^{(c-)} 
(x,u)\,dG_i(u)\right] \, dB_i(x)\\
  & \hspace*{5mm}\mbox{при } k=k_i^{(-)},\ldots , N_1,\
\alpha_wh-kL\geq 0\,,\\
&\hspace*{37mm}\alpha_wh -(k+1)L<0\,;\\
  d_{ik}^{(-)} &= \int\limits_{iL}^{(i+1)L} \left[ \int\limits_{\max \left(0, \sigma_k^{(-
)}(x)\right)}^{\sigma^{(-)}_{k+1}(x)} 
\hspace*{-8mm}D_{ik}^{(w-)} (x,u)\,dG_i(u)\right]\,dB_i(x)\\
  &\hspace*{5mm}\mbox{при } k=k_i^{(-)},\ldots ,N_1, \enskip \alpha_wh-(k+1)L\leq 0\,,
  \end{align*}
\textit{где функции $D_{ik}^{(+)}(x,u)$, $D_{ik}^{(c-)}(x,u)$  
и~$D_{ik}^{(w-)}(x,u)$ аналитически определяются через известные функции 
доходов $d_1(x)$ и~$d_2(x)$ и~затрат
$c_1(x)$ и~$c_2(x)$, которые были введены в~разд.~$2$. В~связи 
с~громоздкостью соответствующих формул они приводятся 
в~приложении}~\cite{1-sh}.

\section{Заключение}

  Подведем краткие итоги первой части проведенного исследования, 
изложенной в~настоящей\linebreak
 работе. Построена и~идейно обоснована новая 
стохастическая полумарковская модель управ\-ле\-ния запасом непрерывного 
продукта. Проб\-ле\-ма управ\-ле\-ния формализуется в~виде задачи безусловного\linebreak 
экстремума стационарного стоимостного показателя~$I$, определяемого 
соотношением~(\ref{e3-sh}), который представляет собой среднюю удельную 
прибыль. Для получения явного представления этого показателя необходимо 
найти формулы для ряда вероятностных характеристик разработанной 
полумарковской модели. Такие формулы были доказаны в~данной работе.
  
   Для завершения решения задачи оптимального управления запасом 
в~предложенной полумарковской модели необходимо доказать, что 
стационарный стоимостный показатель эффективности управле\-ния~$I$ 
представим в~виде дроб\-но-ли\-ней\-но\-го интегрального функционала от набора 
вероятностных распределений, задающих стратегию управ\-ле\-ния. Тогда 
решение задачи управ\-ле\-ния будет основано на использовании тео\-ре\-мы об 
экстремуме дроб\-но-ли\-ней\-но\-го интегрального функционала, доказанной 
П.\,В.~Шнурковым~\cite{9-sh, 10-sh}. В~результате проб\-ле\-ма оптимального 
управления будет сведена к~исследованию на глобальный экстремум некоторой 
заданной функции от конечного чис\-ла действительных неотрицательных 
переменных. 

Соответствующие аналитические выводы и~доказательства 
составляют вторую часть проведенного исследования, которую авторы 
планируют изложить в~следующей статье. 

\vspace*{-10pt}
  
{\small\frenchspacing
 {%\baselineskip=10.8pt
 \addcontentsline{toc}{section}{References}
 \begin{thebibliography}{99}
  \bibitem{1-sh}
  \Au{Шнурков П.\,В., Егоров~А.\,Ю.} Приложение к~статье <<Разработка 
и~предварительное исследование стоха-\linebreak\vspace*{-12pt}

\columnbreak

\noindent
стической полумарковской модели управ\-ле\-ния 
запасом непрерывного продукта при постоянно происходящем потреблении>>, 2017. 49~с. 
{\sf http://www.\linebreak ipiran.ru/publications/Приложение\_upd.pdf}.
  \bibitem{2-sh}
  \Au{Королюк В.\,С., Турбин~А.\,Ф.} Полумарковские процессы и~их приложения.~--- Киев: 
Наукова думка, 1976. 184~с.
\bibitem{3-sh}
\Au{Janssen J., Manca R.} Applied semi-Markov process.~--- New York, NY, USA: Springer, 
2006. 309~p.
  \bibitem{4-sh}
  \Au{Джевелл~В.} Управляемые полумарковские процессы~// Кибернетический сборник. 
Новая серия.~--- М.: Мир, 1967. Вып.~4. С.~97--134.
\bibitem{5-sh}
\Au{Майн Х., Осаки~С.} Марковские процессы принятия решений~/ Пер. с~англ.~--- М.: 
Наука, 1977. 176~с. (\Au{Mine~H., Osaki~S.} Markovian decision processes.~--- New York, NY, 
USA: Elsevier, 1970. 142~p.)
  \bibitem{6-sh}
  \Au{Шнурков П.\,В., Иванов~А.\,В.} Анализ дискретной полумарковской модели 
управления запасом непрерывного продукта при периодическом прекращении 
потребления~// Дискретная математика, 2014. Т.~26. №\,1. С.~143--154. 
  \bibitem{7-sh}
  \Au{Шнурков П.\,В.} Стохастическая модель планового технического обслуживания~// 
  Стохастические сис\-те\-мы и~их приложения.~---
   Киев: Институт  математики АН УССР, 1990. С.~98--105.

  \bibitem{8-sh}
  \Au{Luque-Vasquez F., Herndndez-Lerma~O.} Semi-Markov control models with average 
costs~// Appl. Math., 1999. Vol.~26. No.\,3. P.~315--331.
  \bibitem{9-sh}
  \Au{Шнурков П.\,В.} О~решении проблемы безусловного экстремума для  
дроб\-но-ли\-ней\-но\-го интегрального функционала на множестве вероятностных мер~// 
Докл. РАН. Сер. Математика, 2016. Т.~470. №\,4. С.~387--392.
  \bibitem{10-sh}
  \Au{Шнурков П.\,В., Горшенин~А.\,К., Белоусов~В.\,В.} Аналитическое решение задачи 
оптимального управления полумарковским процессом с~конечным множеством состояний~// 
Информатика и~её применения, 2016. Т.~10. Вып.~4. С.~72--88.

 \end{thebibliography}

 }
 }

\end{multicols}

\vspace*{-6pt}

\hfill{\small\textit{Поступила в~редакцию 31.05.17}}

%\vspace*{8pt}

\newpage

\vspace*{-28pt}

%\hrule

%\vspace*{2pt}

%\hrule

%\vspace*{8pt}


\def\tit{DEVELOPMENT AND PRELIMINARY STUDY\\ OF~A~STOCHASTIC SEMI-MARKOV MODEL\\ 
OF~CONTINUOUS SUPPLY OF~PRODUCT MANAGEMENT\\ UNDER~THE~CONDITION OF~CONSTANT 
CONSUMPTION}

\def\titkol{Development and preliminary study of~a~stochastic semi-Markov model 
of~continuous supply of~product management}
% under~the~condition of~constant  consumption}

\def\aut{P.\,V.~Shnurkov and A.\,Y.~Egorov}

\def\autkol{P.\,V.~Shnurkov and A.\,Y.~Egorov}

\titel{\tit}{\aut}{\autkol}{\titkol}

\vspace*{-9pt}


\noindent
  National Research University Higher School of Economics, 34~Tallinskaya Str., Moscow 123458, 
Russian Federation



\def\leftfootline{\small{\textbf{\thepage}
\hfill INFORMATIKA I EE PRIMENENIYA~--- INFORMATICS AND
APPLICATIONS\ \ \ 2018\ \ \ volume~12\ \ \ issue\ 1}
}%
 \def\rightfootline{\small{INFORMATIKA I EE PRIMENENIYA~---
INFORMATICS AND APPLICATIONS\ \ \ 2018\ \ \ volume~12\ \ \ issue\ 1
\hfill \textbf{\thepage}}}

\vspace*{3pt}

  
  
\Abste{The paper deals with a discrete semi-Markov stochastic model describing the 
operation of a control system of continuous supply of product with constant 
consumption. The model is a couple of random processes ($x(t), \zeta (t)$) where the 
main process $x(t)$ describes the amount of stock in the system at time~$t$ and the 
accompanying random process is a semi-Markov process with a finite set of states. 
The optimal control problem is put in relation to the stationary indicators related to 
the accompanying process. This indicator is the average of the specific nature of the 
profits earned in the evolution of the initial inventory control system. An explicit 
analytical representation for the probability characteristics of semi-Markov models is 
obtained. In the future, the results will allow to find an explicit representation of the 
Quality Score and solving the problem of optimal control.}
  
  \KWE{inventory management; semi-Markov stochastic process; stationary value 
functional; optimal control of stochastic systems}
  
\DOI{10.14357/19922264180114}

%\vspace*{-12pt}

%\Ack
%\noindent




%\vspace*{3pt}

  \begin{multicols}{2}

\renewcommand{\bibname}{\protect\rmfamily References}
%\renewcommand{\bibname}{\large\protect\rm References}

{\small\frenchspacing
 {%\baselineskip=10.8pt
 \addcontentsline{toc}{section}{References}
 \begin{thebibliography}{99} 
  \bibitem{1-sh-1}
  \Aue{Shnurkov, P.\,V., and A.\,Y.~Egorov.} 2017. Prilozhenie k~stat'e 
``Razrabotka i~predvaritel'noe issle\-do\-va\-nie\linebreak
 stokhasticheskoy polumarkovskoy 
modeli uprav\-le\-niya\linebreak
 zapasom nepreryvnogo produkta pri po\-sto\-yan\-no 
pro\-is\-kho\-dya\-shchem potreblenii'' [Appendix to article ``Development and preliminary 
study of stochastic semi-Markov model of continuous supply of product management at 
constantly happening consumption'']. 49~p. Available at: {\sf 
http://www.ipiran.ru/publications/\linebreak Приложение\_upd.pdf} (accessed December~22, 2017).
  \bibitem{2-sh-1}
  \Aue{Korolyuk, V.\,S., and A.\,F.~Turbin.} 1976. \textit{Polumarkovskie 
protsessy i~ikh prilozheniya} [Semi-Markov processes and their applications]. Kiev: 
Naukova Dumka. 184~p.
  \bibitem{3-sh-1}
  \Aue{Janssen, J., and R.~Manca.} 2006. \textit{Applied semi-Markov process}. 
New York, NY: Springer. 309~p.
  \bibitem{4-sh-1}
  \Aue{Jewell, W.\,S.} 1963. Markov-renewal programming. \textit{Oper. Res.}  
11:938--971.
  \bibitem{5-sh-1}
  \Aue{Mine, H., and S.~Osaki.} 1970. \textit{Markovian decision processes}. New 
York, NY: Elsevier. 142~p.
  \bibitem{6-sh-1}
  \Aue{Shnurkov, P.\,V., and A.\,V.~Ivanov.} 2015. Analysis of a discrete  
semi-Markov model of continuous inventory control with periodic interruptions of 
consumption. \textit{Discrete Math. Appl.} 25(1):59--67.
  \bibitem{7-sh-1}
  \Aue{Shnurkov, P.\,V.}  1990. Stokhasticheskaya model' 
  planovogo tekhnicheskogo obsluzhivaniya [Stochastic model of scheduled maintenance]. 
  \textit{Stokhasticheskie sis\-te\-my i~ikh prilozheniya} 
  [Stochastic systems and their applications]. 
  Kiev: Institute of Mathematics of the Ukrainian Academy of Sciences. 98--105.

  \bibitem{8-sh-1}
  \Aue{Luque-Vasquez, F., and O.~Herndndez-Lerma.} 1999. Semi-Markov control 
models with average costs. \textit{Appl. Math.} 26(3):315--331.
  \bibitem{9-sh-1}
  \Aue{Shnurkov, P.\,V.} 2016. Solution of the unconditional extremum problem for 
a~linear-fractional integral functional on a~set of probability measures. 
\textit{Doklady Mathematics} 94(2):550--554.
  \bibitem{10-sh-1}
  \Aue{Shnurkov, P.\,V., A.\,K.~Gorshenin, and V.\,V.~Belousov.} 2016. 
Analiticheskoe reshenie zadachi optimal'nogo upravleniya polumarkovskim 
protsessom s~konechnym mnozhestvom sostoyaniy [An analytic solution of the 
optimal control problem for a semi-Markov process with a finite set of states]. 
\textit{Informatika i~ee Primeneniya~--- Inform. Appl.} 10(4):72--88.

\end{thebibliography}

 }
 }

\end{multicols}

\vspace*{-6pt}

\hfill{\small\textit{Received May 31, 2017}}

\vspace*{-18pt}
  
  \Contr
  
  \noindent
  \textbf{Shnurkov Peter V.} (b.\ 1953)~--- Candidate of Science (PhD) in physics and 
mathematics, associate professor, National Research University Higher School of Economics, 
34~Tallinskaya Str., Moscow 123458, Russian Federation; \mbox{pshnurkov@hse.ru}
  
  %\vspace*{3pt}
  
  \noindent
  \textbf{Egorov Artem Y.} (b.\ 1992)~--- Master student, National Research University 
Higher School of Economics, 34~Tallinskaya Str., Moscow 123458, Russian Federation; 
\mbox{eyupimenova@edu.hse.ru}

  
\label{end\stat}


\renewcommand{\bibname}{\protect\rm Литература} 