\def\stat{serebr+ataeva}

\def\tit{ОНТОЛОГИЯ ЦИФРОВОЙ СЕМАНТИЧЕСКОЙ БИБЛИОТЕКИ LibMeta}

\def\titkol{Онтология цифровой семантической библиотеки LibMeta}

\def\aut{О.\,М.~Атаева$^1$, В.\,А.~Серебряков$^2$}

\def\autkol{О.\,М.~Атаева, В.\,А.~Серебряков}

\titel{\tit}{\aut}{\autkol}{\titkol}

\index{Атаева О.\,М.}
\index{Серебряков В.\,А.}
\index{Serebryakov V.\,A.}
\index{Ataeva O.\,M.}




%{\renewcommand{\thefootnote}{\fnsymbol{footnote}} \footnotetext[1]
%{Работа выполнена при финансовой поддержке РФФИ (проект 17-01-00816).}}


\renewcommand{\thefootnote}{\arabic{footnote}}
\footnotetext[1]{Вычислительный центр им.\ А.\,А.~Дородницына Федерального исследовательского 
центра <<Информатика и~управ\-ле\-ние>> Российской академии наук, 
\mbox{oli@ultimeta.ru}}
\footnotetext[2]{Вычислительный центр им.\ А.\,А.~Дородницына Федерального исследовательского 
центра <<Информатика и~управ\-ле\-ние>> Российской академии наук, \mbox{serebr@ultimeta.ru}}
%\vspace*{-6pt}



\Abst{При разработке цифровых библиотек особое внимание уделяют модели данных 
содержимого библиотеки. При этом контент цифровых библиотек может быть описан 
различными форматами и~представлен различными способами. Библиотека, определяемая 
с~по\-мощью системы LibMeta, рассматривается как хранилище структурированных 
разнообразных данных с~возможностью их интеграции с~другими источниками данных 
и~предполагает возможность специфицирования своего контента за счет описания 
предметной области. В~качестве средства формализации выступает онтология контента 
семантической библиотеки. Также вводятся основные понятия для описания задачи 
интеграции данных из источников Linked Open Data (LOD), понятия для определения 
произвольного тезауруса. Онтология построена таким образом, чтобы иметь возможность 
определения семантической библиотеки в~произвольной предметной области.}

\KW{семантические библиотеки; модель данных; онтологии; источники данных; поиск 
в~LOD}

  \DOI{10.14357/19922264180101} 
  
%\vspace*{9pt}


\vskip 10pt plus 9pt minus 6pt

\thispagestyle{headings}

\begin{multicols}{2}

\label{st\stat}

\section{Введение}

     В различных предметных областях модель данных содержимого 
цифровых семантических биб\-лио\-тек может существенно отличаться как по 
типам ресурсов, так и~по их структуре. При разработке таких биб\-лио\-тек особое 
внимание уделяют модели данных содержимого биб\-лио\-теки.
     
     Говоря о библиотеках, авторы прежде всего \mbox{имеют} в~виду 
разработанную информационную сис\-те\-му для создания семантических 
библиотек LibMeta~[1--3], с~по\-мощью которой создается и~описывается 
семантическая биб\-лио\-те\-ка некоторой предметной об\-ласти. 
     
     LibMeta представляет собой информационную сис\-те\-му, которая 
реализует функциональность, необходимую для работы с~контентом 
семантической биб\-лио\-те\-ки. LibMeta не является традиционной сис\-те\-мой 
управления электронными библиотеками (СУЭБ). 

Развитие современных 
технологий подталкивает к~переопределению как понятия биб\-лио\-те\-ки, так 
и~контента биб\-лио\-те\-ки, в~качестве которых не обязательно могут выступать 
традиционные описания печатных изданий, но и~любые другие типы 
циф\-ро\-вых объектов. При этом контент циф\-ро\-вых биб\-лио\-тек может быть 
описан различными форматами и~пред\-став\-лен различными способами. 
Биб\-лио\-те\-ка, реализуемая с~помощью LibMeta, рассматривается как 
хранилище структурированных разнообразных\ данных с~воз\-мож\-ностью их 
интеграции с~другими источниками данных и~предполагает возможность 
специфицирования своего контента путем описания предметной об\-ласти. 
     
     Определение предметной области задается тезаурусом~[4], который 
содержит основные термины этой предметной об\-ласти, связанные 
иерархическими и~горизонтальными связями между собой. Содержимое 
биб\-лио\-те\-ки задается типами ресурсов, описание которых задает,
в~свою очередь, множество 
допустимых объектов, возможно объединенных в~разнообразные коллекции, 
со\-став\-ля\-ющие вместе с~тезаурусом ее контент.
     
     Статья посвящена исследованию средств пред\-став\-ле\-ния знаний 
о~контенте семантической биб\-лио\-те\-ки. Эти средства необходимы для 
автоматизации описания ресурсов биб\-лио\-те\-ки конкретной предметной 
области и~воз\-мож\-ности их автоматизированной интеграции с~данными 
внеш\-них открытых источников. Необходимым условием для этого является 
структуризация и~формализация знаний в~об\-ласти описания контента 
семантической биб\-лио\-теки. 
     
     При реализации LibMeta авторы руководствовались набором основных 
задач, которые должна решать разрабатываемая система:
     \begin{enumerate}[(1)]
\item библиотека должна поддерживать возможность использования 
медийных объектов или ссылки на них при описании своих объектов, 
включая текст, аудио- и~видеофайлы или любую их комбинацию. Это 
требование отражается в~названии словом <<цифровая>>;\\[-10pt]
\item типы используемых ресурсов и~связи между ними должны быть 
описаны средствами сис\-те\-мы в~рамках определенных в~предыду\-щей работе 
понятий, составляющих семантическое описание ресурсов контента 
биб\-лио\-те\-ки. При этом, согласно принципам LOD, при описании ресурсов 
поддерживается использование классов и~свойств ранее используемых 
онтологий в~сообществе, поддерживающем LOD. Эта поддержка 
выражается либо в~непосредственном использовании готовых онтологий 
при описании ресурсов и~связей между ними, либо в~возможности ссылок 
на их элементы, используя связи на уровне описания ресурсов. Это 
требование отражается в~названии словом <<семантическая>>;\\[-10pt]
\item библиотека должна служить интеграционным узлом, предоставляя 
возможность связывания своих данных с~данными из разных источников, 
которые включены в~облако LOD. Должна также обеспечиваться 
возможность извлекать данные этой биб\-лио\-те\-ки в~машиночитаемом 
формате. Это требование отражается в~на\-зва\-нии словом <<открытая>>;\\[-10pt]
\item пользователи биб\-лио\-те\-ки должны иметь возможность организовывать 
свои коллекции по интересующему их научному на\-прав\-ле\-нию, добавляя 
новые термины в~предметный тезаурус, уточняя таким образом об\-ласть 
своих интересов. Пользователи должны также иметь возможность 
осуществлять поиск не только среди объектов в~рамках сис\-те\-мы, но и~по 
источникам данных, без необходимости использования 
специализированного языка для поисковых запросов. Это требование 
отражается в~названии словом <<персональная>>.
\end{enumerate}

     Основные требования, предъявляемые при этом к~контенту  
системы,~--- \textit{универсальность}, \textit{структурированность}, 
\textit{адаптируемость}~--- не противоречат этим свойствам и~обеспечивают 
поддержку настраиваемого хранилища метаданных для объектов 
и~расширяемый набор информационных ресурсов. \textit{Универсальность} 
обеспечивает описания типов ее ресурсов и~объектов независимо от 
предметной об\-ласти и~об\-ласти интересов пользователей. 
\textit{Структурированность} описания обеспечивает\linebreak\vspace*{-12pt}

\columnbreak

\noindent
 поддержку связей 
между различными типами ресурсов как внутри сис\-те\-мы, так и~вне нее, 
исходя из определений LOD. \textit{Адаптируемость} описания ресурсов 
обеспечивает возможность добавления новых свойств и~связей в~процессе 
развития сис\-те\-мы и~обеспечивает настройку пользовательских интерфейсов 
под эти изменения. 
{ %\looseness=1

}
     
     В качестве средства формализации выступает онтология~\cite{5-ser} 
контента семантической биб\-лио\-те\-ки. На основе этого описания можно 
выделить основные понятия описания задачи интеграции данных из 
открытых источников.
     
     В качестве открытых источников рас\-смат\-ри\-ва\-ют\-ся источники данных, 
включенные в~LOD~\cite{6-ser} и~соответствующие основным 
требованиям, предъявляемым к~таким источникам данных.
     
     В~качестве основных разделов освещаемой задачи рассмотрим 
определение тезауруса и~основные стандарты, выделим понятия, 
необходимые для описания контента семантической биб\-лио\-те\-ки 
в~произвольной предметной об\-ласти, определим основные понятия, 
необходимые для описания задачи интеграции данных из открытых 
источников и~выделим основные типы связей между этими понятиями.

\section{Тезаурус и~стандарты}

     Для описания какой-либо предметной области всегда используется 
определенный набор терминов, каждый из которых обозначает или 
описывает ка\-кую-ли\-бо концепцию из этой предметной об\-ласти. 
Совокупность терминов, описывающих предметную область с~указанием 
семантических отношений (связей) между ними, является тезаурусом. Такие 
отношения в~тезаурусе всегда указывают на наличие смысловой 
(семантической) связи между терминами.
     
     При этом модель тезауруса не должна быть ориентирована ни на одну 
из конкретных предметных областей и~быть достаточно гибкой для того, 
чтобы позволить всегда сохранять актуальность словаря и~удобство его 
использования для определения любой предметной об\-ласти.
     
     Тезаурус с~наличием связей различных типов позволяет реализовать 
гибкий настраиваемый поиск, результатом которого будет список объектов 
предметной об\-ласти, со\-от\-вет\-ст\-ву\-ющий выбранным терминам. 
     
     Рассматриваемая в~статье модель тезауруса соответствует стандарту 
ISO~2788-1986. Этот стандарт определяет тезаурус как набор терминов, 
связанных между собой со\-от\-вет\-ст\-ву\-ющи\-ми связями (отношениями). 

Термины могут иметь следующие атрибуты:
     \begin{itemize}
\item $\mathrm{SN}$~--- Scope Note. Комментарий к~термину. Например, представляет 
вербальное пояснение термина или правила его использования;
\item $\mathrm{TT}$~--- Top Term. Признак, выделяющий термины на самом верхнем 
уровне иерархии (термины наиболее общих понятий в~иерархии понятий).
\end{itemize}

     Связи между терминами могут быть сле\-ду\-ющими:
     \begin{itemize}
\item $\mathrm{USE}$~--- связывает термин с~наиболее предпочтительным термином 
для понятия. $A$~$\mathrm{USE}$~$B$ означает, что термин~$B$ является наиболее 
предпочтительным для понятия, обозначаемого термином~$A$;
\item $\mathrm{UF}$~--- Used For. Обращение связи $\mathrm{USE}$. Связывает наиболее 
подходящий термин с~синонимами и~квазисинонимами (менее 
подходящими терминами);
\item $\mathrm{BT}$~--- Broader Term. Связь термина с~термином более общего 
понятия. $A$~$\mathrm{BT}$~$B$ означает, что термин~$B$ обозначает более общее 
понятие по сравнению с~понятием, обозначаемым термином~$A$;
\item $\mathrm{BTG}$~--- Broader Term Generic. Вариант связи~$\mathrm{BT}$ в~случае, 
когда 
термин характеризует разно\-вид\-ность понятия, определяемого более 
общим термином. Например, <<попугаи>> и~<<птицы>>. Наличие связи 
$\mathrm{BTG}$ подразумевает наличие связи~$\mathrm{BT}$; 
\item $\mathrm{BTP}$~--- Broader Term Partitive. Вариант связи $\mathrm{BT}$ в~случае, когда 
термин характеризует часть понятия, определяемого более общим 
термином. Например, <<математика>> и~<<тео\-рия чисел>>. Наличие 
связи $\mathrm{BTP}$ подразумевает наличие связи~BT; 
\item $\mathrm{NT}$, $\mathrm{NTG}$ и~$\mathrm{NTP}$~--- Narrower Term, 
Narrower Term Generic и~Narrower 
Term Partitive~--- обращение связей $\mathrm{BT}$, $\mathrm{BTG}$ и~$\mathrm{BTP}$ 
со\-от\-вет\-ст\-венно; 
{\looseness=1

}
\item $\mathrm{RT}$~--- Related Term. Ассоциативная связь. Используется для 
семантически связанных между собою терминов, не находящихся при 
этом в~одной иерархии и~не являющихся синонимами или 
квазисинонимами. Эта связь проставляется в~тех случаях, когда 
пользователю тезауруса может быть полезно осуществлять поиск или 
индексацию не только по данному термину, но и~по связанному с~ним.
\end{itemize}

\section{Онтология}

     Исходя из вышесказанного, тезаурус~--- это \mbox{полный} 
сис\-те\-ма\-ти\-зи\-ро\-ван\-ный набор терминов о~ка\-кой-ли\-бо об\-ласти знаний 
и~больше относится к~лексике, используемой в~конкретной об\-ласти, тогда 
как онтология описывает ресурсы предметной об\-ласти и~их взаимосвязи. Для 
каждой предметной об\-ласти набор ресурсов может отличаться как по 
формату, так и~по набору самих ресурсов. Поэтому, задавая определение 
самой библиотеки, предлагается использовать для описания ресурсов, 
со\-став\-ля\-ющих контент конкретной предметной об\-ласти, понятия, общие для 
любой из них, т.\,е.\ набор понятий, формирующих описание контента 
биб\-лио\-те\-ки, должен быть настолько универсальным, чтобы мог 
адаптироваться под нужды конкретной об\-ласти. 

Так как одной из основных 
задач, решаемых в~рамках биб\-лио\-те\-ки, как было сказано выше, является 
интеграция данных из различных источников, такой подход поз\-во\-ля\-ет 
реализовать средства интеграции данных в~рамках биб\-лио\-те\-ки, 
адап\-ти\-ру\-емые под условия любой предметной об\-ласти без оглядки на ее 
специфику.
     
     Понятия, составляющие онтологию библиотеки LibMeta, условно 
делятся на предназначенные для:
     \begin{itemize}
\item описания контента предметной об\-ласти;
\item формирования тезауруса любой предметной об\-ласти;
\item описания тематических коллекций; 
\item описания задачи интеграции контента биб\-лио\-те\-ки с~данными 
источников из LOD.
\end{itemize}

     Между этими группами понятий определены семантически значимые 
связи.
     
     Рассмотрим далее основные формальные определения, необходимые 
для описания онтологии.
     
     \smallskip
     
     \noindent
\textbf{Определение~1.} \textit{Контент библиотеки} $C\hm=\langle \mathrm{IR}, A, 
\mathrm{IO}\rangle$ определяется типами ее информационных ресурсов, описанных 
связанными с~ними наборами атрибутов~$A$ и~набором входных данных, 
опре\-де\-ля\-ющих информационные объекты~$\mathrm{IO}$, которые являются 
непосредственно объектами, хранящимися в~биб\-лио\-теке.
\smallskip

\noindent
\textbf{Определение~2.} \textit{Тезаурус библиотеки} $\mathrm{TH}\hm=\langle T, 
R\rangle$ определяется терминами~$T$ и~связями~$R$ между ними. Набор 
терминов~$T$, со\-став\-ля\-ющих описание предметной об\-ласти, строго задан.

\smallskip

\noindent
\textbf{Определение~3.} \textit{Семантические метки} $M\hm=\left\{ 
m_i\right\}$ информационного объекта~--- это термины, которые не попали 
в~тезаурус, но являются необходимыми для специфицирования тематики 
информационного объекта. Семантические метки не связаны, в~отличие от 
терминов тезауруса, связями между собой или с~терминами тезауруса, но 
дают возможность дополнительного тематического разделения 
информационных объектов в~рамках предметной об\-ласти.

\smallskip

\noindent
\textbf{Определение~4.} \textit{Задача интеграции данных биб\-лио\-те\-ки} 
$\mathrm{IT}\hm = \langle \mathrm{DS}, R, A, M, D, D_S\rangle$ \textit{с~внешними 
источниками}~$\mathrm{DS}$ определяется типами ресурсов биб\-лио\-те\-ки и~набором 
их атрибутов~$A$, отображением~$M$ ресурсов~$R$ на схему источника 
данных~$S$ и~набором связей~$D_S$ с~данными из источника.

\smallskip

\noindent
\textbf{Определение~5.} \textit{Коллекция информационных объектов} 
$C\hm= \langle \mathrm{IO}, T, M, \mathrm{DS}\rangle$ представляет собой набор объектов, 
объединенных на основе совокупности признаков:
\begin{enumerate}[(1)]
\item по их термину тезауруса предметной об\-ласти; 
\item по семантическим меткам; 
\item по источнику данных, из которого поступили объекты.
\end{enumerate}

В коллекцию могут входить объекты различных типов ресурсов, заданных 
при описании контента биб\-лио\-те\-ки. При этом коллекции по каждому 
признаку могут формироваться автоматически и~будем называть их 
автоматическими коллекциями. В~случае, когда признаки определяет 
пользователь, будем называть такие коллекции просто \textit{коллекциями.}

\smallskip

\noindent
\textbf{Определение~6.} \textit{Семантически значимыми связями 
библиотеки} $P\hm=\left\{P_i\right\}$ назовем связи, определенные между 
контентом библиотеки, ее предметной об\-ластью (тезаурусом), 
семантическими метками и~объектами источника данных. Выделим 
сле\-ду\-ющие основные связи:
\begin{itemize}
\item $P_1(t, \mathrm{io})$~--- термин те\-за\-у\-ру\-са\,--\,ин\-фор\-ма\-ци\-он\-ный 
объект;
\item $P_2(\mathrm{io}, t)$~--- информационный объ\-ект\,--\,тер\-мин тезауруса;
\item $P_3(r, s)$~--- информационный ре\-сурс\,--\,класс объектов 
источника, где информационный ресурс~--- это общее определение для 
информационных объектов, хранящихся в~сис\-те\-ме; таким образом, 
фактически информационные объекты являются экземплярами 
информационных ресурсов;
\item $P_4(a, s_a)$~--- атрибут информационного ре\-сур\-са\,--\,свой\-ст\-во 
класса источника;
\item $P_5(\mathrm{io}, o_s)$~--- информационный объ\-ект\,--\,эк\-земп\-ляр класса из 
источника данных;
\item $P_6(m, \mathrm{io})$~--- семантическая мет\-ка\,--\,ин\-фор\-ма\-ци\-он\-ный 
объект;
\item $P_7(\mathrm{io}, m)$~--- информационный объ\-ект\,--\,се\-ман\-ти\-че\-ская 
метка.
\end{itemize}

На основе введенных явных связей можно определить связи, которые 
назовем \textit{неявными значимыми связями} (т.\,е.\ заданными по 
некоторым определенным заранее правилам) между семантическими 
метками и~терминами тезауруса и~объектами как самой биб\-лио\-те\-ки, так 
и~экземплярами связанных данных из источников: 
\begin{itemize}
\item $P_8(m, t) \leftarrow P_6(m, \mathrm{io}) \wedge P_2(\mathrm{io}, t)$ семантическая 
мет\-ка\,--\,ин\-фор\-ма\-ци\-он\-ный объект\,--\,тер\-мин тезауруса;
\item $P_9(t, m) \leftarrow P_1(t, \mathrm{io}) \wedge P_7(\mathrm{io}, m)$ термин  
те\-за\-у\-ру\-са\,--\,ин\-фор\-ма\-ци\-он\-ный  
объ\-ект\,--\,се\-ман\-ти\-че\-ская метка;
\item $P_{10}(m, o_s) \leftarrow P_6(m, \mathrm{io}) \wedge P_5(\mathrm{io}, o_s)$ семантическая  
мет\-ка\,--\,ин\-фор\-ма\-ци\-он\-ный объект\,--\,эк\-земп\-ляр класса из 
источника данных;
\item $P_{11}(t, o_s) \leftarrow P_1(t, \mathrm{io}) \wedge P_5(\mathrm{io}, o_s)$ термин  
те\-за\-у\-ру\-са\,--\,ин\-фор\-ма\-ци\-он\-ный объект~--- экземпляр класса из 
источника данных.
\end{itemize}

     Для представления онтологии LibMeta был выбран язык описания 
онтологий OWL (Web Ontology Language)\footnote{{\sf https://www.w3.org/TR/owl-ref}.}. Такая онтология 
со\-сто\-ит из классов, свойств классов и~индивидов. В~терминах 
OWL~$P_1$~\textit{инверсивно}~$P_2$, $P_6$~\textit{инверсивно}~$P_7$, 
$P_8$~\textit{инверсивно}~$P_9$ и~$P_{10}$ \textit{инверсивно}~$P_{11}$. При этом 
правила для неявных связей задаются с~по\-мощью правил SWRL
(Semantic Web Rule Language)\footnote{ {\sf 
https://www.w3.org/Submission/SWRL}.}. Правила SWRL как расширение 
OWL помогают описать 
абстрактный механизм оперирования объектами предметной области и~ее 
закономерности. Правила SWRL дают возможность выводить новые факты из 
существующих утверж\-де\-ний, что повышает эффективность описания 
предметной об\-ласти.
     
     В соответствии с~определениями были введены основные классы 
онтологии. Исходя из определения~1, вводятся классы:
     \begin{enumerate}[1.]
\item $\mathrm{IResource}$ (информационный ресурс биб\-лио\-те\-ки), который 
содержит общую информацию о~типе ресурса, название, 
$URI$ (Universal Resource Identifier)\footnote{{\sf https://tools.ietf.org/html/rfc3986}.} и~информацию об 
ис\-поль\-зу\-емом наборе атрибутов для описания структуры ресурса.
\item $\mathrm{IObject}$ (информационный объект библиотеки), который 
фактически пред\-став\-ля\-ет собой экземпляр некоторого ресурса и~по 
составу ат-\linebreak рибутов соответствует набору атрибутов связанного с~ним 
ресурса. Для описания со\-от\-вет\-ст\-ву\-ющих значений для информационного 
объекта имеется многозначное свойство $\mathrm{value}$, значениями 
которого являются экземпляры вспомогательного класса $\mathrm{AttributeValue}$, 
содержащие информацию о конкретном значении объекта и~соответствующем атрибуте.
\item $\mathrm{Attribute}$ (атрибут, элемент описания информационного ресурса), 
который имеет следующие свойства:
\begin{itemize}
\item[(а)] $\mathrm{name}$~--- название;
\item[(б)] $\mathrm{type}$~--- содержит информацию о типе значений 
этого атрибута и~может включать такие значения, как 
\textit{строка}, \textit{число}, \textit{дата}, \textit{тип ресурса} 
(т.\,е.\ значениями являются объекты некоторого выбранного типа 
ресурса);
\item[(в)] $\mathrm{view}$~--- указывает на об\-ласть 
применения ат-\linebreak рибута в~рамках системы. Может иметь зна-\linebreak чения 
\textit{поисковый} 
(участвует в~формировании поисковых форм), 
\textit{иден\-ти\-фи\-ци\-ру\-ющий} (является обязательным) 
и~\textit{описательный} (содержит дополнительную информацию 
об описываемом объекте).
\end{itemize}
\item $\mathrm{AttributeSet}$ (набор атрибутов, группирующий атрибуты, 
со\-от\-вет\-ст\-ву\-ющие одному пред\-став\-ле\-нию ресурса).
\end{enumerate}

\begin{figure*}[b] %fig1
\vspace*{1pt}
 \begin{center}
 \mbox{%
 \epsfxsize=105.266mm 
 \epsfbox{ata-1.eps}
 }
 \end{center}
\vspace*{-9pt}
\caption{Пример описания информационного ресурса в~терминах онтологии LibMeta}
\end{figure*}

Исходя из определения~2, согласно описанному ранее стандарту  
ISO~2788-1986 для тезаурусов, вводятся классы:
\begin{enumerate}[1.]
\setcounter{enumi}{4}
\item $\mathrm{Thesaurus}$ (тезаурус предметной области)~--- содержит в~себе общую 
информацию о~тезаурусе: название и~авторов (организации и~персоны). 
Наличие этой сущности позволяет загружать готовые тезаурусы, не 
смешивая их с~теми, что уже, быть может, есть в~сис\-теме.
\item $\mathrm{Concept}$~--- сущность, содержащая информацию о~понятиях 
тезауруса. Содержит следующие атрибуты:
\begin{itemize}
\item[(a)] $\mathrm{Name}$~--- название понятия. В~случае, если понятие 
не может иметь названия, пред\-став\-лен\-но\-го в~виде текста, 
используется ка\-кой-ли\-бо идентификатор;
\item[(б)] $\mathrm{RepresentationType}$~--- тип представления понятия. 
Понятие не всегда можно описать словами, иногда для этого 
гораздо больше подходит формула или изображение, поэтому 
необходимо иметь возможность добавления понятия в~любом виде;
\item[(в)] $\mathrm{Image}$~--- изображение;
\item[(г)] $\mathrm{Note}$~--- примечание.
\end{itemize}
\item $\mathrm{ConceptGroup}$~--- тематическое разделение понятий тезауруса. 
\item $\mathrm{HierarchicalRel}$~--- иерархические связи, определяющие 
древовидную структуру словаря. Содержит атрибуты, определяющие 
связи в~соответствии со стандартом ($\mathrm{BT}$, $\mathrm{BTG}$, $\mathrm{BTP}$).
\item $\mathrm{FamilyRel}$~--- горизонтальные связи. Они задают родственные 
отношения между понятиями и~позволяют находить публикации по 
похожим тематикам. Содержит также атрибуты, определяющие связи 
в~соответствии со стандартом ($\mathrm{NT}$, $\mathrm{NTG}$, $\mathrm{NTP}$).
\item $\mathrm{PrefferedTerm}$~--- дескрипторы понятия. Каж\-до\-му понятию 
соответствует единственный дескриптор на каждом языке. 
\item $\mathrm{NonPrefferedTerm}$~--- сюда включаются синонимы. Один 
дескриптор может иметь множество синонимов. В~этот класс объектов 
добавлен атрибут $\mathrm{Visibility}$~--- свойство, от\-ве\-ча\-ющее за видимость 
термина. Имеет два значения~--- $\mathrm{global}$ и~$\mathrm{private}$, 
глобальная и~приватная об\-ласти видимости соответственно. Этот атрибут 
введен для решения проблемы множественных терминологий~--- разные 
люди могут называть одни и~те же объекты по-раз\-но\-му (пусть даже эти 
названия будут похожи). Для того чтобы каж\-до\-му пользователю было 
комфортно работать в~сис\-те\-ме, ему дается возможность создавать свои 
термины, если таковых нет в~глобальной части тезауруса. Эти термины он 
может связывать с~другими терминами из глобальной части и~размечать 
ими свои публикации. Таким образом, если два пользователя создали 
в~своих локальных репозиториях удобные для них ключевые слова, 
разметили ими свои публикации и~связали эти ключевые слова с~одним 
и~тем же термином из глобального тезауруса, то они смогут находить 
и~получать публикации друг друга, пользуясь при этом своими 
терминологиями.
\item $\mathrm{Term}$~--- общий класс, объединяющий дескрипторы и~синонимы. 
Содержит набор свойств, который при необходимости позволяет 
произвольно расширять текстовые описания терминов и~определять связи 
с~информационными объектами системы.
\end{enumerate}

Исходя из определений~3 и~5, вводятся классы:
\begin{enumerate}[1.]
\setcounter{enumi}{12}
\item  $\mathrm{SemanticTag}$~--- класс семантических меток, который обладает 
следующими свойствами:
\begin{itemize}
\item[(а)] $\mathrm{title}$~--- краткое название семантической метки;
\item[(б)] $\mathrm{description}$~--- расширенное описание 
семантической метки.
\end{itemize}
\item $\mathrm{ICollection}$~--- класс коллекций, определенных человеком, 
который обладает следующими свойствами:
\begin{itemize}
\item[(а)] $\mathrm{name}$~--- название коллекции;
\item[(б)] $\mathrm{definition}$~--- описание коллекции;
\item[(в)] $\mathrm{resources}$~--- типы ресурсов, включаемых в~эту 
коллекцию.
\end{itemize}
\end{enumerate}

Исходя из определения 4 вводятся классы:
\begin{enumerate}[1.]
\setcounter{enumi}{14}
\item $\mathrm{DataSource}$ (источники данных LOD)~--- класс, который имеет 
следующие свойства:
\begin{itemize}
\item[(а)]  $\mathrm{name}$~--- название источника;
\item[(б)] $\mathrm{description}$~--- описание источника;
\item[(в)] $\mathrm{url}$~--- точка входа для извлечения данных;
\item[(г)] $\mathrm{resourceMapping}$~--- содержит информацию о~типах 
ресурсов, отображаемых на этот источник, и~со\-от\-вет\-ст\-ву\-ющие классы 
источника. Значениями являются экземпляры класса 
$\mathrm{ResourceMapping}$.
\end{itemize}
\item $\mathrm{ResourceMapping}$~--- класс, содержащий информацию об 
отображаемых на источник данных информационных ресурсах 
библиотеки:
\begin{itemize}
\item[(а)] $\mathrm{resource}$~--- тип ресурсов, отображаемых на этот 
источник;
\item[(б)] $\mathrm{class}$~--- ссылка на соответствующий класс 
источника данных;
\item[(в)] $\mathrm{attributeMappings}$~--- содержит экземпляры класса 
$\mathrm{AttributeMapping}$, содержащих информацию об отображении 
со\-от\-вет\-ст\-ву\-ющих ресурсу атрибутов.
\end{itemize}
\item $\mathrm{AttributeMapping}$~--- класс, содержащий информацию об 
отображаемых на источник данных атрибутах из набора атрибутов, 
со\-от\-вет\-ст\-ву\-юще\-го информационному ресурсу библиотеки:
\begin{itemize}
\item[(а)] $\mathrm{attribute}$~--- атрибут, отображаемый на этот 
источник;
\item[(б)] $\mathrm{property}$~--- ссылка на соответствующее свойство 
класса источника данных.
\end{itemize}
\end{enumerate}

     На рис.~1 и~2 приведены примеры описания конкретного 
информационного ресурса и~информационного объекта в~терминах этой 
онтологии согласно определению~1. 



\begin{figure*} %fig2
\vspace*{1pt}
 \begin{center}
 \mbox{%
 \epsfxsize=155.106mm 
 \epsfbox{ata-2.eps}
 }
 \end{center}
\vspace*{-9pt}
\Caption{Пример описания информационного объекта в~терминах онтологии LibMeta}
\vspace*{6pt}
\end{figure*}

\section{Использование онтологии контента библиотеки и~тезауруса
предметной области при~конструировании семантической библиотеки 
в~LibMeta}

     Для применения тезауруса конкретной предметной 
об\-ласти и~онтологии контента биб\-лио\-те\-ки необходимо придерживаться 
сле\-ду\-ющей последовательности их использования при конструировании 
семантической биб\-лио\-те\-ки в~рамках LibMeta:
     \begin{enumerate}[(1)]
\item на основе введенной модели задается набор информационных 
ресурсов, ис\-поль\-зу\-емых в~биб\-лио\-те\-ке. Для этого необходимо пред\-ста\-вить 
описания содержимого будущей биб\-лио\-те\-ки в~терминах предложенной 
модели. На базе классов, заданных для описания контента биб\-лио\-те\-ки, 
реализован модуль, в~котором задаются базовые свойства, атрибуты для 
ресурсов и~связи между ними;
\item осуществляется окончательная настройка структуры тезауруса. На 
базе определенных классов согласно определению тезауруса реализован 
модуль для его по\-стро\-ения, в~котором задаются используемые связи 
между терминами, расширяется при необходимости описание термина, 
определяются связи с~ресурсами сис\-темы; 
\item для выбора семантических меток можно использовать 
дополнительные словари по предметной области или оставить 
возможность их определения (доопределения) позднее; 
\item на основе заданных классов согласно определению задачи 
интеграции реализован модуль, в~рамках которого осуществляется 
подключение внешних источников данных. Это действие можно 
выполнить на любом этапе жизнедеятельности системы;
\item на основе заданных классов согласно определению коллекций 
реализован модуль, в~рамках которого осуществляются создание 
коллекций и~их наполнение; это можно выполнить также на 
любом этапе.
\end{enumerate}

     На основе выполненных действий происходит автоматическая 
адаптация пользовательских интер\-фей\-сов системы под заданные описания 
ресурсов, со\-став\-ля\-ющих содержимое биб\-лио\-те\-ки. Пользовательский 
интерфейс делится услов\-но на сле\-ду\-ющие категории:
     \begin{itemize}
\item интерфейсы поиска;\\[-13.5pt]
\item интерфейсы просмотра;\\[-13.5pt]
\item интерфейсы редактирования;\\[-13.5pt]
\item интерфейсы загрузки данных.
\end{itemize}

\vspace*{-8pt}

     \subsection*{Пример}
     
     \vspace*{-1pt}
     
     На основе предложенной модели была сконструирована 
библиотека для предметной об\-ласти обыкновенных дифференциальных 
урав\-нений (ОДУ). В~качестве тезауруса использован тезаурус ОДУ, 
разработанный коллективом специалистов в~этой области~\cite{7-ser}. 
     
     Объектами библиотеки рассматривались журнальные математические 
статьи. В~качестве примеров типов ресурсов, соответственно, 
рассматривались \textit{Авторы} и~\textit{Публикации}. Был определен набор 
атрибутов для каждого типа ресурсов в~рамках минимального набора свойств 
на основе Dublin Core\footnote{{\sf  http://dublincore.org.}} для публикаций 
и~FOAF (Friend of a~Friend)\footnote{{\sf http://xmlns.com/foaf/spec.}} для описания авторов. В~качестве 
примера источника данных рассматривались данные о~персонах из сис\-те\-мы 
MathNet\footnote{{\sf http://www.mathnet.ru}.}, которые были смоделированы в~виде 
источника, интегрированного в~LOD. Были определены отобра\-же\-ния 
атрибутов \textit{Авторов} на свойства персон из этого источника 
и~выявлены связи у~почти~50\% авторов рас\-смат\-ри\-ва\-емых пуб\-ли\-ка\-ций, при 
этом авторов было около~700.
     Средствами системы для каждой публикации на основе ее названия, 
аннотации и~ключевых слов были выявлены связи с~тезаурусом ОДУ. 
В~качестве семантических меток были использованы термины 
математической энциклопедии\footnote{{\sf 
https://www.encyclopediaofmath.org.}}~\cite{8-ser}, что позволило дополнительно 
выявить смежные предметные области и~произвести дополнительное 
тематическое разбиение пуб\-ли\-ка\-ций в~рамках предметной области. Такое 
связывание позволило выявить с~некоторой долей вероятности статьи, 
относящиеся к~предметной области ОДУ, и~организовать их в~коллекции на 
основе тезауруса и~выявленных семантических меток. Было использовано 
описание около~2000~пуб\-ли\-ка\-ций, из них около~30\% были отнесены 
к~об\-ласти ОДУ и~имели связи со смежными предметными областями, 
выявленными согласно семантическим меткам.

\vspace*{-10pt}
     
\section{Дальнейшее направление работ}

\vspace*{-3pt}

Работа с~полными текстами предоставленных статей пока находится 
в~активной стадии. Предполагается создание информационного образа 
статей для выделения мик\-ро\-те\-зау\-ру\-са на основе семантических меток 
и~терминов предметной области по каж\-дой статье с~дальнейшим 
определением возможностей расширения используемых тезаурусов или для 
создания облака ключевых понятий отдельных областей знания. 

Отдельной 
задачей является семантическая обработка формул из полных текстов 
и~определение их ключевых слов с~воз\-мож\-ностью дальнейшего поиска по 
формулам, а~также выделение отдельных направлений и~математических школ. При 
этом формулы рас\-смат\-ри\-ва\-ют\-ся как отдельный тип ресурсов сис\-темы.

\vspace*{-6pt}
     
{\small\frenchspacing
 {%\baselineskip=10.8pt
 \addcontentsline{toc}{section}{References}
 \begin{thebibliography}{9}
 
 \vspace*{-2pt}

\bibitem{2-ser} %1
\Au{Серебряков В.\,А., Атаева~О.\,М.} Персональная цифровая библиотека LibMeta как 
среда интеграции связанных открытых данных~// Электронные библиотеки: 
перспективные методы и~технологии, электронные коллекции: Тр. XVI Всеросс. науч. 
конф. RCDL'2014.~--- Дубна: ОИЯИ, 2014. С.~66--71.
\bibitem{1-ser} %2
\Au{Серебряков~В.\,А., Атаева О.\,М.} Основные понятия формальной 
модели семантических биб\-лио\-тек и~формализация процессов интеграции в~ней~// 
Программные продукты и~сис\-те\-мы, 2015. №\,4. С.~180--187.

\bibitem{3-ser}
\Au{Серебряков В.\,А., Атаева~О.\,М.} Информационная модель открытой персональной 
семантической библиотеки LibMeta~// Научный сервис в~сети Интернет: Тр. XVIII 
Всеросс. науч. конф.~--- М.: ИПМ им.\ М.\,В.~Келдыша, 2016. С.~304--313.
\bibitem{4-ser}
\Au{Нгуен М.\,Х., Аджиев~А.\,С.} Описание и~использование тезаурусов 
в~информационных системах, подходы и~реализация~// Электронные библиотеки, 2004. 
Т.~7. №\,1. С.~16--45.
\bibitem{5-ser}
\Au{Gruber T.\,R.} A~translation approach to portable ontologies~// Knowl. Acquis., 
1993. Vol.~5. No.\,2. P.~199--220.
\bibitem{6-ser}
\Au{Bizer C., Heath~T., Berners-Lee~T.} Linked data~--- the story so far~// Int. J.~Semantic 
Web Inf., 2009. Vol.~5. No.\,3. P.~1--22.
\bibitem{7-ser}
\Au{Моисеев~Е.\,И., Муромский~А.\,А., Тучкова~Н.\,П.} Тезаурус  
ин\-фор\-ма\-ци\-он\-но-по\-иско\-вый по предметной области <<обыкновенные 
дифференциальные уравнения>>.~--- М.: МАКС Пресс, 2005. 116~с.
\bibitem{8-ser}
Математическая энциклопедия: В~5~т.~/ Гл.\ ред. И.\,М.~Виноградов.~--- М.: Советская 
энциклопедия, 1977.
 \end{thebibliography}

 }
 }

\end{multicols}

\vspace*{-6pt}

\hfill{\small\textit{Поступила в~редакцию 03.05.17}}

%\vspace*{8pt}

\newpage

\vspace*{-28pt}

%\hrule

%\vspace*{2pt}

%\hrule

%\vspace*{8pt}


\def\tit{ONTOLOGY OF~THE~DIGITAL SEMANTIC LIBRARY LibMeta}

\def\titkol{Ontology of~the~digital semantic library LibMeta}

\def\aut{V.\,A.~Serebryakov and O.\,M.~Ataeva}

\def\autkol{V.\,A.~Serebryakov and O.\,M.~Ataeva}

\titel{\tit}{\aut}{\autkol}{\titkol}

\vspace*{-9pt}


\noindent
A.\,A.~Dorodnicyn Computing Center, Federal Research Center ``Computer Science and 
Control'' of the Russian Academy of Sciences,  40~Vavilov Str., Moscow 119333, Russian 
Federation 



\def\leftfootline{\small{\textbf{\thepage}
\hfill INFORMATIKA I EE PRIMENENIYA~--- INFORMATICS AND
APPLICATIONS\ \ \ 2018\ \ \ volume~12\ \ \ issue\ 1}
}%
 \def\rightfootline{\small{INFORMATIKA I EE PRIMENENIYA~---
INFORMATICS AND APPLICATIONS\ \ \ 2018\ \ \ volume~12\ \ \ issue\ 1
\hfill \textbf{\thepage}}}

\vspace*{3pt}



\Abste{During development of digital libraries, рarticular attention is paid to the 
library content data model. In this case, the content of digital libraries can be 
described in various formats and presented in various ways. The library defined by 
the LibMeta system is considered as a storehouse of structured diverse data with 
the possibility of their integration with other data sources and assumes the 
possibility of specifying its content by describing the subject area. The ontology of 
the semantic library content serves as a means of formalization. It also introduces 
the basic concepts for describing the task of data integration from sources of 
Linked Open Data (LOD), concepts for defining an arbitrary thesaurus. The 
ontology is constructed in such a~way that it is possible to determine the semantic 
library in an arbitrary domain.}

\KWE{semantic library; data model; ontology; data source; search in LOD}

  \DOI{10.14357/19922264180101} 

%\vspace*{-12pt}

%\Ack
%\noindent



%\vspace*{3pt}

  \begin{multicols}{2}

\renewcommand{\bibname}{\protect\rmfamily References}
%\renewcommand{\bibname}{\large\protect\rm References}

{\small\frenchspacing
 {%\baselineskip=10.8pt
 \addcontentsline{toc}{section}{References}
 \begin{thebibliography}{9} 

\bibitem{2-ser-1}
\Aue{Serebryakov, V.\,A., and O.\,M.~Ataeva.} 2014. Personal'naya tsifrovaya 
biblioteka LibMeta kak sreda integratsii svyazannykh otkrytykh dannykh [Personal 
Digital Library Libmeta as an integration environment of linked data]. \textit{Tr. XVI 
Vseross. nauch. konf. RCDL'2014}
[16th All-Russia Scientific Conference RCDL'2014 Proceedings]. Dubna: OIYI. 66--71.
\bibitem{1-ser-1}
\Aue{Serebryakov, V.\,A., and O.\,M.~Ataeva.} 2015. Osnovnye ponyatiya  
formal'noy modeli se\-man\-ti\-che\-skikh bib\-lio\-tek i~formalizatsiya protsessov 
integratsii v~ney [The basic concepts of a~formal model of semantic libraries 
and formalization of the integration processes in it]. \textit{Programmnye produkty 
i~sistemy} [Software Systems] 4:180--187.

\bibitem{3-ser-1}
\Aue{Serebryakov, V.\,A., and O.\,M.~Ataeva.} 2016. Informatsionnaya model' 
otkrytoy personal'noy semanticheskoy biblioteki LibMeta
[Information model of the open
personal semantic library LibMeta]. 
\textit{Nauchnyy servis v~seti Internet: Tr.\ XVIII Vseross. nauch. konf.}   
[Scietifical service in the Internet: 18th All-Russia Scientific Conference Proceedings]. 
Moscow: IPM.  304--313.
\bibitem{4-ser-1}
\Aue{Nguen, M.\,H., and A.\,S.~Adzhiev.} 2004. Opisanie i~ispol'zovanie tezaurusov 
v~informatsionnykh sistemakh, podkhody i~realizatsiya [Description and use of thesauri 
in information systems, approaches and implementation].\linebreak
 \textit{Elektronnye biblioteki} 
[Digital Library] 7(1):16--45.
\bibitem{5-ser-1}
\Aue{Gruber, T.\,R.} 1993. A~translation approach to portable ontologies. 
\textit{Knowl. Acquis.} 5(2):199--220.
\bibitem{6-ser-1}
\Aue{Bizer, C., T.~Heath, and T.~Berners-Lee.} 2009. Linked data~--- the story so far. 
\textit{Int. J.~Semantic Web Inf.} 5(3):1--22.
\bibitem{7-ser-1}
\Aue{Moiseev, E.\,I., A.\,A.~Muromskiy, and N.\,P.~Tuchkova.} 2005. \textit{Tezaurus 
informatsionno-poiskovyy po predmetnoy oblasti ``obyknovennye differentsial'nye 
uravneniya''} [Information search with thesaurus in application area of ordinary 
differential equations].  Moscow: MAKS Press. 116~p.
\bibitem{8-ser-1}
Vinogradov, I.\,M., ed. 1977.
\textit{Matematicheskaya enciklopediya: V~5~t.} [Mathematical encyclopedia: In 5~vols.]. 
Moscow: Sovetskaya Entsiklopediya.

\end{thebibliography}

 }
 }

\end{multicols}

\vspace*{-6pt}

\hfill{\small\textit{Received May 3, 2017}}

%\vspace*{-10pt}

\Contr

\noindent
\textbf{Ataeva Olga M.} (b.\ 1978)~--- junior scientist, A.\,A.~Dorodnicyn 
Computing Centre, Federal Research Center 
``Computer Science and Control'' of the Russian Academy of Sciences, 
40~Vavilov Str., Moscow 119333, Russian Federation; \mbox{oli@ultimeta.ru}

\vspace*{3pt}


\noindent
\textbf{Serebryakov Vladimir A.} (b.\ 1946)~--- Doctor of Science in physics 
and mathematics, professor, Head of Department, A.\,A.~Dorodnicyn Computing 
Centre, Federal Research Center ``Computer Science and Control'' of the Russian 
Academy of Sciences, 40~Vavilov Str., Moscow 119333, Russian Federation; 
\mbox{serebr@ultimeta.ru}


\label{end\stat}


\renewcommand{\bibname}{\protect\rm Литература} 