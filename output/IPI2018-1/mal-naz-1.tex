\def \SS{{\frak S}}
\def \SK{{\frak K}}
\def \SL{{\frak L}}
\def\wl{\widetilde}


%\def\e{\exists}
%\def\c{\cdot}

\def\stat{mal-naz}

\def\tit{ДИАГРАММЫ УЯЗВИМОСТИ ПОТОКОВЫХ СЕТЕВЫХ СИСТЕМ}

\def\titkol{Диаграммы уязвимости потоковых сетевых систем}

\def\aut{Ю.\,Е.~Малашенко$^1$, И.\,А.~Назарова$^2$, Н.\,М.~Новикова$^3$}

\def\autkol{Ю.\,Е.~Малашенко, И.\,А.~Назарова, Н.\,М.~Новикова}

\titel{\tit}{\aut}{\autkol}{\titkol}

\index{Малашенко Ю.\,Е.}
\index{Назарова И.\,А.}
\index{Новикова Н.\,М.}
\index{Malashenko Yu.\,E.}
\index{Nazarova I.\,A.}
\index{Novikova N.\,M.}




%{\renewcommand{\thefootnote}{\fnsymbol{footnote}} \footnotetext[1]
%{Работа выполнена при поддержке РФФИ (проект 17-20-02200).}}


\renewcommand{\thefootnote}{\arabic{footnote}}
\footnotetext[1]{Вычислительный центр им.\ А.\,А.~Дородницына Федерального исследовательского центра 
<<Информатика и~управление>> Российской академии наук, \mbox{malash09@ccas.ru}}
\footnotetext[2]{Вычислительный центр им.\ А.\,А.~Дородницына Федерального исследовательского центра 
<<Информатика и~управление>> Российской академии наук, 
\mbox{irina-nazar@yandex.ru}}
\footnotetext[3]{Вычислительный центр им.\ А.\,А.~Дородницына Федерального исследовательского центра 
<<Информатика и~управление>> Российской академии наук, \mbox{n\_novikova@umail.ru}}


\vspace*{-6pt}


\Abst{Рассматривается метод анализа изменений функциональных возможностей 
сетевой системы  после  разрушающих воздействий.   
Для  описания процессов передачи  потоков взаимозаменяемых продуктов 
различным равноправным пользователям используется модель однопродуктовой  сети. 
Для  каждого  возможного  случая выхода из строя   физических или логических 
элементов системы вычисляются  оценки ущерба для всех пользователей. 
Под ущербом от повреждений  понимается величина невыполненных требований 
потребителей на объемы  поставок. В~качестве оценок используются  оптимальные  
решения последовательности  задач минимизации суммарных  относительных ущербов. 
На основании полученных  результатов  для различных конфигураций разрушений 
строятся диаграммы  уязвимости исходной  сетевой системы.  Диаграммы позволяют 
проводить априорный  анализ  как  небольших,  так и~критически  опасных повреждений, 
вследствие которых  потоки к~некоторым пользователям оказываются  равными нулю. 
Предложенный метод   может  быть  использован  при  исследовании  
струк\-тур\-но-функ\-цио\-наль\-ной   уязвимости  потоковых сетевых систем.}

\KW{однопродуктовая потоковая сеть; функциональная уязвимость; оценки ущерба}

\DOI{10.14357/19922264180102} 
  
\vspace*{-6pt}


\vskip 10pt plus 9pt minus 6pt

\thispagestyle{headings}

\begin{multicols}{2}

\label{st\stat}

\section{Введение}

В литературе в~качестве меры уязвимости обычно используется оценка снижения  
работоспособности системы вследствие полного или частичного  разрушения ее  
элементов~\cite{Mur13}. В~\cite{Naz03, Naz06} для перераспределения потоков после 
повреждения, оценки ущерба  пользователей и~поиска узких мест сети решалась 
последовательность задач дискретной оптимизации. В~\cite{Mal17} на примере 
однопродуктовой потоковой модели изучались возможности анализа изменений 
функциональных характеристик при понижении пропускной способности дуг  
сетевой системы.

В настоящей работе рассматривается многостоковая  потоковая система, 
структура которой задается графом. Функциональные возможности многополюсной  
сети определяются величинами потоков, которые могут  передаваться одновременно 
по всем стоковым дугам, направленным в~единый фиктивный узел-сток. Считается, 
что точно известен  вектор требований на передачу потока  каждому из пользователей 
сети. Повреждение задается подмножеством разрушенных дуг, пропускная способность 
которых полагается  равной   нулю.

В \cite{MalInf17} был предложен метод оценки изменения функциональных возможностей 
сети. Для по\-стро\-ения  гарантированных оценок~---  апостериори~---   
последовательно  определяется лексикографический векторный  максимин величин потоков, 
пе\-ре\-да\-ва\-емых в~сети после повреждения.   В~настоящей работе предлагается~--- априори~--- 
находить оценки  ущерба для всех пользователей сети при различных  по\-вреж\-да\-ющих 
воздействиях и~на их основе строить диаграммы уязвимости исходной системы. 
Перейдем к~описанию модели. 

\vspace*{-9pt}

\section{Модель сети}

Рассмотрим проблему оценки ущерба пользователей потоковой системы 
после крупномасштабных повреждений и/или целенаправленных разрушающих воздействий.
Следуя~\cite{ford}, сеть передачи единственного вида продукта, или 
однопродуктовую многополюсную сетевую систему, будем описывать ориентированным 
графом $\overline {\mathcal{G}} \hm= \langle \overline{\cal V}, \overline {\cal L} 
\rangle$ без петель, который определяется множеством вершин (узлов) $\overline{\cal V} 
\hm= \{v_1,v_2,\ldots,v_ N\}$, где $|\overline{\cal V}| \hm=  N$, 
и~множеством направленных дуг
$\overline {\cal L} \hm=\{l_{ij} \ | \ i \hm\in {\cal N}, \ j \hm\in {\cal N},
 i \not=$\linebreak $\not= j \}$,
соединяющих вершины, где
${\cal N}$~--- множество индексов вершин. Здесь
$l_{ij} \hm= (v_i, v_j)$~---  дуга, ведущая из вершины~$v_i$ в~вершину~$v_j$, 
 $|\overline{\cal L}|\hm = L$.

Обозначим через ${\cal V}_\SS$ и~${\cal V}_\SK$  
множества вершин графа~$\overline {\mathcal{G}}$, являющихся соответственно 
источниками и~стоками для потока, который передается по многополюсной  сети;
${\cal N}_\SS$ и~${\cal N}_\SK$~---  множества индексов вер\-шин-ис\-точ\-ни\-ков 
и~вер\-шин-сто\-ков:
${\cal V}_\SS = \{v_i | \ i \hm\in {\cal N}_\SS \}$,  $|{\cal V}_\SS|\hm = S$,  
${\cal V}_\SS \hm\subset 
{\cal V}$,  ${\cal N}_\SS \hm\subset {\cal N}$, $S \hm\geq 1$, 
${\cal V}_\SK \hm= \{v_i | \ i \in {\cal N}_\SK\}$,   
$|{\cal V}_\SK | \hm= K$, ${\cal V}_\SK \hm\subset {\cal V}$,  
${\cal N}_\SK \subset {\cal N}$, $K \hm\geq $1, 
${\cal V}_\SS \bigcap {\cal V}_\SK \hm= \emptyset.$
Считается, что на дугах графа $\overline {\mathcal{G}}$ заданы веса~--- 
значения $d_{ij}$ пропускной способности дуг~$l_{ij}$. Вектор~$d$ 
определяет максимально достижимую величину потока  по дугам,
$d\hm = \{d_{ij} \ |\  d_{ij} \hm\geq 0$,  $l_{ij}\hm\in \overline{\cal L}\}$.
К~графу~$\overline {\mathcal{G}}$ добавим:
\begin{description}
\item[\,] $v_0$~---  единственный источник потока бесконечной мощности  
и~дуги $(v_0, v_j)$,   $j \hm\in {\cal N}_\SS$, соединяющие~$v_0$ 
с~каж\-дым уз\-лом-ис\-точ\-ни\-ком. Для каждой дуги~$l_{0j}$ 
определим верхнее ограничение~$d_{0j}$, которое соответствует величине 
максимального потока из источника~$v_j$ в~сис\-те\-му. Будем считать, что  
дуги $(v_0, v_j)$, $j \hm\in 
{\cal N}_\SS$, являются ду\-га\-ми-ис\-точ\-ни\-ка\-ми, и~обозначим их множество через
$ \hat{\cal L}\hm=\{l_{0 j} \ | \  j \hm\in {\cal N}_\SS \}$, $| \hat{\cal L}| \hm=  S; $\\[-15pt]

\item[\,] $v_{N+1}$~---  единственный узел-сток бесконечного объема 
и~дуги $(v_i, v_{N+1})$,  $i \hm\in 
{\cal N}_\SK$, со\-еди\-ня\-ющие каждый узел-сток с~$v_{N+1}$.  
Для каж\-дой дуги~$l_{i(N+1)}$ определим верхнее ограничение~$d_{i(N+1)}$, 
которое соответствует величине\linebreak
 максимально возможного потока из системы в~$v_{N+1}$. 
Значение~$d_{i(N+1)}$ задает верхний предел для величины потока, который покидает 
систему по дуге из   узла~$v_i$.
Назовем  дуги $(v_i, v_{N+1})$, $i \hm\in {\cal N}_\SK$,  ду\-га\-ми-сто\-ка\-ми, 
или стоковыми дугами, и~обозначим их множество через
$\tilde{\cal L}\hm=\{\ l_{iN + 1} \ | \ i  \hm\in {\cal N}_\SK\}$, 
$ |\tilde{\cal L}| \hm=  K.$
Ориентированный граф, который определяется множествами вершин ${\cal V} \hm= \overline 
{\cal V} \bigcup \{v_0, v_{N+1}\}$ и~дуг 
${\cal L}\hm = \tilde{\cal L} \bigcup \overline {\cal L} \bigcup \hat {\cal L}$, 
обозначим ${\mathcal{G}} \hm= \langle {\cal V}, {\cal L} \rangle $.
\end{description}

Для графа ${\mathcal{G}}$ введем обозначения:
$x_{ij}$~--- поток по дуге~$l_{ij}$, $l_{ij}\hm\in {\cal L}$, протекающий 
в~соответствии с~ее направлением;
${\cal N}^{-}_j$~--- множество индексов уз\-лов-пред\-шест\-вен\-ни\-ков $j$-го 
(узлов, из которых исходят дуги, ведущие в~$j$-й узел), 
${\cal N}^{-}_j \hm\subset {\cal N} \cup \{0\} $;
${\cal N}^{+}_j$~--- множество индексов уз\-лов-по\-сле\-до\-ва\-те\-лей $j$-го 
(узлов, в~которые ведут дуги, исходящие из $j$-го узла), 
${\cal N}^{+}_j \hm\subset {\cal N} \cup \{N+1\}$.

Поток
$x= \langle x_{0j},\ldots, x_{ij},\ldots, x_{i(N+1)}\rangle$,  где

\noindent    
\begin{multline}
i \in {\cal N} \cup \{0\}, \ j \in {\cal N} \cup \{N+1\}, \ i \neq j, \ \ 
 l_{ij} \in {\cal L},\\
   \mbox{~и~если~}  i  = 0, \mbox{~то~} j \not = N+1,   
\label{e1-mal}
 \end{multline}
проходящий по дугам $l_{ij}\hm\in {\cal L}$, должен удовлетворять: 
\begin{itemize}
\item условию сохранения потока в~транзитных узлах, т.\,е.


\noindent
\begin{equation}
 \sum\limits_{i \in {\cal N}^{-}_j}^{}{x_{ij}}= 
 \sum\limits_{i \in {\cal N}^{+}_j}^{}{x_{ji}}\,,  \enskip  j \in 
{\cal N}\,,
\label{e2-mal}
\end{equation}
 \item ограничению на пропускную способность соответствующих дуг, т.\,е.
\begin{equation}
\hspace*{-8mm}0 \le x_{ij} \le d_{ij}, \enskip l_{ij}\in {\cal L},  
\mbox{~для~} i, j \mbox{~выполняется~(1)}.\! 
\label{e3-mal}
\end{equation}
\end{itemize}

Обозначим через ${\cal X}$ множество всех допустимых потоков в~сети,
$$
{\cal X} = \{x \ | \mbox{~выполняется~(1)--(3)}  \}\,.
$$

При анализе функциональных возможностей системы будем рассматривать 
потоки по стоковым дугам.
Последние перенумеруем по некоторому правилу натуральными числами от~1 до~$K$, 
т.\,е.\ установим взаимно однозначное  соответствие $l_{k} \hm= l _{j (N+1)}$,
$k\hm = \overline{1, K}$, $j \hm \in {\cal N}_\SK.$

Пусть $d_{k}$~--- пропускная способность стоковой дуги~$l_{k}$,
$d_{k} \hm= d _{j (N+1)},$ $ j \hm\in {\cal N}_\SK,$
$\overline x_{k}$~--- величина потока по стоковой дуге~$l_{k}$,
$\overline x_{k} \hm= x _{j (N+1)}$,  $j \hm\in {\cal N}_\SK. $
Таким образом, вектор
$ \overline x \hm= \langle \overline x_{1}, \ldots, \overline x_{k}, 
\ldots, \overline x_{K}\rangle$
покомпонентно определяет величины потоков, которые передаются по каждой 
стоковой дуге сети  в~соответствии с~некоторым допустимым потоком $x 
\hm\in {\cal X}$.
Обозначим множество всех допустимых векторов величин потоков 
по стоковым дугам~$\overline x$ через
\begin{equation*}
\overline{\cal X} = \left\{ \overline x \ | \ \overline x_{k} = 
x_{j (N+1)}\,, \ k = \overline{1, K}\,, \  j \in 
{\cal N}_\SK\,, \  x \in {\cal X} \right\}. 
%\label{e4-mal}
\end{equation*}

Предположим, что за каждой стоковой дугой стоит некий абстрактный пользователь 
(потребитель), имеющий определенное требование (запрос)~$f_k$, 
$k \hm= \overline{1, K}$, на обеспечение продуктом, который передается по сети. 
Вектор
$ f \hm= \langle f_{1}, \ldots, f_{k}, \ldots, f_{K}\rangle$, $0 \hm\le f_{k} \hm\le 
d_k$, $k \hm= \overline{1, K},$
описывает требования всех потребителей сети.  Считается, что недопоставка продукта 
одному пользователю не может быть восполнена за счет поставки дополнительного 
объема продукта другому пользователю. Указанное предположение делает пользователей 
сети в~определенном смысле равноправными, а~их требования и~потоки~--- 
невзаимозаменяемыми. 

\vspace*{-9pt}

\section{Оценка ущерба поврежденной~сети}

Пусть до момента~$t_0$ однопродуктовая сеть работала в~некотором стационарном режиме 
и~по дугам сети передавался поток 
$x^0\hm= \langle x_{0j}^0,\ldots, x_{ij}^0,\ldots, x_{i(N+1)}^0\rangle$, 
$x^0 \hm\in {\cal X}$, в~том числе по стоковым дугам~--- вектор  $\overline x^0 
\hm= \langle \overline x_1^0, \ldots, \overline x_k^0,\ldots,  
\overline x_K^0\rangle$, $\overline x^0 \hm\in \overline{\cal X}$, 
$\overline x_k^0 \hm> 0$, $k \hm= \overline{1, K}.$
Поток~$x^0$ будем называть исходным, а~величины компонент~$x_{ij}^0$~--- 
исходными значениями~$x^0$ до повреждающего воздействия.

Пусть стационарно работающая однопродуктовая сеть подверглась в~момент~$t_0$ 
воздействию~$W$, в~результате которого были разрушены несколько дуг~$l_{ij}$. 
В~рамках рассматриваемой постановки предполагается, что могут быть повреждены 
любые дуги, кроме стоковых. Однако, какие дуги сети будут разбиты, заранее не 
известно, а~множество вершин  остается неизменным. Пропускная способность 
разрушенных дуг полагается равной нулю. Обозначим через~${\cal L}(W)$ 
множество по\-вреж\-ден\-ных дуг. Тогда
${\cal L}(W) \hm\subset [\overline{\cal L} \bigcup \hat {\cal L}], 
{\cal L}(W) \bigcap  \tilde{\cal L}\hm = \emptyset;$
${\mathcal{G}}(W)$~--- граф сети; ${\cal S}(W) \hm= \langle {\mathcal{G}}(W); 
f \rangle$~--- сеть после разрушающего воздействия~$W$.
Таким образом, воздействие полностью определяется множеством 
дуг~${\cal L}(W)$. Пусть
$d({\cal W})$~--- вектор пропускной способности дуг поврежденной сети~${\cal S}(W)$,  
для каждой компоненты~$d_{ij}(W)$ которого выполняется:
\begin{multline}
 d_{ij}(W) ={}\\
 {}=
\begin{cases}
0 , & \mbox{если~} \ l_{ij} \in {\cal L}(W), \ 
{\cal L}(W) \subset [\overline {\cal L} \bigcup \hat {\cal L}]\,;  \\
d_{ij}, & \mbox{если~} \ l_{ij} \in {\cal L}  \backslash {\cal L}(W).
\end{cases}
\label{e5-mal}
\end{multline}

Для допустимого потока~$x(W)$ 
после разрушения должно выполняться~(\ref{e1-mal}). 
Распределение потоков по дугам сети~${\cal S}(W)$ 
описывается компонентами  вектора
$x (W)=\langle x_{0j}(W),\ldots, x_{ij}(W),\ldots, x_{i(N+1)}(W)\rangle$.  

Для любых допустимых потоков~$x(W)$ в~поврежденной сети~${\cal S}(W)$ 
должны выполняться стандартные ограничения
\begin{equation}
 0 \le x_{ij}(W) \le d_{ij}(W), 
 \label{e6-mal}
\end{equation}
где $d_{ij}(W)$ определяется~(\ref{e5-mal}), $l_{ij}\in 
{\cal L}$,
и закон сохранения потока в~каждом узле
\begin{equation}
 \sum\limits_{i \in {\cal N}^{-}_j}^{}{x_{ij}(W)}= \sum\limits_{i \in 
{\cal N}^{+}_j}^{}{x_{ji}(W)},  \enskip  j \in {\cal N}\,. 
\label{e7-mal}
\end{equation}
Множество ${\cal X}(W)$  допустимых потоков  в~по\-вреж\-ден\-ной сети~${\cal S}(W)$ 
определяется условиями~(\ref{e1-mal}), (\ref{e6-mal}) и~(\ref{e7-mal}):
$$
{\cal X}(W) = \{x(W) \ | \mbox{~выполняется~(1), (5), (6)} \}\,.
$$

Пусть $\overline x_k(W)$~--- величина потока по $k$-й стоковой дуге после 
разрушающего воздействия. Тогда
$\overline x(W)$~--- вектор величин потоков по всем стоковым дугам,
$\overline x(W) \hm= \langle \overline x_1(W), \ldots, \overline x_k(W),
\ldots,  \overline x_K(W)\rangle.$
Множество
\begin{multline*}
\overline{\cal X}(W) = \left\{ \overline x(W) \ | \ \overline x_{k}(W) = 
x _{j (N+1)}(W),\right.\\ 
\left.k = \overline{1, K},  \ j \in 
{\cal N}_\SK, \ \ x(W) \in {\cal X} (W)\right\} 
\end{multline*}
описывает достижимые величины потоков по стоковым дугам после разрушения.

В дальнейших рассуждениях значения компонент вектора требований потоков по 
стоковым дугам сети  положим равными соответствующим исходным величинам до 
повреждения, т.\,е.\ $ f_k \hm=  \overline x_k^0$, $k \hm= \overline{1, K}$.
 Кроме того, поскольку считается, что недопоставка продукта одному пользователю 
 не может быть восполнена за счет поставки дополнительного объема продукта 
 другому, то выполняется $\overline x_k(W)\hm\le \overline x_k^0$, $k \hm= 
 \overline {1, K}$.
Введем величину~$u_k(W)$, численно равную разности между исходным значением, равным
требованию $k$-го пользователя, и~значением потока, протекающего по $k$-й стоковой дуге после разрушающего воздействия:
\begin{multline*}
u_k(W) = f_k -  \overline x_k(W) = \overline x_k^0 - \overline x_k(W)\,, \\
 \overline x_k(W)\le \overline x_k^0\,,  \enskip k = \overline {1, K}\,,
\end{multline*}
которую назовем недопоставкой потока $k$-му  потребителю, или $k$-й недопоставкой.
Вектор $u(W)\hm = \langle u_1(W), \ldots, u_k(W), \ldots, u_K(W)  \rangle$
описывает возможные недопоставки всем пользователям сети \mbox{после} повреждения.

Назовем  ущербом  $k$-го потребителя отношение величины недопоставки $u_k(W)$ к~соответствующему  запрошенному количеству потока 
$f_k \hm=  \overline x_k^0$, $k \hm= \overline{1, K}$, 
и~обозначим эту величину через
\begin{multline*}
\omega_k(W) = \fr{u_k(W)}{f_k}= \fr{f_k - \overline x_k(W)}{f_k}\,, \\ 
0 \le  \omega_k \le 1\,,  \enskip k = \overline{1, K}\,.
\end{multline*}
Для оценки суммарного ущерба пользователей  решим следующую  задачу оптимизации.

\smallskip

\noindent
\textbf{Задача}~{\boldmath{$A$}}. Найти
$$
 \Omega^* =  \min\limits_{x(W) \in {\cal X}(W)} 
\sum\limits_{k = \overline{1, K}} \left(\fr{f_k - \overline x_k(W)}{f_k}\right)^2 
$$
при условии
$f_k\hm =  \overline x_k^0 \neq 0, \enskip k = \overline{1, K}.$

Оптимальное решение задачи~$A$ обозначим~$\Omega^*$. Если $\Omega^* \hm= 0$, 
то исходный вектор~$\overline x^0$ можно передать в~поврежденной сети.
В~противном случае значение~$\Omega^*$ равно минимальной сумме квадратов 
ущерба пользователей от повреждения~$W$.
Поток, обеспечивающий решение задачи~--- достижение минимального значения~$\Omega^*$, 
обозначим~$x^*(W)$, $x^*(W) \hm\in 
{\cal X}(W)$. Вектору $x^*(W)$ соответствует
вектор~$\overline {x}^*(W)$, принадлежащий множеству~$\overline {\cal X}(W)$, 
каж\-дая компонента которого отвечает величине потока по стоковой дуге. 
В~случае, когда $\Omega^* \hm\neq 0$, $\overline {x}^*(W)$ является точкой, 
ближайшей к~вектору требований~$f$.

Для каждой $k$-й стоковой дуги ($k = \overline{1, K}$) вычислим 
величину~$ \omega_k^*(W)$~--- значение ущерба $k$-го  пользователя сети 
при разрушении~$W$:
$$
\omega_k^*(W)=\fr{f_k - \overline x_k^*(W)}{f_k}
$$
при условии 
$f_k \hm=\overline x_k^0 \hm\neq 0$, $k \hm= \overline{1, K}$. 

\pagebreak

Следуя~\cite{Mal99}, отношение величины потока, протекающего по $k$-й стоковой 
дуге после разрушения, к~исходному обозначим через~$\theta_k^*(W)$:
$$
 \theta_k^*(W) = \fr{\overline x_k^*(W)}{f_k} = 1  -  \omega_k^*(W)
 $$  
 при условии 
 $f_k =  \overline x_k^0\hm\neq 0$, $k \hm= \overline{1, K}$.
Величину $\theta_k^*(W)$ будем называть мерой обеспеченности требований $k$-го 
потребителя после разрушающего воздействия~$W$.

Для фиксированного повреждения~$W$ определим величины 
максимального~$\overline {\omega}^*(W)$  
и~минимального~$\underline{\omega}^*(W)$ ущерба пользователей сети. По определению
$$
 \overline {\omega}^*(W) = \max\limits_{k = \overline{1, K}} \omega_k^*(W)\,; 
 \enskip \underline {\omega}^*(W) = \min\limits_{k = \overline{1, K}} \omega_k^*(W)\,.
 $$
Тогда
\begin{multline*}
 \overline {\theta}^*(W) = \max\limits_{k = \overline{1, K}} \theta_k^*(W) = 
\max\limits_{k = \overline{1, K}}\left(1 - \omega_k^*(W)\right) ={}\\
{}= 1 - 
\min\limits_{k = \overline{1, K}}\omega_k^*(W) = 1 - \underline {\omega}^*(W)\,; 
\end{multline*}

\vspace*{-12pt}

\noindent
\begin{multline*}
\underline {\theta}^*(W) = \min\limits_{k = \overline{1, K}} \theta_k^*(W) = 
\min\limits_{k = \overline{1, K}}\left(1 - \omega_k^*(W)\right) ={}\\
{}= 1 - 
\max_{k = \overline{1, K}}\omega_k^*(W) = 1 - \overline {\omega}^*(W)\,.  
\end{multline*}
Заметим, что пользователей, получивших максимальный (минимальный) 
ущерб от повреждения~$W$ и,~соответственно, имеющих минимальную (максимальную) 
обеспеченность требований,  в~зависимости от конфигурации сети может 
быть любое число от~1 до~$K$.

Еще одним важным показателем для фиксированного повреждения~$W$ является 
медиана ущерба~$\wl{\omega}(W)$. Для построения медианы~$\wl{\omega}(W)$  
для каждого повреждения~$W$ определим такое значение~$\wl{\omega}(W)$, 
что последнее делит множество значений~$\omega_k^*(W)$ для выбранного~$W$ 
на две равные части, т.\,е. 
\begin{itemize}
\item в~половине~случаев~значения $\omega_k^*(W) \hm\le \wl{\omega}(W)$;
\item
в~другой~половине~случаев $\wl{\omega}(W) \hm\leq \omega_k^*(W)$.
\end{itemize}
При нечетном~$k$ в~качестве оценки медианы выберем  среднее по порядку значение, 
а~при четном~--- полусумму двух средних по порядку значений. Медиану обеспеченности 
требований~$\wl{\theta}(W)$ определим аналогично~$\wl{\omega}(W)$. 

\section{Диаграммы структурно-функциональной уязвимости}

В настоящей работе под мощностью повреждения~$W$ будем 
понимать число разрушенных дуг. Напомним, что разбиты могут быть любые дуги, 
кроме стоковых. Рассмотрим все повреждения, имеющие мощность~3.
Пусть $R(3)$~--- общее число таких повреждений:
$$
R(3) = \fr{(L+S) (L+S - 1) (L+S - 2)}{6}\,. 
$$
Перенумеруем по некоторому правилу все по\-вреж\-де\-ния, имеющие мощность~3, 
и~обозначим через~$\overline{\cal W}(3)$ множество  всех таких повреждений: 
$$
\overline{\cal W}(3) = \left\{ \overline W_1, \ldots, 
\overline {W}_w, \ldots, \overline {W}_{R(3)}\right\}. 
$$
Каждое из повреждений $\overline {W}_w \hm\in \overline{\cal W}(3)$ 
пол\-ностью определяется тройкой разрушенных дуг сети $(l_{ab}, l_{ij}, l_{km})$.
Таким образом, устанавливается  взаимно однозначное соответствие между 
тройками разрушенных дуг и~номером~$w$ разрушающего воздействия~$\overline {W}_w$.
Для каждого повреждения $\overline {W}_w \hm\in \overline{\cal W}(3)$ решим задачу~$A$ и~вычислим вектор величин потоков $\overline {x}^*(\overline {W}_w)$, 
далее для $k = \overline{1, K}$ найдем значения переменных~$\omega_k^*(\overline{W}_w), 
\theta_k^*(\overline{W}_w)$  и~со\-от\-вет\-ст\-ву\-ющие $\overline {\omega}^*
(\overline {W}_w)$,  $\underline{\omega}^*(\overline {W}_w)$, 
$\wl {\omega}^*(\overline{W}_w)$, \ $\overline {\theta}^*(\overline {W}_w)$, 
$\underline {\theta}^*(\overline {W}_w)$ и~$\wl {\theta}^*(\overline {W}_w)$.

Исходя из полученных значений $\underline {\theta}^*(\overline {W}_w)$ 
и~$\overline {\omega}^*(\overline {W}_w)$, $w \hm= \overline{1, R(3)}$, 
переставим и~пе\-ре\-обозначим элементы множества $\overline{\cal W}(3)$ 
так, чтобы\linebreak значения $\underline {\theta}^*(\overline {W}_w)$ 
были упорядочены по неубыванию. Тогда первым элементом множества 
${\cal W}(3) \hm= \{ W_1, \ldots W_w, \ldots, W_{R(3)} \}$ 
будем считать такое разрушение мощности~3~--- тройку $(l_{ab}, l_{ij}, l_{km})$,\linebreak 
которое обеспечивает минимальное значение $\underline {\theta}^*(\overline {W}_w)$ 
и,~соответственно, максимальное значение $\overline {\omega}^*(\overline {W}_w)$ 
среди всех $\overline {W}_w \in \overline{\cal W}(3)$:
$$
W_1 = \overline {W}_j:  \underline {\theta}^*(\overline {W}_j) = 
\min\limits_{\overline {W}_w \in \overline{\cal W}(3)} \underline {\theta}^*
\left(\overline {W}_w\right). 
$$
При этом будем считать, что для всех $k \hm= \overline{1, K}$   
выполнено  $\omega_k^*(W_1) \hm= \omega_k^*(\overline {W}_{j})$, 
$\theta_k^*( W_1) \hm= \theta_k^*(\overline {W}_{j})$
и~$\overline {\omega}^*(W_1) \hm= \overline {\omega}^*(\overline {W}_{j})$, 
$\underline {\omega}^*( W_1)\hm = \underline {\omega}^*(\overline {W}_{j})$, 
$\wl {\omega}^*(W_{1})\;=$
$=\;\wl {\omega}^*(\overline {W}_{j})$, 
$\overline {\theta}^*(W_{1}) \hm= \overline {\theta}^*(\overline {W}_{j})$, 
$\underline {\theta}^*(W_{1}) \hm= \underline {\theta}^*(\overline {W}_{j})$ 
и~$\wl {\theta}^*( W_{1}) \hm= \wl {\theta}^*(\overline {W}_{j})$.

Аналогично
$$
W_{2}= \overline {W}_g:  \underline {\theta}^*(\overline {W}_g) = 
 \min\limits_{\overline {W}_w \in [\overline {\cal W}(3) 
 \backslash \{\overline {W}_{j}\} ]} \underline {\theta}^*\left(\overline {W}_w\right)\,. 
 $$
 
 \vspace*{-3pt}

И~так далее.

Данные о повреждениях~--- множестве ${\cal W}(3)$~--- представлены на рис.~1 
в~виде диаграммы, построенной по следующему правилу. На рис.~1 
проведена горизонтальная жирная 
линия, проходящая через точку $(0, 1)$.
По оси абсцисс отложены числа
$$
 \fr{1}{R(3)}, \fr{2}{R(3)}, \fr{3}{R(3)}, \ldots,  \fr{w}{R(3)}, \ldots,   
 \fr{R(3)}{R(3)},
 $$
 
%\vspace*{-3pt}



 { \begin{center}  %fig1
\vspace*{1pt}
\mbox{%
 \epsfxsize=71.885mm 
 \epsfbox{mal-1.eps}
 }


\vspace*{6pt}


\noindent
{{\figurename~1}\ \ \small{Диаграмма ущерба}}
\end{center}
}



\vspace*{9pt}

\addtocounter{figure}{1}

 


\noindent
с~помощью которых определяется ширина столбцов, соответствующих номерам повреждений, 
а~именно:  повреждению~$W_1$ соответствует отрезок $(0, {1}/{R(3)})$, 
повреждению~$W_2$~--- отрезок 
$({1}/{R(3)}, {2}/{R(3)})$, \ldots, повреждению $W_{R(3)}$~--- 
отрезок  $({(R(3) - 1)}/{R(3)}, {R(3)}/R(3))$. 
Высота столбца по оси ординат численно равна величине~$\underline {\theta}^*(W_w)$, 
$w\hm = \overline{1, R(3)}$, и~отмечена линией со стрелкой.  Высота %\linebreak 
столбца от линии со стрелкой до горизонтальной жирной линии численно равна 
$\overline {\omega}^*(W_w) \hm= 1 \hm-   \underline {\theta}^*(W_w)$, 
$w \hm= \overline{1, R(3)}. $
Штриховые линии соответствуют значениям $\wl {\theta}^*(W_w)$,\linebreak 
а~линии со звездочками~--- значениям $\overline {\theta}^*(W_w)$, 
$w \hm= \overline{1, R(3)}$.
На  рис.~1 величина максимального ущерба пользователей после повреждения 
любых трех дуг сети слева направо не увеличивается. 
При этом $\underline {\theta}^*(W_w)$, $w \hm= \overline{1, R(3)}$, 
численно равна доле требований, гарантированно выполненных для 
любого пользователя при повреждении $W_w \hm\in  {\cal W}(3)$,  
и~эти значения слева направо не убывают.

В литературе ущерб пользователей при по\-вреж\-де\-ни\-ях трактуется как мера уязвимости: 
чем больше ущерб, тем более уязвимой считается сеть. 
И~обратно~--- чем больше поток, который гарантированно можно передать по сети 
после повреждения, тем сеть менее уязвима. Диаграмму на рис.~1 
будем называть VINV(3)-диа\-грам\-мой (по числу поврежденных дуг и~от 
английских слов vulnerable~--- уязвимый, invulnerable~---  неуязвимый). 
Обозначим через~$\underline {\mathrm{vinv}}(3)$ ку\-соч\-но-по\-сто\-ян\-ную 
ступенчатую линию, отвечающую значениям $\underline {\theta}^*(W_w)$, 
$w \hm= \overline{1, R(3)}$. Выше линии  $\underline {\mathrm{vinv}}(3)$  
лежит область значений ущерба, который может быть нанесен 
пользователям сети, а~ниже~--- область, отвечающая доле требований, которые 
гарантированно могут быть до\-став\-ле\-ны пользователям при повреждениях,
 имеющих мощность три.

Обозначим через $\overline {\mathrm{vinv}}(3)$ ку\-соч\-но-по\-сто\-ян\-ную 
ступенчатую линию со звездочками, отвечающую значениям $\overline {\theta}^*(W_w)$, 
$w \hm= \overline{1, R(3)}$.
Выше линии~$\overline {\mathrm{vinv}}(3)$ лежит область, отвечающая пользователям 
с~<<минимальным>>  ущербом, или тем, кто получил максимальную долю от 
требований при имеющихся повреждениях сети. Между кривыми $\underline {\mathrm{vinv}}(3)$ 
и~$\overline {\mathrm{vinv}}(3)$ лежит область, поз\-во\-ля\-ющая анализировать  ущерб пользователей, 
опираясь на ступенчатую пунктирную линию значений медиан 
$\wl {\theta}^*(W_w), \wl {\omega}^*(W_w)$. Таким образом, 
с~по\-мощью кривых~$\underline {\mathrm{vinv}}(3)$ и~$\overline {\mathrm{vinv}}(3)$ 
можно анализировать спо\-соб\-ность/не\-спо\-соб\-ность системы обеспечивать передачу 
определенной доли потока в~разрушенной\linebreak сети, т.\,е.\
 судить об уяз\-ви\-мости/не\-уяз\-ви\-мости исследуемой сети в~терминах 
 сохраненный по\-ток\,/\,по\-не\-сен\-ный ущерб, а~также выявлять критически  опасные 
 повреждения, вследствие которых  потоки к~некоторым пользователям оказываются  
 равными нулю. Заметим, что процедуру, описанную выше для построения 
 VINV(3)-диаграммы, можно использовать для изучения функциональных возможностей 
 сети после повреждения любого числа дуг.
 {\looseness=-1
 
 }
 




На рис.~2 представлена диаграмма, поз\-во\-ля\-ющая сравнивать последствия 
разрушений по двум расчетным  параметрам:  максимальный ущерб  и~величина  
медианы ущерба. По оси абсцисс отложены значения медиан, по оси ординат~--- 
соответствующие значения максимального ущерба. Каждое разрушение  отмечено   
крестиком,  крестики в~кружочках соответствуют  угловым точкам, лежащим на 
выпуклой оболочке множества всех повреждений мощности три. Повреждения, 
принадлежащие выпуклой оболочке, будем называть эффективными, поскольку 
их показатели  не доминируются  сразу по всем параметрам другими 
разрушающими воздействиями.

К построению VINV($n$)-диа\-грам\-мы для $n \hm\geq 1$ можно подойти иначе. 
Формирование VINV($n$)-\linebreak\vspace*{-12pt}

{ \begin{center}  %fig2
 \vspace*{9pt}
\mbox{%
 \epsfxsize=65.351mm 
 \epsfbox{mal-2.eps}
 }


\vspace*{6pt}


\noindent
{{\figurename~2}\ \ \small{Диаграмма эффективных повреждений}}
\end{center}
}



%\vspace*{9pt}

\addtocounter{figure}{1}

\noindent
{ \begin{center}  %fig3
 \vspace*{-1pt}
\mbox{%
 \epsfxsize=71.885mm 
 \epsfbox{mal-3.eps}
 }


\end{center}


\noindent
{{\figurename~3}\ \ \small{Диаграмма ущерба для различной мощности повреждения}}

}



\vspace*{14pt}

\addtocounter{figure}{1}

\noindent
диа\-грам\-мы можно начинать с~анализа разрушения одной, 
двух, \ldots, $n$ дуг,  или до получения критических повреждений, 
при этом рассматривая каждую предыдущую диаграмму как детализацию сле\-ду\-ющей.
Остановимся на этой вычислительной схеме последовательного построения подробнее.
Обозначим через~$\zeta_{ij}$ отношение исходной величины потока~$x_{ij}^0$ 
по выбранной дуге~$l_{ij}$ к~ее пропускной способности~$d_{ij}$ и~назовем 
указанную величину загруженностью дуги~$l_{ij}$:

\vspace*{2pt}

\noindent
\begin{equation}
\zeta_{ij} = \fr{x_{ij}^0}{d_{ij}}\,,
\label{e8-mal}
\end{equation}

\vspace*{-2pt}

\noindent
где $l_{ij} \in [\overline 
{\cal L} \bigcup \hat {\cal L}]$, $x^0 \hm\in {\cal X}$.



Вычислим загруженность~$\zeta_{ij}$ всех дуг сети, кроме стоковых.
Затем, переупорядочив полученные~$\zeta_{ij}$ по невозрастанию, построим 
VINV($n$)-диа-\linebreak грам\-мы в~первую очередь для дуг с~б$\acute{\mbox{о}}$льшими 
значени\-ями~$\zeta_{ij}$.

Далее исследование уязвимости сети можно про\-во\-дить по двум сценариям. 
В~первом случае~--- выбирать пары, тройки и~т.\,д.\  
разрушенных дуг исходя из вычисленных значений загруженности $\zeta_{ij}$, 
$l_{ij} \hm\in [\overline {\cal L} \bigcup \hat {\cal L}]$. Во втором~--- 
корректировать значения загруженности дуг исходя из повреждений\linebreak
 сети, 
а~именно: для вычисления загруженности не\-по\-вреж\-ден\-ной дуги~$l_{ij}$ 
после фиксированного разруше\-ния~$W$ в~формуле~(\ref{e8-mal}) 
вместо исходного значения~$x_{ij}^0$ компоненты вектора потока~$x^0$  
использовать соответствующую компоненту~$x_{ij}^*(W)$ вектора~$x^*(W)$~--- 
решения задачи~$A$, $x^*(W) \hm\in 
{\cal X}(W)$:

\vspace*{3pt}

\noindent
\begin{equation}
\zeta_{ij}(W) = \fr{x_{ij}^*(W)}{d_{ij}},
\label{e9-mal}
\end{equation}

\vspace*{-3pt}

\noindent
где $l_{ij} \in 
[\overline {\cal L} \bigcup \hat {\cal L}] \backslash {\cal L}(W)$, $x^*(W) \hm\in 
{\cal X}(W).$



Далее выбор очередной дуги для анализа разрушений производить 
исходя из полученных~$\zeta_{ij}(W)$,\linebreak\vspace*{-12pt}

\columnbreak

\noindent
 выбирая, как и~выше, дуги 
с~б$\acute{\mbox{о}}$льшими значениями. Таким образом, для поиска 
каждой следующей дуги множества~${\cal L}(W)$ предлагается использовать 
формулу~(\ref{e9-mal}), а~дальнейший анализ уязвимости сети проводить 
исходя из решения задачи~$A$.
Пример возможного расположения кривых $\underline {\mathrm{vinv}}(1)$, 
$\underline {\mathrm{vinv}}(2)$, $\underline {\mathrm{vinv}}(3)$ приведен на рис.~3.

\vspace*{-6pt}

\section{Заключение}

Предложенный метод позволяет проводить априорный анализ уязвимости сетевой 
системы для произвольного фиксированного вектора требований и~любых разрушений.
Для вычисления ущерба  пользователей сети при  удалении произвольных наборов   
дуг с~помощью эффективных алгоритмов~\cite{Yen} решается задача~$A$, причем 
столько раз, сколько потребуется для получения значимого описания.  

На основании полученных   результатов после анализа и~преобразований  
строятся  подробные  диаграм\-мы  оценок  возможного  ущерба для различных  
значений мощности повреждений.
Диаграммы на качественном уровне демонстрируют\linebreak
 зависимость функциональных 
характеристик от структурных повреждений. Особенно важно,  что  диаграммы точно 
отражают критические по\-вреж\-де\-ния, при которых потоки по некоторым стоковым дугам 
становятся равными нулю. Появление   нулевых значений в~век\-то\-ре-ре\-ше\-нии  задачи~$A$  
указывает на утрату  возможности передачи потока соответствующему пользователю.  
Анализ списка дуг, приводящих к~эффективным и~критическим повреждениям, позволяет 
выявить  слабые места и~структурные особенности сетевой системы.
Более детальное  исследование  струк\-тур\-но-функ\-цио\-наль\-ной уяз\-ви\-мости  и~получение 
гарантированных оценок ущерба можно проводить с~помощью ресурсоемких 
методов~\cite{MalInf17}. 

\vspace*{-6pt}


{\small\frenchspacing
 {%\baselineskip=10.8pt
 \addcontentsline{toc}{section}{References}
 \begin{thebibliography}{9}

\bibitem{Mur13} \Au{Murray A.\,T.} 
An overview of network vulnerability modeling approaches~// GeoJ., 2013. 
Vol.~78. P.~209--221.

\bibitem{Naz03} \Au{Назарова И.\,А.} Лексикографическая задача анализа уязвимости
многопродуктовой сети~// Изв. РАН. \mbox{ТиСУ}, 2003. №\,5. С.~123--134.

\bibitem{Naz06} \Au{Назарова~И.\,А. } Модели и~методы решения задачи анализа
уязвимости сетей~// Изв. РАН. ТиСУ, 2006. №\,4. С.~61--72.

\bibitem{Mal17} \Au{Козлов М.\,В., Малашенко~Ю.\,Е., Назарова~И.\,А.,
Новикова~Н.\,М.} 
Управление   топ\-лив\-но-энер\-ге\-ти\-че\-ской  сис\-те\-мой  
при  крупномасштабных повреждениях. I.~Сетевая  модель  и~программная реализация~// 
Изв. РАН. ТиСУ, 2017. №\,6. С.~50--73.

\bibitem{MalInf17} \Au{Малашенко~Ю.\,Е., Назарова~И.\,А., Новикова~Н.\,М.} 
Метод анализа функциональной уязвимости потоковых сетевых систем~// Информатика 
и~её применения, 2017. Т.~11. Вып.~4. С.~47--54.

\bibitem{ford} \Au{Форд Л., Фалкерсон~Д.} Потоки в~сетях~/ Пер. с~англ.~--- 
М.: Мир, 1966. 277~с. (\Au{Ford~L.\,R.,   Fulkerson~D.\,R.} Flows in networks.~--- 
Princeton, NJ, USA: Princeton University Press, 1962. 332~p.)

\bibitem{Mal99} \Au{Малашенко Ю.\,Е., Новикова~Н.\,М. } 
Модели неопределенности в~многопользовательских сетях.~--- 
М: Эдиториал УРСС, 1999. 160~с.

\bibitem{Yen} \Au{Йенсен П., Барнес~Д.} 
Потоковое программирование~/ Пер. с~англ.~--- М.: Радио и~связь, 1984. 392~с. 
(\Au{Jensen~P.\,A., Barnes~J.\,W. } Network flow programming.~--- 
New York, NY, USA: Wiley, 1980. 408~p. )

 \end{thebibliography}

 }
 }

\end{multicols}

\vspace*{-6pt}

\hfill{\small\textit{Поступила в~редакцию 04.12.17}}

\vspace*{4pt}

%\newpage

%\vspace*{-24pt}

\hrule

\vspace*{2pt}

\hrule

\vspace*{-4pt}


\def\tit{DIAGRAMS OF THE FUNCTIONAL VULNERABILITY OF~FLOW~NETWORK~SYSTEMS}

\def\titkol{Diagrams of the functional vulnerability of flow network systems}

\def\aut{Yu.\,E.~Malashenko, I.\,A.~Nazarova, and N.\,M.~Novikova}

\def\autkol{Yu.\,E.~Malashenko, I.\,A.~Nazarova, and N.\,M.~Novikova}

\titel{\tit}{\aut}{\autkol}{\titkol}

\vspace*{-11pt}


\noindent
A.\,A.~Dorodnicyn Computing Center, Federal Research Center 
``Computer Science and Control'' of the Russian Academy of Sciences, 
40~Vavilov Str., Moscow 119333, Russian Federation



\def\leftfootline{\small{\textbf{\thepage}
\hfill INFORMATIKA I EE PRIMENENIYA~--- INFORMATICS AND
APPLICATIONS\ \ \ 2018\ \ \ volume~12\ \ \ issue\ 1}
}%
 \def\rightfootline{\small{INFORMATIKA I EE PRIMENENIYA~---
INFORMATICS AND APPLICATIONS\ \ \ 2018\ \ \ volume~12\ \ \ issue\ 1
\hfill \textbf{\thepage}}}

\vspace*{3pt}



\Abste{The method of analysis of changes in functional 
capabilities of the flow network system after the damaging effects 
is considered. To describe the processes of streams of different interchangeable 
products to various peer users, the model of single-product network is used. 
The damage is defined as the total amount of unsatisfied demands. 
Damage estimates are calculated for all users and all possible cases of 
failure of physical or logical elements of the system. Optimal solutions 
of the sequence of problems of minimization of total relative damages are 
used as damage estimates. Based on the obtained results for various destruction 
configurations, the vulnerability diagrams of the initial network system are 
constructed. Diagrams allow an \textit{a~priori} analysis of both small and critical 
damages, due to which the flows to some users vanish. The proposed method can 
be used to study the structural and functional vulnerability of flow network systems.}

\KWE{single-product flow network; functional vulnerability; damage assessment}




\DOI{10.14357/19922264180102} 

%\vspace*{-12pt}

%\Ack



%\vspace*{3pt}

  \begin{multicols}{2}

\renewcommand{\bibname}{\protect\rmfamily References}
%\renewcommand{\bibname}{\large\protect\rm References}

{\small\frenchspacing
 {%\baselineskip=10.8pt
 \addcontentsline{toc}{section}{References}
 \begin{thebibliography}{9}
\bibitem{1-mal-1}
\Aue{Murray, A.\,T.} 2013.  
An overview of network vulnerability modeling approaches. \textit{GeoJ.}
78:209--221.
 
\bibitem{2-mal-1}
\Aue{Nazarova, I.\,A.} 2003. Lexicographical problem of the analysis of the 
vulnerability of a multicommodity network.
\textit{J.~Comput. Syst. Sci. Int.} 42(5):778--789.   

\bibitem{3-mal-1}
\Aue{Nazarova, I.\,A.} 2006. 
Models and methods for solving the problem of network vulnerability. 
\textit{J.~Comput. Syst. Sci. Int.} 45(4):567--578.

\bibitem{4-mal-1}
\Aue{Kozlov, M.\,V., Yu.\,E.~Malashenko, I.\,A.~Nazarova, and N.\,M.~Novikova.} 
2017. Fuel and energy system control at large-scale damages. 
I.~Network model and software implementation.  
\textit{J.~Comput. Syst. Sci. Int.} 56(6):945--968. 

\bibitem{5-mal-1}
\Aue{Malashenko, Yu.\,E., I.\,A.~Nazarova, and N.\,M.~Novikova.} 
2017. Metod analiza funktsional'noy uyazvimosti potokovykh setevykh system 
[Method of the analysis of the functional vulnerability of flow network systems]. 
\textit{Informatika i~ee Primeneniya~--- Inform. Appl.} 11(4):50--73.

\bibitem{6-mal-1}
\Aue{Ford, L.\,R., and D.\,R.~Fulkerson.} 1962.  Flows in networks. 
Princeton, NJ: Princeton University Press. 332~p.

\bibitem{7-mal-1}
\Aue{Malashenko, Ju.\,E., and N.\,M.~Novikova.} 1999. Modeli neopredelennosti 
v~mnogopol'zovatel'skikh setyakh [Indeterminacy models in the multiuser networks].  
Moscow: Editorial URSS Publ. 160~p.

\bibitem{8-mal-1}
\Aue{Jensen, P.\,A., and J.\,W.~Barnes.} 1980. Network flow programming. 
New York, NY: Wiley. 408~p.

\end{thebibliography}

 }
 }

\end{multicols}

\vspace*{-9pt}

\hfill{\small\textit{Received December 4, 2017}}

%\pagebreak

\vspace*{-18pt}

\Contr

\noindent
\textbf{Malashenko Yuri E.} (b.\ 1946)~--- 
Doctor of Science in physics and mathematics, Head of Laboratory, 
A.\,A.~Dorodnicyn Computing Center, Federal Research Center 
``Computer Science and Control'' of the Russian Academy of Sciences, 
40~Vavilov Str., Moscow 119333, Russian Federation; \mbox{malash09@ccas.ru}

%\vspace*{5pt}

\noindent
\textbf{Nazarova Irina A.} (b.\ 1966)~--- 
Candidate of Science (PhD) in physics and mathematics, scientist, 
A.\,A.~Dorodnicyn Computing Center, Federal Research Center 
``Computer Science and Control'' of the Russian Academy of Sciences, 
40~Vavilov Str., Moscow 119333, Russian Federation; \mbox{irina-nazar@yandex.ru}

%\vspace*{5pt}

\noindent
\textbf{Novikova Natalya M.} (b.\ 1953)~--- 
Doctor of Science in physics and mathematics, professor; leading scientist, 
A.\,A.~Dorodnicyn Computing Center, Federal Research Center 
``Computer Science and Control'' of the Russian Academy of Sciences, 
40~Vavilov Str., Moscow 119333, Russian Federation; \mbox{n\_novikova@umail.ru}
\label{end\stat}


\renewcommand{\bibname}{\protect\rm Литература} 