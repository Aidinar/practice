\def\re{{\mathrm{э}}}
\def\aa{{\cal A}}
\def\jj{{\cal J}}

\def\stat{sinitsin}

\def\tit{МЕТОД ИНТЕРПОЛЯЦИОННОГО АНАЛИТИЧЕСКОГО 
МОДЕЛИРОВАНИЯ ОДНОМЕРНЫХ РАСПРЕДЕЛЕНИЙ В~СТОХАСТИЧЕСКИХ СИСТЕМАХ}

\def\titkol{Метод интерполяционного аналитического 
моделирования одномерных распределений в~стохастических системах}

\def\aut{И.\,Н.~Синицын$^1$}

\def\autkol{И.\,Н.~Синицын}

\titel{\tit}{\aut}{\autkol}{\titkol}

\index{Синицын И.\,Н.}
\index{Sinitsyn I.\,N.}




%{\renewcommand{\thefootnote}{\fnsymbol{footnote}} \footnotetext[1]
%{Работа выполнена при финансовой поддержке РФФИ (проект 17-01-00816).}}


\renewcommand{\thefootnote}{\arabic{footnote}}
\footnotetext[1]{Институт проблем
информатики Федерального исследовательского центра
<<Информатика и~управление>> Российской академии наук, \mbox{sinitsin@dol.ru}}

%\vspace*{-6pt}


\Abst{Среди известных методов аналитического моделирования одномерных 
распределений стохастических процессов (СтП) в~дифференциальных 
и~ин\-тег\-ро\-диф\-фе\-рен\-ци\-аль\-ных стохастических системах (СтС), основанных 
на прямом численном решении уравнения для одномерной характеристической функции (х.ф.), 
важное место занимают интерполяционные методы, восходящие к~работам С.\,В.~Мальчикова. 
При этом для интерполирования х.ф.\ использовалась теорема отсчетов В.\,А.~Котельникова.
В~статье для двух типов скалярных нелинейных дифференциальных негауссовских СтС 
на базе уравнения для одномерной х.ф.\ разработан  метод  интерполяционного
аналитического моделирования (МИАМ). Получены уравнения для функций влияния 
(чувствительности) параметров, входящих в~уравнения СтС. Приведенный 
тестовый пример показывает эффективность метода для существенно нелинейной 
СтС при небольшом числе членов интерполяционного разложения. Формулируются 
некоторые обобщения метода.}


\KW{одномерная плотность вероятности (п.в.);
одномерная характеристическая функция (х.ф.);
стохастическая система (СтС);
стохастический процесс (СтП)}

\DOI{10.14357/19922264180107} 
  
%\vspace*{6pt}


\vskip 10pt plus 9pt minus 6pt

\thispagestyle{headings}

\begin{multicols}{2}

\label{st\stat}

\section{Введение}

Среди известных методов аналитического моделирования СтП 
в~дифференциальных и~ин\-тег\-ро-диф\-фе\-рен\-ци\-аль\-ных СтС, 
основанных на прямом численном решении уравнения для одномерной 
х.ф.~[1], важное место занимают интерполяционные методы~[2, 3]. 
При этом для интерполирования х.ф.\ используется теорема отсчетов В.\,А.~Котельникова~[4].

Рассмотрим обобщение результатов С.\,В.~Мальчикова~[2, 3] на 
случай скалярных нелинейных дифференциальных СтС с~винеровскими и~пуассоновскими 
компонентами. Особое внимание уделим вопросам чувствительности метода. 
Применение метода проиллюстрируем тестовым примером.

\section{Уравнения скалярных нелинейных дифференциальных стохастических систем}


Пусть скалярный СтП  $Y(t) \hm= Y_t$ определяется скалярным стохастическим 
дифференциальным уравнением Ито следующего вида~[1]:

\noindent
\begin{multline}
dY_t = a \left(Y_t, \Theta, t\right) dt + b \left(Y_t, \Theta, t\right) d W_0 
    (\Theta, t)+ {}\\
    {}+\iii_{R_0^1} c\left(Y_t, \Theta, t, v\right) 
    P^0 (\Theta, dt, dv),\enskip Y\left(t_0\right)=Y_0\,.
    \label{e2.1-s}
    \end{multline}
Здесь $W_0 = W_0 (\Theta, t)$~--- скалярный винеровский СтП 
интенсивности $\nu_0\hm=\nu_0 (\Theta, t)$; $\Theta$~--- 
$n^\Theta$-мер\-ный вектор па\-ра\-мет\-ров; $P^0(\Theta, \jj, \aa)$ 
 представляет собой для любого множества~$\aa$ 
 простой пуассоновский СтП, причем 
 $$
 P^0 (\Theta, \jj,\aa)\hm= P(\Theta, \jj,\aa)
 \hm- \mu_P (\Theta, \jj,\aa),$$
  где $\mu_P (\Theta, \jj,\aa)$~--- 
 его математическое ожидание, равное
    $$
    \mu_P =\mu_P(\Theta, \jj,\aa)=\mm P(\Theta, \jj,\aa)= 
\!\! \!   \iii_\jj \!\!\nu_P (\Theta, \tau,\aa)\, d\tau,
    $$
$\nu_P=\nu_P(\Theta, \jj,\aa)$~--- 
интенсивность соответ\-ст\-ву\-юще\-го пуассоновского потока событий, 
$\jj\hm= (t_1, t_2]$; 
интегрирование по~$v$ распространяется на пространство~$R^1$ 
с~выколотым началом координат; $a\hm=a(Y_t, \Theta, t)$, $b\hm=b(Y_t, \Theta, t)$ 
и~$c\hm=c(Y_t, \Theta, t,v)$~--- скалярные нелинейные функции отмеченных аргументов.

Предположим, что при фиксированном~$\Theta$ выполнены условия существования 
и~единственности\linebreak\vspace*{-12pt}

\pagebreak

\noindent
 СтП, определяемого~(\ref{e2.1-s}) 
при соответствующем начальном условии~\cite{1-s, 5-s}. 
Кроме того, функции~$a$, $b$ и~$c$ и~характеристики СтП $W_0$ и~$P^0$ 
будем считать дифференцируемыми по~$\Theta$.

В дальнейшем для белого шума $V=\dot W$ (понимаемого в~строгом смысле и~являющегося производной по времени от произвольного СтП с~независимыми приращениями $W(t)$) будем использовать уравнение
\begin{equation}
\dot Y_t =a\left(Y_t, \Theta, t\right)+ b\left(Y_t, \Theta, t\right)V\,, 
\enskip V=\dot W\,.
    \label{e2.2-s}
    \end{equation}

Для линейной СтС с~параметрическим шумом уравнение~(\ref{e2.2-s}) 
записывается в~виде:
   \begin{multline}
    \dot Y_t = a(\Theta, t) Y_t + a_0 (\Theta, t) + \left[ b_0(\Theta, t) + 
    b_1(\Theta, t) Y_t\right] V\,,\\ Y(t_0) = Y_0\,.
    \label{e2.3-s}
    \end{multline}

\section{Уравнения для~одномерной характеристической функции}


Как известно~\cite{1-s}, если существует одномерная плотность $f\hm=f(y;\Theta,t)$, 
то уравнение для одномерной х.ф.\ $g\hm=g(\lambda;\Theta,t)$  
при фиксированном~$\Theta$ имеет вид:
\begin{multline}
\fr{\prt g (\lambda;\Theta,t)}{\prt t} = \iin \left[ i\lambda a 
    (y,\Theta,t)+{}\right.\\
\left.    {}+ \chi( b( y,\Theta,t ) \lambda;t )\right] 
    e^{i\lambda y} f ( y; \Theta,t)\, dy
    \label{e3.1-s}
    \end{multline}
при начальном условии
\begin{equation}
g(\lambda; \Theta,t_0) = g_0 (\lambda;\Theta)\,.
    \label{e3.2-s}
    \end{equation}
Здесь $f=f(y;\Theta,t)$~--- одномерная плотность вероятности (п.в.), 
связанная с~х.ф.\ $g(\lambda;\Theta,t)$ следующим  преобразованием Фурье:
      \begin{equation}
      f_1( y;\Theta,t) = \fr{1}{2\pi} \iin e^{-i\lambda y} 
    g (\lambda;\Theta,t)\, d\lambda\,;
    \label{e3.3-s}
    \end{equation}
    
    \vspace*{-12pt}
    
    \noindent
    \begin{multline*}
    \chi(b(y,\Theta,t)\lambda;t) =
    -\fr{\lambda^2 \nu_0 (\Theta,t)}{2}\, b^2 (y,\Theta,t)+{}\\
{}+ \iii_{R_0'} \exp \left\{
\left[ i\lambda c(y,\Theta,t,v)\right] -
    1-{}\right.\\
\left.    {}-i\lambda c(y,\Theta,t,v)\right\} \nu_P (\Theta,t,dv)\,.
   % \label{e3.4-s}
    \end{multline*}

Для СтС~(\ref{e2.2-s}) уравнения для х.ф.\ имеют вид~(\ref{e3.1-s})--(\ref{e3.3-s}). 
При этом предполагается задание функции $\chi(\mu;t)$ в~явном виде.

Так, например, для винеровского белого шума $V\hm=V_0\hm=\dot W_0$ имеем:
    \begin{equation*}
    \chi (\mu;t) =- \fr{\mu^2 \nu_0(t)}{2}\,.
   % \label{e3.5-s}
    \end{equation*}
Если $V(t)= V_P(t) = \dot W_P (t)$~--- общий пуассоновский СтП $W_P$, то
    \begin{equation*}
    \chi (\mu;t) = \lk g(\mu) -1\rk \nu_C (t)\,,
  %  \label{e3.6-s}
    \end{equation*}
где $g_C (\mu)$ и~$\nu_C(t)$~--- х.ф.\ и~интенсивность потока скачков.

Для белого шума $V$, представляющего 
собой линейную комбинацию независимых винеровского и~общего пуассоновского СтП
    \begin{equation*}
    V(t)= \dot W(t)\,,\enskip 
    W(t) = W_0(t) + \sss_{k=1}^n c_k P_k (t)\,,
   % \label{e3.7-s}
    \end{equation*}
имеем:
    \begin{equation*}
    \chi(\mu;t) = - \fr{ \mu^2}{2}\, \nu_0 (t) 
    \sss_{k=1}^n \lk g_k\left(c_k^{\mathrm{T}} \mu \right) - 1 \rk \nu_k(t)\,.
%\label{e3.8-s}
\end{equation*}
Здесь $\nu_0(t)$~--- интенсивность винеровского СтП~$W_0$; 
$g_{ck} (\lambda)$~--- х.ф.\ скачков общего пуассоновского СтП  $P_k (t)$; 
$\nu_k(t)$~--- интенсивность потока скачков.

\smallskip

\noindent
\textbf{Замечание~1.}
   Для стационарных дифференциальных СтС уравнения для  х.ф.\
    $g^*\hm=g^*(\lambda;\Theta)$  и~п.в.\ $f^*(y;\Theta)$ имеют вид:
       \begin{equation*}
    \iin \lk i \lambda a^* (y,\Theta) + 
    \chi\left( b^*(y,\Theta)\lambda\right)\rk e^{i\lambda y} f^* ( y;\Theta) \,dy =0\,;
    %\label{e3.9-s}
    \end{equation*}
      \begin{equation*}
    f^*(y;\Theta) = \fr{1}{2\pi} \iin e^{i\la y} g^* ( \lambda;\Theta)\, dy\,.
 %   \label{e3.10-s}
    \end{equation*}
Здесь звездочкой отмечены стационарные значения отмеченных функций.


\section{Метод интерполяционного аналитического моделирования в~скалярных системах}

Для определения х.~ф. $g=g(\lambda;\Theta, t)$ для СтС~(\ref{e2.1-s}) 
выразим ее через значение в~фиксированных точках отсчета, применив для 
этого какую-нибудь интерполяционную формулу [6]:
        \begin{equation}
    g(\lambda; \Theta, t) = \sss_{l=-N}^N g_l 
    (\lambda_l;\Theta, t) \alpha_l (\lambda)+ 
    R_N\,.
    \label{e4.1-s}
    \end{equation}
Здесь $R_N$~--- остаточный член;  $\lambda_l$~--- 
некоторые фиксированные значения аргумента~$\lambda$ в~точках отсчета;
$\alpha_l(\lambda)$~--- известные функции, удовлетворяющие условию
\begin{equation*}
\alpha_l (\lambda_r) =\delta_{lr}\enskip (l,r=0. \pm 1\tr \pm N)\,,
    %\label{e4.2-s}
    \end{equation*}
где $\delta_{lr}$~--- символ Кронекера.

Таким образом, задача сводится к~вычислению значений х.ф.\  
в~точках отсчета х.ф. После определения х.ф.\
$g(\lambda;\Theta,t)$ п.в.\ определяется преобразованием Фурье~(\ref{e3.3-s}). 
Подставляя в~(\ref{e3.3-s}) выражение~(\ref{e4.1-s}), найдем:
      \begin{align}
      f(y;\Theta, t) &= \sss_{l=-N}^N g_l (\lambda_l ; \Theta, t) \beta_l (y)\,;
    \label{e4.3-s}
\\
     \beta_l &= \fr{1}{2\pi} \iin e^{-i\lambda y} \alpha_l (\lambda)\, d\lambda\,,
     \notag
%    \label{e4.4-s}
    \end{align}
где функции $\beta_l(y)$ являются преобразованием \mbox{Фурье} 
функций $\alpha_l(\lambda)$ и,~следовательно, известны,
а~значит, искомая п.в.\ выражается через значения х.ф.\ 
в~точках отсчета. Полагая в~(\ref{e3.1-s}) $\lambda\hm=\lambda_r$ 
и~подставляя в~него~(\ref{e4.3-s}), получаем:
    \begin{multline}
    \!\!\dot g_r (\lambda_r; |\Theta, t) = - 
    \fr{\lambda^2}{2}\, \nu_0 (\Theta,t) \gamma_{0r} (y,\Theta, t) g_r (\lambda_r; 
    \Theta, t) + {}\\
    {}+\gamma_{1r} (\Theta , t)g_r \left(\lambda_r;\Theta, t\right)+\!\!
     \sss_{l=-N}^N\!\! \varepsilon_{lr} (\Theta,t) g_l (\lambda_r;\Theta,t) .\!\!
    \label{e4.5-s}
    \end{multline}
В уравнении~(\ref{e4.5-s}) введены следующие обозначения:
  \begin{align}
  \gamma_{0r} (y,\Theta,t) &= 
    \iin b^2 (y,\Theta,t) \beta_r (y) e^{i\lambda y} \,dy\,;
    \label{e4.6-s}
    \\
 \gamma_{1r} (y,\Theta, t)& = \iin \iii_{R_0^q} \left\{ 
    \exp \left[ i\lambda_y c (y,\Theta, t,v)\right] -1-{}\right.\notag\\
&\hspace*{-10mm}\left.    {}- i\lambda 
    c (y,\Theta, t,v)\right\} e^{i\lambda Y} \nu_P (\Theta, t, dv)\,dy\,;
    \label{e4.7-s}
    \\
 \varepsilon_{lr} (\Theta ,t) &= i\lambda_r \iin a(y,\Theta, t) 
    \beta_l (y) e^{i\lambda y}\, dy\,.
    \label{e4.8-s}
    \end{align}

Интегралы~(\ref{e4.6-s})--(\ref{e4.8-s}) 
могут быть вычислены заранее, так как подынтегральные функции из\-вестны.

Из уравнения~(\ref{e4.5-s}), вводя обозначения для эффективной 
интенсивности~$\nu_r^\re (\Theta, t)$ со\-от\-вет\-ст\-ву\-юще\-го 
гауссовского белого шума, согласно определению
      \begin{equation*}
      \fr{\lambda_r^2}{2}\, \nu_0 (\Theta, t) \gamma_{0r} (\Theta, t) - 
    \gamma_{1r} (\Theta, t) = \fr{\lambda_r^2}{2}\, 
    \nu_r^\re (\Theta, t)\,,
    %\label{e4.9-s}
    \end{equation*}
перепишем систему линейных обыкновенных дифференциальных уравнений~(\ref{e4.5-s}) 
в~виде:
       \begin{multline}
       \dot g_r \left( \lambda_r; \Theta, t\right) = -\fr{\lambda^2_r}{2}\,
     \nu_r^\re (\Theta, t) g_r \left(\lambda_r; \Theta, t\right) + {}\\
   \hspace*{-3mm} {}+\!\!
     \sss_{l=-N}^N \!\!\!\varepsilon_{lr} (\Theta, t) g_r (\lambda_r;\Theta, t)\ \ 
     (r=0,\pm 1\tr \pm N).\!\!\!
     \label{e4.10-s}
     \end{multline}
Так как х.~ф.\ и~коэффициенты $\varepsilon_{lr}$ в~общем случае являются 
комплексными, то при численном решении уравнений~(\ref{e4.1-s}) 
следует перейти к~действительной форме записи.

Полагая
    \begin{align*}
    g_r (\lambda_r; \Theta, t) &= g_{r1} (\lambda_r; \Theta, t)+ i g_{r2} 
    (\lambda_r; \Theta, t)\enskip (i^2=-1)\,;
%    \label{e4.11-s}
 \\
         \beta_l (y) &= \beta_{l1} (y) + i \beta_{l2}(y)\,;\\
    \beta_{l1} (y) &= \fr{1}{2\pi} \iin \alpha_l (\lambda) \cos \lambda y \,d\lambda\,;
    \\
    \beta _{l2} (y) &= \fr{1}{2\pi} \iin \alpha_l (\lambda)\sin \lambda y \,d\lambda\,;
       %\label{e4.12-s}
    \\
\varepsilon_{lr} &= \varepsilon_{lr1} (\Theta, t) + i \varepsilon_{lr2}
    (\Theta, t)\,;
\\
    \varepsilon_{lr1}  (\Theta, t) &=\lambda_r \!\!\iin\!\! \left[ \beta_{l1} (y) 
    \sin \lambda_r y +{}\right.\\
    &\hspace*{10mm}\left.{}+ \beta_{l2} (y) \cos \lambda_r y\right] a  (y,\Theta, t) \,dy\,,
    \\
    \varepsilon_{lr2}  (\Theta, t) &=\lambda_r \!\!\iin \!\!\left[ \beta_{l1} (y) 
    \cos \lambda_r y +{}\right.\\
    &\hspace*{10mm}\left.{}+ \beta_{l2} (y) \sin \lambda_r y\right] a  
    (y,\Theta, t) \,dy\,,
    %    \label{e4.13-s}
    \end{align*}
после приравнивания в~(\ref{e4.10-s}) действительных и~мнимых частей получаем 
линейную систему обыкновенных дифференциальных уравнений для значений х.ф.\ 
в~точках отсчета в~действительной форме:
    \begin{equation}
    \left.
    \begin{array}{rl}
    \dot g_{r1} (\lambda_r; \Theta, t)& =      \displaystyle- \fr{\nu_r^\re (\Theta, t)}{2}\,
     \lambda_r^2 g_{r1} (\lambda_r; \Theta, t)+ {}\\[6pt]
    & {}+\!\!
          \displaystyle \sss_{l=-N}^N \!\left[ \varepsilon_{lr1} (\Theta, t)g_{l1} 
     (\lambda_l; \Theta, t)-{}\right.\\[6pt]
&\hspace*{10mm}\left.     {}-\varepsilon_{lr2} ( \Theta, t) g_{l2} 
     (\lambda_l; \Theta, t)\right];\\[6pt]
%     \label{e4.14-s}
           \displaystyle  \dot g_{r2} (\lambda_r; \Theta, t) &=- \fr{\nu_r^\re (\Theta, t)}{2}\,
      \lambda_r^2 g_{r2} (\lambda_r; \Theta, t)+ {}\\[6pt]
     & {}+\!\!
      \displaystyle\sss_{l=-N}^N\! \left[ \varepsilon_{lr1} (\Theta, t)g_{l1} 
      (\lambda_l; \Theta, t)+{}\right.\\[6pt]
&\hspace*{10mm}\left.      {}+\varepsilon_{lr2} ( \Theta, t) g_{l2} 
      (\lambda_l; \Theta, t)\right]\,.
      \end{array}
      \right\}\!\!
      \label{e4.15-s}
\end{equation}
Упростим систему уравнений~(\ref{e4.15-s}), 
используя, во-пер\-вых, известные свойства х.ф.
       \begin{equation*}
    g(0;t) = 1, \enskip g(-\lambda;t) = \overline{g(\lambda,t)} 
   % \label{e4.16-s}
\end{equation*}
и вытекающие отсюда соотношения
        \begin{equation}
    g_{01} =1\,, \enskip 
    g_{02} =0\,,\enskip 
    g_{-r1}= g_{r1}\,,\enskip 
    g_{-r2}= -g_{r2}\,,
 \label{e4.17-s}
\end{equation}
во-вторых, выделим слагаемые, отвечающие условию $l\hm=0$, разбив 
оставшиеся части на две суммы и~изменив в~суммах по~$l$ индекс суммирования по~$l$ 
на противоположный по знаку. При этом нужно будет решать соответствующие уравнения 
для значений  $g_{r1}(\lambda_r; \Theta, t)$ и~$g_{r2} (\lambda_r; \Theta, t)$ 
при $r\hm>0$. В~результате придем к~следующей искомой системе уравнений:
    \begin{multline}
    \dot g_{r1}(\lambda_r; \Theta, t)=G_{r1}= 
    - \fr{\nu_r^\re (\Theta, t)}{2}\, 
    \lambda_r^2 g_{r} (\lambda_r; \Theta, t)+{}\\
    {}+\varepsilon_{0r1}(\Theta, t)+
    \sss_{l=1}^N \left\{ \left[ \varepsilon_{lr1} (\Theta, t)+{}\right.\right.\\
\left.{}+
     \varepsilon_{-lr1} (\Theta, t)\right] g_{l1} (\lambda_l; \Theta, t)-{}\\
    \left. {}-\left[ 
     \varepsilon_{lr2} ( \Theta, t)-\varepsilon_{lr2}(\Theta, t)\right] g_{l2}
      (\lambda_l; \Theta, t)\right\}\,;
      \label{e4.18-s}
      \end{multline}
      
      \vspace*{-12pt}
      
      \noindent
      \begin{multline}
\dot g_{r2}(\lambda_r; \Theta, t)=G_{r2}= {}\\
{}=
-\fr {\nu_r^\re (\Theta, t)}{2}\, \lambda_r^2 g_{r} 
(\lambda_r; \Theta, t)+\varepsilon_{0r2}(\Theta, t)+{}\\
{}+
\sss_{l=1}^N \left\{ \left[ 
     \varepsilon_{lr2} (\Theta, t)-\varepsilon_{-lr2} (\Theta, t)\right]\right.
 g_{l1} 
     (\lambda_l; \Theta, t)+{}\\
     \left.{}+\left[ \varepsilon_{lr1} ( \Theta, t)+
     \varepsilon_{lr1}(\Theta, t) \right] g_{l2} (\lambda_l; \Theta, t)\right\}\\ 
     (r=1\tr N)\,.
     \label{e4.19-s}
     \end{multline}
Теперь нетрудно из~(\ref{e4.3-s}) получить выражения для п.в. Имеем:
   \begin{multline}
   f(y;\Theta, t) = 
    \sss_{r=-N}^N \left\{ \left[ g_{r1} \left(\lambda_r;\Theta,t\right) 
    \beta_{r1}(y)-{}\right.\right.\\
    \left.{}-g_{r2} 
    \left(\lambda_r;\Theta,t\right) \beta_{r2}(y)\right] +{}\\
\hspace*{-4mm}\left.{}+i\left[ g_{r1} \left(\lambda_r;\Theta,t\right) \beta_{r2}(y)-g_{r2} 
    \left(\lambda_r;\Theta,t\right) \beta_{r1}(y)\right]\right\}.\!\!
     \label{e4.20-s}
     \end{multline}

Преобразуем~(\ref{e4.20-s}), учитывая, что, во-пер\-вых, п.в.\ является 
действительной функцией (поэтому мнимая часть при расчетах обращается в~нуль), 
во-вто\-рых, выделим в~действительной части слагаемое, в~котором  $l\hm=0$, 
разбив оставшуюся сумму на две части, соответствующие положительным и~отрицательным 
значениям индекса суммирования, \mbox{в-треть}\-их, примем во внимание~(\ref{e4.17-s}). 
В~результате получим искомое выражение для п.в.:
    \begin{multline}
    f(y;\Theta, t) = \beta_{01} (y) +{}\\
    {}+\sss_{r=1}^N \left\{ \lk \beta_{r1}(y)+ 
   \beta_{-r1}(y)\rk  g_{r1} \left(\lambda_r;\Theta,t\right) -{}\right.\\
\left.   {}-\lk \beta_{r2}(y)-
    \beta_{-r2}(y)\rk g_{r2} \left(\lambda_r;\Theta,t\right)\right\}\,. 
    \label{e4.21-s}
    \end{multline}

Таким образом, получены системы линейных обыкновенных дифференциальных~(\ref{e4.18-s}), 
(\ref{e4.19-s}) и~(\ref{e4.21-s}) для вычисления  х.ф.\ и~п.в.\
  при соответствующих начальных условиях:
    \begin{equation}
    g_r \left(0; \Theta , t_0\right) =g_{r1} \left(0;\Theta,t_0\right)+ig_{r2} 
    \left(0;\Theta,t_0\right)\,.
    \label{e4.22-s}
    \end{equation}
В случае гауссовского начального распределения имеем:
      \begin{align*}
      g_r (0; \Theta , t_0) &=\exp \lk i\lambda_r m (\Theta, t_0) - 
    \fr{\lambda_r^2}{2}\, D(\Theta, t_0)\rk\,;
   \\
    g_{r1} (0; \Theta , t_0) &=\exp \lk -\fr{\lambda_r^2}{2}\,
    D(\Theta, t_0)\rk \cos \lambda_r m(t_0)\,;
   \\
    g_{r2} (0; \Theta , t_0) &=\exp \lk -\fr{\lambda_r^2}{2}\,
      D(\Theta, t_0)\rk \sin \lambda_r m(t_0)\,,
%      \label{e4.23-s}
      \end{align*}
где $m(\Theta, t_0) = m_0$ и~$D(\Theta, t_0)\hm= D_0$~--- 
математическое ожидание и~дисперсия~$Y(t_0)$.


Так как  х.ф.\  является преобразованием Фурье п.в., то теорема Котельникова~\cite{4-s} 
полностью определяется последовательностью ее значений в~дискретном ряде точек. 
Поэтому если распределение~$Y_t$ сосредоточено на интервале $2\Delta\hm=2\Delta_y$, 
то $\alpha_r \hm= \alpha_r (\lambda)$ и~$\beta_l\hm = \beta_l(y)$ 
можно выбрать согласно~\cite{4-s} в~виде:
    \begin{align}
    \alpha_r = \alpha_r(\lambda)& = 
    fr{\sin \lk \Delta \left(\lambda-\lambda^*\right)\rk}
    {\Delta (\lambda-\lambda^*)}\,, \notag
    \\
    & \hspace*{37mm}\lambda^* = \fr{\pi r}{\Delta}\,;
    \label{e4.24-s}
   \\
    \beta_r = \beta_r(y)& =
    \begin{cases}
    \fr{1}{2\Delta} \exp (-i\lambda^*)&\  \hbox{при}\  \vert y\vert \le\Delta\,;\\
    0 &\  \hbox{при}\  \vert y\vert >\Delta\,.
    \end{cases}
    \label{e4.25-s}
\end{align}
При этом п.в.\ будет определяться формулой:
     \begin{multline}
     f(y;\Theta,t)= {}\\
     {}=\begin{cases}
    \fr{1}{\Delta} \sss_{r=1}^N \left[ g_r (\lambda_r;\Theta,t) \cos 
    \lambda^* y +{}\right.&\\
    \left.{}+ g_{r2} (\lambda_r;\Theta,t) \sin \lambda^*y\right] 
    &\hspace*{-10mm} \hbox{при}\ \vert y\vert \le\Delta\,;\\
    0 &\hspace*{-10mm} \hbox{при}\  \vert y\vert >0\,.\end{cases}
    \label{e4.26-s}
    \end{multline}

Таким образом, в~основе МИАМ одномерной скалярной дифференциальной СтС~(\ref{e2.1-s}) 
лежит следующее утверждение.

\smallskip

\noindent
\textbf{Теорема~1.}\
\textit{Пусть в~СтС}~(\ref{e2.1-s}) \textit{существует единственный СтП с~конечной 
одномерной п.в. Тогда при интерполяции х.ф.\ по Котельникову п.в.\ и~х.ф.\
 определяются линейными уравнениями}~(\ref{e4.18-s}) и~(\ref{e4.19-s}) 
 \textit{при начальных условиях}~(\ref{e4.22-s}).

\smallskip

Чувствительность МИАМ к~параметрам~$\Theta$ оценивается согласно~\cite{7-s} 
путем вычисления первых векторных функций влияния (чувствительности):
       \begin{align*}
       \nabla^\Theta g_{r1} &= \nabla^\Theta g_{r1} \left( \lambda_r; \Theta, t\right)
        \,;\\ 
    \nabla^\Theta g_{r2} &= \nabla^\Theta g_{r2} \left(\lambda_r;\Theta, t\right)\,,
   % \label{e4.27-s}
    \end{align*}
где $\nabla^\Theta$~--- символ вектора производных по векторному параметру~$\Theta$. 
В~частности, в~силу~(\ref{e4.18-s}) и~(\ref{e4.19-s}) 
соответствующие уравнения для первых функций влияния имеют вид:
    \begin{equation*}
    \dot \nabla^\Theta g_{r1} = \nabla^\Theta G_{r1}\,; \qquad
    \dot \nabla^\Theta g_{r2} = \nabla^\Theta G_{r2}
   % \label{e4.28-s}
    \end{equation*}
при нулевых начальных условиях.

\smallskip

\noindent
\textbf{Замечание~2.}\
Для стационарных СтС $g_{r1}^*$ и~$g_{r2}^*$ находятся из линейных 
уравнений~(\ref{e4.18-s}) и~(\ref{e4.19-s}) при нулевых левых частях.

\smallskip

\noindent
\textbf{Замечание~3.}\
Необходимое число~$N$ членов ряда, обеспечивающее достаточную малость 
остаточного члена~$R_N$ и~чувствительность расчетов к~па\-ра\-мет\-рам~$\Theta$, 
существенно зависит от выбора интервала~$\Delta$ в~(\ref{e4.24-s}) 
и~(\ref{e4.25-s}) и~определяется  аналитической природой задачи.

\smallskip

\noindent
\textbf{Замечание~4.}\
Для СтС~(\ref{e2.2-s}) теорема имеет аналогичный вид, если известна функция $\chi(\mu;t)$.

\section{О~методе интерполяционного моделирования в~векторных системах}

Следуя~\cite{2-s, 3-s}, воспользуемся принципом эквивалентности векторной СтС 
скалярной СтС. В~задачах практики обычно не требуется определения х.ф.\ и~п.в.\
 всех составляющих вектора~$Y_t$, а достаточно определить одномерное распределение 
 некоторых из них. Поэтому, следуя С.\,В.~Мальчикову~\cite{2-s, 3-s}, 
  будем искать распределения отдельных составляющих. Для определения одномерных х.ф.\
   и~п.в.\ некоторой $h$-й составляющей вектора~$Y_t$ заменим 
   исходную многомерную СтС эквивалентной одномерной СтС, описываемой стохастическим 
   дифференциальным уравнением Ито (или уравнением с~$\theta$-дифференциалом~\cite{1-s}), 
   выходная переменная~$\tilde Y_{ht}$  которого имеет те же х.ф.\ и~п.в., 
   что и~$h$-я составляющая~$Y_{ht}$. Далее заменим негауссовский белый шум 
   эквивалентным гауссовским, а затем воспользуемся теоремой~1. 
   В~результате придем к~следующему утверждению.
   
   \smallskip
   
   \noindent
   \textbf{Теорема~2.}\
\textit{Пусть векторная дифференциальная сис\-те\-ма приводится к~эквивалентной 
системе скалярных СтС Ито. Тогда в~основе МИАМ лежат уравнения тео\-ре\-мы}~1.

\smallskip

\noindent
\textbf{Замечание~5.}\
Выходные переменные  $Y_{ht}, \tilde Y_{ht}$ в~общем случае~--- 
различные СтП, поскольку $ \tilde Y_{ht}$ является марковским СтП и,~следовательно, 
сам по себе может и~не быть марковским СтП. Однако одномерные распределения 
у~них одинаковы.

\smallskip

\noindent
\textbf{Замечание~6.}\
Для квазилинейных СтС с~аддитивным шумом замена многомерной сис\-те\-мы одномерной 
допустима в~случае асимптотической устой\-чи\-вости линейной детерминированной части. 
Для ком\-плекс\-но-со\-пря\-жен\-ных корней эквивалентная сис\-те\-ма будет уже двумерной. 
Случаи кратных корней требуют специального рассмотрения.

\section{Тестовый пример}


Рассмотрим скалярную дифференциальную СтС с~негауссовским (в~общем случае) шумом~$V$ 
следующего вида~\cite{4-s}:
      \begin{equation*}
      \dot Y_t = - A \, \mathrm{sgn}\, Y_t + V\,.
   % \label{e6.1-s}
    \end{equation*}
Согласно~(\ref{e4.8-s}) имеем следующие выражения для коэффициентов~$\varepsilon_{lr}$, 
определяющих значения х.ф.\ в~точках отсчета:
\begin{multline}
\varepsilon_{lr} =- i\lambda_r A \iii_{-\Delta}^\Delta \mathrm{sgn}\, y 
    \fr{1}{2\Delta}\, e^{i(\lambda_r - \lambda_l)y}\, dy ={}\\
{}=\fr{i\lambda_r A}{2 \Delta}\left\{ \iii_{-\Delta}^0 \exp \lk i 
    \left(\lambda_r-\lambda_l\right) y\rk \,dy - {}\right.\\
\left.    {}-
    \iii_0^\Delta \exp \lk i \left(\lambda_r-\lambda_l\right) y \rk\, dy\right\}\,.
    \label{e6.2-s}
    \end{multline}
Учитывая, что
    $$
    \lambda_r =\fr{r\pi}{\Delta}\,; \enskip \lambda_l = \fr{l\pi}{\Delta}\,; 
    \enskip \varepsilon_{ll}=0\,,
    $$
из~(\ref{e6.2-s}) находим:
     \begin{multline*}
    \varepsilon_{lr} = \fr{A}{\Delta}\,\fr {\lambda_r}{\lambda_r-\lambda_l} 
    \lk 1-\cos \left(\lambda_r -\lambda_l\right)\Delta\rk = {}\\
    {}=
    \fr{A}{r}\, \fr{r}{r-l} \lk 1-(-1)^{r-l}\rk\,.
    %\label{e6.3-s}
    \end{multline*}

Таким образом, все $\varepsilon_{lr}$ действительны, поэтому 
$\varepsilon_{lr1}\hm= \varepsilon_{lr}$, $\varepsilon_{lr2}\hm=0$. 
Следовательно, уравнения~(\ref{e4.18-s}) и~(\ref{e4.19-s}) 
для действительной и~мнимой частей х.ф.\ разделяются и~принимают следующий вид:
       \begin{equation*}
       \dot g_{r1} = - \fr{\nu_r^\re}{2} \left(
    \fr{r\pi}{\Delta}\right)^2 g_{r1} + \varepsilon_{0r1} + 
    \sss_{l=1}^N g_{l1} \left( \varepsilon_{lr}+\varepsilon_{-lr}\right)\,;
   % \label{e6.4-s}
    \end{equation*}
    \begin{equation*}
    \dot g_{r2} = -\fr{\nu_r^\re}{2}  \left(
    \fr{r\pi}{\Delta}\right)^2 g_{r2} + \varepsilon_{0r2} + 
    \sss_{l=1}^N g_{l2} \left( \varepsilon_{lr}+\varepsilon_{-lr}\right)\,.
    %\label{e6.5-s}
    \end{equation*}
Здесь аргументы для краткости опущены.

В случае если начальное распределение имеет нулевое математическое ожидание, 
$g_{rr}\hm\equiv 0$. В~этом случае, согласно~(\ref{e4.26-s}), 
одномерная плотность выражается формулой:
     \begin{equation*}
     f(y;t) = \fr{1}{2\Delta} + \sss_{l=1}^N g_{l1} \cos 
    \left(
    \fr{l\pi}{\Delta} y\right)\,.
   % \label{e6.6-s}
    \end{equation*}
В~стационарном случае известно точное выражение для стационарной плотности~\cite{1-s}:
    \begin{equation*}
    f(y,\infty) = \fr{A}{\nu_r^\re} \exp \lk -\fr{2A}{\Delta} \vert y\vert \rk\,.
   % \label{e6.7-s}
    \end{equation*}

Как показали расчеты, согласно~(\ref{e4.21-s}) при $A\hm=1$, $\nu_r^\re\hm =1$, 
$N\hm=5$ и~$t\hm\to \infty$ относительная ошибка при $y\hm=0$ составляет~20\%, 
при $y\hm=1$~---  менее~10\%, а~при $y\hm=2$~--- менее~1\%.

\section{Заключение}


Для скалярных нелинейных дифференциальных гауссовских и~негауссовских СтС~(\ref{e2.1-s}) 
и~(\ref{e2.3-s}) при условии существования одномерной п.в.~СтП 
на базе уравнения для одномерной х.ф.\ разработан 
\mbox{МИАМ}. Получены уравнения для функций влияния 
вектора параметров, входящих в~(\ref{e2.1-s}) и~(\ref{e2.3-s}). 
Тестовый пример показывает эффективность метода уже 
при~3--5~членах разложения.

Результаты допускают обобщение на интегральные и~ин\-тег\-ро\-диф\-фе\-рен\-ци\-аль\-ные СтС,
 приводимые к~дифференциальным~\cite{1-s}.

Особого интереса заслуживает обобщение результатов на некоторые классы векторных 
нелинейных дифференциальных СтС.

{\small\frenchspacing
 {%\baselineskip=10.8pt
 \addcontentsline{toc}{section}{References}
 \begin{thebibliography}{9}
\bibitem{1-s}
\Au{Пугачёв В.\,С., Синицын И.\,Н.}
Теория стохастических систем.~--- М.: Логос, 2000; 2004. 1000~с.
%[Англ. пер. Stochastic Systems. Theory and  Applications. --
%Singapore: World Scientific, 2001. 908~p.].

\bibitem{2-s}
\Au{Мальчиков С.\,В.}
Приближенный метод определения законов распределения фазовых координат нелинейных 
автоматических систем~// Автоматика и~телемеханика, 1970. №\,5. С.~43--50.

\bibitem{3-s}
\Au{Казаков И.\,Е., Мальчиков~С.\,В.}
Анализ стохастических систем в~пространстве состояний.~--- М.: Наука, 1983. 348~с.

\bibitem{4-s}
\Au{Котельников В.\,А.}
Теория потенциальной помехоустойчивости.~--- М.: Госэнергоиздат, 1956. 412~с.

\bibitem{5-s}
Справочник по теории вероятностей и~математической статистике~/ 
Под ред.\ В.\,С.~Королюка, Н.\,И.~Портенко, А.\,В.~Скорохода, А.\,Ф.~Турбина.~--- 
М.: Наука, 1985. 640~с.

\bibitem{6-s}
\Au{Корн Г., Корн~Т.}
Справочник по  математике (для научных работников и~инженеров)~/
Пер с~англ.~--- 
М.: Наука, 1974. 832~с.
(\Au{Korn~G., Korn~T.} 
Mathematical handbook for scientists and engineers.~--- 
New York\,--\,San Francisco\,--\,Toronto\,--\,London\,--\,Sydney:
McGraw Hill Book Co., 1968. 1147~p.)

\bibitem{7-s}
\Au{Евланов Л.\,Г., Константинов~В.\,М.}
Системы со случайными параметрами.~--- М.:
Наука,  1976.  585~с.

 \end{thebibliography}

 }
 }

\end{multicols}

\vspace*{-6pt}

\hfill{\small\textit{Поступила в~редакцию 29.08.17}}

\vspace*{8pt}

%\newpage

%\vspace*{-24pt}

\hrule

\vspace*{2pt}

\hrule

%\vspace*{8pt}


\def\tit{METHOD OF INTERPOLATIONAL
ANALYTICAL MODELING OF~PROCESSES IN~STOCHASTIC SYSTEMS}

\def\titkol{Method of interpolational
analytical modeling of~processes in~stochastic systems}

\def\aut{I.\,N.~Sinitsyn}

\def\autkol{I.\,N.~Sinitsyn}

\titel{\tit}{\aut}{\autkol}{\titkol}

\vspace*{-9pt}


\noindent
Institute of Informatics Problems, Federal Research Center 
``Computer Science and Control'' of the Russian Academy of Sciences,
44-2~Vavilov Str., Moscow 119333, Russian Federation



\def\leftfootline{\small{\textbf{\thepage}
\hfill INFORMATIKA I EE PRIMENENIYA~--- INFORMATICS AND
APPLICATIONS\ \ \ 2018\ \ \ volume~12\ \ \ issue\ 1}
}%
 \def\rightfootline{\small{INFORMATIKA I EE PRIMENENIYA~---
INFORMATICS AND APPLICATIONS\ \ \ 2018\ \ \ volume~12\ \ \ issue\ 1
\hfill \textbf{\thepage}}}

\vspace*{3pt}



\Abste{Among known methods of stochastic processes (StP) analytical modeling in 
differential and integrodifferential stochastic systems (StS) based on the 
direct numerical solution of equation for one-dimensional characteristic function 
(c.f.), it is necessary to distinguish interpolational S.\,V.~Mal'chikov method. 
In this case, for c.f.\ interpolation, the Kotelnikov theorem was implemented. 
The paper contains the treatment of interpolational methods of StP analytical 
modeling for two classes of nonlinear non-Gaussian StS. Special attention in paid to 
sensitivity analysis. Test example for discontinuous nonlinearity confirms the
 method efficiency. Some generalizations are mentioned.}



\KWE{one dimensional characteristic functions (c.f.);
one-dimensional probability density (p.d.);
stochastic processes (StP);
stochastic system (StS)}

\DOI{10.14357/19922264180107} 

\pagebreak

%\vspace*{-12pt}

%\Ack
%\noindent





  \begin{multicols}{2}

\renewcommand{\bibname}{\protect\rmfamily References}
%\renewcommand{\bibname}{\large\protect\rm References}

{\small\frenchspacing
 {%\baselineskip=10.8pt
 \addcontentsline{toc}{section}{References}
 \begin{thebibliography}{9}
\bibitem{1-s-1}
\Aue{Pugachev, V.\,S., and I.\,N.~Sinitsyn.} 2000, 2004.
Teoriya stokhasticheskikh sistem [Stochastic systems: Theory and  applications]. 
Moscow: Logos. 1000~p.  %[Angl. per. . -- Singapore: World Scientific, 2001].


\bibitem{2-s-1}
\Aue{Mal'chikov, S.\,V.} 1970.
Priblizhennyy metod opredeleniya zakonov raspredeleniya fazovykh koordinat 
nelineynykh avtomaticheskikh sistem [Approximate method for phase coordinates 
analysis in nonlinear control systems]. 
\textit{Avtomatika i~telemekhanika} [Automat. Rem. Contr.]
5:43--50.

\bibitem{3-s-1}
\Aue{Kazakov, I.\,E., and S.\,V.~Mal'chikov.} 1983.
\textit{Analiz sto\-kha\-sti\-che\-skikh sistem v~prostranstve sostoyaniy} [Stochastic 
systems analysis in state space]. Moscow: Nauka. 348~p.

\bibitem{4-s-1}
\Aue{Kotel'nikov, V.\,A. } 1956.
\textit{Teoriya potentsial'noy po\-me\-kho\-us\-toy\-chi\-vosti} [Theory of potential noiseproof]. 
Moscow: Gosenergoizdat. 412~p.

\bibitem{5-s-1}
Korolyuk, V.\,S., N.\,I.~Portenko, A.\,V.~Skorokhod, and A.\,F.~Turbin, eds. 1985.
\textit{Spravochnik po teorii veroyatnostey v~matematicheskoy statistike}
[Handbook on probability theory and mathematical statistics]. 
 Moscow: Nauka. 640~p.

\bibitem{6-s-1}
\Aue{Korn, G., and T.~Korn.} 1968.
\textit{Mathematical handbook for scientists and engineers}. 
New York\,--\,San Francisco\,--\,Toronto\,--\,London\,--\,Sydney:
McGraw Hill Book Co. 1147~p.

\bibitem{7-s-1}
\Aue{Evlanov, L.\,G., and V.\,M.~Konstantinov.} 1976.
\textit{Sistemy so sluchaynymi parametrami} [Systems with random parameters]. Moscow:
Nauka.  585~p.
\end{thebibliography}

 }
 }

\end{multicols}

\vspace*{-6pt}

\hfill{\small\textit{Received August 29, 2017}}

%\vspace*{-10pt}
  

\Contrl

\noindent
\textbf{Sinitsyn Igor N.} (b.\ 1940)~--- 
Doctor of Science in technology, professor, Honored Scientist of RF, 
principal scientist, Institute of Informatics Problems, Federal Research 
Center ``Computer Science and Control'' of the Russian Academy of Sciences, 
44-2~Vavilov Str., Moscow 119333, Russian Federation; \mbox{sinitsin@dol.ru}
\label{end\stat}


\renewcommand{\bibname}{\protect\rm Литература} 