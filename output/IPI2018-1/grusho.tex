\def\stat{grusho}

\def\tit{О НЕКОТОРЫХ ВОЗМОЖНОСТЯХ УПРАВЛЕНИЯ РЕСУРСАМИ ПРИ~ОРГАНИЗАЦИИ 
ПРОАКТИВНОГО ПРОТИВОДЕЙСТВИЯ КОМПЬЮТЕРНЫМ АТАКАМ$^*$}

\def\titkol{О некоторых возможностях управления ресурсами при~организации 
проактивного противодействия %компьютерным 
атакам}

\def\aut{А.\,А.~Грушо$^1$, М.\,И.~Забежайло$^2$, А.\,А.~Зацаринный$^3$, 
Е.\,Е.~Тимонина$^4$}

\def\autkol{А.\,А.~Грушо, М.\,И.~Забежайло, А.\,А.~Зацаринный, 
Е.\,Е.~Тимонина}

\titel{\tit}{\aut}{\autkol}{\titkol}

\index{Грушо А.\,А.}
\index{Забежайло М.\,И.}
\index{Зацаринный А.\,А.}
\index{Тимонина Е.\,Е.}
\index{Grusho A.\,A.}
\index{Zabezhailo M.\,I.}
\index{Zatsarinny A.\,A.}
\index{Timonina E.\,E.}




{\renewcommand{\thefootnote}{\fnsymbol{footnote}} \footnotetext[1]
{Работа выполнена при поддержке РФФИ (проект 15-29-07981).}}


\renewcommand{\thefootnote}{\arabic{footnote}}
\footnotetext[1]{Институт проблем информатики Федерального исследовательского центра 
<<Информатика и~управ\-ле\-ние>> 
Российской академии наук, \mbox{grusho@yandex.ru}}
\footnotetext[2]{Институт проблем информатики Федерального исследовательского центра 
<<Информатика и~управ\-ле\-ние>> 
Российской академии наук, \mbox{m.zabezhailo@yandex.ru}}
\footnotetext[3]{Институт проблем информатики Федерального исследовательского центра 
<<Информатика и~управ\-ле\-ние>> 
Российской академии наук, \mbox{alex250451@mail.ru}}
\footnotetext[4]{Институт проблем информатики Федерального исследовательского центра 
<<Информатика и~управ\-ле\-ние>> 
Российской академии наук, \mbox{eltimon@yandex.ru}}

\vspace*{-6pt}

  
  \Abst{Обсуждаются возможности рационального противодействия компьютерным атакам, 
описываемым как последовательности событий. В~основе подхода~--- математическая 
техника обучения на прецедентах, использующая уточнение сходства объектов как бинарной 
алгебраической операции. Анализируются сходства цепочек событий. Порождаемые 
в~процессе обучения классы сходства (толерантности) используются для идентификации атак 
рассматриваемого типа на ранних стадиях их развертывания. Предлагается технология 
рационального управления ресурсами при организации противодействия подобным 
вредоносным воздействиям.}
  
\KW{информационная безопасность; анализ данных; сходство как алгебраическая операция; 
сходство цепочек; управление ресурсами}

\DOI{10.14357/19922264180108} 
  
\vspace*{-6pt}


\vskip 10pt plus 9pt minus 6pt

\thispagestyle{headings}

\begin{multicols}{2}

\label{st\stat}

\section{Введение}

  Весьма полезно организовать противодействие атакам на информационные 
технологии в~проактивном режиме, позволяющем, в~частности, 
идентифицировать предстоящие (прогнозируемые) вредонос\-ные воздействия на 
ранних стадиях их развертывания и,~как следствие, иметь возможность 
препятствовать их развертыванию в~полном объеме~[1].
  
  Возможности проследить по первым наблюда\-емым стадиям развития 
(потенциальной) атаки (согласуемые с~уже накопленным опытом) варианты ее 
продолжения позволяют выделить те группы признаков, которые должны будут 
сопутствовать последовательно проявляющимся в~ходе развертывания текущей 
атаки признакам. Это позволяет организовать противодействие атаке, фокусируя 
усилия в~первую очередь на \textit{сопутствующих} признаках, препятствуя 
таким образом формированию тех комбинаций признаков (КП), которые 
и~вызывают появление соответствующего вредоносного эффекта (ВЭ).
  
  Фокусировка на \textit{сопутствующих} уже \textit{наблюда\-емым} факторам 
влияния (\textit{признакам}), чтобы не допустить формирования уже известных 
КП, которые ведут к~конкретным ВЭ, дает основания охарактеризовать 
управление ресурсами в~рассматриваемом подходе как рациональное действие.
  %
  Тогда можно говорить о предлагаемом подходе как о составной части 
инструментария представления знаний средствами так называемых 
квазиаксиоматических теорий (см., например,~[2, 3]) и~построения теорий для 
открытых, пополняемых новыми данными предметных областей.

\section{Цепочки состояний: основные понятия и~определения}

  Обратимся к~простейшему тео\-ре\-ти\-ко-мно\-же\-ст\-вен\-но\-му варианту 
описания процедурных конструкций предлагаемого типа. Будем рассматривать 
\textit{событие} как последовательность \textit{состояний} объекта защиты, где 
каждое \textit{состояние} представлено некоторым множеством 
\textit{признаков}~--- фактов из некоторого <<базового>> множества, 
наблюдаемых в~текущем со\-сто\-янии объекта защиты.
  
  Пусть $\mathbf{U}\hm= \left\{ a_1, a_2, \ldots, a_n\right\}$~--- исходное 
множество признаков, которые будут использоваться для описания событий.
  
  \smallskip
  
  \noindent
  \textbf{Определение~1.} 
  \begin{enumerate}[1.]
  \item \textit{Состоянием} $\mathbf{St}$ объекта защиты~$\mathbf{O}$ будем 
называть каждое непустое подмножество исходного множества 
признаков~$\mathbf{U}: \mathbf{St}\hm\subseteq \mathbf{U}\not= \varnothing$.
\item \textit{Событием} (описанием события) $\mathbf{E}\hm = 
\langle\mathbf{St}_1, \mathbf{St}_2, \ldots , \mathbf{St}_k\rangle$, наблюдаемого 
в~поведении~$\mathbf{O}$, будем называть всякую конечную 
последовательность $\langle \mathbf{St}_1, \mathbf{St}_2, \ldots, 
\mathbf{St}_k\rangle$ со\-сто\-яний $\mathbf{St}_i \hm\in \mathbf{St}\hm = 
(2^{\mathbf{U}}\backslash \varnothing) \mathbf{St}$ для каждого $i \hm\in \{1, 2, \ldots, 
k\}$. 
\end{enumerate}

  Пусть $\mathbf{Е}\vert_l \hm= \langle\mathbf{St}_{l_1}, 
\mathbf{St}_{l_2},\ldots , \mathbf{St}_{l_m}\rangle$ есть подпоследовательность 
представляющей произвольным образом выбранное событие~$\mathbf{E}$ 
по\-сле\-до\-ва\-тель\-ности $\langle \mathbf{St}_1, \mathbf{St}_2, \ldots, 
\mathbf{St}_k\rangle$, такая что в~$\mathbf{E}\vert_l$ сохраняется 
фор\-ми\-ру\-ющий~$\mathbf{E}$ порядок вхождения со\-от\-вет\-ст\-ву\-ющих 
со\-сто\-яний~$\mathbf{St}_i$. Выберем из каждого множества 
признаков~$\mathbf{St}_{l_i}$ ровно по одному элементу~$a_{l_i}$ 
и~рас\-смот\-рим цепочку (по\-сле\-до\-ва\-тель\-ность) вида $\vec{s}$ $\langle a_{l_1}, 
a_{l_2}, \ldots , a_{l_m}\rangle$, элементы которой упорядочены в~соответствии 
с~уже имеющимся порядком со\-сто\-яний в~по\-сле\-до\-ва\-тель\-ности~$\mathbf{E}$. 
Фор\-ми\-ру\-емое таким путем отношение~$\prec_1$ час\-тич\-но\-го порядка на 
элементах цепочек вида~$\vec{s}$ будем интерпретировать как $a_{l_i}\prec_1 
a_{l_j}$. Это означает, что в~цепочку~$\vec{s}$ признак~$a_{l_i}$ входит 
раньше, чем признак~$a_{l_j}$.
  
  Пусть имеются два события: $\mathbf{E}_1$, задающее порядок~$\prec_1^1$ 
входящих в~него состояний, и~$\mathbf{E}_2$ с~со\-от\-вет\-ст\-ву\-ющим 
порядком~$\prec_1^2$. Пусть также $S(\mathbf{Е}_1)$ и~$S(\mathbf{Е}_2)$~--- 
множества цепочек, пред\-став\-ля\-ющие соответственно события~$\mathbf{E}_1$ 
и~$\mathbf{E}_2$, а~$\vec{s}_1$ и~$\vec{s}_2$~--- цепочки из 
множеств~$S(\mathbf{Е}_1)$ и~$S(\mathbf{Е}_2)$ соответственно. Наложением 
цепочек~$\vec{s}_1$ и~$\vec{s}_2$ будем называть 
множество~$\mathbf{s}_{12}$ максимальных по вложению общих подцепочек 
для~$\vec{s}_1$ и~$\vec{s}_2$, каж\-дый элемент которого сохраняет при этом 
име\-ющий\-ся в~каж\-дой из сравниваемых исходных цепочек час\-тич\-ный 
порядок~$\prec_1^1$ и~$\prec_1^2$.
  
  Для любой цепочки~$\vec{s}_i$ результатом автоналожения, т.\,е.\ 
наложения~$\vec{s}_i$ на~$\vec{s}_i$, будет множество~$\mathbf{s}_{ii}$ всех 
максимальных по вложению в~исходную цепочку~$\vec{s}_i$ подцепочек, 
со\-хра\-ня\-ющих уже име\-ющий\-ся в~$\vec{s}_i$ порядок~$\prec_1^i$. Пусть 
$S(\mathbf{E})$~--- множество цепочек, по\-рож\-да\-емых событием~$\mathbf{E}$. 
Сформируем для каждой цепочки $\vec{s}_i\hm\in  S(\mathbf{E})$ 
множество~$\mathbf{s}(\mathbf{E})$ всех автоналожений~$\mathbf{s}_{ii}$ 
входящих в~$S(\mathbf{E})$ цепочек~$\vec{s}_i$.
  
  Нетрудно убедиться, что любая пара несовпадающих цепочек порождает 
несовпадающие автоналожения. Если эти цепочки имеют разные длины, то 
более длинная цепочка не может содержаться в~автоналожении более короткой 
цепочки. При одинаковой длине цепочек каждая из них не может содержаться 
в~автоналожении другой (они ведь не совпадают). Таким образом, каждое 
событие~$\mathbf{E}$ может быть однозначным образом пред\-став\-ле\-но 
множеством~$\mathbf{s}(\mathbf{E})$ автоналожений всех входящих в~него 
цепочек вида~$\vec{s}_i$.
  
  Ситуацию, когда результат~$\mathbf{s}_{12}$ наложения двух 
цепочек~$\vec{s}_1$ и~$\vec{s}_2$ не является одноэлементным множеством 
подцепочек, демонстрирует сле\-ду\-ющий пример.
  
  \smallskip
  
  \noindent
  \textbf{Пример.} 
  
  \begin{enumerate}
\item Пусть $\mathbf{E}_1 \hm= \langle \mathbf{St}_1, \mathbf{St}_2, 
\mathbf{St}_3, \mathbf{St}_4\rangle \hm = \langle \{a\},\{b\},\linebreak \{a\},\{c\}\rangle$, 
а~$\mathbf{E}_2\hm = \langle \{a\},\{b\},\{c\}\rangle$. В~данном случае 
$S(\mathbf{E}_1) \hm= \{\langle \{a\},\{b\},\{a\},\{c\}\rangle\} \hm= 
\{\vec{s}_1\}$, $S(\mathbf{E}_2) \hm= \{\langle \{a\},\{b\},\{c\}\rangle\} \hm= 
\{\vec{s}_2\}$. Рассмотрим~$\mathbf{s}_{12}$~--- результат наложения 
цепочек~$\vec{s}_1$ и~$\vec{s}_2$. Легко проверить, что
  $$
\mathbf{s}_{12} =\left\{\left\{ \langle \{a^1\},\{b\},\{c\}\rangle\right\}, 
\left\{\langle \{a^2\},\{c\}\rangle\right\}\right\},
$$
где $a^1$ и~$a^2$ соответственно первое и~второе (в~смыс\-ле~$\prec_1$) 
вхож\-де\-ние~$a$ в~цепочку~$\vec{s}_1$. Однако при этом цепочки, например, 
$\langle \{a^1\},\{c\}\rangle$ и~$\{\langle \{b\},\{c\}\rangle$ не входят 
в~$\mathbf{s}_{12}$, так как могут быть расширены до вошедшей 
в~$\mathbf{s}_{12}$ цепочки $\langle \{a^1\},\{b\},\{c\}\rangle$. Таким образом, 
наложение~$\mathbf{s}_{12}$ содержит сразу две максимальные по вложению 
в~$\vec{s}_1$ и~$\vec{s}_2$ общие для них подцепочки.
 
 \item Пусть $\mathbf{E}\hm = \langle\{a\},\{b\},\{a\},\{b\}\rangle$. 
Рас\-смот\-рим~$\mathbf{s}_{11}$~--- результат наложения цепочки~$\vec{s}_1$ на 
саму себя. Легко проверить, что
  $$
  \mathbf{s}_{11} = \left\{\langle \{a\},\{b\},\{a\},\{b\}\rangle, \langle 
\{a^{12}\},\{b^{12}\}\rangle\right\}\,,
  $$
где $a^{12}$ порождено наложением~$a^1$ и~$a^2$ соответственно первого 
и~второго (в~смысле отношения~$\prec_1$) вхож\-де\-ния~$a$ 
в~цепочку~$\vec{s}_1$, $b^{12}$ по\-рож\-де\-но наложением~$b^1$ и~$b^2$~--- 
первого и~второго вхождения~$b$ в~цепочку~$\vec{s}_1$, по\-рож\-ден\-ную 
событием~$\mathbf{E}$.
  
  Сходство пар событий формализуем как результат наложения 
пред\-став\-ля\-ющих эти события множеств (автоналожений) цепочек, определив 
для этого бинарную алгебраическую операцию на множествах цепочек 
и~по\-рож\-да\-емое ею отношение на\-ло\-жи\-мости.
  
  \end{enumerate}
  
  \noindent
  \textbf{Определение~2.}\ Пусть на общем множестве признаков~$\mathbf{U}$ 
заданы два события~$\mathbf{E}_1$ и~$\mathbf{E}_2$ и~порождаемая ими пара 
множеств цепочек $S_1 \hm= S(\mathbf{E}_1)$ и~$S_2 \hm= S(\mathbf{E}_2)$ 
соответственно. Результатом наложения множеств~$S_1$ и~$S_2$ будем считать 
множество цепочек~$S_{12}$, образованное наложением каждого элемента 
из~$S_1$ на каждый элемент из~$S_2$.
  
  \smallskip
  
  Пусть заданы множество признаков~$\mathbf{U}$ и~множество 
$\boldsymbol{\Omega}\hm= \{\mathbf{E}_1, \mathbf{E}_2, \ldots , \mathbf{E}_r\}$ 
сформированных на~$\mathbf{U}$ событий $\mathbf{E}_i$, $i\hm\in\{1, 2, \ldots , 
r\}$. Множество $\mathbf{Dom}(\mathbf{U}, \boldsymbol{\Omega})$ определим 
сле\-ду\-ющим образом.
  
%  \smallskip
\pagebreak
  
  \noindent
  \textbf{Определение~3.}\ 
  \begin{enumerate}[1.]
  \item  Для каждого события $\mathbf{E}_i \hm\in \boldsymbol{\Omega}$ 
автоналожение~$\mathbf{s}^i_{jj}$ каждой цепочки~$\vec{s}^{\,i}_j$ из множества 
цепочек $S_i\hm=  S(\mathbf{E}_i)$ принадлежит 
$\mathbf{Dom}(\mathbf{U},\boldsymbol{\Omega})$, т.\,е.
  $$
  S_i =S(\mathbf{E}_i)\subseteq 
\mathbf{Dom}(\mathbf{U},\boldsymbol{\Omega})\,.
  $$
  \item  Для каждой пары $S_i$ и~$S_j$ множеств из $\mathbf{Dom}(\mathbf{U}, 
\boldsymbol{\Omega})$ их наложение (т.\,е.\ результат наложения каждой 
цепочки из~$S_i$ на каждую цепочку из~$S_j$) также принадлежит 
$\mathbf{Dom}(\mathbf{U}, \boldsymbol{\Omega})$.
  \item  Других элементов в~множестве 
$\mathbf{Dom}(\mathbf{U},\boldsymbol{\Omega})$ нет.
  \end{enumerate}
  
  \noindent
  \textbf{Определение~4.}\ Для каж\-дой пары~$S_i$ и~$S_j$ множеств цепочек 
из~$\mathbf{Dom}(\mathbf{U}, \boldsymbol{\Omega})$ определена бинарная 
операция $S_{ij}\hm= S_i\otimes S_j$ поэлементного наложения каж\-дой цепочки 
из~$S_i$ на каж\-дую цепочку из~$S_j$.
  
  \smallskip
  
  \noindent
  \textbf{Утверждение~1.} \textit{Для любых $S_i, S_j, S_k \hm\subseteq 
\mathbf{Dom}(\mathbf{U}, \boldsymbol{\Omega})$ операция~$\otimes$ обладает 
сле\-ду\-ющи\-ми свойствами}:
  \begin{itemize}
  \item[(a)] \textit{идемпотентность}:
$$
 S_i = S_i \otimes S_i\,,
$$
\item[(б)] \textit{коммутативность}:
$$
S_i \otimes S_j = S_j \otimes S_i
$$
\item[(в)] \textit{ассоциативность}:
\begin{equation}
S_i \otimes (S_j \otimes S_k) = (S_i \otimes S_j) \otimes S_k\,.
\label{e1-gr}
\end{equation}
\end{itemize}

\textit{Таким образом, алгебра $\wp \hm= \langle 
\mathbf{Dom}(\mathbf{U},\boldsymbol{\Omega}), \varnothing\rangle$ есть 
нижняя полурешетка}.
  
  \noindent
  Д\,о\,к\,а\,з\,а\,т\,е\,л\,ь\,с\,т\,в\,о\,.\ \ Идемпотентность является очевидным 
след\-ст\-ви\-ем пред\-став\-ле\-ния в~множестве 
$\mathbf{Dom}(\mathbf{U},\boldsymbol{\Omega})$ цепочек из~$\mathbf{E}_i$ их 
автоналожениями.
  
   Коммутативность обеспечивается выделением в~каждых двух на\-кла\-ды\-ва\-емых 
цепочках такого общего подмножества элементов, на котором обеспечивается 
одновременное сохранение со\-от\-вет\-ст\-ву\-ющих обеим цепочкам час\-тич\-ных 
порядков~$\prec_1^1$ и~$\prec_1^2$.
  
\textbf{Ассоциативность.} Предположим противное: условие ассоциативности не 
выполняется. Тогда должны найтись три цепочки~$\vec{s}_i$, $\vec{s}_j$ 
и~$\vec{s}_k$ и~некоторая максимальная по вложению в~каж\-дую из них 
подцепочка $\vec{s}_0\hm=\langle a, b, \ldots , c, d\rangle$ такие, что в~одной из 
час\-тей равенства~(\ref{e1-gr})~$\vec{s}_0$ содержится, например, в~левой  
час\-ти~(\ref{e1-gr}), а~в~правой нет. При этом
  \begin{gather*}
(a \prec_1^i b), (a \prec_1^j b)\ \mbox{и } (a \prec_1^k b)\,;\\
\ldots\\
(c \prec_1^i d), (c \prec_1^j d)\ \mbox{и } (c \prec_1^k d)\,.
\end{gather*}
По определению процедуры наложения эти же упорядочения сохранятся на 
элементах множества $\{a, b, \ldots, c, d\}$ также и~в~правой час\-ти~(\ref{e1-gr}). 
Принимая во внимание ассоциативность операции~$\cap$ и~наличие множества 
$\{a, b, \ldots, c, d\}$ одновременно в~множествах признаков, обра\-зо\-вав\-ших 
соответственно цепочки~$\vec{s}_i$, $\vec{s}_j$ и~$\vec{s}_k$, получим, что 
$\{a, b, \ldots, c, d\}$ войдет и~в~правую часть, т.\,е.\ в~(\ref{e1-gr}). Заметим, 
что в~силу ассоциативности операции~$\cap$ и~сохранения на $\{a, b, \ldots, c, 
d\}$ всех трех упорядочений $\prec_1^i$, $\prec_1^j$ и~$\prec_1^k$ множество 
$\{a, b, \ldots, c, d\}$ нерасширяемо ни одним дополнительным элементом~$z$ 
из~$\mathbf{U}$. Следовательно, оно представляет собою максимальное по 
вложению мно\-же\-ст\-во признаков, входящих и~в~правую часть~(\ref{e1-gr}) как 
общая часть цепочек~$\vec{s}_i$, $\vec{s}_j$ и~$\vec{s}_k$, что противоречит 
предположению. Утверждение доказано.
  
  \smallskip
  
  Далее с~помощью операции~$\otimes$ может быть определено 
отношение~$R^{\otimes}$ наложимости множеств цепочек 
из~$\mathbf{Dom}(\mathbf{U},\boldsymbol{\Omega})$.
  
  \smallskip
  
  \noindent
  \textbf{Определение~5.}\ Для каж\-дой пары~$O_{i_1}$ и~$O_{i_2}$ элементов 
множества $\mathbf{Dom}(\mathbf{U}, \boldsymbol{\Omega})$ определим 
отношение~$R^{\otimes}$ их взаимной на\-ло\-жи\-мости сле\-ду\-ющим образом: 
$O_{i_1} R^{\otimes} O_{i_2}$ тогда и~только тогда, когда результат их 
наложения $O_{i_1}\otimes O_{i_2}\hm\not= \varnothing$.
  
  \smallskip
  
  Принимая во внимание утверждение~1, можно заключить, что отношение 
на\-ло\-жи\-мости~$R^{\otimes}$ представляет собой отношение типа 
\textit{сходства} (\textit{то\-ле\-рант\-ности}~--- см., например,~\cite{4-gr, 5-gr}). 
Таким образом, на элементах множества $\mathbf{Dom}(\mathbf{U}, 
\boldsymbol{\Omega})$ могут быть сформированы классы сходства вида
  \begin{equation*}
  \mathbf{T}(O_{i_1})=\left\{ O_{i_2}\vert O_{i_1} R^{\otimes} O_{i_2}\right\}\,.
  %\label{e2-gr}
  \end{equation*}
  
  Далее, для фиксированного $V \hm= V_0 \hm= A\otimes B$, $A, B \hm\in 
\mathbf{Dom}(\mathbf{U}, \boldsymbol{\Omega})$, мож\-но найти 
соответствующие подклассы фор\-ми\-ру\-емых классов сходства:
  $$
  \mathbf{E}\mathbf{Q}_{V_0} =\left\{ \langle O_{i_1}, O_{i_2}\rangle \vert 
O_{i_1}\otimes O_{i_2} \otimes V_0=V_0\right\}\,.
  $$
  
  \noindent
  \textbf{Утверждение~2.}\ \textit{Пусть заданы множество 
$\mathbf{Dom}(\mathbf{U}, \boldsymbol{\Omega})$, операция~$\otimes$, 
некоторый объект $O_r \hm\in \mathbf{Dom}(\mathbf{U}, \boldsymbol{\Omega})$, 
а~также некоторый подобъект~$V$, характеризующий сходство объекта~$O_r$ 
по крайней мере с~одним из элементов класса сходства 
$\mathbf{T}_{\mathbf{Dom}(\mathbf{U}, \boldsymbol{\Omega})}(O_r)$. Множество всех таких 
элементов $O_s \hm\in \mathbf{T}_{\mathbf{Dom}(\mathbf{U}, \boldsymbol{\Omega})}(O_r)$, 
что $O_r\otimes O_s\otimes V \hm= V$, пред\-став\-ля\-ет собой класс экви\-ва\-лент\-ности} 
$\mathbf{E}\mathbf{Q}_{\mathbf{Dom}(\mathbf{U}, \boldsymbol{\Omega}),v}(O_r)$.
  
  \smallskip
  
  \noindent
  Д\,о\,к\,а\,з\,а\,т\,е\,л\,ь\,с\,т\,в\,о\,.\ \ Рефлексивность и~сим\-мет\-рич\-ность 
отношения $E_v^{\otimes}$, по\-рож\-да\-емо\-го из уже определенного 
отношения~$R^\otimes$ фиксацией не\-пус\-то\-го результата~$V$ применения 
операции сходства~$\otimes$, определяется со\-от\-вет\-ст\-ву\-ющи\-ми свойствами 
отношения~$R^\otimes$. Транзитивность следует из того, что, будучи общей 
частью одновременно пар $\langle O_r, O_s\rangle$ и~$\langle O_s, O_t\rangle$, 
заданный подобъект~$V$ оказывается также общей частью объектов~$O_r$ 
и~$O_t$. Утверждение~2 доказано.
  
  \smallskip
  
  Таким образом, имея исходно заданное множество событий, можно 
корректным образом уточнить понимание сходства таких событий и~получить\linebreak 
воз\-мож\-ность выделять классы эк\-ви\-ва\-лент\-ности множеств используемых 
цепочек. Другими словами, можно сформировать процедурную схему обуче\-ния 
на прецедентах (описаниях событий) и~классификации вновь наблю\-да\-емых 
событий~\cite{6-gr}, в~которой диа\-гнос\-ти\-ка вновь наблюдаемых событий будет 
опираться на возможность их корректного отнесения к~уже построенным ранее 
на име\-ющей\-ся обуча\-ющей выборке (и~дополняемым при ее расширениях) 
классам эк\-ви\-ва\-лент\-ности описаний прецедентов.
  
  При построении программной реализации предлагаемой процедурной схемы 
существенными оказываются оценки слож\-ности вы\-чис\-ле\-ний. Для демонстрации 
особенностей рассматриваемой ситуации обратимся к~двум ха\-рак\-те\-ри\-зу\-ющим ее 
аспектам, а~именно: к~проб\-ле\-ме ско\-рости рос\-та размеров автоналожений с~рос\-том 
длины исходных цепочек со\-сто\-яний, а~так\-же к~оценке ем\-кости ре\-зуль\-ти\-ру\-ющих 
мно\-жеств наложений, ха\-рак\-те\-ри\-зу\-ющих сходства ана\-ли\-зи\-ру\-емых событий.
  
  При обсуждении скорости роста размеров автоналожений полезно учитывать 
также и~содержа\-тельные особенности рас\-смат\-ри\-ва\-емой задачи~--- анализ 
<<треков>> развертывания компьютерных атак. Заметим, что если оста\-вить 
в~стороне легко <<узнаваемые>> массовые атаки со значительными объемами 
по\-вто\-ря\-ющих\-ся <<типовых>> действий, то, сфокусировавшись на анализе 
небольших выборок однородных по вызываемым вредоносным эффектам 
целенаправленных воз\-дей\-ст\-вий, мож\-но ожидать не очень <<длинных>> 
последовательностей со\-сто\-яний, которые характеризуются <<не очень час\-то>> 
повторяющимися вхождениями отдельных признаков. Таким образом, остается 
надежда на <<по\-гру\-жа\-емость>> вы\-чис\-ле\-ний как со\-от\-вет\-ст\-ву\-ющих 
автоналожений, так и~сходств ана\-ли\-зи\-ру\-емых цепочек в~разумные ресурсы 
(необходимые объемы памяти и~быст\-ро\-дей\-ст\-вия) современных вы\-чис\-ли\-тель\-ных 
устройств.
  
  В части оценки емкости характерных классов объектов, порождаемых при 
формировании сходств описаний событий, в~част\-ности чис\-ла классов 
экви\-ва\-лент\-ности, по\-рож\-да\-емых отношением на\-ло\-жи\-мости~$R^\otimes$ и~его 
суже\-ни\-ем~$E_v^\otimes$, ситуация оказывается, к~сожалению, существенно 
хуже. При этом ситуация со\-от\-вет\-ст\-ву\-ет положению дел, известному и~при 
работе с~другими типами данных (см., например,~\cite{7-gr}). Здесь можно 
предъявить некоторые переборные задачи, для которых доказуема 
при\-над\-леж\-ность к~классу $\#\mathrm{PC}$ так на\-зы\-ва\-емых \textit{перечислительно 
полных} задач (см., например,~\cite{8-gr, 9-gr, 10-gr, 11-gr}).
  
  Рассмотрим частный случай задачи о~сходстве событий, пред\-став\-ля\-емых 
последовательностями со\-сто\-яний, в~котором для каждого входящего в~исходную 
обуча\-ющую выборку $\boldsymbol{\Omega} \hm = \{\mathbf{Е}_1, \mathbf{Е}_2, 
\ldots, \mathbf{Е}_r\}$ события $\mathbf{E}_i \hm= \langle \mathbf{St}_1^i, 
\mathbf{St}_2^i, \ldots, \mathbf{St}_k^i \rangle$ каждое из ха\-рак\-те\-ри\-зу\-ющих его 
со\-сто\-яний~$\mathbf{St}_r^i$ может иметь общие признаки из исходного 
множества~$\mathbf{U}$ лишь с~признаками из со\-от\-вет\-ст\-ву\-юще\-го (т.\,е.\ 
име\-юще\-го тот же номер~$r$) со\-сто\-яния $\mathbf{St}_r^j$ любого другого 
события~$\mathbf{E}_j$ из~$\boldsymbol{\Omega}$.
  
  \smallskip
  
  \noindent
  \textbf{Утверждение~3.}\ \textit{Задача о числе классов эквивалентности 
$\mathbf{E}\mathbf{Q}_{V_0}$, фор\-ми\-ру\-емых в~множестве 
$\mathbf{Dom}(\mathbf{U}, \boldsymbol{\Omega})$ с~помощью 
операции~$\otimes$ и~по\-рож\-да\-емых ею отношений~$R^\otimes$ 
и~$E_v^\otimes$, принадлежит классу $\#\mathrm{PC}$ перечислительно полных задач.}
  
  \smallskip
  
  \noindent
  Д\,о\,к\,а\,з\,а\,т\,е\,л\,ь\,с\,т\,в\,о\,.\ \ Предложенная част\-ная версия задачи 
о~сходстве событий, представленных последовательностями со\-сто\-яний, может 
быть сведена к~задаче о~сходстве множеств признаков (см.,  
например,~\cite{7-gr}). Действительно, условие непустоты пересечения 
подмножеств признаков только в~со\-от\-вет\-ст\-ву\-ющих, име\-ющих один и~тот же 
номер~$r$, со\-сто\-яни\-ях $\mathbf{St}_r^i, \ldots , \mathbf{St}_r^j$ поз\-во\-ля\-ет для 
каж\-до\-го события~$\mathbf{E}_i$ собрать все фор\-ми\-ру\-ющие его признаки по 
всем со\-сто\-яни\-ям в~единое множество 
  $$
  \mathbf{U}(\mathbf{E}_i) = \mathbf{St}_1^i \cup \mathbf{St}_2^i \cup \cdots 
\cup \mathbf{St}_k^i\,,
  $$
  не опасаясь утраты порядков~$\prec_1^i$ их вхож\-де\-ния 
  в~описание~$\mathbf{E}_i$. Такой общий для всех событий 
из~$\boldsymbol{\Omega}$ порядок~$\prec_1$ может быть автоматически 
вос\-ста\-нов\-лен с~учетом вхождений для каж\-до\-го $s\hm\in\{1, 2, \ldots , k\}$ всех 
попарно не\-пе\-ре\-се\-ка\-ющих\-ся объединений вида $\mathbf{St}_s\hm = 
\mathop{\cup}\nolimits^r_{i=1} \mathbf{St}_s^i$ в~исходное множество 
признаков~$\mathbf{U}$, т.\,е.\ порядок на признаках в~данном случае 
оказывается несущественным, и~анализу (см.\ задачу о~чис\-ле классов 
эквивалентности в~\cite{7-gr}) подлежат лишь общие час\-ти со\-от\-вет\-ст\-ву\-ющих 
множеств признаков $\mathbf{U}(\mathbf{Е}_i)$, пред\-став\-ля\-ющих в~данном 
случае события из обуча\-ющей выборки~$\boldsymbol{\Omega}$.

  Для доказательства принадлежности задачи о~чис\-ле классов эк\-ви\-ва\-лент\-ности 
классу $\#\mathrm{PC}$ продемонстрируем полиномиальную сводимость к~ней известной 
(см., например,~\cite{10-gr}) переборной задачи о~вы\-пол\-ни\-мости монотонной 
булевской функции~${\boldsymbol{\Phi}}$. Задача со\-сто\-ит в~следующем. Для 
функции~${\boldsymbol{\Phi}}$, 
пред\-став\-лен\-ной в~виде \mbox{2-КНФ}
  \begin{multline}
{\boldsymbol{\Phi}}\left(x_1, x_2, \ldots , x_n\right) = 
   {\boldsymbol{\Phi}}_1\&{\boldsymbol{\Phi}}_2 \& \cdots \&{\boldsymbol{\Phi}}_m ={}\\
   \!\!{}= \left(x_{11}\vee  x_{12}\right)\&\left(x_{21} \vee  x_{22}\right)\& 
   \cdots \&\left(x_{m1} \vee  x_{m2}\right)\,,\!\!
   \label{e3-gr}
   \end{multline}
   найти число наборов значений переменных $x_1, x_2, \ldots , x_n$, 
   вы\-пол\-ня\-ющих функцию~${\boldsymbol{\Phi}}$.

  Схема сведения одной задачи к~другой:
  \begin{itemize}
\item по заданной функции~${\boldsymbol{\Phi}}$ полиномиально быст\-ро строится пара 
множеств $\mathbf{U}_{\Phi}, {\boldsymbol{\Omega}}_{\Phi}$ такая, что 
классы эквивалентности объектов из~$\boldsymbol{\Omega}_{\Phi}$ над 
множеством~$\mathbf{U}_{\Phi}$ <<пе\-ре\-чис\-ля\-ют>> все нули 
функ\-ции~${\boldsymbol{\Phi}}$;
\item размер множества единиц функ\-ции~${\boldsymbol{\Phi}}$ (т.\,е.\ множества наборов, 
вы\-пол\-ня\-ющих эту функ\-цию) есть размер дополнения множества нулей 
функ\-ции~${\boldsymbol{\Phi}}$ до множества всех возможных наборов значений переменных 
$x_1, x_2, \ldots , x_n$.
  \end{itemize}
  
  Итак, пусть~${\boldsymbol{\Phi}}$ имеет вид~(\ref{e3-gr}). Не теряя общ\-ности, будем считать, 
что в~записи функции~${\boldsymbol{\Phi}}$ все конъюнкты~${\boldsymbol{\Phi}}_i$ различны. В~противном 
случае функция~${\boldsymbol{\Phi}}$ может быть приведена к~нужному виду полиномиально 
слож\-ной процедурой~--- сортировкой образующих ее конъюнктов.
  
  \textit{Примитивным} нулем номер~$i$ функции 
  $$
  {\boldsymbol{\Phi}}\left(x_1, x_2, \ldots , x_n\right) 
= {\boldsymbol{\Phi}}_1\&{\boldsymbol{\Phi}}_2\& \cdots  \&{\boldsymbol{\Phi}}_i\& 
\cdots \&{\boldsymbol{\Phi}}_m
$$ 
будем называть такой 
булевский вектор~$\boldsymbol{\sigma}_i$, что $x_{i1}\hm = 0$, $x_{i2}\hm = 0$, 
а~все остальные переменные равны~1. 
  
  \textit{Первым сопряженным} с~примитивным нулем $\boldsymbol{\sigma}_i$ 
будем называть всякий такой булевский вектор $\boldsymbol{\sigma}_i^*$, что 
ровно одна из $n\hm-2$ переменных множества $\mathbf{X}_i \hm= \{x_1, x_2, 
\ldots, x_n\} \backslash  \{x_{i1}, x_{i2}\}$ принимает значение~0. 
  
  Положим $\mathbf{U}_{{\boldsymbol{\Phi}}}\hm= \{a^1, a^2, \ldots, a^n\}$, 
а~множество~$\boldsymbol{\Omega}_{\mathbf{\Phi}, i}$ для 
каждого~$\boldsymbol{\sigma}_i$ определим как
  $$
\boldsymbol{\Omega}_{\mathbf{\Phi}, i} = \left\{\mathbf{U}_{\boldsymbol{\Phi}}, 
A^i_0, A_1^i, A_2^i,\ldots A_n^i\right\},
$$
где
\begin{align*}
A_0^i &= \mathbf{U}_{\boldsymbol{\Phi}}\backslash \left\{a^{i1}, a^{i2}\right\}\,;\\
A_1^i &= \mathbf{U}_{\boldsymbol{\Phi}}\backslash \left\{a^{i1}, a^{i2}\right\} 
\backslash \{a^1\}\,;\\
A_2^i &= \mathbf{U}_{\boldsymbol{\Phi}}\backslash \left\{a^{i1}, a^{i2}\right\} 
\backslash \{a^2\}\,;\\
&\hspace*{13mm}\ldots\\
A_n^i &= \mathbf{U}_{\boldsymbol{\Phi}}\backslash  \left\{a^{i1}, a^{i2}\right\} 
\backslash \{a^n\}.
\end{align*}
При этом, как нетрудно убедиться, для каждого~$i$ имеет место равенство: 
$$
A_0^i = A^i_{i1} = A^i_{i2} = \mathbf{U}_{\boldsymbol{\Phi}}\backslash  
\left\{a^{i1}, a^{i2}\right\}.
$$
  
  В рассматриваемой ситуации всякий нуль функции~$\boldsymbol{\Phi}$, 
который сопряжен с~заданным примитивным нулем~$\boldsymbol{\sigma}_i$, 
может быть представлен соответствующим сходством, которое будем называть 
его \textit{исчерпывающим}, и~порождаемым им классом эквивалентности по 
отношению эквивалентности $\mathbf{E}^{\cap}$. $\mathbf{E}^{\cap}$ 
порождено операцией~$\cap$ пересечения множеств 
из~$\boldsymbol{\Omega}_{\boldsymbol{\Phi}, i}$. Действительно, с~точностью до 
совпадения~$A_0^i$, $A^i_{i1}$ и~$A^i_{i2}$ любое двух- и~более элементное 
подмножество множества~$\boldsymbol{\Omega}_{\boldsymbol{\Phi}, i}$ 
порождает уникальное множество общих для всех входящих 
в~$\boldsymbol{\Omega}_{\boldsymbol{\Phi}, i}$ элементов 
из~$\mathbf{U}_{\boldsymbol{\Phi}}$. Это обусловлено <<диагональной>> по 
вхождению в~него элементов из~$\mathbf{U}_{\boldsymbol{\Phi}}$ структурой 
множества~$\boldsymbol{\Omega}_{\boldsymbol{\Phi}, i}$. 
  
  При этом имеет место и~обратное соответствие. Каждое исчерпывающее 
сходство элементов множества~$\boldsymbol{\Omega}_{\boldsymbol{\Phi}, i}$ 
может быть сопоставлено одно\-му и~только одно\-му из нулей исходной булевской 
функции~$\boldsymbol{\Phi}$. Однозначность задействованного сопоставления 
есть следствие используемого здесь способа <<кодировки>> единиц каждого 
восстанавливаемого булевского вектора~${\boldsymbol{\sigma}}$. 
  
  Пусть далее
  \begin{equation}
  {\boldsymbol{\Omega}}_{\boldsymbol{\Phi}} =\mathop{\bigcup}\limits^m_{i=1} 
\boldsymbol{\Omega}_{\boldsymbol{\Phi}, i}\,.
   \label{e4-gr}
   \end{equation}
Таким образом, нуль каждого конъюнкта~$\boldsymbol{\Phi}_i$ из  
\mbox{2-КНФ}-пред\-став\-ле\-ния функции~$\boldsymbol{\Phi}$ <<кодируется>> 
$n\hm+1$ множеством $A^i\hm = \{A_0^i, A_1^i, A_2^i, \ldots , A_n^i\}$ по 
следующей схеме:
\begin{itemize}
\item $A_0^i$ представляет ситуацию, когда $x_{i1}\hm = x_{i2}\hm = 0$, а~все 
остальные переменные функции~$\boldsymbol{\Phi}$ принимают значение~1;
\item каждое из множеств~$A_j^i$, где $j\hm\in\{1, 2, \ldots , n\}$, 
характеризует ситуацию, когда одна отличная от~$x_{xi1}$ и~$x_{i2}$ 
переменная~$x_j$ (и~только она) обращается в~0 одновременно с~$x_{i1}\hm = 
x_{i2}\hm = 0$;
\item по значениям индекса~$i$ подобных коллекций множеств вида~$A_j^i$, 
$j\hm\in \{1, 2, \ldots, n\}$, будет ровно~$m$ по числу 
конъюнктов~$\boldsymbol{\Phi}_i$ в~2-КНФ исходно заданной 
функции~$\boldsymbol{\Phi}$;
\item множество всех нулей функции~$\boldsymbol{\Phi}$, являющихся 
одновременно нулями конъюнкта~$\boldsymbol{\Phi}_i$, будет 
перечисляться множеством всех ис\-чер\-пы\-вающих пересечений (сходств) 
объектов из множества~$\boldsymbol{\Omega}_{\boldsymbol{\Phi}, i}$. Можно 
показать, например индукцией по числу пересекаемых элементов 
из~$\boldsymbol{\Omega}_{\boldsymbol{\Phi}, i}$, что с~точностью до 
равенства $A_0^i\hm= A^i_{i1}\hm = A^i_{i2}$ каж\-дое пересечение объектов 
из множества~$\boldsymbol{\Omega}_{\boldsymbol{\Phi}, i}$ будет 
исчерпывающим.
\end{itemize}
  
  В ситуации, когда~$\boldsymbol{\Phi}$ обращается в~нуль, по крайней мере 
один из ее конъюнктов~$\boldsymbol{\Phi}_i$ непременно также обращается 
в~нуль. Нетрудно убедиться, что каждый из нулей функции~$\boldsymbol{\Phi}$ 
может быть рассмотрен как суперпозиция примитивных и~сопряженных с~ними 
нулей, т.\,е.\ в~ситуации, когда одновременно реализуются соответствующие 
примитивные нули функции~~$\boldsymbol{\Phi}$.
  
  Разумеется, в~объединенном 
множестве~$\boldsymbol{\Omega}_{\boldsymbol{\Phi}}$ будут встречаться 
одинаковые элементы. Такие по\-вто\-ры порождаются для сопряженных 
с~примитивными нулями, каждый из которых формируется суперпозициями 
примитивных нулей. Очевидно, что наличие таких <<повторов>> не может 
расширить множество исчерпывающих сходств (пересечений) элементов 
сводного множества~~$\boldsymbol{\Omega}_{\boldsymbol{\Phi}}$  
в~(\ref{e4-gr}).
  
  
  Итак, каждое исчерпывающее пересечение (сходство) объектов из 
множества~$\boldsymbol{\Omega}_{\boldsymbol{\Phi}}$ будет представлять  
ка\-кой-ли\-бо из нулей функции~$\boldsymbol{\Phi}$. И~наобо\-рот: каждый из 
нулей функции~$\boldsymbol{\Phi}$ как соответствующая суперпозиция 
примитивных и~сопряженных\linebreak с~ними нулей функции~$\boldsymbol{\Phi}$ 
может быть пред\-став\-лен (исчерпывающим) сходством объектов из\linebreak 
множества~$\boldsymbol{\Omega}_{\boldsymbol{\Phi}}$. Таким образом, 
множество нулей функции~$\boldsymbol{\Phi}$ и~множество исчерпывающих 
сходств объектов из множества~$\boldsymbol{\Omega}_{\boldsymbol{\Phi}}$ 
могут быть взаимно однозначным образом сопоставлены друг другу, т.\,е.\ по 
числу исчерпывающих сходств~$\boldsymbol{\Omega}_{\boldsymbol{\Phi}}$ 
можно полиномиально быстро определить число наборов, выполняющих 
функцию~$\boldsymbol{\Phi}$.
  
  Осталось убедиться, что множества~$\mathbf{U}_{\boldsymbol{\Phi}}$ 
и~$\boldsymbol{\Omega}_{\boldsymbol{\Phi}}$ полиномиально быстро 
порождаются по заданной функции~$\boldsymbol{\Phi}$. Действительно, 
в~~$\mathbf{U}_{\boldsymbol{\Phi}}$ имеется $n\hm = \vert \{x_1, x_2, \ldots , 
x_n\}\vert$ элементов, а~в~$\boldsymbol{\Omega}_{\boldsymbol{\Phi}}$ 
содержится не более чем $(n\hm+1)m$ элементов, каждый из которых содержит 
не более чем $n\hm-2$ элементов из~~$\mathbf{U}_{\boldsymbol{\Phi}}$.

\section{Свертки цепочек состояний}

  Обратимся к~анализу эффектов совместного вхождения признаков из 
исходного~$\mathbf{U}$ в~описания событий обучающей выборки 
$\boldsymbol{\Omega}\hm = \{\mathbf{Е}_1, \mathbf{Е}_2, \ldots , \mathbf{Е}_r\}$.
  
  \smallskip
  
  \noindent
  \textbf{Определение~6.}\ Сверткой $\mathbf{U}(\mathbf{Е})$ события 
$\mathbf{Е}\hm = \langle \mathbf{St}_1, \mathbf{St}_2, \ldots, \mathbf{St}_k\rangle$ 
будем называть объединение всех признаков $\mathbf{U}(\mathbf{Е}) \hm= 
\mathbf{St}_1\cup \mathbf{St}_2\cup \cdots \cup  \mathbf{St}_k$, встречающихся 
в~описаниях всех его состояний.
  
  \smallskip
  
  Соберем для $\boldsymbol{\Omega}\hm = \{\mathbf{Е}_1, \mathbf{Е}_2, \ldots , 
\mathbf{Е}_r\}$ множества $\mathbf{U}(\mathbf{Е}_i)$ в~новую обучающую 
выборку $\boldsymbol{\Omega}^*\hm = \{\mathbf{U}(\mathbf{Е}_1), 
\mathbf{U}(\mathbf{Е}_2), \ldots, \mathbf{U}(\mathbf{Е}_r)\}$. Операцию 
сходства на элементах~$\boldsymbol{\Omega}^*$ определим как  
тео\-ре\-ти\-ко-мно\-же\-ст\-вен\-ное пересечение~$\cap$. Далее построим 
соответствующие классы эквивалентности и~сформируем\linebreak из них, используя 
отношение~$\subseteq$ взаимной вложимости соответствующих таким классам 
эквивалентности подмножеств объектов из~$\boldsymbol{\Omega}^*$, частично 
упорядоченные множества. Будем рассматривать\linebreak диаграммы таких частично 
упорядоченных множеств, сфокусировав внимание на совместном вхож\-де\-нии 
признаков из исходного~$\mathbf{U}$ одновременно в~каждое из множеств 
объектов, порождающих какую-либо вершину такой диаграммы.
  
  \smallskip
  
  \noindent
  \textbf{Определение~7.}\ Замыканием признака $a\hm = \{a_1, a_2, \ldots , 
a_n\}\hm\in \mathbf{U}$ относительно множества 
объектов~$\boldsymbol{\Omega}^*$ будем называть множество  
$[a]_{ \mathbf{U},{\boldsymbol{\Omega}}^*}$, представляющее собой сходство 
(пересечение) всех тех и~только тех объектов~$\mathbf{U}(\mathbf{Е})$ 
из~$\boldsymbol{\Omega}^*$, которые содержат данный признак~$a$.
  
  Используя отношение~$\subseteq$ взаимной вложимости элементов 
множества 
  \begin{equation}
  \hspace*{-2.8mm}\mathrm{GC}^1(\mathbf{U},\boldsymbol{\Omega}^*) =\left\{ \left[ 
a_1\right]_{\mathbf{U},{\boldsymbol{\Omega}}^*}\!, \left[ 
a_2\right]_{\mathbf{U},{\boldsymbol{\Omega}}^*}\!, \ldots , \left[ 
a_n\right]_{\mathbf{U},{\boldsymbol{\Omega}}^*}\!\right\}\!\!\!
  \label{e5-gr}
  \end{equation}
как (непустых) подмножеств заданного множества~$\mathbf{U}$, сформируем 
частично упорядоченное множество $\langle \mathrm{GC}^1(\mathbf{U}, 
{\boldsymbol{\Omega}}^*), \subseteq\rangle$, диаграмму 
$\mathrm{D}\_{\mathrm{GC}}^1(\mathbf{U},{\boldsymbol{\Omega}}^*)$ частичного порядка, которую 
будем (как и~в~\cite{12-gr}) называть \textit{каркасом} для 
$\mathrm{GC}^1(\mathbf{U},{\boldsymbol{\Omega}}^*)$.

  Порядок расположения элементов множества~(\ref{e5-gr}) на диаграмме 
$\mathrm{D}\_\mathrm{GC}^1(\mathbf{U},{\boldsymbol{\Omega}}^*)$ задает естест\-вен\-ную 
последовательность первых вхождений признаков~$a_i$ из исходного 
множества~$\mathbf{U}$ в~объекты~$\mathbf{U}(\mathbf{Е})$ 
из~$\boldsymbol{\Omega}^*$. Будем обозначать это отношение частичного 
порядка на элементах множест-\linebreak ва~$\mathbf{U}$ как~$\prec_2$. Выражение 
$a_{l_i}\prec_2 a_{l_j}$ означает, что\linebreak на диаграмме 
$\mathrm{D}\_\mathrm{GC}^1(\mathbf{U},{\boldsymbol{\Omega}}^*)$ первое вхождение\linebreak 
признака~$a_{l_i}$ доминируется первым вхождением признака~$a_{l_j}$. 
Дополнительно каждая из вершин диаграммы 
$\mathrm{D}\_\mathrm{GC}^1(\mathbf{U},{\boldsymbol{\Omega}}^*)$ демонстрирует, какие 
признаки входят в~описания событий из исходного 
множества~$\boldsymbol{\Omega}$ \textit{вместе} с~порождающим путем 
формирования его замыкания $[a_i]_{\mathbf{U},{\boldsymbol{\Omega}}^*}$ эту вершину 
признаком~$a_i$.
  
\section{Проблема управления ресурсами при~противодействии 
атакам}

  При организации противодействия компьютерным атакам представляется 
естественным задействовать любые возможности для рационального 
использования имеющихся ресурсов. В~ситуации, когда атаки могут быть 
описаны как последовательности состояний объекта защиты, один из вариантов 
такого управления ресурсами противодействия может быть основан на 
совместном использовании введенных выше отношений~$\prec_1$ и~$\prec_2$. 
Сделать это можно, в~частности, по следующей схеме:
  \begin{itemize}
\item описывается каждый факт атаки как последовательности состояний, 
образующих формализованное представление соответствующего события. 
Такие описания будем называть \textit{примерами};
\item дополнительно формируются такие последовательности состояний, 
которые похожи по своей структуре на факты примеров, но не являются 
отражением действий, ведущих к~вредоносным эффектам. Эти описания 
будем называть \textit{контрпримерами}. Примеры и~контрпримеры будем 
называть \textit{прецедентами};
\item используя порождаемое на описаниях событий отношение сходства, 
выделяется каждая такая цепочка признаков~$\vec{s}$, которая является 
общей для всех событий, попадающих в~соответствующий класс 
эквивалентности $\mathbf{EQ}_s$ по отношениям~$R^\otimes$ 
и~$E_v^\otimes$. Каждую порожденную таким образом цепочку фиксируем 
в~базе знаний (БЗ) системы противодействия анализируемым атакам. При 
этом при анализе текущего состояния~$\mathbf{O}$ можно при наблюдении 
появления признаков $a_1^i, a_2^i, \ldots$ идентифицировать в~БЗ все такие 
цепочки вида $\vec{s}^{\,i}\hm= \langle a_1^i, a_2^i, \ldots , a_t^i,\ldots \rangle$, 
которые позволяют предположить дальнейшее\linebreak в~смысле 
отношения~$\prec_1^i$ <<появление>> в~рас\-смат\-ри\-ва\-емой текущей ситуации 
каждого из соответствующих каждой цепочке признаков~$a^i_t$.\linebreak Далее 
принимается во внимание заранее по\-рож\-да\-емая на свертках описаний 
исходной выборки прецедентов и~сохраняемая в~БЗ информация 
о~совместном (с~учетом отношения~$\prec_2$) вхождении признаков 
из~$\mathbf{U}$ в~сходства событий исходной обучающей выборки 
прецедентов; 
\item последовательно противодействовать развивающейся атаке и~ее 
продолжениям, известным по уже накопленной базе прецедентов, на тех 
направлениях, которые соответствуют появлению совместно с~уже 
зафиксированными, а~также далее ожидаемыми в~смысле каждого 
отношения~$\prec_1^i$ признаками из каждой цепочки $\vec{s}^{\,i}\hm= \langle 
a_1^i, a_2^i, \ldots , a_t^i,\ldots \rangle$ сопутствующих им, т.\,е.\ совместно 
встречающихся с~ними на диаграмме 
$\mathrm{D}\_\mathrm{GC}^1(\mathbf{U},{\boldsymbol{\Omega}}^*)$ признаков, а~именно: 
сперва признакам $b_{11}^i, b_{12}^i, \ldots$, сопутствующим 
признаку~$a_1^i$ в~смысле отношения~$\prec_2$, затем признакам $b_{21}^i, 
b_{22}^i,\ldots ,$ сопутствующим признаку~$a_2^i$ в~смыс\-ле 
отношения~$\prec_2$, и~т.\,д. 
\end{itemize}
  
  Очевидным аргументом в~пользу рациональ\-ности именно такой тактики 
выбора приоритетов противодействия может быть следующее соображение. 
Если, как показывают результаты обучения на прецедентах исходной 
обучающей выборки, отраженные в~отношениях~$\prec_1$ и~$\prec_2$ 
признаки~$a_j^i$\linebreak и~$b^i_{j1}, b^i_{j2}, \ldots$ в~ходе результативных атак 
проявляются вместе и~тем более если это совместное вхож\-дение имеет также 
и~\textit{ясную содержательную ин\-терпретацию}, то при фиксации 
появления~$a^i_j$\linebreak представляет\-ся естественным в~приоритетном порядке 
противостоять появлению $b^i_{j1}, b^i_{j2},\ldots$ При этом реакция на 
возможные разворачивания различных вредоносных цепочек $\vec{s}^{\,i}\hm= 
\langle a_1^i, a_2^i, \ldots , a_t^i,\ldots\rangle$ на ранних стадиях их 
формирования оставляет больше надежд на эффективное использование 
ресурсов, имеющихся для организации противодействия.
  
  Следует отметить существенную роль анализа контрпримеров в~усечении 
артефактов, порож\-да\-емых при обучении на примерах. Порождение одной и~той 
же цепочки при формировании классов эквивалентности как на примерах, так 
и~на контрпримерах можно рассматривать как сигнал о ее несущественности 
при анализе имеющейся выборки прецедентов, т.\,е.\ о~возможности ее 
отнесения к~\textit{артефактам} переобучения. Подобное игнорирование 
артефактов~--- один из способов борьбы с~\textit{эффектом} так называемого 
\textit{переобучения} (см., например,~\cite{13-gr}), характерным для 
использования различных методов машинного обучения.

\section{Заключение}

  Представлен подход к~организации управления ресурсами при формировании 
противодействия компьютерным атакам, характеризуемым разворачивающимися 
во времени последовательностями действий (событий).
  
  В его основу положена техника машинного обуче\-ния на прецедентах, 
базирующаяся на формализации сходства как бинарной алгебраической 
операции и~порождении на элементах описаний наблюдаемых атак двух 
отношений частичного порядка~$\prec_1$ и~$\prec_2$. Отношение~$\prec_1$ 
отражает \textit{последовательность} появления наблюдаемых активностей, 
отношение~$\prec_2$ отражает \textit{совместное воздействие} наблюдаемых 
активностей на объект защиты.
  
  Приведенный тео\-ре\-ти\-ко-мно\-же\-ст\-вен\-ный вариант формализованного 
описания анализируемых атак может быть расширен за счет использования 
более сложных структур данных. Так, в~частности, для описания стадий, 
последовательности которых представляют фиксируемые  
со\-бы\-тия-пре\-це\-ден\-ты, могут быть использованы не только собственно 
множества признаков, но и,~например, множества признаков вместе 
с~отношениями на них, т.\,е.\ графы, в~том чисде с~числовыми метками на 
вершинах и~ребрах.
  
  Интересное расширение представленного подхода состоит в~использовании 
тернарного отношения~\cite{14-gr}, связывающего фиксируемые вредоносные 
воздействия с~возникающими негативными эффектами в~условиях (т.\,е.\ 
с~непременным учетом) тех или иных вариантов противодействия влиянию 
наблюдаемых вредоносных воздействий, т.\,е.\ отношения вида ($\langle$вариант 
вредоносных действий$\rangle$, $\langle$вариант  
про\-ти\-во\-дей\-ст\-вия$\rangle$)\;$\Rightarrow$\;($\langle$воз\-ни\-ка\-ющий 
в~результате эффект$\rangle$).
  
  Для эффективной организации управления перебором вариантов при 
формировании соответствующих классов эквивалентности в~рассматриваемой 
процедуре, как и~в~работах~\cite{12-gr, 15-gr}, может быть использована 
техника целенаправленной навигации на последовательно порождаемых 
диаграммах взаимной вложимости классов эквивалентности.

{\small\frenchspacing
 {%\baselineskip=10.8pt
 \addcontentsline{toc}{section}{References}
 \begin{thebibliography}{99}
     \bibitem{1-gr}
     \Au{Grusho A., Levykin~M., Timonina~E., Piskovski~V., Timonina~A.} Architecture of 
consecutive identification of attack to information resources~// 7th Congress (International) on Ultra 
Modern Telecommunications and Control Systems Proceedings.~--- 
Piscataway, NJ, USA: IEEE, 2015. P.~265--268.
    
     \bibitem{3-gr} %2
     \Au{Финн В.\,К.} Синтез познавательных процедур и~проб\-ле\-ма
     индукции~// На\-уч\-но-тех\-ни\-че\-ская информация. Сер.~2, 2009.
     №\,6. С.~1--37.
     
      \bibitem{2-gr} %3
     \Au{Финн В.\,К.} Об интеллектуальном анализе данных~// Новости искусственного 
интеллекта, 2014. №\,3. С.~3--18.

          \bibitem{4-gr}
     \Au{Шрейдер Ю.\,А.} Равенство, сходство, порядок.~--- М.: Наука, 1971. 255~с.
     \bibitem{5-gr}
     \Au{Гусакова С.\,М., Финн~В.\,К.} Сходства и~правдоподобный вывод~// Известия АН 
СССР. Сер. Техническая кибернетика, 1987. №\,5. С.~42--63. 
     \bibitem{6-gr}
     \Au{Финн В.\,К.} Индуктивные методы Д.\,С.~Милля в~сис\-те\-мах искусственного 
интеллекта~// Искусственный интеллект и~принятие решений, 2010. Ч.~I. №\,3. С.~3--21; Ч.~II. 
№\,4. С.~14--40.
     \bibitem{7-gr}
     \Au{Забежайло М.\,И.} О~некоторых оценках сложности вычислений  
в~ДСМ-рас\-суж\-де\-ни\-ях~// Искусственный интеллект и~принятие решений, 2015. Ч.~I. 
№\,1. С.~3--17; Ч.~II. №\,2. С.~3--17. 
    
     \bibitem{9-gr} %8
     \Au{Simon J.} On the difference between one and many (preliminary version)~// 
     Automata, languages and programming~/ Eds. A.~Salomaa, M.~Steinby.~---
     Lecture notes in computer 
science ser.~--- Berlin--Heidelberg: Springer, 1977. Vol.~52. P.~480--491.
     \bibitem{10-gr} %9
     \Au{Valiant L.\,G.} The complexity of enumeration and reliability problems~// SIAM~J. 
Comput., 1979. Vol.~8. P.~410--421.
     \bibitem{11-gr} %10
     \Au{Valiant L.\,G.} The complexity of computing the permanent~// Theor. Comput. Sci., 
1979. Vol.~8. P.~189--201.
 \bibitem{8-gr} %11
     \Au{Гэри М., Джонсон~Д.\,С.} Вычислительные машины и~труд\-но-ре\-ша\-емые 
задачи~/ Пер. с~англ.~--- М.: Мир, 1982. 416~с.
(\Au{Garey~M.\,R., Johnson~D.\,S.} {Computers and intractability: A~guide 
to the theory of NP-completeness}.~--- San Francisco, CA, USA: W.\,H.~Freeman and Co.,
1979. 338~p.)
     \bibitem{12-gr}
     \Au{Забежайло М.\,И.} О~некоторых возможностях управ\-ле\-ния перебором  
в~ДСМ-ме\-то\-де~// Искусственный интеллект и~принятие решений, 2014. Ч.~I. №\,1.  
С.~95--110; Ч.~II. №\,3. С.~3--21.
     \bibitem{13-gr}
     \Au{Everitt B.\,S.} Cambridge dictionary of statistics.~--- Cambridge: Cambridge University 
Press, 2002. 410~p.
     \bibitem{14-gr}
     \Au{Финн В.\,К., Шестерникова~О.\,П.} О~новом варианте обобщенного  
ДСМ-ме\-то\-да~// Искусственный интеллект и~принятие решений, 2016. №\,1. С.~57--63. 
     \bibitem{15-gr}
     \Au{Забежайло М.\,И.} Приближенный ДСМ-ме\-тод на примерах~//  
На\-уч\-но-тех\-ни\-че\-ская информация. Сер.~2, 2014. №\,10. С.~1--12.

 \end{thebibliography}

 }
 }

\end{multicols}

\vspace*{-6pt}

\hfill{\small\textit{Поступила в~редакцию 03.01.18}}

\vspace*{8pt}

%\newpage

%\vspace*{-24pt}

\hrule

\vspace*{2pt}

\hrule

%\vspace*{8pt}


\def\tit{ON~SOME POSSIBILITIES OF~RESOURCE MANAGEMENT 
FOR~ORGANIZING ACTIVE COUNTERACTION TO~COMPUTER 
ATTACKS}

\def\titkol{On~some possibilities of~resource management 
for~organizing active counteraction to~computer 
attacks}

\def\aut{A.\,A.~Grusho, M.\,I.~Zabezhailo, A.\,A.~Zatsarinny, and E.\,E.~Timonina}

\def\autkol{A.\,A.~Grusho, M.\,I.~Zabezhailo, A.\,A.~Zatsarinny, and E.\,E.~Timonina}

\titel{\tit}{\aut}{\autkol}{\titkol}

\vspace*{-9pt}


\noindent
\noindent
Institute of Informatics Problems, Federal Research Center ``Computer Sciences and Control'' of the Russian
Academy of Sciences, 44-2~Vavilov Str., Moscow 119333, Russian Federation



\def\leftfootline{\small{\textbf{\thepage}
\hfill INFORMATIKA I EE PRIMENENIYA~--- INFORMATICS AND
APPLICATIONS\ \ \ 2018\ \ \ volume~12\ \ \ issue\ 1}
}%
 \def\rightfootline{\small{INFORMATIKA I EE PRIMENENIYA~---
INFORMATICS AND APPLICATIONS\ \ \ 2018\ \ \ volume~12\ \ \ issue\ 1
\hfill \textbf{\thepage}}}

\vspace*{3pt}


\Abste{Rational counteraction to computer attacks, described as event sequences, is discussed. The 
approach is based on the mathematical technique of learning by precedents, formalizing similarity as 
a~binary algebraic operation. Similarities of event sequences are analyzed. The learned classes of 
similarity (tolerance classes) are used to recognize computer attacks on initial steps of their life cycle. 
A~problem-oriented resource management technology aimed at developing rational counteraction to 
attacks of the discussed type is presented.}

\KWE{information security; data analysis; similarity as binary algebraic operation; similarity of 
sequences; resource management}



    \DOI{10.14357/19922264180108} 

%\vspace*{-12pt}

\Ack
\noindent
The paper was supported by the
Russian Foundation for Basic Research (project 15-29-07981).



%\vspace*{3pt}

  \begin{multicols}{2}

\renewcommand{\bibname}{\protect\rmfamily References}
%\renewcommand{\bibname}{\large\protect\rm References}

{\small\frenchspacing
 {%\baselineskip=10.8pt
 \addcontentsline{toc}{section}{References}
 \begin{thebibliography}{99} 
     \bibitem{1-gr-1}
     \Aue{Grusho, A., M.~Levykin, E.~Timonina, V.~Piskovski, and A.~Timonina}. 2015. 
Architecture of consecutive identification of attack to information resources. \textit{7th Congress 
(International) on Ultra Modern Telecommunications and Control Systems Proceedings}.
 Piscataway, NJ: IEEE. 265--268.
   
     \bibitem{3-gr-1}
     \Aue{Finn, V.\,K.} 2009. The synthesis of cognitive procedures and the problem of induction. 
\textit{Automatic Documentation Math. Linguistics} 43(3):149--195.
  \bibitem{2-gr-1}
     \Aue{Finn, V.\,K.} 2014. Ob intellektual'nom analize dannykh [On intelligent data analysis]. 
\textit{Novosti iskusstvennogo intellekta} [Artificial Intelligence News] 3:3--18.
     \bibitem{4-gr-1}
     \Aue{Shreyder, J.\,A.} 1971. \textit{Ravenstvo, skhodstvo, poryadok} [Equality, similarity, 
order]. Мoscow: Nauka. 255~p.
     \bibitem{5-gr-1}
     \Aue{Gusakova, S.\,M., and V.\,K.~Finn.} 1987. Skhodstvo i~pravdopodobnyy vyvod 
[Similarity and plausible inference]. \textit{Izvestiya AN SSSR. Ser.~Tekhnicheskaya kibernetika} 
[News of Academy of Sciences of the USSR. Ser. engineering cybernetics] 5:42--63. 
     \bibitem{6-gr-1}
     \Aue{Finn, V.\,K.} 2011--2012. Induktivnye metody D.\,S.~Millya v~sistemakh 
iskusstvennogo intellekta [J.\,S.~Mill's inductive methods in artificial intelligence systems]. 
\textit{Iskusstvennyy intellekt i~prinyatie resheniy} 
[Scientific and Technical Information Processing]. 
Part~I (2011). 38(6):385--402. Part~II (2012). 39(5):241--260.
     \bibitem{7-gr-1}
     \Aue{Zabezhailo, M.\,I.} 2015. O~nekotorykh otsenkakh slozhnosti vichisleniy  
v~DSM-rassuzhdeniyakh [To the computational complexity of hypotheses generation in\linebreak  
JSM-method]. \textit{Iskusstvennyy intellect i~prinyatie resheniy} [Artificial Intelligence and 
Decision Making]. Part~I. 1:3--17. Part~II. 2:3--17. 
     
     \bibitem{9-gr-1} %8
     \Aue{Simon, J.} 1977. On the difference between one and many (preliminary version).
     \textit{Automata, languages and programming}. Eds. A.~Salomaa and M.~Steinby.
     Lecture notes in computer 
science ser. Berlin--Heidelberg: Springer. 52:480--491.
     \bibitem{10-gr-1} %9
     \Aue{Valiant, L.\,G.} 1979. The complexity of enumeration and reliability problems. 
\textit{SIAM J.~Comput.} 8:410--421.
     \bibitem{11-gr-1} %10
     \Aue{Valiant, L.\,G.} 1979. The complexity of computing the permanent. 
     \textit{Theor. Comput. Sci.} 8:189--201.
     
     \bibitem{8-gr-1} %11
     \Aue{Garey, M.\,R., and D.\,S.~Johnson.} 1979. \textit{Computers and intractability: A~guide 
to the theory of NP-completeness}. San Francisco, CA: W.\,H.~Freeman and Co. 338~p.
     \bibitem{12-gr-1}
     \Aue{Zabezhailo, M.\,I.} 2014. O~nekotorykh vozmozhnostyakh upravleniya pereborom 
v~DSM-metode [On some possibilities to control computational complexity of hypotheses generation 
in JSM-method]. \textit{Iskusstvennyy intellect i~pri\-nya\-tie resheniy} [Artificial Intelligence and 
Decision Making]. Part~I. 1:95--110. Part~II. 3:3--21.
     \bibitem{13-gr-1}
     \Aue{Everitt, B.\,S.} 2002. \textit{Cambridge dictionary of statistics}. Cambridge: Cambridge 
University Press. 410~p.
     \bibitem{14-gr-1}
     \Aue{Finn, V.\,K., and O.\,P.~Shesternikova.} 2016. O novom variante obobschennogo  
DSM-metoda [On the new variant of generalized JSM-method]. \textit{Iskusstvennyy intellect 
i~prinyatie resheniy} [Artificial Intelligence and Decision Making] 1:57--63.
     \bibitem{15-gr-1}
     \Aue{Zabezhailo, M.\,I.} 2014. Priblizhennyy DSM-metod na primerakh [Approximate  
JSM-method by examples]. \textit{Nauchno-tekhnicheskaya informatsiya} [Scientific and Technical 
Information Processing]. Ser.~2. 10:1--12.
\end{thebibliography}

 }
 }

\end{multicols}

\vspace*{-6pt}

\hfill{\small\textit{Received January 3, 2018}}

%\vspace*{-10pt}
  
  
  \Contr
  
\noindent
\textbf{Grusho Alexander A.} (b.\ 1946)~--- Doctor of Science in physics and 
mathematics, professor, Head of Laboratory, Institute of Informatics Problems, 
Federal Research Center ``Computer Sciences and Control'' of the Russian 
Academy of Sciences, 44-2~Vavilov Str.,Moscow 119333, Russian Federation; 
\mbox{grusho@yandex.ru}

\vspace*{3pt}

\noindent
\textbf{Zabezhailo Michael I.} (b.\ 1956)~--- Doctor of Science in physics and 
mathematics, Head of Laboratory, Institute of Informatics Problems, Federal 
Research Center ``Computer Sciences and Control'' of the Russian Academy of 
Sciences, 44-2~Vavilov Str.,Moscow 119333, Russian Federation; 
\mbox{m.zabezhailo@yandex.ru}

\vspace*{3pt}

\noindent
\textbf{Zatsarinny Alexander A.} (b.\ 1951)~--- Doctor of Science in technology, 
professor, Deputy Director, Federal Research Center ``Computer Sciences and 
Control'' of the Russian Academy of Sciences, 44-2~Vavilov Str.,Moscow 119333, 
Russian Federation; \mbox{alex250451@mail.ru}

\vspace*{3pt}

\noindent
\textbf{Timonina Elena E.} (b.\ 1952)~--- Doctor of Science in technology, 
professor, leading scientist, Institute of Informatics Problems, Federal Research 
Center ``Computer Sciences and Control'' of the Russian Academy of Sciences,  
44-2~Vavilov Str.,Moscow 119333, Russian Federation; \mbox{eltimon@yandex.ru} 
  
\label{end\stat}


\renewcommand{\bibname}{\protect\rm Литература} 