\def\stat{kovalev}

\def\tit{ТЕОРИЯ КАТЕГОРИЙ КАК МАТЕМАТИЧЕСКАЯ ПРАГМАТИКА  
МОДЕЛЬНО-ОРИЕНТИРОВАННОЙ СИСТЕМНОЙ ИНЖЕНЕРИИ}

\def\titkol{Теория категорий как математическая прагматика  
модельно-ориентированной системной инженерии}

\def\aut{С.\,П.~Ковалёв$^1$}

\def\autkol{С.\,П.~Ковалёв}

\titel{\tit}{\aut}{\autkol}{\titkol}

\index{Ковалёв С.\,П.}
\index{Kovalyov S.}


%{\renewcommand{\thefootnote}{\fnsymbol{footnote}} \footnotetext[1]
%{Работа выполнена при финансовой поддержке РФФИ (проект 17-01-00816).}}


\renewcommand{\thefootnote}{\arabic{footnote}}
\footnotetext[1]{Институт проблем управления им.\
 В.\,А.~Трапезникова Российской академии наук, \mbox{kovalyov@nm.ru}}

%\vspace*{-6pt}
    





\Abst{Развивается предложенный ранее математический аппарат на базе 
теории категорий, который позволяет формально описывать и~строго исследовать 
процедуры применения моделей в~инженерной деятельности, со\-став\-ля\-ющие 
прагматику мо\-дель\-но-ори\-ен\-ти\-ро\-ван\-ной сис\-тем\-ной инженерии (Model-Based Systems 
Engineering, MBSE). В~основе аппарата лежит математическое пред\-став\-ле\-ние 
сборочных чертежей (мегамоделей сис\-тем) диаграммами в~категориях, объектами 
которых служат модели, а~морфизмы представляют действия по сборке моделей 
сис\-тем из моделей компонентов. Предложены, исследованы и~проиллюстрированы 
тео\-ре\-ти\-ко-ка\-те\-гор\-ные методы решения прямых и~обратных прагматических задач 
сборки систем. Выявлена ключевая роль монады диаграмм. Особое внимание уделено 
задаче восстановления конфигурации заданной сис\-те\-мы с~учетом технологических 
ограничений, накладываемых на способы и~процедуры сборки. Приведено 
сопоставление ряда ключевых понятий сис\-тем\-ной инженерии конструкциям теории 
категорий.}

\KW{модельно-ориентированная сис\-тем\-ная инженерия; прагматика; мегамодель; тео\-рия 
категорий; задача восстановления конфигурации; монада диаграмм}

\DOI{10.14357/19922264180112} 
  
%\vspace*{6pt}


\vskip 10pt plus 9pt minus 6pt

\thispagestyle{headings}

\begin{multicols}{2}

\label{st\stat}

\section{Введение}

   Технологии MBSE в~настоящее время интенсивно 
развиваются~\cite{1-kov}. Они поддерживают работу инженеров с~разнообразными 
моделями сложных изделий~--- абстрактными формальными пред\-став\-ле\-ни\-ями, 
пригодными к~автоматической обработке на компьютерах. Модели 
записываются на языках, имеющих хорошо проработанный синтаксис (форму) 
и~семантику (смысл). Всеобъемлющий комплекс моделей образует  
элект\-рон\-но-циф\-ро\-вой макет (digital mock-up)~--- виртуальную копию 
изделия, на которой можно исследовать и~оптимизировать все аспекты 
жизненного цикла <<в~цифре>> прежде, чем воплощать <<в~железе>>, снижая 
затраты и~сокращая сроки выполнения работ.
   
   И все же внедрение технологий MBSE в~инженерную деятельность 
происходит фрагментарно и~часто не выходит за рамки <<пилотных>> 
проектов. Такое положение дел во многом обусловлено факторами 
неопределенности прагматического характера, которые снижают отдачу от 
вложений в~перевод всех работ на мо\-дель\-но-ори\-ен\-ти\-ро\-ван\-ные 
<<рельсы>>~\cite{2-kov, 3-kov}:
   \begin{itemize}
\item слабая совместимость языков и~инструментов моделирования от разных 
поставщиков;
\item недостаточно внятное отражение принципов и~приемов сис\-тем\-но\-го 
мышления в~языках и~инструментах;
\item нехватка мощных типовых механизмов коллективной модульной 
разработки больших моделей;
\item неполнота автоматизации жизненного цикла артефактов MBSE.
\end{itemize}

   Эти факторы неопределенности известны более~10~лет, 
и~в~направлении их устранения ведутся активные исследования. Однако 
результаты в~основном ограничиваются совершенствованием частных языков 
и~технологий: лишь немногие вносят вклад в~методологический базис MBSE 
в~целом. Естественный, хотя и~<<трудный>> подход к~получению результатов 
общего характера, уни\-фи\-ци\-ру\-ющих разнородные технологии, состоит в~том, 
чтобы как можно строже формализовать процедуры применения моделей 
в~инженерии, составляющие прагматику  
мо\-дель\-но-ори\-ен\-ти\-ро\-ван\-но\-го подхода. Формализация, особенно 
с~привлечением математического аппарата, позволит и~совершенствовать 
процедуры MBSE, и~передавать их на исполнение компьютеру без пробелов 
и~искажений.
   
   В настоящей работе, в~развитие результатов работы~\cite{4-kov}, применяется 
аппарат, основанный на математическом представлении сборочных чертежей 
(<<мегамоделей>> сис\-тем) ориентированными графами (диаграммами). Узлы 
такого графа помечаются обозначениями моделей частей, а ребра помечаются 
обозначениями действий (activities), посредством которых части собираются 
в~сис\-те\-му. Представление структуры сис\-тем графами регламентируется 
стандартами, такими как IEC 81346~\cite{5-kov}. Естественным источником 
математических методов конструирования и~анализа мегамоделей служит 
теория категорий (см., например,~\cite{6-kov}): модели\linebreak
 рассматриваются как объекты 
подходящих ка\-те\-горий, а~действия формально описываются мор\-физмами. 
В~работе строятся и~исследуются тео\-ретико-ка\-те\-гор\-ные конструкции, 
описывающие процедуры MBSE на абстрактном концептуальном уровне. 
Например, сборке сис\-те\-мы из мегамодели отвечает построение копредела 
диаграммы~--- универсальной конструкции~\cite{6-kov}. Актуальны в~сис\-тем\-ной 
инженерии и~обратные задачи сборки \mbox{сис\-тем}, решение которых со\-сто\-ит 
в~реконструкции диаграмм по реб\-рам их копределов. Насколько известно 
автору, реконструкция такого рода ранее не рассматривалась в~литературе.
   
   Работа построена следующим образом. Раздел~2 посвящен прагматическим 
проблемам мегамоделирования и~сборке сис\-тем. В~разд.~3 и~4 вводятся 
конструкции теории категорий, позволяющие решать прямые и~обратные 
задачи мегамоделирования соответственно. В~заключении приводятся выводы 
и~намечаются направления дальнейших исследований.

\section{Прямые и~обратные задачи сборки систем}

   Для анализа структуры сис\-тем и~организации сборки в~MBSE необходимо 
знать не столько внут\-рен\-нюю структуру моделей, сколько ассортимент их 
возможностей соединяться с~другими моделями в~целях формирования моделей 
составных объектов. Иными словами, модели рассматриваются как <<черные 
ящики>> с~известным поведением по отношению к~другим моделям. 
Прагматический контекст сборки сис\-тем в~условиях применения некоторой 
частной технологии MBSE можно представлять в~виде каталога, состоящего из 
моделей и~описаний действий по их соединению.
   
   Напомним, что структуры сис\-тем и~сборочные чертежи представляют собой 
частные случаи мегамоделей (megamodel)~--- моделей, состоящих из моделей 
и~связей между ними~\cite{8-kov}. Мегамодель, в~которой связи описывают соединение 
моделей,\linebreak
 образующих некоторую сис\-те\-му, называется конфигурацией этой 
сис\-те\-мы~\cite{6-kov}. Прямая задача мегамоделирования сборки сис\-тем формулируется\linebreak 
следующим образом~\cite{4-kov}. По мегамодели, пред\-став\-ля\-ющей конфигурацию 
некоторой сис\-те\-мы, требуется сконструировать модель сис\-те\-мы как це-\linebreak\vspace*{-12pt}

 { \begin{center}  %fig1
 \vspace*{-1pt}
  \mbox{%
 \epsfxsize=61.131mm 
 \epsfbox{kov-1.eps}
 }


\vspace*{6pt}


\noindent
{{\figurename~1}\ \ \small{Схема склеивания}}
\end{center}
}


\vspace*{9pt}

\addtocounter{figure}{1}



\noindent
лого. 
Например, рас\-смот\-рим сборку сис\-те\-мы из двух объектов~$P$ и~$S$  
с~по\-мощью объ\-ек\-та-клея~$G$. Действие клея описывается конфигурацией 
следующего вида: объекты~$G$ и~$P$ по\-рож\-да\-ют в~результате соединения 
известный промежуточный комплексный объект~$P_G$, содержащий их, 
а~объекты $G$ и~$S$ порождают объект~$S_G$. Система~$R$, полученная 
путем склеивания~$P$ с~$S$ при помощи~$G$, отбирается среди объектов, 
содержащих~$P_G$ и~$S_G$, по сле\-ду\-юще\-му структурному критерию: 
объект~$R$ должен содержаться в~любом объекте~$T$, содержащем~$P_G$ 
и~$S_G$. Схематически этот критерий изображен на рис.~1, где в~кружки 
заключены элементы структурного представления сис\-те\-мы по стандарту 
IEC~81346.
    
\begin{figure*} %fig2
\vspace*{1pt}
 \begin{center}
 \mbox{%
 \epsfxsize=115.833mm 
 \epsfbox{kov-2.eps}
 }
 \end{center}
\vspace*{-9pt}
\Caption{Пример коллективной модификации сложной модели}
\end{figure*}

   
   Хотя прямой расчет сис\-те\-мы по конфигурации имеет большое значение, 
в~прагматике MBSE он играет вспомогательную роль. Согласно стандарту 
IEC~81346 и~практикам сис\-тем\-ной инженерии, сис\-те\-ма обычно проектируется 
сверху вниз~--- от корня структурной иерархии к~составляющим~\cite{9-kov}. Это 
означает, что технолог в~основном решает не прямую, а~обратную задачу: 
модель сис\-те\-мы, которую нужно собрать, известна, а~нуж\-но по\-строить 
(восстановить) конфигурацию, из которой такая сис\-те\-ма может быть получена 
путем сборки. При этом ситуация, когда известна только целевая сис\-те\-ма, 
встречается редко. Обычно конструктор заранее знает некоторые из 
составляющих ее объектов и~способов их вклю\-че\-ния. В~свою очередь, 
технолог априори знает ассортимент и~возможности име\-ющих\-ся в~его 
распоряжении средств производства, поэтому выбирает конфигурацию целевой 
сис\-те\-мы не произвольно, а~из некоторого заранее известного класса. Учет этих 
знаний приводит к~следующей формулировке обратной задачи 
мегамоделирования. По заданному фрагменту мегамодели сборки, 
содержащему целевую си\-сте\-му, требуется восстановить мегамодель целиком, 
причем образующая ее конфигурация должна находиться в~заданном классе, 
пред\-став\-ля\-ющем все технологически реализуемые конфигурации.
   
   Например, обратная задача поиска клея ставится следующим образом. 
В~соответствии со струк\-турой целевой сис\-те\-мы~$R$, описанной согласно 
стандар\-ту IEC~81346, дан фрагмент мегамодели, состоящий из трех 
объектов~$P$, $S$ и~$R$ и~двух стрелок $P\hm\to R$ и~$S\hm\to R$. 
Конфигурацией склеивания служит любая комбинация из пяти объектов 
и~четырех стрелок, форма которой совпадает с~частью рис.~1, образованной 
всеми сплошными стрелками. Требуется найти среди всех конфигураций 
указанного вида такую, для которой решение прямой задачи 
мегамоделирования порождает заданные стрелки $P\hm\to R$ и~$S\hm\to R$ 
в~качестве композитных включений двух <<крайних>> объектов 
конфигурации в~целевую сис\-те\-му (в~част\-ности, крайними объектами искомой 
конфигурации должны быть~$P$ и~$S$, а~целевой сис\-те\-мой~--- $R$).
   
   Обратные задачи мегамоделирования возникают также при организации 
управления моделями (model management)~\cite{7-kov}. Так, на вход процедур 
коллективной модульной разработки поступает цельная модель, которую также 
нужно разделить на части. Однако критерием разделения выступает не 
доступность частей в~виде деталей, а удобство модификации и~локальной 
верификации каждой части. После модификации частей производится сборка 
новой версии цельной модели, т.\,е.\ решается прямая задача 
мегамоделирования. При этом требуется обеспечить структурную корректность 
всей процедуры: должна иметься возможность представить процедуры 
модификации частей как составляющие гипотетической общей процедуры 
модификации модели как целого. 
Такая общая процедура может быть 
практически невыполнима ввиду чрезмерной ре\-сур\-со\-ем\-кости, однако она 
должна существовать в~принципе. 

В~простом случае, когда модификация 
каж\-дой части может быть описана действием из каталога моделей, условие 
структурной кор\-рект\-ности может выглядеть, как представлено на рис.~2. Здесь 
изображена процедура коллективной модификации некоторой большой 
модели~$M$ по частям $M_1, \ldots ,M_5$, превращающая ее 
в~модель~$M^\prime$.


   В MBSE предлагается ряд подходов к~решению обратных задач 
моделирования, известных под общим названием исследования пространства 
проектирования (Design Space Exploration, DSE)~\cite{10-kov}. 
В~мегамоделировании пространством проектирования служит класс всех 
мегамоделей, из которого допускается выбирать искомые конфигурации. 
Методы DSE, по существу, предлагают перебирать пространство 
проектирования, для каждой выбранной из него конфигурации решать прямую 
задачу мегамоделирования и~проверять, включает ли получившаяся мегамодель 
сборки исходно заданный фрагмент, содержащий целевую сис\-те\-му. Конечно, 
полный перебор может быть неосуществим в~принципе, поскольку 
пространство проектирования может быть теоретически бесконечным 
(содержать мегамодели, включающие сколь угодно много составляющих). 
Поэтому предлагаются методы и~автоматизированные инструменты DSE, 
основанные на представлении постановки обратной задачи набором 
ограничений, записанных на подходящем формальном языке 
метамоделирования~\cite{11-kov}\linebreak (в~данном случае требуется язык\ \ 
\textit{метамегамоделирования}). В~некоторых случаях возможна 
непосредственная генерация элементов пространства проектирования по такой 
метамегамодели, в~других\linebreak случаях~--- разделение пространства проектирования 
на обозримые части путем уточнения ограничений и~т.\,п.
   
   Если пространство проектирования достаточно велико, то решение 
некоторой обратной задачи мегамоделирования может дать большое число 
конфигураций. Выбор итоговой среди них осуществляется путем ранжирования 
по значениям различных количественных показателей, отражающих расход 
труда, материалов, финансовых средств, времени и~других ресурсов на сборку 
сис\-те\-мы. Такие показатели являются предметом стандартизации, в~частности 
в~рамках ГОСТов серии~14, регламентирующих технологическую подготовку 
производства. В~фокусе внимания стандартов группы~2 серии~14 находится 
технологичность конструкции изделия~--- <<совокупность свойств 
конструкции изделия, определяющих ее приспособленность к~достижению 
оптимальных затрат при производстве, техническом обслуживании и~ремонте 
для заданных показателей качества, объема выпуска и~условий выполнения 
работ>>~\cite{12-kov}. Многие показатели технологичности конструкции 
изделия можно вычислять, основываясь на формальном пред\-став\-ле\-нии 
мегамоделей графами: подсчитывать число узлов, длину критического пути 
и~т.\,д. Например, коэффициентом сборности называется отношение числа 
узлов, имеющих входящие ребра, к~общему числу узлов графа мегамодели 
сборки. Коэффициент сборности мегамодели склеивания, изображенной на 
рис.~1, равен~0,5.
   
   Существуют и~менее формальные методы восстановления конфигурации 
сис\-тем, возникшие задолго до MBSE. В~основе таких методов лежат 
разнообразные эвристики, направленные на сужение пространства 
проектирования до области применения того или иного технологического 
приема или образца. Широкий набор таких эвристик накапливается начиная 
с~1950-х~гг.\ в~рамках дисциплины, известной как теория решения 
изобретательских задач (ТРИЗ)~\cite{13-kov}.

\section{Теория категорий в~мегамоделировании}

   Как указывалось во введении, естественным\linebreak
    источником математических 
методов конструирования и~анализа мегамоделей служит теория кате\-горий. 
Будем пользоваться тео\-ре\-ти\-ко-ка\-те\-гор\-ны\-ми\linebreak конструкциями 
и~обозначениями, введенными в~работе~\cite{4-kov}. Зафиксируем 
произвольную категорию~$C$, представляющую некоторый каталог моделей 
и~описаний действий по их соединению. Примерами служат категория 
твердотельных геометрических моделей \textbf{MBS}, содержащаяся в~категории 
множеств \textbf{Set}, и~категория дис\-крет\-но-со\-бы\-тий\-ных имитационных 
моделей \textbf{Pomset}, которая является конкретной категорией над Set. Задачи 
мегамоделирования рассматриваются в~категории $C$-диа\-грамм 
$\mathbf{D}C$. Морфизмы диаграмм описывают структурные преобразования 
мегамоделей, выполняемые при помощи инструментов MBSE, 
поддерживающих сборку сис\-тем (например, морфизм диаграмм изображен на 
схеме коллективной модификации моделей на рис.~2). Таким образом, теория 
категорий предоставляет тот самый язык метамегамоделирования, который 
необходим для проведения DSE в~задачах восстановления конфигурации 
сис\-тем.
   
   Для начала напомним, что прямая задача построения сис\-те\-мы~$R$ по 
заданной конфигурации~$\Delta$ решается путем конструирования  
копредела~---\linebreak
 $\mathbf{D}C$-мор\-физ\-ма ${\mathrm{colim}}\,\Delta : \Delta \hm\to 
\ulcorner R\urcorner$, универсального среди всех коконусов 
с~областью~$\Delta$. (Через $\ulcorner R\urcorner$ обозначается одноточечная 
$C$-диа\-грам\-ма, состоящая из единственной вершины~$R$ и~не имеющая 
ребер.) Конструирование копределов описано в~работе~\cite{4-kov} функтором 
$\mathrm{colim}$, который переводит в~$C$ подходящую подкатегорию 
в~$\mathbf{D}C$ согласно сле\-ду\-ющей диаграмме:

\vspace*{6pt}
   \begin{center}
   \vspace*{1pt}
 \mbox{%
 \epsfxsize=54.085mm 
 \epsfbox{kov-3.eps}
 }
 \end{center}

\vspace*{6pt}


   Для прагматики MBSE функтор $\mathrm{colim}$ полезен, в~частности, тем, что он 
позволяет обеспечить структурную корректность описанной в~разд.~2\linebreak 
процедуры коллективной модификации большой модели. Действительно, если 
можно <<упаковать>> модификации частей модели в~некоторый 
$\mathbf{D}C$-мор\-физм~$\theta$, то порожденная ими процедура 
модификации модели как целого описывается морфизмом $\mathrm{colim}\,(\theta)$. Легко 
увидеть это из вышеприведенной диаграммы: если повернуть ее на~90$^\circ$ 
против часовой стрелки и~изобразить копределы диаграммами 
соответствующего вида, то получится рис.~2. Предлагаются  
тео\-ре\-ти\-ко-ка\-те\-гор\-ные конструкции и~для более сложных процедур 
управления моделями~\cite{8-kov}.
   
   Обратим внимание на одно существенное отличие диаграммы, 
определяющей действие функтора $\mathrm{colim}$, от диаграммы коллективной 
модификации модели из разд.~2: первая является  
$\mathbf{D}C$-диа\-грам\-мой, а~вторая~--- $C$-диа\-грам\-мой. Чтобы 
проиллюстрировать приложение функтора $\mathrm{colim}$, можно мысленно 
преобразовать $\mathbf{D}C$-диа\-грам\-му в~$C$-диа\-грам\-му путем 
<<отрисовки>>. В~действительности это преобразование является строго 
формальным: отрисовка произвольной диаграммы~$\Gamma : Z\hm\to 
\mathbf{D}C$ заключается в~замене каждой вершины~$A$ графа схемы~$Z$ 
графом $C$-диа\-грам\-мы~$\Gamma(A)$, а~каждого ребра~$f : A\hm\to B$~---
совокупностью ребер, по одному для каждой вершины~$I$ графа диаграммы 
$\Gamma (A)$, направленному из~$I$ в~вершину $fd(I)$ графа 
диаграммы~$\Gamma(B)$ и~помеченному $C$-мор\-физ\-мом~$\varepsilon_I$, 
где $\langle \varepsilon, fd\rangle \hm= \Gamma(f)$. Ясно, что вместе 
   с~$\mathbf{D}C$-диа\-грам\-ма\-ми можно отрисовывать и~их морфизмы, 
причем так, что возникает функтор отрисовки~$\mathbf{K}_C: 
\mathbf{DD}C\hm\to  \mathbf{D}C$. В~свою очередь, процедура 
конструирования категории диаграмм~$\mathbf{D}C$ может быть задана 
функтором: любой функтор $\mathrm{fun}: C\hm\to D$ порождает функтор
  \begin{multline*}
\mathrm{fun}\, \circ - : \mathbf{D}C \to \mathbf{D}D : \Delta \mapsto 
\mathrm{fun}\, \circ \Delta, 
\langle\varepsilon, fd\rangle\mapsto{}\\
{}\mapsto \langle \mathrm{fun}\,(\varepsilon), fd\rangle\,,
\end{multline*}
так что имеется функтор:
$$
\mathbf{D}: \mathbf{CAT} \to \mathbf{CAT}: C \mapsto \mathbf{D}C, \mathrm{fun} 
\mapsto (\mathrm{fun}\, \circ -)\,.
$$
   
   Можно распространить действие этого функтора на естественные 
преобразования, а~именно: произвольному естественному преобразованию 
$\xi~:~\mathrm{fun} \hm\to  \mathrm{fun}^\prime$, где 
$\mathrm{fun}, \mathrm{fun}^\prime : C \hm\to D$~--- 
произвольные функторы, соответствует естественное 
преобразование~$\mathbf{D}\xi~:~\mathbf{D}\,\mathrm{fun} \hm \to \mathbf{D}\,\mathrm{fun}^\prime$ 
с~компонентами вида
   $$
(\mathbf{D}\xi)_{\Delta} = \langle \xi_{\Delta(-)}, 1_X\rangle~: \mathrm{fun}\,\circ \Delta \to 
\mathrm{fun}^\prime \circ \Delta\,,
$$
где $\Delta~:~X \hm\to C$~--- произвольная диаграмма. Читатель, искушенный 
в~теории категорий, заметит, что таким путем~$\mathbf{D}$ определен как  
2-функ\-тор. Более того, как установлено в~\cite{14-kov}, тройка $\cal{D}\hm= 
\langle \mathbf{D}, \ulcorner - \urcorner, \mathbf{K}\rangle$ образует 2-мо\-на\-ду 
в~\textbf{CAT}, причем она по ряду свойств аналогична монаде степени $\langle 2^-, \{-
\}, \cup\rangle$ в~категории \textbf{Set}. Стрелка $\mathrm{colim}~:~\mathbf{D}C \hm\to C$ для 
любой кополной категории~$C$ является\linebreak
 $\mathcal{D}$-ал\-геб\-рой~--- 
объектом категории Эй\-лен\-бер\-га--Му\-ра, ассоциированной 
с~монадой~$\mathcal{D}$. Если в~категории~$C$ есть суммы, то имеется 
функтор $\mathrm{sum}~:~\mathbf{D}C \hm\to C$, сопоставляющий каждой  
\mbox{$C$-диа}\-грам\-ме сумму всех ее вершин, и~он также является  
$\mathcal{D}$-ал\-геб\-рой. И~кроме того, произвольная категория~$C$ 
порождает свободную $\mathcal{D}$-ал\-геб\-ру~$\mathbf{K}_C~:~\mathbf{DD}C \hm\to 
\mathbf{D}C$~\cite[предложение~20.7]{15-kov}. Например, категория \textbf{Pomset} 
выступает носителем попарно неизоморфных $\mathcal{D}$-ал\-гебр всех трех 
вышеуказанных видов. В~целом, $\mathcal{D}$-ал\-геб\-ры задают шаблоны 
сис\-тем\-ных процедур общего вида, состоящих в~свертке мегамоделей в~модели 
со\-став\-ных объектов с~соблюдением условий естественности относительно 
модификаций. Монаду~$\mathcal{D}$ правомерно рас\-смат\-ри\-вать как 
метамодель мегамоделирования, поскольку она включает в~себя все средства 
языка теории категорий, применяемые для описания процедур, связанных со 
сборкой сис\-тем.

\section{Теоретико-категорные методы восстановления 
конфигурации}

   Общее категорное решение обратных задач мегамоделирования неизвестно. 
Сформулируем на языке теории категорий постановку задачи восстановления 
конфигурации и~исследуем ряд частных случаев. Для этого понадобится 
понятие подкоконуса. Коконус~$\delta$ называется подкоконусом 
в~ко\-ко\-ну\-се~$\beta$, если существует $\mathbf{D}C$-мор\-физм вида $\langle\mu, 
iy\rangle~: \mathrm{dom}\, \delta \hm\to \mathrm{dom}\, \beta$ такой, что 
естественное преобразование~$\mu$ состоит из тождественных  
$C$-мор\-физ\-мов, функтор~$iy$ инъективен и~$\beta \circ \langle \mu, 
iy\rangle\hm= \delta$. 

Обратная задача мегамоделирования ма\-те\-ма\-тически 
ставится следующим образом. Даны коконус~$\delta$, пред\-став\-ля\-ющий 
известный фрагмент\linebreak
 мегамодели сборки, и~класс диаграмм~$\mathrm{Cd}$, 
представляющий пространство проектирования. Требуется найти 
в~классе~$\mathrm{Cd}$ диаграмму, обладающую копределом, подкоконусом в~котором 
является~$\delta$.
   
   Как упоминалось в~разд.~2, большую пользу для решения такой задачи 
приносит сужение класса~$\mathrm{Cd}$, представляющего пространство 
проектирования, и~упрощение структуры составляющих его диаграмм. В~этом 
направлении получен следующий результат.
   
   \smallskip
   
   \noindent
   \textbf{Теорема~1.} \textit{Пусть $\Delta~: X\hm\to C$~--- произвольная 
диаграмма, $Y$~--- полная рефлективная подкатегория в~$X$, $\mathrm{isc}~: 
Y\hookrightarrow X$~--- вложение. Для любого коконуса~$\theta~: \Delta \circ 
\mathrm{isc} \hm\to \ulcorner S\urcorner$ существует единственный коконус 
$\theta^\prime~: \Delta\hm\to \ulcorner S\urcorner$ такой, что 
$\theta\hm=\theta^\prime \circ \langle 1_{\Delta \circ \mathrm{isc}}, \mathrm{isc}\rangle$. При этом 
коконус~$\theta$ является копределом диаграммы $\Delta \circ~\mathrm{isc}$ тогда 
и~только тогда, когда коконус~$\theta^\prime$ является копределом 
диаграммы~$\Delta$.}
   
   \smallskip
   
   \noindent
   Д\,о\,к\,а\,з\,а\,т\,е\,л\,ь\,с\,т\,в\,о\,.\ \ Напомним, что подкатегория~$Y$ 
называется рефлективной в~$X$, если вложение $\mathrm{isc}~: Y\hookrightarrow X$ 
обладает левым сопряженным $\mathrm{isc}^*~: X\hm\to Y$ 
рефлектором~\cite[разд.~4.3]{16-kov}. Пусть $\eta : 1_X \hm\to \mathrm{isc} \circ  
\mathrm{isc}^*$~--- единица рефлексии, она индуцирует $\mathbf{D}C$-мор\-физм 
$\langle\Delta (\eta), \mathrm{isc}^*\rangle : \Delta\hm\to \Delta \circ~\mathrm{isc}$. Согласно 
определению коконуса, для любого \mbox{$X$-объ}\-ек\-та $I$~ребро $f : \Delta 
(I)\hm\to S$ искомого коконуса~$\theta^\prime$ должно удовлетворять условию 
$f = h \circ \Delta (\eta_I)$, где $h : \Delta (\mathrm{isc}\,(\mathrm{isc}^*(I))) \hm\to S$~--- ребро 
коконуса~$\theta$, откуда $\theta^\prime\hm = \theta \circ \langle \Delta (\eta), 
\mathrm{isc}^*\rangle$.
   
   Предположим, что $\theta$~--- копредел, выберем произвольно 
коконус~$\rho : \Delta\hm \to \ulcorner T\urcorner$. Пусть $w : S \hm\to T$~--- 
единственный $C$-мор\-физм, удовлетворяющий условию $\rho \circ \langle 1_{\Delta 
\circ  \mathrm{isc}}, 
\mathrm{isc}\rangle\hm = \ulcorner w\urcorner \circ \theta$. Тогда $\ulcorner 
w\urcorner \circ \theta^\prime \hm= \rho\circ \langle 1_{\Delta\circ \mathrm{isc}}, 
\mathrm{isc}\,\rangle 
\circ \langle \Delta (\eta), \mathrm{isc}^*\rangle \hm= \rho$ и,~кро-\linebreak\vspace*{-12pt}

\pagebreak

\noindent
ме того, 
если некоторый 
$C$-мор\-физм $\tilde{w} : S \hm\to T$ удовле\-тво\-ря\-ет условию $\rho\hm = 
\ulcorner \tilde{w}\urcorner \circ \theta^{\prime}$, то
 $\rho\circ \langle 1_{\Delta\circ 
\mathrm{isc}}, \mathrm{isc}\,\rangle\hm = \ulcorner\tilde{w}\urcorner 
\circ \theta^\prime\circ\langle 
1_{\Delta\circ~\mathrm{isc}}, \mathrm{isc}\,\rangle\hm = 
\ulcorner \tilde{w}\urcorner \circ \theta$, 
откуда 
$\tilde{w}\hm=w$. Следовательно, $\theta^\prime$~--- копредел 
диаграммы~$\Delta$.
   
   Обратно, предположим, что $\theta^\prime$~--- копредел, и~выберем 
произвольно коконус $\gamma : \Delta \circ \mathrm{isc} \hm\to \ulcorner U\urcorner$. 
Пусть $u : S \hm\to U$~--- единственный $C$-мор\-физм, удовлетворяющий 
условию 
$$
\gamma^\prime = \ulcorner u\urcorner \circ \theta^\prime\,,
$$
 где 
$\gamma^\prime\hm = \gamma \circ \langle \Delta (\eta), \mathrm{isc}^*\rangle$. Тогда 
$$
\ulcorner u\urcorner \circ \theta = \gamma\circ \langle \Delta (\eta), 
\mathrm{isc}^*\rangle 
\circ \langle 1_{\Delta\circ~\mathrm{isc}}, 
\mathrm{isc}\,\rangle\hm = \gamma$$
 и,~кроме того, если 
некоторый $C$-мор\-физм $\tilde{u} : S \hm\to U$ удовлетворяет условию 
$\gamma\hm = \ulcorner \tilde{u}\urcorner \circ\theta$, то 
$$
\gamma^\prime = \ulcorner 
\tilde{u}\urcorner \circ \theta \circ \langle\Delta (\eta), 
\mathrm{isc}^*\rangle = \ulcorner 
\tilde{u}\urcorner\circ\theta^\prime\,,
$$
 откуда $\tilde{u}\hm= u$. Следовательно, 
$\theta$~--- копредел диаграммы $\Delta \circ~\mathrm{isc}$.\hfill$\square$
   
   \smallskip
   
   Исследуем при помощи теоремы~1 задачу поиска клея из разд.~2. 
Обозначим через~$M$ схему конфигурации склеивания~--- категорию, 
порожденную графом следующего вида:
   \begin{center} %fig4
   \vspace*{1pt}
 \mbox{%
 \epsfxsize=37.099mm 
 \epsfbox{kov-4.eps}
 }
   \end{center}


   Все диаграммы со схемой~$M$ образуют пространство проектирования 
склеивания~--- класс~$\mathrm{Cd}$ из общей постановки задачи восстановления 
конфигурации. Коконус~$\delta$, фигурирующий в~общей постановке, 
реконструируется по структуре сис\-те\-мы, представленной согласно стандарту 
IEC~81346 на рис.~1: основанием коконуса служит диаграмма $\Delta : a 
\mapsto P$, $a^\prime\mapsto S$ с~двухточечной дискретной схемой $\{a, 
a^\prime\} \hm\subseteq M$, а~вершиной~--- собираемая сис\-те\-ма~$R$.
   
   Пусть $V$~--- полная подкатегория в~$M$, состоящая из трех 
<<центральных>> $M$-объ\-ек\-тов: $V$ порождается графом $b\leftarrow 
c\rightarrow b^\prime$. Ясно, что подкатегория~$V$ рефлективна в~$M$, 
причем ребра $a\hm\to b$ и~$a^\prime\hm\to b^\prime$ графа категории~$M$ 
являются компонентами единицы рефлексии, соответствующими объектам~$a$ 
и~$a^\prime$ соответственно. По теореме~1 копредел диаграммы со 
схемой~$M$ взаимно однозначно задается копределом диаграммы со 
схемой~$V$, который хорошо известен в~теории категорий под названием 
кодекартова квадрата (pushout). С~учетом этого получается общее решение 
задачи поиска клея.
   
   \smallskip
   
   \noindent
   \textbf{Предложение~1.}\ \textit{Для любых объектов $P$, $S$, $R$ и~пары 
морфизмов $p : P \hm\to R \hm\leftarrow S : s$ существует диаграмма $\Delta : M 
\hm\to C$, имеющая копредел вида $\langle \kappa, !_M\rangle : \Delta\hm\to 
\ulcorner R\urcorner$ такой, что $\kappa_a\hm= p$ и~$\kappa_{a^\prime}\hm = s$. 
Указанная диаграмма единственна тогда и~только тогда, когда $p \hm= s \hm= 
1_R$ является единственным $C$-эпи\-мор\-физ\-мом с~кообластью}~$R$.
   
   \smallskip
   
   \noindent
   Д\,о\,к\,а\,з\,а\,т\,е\,л\,ь\,с\,т\,в\,о\,.\ \
   Произвольный $C$-мор\-физм~$e$ с~кообластью~$R$ является 
эпиморфизмом тогда и~только тогда, когда равенство $1_R \circ e \hm= 1_R \circ 
e$ задает кодекартов квадрат в~$C$~\cite[упражнение~3.4.4]{16-kov}. Поэтому 
если $e$~--- эпиморфизм, то ввиду теоремы~1 в~качестве~$\Delta$ можно 
выбрать следующую диаграмму:
\begin{center} %fig5
\vspace*{1pt}
 \mbox{%
 \epsfxsize=36.995mm 
 \epsfbox{kov-5.eps}
 }
\end{center}

В частности, при $e\hm = 1_R$ получается решение задачи поиска клея в~стиле 
ТРИЗ: если уже имеется сис\-те\-ма~$R$, включающая в~себя объекты~$P$ и~$S$, 
то пусть она и~послужит для склеивания этих объектов!

   Если $p\not=  1_R$, то существует еще одно решение, которое выглядит 
следующим образом:
\begin{center} %fig6
\vspace*{1pt}
 \mbox{%
 \epsfxsize=36.995mm 
 \epsfbox{kov-6.eps}
 }
\end{center}

   Если же $p = s \hm= 1_R$, то решение задачи включает нахождение 
разложений вида $1_R \hm= o\circ q$ (чтобы\linebreak пометить морфизмом~$q$ 
внешнюю стрелку в~искомой
 диаграмме~$\Delta$). Морфизм~$o$ в~любом 
таком разложе\-нии является эпиморфизмом~\cite[предложение~7.42]{15-kov} 
и~имеет кообласть~$R$, поэтому если единственным эпиморфизмом 
с~кообластью~$R$ является~$1_R$, то искомая диаграмма~$\Delta$ 
единственна: она может быть составлена только из 
морфиз-\linebreak мов~$1_R$.\hfill$\square$
   
   Например, в~категории множеств \textbf{Set} критерий единственности решения 
задачи поиска клея равносилен тому, что множество~$R$ пусто.
   
   Простота решения общей задачи поиска клея обусловлена широким 
произволом в~выборе объекта, выполняющего функцию склеивания. Однако на 
практике на выбор клея накладываются те или иные технологические 
ограничения. Например, часто требуется, чтобы клей не вносил существенных 
изменений в~объекты~$P$ и~$S$, т.\,е.\ чтобы крайние стрелки конфигурации 
склеивания были изоморфизмами. Так устроено склеивание в~твер\-до\-тель\-ном 
моделировании: нанесение тонкого слоя клеящего вещества на участок 
по\-верх\-ности тела не меняет его геометрии. В~таком случае можно упрос\-тить 
конфигурацию, отбросив крайние стрелки: составить класс~$\mathrm{Cd}$, 
фигурирующий в~постановке обратной задачи, из всех диаграмм вида 
$P\hm\leftarrow G\hm\to S$ со схемой~$V$, а~в~качестве~$\delta$ выбрать 
коконус $[\ulcorner p\urcorner, \ulcorner s\urcorner] : \Omega\hm\to  \ulcorner 
R\urcorner$, где $\Omega : b\mapsto P$, $b^\prime\mapsto S$~--- диаграмма 
с~двухточечной дискретной схемой $\{b, b^\prime\} \hm\subseteq V$. Решить 
такую задачу вос\-ста\-нов\-ле\-ния кодекартова квадрата гораздо труднее, чем 
предыдущую. Ее решение будет приведено в~категории \textbf{Set}.
   
   \smallskip
   
   \noindent
   \textbf{Предложение~2.}\ \textit{Пусть даны произвольные множества~$P$, 
$S$, $R$ и~пара отображений $p : P \hm\to R \hm\leftarrow S : s$. 
Множество~$G$ и~пара отображений $f : P \hm\leftarrow G \hm\to S : g$ такие, 
что соотношение $p\circ f = s \circ g$\linebreak выполняется и~задает кодекартов квадрат 
в~категории} \textbf{Set}, \textit{существуют тогда и~только тогда, когда для любого $r\hm\in 
R$ выполняется следующее условие}:
  \begin{multline*}
   \left( \left\vert p^{-1}(r)\right\vert  = 0 \Rightarrow \left\vert s^{-1}(r)\right\vert  
= 1\right) \wedge\\
\wedge \left(\left\vert s^{-1}(r)\right\vert  =
 0 \Rightarrow \left\vert p^{-
1}(r)\right\vert  = 1\right)\,.
\end{multline*}
   
   \textit{При выполнении этого условия, искомые множество и~пара 
отображений единственны тогда и~только тогда, когда} $\left\vert p(P) \cap 
s(S)\right\vert \hm= 0$.
   
   \smallskip
   
   \noindent
   Д\,о\,к\,а\,з\,а\,т\,е\,л\,ь\,с\,т\,в\,о\,.\ \ Напомним, что кодекартов квадрат 
в~\textbf{Set} над некоторой парой отображений $f : P \hm\leftarrow G\hm\to S : g$ 
строится следующим образом~\cite[разд.~3.3]{16-kov}. В~качестве 
вершины~$R$ берется множество, полученное из раздельного объединения $P 
\coprod  S$ путем факторизации по отношению эквивалентности, 
порожденному множеством пар $\{(f(x), g(x)) \vert x \hm\in G\}$. Ребро $p : P 
\hm\to R$ сопоставляет каждому элементу множества~$P$ класс 
эквивалентности элемента по этому отношению, а ребро $s : S \hm\to R$ 
действует аналогично на~$S$. Поэтому если $\vert p^{-1}(r)\vert  \hm= 0$ для 
некоторого $r\hm\in R$, то элемент~$r$ <<попал>> в~$R$ из множества~$S 
\backslash g(G)$: существует единственный $t \hm\in S$ такой, что $s(t) \hm= r$. 
Аналогично рассматривается случай $\vert s^{-1}(r)\vert\hm = 0$.
   
   Обратно, предположим, что условие, приведен\-ное в~формулировке 
доказываемого утверждения,\linebreak
 выполняется для некоторой пары отображений $p 
: P \hm\to R \hm\leftarrow S : s$; в~частности, $p(P) \cup s(S) \hm= R$. Построим 
\textit{декартов} квадрат (предел) указанной пары~\cite[разд.~3.4]{16-kov}: 
положим 
\begin{multline*}
G = \{(x, y) \vert p(x) = s(y)\} \subseteq  P \times S,\\
 f : G 
\to P : (x, y) \mapsto x, g : G \to S : (x, y) \mapsto y\,.
\end{multline*}
 Пусть пара 
отображений $p^\prime : P \hm\to R^\prime\hm\leftarrow S : s^\prime$ задает 
ребра кодекартова квадрата, построенного стандартным способом над парой 
отображений $f : P \hm\leftarrow G \hm\to S : g$. Построим отоб\-ра\-же\-ние $i : R 
\hm\to R^\prime$ следующим образом: если $r\hm\in R \backslash p(P)$, то 
полагаем $i(r) \hm= s^\prime(t)$, где $t$~--- единственный элемент 
множества~$S$ такой, что $s(t) \hm= r$ (такой элемент существует по 
предположению), а~в~противном случае полагаем $i(r) \hm= p^\prime(v)$ для 
произвольного $v\hm\in p^{-1}(r)$ (ясно, что значение $i(r)$ не зависит от 
выбора v). Легко видеть, что $i$~--- биекция, $p\hm= i^{-1}\circ p^\prime$ 
и~$s\hm = i^{-1}\circ  s^\prime$, так что пара $p : P \hm\to R \hm\leftarrow S : s$ 
действительно задает ребра кодекартова квадрата над парой $f : P \hm\leftarrow 
G \hm\to S : g$. Более того, если $\vert p(P) \cap  s(S)\vert\hm = 0$, то~$R$ 
совпадает с~точностью до изоморфизма с~раздельным объединением $P\coprod 
S$, а~$p$ и~$s$~--- с~инъекциями компонентов в~него, поэтому пара $p : P 
\hm\to R \hm\leftarrow S : s$ задает ребра кодекартова квадрата только над парой 
$P\hm\leftarrow \varnothing\hm\to S$. Если же $\vert p(P) \cap  s(S)\vert \hm> 0$, 
т.\,е.\ если множество~$G$ непусто, то искомых пар бесконечно много: 
очевидно, что для любой сюръекции $e : G^\prime\hm\to G$ отображения~$p$ 
и~$s$ являются ребрами кодекартова квадрата над парой $f \circ e : P 
\hm\leftarrow G^\prime\hm\to S : g \circ e$.\hfill$\square$
{\looseness=1

}
   
   
   
   Если отображения~$p$ и~$s$ инъективны, то отоб\-ра\-же\-ния~$f$ и~$g$, 
построенные в~доказательстве предложения~2, также будут инъективными. 
Поэтому\linebreak метод из этого доказательства позволяет в~ряде случаев 
восстанавливать конфигурации склеивания в~категории твердотельных моделей 
\textbf{MBS}. 

Можно приспособить этот метод и~для применения во многих известных 
категориях множеств со структурой. Здесь в~условие существования решения 
добавляется требование совпадения структуры, индуцированной 
отображением~$p$ на подмножестве $p(P) \hm\subseteq R$, с~ограничением 
структуры объекта~$R$ на это подмножество и~аналогичное требование 
для~$s$. 
{\looseness=1

}

Объ\-ект-клей~$G$, как в~\textbf{Set}, можно построить при помощи декартова 
квадрата (можно проверить, что и~в~произвольной категории, если для 
некоторой пары морфизмов с~общей кообластью существует как решение 
задачи восстановления кодекартова квадрата, так и~декартов квадрат, то 
последний является и~кодекартовым, т.\,е.\ решением задачи). Например, так 
можно действовать в~категории дискретно-событийных имитационных моделей 
\textbf{Pomset}. В~ней задача восстановления кодекартова квадрата над парой $p : P 
\hm\to R \hm\leftarrow S : s$ имеет решение тогда и~только тогда, когда пара 
удовлетворяет условию из предложения~2 и,~кроме того, каноническое 
отображение $[p, s] : P \coprod S \hm\to R$ является регулярным  
\textbf{Pomset}-эпи\-мор\-физ\-мом. Клей представляет собой множество~$G$, 
сконструированное в~доказательстве предложения~2, порядок и~разметка на 
котором индуцированы из произведения $P\times  S$. Известны примеры 
практической реализации сце\-на\-рия-<<клея>> такого рода при сборке 
имитационных моделей больших производственных сис\-тем~\cite{17-kov}.

\section{Заключение}

   Аппарат теории категорий обладает большим потенциалом в~области 
повышения полезной отдачи от MBSE, в~том числе путем математически 
строгого решения прямых и~обратных задач мегамоделирования. Можно 
наглядно представить принципы применения аппарата в~форме таб\-ли\-цы 
соответствия между ключевыми понятиями теории категорий и~сис\-тем\-ной 
инженерии. Для физики, топологии, логики и~информатики таблица такого рода 
построена в~известном тео\-ре\-ти\-ко-ка\-те\-гор\-ном <<розеттском 
камне>>~\cite{18-kov}. В~настоящей работе установлены следующие 
соответствия, связанные с~процедурами сборки сис\-тем  
в~мо\-дель\-но-ори\-ен\-ти\-ро\-ван\-ном подходе:

\vspace*{6pt}

   \begin{center}
   \begin{tabular}{ll}
   \hline
Теория категорий&Сборка систем в~MBSE\\
\hline
Объект&Модель\\
\hline
Морфизм&\tabcolsep=0pt\begin{tabular}{l}Действие по сборке\\ моделей систем\end{tabular}\\
\hline
Диаграмма&Мегамодель\\
\hline
Копредел диаграммы&\tabcolsep=0pt\begin{tabular}{l}Сборка системы\\ из конфигурации\end{tabular}\\
\hline
Функтор colim&\tabcolsep=0pt\begin{tabular}{l}Коллективная\\ модификация моделей\end{tabular}\\
\hline
Монада диаграмм&\tabcolsep=0pt\begin{tabular}{l}Метамодель\\ мегамоделирования\end{tabular}\\
\hline
\end{tabular}
\end{center}

\vspace*{6pt}
   
   Конечно, чтобы передать аппарат в~руки сис\-тем\-ных инженеров, 
недостаточно сопоставить понятия: необходимо реализовать категорные 
методы в~программных инструментах~\cite{19-kov}. В~част\-ности, 
целесообразно записать методы решения типовых прагматических задач MBSE 
на входном языке решателей~--- автоматизированных средств доказательства 
теорем. Существует ряд категорных решателей (см., например,~\cite{20-kov}), 
которые можно взять за основу. На базе решателя можно создать про\-грам\-мный 
процессор мегамоделей <<общего назначения>>. Чтобы настроить такой 
процессор на конкретную технологию моделирования, нужно будет описать 
технологию категорией, представляющей каталог моделей и~действий по их 
соединению. Таким путем может быть реализован перспективный 
универсальный язык сис\-тем\-но\-го моделирования~\cite{3-kov}. Здесь 
открывается широкий спектр на\-прав\-ле\-ний для дальнейших исследований.
   
{\small\frenchspacing
 {%\baselineskip=10.8pt
 \addcontentsline{toc}{section}{References}
 \begin{thebibliography}{99}
\bibitem{1-kov}
Modeling and simulation-based systems engineering handbook~/ Eds. D.~Gianni, A.~D'Ambrogio, 
A.~Tolk.~--- London: CRC Press, 2014. 513~p.
\bibitem{2-kov}
\Au{Selic B.} The pragmatics of model-driven development~// IEEE Software, 2003. Vol.~20. 
Iss.~5. P.~19--25.
\bibitem{3-kov}
\Au{Левенчук А.\,И.} Системноинженерное мышление.~--- М.: TechInvestLab, 2015. 305~с.
\bibitem{4-kov}
\Au{Ковалёв С.\,П.} Методы теории категорий в~модельно-ори\-ен\-ти\-ро\-ван\-ной сис\-тем\-ной 
инженерии~// Информатика и~её применения, 2017. Т.~11. Вып.~3. С.~42--50.
\bibitem{5-kov}
IEC 81346-1:2009. Industrial Systems, Installations and Equipment and Industrial Products~--- 
Structuring Principles and Reference Designations~--- Part~1: Basic Rules.~--- Geneva: ISO, 2009. 
168~p.
\bibitem{6-kov} %6
\Au{Ginali S., Goguen~J.} A~categorical approach to general systems~// 
 Applied general systems research: Recent development and trends~/ Ed. 
G.\,J.~Klir.~--- NATO conference ser.~--- Boston, MA, USA: Springer U.S., 1978. Vol.~5.  P.~257--270.

\bibitem{8-kov} %7
\Au{Diskin Z., Kokaly~S., Maibaum~T.} Mapping-aware megamodeling: Design patterns and 
laws~//  Software language engineering: 6th Conference (International) Proceedings / Eds. 
M.~Erwig, R.\,F.~Paige, E.~Van Wyk.~--- Lecture notes in computer science ser.~--- Springer, 
2013. Vol.~8225. P.~322--343.

\bibitem{9-kov} %8
\Au{Косяков А., Свит~У., Сеймур~С., Бимер~С.} Системная инженерия. Принципы 
и~практика~/ Пер. с~англ.~--- М.: ДМК-Пресс, 2014. 636~с. (\Au{Kossiakoff~A., Sweet~W.\,N., 
Seymour~S., Biemer~S.\,M.} Systems engineering principles and practice.~--- 2nd ed.~--- New 
York, NY, USA: John Wiley, 2011. 560~p.)

\bibitem{7-kov} %9
\Au{B$\acute{\mbox{e}}$zivin J., Jouault~F., Rosenthal~P., Valduriez~P.} Modeling in the large 
and modeling in the small~// Model driven architecture: European MDA Workshops on 
Foundations and Applications Proceedings~/ Eds.  U.~\mbox{A{\!\ptb{\ss}}mann}, M.~Aksit, 
A.~Rensink.~--- Lecture notes in computer science ser.~--- Springer, 2005.  Vol.~3599. 
P.~33--46.

\bibitem{10-kov}
\Au{Neema S., Sztipanovits~J., Karsai~G., Butts~K.} Constraint-based design-space exploration 
and model synthesis~// 3rd Conference (International) on Embedded Software Proceedings~/ Eds. 
R.~Alur, I.~Lee.~--- Lecture notes in computer science ser.~--- Springer, 2003. 
Vol.~2855. P.~290--305.
\bibitem{11-kov}
\Au{Vanherpen K., Denil~J., De~Meulenaere~P., Vangheluwe~H.} Design-space exploration in 
MDE: An initial pattern catalogue~// 1st Workshop (International) on Combining Modelling with 
Search- and Example-Based Approaches Proceedings~/ Eds. R.~Paige, M.~Kessentini, P.~Langer, 
M.~Wimmer.~--- CEUR Workshop Proceedings ser.~---  Valencia, Spain, 2014. Vol.~1340. 
P.~42--51.
\bibitem{12-kov}
ГОСТ 14.205-83. Технологичность конструкции изделий. Термины и~определения.~--- М.: 
Стандартинформ, 2009. 22~с.
\bibitem{13-kov}
\Au{Альтшуллеp Г.\,С.} Творчество как точная наука.~--- М.: Советское pадио, 1979. 116~с.
\bibitem{14-kov}
\Au{Guitart R., van den~Bril~L.} D$\acute{\mbox{e}}$compositions et lax-
compl$\acute{\mbox{e}}$tions~// Cah. Topologie G$\acute{\mbox{e}}$om$\acute{\mbox{e}}$trie Diff$\acute{\mbox{e}}$rentielle 
Cat$\acute{\mbox{e}}$goriques, 1977. Vol.~18. No.\,4. P.~333--407.
\bibitem{15-kov}
\Au{Ad$\acute{\mbox{a}}$mek~J., Herrlich~H., Strecker~G.\,E.} Abstract and concrete 
categories.~--- New York, NY, USA: John Wiley, 1990. 507~p.
\bibitem{16-kov}
\Au{Маклейн С.} Категории для работающего математика~/ Пер. с~англ.~--- М.: Физматлит, 
2004. 352~с. (\Au{Mac Lane~S.} Categories for the working mathematician.~--- New York, NY, 
USA: Springer, 1978. 317~p.)
\bibitem{17-kov}
\Au{Андрюшкевич С.\,К., Журавлев~С.\,С., Золотухин~Е.\,П., Ковалев~С.\,П., 
Окольнишников~В.\,В., Рудометов~С.\,В.} Разработка сис\-те\-мы мониторинга 
с~использованием имитационного моделирования~// Проблемы информатики, 2010. №\,4. 
С.~65--75.
\bibitem{18-kov}
\Au{Baez J., Stay~M.} Physics, topology, logic and computation: A~Rosetta stone~// New 
structures for physics~/ Ed. B.~Coecke.~--- Lecture notes in physics ser.~--- 
Springer, 2011. Vol.~813.  
P.~95--172.
\bibitem{19-kov}
\Au{Ковалёв С.\,П.} Системный анализ жизненного цикла больших  
ин\-фор\-ма\-ци\-он\-но-управ\-ля\-ющих сис\-тем~// Автоматика и~телемеханика, 2013. №\,9. 
С.~98--118.
\bibitem{20-kov}
\Au{Gross J., Chlipala~A., Spivak~D.\,I.} Experience implementing a performant category-theory 
library in Coq~// 5th Conference (International) on Interactive Theorem Proving Proceedings~/ Eds. 
G.~Klein, R.~Gamboa.~--- Lecture notes in computer science ser.~--- Springer, 2014. Vol.~8558.  
P.~275--291.
 \end{thebibliography}

 }
 }

\end{multicols}

\vspace*{-6pt}

\hfill{\small\textit{Поступила в~редакцию 11.12.17}}

\vspace*{8pt}

%\newpage

%\vspace*{-24pt}

\hrule

\vspace*{2pt}

\hrule

%\vspace*{8pt}


\def\tit{CATEGORY THEORY AS~A~MATHEMATICAL 
PRAGMATICS OF~MODEL-BASED SYSTEMS ENGINEERING}

\def\titkol{Category theory as a mathematical pragmatics of model-based systems engineering}

\def\aut{S.~Kovalyov}

\def\autkol{S.~Kovalyov}

\titel{\tit}{\aut}{\autkol}{\titkol}

\vspace*{-9pt}


\noindent
Institute of Control Sciences, Russian Academy of Sciences, 65 Profsoyuznaya Str., 
Moscow 117997, Russian 
Federation 



\def\leftfootline{\small{\textbf{\thepage}
\hfill INFORMATIKA I EE PRIMENENIYA~--- INFORMATICS AND
APPLICATIONS\ \ \ 2018\ \ \ volume~12\ \ \ issue\ 1}
}%
 \def\rightfootline{\small{INFORMATIKA I EE PRIMENENIYA~---
INFORMATICS AND APPLICATIONS\ \ \ 2018\ \ \ volume~12\ \ \ issue\ 1
\hfill \textbf{\thepage}}}

\vspace*{3pt}
   






\Abste{Mathematical device built upon the category theory is developed which was previously 
proposed to formally describe and rigorously explore procedures of employing models in 
engineering that constitute the pragmatics of model-based systems engineering. The 
essence of the device consists in mathematical representation of assembly drawings (megamodels of 
systems) as diagrams in categories whose objects are models, and morphisms represent actions 
associated with assembling system models from component models. Category-theoretical methods 
for solving direct and inverse pragmatic problems of assembling systems are proposed and 
explored. The key role of the diagram monad is revealed. Special attention is paid to the problem of 
recovering the configuration of a given system, taking into account technological limitations of the 
assembling means and procedures. A number of key systems engineering concepts are matched 
with relevant constructions of the category theory.}

\KWE{model-based systems engineering; pragmatics; megamodel; category theory; configuration 
recovery problem; diagram monad}

\DOI{10.14357/19922264180112} 

%\vspace*{-12pt}

%\Ack
%\noindent



%\vspace*{3pt}

  \begin{multicols}{2}

\renewcommand{\bibname}{\protect\rmfamily References}
%\renewcommand{\bibname}{\large\protect\rm References}

{\small\frenchspacing
 {%\baselineskip=10.8pt
 \addcontentsline{toc}{section}{References}
 \begin{thebibliography}{99} 
\bibitem{1-kov-1}
Gianni, D., A.~D'Ambrogio, and A.~Tolk, eds. 2014. \textit{Modeling and simulation-based 
systems engineering handbook}. London: CRC Press. 513~p.
\bibitem{2-kov-1}
\Aue{Selic, B.} 2003. The pragmatics of model-driven development. \textit{IEEE Software} 
20(5):19--25.
\bibitem{3-kov-1}
\Aue{Levenchuk, A.\,I.} 2015. \textit{Sistemnoinzhenernoe myshlenie} [Systems engineering 
thinking]. Moscow: TechInvestLab. 305~p.
\bibitem{4-kov-1}
\Aue{Kovalyov, S.\,P.} 2017. Metody teorii kategoriy v~model'no-orientirovannoy sistemnoy 
inzhenerii [Methods of category theory in model-based systems engineering]. \textit{Informatika 
i~ee Primeneniya~--- Inform. Appl.} 11(3):42--50.
\bibitem{5-kov-1}
IEC 81346-1:2009. 2009. Industrial Systems, Installations and Equipment and Industrial  
Products~--- Structuring Principles and Reference Designations~--- Part~1: Basic Rules. Geneva: 
ISO. 168~p.
\bibitem{6-kov-1}
\Aue{Ginali, S., and J.~Goguen.} 1978. A~categorical approach to general systems. 
\textit{Applied general systems research: Recent development and trends}. Ed.\ 
G.\,J.~Klir. NATO conference ser. Boston, MA: Springer U.S. 5:257--270.

\bibitem{8-kov-1} %7
\Aue{Diskin, Z., S.~Kokaly, and T.~Maibaum.} 2013. Mapping-aware megamodeling: Design 
patterns and laws. 
\textit{Software language engineering: 6th Conference (International) 
Proceedings}. Eds. M.~Erwig, R.\,F.~Paige, and E.~Van Wyk. Lecture notes in computer science 
ser. Springer. 8225:322--343.
\bibitem{9-kov-1} %8
\Aue{Kossiakoff, A., W.\,N.~Sweet, S.~Seymour, and S.\,M.~Biemer.} 2011. \textit{Systems 
engineering principles and practice}. 2nd ed. New York, NY: John Wiley. 560~p.

\bibitem{7-kov-1} %9
\Aue{B$\acute{\mbox{e}}$zivin, J., F.~Jouault, P.~Rosenthal, and P.~Valduriez.} 2005. Modeling 
in the large and modeling in the small. \textit{Model driven architecture: European MDA 
Workshops on Foundations and Applications Proceedings}. Eds. U.~\mbox{A{\!\ptb{\ss}}mann}, M.~Aksit,
and A.~Rensink. Lecture notes in computer science ser. Springer. 3599:33--46.
\bibitem{10-kov-1}
\Aue{Neema, S., J.~Sztipanovits, G.~Karsai, and K.~Butts.} 2003. Constraint-based design-space 
exploration and model synthesis. \textit{3rd Conference (International) on Embedded Software 
Proceedings}. Eds. R.~Alur and I.~Lee. Lecture notes in computer science ser. Springer.  
2855:290--305.
\bibitem{11-kov-1}
\Aue{Vanherpen, K., J.~Denil, P.~De~Meulenaere, and H.~Vangheluwe.} 2014. Design-space 
exploration in MDE: An initial pattern catalogue. \textit{1st Workshop (International) on 
Combining Modelling with Search- and Example-Based Approaches Proceedings}. Eds. R.~Paige, 
M.~Kessentini, P.~Langer, and M.~Wimmer. CEUR Workshop Proceedings ser. Valencia, Spain. 
1340:42--51.
\bibitem{12-kov-1}
GOST 14.205-83. 2009. Tekhnologichnost' konstruktsii izdeliy. Terminy i~opredeleniya 
[Technological efficiency of products design. Terms and definitions]. Moscow: Standardinform 
Publs. 22~p.
\bibitem{13-kov-1}
\Aue{Altshuller, G.} 1984. \textit{Creativity as an exact science: The theory of the solution of 
inventive problems}. Amsterdam: Gordon and Breach Science Publs. 324~p.
\bibitem{14-kov-1}
\Aue{Guitart, R., and L.~van den Bril.} 1977. D$\acute{\mbox{e}}$compositions et 
lax-compl$\acute{\mbox{e}}$tions. \textit{Cah. Topologie 
G$\acute{\mbox{e}}$om$\acute{\mbox{e}}$trie Diff$\acute{\mbox{e}}$rentielle 
Cat$\acute{\mbox{e}}$goriques} 18(4):333--407.
\bibitem{15-kov-1}
\Aue{Ad$\acute{\mbox{a}}$mek, J., H.~Herrlich, and G.\,E.~Strecker.} 1990. \textit{Abstract and 
concrete categories}. New York, NY: John Wiley. 507~p.
\bibitem{16-kov-1}
\Aue{Mac Lane, S.} 1978. \textit{Categories for the working mathematician}. New York, NY: 
Springer. 317~p.
\bibitem{17-kov-1}
\Aue{Andryushkevich, S.\,K., S.\,S.~Zhuravlev, E.\,P.~Zolotukhin, S.\,P.~Kovalyov, 
V.\,V.~Okol'nishnikov, and S.\,V.~Rudometov.} 2010. Razrabotka sistemy monitoringa 
s~ispol'zovaniem imitatsionnogo modelirovaniya [Development of a monitoring system using 
simulation]. \textit{Problemy informatiki} [Problems of Informatics] 4:65--75.
\bibitem{18-kov-1}
\Aue{Baez, J., and M.~Stay.} 2011. Physics, topology, logic and computation: A~Rosetta stone. 
\textit{New structures for physics}. Ed. B.~Coecke. Lecture notes in physics ser. Springer. 
 813:95--172.
\bibitem{19-kov-1}
\Aue{Kovalev, S.\,P.} 2013. Systems analysis of life cycle of large-scale information-control 
systems. \textit{Automat. Rem. Contr.} 74(9):1510--1524.
\bibitem{20-kov-1}
\Aue{Gross, J., A.~Chlipala, and D.\,I.~Spivak.} 2014. Experience implementing a performant 
category-theory library in Coq. \textit{5th Conference (International) on Interactive Theorem 
Proving Proceedings}. Eds. G.~Klein and R.~Gamboa. Lecture notes in computer science ser. 
Springer. 8558:275--291.
\end{thebibliography}

 }
 }

\end{multicols}

\vspace*{-6pt}

\hfill{\small\textit{Received December 11, 2017}}

%\vspace*{-10pt}
  

\Contrl

\noindent
\textbf{Kovalyov Sergey P.} (b.\ 1972)~--- Doctor of Science in physics and mathematics, leading 
scientist, Institute of Control Sciences, Russian Academy of Sciences, 65~Profsoyuznaya Str., 
Moscow 117997, Russian Federation; \mbox{kovalyov@nm.ru}

\label{end\stat}


\renewcommand{\bibname}{\protect\rm Литература} 
   