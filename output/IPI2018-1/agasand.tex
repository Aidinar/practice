
\def\stat{agasand}

\def\tit{КОНТИНУАЛЬНЫЙ КРИТЕРИЙ VaR НА СЦЕНАРНЫХ РЫНКАХ$^*$}

\def\titkol{Континуальный критерий VaR на сценарных рынках}

\def\aut{Г.\,А.~Агасандян$^1$}

\def\autkol{Г.\,А.~Агасандян}

\titel{\tit}{\aut}{\autkol}{\titkol}

\index{Агасандян Г.\,А.}
\index{Agasandyan G.\,A.}




{\renewcommand{\thefootnote}{\fnsymbol{footnote}} \footnotetext[1]
{Работа выполнена при финансовой поддержке РФФИ (проект 17-01-00816).}}


\renewcommand{\thefootnote}{\arabic{footnote}}
\footnotetext[1]{Вычислительный центр им.\ А.\,А.~Дородницына Федерального исследовательского 
центра <<Информатика и~управ\-ле\-ние>> Российской академии наук, 
\mbox{agasand17@yandex.ru}}

%\vspace*{-6pt}
    
      
  
  \Abst{Работа посвящена изучению проблем использования континуального критерия VaR 
(CC-VaR) на сценарных рынках~--- дискретном аналоге идеального теоретического 
однопериодного рынка опционов. Участие инвестора на рынке состоит в~задании им 
прогноза распределения цены базового актива и~формировании функции рисковых 
предпочтений (ф.р.п.). Предлагается дискретный алгоритм оптимизации как результат 
проецирования теоретического алгоритма, основанного на применении процедуры  
Ней\-ма\-на--Пир\-со\-на, на сценарный рынок. Приводится пример нарушения 
оптимальности для небольшого числа сценариев, однако такие нарушения происходят редко и~они весьма незначительны. Предлагается использовать рандомизацию весов найденного 
портфеля как средство сглаживания функций распределения и~повышения качества 
получаемых решений. Специальные алгоритмы предлагаются для расчетов, связанных 
с~доходностью портфелей с~рандомизацией. Изложение иллюстрируется графиками.} 
  
  \KW{континуальный критерий VaR (CC-VaR); сценарии; прогнозная плотность; 
стоимостная плотность; функция рисковых предпочтений (ф.р.п.); оптимальный портфель; 
инвестиционная сумма; доход; доходность; рандомизация}

\DOI{10.14357/19922264180104} 
  
\vspace*{6pt}


\vskip 10pt plus 9pt minus 6pt

\thispagestyle{headings}

\begin{multicols}{2}

\label{st\stat}
   
  \section{Введение}
  
\vspace*{-2pt}
  
  В работе излагается методология применения инвестором в~своих решениях 
на финансовых рынках так называемого \textit{континуального критерия VaR} 
(CC-VaR). Сферой его приложения служат высокоразвитые по многообразию 
торгуемого инструментария рынки, такие как рынки опционов со страйками, 
плотно заполняющими диапазон возможных будущих цен базового актива. 
  
  Критерий CC-VaR~--- естественное континуальное обобщение привычного 
\textit{одноступенчатого} критерия VaR, он должен противостоять как его 
недостаткам, так и~дисперсии в~отношении проблем риска в~задачах условной 
максимизации среднего дохода (см., например,~[1--3]). Суть в~том, что 
применение обычного критерия VaR на рынках опционов по\-рож\-да\-ет доходы, 
обычно меньшие размера инвестиции, с~вероятностью, близкой к~единице (при 
частой решетке страйков), что едва ли может устроить инвестора.
   
Использование дисперсии, как в~классическом подходе Марковица~[1], 
основанной на свойствах случайности не выше второго порядка, не позволяет 
учитывать такие важные нюансы распределений, как, например, часто 
наблюдаемые на рынках тяжелые хвосты распределений, обусловленные 
свойствами случайности не ниже четвертого порядка (куртозис). 
  
  Предлагаемый \textit{континуальный критерий VaR} должен обеспечить 
инвестору (во всяком случае, на теоретическом рынке) получение функции 
распределения доходов с~заранее заданными им же свойствами.
  
  В наиболее простой теоретической модели рынок предполагается 
однопериодным с~одним базовым активом, континуальным по мощности 
инструментария и~идеальным в~том смысле, что цены продавца и~покупателя 
считаются равными, а комиссионные равны нулю~[4, 5].
  
  Вкратце приводятся основные результаты и~формулы для такого рынка, но 
основное внимание уделяется проблемам их приложения к~дискретным по 
инструментам рынкам, рас\-смат\-ри\-ва\-емым как приближение к~континуальным. 

\vspace*{-4pt}
   
  \section{Континуальный $\delta$-рынок и~его~инструменты}
  
  В начале периода цена базового актива~$\mathbf{X}$ известна, а~в~конце 
периода она образует случайную величину~$X$, принимающую значения~$x$ 
из континуального множества ${\sf X}\subset \Re_+$ (или даже~$\Re$). На 
рынке, называемом $\delta$-\textit{рын\-ком}, можно торговать любым 
инструментом~$\mathbf{G}$ с~доходом, представимым в~виде произвольной 
неотрицательной измеримой функции $g(x)$, $x\hm\in {\sf X}$. Ее называем 
\textit{платежной функцией} инструмента и~обозначаем $\pi (x; \mathbf{G})$, $x\hm\in 
{\sf X}$, т.\,е.\ $g(x)\hm= \pi(x; \mathbf{G})$; в~част\-ности, 
$\pi(x; \mathbf{X})\hm=
x$. 
Цена инструмента~$\mathbf{G}$ обозначается $\vert \mathbf{G}\vert$, а средний доход от него~--- 
$\| \mathbf{G}\|$. 
  
  На рынке обращаются так называемые $\delta$-\textit{ин\-стру\-мен\-ты}~$\mathbf{D}(s)$, 
  $s\hm\in {\sf X}$. Платежная функция $\delta$-ин\-стру\-мен\-та 
равна $\delta$-функ\-ции относительно~$s$: $\pi(x;\mathbf{D}(s))\hm\equiv \delta (x\hm-
s)$; такой инструмент порождает нулевой доход, если $x\hm\not= s$, 
и~бесконечный, если $x\hm=s$, $s\hm\in {\sf X}$, притом интеграл от 
платежной функции по $x\hm\in{\sf X}$ равен единице. 
  
  Нетрудно усмотреть в~$\delta$-ин\-стру\-мен\-тах предельный аналог 
специально нормированных баттерфляев обычного рынка опционов при 
стремлении расстояния между соседними страйками к~нулю. 
  
  Заданы \textit{стоимостная} $c(x)$ и~\textit{прогнозная}~$p(x)$ плотности, 
$x\hm\in {\sf X}$. Первая формируется рынком к~началу периода, вторая 
задается инвестором на конец периода. Стоимость (цена)  
$\delta$-ин\-стру\-мен\-тов и~их средний доход соответственно равны:
  $$
  \vert \mathbf{D}(s)\vert =c(s)\,;\enskip \| \mathbf{D}(s)\|=p(s)\,,\enskip s\in{\sf X}\,.
  $$
  
  Портфель $\mathbf{G}$ с~$g(x)\hm=\pi(x;\mathbf{G})$, его цена и~средний доход имеют 
представления: 
  \begin{align*}
  \mathbf{G}&= \int\limits_{\sf X} g(s) \mathbf{D}(s)\,ds\,;\\
  \vert \mathbf{G}\vert &= \int\limits_{\sf X} g(s) \vert \mathbf{D}(s)\vert\,ds=\int\limits_{\sf X} g(s) 
c(s)\,ds\,;\\
  \| \mathbf{G}\| &= \int\limits_{\sf X} g(s)\| \mathbf{D}(s)\|\,ds= 
  \int\limits_{\sf x} g(s) p(s)\,ds\,.
  \end{align*}
  
  Имея в~виду последующий переход к~сценарным рынкам и~сравнение с~ними, 
введенный континуальный рынок будем называть \textit{эталонным}, а~его 
характеристики при сравнении с~дискретными рынками помечать 
\textit{звездочкой}~(*). 
  
  Особый интерес представляют такие континуальные комбинации  
$\delta$-ин\-стру\-мен\-тов, как \textit{индикаторы}~$\mathbf{H}\{M\}$, $M\hm\subset 
{\sf X}$. Платежной функцией $\mathbf{H}\{M\}$ служит характеристическая функция 
множества~$M$, равная единице, если $x\hm\in M$, и~нулю, если $x\hm\not\in 
M$. В~частности, безрисковый инструмент единичного объема~$\mathbf{U}$ является 
индикатором~$\mathbf{H}\left\{{\sf X}\right\}$. Его платежная функция тождественно равна 
единице. Имеют место соотношения:
  $$
  \mathbf{H}\{M\} =\int\limits_M \mathbf{D}(s)\,ds\,;\enskip  
  \mathbf{U}=\mathbf{H}\{{\sf X}\}=\int\limits_{\sf X} 
\mathbf{D}(s)\,ds\,.
  $$
  
  Очевидно, из специально нормированных индикаторов также можно строить 
сходящиеся\linebreak к~$\delta$-ин\-стру\-мен\-там последовательности.
  
  Критерий CC-VaR требует, чтобы порождаемый портфелем инвестора доход~$q$ 
удовлетворял неравенствам: 
  \begin{equation}
  {\sf P}\left\{ q\geq \phi(\varepsilon)\right\} \geq 1-\varepsilon
  \label{e1-ag}
  \end{equation}
  для \textit{всех} $\varepsilon \in [0, 1]$.   
  %
  Здесь ${\sf P}\{M\}$~--- вероятностная мера множества~$M$ в~соответствии 
с~прогнозом инвестора, а $\phi(\varepsilon)$~--- неотрицательная  
мо\-но\-тон\-но воз\-рас\-та\-ющая и~непрерывная \textit{функция рисковых 
предпочтений}\ инвестора. 
  
  Алгоритм оптимизации приводится для \textit{задачи~CB}, основной для 
прочих возможных постановок, в~которой требуется минимизировать 
инвестиционную сумму лишь при выдерживании всех ограничений  
CC-VaR~[5]. 
  
  Применением известной из математической статистики процедуры 
  Ней\-ма\-на--Пир\-со\-на~[6] для всех $\varepsilon\hm\in [0, 1]$ находятся 
оптимальные множества $X_\varepsilon\hm\subset {\sf X}$ прогнозной 
вероятности~$\varepsilon$, а~также прогнозная и~стоимостная функции 
($x\hm\in{\sf X}$): 
  \begin{align*}
 \hspace*{-15mm}{\sf X}_\varepsilon &=\left\{ \rho(x)\leq \tau\right\}\,,\enskip \tau\in \left[ 
\tau^\prime, \tau^{\prime\prime}\right]\,,\\
  &\hspace*{20mm}\tau^\prime=\min\limits_{x\in{\sf X}} \rho(x)\,,\enskip 
\tau^{\prime\prime}=\max\limits_{x\in{\sf X}} \rho(x)\,;\\
   \hspace*{-5mm}{\sf f}_{\sf P} (\tau) &= {\sf P}\left\{ \rho(x)\leq\tau\right\} {\sf F}_{{\sf 
P};\rho(X)}(\tau)\,;\\ 
 \hspace*{-5mm}{\sf f}_{\sf C}(\tau) &={\sf C} \left\{ \rho(x)\leq\tau\right\} 
= {\sf F}_{{\sf C};\rho(X)}(\tau)
  \end{align*}
 (здесь ${\sf F}_{{\sf M};\zeta}(\cdot)$~--- функция распределения случайной 
величины~$\zeta$ по мере~${\sf M}\{\cdot\}$). 
  
  Далее определяются диссонанта, ее производная, функция упорядочения, 
оптимальная весовая функция портфеля и~сам портфель соответственно:
  \begin{gather*}
  \gamma(\varepsilon) ={\sf f}_{{\sf C}} \left( {\sf f}_{\sf 
P}^{\leftarrow}(\varepsilon)\right)\,,\enskip \gamma^\prime(\varepsilon) 
=\fr{1}{{\sf f}_{\sf P}^{\leftarrow}(\varepsilon)}\,;\\
  w(x)={\sf f}_{{\sf P}} (\rho(x))\,;\\
   g(x)=\phi(w(x))\,;\\
 \mathbf{G}=\int\limits_{\sf X} 
g(x)\mathbf{D}(x)\,dx\,.
  \end{gather*}
  
  Наконец, выписываются представления для дохода, среднего дохода, 
инвестиционной суммы, дисперсии доходности и~функции распределения 
дохода (здесь ${\sf E}_{\sf M}$ и~${\sf D}_{\sf M}$~--- символы соответственно 
математического ожидания и~дисперсии по вероятностной мере~${\sf 
M}\{\cdot\}$, $\sigma$~--- стандартное отклонение доходности):
  \begin{equation}
  \left.
  \hspace*{-2mm}\begin{array}{rl}
  q &=g(X)=\phi(w(X))\,;\\[6pt]
  R&=\int\limits_0^1 \phi(\varepsilon)\,d\varepsilon 
=\displaystyle\int\limits_{\tau^\prime}^{\tau^{\prime\prime}} \phi\left( {\sf f}_{\sf 
P}(\tau)\right) {\sf f}_{\sf P}^\prime (\tau)\,d\tau \left( ={\sf E}_{\sf P}q\right)\,;
  \end{array}\!
  \right\}\!\!
  \label{e2-ag}
  \end{equation}
  \begin{equation}
  \left.
  \hspace*{-3mm}\begin{array}{rl}
  A&= \displaystyle\int\limits_0^1 \phi(\varepsilon) d\gamma(\varepsilon) = {}\\[6pt]
  &\hspace*{3mm}{}=
\displaystyle\int\limits_{\tau^\prime}^{\tau^{\prime\prime}} 
\phi\left( {\sf f}_{\sf P}(\tau)\right) 
{\sf f}^\prime_{\sf C}(\tau)\,d\tau \left( ={\sf E}_{\sf C}q\right)\,;\\[6pt]
  y&=\displaystyle\fr{R}{A}-1\,;
  \end{array}
  \right\}
  \label{e3-ag}
  \end{equation}
  
  \vspace*{-12pt}
  
  \noindent
  \begin{multline}
  \sigma^2 =\fr{{\sf D}_{\sf P} q}{A^2} =\int\limits_{\sf X} \fr{ \left(g(x)-
R\right)^2 p(x)\,dx}{A^2} =\fr{\mu_2-R^2}{A^2}\,,\\
  \mu_2= \int\limits_0^1 \phi^2(\varepsilon)\,d\varepsilon\,;
  \label{e4-ag}
  \end{multline}
  
  \vspace*{-6pt}
  
  \noindent
  \begin{equation}
  \left.
  \begin{array}{rl}
  {\sf F}_{{\sf P};q} (z) &=\phi^{\leftarrow}(z)\,,\enskip
  z\in \left[ \phi(0),\phi(1)\right)\,;\\[6pt]
  {\sf P}\left\{ q\geq \phi(\varepsilon)\right\}&\equiv 1-\varepsilon\
  \mbox{для\ \textit{всех }} \varepsilon\in [0,1]\,.
  \end{array}
  \right\}
  \label{e5-ag}
  \end{equation}
  
  Конструкция теоретического $\delta$-рын\-ка и~алгоритм оптимизации 
портфеля используются теперь для изучения дискретного сценарного рынка. 
   
  \section{Сценарный рынок и~его~агрегаты}
  
  Сценарная дискретизация эталонного рынка вводится разбиением множества 
${\sf X}\hm=[x_0, x_n)$ на~$n$~сценариев $S_i\hm=[x_{i-1}, x_i)\hm\subset {\sf X}$, 
$x_{i-1}\hm< x_i$, $i\hm\in I\hm= \{1,\ldots , n\}$. Равномерное разбиение 
выделяется правилом $x_i \hm=x_0 \hm+ ih$, $h\hm=(x_n\hm- x_0)/n$, $i\hm\in I$. 
  
  На сценарном рынке базисными инструментами служат дискретные аналоги 
инструментов $\mathbf{D}(s)$ теоретического рынка~--- индикаторы сценариев 
$\mathbf{D}_i\hm=\mathbf{H}\{S_i\}$, $i\hm\in I$; при этом $\sum\nolimits_{i\in I} 
\mathbf{D}_i\hm=\mathbf{U}$. Их 
платежные функции $\pi(x;\mathbf{D}_i)\hm=\chi_i(x)$, где $\chi_i(x)$~--- 
характеристическая функция множества~$S_i$, а~стоимости~--- 
\textit{рыночные цены} индикаторов~--- получаются из стоимостной плотности 
и~образуют неотрицательный вектор $\mathbf{c}\hm\equiv \{c_i, i\hm\in I\}$, 
где
  $$
  c_i=\left\vert \mathbf{D}_i\right\vert =\int\limits_{x_{i-1}}^{x_i} c(x)\,dx=\int\limits_{\sf 
X} \chi_i(x) c(x)\,dx\,,\enskip i\in I\,.
  $$
    Строго говоря, только сам этот вектор, а~не плот\-ность~$c(x)$ определяет 
рыночную картину текущих цен, а~плот\-ность в~дискретном случае задается 
для удобства моделирования и~сравнительного анализа. Этому вектору 
противостоит прогноз инвестора в~форме получаемого из плотности~$p(x)$ 
вектора \textit{прогнозных вероятностей} сценариев $\mathbf{p}\hm\equiv 
\{p_i\}$, где 
  $$
  p_i=\left\| \mathbf{D}_i\right\| =\int\limits^{x_i}_{x_{i-1}} p(x)\,dx =\int\limits_{\sf X} 
\chi_i(x) p(x)\,dx\,,\enskip i\in I\,,
  $$
т.\,е.\ вероятность сценария~$S_i$ совпадает со средним доходом от 
инструмента~$D_i$ (ее можно считать \textit{справедливой} ценой 
индикатора~$D_i$). 

  Очевидно, $c_i={\sf E}_{\sf C} \chi_i(X)$, $p_i\hm={\sf E}_{\sf P}\chi_i(X)$, 
$i\hm\in I$.
  
  Портфель инвестора $\mathbf{G}\hm=\sum\nolimits_{i\in I} g_i \mathbf{D}_i$ 
  с~$g_i\hm\geq 0$, 
$i\hm\in I$, порождает вектор доходов (весов) $\mathbf{g}\hm= \{g_1,\ldots$\linebreak
$\ldots , 
g_n\}$. (Неотрицательность весов обуслов\-ле\-на выставлением инвестору 
маржевых требований во избежание необеспеченных потерь.) Рыночная цена 
портфеля $\vert \mathbf{G}\vert \hm= \sum\nolimits_{i\in I} g_i c_i$, а~средний 
доход $\| \mathbf{G}\| \hm= \sum\nolimits_{i\in I} g_i p_i$. 
  
  В соответствии с~(1) рисковые интересы ин\-вес\-то\-ра на рынке задаются его 
ф.р.п.~$\phi(\varepsilon)$, $\varepsilon\hm\in [0, 1]$, совместно с~континуальным 
критерием~VaR. 
  
  В связи с~наличием некоторого разнобоя в~формировании прогноза и~ценовой 
картины рынка и~в~отсутствие иной информации вводится естественное
  
  \noindent
  \textit{Предположение~C.} Стоимостная мера ${\sf C}\{\cdot\}$ (независимо 
от ее происхождения) трансформируется в~меру с~иной плотностью~$c(x)$, 
$x\hm\in{\sf X}$, так, чтобы выполнялось тождество $\rho(x)\hm\equiv 
p(x)/c(x)\hm\equiv p_i/c_i$ по $x\hm\in S_i$, $i\hm\in I$.\hfill $\square$
  
  Действие этого предположения меняет исходный \textit{эталонный} 
теоретический рынок, основанный на изначальной паре $\{p(x), c(x)\}$, 
привязывая его к~данному сценарному рынку, но и~оставляя возможность 
считать его теоретическим. Будем назы\-вать новую схему 
\textit{вариантом}~\#0. Отметим, что\linebreak предлагаемая \textit{предположением~C} 
трансформация, пожалуй, самая простая из всех допускающих естественную 
упорядоченность элементов~{\sf X} по величине~$\rho(x)$ внутри сценариев. 
   
  \section{Дискретный алгоритм оптимизации портфелей}
  
  Предлагается адаптировать континуальный алгоритм построения 
оптимального портфеля, рассмотренного в~\cite{5-ag}, к~сценарному рынку. 
При этом адаптация возможна в~нескольких вариантах назначения 
портфельных весов. 
  
  Дискретный алгоритм строится как результат проецирования континуального 
алгоритма на сценарный рынок и~состоит из общей и~специфической частей. 
При этом целью алгоритма, как и~для эталонного рынка, служит решение 
\textit{задачи~CB}, и~также применяется процедура Ней\-ма\-на--Пир\-сона.
  
  Итак, заданы векторы~$\mathbf{c}$ и~$\mathbf{p}$ размерности~$n$ для 
сценариев~$S_i$, $i\hm\in I$. Дискретный алгоритм описывается следующей 
последовательностью обозначений и~операций (при этом $i, j\hm\in I$, все 
векторы имеют длину~$n$, а~матрицы~---  размер $n\times n$): 
  \begin{description}
  \item[\,] $\boldsymbol{\rho}=\mathbf{p}/\mathbf{c}$~--- вектор относительных 
доходов $\rho_i\hm=p_i/c_i$, дискретный аналог функции относительных 
доходов~$\rho(x)$;
  \item[\,] $\boldsymbol{\xi}=\mathbf{O}(\boldsymbol{\rho})$~--- вектор, 
задающий на множестве сценариев позиции компонент 
вектора~$\boldsymbol{\rho}$ в~порядке возрастания: первый элемент указывает 
позицию (номер) наименьшего относительного дохода, $n$-й~--- 
наибольшего, $\mathbf{O}$~--- со\-от\-вет\-ст\-ву\-ющее преобразование; 
  \item[\,] $\boldsymbol{\eta}= \mathbf{O}(\boldsymbol{\xi})$~--- вектор, 
обратный к~$\boldsymbol{\xi}$: если $\xi_i\hm=j$, то $\eta_{j}\hm=i$; 
компонента~$\eta_j$ означает номер дохода~$\rho_j$ в~вариационном ряду для 
вектора~$\boldsymbol{\rho}$; при этом также 
$\boldsymbol{\xi}\hm=\mathbf{O}(\boldsymbol{\eta})$; 
  \item[\,] $\boldsymbol{\tau} =\boldsymbol{\rho}(\boldsymbol{\xi})$~--- вектор 
относительных доходов с~упорядоченными по возрастанию компонентами; 
  \item[\,] $\boldsymbol{\Xi}\hm=\{y_{ij}\}$, где 
  $$
  y_{ij}=
  \begin{cases}
  1, &\ \mbox{если } \xi_i=j\,;\\ 
0, &\ \mbox{если } \xi_i\not=j;
\end{cases}
$$
эта матрица реализует 
преобразование~$\mathbf{O}$, и~име\-ют место равенства 
$\boldsymbol{\eta}\hm=\boldsymbol{\Xi}\boldsymbol{\xi}$, 
$\boldsymbol{\xi}\hm= \boldsymbol{\Xi}^{-1}\boldsymbol{\eta}$; 
  \item[\,] $\mathbf{T} = \{t_{ij}\}$, где
  $$
  t_{ij}=\begin{cases}
  1, &\ \mbox{если } i\leq j\,;\\
  0, &\ \mbox{если } i>  j.
  \end{cases}
  $$ 
  Это треугольная матрица для последовательного суммирования компонент 
векторов, начиная с~первой; 
  \item[\,] 
$\mathbf{d}=\boldsymbol{\Xi}\mathbf{p}=\mathbf{p}(\boldsymbol{\xi})$~--- 
подстановка вектора~$\mathbf{p}$, компоненты которой упорядочены по 
возрастанию компонент вектора~$\boldsymbol{\rho}$; обратно, верны 
равенства $\mathbf{p}\hm= \boldsymbol{\Xi}^{-1} \mathbf{d}\hm= 
\mathbf{d}(\boldsymbol{\eta})$;
  \item[\,] $\boldsymbol{\varepsilon}=\mathbf{T}\mathbf{d}$~--- вектор 
кумулятивных вероятностей для вектора~$\mathbf{d}$; 
  \item[\,] $\mathbf{f}=\boldsymbol{\Xi}\mathbf{c}\hm= 
\mathbf{c}(\boldsymbol{\xi})$~--- подстановка~$\mathbf{c}$, компоненты 
которой упорядочены по возрастанию компонент вектора~$\boldsymbol{\rho}$; 
$\mathbf{c}\hm= \boldsymbol{\Xi}^{-1} \boldsymbol{f}\hm= 
\mathbf{f}(\boldsymbol{\eta})$;
  \item[\,] $\gamma=\mathbf{T}\mathbf{f}$~--- вектор кумулятивных цен для 
вектора~$\mathbf{f}$.
  \end{description}
  
  Считаем еще, что $\varepsilon_0\hm=\gamma_0\hm=0$, но 
в~векторы~$\boldsymbol{\varepsilon}$ и~$\boldsymbol{\gamma}$ эти величины 
не включаются. И~справедливыми будут, например, формулы $d_i\hm=
  \varepsilon_i \hm- \varepsilon_{i-1}$, $f_i\hm=\gamma_i \hm- \gamma_{i-1}$, $i,j\hm\in I$.
  
  Можно добавить, что для сценарного рынка аналогами прогнозной 
функции~${\sf f}_{\sf P}(\tau)$, стоимостной функции~${\sf f}_{\sf C}(\tau)$, 
диссонанты~$\gamma(\varepsilon)$ и~функции упорядочения~$w(x)$~\cite{5-ag}, 
важных для теоретической модели характеристик, служат соответственно пары 
векторов $\{\boldsymbol{\tau},\boldsymbol{\varepsilon}\}$, 
$\{\boldsymbol{\tau},\boldsymbol{\gamma}\}$, 
$\{\boldsymbol{\varepsilon},\boldsymbol{\gamma}\}$ 
и~суперпозиция~$\boldsymbol{\varepsilon}(\boldsymbol{\eta})$. 
  
  Все эти операции составляют \textit{общую} часть алгоритма. 
  
  \textit{Специфическая} часть алгоритма связана с~назначением весов 
базисных инструментов в~искомом портфеле~--- с~векторами~$\mathbf{b}$ 
и~$\mathbf{g}$. Первый из них показывает веса в~порядке возрастания 
компонент вектора~$\boldsymbol{\rho}$, а~второй~--- в~исходном порядке. 
Вектору~$\mathbf{b}$ соответствует вектор вероятностей~$\mathbf{d}$, 
а~вектору~$\mathbf{g}$~--- вектор~$\mathbf{p}$.
  
  В дискретном случае назначать веса можно не единственным способом. 
Обязательно лишь, чтобы выдерживался единый порядок возрастания 
относительных доходов от сценария к~сценарию. Сначала назначается 
вектор~$\mathbf{b}$, а~на его основе уже однозначно определяется 
вектор~$\mathbf{g}$ по формуле: 
  $$
  \mathbf{g}=\boldsymbol{\Xi}^{-1} 
\mathbf{b}=\mathbf{b}(\boldsymbol{\eta})\ (\mbox{обратно},\ 
\mathbf{b}=\boldsymbol{\Xi}\mathbf{g}=\mathbf{g}(\boldsymbol{\xi})).
  $$ 
  Портфель, получаемый применением дискретного алгоритма оптимизации 
для сценарного рынка, представляется в~виде взвешенной суммы индикаторов:
  $$
  \mathbf{G} = \sum\limits_{i\in I} g_i \mathbf{D}_i\,.
  $$
  
  Основными числовыми показателями инвестиции служат ее сумма~$A$, 
средний доход~$R$ (сам доход~---~$\zeta$), средняя доходность~$y$ 
и~стандартное отклонение доходности~$\sigma$. Они образуют запись 
$\mathbf{J}\hm= \langle A, R, y, \sigma\rangle$. Для \textit{дискретного} рынка 
имеем: 
  \begin{equation}
  \left.
 \hspace*{-3mm} \begin{array}{rlrl}
  A&=(\mathbf{g},\mathbf{c})= (\mathbf{b},\mathbf{f})\,;&\enskip
  R&=(\mathbf{g},\mathbf{p})=(\mathbf{b},\mathbf{d})\,;\\[6pt]
  y&=\displaystyle\fr{R}{A}-1\,;&\enskip 
  \sigma&=\left( \fr{(\mathbf{g} -R)^2}{A^2},\mathbf{p}\right)^{1/2}\,.
  \end{array}\!
  \right\}\!
  \label{e6-ag}
  \end{equation}
Здесь $(\mathbf{u},\mathbf{v})$ означает скалярное произведение 
векторов~$\mathbf{u}$ и~$\mathbf{v}$. 
  
  По векторам~$\mathbf{b}$ или~$\mathbf{g}$ можно определять и~функцию 
распределения~${\sf F}(z)$ дохода. Ее представления с~суммированием 
в~порядке возрастания весов портфеля и~в исходном порядке сценариев даются 
соответственно формулами:
  $$
  {\sf F}(z)=\sum\limits_{i\in I} d_i \chi_{[b_i\infty)}(z)
  $$
  и
  $$
  {\sf F}(z)=\sum\limits_{i\in I} p_i \chi_{[g_i\infty)}(z)\,.
  $$
  
  Теперь обратимся к~наиболее интересным ва\-риантам назначения весов. Для 
\textit{варианта}~\#0, вве\-денного в~разд.~3, производная диссонанты 
и~ин\-вестиционная сумма в~терминах характеристик сценарно\-го рынка 
приобретают вид: 
  \begin{align}
  \hspace*{-1mm}\gamma_0^\prime(z) &= \gamma_i^\prime =\fr{\gamma_i-\gamma_{i-1}} 
{\varepsilon_i-\varepsilon_{i-1}}\,,\ z\in \left( \varepsilon_{i-
1},\varepsilon_i\right)\,,\ i\in I\,;\label{e7-ag}\\
   \hspace*{-1mm} A_0 &= \int\limits_0^1\phi(z)\gamma_0^\prime(z)\,dz=\sum\limits_{i\in I} \left( 
\gamma_i^\prime \int\limits_{\varepsilon_{i-1}}^{\varepsilon_i} 
\phi(z)\,dz\right).\label{e8-ag}
  \end{align}
  
  Среди собственно дискретных наиболее естественным выглядит 
\textit{вариант}~\#1, для которого, как нетрудно видеть, выдерживаются все 
ограничения CC-VaR: 
  \begin{equation}
  \mathbf{b}_1=\phi(\varepsilon)\,;\quad 
\mathbf{g}_1=\mathbf{b}_1(\boldsymbol{\eta})\,.
  \label{e9-ag}
  \end{equation}
  
  В множестве вариантов, сохраняющих тот же порядок сценариев, выделяется 
и~\textit{вариант}~\#2, определяющий вектор~$\mathbf{b}$ интегральным 
осреднением: 
  \begin{equation}
  \left.
  \begin{array}{rl}
  \mathbf{b}_2 &= \left\{ b_i,\ i\in I\right\}\,,\enskip 
  b_{2;i} =\displaystyle\int\limits^{\varepsilon_i}_{\varepsilon_{i-1}} 
\fr{\phi(z)\,dz}{d_i}\,,\\[6pt]
 &\hspace*{30mm}d_i=\varepsilon_i-\varepsilon_{i-1}\,;\\[6pt] 
\mathbf{g}_2&=\mathbf{b}_2(\boldsymbol{\eta})\,.
  \end{array}
  \right\}
  \label{e10-ag}
  \end{equation}
  
  Этот вариант представляет не только определенный теоретический интерес. 
Хотя в~нем неравенства CC-VaR, как правило, нарушаются, но во многих иных 
отношениях \textit{вариант}~\#2 демонстрирует большее сходство 
с~теоретическим \textit{вариантом}~\#0 и~вполне может рассматриваться как 
приближение к~нему.
  
  Отметим еще, что для \textit{варианта}~\#0, как для тео\-ре\-ти\-че\-ско\-го, 
справедлива формула для функции распределения дохода портфеля~$\zeta$ 
(ср.\ также~(\ref{e5-ag})):
  $$
  {\sf F}_{\zeta;0}(z)={\sf F}^*_{\zeta}(z) ={\sf F}_{{\sf 
P};q}(z)=\phi^{\leftarrow}(z)\,,\ z\in [\phi(0),\phi(1))\,.
  $$
  
  Сравнение записей~$\mathbf{J}_0$ и~$\mathbf{J}_2$ весьма показательно. 
Из~(\ref{e2-ag}), (\ref{e5-ag}) и~(\ref{e10-ag}) имеем:
  \begin{multline}
  R_0\left(=R^*\right) =\sum\limits_{i\in I}\left(\ \, \int\limits_{\varepsilon_{i-
1}}^{\varepsilon_i} \phi(z)\,dz\right) ={}\\
  {}=\sum\limits_{i\in I} b_{0,i}\left( \varepsilon_i -\varepsilon _{i-
1}\right)=\sum\limits_{i\in I} b_{2,i} d_i =R_2\,.
  \label{e11-ag}
  \end{multline}
  
  Аналогичным образом, применяя к~(\ref{e3-ag}) соотношения~(\ref{e7-ag}) 
и~(\ref{e8-ag}) и~вновь учитывая результаты дискретного алгоритма, получаем 
для инвестиционных сумм:
  \begin{multline}
  A_0=\int\limits_{\varepsilon_{i-1}}^{\varepsilon_i} \phi(z)\gamma^\prime(z)\,dz 
={}\\
{}=\sum\limits_{i\in I} \left( \fr{\gamma_i-\gamma_{i-1}} {\varepsilon_i - 
\varepsilon_{i-1}} \int\limits_{\varepsilon_{i-1}}^{\varepsilon_i} \!\phi(z)\,dz\!\right) 
={}\\
{}=\sum\limits_{i\in I} \left( b_{2,i} f_i\right) =A_2.\!\!
  \label{e12-ag}
  \end{multline}
  
  Таким образом, в~\textit{вариантах}~\#0 и~\#2 совпадают инвестиционные 
суммы и~средние доходы, а~потому и~средние доходности. Однако оптимальные 
портфели в~континуальном и~дискретном вариантах образуются по-раз\-но\-му, 
что проявляется в~стандартном отклонении (и~дисперсии) до\-ход\-ности, и,~как 
правило, $\sigma_0\hm\not= \sigma_2$. Для \textit{варианта}~\#0 используются 
формулы~(\ref{e4-ag}), для \textit{варианта}~\#2~--- (\ref{e6-ag}). 
  
  Предложим еще краткую запись алгоритма (приводится для варианта~\#1): 
  \begin{gather*}
  \boldsymbol{\rho}=\fr{\mathbf{p}}{\mathbf{c}},\enskip 
\boldsymbol{\xi}=\mathbf{O}(\boldsymbol{\rho}),\enskip 
\boldsymbol{\eta}=\mathbf{O}(\boldsymbol{\xi}),\\ 
\mathbf{d}=\mathbf{p}(\boldsymbol{\xi}),\enskip \mathbf{T}=[t_{ij}],\enskip
t_{ij}=\begin{cases}
1, &\ \mbox{если } i\leq j;\\ 
0, &\ \mbox{если } i>j,
\end{cases}\\ 
\boldsymbol{\varepsilon}=\mathbf{T}\mathbf{d};\enskip 
\mathbf{b} =\phi(\boldsymbol{\varepsilon}),\enskip 
\mathbf{g} = \mathbf{b}(\boldsymbol{\eta}). 
  \end{gather*}
   
  \section{Проблемы оптимальности дискретного портфеля}
  
  Если для теоретического \textit{варианта}~\#0 гарантируется оптимальность 
получаемого портфеля, то для прочих вариантов она остается под вопросом. 
  
  Без труда доказываются простые утверждения. 
  
  \smallskip
  
  \noindent
  \textbf{Лемма~1.}\ \textit{Если $p_i\hm\equiv 1/n$, $i\hm\in I$, то решение, 
доставляемое дискретным алгоритмом в}~\textit{варианте}~\#1, 
\textit{оптимально.}
  
  \smallskip
  
  \noindent
  \textbf{Лемма~2.}\ \textit{При $n\hm=2$ решение, доставляемое дискретным 
алгоритмом в}~\textit{варианте}~\#1, \textit{оптимально}. 
  
  \smallskip
  
  Однако в~общем случае гарантировать оптимальность не удается, хотя 
пример неоптимальности портфеля обнаруживается непросто (в~нем $n\hm = 
3$). 
  
  Пусть $\mathbf{c}=\{0{,}7; 0{,}2; 0{,}1\}$, $\mathbf{p}\hm =\{0{,}1; 0{,}599; 
0{,}301\}$. Имеем $p_1\hm/c_1\hm<p_2 /c_2\hm\leq p_3/c_3$, и~потому 
$\mathbf{d}\hm=\mathbf{p}$, $\mathbf{f}\hm=\mathbf{c}$. Применение 
дискретного алгоритма дает: 
  \begin{gather*}
  \boldsymbol{\rho}=\left\{
  0{,}142857; 2{,}995; 3{,}01\right\};\enskip \boldsymbol{\xi}=\{1,\ 2,\ 3\};\\ 
  \mathbf{b} = \mathbf{g} = \{0{,}01; 0{,}488601; 1{,}0\};\\ 
  A = 0{,}20472;\enskip R = 0{,}594672;\enskip y = 1{,}9048\,. 
  \end{gather*}
    Однако в~этом случае при $\boldsymbol{\xi}^\prime \hm=\{1, 3, 2\}$ 
доходность выше: 
  $$
  A^\prime=0{,}22308;\enskip R^\prime = 0{,}648401;\enskip y^\prime = 1{,}90658\,. 
  $$
  
  Нелишне отметить, что в~данном примере область параметров 
неоптимальности алгоритма весьма узкая и~содержится для~$p_2$ в~интервале 
$[0{,}598; 0{,}6]$. 
  
  Можно предложить простую процедуру, позволяющую устанавливать 
оптимальность портфеля, но сводящуюся к~полному перебору. 
  
  Пусть в~результате применения дискретного алгоритма 
  к~вектору~$\boldsymbol{\rho}$ получены 
$\boldsymbol{\xi}\hm=\mathbf{O}(\boldsymbol{\rho})$ и~средний 
относительный доход~$r$. К~произвольной 
подстановке~$\boldsymbol{\xi}^\prime$ вектора~$\boldsymbol{\xi}$, 
$\boldsymbol{\xi}^\prime \hm \not= \boldsymbol{\xi}$, можно применить тот же 
алгоритм (опуская операции с~вектором~$\boldsymbol{\rho}$) 
и~определить~$\boldsymbol{\eta}^\prime$, $\mathbf{b}^\prime$ 
и~$\mathbf{g}^\prime\hm=\mathbf{b}^\prime(\boldsymbol{\eta}^\prime)$ 
и,~наконец, $A^\prime$, $R^\prime$ и~$r^\prime$. Если для всех возможных 
подстановок оказывается, что $r^\prime\hm\leq r$, то искомый портфель~--- 
оптимальный. 
  Однако такая проверка технически осуществима лишь при небольших 
значениях~$n$, что, по существу, оправдывает проводимые в~работе 
исследования. 
  
  Несмотря на свою ограниченную непосредст\-венную применимость, лемма~1 
позволяет про\-ана\-ли\-зи\-ро\-вать оптимальность получаемого алгорит\-мом решения 
при возрастании степени дробления\linebreak сценариев на подсценарии. 
  
  Пусть $\Delta$~--- сколь угодно малая положительная величина. Вводится 
$\Delta$-ры\-нок, образуемый разбиением каждого сценария~$S_i$, $i\hm\in I$, 
на стандартные подсценарии вероятности~$\Delta$ и~один нестандартный 
вероятности $\Delta_i^\prime\hm< \Delta$. Можно показать, что для 
ограниченной ф.р.п.\ портфель, получаемый дискретным алгоритмом на таком 
$\Delta$-рын\-ке, при $\Delta\hm\to 0$ асимптотически оптимален.
   
  \section{Рандомизация портфеля на~сценарном рынке}
  
  Вернемся к~исходному дискретному рынку. Очевидно, платежная функция 
такого портфеля является ступенчатой функцией, при этом выбор весов 
в~пределах допускаемых дискретным алгоритмом диапазонов сказывается на 
степени выполнения ограничений CC-VaR и~близости функции распределения 
дохода к~теоретической функции~${\sf F}_0(x)$. Так, ограничения CC-VaR 
выдерживаются пол\-ностью лишь для \textit{варианта}~\#1, но в~отношении 
бли\-зости к~функции~${\sf F}_0(x)$, во всяком случае <<на глаз>>,~--- для 
\textit{варианта}~\#2. 
  
  Попытаемся улучшить качество функции распределения посредством 
рандомизации портфеля. Такую схему будем именовать \textit{вариантом}~\#3. 
Рандомизацию проведем, придавая весам портфеля случайный характер:
  \begin{equation}
  \mathbf{G}_3=\sum\limits_{i\in I} \omega_i \mathbf{D}_i\,,\quad \omega_i=\phi(\theta_i)\,,
  \label{e13-ag}
  \end{equation}
где $\theta_i\sim {\sf R}\{{\sf E}_i\}$, $i\hm\in I$,~--- равномерно 
распределенные случайные величины на последовательных полуинтервалах: 
\begin{multline*}
  {\sf E}_i=\left[ \underline{e}_i, \overline{e}_i\right)\,,\ 
\underline{e}_i=\varepsilon_{\eta_i-1}\left( =0,\ \eta_i=1\right)\,,\\
 \overline{e}_i 
=\varepsilon_{\eta_i}\,,\ \overline{e}_i-\underline{e}_i=p_i\,,\ i\in I\,.
\end{multline*}
  
  Случайный доход для портфеля~(\ref{e13-ag}) представляется в~виде:
  $$
  \zeta_3=\sum\limits_{i\in I} \phi(\theta_i)\pi(x;\mathbf{D}_i)\,,
  $$
а~функции распределения ${\sf F}_{\omega;i}(z)$ весов~$\omega_i$ 
базисных индикаторов на интервалах их роста (для упрощения формулы 
приводятся лишь для таких интервалов):
\begin{multline*}
  {\sf F}_{\omega;i} (z) ={\sf P}\left\{ \omega_i\leq z\right\} ={}\\
  {}={\sf P}\left\{ 
\phi(\theta_i) \leq z\right\} ={\sf P} \left\{ \theta_i\leq \phi^{\leftarrow}(z)\right\} 
=\fr{\phi^{\leftarrow}(z)-\underline{e}_i}{p_i}\,,\\
 i\in I\,.
  \end{multline*}
  
  Поскольку~$\omega_i$ является условным доходом при реализации~$S_i$, 
  то 
  $$
  {\sf F}_{\omega; i}(z)= {\sf F}_{\zeta;3}(z\vert S_i)\,,\enskip i\in I\,,
  $$
и по формуле полной вероятности находим: 
\begin{multline*}
{\sf F}_{\zeta;3}(z)=\sum\limits_{i\in I}{\sf F}_{\zeta;3} (z\vert S_i) p_i
 =\sum\limits_{i\in I} {\sf F}_{\zeta;3}(z\vert 
S_{\xi_i})d_i={}\\
{}= \phi^{\leftarrow}(z)\,,\ z\in [0,1]\,.
\end{multline*}
  
  Для \textit{варианта}~\#3 функция распределения дохода портфеля ${\sf 
F}_{\zeta;3}(z)$ совпадает с~функцией~${\sf F}_{{\sf P};q}(z)$ 
в~формуле~(\ref{e5-ag}). Поэтому, очевидно, имеет место  
(с~учетом~(\ref{e11-ag})): 
  \begin{equation}
  R_3=R_0\left(=R_2\right)\,.
  \label{e14-ag}
  \end{equation}
  
  При этом также инвестиционная сумма~$A_3$ для рандомизированного 
портфеля является случайной величиной. Тем не менее ее среднее значение 
  $$
  A_3={\sf E} A_3 =\sum\limits_{i\in I} c_i {\sf E}\omega_i\,,
  $$
  где
  $$
  {\sf E} \omega_i =\int\limits_{\underline{e}_i}^{\overline{e}_i}
  \fr{\phi(t)dt}{p_i^S}\,,
  $$
  и
  \begin{equation}
  A_3=\sum\limits_{i\in I} \fr{c_i}{p_i} 
\int\limits_{\underline{e}_i}^{\overline{e}_i} \phi(t)\,dz =A_2=A_0\,.
  \label{e15-ag}
  \end{equation}
  
  Такое совпадение средних имеет место при \textit{любом} совместном 
распределении вероятностей компонент вектора~$\boldsymbol{\theta}$ 
(и~$\boldsymbol{\omega}$). Но оно не распространяется уже на средние 
значения \textit{относительного} дохода. Даже в~самом простом 
и~естественном случае принимаемой далее \textit{взаимной независимости} 
компонент вектора~$\boldsymbol{\theta}$ функции распределения 
\textit{относительного дохода}, их математические ожидания и~дисперсии 
в~\textit{вариантах}~\#0 и~\#3 могут различаться, а~требования CC-VaR~--- не 
выполняться. Более того, в~силу высокой кратности интегралов, возникающих 
при расчетах, понадобятся специальные численные методы (см.\ разд.~8). 
   
  \section{Иллюстративный пример}
  
  Допустим, что плотности $p(x)$ и~$c(x)$ подчиняются  
бе\-та-рас\-пре\-де\-ле\-нию: $p(x)\sim \mathrm{Be}\,(\alpha_1, \alpha_2)$, 
$c(x) \sim \mathrm{Be}\,(\beta_1, \beta_2)$, при этом $\alpha_1\hm = 2$; $\alpha_2 \hm= 1{,}5$; 
$\beta_1\hm= 3$; $\beta_2 \hm= 2{,}5$, т.\,е. \
  \begin{multline}
  p(x) =\fr{x(1-x)^{0{,}5}}{B(2, 1{,}5)}\,;\enskip c(x) = \fr{x^2(1-
x)^{1{,}5}}{B(3,2{,}5)}\,,\\ x\in {\sf X} = [0,1)\,,
  \label{e16-ag}
  \end{multline}
где $B(\alpha_1, \alpha_2)$~--- бе\-та-функ\-ция: $B(2, 1{,}5)\hm\approx 3{,}75$; 
$B(3, 2{,}5)\hm\approx 19{,}6875$.
  
  Также считаем, что $\phi(\varepsilon)\hm=\varepsilon^2$, $\varepsilon\hm\in [0, 
1]$, и~потому функция распределения доходов оптимального теоретического 
портфеля (кстати, как и~для эталонного) ${\sf F}_0(z)\hm= 
\phi^{\leftarrow}(z)\hm= z^{1/2}$, $z\hm\in [0, 1]$. 
  
  Расчет записи~$\mathbf{J}^*$ для \textit{эталонного} варианта (т.\,е.\ 
в~соответствии с~плотностями~(\ref{e16-ag}) и~вне сферы действия 
\textit{предположения~C}) должен проводиться по  
формулам~(\ref{e2-ag})--(\ref{e5-ag}). Но для  
бе\-та-рас\-пре\-де\-лен\-ных плотностей точного аналитического решения 
получить не удается. Поэтому потребуется либо приближенно вычислять 
трудоемкие па\-ра\-мет\-ри\-че\-ские интегралы, либо находить решение для 
дискретной модели с~достаточно большим~$n$. Второй способ определенно 
предпочтительнее, и~в~результате при $n\hm=5000$ получаем: 
  $$
  \mathbf{J}^*\approx \langle 0{,}239612; 0{,}333441; 0{,}391586; 
1{,}24444\rangle\,. 
  $$
  
  Заметим для сравнения, что теоретическое значение 
$R^*\hm=1/(1+\lambda)\hm\approx 0{,}333333$. 
  
  Сценарный рынок определяется равномерным разбиением диапазона цен 
(для бе\-та-рас\-пре\-де\-ле\-ния) ${\sf X}\hm=[0,1)$ на $n\hm=5$ сценариев. 
Вычисления дают: 
  \begin{align*}
  \mathbf{p} &= \{0{,}070; 0{,}187; 0{,}263; 0{,}284; 0{,}197\}\,;\\
  \mathbf{c}&= \{0{,}041; 0{,}206; 0{,}341; 0{,}309; 0{,}104\}\,.
\end{align*}
  
  Затем на основе \textit{предположения~С} и~плотности $p(\cdot)$ 
из~(\ref{e16-ag}) рассчитывается \textit{вариант}~\#0, для чего каждый из 
$n\hm=5$ сценариев делится на $k\hm=1000$ равных подсценариев. 
Результатом расчета с~5000~подсценариев будет запись:
  $$
  \mathbf{J}_0\approx \langle 0{,}253719; 0{,}333446; 0{,}314232; 1{,}1753\rangle\,. 
  $$
  
  Как и~должно быть, функция распределения \textit{дохода} и~средний 
\textit{доход} для варианта~\#0 совпадают с~\textit{эталонными}. Однако 
большая (в~данном случае) \textit{стоимость} портфеля растягивает график 
функции распределения \textit{доходности} по оси абсцисс в~меньшей степени, 
а также снижает ее среднее и~дис\-пер\-сию. 
  
  Теперь применяется алгоритм оптимизации для $n\hm=5$. Получаем: 
  
  \noindent
  \begin{align*}
  {\boldsymbol\xi}&= \{ 3, 2, 4, 1, 5\}\,;\\
   \boldsymbol{\eta}&=\{4, 2, 1, 3, 5\}\,;\\
  \boldsymbol{\varepsilon}&=\{0{,}263; 0{,}450; 0{,}733; 0{,}803; 1{,}0\}\,; \\
  \underline{\mathbf{e}}&=\{0{,}733; 0{,}263; 0; 0{,}450; 0{,}803\}\,;\\
  \overline{\mathbf{e}}&=\{0{,}803; 0{,}450; 0{,}263; 0{,}733; 1{,}0\}.
  \end{align*}
  
  Веса портфеля для \textit{вариантов}~\#1 и~\#2 назначаются по 
формулам~(\ref{e9-ag}) и~(\ref{e10-ag}): 

\noindent
  \begin{align*}
  \mathbf{b}_1&= \{0{,}069; 0{,}202; 0{,}538; 0{,}645; 1{,}0\}\,;\\
  \mathbf{g}_1&= \{0{,}645; 0{,}202; 0{,}069; 0{,}538; 1{,}0\}\,; \\
  \mathbf{J}_1&\approx \langle 0{,}361; 0{,}450; 0{,}246; 0{,}934\rangle\,; \\
  \mathbf{b}_2&= \{0{,}023; 0{,}130; 0{,}357; 0{,}591; 0{,}816\}\,; \\
  \mathbf{g}_2&=\{0{,}591; 0{,}130; 0{,}023; 0{,}357; 0{,}816\}\,; \\
  \mathbf{J}_2 &\approx \langle 0{,}254; 0{,}333; 0{,}314; 1{,}141\rangle\,. 
  \end{align*}
  
  В соответствии с~соотношениями~(\ref{e4-ag}), (\ref{e6-ag}), (\ref{e11-ag}) 
и~(\ref{e12-ag}) первые три компоненты в~записях~$\mathbf{J}_0$ 
и~$\mathbf{J}_2$ должны совпадать, но 
 четвертые~--- различаться, $\sigma_0\hm\not=\sigma_2$. 
 
 
  
  Для \textit{вариантов}~\#1 и~\#2 графики функций распределения доходов 
${\sf F}_{\zeta;1}(\cdot)$ и~${\sf F}_{\zeta;2}(\cdot)$ совместно  
с~${\sf F}_{\zeta;0}(\cdot)$ приводятся на рис.~1 (сплошная, штриховая 
и~пунктирная линии соответственно). Поскольку функция распределения~${\sf 
F}_{\zeta;3}(\cdot)$ совпадает с~${\sf F}_{\zeta;0}(\cdot)$, она на рисунке уже не 
отображается. 

 { \begin{center}  %fig1
 \vspace*{1pt}
 \mbox{%
 \epsfxsize=77.698mm 
 \epsfbox{aga-1.eps}
 }


\end{center}


\noindent
{{\figurename~1}\ \ \small{Графики функций распределения доходов:
\textit{1}~--- ${\sf F}_{\zeta;0}$; \textit{2}~--- ${\sf F}_{\zeta;1}$; \textit{3}~---
${\sf F}_{\zeta;2}$}}
}

\pagebreak

%\vspace*{9pt}

\addtocounter{figure}{1}
 
  
  Функции распределения для \textit{относительного} дохода в~вариантах~\#0, 
\#1 и~\#2 получаются из аналогичных функций для \textit{дохода} простым 
делением аргумента на разные (по вариантам) детерминированные стоимости. 
И~лишь для варианта~\#3 из-за случайности стоимости требуются специальные 
вычисления.



\section{Вычисление функции распределения 
относительного дохода} 
  
  Относительный доход рандомизированного портфеля $\psi_3\hm=\zeta_3/A_3$, 
где $\zeta_3$~--- случайный доход портфеля; $A_3$~--- его стоимость (на этот 
раз \textit{случайная}~--- отсюда и~ее иное обозначение). Для этого 
\textit{варианта}~\#3 имеем:
  $$
  A_3\left(\theta_1,\theta_2,\ldots ,\theta_n\right) =\sum\limits_{l\in I} 
c_l\phi(\theta_l)\,.
  $$
  
  Случайность стоимости и~взаимозависимость величин~$\xi_3$ и~$A_3$ 
усложняют расчеты функции распределения~${\sf F}_{\psi;3}(\cdot)$, которая 
в~данном случае записывается в~виде многократного интеграла с~аргументом 
функции распределения в~качестве его параметра. Для реализации 
приближенных вы\-чис\-ле\-ний предлагается применять разновидность метода 
Мон\-те Кар\-ло с~детерминированной решеткой значений переменных 
интегрирования. 
  
  Для вектора рандомизации~$\boldsymbol{\theta}$ для каждого 
сценария~$S_i$, $i\hm\in I$, вводится равномерная решетка с~$n_J$~узлами 
(всего $n_J^n$ $n$-мер\-ных узлов) по правилу
  $$
  t_{i,j}=\underline{e}_i+\fr{p_i\left( j-1/2\right)}{n_J}\,,\enskip j\in J=\{1,\ldots 
,n_J\}\,,\ i\in I\,,
  $$
порождающая решетки значений дохода и~стои\-мости портфеля соответственно:
\begin{align*}
r_{i,j}&=\phi(t_{i,j})\,,\ j\in J\,,\ i\in I\,,\\
 a_{j_1,\ldots, j_n}& =
\sum\limits_{l\in I} 
c_{l} \phi(t_{l}, j_{l})\,,\enskip 
j_i\in J\,,\enskip i\in I\,.
\end{align*}
  
  Имеет место аппроксимирующее выборочное представление: 
  \begin{equation}
{\sf F}_{\psi;3}(z)\approx \hspace*{-7mm}
  \sum\limits_{\substack{{j_1, j_2, \ldots, j_5\in J;}\\{i\in I}} }
\fr{p_i}{n_J^n}\, {\sf u}\left(z-
\fr{r_{i,j_i}}{\sum\nolimits_{l\in I} 
  c_i r_{l}\,, 
j_{l}}\right),\!\!
  \label{e17-ag}
  \end{equation}
где ${\sf u}(\cdot)$~--- характеристическая функция множества 
положительности аргумента. 

 { \begin{center}  %fig2
 \vspace*{-1pt}
  \mbox{%
 \epsfxsize=77.698mm 
 \epsfbox{aga-2.eps}
 }

\end{center}


\noindent
{{\figurename~2}\ \ \small{Графики функций распределения относительных доходов:
\textit{1}~--- ${\sf F}_{\psi;0}$; \textit{2}~--- ${\sf F}_{\psi;1}$;
\textit{3}~--- ${\sf F}_{\psi;2}$; \textit{4}~--- ${\sf F}_{\psi;3}$}
}

}

\vspace*{12pt}
  
  На рис.~2 изображен график функции ${\sf F}_{\psi;3}(\cdot)$ при $n_J\hm=8$ 
(кривая~\textit{4}) совместно с~графиками функций  
${\sf F}_{\psi;0}(\cdot)$, ${\sf F}_{\psi;1}(\cdot)$ и~${\sf F}_{\psi;2}(\cdot)$.
  
  Аналогично образуются и~формулы приближенного вычисления основных 
числовых показателей~$\psi_3$ и~$A_3$ в~варианте~\#3. Имеем:
  \begin{multline*}
  \mu_m\approx \sum\limits_{\substack{
  {j_1,j_2,\ldots,j_5\in J;}\\ {i\in I}}} \fr{p_i^S}{n_J^n}\, 
\fr{r^m_{i,j_i}}{\left( \sum\nolimits_{l\in I} c_l^S r_{l,j_l}\right)^m}\,,\\
  m=1,2,\enskip  \sigma_3^2=\mu_2-\mu_1^2\,; 
   \end{multline*}
   
   \noindent
   \begin{align*}
   {\sf E} (A_3) &\approx \hspace*{-4.5pt}\sum\limits_{j_1,j_2, \ldots, j_5\in J} 
   \sum\limits_{l\in I} \fr{c_l^S 
r_{l,j_l}}{n_J^n}\,,\enskip n_J=8\,; \ nX=50\,;
   \\
  {\sf E}\left(\fr{1}{A_3}\right)&\approx \hspace*{-4.5pt}
  \sum\limits_{\substack{{j_1,j_2, \ldots,j_5\in J;}\\ {i\in I}}} 
\fr{p_i^S/n_J^n}{\sum\nolimits_{l\in I} c_l^S r_{l,j_l}}\,.
 \end{align*}
  Здесь $\mu_m$~--- $m$-й начальный момент величины~$\psi_3$, ${\sf E}(A_3)$ и~${\sf E}(1/A_3)$~--- средняя инвестиционная сумма и~среднее 
число портфелей, которые можно приобрести на денежную единицу, 
соответственно. Формулы для них получаются очевидными трансформациями 
формулы для выборочного распределения~(\ref{e17-ag}).
   
  Используя их, находим:
  \begin{gather*}
  {\sf E}\psi_3=\mu_1 = 1{,}31971;\enskip
  {\sf E}\psi_3^2 = \mu_2 = 3{,}15321;\\
  y_3 = \mu_1 - 1 = 0{,}31971;\enskip  \sigma_3 = 1{,}18809; \\
  {\sf E}(A_3) \approx 0{,}253541;\enskip {\sf E}\left(\fr{1}{A_3}\right) \approx 4{,}0136; \\
   A_3 = {\sf E}(A_3) \approx 0{,}253541\,. 
  \end{gather*}
  
  Объединяя~(\ref{e14-ag}) и~(\ref{e15-ag}) (ср.~$A_3$), найденные 
значения~$y_3$ и~$\sigma_3$, получаем запись: 
  $$
  \mathbf{J}_3\approx \langle 0{,}253541; 0{,}333446; 0{,}31971; 
1{,}18809\rangle\,. 
  $$
  
  Интересно обратить внимание на то, что при переходе к~относительным 
доходам, как можно усмотреть из рис.~1 и~2, наибольшие искажения графиков, 
отражающие масштаб их растяжения по оси абсцисс, наступают для 
варианта~\#1. 
   
{\small\frenchspacing
 {%\baselineskip=10.8pt
 \addcontentsline{toc}{section}{References}
 \begin{thebibliography}{9}
  \bibitem{1-ag}
  \Au{Markowitz H.} Portfolio selection~// J.~Financ., 1952. Vol.~7. Iss.~1. P.~77--91.
  \bibitem{2-ag}
  \Au{Касимов Ю.\,Ф.} Основы теории оптимального портфеля ценных бумаг.~--- М.: 
Филинъ, 1998. 140~с.
  \bibitem{3-ag}
  \Au{Artzner P., Delbaen~F., Eber~J.-M., Heath~D.} Coherent measures of risk~// Math. 
Financ., 1999. Vol.~9. Iss.~3. P.~203--228.
  \bibitem{4-ag}
  \Au{Agasandian G.\,A.} Optimal behavior of an investor in option market~// Joint Conference 
(International) on Neural Networks. IEEE World Congress on Computational Intelligence.~--- 
Honolulu, Hawaii, 2002. P.~1859--1864. 
  \bibitem{5-ag}
  \Au{Агасандян Г.\,А.} Применение континуального критерия VaR на финансовых 
рынках.~--- М.: ВЦ РАН, 2011. 299~с. 
  \bibitem{6-ag}
  \Au{Крамер Г.} Математические методы статистики~/ Пер. с~англ.~--- М.: Мир, 1975. 
750~с. (\Au{Cramer~H.} Mathematical methods of statistics.~--- Princeton, NJ, USA: Princeton 
University Press, 1946. 575~p.)

 \end{thebibliography}

 }
 }

\end{multicols}

\vspace*{-6pt}

\hfill{\small\textit{Поступила в~редакцию 17.04.17}}

\vspace*{8pt}

%\newpage

%\vspace*{-24pt}

\hrule

\vspace*{2pt}

\hrule

%\vspace*{8pt}


\def\tit{CONTINUOUS VaR-CRITERION 
IN~SCENARIO MARKETS}

\def\titkol{Continuous VaR-criterion 
in~scenario markets}

\def\aut{G.\,A.~Agasandyan}

\def\autkol{G.\,A.~Agasandyan}

\titel{\tit}{\aut}{\autkol}{\titkol}

\vspace*{-9pt}


\noindent
A.\,A.~Dorodnicyn Computing Center, Federal Research Center ``Computer Science and 
Control'' of the Russian Academy of Sciences,  40~Vavilov Str., Moscow 119333, Russian 
Federation 



\def\leftfootline{\small{\textbf{\thepage}
\hfill INFORMATIKA I EE PRIMENENIYA~--- INFORMATICS AND
APPLICATIONS\ \ \ 2018\ \ \ volume~12\ \ \ issue\ 1}
}%
 \def\rightfootline{\small{INFORMATIKA I EE PRIMENENIYA~---
INFORMATICS AND APPLICATIONS\ \ \ 2018\ \ \ volume~12\ \ \ issue\ 1
\hfill \textbf{\thepage}}}

\vspace*{3pt}

 
   
\Abste{The paper investigates problems of using continuous VaR-criterion (CC-VaR) in scenario market as a discrete 
analog of ideal theoretical one-period option market. The participation of an investor in the market supposes that the 
investor prepares a forecast of future underlier's price distribution and sets the risk-preferences function. A~discrete 
optimization algorithm as the result of projecting the theoretical algorithm based on the 
Newman--Pearson procedure 
onto scenario market is suggested. An example of the market with three scenarios, for which the optimality can be 
broken, is adduced. However, such violations occur seldom and are insignificant. To improve the quality of solutions, 
randomization of portfolio weights as remedy of smoothing the distribution function is proposed. Special algorithms for 
calculations connected with yield of randomized portfolios are suggested. The exposition is illustrated by diagrams.} 
   
\KWE{continuous VaR-criterion (CC-VaR); scenario; forecast density; price density; investor's risk-preferences 
function (r.p.f.); optimal portfolio; investment amount; income; yield; randomization} 
  
\DOI{10.14357/19922264180104} 

\vspace*{-14pt}

\Ack
\noindent
The work was supported by the
Russian Foundation for Basic Research (project 17-01-00816).



%\vspace*{-3pt}

  \begin{multicols}{2}

\renewcommand{\bibname}{\protect\rmfamily References}
%\renewcommand{\bibname}{\large\protect\rm References}

{\small\frenchspacing
 {%\baselineskip=10.8pt
 \addcontentsline{toc}{section}{References}
 \begin{thebibliography}{9} 
  \bibitem{1-ag-1}
  \Aue{Markowitz, H.} 1952. Portfolio selection. \textit{J.~Financ.} 7(1):77--91.
  \bibitem{2-ag-1}
  \Aue{Kasimov, Yu.\,F.} 1998. \textit{Osnovy teorii optimal'nogo portfelya tsennykh bumag} 
[Fundamentals of the theory of optimal security portfolio]. Moscow: Filin. 140~p.
  \bibitem{3-ag-1}
  \Aue{Artzner, P., F.~Delbaen, J.-M.~Eber, and D.~Heath.} 1999. Coherent measures of risk. 
\textit{Math. Financ.} 9(3):203--228.
  \bibitem{4-ag-1}
  \Aue{Agasandian, G.\,A.} 2002. Optimal behavior of an investor in option market. \textit{Joint 
Conference (International) on Neural Networks. The IEEE World Congress on Computational 
Intelligence}. Honolulu, Hawaii. 1859--1864. 
  \bibitem{5-ag-1}
  \Aue{Agasandyan, G.\,A.} 2011. Primenenie kontinual'nogo kriteriya VaR na finansovykh 
rynkakh [Application of continuous VaR-criterion in financial markets]. Moscow: CC RAS. 
299~p. 
  \bibitem{6-ag-1}
  \Aue{Cramer, H.} 1946. \textit{Mathematical methods of statistics}. Princeton, NJ: Princeton 
University Press. 575~p.

\end{thebibliography}

 }
 }

\end{multicols}

\vspace*{-9pt}

\hfill{\small\textit{Received April 17, 2017}}

\vspace*{-24pt}
  
  \Contrl
  
   \noindent
   \textbf{Agasandyan Gennady A.} (b.\ 1941)~--- Doctor of Science in physics 
and mathematics, leading scientist, A.\,A.~Dorodnicyn Computing Center, Federal 
Research Center ``Computer Science and Control'' of the Russian Academy of 
Sciences, 40~Vavilov Str., Moscow 119333, Russian Federation; 
\mbox{agasand17@yandex.ru}

   
\label{end\stat}


\renewcommand{\bibname}{\protect\rm Литература} 
   