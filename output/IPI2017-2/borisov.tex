%\newcommand {\ebd}{\stackrel{\triangleq}{=}}
\newcommand {\col}{\mathop{\mathrm{col}}}
\newcommand{\me}[2]{\mathbf{E}_{ #1 }\left\{ \mathop{#2} \right\} }
\newcommand{\pp}[1]{\mathbf{P}\left\{ #1 \right\}}
\newcommand {\ppp}{{\mathcal P}}



\def\stat{borisov}

\def\tit{КЛАССИФИКАЦИЯ ПО НЕПРЕРЫВНЫМ НАБЛЮДЕНИЯМ С~МУЛЬТИПЛИКАТИВНЫМИ ШУМАМИ~II:
АЛГОРИТМ ЧИСЛЕННОЙ РЕАЛИЗАЦИИ ОЦЕНКИ$^*$}

\def\titkol{Классификация по непрерывным наблюдениям с~мультипликативными шумами~II:
алгоритм численной реализации} % оценки}

\def\aut{А.\,В.~Борисов$^1$}

\def\autkol{А.\,В.~Борисов}

\titel{\tit}{\aut}{\autkol}{\titkol}

\index{Борисов А.\,В.}
\index{Borisov A.\,V.}


{\renewcommand{\thefootnote}{\fnsymbol{footnote}} \footnotetext[1]
{Работа выполнена при финансовой поддержке РФФИ (проекты 16-07-00677 
и~15-37-20611-мол\_а\_вед).}}


\renewcommand{\thefootnote}{\arabic{footnote}}
\footnotetext[1]{Институт проблем информатики Федерального исследовательского центра <<Информатика и~управление>> Российской академии наук,
\mbox{aborisov@frccsc.ru}}

\vspace*{12pt}



\Abst{Данная работа является второй частью статьи <<Классификация по непрерывным 
наблюдениям с~мультипликативными шумами~I: формулы байесовской оценки>>, опубликованной
 в~журнале <<Информатика и~её применения>>, 2017, том~11, выпуск~1. 
 Исследования посвящены решению задачи оценивания случайного вектора 
 с~конечным множеством состояний по непрерывным зашумленным наблюдениям. Особенностью 
 модели является то, что интенсивность шумов в~наблюдениях зависит от оцениваемого 
 вектора, что не позволяет применять классические результаты оптимальной нелинейной 
 фильтрации.
В~первой части статьи искомая оценка получена как в~явной интегральной форме, так 
и~в~виде решения некоторой стохастической дифференциальной системы со скачкообразными 
процессами в~правой части. Во второй части представлен алгоритм приближенного 
вычисления оценки и~характеристики его точности. Модельный пример иллюстрирует 
качество предлагаемой оценки и~соответствующей численной процедуры.}

\KW{оптимальная фильтрация; идентифицируемость; рекуррентная схема; 
порядок точ\-ности; дискретизация по времени}

\vspace*{12pt}

\DOI{10.14357/19922264170204} 


\vskip 10pt plus 9pt minus 6pt

\thispagestyle{headings}

\begin{multicols}{2}

\label{st\stat}

 \section{Введение}
 
 Статья является окончанием~\cite{B_17} и~имеет следующую структуру. 
 
 Раздел~2 
 содержит постановку исследуемой задачи классификации, а~также краткое изложение 
 теоретических результатов, представленных в~первой части работы. 
 
 В~разд.~3 предлагается алгоритм приближенного вычисления оценки и~характеристики 
 точности соответствующих аппроксимаций. 
 Раздел~4 содержит модельный пример, 
 иллюстрирующий свойства предлагаемой оценки классификации и~алгоритма ее вычисления. 
 
 Обсуждение результатов и~заключительные комментарии представлены в~разд.~5.

  \section{Постановка задачи и~сводка имеющихся результатов}
  
  Рассматривается следующая система наблюдения:
  
  \noindent
  \begin{equation}
\left.
\begin{array}{l}
\hspace*{-10mm}\displaystyle dX_t = 0\,, \enskip X_0 = X\,;\\[6pt]
\hspace*{-10mm}\displaystyle dY_t = \sum\limits_{n=1}^Ne_n^{\top}X_t f_t(n)\,dt + {}\\[6pt] 
\displaystyle{}+\sum\limits_{n=1}^Ne_n^{\top}X_tg_t(n)\,dW_t\,, \enskip
Y_0=0\,,
\end{array}
\right\}
\label{eq:obsys_1}
\end{equation}
где
  \begin{itemize}
  \item $X_t$ --- ненаблюдаемое состояние системы: начальное условие~$X$ 
  принимает значения из\linebreak множества $\mathbb{S}^N \triangleq \{e_1,\ldots,e_N\}$ 
  единичных векторов евклидова пространства~$\mathbb{R}^N$ 
  с~вероятностями~$\{p_n\}_{n=\overline{1,N}}$, $p\triangleq \col\,(p_1,\ldots,p_N)$;
  \item $Y_t$ --- $M$-мер\-ный процесс наблюдений;
  \item $W_t \in \mathbb{R}^M$~--- независимый от~$X$ векторный 
  стандартный винеровский процесс, характери\-зу\-ющий ошибки наблюдений;
  \item $f_t(n):\;\mathbb{S}^N \times [0,+\infty) \hm\to \mathbb{R}^{M \times 1}$ 
  $(n\hm=\overline{1,N})$~--- набор неслучайных  
  век\-тор-функ\-ций, характеризующий <<план наблюдений>>,
  \item
  $g_t(n):\;\mathbb{S}^N \times [0,+\infty)\hm \to \mathbb{R}^{M \times M}$ 
  $(n\hm=\overline{1,N})$~--- набор равномерно невырожденных неслучайных 
 матричнозначных функций,
характеризующий условную интенсивность шумов в~наблюдениях в~зависимости от 
значения состояния~$X_t$.
  \end{itemize}
  Функции $\{f_t(n)\}_n$ и~$\{g_t(n)\}_n$ непрерывны справа и~имеют конечные пределы слева.

  Через $\mathcal{Y}_t \triangleq \sigma \{Y_s: s \in [0,t] \}$ 
  обозначен естественный поток $\sigma$-ал\-гебр, порожденный наблюдениями~$Y$ 
  до момента~$t$ включительно.

  \textit{Задача байесовской классификации вектора~$X$ по наблюдениям~$Y$, 
  полученным на отрезке времени $[0,T]$, заключается в~нахождении 
  $\widehat{X}_T \triangleq \me{}{X|\mathcal{Y}_T}$.}

  Известно~\cite{S_14}, что поток $\{\mathcal{Y}_t\}_{t \geqslant 0}$ 
  не является непрерывным справа, т.\,е.\ 
  $\displaystyle \bigcap_{s > t}\mathcal{Y}_s \hm\neq \mathcal{Y}_t$, 
  что не позволяет непосредственно применять эффективный аппарат 
  стохастического анализа~\cite{LS_86} для решения поставленной задачи оценивания, 
  а~требует некоторой ее модификации.

    \textit{Задача локального сглаживания вектора~$X$ по наблюдениям~$Y$ 
    заключается в~нахождении $\widehat{X}^+_T \triangleq$\linebreak
    $\triangleq \me{}{X|\mathcal{Y}_{T+}}$.}

    Обозначим
    \begin{equation*}
 G_t(n) \triangleq g_t(n)g_t^{\top}(n)\,,\enskip 
 \mathbf{G}_t(n) \triangleq \int\limits_0^t g_s(n)g^{\top}_s(n)\,ds\,.
  \label{eq:nots_2}
    \end{equation*}
Лемма~1~\cite{B_17} определяет решение задачи байесовской классификации:
\begin{equation}
   \widehat{X}_T(n) = \fr{\widetilde{X}_T(n)}{\sum\nolimits_{\ell=1}^N 
   \widetilde{X}_T(\ell)}\,,
     \label{eq:int_class}
  \end{equation}
  где $\widetilde{X}_T(n)$~---
   ненормированная условная вероятность события $\{\omega:\;X(\omega)=e_n\}$ 
  относительно $\mathcal{Y}_T$:
  \begin{equation}
  \widetilde{X}_T(n)=
   \begin{cases}
  \displaystyle p_n\exp
 \left\{\int\limits_0^{\mathrm{T}}
   \left[
       f_s^{\top}(n)G_s^{-1}(n)\,dY_s -{}\right.\right.\\
        \displaystyle \left.\left.{}- \fr{1}{2}\,\|f_s(n)\|^2_{G_s^{-1}(n)}\,ds
        \right] \vphantom{\int_0^{\mathrm{T}}}
        \right\},  & 
\hspace*{-13mm}\mbox{ если }\\
 \hspace*{7mm}\langle Y,Y\rangle_t \equiv \mathbf{G}_t(n), \;\; t \in [0,T);&\\
 0 & \hspace*{-52mm}\mbox{ в~противном~случае.}
   \end{cases}
  \label{eq:int_class_unnorm}
  \end{equation}
 Здесь $\|x\|_A^2 \triangleq x^{\top}Ax$ и~$|A| \triangleq \det A$.

  Для определения локально сглаженной оценки определим неслучайные 
  моменты $u(\ell,n)$ ($\ell \hm\neq n$):
  \begin{equation*}
  u(\ell,n) \triangleq  \left\{
  \begin{array}{l}
  \displaystyle
  \inf \left\{ t\geqslant 0: \; \mathbf{G}_t(\ell) \neq \mathbf{G}_t(n)\right\}\,; \\
  +\infty, \; \mbox{если $\mathbf{G}_t(\ell) \equiv \mathbf{G}_t(n)$ для $\forall \; t\geqslant 0 $},
  \end{array}
  \right.
  \label{eq:u_def}
  \end{equation*}
  множества $\Xi(n)\triangleq\{u(\ell,n)\}_{\ell: \ell \neq n}$, $n\hm=\overline{1,N}$, 
  и~множество
   $\Xi \triangleq \{u(\ell,n)\}_{{(\ell,n):}\ {\ell \neq n}}$.

   Наблюдения $\{Y_t\}_{t \geqslant 0}$ являются квадратично интегрируемым 
   субмартингалом с~$\mathcal{Y}_t$-пред\-ска\-зу\-емой квадратической характеристикой
  \begin{multline*}
  \langle Y,Y\rangle_t =
  Y_t Y_t^{\top}- \int\limits_0^t Y_s \,dY_s^{\top} - \int\limits_0^t \,dY_s Y_s^{\top}
  ={}\\
  {}=
  \int\limits_0^t \sum\limits_{n=1}^Ne_n^{\top}X_s G_s(n)\,ds\,.
  %\label{eq:sq_char}
  \end{multline*}
  Случайные моменты времени
  \begin{equation*}
  \xi(n) \triangleq  \left\{
  \begin{array}{l}
  \displaystyle
  \inf \left\{ t\geqslant 0: \; \mathbf{G}_t(n) \neq \langle Y,Y\rangle_t \right\}\,; \\[6pt]
  +\infty, \; \mbox{если $\displaystyle \mathbf{G}_t(n) \equiv \langle Y,Y\rangle_t$ для $\; t\geqslant 0 $},
  \end{array}
  \right.
 % \label{eq:chi_def}
  \end{equation*}
  используют~$\Xi(n)$ в~качестве множества возможных значений и~определяют 
  следующие скачкообразные $\mathcal{Y}_{t+}$-со\-гла\-со\-ван\-ные процессы:
    \begin{equation*}
  \mathcal{I}_t(n)
   \triangleq \begin{cases}
   1, &\ \mbox{если } t < \xi(n)\,;\\
   0, &\ \mbox{если } t \geqslant \xi(n)\,.
\end{cases}
  \end{equation*}
Теорема~1~\cite{B_17} определяет локально сглаженную оценку $\widehat{X}^+_t
\hm=\mathcal{I}_t(n)\widehat{X}_t$ как решение стохастической дифференциальной 
системы с~наблюде\-ниями~$Y_t$ и~скачкообразными процессами~$\mathcal{I}_t(n)$ 
в~правой\linebreak \mbox{части} ($n=\overline{1,N}$):
  \begin{multline}
  \widehat{X}_t^+(n) = \fr{p_n\mathcal{I}_{0}(n)}
  {\sum\nolimits_{k=1}^N p_k\mathcal{I}_{0}(k)} +{}\\
  {}+
  \int\limits_0^t \widehat{X}_{s}^+(n)\left(
  f_n^{\top}(s) - \sum\limits_{\ell=1}^N \widehat{X}_{s-}^+(\ell)f_{\ell}^{\top}(s)
  \right)\times{}\\
  {}\times \left( \fr{d\langle Y,Y\rangle_s}{ds}\right)^{-1/2}dZ_s+{} \\
\!{}  + \sum\limits_{s \leqslant t} \widehat{X}^+_{s-}(n)
  \left(
  \fr{1+\Delta\mathcal{I}_{s}(n)}
  {1+\sum\nolimits_{\ell=1}^N \widehat{X}_{s-}^+(\ell)\Delta\mathcal{I}_{s}(\ell)} 
  - 1  \! \right)\!,\!\!
  \label{eq:dif_class_norm}
  \end{multline}
  где $Z_t$~--- обновляющий процесс:
  \begin{equation*}
Z_t \triangleq \int\limits_0^t \! \left(\fr{d\langle Y,Y\rangle_s}{ds}\right)^{-1/2}\!
\left( \!dY_s - \sum\limits_{n=1}^Ne_n^{\top}\widehat{X}_{s-}^+ f_s(n)ds \!\right).
%\label{eq:innov}
\end{equation*}

  \section{Алгоритм вычисления оценок классификации и~его точность}
  
  На первый взгляд, для численного решения сис\-те\-мы~(\ref{eq:dif_class_norm})
  применимы известные алгоритмы~\cite{PB_10}.
Однако в~правую часть системы входят~$\mathcal{Y}_{t+}$-со\-гла\-со\-ван\-ные 
процессы ${d\langle Y,Y\rangle_t}/{dt}$ и~$\{\mathcal{I}_t(n)\}_{n=\overline{1,N}}$.
Они недоступны прямому наблюдению, и~для их вы\-чис\-ле\-ния требуются 
нетривиальные преобразования наблюдений, по сложности эквивалентные решению 
исходной задачи оценивания.

В данной работе предлагается дискретизовать непрерывный процесс и~строить 
оптимальные оценки уже по дискретизованным наблюдениям. Такой подход близок 
по смыслу к~численным методам, предложенным 
в~\cite{PR_10}.
Введем в~рассмотрение последовательность вложенных двоичных разбиений отрезка $[0,T]$, 
порожденных множествами точек~$\mathcal{T}^K$, $K\hm \in \mathbb{N}$:
$$
  \mathcal{T}^K \triangleq \{\tau_k^K\}_{k=\overline{0,2^K}}: \enskip
  \tau_k^K \triangleq kh_K\,, \enskip h_K \triangleq \fr{T}{2^K}\,,
 $$
  соответствующие наборы дискретизованных наблюдений:
  \begin{multline*}
  \Delta Y^K_k \triangleq Y_{\tau_k^K}-Y_{\tau_{k-1}^K} =
  \int\limits_{\tau_{k-1}^K}^{\tau_{k}^K}\sum\limits_{n=1}^Ne_n^{\top}X_s f_s(n)\,ds +{}\\
  {}+  \int\limits_{\tau_{k-1}^K}^{\tau_{k}^K}\sum\limits_{n=1}^Ne_n^{\top}X_sg_s(n)\,
  dW_s\,, \enskip k=\overline{1,2^K}\,,
  %\label{eq:discr_obs}
  \end{multline*}
  и~семейство вложенных $\sigma$-ал\-гебр
  $\{\mathcal{Y}^K\}_{K \in \mathbb{N}}$: $\mathcal{Y}^K \triangleq$\linebreak $\triangleq \sigma \{\Delta Y^K_k, 
  \; k=\overline{1,2^K}\}$. Так как процесс~$Y_t$ является сепарабельным, то 
  $\mathcal{Y}^K \uparrow \mathcal{Y}_T$ при $K \hm\to \infty$ и~по теореме
   Леви~\cite{LS_74}
  $$
  \widehat{X}^K \triangleq \me{}{X |\mathcal{Y}^K} \to \me{}{X |\mathcal{Y}_T} = 
  \widehat{X}_T \enskip\mbox{$\ppp$-п.~н.}
  $$

  Тогда компоненты $\widehat{X}^K(n) \triangleq \pp{X=e_n|\mathcal{Y}^K}$ 
  вектора~$\widehat{X}^K$ определяются формулами:
    \begin{equation}
    \widehat{X}^K(n) = \fr{\widetilde{X}^K(n)}
  {\sum\nolimits_{\ell=1}^N \widetilde{X}^K(\ell)
  }
  \label{eq:discr_class}
  \end{equation}
  и
    \begin{multline}
  \widetilde{X}^K(n) \triangleq p_n\exp\left\{
  -\fr{1}{2h_K}\sum\limits_{k=1}^{2^K}\left[
  \vphantom{\left\|\Delta Y^K_k-h_K F_k^K(n)\right\|^2_{
  \left(G_k^K(n)\right)^{-1}}}
  \ln\left\vert G_k^K(n)\right\vert h_K +{}\right.\right.\\
\left.\left.  {}+ \left\|\Delta Y^K_k-h_K F_k^K(n)\right\|^2_{
  \left(G_k^K(n)\right)^{-1}}
  \right]
  \vphantom{\sum\limits_{k=1}^{2^K}}
  \right\},
 \label{eq:discr_class_22}
  \end{multline}
где
\begin{align*}
  F_k^K(n) &
  \triangleq     \fr{1}{h_K}  \int\limits_{\tau_{k-1}^K}^{\tau_{k}^K}f_s(n)\,ds\,, 
  \\
  G_k^K(n) &\triangleq \fr{1}{h_K} \int\limits_{\tau_{k-1}^K}^{\tau_{k}^K}g_s(n)g_s^{\top}(n)\,ds\,.
  \end{align*}
%  \label{eq:FG}

Для упрощения выкладок будем дополнительно считать, что существует такое 
целое число~$K^*$, что $\Xi \backslash \{+\infty\} \subset \mathcal{T}^K$ 
для всех  $K \hm> K^*$.
Формулы~(\ref{eq:discr_class}) и~(\ref{eq:discr_class_22}) могут быть реализованы 
с~по\-мощью сле\-ду\-ющей рекуррентной схемы ($n=\overline{1,N}$):
\begin{equation}
\left.
\begin{array}{l}
\widehat{\varkappa}_0^K(n) = p_n\,;\\[6pt]
\widetilde{\varkappa}_{\tau_{k}^{K}}^K(n) = \widehat{\varkappa}_{\tau_{k-1}^{K}}^K(n)
\left|G_k(n)\right|^{-1/2}\times{}\\
\hspace*{-1.5mm}{}\times
\exp \displaystyle\left\{\! -\fr{1}{2h_K}\|\Delta Y_k^K-h_K F_k^K(n)\|^2_{(G_k^K(n))^{-1}}
\right\};\\[6pt]
\displaystyle \widehat{\varkappa}_{\tau_{k}^{K}}^K(n) = 
\fr{\widetilde{\varkappa}_{\tau_{k}^{K}}^K(n)}{\sum\nolimits_{\ell=1}^N \widetilde{\varkappa}_{\tau_{k}^{K}}^K(\ell)}\,, \enskip
\tau_{k}^{K} \in \mathcal{T}^K\,.
\end{array}\!
\right\}\!\!
\label{eq:sch_norm_2}
\end{equation}

\vspace*{-2pt}

\noindent
\textbf{Теорема~1.}\
\textit{Набор вектров $\{\widehat{\varkappa}^K_t\}_{t \in \mathcal{T}^K}$ 
является аппроксимацией оценки~$\widehat{X}_t$ на сетке~$\mathcal{T}^K$ 
$\left(\widehat{X}_{t}\hm \approx \widehat{\varkappa}^K_{t},\ t \hm\in \mathcal{T}^K\right)$. 
Аппроксимация обеспечивает точность порядка~${1}/{2}$, т.\,е.}
  \begin{equation}
  \sqrt{\me{}{\|\widehat{\varkappa}^K_{t}-\widehat{X}_{t}\|^2}} = O\left(h_K^{1/2}\right) \enskip 
  \forall \; \tau_k^K \in \mathcal{T}^K.
  \label{eq:approx_ord}
  \end{equation}

\noindent
Д\,о\,к\,а\,з\,а\,т\,е\,л\,ь\,с\,т\,в\,о\ \  \ теоремы~1 приведено в~при\-ложении.

\smallskip

Порядок аппроксимации ${1}/{2}$ является обычным для традиционных схем 
численного решения стохастических дифференциальных систем с~винеровскими 
и~пуассоновскими процессами в~правой\linebreak части~\cite{PB_10}.
Однако <<локальный порядок точности>> аппроксимации $\widehat{\varkappa}_t^K(n)$ 
компоненты~$\widehat{X}_t(n)$ сразу после момента скачка~$\mathcal{I}_t(n)$ 
абсолютно другой.

\smallskip

\noindent
\textbf{Теорема~2.}\ \textit{Пусть множество~$\Xi(n)$ возможных значений 
момента~$\xi(n)$ и~значение временн$\acute{\mbox{о}}$го лага $H_K \hm> 0$ удовлетворяют следующим 
условиям}:
\begin{itemize}
\item[(а)]
$\max\{u:\; u \in \Xi(n)\} < + \infty$;
\item[(б)]
$\Xi(n) \subseteq \mathcal{T}^{K^*}$, $\{t+H_k\}_{t \in \Xi(n)}\subseteq 
\mathcal{T}^{K^*}$ \textit{для некоторого} $K^* \hm\in \mathbb{N}$;
\item[(в)]
\textit{неравенство}
\begin{equation*}
\!\!\!\!\!\!\!\min\limits_{t \in \Xi(n)}\!\!\!\!\int\limits_{t}^{t+H_K}\left[
\ln\!\fr{|G_s(n)|}{|G_s(\ell)|}
+
\mathrm{tr}\,
\left(
G_s(\ell)
G_s^{-1}(n)
\right)
\right]\,ds \geqslant C_1
\end{equation*}
\textit{выполняется для некоторой константы} $C_1\hm>0$.
\end{itemize}
\textit{Тогда аппроксимация $\widehat{\varkappa}^K_{\xi(n) + H_K}$ 
оценки $\widehat{X}_{\xi(n)+H_k}$ имеет экспоненциальную точность, т.\,е.}
  \begin{multline*}
 \left( \mathbf{E}\left\{\left(\widehat{\varkappa}^K_{\xi(n) + H_K}(n)-\widehat{X}_{\xi(n) + 
  H_K}(n)\right)^2\times{}\right.\right.\\
\left.\left.{}\times  \mathbf{I}_{\{\xi(n)< + \infty\}}(\omega)
\vphantom{\left(\widehat{\varkappa}^K_{\xi(n) + H_K}(n)-\widehat{X}_{\xi(n) + 
  H_K}(n)\right)^2}
  \right\}
\right)^{1/2} 
  \leqslant C_2 \exp\left(-{\fr{C_1}{2h_K}} \right)
  \label{eq:approx_ord_2}
  \end{multline*}
 \textit{для некоторого $C_2>0$.}
 
\pagebreak

  \textit{Если дополнительно}
  \begin{equation*}  %\substack{{i=\overline{1,n}}\\ {j=\overline{1,l}}}
  \min\limits_{\substack{{(s,t):\; t \in \Xi(n),}\\
{s \in (t, t+H_K]}}}
 \left[ \ln\fr{|G_s(n)|}{|G_s(\ell)|}
+\mathrm{tr}\,
\left(
G_s(\ell)
G_s^{-1}(n)\right)\right] \geqslant C_3
%\label{eq:neq_1}
  \end{equation*}
  \textit{для некоторого $C_3>0$,
   то}
    \begin{multline*}
\left(\mathbf{E}\left\{\left(\widehat{\varkappa}^K_{\xi(n) + H_K}(n)-\widehat{X}_{\xi(n) + 
  H_K}(n)\right)^2\times{}\right.\right.\\
 \left. \left.{}\times
  \mathbf{I}_{\{\xi(n)< + \infty\}}(\omega)
  \vphantom{\left(\widehat{\varkappa}^K_{\xi(n) + H_K}(n)-\widehat{X}_{\xi(n) + 
  H_K}(n)\right)^2}
  \right\}
\right)^{1/2} \leqslant  C_2 \exp\left(-{\fr{C_3H_K}{2h_K}} \right)\,.
  %\label{eq:approx_ord_3}
  \end{multline*}

\noindent
Д\,о\,к\,а\,з\,а\,т\,е\,л\,ь\,с\,т\,в\,о\ \ теоремы~2 дано в~прило-\linebreak жении.

Так как оценки $\{\widehat{X}_t\}_{t \geqslant 0}$ 
и~$\{\widehat{X}_t^+\}_{t \geqslant 0}$
различаются не более чем на множестве~$\Xi$
потенциальных моментов скачков,
численная схема~(\ref{eq:sch_norm_2})
может быть использована также и~для вычисления локально сглаженной 
оценки~$\widehat{X}^+$. Действительно, ввиду тео\-ре\-мы~2,
аппроксимации оценок классификации~(\ref{eq:sch_norm_2})
могут быть легко преобразованы в~аппроксимации оценок сглаживания с~фиксированным 
лагом путем очевидного переопределения:
$$
\widehat{X}_{\tau_k^K}^+ \approx \widehat{\varkappa}_{\tau_k^K+H_K}^K\,,
$$
где $H_K$~--- величина лага. Процедура локального сглаживания позволит значительно 
увеличить точность аппроксимации в~точках разрыва 
решения при малых абсолютных значениях временн$\acute{\mbox{о}}$го лага~$H_K$.
Если выполнены условия второй час\-ти теоремы~2, то лаг следует выбирать из 
условия: 
$$
\lim\limits_{K \to +\infty} \fr{H_K}{h_K}=+\infty\,.
$$

%\vspace*{-9pt}

  \section{Численный пример}
  
  Данный численный пример является абстрактным, однако он иллюстрирует 
  свойства предложенной оценки и~алгоритма ее аппроксимации.

  Рассматривается система наблюдения~(\ref{eq:obsys_1}) со следующими параметрами:
 $N\hm=4$; $M\hm=1$; $T \hm= 1$; $p=\mathrm{col}({1}/{4},{1}/{4},{1}/{4},{1}/{4})$;
 \begin{equation*}
\begin{array}{l}
f_t(1) = 1\cdot\mathbf{I}_{[0,{1}/{4})}(t) +
1\cdot\mathbf{I}_{[{1}/{2},{3}/{4})}(t)\,;
\\[6pt]
f_t(2) = 2\cdot\mathbf{I}_{[0,{1}/{4})}(t) +
2\cdot\mathbf{I}_{[{1}/{2},{3}/{4})}(t)\,;
\\[6pt]
f_t(3) = 3\cdot\mathbf{I}_{[0,{1}/{4})}(t) +
3\cdot\mathbf{I}_{[{1}/{2},{3}/{4})}(t)\,;
\\[6pt]
f_t(4) = 4\cdot\mathbf{I}_{[0,{1}/{4})}(t) +
4\cdot\mathbf{I}_{[\frac{1}{2},{3}/{4})}(t)\,;\\[6pt]
g_t(1) = 1\cdot\mathbf{I}_{[0,{3}/{4})}(t) + 
1{,}1\cdot\mathbf{I}_{[{3}/{4},{1}/{2}]}(t)\,;
\\[6pt]
g_t(2) = 1\cdot\mathbf{I}_{[0,{3}/{4})}(t) + 1{,}2\cdot\mathbf{I}_{[{3}/{4},{1}/{2}]}
(t)\,; \\[6pt]
g_t(3) = 1\cdot\mathbf{I}_{[0,{3}/{4})}(t) + 1{,}3\cdot\mathbf{I}_{[{3}/{4},{1}/{2}]}(t)\,;
\end{array}
\end{equation*}

 { \begin{center}  %fig1
 \vspace*{1pt}
 \mbox{%
\epsfxsize=77.118mm
\epsfbox{bor-1.eps}
}


\vspace*{9pt}


\noindent
{{\figurename~1}\ \ \small{Интенсивности шумов $g_t(n)$ ($n=\overline{1,4}$)}}
\end{center}
}

\vspace*{6pt}

\addtocounter{figure}{1}


\noindent
\begin{multline*}
g_t(4) = 1{,}0\cdot\mathbf{I}_{[0,{1}/{2})}(t) + (t-{1}/{4})\cdot\mathbf{I}_{[{1}/{4},
{1}/{2})}(t) +{}\\
{}+
1{,}0\cdot\mathbf{I}_{[{1}/{2},{3}/{4})}(t)
+ 1{,}4\cdot\mathbf{I}_{[{3}/{4},1]}(t)\,.
\end{multline*}
Графики $\{g_n\}$ приведены на рис.~1.



Точки разрывности потока $\sigma$-ал\-гебр~$\mathcal{Y}_t$ образуют множество 
$\Xi\hm=\left\{{1}/{4},{1}/{2},{3}/{4}\right\}$.
Система наблюдения такова, что компонента~$X(4)$ может быть точно восстановлена 
по~$\mathcal{Y}_{({1}/{4})+}$,
в~то время как весь вектор~$X$ может быть точно восстановлен 
по~$\mathcal{Y}_{({3}/{4})+}$. При этом в~момент $t\hm={1}/{4}$ для компоненты~$X(4)$ 
выполняется первое условие идентифицируемости теоремы~2, а~в~момент 
$t\hm={3}/{4}$ для оставшихся компонент~--- второе условие идентифицируемости.

%Далее, на интервалах $(0,{1}/{4})$ и~$({1}/{2},{3}/{4})$ интенсивность шумов
 %в~наблюдениях не зависит от состояния~$X$.
%Это классический случай аддитивных шумов. 
Еще одной из целей данного примера является 
сравнение точности фильтрации при наличии аддитивных и~мультипликативных шумов.

Наконец, в~примере компоненты сноса~$\{f_n\}$ значительно отличаются друг 
от друга (до~400\%), в~то время как компоненты диффузии~$\{g_n\}$ варьируются 
значительно меньше (до~40\%).

Результаты примера демонстрируют, что даже малые различия интенсивностей шумов 
в~наблюдениях
обеспечивают быстрое и~точное восстановление вектора~$X$.

Аппроксимации $\widehat{\varkappa}_t^K$ были вычислены с~по\-мощью 
алгоритма~(\ref{eq:sch_norm_2}) при разных значениях шага дискретизации 
$h_K\hm=2^{-K}$ ($K\hm=10$, 12, 14, 16).
Результаты оценивания представлены на рис.~\ref{pic:pic_2}. В~этом эксперименте 
$X\hm=e_1$. Графики демонстрируют сходимость $\widehat{\varkappa}_t^K \hm\to 
\widehat{X}_t$ при $K\hm \to +\infty$.


Точность оценок фильтрации и~их численной реализации представлены на 
рис.~3.

На нем приведены графики среднеквадратических отклонений (СКО)
$\widehat{\sigma}^{16}_t$  ошибок оценивания компонент~$X(n)$, вычисленных 
при шаге дискретизации $h_K\hm=2^{-16}$. Значения СКО рассчитываются
 с~помощью 
метода Мон\-те Кар\-ло путем осредне-\linebreak\vspace*{-12pt}

\pagebreak

\end{multicols}

\begin{figure*} %fig2
%\includegraphics[width=1.0\columnwidth]{pic_2.pdf}
\vspace*{1pt}
\begin{center}
\mbox{%
\epsfxsize=146.393mm
\epsfbox{bor-2.eps}
}
\end{center}
\vspace*{-9pt}
\Caption{Оценки $\widehat{\varkappa}^K_t(n)$, вычисленные при $h_K=2^{-K}$:
\textit{1}~--- $K\hm=10$;
\textit{2}~--- 12; \textit{3}~--- 14; \textit{4}~--- $K\hm=16$} 
\label{pic:pic_2}
\vspace*{5pt}
\end{figure*}


\begin{multicols}{2}


{ \begin{center}  %fig3
 \vspace*{1pt}
 \mbox{%
\epsfxsize=77.371mm
\epsfbox{bor-3.eps}
}
\end{center}

\vspace*{-3pt}


\noindent
{{\figurename~3}\ \ \small{Среднеквадратические отклонения~$\widehat{\sigma}^{16}_t(n)$ 
ошибок оценок $\widehat{X}_t(n)$: \textit{1}~--- $n\hm=1$;
\textit{2}~--- 2; \textit{3}~--- 3; \textit{4}~--- $n\hm=4$}}
}

\vspace*{12pt}

\addtocounter{figure}{1}
 


\noindent
ния~1000~траекторий.
Графики зависимости СКО от
 времени подтверждают представленные теоретические 
результаты. Иследуемая система наблюдения сконструирована таким 
образом, что на интервалах времени $(0,{1}/{4})$ и~$({1}/{2},{3}/{4})$ 
интенсивности шумов в~наблюдениях одинаковы, а различаются только коэффициенты сноса.
 Это классический случай
   наблюдений с~аддитивными шумами. 
   
   Из графиков можно 
 заключить, что в~данном примере на интервалах аддитивных шумов точность 
 оценивания
 с~ростом времени наблюдения меняется незначительно: не более 
 чем на~10\% за интервал наблюдения длиной~${1}/{4}$.
В~то же время после момента $t\hm={1}/{4}$ компонента~$X(4)$ идентифицируется точно 
по наблюдениям, полученным на интервале времени длиной менее~0,12. 
Оставшиеся компоненты точно идентифицируются после момента $t\hm={3}/{4}$ 
за время не более~0,06. Причиной такой точности оценивания является наличие
 мультипликативных шумов и~выполнение условий иден\-ти\-фи\-ци\-ру\-емости. Компоненты~$X$ 
 не идентифицируются точно в~моменты~${1}/{4}$ и~${3}/{4}$ из-за того, что в~примере 
 используются аппроксимации оценок, а~не их точные значения. При этом время точной 
 идентификации компоненты~$X(4)$ больше, чем время идентификации остальных компонент. 
 Это связано с~тем, что в~случае идентификации~$X(4)$ выполняются мягкие 
 условия идентифицируемости, определенные первой частью теоремы~2, 
 а~для остальных компонент~--- более ограничительные, но <<эффективные>> для 
 идентификации условия второй части теоремы.

Одной из целей рассматриваемого примера являлась проверка 
порядка точ\-ности ${1}/{2}$ численной аппроксимации~(\ref{eq:sch_norm_2}). 
Согласно следствию~1~\cite{B_17},  $\widehat{X}^+_t\hm=X$ $\ppp$-п.~н.\ 
для любого $t \hm> {3}/{4}$. В~качестве временн$\acute{\mbox{о}}$го лага была выбрана 
величина $H\hm=2^{-7}$. Поэтому для исследования порядка точности был выбран 
момент времени $t^*\hm={3}/{4}+{1}/{2^7}$. Среднеквадратическая ошибка 
аппроксимации (СКОА) $\Delta_{t^*}^K(n) \triangleq \me{}{(\widehat{\varkappa}^K_{t^*}
(n)-X(n))^2}$ компонент вектора вычислялась методом Мон\-те Кар\-ло 
путем осреднения пучка~1000~траекторий.
Согласно теореме~1 $\Delta_{t^*}^K(n)\hm \leqslant C_4 h_K$ для некоторого $C_4\hm>0$ 
и~любого $K \hm> K^*$.
Так как $h_K\hm=2^{-K}$, то $\log_2\Delta_{t^*}^K(n) \hm\leqslant \log_2 C_4 \hm- K$. 
Рисунок~4 демонстрирует зависимость $\log_2\Delta_{t^*}^K$
 от~$K$ ($K\hm=\overline{7,20}$) для всех ком понент вектора.
Графики зависимостей являются
 вогнутыми, что соответствует утверждению теоремы~1, 
и~демонстрируют консервативный характер
 оценки~(\ref{eq:approx_ord}): 
порядок аппроксимации, полученный в~рассмотренном примере, выше чем~${1}/{2}$.

   { \begin{center}  %fig4
 \vspace*{18pt}
 \mbox{%
\epsfxsize=78.033mm
\epsfbox{bor-4.eps}
}
\end{center}

%\vspace*{-3pt}


\noindent
{{\figurename~4}\ \ \small{Зависимость СКОА $\Delta_{t^*}^K$ от $K$:
\textit{1}~--- $n\hm=1$;
\textit{2}~--- 2; \textit{3}~--- 3; \textit{4}~--- $n\hm=4$}}
}

%\vspace*{12pt}

\addtocounter{figure}{1}
 
  
% \noindent



  \section{Заключение}

  Вторая часть статьи посвящена разработке чис\-лен\-но\-го алгоритма, реализующего 
  решение за\-да\-чи байесовской классификации. 
  
  Предлагается отказаться от 
  непосредственной численной ап\-прок\-си\-ма\-ции решения стохастической дифференциальной 
  системы, определяющей искомую оценку классификации, в~пользу по\-стро\-ения 
  байесовской оценки по дискретизованным наблюдениям. Помимо формул, 
  определяющих чис\-лен\-ную схему аппроксимации искомой оценки, в~работе также 
  исследована точность этой аппроксимации. Предложенный численный пример 
  иллюстрирует свойства оценки и~качество ее аппроксимаций.

  Полученные в~обеих частях работы теоретические результаты и~схемы 
  численной ап\-прок\-си\-мации могут рассматриваться как часть алгоритмического 
  обеспечения систем информатики при реше\-нии задач ве\-ро\-ят\-ност\-но-ста\-ти\-cти\-че\-ско\-го 
  экс\-пресс-мо\-де\-ли\-ро\-ва\-ния и~оценивания по вы\-со\-ко\-час\-тотным данным.

 {\small \section*{\raggedleft Приложение}


\noindent
Д\,о\,к\,а\,з\,а\,т\,е\,л\,ь\,с\,т\,в\,о\ \ 
теоремы~1.\ %\ \begin{proofoftheorem}{\ref{th:thm3}}
Доказательство теоремы опирается на формулы~(\ref{eq:int_class}), 
(\ref{eq:int_class_unnorm}), (\ref{eq:discr_class}) и~(\ref{eq:discr_class_22}).
 Заметим, что на промежутке времени $(0,\xi(n))$ 
 производная ${d\langle Y,Y\rangle_t}/{dt}$ существует почти всюду по мере Лебега
  и~также почти всюду верно равенство
 \begin{equation*}
 \fr{d\langle Y,Y\rangle_t}{dt} \equiv G_t(n)\,.
 %\label{eq:qch}
 \end{equation*}
 Введем следующие обозначения:
 \begin{align*}
 \psi(n) &\triangleq \displaystyle \int\limits_0^{\mathrm{T}} \left[ \ln|G_s(n)|+ \mathrm{tr}\left( 
 G_s(\omega) G_s^{-1}(n)\right)\right]\,ds\,;\\
 \displaystyle
 \eta(n) &\displaystyle\triangleq -\int\limits_0^{\mathrm{T}} \left[
\fr{1}{2}\|f_s(n)\|^2_{
  G_s^{-1}(n)}ds - f_s^{\top}(n)G_s^{-1}(n)\,dY_s
  \right]\,.
% \label{eq:eqpr_1}
 \end{align*}

 Зафиксируем $n \in \{1,\ldots,N\}$ и~построим множества
 \begin{align*}
 S_T &\triangleq \{\ell: \; 1\leqslant \ell \leqslant N:\; 
 \mathbf{G}_t(\ell)\equiv \langle Y,Y\rangle_t, \; t \in [0,T] \}; \\
 \overline{S}_T &\triangleq \{1,\ldots,N\} \backslash S_T\,.
% \label{eq:not_2}
 \end{align*}
 Очевидно, что $S_T \hm\neq \varnothing$, так как гарантированно $n \hm\in S_T$. 
 Из определения случайного момента $\xi(n)$ следует, что $\{\omega:\; \xi(n) 
 \hm\geqslant T\} \hm=\{\omega:\; X(\omega)\hm=e_q,\; q \hm\in S_T\}$. При этом 
 из доказательства леммы~1~\cite{B_17} следует, что 
 на множестве $\{\omega:\; X(\omega)\hm=
 e_q,\; q\hm\in S_T\}$
 $\psi(\ell)\hm - \psi(q)\hm \equiv 0$ $\forall \; \ell \hm\in S_T$ и~$\psi(j)\hm - 
 \psi(q) \hm> 0$ $\forall \; j\hm \in \overline{S}_T$.

 Далее без ограничения общности для упрощения выкладок будем считать, 
 что 
 $$\pp{\omega:\; \xi(n) < T}>0\,.
 $$
 Оценим сверху средний квадрат ошибки аппроксимации 
 $I^K(n) \triangleq \me{}{(\widehat{X}^K(n)-\widehat{X}(n))^2}$:
 \begin{multline*}
 I^K(n) = \sum\limits_{q \in S_T}
 \me{}{(\widehat{X}^K(n)-\widehat{X}(n))^2\mathbf{I}_{\{X=e_q\}}(\omega)} + {}
 \\ 
 {}+\sum\limits_{j \in \overline{S}_T}\me{}{(\widehat{X}^K(n)-
 \widehat{X}(n))^2\mathbf{I}_{\{X=e_j\}}(\omega)}.
 %\label{eq:eqpr_3}
 \end{multline*}
 Сначала для фиксированного $q \hm\in S_T$ рассмотрим на множестве 
 $\{\omega: \; X(\omega)\hm=e_q\}$ случайную величину $|\widehat{X}^K(n)\hm-
 \widehat{X}(n)|\mathbf{I}_{\{X=e_q\}}(\omega)$, учитывая, что на этом множестве
 \begin{equation*}
 \widehat{X}(n) = \fr{\widetilde{X}(n)}{\sum\nolimits_{\ell \in S_T}\widetilde{X}(\ell)} =
 \fr{p_n e^{\eta(n)}}{\sum\nolimits_{\ell \in S_T}p_{\ell} e^{\eta(\ell)}}\,.
% \label{eq:eqpr_4}
 \end{equation*}
 Тогда верна следующая цепочка неравенств:
 \begin{multline*}
 \left\vert \widehat{X}^K(n)-\widehat{X}(n)\right\vert
 \mathbf{I}_{\{X=e_q\}}(\omega) ={}\\[1pt]
 {}= \widehat{X}(n)\mathbf{I}_{\{X=e_q\}}(\omega)
 \left|\fr{\widehat{X}^K(n)}{\widehat{X}(n)} -1\right| = {}
 \\[1pt]
 {}=
 \widehat{X}(n)\mathbf{I}_{\{X=e_q\}}(\omega)
 \left|\fr{\widetilde{X}^K(n)\sum\nolimits_{\ell \in S_T} \widetilde{X}(\ell)}
 {\sum\nolimits_{m=1}^N\widetilde{X}^K(m)\widetilde{X}(n) } -1\right| ={}\\[1pt]
 {}=
 \widehat{X}(n)\mathbf{I}_{\{X=e_q\}}(\omega)
  \left(\left|\sum\limits_{\ell \in S_T}\left( 
 \fr{\widetilde{X}^K(n)}{\widetilde{X}(n)}\,
 \widetilde{X}(\ell)- {}\right.\right.\right.\\[1pt]
\left.\left.\left. {}-\widetilde{X}^K(\ell)\right)-\sum\limits_{j \in 
 \overline{S}_T}\widetilde{X}^K(j)\right|\right)\Bigg /
 \sum\limits_{m=1}^N\widetilde{X}^K(m) \leqslant{}\\[1pt]
 {} \leqslant
 \mathbf{I}_{\{X=e_q\}}(\omega)
 \left(\left|\sum\limits_{\ell \in S_T}\left( 
 \fr{\widetilde{X}^K(n)}{\widetilde{X}(n)}\,\widetilde{X}(\ell)-{} \right.\right.\right.\\[1pt]
\left.\left.\left.{}- \widetilde{X}^K(\ell)\right)-
\sum\limits_{j \in \overline{S}_T}
 \widetilde{X}^K(j)\right| \right) \Bigg / 
 \widetilde{X}^K(q) \leqslant{} \\[1pt]
 {}\leqslant
 \mathbf{I}_{\{X=e_q\}}(\omega) \left[
 \sum\limits_{\ell \in S_T}\left|
 \fr{\widetilde{X}^K(n)}{\widetilde{X}^K(q)}\,
 \fr{\widetilde{X}(\ell)}{\widetilde{X}(n)}-
 \fr{\widetilde{X}^K(\ell)}{\widetilde{X}^K(q)}
 \right| +{}\right.\\[1pt]
 \left. {}+ \sum\limits_{j \in \overline{S}_T} 
 \fr{\widetilde{X}^K(j)}{\widetilde{X}^K(q)}
 \right] = {}\\[1pt]
 \! \!{}=
 \mathbf{I}_{\{X=e_q\}}(\omega)\! \left(\sum\limits_{\ell \in S_T}\!\!
 e^{\eta(\ell)-\eta(q)-\phi^K(q)}\left\vert e^{\phi^K(n)}-e^{\phi^K(q)}\right\vert +{}\right.\hspace*{-3.99pt}\\[1pt]
  \left. {}+
 \!\sum\limits_{j \in \overline{S}_T}\!
 e^{-(({\psi(j)-\psi(q)})/({2h_K}))+\eta(j)-\eta(q)+\phi^K(j)-\phi^K(q)}\right) 
 \leqslant{} \hspace*{-1.65pt}
% \label{eq:eqpr_5}
 \\[1pt]
 \hspace*{-8.60881pt}{}\leqslant
 \mathbf{I}_{\{X=e_q\}}(\omega) \!\left(\!\sum\limits_{\ell \in S_T}
 e^{\eta(\ell)-\eta(q)-\phi^K(q)}\left\vert
 e^{\phi^K(n)}-e^{\phi^K(q)}\right\vert +{}\right.\\[1pt]
\left. {}+
 e^{-({\mu(q)}/({2h_K}))}
 \sum\limits_{j \in \overline{S}_T}
 e^{\eta(j)-\eta(q)+\phi^K(j)-\phi^K(q)}\right),
 %\label{eq:eqpr_6}
 \end{multline*}
 где
 \begin{equation*}
 \mu(q) \triangleq \min\limits_{j \in \overline{S}_T} \int\limits_0^T \left[ 
 \ln \fr{|G_s(j)|}{|G_s(q)|}
 + \mathrm{tr}\left (G_s(q)G_s^{-1}(j)\right)
 \right]ds > 0\,.
 %\label{eq:eqpr_7}
 \end{equation*}
Тогда $\forall \; q \in S_T$
\begin{multline*}
\me{}{(\widehat{X}^K(n)-\widehat{X}(n))^2\mathbf{I}_{\{X=e_q\}}(\omega)} 
\leqslant {}\\[1pt] 
{}\leqslant
N \sum\limits_{\ell \in S_T} \mathbf{E}\left\{
\vphantom{\left(e^{\phi^K(n)}-e^{\phi^K(q)}\right)^2}
\mathbf{I}_{\{X=e_q\}}
(\omega)e^{2(\eta(\ell)-
\eta(q)-\phi^K(q))}\times{}\right.\\[1pt]
\left.{}\times\left(e^{\phi^K(n)}-e^{\phi^K(q)}\right)^2\right\}+
Ne^{-{\mu(q)}/{h_K}}\times{}\\[1pt]
{}\times
\sum\limits_{j \in \overline{S}_T}\me{}{\mathbf{I}_{\{X=e_q\}}
(\omega)e^{2(\eta(j)-\eta(q)+\phi^K(j)-\phi^K(q))}}.
%\label{eq:eqpr_7}
 \end{multline*}
 При условии $X(\omega)=e_q$ величины 
 $\{\phi^K(\ell)\}_{\ell=\overline{1,N}}$ имеют гауссовское распределение. 
 Можно показать, что в~случае, когда $\me{}{(\phi^K(\ell))^2}\hm=O(h_K)$, 
 оценка $\me{}{(\phi^K(\ell)\hm-\phi^K(m))^4}\hm=O(h_K^2)$ верна для 
 любых $\ell,m \hm\in \{1,\ldots,N\}$ и~любой зависимости $\phi^K(\ell)$ 
 и~$\phi^K(m)$. При этом случайные величины $\{e^{4(\eta(\ell)\hm-
 \eta(q)-\phi^K(q))}\}_{\ell \in S_T}$ и~$\{e^{2(\eta(j)\hm-
 \eta(q)+\phi^K(j)-\phi^K(q))}\}_{j \in \overline{S}_T}$ имеют 
 логнормальное распределение с~ограниченными по~$\ell$ и~$j$ математическими 
 ожиданиями. Одновременно с~этим функция $e^{-({\mu(q)}/{h_K})}$ стремится к~0 
 при~$h_K \hm\to 0$ быстрее любой степени~$h_K$. Используя все вышесказанное, 
 а~также неравенство Ко\-ши--Бу\-ня\-ков\-ско\-го, 
 можно построить следующую цепочку неравенств:
  \begin{multline}
\me{}{(\widehat{X}^K(n)-\widehat{X}(n))^2\mathbf{I}_{\{X=e_q\}}(\omega)} 
\leqslant {}\\[1pt] 
{}\leqslant
N \sum\limits_{\ell \in S_T}
\sqrt{
\me{}{\mathbf{I}_{\{X=e_q\}}(\omega)e^{4(\eta(\ell)-\eta(q)-\phi^K(q))}}} 
C_{nq}h_K+{}\\[1pt]
{}+
Ne^{-({\mu(q)}/{h_K})}\times{}\\[1pt]
{}\times \sum\limits_{j \in \overline{S}_T}\me{}{\mathbf{I}_{\{X=e_q\}}
(\omega)e^{2(\eta(j)-\eta(q)+\phi^K(j)-\phi^K(q))}} \leqslant{}\\[1pt]
{}\leqslant Q_{nq}h_K
\label{eq:eqpr_8}
 \end{multline}
 для некоторых положительных констант~$C_{nq}$ и~$Q_{nq}$,
  что и~доказывает истинность оценки~(\ref{eq:approx_ord}).

  Далее рассмотрим величину $|\widehat{X}^K(n)\hm-\widehat{X}(n)|\mathbf{I}_{\{X=e_q\}}
  (\omega)$ на множестве $\{\omega: X(\omega)\hm=e_q\}$ для некоторого 
  фиксированного~$q \hm\in \overline{S}_T$. Из~(\ref{eq:int_class}) 
  и~(\ref{eq:int_class_unnorm}) следует, что $\widehat{X}(n)\mathbf{I}_{\{X=e_q\}}
  (\omega)=0$ $\ppp$-п.~н., поэтому
    \begin{multline}
\left\vert \widehat{X}^K(n)-\widehat{X}(n)\right\vert\mathbf{I}_{\{X=e_q\}}(\omega) ={}\\[1pt]
{}=
\mathbf{I}_{\{X=e_q\}}(\omega)
\fr{p_n\widetilde{X}^K(n)}{\sum\nolimits_{\ell=1}^Np_n\widetilde{X}^K(\ell)} \leqslant
\mathbf{I}_{\{X=e_q\}}(\omega) \fr{p_n}{p_q}\times{}\\[1pt]
{}\times
e^{-(({\psi(n)-\psi(q)})/({2h_k}))+\eta(n)-\eta(q)+\phi^K(n)-\phi^K(q)}\,,
\label{eq:eqpr_9}
 \end{multline}
причем из леммы~1~\cite{B_17} следует, что $\psi(n)\hm-\psi(q)\hm>0$ на 
множестве $\{\omega:\;X(\omega)\hm=e_q\}$. Вычисляя математические ожидания квадратов 
левой и~правой частей~(\ref{eq:eqpr_9}) и~используя ту же аргументацию, 
можно видеть, что оценка~(\ref{eq:eqpr_8}) справедлива также и~для всех $q \hm\in 
\overline{S}_T$. Таким образом,
\begin{equation*}
I^K(n) \leqslant \sum\limits_{q=1}^N
Q_{nq}h_K\,,
%\label{eq:eqpr_10}
\end{equation*}

\noindent
что доказывает истинность оценки~(\ref{eq:approx_ord}).
 Теорема~1 дока\-зана.

\smallskip

\noindent
Д\,о\,к\,а\,з\,а\,т\,е\,л\,ь\,с\,т\,в\,о\ \ теоремы~2. 
%\begin{proofoftheorem}{\ref{th:thm4}}
Воспользуемся приемами доказательства теоремы~1. Прежде всего, 
$\widehat{X}_{\xi(n)+H_K}(n)\hm=0$ $\ppp$-п.~н., а~также
\begin{multline*}
\widehat{\varkappa}^K_{\xi(n)+H_K}(n)\mathbf{I}_{\{\xi(n)<T\}}(\omega)={}\\[1pt]
{}=
\sum\limits_{q:q\neq n}\widehat{\varkappa}^K_{\xi(n)+H_K}(n)\mathbf{I}_{\{X=e_q\}}
(\omega)= {}\\[1pt] 
{}=
\sum\limits_{q:q\neq n} \mathbf{I}_{\{X=e_q\}}(\omega)
\left(\widehat{\varkappa}^K_{u(q,n)}(n)\exp\left\{
-\fr{\psi_{H_K}(n)}{2h_K}+{}\right.\right.\\[1pt]
\!\left.\left.{}+\eta_{H_K}(n)+\phi_{H_K}^K(n)
\vphantom{\fr{\psi_{H_K}(n)}{2h_K}}\!
\right\}\!
\right)\!\!\Bigg/ \!\!
\sum\limits_{\ell=1}^N\!
\widehat{\varkappa}^K_{u(q,n)}(\ell)\exp\left\{\!
-\fr{\psi_{H_K}(\ell)}{2h_K}
+{}\right.\hspace*{-1.55pt}\\[1pt]
\left.{}+\eta_{H_K}(\ell)+\phi_{H_K}^K(\ell)
\vphantom{\fr{\psi_{H_K}(n)}{2h_K}}
\right\}\,,
%\label{eq:eqpr_11}
 \end{multline*}
 где
 \begin{align*}
 \psi_{H_K}(\ell) &\triangleq
\displaystyle \int\limits_{u(q,n)}^{u(q,n)+H_K}\left[
\left\vert G_s(n)\right\vert
+
\mathrm{tr}\left(G_s(q) G_s^{-1}(n)
\right)
\right]\,ds\,;
 \\[1pt]
  \eta_{H_K}(\ell) &\triangleq
\int\limits_{u(q,n)}^{u(q,n)+H_K}
  \left(
{\fr{1}{2}}\,\|f_s(n)\|^2_{G^{-1}_s(n)}-{}\right.\\
&\hspace*{30mm}\left.{}-f_s^{\top}(n)G^{-1}_s(n)\,dY_s
  \vphantom{\fr{1}{2}}\right),
 %\label{eq:eqpr_12}
 \end{align*}
 а $\{\phi_{H_K}^K(\ell)\}_{\ell=\overline{1,N}}$~--- 
 набор таких случайных последовательностей, что $\me{}{(\phi_{H_K}^K(\ell))^2}\hm=O(h_K)$.

 Тогда в~силу условия~(в) теоремы верно следующее неравенство:
 \begin{multline*}
\widehat{\varkappa}^K_{\xi(n)+H_K}(n)\mathbf{I}_{\{\xi(n)<T\}}(\omega) \leqslant {}\\[1pt] 
{}\leqslant
\mathbf{I}_{\{X=e_q\}}(\omega) e^{-{C_1}/({2h_K})}\!
\sum\limits_{q:q\neq n} \fr{\widehat{\varkappa}^K_{u(q,n)}(n)}
{\widehat{\varkappa}^K_{u(q,n)}(q)} \exp
\left\{\!
\vphantom{\phi_{H_K}^K(q)}
\eta_{H_K}(n)+{}\right.\\[1pt]
\left.{}+\phi_{H_K}^K(n)-\eta_{H_K}(q)-\phi_{H_K}^K(q)\right\}\,.
%\label{eq:eqpr_12}
 \end{multline*}
 Далее
 \begin{multline}
\mathbf{E}\left\{(\widehat{\varkappa}^K_{\xi(n)+H_K}(n)- 
 \widehat{X}_{\xi(n)+H_K}(n))^2\mathbf{I}_{\{\xi(n)<T\}}(\omega)\right\}
 \leqslant {}\\
 {}\leqslant \!
e^{-{C_1}/{h_K}}(N-1)\times{}\\[2pt]
{}\times\sum\limits_{q:q\neq n}\!\! 
\mathbf{E}\left\{\!\mathbf{I}_{\{X=e_q\}}(\omega)
\left(\fr{\widehat{\varkappa}^K_{u(q,n)}(n)}{\widehat{\varkappa}^K_{u(q,n)}(q)}
\right)^2
\exp\left\{2\left(
\vphantom{\phi_{H_K}^K(q)}
\eta_{H_K}(n)+{}\right.\right.\right.\\
\left.\left.\left.{}+\phi_{H_K}^K(n)-\eta_{H_K}(q)-
\phi_{H_K}^K(q)\right)\right\}
\vphantom{\fr{\widehat{\varkappa}^K_{u(q,n)}(n)}{\widehat{\varkappa}^K_{u(q,n)}(q)}}
\right\}\!.
\label{eq:eqpr_13}
 \end{multline}
 Математические ожидания в~правой части~(\ref{eq:eqpr_13}) ограничены по $K \hm\in 
 \mathbb{N}$ некоторой положительной константой~$Q$, поэтому первое утверждение 
 теоремы непосредственно следует из~(\ref{eq:eqpr_13}), если обозначить\linebreak 
 $C_2 \hm= \sqrt{(N-1)Q}$. Второе утверждение теоремы доказывается абсолютно 
 аналогично путем замены константы~$C_1$ на~$C_3H_K$.
 { %\looseness=1
 
 }

 Теорема~2 доказана.
 
 }


{\small\frenchspacing
 {%\baselineskip=10.8pt
 \addcontentsline{toc}{section}{References}
 \begin{thebibliography}{9}
\bibitem{B_17}
\Au{Борисов А.\,В.} Классификация по непрерывным наблюдениям с~мультипликативными 
шумами~I: формулы байесовской оценки~// Информатика и~её применения, 2017. 
Т.~11. Вып.~1. C.~11--19.

\bibitem{S_14}
\Au{Стоянов Й.} Контрпримеры в~теории вероятностей~/
Пер. с~англ.~--- М.: МЦНМО, 2014. 296~с.
(\Au{Stoyanov~J.} {Counterexamples in probability.}~--- 
New York, NY, USA: John Wiley, 1987. 313~p.)

\bibitem{LS_86}
\Au{Липцер Р.\,Ш., Ширяев А.\,Н.} Теория мартингалов.~---~ М.: Наука, 1986. 512~c.

\bibitem{PB_10}
\Au{Platen E., Bruti-Liberati~N.}
Numerical solution of stochastic differential equations with jumps in finance.~--- 
New York, NY, USA: Springer, 2010. 868~p.

\bibitem{PR_10}
\Au{Platen E., Rendek~R.}
Quasi-exact approximation of hidden Markov chain filters~//
Commun. Stoch. Anal., 2010. Vol.~4. No.\,1. P.~129--142.

\bibitem{LS_74}
\Au{Липцер Р.\,Ш., Ширяев~А.\,Н.} Статистика случайных процессов.~--- 
М.: Наука, 1974. 512~с.
 \end{thebibliography}

 }
 }

\end{multicols}

\vspace*{-3pt}

\hfill{\small\textit{Поступила в~редакцию 19.12.16}}

\vspace*{8pt}

%\newpage

%\vspace*{-24pt}

\hrule

\vspace*{2pt}

\hrule

\vspace*{8pt}


\def\tit{CLASSIFICATION BY CONTINUOUS-TIME OBSERVATIONS 
IN~MULTIPLICATIVE NOISE~II: NUMERICAL ALGORITHM}

\def\titkol{Classification by continuous-time observations in multiplicative noise~II: Numerical algorithm}

\def\aut{A.\,V.~Borisov}

\def\autkol{A.\,V.~Borisov}

\titel{\tit}{\aut}{\autkol}{\titkol}

\vspace*{-9pt}


\noindent
Institute of Informatics Problems, Federal Research Center 
``Computer Science and Control'' of the Russian
Academy of Sciences,  44-2~Vavilov Str., Moscow 119333, Russian Federation



\def\leftfootline{\small{\textbf{\thepage}
\hfill INFORMATIKA I EE PRIMENENIYA~--- INFORMATICS AND
APPLICATIONS\ \ \ 2017\ \ \ volume~11\ \ \ issue\ 2}
}%
 \def\rightfootline{\small{INFORMATIKA I EE PRIMENENIYA~---
INFORMATICS AND APPLICATIONS\ \ \ 2017\ \ \ volume~11\ \ \ issue\ 2
\hfill \textbf{\thepage}}}

\vspace*{3pt}


\Abste{This is the second part of the paper ``Classification by continuous-time 
observations in multiplicative noise~I: Formulae for Bayesian estimate'' 
published in ``Informatics and Applications,'' 2017, 11(1). Investigations 
are aimed at estimation of a~finite-state random vector given
continuous-time noised observations. The key feature\linebreak\vspace*{-12pt}}

\Abstend{is that the 
observation noise intensity is a function of the estimated vector, 
which makes useless the known results in the optimal filtering.
In the first part of the paper, the required estimate is obtained 
both in the explicit integral form and as a~solution to 
a~stochastic differential system with some jump processes in the right-hand side.
The second part contains a~numerical algorithm of the estimate approximate 
calculation together with its accuracy analysis. An example illustrating the 
performance of the proposed estimate is also presented.}

\KWE{optimal filtering; identifiability; recursive scheme; approximation order;
 time discretization}



\DOI{10.14357/19922264170204} 

%\vspace*{-18pt}

\Ack
\noindent
The work was supported in part by the Russian Foundation
for Basic Research (projects Nos.\,15-37-20611 and
16-07-00677).



%\vspace*{3pt}

  \begin{multicols}{2}

\renewcommand{\bibname}{\protect\rmfamily References}
%\renewcommand{\bibname}{\large\protect\rm References}

{\small\frenchspacing
 {%\baselineskip=10.8pt
 \addcontentsline{toc}{section}{References}
 \begin{thebibliography}{9}
\bibitem{B_17-1}
\Aue{Borisov, A.\,V.} 2017. 
Klassifikatsiya po nepreryvnym nablyudeniyam s~mul'tiplikativnymi shumami~I:
Formuly Bayesovskoy otsenki 
[Classification by continuous-time observations in multiplicative noise~I: 
Formulae for Bayesian estimate]~//
\textit{Informatika i~ee Primeneniya~--- Inform. Appl.} 11(1):11--19.

 \bibitem{S_14-1}
\Aue{Stoyanov, J.} 1987. \textit{Counterexamples in probability.} 
New York, NY: John Wiley. 313~p.

\bibitem{LS_86-1}
\Aue{Liptser, R.\,Sh., and A.\,N.~Shiryayev.} 1989. \textit{Theory of martingales.}
New York, NY: Springer. 812~p.

\bibitem{PB_10-1}
\Aue{Platen, E., and N.~Bruti-Liberati}. 2010.
\textit{Numerical solution of stochastic differential equations with jumps in finance.}
New York, NY: Springer. 868~p.

\bibitem{PR_10-1}
\Aue{Platen, E., and R.~Rendek}. 2010.
Quasi-exact approximation of hidden Markov chain filters.
\textit{Commun. Stoch. Anal.} 4(1):129--142.

\bibitem{LS_74-1}
\Aue{Liptser, R.\,Sh., and A.\,N.~Shiryayev.} 2001. 
\textit{Statistics of random processes: I.~General theory.} Berlin: Springer. 427~p.
\end{thebibliography}

 }
 }

\end{multicols}

\vspace*{-3pt}

\hfill{\small\textit{Received December 19, 2016}}

\Contrl

\noindent
\textbf{Borisov Andrey V.} (b.\ 1965)~--- 
Doctor of Science in physics and mathematics, principal scientist, Institute of
Informatics Problems, Federal Research Center ``Computer Science and Control'' 
of the Russian Academy of
Sciences, 44-2~Vavilov Str., Moscow 119333, Russian Federation; \mbox{aborisov@frccsc.ru}

\label{end\stat}


\renewcommand{\bibname}{\protect\rm Литература} 