
\def\stat{serebr}

\def\tit{ПЕРСОНАЛЬНАЯ ОТКРЫТАЯ СЕМАНТИЧЕСКАЯ ЦИФРОВАЯ БИБЛИОТЕКА 
LibMeta. КОНСТРУИРОВАНИЕ КОНТЕНТА. ИНТЕГРАЦИЯ 
С~ИСТОЧНИКАМИ LOD$^*$}

\def\titkol{Персональная открытая семантическая цифровая библиотека 
LibMeta. Конструирование контента} %. Интеграция с~источниками LOD}

\def\aut{О.\,М.~Атаева$^1$, В.\,А.~Серебряков$^2$}

\def\autkol{О.\,М.~Атаева, В.\,А.~Серебряков}

\titel{\tit}{\aut}{\autkol}{\titkol}

\index{Атаева О.\,М.}
\index{Серебряков В.\,А.}
\index{Ataeva O.\,M.}
\index{Serebryakov V.\,A.}


{\renewcommand{\thefootnote}{\fnsymbol{footnote}} \footnotetext[1]
{Работа выполнена при финансовой поддержке РФФИ (проект 14-07-00058~А).}}


\renewcommand{\thefootnote}{\arabic{footnote}}
\footnotetext[1]{Вычислительный центр им.\ А.\,А.~Дородницына Федерального исследовательского 
центра <<Информатика и~управ\-ле\-ние>> Российской академии наук, 
\mbox{oli@ultimeta.ru}}
\footnotetext[2]{Вычислительный центр им.\ А.\,А.~Дородницына Федерального исследовательского 
центра <<Информатика и~управ\-ле\-ние>> Российской академии наук, 
\mbox{serebr@ultimeta.ru}}

\vspace*{-3pt}

    

\Abst{Развитие семантических технологий вывело цифровые библиотеки на уровень, на 
котором на первый план выступила необходимость осмысленного представления контента 
цифровых библиотек. Одновременно возникает необходимость ограничения его 
в~терминах некоторой предметной области. В~работе рассматривается конструирование 
контента библиотеки для некоторой предметной области в~рамках разработанной системы 
LibMeta.
Персональная открытая семантическая цифровая библиотека LibMeta с~сис\-те\-мой 
поддержки работы пользователей с~цифровыми ресурсами библиотек и~их коллекциями 
для некоторой предметной об\-ласти, ограниченной терминологически с~по\-мощью 
тезауруса, предоставляет функциональность конструирования контента библиотеки 
согласно определенным требованиям и~требует всего лишь произвести начальную 
настройку сис\-те\-мы под конкретную предметную об\-ласть, ограниченную 
терминологически с~по\-мощью тезауруса. В~качестве примера предметной об\-ласти 
в~работе используется узкоспециализированный тезаурус обыкновенных 
дифференциальных урав\-не\-ний (ОДУ).}

\KW{семантические библиотеки; модель данных; онтологии; источники данных; 
поиск в~LOD}

\DOI{10.14357/19922264170210} 


\vskip 10pt plus 9pt minus 6pt

\thispagestyle{headings}

\begin{multicols}{2}

\label{st\stat}

\section{Введение}

     Взрывное развитие технологий в~последние десятилетия повлияло на 
все аспекты деятельности человека. Накопленные в~библиотеках данные 
стали через сеть доступны широкому кругу пользователей, удовлетворяя 
информационные потребности которых, разработчики расширяли 
функциональность цифровых библиотек. 
     
     Развитие семантических технологий вывело циф\-ро\-вые библиотеки на 
новый уровень, на котором на первый план выступила необходимость 
осмыс\-лен\-но\-го представления контента цифровых биб\-лио\-тек. В~решении этих 
задач ключевую роль стали играть онтологии~[1], позволяя представлять 
концептуальные модели для описания самого контента этих библиотек, 
основываясь на ранее разработанных форматах описания, таких как 
MARC\footnote[3]{{\sf http://www.loc.gov/marc/unimarctomarc21.html.}}. Такие 
онтологии получили название библиографических, дополняя семантикой эти 
форматы. Фактически в~библиографических онтологиях фиксируются 
ключевые понятия объектов, составляющих наполнение библиотеки, и~связи 
между ними. Этих понятий достаточно для описания обычной классической 
цифровой библиотеки для любой предметной области, в~которой 
представлена информация о~различных печатных изданиях и,~возможно, их 
электронные версии. Но развитие семантических библиотек~[2] способствует 
расширению модели, определяющей наполнение библиотеки, в~которой 
теперь могут содержаться самые различные типы объектов. 
     
     Одновременно с~расширением модели биб\-ли\-о\-теч\-но\-го наполнения 
возникает необходимость ограничения его в~рамках некоторой предметной 
области. Для этого вводится набор терминов, используемых для описания 
этой предметной об\-ласти. Чаще всего эти термины организованы в~виде 
некоторой таксономии с~поддержкой разнообразных связей между ними. 
В~дальнейшем будем называть наполнение библиотеки с~такой 
терминологической поддержкой некоторой предметной области контентом 
семантической цифровой библиотеки, или просто контентом.
     
Для тематической классификации ресурсов биб\-ли\-о\-те\-ки используются различные классификаторы, которые 
отличаются друг от друга охватом предметных областей и~степенью гранулярности при классификации этих 
областей. Для этих целей может использоваться один из широко распространенных классификаторов, таких 
как УДК\footnote[1]{{\sf http://nlib.sakha.ru/Cataloque/udk/index.shtml.}} (универсальная десятичная 
классификация), ББК\footnote[2]{{\sf http://roslavl.library67.ru/files/382/bbk.pdf.}} 
(биб\-лио\-теч\-но-биб\-лио\-гра\-фи\-че\-ская классификация), ГРНТИ\footnote[3]{{\sf 
http://www2.viniti.ru/index.php?option=com\_content\&view=article\&id=39:rubrikator-nti.}} 
(государственный рубрикатор на\-уч\-но-тех\-ни\-че\-ской информации). Эти классификаторы охватывают 
почти все области научного знания и~перечень понятий, характерных для этих областей. Обычно эти понятия 
носят довольно общий характер и~не отражают всего разнообразия направлений в~каждой отдельной 
области научного знания.
     
     Специализированные по конкретным областям библиотеки используют 
обычно свои классификаторы для систематизации своих ресурсов. Такой 
подход обеспечивает более детальный анализ содержания документов 
и~соотношение смысловых понятий в~документе с~определенным 
направлением специализированной области знания. К~таким 
классификаторам можно, например, отнести MSC\footnote[4]{{\sf 
http://www.ams.org/msc/pdfs/classifications2010.pdf.}} (Mathematics Subject 
Classification), который используется для классификации разделов 
математики. 
     
     Семантические библиотеки предоставляют своим пользователям 
большой арсенал возможностей для удовлетворения их информационных 
потребностей~[2]. Это разнообразные средства поиска: атрибутный поиск, 
полнотекстовый поиск, поиск по коллекциям на основе тематических 
классификаторов, поиск по разнообразным типам ресурсов, включенных 
в~библиотеку. Для пользователей семантических библиотек, являющихся 
активными потребителями информации, во многих современных решениях 
предоставляется возможность создать собственную коллекцию. 
     
     Возникает необходимость дать пользователям специфицировать свои 
предпочтения, развивая возможность определения собственных терминов 
в~рамках некоторого направления научного знания, уточняя и~очерчивая круг 
своих интересов, позволяя организовывать группы пользователей со 
сходными интересами для возможности отслеживания всей информации по 
определенным направлениям.
     
     Широкое применение онтологий позволяет интегри\-ровать данные 
библиотек с~данными из различных источников, основываясь на их 
семантике~[3]. Эти источники не обязательно сами являются библиотеками. 
Множество таких источников подключено к~облаку LOD (Linked Open 
Data)~[4]. Основная идея LOD заключается в~решении задач интеграции 
данных, представленных в~сети, для чего предлагается представить 
информацию в~формализованном виде с~по\-мощью онтологий, что делает ее 
доступной для машинной обработки. В~этих источниках данных провязаны 
самые различные типы ресурсов, которые представляют интерес для 
пользователей библиотек с~точки зрения обогащения данных как структурно, 
так и~семантически.
     
     На основе модели понятий, описанной в~предыду\-щих работах~[5], 
а~также идей Semantic Web и~LOD была разработана 
персональная открытая семантическая циф\-ро\-вая библиотека \mbox{LibMeta} 
с~системой поддержки работы пользователей с~циф\-ро\-вы\-ми ресурсами 
библиотек и~их коллекциями для некоторой предметной области, 
ограниченной терминологически с~помощью тезауруса~\cite{3-ser, 6-ser}.

\vspace*{-4pt}
     
\section{LibMeta~--- основные идеи}

\vspace*{-2pt}

     При реализации LibMeta авторы руководствовались набором основных 
задач, которые должна решать разрабатываемая система:
     \begin{enumerate}[(1)]
\item библиотека должна поддерживать возможность использования 
медийных объектов или ссылки на них при описании своих объектов, 
включая текст, аудио-, видеофайлы или любую их комбинацию. Это 
требование отражается в~названии словом <<цифровая>>;
\item типы используемых ресурсов и~связи между ними должны быть 
описаны средствами сис\-те\-мы в~рамках определенных в~предыдущей работе 
понятий, составляющих семантическое описание ресурсов контента 
библиотеки. При этом согласно принципам LOD при описании ресурсов 
поддерживается использование классов и~свойств ранее используемых 
онтологий в~сообществе, поддерживающем LOD. Эта поддержка 
выражается либо в~непосредственном использовании готовых онтологий 
при описании ресурсов и~связей между ними, либо возможностью ссылок 
на их элементы, используя связи на уровне описания ресурсов. Это 
требование отражается в~названии словом <<семантическая>>;
\item библиотека должна служить интеграционным узлом, предоставляя 
возможность связывания своих данных с~данными из разных источников, 
которые включены в~облако LOD. Должна также обеспечиваться 
возможность извлекать данные этой библиотеки в~машиночитаемом 
формате. Это требование отражается в~названии словом <<открытая>>;
\item пользователи библиотеки должны иметь возможность организовывать 
свои коллекции по интересующему их научному направлению, добавляя 
новые термины в~предметный тезаурус, уточняя таким образом область 
своих интересов. Пользователи должны также иметь возможность 
осуществлять поиск не только среди объектов в~рамках системы, но и~по 
источникам данных без необходимости использования 
специализированного языка для поисковых запросов. Это требование 
отражается в~названии словом <<персональная>>.
\end{enumerate}

     Основные требования, предъявляемые при этом\linebreak к~контенту  
сис\-те\-мы,~--- универсальность, структурированность, адаптируемость~--- 
не противо\-речат этим свойствам и~обеспечивают поддержку\linebreak настраива\-емого 
хранилища метаданных для объектов и~расширяемый набор 
информационных ресурсов. Универсальность обеспечивает описание типов 
ее ресурсов и~объектов независимо от предметной области и~области 
интересов пользователей. Структурированность описания обеспечивает 
поддержку связей между различными типами ресурсов как внутри системы, 
так и~вне ее, исходя из определений LOD. Адаптируемость описания 
ресурсов обеспечивает возможность добавления новых свойств и~связей 
в~процессе развития системы и~обеспечивает настройку пользовательских 
интерфейсов под эти изменения. 
     
     Фактически LibMeta предоставляет функциональность 
конструирования контента библиотеки согласно этим требованиям, и~на 
начальном этапе при установке системы требуется всего лишь произвести 
настройку системы под конкретную предметную область, описав ее ресурсы и~таксономии, которые будут очерчивать тематически предметную область 
ее ресурсов и~таким образом составлять ее тезаурус. 


     
\section{LibMeta~--- первый пример конструирования}



     Рассмотрим простой пример реализации биб\-лио\-те\-ки LibMeta, 
основанной на данных публикаций из электронной библиотеки <<Научное 
наследие России>>~\cite{7-ser}. Основных типов ресурсов, которые 
определены для этих данных, всего два: персоны и~публикации. Для 
тематической классификации этих публикаций используется классификатор 
\mbox{ГРНТИ}, и~каждая публикация снабжена номером УДК.
     
     Авторы не ставили своей целью создание уменьшенной копии 
<<Научного наследия>>. Основная цель, преследуемая в~контексте 
предлагаемой сис\-те\-мы,~--- это связывание этих данных с~данными, 
опубликованными в~LOD, и~их публикация для возможности доступа к~ним 
других систем. В качестве источника данных для связывания в~этом примере 
используется DBpedia\footnote{{\sf http://dbpedia.org.}}, служащая ядром LOD. 
     
     Итак, основная цель при конструировании описа\-ния контента 
заключается в~том, чтобы пред\-став\-лен\-ное описание по возможности 
макси\-мально облегчало реализацию процедуры поиска данных в~узлах LOD. 
Жертвой этой идеи становится,\linebreak возможно, некоторая упро\-щен\-ность 
структуры контента, в~отличие от выразительности, пред\-став\-ля\-емой 
средствами языка OWL\footnote{{\sf https://www.w3.org/2001/sw/wiki/OWL.}}, как 
будет показано ниже, но при этом получаем гибкость при по\-стро\-ении 
интеграционного узла для различных типов ресурсов, описание которых 
можно расширять в~процессе жизнедеятельности системы. 
     
     Фактически понятия \textit{персоны} и~\textit{пуб\-ли\-ка\-ции} 
представляют собой экземпляры класса \textit{информационный ресурс}, 
определенного как базовая единица контента семантической библиотеки. Так 
как каждый ресурс обладает набором атрибутов, для каждого из этих 
экземпляров задается собственный набор из множества атрибутов, 
предварительно описанных в~системе. Множество атрибутов для 
информационных ресурсов состоит из следующих элементов: \textit{название 
на языке оригинала, название на русском, фамилия, имя, отчество, 
электронный адрес, дата рождения, аннотация, идентификатор, автор, 
деятельность, тип публикации, место рождения, биография, описание, 
дополнительное заглавие, язык}. 
     
    Конкретные персоны~--- это объекты, пред\-ставля\-ющие экземпляры 
класса \textit{информационный объект}, они определяются 
информационным ре\-сурсом \textit{персона} и~представляются значениями %\linebreak 
атрибутов соответствующего ресурса. Помимо свойств, заданных 
атрибутами, представленными в~наборе атрибутов своего информационного 
ресурса, каж\-дый объект обладает также свойствами, общими для всех 
информационных объектов, такими как \textit{теги, описание, дата 
создания, дата изменения, владелец, уникальный идентификатор}. 
    
    На рис.~1 приведена упрощенная схема, сконструированная для этих 
типов ресурсов. На схе-\linebreak\vspace*{-12pt}

\pagebreak

\end{multicols}

     \begin{figure*} %fig1
      \vspace*{1pt}
\begin{center}
\mbox{%
\epsfxsize=158.401mm
\epsfbox{ser-1.eps}
}
\end{center}
\vspace*{-9pt}
\Caption{Конструирование информационных ресурсов}
\vspace*{4pt}
\end{figure*}

\begin{multicols}{2}

\noindent 
ме проиллюстрированы связи между экземплярами
информационных ресурсов \textit{персона} и~\textit{публикация} 
и~конкретными экземплярами класса информационного объекта (названия 
объектов \textit{объект}-\textit{п1}, \textit{объект}-\textit{п2},  
\textit{объект}-\textit{пб1} подчеркнуты). Префиксы <<ио>>, <<ир>>, 
<<к>>, <<т>>, <<тт>>, <<на>>, <<а>>, отделяемые двоеточием, 
указывают на принадлежность экземпляра соответственно к~классам 
\textit{информационный объект, информационный ресурс, коллекция, 
таксономия, таксон, набор атрибутов, атрибут}. 

Для тематической 
классификации объектов публикации используется коллекция, основанная на 
классификаторе ГРНТИ. 

        
    Серые стрелки, исходящие из экземпляров атрибутов, указывают на 
область возможных значений для них. Областью значений остальных 
атрибутов являются простые типы данных. На схеме значения атрибутов 
представлены с~помощью объектов вспомогательного класса \textit{значение 
атрибута} с~префиксом <<за>>. Объекты этого класса содержат для 
простых типов атрибутов их значения (например, значения текстовых 
атрибутов \textit{фамилия, имя, название} представлены на схеме 
в~кавычках).

 Для объектного атрибута \textit{автор} его значение содержит 
ссылку на соответствующий экземпляр информационного объекта с~типом 
персона,\linebreak\vspace*{-12pt}

\pagebreak

\end{multicols}

     \begin{figure*} %fig2
      \vspace*{1pt}
\begin{center}
\mbox{%
\epsfxsize=112mm
\epsfbox{ser-2.eps}
}
\end{center}
\vspace*{-3pt}
\Caption{Описание ресурсов в~формате RDF/XML}
\vspace*{9pt}
\end{figure*}

\begin{multicols}{2}

\noindent 
 что отображено
на схеме пунктирной стрелкой. Таксономические 
атрибуты \textit{тип} и~\textit{язык} в~качестве области значений указывают на 
соответствующие таксономии \textit{тип пуб\-ли\-ка\-ции} и~\textit{язык}, 
пред\-став\-ля\-ющие собой линейные словари, элементы которых (таксоны) 
используются в~качестве значений атрибутов. 
    
    Для каждого атрибута указан его вид: \textit{описательный, 
идентификационный} или \textit{поисковый}. Атрибут может относиться 
к~нескольким видам одновременно. Поисковые атрибуты используются для 
динамической генерации формы поиска по объектам определенного типа 
ресурсов. Описательные атрибуты используются для генерации формы 
представления информации об объекте для пользова\-теля.
{\looseness=1

} 

Набор значений 
идентификационных атрибутов необходим, как понятно из названия, для 
идентификации объекта. В~наборе атрибутов для публикаций атрибут 
\textit{автор} помечен как \textit{множественный}. Этот атрибут может 
иметь при описании информационных объектов, соответствующих по типу 
ресурса \textit{публикациям}, несколько значений, что отражено в~качестве 
примера на схеме.
     


    Описание структуры контента в~терминах \mbox{LibMeta} в~формате 
RDF/XML\footnote{{\sf http://www.w3.org/RDF/}; {\sf http://www.w3.org/XML/}; {\sf 
http://www.w3.org/TR/rdf-syntax-grammar.}} представлено на рис.~2. Задание 
структуры может осуществляться с~по\-мощью пользовательских 
интерфейсов системы или с~помощью загрузки RDF/XML с~описанием 
структуры контента в~соответствующем разделе системы пользователем, 
наделенным соответствующим уровнем прав.
    


    Основные понятия для описания контента биб\-лио\-те\-ки представлены 
в~работе~\cite{5-ser}. Фактически исходная онтология контента LibMeta 
содержит необходимые понятия, отношения и~аксиомы. При описании 
конкретной предметной области в~эту онтологию добавляются отдельные 
экземпляры определенных в~ней понятий, которые и~составляют контент 
создаваемой библиотеки. 



    
    

\pagebreak

\end{multicols}

     \begin{figure*} %fig3
      \vspace*{1pt}
\begin{center}
\mbox{%
\epsfxsize=160mm
\epsfbox{ser-3.eps}
}
\end{center}
\vspace*{-9pt}
\Caption{Информационный объект в~формате RDF/XML}
\vspace*{12pt}
\end{figure*}

\begin{multicols}{2}

%\noindent
На рис.~3 приведен пример представления экземпляра 
информационного объекта по заданному набору атрибутов из 
соответствующего ему экземпляра информационного ресурса.

\section{LibMeta~--- второй пример конструирования}

    Рассмотрим пример, когда в~качестве терминов предметной области 
используется узкоспециализированный тезаурус ОДУ~\cite{8-ser}. Особенность этого 
тезауруса заключается в~том, что он содержит не только сами понятия 
и~термины, но и~ссылки на публикации, в~которых  
вво\-дят\-ся/опре\-де\-ля\-ют\-ся эти понятия, их математическая запись. Был 
введен новый информационный ресурс \textit{литература} для описания 
публикаций, ставших основой построения этого тезауруса. В~соответствие 
ему был\linebreak поставлен тот же набор атрибутов, что и~в~преды\-ду\-щем примере. 
На рис.~4 представлены понятия тезауруса, связанные иерархически, и~для 
каж\-дого понятия отображаются его горизонтальные\linebreak связи.
{ %\looseness=1

}


    С помощью этого тезауруса был размечен набор публикаций со схожей 
тематикой. Схожесть тематики публикации тезаурусу ОДУ определялась по 
ее ключевым словам, соответствующим терминам тезауруса. 

На рис.~5 
представлен пример связи понятия из ОДУ и~найденных публикаций. 
В~качестве связанных объектов могут выступать не только 
\textit{публикации}, но и,~например, персоны, в~описании деятельности 
которых могут встречаться соответствующие понятию из ОДУ ключевые 
слова.
    
    В качестве модели информационных ресурсов было использовано то же 
описание \textit{персоны} и~\textit{пуб\-ли\-ка\-ции}, что и~в~предыдущем 
примере. В~этом случае один и~тот же набор атрибутов использовался как 
для описания ресурса \textit{литература}, так и~для описания ресурса 
\textit{пуб\-ли\-ка\-ция}. Это позволило отдельно настроить права доступа для 
всех объектов \textit{литературы}, запретив их модификацию\linebreak или удаление 
пользователям, не являющимся\linebreak редакторами предметной области. Сами 
публикации извлечены из Единого научного информа\-ционного пространства 
(ЕНИП) РАН~--- это ин\-тегрированное информационное пространство\linebreak\vspace*{-12pt}

\pagebreak

\end{multicols}

\begin{figure*} %fig4
 \vspace*{1pt}
\begin{center}
\mbox{%
\epsfxsize=160mm
\epsfbox{ser-4.eps}
}
\end{center}
\vspace*{-9pt}
\Caption{Тезаурус ОДУ}
%\end{figure*}
%\begin{figure*} %fig5
 \vspace*{14pt}
\begin{center}
\mbox{%
\epsfxsize=160mm
\epsfbox{ser-5.eps}
}
\end{center}
\vspace*{-9pt}
\Caption{Информационные объекты и~термины ОДУ}
\vspace*{12pt}
\end{figure*}


\begin{multicols}{2}

\noindent
распределенных и~локальных цифровых (электронных) ресурсов организаций 
РАН и~комплекс про\-граммно-тех\-ни\-че\-ских средств, обеспе\-чи\-ва\-ющих 
использование этих ресурсов и~полнофункциональное управление 
ими~\cite{9-ser}. 

Для извлечения информации о~публикациях и~авторах 
использовался
 протокол OAI-PMH\footnote[1]{{\sf 
https://www.openarchives.org/pmh.}}. Данные были представлены в~формате Dublin 
Core\footnote[2]{{\sf http://dublincore.org.}}. 
%
Часть публикаций была размечена 
ключевыми словами, однако термины не разделены между собой и~просто 
перечислялись через запятую в~одном поле. 
%
Ключевые слова пуб\-ли\-ка\-ции 
были преобразованы в~набор ключевых слов соответствующего 
информационного объекта для каждой извлеченной публикации из ЕНИП. 

В~коллекцию публикаций тезауруса ОДУ добавлялись те объекты, в~наборе 
ключевых слов которых находились термины ОДУ. 
    
    Размещение публикации в~ту или иную ветвь коллекции может 
осуществлять как сам пользователь, так и~соответствующий модуль 
автоматической разметки информационных объектов по тезаурусу, в~котором 
задаются простые правила разметки в~рамках описания тезауруса.
    
\section{LibMeta~--- взаимодействие с~источниками LOD}

    Основная проблема приложений, разрабатыва\-емых для работы 
с~данными из источников, интегрированных в~LOD, состоит в~том, что 
данные в~этих источниках очень слабо провязаны с~данными других 
источников. Большинство имеющихся связей расположены на уровне самих 
данных, при этом на уровне схем такие связи практически отсутствуют. Для 
решения этой проблемы предлагаются разные подходы, которые основаны на 
методах сравнения онтологий~\cite{1-ser, 4-ser}. Некоторые из них 
используют для сравнения онтологий данные, доступные в~сети. В~част\-ности, 
используется Wikipedia и~ее иерархический рубрикатор\footnote{{\sf  
https://en.wikipedia.org/wiki/Special:Categories.}}. 
    
    В отличие от других работ, построение иерархий классов при 
трансляции ресурсов на источник данных в~данной ситуации не представляет 
интереса, поэтому для проставления связей используется связь, которая 
указывает, что два разных класса \textit{могут} иметь одинаковых 
представителей. Эта связь может указывать на класс в~источнике данных 
LOD,\linebreak который является источником дополнительной информа\-ции  
о~ре\-сур\-се-субъ\-ек\-те, или на эквивалентный ему класс, возможно, 
с~разной степенью детализации описания объектов. Фактически 
предполагается, что онтология источника данных \textit{час\-тич\-но} 
совместима со структурой ресурсов описываемой библиотеки. Это означает, 
что хотя бы один ресурс ее онтологии может быть транслирован в~некоторый 
класс в~онтологии источника данных. Требуется лишь минимальное 
частичное соответствие ресурсу LibMeta. Это означает, что для однозначной 
идентификации экземпляров соответствующего класса из источника данных 
отображаться должны как минимум идентифицирующие атри\-буты. 

    
    В~связи с~гибкостью схемы LibMeta предполагается возможным 
сценарий создания дополнительных типов ресурсов для подключаемых 
источников, информацию из которых можно использовать как значения 
некоторых атрибутов основных ресурсов.
{\looseness=1

}
    
    Для трансляции атрибутов ресурса в~свойства выбранного класса 
источника данных будет использоваться связь, которая указывает, что 
значения атрибута и~свойства полностью или частично совпадают в~рамках 
установленного соответствия на уровне ресурса библиотеки и~класса 
онтологии. При совпадении значений все очевидно, проблема возникает при 
разной детализации данных, когда возможно отображение значения свойства 
класса в~несколько атрибутов и~наоборот. В~связи с~гиб\-костью модели 
данных может быть принято решение о~расширении схемы ресурса. 
В~другом случае пользователь может использовать набор вспомогательных 
функций, например для расщепления или слияния данных. Простой пример 
такого рода преобразований связан с~трансформацией имени персоны. 
В~первом случае, когда имя персоны описывается одним значением 
в~источнике данных, а транслируется в~три отдельных атрибута, 
используется функция расщепления (\textit{ФИО}\;$\to$\;\textit{Ф, И, O}). Во 
втором случае значения отдельных свойств преобразуются в~значения одного 
атрибута (\textit{Ф, И, O}\;$\to$\;\textit{ФИО}), отображение свойств класса 
производится в~один атрибут и~тогда данные будут склеиваться как одно 
значение этого атрибута именно в~том порядке, в~котором они были 
перечислены при описании трансляции.
    
    Для преобразования данных в~соответствующие типы значений, которые 
указаны при описании\linebreak атрибута, использованы встроенные функции 
пре\-об\-ра\-зо\-ва\-ния, которые нет нужды настраивать пользователю. В~случае 
если такое преобразование заканчивается неудачно, то информация об этом\linebreak 
будет сохранена в~соответствующем административном атрибуте объекта для 
возможности дальнейшей обработки и~исправления ошибок.
    
    Поиск эквивалентных классов в~источниках данных пользователь может 
выполнить: 
    \begin{enumerate}[(1)]
\item вручную, выбирая из списка доступных классов в~указанном 
источнике данных;
\item полуавтоматически, используя имеющиеся описания связей 
с~другими классами внешних онтологий, заранее определенными при 
описании структуры ресурсов;
\item автоматически. 
\end{enumerate}

    В первых двух случаях пользователь на первом шаге предварительно 
указывает, с~каким типом ресурсов он предполагает работать, привязывая тот 
или иной источник данных. На втором шаге он определяет соответствие 
атрибутов и~свойств. В~треть\-ем варианте он получает возможность получить 
общую оценку соответствия схемы ресурсов библиотеки некоторой 
онтологии и~на основе этой оценки принимать решение о~трансляции 
ресурсов на тот или иной источник данных, использующий эту онтологию.
    
    Назовем проекцию понятия библиотеки $\mathrm{IR}$ на понятие~$C$ источника 
данных из LOD \textit{допустимой}, если возможно установить между ними 
хотя бы одно отношение из $\{R_1, R_2, R_3, R_4\}$: 
    \begin{itemize}
\item $R_1(C, \mathrm{IR})$ означает, что понятие~$C$ включает в~себя $\mathrm{IR}$;
\item $R_2(C, \mathrm{IR})$ означает, что понятие $\mathrm{IR}$ включает в~себя~$C$;
\item $R_3(C, \mathrm{IR})$ означает, что понятие $\mathrm{IR}$ связано отношением 
эквивалентности с~$C$;
\item $R_4(C, \mathrm{IR})$ означает, что понятие $\mathrm{IR}$ связано отношением 
частичной эквивалентности с~$C$.
\end{itemize}

    Все эти четыре отношения говорят о том, что понятия $\mathrm{IR}$ и~$C$ могут 
иметь одинаковых представителей. По семантике~$R_1$ соответствует 
skos:broader (например, $\mathrm{IR}$\;=\;\textit{Человек}, $C$\;=\;\textit{Студент}); 
$R_2$ соответствует skos:narrower (например, $\mathrm{IR}$\;=\;\textit{Студент}, 
$C$\;=\;\textit{Человек}); $R_3$ соответствует skos:exactMatch (например, 
$\mathrm{IR}$\;=\;\textit{Студент}, $C$\;=\;\textit{Студент}); $R_4$ соответствует 
skos:closeMatch (например, $\mathrm{IR}$\;=\;\textit{Студент}, 
$C$\;=\;\textit{Учащийся}).
    
    Отображение $\mathrm{IR}$ на~$C$ \textit{возможно}, если трансляция набора 
атрибутов $\mathrm{IR}$ на свойства класса~$C$ выполнена хотя бы для 
идентифицирующих атрибутов, т.\,е.\ любой идентифицирующий атрибут 
из $a_1, \ldots,  a_k$, принадлежащих набору атрибутов~$\mathrm{IR}$, где $k\hm< n$, 
$n$~--- число атрибутов соответствующего набора, транслируется хотя бы на 
одно свойство~$c_1, \ldots , c_m$ класса~$C$.
    
    Трансляция атрибута на свойство источника $t(a_i, c_j)$ для искомых 
или связываемых объектов может быть:
    \begin{itemize}
\item \textit{прямой}, когда атрибут отображается на свойство: 
$t_1\hm=\{t(a_i, c_j):\ a_i \hm= c_j\}$, т.\,е.\ значения должны быть 
эквивалентны;
\item \textit{неполной}, когда атрибут отображается на свойство лишь 
частично: $t_2\hm=\{t(a_i, c_j):\ a_i\subset c_j\}$, т.\,е.\ значение атрибута 
включается в~значение свойства;
\item \textit{избыточной}, когда атрибут шире свойства: $t_3\hm=\{t(a_i, 
c_j): a_i \supset c_j\}$, т.\,е.\ значение атрибута содержит больше 
информации, чем значение свойства.
\end{itemize}

    Из определения проекции некоторого понятия~$\mathrm{IR}$ и~трансляции его 
атрибутов, задающих отоб\-ра\-же\-ние понятия на некоторый набор данных 
источника, следует, что это отображение сюръективно и~неинъективно для 
наборов его атрибутов. 
    
    При использовании трансляции для поиска связанных объектов в~случае 
\textit{полной} трансляции идентифицирующего атрибута все ограничивается 
выбо\-ром соответствующего свойства и~значения могут сравниваться. 
В~случае \textit{неполной} и~\textit{избыточной} трансляции 
идентифицирующего атрибута явного сравнения значений недостаточно. 
В~любом случае возможно использование функций предобработки данных: 
преобразование форматов, извлечение подстрок и~т.\,д.
    
    В общем виде извлечение объектов из источника для сохранения 
в~качестве информационных объектов библиотеки задается функцией
    \begin{multline*}
    z\left( R_l\cap f(t)\right) = {}\\
    {}=\left\{ o\in \mathrm{IR}\vert R_l(\mathrm{IR},C):\ \forall\ a_i\exists\ 
f(t(a_i, c_j))\right\}\,,
    \end{multline*}
где функция $f$ зависит от типа трансляции атрибутов:
\begin{gather*}
f(t_1)=\begin{cases}
\mbox{true}, &\ a_i=c_j\,;\\
\mbox{false}, &\  a_i\not=c_j\,;
\end{cases}\\
f(t_2)= \begin{cases}
\mbox{true}, &\ a_i\subset c_j\,;\\
\mbox{false}, &\ a_i \not\subset c_j\,;
\end{cases}
\\
f(t_3) = \begin{cases}
\mbox{true}, &\ c_j\subset a_i\,;\\
\mbox{false}, &\ c_j \not\subset a_i\,.
\end{cases}
\end{gather*}
    
    Для того чтобы выполнять поисковые запросы по источникам данных, 
необходимо, чтобы отображение понятий библиотеки на понятия источника 
было и~\textit{возможным}, и~\textit{допустимым}.
    
    Если оно возможно, то для $R_1(C, \mathrm{IR})$ это означает, что все 
характерные признаки~$\mathrm{IR}$ наследуются~$C$, при этом набор 
признаков~$C$ шире набора~$\mathrm{IR}$, так как $\mathrm{IR}$ является более
 объемлющим 
понятием и,~следовательно, класс~$C$ всегда включает признаки, которые 
являются идентифицирующими для более широкого понятия~$\mathrm{IR}$. Если оно 
возможно для $R_2(C, \mathrm{IR})$, это означает, что все признаки класса~$C$ 
наследуются~$\mathrm{IR}$, при этом набор признаков~$\mathrm{IR}$ может быть 
шире набора 
класса~$C$, так как~$\mathrm{IR}$ является более узким понятием. Набор 
идентифицирующих признаков класса~$C$, необходимых для 
идентификации объектов в~источнике данных, включается в~набор 
признаков~$\mathrm{IR}$, что достаточно для идентификации эквивалентных объектов 
в~источнике. Если набор необходимых идентифицирующих признаков~$\mathrm{IR}$ 
шире, то это означает, что в~контексте источника этот набор избыточен 
и~можно его переопределить (сузить) для по\-стро\-ения допустимой 
трансляции. Если оно возможно для $R_3(C, \mathrm{IR})$, это означает, что все 
характерные признаки совпадают, в~том числе и~идентифицирующие. Если 
трансляция возможна, то для понятий, связанных отношением $R_4(C, \mathrm{IR})$, 
это означает, что хотя бы идентифицирующие характеристики у~них 
совпадают.
    
    Исходя из сказанного, для любых понятий, связанных одним из этих 
отношений, возможно построение допустимой трансляции и~можно строить 
поисковые запросы, результатом которых являются интерпретируемые 
в~терминах ресурсов библиотеки объекты.


    
    Если же отображение недопустимо, то, значит, для всех четырех 
вариантов нарушается отображение идентифицирующих атрибутов, т.\,е.\ 
невозможно извлечь интерпретируемые данные (пример: достанем всех 
персон по имени Лев, но не сможем понять, кто из них Толстой, если не 
отобразим идентифицирующие атрибуты, которые определены в~библиотеке, 
например, как Ф, И, О, ДР). 
    
    Если отображение невозможно, но допустимо, т.\,е.\ некоторый набор 
атрибутов можно отобразить на некоторые свойства, то, так как один набор 
атрибутов может соответствовать нескольким по\-ня\-ти\-ям/ти\-пам ресурсов 
в~библиотеке, невозможно будет идентифицировать тип ресурса 
извлекаемого объекта (например, если имеются понятия \textit{Мужчина} 
и~\textit{Женщина}, набор атрибутов которых одинаков, то, извлекая персону 
с~именем <<\textit{Джойс Кэрол Оутс}>> с~датой рождения 
<<16.06.1938>>, нельзя определить ее принадлежность к~понятию). Поэтому 
при построении отображения надо последовательно проходить этап 
построения возможных проекций ресурсов, а затем этап построения 
допустимых трансляций атрибутов этих ресурсов.
    
    Авторы не претендуют на идеальную модель отоб\-ра\-же\-ния в~любой 
источник, но, по крайней мере, имея адаптивную модель данных, всегда 
можно выполнить настройку таким образом, чтобы иметь возможность 
извлечь интересующие данные для определенного круга задач.
    
    Введя функцию интерпретации~$I_z$, которая по построенным 
отображениям~$T_i$ для источников данных~$D_i$ и~информационного 
ресурса~$\mathrm{IR}$ сопоставляет \textit{информационным объектам}, 
со\-от\-вет\-ст\-ву\-ющим этому ресурсу, объекты источников данных, можно 
построить множество связей этих объектов. Функция~$I_z$ называется 
моделью связанных данных источников данных и~LibMeta. 

Если не удается 
построить функцию интерпретации некоторого источника для хотя бы 
одного ресурса, то надо либо расширять модель LibMeta, либо источник 
отбрасывается как источник с~данными, не интерпретируемыми в~данной 
библиотеке. Таким образом, можно выявить скрытые связи между 
источниками данных как на уровне схем, так и~на уровне данных через 
ресурсы библио\-теки. 
{\looseness=1

}


     
\section{LibMeta~--- пример использования внешних онтологий 
при описании ресурсов и~подключении к~источникам}

    Существуют два пути настройки системы под конкретные понятия 
предметной области: (1)~воспользоваться формами создания 
и~редактирования ресурсов, что удобно, когда экземпляров ресурсов мало; 
(2)~воспользоваться имеющимися онтологиями, описывающими понятия 
предметной области. Если в~первом случае все достаточно тривиально, то на 
втором случае следует остановиться подробней.
{\looseness=1

}
    
    В~системе предусмотрена подсистема загрузки онтологии, 
представленной на языке описания онтологий OWL, для автоматического 
создания экземпляров информационных ресурсов. При загрузке онтологии 
можно указать, какие классы онтологии извлекаются из нее, при этом они 
становятся экземплярами класса \textit{информационный ресурс} в~\mbox{LibMeta}, 
а~их свойства преобразуются в~экземпляры атрибутов и~включаются в~один 
набор атрибутов для конкретного ресурса. Естественно, \mbox{LibMeta} не 
поддерживает всю семантику отношений и~ограничений, накладываемых на 
свойства и~классы в~OWL, но отображает основные свойства и~ограничения 
на свою схему. Этого достаточно для выполнения задач интеграции 
и~публикации данных в~облаке LOD. Так, в~наборе данных, полученном из 
<<Научного наследия России>> с~по\-мощью подсистемы харвестинга по 
протоколу OAI-PMH, вся исходная информация о~том, где находится 
исходный объект со всем своим описанием, сохраняется. С~по\-мощью 
\mbox{LibMeta} выполняется только провязывание данных с~данными из LOD 
и~пользователям предоставляется возможность организовывать их 
в~собственные коллекции, возможно, дополняя описания данных на свое 
усмотрение, добавляя новые теги или определяя более точно тематическую 
направленность, используя в~качестве базовых расширяемых понятий 
элементы из ГРНТИ. Перечислим правила отоб\-ра\-же\-ния внешней онтологии 
в~описание контента библиотеки:
    \begin{itemize}
\item классы онтологии становятся экземплярами класса 
\textit{информационный ресурс};
\item свойства класса онтологии становятся экземплярами класса 
\textit{атрибут};
\item все свойства, относящиеся к~одному классу, группируются 
в~\textit{наборы атрибутов};
\item для простых свойств в~качестве области значений атрибута указывается 
соответствующий тип (строка, дата и~т.\,д.);
\item по умолчанию все атрибуты относятся к~виду \textit{описательный} 
и~\textit{поисковый}, соответствующие настройки можно изменить 
в~дальнейшем через интерфейсы системы;
\item для сложных свойств, областью значений которых являются 
экземпляры некоторого класса, выбирается соответствующий ресурс; если 
ресурс не был загружен в~систему, то дальнейшие решения принимает 
пользователь, отвечающий за создание структуры контента библиотеки;
\item все однозначные свойства становятся однозначными атрибутами, для 
атрибутов многозначных свойств ставится пометка о~воз\-мож\-ности 
подключения нескольких значений;
\item иерархические связи между классами не сохраняются в~явном виде, но 
для ресурсов создается атрибут с~соответствующим назнанием 
<<\textit{вышестоящий объект}>> и~<<\textit{нижестоящий объект}>>, 
кото\-рый позволяет устанавливать подобные связи на уровне объектов 
(практически идентичны по смыслу связям из онтологии SKOS\footnote{{\sf  
https://www.w3.org/2004/02/skos.}} skos:broader, skos:narrower).
\end{itemize}

    Инверсивность, транзитивность и~симметричность свойств не 
отображаются явно в~описании ресурсов LibMeta, но для каждого 
информационного объекта можно всегда получить список ссылающихся на 
него объектов и~информацию о~том, посредством какого атрибута это 
делается. Фактически онтологическое описание классов сводится к~набору 
\textit{поисковых} и~\textit{описательных} атрибутов выделением среди них 
\textit{идентифицирующих}, используемых, например, в~задачах выявления 
дубликатов.
    
    Важно отметить, что все идентификаторы классов и~свойств онтологии 
сохраняются в~описании\linebreak соответствующих экземпляров ресурсов 
и~атрибутов с~помощью использования связи, опре\-де\-ля\-ющей их 
эквивалентность. Это позволяет в~дальнейшем при настройке отображения 
ресурса на\linebreak некоторый источник в~LOD, который в~своей схеме использует 
эти классы и~свойства, выполнять эту процедуру почти полностью 
автоматически. При конструировании структуры контента через интерфейсы 
также имеется возможность для каждого ресурса и~его атрибута указать 
соответствующие им URI\footnote[2]{{\sf https://tools.ietf.org/html/rfc3986.}} из 
общеиспользуемых онтологий. 
    
    Рассмотрим в~качестве примера онтологии, ко\-то\-рые широко 
распространены для описания основ\-ных типов ресурсов рассматриваемых 
понятий \textit{персона} и~\textit{пуб\-ли\-ка\-ция} в~сообществе LOD, 
и~оценим их описания на соответствие имеющимся в~наличии метаданным. 
Чаще всего эти онтологии содер\-жат десятки классов и~свойств и~являются 
избыточными для описания нужных объектов, выделяя подмножество 
необходимых классов и~свойств, используя для отображения лишь малую 
часть их свойств, необходимых для подключения к~источникам данных, 
которые они охватывают.

\vspace*{-4pt}
    
     \subsection*{AKT}
     
\vspace*{-2pt}
     
    Онтология AKT Reference Ontology, или кратко AKT\footnote[3]{ {\sf 
http://projects.kmi.open.ac.uk/akt/ref-onto.}} (доступна по адресу {\sf 
http://swl. slis.indiana.edu/repository/owl/aktportal.owl}), разработана в~целях 
унификации доступа к~библиографической информации в~2003~г. И~хотя 
проект был закрыт, данные AKT на сегодняшний момент представлены более 
чем в~200~источниках, таких как DBLP, Citeseer, CORDIS, NSF, EPSRC, 
ACM, IEEE и~др.
    
    Объединяет несколько онтологий; из них интерес представляет 
основная онтология Portal Ontology, которая содержит понятия для 
описания \textit{персон} и~\textit{пуб\-ли\-ка\-ций}. Данные разнородны 
и~опираются на очень узкие подмножества этой онтологии. Многие поля, 
имеющиеся в~этой богатой онтологии, остаются незаполненными при 
описании реальных данных.

\vspace*{-4pt}
     
     \subsection*{Dublin Core}
     
\vspace*{-2pt}
     
    Исторически Dublin Core представляет собой набор понятий, 
используемых для описания разнообразных типов ресурсов, из 
которых~15~являются обязательными для описания. Практически можно 
описать метаданные о~\textit{персонах} и~\textit{пуб\-ли\-ка\-ци\-ях} из 
рассматриваемого примера в~терминах этих понятий. Элементы Dublin Core часто 
повторно используются, дополняются и~конкретизируются в~других 
онтологиях. Охватывает огромное число источников, включая DBpedia.

\vspace*{-4pt}
    
     \subsection*{FOAF}
     
\vspace*{-2pt}
     
    Онтология FOAF\footnote[4]{{\sf http://xmlns.com/foaf/spec.}}
    (Friend-of-a-Friend)  уже является 
практически стандартом для описания людей и~их отношений с~другими 
ресурсами. Используется в~разнообразных контекстах и~может 
использоваться для описания в~любых сценариях с~участием персон. Часто 
также включается и~конкретизируется в~других онтологиях.

%\vspace*{-4pt}
    
     \subsection*{BIBO}
     
     
     %\vspace*{-2pt}
     
     Онтология BIBO\footnote[1]{ {\sf http://bibliontology.com.}} 
     (Bibliographic Ontology)
     предназначена для 
описания библиографических данных, включает в~себя понятия из других 
онтологий, таких как Dublin Core и~FOAF, расширяя и~конкретизируя их 
понятия, которые используются при описании ее классов. 
Содержит~38~видов документов, вклю\-ча\-ет понятия, необходимые для 
описания \textit{персон} и~\textit{пуб\-ли\-ка\-ций}. Можно представить 
описание собственных ресурсов в~терминах этой онтологии, ограничившись 
лишь частью ее терминов. Охватывает такие источники, как Британская 
национальная библиотека\footnote{{\sf http://www.bl.uk.}}, DBpedia и~т.\,д.

%\vspace*{-4pt}
    
     \subsection*{DBpedia}
     
     %\vspace*{-2pt}
     
    Онтология Dbpedia, разработанная в~рамках проекта Dbpedia, содержит 
большое число классов для описания самых разнообразных объектов, 
включая понятия \textit{пуб\-ли\-ка\-ция} и~\textit{персона}. Она также 
включает в~себя понятия из других онтологий, которые используются при 
описании ее классов. DBpedia является центральным узлом LOD и~связывает 
информацию из самых разных источников, которые ссылаются на нее. 


    
    Таблица~1 показывает отображение информационного ресурса 
<<Публикация>> в~термины источников данных, схема данных которых 
опирается на
 перечисленные онтологии. Если для указанных онтологий нет 
соответствующего класса, то название класса не указывается. Это всего лишь 
означает, что элементы этой онтологии могут использоваться в~другой 
онтологии, где они конкретизируются в~рамках используемого класса. Если 
не указывается свойство, значит, в~терминах этой онтологии нет такого 
свойства или близкого ему по смыслу. В~случае с~BIBO один из 
перечисленных классов определяет тип публикации. Поиск публикаций 
в~DBpedia представляется бессмысленным в~рассматриваемых примерах, 
поэтому в~табл.~1 информация из этой онтологии не включалась


    В табл.~2 представлено отображение информационного ресурса 
<<Персона>> в~термины источников данных, схема данных которых 
опирается на перечисленные онтологии. Например, для персон из DBpedia 
представлены элементы из собственного пространства имен, но DBpedia 
также включает и~FOAF-он\-то\-ло\-гию, поэтому можно было отобразить 
значения и~на пространство имен FOAF в~рамках источника данных DBpedia.


    
    Эта информация об отображении атрибутов \mbox{LibMeta} на свойства других 
онтологий также может быть включена в~описание каждого атрибута 
с~помощью использования соответствующей связи. Эта информация 
позволит быстро подключаться к~нужным источникам данных 
и~формировать описания объектов в~терминах нужной онтологии.
В~рас\-смат\-ри\-ва\-емом примере были подключены два источника данных~--- 
это Dbpedia и~данные о персонах из системы MathNet\footnote[3]{{\sf  
http://www.mathnet.ru.}}, выгруженные предварительно в~отдельное хранилище 
в~виде RDF-тро\-ек в~формате Dublin Core.

%\columnbreak
 %\vspace*{-12pt}


\end{multicols}

\begin{table*}[h]\small
%\vspace*{-12pt}
\begin{center}
\Caption{Элементы описания ресурса <<Публикация>>}
\vspace*{2ex}

\tabcolsep=2.4pt
\begin{tabular}{|l|c|c|c|c|}
\hline
\multicolumn{1}{|c|}{Libmeta}&AKT&Dublin Core&FOAF&BIBO\\
\hline
&\tabcolsep=0pt\begin{tabular}{c}Класс\\
Akt:Publication-Reference
\end{tabular}&\tabcolsep=0pt\begin{tabular}{c}Класс\\
Dc:bibliographicresource\end{tabular}&Класс
&\tabcolsep=0pt\begin{tabular}{c}Класс\\
Bibo:Article,\\ bibo:academicarticle, \\bibo:Proceedings\end{tabular}\\
\hline
Название&akt:has-title&dc:title&foaf:title&dc:title\\
\hline
Аннотация&akt:has-abstract&dc:description&&bibo:abstract\\
\hline
\tabcolsep=0pt\begin{tabular}{l}Дополнительное\\ заглавие\end{tabular}
&&&&bibo:shortTitle\\
\hline
Тип&akt:article-of-journal&dc:type&&\\
\hline
Язык&&dc:contributor&&dc:contributor \\
\hline
Автор&akt:has-author&dc:language&&dc:language\\
\hline
\tabcolsep=0pt\begin{tabular}{l}Исходная\\ страница\end{tabular}&akt:has-web-address&dc:source&foaf:homepage&foaf:homepage\\
\hline
Описание&akt:addresses-generic-area-of-interest&&&bibo:shortDescription\\
\hline
\end{tabular}
\end{center}
\vspace*{-6pt}
\end{table*}
    
    
    \begin{table*}\small %tabl2
\begin{center}
\Caption{Элементы описания ресурса <<Персона>>}
\vspace*{2ex}

\begin{tabular}{|l|c|c|c|c|c|}
\hline
\multicolumn{1}{|c|}{LibMeta}&AKT&Dublin Core&FOAF&BIBO&DBpedia\\
\hline
&Класс
&\tabcolsep=0pt\begin{tabular}{c}Класс\\
Dc:agent\end{tabular}&\tabcolsep=0pt\begin{tabular}{c}Класс\\
Foaf:agent, \\foaf:person \end{tabular}&\tabcolsep=0pt\begin{tabular}{c}Класс\\
Foaf:agent, \\foaf:person\end{tabular}&\tabcolsep=0pt\begin{tabular}{c}Класс\\
Dbo:person\end{tabular}\\
\hline
Фамилия&\tabcolsep=0pt\begin{tabular}{c}
akt:full-name,\\
akt:family-name
\end{tabular}&dc:title&\tabcolsep=0pt\begin{tabular}{c}foaf:name, \\
foaf:lastname, \\
foaf:family\_name\end{tabular}&foaf:family\_name&dbo:birthName\\
\hline
Имя&\tabcolsep=0pt\begin{tabular}{c}
akt:full-name,\\
akt:given-name
\end{tabular}&dc:title&\tabcolsep=0pt\begin{tabular}{c}
foaf:name, \\
foaf:given\_name
\end{tabular}&foaf:given\_name&dbo:birthName\\
\hline
Отчество&akt:full-name&dc:title&\tabcolsep=0pt\begin{tabular}{c}
foaf:name, \\
foaf:surname
\end{tabular}&&dbo:birthName\\
\hline
Дата рождения&&dc:date&foaf:birthData&&dbo:birthDate\\
\hline
Место рождения&&&&&dbo:birthPlace\\
\hline
Биография&&dc:description&&&dbo:abstract\\
\hline
Деятельность&&dc:description&&&dbo:occupation\\
\hline
\end{tabular}
\end{center}
\end{table*}

\begin{multicols}{2}





    
\section{LibMeta~--- роль пользователей в~системе}

    Несомненно, самыми главными действующими лицами в~любой 
библиотеке являются ее пользователи, и~LibMeta не исключение. 
Пользователи LibMeta делятся на несколько категорий в~соответствии со 
своими ролями: 
    \begin{enumerate}[(1)]
\item администраторы контента библиотеки; 
\item редакторы предметной области; 
\item администраторы источников данных; 
\item редакторы информационных объектов; 
\item простые пользователи.
\end{enumerate}




    Администраторы контента библиотеки отвечают за создание 
информационных ресурсов, атрибутов и~их наборов и~получают доступ ко 
всей функциональности системы. Редактор предметной\linebreak об\-ласти имеет право 
на редактирование тезауруса и~определение основных коллекций системы.\linebreak 
Адми\-нистраторы источников данных отвечают за их подключение 
и~настройку отображения. Редак\-тор информационных объектов может 
создавать, редактировать и~удалять любой информационный объект. 
В~отличие от всех остальных ролей, для редактора информационных 
объектов роль определяет доступ к~конкретным функциям подсистемы 
рабо\-ты с~объектами (редактирование, создание, и~удаление), а~не просто 
доступ к~функциональ\-ности для сопровождения отдельных объектов. 
Зарегистрированные простые пользователи могут описывать свою об\-ласть 
интересов на основе терминов тезауруса предметной об\-ласти, очерчивая тем 
самым интересующий их круг информационных объектов. Эти пользователи 
также могут добавлять в~тезаурус уточняющие об\-ласть их интересов 
термины, которые не отображаются для остальных пользователей, добавлять 
собственные объекты в~свою коллекцию в~рамках своей об\-ласти интересов, 
сохранять свои результаты поиска по источникам данных. Задание об\-ласти 
интересов пользователя позволяет группировать пользователей со сходными 
интересами, строить связи между ними, анализируя круг 
используемых терминов, добавленных объектов, и~давать 
рекомендации на основе этих связей. Таким образом, можно отслеживать 
и~выделять взаимосвязи между разными об\-лас\-тя\-ми интересов. Понятно, что 
один и~тот же пользователь может обладать несколькими ролями 
и~выступать, например, в~качестве редактора предметной об\-ласти 
и~администратора источников данных. Но при этом любой пользователь 
системы независимо от роли получает доступ к~функциям поиска 
и~навигации по объектам системы.
    
\begin{figure*} %fig6
 \vspace*{1pt}
\begin{center}
\mbox{%
\epsfxsize=163.09mm
\epsfbox{ser-6.eps}
}
\end{center}
\vspace*{-9pt}
\Caption{Пользователи}
\vspace*{6pt}
\end{figure*}

    На рис.~6 отображены роли пользователей и~в~виде четырехугольников 
очерчивается их область влияния для рассмотренного выше примера 
сконструированного контента.
    
\section{Заключение}

    Разрабатывая модель информационной системы LibMeta, авторы хотели 
получить гибкую сис\-тему интеграции различных типов ресурсов 
с~возможностью интеграции с~внешними системами. Одним из 
определяющих условий было отсутствие тре\-бования специальной 
подготовки у~простых пользователей системы. Основные идеи при 
разработке системы были позаимствованы из концепции адап\-тив\-ных 
моделей данных, разработанной в~\mbox{1990-х}~гг.~[10, 11]. Эта модель данных 
подходит для определенного круга задач, для решения которых нужно 
разрабатывать довольно сложные частные модели. К~таким задачам 
относится ставшая классической задача интеграции данных из источников 
с~разными моделями. Внесение изменений в~интеграционную модель чаще 
всего выполняется на программном уровне, требуя значительных 
усилий от разработчиков для внедрения изменений. Применение адаптивной 
модели позволяет понизить сложность как самой модели данных, так 
и~разрабатываемых на их основе систем, в~которых попутно решается задача 
создания динамических (адаптивных) пользовательских интерфейсов. 
    
    Применение этой модели данных делает воз\-мож\-ной динамическую 
трансформацию и~ин\-терпретацию модели данных в~приложении, ре\-ша\-ющем 
задачу интеграции данных, позволяя\linebreak настраивать используемые решения под 
определенную предметную область. В~таких задачах часто меняются 
требования к~модели данных, меняется детализация схемы, детализируя или, 
наоборот, обобщая ее описание. Реализация приложений для конечных 
пользователей под эти требования занимает больше времени, чем хотелось 
бы. Поиск решений для такого рода задач на более высоком уровне 
абстракции привел к~появлению концепции адап\-тив\-ной модели данных, 
в~которой декларируется лишь стиль моделирования данных для 
приложений. Получаемые модели более абстрактны, состоят из меньшего 
числа понятий с~более простыми связями и~не привязаны к~определенным 
предметным областям. Поведенческие моменты определяются операциями 
создать, сохранить и~т.\,д., ролями пользователя, которые отражают 
условия доступа к~этим операциям. Пользовательский интерфейс 
представляется адаптивным в~соответствии с~мо\-делью данных и~использует 
для своего вос\-про\-из\-ве\-де\-ния/по\-стро\-ения заранее определенные 
паттерны уровня представления. В~результате применения этой модели, 
когда меняется ее структура, система немедленно подстраивается под эти 
изменения.
    
    В статье был приведен пример конструирования контента семантической 
библиотеки для авторов и~их публикаций в~рамках библиотеки LibMeta на 
основе данных из системы <<Научное наследие России>>. На имеющемся 
наборе данных, пред\-став\-лен\-ном примерно~7000~публикациями и~их 
авторами (около~1300~персон), было проведено тестовое исследование. 
Из~1300~авторов оказалась пред\-став\-ле\-на в~DBPedia примерно треть, и~около 
половины из них были представлены в~VIAF\footnote{{\sf http://viaf.org.}}
(Virtual International Authority File). Часть ссылок на данные 
VIAF была получена из набора данных DBPedia, 
другая часть была непосредственно извлечена из самого VIAF, 
подключенного как источник данных. 
    
    Во втором примере, сконструированном для тезауруса ОДУ 
из~10\,000~публикаций базы ЕНИП, было загружено~100~подходящих по 
тематике. Загруженные публикации были размечены ключевыми словами 
тезауруса, всего было получено~789~элементов разметки.
    
    Провести связывание с~данными из источников, охваченных 
онтологиями AKT, оказалось невозможным из-за специфики данных, но эти 
источники были подключены к~LibMeta в~качестве тестовых для 
использования их в~качестве источников для поиска. В~статью не вошли 
задачи поиска по семантическим тегам или наборам ключевых слов не 
только по строго заданным правилам отоб\-ра\-же\-ния ресурсов, а также поиск 
по формулам в~математических данных. Предполагается 
осветить это направление работ в~рамках предложенной системы 
в~дальнейших работах.
    
{\small\frenchspacing
 {%\baselineskip=10.8pt
 \addcontentsline{toc}{section}{References}
 \begin{thebibliography}{99}
\bibitem{1-ser}
\Au{Gruber T.\,R.} A translation approach to portable ontologies~// 
Knowl. Acquis., 
1993. Vol.~5. No.\,2. P.~199--220. 
\bibitem{2-ser}
Semantic digital libraries~/ Eds. S.\,R.~Kruk, B.~McDaniel.~--- Berlin--Heidelberg:  
Springer-Verlag, 2009. 245~p.
\bibitem{3-ser}
\Au{Антопольский~А.\,Б., Каленкова~А.\,А., Каленов~Н.\,Е., Серебряков~В.\,А., 
Сотников~А.\,Н.} Принципы разработки интегрированной системы для научных 
библиотек, архивов и~музеев~// Информационные ресурсы России, 2012. №\,1. С.~2--6.
\bibitem{4-ser}
\Au{Bizer C., Heath~T., Berners-Lee~T.} Linked data~--- the story so far~// 
Int. J.~Semant. 
Web Inf. Syst., 2009. Vol.~5. No.\,3. P.~1--22.
\bibitem{5-ser}
\Au{Серебряков~В.\,А., Атаева О.\,М.} Основные понятия формальной 
модели семантических библиотек и~формализация процессов интеграции в~ней~// 
Программные продукты и~системы, 2015. №\,4. С.~180--187.
\bibitem{6-ser}
\Au{Серебряков В.\,А., Атаева~О.\,М.} Персональная циф\-ровая библиотека LibMeta как 
среда интеграции связан\-ных открытых данных~// Электронные биб\-лио\-те\-ки: 
перспективные методы и~технологии, электронные коллекции: Тр. XVI Всеросс. научной 
конф. RCDL'2014.~--- Дубна: ОИЯИ, 2014. С.~66--71.
\bibitem{7-ser}
\Au{Каленов Н.\,Е., Савин~Г.\,И., Сотников~А.\,Н.} Электронная библиотека <<Научное 
наследие России>>~// Информационные ресурсы России, 2009. №\,2(108). С.~19--20.
\bibitem{8-ser}
\Au{Моисеев Е.\,И., Муромский~А.\,А., Тучкова~Н.\,П.} Тезаурус  
ин\-фор\-ма\-ци\-он\-но-по\-ис\-ко\-вый по предметной области <<обыкновенные 
дифференциальные уравнения>>.~--- М.: МАКС Пресс, 2005. 116~с.
\bibitem{9-ser}
\Au{Бездушный А.\,Н., Бездушный~А.\,А., Серебряков~В.\,А., Филиппов~В.\,И.} Интеграция 
метаданных Единого Научного Информационного Пространства РАН.~--- М.: ВЦ РАН, 
2006.  238~с.
\bibitem{10-ser}
\Au{Yoder J.\,W., Balaguer~F., Johnson~R.} Architecture and design of adaptive  
object-model~// Adaptive Object Model, 2000. 11~p. {\sf 
http://www.adaptiveobjectmodel.com/ OOPSLA2001/AOMIntriguingTechPaper.pdf}.
\bibitem{11-ser}
\Au{Welick L., Yode~J.\,W., Wirfs-Broc~R.} Adaptive object-model builder~// Adaptive Object 
Model, 2009. 8~p. {\sf http://joeyoder.com/PDFs/04welicki.pdf}.
 \end{thebibliography}

 }
 }

\end{multicols}

\vspace*{-3pt}

\hfill{\small\textit{Поступила в~редакцию 01.12.16}}

\vspace*{8pt}

%\newpage

%\vspace*{-24pt}

\hrule

\vspace*{2pt}

\hrule

%\vspace*{8pt}


\def\tit{PERSONAL SEMANTIC OPEN DIGITAL LIBRARY 
LibMeta. CONSTRUCTION~OF~THE~CONTENT. 
INTEGRATION~WITH~LOD~SOURCES}

\def\titkol{Personal semantic open digital library 
LibMeta. Construction 
of~the~content. Integration with~LOD sources}

\def\aut{O.\,M.~Ataeva and V.\,A.~Serebryakov}

\def\autkol{O.\,M.~Ataeva and V.\,A.~Serebryakov}

\titel{\tit}{\aut}{\autkol}{\titkol}

\vspace*{-9pt}


\noindent
A.\,A.~Dorodnicyn Computing Center, Federal Research Center ``Computer 
Science and Control'' of the Russian Academy of Sciences, 44-2~Vavilov Str., 
Moscow 119333, Russian Federation



\def\leftfootline{\small{\textbf{\thepage}
\hfill INFORMATIKA I EE PRIMENENIYA~--- INFORMATICS AND
APPLICATIONS\ \ \ 2017\ \ \ volume~11\ \ \ issue\ 2}
}%
 \def\rightfootline{\small{INFORMATIKA I EE PRIMENENIYA~---
INFORMATICS AND APPLICATIONS\ \ \ 2017\ \ \ volume~11\ \ \ issue\ 2
\hfill \textbf{\thepage}}}

\vspace*{3pt}



\Abste{Semantic technologies development has brought digital libraries to the 
level where a~meaningful representation of the content of digital libraries came to 
the forefront. At the same time, it is necessary to limit it in terms of a certain 
subject area. The paper describes the libraries content construction with 
a~thesaurus supporting the domain terminology within the developed system 
LibMeta. The personal semantic open digital library LibMeta provides the 
functionality of the construction of the library content in accordance with the 
specific requirements. LibMeta supports users working with resources of digital 
libraries and their collections in a~certain subject area. One needs just to make the 
initial setup of the system for a specific subject area. For the description of a 
subject area, the system uses its limited terminology collected in a thesaurus. The 
domain used as an example is a highly specialized thesaurus of ordinary 
differential equations.}

\KWE{semantic library; data model; ontology; data sources; search in LOD}

\DOI{10.14357/19922264170210} 

%\vspace*{-18pt}

\Ack
\noindent
The work was supported by the Russian Foundation for
Basic Research (project 14-07-00058~A).



%\vspace*{3pt}

  \begin{multicols}{2}

\renewcommand{\bibname}{\protect\rmfamily References}
%\renewcommand{\bibname}{\large\protect\rm References}

{\small\frenchspacing
 {%\baselineskip=10.8pt
 \addcontentsline{toc}{section}{References}
 \begin{thebibliography}{99}
\bibitem{1-ser-1}
\Aue{Gruber, T.\,R.} 1993. A~translation approach to portable ontologies. 
\textit{Knowl. Acquis.} 5(2):199--220. 
\bibitem{2-ser-1}
Kruk, S.\,R., and B.~McDaniel, eds. 2009.
\textit{Semantic digital libraries}. Berlin--Heidelberg: Springer-Verlag. 245~p.
\bibitem{3-ser-1}
\Aue{Antopolsky, A.\,B., A.\,A.~Kalenkova, N.\,E.~Kalenov, 
V.\,A.~Serebryakov, and A.~Sotnikov}. 2012. Printsipy razrabotki 
integrirovannoy sistemy dlya nauchnykh bibliotek, arkhivov i~muzeev [Principles 
for the development of an integrated system for academic libraries, archives and 
museums]. \textit{Informatsionnye resursy Rossii} [Information Resources of 
Russia] 1:2--6. 
\bibitem{4-ser-1}
\Aue{Bizer, C., T.~Heath, and T.~Berners-Lee.} 2009. Linked data~--- the story 
so far. \textit{Int. J.~Semant. Web Inf. Syst.} 5(3):1--2.
\bibitem{5-ser-1}
\Aue{Serebryakov, V.\,A., and  O.\,M.~Ataeva.} 2015. Osnovnye ponyatiya dlya 
postroeniya formal'noy modeli se\-man\-ti\-che\-skikh bibliotek i~opisaniya protsessov 
integratsii v~ney [The basic concepts for building a~formal model of semantic 
libraries and description of the integration processes in it]. \textit{Programmnye 
produkty i~sistemy} [Software and Systems] 4:180--187.
\bibitem{6-ser-1}
\Aue{Serebryakov, V.\,A., and O.\,M.~Ataeva.} 2014. Personal'naya tsifrovaya 
biblioteka LibMeta kak sreda integratsii svyazannykh otkrytykh dannykh 
[Personal digital library LibMeta as an integration environment of linked data]. 
\textit{RCDL Proceedings}.  66--71.
\bibitem{7-ser-1}
\Aue{Kalyonov, N.\,E., G.\,I.~Savin, and A.\,N.~Sotnikov.} 2009. 
{Elektronnaya biblioteka ``Nauchnoe nasledie Rossii''} [Electronic library 
``The scientific heritage of Russia''].
\textit{Informacionnye resursy Rossii} [Information Resources of Russia] 
2:19--20.
\bibitem{8-ser-1}
\Aue{Moiseev, E.\,I., A.\,A.~Muromskij, and N.\,P.~Tuchkova.} 2005. 
\textit{Tezaurus informatsionno-poiskovyy po predmetnoy oblasti 
``obyknovennye differentsial'nye uravneniya''} [Information search thesaurus of 
subject area ``ordinary differential equations'']. Moscow: MAKS Press. 116~p.
\bibitem{9-ser-1}
\Aue{Bezdushnyj, A.\,N., A.\,A.~Bezdushnyj, V.\,A.~Serebryakov, and 
V.\,I.~Filippov.} 2006. \textit{Integratsiya metadannykh Edinogo Nauchnogo 
Informatsionnogo Prostranstva RAN} [The integration of metadata for common 
scientific information space of RAS].~--- Moscow: Computing Centre of the 
Russian Academy of Sciences. 238~p.
\bibitem{10-ser-1}
\Aue{Yoder, J.\,W., F.~Balaguer, and R.~Johnson.} 2000. Architecture and 
design of adaptive object-model. \textit{Adaptive Object Model.} 
Available at: {\sf http://www.adaptiveobjectmodel. com/OOPSLA2001/AOMIntriguingTechPaper.pdf} (accessed 
April~17, 2017).
\bibitem{11-ser-1}
\Aue{Welick, L., J.\,W.~Yode, and R.~Wirfs-Broc}. 2009. Adaptive  
object-model builder. \textit{Adaptive Object Model}. Available at: {\sf 
http://joeyoder.com/PDFs/04welicki.pdf} (accessed April~17, 2017). 
\end{thebibliography}

 }
 }

\end{multicols}

\vspace*{-3pt}

\hfill{\small\textit{Received December 1, 2016}}

\Contr

\noindent
\textbf{Ataeva Olga M.} (b.\ 1978)~--- junior scientist, A.\,A.~Dorodnicyn 
Computing Center, Federal Research Center ``Computer Science and Control'' 
of the Russian Academy of Sciences, 44-2~Vavilov Str., Moscow 119333, 
Russian Federation; \mbox{oli@ultimeta.ru}

\vspace*{3pt}

\noindent
\textbf{Serebryakov Vladimir A.} (b.\ 1946)~--- Doctor of Science in physics 
and mathematics, professor, Head of Department, A.\,A.~Dorodnicyn Computing 
Center, Federal Research Center ``Computer Science and Control'' 
of the Russian Academy of Sciences, 44-2~Vavilov Str., Moscow 119333, 
Russian Federation; \mbox{serebr@ultimeta.ru}
\label{end\stat}


\renewcommand{\bibname}{\protect\rm Литература} 