\def\stat{ushakovi}

\def\tit{ОДНОКАНАЛЬНАЯ СИСТЕМА ОБСЛУЖИВАНИЯ С~ЗАВИСИМЫМИ ИНТЕРВАЛАМИ ВРЕМЕНИ 
МЕЖДУ~ПОСТУПЛЕНИЯМИ ТРЕБОВАНИЙ$^*$}

\def\titkol{Одноканальная система обслуживания с~зависимыми интервалами времени между поступлениями требований}

\def\aut{В.\,Г.~Ушаков$^1$,  Н.\,Г.~Ушаков$^2$}

\def\autkol{В.\,Г.~Ушаков,  Н.\,Г.~Ушаков}

\titel{\tit}{\aut}{\autkol}{\titkol}

\index{Ушаков В.\,Г.}
\index{Ушаков Н.\,Г.}
\index{Ushakov V.\,G.}
\index{Ushakov N.\,G.}


{\renewcommand{\thefootnote}{\fnsymbol{footnote}} \footnotetext[1]
{Работа выполнена при финансовой поддержке РФФИ (проект 15-07-02354).}}


\renewcommand{\thefootnote}{\arabic{footnote}}
\footnotetext[1]{Факультет вычислительной математики и~кибернетики Московского 
государственного университета им.\ М.\,В.~Ломоносова; 
Институт проб\-лем информатики Федерального исследовательского центра 
<<Информатика и~управ\-ле\-ние>> Российской академии наук, \mbox{vgushakov@mail.ru}}
\footnotetext[2]{Институт проблем технологии микроэлектроники и~особочистых 
материалов Российской академии наук;
Норвежский на\-уч\-но-тех\-но\-ло\-ги\-че\-ский университет, Тронхейм, \mbox{ushakov@math.ntnu.no}}

\vspace*{6pt}


\Abst{Изучена одноканальная система массового обслуживания с~бесконечным 
числом мест для ожидания и~произвольным распределением времени обслуживания. 
Входящий поток требований является пуассоновским потоком со случайной интенсивностью. 
Текущее значение интенсивности выбирается из конечного множества с~заданными 
вероятностями в~момент начала отсчета времени до следующего поступления требования. 
Последовательные интенсивности образуют цепь Маркова специального вида. Частными 
случаями таких потоков являются гиперэкспоненциальные потоки и~потоки, возникающие 
при исследовании байесовских моделей систем обслуживания с~дискретным априорным 
распределением. Рассматриваемые потоки хорошо описывают работу систем массового 
обслуживания, функци\-о\-ни\-ру\-ющих в~случайной среде с~конечным множеством различных 
состояний и~марковской зависимостью между ними. Кроме того, такими потоками можно 
достаточно точно аппроксимировать реальные потоки в~сетях передачи данных. Исследовано 
поведение длины очереди в~нестационарном режиме.}

\KW{пуассоновский поток; случайная интенсивность; гиперэкспоненциальный поток; 
цепь Маркова; одноканальная система; длина очереди}

\vspace*{6pt}

\DOI{10.14357/19922264170212} 


\vskip 10pt plus 9pt minus 6pt

\thispagestyle{headings}

\begin{multicols}{2}

\label{st\stat}


\section{Введение} 

Статистический анализ трафика в~различных телекоммуникационных сетях 
показывает ярко выраженную зависимость интервалов времени между поступлениями 
пакетов данных. В последние годы появилось много работ, в~которых при 
построении математических моделей учитывается это явление. При моделировании 
процесса передачи информации важно концентрироваться на тех
характеристиках трафика, которые могут быть эффективно оценены из реальных 
данных и~поддаются физической интерпретации. С этой точки
зрения перспективными являются модели, в~которых те или иные параметры входящих 
потоков предполагаются случайными величинами, связанными регрессионной 
зависимостью небольшого порядка. 

В~работах~[1--6] содержатся различные постановки 
задач в~этом направлении и~приведена обширная библиография. Потоки, рассматриваемые 
в~настоящей статье, также относятся к~потокам с~регрессионной зависимостью 
параметров, в~качестве которых выступают интенсивности. Кроме того, рассматриваемый 
класс потоков хорошо описывает работу сис\-тем обслуживания, функционирующих в~случайной 
среде. Множество состояний среды конечно, и~последовательные состояния связаны 
марковской зависимостью.



\section{Входящий поток}

В изучаемой системе обслуживания входящий поток имеет следующую структуру. 
Интервал времени до поступления первого требования~$z_1$ и~интервалы между 
поступлениями $(n-1)$-го и~$n$-го требований~$z_n$ имеют показательное распределение
со случайным параметром $a^{(n)}$, $n\hm=1, 2,\ldots$ Значение~$a^{(n)}$ выбирается 
непосредственно перед началом промежутка~$z_n$, причем $\mathbf{P}(a^{(1)}\hm=
a_j)=c_j$, $a_i\hm\ne a_j$, $i\hm\ne j$, $c_j\hm>0$, $j\hm=1,\ldots,N$, 
$\sum\nolimits_{j=1}^N c_j\hm=1$ и~$a^{(n)}\hm=\alpha a^{(n-1)}+(1-\alpha)b^{(n)}$, 
где $b^{(n)}$, $n=1,2,\ldots,$~--- последовательность независимых и~независящих 
от последовательности $a^{(n)}$, $n\hm=1,2,\ldots,$ одинаково 
распределенных случайных величин, распределение которых такое же, как у~$a^{(1)}$, 
а~$\alpha$ не зависит от~$a^{(n)}$ и~$b^{(n)}$, $n\hm=1,2,\ldots$, $\alpha\hm=\left\{
\begin{array}{cc}
1,&p,\\0,&1-p.
\end{array}
\right.
$

Легко видеть, что
$$
\mathbf{P}(z_n<t)= \sum\limits_{j=1}^Nc_j \left(1-e^{-a_jt}\right)\,;
$$

\vspace*{-12pt}

\noindent
\begin{multline*}
\mathbf{P}(z_n<t_1,z_{n+1}<t_2)={}\\
{}=(1-p)
\sum\limits_{j=1}^Nc_j\left(1-e^{-a_jt_1}\right)\sum\limits_{k=1}^Nc_k\:\left(1-e^{-a_kt_2}\right)+{}
\\
{}+p\sum\limits_{k=1}^Nc_k\left(1-e^{-a_kt_1}\right)\left(1-e^{-a_kt_2}\right),\ n=1,\:2,\ldots
\end{multline*}
В частности, при $p=0$ входящий поток будет гиперэкспоненциальным. При $p\hm=1$ 
получается система, в~которой в~начальный
момент времени случайно выбирается значение интенсивности из множества 
$\{a_1,\ldots,a_N\}$ с~вероятностями $c_1,\ldots,c_N$ и~в~дальнейшем система 
функционирует как система с~пуассоновским входящим потоком с~выбранной интенсивностью. 
В~статье этот случай рассматриваться не будет.

Известно, что для любых $\mu\hm>0$ и~$\sigma\hm>\mu$ существует гиперэкспоненциальный 
поток второго порядка $(N\hm=2)$, у~которого математическое ожидание и~дисперсия 
интервалов между поступлениями требований равны~$\mu$ и~$\sigma^2$. Коэффициент
корреляции двух соседних интервалов для рас\-смат\-ри\-ва\-емых в~статье потоков при $N\hm=2$ 
со\-став\-ля\-ет $({p}/{2})\left(1-\left({\mu}/{\sigma}\right)^2\right)$. Таким образом, 
появляется возможность не только подогнать первые два момента интервалов между 
поступления реального потока, но и~учесть их зависимость.

\section{Обозначения и~определения}

Пусть $B(x)$ и~$b(x)$~--- соответственно функция распределения и~плотность
 распределения времен обслуживания требований.  Обозначим
  $$
  \beta(s)=\int\limits_0^{\infty}e^{-sx}b(x)\,dx;\quad
   \eta(x)=\fr{b(x)}{1-B(x)}\,.
$$

Введем следующие случайные процессы:
\begin{description}
\item[\,] $L(t)$~--- число требований в~системе в~момент времени~$t$;
\item[\,] 
$j(t)$~--- процесс с~состояниями $1,\ldots,N$ такой, что $j(t)\hm=j,$ если  
в~момент времени~$t$ интенсивность входящего потока есть~$a_j$;
\item[\,] 
$x(t)$ --- время, прошедшее с~начала обслуживания требования, находящегося на приборе,
 до момента~$t$, если $L(t)\hm>0,$ и~$x(t)\hm=0,$ если $L(t)\hm=0$.
 \end{description}

Трехмерный случайный процесс $(L(t),\:x(t),\:j(t))$ является однородным 
марковским процессом. Положим
\begin{align*}
P_j(n,x,t)&={}\\
&\hspace*{-28pt}{}=\fr{\partial}{\partial x}\,\mathbf{P}(L(t)=n\,,
\ 
j(t)=j\,,\
\nu(t)=1,\:x(t)<x)\,,\\
&\hspace*{43mm}n>0\,,\enskip x\geqslant 0\,,
\\
p_j(z,x,s)&=\sum\limits_{n=1}^{\infty}z^n\int\limits_0^{\infty}e^{-st}P_j(n,x,t)\,dt\,;
\\
P_{0j}(t)&=\mathbf{P}(L(t)=0\,, \ j(t)=j)\,;\\
p_{0j}(s)&=\int\limits_0^{\infty}e^{-st}P_{0j}(t)\,dt\,,\enskip
 j=1,\ldots,N\,.
\end{align*}
Справедливы следующие леммы.

\smallskip

\noindent
\textbf{Лемма1.}\
\textit{Уравнение
\begin{equation}
\label{1}
(1-p)z\sum\limits_{j=1}^{N}\fr{c_ja_j}{\mu+a_j(1-pz)}=1
\end{equation}
имеет $N$ непрерывных в~области $|z|\hm\leqslant 1$ решений 
$\mu_1(z),\ldots,\mu_N(z)$, причем}:
\begin{itemize}
\item[(а)] \textit{только одна из этих функций  обращается в~нуль в~точке} $z\hm=1$;

\item[(б)] \textit{для всех $j=1,\ldots,N$ при $|z|< 1$ справедливы 
неравенства} $\mathrm{Re} (\mu_j(z))\hm< 0$;

\item[(в)]  \textit{функции $\mu_i(z)\ne\mu_j(z)$ при $i\ne j$}.
\end{itemize}

Не ограничивая общности, будем считать, что $\mu_1(1)\hm=0$.
Обозначим
$$
\alpha_k(z)=\prod\limits_{j\ne k}(\mu_k(z)-\mu_j(z))\,;\enskip \delta_{ij}=\begin{cases}
1,&\ i=j\,,\\
0,&\ i\ne j\,.
\end{cases}
$$

\noindent
\textbf{Лемма~2.}\
\textit{При каждом $k,\ k=1,\ldots,N,$ уравнение}
$$
z=\beta\left(s-\mu_k(z)\right)
$$
\textit{имеет в~области $\mathrm{Re} s\hm>0$ единственное решение $z\hm=z_k(s)$ такое, 
что $|z_k(s)|\hm<1.$}

\smallskip

Положим
$$
\mu_k^{(\ast)}(s)=\mu_k\left(z_k(s)\right)\,.
$$

\section{Распределение длины очереди}

Прямые уравнения Колмогорова для распределения процесса $(L(t)$, $x(t)$,
$j(t))$ при $x\hm>0$ имеют вид:
\begin{multline}
\label{3}
\fr{\partial P_j(n,x,t)}{\partial t}+\fr{\partial P_j(n,x,t)}{\partial x}={}\\
{}=-(a_j+\eta(x))P_j(n,x,t)+(1-\delta_{n,1})\times
\\
\times\left(\vphantom{\sum\limits_{k=1}^N }
p\, a_jP_j(n-1,x,t)+{}\right.\\
\left.{}+(1-p) c_j\sum\limits_{k=1}^N a_kP_k(n-1,x,t)\right)\,,
\end{multline}
а краевые условия при $x\hm=0$ и~уравнения для~$P_{0j}(t)$:
\begin{multline}
\label{4}
P_j(n,0,t)=\int\limits_0^{\infty}P_j(n+1,x,t)\eta(x)\,dx+
\delta_{n,1}\times{}\\
{}\times \left(p\, a_j P_{0j}(t)+
(1-p)c_j\sum\limits_{k=1}^Na_kP_{0k}(t)\right)\,;
\end{multline}
\begin{equation}
\label{5}
\fr{\partial P_{0j}(t)}{\partial t}=-a_jP_{0j}(t)+
\int\limits_0^{\infty}P_j(1,x,t)\eta(x)\,dx\,.
\end{equation}
Начальные условия при $t\hm=0$ имеют вид:
$$
P_j(n,x,0)=0,\enskip n>0,\enskip P_{0j}(0)=c_j,\enskip j=1,\ldots,N.
$$
Переходя в~\eqref{3}--\eqref{5} к~производящим функциям и~преобразованиям Лапласа, 
получаем:
\begin{multline}
\label{6}
\fr{\partial p_j(z,x,s)}{\partial x}={}\\
{}=-\left(s+a_j-p\,a_jz+\eta(x)\right)p_j(z,x,s)+{}\\
{}+
(1-p)c_jz\sum\limits_{k=1}^Na_kp_k(z,x,s)\,;
\end{multline}

\vspace*{-12pt}

\noindent
\begin{multline}
\label{7}
p_j(z,0,s)=z^{-1}\int\limits_0^{\infty}p_j(z,x,s)\eta(x)\,dx+c_j-{}\\
{}-(s+a_j)p_{0j}(s)+{}\\
{}+
z\left(p\, a_jp_{0j}(s)+(1-p) c_j\sum\limits_{k=1}^Na_kp_{0k}(s)\right),\\  
j=1,\ldots,N.
\end{multline}
Общее решение системы дифференциальных уравнений~\eqref{6} имеет вид:
\begin{multline}
\label{8}
p_j(z,x,s)=(1-B(x))c_j\times{}\\
{}\times \sum\limits_{k=1}^N\fr{\gamma^{(k)}(z,s)}{\mu_k(z)+a_j(1-pz)}\,
e^{-(s-\mu_k(z))x}\,,
\end{multline}
где функции  $\gamma^{(k)}(z,s)$ определяются из краевых условий.  Подставляя~\eqref{8} 
в~\eqref{7}, получаем:
\begin{equation}
\label{9}
\sum\limits_{k=1}^N\fr{1-z^{-1}\beta(s-\mu_k(z))}{\mu_k(z)+a_j(1-p z)}\,
\gamma^{(k)}(z,s)=f_j(z,s)\,,
\end{equation}
где
\begin{multline*}
f_j(z,s)=1-\left(s+a_j-a_jpz)\right)c_j^{-1}p_{0j}(s)+{}\\
{}+
(1-p)z\sum\limits_{k=1}^Na_kp_{0k}(s)\,.
\end{multline*}
Решая систему уравнений~\eqref{9} относительно  $\gamma^{(k)}(z,s)$, находим:
\begin{multline*}
%\label{10}
\left(1-z^{-1}\beta(s-\mu_k(z))\right)\gamma^{(k)}(z,s)=\alpha_k^{-1}(z)\times
\\
\times\sum\limits_{l=1}^Nf_l(z,s)
\left(\prod\limits_{j=1}^N \left(\mu_k(z)+a_j(1-pz)\right)
\left(\mu_j(z)+{}\right.\right.\\
\left.\left.{}+a_l(1-pz)\right)
\vphantom{\prod\limits_{j=1}^N}\!\!
\right)\!\!\Bigg/\!\!
\left(\prod\limits_{\nu\ne l}(1-pz)\left(
a_l-a_{\nu}\right)\left(\mu_k(z)+{}\right.\right.\\
\left.\left.{}+a_l(1-pz)\right)
\vphantom{\prod\limits_{j=1}^N}\!\!\right),\enskip
 k=1,\ldots,N\,.
\end{multline*}

Так как $\mu_1(z),\ldots,\mu_N(z)$ являются решениями уравнения~\eqref{1},
\begin{multline*}
\prod\limits_{j=1}^N(\mu+a_j(1-pz))-{}\\
{}-(1-p)z\sum\limits_{l=1}^N\!
c_la_l\prod\limits_{j\ne l}(\mu+a_j(1-pz))=
\prod\limits_{\nu=1}^N(\mu-\mu_{\nu}(z))\,.\hspace*{-3.1708pt}
\end{multline*}
Подставляя сюда $\mu\hm=-a_l(1\hm-pz)$, получаем:
$$
\fr{\prod\nolimits_{j=1}^N(\mu_j(z)+a_l(1-pz))}{\prod\nolimits_{\nu\ne l}((1-pz)
(a_l-a_{\nu}))}=(1-p)zc_la_l\,.
$$
Следовательно,
\begin{multline}
\label{11}
\left(1-z^{-1}\beta\left(s-\mu_k(z)\right)\right)\gamma^{(k)}(z,s)=
\alpha_k^{-1}(z)(1-)z\times{}\\
{}
\times\prod\limits_{j=1}^N(\mu_k(z)+a_j(1-pz))\sum\limits_{l=1}^N
\fr{c_la_lf_l(z,s)}{\mu_k(z)+a_l(1-pz)},\\ k=1,\ldots,N\,.
\end{multline}
В силу леммы~2 левая часть~\eqref{11} обращается в~0 при $z\hm=z^{(k)}(s).$ Значит,
$$
\sum\limits_{l=1}^N \fr{c_l\:a_l\:f_l(z^{(k)}(s),s)}{\mu^{\ast}_k(s)+a_l
(1-pz^{(k)}(s))}=0,\enskip k=1,\ldots,N\,.
$$
Отсюда, учитывая~\eqref{1} и~определение~$f_l(z,s)$, 
получаем систему линейных уравнений для нахождения функций $p_{0j}(s)$, 
$j\hm=1,\ldots,N$:
\begin{multline*}
\sum\limits_{l=1}^N \fr{c_l a_l}{\mu^{\ast}_k(s)+a_l(1-pz^{(k)}(s))}
\left(
\vphantom{(s+a_l(1-pz^{(k)}(s)))c_l^{-1} p_{0l}(s)}
1-{}\right.\\
\left.{}-(s+a_l(1-pz^{(k)}(s)))c_l^{-1} p_{0l}(s)\right)+{}
\\
{}+\sum\limits_{\nu=1}^Na_{\nu}p_{0\nu}(s)=0,\enskip k=1,\ldots,N\,.
\end{multline*}
Таким образом, все функции, определяющие $p_j(z,x,s)$ и~$p_{0j}(s)$, 
$j\hm=1,\ldots,N$, найдены.


{\small\frenchspacing
 {%\baselineskip=10.8pt
 \addcontentsline{toc}{section}{References}
 \begin{thebibliography}{9}
\bibitem{1-us}
\Au{Hwang G.\,U., Choi B.\,D., Kim~J.-K.} The waiting time analysis of 
a~discrete-time queue with arrivals as an autoregressive process of order~1~// 
J.~Appl. Probab., 2002. Vol.~39. No.\,3. P.~619--629.
\bibitem{2-us}
\Au{Hwang G.\,U., Sohraby~K.} On the exact analysis of a discrete-time queueing
system with autoregressive inputs~// Queueing Syst., 2003. Vol.~43.  P.~29--41.
\bibitem{3-us}
\Au{Kamoun F.} The discrete-time queue with autoregressive inputs revisited~// 
Queueing Syst., 2006. Vol.~54. P.~185--192.
\bibitem{4-us}
\Au{Леонтьев Н.\,Д., Ушаков~В.\,Г. } Анализ системы обслуживания с~входящим
потоком авторегрессионного типа~// Информатика и~её применения, 2014.
Т.~8. Вып.~3. С.~39--44.
\bibitem{5-us}
\Au{Леонтьев Н.\,Д., Ушаков В.\,Г.} Исследование систем обслуживания с~дискретным временем, входящим потоком авторегрессионного типа и~обратной
связью~// Системы и~средства информатики, 2015. Т.~25. №\,2. С.~61--71.
\bibitem{6-us}
\Au{Леонтьев Н.\,Д., Ушаков~В.\,Г.} Анализ системы обслуживания с~входящим потоком 
авторегрессионного типа и~относительным приоритетом~// Информатика и~её применения, 
2016. Т.~10. Вып.~3. С.~15--22.
 \end{thebibliography}

 }
 }

\end{multicols}

\vspace*{-3pt}

\hfill{\small\textit{Поступила в~редакцию 02.03.17}}

\vspace*{8pt}

%\newpage

%\vspace*{-24pt}

\hrule

\vspace*{2pt}

\hrule

%\vspace*{8pt}


\def\tit{SINGLE SERVER QUEUEING SYSTEM WITH~DEPENDENT INTERARRIVAL TIMES}

\def\titkol{Single server queueing system with~dependent interarrival times}

\def\aut{V.\,G.~Ushakov$^{1,2}$ and~N.\,G.~Ushakov$^{3,4}$}

\def\autkol{V.\,G.~Ushakov and~N.\,G.~Ushakov}

\titel{\tit}{\aut}{\autkol}{\titkol}

\vspace*{-9pt}


\noindent
$^1$Department of Mathematical 
Statistics, Faculty of Computational Mathematics and Cybernetics,
M.\,V.~Lo-\linebreak 
$\hphantom{^1}$monosov  Moscow State University, 1-52~Leninskiye Gory, Moscow 119991, GSP-1, 
Russian Federation

\noindent
$^2$Institute of Informatics Problems, 
Federal Research Center ``Computer Science and Control'' 
of the Russian\linebreak
$\hphantom{^1}$Academy of Sciences, 44-2~Vavilov Str., Moscow 119333, Russian Federation

\noindent
$^3$Institute of 
Microelectronics Technology and High-Purity Materials of the Russian Academy of 
Sciences,\linebreak
$\hphantom{^1}$6~Academician Osipyan Str., Chernogolovka, Moscow Region 142432, 
Russian Federation

\noindent
$^4$Norwegian University of Science and Technology, 
15A~S.\,P.~Andersensvei, Trondheim 7491, Norway


\def\leftfootline{\small{\textbf{\thepage}
\hfill INFORMATIKA I EE PRIMENENIYA~--- INFORMATICS AND
APPLICATIONS\ \ \ 2017\ \ \ volume~11\ \ \ issue\ 2}
}%
 \def\rightfootline{\small{INFORMATIKA I EE PRIMENENIYA~---
INFORMATICS AND APPLICATIONS\ \ \ 2017\ \ \ volume~11\ \ \ issue\ 2
\hfill \textbf{\thepage}}}

\vspace*{3pt}



\Abste{The paper studies a single server queueing system with 
an infinite number of positions in the queue and random distribution 
of the service time. The incoming flow of claims is a Poisson flow with
 a~random intensity. The current intensity value is selected from 
 a~finite set with given probabilities at the start of the countdown 
 to the next receipt of the claim. Sequential intensities form 
 a~Markov chain of a special kind. Particular cases of such flows are 
 hyperexponential flows and flows arising in the study of Bayesian models 
 of queueing systems with a discrete prior distribution. Considered flows 
 describe well the work of queueing systems operating in a~random environment 
 with a finite set of different states and Markov relationship between them.
  Furthermore, such flows can accurately approximate real flows in data networks. 
The nonstationary behavior of the queue length is studied.}

\KWE{Poisson flow; random intensity; hyperexponential flow; Markov chain; single server; queue length}



\DOI{10.14357/19922264170212} 

%\vspace*{-18pt}

\Ack
\noindent
This work was supported by the Russian Foundation for Basic Research (project 
No.\,15-07-02354).



%\vspace*{3pt}

  \begin{multicols}{2}

\renewcommand{\bibname}{\protect\rmfamily References}
%\renewcommand{\bibname}{\large\protect\rm References}

{\small\frenchspacing
 {%\baselineskip=10.8pt
 \addcontentsline{toc}{section}{References}
 \begin{thebibliography}{9}
\bibitem{1-us-1}
\Aue{Hwang, G.\,U., B.\,D.~Choi, and J.-K.~Kim.} 
2002. The waiting time analysis of a discrete-time queue with arrivals as an 
autoregressive process of order~1. \textit{J.~Appl. Probab.} 39(3):619--629.
\bibitem{2-us-1}
\Aue{Hwang, G.\,U., and K.~Sohraby.} 2003. 
On the exact analysis of a discrete-time queueing system with autoregressive inputs. 
\textit{Queueing Syst.} 43:29--41.
\bibitem{3-us-1}
\Aue{Kamoun, F.} 2006. The discrete-time queue with autoregressive inputs revisited. 
\textit{Queueing Syst.} 54:185--192.
\bibitem{4-us-1}
\Aue{Leont'ev, N.\,D., and V.\,G.~Ushakov.} 2014. Analiz sistemy obsluzhivaniya 
s~vkhodyashchim potokom avtoregressionnogo tipa [Analysis of 
a~queueing system with autoregressive arrivals]. 
\textit{Informatika i~ee Primeneniya~--- Inform. Appl.} 8(3):39--44.
\bibitem{5-us-1}
\Aue{Leont'ev, N.\,D., and V.\,G.~Ushakov}. 2015. Issledovanie sistem obsluzhivaniya 
s~diskretnym vremenem, vkhodyashchim potokom avtoregressionnogo tipa i~obratnoy 
svyaz'yu [A~study of queueing systems with discrete time, autoregressive arrivals, 
and feedback]. \textit{Sistemy i~Sredstva Informatiki~--- Systems and Means of 
Informatics} 25(2):61--71.
\bibitem{6-us-1}
\Aue{Leont'ev, N.\,D., and V.\,G.~Ushakov.} 2016. Analiz sistemy obsluzhivaniya 
s~vkhodyashchim potokom avtoregressionnogo tipa i~otnositel'nym prioritetom 
[Analysis of a~queueing system with autoregressive arrivals and nonpreemptive priority].
\textit{Informatika i~ee Primeneniya~--- Inform. Appl.} 10(3):15--22.
\end{thebibliography}

 }
 }

\end{multicols}

\vspace*{-3pt}

\hfill{\small\textit{Received March 2, 2017}}


\Contr



\noindent
\textbf{Ushakov Vladimir G.} (b.\ 1952)~--- 
Doctor of Science in physics and mathematics, professor, Department of Mathematical 
Statistics, Faculty of Computational Mathematics and Cybernetics, M.\,V.~Lomonosov 
Moscow State University, 1-52~Leninskiye Gory, Moscow 119991, GSP-1, 
Russian Federation; senior scientist, Institute of Informatics Problems, 
Federal Research Center ``Computer Science and Control'' 
of the Russian Academy of Sciences, 44-2~Vavilov Str., Moscow 119333, Russian Federation; 
\mbox{vgushakov@mail.ru}

\vspace*{3pt}

\noindent
\textbf{Ushakov Nikolai G.} (b.\ 1952)~--- 
Doctor of Science in physics and mathematics, leading scientist, Institute of 
Microelectronics Technology and High-Purity Materials of the Russian Academy of 
Sciences, 6~Academician Osipyan Str., Chernogolovka, Moscow Region 142432, 
Russian Federation; professor, Norwegian University of Science and Technology, 
15A~S.\,P.~Andersensvei, Trondheim 7491, Norway; \mbox{ushakov@math.ntnu.no}


\label{end\stat}


\renewcommand{\bibname}{\protect\rm Литература} 