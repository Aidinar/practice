\def\stat{parkhom}

\def\tit{ПРИМЕНЕНИЕ КВАЗИСЛУЧАЙНОГО ПОДХОДА И~АНСАМБЛЕВЫХ ВЫЧИСЛЕНИЙ 
ДЛЯ~ОПРЕДЕЛЕНИЯ ОПТИМАЛЬНЫХ НАБОРОВ ЗНАЧЕНИЙ ПАРАМЕТРОВ 
КЛИМАТИЧЕСКОЙ МОДЕЛИ$^*$}

\def\titkol{Применение квазислучайного подхода и~ансамблевых вычислений 
для~определения оптимальных наборов значений} % параметров  климатической модели}

\def\aut{В.\,П.~Пархоменко$^1$}

\def\autkol{В.\,П.~Пархоменко}

\titel{\tit}{\aut}{\autkol}{\titkol}

\index{Пархоменко В.\,П.}
\index{Parkhomenko V.\,P.}


{\renewcommand{\thefootnote}{\fnsymbol{footnote}} \footnotetext[1]
{Работа выполнена при поддержке РФФИ (проекты 16-01-0466, 17-01-00693, 17-07-00035).}}


\renewcommand{\thefootnote}{\arabic{footnote}}
\footnotetext[1]{Вычислительный центр им.\ А.\,А.~Дородницына Федерального исследовательского центра <<Информатика 
и~управ\-ле\-ние>> Российской академии наук; Московский государственный технический университет им.\ 
Н.\,Э.~Баумана, \mbox{parhom@ccas.ru}}

\vspace*{-6pt}


  \Abst{В условиях неопределенности значений большого числа параметров 
гидродинамической трехмерной глобальной климатической модели реализована процедура 
их одновременной оценки для близости результатов моделирования к~данным наблюдений. 
Модель включает блоки атмосферы, термохалинной крупномасштабной циркуляции океана и~морского льда. В квазислучайном подходе по методу латинского гиперкуба генерируется 
ансамбль из~200~расчетов путем равномерного полного покрытия диапазона изменения 
каждого из~12~параметров модели. Параметры определяют перемешивание и~перенос 
в~атмосфере, океане и~морском льду, но их комбинации выбираются случайным образом. 
Исследование количественной меры ошибки модели позволило решить обратную задачу 
оценки параметров модели и~прямую задачу прогнозных расчетов по модели.}
  
  \KW{глобальная климатическая модель; оценка параметров; метод латинского гиперкуба}
  
  \DOI{10.14357/19922264170208}
  
%  \vspace*{-6pt} 


\vskip 10pt plus 9pt minus 6pt

\thispagestyle{headings}

\begin{multicols}{2}

\label{st\stat}
  
\section{Введение}

  Климатические модели имеют ряд на\-стра\-и\-ва\-емых параметров, значения 
которых не всегда определяются из теории или данных наблюдений при 
исследовании соответствующих процессов~[1]. Даже характер физических 
процессов может быть неясен и~зависеть от пространственного разрешения 
модели, а параметризации подсеточных процессов\linebreak представляют собой самые 
различные физические явления (вихри и~мелкомасштабные движения, 
инерционные гравитационные волны, приливы и~т.\,п.). В~таких случаях 
значения параметров могут быть определены путем выбора оптимального 
ансамбля модельных результатов для соответствия данным наблюдений. Это, 
естественно, влечет за собой поиск оптимальных квазистационарных решений 
в~многомерном пространстве всех па\-ра\-мет\-ров модели. Использование 
стандартного метода Мон\-те Кар\-ло потребует десятков или сотен тысяч 
интегрирований модели до достижения квазистационарных состояний. 

Для 
моделей с~высоким или умеренным разрешением вычислительные затраты 
даже одного такого расчета могут оказаться непомерно высокими~[2, 3]. 
Вместо этого большие модели, как правило, настроены на последовательность 
расчетов для подробного исследования влияния одного параметра. Однако 
взаимозависимость параметров почти наверняка означает, что даже порядок, 
в~котором такие исследования проводятся, повлияет на конечный результат 
и,~следовательно, на модельные прогнозы. 

Вычислительно эффективные 
модели имеют значительный потенциал для выполнения большого числа 
расчетов за разумное время и~позволяют исследовать большие диапазоны 
в~пространстве их параметров. Если параметры имеют явную физическую 
интерпретацию или близкие аналоги в~модели с~более высоким 
пространственным разрешением, то результаты могут иметь и~более общее 
значение. Вычислительно эффективные модели также полезны для понимания 
долгосрочной естественной изменчивости климата, в~этом случае оптимальный 
баланс сложности блоков модели может зависеть от временн$\acute{\mbox{ы}}$х масштабов, 
интересующих исследователя. 
  
  В статье рассматривается модель океана с~произвольным рельефом дна 
в~глобальной постановке в~геострофическом приближении с~фрикционным 
членом и~с~расширением за счет добавления энерго- и~влагобалансовой 
модели атмосферы и~динамической и~термодинамической модели морского 
льда~[4]. В~данной реализации увеличено горизонтальное разрешение модели 
до~$72 \times 72$~расчетных ячеек~[5]; тем не менее, учитывая достаточно простое 
представление процессов в~атмосфере, в~результате совместная модель имеет 
высокую вычислительную эффективность.

\vspace*{-3pt}
  
\section{Описание модели}

\vspace*{-2pt}

  Представлена глобальная модель климата, которая включает полностью 
трехмерную, с~трением геострофическую модель океана, обладающая высокой 
эффективностью интегрирования по сравнению со значительно более 
ресурсоемкими климатическими моделями с~трехмерными примитивными 
уравнениями океана. Модель включает также динамическую 
и~термодинамическую модель морского льда и~энерго- и~влагобалансовую 
модель ат\-мо\-сферы.
  
  Система уравнений модели океана рассматривается в~геострофическом 
приближении с~фрикционным членом в~уравнениях импульса по 
горизон\-тали~[4, 5]. Значения температуры и~солености удовле\-творяют  
ад\-век\-ци\-он\-но-диф\-фу\-зи\-он\-ным уравнениям, что позволяет описать 
термохалинную циркуляцию океана. Приближенным образом учитываются 
также конвективные процессы. Таким образом, система основных уравнений, 
записанных для наглядности в~локальных декартовых координатах $(x, y, z)$, 
где $x, y$~--- горизонтальные координаты и~$z$~--- высота, направленная 
вверх, имеет следующий вид:
  \begin{itemize}
  \item уравнения импульса по горизонтали
  \begin{align*}
  -lv +\lambda u &=-\fr{1}{\rho}\,\fr{\partial p}{\partial x} 
+\fr{1}{\rho}\,\fr{\partial(k_w\tau_x)}{\partial z}\,;\\
  lu+\lambda v &= -\fr{1}{\rho}\,\fr{\partial p}{\partial y} +\fr{1}{\rho}\, 
\fr{\partial (k_w\tau_y)}{\partial z}\,;
  \end{align*}
  \item уравнение неразрывности
  $$
  \fr{\partial u}{\partial x} +\fr{\partial v}{\partial y} +\fr{\partial w}{\partial y} 
=0\,;
  $$
  \item уравнение гидростатики
  $$
  \fr{\partial p}{\partial z} =-\rho \g\,;
  $$
  \item уравнение состояния морской воды
  $$
  \rho= \rho(S,T)\,;
  $$
  \item уравнение переноса и~диффузии трассеров~$X$ (температуры 
и~солености)
  $$
  \fr{d}{dt}\,X= k_h\nabla^2 X+\fr{\partial}{\partial z}\left( k_v\fr{\partial 
X}{\partial z} \right) +C\,,
  $$
  \end{itemize}
в которых $u$, $v$ и~$w$~--- горизонтальные и~вертикальная компоненты вектора 
скорости соответственно; $\lambda$~--- переменный в~пространстве параметр, 
увеличивающийся к~береговым границам и~экватору и~определяющий влияние 
фрикционного члена; $T$, $S$ и~$p$~--- температура, соленость и~дав\-ле\-ние 
соответственно; $\tau_x$ и~$\tau_y$~--- компоненты напряжения трения ветра; 
$\rho$~--- плотность воды; $l$~--- параметр Кориолиса; $\g$~--- ускорение 
свободного падения; $k_h$ и~$k_v$~--- коэффициенты турбулентной диффузии 
трассеров по горизонтали и~вертикали соответственно; $C$~--- источники.

  Указанная система уравнений решается в~сферической системе координат 
для всего Мирового океана с~реальной аппроксимированной глубиной. На 
границах материков принимаются равными нулю нормальные составляющие 
потоков тепла и~солей. Океан подвергается воздействию напряжения трения 
ветра на поверхности. Потоки~$T$ и~$S$ у~дна полагаются равными нулю, 
а~на поверхности определяются взаимодействием с~атмосферой. 
  
  В термодинамической модели морского льда динамические уравнения 
решаются для сплоченности льда и~для средней толщины льда. Рост и~таяние 
льда в~модели зависят только от разности между потоком тепла из атмосферы 
в~морской лед и~потока тепла изо льда в~океан. Для температуры по\-верх\-ности 
льда решается диагностическое урав\-нение.
{\looseness=1

} 
  
  Для описания процессов, протекающих в~атмосфере, используется энерго- 
и~влагобалансовая модель. В~модели решается вертикально 
проинтегрированное уравнение для температуры, определяющее баланс 
приходящего и~уходящего радиационных потоков, явных (турбулентных) 
обменов потоками тепла с~подстилающей по\-верх\-ностью, высвобождения 
скрытого тепла из-за осадков и~прос\-той однослойной параметризации 
горизонтальных процессов переноса. Источники в~уравнении переноса для 
удельной влажности определяются осадками, испарением и~сублимацией 
с~подстилающей поверхности.

\begin{table*}[b]\small
%\vspace*{3pt}
\begin{center}

\begin{tabular}{|c|l|c|c|c|}
\multicolumn{5}{c}{Диапазон изменения параметров модели климата для ансамблевых 
экспериментов}\\
\multicolumn{5}{c}{\ }\\[-6pt]
\hline
&\multicolumn{1}{c|}{Параметр модели}&Минимум&Максимум&Приемлемый 
диапазон\\
\hline
\multicolumn{5}{|c|}{Океан}\\
\hline
&&&&\\[-9pt]
1.& Горизонтальная диффузия, м$^2$/с & 300 & 10$^4$ & 4200--8500\\
2. & Вертикальная диффузия, м$^2$/с & $2\cdot 10^{-6}$ & $2\cdot 10^{-4}$ &  
$3\cdot 10^{-5}$--$1{,}9\cdot 10^{-4}$\\
3. & Коэффициент трения, сут$^{-1}$ & 1/5 & 2 & 0,6--1,90\\
4. & Ветровое воздействие & 1& 3& 1,14--2,58\\
\hline
\multicolumn{5}{|c|}{Атмосфера}\\
\hline
&&&&\\[-9pt]
5. & Диффузия тепла, м$^2$/с & 10$^6$ & 10$^7$ & $4{,}35\cdot 10^6$--$9\cdot 10^6$\\
6. & Угловой коэффициент, рад & 0,5& 2& 0,7--1,45\\
7. & Коэффициент наклона & 0& 0,25& 0,023--0,230\\
8. & Диффузия влажности, м$^2$/с & $5\cdot 10^6$ & $5\cdot 10^6$ & $1\cdot 10^5$--$3\cdot 10^5$\\
9. & Коэффициент адвекции тепла & 0 & 1& 0,050--0,815\\
10.\hphantom{9} & Коэффициент адвекции влажности & 0 &1& 0,255--0,850\\
11.\hphantom{9} & Поток между океанами, $S_v$ & 0 & 0,64& 0--0,75\\
\hline
\multicolumn{5}{|c|}{Морской лед}\\
\hline
&&&&\\[-9pt]
12.\hphantom{9} &Диффузия морского льда, м$^2$/с & 300& 10$^4$ & 300--9320\\
\hline
\end{tabular}
\end{center}
\end{table*}
  
  Все блоки модели связаны между собой обменом импульсом, теплом 
и~влагой. Используется реальная конфигурация материков и~распределение 
глубин Мирового океана~[5]. Уравнения в~сферической системе координат 
решаются численным ко\-неч\-но-раз\-ност\-ным методом. По горизонтали 
применяется равномерная по долготе и~синусу широты расчетная сетка 
размерностью~$72\times72$. Глубина океана представляется в~виде 
восьмиуровневой логарифмической шкалы до максимального 
значения~5000~м.\linebreak Начальное состояние системы характеризуется постоянными 
температурами океана, атмосферы и~нулевыми скоростями течений океана. 
Численные эксперименты показывают, что модель выходит на равновесие за 
период около~2000~лет~[5].

  %\vspace*{-6pt}
  
\section{Постановка задачи оценки параметров и~результаты}

  %\vspace*{-2pt}
  
  В предлагаемом квазислучайном подходе генерируется ансамбль расчетов 
путем равномерного полного покрытия диапазона изменения каждого 
индивидуального параметра модели, которые перечислены далее, но 
комбинации параметров выбираются случайным образом. Это соответствует 
равномерному разбиению вероятностного пространства значений параметров 
при равномерном распределении плотности вероятности. Таким образом,\linebreak 
при~$M$~расчетах и~$N$~параметрах каждый параметр 
принимает~$M$~значений, равномерно (или по логарифмическому закону) 
покрывающих весь диа\-пазон его изменения, но порядок, в~котором выбираются 
эти значения, определяется случайным\linebreak образом. Это соответствует понятию так 
называ\-емого <<латинского гиперкуба>> в~статистике и~планировании 
эксперимента~[6]. Выборки из латинских гиперкубов начали активно 
применяться\linebreak после удачных решений в~области планирования эксперимента, 
где их использование позволяет уменьшить взаимную зависимость факторов 
без увеличения числа экспериментов~[6]. Каждый расчет представляет собой 
отдельное интегрирование модели на~2000~лет от однородного состояния 
климатической системы с~нулевыми скоростями течений до установившегося 
состояния при стандартных условиях, соответствующих современному 
климату~[5]. 

Как показывают расчеты, окончательное квазистационарное 
состояние может быть не единственным для данного набора параметров. 
Другие квазистационарные состояния могут быть получены с~использованием 
различных начальных условий, в~частности различных начальных температур 
океана. Однако в~настоящей работе прежде всего изучается влияние изменения 
параметров модели и~поэтому фиксируется начальная температура океана 
на~20~$^\circ$C. Такая постановка приводит к~быстрому конвективному 
механизму начала процессов установления в~океане. 

Для обработки результатов 
такого большого количества численных экспериментов необходимо определить 
объективную меру ошибки модели.\linebreak Для этого используется взвешенная 
сред\-не\-квад\-ратическая ошибка на множестве всех динамических переменных 
в~океане и~атмосфере по\linebreak сравнению с~интерполированными данными 
наблюдений, а~именно: температуры и~влажности воздуха на поверхности 
(1000~мб), в~среднем за период с~1948 до~2002~гг., и~температуры и~солености 
океана~[7].
  
  В таблице перечислены~12~основных параметров модели (первый столбец) 
и~принимаемые диапазоны их возможного изменения (второй и~третий 
столбцы)~[8]. Если изменять каждый из этих параметров в~отдельности, то 
будет изучена только очень ограниченная область пространства па\-ра\-мет\-ров. 
Поэтому допускаем, чтобы все~12~па\-ра\-мет\-ров изменялись сразу в~указанных 
диапазонах, которые приведены в~таб\-ли\-це. Предельные значения выбираются 
таким образом, чтобы покрывать или превышать диапазон разумного выбора 
со\-от\-вет\-ст\-ву\-ющих значений для такой модели.

\end{multicols}

 \begin{figure*}[b] %fig1
   \vspace*{-7pt}
\begin{center}
\mbox{%
\epsfxsize=157.963mm
\epsfbox{par-1.eps}
}
\end{center}
\vspace*{-11pt}
\Caption{Среднеквадратичные ошибки в~зависимости от величины исследуемых параметров 
под номерами~1--4 из таблицы}
\end{figure*}

\begin{multicols}{2}
  
  Всего по модели было проведено $M\hm= 200$~расче\-тов. Для определения 
ошибки модельных результа\-тов используется взвешенная среднеквадратичная 
ошибка, вычисляемая по набору всех динамических переменных для 
атмосферы и~океана при сравнении с~данными наблюдений:
  $$
  \varepsilon^2 =\sum\limits_{i=1}^n w_i \left( X_i-D_i\right)^2\,,
  $$
где $X_i$ и~$D_i$~--- соответственно модельные результаты и~данные 
наблюдений для этих переменных (температура и~влажность атмосферы, 
температура и~соленость океана). Суммирование ведется по всем точкам 
трехмерной сетки и~по всем указанным переменным ($n\hm=30\,008$). 
Величины  $w_i\hm= 1/(n\sigma^2_X)$~--- весовые множители, зависящие от 
со\-от\-вет\-ст\-ву\-ющей переменной~$X_i$, но не зависящие от точки сетки; 
 $\sigma_X$~--- среднеквадратичная ошибка данных наблюдений. Вычисляется 
также альтернативная ошибка~$\varepsilon_A$~--- по той же формуле, но только 
для расчетных точек и~переменных атмосферы.

 \begin{figure*} %fig2
  \vspace*{1pt}
\begin{center}
\mbox{%
\epsfxsize=159.825mm
\epsfbox{par-2.eps}
}
\end{center}
\vspace*{-9pt}
\Caption{Среднеквадратичные ошибки в~зависимости от величины исследуемых параметров 
под номерами~5--12 из таблицы}
\end{figure*}
   
  
  На рис.~1 и~2 приведены~12~графиков со значениями вычисленных ошибок 
в~зависимости от исследуемых параметров. На этих рисунках символами~\textit{1} 
отмечены значения параметров с~ошибками $\varepsilon\hm>0{,}6$; 
\textit{2}--\textit{4} соответствуют меньшим значениям ошибки. Среди 
последних символами~\textit{2} отмечены значения параметров с~ошибкой 
$\varepsilon_A\hm> 0{,}1$; \textit{3} (всего~4~штуки) отмечены 
значения $\varepsilon\hm< 0{,}6$ и~$\varepsilon_A\hm< 0{,}1$ одновременно, 
при этом исследование климатических распределений показывает, что 
достигнуто состо\-яние климатической системы, не соответствующее 
современному. Эти результаты исключаются из рассмотрения. Наконец, 
символами~\textit{4} (всего~7~штук) отмечены приемлемые 
результаты расчетов ($\varepsilon\hm< 0{,}6$ и~$\varepsilon_A\hm= 0{,}1)$ 
с~минимальными ошибками, описывающие современный климат. 

\begin{figure*}[b] %fig3
\vspace*{3pt}
\begin{center}
\mbox{%
\epsfxsize=145.144mm
\epsfbox{par-4.eps}
}
\end{center}
\vspace*{-12pt}
\Caption{Температура поверхности океана, осредненная по результатам~7~расчетов 
с~минимальной ошибкой}
%\end{figure*}
% \begin{figure*} %fig4
  \vspace*{9pt}
\begin{center}
\mbox{%
\epsfxsize=145.144mm
\epsfbox{par-5.eps}
}
\end{center}
\vspace*{-12pt}
\Caption{Среднеквадратичное отклонение температуры поверхности океана в~январе, вычисленное по 
набору~7~рас\-че\-тов с~минимальной ошибкой}
\end{figure*}

Таким 
образом, результаты показывают, что сформулированным критериям 
удовлетворяют~7~наборов значений~12~па\-ра\-мет\-ров. Граничные значения 
ошибок $\varepsilon\hm=0{,}6$ и~$\varepsilon_A\hm= 0{,}1$ соответствуют 
ошибкам данных наблюдений. По этой причине нет оснований в~расчетах 
предпочесть только один набор значений параметров. Предлагается вести 
ансамблевые расчеты по модели сразу с~7~оптимальными наборами 
параметров и~в качестве результатов
 рассматривать средние по ансамблю 
и~отклонения от них. В~соответствии с~таблицей и~рис.~1 и~2 в~наборы 
параметров входят значения па\-ра\-мет\-ров, меняющиеся в~широком диапазоне 
(последний стол\-бец в~таб\-ли\-це). Это может означать, что предположение 
о~постоянных значениях параметров достаточно
 грубое. В~зависимости от 
расчетных характеристик климата, градиентов, географических координат 
и~некоторых других причин значения параметров могут меняться во времени 
и~пространстве. 

В~силу заложенных в~постановку ограничений модели 
и~сложности описываемых процессов эти зависимости неизвестны. Однако 
пред\-ла\-га\-емая процедура проведения ансамблевых расчетов в~некоторой 
степени учитывает эти зависимости и~позволяет уточнить результаты, 
поскольку дает диапазон изменения климатических характеристик в~рамках 
ансамбля. 

\begin{figure*} %fig5
\vspace*{1pt}
\begin{center}
\mbox{%
\epsfxsize=164.593mm
\epsfbox{par-6.eps}
}
\end{center}
\vspace*{-9pt}
\Caption{Распределение зонально осредненной температуры атмосферы для 
января~(\textit{а}) и~июля~(\textit{б}): \textit{1}~--- данные наблюдений; \textit{2}~--- максимум  
в~ансамбле расчетов; \textit{3}~--- минимум в~ансамбле расчетов}
\end{figure*}

\vspace*{-7pt}
  
\section{Ансамблевые расчеты с~оптимальными наборами 
параметров модели}

\vspace*{-3pt}
 
  Далее приведены результаты расчетов по модели 
с~использованием~7~приемлемых наборов па\-ра\-мет\-ров, обеспечивающих 
минимальную ошибку по сравнению с~данными наблюдений. Расчеты ведутся 
в~постановке, описанной выше, до установившегося состояния, 
соответствующего современному климату (рис.~3). Сравнение с~данными 
наблюдений показывает хорошее совпадение (рис.~4 и~5). Среднеквадратичное 
отклонение температуры поверхности океана, вычисленное по 
набору~7~расчетов с~минимальной ошибкой, практически во всей области не 
превышает 0,5--1,0~$^\circ$C (см.\ рис.~4).


  
  Расчетный разброс климатических откликов на глобальное потепление при 
100-лет\-нем прогнозе (для приземной температуры воздуха разброс 
около~0,3~$^\circ$C, см.\ рис.~6) является существенным, учитывая, что он 
представляет собой диапазон предсказаний, возникающий только с~изменением 
парамет\-ров перемешивания и~транспорта в~модели.
{ %\looseness=1

}

 { \begin{center}  %fig6
 \vspace*{18pt}
 \mbox{%
\epsfxsize=78.036mm
\epsfbox{par-7.eps}
}
\end{center}

%\vspace*{-3pt}


\noindent
{{\figurename~6}\ \ \small{Изменение средней глобальной температуры атмосферы для~7~расчетов 
с~минимальной ошибкой при прогнозируемом увеличении концентрации СО$_2$ с~2010 
до~2100~г.}}
}

%\vspace*{12pt}

%\addtocounter{figure}{1}

  
\section{Заключение}

\vspace*{-2pt}

  Посредством анализа случайным образом сгенерированных расчетов 
на~2000~лет рассмотрены неопределенности, связанные с~12~параметрами 
модели, определяющими перемешивание и~перенос в~атмосфере, океане 
и~морском льду. Исследование количественной меры ошибки модели 
позволило\linebreak
 решить обратную задачу оценки параметров модели\linebreak и~прямую 
задачу прогнозных расчетов по модели. Результаты представляют собой 
попытку настройки трехмерной климатической модели жестко определенной 
процедурой, но в~которой, тем не менее, рассматривается соответствующее 
пространство квазислучайного изменения параметров модели. Этот подход 
обеспечивает соответствие результатов моделирования данным наблюдений, 
хотя модельные входные параметры исходно точно не известны и~могут 
меняться в~широких пределах. Неопределенность предсказаний модели 
преодолевается двумя различными способами: во-пер\-вых, рассмотрением 
множества прогнозов по подмножеству примерно одинаково правдоподобных 
моделей и,~во-вто\-рых, достаточно статистически обосно\-ван\-ной процедурой 
взвешивания всех результатов в~соответствии со средней ошибкой. Меньшее 
значение ошибки, вероятно, означает лучшее качество моделирования, 
и~поэтому, если модель в~динамике надежна, приемлемые прогнозы находятся 
в~пределах неопределенности порядка ошибки. 

\vspace*{-20pt}
  
{\small\frenchspacing
 {%\baselineskip=10.8pt
 \addcontentsline{toc}{section}{References}
 \begin{thebibliography}{9}
 
 \vspace*{-2pt}
 
  \bibitem{1-par}
  \Au{Edwards N.\,R., Marsh~R.} Uncertainties due to transport-parameter 
sensitivity in an efficient 3-D ocean-climate model~// Clim. Dynam., 2005. 
Vol.~24. No.\,4. P.~415--433.
\bibitem{2-par}
\Au{Randall D.\,A.} General circulation model development.~--- Gardners Books, 
2010. 416~p. 
  \bibitem{3-par}
  \Au{Satoh M.} Atmospheric circulation dynamics and general circulation 
models.~--- Berlin: Springer-Verlag, 2014.\linebreak 730~p.
  \bibitem{4-par}
  \Au{Marsh R., Edwards~N.\,R., Shepherd~J.\,G.} Development of a fast climate 
model (C-GOLDSTEIN) for Earth System Science~// SOC, 2002. No.\,83. 54~p.
  \bibitem{5-par}
  \Au{Пархоменко В.\,П.} Глобальная модель климата с~описанием 
термохалинной циркуляции Мирового океана~// Математическое 
моделирование и~численные методы, 2015. №\,1. С.~94--108.

%\columnbreak

  \bibitem{6-par}
  \Au{Montgomery D.\,C.} Design and analysis of experiments.~--- 5th ed.~--- New 
York, NY, USA: John Wiley \& Sons, 2001. 684~p.
  \bibitem{7-par}
  \Au{Levitus S., Boyer~T.\,P., Conkright~M.\,E., O'Brien~T., Antonov~J., 
Stephens~C., Stathoplos~L., Johnson~D., Gelfeld~R.} Noaa Atlas Nesdis~18, World 
Ocean database.~--- Washington, D.C., USA: U.S.\ Government Printing, 1998. 
 Vol.~1. 346~p.
  \bibitem{8-par}
  \Au{Parkhomenko V.} Ensemble calculations application for estimation and 
optimization of climate model parameters~//  3rd Conference (International) on 
Optimization Methods and Applications Proceedings.~--- Moscow: 
Computing Center of RAS, 2012. P.~203--207.
 \end{thebibliography}

 }
 }

\end{multicols}

\vspace*{-3pt}

\hfill{\small\textit{Поступила в~редакцию 26.01.17}}

\vspace*{10pt}

%\newpage

%\vspace*{-24pt}

\hrule

\vspace*{2pt}

\hrule

\vspace*{8pt}


\def\tit{APPLICATION OF~QUASI-RANDOM ENSEMBLE CALCULATIONS 
FOR~DETERMINATION OF~CLIMATE MODEL OPTIMAL PARAMETERS}

\def\titkol{Application of~quasi-random ensemble calculations 
for~determination of~climate model optimal parameters}

\def\aut{V.\,P.~Parkhomenko$^{1,2}$}

\def\autkol{V.\,P.~Parkhomenko}

\titel{\tit}{\aut}{\autkol}{\titkol}

\vspace*{-9pt}


\noindent
$^1$A.\,A.~Dorodnicyn Computing Center, Federal Research Center ``Computer 
Science and Control'' of the Russian\linebreak
$\hphantom{^1}$Academy of Sciences, 40~Vavilov Str., Moscow 
119333, Russian Federation

\noindent
$^2$N.\,E.~Bauman Moscow State Technical University, 5  Baumanskaya 2nd Str.,
Moscow 105005, Russian 
Federation



\def\leftfootline{\small{\textbf{\thepage}
\hfill INFORMATIKA I EE PRIMENENIYA~--- INFORMATICS AND
APPLICATIONS\ \ \ 2017\ \ \ volume~11\ \ \ issue\ 2}
}%
 \def\rightfootline{\small{INFORMATIKA I EE PRIMENENIYA~---
INFORMATICS AND APPLICATIONS\ \ \ 2017\ \ \ volume~11\ \ \ issue\ 2
\hfill \textbf{\thepage}}}

\vspace*{3pt} 



\Abste{By analyzing a randomly generated set of runs, 
each~2000~years in length, the author has considered the uncertainty in~12~mixing 
and transport parameters. Constructing a quantitative measure for the model 
error made it possible to address both the inverse problem of estimation 
of model parameters and the direct problem of model predictions. 
The results represent an attempt at tuning a~three-dimensional climate model by 
a~strictly defined procedure which, nevertheless, considers the whole of the 
appropriate parameter space. The modeling approach is thus to match 
model outputs to observations while 
model inputs (parameters) are initially only weakly constrained.}

\KWE{global climate model; model parameters estimation; latin hypercube sampling}

\DOI{10.14357/19922264170208} 

\vspace*{-3pt}

\Ack
\noindent
The work was supported by the Russian Foundation
for Basic Research (projects 16-01-0466, 17-01-00693, and 17-07-00035).



\vspace*{9pt}

  \begin{multicols}{2}

\renewcommand{\bibname}{\protect\rmfamily References}
%\renewcommand{\bibname}{\large\protect\rm References}

{\small\frenchspacing
 {%\baselineskip=10.8pt
 \addcontentsline{toc}{section}{References}
 \begin{thebibliography}{9}
  \bibitem{1-par-1}
  \Aue{Edwards, N.\,R., and R.~Marsh.} 2005. Uncertainties due to 
  transport-parameter sensitivity in an efficient 3-D ocean-climate model. 
  \textit{Clim. Dynam.} 24(4):415--433.
  \bibitem{2-par-1}
  \Aue{Randall, D.\,A.} 2010. \textit{General circulation model development}. 
Gardners Books. 416~p.
  \bibitem{3-par-1}
  \Aue{Satoh, M.} 2014. \textit{Atmospheric circulation dynamics and general 
circulation models}. Berlin: Springer-Verlag. 730~p.

\columnbreak

  \bibitem{4-par-1}
  \Aue{Marsh, R., N.\,R. Edwards, and J.\,G.~Shepherd.} 2002. Development of 
a~fast climate model (C-GOLDSTEIN) for Earth System Science. \textit{SOC}  83. 
54~p.
  \bibitem{5-par-1}
  \Aue{Parkhomenko, V.\,P.} 2015. Global'naya model' klimata s~opisaniem 
termokhalinnoy tsirkulyatsii Mirovogo okeana [Global climate model including 
description of thermohaline circulation of the World Ocean].
  \textit{Matematicheskoe modelirovanie i~chislennye metody} [Mathematical 
Modeling and Numerical Methods] 1:94--108.
  \bibitem{6-par-1}
  \Aue{Montgomery, D.\,C.} 2001. \textit{Design and analysis of experiments}. 5th 
ed. New York, NY: John Wiley \& Sons, Inc. 684~p.
{\looseness=-1

}

\pagebreak

  \bibitem{7-par-1}
  \Aue{Levitus, S., T.\,P.~Boyer, M.\,E.~Conkright, T.~O'Brien, J.~Antonov, 
C.~Stephens, L.~Stathoplos, D.~Johnson, and R.~Gelfeld}. 1998. \textit{Noaa Atlas 
Nesdis~18, World Ocean database 1998.} Washington, D.C.: U.S. Government 
Printing. Vol.~1. 346~p.
  \bibitem{8-par-1}
  \Aue{Parkhomenko, V.} 2012. Ensemble calculations application for estimation 
and optimization of climate model parameters. \textit{3rd~Conference (International) 
on Optimization Methods and Applications Proceedings}. Moscow: 
Computing Center of RAS P.~203--207. 
  \end{thebibliography}

 }
 }

\end{multicols}

\vspace*{-3pt}

\hfill{\small\textit{Received January 26, 2017}}
  
  \Contrl
  
  \noindent
\textbf{Parkhomenko Valery P.} (b.\ 1951)~--- Candidate of Science (PhD) in physics and mathematics, 
head of laboratory, A.\,A.~Dorodnicyn Computing Center, Federal Research Center ``Computer Science and 
Control'' of the Russian Academy of Sciences, 40~Vavilov Str., Moscow 119333, Russian Federation; 
associate professor, N.\,E.~Bauman Moscow State Technical University, 5~Baumanskaya 
2nd Str., Moscow 
105005, Russian Federation; \mbox{parhom@ccas.ru}
  
\label{end\stat}


\renewcommand{\bibname}{\protect\rm Литература} 