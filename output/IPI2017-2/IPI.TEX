\documentclass[10pt]{book}
\usepackage[utf8]{inputenc}

\usepackage{latexsym,amssymb,amsfonts,amsmath,indentfirst,shapepar,%fleqn,%
picinpar,shadow,floatflt,enumerate,multicol,colortbl,moreverb,cite,ipi}

\usepackage{rotating}
\usepackage{mathrsfs}
\usepackage[noend]{algorithmic}
\usepackage{ulem}
%\usepackage{graphicx}
%\usepackage{algorithm2e}

\input{epsf}

%\nofiles

%\includeonly{avtor} %+pdf
%\includeonly{obchak,avtor}
%\includeonly{pred}                 %+
%\includeonly{podgot-rus-site,podgot-eng-site}  
%\includeonly{ocherk} 
%\includeonly{nekrol} 
%\includeonly{ipi-ind} 
%\includeonly{toc-rus}
%\includeonly{toc-en} 



%\includeonly{kabanov}    %++pdfотпр-авт+
%\includeonly{lukashenko} %+pdfотпр-авт+
%\includeonly{agalarov}   %+pdfотпр-авт+
%\includeonly{borisov}    %++pdfотпр-авт+
%\includeonly{vasiliev}   %+pdfотпр-авт+
%\includeonly{gaidamaka}  %+pdfотпр-авт+
%\includeonly{krivenko}   %+pdfотпр-авт+
%\includeonly{parkhom}    %++pdfотпр-авт+
%\includeonly{rudoy}      %+pdfотпр
%\includeonly{serebr-1}   %+pdfотпр
%\includeonly{sinitsin}   %+pdfотпр
%\includeonly{ushakov}    %++pdfотпр-авт+
%\includeonly{shest-1}    %+pdfотпр-авт+
%\includeonly{shest-2}    %+pdfотпр-авт+


%\includeonly{toc-rus, toc-en}
%\includeonly{obchak} %,toc-en}
%\includeonly{rekl}
%\includeonly{rekl-1}
%\includeonly{reshal}  %
%\includeonly{eng-index}
%\includeonly{cover3}

\usepackage{acad}
%\usepackage{courier}
\usepackage{decor}
\usepackage{newton}
\usepackage{pragmatica}
\usepackage{zapfchan}
\usepackage{petrotex}
\usepackage{bm}                     % полужирные греческие буквы
\usepackage{upgreek}                % прямые греческие буквы
\usepackage{eufrak}
\usepackage{verbatim}

\renewcommand{\bottomfraction}{0.99}
\renewcommand{\topfraction}{0.99}
\renewcommand{\textfraction}{0.01}

\setcounter{secnumdepth}{1} %здесь - 3 + chapter = 4

\arraycolsep=1.5pt

%\usepackage[pdftex]{graphicx}

%\usepackage{oz}

%NEW COMMANDS


\renewcommand*{\hm}[1]{#1\nobreak\discretionary{}%
            {\hbox{$\mathsurround=0pt #1$}}{}} %% Дублирует знаки операций
                               %при переносе в формуле (перед знаком, который
                               %надо продублировать ставится команда \hm)

%\newcommand{\endproof}{\hfill$\Box$}
%\renewcommand{\r}{\mathbb{R}}
\newcommand{\I}{{\rm I\hspace{-0.7mm}I}}
%\newcommand{\Ikl}{{\tt{1}}\hspace*{-1.44mm}\mathtt{1}}
\newcommand{\Ik}{\mbox{{\small \tt {1}}\hspace{-1.3mm}{\tt 1}}}
\newcommand{\argmin}{\mathop{\mathrm{arg}\,\mathrm{min}}}
\newcommand{\argmax}{\mathop{\mathrm{arg}\,\mathrm{max}}}
%\newcommand{\capr}{\mathop{\cap\,}}
%\newcommand{\cupr}{\mathop{\cup\,}}
%\def\argmin{\mathop{arg\,min}}

\def\vrp{\varphi}
\def\prt{\partial}
\def\mm{{\sf M}}
\def\modnop#1{\mathop{#1}\limits_{n}}
\def\eam{\mathbin{{\mathop{=}\limits^{\mathrm{def}}}}}
\def\dey#1#2{#1 (#2)}
\def\deyc#1#2{#1 \cdot  #2}
\def\ra#1{\;\mathop{\to}\limits^{#1}\;}
\def\raz#1{\;\mathop{\longrightarrow}\limits^{\!\!\!#1}\;}
\def\ral#1{\;\mathop{\longrightarrow}\limits^{#1}\;}

\newcommand{\Nor}{\mathcal{N}}
\newcommand{\T}{\mathbb{T}}
\newcommand{\Z}{\mathbb{Z}}



\newcommand{\il}[2]{\int\limits_{#1}^{#2}}%интеграл с пределами #1 и #2

\def\sm2{\mathop {\sum\limits^{n^\Theta}\sum\limits^{n^\Theta}}}
\def\sss{\sum\limits}
\def\tr{,\,\ldots\,,\,}
\def\rk{\right]}
\def\lk{\left[}
\def\rf{\right\}}
\def\lf{\left\{}
\def\lv{\,\left\vert}
\def\rv{\right\vert\,}
\def\iii{\int\limits}
\def\iin{\int\limits_{-\infty}^\infty}
\def\rrv{\right\vert}


\def\ee{{\cal E}}
\def\ww{{\cal W}}
\def\yy{{\cal Y}}
\def\vv{{\cal V}}

\newcommand{\R}{\mathbb R}
\newcommand{\E}{\mathbb E}
\newcommand{\N}{\mathbb N}

\renewcommand{\P}{\mathbb{P}}

\newcommand{\h}{{\bf H}}
\newcommand{\p}{{\sf P}}  % вероятность

\newcommand{\e}{{\sf E}}  % мат. ожидание
\newcommand{\D}{{\sf D}}  % дисперсия
\newcommand{\eps}{\varepsilon}
\newcommand{\vp}{{\mathbf p}}
\newcommand{\vz}{{\mathbf z}}
\newcommand{\vx}{{\mathbf x}}
\newcommand{\vf}{{\mathbf f}}
\newcommand{\F}{{\mathcal F}}
\def\ap{{\mathrm{ЭР}}}
\newcommand{\ud}{\Delta_n} %uniform ditance
\newcommand{\nud}{\Delta_n(x)}
\renewcommand{\Re}{\mathrm{Re}\,}

\newcommand{\abs}[1]{\left\vert#1\right\vert}

\newcommand{\norm}[1]{\left\Vert#1\right\Vert}
\def\da{(\Delta_t,A)}

\newcommand{\corr}{\mathrm{corr}}

\newcommand{\cov}{\mathrm{cov}}
\newcommand{\Expect}{\mathbb{E}}

\def\w{\omega}
\def\W{\Omega}

\def\inh{\int\limits_{nh}^{(n+1)h}}

\def\sumin{\sum_{i=1}^N}


\def\bxt{(Y,t)}
\def\xt{(y,t)}

\def\ovth{{\fr{\tau-nh}{h}}}
\def\ov{\overline}
\def\tm{\tilde m}
\def\tl{\tilde\lambda}
\def\tB{\widetilde B}
\def\tb{\tilde b}
\def\ld{\ldots}
\def\cd{\cdots}


\DeclareMathOperator{\sign}{sign}

%\newcommand{\gr}{{\geqslant}}


\newcommand{\g}{\mbox{\textit{g}}}

\renewcommand{\la}{\lambda}
\newcommand{\si}{\sigma}
\newcommand{\alp}{\alpha}

%\newcommand{\pto}{\stackrel{P}{\longrightarrow}} % сходимость по веpоятности

\newcommand{\eqd}{\stackrel{\mathrm{d}}{=}} % равенство по pаспpеделению
\newcommand{\eqdelta}{\stackrel{\triangle}{=}} % равенство по pаспpеделению

\def\be#1{\begin{equation}\label{#1}}
\def\ee{\end{equation}}
\def\re#1{(\ref{#1})}

\def\bn{\begin{enumerate}}
\def\en{\end{enumerate}}
\def\bi{\begin{itemize}}
\def\ei{\end{itemize}}
%\def\i{\item}

%\newcommand{\kp}{\kappa}
%\def\Q{{\cal Q}} \def\H{{\cal H}}
%\newcommand{\bet}{\beta_{2+\delta}}


%\newtheorem{definition}{Определение}
%\renewcommand{\thedefinition}{\arabic{definition}.}
%END NEW COMMANDS

%\renewcommand{\baselinestretch}{1.2}

%\pagestyle{myheadings}

\setlength{\textwidth}{167mm}      % 122mm
\setlength{\textheight}{658pt}
%\setlength{\textheight}{635.6pt}
\setlength{\columnsep}{4.5mm}

\setcounter{secnumdepth}{4}

%\addtolength{\headheight}{2pt}
%\addtolength{\headsep}{-2mm}

\addtolength{\topmargin}{-7mm}  % for printing


%\hoffset=-30mm  % From Yap
\hoffset=-23mm  % From Acrobat

%\voffset=0mm % From Yap
\voffset=-5mm   % From Acrobat

%\addtolength{\evensidemargin}{-2.5mm} % for printing
%\addtolength{\oddsidemargin}{2.5mm}  % for printing

\addtolength{\evensidemargin}{-12mm} % for printing
\addtolength{\oddsidemargin}{8mm}  % for printing

%\renewcommand{\thefootnote}{\fnsymbol{footnote}}
%\renewcommand{\thefootnote}{\arabic{footnote}}
\renewcommand{\figurename}{\protect\bf Рис.}
\renewcommand{\tablename}{\protect\bf Таблица}

\newcommand{\Caption}[1]{\caption{\protect\small %\baselineskip=2.5ex
#1}}

\renewcommand{\thefigure}{\arabic{figure}}
\renewcommand{\thetable}{\arabic{table}}
\renewcommand{\theequation}{\arabic{equation}}
\renewcommand{\thesection}{\arabic{section}}

\renewcommand{\contentsname}{СОДЕРЖАНИЕ}
\newcommand{\fr}[2]{\displaystyle\frac{\displaystyle #1\mathstrut}{\displaystyle #2\mathstrut}}

%\renewcommand{\thefootnote}{\fnsymbol{footnote}}
%\newcommand{\g}{\mbox{\textit{g}}}

%\newcommand{\Caption}[1]{\caption{\protect\small\baselineskip=2ex #1}}
\newcounter{razdel}
\setcounter{razdel}{0}


\newcommand{\titel}[4]{%
\

\vspace*{5pt}

\ifodd\therazdel {\raggedright\noindent\Large\textrm\textbf
 \lineskip .75em
  \baselineskip=3.2ex #1 \par}
\vskip 1em {\noindent\large\textrm\textbf #2 \par}
\addcontentsline{toc}{subsection}{{\textrm\textbf #1}\protect\newline #2}
\def\rightheadline{\underline{\noindent\hbox to \textwidth{\hfill\small\textrm{#4}
%\hfill \large\bf\thepage
}}}
\def\leftheadline{\underline{\noindent\parbox{\textwidth}{
%\raggedleft\large\bf\thepage \hfill
\small\textit{#3}\hfill}}}
\def\leftfootline{\small{\textbf{\thepage}
\hfill ИНФОРМАТИКА И ЕЁ ПРИМЕНЕНИЯ\ \ \ том~11\ \ \ выпуск 2\ \ \ 2017}
}%
 \def\rightfootline{\small{ИНФОРМАТИКА И ЕЁ ПРИМЕНЕНИЯ\ \ \ том~11\ \ \ выпуск~2\ \ \ 2017
\hfill \textbf{\thepage}}}
\vskip 2em \setcounter{figure}{0}
\setcounter{table}{0}
\setcounter{equation}{0}
\setcounter{section}{0}
\setcounter{subsection}{0}
\setcounter{subsubsection}{0}
\setcounter{footnote}{0}
\setcounter{razdel}{0}
%\end{flushleft}
\else {
 \raggedright\noindent\Large\textrm\textbf
 \lineskip .75em
\baselineskip=3.2ex #1 \par} \vskip 1em
%\begin{flushleft}
{\noindent\large\textrm\textbf #2 \par}
\addcontentsline{toc}{subsection}{{\textrm\textbf #1}\protect\newline #2}
\def\rightheadline{\underline{\noindent\hbox to \textwidth{\hfill\small\textrm{#4}
%\hfill \large\bf\thepage
}}}
\def\leftheadline{\underline{\noindent\parbox{\textwidth}{%\raggedleft\large\bf\thepage \hfill
\small\textit{#3}\hfill}}}
\def\leftfootline{\small{\textbf{\thepage}
\hfill ИНФОРМАТИКА И ЕЁ ПРИМЕНЕНИЯ\ \ \ том~11\ \ \ выпуск~2\ \ \ 2017}
}%
 \def\rightfootline{\small{ИНФОРМАТИКА И ЕЁ ПРИМЕНЕНИЯ\ \ \ том~11\ \ \ выпуск~2\ \ \ 2017
\hfill \textbf{\thepage}}} \vskip 2em \setcounter{figure}{0}
\setcounter{table}{0} \setcounter{equation}{0} \setcounter{section}{0}
\setcounter{subsection}{0} \setcounter{subsubsection}{0}
\setcounter{footnote}{0}
%\end{flushleft}
\fi}

\newcommand{\titelr}[2]{%
\

\vspace*{5pt}

\ifodd\therazdel {\raggedright\noindent%\Large\textrm\textbf
 \lineskip .75em
  \baselineskip=3.2ex #1 \par}
\vskip 1em {\noindent\normalsize\textrm\textbf #2 \par}
\else {
 \raggedright\noindent\Large\textrm\textbf
 \lineskip .75em
\baselineskip=3.2ex #1 \par} \vskip 1em
%\begin{flushleft}
{\noindent\large\textrm\textbf #2 \par
%\noindent\normalsize\textrm\textbf #2 \par
} \fi}

\newcommand{\titele}[5]{%
\

%\vspace*{5pt}

\ifodd\therazdel {\raggedright\noindent\large
\textrm\textbf
 \lineskip .75em
%  \baselineskip=3.2ex
#1 \par}
\vskip .5em {\noindent\large\textrm\textbf #2 \par}
\vskip .5em
 {\noindent\textrm #3 \par}
\addcontentsline{toc}{subsection}{{\textrm\textbf #1}\protect\newline #2}
\def\rightheadline{\underline{\noindent\hbox to \textwidth{\hfill\small\textrm{#4}
%\hfill \large\bf\thepage
}}}
\def\leftheadline{\underline{\noindent\parbox{\textwidth}{
%\raggedleft\large\bf\thepage \hfill
\small\textrm{#5}\hfill}}}
\def\leftfootline{\small{\textbf{\thepage}
\hfill ИНФОРМАТИКА И ЕЁ ПРИМЕНЕНИЯ\ \ \ том~11\ \ \ выпуск~2\ \ \ 2017}
}%
 \def\rightfootline{\small{ИНФОРМАТИКА И ЕЁ ПРИМЕНЕНИЯ\ \ \ том~11\ \ \ выпуск~2\ \ \ 2017
\hfill \textbf{\thepage}}} \vskip 1em \setcounter{figure}{0}
\setcounter{table}{0} \setcounter{equation}{0} \setcounter{section}{0}
\setcounter{subsection}{0} \setcounter{subsubsection}{0}
\setcounter{footnote}{0} \setcounter{razdel}{0}
%\end{flushleft}
\else {
 \raggedright\noindent\large
 \textrm\textbf
 \lineskip .75em
%\baselineskip=3.2ex
#1 \par} \vskip .5em
%\begin{flushleft}
{\noindent\large\textrm\textbf #2 \par} \vskip .5em
 {\noindent\textrm #3 \par}
\addcontentsline{toc}{subsection}{{\textrm\textbf #1}\protect\newline #2}
\def\rightheadline{\underline{\noindent\hbox to \textwidth{\hfill\small\textrm{#4}
%\hfill \large\bf\thepage
}}}
\def\leftheadline{\underline{\noindent\parbox{\textwidth}{%\raggedleft\large\bf\thepage \hfill
\small\textrm{#5}\hfill}}}
\def\leftfootline{\small{\textbf{\thepage}
\hfill ИНФОРМАТИКА И ЕЁ ПРИМЕНЕНИЯ\ \ \ том~11\ \ \ выпуск~2\ \ \ 2017}
}%
 \def\rightfootline{\small{ИНФОРМАТИКА И ЕЁ ПРИМЕНЕНИЯ\ \ \ том~11\ \ \ выпуск~2\ \ \ 2017
\hfill \textbf{\thepage}}} \vskip 1em \setcounter{figure}{0}
\setcounter{table}{0} \setcounter{equation}{0} \setcounter{section}{0}
\setcounter{subsection}{0} \setcounter{subsubsection}{0}
\setcounter{footnote}{0}
%\end{flushleft}
\fi}

\def\Abst#1{
\begin{center}\small\nwt
\parbox{150mm}{%\baselineskip=2.5ex
\textbf{Аннотация:}\ \
%\hspace*{\parindent}
#1}
\end{center}}
\def\Abste#1{
\begin{center}\small\nwt
\parbox{150mm}{%\baselineskip=2.5ex
\textbf{Abstract:}\ \
%\hspace*{\parindent}
#1}
\end{center}}

\def\DOI#1{
\begin{center}\small\nwt
\parbox{150mm}{%\baselineskip=2.5ex
\textbf{DOI:}\ \
%\hspace*{\parindent}
#1}
\end{center}}

\def\Abstend#1{
\begin{center}\small\nwt
\parbox{150mm}{%\baselineskip=2.5ex
%\hspace*{\parindent}
#1}
\end{center}}


\def\KW#1{
\begin{center}\small\nwt
\parbox{150mm}{%\baselineskip=2.5ex
\textbf{Ключевые слова:}\ \ #1}
\end{center}}

\def\KWE#1{
\begin{center}\small\nwt
\parbox{150mm}{%\baselineskip=2.5ex
\textbf{Keywords:}\ \ #1}
\end{center}}


\def\KWN#1{
%\begin{center}
%\small
%\parbox{150mm}\end{center}
}

\newcommand{\Avtors}[1]{%\smallskip
%\vspace*{.5pt}
\hangindent=23pt\noindent
%\nwt
{\bfseries#1}\
}


\renewcommand{\thesubsection}{\thesection.\arabic{subsection}\hspace*{-5pt}}
\renewcommand{\thesubsubsection}{\thesubsection\hspace*{5pt}.\arabic{subsubsection}\hspace*{-3pt}}

\newcommand{\Ack}{\section*{\protect\rmfamily Acknowledgments}\noindent}
\newcommand{\Contr}{\section*{\protect\rmfamily Contributors}\noindent}
\newcommand{\Contrl}{\section*{\protect\rmfamily Contributor}\noindent}

\makeindex


\begin{document}
\Rus

\nwt
%\ptb


%\renewcommand{\contentsname}{\protect\Large\bf Содержание}

\setcounter{tocdepth}{2}

%\tableofcontents

\renewcommand{\bibname}{\protect\rmfamily Литература}
  \def\Au#1{{\it #1}}
    \def\Aue#1{{#1}}

%\newcommand{\No}{№}
  \newcommand{\tg}{\,\mathrm{tg}\,}
    \newcommand{\ctg}{\,\mathrm{ctg}\,}
  \newcommand{\arctg}{\,\mathrm{arctg}\,}

\def\forallb{\mathop{\forall}}
\def\cupb{\mathop{\cup}}
\def\existsb{\mathop{\exists}}


\newpage
\addtocounter{razdel}{1}
%\def\razd{РЕГУЛИРУЕМЫЙ ЭЛЕКТРОПРИВОД ДЛЯ ЭЛЕКТРОЭНЕРГЕТИКИ}


\setcounter{page}{2}

%   { %\Large  
   { %\baselineskip=16.6pt
   
   \vspace*{-48pt}
   \begin{center}\LARGE
   \textit{Предисловие}
   \end{center}
   
   %\vspace*{2.5mm}
   
   \vspace*{25mm}
   
   \thispagestyle{empty}
   
   { %\small 

    
Вниманию читателей журнала <<Информатика и её применения>> предлагается 
очередной тематический выпуск <<Вероятностно-статистические методы и 
задачи информатики и информационных технологий>>. Предыдущие тематические 
выпуски журнала по данному направлению вышли в 2008~г.\ (т.~2, вып.~2), 
в 2009~г.\ (т.~3, вып.~3) и в 2010~г.\ (т.~4, вып.~2). 

Статьи, собранные в данном журнале, посвящены разработке новых вероятностно-статистических 
методов, ориентированных на применение к решению конкретных задач информатики и информационных 
технологий, а также~--- в ряде случаев~--- и других прикладных задач. Проблематика, охватываемая 
публикуемыми работами, развивается в рамках научного сотрудничества между Институтом проблем 
информатики Российской академии наук (ИПИ РАН) и Факультетом вычислительной математики и 
кибернетики Московского государственного университета им.\ М.\,В.~Ломоносова в ходе работ 
над совместными научными проектами (в том числе в рамках функционирования 
Научно-образовательного центра <<Вероятностно-статистические методы анализа рисков>>). 
Многие из авторов статей, включенных в данный номер журнала, являются активными участниками 
традиционного международного семинара по проблемам устойчивости стохастических моделей, 
руководимого В.\,М.~Золотаревым и В.\,Ю.~Королевым; регулярные сессии этого семинара 
проводятся под эгидой МГУ и ИПИ РАН (в 2011~г.\ указанный семинар проводится в октябре 
в Калининградской области РФ). 

Наряду с представителями ИПИ РАН и МГУ в число авторов данного выпуска журнала входят 
ученые из Научно-исследовательского института системных исследований РАН, Института 
проблем технологии микроэлектроники и особочистых материалов РАН, Института 
прикладных математических исследований Карельского НЦ РАН, Московского 
авиационного института, Вологодского государственного педагогического университета, 
НИИММ им.\ Н.\,Г.~Чеботарева, Казанского государственного университета, Дебреценского 
университета (Венгрия).

Несколько статей выпуска посвящено разработке и применению стохастических методов и 
информационных технологий для решения различных прикладных задач. В~работе В.\,Г.~Ушакова 
и О.\,В.~Шестакова рассмотрена задача определения вероятностных характеристик случайных 
функций по распределениям интегральных преобразований, возникающих в задачах эмиссионной 
томографии. В~статье Д.\,О.~Яковенко и М.\,А.~Целищева рассмотрены некоторые вопросы 
математической теории риска и предложен новый подход к диверсификации инвестиционных 
портфелей. Работа И.\,А.~Кудрявцевой и А.\,В.~Пантелеева посвящена построению и 
исследованию математической модели, описывающей динамику сильноионизованной плазмы. 
В~статье П.\,П.~Кольцова изучается качество работы ряда алгоритмов сегментации изображений. 
Статья А.\,Н.~Чупрунова и И.~Фазекаша посвящена вероятностному анализу числа без\-оши\-бочных 
блоков при помехоустойчивом кодировании; получены усиленные законы больших чисел для указанных 
величин.

В данном выпуске традиционно присутствует тематика, весьма активно разрабатываемая в течение 
многих лет специалистами ИПИ РАН и МГУ,~--- методы моделирования и управления для 
информационно-телекоммуникационных и вычислительных систем, в частности методы 
теории массового обслуживания. В~статье А.\,И.~Зейфмана с соавторами рассматриваются 
модели обслуживания, описываемые марковскими цепями с непрерывным временем в случае 
наличия катастроф. В~работе М.\,М.~Лери и И.\,А.~Чеплюковой рассматриваются случайные 
графы Интернет-типа, т.\,е.\ графы, степени вершин которых имеют степенные распределения; 
такие задачи находят применение при исследовании глобальных сетей передачи данных. 
Работа Р.\,В.~Разумчика посвящена исследованию систем массового обслуживания специального 
вида~--- с отрицательными заявками и хранением вытесненных заявок.

Ряд статей посвящен развитию перспективных теоретических 
вероятностно-статистических методов, которые находят широкое применение в различных 
задачах информатики и информационных технологий. В~работе В.\,Е.~Бенинга, А.\,К.~Горшенина 
и В.\,Ю.~Королева рассмотрена задача статистической проверки гипотез о числе компонент 
смеси вероятностных распределений, приводится конструкция асимптотически наиболее мощного 
критерия. Результаты этой работы найдут применение в ряде прикладных задач, использующих 
математическую модель смеси вероятностных распределений (в информатике, моделировании 
финансовых рынков, физике турбулентной плазмы и~т.\,д.). В~статье В.\,Ю.~Королева, 
И.\,Г.~Шевцовой и С.\,Я.~Шоргина строится новая, улучшенная оценка точности нормальной 
аппроксимации для пуассоновских случайных сумм; как известно, указанные случайные суммы 
широко используются в качестве моделей многих реальных объектов, в том числе в информатике, 
физике и других прикладных областях. Работа В.\,Г.~Ушакова и Н.\,Г.~Ушакова посвящена 
исследованию ядерной оценки плотности распределения; эти результаты могут применяться, 
в част\-ности, при анализе трафика в телекоммуникационных системах. Серьезные приложения 
в статистике могут получить результаты работы О.\,В.~Шестакова, в которой доказаны оценки 
скорости сходимости распределения выборочного абсолютного медианного отклонения к нормальному 
закону. 

\smallskip

Редакционная коллегия журнала выражает надежду, что данный тематический  выпуск 
будет интересен специалистам в области теории вероятностей и математической статистики 
и их применения к решению задач информатики и информационных технологий.
     
     %\vfill 
     \vspace*{20mm}
     \noindent
     Заместитель главного редактора журнала <<Информатика и её 
применения>>,\\
     директор ИПИ РАН, академик  \hfill
     \textit{И.\,А.~Соколов}\\
     
     \noindent
     Редактор-составитель тематического выпуска,\\
     профессор кафедры математической статистики факультета\\
      вычислительной математики и кибернетики МГУ им.\ М.\,В.~Ломоносова,\\
     ведущий научный сотрудник ИПИ РАН,\\ 
доктор физико-математических наук \hfill
      \textit{В.\,Ю.~Королев}
     
     } }
     }


\renewcommand{\figurename}{\protect\bf Figure}
\renewcommand{\tablename}{\protect\bf Table}

\def\stat{kabanov}


\def\tit{ON UNIQUENESS OF CLEARING VECTORS REDUCING~THE~SYSTEMIC RISK}

\def\titkol{On uniqueness of clearing vectors reducing the systemic risk}

\def\autkol{Kh.\ El Bitar,  Yu.~Kabanov, and~R.~Mokbel}

\def\aut{Kh.\ El Bitar$^1$,  Yu.~Kabanov$^2$, and~R.~Mokbel$^3$}

\titel{\tit}{\aut}{\autkol}{\titkol}

%{\renewcommand{\thefootnote}{\fnsymbol{footnote}}
%\footnotetext[1] {The 
%research of Yuri Kabanov was done under partial financial support   of the grant 
%of  RSF No.\,14-49-00079.}}

\renewcommand{\thefootnote}{\arabic{footnote}}
\footnotetext[1]{Laboratoire de Math$\acute{\mbox{e}}$matiques, Universit$\acute{\mbox{e}}$ de 
Franche-Comt$\acute{\mbox{e}}$, 16~Route de Gray, 25030 \mbox{Besan{\!\ptb{\c{c}}}on}, CEDEX, France, 
\mbox{khalilbitar\_aw@hotmail.com}}
\footnotetext[2]{Laboratoire de 
Math$\acute{\mbox{e}}$matiques, Universit$\acute{\mbox{e}}$ de
 Franche-Comt$\acute{\mbox{e}}$, 16~Route de Gray, 25030 
\mbox{Besan{\!\ptb{\c{c}}}on}, CEDEX, France; 
Institute of Informatics Problems, Federal Research 
Center ``Computer Science and Control'' of the Russian Academy of Sciences, 
44-2~Vavilov Str., Moscow 119333, Russian Federation; 
National Research University 
``MPEI,'' 14~Krasnokazarmennaya Str., Moscow, 111250, Russian Federation, 
\mbox{Youri.Kabanov@univ-fcomte.fr}}
\footnotetext[3]{Laboratoire de 
Math$\acute{\mbox{e}}$matiques, Universit$\acute{\mbox{e}}$ de 
Franche-Comt$\acute{\mbox{e}}$, 
16~Route de Gray, 25030  \mbox{Besan{\!\ptb{\c{c}}}on}, CEDEX, France,
\mbox{ritamokbel@hotmail.com}}

\index{El Bitar Kh.}
\index{Kabanov Yu.}
\index{Mokbel R.}
\index{Эль Битар Х.}
\index{Кабанов Ю.}
\index{Мокбель Р.}


\vspace*{-12pt}

\def\leftfootline{\small{\textbf{\thepage}
\hfill INFORMATIKA I EE PRIMENENIYA~--- INFORMATICS AND APPLICATIONS\ \ \ 2017\ \ \ volume~11\ \ \ issue\ 1}
}%
 \def\rightfootline{\small{INFORMATIKA I EE PRIMENENIYA~--- INFORMATICS AND APPLICATIONS\ \ \ 2017\ \ \ volume~11\ \ \ issue\ 1
\hfill \textbf{\thepage}}}




\Abste{Clearing of financial system, i.\,e., of a~network of interconnecting banks, is 
a~procedure of simultaneous repaying debts to reduce their total volume. The 
vector whose components are  repayments of each bank
is called clearing vector.  In  simple models  considered  by Eisenberg and Noe 
(2001) and, independently,  by Suzuki (2002), it was shown that
the  clearing  to the minimal value of debts  accordingly to natural rules  can 
be formulated as fixpoint problems.
The existence
of their solutions, i.\,e., of clearing vectors,  is rather straightforward and can 
be obtained by a~direct reference to the Knaster--Tarski or Brouwer theorems.  
The uniqueness of clearing vectors is a~more delicate problem which was solved 
by Eisenberg and Noe  using a~graph structure of the financial network.  
The uniqueness  results have been proved in two generalizations of the  Eisenberg--Noe model:  
in the Elsinger model with seniority of liabilities and in the Amini--Filipovic--Minca 
type model with several
types of illiquid assets whose firing sale has a~market impact.}

\KWE{systemic risk;  financial networks; clearing; Knaster--Tarski 
theorem; Eisenberg--Noe model; debt seniority; price impact}

\DOI{10.14357/19922264170110} 

\vspace*{7pt}


\vskip 12pt plus 9pt minus 6pt

      \thispagestyle{myheadings}

      \begin{multicols}{2}

                  \label{st\stat}


\section{Introduction}

\noindent
To explain the clearing problem, let us start with the simplest example of 
a~financial
system with two agents each having in a~cash 10 dollars. The first agent gets 
from the second a~credit of~1M  dollars, the second gets from the first  a~credit 
of~1~M and 1~dollars. Apparently, as a~result, both agents have a~huge liabilities with 
respect to each other. Of course, the agents can be asked to reduce their 
liabilities by reimbursing credits partially (e.\,g., to the levels~0.5~M and 
0.5~M\;+\;1 in liabilities and~10~dollars both in cash) or completely, with zero 
liabilities and cash reserves~11 and~9~dollars, respectively. Intuitively, the 
situation where the liability is reduced (i.\,e., the system is cleared) seems to 
be less risky: if one of the agents became bankrupt and only the percentage of the 
huge debt value  can be reimbursed, the creditor's losses will be also huge. For 
complex financial systems involving large numbers of agents
with chains of borrowing,  the clearing problem, that is, the reduction of 
absolute values by reimbursement, looks much more complicated.
{ %\looseness=1

}

In the influential paper~\cite{Eisenberg-Noe} published in 2001, Eisenberg and 
Noe suggested a~clearing procedure in the model describing a~financial system 
composed by~$N$~banks (under ``banks''  can be understood  various financial 
institutions); a~more general model was introduced independently at the same 
time by Suzuki~\cite{Suzuki}.   The assets of the bank are cash and interbank 
exposures which are, in turn, liabilities for its debtors.  The clearing 
consists in simultaneous paying all debts. Each bank pays to its counterparties 
the debts \textit{pro rata} of their relative volume using its cash reserve and 
money collected from the credited banks. The rule is: either all debts are payed 
in full or the zero level of the equity is attained and the bank defaults. The 
totals reimbursed by banks form an $N$-dimensional clearing vector. A~remarkable 
feature is that this vector is a~fixed point of a~monotone mapping of a~complete 
lattice into itself and its existence follows immediately from the 
Knaster--Tarski  theorem, a~beautiful and fairy simple result which proof needs only 
a~few lines of arguments~\cite{Tarski}. The uniqueness of the clearing vector is a~more 
delicate 
result involving the graph structure of the system.

The ideas of the  Eisenberg--Noe paper happened to be very fruitful and their 
model
was generalized in many directions having not only financial importance but 
posing  interesting mathematical questions. One of them is the question on 
uniqueness of clearing vector or   equilibrium  on financial market.

The first theorem provides a~new sufficient condition for the Elsinger model of 
clearing with debts priority structure. This model is given by a~set of 
liability matrices corresponding to each seniority. The idea of the present approach is 
to use the largest clearing vector which always exists to construct a~new 
liability matrix generating a~graph structure with which one can work in 
a~similar way as in the Eisenberg--Noe model.
The second theorem deals with the uniqueness of  equilibrium in a~clearing  
model with several illiquid assets and a~market impact.  In the presence of 
several illiquid assets,  the banks are faced the choice of  asset selling 
strategies. The proportional scheme of selling similar to that in the 
paper by Cont--Wagalath~\cite{Cont-Wag} has been used 
leaving game-theoretical versions for 
future studies.  In the case of one illiquid asset, the
obtained result is close to that  
of the study by Amini--Filipovic--Minca~\cite{AFM}, but the present definition of the 
equilibrium is different (but equivalent).

The structure of the note is as follows. In the introductory section~2, 
 the general principle and results are discussed briefly in the framework of the 
Eisenberg--Noe model. To facilitate the comparison with further development, 
also, short proofs are provided.
In section~3, a~uniqueness of the clearing vector for the Elsinger model 
where senior  liabilities should be reimbursed before the juniors ones. Section~4 
contains
the sufficient condition  for the uniqueness of the equilibrium in the model
where clearing requires selling of the illiquid assets with price impact.  
Economically speaking, it is  oriented to the recovering of the market  after 
fire sales.  For the reader convenience,  in Appendix, 
a~short information about the Knaster--Tarski theorem adapted to 
the present authors' needs is provided.



\noindent
\textbf{Notations.}\ The partial ordering in~$\mathbb{R}^n$ and its 
subsets  induced by the cone~$\mathbb{R}^n_+$ is denoted by $\ge$. In other words, the inequality $y\ge x$ 
is understood componentwise. Also, the symbols $x\wedge y$ and $x\vee y$ mean, 
respectively, the componentwise minimum and maximum, $x^+:=x\vee 0$ and\linebreak 
$x^-:=(- x)^+$.
The notation $[x,z]$ is used for the order interval, i.\,e.,
$[x,z]=\{y\in \mathbb{R}^n:\ x\le y\le z \}$.
If $A\subseteq [x,z]$, then $\inf A$ is the unique element $\underline y\in 
[x,z]$ such
that $\underline y\le y$ for all $y\in A$ and for any $\tilde y$  such that 
$\tilde y\le y$ for all $y\in A$, one has $\tilde y\le \underline y$, that 
is, the component $\underline y^i=\inf \{y^i:\ y\in A\}$ for  $i=1,\dots,n$.

The matrix notations are used where the vectors are columns, $'$ is the symbol of 
transpose, and\linebreak  ${\bf 1}':=(1,\dots,1)$ (the dimension of the vector is supposed 
to be clear from the context).

\vspace*{-9pt}

\section{The Eisenberg--Noe Model}

\noindent
In~\cite{Eisenberg-Noe}, Eisenberg and Noe investigated the model 
describing a~financial system composed of $N$ banks (under ``banks"  can be 
understood  various financial institutions). In the aggregate oversimplified  
form, the balance sheet of the bank $i$ can be split into two parts: assets and 
liabilities. The assets are of two types:  interbank assets (exposures)~$\tilde X^i$ 
and cash~$e^i$.  The liabilities are: interbank debts (liabilities)~$\tilde L^i$ 
and the equity~$C ^i$ (or proper capital reserve) equalizing the two sides 
of the balance sheet:
\vspace*{2pt}

\noindent
$$
e^i+\tilde X^i= \tilde L^i + C^i\,.
$$

\vspace*{-2pt}

\noindent
All these values are assumed to be greater or equal to zero. The condition that 
$C^i\ge 0$ means that the bank is solvent.

More detailed balance sheet provides the information on the values  of 
liabilities of the bank  $i$ to the bank $j$, namely,  vectors 
$(L^{i1},\ldots,L^{iN})'$ of liabilities and  $(X^{i1},\ldots,X^{iN})$ of exposures.
 With this, one 
has  $\tilde X^i=X^{i1}+\cdots+X^{iN}$ and $\tilde L^i =L^{i1}+\cdots+L^{iN}$.

The matrix $L=(L^{ij})$ with positive entries and zero diagonal defines the
total interbank exposures. Since the value of the exposure of~$i$ to~$j$ is the 
value of the liability of~$j$ to~$i$, one has that $L'=X$.  So, 
the matrix $L$ and the vector~$e$ give a~description of a~financial system in 
this model.

Put

\vspace*{-3pt}

\noindent
$$
\Pi^{ij}:=
\begin{cases}
\fr {L^{ij}}{\tilde L^i}=\displaystyle \fr {L^{ij}}{\sum\nolimits_j L^{ij}} 
&\ \mbox{if } \tilde L^i\neq 0\,; \\
\delta^{ij} &\  \mbox{otherwise}
\end{cases}
$$
where the Kronecker symbol $\delta^{ij}=0$ for $i\neq j$ and $\delta^{ii}=1$.
Then,~$\Pi^{ij}$  describes the proportion of the value debtor $i$ due to the 
creditor~$j$ of the total interbank debt of~$i$; $\Pi=(\Pi^{ij})$  is called 
relative liabilities matrix. Note that in this definition, to get a~stochastic 
matrix $\Pi$, we deviate from~\cite{Eisenberg-Noe} where $\Pi^{ii}=0$ when 
$L^i= 0$.

%As an example consider the simplest system with two banks where 
%$L^{12}=L^{21}+\varepsilon$ where $\e<0$ can be thought small with respect to~$L^{21}$.  
%After paying debts in the cleared system the matrix of liabilities will have the 
%entries $L^{12}_{c}=\e$, $L^{21}_c=0$. That is the values of debts are reduced 
%and so are eventual values of losses in the case of defaults of a~partner.

In general, financial system   may have a~complicated structure with cyclical 
interdependences and  banks may have large exposures within cycles. To reduce 
them, one can impose a~clearing mechanism satisfying several natural 
requirements: limited liability and proportionality. Formally,  this  leads to 
the concept of a~\textit{clearing payment vector} $p^*\in \prod_i[0,\tilde L^i]$ 
satisfying the following properties:
\begin{itemize}
\item[$a.$] \textit{Limiting liability}. For every $i$,
$$
p_i^*\le e^i+\sum\limits_j\Pi^{ji}p_j^*\,.
$$

\item[$b.$] \textit{Absolute priority.} For every $i$, either $p^*_i=\tilde L^i$, or
$$
p_i^*= e^i+ \sum\limits_j\Pi^{ji}p_j^*.
$$
\end{itemize}
One may think that the  central clearing authority forces  each bank to make 
a~``fair'' payment of debts in such\linebreak\vspace*{-12pt}

\pagebreak

\noindent
 a~way that, having  the total payment~$p_i^*$, 
the bank~$i$ remains solvent and  pays to~$j$ the fraction $p_i^*\Pi^{ij}$ in 
such a~way that either its total debts are paid,  or all the resources are 
exhausted.

Alternatively, the conditions~$a$ and~$b$ can be written in the following way:
\begin{equation}
\label{p^*} p^*=\min \left\{ e+\Pi' p^*, \tilde L\right\}
\end{equation}
where the minimum is understood in the componentwise sense, i.\,e., accordingly to 
the partial ordering defined by the cone~${\mathbb{R}}^N_+$.

The main result of Eisenberg and Noe asserts that the set of clearing vectors is 
nonempty. Moreover, there are the minimal and the maximal clearing vectors,  
denoted here~$\underline p$ and~$\bar p$, respectively.  This assertion follows
immediately from the Knaster--Tarski fixed point theorem: the monotone mapping 
$f:p\mapsto (e+\Pi'p)\wedge \tilde L$ of a~complete lattice $[0,\tilde L]$ into 
itself has the largest and the smallest fixed points (for 
details, see section~5). The set $[0,\tilde L]$ is convex and compact and~$f$ is a~continuous 
mapping. So, the existence of its fixed point follows also from the classical  
Brouwer theorem.

Using the obvious identity $(x-y)^+=x -x\wedge y$, one can rewrite 
Eq.~(\ref{p^*}) in the following equivalent form:
\begin{equation}
\label{alt1}
\left(e+\Pi'p^*-\tilde L\right)^+=e+\Pi'p^*-p^*
\end{equation}
where the left-hand side is the equity vector of the system after clearing.

%After clearing by an arbitrary clearing (outflow) vector $p^*$ the equity 
%vector of the system  is
%$$(e+\Pi'p^*-\tilde L)^+=e+\Pi'p^*-p^*; $$
%this equality is nothing but the equation (\ref{p^*}) written in an equivalent 
%form.

An important but simple observation: {\it the equity (after clearing) does not 
depend on the clearing vector}.
Indeed,~$\Pi$~being  a~stochastic matrix, ${\bf 1}'\Pi'={\bf 1}'$.  Therefore, 
multiplying  the above representation~(\ref{alt1}) from the left by~${\bf 1}'$, 
one gets that  the sum of equities
$$
{\bf 1}'\left(e+\Pi'p^*-\tilde L\right)^+={\bf 1}'e
$$
is equal to the sum of the initial cash reserves, that is, invariant with respect 
to the choice of the clearing vector.
On the other hand, by monotonicity, one has that
$$
\left(e+\Pi'p^*-\tilde L\right)^+\le \left(e+\Pi'\bar p-\tilde L\right)^+.
$$
If the both sides here are not equal, then
$$
{\bf 1}'\left(e+\Pi'p^* - \tilde L\right)^+< {\bf 1}'\left(e+\Pi'\bar p-\tilde L\right)^+$$
in contradiction with the invariance of the  total of equities.

\smallskip

\noindent
\textbf{Sufficient condition for the uniqueness of the clearing vector.}
As in~\cite{Eisenberg-Noe}, let us assume for simplicity that $\tilde L^i>0$ 
for all~$i$.

For a~stochastic matrix~$\Pi$,  we say that
$I\subseteq  \{1,\ldots,N\}$ is
a~($\Pi$-)\textit{surplus set} if $\Pi^{ij}=0$ for all $i\in I$, $j\in I^c$, 
and~$\sum_{j\in I}e^j>0$.

\columnbreak

Recall that~$j$ is the creditor of $i$ if $\Pi^{ij}>0$ (i.\,e., $\Pi^{ij}>0$); in 
this case, let us use, as in the  theory of Markov chains or in the graph 
theory,  the notation $i\to j$.

Let us denote by $o(i)$ {\it the orbit of $i$} that is the set of all~$j$ for which 
there is a~directed path 
$$
i\to i_1\to i_2\to\cdots\to j\,,$$ 
i.\,e.,  $o(i)$ is the set 
of all direct or indirect creditors of~$i$.

Note that the orbit $o(i)$ with $\sum_{j\in I}e^j>0$ is a~surplus set. Indeed,
if\ $\Pi^{jj'}>0$ for some $j\in o(i)$, $j'\notin o(i)$, i.\,e.,  $j\to j'$, then 
there is
a~path 
$$i\to i_1\to i_2\to\cdots\to j \to  j'\,.
$$


\noindent
\textbf{Lemma~1.}\
%\label{equity>0}
 \textit{Suppose that the market is cleared by a~vector $p^*\in [0,\tilde L]$. Let~$I$ 
be a~surplus set.  Then, at least one node of~$I$ has a~strictly positive equity 
value}.

\textit{In particular,
any orbit~$o(i)$ with $\sum_{j\in o(i)}e^j>0$ has an element with strictly  
positive equity value}.

\smallskip

\noindent
P\,r\,o\,o\,f\,.\ \  Multiplying the identity~(\ref{alt1}) by~${\bf 1}'_I$ and noticing 
that
$({\bf 1}'_I\Pi')^i=1$ for $i\in I$,
one obtains that
$$
{\bf 1}'_I \left(e+\Pi'p^*-\tilde L\right)^+\ge {\bf 1}'_I e>0
$$
implying the claim.~$\square$

\smallskip

A financial system is called \textit{regular} if for  every~$i$, the orbit~$o(i)$ is 
a~surplus set.

\smallskip

\noindent
\textbf{Theorem~1.}\
%\label{uni1}
\textit{Suppose that the financial system is regular.
Then}, $\underline p=\bar p$.

\smallskip

\noindent
P\,r\,o\,o\,f\,.\ \  Suppose that~$\underline p$ and~$\bar p$ are not equal, i.\,e., 
$\underline p\le \bar p$ but for some~$i$, one 
has the strict inequality  $\underline p^i<\bar p^i$.
Denote by~$C$ the vector of equities (it is common for all clearing vectors).
By assumption, the orbit~$o(i)$ is a~surplus set and by Lemma~1, it 
contains an element~$m$ with the equity value $C^m>0$. By definition of the 
orbit, there is a~path $i\to i_1\to \cdots \to m$ and one may assume without loss of 
generality that in this path,~$m$ is  the first node with strictly positive 
equity value.

First, let us prove that  one may consider only the case where the path
consists of one step,  i.\,e., $i\to m$.  To this end, let us check that
$\underline p^{i_1}<\bar p^{i_1}$ if $i_1\neq m$. In other words, the property 
that $\underline p^i\neq \bar p^i$ propagates along the path.

Suppose that $\bar p^{i_1}< \tilde L^{i_1}$. Then, also, $\underline p^{i_1}< 
\tilde L^{i_1}$.  In such a~case,

\vspace*{3pt}

\noindent
$$
 \underline p^{i_1}=e^{i_1}+ \sum\limits_j\Pi^{ji_1}\underline p^j\,, \enskip \bar 
p^{i_1}=e^{i_1}+\sum\limits_j\Pi^{ji_1}\bar p^j
$$
and one has  that

\vspace*{3pt}

\noindent
$$
\bar p^{i_1}-\underline p^{i_1}=\sum\limits_j\Pi^{ji_1}\left(\bar p^j-\underline p^j\right)>0
$$

\vspace*{-6pt}

\noindent
because   $\Pi^{ii_1}>0$, that is, $\underline p^{i_1}<  \bar p^{i_1}$. This 
inequality also holds trivially, if
$\bar p^{i_1}= \tilde L^{i_1}$ but $\underline p^{i_1}< \tilde L^{i_1}$.
 The remaining\linebreak\vspace*{-12pt}
 
 \pagebreak
 
 \noindent
  case where
$\underline p^{i_1}=\bar p^{i_1}=\tilde L^{i_1}$ is excluded as it is supposed that 
$C^{i_1}=0$.  Indeed, according to~(\ref{alt1}),  this leads to the equalities:
$$
e^{i_1}+ \sum\limits_j\Pi^{ji_1}\bar p^j - \tilde L^{i_1}=0\,;\enskip
e^{i_1}+  \sum\limits_j\Pi^{ji_1}\underline p^j - \tilde L^{i_1}=0\,,
$$
implying the identity
$$
\sum\limits_j\Pi^{ji_1}\left(\bar p^j-\underline p^j\right)=0
$$
which cannot be true since in the above sum, the term corresponding to $j=i$ is 
strictly positive.

So, it is sufficient to consider only one-step case. Since $C^m>0$, one has the 
representations:
\begin{align*}
C^m&=e^{m}+ \sum\limits_j\Pi^{jm}\underline p^j - \tilde L^{m}\,; \\
C^m&=e^{m}+ \sum\limits_j\Pi^{jm}\bar p^j- \tilde L^{m}\,.
\end{align*}
As above, one again obtains the impossible equality:
$$
\sum\limits_j\Pi^{jm}\left(\bar p^j-\underline p^j\right)=0\,.
$$
Therefore, the  assumption $\underline p^i<\bar p^i$ leads to a~contradiction. 
The
uniqueness of clearing vector is proven.~$\square$


\smallskip

\noindent
\textbf{Remark~1.}\
The above  theorem reveals that the problem to find a~clearing 
vector is ill-posed. Indeed, adding an infinitesimally small amount $\varepsilon>0$ 
(say,  one cent) to the initial endowments leads to a~unique clearing vector. Similar 
effect will have small increase in liabilities. One can think that the ``true'' 
liability matrix has all elements strictly positive and that in the model matrix, zero 
elements appeared because liabilities are neglected.
These phenomena are related to the ill-posedness of the spectral problem for 
stochastic matrices. Another question is which clearing vector is natural.


\smallskip


The above proof  is rather straightforward and uses graph-theoretical language.  
One can get another one  appealing to the contraction property of the mapping 
$f:p\mapsto (e+\Pi'p)\wedge \tilde L$ defined on the set $[0,\tilde L]$ equipped 
with $l_1$-distance $|p-\tilde p|_1$.

\smallskip

\noindent
\textbf{Proposition.}\
For every $p,\tilde p\in [0,\tilde L]$
\begin{equation*}
%\label{non-exp}
\left\vert f(p)-f(\tilde p)\right\vert_1\le \left\vert\Pi' (p-\tilde p)
\right\vert_1\le \left\vert p-\tilde p\right\vert_1\,.
\end{equation*}
Moreover, the first relation above is the equality if and only if the
union of subsets $A:=\{i:\ (\Pi'p)^i=(\Pi'\tilde p)^i\}$ and $B:=\{i:\ 
(\Pi'p)^i,(\Pi'\tilde p)^i\le \tilde L^i-e^i\}$ is the set of indices 
$\{1,\dots, N\}$.

\smallskip
%Moreover, if for each $i$ the sum $\sum_{j\in o(i)} e^i>0$, then the mapping 
%$f$ is a~contraction %on the set ${\rm Fix}_f$, i.e. the above inequality is 
%strict when the fixed points $p\neq \tilde p$.

\noindent
P\,r\,o\,o\,f\,.\ \ Using the elementary inequality $|a\wedge c-b\wedge c|$\linebreak $\le |a-b|$ 
which holds as  the
equality if and only if when\linebreak $a=b$ or $a,b\le c$, one obtains that
$|f(p)-f(\tilde p)|_1$\linebreak $\le |\Pi'p-\Pi'\tilde p|_1$
where the equality holds if and only if for every~$i$, one has 
$(\Pi'p)^i=(\Pi'\tilde p)^i$ or
$(\Pi'p)^i,(\Pi'\tilde p)^i$\linebreak $\le \tilde L^i-e^i$. Since $|\Pi'y|_1\le 
|\Pi'|_1|y|_1$ and $|\Pi'|_1=1$, one has the claim.~$\square$

\smallskip

Let us consider  the case where the matrix~$\Pi$ is irreducible. Suppose that 
${\bf 1}'e>0$ and~$p$ and~$\tilde p$ are two different fixed points of the 
mapping~$f$. According to above proposition,
$$
\sum\limits_{j\in B}\Pi^{ji}\left(p^j-\tilde p^j\right)=p^ i-\tilde p^i\,, \enskip i\in B\,.
$$
This means that  the nonzero vector with the coordinates $p^ i-\tilde p^i$, 
$i\in B$, is a~left eigenvector of the matrix
$(\Pi^{ij})_{i,j\in B}$ corresponding to unit eigenvalue. This is possible only 
if the latter matrix coincides with~$\Pi$. Thus, $p=f(p)=e+\Pi'p$. Since  
${\bf 1}'\Pi'p={\bf 1}'p$, one gets that ${\bf 1}'e=0$
which is a~contradiction.  Using the decomposition of the matrix~$\Pi$ on the 
irreducible component, one gets that  the clearing vector  is unique if for any 
irreducible component, there is a~node with strictly positive initial endowment.



\section{The Elsinger Model}

\noindent
In the present paper,  a~simplified version of the Elsinger model
introduced in~\cite{Elsinger2011}, where the interbank debts may be 
senior and junior, is considered. In this model, the system of~$N$ banks is described by the 
vector
of cash reserves and by~$M$~matrices $L_1=(L^{ij}_1), \ldots, L_M=(L^{ij}_M)$ 
representing the hierarchy of liabilities with decreasing seniority,  that is, 
the element~$L^{ij}_1$ represents the debt of the bank~$i$ to the bank~$j$ of the 
highest seniority, etc.,  $\sum_jL^{ij}_S$ is the total of  debts of the bank~$i$ 
of the seniority~$S$.

The relative liabilities are defined by  the matrix~$\Pi_S$ with
$$
\Pi_S^{ij}=\fr {L_S^{ij}}{\tilde L_S^i}=\fr {L_S^{ij}}{\sum\nolimits_j L_S^{ij}}\,.
$$
The clearing procedure requires the complete reimbursement of the debts starting 
from the highest priority and for each seniority level, the distribution is 
proportional
to the volume of debts of this seniority. For the bank~$i$, let us denote  by $p^i_S$ 
the value distributed to cover the debts of the seniority~$S$. So, the clearing 
can be described by the set of vectors~$p_S$, $S=1,\ldots, M$, which can be 
considered as a~``long'' vector from~$(\mathbb{R}^N)^M$  satisfying the system of 
equations:
\begin{equation*}
p_{1}^{i}=\min \left\{e^i+\sum\limits_S \sum\limits_j\Pi_S^{ji}p_{S}^{j}, \tilde L_1^i 
\right\}\,;
\end{equation*}
\begin{align*}
p_{S}^{i}&=\min\left\{\left(e^i+\sum\limits_S \sum\limits_j\Pi_S^{ji}p_{S}^{j}-
\sum\limits_{r<S}\tilde 
L_r^i\right)^+, \tilde L_S^i \right\}\,,  \\
&\hspace*{57mm}1<S\le M\,.
\end{align*}
In a~vector form, these equations can be written as follows:

\vspace*{-4pt}

\noindent
\begin{multline}
\label{SM}
p_{S}^{}=\left(e+\sum\limits_S \hspace*{-1.2pt}
\Pi_S'p_{S}-\sum\limits_{r<S}\hspace*{-1.2pt}\tilde L_r\right)^+\wedge  
\tilde  L_S\,,  \\ S=1,\ldots,M\,.
\end{multline}
It is clear that for the partial ordering in~$(\mathbb{R}^N)^M$ induced by the 
cone~$(\mathbb{R}^N_+)^M$, the function

\vspace*{-4pt}

\noindent
\begin{multline*}
\left(p_1,\ldots,p_M\right)\mapsto \left(
\left(e+\sum\limits_S \Pi_S'p_{S}^* \right)^+\wedge \tilde L_1 
,\ldots\right.\\
\left.\ldots,\left(e+\sum\limits_S \Pi_S'p_{S}^*-\sum\limits_{r<M}\tilde L_r\right)^+ 
\wedge L_M 
\right)
\end{multline*}
is a~monotone mapping of the order interval 
$$
[0,\tilde L_1] \times\cdots\times 
[0,\tilde L_M]\subset (\mathbb{R}^N)^M
$$ 
into itself.
 Thus, according to the Knaster--Tarski theorem, the set of fixed points of this 
mapping, i.\,e., the solutions of Eq.~(\ref{SM}), is nonempty and has the 
maximal and the minimal elements.

In the case of liabilities of different seniority after clearing by the vector 
$p\in (\mathbb{R}^N)^M$,  the equity vector $C\in \mathbb{R}^N$ has the form:
$$
C=\left(e+\sum\limits_S \Pi_S'p_{S}-\sum\limits_S \tilde L_S\right)^+\,.
$$

%\smallskip

\noindent
\textbf{Lemma~2.}\
\textit{The equity vector does not depend on the clearing vector}.

\vspace*{2pt}

\noindent
P\,r\,o\,o\,f\,.\ \  Note that
$$
\left(e+\sum\limits_S\Pi'_Sp_S\right)\wedge \sum\limits_S \tilde L^i_S=\sum\limits_S p_S\,.
$$
Therefore,
$$
\left(e+\sum\limits_S \Pi_S'p_{S}-\sum\limits_S \tilde L_S\right)^+=
e+\sum\limits_S \Pi_S'p_{S}-\sum\limits_S  p_{S}\,.
$$
With this identity, the reasoning is analogous to that with a~single seniority 
class.~$\square$

\vspace*{2pt}

The aim of this section is to provide a~sufficient condition for the uniqueness 
of clearing vector using a~specific graph structure induced by the matrices~$\Pi_S$.

For a~given clearing vector~$p$, let us define the \textit{default index}~$d^i$ of the 
node~$i$ as the smallest~$r$  such that
$$
\bar p_r^i=e^i+ \sum\limits_S \sum\limits_j\Pi_S^{ji}\bar p_{S}^j-\sum\limits_{r'< r}\tilde 
L_{r'}^{i}\,.
$$
In another words,~$d^i$ is the lowest seniority for which the bank equity after 
clearing is equal to zero. Define the matrix $\Delta=\Delta(p)$ by putting 
$$
\Delta^{ij}=
\begin{cases}
1 &\ \mbox{if\ \ } \Pi_{d(i)}^{ij}>0\,;\\
0 &\ \mbox{otherwise}.
\end{cases}
$$

%\columnbreak

\noindent
Let us use 
the notation $i\leadsto j$ if $\Delta^{ij}=1$ and  denote by $O(i)$ \textit{the 
$\Delta $-orbit of $i$} that is the set of all~$j$ for which there is 
a~directed path $i\leadsto i_1\leadsto i_2\leadsto\cdots\leadsto j$.

\vspace*{2pt}

\noindent
\textbf{Theorem~2.}\
\textit{Suppose that for the clearing vector $\bar p$, any $\Delta $-orbit is a~surplus 
set.
Then, the clearing vector is unique}.

\vspace*{2pt}

\noindent
P\,r\,o\,o\,f\,.\ \  By definition, the default index
$$
d^i:=\min\left\{r:\ \bar p_r^i=e^i+ \sum\limits_S \sum\limits_j\Pi_S^{ji}
\bar p_{S}^j-\sum\limits_{r'<  r}\tilde L_{r'}^{i}\right \}\,.
$$
It follows that $\bar p_r^i=0$; hence,  $\underline p_r^i=0$ for every $r>d^i$.
Suppose that
$\underline p_{d^i}^i<\bar p_{d^i}^i$ and consider a~path 
$$
i\leadsto  i_1\leadsto i_2\leadsto\cdots \leadsto m
$$ 
ending up at the node with strictly  positive equity value.

First, let us show that at least for one seniority $\underline p^{i_1}_S<\bar 
p^{i_1}_S$.

Let $r':=d^{i_1}$.  By definition, one has: 
$$
\bar p^{i_1}_r=\begin{cases}
\tilde L^{i_1}_r\,, & r\le r'\,;\\
0\,,  & r>r'\,.
\end{cases}
$$
 The claim 
holds, if  $\underline p^{i_1}_r<\tilde L^{i_1}_r$
for some $r<r'$. Thus, it remains to consider only the case where $\underline 
p^{i_1}_r=\bar p^{i_1}_r = \tilde L^{i_1}_r$
for all $r<r'$ and prove that  $\underline p^{i_1}_{r'}<\bar p^{i_1}_{r'}$.
One has the alternative: either $\underline p^{i_1}_{r'}<\bar p^{i_1}_{r'}\le  
\tilde L^{i_1}_r$ (what is needed), or
$\underline p^{i_1}_{r'}=\bar p^{i_1}_{r'}\le  \tilde L^{i_1}_r$. The second 
case is impossible, since the equalities

\noindent
\begin{align*}
\bar p^{i_1}_{r'}&=e^{i_1}+ \sum\limits_S \sum\limits_j\Pi_S^{ji_1}\bar p_{S}^j-
\sum\limits_{r<  r'}\tilde L_{r}^{i_1}\,;\\
\underline p^{i_1}_{r'}&=e^{i_1}+ \sum\limits_S 
\sum\limits_j\Pi_S^{ji_1}\underline p_{S}^j-
\sum\limits_{r< r'}\tilde L_{r}^{i_1}
\end{align*}
imply that

\noindent
\begin{multline*}
\bar p^{i_1}_{r'}-\underline p^{i_1}_{r'}=\sum\limits_S \sum\limits_j\Pi_S^{ji_1}
\left(\bar  p_{S}^j-\underline  p_{S}^j\right)\\
{}\ge \Pi_{d^i}^{ii_1}
\left(\bar p_{d^i}^i-\underline   p_{d^i}^i\right)>0\,.
\end{multline*}
This is a~contradiction.

\pagebreak

The above argument reduces the problem to the case $i\leadsto m$ and the node~$m$ 
has a~strictly positive equity.  The equity~$C^m$ does not depend on the 
clearing vector.  Therefore,

\noindent
\begin{align*}
C^m&=e^{m}+ \sum\limits_S \sum\limits_j\Pi_S^{jm}\bar p_{S}^j-
\sum\limits_{S}\tilde L_{S}^{m}\,;\\
C^m&=e^{m}+ \sum\limits_S \sum\limits_j\Pi_S^{jm}\underline p_{S}^j-
\sum\limits_{S}\tilde L_{S}^{m}\,.
\end{align*}


\noindent
It follows that
$$
0=\sum\limits_S \sum\limits_j\Pi_S^{jm}\left(\bar p_{S}^j-\underline p_{S}^j\right)\ge 
\Pi_{d^i}^{im}\left(\bar p_{d^i}^i-\underline  p_{d^i}^i\right)>0\,.
$$
This contradiction shows that $\underline p=\bar p$.

\subsection{Example~1}

\noindent
Let us consider the system consisting of~3~nodes with the initial cash 
endowments
given by the vector $e=(0.1,0,0)$ and the liability and the "distribution"  
matrices corresponding to senior and junior debts:
\begin{alignat*}{2}
L_S&=
\begin{pmatrix}
0 & 1 & 0\\
1 & 0 & 1\\
0 & 2 & 0
\end{pmatrix}\,; &\enskip
L_J&=\begin{pmatrix}
0 & 0 & 0\\
0& 0 & 2\\
0 & 0 & 0
\end{pmatrix}\,;
\\[9pt]
\Pi_S&=
\begin{pmatrix}
0 & 1 & 0\\
0.5 & 0 & 0.5\\
0 & 1 & 0
\end{pmatrix}\,; &\enskip
\Pi_J&=\begin{pmatrix}
0 & 0 & 0\\
0& 0 & 1\\
0 & 0 & 0
\end{pmatrix}.
\end{alignat*}
For this model, the vectors of total liabilities corresponding to the senior and 
junior debts are, respectively, $\tilde L_S=(1,2,2)$ and   $\tilde L_J=(0,2,0)$.

The equations for clearing vectors are:
\begin{align*}
p_S^1 & =  \left(0.1+0.5\, p_S^2 \right)\wedge 1\,;\\
p_S^2 & =  \left(p_S^1+p_S^3 \right)\wedge 2\,;\\
p_S^3 & =  \left(0.5\, p_S^2+p_J^2\right)\wedge 2\,;\\
p_J^1 & = 0\,;\\
p_J^2 & = \left(p_S^1+p_S^3-2\right)^+\wedge 2\,;\\
p_J^3 & = 0.
\end{align*}
It is not difficult to check that there are infinite set of clearing vectors.
Namely, one has that $p_S=(1,2,1+t)$ and $p_J=(0,t,0)$ where $t\in [0,1]$.
The minimal clearing vector corresponds to $t=0$ and the maximal corresponds to 
$t=1$.

\subsection{Example~2}

\noindent
The vector of cash endowments and the matrix of the senior debts  is the same as 
in Example~1. The junior debts matrix $L_J$ and the corresponding 
distribution matrix~$\Pi_J$ are now:
$$
L_J=\begin{pmatrix}
0 & 0 & 0\\
0.4& 0 & 1.6\\
0 & 0 & 0
\end{pmatrix}\,;
 \enskip
\Pi_J=\begin{pmatrix}
0 & 0 & 0\\
0.2& 0 & 0.8\\
0 & 0 & 0
\end{pmatrix}\,.
$$
We are looking for positive solutions of the following  equations:
\begin{align*}
p_S^1 & =  \left(0.1+0.5\, p_S^2 + 0.2\, p_J^2\right)\wedge 1\,;\\
p_S^2 & =  \left(p_S^1+p_S^3 \right)\wedge 2\,;\\
p_S^3 & =  \left(0.5\, p_S^2+0.8\, p_J^2\right)\wedge 2\,;\\
p_J^1 & = 0\,;\\
p_J^2 & = \left(p_S^1+p_S^3-2\right)^+\wedge 2\,;\\
p_J^3 & = 0\,.
\end{align*}
Note that $p_S^1\le 1$ and $p_S^2\le 2$; hence, $p_J^2\le 1$ and the 3rd equation 
is linear:
\begin{equation}
\label{pS3}
p_S^3  =  0.5\, p_S^2+0.8\, p_J^2.
\end{equation}
Substituting into the 2nd equation this expression for~$p_S^3$ and the 
expression for~$p_S^1$ from the 1st equation, one gets that
\begin{equation*}
p_S^2 \!=\!\left(\!\left(0.1+0.5\, p_S^2 + 0.2\, p_J^2\right)\wedge 1+
0.5\, p_S^2+0.8\, p_J^2 \right)\wedge 2.
\end{equation*}
The inequality $p_S^1< 1$ is impossible since in this case, $0.1+0.5\, p_S^2 + 
0.2\, p_J^2<1$, implying that
$$
p_S^2 =\left(0.1+p_S^2 + p_J^2\right)\wedge 2\,.
$$
For positive values of unknown variables, the last equality may hold only if  
$p_S^2=2$ but then, the 1st equation tells one that  $p_S^1=1$.

Thus, it was determined that $p_S^1=1$.

Combining the 2nd equation with~(\ref{pS3}), one obtains the equality
$$
p_S^2  =  \left(1+0.5\, p_S^2+0.8\, p_J^2\right)\wedge 2
$$
implying that $p_S^2=2$.

Available information allows one to reduce
the 5th equation to the simple one of the  form
$p_J^2  = 0.8\left(p_J^2\right)^+\wedge 2$ having the unique solution  $p_J^2=0$.

Summarizing, one gets that  $p_S=(1,2,1)$ and $p_J=(0,0,0)$.

\smallskip

\noindent
\textbf{Comment.} In the first example, the bank 1 has met all liabilities and 
finished with a~positive equity,  the bank~2 has payed the senior liabilities 
but defaulted on the junior debts, the bank~3 has defaulted already at the 
senior debts; and the 
bank~2 has no junior liabilities with the bank~1.  So, the $\Delta$-orbit of the 
banks~2 and~3 are not surplus sets and there are infinite many clearing vectors. 
In the second example, the bank~2 has a~junior debt to bank~1, 
all  $\Delta$-orbits are surplus sets, and the clearing vector is unique.


\section{Models with Illiquid Assets and~a~Price Impact}

\noindent
Let us consider the clearing problem without seniority structure where the bank~$i$ 
owns not only cash~$e^i$ but also~$K$~illiquid assets, in quantities 
$y^{i1},\dots y^{iK}$ represented in  the model by the row~$i$ of the matrix 
$Y=(y^{im})$, $i\le N$, $m\le K$. The nominal prices per unit  of illiquid 
assets are strictly positive  numbers $Q^1,\ldots,Q^K$.  The clearing might  
require their partial   sale  influencing   the market price. If the bank sells  
$u^{im}\in [0,y^{im}]$ units of the $m$th assets for the price~$q_m$, its 
total increase in cash is
$$
(Uq)^i=\sum\limits_{m=1}^K u^{im}q^m\,.
$$

\textbf{The price formation}  is modeled by the inverse demand function 
$F_0:\mathbb{R}^K\to \mathbb{R}^K$ assumed to be continuous and monotone 
decreasing ($F_0(z)\le F_0(x)$ when $z\ge x$ in the sense of partial ordering 
defined by~$\mathbb{R}^K_+$) and 
such that $F_0(0)=Q$ and $F^m_0(Y'{\bf 1})>0$ for $m=1,\ldots , K$.  The first 
condition means that in the absence
of supply, the prices are just the nominal prices while  the second one shows 
that even in the case of total sale, the prices of illiquid assets remain strictly 
positive.


\textbf{The clearing rules:} each bank pays  debts in accordance to the matrix of 
relative liabilities
and sells illiquid assets if it has insufficient amount of cash. The result of 
clearing should be: all
debts of the bank are covered or its equity falls down  to zero.



In the case of several illiquid assets,  there is a~problem how the banks chose 
their strategies of selling. In principle, one can imagine the situation that 
they have  full freedom and, acting in the noncooperative way, drop down the 
market of  illiquid assets because of an excessive supply. It seems reasonable 
that the central authority may  impose extra rules on selling illiquid assets. 
Let us suppose that this is done by prescribing that the bank~$i$ must sell all 
assets in the same proportion~$\alpha^{i}$:
\begin{equation*}
\alpha^i(q)=\fr{\left(\tilde L^i -e^i - \sum\nolimits_j\Pi^{ji}p^j \right)^+}
{ \sum\nolimits_k  y^{ik} q^k}\,\wedge 1\,,\enskip i=1,\dots, N\,.
\end{equation*}
This formula means that for a~fixed market price, the bank does not sell illiquid 
assets
if its  cash reserve together with collected debts covers the liabilities.
In the another extreme case where
$$
\tilde L^i -e^i - \sum\limits_j\Pi^{ji}p^j \ge \sum\limits_k y^{ik} q^k=(Yq)^i\,,
$$
all illiquid assets have to be sold and the bank defaults. In the intermediate
case, the bank sells a~share $\alpha^i\in ]0,1[$ of the $m$th asset adding to its 
cash an extra amount
$(({\tilde L^i -e^i - \sum\nolimits_j\Pi^{ji}p^j})/{\sum\nolimits_k y^{ik} 
q^k})\,y^{im}q_m$.
The total increase in cash allows to cover the liabilities.

Under such a~rule, the  $i$th bank sells~$u^{im}:=u^{im}(p,q)$ units of the $m$th asset where
\begin{equation*}
u^{im}
{}:=\fr{y^{im}\left(\tilde L^i -e^i - \sum\nolimits_j\Pi^{ji}p^j 
\right)^+}{ \sum\nolimits_k y^{ik} q^k}\,\wedge y^{im}.
\end{equation*}
The total supply of the illiquid assets is given by the vector ${\bf 1}'U(p,q)$ 
where
$U(p,q)$ is the matrix with entries given by the above formula.

Define the equilibrium vector 
$$
\left(p^*,q^*\right)\in \left[0,\tilde L\right] \times \left[ F_0(1Y),Q\right]
$$ 
as 
the solution of the system of $N+K$ equations written in the matrix form as
\begin{align}
\label{firstM}
p&=(e+U(p,q)q+\Pi'p)\wedge \tilde L\,;\\
\label{secondM}
q&=F_0(U'(p,q){\bf 1})\,.
\end{align}
The existence of the equilibrium is easy. Indeed,
check that 
\begin{gather*}
U'(p,q){\bf 1}\ge U'\left(\tilde p,\tilde q\right){\bf 1}\,;\\
U(p,q)q+\Pi'p\le  U\left(\tilde p,\tilde q\right)\tilde q+\Pi'\tilde p
\end{gather*}
when $(\tilde p,\tilde q)\ge (p,q)$. Denoting  $F(p,q)$ the right-hand side 
of the first equation, one obtains that  
$$
(p,q)\mapsto \left(F(p,q),F_0\left(U'(p,q)\right){\bf  1}\right)
$$ 
is a~monotone  mapping of the order interval $[0,\tilde L]\linebreak\times [ F_0(1Y),Q]$ into 
itself.  According to Knaster--Tarski theorem, the set of its fixed points is nonempty 
and contains the minimal and maximal elements $(\underline p^*, \underline q^*)$ 
and $(\bar p^*,\bar q^*)$.

For a~fixed $q$, the function $p\to F(p,q)$ is monotone. Thus, by the 
Knaster--Tarski theorem, the set of solutions of Eq.~(\ref{firstM}) is nonempty 
and contains, in particular, the maximal element~$\bar p(q)$.

For any fixed $q\in [F_0(Y),Q]$, the largest solution $\bar p=\bar p(q)$ 
of~(\ref{firstM}) is given by formula:
$$
\bar p=\sup\left\{p\in [0,\tilde L]:\ p\le \left(e+U(p,q)q+\Pi'p\right)\wedge \tilde L\right\}
$$
implying that $q\mapsto \bar p(q)$ is an increasing (and continuous) function on 
$[F_0(Y),Q]$.  It follows that the supply function
$$
q\mapsto \zeta(q):=U'(\bar p(q),q){\bf 1}
$$
is decreasing and, therefore, $q\mapsto F_0(\zeta(q))$ is an increasing 
(and continuous) mapping of the interval  $[F_0(Y),Q]$ into itself and, 
therefore, it has  the minimal and maximal fixed points that will be denoted by~$q_1$ 
and~$q_2$.

\smallskip

\noindent
\textbf{Lemma~3.}\
\textit{Suppose that the scalar function $x\to x'F_0(x)$ is strictly increasing on 
$[F_0(Y),Q]$. Then, the
solution of the equation  $q=F_0(\zeta(q))$ is unique, i.\,e.}, $q_1=q_2$.

\smallskip

\noindent
P\,r\,o\,o\,f\,.\ \
Arguing by contradiction, suppose that  $q_1\neq q_2$.     Since $q_1\le q_2$ 
and $\zeta(\cdot)$ is decreasing,   $\zeta(q_1)\ge \zeta(q_2)$. Moreover, 
$\zeta(q_1)\neq \zeta(q_2)$ as the values of~$F_0$ at these points are~$q_1$ 
and~$q_2$.
 The assumed strict monotonicity  implies that
 $$
 \zeta'(q_1)F_0( \zeta(q_1))> \zeta'(q_2)F_0( \zeta(q_2)).
 $$
It follows that
$$
\zeta'\left(q_1\right) q_1> \zeta'\left(q_2\right)q_2\,.
 $$
To get a~contradiction, it is sufficient to show that
$$
\Delta:= \zeta'\left(q_2\right)q_2-\zeta'\left(q_1\right)q_1\ge 0\,.
$$
Let $\bar p_k:=\bar p(q_k)$ and let
$$
D_k:=\left\{i:\ \left(\tilde L-e-\Pi'\bar p\left(q_k\right)\right)^i\ge 
\left(Yq_k\right)^i\right\}\,,
$$
i.\,e., $D_k$ is the set of banks that are forced to sell all their illiquid assets 
for the price~$q_k$, $k=1,2$. Since~$\bar p(\cdot)$ is increasing, $D_2\subseteq D_1$.  
With the 
notation~${\bf 1}'_{A}$ for the row-vector representing the indicator function
of the subset $A\subseteq \{1,\dots, N\}$, one has, taking into account that 
$a^+=a+a^-$:
\begin{multline*}
\zeta'\left(q_k\right)q_k={\bf 1}'_{D_k}Yq_k\\
{}+{\bf 1}'_{D_k^c}\left(\tilde L-e-\Pi'\bar 
p_k\right)+{\bf 1}'_{D_k^c}\left(\tilde L-e-\Pi'\bar p_k\right)^-.
\end{multline*}
This formula leads to the representation:
\begin{multline*}
\Delta={\bf 1}'_{D_2}Y(q_2-q_1)-{\bf 1}'_{D_1\setminus D_2}Yq_1\\
{}- {\bf 1}'_{D_1^c}
\Pi'\left(\bar p_2-  \bar p_1\right)+{\bf 1}'_{D_2^c\setminus D_1^c}
\left(\tilde L-e -\Pi'\bar p_2\right)\\
{}+ {\bf 1}'_{D_1^c}\left(\left(\tilde L-e -\Pi'\bar p_2\right)^- -
\left(\tilde L-e -\Pi'\bar p_1\right)^-\right)\\
+
{\bf 1}'_{D_2^c\setminus D_1^c}\left(\tilde L-e -\Pi'\bar p_2\right)^-.
\end{multline*}
Since the function $x\to x^-$ (on ${\mathbb{R}}^N$) is positive and decreasing, the 
last two terms in the right-hand side are positive. Regrouping  the third and 
the forth  terms, one gets that
\begin{multline}
\label{ineq1}
\Delta\ge{\bf 1}'_{D_2}Y\left(q_2-q_1\right)-{\bf 1}'_{D_1\setminus D_2}q_1Y
- {\bf 1}'_{D_2^c}\Pi'(\bar p_2-
 \bar p_1)\\
 {}+{\bf 1}'_{D_1\setminus D_2}\left(\tilde L-e -\Pi'\bar p_1\right)\,.
\end{multline}
From Eq.~(\ref{firstM}), it follows that
\begin{multline*}
{\bf 1}'\Pi'\left(\bar p_2-  \bar p_1\right)=
{\bf 1}'\left(\bar p_2-  \bar p_1\right)={\bf 1}'_{D_1}\left(\bar p_2-  \bar p_1\right)\\
{}={\bf 1}'_{D_2}\left(q_2u\left(\bar p_2,q_2\right)-q_1u
\left(\bar p_1,q_1\right)+\Pi'\left(\bar p_2-  \bar p_1\right)\right)\\
{}+{\bf 1}'_{D_1\setminus D_2}\left(\tilde L -\left(e+q_1u\left(\bar p_1,q_1
\right) +\Pi'\bar p_1\right)\right)
\end{multline*}
implying that

\columnbreak

\noindent
\begin{multline*}
 {\bf 1}'_{D_2^c}\Pi'\left(\bar p_2-
 \bar p_1\right)={\bf 1}'_{D_2}\left(U\left(\bar p_2,q_2\right)q_2\right.\\
 \left.{}-
 U\left(\bar p_1,q_1\right)q_1\right)-{\bf  1}'_{D_1\setminus D_2}
 U\left(\bar p_1,q_1\right)q_1\\
{}+{\bf 1}'_{D_1\setminus D_2}\left(\tilde L-e -\Pi'\bar p_1\right)\,.
\end{multline*}
Substituting this expression in~(\ref{ineq1}), one has:
\begin{multline*}
\Delta\ge{\bf 1}'_{D_2}Y\left(q_2-q_1\right)-{\bf 1}'_{D_1\setminus D_2}Yq_1\\
{}-{\bf 1}'_{D_2}\left(U\left(
\bar p_2,q_2\right)q_2-U\left(\bar p_1,q_1\right)q_1\right)\\
{}+
{\bf 1}'_{D_1\setminus D_2}q_1u\left(\bar p_1,q_1\right)=0
\end{multline*}
since the cash increment $(U(\bar p_2,q_2)q_2)^i=(Yq)^i$ for the bank $i\in D_2$ 
and $(U(\bar p_1,q_1)q_1)^i=(Yq_1)^i$ for $i\in D_1\supseteq D_2$.~$\square$


\smallskip

\noindent
\textbf{Theorem~3.}\
\textit{Suppose that the scalar function $x\to x'F_0(x)$ is strictly increasing on 
$[F_0(Y),Q]$. Then, there is $q^*$ such that  the set of solutions of the 
system}~(\ref{firstM}),  (\ref{secondM})
\textit{is contained in the interval  with the extremities $(\underline p(q^*),q^*)$ and 
$(\bar p(q^*),q^*)$.
In particular, if for each~$q$ the solution of}~(\ref{firstM}) \textit{is unique, then 
the solution of the system is also unique}.

\smallskip

\noindent
P\,r\,o\,o\,f\,.\ \ 
Let~$\Gamma$ be the set of~$q$ for which $(p,q)$ is a~solution  of 
the system~(\ref{firstM}),  (\ref{secondM}). If $q^*\in \Gamma$, then $(\bar 
p(q^*),q^*)$
is the solution of~(\ref{firstM}),  (\ref{secondM}). According to  
lemma~3, the point~$q^*$ is uniquely defined. This implies the result.~$\square$

\smallskip

Note that the uniqueness of the solution of~(\ref{firstM}) is guarantied if  for 
each~$i$,
the orbit of~$i$ contains an element with positive cash reserve.

\smallskip

\noindent
\textbf{Remark~2.}
In~\cite{AFM},  it was considered  a~model coinciding with studied 
above
in the case of a~single illiquid asset. The difference is that in the cited 
paper, the equilibrium is defined  as a~vector $(p,q)$ satisfying the
system of equations:
\begin{align}
\label{firstAFM}
p&=\left(e+qy+ \Pi'p\right)^+\wedge \tilde L\,; \\
%\label{secondAFM}
q&=F_0\left({\bf 1}'\left(\left(q^{-1}
\left(\tilde L-e-\Pi'p\right)^+\right)\wedge y\right)\right).\notag
\end{align}
To our opinion, the definition of the equilibrium given 
by the system~(\ref{firstM}), 
(\ref{secondM}), which is in the one liquid asset case has the  form:
\begin{align}
p&=\left(e+\left(\tilde L-e-\Pi'p\right)^+\wedge (qy)+ 
\Pi'p\right)\wedge \tilde L\,; \label{firstAFM1}
\\
%\label{secondAFM1}
q&=F_0\left({\bf 1}'\left(\left(q^{-1}\left(\tilde L-e-\Pi'p\right)^+
\right)\wedge y\right)\right), \notag
\end{align}
 is more natural.  In fact, the   right-hand sides of~(\ref{firstAFM}) 
 and~(\ref{firstAFM1}) as functions $R_1(p,q)$ and $R_2(p,q)$ defined
 on $[0,\tilde L]\times [ F_0(1Y),Q]$ coincide.  To see this, fix~$i$ and  
consider the three possible cases.
\begin{enumerate}[1.]
\item  Let  $e^i+qy+ (\Pi'p)^i\le \tilde L^i$. Then, the expressions for 
$R^i_1(p,q)$ and $R^i_2(p,q)$ have the same form  $e^i+qy+ (\Pi'p)^i$.

\item Let $e^i+qy+ (\Pi'p)^i> \tilde L^i$ and $\tilde L^i-e^i - (\Pi'p)^i\ge 0$. 
Then, the values $R^i_1(p,q)$ and $R^i_2(p,q)$ are equal to~$\tilde L^i$.

\item Let $e^i+qy+ (\Pi'p)^i> \tilde L^i$ and $\tilde L^i-e^i - (\Pi'p)^i<0$. 
Then, the value of $R^i_1(p,q)$ is $\tilde L^i$ and the value of $R^2_1(p,q)$ is 
$(e^i + (\Pi'p)^i)\wedge \tilde L^i=\tilde L^i$.
\end{enumerate}

\vspace*{-18pt}


{\small
\section*{\raggedleft Appendix}

%\vspace*{-6pt}

\subsection*{Knaster--Tarski Fixpoint Theorem}
%\label{app}

\noindent
Let $X$ be a~set with a~partial ordering~$\ge$ and let~$A$ be its nonempty 
subset.
By definition,~$\sup A$ is an element~$\bar x$ such that $\bar x\ge x$ for all 
$x\in A$ and if~$\bar x'$ is such that  $\bar x'\ge x$ for all $x\in A$, then 
$\bar x'\ge \bar x$. The definition of~$\inf A$ follows the same pattern but 
with the dual ordering~$\le$.  A~partially ordered set~$X$ is  {\it complete 
lattice} if for any its nonempty subset~$A$,
there exist~$\inf A$ and~$\sup A$.

\smallskip

\noindent
\textbf{Theorem~4.}\
\textit{Let $X$ be a~complete lattice and let $f : X \mapsto X$ be an order-preserving 
mapping, $L:=\{x:\  f(x)\le x\}$, $U:=\{x:\ f(x)\ge x\}$.   The set
$L\cap U$ of fixed points of~$f$
is nonempty and has the smallest and the largest fixed points  which are, 
respectively, $\underline x:=\inf L$ and}   $\bar x:=\sup U$.

\smallskip

 \noindent
P\,r\,o\,o\,f\,.\ \  
Note that $L\neq \emptyset$ since it contains the element~$\sup X$.
Take arbitrary $x\in L$. Then, $\underline x\le x$
implying that $f(\underline x)\le f(x) \le x$. Thus, $f(\underline x)\le 
\underline x$ as~$\underline x$ is~$\inf L$. So, $\underline x\in L$. 
Since $f(L)\subseteq L$, 
also $f(\underline x)\in L$; hence,  $\underline x\le f(\underline x)$, i.\,e., 
$\underline x= f(\underline x)$. All fixed points belong to~$L$ and, 
therefore,~$\underline x$ is the smallest one.

The proof of the statement for the largest fixed point is analogous.~$\square$

\smallskip

 \noindent
 \textbf{Corollary.}\
\textit{Let $f(\cdot;y)$ be an order-preserving mapping of a~complete lattice $(X,\ge)$ into 
itself, depending on the parameter~$y$ from a~partially ordered set 
$(Y,\succeq)$.
Suppose that $f(\cdot,y)$ is increasing in~$y$, that is, $f(x,y')\ge f(x,y)$ for all 
$x\in X$ when $y'\succeq y$.  Then, the smallest and the largest fixed points are 
also increasing in}~$y$.

\smallskip

\noindent
P\,r\,o\,o\,f\,.\ \ The claim is obvious because the sets   
$$
L(y):=\{x:\  f(x,y)\le x\}
$$ 
are decreasing and the sets 
$$
U(y):=\{x:\ f(x,y)\ge x\}$$ 
are increasing in~$y$
(see~\cite{Milgrom-Roberts}).

These general results are applied to the order intervals $[a,b]\subset \mathbb{R}^d$
with the ordering induced by~$\mathbb{R}^d_+$.

}

\vspace*{-6pt}

\Ack
\noindent
The 
research of Yuri Kabanov was done under partial financial support   of the grant 
of  the Russian Science Foundation No.\,14-49-00079.


\renewcommand{\bibname}{\protect\rmfamily References}

\vspace*{-6pt}

{\small\frenchspacing
{%\baselineskip=10.8pt
\begin{thebibliography}{9}

\bibitem{Eisenberg-Noe} %1
\Aue{Eisenberg, L., and T.\,H.~Noe}. 2001. Systemic risk in financial systems. 
\textit{Manag. Sci.} 47(2):236--249.

\bibitem{Suzuki} %2
\Aue{Suzuki, T.} 2002. Valuing corporate debt: The effect of cross-holdings of stock 
and debt. \textit{J.~Oper. Res. Soc. Japan} 45(2):123--144.

\bibitem{Tarski} %3
\Aue{Tarski, A.} 1955. A~lattice-theoretical fixpoint theorem and its applications. 
\textit{Pacific J.~Math.} 5(2):285--309.


\bibitem{Cont-Wag} %4
\Aue{Cont, R., and L.~Wagalath}. 2015. Fire sale forensics: Measuring endogenous risk. 
\textit{Math. Finance} 26:835--866. %doi: 10.1111/mafi.12071.

\bibitem{AFM} %5
\Aue{Amini, H., D.~Filipovi$\acute{\mbox{c}}$, and A.~Minca.} 2015. To fully net or not to net: Adverse 
effects of partial multilateral netting. %Swiss Finance Institute Research Paper  series. No.~14-63. Forthcoming in ``
\textit{Oper. Res.} 64(5):1135--1142.

\bibitem{Elsinger2011} %6
\Aue{Elsinger, H.} 2009. Financial networks, cross holdings, and limited liability. 
Working paper from Oesterreichische Nationalbank.

\bibitem{Milgrom-Roberts} %7
\Aue{Milgrom, J., and J.~Roberts.} 1994. Comparing equilibria. 
\textit{Am. Econ. Rev.}  84:441--454.




\end{thebibliography} } }

\end{multicols}

\vspace*{-6pt}

\hfill{\small\textit{Received September 25, 2016}}

\vspace*{-18pt}

\Contr

%\vspace*{-3pt}

\noindent
\textbf{El Bitar  Khalil} (b.\ 1981)~--- 
PhD student, Laboratoire de Mathematiques, Universite de Franche-Comte, 
16~Route de Gray, 25030, \mbox{Besan{\!\ptb{\c{c}}}on}, CEDEX, France; 
\mbox{khalilbitar\_aw@hotmail.com}  

 \vspace*{1pt}
 
 \noindent
 \textbf{Kabanov Yuri M.} (b.\ 1948)~---
  professor, Laboratoire de Mathematiques, Universite de Franche-Comte, 
  16~Route de Gray, 25030, Besancon, CEDEX, France; leading scientist, 
  Institute of Informatics Problems, Federal Research Center 
  ``Computer Science and Control'' of the Russian Academy of Sciences,  
  44-2~Vavilov Str., Moscow 119333, Russian Federation; 
  National Research University ``MPEI,'' 14~Krasnokazarmennaya Str., 
  Moscow 111250, Russian Federation; \mbox{Youri.Kabanov@univ-fcomte.fr} 

\vspace*{1pt}
 
 \noindent
 \textbf{Mokbel Rita} (b.\ 1981)~--- 
 PhD student, Laboratoire de Mathematiques, Universite de Franche-Comte, 
 16~Route de Gray, 25030, Besancon, CEDEX, France; \mbox{ritamokbel@hotmail.com}




%\vspace*{8pt}

%\hrule

%\vspace*{2pt}

%\hrule

\newpage

\vspace*{-24pt}



\def\tit{О~ЕДИНСТВЕННОСТИ КЛИРИНГОВЫХ ВЕКТОРОВ, РЕДУЦИРУЮЩИХ 
СИСТЕМНЫЙ РИСК$^*$}

\def\aut{Х.~Эль Битар$^1$, Ю.~Кабанов$^{1,2,3}$, Р.~Мокбель$^1$}


\def\titkol{О~единственности клиринговых векторов, редуцирующих 
системный риск}

\def\autkol{Х.~Эль Битар, Ю.~Кабанов, Р.~Мокбель}

{\renewcommand{\thefootnote}{\fnsymbol{footnote}}
\footnotetext[1]{Представленные в настоящей статье результаты исследований, проведенных 
Ю.\,М.~Кабановым, были получены при частичной финансовой поддержке 
Российского научного фонда (проект №\,14-49-00079).}}


\titel{\tit}{\aut}{\autkol}{\titkol}

\vspace*{-12pt}

\noindent
$^1$Лаборатория математики Университета Франш-Кон\-те, г.~Безансон, Франция

\noindent
$^2$Институт проблем информатики Федерального исследовательского
центра <<Информатика и~управление>>\linebreak
$\hphantom{^1}$Российской академии наук, Российский
университет дружбы народов

\noindent
$^3$Национальный исследовательский университет <<МЭИ>>

\vspace*{6pt}

\def\leftfootline{\small{\textbf{\thepage}
\hfill ИНФОРМАТИКА И ЕЁ ПРИМЕНЕНИЯ\ \ \ том\ 11\ \ \ выпуск\ 1\ \ \ 2017}
}%
 \def\rightfootline{\small{ИНФОРМАТИКА И ЕЁ ПРИМЕНЕНИЯ\ \ \ том\ 11\ \ \ выпуск\ 1\ \ \ 2017
\hfill \textbf{\thepage}}}


\Abst{В~финансовых системах, т.\,е.\ в сети взаимосвязанных банков, 
процедура взаимозачета, или клиринга, состоит в~одновременной выплате 
задолженностей с~целью уменьшения общей их суммы в~системе. Вектор, компоненты 
которого есть суммарные выплаты каждого банка системы, называется клиринговым 
вектором. В~простых моделях, предложенных Айзенбергом и Ноэ (2001) и~независимо 
Судзуки (2002) было показано, что полный клиринг описывается вектором, который 
является неподвижной точкой некоторого отображения. Существование клирингового 
вектора может быть получено прямой ссылкой на теоремы о~неподвижной точке 
Кнас\-те\-ра--Тар\-скo\-го или Брауэра. Вопрос о~его единственности является более 
деликатным. Айзенберг и Ноэ получили достаточное условие единственности 
в~терминах графа связей финансовой системы. В~настоящей работе доказывается 
единственность для двух более общих моделей: модели Эльсингера с~приоритетами 
долгов и~модели типа Ами\-ни--Фи\-ли\-по\-ви\-ча--Мин\-ки, 
в~которой банки имеют неликвидные 
активы, продажа которых влияет на их рыночную цену.}

\KW{системный риск; финансовые сети; клиринг; теорема 
Кнас\-те\-ра--Тар\-ско\-го; модель Ай\-зен\-бер\-га--Ноэ; приоритет финансовых обязательств; 
влияние на ценообразование}



\DOI{10.14357/19922264170110}

%\vspace*{6pt}


 \begin{multicols}{2}

\renewcommand{\bibname}{\protect\rmfamily Литература}
%\renewcommand{\bibname}{\large\protect\rm References}

{\small\frenchspacing
{%\baselineskip=10.8pt
\begin{thebibliography}{9}
\bibitem{3-kab} %1
\Au{Eisenberg L., Noe~T.\,H.} Systemic risk in financial systems~// 
Manag. Sci., 2001. Vol.~47. No.\,2. P.~236--249.
\bibitem{6-kab} %2
\Au{Suzuki T.} Valuing corporate debt: The effect of cross-holdings of stock and debt~// 
J.~Oper. Res. Soc. Japan, 2002. Vol.~45. No.\,2. P.~123--144.
\bibitem{7-kab} %3
\Au{Tarski A.} A~lattice-theoretical fixpoint theorem and its applications~// 
Pacific J.~Math., 1955. Vol.~5. No.\,2. P.~285--309.

\bibitem{2-kab} %4
\Au{Cont R., Wagalath~L.} Fire sale forensics: Measuring endogenous risk~// 
Math.  Finance, 2015. Vol.~26. P.~835--866. %doi: 10.1111/mafi.12071.
\bibitem{1-kab} %5
\Au{Amini H., Filipovi$\acute{\mbox{c}}$~D., Minca~A.} To fully net or not to net: 
Adverse effects of partial multilateral netting~// Oper. Res., 2015. Vol.~62.
No.\,5. P.~1135--1142.

\bibitem{4-kab} %6
\Au{Elsinger H.} Financial networks, cross holdings, and limited liability. 
Working paper from Oesterreichische Nationalbank, 2009.
\bibitem{5-kab} %7
\Au{Milgrom J., Roberts~J.} Comparing equilibria~// Am. Econ. Rev., 1994. 
Vol.~84. P.~441--454.


\end{thebibliography}
} }

\end{multicols}

 \label{end\stat}

 \vspace*{-3pt}

\hfill{\small\textit{Поступила в редакцию  25.09.2016}}
%\renewcommand{\bibname}{\protect\rm Литература}
\renewcommand{\figurename}{\protect\bf Рис.}
\renewcommand{\tablename}{\protect\bf Таблица}  %
%\newcommand{\Var}{\ensuremath{{\rm\mathbb{V}ar}}}

\renewcommand{\figurename}{\protect\bf Figure}
\renewcommand{\tablename}{\protect\bf Table}

\def\stat{lukashenko}


\def\tit{A~GAUSSIAN APPROXIMATION OF THE DISTRIBUTED COMPUTING PROCESS}

\def\titkol{A~Gaussian approximation of the distributed computing process}

\def\autkol{O.\,V.~Lukashenko, E.\,V.~Morozov,  and~M.~Pagano}

\def\aut{O.\,V.~Lukashenko$^{1}$, E.\,V.~Morozov$^2$,  and~M.~Pagano$^{3}$}

\titel{\tit}{\aut}{\autkol}{\titkol}

%{\renewcommand{\thefootnote}{\fnsymbol{footnote}}
%\footnotetext[1] {The study was carried out under state order to the Karelian Research 
%Centre of the Russian Academy of Sciences (Institute of Applied Mathematical 
%Research KarRC RAS) and supported by the Russian Foundation for Basic Research, 
%projects 18-07-00187, 18-07-00147, 18-07-00156, 19-07-00303.}}

\renewcommand{\thefootnote}{\arabic{footnote}}
\footnotetext[1]{Institute of  Applied Mathematical Research of Karelian Research Centre of RAS, 
11~Pushkinskaya Str.,  Petrozavodsk 185910, Republic of Karelia, Russian Federation; 
Petrozavodsk State University, 33~Lenin Str., Petrozavodsk 185910, Republic of Karelia, 
Russian Federation,  \mbox{lukashenko@krc.karelia.ru}}
\footnotetext[2]{Institute of  Applied Mathematical Research of Karelian Research Centre of RAS, 
11~Pushkinskaya Str.,  Petrozavodsk 185910, Republic of Karelia, Russian Federation; 
Petrozavodsk State University, 33~Lenin Str., Petrozavodsk 185910, Republic of Karelia, 
Russian Federation, \mbox{emorozov@karelia.ru}}
\footnotetext[3]{University of Pisa, 43~Lungarno Pacinotti, Pisa 56126, Italy, \mbox{m.pagano@iet.unipi.it}}


\index{Lukashenko O.\,V.}
\index{Morozov E.\,V.}
\index{Pagano M.}
\index{Лукашенко О.\,В.}
\index{Морозов Е.\,В.}
\index{Пагано М.}


\def\leftfootline{\small{\textbf{\thepage}
\hfill INFORMATIKA I EE PRIMENENIYA~--- INFORMATICS AND
APPLICATIONS\ \ \ 2019\ \ \ volume~13\ \ \ issue\ 2}
}%
 \def\rightfootline{\small{INFORMATIKA I EE PRIMENENIYA~---
INFORMATICS AND APPLICATIONS\ \ \ 2019\ \ \ volume~13\ \ \ issue\ 2
\hfill \textbf{\thepage}}}

%\vspace*{-2pt}



 
\Abste{The authors propose a~refinement of the stochastic model 
describing the dynamics of the Desktop Grid (DG) project with many hosts and many
 workunits to be performed, originally proposed  by Morozov \textit{et al.}\ in 2017.
The target performance measure  is the mean  duration of the runtime of the project. 
To this end,  the authors derive  an asymptotic expression for the  amount 
of the accumulated work to be done by means of 
    limit theorems for  superposed on-off sources that   lead to a~Gaussian 
    approximation. In more detail, depending on the distribution of active 
    and idle periods, Brownian  or fractional Brownian processes are obtained.
    The authors present the analytic results related to the hitting time of 
    the considered processes (including the case in which the overall amount of
work is only known in a~probabilistic way), and highlight how the 
     runtime tail distribution could be estimated by simulation. Taking 
     advantage of the properties of Gaussian processes and the Conditional 
     Monte-Carlo (CMC) approach, the authors present a~theoretical framework for 
     evaluating the runtime tail distribution.}


\KWE{Gaussian approximation; distributed computing; fractional Brownian motion}

 \DOI{10.14357/19922264190215}


%\vspace*{8pt}


\vskip 12pt plus 9pt minus 6pt

 \thispagestyle{myheadings}

 \begin{multicols}{2}

 \label{st\stat}



\section{Introduction}

\vspace*{-4pt}

\noindent
Gaussian processes are widely used in the performance analysis of telecommunication 
systems for their analytic tractability and arguments based on the central-limit 
theorem that make them suitable in case of a~large number of independent contributions.  For instance, these  models are able to capture, in a simple and parsimonious way,
the properties of self-similarity and long-range dependence, inherent to multimedia
network traffic~\cite{2-luk-1, 3-luk-1}. 
These properties dramatically increase the difficulty of the probabilistic 
analysis and, as a consequence, in many cases only Monte-Carlo simulation can be used.
The \textit{fractional Brownian motion} (FBM) is one of the most studied 
self-similar long-range dependent Gaussian processes due to its simplicity. 
Its use as traffic model is supported  by the following theoretical analysis~\cite{4-luk-1}: 
the sum of an increasing  number of the so-called on-off inputs,
with either on-times or off-times having a~heavy-tailed distribution
with infinite variance, converges weakly to an~FBM, after an
appropriate time  scaling.


In this paper,  the applicability of FBM for high-performance computing is considered. 
In that framework, computing clusters and computational Grid systems are the 
main tools: computing clusters are based on computing nodes connected by 
a~high-speed network, while computational Grid systems include geographically 
dispersed computing nodes connected by a relatively slow network. 

Desktop Grid belongs to the latter class. The DG combines nondedicated 
\textit{hosts} (typically, desktops/laptops owned by volunteers) over the Internet 
to process loosely coupled \textit{workunits} (computational tasks). 
Desktop Grids utilize the idle host resources, providing  potentially huge, although highly 
variable, computing power. (For example, the DG project Einstein\@HOME aggregates 
peak performance at about~1~PetaFLOPS~[4].) Typically,  DGs are 
managed by a scientific community that utilizes the resources to complete 
a~\textit{DG project} which consists of a~(usually finite) number of workunits.
 Thus, the \textit{runtime} of the DG project  is the time to complete all the 
 workunits and it is desirable to minimize it.

Minimization of the DG project runtime may be performed by means of 
scheduling optimization~\cite{6-luk-1, 7-luk-1, 8-luk-1}. Additional information on the hosts, 
such as reliability and availability, can be used to improve the efficiency of
 DGs~\cite{9-luk-1, 10-luk-1}, In~\cite{11-luk-1, 12-luk-1}, the focus is placed on the so-called 
 workunit replication mechanism for reliability purposes. However,  to the 
 best of our knowledge, the estimation of the runtime of a~DG project remains
  generally an unsolved issue, and it is the main motivation of this paper.

Desktop Grids have several important distinctive  features 
when compared to computational Grids or computing clusters. First of all, 
hosts, being nondedicated, possess individual availability periods. Moreover, 
the management server of a~DG is not able to obtain information on the current 
state of the hosts (such as ``computing,'' ``suspended,'' etc.). 
These two issues  make the estimation of the runtime of a~DG project a~hard problem. 

The execution of a~DG project can be divided into two stages. 
During the first phase, the number of work\-units is greater than the number of 
hosts and, thus,  each host will receive at least one workunit. 
In the second stage, all the available workunits are dispatched and
 there are available (idle) hosts. In this paper, the focus is on the duration of  
 the first phase  which is studied by means of a Gaussian approximation of the 
 overall work.  The   study of the second stage requires a completely different 
 probabilistic technique, which relies on  the theory of order statistics and the 
 asymptotic properties of renewal processes, and is postponed for a future work.
Thus, in what follows, runtime will relate to the first stage of the project solely.


We describe the availability patterns of the hosts by treating each of them as an 
individual on-off source which processes workunits during on periods.  
Our approach  is based on the asymptotics of the (properly scaled) superposition 
of a large number of independent on-off sources. It is well-known~\cite{4-luk-1} that 
after an appropriate scaling, the limiting process describing the summary workload 
in the system turns out to be \textit{Brownian motion} (BM), when the sojourn times 
are light-tailed, while it becomes \textit{fractional Brownian motion} 
in case of heavy-tailed sojourn times.  
Then, the problem reduces to the calculation of the hitting time of the given
 threshold~$D$ by the process of accumulated work which is a~well-known topic 
 in probability theory.

The paper is organized as follows. Section 2 presents the theoretical background 
related to  FBM, including functional limit theorems for the cumulative  work 
performed  by an increasing  number of on-off sources.    
Then, Section~3 describes the model and summarizes the available analytic results, 
while Section~4 is devoted to the evaluation of the runtime tail distribution 
by means of the CMC method which potentially leads to variance 
reduction of the estimate of the runtime. Finally, in Section~5, the main 
contributions of the paper are presented and some future research issues are discussed. 

\section{Theoretical Background}

\noindent
In this section, let us recall the basic definitions about FBM and how it is 
related to the limiting theorems for the superposition of independent 
\textit{on-off sources}.

\subsection{Fractional Brownian motion}

\noindent
The FBM $\left\{B_H(t), t \in \mathbb{R} \right\}$ is a~Gaussian centered process 
with $B_H(0)=0$, stationary increments, and the following covariance function:
  \begin{multline*}
       K_H(t,s) := \mathbb{E} \left[ B_H(t) \: B_H(s) \right]\\
       {}=
       \fr{1}{2} \left[
       |t|^{2H} - |t-s|^{2H} + |s|^{2H} \right],\enskip s,\,t\ge 0 
     \end{multline*}
where $H \in (0,1)$ is the so-called \textit{Hurst parameter}. 
It is easy to verify that~$B_H(t)$  is a self-similar process with 
self-similarity parameter~$H$, i.\,e., for each $c>0$,
$$
  c^{-H}B_H(ct) \stackrel{d}{=} B_H(t)
$$
where  $\stackrel{d}{=}$  denotes equality in distribution.

Fractional Brownian motion is widely used for modeling purposes due to its 
Gaussianity (that typically arises under aggregation conditions) 
and parsimonious description (apart from mean and variance, its behavior 
is unambiguously determined by~$H$).

When $H>1/2$,  FBM is a long-range dependent process since the
 autocorrelation of the corresponding increment process is nonsummable. 
 For more details on FBM and its properties, see~\cite{13-luk-1}.


\subsection{Limit theorems for distributed computing processes}



\noindent
Let us assume that the DG consists of~$N$ heterogeneous hosts which can be 
considered as  independent \textit{on-off sources}. In more detail, let us suppose 
that there are~$n$~types of hosts ($n<N$) and denote by~$N_i$ the number of  
$i$-type hosts, i.\,e., $\sum\nolimits_{i=1}^n N_i=N$. 
Moreover, let~$R_i$ denote the amount of processed work per unit time for $i$-type 
hosts and let $\left\{I^{(i)}(t),\ t\geq0\right\}$,
\begin{equation*}
I^{(i)}(t)=\begin{cases}
 R_i\,, &t\in \mbox{on-period}\,; \\
 0\,, & t\in \mbox{off-period}\,, 
\end{cases}
 %\label{taq1}
\end{equation*}
be the  \textit{on-off} process that characterizes the activity/silent 
periods of the corresponding hosts (Fig.~1).
For sake  of simplicity, it is assumed that for each host, both \textit{on} 
and  \textit{off} periods are sequences of i.i.d.\ (independent
and identically distributed)
random variables (RVs) and mutually independent.
Moreover, as already stated, the\linebreak\vspace*{-12pt}

{ \begin{center}  %fig1
 \vspace*{9pt}
   \mbox{%
 \epsfxsize=79mm 
 \epsfbox{luk-1.eps}
 }


\vspace*{6pt}


\noindent
{{\figurename~1}\ \ \small{On-off model}}
\end{center}
}

%\vspace*{9pt}

\addtocounter{figure}{1}


\noindent
on-off processes modeling the contribution of 
different hosts are assumed to be independent.



The \textit{cumulative processed work}, i.\,e., the aggregated amount of  
work provided by all $N$ hosts, during the time interval $[0,t]$ is given by
$$
A(t)=\int\limits_0^{t} \left( {\sum\limits_{i=1}^n\sum\limits_{k=1}^{N_i} 
{I_{k}^{(i)}(u)} } \right)\,du
$$
where $I_k^{(i)}$ are the independent copies of~$I^{(i)}$, $i=1,\ldots\linebreak \ldots,n$. 
Moreover, for the $i$-type ($i=1,\dots,n$) hosts, let us denote by
$\mu_{\mathrm{on}}^i$, $\sigma_{\mathrm{on}}^i$, $\mu_{\mathrm{off}}^i$, 
and~$\sigma_{\mathrm{off}}^i$
the mean length and  standard deviation (that may be infinite) of 
the duration of the  on and  off periods, respectively.


The statistical behavior of~$A(t)$ is determined by the distributions~$F_{\mathrm{on}}^i$ 
and~$F_{\mathrm{off}}^i$ of the on and off periods for each type of hosts, namely,
 by their tail. 
In more detail, in case of infinite variance, let us assume that as $x \to \infty$,
\begin{align*}
1-F_{\mathrm{on}}^i(x)& \sim  \ell_{\mathrm{on}}^i x^{-\alpha_{\mathrm{on}}^i}L_{\mathrm{on}}^i(x)\,;\\
1-F^i_{\mathrm{off}}(x)& \sim  \ell_{\mathrm{off}}^i x^{-\alpha_{\mathrm{off}}^i}L_{\mathrm{off}}^i(x)
%  \label{3}
\end{align*}
where   $a \sim b$  means that $a/b\to 1$;    
$\ell_{\mathrm{on}}^i$ and~$\ell_{\mathrm{off}}^i$ are the positive constants; the 
exponents~$\alpha_{\mathrm{on}}^i$ and~$\alpha_{\mathrm{off}}^i\in (1,\,2)$; 
and the functions~$L_{\mathrm{on}}^i$ 
and~$L_{\mathrm{\mathrm{off}}}^i$ are slowly varying at infinity, i.\,e., for any $t >0$,
$$
\lim_{x \to \infty} \fr{L^i(tx)}{L^i(x)}=1\,,\enskip i=1,\ldots,n\,.
$$
Instead, if $\sigma_{\mathrm{on}}^i$ and~$\sigma_{\mathrm{off}}^i <\infty$, we  set 
$\alpha_{\mathrm{on}}^i=\alpha_{\mathrm{off}}^i=2$. 

It has been  shown in~\cite{4-luk-1} that the scaled process of cumulative  
work arrived during interval  $[0,\,Tt]$ converges weakly to a sum of the i.i.d.\
 FBM's,  provided that 
\begin{enumerate}[(1)]
    \item $N_i\to \infty$ such that
$\lim\nolimits_{N\to \infty}N_i/N>0$, $i=1,\ldots\linebreak \ldots , n$; and
\item  the scaling factor $T\to \infty$.
\end{enumerate}

This \textit{functional limit theorem} leads to the following approximation:
\begin{multline*}
A(tT)\approx T\left( \sum\limits_{i=1}^n R_i N_i
\fr{\mu_{\mathrm{on}}^i}{\mu_{\mathrm{on}}^i+\mu_{\mathrm{off}}^i} \right)t \\
{}+ \sum\limits_{i=1}^n
T^{H_i}R_i \sqrt{L_i(T)N_i}c_i B_{H_i}(t) 
%\label{approx1}    
\end{multline*}
where $c_i$ are the positive constants; $L_i$ are the slowly varying at
infinity functions (expressed in terms of the given  parameters); and
$B_{H_i}$ are the independent FBMs with the Hurst parameters~$H_i$ given by
$$
H_i=\fr{3-\min(\alpha_{\mathrm{on}}^i,\,\alpha_{\mathrm{off}}^i)}{2}\in
\left(\fr{1}{2},\,1 \right),\enskip i=1,\ldots, n\,.
$$
Thus, the cumulative work processed by a~large number of independent hosts 
(with heavy-tailed distributions of the on-off periods) is approximated by 
a~superposition of independent FBMs  $\{B_{H_i}(t)\}$, 
$i=1,\dots,n$,  with a~linear drift that depends on the rates~$R_i$ and 
the average duty cycle.

Instead, if for all types of hosts the variances of the sojourn times 
are finite (i.\,e., $\sigma_{\mathrm{on}}^i,\,\sigma_{\mathrm{off}}^i<\infty$ 
$\forall i=1,\dots,n$), then the limiting (scaled) process becomes
\begin{equation*}
T\left( \sum\limits_{i=1}^n \fr{R_i N_i \mu_{\mathrm{on}}^i}{\mu_{\mathrm{on}}^i+\mu_{\mathrm{off}}^i}
\right)t + \left(\sqrt{T}\sum\limits_{i=1}^n R_i \sqrt{N_i}c_i \right)W(t)
%\label{bm-l1}
\end{equation*}
where $W(t)$ is the Wiener process, and the constants~$c_i$ are given by
$$
c_i = \sqrt{ \fr{ (\mu_{\mathrm{off}}^i\sigma_{\mathrm{on}}^i)^2 + 
\left(\mu_{\mathrm{on}}^i\sigma_{\mathrm{off}}^i\right)^2 }{\left(\mu_{\mathrm{on}}^i+\mu_{\mathrm{off}}^i\right)^3}}\,.
$$
Finally, it is worth mentioning that taking the limits in reverse order, 
the (scaled) process of cumulative work converges to a~\textit{Levy stable motion}, 
an infinite variance process with stationary and independent increments~\cite{14-luk-1}; 
however, such  model is beyond the scope of this paper as in DG, the experimental 
data confirmed the convergence to processes with finite variance.


\section{Model Description and~Performance Measures}


\noindent
The above functional limit theorems provide a~theoretical motivation to 
consider the following model for the cumulative processed work: 
\begin{equation}
    A(t) = m t + X(t)
    \label{6}
\end{equation}
where $X$ is the centered Gaussian process with stationary increments 
(FBM or the sum of independent FBM, in case of heterogeneous systems),
 which describes random fluctuations around the linearly  increasing mean. 
 Such type of stochastic process was previously suggested as the model of 
 network traffic (see~\cite{15-luk-1} for more details). 


Let us denote  by~$\tau_D$ the  runtime of the  DG project where~$D$ 
denotes the required amount of work. Thus,~$\tau_D$ represents the  
\textit{hitting time} of  the process~$\{A(t)\}$:  
\begin{equation*}
\tau_D = \min\{t:\,\, A(t)\ge D \} \, ,
\end{equation*}
i.\,e., the first time the process~$\{A(t)\}$ hits the threshold~$D$.
Then, the original problem is reduced to the calculation (or estimation)
 of some useful performance characteristics, such as the mean  hitting time.

\subsection{Available analytic results}

\noindent
Let us recall the available analytic results for different types 
of Gaussian processes, corresponding to the different limiting cases. 

\vspace*{-4pt}

\subsubsection{Wiener case}

\noindent
When $X$ is a Wiener process (i.\,e., $X = \sigma B_{1/2}$), 
the  density of~$\tau_D$ is available in  explicit form~\cite{16-luk-1}:
\begin{multline}
\mathbb{P}(\tau_D \in dt)= \fr{D}{\sqrt{2\pi}\sigma t^{3/2}} 
\exp \left( -\fr{(D-mt)^2}{2\sigma^2 t} \right)\,dt\\
{=:} f_\tau(t|D)\,dt\,.
\label{8}
\end{multline}
In this case, the corresponding expected value $\mathbb{E} [\tau_D ]$ 
is  the ratio between the given  amount of the work~$D$ and the mean processing 
rate~\cite{16-luk-1}:
$$
\mathbb{E} \left[\tau_D\right] = \fr{D}{m}\,.
$$

\vspace*{-12pt}


\subsubsection{Fractional Brownian motion case}



\noindent
When the limiting process is an FBM,  only asymptotic results and some bounds
 for the  distribution of~$\tau_D$ are available.

In~\cite{17-luk-1}, the following bounds (quite inaccurate when~$H$ is close to~1, 
see Fig.~2)
 for the moments of the hitting time were obtained for $1/2 \le H < 1$:
\begin{multline*}
\fr{1}{\sqrt{2\pi}} \left( \fr{2 H D}{n-H}\, L_n (D,H,m)\right.\\ - 
\left.\fr{(2H-1)m}{n+1-H} \,L_{n+1}(D,H,m) \right) \le \mathbb{E} \left[\tau_D^n\right]\\
   \le \fr{1}{\sqrt{2\pi}} \left( \fr{ H D}{n-H} \,L_n (D,H,m)\right.\\
   \left.{} + 
   \fr{(1-H)m}{n+1-H}\, L_{n+1}(D,H,m) \right)
\end{multline*}

{ \begin{center}  %fig2
% \vspace*{9pt}
   \mbox{%
 \epsfxsize=79mm 
 \epsfbox{luk-2.eps}
 }


\end{center}


\noindent
{{\figurename~2}\ \ \small{Bounds for the mean hitting time ($D=10$, $m=3$): 
\textit{1}~--- $D/m$;
\textit{2}~--- lower bound;
and \textit{3}~--- upper bound}}
}

%\vspace*{9pt}

\addtocounter{figure}{1}


\noindent
where
\begin{multline*}
L_n(D,H,m) \\
{}=\!\! \int\limits_0^\infty \!\exp\left\{ -\fr{1}{2} \!
\left(\!Dt^{-H/(n-H)}-mt^{(1-H)/(n-H)} \right)^2 \!\right\}dt.\hspace*{-3.80858pt}
\end{multline*}
Additionally, the following asymptotic was derived for the large values of level~$D$:
\begin{equation*}
    \lim\limits_{D \to \infty} \fr{\mathbb{E} \left[\tau_D^n\right]}{D^n} = m^{-n}
\end{equation*}
for all $n \ge 1$, $m>0$, from which it is quite straightforward to show that 
for all $n \ge 1$, 
\begin{equation*}
     \fr{\tau_D}{D} \overset{L_n}{\longrightarrow} \fr{1}{m}\enskip 
     \mbox{as} \enskip D \to \infty
\end{equation*}
where $\overset{L_n}{\longrightarrow}$ means convergence in~$L_n$~space.

\subsubsection{General case}

\noindent
In the general case,  to derive asymptotic (for large values of~$D$) 
for the distribution of~$\tau_D$,  it is possible to take advantage of the 
following identity: 
\begin{equation*}
\mathbb{P}\left(\tau_D \le  T\right) = 
\mathbb{P}\left(\sup\limits_{t \in [0,T]}A(t)\ge D\right)\,.
\end{equation*}

The distribution of the maximum of Gaussian processes over a~finite interval is 
a~well-studied  problem. In more detail, for any Gaussian process with stationary 
increments and strictly monotonically increasing and
convex variance such that $\lim\nolimits_{t\to 0} \mathrm{Var}(X(t))/t=0$, the following asymptotic 
holds~\cite{18-luk-1}:
\begin{multline*}
\mathbb{P}\left(\sup\limits_{t \in [0,T]}A(t)\ge D\right) \sim 
\Phi \left( \fr{D-mT}{\sqrt{\mathrm{Var}(X(T))}} \right)\\
 \mbox{ as } D \to \infty
\end{multline*}
where~$\Phi$ denotes the tail distribution of the standard normal RV~$N(0,1)$.

\subsection{A~possible  generalization} 



 \noindent
 It seems quite natural to consider the setting in which the threshold~$D$  
 is an~RV which is independent of the process~$X$ in~(\ref{6}). 
 Such a~setting seems to be highly motivated by practice because it is more 
 realistic that the exact value of   the quantity~$D$ is not available, 
 and it is known in part. This incomplete information can be reflected by  
 introducing the probability density
 function (PDF)~$f_D$ of~$D$, which is assumed to be predefined.   
 Provided that~$X$ in~(\ref{6}) is  a~Wiener process and, hence, the conditional 
 density $f_\tau(t|D)$ in~(\ref{8}) is known, one can write  the density of the RV~$\tau_D$ as
$$
f_\tau(x)=\int\limits_{y=0}^\infty 
f_\tau(x|y)f_D(y)\,dy\,.
$$
In general, one
can calculate this density only by numerical methods 
but for some cases, it is possible to derive its expression in terms of special
 functions. For example, when~$D$  is  exponential with parameter~$\lambda$, 
 one can obtain the following expression:
\begin{multline}
f_\tau(x) = \fr{\lambda}{\sqrt{2\pi}\sigma x^{3/2}}\exp
\left( -\fr{m^2 x}{2\sigma^2} +\fr{\gamma^2}{8\beta(x)}\right)\\
 \times
(2\beta(x))^{-1/2} D_{-1}\left( \fr{\gamma}{\sqrt{2\beta(x)}} \right) 
\label{dens}
\end{multline}
where 
$$
\gamma=\lambda-\fr{m}{\sigma^2}\,;\qquad \beta(x) = \fr{1}{2\sigma^2 x}\,;
$$
and $D_p$, $\mathrm{Re}\, p <0$, is the parabolic cylinder function~\cite{19-luk-1}. 
Numeric calculation of the expression~\eqref{dens} is shown in 
Fig.~3.

{ \begin{center}  %fig3
 \vspace*{6pt}
  \mbox{%
 \epsfxsize=78.984mm 
 \epsfbox{luk-3.eps}
 }


\end{center}


\noindent
{{\figurename~3}\ \ \small{Probability density function of~$\tau_D$ for different values of~$m$ 
($\lambda=1$): \textit{1}~--- $m=1$; \textit{2}~--- $2$;  and \textit{3}~--- $m=3$}}
}

\vspace*{-3pt}

\addtocounter{figure}{1}


\section{Estimation via Monte Carlo}

\noindent
A more flexible alternative to analytic results is represented  by simulation 
that in our case can be used to estimate 
\begin{equation*}
\pi(T):=\mathbb{P} \left(\tau_D > T\right)\,.
\end{equation*}
Such probability could be extremely small for large values of~$T$; 
thus, its estimation with a~given accuracy requires to generate a~large number of 
sample paths of the process~$X$. However, for such type of rare events, it is 
possible to apply a special case of the well-known CMC 
method which always leads to variance reduction.

The method, originally proposed by some of the authors 
in~\cite{20-luk-1, 21-luk-1, 22-luk-1} and named Bridge Monte Carlo (BMC), 
is based on the idea of expressing the target probability as the
expectation of a function of the {Bridge} $Y:=\{Y_t\}$ of the
Gaussian process~$X$, i.\,e.,  the process obtained by
conditioning~$X$ to reach a certain level at some prefixed time instant~$\tau$:
\begin{equation*}
Y(t) = X(t) - \psi(t) X(\tau)
\end{equation*}
where $\psi$ can be easily  expressed in terms of the the covariance 
function~$\Gamma(s,t)$ of the process~$X$ 
$$
\psi(t)   :=
\fr{\Gamma(t,\tau)}{\Gamma(\tau, \tau)} \,.  
$$
Since the variance of~$X$ is an increasing function of~$t$ in all models we consider,
it is easy to see that  $\psi(t)>0$ for all $t \ge 0$.
Moreover, for any~$t$, $Y(t)$ is
independent of~$X(\tau)$ since
$$
\mathbb{E} \left[X(\tau)Y(t)\right]=
\Gamma(\tau,t)-
\fr{\Gamma(t,\tau)}{\Gamma(\tau,\tau)}\,\Gamma(\tau, \tau)=0
$$
and $(X(\tau),Y(t))$ has bivariate normal distribution.





Let $\mathbb{T} = [0,T]$, then the target probability can 
be expressed in the following way:
\begin{multline*}
\pi(T)\, =\mathbb{P}\left(\sup\limits_{t \in [0,T]}A(t)\ge D\right)\\
{}=\mathbb{P}\left(\forall t \in \mathbb{T}:\,\,mt+X(t) \le D\right)\\
{}=\mathbb{P}\left( \forall t \in \mathbb{T}:\,\, X(\tau) \le \fr{D-Y(t)-mt}{\psi(t)}\right)\\
{}=\mathbb{P}\left( X(\tau)\le \inf_{t \in T} \fr{D-Y(t)-mt}{\psi(t)}\right)\\
{}=\mathbb{P}\left( X(\tau)\le\overline{Y} \right)
\end{multline*}
where
\begin{equation*}
\overline{Y}:=\inf\limits_{t \in \mathbb{T}}\fr{D-Y(t)-mt}{\psi(t)}\,. 
%\label{BMC_4}
\end{equation*}

Finally, the  considered probability can be rewritten as follows:

\begin{equation*}
\pi(T)=\mathbb{P}\left(X_{\tau}\le\overline{Y}\right)=
\mathbb{E}\left[\Psi\left(\fr{\overline{Y}}{ \sqrt{\Gamma(\tau,\,\tau)}}\right)\right]
\end{equation*}
where independence between~$\overline{Y}$ and~$X_{\tau}$ is used and~$\Psi$ denotes 
the cumulative distribution function of a~standard normal variable.

Hence, given $N$ samples $\{\overline{Y}^{(n)},\,\,n=1,\ldots,N\}$  of~~$\overline{Y}$,
the estimator of~$\pi(T)$ is defined as follows:
\begin{equation*}
\widehat{\pi}_N^{\mathrm{BMC}} \: := \: \fr{1}{N}\sum\limits_{n=1}^N
\Psi
\left(\fr{\overline{Y}^{(n)}}{\sqrt{\Gamma(\tau,\tau)}}\right).
%\label{estimator}
\end{equation*}


Note that
\begin{equation*}
\Psi\left(\fr{\overline{Y}}{ \sqrt{\Gamma(\tau,\,\tau)}}\right)=
\mathbb{E} \left[ I(X(\tau) \le \overline{Y}) | \overline{Y}\right]
%\label{BMC-5}
\end{equation*}
and, therefore, the BMC approach is actually a special case of the  
CMC method;
so, one can expect that the BMC estimator implies variance reduction  (with
regard to crude 
Monte-Carlo simulation) in   the estimation of the target probability~$\pi(T)$ as also 
justified by the previous experience when such a~method was successfully applied for 
estimation some other rare-event probabilities related to Gaussian 
processes~\cite{23-luk-1}.  


\section{Concluding Remarks and~Future~Research}

\noindent
 In this paper,   a~stochastic model describing the dynamics of 
 a~DG project with many hosts and many work\-units to be performed, originally
 proposed in~\cite{1-luk-1},  is presented. 
 It is assumed that the project  can be   described by the so-called on-off model 
 where the hosts are on-off  sources of the work\-units  and the basic process is the 
 completed work. It is assumed that  the hosts' working sessions can have both 
  light- and heavy-tailed distributions.
 Then, an approximation of the  basic process, based on  the asymptotics of 
 the superposed on-off sources, is applied.    
 The suggested approach   leads to a~Gaussian approximation of the process of the 
 completed work. Finally,  a~simulation framework for the evaluation 
 of the runtime of the project, using the properties of Gaussian processes and 
 CMC simulation, is presented. 

Although  this note is focused on estimation of the runtime related to the 
1st stage of the project completion when the number of workunits is bigger 
than the number of hosts,  the 2nd  stage 
could also be relevant.
In more detail, it can be considered as a~collection of the ``tails''  
of the  workunit remaining times. From this point of view, the completion time of the 
2nd stage of the project  can be interpreted as  the \textit{longest} remaining time 
and analyzed by means of the asymptotic results of \textit{renewal theory}.
 Moreover, since the workunits are assumed to be independent, to evaluate the  
 duration of the  2nd stage, it seems promising to  apply the theory of 
 \textit{order statistics} and interpret the completion  time as the maximal 
 order statistics. 


\Ack
\noindent
The study was carried out under state order to the Karelian Research 
Centre of the Russian Academy of Sciences (Institute of Applied Mathematical 
Research KarRC RAS) and supported by the Russian Foundation for Basic Research, 
projects 18-07-00187, 18-07-00147, 18-07-00156, and 19-07-00303.


\renewcommand{\bibname}{\protect\rmfamily References}


%\vspace*{-6pt}

{\small\frenchspacing
{ %\baselineskip=10.35pt
\begin{thebibliography}{99}


\bibitem{2-luk-1} %1
\Aue{Leland, W.\,E., M.\,S.~Taqqu, W.~Willinger., and D.\,V.~Wilson.}
 1994. On the self-similar nature of ethernet traffic (extended version). 
 \textit{IEEE ACM~T. Network.} 2(1):1--15.
\bibitem{3-luk-1} %2
\Aue{Willinger, W., M.\,S.~Taqqu, W.\,E.~Leland, and D.~Wilson.}
1995. Self-similarity in high-speed packet traffic: Analysis and modeling of Ethernet 
traffic measurements. \textit{Stat. Sci.} 10(1):67--85.
\bibitem{4-luk-1}  %3
\Aue{Taqqu, M.\,S., W.~Willinger, and R.~Sherman.} 1997. 
Proof of a~fundamental result in self-similar traffic modeling. 
\textit{Comp. Comm.~R.} 27:5--23.
\bibitem{5-luk-1} %4
BOINCstats. 2017. Available at: {\sf https://boincstats.com} (accessed May~7, 2019).

\bibitem{7-luk-1} %5
\Aue{Kondo, D., D.\,P.~Anderson, and J.~McLeod~VII.} 
2007. Performance evaluation of scheduling policies for volunteer computing.  
\textit{3rd IEEE Conference (International) on e-Science and Grid Computing 
Proceedings}. IEEE. 221--227.
\bibitem{6-luk-1} %6
\Aue{Estrada, T., and M.~Taufer.} 2012. Challenges in 
designing scheduling policies in volunteer computing. 
\textit{Desktop grid computing}. Eds C.~C$\acute{\mbox{e}}$rin and G.~Fedak. 
CRC Press. 167--190.

\bibitem{8-luk-1} %7
\Aue{Durrani, N., and J.~Shamsi.} 2014. Volunteer computing: 
Requirements, challenges, and solutions. \textit{J.~Netw. Comput. Appl.} 39:369--380.
\bibitem{9-luk-1} %8
\Aue{Sonnek, J., M.~Nathan, A.~Chandra, and J.~Weissman.}
 2006. Reputation-based scheduling on unreliable distributed infrastructures in 
 distributed computing systems. \textit{26th IEEE Conference 
 (International) on Distributed Computing Systems Proceedings}. IEEE. Art.~No.\,30. P.~1--8.
\bibitem{10-luk-1} %9
\Aue{Watanabe, K., M.~Fukushi, and M.~Kameyama.}
 2011. Adaptive group-based job scheduling for high performance and reliable 
 volunteer computing.  \textit{J.~Information Processing} 19:39--51.
 
 \bibitem{12-luk-1} %10
\Aue{Xavier, E., R.~Peixoto, and J.~da~Silveira.}
 2013. Scheduling with task replication on desktop grids: 
 Theoretical and experimental analysis.  \textit{J.~Comb. Optim.}
 30(3):520--544.
\bibitem{11-luk-1} %11
\Aue{Chernov, I.\,A., and N.\,N.~Nikitina.}
2015. Virtual screening in a~Desktop Grid: Replication and the optimal quorum. 
\textit{Parallel computing technologies}. Ed. V.~Malyshkin.
Lecture notes in computer science ser. Springer.  9251:258--267.

\bibitem{13-luk-1} %12
\Aue{Samorodnitsky, G., and M.\,S.~Taqqu.} 1994.  \textit{Stable non-Gaussian random processes: 
Stochastic models with infinite variance}. Chapman \& Hall. 632~p.
\bibitem{14-luk-1} %13
\Aue{Mikosch, T., S.~Resnick, H.~Rootz$\acute{\mbox{e}}$n, and A.~Stegeman.}
2002. Is network traffic approximated by stable Levy motion or fractional Brownian 
motion? \textit{Ann. Appl. Probab.} 12(1):23--68.
\bibitem{15-luk-1} %14
\Aue{Norros, I.} 1994. A~storage model with self-similar input.  
\textit{Queueing Syst.} 16:387--396.
\bibitem{16-luk-1} %15
\Aue{Borodin, A.\,N., and P.~Salminen.} 2002.  \textit{Handbook of Brownian motion~--- 
facts and formulae}. Birkh$\ddot{\mbox{a}}$user. 685~p.
\bibitem{17-luk-1} %16
\Aue{Michna, Z.} 1999. On tail probabilities and first passage times for fractional 
Brownian motion.  \textit{Math. Method. Oper. Res.} 49(2):335--354.
\bibitem{18-luk-1} %17
\Aue{Caglar, M., and C.~Vardar.} 2013. Distribution of maximum loss of fractional 
Brownian motion with drift.  \textit{Stat. Probabil. Lett.} 83:2729--2734.
\bibitem{19-luk-1} %18
Gradshtein, I.\,S., I.\,M.~Ryzhik, and A.~Jeffrey, eds.
%D.~Zwillinger, associate ed. 
2015. 
\textit{Table of integrals, series and products}. 8th ed. 
San Diego, CA: Academic Press. 1220~p.
\bibitem{20-luk-1} %19
\Aue{Giordano, S., M.~Gubinelli, and M.~Pagano.}
 2005. Bridge Monte-Carlo: A~novel approach to rare events of Gaussian processes. 
 \textit{5th St. Petersburg Workshop on Simulation Proceedings}. St.\ Petersburg: 
 St. Petersburg State University. 281--286.
\bibitem{21-luk-1} %20
\Aue{Giordano, S., M.~Gubinelli, and M.~Pagano.}
 2007. Rare events of Gaussian processes: A~performance comparison between 
 bridge Monte-Carlo and importance sampling. 
 \textit{Next generation teletraffic and wired/wireless advanced networking}.
 Eds.\ Y.~Koucheryavy, J.~Harju, and A.~Sayenko.
 Lecture notes in computer science ser.
 Springer. 4712:269--280.
\bibitem{22-luk-1} %21
\Aue{Lukashenko, O.\,V., E.\,V.~Morozov, and M.~Pagano.} 
2012. Performance analysis of Bridge Monte-Carlo estimator. 
\textit{Transactions of KarRC RAS} 5:54--60.
\bibitem{23-luk-1} %22
\Aue{Lukashenko, O.\,V., E.\,V.~Morozov, and M.~Pagano.}
 2017. On the efficiency of bridge Monte-Carlo estimator.  
 \textit{Informatika i~ee Primeneniya~--- Inform.  Appl.} 11(2):16--24.
 
 \bibitem{1-luk-1} %23
\Aue{Morozov, E., O.~Lukashenko, A.~Rumyantsev, and E.~Ivashko.}
2017. A~Gaussian approximation of runtime estimation in a desktop grid project. 
\textit{9th  Congress (International) on Ultra Modern Telecommunications and 
Control Systems and Workshops}. IEEE. 107--111.


\end{thebibliography} } }

\end{multicols}

\vspace*{-6pt}

\hfill{\small\textit{Received April 15, 2019}}

\vspace*{-18pt}

\Contr

%\vspace*{-3pt}

\noindent
\textbf{Lukashenko Oleg  V.} (b.\ 1986)~--- 
Candidate of Science (PhD) in physics and mathematics, scientist, 
Institute of  Applied Mathematical Research of Karelian Research Centre of 
the Russian Academy of Sciences, 
11~Pushkinskaya Str.,  Petrozavodsk 185910, Republic of Karelia, 
Russian Federation; associate professor, Petrozavodsk State University, 33~Lenin Str., 
Petrozavodsk 185910, Republic of Karelia, Russian Federation; 
\mbox{lukashenko@krc.karelia.ru}

\vspace*{3pt}

\noindent
\textbf{Morozov  Evsei  V.} (b.\ 1947)~--- 
Doctor of Science in physics and mathematics, professor, leading scientist,
 Institute of  Applied Mathematical Research of Karelian Research Centre of 
 the Russian Academy of Sciences, 11~Pushkinskaya Str.,  Petrozavodsk 185910, 
 Republic of Karelia, Russian Federation; professor, Petrozavodsk State University, 
 33~Lenin Str., Petrozavodsk 185910, Republic of Karelia, Russian Federation; 
 \mbox{emorozov@karelia.ru}
 
 \vspace*{3pt}
 
 \noindent
 \textbf{Pagano Michele} (b.\ 1968)~--- 
 PhD in Information Engineering, associate professor, University of Pisa, 
 43~Lungarno Pacinotti, Pisa 56126, Italy; \mbox{m.pagano@iet.unipi.it}

 


\vspace*{8pt}

\hrule

\vspace*{2pt}

\hrule

%\vspace*{-7pt}

%\newpage

%\vspace*{-28pt}

\def\tit{ГАУССОВСКАЯ АППРОКСИМАЦИЯ ПРОЦЕССА РАСПРЕДЕЛЕННЫХ ВЫЧИСЛЕНИЙ$^*$}

\def\titkol{Гауссовская аппроксимация процесса распределенных вычислений}

\def\aut{О.\,В.~Лукашенко$^{1,2}$, Е.\,В.~Морозов$^{1,2}$,
М.~Пагано$^3$}

\def\autkol{О.\,В.~Лукашенко, Е.\,В.~Морозов,
М.~Пагано}

{\renewcommand{\thefootnote}{\fnsymbol{footnote}} \footnotetext[1]
{Финансовое обеспечение исследований осуществлялось из 
средств федерального бюджета на выполнение государственного задания 
КарНЦ РАН (Институт прикладных математических исследований КарНЦ РАН) 
и~при финансовой поддержке РФФИ (проекты 18-07-00187, 18-07-00147, 18-07-00156
и~19-07-00303).}}



\titel{\tit}{\aut}{\autkol}{\titkol}

\vspace*{-11pt}

\noindent
$^1$Институт прикладных математических исследований Карельского научного центра 
Российской акаде-\linebreak
$\hphantom{^1}$мии наук 

\noindent
$^2$Петрозаводский государственный университет

\noindent
$^3$Университет г.\ Пиза, Италия
%, danielkh@post.bgu.ac.il 

\vspace*{1pt}

\def\leftfootline{\small{\textbf{\thepage}
\hfill ИНФОРМАТИКА И ЕЁ ПРИМЕНЕНИЯ\ \ \ том\ 13\ \ \ выпуск\ 2\ \ \ 2019}
}%
 \def\rightfootline{\small{ИНФОРМАТИКА И ЕЁ ПРИМЕНЕНИЯ\ \ \ том\ 13\ \ \ выпуск\ 2\ \ \ 2019
\hfill \textbf{\thepage}}}

\vspace*{-1pt}


 
\Abst{Продолжено изучение стохастической модели процесса динамики 
выполнения задачи в~сис\-те\-ме Desktop Grid при наличии многих пользователей, 
предложенной в~2017~г.\ Морозовым с~соавт. Тре\-бу\-емой характеристикой 
выступает средняя 
про\-дол\-жи\-тель\-ность времени выполнения проекта. Гауссовская аппроксимация искомого 
процесса производится на основе предельных тео\-рем для суперпозиции on-off 
источников. Приведен обзор известных аналитических результатов для 
тре\-бу\-емой характеристики, вклю\-чая результаты для броуновского 
и~дроб\-но\-го броуновского движения. Также показывается, как с~по\-мощью условного метода 
Мон\-те-Кар\-ло оценить хвост распределения времени выполнения проекта.}


\KW{гауссовская аппроксимация; распределенные вычисления; дробное броуновское движение}

 \DOI{10.14357/19922264190215}



%\vspace*{-3pt}


 \begin{multicols}{2}

\renewcommand{\bibname}{\protect\rmfamily Литература}
%\renewcommand{\bibname}{\large\protect\rm References}

{\small\frenchspacing
{\baselineskip=10.5pt
\begin{thebibliography}{99}
%\vspace*{-3pt}


\bibitem{2-luk} %1
\Au{Leland W.\.E., Taqqu~M.\,S., Willinger~W., Wilson~D.\,V.}
On the self-similar nature of Ethernet traffic (extended version)~// 
IEEE ACM~T. Network., 1994. Vol.~2. Iss.~1. P.~1---15.
\bibitem{3-luk} %2
\Au{Willinger W., Taqqu~M.\,S., Leland~W.\,E., Wilson~D.}
 Self-similarity in high-speed packet traffic: Analysis and modeling of Ethernet 
 traffic measurements~// Stat. Sci., 1995. Vol.~10. Iss.~1. P.~67--85.
\bibitem{4-luk} %3
\Au{Taqqu M.\,S., Willinger~W., Sherman~R.} 
Proof of a~fundamental result in self-similar traffic modeling~// 
Comp. Comm.~R., 1997. Vol.~27. P.~5--23.
\bibitem{5-luk} %4
 BOINCstats, 2017. {\sf https://boincstats.com}.

\bibitem{7-luk} %5
\Au{Kondo D., Anderson~D.\,P., McLeod~VII~J.}
Performance evaluation of scheduling policies for volunteer computing~// 
3rd IEEE  Conference (International) 
on e-Science and Grid Computing Proceedings.~--- IEEE, 2007. P.~221--227.

\bibitem{6-luk} %6
\Au{Estrada T., Taufer~M.}
Challenges in designing scheduling policies in volunteer computing~// 
Desktop grid computing~/ Eds. C.~C$\acute{\mbox{e}}$rin, G.~Fedak.~--- 
CRC Press, 2012. P.~167--190.
\bibitem{8-luk} %7
\Au{Durrani N., Shamsi~J.} 
Volunteer computing: Requirements, challenges, and solutions~// 
J.~Netw. Comput. Appl., 2014. Vol.~39. P.~369--380.
\bibitem{9-luk} %8
\Au{Sonnek J., Nathan~M., Chandra~A., Weissman~J.}
 Reputation-based scheduling on unreliable distributed infrastructures in 
 distributed computing systems~// 26th IEEE Conference (International)
 on Distributed Computing Systems Proceedings.~--- IEEE, 2006. Art. No.\,30. P.~1--8.
\bibitem{10-luk} %9
\Au{Watanabe K., Fukushi~M., Kameyama~M.}
Adaptive group-based job scheduling for high performance and reliable volunteer 
computing~// J.~Information Processing, 2011. Vol.~19. P.~39--51.

\bibitem{12-luk} %10
\Au{Xavier E., Peixoto R., da~Silveira~J.}
 Scheduling with task replication on desktop grids: Theoretical and experimental 
 analysis~// J.~Comb. Optim., 2013.  Vol.~30. Iss.~3. P.~520--544.
 
 \bibitem{11-luk} %11
\Au{Chernov I.\,A., Nikitina~N.\,N.}
 Virtual screening in a desktop grid: Replication and the optimal quorum~// 
  Parallel computing technologies~/ Ed. V.~Malyshkin.~---
 Lecture notes in computer science ser.~---  Springer, 2015.  
 Vol.~9251. P.~258--267.
 
\bibitem{13-luk} %12
\Au{Samorodnitsky G., Taqqu~M.\,S.} Stable non-Gaussian random processes: Stochastic 
models with infinite variance.~--- Chapman \& Hall, 1994. 632~p.
\bibitem{14-luk} %13
\Au{Mikosch T., Resnick~S., Rootz$\acute{\mbox{e}}$n~H., Stegeman~A.}
Is network traffic approximated by stable Levy motion or fractional Brownian motion?~// 
Ann. Appl. Probab., 2002. Vol.~12. Iss.~1. P.~23--68.
\bibitem{15-luk} %14
\Au{Norros I.} A~storage model with self-similar input~// Queueing Syst., 1994. 
Vol.~16. P.~387--396.
\bibitem{16-luk} %15
\Au{Borodin A.\,N., Salminen~P.}
 Handbook of Brownian motion~--- facts and formulae.~--- Birkh$\ddot{\mbox{a}}$user, 2002. 685~p.
\bibitem{17-luk} %16
\Au{Michna Z.} On tail probabilities and first passage times for fractional 
Brownian motion~// Math. Method. Oper. Res., 1999. Vol.~49. Iss.~2. 
P.~335--354.
\bibitem{18-luk} %17
\Au{Caglar M., Vardar~C.} Distribution of maximum loss of fractional 
Brownian motion with drift~// Stat. Probabil. Lett., 2013. Vol.~83. P.~2729--2734.
\bibitem{19-luk} %18
Table of integrals, series and products~/
Eds.  I.\,S.~Gradshtein, I.\,M.~Ryzhik, A.~Jeffrey.~--- 8 ed.~---
%; associate editor D.~Zwillinger.   
San Diego, CA, USA: Academic Press, 2015. 1220~p.
\bibitem{20-luk} %19
\Au{Giordano S., Gubinelli~M., Pagano~M.} Bridge Monte-Carlo: 
A~novel approach to rare events of Gaussian processes~// 5th St.\ 
Petersburg Workshop on Simulation Proceedings.~--- St.\ Petersburg: St. Petersburg 
State University, 2005. P.~281--286.
\bibitem{21-luk} %20
\Au{Giordano S., Gubinelli~M., Pagano~M.} Rare events of Gaussian processes: 
A~performance comparison between bridge Monte-Carlo and importance sampling~// 
Next generation teletraffic and wired/wireless advanced networking~/
 Eds.\ Y.~Koucheryavy, J.~Harju, A.~Sayenko.~--- 
 Lecture notes in computer science ser.~--- Springer, 2007.
Vol.~4712. P.~269--280.
\bibitem{22-luk} %21
\Au{Lukashenko O.\,V., Morozov~E.\,V., Pagano~M.}
Performance analysis of bridge Monte-Carlo estimator~// 
Труды Карельского научного центра Российской академии наук, 
2012. Т.~5. С.~54--60.
\bibitem{23-luk} %22
\Au{Lukashenko O.\,V., Morozov~E.\,V., Pagano~M.}
 On the efficiency of bridge Monte-Carlo estimator~// Информатика и её применения,
  2017.  Т.~11. Вып.~2. С.~16--24.

\bibitem{1-luk} %23
\Au{Morozov E., Lukashenko~O., Rumyantsev~A., Ivashko~E.}
A~Gaussian approximation of runtime estimation in a~desktop grid project~// 
9th  Congress (International) on Ultra Modern Telecommunications and Control Systems 
and Workshops.~--- IEEE, 2017. P.~107--111.

\end{thebibliography}
} }

\end{multicols}

 \label{end\stat}

 \vspace*{-9pt}

\hfill{\small\textit{Поступила в~редакцию 15.04.2019}}


%\renewcommand{\bibname}{\protect\rm Литература}
\renewcommand{\figurename}{\protect\bf Рис.}
\renewcommand{\tablename}{\protect\bf Таблица}  %
\def\stat{agalarov}


\def\tit{ПРИБЛИЖЕННЫЙ МЕТОД ВЫЧИСЛЕНИЯ ХАРАКТЕРИСТИК УЗЛА 
ТЕЛЕКОММУНИКАЦИОННОЙ СЕТИ С~ПОВТОРНЫМИ ПЕРЕДАЧАМИ}
\def\titkol{Приближенный метод вычисления характеристик узла 
телекоммуникационной сети с~повторными передачами} 

\def\autkol{Я.\,М.~Агаларов}
\def\aut{Я.\,М.~Агаларов$^1$}

\titel{\tit}{\aut}{\autkol}{\titkol}

%{\renewcommand{\thefootnote}{\fnsymbol{footnote}}\footnotetext[1]
%{Работа выполнена при поддержке РФФИ, проекты 08--07--00152 и 08--01--00567.}}

\renewcommand{\thefootnote}{\arabic{footnote}}
\footnotetext[1]{Институт проблем
информатики Российской академии наук, agglar@yandex.ru}

%\vspace*{-6pt}


\Abst{Рассмотрена модель узла коммутации пакетов c повторными передачами для двух 
схем распределения буферной памяти: полнодоступной и полного разделения. Предложен 
приближенный метод вычисления интенсивностей потоков и вероятностей блокировок узла. 
Получены необходимые и достаточные условия существования и единственности решения 
уравнения для потоков в узле при установившемся режиме работы и доказана сходимость 
итерационного метода решения указанного уравнения.}

\KW{узел коммутации пакетов; буферная память; повторные передачи; вероятности 
блокировок; итерационный метод}

      \vskip 18pt plus 9pt minus 6pt

      \thispagestyle{headings}

      \begin{multicols}{2}

      \label{st\stat}


\section{Введение}

    Одной из основных задач предварительного анализа 
телекоммуникационных сетей коммутации пакетов с ограниченной буферной 
памятью является расчет характеристик потоков и вероятностей блокировок в 
узлах связи. Важность указанных характеристик определяется тем, что от их 
значений существенным образом зависят другие основные показатели сети 
(пропускная способность, задержки пакетов и~др.). 

    Существует множество различных моделей узлов коммутации пакетов и 
методов их расчета (см., например,~[1--6]). Для моделей, рассматривающих 
узел с ограниченной буферной памятью как систему массового обслуживания 
(CMO) типа 
$
\begin{matrix}
M \\ \lambda
\end{matrix}
\left |
\begin{matrix}
M \\ \lambda
\end{matrix}
\right |
\overline{m} \vert N
$ или  $\vert PH\vert PH\vert 1\vert r$, в предположении отсутствия повторных 
передач пакетов получены точные методы вычисления характеристик 
узлов~[1, 3, 4, 6]. Приближенные методы расчета узлов, учитывающие повторные 
попытки передачи, используют модели типа $\vert PH\vert PH\vert 1\vert r$ или 
$
\begin{matrix}
M \\ \lambda
\end{matrix}
\left |
\begin{matrix}
M \\ \lambda
\end{matrix}
\right |
1 \vert N
$ и являются 
итерационными~[2, 3, 5, 7]. Для моделей типа 
$BM\!AP\vert PH\vert 1$, $M\vert G\vert 1\vert r$ и $M\!AP\vert 
(PH,PH)\vert 1$ с повторными заявками получены точные методы вычисления 
характеристик (например, в работах~[8--10]), которые также могут быть 
использованы при расчете узлов.

    Ниже будут рассмотрены модели узла коммутации пакетов с повторными 
передачами для двух схем распределения буферной памяти: с 
полнодоступными буферами и с полным разделением буферной памяти. 
Предлагается приближенный метод расчета характеристик, который в качестве 
модели узла использует СМО типа $
\begin{matrix}
M \\ \lambda
\end{matrix}
\left |
\begin{matrix}
M \\ \lambda
\end{matrix}
\right |
\overline{m} \vert N
$ с повторными заявками. Доказаны утверждения о 
достаточных и необходимых условиях существования и единственности 
решения уравнения для вероятности блокировки в установившемся режиме 
работы и сходимости предлагаемого итерационного метода. 

\section{Модель узла}

    Математическая модель узла представляется в виде СМО с ограниченной 
буферной памятью и различными потоками заявок, каждая из которых требует 
обслуживания только на одной из многоканальных линий связи. 

    Пусть $0<N<\infty$~--- число мест хранения в буферной памяти, $u$~--- 
узел связи, $v$~--- линия связи, $\Omega_u^+$~--- множество исходящих из 
узла~$u$ линий, $c_v$~--- канальная емкость линии~$v$. Поток заявок, 
тре\-бу\-ющих обслуживания на линии~$v$, назовем $v$-по\-то\-ком, заявки этого 
потока~--- $v$-за\-яв\-ка\-ми.


    Пусть выполняются следующие предположения: 
\begin{enumerate}[1.]
\item Места в буферной памяти распределяются согласно одной из двух 
схем:
\begin{enumerate}[($i$)]
\item полнодоступная схема~--- каждое свободное место хранения доступно 
любой заявке;
\item схема полного разделения памяти~--- $v$-за\-яв\-кам доступны всего 
$N_v$ мест, где $\sum\limits_{v\in\Omega_u^+} N_v=N$.
\end{enumerate}
\item Если в момент поступления $v$-заявки в буферной памяти есть 
доступное свободное место, то она сразу занимает это место. Если в момент 
поступления $v$-заявки в системе нет свободного доступного места 
хранения, то поступившая заявка через некоторое время повторно поступает 
на систему, оставаясь $v$-заявкой. 
\item Интенсивности первичных потоков $v$-заявок~--- заданные величины 
$0<\Lambda_v<\infty$, $v\in \Omega_u^+$. Суммарные потоки первичных и 
повторных $v$-заявок являются независимыми в совокупности 
пуассоновскими потоками. Для обслуживания $v$-заявки требуется 
одновременно одно место хранения и один канал типа~$v$, $v\in 
\Omega_u^+$.
\item Первичные нагрузки~--- реализуемые, т.\,е.\ в данном случае 
интенсивности входных первичных потоков равны интенсивностям 
выходных потоков выполненных заявок. 
\item Принятые в СМО $v$-заявки обслуживаются линией~$v$ в порядке 
поступления. 
\item Время занятия канала $v$-заявкой~--- экспоненциально 
распределенная случайная величина с параметром $0<\mu_v<\infty$, 
$v\in\Omega_u^+$, независимая от других случайных событий в узле.
\item Выполненная $v$-заявка с вероятностью~$B_v$ повторяется через 
заданное время~$\tau_v$ (тайм-аут) и с вероятностью $1-B_v$ покидает 
систему через время~$t_v$ навсегда, сразу освободив занятый канал и место 
буферной памяти.
\end{enumerate}

   Будем говорить, что узел блокирован для $v$-за\-яв\-ки, если в буферной 
памяти отсутствует доступное место хранения. Ставится задача вычисления 
вероятностей блокировок и интенсивностей потоков в узле.

\section{Вычисление вероятности блокировки и~интенсивностей~потоков} 

   Пусть $\Lambda_v^*$~--- интенсивность суммарного потока внешних 
заявок, требующих передачи по линии~$v$, $\pi_v$~--- вероятность блокировки 
узла для заявок, требующих передачи по исходящей из узла линии~$v$. 

    Пусть в узле используется полнодоступная схема распределения 
буферной памяти. Тогда, как следует из описания модели, $\pi_v 
=\pi_{v^\prime},\,v,\,v^\prime\in \Omega_u^+$, и для 
интенсивностей~$\Lambda_v^*$, $v\in\Omega_u^+$, справедливы соотношения:
\begin{equation*}
\Lambda_v^* = \fr{\Lambda_v}{1-\pi}\,,
%\label{e1aga}
\end{equation*}
    где
    $\pi =\pi_v$, $v\in\Omega_u^+$.

    Пусть 
    $\overline{k} = \{\overline{k}_v$, $v\in\Omega_u^+\}$~--- состояние 
буферной памяти узла, $\overline{k}_v =\left ( k_v,\,k_v^\prime,\,k_v^{\prime\prime}\right )$; 
$k_v$~--- число $v$-заявок в буферной 
памяти, ожидающих выполнения линией~$v$; $k^\prime_v$~--- число 
$v$-заявок в буферной памяти, ожидающих тайм-аут и неуспешно переданных 
в последующий узел; $k_v^{\prime\prime}$~--- число $v$-за\-явок в буферной 
памяти, успешно переданных в последующий узел и ожидающих 
потверждения; 
$A_m = \left \{ \overline{k}:\ \sum\limits_{v\in\Omega_u^+} \left ( 
k_v+k_v^\prime + k_v^{\prime\prime}\right ) =m \right \}$~--- множество различных 
состояний, при которых в памяти узла занято ровно $m$~буферов. Тогда с 
учетом введенных выше обозначений и предположений для ве\-ро\-ят\-ности 
блокировки узла можно написать формулу~\cite{1aga, 2aga}:
\begin{equation}
\pi = \fr{1}{G_N}\sum\limits_{\overline{k}\in A_N} 
p\left (\overline{k},\overline{\rho}^*\right )\,,
\label{e2aga}
\end{equation}
где  
\begin{gather}
p(\overline{k},\overline{\rho}^*) = \prod\limits_{v\in\Omega_u^+} z_v (\pi, 
\rho_v , k_v , k_v^\prime , k_v^{\prime\prime})\,;\\
z_v (\pi, \rho_v , k_v , k_v^\prime , k_v^{\prime\prime}) ={}\notag\\
\!\!{}=
\begin{cases}
 \fr{\rho_v^{\prime *k_v^\prime}}{k_v^{\prime}!}\,
\fr{\rho_v^{\prime\prime * k_v^{\prime\prime}}}{ k_v^{\prime\prime}!}  \,
\fr{\rho_v^{*k_v}}{ k_{v}!} 
&\mbox{при}\ k_v<c_v\,,\\
 \fr{\rho_v^{\prime * k_v^\prime}}{k_v^{\prime}!} \,
\fr{\rho_v^{\prime\prime * k_v^{\prime\prime}}} { k_v^{\prime\prime}!} 
\fr{\rho_v^{*k_v}}{ c_{v}!c_v^{k_v- c_v}} 
& \mbox{при}\ k_v\geq c_v\,;
\end{cases}\\
G_N = \sum\limits_{m=0}^N\sum\limits_{\overline{k}\in A_m}
p(\overline{k},\overline{\rho}^*)\,;\\ 
\overline{\rho}^*=\{\rho_v^*,\,v\in\Omega_u^+\}\,;\\
\rho_v^* = \fr{\rho_v}{1-\pi}\,;\quad \rho_v =\fr{\Lambda_v}{\mu_v(1- B_v)}\,;\\
\rho_v^{\prime *} =\rho_v^*\mu_v\tau_vB_v\,;\quad \rho_v^{\prime\prime *}=
p_v^* \mu_vt_v,\,\quad  v\in \Omega_u^+\,.\label{e3aga}
\end{gather}

Переобозначив $1-\pi$ через $y$, выражение в правой части равенства~(2)~--- через 
$p_{\overline{k}}(\overline{\rho},y)$, выражение в правой части равенства~(4)~--- 
через $g_N(\overline{\rho},y)$, а выражение в правой 
части равенства~(1)~--- через $1-q_N (\overline{\rho},y)$, 
где $\overline{\rho} = (\rho_v,\,v\in \Omega_u^+)$, $\rho_v = \rho_v^*y\;=$\linebreak 
$=\;\Lambda_v/(\mu_v(1-B_v))$, $v\in\Omega_u^+$, получим нелинейное уравнение 
относительно неизвестной переменной~$y$:
\begin{equation}
y=q_N(\overline{\rho},y)\,.
\label{e4aga}
\end{equation}

    Решим уравнение~(8). Как следует из~(2)--(7), верно 
равенство
\begin{equation}
q_N(\overline{\rho},y) = \fr{g_{N-1}(\overline{\rho},y )}{g_N(\overline{\rho},y)}\,.
\label{e5aga}
\end{equation}
Введем функцию  $d_n(\overline{\rho} ,y)$ среднего числа заявок в узле с 
буферной памятью емкости $n\geq 0$:
$$
d_n(\overline{\rho} ,y) = 
\fr{1}{g_n(\overline{\rho},y)}\,\sum\limits_{m=0}^n m\sum\limits_{\overline{k}\in 
A_m} p_{\overline{k}}(\overline{\rho},y)\,.
$$
Заметим, что $g_n$, $d_n$ и $q_n$, 
$n\geq 0$,~--- непрерывно-дифференцируемые функции по $y\in (0,\,1]$. Взяв 
производную функции~$g_n$ по~$y$, из~(2)--(7) получим
\begin{multline}
\fr{\partial g_n(\overline{\rho},y)}{\partial y} ={}\\
{}= -\sum\limits_{m=0}^n m 
\sum\limits_{\overline{k}\in A_m}\fr{\prod\limits_{v\in\Omega_u^+} z_n 
(0,\rho_v, k_v, k_v^\prime , k_v^{\prime\prime})}{y^{m+1}}={}\\
{}= -\fr{1}{y}\,g_n (\overline{\rho},y)d_n(\overline{\rho},y)\,.
\label{e6aga}
\end{multline}
Взяв производную функции $q_N$ по $y$, из~(\ref{e5aga}) и~(\ref{e6aga}) 
получим
\begin{equation}
\fr{\partial q_N(\overline{\rho},y)}{\partial y} = \fr{q_N(\overline{\rho},y)}{y}\left 
[ d_N (\overline{\rho},y)-d_{N-1}(\overline{\rho},y)\right ]\,.
\label{e7aga}
\end{equation}
    Докажем несколько утверждений о свойствах 
функции~$q_N(\overline{\rho},y)$.
\medskip

\noindent
\textbf{Утверждение 1.} \textit{Справедливы неравенства}
\begin{multline}
0<d_{n+1}(\overline{\rho},y)-d_n(\overline{\rho},y) <1\,,\\
\ \ \ \ \ \ \ \ \ \ \ \ \ \ \ \ \ \ \ \ y\in (0,\,1]\,, \ n\geq 0\,.
\label{e8aga}
\end{multline}


\noindent

Д\,о\,к\,а\,з\,а\,т\,е\,л\,ь\,с\,т\,в\,о\,.\ Подставив выражение для функции 
$d_n(\overline{\rho},y)$ и проведя преобразования, получим
\begin{multline*}
d_{n+1}(\overline{\rho},y) -d_n(\overline{\rho},y) = 
\fr{\sum\limits_{m=0}^{n+1}m\sum\limits_{\overline{k}\in A_m} 
p_{\overline{k}}(\overline{\rho},y)}
{\sum\limits_{m=0}^{n+1}
\sum\limits_{\overline{k}\in A_m} p_{\overline{k}}(\overline{\rho},y)} - {}\\
{}-
\fr{\sum\limits_{m=0}^n m \sum\limits_{\overline{k}\in A_m} p_{\overline{k}} 
(\overline{\rho},y)}{\sum\limits_{m=0}^n
\sum\limits_{\overline{k}\in A_m}p_{\overline{k}}(\overline{\rho},y)}={}\\
{}=\fr{\sum\limits_{m=1}^n m \sum\limits_{\overline{k}\in 
A_m}p_{\overline{k}}(\overline{\rho},y)+(n+1)\sum\limits_{\overline{k}\in 
A_{n+1}}  p_{\overline{k}}(\overline{\rho},y)}{\sum\limits_{m=0}^n\sum\limits_{\overline{k
}\in A_m}p_{\overline{k}}(\overline{\rho},y)+\sum\limits_{\overline{k}\in 
A_{n+1}}p_{\overline{k}}(\overline{\rho},y)} -{}
\end{multline*}
\begin{multline}
{}-
\fr{\sum\limits_{m=0}^n m 
\sum\limits_{\overline{k}\in A_m}p_{\overline{k}}(\overline{\rho},y)}
{\sum\limits_{m=0}^n\sum\limits_{\overline{k}\in A_m} 
p_{\overline{k}}(\overline{\rho},y)}={}\\
{}=\fr{(n+1)\sum\limits_{\overline{k}\in 
A_{n+1}}p_{\overline{k}}(\overline{\rho},y)g_n(\overline{\rho},y)}{g_{n+1}(\overline{\rho},y) g_n(\overline{\rho},y)} -{}\\
{}-
\fr{\sum\limits_{\overline{k}\in 
A_{n+1}}p_{\overline{k}}(\overline{\rho},y)\sum\limits_{m=0}^n  m 
\sum\limits_{\overline{k}\in A_m} p_{\overline{k}}(\overline{\rho},y) }
{g_{n+1}(\overline{\rho},y) g_n(\overline{\rho},y)}
={}\\
{}=\left [ 1-q_{n+1}(\overline{\rho},y)\right ] \left [n+1-d_n(\overline{\rho},y)\right ]\,.
\label{e9aga}
\end{multline}


    Докажем утверждение~1 методом индукции. При $n = 0$, как следует 
из~(\ref{e9aga}), имеем
$$
d_2(\overline{\rho},y) - d_1 (\overline{\rho},y) =1-q_1(\overline{\rho},y)\,,
$$
    т.\,е.\ утверждение~1 при $n = 0$ справедливо. 

    Пусть неравенства~(\ref{e8aga}) справедливы для некоторого $n > 0$. 
Докажем, что они справедливы и для $n + 1$. Из~(\ref{e9aga}) получаем
\begin{multline*}
d_{n+1}(\overline{\rho},y)- d_n(\overline{\rho},y)={}\\
{}=\left [ 1-
q_{n+1}(\overline{\rho},y)\right ] \left [n+1-d_n(\overline{\rho},y)\right ] ={}\\
{}= \left [ 1-
1-q_{n+1}(\overline{\rho},y)\right ] \left [ n-{}\right.\\
{}-\left. d_{n-1}(\overline{\rho},y)+d_{n-1}(\overline{\rho},y)-
d_n(\overline{\rho},y)+1\right ] ={}\\
{}=\left [ 1-q_{n+1}(\overline{\rho},y)\right ] 
\left [ n-d_{n-1}(\overline{\rho},y)-{}\right.\\
{}-\left. \left ( d_n(\overline{\rho},y)-d_{n-1}(\overline{\rho},y)\right )+1\right] = {}\\
{}=
\left [ 1-q_{n+1}(\overline{\rho},y)\right ]
\left [ 
\fr{d_n(\overline{\rho},y) -d_{n-1}(\overline{\rho},y)}{1-
q_n(\overline{\rho},y)}\right.-{}\\
{}-\left.
\left ( d_n(\overline{\rho},y)-d_{n-1}(\overline{\rho},y)\right )+1
\vphantom{\fr{d_n(\overline{\rho})}{(q_n)}}
\right ]={}\\
{}=
\left [ 1-q_{n+1}(\overline{\rho},y)\right ]
\left [ 
\vphantom{\fr{d_n(\overline{\rho})}{(q_n)}}
\left ( d_n(\overline{\rho},y\right)\right. -{}\\
 {}-\left.
d_{n-1}\left(\overline{\rho},y)\right )\fr{q_n(\overline{\rho},y)}{1-
q_n(\overline{\rho},y)}+1\right ]\,.
\end{multline*}
Так как по предположению $d_n (\overline{\rho},y) -d_{n-1}(\overline{\rho},y) 
>0$, то правая часть последнего равенства больше нуля; следовательно, 
$d_{n+1}(\overline{\rho},y)-d_n(\overline{\rho},y)>0$. 

    Продолжив преобразование правой части последнего равенства и 
учитывая предположение $d_n(\overline{\rho},y) -d_{n-1}(\overline{\rho},y)<1$, 
получим
\begin{multline*}
d_{n+1}((\overline{\rho},y) -d_n(\overline{\rho},y)<{}\\
{}< \left [ 1-
q_{n+1}(\overline{\rho},y)\right ]
\left ( \fr{q_n(\overline{\rho},y)}{1-q_n(\overline{\rho},y)}+1\right )={}\\
{}=
\fr{1-q_{n+1}(\overline{\rho},y)}{1-q_n(\overline{\rho},y)}<1\,,
\end{multline*}
так как $0< q_n(\overline{\rho},y)<q_{n+1}(\overline{\rho},y)<1$, $n>0$, $y\in 
(0,\,1]$.

Следовательно, утверждение~1 доказано.

\medskip

\noindent
\textbf{Утверждение 2.} $q_N(\overline{\rho},y)$~--- \textit{монотонно-воз\-рас\-та\-ющая 
функция по $y\in (0,\,1]$. При этом $0< q_N(\overline{\rho},y)\;\leq $\linebreak 
$\leq\;q_N(\overline{\rho},1) <1$, $y\in (0,\,1]$,  и $\underset{y\rightarrow 
0}{\mathrm{lim}}\,q_N(\overline{\rho},y) =0$}.

\medskip

\noindent
Д\,о\,к\,а\,з\,а\,т\,е\,л\,ь\,с\,т\,в\,о\,.\  Возрастание функции 
$q_N(\overline{\rho},y)$ следует непосредственно из~(\ref{e7aga}) и 
утверж\-де\-ния~1. Доказательство неравенств в условии утверждения очевидно 
следует из~(\ref{e5aga}) и вида функции $g_n (\overline{\rho},y)$, $n\geq 0$. 
Для предела имеем:
\begin{multline*}
\underset{y\rightarrow 0}{\mathrm{lim}}\,q_N(\overline{\rho},y) 
=\underset{y\rightarrow 0}{\mathrm{lim}}\,\fr{g_{N- 1}(\overline{\rho},y)}{g_N(\overline{\rho},y)} = {}\\
{}= \underset{y\rightarrow 0}{\mathrm{lim}}\,\left (
g_{N-1}(\overline{\rho},y)\Bigg / \left ( 
\vphantom{\prod\limits_{v\in\Omega_u^+}}
g_{N-1}(\overline{\rho},y)\right.\right.+{}\\
{}+\left.\left.\sum\limits_{\overline{k}\in A_N}\prod\limits_{v\in\Omega_u^+} 
\fr{z_v(0,\rho_v,k_v,k^\prime_v,k^{\prime\prime}_v)}{y^N}\right )\right ) = {}\\
{}= \underset{y\rightarrow 0}{\mathrm{lim}}\,\left (
y^N g_{N-1}(\overline{\rho},y)\Bigg / 
\left ( 
\vphantom{\prod\limits_{v\in\Omega_u^+}}
y^N g_{N-1}(\overline{\rho},y)+{}\right.\right.\\
{}+\left.\left.\sum\limits_{\overline{k}\in A_N}
\prod\limits_{v\in\Omega_u^+} z_v(0,\rho_v,k_v,k_v^\prime , k_v^{\prime\prime}) 
\right ) \right )=0\,.
\end{multline*}
    
\medskip

\noindent
\textbf{Утверждение 3.} \textit{Производная функции~$q_N (\overline{\rho},y)$ по 
$y\in (0,\,1]$ удовлетворяет следующим соотношениям}:
\begin{align}
\underset{y\rightarrow 0}{\mathrm{lim}}\fr{\partial q_N(\overline{p},y)}
{\partial  y} &= \fr{\sum\limits_{\overline{k}\in A_{N-1}} 
p_{\overline{k}}(\overline{\rho},1)}{\sum\limits_{\overline{k}\in 
A_N}p_{\overline{k}}(\overline{\rho},1)}\,;\label{e10aga}\\
\fr{\partial q_N(\overline{\rho},y)}{\partial y}\Big |_{y=1}&<1\,.\label{e11aga}
\end{align}

\medskip

\noindent
Д\,о\,к\,а\,з\,а\,т\,е\,л\,ь\,с\,т\,в\,о\,.\ Проведя преобразования 
функции~$q_N(\overline{\rho},y)$, получим:
\begin{multline*}
\underset{y\rightarrow 0}{\mathrm{lim}}\fr{q_N(\overline{\rho},y)}{y} = {}\\
\!\!{}=
\underset{y\rightarrow 0}{\mathrm{lim}}
\fr{\sum\limits_{m=0}^{N-1}\sum\limits_{\overline{k}\in A_m}
\!\!\left (\prod\limits_{v\in\Omega_u^+}\!\! 
z_v(0,\rho_v,k_v,k_v^\prime , k_v^{\prime\prime})\right )\!\!\Bigg /\!\! y^m}
{y\sum\limits_{m=0}^{N}\sum\limits_{\overline{k}\in A_m}
\!\!\left(\prod\limits_{v\in\Omega_u^+}\!\! z_v\left (0,\rho_v,k_v,k_v^\prime , 
k_v^{\prime\prime}\right )\right )\!\!\Bigg /\!\!y^m} = \!\!\!
\end{multline*}
\begin{multline*}
\!\!\!\!\!\!{}=\underset{y\rightarrow 0}{\mathrm{lim}}\,
\fr{\sum\limits_{m=0}^{N-1}\sum\limits_{\overline{k}\in A_m}
y^{N-1-m}\prod\limits_{v\in\Omega_u^+} z_v(0,\rho_v,k_v,k_v^\prime , 
k_v^{\prime\prime})}{\sum\limits_{m=0}^{N}\sum\limits_{\overline{k}
\in A_m} y^{N-m}
\prod\limits_{v\in\Omega_u^+} z_v(0,\rho_v,k_v,k_v^\prime , 
k_v^{\prime\prime})}={}\!\\
{}=\fr{\sum\limits_{\overline{k}\in A_{N-1}} p_{\overline{k}}(\overline{\rho},1)}{ 
\sum\limits_{\overline{k}\in A_{N}} p_{\overline{k}}(\overline{\rho},1)}\,.
\end{multline*}
Очевидно, $\underset{y\rightarrow 0}{\mathrm{lim}} \,[d_N (\overline{\rho},y) -
d_{N-1} (\overline{\rho},y)]=1$, так как $\underset{y\rightarrow 
0}{\mathrm{lim}}\,d_n (\overline{\rho},y)=n$, $n>0$.

Следовательно, учитывая~(\ref{e7aga}), получаем~(\ref{e10aga}). 
Справедливость~(\ref{e11aga}) непосредственно следует из~(\ref{e7aga}) и 
утверждения~1.

\medskip

\noindent
\textbf{Утверждение 4.} \textit{Пусть $y^*\in (0,\,1]$~--- решение 
уравнения}~(\ref{e4aga}). \textit{Тогда}
\begin{equation*}
\fr{\partial q_N(\overline{\rho},y)}{\partial y}\Big |_{y=y^*}<1\,.
%\label{e12aga}
\end{equation*}

\medskip

\noindent
Д\,о\,к\,а\,з\,а\,т\,е\,л\,ь\,с\,т\,в\,о\,.\ \ Доказательство следует из~(\ref{e7aga}), 
так как $q_N(\overline{\rho},y^*)/y^* =1$.
\medskip

\noindent
\textbf{Утверждение 5.} \textit{Уравнение}~(\ref{e4aga}) \textit{имеет решение $y^*\in 
(0,\,1)$ тогда и только тогда, когда} 
\begin{equation}
\fr{\sum\limits_{\overline{k}\in A_{N-1}} p_{\overline{k}}(\overline{\rho},1)}{ 
\sum\limits_{\overline{k}\in A_{N}} p_{\overline{k}}(\overline{\rho},1)} >1\,.
\label{e13aga}
\end{equation}
\textit{Если уравнение}~(\ref{e4aga}) \textit{имеет решение $y^*\in (0,\,1)$, то оно 
единственное положительное решение}.
\medskip

\noindent
Д\,о\,к\,а\,з\,а\,т\,е\,л\,ь\,с\,т\,в\,о\,.\ Пусть выполняется 
неравенство~(\ref{e13aga}). Тогда, как следует из утверждения~3, 
$\underset{y\rightarrow 0}{\mathrm{lim}} (\partial q_N(\overline{\rho},y)/\partial y) 
>1$. Кроме того, как следует из утверждения~2, 
$\underset{y\rightarrow 0}{\mathrm{lim}} q_N(\overline{\rho},y)=0$. Тогда, так 
как $q_N(\overline{\rho},y)$~--- непрерывно-дифференцируемая функция по 
$y\in (0,\,1]$, существует значение $y^\prime \in (0,\,1)$ такое, что 
$q_N(\overline{\rho},y)>y$ для всех $y\in (0,\,y^\prime]$ (следует из теоремы о 
конечном приращении~\cite{11aga}). В то же время, согласно утверждению~2, 
$q_N(\overline{\rho},y)<y$ в окрестности точки $y=1$ (рис.~\ref{f1aga},\,\textit{а}). 
Следовательно, кривая $x=q_N(\overline{\rho},y)$ пересекает прямую $x=y$ 
хотя бы в одной точке $y=y^*\in (0,\,1)$, т.\,е.\ уравнение~(\ref{e4aga}) имеет 
хотя бы одно решение $y^*\in (0,\,1)$.

\begin{figure*}
\vspace*{1pt}
\begin{center}
\vspace*{1pt}
\mbox{%
\epsfxsize=158mm
\epsfbox{aga-1.eps}
}
\end{center}
\vspace*{-9pt}
\Caption{Примеры кривых $x=q_N(\overline{\rho},y)$ и $x=y$ (\textit{а})~при существовании решения 
уравнения~(\ref{e5aga}) и (\textit{б})~при выполнении условий~(17)
\label{f1aga}}
\vspace*{6pt}
\end{figure*}

Пусть уравнение~(\ref{e4aga}) имеет решение $y^*\in (0,\,1)$ и 
\begin{equation}
\fr{\sum\limits_{\overline{k}\in A_{N-1}}p_{\overline{k}}(\overline{\rho},1)}{ 
\sum\limits_{\overline{k}\in A_{N}}p_{\overline{k}}(\overline{\rho},1)}\leq 
1\,.\label{e14aga}
\end{equation}
Тогда из условий утверждений~2 и~3 следует, что 
уравнение~(\ref{e4aga}) в интервале $(0,\,1)$ имеет более одного решения, что 
может быть только при существовании решения $y^\prime \in (0,\,1)$ такого, 
что в окрестности точки $y=y^\prime$ выполняются неравенства: 
$q_N(\overline{\rho},y)<y$ при $y<y^\prime$ и $q_N(\overline{\rho},y)>y$ при 
$y>y^\prime$, где $y$ принадлежит указанной окрест\-ности точки~$y^\prime$ 
(рис.~\ref{f1aga},\,\textit{б}). Тогда в точке $y=y^\prime$ производная функции 
$q_N(\overline{\rho},y)$ по $y$ больше~1, что противоречит утверждению~4. 
Следовательно, неравенство~(\ref{e13aga}) справедливо.


Пусть уравнение~(\ref{e4aga}) имеет более одного положительного 
решения. Рассуждая точно так же, как и выше (в случае выполнения 
условий~(\ref{e14aga})), получим противоречие утверждению~4. 
Следовательно, утверждение~5 справедливо.
\medskip

\noindent
\textbf{Следствие.} \textit{Неравенства}
\begin{gather*}
\fr{\mu_v c_v (1-B_v)}{\Lambda_v}>1\,,\quad \fr{1-B_v}{\Lambda_v \tau_v B_v}>1\,,\\ 
\fr{1-B_v}{\Lambda_v t_v}>1\,,\ v\in\Omega_u^+\,,
\end{gather*}
\textit{являются необходимым условием существования решения 
уравнения}~(\ref{e4aga}).

\medskip
\noindent
Д\,о\,к\,а\,з\,а\,т\,е\,л\,ь\,с\,т\,в\,о\,.\ Пусть $\overline{k}_v$~--- это 
набор~$\overline{k}$, у которого $k_v=0$. Преобразовав левую 
часть~(\ref{e13aga}), получим

\noindent
\begin{multline*}
\fr{\sum\limits_{\overline{k}\in A_{N-1}} p_{\overline{k}} (\overline{\rho},1)}
{ \sum\limits_{\overline{k}\in A_{N}} 
 p_{\overline{k}}(\overline{\rho},1)} 
={}
\\
{}=
\fr{\sum\limits_{\overline{k}\in A_{N-1}}\prod\limits_{v\in \Omega_u^+} 
z_v\left(0,\rho_v,k_v,k_v^\prime , k_v^{\prime\prime}\right)}
{\sum\limits_{\overline{k}\in A_{N}}
\prod\limits_{v\in \Omega_u^+} z_v\left (0,\rho_v,k_v,k_v^\prime , k_v^{\prime\prime}\right )} \leq{}
\\
{}\leq
\left ( 
\vphantom{\prod\limits_{v^\prime\in\Omega_u^+\backslash v}}
\fr{\mu_v c_v(1-B_v)}{\Lambda_v}\right. \times{}\\
{}\times \sum\limits_{k_v=0}^{N-1}\sum\limits_{\overline{k}_v\in A_{N-1-k_v}} z_v\left(0,\rho_v,k_v+1,k_v^\prime , 
k_v^{\prime\prime}\right )\times{}\\
{}\times \left.\prod\limits_{v^\prime\in\Omega_u^+\backslash v} z_v^\prime 
\left(0,\rho_v,k_v,k_v^\prime , k_v^{\prime\prime}\right) \right)
\Bigg /{}\\
\Bigg / \left ( 
\vphantom{\prod\limits_{v^\prime\in\Omega_u^+\backslash v}}
\sum\limits_{k_v=0}^{N-1} \sum\limits_{\overline{k}_v\in A_{N-1-k_v}} z_v 
\left (0,\rho_v,k_v+1,k_v^\prime , 
k_v^{\prime\prime}\right )\right. \times{}\\
{}\times \prod\limits_{v^\prime\in\Omega_u^+\backslash v} 
z_{v^\prime}\left(0,\rho_v,k_v,k^\prime , k_v^{\prime\prime}\right)+{}\\
{}+
\sum\limits_{\overline{k}_v\in A_N} z_v\left (0,\rho_v, 0,k_v^\prime , 
k_v^{\prime\prime}\right)\times{}\\
\left.{}\times \prod\limits_{v^\prime\in\Omega_u^+\backslash v}z_{v^\prime} 
\left(0,\rho_v,k_v,k_v^\prime , k_v^{\prime\prime}\right )\right )\,.
\end{multline*}
Как следует из правой части последнего неравенства, если 
$\mu_v c_v (1-B_v)/\Lambda_v \leq 1$, то она меньше~1. Поэтому, чтобы 
выполнилось условие~(\ref{e13aga}), необходимо выполнение первого 
неравенства в условии следствия для каждого $v\in\Omega_u^+$. Точно так же 
доказывается необходимость выполнения второго и третьего неравенств в 
условии следствия.

    Пусть $y[n]$, $n\geq 0$, последовательность, полученная по формуле 
$y[n+1]=q_N(\overline{\rho},y[n])$, $y[0]=1$.

\medskip

\noindent
\textbf{Утверждение 6.} \textit{Пусть $y^*\in (0,\,1)$~--- решение 
уравнения}~(8). \textit{Тогда последовательность $y[n]$, $n\geq 0$, сходится 
к решению~$y^*$}.

\medskip

\noindent
Д\,о\,к\,а\,з\,а\,т\,е\,л\,ь\,с\,т\,в\,о\,.\ Отметим, что $y[1]<y[0]$ (это следует из 
утверждения~2, так как $y[0]=1$). Пусть для некоторого $n>1$ выполняется 
условие $y[n]<y[n-1]$. Тогда, как следует из утверждения~2, указанное условие 
выполняется и для $n+1$, т.\,е.\ по индукции следует, что последовательность 
$y[n]$, $n\geq 0$, монотонно убывает. 

    Пусть для некоторого $n>0$ $y[n]>y^*$ (существование такого $n$ 
следует из равенства $y[0]=1$). Тогда, как следует из утверждения~2, 
$y[n+1]\;=$\linebreak $=\;q_N(\overline{\rho},y[n])>q_N(\overline{\rho},y^*) =y^*$, т.\,е.\ 
последовательность ограничена снизу величиной~$y^*$. Значит, существует 
$\underset{n\rightarrow \infty}{\mathrm{lim}}\,y[n]=y^0\geq y^*$. Так как 
$q_n(\overline{\rho},y)$~--- непрерывная по~$y$ функция, то можно написать 
$\underset{n\rightarrow 
\infty}{\mathrm{lim}}\,q_N(\overline{\rho},y[n])=q_N(\overline{\rho},y^0)=y^0$, 
т.\,е.\ $y^0$~--- решение уравнения~(\ref{e4aga}). Из единственности 
положительного решения уравнения~(\ref{e4aga}) получаем $y^0=y^*$.

    Пусть в узле используется схема полного разделения буферной памяти. 
Тогда для интенсив\-ностей~$\Lambda_v^*$, $v\in\Omega_u^+$, справедливы 
соотношения:
$$
\Lambda_v^* = \fr{\Lambda_v}{1-\pi_v}\,,
$$
где $v\in\Omega_u^+$.


Фиксируем произвольную линию сети~$v$. Пусть $\overline{k}_v = (k_v, 
k_v^\prime, k_v^{\prime\prime})$~--- состояние буферной памяти линии~$v$; 
$k_v$, $k_v^\prime$, $k_v^{\prime\prime}$ определены выше. Тогда с 
учетом введенных ранее предположений и обозначений для вероятности 
блокировки линии справедлива формула~\cite{4aga}:
\begin{equation}
\pi_v = \fr{1}{G_{N_v}}\sum\limits_{k_v=N_v} 
z_v(\pi_v,\rho_v,\overline{k}_v)\,,
\label{e15aga}
\end{equation}
где 
\begin{multline*}
z_v(\pi_v, \rho_v, \overline{k}_v)={}\\
{}=
\begin{cases}
\fr{\rho_v^{\prime * k_v^\prime}}{k_v^\prime !}\,
 \fr{\rho_v^{\prime\prime * k_v^{\prime\prime}}}{k_v^{\prime\prime}!}\,
 \fr{\rho_v^{*k_v}}{k_v !} & \mbox{при}\ k_v<c_v\,,\\
 \fr{\rho_v^{\prime *k_v^\prime}}{k_v^{\prime }! }
 \fr{\rho_v^{\prime\prime * k_v^{\prime\prime}}}{k_v^{\prime\prime}!}
\fr{\rho_v^{*k_v}}{c_v !c_v^{k_v-c_v}} & \mbox{при}\ k_v\geq c_v\,;
\end{cases}
\end{multline*}
\begin{align*}
G_{N_v} &= \sum\limits_{m=0}^{N_v} z_v (\pi_v ,\rho_v , \overline{k}_v)\,;\\ 
\rho_v^*&=\fr{\rho_v}{1-\pi_v}\,;
\end{align*}
$\rho_v$, $\rho_v^{\prime *}$, 
$\rho_v^{\prime\prime *}$, $v\in\Omega_u^+$ определены выше.
    
Пусть $y_v=1-\pi_v$, а $q_{N_v} (\rho_v, y_v)$~--- выражение в правой 
части~(\ref{e15aga}). Тогда из равенств~(\ref{e15aga}), взяв~$y_v$ в качестве 
неизвестной переменной, получим систему независимых уравнений:
\begin{equation}
y_v = q_{N_v}(\rho_v, y_v)\,, \quad v\in \Omega_u^+\,.
\label{e16aga}
\end{equation}
    
    Заметим, что для фиксированной $v$ и заданных параметров $\Lambda_v$, 
$\mu_v$, $\tau_v$, $t_v$, $N_v$, $v\in\Omega_u^+$, уравнение в~(\ref{e16aga}) 
является частным случаем уравнения~(\ref{e4aga}) и совпадает с последним, 
когда число исходящих линий из узла равно~1. Следовательно, для схемы 
полного разделения памяти также справедливы все приведенные выше 
утверждения~1--6 и следствие. Заметим, что неравенство~(\ref{e13aga}) в 
условии утверждения~5 при $B_v=0$ и $t_v=0$ преобразуется в неравенство 
$\Lambda_v / (\mu_v c_v) >1$, $v\in\Omega_u^+$. Последовательность 
$\overline{y}[n]$, $n\geq 0$, в утверждении~6 определяется по формуле:
    \begin{gather*}
    \overline{y}[n] =\{y_v[n],\ v\in\Omega_u^+\}\,,\
    y_v[n+1]=q_{N_v} (\rho_v,\,y_v[n])\,,\\
    y_v[0] =1\,,\quad n\geq 0\,,\quad v\in \Omega_u^+\,.
    \end{gather*}


\section{Алгоритм расчета} %4

    Для вычисления интенсивностей потоков и вероятностей блокировок в 
узле предлагается следующий алгоритм, описывающий изложенную выше 
итерационную процедуру. Введем обозначения:
$y_u^v$~--- вероятность блокировки узла для заявок, направляемых на 
линию~$v$,
\begin{gather*}
y_u^v  = 
\begin{cases}
y_u & \mbox{для}\ v\in\Omega_u^+\ \mbox{при}\\
&\mbox{полнодоступной схеме},\\
y_v & \mbox{при схеме полного распределения}\\
&\mbox{памяти};
\end{cases}
\\
q_N^v(\overline{\rho}_u^{-v}, y_u^v)  = 
\begin{cases}
q_N(\overline{\rho},y) & \mbox{для}\ v\in\Omega_u^+\ \mbox{при пол-}\\ 
&\mbox{нодоступной схеме},\\
q_{N_v}(\rho_v, y_v) & \mbox{при схеме полного}\\
&\mbox{распределения}\\ 
&\mbox{памяти},  v\in\Omega_u^+\,.
\end{cases}
\end{gather*}



Тогда уравнения~(\ref{e4aga}) и~(\ref{e16aga}) записываются в виде:
$$
y_u^v = q_N^v (\overline{\rho}^v_u, y^v_u)\,,\quad v\in \Omega_u^+\,.
$$
Для значений, вычисляемых на $k$-м шаге алгоритма, к 
обозначениям соответствующих параметров приписывается знак~$[k]$.
\pagebreak

\textbf{Шаг 0.} 
\begin{enumerate}[1.]
\item  \textit{Инициализация}. Вычисление начальных значений 
параметров~$\rho_v$, $v\in\Omega_u^+$: $\Lambda_v[0]=\Lambda_v$, 
$\rho_v[0]=\Lambda_v[0]/(\mu_v(1-B_v))$, $y_u^v[0]=1$.
\item \textit{Проверка условий существования решения}. Если для некоторой 
линии $v\in\Omega_u^+$ выполняется хотя бы одно неравенство $(c_v\mu_v(1-
B_v))/\Lambda_v[0]\;\leq$\linebreak $\leq\;1$, или $(1-B_v)/(\Lambda_v\tau_v B_v) \leq 1$, или 
$(t_v(1\;-$\linebreak $-\;B_v))/\Lambda_v[0] \leq 1$, то алгоритм заканчивает работу с 
результатом <<нагрузка не реализуема>>. Если в узле используется 
полнодоступная схема и $(c_v\mu_v(1-B_v))/\Lambda_v[0] > 1$, $(1-
B_v)/(\Lambda_v\tau_v B_v)\;>$\linebreak $>\;1$, $(t_v(1-B_v))/\Lambda_v[0] > 1$ для всех 
$v\in\Omega_u^+$, то проверяется условие~(\ref{e13aga}) для $\Lambda_v =
\Lambda_v[0]$, $v\in\Omega_u^+$, и при невыполнении этого условия алгоритм 
заканчивает работу с результатом <<нагрузка не реализуема>>.
\end{enumerate}

    При вычислении левой части неравенства~(\ref{e13aga}) рекомендуется 
использовать метод свертки Базена (см.~\cite{12aga}), позволяющий 
производить рекуррентные вычисления (подробно этот метод описан также 
в~[1, 3--6]).



\medskip
\textbf{Шаг~$k$} ($k > 0$):
\begin{enumerate}[1.]
\item \textit{Вычисление вероятностей блокировок}. Используя значения 
параметров $\overline{\rho}_u^v[k-1]$, $y_u^v[k-1]$, $v\in\Omega_u^+$, 
вычисление с помощью формул~(1)--(7) значений 
вероятностей $y[k]=1- \pi [k]$~--- в случае полнодоступной памяти, или 
$y_v[k]=1- \pi_v[k]$, $v\in\Omega_u^+$, с помощью формул~(\ref{e15aga})~--- в 
случае полного разделения памяти. При вычислении этих значений 
рекомендуется использовать метод свертки Базена.
    \item \textit{Проверка условий останова алгоритма}. Если хотя бы для 
одной $v\in\Omega_u^+$ для заданного значения точности   выполняется 
условие
$$
\fr{\vert \Lambda_v^*[k]-\Lambda_v^*[k-1]\vert}{\Lambda_v^*[k]}> \varepsilon\,,
$$
то вычисление параметров $\overline{\rho}_u^v[k]$, $v\in\Omega_u^+$, и 
переход к шагу~$k$, положив $k$ равным $k+1$, иначе алгоритм завершает 
работу. 
\end{enumerate}

    По завершении алгоритма либо выявится, что нагрузка в системе не 
реализуема, либо будут вычислены интенсивности потоков, поступающих на 
линии узла, и стационарные вероятности блокировок для заявок каждого типа. 
    
\section{Примеры расчета}

    Для проверки точности вычисления результатов с помощью 
предложенного выше алгоритма и приемлемости введенных предположений 
были проведены вычислительные эксперименты с использованием 
аналитических и имитационных моделей. Во всех рассмотренных ниже 
примерах потоки внешних заявок считаются пуассоновскими. 
В~табл.~1 приведены значения вероятности блокировок вновь 
поступивших извне заявок, полученные на основании точной формулы, 
приведенной в~\cite{4aga} для СМО типа $M\vert M\vert 1\vert 0$ с повторными 
заявками при экспоненциальном распределении интервала времени между 
повторными попытками (первая строка таблицы), алгоритма из подраздела~5 
настоящей статьи (вторая строка) и имитационной модели при постоянном 
интервале времени между повторными попытками, равном~10 (третья строка). 
Расчет табл.~1 проведен для узла с одной исходящей одноканальной 
линией при интенсивности первичного потока $\Lambda =1$ и емкости 
накопителя $N_v=1$. Таблицы~2 и~3 вычислены с помощью 
алгоритма из подраздела~5 и имитационной модели соответственно при одной 
исходящей линии с числом каналов~10.


    В табл.~\ref{t4aga} и~\ref{t5aga} приведены значения вероятности 
блокировки узла с тремя исходящими линиями канальной емкости~10 каждая 
при $\mu_v =0{,}2$, $v\in\Omega_u^+$,  вычисленные с помощью алгоритма из 
подраздела~5 и имитационной модели с интервалом повторной попытки, 
равным~10, соответственно. В табл.~\ref{t4aga} и~\ref{t5aga} знак <<--->> в 
ячейках означает, что предложенная нагрузка $\Lambda_v$, $v\in\Omega_u^+$, 
не реализуема.



В табл.~\ref{t6aga} отражены вероятности блокировки такого же узла с 
накопителем $N = 35$ при экспоненциальном распределении интервала 
времени между повторными попытками со средним значением~$\tau$. 


Результаты вычислительного эксперимента показывают, что с  увеличением 
длины интервала между повторными попытками  вероятность блокировки 
увеличивается и приближается к значению,\linebreak
вычисленному с помощью 
алгоритма из подраздела~5 (см.\ табл.~\ref{t4aga} и~\ref{t6aga}), т.\,е.\ при 
пуассоновском внешнем потоке заявок предположение, что суммарный 
входной поток заявок  является пуассоновским, вполне приемлемо для 
предварительного анализа характеристик узла (например, при  $\tau c_v\mu_v > 
10$). Как показывают табл.~1--3, вероятность блокировки 
узла существенно зависит от\linebreak 

\vspace*{6pt}
\noindent
%\begin{table*}\small %tabl1
{\small
{{\tablename~1}\ \ \small{Вероятности блокировок при одной исходящей одноканальной линии}}
%\label{t1aga}}
\vspace*{-3pt}

\begin{center}
{\tabcolsep=7.3pt
\begin{tabular}{|c|c|c|c|c|c|}
\hline
&\multicolumn{5}{c|}{$\mu$}\\
\cline{2-6}
\multicolumn{1}{|c|}{\raisebox{4pt}[0pt][0pt]{№}}&1{,}1&1{,}2&2&3&4\\
\hline
1&0,9091&0,8333&0,5000&0,3333&0,2500\\
2&0,9091&0,8333&0,5000&0,3333&0,2500\\
3&0,8867&0,8452&0,4944&0,3167&0,2396\\
\hline
\end{tabular}}
\end{center}
%\vspace*{-6pt}
%\end{table*}
}
%\bigskip
%\medskip
\addtocounter{table}{1}
\pagebreak

\end{multicols}

\renewcommand{\figurename}{\protect\bf Таблица}
%\renewcommand{\tablename}{\protect\bf Рис.}
\begin{figure*}
{\small
\begin{minipage}[t]{76mm}
%\begin{table*}\small %tabl2
\begin{center}
\Caption{Вероятности блокировок при одной исходящей многоканальной линии ($\varepsilon 
=0{,}0001$)
\label{t2aga}}
\vspace*{2ex}

\tabcolsep=6.5pt
\begin{tabular}{|c|c|c|c|c|c|}
\hline
&\multicolumn{5}{c|}{$\mu$}\\
\cline{2-6}
\multicolumn{1}{|c|}{\raisebox{4pt}[0pt][0pt]{$N$}}&0{,}11&0{,}12&0{,}2&0{,}3&0{,}4\\
\hline
10&0,4845&0,2935&0,0204&0,0017&0,0002\\
15&0,1181&0,0545&0,0006&0,0000&0,0000\\
20&0,0489&0,0167&0,0000&0,0000&0,0000\\
\hline
\end{tabular}
\end{center}
%\end{table*}
\end{minipage}
\hfill
\begin{minipage}[t]{76mm}
%\begin{table*}\small %tabl3
\begin{center}
\Caption{Вероятности блокировок при одной исходящей линии
\label{t3aga}}
\vspace*{2ex}

\tabcolsep=6.5pt
\begin{tabular}{|c|c|c|c|c|c|}
\hline
&\multicolumn{5}{c|}{$\mu_v$}\\
\cline{2-6}
\multicolumn{1}{|c|}{\raisebox{4pt}[0pt][0pt]{$N$}}&0{,}11&0{,}12&0{,}2&0{,}3&0{,}4\\
\hline
10&0,5247&0,3238&0,0219&0,0019&0,0001\\
15&0,1726&0,0912&0,0004&0,0001&0,0000\\
20&0,1180&0,0371&0,0000&0,0000&0,0000\\
\hline
\end{tabular}
\end{center}
%\end{table*}
\end{minipage}
}
\vspace*{6pt}
\end{figure*}

\renewcommand{\figurename}{\protect\bf Рис.}
\renewcommand{\tablename}{\protect\bf Таблица}
\addtocounter{table}{2}

\begin{table}\small %tabl4
\begin{center}
\parbox{400pt}{\Caption{Вероятности блокировок при трех исходящих линиях, вычисленные алгоритмом из 
подраздела~5 ($\varepsilon =0{,}0001$)
\label{t4aga}}
}

\vspace*{2ex}

\tabcolsep=8pt
\begin{tabular}{|c|c|c|c|c|c|c|c|c|c|}
\hline
&\multicolumn{9}{c|}{$\Lambda_v$}\\
\cline{2-10}
\multicolumn{1}{|c|}{\raisebox{4pt}[0pt][0pt]{$N$}}&1&1{,}1&1{,}2&1{,}3&1{,}4&1{,}5&1{,}6&1{,}7&1{,}8\\
\hline
20&0,0677&0,1423&0,2975&0,7653&---&---&---&---&---\\
25&0,0065&0,0173&0,0394&0,0827&0.1690&0.3827&---&---&---\\
30&0,0005&0,0019&0,0059&0,0155&0.0361&0.0790&0.1792&0,7259&---\\
35&0,0000&0,0002&0,0008&0,0030&0,0089&0,0234&0,0574&0,1505&---\\
40&0,0000&0,0000&0,0001&0,0005&0,0022&0,0075&0,0220&0,0617&0,2449\\
\hline
\end{tabular}
\end{center}
%\end{table}
\vspace*{6pt}
%\begin{table}\small %tabl5
\begin{center}
\parbox{400pt}{\Caption{Вероятности блокировок при трех исходящих линиях, вычисленные с помощью 
имитационной модели
\label{t5aga}}
}

\vspace*{2ex}

\tabcolsep=8pt
\begin{tabular}{|c|c|c|c|c|c|c|c|c|c|}
\hline
&\multicolumn{9}{c|}{$\Lambda_v$}\\
\cline{2-10}
\multicolumn{1}{|c|}{\raisebox{4pt}[0pt][0pt]{$N$}}&1&1{,}1&1{,}2&1{,}3&1{,}4&1{,}5&1{,}6&1{,}7&1{,}8\\
\hline
20&0,0786&0,1695&0,3549&0,7056&---&---&---&---&---\\
25&0,0069&0,0190&0,0441&0,0998&0,2266&0,4583&---&---&---\\
30&0,0007&0,0024&0,0075&0,0184&0,0462&0,1025&0,2380&0,6931&---\\
35&0,0000&0,0003&0,0007&0,0040&0,0129&0,0307&0,0890&0,2981&---\\
40&0,0000&0,0000&0,0000&0,0011&0,0041&0,0095&0,0346&0,0790&0,3179\\
\hline
\end{tabular}
\end{center}
%\end{table}
\vspace*{6pt}
%\begin{table}\small %tabl6
\begin{center}
\parbox{356pt}{\Caption{Зависимость вероятности блокировки при трех исходящих линиях, вы\-чис\-лен\-ные с 
помощью имитационной модели со случайным интервалом между повторными попытками
\label{t6aga}}
}

\vspace*{2ex}

\tabcolsep=8pt
\begin{tabular}{|c|c|c|c|c|c|c|c|c|}
\hline
&\multicolumn{8}{c|}{$\Lambda_v$}\\
\cline{2-9}
\multicolumn{1}{|c|}{\raisebox{6pt}[0pt][0pt]{$\tau$}}&1&1{,}1&1{,}2&1{,}3&1{,}4&1{,}5&1{,}6&1{,}7\\
\hline
\hphantom{9}1&0.0001&0,0001&0,0017&0,0063&0,0210&0,0733&0,1996&0,4222\\
\hphantom{9}5&0.0000&0,0002&0,0016&0,0036&0,0446&0,0159&0,1360&0,3273\\
10&0.0000&0,0002&0,0011&0,0036&0,0101&0,0430&0,0818&0,2774\\
20&0.0000&0,0003&0,0007&0,0029&0,0089&0,0257&0,0863&0,2045\\
     \hline
\end{tabular}
\end{center}
\end{table}


\begin{multicols}{2}


\noindent
числа каналов в линии при равной суммарной 
производительности. Кроме того, как видно из табл.~\ref{t5aga} и~\ref{t6aga}, 
вероятность блокировки в большей степени зависит от среднего значения 
длины интервала между повторными попытками передачи, чем от закона 
распределения длины интервала. Таким образом, предложенный в работе 
алгоритм позволяет вы\-чис\-лить с достаточной точностью вероятность 
блокировки узла, интенсивности повторных передач и предельную величину 
реализуемой нагрузки. Отметим, что полученные в данной статье результаты 
могут быть использованы для расчета нагрузок в телекоммуникационной сети с 
повторами заявок в предыдущем узле или из источника. 


{\small\frenchspacing
{%\baselineskip=10.8pt
\addcontentsline{toc}{section}{Литература}
\begin{thebibliography}{99}    
\bibitem{1aga}
\Au{Kamoun~F., Kleinrock~L.}
Analysis of shared finite storage in a computer networks node environment under 
general traffic conditions~// IEEE Trans. on Commun., 1980. Vol.~28. No.\,7. 
P.~992--1003.

\bibitem{6aga} %2
\Au{Агаларов~Я.\,М., Шоргин~С.\,Я.}
Рекуррентный метод вычисления параметров сетей связи~// Техника средств 
связи, 1986. Сер. <<Системы связи>>. Вып.~6. С.~42--46.

\bibitem{3aga}
\Au{Башарин Г.\,П., Бочаров~П.\,П., Коган~Я.\,А.}
Анализ очередей в вычислительных сетях.~--- М.: Наука, 1989. 

\bibitem{4aga}
\Au{Бочаров~П.\,П., Печинкин~А.\,В.}
Теория массового обслуживания.~--- М.: Изд-во РУДН, 1995. 

\bibitem{5aga}
\Au{Вишневский~В.\,М.} 
Теоретические основы проектирования компьютерных сетей.~--- М.: 
Техносфера, 2003. 

\bibitem{2aga} %6
\Au{Башарин Г.\,П.}
Лекции по математической теории телетрафика.~--- М.: Изд-во РУДН, 2007. 

\bibitem{7aga}
\Au{Таранцев~А.\,А.}
Инженерные методы теории массового обслуживания.~--- М.: Наука, 2007.

\bibitem{9aga} %8
\Au{D'Apice~C., De~Simone~T., Manzo~R., Rizelian~G.}
$M\vert G\vert 1\vert r$ retrial queueing system with priority service of primary 
customers and a customers-searching server~// Distributed Computer and 
Communication Networks. Stochastic Modelling and Optimization.~--- М.: 
Техносфера, 2003. P.~106--117.

\bibitem{8aga} %9
\Au{Klimenok~V.\,I., Kim~C.\,S.}
$BM\!AP$/$PH$/1 retrial system operating in random environment~// Proceedings of 
the 5th St.-Petersburg Workshop on Simulation, St.-Petersburg, June~26\,--\,July~2, 
2005.~--- St.-Petersburg: NII Chemistry St.-Petersburg University Publs., 
2005. P.~367--372.   

\bibitem{10aga}
\Au{Krishnamoorthy~A., Babu~S.}
$M\!AP\vert (PH,PH)/c$ retrial queue with selegeneration of priorities 
and non-preemptive service~// Proceedings of the 14th International Conference on 
Analytical and Stochastic Modeling Techniques and Applications, June~4--6, 
2007. Prague, Czech Republic.~--- Sbr.-Dudweiler: Digitaldruck Pirrot GmbH, 
2007. P.~70--74.

\bibitem{11aga}
\Au{Корн~Г., Корн~Т.}
Справочник по математике.~--- М.: Наука, 1974.

\label{end\stat}


\bibitem{12aga}
\Au{Buzen~J.\,P.}
Computational algorithm for closed queuing networks with exponential servers~// 
Communications ACM, 1973. Vol.~16. No.\,9. P.~527--531.
 \end{thebibliography}
}
}
\end{multicols}
 
 
   %
%\newcommand {\ff}{{\mathcal F}}
\newcommand {\ebd}{\triangleq}
\newcommand{\me}[2]{\mathbf{E}_{ #1 }\left\{ \mathop{#2} \right\} }



\def\stat{borisov}

\def\tit{ФИЛЬТРАЦИЯ СОСТОЯНИЙ МАРКОВСКИХ СКАЧКООБРАЗНЫХ ПРОЦЕССОВ 
ПО~ДИСКРЕТИЗОВАННЫМ НАБЛЮДЕНИЯМ$^*$}

\def\titkol{Фильтрация состояний марковских скачкообразных процессов 
по~дискретизованным наблюдениям}

\def\aut{А.\,В.~Борисов$^1$}

\def\autkol{А.\,В.~Борисов}

\titel{\tit}{\aut}{\autkol}{\titkol}

\index{Борисов А.\,В.}
\index{Borisov A.\,A.}




{\renewcommand{\thefootnote}{\fnsymbol{footnote}} \footnotetext[1]
{Работа выполнена при частичной поддержке РФФИ (проект 16-07-00677).}}


\renewcommand{\thefootnote}{\arabic{footnote}}
\footnotetext[1]{Институт проблем информатики Федерального исследовательского центра <<Информатика 
и~управление>> Российской академии наук,
\mbox{aborisov@frccsc.ru}}

%\vspace*{8pt}



\Abst{Статья посвящена решению задачи оптимальной 
фильтрации состояний однородного марковского скачкообразного процесса (МСП). 
Наблюдения представляют собой приращения случайных процессов~--- интегральных 
преобразований состояний, зашумленные винеровскими процессами, интенсивность 
которых также зависит от оцениваемого состояния. Оптимальная оценка в~моменты 
получения нового наблюдения вычисляется как функция предыдущей оценки и~новых 
наблюдений, а~между моментами наблюдений~--- простейшим прогнозом в~силу системы 
уравнений Колмогорова. Рекуррентная формула пересчета ресурсозатратна, так как 
содержит  интегралы~--- мас\-штаб\-но-сдви\-го\-вые смеси многомерных гауссиан, 
где в~качестве смешивающих выступают распределения времени пребывания 
состояния в~каждом из возможных значений. Предложены более простые аппроксимации, 
основанные на предположении об ограниченности числа скачков состояния за время между 
наблюдениями. Получены универсальные локальная и~глобальная характеристики точности 
аппроксимаций, зависящие от па\-ра\-мет\-ров оцениваемого процесса, величины 
временн$\acute{\mbox{о}}$го шага  между наблюдениями и~максимального числа учитываемых скачков.}

\KW{марковский скачкообразный процесс; оптимальная фильтрация; мультипликативные 
шумы в~наблюдениях; стохастическое дифференциальное уравнение; численная аппроксимация}

\DOI{10.14357/19922264180316}
  
%\vspace*{4pt}


\vskip 10pt plus 9pt minus 6pt

\thispagestyle{headings}

\begin{multicols}{2}

\label{st\stat}



 \section{Введение}
 
 Фильтр Вонэма~\cite{Won_65}~--- один из редких удачных случаев, когда 
 оценка оптимальной фильтрации состо\-яния стохастической системы наблюдения 
 выражается в~виде решения некоторой замк\-ну\-той\linebreak конечномерной сис\-те\-мы 
 стохастических дифференциальных уравнений. 
 
 Алгоритм данного фильт\-ра 
 позволяет вычислить оценку фильт\-ра\-ции со\-сто\-яния \textit{марковского скачкообразного 
 процесса} с~\mbox{конечным} множеством состояний по наблюдениям в~присутствии 
 аддитивных винеровских шумов. Теоретически оптимальная оценка со\-сто\-яния~--- 
 его условное распределение в~текущий момент времени~--- 
 обладает очевидными свойствами неотрицательности и~нормировки. 
 При чис\-лен\-ной реализации данного фильтра классическим методом 
 Эй\-ле\-ра--Ма\-ру\-ямы~\cite{KP_92} данные свойства могут не сохраняться и~процедура 
 вы\-чис\-ле\-ния становится неустойчивой.  В~связи с~этим обстоятельством разрабатывались 
 другие алгоритмы чис\-лен\-но\-го решения уравнения фильтра Вонэма, обладающие 
 требуемыми свойствами устойчивости (см.~\cite{YZL_04, PR_10} и~библиографию в~них). 
 В~час\-ти этих работ доказана лишь слабая сходимость пред\-ла\-га\-емых аппроксимационных 
 схем к~оценке фильт\-ра Вонэма, в~то время как ка\-кая-ли\-бо 
 характеризация точ\-ности этих приближений отсутствует.
 
 В~\cite{B_18} было представлено распространение фильт\-ра Вонэма на случай 
 наблюдений с~мультипликативными шумами. При этом уравнение обобщенного 
 фильт\-ра содержит в~правой части квадратическую характеристику шумов в~наблюдениях. 
 Данный процесс на практике никогда не наблюдается непосредственно, а~является лишь 
 некоторым нелинейным интегральным преобразованием наблюдений. Очевидно, что 
 имеющиеся в~настоящий момент времени алгоритмы приближенного вычисления оценки 
 фильтрации Вонэма для данной системы не подходят. 
 
 Целью предлагаемой работы является ис\-поль\-зование результатов оптимальной 
 фильтрации со\-стояний сис\-тем с~дискретным временем для аппроксимации решения 
 аналогичной задачи для\linebreak стохастических дифференциальных сис\-тем. 
 
 Статья организована следующим образом. Раздел~2 содержит формальную постановку 
 задачи фильт\-ра\-ции со\-сто\-яний однородного МСП с~конечным множеством со\-сто\-яний 
 по наблюдениям, полученным путем временн$\acute{\mbox{о}}$й дискретизации процессов с~непрерывным 
 временем~--- интегральных преобразований со\-сто\-яния сис\-те\-мы в~присутствии 
 мультипликативных винеровских шумов.\linebreak
  В~разд.~3 пред\-став\-ле\-но решение поставленной 
 задачи фильт\-ра\-ции: пересчет оценок со\-сто\-яний в~момент получения новых 
 дискретизованных наблюдений выполняется в~соответствии с~некоторыми\linebreak 
 рекуррентными интегральными соотношениями, в~то время как между 
 моментами наблюдений оценка корректируется в~соответствии с~прогнозом в~силу 
 сис\-те\-мы уравнений Колмогорова. Вы\-чис\-ли\-тель\-ная слож\-ность 
 упомянутых выше интегральных\linebreak 
 соотношений связана с~тем, что в~расчет принимается воз\-мож\-ность того, что между 
 моментами наблюдений оцениваемое со\-сто\-яние может совершить произвольное чис\-ло 
 скачков. В~разд.~4 пред\-став\-лен более простой алгоритм приближенного вы\-чис\-ле\-ния 
 оценки фильт\-ра\-ции, основанный на ограничении возможного числа учитываемых скачков 
 МСП. Доказана тео\-ре\-ма, опре\-де\-ля\-ющая как\linebreak
  локальную (одношаговую), так и~глобальную 
 (многошаговую) характеристики точ\-ности предложенного при\-бли\-же\-ния~--- 
 $\ell_1$-нор\-мы ошибки аппроксимации. Полученные характеристики являются\linebreak 
 универсальными, т.\,е.\ не асимптотическими по шагу дискретизации, и~зависят от характеристик 
 самого МСП, %\linebreak
  шага временн$\acute{\mbox{о}}$й дискретизации и~чис\-ла
  скачков со\-сто\-яния, учи\-ты\-ва\-емых 
 на шаге. Об\-суж\-де\-ние результатов и~заключительные комментарии пред\-став\-ле\-ны 
 в~разд.~5.
 
 \section{Постановка задачи фильтрации}
 
 На полном вероятностном пространстве с~фильт\-ра\-цией 
 $(\Omega,\mathcal{F},\mathcal{P},\{\mathcal{F}_{t}\}_{t \geqslant 0})$ рассматривается система наблюдений
\begin{equation}
 \left.
 \begin{array}{rl}
 \displaystyle X_t &=X_0 +  \displaystyle
 \int\limits_0^t \Lambda^{\top}X_{s}\,ds + \mu_s\,;  \\[6pt]
 \displaystyle Y_k &=  \displaystyle\int\limits_{t_{k-1}}^{t_k}fX_s\,ds+
 \int\limits_{t_{k-1}}^{t_k} 
 \sum\limits_{n=1}^NX_s^ng_n \,dW_s, \\[6pt]
 &\hspace*{10mm}\{t_k\}_{k \geqslant 0}: \; 0 = t_0 < t_1 < t_2\cdots,
 \end{array}
 \right\}
 \label{eq:obsys_1}
 \end{equation}
 где
  \begin{itemize}
  \item
  $X_t \ebd \mathrm{col}\left(X_t^1,\ldots,X_t^N\right) \hm\in \mathbb{S}^N$~--- 
  ненаблюда\-емое состояние системы, являющееся однородным МСП с~конечным 
  множеством состояний $ \mathbb{S}^N \ebd$\linebreak $\ebd \{e_1,\ldots,e_N\}$ ($\mathbb{S}^N$~--- 
  множество единичных векторов евклидова пространства~$\mathbb{R}^N$), 
  матрицей интенсивностей переходов~$\Lambda$ и~начальным распределением~$\pi$;
  \item
  $\mu_t \ebd \mathrm{col}\left(
  \mu_t^1,\ldots,\mu_t^N\right)\hm\in \mathbb{R}^N$~--- 
  ${\mathcal{F}}_t$-со\-гла\-со\-ван\-ный мартингал;
  \item
  $\{Y_k\}_{k \in \mathbb{N}}:\;  Y_k \ebd \mathrm{col}\left(Y_k^1,\ldots,Y_k^M\right) 
  \hm\in \mathbb{R}^M$~--- последовательность дискретизованных наблюдений, 
  доступных в~известные неслучайные  моменты времени~$\{t_k\}_{k \in \mathbb{N}}$,
в~которых $W_t \ebd$\linebreak $\ebd \mathrm{col}\left(W_t^1,\ldots,W_t^M\right) \hm\in \mathbb{R}^M$
 является ${\mathcal{F}}_t$-со\-гла\-со\-ван\-ным стандартным винеровским процессом, 
 определяющим шумы в~наблюдениях,\linebreak  $f$~--- $(M \times N)$-мер\-ная 
 мат\-ри\-ца плана наблюдений, а~набор мат\-риц~$\{g_n\}_{n=\overline{1,N}}$ 
 характеризует интенсивности шумов в~зависимости от текущего состояния~$X_t$.
  \end{itemize}
  
  Введем также в~рассмотрение неубывающие семейства $\sigma$-ал\-гебр 
  $\mathcal{O}_k \ebd \sigma\{ Y_{\ell}: \; 1 \hm\leqslant \ell \hm\leqslant k\}$ 
  и~$\mathcal{O}_t \ebd  \mathcal{O}_{k(t)}$, где 
  $k(t) \ebd \sum\nolimits_{j \in \mathbb{N}}\mathbf{I}(t-t_{j})$; 
  $\mathcal{O}_0 \ebd \{\varnothing,\; \Omega\}$.
  
   \textit{Задача оптимальной фильтрации состояния~$X$ по наблюдениям~$Y$} 
   заключается в~нахождении \textit{условного математического ожидания} (УМО)
  \begin{equation*}
  \widehat{X}_t \ebd {\sf E}\left\{X_t|\mathcal{O}_{t} \right\}\,.
 % \label{eq:fest_1}
  \end{equation*}
  
  Относительно системы~(\ref{eq:obsys_1})  сделаны следующие предположения:
   \begin{itemize}
 \item[(а)]
 ${\mathcal{F}}_t \equiv {\mathcal{F}}_{t}^X \bigvee 
 {\mathcal{F}}_{t}^W $ для любого $t \hm\geqslant 0$;
 \item[(б)]
 шумы в~наблюдениях равномерно невырожденные, т.\,е.\
  $g_ng_n^{\top} \hm\geqslant \alpha I \hm> 0$ для всех $n\hm=\overline{1,N}$ 
  и~некоторого $\alpha\hm>0$.
% \item
 % Верно неравенство
  %\begin{equation}
  %\min_{1\leqslant k \leqslant N}|\lambda_{kk}| > 0.
  %\label{eq:ineq_0}
  % \end{equation}
 %\item
 %Для любого $t \geqslant 0$ все компоненты вектора $p_t \ebd \me{}{X_t}$ строго %положительны. 
 \end{itemize} 

 \section{Уравнения оптимального фильтра} 
 
 Для получения уравнений оптимального фильт\-ра воспользуемся подходом, 
 применяемым для решения аналогичной задачи в~стохастических сис\-те\-мах 
 наблюдения с~дискретным временем~\cite{BSh_85}. 
 Воспользу\-ем\-ся методом математической индукции. 
 
 При $r=0$ 
 \begin{equation}
 \widehat{X}_{t_0}={\sf E}\{X_0|\mathcal{O}_0\}={\sf E}\{X_0\}=\pi\,.
 \label{eq:in_cond}
 \end{equation} 
 
 Пусть для некоторого $ r \hm\geqslant 0$ известна оценка оптимальной 
 фильтрации~$\widehat{X}_{t_r} \hm= {\sf E}{X_{t_r} |\mathcal{O}_r}$. 
 Определим оценку оптимальной фильтрации~$\widehat{X}_{t} $ для $t\hm \in (t_r,t_{r+1}]$. 
 
 Для произвольного момента $t \hm\in (t_r,t_{r+1})$ в~силу мартингального 
 разложения МСП~$X_t$ и~свойств УМО верна следующая цепочка равенств:
 \begin{multline*}
 \widehat{X}_{t} = {\sf E}\left\{X_t | \mathcal{O}_r\right\}={}\\
 {}=
 {\sf E}\left\{{\cal P}^{\top}(t_r,t)X_{t_r}+
 \int\limits_{t_r}^t{\cal P}^{\top}(t_r,s)\,dM_s\big\vert \mathcal{O}_r\right\} = {}
\end{multline*}

\noindent
   \begin{multline}
 \hspace*{-11.66pt}{}=\mathcal{P}^{\top}(t_r,t)\widehat{X}_{t_r} + {\sf E}\hspace*{-2pt}
 \left\{{\sf E}\hspace*{-2pt}\left\{\int\limits_{t_r}^t\hspace*{-2pt}\mathcal{P}^{\top}(t_r,s)\,dM_s |
 {\mathcal{F}}_{t_r}\right\}\!\big\vert 
 \mathcal{O}_r\!\right\} ={}\hspace*{-4.24124pt}\\
 {}=
  \mathcal{P}^{\top}(t_r,t)\widehat{X}_{t_r}\,,
 \label{eq:bw_obs}
 \end{multline}
 где $\mathcal{P}(s,t)$ $(s \hm\leqslant t)$~--- матрица переходной ве\-ро\-ят\-ности МСП 
 на промежутке $[s,t]$, являющаяся решением сис\-те\-мы дифференциальных 
 уравнений Колмогорова
 \begin{equation*}
 \mathcal{P}'_t(s,t) = \mathcal{P}(s,t) \Lambda, \enskip t > s, \enskip \mathcal{P}(s,s) = I.
 \end{equation*}
 В случае однородного МСП $\mathcal{P}(s,t) \hm= e^{(t-s)\Lambda}$.
 
 Далее необходимо определить совместное распределение $(X_{t_{r+1}},Y_{r+1})$ 
 относительно~$ \mathcal{O}_r$. Из модели наблюдений следует, что 
 распределение~$Y_{r+1}$ относительно 
 $\sigma$-ал\-геб\-ры~$\mathcal{F}^X_{t_{r+1}} \vee \mathcal{O}_r$~---
 гауссовское с~параметрами 
 \begin{align*}
{\sf E}\left\{Y_{r+1}|{\mathcal{F}}^X_{t_{r+1}}\right\}& = f \tau_{r+1}\,; \\[6pt]
 \mathrm{cov} \left(Y_{r+1},Y_{r+1}|{\mathcal{F}}^X_{t_{r+1}}\right) &= 
 \displaystyle\sum\limits_{n=1}^N \tau_{r+1}^n g_ng_n^{\top}\,,
% \label{eq:occup_1}
 \end{align*}
 где $\tau_{r+1} \hm= \tau_{r+1}(X(\omega))=
 \mathrm{col}\left(\tau_{r+1}^1,\ldots,\tau_{r+1}^N\right) \ebd$\linebreak
 $\ebd 
 \int\nolimits_{t_r}^{t_{r+1}}X_s\,ds$~--- случайный вектор, $n$-я 
 компонента которого равна времени пребывания процесса~$X$ в~со\-сто\-янии~$e_n$ 
 на  интервале времени $[t_r, t_{r+1}]$. 
 Обозначим через $\mathcal{D}_{r+1} \ebd \{u=\mathrm{col}\,(u^1,\ldots,u^N):\; 
 u_m \hm\geqslant 0,\; \sum\nolimits_{m=1}^Mu_m\hm= t_{r+1}-t_r\}$ $(M-1)$-мер\-ный 
 симплекс в~пространстве~$\mathbb{R}^M$, являющийся носителем распределения 
 вектора~$\tau_{r+1}$. Пусть $\rho^{k,\ell}_{r+1}(du)$~--- 
 распределение вектора $\tau_{r+1} X_{t_{r+1}}^{\ell}$ при условии $X_{t_r}\hm=e_k$, 
 т.\,е.\ 
 для любого $\mathcal{A} \hm\in \mathcal{B}(\mathbb{R}^M)$ верно тождество:
\begin{multline*}
 \mathbf{P}\left\{\omega: \; X_{t_{r+1}}(\omega)=e_{\ell},\right.\\
 \left. 
 \tau_{r+1}(X(\omega)) \in \mathcal{A}\;|\;X_{t_r}=e_k\right\} \equiv
   \rho^{k,\ell}_{r+1}(\mathcal{A})\,.
\end{multline*}
 
Обозначим через
\begin{multline*}
 \mathcal{N}(y,m,K) \ebd (2\pi)^{-M/2} \mathrm{ det}^{-1/2} K\times{}\\
 {}\times\exp
 \left\{ -\fr{1}{2}\left(y-m\right)^{\top}K^{-1}(y-m)\right\}
\end{multline*}
 $M$-мер\-ную плот\-ность гауссовского распределения с~математическим 
 ожиданием~$m$ и~ковариационной матрицей~$K$.
 
 Из марковского свойства  $\{X_{t_{r}},Y_{r})\}_{r \geqslant 0}$ 
 относительно~${\mathcal{F}}_{t_{r}}$~\cite{ZhSh_95} и~теоремы Фубини следует, что 
 для любого  множества $\mathcal{A} \hm\in \mathcal{B}(\mathbb{R}^M)$ 
 верна следующая цепочка равенств:
 \begin{multline*}
 {\sf E}\left\{X_{t_{r+1}}\mathbf{I}_{\mathcal{A}}
 \left(Y_{r+1}\right)\big|\mathcal{O}_r\right\}={}\\
 {}=
{\sf E}\left\{{\sf E}\left\{X_{t_{r+1}}\mathbf{I}_{\mathcal{A}}
\left(Y_{r+1}\right)\big|
\mathcal{F}^X_{t_{r+1}} \vee \mathcal{O}_r\right\}
 \big|\mathcal{O}_r\right\} = {}
\end{multline*}

\noindent
\begin{multline*}
 %{}=
% {\sf E}\left\{{\sf E}\left\{X_{t_{r+1}}\mathbf{I}_{\mathcal{A}}
% \left(Y_{r+1}\right)\vert X_{t_r}\right\}
% \vert\mathcal{O}_r\right\} = {}\\
% {}=
%{\sf E}\left\{\sum\limits_{k=1}^N {\sf E}\left\{X_{t_{r+1}}\mathbf{I}_{\mathcal{A}}
%\left(Y_{r+1}\right)  \big| X_{t_r}=e_k\right\}X_{t_r}^k
% \big|\mathcal{O}_r\right\} = {}\\ 
% {}=
% \sum\limits_{k=1}^N{\sf E}
% \left\{X_{t_{r+1}}\mathbf{I}_{\mathcal{A}}\left(Y_{r+1}\right)\bigl| X_{t_r}=e_k\right\} 
% \widehat{X}_{t_r}^k ={}\\
% {}=\!
% \sum\limits_{k=1}^N{\sf E}
% \left\{{\sf E}\left\{X_{t_{r+1}}\mathbf{I}_{\mathcal{A}}
% \left(Y_{r+1}\right)\!\bigl| {\mathcal{F}}_{t_{r+1}}\right\}\!\bigl| 
% X_{t_r}\!=e_k\right\} \widehat{X}_{t_r}^k ={}\\
% {}=
% \sum\limits_{k=1}^N {\sf E}\left\{
% \vphantom{\int\limits_A\left(\sum\limits_{p=1}^N\right)}
% X_{t_{r+1}} \times{}\right.\\
% {}\times\int\limits_{\mathcal{A}}  
% \mathcal{N}\left(y,f \tau_{r+1}(X),\sum\limits_{p=1}^N \tau_{r+1}^p(X) g_pg_p^{\top}\right)dy
% \Biggl| X_{t_r}={}\\
%\left. {}=e_k
% \vphantom{\int\limits_A\left(\sum\limits_{p=1}^N\right)}
%\right\} \widehat{X}_{t_r}^k = 
% \sum\limits_{k=1}^N \int\limits_{\mathcal{A}}{\sf E}\left\{ 
% \vphantom{\sum\limits_{p=1}^N}
% X_{t_{r+1}} \times{}\right.\\
% {}\times\mathcal{N}\left(y,f \tau_{r+1}(X),\sum\limits_{p=1}^N \tau_{r+1}^p(X) 
% g_p g_p^{\top}\right)
% \Biggl| X_{t_r}={}\\
%\left. {}=e_k
%\vphantom{\sum\limits^N_{p=1}}
%\right\} \widehat{X}_{t_r}^k\, dy
 %={}\\
 {}=
 \sum\limits_{\ell=1}^N e_{\ell} \int\limits_{\mathcal{A}} 
 \left[ \sum\limits_{k=1}^N 
 \int\limits_{\mathcal{D}_{r+1}} 
 \mathcal{N}\left(y,f u,\sum_{p=1}^N u^p g_pg_p^{\top}\right)\times{}\right.\\
\left. {}\times
 \rho^{k,\ell}_{r+1}(du)\widehat{X}_{t_r}^k
 \vphantom{\int\limits_A\sum\limits_{p=1}^N}
 \right] 
 dy,
 \end{multline*}
 из чего следует, что интегранд в~квадратных скобках в~последнем выражении 
 определяет искомое совместное распределение $(X_{t_{r+1}},Y_{r+1})$ 
 относительно~$ \mathcal{O}_r$. Оценка~$\widehat{X}_{t_{r+1}}$ покомпонентно 
 определяется~\cite{BSh_85} с~помощью обобщенного варианта формулы Байеса:
 \begin{multline}
 \widehat{X}_{t_{r+1}}^j = {}\\
 \hspace*{-1mm}{}=
 \fr{\int\nolimits_{\mathcal{D}_{r+1}}\hspace*{-6mm} 
 \mathcal{N}\left(Y_{r+1},f u,\sum\nolimits_{p=1}^N \hspace*{-2mm}
 u^p g_pg_p^{\top}\!\right)\hspace*{-1mm}
 \sum\nolimits_{k=1}^N \hspace*{-2mm}
 \widehat{X}_{t_r}^k
 \rho^{k,j}_{r+1}(du)
 }
 { \int\nolimits_{\mathcal{D}_{r+1}} \hspace*{-6mm}
 \mathcal{N}\left(Y_{r+1},f v,\sum\nolimits_{q=1}^N \hspace*{-2mm}
 v^q g_qg_q^{\top}\!\right)\hspace*{-1mm}
 \sum\nolimits_{i,\ell=1}^N \hspace*{-2mm}
 \widehat{X}_{t_r}^i
 \rho^{i,\ell}_{r+1}(dv)
  },  \\ 
  j = \overline{1,N}\,.
 \label{eq:filt_1}
 \end{multline}
 Таким образом, доказана следующая
 
 %\smallskip
 
 \noindent
 \textbf{Лемма~1.}
\textit{Если для системы наблюдения}~(\ref{eq:obsys_1}) 
\textit{верны условия~(а) и~(б), то оценка~$\widehat{X}_t$ оптимальной фильтрации 
определяется формулой}~(\ref{eq:in_cond}) 
\textit{при $t\hm=0$, рекуррентным соотношением}~(\ref{eq:filt_1})~---
\textit{в~моменты~$t_{r+1}$ получения наблюдений~$Y_{r+1}$ 
и~формулой}~(\ref{eq:bw_obs})~--- 
\textit{в~промежутках времени между моментами получения наблюдений}.


\smallskip
 

 
 Несмотря на компактную запись~(\ref{eq:filt_1}), их прямая численная реализация 
 ресурсозатратна. Во-пер\-вых, в~(\ref{eq:filt_1}) требуется вычислять 
 распределения мас\-штаб\-но-сдви\-го\-вых смесей многомерных нормальных 
 распределений, что является трудоемкой\linebreak процедурой. Во-вто\-рых, 
 распределения~$\rho^{k,j}_{r+1}$ вре-\linebreak мени пребывания представляют собой 
 сумму\linebreak бесконечного ряда, слагаемые которого вычис\-ляются с~помощью 
 некоторой рекуррентной про\-це\-дуры~\cite{S_00}. В-третьих, 
 распределения~$\rho^{k,j}_{r+1}$ не являются абсолютно непрерывными 
 относительно меры Ле\-бега.
 { %\looseness=1
 
 }
 
 Следующий раздел посвящен численной аппроксимации~(\ref{eq:filt_1}) и~исследованию 
 ее точностных характеристик.
 
 \section{Приближенное вычисление оценки фильтрации}
 
 Без ограничения общности будем считать, что сетка~$\{t_r\}_{r \geqslant 0}$ 
 является равномерной с~шагом~$\Delta$, т.\,е.\ $t_r \hm= r \Delta$ 
 и~$\mathcal{D}_r \hm\equiv \mathcal{D}$.
 Обозначим через~$N_{r+1}$ об-\linebreak\vspace*{-12pt}
 
 \pagebreak
 
 \noindent
 щее число скачков процесса~$X_t$, имевших место 
 на промежутке $(t_r,t_{r+1}]$. Тогда из формулы полной вероятности следует, 
 что~(\ref{eq:filt_1}) представима в~виде:
 \begin{multline}
 \widehat{X}_{t_{r+1}}^j =  \left(
 \int\limits_{\mathcal{D}} 
 \mathcal{N}\left(Y_{r+1},f u,\sum\limits_{p=1}^N u^p g_pg_p^{\top}\right)\times{}\right.\\
\left. {}\times
 \sum\limits_{h=0}^{\infty}\sum\limits_{k=1}^N \widehat{X}_{t_r}^k
 \rho^{k,j,h}_{r+1}(du)
 \right)\Bigg/ \\
 \left(
 \vphantom{\sum\limits_{m=0}^{\infty}
 \sum\limits_{i,\ell=1}^N \widehat{X}_{t_r}^i
 \rho^{i,\ell,m}_{r+1}(dv)}
 \int\limits_{\mathcal{D}} 
 \mathcal{N}\left(Y_{r+1},f v,\sum\limits_{q=1}^N v^q g_qg_q^{\top}\right)\times{}\right.\\
\left.{}\times \sum\limits_{m=0}^{\infty}
 \sum\limits_{i,\ell=1}^N \widehat{X}_{t_r}^i
 \rho^{i,\ell,m}_{r+1}(dv)
 \right)
  \,, \enskip j = \overline{1,N}\,,
  \label{eq:filt_1_1}
 \end{multline}
 где 
 $ \rho^{k,j,h}_{r+1}(du)$~--- распределение вектора 
 $\tau_{r+1}X_{t_{r+1}}^{j}\mathbf{I}_{\{h\}}(N_{r+1})$ при 
 условии $X_{t_r}\hm=e_k$, т.\,е.\ 
 для любого $\mathcal{A} \hm\in \mathcal{B}(\mathbb{R}^M)$ верно тождество
\begin{multline*}
 \mathbf{P}\left\{\omega: \; X_{t_{r+1}}(\omega)=e_{j}, \; N_{r+1} = h,\right.\\ 
\left. \tau_{r+1}(X(\omega)) \in \mathcal{A}\;|\;X_{t_r}=e_k\right\} \equiv
  \rho^{k,j,h}_{r+1}(\mathcal{A}).
\end{multline*}
В качестве аппроксимации оценок можно использовать  
 $\overline{X}_{t_{r+1}}^n \ebd 
 \mathrm{col}\,(\overline{X}_{t_{r+1}}^{n,1},\ldots,\overline{X}_{t_{r+1}}^{n,N})$, 
 полученные из~(\ref{eq:filt_1_1}) путем урезания сумм ряда в~числителе и~знаменателе:
 
 \noindent
 \begin{multline}
 \overline{X}_{t_{r+1}}^{n,j} = 
 \left(
 \int\limits_{\mathcal{D}} 
 \mathcal{N}\left(Y_{r+1},f u,\sum\limits_{p=1}^N u^p g_pg_p^{\top}\right)\times{}\right.\\[-1pt]
\left.{}\times \sum\limits_{h=0}^{n}\sum\limits_{k=1}^N \overline{X}_{t_r}^k
 \rho^{k,j,h}_{r+1}(du)
 \right)\Bigg/ \\[-1pt]
 \left(
 \int\limits_{\mathcal{D}} 
 \mathcal{N}\left(Y_{r+1},f v,\sum\limits_{q=1}^N v^q g_qg_q^{\top}\right)\times{}\right.\\[-1pt]
\left. {}\times
 \sum\limits_{m=0}^{n}
 \sum\limits_{i,\ell=1}^N \overline{X}_{t_r}^i
 \rho^{i,\ell,m}_{r+1}(dv)
  \right)\,, \enskip
   j = \overline{1,N}.
  \label{eq:filt_2}
 \end{multline}
 Ниже по формуле полной вероятности получены интегралы из~(\ref{eq:filt_2}) для 
 $h\hm=0,1,2$:
 
\vspace*{-3pt}

 \noindent
  \begin{multline*}
 \int\limits_{\mathcal{D}}  \mathcal{N}
 \left(Y_{r+1},f u,\sum\limits_{p=1}^N u^p g_pg_p^{\top}\right) 
 \rho^{k,j,0}_{r+1}(du) = {}\\[-1pt]
 {}=
 \delta_{kj}\mathcal{N}\left(Y_{r+1},\Delta f^j,\Delta g_jg_j^{\top}\right)
 e^{\lambda_{jj}\Delta};
 %\label{eq:h0}
\\[-1pt]
 \int\limits_{\mathcal{D}}  \mathcal{N}\left(
 Y_{r+1},f u,\sum\limits_{p=1}^N u^p g_pg_p^{\top}\right) 
 \rho^{k,j,1}_{r+1}(du) ={} 
 \end{multline*}
 
 \noindent
 \begin{multline}
 \hspace*{-6.7pt}{}=\left(1-\delta_{kj}\right)\lambda_{kj}e^{\lambda_{jj}\Delta}
\! \int\limits_0^{\Delta}\!
 e^{(\lambda_{kk}-\lambda_{jj})u^k}
 \mathcal{N}\left(Y_{r+1},u^kf^k +{}\right.\hspace*{-0.28818pt}\\[-1pt]
\hspace*{-3mm}\left. {}+ \left(\Delta - u^k\right)f^j, u^k g_kg_k^{\top}+
 \left(\Delta-u^k\right)g_jg_j^{\top}\right)\,du^k;
 \label{eq:h1}
 \end{multline}
 
 \vspace*{-12pt}
 
 \noindent
 \begin{multline}
 \int\limits_D \mathcal{N}\left( 
Y_{r+1},f u,\sum\limits_{p=1}^N u^p g_pg_p^{\top}\right)du ={}\\[-1pt]
{}=
\sum\limits_{\substack{{\ell:\ell \neq k,}\\ {\ell \neq j}}}
 \lambda_{k\ell}\lambda_{\ell j} e^{\lambda_{jj}\Delta}\times {}\\[-1pt] 
 {}\times
 \int\limits_0^{\Delta} \int\limits_0^{\Delta-u^k} \!
e^{(\lambda_{kk}-\lambda_{\ell\ell})u^k+(\lambda_{\ell\ell}-
 \lambda_{jj})u^{\ell}}\times{} \\[-1pt] 
{}  \times
 \mathcal{N}\left(Y_{r+1},u^k f^k+u^{\ell}f^{\ell}+\left(
 \Delta-u^k-u^{\ell} \right)f^j,\right.\\[-1pt]
 \hspace*{-1mm}\left.
 u^k g_kg_k^{\top}+u^{\ell}g_{\ell}g_{\ell}^{\top}+\left(
 \Delta-u^k-u^{\ell} \right)
 g_jg_j^{\top}
 \right) du^{\ell}du^{k}, \!\!
  \label{eq:h2}
 \end{multline} 
 
\vspace*{-2pt}
 
 \noindent
  где  $\delta_{ij}$~--- символ Кронекера. Интегралы для $h\hm>2$ также могут 
  быть получены в~явном виде, однако их сложность резко возрастает.
 

   Так как система~(\ref{eq:obsys_1}) является автономной, то в~качестве локальной 
   характеристики бли\-зости~$\{\overline{X}_{t_r}\}$ 
   к~$\{\widehat{X}_{t_r}\}$ может быть выбрана величина
   
\noindent
 \begin{multline*}
 \overline{\sigma}(\pi) \ebd {\sf E}\left\{
 \|\widehat{X}_{t_{1}}(\pi, Y_{1}) - \overline{X}_{t_{1}}
 \left(\pi,Y_{1}\right)\|_{1}\right\} = {}\\
 {}=
 \sum\limits_{j=1}^N{\sf E}
 \left\{\left\vert \widehat{X}^j_{t_{1}}\left(\pi, Y_{1}\right) - \overline{X}^{n,j}_{t_{1}}
 \left(\pi,Y_{1}\right)\right\vert\right\}.
 %\label{eq:prec_1}
 \end{multline*}
 При этом начальное распределение $\pi \hm\in \mathcal{D}_1 \ebd $\linebreak $\ebd
 \{\mathrm{col}\,(\pi^1,\ldots,\pi^N):\;\pi^j > 0$, 
 $\sum\nolimits_{j=1}^N\pi^j\hm=1\}$ является начальным условием применения 
 одного шага рекурсии~(\ref{eq:filt_1}) или~(\ref{eq:filt_2}) для вычисления 
 оценки~$\widehat{X}_{t_{1}}$
   или~$\overline{X}_{t_{1}}$ соответственно. Фактически, 
 характеристика~$\overline{\sigma}(\pi)$ определяет, насколько сильно 
 рекурсивные схемы~(\ref{eq:filt_1}) и~(\ref{eq:filt_2}) разойдутся за 
 один шаг, стартуя из общей точки~$\pi$.
 
 Рекуррентные схемы~(\ref{eq:filt_1}) и~(\ref{eq:filt_2}), примененные~$r$~раз, 
 позволяют вычислить оценки~$\widehat{X}_{t_r}$ и~$\overline{X}_{t_r}$ 
 в~точке~$t_r$. В~качестве характеристики точности глобальной аппроксимации в~этом 
 случае естественно рассмотреть величину
 
 \vspace*{-2pt}
 
 \noindent
 \begin{equation*}
 \overline{\Sigma}_{t_r}(\pi) \ebd {\sf E}
 \left\{\|\widehat{X}_{t_{r}} - \overline{X}_{t_{r}}\|_{1}\right\} = 
 \!\sum\limits_{j=1}^N\!{\sf E}
 \left\{\left\vert \widehat{X}^j_{t_{r}} - 
 \overline{X}^{n,j}_{t_{r}}\right\vert \right\}.
% \label{eq:prec_2}
 \end{equation*}
 
 Следующее утверждение определяет оценки локальной и~глобальной 
 точности схемы аппроксимации~(\ref{eq:filt_2}).
 
 %\smallskip
 
 \noindent
 \textbf{Теорема~1.}\
\textit{Выполняются неравенства} 

%\vspace*{-2pt}

\noindent
 \begin{equation}
 \sup_{\pi \in \mathcal{D}_1} \overline{\sigma}(\pi) 
 \leqslant 2 \fr{(\overline{\lambda}\Delta)^{n+1}}{(n+1)!}\,;
 \label{eq:prec_loc}
\end{equation}

\noindent
\begin{align}
  \sup\limits_{\pi \in \mathcal{D}_1} \overline{\Sigma}_{t_r}(\pi)
   &\leqslant 2r \fr{(\overline{\lambda}\Delta)^{n+1}}{(n+1)!} +{}\notag\\[-0.5pt]
   &\hspace*{-20mm}{}+
  r(r-1)\left(
  \fr{(\overline{\lambda}\Delta)^{n+1}}{(n+1)!}
  \right)^2
  \left(
  1-\fr{(\overline{\lambda}\Delta)^{n+1}}{(n+1)!}
  \right)^{r-2},
 \label{eq:prec_glob}
 \end{align}
 
 \vspace*{-2pt}
 
 \noindent
 \textit{где} $\overline{\lambda} \ebd \max_{1 \leqslant j \leqslant N}|\lambda_{jj}|$.


%\smallskip

 Доказательство теоремы~1 приведено в~приложении.
 
 Данное утверждение представляет полезные оценки точности. Во-пер\-вых, 
 они являются равномерными по начальному распределению $\pi \hm\in \mathcal{D}_1$. 
 Во-вто\-рых, оценки носят универсальный, а~не асимптотический характер. Это 
 существенно в~практических задачах оценивания по дискретизованным 
 наблюдениям с~физическими или алгоритмическими ограничениями на шаг 
 по времени. Например, в~случае наблюдаемого процесса восстановления в~силу 
 центральной предельной теоремы для процессов восстановления~\cite{B_80} его
  приращения можно рассматривать как гауссовские случайные величины. 
  Однако данная аппроксимация обладает удовлетворительной точностью 
  только в~случае, когда шаг дискретизации по времени достаточно большой. 
 %
 В-третьих, неравенство~(\ref{eq:prec_glob}) позволяет получить порядок 
 аппроксимации при $\Delta \hm\to 0$. Зафиксируем момент времени $t\hm=T$ и~рассмотрим 
 характеристику $\sup\nolimits_{\pi \in \mathcal{D}_1} 
 \overline{\Sigma}_{T}(\pi)$ при $r\hm={T}/{\Delta}$ и~$\Delta \hm\to 0$. 
 Как только~$\Delta$ становится настолько мало, что 
 $\max\left({(\overline{\lambda}\Delta)^{n+1}}/{(n+1)!}, 
 \Delta ({T\lambda^{n+1}}/{(n+1)!})\right)\hm< 1$, из~(\ref{eq:prec_glob}) 
 следует неравенство
  %\begin{equation}
  $\sup\nolimits_{\pi \in \mathcal{D}_1} \overline{\Sigma}_{T}(\pi) 
  \hm\leqslant  ({3\overline{\lambda}^{n+1}}/{(n+1)!}) T\Delta^n.$
 %\label{eq:prec_asympt}
 %\end{equation}
 Это значит, что с~ростом времени~$T$ 
 ошибка аппроксимации копится пропорционально~$T$ и~при этом порядок точности 
 по~$\Delta$ равен~$n$.
 
 %\vspace*{-7pt}
 
  \section{Заключение}
  
  \vspace*{-4pt}
 
  В работе решена задача оценивания состояния однородного МСП по 
  дискретизованным наблюдениям. Получено аналитическое решение и~его 
  чис\-лен\-ные аппроксимации. Локальные и~глобальные показатели точ\-ности этих 
  приближений в~статье так\-же пред\-став\-ле\-ны. Примечательно, что  част\-ный случай 
  аппроксимаций~(\ref{eq:filt_2}) при $n\hm=0$ и~$\Lambda\hm=0$ был ранее 
  пред\-став\-лен в~\cite{B_17_1,B_17_2} для решения задачи байесовской классификации 
  случайного вектора по непрерывным наблюдениям с~мультипликативными шумами. 
 % 
Алгоритм оптимальной фильт\-ра\-ции и~его субоптимальные версии могут 
рас\-смат\-ри\-вать\-ся в~качестве основы чис\-лен\-ной реализации обобщения фильт\-ра 
Вонэма для сис\-тем с~мультипликативными шумами в~наблюдениях. 
Однако для их непосредственного использования необходимо решить 
следующие проб\-ле\-мы. Во-пер\-вых, в~(\ref{eq:h1}) и~(\ref{eq:h2}) присутствуют
 многомерные интегралы. Следует выяснить, какую результирующую погрешность 
 будут вносить ошибки их вы\-чис\-ле\-ния. Во-вто\-рых, представляется интересным 
 определить характеристики точ\-ности оптимальной фильт\-ра\-ции по дискретизованным 
 наблюдениям по отношению к~оптимальной фильт\-ра\-ции по непрерывным наблюдениям: 
 каков порядок точ\-ности по шагу временной дискретизации~$\Delta$? Для случая 
 вы\-чис\-ле\-ния классического фильт\-ра Вонэма с~по\-мощью алгоритма Эй\-ле\-ра--Ма\-ру\-ямы 
 подобный результат известен: порядок глобальной ошибки равен~${1}/{2}$. 
 Перечисленные задачи являются предметом дальнейших исследований.
 
 
  \vspace*{-10pt}
 
{\small
\subsection*{\raggedleft Приложение} 

\vspace*{-2pt}


\noindent
Д\,о\,к\,а\,з\,а\,т\,е\,л\,ь\,с\,т\,в\,о\ \ теоремы~1.\ \ Введем следующие 
обозначения для случайных величин и~мат\-риц, составленных из них:
\begin{align*}
\xi^{ji}(\ell)&\ebd 
\sum\limits_{h=0}^n \int\limits_{\mathcal{D}} 
 \mathcal{N}\left(Y_{\ell},f u,\sum\limits_{p=1}^N u^p g_pg_p^{\top}\right)
 \rho^{j,i,h}_{1}(du)\,; \\
  \theta^{ji}(\ell)&\ebd 
\sum\limits_{h=n+1}^{\infty} \int\limits_{\mathcal{D}} 
 \mathcal{N}\left(Y_{\ell},f u,\sum\limits_{p=1}^N u^p g_pg_p^{\top}\right)
 \rho^{j,i,h}_{1}(du)\,;
\\
 \xi(\ell)&\ebd \|\xi^{ji}(\ell)\|_{j,i=\overline{1,N}}\,,\quad 
 \Xi(r) \ebd \xi(r) \xi(r-1)\cdots \xi(1)\,;
 \\
 \theta(\ell)&\ebd \|\theta^{ji}(\ell)\|_{j,i=\overline{1,N}}\,, \quad 
 \Theta(r) \ebd \theta(r) \theta(r-1)\cdots \theta(1)\,.
%\label{eq:not_1}
\end{align*}
 
 Рекуррентные формулы~(\ref{eq:filt_1}) и~(\ref{eq:filt_2}) можно записать в~явной 
 форме
 
 
\noindent
\begin{align*}
 \widehat{X}_{t_r}& = \left( \mathbf{1}\left(\Xi(r) + 
 \Theta(r)\right)\pi\right)^{-1} \left(\Xi(r) + \Theta(r)\right)\pi\,;
\\
 \overline{X}_{t_r} &= \left( \mathbf{1}\Xi(r)\pi\right)^{-1} \Xi(r) \pi,
\end{align*}

\vspace*{-2pt}

\noindent
где $\mathbf{1} \ebd (1,\ldots,1)$~--- век\-тор-стро\-ка 
подходящей раз\-мер\-ности, составленная из единиц.

%Далее для краткости записи зависимость от~$r$ в~обозначениях~$\Xi(r)$ 
%и~$\Theta(r)$ будет опущена. 
Верна следующая цепочка неравенств:

 \vspace*{-3pt}

\noindent
\begin{multline}
\overline{\Sigma}_{t_r}(\pi)=%
%\me{}{\left\| 
%\widehat{X}_{t_r}(\pi, Y_1,\ldots,Y_r) - \overline{X}_{t_r}(\pi, Y_1,\ldots,Y_r)
%\right\|_1} =\\=
{\sf E}\left\{\left\| 
\fr{1}{\mathbf{1}\left(\Xi(r) + \Theta(r)\right)\pi} \left(\Xi(r) +{}\right.\right.\right.\\[-1pt]
\left.\left.\left.{}+ \Theta(r)\right)\pi
- \fr{1}{\mathbf{1}\Xi(r)\pi}\,\Xi(r) \pi
\right\|_1\right\} ={} \\[-1pt]
{}=
{\sf E}\left\{\fr{1}{\mathbf{1}\left(\Xi(r) + \Theta(r)\right)\pi \mathbf{1}\Xi(r)\pi}
\left\|
 \mathbf{1}\Xi(r) \pi \Theta(r)\pi -{}\right.\right.\\[-1pt]
\left.\left. {}- \mathbf{1}\Theta(r)\pi \Xi(r) \pi
 \right\|_1
 \vphantom{\fr{1}{\mathbf{1}\left(\Xi(r) + \Theta(r)\right)\pi \mathbf{1}\Xi(r)\pi}}
\right\} \leqslant {}\\[-1pt]
{}\leqslant 
{\sf E}\left\{\fr{1}{\mathbf{1}\left(\Xi(r) + \Theta(r)\right)\pi \mathbf{1}\Xi(r)\pi}
\left(
\mathbf{1}\Xi(r)\pi \| \Theta(r)\pi \|_1 +{}\right.\right.\\[-1pt]
\left.\left.{}+ \mathbf{1}\Theta(r)\pi 
\|
\Xi(r) \pi
\|_1
\right)
 \vphantom{\fr{1}{\mathbf{1}\left(\Xi(r) + \Theta(r)\right)\pi \mathbf{1}\Xi(r)\pi}}
\right\} ={}\\[-1pt]
{}=
2\,{\sf E}\left\{\fr{1}{\mathbf{1}\left(\Xi(r) + \Theta(r)\right)\pi}\mathbf{1}\Theta(r)\pi 
\right\}.
\label{eq:ineq_1}
\end{multline}

 
 \noindent
 Рассмотрим случайные события $a_{\ell} \ebd \{\omega \in \Omega: 
 N_{\ell}(\omega) \hm\leqslant n\}$, $\ell \hm= \overline{1,r}$, и~$A_r \ebd \{
 \omega\hm \in \Omega: \max_{1 \leqslant {\ell} \leqslant r}N_{\ell}(\omega) 
 \hm\leqslant n
 \}\hm=\prod\nolimits_{\ell=1}^r a_{\ell}$ и~оценку 
 $
 \widetilde{X}_{t_r}(\pi, Y_1,\ldots,Y_r)\ebd$\linebreak $\ebd
 {\sf E}\left\{X_{t_r}(\omega)\mathbf{I}_{A_r}(\omega)|\mathcal{O}_r\right\}.
 $
 Используя введенные выше обозначе\-ния и~абстрактный вариант формулы Байеса, 
 получаем, что
 
 \noindent
\begin{align}
\widetilde{X}_{t_r}& = \fr{1}{{\mathbf{1}\left(\Xi(r) + 
 \Theta(r)\right)\pi}}\,\Xi(r)\pi\,;\notag
 \\
\widehat{X}_{t_r} - \widetilde{X}_{t_r} &=
{\sf E}\left\{X_{t_r}(\omega)\mathbf{I}_{\overline{A}_r}(\omega)|\mathcal{O}_r\right\} ={}\notag\\[-1pt]
&\hspace*{17mm}{}= 
\fr{1}{\mathbf{1}\left(\Xi(r) + \Theta(r)\right)\pi}\Theta(r)\pi\,. 
\label{eq:eq_2}
 \end{align}
 Из (\ref{eq:ineq_1}) и~(\ref{eq:eq_2}) для $r\hm=1$ следует, что
 
 \vspace*{-4pt}
 
 \noindent
 \begin{multline}
 \overline{\sigma}(\pi) \leqslant 2\,{\sf E}
 \left\{\|{\sf E}\left\{X_{t_1}(\omega)\mathbf{I}_{\overline{a}_1}(\omega)|\mathcal{O}_1
 \right\}\|_1
 \right\} ={}\\[-1.5pt]
 {}=
 2\,{\sf E}\left\{\sum\limits_{n=1}^N {\sf E}
 \left\{X^n_{t_1}(\omega)\mathbf{I}_{\overline{a}_1}
 (\omega)|\mathcal{O}_1\right\}\right\} ={} \\[-2pt] 
 {}=
  2\,{\sf E}\left\{{\sf E}\left\{\mathbf{I}_{\overline{a}_1}(\omega)|\mathcal{O}_1
  \right\}\right\} =
   2 \mathbf{P}\left\{\overline{a}_1(\omega)\right\}.
\label{eq:ineq_3}
\end{multline}

 \vspace*{-2pt}
 
 \noindent
 Процесс $N^X_t$ общего числа скачков состояния~$X_t$ является считающим, и~его
  квадратическая характеристика равна 
  
\vspace*{-2pt}
  
  \noindent
 $$
 \langle N^X, N^X\rangle_t = - \int\limits_0^t \sum\limits_{n=1}^N \lambda_{nn} X_s^n\,ds\,,
 $$
 поэтому искомая вероятность ограничена сверху:
 $$ 
 \mathbf{P}\left\{\overline{a}_1(\omega)\right\} \leqslant 
 e^{-\overline{\lambda}\Delta}\sum\limits_{k=n+1}^{\infty} 
 \fr{(\overline{\lambda}\Delta)^{k}}{k!} <
 \fr{(\overline{\lambda}\Delta)^{n+1}}{(n+1)!}.
 $$
 
  \vspace*{-2pt}
  
  \noindent
 Из последнего неравенства и~(\ref{eq:ineq_3}) следует, что  для любого 
 начального распределения~$\pi$ выполняется неравенство $\overline{\sigma}(\pi)  
 \hm< 2({(\overline{\lambda}\Delta)^{n+1}}/{(n+1)!})$, т.\,е.\ 
 локальная оценка~(\ref{eq:prec_loc}) верна.
 
 С помощью марковского свойства пары $(X_t, N^X_t)$ и~последнего 
 неравенства можно оценить сверху вероятность 
 $\mathbf{P}\left\{\overline{A}_r(\omega)\right\}$:
 
  \vspace*{-2pt}
 
 \noindent
 \begin{multline*}
 \mathbf{P}\left\{\overline{A}_r(\omega)\right\} \leqslant 1 - \left(
 1- \fr{(\overline{\lambda}\Delta)^{n+1}}{(n+1)!}
 \right)^r \leqslant r \fr{(\overline{\lambda}\Delta)^{n+1}}{(n+1)!} + {}\\[-1pt]
 {}+\left|
 \sum\limits_{k=2}^r C_r^k \left(-\fr{(\overline{\lambda}\Delta)^{n+1}}{(n+1)!}
 \right)^k
 \right| \leqslant
 r \fr{(\overline{\lambda}\Delta)^{n+1}}{(n+1)!} +{}\\[-1pt]
 {}+\fr{r(r-1)}{2}
 \left(
 \fr{(\overline{\lambda}\Delta)^{n+1}}{(n+1)!}
 \right)^2
 \left(
 1-\fr{(\overline{\lambda}\Delta)^{n+1}}{(n+1)!}
 \right)^{r-2},
 \end{multline*} 
 из чего следует истинность глобальной оценки~(\ref{eq:prec_glob}).
Теорема~1 доказана.

}

%\vspace*{-12pt}

{\small\frenchspacing
 {%\baselineskip=10.8pt
 \addcontentsline{toc}{section}{References}
 \begin{thebibliography}{99}

\bibitem{Won_65}
\Au{Wonham W.} 
Some applications of stochastic differential equations to optimal
  nonlinear filtering~//
SIAM~J.~Control, 1965. Vol.~2. P.~347--369. 

\bibitem{KP_92}
\Au{Kloeden P., Platen E.} Numerical solution of stochastic
differential equations.~--- Berlin: Springer, 1992.~636~p.

\bibitem{YZL_04}
\Au{Yin G., Zhang Q., Liu Y.} 
Discrete-time approximation of Wonham filters~//
J.~Control Theory Applications, 2004. Iss.~2. P.~1--10.

\bibitem{PR_10}
\Au{Platen E., Rendek R.}
Quasi-exact approximation of hidden Markov chain filters~//
Communicat.~Stoch.~Analys., 2010. Vol.~4. Iss.~1. P.~129--142.

\bibitem{B_18}
\Au{Борисов А.} Фильтрация Вонэма по наблюдениям с~мультипликативными шумами~// 
Автоматика и~телемеханика, 2018.
№~1. C.~52--65. 
 
  \bibitem{BSh_85} %6
\Au{Бертсекас Д., Шрив С.} Стохастическое оптимальное управление. 
Случай дискретного времени~/ Пер. с~англ.~--- М.: Наука, 1985.~280~c.
(\Au{Betsekas~D.\,P., Shreve~S.\,E.} Stochastic optimal control:
The discrete-time case.~--- Orlando, FL, USA:
Academic Press Inc., 1978. 323~p.)

  \bibitem{ZhSh_95} %7
\Au{Жакод Ж., Ширяев А.} Предельные теоремы для случайных процессов,~I.~/
Пер. с~англ.~--- 
М.: Физматлит, 1995.~544~c.
(\Au{Jacod~J., Shiryaev~A.} Limit theorems for stochastic processes.~---
Berlin: Springer, 2003. 664~p.)

\bibitem{S_00}
\Au{Sericola B.} Occupation times in Markov processes~//
Commun. Stat. Stochastic Models, 2000. Vol.~16. Iss.~5. P.~479--510. 

  \bibitem{B_80}
\Au{Боровков А.} Асимптотические методы в~тео\-рии массового обслуживания.~--- 
М.: Физматлит, 1995.~384~c.

  \bibitem{B_17_1}
\Au{Борисов А.} Классификация по непрерывным наблюдениям с~мультипликативными шумами.~I. 
Формулы байесовской оценки~// Информатика и~её применения, 2017. Т.~11. Вып.~1. C.~11--19.
doi: 10.14357/19922264170102.

  \bibitem{B_17_2}
\Au{Борисов А.} Классификация по непрерывным наблюдениям с~мультипликативными 
шумами.~II. Алгоритм численной реализации оценки~// Информатика и~её 
применения, 2017. Т.~11. Вып.~2. C.~33--41.
doi: 10.14357/19922264170204.

 \end{thebibliography}

 }
 }

\end{multicols}

\vspace*{-4pt}

\hfill{\small\textit{Поступила в~редакцию 10.07.18}}

\vspace*{6pt}

%\pagebreak

%\newpage

%\vspace*{-28pt}

\hrule

\vspace*{2pt}

\hrule

%\vspace*{-2pt}

\def\tit{FILTERING OF~MARKOV JUMP PROCESSES\\ BY~DISCRETIZED OBSERVATIONS}

\def\titkol{Filtering of Markov jump processes by discretized observations}

\def\aut{A.\,V.~Borisov}

\def\autkol{A.\,V.~Borisov}

\titel{\tit}{\aut}{\autkol}{\titkol}

\vspace*{-11pt}


\noindent
Institute of Informatics Problems, Federal Research Center ``Computer Science 
and Control'' of the Russian Academy of Sciences, 44-2~Vavilov Str., Moscow 
119333, Russian Federation


\def\leftfootline{\small{\textbf{\thepage}
\hfill INFORMATIKA I EE PRIMENENIYA~--- INFORMATICS AND
APPLICATIONS\ \ \ 2018\ \ \ volume~12\ \ \ issue\ 3}
}%
 \def\rightfootline{\small{INFORMATIKA I EE PRIMENENIYA~---
INFORMATICS AND APPLICATIONS\ \ \ 2018\ \ \ volume~12\ \ \ issue\ 3
\hfill \textbf{\thepage}}}

\vspace*{6pt}



\Abste{The article is devoted to a~solution of the optimal filtering problem 
of a~homogenous Markov
jump process state. The available observations represent 
time increments of the integral transformations of the Markov\linebreak\vspace*{-12pt}}

\Abstend{state corrupted by 
Wiener processes. The noise intensity is also state-dependent. At the instant of 
the consecutive
observation obtaining, the optimal estimate is calculated recursively 
as a~function of previous estimate and the new observation, meanwhile between 
observations the filtering estimate is a simple forecast by virtue of the Kolmogorov 
differential system. The recursion is rather expensive because of  need to calculate 
the integrals, which are the location-scale mixtures of Gaussians. The mixing 
distributions represent the occupation of the state in each of possible values 
during the mid-observation intervals. The paper contains numerically cheaper 
approximations, based on the restriction of the state transitions number between 
the observations. Both the local and global characteristics of approximation 
accuracy are obtained as functions of the dynamics parameters, mid-observation 
interval length, and upper bound of transitions number.}

\KWE{Markov jump process; optimal filtering; multiplicative observation noises; 
stochastic differential equation; numerical approximation}




\DOI{10.14357/19922264180316}

%\vspace*{-14pt}

\Ack
\noindent
The work was supported in part by the Russian Foundation
for Basic Research (Project No.\,16-07-00677).



%\vspace*{6pt}

  \begin{multicols}{2}

\renewcommand{\bibname}{\protect\rmfamily References}
%\renewcommand{\bibname}{\large\protect\rm References}

{\small\frenchspacing
 {%\baselineskip=10.8pt
 \addcontentsline{toc}{section}{References}
 \begin{thebibliography}{99}
\bibitem{Won_65-1}
\Aue{Wonham, W.} 1965.
Some applications of stochastic differential equations to optimal
  nonlinear filtering.
\textit{SIAM~J.~Control} 2:347--369. 

\bibitem{KP_92-1}
\Aue{Kloeden,~P., and E.~Platen.} 1992. \textit{Numerical solution of stochastic
differential equations.} Berlin: Springer. 636~p.

\bibitem{YZL_04-1}
\Aue{Yin,~G., Q.~Zhang, and Y.~Liu.} 2004.
Discrete-time approximation of Wonham filters.
\textit{J.~Control Theory Applications} 2:1--10.

\bibitem{PR_10-1}
\Aue{Platen, E., and R.~Rendek.} 2010.
Quasi-exact approximation of hidden Markov chain filters.
\textit{Communicat. Stoch. Analys.} 4(1):129--142.

\bibitem{B_18-1}
\Aue{Borisov, A.} 2018. Wonham filtering by observations
with multiplicative noises. \textit{Automat.~Rem.~Contr.} 79(1):39--50.  
doi: 10.1134/ S0005117918010046.
 
  \bibitem{BSh_85-1}
\Aue{Bertsekas, D., and S.~Shreve.} 1996.
\textit{Stochastic optimal control: The discrete-time case}.
Nashua, NH: Athena Scientific. 330~p.
  
  \bibitem{ZhSh_95-1}
  \Aue{Jacod,~J., and A.~Shiryaev.} 2003.
\textit{Limit theorems for stochastic processes.}
Berlin: Springer. 664~p.

\bibitem{S_00-1}
\Aue{Sericola, B.}
2000. Occupation times in Markov processes.
\textit{Commun. Stat.} 16(5):479--510. 

  \bibitem{B_80-1}
\Aue{Borovkov, A.} 1984.
 \textit{Asymptotic methods in queueing theory}. 
 Hoboken, NJ: Wiley-Blackwell.~304~p.

  \bibitem{B_17_1-1}
  \Aue{Borisov, A.} 2017. 
  Klassifikatsiya po ne\-pre\-ryv\-nym nablyu\-de\-miyam s~mul'tiplikativnymi shumami. I. 
  Formuly bayesov\-skoy otsenki [Classification by continuous-time observations
in multiplicative noise. I.~Formulae for Bayesian 
estimate]. \textit{Informatika i~ee Primeneniya~--- Inform.~Appl.}
11(1):11--19. doi: 10.14357/19922264170102.

  \bibitem{B_17_2-1}
\Aue{Borisov, A.} 2017. Klassifikatsiya po nepreryvnym nablyudemiyam 
s~mul'tiplikativnymi summami. II.~Formuly bayesovskoy otsenki 
[Classification by continuous-time observations
in multiplicative noise. II.~Numerical algorithm].
\textit{Informatika i~ee Primeneniya~--- Inform.~Appl.}
11(2):33--41. doi: 10.14357/19922264170204.

\end{thebibliography}

 }
 }

\end{multicols}

\vspace*{-6pt}

\hfill{\small\textit{Received July 10, 2018}}

%\pagebreak

%\vspace*{-18pt}

\Contrl

\noindent
\textbf{Borisov Andrey V.} (b.\ 1965)~--- 
Doctor of Science in physics and mathematics, principal scientist, Institute of
Informatics Problems, Federal Research Center ``Computer Science and Control''
 of the Russian Academy of
Sciences, 44-2 Vavilov Str., Moscow 119333, Russian Federation; 
\mbox{aborisov@frccsc.ru}
\label{end\stat}

\renewcommand{\bibname}{\protect\rm Литература}         %
\def\stat{vasiliev}

\def\tit{О ФУНКТОРНОМ ПРЕДСТАВЛЕНИИ ОПТИМИЗИРУЕМЫХ ДИНАМИЧЕСКИХ 
МУЛЬТИАГЕНТНЫХ СИСТЕМ}

\def\titkol{О функторном представлении оптимизируемых динамических 
мультиагентных систем}

\def\aut{Н.\,С.~Васильев$^1$}

\def\autkol{Н.\,С.~Васильев}

\titel{\tit}{\aut}{\autkol}{\titkol}

\index{Васильев Н.\,С.}
\index{Vasilyev N.\,S.}


%{\renewcommand{\thefootnote}{\fnsymbol{footnote}} \footnotetext[1]
%{Исследование выполнено за счет гранта Российского научного фонда №\,22-28-00588, {\sf 
%https://rscf.ru/project/22-28-00588/}. Работа проводилась с~использованием инфраструктуры Центра 
%коллективного пользования <<Высокопроизводительные вычисления и~большие данные>> (ЦКП 
%<<Информатика>> ФИЦ ИУ РАН, Москва).}}


\renewcommand{\thefootnote}{\arabic{footnote}}
\footnotetext[1]{Московский государственный технический университет имени Н.\,Э.~Баумана, \mbox{nik8519@yandex.ru}}

\vspace*{2pt}






    \Abst{Топос функторов выбран в~качестве компьютерного инструмента синтеза 
динамических игр многих лиц. Задаваемая шкала упорядочивает объекты, 
отвечающие сопутствующим статическим подыграм. Последние служат 
состояниями динамической мультиагентной сис\-те\-мы (ДМАС). Исходная 
динамическая игра и~все статические подзадачи пред\-став\-ля\-ют\-ся в~моноидальной 
категории бинарных отношений. Под рациональным решением игры понимается 
равновесие. Композициональное строение оп\-ти\-ми\-зи\-ру\-емой ДМАС выражено 
в~форме динамического результирующего отношения (ДРО) игры. Поиску равновесия 
отвечает максимизация ДРО. Это делается методом Беллмана, обобщенным на 
задачи оптимального управления, поставленные в~форме отношений. Программная 
реализация предложенного подхода может быть основана на нейросетевых 
вычислениях ввиду согласованности архитектур применяемых графов отношений и~нейросетей.}
    
    \KW{категория функторов; композициональность; моноидальная категория; 
обратный образ; динамическое отношение игры; статическая подыгра; отношение 
предпочтения; динамическое результирующее отношение; рациональное решение; 
морфизм Беллмана}

\DOI{10.14357/19922264240201}{CLMBXC}
  
%\vspace*{-6pt}


\vskip 10pt plus 9pt minus 6pt

\thispagestyle{headings}

\begin{multicols}{2}

\label{st\stat}
    
\section{Введение}

    В ДМАС агенты принимают 
решения в~каждом состоянии системы в~за\-ви\-си\-мости от рас\-по\-ла\-га\-емой ими 
информации о~поведении других участников конфликта~[1--4]. Воздействие 
стратегий на ДМАС распределено во времени и~связано с~выбором ситуаций 
в~череде со\-пут\-ст\-ву\-ющих статических игровых подзадач. 
    
    Процесс изменения состояний ДМАС традиционно задают с~по\-мощью 
дифференциальных или итерационных уравнений, в~которые входят управ\-ля\-ющие 
воздействия игроков. От этого описания динамики приходится отходить даже 
в~антагонистических дифференциальных играх с~полной ин\-фор\-ми\-ро\-ван\-ностью 
игроков. Управляемую систему представляют дифференциальным уравнением 
в~контингенциях~\cite{3-vas}, т.\,е.\ применяют многозначные отоб\-ра\-же\-ния. 
В~стохастических постановках так\-же возникает дополнительная не\-опре\-де\-лен\-ность, 
связанная с~прогнозированием будущих со\-сто\-яний сис\-те\-мы~\cite{4-vas}. 
    
    Зачастую целесообразно отказаться от функционального описания динамики 
игры и~перейти на\linebreak язык отношений~[5]. Благодаря этому значительно расширяется 
круг приложений~\cite{4-vas, 6-vas, 7-vas, 8-vas, 9-vas}, приобретается свойство 
композициональности моделей ДМАС, удобное для модификации игровых \mbox{задач}. 
Этот подход поддерживается компьютерной алгеброй тео\-рии категорий~\cite{10-vas, 11-vas}. 
Кроме того, графовая структура отношений и~сетевая структура игровой задачи 
допускают эффективную нейросетевую программную реализацию~\cite{6-vas, 7-vas, 12-vas, 13-vas}. 
    
    Категорный подход охватывает общий случай динамических игр многих лиц 
    с~разнообразными классами допустимых стратегий игроков~\cite{5-vas, 11-vas}. 
В~качестве со\-сто\-яний динамической системы рассматриваются со\-пут\-ст\-ву\-ющие 
статические игры, связанные динамическим отношением.
    
    Традиционное моделирование игровой задачи проводится средствами 
категории множеств SET. Естественным обобщением классического подхода 
становится формализация игровой операции, вы\-пол\-ня\-емая на языке моноидальной 
категории бинарных отношений REL. Применяемый аппарат позволяет 
единообразно выражать и~модифицировать интересы агентов, проводить 
эквивалентные преобразования игровой задачи, описывать поиск рационального 
решения~\cite{5-vas}. С~по\-мощью введения ДРО игры исследование ДМАС может быть сведено 
к~последовательной максимизации ДРО.
    
    Итак, вместо многошагового процесса игры предлагается перейти 
    к~представлению динамики задачи с~по\-мощью функтора $\tau: S\hm\to \mathrm{SET}_T$, 
который преобразует вы\-би\-ра\-емую шкалу~$S$ в~категорию $\mathrm{SET}_T$ монады 
$(T,\eta,\psi)$, $T: A\hm\to 2^A$~\cite{11-vas}. Объектами монады служат конечные 
множества, а~морфизмами~--- отношения между ними. Равенство $\mathrm{SET}_T\hm=\mathrm{REL}$ 
означает, что функторная модель ДМАС строится на языке отношений. 
Единообразно выражаются правила игры, интересы участников, классы стратегий 
агентов, динамика задачи и~даже алгоритм ее решения. Строгость функтора 
вложения $\mathrm{SET}\hm\to \mathrm{SET}_T$ обеспечивает пре\-емст\-вен\-ность новой формы 
пред\-став\-ле\-ния задачи и~классической.
    
    Функтор $\tau$ порождает динамическое отношение игры $R\hm= R(\tau)$, 
регламентирующее смену со\-сто\-яний~--- ре\-ша\-емых в~текущий момент времени 
статических подыгр $\Gamma(X)$ с~множеством \mbox{до\-пус\-ти\-мых} ситуаций~$X$. 
Динамическое отношение определяет по\-сле\-до\-ва\-тель\-ность смены со\-сто\-яний 
$\Gamma(X_0), \ldots , \Gamma(X_T)$ и~вводит отношение предшествования подыгр 
$\Gamma(X)\overset{R}{\prec} \Gamma(Y)$. В~соответствии с~ним определена 
преемственность ситуаций $x\overset{R}{\prec} y$, причем выбор любой пары 
ситуаций $x\hm\in X$ и~$y\hm\in Y$, относящихся к~разным состояниям системы, 
возможен, только если выполнено включение $(x,y)\hm\in R$. Таким образом, из 
решений подзадач $\Gamma(X_0), \ldots , \Gamma(X_T)$ строится 
по\-сле\-до\-ва\-тель\-ность $x_0, \ldots , x_T$ вы\-би\-ра\-емых ситуаций подобно тому, как это 
делается в~многошаговых играх с~по\-мощью уравнений. 
    
    Интересы агентов в~задаче заданы в~форме отношений предпочтения для 
терминального со\-сто\-яния сис\-те\-мы~$\Gamma(X^T)$: 
    \begin{equation}
    \tilde{\rho}^{kT} \in \mathrm{REL} \left( X^T, X^T\right)\,,\enskip k\in K\,.
    \label{e1.1-vas}
    \end{equation}
    
    Правила функционирования ДМАС формулируются во всех подыграх 
$\Gamma(X_i)$. Во-пер\-вых, определены цели игроков~--- максимизация 
отношений предпочтения $\rho_i^k\hm\in \mathrm{REL}\left( X_i, X_i\right)$, $k\hm\in 
K(X_i)$, по\-лу\-ча\-емых из заданных~(\ref{e1.1-vas}) с~по\-мощью вы\-чис\-ле\-ния 
обратных образов морфизмов~$\tilde{\rho}^{kT}$ относительно динамического 
отношения~$R$. Во-вто\-рых, агенты выбирают классы допустимых стратегий 
посредством графа коммуникаций и~функции, размечающей его дуги. Тем самым 
задается сетевая структура в~статических подыграх~\cite{12-vas, 13-vas}. Разметка 
графа детализирует информацию, которой обмениваются партнеры при совершении 
ходов и~формировании коалиций~\cite{5-vas}. Стратегии агентов в~исходной 
динамической задаче включают процедуру принятия решений в~каж\-дом состоянии 
сис\-темы. 
    
    Далее изучено композициональное стро\-ение динамического результирующего 
отношения игры, поз\-во\-ля\-ющее искать рациональное решение, т.\,е.\ такое, которое 
оптимизирует функционирование ДМАС. Поиск проводится методом 
динамического программирования, обоб\-щен\-ным на задачи оптимального 
управ\-ле\-ния, по\-став\-лен\-ные в~форме отношений. 

\section{Постановка задачи}

    В качестве шкалы~$S$ выберем категорию 
    \begin{equation}
 S: L \overset{\underset{\mathrm{out}}{\longrightarrow}}{\underset{\overset{\mathrm{in}}{\longrightarrow}}
 {\vphantom{{\mbox{\scriptsize{$\circ$}}}} \ }} M\,.
    \label{e2.1-vas}
    \end{equation}
    
    Динамику мультиагентной системы зададим функтором $\tau: S\hm\to \mathrm{SET}_T$, 
которому отвечает ориентированный граф динамического отношения~$\Gamma_R$, 
$R\hm=R(\tau)$. Граф должен быть ацик\-ли\-че\-ским и~связ\-ным. В~его вершинах 
находятся множества до\-пус\-ти\-мых ситуаций статических подыгр~$\Gamma(X_i)$, 
а~дуги отвечают отношениям $R_{ij}\hm\in \mathrm{REL}\left( X_i, X_j\right)$. 
    
    \smallskip
    
    \noindent
    \textbf{Определение~2.1.}\  Скажем, что ситуация $x_i\hm\in X_i$ 
непосредственно предшествует ситуации $x_j\hm\in X_j$, если и~только если $(x_i, 
x_j)\hm\in R_{ij}$. Отношением предшествования со\-сто\-яний назовем транзитивное 
замыкание $\overline{R}\hm= \overline{R}_{ij}$ динамического отношения. 
    
    \smallskip
    
    Последовательная смена состояний ДМАС происходит в~соответствии 
с~отношением непосредственного предшествования $\Gamma(X_i) 
\overset{R_{ij}}{\prec} \Gamma(X_j)$. Каж\-до\-му со\-сто\-янию соответствует некоторая 
\mbox{статическая} подыгра. Процесс функционирования сис\-те\-мы складывается из 
последовательного выбора игроками ситуаций $x_0, \ldots , x_T$, где  
$x_0$ и~$x_T$~--- решения начальной и~терминальной подыгр. Итак, имеем 
множество всех допустимых ситуаций исходной динамической задачи вида
 \begin{multline*}
    \tilde{X} ={}\\
    {}=\!\left\{\! \tilde{x} =\left( x_0, \ldots , x_T\right): \left( \forall_i x_i\in 
X_i\right) \wedge x_0\overset{R}{\prec} \cdots \overset{R}{\prec} x_T\!\right\}\!.\hspace*{-5.73415pt}
    \end{multline*}
    
    Правила проведения каждой подыгры $\Gamma(X_i)$ задаются сле\-ду\-ющим 
функтором и~функ\-цией: 
    \begin{equation}
    g_i: S\to \mathrm{SET}\,;\enskip h_i: E\to \overline{2}\,.
    \label{e2.2-vas}
    \end{equation}
Игроки стремятся по возможности максимизировать свои отношения предпочтения 
$\rho_l^k: X_l\hm\to X_l$,\linebreak $k\hm\in K_l$. (Иначе рас\-смат\-ри\-ва\-ют\-ся противоположные 
отношения $(\rho_i^k)^{\mathrm{op}}$.) Функторы~(\ref{e2.1-vas}) и~(\ref{e2.2-vas}) 
опре\-де\-ля\-ют сетевую структуру игры. Шкала~$S$ выделяет множество дуг~$E$, по 
которым участники\linebreak $k\hm\in K(X_i)$ статической игры осуществляют 
коммуникацию. Функциями in и~out в~(\ref{e2.1-vas}) вводят порядок ходов. 
Функция~$h_i$ из формулы~(\ref{e2.2-vas}) осуществляет разметку дуг графа 
коммуникаций~$\gamma_i$, определяя данные, которыми обмениваются игроки при 
своих ходах~\cite{5-vas}. Не\-пус\-тые сообщения могут содержать либо выбранную 
стра\-те\-гию-конс\-тан\-ту, либо стра\-те\-гию-функ\-цию~\cite{1-vas, 5-vas}. 
Предполагается так\-же, что никто из агентов не блефует. 
    
    Понятие рационального поведения игроков зависит от сетевой 
структуры~(\ref{e2.2-vas}) статических игр~\cite{5-vas, 12-vas, 13-vas}. Будем 
применять принцип равновесия как к~исходной динамической задаче, так и~ко всем 
по отдельности подыграм~$\Gamma(X_i)$.
    
    \smallskip
    
    \noindent
    \textbf{Пример~2.1.}\ Рассмотрим функтор~(\ref{e2.1-vas}), которому отвечает 
граф
    \begin{equation}
    \Gamma_R: \mbox{\fbox{$X_1$}} \xrightarrow{R_{12}} \mbox{\fbox{$X_2$}} 
\xrightarrow{R_{23}} \cdots \xrightarrow{R_{n-1,n}} \mbox{\fbox{$X_n$}}\,.
    \label{e2.3-vas}
    \end{equation}
    
    Функции out и~in определяют начало $X_i\hm\in M$ и~конец $X_j\hm\in M$ 
каж\-дой дуги, обозначенной стрелкой $R_{i,i+1}\hm\in L$. В~моменты времени 
$i\hm= 1,2,\ldots , n$ решаются игры~$\Gamma(X_i)$.
    
    Схема~(\ref{e2.3-vas}) свойственна многошаговым опе\-ра\-циям.
    
    \smallskip
    
    \noindent
    \textbf{Пример~2.2.}\ Пусть в~позиционной многошаговой игре двух лиц 
известна сле\-ду\-ющая динамическая сис\-те\-ма, начальное условие и~фазовое 
ограничение:
    \begin{gather*}
    \left\{
    \begin{array}{c}
    y_{t+1}^1 =y_t^1+u_t^1\,;\\[6pt]
    y_{t+1}^2 = y_t^2-u_t^2\,;
    \end{array}
    \right.\\[2pt]
    y_0^1 =0\,,\enskip y_0^2 =0\,;\\[2pt]
    \left\vert y_t^1 -y_t^2\right\vert \leq 2\,,\enskip t=\overline{0, T}\,.
    \end{gather*}
Агенты $k=1, 2$ стремятся по возможности максимизировать функции выигрыша
\begin{equation*}
J_1=-\sum\limits_{t=1}^{T-1} y_t u_t^1;\quad J_2= \sum\limits_{t=1}^{T-1} y_t u_t^2,
\end{equation*}
где
$$
y_t=y_t^1 - y_t^2,\quad u_t^k \in U_0\equiv \{ -1, 0, 1\}.
$$
    
    Интересы игроков $\tilde{\rho}^k \hm\in \mathrm{REL}( \tilde{X}, 
\tilde{X})$~(\ref{e1.1-vas}) заданы функционалами~$J_k$. В~статических 
подыг\-рах~$\Gamma(X_t)$, $X_t\hm= \cup y_t X(y_t)$, $X_0\hm= U_0^2$, с~по\-мощью 
функций $f^1\hm= -y_t u_t^1$, $f^2\hm= y_t u_t^2$ введем отношения предпочтения 
игроков~$\rho_t^k$. Видно, что интересы участников подыгр противоположны 
$\rho_t^2\hm= \rho_t^{1\mathrm{op}}$, а~множества допустимых ситуаций расслоены 
в~за\-ви\-си\-мости от позиций $y_t\hm= y_t^1\hm- y_t^2$, $\vert y_t\vert \hm\leq 2$, 
в~которых может пребывать динамическая сис\-те\-ма. Ситуация $(u_t^1, u_t^2)\hm\in 
X(y_t)$ допустима, только если ее выбор не нарушает заданное фазовое 
ограничение. 
    
    Во все моменты времени отношения предпочтения $\rho_t^k(y_t): X(y_t) \hm\to 
X(y_t)$, $k\hm=1,2$,  одинаковы: $\rho_t^k \hm= \rho_1^k$, $X_t\hm= X_1$, $t\hm= 
\overline{1,T-1}$, причем
    $$
    \rho_t^k\triangleq \bigcup\limits_{y_t} \rho_t^k (y_t).
    $$

    
    Вычисления показывают, что при $t\hm= \overline{1, T-1}$ имеем
    
    \vspace*{-8pt}
    
    \noindent
    \begin{multline*}
    X_t=\coprod\limits^2_{y_t=-2} X(y_t),\ X(\pm 2) = \{ (\pm 1,\pm1)\},\\
     X(\pm1) = \{ (\pm1,0), (0,\pm1)\},\\
      X(0)=\{ (1,-1), (0,0), (-1,1)\}\,;
    \end{multline*}
    
    \vspace*{-13pt}
    
    \noindent
    \begin{multline*}
    \rho_t^1(0): (0,-1)\sim (-1,0) \sim (0,0),\\
     \rho_t^1(1)=\{ ((0,-1), (-1,0))\},\\
    \rho^1_t(-1) =\{ ((-1,0), (0,-1))\}\,,\\ 
    \rho_1^1(\pm2) =\{ (\pm 1, \pm1)\}.
    \end{multline*}
    
    \vspace*{-4pt}
    
    Динамическое отношение игры, равное $R_{t,t+1}: X_t\hm\to X_{t+1}$, $t\hm= 
\overline{1, T-1}$, связывает до\-пус\-ти\-мую текущую $u_m\hm\equiv (u_m^1, u_m^2) 
\hm\in X(y_m)$ и~выбираемую сле\-ду\-ющую $u_{m+1}\hm\in X(y_{m+1})$ ситуации 
в~состояниях сис\-те\-мы $\Gamma(X_m)$, $m\hm= t, t\hm+1$. Таким образом, 
отношение непосредственного пред\-шест\-во\-ва\-ния ситуаций определено включением
    $$
    u_t \xrightarrow{R_{tt+1}} u_{t+1},\ u_{t+1} \in X(y_{t+1}),\ y_{t+1} \!=\! y_t+\left( 
u_t^1+ u_t^2\right).
    $$
    
    \vspace*{-2pt}
    
    Пусть сетевая структура~(\ref{e2.2-vas}) такова, что во всех играх 
$\Gamma(X_t)$ один из участников сообщает другому свое управ\-ля\-ющее 
воздействие~$u_t^k$. Тогда рациональное решение игры состоит в~выборе 
ситуации, ста\-би\-ли\-зи\-ру\-ющей траекторию сис\-те\-мы $y_t\hm\equiv 0$, а~многошаговая 
игра имеет седловые точки  $\forall_t u_t^1 \hm= -u_t^2$.
    
    При поиске рационального решения игровых задач применяются 
вспомогательные морфизмы, стро\-ящи\-еся из исходных~(\ref{e1.1-vas}) с~по\-мощью 
операций алгебры отношений $A\hm= \left( \circ, \cup, \cap, {}^{\mathrm{op}}, \times; 
\sigma,\varnothing \right)$~\cite{5-vas, 10-vas, 11-vas}. Получение ка\-ким-ли\-бо 
игроком дополнительной информации уменьшает неопределенность выбора 
подходящей стратегии (см.\ пример~2.2). Выбор сетевой структуры  
игры~(\ref{e2.2-vas}) изменяет отношения предпочтений участников~$\rho$. Игроки 
руководствуются сужениями $\rho\vert_A\hm= \rho \cap A^2$, $A\hm\subset X$, 
исходных отношений~$\rho$ на некоторые, вполне определенные подмножества 
ситуаций. 
    
    При условии сделанных предположений граф динамического отношения 
содержит минимальные и~максимальные элементы~--- вершины $X^0$ и~$X^T$ 
(см.~пример~2.1). Процесс функционирования ДМАС начинается с~решения 
статических \mbox{подыгр}~$\Gamma(X^0)$ и~заканчивается подзадачами~$\Gamma(X^T)$. 
Они названы начальным и~терминальным со\-сто\-яни\-ями сис\-те\-мы соответственно. 
В~отличие от примера~2.2, в~рас\-смат\-ри\-ва\-емой по\-ста\-нов\-ке задаются терминальные 
отношения предпочтения агентов~(\ref{e1.1-vas}).

     \begin{figure*}[b] %fig1
\vspace*{1pt}
\begin{minipage}[t]{79mm}
\begin{center}  
    \mbox{%
\epsfxsize=17.323mm 
\epsfbox{vas-1.eps}
}
\end{center}
\end{minipage}
\hfill
\vspace*{1pt}
\begin{minipage}[t]{79mm}
\begin{center}
     \mbox{%
\epsfxsize=25.462mm 
\epsfbox{vas-2.eps}
}
\end{center}
\end{minipage}
\begin{minipage}[t]{79mm}
\vspace*{-12pt}
\Caption{Образ $r=R\circ r^\prime$ и~обратный образ $r^\prime \hm= R^{\mathrm{op}}\circ r$ 
относительно функтора~$R$}
\end{minipage}
%\end{figure*}
\hfill
% \begin{figure*} %fig3.2
      \begin{minipage}[t]{79mm}
      \vspace*{-12pt}
           \Caption{Граф $\Gamma_R$ динамического отношения~$R$: $R_{ij}\hm\in \mathrm{REL}\,(X_i, X_j)$ }
     \label{f3.2-vas}
     \end{minipage}
%     \vspace*{-24pt}
     \end{figure*} 

    
    Рациональное поведение игроков заключается в~том, что они стремятся 
максимизировать свои 
 предпочтения на множестве всех допустимых ситуаций:
    \begin{equation}
    \forall_k \tilde{\rho}^{kT} \to \mathop{\mathrm{MAX}}\limits_{\tilde{X}}\,.
    \label{e2.4-vas}
    \end{equation}
В~задаче~(\ref{e1.1-vas}), (\ref{e2.1-vas}), (\ref{e2.2-vas}), (\ref{e2.4-vas}) требуется 
найти ситуацию равновесия~\cite{1-vas, 5-vas}. 
    
    Существование равновесия во многом зависит от свойств 
морфизмов~$\rho_i^k$, $k\hm\in K$. Наложим одно из таких требований. Пусть 
партнеры предлагают некоторому игроку~$k$ принять решение в~ситуации~$s$, для 
которой $\rho_i^k\vert_{s_k} \hm= \varnothing$, $s\hm\in X_i$. Предполагается, что 
возникшую неопределенность выбора каж\-дый игрок $k\hm \in K$ разрешит  
с~по\-мощью изменения своего отношения предпочтения, положив 
$\rho_i^k\vert_{s_k} \hm= \{ s\}$. (Игрок соглашается с~выбором ситуации~$s$.) Это 
требование выполняется для рефлексивных отношений.
    
\section{Эквивалентные преобразования динамических~игр}

    Категорное представление допускает применение моноидальных операций для 
преобразования игровой задачи~\cite{11-vas}. Морфизмы $R: \mathrm{REL}\left(Y\right)\hm\to 
\mathrm{REL}\,(X)$ представляют собой функторы, срав\-ни\-ва\-ющие однообъектные 
подкатегории в~категории REL. Образ морфизма $\rho\hm\in \mathrm{REL}\,(Y,Y)$ 
относительно~$R$ будем записывать как композицию $R\circ \rho \hm\in \mathrm{REL}\,(X,X)$.
    
    \smallskip
    
    \noindent
    \textbf{Определение~3.1.} Обратным образом (или кообразом) отношения $r: 
X\hm\to X$ относительно морфизма $R: Y\hm\to X$ назовем стрелку $r^\prime: 
Y\hm\to Y$, превращающую сле\-ду\-ющий квад\-рат в~коммутативный  
(рис.~1).
    
    
     
     
    Ввиду того что морфизм $R^{\mathrm{op}} \hm\in \mathrm{REL}\,(X,Y)$ сохраняет композиции 
стрелок, он сам становится функтором $R^{\mathrm{op}}: \mathrm{REL}\,(X)\hm\to \mathrm{REL}\,(Y)$. Так как 
$\forall_{ij} (x,y) \hm\in R_{ij}^{\mathrm{op}} \hm\Leftrightarrow  (y,x)\hm\in R_{ij}$, то 
назовем~$R^{\mathrm{op}}$ противоположным к~$\{ R_{ij}\}$ динамическим отношением. 
Его можно изобразить, изменив на\-прав\-ле\-ния всех дуг в~графе~$\Gamma_R$ 
(вертикальных стрелок на рис.~1). 




    
    Пусть вершины $X_1$ и~$X_l$ графа динамического отношения~$\Gamma_R$ 
связаны некоторым путем $L\hm= (X_1, \ldots, X_l)$. Взяв композицию $R_L\hm= 
R_{l-1}\circ\cdots\linebreak \cdots \circ R_1$ морфизмов $R_i\hm \in \mathrm{REL}\,(X_i, X_{i+1})$ вдоль цепи
$L\hm= (X_1, \ldots , X_l)$, мож\-но <<опустить>> произвольное бинарное отношение 
$r_l: X_l\hm\to X_l$ с~$X_l$ на множество~$X_1$ и~по\-стро\-ить его обратный образ 
$r_1\hm= R_L^{\mathrm{op}} \circ r_l$.
    
    Пусть теперь множество $\{ L\hm= (X_1, \ldots , X_l)\}$ всех попарно 
различных путей, свя\-зы\-ва\-ющих вершины $X_1$ и~$X_l$, состоит более чем из одного 
элемента. Под обратным образом отношения $\rho_l: X_l\hm\to X_l$, переносимого 
на множество~$X_1$, будем понимать копроизведение
    \begin{multline}
    \rho_1^\prime =R^{\mathrm{op}}\circ \rho_l =\coprod\limits_{\{L\}} R_L^{\mathrm{op}} \circ 
\rho_l;\\
 \rho_1^\prime: \coprod\limits_{\{ L\}} X_1\to \coprod\limits_{\{L\}} X_1\,.
    \label{e3.1-vas}
    \end{multline}
    
    <<Поднятием>> морфизма~$\rho_1$ вдоль путей $\{L\}$ мож\-но по\-стро\-ить 
образ $\rho_L\hm= R\circ \rho_1$. Это отношение $\coprod_{\{L\}} R_L\circ \rho_1$ на 
копроизведении объектов $\coprod_{\{L\}} X_l$ с~числом сомножителей, рав\-ным 
мощ\-ности набора~$\{L\}$.
    
    По формуле~(\ref{e3.1-vas}) заданные в~терминальных со\-сто\-яни\-ях сис\-те\-мы 
морфизмы~(\ref{e1.1-vas}) переносятся во все остальные со\-сто\-яния. Так вводятся 
отношения предпочтения игроков~$\rho_i^k$ в~играх~$\Gamma(X_i)$:
    \begin{equation*}
    \forall_{ik} \rho_i^k =R^{\mathrm{op}}\circ \tilde{\rho}_i^{kT}.
%    \label{e3.2-vas}
    \end{equation*}
    
    \noindent
    \textbf{Замечание~3.1.}\ Общее определение кообраза~(\ref{e3.1-vas}) мож\-но 
заменить двойственной конструкцией, основанной на произведении 
отношений~\cite{11-vas}: 
    \begin{equation}
    \rho^{\prime\prime} \!=\!R^{\mathrm{op}} \circ \rho \!=\!\prod\limits_{\{L\}} R^{\mathrm{op}}_L \circ \rho\,;\ \rho\! =\!R\circ \rho^{\prime\prime}= 
\prod\limits_{\{L\}} R_L\circ \rho^{\prime\prime}\!.\!
    \label{e3.3-vas}
    \end{equation}
Получим морфизмы~(\ref{e3.3-vas}) вида 
$$
\rho: \prod\limits_{\{L\}} X_1\to \prod\limits_{\{L\}} X_1,\ \rho^{\prime\prime}: \prod\limits_{\{L\}} X_l\to \prod\limits _{\{L\}} X_l.
$$

    \smallskip
    
    \noindent
    \textbf{Пример~3.1.}\ Воспользуемся формулой~(\ref{e3.1-vas}) 
применительно к~ДМАС, пред\-став\-лен\-ной на рис.~2. 
    
    % \end{multicols}
     


     
  %   \begin{multicols}{2}
    
     
Кообразом $R^{\mathrm{op}}\circ \rho_4$ отношения $\rho_4: X_4\hm\to X_4$ служит морфизм 
\begin{multline*}
\rho_1^\prime: X_1\coprod X_1 \to X_1\coprod X_1;
\\
\rho_1^\prime =\left( R_{13}^{\mathrm{op}} \circ R_{34}^{\mathrm{op}}\coprod R_{12}^{\mathrm{op}} \circ 
R_{24}^{\mathrm{op}}\right)\circ \rho_4.
\end{multline*}
Опуская~$\rho_4$ на множества $X_2$ и~$X_3$, получим $\rho_2^\prime \hm= 
R_{24}^{\mathrm{op}} \circ \rho_4$, $\rho_3^\prime\hm= R_{34}^{\mathrm{op}} \circ \rho_4$ 
соответственно. Поднятием отношения $\rho_1^\prime: X_1\hm\to X_1$ на 
множество~$X_4$ строится образ  $R\circ \rho_1^\prime$ как
\begin{multline*}
\rho_4: X_4\coprod X_4 \to X_4\coprod X_4\,;\\
 \rho_4=\left( R_{24}\circ R_{12} \coprod 
R_{34} \circ R_{13}\right) \circ \rho_1^\prime.
\end{multline*}
    
    Конструкция~(\ref{e3.3-vas}) дает структуру
\begin{multline*}
    \rho_4: X_4\times X_4 \to X_4\times X_4\,;\\
     \rho_4=\left( R_{24}\circ R_{12} \prod 
R_{34} \circ R_{13}\right) \circ \rho_1^{\prime\prime}.
\end{multline*}
    
    Формулы~(\ref{e3.1-vas}) и~(\ref{e3.3-vas}) позволяют строить эквивалентные 
модели игры. Свойство универсальности объектов $X_2\times X_3$ и~$X_2\coprod 
X_3$, вы\-ра\-жа\-емое по\-средст\-вом коммутативных диаграмм~\cite{11-vas}, однозначно 
определяет новое динамическое отношение с~со\-от\-вет\-ст\-ву\-ющи\-ми морфизмами. 
Например, вместо графа из рис.~2 мож\-но работать с~более прос\-той 
графовой структурой, пред\-став\-лен\-ной в~любой из сле\-ду\-ющих форм:
    \begin{gather}
    \mbox{\fbox{$X_1$}} \xrightarrow{\left\langle R_{12}, R_{13}\right\rangle} 
\mbox{\fbox{$X_2\times X_3$}} \xrightarrow{\left\langle R_{24}^{\mathrm{op}}, 
R_{34}^{\mathrm{op}}\right\rangle^{\mathrm{op}}} \mbox{\fbox{$X_4$}}\,;
    \label{e3.4-vas}\\
    \mbox{\fbox{$X_1$}} \xrightarrow{[ R_{12}, R_{13}]^{\mathrm{op}}} 
\mbox{\fbox{$X_2\coprod X_3$}} \xrightarrow{[ R_{24}, R_{34}]} 
\mbox{\fbox{$X_4$}}\,.
    \label{e3.5-vas}
    \end{gather}
Возможность эквивалентного перехода от произвольного графа~$\Gamma_R$ (см.\ 
рис.~2) к~цепи~(\ref{e2.3-vas}) (см.\ (\ref{e3.4-vas}) и~(\ref{e3.5-vas})) 
обоснована в~тео\-ре\-ме~3.1. 
    
    \smallskip
    
    \noindent
    \textbf{Теорема~3.1.} \textit{Граф всякого динамического отношения 
приводится к~виду}~(\ref{e2.3-vas}).

\vspace*{-6pt}

\section{Результирующее отношение динамической игры}

    Введением результирующего отношения игра сводится к~проблеме 
оптимального управ\-ле\-ния, по\-став\-лен\-ной в~форме отношений. Для ее решения\linebreak 
мож\-но применить обобщенный метод динамического программирования. 
Напомним~\cite{5-vas}, что в~любой статической игре $\Gamma(Y)$ существует 
ре\-зуль\-ти\-ру\-ющее отношение~$P_Y$. Оно выражает \mbox{композициональное} свойство 
игры, учи\-ты\-ва\-ющее интересы всех участников операции, сетевую структуру их 
взаимодействия и~ра\-ци\-о\-наль\-ность поведения. Решение игры $\Gamma(Y)$ сводится
 к~поиску максимальных элементов $P_Y\hm\to \mathrm{MAX}_Y$. Благодаря\linebreak этому, мож\-но 
по-но\-во\-му определить состояния \mbox{исходной} ДМАС. Вместо игр $\Gamma(X_i)$ 
будем рас\-смат\-ри\-вать оптимизационные задачи $(\tilde{X}_i, P_{\tilde{X}_i})$, 
$P_{\tilde{X}_i} \hm\in \mathrm{REL}\,(\tilde{X}_i, \tilde{X}_i)$. Иначе говоря, теперь во всех 
со\-сто\-яни\-ях сис\-те\-мы решение принимает единственный агент. 
    
    %\smallskip
    
    \noindent
    \textbf{Определение~4.1.}\ Пусть $(\tilde{Y}_1, P_{\tilde{Y}_1} ) 
\overset{R}{\prec} (\tilde{Y}_2, P_{\tilde{Y}_2})$; $\tilde{y}_1, \tilde{y}_2 \hm\in 
\tilde{Y}_1$. Ситуация~$\tilde{y}_2$ называется более перспективной по сравнению 
с~$\tilde{y}_1$, если выполнено свойство
    \begin{equation}
    \left( \tilde{y}_1, \tilde{y}_2\right) \in \left( R^{\mathrm{op}}\circ P_{\tilde{Y}_2} \right)\circ 
P_{\tilde{Y}_1}.
    \label{e4.1-vas}
    \end{equation}
    
    В каждом состоянии ДМАС игрокам целесообразно использовать более 
перспективные ситуации и~из них формировать рациональное решение 
задачи~(\ref{e1.1-vas}), (\ref{e2.1-vas}), (\ref{e2.2-vas}), (\ref{e2.4-vas}). В~этом 
заключается прин\-цип Бел\-лмана. 
    
    В случае динамического отношения с~графом~(\ref{e2.3-vas}) (см.\ 
тео\-ре\-му~3.1) рас\-смот\-рим сле\-ду\-ющую итерационную схему по\-стро\-ения 
<<оп\-ти\-ми\-зи\-ру\-ющих>> морфизмов $\{ \tilde{P}_{\tilde{X}_k}\}$:

\vspace*{-4pt}

\noindent
    \begin{multline}
    \tilde{P}_{\tilde{X}_T} =P_{X_T},\ \tilde{P}_{\tilde{X}_{T-k}} =\left( 
R^{\mathrm{op}}\circ \tilde{P}_{\tilde{X}_{T-k+1}}\right) \circ P_{\tilde{X}_{T-k}},\\
    k=\overline{1, T}\,.
    \label{e4.2-vas}
    \end{multline}
    
    \vspace*{-4pt}

\noindent
Назовем~(\ref{e4.2-vas}) уравнениями Беллмана в~форме отношений. 
    
    \smallskip
    
    \noindent
    \textbf{Определение~4.2.} Динамическим ре\-зуль\-ти\-ру\-ющим отношением 
называется семейство морфизмов Беллмана $\{ \tilde{P}_{\tilde{X}_k},\ k\hm= 
\overline{1, T}\}$ из уравнений~(\ref{e4.2-vas}).
    
    \smallskip
    
    \noindent
    \textbf{Теорема~4.1.} \textit{Во всякой ДМАС существует ре\-зу\-ль\-ти\-ру\-ющее 
отношение.}
    
    \smallskip
    
    \noindent
    Д\,о\,к\,а\,з\,а\,т\,е\,л\,ь\,с\,т\,в\,о\,.\ \ Построение ДРО проведем методом 
математической индукции. На начальном шаге, в~терминальном со\-сто\-янии сис\-те\-мы, 
ДРО совпадает с~ре\-зуль\-ти\-ру\-ющим отношением $\tilde{P}_{X^T} \triangleq 
P_{X^T}$, $\tilde{X}^T\hm= X^T$, статической игры~$\Gamma(X^T)$. Опус\-тим 
этот морфизм на все множества $\tilde{X}^{T-1}$, для которых имеет мес\-то 
непосредственное предшествование $\Gamma(\tilde{X}^{T-1}) \overset{R}{\prec} 
\Gamma(\tilde{X}^T)$. Композиция морфизмов~(\ref{e4.1-vas}), $\tilde{Y}_1\hm= 
\tilde{X}^{T-1}$, $\tilde{Y}_2\hm= \tilde{X}^T$, определяет компоненту ДРО 
$\tilde{P}_{\tilde{X}^{T-1}}$, от\-ве\-ча\-ющую со\-сто\-янию сис\-те\-мы $(\tilde{X}^{T-1}, 
P_{\tilde{X}^{T-1}})$. Пусть отношение~$\tilde{P}_{\tilde{X}_i}$ уже построено. 
Тогда опять по формуле~(\ref{e4.1-vas}) его можно продолжить на все со\-сто\-яния 
$\Gamma(\tilde{X}_j) \overset{R_{ij}}{\prec} \Gamma(\tilde{X}_i)$ исходной ДМАС, 
т.\,е.\ построить очередные морфизмы~$\tilde{P}_{\tilde{X}_j}$. Процесс 
завершается в~начальных состояниях сис\-темы. 

\vspace*{-11pt}

\section{Поиск рационального решения игры}

\vspace*{-1pt}

    Для решения задачи~(\ref{e1.1-vas}), (\ref{e2.1-vas}), (\ref{e2.2-vas}),  
(\ref{e2.4-vas}) предложим сле\-ду\-ющее обобщение метода динамического\linebreak 
программирования. Сначала из системы уравнений Беллмана~(\ref{e4.2-vas}) 
найдем ДРО игры. Затем последовательно для моментов времени $1,\ldots, T$ 
вычислим сле\-ду\-ющие множества максимальных \mbox{элементов} отношения Беллмана:

\pagebreak



\noindent
    \begin{multline}
    X_1^*=\mathop{\mathrm{ARGMAX}}\,\tilde{P}_{\tilde{X}_1};\enskip 
    X_2^*=\mathop{\mathrm{ARGMAX}}\limits_{RX_1^*}\, \tilde{P}_{\tilde{X}_2}; \ldots  \\
\ldots ;    X_T^* = \mathop{\mathrm{ARGMAX}}\limits_{RX^*_{T-1}}\,\tilde{P}_{\tilde{X}_T}.
       \label{e5.1-vas}
    \end{multline}
    
\vspace*{-3pt}

    \noindent
    \textbf{Теорема~5.1.}\ \textit{Рациональному решению динамической игровой 
задачи}~(\ref{e1.1-vas})--(\ref{e2.4-vas}) \textit{отвечает ситуация 
$\tilde{x}^*\hm= (x_1^*, \ldots , x_T^*)$; $x_s^*\hm\in X_s^*$, $s\hm= \overline{1, T}$, 
найденная методом Беллмана}~(\ref{e4.2-vas}), (\ref{e5.1-vas}).
    
    \smallskip
    
    \noindent
    \textbf{Пример~5.1.} Графом~$\Gamma_R$~(\ref{e2.3-vas}), $n\hm=2$, 
динамического отношения задана ДМАС 

\vspace*{-4pt}

\noindent
    \begin{multline*}
    R: (x_1, x_2, x_3) \to (x_3, x_1, x_1\vee x_2) \cup (x_2, x_3, x_1\wedge x_2);\\
    x=(x_1, x_2, x_3)\in\underline{8}\simeq \{0,1\}^3.
    \end{multline*}
    
    \vspace*{-3pt}
    
    \noindent
Терминальная задача представляет собой игру Гермейера трех лиц 
$\Gamma^{2,3}$~\cite{1-vas, 5-vas}, где $X^T\hm=\underline{8}$~--- бинарный куб. 
Отношения предпочтений агентов $\tilde{\rho}_2^k \hm\in \mathrm{REL}\,(\underline{8},\underline{8})$ равны:
\begin{align*}
\tilde{\rho}_2^1 &=\{ 01, 05, 40, 34, 32, 76\};\\
\tilde{\rho}_2^2&= \{ 02, 10, 24, 32, 35, 45, 64, 67\};\\
\tilde{\rho}_2^3 &=\{ 20, 40, 35, 31, 64, 76\}.
\end{align*}
Найдем интересы игроков $\rho_1^k\hm= R^{\mathrm{op}}(\tilde{\rho}_2^k)$ в~начальном 
состоянии сис\-те\-мы~$\Gamma^{1,3}$, $X^0\hm= \underline{8}$: 

\vspace*{-4pt}

\noindent
\begin{multline*}
\rho_1^1= \{ 01, 02, 03, 04, 06, 12, 13, 14, 16, 20, 21, 62, 63,\\
 64, 65, 72, 73, 74, 75\}\,;
 \end{multline*}
 
 \vspace*{-14pt}
 
 \noindent
 \begin{multline*}
 \rho_1^2=\{ 03, 04, 05, 13, 14, 15, 20, 21,23, 26, 32, 40, 41,\\
  42, 52, 61, 63, 64, 65, 71, 73, 74, 75, 76\};
  \end{multline*}
  
  \vspace*{-14pt}
 
 \noindent
 \begin{multline*}
\rho_1^3 = \{ 20, 21, 30, 31, 40, 41, 50, 51, 61, 62, 63, 64, 71,\\
 72, 73, 74, 76\}.
\end{multline*}

\vspace*{-5pt}

\noindent
Вычислим результирующие отношения статических подыгр $\Gamma(X^T)$, 
$\Gamma(X^0)$ ~\cite{5-vas}:

\vspace*{-4pt}

\noindent
\begin{multline*}
P_{X^T} =\tilde{\rho}_1^3 \vert_{x_3} \circ \tilde{\rho}_1^2 \circ \left( \tilde{\rho}_1^1 
\cup \tilde{\rho}_1^{2G}\right),\\
\rho_{1}^{2G} =\tilde{\rho}_1^2 \vert_{\mathrm{MIN} \tilde{\rho}_1^2\vert_{{x_1}}}
\Rightarrow P_{X^T} =\{ 40, 05, 42, 35, 30, 70\},\\
P_{X^0} =\rho_1^1\vert_{x_1} \circ \rho_1^2\vert_{x_2}\circ\rho_1^3\vert_{x_3} =\{ 41, 53, 61, 73\}.
\end{multline*}

\vspace*{-4pt}

\noindent
Из уравнений Беллмана~(\ref{e4.2-vas}) найдем искомое ДРО игры:
$$
\tilde{P}_{X^T} =\{ 40, 05, 42, 35, 30, 70\},\ \tilde{P}_{X^0} = \{40, 00, 60, 70\}.
$$

\vspace*{-3pt}

\noindent
По формулам~(\ref{e5.1-vas}) и~тео\-ре\-ме~5.1 рациональное решение игры равно 
$\tilde{x}^*\hm=(0,0)$.

\vspace*{-12pt}

\section{Заключение}

\vspace*{-4pt}

    Функторная модель стала наследником традиционной формы представления 
ДМАС. Обладая композициональной структурой, она создает условия для 
модификации игровой задачи и~выполнения эквивалентных преобразований 
средствами алгебры тео\-рии категорий. Доказано существование ре\-зуль\-ти\-ру\-юще\-го 
отношения динамической игры многих лиц. Для задач оптимального управ\-ле\-ния 
в~форме динамических отношений предложено обобщение метода Беллмана.  
С~по\-мощью ДРО этим методом строится рациональное решение задачи. 
Функторный подход поддерживается компьютерной алгеброй тео\-рии категорий. 
Сетевая архитектура применяемых морфизмов допускает эффективную 
нейросетевую программную реализацию, которую еще только пред\-сто\-ит 
осуществить. 

\vspace*{-12pt}

{\small\frenchspacing
 {\baselineskip=10.6pt
 %\addcontentsline{toc}{section}{References}
 \begin{thebibliography}{99}
 
\vspace*{-3pt}

\bibitem{2-vas}
\Au{Моисеев~Н.\,Н.} Элементы теории оптимальных сис\-тем.~--- М.: Наука, 1974. 526~с.

\bibitem{1-vas} %2
\Au{Гермейер~Ю.\,Б.} Игры с~непротивоположными интересами.~--- М.: Наука, 1976. 326~с.

\bibitem{3-vas}
\Au{Красовский Н.\,Н., Субботин~А.\,И.} Позиционные дифференциальные игры.~--- M.: Наука, 
1976. 456~с.
\bibitem{4-vas}
\Au{Dockner E.\,J., Jorgensen~S., Long~N.\,V., Sorger~G.} Differential games in economics and management
science.~--- Cambridge: Cambridge University 
Press, 2000. 382~p. doi: 10.1017/CBO9780511805127.
\bibitem{5-vas}
\Au{Васильев Н.\,С.} Композициональное представление структуры игры многих лиц 
в~моноидальной категории бинарных отношений~// Информатика и~её применения, 2023. Т.~17. 
Вып.~2. С.~18--26. doi: 10.14357/19922264230203. EDN: GPMZTS.
\bibitem{9-vas}  %6
\Au{Dixit~A.\,K., Natebuff~B.\,J.} The art of strategy.~--- New York; London: W.\,W.~Norton~\&~Co., 
2008. 446~p.

\bibitem{7-vas}
\Au{Shoham~Y., Leyton-Brown~R.} Multiagent systems: Algorithmic, game-theoretic, and logical 
foundations.~--- Cambridge: Cambridge University Press, 2010. 532~p.
\bibitem{6-vas} %8
\Au{Bai~Q., Ren~F., Fujita~K., Znang~M.} Multi-agent and complex systems.~--- Studies in computational 
intelligence ser.~--- Springer Singapore, 2016. Vol.~670. 210~p.
\bibitem{8-vas} %9
\Au{Dixit~A.\,K., Skeath~S., Reiley~W.\,W., Jr.} Games of strategy.~--- New York; London: 
W.\,W.~Norton \&~Co., 2017. 880~p.

\bibitem{10-vas}
\Au{Скорняков Л.\,А.} Элементы общей алгебры.~--- М.: Наука, 1983. 272~с.

%\pagebreak

\bibitem{11-vas}
\Au{Маклейн~С.} Категории для работающего математика~/ Пер.\ с~англ.~--- М.: Физматлит, 2004. 352~с. 
(\Au{Mac Lane~S.} Categories for the working mathematician.~--- 2nd ed.~--- New York, NY, USA: 
Springer, 1998. 318~p.)
\bibitem{12-vas}
\Au{Губко М.\,В.} Управление организационными системами с~сетевым взаимодействием агентов. 
Обзор теории сетевых игр~// Автоматика и~телемеханика, 2004. №\,8. С.~115--132.
\bibitem{13-vas}
Group formation in economics: Networks, clubs, and coalitions~/ Eds.\ G.~Demange, M.~Wooders.~--- 
Cambridge: Cambridge University Press, 2005. 475~p.

\end{thebibliography}

 }
 }

\end{multicols}

\vspace*{-9pt}

\hfill{\small\textit{Поступила в~редакцию 02.02.24}}

%\vspace*{6pt}

\pagebreak

%\newpage

\vspace*{-28pt}

%\hrule

%\vspace*{2pt}

%\hrule



\def\tit{ON FUNCTOR REPRESENTATION OF~OPTIMIZED\\ DYNAMIC~MULTIAGENT~SYSTEMS}


\def\titkol{On functor representation of~optimized dynamic multiagent systems}


\def\aut{N.\,S.~Vasilyev}

\def\autkol{N.\,S.~Vasilyev}

\titel{\tit}{\aut}{\autkol}{\titkol}

\vspace*{-15pt}


\noindent
N.\,E.~Bauman Moscow State Technical University, 5-1~Baumanskaya 2nd Str., Moscow 105005, Russian 
Federation

\def\leftfootline{\small{\textbf{\thepage}
\hfill INFORMATIKA I EE PRIMENENIYA~--- INFORMATICS AND
APPLICATIONS\ \ \ 2024\ \ \ volume~18\ \ \ issue\ 2}
}%
 \def\rightfootline{\small{INFORMATIKA I EE PRIMENENIYA~---
INFORMATICS AND APPLICATIONS\ \ \ 2024\ \ \ volume~18\ \ \ issue\ 2
\hfill \textbf{\thepage}}}

\vspace*{4pt}



\Abste{Functors' topoi is chosen as a computational tool for synthesizing dynamic multiagent systems (DMAS). The scale orders the objects as 
multiagent system states to solve attendant static subgames in them. 
The initial dynamic game and all static subproblems are represented in the monoidal category of binary relations. 
Players' preference relations might be maximized in DMAS. The game rational solution is understood as
 equilibrium. The 
compositional structure of the optimized DMAS can be described in the form of the game dynamic resulting relation (DRR). 
Players' rational behavior search is reduced to DRR subsequent maximization. For this purpose, the Bellman's method 
is generalized to solve control problems stated in the form of relations. 
The program implementation of the approach can be based on neural networks due to the consistency of the architectures 
of the applied relation graphs and neural networks.}

\KWE{functor category; compositionality; monoidal category; opposite image; game dynamic relation; 
static subgame; preference relation; dynamic resulting relation; rational solution; Bellman morphism}

\DOI{10.14357/19922264240201}{CLMBXC}

%\vspace*{-12pt}

%\Ack

%\vspace*{-3pt}

%    \noindent
 

  \begin{multicols}{2}

\renewcommand{\bibname}{\protect\rmfamily References}
%\renewcommand{\bibname}{\large\protect\rm References}

{\small\frenchspacing
 {%\baselineskip=10.8pt
 \addcontentsline{toc}{section}{References}
 \begin{thebibliography}{99} 

\bibitem{2-vas-1}
\Aue{Moiseev, N.\,N.} 1975. \textit{Elementy teorii optimal'nykh sistem} [Elements of optimal systems 
theory]. Moscow: Nauka. 527~p.

\bibitem{1-vas-1}
\Aue{Germeyer, Yu.\,B.} 1976. \textit{Igry s~neprotivopolozhnymi interesami} [Games with  
nonopposing interests]. Moscow: Nauka. 326~p.

\bibitem{3-vas-1}
\Aue{Krasovskiy, N.\,N., and A.\,I.~Subbotin.} 1974. \textit{Pozitsionnye differentsial'nye igry} 
[Positional differential games]. Moscow: Nauka. 456~p.
\bibitem{4-vas-1}
\Aue{Dockner, E.\,J., S.~Jorgensen, N.\,V.~Long, and G.~Sorger.} 2000. \textit{Differential games in 
economics and management science}. Cambridge: Cambridge University Press. 382~p. doi: 
10.1017/CBO9780511805127.
\bibitem{5-vas-1}
\Aue{Vasilyev, N.\,S.} 2023. Kompozitsional'noe predstavlenie struktury igry mnogikh lits 
v~monoidal'noy kategorii binarnykh otnosheniy [Multiplayers' games compositional structure in the 
monoidal category of binary relations]. \textit{Informatika i~ee Primeneniya~--- Inform. Appl.} 
 17(2):18--26. doi: 10.14357/19922264230203. EDN: GPMZTS.
 
 \bibitem{9-vas-1} %6
\Aue{Dixit, A.\,K., and B.\,J.~Nalebuff.} 2008. \textit{The art of strategy}. New York, London: 
W.\,W.~Norton \&~Co. 446~p.


\bibitem{7-vas-1}
\Aue{Shoham, Y., and R.~Leyton-Brown.} 2010. \textit{Multiagent systems: Algorithmic,  
game-theoretic, and logical foundations}. Cambridge University Press. 532~p.

\bibitem{6-vas-1} %8
\Aue{Bai, Q., F.~Ren, K.~Fujita, and M.~Znang.} 2016. \textit{Multi-agent and complex systems}. Studies 
in computational intelligence ser.  Springer Singapore. 210~p.

\bibitem{8-vas-1} %9
\Aue{Dixit, A.\,K., S.~Skeath, and D.\,H.~Reiley, Jr.} 2017. \textit{Games of strategy}. New York, 
London: W.\,W.~Norton \&~Co.\linebreak 880~p.

\bibitem{10-vas-1}
\Aue{Skornyakov, L.\,A.} 1983. \textit{Elementy obshchey algebry} [Elements of general algebra]. 
Moscow: Nauka. 272~p.
\bibitem{11-vas-1}
\Aue{Mac Lane, S.} 1998. \textit{Categories for the working mathematician}. 2nd ed. New York, NY: 
Springer. 318~p. 
\bibitem{12-vas-1}
\Aue{Gubko, M.\,V.} 2004. Control of organizational systems with network interaction of agents. 
II.~Stimulation problems. \textit{Automat. Rem. Contr.} 65(9):1470--1485. doi: 
10.1023/B:AURC.0000041425.34118.7d. EDN: \mbox{LFMUCG}.
\bibitem{13-vas-1}
Demange, G., and M.~Wooders, eds. 2005. \textit{Group formation in economics: Networks, clubs, and 
coalitions}. Cambridge: Cambridge University Press. 475~p. 

\end{thebibliography}

 }
 }

\end{multicols}

\vspace*{-8pt}

\hfill{\small\textit{Received February 2, 2024}} 

\vspace*{-18pt}


\Contrl

\vspace*{-3pt}

\noindent
\textbf{Vasilyev Nikolai S.} (b.\ 1952)~--- Doctor of Science in physics and mathematics, professor, 
N.\,E.~Bauman Moscow State Technical University, 5-1 Baumanskaya 2nd Str., Moscow 105005, Russian 
Federation; \mbox{nik8519@yandex.ru}





\label{end\stat}

\renewcommand{\bibname}{\protect\rm Литература}   %
\def\stat{gaidamaka}

\def\tit{МЕТОД РАСЧЕТА ХАРАКТЕРИСТИК ИНТЕРФЕРЕНЦИИ
ДВУХ ВЗАИМОДЕЙСТВУЮЩИХ УСТРОЙСТВ
В~БЕСПРОВОДНОЙ ГЕТЕРОГЕННОЙ СЕТИ$^*$}

\def\titkol{Метод расчета характеристик интерференции
двух взаимодействующих устройств
в~беспроводной гетерогенной сети}

\def\aut{Ю.\,В.~Гайдамака$^1$, А.\,К.~Самуйлов$^2$}

\def\autkol{Ю.\,В.~Гайдамака, А.\,К.~Самуйлов}

\titel{\tit}{\aut}{\autkol}{\titkol}

{\renewcommand{\thefootnote}{\fnsymbol{footnote}} \footnotetext[1]
{Работа выполнена при финансовой поддержке РФФИ (проекты 14-07-00090
и~15-07-03051).}}


\renewcommand{\thefootnote}{\arabic{footnote}}
\footnotetext[1]{Российский университет дружбы народов, ygaidamaka@sci.pfu.edu.ru}
\footnotetext[2]{Российский университет дружбы народов; Технологический университет г.\ Тампере, Финляндия,
aksamuylov@gmail.com}


\Abst{Одним из показателей качества функционирования современных беспроводных сетей
является отношение сигнала к~сумме интерференции и шума (SINR, Signal to Interference plus
Noise Ratio) в~беспроводных каналах связи. Поскольку значение этой характеристики
существенно зависит от расстояния между интерферирующими устройствами, задача оценки
значения SINR часто сводится к~вычислению длины одной из сторон треугольника,
в~вершинах которого находятся взаимодействующие устройства. В~данной статье решается
задача нахождения математического ожидания и~среднеквадратического отклонения
отношения сигнал/интерференция пары взаимодействующих устройств в достаточно общих
предположениях о~распределении случайных величин (с.в.)\ расстояний между
интерфери\-ру\-ющи\-ми устройствами. Предложенный метод может быть использован
в~качестве основы для анализа интерференции в~гетерогенной сети с~применением различных
беспроводных технологий, включая анализ беспроводных взаимодействий оконечных
устройств, на которые интерференция оказывает наиболее сильное воздействие.}

\KW{беспроводная сеть; LTE; интерференция; SINR; взаимодействие устройств; D2D}

\DOI{10.14357/19922264150102}


\vskip 14pt plus 9pt minus 6pt

\thispagestyle{headings}

\begin{multicols}{2}

\label{st\stat}

\section{Постановка задачи}

  В современных беспроводных сетях, построенных на базе технологии LTE
(Long Term Evolution), оценка интерференции между взаимодействующими
устройствами является одной из основных задач анализа показателей качества
функционирования~[1,~2]. Под интерференцией понимается взаимодействие
сигналов, передаваемых разными\linebreak источниками на одном и~том же канале.
Интерференция вызывает искажение сигнала рас\-смат\-ри\-ва\-емо\-го источника под
воздействием сигнала сторонне\-го источника. В~гетерогенных сетях
беспроводного взаимодействия оконечных устройств D2D
  (device-to-device)~[3], где плот\-ность интерферирующих объектов высока,
интерференция оказывает существенное влияние на принимаемый оконечным
устройством сигнал. При анализе беспроводных взаимодействий устройств
обычно рассматривается несколько источников сигнала (передатчиков),
распределенных на плоскости согласно некоторому закону~[4]. Упрощение
задачи состоит в~том, что, рассмотрев один передатчик и~оценив
характеристики интерференции на соответствующем ему приемном устройстве
(приемнике), можно предположить, что основные показатели будут идентичны
и~для остальных пар <<пе\-ре\-дат\-чик--при\-ем\-ник>>. В~данной статье
решается задача нахождения числовых характеристик отношения
сигнал/ин\-тер\-фе\-рен\-ция пары взаимодействующих устройств.

  Отношение сигнала к сумме интерференции и~шума, SINR,
  является одной из основных характеристик качества канала
  в~беспроводных сетях связи~[5--7]. Отношение сигнала к~сумме интерференции 
и~шума на стороне приемника определяется по следующей формуле:
  \begin{equation}
  \mathrm{SINR} = \fr{S}{\sigma^2 +I}\,,
  \label{e1-gai}
  \end{equation}
где $S$~--- мощность принимаемого сигнала от соответствующего
передатчика; $\sigma^2$~--- мощность шума; $I$~--- мощность принимаемого
сигнала от интерферирующих передатчиков. Согласно линейной модели~[4]
\begin{equation}
S=gl^{-\alpha}\,,
\label{e2-gai}
\end{equation}
где $g$~--- базовая мощность сигнала передатчика, соответствующего
рассматриваемому приемнику; $l$~--- расстояние между передатчиком
и~приемником; $\alpha$~--- коэффициент потерь (path loss exponent),
принимающий значение от~2 (при условии прямой видимости) до~6 (в~худшем
случае). Величина~$I$ в~знаменателе формулы~(1) соответствует суммарной
мощности сигнала от всех интерферирующих передатчиков, где каждое
слагаемое имеет вид~(2). Заметим, что принцип повторного использования
частот (frequency reuse) в~беспроводных сетях связи поколения 4G (4th
Generation) позволяет назначать одну и~ту же единицу ресурса сети (например,
один и~тот же ресурсный блок LTE) нескольким парам взаимодействующих
устройств, если интерференция не превосходит определенного стандартами
уровня.

  Рассмотрим случай, когда несколько принимающих устройств (приемников)
и~одно передающее устройство (передатчик), образующие кластер,
расположены на плоскости внутри круга радиуса~$r_0$, причем передатчик
расположен в центре круга. Такой кластер образуется, например, при
проведении интерактивного занятия преподавателя с учениками, когда можно
предположить, что передатчик располагается в центре круга, а приемники
равномерно распределены внутри круга. Для передачи данных на каждую пару
взаимодействующих устройств внутри кластера планировщиком распределения
радиоресурсов в беспроводной сети 4G назначается по одному ресурсному
блоку LTE, и тогда сигналы взаимодействующих пар не интерферируют друг
с~другом. Но если в соседнем помещении также проходит интерактивное
занятие и там использованы те же ресурсные блоки, то пары из соседних
кластеров, использующие один и тот же ресурсный блок, будут создавать
помехи друг другу. Сведем задачу к анализу взаимодействия двух пар
устройств в двух кластерах, как показано на рис.~1.



  Пару взаимодействующих устройств, для которой будем рассчитывать
показатели качества канала, назовем целевой, а соответствующую ей пару
устройств обозначим TR$_0\hm= \langle \mathrm{Tx}_0, \mathrm{Rx}_0\rangle$.
Остальные пары, которые создают помехи целевой паре 
$\mathrm{TR}_0$,\linebreak\vspace*{-12pt}
\begin{center}  %fig1
\vspace*{8pt}
\mbox{%
 \epsfxsize=77.569mm
 \epsfbox{gai-1.eps}
 }
\end{center}

\noindent
{{\figurename~1}\ \ \small{Схема взаимодействия интерферирующих устройств}}


%\vspace*{9pt}


\addtocounter{figure}{1}


\noindent
 обозначим $\mathrm{TR}_i\hm= \langle
\mathrm{Tx}_i, \mathrm{Rx}_i\rangle$ и~будем называть их интерферирующими. Расстояние
между Rx$_i$ и~Tx$_i$ обозначим $R_i$, а~расстояние между Tx$_0$ и~Tx$_i$
обозначим~$U_i$. Мощность интерферирующего сигнала от пары TR$_i$
является функцией расстояния между приемником Rx$_0$ из целевой пары и
интерферирующим передатчиком~Tx$_i$, которое обозначим~$D_i$. Угол
между прямой, соединяющей целевые передатчик~Tx$_0$ и~приемник~Rx$_0$,
и~прямой, соеди\-ня\-ющей передатчики~Tx$_0$ и~Tx$_i$,
обозначим~$\gamma_i$.

  Рассмотрим систему двух кластеров, показанную на рис.~1. В~условиях
отсутствия шума и~одинаковой базовой мощности~$g$ сигналов обоих
передатчиков искомой характеристикой является отношение
  сигнал/ин\-тер\-фе\-рен\-ция SIR для приемника~Rx$_0$, вычисляемое по
формуле:
  \begin{equation}
\mathrm{SIR}=\left( \fr{D_1}{R_0}\right)^{\alpha}\,.
  \label{e3-gai}
  \end{equation}

  Будем считать, что $R_0$, $U_i$ и~$\gamma_i$ являются
с.в.\ с~заданными функциями распределения. Задача состоит
в~нахождении числовых характеристик с.в.~SIR. Для
решения задачи в следующем разделе статьи предлагается метод нахождения
совместной плотности распределения с.в.~$R_0$ и~$D_i$, что позволяет
вычислять начальные моменты ${\sf E}[\mathrm{SIR}^n]$ с.в.~SIR.

\vspace*{-6pt}

\section{Метод расчета отношения сигнал/интерференция}

%\vspace*{-2pt}

  Как видно из формулы~(3), с.в.~SIR пропорциональна с.в.~$D_1$, которая,
в~свою очередь, зависит от с.в.~$R_0$. В~этом случае для нахождения
характеристик с.в.~SIR необходимо найти совместное распределение
с.в.~$R_0$ и~$D_1$.

  Введем обозначения $\xi_1{:=} R_0$, $\xi_2 {:=} U_1$, $\xi_3 {:=}
\gamma_1$, $\eta_1 {:=} D_1$. Тогда $w_{\xi_1,\xi_2,\xi_3}(x_1,x_2,x_3) {:=}$\linebreak
${=:}\;f_{R_0, U_1, \gamma_1}(x_1,x_2,x_3)$~--- совместная плот\-ность
распределения с.в.~$R_0$, $U_1$ и~$\gamma_1$, а~$W_{\xi_1,\eta_1}(x_1,y_1)
{:=} f_{R_0, D_1}(x_1,y_1)$~--- искомое совместное распределение с.в.~$R_0$
и~$D_1$. По теореме косинусов с.в.~$\eta_1$ является функцией с.в.~$\xi_1$,
$\xi_2$ и~$\xi_3$:
  \begin{equation}
  \eta_1=\sqrt{\xi_1^2+\xi_2^2-2\xi_1\xi_2\cos \xi_3}\,.
  \label{e4-gai}
  \end{equation}

  Следуя~\cite{8-gai, 9-gai}, введя вспомогательную
переменную $\eta_2\hm=\xi_3$, искомое распределение можно найти по
следующей формуле:
  \begin{multline}
W_{\xi_1, \eta_1} (y_1,y_2) ={}\\
{}=\sum\limits_{i=1}^2
\int\limits_{\mathrm{Y}_{3,j}}\!\!\! w_{\xi_1,\xi_2,\xi_3}\left(
y_1,\varphi_i(y_1,y_2,y_3),y_3\right) \times{}\\[-6pt]
{}\times
\left\vert \fr{\partial \varphi_j(y_1,y_2,y_3)}
{\partial y_2}\right\vert\,dy_3\,,
  \label{e5-gai}
  \end{multline}
где $\varphi_j$~--- обратное преобразование правой части формулы~(\ref{e4-gai})
относительно~$\xi_2$:
\begin{align*}
\varphi_1(y_1,y_2,y_3) &= y_1\cos y_3 +\sqrt{y_2^2-y_1^2+y_1^2\cos^2 y_3}\,;\\
\varphi_2(y_1,y_2,y_3) &= y_1\cos y_3 -\sqrt{y_2^2-y_1^2+y_1^2\cos^2 y_3}\,.
\end{align*}

  В формуле~(\ref{e5-gai}) области значений Y$_{3,j}$ переменной~$y_3$ для
$j$-й вет\-ви обратного преобразования определяются системой неравенств:
  \begin{equation}
  \left.
  \begin{array}{c}
  \varphi_j(y_1,y_2,y_3)\geq0\,;\\[6pt]
  y_1\geq 0\,;\\[6pt]
  y_2\geq 0\,;\\[6pt]
  0\leq y_3\leq 2\pi\,.
  \end{array}
  \right\}
  \label{e6-gai}
  \end{equation}

  Решая систему~(\ref{e6-gai}), нетрудно убедиться, что для первой ветви
обратного преобразования  $\mathrm{Y}_{3,1}\hm= \mathrm{Y}_{3,1}^1\cup
\mathrm{Y}_{3,1}^2\cup \mathrm{Y}_{3,1}^3$, где
  \begin{align}
  \hspace*{-2mm}\mathrm{Y}_{3,1}^1 &=\begin{cases}
  0\leq y_2\leq y_1;\\
  0\leq y_3\leq \fr{1}{2}\,\mathrm{arccos}\,\left( \fr{y_1^2-
2y_2^2}{y_1^2}\right);\end{cases}
  \label{e7-1-gai}
\\
\hspace*{-2mm}\mathrm{Y}_{3,1}^2 &= \begin{cases}
  0\leq y_2\leq y_1;\\
  2\pi -\fr{1}{2}\,\mathrm{arccos}\left( \fr{y_1^2-2y_2^2}{y_1^2}\right) \leq
y_3\leq 2\pi;
  \end{cases}\!\!\!\!\!
  \label{e7-2-gai}
  \\
\hspace*{-2mm}\mathrm{Y}_{3,1}^3 &= \begin{cases}
  y_2\geq y_1;\\
  0\leq y_3\leq 2\pi,
  \end{cases}\!\!\!\!\!\!\!\!\!
  \label{e7-3-gai}
  \end{align}
а для второй ветви  $\mathrm{Y}_{3.2}=\mathrm{Y}_{3,2}^1\cup
\mathrm{Y}_{3,2}^2$, где
\begin{align}
\label{e8-1-gai}
\hspace*{-2mm}\mathrm{Y}_{3,2}^1 &= \begin{cases}
0\leq y_2\leq y_1\,;\\
0\leq y_3\leq \fr{1}{2}\,\mathrm{arccos}\left( \fr{y_1^2-2y_2^2}{y_1^2}\right);
\end{cases}
\\
\hspace*{-2mm}\mathrm{Y}_{3,2}^2 &=\begin{cases}
0\leq y_2\leq y_1\,;\\
2\pi -\fr{1}{2}\,\mathrm{arccos} \left( \fr{y_1^2-2y_2^2}{y_1^2}\right) \leq y_3\leq
2\pi.\!\!\!\!\!\!\!\!
\end{cases}
\label{e8-2-gai}
\end{align}

  Таким образом, получена формула для вычисления совместной плотности
с.в.~$R_0$ и~$D_1$:
  \begin{multline}
  W_{\xi_1,\eta_1}(y_1,y_2) ={}\\
  {}=\sum\limits_{i=1}^2 \int\limits_{\mathrm{Y}_{3,i}}
\fr{w_{\xi_1,\xi_2,\xi_3} (y_1,\varphi_i(y_1,y_2,y_3),y_3) y_2} {\sqrt{y_2^2-
y_1^2+y_1^2\cos^2 y_3}}\,dy_3\,,
  \label{e9-gai}
  \end{multline}
где $\mathrm{Y}_{3,j}$ вычисляются по
формулам~(\ref{e7-1-gai})--(\ref{e8-2-gai}).

  В следующем разделе приведен пример численного анализа
с~использованием формул~(\ref{e7-1-gai})--(\ref{e9-gai}).

\section{Пример численного анализа}

  В рассматриваемом примере предложенный выше метод использован для
расчета начальных моментов ${\sf E}[\mathrm{SIR}^n]$ отношения сигнал/интерференция,
которые определяются следующей формулой:
  \begin{multline}
{\sf   E}[\mathrm{SIR}^n] ={}\\
{}=\int\limits_{0\leq y_1\leq r_0} \int\limits_{y_2\geq0} \left(
\fr{y_2}{y_2}\right)^{n\alpha} W_{\xi_1,\eta_1}(y_1,y_2)\,dy_2dy_1\,.
  \label{e10-gai}
  \end{multline}

  Рассматривается случай, когда целевой приемник Rx$_0$ находится внутри
круга единичного \mbox{радиуса} ($r_0\hm=1$), в центре которого расположен
передат\-чик~Tx$_0$, а~интерферирующий передатчик~Tx$_1$~--- в~кольце
вокруг передатчика~Tx$_0$ с~внутренним радиусом~$r_0$ и~внешним
радиусом~$h_0$ (рис.~2).

\begin{center}  %fig2
\vspace*{8pt}
\mbox{%
 \epsfxsize=77.111mm
 \epsfbox{gai-2.eps}
 }


\noindent
{{\figurename~2}\ \ \small{Пример взаимодействия двух устройств}}

\end{center}


\vspace*{9pt}


\addtocounter{figure}{1}


%\noindent


  Тогда с.в.~$R_0$ расстояния от целевого передатчика Tx$_0$ до
соответствующего ему приемника Rx$_0$ и~с.в.~$U_1$ расстояния от целевого
передатчика~Tx$_0$ до интерферирующего передатчика~Tx$_1$ имеют
распределения
  \begin{alignat*}{2}
  f_{R_0}(r) &= 2r\,,&\quad 0&\leq r\leq1\,;\\
  f_{U_1}(u) &= \fr{2u}{h_0^2-1}\,,&\quad 1&\leq u\leq h_0\,.
  \end{alignat*}
Будем считать, что с.в.\ угла~$\gamma_1$ равномерно распределена на отрезке
$[0,\,2\pi]$, а~коэффициент потерь в~формуле~(2) принимает значение
$\alpha\hm=2$. Приняты условные единицы измерения: например, расстояние
между взаимодействующими устройствами может измеряться в~метрах,
а~величина SIR~--- в~децибелах.

  По формулам~(\ref{e7-1-gai})--(\ref{e10-gai}) рассчитано математическое
ожидание отношения сигнал/ин\-тер\-фе\-рен\-ция ${\sf E}[\mathrm{SIR}]$, представленное в
таблице в зависимости от радиуса внешней границы кольца, внутри которого
распределены интерферирующие передатчики. В~таблице также показаны
значения математического ожидания расстояния  ${\sf E}[U_1]$ от целевого
передатчика~Tx$_0$ до интерферирующего передатчика~Tx$_1$.

%  \begin{table*}\small
  \begin{center}
  \begin{tabular}{|c|c|c|}
  \multicolumn{3}{p{48mm}}{Математическое ожидание величины~SIR}\\
  \multicolumn{3}{c}{\ }\\[-5pt]
  \hline
\ \ \ \ $h_0$\ \ \ \ &\ \ \ \ ${\sf E}[U_1]$\ \ \ \ &${\sf E}[\mathrm{SIR}]$\\
\hline
2&1,56&4,84985\\
3&2,17&7,41701\\
4&2,8\hphantom{9}&9,54562\\
5&3,44&11,30286\hphantom{9}\\
\hline
\end{tabular}
\end{center}
%\end{table*}

\begin{center}  %fig3
\vspace*{18pt}
\mbox{%
 \epsfxsize=77.754mm
 \epsfbox{gai-3.eps}
 }
 \end{center}


\noindent
{{\figurename~3}\ \ \small{Числовые характеристики отношения сигнал/ин\-тер\-фе\-рен\-ция: \textit{1}~---
${\sf E}[\mathrm{SIR}]$; \textit{2}~--- $\sigma_{\mathrm{SIR}}$}}

\vspace*{18pt}

  Также были рассчитаны математическое ожидание ${\sf E}[\mathrm{SIR}]$
  и~среднеквадратическое \mbox{отклонение} $\sigma_{\mathrm{SIR}}\hm= \sqrt{{\sf E}[\mathrm{SIR}^2]-
{\sf E}[\mathrm{SIR}]^2}$ отношения сигнал/ин\-тер\-фе\-рен\-ция, показанные на рис.~3 в
зави\-си\-мости от математического ожидания расстояния ${\sf E}[U_1]$ между
целевым передатчиком~Tx$_0$ и~интерферирующим передатчиком~Tx$_1$. Из
таблицы и~графиков видно, что с~ростом расстояния между целевым
и~интерферирующим передатчиком обе \mbox{числовые} характеристики отношения
сиг\-нал/ин\-тер\-фе\-рен\-ция растут, поскольку мощность интерферирующего сигнала
убывает. Вычисления проводились с~использованием встроенных средств
пакета программ Wolfram Mathematica~[10].



\section{Заключение}

  В настоящей статье метод преобразования с.в.\ применен для
анализа основной характеристики качества функционирования беспроводных
сетей, а~именно: отношения сигнал/ин\-тер\-фе\-рен\-ция при заданных
распределениях расстояний между интерферирующими устройствами.
Приведенный пример показывает, что чис\-лен\-ный анализ является достаточно
трудоемким даже в простейших предположениях о~распределении исходных
с.в., а~для оценки характеристик интерференции в~условиях
наличия в~беспроводной сети нескольких источников интерференции требуется
разработка приближенных методов и~имитационных моделей, как это сделано,
например, в~[11]. Задача с несколькими источниками интерференции
в~беспроводных гетерогенных сетях взаимодействующих устройств
представляется особенно актуальной ввиду быст\-ро\-го развития сетей 4G
и~принятия в~ближайшем будущем стандартов для беспроводных сетей 5G~[12].
{\looseness=1

}

  \bigskip

Авторы выражают благодарность проф.\ К.\,Е.~Самуйлову за
плодотворное обсуждение и ценные советы.


{\small\frenchspacing
 {%\baselineskip=10.8pt
 \addcontentsline{toc}{section}{References}
 \begin{thebibliography}{99}
\bibitem{1-gai}
\Au{Гайдамака~Ю.\,В., Ефимушкина~Т.\,В., Самуйлов~А.\,К., Самуйлов~К.\,Е.} Задачи
оптимального планирования межуровневого интерфейса в беспроводных сетях~//
Информатика и~её применения, 2012. Т.~6. Вып.~3. С.~75--81.
\bibitem{2-gai}
\Au{Basharin G.\,P., Gaidamaka Yu.\,V., Samouylov~K.\,E.} Mathematical theory of teletraffic and
its application to the analysis of multiservice communication of next generation networks~//
Autom. Control Comp. Sci., 2013. Vol.~47. No.\,2. P.~62--69.
\bibitem{3-gai}
\Au{Andreev S., Pyattaev A., Johnsson~K., Galinina~O., Koucheryavy~Y.} Cellular traffic
offloading onto network-assisted device-to-device connections~// IEEE Commun. Mag.,
2014. Vol.~52. No.\,4. {\sf http://ieeexplore.ieee.org/\linebreak xpl/tocresult.jsp?isnumber=6807935}.
\bibitem{4-gai}
\Au{Baccelli F., Blaszczyszyn B.} Stochastic geometry and wireless networks. Vol.~I: Theory.~---
Boston: NoW Publs. Inc., 2009. 164~p.


\bibitem{6-gai} %5
\Au{Erturk M.\,C., Mukherjee S., Ishii~H., Arslan~H.} Distributions of transmit power and SINR in
device-to-device networks~// IEEE Commun. Lett., 2013. Vol.~17. No.\,2. {\sf
http://ieeexplore.ieee.org/xpl/tocresult.jsp?isnumber=\linebreak 6472443}.

\bibitem{7-gai} %6
\Au{Kim M., Han Y., Yoon~Y., Chong~Y., Lee~H.} Modeling of adjacent channel interference in
heterogeneous wireless networks~// IEEE Commun. Lett., 2013. Vol.~17. No.\,9. {\sf
http://ieeexplore.ieee.org/xpl/tocresult.jsp?isnumber=\linebreak 6604524}.

\bibitem{5-gai} %7
\Au{Andrews J.\,G., Singh S., Ye~Q., Lin~X., Dhillon~H.\,S.} An overview of load balancing in
hetnets: Old myths and open problems~// IEEE Wirel. Commun., 2014. Vol.~21. No.\,2.
{\sf http://ieeexplore.ieee.org/xpl/tocresult.\linebreak jsp?isnumber=6812279}.


\bibitem{8-gai}
\Au{Левин Б.\,Р.} Теоретические основы статистической радиотехники.~--- 3-е изд.~--- М.:
Радио и связь, 1989. 656~с.
\bibitem{9-gai}
\Au{Mardia K., Jupp P.} Directional statistics.~--- Wiley Press, 1999. 441~p.
\bibitem{10-gai}
Wolfram Mathematica: Программное обеспечение для технических вычислений. {\sf
http://www.wolfram.\linebreak com/mathematica}.
\bibitem{11-gai}
\Au{Гайдамака Ю.\,В., Печинкин А.\,В., Разумчик~Р.\,В., Самуйлов~А.\,К., Самуйлов~К.\,Е.,
Соколов~И.\,А., Сопин~Э.\,С., Шоргин~С.\,Я.} Распределение времени выхода из множества
состояний перегрузки в системе $M\vert M\vert 1\vert \langle L,H\rangle \vert \langle
H,R\rangle$ с~гистерезисным управлением нагрузкой~// Информатика и~её применения,
2013. Т.~7. Вып.~4. С.~20--33.
\bibitem{12-gai}
\Au{Tehrani M., Uysal M., Yanikomeroglu~H.} Device-to-device communication in 5G cellular
networks: Challenges, solutions, and future directions~// IEEE Commun. Mag., 2014.
Vol.~52. No.\,5. {\sf http://ieeexplore. ieee.org/xpl/tocresult.jsp?isnumber=6815882}.
 \end{thebibliography}

 }
 }

\end{multicols}

\vspace*{-3pt}

\hfill{\small\textit{Поступила в редакцию 20.01.15}}

%\newpage

\vspace*{12pt}

\hrule

\vspace*{2pt}

\hrule

%\vspace*{12pt}

\def\tit{METHOD FOR CALCULATING NUMERICAL
CHARACTERISTICS OF~TWO DEVICES INTERFERENCE
FOR~DEVICE-TO-DEVICE COMMUNICATIONS
IN~A~WIRELESS HETEROGENEOUS NETWORK}

\def\titkol{Method for calculating numerical
characteristics of~two devices interference
for~D2D communications
in~a~wireless %heterogeneous
network}

\def\aut{Yu.~Gaidamaka$^1$ and A.~Samuylov$^{1,2}$}

\def\autkol{Yu.~Gaidamaka and A.~Samuylov}

\titel{\tit}{\aut}{\autkol}{\titkol}

\vspace*{-9pt}

 \noindent
$^1$Peoples' Friendship University of Russia,
Applied Probability and Informatics Department,
6~Miklukho-Maklaya\linebreak
$\hphantom{^1}$Str., Moscow 117198, Russian Federation

\noindent
$^2$Tampere University of Technology,
Department of Electronics and Communications Engineering,
10 Korkeak-\linebreak
$\hphantom{^1}$oulunkatu,  Tampere 33720, Finland


\def\leftfootline{\small{\textbf{\thepage}
\hfill INFORMATIKA I EE PRIMENENIYA~--- INFORMATICS AND
APPLICATIONS\ \ \ 2015\ \ \ volume~9\ \ \ issue\ 1}
}%
 \def\rightfootline{\small{INFORMATIKA I EE PRIMENENIYA~---
INFORMATICS AND APPLICATIONS\ \ \ 2015\ \ \ volume~9\ \ \ issue\ 1
\hfill \textbf{\thepage}}}

\vspace*{3pt}


\Abste{In wireless networks, one of the key performance metrics is the signal to noise ratio, SINR. As this metric
highly depends on the distance between the interfering devices, the problem of SINR estimation is often reduced to the
calculation of a triangle's side length, where the vertices represent the interacting devices. This paper addresses the
problem of calculating the numerical characteristics of the signal to interference ratio for a pair of interfering devices
determined by the known distributions of distances between the entities in question. The proposed method can be used
as a basis for analyzing heterogeneous networks, including the analysis of
device-to-device (D2D) communications as one of
the interference-limited cases.}

\KWE{wireless network; LTE; interference; SINR; D2D}




\DOI{10.14357/19922264150102}

\Ack
\noindent
The reported study was partially supported by the Russian Foundation for Basic
Research,  research projects Nos.\,14-07-00090 and
15-07-03051.



%\vspace*{3pt}

  \begin{multicols}{2}

\renewcommand{\bibname}{\protect\rmfamily References}
%\renewcommand{\bibname}{\large\protect\rm References}



{\small\frenchspacing
 {%\baselineskip=10.8pt
 \addcontentsline{toc}{section}{References}
 \begin{thebibliography}{99}
\bibitem{1-gai-1}
\Aue{Gaidamaka, Yu.\,V., T.\,V. Efimushkina, A.\,K.~Samuylov, and K.\,E.~Samouylov}. 2012.
Zadachi optimal'nogo planirovaniya mezhurovnevogo interfeysa v besprovodnykh setyakh
[Cross-layer optimization planning problems in wireless networks]. \textit{Informatika i~ee
Primeneniya}~--- \textit{Inform. Appl.} 6(3):75--81.
\bibitem{2-gai-1}
\Aue{Basharin, G.\,P., Yu.\,V. Gaidamaka, and K.\,E.~Samouylov}. 2013. Mathematical theory of
teletraffic and its application to the analysis of multiservice communication of next generation
networks. \textit{Autom. Control Comp. Sci.} 47 (2):62--69.
\bibitem{3-gai-1}
\Aue{Andreev, S., A. Pyattaev, K.~Johnsson, O.~Galinina, and Y.~Koucheryavy}. 2014. Cellular
traffic offloading onto network-assisted device-to-device connections. \textit{IEEE
Commun. Mag.} 52(4). Available at: {\sf
http://ieeexplore.ieee.\linebreak org/xpl/tocresult.jsp?isnumber=6807935} (accessed January~10, 2015).
\bibitem{4-gai-1}
\Aue{Baccelli, F., and B. Blaszczyszyn.} 2009. \textit{Stochastic geometry and wireless networks}.
Vol.~I: Theory. Boston: NoW Publs. Inc. 164~p.



January~10, 2015). %5
\bibitem{6-gai-1}
\Aue{Erturk, M.\,C., S. Mukherjee, H.~Ishii, and H.~Arslan}. 2013. Distributions of transmit
power and SINR in device-to-device networks. \textit{IEEE Commun. Lett.} 17(2).
Available at: {\sf http://ieeexplore.ieee.org/xpl/tocresult.jsp?isnumber=\linebreak 6472443} (accessed
January~10, 2015).
\bibitem{7-gai-1} %6
\Aue{Kim, M., Y. Han, Y.~Yoon, Y.~Chong, and H.~Lee}. 2013. Modeling of adjacent channel
interference in heterogeneous wireless networks. \textit{IEEE Commun. Lett.} 17(9).
Available at: {\sf http://ieeexplore.ieee.org/\linebreak xpl/tocresult.jsp?isnumber=6604524} (accessed
January~10, 2015).

\bibitem{5-gai-1} %7
\Aue{Andrews, J.\,G., S. Singh, Q.~Ye, X.~Lin, and H.\,S.~Dhillon}. 2014. An overview of load
balancing in hetnets: Old myths and open problems. \textit{IEEE Wirel. Commun.} 21(2).
Available at: {\sf http://ieeexplore.ieee.org/\linebreak xpl/tocresult.jsp?isnumber=6812279} (accessed

\bibitem{8-gai-1}
\Aue{Levin, B.\,R.} 1989. \textit{Teoreticheskie osnovy statisticheskoy radiotekhniki} [Theoretical
basis of statistical radiotechnics]. 3rd ed. Moscow: Radio and Communications. 656~p.
\bibitem{9-gai-1}
\Aue{Mardia, K., and P. Jupp}. 1999. \textit{Directional statistics}. 1st ed. Wiley Press. 441~p.
\bibitem{10-gai-1}
Wolfram mathematica: Software for technical computing. [Free access] Available at: {\sf
http://www.wolfram.\linebreak com/mathematica} (accessed December~1, 2014).
\bibitem{11-gai-1}
\Aue{Gaidamaka, Yu.\,V., A.\,V. Pechinkin, R.\,V.~Razumchik, A.\,K.~Samuylov,
K.\,E.~Samouylov, I.\,A.~Sokolov, E.\,S.~Sopin, and S.\,Ya.~Shorgin}. 2013. Raspredelenie
vremeni vykhoda iz mnozhestva sostoyaniy peregruzki v~sisteme $M\vert M\vert 1\vert \langle
L,H\rangle \vert \langle H,R\rangle$ s~gisterezisnym upravleniem nagruzkoy [The distribution of
the return time from the set of overload states to the set of normal load states in a system $M\vert
M\vert 1\vert \langle L,H\rangle \vert \langle H,R\rangle$ with hysteretic load control].
\textit{Informatika i~ee~Primeneniya}~--- \textit{Inform. Appl.} 7(4):20--33.
\bibitem{12-gai-1}
\Aue{Tehrani, M., M. Uysal, and H.~Yanikomeroglu.} 2014. Device-to-device communication in
5G cellular networks: Challenges, solutions, and future directions.
\textit{IEEE Commun. Mag.} 52(5). Available at: {\sf http://ieeexplore.\linebreak ieee.org/xpl/tocresult.jsp?isnumber=6815882}
(accessed January~10, 2015).
\end{thebibliography}

 }
 }

\end{multicols}

\vspace*{-3pt}

\hfill{\small\textit{Received January 20, 2015}}

%\vspace*{-18pt}


\Contr

\noindent
\textbf{Gaidamaka Yuliya V.} (b.\ 1971)~---
Candidate of Science (PhD) in physics and mathematics, associate
professor, Applied Probability and Informatics Department, Peoples' Friendship University of Russia,
6~Miklukho-Maklaya Str., Moscow 117198, Russian Federation;
ygaidamaka@sci.pfu.edu.ru

\vspace*{3pt}

\noindent
\textbf{Samuylov Andrey K.} (b.\ 1988)~---
PhD student, Peoples' Friendship University of Russia, Moscow 117198, Russian
Federation; researcher, Department of Electronics and Communications Engineering,  Tampere
University of Technology, 10 Korkeakoulunkatu, Tampere 33720, Finland;
aksamuylov@gmail.com

\label{end\stat}

\renewcommand{\bibname}{\protect\rm Литература}  
\def\stat{krivenko}

\def\tit{МНОГОМЕРНЫЙ РЕФЕРЕНСНЫЙ РЕГИОН\\ ВЫСОКОЙ ПЛОТНОСТИ}

\def\titkol{Многомерный референсный регион высокой плотности}

\def\aut{М.\,П.~Кривенко$^1$}

\def\autkol{М.\,П.~Кривенко}

\titel{\tit}{\aut}{\autkol}{\titkol}

\index{Кривенко М.\,П.}
\index{Krivenko M.\,P.}


%{\renewcommand{\thefootnote}{\fnsymbol{footnote}} \footnotetext[1]
%{Работа выполнена при финансовой поддержке РФФИ (проекты 16-07-00677 
%и~15-37-20611-мол\_а\_вед).}}


\renewcommand{\thefootnote}{\arabic{footnote}}
\footnotetext[1]{Институт проблем информатики Федерального исследовательского центра <<Информатика и~управление>> Российской академии наук,
\mbox{mkrivenko@ipiran.ru}}

\vspace*{4pt}



\Abst{Рассматриваются принципы построения многомерных референсных регионов
(MRR~--- multivariate reference region). 
Предложен оригинальный метод построения региона на основе областей с~высокой 
плотностью точек и~аппроксимации распределения данных с~помощью смеси нормальных 
распределений. Для оценки порога для плотности распределения используется  
бут\-стреп-ме\-тод. В~качестве эксперимента рассмотрена задача построения 
и~использования эталонной области для прогнозирования типа мочевого камня. Обработка 
реальных данных продемонстрировала преимущества предлагаемых решений.}

\KW{многомерный референсный регион; область высокой плотности; бут\-стреп-ме\-тод; 
смесь многомерных нормальных распределений}

\vspace*{6pt}

\DOI{10.14357/19922264170207} 


\vskip 10pt plus 9pt minus 6pt

\thispagestyle{headings}

\begin{multicols}{2}

\label{st\stat}

\section{Введение}

     Многомерный референсный регион 
был предложен в~литературе по клинической химии в~начале 1970-х~гг.\ как 
альтернатива одномерным референсным интервалам~[1]. Там излагались 
преимущества предлагаемых множественных тестов, хоть и~имеющих 
упрощенный вид, но снижающих (по отношению к~одномерным вариантам) 
число ложных положительных результатов. Появление MRR оказалось 
особенно привлекательным для интерпретации результатов наборов 
медицинских тестов. Тем не менее возникали трудности в~построении 
и~использовании процедур многомерного анализа (см., например,~[2]), 
связанные, в~частности, с~быстрым увеличением числа параметров, которые 
должны быть оценены. Немногие лаборатории использовали MRR в~своей 
практике, причем в~экспериментальном режиме, и,~как следствие, на 
сегодняшний день имеется относительно малое количество соответствующих 
публикаций. 

\vspace*{-6pt}

\section{Многомерный референсный регион на основе расстояния Махалонобиса}

\vspace*{-2pt}

     Одномерный референсный интервал, полученный статистическим путем, 
использует центральную часть значений анализируемого показателя, обычно 
соответствующую~95\% некоторой популяции~--- совокупности особей 
определенного вида (например, здоровой части населения определенного пола 
из некоторого диапазона возрастов). Одномерные референсные интервалы 
применялись в~течение многих лет в~качестве стандартного приема 
интерпретации лабораторных данных. Они легко формируются, хранятся, 
извлекаются и~передаются в~лабораторных информационных системах, просты 
в~понимании, хорошо воспринимаются медицинским сообществом в~ходе 
длительного использования. Тем не менее одномерные референсные интервалы 
при классификации данных могут дать большое число ложно аномальных 
результатов. Этот далеко не единственный недостаток однофакторного 
референсного интервала может быть полностью или частично устранен 
с~помощью MRR.
     
     Простейшим и~весьма распространенным способом построения MRR 
является использование прямого произведения отдельных референсных 
интервалов в~предположении, что они статистически независимы. Пусть 
$(1\hm-\alpha)$~--- вероятность попадания в~MRR, а~$p_0$~--- вероятность 
попадания в~референсный интервал для любого из~$d$~признаков, тогда 
$p_0\hm= \sqrt[d]{1-\alpha}$. С~ростом размерности~$d$ значения~$p_0$ 
быстро приближаются к~1, что фактически лишает смысла применение MRR.
     
     Как и~в одномерном случае, отправной точкой для построения MRR 
может стать нормальное распределение. Идеи центрального расположения 
референсного региона и~заданной вероятности попадания в~него приводят для 
$d$-мер\-но\-го нормального распределения, имеющего плотность 
распределения
     \begin{multline*}
     \varphi(y,\mu,\Sigma) ={}\\
     {}=(2\pi)^{-d/2}\vert\Sigma\vert^{-1/2}\exp \left( -\fr{\left(y-
\mu\right)^{\mathrm{T}} \Sigma^{-1}(y-\mu)}{2}\right),
   \end{multline*}
где величина $(y-\mu)^{\mathrm{T}} \Sigma^{-1} (y-\mu)$ есть квадрат так 
называемого расстояния Махаланобиса между~$y$ и~$\mu$, к~использованию 
многомерного эллипсоида
\begin{multline*}
(2\pi)^{-d/2}\vert\Sigma\vert^{-1/2}\exp \left( -\fr{\left(y-\mu\right)^{\mathrm{T}}
\Sigma^{-1} 
(y-\mu)}{2}\right) ={}\\
{}=const
\end{multline*}
или, что то же самое, 
$$ 
(y-\mu)^{\mathrm{T}} \Sigma^{-1}(y-\mu)=const\,.
$$
Его называют эллипсоидом равной плотности распределения (или просто 
эллипсоидом равной вероятности). 
     
     Если задаться вероятностью $(1\hm-\alpha)$ попадания в~эллипсоид 
равной вероятности вида $(y\hm-\mu)^{\mathrm{T}}\Sigma^{-1} (y\hm-\mu)\hm= 
\rho$, то параметр~$\rho$ удовлетворяет уравнению $\mathrm{Pr}\left\{ 
\chi_d^2\leq \rho\right\} \hm=1\hm-\alpha$.
     
     Использование эллипсоида в~качестве MRR будет оправдано только 
тогда, когда исходное распределение данных есть многомерное нормаль-\linebreak ное. 
Поэтому становятся актуальными критерии\linebreak подгонки, а~также использование 
процедур норма\-ли\-зации распределения данных в~многомерном\linebreak случае.
 Если 
с~помощью тестов выявляется, что распределение не является нормальным, то 
Международная федерация клинической химии и~лабораторной медицины 
рекомендует, согласно~[3], использовать двухступенчатую процедуру 
нормализации. Следует обратить внимание, что многошаговость здесь 
относится не к~многомерности, а касается лишь покоординатного 
преобразования распределения данных к~нормальному.
     
     Первые же попытки применения MRR на основе расстояния 
Махалонобиса (фактически это означает принятие модели нормального 
распределения референсных значений) выявили ряд недостатков (более 
подробно смотри в~\cite[разд.~6.2]{4-kri}):
     \begin{itemize}
\item проявление <<проклятий>> размерности при механическом 
увеличении~$d$, в~особенности если игнорируется этап анализа состава 
признаков~[1, 5, 6];
\item из-за небольших объемов обучающей выборки невысокая устойчивость 
при применении, в~частности чувствительность к~увеличению неточностей 
измерений после того, как регион был установлен~\cite{5-kri, 7-kri}. 
\item предположение о нормальном распределении и~попытки <<подправить>> 
действительность с~помощью преобразований реальных данных для их 
нормализации при увеличении размерности данных становятся все более 
шаткими~\cite{5-kri};
\item представление и~интерпретация выводов на основе MRR трудно 
понимаемы не только для специалистов в~предметной области~[8].
\end{itemize}

\vspace*{-9pt}

\section{Многомерный референсный регион высокой плотности}

\vspace*{-2pt}

     Заметим, что в~случае нормального распределения референсных значений 
для точек внут\-ри построенного эллипсоида значения плотности\linebreak распределения 
больше, чем на границе, а~вне~--- меньше. Это замечание позволяет 
предложить другой подход к~построению MRR.
     
     \smallskip
     
     \noindent
     \textbf{Определение.}\ Eсли плотность распределения референсных 
значений есть $f(y)$, то MRR есть область $A_t\hm= \left\{ y\in 
\mathcal{R}^d\vert f(y)\hm\geq t\right\}$ для некоторого порогового 
значения~$t$. 
     
     \smallskip
     
     Для нормального распределения это уже упомянутый эллипсоид равной 
вероятности. Если задается вероятность $(1\hm-\alpha)$ попадания в~$A_t$, то 
пороговое значение~$t$ есть решение уравнения $\int\nolimits_{A_t} 
f(u)\,du\hm=1\hm-\alpha$, получить которое аналитически в~случае 
произвольной плотности распределения вряд ли возможно. Здесь присутствуют 
две проблемы: вычисление многомерного интеграла и~зависимость области 
интегрирования от неизвестного значения. Для решения их предлагается 
привлечь метод моделирования.
     
     Сгенерируем выборку из $f(y)$, которую обозначим как $Y^f\hm= \left\{ 
y_1^f, \ldots, y_m^f\right\}$. Для оценки $\int\nolimits_{A_t} f(u)\,du$ 
используем отношение:

\noindent
\begin{multline*}
     \fr{\left\vert \left\{ y_i^f\vert y_i^f\in A_t\right\}\right\vert }{m} =
      \fr{\left\vert\left\{ y_i^f\vert 
f\left(y_i^f\right) \geq t\right\}\right\vert }{m} ={}\\
{}= 1-\fr{\left\vert \left\{ y_i^f\vert f(y_i^f)<t\right\}\right\vert }{m}=1-
F_m(t)\,,
     \end{multline*}
где $F_m(t)$~--- эмпирическая функция распределения случайной 
величины~$f(y)$, т.\,е.\ случайной величины, являющейся результатом 
преобразования с~помощью функции~$f(\cdot)$ случайной величины, име\-ющей 
плотность распределения~$f(u)$.

     Таким образом, искомая оценка~$t^*$ должна удовле\-тво\-рять уравнению 
$F_m(t^*)\hm=\alpha$ и~может быть получена как непараметрическая оценка 
квантиля\linebreak\vspace*{-12pt}

\pagebreak

\noindent
 порядка~$\alpha$ из распределения $F_m(\cdot)$. Если обозначить 
$f_i\hm= f(y_i^f)$, то~$t^*$ есть~$f_{(r)}$, где
     $$
     r= \begin{cases}
     m\alpha, &\ m\alpha~\mbox{---~целое}\,;\\
     \lfloor m\alpha+1\rfloor\,, & m\alpha~\mbox{--- не целое}\,.
     \end{cases}
     $$
     Заметим, что для такой оценки можно указать доверительный интервал.
     
     Для построения MRR необходимо знать распределение данных. При 
реализации принципа точек высокой плотности в~первую очередь следует 
обратиться к~параметрическим моделям, в~част\-ности к~смеси нормальных 
распределений, име\-ющей плотность распределения
     $$
     f(u) =\sum\limits_{j=1}^k p_j \varphi\left (u,\mu_j, \Sigma_j\right)\,.
     $$
Если $\hat{f}(u)$~--- оценка смеси, то~$t^*$ строится сле\-ду\-ющим образом:
\begin{itemize}
\item генерируется выборка $\left\{ y_1^f,\ldots , y_m^f\right\}$ из $\hat{f}(u)$ и~
для каждого ее $i$-го элемента подсчитывается значение $\hat{f}\left( 
y_i^f\right)$;
\item в~качестве~$t^*$ берется непараметрическая оценка квантиля 
порядка~$\alpha$ (в случае необходимости дополнительно находится 
непараметрическая оценка доверительного интервала для~$t^*$, что 
может характеризовать правильность выбранного объема для 
генерируемой выборки).
\end{itemize}

     Пусть для $f(u)$ имеется~$A_t$, а также получена $\hat{f}(u)$ 
и~соответствующий MRR вида~$\hat{A}_t$. Качество аппроксимации~$A_t$ 
с~по\-мощью~$\hat{A}_t$ можно оценить с~по\-мощью вероятности совпадения 
этих областей, т.\,е. 
     $$
     P_c= \int\limits_{\{ u\in A_t\}\cup \{u\in \hat{A}_t\}} \hspace*{-6mm}
f(u)\,du+\int\limits_{\{u\not\in A_t\} \cup\{ u\not\in \hat{A}_t\}}\hspace*{-6mm} f(u)\,du\,.
     $$
     
     Для оценки  $P_c$ можно использовать величину
     \begin{multline*}
     \hat{P}_c= \fr{\left\vert \left\{ 
     y_i^f\vert y_i^f \in \left\{\left\{ y_i^f\in A_t\right\}\cup \left\{y_i^f\in 
\hat{A}_t\right\}\right\}\right\}\right\vert}{m}+{}\\
{}+\fr{\left\vert \left\{ y_i^f\vert y_i^f \in \left\{\left\{ y_i^f\not\in A_t\right\}\cup 
\left\{ y_i^f\not\in \hat{A}_t\right\}\right\}\right\}\right\vert}{m}\,.
     \end{multline*}
     
     Использование MRR высокой плотности для диагностирования сводится 
к~реализации так называемого слабого критерия значимости для наблюденного 
значения~$x$: нулевая гипотеза заключается в~том, что $x\hm\in A_t$, 
статистика критерия есть $\hat{f}(x)$ и~решение о~принадлежности 
критической об\-ласти~$A_t$ принимается при больших значениях~$\hat{f}(x)$.
     
     Для медицинской практики важна возможность использования 
референсного региона при интерпретации результатов обследования 
некоторого пациента с~вектором признаков~$x$. В~подобных случаях 
сложившейся практикой для слабых критериев значимости является 
использование критического уровня~$\alpha_{\mathrm{cr}}$ (более распространенным 
в~медицине является употребление термина $p$-зна\-че\-ние)  $\alpha_{\mathrm{cr}}\hm= 
\mathrm{Pr}\left\{ \hat{f}(y)\hm\leq \hat{f}(x)\right\}$, где $y$~--- случайная 
величина, имеющая плотность распределения~$\hat{f}(u)$, а $\hat{f}(x)$~--- 
значение плотности распределения~$\hat{f}(u)$ в~точке~$x$. Эта 
характеристика дает представление о~том, насколько сильно данное 
наблюденное значение~$x$ противоречит гипотезе (или подкрепляет ее) 
о~принадлежности данных MRR. При выбранном же заранее уровне 
значимости с~помощью~$\alpha_{\mathrm{cr}}$ сразу же можно принять конкретное 
решение. 

\vspace*{-9pt}

\section{Эксперименты}

\vspace*{-2pt}

     Для демонстрации возможностей MRR использовались данные по 
прогнозу химического состава мочевых камней по метаболическим 
показателям мочи и~сыворотки крови, а также антропологическим 
характеристикам пациентов~[9]. В качестве исходной классификации камней 
рассматривалась следующая: чисто оксалатные (далее обозначены как O), чисто 
уратные (U), чисто фосфатные (P), смесь только оксалатных и~уратных (OU), 
смесь только оксалатных и~фосфатных (OP), смесь только уратных 
и~фосфатных (UP), все остальные. Данная классификация была построена 
в~[10] на основе доминирующих частот встречаемости основных компонентов. 
В~качестве референсных значений рассматривались наборы метаболических 
и~антропологических показателей (их всего было~14), соответствующих 
определенному классу камней.

\begin{table*}\small
\begin{center}


\begin{tabular}{|c|c|c|c|c|c|c|}
\multicolumn{7}{c}{Качество классификации с~помощью MRR}\\
\multicolumn{7}{c}{\ }\\[-6pt]
\hline
\multicolumn{1}{|c|}{\raisebox{-6pt}[0pt][0pt]{\tabcolsep=0pt\begin{tabular}{c}Тип\\ камня\end{tabular}}}&
\multicolumn{1}{c|}{\raisebox{-6pt}[0pt][0pt]{$N$}}&$(1-\alpha)$, 
&\multicolumn{2}{c|}{MRR(5)}&\multicolumn{2}{c|}{MRR(1)}\\
\cline{4-7}
&&&&&&\\[-9pt]
&&\%&$(1-\hat{\alpha})$, \%&$\hat{\beta}$, \%&$(1-\hat{\alpha})$, \%&$\hat{\beta}$, \%\\
\hline
\multicolumn{1}{|c|}{\raisebox{-18pt}[0pt][0pt]{O}}&
\multicolumn{1}{c|}{\raisebox{-18pt}[0pt][0pt]{82}}
&95&100\hphantom{9}&71&90&24\\
&&85&96&78&89&36\\
&&75&91&85&77&44\\
&&65&76&88&74&50\\
\hline
\multicolumn{1}{|c|}{\raisebox{-18pt}[0pt][0pt]{U}}&
\multicolumn{1}{c|}{\raisebox{-18pt}[0pt][0pt]{76}}&95&100\hphantom{9}&75&91&24\\
&&85&99&85&80&35\\
&&75&82&89&74&48\\
&&65&71&91&68&56\\
\hline
\multicolumn{1}{|c|}{\raisebox{-18pt}[0pt][0pt]{P}}&
\multicolumn{1}{c|}{\raisebox{-18pt}[0pt][0pt]{83}}&95&100\hphantom{9}&66&87&25\\
&&85&94&78&86&33\\
&&75&86&82&82&41\\
&&65&77&87&75&47\\
\hline
\end{tabular}
\end{center}
\end{table*}
     
     
     Для каждого из основных классов O, U, P, OU, OP и~UP перед построением 
MRR проводилась селекция признаков и~принималось то значение размерности 
признакового пространства~$d$ и~соответствующий набор показателей, 
которые позволяли прогнозировать состав камней без потери качества 
(методика описана в~\cite{9-kri} и~привела к~значению $d\hm=9$). В~качестве 
модели данных в~первую очередь рассматривалась смесь многомерных 
нормальных распределений из пяти элементов (подбор числа элементов смеси 
проводился с~по\-мощью AIC~--- Akaike information criterion), для соответствующего региона было принято 
обозначение MRR(5). Для сравнения также использовалась модель 
нормального распределения, которой соответствовал MRR(1). Полученные 
результаты приводятся час\-тич\-но в~таблице, где $N$~--- объем 
классифицируемых данных; $\hat{\alpha}$~--- оценка для~$\alpha$; 
$\hat{\beta}$~--- оценка мощности критерия при определении типа камня на 
основании MRR.


     Одной из базовых характеристик является вероятность попадания в~MRR 
$(1\hm-\alpha)$ и~ее оценка $(1\hm-\hat{\alpha})$. Сравнение соответствующих 
столбцов с~учетом значений~$N$ и~ориентировочных значений разброса 
(стандартные отклонения на основе биномиального распределения) не 
позволило выявить явных отклонений. Необходимо, правда, отметить, что во 
всех проанализированных случаях для MRR(5) оказалось, что $1\hm-
\hat{\alpha}\hm\geq 1\hm-\alpha$.
     
     Назначение MRR, заключающееся в~сжатом представлении референсных 
значений, в~многомерном случае практически не проявляется. Для задания 
MRR(5) необходимо указать следующие величины: $1\hm-\alpha$, $t$, 
$p_1,\ldots, p_{k-1}$, $\mu_1, \Sigma_1,\ldots , \mu_k,\Sigma_k$, общее 
количество которых равно  $[2\hm+ (k\hm-1)\hm+ k(d\hm+ d(d\hm+1)/2)]$ 
и,~в~частности, в~рассматриваемых экспериментах~--- 276. Для MRR(1) это 
значение меньше и~равно~56. При этом для обрабатываемой обучающей 
выборки в~зависимости от класса камней речь идет о~порядка~10$^2$ векторах 
данных (см.\ столбец со значениями~$N$), что приблизительно 
дает~10$^3$~скалярных величин.
     
     Другое назначение MRR состоит в~его использовании для 
диагностирования (классификации). В~этой связи в~первую очередь 
проводился сравнительный анализ MRR(1) (фактически это означает, что 
построение региона осуществляется на основе расстояния Махаланобиса) 
и~MRR(5) (модель смеси нормальных распределений и~предложенный 
в~данной работе метод оценивания па\-ра\-мет\-ров региона). Показателем 
информативности метода построения многомерного региона выступала 
мощность соответствующего слабого критерия значимости, а~именно: 
вероятность не попасть в~MRR при условии, что данные берутся из дополнения 
к~классу, для которого построена MRR. Сравнение соответствующих столбцов 
говорит о~явном преимуществе двух предложенных моментов: усложнение 
модели данных путем перехода от нормального распределения к~смеси 
нормальных распределений и~построение региона высокой плотности.
     
     Использование критического уровня можно продемонстрировать  
с~по\-мощью зависимости результатов сравнения двух классов от того, какой 
класс взять за основу. Введем для возможных значений $p$-ве\-ли\-чи\-ны три 
интервала: $(-\infty, 1\%)$, $[1\%, 5\%)$, $[5\%, 100\%)$ с~соответствующей 
интерпретацией положения наблюденного набора показателей для пациента 
относительно построенного MRR: уверенное непопадание, неуверенное 
попадание, уверенное попадание. Если MRR построить для оксалатных камней, 
то результаты для анализа пациентов с~фосфатными камнями дадут следующий 
вектор относительных частот попадания $p$-ве\-ли\-чин в~указанные 
интервалы: $(60\%, 18\%, 22\%)$. Если же MRR строить для фосфатных 
камней, то получим $(71\%, 5\%, 24\%)$. Таким образом, для классификации 
указанных камней при приблизительно одинаковых частотах попадания в~MRR 
(22\% или~24\%) уверенный отказ от референсного региона происходит чаще, 
если принять за базовый MRR регион для фосфатных камней. Построение 
шкалы, подобной рассмотренной, является прерогативой специалистов 
в~предметной области, в~данной работе она использовалась только для 
иллюстрации. 

\vspace*{-6pt}

\section{Заключение}

\vspace*{-2pt}

     На настоящий момент имеется относительно мало примеров применения 
MRR в~клинической практике. Тому есть несколько причин. Математическое 
обеспечение, необходимое для получения и~применения MRR, не отвечает 
возможностям большинства клинических лабораторий. Лаборатории слабо 
оснащены программными средствами\linebreak для реализации достаточно сложного 
математического аппарата многомерного анализа, а~еще важнее, что 
отсутствуют методики, инструкции по\linebreak использованию соответствующих 
средств. Лишь немногие клинические применения демонстрируют 
преимущества MRR, хотя свидетельств неудачных попыток больше.
     
     Несмотря на сложности внедрения мно\-го\-мерно\-го анализа референсных 
значений, можно сформулировать некоторые рекомендации по иссле\-до\-ва\-нию 
и~разработке MRR. Во-пер\-вых, эффективная размерность в~MRR должна 
быть как можно меньше, чтобы избежать затенения диагностически полезной 
информации тестами, со\-зда\-ющи\-ми шум. Низкая размерность также должна 
уменьшить неблагоприятные последствия увеличения неточности результатов 
в~связи с~ростом числа анализируемых показателей. Во-вто\-рых, показатели 
(тес\-ты), включенные в~MRR, должны быть физиологически релевантными 
исследуемому кругу расстройств, чтобы максимизировать информацию, 
полученную от MRR. В-треть\-их, чтобы учесть эффекты долгосрочной 
лабораторной из\-мен\-чи\-вости, данные, используемые для получения MRR, 
долж\-ны быть собраны и~проанализированы в~течение достаточно большого 
периода времени (от нескольких недель до нескольких месяцев).  
В-чет\-вер\-тых, представление результатов лабораторных исследований 
следует осуществлять в~графическом виде, чтобы помочь врачам лучше понять 
MRR. Различные подходы к~уменьшению размерности помогут выполнить это 
требование.
     
     Необходима дальнейшая разработка пояснительных инструментов, 
способных воспринять результаты анализа MRR. При этом дополнительно 
необходима информация о~том, какие именно тес\-ты являются важнейшими 
факторами нарушения нормы. Надо признать, что соответствующий 
математический аппарат еще предстоит разработать. Решение перечисленных 
вопросов играет важную роль для обеспечения постоянного клинического 
применения MRR. 

\vspace*{-6pt}
     
{\small\frenchspacing
 {%\baselineskip=10.8pt
 \addcontentsline{toc}{section}{References}
 \begin{thebibliography}{99}
 
 \vspace*{-2pt}
 
\bibitem{1-kri}
\Au{Boyd J.\,C.} Reference regions of two or more dimensions~// Clin. Chem. Lab. 
Med., 2004. Vol.~42. No.\,7. P.~739--746.
\bibitem{2-kri}
\Au{Winkel P.} Patterns and clusters~--- multivariate approach for interpreting 
clinical chemistry results~// Clin. Chem., 1973. Vol.~19. No.\,12. P.~1329--1333.
\bibitem{3-kri}
IFCC. Expert panel on theory of reference values. Approved recommendation on the 
theory of reference values. Part~5. Statistical treatment of collected reference values. 
Determination of reference limits~// J.~Clin. Chem. Clin. Biochem., 1987. Vol.~25. 
No.\,9. P.~645--656.
\bibitem{4-kri}
\Au{Кривенко М.\,П.} Статистические методы представления и~предварительной 
обработки референсных значений.~--- М.: ФИЦ ИУ РАН, 2016. 160~с.
\bibitem{5-kri}
\Au{Boyd J.\,C., Lacher~D.\,A.} The multivariate reference range: An alternative 
interpretation of multi-test profiles~// Clin. Chem., 1982. Vol.~28. No.\,2.  
P.~259--265.
\bibitem{6-kri}
\Au{Albert A., Harris~E.\,K.} Multivariate interpretation of clinical laboratory  
data.~--- New York, NY, USA: CRC Press, 1987. 328~p.
\bibitem{7-kri}
\Au{Linnet K.} Influence of sampling variation and analytical errors on the 
performance of the multivariate reference region~// Meth. Inf. Med., 1988. Vol.~27. 
No.\,1. P.~37--42.
\bibitem{8-kri}
\Au{Durbridge T.\,C.} Clinical acceptance of a multi-test reference region for 
biochemical-panel results~// Clin. Chem., 1983. Vol.~29. No.\,10. P.~1724--1726.
\bibitem{9-kri}
\Au{Кривенко М.\,П.} Критерии значимости отбора признаков классификации~// 
Информатика и~её применения, 2016. Т.~10. Вып.~3. С.~32--40.
\bibitem{10-kri}
\Au{Кривенко М.\,П., Голованов~С.\,А., Сивков~А.\,В.} Анализ однородности 
данных о химическом составе камней при уролитиазе~// Информатика и~её 
применения, 2013. Т.~7. Вып.~4. С.~94--104.
 \end{thebibliography}

 }
 }

\end{multicols}

\vspace*{-10pt}

\hfill{\small\textit{Поступила в~редакцию 5.12.16}}

\vspace*{4pt}

%\newpage

%\vspace*{-24pt}

\hrule

\vspace*{2pt}

\hrule

\vspace*{-3pt}


\def\tit{HIGH-DENSITY MULTIVARIATE REFERENCE REGION\\[-5pt]}

\def\titkol{High-density multivariate reference region}

\def\aut{M.\,P.~Krivenko\\[-7pt]}

\def\autkol{M.\,P.~Krivenko}

\titel{\tit}{\aut}{\autkol}{\titkol}

\vspace*{-16pt}


\noindent
Institute of Informatics Problems, Federal Research Center 
``Computer Science and Control'' of the Russian
Academy of Sciences,  44-2~Vavilov Str., Moscow 119333, Russian Federation



\def\leftfootline{\small{\textbf{\thepage}
\hfill INFORMATIKA I EE PRIMENENIYA~--- INFORMATICS AND
APPLICATIONS\ \ \ 2017\ \ \ volume~11\ \ \ issue\ 2}
}%
 \def\rightfootline{\small{INFORMATIKA I EE PRIMENENIYA~---
INFORMATICS AND APPLICATIONS\ \ \ 2017\ \ \ volume~11\ \ \ issue\ 2
\hfill \textbf{\thepage}}}

\vspace*{2pt}




\Abste{The paper considers the principles of construction of multivariate 
reference regions. An original method of construction of 
a~region on the basis of areas of high density of points and approximation 
of data distribution with a~mixture of normal distributions is suggested. 
To estimate the threshold for the probability density, the bootstrap method is used. 
As an experiment, the paper considers the problem of description and use of 
the reference region for predicting the type of urinary stones. 
Real data treatment demonstrated the benefits of the proposed solutions.}

\KWE{multivariate reference region; high-density region; bootstrap method; 
multivariate normal mixture}

\DOI{10.14357/19922264170207} 

%\vspace*{-18pt}

%\Ack
%\noindent



%\vspace*{3pt}

  \begin{multicols}{2}

\renewcommand{\bibname}{\protect\rmfamily References}
%\renewcommand{\bibname}{\large\protect\rm References}

{\small\frenchspacing
 {%\baselineskip=10.8pt
 \addcontentsline{toc}{section}{References}
 \begin{thebibliography}{99}
\bibitem{1-kri-1}
\Aue{Boyd, J.\,C.} 2004. Reference regions of two or more dimensions. \textit{Clin. 
Chem. Lab. Med.} 42(7):739--746.

\bibitem{2-kri-1}
\Aue{Winkel, P.} 1973. Patterns and clusters~--- multivariate approach for interpreting 
clinical chemistry results. \textit{Clin. Chem.} 19(12):1329--1333.
\bibitem{3-kri-1}
IFCC. 1987. Expert panel on theory of reference values. Approved recommendation on the 
theory of reference values. Part~5. Statistical treatment of collected reference values. 
Determination of reference limits. \textit{J.~Clin. Chem. Clin. Biochem.} 
25(9):645--656.
\bibitem{4-kri-1}
\Aue{Krivenko, M.\,P.} 2016. \textit{Statisticheskie metody predstavleniya 
i~predvaritel'noy obrabotki referensnykh znacheniy}
[Statistical methods for representation and preliminary processing of
reference values]. Moscow: FRC CSC RAS. 160~p.

\bibitem{5-kri-1}
\Aue{Boyd, J.\,C., and D.\,A.~Lacher.} 1982. The multivariate reference range: An 
alternative interpretation of multi-test profiles. \textit{Clin. Chem.}  
28(2):259--265.
\bibitem{6-kri-1}
\Aue{Albert, A., and E.\,K.~Harris.} 1987. \textit{Multivariate interpretation of 
clinical laboratory data}. New York, NY: CRC Press. 328~p.
\bibitem{7-kri-1}
\Aue{Linnet, K.} 1988. Influence of sampling variation and analytical errors on the 
performance of the multivariate reference region. \textit{Meth. Inf. Med.}  
27(1):37--42.
\bibitem{8-kri-1}
\Aue{Durbridge, T.\,C.} 1983. Clinical acceptance of a multi-test reference region 
for biochemical-panel results. \textit{Clin. Chem.} 29(10):1724--1726.
\bibitem{9-kri-1}
\Aue{Krivenko, M.\,P.} 2016. Kriterii znachimosti otbora priznakov klassifikatsii
[Significance tests of feature selection for~classification]. \textit{Informatika i~ee 
Primeneniya~--- Inform. Appl.} 10(3):32--40.
\bibitem{10-kri-1}
\Aue{Krivenko, M.\,P., S.\,A.~Golovanov, and A.\,V.~Sivkov}. 2013. Analiz 
odnorodnosti dannykh o~khimicheskom sostave kamney pri urolitiaze
[Analysis of data homogeneity of~the~chemical compositions 
of~stones in~case of~urolithiasis]. \textit{Informatika i~ee Primeneniya~---
Inform Appl.} 7(4):94--104.
\end{thebibliography}

 }
 }

\end{multicols}

\vspace*{-3pt}

\hfill{\small\textit{Received December 5, 2016}}


\Contrl

\noindent
\textbf{Krivenko Michail P.} (b.\ 1946)~--- Doctor of Science in technology, 
professor, leading scientist, Institute of Informatics Problems, Federal Research 
Center ``Computer Science and Control'' of the Russian Academy of Sciences, 
\mbox{44-2}~Vavilov Str., Moscow 119333, Russian Federation; \mbox{mkrivenko@ipiran.ru}

\label{end\stat}


\renewcommand{\bibname}{\protect\rm Литература}   %
\def\stat{parkhom}

\def\tit{ПРИМЕНЕНИЕ КВАЗИСЛУЧАЙНОГО ПОДХОДА И~АНСАМБЛЕВЫХ ВЫЧИСЛЕНИЙ 
ДЛЯ~ОПРЕДЕЛЕНИЯ ОПТИМАЛЬНЫХ НАБОРОВ ЗНАЧЕНИЙ ПАРАМЕТРОВ 
КЛИМАТИЧЕСКОЙ МОДЕЛИ$^*$}

\def\titkol{Применение квазислучайного подхода и~ансамблевых вычислений 
для~определения оптимальных наборов значений} % параметров  климатической модели}

\def\aut{В.\,П.~Пархоменко$^1$}

\def\autkol{В.\,П.~Пархоменко}

\titel{\tit}{\aut}{\autkol}{\titkol}

\index{Пархоменко В.\,П.}
\index{Parkhomenko V.\,P.}


{\renewcommand{\thefootnote}{\fnsymbol{footnote}} \footnotetext[1]
{Работа выполнена при поддержке РФФИ (проекты 16-01-0466, 17-01-00693, 17-07-00035).}}


\renewcommand{\thefootnote}{\arabic{footnote}}
\footnotetext[1]{Вычислительный центр им.\ А.\,А.~Дородницына Федерального исследовательского центра <<Информатика 
и~управ\-ле\-ние>> Российской академии наук; Московский государственный технический университет им.\ 
Н.\,Э.~Баумана, \mbox{parhom@ccas.ru}}

\vspace*{-6pt}


  \Abst{В условиях неопределенности значений большого числа параметров 
гидродинамической трехмерной глобальной климатической модели реализована процедура 
их одновременной оценки для близости результатов моделирования к~данным наблюдений. 
Модель включает блоки атмосферы, термохалинной крупномасштабной циркуляции океана и~морского льда. В квазислучайном подходе по методу латинского гиперкуба генерируется 
ансамбль из~200~расчетов путем равномерного полного покрытия диапазона изменения 
каждого из~12~параметров модели. Параметры определяют перемешивание и~перенос 
в~атмосфере, океане и~морском льду, но их комбинации выбираются случайным образом. 
Исследование количественной меры ошибки модели позволило решить обратную задачу 
оценки параметров модели и~прямую задачу прогнозных расчетов по модели.}
  
  \KW{глобальная климатическая модель; оценка параметров; метод латинского гиперкуба}
  
  \DOI{10.14357/19922264170208}
  
%  \vspace*{-6pt} 


\vskip 10pt plus 9pt minus 6pt

\thispagestyle{headings}

\begin{multicols}{2}

\label{st\stat}
  
\section{Введение}

  Климатические модели имеют ряд на\-стра\-и\-ва\-емых параметров, значения 
которых не всегда определяются из теории или данных наблюдений при 
исследовании соответствующих процессов~[1]. Даже характер физических 
процессов может быть неясен и~зависеть от пространственного разрешения 
модели, а параметризации подсеточных процессов\linebreak представляют собой самые 
различные физические явления (вихри и~мелкомасштабные движения, 
инерционные гравитационные волны, приливы и~т.\,п.). В~таких случаях 
значения параметров могут быть определены путем выбора оптимального 
ансамбля модельных результатов для соответствия данным наблюдений. Это, 
естественно, влечет за собой поиск оптимальных квазистационарных решений 
в~многомерном пространстве всех па\-ра\-мет\-ров модели. Использование 
стандартного метода Мон\-те Кар\-ло потребует десятков или сотен тысяч 
интегрирований модели до достижения квазистационарных состояний. 

Для 
моделей с~высоким или умеренным разрешением вычислительные затраты 
даже одного такого расчета могут оказаться непомерно высокими~[2, 3]. 
Вместо этого большие модели, как правило, настроены на последовательность 
расчетов для подробного исследования влияния одного параметра. Однако 
взаимозависимость параметров почти наверняка означает, что даже порядок, 
в~котором такие исследования проводятся, повлияет на конечный результат 
и,~следовательно, на модельные прогнозы. 

Вычислительно эффективные 
модели имеют значительный потенциал для выполнения большого числа 
расчетов за разумное время и~позволяют исследовать большие диапазоны 
в~пространстве их параметров. Если параметры имеют явную физическую 
интерпретацию или близкие аналоги в~модели с~более высоким 
пространственным разрешением, то результаты могут иметь и~более общее 
значение. Вычислительно эффективные модели также полезны для понимания 
долгосрочной естественной изменчивости климата, в~этом случае оптимальный 
баланс сложности блоков модели может зависеть от временн$\acute{\mbox{ы}}$х масштабов, 
интересующих исследователя. 
  
  В статье рассматривается модель океана с~произвольным рельефом дна 
в~глобальной постановке в~геострофическом приближении с~фрикционным 
членом и~с~расширением за счет добавления энерго- и~влагобалансовой 
модели атмосферы и~динамической и~термодинамической модели морского 
льда~[4]. В~данной реализации увеличено горизонтальное разрешение модели 
до~$72 \times 72$~расчетных ячеек~[5]; тем не менее, учитывая достаточно простое 
представление процессов в~атмосфере, в~результате совместная модель имеет 
высокую вычислительную эффективность.

\vspace*{-3pt}
  
\section{Описание модели}

\vspace*{-2pt}

  Представлена глобальная модель климата, которая включает полностью 
трехмерную, с~трением геострофическую модель океана, обладающая высокой 
эффективностью интегрирования по сравнению со значительно более 
ресурсоемкими климатическими моделями с~трехмерными примитивными 
уравнениями океана. Модель включает также динамическую 
и~термодинамическую модель морского льда и~энерго- и~влагобалансовую 
модель ат\-мо\-сферы.
  
  Система уравнений модели океана рассматривается в~геострофическом 
приближении с~фрикционным членом в~уравнениях импульса по 
горизон\-тали~[4, 5]. Значения температуры и~солености удовле\-творяют  
ад\-век\-ци\-он\-но-диф\-фу\-зи\-он\-ным уравнениям, что позволяет описать 
термохалинную циркуляцию океана. Приближенным образом учитываются 
также конвективные процессы. Таким образом, система основных уравнений, 
записанных для наглядности в~локальных декартовых координатах $(x, y, z)$, 
где $x, y$~--- горизонтальные координаты и~$z$~--- высота, направленная 
вверх, имеет следующий вид:
  \begin{itemize}
  \item уравнения импульса по горизонтали
  \begin{align*}
  -lv +\lambda u &=-\fr{1}{\rho}\,\fr{\partial p}{\partial x} 
+\fr{1}{\rho}\,\fr{\partial(k_w\tau_x)}{\partial z}\,;\\
  lu+\lambda v &= -\fr{1}{\rho}\,\fr{\partial p}{\partial y} +\fr{1}{\rho}\, 
\fr{\partial (k_w\tau_y)}{\partial z}\,;
  \end{align*}
  \item уравнение неразрывности
  $$
  \fr{\partial u}{\partial x} +\fr{\partial v}{\partial y} +\fr{\partial w}{\partial y} 
=0\,;
  $$
  \item уравнение гидростатики
  $$
  \fr{\partial p}{\partial z} =-\rho \g\,;
  $$
  \item уравнение состояния морской воды
  $$
  \rho= \rho(S,T)\,;
  $$
  \item уравнение переноса и~диффузии трассеров~$X$ (температуры 
и~солености)
  $$
  \fr{d}{dt}\,X= k_h\nabla^2 X+\fr{\partial}{\partial z}\left( k_v\fr{\partial 
X}{\partial z} \right) +C\,,
  $$
  \end{itemize}
в которых $u$, $v$ и~$w$~--- горизонтальные и~вертикальная компоненты вектора 
скорости соответственно; $\lambda$~--- переменный в~пространстве параметр, 
увеличивающийся к~береговым границам и~экватору и~определяющий влияние 
фрикционного члена; $T$, $S$ и~$p$~--- температура, соленость и~дав\-ле\-ние 
соответственно; $\tau_x$ и~$\tau_y$~--- компоненты напряжения трения ветра; 
$\rho$~--- плотность воды; $l$~--- параметр Кориолиса; $\g$~--- ускорение 
свободного падения; $k_h$ и~$k_v$~--- коэффициенты турбулентной диффузии 
трассеров по горизонтали и~вертикали соответственно; $C$~--- источники.

  Указанная система уравнений решается в~сферической системе координат 
для всего Мирового океана с~реальной аппроксимированной глубиной. На 
границах материков принимаются равными нулю нормальные составляющие 
потоков тепла и~солей. Океан подвергается воздействию напряжения трения 
ветра на поверхности. Потоки~$T$ и~$S$ у~дна полагаются равными нулю, 
а~на поверхности определяются взаимодействием с~атмосферой. 
  
  В термодинамической модели морского льда динамические уравнения 
решаются для сплоченности льда и~для средней толщины льда. Рост и~таяние 
льда в~модели зависят только от разности между потоком тепла из атмосферы 
в~морской лед и~потока тепла изо льда в~океан. Для температуры по\-верх\-ности 
льда решается диагностическое урав\-нение.
{\looseness=1

} 
  
  Для описания процессов, протекающих в~атмосфере, используется энерго- 
и~влагобалансовая модель. В~модели решается вертикально 
проинтегрированное уравнение для температуры, определяющее баланс 
приходящего и~уходящего радиационных потоков, явных (турбулентных) 
обменов потоками тепла с~подстилающей по\-верх\-ностью, высвобождения 
скрытого тепла из-за осадков и~прос\-той однослойной параметризации 
горизонтальных процессов переноса. Источники в~уравнении переноса для 
удельной влажности определяются осадками, испарением и~сублимацией 
с~подстилающей поверхности.

\begin{table*}[b]\small
%\vspace*{3pt}
\begin{center}

\begin{tabular}{|c|l|c|c|c|}
\multicolumn{5}{c}{Диапазон изменения параметров модели климата для ансамблевых 
экспериментов}\\
\multicolumn{5}{c}{\ }\\[-6pt]
\hline
&\multicolumn{1}{c|}{Параметр модели}&Минимум&Максимум&Приемлемый 
диапазон\\
\hline
\multicolumn{5}{|c|}{Океан}\\
\hline
&&&&\\[-9pt]
1.& Горизонтальная диффузия, м$^2$/с & 300 & 10$^4$ & 4200--8500\\
2. & Вертикальная диффузия, м$^2$/с & $2\cdot 10^{-6}$ & $2\cdot 10^{-4}$ &  
$3\cdot 10^{-5}$--$1{,}9\cdot 10^{-4}$\\
3. & Коэффициент трения, сут$^{-1}$ & 1/5 & 2 & 0,6--1,90\\
4. & Ветровое воздействие & 1& 3& 1,14--2,58\\
\hline
\multicolumn{5}{|c|}{Атмосфера}\\
\hline
&&&&\\[-9pt]
5. & Диффузия тепла, м$^2$/с & 10$^6$ & 10$^7$ & $4{,}35\cdot 10^6$--$9\cdot 10^6$\\
6. & Угловой коэффициент, рад & 0,5& 2& 0,7--1,45\\
7. & Коэффициент наклона & 0& 0,25& 0,023--0,230\\
8. & Диффузия влажности, м$^2$/с & $5\cdot 10^6$ & $5\cdot 10^6$ & $1\cdot 10^5$--$3\cdot 10^5$\\
9. & Коэффициент адвекции тепла & 0 & 1& 0,050--0,815\\
10.\hphantom{9} & Коэффициент адвекции влажности & 0 &1& 0,255--0,850\\
11.\hphantom{9} & Поток между океанами, $S_v$ & 0 & 0,64& 0--0,75\\
\hline
\multicolumn{5}{|c|}{Морской лед}\\
\hline
&&&&\\[-9pt]
12.\hphantom{9} &Диффузия морского льда, м$^2$/с & 300& 10$^4$ & 300--9320\\
\hline
\end{tabular}
\end{center}
\end{table*}
  
  Все блоки модели связаны между собой обменом импульсом, теплом 
и~влагой. Используется реальная конфигурация материков и~распределение 
глубин Мирового океана~[5]. Уравнения в~сферической системе координат 
решаются численным ко\-неч\-но-раз\-ност\-ным методом. По горизонтали 
применяется равномерная по долготе и~синусу широты расчетная сетка 
размерностью~$72\times72$. Глубина океана представляется в~виде 
восьмиуровневой логарифмической шкалы до максимального 
значения~5000~м.\linebreak Начальное состояние системы характеризуется постоянными 
температурами океана, атмосферы и~нулевыми скоростями течений океана. 
Численные эксперименты показывают, что модель выходит на равновесие за 
период около~2000~лет~[5].

  %\vspace*{-6pt}
  
\section{Постановка задачи оценки параметров и~результаты}

  %\vspace*{-2pt}
  
  В предлагаемом квазислучайном подходе генерируется ансамбль расчетов 
путем равномерного полного покрытия диапазона изменения каждого 
индивидуального параметра модели, которые перечислены далее, но 
комбинации параметров выбираются случайным образом. Это соответствует 
равномерному разбиению вероятностного пространства значений параметров 
при равномерном распределении плотности вероятности. Таким образом,\linebreak 
при~$M$~расчетах и~$N$~параметрах каждый параметр 
принимает~$M$~значений, равномерно (или по логарифмическому закону) 
покрывающих весь диа\-пазон его изменения, но порядок, в~котором выбираются 
эти значения, определяется случайным\linebreak образом. Это соответствует понятию так 
называ\-емого <<латинского гиперкуба>> в~статистике и~планировании 
эксперимента~[6]. Выборки из латинских гиперкубов начали активно 
применяться\linebreak после удачных решений в~области планирования эксперимента, 
где их использование позволяет уменьшить взаимную зависимость факторов 
без увеличения числа экспериментов~[6]. Каждый расчет представляет собой 
отдельное интегрирование модели на~2000~лет от однородного состояния 
климатической системы с~нулевыми скоростями течений до установившегося 
состояния при стандартных условиях, соответствующих современному 
климату~[5]. 

Как показывают расчеты, окончательное квазистационарное 
состояние может быть не единственным для данного набора параметров. 
Другие квазистационарные состояния могут быть получены с~использованием 
различных начальных условий, в~частности различных начальных температур 
океана. Однако в~настоящей работе прежде всего изучается влияние изменения 
параметров модели и~поэтому фиксируется начальная температура океана 
на~20~$^\circ$C. Такая постановка приводит к~быстрому конвективному 
механизму начала процессов установления в~океане. 

Для обработки результатов 
такого большого количества численных экспериментов необходимо определить 
объективную меру ошибки модели.\linebreak Для этого используется взвешенная 
сред\-не\-квад\-ратическая ошибка на множестве всех динамических переменных 
в~океане и~атмосфере по\linebreak сравнению с~интерполированными данными 
наблюдений, а~именно: температуры и~влажности воздуха на поверхности 
(1000~мб), в~среднем за период с~1948 до~2002~гг., и~температуры и~солености 
океана~[7].
  
  В таблице перечислены~12~основных параметров модели (первый столбец) 
и~принимаемые диапазоны их возможного изменения (второй и~третий 
столбцы)~[8]. Если изменять каждый из этих параметров в~отдельности, то 
будет изучена только очень ограниченная область пространства па\-ра\-мет\-ров. 
Поэтому допускаем, чтобы все~12~па\-ра\-мет\-ров изменялись сразу в~указанных 
диапазонах, которые приведены в~таб\-ли\-це. Предельные значения выбираются 
таким образом, чтобы покрывать или превышать диапазон разумного выбора 
со\-от\-вет\-ст\-ву\-ющих значений для такой модели.

\end{multicols}

 \begin{figure*}[b] %fig1
   \vspace*{-7pt}
\begin{center}
\mbox{%
\epsfxsize=157.963mm
\epsfbox{par-1.eps}
}
\end{center}
\vspace*{-11pt}
\Caption{Среднеквадратичные ошибки в~зависимости от величины исследуемых параметров 
под номерами~1--4 из таблицы}
\end{figure*}

\begin{multicols}{2}
  
  Всего по модели было проведено $M\hm= 200$~расче\-тов. Для определения 
ошибки модельных результа\-тов используется взвешенная среднеквадратичная 
ошибка, вычисляемая по набору всех динамических переменных для 
атмосферы и~океана при сравнении с~данными наблюдений:
  $$
  \varepsilon^2 =\sum\limits_{i=1}^n w_i \left( X_i-D_i\right)^2\,,
  $$
где $X_i$ и~$D_i$~--- соответственно модельные результаты и~данные 
наблюдений для этих переменных (температура и~влажность атмосферы, 
температура и~соленость океана). Суммирование ведется по всем точкам 
трехмерной сетки и~по всем указанным переменным ($n\hm=30\,008$). 
Величины  $w_i\hm= 1/(n\sigma^2_X)$~--- весовые множители, зависящие от 
со\-от\-вет\-ст\-ву\-ющей переменной~$X_i$, но не зависящие от точки сетки; 
 $\sigma_X$~--- среднеквадратичная ошибка данных наблюдений. Вычисляется 
также альтернативная ошибка~$\varepsilon_A$~--- по той же формуле, но только 
для расчетных точек и~переменных атмосферы.

 \begin{figure*} %fig2
  \vspace*{1pt}
\begin{center}
\mbox{%
\epsfxsize=159.825mm
\epsfbox{par-2.eps}
}
\end{center}
\vspace*{-9pt}
\Caption{Среднеквадратичные ошибки в~зависимости от величины исследуемых параметров 
под номерами~5--12 из таблицы}
\end{figure*}
   
  
  На рис.~1 и~2 приведены~12~графиков со значениями вычисленных ошибок 
в~зависимости от исследуемых параметров. На этих рисунках символами~\textit{1} 
отмечены значения параметров с~ошибками $\varepsilon\hm>0{,}6$; 
\textit{2}--\textit{4} соответствуют меньшим значениям ошибки. Среди 
последних символами~\textit{2} отмечены значения параметров с~ошибкой 
$\varepsilon_A\hm> 0{,}1$; \textit{3} (всего~4~штуки) отмечены 
значения $\varepsilon\hm< 0{,}6$ и~$\varepsilon_A\hm< 0{,}1$ одновременно, 
при этом исследование климатических распределений показывает, что 
достигнуто состо\-яние климатической системы, не соответствующее 
современному. Эти результаты исключаются из рассмотрения. Наконец, 
символами~\textit{4} (всего~7~штук) отмечены приемлемые 
результаты расчетов ($\varepsilon\hm< 0{,}6$ и~$\varepsilon_A\hm= 0{,}1)$ 
с~минимальными ошибками, описывающие современный климат. 

\begin{figure*}[b] %fig3
\vspace*{3pt}
\begin{center}
\mbox{%
\epsfxsize=145.144mm
\epsfbox{par-4.eps}
}
\end{center}
\vspace*{-12pt}
\Caption{Температура поверхности океана, осредненная по результатам~7~расчетов 
с~минимальной ошибкой}
%\end{figure*}
% \begin{figure*} %fig4
  \vspace*{9pt}
\begin{center}
\mbox{%
\epsfxsize=145.144mm
\epsfbox{par-5.eps}
}
\end{center}
\vspace*{-12pt}
\Caption{Среднеквадратичное отклонение температуры поверхности океана в~январе, вычисленное по 
набору~7~рас\-че\-тов с~минимальной ошибкой}
\end{figure*}

Таким 
образом, результаты показывают, что сформулированным критериям 
удовлетворяют~7~наборов значений~12~па\-ра\-мет\-ров. Граничные значения 
ошибок $\varepsilon\hm=0{,}6$ и~$\varepsilon_A\hm= 0{,}1$ соответствуют 
ошибкам данных наблюдений. По этой причине нет оснований в~расчетах 
предпочесть только один набор значений параметров. Предлагается вести 
ансамблевые расчеты по модели сразу с~7~оптимальными наборами 
параметров и~в качестве результатов
 рассматривать средние по ансамблю 
и~отклонения от них. В~соответствии с~таблицей и~рис.~1 и~2 в~наборы 
параметров входят значения па\-ра\-мет\-ров, меняющиеся в~широком диапазоне 
(последний стол\-бец в~таб\-ли\-це). Это может означать, что предположение 
о~постоянных значениях параметров достаточно
 грубое. В~зависимости от 
расчетных характеристик климата, градиентов, географических координат 
и~некоторых других причин значения параметров могут меняться во времени 
и~пространстве. 

В~силу заложенных в~постановку ограничений модели 
и~сложности описываемых процессов эти зависимости неизвестны. Однако 
пред\-ла\-га\-емая процедура проведения ансамблевых расчетов в~некоторой 
степени учитывает эти зависимости и~позволяет уточнить результаты, 
поскольку дает диапазон изменения климатических характеристик в~рамках 
ансамбля. 

\begin{figure*} %fig5
\vspace*{1pt}
\begin{center}
\mbox{%
\epsfxsize=164.593mm
\epsfbox{par-6.eps}
}
\end{center}
\vspace*{-9pt}
\Caption{Распределение зонально осредненной температуры атмосферы для 
января~(\textit{а}) и~июля~(\textit{б}): \textit{1}~--- данные наблюдений; \textit{2}~--- максимум  
в~ансамбле расчетов; \textit{3}~--- минимум в~ансамбле расчетов}
\end{figure*}

\vspace*{-7pt}
  
\section{Ансамблевые расчеты с~оптимальными наборами 
параметров модели}

\vspace*{-3pt}
 
  Далее приведены результаты расчетов по модели 
с~использованием~7~приемлемых наборов па\-ра\-мет\-ров, обеспечивающих 
минимальную ошибку по сравнению с~данными наблюдений. Расчеты ведутся 
в~постановке, описанной выше, до установившегося состояния, 
соответствующего современному климату (рис.~3). Сравнение с~данными 
наблюдений показывает хорошее совпадение (рис.~4 и~5). Среднеквадратичное 
отклонение температуры поверхности океана, вычисленное по 
набору~7~расчетов с~минимальной ошибкой, практически во всей области не 
превышает 0,5--1,0~$^\circ$C (см.\ рис.~4).


  
  Расчетный разброс климатических откликов на глобальное потепление при 
100-лет\-нем прогнозе (для приземной температуры воздуха разброс 
около~0,3~$^\circ$C, см.\ рис.~6) является существенным, учитывая, что он 
представляет собой диапазон предсказаний, возникающий только с~изменением 
парамет\-ров перемешивания и~транспорта в~модели.
{ %\looseness=1

}

 { \begin{center}  %fig6
 \vspace*{18pt}
 \mbox{%
\epsfxsize=78.036mm
\epsfbox{par-7.eps}
}
\end{center}

%\vspace*{-3pt}


\noindent
{{\figurename~6}\ \ \small{Изменение средней глобальной температуры атмосферы для~7~расчетов 
с~минимальной ошибкой при прогнозируемом увеличении концентрации СО$_2$ с~2010 
до~2100~г.}}
}

%\vspace*{12pt}

%\addtocounter{figure}{1}

  
\section{Заключение}

\vspace*{-2pt}

  Посредством анализа случайным образом сгенерированных расчетов 
на~2000~лет рассмотрены неопределенности, связанные с~12~параметрами 
модели, определяющими перемешивание и~перенос в~атмосфере, океане 
и~морском льду. Исследование количественной меры ошибки модели 
позволило\linebreak
 решить обратную задачу оценки параметров модели\linebreak и~прямую 
задачу прогнозных расчетов по модели. Результаты представляют собой 
попытку настройки трехмерной климатической модели жестко определенной 
процедурой, но в~которой, тем не менее, рассматривается соответствующее 
пространство квазислучайного изменения параметров модели. Этот подход 
обеспечивает соответствие результатов моделирования данным наблюдений, 
хотя модельные входные параметры исходно точно не известны и~могут 
меняться в~широких пределах. Неопределенность предсказаний модели 
преодолевается двумя различными способами: во-пер\-вых, рассмотрением 
множества прогнозов по подмножеству примерно одинаково правдоподобных 
моделей и,~во-вто\-рых, достаточно статистически обосно\-ван\-ной процедурой 
взвешивания всех результатов в~соответствии со средней ошибкой. Меньшее 
значение ошибки, вероятно, означает лучшее качество моделирования, 
и~поэтому, если модель в~динамике надежна, приемлемые прогнозы находятся 
в~пределах неопределенности порядка ошибки. 

\vspace*{-20pt}
  
{\small\frenchspacing
 {%\baselineskip=10.8pt
 \addcontentsline{toc}{section}{References}
 \begin{thebibliography}{9}
 
 \vspace*{-2pt}
 
  \bibitem{1-par}
  \Au{Edwards N.\,R., Marsh~R.} Uncertainties due to transport-parameter 
sensitivity in an efficient 3-D ocean-climate model~// Clim. Dynam., 2005. 
Vol.~24. No.\,4. P.~415--433.
\bibitem{2-par}
\Au{Randall D.\,A.} General circulation model development.~--- Gardners Books, 
2010. 416~p. 
  \bibitem{3-par}
  \Au{Satoh M.} Atmospheric circulation dynamics and general circulation 
models.~--- Berlin: Springer-Verlag, 2014.\linebreak 730~p.
  \bibitem{4-par}
  \Au{Marsh R., Edwards~N.\,R., Shepherd~J.\,G.} Development of a fast climate 
model (C-GOLDSTEIN) for Earth System Science~// SOC, 2002. No.\,83. 54~p.
  \bibitem{5-par}
  \Au{Пархоменко В.\,П.} Глобальная модель климата с~описанием 
термохалинной циркуляции Мирового океана~// Математическое 
моделирование и~численные методы, 2015. №\,1. С.~94--108.

%\columnbreak

  \bibitem{6-par}
  \Au{Montgomery D.\,C.} Design and analysis of experiments.~--- 5th ed.~--- New 
York, NY, USA: John Wiley \& Sons, 2001. 684~p.
  \bibitem{7-par}
  \Au{Levitus S., Boyer~T.\,P., Conkright~M.\,E., O'Brien~T., Antonov~J., 
Stephens~C., Stathoplos~L., Johnson~D., Gelfeld~R.} Noaa Atlas Nesdis~18, World 
Ocean database.~--- Washington, D.C., USA: U.S.\ Government Printing, 1998. 
 Vol.~1. 346~p.
  \bibitem{8-par}
  \Au{Parkhomenko V.} Ensemble calculations application for estimation and 
optimization of climate model parameters~//  3rd Conference (International) on 
Optimization Methods and Applications Proceedings.~--- Moscow: 
Computing Center of RAS, 2012. P.~203--207.
 \end{thebibliography}

 }
 }

\end{multicols}

\vspace*{-3pt}

\hfill{\small\textit{Поступила в~редакцию 26.01.17}}

\vspace*{10pt}

%\newpage

%\vspace*{-24pt}

\hrule

\vspace*{2pt}

\hrule

\vspace*{8pt}


\def\tit{APPLICATION OF~QUASI-RANDOM ENSEMBLE CALCULATIONS 
FOR~DETERMINATION OF~CLIMATE MODEL OPTIMAL PARAMETERS}

\def\titkol{Application of~quasi-random ensemble calculations 
for~determination of~climate model optimal parameters}

\def\aut{V.\,P.~Parkhomenko$^{1,2}$}

\def\autkol{V.\,P.~Parkhomenko}

\titel{\tit}{\aut}{\autkol}{\titkol}

\vspace*{-9pt}


\noindent
$^1$A.\,A.~Dorodnicyn Computing Center, Federal Research Center ``Computer 
Science and Control'' of the Russian\linebreak
$\hphantom{^1}$Academy of Sciences, 40~Vavilov Str., Moscow 
119333, Russian Federation

\noindent
$^2$N.\,E.~Bauman Moscow State Technical University, 5  Baumanskaya 2nd Str.,
Moscow 105005, Russian 
Federation



\def\leftfootline{\small{\textbf{\thepage}
\hfill INFORMATIKA I EE PRIMENENIYA~--- INFORMATICS AND
APPLICATIONS\ \ \ 2017\ \ \ volume~11\ \ \ issue\ 2}
}%
 \def\rightfootline{\small{INFORMATIKA I EE PRIMENENIYA~---
INFORMATICS AND APPLICATIONS\ \ \ 2017\ \ \ volume~11\ \ \ issue\ 2
\hfill \textbf{\thepage}}}

\vspace*{3pt} 



\Abste{By analyzing a randomly generated set of runs, 
each~2000~years in length, the author has considered the uncertainty in~12~mixing 
and transport parameters. Constructing a quantitative measure for the model 
error made it possible to address both the inverse problem of estimation 
of model parameters and the direct problem of model predictions. 
The results represent an attempt at tuning a~three-dimensional climate model by 
a~strictly defined procedure which, nevertheless, considers the whole of the 
appropriate parameter space. The modeling approach is thus to match 
model outputs to observations while 
model inputs (parameters) are initially only weakly constrained.}

\KWE{global climate model; model parameters estimation; latin hypercube sampling}

\DOI{10.14357/19922264170208} 

\vspace*{-3pt}

\Ack
\noindent
The work was supported by the Russian Foundation
for Basic Research (projects 16-01-0466, 17-01-00693, and 17-07-00035).



\vspace*{9pt}

  \begin{multicols}{2}

\renewcommand{\bibname}{\protect\rmfamily References}
%\renewcommand{\bibname}{\large\protect\rm References}

{\small\frenchspacing
 {%\baselineskip=10.8pt
 \addcontentsline{toc}{section}{References}
 \begin{thebibliography}{9}
  \bibitem{1-par-1}
  \Aue{Edwards, N.\,R., and R.~Marsh.} 2005. Uncertainties due to 
  transport-parameter sensitivity in an efficient 3-D ocean-climate model. 
  \textit{Clim. Dynam.} 24(4):415--433.
  \bibitem{2-par-1}
  \Aue{Randall, D.\,A.} 2010. \textit{General circulation model development}. 
Gardners Books. 416~p.
  \bibitem{3-par-1}
  \Aue{Satoh, M.} 2014. \textit{Atmospheric circulation dynamics and general 
circulation models}. Berlin: Springer-Verlag. 730~p.

\columnbreak

  \bibitem{4-par-1}
  \Aue{Marsh, R., N.\,R. Edwards, and J.\,G.~Shepherd.} 2002. Development of 
a~fast climate model (C-GOLDSTEIN) for Earth System Science. \textit{SOC}  83. 
54~p.
  \bibitem{5-par-1}
  \Aue{Parkhomenko, V.\,P.} 2015. Global'naya model' klimata s~opisaniem 
termokhalinnoy tsirkulyatsii Mirovogo okeana [Global climate model including 
description of thermohaline circulation of the World Ocean].
  \textit{Matematicheskoe modelirovanie i~chislennye metody} [Mathematical 
Modeling and Numerical Methods] 1:94--108.
  \bibitem{6-par-1}
  \Aue{Montgomery, D.\,C.} 2001. \textit{Design and analysis of experiments}. 5th 
ed. New York, NY: John Wiley \& Sons, Inc. 684~p.
{\looseness=-1

}

\pagebreak

  \bibitem{7-par-1}
  \Aue{Levitus, S., T.\,P.~Boyer, M.\,E.~Conkright, T.~O'Brien, J.~Antonov, 
C.~Stephens, L.~Stathoplos, D.~Johnson, and R.~Gelfeld}. 1998. \textit{Noaa Atlas 
Nesdis~18, World Ocean database 1998.} Washington, D.C.: U.S. Government 
Printing. Vol.~1. 346~p.
  \bibitem{8-par-1}
  \Aue{Parkhomenko, V.} 2012. Ensemble calculations application for estimation 
and optimization of climate model parameters. \textit{3rd~Conference (International) 
on Optimization Methods and Applications Proceedings}. Moscow: 
Computing Center of RAS P.~203--207. 
  \end{thebibliography}

 }
 }

\end{multicols}

\vspace*{-3pt}

\hfill{\small\textit{Received January 26, 2017}}
  
  \Contrl
  
  \noindent
\textbf{Parkhomenko Valery P.} (b.\ 1951)~--- Candidate of Science (PhD) in physics and mathematics, 
head of laboratory, A.\,A.~Dorodnicyn Computing Center, Federal Research Center ``Computer Science and 
Control'' of the Russian Academy of Sciences, 40~Vavilov Str., Moscow 119333, Russian Federation; 
associate professor, N.\,E.~Bauman Moscow State Technical University, 5~Baumanskaya 
2nd Str., Moscow 
105005, Russian Federation; \mbox{parhom@ccas.ru}
  
\label{end\stat}


\renewcommand{\bibname}{\protect\rm Литература}    %
\newcommand{\bomega}{\boldsymbol{\omega}}

\def\stat{rudoy}

\def\tit{МОДИФИКАЦИЯ ФУНКЦИОНАЛА КАЧЕСТВА\\ В~ЗАДАЧАХ НЕЛИНЕЙНОЙ РЕГРЕССИИ\\ 
ДЛЯ~УЧЕТА ГЕТЕРОСКЕДАСТИЧНЫХ ПОГРЕШНОСТЕЙ\\ ИЗМЕРЯЕМЫХ ДАННЫХ}

\def\titkol{Модификация функционала качества в~задачах нелинейной регрессии 
для~учета гетероскедастичных погрешностей} % измеряемых данных}

\def\aut{Г.\,И.~Рудой$^1$}

\def\autkol{Г.\,И.~Рудой}

\titel{\tit}{\aut}{\autkol}{\titkol}

\index{Рудой Г.\,И.}
\index{Rudoy G.\,I.}


%{\renewcommand{\thefootnote}{\fnsymbol{footnote}} \footnotetext[1]
%{Работа выполнена при финансовой поддержке РФФИ (проекты 16-07-00677 
%и~15-37-20611-мол\_а\_вед).}}


\renewcommand{\thefootnote}{\arabic{footnote}}
\footnotetext[1]{Московский физико-технический институт, 
\mbox{0xd34df00d@gmail.com}}

\vspace*{-10pt}



\Abst{Рассматривается задача восстановления нелинейной
  регрессионной зависимости по данным, имеющим погрешности определения как 
зависимых,   так и~независимых переменных, при этом распределения погрешностей
  разных измерений могут иметь разную дисперсию.
  Предлагается модифицированный функционал качества,
  учитывающий по\-греш\-но\-сти определения независимых переменных и~разные 
рас\-пре\-де\-ле\-ния   погрешностей в~разных точ\-ках.
  Приводятся результаты численного моделирования на данных, полученных в~ходе
  эксперимента по измерению зависимости мощ\-ности лазера от прозрачности
  резонатора. Рассматривается сходимость вектора па\-ра\-мет\-ров, минимизирующего
  предлагаемый функционал качества, к~оптимальному для классического
  функционала сред\-не\-квад\-ра\-тич\-ной ошиб\-ки.
  Сравнивается сходимость па\-ра\-мет\-ров, оптимальных для предлагаемого
  и~классического функционалов, к~некоторым <<истинным>> параметрам
  модели на данных, сгенерированных со\-глас\-но этим <<истинным>> параметрам 
  и~зашумленным   со\-глас\-но предположениям о~погрешностях измерений, 
  в~зависимости от па\-ра\-мет\-ров  этих   погрешностей.}

\KW{гетероскедастичные ошибки;
  ошибки измерения независимых переменных; символьная регрессия; нелинейная 
регрессия}

%\vspace*{-6pt}

\DOI{10.14357/19922264170209} 


\vskip 10pt plus 9pt minus 6pt

\thispagestyle{headings}

\begin{multicols}{2}

\label{st\stat}


\section{Введение}

\vspace*{-2pt}

В ряде приложений (см., например,~\cite{Gladun2004Labs,Rudoy15MonteCarlo})
возникает задача нахождения оптимальных
коэффициентов~$\bomega$ некоторой регрессионной модели~$f$, заданной
в~виде аналитической формулы, по набору экспериментальных данных. Для этого
в~предположении о~нормальном распределении регрессионных остатков
строится функционал $\sum\nolimits_i (y_i \hm- f(x_i, \bomega))^2$,
представляющий сумму квад\-ра\-тов отклонений экспериментальных точек~$y_i$ от
значения регрессионной кривой $f(x, \bomega)$ в~точке~$x_i$,
и~находится вектор параметров~$\bomega$, его минимизирующий.

Однако данный функционал корректен только для точно измеренных независимых
переменных и~гомоскедастичных ошибок измерения зависимой переменной:
в~част\-ности, для линейных моделей соответствующая оценка параметров является
несмещенной, со\-сто\-ятель\-ной и~наиболее эффективной только при выполнении этих
условий. В~случае\linebreak нелинейных моделей вывод функционала среднеквадратичной
ошибки согласно методу наибольшего прав\-до\-по\-до\-бия также опирается на
эти предположения (с~обобщением в~виде взвешенного метода\linebreak наименьших
квад\-ра\-тов (МНК) в~случае разных стандартных отклонений зависимой
переменной).
Иными словами, предполагается существование лишь ошибок измерения зависимой
переменной, распределение которых принимается одинаковым.

На практике, как правило, это предположение не выполняется,
особенно при измерениях в~достаточно широких диапазонах.
Например, в~задаче на\-хож\-де\-ния за\-ви\-си\-мости коэффициента преломления~$n$ 
прозрачного
полимера от длины волны~$\lambda$ по\-греш\-ности измерения каждого физического
па\-ра\-мет\-ра в~разных точках, вообще говоря, различны~\cite{Rudoy15MonteCarlo}.
Так, если для измерения длины волны~$\lambda$ используется дифракционная
решетка, то постоянной является относительная погрешность определения длины 
волны
${\sigma_{\lambda_i}}/{\lambda_i} \hm\approx {const}$ и,~следовательно,
погрешность определения длины волны зависит от самой длины волны. Подобная 
ситуация
фиксированной относительной (а~не абсолютной) ошибки является типичной для
физических экспериментов.

Таким образом, возникает задача поиска оптимальных коэффициентов регрессионной
формулы с~учетом различающихся погрешностей измерения в~разных экспериментальных 
точках. Для некоторых частных случаев эта задача решена.

Так, детальный обзор методов решения этой задачи
в~случае линейной регрессии приведен в~\cite{gillard2006historical}.
В~частности, для линейных моделей рассматривается даже более общая задача,
когда распределение ошибок не является точно известным.
Однако, по\linebreak крайней мере для ряда методов, априорная инфор\-мация все
равно необходима, как то: значение\linebreak отношения стандартных отклонений
зависимой и~независимой переменных
в~случае регрессии Деминга~\cite{Deming1943Statistical}
либо наличие инструментальных переменных
при использовании одноименного метода~\cite{Bowden1990Instrumental}.
Отметим, что дополнительная априорная информация необходима
для обеспечения воз\-мож\-ности однозначного определения параметров
модели, иначе модель становится неидентифицируемой. При этом
условие идентифицируемости модели для случая многомерной
линейной регрессии в~общем виде до сих пор неизвестно~\cite{Bekker1986Comment}.

Обзор методов решения аналогичной задачи для случая нелинейной
регрессии приведен в~\cite{Carrol06MeasurementErrors}.
Так, например, метод инструментальных переменных
обобщается на случай нелинейных моделей, при этом
опять же требуется наличие дополнительных наблюдаемых переменных,
пропорциональных регрессору с~точ\-ностью до некоторой аддитивной ошибки.
Заметим, что условия иден\-ти\-фи\-ци\-ру\-емости модели при этом неизвестны.

В ряде работ изучаются конкретные нелинейные
регрессионные модели, и~соответствующие ошибки измерений
предполагаются экспертно заданными.
Например, в~\cite{jukic2013nonlinear} рассматривается модель Басса,
описывающая динамику процесса распространения новых потребительских продуктов,
для которой вводится предположение о~неравной точ\-ности измерений в~разных
экспериментальных точках, что описывается разными весовыми коэффициентами при
соответствующих регрессионных остат\-ках. При этом весовые коэффициенты имеют 
достаточно общий вид и~вводятся произвольно в~виде экспертно указанных значений.

Другим примером является~\cite{jukic2010nonlinear}, где рассматривается задача 
оценки коэффициентов трех\-па\-ра\-мет\-ри\-че\-ско\-го распределения Вейбулла по неточно 
измеренным данным.
Для этого используется метод латентных переменных: 
к~независимым переменным~$t_i$ добавляются <<свободные>> переменные~$\delta_i$,
предоставляющие степень свободы в~про\-стран\-ст\-ве независимых переменных, 
и~минимизируется функционал вида
\begin{multline*}
  T(\alpha, \beta, \eta, \boldsymbol{\delta}) ={}\\
  {}= \sum\limits_{i = 0}^n w_i \left[f(t_i + 
\delta_i; \alpha, \beta, \eta) - y_i\right]^2 +\sum\limits_{i = 0}^n p_i \delta_i^2,
\end{multline*}
где $\alpha$, $\beta$ и~$\eta$~--- параметры распределения, а~$w_i$ и~$p_i$ являются
некоторыми экспертно заданными весами, соответствующими относительной точ\-ности
\mbox{$i$-го} измерения аналогично~\cite{jukic2013nonlinear}.

В настоящей работе рассмотрена более общая ситуация, в~которой не только 
зависимые, но и~независимые переменные определяются неточно и~каж\-дая переменная
имеет свою погрешность измерения, заданную экспертно.
Исследуется случай нелинейной регрессионной зависимости, в~отличие, 
например,  от~\cite{kiryati2000heteroscedastic}, где изучается линейная модель.
Предлагается модифицированный функционал качества, учитывающий погрешности как
зависимых, так и~независимых переменных в~виде, достаточном для большинства
практических приложений. Весовые коэффициенты при регрессионных остатках
в~настоящей работе выводятся из базовых предположений о~рас\-пре\-де\-ле\-нии
погрешностей измерения и~о~поведении регрессионной модели в~окрест\-ности каж\-дой
экспериментальной точки. В~част\-ности, оказывается, что весовые коэффициенты
зависят не только от самой по\-греш\-ности измерений в~данной точке, но и~от
производных регрессионной модели в~окрестности этой точки.

Предложенный функционал наиболее близок к~описанному в~\cite{Boggs1987Stable}.
Однако, кроме того, в~настоящей работе предлагается вероятностная интерпретация
 функционала для случая нормально распределенных ошибок.

В разд.~2 настоящей работы формально по\-став\-ле\-на задача нахождения
оптимальных па\-ра\-мет\-ров регрессионной модели с~учетом гетероскедастичных
по\-греш\-но\-стей определения как зависимых, так и~независимых переменных.
В~разд.~3 выводится предлагаемый функционал качества.
Затем, в~разд.~4, описывается метод использования имеющихся
алгоритмов оптимизации, при\-ме\-ня\-емых в~подобных задачах (как, например, алгоритм
Ле\-вен\-бер\-га--Марк\-вард\-та~\cite{Marquardt1963Algorithm}), для минимизации
предлагаемого функционала. В~разд.~5 приводятся результаты
вычислительного эксперимента, со\-сто\-ящие из трех частей: во-пер\-вых,
приводятся результаты анализа экспериментальных данных по измерению
параметров усиливающей среды газового лазера; затем сравнивается сходимость
оптимальных параметров для предложенного функционала качества к~параметрам,
минимизирующим классический функционал среднеквадратичной ошибки,
в~зависимости от па\-ра\-мет\-ров распределения ошибок; кроме того, фиксируется 
некоторый вектор па\-ра\-мет\-ров
модели, принимаемый <<истинным>>, согласно которому генерируется набор
зашумленных обучающих выборок, для которых анализируется сходимость па\-ра\-мет\-ров,
оптимальных для классического и~для предлагаемого функционалов качества,
к~<<истинным>> в~за\-ви\-си\-мости от параметров шума. Показано, что в~подавляющем 
большинстве рассмотренных случаев предложенный функционал дает лучшие приб\-ли\-жения.

\section{Постановка задачи}

Дана обучающая выборка~$D$:
\begin{equation}
  D = \left\{ \mathbf{x}_i, y_i\right \} | i \in \{ 1, \dots, \ell \}\,,\enskip
   \mathbf{x}_i \in  \mathbb{R}^m, y_i \in \mathbb{R}.
  \label{eq:d}
\end{equation}
Для каждой зависимой переменной~$y_i$ известно
стандартное отклонение ошиб\-ки ее измерения~$\sigma_{y_i}$, а для 
соответствующего вектора независимых переменных~$\mathbf{x}_i$ 
аналогично известны стандартные
отклонения его компонент $\sigma_{x_{ij}} | j \hm\in \{ 1, \dots, m \}$.
При этом допускается, что близ\-кие точки могут иметь сколь угодно различные  ошиб\-ки.
Кроме того, различные ошиб\-ки измерения независимы.

Для удобства введем вектор ошибок измерений зависимых переменных~$\sigma_{y_i}$:
$$
  \boldsymbol{\sigma}_y =\left \{ \sigma_{y_1}, \dots, \sigma_{y_{\ell}} \right\}\,.
$$

Аналогично введем матрицу ошибок измерений независимых переменных~$\sigma_{x_{ij}}$:
$$
  \Sigma_x = \| \sigma_{x_{ij}} \|\, | i \in \{ 1, \dots, \ell \}\,,\enskip j \in \{ 1, 
\dots, m \}\,.
$$
Отметим, что эта мат\-ри\-ца не является ковариационной мат\-ри\-цей ошибок
каждого конкретного объекта из обучающей выборки,
поэтому нельзя утверждать, что она является
диагональной (и,~более того, квадратной).

Пусть выбрана некоторая регрессионная модель
$y \hm= f (\mathbf{x}, \bomega)$, параметризованная вектором~$\bomega$.
Требуется построить функционал ошибки~$\breve{S}(\bomega)$ вектора 
па\-ра\-мет\-ров~$\bomega$ 
модели~$f$, учитывающий ошибки измерений~$\boldsymbol{\sigma}_y$ и~$\Sigma_x$:
\begin{equation}
  \breve{S}(\bomega) = \breve{S}\left(\bomega, \boldsymbol{\sigma}_y, \Sigma_x, D\right)\,,
  \label{eq:s_modified}
\end{equation}
и,~кроме того, найти вектор па\-ра\-мет\-ров~$\omega$, минимизирующий 
функционал~\eqref{eq:s_modified}:
\begin{equation*}
  \hat{\bomega} = \mathop{\arg \min}\limits_{\bomega} \breve{S}(\bomega)\,.
\end{equation*}

\section{Модифицированный функционал качества}

Воспользуемся следующим качественным соображением:
чем больше погрешность определения переменных (зависимых или независимых)
для некоторой экспериментальной точки, тем в~меньшей степени соответствующий
регрессионный оста\-ток должен учитываться при оптимизации па\-ра\-мет\-ров\linebreak\vspace*{-12pt}

\columnbreak

 { \begin{center}  %fig1
 \vspace*{1pt}
 \mbox{%
\epsfxsize=77.346mm
\epsfbox{rud-1.eps}
}
\end{center}

\vspace*{-1pt}


\noindent
{{\figurename~1}\ \ \small{Различные способы определения расстояния от точки до прямой: 
$\tilde{\rho}$~---
    истинное расстояние как минимум расстояния от точки $(x_i, y_i)$ до ка\-кой-ли\-бо
    точки на прямой; $y_i \hm- f(x_i, \bomega)$~--- расстояние в~классическом
    функционале среднеквадратичной ошибки в~предположении об отсутствии ошибок 
измерения     независимой переменной~$x$; $\rho$~--- предлагаемое расстояние}}
}

\vspace*{12pt}

\addtocounter{figure}{1}



\noindent
 модели.
Кроме того, с~физической точки зрения\linebreak складывать
можно только величины, име\-ющие одинаковую раз\-мер\-ность, либо безразмерные
величины, поэтому необходима соответствующая нормировка невязок по каждой
из переменных.

Для упрощения изложения рассмотрим случай одной независимой переменной:
$x \hm\in \mathbb{R}$. С~учетом приведенных выше соображений введем
сле\-ду\-ющее определение расстояния~$\rho(x, i)$
от точки $(x_i, y_i)$ до некоторой точ\-ки
$(x, f(x, \bomega))$ на кривой, описываемой регрессионной моделью $y \hm= f(x, 
\bomega)$:
\begin{equation}
  \tilde{\rho}^2(x, i) = \fr{(x_i - x)^2}{\sigma_{x_i}^2} + \fr{(y_i - f(x, 
\bomega))^2}{\sigma_{y_i}^2}\,.
  \label{eq:dist0}
\end{equation}



Непосредственное точное определение расстояния от экспериментальной
точки до регрессионной кривой представляется отдельной
сложной вы\-чис\-ли\-тель\-ной задачей
оптимизации (решаемой, например, итерационно),
поэтому предлагается рассматривать
расстояние от точки не до самой кривой, а~до
линеаризованной кривой в~окрестности этой точки. На рис.~1
показаны различные варианты определения рас\-сто\-яния, при этом
в~иллюстративных целях раз\-мер\-ности и~по\-греш\-ности определения~$x$ и~$y$ приняты
одинаковыми.

Итак, линеаризуем~$f(x, \bomega)$ в~окрестности точки $(x_i, f(x_i, \bomega))$,
обозначив оператор линеаризации в~окрестности этой точки~$\mathbb{L}_i$:
\begin{multline}
  f(x, \bomega) \approx \mathbb{L}_{i}[f](x, \bomega) = {}\\
  {}=f(x_i, \bomega) + \left(x - 
x_i\right) \fr{\partial f}{\partial x}\left(x_i, \bomega\right)\,.
  \label{eq:f_linear}
\end{multline}
Расстояние~\eqref{eq:dist0} выражается для линеаризованной функции~\eqref{eq:f_linear} 
сле\-ду\-ющим образом:

\noindent
\begin{multline*}
  \rho^2(x, i) = \fr{(x_i - x)^2}{\sigma_{x_i}^2} +{}\\
  {}+ \fr{\left(y_i - f\left(x_i, 
\bomega\right) - ({\partial f}/{\partial x})\left(x_i, \bomega\right) \left(x - 
x_i\right)\right)^2}{\sigma_{y_i}^2}\,.
  %\label{eq:dist_linear}
\end{multline*}
Минимизируя это выражение по~$x$:

\noindent
$$
  \hat{x} = \mathop{\arg \min}\limits_x \rho^2(x, i)\,,
$$
находим расстояние от точки $(x_i, y_i)$ из обуча\-ющей выборки до
линеаризованной в~ее окрест\-ности регрессионной за\-ви\-си\-мости~$f$ при
данном векторе па\-ра\-мет\-ров~$\bomega$:

\noindent
\begin{multline}
  \rho^2(f, \bomega, i) = \rho^2\left(\hat{x}, i\right) = {}\\
  {}=\fr{(y_i - f(x_i, 
\bomega))^2}{\sigma^2_{y_i} + ({\partial f}/{\partial x})\left(x_i, \bomega\right)^2 
\sigma^2_{x_i}}\,.
  \label{eq:rho_univar}
\end{multline}

Отметим, что решение~\eqref{eq:rho_univar} корректируется при
последовательном изменении линеаризации в~связи с~изменением вектора
па\-ра\-мет\-ров~$\bomega$ согласно выбранному итерационному методу решения
этой задачи.

Аналогично можно получить выражение для расстояния в~случае, когда объекты 
в~обучающей выборке представлены~$m$ независимыми переменными ($\mathbf{x}
\hm \in \mathbb{R}^m$):

\noindent
$$
  \rho^2(f, \bomega, i) = \fr{(y_i - f(\mathbf{x}_i, 
\bomega))^2}{\sigma_{y_i}^2 + \sum\nolimits_{j = 1}^m (({\partial f}/{\partial 
x_j})(\mathbf{x}_i, \bomega))^2 \sigma^2_{x_{ij}}}\,.
$$

Таким образом, предлагаемый функционал, ми\-ни\-ми\-зи\-ру\-ющий сумму введенных
со\-глас\-но~\eqref{eq:dist0} рас\-сто\-яний с~учетом их линеаризации,
для достаточно глад\-ких функций выглядит следующим образом:
\begin{equation}
  \breve{S}(\bomega) = \sum\limits_{i = 1}^\ell\! \fr{(y_i - f(\mathbf{x}_i, 
\bomega))^2}{\sigma_{y_i}^2 + \sum\nolimits_{j = 1}^m (({\partial f}/{\partial 
x_j})(\mathbf{x}_i, \bomega))^2 \sigma^2_{x_{ij}}}.\!
  \label{eq:s}
\end{equation}

Отметим следующее:
\begin{itemize}
  \item функционал~\eqref{eq:s} соответствует классической сумме квад\-ра\-тов 
регрессионных     остат\-ков при условии нормировки 
квад\-ра\-та каждого остатка на сумму 
квад\-ра\-тов по\-греш\-ности
    определения зависимой величины $\sigma_{y_i}$ и~произведения част\-ной 
производной     регрессионной модели по \mbox{$j$-й} компоненте вектора независимых величин на 
по\-греш\-ность     определения соответствующей компоненты~$\sigma_{x_{ij}}$;

  \item при прочих равных условиях в~выражении для расстояния~\eqref{eq:rho_univar} 
  и,~соответственно, в~функционале~\eqref{eq:s} с~большим весом учитываются те 
точ\-ки, в~которых производная регрессионной модели 
${\partial f}/{\partial x_j}$ по  соответствующей
    компоненте~$x_j$ больше, что соответствует соображениям здравого смысла: 
чем меньше наклон регрессионной зависимости в~окрест\-ности данной точки, тем меньше влияние 
неточного измерения соответствующей независимой переменной на значение регрессионной 
зависимости в~этой точке;

  \item если все независимые переменные измерены точно, т.\,е.\
    $\forall i, j : \sigma_{x_{ij}} \hm= 0$, то предложенный функционал переходит 
в~рассмотренный в~\cite{jukic2013nonlinear}. Если же, кроме того, все зависимые переменные 
имеют одну и~ту  же по\-греш\-ность~$\sigma_y$,
    то предложенный функционал переходит в~известную сумму квадратов 
регрессионных остатков с~точ\-ностью до некоторого множителя 
(а~именно ${1}/{\sigma_y}$), не 
влияющего на положения минимумов функционала среднеквадратичной ошибки.
\end{itemize}

Следует отметить возможность вероятностной интерпретации предложенного
выражения для рассто\-яния~\eqref{eq:dist0}.
Для случая одной независимой переменной предположим, что
вероятность соответствия некоторой точки $(\tilde{x}_i, f(\tilde{x}_i,  \bomega))$
на регрессионной кривой $y \hm= f(x, \bomega)$
данной экспериментальной точке $(x_i, y_i)$
описывается двумерным нормальным распределением
с~центром в~этой экспериментальной точке $(x_i, y_i)$
и~диагональной ковариационной мат\-ри\-цей
$$
\Sigma_i = \begin{Vmatrix} \sigma_{x_i}^2 & 0 \\[2pt] 
0 & \sigma_{y_i}^2  \end{Vmatrix}
$$
(т.\,е.\ ошибки измерения каждой координаты независимы):
\begin{multline*}
  P\left(\tilde{x}_i, f(\tilde{x}_i, \bomega)\right) \sim
   \mathcal{N}\left(\left(x_i, y_i\right), \Sigma_i\right)
    = \fr{1}{2 \pi \sqrt{\mathrm{det}\,\Sigma_i}}\times{}\\
    {}\times
            \exp \left\{\! -\fr{1}{2}\,
                        \left\|\begin{matrix} x_i - \tilde{x}_i \\ 
y_i - f(\tilde{x}_i, \bomega) \end{matrix}\right\|^\mathrm{T}
                        \Sigma_i^{-1}
                        \left\|\begin{matrix} x_i - \tilde{x}_i \\ 
y_i - f(\tilde{x}_i, \bomega) \end{matrix}\right\|
                 \right\}.\hspace*{-1.14958pt}
\end{multline*}
Максимизация логарифма правдоподобия
с~аналогичной~\eqref{eq:f_linear} линеаризацией
позволяет получить те же выражения~\eqref{eq:dist0} и~\eqref{eq:s}.
Более подробное рас\-смот\-ре\-ние такого подхода
и~следствий из него станет предметом дальнейшей работы.

\section{Метод оптимизации предложенного функционала}

Для численной оптимизации функционала~\eqref{eq:s} представим его в~виде
суммы квадратов регрессионных остатков путем следующего переобозначе-\linebreak\vspace*{-12pt}

\pagebreak

\noindent
ния 
переменных. Вместо выборки~\eqref{eq:d}
рас\-смот\-рим выборку 
$$
  \tilde{D} = \left\{ \tilde{\mathbf{x}}_i, \tilde{y}_i \right\} | i \in 
  \{ 1, \dots, \ell 
\},\enskip
 \tilde{\mathbf{x}}_i \in \mathbb{R}^{m + 1}\,,\enskip
  \tilde{y}_i \in \mathbb{R}\,,
$$
где $\tilde{y}_i \equiv 0$, а~$\tilde{\mathbf{x}}_i \hm= \{ \mathbf{x}_i, y_i \}$~--- 
исходный вектор~$\mathbf{x}_i$
с~дополнительно приписанным к~нему значением~$y_i$. Кроме того, примем
$$
  \tilde{f}(\tilde{\mathbf{x}}_i, \bomega) = \fr{f(\mathbf{x}_i, \bomega) - 
y_i}{\sqrt{\sigma_{y_i}^2 + \sum\nolimits_{j = 1}^m (({\partial f}/{\partial 
x_j})(\mathbf{x}_i, \bomega))^2 \sigma^2_{x_{ij}}}}.
$$
Тогда минимизация функционала~\eqref{eq:s} возможна известными методами 
оптимизации, так как прямой подстановкой можно убедиться, что~\eqref{eq:s} 
в~этом случае эквивалентен
$$
  S(\bomega) = \sum\limits_{i = 1}^\ell \left(
  \tilde{y}_i - \tilde{f}\left(\tilde{\mathbf{x}}_i,  \bomega\right)\right)^2.
$$

\begin{comment}
Легко показать, что градиент~$\tilde{f}$ по па\-ра\-мет\-рам выглядит следующим 
образом:
\begin{footnotesize}
\[
  \frac{\partial\tilde{f}}{\partial \omega_k}(\mathbf{x}_i, \bomega) = \frac{
        \frac{\partial f}{\partial \omega_k}(\mathbf{x}_i, \bomega) 
\Big(\sigma_{y_i}^2 + \sum_{j = 1}^m \big(\frac{\partial f}{\partial x_j} 
(\mathbf{x}_i, \bomega)\big)^2 \sigma_{x_{ij}}^2 \Big)-
        \big(f(\mathbf{x}_i, \bomega) - y_i\big) \sum_{j = 1}^m 
\sigma_{x_{ij}}^2 \frac{\partial f}{\partial x_j}(\mathbf{x}_i, \bomega) 
\frac{\partial^2 f}{\partial x_j \partial \omega_k}(\mathbf{x}_i, \bomega)}
  {\Big( \sigma_{y_i}^2 + \sum_{j = 1}^m \big(\frac{\partial f}{\partial x_j} 
(\mathbf{x}_i, \bomega)\big)^2 \sigma_{x_{ij}}^2 \Big)^{\sfrac{3}{2}}}.
\]
\end{footnotesize}
\end{comment}

Для таким образом преобразованного функционала
в~качестве базового алгоритма оптимизации может
быть использован любой метод решения зада\-чи о~наименьших квад\-ра\-тах, как, 
например,
метод градиентного спуска или алгоритм Ле\-вен\-бер\-га--Марк\-вард\-та~\cite{dlib09}.
В~этом случае при
соответствующих условиях глад\-кости частных производных функции~$f$ (что 
практически всегда выполняется в~реальных физических приложениях) 
сохраняются все свойства
исходного алгоритма.

Отметим, что предложенная идея введения весовых коэффициентов, отвечающих разным
измерениям и~зависящих от точности этих измерений, вообще говоря, применима 
и~для прочих методов решения задачи вос\-ста\-нов\-ле\-ния регрессии, отличных от символьной 
регрессии. Подробное рас\-смот\-ре\-ние этих методов в~совокупности с~предлага-\linebreak емым подходом 
выходит за рамки статьи, однако укажем, что при невозможности выполнить аналитическое
дифференцирование функции~$f$ предлагается использовать следующий
итеративный алгоритм, предназначенный для использования с~уже имеющимися
реализациями соответствующих методов оптимизации. Предполагается, что реализация
<<принимает на вход>> массив значений~$y_i$,
функцию вычисления значения~$f$ в~точках~$\mathbf{x}_i$ с~вектором па\-ра\-мет\-ров~$\bomega$.

Алгоритм выглядит следующим образом.
\begin{enumerate}[1.]
  \item Выбирается некоторое начальное приближение вектора па\-ра\-мет\-ров~$\bomega$.
  \item Для каждой пары $(\mathbf{x}_i, y_i)$ из обуча\-ющей выборки численно или
    аналитически рассчитывается значение част\-ной производной
    ${\partial f}/{\partial x}$ в~точке~$(\mathbf{x}_i, \bomega)$.
  \item Каждое значение зависимой переменной $y_i$ и~значение функции 
$f(\mathbf{x}_i, \bomega)$
    нормируется на соответствующую величину
$$
      \sigma_{y_i}^2 + \sum\limits_{j = 1}^m \left(\fr{\partial f}{\partial 
x_j}\left(\mathbf{x}_i, \bomega\right)\right)^2 \sigma^2_{x_{ij}}\,.
$$
  \item Выполняется итерация классического алгоритма оптимизации для таким 
образом модифицированных значений функции~$f$ и~зависимых переменных~$y_i$, 
получая     новое значение вектора~$\bomega$.
  \item Если критерий останова не достигнут, алгоритм продолжает выполнение 
  с~п.~2.
\end{enumerate}

Отметим следующее:
\begin{itemize}
  \item критерием останова могут служить обычные критерии, такие как достижение 
некоторого     числа итераций, порог нормы изменения вектора~$\bomega$ и~т.\,п.;
  \item если известно, что производная ${\partial f}/{\partial x}$ является 
достаточно гладкой в~окрестности $(\mathbf{x}_i, \bomega) \mid i \hm\in 
\{ 1, \dots, \ell \}$, 
на шаге ~4 алгоритма представляется разумным выполнить сразу несколько итераций
    классического алгоритма во избежание потенциально ресурсоемкого пересчета 
производных и~перенормировки значений~$y_i$ и~$f$.
\end{itemize}

\section{Вычислительный эксперимент}

В вычислительном эксперименте рассматриваются данные, полученные в~ходе 
измерения зависимости интенсивности излучения~$I$ лазера от прозрачности его резонатора.
Изучался лазер высокого давления ($\approx 3$~атм He, $\approx 60$~Торр\ Ne, 
$\approx 20$~Торр\ Ar) на
$3p$--$3s$ переходах неона (основной переход~--- 585~нм), возбуждаемый электронным 
пучком~\cite{alexandrov1991kinetics}.

Пусть насыщающая переход интенсивность излуче\-ния~--- $I_s$, наблюдаемая 
интенсивность~--- $I_l$. В~таком случае для безразмерной величины 
$y \hm= {I_l}/{I_s}$ с~учетом  однородного
уширения линии усиле\-ния при высоком давлении газа и~хорошей однородности 
возбуждения, обеспечиваемой электронным пучком, можно получить нелинейное 
уравнение~\cite{champagne1982transient}:
\begin{multline}
  \alpha_0 L - \fr{1}{2} \ln R_0 = g_0 L \fr{1 + \sqrt{R_0}}{1 - \sqrt{R_0}} \,
\fr{1}{y} \times{}
\\
{}\times \ln \left( 1 + \fr{y ({1 - \sqrt{R_0}})/({1 + \sqrt{R_0}})}{1 + y 
({2 \sqrt{R_0}})/({1 - R_0})} \right)\,,
  \label{eq:y_exact}
\end{multline}
где $\alpha_0$~--- распределенные потери (например, на рассеяние света);
$g_0$~--- коэффициент усиления слабого сигнала; $R_0$~--- коэффициент отражения 
выходного зеркала лазера. Однородность накачки означает, что~$g_0$ 
и~$\alpha_0$ одинаковы  по всему объему с~хорошей точностью.

Значение $R_0$ является независимой переменной, изменяемой экспериментаторами, 
и~в~данном разделе также обозначается~$x$ сообразно остальной части работы.

Для достаточно больших~$R_0$, близких к~единице (фактически для 
$R_0 \hm\geq 0{,}6\ldots0{,}7$),
можно упрос\-тить~\eqref{eq:y_exact}, заменив $2 \sqrt{R_0} \hm\approx 1 \hm+ R_0$ 
и~получив хорошо известное выражение~\cite{champagne1982transient}:
\begin{equation}
  y(R_0) = \gamma \fr{1 - R_0}{1 + R_0} \left(\fr{g_0}{\alpha_0 - 
({1}/({2L})) \ln R_0} - 1\right)\,,
  \label{eq:y_approx}
\end{equation}
где $\gamma$~--- нормировочный коэффициент.

В рассматриваемом физическом эксперименте длина активной среды $L$~--- 150~см,
точность определения мощности лазера~$y$ имеет
относительную погрешность в~2\%, точ\-ность определения прозрачности~$R_0$ имеет
абсолютную погрешность и~со\-став\-ля\-ет~0,01 при $R_0 \hm\geq 0{,}6$ и~0,02 при $R_0  \hm< 0{,}6$. 

В ходе измерений получены значения~$y(R_0)$, приведенные в~табл.~1.

\vspace*{6pt}

 {\small \begin{center}  %tabl1
 \noindent
\parbox{32mm}{{{\tablename~1}\ \ \small{Экспериментальные значения $y(R_0)$}}
}

\vspace*{2ex}


    \tabcolsep=12pt
    \begin{tabular}{|c|c|}
    \hline
    $R_0$   &   $y$\\
    \hline
    0,48 & \hphantom{9,}3,25\\
    0,56  &10,2\\
    0,65  &16,5\\
    0,73  &20,5\\
    0,80  &22,5\\
    0,87 &23,2\\
    0,94 &18,2 \\ 
    \hline
  \end{tabular}
  \end{center}
  }
%\end{table*}

%\vspace*{12pt}

\vspace*{6pt}

\addtocounter{table}{1}

Таким образом, решается задача минимизации функционала~\eqref{eq:s} при
\begin{align*}
  \bomega &= \left(\omega_1, \omega_2, \omega_3\right) = \left(\gamma, \alpha_0, g_0\right)\,;
\\
  f(x, \bomega) &= y\left(R_0, \gamma, \alpha_0, g_0\right)\,;
\end{align*}
\begin{equation}
\left.
  \begin{array}{rl}
    \sigma_{y_i} &= 0{,}02 y_i;\\
    \sigma_{x_i} &= \begin{cases}
        0{,}01 & \mid x_i \geq 0{,}6\,; \\
        0{,}02 & \mid x_i < 0{,}6\,.
        \end{cases}
      \end{array}
    \right\}
  \label{eq:sigmas_definition}
\end{equation}

\subsection{Оптимальные параметры модели}

Кроме предложенного в~настоящей работе функционала~\eqref{eq:s} рассмотрен
также и~классический функционал среднеквадратичной ошибки:
\begin{equation}
  S = \sum\limits_{i = 1}^\ell \left(y_i - f\left(x_i, \bomega\right)\right)^2\,.
  \label{eq:s_classic}
\end{equation}

 {\small \begin{center}  %tabl2
 \noindent
{{\tablename~2}\ \ \small{Оптимальные значения параметров модели}}


\vspace*{2ex}


    \begin{tabular}{|c|c|c|c|}
    \hline
    &&&\\[-9pt]
              Параметры  &  $\bomega$    &$\bomega^0$   & $\fr{|\omega_i - \omega^0_i|}{\omega^0_i}$\\
              \hline
            &&&\\[-9pt]
$g_0$ &  $2{,}93\cdot 10^{-3}$ & $2{,}92\cdot 10^{-3}$ & 0,31\%\\
$\alpha_0$   &$2{,}07\cdot 10^{-4}$& $2{,}22\cdot 10^{-4}$& 6,59\%\\
 $\gamma$      & 98,6& 101,5& 2,9\%\hphantom{9}\\
        $\breve{S}$~\eqref{eq:s} &  0,542& 0,645& 16\%\hphantom{999}\\
        $S$~\eqref{eq:s_classic} & 0,328& 0,183& 80\%\hphantom{999}\\ 
        \hline
  \end{tabular}
  \end{center}
  }
%\end{table*}

%\vspace*{12pt}

\vspace*{3pt}

\addtocounter{table}{1}

{ \begin{center}  %fig2
 \vspace*{1pt}
 \mbox{%
\epsfxsize=77.694mm
\epsfbox{rud-2.eps}
}
\end{center}

\vspace*{-1pt}


\noindent
{{\figurename~2}\ \ \small{Графики~\eqref{eq:y_approx}, соответствующие параметрам,
    минимизирующим~\eqref{eq:s} и~\eqref{eq:s_classic}:
    \textit{1}~--- экпериментальные данные;
    \textit{2}~--- $\omega^0$; \textit{3}~--- $\omega$}}
}

\vspace*{12pt}

\addtocounter{figure}{1}



В табл.~2 приведены значения параметров~$\bomega$ и~$\bomega^0$
функции~\eqref{eq:y_approx},
минимизирующие~\eqref{eq:s} и~\eqref{eq:s_classic} соответственно, а также 
относительные разности их компонент. Кроме того, приведены значения функционалов~\eqref{eq:s} 
и~\eqref{eq:s_classic} для обоих векторов параметров.



Отдельно отметим, что сравнивать непосредственные значения функционалов~\eqref{eq:s} 
и~\eqref{eq:s_classic} не имеет смысла. Вместо этого необходимо сравнивать 
различные модели по каждому из этих функционалов в~отдельности. Так, результаты, 
приведенные
в~табл.~2, показывают вполне естественный результат: каждый
из двух векторов параметров ($\bomega$ и~$\bomega^0$) является оптимальным лишь
для того функционала, который он минимизирует.

Графики регрессионной модели~\eqref{eq:y_approx}, соответствующие~$\bomega$ 
и~$\bomega^0$, приведены на рис.~2.


\subsection{Сходимость оптимальных параметров к~классическим}

Численно исследована зависимость сходимости параметров~$\bomega$ к~параметрам~$\bomega^0$,
получаемым минимизацией функционалов~\eqref{eq:s} и~\eqref{eq:s_classic} 
соответственно, от
погрешности~$\mathbf{\sigma}_y$ измерения зависимой переменной~$y$.

\begin{figure*}[b] %fig3
\vspace*{1pt}
\begin{center}
\mbox{%
\epsfxsize=162.722mm
\epsfbox{rud-3.eps}
}
\end{center}
\vspace*{-9pt}
   \Caption{Зависимость оптимальных параметров от $k \in [1; 100]$:
  (\textit{а})~$\sigma_{y_i} \hm= 0{,}02ky_i$;
  (\textit{б})~$\sigma_{y_i} \hm= 0{,}02ky_{\max}$; \textit{1}~--- классическое значение;
  \textit{2}~--- $g_0$; \textit{3}~--- $\alpha_0$;
  \textit{4}~--- $\gamma$}
  \label{fig:conv_varY}
\end{figure*}

Следует ожидать, что при увеличении погрешности измерения величины~$y$ при
фиксированной погрешности измерения~$R_0$ оптимальный вектор~$\bomega$
будет приближаться к~$\bomega^0$, так как тем более незначителен
вклад ошибки измерения независимой переменной.

Рассматриваются два случая.
\begin{enumerate}
  \item Погрешность $i$-го измерения~$y_i$ задается как $\sigma_{y_i} \hm= 
0{,}02ky_i$, т.\,е.\     погрешность зависит от значения самого~$y_i$.
  \item Погрешность $i$-го измерения~$y_i$ задается как $\sigma_{y_i} \hm= 
0{,}02ky_{\max}$,
    т.\,е.\ погрешность от значения конкретного~$y_i$ не зависит. Заметим, что 
выбор\linebreak конкретного значения~$y$, определяющего %\linebreak 
погрешность, является в~данном 
случае  достаточ\-но произвольным и~соответствует умножению всех погрешностей на 
некоторую константу     (что нивелируется соответствующим изменением выбора диапазона~$k$).
\end{enumerate}

В первом случае ошибки измерения~$y$ распределены неодинаково; следовательно, 
применение стандартного МНК не обосно\-ва\-но. В~то же время во втором 
случае ошибки принадлежат одному и~тому же распределению и,~кроме того, независимы, 
поэтому в~данном случае МНК-оцен\-ка применима (с~точностью до ошибки измерения 
независимой переменной).

Для обоих случаев подробно рассматривалась область $k\hm \in [1; 100]$, значение~$k$
изменялось с~шагом~0,01. Отметим, что уже при $k \hm\approx 25$ характерная 
погрешность измерения величины~$y$ сопоставима с~самой величиной~$y$, а при $k \hm> 50$ 
превышает ее.

Результаты приведены на рис.~\ref{fig:conv_varY}.
На графиках отображены компоненты вектора~$\bomega$, нормированные на
соответствующие значения~$\bomega^0$, в~за\-ви\-си\-мости от значения~$k$.



В случае фиксированной погрешности~$\sigma_{y_i}$ значения~$\bomega$
действительно стремятся к~$\bomega^0$ для разумных значений~$k$, а~в~случае
гетероскедастичных ошибок такой зависимости не наблюдается, хотя значения~$\bomega$
и~оказываются достаточно близки к~$\bomega^0$. По мнению автора, такое поведение
вектора оптимальных параметров является вполне ожидаемым и~демонстрирует
несостоятельность классического функционала качества в~случае неодинаково
распределенных ошибок.

\subsection{Сходимость параметров к~истинным}

Численно исследована зависимость сходимости параметров $\bomega \hm= \arg \min 
\breve{S}$
и~$\bomega^0 \hm= \arg \min S$ к~некоторому <<истинному>> значению вектора 
па\-ра\-мет\-ров~$\hat{\bomega}$ от числа точек~$\ell$ в~обучающей выборке и~от погрешности 
определения независимой переменной.

Для этого вектор параметров~$\bomega$, полученный минимизацией обучающей выборки 
из табл.~1, принимается за некоторый <<истинный>> вектор 
параметров~$\hat{\bomega}$
и~на каждой $j$-й итерации генерируется обучающая выборка $D_j(\ell, k)$:
\begin{multline*}
  D_j(\ell, k) = {}\\
  {}=\left\{ (x_i + \xi^x_i, y(x_i, \hat{\bomega}) + \xi^y_i) \right\} \mid 
\xi^x_i \sim \mathcal{N}(0, k \sigma_{x_i}),\\
\xi^y_i \sim \mathcal{N}(0, 
\sigma_{y_i})\,,\enskip i \in \{ 1, \dots, \ell \},
\end{multline*}
где $y(x, \bomega)$ задано соотношением \eqref{eq:y_approx}, а $\sigma_{x_i}$ 
и~$\sigma_{y_i}$
определяются соотношениями~\eqref{eq:sigmas_definition}.

\begin{figure*}[b] %fig4
\vspace*{3pt}
\begin{center}
\mbox{%
\epsfxsize=162.35mm
\epsfbox{rud-4.eps}
}
\end{center}
\vspace*{-9pt}
  \Caption{Сходимость параметров $g_0$~(\textit{a}),
  $\alpha_0$~(\textit{б}) и~$\gamma$~(\textit{в})
  к~истинным при $k \hm= 0{,}2$ (левый столбец) и~0,65
  (правый столбец):
   \textit{1}~--- $\omega^0$;
  \textit{2}~--- $\omega$}
  \label{fig:comparison_0.2}
\end{figure*}

Иными словами, генерируется обучающая выборка согласно искомой модели 
с~известным и~фиксированным вектором параметров, которая затем зашумляется 
нормально распределенными случайными величинами. При этом стандартное отклонение
шума для зависимой величины совпадает с~экспертно предложенной погрешностью
измерений для реального эксперимента, а~стандартное отклонение независимой
величины отличается от экспертно предложенной погрешности для этой величины
в~$k$~раз.

После генерации выборки $D_j(\ell, k)$ по ней находятся значения $\bomega_j \hm= 
\arg \min \breve{S}(D_j)$
и $\bomega_j^0 \hm= \arg \min S(D_j)$, что повторяется~$N$~раз, и~$\forall\ i$ 
рассматриваются значения:

\noindent
\begin{align*}
  \overline{\delta \omega_i} &= \fr{\sum\nolimits_{j = 1}^N (\omega_{ji} - 
\hat{\omega}_i)}{N}\,;
\\
  \overline{\delta \omega^0_i} &= \fr{\sum\nolimits_{j = 1}^N (\omega^0_{ji} - 
\hat{\omega}_i)}{N}\,.
\end{align*}

\vspace*{-4pt}

Обозначим, кроме того, $\overline{\delta \bomega} \hm= \{ \overline{\delta 
\omega_1}, \overline{\delta \omega_2}, \overline{\delta \omega_3} \}$.

Таким образом, при варьировании~$k$ и~изуче\-нии поведения~$\overline{\delta 
\bomega}$ и~$\overline{\delta \bomega^0}$ исследуется влияние по\-греш\-ности определения
независимой переменной на раз\-ность между~$\hat{\bomega}$ и~оптимальными
па\-ра\-мет\-ра\-ми~$\bomega$ и~$\bomega^0$ согласно~\eqref{eq:s} и~\eqref{eq:s_classic}
соответст\-венно.
{\looseness=1

}

Так как при уменьшении~$k$ монотонно уменьшается погрешность определения 
независимой перемен\-ной, естественно ожидать, что различие между~$\bomega$ и~$\bomega^0$ 
будет уменьшаться. Однако вы\-чис\-ли\-тельный эксперимент демонстрирует, что это не так.

В проведенном эксперименте $N \hm= 1000$, $\ell \hm\in \{ 10; \dots; 5000 \}$,
$k \hm\in \{ 0{,}2; 0{,}25; \dots; 1 \}$.

Результаты представлены на рис.~4 и~5.






Результаты на рис.~4 (левый столбец) характерны для всех
значений $k \hm\in [0{,}2; 0{,}6]$. Однако при $k\hm \geq 0{,}65$ 
поведение оптимальных
параметров резко меняется. Так, на рис.~4 (правый столбец) приведены
графики сходимости для $k\hm = 0{,}65$. Видно, что поведение параметров,
оптимизированных согласно классическому функционалу качества~$S$, является
существенно более хаотическим, что может говорить о меньшей 
устойчивости~\cite{Rudoy16StabilityAnalysis} модели, оптимизированной согласно~$S$.

Более того, для оценок параметров~$g_0$ и~$\alpha_0$ соответствующее
приближение на несколько порядков хуже, чем полученное минимизацией~$\breve{S}$,
вплоть до того, что кривые, соответствующие минимизирующим~$\breve{S}$ 
параметрам,
практически не видны на графиках (см.\ рис.~4,\,\textit{а} 
и~4,\,\textit{б}, правый столбец), поскольку в~выбранном масштабе они
практически совпадают с~осью абсцисс. С~другой стороны, важно отметить, что 
оценка параметра~$\gamma$, полученная минимизацией~$S$, 
является несколько лучшей для
$k \hm= 0{,}65$ (для $k \hm= 0{,}7$ график выглядит аналогично), но 
минимизация~$\breve{S}$
дает все лучшие и~лучшие приближения с~ростом~$k$ (см.\ 
рис.~5).

Отметим следующее:
\begin{itemize}
  \item практически во всех случаях (кроме оценки~$\gamma$ для $k \hm= 0{,}8$)
    предложенный в~настоящей работе функционал~\eqref{eq:s}
    дает лучшее приближение, в~том числе при разумно малом объеме обуча\-ющей 
выборки. Кроме того, в~подав\-ля\-ющем большинстве случаев предпочтительность 
предложенного  функционала  сохраняется и~для большего числа экспериментальных точек;\\[-8pt]
  \item для малых $k \hm\leq 0{,}6$ ошибка оценки параметров при помощи классического
    функционала~\eqref{eq:s_classic} имеет ярко выраженный минимум 
    в~окрест\-ности 60--100 для~$\alpha_0$ и~$\gamma$ и~400 для~$g_0$ 
    экспериментальных точек  (см.\ рис.~4, левый столбец).
    Дальнейшее увеличение обучающей выборки ведет к~ухудшению приближения, 
получаемого минимизацией~\eqref{eq:s_classic};
 \end{itemize}
 
 %\columnbreak

 { \begin{center}  %fig5
 \vspace*{1pt}
 \mbox{%
\epsfxsize=70.982mm
\epsfbox{rud-6.eps}
}
\end{center}

\vspace*{-1pt}


\noindent
{{\figurename~5}\ \ \small{Сходимость параметра $\gamma$ при $k \hm= 0{,}8$~(\textit{а}),
    0,9~(\textit{б}) и~1,0~(\textit{в}): \textit{1}~--- $\omega^0$;
    \textit{2}~--- $\omega$}}
}

\vspace*{3pt}

\addtocounter{figure}{1}

\

\begin{itemize}
 \item для некоторых~$k$ ошибка приближения, получаемого минимизацией 
предложенного функционала~\eqref{eq:s}, имеет явную горизонтальную асимптоту 
(см.\ рис.~4,\,\textit{а}, 4,\,\textit{б} и~4,\,\textit{в} (левый столбец)).
\end{itemize}

Причины подобного поведения оптимальных параметров являются предметом
дальнейших исследований.

\section{Заключение}

Предложен модифицированный функционал среднеквадратичной ошибки для задач
регрессии, применимый в~случае наличия ошибок измерения независимых
переменных и~различных распределений, к~которым принадлежат ошибки, в~разных
точках обучающей выборки. Предложена вероятностная интерпретация этого
функционала для случая нормального распределения ошибок измерения.

Показана сходимость предложенного функционала к~классическому функционалу
сред\-не\-квад\-ра\-тич\-ной ошибки для случая гомоскедастичности погрешностей
зависимой переменной и~пренебрежимо малой погрешности измерения независимых
переменных.

Исследовано поведение оптимального вектора параметров для предлагаемого
функционала в~зависимости от параметров распределений ошибок независимых
переменных, в~том числе в~сравнении с~вектором параметров, минимизирующим
классический функционал качества.

Представляется разумным использовать предложенный в~настоящей работе функционал
при оптимизации параметров регрессионных моделей и~анализе их
устойчивости к~погрешностям как зависимых, так и~независимых
переменных~\cite{Rudoy15MonteCarlo,Rudoy16StabilityAnalysis}.


{\small\frenchspacing
 {%\baselineskip=10.8pt
 \addcontentsline{toc}{section}{References}
 \begin{thebibliography}{99}
\bibitem{Gladun2004Labs}
\Au{Гладун А.\,В.}  
Лабораторный практикум по общей физике.~---
М.: МФТИ, 2004. 316~с.

\bibitem {Rudoy15MonteCarlo}
\Au{Рудой Г.\,И.}  О~возможности применения методов Мон\-те-Кар\-ло в~анализе
нелинейных регрессионных моделей~//
Сиб. ж. вычисл. мат., 2015. Т.~4. С.~425---434.

\bibitem {gillard2006historical}
\Au{Gillard J.\,W.} 2006.  
An historical overview of linear regression with errors in both variables.
Cardiff University School of Mathematics. Technical Report.

\bibitem {Deming1943Statistical}
\Au{Deming W.\,E.}  {Statistical adjustment of data.}~---
New York, NY, USA: Wiley, 1943.  216~p.

\bibitem {Bowden1990Instrumental}
\Au{Bowden R.\,J., Turkington D.\,A.}
{Instrumental variables}.~---
Cambridge: Cambridge University Press, 1990. 236~p.

\bibitem {Bekker1986Comment}
\Au{Bekker~P.\,A.}  Comment on
  identification in the linear errors in variables model~//
{Econometrica}, 1986. Vol.~54. No.\,1. P.~215--217.

\bibitem {Carrol06MeasurementErrors}
\Au{Carroll~R.\,J., Ruppert~D, Stefanski~L.\,A., Crainiceanu~C.\,M.} 
{Measurement error in nonlinear models: A~modern perspective}.~---
London: Chapman and Hall/CRC, 2006. 484~p.

\bibitem {jukic2013nonlinear}
\Au{Juki$\acute{\mbox{c}}$~D.} 
On nonlinear weighted least squares
  estimation of bass diffusion model~//
{Appl. Math. Comput.}, 2013.  Vol.~219. No.\,14. P.~7891--7900.

\bibitem {jukic2010nonlinear}
\Au{Juki$\acute{\mbox{c}}$~D.,
Markovi$\acute{\mbox{c}}$~D.} 
On nonlinear weighted errors-in-variables
  parameter estimation problem in the three-parameter Weibull model~//
{Appl. Math. Comput.}, 2010. Vol.~215. No.\,10. P.~3599--3609.

\bibitem {kiryati2000heteroscedastic}
\Au{Kiryati N., Bruckstein~A.\,M.} 
Heteroscedastic hough transform (HtHT): An
  efficient method for robust line fitting in the `errors in the
  variables' problem~//
{Comput. Vis. Image Und.}, 2000. Vol.~78. No.\,1. P.~69--83.

\bibitem {Boggs1987Stable}
\Au{Boggs~P.\,T., Byrd~R.\,H., Schnabel~R.\,B.}
A~stable and efficient algorithm for nonlinear orthogonal distance regression~//
{SIAM J.~Sci. Stat.  Comp.},   1987. Vol.~8. No.\,6. P.~1052--1078.

\bibitem {Marquardt1963Algorithm}
\Au{Marquardt~D.\,W.}  An algorithm for
  least-squares estimation of non-linear parameters~//
{J.~Soc. Ind. Appl.   Math.}, 1963. Vol.~11. No.\,2. P.~431--441.

\bibitem {dlib09}
\Au{King D.\,E.}  Dlib-ml: A~machine
learning toolkit~// {J.~Mach. Learn. Res.}, 2009. Vol.~10. P.~1755--1758.

\bibitem {alexandrov1991kinetics}
\Au{Александров~А.\,Ю., Долгих~В.\,А.,
Рудой~И.\,Г., Сорока~А.\,М.} Кинетика возбуждаемого электронным пучком 
лазера высокого давления на <<желтой>> линии неона~//
Квантовая электроника, 1991. Т.~18. №\,9. С.~1029--1033.

\bibitem {champagne1982transient}
\Au{Champagne L.\,F.}  Transient optical absorption 
in the ultraviolet~// \textit{Applied atomic collision physics. Vol.~3: Gas lasers}~/
Eds. E.\,W.~McDaniel, W.\,L.~Nighan.~--- Amsterdam, Netherlands:
Elsevier, 1982.  349--386.

\bibitem {Rudoy16StabilityAnalysis}
\Au{Rudoy~G.\,I.}  Analysis of the
  stability of nonlinear regression models to errors in measured data~//
{Pattern Recognit. Image Anal.}, 2016. Vol.~26. No.\,3. P.~608--616.
 \end{thebibliography}

 }
 }

\end{multicols}

\vspace*{-3pt}

\hfill{\small\textit{Поступила в~редакцию 15.09.16}}

%\vspace*{8pt}

\newpage

\vspace*{-28pt}

%\hrule

%\vspace*{2pt}

%\hrule

\vspace*{-2pt}


\def\tit{ON MODIFICATION OF THE MEAN SQUARED ERROR LOSS FUNCTION FOR~SOLVING NONLINEAR 
HETEROSCEDASTIC ERRORS-IN-VARIABLES PROBLEMS}

\def\titkol{On modification of the mean squared error loss function for~solving nonlinear 
heteroscedastic errors-in-variables problems}

\def\aut{G.\,I.~Rudoy}

\def\autkol{G.\,I.~Rudoy}

\titel{\tit}{\aut}{\autkol}{\titkol}

\vspace*{-9pt}


\noindent
Moscow Institute of Physics and Technology, 9~Institutskiy Per.,
Dolgoprudny, Moscow Region 141700, Russian Federation


\def\leftfootline{\small{\textbf{\thepage}
\hfill INFORMATIKA I EE PRIMENENIYA~--- INFORMATICS AND
APPLICATIONS\ \ \ 2017\ \ \ volume~11\ \ \ issue\ 2}
}%
 \def\rightfootline{\small{INFORMATIKA I EE PRIMENENIYA~---
INFORMATICS AND APPLICATIONS\ \ \ 2017\ \ \ volume~11\ \ \ issue\ 2
\hfill \textbf{\thepage}}}

\vspace*{3pt}


\Abste{The paper considers the problem of finding the optimal parameters of 
a~nonlinear regression model accounting for errors in both dependent and
  independent variables. The errors of different measurements are assumed to
  belong to different probability distributions with different variances.
  A~modified mean squared error-based loss function is derived and analyzed
  for this case.
  In the computational experiment, the measurements of the laser's radiation
  power as a~nonlinear function of the resonator's transparency are used to
  compare the parameters vectors minimizing the presented
  loss function and the classical mean squared error.
  The convergence of the parameters minimizing the presented loss function
  to the optimal parameters for the classical loss function is studied.
  In addition, some values of the parameters are considered to be ``true''
  ones and are used to generate synthetic data using the physical model and
  Gaussian noise, which is then used to study the convergence of the parameters
  minimizing the presented and the classical loss function, respectively, as
  the function of the noise parameters.}

\KWE{errors-in-variables models; heteroscedastic errors;
  symbolic regression; nonlinear regression}
  
\DOI{10.14357/19922264170209} 

%\vspace*{-18pt}

%\Ack
%\noindent



%\vspace*{3pt}

  \begin{multicols}{2}

\renewcommand{\bibname}{\protect\rmfamily References}
%\renewcommand{\bibname}{\large\protect\rm References}

{\small\frenchspacing
 {%\baselineskip=10.8pt
 \addcontentsline{toc}{section}{References}
 \begin{thebibliography}{99}

\bibitem{Gladun2004Labs-en}
\Aue{Gladun, A.\,V.} 2004. \textit{Laboratornyy praktikum po obshchey fizike} 
[Laboratory classes on general  physics].
Moscow: MFTI. 316~p.

\bibitem {Rudoy15MonteCarlo-en}
\Aue{Rudoy, G.\,I.} 2015. {Applying
    Monte Carlo methods to analysis of nonlinear regression
    models}.
\textit{Numer. Anal. \mbox{Appl.}} 4:344--350.

\bibitem {gillard2006historical-en}
\Aue{Gillard, J.\,W.} 2006.  
{An historical overview of linear regression with errors in both variables}.
Cardiff University School of Mathematics. Technical Report.

\bibitem {Deming1943Statistical-en}
\Aue{Deming, W.\,E.} 1943. \textit{Statistical adjustment of data.}
New York, NY: Wiley. 216~p.

\bibitem {Bowden1990Instrumental-en}
\Aue{Bowden, R.\,J., D.\,A.~Turkington}.
 1990. \textit{Instrumental variables}.
Cambridge: Cambridge University Press. 236~p.

\bibitem {Bekker1986Comment-en}
\Aue{Bekker, P.\,A.} 1986. Comment on
  identification in the linear errors in variables model.
\textit{Econometrica} 54(1):215--217.

\bibitem {Carrol06MeasurementErrors-en}
\Aue{Carroll, R.\,J., D.~Ruppert, L.\,A.~Stefanski, and C.\,M.~Crainiceanu.} 2006.
\textit{Measurement error in nonlinear models: A~modern perspective}.
London: Chapman and Hall/CRC. 484~p.

\bibitem {jukic2013nonlinear-en}
\Aue{Juki$\acute{\mbox{c}}$,~D.} 2013.
On nonlinear weighted least squares
  estimation of bass diffusion model.
\textit{Appl. Math. Comput.}  219(14):7891--7900.

\bibitem {jukic2010nonlinear-en}
\Aue{Juki$\acute{\mbox{c}}$, D.,
and D.~Markovi$\acute{\mbox{c}}$.} 2010.
On nonlinear weighted errors-in-variables
  parameter estimation problem in the three-parameter Weibull model.
\textit{Appl. Math. Comput.} 215(10):3599--3609.

\bibitem {kiryati2000heteroscedastic-en}
\Aue{Kiryati, N., and A.\,M.~Bruckstein.} 2000.
Heteroscedastic hough transform (HtHT): An
  efficient method for robust line fitting in the errors in the
  variables' problem.
\textit{Comput. Vis. Image Und.} 78(1):69--83.

\bibitem {Boggs1987Stable-en}
\Aue{Boggs, P.\,T., R.\,H.~Byrd, and R.\,B.~Schnabel.}
  1987. A~stable and efficient algorithm for nonlinear orthogonal distance regression.
\textit{SIAM J.~Sci. Stat.  Comp.} 8(6):1052--1078.

\bibitem {Marquardt1963Algorithm-en}
\Aue{Marquardt, D.\,W.} 1963. An algorithm for
  least-squares estimation of non-linear parameters.
\textit{J.~Soc. Ind. Appl.  Math.} 11(2):431--441.

\bibitem {dlib09-en}
\Aue{King, D.\,E.} 2009. Dlib-ml: A~machine
  learning toolkit.
\textit{J.~Mach. Learn. Res.} 10:1755--1758.

\bibitem {alexandrov1991kinetics-en}
\Aue{Aleksandrov, A.\,Yu., V.\,A.~Dolgikh,
I.\,G.~Rudoy, and A.\,M.~Soroka.} 1991.
Kinetics of a high-pressure electron-beam-excited laser emitting the ``yellow'' 
neon line. \textit{Sov. J.~Quantum Electronics} 21(9):933--937.

\bibitem {champagne1982transient-en}
\Aue{Champagne, L.\,F.} 1982. {Transient optical absorption 
in the ultraviolet}. \textit{Applied atomic collision physics. Vol.~3: Gas lasers}.
Eds. E.\,W.~McDaniel and W.\,L.~Nighan. Amsterdam, Netherlands: Elsevier.
349--386.

\bibitem {Rudoy16StabilityAnalysis-en}
\Aue{Rudoy, G.\,I.} 2016. Analysis of the
  stability of nonlinear regression models to errors in measured data.
\textit{Pattern Recognit. Image Anal.} 26(3):608--616.
\end{thebibliography}

 }
 }

\end{multicols}

\vspace*{-6pt}

\hfill{\small\textit{Received September 15, 2016}}

\vspace*{-18pt}

\Contrl

\noindent
\textbf{Rudoy Georg I.} (b.\ 1991)~--- PhD student,
Moscow Institute of Physics and Technology, 9~Institutskiy Per., Dolgoprudny,
Moscow Region 141700, Russian Federation;
\mbox{0xd34df00d@gmail.com}


\label{end\stat}


\renewcommand{\bibname}{\protect\rm Литература}   %

\def\stat{serebr}

\def\tit{ПЕРСОНАЛЬНАЯ ОТКРЫТАЯ СЕМАНТИЧЕСКАЯ ЦИФРОВАЯ БИБЛИОТЕКА 
LibMeta. КОНСТРУИРОВАНИЕ КОНТЕНТА. ИНТЕГРАЦИЯ 
С~ИСТОЧНИКАМИ LOD$^*$}

\def\titkol{Персональная открытая семантическая цифровая библиотека 
LibMeta. Конструирование контента} %. Интеграция с~источниками LOD}

\def\aut{О.\,М.~Атаева$^1$, В.\,А.~Серебряков$^2$}

\def\autkol{О.\,М.~Атаева, В.\,А.~Серебряков}

\titel{\tit}{\aut}{\autkol}{\titkol}

\index{Атаева О.\,М.}
\index{Серебряков В.\,А.}
\index{Ataeva O.\,M.}
\index{Serebryakov V.\,A.}


{\renewcommand{\thefootnote}{\fnsymbol{footnote}} \footnotetext[1]
{Работа выполнена при финансовой поддержке РФФИ (проект 14-07-00058~А).}}


\renewcommand{\thefootnote}{\arabic{footnote}}
\footnotetext[1]{Вычислительный центр им.\ А.\,А.~Дородницына Федерального исследовательского 
центра <<Информатика и~управ\-ле\-ние>> Российской академии наук, 
\mbox{oli@ultimeta.ru}}
\footnotetext[2]{Вычислительный центр им.\ А.\,А.~Дородницына Федерального исследовательского 
центра <<Информатика и~управ\-ле\-ние>> Российской академии наук, 
\mbox{serebr@ultimeta.ru}}

\vspace*{-3pt}

    

\Abst{Развитие семантических технологий вывело цифровые библиотеки на уровень, на 
котором на первый план выступила необходимость осмысленного представления контента 
цифровых библиотек. Одновременно возникает необходимость ограничения его 
в~терминах некоторой предметной области. В~работе рассматривается конструирование 
контента библиотеки для некоторой предметной области в~рамках разработанной системы 
LibMeta.
Персональная открытая семантическая цифровая библиотека LibMeta с~сис\-те\-мой 
поддержки работы пользователей с~цифровыми ресурсами библиотек и~их коллекциями 
для некоторой предметной об\-ласти, ограниченной терминологически с~по\-мощью 
тезауруса, предоставляет функциональность конструирования контента библиотеки 
согласно определенным требованиям и~требует всего лишь произвести начальную 
настройку сис\-те\-мы под конкретную предметную об\-ласть, ограниченную 
терминологически с~по\-мощью тезауруса. В~качестве примера предметной об\-ласти 
в~работе используется узкоспециализированный тезаурус обыкновенных 
дифференциальных урав\-не\-ний (ОДУ).}

\KW{семантические библиотеки; модель данных; онтологии; источники данных; 
поиск в~LOD}

\DOI{10.14357/19922264170210} 


\vskip 10pt plus 9pt minus 6pt

\thispagestyle{headings}

\begin{multicols}{2}

\label{st\stat}

\section{Введение}

     Взрывное развитие технологий в~последние десятилетия повлияло на 
все аспекты деятельности человека. Накопленные в~библиотеках данные 
стали через сеть доступны широкому кругу пользователей, удовлетворяя 
информационные потребности которых, разработчики расширяли 
функциональность цифровых библиотек. 
     
     Развитие семантических технологий вывело циф\-ро\-вые библиотеки на 
новый уровень, на котором на первый план выступила необходимость 
осмыс\-лен\-но\-го представления контента цифровых биб\-лио\-тек. В~решении этих 
задач ключевую роль стали играть онтологии~[1], позволяя представлять 
концептуальные модели для описания самого контента этих библиотек, 
основываясь на ранее разработанных форматах описания, таких как 
MARC\footnote[3]{{\sf http://www.loc.gov/marc/unimarctomarc21.html.}}. Такие 
онтологии получили название библиографических, дополняя семантикой эти 
форматы. Фактически в~библиографических онтологиях фиксируются 
ключевые понятия объектов, составляющих наполнение библиотеки, и~связи 
между ними. Этих понятий достаточно для описания обычной классической 
цифровой библиотеки для любой предметной области, в~которой 
представлена информация о~различных печатных изданиях и,~возможно, их 
электронные версии. Но развитие семантических библиотек~[2] способствует 
расширению модели, определяющей наполнение библиотеки, в~которой 
теперь могут содержаться самые различные типы объектов. 
     
     Одновременно с~расширением модели биб\-ли\-о\-теч\-но\-го наполнения 
возникает необходимость ограничения его в~рамках некоторой предметной 
области. Для этого вводится набор терминов, используемых для описания 
этой предметной об\-ласти. Чаще всего эти термины организованы в~виде 
некоторой таксономии с~поддержкой разнообразных связей между ними. 
В~дальнейшем будем называть наполнение библиотеки с~такой 
терминологической поддержкой некоторой предметной области контентом 
семантической цифровой библиотеки, или просто контентом.
     
Для тематической классификации ресурсов биб\-ли\-о\-те\-ки используются различные классификаторы, которые 
отличаются друг от друга охватом предметных областей и~степенью гранулярности при классификации этих 
областей. Для этих целей может использоваться один из широко распространенных классификаторов, таких 
как УДК\footnote[1]{{\sf http://nlib.sakha.ru/Cataloque/udk/index.shtml.}} (универсальная десятичная 
классификация), ББК\footnote[2]{{\sf http://roslavl.library67.ru/files/382/bbk.pdf.}} 
(биб\-лио\-теч\-но-биб\-лио\-гра\-фи\-че\-ская классификация), ГРНТИ\footnote[3]{{\sf 
http://www2.viniti.ru/index.php?option=com\_content\&view=article\&id=39:rubrikator-nti.}} 
(государственный рубрикатор на\-уч\-но-тех\-ни\-че\-ской информации). Эти классификаторы охватывают 
почти все области научного знания и~перечень понятий, характерных для этих областей. Обычно эти понятия 
носят довольно общий характер и~не отражают всего разнообразия направлений в~каждой отдельной 
области научного знания.
     
     Специализированные по конкретным областям библиотеки используют 
обычно свои классификаторы для систематизации своих ресурсов. Такой 
подход обеспечивает более детальный анализ содержания документов 
и~соотношение смысловых понятий в~документе с~определенным 
направлением специализированной области знания. К~таким 
классификаторам можно, например, отнести MSC\footnote[4]{{\sf 
http://www.ams.org/msc/pdfs/classifications2010.pdf.}} (Mathematics Subject 
Classification), который используется для классификации разделов 
математики. 
     
     Семантические библиотеки предоставляют своим пользователям 
большой арсенал возможностей для удовлетворения их информационных 
потребностей~[2]. Это разнообразные средства поиска: атрибутный поиск, 
полнотекстовый поиск, поиск по коллекциям на основе тематических 
классификаторов, поиск по разнообразным типам ресурсов, включенных 
в~библиотеку. Для пользователей семантических библиотек, являющихся 
активными потребителями информации, во многих современных решениях 
предоставляется возможность создать собственную коллекцию. 
     
     Возникает необходимость дать пользователям специфицировать свои 
предпочтения, развивая возможность определения собственных терминов 
в~рамках некоторого направления научного знания, уточняя и~очерчивая круг 
своих интересов, позволяя организовывать группы пользователей со 
сходными интересами для возможности отслеживания всей информации по 
определенным направлениям.
     
     Широкое применение онтологий позволяет интегри\-ровать данные 
библиотек с~данными из различных источников, основываясь на их 
семантике~[3]. Эти источники не обязательно сами являются библиотеками. 
Множество таких источников подключено к~облаку LOD (Linked Open 
Data)~[4]. Основная идея LOD заключается в~решении задач интеграции 
данных, представленных в~сети, для чего предлагается представить 
информацию в~формализованном виде с~по\-мощью онтологий, что делает ее 
доступной для машинной обработки. В~этих источниках данных провязаны 
самые различные типы ресурсов, которые представляют интерес для 
пользователей библиотек с~точки зрения обогащения данных как структурно, 
так и~семантически.
     
     На основе модели понятий, описанной в~предыду\-щих работах~[5], 
а~также идей Semantic Web и~LOD была разработана 
персональная открытая семантическая циф\-ро\-вая библиотека \mbox{LibMeta} 
с~системой поддержки работы пользователей с~циф\-ро\-вы\-ми ресурсами 
библиотек и~их коллекциями для некоторой предметной области, 
ограниченной терминологически с~помощью тезауруса~\cite{3-ser, 6-ser}.

\vspace*{-4pt}
     
\section{LibMeta~--- основные идеи}

\vspace*{-2pt}

     При реализации LibMeta авторы руководствовались набором основных 
задач, которые должна решать разрабатываемая система:
     \begin{enumerate}[(1)]
\item библиотека должна поддерживать возможность использования 
медийных объектов или ссылки на них при описании своих объектов, 
включая текст, аудио-, видеофайлы или любую их комбинацию. Это 
требование отражается в~названии словом <<цифровая>>;
\item типы используемых ресурсов и~связи между ними должны быть 
описаны средствами сис\-те\-мы в~рамках определенных в~предыдущей работе 
понятий, составляющих семантическое описание ресурсов контента 
библиотеки. При этом согласно принципам LOD при описании ресурсов 
поддерживается использование классов и~свойств ранее используемых 
онтологий в~сообществе, поддерживающем LOD. Эта поддержка 
выражается либо в~непосредственном использовании готовых онтологий 
при описании ресурсов и~связей между ними, либо возможностью ссылок 
на их элементы, используя связи на уровне описания ресурсов. Это 
требование отражается в~названии словом <<семантическая>>;
\item библиотека должна служить интеграционным узлом, предоставляя 
возможность связывания своих данных с~данными из разных источников, 
которые включены в~облако LOD. Должна также обеспечиваться 
возможность извлекать данные этой библиотеки в~машиночитаемом 
формате. Это требование отражается в~названии словом <<открытая>>;
\item пользователи библиотеки должны иметь возможность организовывать 
свои коллекции по интересующему их научному направлению, добавляя 
новые термины в~предметный тезаурус, уточняя таким образом область 
своих интересов. Пользователи должны также иметь возможность 
осуществлять поиск не только среди объектов в~рамках системы, но и~по 
источникам данных без необходимости использования 
специализированного языка для поисковых запросов. Это требование 
отражается в~названии словом <<персональная>>.
\end{enumerate}

     Основные требования, предъявляемые при этом\linebreak к~контенту  
сис\-те\-мы,~--- универсальность, структурированность, адаптируемость~--- 
не противо\-речат этим свойствам и~обеспечивают поддержку\linebreak настраива\-емого 
хранилища метаданных для объектов и~расширяемый набор 
информационных ресурсов. Универсальность обеспечивает описание типов 
ее ресурсов и~объектов независимо от предметной области и~области 
интересов пользователей. Структурированность описания обеспечивает 
поддержку связей между различными типами ресурсов как внутри системы, 
так и~вне ее, исходя из определений LOD. Адаптируемость описания 
ресурсов обеспечивает возможность добавления новых свойств и~связей 
в~процессе развития системы и~обеспечивает настройку пользовательских 
интерфейсов под эти изменения. 
     
     Фактически LibMeta предоставляет функциональность 
конструирования контента библиотеки согласно этим требованиям, и~на 
начальном этапе при установке системы требуется всего лишь произвести 
настройку системы под конкретную предметную область, описав ее ресурсы и~таксономии, которые будут очерчивать тематически предметную область 
ее ресурсов и~таким образом составлять ее тезаурус. 


     
\section{LibMeta~--- первый пример конструирования}



     Рассмотрим простой пример реализации биб\-лио\-те\-ки LibMeta, 
основанной на данных публикаций из электронной библиотеки <<Научное 
наследие России>>~\cite{7-ser}. Основных типов ресурсов, которые 
определены для этих данных, всего два: персоны и~публикации. Для 
тематической классификации этих публикаций используется классификатор 
\mbox{ГРНТИ}, и~каждая публикация снабжена номером УДК.
     
     Авторы не ставили своей целью создание уменьшенной копии 
<<Научного наследия>>. Основная цель, преследуемая в~контексте 
предлагаемой сис\-те\-мы,~--- это связывание этих данных с~данными, 
опубликованными в~LOD, и~их публикация для возможности доступа к~ним 
других систем. В качестве источника данных для связывания в~этом примере 
используется DBpedia\footnote{{\sf http://dbpedia.org.}}, служащая ядром LOD. 
     
     Итак, основная цель при конструировании описа\-ния контента 
заключается в~том, чтобы пред\-став\-лен\-ное описание по возможности 
макси\-мально облегчало реализацию процедуры поиска данных в~узлах LOD. 
Жертвой этой идеи становится,\linebreak возможно, некоторая упро\-щен\-ность 
структуры контента, в~отличие от выразительности, пред\-став\-ля\-емой 
средствами языка OWL\footnote{{\sf https://www.w3.org/2001/sw/wiki/OWL.}}, как 
будет показано ниже, но при этом получаем гибкость при по\-стро\-ении 
интеграционного узла для различных типов ресурсов, описание которых 
можно расширять в~процессе жизнедеятельности системы. 
     
     Фактически понятия \textit{персоны} и~\textit{пуб\-ли\-ка\-ции} 
представляют собой экземпляры класса \textit{информационный ресурс}, 
определенного как базовая единица контента семантической библиотеки. Так 
как каждый ресурс обладает набором атрибутов, для каждого из этих 
экземпляров задается собственный набор из множества атрибутов, 
предварительно описанных в~системе. Множество атрибутов для 
информационных ресурсов состоит из следующих элементов: \textit{название 
на языке оригинала, название на русском, фамилия, имя, отчество, 
электронный адрес, дата рождения, аннотация, идентификатор, автор, 
деятельность, тип публикации, место рождения, биография, описание, 
дополнительное заглавие, язык}. 
     
    Конкретные персоны~--- это объекты, пред\-ставля\-ющие экземпляры 
класса \textit{информационный объект}, они определяются 
информационным ре\-сурсом \textit{персона} и~представляются значениями %\linebreak 
атрибутов соответствующего ресурса. Помимо свойств, заданных 
атрибутами, представленными в~наборе атрибутов своего информационного 
ресурса, каж\-дый объект обладает также свойствами, общими для всех 
информационных объектов, такими как \textit{теги, описание, дата 
создания, дата изменения, владелец, уникальный идентификатор}. 
    
    На рис.~1 приведена упрощенная схема, сконструированная для этих 
типов ресурсов. На схе-\linebreak\vspace*{-12pt}

\pagebreak

\end{multicols}

     \begin{figure*} %fig1
      \vspace*{1pt}
\begin{center}
\mbox{%
\epsfxsize=158.401mm
\epsfbox{ser-1.eps}
}
\end{center}
\vspace*{-9pt}
\Caption{Конструирование информационных ресурсов}
\vspace*{4pt}
\end{figure*}

\begin{multicols}{2}

\noindent 
ме проиллюстрированы связи между экземплярами
информационных ресурсов \textit{персона} и~\textit{публикация} 
и~конкретными экземплярами класса информационного объекта (названия 
объектов \textit{объект}-\textit{п1}, \textit{объект}-\textit{п2},  
\textit{объект}-\textit{пб1} подчеркнуты). Префиксы <<ио>>, <<ир>>, 
<<к>>, <<т>>, <<тт>>, <<на>>, <<а>>, отделяемые двоеточием, 
указывают на принадлежность экземпляра соответственно к~классам 
\textit{информационный объект, информационный ресурс, коллекция, 
таксономия, таксон, набор атрибутов, атрибут}. 

Для тематической 
классификации объектов публикации используется коллекция, основанная на 
классификаторе ГРНТИ. 

        
    Серые стрелки, исходящие из экземпляров атрибутов, указывают на 
область возможных значений для них. Областью значений остальных 
атрибутов являются простые типы данных. На схеме значения атрибутов 
представлены с~помощью объектов вспомогательного класса \textit{значение 
атрибута} с~префиксом <<за>>. Объекты этого класса содержат для 
простых типов атрибутов их значения (например, значения текстовых 
атрибутов \textit{фамилия, имя, название} представлены на схеме 
в~кавычках).

 Для объектного атрибута \textit{автор} его значение содержит 
ссылку на соответствующий экземпляр информационного объекта с~типом 
персона,\linebreak\vspace*{-12pt}

\pagebreak

\end{multicols}

     \begin{figure*} %fig2
      \vspace*{1pt}
\begin{center}
\mbox{%
\epsfxsize=112mm
\epsfbox{ser-2.eps}
}
\end{center}
\vspace*{-3pt}
\Caption{Описание ресурсов в~формате RDF/XML}
\vspace*{9pt}
\end{figure*}

\begin{multicols}{2}

\noindent 
 что отображено
на схеме пунктирной стрелкой. Таксономические 
атрибуты \textit{тип} и~\textit{язык} в~качестве области значений указывают на 
соответствующие таксономии \textit{тип пуб\-ли\-ка\-ции} и~\textit{язык}, 
пред\-став\-ля\-ющие собой линейные словари, элементы которых (таксоны) 
используются в~качестве значений атрибутов. 
    
    Для каждого атрибута указан его вид: \textit{описательный, 
идентификационный} или \textit{поисковый}. Атрибут может относиться 
к~нескольким видам одновременно. Поисковые атрибуты используются для 
динамической генерации формы поиска по объектам определенного типа 
ресурсов. Описательные атрибуты используются для генерации формы 
представления информации об объекте для пользова\-теля.
{\looseness=1

} 

Набор значений 
идентификационных атрибутов необходим, как понятно из названия, для 
идентификации объекта. В~наборе атрибутов для публикаций атрибут 
\textit{автор} помечен как \textit{множественный}. Этот атрибут может 
иметь при описании информационных объектов, соответствующих по типу 
ресурса \textit{публикациям}, несколько значений, что отражено в~качестве 
примера на схеме.
     


    Описание структуры контента в~терминах \mbox{LibMeta} в~формате 
RDF/XML\footnote{{\sf http://www.w3.org/RDF/}; {\sf http://www.w3.org/XML/}; {\sf 
http://www.w3.org/TR/rdf-syntax-grammar.}} представлено на рис.~2. Задание 
структуры может осуществляться с~по\-мощью пользовательских 
интерфейсов системы или с~помощью загрузки RDF/XML с~описанием 
структуры контента в~соответствующем разделе системы пользователем, 
наделенным соответствующим уровнем прав.
    


    Основные понятия для описания контента биб\-лио\-те\-ки представлены 
в~работе~\cite{5-ser}. Фактически исходная онтология контента LibMeta 
содержит необходимые понятия, отношения и~аксиомы. При описании 
конкретной предметной области в~эту онтологию добавляются отдельные 
экземпляры определенных в~ней понятий, которые и~составляют контент 
создаваемой библиотеки. 



    
    

\pagebreak

\end{multicols}

     \begin{figure*} %fig3
      \vspace*{1pt}
\begin{center}
\mbox{%
\epsfxsize=160mm
\epsfbox{ser-3.eps}
}
\end{center}
\vspace*{-9pt}
\Caption{Информационный объект в~формате RDF/XML}
\vspace*{12pt}
\end{figure*}

\begin{multicols}{2}

%\noindent
На рис.~3 приведен пример представления экземпляра 
информационного объекта по заданному набору атрибутов из 
соответствующего ему экземпляра информационного ресурса.

\section{LibMeta~--- второй пример конструирования}

    Рассмотрим пример, когда в~качестве терминов предметной области 
используется узкоспециализированный тезаурус ОДУ~\cite{8-ser}. Особенность этого 
тезауруса заключается в~том, что он содержит не только сами понятия 
и~термины, но и~ссылки на публикации, в~которых  
вво\-дят\-ся/опре\-де\-ля\-ют\-ся эти понятия, их математическая запись. Был 
введен новый информационный ресурс \textit{литература} для описания 
публикаций, ставших основой построения этого тезауруса. В~соответствие 
ему был\linebreak поставлен тот же набор атрибутов, что и~в~преды\-ду\-щем примере. 
На рис.~4 представлены понятия тезауруса, связанные иерархически, и~для 
каж\-дого понятия отображаются его горизонтальные\linebreak связи.
{ %\looseness=1

}


    С помощью этого тезауруса был размечен набор публикаций со схожей 
тематикой. Схожесть тематики публикации тезаурусу ОДУ определялась по 
ее ключевым словам, соответствующим терминам тезауруса. 

На рис.~5 
представлен пример связи понятия из ОДУ и~найденных публикаций. 
В~качестве связанных объектов могут выступать не только 
\textit{публикации}, но и,~например, персоны, в~описании деятельности 
которых могут встречаться соответствующие понятию из ОДУ ключевые 
слова.
    
    В качестве модели информационных ресурсов было использовано то же 
описание \textit{персоны} и~\textit{пуб\-ли\-ка\-ции}, что и~в~предыдущем 
примере. В~этом случае один и~тот же набор атрибутов использовался как 
для описания ресурса \textit{литература}, так и~для описания ресурса 
\textit{пуб\-ли\-ка\-ция}. Это позволило отдельно настроить права доступа для 
всех объектов \textit{литературы}, запретив их модификацию\linebreak или удаление 
пользователям, не являющимся\linebreak редакторами предметной области. Сами 
публикации извлечены из Единого научного информа\-ционного пространства 
(ЕНИП) РАН~--- это ин\-тегрированное информационное пространство\linebreak\vspace*{-12pt}

\pagebreak

\end{multicols}

\begin{figure*} %fig4
 \vspace*{1pt}
\begin{center}
\mbox{%
\epsfxsize=160mm
\epsfbox{ser-4.eps}
}
\end{center}
\vspace*{-9pt}
\Caption{Тезаурус ОДУ}
%\end{figure*}
%\begin{figure*} %fig5
 \vspace*{14pt}
\begin{center}
\mbox{%
\epsfxsize=160mm
\epsfbox{ser-5.eps}
}
\end{center}
\vspace*{-9pt}
\Caption{Информационные объекты и~термины ОДУ}
\vspace*{12pt}
\end{figure*}


\begin{multicols}{2}

\noindent
распределенных и~локальных цифровых (электронных) ресурсов организаций 
РАН и~комплекс про\-граммно-тех\-ни\-че\-ских средств, обеспе\-чи\-ва\-ющих 
использование этих ресурсов и~полнофункциональное управление 
ими~\cite{9-ser}. 

Для извлечения информации о~публикациях и~авторах 
использовался
 протокол OAI-PMH\footnote[1]{{\sf 
https://www.openarchives.org/pmh.}}. Данные были представлены в~формате Dublin 
Core\footnote[2]{{\sf http://dublincore.org.}}. 
%
Часть публикаций была размечена 
ключевыми словами, однако термины не разделены между собой и~просто 
перечислялись через запятую в~одном поле. 
%
Ключевые слова пуб\-ли\-ка\-ции 
были преобразованы в~набор ключевых слов соответствующего 
информационного объекта для каждой извлеченной публикации из ЕНИП. 

В~коллекцию публикаций тезауруса ОДУ добавлялись те объекты, в~наборе 
ключевых слов которых находились термины ОДУ. 
    
    Размещение публикации в~ту или иную ветвь коллекции может 
осуществлять как сам пользователь, так и~соответствующий модуль 
автоматической разметки информационных объектов по тезаурусу, в~котором 
задаются простые правила разметки в~рамках описания тезауруса.
    
\section{LibMeta~--- взаимодействие с~источниками LOD}

    Основная проблема приложений, разрабатыва\-емых для работы 
с~данными из источников, интегрированных в~LOD, состоит в~том, что 
данные в~этих источниках очень слабо провязаны с~данными других 
источников. Большинство имеющихся связей расположены на уровне самих 
данных, при этом на уровне схем такие связи практически отсутствуют. Для 
решения этой проблемы предлагаются разные подходы, которые основаны на 
методах сравнения онтологий~\cite{1-ser, 4-ser}. Некоторые из них 
используют для сравнения онтологий данные, доступные в~сети. В~част\-ности, 
используется Wikipedia и~ее иерархический рубрикатор\footnote{{\sf  
https://en.wikipedia.org/wiki/Special:Categories.}}. 
    
    В отличие от других работ, построение иерархий классов при 
трансляции ресурсов на источник данных в~данной ситуации не представляет 
интереса, поэтому для проставления связей используется связь, которая 
указывает, что два разных класса \textit{могут} иметь одинаковых 
представителей. Эта связь может указывать на класс в~источнике данных 
LOD,\linebreak который является источником дополнительной информа\-ции  
о~ре\-сур\-се-субъ\-ек\-те, или на эквивалентный ему класс, возможно, 
с~разной степенью детализации описания объектов. Фактически 
предполагается, что онтология источника данных \textit{час\-тич\-но} 
совместима со структурой ресурсов описываемой библиотеки. Это означает, 
что хотя бы один ресурс ее онтологии может быть транслирован в~некоторый 
класс в~онтологии источника данных. Требуется лишь минимальное 
частичное соответствие ресурсу LibMeta. Это означает, что для однозначной 
идентификации экземпляров соответствующего класса из источника данных 
отображаться должны как минимум идентифицирующие атри\-буты. 

    
    В~связи с~гибкостью схемы LibMeta предполагается возможным 
сценарий создания дополнительных типов ресурсов для подключаемых 
источников, информацию из которых можно использовать как значения 
некоторых атрибутов основных ресурсов.
{\looseness=1

}
    
    Для трансляции атрибутов ресурса в~свойства выбранного класса 
источника данных будет использоваться связь, которая указывает, что 
значения атрибута и~свойства полностью или частично совпадают в~рамках 
установленного соответствия на уровне ресурса библиотеки и~класса 
онтологии. При совпадении значений все очевидно, проблема возникает при 
разной детализации данных, когда возможно отображение значения свойства 
класса в~несколько атрибутов и~наоборот. В~связи с~гиб\-костью модели 
данных может быть принято решение о~расширении схемы ресурса. 
В~другом случае пользователь может использовать набор вспомогательных 
функций, например для расщепления или слияния данных. Простой пример 
такого рода преобразований связан с~трансформацией имени персоны. 
В~первом случае, когда имя персоны описывается одним значением 
в~источнике данных, а транслируется в~три отдельных атрибута, 
используется функция расщепления (\textit{ФИО}\;$\to$\;\textit{Ф, И, O}). Во 
втором случае значения отдельных свойств преобразуются в~значения одного 
атрибута (\textit{Ф, И, O}\;$\to$\;\textit{ФИО}), отображение свойств класса 
производится в~один атрибут и~тогда данные будут склеиваться как одно 
значение этого атрибута именно в~том порядке, в~котором они были 
перечислены при описании трансляции.
    
    Для преобразования данных в~соответствующие типы значений, которые 
указаны при описании\linebreak атрибута, использованы встроенные функции 
пре\-об\-ра\-зо\-ва\-ния, которые нет нужды настраивать пользователю. В~случае 
если такое преобразование заканчивается неудачно, то информация об этом\linebreak 
будет сохранена в~соответствующем административном атрибуте объекта для 
возможности дальнейшей обработки и~исправления ошибок.
    
    Поиск эквивалентных классов в~источниках данных пользователь может 
выполнить: 
    \begin{enumerate}[(1)]
\item вручную, выбирая из списка доступных классов в~указанном 
источнике данных;
\item полуавтоматически, используя имеющиеся описания связей 
с~другими классами внешних онтологий, заранее определенными при 
описании структуры ресурсов;
\item автоматически. 
\end{enumerate}

    В первых двух случаях пользователь на первом шаге предварительно 
указывает, с~каким типом ресурсов он предполагает работать, привязывая тот 
или иной источник данных. На втором шаге он определяет соответствие 
атрибутов и~свойств. В~треть\-ем варианте он получает возможность получить 
общую оценку соответствия схемы ресурсов библиотеки некоторой 
онтологии и~на основе этой оценки принимать решение о~трансляции 
ресурсов на тот или иной источник данных, использующий эту онтологию.
    
    Назовем проекцию понятия библиотеки $\mathrm{IR}$ на понятие~$C$ источника 
данных из LOD \textit{допустимой}, если возможно установить между ними 
хотя бы одно отношение из $\{R_1, R_2, R_3, R_4\}$: 
    \begin{itemize}
\item $R_1(C, \mathrm{IR})$ означает, что понятие~$C$ включает в~себя $\mathrm{IR}$;
\item $R_2(C, \mathrm{IR})$ означает, что понятие $\mathrm{IR}$ включает в~себя~$C$;
\item $R_3(C, \mathrm{IR})$ означает, что понятие $\mathrm{IR}$ связано отношением 
эквивалентности с~$C$;
\item $R_4(C, \mathrm{IR})$ означает, что понятие $\mathrm{IR}$ связано отношением 
частичной эквивалентности с~$C$.
\end{itemize}

    Все эти четыре отношения говорят о том, что понятия $\mathrm{IR}$ и~$C$ могут 
иметь одинаковых представителей. По семантике~$R_1$ соответствует 
skos:broader (например, $\mathrm{IR}$\;=\;\textit{Человек}, $C$\;=\;\textit{Студент}); 
$R_2$ соответствует skos:narrower (например, $\mathrm{IR}$\;=\;\textit{Студент}, 
$C$\;=\;\textit{Человек}); $R_3$ соответствует skos:exactMatch (например, 
$\mathrm{IR}$\;=\;\textit{Студент}, $C$\;=\;\textit{Студент}); $R_4$ соответствует 
skos:closeMatch (например, $\mathrm{IR}$\;=\;\textit{Студент}, 
$C$\;=\;\textit{Учащийся}).
    
    Отображение $\mathrm{IR}$ на~$C$ \textit{возможно}, если трансляция набора 
атрибутов $\mathrm{IR}$ на свойства класса~$C$ выполнена хотя бы для 
идентифицирующих атрибутов, т.\,е.\ любой идентифицирующий атрибут 
из $a_1, \ldots,  a_k$, принадлежащих набору атрибутов~$\mathrm{IR}$, где $k\hm< n$, 
$n$~--- число атрибутов соответствующего набора, транслируется хотя бы на 
одно свойство~$c_1, \ldots , c_m$ класса~$C$.
    
    Трансляция атрибута на свойство источника $t(a_i, c_j)$ для искомых 
или связываемых объектов может быть:
    \begin{itemize}
\item \textit{прямой}, когда атрибут отображается на свойство: 
$t_1\hm=\{t(a_i, c_j):\ a_i \hm= c_j\}$, т.\,е.\ значения должны быть 
эквивалентны;
\item \textit{неполной}, когда атрибут отображается на свойство лишь 
частично: $t_2\hm=\{t(a_i, c_j):\ a_i\subset c_j\}$, т.\,е.\ значение атрибута 
включается в~значение свойства;
\item \textit{избыточной}, когда атрибут шире свойства: $t_3\hm=\{t(a_i, 
c_j): a_i \supset c_j\}$, т.\,е.\ значение атрибута содержит больше 
информации, чем значение свойства.
\end{itemize}

    Из определения проекции некоторого понятия~$\mathrm{IR}$ и~трансляции его 
атрибутов, задающих отоб\-ра\-же\-ние понятия на некоторый набор данных 
источника, следует, что это отображение сюръективно и~неинъективно для 
наборов его атрибутов. 
    
    При использовании трансляции для поиска связанных объектов в~случае 
\textit{полной} трансляции идентифицирующего атрибута все ограничивается 
выбо\-ром соответствующего свойства и~значения могут сравниваться. 
В~случае \textit{неполной} и~\textit{избыточной} трансляции 
идентифицирующего атрибута явного сравнения значений недостаточно. 
В~любом случае возможно использование функций предобработки данных: 
преобразование форматов, извлечение подстрок и~т.\,д.
    
    В общем виде извлечение объектов из источника для сохранения 
в~качестве информационных объектов библиотеки задается функцией
    \begin{multline*}
    z\left( R_l\cap f(t)\right) = {}\\
    {}=\left\{ o\in \mathrm{IR}\vert R_l(\mathrm{IR},C):\ \forall\ a_i\exists\ 
f(t(a_i, c_j))\right\}\,,
    \end{multline*}
где функция $f$ зависит от типа трансляции атрибутов:
\begin{gather*}
f(t_1)=\begin{cases}
\mbox{true}, &\ a_i=c_j\,;\\
\mbox{false}, &\  a_i\not=c_j\,;
\end{cases}\\
f(t_2)= \begin{cases}
\mbox{true}, &\ a_i\subset c_j\,;\\
\mbox{false}, &\ a_i \not\subset c_j\,;
\end{cases}
\\
f(t_3) = \begin{cases}
\mbox{true}, &\ c_j\subset a_i\,;\\
\mbox{false}, &\ c_j \not\subset a_i\,.
\end{cases}
\end{gather*}
    
    Для того чтобы выполнять поисковые запросы по источникам данных, 
необходимо, чтобы отображение понятий библиотеки на понятия источника 
было и~\textit{возможным}, и~\textit{допустимым}.
    
    Если оно возможно, то для $R_1(C, \mathrm{IR})$ это означает, что все 
характерные признаки~$\mathrm{IR}$ наследуются~$C$, при этом набор 
признаков~$C$ шире набора~$\mathrm{IR}$, так как $\mathrm{IR}$ является более
 объемлющим 
понятием и,~следовательно, класс~$C$ всегда включает признаки, которые 
являются идентифицирующими для более широкого понятия~$\mathrm{IR}$. Если оно 
возможно для $R_2(C, \mathrm{IR})$, это означает, что все признаки класса~$C$ 
наследуются~$\mathrm{IR}$, при этом набор признаков~$\mathrm{IR}$ может быть 
шире набора 
класса~$C$, так как~$\mathrm{IR}$ является более узким понятием. Набор 
идентифицирующих признаков класса~$C$, необходимых для 
идентификации объектов в~источнике данных, включается в~набор 
признаков~$\mathrm{IR}$, что достаточно для идентификации эквивалентных объектов 
в~источнике. Если набор необходимых идентифицирующих признаков~$\mathrm{IR}$ 
шире, то это означает, что в~контексте источника этот набор избыточен 
и~можно его переопределить (сузить) для по\-стро\-ения допустимой 
трансляции. Если оно возможно для $R_3(C, \mathrm{IR})$, это означает, что все 
характерные признаки совпадают, в~том числе и~идентифицирующие. Если 
трансляция возможна, то для понятий, связанных отношением $R_4(C, \mathrm{IR})$, 
это означает, что хотя бы идентифицирующие характеристики у~них 
совпадают.
    
    Исходя из сказанного, для любых понятий, связанных одним из этих 
отношений, возможно построение допустимой трансляции и~можно строить 
поисковые запросы, результатом которых являются интерпретируемые 
в~терминах ресурсов библиотеки объекты.


    
    Если же отображение недопустимо, то, значит, для всех четырех 
вариантов нарушается отображение идентифицирующих атрибутов, т.\,е.\ 
невозможно извлечь интерпретируемые данные (пример: достанем всех 
персон по имени Лев, но не сможем понять, кто из них Толстой, если не 
отобразим идентифицирующие атрибуты, которые определены в~библиотеке, 
например, как Ф, И, О, ДР). 
    
    Если отображение невозможно, но допустимо, т.\,е.\ некоторый набор 
атрибутов можно отобразить на некоторые свойства, то, так как один набор 
атрибутов может соответствовать нескольким по\-ня\-ти\-ям/ти\-пам ресурсов 
в~библиотеке, невозможно будет идентифицировать тип ресурса 
извлекаемого объекта (например, если имеются понятия \textit{Мужчина} 
и~\textit{Женщина}, набор атрибутов которых одинаков, то, извлекая персону 
с~именем <<\textit{Джойс Кэрол Оутс}>> с~датой рождения 
<<16.06.1938>>, нельзя определить ее принадлежность к~понятию). Поэтому 
при построении отображения надо последовательно проходить этап 
построения возможных проекций ресурсов, а затем этап построения 
допустимых трансляций атрибутов этих ресурсов.
    
    Авторы не претендуют на идеальную модель отоб\-ра\-же\-ния в~любой 
источник, но, по крайней мере, имея адаптивную модель данных, всегда 
можно выполнить настройку таким образом, чтобы иметь возможность 
извлечь интересующие данные для определенного круга задач.
    
    Введя функцию интерпретации~$I_z$, которая по построенным 
отображениям~$T_i$ для источников данных~$D_i$ и~информационного 
ресурса~$\mathrm{IR}$ сопоставляет \textit{информационным объектам}, 
со\-от\-вет\-ст\-ву\-ющим этому ресурсу, объекты источников данных, можно 
построить множество связей этих объектов. Функция~$I_z$ называется 
моделью связанных данных источников данных и~LibMeta. 

Если не удается 
построить функцию интерпретации некоторого источника для хотя бы 
одного ресурса, то надо либо расширять модель LibMeta, либо источник 
отбрасывается как источник с~данными, не интерпретируемыми в~данной 
библиотеке. Таким образом, можно выявить скрытые связи между 
источниками данных как на уровне схем, так и~на уровне данных через 
ресурсы библио\-теки. 
{\looseness=1

}


     
\section{LibMeta~--- пример использования внешних онтологий 
при описании ресурсов и~подключении к~источникам}

    Существуют два пути настройки системы под конкретные понятия 
предметной области: (1)~воспользоваться формами создания 
и~редактирования ресурсов, что удобно, когда экземпляров ресурсов мало; 
(2)~воспользоваться имеющимися онтологиями, описывающими понятия 
предметной области. Если в~первом случае все достаточно тривиально, то на 
втором случае следует остановиться подробней.
{\looseness=1

}
    
    В~системе предусмотрена подсистема загрузки онтологии, 
представленной на языке описания онтологий OWL, для автоматического 
создания экземпляров информационных ресурсов. При загрузке онтологии 
можно указать, какие классы онтологии извлекаются из нее, при этом они 
становятся экземплярами класса \textit{информационный ресурс} в~\mbox{LibMeta}, 
а~их свойства преобразуются в~экземпляры атрибутов и~включаются в~один 
набор атрибутов для конкретного ресурса. Естественно, \mbox{LibMeta} не 
поддерживает всю семантику отношений и~ограничений, накладываемых на 
свойства и~классы в~OWL, но отображает основные свойства и~ограничения 
на свою схему. Этого достаточно для выполнения задач интеграции 
и~публикации данных в~облаке LOD. Так, в~наборе данных, полученном из 
<<Научного наследия России>> с~по\-мощью подсистемы харвестинга по 
протоколу OAI-PMH, вся исходная информация о~том, где находится 
исходный объект со всем своим описанием, сохраняется. С~по\-мощью 
\mbox{LibMeta} выполняется только провязывание данных с~данными из LOD 
и~пользователям предоставляется возможность организовывать их 
в~собственные коллекции, возможно, дополняя описания данных на свое 
усмотрение, добавляя новые теги или определяя более точно тематическую 
направленность, используя в~качестве базовых расширяемых понятий 
элементы из ГРНТИ. Перечислим правила отоб\-ра\-же\-ния внешней онтологии 
в~описание контента библиотеки:
    \begin{itemize}
\item классы онтологии становятся экземплярами класса 
\textit{информационный ресурс};
\item свойства класса онтологии становятся экземплярами класса 
\textit{атрибут};
\item все свойства, относящиеся к~одному классу, группируются 
в~\textit{наборы атрибутов};
\item для простых свойств в~качестве области значений атрибута указывается 
соответствующий тип (строка, дата и~т.\,д.);
\item по умолчанию все атрибуты относятся к~виду \textit{описательный} 
и~\textit{поисковый}, соответствующие настройки можно изменить 
в~дальнейшем через интерфейсы системы;
\item для сложных свойств, областью значений которых являются 
экземпляры некоторого класса, выбирается соответствующий ресурс; если 
ресурс не был загружен в~систему, то дальнейшие решения принимает 
пользователь, отвечающий за создание структуры контента библиотеки;
\item все однозначные свойства становятся однозначными атрибутами, для 
атрибутов многозначных свойств ставится пометка о~воз\-мож\-ности 
подключения нескольких значений;
\item иерархические связи между классами не сохраняются в~явном виде, но 
для ресурсов создается атрибут с~соответствующим назнанием 
<<\textit{вышестоящий объект}>> и~<<\textit{нижестоящий объект}>>, 
кото\-рый позволяет устанавливать подобные связи на уровне объектов 
(практически идентичны по смыслу связям из онтологии SKOS\footnote{{\sf  
https://www.w3.org/2004/02/skos.}} skos:broader, skos:narrower).
\end{itemize}

    Инверсивность, транзитивность и~симметричность свойств не 
отображаются явно в~описании ресурсов LibMeta, но для каждого 
информационного объекта можно всегда получить список ссылающихся на 
него объектов и~информацию о~том, посредством какого атрибута это 
делается. Фактически онтологическое описание классов сводится к~набору 
\textit{поисковых} и~\textit{описательных} атрибутов выделением среди них 
\textit{идентифицирующих}, используемых, например, в~задачах выявления 
дубликатов.
    
    Важно отметить, что все идентификаторы классов и~свойств онтологии 
сохраняются в~описании\linebreak соответствующих экземпляров ресурсов 
и~атрибутов с~помощью использования связи, опре\-де\-ля\-ющей их 
эквивалентность. Это позволяет в~дальнейшем при настройке отображения 
ресурса на\linebreak некоторый источник в~LOD, который в~своей схеме использует 
эти классы и~свойства, выполнять эту процедуру почти полностью 
автоматически. При конструировании структуры контента через интерфейсы 
также имеется возможность для каждого ресурса и~его атрибута указать 
соответствующие им URI\footnote[2]{{\sf https://tools.ietf.org/html/rfc3986.}} из 
общеиспользуемых онтологий. 
    
    Рассмотрим в~качестве примера онтологии, ко\-то\-рые широко 
распространены для описания основ\-ных типов ресурсов рассматриваемых 
понятий \textit{персона} и~\textit{пуб\-ли\-ка\-ция} в~сообществе LOD, 
и~оценим их описания на соответствие имеющимся в~наличии метаданным. 
Чаще всего эти онтологии содер\-жат десятки классов и~свойств и~являются 
избыточными для описания нужных объектов, выделяя подмножество 
необходимых классов и~свойств, используя для отображения лишь малую 
часть их свойств, необходимых для подключения к~источникам данных, 
которые они охватывают.

\vspace*{-4pt}
    
     \subsection*{AKT}
     
\vspace*{-2pt}
     
    Онтология AKT Reference Ontology, или кратко AKT\footnote[3]{ {\sf 
http://projects.kmi.open.ac.uk/akt/ref-onto.}} (доступна по адресу {\sf 
http://swl. slis.indiana.edu/repository/owl/aktportal.owl}), разработана в~целях 
унификации доступа к~библиографической информации в~2003~г. И~хотя 
проект был закрыт, данные AKT на сегодняшний момент представлены более 
чем в~200~источниках, таких как DBLP, Citeseer, CORDIS, NSF, EPSRC, 
ACM, IEEE и~др.
    
    Объединяет несколько онтологий; из них интерес представляет 
основная онтология Portal Ontology, которая содержит понятия для 
описания \textit{персон} и~\textit{пуб\-ли\-ка\-ций}. Данные разнородны 
и~опираются на очень узкие подмножества этой онтологии. Многие поля, 
имеющиеся в~этой богатой онтологии, остаются незаполненными при 
описании реальных данных.

\vspace*{-4pt}
     
     \subsection*{Dublin Core}
     
\vspace*{-2pt}
     
    Исторически Dublin Core представляет собой набор понятий, 
используемых для описания разнообразных типов ресурсов, из 
которых~15~являются обязательными для описания. Практически можно 
описать метаданные о~\textit{персонах} и~\textit{пуб\-ли\-ка\-ци\-ях} из 
рассматриваемого примера в~терминах этих понятий. Элементы Dublin Core часто 
повторно используются, дополняются и~конкретизируются в~других 
онтологиях. Охватывает огромное число источников, включая DBpedia.

\vspace*{-4pt}
    
     \subsection*{FOAF}
     
\vspace*{-2pt}
     
    Онтология FOAF\footnote[4]{{\sf http://xmlns.com/foaf/spec.}}
    (Friend-of-a-Friend)  уже является 
практически стандартом для описания людей и~их отношений с~другими 
ресурсами. Используется в~разнообразных контекстах и~может 
использоваться для описания в~любых сценариях с~участием персон. Часто 
также включается и~конкретизируется в~других онтологиях.

%\vspace*{-4pt}
    
     \subsection*{BIBO}
     
     
     %\vspace*{-2pt}
     
     Онтология BIBO\footnote[1]{ {\sf http://bibliontology.com.}} 
     (Bibliographic Ontology)
     предназначена для 
описания библиографических данных, включает в~себя понятия из других 
онтологий, таких как Dublin Core и~FOAF, расширяя и~конкретизируя их 
понятия, которые используются при описании ее классов. 
Содержит~38~видов документов, вклю\-ча\-ет понятия, необходимые для 
описания \textit{персон} и~\textit{пуб\-ли\-ка\-ций}. Можно представить 
описание собственных ресурсов в~терминах этой онтологии, ограничившись 
лишь частью ее терминов. Охватывает такие источники, как Британская 
национальная библиотека\footnote{{\sf http://www.bl.uk.}}, DBpedia и~т.\,д.

%\vspace*{-4pt}
    
     \subsection*{DBpedia}
     
     %\vspace*{-2pt}
     
    Онтология Dbpedia, разработанная в~рамках проекта Dbpedia, содержит 
большое число классов для описания самых разнообразных объектов, 
включая понятия \textit{пуб\-ли\-ка\-ция} и~\textit{персона}. Она также 
включает в~себя понятия из других онтологий, которые используются при 
описании ее классов. DBpedia является центральным узлом LOD и~связывает 
информацию из самых разных источников, которые ссылаются на нее. 


    
    Таблица~1 показывает отображение информационного ресурса 
<<Публикация>> в~термины источников данных, схема данных которых 
опирается на
 перечисленные онтологии. Если для указанных онтологий нет 
соответствующего класса, то название класса не указывается. Это всего лишь 
означает, что элементы этой онтологии могут использоваться в~другой 
онтологии, где они конкретизируются в~рамках используемого класса. Если 
не указывается свойство, значит, в~терминах этой онтологии нет такого 
свойства или близкого ему по смыслу. В~случае с~BIBO один из 
перечисленных классов определяет тип публикации. Поиск публикаций 
в~DBpedia представляется бессмысленным в~рассматриваемых примерах, 
поэтому в~табл.~1 информация из этой онтологии не включалась


    В табл.~2 представлено отображение информационного ресурса 
<<Персона>> в~термины источников данных, схема данных которых 
опирается на перечисленные онтологии. Например, для персон из DBpedia 
представлены элементы из собственного пространства имен, но DBpedia 
также включает и~FOAF-он\-то\-ло\-гию, поэтому можно было отобразить 
значения и~на пространство имен FOAF в~рамках источника данных DBpedia.


    
    Эта информация об отображении атрибутов \mbox{LibMeta} на свойства других 
онтологий также может быть включена в~описание каждого атрибута 
с~помощью использования соответствующей связи. Эта информация 
позволит быстро подключаться к~нужным источникам данных 
и~формировать описания объектов в~терминах нужной онтологии.
В~рас\-смат\-ри\-ва\-емом примере были подключены два источника данных~--- 
это Dbpedia и~данные о персонах из системы MathNet\footnote[3]{{\sf  
http://www.mathnet.ru.}}, выгруженные предварительно в~отдельное хранилище 
в~виде RDF-тро\-ек в~формате Dublin Core.

%\columnbreak
 %\vspace*{-12pt}


\end{multicols}

\begin{table*}[h]\small
%\vspace*{-12pt}
\begin{center}
\Caption{Элементы описания ресурса <<Публикация>>}
\vspace*{2ex}

\tabcolsep=2.4pt
\begin{tabular}{|l|c|c|c|c|}
\hline
\multicolumn{1}{|c|}{Libmeta}&AKT&Dublin Core&FOAF&BIBO\\
\hline
&\tabcolsep=0pt\begin{tabular}{c}Класс\\
Akt:Publication-Reference
\end{tabular}&\tabcolsep=0pt\begin{tabular}{c}Класс\\
Dc:bibliographicresource\end{tabular}&Класс
&\tabcolsep=0pt\begin{tabular}{c}Класс\\
Bibo:Article,\\ bibo:academicarticle, \\bibo:Proceedings\end{tabular}\\
\hline
Название&akt:has-title&dc:title&foaf:title&dc:title\\
\hline
Аннотация&akt:has-abstract&dc:description&&bibo:abstract\\
\hline
\tabcolsep=0pt\begin{tabular}{l}Дополнительное\\ заглавие\end{tabular}
&&&&bibo:shortTitle\\
\hline
Тип&akt:article-of-journal&dc:type&&\\
\hline
Язык&&dc:contributor&&dc:contributor \\
\hline
Автор&akt:has-author&dc:language&&dc:language\\
\hline
\tabcolsep=0pt\begin{tabular}{l}Исходная\\ страница\end{tabular}&akt:has-web-address&dc:source&foaf:homepage&foaf:homepage\\
\hline
Описание&akt:addresses-generic-area-of-interest&&&bibo:shortDescription\\
\hline
\end{tabular}
\end{center}
\vspace*{-6pt}
\end{table*}
    
    
    \begin{table*}\small %tabl2
\begin{center}
\Caption{Элементы описания ресурса <<Персона>>}
\vspace*{2ex}

\begin{tabular}{|l|c|c|c|c|c|}
\hline
\multicolumn{1}{|c|}{LibMeta}&AKT&Dublin Core&FOAF&BIBO&DBpedia\\
\hline
&Класс
&\tabcolsep=0pt\begin{tabular}{c}Класс\\
Dc:agent\end{tabular}&\tabcolsep=0pt\begin{tabular}{c}Класс\\
Foaf:agent, \\foaf:person \end{tabular}&\tabcolsep=0pt\begin{tabular}{c}Класс\\
Foaf:agent, \\foaf:person\end{tabular}&\tabcolsep=0pt\begin{tabular}{c}Класс\\
Dbo:person\end{tabular}\\
\hline
Фамилия&\tabcolsep=0pt\begin{tabular}{c}
akt:full-name,\\
akt:family-name
\end{tabular}&dc:title&\tabcolsep=0pt\begin{tabular}{c}foaf:name, \\
foaf:lastname, \\
foaf:family\_name\end{tabular}&foaf:family\_name&dbo:birthName\\
\hline
Имя&\tabcolsep=0pt\begin{tabular}{c}
akt:full-name,\\
akt:given-name
\end{tabular}&dc:title&\tabcolsep=0pt\begin{tabular}{c}
foaf:name, \\
foaf:given\_name
\end{tabular}&foaf:given\_name&dbo:birthName\\
\hline
Отчество&akt:full-name&dc:title&\tabcolsep=0pt\begin{tabular}{c}
foaf:name, \\
foaf:surname
\end{tabular}&&dbo:birthName\\
\hline
Дата рождения&&dc:date&foaf:birthData&&dbo:birthDate\\
\hline
Место рождения&&&&&dbo:birthPlace\\
\hline
Биография&&dc:description&&&dbo:abstract\\
\hline
Деятельность&&dc:description&&&dbo:occupation\\
\hline
\end{tabular}
\end{center}
\end{table*}

\begin{multicols}{2}





    
\section{LibMeta~--- роль пользователей в~системе}

    Несомненно, самыми главными действующими лицами в~любой 
библиотеке являются ее пользователи, и~LibMeta не исключение. 
Пользователи LibMeta делятся на несколько категорий в~соответствии со 
своими ролями: 
    \begin{enumerate}[(1)]
\item администраторы контента библиотеки; 
\item редакторы предметной области; 
\item администраторы источников данных; 
\item редакторы информационных объектов; 
\item простые пользователи.
\end{enumerate}




    Администраторы контента библиотеки отвечают за создание 
информационных ресурсов, атрибутов и~их наборов и~получают доступ ко 
всей функциональности системы. Редактор предметной\linebreak об\-ласти имеет право 
на редактирование тезауруса и~определение основных коллекций системы.\linebreak 
Адми\-нистраторы источников данных отвечают за их подключение 
и~настройку отображения. Редак\-тор информационных объектов может 
создавать, редактировать и~удалять любой информационный объект. 
В~отличие от всех остальных ролей, для редактора информационных 
объектов роль определяет доступ к~конкретным функциям подсистемы 
рабо\-ты с~объектами (редактирование, создание, и~удаление), а~не просто 
доступ к~функциональ\-ности для сопровождения отдельных объектов. 
Зарегистрированные простые пользователи могут описывать свою об\-ласть 
интересов на основе терминов тезауруса предметной об\-ласти, очерчивая тем 
самым интересующий их круг информационных объектов. Эти пользователи 
также могут добавлять в~тезаурус уточняющие об\-ласть их интересов 
термины, которые не отображаются для остальных пользователей, добавлять 
собственные объекты в~свою коллекцию в~рамках своей об\-ласти интересов, 
сохранять свои результаты поиска по источникам данных. Задание об\-ласти 
интересов пользователя позволяет группировать пользователей со сходными 
интересами, строить связи между ними, анализируя круг 
используемых терминов, добавленных объектов, и~давать 
рекомендации на основе этих связей. Таким образом, можно отслеживать 
и~выделять взаимосвязи между разными об\-лас\-тя\-ми интересов. Понятно, что 
один и~тот же пользователь может обладать несколькими ролями 
и~выступать, например, в~качестве редактора предметной об\-ласти 
и~администратора источников данных. Но при этом любой пользователь 
системы независимо от роли получает доступ к~функциям поиска 
и~навигации по объектам системы.
    
\begin{figure*} %fig6
 \vspace*{1pt}
\begin{center}
\mbox{%
\epsfxsize=163.09mm
\epsfbox{ser-6.eps}
}
\end{center}
\vspace*{-9pt}
\Caption{Пользователи}
\vspace*{6pt}
\end{figure*}

    На рис.~6 отображены роли пользователей и~в~виде четырехугольников 
очерчивается их область влияния для рассмотренного выше примера 
сконструированного контента.
    
\section{Заключение}

    Разрабатывая модель информационной системы LibMeta, авторы хотели 
получить гибкую сис\-тему интеграции различных типов ресурсов 
с~возможностью интеграции с~внешними системами. Одним из 
определяющих условий было отсутствие тре\-бования специальной 
подготовки у~простых пользователей системы. Основные идеи при 
разработке системы были позаимствованы из концепции адап\-тив\-ных 
моделей данных, разработанной в~\mbox{1990-х}~гг.~[10, 11]. Эта модель данных 
подходит для определенного круга задач, для решения которых нужно 
разрабатывать довольно сложные частные модели. К~таким задачам 
относится ставшая классической задача интеграции данных из источников 
с~разными моделями. Внесение изменений в~интеграционную модель чаще 
всего выполняется на программном уровне, требуя значительных 
усилий от разработчиков для внедрения изменений. Применение адаптивной 
модели позволяет понизить сложность как самой модели данных, так 
и~разрабатываемых на их основе систем, в~которых попутно решается задача 
создания динамических (адаптивных) пользовательских интерфейсов. 
    
    Применение этой модели данных делает воз\-мож\-ной динамическую 
трансформацию и~ин\-терпретацию модели данных в~приложении, ре\-ша\-ющем 
задачу интеграции данных, позволяя\linebreak настраивать используемые решения под 
определенную предметную область. В~таких задачах часто меняются 
требования к~модели данных, меняется детализация схемы, детализируя или, 
наоборот, обобщая ее описание. Реализация приложений для конечных 
пользователей под эти требования занимает больше времени, чем хотелось 
бы. Поиск решений для такого рода задач на более высоком уровне 
абстракции привел к~появлению концепции адап\-тив\-ной модели данных, 
в~которой декларируется лишь стиль моделирования данных для 
приложений. Получаемые модели более абстрактны, состоят из меньшего 
числа понятий с~более простыми связями и~не привязаны к~определенным 
предметным областям. Поведенческие моменты определяются операциями 
создать, сохранить и~т.\,д., ролями пользователя, которые отражают 
условия доступа к~этим операциям. Пользовательский интерфейс 
представляется адаптивным в~соответствии с~мо\-делью данных и~использует 
для своего вос\-про\-из\-ве\-де\-ния/по\-стро\-ения заранее определенные 
паттерны уровня представления. В~результате применения этой модели, 
когда меняется ее структура, система немедленно подстраивается под эти 
изменения.
    
    В статье был приведен пример конструирования контента семантической 
библиотеки для авторов и~их публикаций в~рамках библиотеки LibMeta на 
основе данных из системы <<Научное наследие России>>. На имеющемся 
наборе данных, пред\-став\-лен\-ном примерно~7000~публикациями и~их 
авторами (около~1300~персон), было проведено тестовое исследование. 
Из~1300~авторов оказалась пред\-став\-ле\-на в~DBPedia примерно треть, и~около 
половины из них были представлены в~VIAF\footnote{{\sf http://viaf.org.}}
(Virtual International Authority File). Часть ссылок на данные 
VIAF была получена из набора данных DBPedia, 
другая часть была непосредственно извлечена из самого VIAF, 
подключенного как источник данных. 
    
    Во втором примере, сконструированном для тезауруса ОДУ 
из~10\,000~публикаций базы ЕНИП, было загружено~100~подходящих по 
тематике. Загруженные публикации были размечены ключевыми словами 
тезауруса, всего было получено~789~элементов разметки.
    
    Провести связывание с~данными из источников, охваченных 
онтологиями AKT, оказалось невозможным из-за специфики данных, но эти 
источники были подключены к~LibMeta в~качестве тестовых для 
использования их в~качестве источников для поиска. В~статью не вошли 
задачи поиска по семантическим тегам или наборам ключевых слов не 
только по строго заданным правилам отоб\-ра\-же\-ния ресурсов, а также поиск 
по формулам в~математических данных. Предполагается 
осветить это направление работ в~рамках предложенной системы 
в~дальнейших работах.
    
{\small\frenchspacing
 {%\baselineskip=10.8pt
 \addcontentsline{toc}{section}{References}
 \begin{thebibliography}{99}
\bibitem{1-ser}
\Au{Gruber T.\,R.} A translation approach to portable ontologies~// 
Knowl. Acquis., 
1993. Vol.~5. No.\,2. P.~199--220. 
\bibitem{2-ser}
Semantic digital libraries~/ Eds. S.\,R.~Kruk, B.~McDaniel.~--- Berlin--Heidelberg:  
Springer-Verlag, 2009. 245~p.
\bibitem{3-ser}
\Au{Антопольский~А.\,Б., Каленкова~А.\,А., Каленов~Н.\,Е., Серебряков~В.\,А., 
Сотников~А.\,Н.} Принципы разработки интегрированной системы для научных 
библиотек, архивов и~музеев~// Информационные ресурсы России, 2012. №\,1. С.~2--6.
\bibitem{4-ser}
\Au{Bizer C., Heath~T., Berners-Lee~T.} Linked data~--- the story so far~// 
Int. J.~Semant. 
Web Inf. Syst., 2009. Vol.~5. No.\,3. P.~1--22.
\bibitem{5-ser}
\Au{Серебряков~В.\,А., Атаева О.\,М.} Основные понятия формальной 
модели семантических библиотек и~формализация процессов интеграции в~ней~// 
Программные продукты и~системы, 2015. №\,4. С.~180--187.
\bibitem{6-ser}
\Au{Серебряков В.\,А., Атаева~О.\,М.} Персональная циф\-ровая библиотека LibMeta как 
среда интеграции связан\-ных открытых данных~// Электронные биб\-лио\-те\-ки: 
перспективные методы и~технологии, электронные коллекции: Тр. XVI Всеросс. научной 
конф. RCDL'2014.~--- Дубна: ОИЯИ, 2014. С.~66--71.
\bibitem{7-ser}
\Au{Каленов Н.\,Е., Савин~Г.\,И., Сотников~А.\,Н.} Электронная библиотека <<Научное 
наследие России>>~// Информационные ресурсы России, 2009. №\,2(108). С.~19--20.
\bibitem{8-ser}
\Au{Моисеев Е.\,И., Муромский~А.\,А., Тучкова~Н.\,П.} Тезаурус  
ин\-фор\-ма\-ци\-он\-но-по\-ис\-ко\-вый по предметной области <<обыкновенные 
дифференциальные уравнения>>.~--- М.: МАКС Пресс, 2005. 116~с.
\bibitem{9-ser}
\Au{Бездушный А.\,Н., Бездушный~А.\,А., Серебряков~В.\,А., Филиппов~В.\,И.} Интеграция 
метаданных Единого Научного Информационного Пространства РАН.~--- М.: ВЦ РАН, 
2006.  238~с.
\bibitem{10-ser}
\Au{Yoder J.\,W., Balaguer~F., Johnson~R.} Architecture and design of adaptive  
object-model~// Adaptive Object Model, 2000. 11~p. {\sf 
http://www.adaptiveobjectmodel.com/ OOPSLA2001/AOMIntriguingTechPaper.pdf}.
\bibitem{11-ser}
\Au{Welick L., Yode~J.\,W., Wirfs-Broc~R.} Adaptive object-model builder~// Adaptive Object 
Model, 2009. 8~p. {\sf http://joeyoder.com/PDFs/04welicki.pdf}.
 \end{thebibliography}

 }
 }

\end{multicols}

\vspace*{-3pt}

\hfill{\small\textit{Поступила в~редакцию 01.12.16}}

\vspace*{8pt}

%\newpage

%\vspace*{-24pt}

\hrule

\vspace*{2pt}

\hrule

%\vspace*{8pt}


\def\tit{PERSONAL SEMANTIC OPEN DIGITAL LIBRARY 
LibMeta. CONSTRUCTION~OF~THE~CONTENT. 
INTEGRATION~WITH~LOD~SOURCES}

\def\titkol{Personal semantic open digital library 
LibMeta. Construction 
of~the~content. Integration with~LOD sources}

\def\aut{O.\,M.~Ataeva and V.\,A.~Serebryakov}

\def\autkol{O.\,M.~Ataeva and V.\,A.~Serebryakov}

\titel{\tit}{\aut}{\autkol}{\titkol}

\vspace*{-9pt}


\noindent
A.\,A.~Dorodnicyn Computing Center, Federal Research Center ``Computer 
Science and Control'' of the Russian Academy of Sciences, 44-2~Vavilov Str., 
Moscow 119333, Russian Federation



\def\leftfootline{\small{\textbf{\thepage}
\hfill INFORMATIKA I EE PRIMENENIYA~--- INFORMATICS AND
APPLICATIONS\ \ \ 2017\ \ \ volume~11\ \ \ issue\ 2}
}%
 \def\rightfootline{\small{INFORMATIKA I EE PRIMENENIYA~---
INFORMATICS AND APPLICATIONS\ \ \ 2017\ \ \ volume~11\ \ \ issue\ 2
\hfill \textbf{\thepage}}}

\vspace*{3pt}



\Abste{Semantic technologies development has brought digital libraries to the 
level where a~meaningful representation of the content of digital libraries came to 
the forefront. At the same time, it is necessary to limit it in terms of a certain 
subject area. The paper describes the libraries content construction with 
a~thesaurus supporting the domain terminology within the developed system 
LibMeta. The personal semantic open digital library LibMeta provides the 
functionality of the construction of the library content in accordance with the 
specific requirements. LibMeta supports users working with resources of digital 
libraries and their collections in a~certain subject area. One needs just to make the 
initial setup of the system for a specific subject area. For the description of a 
subject area, the system uses its limited terminology collected in a thesaurus. The 
domain used as an example is a highly specialized thesaurus of ordinary 
differential equations.}

\KWE{semantic library; data model; ontology; data sources; search in LOD}

\DOI{10.14357/19922264170210} 

%\vspace*{-18pt}

\Ack
\noindent
The work was supported by the Russian Foundation for
Basic Research (project 14-07-00058~A).



%\vspace*{3pt}

  \begin{multicols}{2}

\renewcommand{\bibname}{\protect\rmfamily References}
%\renewcommand{\bibname}{\large\protect\rm References}

{\small\frenchspacing
 {%\baselineskip=10.8pt
 \addcontentsline{toc}{section}{References}
 \begin{thebibliography}{99}
\bibitem{1-ser-1}
\Aue{Gruber, T.\,R.} 1993. A~translation approach to portable ontologies. 
\textit{Knowl. Acquis.} 5(2):199--220. 
\bibitem{2-ser-1}
Kruk, S.\,R., and B.~McDaniel, eds. 2009.
\textit{Semantic digital libraries}. Berlin--Heidelberg: Springer-Verlag. 245~p.
\bibitem{3-ser-1}
\Aue{Antopolsky, A.\,B., A.\,A.~Kalenkova, N.\,E.~Kalenov, 
V.\,A.~Serebryakov, and A.~Sotnikov}. 2012. Printsipy razrabotki 
integrirovannoy sistemy dlya nauchnykh bibliotek, arkhivov i~muzeev [Principles 
for the development of an integrated system for academic libraries, archives and 
museums]. \textit{Informatsionnye resursy Rossii} [Information Resources of 
Russia] 1:2--6. 
\bibitem{4-ser-1}
\Aue{Bizer, C., T.~Heath, and T.~Berners-Lee.} 2009. Linked data~--- the story 
so far. \textit{Int. J.~Semant. Web Inf. Syst.} 5(3):1--2.
\bibitem{5-ser-1}
\Aue{Serebryakov, V.\,A., and  O.\,M.~Ataeva.} 2015. Osnovnye ponyatiya dlya 
postroeniya formal'noy modeli se\-man\-ti\-che\-skikh bibliotek i~opisaniya protsessov 
integratsii v~ney [The basic concepts for building a~formal model of semantic 
libraries and description of the integration processes in it]. \textit{Programmnye 
produkty i~sistemy} [Software and Systems] 4:180--187.
\bibitem{6-ser-1}
\Aue{Serebryakov, V.\,A., and O.\,M.~Ataeva.} 2014. Personal'naya tsifrovaya 
biblioteka LibMeta kak sreda integratsii svyazannykh otkrytykh dannykh 
[Personal digital library LibMeta as an integration environment of linked data]. 
\textit{RCDL Proceedings}.  66--71.
\bibitem{7-ser-1}
\Aue{Kalyonov, N.\,E., G.\,I.~Savin, and A.\,N.~Sotnikov.} 2009. 
{Elektronnaya biblioteka ``Nauchnoe nasledie Rossii''} [Electronic library 
``The scientific heritage of Russia''].
\textit{Informacionnye resursy Rossii} [Information Resources of Russia] 
2:19--20.
\bibitem{8-ser-1}
\Aue{Moiseev, E.\,I., A.\,A.~Muromskij, and N.\,P.~Tuchkova.} 2005. 
\textit{Tezaurus informatsionno-poiskovyy po predmetnoy oblasti 
``obyknovennye differentsial'nye uravneniya''} [Information search thesaurus of 
subject area ``ordinary differential equations'']. Moscow: MAKS Press. 116~p.
\bibitem{9-ser-1}
\Aue{Bezdushnyj, A.\,N., A.\,A.~Bezdushnyj, V.\,A.~Serebryakov, and 
V.\,I.~Filippov.} 2006. \textit{Integratsiya metadannykh Edinogo Nauchnogo 
Informatsionnogo Prostranstva RAN} [The integration of metadata for common 
scientific information space of RAS].~--- Moscow: Computing Centre of the 
Russian Academy of Sciences. 238~p.
\bibitem{10-ser-1}
\Aue{Yoder, J.\,W., F.~Balaguer, and R.~Johnson.} 2000. Architecture and 
design of adaptive object-model. \textit{Adaptive Object Model.} 
Available at: {\sf http://www.adaptiveobjectmodel. com/OOPSLA2001/AOMIntriguingTechPaper.pdf} (accessed 
April~17, 2017).
\bibitem{11-ser-1}
\Aue{Welick, L., J.\,W.~Yode, and R.~Wirfs-Broc}. 2009. Adaptive  
object-model builder. \textit{Adaptive Object Model}. Available at: {\sf 
http://joeyoder.com/PDFs/04welicki.pdf} (accessed April~17, 2017). 
\end{thebibliography}

 }
 }

\end{multicols}

\vspace*{-3pt}

\hfill{\small\textit{Received December 1, 2016}}

\Contr

\noindent
\textbf{Ataeva Olga M.} (b.\ 1978)~--- junior scientist, A.\,A.~Dorodnicyn 
Computing Center, Federal Research Center ``Computer Science and Control'' 
of the Russian Academy of Sciences, 44-2~Vavilov Str., Moscow 119333, 
Russian Federation; \mbox{oli@ultimeta.ru}

\vspace*{3pt}

\noindent
\textbf{Serebryakov Vladimir A.} (b.\ 1946)~--- Doctor of Science in physics 
and mathematics, professor, Head of Department, A.\,A.~Dorodnicyn Computing 
Center, Federal Research Center ``Computer Science and Control'' 
of the Russian Academy of Sciences, 44-2~Vavilov Str., Moscow 119333, 
Russian Federation; \mbox{serebr@ultimeta.ru}
\label{end\stat}


\renewcommand{\bibname}{\protect\rm Литература} 
\def\stat{sinits}

\def\tit{АНАЛИТИЧЕСКОЕ МОДЕЛИРОВАНИЕ  РАСПРЕДЕЛЕНИЙ С~ИНВАРИАНТНОЙ
МЕРОЙ В~СТОХАСТИЧЕСКИХ СИСТЕМАХ С~РАЗРЫВНЫМИ ХАРАКТЕРИСТИКАМИ$^*$}

\def\titkol{Аналитическое моделирование  распределений с~инвариантной
мерой в~стохастических системах} % с~разрывными характеристиками}

\def\autkol{И.\,Н.~Синицын}

\def\aut{И.\,Н.~Синицын$^1$}

\titel{\tit}{\aut}{\autkol}{\titkol}

{\renewcommand{\thefootnote}{\fnsymbol{footnote}}\footnotetext[1]
{Работа выполнена при финансовой поддержке РФФИ
(проект №\,13-07-00036) и программой <<Интеллектуальные информационные 
технологии, системный анализ и автоматизация>> (проект~1.7).}}

\renewcommand{\thefootnote}{\arabic{footnote}}
\footnotetext[1]{Институт проблем информатики Российской академии наук, sinitsin@dol.ru}



\Abst{На базе методов нормальной аппроксимации и статистической линеаризации разработаны 
точные и приближенные алгоритмы аналитического моделирования плотностей стохастических 
режимов с инвариантной мерой в гауссовых и негауссовых стохастических системах (СтС)
с разрывными 
характеристиками. Рассмотрены особенности моделирования в СтС с 
пуассоновскими шумами. На тестовых примерах показана достаточная для многих приложений 
точность алгоритмов.}

\KW{автокоррелированная помеха; аналитическое моделирование;
интегродифференциальные уравнения Пугачёва; метод нормальной аппроксимации;
метод статистической линеаризации; нелинейная гауссовская и негауссовская стохастическая система в смысле Ито;
пуассоновская стохастическая сис\-те\-ма; распределение с инвариантной мерой;
стохастический режим}

\vskip 14pt plus 9pt minus 6pt

      \thispagestyle{headings}

      \begin{multicols}{2}

            \label{st\stat}



\section{Введение}

Следуя [1, 2], будем рассматривать нестационарный стохастических режим $Z\hm=Z(t)$ 
в нелинейной дифференциальной СтС, понимаемой в смысле Ито:
    \begin{equation}
    \dot Z = a(Z,t) + b (Z,t) V\,, \enskip Z(t_0) = Z_0\,.
    \label{e1.1-sin}
    \end{equation}
Здесь $Z$~--- $k$-мер\-ный вектор состояния СтС, $Z\hm\in \Delta$ ($\Delta$~--- 
многообразие состояний); $a\hm=a(Z,t)$ и $b\hm= b(Z,t)$~--- детерминированные  
$(k\times 1)$- и $(k\times m)$-мер\-ные  функции  отмеченных аргументов; 
$V\hm=V(t)$~--- $m$-мер\-ный вектор негауссовских (в общем случае) белых шумов 
с нулевыми математическими ожиданиями и представляющий собой среднеквадратичную 
(с.к.)\ производную процесса с независимыми приращениями  $W\hm=W(t)$, 
$V\hm=\dot W$. Обозначим через $\chi\hm=\chi(\mu;t)$ логарифмическую производную 
одномерной характеристической функции $h_1\hm=h_1(\mu;t)$ процесса $W\hm=W(t)$, определяемую формулой
    \begin{equation}
    \chi(\mu;t)=\fr{\prt \ln h_1 (\mu;t)}{\prt t}=
    \fr{1}{h_1(\mu;t)}\,\fr{\prt h_1(\mu;t)}{\prt t}\,.
    \label{e1.2-sin}
    \end{equation}

Начальное состояние $Z_0$ будем считать случайной величиной (СВ), не зависящей 
от $W(t)$ для $t\hm>t_0$. Предположим, что стохастический режим $Z(t)$ является 
сильным решением~(\ref{e1.1-sin}), а функции $a,b$ и $\chi$ удовлетворяют известным 
условиям существования и единственности~[1, 2].

Пусть существуют одно- и $n$-мерные плот\-ности\linebreak $f_1\hm=f_1(z;t)$ и 
$f_n\hm= f_n(z_1\tr z_n; t_1 \tr t_n)$ и характеристические функции $g_1\hm=g_1(\la;t)$ и 
$g_n\hm=g_n(\la_1\tr \la_n; t_1\tr t_n)$ $(n\hm\ge 2)$, удовлетворяющие 
интегродифференциальным уравнениям\linebreak Пугачева~[1, 2]:
    \begin{multline}
    \fr{\prt f_1(z;t)}{\prt t}+\fr{\prt^{\mathrm{T}}}{\prt z}\lk a(z,t)f_1(z;t)\rk = 
\fr{1}{(2 \pi)^k} \times{}\\
{}\times \iin\iin \chi(b(\xi,t)^{\mathrm{T}}\la;t) e^{i\la^{\mathrm{T}}(\xi-z)} f_1(z;t) \,
    d\xi d\la\,;
    \label{e1.3-sin}
    \end{multline}
   \begin{equation}
    f_1(z;t_0)=f_0(z)\,;\label{e1.4-sin}
    \end{equation}
    
    \vspace*{-12pt}
    
    \begin{multline*}
\fr{\prt f_n(z_1\tr z_n;t_1\tr t_n)}{\prt t_n}+{}\\
{}+\fr{\prt^{\mathrm{T}}}{\prt z_n}\left[
a(z_n, t_n) f_n (z_1\tr z_n; t_1\tr t_n)\right]={}\\
{}= \fr{1}{(2\pi)^{kn}} \iin\iin \chi(b(\xi_n, t_n)^{\mathrm{T}} \la_n;t_n) \times{}\\
{}\times \exp\lf i \sss_{l=1}^n \la_l^{\mathrm{T}} (\xi_l-z_l)\rf \times{}\\
{}\times
f_n (\xi_1\tr \xi_n; t_1\tr t_n)\,d\xi_1\cdots d\xi_n d\la_1\cdots d\la_n\,;
%\label{e1.5-sin}
\end{multline*}

\vspace*{-12pt}

\begin{multline*}
f_n(z_1\tr z_{n-1},z_n;t_1\tr t_{n-1},t_{n})={}\\
{}= f_{n-1} (z_1\tr z_{n-1};t_1\tr t_{n-1})\delta (z_n - z_{n-1})\,;
%\label{e1.6-sin}
\end{multline*}
       
        
\noindent        
\begin{multline}
\fr{\prt g_1 (\la;t)}{\prt t} -{}\\
{}-\fr{1}{(2\pi)^k} \iin \iin i\la^{\mathrm{T}} a (z,t) 
e^{i(\la^{\mathrm{T}} -\mu^{\mathrm{T}})z} g_1 (\mu;t)\, d\mu dz={}\\
{}=\fr{1}{(2\pi)^k} \iin \iin \chi(b(z,t)^{\mathrm{T}} \la^{\mathrm{T}};t) 
e^{i(\la^{\mathrm{T}} -\mu^{\mathrm{T}})z} \times{}\\
{}\times
g_1 (\mu;t)\, d\mu dz\,;
\label{e1.7-sin}
\end{multline}
\begin{equation}
g_1(\la;t_0) = g_0(\la)\,\,; \label{e1.8-sin}
\end{equation}

\vspace*{-12pt}

\begin{multline*}
\fr{\prt g_n (\la_1\tr \la_n; t_1\tr t_n)}{\prt t_n} -{}\\
{}-
\fr{1}{(2\pi)^{kn}} \iin \cdots \iin i\la^{\mathrm{T}} a (z_n,t_n) \times{}\\
{}\times \exp \lk i \sss\limits_{k=1}^n (\la_k^{\mathrm{T}} - \mu_k^{\mathrm{T}}) z_k\rk \times{}\\
{}\times g_n 
(\mu_1\tr \mu_n; t_1\tr t_n)\, d\mu_1 \cdots d \mu_n dz_1\cdots dz_n={}\\
{}= \fr{1}{(2\pi)^{kn}} \iin\cdots \iin \chi (b(z_n;t)^{\mathrm{T}} \la_n;t_n)\times{}\\
{}\times\exp \lk i \sss_{k=1}^n (\la_k^{\mathrm{T}} - \mu_k^{\mathrm{T}}) z_k\rk \times{}\\
{}\times g_n 
(\mu_1\tr \mu_n; t_1\tr t_n) \,d\mu_1 \cdots d \mu_n dz_1\cdots dz_n;
%\label{e1.9-sin}
\end{multline*}

\vspace*{-12pt}

\noindent
\begin{multline*}
g_n (\la_1\tr \la_n; t_1\tr t_{n-1},t_{n-1})= {}\\
{}=
g_{n-1} (\la_1\tr \la_{n-2},\la_{n-1}+\la_n; t_1\tr t_{n-1})\,, %\label{e1.10-sin}
\end{multline*}
 $$
        t_1\le t_2 \le \cdots \le t_n,\enskip n=2,3,\ldots
        $$

При этом одно- и $n$-мер\-ные плотности и характеристические функции связаны 
между собой известными соотношениями:
\begin{equation*}
f_1(z;t) = \fr{1}{(2\pi)^{k}} \iin e^{-i\mu^{\mathrm{T}} z} g_1(\mu;t) d\mu\,; %\label{e1.11-sin}
    \end{equation*}
      \begin{equation*}
   g_1(\la;t) = \iin e^{i\la^{\mathrm{T}} z} f_1(z;t)\, dz\,; %\label{e1.12-sin}
   \end{equation*}
   
   \vspace*{-12pt}

\noindent
\begin{multline}
f_n( z_1\tr z_n; t_1\tr t_n) ={}\\
{}=
\fr{1}{(2\pi)^{kn}} 
\iin\cdots \iin \exp \lf - i \sss_{l=1}^n \la_l^{\mathrm{T}} z_l\rf \times{}\\
{}\times g_n (\la_1\tr \la_n; t_1\tr t_n)\, d\la_1\cdots d\la_n\,;\label{e1.13-sin}
\end{multline}


\vspace*{-12pt}

\noindent
\begin{multline*}
g_n (\la_1\tr \la_n; t_1\tr t_n) ={}\\
{}=\iin\cdots \iin \exp\lf i \sss_{l=1}^n \la_l^{\mathrm{T}} z_l\rf \times{}\\
{}\times f_n (z_1\tr z_n; t_1\tr t_n)\, dz_1\cdots dz_n\,. %\label{e1.14-sin}
\end{multline*}

Для нахождения одномерных плотностей $f_1(z,t) \hm= f_1^* (z)$ и характеристических функций 
$g_1(\la;t) \hm= g_1^* (\la)$ стохастических режимов в стационарных СтС~(\ref{e1.1-sin}) при
    \begin{equation}
    a(z,t) = a^*(z)\,;\ b(z,t)=b^*(z)\,;\ \chi(\mu;t)= \chi^*(\mu)
    \label{e1.15-sin}
    \end{equation}
следует в~(\ref{e1.3-sin}) и~(\ref{e1.7-sin}) положить 
$\prt f_1/\prt t \hm= 0$ и $\prt g_1/ \prt t \hm=0$. В~результате получим соответственно
\begin{multline*}
\fr{\prt^{\mathrm{T}}}{\prt z}\lk a^* (z) f_1^* (z)\rk = {}\\
{}=
\fr{1}{(2\pi)^k} \iin \iin \chi^* (b^*(\xi)^{\mathrm{T}} \la) e^{i\la^{\mathrm{T}}(\xi-z)} f_1^* (\xi)\, d\xi d\la\,;
%\label{e1.16-sin}
\end{multline*}

\vspace*{-12pt}

\noindent
\begin{multline*}
-\fr{1}{(2\pi)^k} \iin  \iin i\la^{\mathrm{T}} a^*(z) e^{i(\la^{\mathrm{T}}-\mu^{\mathrm{T}})z} g_1^*(\mu)\, d\mu dz={}\\
{}=\fr{1}{(2\pi)^k} \iin  \iin \chi^*(b^*(z)^{\mathrm{T}}\la) e^{i(\la^{\mathrm{T}}-\mu^{\mathrm{T}})z} g_1^*(\mu)\, d\mu dz.
%\label{e1.17-sin}
\end{multline*}
Поставим задачу разработки точных и приближенных  алгоритмов
аналитического моделирования распределений (плотностей и
характеристических функций) стохастических режимов  $Z\hm=Z(t)$ в
нелинейных гауссовских и негауссовских СтС~(\ref{e1.1-sin})  с разрывными
характеристиками $a\hm=a(z,t)$ и $b\hm=b(z,t)$, обладающих свойством
сохранения инвариантной меры, т.\,е.\ удовлетворяющих уравнениям~(\ref{e1.3-sin})
и~(\ref{e1.7-sin}) при $\chi\hm=0$.

Условия сохранения инвариантной меры можно представить в следующем развернутом виде:
\begin{equation}
\left.
\begin{array}{c}
\displaystyle\fr{\prt f_1 (z;t)}{\prt t} + A_a f_1 (z;t) =0\,;\\[9pt] 
\hspace*{-4.5mm}\displaystyle A_a f_1(z;t) = 
    \fr{\prt^{\mathrm{T}}}{\prt z} \lk a(z,t) f_1(z;t)\rk = \mathrm{div}\, \pi(z;t)\,;
    \end{array}
    \right\}
    \label{e1.18-sin}
    \end{equation}
\begin{equation}
\left.
\begin{array}{c}
A_a^* f_1^*(z) =0\,;\\[9pt]
\displaystyle A_a^* f_1^* (z) = \fr{\prt^{\mathrm{T}}}{ \prt z} \lk a^* 
(z) f_1^* (z)\rk =\mathrm{div}\, \pi^* (z)\,;
\end{array}
\right\}
\label{e1.19-sin}
\end{equation}
$$
\fr{\prt g_1 (\la;t)}{\prt t} - B_a g_1(\la;t) =0\,;
$$

\vspace*{-14pt}

\noindent
\begin{multline}
B_a g_1(\la;t) ={}\\[2pt]
{}=\fr{1}{(2\pi)^k} \iin\iin i\la^{\mathrm{T}} a(z,t) e^{i(\la^{\mathrm{T}}-\mu^{\mathrm{T}})z}
 g_1(\mu;t)\, d\mu dz={}\\[2pt]
{}= \iin i\la^{\mathrm{T}} a(z,t) e^{i\la^{\mathrm{T}}z} f_1(z;t)\, dz={}\\[2pt]
{}= \iin e^{i\la^{\mathrm{T}} z} i\la^{\mathrm{T}} \pi(z;t)\, dz\,;
\label{e1.20-sin}
\end{multline}

\vspace*{-9pt}

\noindent
\begin{equation}
\left.
\begin{array}{c}
\hspace*{-45mm}B_a^* g_1^* (\la)=0\,;\\[12pt]
\hspace*{-48mm}B_a^* g_1^* (\la) = {}\\[10pt]
\hspace*{-3mm}{}=\fr{1}{(2\pi)^k} \iin\! i\la^{\mathrm{T}} a^* (z) e^{i(\la^{\mathrm{T}} -\mu^{\mathrm{T}})z} g_1^* (\mu)\, d\mu dz={}\\[10pt]
{}=\displaystyle\iin\! i\la^{\mathrm{T}} a^*(z) e^{i\la^{\mathrm{T}}z} f_1^* (z)\, dz = {}\\[10pt]
\displaystyle{}=
\iin e^{i\la^{\mathrm{T}} z} i\la^{\mathrm{T}} \pi^* (z)\, dz\,.
\end{array}
\right\}
\label{e1.21-sin}
\end{equation}
Для гауссовских (нормальных) СтС с гладкими характери\-стиками точные и приближенные 
методы  и алгоритмы аналитического моделирования рассмотрены в~[1--15]. 

Особое внимание 
уделим приближенным методам, основанным на методах нор\-маль\-ной аппроксимации и статистической 
линеаризации. Подробно рассмотрим их применение к пуассоновским СтС.



\section{Точные методы и~алгоритмы аналитического моделирования распределений 
с~инвариантной мерой}

Пусть функция~$a$ в СтС~(\ref{e1.1-sin}) допускает пред\-став\-ле\-ние
\begin{equation}
a= a(z,t) = a_1(z,t) +a_2 (z,t) \label{e2.1-sin}
\end{equation}
такое, что функция  $f_1\hm=f_1(z;t)$ является плот\-ностью инвариантной меры 
невозмущенной шумами системы, описываемой векторным обыкновенным дифференциальным 
уравнением вида
   \begin{equation}
   \dot z = a_1 (z,t)\,,\label{e2.2-sin}
   \end{equation}
т.\,е.\ удовлетворяет условию~(\ref{e1.18-sin}):
\begin{equation}
\fr{\prt f_1 (z;t)}{\prt t}+ \fr{\prt^{\mathrm{T}}}{\prt z} \lk a_1 (z,t) f_1(z;t)\rk =0\,.
\label{e2.3-sin}
\end{equation}

Для гладких функций $a_1\hm=a_1(z,t)$ вопросы существования и основные свойства 
интегральных 
 инвариантов изучены в~\cite{16-sin, 17-sin}. При этом в~(\ref{e2.1-sin}) 
функция $a_2 \hm= a_2(z,t)$ определяется путем решения следующего интегродифференциального 
уравнения:
\begin{multline}
\fr{\prt^{\mathrm{T}}}{\prt z}\lk a_2 (z,t) f_1(z;t) \rk 
=
\fr{1}{(2\pi)^k}\times{}\\
{}\times \iin\iin \chi(b(\xi,t)^{\mathrm{T}} \la;t) 
e^{i\la^{\mathrm{T}}(\xi-z)} f_1(\xi;t)\, d\xi d\la\,.\label{e2.4-sin}
\end{multline}
В общем случае нахождение функций $a_1$ и~$a_2$ в~(\ref{e2.1-sin})~--- такая же
трудная задача, как решение уравнений~(\ref{e1.3-sin}) и~(\ref{e1.4-sin}).

Для стационарных СтС, когда выполнены условия~(\ref{e1.15-sin}), 
уравнения~(\ref{e2.1-sin})--(\ref{e2.4-sin}) имеют вид:
\begin{align}
a(z)&= a_1(z) + a_2(z)\,;\label{e2.5-sin}
\\
\dot z &= a_1(z)\,,\label{e2.6-sin}
\\
\fr{\prt^{\mathrm{T}}}{\prt z}\lk a_2^*(z) f_1^*(z)\rk &= {}\notag\\
&\hspace*{-28mm}{}=
\fr{1}{(2\pi)^k} \!\!\iin \iin\!\! \chi^* (b^*(\xi)^{\mathrm{T}} \la) 
e^{i\la^{\mathrm{T}}(\xi-z)} f_1^*(\xi)\, d\xi d\la\,.\!\!\!\label{e2.7-sin}
\end{align}
В этом случае можно выбирать невозмущенную сис\-те\-му~(\ref{e2.6-sin}) так, чтобы
она имела первые интегралы.

В терминах характеристических функций соотношения~(\ref{e2.3-sin}), (\ref{e2.4-sin})
и~(\ref{e2.7-sin}) могут быть записаны следующим образом:

\noindent
\begin{equation}
\fr{\prt g_1 (\la;t)}{\prt t} - B_{a_1} g_1(\la;t) =0\,;\label{e2.8-sin}
\end{equation}
\begin{equation*}
B_{a_1}^* g_1^*(\la) =0\,. %\label{e2.9-sin}
\end{equation*}
Для составляющих $a_2(z,t)$ и $a_2^*(z)$ имеют место уравнения
\begin{multline}
B_{a_2} g_1(\la;t) 
= \fr{1}{(2\pi)^k} \times{}\\
\hspace*{-2.5mm}{}\times\iin\iin \!\chi(b(z,t)^{\mathrm{T}} \la;t) 
e^{i(\la^{\mathrm{T}}-\mu^{\mathrm{T}})z} g_1(\mu;t) \,d\mu dz;\label{e2.10-sin}
\end{multline}

\vspace*{-16pt}

\noindent
\begin{multline}
B_{a_2}^* g_1^*(\la) 
= \fr{1}{(2\pi)^k} \times{}\\
{}\times\iin\iin 
\chi^*(b^*(z)^{\mathrm{T}} \la) e^{i(\la^{\mathrm{T}}-\mu^{\mathrm{T}})z} g_1^*(\mu)\, d\mu dz\,.
\label{e2.11-sin}
\end{multline}

Отсюда вытекают конструктивные точные алгоритмы аналитического
моделирования распределений с инвариантной мерой. В~их основе лежат
следующие теоремы.

%\pagebreak

\medskip

\noindent
\textbf{Теорема~2.1.} \textit{Функция $f_1\hm=f_1(z;t)$ будет решением}~(\ref{e1.3-sin})
\textit{и}~(\ref{e1.4-sin}) \textit{тогда и только тогда, когда $a\hm=a(z,t)$ допускает
представление}~(\ref{e2.1-sin}) \textit{такое, что $f_1\hm=f_1(z;t)$ является плотностью
инвариантной меры обыкновенного дифференциального уравнения}~(\ref{e2.2-sin}),
\textit{т.\,е.\ удовле\-тво\-ря\-ет условию}~(\ref{e2.3-sin}). \textit{При этом со\-став\-ля\-ющая $a_2$
определяется из решения интегродифференциального уравнения}~(\ref{e2.4-sin}).

\medskip

\noindent
\textbf{Теорема~2.2.} \textit{Функция $f_1^*\hm=f_1^*(z)$ будет решением}~(\ref{e1.3-sin}) 
\textit{тогда и только тогда, когда $a^*\hm=a^*(z)$ допускает
представление}~(\ref{e2.5-sin}) \textit{такое, что $f_1^*\hm=f_1^*(z)$ является плотностью
инвариантной меры}~(\ref{e2.6-sin}). \textit{При этом составляющая $a_2^{*}$
определяется из решения  уравнения}~(\ref{e2.7-sin}).

\medskip

\noindent
\textbf{Теорема~2.3.} \textit{Функция $g_1\hm=g_1(\la;t)$ будет ре\-ше\-нием}~(\ref{e1.7-sin}), 
(\ref{e1.8-sin}) \textit{тогда и только тогда, когда недиф\-фе\-ренцируемая функция
$a\hm=a(z,t)$  допускает пред\-став\-ление}~(\ref{e2.1-sin}) \textit{такое, что
$g_1\hm=g_1(\la;t)$ является ха\-рак\-теристической функцией инвариантной
меры \mbox{уравнения}}~(\ref{e2.2-sin}), \textit{т.\,е.\ удовлетворяет условию}~(\ref{e2.8-sin}). 
\textit{При этом составляющая $a_2$ определяется из уравнения}~(\ref{e2.10-sin}).

\medskip

\noindent
\textbf{Теорема 2.4.} \textit{Функция $g_1^*\hm=g_1^*(\la)$  будет решением}~(\ref{e1.13-sin}) 
\textit{тогда и только тогда, когда недифференцируемая функция $a^*\hm=a^*(z)$  
допускает представление}~(\ref{e2.5-sin}) \textit{такое, что $g_1^*$ является  
характеристической функцией инвариантной меры}~(\ref{e2.2-sin}). 
\textit{При этом $a_2^*$ определяется из решения}~(\ref{e2.11-sin}).

\smallskip

Теоремы~2.1--2.4 легко обобщаются на случай многомерных распределений с инвариантной мерой.

\section{Приближенные методы и~алгоритмы аналитического моделирования распределений 
с~инвариантной мерой, основанные на~нормальной аппроксимации и статистической линеаризации}

Пусть нелинейная СтС~(\ref{e1.1-sin}) допускает применение метода нормальной аппроксимации 
(МНА)~[1, 2]. Тогда одно- и двумерные нормальные плот\-ности $f_1^{\mathrm{МНА}}$,
 $f_2^{\mathrm{МНА}}$ и характеристические функции  $g_1^{\mathrm{МНА}}$,  
 $g_2^{\mathrm{МНА}}$, а также вектор математического ожидания $m_t = M^{\mathrm{МНА}} Z(t)$, 
 ковариационная мат\-ри\-ца $K_t \hm= M^{\mathrm{МНА}} Z^{0\mathrm{T}} Z^0 (t)$ 
 $(Z^0 (t) \hm= Z(t) \hm- m_t)$ и матрица ковариационных функций 
 $K(t_1, t_2) \hm= M^{\mathrm{МНА}} Z^{0\mathrm{T}} (t_1) Z^0 (t_2)$ $(t_1\hm< t_2)$ определяются 
 следующими уравнениями:
    \begin{multline}
    f_1^{\mathrm{МНА}} = f_1^{\mathrm{МНА}} (z;t, m_t, K_t) =
    \lk (2\pi)^k |K_t|\rk^{-1/2}\times{}\\
    {}\times \exp \lf -  \fr{1}{ 2} 
    \left(z^{\mathrm{T}} - m_t^{\mathrm{T}}\right) K_t^{-1}(z-m_t)\rf\,;\label{e3.1-sin}
    \end{multline}
    
    \vspace*{-12pt}
    
    \noindent
\begin{multline}
f_2^{\mathrm{МНА}} ={}\\
= f_2^{\mathrm{МНА}} (z_1, z_2;t_1, t_2, m_{t_1}, m_{t_2}, K_{t_1}, K_{t_2}, K(t_1, t_2))=\\
{}=\lk (2\pi)^k |\bar K_2|\rk^{-1/2}\times{}\\
\hspace*{-2mm}{}\times \exp \lf - 
([z_1^{\mathrm{T}} z_2^{\mathrm{T}}] - \bar m_2^{\mathrm{T}}) 
\bar K_2^{-1}([z_1^{\mathrm{T}} z_2^{\mathrm{T}}]^{\mathrm{T}}-\bar m_2)\rf;
\!\!\label{e3.2-sin}
\end{multline}
\begin{equation}
g_1^{\mathrm{МНА}} (\la;t)=
\exp\lf i\la^{\mathrm{T}} m- \fr{1}{2}\,\la^{\mathrm{T}} K_t \la\rf\,;\label{e3.3-sin}
\end{equation}

\vspace*{-12pt}

\noindent
\begin{multline}
g_2^{\mathrm{МНА}} (\la_1, \la_2; t_1,t_2) ={}\\
{}= \exp \lf i \bar \la^{\mathrm{T}} \bar m_2 - 
    \fr{1}{2} \,\bar \la^{\mathrm{T}} \bar K_2 \bar \la\rf\,;\label{e3.4-sin}
    \end{multline}
$$
    \bar \la =\lk \la_1^{\mathrm{T}} \la_2^{\mathrm{T}}\rk^{\mathrm{T}}\,;\enskip 
    \bar m_2 =\lk m_{t_1}^{\mathrm{T}} m_{t_2}^{\mathrm{T}}\rk^{\mathrm{T}}\,;
    $$
    $$
    \bar K_2 =\begin{bmatrix}
        K(t_1, t_1)&K(t_1, t_2)\\[3pt]
        K(t_2, t_1)& K(t_2, t_2)
        \end{bmatrix}\,;
        $$
  \begin{multline}
  \dot m_t = a_1 (t, m_t, K_t) ={}\\
  {}=\iin a(z,t) f_1^{\mathrm{МНА}} (z; t, m_t, K_t) \,dz\,;
  \label{e3.5-sin}
  \end{multline}

\vspace*{-12pt}

\noindent
\begin{multline}
\dot K_t = a_2(t, m_t, K_t) = a_{21} + a_{12}+a_{22}={}\\
{}=\left[ \iin a(z,t) (z^{\mathrm{T}}-m_t^{\mathrm{T}}) + (z-m_t) a^{\mathrm{T}} (z,t) +{}\right.\\
\left.{}+ \sigma (z,t)
\vphantom{\iin}\right] f_1^{\mathrm{МНА}} (z;t, m_t, K_t)\, dz\,;
\label{e3.6-sin}
\end{multline}

\vspace*{-12pt}

\noindent
\begin{multline}
\fr{\prt K(t_1, t_2)}{\prt t_2} ={}\\
{}= a_3 (t_1, t_2, m_{t_1},m_{t_2}, K_{t_1}, K_{t_2}, K(t_1,t_2))={}\\
{}=\lk (2\pi)^{2k} |\bar K_2|\rk^{-1/2}\times{}\\
{} \times\iin\iin (z_1-m_{t_1}) a(z_2, t_2)
\exp\left\{ - ([z_1^{\mathrm{T}} z_2^{\mathrm{T}}]-\bar m_2^{\mathrm{T}})\times{}\right.\\
\left.{}\times\bar K_2^{-1} 
([z_1^{\mathrm{T}} z_2^{\mathrm{T}}]-\bar m_2)\right\} dz_1 dz_2\,.
\label{e3.7-sin}
\end{multline}
Здесь введены следующие обозначения:
\begin{equation}
\left.
\begin{array}{c}
z_1=z_{t_1}\,;\enskip  z_2=z_{t_2}\,;\enskip \bar m_2 =\lk m_{t_1}^{\mathrm{T}} m_{t_2}^{\mathrm{T}}\rk^{\mathrm{T}}\,;\\[9pt]
\displaystyle \bar K_2 =\begin{bmatrix}
        K(t_1,t_1)&K(t_1, t_2)\\[3pt]
        K(t_2, t_1)& K(t_2, t_2)
        \end{bmatrix}\,,
        \end{array}
        \right\}
        \label{e3.8-sin}
        \end{equation}
\begin{equation}
\sigma(z,t) = b(z,t) \nu(t) b(z,t)^{\mathrm{T}}\,,\label{e3.9-sin}
\end{equation}
где $\nu=\nu(t)$~--- интенсивность негауссовского белого шума $V\hm=V(t)$.

Для стационарных СтС  при $\dot m^* \hm=0$, $\dot K^* \hm=0$, 
$K(t_1, t_2)\hm= k(\tau)$ $(\tau\hm=t_1-t_2)$  соотношения~(\ref{e3.5-sin})--(\ref{e3.9-sin}) 
принимают вид:
\begin{equation}
a_1^* (m^*, K^*) =0\,;\label{e3.10-sin}
\end{equation}
\begin{equation}
    a_2^*(m^*, K^*) =0\,;\label{e3.11-sin}
    \end{equation}
    \begin{equation}
    \fr{dk(\tau) }{d\tau} = a_{11}^{\mathrm{МНА}} (m^*, K^*) k(\tau)\,;\label{e3.12-sin}
    \end{equation}
$$
k(\tau) = k(-\tau^{\mathrm{T}})\,;\enskip k(0)=K\,.
$$
Из уравнения~(\ref{e3.12-sin}) следует, что алгоритм МНА будет устойчивым, если матрица 
$a_{11}^{\mathrm{МНА}} (m_t, K_t, t)$ будет асимптотически устойчива.

Для $m$ и $K$ уравнения метода статистической линеаризации (МСЛ) в 
нелинейных СтС  при аддитивных шумах, когда $b(z,t) \hm= b_0(t)$, $b^*(z)\hm=b_0^*$ 
получаются из~(\ref{e3.5-sin})--(\ref{e3.7-sin}) и (\ref{e3.10-sin})--(\ref{e3.12-sin}) 
как частный случай.

Условия наличия нормального распределения с инвариантной мерой~(\ref{e1.18-sin}) 
и~(\ref{e1.19-sin}), если заменить $a(z,t)$ статистически
линеаризованным выраже\-нием
\begin{equation*}
    a(Z,t)\approx a_{10}^{\mathrm{МНА}} (t, m_t, K_t) + a_{11}^{\mathrm{МНА}} (t, m_t, K_t) 
    (Z-m_t)\,, %\label{e3.13-sin}
    \end{equation*}
где
\begin{equation*}
a_{10}^{\mathrm{МНА}} =a_{10}^{\mathrm{МНА}} (t, m_t, K_t)\equiv a_1\,; %\label{e3.14-sin}
\end{equation*}
    
    
   
    \noindent
    \begin{multline*}
    a_{11}^{\mathrm{МНА}}=a_{11}^{\mathrm{МНА}} (t, m_t, K_t) = {}\\
    {}=\lk \iin a(z,t) (z^{\mathrm{T}}-m_t^{\mathrm{T}}) 
        f_1^{\mathrm{МНА}} (z; t , m_t, K_t)\, dz\rk\times{}\\
        {}\times K_t^{-1} 
=\left(\fr{\prt}{\prt m_t} a_1^{\mathrm{T}}\right)^{\mathrm{T}}\,, %\label{e3.15-sin}
\end{multline*}
приводят к следующим соотношениям:
        \begin{multline}
\fr{\prt f_1^{\mathrm{МНА}} (z; t, m_t, K_t)}{\prt t} +\fr{\prt^{\mathrm{T}}}{ \prt z} 
\left\{ \left[ a_{10}^{\mathrm{МНА}} (t, m_t, K_t) 
+{}\right.\right.\\
\left.{}+ a_{11}^{\mathrm{МНА}} (t, m_t, K_t) (z-m_t) \vphantom{a_{10}^{\mathrm{МНА}}}
\right]\times{}\\
\left.{}\times 
     f_1^{\mathrm{МНА}} ( z; t , m_t, K_t)\right\} =0\,;
     \label{e3.16-sin}
     \end{multline}
     
     
     \noindent
\begin{multline}
\hspace*{-9.81628pt}\fr{\prt^{\mathrm{T}}}{\prt z} \left\{ \left[ a_{10}^{*{\mathrm{МНА}}}(m^*, K^*) + 
 a_{11}^{*{\mathrm{МНА}}}(m^*, K^*) (z-m^*)\right] \times{}\right.\\
\left.{}\times f_1^{*{\mathrm{МНА}}}(z; m^*, K^*)\right\} =0\,,\label{e3.17-sin}
 \end{multline}
где
\begin{multline*}
f_1^{*{\mathrm{МНА}}} (z; m^*, K^*) = \lk (2\pi)^k |K^*|\rk^{-1/2}\times{}\\
{}\times \exp \lf -
    \fr{1}{2} (z^{\mathrm{T}}-m^{*\mathrm{T}})(K^*)^{-1} (z-m^*)\rf\,.
    \end{multline*}

Аналогично в развернутом виде выписываются условия~(\ref{e1.20-sin}) и~(\ref{e1.21-sin}):
\begin{multline}
\fr{\prt g_1^{\mathrm{МНА}} (\la;t)}{\prt t} -\iin i\la^{\mathrm{T}} \left[ a_{10}^{\mathrm{МНА}} 
    (m_t, K_t, t) +{}\right.\\[2pt]
\left.    {}+ a_{11}^{\mathrm{МНА}} (m_t, K_t, t) (z- m_t) \right]\times{}\\[2pt]
{}\times e^{i\la^{\mathrm{T}} z} f_1^{\mathrm{МНА}} (z; m_t, K_t, t)\, dz=0\,;\label{e3.18-sin}
\end{multline}


\noindent
\begin{multline}
\iin i\la^{\mathrm{T}} \left[ a_{10}^{*{\mathrm{МНА}} } (m^*, K^*) 
+{}\right.\\[2pt]
\left.{}+a_{11}^{*{\mathrm{МНА}} } 
    (m^*, K^*) (z-m^*)\right]\times{}\\[2pt]
    {}\times
     e^{i\la^{\mathrm{T}}z} f_1^{*{\mathrm{МНА}} } (z; m^*, K^*)\, dz =0\,.
    \label{e3.19-sin}
    \end{multline}

Отсюда вытекают следующие теоремы.

\bigskip

\noindent
\textbf{Теорема~3.1.}\ \textit{Если существуют одно- и двумерные  плотности
стохастического режима, а  матрица $a_{11}^{\mathrm{МНА}}$ коэффициентов
статистической (нормальной) линеаризации асимптотически устойчива,
то приближенный алгоритм аналитического моделирования МНА
нестационарных стохастических режимов в СтС}~(\ref{e1.1-sin}) \textit{с инвариантной
мерой определяется выражениями}~(\ref{e3.1-sin})--(\ref{e3.7-sin}) и~(\ref{e3.16-sin}).

\bigskip

\noindent
\textbf{Теорема 3.2.}\ \textit{Если существуют стационарные одно- и
двумерные плотности стохастического режима, а матрица
$a_{11}^{*{\mathrm{МНА}}}$  коэффициентов статистической (нормальной)
линеаризации асимптотически устойчива, то приближенный алгоритм
аналитического моделирования стационарных стохастических режимов с
инвариантной мерой в стационарной СтС}~(\ref{e1.1-sin}) \textit{определяется 
выражениями}~(\ref{e3.10-sin})--(\ref{e3.12-sin}) и~(\ref{e3.17-sin}).

\bigskip

Как известно~[1, 2], одно- и двумерные нормальные распределения
определяют и все  $n$-мер\-ные распределения $(n\hm\ge 3)$, поэтому МНА и
МСЛ дают приближенные алгоритмы для любых многомерных плотностей
стохастических режимов, если они существуют. Аналогично
формулируются теоремы~3.3 и~3.4 на основе условий~(\ref{e3.18-sin}) и~(\ref{e3.19-sin}).


\section{О других приближенных методах и~алгоритмах аналитического моделирования 
распределений с~инвариантной мерой}

\vspace*{-2pt}

 Обобщением МНА являются различные
приближенные методы, основанные на параметризации распределений~[1, 2].
Аппроксимируя одномерную характеристическую функцию $g_1 (\la;t)$
и соответствующую плотность $f_1 (z,t)$ известными функциями
 $g_1^* (\la;\theta)$, $f_1^* (z;\theta)$,  зависящими от
конечномерного векторного параметра~$\theta$, можно свести задачу
приближенного определения одномерного распределения к выводу из
уравнения для характеристических функций обыкновенных
дифференциальных уравнений, определяющих~$\theta$ как функцию
времени. Это относится и к остальным многомерным распределениям.
При аппроксимации многомерных распределений целесообразно выбирать
последовательности функций $\{ f_n^* (z_1,\ldots,z_n;\theta_n)\}$ и 
$\{g_n^* (\la_1\tr \la_n;\theta_n)\}$, каждая пара
которых находилась бы в такой  зависимости от векторного параметра~$\theta_n$, 
чтобы при любом~$n$ множество параметров, образующих
вектор~$\theta_n$, включало в качестве подмножества множество
параметров, образующих вектор~$\theta_{n-1}$. Тогда при
аппроксимации $n$-мер\-но\-го распределения придется определять только
те координаты вектора~$\theta_n$, которые не были определены ранее
при аппроксимации функций $f_1, g_1\tr f_{n-1}$, $g_{n-1}$.

В зависимости от того, что представляют собой параметры, от
которых зависят функции $f_n^* (z_1\tr z_n;\theta_n)$ и 
$g_n^* (\la_1\tr \la_n;\theta_n)$, аппрок-\linebreak симирующие неизвестные
многомерные плотности $f_n (z_1,  \ldots,z_n; t_1 \tr t_n)$ и
характеристические функции $g_n (\la_1\tr \la_n; t_1,\ldots,t_n)$,
используются различные приближенные методы решения
 уравнений при условиях~(9)--(12), определяющих\linebreak многомерные
распределения вектора состояния сис\-те\-мы~$X_t$, в частности методы
моментов (ММ), семиинвариантов (МСИ), ортогональных разложений
(МОР), квазимоментов (МКМ) и~др.~[1, 2].

\vspace*{-6pt}


\section{Обобщение на~случай стохастических систем с~автокоррелированными шумами}

\vspace*{-2pt}

Пусть  СтС описывается нелинейным, в общем случае векторным дифференциальным 
стохастическим уравнением Ито~\cite{1-sin, 2-sin, 15-sin, 18-sin}

\noindent
\begin{equation}
\left.
\begin{array}{c}
    \dot Z = a(Z,t) + b_U(Z,t) U\,;\\[6pt] 
\displaystyle    \sss_{i=0}^l \alpha_i U^{(i)} =
\displaystyle\sss_{j=0}^h \beta_j V^{(j)}\enskip (h<l)\,.
\end{array}
\right\}
    \label{e5.1-sin}
    \end{equation}
    Здесь $U=U(t)$~--- векторная помеха размерности  $m\times 1$; $V\hm=V(t)$~--- 
    негауссовский белый шум с нулевым математическим ожиданием и известной функцией  
    $\chi\hm=\chi(\mu;t)$. В~таком случае в за\-ви\-си\-мости от степени <<гладкости>> 
    стохастического режима $Z\hm=Z(t)$ и помехи $U\hm=U(t)$ уравнения~(\ref{e5.1-sin})  
    путем расширения вектора состояния согласно~[1, 2] приводятся к виду~(\ref{e1.1-sin}) 
    для расширенного вектора состояния~$\bar Z$. Тогда, но уже для расширенного вектора 
    состояния СтС, при решении уравнений~(\ref{e5.1-sin}) могут быть использованы точные 
    (разд.~2) и приближенные (разд.~3) методы и алгоритмы аналитического моделирования 
    нестационарных и стационарных распределений с инвариантной мерой.

\section{Особенности аналитического моделирования распределений с~инвариантной мерой 
в~пуассоновских стохастических системах}

Рассмотрим СтС~(\ref{e1.1-sin}) при $b(z,t) \hm=I_m$ для обобщенного пуассоновского 
белого шума  $V^{\mathrm{OP}}\hm=  V^{\mathrm{OP}}(t)$, когда функция~(\ref{e1.2-sin}) 
определяется формулой
\begin{equation*}
\chi^{\mathrm{OP}} (\mu;t) =\lk g_c^{\mathrm{OP}} (\mu) -
1\rk \nu^{\mathrm{OP}} (t)\,, %\label{e6.1-sin}
\end{equation*}
где $g_c^{\mathrm{OP}} \hm=g_c^{\mathrm{OP}} (\mu)$~--- характеристическая 
функция скачков; $\nu^{\mathrm{OP}} \hm= \nu^{\mathrm{OP}} (t)$~--- 
интенсивность пуассоновского белого шума 
$V^{\mathrm{OP}}\hm=V^{\mathrm{OP}} (t)$. Обозначим через $f_c^{\mathrm{OP}} \hm=
 f_c^{\mathrm{OP}} (z)$ плотность скачков обобщенного пуассоновского процесса. 
 Тогда~(\ref{e1.3-sin}) будет представлять собой известное уравнение Фел\-ле\-ра--Кол\-мо\-го\-ро\-ва
\begin{multline}
\fr{\prt f_1(z;t)}{\prt t} + \fr{\prt^{\mathrm{T}}}{\prt z} 
    \lk a(z,t) f_1(z;t)\rk ={}\\
    \hspace*{-3mm}{}= \nu^{\mathrm{OP}} (t) \lk \iin f_c^{\mathrm{OP}} (z-\xi) f_1 (\xi;t)\, d\xi - f_1(z;t)\rk
    \label{e6.2-sin}
    \end{multline}
с начальным условием~(\ref{e1.4-sin}). В~случае простого пуассоновского белого шума 
с единичными скачками $g_c (\mu) \hm= e^{i\mu}$.

Для  стационарной пуассоновской СтС~(\ref{e1.1-sin}) уравнение~(\ref{e6.2-sin}) имеет следующий вид:
\begin{multline}
\fr{\prt^{\mathrm{T}}}{\prt z} \lk a^* (z) f_1^* (z)\rk = {}\\
{}=
\nu^{\mathrm{OP} *} \lk \iin f_c^{\mathrm{OP}} (z-\xi) f_1^* (\xi)\, d\xi- 
f_1^* (z)\rk\,.\label{e6.3-sin}
\end{multline}

Пользуясь уравнениями~(\ref{e6.2-sin}), (\ref{e6.3-sin})  
и результатами разд.~1 и~2, нетрудно сформулировать следующие утверждения.

\medskip

\noindent
\textbf{Теорема 6.1.}\ \textit{Функция $f_1 \hm= f_1(z;t)$ будет
нестационарным решением}~(\ref{e6.2-sin}), (\ref{e1.4-sin}) \textit{тогда и только тогда, 
когда $a$ допускает представление}~(\ref{e2.1-sin}) \textit{такое, что $f_1$ является плот\-ностью
инвариантной меры обыкновенного дифференциального уравнения}~(\ref{e2.2-sin}),
\textit{т.\,е.\ удовле\-тво\-ря\-ет условию}~(\ref{e2.3-sin}), \textit{а составляющая $a_2$ определяется
из решения следующего уравнения}:
\begin{multline*}
    \fr{\prt^{\mathrm{T}}}{\prt z} \lk a_2 (z,t) f_1 (z;t)\rk =
     \fr{1}{(2\pi)^k}\times{}\\
     {}\times \iin\iin \chi^{\mathrm{OP}} 
    \left(b(\xi,t)^{\mathrm{T}} \la;t\right) e^{i\la^{\mathrm{T}}(\xi-z)} f_1(\xi,t)\,d\xi d\la\,.
%    \label{e6.4-sin}
    \end{multline*}

%\smallskip

\noindent
\textbf{Теорема 6.2.}\ \textit{Функция $f_1^* \hm= f_1^* (z)$ будет стационарным 
решением}~(\ref{e6.3-sin}) \textit{тогда и только тогда, когда $a_2^*$ допускает 
представление}~(\ref{e2.5-sin}) \textit{такое, что  $f_1^*$ является плот\-ностью 
инвариантной меры}~(\ref{e2.6-sin}), \textit{а составляющая $a_2^{*}$ определяется 
из решения следующего уравнения}:
\begin{multline*}
\fr{\prt^{\mathrm{T}} }{\prt z} \lk a_2^{*} (z) f_1^* (z)\rk ={}\\
{}=
    \fr{1}{(2\pi)^k} \iin\iin \chi^{\mathrm{OP} *} (b(\xi)^{\mathrm{T}} \la) 
    e^{i\la^{\mathrm{T}}(\xi-z)} f_1^*(\xi)\,d\xi d\la\,.
%    \label{e6.5-sin}
    \end{multline*}

При использовании МНА и МСЛ для пуассоновских СтС непосредственно применяются теоремы~3.1--3.4, 
причем в формулу~(\ref{e3.9-sin}) для  
$\sigma(z,t)$ входит интенсивность 
$\nu^{\mathrm{OP}} (t)$ обобщенного пуассоновского белого шума.

\section{Тестовые примеры}

\noindent
\textbf{Пример~1}. Рассмотрим осциллятор Дуффинга в обобщенной пуассоновской 
стохастической среде:
\begin{equation}
\ddot X +\w^2 X -\mu X^3 =-\delta^{\mathrm{OP}} \dot X + V^{\mathrm{OP}} (t)\,.\label{e7.1-sin}
\end{equation}
Уравнения МСЛ для~(\ref{e7.1-sin}) имеют следующий вид:
\begin{equation}
\dot m_X = m_{\dot X}\,;\enskip 
\dot m_{\dot X} =- \w_{\mathrm{э}}^2 m_X -\delta^{\mathrm{OP}} m_{\dot X}\,;
\label{e7.2-sin}
\end{equation}
    \begin{equation}
    \left.
    \begin{array}{rl}
    \dot D_{X} &= 2 K_{X\dot X}\,;\\[6pt] 
    \dot D_{\dot X} &=\nu^{\mathrm{OP}} - 2 (\w_{1 \mathrm{э}}^2 K_{X\dot X} + 
    \delta^{\mathrm{OP}} D_{\dot X})\,;\\[6pt]
\dot K_{X\dot X} &= D_{\dot X} -\w_{1 \mathrm{э}}^2 D_X - 
\delta^{\mathrm{OP}} K_{X\dot X}\,.
\end{array}
\right\}
 \label{e7.3-sin}
\end{equation}
Здесь кубическая функция $X^3$ была заменена на статистически линеаризованную при 
гауссовом распределении с дисперсией  $D_X$ согласно~[1, 2]:
\begin{equation*}
X^3 \approx k_0 (m_X, D_X) m_X + k_1 (m_X, D_X) X^0\,,\label{e7.4-sin}
\end{equation*}
где
\begin{align*}
k_0 (m_X, D_X) &= m_X^2 + 3 D_X\,;\\ 
k_1 (m_X, D_X) &= 3 (m_X^2 + D_X)\,;\\
%\label{e7.5-sin}
\w_{\mathrm{э}}^2 &=\w^2 \lk 1- \fr{\mu (m_X^2 + 3D_X)}{\w^2}\rk\,;\\
\w_{1 \mathrm{э}}^2 &=\w^2 \lk 1-  \fr{3\mu (m_X^2 + D_X)}{\w^2}\rk \enskip 
(\w_{\mathrm{э}}>\w_{1 \mathrm{э}})\,.
\end{align*}
%\label{e7.6-sin}
Из~(\ref{e7.2-sin}) и~(\ref{e7.3-sin}) в стационарном режиме имеем:
\begin{gather*}
m_X^* =0\,;\enskip 
m_{\dot X}^* =0\,;\enskip 
K_{X\dot X}^* =0\,;\\
D_{\dot X}^* =\vartheta\,;\enskip 
\vartheta =  \fr{\nu^{\mathrm{OP}}}{ 2\delta^{\mathrm{OP}}}\,,
\end{gather*}
%\label{e7.7-sin}
а $D_X^*$ определяется из уравнения:
    \begin{equation*}
    \w_{1 \mathrm{э}}^2 (D_X^*) D_X^* =\vartheta\,. %\label{e7.8-sin}
    \end{equation*}
Условие наличия стационарного распределения с инвариантной мерой~(\ref{e3.17-sin}) 
требует консерватизма линеаризованной левой части~(\ref{e7.1-sin}). 
Процесс установления стационарных стохастических колебаний происходит 
в два этапа: сначала устанавливается $D_{\dot X}^*$, а затем $D_X^*$.

Интересно отметить, что уравнения МСЛ~(\ref{e7.2-sin}) и~(\ref{e7.3-sin}) сохраняют свой
вид и для любого белого шума интенсивности  $\nu(t)$,
представляющего собой с.к., производную от произвольного процесса с
независимыми приращениями~$W(t)$. Для гауссовского белого шума
$\nu\hm=\nu^G$ соответствующие результаты получены в~\cite{1-sin, 2-sin, 15-sin}. Как
показали вычислительные эксперименты для значений~$\mu$, отвечающих
стохастическим колебаниям, точность составляет около 10\%~\cite{15-sin}.

\medskip

\noindent
\textbf{Пример~2}.\  Для осциллятора Дуффинга в автокоррелированной  пуассоновской среде, когда
\begin{equation*}
\ddot X+ \w^2 X -\mu X^3 =-\delta^{\mathrm{OP}} \dot X + U\,;\enskip 
\dot U +\gamma U =V^{\mathrm{OP}} (t)\,, %\label{e7.9-sin}
\end{equation*}
уравнения МСЛ для  $Z\hm= [X\dot X U]^{\mathrm{T}}$ имеют вид~(\ref{e3.5-sin}) и~(\ref{e3.6-sin}) при
    \begin{gather*}
   a_1 = \begin{bmatrix}
        m_{\dot X}\\
        -\w_{ \mathrm{э}}^2 m_X-\delta^{\mathrm{OP}} m_{\dot X}\\
        -m_U\end{bmatrix}\,;\\
    \alpha=  \begin{bmatrix}
            0&1&0\\
            -\w_{1 \mathrm{э}}^2&-\delta^{\mathrm{OP}}&0\\
            0&0&-\gamma\end{bmatrix}\,;\enskip
    \beta= \begin{bmatrix}
        0&0&0\\
        0&0&0\\
        0&0&1\end{bmatrix}\,;
%        \label{e7.10-sin}
\\
a_2 =\alpha K_t+ K_t \alpha^{\mathrm{T}} +\beta \nu^{\mathrm{OP}} \beta^{\mathrm{T}}\,.
        \end{gather*}
Здесь $\nu^{\mathrm{OP}} =\nu^{\mathrm{OP}}(t)$~--- интенсивность белого шума 
$V^{\mathrm{OP}}(t)$. 
Отсюда аналитическим мо\-де\-ли\-ро\-ванием определяются стационарные
режимы, а также режимы их установления. Так же, как в\linebreak случае
автокоррелированных гауссовских белых шумов~\cite{1-sin, 2-sin, 15-sin}, точность МСЛ
за счет <<профильтрованности>> помех значительно повышается и
достигает 2\%--4\%. Результат справедлив и для произвольных
негауссовских белых шумов.

\medskip

\noindent
\textbf{Пример 3}.\  Для релейного осциллятора в гауссовской стохастической среде
\begin{equation}
\ddot X + \w^2 {\mathrm{sgn}} X = -\delta^G \dot X + V^G + U_0\label{e7.11-sin}
\end{equation}
плотность распределения стационарного режима стохастических колебаний при $U_0\hm=0$ 
определяется формулой Гиббса~[1, 2]:
\begin{equation}
f^* (x,\dot x) = c \exp \lf - 
    \fr{H(x,\dot x)}{\vartheta^G}\rf\,,\enskip \vartheta^G = 
    \fr{\nu^G}{ 2\delta^G}\,.\label{e7.12-sin}
    \end{equation}
Здесь через
\begin{equation*}
H(x,\dot x) = \fr{\dot x^2}{2} +\Pi(x)\,,\enskip \Pi (x) =\w^2 |x|\,, %\label{e7.13-sin}
\end{equation*}
обозначена полная энергия осциллятора.

Для~(\ref{e7.11-sin}) при  $U_0\hm\ne 0$, если заменить релейную характеристику 
статистически линеаризованной, согласно~[1, 2]
\begin{equation*}
\mathrm{sgn}\, X = k_0 (m_X, D_X) m_X + k_1 (m_X, D_X) (X^0 - m_X)\,; %\label{e7.14-sin}
\end{equation*}
    $$
    k_0(m_X, D_X) =\fr{2}{ m_X} \Phi \left( \fr{m_X}{\sqrt{D_X}}\right)\,;
    $$
    $$ 
    k_1 (m_X,D_X) = \fr{1}{\sqrt{D_X}} \sqrt{\fr{2}{\pi}}\, \exp \left( -\fr{m_X^2}{2D_X}\right)\,;
    $$
\begin{equation}
\Phi (\tau) = \fr{1}{2\pi} \int\limits_0^\tau e^{-t^2/2} dt\,.\label{e7.15-sin}
\end{equation}
Тогда уравнения МСЛ будут иметь вид:
\begin{equation}
\left.
\begin{array}{rl}
\dot m_X &= m_{\dot X}\,;\\[9pt]
\dot m_X &= U_0 - \w^2 k_0 (m_X, D_X) m_X -\delta m_{\dot X}\,;
\end{array}
\right\}
\label{e7.16-sin}
\end{equation}
    \begin{equation}
\left.
\hspace*{-3.5mm}\begin{array}{c}
    \dot D_X = 2 K_{X\dot X}\,;
\\
    \dot D_{\dot X} = \nu^G - 2\lk \delta D_{\dot X} + \w^2 k_1(m_X,D_X) K_{X\dot X}\rk\,;\\[9pt]
    \dot K_{X\dot X} = D_{\dot X} - \w^2 k_1 (m_X, D_X) D_X - \delta K_{X\dot X}\,,
    \end{array}
    \right\}\!\!
    \label{e7.17-sin}
    \end{equation}
где $\delta \hm= \delta^G$, $\nu\hm=\nu^G$.
Отсюда для стационарных стохастических колебаний имеем связанную систему уравнений:
\begin{equation}
m_{\dot X}^* =0\,;\enskip \w^2 k_0 (m_X^*, D_X^*) = U_0\,;\label{e7.18-sin}
\end{equation}
\begin{equation}
\left.
\begin{array}{c}
K_{X\dot X}^* =0\,;\enskip 
D_X^* =\vartheta=\displaystyle \fr{\nu}{ 2\delta}\,;\\[9pt]
k_1(m_X^*, D_X^*) D_X^* =\rho= \displaystyle \fr{\vartheta}{\w^2} =\fr{\nu}{ 2\delta \w^2}\,.
\end{array}
\right\}
\label{e7.19-sin}
\end{equation}

При $U_0 =0$ из~(\ref{e7.15-sin}), (\ref{e7.18-sin}) и~(\ref{e7.19-sin}) находим:
\begin{equation*}
m_X^* =0\,;\enskip 
m_{\dot X}^* =0\,; \enskip 
D_{\dot X}^* =\vartheta\,;\enskip 
D_X^* =  \fr{\pi}{2}\,\rho^2\,. %\label{e7.20-sin}
\end{equation*}
Отсюда видно, что стационарная дисперсия скорости совпадает с точным
решением~(\ref{e7.12-sin}). Стационарная дисперсия координаты, найденная
согласно МСЛ, отличается от следующего точного решения, полученного
согласно~(\ref{e7.12-sin}). При $\rho\hm \le 1$ относительная ошибка составляет
10\%. Стационарные колебания по~$X$ и $\dot X$ не коррелированы.

Уравнения~(\ref{e7.16-sin}) и~(\ref{e7.17-sin}) показывают, что процесс установления 
режима стохастических колебаний происходит в две стадии: сначала устанавливается 
стационарное распределение по ско\-рости~$\dot X$, а затем по координате~$X$.

\medskip

\noindent
\textbf{Пример 4}.  В~условиях примера~3, но для пуассоновской среды, когда
    \begin{equation*}
    \ddot X +\w^2 {\mathrm{sgn}} X =-\delta^{\mathrm{OP}} \dot X + 
    V^{\mathrm{OP}} + U_0\,,
%    \label{e7.21-sin}
    \end{equation*}
уравнения МСЛ имеют вид~(\ref{e7.16-sin}), (\ref{e7.17-sin}), если принять 
$\delta\hm= \delta^{\mathrm{OP}}$, $ \nu\hm=\nu^{\mathrm{OP}}$, 
$\vartheta\hm=\vartheta^{\mathrm{OP}}\hm=\nu^{\mathrm{OP}}/(2\delta^{\mathrm{OP}})$, 
$\rho \hm=\vartheta^{\mathrm{OP}}/\w^2$. Точного аналитического уравнения 
Фел\-ле\-ра--Кол\-мо\-го\-ро\-ва не обнаружено.

Другие тестовые примеры можно найти в~[10, 12--14].

\section{Заключение}

Дано обобщение точных и приближенных (основанных на параметризации распределений)\linebreak 
методов и алгоритмов теории распределений с инвари\-антной мерой на случай нелинейных 
дифференциальных гауссовых и негауссовых стохастических систем с гладкими и разрывными 
характеристиками.

Особое внимание уделено пуассоновским стохастическим системам с разрывными характеристиками.

На тестовых примерах показана достаточная точность для практических приложений в стохастической 
информатике.

{\small\frenchspacing
{%\baselineskip=10.8pt
\addcontentsline{toc}{section}{Литература}
\begin{thebibliography}{99}
\bibitem{1-sin}
\Au{Пугачёв В.\,С., Синицын И.\,Н.} Стохастические дифференциальные системы. 
Анализ и фильтрация.~--- 2-е изд., доп.~--- М.: Наука, 1990.

\bibitem{2-sin}
\Au{Пугачёв В.\,С., Синицын И.\,Н.} Теория стохастических систем.~--- 2-е изд.~--- М.: Логос,  2004.

\bibitem{3-sin}
\Au{Moshchuk N.\,K., Sinitsyn I.\,N.} On stationary distributions in nonlinear 
stochastic differential systems: Preprint.~--- Coventry, UK: 
University of Warwick, Mathematics Institute, 1989. 15~p.

\bibitem{4-sin}
\Au{Moshchuk N.\,K., Sinitsyn I.\,N.} On stochastic nonholonomic systems: Preprint.~--- 
Coventry, UK: University of Warwick, Mathematics Institute, 1989. 32~p.

\bibitem{5-sin}
\Au{Мощук Н.\,К., Синицын И.\,Н.} О~стохастических неголономных системах~// 
Прикладная механика и математика, 1990. Т.~54. Вып.~2. С.~213--223.

\bibitem{6-sin}
\Au{Moshchuk N.\,K., Sinitsyn I.\,N.} On stationary distributions in 
nonlinear stochastic differential systems~// Quart. J. Mech. Appl. Math., 1991. Vol.~44.  
Pt.~4.  P.~571--579.

\bibitem{7-sin}
\Au{Мощук Н.\,К., Синицын И.\,Н.} О~стационарных и приводимых к стационарным 
режимах в нормальных стохастических системах~// 
Прикладная механика и математика, 1991. Т.~55. Вып.~6. С.~895--903.

\bibitem{8-sin}
\Au{Мощук Н.\,К., Синицын И.\,Н.} Распределения с инвариантной мерой в механических 
стохастических нормальных сис\-те\-мах~// Докл. АН СССР, 1992. Т.~322. №\,4. С.~662--667.

\bibitem{9-sin}
\Au{Синицын И.\,Н.} Конечномерные распределения с инвариантной мерой в стохастических 
механических сис\-те\-мах~// Докл. РАН, 1993. Т.~328. №\,3. С.~308--310.

\bibitem{13-sin} %10
\Au{Soize C.} The Fokker--Plank equation for stochastic dynamical systems 
and its explicit steady state solutions.~--- Singapore: World Scientific,  1994.

\bibitem{10-sin} %11
\Au{Синицын И.\,Н.} Конечномерные распределения с инвариантной мерой в 
стохастических нелинейных дифференциальных системах.~--- М.: Диалог--МГУ, 1997. С.~129--140.

\bibitem{11-sin} %12
\Au{Синицын И.\,Н., Корепанов Э.\,Р., Белоусов~В.\,В.} 
Точные методы расчета стационарных режимов с инвариантной мерой в стохастических 
сис\-те\-мах управ\-ле\-ния~// Кибернетика и технологии XXI~ве\-ка: Тр.\ II Междунар. 
науч.-техн. конф. C\&T'2002.~--- Воронеж: Саквое, 2002. С.~124--131.

\bibitem{12-sin} %13
\Au{Синицын И.\,Н., Корепанов Э.\,Р., Белоусов~В.\,В.} 
Точные аналитические методы в статистической динамике нелинейных 
ин\-фор\-ма\-ци\-он\-но-управ\-ля\-ющих сис\-тем~// Сис\-те\-мы и средства информатики. 
Спец. вып. Математическое и алгоритмическое обеспечение 
ин\-фор\-ма\-ци\-он\-но-те\-ле\-ком\-му\-ни\-ка\-ци\-он\-ных сис\-тем.~--- М.: Наука, 2002. С.~112--121.

\bibitem{14-sin}
\Au{Синицын И.\,Н.} Развитие методов аналитического моделирования распределений с 
инвариантной мерой в стохастических сис\-те\-мах~// Современные проб\-ле\-мы 
прикладной математики, информатики и автоматизации: Тр. Междунар. науч.-техн. семинара.~--- 
Севастополь, 2012. С.~24--35.

\bibitem{15-sin}
\Au{Синицын И.\,Н.} Аналитическое моделирование распределений с инвариантной мерой 
в стохастических сис\-те\-мах с автокоррелированными шумами~// 
Информатика и её применения, 2012. Т.~6. Вып.~4. С.~4--8.

\bibitem{16-sin}
\Au{Немыцкий В.\,В., Степанов В.\,В.} Качественная теория дифференциальных уравнений.~--- 
М.--Л.: Гостехиздат, 1949.


\bibitem{17-sin}
\Au{Козлов В.\,В.} О~существовании интегрального инварианта гладких динамических систем~// 
ПММ, 1987. №\,1. С.~538--545.

\label{end\stat}

\bibitem{18-sin}
\Au{Синицын И.\,Н.} Фильтры Калмана и Пугачёва.~--- 2-е изд.~--- М.: Логос, 2007.
\end{thebibliography}
}
}

\end{multicols}  %
\def\stat{kondranin+ushakov}

\def\tit{СИСТЕМА ОБСЛУЖИВАНИЯ С~ОТНОСИТЕЛЬНЫМ ПРИОРИТЕТОМ  И~ПРОФИЛАКТИКАМИ ПРИБОРА$^*$}

\def\titkol{Система обслуживания с~относительным приоритетом  и~профилактиками прибора}

\def\aut{Е.\,С.~Кондранин$^1$,  В.\,Г.~Ушаков$^2$}

\def\autkol{Е.\,С.~Кондранин,  В.\,Г.~Ушаков}

\titel{\tit}{\aut}{\autkol}{\titkol}

\index{Кондранин Е.\,С.}
\index{Ушаков В.\,Г.}
\index{Kondranin E.\,S.}
\index{Ushakov V.\,G.}




{\renewcommand{\thefootnote}{\fnsymbol{footnote}} \footnotetext[1]
{Работа выполнена при финансовой поддержке РФФИ (проект 18-07-00678).}}


\renewcommand{\thefootnote}{\arabic{footnote}}
\footnotetext[1]{Факультет вычислительной математики и~кибернетики Московского государственного 
университета им.\ М.\,В.~Ломоносова, \mbox{ekondranin@yandex.ru}}
\footnotetext[2]{Факультет вычислительной математики и~кибернетики
Московского государственного университета им.\ М.\,В.~Ломоносова;
Институт проб\-лем информатики Федерального исследовательского
центра <<Информатика и~управ\-ле\-ние>> Российской академии наук,
\mbox{vgushakov@mail.ru}}

\vspace*{-10pt}




\Abst{Изучена одноканальная система
массового обслуживания с~двумя типами требований, бесконечным
числом мест для ожидания, гиперэкспоненциальным входящим потоком 
и~профилактиками обслуживающего прибора при освобождении системы.
Тип  требования определяется случайно с~заданными вероятностями 
в~момент его поступления в~систему обслуживания. Требования первого
типа имеют относительный приоритет перед требованиями второго
типа. Найдено нестационарное совместное распределение числа
требований каждого типа в~системе. Профилактики прибора
заключаются в~том, что в~момент освобождения системы от требований
прибор на случайное время с~заданным распределением становится
недоступным для обслуживания. Если за время профилактики поступает
хотя бы одно требование, то начинается нормальное функционирование
системы. Если требования не поступают, то прибор отправляется на
новую профилактику. Такие системы хорошо описывают
функционирование большого числа реальных вычислительных и~информационных систем.}

\KW{гиперэкспоненциальный поток; профилактики
обслуживающего прибора; одноканальная система; относительный
приоритет; длина очереди}

\DOI{10.14357/19922264180405}
  
%\vspace*{4pt}


\vskip 10pt plus 9pt minus 6pt

\thispagestyle{headings}

\begin{multicols}{2}

\label{st\stat}

\section{Введение}

В классической системе массового обслуживания ожидание требований
в очереди связано только с~занятостью обслуживающего прибора. В~то
же время в~реальных системах сам  прибор может пребывать как 
в~активном, так и~в~неактивном состоянии. Такое неактивное
состояние прибора (в~литературе на английском языке используется
термин vacation, а~на русском~--- профилактика или прогулка) может
быть связано со многими причинами. В~част\-ности, сис\-те\-мы
обслуживания с~профилактиками прибора хорошо описывают
функционирование  реальных вычислительных и~информационных систем,
в которых наряду с~основными требованиями имеются второстепенные.
Второстепенные требования всегда присутствуют в~сис\-те\-ме, а~их
обслуживание может проводиться только тогда, когда нет основных,
т.\,е.\ в~фоновом режиме.

С точки зрения самого процесса профилактики прибора существует
несколько ее разновидностей. Во-пер\-вых, могут быть разными
правила, задающие условия начала профилактики: прибор может брать
перерыв только при  полном исчерпании требований в~очереди
(exhaustive service) либо при наличии определенного их числа
(nonexhaustive service). Во-вто\-рых, могут быть разными правила
возвращения прибора в~работу. С~этой точки зрения различают случаи
однократного (single vacation) и~многократного (multiple vacation)
перерыва в~работе. В~первом случае ушедший на профилактику прибор
после ее окончания находится в~рабочем состоянии независимо от
наличия требований в~системе. Во втором случае прибор, не
обнаружив новых требований в~очереди, уходит на новую
профилактику.


В работах~[1--4] можно найти обзор известных результатов, большое
число постановок задач, описание различных приложений и~обширную
библиографию по анализу систем с~профилактиками обслуживающего
прибора.


В настоящей работе исследуется совместное распределение длин
очередей в~нестационарном режиме в~однолинейной системе 
с~ожиданием, гиперэкспоненциальным входящим потоком, двумя типами
требований и~относительным приоритетом. Аналогичная неприоритетная
система обслуживания исследована в~[5].

\vspace*{-6pt}

\section{Описание модели}

Рассматривается однолинейная система массового обслуживания 
с~двумя приоритетными классами требований. Входящий поток~---
гиперэкспоненциальный с~функцией распределения интервалов между
поступлениями требований вида:
\begin{multline*}
A(t)=\sum\limits_{i=1}^kc_i\left(1-e^{-a_it}\right),\enskip t>0,\enskip
a_i>0,\enskip c_i>0,\\
a_i\ne a_j\,,\enskip i\ne j\,,\enskip  \sum\limits_{i=1}^k c_i=1\,.
\end{multline*}

Каждое поступившее требование направляется в~первый класс 
с~вероятностью~$p$ и~во второй класс с~вероятностью $1\hm-p$
независимо от остальных требований. Требования первого класса
обладают относительным приоритетом перед требованиями второго
класса. Длительности обслуживания требований $i$-го приоритетного
класса~--- независимые в~совокупности и~не зависящие от входящего
потока случайные величины с~функцией распределения~$B_i(x)$,
$i\hm=1,2.$
 Если в~некоторый момент времени система освободилась от требований, 
 то обслуживающий прибор
 отправляется на профилактику, которая длится случайное время с~функцией 
 распределения~$C(x).$
 Не ограничивая общности, будем считать, что $B_i(x)\hm<1$
 и~$C(x)\hm<1$  для любого~$x$ 
 и~существуют плотности
 распределения~$b_i(x)$ и~$c(x).$
  Обозначим:
$$
 \beta_i(s)=\int\limits_0^{\infty}e^{-sx}b_i(x)\,dx\,;\enskip 
  \gamma(s)=\int\limits_0^{\infty}e^{-sx}c(x)\,dx\,.
$$
Пока прибор находится на профилактике, он не доступен для
обслуживания. Если за время профилактики поступают требования,
после ее завершения начинается их обслуживание. Если ни одно
требование не поступает, то прибор отправляется на новую
профилактику. Длительности различных профилактик являются
независимыми случайными величинами 
и~не зависят от входящего потока и~времен обслуживания.

\section{Вспомогательные результаты}

  Рассмотрим многочлен по $\mu$ степени $k$ вида:
\begin{multline}
\label{1}
\prod\limits_{i=1}^k\left(\mu+a_i\right)-{}\\
{}-
\left(pz_1+(1-p)z_2\right)\sum\limits_{j=1}^kc_ja_j\prod\limits_{i\ne
j}\left(\mu+a_i\right)\,.
\end{multline}
Занумеруем его корни $\mu_1(z_1,z_2),\ldots,\mu_k(z_1,z_2)$ таким образом,
чтобы они были непрерывными функциями и~$\mu_1(1,1)\hm=0.$ Тогда
$\mathrm{Re}\, \mu_j\left(z_1,z_2\right)\hm<0$, $|z_1|\hm<1$, 
$|z_2|\hm<1,$ $\mu_i(z_1,z_2)\hm\ne \mu_j(z_1,z_2),$ $ i\hm\ne j$,
$j\hm=1,\ldots,k.$ Обозначим:
$$
\alpha_m(z_1,z_2)=\prod\limits_{j\ne m}\left(\mu_m\left(z_1,z_2\right)-
\mu_j\left(z_1,z_2\right)\right)\,.
$$
Справедливы следующие леммы.

\smallskip

\noindent
\textbf{Лемма~1.}\
\textit{Для любого $l=1,\ldots,\:k$ система уравнений}
$$
z_j=\beta_j(s-\mu_l(z_1,z_2)),\ \ j=1,2,
$$
\textit{имеет единственное решение $z_i=z_{il}(s)$ такое, 
что $|z_{il}(s)|\hm<1$ при $l\hm=2,\ldots, k,$ $\mathrm{Re}\, s\hm\geqslant 0,$ 
а~$z_{i1}(0)\hm=1$, $|z_{i1}(s)|\hm<1$ при} $\mathrm{Re}\, s\hm> 0$, $i\hm=1,2.$

\smallskip

\noindent
\textbf{Лемма~2.}\
\textit{При каждом $l\hm=1,\ldots,k$ уравнение}
$$
z_1=\beta_1\left(s-\mu_l(z_1,z_2)\right)
$$
\textit{имеет единственное решение $z_1\hm=z_{1l}(z_2,s),$ 
аналитическое в~области $\mathrm{Re}\, s\hm>0$, $|z_2|\hm<1.$
}

\smallskip

Положим
$$
\lambda_l(s)=\mu_l\left(z_{1l}(s),z_{2l}(s)\right)\,.
$$




\section{Распределение длины очереди}

  Гиперэкспоненциальный поток можно рас\-смат\-ри\-вать как
пуассоновский поток со случайной интен\-сив\-ностью~$a,$ которая
принимает $k$ различных значений $a_1,\ldots,a_k$  с~вероятностями
$c_1,\ldots,c_k.$ Текущее значение~$a$ разыгрывается в~момент
поступления требования и~не меняется между двумя соседними
поступлениями. Введем случайный процесс~$j(t)$ такой, что если
$a\hm=a_j$ в~момент времени $t,$ то $j(t)\hm=j.$

Целью работы является нахождение распределения случайного процесса
$\left(L_1(t),L_2(t)\right),$ где $L_i(t)$~--- число требований из
$i$-го приоритетного класса, находящихся в~системе в~момент
времени~$t.$

При сделанных предположениях относительно параметров изучаемой
системы обслуживания\linebreak процесс $\left(L_1(t),L_2(t)\right)$ не
является, вообще говоря, марковским. Пусть $i(t)=i$, $i\hm=1,2,$ если
в~момент времени~$t$ обслуживается требование из $i$-го
приоритетного класса, и~$i(t)\hm=0,$ если в~момент времени~$t$ прибор
находится на профилактике. Случайный процесс~$x(t)$ определим
следующим образом. Если $i(t)\hm\ne 0,$ то $x(t)$ есть
время, прошедшее с~начала обслуживания требования, находящегося на
приборе, до момента~$t.$ Если $i(t)\hm=0,$ то $x(t)$ есть время,
прошедшее с~начала профилактики прибора до момента~$t.$ Случайный
процесс $\left(L_1(t),L_2(t),i(t),j(t),x(t)\right)$ является
однородным марковским процессом. Положим
\begin{multline*}
P_{ij}(n_1,n_2,x,t)=\fr{\partial}{\partial x}
\mathbf{P}\left(L_1(t)=n_1,L_2(t)=n_2,\right.\\
\left. i(t)=i,j(t)=j,x(t)<x
\vphantom{L_1}\right)\,,\enskip 
 x\geqslant 0,\\ 
 j=1,\ldots,k,\enskip i=0,1,2;
\end{multline*}
\begin{gather*}
\eta_i(x)=\fr{b_i(x)}{1-B_i(x)},\ i=1,2;\enskip 
\eta_0(x)=\fr{c(x)}{1-C(x)}\,;\\
\delta_{i,j}=\begin{cases}
1,&\ i=j;\\ 
0,&\ i\ne j\,.
\end{cases}
\end{gather*}
Функции $P_{ij}(n_1,n_2,x,t)$  удовлетворяют при $x\hm>0$
системам дифференциальных уравнений:
\begin{multline}
\label{3}
\fr{\partial P_{ij}(n_1,n_2,x,t)}{\partial t}+\fr{\partial
P_{ij}(n_1,n_2,x,t)}{\partial
x}={}\\
{}=-(a_j+\eta_i(x))P_{ij}(n_1,n_2,x,t)+ {}\\
{}+
c_j\sum\limits_{l=1}^ka_l\left(p\:P_{il}(n_1-1,n_2,x,t)+{}\right.\\
\left.{}+
(1-p)P_{il}(n_1,n_2-1,x,t)\right)
\end{multline}
и краевым условиям при $x\hm=0$:
\begin{multline}
\label{5}
P_{0j}(n_1,n_2,0,t)=0,\ n_1+n_2>0;\\
P_{0j}(0,0,0,t)=\int\limits_0^{\infty}P_{0j}(0,0,x,t)\eta_0(x)\,dx+{}\\
 {}+\int\limits_0^{\infty}P_{1j}(1,0,x,t)\eta_1(x)dx+{}\\
 {}+
\int\limits_0^{\infty}P_{2j}(0,1,x,t)\eta_2(x)\,dx\,;
\end{multline}

\vspace*{-12pt}

\noindent
\begin{multline}
\label{6}
P_{1j}(n_1,n_2,0,t)+P_{2j}(n_1,n_2,0,t)={}\\
{}=\int\limits_0^{\infty}P_{1j}(n_1+1,n_2,x,t)\eta_1(x)\,dx+{}\\
{}+
\int\limits_0^{\infty}P_{2j}(n_1,n_2+1,x,t)\eta_2(x)\,dx+{}\\
{}+\int\limits_0^{\infty}P_{0j}(n_1,n_2,0,t)\eta_0(x)\,dx\,.
\end{multline}

Будем предполагать, что в~начальный момент времени $t\hm=0$ система
свободна от требований, а~с~начала профилактики прибора прошло
случайное время с~заданным распределением с~плотностью $d(x).$
Таким образом,
\begin{align*}
P_{ij}\left(n_1,n_2,x,0\right)&=0,\ i=1,2;
\\
P_{0j}\left(n_1,n_2,x,0\right)&=c_jd(x)\delta_{n_1+n_2,0},\ \
j=1,\ldots,k\,.
\end{align*}
Положим
\begin{multline*}
p_{ij}\left(z_1,z_2,x,s\right)={}\\
{}=\sum\limits_{n_1=0}^{\infty}
\sum\limits_{n_2=0}^{\infty}z_1^{n_1}z_2^{n_2}\!
\int\limits_0^{\infty}e^{-st}P_{ij}(n_1,n_2,x,t)\,dt\,;
\end{multline*}
$$
  \psi(s)=\int\limits_0^{\infty}e^{-sx}\,dx
  \int\limits_0^{\infty}\fr{c(u+x)d(u)}{1-C(u)}\,du\,.
$$
Тогда, учитывая начальные условия,  из \eqref{3}
получаем:
\begin{multline}
\label{7} 
\fr{\partial p_{ij}(z_1,z_2,x,s)}{\partial x}={}\\
{}=-\left(s+a_j+\eta_i(x)\right)p_{ij}
\left(z_1,z_2,x,s\right)+{}\\
{}+c_j\left(pz_1+(1-p)z_2\right)
\sum\limits_{l=1}^ka_lp_{il}\left(z_1,z_2,x,s\right),\\ 
i=1,2;
\end{multline}

\vspace*{-12pt}

\noindent
\begin{multline}
\label{8} 
\fr{\partial p_{0j}(z_1,z_2,x,s)}{\partial x}={}\\
{}=-\left(s+a_j+\eta_0(x)\right)p_{0j}\left(z_1,z_2,x,s\right)+{}\\
{}+c_j\left(pz_1+(1-p)z_2\right)\sum\limits_{l=1}^ka_lp_{0l}\left(z_1,z_2,x,s\right)+{}\\
{}+ c_jd(x).
\end{multline}
Решения \eqref{7} и~\eqref{8} имеют вид:
\begin{multline}
\label{9}
p_{ij}\left(z_1,z_2,x,s\right)=\left(1-B_i(x)\right)c_j\times{}\\
{}\times \sum\limits_{m=1}^k\fr{\gamma_i^{(m)}(z_1,z_2,s)}{\mu_m(z_1,z_2)+a_j}\,
e^{-(s-\mu_m(z_1,z_2))x}\,,\\
 i=1,2\,,
\end{multline}
\vspace*{-12pt}

\noindent
\begin{multline}
\label{10}
p_{0j}\left(z_1,z_2,x,s\right)={}\\
{}=\left(1-C(x)\right)
c_j\!\!\sum\limits_{m=1}^k\!\! e^{-(s-\mu_m(z_1,z_2))x}\!
\!\left(\!
\vphantom{\int\limits_{l=1}^k}
\delta^{(m)}\left(z_1,z_2,s\right)+{}\right.\\
%\left.
{}+\alpha_m^{-1}\left(z_1,z_2\right)
\prod\limits_{l=1}^k
\left(\mu_m\left(z_1,z_2\right)+a_l\right)\times{}\\
\left.{}\times \int\limits_0^x\!
e^{(s-\mu_m(z_1,z_2))u}
\fr{d(u)}{1-C(u)}\,du
\right)
\!\Bigg/ \!\left(\mu_m\left(z_1,z_2\right)+{}\right.\\
\left.{}+a_j\right)\,,
\end{multline}
где функции $\gamma_i^{(m)}(z_1,z_2,s)$  и~$\delta^{(m)}(z_1,z_2,s)$ являются
произвольными функциями указанных переменных и~определяются из
краевых условий. Из~\eqref{5} и~\eqref{6} получаем:
\begin{multline}
\label{11}
p_{1j}\left(z_1,z_2,0,s\right)+p_{2j}\left(z_1,z_2,0,s\right)={}\\
{}=z_1^{-1}\int\limits_0^{\infty}p_{1j}\left(z_1,z_2,x,s\right)\eta_1(x)\,dx+{}
\\
+z_2^{-1}\int\limits_0^{\infty}p_{2j}\left(z_1,z_2,x,s\right)\eta_2(x)\,dx+{}\\
{}+
\int\limits_0^{\infty}p_{0j}\left(z_1,z_2,x,s\right)\eta_0(x)\,dx
-p_{0j}\left(z_1,z_2,0,s\right)\,.
\end{multline}
Заметим, что $p_{0j}(z_1,z_2,0,s)$ не зависит от $z_1$ и~$z_2,$ т.\,е.\
$p_{0j}(z_1,z_2,0,s)\hm=q_j(s).$ 
Подставляя~\eqref{9} и~\eqref{10} в~\eqref{11}, получаем:
\begin{multline}
\label{12}
\gamma_1^{(m)}\left(z_1,z_2,s\right)\left(1-z_1^{-1}\beta_1(s-\mu_m(z_1,z_2))\right)+{}\\
{}+
\gamma_2^{(m)}(z_1,z_2,s)\left(1-z_2^{-1}\beta_2(s-\mu_m(z_1,z_2))\right)={}\\
{} =
\delta^{(m)}\left(z_1,z_2,s\right)\left(\gamma\left(s-\mu_m\left(z_1,z_2\right)\right)-1\right)+{}\\
{}+
\alpha_m^{-1}\left(z_1,z_2\right)\prod\limits_{l=1}^k
\left(\mu_m\left(z_1,z_2\right)+a_l\right)\psi\left(s-\mu_m(z_1,z_2)\right),\\
j=1,\ldots,k.
\end{multline}
В силу леммы~1 левая часть~\eqref{12} обращается в~0 при
$z_1\hm=z_{1m}(s)$ и~$z_2\hm=z_{2m}(s)$, $m\hm=1,\ldots,k.$ Следовательно,
\begin{multline}
\label{13}
\delta^{(m)}\left(z_{1m}(s),z_{2m}(s),s\right)={}\\
{}=\fr{\psi(s-\lambda_m(s))}{\alpha_m(z_{1m}(s),z_{2m}(s))
(1-\gamma(s-\lambda_m(s)))}\times{}\\
{}\times \prod\limits_{l=1}^k\left(\lambda_m(s)+a_l\right).
\end{multline}
Из \eqref{10} следует, что
$$
q_j(s)=c_j\sum\limits_{m=1}^k\fr{\delta^{(m)}(z_1,z_2,s)}{\mu_m(z_1,z_2)+a_j},\
j=1,\ldots,k .
$$
Решая эту систему уравнений относительно
$\delta^{(m)}(z_1,z_2,s),$ получаем:
\begin{multline}
\label{n1}
\delta^{(m)}(z_1,z_2,s)=\left(pz_1+(1-p)z_2\right)\times{}\\
{}\times
\fr{\prod\nolimits_{j=1}^k(\mu_m(z_1,z_2)+a_j)}
{\alpha_m(z_1,z_2)}\sum\limits_{l=1}^k\frac{a_lq_l(s)}{\mu_m(z_1,z_2)+a_l}.
\end{multline}
Подставляя в~\eqref{n1} $z_1\hm=z_{1m}(s)$ и~$z_2\hm=z_{2m}(s),$ имеем:
\begin{multline}
\label{14}
\delta^{(m)}\left(z_{1m}(s),z_{1m}(s),s\right)={}\\
{}=
\left(pz_{1m}(s)+(1-p)z_{2m}(s)\right)\times{}\\
{}\times
\fr{\prod\nolimits_{j=1}^k
(\lambda_m(s)+a_j)}{\alpha_m(z_{1m}(s),z_{1m}(s))}
\sum\limits_{l=1}^k\fr{a_lq_l(s)}{\lambda_m(s)+a_l}\,.
\end{multline}
Сравнивая два представления~\eqref{13} в~\eqref{14} для
$\delta^{(m)}(z_m(s),s),$ получаем систему уравнений для~$q_l(s)$:
\begin{multline*}
\sum\limits_{l=1}^k\fr{a_lq_l(s)}{\lambda_m(s)+a_l}={}\\
{}=\fr{\psi(s-\lambda_m(s))}{(pz_{1m}(s)+(1-p)z_{2m}(s))
(1-\gamma(s-\lambda_m(s)))},\\
m=1,\ldots,k\,,
\end{multline*}
из которой находим
\begin{multline}
\hspace*{-3pt}q_l(s)=c_l\prod\limits_{j=1}^k
\left(\lambda_l(s)+a_j\right) 
\sum\limits_{m=1}^k
%\fr
\psi(s-\lambda_m(s))\!\Bigg/ \!
\Bigg(\left(1-{}\right.\\
\left.
{}-\gamma\left(s-\lambda_m(s)\right)\right)(\lambda_m(s)+a_l)\times{}\\
{}\times \prod\limits_{n\ne m}(\lambda_m(s)-\lambda_n(s))\!\Bigg).
\label{15}
\end{multline}
Подставляя \eqref{15} в~\eqref{n1} и~учитывая~\eqref{1}, получаем:
\begin{multline*}
\delta^{(m)}(z_1,z_2,s)=\fr{(pz_1+(1-p)z_2)}{\alpha_m(z_1,z_2)}\times
\\
\times\sum\limits_{j=1}^k
\fr{\psi(s-\lambda_j(s))\prod\nolimits_{l=1}^k(\lambda_j(s)+a_l)}
{(pz_{1j}(s)+(1-p)z_{2j}(s))(1-\gamma(s-\lambda_j(s)))}\times{}\\
{}\times\prod\limits_{\nu\ne j}
\fr{\mu_m(z_1,z_2)-\lambda_{\nu}(s)}{\lambda_j(s)-\lambda_{\nu}(s)}\,.
\end{multline*}
Положим
$$
\lambda_m(z_2,s)=\mu_m\left(z_{1m}(z_2,s),z_2\right),\enskip m=1,\ldots,k\,.
$$
Подставляя в~\eqref{12} $z_1\hm=z_{1m}(z_2,s)$, имеем:
\begin{multline}
\label{1q}
\gamma_2^{(m)}\left(z_{1m}(z_2,s),z_2,s\right)={}\\
{}=\fr{\delta^{(m)}(z_{1m}(z_2,s),z_2,s)(\gamma_m(s-\lambda_m(z_2,s))-1)}
{1-z_2^{-1}\beta_2(s-\lambda_m(z_2,s))}+{}
\\
{}+\alpha_m^{-1}(z_{1m}(z_2,s),z_2)\psi(s-\lambda_m(z_2,s))
\prod\limits_{l=1}^k\left(\lambda_m(z_2,s)+{}\right.\\
\left.{}+a_l\right)\!\Bigg/\!
\left(
1-z_2^{-1}\beta_2(s-\lambda_m(z_2,s))\right).
\end{multline}
Далее, из~\eqref{9} следует:
$$
p_{2j}(z_1,z_2,0,s)=c_j\sum\limits_{m=1}^k
\fr{\gamma_2^{(m)}(z_1,z_2,s)}{\mu_m(z_1,z_2)+a_j}\,.
$$
Отсюда
\begin{multline}
\label{2q}
\gamma_2^{(m)}(z_1,z_2,s)=\fr{pz_1+(1-p)z_2}{\alpha_m(z_1,z_2)}\times{}\\
{}\times
\prod\limits_{j=1}^k(\mu_m(z_1,z_2)+a_j)
\sum\limits_{l=1}^k\fr{a_lp_{2l}(z_1,z_2,0,s)}{\mu_m(z_1,z_2)+a_l}\,.
\end{multline}
Так как $p_{2j}(z_1,z_2,0,s)$ не зависит от $z_1$, то
\begin{multline}
\label{3q}
p_{2j}\left(z_1,z_2,0,s\right)={}\\
{}=c_j
\sum\limits_{m=1}^k\fr{\gamma_2^{(m)}\left(z_{1m}(z_2,s),z_2,s\right)}{\lambda_m(z_2,s)+a_j}\,.
\end{multline}
Таким образом, соотношения~\eqref{1q}--\eqref{3q} полностью
определяют $\gamma_2^{(m)}(z_1,z_2,s)$ при любых $z_1$ и~$z_2$.
Теперь из~\eqref{12} можно найти $\gamma_2^{(m)}(z_1,z_2,s)$.

Все функции, необходимые для вычисления $p_{ij}(z_1,z_2,x,s)$,
$i\hm=0,1,2$, $j\hm=1,\ldots,k,$ найде-\linebreak\vspace*{-12pt}

\columnbreak

\noindent
ны. Искомая производящая функция
процесса $(L_1(t),L_2(t))$ равна:

\noindent
\begin{multline*}
\int\limits_0^{\infty}e^{-st}\mathbf{E}
z_1^{L_1(t)} z_2^{L_2(t)}\,dt={}\\
{}=
\sum\limits_{i=0}^2\sum\limits_{j=1}^k\int\limits_0^{\infty}p_{ij}
\left(z_1,z_2,x,s\right)\,dx\,.
\end{multline*}

\vspace*{-18pt}

{\small\frenchspacing
 {%\baselineskip=10.8pt
 \addcontentsline{toc}{section}{References}
 \begin{thebibliography}{9}
\bibitem{1-u}
\Au{Doshi B.\,T.} Queueing systems with vacations~--- a~survey~// 
Queueing Syst., 1986. Vol.~1.  P.~29--66.
\bibitem{2-u}
\Au{Takagi H.} Time-dependent analysis of $M\vert G\vert 1$ vacation models 
with exhaustive service~// Queueing Syst.,
1990. Vol.~6.  P.~369--390.
\bibitem{3-u}
\Au{Li J., Tian N., Zhang~Z.\,G. , Luh~H.\,P.} 
Analysis of the $M\vert G\vert 1$ queue with exponentially working vacations~--- 
a~matrix analytic approach~// Queueing Syst., 2009. Vol.~61.
P.~139--166.
\bibitem{4-u}
\Au{Bouman N., Borst S.\,C., Boxma~O.\,J., Leeuwaarden~J.\,S.\,H.} 
Queues with random back-offs~// Queueing Syst.,
2014. Vol.~77. P.~33--74.
\bibitem{5-u}
\Au{Ушаков~В.\,Г.} Система обслуживания с~гиперэкспоненциальным входящим потоком 
и~профилактиками прибора~// Информатика и~её применения, 2016. Т.~10. 
Вып.~2. С.~93--98.
 \end{thebibliography}

 }
 }

\end{multicols}

\vspace*{-9pt}

\hfill{\small\textit{Поступила в~редакцию 11.05.18}}

\vspace*{6pt}

%\pagebreak

%\newpage

%\vspace*{-28pt}

\hrule

\vspace*{2pt}

\hrule

%\vspace*{-2pt}

\def\tit{A~HEAD OF~THE~LINE PRIORITY QUEUE\\ WITH~WORKING VACATIONS}

\def\titkol{A head of the line priority queue with working vacations}

\def\aut{E.\,S.~Kondranin$^1$ and~V.\,G.~Ushakov$^{1,2}$}

\def\autkol{E.\,S.~Kondranin and~V.\,G.~Ushakov}

\titel{\tit}{\aut}{\autkol}{\titkol}

\vspace*{-11pt}


\noindent
$^1$Department of 
Mathematical Statistics, Faculty of Computational Mathematics and Cybernetics, 
M.\,V.~Lo\-mo-\linebreak
$\hphantom{^1}$no\-sov Moscow State University, 1-52~Leninskiye Gory, 
Moscow 119991, GSP-1, Russian Federation

\noindent
$^2$Institute of Informatics Problems, Federal Research Center 
``Computer Science and Control'' of the Russian\linebreak
$\hphantom{^1}$Academy of Sciences,  44-2~Vavilov Str., Moscow 119333, Russian Federation

\def\leftfootline{\small{\textbf{\thepage}
\hfill INFORMATIKA I EE PRIMENENIYA~--- INFORMATICS AND
APPLICATIONS\ \ \ 2018\ \ \ volume~12\ \ \ issue\ 4}
}%
 \def\rightfootline{\small{INFORMATIKA I EE PRIMENENIYA~---
INFORMATICS AND APPLICATIONS\ \ \ 2018\ \ \ volume~12\ \ \ issue\ 4
\hfill \textbf{\thepage}}}

\vspace*{3pt}



\Abste{The authors analyze the single-server queueing system with 
two types of customers, head of the line priority, hyperexponential 
input stream, and working vacations. The authors obtain the Laplace 
transform (with respect to an arbitrary point in time) of the joint 
distribution of server state, queue size, and elapsed time in that state. 
The authors restrict themselves to a~system with exhaustive service (the 
queue must be empty when the server starts a vacation) and multiple vacations. 
The queueing systems with vacations have been well studied because of their 
applications in modeling computer networks, communication, and manufacturing 
systems. For example, in many digital systems, the processor is multiplexed 
among a~number of jobs and, hence, is not available all the time to handle one job type. 
Besides such an application, theoretical interest in vacation models 
has been aroused with respect to their relationship with polling models.}

\KWE{hyperexponential input stream; working vacations; single server; 
head of the line priority; queue length}



\DOI{10.14357/19922264180405}

\vspace*{-14pt}

\Ack
\noindent
This work was supported by the Russian Foundation for Basic Research 
(project 18-07-00678).


%\vspace*{6pt}

  \begin{multicols}{2}

\renewcommand{\bibname}{\protect\rmfamily References}
%\renewcommand{\bibname}{\large\protect\rm References}

{\small\frenchspacing
 {%\baselineskip=10.8pt
 \addcontentsline{toc}{section}{References}
 \begin{thebibliography}{9}
\bibitem{1-u-1}
\Aue{Doshi, B.\,T.} 1986. Queueing systems with vacations~--- a~survey. 
\textit{Queueing Syst.} 1:29--66.
\bibitem{2-u-1}
\Aue{Takagi, H.} 1990. Time-dependent analysis of $M\vert G\vert M\vert 1$ 
vacation models with exhaustive service. \textit{Queueing Syst.} 6:369--390.
\bibitem{3-u-1}
\Aue{Li, J., N. Tian, Z.\,G.~Zhang,  and H.\,P.~Luh.} 2009. Analysis of the 
$M\vert G\vert 1$ queue with exponentially working vacations~--- 
a~matrix analytic approach. \textit{Queueing Syst.} 61:139--166.
{\looseness=1

}
\bibitem{4-u-1}
\Aue{Bouman, N., S.\,C.~Borst, O.\,J.~Boxma, and J.\,S.\,H.~Leeuwaarden.} 
2014. Queues with random back-offs. \textit{Queueing Syst.} 77:33--74.
\bibitem{5-u-1}
\Aue{Ushakov, V.\,G.} 2016. Sistema obsluzhivaniya s~gipereksponentsialnym 
vkhodyashchim potokom i~profilaktikami\linebreak pribora [Queueing system with working 
vacations and hyperexponential input stream]. 
\textit{Informatika i~ee Primeneniya~--- Inform. Appl.} 10(2):93--98.
\end{thebibliography}

 }
 }

\end{multicols}

\vspace*{-6pt}

\hfill{\small\textit{Received May 11, 2018}}

%\pagebreak

%\vspace*{-18pt}

\Contr

\noindent
\textbf{Kondranin Egor S.} (b.\ 1995)~---  MSc student, Department of 
Mathematical Statistics, Faculty of Computational Mathematics and Cybernetics, 
M.\,V.~Lomonosov Moscow State University, 1-52~Leninskiye Gory, 
Moscow 119991, GSP-1, Russian Federation; \mbox{ekondranin@yandex.ru}

\vspace*{6pt}

\noindent
\textbf{Ushakov Vladimir G.} (b.\ 1952)~--- 
Doctor of Science in physics and mathematics, professor, Department of Mathematical 
Statistics, Faculty of Computational Mathematics and Cybernetics, 
M.\,V.~Lomonosov Moscow State University, 1-52~Leninskiye Gory, Moscow 119991, 
GSP-1, Russian Federation; 
senior scientist, Institute of Informatics Problems, Federal Research Center 
``Computer Science and Control'' of the Russian Academy of Sciences, 
44-2~Vavilov Str., Moscow 119333, Russian Federation; \mbox{vgushakov@mail.ru}
\label{end\stat}

\renewcommand{\bibname}{\protect\rm Литература}        %
\def\stat{shestakov-1}

\def\tit{СИЛЬНАЯ СОСТОЯТЕЛЬНОСТЬ ОЦЕНКИ СРЕДНЕКВАДРАТИЧНОЙ
ПОГРЕШНОСТИ ПРИ~РЕШЕНИИ~ОБРАТНЫХ СТАТИСТИЧЕСКИХ ЗАДАЧ$^*$}

\def\titkol{Сильная состоятельность оценки среднеквадратичной
погрешности при~решении обратных статистических задач}

\def\aut{О.\,В.~Шестаков$^1$}

\def\autkol{О.\,В.~Шестаков}

\titel{\tit}{\aut}{\autkol}{\titkol}

\index{Шестаков О.\,В.}
\index{Shestakov O.\,V.}


{\renewcommand{\thefootnote}{\fnsymbol{footnote}} \footnotetext[1]
{Работа выполнена при частичной финансовой поддержке РФФИ (проект 15-07-02652).}}


\renewcommand{\thefootnote}{\arabic{footnote}}
\footnotetext[1]{Московский государственный университет им.\ М.\,В.~Ломоносова, кафедра 
математической статистики факультета вычислительной математики и~кибернетики; 
Институт проб\-лем информатики Федерального исследовательского центра 
<<Информатика и~управ\-ле\-ние>> Российской академии наук, \mbox{oshestakov@cs.msu.su}}

\vspace*{-18pt}




\Abst{Нелинейные методы обработки сигналов и~изображений с~помощью процедур 
пороговой обработки коэффициентов вейвлет-раз\-ло\-же\-ний стали популярным аппаратом 
для задач подавления шума и~компрессии. Объясняется это тем, что вейвлет-ана\-лиз 
позволяет гораздо более эффективно исследовать нестационарные сигналы, 
чем традиционный Фурье-ана\-лиз, благодаря возможности лучшей адаптации к~функциям, 
имеющим на разных участках различную степень регулярности. Анализ погрешностей 
этих методов представляет собой важную практическую задачу, поскольку позволяет 
оценить качество как самих методов, так и~используемого оборудования. 
В~некоторых приложениях данные наблюдаются не напрямую, а после применения некоторого
 линейного преобразования. Задача обращения такого преобразования, как правило, 
 некорректно поставлена, что приводит к~росту дисперсии шума. В~работе исследуются 
 асимптотические свойства оценки среднеквадратичной погрешности при обращении 
 линейных однородных операторов методами вейв\-лет-вейг\-лет-раз\-ло\-же\-ния 
 и~пороговой обработки. При довольно слабых ограничениях доказывается сильная 
 состоятельность этой оценки.}


\KW{вейвлеты; пороговая обработка; несмещенная оценка риска; коррелированный шум; асимптотическая нормальность}

\vspace*{-8pt}

\DOI{10.14357/19922264170213} 


\vskip 10pt plus 9pt minus 6pt

\thispagestyle{headings}

\begin{multicols}{2}

\label{st\stat}


\section{Введение}

 \vspace*{-4pt}

Методы вейвлет-анализа широко применяются при анализе и~обработке зашумленных данных. 
Во многих статистических задачах эти данные измеряются не напрямую, а после некоторого 
преобразования. В~таких случаях построение вейв\-лет-оце\-нок осуществляется с~по\-мощью 
вейв\-лет-вейг\-лет-раз\-ло\-же\-ния и~процедуры мягкой пороговой обработки. Порог 
обычно зависит от уровня разложения, и~его можно выбирать различными способами,
 исходя из постановки задачи и~целей обработки. Наличие шума неизбежно 
 приводит к~погрешностям. Свойства оценки таких погрешностей 
 (риска) в~модели с~независимым шумом подробно исследовались в~[1--6]. 
 Модель с~коррелированным шумом исследовалась в~[7--9]. Показано, что при 
 определенных условиях оценка риска является состоятельной и~асимптотически 
 нормальной. В~данной работе доказывается сильная состоятельность оценки риска.
 
 \vspace*{-6pt}

\section{Модель данных и~вейвлет-вейглет-разложение}

 \vspace*{-4pt}

В~данной работе рассматривается следующая модель данных:

\noindent
\begin{equation}
\label{Data_Model_Operator}
Y_i=(Kf)_i+\epsilon_i\,,\enskip i=1,\ldots, 2^J\,,
\end{equation}
где $K$~--- некоторый линейный оператор; $f$~--- 
функция, которую необходимо оценить (предполагается, что~$f$ 
принадлежит области определения~$K$); $\{\epsilon_i$, $i \in Z\}$~--- 
стационарный гауссовский процесс с~ковариационной последовательностью 
$r_k \hm= \cov(\epsilon_i,\epsilon_{i+k})$ 
(предполагается, что~$\epsilon_i$ имеют нулевое среднее и~единичную дисперсию). 

В~данной работе рассматривается модель долгосрочной зависимости 
$r_k \sim Ak^{-\alpha}$, $0\hm < \alpha \hm<1$, $A\hm>0$. Как показано в~работе~[10], 
случай краткосрочной зависимости, т.\,е.\ когда $\sum\nolimits_{-\infty}^{+\infty} 
|r_k| \hm< \infty$, эквивалентен модели с~независимым шумом.

 Во многих случаях нельзя оценить функцию~$f$, просто применив к~данным обратный 
 оператор~$K^{-1}$, поскольку такой оператор либо не существует, либо не 
 является ограниченным. Статистические задачи такого рода называются \mbox{некорректно} 
 поставленными. 
 
 Для решения такого рода задач в~работе~[11] предложен метод так 
 называемого вейв\-лет-вейг\-лет-раз\-ло\-же\-ния, 
 который хорошо зарекомендовал себя при обращении линейных однородных операторов, т.\,е.\
  таких операторов~$K$, для которых выпол-\linebreak нено
  
 
  
  \noindent
\begin{equation}
K\left[f\left(a\left(x-x_0\right)\right)\right]=a^{-\beta}(Kf)\left[a\left(x-x_0\right)\right]\notag
\end{equation}
для любого~$x_0$ и~любого $a\hm>0$. Параметр~$\beta$ называется 
показателем однородности. Примерами линейных однородных операторов служат 
оператор интегрирования, преобразование Абеля и~операторы Рисса.

 Вейвлет-разложение функции $f\hm\in L^2(\mathbb{R})$ представляет собой ряд
\begin{align}
f=\sum\limits_{j,k\in Z}\langle f,\psi_{jk}\rangle\psi_{jk},
\label{Wavelet_Decomp}
\end{align}
где $\psi_{jk}(t)\hm=2^{j/2}\psi(2^jt-k)$, а $\psi(t)$~--- 
некоторая материнская вейв\-лет-функ\-ция (семейство $\{\psi_{jk}\}_{jk\in Z}$ 
образует ортонормированный базис в~$L^2(\mathbb{R})$). Индекс~$j$ 
в~\eqref{Wavelet_Decomp} называется масштабом, а~индекс~$k$~--- сдвигом.

В дискретной постановке задачи функция задана в~отсчетах на конечном отрезке. 
Дискретное вейв\-лет-пре\-обра\-зо\-ва\-ние представляет собой умножение вектора 
значений функции~$f$ на орто\-гональную матрицу~$W$, определяемую вейв\-лет-функ\-цией~$\psi$. 
При этом в~силу ортогональности матри\-цы дискретные вейв\-лет-ко\-эф\-фи\-ци\-ен\-ты 
\mbox{связаны} с~непрерывными следующим образом: 
$\mu_{jk}\hm\approx 2^{J/2}\langle f,\psi_{jk}\rangle$, где $2^J$~--- 
число отсчетов функции~$f$~[12]. Всюду далее предполагается, что используются 
вейвлеты Мейера~[12], преобразование\linebreak \mbox{Фурье} которых обладает необходимым количеством 
непрерывных производных.

Поскольку наблюдается не функция~$f$, а~$Kf$, коэффициенты разложения 
в~(\ref{Wavelet_Decomp}) вычислить
напрямую нельзя. Идея метода вейв\-лет-вейг\-лет-раз\-ло\-же\-ния 
заключается в~том, чтобы выразить коэффициенты разложения в~(\ref{Wavelet_Decomp}) 
через~$Kf$. Если \mbox{оператор}~$K$ однороден, то существует последовательность
 функций~$\xi_{jk}$ такая, что $\langle Kf, \xi_{jk}\rangle \hm= \langle f, 
 \psi_{jk}\rangle$. Нормированные функции $u_{jk}\hm=2^{-\beta j}\xi_{jk}$ 
 получили название\linebreak <<вейглеты>>. Своими свойствами они очень похо\-жи на вейвлеты 
 (если соответствующие вейвлеты\linebreak удовлетворяют определенным условиям глад\-кости~[11]), 
 за исключением свойства ортогональности. Таким образом, ряд~(\ref{Wavelet_Decomp}) 
 можно переписать в~виде
\begin{equation*}
f=\sum\limits_{j,k\in Z}2^{-\beta j}\langle Kf, u_{jk}\rangle\psi_{jk}\,,
%\label{Wavelet_Vague_Decomp}
\end{equation*}
которое и~представляет собой вейв\-лет-вейг\-лет-разложение.

При применении дискретного аналога этого разложения 
к~модели~(\ref{Data_Model_Operator}) по аналогии с~дискретным вейв\-лет-пре\-обра\-зо\-ва\-ни\-ем 
получается следующая модель дискретных вейг\-лет-ко\-эф\-фи\-ци\-ен\-тов~\cite{9-sh1}:

\columnbreak 

\noindent
\begin{multline*}
X_{jk} = \mu_{j,k} + 2^{J(1-\alpha)/2} 
\eps_{jk}\,, \\
j=0,\ldots,J-1;\enskip k=0,\ldots,2^j-1\,,
\end{multline*}
где $\mu_{j,k} = 2^{J/2}\langle Kf, \xi_{j,k}\rangle$, 
$\eps_{jk} \hm= \int \xi_{j,k}\, d \mathbf{B}_H$, а $\mathbf{B}_H(t)$~--- 
процесс дробного броуновского движения с~$H \hm= 1-\alpha/2$.
Дисперсии коэффициентов~$X_{jk}$ не зависят от~$k$ и~равны~\cite{10-sh1}
$$
\sigma^2_{j} = C 2^{(J - j)(1 - \alpha)} 2^{2\beta j}\,,
$$
где $C$~--- положительная константа, зависящая от параметров~$A$, $\alpha$ и~$\beta$.

\section{Оценка среднеквадратичной погрешности}

Для подавления шума в~методе вейв\-лет-вейг\-лет-раз\-ло\-же\-ния 
используется процедура пороговой\linebreak обработ\-ки коэффициентов, смысл которой
 заклю\-чается 
в~удалении достаточно маленьких коэффициентов, которые считаются шумом. 
В~данной работе рассматривается так называемая мягкая пороговая обработка. 
К~каждому ко\-эф\-фи\-ци\-ен\-ту применяется функция $\rho_{T}(x)\hm=
\textrm{sgn}\,(x)\left(\abs{x}-T\right)_{+}$, т.\,е.\ 
коэффициенты, которые по модулю меньше порога~$T$, 
обнуляются, а абсолютные величины остальных коэффициентов уменьшаются на 
величину порога.

Среднеквадратичная погрешность (или риск) мягкой пороговой обработки 
определяется следующим образом:
\begin{equation*}
R_J(f)=\sum\limits_{j=0}^{J-1}\sum\limits_{k=0}^{2^j-1}
\Expect\left(\mu_{jk}-\rho_{T}(X_{jk})\right)^2\,.
\end{equation*}

В~\cite{13-sh1} было предложено использовать порог $T_{j} \hm= 
\sigma_{j}\sqrt{2\ln 2^{j}}$ и~показано, что при таком пороге среднеквадратичная 
ошибка близка к~минимальной~\cite{12-sh1}. Этот порог получил название 
<<универсальный>>. В~дальнейшем будет использоваться именно такой вид порога.

Вычислить $R_J(f)$ в~явном виде нельзя, так как в~выражении присутствуют неизвестные 
<<чистые>> коэффициенты~$\mu_{jk}$. Однако его можно оценить с~по\-мощью 
следующей величины:
\begin{equation}
\widehat{R}_J(f)=\sum\limits_{j=0}^{J-1}\sum\limits_{k=0}^{2^j-1}F\left[X_{jk}^2,T\right]\,,
\label{Risk_Estimate}
\end{equation}
где $F[x,T]=(x-\sigma^2)\Ik(|x|\leqslant T^2)\hm+
(\sigma^2+T^2)\Ik(|x|\hm>T^2)$. В~работе~\cite{1-sh1} 
было показано, что~$\widehat{R}_J(f)$ является несмещенной оценкой~$R_J(f)$.

\section{Сильная состоятельность оценки среднеквадратичной погрешности}

В работе~\cite{9-sh1} показано, что~(\ref{Risk_Estimate}) при определенных 
условиях гладкости на функцию~$f$ является асимптотически нормальной и~состоятельной 
оценкой~$R_J(f)$. Оказывается, что эта оценка является также сильно состоятельной 
даже при более слабых ограничениях.

Для доказательства этого утверждения потребуется следующая лемма Боска~\cite{14-sh1}, 
в~которой оценивается вероятность отклонения суммы ограниченных слабозависимых 
случайных величин от ее математического ожидания.

\smallskip

\noindent
\textbf{Лемма.}\ \textit{Пусть $\{X_i,\;i\in Z\}$~--- 
последовательность случайных величин таких, что $\Expect X_i\hm=0$ 
и~$\abs{X_i}\leqslant b$ п.в.\ для всех $i\hm\in Z$, где $b\hm>0$~--- 
некоторая константа. Тогда для любого $q\hm\in[1,n/2]$ и~любого $\eps\hm>0$
\begin{multline}
{\sf P}\left(\abs{\sum\limits_{i=1}^n X_i}>n\eps\right)\leqslant 
4\exp\left\{-\fr{\eps^2}{8b^2}q\right\}+{}\\
{}+22\left(1+\fr{4b}{\eps}\right)^{1/2}
q\alpha\left(\left[\fr{n}{2q}\right]\right)\,,
\label{Bosq_inequality}
\end{multline}
где $\alpha(k)$~--- коэффициент $\alpha$-пе\-ре\-ме\-ши\-ва\-ния последовательности
 $\{X_i,\;i\hm\in Z\}$.}
 
 \smallskip

Докажем теперь сильную состоятельность оценки~(\ref{Risk_Estimate}).

\smallskip

\noindent

\textbf{Теорема.}\ \textit{Пусть $f\in  L^2(\mathbb{R})$ задана на 
конечном отрезке, а $K$~--- линейный однородный оператор с~показателем $\beta\hm>0$. 
Тогда имеет место сходимость
\begin{equation}
\label{R_Conv}
\fr{\widehat{R}_J(f)-R_J(f)}{2^{\lambda J}}\rightarrow 0 \;\mbox{ п.в.~при } 
J\rightarrow\infty
\end{equation}
при любом $\lambda>1/2\hm+2\beta$ в~случае $\alpha\hm+2\beta\hm\geqslant1/2$ 
и~любом $\lambda\hm>1\hm-\alpha$ в~случае $\alpha\hm+2\beta\hm<1/2$}.

\smallskip

\noindent
Д\,о\,к\,а\,з\,а\,т\,е\,л\,ь\,с\,т\,в\,о\,.\ \ 
Пусть $0\hm<p\hm<1$ некоторое число, которое будет выбрано позднее. 
Представим числитель~(\ref{R_Conv}) в~виде
$\widehat{R}_J(f)\hm-R_J(f)\hm=\widehat{R}_1\hm+\widehat{R}_2$, где
\begin{align*}
\widehat{R}_1&=\sum\limits_{j=0}^{[pJ]-1}\sum\limits_{k=0}^{2^j-1}\left(F
\left[X_{jk}^2,T\right]-\Expect F\left[X_{jk}^2,T\right]\right)\,;\\
\widehat{R}_2&=\sum\limits_{j=[pJ]}^{J-1}\sum\limits_{k=0}^{2^j-1}\left(F\left[X_{jk}^2,T\right]
-\Expect F\left[X_{jk}^2,T\right]\right)\,.
\end{align*}
Рассмотрим~$\widehat{R}_2$. Для произвольного $\delta\hm>0$ имеем:
\begin{multline}
p_J={\sf P}\left(\abs{\widehat{R}_2}>\delta2^{\lambda J}\right)\leqslant{}\\
{}\leqslant \sum\limits_{j=[pJ]}^{J-1}{\sf P}\left(
\abs{\sum\limits_{k=0}^{2^j-1}\left(F\left[X_{jk}^2,T\right]-
\Expect F\left[X_{jk}^2,T\right]\right)}>{}\right.\\
\left.{}>\delta J^{-1}2^{\lambda J}
\vphantom{\abs{\sum\limits_{k=0}^{2^j-1}\left(F\left[X_{jk}^2,T\right]-
\Expect F\left[X_{jk}^2,T\right]\right)}}
\right)\,.
\label{PJ_inequality}
\end{multline}
Из вида функции $F[x,T]$ следует, что $-\sigma^2_j\hm\leqslant F[X_{jk}^2,T_j]
\hm\leqslant\sigma^2_j\hm+T^2_j$. В~работе~\cite{10-sh1} 
показано, что в~силу свойств вейвлетов Мейера при каждом~$j$ слагаемые в~сумме 
под вероятностью в~(\ref{PJ_inequality}) удовлетворяют свойству $\rho$-пе\-ре\-ме\-ши\-ва\-ния 
с~коэффициентом $\rho(k)\hm\leqslant C k^{-M}$, где $C\hm>0$~--- 
некоторая константа, а~$M$ можно сделать достаточно большим, выбрав соответствующий 
вейвлет Мейера.

Известно~\cite{15-sh1}, что для коэффициентов $\alpha$-пе\-ре\-ме\-ши\-ва\-ния 
и~$\rho$-пе\-ре\-ме\-ши\-ва\-ния справедливо неравенство $4\alpha(k)\hm\leqslant\rho(k)$. 
Применяя неравенство~(\ref{Bosq_inequality}) с~$q\hm=2^{\theta j}$ ($\theta\hm<1$) 
для каждого~$j$ в~сумме~(\ref{PJ_inequality}) и~выбирая~$M$
 достаточно большим, получаем:
\begin{multline}
p_J\leqslant c_1 J\times{}\\
{}\times\max\limits_{[pJ]\leqslant j\leqslant J-1}
\left\{\exp\!\left[-c_2J^{-3} 2^{2(\lambda-1+\alpha)J+(\theta-2\alpha-4\beta)j}\right]
\!\right\}+{}\\
{}+o_J\,.
\label{P_inequality}
\end{multline}
Здесь $c_1$ и~$c_2$~--- некоторые положительные константы, а~$o_J$ убывает по~$J$ 
быстрее, чем~$2^{-M_0 pJ}$, где~$M_0$~--- некоторое положительное число, 
зависящее от~$M$.

Если $\alpha+2\beta\geqslant 1/2$, то $\theta\hm-2\alpha\hm-4\beta\hm<0$, и~при $j\hm=J$ 
правая часть~(\ref{P_inequality}) не превосходит $c_1 J\exp\left[-c_2J^{-3} 
2^{(2\lambda-2+\theta-4\beta)J}\right]$. Поскольку $\theta\hm<1$ 
можно выбрать произвольно, для того чтобы выполнялось неравенство 
$2\lambda\hm-2\hm+\theta\hm-4\beta\hm>0$,
 до\-статочно потребовать $\lambda\hm>1/2\hm+2\beta$. 
Если же\linebreak $\alpha\hm+2\beta\hm<1/2$, то можно выбрать $\theta\hm<1$ так, 
что $\theta\hm-2\alpha\hm-4\beta\hm>0$, и~правая часть~(\ref{P_inequality}) 
не превосходит $c_1 J\exp\left[-c_2J^{-3} 2^{2(\lambda-1+\alpha)J}\right]$. 
Следовательно, чтобы выполнялось неравенство $\lambda\hm-1\hm+\alpha\hm>0$, достаточно 
потребовать $\lambda\hm>1\hm-\alpha$. При таком выборе~$\lambda$
\begin{equation*}
\sum\limits_{J=1}^{\infty}p_J<\infty\,,
\end{equation*}
и в~силу леммы Бо\-ре\-ля--Кан\-тел\-ли для любого $\delta\hm>0$ 
событие $\left\{\abs{\widehat{R}_2}\hm>\delta2^{\lambda J}\right\}$ осуществляется лишь 
конечное число раз. Следовательно, $\widehat{R}_2 2^{-\lambda J}\rightarrow 0$ п.в.

В $\widehat{R}_1$ при каждом фиксированном~$j$ ($0\hm\leqslant j\hm\leqslant [pJ]\hm-1$) 
чис\-ло слагаемых равно~$2^j$, а каждое слагаемое не превосходит по
 модулю $B J2^{J(1-\alpha)}2^{j(2\beta+\alpha-1)}$, где $B\hm>0$~--- 
 некоторая константа. Следовательно, 
 $\abs{\widehat{R}_1}\hm\leqslant B_1J 2^{J(1-\alpha+p(2\beta+\alpha))}$, где 
 $B_1$~--- некоторая положительная константа. Если $2\beta\hm+\alpha\hm\geqslant1/2$, 
 то при $\lambda\hm>1/2\hm+2\beta$ всегда можно выбрать такое $0\hm<p\hm<1$, что 
 $\lambda\hm-(1\hm-\alpha\hm+p(2\beta\hm+\alpha))\hm>0$, и,~следовательно, 
 $\widehat{R}_1 2^{-\lambda J}\rightarrow 0$ п.в. 
 Если же $\alpha\hm+2\beta\hm<1/2$, то при $\lambda\hm>1\hm-\alpha$ 
 всегда можно выбрать такое $0\hm<p\hm<1$, что $\lambda\hm-(1\hm-\alpha\hm+p(2\beta
 \hm+\alpha))\hm>0$, и,~следовательно, $\widehat{R}_1 2^{-\lambda J}\rightarrow 0$ 
 п.в. Теорема до\-ка\-зана.


{\small\frenchspacing
 {%\baselineskip=10.8pt
 \addcontentsline{toc}{section}{References}
 \begin{thebibliography}{99}
\bibitem{1-sh1}
\Au{Donoho D., Johnstone~I.\,M.} Adapting to unknown smoothness via wavelet shrinkage~// 
J.~Amer. Stat. Assoc., 1995. Vol.~90. P.~1200--1224.

\bibitem{3-sh1} %2
\Au{Маркин А.\,В.} Предельное распределение оценки риска при пороговой обработке 
вейв\-лет-ко\-эф\-фи\-ци\-ен\-тов~// Информатика и~её применения, 2009. Т.~3. Вып.~4. С.~57--63.

\bibitem{4-sh1} %3
\Au{Маркин А.\,В., Шестаков~О.\,В.} О~со\-сто\-ятель\-ности оценки риска при пороговой 
обработке вейв\-лет-ко\-эф\-фи\-ци\-ен\-тов~// Вестн. Моск.
ун-та. Сер.~15: Вычисл. матем. и~киберн., 2010. №\,1. C.~26--34.

\bibitem{2-sh1} %4
\Au{Кудрявцев А.\,А., Шестаков~О.\,В.} Асимптотика оценки риска при вейг\-лет-вейв\-лет-раз\-ло\-же\-нии 
наблюдаемого сигнала~// T-Comm: Телекоммуникации и~транспорт, 2011. №\,2. С.~54--57.



\bibitem{5-sh1}
\Au{Шестаков О. \,В.}  Асимптотическая нормальность оценки риска 
пороговой обработки вейв\-лет-ко\-эф\-фи\-ци\-ен\-тов при выборе
адаптивного порога~// Докл. РАН, 2012. Т.~445. №\,5. С.~513--515.

\bibitem{6-sh1}
\Au{Шестаков О.\,В.} О~свойствах оценки среднеквадратичного риска при регуляризации 
обращения линейного однородного оператора с~помощью адаптивной пороговой 
обработки коэффициентов вейг\-лет-вейв\-лет раз\-ло\-же\-ния~// 
Вестн. ТвГУ. Серия: Прикладная математика, 2012. №\,8. С.~117--130.

\bibitem{9-sh1} %7
\Au{Ерошенко А.\,А., Шестаков~О.\,В.} Асимптотическая нормальность оценки риска 
при вейв\-лет-вейг\-лет-раз\-ло\-же\-нии функции сигнала в~модели с~коррелированным 
шумом~// Вестн. Моск. ун-та. Сер.~15: Вычисл. матем. и~киберн., 2014. №\,3. C.~110--117.

\bibitem{7-sh1} %8
\Au{Ерошенко А.\,А.} Состоятельность оценок риска при вейв\-лет-вейг\-лет 
и~вейг\-лет-вейв\-лет-раз\-ло\-же\-ни\-ях функции сигнала в~модели с~коррелированным 
шумом~// Вестн. ТвГУ. Серия: Прикладная математика, 2015. №\,1. С.~103--114.

\bibitem{8-sh1} %9
\Au{Ерошенко А.\,А., Кудрявцев~А.\,А., Шестаков~О.\,В.} 
Предельное распределение оценки риска метода вейг\-лет-вейв\-лет-раз\-ло\-же\-ния 
сигнала в~модели с~коррелированным шумом~// Вестн. Моск. ун-та. Сер.~15: 
Вычисл. матем. и~киберн., 2015. №\,1. C.~12--18.



\bibitem{10-sh1}
\Au{Johnstone I.\,M.} Wavelet shrinkage for correlated data and inverse problems: 
Adaptivity results~// Stat. Sinica, 1999. Vol.~9. No.\,1. P.~51--83.

\bibitem{11-sh1}
\Au{Donoho D.} Nonlinear solution of linear inverse problems by wavelet-vaguelette 
decomposition~// Appl. Comput. Harmon. Anal., 1995. Vol.~2. P.~101--126.

\bibitem{12-sh1}
\Au{Mallat S.} A~wavelet tour of signal processing.~--- New York, NY, USA:
Academic Press, 1999. 857~p.

\bibitem{13-sh1}
\Au{Kolaczyk E.\,D.} Wavelet methods for the inversion of certain homogeneous 
linear operators in the presence of noisy data.~--- Stanford, CA, USA:
Stanford University, 1994.
 PhD Thesis.

\bibitem{14-sh1}
\Au{Bosq D.} Nonparametric statistics for stochastic processes: Estimation and 
prediction.~--- New York, NY, USA: Springer-Verlag, 1996. 169~p.

\bibitem{15-sh1}
\Au{Bradley R.\,C.} Basic properties of strong mixing conditions. 
A~survey and some open questions~// Probab. Surveys, 2005. Vol.~2. P.~107--144.

 \end{thebibliography}

 }
 }

\end{multicols}

\vspace*{-3pt}

\hfill{\small\textit{Поступила в~редакцию 11.11.16}}

\vspace*{8pt}

%\newpage

%\vspace*{-24pt}

\hrule

\vspace*{2pt}

\hrule

%\vspace*{8pt}


\def\tit{STRONG CONSISTENCY OF~THE~MEAN SQUARE RISK ESTIMATE IN~THE~INVERSE STATISTICAL 
PROBLEMS}

\def\titkol{Strong consistency of~the~mean square risk estimate in~the~inverse statistical 
problems}

\def\aut{O.\,V.~Shestakov$^{1,2}$}

\def\autkol{O.\,V.~Shestakov}

\titel{\tit}{\aut}{\autkol}{\titkol}

\vspace*{-9pt}


\noindent
$^1$Department of Mathematical Statistics, Faculty of Computational Mathematics 
and Cybernetics, M.\,V.~Lo-\linebreak
$\hphantom{^1}$monosov Moscow State University, 1-52~Leninskiye Gory,
 GSP-1, Moscow 119991, Russian Federation
 
 \noindent
 $^2$Institute of Informatics Problems, Federal Research Center 
 ``Computer Science and Control'' of the Russian\linebreak
 $\hphantom{^1}$Academy of Sciences, 44-2~Vavilov Str., 
 Moscow 119333, Russian Federation



\def\leftfootline{\small{\textbf{\thepage}
\hfill INFORMATIKA I EE PRIMENENIYA~--- INFORMATICS AND
APPLICATIONS\ \ \ 2017\ \ \ volume~11\ \ \ issue\ 2}
}%
 \def\rightfootline{\small{INFORMATIKA I EE PRIMENENIYA~---
INFORMATICS AND APPLICATIONS\ \ \ 2017\ \ \ volume~11\ \ \ issue\ 2
\hfill \textbf{\thepage}}}

\vspace*{3pt}



\Abste{Nonlinear methods of digital signal processing based on 
thresholding of wavelet coefficients became a popular tool for solving 
the problems of signal de-noising and compression. This is explained by 
the fact that the wavelet methods allow much more effective analysis of 
nonstationary signals than traditional Fourier analysis,
thanks to the 
better adaptation to the functions with varying degrees of regularity. 
Wavelet thresholding risk
analysis is an important practical task, because 
it allows determining the quality of techniques themselves and the equipment\linebreak\vspace*{-12pt}}

\Abstend{which is being used. In some applications, the data are observed not directly 
but after applying a~linear transformation. The problem of inverting this 
transformation is usually set incorrectly, leading to an increase in the 
noise variance. In this paper, the asymptotic properties of the mean square 
error (MSE) estimate are studied when inverting linear homogeneous operators by means of wavelet 
vaguelette decomposition and thresholding. 
The strong consistency of this estimate has been proved under
mild conditions.}

\KWE{wavelets; thresholding; MSE risk estimate; correlated noise;
 asymptotic normality} 
 
\DOI{10.14357/19922264170213} 

\vspace*{-8pt}

\Ack
\noindent
The work was partly supported by the Russian Foundation
for Basic Research projects Nos.\,15-37-20611 and
16-07-00677).



\vspace*{3pt}

  \begin{multicols}{2}

\renewcommand{\bibname}{\protect\rmfamily References}
%\renewcommand{\bibname}{\large\protect\rm References}

{\small\frenchspacing
 {%\baselineskip=10.8pt
 \addcontentsline{toc}{section}{References}
 \begin{thebibliography}{99}

\bibitem{1-sh1-1}
\Aue{Donoho, D., and I.\,M.~Johnstone.} 1995. Adapting to unknown smoothness via
 wavelet shrinkage. \textit{J.~Amer. Stat. Assoc.} 90:1200--1224.
 
 \bibitem{3-sh1-1}
\Aue{Markin, A.\,V.} 2009. Predel'noe raspredelenie otsenki riska pri 
porogovoy obrabotke veyvlet-koeffitsientov [Limit distribution of risk estimate 
of wavelet coefficient thresholding]. \textit{Informatika i~ee Primeneniya~--- 
Inform. Appl.}  3(4):57--63.

\bibitem{4-sh1-1}
\Aue{Markin, A.\,V., and O.\,V.~Shestakov.} 2010. Consistency of risk estimation 
with thresholding of wavelet coefficients. \textit{Moscow Univ. Comput. Math. 
Cybern.} 34(1):22--30.


\bibitem{2-sh1-1} %4
\Aue{Kudryavtsev, A.\,A., and O.\,V.~Shestakov}. 2011. Asimptotika otsenki riska pri 
veyglet-veyvlet-razlozhenii nablyudaemogo signala 
[The asymptotic behavior of the risk estimate under wavelet-vaguelette 
decomposition of the observed signal]. \textit{T-Comm: Telekommunikatsii i Transport}
[T-Comm: Telecommunications and Transport] 2:54--57.



\bibitem{5-sh1-1}
\Aue{Shestakov, O.\,V.} 2012. Asymptotic normality of adaptive wavelet thresholding 
risk estimation. \textit{Dokl. Math.} 86(1):556--558.

\bibitem{6-sh1-1}
\Aue{Shestakov, O.\,V.} 2012. O~svoystvakh otsenki sred\-ne\-kvad\-ra\-ti\-ch\-no\-go riska pri 
regulyarizatsii obrashcheniya lineynogo odnorodnogo operatora s~pomoshch'yu adaptivnoy 
porogovoy obrabotki koeffitsientov veyglet-veyvlet razlozheniya 
[The properties of mean square error estimate when regularizing the inversion of the homogeneous 
linear operator using adaptive thresholding of wavelet-vaguelette decomposition 
coefficients]. \textit{Vestn. TvGU. Seriya: Prikladnaya matematika} 
[Herald of Tver State University. Series: Applied Mathematics] 8:117--130.

\bibitem{9-sh1-1} %7
\Aue{Eroshenko, A.\,A., and O.\,V.~Shestakov}. 2014. Asymptotic normality of 
estimating risk upon the wavelet-vaguelette decomposition of a~signal function 
in a~model with correlated noise. \textit{Moscow Univ. Comput. Math.  Cybern.} 
38(3):110--117.

\bibitem{7-sh1-1} %8
\Aue{Eroshenko, A.\,A.} 2015. Sostoyatel'nost' otsenok riska pri veyvlet-veyglet 
i~veyglet-veyvlet-razlozheniyakh funktsii signala v~modeli s~korrelirovannym shumom 
[Consistency of risk estimates for wavelet-vaguelette and vaguelette-wavelet 
decompositions of signal function in the model of data with correlated noise]. 
\textit{Vestn. TvGU. Seriya: Prikladnaya matematika} 
[Herald of Tver State University. Series: Applied Mathematics] 1:103--114.

\bibitem{8-sh1-1} %9
\Aue{Eroshenko, A.\,A., A.\,A.~Kudryavtsev,  and O.\,V.~Shestakov.} 2015. 
Limit distribution of a~risk estimate using the vaguelette-wavelet decomposition 
of signals in a model with correlated noise. 
\textit{Moscow Univ. Comput. Math. Cybern.} 39(1):6-13.



\bibitem{10-sh1-1}
\Aue{Johnstone, I.\,M.} 1999. Wavelet shrinkage for correlated data and inverse
 problems: Adaptivity results. \textit{Stat. Sinica} 9(1):51--83.

\bibitem{11-sh1-1}
\Aue{Donoho, D.} 1995. Nonlinear solution of linear inverse problems by 
wavelet-vaguelette decomposition. \textit{Appl. Comput. Harmon. Anal.}  
2:101--126.


\bibitem{12-sh1-1}
\Aue{Mallat, S.} 1999. \textit{A~wavelet tour of signal processing.} 
New York, NY: Academic Press. 857~p.

\bibitem{13-sh1-1}
\Aue{Kolaczyk, E.\,D.} 1994. Wavelet methods for the inversion of certain homogeneous 
linear operators in the presence of noisy data. Stanford, CA: Stanford University. 
PhD Thesis. 163~p.

\bibitem{14-sh1-1}
\Aue{Bosq, D.} 1996. \textit{Nonparametric statistics for stochastic processes: 
Estimation and prediction.} New York, NY: Springer-Verlag. 169~p.

\bibitem{15-sh1-1}
\Aue{Bradley, R.\,C.} 2005. Basic properties of strong mixing conditions. 
A~survey and some open questions. \textit{Probab. Surveys} 2:107--144.
\end{thebibliography}

 }
 }

\end{multicols}

\vspace*{-3pt}

\hfill{\small\textit{Received November 11, 2016}}

\vspace*{-12pt}

\Contrl

\noindent
\textbf{Shestakov Oleg V.} (b.\ 1976)~--- 
Doctor of Science in physics and mathematics, associate professor, 
Department of Mathematical Statistics, Faculty of Computational Mathematics 
and Cybernetics, M.\,V.~Lomonosov Moscow State University, 1-52~Leninskiye Gory,
 GSP-1, Moscow 119991, Russian Federation; senior scientist, 
 Institute of Informatics Problems, Federal Research Center 
 ``Computer Science and Control'' of the Russian Academy of Sciences, 44-2~Vavilov Str., 
 Moscow 119333, Russian Federation; \mbox{oshestakov@cs.msu.su}
\label{end\stat}


\renewcommand{\bibname}{\protect\rm Литература}   %

\newcommand{\It}{\mathbf{1}}

\def\stat{shestakov-2}

\def\tit{УНИВЕРСАЛЬНАЯ ПОРОГОВАЯ ОБРАБОТКА В~МОДЕЛЯХ~С~НЕГАУССОВЫМ ШУМОМ}

\def\titkol{Универсальная пороговая обработка в моделях с~негауссовым шумом}

\def\aut{О.\,В.~Шестаков$^1$}

\def\autkol{О.\,В.~Шестаков}

\titel{\tit}{\aut}{\autkol}{\titkol}

\index{Шестаков О.\,В.}
\index{Shestakov O.\,V.}


%{\renewcommand{\thefootnote}{\fnsymbol{footnote}} \footnotetext[1]
%{Работа выполнена при частичной финансовой поддержке РФФИ (проект 15-07-02652).}}


\renewcommand{\thefootnote}{\arabic{footnote}}
\footnotetext[1]{Московский государственный университет им.\ М.\,В.~Ломоносова, кафедра 
математической статистики факультета вычислительной математики и кибернетики; 
Институт проб\-лем информатики Федерального исследовательского центра 
<<Информатика и~управ\-ле\-ние>> Российской академии наук, \mbox{oshestakov@cs.msu.su}}

%\vspace*{-18pt}


\Abst{В~задачах непараметрического оценивания сигнала обычно предполагается, 
что функция сигнала принадлежит некоторому специальному классу. Например, она 
может быть ку\-соч\-но-не\-пре\-рыв\-ной или 
ку\-соч\-но-диф\-фе\-рен\-ци\-ру\-емой и~иметь компактный носитель. Эти 
предположения, как правило, позволяют экономно представить функцию сигнала 
в~некотором специально подобранном базисе таким образом, что полезный сигнал 
оказывается сосредоточенным в относительно небольшом количестве больших по 
абсолютному значению коэффициентов разложения. Затем осуществляется пороговая 
обработка с целью удаления шумовых коэффициентов. Обычно распределение шума 
предполагается гауссовым. Эта модель хорошо изучена в~литературе, и для разных 
классов функций сигналов вычислены оптимальные параметры пороговой обработки. 
В~данной работе рассматривается задача построения оценки функции сигнала по 
наблюдениям, содержащим аддитивный шум, распределение которого принадлежит 
достаточно широкому классу. Вычисляются значения универсальных параметров 
пороговой обработки, при которых среднеквадратичный риск близок 
к~минимальному.}

\KW{пороговая обработка; негауссовый шум; среднеквадратичный риск}

%\vspace*{-6pt}

\DOI{10.14357/19922264170214} 


\vskip 10pt plus 9pt minus 6pt

\thispagestyle{headings}

\begin{multicols}{2}

\label{st\stat}

\section{Введение}

Современные методы построения оценок функции сигнала по зашумленным наблюдениям 
часто основаны на разложении этой функции по базису, обеспечивающему <<экономное>> 
представление данных, т.\,е.\ коэффициенты такого разложения убывают достаточно 
быстро (примерами подобных базисов могут служить различные классы вейвлетов). 
Затем происходит обнуление части коэффициентов, которые по предположению содержат 
в~основном шум. В~предположении о гауссовском распределении шума эти методы хорошо 
разработаны~[1--3]. В~данной работе рассматривается модель с аддитивным шумом, 
который необязательно имеет гауссово распределение, и вычисляются 
универсальные параметры диагональных методов подавления шума, при которых 
среднеквадратичный риск близок к минимальному.

\section{Модель данных и~методы подавления шума}

Предположим, что данные имеют вид:
\begin{equation*}
%\label{data_model}
X_i=f_i+z_i\,, \enskip i=1,\ldots,N\,,
\end{equation*}
где $f_i$~--- <<чистый>> сигнал, а~$z_i$~--- 
<<шумовые>> коэффициенты, относительно которых предполагается, что они 
независимы и имеют распределение с симметричной дифференцируемой плотностью~$h(x)$. 
Также предположим, что $\sup\limits_{x\in {\mathbf R}}\abs{h'(x)}\hm<A$ 
с~некоторой константой $A\hm>0$ и что
$$
h(x)\asymp x^\alpha e^{-\theta x^\beta}\ \mbox{при } x\to\infty\,, \enskip
\alpha\in {\mathbf R}\,, \  \theta>0\,, \  \beta>0\,.
$$
Дисперсию $z_i$ обозначим через~$\sigma^2$. Класс распределений такого вида 
достаточно широк. Распределения из этого класса могут иметь как более легкие, 
так и более тяжелые хвосты, чем гауссово распределение.

При построении оценки сигнала будем рас\-смат\-ри\-вать только диагональные методы, 
т.\,е.\ когда для получения оценки~$\hat{f}_i$ коэффициента~$f_i$ используется 
только величина~$X_i$. Определим среднеквадратичный риск оценки:
\begin{equation}
\label{MSE}
R(\hat{f})=\sum\limits_{i=1}^{N}\e\left(\hat{f}_i-f_i\right)^2\,.
\end{equation}
Рассмотрим метод построения оценки, который заключается в том, 
что каждый коэффициент либо обнуляется, либо остается неизменным:
$$
\hat{f}_i=\rho_\delta\left(X_i\right)=\delta_i X_i\,, \enskip
\delta_i\in\{0,1\}\,,\ i=1,\ldots,N\,.
$$
Предположим, что известны <<идеальные>> па\-ра\-мет\-ры~$\delta_i$, которые 
минимизируют риск~\eqref{MSE}. Поскольку слагаемые в~\eqref{MSE} равны~$f^2_i$, 
если $\delta_i\hm=0$, и~$\sigma^2$, если $\delta_i\hm=1$, минимальный 
среднеквадратичный риск равен

\vspace*{-2pt}

\noindent
\begin{equation}
\label{MSE_Min}
R_{\mathrm{Min}}(\hat{f})=\sum\limits_{i=1}^{N}\min\left(f_i^2,\sigma^2\right)\,,
\end{equation}
а <<идеальные>> параметры равны $\delta_i\hm=\It(\abs{f_i}\hm>\sigma)$. На практике 
вычислить эти параметры нельзя и невозможно построить оценку, риск которой 
равен~\eqref{MSE_Min}. Однако в работе~\cite{DonJ94} показано, что при 
использовании процедуры пороговой обработки в~модели с~гауссовым можно обеспечить 
порядок сред\-не\-квад\-ра\-тич\-но\-го риска, который близок к~\eqref{MSE_Min} с точностью до 
логарифмического множителя.

Пороговая обработка является одним из самых популярных методов подавления шума. 
Ее смысл заключается в обнулении коэффициентов, чьи абсолютные значения не превышают 
заданного порога. Оценка~$\hat{f}_i$ вычисляется с помощью пороговой 
функции $\rho_T(x)$ с порогом~$T$. Наиболее популярными являются функция жесткой 
пороговой обработки $\rho_T^{(h)}(x)\hm=x \cdot\mathbf{1} (|x|>T)$ и мягкой пороговой 
обработки $\rho_T^{(s)}(x)\hm={\mbox{sign}}(x)(|x|-T)_+$. Сред\-не\-квад\-ра\-тич\-ный риск 
пороговой обработки обозначим через~$R_T(\hat{f})$.

\vspace*{-2pt}

\section{Универсальный порог}

\vspace*{-2pt}

Одной из основных проблем при пороговой обработке является стратегия выбора порога. 
В~работе \cite{DonJ94} предложен так называемый универсальный порог для модели 
с~гауссовским шумом. Этот порог является в~некотором смысле максимальным среди 
<<разумных>> порогов, и~среднеквадратичный риск при таком пороге близок 
к~\eqref{MSE_Min}. Более точные значения порога для различных функций потерь при 
дополнительных условиях на гладкость функции сигнала получены в работах~[3--6]. 
В~данной работе предлагается аналог универсального порога для определенной выше 
более общей модели шума и~показывается, что он обладает практически такими же 
свойствами, как в~модели с~гауссовским шумом.

Пусть $T_U\hm=\left(\theta^{-1}\ln N\right)^{1/\beta}$. 
По аналогии с гауссовской моделью шума назовем этот порог универсальным.

\smallskip

\noindent
\textbf{Теорема~1.}\ \textit{Существует такая константа $C\hm>0$, 
зависящая только от~$h(x)$, что начиная с некоторого~$N$ при жесткой пороговой 
обработке}
\begin{equation}
\label{MSE_Hard}
R_{T_U}\left(\hat{f}\right)\leq C(\ln N)^{\delta(\alpha,\beta)}
\left(\sigma^2+R_{\mathrm{Min}}\left(\hat{f}\right)\right),
\end{equation}
где $\delta(\alpha,\beta)=\max({2}/{\beta},({3+\alpha-\beta})/{\beta})$.

\columnbreak

\noindent
Д\,о\,к\,а\,з\,а\,т\,е\,л\,ь\,с\,т\,в\,о\,.\ \ 
Рассмотрим $\e(\rho_{T_U}^{(h)}(X_i)\hm-f_i)^2$. Пусть $\abs{f_i}\hm>T_U$. Тогда
\begin{multline*}
\e\left(\rho_{T_U}^{(h)}\left(X_i\right)-f_i\right)^2=\sigma^2-{}\\
{}-
\e\left(X_i-f_i\right)^2\It\left(\abs{X_i}\leq T_U\right)+f_i^2\e\It
\left(\abs{X_i}\leq T_U\right) \leq{}\\
{}\leq \sigma^2+f_i^2\e\It\left(\abs{X_i}\leq T_U\right)\,.
\end{multline*}

В силу конечности второго момента плот\-ности~$h(x)$ существует 
такая положительная константа~$C^{(h)}$, зависящая от~$h(x)$, что 
$f_i^2\e\It(\abs{X_i}\hm\leq T)\hm\leq C^{(h)}T^2$ при любом $T\hm>0$. Следовательно, 
существует такая константа $C_1\hm>0$, что
\begin{multline*}
\e\left(\rho_{T_U}^{(h)}\left(X_i\right)-f_i\right)^2\leq {}\\
{}\leq
\sigma^2+C^{(h)}T_U^2\leq C_1\left(\fr{\sigma^2}{N}+\sigma^2\right)
(\ln N)^{2/\beta}\,.
\end{multline*}
Пусть теперь $\abs{f_i}\hm\leq T_U$. Тогда
$$
\e\left(\rho_{T_U}^{(h)}\left(X_i\right)-f_i\right)^2\leq\e\left(X_i-f_i\right)^2\It
\left(\abs{X_i}> T_U\right)+f_i^2\,.
$$
Обозначим
$$
g\left(f_i\right)=\e\left(X_i-f_i\right)^2\It\left(\abs{X_i}> T_U\right)\,.
$$
Поскольку $g(f_i)$ симметрична относительно~0 и~$\abs{h'(x)}$ ограничена,
$$
g\left(f_i\right)\leq g(0)+\fr{1}{2}\left(\sup\limits_{f\in{\mathbf R}}
\abs{g''(f)}\right)f_i^2\,.
$$
В силу определения~$h(x)$ существует такая константа $C_0\hm>0$, что
$$
g(0)\leq C_0T_U^{3+\alpha-\beta}e^{-\theta T_U^{\beta}}\,.
$$
Следовательно, существует такая константа $C_2\hm>0$, что
$$
\e\left(\rho_{T_U}^{(h)}\left(X_i\right)-f_i\right)^2\leq 
C_2 \left(\!\fr{\sigma^2(\ln N)^{({3+\alpha-\beta})/{\beta}}}{N}+f_i^2\!\right).
$$
Таким образом, начиная с некоторого~$N$
\begin{multline*}
\e\left(\rho_{T_U}^{(h)}\left(X_i\right)-f_i\right)^2\leq {}\\
{}\leq
C (\ln N)^{\delta(\alpha,\beta)}\left(\fr{\sigma^2}{N}+\min(f_i^2,\sigma^2)\right)\,,
\end{multline*}
где $C$ зависит только от~$h(x)$. Суммируя по~$i$, получаем~\eqref{MSE_Hard}. 
Теорема доказана.

\smallskip

\noindent
\textbf{Теорема~2.}\ \textit{Существует такая константа $C\hm>0$, зависящая только 
от~$h(x)$, что начиная с некоторого~$N$ при мягкой пороговой обработке}
\begin{equation}
\label{MSE_Soft}
R_{T_U}\left(\hat{f}\right)\leq C(\ln N)^{\delta(\alpha,\beta)}\left(\sigma^2+R_{\mathrm{Min}}
\left(\hat{f}\right)\right)\,,
\end{equation}
\textit{где} $\delta(\alpha,\beta)\hm=\max({2}/{\beta},({3+\alpha-3\beta})/{\beta})$.

\smallskip

\noindent
Д\,о\,к\,а\,з\,а\,т\,е\,л\,ь\,с\,т\,в\,о\ \ 
этой теоремы аналогично доказательству теоремы~1. Основное отличие заключается в том, 
что при мягкой пороговой обработке в силу определения~$h(x)$ выполнено
\begin{multline*}
g(0)=\e\left(X_i-T_U\right)^2\It\left(X_i>T_U\right)+{}\\
{}+
\e\left(X_i+T_U\right)^2\It\left(X_i<-T_U\right)\leq 
C_0T_U^{3+\alpha-3\beta}e^{-\theta T_U^{\beta}}
\end{multline*}
с некоторой константой $C_0\hm>0$.

\smallskip

Таким образом, оценки~\eqref{MSE_Hard} и~\eqref{MSE_Soft} демонстрируют, 
что в рассматриваемой модели универсальный порог обеспечивает порядок 
среднеквадратичного риска, который отличается от минимального лишь 
наличием логарифмического множителя в степени, зависящей от 
характеристик распределения шума.

Установим теперь еще одно свойство порога~$T_U$, 
которое показывает, что, как и в случае гауссовского шума, универсальный порог 
является в некотором смысле максимальным среди разумных порогов.

\smallskip

\noindent
\textbf{Теорема~3.} \textit{Пусть случайные величины~$z_i$, $i\hm=1,\ldots,N$, 
независимы и имеют плотность распределения~$h(x)$, удовлетворяющую перечисленным 
выше условиям. Тогда}
\begin{equation}
\label{Prob_T}
{\sf P}\left(T^{-}_{U}\leq\max\limits_{1\leq i\leq N}
\abs{z_i}\leq T^{+}_{U}\right)\to 1 \mbox{ при } N\to\infty\,.
\end{equation}
Здесь
$$
T^{-}_{U}=\left(\fr{\ln N+\gamma'\ln\ln N}{\theta}\right)^{1/\beta}\,,
$$
\textit{где $\gamma'$~--- произвольное число, удовлетворяющее}
 $\gamma'\hm<\beta^{-1}(1\hm+\alpha\hm-\beta)$; 
\begin{equation*}
T^{+}_{U}=\begin{cases}
\left(\fr{\ln N+\gamma''\ln\ln N}{\theta}\right)^{\!\!1/\beta} &
\!\!\mbox{при } 1+\alpha-\beta\geq0\,; \\ 
T_U  & \!\!\mbox{при } 1+\alpha-\beta<0\,,
\end{cases}
\end{equation*}
\textit{где $\gamma''$~--- произвольное число, удовлетворяющее} $\gamma''\hm>
\beta^{-1}(1\hm+\alpha\hm-\beta)$.

\smallskip

\noindent
Д\,о\,к\,а\,з\,а\,т\,е\,л\,ь\,с\,т\,в\,о.\ \ 
Утверждение~\eqref{Prob_T} является простым следствием более общих утверждений 
о~распределениях экстремумов случайных последовательностей. Легко видеть, 
что $NH(T^{-}_{U})\hm\to\infty$ и $NH(T^{+}_{U})\hm\to 0$ при $N\hm\to\infty$, 
где $H(x)$~--- функция распределения, соответствующая плотности~$h(x)$. 
Следовательно, в силу теоремы~1.5.1 из~\cite{Lidb89} при $N\hm\to\infty$
\begin{align*}
{\sf P}\left(\max\limits_{1\leq i\leq N}\abs{z_i}\leq T^{-}_{U}\right)&\to 0\,;\\
{\sf P}\left(\max\limits_{1\leq i\leq N}\abs{z_i}\leq T^{+}_{U}\right)
&\to 1\,,
\end{align*}
т.\,е.\ выполнено~\eqref{Prob_T}. Теорема доказана.

\smallskip

Утверждение теоремы~3 означает, что макси\-маль\-ная амплитуда шума с~вероятностью, 
стре\-мящейся к~единице, находится в~некоторой\linebreak окрест\-ности~$T_U$. Следовательно, при 
достаточно большом~$N$ нет смысла выбирать порог, превосходящий~$T_U$.


{\small\frenchspacing
 {%\baselineskip=10.8pt
 \addcontentsline{toc}{section}{References}
 \begin{thebibliography}{9}
\bibitem{DonJ94}
\Au{Donoho D., Johnstone I.\,M.} Ideal spatial adaptation via wavelet shrinkage~// 
Bio\-met\-ri\-ka, 1994. Vol.~81. No.\,3. P.~425--455.


\bibitem{DJ98}
\Au{Donoho D., Johnstone~I.\,M.} Minimax estimation via wavelet shrinkage~// 
Ann. Stat., 1998. Vol.~26. No.\,3. P.~879--921.

\bibitem{Jan01}
\Au{Jansen M.} Noise reduction by wavelet thresholding.~--- 
Lecture notes in statistics ser.~--- 
New York, NY, USA: Springer Verlag, 2001.  Vol.~161. 217~p.

\bibitem{SH12}
\Au{Шестаков О.\,В.} Асимптотическая нормальность оценки риска пороговой обработки 
вейв\-лет-ко\-эф\-фи\-ци\-ен\-тов при выборе адаптивного порога~// Докл. 
РАН, 2012. Т.~445. №\,5. С.~513--515.

\bibitem{KS16-1} 
\Au{Кудрявцев А.\,А., Шестаков~О.\,В.} Асимптотическое поведение порога, 
минимизирующего усредненную вероятность ошибки вычисления вейв\-лет-ко\-эф\-фи\-ци\-ен\-тов~// 
Докл. РАН, 2016. Т.~468. №\,5. С.~487--491.

\bibitem{KS16-2}
\Au{Кудрявцев А.\,А., Шестаков~О.\,В.} 
Асимптотически оптимальная пороговая обработка вейв\-лет-ко\-эф\-фи\-ци\-ен\-тов 
в~моделях с негауссовым распределением шума~// Докл. РАН, 2016. Т.~471. №\,1. С.~11--15.

\bibitem{Lidb89}
\Au{Лидбеттер М., Линдгрен~Г., Ротсен~Х.} 
Экстремумы случайных последовательностей и процессов~/
Пер с~англ.~--- М.:~Мир, 1989. 392~с.
(\Au{Leadbetter~M., Lindgren~G., Rootzen~H.} 
{Extremes and related properties of random sequences and processes.}~--- 
New York, NY, USA: Springer-Verlag, 1983. 336~p.)

 \end{thebibliography}

 }
 }

\end{multicols}

\vspace*{-3pt}

\hfill{\small\textit{Поступила в~редакцию 01.03.17}}

%\vspace*{8pt}

\newpage

\vspace*{-24pt}

%\hrule

%\vspace*{2pt}

%\hrule

%\vspace*{8pt}


\def\tit{UNIVERSAL THRESHOLDING IN~THE~MODELS WITH~NON-GAUSSIAN NOISE}

\def\titkol{Universal thresholding in the models with non-Gaussian noise}

\def\aut{O.\,V.~Shestakov$^{1,2}$}

\def\autkol{O.\,V.~Shestakov}

\titel{\tit}{\aut}{\autkol}{\titkol}

\vspace*{-9pt}


\noindent
$^1$Department of Mathematical Statistics, Faculty of Computational Mathematics 
and Cybernetics, M.\,V.~Lo-\linebreak
$\hphantom{^1}$monosov Moscow State University, 1-52~Leninskiye Gory,
 GSP-1, Moscow 119991, Russian Federation
 
 \noindent
 $^2$Institute of Informatics Problems, Federal Research Center 
 ``Computer Science and Control'' of the Russian\linebreak
 $\hphantom{^1}$Academy of Sciences, 44-2~Vavilov Str., 
 Moscow 119333, Russian Federation



\def\leftfootline{\small{\textbf{\thepage}
\hfill INFORMATIKA I EE PRIMENENIYA~--- INFORMATICS AND
APPLICATIONS\ \ \ 2017\ \ \ volume~11\ \ \ issue\ 2}
}%
 \def\rightfootline{\small{INFORMATIKA I EE PRIMENENIYA~---
INFORMATICS AND APPLICATIONS\ \ \ 2017\ \ \ volume~11\ \ \ issue\ 2
\hfill \textbf{\thepage}}}

\vspace*{3pt}



\Abste{A common assumption in nonparametric signal estimation 
is that the signal function belongs to a certain class. For example, 
it may be piecewise continuous or piecewise differentiable and have 
a~compact support. These assumptions, as a~rule, make it possible to economically 
represent a~signal function in a~specially selected basis in such 
a~way that the useful signal is concentrated in a~relatively small number of 
large expansion coefficients. Then, threshold processing removes noisy coefficients. 
Typically, the noise distribution is assumed to be Gaussian. This model has been 
well studied in the literature and optimal thresholding parameters have been 
calculated for different classes of signal functions. The paper considers 
the problem of constructing an estimate for the signal function from the 
observations containing additive noise, whose distribution belongs to quite 
a~wide class. The authors calculate the values of universal thresholding 
parameters for which the mean-square risk is close to the minimum.}

\KWE{thresholding; non-Gaussian noise; mean-square risk} 
 
\DOI{10.14357/19922264170214} 

%\vspace*{-18pt}

%\Ack
%\noindent




%\vspace*{3pt}

  \begin{multicols}{2}

\renewcommand{\bibname}{\protect\rmfamily References}
%\renewcommand{\bibname}{\large\protect\rm References}

{\small\frenchspacing
 {%\baselineskip=10.8pt
 \addcontentsline{toc}{section}{References}
 \begin{thebibliography}{9}



\bibitem{DonJ94-1}
\Aue{Donoho, D., and I.\,M.~Johnstone}. 1994. Ideal spatial adaptation via wavelet 
shrinkage. \textit{Biometrika} 81(3):425--455.

\bibitem{DJ98-1}
\Aue{Donoho, D., and I.\,M.~Johnstone}. 1998. Minimax estimation via wavelet
 shrinkage \textit{Ann. Stat.}  26(3):879--921.

\bibitem{Jan01-1}
\Aue{Jansen, M.} 2001. \textit{Noise reduction by wavelet thresholding.} 
Lecture notes in statistics ser.
New York, NY: Springer Verlag.  Vol.~161. 217~p.

\bibitem{SH12-1}
\Aue{Shestakov, O.\,V.} 2012. Asymptotic normality of adaptive wavelet 
thresholding risk estimation. \textit{Dokl. Math.} 86(1):556--558.

\bibitem{KS16-1-1}  
\Aue{Kudryavtsev, A.\,A., and O.\,V.~Shestakov.} 2016. 
Asymptotic behavior of the threshold minimizing the average probability of error 
in calculation of wavelet coefficients. \textit{Dokl. Math.} 93(3):295--299.

\bibitem{KS16-2-1} 
\Aue{Kudryavtsev, A.\,A., and O.\,V.~Shestakov.} 2016. Asymptotically optimal 
wavelet thresholding in the models with non-Gaussian noise distributions. 
\textit{Dokl. Math.} 94(3):615--619.

\bibitem{Lidb89-1}
\Aue{Leadbetter, M., G.~Lindgren, and H.~Rootzen.} 1983. 
\textit{Extremes and related properties of random sequences and processes.} 
New York, NY: Springer-Verlag. 336~p.
\end{thebibliography}

 }
 }

\end{multicols}

\vspace*{-3pt}

\hfill{\small\textit{Received March 1, 2017}}

\Contrl

\noindent
\textbf{Shestakov Oleg V.} (b.\ 1976)~--- 
Doctor of Science in physics and mathematics, associate professor, 
Department of Mathematical Statistics, Faculty of Computational Mathematics 
and Cybernetics, M.\,V.~Lomonosov Moscow State University, 1-52~Leninskiye Gory,
 GSP-1, Moscow 119991, Russian Federation; senior scientist, 
 Institute of Informatics Problems, Federal Research Center 
 ``Computer Science and Control'' of the Russian Academy of Sciences, 44-2~Vavilov Str., 
 Moscow 119333, Russian Federation; \mbox{oshestakov@cs.msu.su}
\label{end\stat}


\renewcommand{\bibname}{\protect\rm Литература}   %



%%%%%%%%%%%%%%%%%%%%%%%%%%%%%%%%%%%%%%%%%%%%%%%

%\def\stat{rez}
{%\hrule\par
%\vskip 7pt % 7pt
\raggedleft\Large \bf%\baselineskip=3.2ex
Р\,Е\,Ц\,Е\,Н\,З\,И\,И \vskip 17pt
    \hrule
    \par
\vskip 6pt plus 6pt minus 3pt }

%\thispagestyle{headings} %с верхним колонтитулом
%\thispagestyle{myheadings} %с нижним колонтитулом, но в верхнем РЕЦЕНЗИИ

\def\tit{НОВАЯ КНИГА И.\,Н.~СИНИЦЫНА, А.\,С.~ШАЛАМОВА <<ЛЕКЦИИ ПО ТЕОРИИ 
ИНТЕГРИРОВАННОЙ ЛОГИСТИЧЕСКОЙ ПОДДЕРЖКИ>> (М.: ТОРУС ПРЕСС, 2012. 624~с.)}

%1
\def\aut{Д.ф.-м.н., профессор С.\,Я.~Шоргин}

\def\auf{\ }

\def\leftkol{\ % РЕЦЕНЗИИ
}

\def\rightkol{ \ } 

%\def\leftkol{\ } % ENGLISH ABSTRACTS}

%\def\rightkol{\ } %ENGLISH ABSTRACTS}

%\def\leftkol{РЕЦЕНЗИИ}

%\def\rightkol{РЕЦЕНЗИИ}

\titele{\tit}{\aut}{\auf}{\leftkol}{\rightkol}
\vspace*{-18pt}


     \label{st\stat}

     \begin{multicols}{2}
     {\small
     {\baselineskip=10.1pt
     

      В книге представлено системное изложение теоретических основ одного из новейших 
направлений в \mbox{об\-ласти} экономики послепродажного обслуживания изделий наукоемкой 
продукции (ИНП) длительного пользования~--- интегрированной логистической поддержки
(ИЛП). 
{\looseness=1

}

Приведены также результаты новых работ, выполненных в Институте проблем информатики 
Российской академии наук в рамках научного направления <<Информационные технологии и 
анализ сложных сис\-тем>>.
 {%\looseness=1

}
     
      Излагаемые в книге научные подходы позво\-ляют карди\-наль\-но реформировать 
существующие системы производства и эксплуатации ИНП путем создания и внед\-ре\-ния 
методов рационального и оптимального управ\-ле\-ния процессами расходования 
вре\-мен\-н$\acute{\mbox{ы}}$х, 
мате\-ри\-аль\-ных, трудовых и других ресурсов на всех стадиях жизненного цикла изделий (ЖЦИ) по 
критериям экономической целесообразности и эф\-фек\-тив\-ности.
  {\looseness=1

}
    
      В книге приведен краткий обзор причин возник\-новения и
      развития CALS-методологии как основы 
современных международных стандартов по созданию и функционированию глобальных 
ин\-фор\-ма\-ци\-он\-но-ком\-му\-ни\-ка\-ци\-он\-ных систем, ее ключевых возможностей и эффективности 
результатов ее использования. 
Авторы %\linebreak 
предлагают ряд научных обоснований для разработки 
единой теории проектирования и управления систем ИЛП для полноценного использования 
преимуществ %\linebreak
 суще\-ст\-ву\-ющей методологии, определяют \mbox{общую} структурную схему 
комплексной системы <<ИНП-СППО>> и необходимость разработки для ее описания 
гибридных стохастических моделей.
{%\looseness=1

}

%\columnbreak
      
      Книга состоит из пяти частей, где последовательно излагается материал по каждой из 
следующих тем: <<Интегрированная логистическая поддержка>>, <<Теория гибридных 
стохастических систем и компьютерная поддержка исследований и разработок>>, <<Основы 
математического моделирования, анализа и синтеза систем послепродажного обслуживания>>, 
<<Определение и анализ показателей экспортного потенциала ИНП при проектировании>>, 
<<Задачи управления поддержкой послепродажного обслуживания>>, а также 
<<Моделирование инвестиционных процессов ИЛП в условиях неравновесных финансовых 
рынков>>. 
   
      В конце каждой главы приведены выводы и даны вопросы и задания для 
самоконтроля. В~приложениях содержатся основные определения по программам работ по 
анализу ИЛП, логистическим базам данных и компьютерным решениям, эквивалентной статистической 
линеаризации нелинейных преобразований ИЛП, справочный материал, а также развернутые 
уравнения для вероятностных характеристик.


      \def\leftkol{РЕЦЕНЗИИ}

\def\rightkol{РЕЦЕНЗИИ} 

      
      Книга заинтересует широкий круг специалистов и может быть использована научными 
проектными организациями в сфере промышленного производства ИНП. Большое количество 
иллюстраций, примеров и вопросов, обращенных к читателю, позволяет использовать книгу 
также в качестве учебного пособия для студентов и аспирантов машиностроительных, 
транспортных и~других специальностей, а также для самостоятельного изучения. 
{%\looseness=-1

}

Книга 
представляет несомненный интерес для специалистов и студентов в области прикладной 
математики и информатики.
    

}

}
\end{multicols}

%\newpage

\def\stat{authorsrus}
{%\hrule\par
%\vskip 7pt % 7pt
\raggedleft\Large \bf%\baselineskip=3.2ex
О\,Б\ \ А\,В\,Т\,О\,Р\,А\,Х \vskip 17pt
    \hrule
    \par
\vskip 21pt plus 8pt minus 4pt }


\def\tit{\ }

\def\aut{\ }

\def\auf{\ }

\def\leftkol{\ } % ENGLISH ABSTRACTS}

\def\rightkol{ОБ АВТОРАХ} %ENGLISH ABSTRACTS}

\titele{\tit}{\aut}{\auf}{\leftkol}{\rightkol}
      
            \label{st\stat}



\vspace*{24pt}

\begin{multicols}{2}




\noindent
\textbf{Архипов Олег Петрович} (р.\ 1948)~---
кандидат технических наук, директор Орловского филиала Института проб\-лем информатики
Российской академии наук
%302025, г.Орел, Московское шоссе, д.137

\vspace*{3pt}

\noindent
\textbf{Бирюкова Татьяна Константиновна} (р.\ 1968)~---
кандидат фи\-зи\-ко-ма\-те\-ма\-ти\-че\-ских наук, старший научный сотрудник Института проб\-лем информатики
Российской академии наук

\vspace*{3pt}

\noindent 
\textbf{Бобков  Сергей Геннадьевич} (р.\ 1955)~---
доктор технических наук,  заведующий отделением На\-уч\-но-ис\-сле\-до\-ва\-тель\-ско\-го 
института системных исследований Российской академии наук
%117218, Москва, Нахимовский просп., 36, к.1 

\vspace*{3pt}

\noindent \textbf{Васильев Николай Семенович} (р.\ 1952)~--- доктор 
фи\-зи\-ко-ма\-те\-ма\-ти\-че\-ских наук, профессор, 
МГТУ им.\ Н.\,Э.~Баумана 
%, Москва 105005, 2-я Бауманская ул., д.~5,

\vspace*{3pt}

\noindent
\textbf{Гершкович Максим Михайлович} (р.\ 1968)~---
старший научный сотрудник Института проб\-лем информатики
Российской академии наук

\vspace*{3pt}

\noindent 
\textbf{Дьяченко Юрий Георгиевич} (р.\ 1958)~--- кандидат технических наук, 
старший научный сотрудник Института проб\-лем информатики
Российской академии наук

\vspace*{3pt}

\noindent 
\textbf{Ерошенко Александр Андреевич} (р.\ 1989)~--- аспирант кафедры 
математической статистики факультета вычисли\-тельной математики и кибернетики 
Московского государственного университета им.\ М.\,В.~Ломоносова
%119991, Москва ГСП-1, Ленинские горы, д.\ 1, стр. 52

\vspace*{3pt}
 
\noindent 
\textbf{Захаров Виктор Николаевич} (р.\ 1948)~--- 
доктор технических наук, доцент, ученый секретарь Института проб\-лем информатики
Российской академии наук

\vspace*{3pt}

\noindent
\textbf{Зейфман Александр Израилевич} (р.\ 1954)~---
доктор фи\-зи\-ко-ма\-те\-ма\-ти\-че\-ских наук, профессор, 
заведующий кафедрой Вологодского государственного университета; 
старший научный сотрудник Института проб\-лем информатики
Российской академии наук; главный научный сотрудник ИСЭРТ Российской академии наук

\vspace*{3pt}

\noindent
\textbf{Зыкин Сергей Владимирович} (р.\ 1959)~--- 
доктор технических наук, профессор, заведующий лабораторией Института математики 
им.\ С.\,Л.~Соболева Сибирского отделения Российской академии наук, Новосибирск 
%630090, пр.\ ак.\ Коптюга, 4 

\vspace*{4pt}

\noindent
\textbf{Киреев Владимир Иванович} (р.\ 1938)~---
доктор фи\-зи\-ко-ма\-те\-ма\-ти\-че\-ских наук, профессор Московского 
государственного горного университета
%Адрес: Россия, 119991, г. Москва, Ленинский проспект, д. 6

%\columnbreak

\vspace*{4pt}

\noindent
\textbf{Козеренко Елена Борисовна} (р.\ 1959)~---
кандидат филологических наук, заведующая лабораторией Института проб\-лем информатики
Российской академии наук

\vspace*{4pt}

\noindent
\textbf{Королев Виктор Юрьевич} (р.\ 1954)~--- доктор
фи\-зи\-ко-ма\-те\-ма\-ти\-че\-ских наук, профессор кафедры математической 
статистики факультета вычисли\-тельной математики и кибернетики 
Московского государственного университета; 
ведущий научный сотрудник Института проб\-лем информатики
Российской академии наук

\vspace*{4pt}

\noindent
\textbf{Коротышева Анна Владимировна} (р.\ 1988)~---
старший преподаватель Вологодского государственного университета

\vspace*{4pt}

\noindent 
\textbf{Кун Де Турк} (р.\ 1981)~--- научный сотрудник 
исследовательской группы SMACS факультета телекоммуникаций и обработки информации
Университета Гента, Бельгия
%В-9000 Гент, Бельгия

\vspace*{4pt}

\noindent
\textbf{Лупенцов Олег Сергеевич} (р.\ 1986)~---
аспирант Омского государственного института сервиса
%Омск 644043, ул.\ Певцова 13

\vspace*{4pt}

\noindent
\textbf{Лучко Олег Николаевич} (р.\ 1961)~---
кандидат педагогических наук, профессор, заведующий кафедрой 
Омского государственного института сервиса
%Омск 644043, ул.\ Певцова 13

\vspace*{4pt}

\noindent
\textbf{Малашенко Юрий Евгеньевич} (р.\ 1946)~---
доктор фи\-зи\-ко-ма\-те\-ма\-ти\-че\-ских наук, заведующий сектором 
Вычислительного центра им.\ А.\,А.~Дородницына Российской академии наук
%Адрес: 119333, Москва, ул. Вавилова, 40,

\vspace*{4pt}

\noindent
\textbf{Маньяков Юрий Анатольевич} (р.\ 1984)~---
кандидат технических наук, научный сотрудник Орловского филиала Института проб\-лем информатики
Российской академии наук
%302025, г.Орел, Московское шоссе, д.137

\vspace*{4pt}

\noindent
\textbf{Маренко Валентина Афанасьевна} (р.\ 1951)~---
кандидат технических наук, доцент, старший научный сотрудник 
Института математики им.\ С.\,Л.~Соболева Сибирского отделения Российской академии наук
%Новосибирск 630090, пр. ак. Коптюга, 4 

\vspace*{3pt}

\noindent 
\textbf{Морозов Евсей Викторович} (р.\ 1947)~--- доктор 
фи\-зи\-ко-ма\-те\-ма\-ти\-че\-ских, профессор, ведущий научный сотрудник 
Института прикладных математических исследований Карельского научного центра Российской
академии наук; 
%%185910 Россия, Республика Карелия, г.\ Петрозаводск, ул.\ Пушкинская, 11
профессор Петрозаводского государственного университета, Петрозаводск
%185910 Россия, Республика Карелия, г.\ Петрозаводск, пр.\ Ленина, 33

%\pagebreak

\vspace*{3pt}

\noindent
\textbf{Назарова Ирина Александровна} (р.\ 1966)~---
кандидат фи\-зи\-ко-ма\-те\-ма\-ти\-че\-ских наук, 
научный сотрудник Вычислительного центра им.\ А.\,А.~Дородницына Российской академии наук 
%Адрес: 119333, Москва, ул. Вавилова, 40

\vspace*{3pt}

\noindent
\textbf{Павлов Игорь Валерианович} (р.\ 1945)~--- 
доктор фи\-зи\-ко-ма\-те\-ма\-ти\-че\-ских наук, профессор МГТУ им.\ Н.\,Э.~Баумана 
%Москва 105005, 2-я Бауманская ул., д.~5 

%\pagebreak

\vspace*{3pt}

\noindent 
\textbf{Потахина Любовь Викторовна} (р.\ 1989)~--- аспирантка
Института прикладных математических исследований Карельского научного центра
Российской академии наук; 
%%185910 Россия, Республика Карелия, г.\ Петрозаводск, ул.\ Пушкинская, 11
инженер Петрозаводского государственного университета, Петрозаводск
%185910 Россия, Республика Карелия, г.\ Петрозаводск, пр.\ Ленина, 33

\vspace*{3pt}

\noindent 
\textbf{Рождественский Юрий Владимирович} (р.\ 1952)~--- 
кандидат технических наук, заведующий сектором Института проб\-лем информатики
Российской академии наук

\vspace*{3pt}

\noindent 
\textbf{Синицын Игорь Николаевич} (р.\ 1940)~--- доктор технических наук,
профессор, заслуженный деятель\linebreak\vspace*{-12pt}

\columnbreak

\noindent
 науки РФ, заведующий отделом Института проб\-лем информатики
Российской академии наук

\vspace*{7pt}


\noindent
\textbf{Сиротинин Денис Олегович} (р.\ 1984)~---
кандидат технических наук, научный сотрудник Орловского филиала Института проб\-лем информатики
Российской академии наук
%302025, г.Орел, Московское шоссе, д.137

\vspace*{7pt}

%\columnbreak

\noindent 
\textbf{Соколов  Игорь Анатольевич} (р.\ 1954)~--- академик (действительный член) Российской 
академии наук, доктор технических наук, директор Института проб\-лем информатики
Российской академии наук

\vspace*{7pt}

\noindent
\textbf{Степченков Юрий Афанасьевич} (р.\ 1951)~---
кандидат технических наук, заведующий отделом Института проб\-лем информатики
Российской академии наук

\vspace*{7pt}

\noindent
\textbf{Сурков Алексей Викторович} (р.\ 1978)~--- 
старший научный сотрудник На\-уч\-но-ис\-сле\-до\-ва\-тель\-ско\-го 
института системных исследований Российской академии наук
%117218, Москва, Нахимовский просп., 36, к.1 

\vspace*{7pt}

\noindent 
\textbf{Шестаков Олег Владимирович} (р.\ 1976)~--- доктор 
фи\-зи\-ко-ма\-те\-ма\-ти\-че\-ских, доцент кафедры математической статистики 
факультета вычисли\-тельной математики и кибернетики Московского 
государственного университета им.\ М.\,В.~Ломоносова; 
%119991, Москва ГСП-1, Ленинские горы, д.\ 1, стр. 52
старший научный сотрудник Института проб\-лем информатики
Российской академии наук
%, Москва 119333, ул. Вавилова, д.~44, корп.~2

\vspace*{7pt}

\noindent 
\textbf{Шоргин Сергей Яковлевич} (р.\ 1952.)~--- доктор
фи\-зи\-ко-ма\-те\-ма\-ти\-че\-ских наук, профессор, заместитель директора Института 
проб\-лем информатики Российской академии наук





%%%%%%%%%%%%%%%%%%%%%%%%%%%%%%%%%%%%%%%%%%%%%%%%%%%%%%%%%%%%%%%%%%%%%%%%%%%%%%%




%\def\rightkol{ОБ АВТОРАХ}
%\def\leftkol{ОБ АВТОРАХ}

 \label{end\stat}





%\def\leftfootline{\small{\textbf{\thepage}
%\hfill ИНФОРМАТИКА И ЕЁ ПРИМЕНЕНИЯ\ \ \ том~7\ \ \ выпуск~1\ \ \ 2013}
%}%
% \def\rightfootline{\small{ИНФОРМАТИКА И ЕЁ ПРИМЕНЕНИЯ\ \ \ том~7\ \ \ выпуск~1\ \ \ 2013
%\hfill \textbf{\thepage}}}


%\thispagestyle{myheadings}



\end{multicols}

\newpage

%\end{document}

%
\def\stat{rekl}
%\label{preobr}

%\def\tit{АКАДЕМИК ПУГАЧЁВ  ВЛАДИМИР СЕМЁНОВИЧ\\
%25.03.1911--25.03.1998}


%   \vspace*{-48pt}
%   \begin{center}\LARGE
%Академик Пугачёв  Владимир Семёнович\\ (25.03.1911--25.03.1998)
%   \end{center}

   %\vspace*{2.5mm}

   \begin{center}

{\prgsh\LARGE
ЮБИЛЕИ}

\end{center}
%\hrule

\vspace*{6pt}


   \vspace*{8mm}

   \thispagestyle{empty}


%\def\stat{emel}


\section*{К 70-летию заместителя директора ИПИ РАН,\\ члена редколлегии журнала
<<Информатика и её применения>>\\ доктора технических наук В.\,И.~Будзко}

\vspace*{18pt}




          \begin{multicols}{2}

%            \label{st\stat}

\begin{center}
\vspace*{1pt}
\mbox{%
\epsfxsize=78mm
\epsfbox{bud-1.eps}
}
\end{center}

\vspace*{12pt}

      14 августа 2014~г.\ исполнилось 70~лет за\-мес\-ти\-те\-лю директора ИПИ РАН по
научной работе доктору технических наук Владимиру Игоревичу Будзко.

      Владимир Игоревич Будзко родился в г.~Москве. Высшее образование получил на факультете
элект\-рон\-но-вы\-чис\-ли\-тель\-ных устройств в Московском
ин\-же\-нер\-но-фи\-зи\-че\-ском институте
(МИФИ), который он окончил в 1968~г., после чего был на\-прав\-лен для прохождения
службы в одну из войс\-ко\-вых частей, где прошел путь от инженера до первого заместителя
командира войсковой части.

      С приходом В.\,И.~Будзко в ИПИ РАН (2001~г.)\ в институте
сформировалось новое научное на\-прав\-ле\-ние теоретических исследований~--- <<Постро\-ение
ин\-фор\-ма\-ци\-он\-но-те\-ле\-ком\-му\-ни\-ка\-ци\-он\-ных\linebreak сис\-тем
высокой до\-ступ\-ности>>. В~рамках этого
направления выполнен широкий круг фундаментальных исследований по поиску подходов и
определению принципов построения средств обеспечения доступности, конфиденциальности
и целостности современных крупномасштабных
ин\-фор\-ма\-ци\-он\-но-те\-ле\-ком\-му\-ни\-ка\-ци\-он\-ных
сис\-тем (ИТС). Разработаны основные сис\-тем\-но-тех\-ни\-че\-ские принципы и базовые
архитектурные решения построения перспективных для условий России ИТС с
централизованной обработкой и хранением информации, сочетающих в себе свойства
высокой доступности, отказо- и катастрофоустойчивости, информационной защищенности.
Определены принципы, методы и математические основы рационального построения и
оптимизации средств восстановления функционирования центров обработки данных (ЦОД)
после возникновения отказов и катастроф, передачи и хранения данных, обеспечения
информационной безопасности при достижении минимальной совокупной стоимости
владения такими системами. Результаты нашли практическое воплощение при реализации
проектов в интересах ряда отечественных государственных и негосударственных
организаций, таких как Банк России (БР), Внешторгбанк, ОАО <<ГМК <<Норильский Никель>>,
<<Газпром>>, Минэкономразвития России, Правительство Москвы, а также ряд силовых
ведомств.

      Под руководством В.\,И.~Будзко начиная с 2001~г.\ выполнен комплекс
      на\-уч\-но-ис\-сле\-до\-ва\-тель\-ских и
      опыт\-но-кон\-ст\-рук\-тор\-ских работ (свыше 100~проектов),
направленных на развитие электронной информационной технологии БР.
Разработаны концепции развития ИТС БР сначала до 2008~г., а затем до 2013~г., которые
были приняты в качестве основы проведения технической политики. За реализацию проекта
<<Катастрофоустойчивая тер\-ри\-то\-ри\-аль\-но-рас\-пре\-де\-лен\-ная
      ин\-фор\-ма\-ци\-он\-но-те\-ле\-ком\-му\-ни\-ка\-ци\-он\-ная сис\-те\-ма централизованной
обработки банковской информации>> В.\,И.~Будзко удостоен Премии Правительства РФ в
области науки и техники за 2010~г.

      В.\,И.~Будзко возглавлял и возглавляет работы по ряду других прикладных проектов,
связанных с созданием, совершенствованием и развитием крупномасштабных ИТС.

      В.\,И.~Будзко~--- генерал-майор, доктор технических наук, член-кор\-рес\-пон\-дент
Академии криптографии РФ, известный ученый в области информатики и применения
информационных технологий при построении территориально распределенных ИТС
различного назначения. Является автором свыше 250~научных работ, опубликованных в
на\-уч\-но-тех\-ни\-че\-ских и специальных изданиях.

    \thispagestyle{empty}

      В.\,И.~Будзко уделяет большое внимание подготовке научных кадров. Под его
руководством защищено 6~диссертаций на соискание ученой степени кандидата
технических наук. Свыше 30~лет он читает лекции в ИКСИ Академии ФСБ, профессор
кафедры НИЯУ МИФИ. Является членом двух диссертационных советов, главным
редактором журнала <<Системы высокой доступности>> и членом редколлегии журнала
<<Информатика и её применения>>.

      \bigskip

      Редакционный совет и Редакционная коллегия журнала <<Информатика и её
применения>> сердечно поздравляют Владимира Игоревича Будзко с 70-ле\-ти\-ем и желают
крепкого здоровья и новых научных достижений.

\end{multicols}

\def\stat{cont}
{%\hrule\par
%\vskip 7pt % 7pt
\raggedleft\Large \bf%\baselineskip=3.2ex
А\,В\,Т\,О\,Р\,С\,К\,И\,Й\ \ У\,К\,А\,З\,А\,Т\,Е\,Л\,Ь\ \ З\,А\ \ 2\,0\,1\,0 г. \vskip 17pt
    \hrule
    \par
\vskip 21pt plus 6pt minus 3pt }

\label{st\stat}

\def\tit{\ }

\def\aut{\ }
\def\auf{\ }

\def\leftkol{\ } % ENGLISH ABSTRACTS}

\def\rightkol{\ } %АВТОРСКИЙ УКАЗАТЕЛЬ ЗА 2010 г.} %ENGLISH ABSTRACTS}

\titele{\tit}{\aut}{\auf}{\leftkol}{\rightkol}

\vspace*{-12pt}

{\tabcolsep=3pt
\begin{tabular}{p{388pt}rr}
&\textbf{Выпуск} & \textbf{Стр.}\\[6pt]
\hangindent=23pt\noindent\textbf{Арутюнян~А.\,Р.} Моделирование влияния деформаций отпечатков пальцев на 
точность\linebreak
\vspace*{-12pt}\\
\hspace*{23pt}дактилоскопической идентификации$\dotfill$&1&51\\
\hangindent=23pt\noindent\textbf{Архипов~О.\,П., Зыкова~З.\,П.} Интеграция гетерогенной информации о цветных 
пикселях\linebreak
\vspace*{-12pt}\\
\hspace*{23pt}и их цветовосприятии$\dotfill$&4&15\\
\hangindent=23pt\noindent\textbf{Баранов~С.\,И., Френкель~С.\,Л., Захаров~В.\,Н.} Полуформальная верификация 
цифрового устройства с конвейером, основанная на использовании алгоритмических машин\linebreak
\vspace*{-12pt}\\
\hspace*{23pt}состояния$\dotfill$&4&49\\
\textbf{Бекетова~И.\,В.} см.~Каратеев~С.\,Л.&&\\
\textbf{Белоусов~В.\,В.} см.~Синицын~И.\,Н.&&\\
\hangindent=23pt\noindent\textbf{Бенинг~В.\,Е., Королев~Р.\,А.} О предельном поведении мощностей критериев в 
случае\linebreak
\vspace*{-12pt}\\
\hspace*{23pt}распределения Лапласа$\dotfill$&2&63\\
\hangindent=23pt\noindent\textbf{Бенинг~В.\,Е., Сипина~А.\,В.} Асимптотическое разложение для мощности 
критерия,\linebreak
\vspace*{-12pt}\\
\hspace*{23pt}основанного на выборочной медиане, в случае распределения Лапласа$\dotfill$&1&18\\
\textbf{Бондаренко~А.\,В.} см.~Каратеев~С.\,Л.&&\\
\hangindent=23pt\noindent\textbf{Бородина~А.\,В., Морозов~Е.\,В.} Об оценивании асимптотики вероятности 
большого\linebreak
\vspace*{-12pt}\\
\hspace*{23pt}уклонения стационарной регенеративной очереди с одним прибором$\dotfill$&3&29\\
\hangindent=23pt\noindent\textbf{Бунтман~Н.\,В., Минель~Ж.-Л., Ле~Пезан~Д., Зацман~И.\,М.} Типология и 
компьютерное\linebreak
\vspace*{-12pt}\\
\hspace*{23pt}моделирование трудностей перевода$\dotfill$&3&77\\
\textbf{Визильтер~Ю.\,В.} см.~Каратеев~С.\,Л.&&\\
\hangindent=23pt\noindent\textbf{Гавриленко~С.\,В.} Оценки скорости сходимости распределений случайных сумм с 
безгранично делимыми индексами к нормальному закону$\dotfill$&4&81\\
\hangindent=23pt\noindent\textbf{Григорьева~М.\,Е., Шевцова~И.\,Г.} Уточнение неравенства 
Каца--Берри--Эссеена$\dotfill$&2&75\\
\hangindent=23pt\noindent\textbf{Грушо~А.\,А., Грушо~Н.\,А., Тимонина~Е.\,Е.} Поиск конфликтов в политиках 
безопасности: модель случайных графов$\dotfill$&3&38\\
\textbf{Грушо~Н.\,А.} см.~Грушо~А.\,А.&&\\
\hangindent=23pt\noindent\textbf{Гудков~В.\,Ю.} Математические модели изображения отпечатка пальца на основе 
описания линий$\dotfill$&1&58\\
\textbf{Гуртов~А.\,В.} см.~Лукьяненко~А.\,С.&&\\
\textbf{Желтов~С.\,Ю.} см.~Каратеев~С.\,Л.&&\\
\hangindent=23pt\noindent\textbf{Захаров~А.\,А., Серебряков~В.\,А.} Система управления электронной библиотекой 
LibMeta$\dotfill$&4&2\\
\textbf{Захаров~В.\,Н.} см.~Баранов~С.\,И.&&\\
\textbf{Захарова~Т.\,В.} см.~Матвеева~С.\,С.&&\\
\hangindent=23pt\noindent\textbf{Зацаринный~А.\,А., Чупраков~К.\,Г.} Некоторые аспекты выбора технологии для 
постро-\linebreak
\vspace*{-12pt}\\
\hspace*{23pt}ения систем отображения информации ситуационного центра$\dotfill$&3&59\\
\textbf{Зацман~И.\,М.} см.~Бунтман~Н.\,В.&&\\
\hangindent=23pt\noindent\textbf{Зейфман~А.\,И., Коротышева~А.\,В., Сатин~Я.\,А., Шоргин~С.\,Я.} Об 
устойчивости нестаци-\linebreak
\vspace*{-12pt}\\
\hspace*{23pt}онарных систем обслуживания с катастрофами$\dotfill$&3&9\\
\textbf{Зыкова~З.\,П.} см.~Архипов~О.\,П.&&\\
\hangindent=23pt\noindent\textbf{Илюшин~Г.\,Я., Соколов~И.\,А.} Организация управляемого доступа пользователей 
к\linebreak
\vspace*{-12pt}\\
\hspace*{23pt}разнородным ведомственным информационным ресурсам$\dotfill$&1&24\\
\hangindent=23pt\noindent\textbf{Кавагучи~Ю., Ульянов~В.\,В., Фуджикоши~Я.} Приближения для статистик, 
описывающих\linebreak
\vspace*{-12pt}\\
\hspace*{23pt}геометрические свойства данных большой размерности, с оценками 
ошибок$\dotfill$&1&12\\
\hangindent=23pt\noindent\textbf{Каратеев~С.\,Л., Бекетова~И.\,В., Ососков~М.\,В., Князь~В.\,А., 
Визильтер~Ю.\,В., Бондаренко~А.\,В., Желтов~С.\,Ю.} Автоматизированный контроль 
качества цифровых\linebreak
\vspace*{-12pt}\\
\hspace*{23pt}изображений для персональных документов$\dotfill$&1&65\\
\end{tabular}
}

\pagebreak

\def\leftkol{АВТОРСКИЙ УКАЗАТЕЛЬ ЗА 2010 г.} % ENGLISH ABSTRACTS}

\def\rightkol{АВТОРСКИЙ УКАЗАТЕЛЬ ЗА 2010 г.} %ENGLISH ABSTRACTS}

{\tabcolsep=3pt
\begin{tabular}{p{388pt}rr}
&\textbf{Выпуск} & \textbf{Стр.}\\[3pt]
\hangindent=23pt\noindent\textbf{Козеренко~Е.\,Б.} Лингвистические фильтры в статистических моделях машинного\linebreak
\vspace*{-12pt}\\
\hspace*{23pt}перевода$\dotfill$&2&83\\
\hangindent=23pt\noindent\textbf{Козеренко~Е.\,Б., Кузнецов~И.\,П.} Когнитивно-лингвистические представления в 
систе-\linebreak
\vspace*{-12pt}\\
\hspace*{23pt}мах обработки текстов$\dotfill$&3&69\\
\textbf{Князь~В.\,А.} см.~Каратеев~С.\,Л.&&\\
\hangindent=23pt\noindent\textbf{Колесников~А.\,В., Солдатов~С.\,А.} Алгоритм координации для гибридной 
интеллектуальной системы решения сложной задачи оперативно-производственного\linebreak
\vspace*{-12pt}\\
\hspace*{23pt}планирования$\dotfill$&4&61\\
\hangindent=23pt\noindent\textbf{Коновалов~М.\,Г.} О планировании потоков в системах вычислительных 
ресурсов$\dotfill$&2&3\\
\textbf{Конушин~А.\,С.} см.~Конушин~В.\,С.&&\\
\hangindent=23pt\noindent\textbf{Конушин~В.\,С., Кривовязь~Г.\,Р., Конушин~А.\,С.} Алгоритм распознавания людей 
в видео-\linebreak
\vspace*{-12pt}\\
\hspace*{23pt}последовательности по одежде$\dotfill$&1&74\\
\textbf{Корепанов~Э.\, Р.} см.~Синицын~И.\,Н.&&\\
\textbf{Королев~В.\,Ю.} см.~Соколов~И.\,А.&&\\
\textbf{Королев~Р.\,А.} см.~Бенинг~В.\,Е.&&\\
\textbf{Коротышева~А.\,В.} см.~Зейфман~А.\,И.&&\\
\hangindent=23pt\noindent\textbf{Кривенко~М.\,П.} Непараметрическое оценивание элементов байесовского 
клас\-си-\linebreak
\vspace*{-12pt}\\
\hspace*{23pt}фикатора$\dotfill$&2&13\\
\textbf{Кривовязь~Г.\,Р.} см.~Конушин~В.\,С.&&\\
\textbf{Крылов~А.\,С.} см.~Павельева~Е.\,А.&&\\
\hangindent=23pt\noindent\textbf{Крылов~В.\,А.} Моделирование и классификация многоканальных дистанционных\linebreak
\vspace*{-12pt}\\
\hspace*{23pt}изображений с использованием копул$\dotfill$&4&34\\
\hangindent=23pt\noindent\textbf{Крючин~О.\,В.} Разработка параллельных эвристических алгоритмов подбора 
весовых\linebreak
\vspace*{-12pt}\\
\hspace*{23pt}коэффициентов искусственной нейтронной сети$\dotfill$&2&53\\
\hangindent=23pt\noindent\textbf{Кудрявцев~А.\,А., Шоргин~С.\,Я.} Байесовские модели массового обслуживания и 
надеж-\linebreak
\vspace*{-12pt}\\
\hspace*{23pt}ности: характеристики среднего числа заявок в системе $M\vert M \vert 1\vert 
\infty$$\dotfill$&3&16\\
\hangindent=23pt\noindent\textbf{Кузнецов~А.\,А.} Связь между временными и структурно-топологическими 
характери-\linebreak
\vspace*{-12pt}\\
\hspace*{23pt}стиками диаграмм ритма сердца здоровых людей$\dotfill$&4&39\\
\textbf{Кузнецов~И.\,П.} см.~Козеренко~Е.\,Б.&&\\
\textbf{Ле~Пезан~Д.} см.~Бунтман~Н.\,В.&&\\
\hangindent=23pt\noindent\textbf{Лукьяненко~А.\,С., Морозов~Е.\,В., Гуртов~А.\,В.} Анализ сетевого протокола с общей 
функ-\linebreak
\vspace*{-12pt}\\
\hspace*{23pt}цией расширения окна передачи сообщения при конфликтах$\dotfill$&2&46\\
\hangindent=23pt\noindent\textbf{Лямин~О.\,О.} О предельном поведении мощностей критериев в случае обобщенного\linebreak
\vspace*{-12pt}\\
\hspace*{23pt}распределения Лапласа$\dotfill$&3&47\\
\hangindent=23pt\noindent\textbf{Маркин~А.\,В., Шестаков~О.\,В.} Асимптотики оценки риска при пороговой 
обработке\linebreak
\vspace*{-12pt}\\
\hspace*{23pt}вейвлет-вейглет коэффициентов в задаче томографии$\dotfill$&2&36\\
\hangindent=23pt\noindent\textbf{Матвеева~С.\,С., Захарова~Т.\,В.} Сети массового обслуживания с наименьшей 
длиной\linebreak
\vspace*{-12pt}\\
\hspace*{23pt}очереди$\dotfill$&3&22\\
\hangindent=23pt\noindent\textbf{Матюшенко~С.\,И.} Стационарные характеристики двухканальной системы 
обслужива-\linebreak
\vspace*{-12pt}\\
\hspace*{23pt}ния с переупорядочиванием заявок и распределениями фазового типа$\dotfill$&4&68\\
\textbf{Минель~Ж.-Л.} см.~Бунтман~Н.\,В.&&\\
\textbf{Морозов~Е.\,В.} см.~Бородина~А.\,В.&&\\
\textbf{Морозов~Е.\,В.} см.~Лукьяненко~А.\,С.&&\\
\textbf{Ососков~М.\,В.} см.~Каратеев~С.\,Л.&&\\
\hangindent=23pt\noindent\textbf{Павельева~Е.\,А., Крылов~А.\,С.} Поиск и анализ ключевых точек радужной 
оболочки\linebreak
\vspace*{-12pt}\\
\hspace*{23pt}глаза методом преобразования Эрмита$\dotfill$&1&79\\
\textbf{Печинкин~А.\,В.} см.~Френкель~С.\,Л.,&&\\
\hangindent=23pt\noindent\textbf{Протасов~В.\,И.} Составление субъективного портрета с использованием 
эволюционно-\linebreak
\vspace*{-12pt}\\
\hspace*{23pt}го морфинга и квалиметрия метода$\dotfill$&1&83\\
\hangindent=23pt\noindent\textbf{Рудаков~К.\,В., Торшин~И.\,Ю.} Вопросы разрешимости задачи распознавания 
вторичной\linebreak
\vspace*{-12pt}\\
\hspace*{23pt}структуры белка$\dotfill$&2&25\\
\textbf{Сатин~Я.\,А.} см.~Зейфман~А.\,И.&&\\
\hangindent=23pt\noindent\textbf{Сейфуль-Мулюков~Р.\,Б.} Нефть как носитель информации о своем 
происхождении,\linebreak
\vspace*{-12pt}\\
\hspace*{23pt}структуре и эволюции$\dotfill$&1&41\\
\end{tabular}
}

{\tabcolsep=3pt
\begin{tabular}{p{388pt}rr}
&\textbf{Выпуск} & \textbf{Стр.}\\[6pt]
\textbf{Семендяев~Н.\,Н.} см.~Синицын~И.\,Н.&&\\
\textbf{Серебряков~В.\,А.} см.~Захаров~А.\,А.&&\\
\textbf{Синицын~В.\,И.} см.~Синицын~И.\,Н.&&\\
\hangindent=23pt\noindent\textbf{Синицын~И.\,Н., Синицын~В.\,И., Корепанов~Э.\, Р., Белоусов~В.\,В., 
Семендяев~Н.\,Н.} Оперативное построение информационных моделей движения полюса 
Земли\linebreak
\vspace*{-12pt}\\
\hspace*{23pt}методами линейных и линеаризованных фильтров$\dotfill$&1&2\\
\textbf{Сипина~А.\,В.} см.~Бенинг~В.\,Е.&&\\
\hangindent=23pt\noindent\textbf{Соколов~И.\,А.} О работах заслуженного деятеля науки Российской Федерации 
И.\,Н.~Синицына в области информационных технологий и автоматизации (к 70-летию\linebreak
\vspace*{-12pt}\\
\hspace*{23pt}со дня рождения)$\dotfill$&3&84\\
\textbf{Соколов~И.\,А.} см.~Илюшин~Г.\,Я.&&\\
\hangindent=23pt\noindent\textbf{Соколов~И.\,А., Королев~В.\,Ю.} Предисловие$\dotfill$&2&2\\
\textbf{Солдатов~С.\,А.} см.~Колесников~А.\,В.&&\\
\hangindent=23pt\noindent\textbf{Степанов~С.\,Ю.} Использование координатного метода фрагментации 
коммутаторной\linebreak
\vspace*{-12pt}\\
\hspace*{23pt}нейронной сети для сокращения трафика$\dotfill$&2&57\\
\textbf{Тимонина~Е.\,Е.} см.~Грушо~А.\,А.&&\\
\textbf{Торшин~И.\,Ю.} см.~Рудаков~К.\,В.&&\\
\textbf{Ульянов~В.\,В.} см.~Кавагучи~Ю.&&\\
\textbf{Фазекаш~И.} см.~Чупрунов~А.\,Н.&&\\
\textbf{Френкель~С.\,Л.} см.~Баранов~С.\,И.&&\\
\hangindent=23pt\noindent\textbf{Френкель~С.\,Л., Печинкин~А.\,В.} Оценка времени самовосстановления в 
цифровых\linebreak
\vspace*{-12pt}\\
\hspace*{23pt}системах после сбоев, вызываемых переходными помехами$\dotfill$&3&2\\
\textbf{Фуджикоши~Я.} см.~Кавагучи~Ю.&&\\
\hangindent=23pt\noindent\textbf{Цискаридзе~А.\,К.} Математическая модель и метод восстановления позы человека 
по\linebreak
\vspace*{-12pt}\\
\hspace*{23pt}стереопаре силуэтных изображений$\dotfill$&4&27\\
\hangindent=23pt\noindent\textbf{Чупраков~К.\,Г.} К вопросу о размещении коллективных средств отображения в 
ситуа-\linebreak
\vspace*{-12pt}\\
\hspace*{23pt}ционном зале с заданными параметрами$\dotfill$&4&89\\
\textbf{Чупраков~К.\,Г.} см.~Зацаринный~А.\,А.&&\\
\hangindent=23pt\noindent\textbf{Чупрунов~А.\,Н., Фазекаш~И.} Законы повторного логарифма для числа 
безошибочных\linebreak
\vspace*{-12pt}\\
\hspace*{23pt}блоков при помехоустойчивом кодировании$\dotfill$&3&42\\
\textbf{Шевцова~И.\,Г.} см.~Григорьева~М.\,Е.&&\\
\hangindent=23pt\noindent\textbf{Шестаков~О.\,В.} Аппроксимация распределения оценки риска пороговой 
обработки вейвлет-коэффициентов нормальным распределением при использовании 
выбо-\linebreak
\vspace*{-12pt}\\
\hspace*{23pt}рочной дисперсии$\dotfill$&4&73\\
\textbf{Шестаков~О.\,В.} см.~Маркин~А.\,В.&&\\
\textbf{Шоргин~С.\,Я.} см.~Зейфман~А.\,И.&&\\
\textbf{Шоргин~С.\,Я.} см.~Кудрявцев~А.\,А.&&\\
\end{tabular}
}

%\thispagestyle{myheadings}
\def\leftfootline{\small{\textbf{\thepage}
\hfill ИНФОРМАТИКА И ЕЁ ПРИМЕНЕНИЯ\ \ \ том~4\ \ \ выпуск~4\ \ \ 2010}
}%
 \def\rightfootline{\small{ИНФОРМАТИКА И ЕЁ ПРИМЕНЕНИЯ\ \ \ том~4\ \ \ выпуск~4\ \ \ 2010
 \hfill \textbf{\thepage}}}
 \label{end\stat}





%Том 10 Выпуск 1-4 Год 2016

\def\stat{cont-e}
{%\hrule\par
%\vskip 7pt % 7pt
\raggedleft\Large \bf%\baselineskip=3.2ex
2\,0\,1\,6\ \ A\,U\,T\,H\,O\,R\ \ I\,N\,D\,E\,X \vskip 17pt
 \hrule
 \par
\vskip 21pt plus 6pt minus 3pt }

\label{st\stat}

\def\tit{\ }

\def\aut{\ }
\def\auf{\ }

\def\leftkol{\ } %2016 AUTHOR INDEX} % ENGLISH ABSTRACTS}

\def\rightkol{\ } %2016 AUTHOR INDEX} %ENGLISH ABSTRACTS}

\titele{\tit}{\aut}{\auf}{\leftkol}{\rightkol}

\def\leftfootline{\small{\textbf{\thepage}
\hfill INFORMATIKA I EE PRIMENENIYA~--- INFORMATICS AND APPLICATIONS\ \ \ 2016\
\ \ volume~10\ \ \ issue\ 4}
}%
 \def\rightfootline{\small{INFORMATIKA I EE PRIMENENIYA~--- INFORMATICS AND APPLICATIONS\ \ \ 2016\ \ \ volume~10\ \ \ issue\ 4
\hfill \textbf{\thepage}}}

\vspace*{-12pt}
\vspace*{-18pt}

{\tabcolsep=2.8pt
\begin{tabular}{p{382pt}cc}
&\textbf{Issue} & \textbf{Page}\\[6pt]
\Avtors{Agalarov~M.\,Ya.} see~Agalarov~Ya.\,M.&&\\
\Avtors{Agalarov~Ya.\,M., Agalarov~M.\,Ya., and
Shorgin~V.\,S.} About the optimal threshold of queue\linebreak
\\[-12pt]
\hspace*{23pt}length in a~particular problem of profit maximization
in the $M/G/1$ queuing system&2&70--79\\
\Avtors{Alexeyevsky~D.\,A.} BioNLP ontology extraction from 
a~restricted language corpus with\linebreak
\\[-12pt]
\hspace*{23pt}context-free grammars&1&119--128\\
\Avtors{Andreev~S.\,D.} see~Gaidamaka~Yu.\,V.&&\\
\Avtors{Andreev~S.\,D.} see~Ometov~A.\,Ya.&&\\
\Avtors{Arkhipov~O.\,P., Arkhipov~P.\,O., and Sidorkin~I.\,I.} The
option to create a~local coordinate\linebreak
\\[-12pt]
\hspace*{23pt}system for synchronization of selected images&3&91--97\\
\Avtors{Arkhipov~P.\,O.} see~Arkhipov~O.\,P.&&\\
\Avtors{Belousov~V.\,V.} see~Shnurkov~P.\,V.&&\\
\Avtors{Belousov~V.\,V.} see~Shnurkov~P.\,V.&&\\
\Avtors{Bening~V.\,E.} Calculation of~the~asymptotic deficiency
of~some statistical procedures based\linebreak
\\[-12pt]
\hspace*{23pt}on~samples with~random sizes&4&34--45\\
\Avtors{Borisov~A.\,V., Bosov~A.\,V., and Miller~G.\,B.} Modeling and
monitoring of VoIP connection&2&\hphantom{1}2--13\\
\Avtors{Bosov~A.\,V.} see~Borisov~A.\,V.&&\\
\Avtors{Briukhov~D.\,O.} see~Stupnikov~S.\,A.&&\\
\Avtors{Callaos~N.\,K.\ and Seyful-Mulyukov~R.\,B.} Complexity and
its information content&1&129--139\\
\Avtors{Chertok~A.\,V., Kadaner~A.\,I., Khazeeva~G.\,T., and
Sokolov~I.\,A.} Regime switching detection\linebreak
\\[-12pt]
\hspace*{23pt}for~the~Levy driven
Ornstein--Uhlenbeck process using CUSUM methods&4&46--56\\
\Avtors{Chichagov~V.\,V.} Asymptotic expansions of mean absolute
error of uniformly minimum variance unbiased and maximum likelihood
estimators on the one-parameter exponential\linebreak
\\[-12pt]
\hspace*{23pt}family model of lattice distributions&3&66--76\\
\Avtors{Danishevsky~V.\,I.} see~Kolesnikov A.\,V.&&\\
\Avtors{Fazliev~A.\,Z.} see~Kalinichenko~L.\,A.&&\\
\Avtors{Fedoseev~A.\,A.} What is behind the concept of ``knowledge in
small packages''&3&105--110\\
\Avtors{Gaidamaka~Yu.\,V., Andreev~S.\,D., Sopin~E.\,S.,
Samouylov~K.\,E., and Shorgin~S.\,Ya.} Interference analysis
of~the~device-to-device communications model with~regard to~a~signal\linebreak
\\[-12pt]
\hspace*{23pt}propagation environment&4&\hphantom{1}2--10\\
\Avtors{Gasilov~A.\,V.} see~Yakovlev~O.\,A.&&\\
\Avtors{Goncharov~A.\,V.\ and Strijov~V.\,V.} Metric time series
classification using weighted dynamic\linebreak
\\[-12pt]
\hspace*{23pt}warping relative to centroids of classes&2&36--47\\
\Avtors{Gordov~E.\,P.} see~Kalinichenko~L.\,A.&&\\
\Avtors{Gorshenin~A.\,K.} Concept of online service for stochastic
modeling of real processes&1&72--81\\
\Avtors{Gorshenin~A.\,K.} see~Shnurkov~P.\,V.&&\\
\Avtors{Gorshenin~A.\,K.} see~Shnurkov~P.\,V.&&\\
\Avtors{Grusho~A.\,A., Grusho~N.\,A., Zabezhailo~M.\,I., and
Timonina~E.\,E.} Integration of statistical and\linebreak
\\[-12pt]
\hspace*{23pt}deterministic methods for
analysis of information security&3&2--8\\
\Avtors{Grusho~A.\,A., Zabezhailo~M.\,I., and Zatsarinny~A.\,A.} On
the advanced procedure to reduce\linebreak
\\[-12pt]
\hspace*{23pt}calculation of Galois closures&4&\hphantom{1}96--104\\
\Avtors{Grusho~N.\,A.} see~Grusho~A.\,A.&&\\
\Avtors{Havanskov~V.\,A.} see~Minin~V.\,A.&&\\
\Avtors{Inkova~O.\,Yu.} see~Zatsman~I.\,M.&&\\
\Avtors{Isachenko~R.\,V.\ and Strijov~V.\,V.} Metric learning in
multiclass time series classification\linebreak
\\[-12pt]
\hspace*{23pt}problem&2&48--57\\
\end{tabular}
}
\pagebreak

\def\leftfootline{\small{\textbf{\thepage}
\hfill INFORMATIKA I EE PRIMENENIYA~--- INFORMATICS AND APPLICATIONS\ \ \ 2016\
\ \ volume~10\ \ \ issue\ 4}
}%
 \def\rightfootline{\small{INFORMATIKA I EE PRIMENENIYA~---
INFORMATICS AND APPLICATIONS\ \ \ 2016\ \ \ volume~10\ \ \ issue\ 4
\hfill \textbf{\thepage}}}

\def\leftkol{2016 AUTHOR INDEX} % ENGLISH ABSTRACTS}

\def\rightkol{2016 AUTHOR INDEX} %ENGLISH ABSTRACTS}


{\tabcolsep=2.83pt
\begin{tabular}{p{382pt}cc}
&\textbf{Issue} & \textbf{Page}\\[6pt]
\Avtors{Kadaner~A.\,I.} see~Chertok~A.\,V.&&\\[.255pt]
\Avtors{Kalinichenko~L.\,A., Volnova~A.\,A., Gordov~E.\,P.,
Kiselyova~N.\,N., Kovaleva~D.\,A., Malkov~O.\,Yu., Okladnikov~I.\,G.,
Podkolodnyy~N.\,L., Pozanenko~A.\,S., Ponomareva~N.\,V.,
Stupnikov~S.\,A.,} \textbf{and Fazliev~A.\,Z.} Data access challenges for data
intensive\linebreak
\\[-12pt]
\hspace*{23pt}research in Russia&1& 2--22\\[.255pt]
\Avtors{Karasikov~M.\,E.\ and Strijov~V.\,V.} Feature-based
time-series classification&4&121--131\\[.255pt]
\Avtors{Khazeeva~G.\,T.} see~Chertok~A.\,V.&&\\[.255pt]
\Avtors{Khokhlov~Yu.\,S.} Multivariate fractional Levy motion and its
applications&2&\hphantom{1}98--106\\[.255pt]
\Avtors{Kirikov~I.\,A., Kolesnikov~A.\,V., Listopad~S.\,V., and
Rumovskaya~S.\,B.} Fine-grained hybrid\linebreak
\\[-12pt]
\hspace*{23pt}intelligent systems. Part 2:
Bidirectional hybridization&1&\hphantom{1}96--105\\[.255pt]
\Avtors{Kirikov~I.\,A., Kolesnikov~A.\,V., Listopad~S.\,V., and
Rumovskaya~S.\,B.} ``Virtual council''~---\linebreak
\\[-12pt]
\hspace*{23pt}source environment
supporting complex diagnostic decision making&3&81--90\\[.255pt]
\Avtors{Kiselyova~N.\,N.} see~Kalinichenko~L.\,A.&&\\[.255pt]
\Avtors{Kolesnikov A.\,V., Listopad~S.\,V., Rumovskaya~S.\,B., and
Danishevsky~V.\,I.} Informal axiomatic\linebreak
\\[-12pt]
\hspace*{23pt}theory of~the~role visual models&4&114--120\\[.255pt]
\Avtors{Kolesnikov~A.\,V.} see~Kirikov~I.\,A.&&\\[.255pt]
\Avtors{Kolesnikov~A.\,V.} see~Kirikov~I.\,A.&&\\[.255pt]
\Avtors{Kolin~K.\,K.} Humanitarian aspects of information
security&3&111--121\\[.255pt]
\Avtors{Konovalov~M.\,G.\ and Razumchik~R.\,V.} Dispatching
to~two parallel nonobservable queues using\linebreak
\\[-12pt]
\hspace*{23pt}only static
information&4&57--67\\[.255pt]
\Avtors{Korchagin~A.\,Yu.} see~Korolev~V.\,Yu.&&\\[.255pt]
\Avtors{Korchagin~A.\,Yu.} see~Korolev~V.\,Yu.&&\\[.255pt]
\Avtors{Korepanov~E.\,R.} see~Sinitsyn~I.\,N.&&\\[.255pt]
\Avtors{Korepanov~E.\,R.} see~Sinitsyn~I.\,N.&&\\[.255pt]
\Avtors{Korolev~V.\,Yu., Korchagin~A.\,Yu., and Zeifman~A.\,I.} The
Poisson theorem for Bernoulli trials\linebreak
\\[-12pt]
\hspace*{23pt}with~a~random probability
of~success and~a~discrete analog of~the~Weibull distribution&4&11--20\\[.255pt]
\Avtors{Korolev~V.\,Yu., Zeifman~A.\,I., and Korchagin~A.\,Yu.}
Asymmetric Linnik distributions as~limit\linebreak
\\[-12pt]
\hspace*{23pt}laws for~random sums
of~independent random variables with~finite variances&4&21--33\\[.255pt]
\Avtors{Koucheryavy~E.\,A.} see~Ometov~A.\,Ya.&&\\[.255pt]
\Avtors{Kovaleva~D.\,A.} see~Kalinichenko~L.\,A.&&\\[.255pt]
\Avtors{Kovalyov~S.\,P.} Metaprogramming to increase
manufacturability of large-scale software-\linebreak
\\[-12pt]
\hspace*{23pt}intensive systems&1&56--66\\[.255pt]
\Avtors{Krivenko~M.\,P.} Significance tests of feature selection for
classification&3&32--40\\[.255pt]
\Avtors{Kruzhkov~M.\,G.} see~Zalizniak~Anna~A.&&\\[.255pt]
\Avtors{Kruzhkov~M.\,G.} see~Zatsman~I.\,M.&&\\[.255pt]
\Avtors{Kudryavtsev~A.\,A.} Bayesian queueing and reliability models:
\textit{A~priori} distributions with\linebreak
\\[-12pt]
\hspace*{23pt}compact support&1&67--71\\[.255pt]
\Avtors{Kudryavtsev~A.\,A.} Characteristics dependent on the balance
coefficient in Bayesian models\linebreak
\\[-12pt]
\hspace*{23pt}with compact support of \textit{a priori}
distributions&3&77--80\\[.255pt]
\Avtors{Kudryavtsev~A.\,A.\ and Palionnaia~S.\,I.} Bayesian recurrent
model of reliability growth:\linebreak
\\[-12pt]
\hspace*{23pt}Parabolic distribution of parameters&2&80--83\\[.255pt]
\Avtors{Kudryavtsev~A.\,A.\ and Titova~A.\,I.} Bayesian queuing
and~reliability models: Degenerate-\linebreak
\\[-12pt]
\hspace*{23pt}Weibull case&4&68--71\\[.255pt]
\Avtors{Leontyev~N.\,D.\ and Ushakov~V.\,G.} Analysis of a queueing
system with autoregressive arrivals\linebreak
\\[-12pt]
\hspace*{23pt}and nonpreemptive priority&3&15--22\\[.255pt]
\Avtors{Listopad~S.\,V.} see~Kirikov~I.\,A.&&\\[.255pt]
\Avtors{Listopad~S.\,V.} see~Kirikov~I.\,A.&&\\[.255pt]
\Avtors{Listopad~S.\,V.} see~Kolesnikov A.\,V.&&\\[.255pt]
\Avtors{Malkov~O.\,Yu.} see~Kalinichenko~L.\,A.&&\\[.255pt]
\Avtors{Markov~A.\,S., Monakhov~M.\,M., and
Ulyanov~V.\,V.} Generalized Cornish--Fisher expansions\linebreak
\\[-12pt]
\hspace*{23pt}for distributions of statistics based on samples
of random size&2&84--91\\[.255pt]
\Avtors{Melnikov~A.\,K.\ and Ronzhin~A.\,F.} Generalized statistical
method of~text analysis based\linebreak
\\[-12pt]
\hspace*{23pt}on~calculation of~probability distributions
of~statistical values&4&89--95\\
\end{tabular}
}
\pagebreak

\def\leftfootline{\small{\textbf{\thepage}
\hfill INFORMATIKA I EE PRIMENENIYA~--- INFORMATICS AND APPLICATIONS\ \ \ 2016\
\ \ volume~10\ \ \ issue\ 4}
}%
 \def\rightfootline{\small{INFORMATIKA I EE PRIMENENIYA~---
INFORMATICS AND APPLICATIONS\ \ \ 2016\ \ \ volume~10\ \ \ issue\ 4
\hfill \textbf{\thepage}}}

\def\leftkol{2016 AUTHOR INDEX} % ENGLISH ABSTRACTS}

\def\rightkol{2016 AUTHOR INDEX} %ENGLISH ABSTRACTS}


{\tabcolsep=3pt
\begin{tabular}{p{381pt}cc}
&\textbf{Issue} & \textbf{Page}\\[6pt]
\Avtors{Meykhanadzhyan~L.\,A.} Stationary characteristics of the finite
capacity queueing system with\linebreak
\\[-12pt]
\hspace*{23pt}inverse service order and generalized
probabilistic priority&2&123--131\\[.23pt]
\Avtors{Miller~G.\,B.} see~Borisov~A.\,V.&&\\[.23pt]
\Avtors{Minin~V.\,A., Zatsman~I.\,M., Havanskov~V.\,A., and
Shubnikov~S.\,K.} Intensity of citation of scientific publications in
inventions on information and computer technologies patented\linebreak
\\[-12pt]
\hspace*{23pt}in Russia by domestic and foreign applicants&2&107--122\\[.23pt]
\Avtors{Monakhov~M.\,M.} see~Markov~A.\,S.&&\\[.23pt]
\Avtors{Naumov~V.\,A.\ and Samouylov~K.\,E.} On relationship
between queuing systems with resources\linebreak
\\[-12pt]
\hspace*{23pt}and Erlang networks&3&\hphantom{1}9--14\\[.23pt]
\Avtors{Okladnikov~I.\,G.} see~Kalinichenko~L.\,A.&&\\[.23pt]
\Avtors{Ometov~A.\,Ya., Andreev~S.\,D., Turlikov~A.\,M., and
Koucheryavy~E.\,A.} Performance analysis of\linebreak
\\[-12pt]
\hspace*{23pt}a wireless data
aggregation system with contention for contemporary sensor
networks&3&23--31\\[.23pt]
\Avtors{Palionnaia~S.\,I.} see~Kudryavtsev~A.\,A.&&\\[.23pt]
\Avtors{Podkolodnyy~N.\,L.} see~Kalinichenko~L.\,A.&&\\[.23pt]
\Avtors{Ponomareva~N.\,V.} see~Kalinichenko~L.\,A.&&\\[.23pt]
\Avtors{Popkova~N.\,A.} see~Zatsman~I.\,M.&&\\[.23pt]
\Avtors{Pozanenko~A.\,S.} see~Kalinichenko~L.\,A.&&\\[.23pt]
\Avtors{Razumchik~R.\,V.} see~Konovalov~M.\,G.&&\\[.23pt]
\Avtors{Ronzhin~A.\,F.} see~Melnikov~A.\,K.&&\\[.23pt]
\Avtors{Rumovskaya~S.\,B.} see~Kirikov~I.\,A.&&\\[.23pt]
\Avtors{Rumovskaya~S.\,B.} see~Kirikov~I.\,A.&&\\[.23pt]
\Avtors{Rumovskaya~S.\,B.} see~Kolesnikov A.\,V.&&\\[.23pt]
\Avtors{Samouylov~K.\,E.} see~Gaidamaka~Yu.\,V.&&\\[.23pt]
\Avtors{Samouylov~K.\,E.} see~Naumov~V.\,A.&&\\[.23pt]
\Avtors{Serebryanskii~S.\,M.} see~Tyrsin~A.\,N.&&\\[.23pt]
\Avtors{Seyful-Mulyukov~R.\,B.} see~Callaos~N.\,K.&&\\[.23pt]
\Avtors{Shestakov~O.\,V.} Statistical properties of the denoising method
based on the stabilized hard\linebreak
\\[-12pt]
\hspace*{23pt}thresholding&2&65--69\\[.23pt]
\Avtors{Shestakov~O.\,V.} The strong law of large numbers for the risk
estimate in the problem of\linebreak
\\[-12pt]
\hspace*{23pt}tomographic image reconstruction from
projections with a correlated noise&3&41--45\\[.23pt]
\Avtors{Shestakov~O.\,V.} see~Zakharova~T.\,V.&&\\[.23pt]
\Avtors{Shnurkov~P.\,V., Gorshenin~A.\,K., and Belousov~V.\,V.}
Analytical solution of~the~optimal control\linebreak
\\[-12pt]
\hspace*{23pt}task of~a~semi-Markov
process with~finite set of~states&4&72--88\\[.23pt]
\Avtors{Shnurkov~P.\,V., Zasypko~V.\,V., Belousov~V.\,V., and
Gorshenin~A.\,K.} Development of the algorithm of numerical solution
of the optimal investment control problem\linebreak
\\[-12pt]
\hspace*{23pt}in the closed dynamical model of three-sector economy&1&82--95\\[.23pt]
\Avtors{Shorgin~S.\,Ya.} see~Gaidamaka~Yu.\,V.&&\\[.23pt]
\Avtors{Shorgin~V.\,S.} see~Agalarov~Ya.\,M.&&\\[.23pt]
\Avtors{Shubnikov~S.\,K.} see~Minin~V.\,A.&&\\[.23pt]
\Avtors{Sidorkin~I.\,I.} see~Arkhipov~O.\,P.&&\\[.23pt]
\Avtors{Sinitsyn~I.\,N.} Analytical modeling of processes in stochastic
systems with complex fractional\linebreak
\\[-12pt]
\hspace*{23pt}order Bessel nonlinearities&3&55--65\\[.23pt]
\Avtors{Sinitsyn~I.\,N.} Orthogonal supoptimal filters for nonlinear
stochastic systems on manifolds&1&34--44\\[.23pt]
\Avtors{Sinitsyn~I.\,N.\ and Korepanov~E.\,R.} Normal Pugachev
conditionally-optimal filters and extra-\linebreak
\\[-12pt]
\hspace*{23pt}polators for state linear stochastic systems&2&14--23\\[.23pt]
\Avtors{Sinitsyn~I.\,N.\ and Sinitsyn~V.\,I.} Analytical modeling of
distributions in stochastic systems on\linebreak
\\[-12pt]
\hspace*{23pt}manifolds based on ellipsoidal approximation&1&45--55\\[.23pt]
\Avtors{Sinitsyn~I.\,N., Sinitsyn~V.\,I., and
Korepanov~E.\,R.} Ellipsoidal suboptimal filters for nonlinear\linebreak
\\[-12pt]
\hspace*{23pt}stochastic systems on manifolds&2&24--35\\[.23pt]
\Avtors{Sinitsyn~V.\,I.} see~Sinitsyn~I.\,N.&&\\[.23pt]
\Avtors{Sinitsyn~V.\,I.} see~Sinitsyn~I.\,N.&&\\[.23pt]
\Avtors{Skvortsov~N.\,A.} see~Stupnikov~S.\,A.&&\\[.23pt]
\Avtors{Sokolov~I.\,A.} see~Chertok~A.\,V.&&\\
\end{tabular}
}
\pagebreak

\def\leftfootline{\small{\textbf{\thepage}
\hfill INFORMATIKA I EE PRIMENENIYA~--- INFORMATICS AND APPLICATIONS\ \ \ 2016\
\ \ volume~10\ \ \ issue\ 4}
}%
 \def\rightfootline{\small{INFORMATIKA I EE PRIMENENIYA~---
INFORMATICS AND APPLICATIONS\ \ \ 2016\ \ \ volume~10\ \ \ issue\ 4
\hfill \textbf{\thepage}}}

\def\leftkol{2016 AUTHOR INDEX} % ENGLISH ABSTRACTS}

\def\rightkol{2016 AUTHOR INDEX} %ENGLISH ABSTRACTS}


{\tabcolsep=3pt
\begin{tabular}{p{382pt}cc}
&\textbf{Issue} & \textbf{Page}\\[6pt]
\Avtors{Sopin~E.\,S.} see~Gaidamaka~Yu.\,V.&&\\
\Avtors{Strijov~V.\,V.} see~Goncharov~A.\,V.&&\\
\Avtors{Strijov~V.\,V.} see~Isachenko~R.\,V.&&\\
\Avtors{Strijov~V.\,V.} see~Karasikov~M.\,E.&&\\
\Avtors{Stupnikov~S.\,A., Briukhov~D.\,O., and Skvortsov~N.\,A.}
Co-lending systemic risk analysis over\linebreak
\\[-12pt]
\hspace*{23pt}heterogeneous data collections&1&23--33\\
\Avtors{Stupnikov~S.\,A.} see~Kalinichenko~L.\,A.&&\\
\Avtors{Suchkov~A.\,P.} see~Zatsarinny~A.\,A.&&\\
\Avtors{Timonina~E.\,E.} see~Grusho~A.\,A.&&\\
\Avtors{Titova~A.\,I.} see~Kudryavtsev~A.\,A.&&\\
\Avtors{Turlikov~A.\,M.} see~Ometov~A.\,Ya.&&\\
\Avtors{Tyrsin~A.\,N.\ and Serebryanskii~S.\,M.} Recognition of
dependences on the basis of inverse\linebreak
\\[-12pt]
\hspace*{23pt}mapping&2&58--64\\
\Avtors{Ulyanov~V.\,V.} see~Markov~A.\,S.&&\\
\Avtors{Ushakov~V.\,G.} Queueing system with working vacations and
hyperexponential input stream&2&92--97\\
\Avtors{Ushakov~V.\,G.} see~Leontyev~N.\,D.&&\\
\Avtors{Volnova~A.\,A.} see~Kalinichenko~L.\,A.&&\\
\Avtors{Yakovlev~O.\,A.\ and Gasilov~A.\,V.} Speeded-up stereo
matching using geodesic support weights&3&\hphantom{1}98--104\\
\Avtors{Zabezhailo~M.\,I.} see~Grusho~A.\,A.&&\\
\Avtors{Zabezhailo~M.\,I.} see~Grusho~A.\,A.&&\\
\Avtors{Zakharova~T.\,V.\ and Shestakov~O.\,V.} Precision analysis of
wavelet processing of aerodynamic\linebreak
\\[-12pt]
\hspace*{23pt}flow patterns&3&46--54\\
\Avtors{Zalizniak~Anna~A.\ and Kruzhkov~M.\,G.} Database
of~Russian impersonal verbal constructions&4&132--141\\
\Avtors{Zasypko~V.\,V.} see~Shnurkov~P.\,V.&&\\
\Avtors{Zatsarinny~A.\,A.\ and Suchkov~A.\,P.} Systems engineering
approaches to~the~establishment of\linebreak
\\[-12pt]
\hspace*{23pt}a~system for~decision support based
on~situational analysis&4&105--113\\
\Avtors{Zatsarinny~A.\,A.} see~Grusho~A.\,A.&&\\
\Avtors{Zatsman~I.\,M., Inkova~O.\,Yu., Kruzhkov~M.\,G., and
Popkova~N.\,A.} Representation of cross-\linebreak
\\[-12pt]
\hspace*{23pt}lingual knowledge about
connectors in supracorpora databases&1&106--118\\
\Avtors{Zatsman~I.\,M.} see~Minin~V.\,A.&&\\
\Avtors{Zeifman~A.\,I.} see~Korolev~V.\,Yu.&&\\
\Avtors{Zeifman~A.\,I.} see~Korolev~V.\,Yu.&&\\
\end{tabular}
}

%\thispagestyle{myheadings}
\def\leftfootline{\small{\textbf{\thepage}
\hfill INFORMATIKA I EE PRIMENENIYA~--- INFORMATICS AND APPLICATIONS\ \ \ 2016\
\ \ volume~10\ \ \ issue\ 4}
}%
 \def\rightfootline{\small{INFORMATIKA I EE PRIMENENIYA~---
INFORMATICS AND APPLICATIONS\ \ \ 2016\ \ \ volume~10\ \ \ issue\ 4
\hfill \textbf{\thepage}}}

 \label{end\stat}

\newpage

%\def\stat{rekl}
%\label{preobr}

%\def\tit{АКАДЕМИК ПУГАЧЁВ  ВЛАДИМИР СЕМЁНОВИЧ\\
%25.03.1911--25.03.1998}


%   \vspace*{-48pt}
%   \begin{center}\LARGE
%Академик Пугачёв  Владимир Семёнович\\ (25.03.1911--25.03.1998)
%   \end{center}
   
   %\vspace*{2.5mm}
   
   \begin{center}

{\prgsh\LARGE
ОБЪЯВЛЕНИЯ О КОНФЕРЕНЦИЯХ}

\end{center}
%\hrule

\vspace*{6pt}

   
   \vspace*{10mm}
   
   \thispagestyle{empty}

\noindent
\begin{tabular}{cc}
%\begin{center}
\multicolumn{1}{c}{\raisebox{-40pt}[0pt][0pt]{\mbox{%
\epsfxsize=33mm
\epsfbox{vspu.eps}
}}}
%\end{center}
&
\tabcolsep=0pt\begin{tabular}{c}
{\prg{\Large\textbf{XII Всероссийское совещание}}}\\[6pt]
{\prg{\Large\textbf{по проблемам управления}}}\\[12pt]
{\prg{\large 16--19 июня 2014~г.}}\\[6pt] 
{\prg{\large Институт проблем управления имени В.\,А.~Трапезникова РАН}}\\[6pt]
{\prg{\large Москва, Россия}}
\end{tabular}
\end{tabular}

\vspace*{60pt}

     
 { %\large    
 XII Всероссийское совещание по проблемам управления (ВСПУ XII), посвященное 75-летию 
Института проблем управления (ИПУ) имени В.\,А.~Трапезникова РАН, проводится 16--19~июня 
2014~г.\ 
в ИПУ РАН (г.~Москва, Россия). ВСПУ XII организуется ИПУ РАН при поддержке РФФИ, Отделения 
энергетики, машиностроения, механики и процессов управления Российской академии наук, 
Российского 
национального комитета по автоматическому управлению, Академии навигации и управ\-ле\-ния 
движением, 
Научного совета РАН по комплексным проблемам управления и автоматизации, Совета по 
мехатронике и робототехнике РАН. Официальный язык Совещания~--- русский.

\vspace*{24pt}
     
     \textbf{Направления работы}
     \begin{enumerate}[1.]
\item Теория систем управления
\item Управление подвижными объектами и навигация
\item Интеллектуальные системы управления
\item Управление в промышленности, транспортом и логистикой
\item Управление системами междисциплинарной природы
\item Средства измерения, вычислений и контроля в управлении
\item Системный анализ и принятие решений в задачах управления
\item Информационные технологии в управлении
\item Проблемы образования в области управления: современное содержание и технологии обучения
\end{enumerate}

\vspace*{24pt}

     Подробная информация о Совещании находится на сайте {\sf http://vspu2014.ipu.ru}. Срок 
окончательной подачи докладов через систему подачи докладов на сайте~--- \textbf{30~ноября} 
2013~г.
}

%\include{rekl-1}

%\end{document}

%   \vspace*{-48pt}

\begin{center}
\vspace*{6pt}
\mbox{%
\epsfxsize=53.502mm
\epsfbox{foto-1.eps}
}
\end{center}

\vspace*{6pt} %Академик


   \begin{center}
\fbox{\Large\textbf{Профессор Игорь Алексеевич Ушаков}}\\[12pt]
\textbf{\large 22.01.1935--27.02.2015}
   \end{center}


   %\vspace*{2.5mm}

   \vspace*{5mm}

   \thispagestyle{empty}

%\

%\vspace*{-12pt}


Редакционный совет и редакционная коллегия журнала <<Информатика и~её применения>> с~глубоким прискорбием извещают, что 27~февраля 2015~г.\ после тяжелой
и~продолжительной болезни скончался Игорь Алексеевич Ушаков~--- доктор технических наук, профессор, член редколлегии журнала <<Информатика и ее применения>>.

Игорь Алексеевич Ушаков окончил Московский авиационный институт, в~1963~г.\ защитил кандидатскую, а~в~1968~г.~--- докторскую диссертацию. С~1958 по 1989~гг.\ работал в~ряде научно-исследовательских организаций СССР, в~том числе руководил отделами в~НИИ АА и~ВЦ АН СССР; с 1969 по 1989 гг. преподавал в~МФТИ (был профессором, а~затем заведующим кафедрой) и~в~МЭИ. С~1989~г.~---- в~США: являлся профессором университета Дж.\ Вашингтона, университета Дж.\ Мэйсона и~Калифорнийского университета, сотрудником компаний MCI, Qualcomm и Hughes.

И.\,А.~Ушаков с момента основания журнала <<Надежность и~контроль качества>> был заместителем ответственного редактора, а~затем на протяжении многих лет членом редколлегии. В~2006~г.\ основал электронный международный журнал ``Reliability: Theory \& Application'', главным редактором которого оставался до конца жизни.

Учебниками и справочниками по теории надежности, написанными И.\,А.~Ушаковым, пользовались и~пользуются несколько поколений ученых и~специалистов в~разных странах мира.

Игорь Алексеевич всегда уделял огромное внимание работе с~молодежью; более~50 его учеников защитили докторские и~кандидатские диссертации.

И.\,А.~Ушаков вел активную научно-про\-све\-ти\-тель\-скую деятельность. В~частности, он был одним из организаторов и~руководителей Московского кабинета качества и~надежности при Политехническом музее (целью этого Кабинета было оказание консультаций работникам промышленных предприятий и~чтение курсов лекций для инженеров, занимающихся проблемой надежности). Находясь в~США, И.\,А.~Ушаков создал международный ин\-тер\-нет-фо\-рум им.\ Б.\,В.~Гнеденко, объединивший около~400~видных специалистов по приложениям теории вероятностей и~математической статистики, преимущественно в~об\-ласти теории надежности и~анализа риска, из десятков стран мира; коллективным членов этого Форума является и~наш журнал. Цели Форума~--- содействие контактам между специалистами из разных стран, организация обмена профессиональными 
новостями и~информацией (новые публикации, предстоящие события и~др.). Также необходимо отметить большое число на\-уч\-но-по\-пу\-ляр\-ных работ, опубликованных И.\,А.~Ушаковым.

И.\,А.~Ушаков обладал большим личным обаянием, имел широкий круг интересов. Все знавшие И.\,А.~Ушакова всегда будут помнить его как замечательного ученого и~прекрасного человека.

\bigskip

Редакционный совет и редакционная коллегия журнала <<Информатика и~её применения>> 
выражают глубокие соболезнования родным и близким покойного, всем, кто его знал и~работал с~ним.



%\end{document}

%\include{IPPM-25}

\def\stat{cont-rus}
{%\hrule\par
%\vskip 7pt % 7pt
\vspace*{-24pt}
\raggedleft\Large \bf%\baselineskip=3.2ex
Правила подготовки рукописей  для публикации в журнале
<<Информатика~и~её~применения>> \vskip 8pt
    \hrule
    \par
\vskip 14pt plus 6pt minus 3pt }

\label{st\stat}

\def\tit{\ }

\def\aut{\ }
\def\auf{\ }

\def\leftkol{\ }
% Правила подготовки рукописей  для публикации в журнале
%<<Информатика и её применения>>

\def\rightkol{\ }
%Правила подготовки рукописей  для публикации в журнале
%<<Информатика и её применения>>}


\titele{\tit}{\aut}{\auf}{\leftkol}{\rightkol}


\vspace*{-60pt}
{ %\small

Журнал <<Информатика и её применения>>
публикует теоретические, обзорные и дискуссионные статьи,
посвященные научным исследованиям и разработкам в области
информатики и ее приложений.

Журнал издается на русском языке. По специальному решению
редколлегии отдельные статьи могут печататься на английском языке.

Тематика журнала охватывает следующие направления:
\begin{itemize}
\item теоретические основы информатики;\\[-15pt]
      \item
математические методы исследования сложных систем и процессов;\\[-15pt]
           \item
информационные системы и сети;\\[-15pt]
                \item
информационные технологии;\\[-15pt]
                     \item
архитектура и программное обеспечение вычислительных комплексов и сетей.\\[-15pt]
\end{itemize}


\noindent
\begin{enumerate}[1.]
\item В журнале печатаются статьи, содержащие результаты, ранее не опубликованные и
не предназначенные к одновременной публикации в других изданиях.

%Публикация не должна нарушать закон об авторских правах.
Публикация предоставленной автором(ами) рукописи не должна нарушать 
положений глав~69, 70 раздела~VII части~IV Гражданского кодекса, 
которые определяют права на результаты интеллектуальной деятельности 
и~средства индивидуализации, в~том числе авторские права, в~РФ.

Ответственность за нарушение авторских прав, в~случае предъявления претензий к~редакции журнала,  
несут авторы статей.



Направляя рукопись в редакцию, авторы сохраняют свои права на данную
рукопись и при этом передают учредителям и редколлегии журнала неисключительные права на
издание статьи на русском языке 
(или на языке статьи, если он отличен от рус\-ско\-го) и~на перевод ее на английский
язык, а~также на
ее распространение в России и за рубежом. 
Каждый автор должен представить в~редакцию подписанный 
с~его стороны <<Лицензионный договор о~передаче неисключительных прав 
на использование произведения>>, текст которого размещен по адресу 
{\sf http://www.ipiran.ru/publications/licence.doc}. 
Этот договор может быть пред\-став\-лен в~бумажном (в~2-х экз.)\ 
или в~электронном виде (отсканированная копия заполненного и~подписанного документа).




Редколлегия вправе запросить у авторов экспертное заключение о возможности
пуб\-ли\-ка\-ции пред\-став\-лен\-ной статьи в открытой печати.\\[-13.5pt]

\item К статье прилагаются данные автора (авторов) (см.\ п.~8). При наличии нескольких
авторов указывается фамилия автора, ответственного за переписку с редакцией.\\[-13.5pt]

\item Редакция журнала осуществляет экспертизу присланных статей в соответствии с
принятой в журнале процедурой рецензирования.

Возвращение рукописи на доработку не означает ее принятия к печати.

Доработанный вариант с ответом на замечания рецензента необходимо прислать в
редакцию.\\[-13.5pt]

\item Решение редколлегии о публикации статьи или ее отклонении сообщается авторам.

Редколлегия может также направить авторам текст рецензии на их статью. Дискуссия по
поводу отклоненных статей не ведется.\\[-13.5pt]

%\pagebreak

\item Редактура статей высылается авторам для просмотра. Замечания к редактуре должны
быть присланы авторами в кратчайшие сроки.\\[-13.5pt]

\item Рукопись предоставляется в электронном виде в форматах MS WORD (.doc или
.docx) или \LaTeX\  (.tex), дополнительно~--- в формате .pdf, на дискете, лазерном диске
или электронной почтой. Предоставление бумажной рукописи необязательно.\\[-13.5pt]

\item При подготовке рукописи в MS Word рекомендуется использовать следующие
настройки.

Параметры страницы:
формат~--- А4; ориентация~--- книжная; поля (см): внутри~--- 2,5, снаружи~--- 1,5,
сверху~--- 2, снизу~--- 2, от края до нижнего колонтитула~--- 1,3.

Основной текст: стиль~--- <<Обычный>>, шрифт~--- Times New Roman, размер~---
14~пунк\-тов, абзацный отступ~--- 0,5~см, 1,5~интервала, выравнивание~--- по ширине.

\pagebreak

\def\leftkol{Правила подготовки рукописей  для публикации в журнале
<<Информатика и её применения>>}

\def\rightkol{Правила подготовки рукописей  для публикации в журнале
<<Информатика и её применения>>}



Рекомендуемый объем рукописи~--- не свыше 10~страниц указанного формата.
При превышении указанного объема редколлегия вправе потребовать от 
автора сокращения объема рукописи.


Сокращения слов, помимо стандартных, не допускаются. Допускается минимальное
количество аббревиатур.


Все страницы рукописи нумеруются.

Шаблоны оформления представлены в интернете:

\noindent
 {\sf
http://www.ipiran.ru/journal/template\_iiep\_ssi\_2024.zip}\\[-14pt]

\item Статья должна содержать следующую информацию на {\bfseries\textit{русском и
английском языках}}:\\[-16pt]

\begin{itemize}
\item название статьи;\\[-15pt]
\item Ф.И.О.\ авторов, на английском можно только имя и фамилию;\\[-15pt]
\item место работы, с указанием почтового адреса организации и электронного адреса каждого
автора;\\[-15pt]
\item сведения об авторах, в соответствии с форматом, образцы которого
представлены на страницах:



\def\leftfootline{\small{\textbf{\thepage}
\hfill ИНФОРМАТИКА И ЕЁ ПРИМЕНЕНИЯ\ \ \ том\ 18\ \ \ выпуск\ 3\ \ \ 2024}
}%
 \def\rightfootline{\small{ИНФОРМАТИКА И ЕЁ ПРИМЕНЕНИЯ\ \ \ том\ 18\ \ \ выпуск\ 3\ \ \ 2024
\hfill \textbf{\thepage}}}



{\sf http://www.ipiran.ru/journal/issues/2013\_07\_01/authors.asp} и

{\sf http://www.ipiran.ru/journal/issues/2013\_07\_01\_eng/authors.asp};
\item аннотация (не менее 100~слов на каждом из языков). Аннотация~--- это краткое
резюме работы, которое может публиковаться отдельно. Она является основным
источником информации в~ин\-фор\-ма\-ци\-он\-ных системах и базах данных. Английская
аннотация должна быть оригинальной, может не быть дословным переводом русского
текста и должна быть написана хорошим английским языком. В~аннотации не должно
быть ссылок на литературу и, по возможности, формул;\\[-15pt]
\item ключевые слова~--- желательно из принятых в мировой
на\-уч\-но-тех\-ни\-че\-ской литературе тематических тезаурусов. Предложения не
могут быть ключевыми словами;\\[-15pt]
\item источники финансирования работы (ссылки на гранты, проекты,
поддерживающие организации и~т.\,п.).
\end{itemize}



%\pagebreak

\item  Требования к спискам литературы.\\[-14pt]

Ссылки на литературу в тексте статьи нумеруются (в квадратных скобках) и
располагаются в каждом из списков литературы в порядке  первых упоминаний. Если источник имеет DOI и/или EDN,
то их необходимо указывать.

Списки литературы представляются в двух вариантах:\\[-14pt]


\noindent
\begin{enumerate}[(1)]
\item \textbf{Список литературы к русскоязычной части}. Русские и английские
работы~---  на языке и в алфавите оригинала;\\[-14.5pt]
\item  \textbf{References}. Русские работы и работы на других языках~--- в латинской
транслитерации с переводом на английский язык; английские работы и работы на других
языках~--- на языке оригинала.
\end{enumerate}

Необходимо для составления списка ``References'' пользоваться размещенной на сайте
{\sf http://www. translit.net/ru/bgn/} бесплатной программой транслитерации русского
 текста в~латиницу. %, при этом в~за\-клад\-ке <<варианты\ldots>> следует выбратьопцию BGN.

Список литературы ``References'' приводится полностью отдельным блоком, повторяя все
позиции из списка литературы к русскоязычной части, независимо от того, имеются или
нет в нем иностранные источники. Если в списке литературы к русскоязычной части есть
ссылки на иностранные публикации, набранные латиницей, они полностью повторяются в
списке ``References''.

Ниже приведены примеры ссылок на различные виды публикаций в списке ``References''.

\def\leftfootline{\small{\textbf{\thepage}
\hfill ИНФОРМАТИКА И ЕЁ ПРИМЕНЕНИЯ\ \ \ том\ 18\ \ \ выпуск\ 3\ \ \ 2024}
}%
 \def\rightfootline{\small{ИНФОРМАТИКА И ЕЁ ПРИМЕНЕНИЯ\ \ \ том\ 18\ \ \ выпуск\ 3\ \ \ 2024
\hfill \textbf{\thepage}}}

{\small

\noindent
\textbf{Описание статьи из журнала:}

\Aue{Zagurenko, A.\,G., V.\,A.~Korotovskikh, A.\,A.~Kolesnikov, A.\,V.~Timonov, and D.\,V.~Kardymon}. 2008.
Tekhniko-ekonomicheskaya optimizatsiya dizayna gidrorazryva plasta [Technical and
economic optimization of the design
of hydraulic fracturing]. \textit{Neftyanoe hozyaystvo} [\textit{Oil Industry}] 11:54--57.

\Aue{Zhang, Z., and D.~Zhu}. 2008. Experimental research on the localized
electrochemical micromachining. \textit{Russ. J.~Electrochem.}  44(8):926--930.
{\sf doi:10.1134/S1023193508080077}.

\noindent
\textbf{Описание статьи из электронного журнала:}

\Aue{Swaminathan, V., E.~Lepkoswka-White, and B.\,P.~Rao}. 1999. Browsers or buyers in cyberspace? An
investigation of electronic factors influencing electronic exchange. \textit{JCMC}
5(2). Available at: {\sf http://www.ascusc.org/jcmc/vol5/issue2/} (accessed April~28, 2011).

\def\leftkol{Правила подготовки рукописей  для публикации в журнале
<<Информатика и её применения>>}

\def\rightkol{Правила подготовки рукописей  для публикации в журнале
<<Информатика и её применения>>}


\noindent
\textbf{Описание статьи из продолжающегося издания (сборника трудов):}

\Aue{Astakhov, M.\,V., and T.\,V.~Tagantsev}. 2006. Eksperimental'noe
issledovanie prochnosti soedineniy ``stal'--kompozit'' [Experimental study of
the strength of joints ``steel--composite'']. \textit{Trudy MGTU
``Matematicheskoe modelirovanie slozhnykh tekh\-ni\-che\-skikh sistem''}
[\textit{Bauman MSTU ``Mathematical Modeling of Complex Technical
Systems'' Proceedings}]. 593:125--130.


\pagebreak



\noindent
\textbf{Описание материалов конференций:}

\Aue{Usmanov, T.\,S., A.\,A.~Gusmanov, I.\,Z.~Mullagalin, R.\,Ju.~Muhametshina, A.\,N.~Chervyakova, and
A.\,V.~Sveshnikov}. 2007. Osobennosti proektirovaniya razrabotki mestorozhdeniy
s primeneniem gidrorazryva
plasta [Features of the design of field development with the use of hydraulic fracturing].
\textit{Trudy 6-go
Mezhdu\-na\-rod\-no\-go Simpoziuma ``Novye resursosberegayushchie tekhnologii nedropol'zovaniya i povysheniya
neftegazootdachi''} [\textit{6th  Symposium (International) ``New Energy Saving Subsoil Technologies and
the Increasing of the Oil and Gas Impact'' Proceedings}]. Moscow. 267--272.



\def\leftfootline{\small{\textbf{\thepage}
\hfill ИНФОРМАТИКА И ЕЁ ПРИМЕНЕНИЯ\ \ \ том\ 18\ \ \ выпуск\ 3\ \ \ 2024}
}%
 \def\rightfootline{\small{ИНФОРМАТИКА И ЕЁ ПРИМЕНЕНИЯ\ \ \ том\ 18\ \ \ выпуск\ 3\ \ \ 2024
\hfill \textbf{\thepage}}}



\noindent
\textbf{Описание книги (монографии, сборники):}



Lindorf, L.\,S., and L.\,G.~Mamikoniants, eds. 1972.
\textit{Ekspluatatsiya turbogeneratorov s neposredstvennym
okhlazhdeniem} [\textit{Operation of turbine generators with direct cooling}].
Moscow: Energy Publs. 352~p.


\Aue{Latyshev, V.\,N.} 2009. \textit{Tribologiya rezaniya. Kn.~1: Friktsionnye protsessy
pri rezanii metallov}
[\textit{Tribology of cutting. Vol.~1: Frictional processes in metal cutting}]. Ivanovo: Ivanovskii
State Univ. 108~p.

\def\leftkol{Правила подготовки рукописей  для публикации в журнале
<<Информатика и её применения>>}

\def\rightkol{Правила подготовки рукописей  для публикации в журнале
<<Информатика и её применения>>}

\noindent
\textbf{Описание переводной книги}
(в списке литературы к русскоязычной части необходимо указать:~/ Пер.\ с англ.~---
после названия книги, а в конце ссылки указать оригинал книги в круглых скобках):
\begin{enumerate}[1.]
\item  В русскоязычной части:

\def\leftfootline{\small{\textbf{\thepage}
\hfill ИНФОРМАТИКА И ЕЁ ПРИМЕНЕНИЯ\ \ \ том\ 18\ \ \ выпуск\ 3\ \ \ 2024}
}%
 \def\rightfootline{\small{ИНФОРМАТИКА И ЕЁ ПРИМЕНЕНИЯ\ \ \ том\ 18\ \ \ выпуск\ 3\ \ \ 2024
\hfill \textbf{\thepage}}}

\Au{Тимошенко С.\,П., Янг Д.\,Х., Уивер~У.}
Колебания в инженерном деле~/ Пер.\ с англ.~--- М.: Машиностроение, 1985. 472~с.
(\Au{Timoshenko~S.\,P., Young~D.\,H., Weaver~W.}
Vibration problems in engineering.~--- 4th ed.~--- New York, NY, USA: Wiley, 1974. 521~p.)\\[-13.5pt]
\item  В англоязычной части:

\Aue{Timoshenko, S.\,P., D.\,H.~Young, and W.~Weaver}.
1974. \textit{Vibration problems in engineering}. 4th ed. New York: 
Wiley. 521~p.
\end{enumerate}

\vspace*{-3pt}


\noindent
\textbf{Описание неопубликованного документа:}


\Aue{Latypov, A.\,R., M.\,M.~Khasanov, and V.\,A.~Baikov}.
2004 (unpubl.). Geologiya i~dobycha (NGT GiD) [Geology and production (NGT GiD)]. Certificate on official registration of the computer program
No.\,2004611198. 

\noindent
\textbf{Описание интернет-ресурса:}


Pravila tsitirovaniya istochnikov [Rules for the citing of sources]. Available at: {\sf
http://www.scribd.com/doc/1034528/} (accessed February~7, 2011).

%\pagebreak

\noindent
\textbf{Описание диссертации или автореферата диссертации:}

\Aue{Semenov, V.\,I.}
2003. Matematicheskoe modelirovanie plazmy v sisteme kompaktnyy tor [Mathematical
modeling of the plasma in the compact torus].  Moscow.  D.Sc.\ Diss. 272~p.

\Aue{Kozhunova, O.\,S.} 2009. Tekhnologiya razrabotki semanticheskogo
slovarya informatsionnogo monitoringa [Technology of development of
semantic dictionary of information monitoring system].  Moscow: IPI RAN. PhD Thesis. 23~p.


\noindent
\textbf{Описание ГОСТа:}

GOST 8.586.5-2005. 2007. Metodika vypolneniya izmereniy. Izmerenie raskhoda i~kolichestva zhidkostey i~gazov
s~pomoshch'yu standartnykh suzhayushchikh ustroystv [Method of measurement.
Measurement of flow rate and volume of liquids and gases by means of orifice devices]. Moscow:
Standardinform  Publs. 10~p.

\noindent
\textbf{Описание патента:}

\Aue{Bolshakov, M.\,V., A.\,V.~Kulakov, A.\,N.~Lavrenov, and M.\,V.~Palkin}.
2006. Sposob orientirovaniya po krenu letatel'nogo
apparata s opti\-che\-skoy golovkoy
samonavedeniya [The way to orient on the roll of aircraft with optical homing head].
Patent RF No.\,2280590.
}

\item Присланные в редакцию материалы авторам не возвращаются.\\[-13.5pt]

\item При отправке файлов по электронной почте просим придерживаться следующих
правил:
\begin{itemize}
\item указывать в поле subject (тема) название журнала и фамилию автора;\\[-13.5pt]
\item указывать в тексте письма название статьи, авторов и~журнал, в~который направляется статья;\\[-13.5pt]
\item использовать attach (присоединение);\\[-13.5pt]
\item в состав электронной версии статьи должны входить: файл, содержащий текст
статьи, и файл(ы), содержащий(е) иллюстрации.\\[-13.5pt]
\end{itemize}

\item Журнал <<Информатика и её применения>> является некоммерческим изданием.
Плата за публикацию не взимается, гонорар авторам не выплачивается.
\end{enumerate}



\def\leftfootline{\small{\textbf{\thepage}
\hfill ИНФОРМАТИКА И ЕЁ ПРИМЕНЕНИЯ\ \ \ том\ 18\ \ \ выпуск\ 3\ \ \ 2024}
}%
 \def\rightfootline{\small{ИНФОРМАТИКА И ЕЁ ПРИМЕНЕНИЯ\ \ \ том\ 18\ \ \ выпуск\ 3\ \ \ 2024
\hfill \textbf{\thepage}}}


\vspace*{-1mm}

\begin{center}

\textbf{Адрес редакции журнала <<Информатика и её применения>>:} \\




Москва 119333, ул.~Вавилова, д.~44, корп.~2, ФИЦ ИУ РАН\\[-10pt]

\

Тел.: +7\,(499)\,135-86-92\ \ Факс:  +7\,(495)\,930-45-05\\[-10pt]

 \

e-mail:   {\sf iiep@frccsc.ru} (Стригина Светлана Николаевна)\\[-10pt]

\

{\sf http://www.ipiran.ru/journal/issues/}
\end{center}
}


\def\leftkol{Правила подготовки рукописей  для публикации в журнале
<<Информатика и её применения>>}

\def\rightkol{Правила подготовки рукописей  для публикации в журнале
<<Информатика и её применения>>}


\def\leftfootline{\small{\textbf{\thepage}
\hfill ИНФОРМАТИКА И ЕЁ ПРИМЕНЕНИЯ\ \ \ том\ 18\ \ \ выпуск\ 3\ \ \ 2024}
}%
 \def\rightfootline{\small{ИНФОРМАТИКА И ЕЁ ПРИМЕНЕНИЯ\ \ \ том\ 18\ \ \ выпуск\ 3\ \ \ 2024
\hfill \textbf{\thepage}}} 
\def\stat{podg-e}
{%\hrule\par
%\vskip 7pt % 7pt
\vspace*{-24pt}
\raggedleft\Large \bf%\baselineskip=3.2ex
Requirements for manuscripts submitted to Journal
``Informatics~and~Applications'' \vskip 8pt
    \hrule
    \par
\vskip 21pt plus 6pt minus 3pt }

\label{st\stat}

\def\tit{\ }

\def\aut{\ }
\def\auf{\ }

\def\leftkol{\ }

\def\rightkol{\ }
%Requirements for manuscripts submitted to Journal
%``Informatics~and~Applications''}

\titele{\tit}{\aut}{\auf}{\leftkol}{\rightkol}

\def\leftfootline{\small{\textbf{\thepage}
\hfill INFORMATIKA I EE PRIMENENIYA~--- INFORMATICS AND APPLICATIONS\ \ \ 2019\
\ \ volume~13\ \ \ issue\ 4}
}%
 \def\rightfootline{\small{INFORMATIKA I EE PRIMENENIYA~--- INFORMATICS AND APPLICATIONS\ \ \ 2019\ \ \ volume~13\ \ \ issue\ 4
\hfill \textbf{\thepage}}}

\vspace*{-60pt}

{\small

\noindent
Journal ``Informatics and Applications'' (Inform.\ Appl.)
publishes theoretical, review, and discussion
articles on the research and development in the
field of informatics and its applications.

The journal is published in Russian.
By a special decision of the editorial
board, some articles can be published in English.


The topics covered include the following areas:
\begin{itemize}
               \item
     theoretical fundamentals of informatics; \\[-14pt]
\item
mathematical methods for studying complex systems and processes; \\[-14pt]
\item
information systems and networks;\\[-14pt]
\item
information technologies; and \\[-14pt]
\item
architecture and software of computational complexes and networks. \\[-14pt]
\end{itemize}

\noindent
\begin{enumerate}[1.]
\item The Journal publishes original articles which have not been published before and are not
intended for simultaneous publication in other editions. An article submitted to the Journal must not violate the
Copyright law. Sending the manuscript to the Editorial Board, the authors retain all rights of the
owners of the manuscript and transfer the nonexclusive rights to publish the article in Russian
(or the language of the article, if not Russian) and its distribution in Russia and abroad to the
Founders and the Editorial Board. Authors should submit a letter to the Editorial Board in the
following form:

{\bfseries\textit{Agreement on the transfer of rights to publish:}}

``\textit{We, the undersigned authors of the manuscript ``\ldots'', pass to the
Founder and the Editorial Board of the Journal ``Informatics and Applications''
the nonexclusive right to publish the manuscript of the article in Russian (or
in English) in both print and electronic versions of the Journal. We affirm
that this publication does not violate the Copyright of other persons or
organizations.}

\textit{Author(s) signature(s): (name(s), address(es), date).}

This agreement should be submitted in paper form or in the form of a scanned copy (signed by
the authors).


%The Editorial Board has the right to request from the authors an official expert conclusion that
%the submitted article has no secret data prohibited for publication. \\[-13.5pt]
\item
A submitted article should be attached with \textbf{the data on the author(s)} (see item~8). If
there are several authors, the contact person should be indicated who is responsible for
correspondence with the Editorial Board and other authors about revisions and final approval
of the proofs.\\[-13.5pt]

\item The Editorial Board of the Journal examines the article according to the established
reviewing procedure. If the authors receive their article for correction after reviewing, it does not
mean that the article is approved for publication. The corrected article should be sent to the
Editorial Board for the subsequent review and approval.\\[-13.5pt]

\item The decision on the article publication or its rejection is communicated to the authors. The
Editorial Board may also send the reviews on the submitted articles to the authors. Any
discussion upon the rejected articles is not possible.\\[-13.5pt]

\item The edited articles will be sent to the authors for proofread. The comments of the authors
to the edited text of the article should be sent to the Editorial Board as soon as possible.\\[-13.5pt]

\item The manuscript of the article should be presented electronically in the MS WORD (.doc or
.docx) or \LaTeX\ (.tex) formats, and additionally in the .pdf format. All documents
 may be sent
by e-mail or provided on a CD or diskette. A~hard copy submission is not necessary.\\[-13.5pt]

\item The recommended typesetting instructions for manuscript.

Pages parameters: format A4, portrait orientation, document margins (cm): left~--- 2.5, right~---
1.5, above~--- 2.0, below~--- 2.0, footer 1.3.

Text: font~---Times New Roman, font size~--- 14, paragraph indent~--- 0.5, line spacing~--- 1.5,
justified alignment.

The recommended manuscript size: not more than 15~pages of the specified format.
If the specified size exceeded, the editorial board is entitled to require the author
to reduce the manuscript.

Use only standard abbreviations. Avoid  abbreviations in the title and
abstract. The full term for which an abbreviation stands should precede
its first use in the text unless it is a standard unit of measurement.

All pages of the manuscript should be numbered.

The templates for the manuscript typesetting are presented on site: {\sf
http://www.ipiran.ru/journal/template.doc}.\\[-13.5pt]


%\def\leftkol{Requirements for manuscripts submitted to Journal
%``Informatics~and~Applications''}

\item The articles should enclose data both in \textbf{Russian and English}:
\begin{itemize}
\item title;\\[-13.5pt]
\item author's name and surname;\\[-13.5pt]
\item affiliation~--- organization, its address with ZIP code, city, country, and
official e-mail address;\\[-13.5pt]
\item data on authors according to the format: (see site)

{\sf http://www.ipiran.ru/journal/issues/2013\_07\_01/authors.asp}  and

{\sf  http://www.ipiran.ru/journal/issues/2013\_07\_01\_eng/authors.asp};\\[-13.5pt]

\pagebreak

\def\leftfootline{\small{\textbf{\thepage}
\hfill INFORMATIKA I EE PRIMENENIYA~--- INFORMATICS AND APPLICATIONS\ \ \ 2019\
\ \ volume~13\ \ \ issue\ 4}
}%
 \def\rightfootline{\small{INFORMATIKA I EE PRIMENENIYA~--- INFORMATICS AND APPLICATIONS\ \ \ 2019\ \ \ volume~13\ \ \ issue\ 4
\hfill \textbf{\thepage}}}


%\def\leftkol{Requirements for manuscripts submitted to Journal
%``Informatics~and~Applications''}

%\def\rightkol{Requirements for manuscripts submitted to Journal
%``Informatics~and~Applications''}



\item abstract (not less than 100 words) both in Russian and in English. Abstract is a short
summary of the article that can be published separately. The abstract is the
main source of information on the article and it could be included in leading information
systems and data bases. The abstract in English has to be an original text and should
not be an exact translation of the Russian one. Good English is required.
In abstracts, avoid references and formulae;\\[-13.5pt]
\item indexing is performed on the basis of keywords. The use of keywords from the
internationally accepted thematic Thesauri is recommended.

%\def\leftkol{Requirements for manuscripts submitted to Journal
%``Informatics~and~Applications''}

%\def\rightkol{Requirements for manuscripts submitted to Journal
%``Informatics~and~Applications''}

Important! Keywords must not be sentences;
\item Acknowledgments.
\end{itemize}

\item References. Russian references have to be presented both in English translation and Latin
transliteration (refer {\sf http://www.translit.net/ru/bgn/}).

Please take into account the following examples of Russian references appearance:

\noindent
\textbf{Article in journal:}

\Aue{Zhang, Z., and D.~Zhu}. 2008. Experimental research on the localized electrochemical
micromachining.
\textit{Rus. J.~Electrochem.}  44(8):926--930. {\sf doi:10.1134/S1023193508080077}.


\noindent
\textbf{Journal article in electronic format:}

\Aue{Swaminathan, V., E.~Lepkoswka-White, and B.\,P.~Rao}. 1999. Browsers or buyers in
cyberspace? An
investigation of electronic factors influencing electronic exchange. \textit{JCMC}
5(2). Available at: {\sf http://www.ascusc.org/jcmc/vol5/issue2/} (accessed April~28, 2011).




\noindent
\textbf{Article from the continuing publication (collection of works, proceedings):}

\Aue{Astakhov, M.\,V., and T.\,V.~Tagantsev}. 2006. Eksperimental'noe
issledovanie prochnosti soedineniy ``stal'--kompozit'' [Experimental study of
the strength of joints ``steel--composite'']. \textit{Trudy MGTU
``Matematicheskoe modelirovanie slozhnykh tekh\-ni\-che\-skikh sistem''}
[\textit{Bauman MSTU ``Mathematical Modeling of Complex Technical
Systems'' Proceedings}]. 593:125--130.

\def\leftfootline{\small{\textbf{\thepage}
\hfill INFORMATIKA I EE PRIMENENIYA~--- INFORMATICS AND APPLICATIONS\ \ \ 2019\
\ \ volume~13\ \ \ issue\ 4}
}%
 \def\rightfootline{\small{INFORMATIKA I EE PRIMENENIYA~--- INFORMATICS AND APPLICATIONS\ \ \ 2019\ \ \ volume~13\ \ \ issue\ 4
\hfill \textbf{\thepage}}}

\def\leftkol{Requirements for manuscripts submitted to Journal
``Informatics~and~Applications''}

\def\rightkol{Requirements for manuscripts submitted to Journal
``Informatics~and~Applications''}

\noindent
\textbf{Conference proceedings:}

\Aue{Usmanov, T.\,S., A.\,A.~Gusmanov, I.\,Z.~Mullagalin, R.\,Ju.~Muhametshina,
A.\,N.~Chervyakova, and
A.\,V.~Sveshnikov}. 2007. Osobennosti proektirovaniya razrabotki mestorozhdeniy
s primeneniem gidrorazryva
plasta [Features of the design of field development with the use of hydraulic fracturing].
\textit{Trudy 6-go
Mezhdu\-na\-rod\-no\-go Simpoziuma ``Novye resursosberegayushchie tekhnologii
nedropol'zovaniya i povysheniya
neftegazootdachi''} [\textit{6th  Symposium (International) ``New Energy Saving Subsoil
Technologies and
the Increasing of the Oil and Gas Impact'' Proceedings}]. Moscow. 267--272.


\noindent
\textbf{Books and other monographs:}




Lindorf, L.\,S., and L.\,G.~Mamikoniants, eds. 1972.
\textit{Ekspluatatsiya turbogeneratorov s neposredstvennym
okhlazhdeniem} [\textit{Operation of turbine generators with direct cooling}].
Moscow: Energy Publs. 352~p.


%\Aue{Latyshev, V.\,N.} 2009. \textit{Tribologiya rezaniya. Kn.~1: Frikcionnye prosessy
%pri rezanii metallov}
%[\textit{Tribology of cutting. Vol.~1: Frictional processes in metal cutting}]. Ivanovo: Ivanovskii
%State Univ. 108~p.


%\noindent
%\textbf{Unpublished material:}

%\Aue{Latypov, A.\,R., M.\,M.~Khasanov, and V.\,A.~Baikov}.
%2004. Geology and production (NGT GiD). Certificate on official registration of the computer
%program
%No.\,2004611198. (In Russian, unpubl.)

%\noindent
%\textbf{Internet-source:}

%APA Style. 2011. Available at: {\sf http://www.apastyle.org/apa-style-help.aspx} (accessed
%February~5, 2011).

%Pravila citirovaniya istochnikov [Rules for the citing of sources]. Available at: {\sf
%http://www.scribd.com/doc/1034528/} (accessed February~7, 2011).


\noindent
\textbf{Dissertation and Thesis:}

%\Aue{Semenov, V.\,I.}
%2003. Matematicheskoe modelirovanie plazmy v sisteme kompaktnyy tor. [Mathematical
%modeling of the plasma in the compact torus]. D.Sc.\ Diss. Moscow. 272~p.

\Aue{Kozhunova, O.\,S.} 2009. Tekhnologiya razrabotki semanticheskogo
slovarya informatsionnogo monitoringa [Technology of development of
semantic dictionary of information monitoring system]. PhD Thesis. Moscow: IPI RAN. 23~p.


\noindent
\textbf{State standards and patents:}

GOST 8.586.5-2005. 2007. Metodika vypolneniya izmereniy. Izmerenie raskhoda i~kolichestva
zhidkostey i gazov 
s~pomoshch'yu standartnykh suzhayushchikh ustroystv [Method of measurement.
Measurement of flow rate and volume of liquids and gases by means of orifice devices]. M.:
Standardinform
Publs. 10~p.

%\noindent
%\textbf{Patent:}

\Aue{Bolshakov, M.\,V., A.\,V.~Kulakov, A.\,N.~Lavrenov, and M.\,V.~Palkin}.
2006. Sposob orientirovaniya po krenu letatel'nogo
apparata s opti\-che\-skoy golovkoy
samonavedeniya [The way to orient on the roll of aircraft with optical homing head].
Patent RF No.\,2280590.

References in Latin transcription are presented in the original language.

References in the text are numbered according to the order of their
first appearance; the number is
placed in square brackets. All items from the reference list should be
cited.\\[-13.5pt]

\item Manuscripts and additional materials are not returned to Authors by the Editorial Board.\\[-13.5pt]

\item Submissions of files by e-mail must include:\\[-13.5pt]
\begin{itemize}
\item   the journal title and author's name in the ``Subject'' field; \\[-13.5pt]
\item   an article and additional materials have to be attached using the ``attach'' function;\\[-13.5pt]
\item   an electronic version of the article should contain the file with the text and a separate file
with figures.\\[-13.5pt]
\end{itemize}

\item ``Informatics and Applications'' journal is not a profit publication. There are no
charges for the authors as well as there are no royalties.\\[-13.5pt]
\end{enumerate}

\def\leftfootline{\small{\textbf{\thepage}
\hfill INFORMATIKA I EE PRIMENENIYA~--- INFORMATICS AND APPLICATIONS\ \ \ 2019\
\ \ volume~13\ \ \ issue\ 4}
}%
 \def\rightfootline{\small{INFORMATIKA I EE PRIMENENIYA~--- INFORMATICS AND APPLICATIONS\ \ \ 2019\ \ \ volume~13\ \ \ issue\ 4
\hfill \textbf{\thepage}}}

\def\leftkol{Requirements for manuscripts submitted to Journal
``Informatics~and~Applications''}

\def\rightkol{Requirements for manuscripts submitted to Journal
``Informatics~and~Applications''}


%\vspace*{5mm}


\begin{center}
\textbf{Editorial Board address:} \\

%ABOUT AUTHORS



FRC CSC RAS, 44, block~2, Vavilov Str., Moscow 119333, Russia\\[-10pt]

\

Ph.: +7\,(499)\,135\,86\,92,\ \ Fax: +7\,(495)\,930\,45\,05\\[-10pt]

\

 e-mail: {\sf rust@ipiran.ru} (to Prof.\ Rustem Seyful-Mulyukov)\\[-10pt]

\

 {\sf http://www.ipiran.ru/english/journal.asp}
\end{center}
 }
%\thispagestyle{myheadings}

\def\leftkol{Requirements for manuscripts submitted to Journal
``Informatics~and~Applications''}

\def\rightkol{Requirements for manuscripts submitted to Journal
``Informatics~and~Applications''}

\def\leftfootline{\small{\textbf{\thepage}
\hfill INFORMATIKA I EE PRIMENENIYA~--- INFORMATICS AND APPLICATIONS\ \ \ 2019\
\ \ volume~13\ \ \ issue\ 4}
}%
 \def\rightfootline{\small{INFORMATIKA I EE PRIMENENIYA~--- INFORMATICS AND APPLICATIONS\ \ \ 2019\ \ \ volume~13\ \ \ issue\ 4
\hfill \textbf{\thepage}}}

 \label{end\stat}

\newpage

%\vspace*{-60pt} {\small
{\baselineskip=9.1pt
\section*{Правила подготовки рукописей статей для публикации в журнале
<<Информатика и её применения>>}

\thispagestyle{empty}

 Журнал <<Информатика и её применения>> публикует
теоретические, обзорные и дискуссионные статьи, посвященные научным
исследованиям и разработкам в области информатики и ее приложений. Журнал
издается на русском языке. По специальному решению редколлегии отдельные статьи,
в виде исключения, могут печататься на английском языке.
Тематика журнала охватывает следующие направления:
\begin{itemize}
\item теоретические основы информатики; %\\[-13.5pt]
\item математические методы исследования сложных систем и процессов; %\\[-13.5pt]
\item информационные системы и сети; %\\[-13.5pt]
\item информационные технологии; %\\[-13.5pt]
\item архитектура и программное
обеспечение вычислительных комплексов и сетей.
\end{itemize}
\begin{enumerate}
\item В журнале печатаются результаты, ранее не
опубликованные и не предназначенные к одновременной публикации в других
изданиях. Публикация не должна нарушать закон об авторских правах. Направляя
свою рукопись в редакцию, авторы автоматически передают учредителям и
редколлегии неисключительные права на издание данной статьи на русском языке и
на ее распространение в России и за рубежом. При этом за авторами сохраняются
все права как собственников данной рукописи. В связи с этим авторами должно
быть представлено в редакцию письмо в следующей форме:
Соглашение о передаче права на публикацию:

\textit{<<Мы, нижеподписавшиеся, авторы рукописи <<$\qquad\qquad$>>, передаем
учредителям и редколлегии журнала <<Информатика и её применения>>
неисключительное право опубликовать данную рукопись статьи на русском языке как
в печатной, так и в электронной версиях журнала. Мы подтверждаем, что данная
публикация не нарушает авторского права других лиц или организаций. Подписи
авторов: (ф.\,и.\,о., дата, адрес)>>.}

Указанное соглашение может быть представлено 
как в бумажном виде, так и в виде отсканированной копии (с подписями авторов).


Редколлегия вправе запросить у авторов экспертное заключение о возможности
опубликования представленной статьи в открытой печати. %\\[-13.5pt]
\item Статья
подписывается всеми авторами. На отдельном листе представляются данные автора
(или всех авторов): фамилия, полные имя и отчество, телефон, факс, e-mail,
почтовый адрес. Если работа выполнена несколькими авторами, указывается фамилия
одного из них, ответственного за переписку с редакцией. %\\[-13.5pt]
\item Редакция журнала
осуществляет самостоятельную экспертизу присланных статей. Возвращение рукописи
на доработку не означает, что статья уже принята к печати. Доработанный вариант
с ответом на замечания рецензента необходимо прислать в редакцию. %\\[-13.5pt]
\item Решение
редакционной коллегии о принятии статьи к печати или ее отклонении сообщается
авторам. Редколлегия не обязуется направлять рецензию авторам отклоненной
статьи. %\\[-13.5pt]
\item Корректура статей высылается авторам для просмотра. Редакция
просит авторов присылать свои замечания в кратчайшие сроки. %\\[-13.5pt]
\item При
подготовке рукописи в MS Word рекомендуется использовать следующие настройки.
Параметры страницы: формат~--- А4; ориентация~--- книжная; поля (см): внутри~---
2,5, снаружи~--- 1,5, сверху~--- 2, снизу~--- 2, от края до нижнего
колонтитула~--- 1,3. Основной текст: стиль~--- <<Обычный>>: шрифт Times New
Roman, размер 14~пунктов, абзацный отступ~--- 0,5~см, 1,5 интервала,
выравнивание~--- по ширине. Рекомендуемый объем рукописи~--- не свыше
25~страниц указанного формата. Ознакомиться с шаблонами, содержащими примеры
оформления, можно по адресу в Интернете:
\textsf{http://www.ipiran.ru/journal/template.doc}.
\item К рукописи, предоставляемой в 2-х
экземплярах, обязательно прилагается электронная версия статьи (как правило, в
форматах MS WORD (.doc) или \LaTeX\ (.tex), а также~--- дополнительно~--- в
формате .pdf) на дискете, лазерном диске или по электронной почте. Сокращения
слов, кроме стандартных, не применяются. Все страницы рукописи должны быть
пронумерованы. %\\[-13.5pt]
\item Статья должна содержать следующую информацию на русском и
английском языках: название, Ф.И.О. авторов, места работы авторов и их
электронные адреса, подробные сведения об авторах, оформленные в соответствии с форматом, 
определяемым файлами {\sf http://www.ipiran.ru/journal/issues/2011\_05\_01/authors.asp} и 
{\sf http://www.ipiran.ru/journal/issues/2011\_01\_eng/authors.asp},
аннотация (не более 100~слов), ключевые слова. Ссылки на
литературу в тексте статьи нумеруются (в квадратных скобках) и располагаются в
порядке их первого упоминания. В~списке литературы не должно быть позиций, на которые нет ссылки в тексте статьи.
Все фамилии авторов, заглавия статей, названия
книг, конференций и~т.\,п.\ даются на языке оригинала, если этот язык
использует кириллический или латинский алфавит. %\\[-13.5pt]
\item Присланные в редакцию материалы авторам не возвращаются.
\item При отправке файлов по электронной
почте просим придерживаться следующих правил:
\begin{itemize}
\item указывать в поле subject (тема) название журнала и фамилию автора; %\\[-13.5pt]
\item использовать attach (присоединение); %\\[-13.5pt]
\item в случае больших объемов информации возможно
использование общеизвестных архиваторов (ZIP, RAR); %\\[-13.5pt]
\item в состав электронной версии статьи должны входить: файл, содержащий текст статьи, и файл(ы),
содержащий(е) иллюстрации. %\\[-13.5pt]
\end{itemize}
\item Журнал <<Информатика и её применения>> является некоммерческим изданием. 
Плата за публикацию с авторов не взимается, гонорар авторам не выплачивается.
\end{enumerate}
\thispagestyle{empty}
\textbf{Адрес редакции:} Москва 119333,
ул.~Вавилова, д.~44, корп.~2, ИПИ РАН\\
\hphantom{\textbf{Адрес редакции:} }Тел.: +7 (499) 135-86-92\ \
Факс:  +7 (495) 930-45-05\ \  E-mail:   rust@ipiran.ru }
}

%\include{ipi-ind}

%\tableofcontents

\end{document}


%\tableofcontents

%\end{document}





%\def\stat{cont}
{%\hrule\par
%\vskip 7pt % 7pt
\raggedleft\Large \bf%\baselineskip=3.2ex
А\,В\,Т\,О\,Р\,С\,К\,И\,Й\ \ У\,К\,А\,З\,А\,Т\,Е\,Л\,Ь\ \ З\,А\ \ 2\,0\,0\,7 г. \vskip 17pt
    \hrule
    \par
\vskip 21pt plus 6pt minus 3pt }

\label{st\stat}

\def\tit{\ }

\def\aut{\ }
\def\auf{\ }

\def\leftkol{\ } % ENGLISH ABSTRACTS}

\def\rightkol{\ } %ENGLISH ABSTRACTS}

\titele{\tit}{\aut}{\auf}{\leftkol}{\rightkol}


\contentsline {chapter}{\ }{Выпуск \quad Стр.} 
\contentsline {section}{\textbf{Батракова Д.\,А., Королев В.\,Ю., Шоргин С.\,Я.}\ \ Новый метод вероятностно-ста\-ти\-сти\-че\-ско\-го анализа информационных потоков в\nobreakspace {}телекоммуникационных сетях}{\qquad 1 \qquad 40} 
\contentsline {section}{\textbf{Борисов А.\,В.}\ \ Байесовское оценивание в системах наблюдения с\nobreakspace {}марковскими скачкообразными процессами: игровой подход}{\qquad 2 \qquad 65}
\contentsline {section}{\textbf{Босов А.\,В., Иванов А.\,В.}\ \ Программная инфраструктура информационного Web-пор\-тала}{\qquad 2 \qquad 50}
\contentsline {section}{\textbf{Захаров В.\,Н., Калиниченко Л.\,А., Соколов И.\,А., Ступников С.\,А.}\ \ Конструирование канонических информационных моделей для интегрированных информационных систем}{\qquad 2 \qquad 15}
\contentsline {section}{\textbf{Захаров В.\,Н., Козмидиади В.\,А.}\ \ Средства обеспечения отказоустойчивости при\-ло\-жений}{\qquad 1 \qquad 14} 
\contentsline {section}{\textbf{Иванов А.\,В.}\ \ см. Босов А.\,В.\hfill\hfill\hfill\hfill\hfill\hfill\hfill\hfill\hfill\hfill\hfill\hfill\hfill\hfill\hfill\hfill\hfill\hfill\hfill\hfill\hfill\hfill\hfill\hfill\hfill\hfill\hfill\hfill\hfill\hfill\hfill\hfill\hfill\hfill\hfill}{\ }
\contentsline {section}{\textbf{Ильин В.\,Д., Соколов И.\,А.}\ \ Символьная модель системы знаний информатики в\nobreakspace {}че\-ло\-ве\-ко-автоматной среде}{\qquad 1 \qquad 66} 
\contentsline {section}{\textbf{Калиниченко Л.\,А.}\ \ см. Захаров В.\,Н.\hfill\hfill\hfill\hfill\hfill\hfill\hfill\hfill\hfill\hfill\hfill\hfill\hfill\hfill\hfill\hfill\hfill\hfill\hfill\hfill\hfill\hfill\hfill\hfill\hfill\hfill\hfill\hfill\hfill\hfill\hfill\hfill\hfill\hfill\hfill}{\ }
\contentsline {section}{\textbf{Козеренко Е.\,Б.}\ \ Лингвистическое моделирование для систем машинного перевода и обработки знаний}{\qquad 1 \qquad 54} 
\contentsline {section}{\textbf{Козмидиади В.\,А.}\ \ см. Захаров В.\,Н.\hfill\hfill\hfill\hfill\hfill\hfill\hfill\hfill\hfill\hfill\hfill\hfill\hfill\hfill\hfill\hfill\hfill\hfill\hfill\hfill\hfill\hfill\hfill\hfill\hfill\hfill\hfill\hfill\hfill\hfill\hfill\hfill\hfill\hfill\hfill }{\ } 
\contentsline {section}{\textbf{Королев В.\,Ю.}\ \ см. Батракова Д.\,А.\hfill\hfill\hfill\hfill\hfill\hfill\hfill\hfill\hfill\hfill\hfill\hfill\hfill\hfill\hfill\hfill\hfill\hfill\hfill\hfill\hfill\hfill\hfill\hfill\hfill\hfill\hfill\hfill\hfill\hfill\hfill\hfill\hfill\hfill\hfill}{\ } 
\contentsline {section}{\textbf{Кудрявцев А.\,А., Шоргин С.\,Я.}\ \ Байесовский подход к\nobreakspace {}анализу систем массового обслуживания и\nobreakspace {}показателей надежности}{\qquad 2 \qquad 76}
\contentsline {section}{\textbf{Печинкин А.\,В., Соколов И.\,А., Чаплыгин В.\,В.}\ \ Многолинейная система массового обслуживания с конечным накопителем и ненадежными приборами}{\qquad 1 \qquad 27} 
\contentsline {section}{\textbf{Печинкин А.\,В., Соколов И.\,А., Чаплыгин В.\,В.}\ \ Стационарные характеристики многолинейной\nobreakspace {}системы массового обслуживания с\nobreakspace {}одновременными отказами приборов}{\qquad 2 \qquad 39}
\contentsline {section}{\textbf{Синицын И.\,Н.}\ \ Корреляционные методы построения аналитических информационных моделей флуктуаций полюса Земли по априорным данным}{\qquad 2 \qquad \hphantom{9}2}
\contentsline {section}{\textbf{Синицын И.\,Н.}\ \ Развитие теории фильтров Пугачева для оперативной обработки информации в стохастических системах}{{\qquad 1 \qquad \hphantom{9}3}} 
\contentsline {section}{\textbf{Соколов И.\,А.}\ \ см. Захаров В.\,Н.\hfill\hfill\hfill\hfill\hfill\hfill\hfill\hfill\hfill\hfill\hfill\hfill\hfill\hfill\hfill\hfill\hfill\hfill\hfill\hfill\hfill\hfill\hfill\hfill\hfill\hfill\hfill\hfill\hfill\hfill\hfill\hfill\hfill\hfill\hfill}{\ }
\contentsline {section}{\textbf{Соколов И.\,А.}\ \ см. Ильин В.\,Д.\hfill\hfill\hfill\hfill\hfill\hfill\hfill\hfill\hfill\hfill\hfill\hfill\hfill\hfill\hfill\hfill\hfill\hfill\hfill\hfill\hfill\hfill\hfill\hfill\hfill\hfill\hfill\hfill\hfill\hfill\hfill\hfill\hfill\hfill\hfill}{\ } 
\contentsline {section}{\textbf{Соколов И.\,А.}\ \ см. Печинкин А.\,В.\hfill\hfill\hfill\hfill\hfill\hfill\hfill\hfill\hfill\hfill\hfill\hfill\hfill\hfill\hfill\hfill\hfill\hfill\hfill\hfill\hfill\hfill\hfill\hfill\hfill\hfill\hfill\hfill\hfill\hfill\hfill\hfill\hfill\hfill\hfill}{\ } 
\contentsline {section}{\textbf{Соколов И.\,А.}\ \ см. Печинкин А.\,В.\hfill\hfill\hfill\hfill\hfill\hfill\hfill\hfill\hfill\hfill\hfill\hfill\hfill\hfill\hfill\hfill\hfill\hfill\hfill\hfill\hfill\hfill\hfill\hfill\hfill\hfill\hfill\hfill\hfill\hfill\hfill\hfill\hfill\hfill\hfill}{\ }
\contentsline {section}{\textbf{Ступников С.\,А.}\ \ см. Захаров В.\,Н.\hfill\hfill\hfill\hfill\hfill\hfill\hfill\hfill\hfill\hfill\hfill\hfill\hfill\hfill\hfill\hfill\hfill\hfill\hfill\hfill\hfill\hfill\hfill\hfill\hfill\hfill\hfill\hfill\hfill\hfill\hfill\hfill\hfill\hfill\hfill}{\ }
\contentsline {section}{\textbf{Чаплыгин В.\,В.}\ \ см. Печинкин А.\,В.\hfill\hfill\hfill\hfill\hfill\hfill\hfill\hfill\hfill\hfill\hfill\hfill\hfill\hfill\hfill\hfill\hfill\hfill\hfill\hfill\hfill\hfill\hfill\hfill\hfill\hfill\hfill\hfill\hfill\hfill\hfill\hfill\hfill\hfill\hfill}{\ } 
\contentsline {section}{\textbf{Чаплыгин В.\,В.}\ \ см. Печинкин А.\,В.\hfill\hfill\hfill\hfill\hfill\hfill\hfill\hfill\hfill\hfill\hfill\hfill\hfill\hfill\hfill\hfill\hfill\hfill\hfill\hfill\hfill\hfill\hfill\hfill\hfill\hfill\hfill\hfill\hfill\hfill\hfill\hfill\hfill\hfill\hfill}{\ }
\contentsline {section}{\textbf{Шоргин С.\,Я.}\ \ см. Батракова Д.\,А.\hfill\hfill\hfill\hfill\hfill\hfill\hfill\hfill\hfill\hfill\hfill\hfill\hfill\hfill\hfill\hfill\hfill\hfill\hfill\hfill\hfill\hfill\hfill\hfill\hfill\hfill\hfill\hfill\hfill\hfill\hfill\hfill\hfill\hfill\hfill}{\ } 
\contentsline {section}{\textbf{Шоргин С.\,Я.}\ \ см. Кудрявцев А.\,А.\hfill\hfill\hfill\hfill\hfill\hfill\hfill\hfill\hfill\hfill\hfill\hfill\hfill\hfill\hfill\hfill\hfill\hfill\hfill\hfill\hfill\hfill\hfill\hfill\hfill\hfill\hfill\hfill\hfill\hfill\hfill\hfill\hfill\hfill\hfill}{\ }
%\thispagestyle{myheadings}
\def\leftfootline{\small{\textbf{\thepage}
\hfill ИНФОРМАТИКА И ЕЁ ПРИМЕНЕНИЯ\ \ \ том~1\ \ \ выпуск~2\ \ \ 2007}
}%
 \def\rightfootline{\small{ИНФОРМАТИКА И ЕЁ ПРИМЕНЕНИЯ\ \ \ том~1\ \ \ выпуск~2\ \ \ 2007
 \hfill \textbf{\thepage}}}
 \label{end\stat}

%\def\stat{cont-e}
{%\hrule\par
%\vskip 7pt % 7pt
\raggedleft\Large \bf%\baselineskip=3.2ex
2\,0\,0\,7\ \ A\,U\,T\,H\,O\,R\ \ I\,N\,D\,E\,X \vskip 17pt
    \hrule
    \par
\vskip 21pt plus 6pt minus 3pt }

\label{st\stat}

\def\tit{\ }

\def\aut{\ }
\def\auf{\ }

\def\leftkol{\ } % ENGLISH ABSTRACTS}

\def\rightkol{\ } %ENGLISH ABSTRACTS}

\titele{\tit}{\aut}{\auf}{\leftkol}{\rightkol}


\contentsline {chapter}{\ }{Issue \quad Page} 
\contentsline {subsection}{\textbf{Batrakova D.\,A., Korolev V.\,Yu., Shorgin S.\,Ya.}\ \ A New Method for the Probabilistic and Statistical Analysis of Information Flows in Telecommunication Networks}{\qquad 1 \qquad 40} 
\contentsline {subsection}{\textbf{Borisov A.\,V.}\ \ Bayesian Estimation in\nobreakspace {}Observation Systems with\nobreakspace {}Markov Jump Processes: Game-Theoretic Approach}{\qquad 2 \qquad 65} 
\contentsline {subsection}{\textbf{Bosov A.\,V., Ivanov A.\,V.}\ \ Linguistic Simulation for Machine Translation and Knowledge Management Systems}{\qquad 2 \qquad 50} 
\contentsline {subsection}{\textbf{Chaplygin V.\,V.} see Pechinkin A.\,V.\hfill\hfill\hfill\hfill\hfill\hfill\hfill\hfill\hfill\hfill\hfill\hfill\hfill\hfill\hfill\hfill\hfill\hfill\hfill\hfill\hfill\hfill\hfill\hfill\hfill\hfill\hfill\hfill\hfill\hfill\hfill\hfill\hfill\hfill\hfill}{\ }
\contentsline {subsection}{\textbf{Chaplygin V.\,V.} see Pechinkin A.\,V.\hfill\hfill\hfill\hfill\hfill\hfill\hfill\hfill\hfill\hfill\hfill\hfill\hfill\hfill\hfill\hfill\hfill\hfill\hfill\hfill\hfill\hfill\hfill\hfill\hfill\hfill\hfill\hfill\hfill\hfill\hfill\hfill\hfill\hfill\hfill}{\ }
\contentsline {subsection}{\textbf{Ilyin V.\,D., Sokolov I.\,A.}\ \ The Symbol Model of Informatics Knowledge System in Human-Automaton Environment}{\qquad 1 \qquad 66} 
\contentsline {subsection}{\textbf{Ivanov A.\,V.} see Bosov A.\,V.\hfill\hfill\hfill\hfill\hfill\hfill\hfill\hfill\hfill\hfill\hfill\hfill\hfill\hfill\hfill\hfill\hfill\hfill\hfill\hfill\hfill\hfill\hfill\hfill\hfill\hfill\hfill\hfill\hfill\hfill\hfill\hfill\hfill\hfill\hfill}{\ }
\contentsline {subsection}{\textbf{Kalinichenko L.\,A.} see Zakharov V.\,N.\hfill\hfill\hfill\hfill\hfill\hfill\hfill\hfill\hfill\hfill\hfill\hfill\hfill\hfill\hfill\hfill\hfill\hfill\hfill\hfill\hfill\hfill\hfill\hfill\hfill\hfill\hfill\hfill\hfill\hfill\hfill\hfill\hfill\hfill\hfill}{\ }
\contentsline {subsection}{\textbf{Korolev V.\,Yu.} see Batrakova D.\,A.\hfill\hfill\hfill\hfill\hfill\hfill\hfill\hfill\hfill\hfill\hfill\hfill\hfill\hfill\hfill\hfill\hfill\hfill\hfill\hfill\hfill\hfill\hfill\hfill\hfill\hfill\hfill\hfill\hfill\hfill\hfill\hfill\hfill\hfill\hfill}{\ }
\contentsline {subsection}{\textbf{Kozerenko E.\,B.}\ \ Linguistic Simulation for Machine Translation and Knowledge Management Systems}{\qquad 1 \qquad 54} 
\contentsline {subsection}{\textbf{Kozmidiady V.\,A.} see Zakharov V.\,N.\hfill\hfill\hfill\hfill\hfill\hfill\hfill\hfill\hfill\hfill\hfill\hfill\hfill\hfill\hfill\hfill\hfill\hfill\hfill\hfill\hfill\hfill\hfill\hfill\hfill\hfill\hfill\hfill\hfill\hfill\hfill\hfill\hfill\hfill\hfill}{\ }
\contentsline {subsection}{\textbf{Kudryavtsev A.\,A., Shorgin S.\,Ya.}\ \ Bayesian Approach to Queueing Systems and Reliability Characteristics}{\qquad 2 \qquad 76} 
\contentsline {subsection}{\textbf{Pechinkin A.\,V., Sokolov I.\,A., Chaplygin V.\,V.}\ \ Multichannel Queuing System with Finite Buffer and Unreliable Servers}{\qquad 1 \qquad 27} 
\contentsline {subsection}{\textbf{Pechinkin A.\,V., Sokolov I.\,A., Chaplygin V.\,V.}\ \ Stationary Characteristics of a Multichannel Queueing System with\nobreakspace {}Simultaneous Refusals of Servers}{\qquad 2 \qquad 39} 
\contentsline {subsection}{\textbf{Shorgin S.\,Ya.} see Batrakova D.\,A.\hfill\hfill\hfill\hfill\hfill\hfill\hfill\hfill\hfill\hfill\hfill\hfill\hfill\hfill\hfill\hfill\hfill\hfill\hfill\hfill\hfill\hfill\hfill\hfill\hfill\hfill\hfill\hfill\hfill\hfill\hfill\hfill\hfill\hfill\hfill}{\ }
\contentsline {subsection}{\textbf{Shorgin S.\,Ya.} see Kudryavtsev A.\,A.\hfill\hfill\hfill\hfill\hfill\hfill\hfill\hfill\hfill\hfill\hfill\hfill\hfill\hfill\hfill\hfill\hfill\hfill\hfill\hfill\hfill\hfill\hfill\hfill\hfill\hfill\hfill\hfill\hfill\hfill\hfill\hfill\hfill\hfill\hfill}{\ }
\contentsline {subsection}{\textbf{Sinitsyn I.\,N.}\ \ Correlational Methods for Analytical Informational Models of the Earth Pole Fluctuations Design Based on a priori Data}{\qquad 2 \qquad \hphantom{9}2}
\contentsline {subsection}{\textbf{Sinitsyn I.\,N.}\ \ Development of Pugachev Filtering for Stochastic Systems}{\qquad 1 \qquad \hphantom{9}3}
\contentsline {subsection}{\textbf{Sokolov I.\,A.} see Ilyin V.\,D.\hfill\hfill\hfill\hfill\hfill\hfill\hfill\hfill\hfill\hfill\hfill\hfill\hfill\hfill\hfill\hfill\hfill\hfill\hfill\hfill\hfill\hfill\hfill\hfill\hfill\hfill\hfill\hfill\hfill\hfill\hfill\hfill\hfill\hfill\hfill}{\ }
\contentsline {subsection}{\textbf{Sokolov I.\,A.} see Pechinkin A.\,V.\hfill\hfill\hfill\hfill\hfill\hfill\hfill\hfill\hfill\hfill\hfill\hfill\hfill\hfill\hfill\hfill\hfill\hfill\hfill\hfill\hfill\hfill\hfill\hfill\hfill\hfill\hfill\hfill\hfill\hfill\hfill\hfill\hfill\hfill\hfill}{\ }
\contentsline {subsection}{\textbf{Sokolov I.\,A.} see Pechinkin A.\,V.\hfill\hfill\hfill\hfill\hfill\hfill\hfill\hfill\hfill\hfill\hfill\hfill\hfill\hfill\hfill\hfill\hfill\hfill\hfill\hfill\hfill\hfill\hfill\hfill\hfill\hfill\hfill\hfill\hfill\hfill\hfill\hfill\hfill\hfill\hfill}{\ }
\contentsline {subsection}{\textbf{Sokolov I.\,A.} see Zakharov V.\,N.\hfill\hfill\hfill\hfill\hfill\hfill\hfill\hfill\hfill\hfill\hfill\hfill\hfill\hfill\hfill\hfill\hfill\hfill\hfill\hfill\hfill\hfill\hfill\hfill\hfill\hfill\hfill\hfill\hfill\hfill\hfill\hfill\hfill\hfill\hfill}{\ }
\contentsline {subsection}{\textbf{Stupnikov S.\,A.} see Zakharov V.\,N.\hfill\hfill\hfill\hfill\hfill\hfill\hfill\hfill\hfill\hfill\hfill\hfill\hfill\hfill\hfill\hfill\hfill\hfill\hfill\hfill\hfill\hfill\hfill\hfill\hfill\hfill\hfill\hfill\hfill\hfill\hfill\hfill\hfill\hfill\hfill}{\ }
\contentsline {subsection}{\textbf{Zakharov V.\,N., Kalinichenko L.\,A., Sokolov I.\,A., Stupnikov S.\,A.}\ \ Development of Canonical Information Models for Integrated Information Systems}{\qquad 2 \qquad 15} 
\contentsline {subsection}{\textbf{Zakharov V.\,N., Kozmidiady V.\,A.}\ \ Means Providing Applications Fault Tolerance}{\qquad 1 \qquad 14} 
\def\leftfootline{\small{\textbf{\thepage}
\hfill ИНФОРМАТИКА И ЕЁ ПРИМЕНЕНИЯ\ \ \ том~1\ \ \ выпуск~2\ \ \ 2007}
}%
 \def\rightfootline{\small{ИНФОРМАТИКА И ЕЁ ПРИМЕНЕНИЯ\ \ \ том~1\ \ \ выпуск~2\ \ \ 2007
 \hfill \textbf{\thepage}}}
 \label{end\stat}


%\tableofcontents


\end{document}

\newcommand{\Ack}{\subsection*{\protect\large\bf Acknowledgments}}