\def\stat{shestakov-1}

\def\tit{СИЛЬНАЯ СОСТОЯТЕЛЬНОСТЬ ОЦЕНКИ СРЕДНЕКВАДРАТИЧНОЙ
ПОГРЕШНОСТИ ПРИ~РЕШЕНИИ~ОБРАТНЫХ СТАТИСТИЧЕСКИХ ЗАДАЧ$^*$}

\def\titkol{Сильная состоятельность оценки среднеквадратичной
погрешности при~решении обратных статистических задач}

\def\aut{О.\,В.~Шестаков$^1$}

\def\autkol{О.\,В.~Шестаков}

\titel{\tit}{\aut}{\autkol}{\titkol}

\index{Шестаков О.\,В.}
\index{Shestakov O.\,V.}


{\renewcommand{\thefootnote}{\fnsymbol{footnote}} \footnotetext[1]
{Работа выполнена при частичной финансовой поддержке РФФИ (проект 15-07-02652).}}


\renewcommand{\thefootnote}{\arabic{footnote}}
\footnotetext[1]{Московский государственный университет им.\ М.\,В.~Ломоносова, кафедра 
математической статистики факультета вычислительной математики и~кибернетики; 
Институт проб\-лем информатики Федерального исследовательского центра 
<<Информатика и~управ\-ле\-ние>> Российской академии наук, \mbox{oshestakov@cs.msu.su}}

\vspace*{-18pt}




\Abst{Нелинейные методы обработки сигналов и~изображений с~помощью процедур 
пороговой обработки коэффициентов вейвлет-раз\-ло\-же\-ний стали популярным аппаратом 
для задач подавления шума и~компрессии. Объясняется это тем, что вейвлет-ана\-лиз 
позволяет гораздо более эффективно исследовать нестационарные сигналы, 
чем традиционный Фурье-ана\-лиз, благодаря возможности лучшей адаптации к~функциям, 
имеющим на разных участках различную степень регулярности. Анализ погрешностей 
этих методов представляет собой важную практическую задачу, поскольку позволяет 
оценить качество как самих методов, так и~используемого оборудования. 
В~некоторых приложениях данные наблюдаются не напрямую, а после применения некоторого
 линейного преобразования. Задача обращения такого преобразования, как правило, 
 некорректно поставлена, что приводит к~росту дисперсии шума. В~работе исследуются 
 асимптотические свойства оценки среднеквадратичной погрешности при обращении 
 линейных однородных операторов методами вейв\-лет-вейг\-лет-раз\-ло\-же\-ния 
 и~пороговой обработки. При довольно слабых ограничениях доказывается сильная 
 состоятельность этой оценки.}


\KW{вейвлеты; пороговая обработка; несмещенная оценка риска; коррелированный шум; асимптотическая нормальность}

\vspace*{-8pt}

\DOI{10.14357/19922264170213} 


\vskip 10pt plus 9pt minus 6pt

\thispagestyle{headings}

\begin{multicols}{2}

\label{st\stat}


\section{Введение}

 \vspace*{-4pt}

Методы вейвлет-анализа широко применяются при анализе и~обработке зашумленных данных. 
Во многих статистических задачах эти данные измеряются не напрямую, а после некоторого 
преобразования. В~таких случаях построение вейв\-лет-оце\-нок осуществляется с~по\-мощью 
вейв\-лет-вейг\-лет-раз\-ло\-же\-ния и~процедуры мягкой пороговой обработки. Порог 
обычно зависит от уровня разложения, и~его можно выбирать различными способами,
 исходя из постановки задачи и~целей обработки. Наличие шума неизбежно 
 приводит к~погрешностям. Свойства оценки таких погрешностей 
 (риска) в~модели с~независимым шумом подробно исследовались в~[1--6]. 
 Модель с~коррелированным шумом исследовалась в~[7--9]. Показано, что при 
 определенных условиях оценка риска является состоятельной и~асимптотически 
 нормальной. В~данной работе доказывается сильная состоятельность оценки риска.
 
 \vspace*{-6pt}

\section{Модель данных и~вейвлет-вейглет-разложение}

 \vspace*{-4pt}

В~данной работе рассматривается следующая модель данных:

\noindent
\begin{equation}
\label{Data_Model_Operator}
Y_i=(Kf)_i+\epsilon_i\,,\enskip i=1,\ldots, 2^J\,,
\end{equation}
где $K$~--- некоторый линейный оператор; $f$~--- 
функция, которую необходимо оценить (предполагается, что~$f$ 
принадлежит области определения~$K$); $\{\epsilon_i$, $i \in Z\}$~--- 
стационарный гауссовский процесс с~ковариационной последовательностью 
$r_k \hm= \cov(\epsilon_i,\epsilon_{i+k})$ 
(предполагается, что~$\epsilon_i$ имеют нулевое среднее и~единичную дисперсию). 

В~данной работе рассматривается модель долгосрочной зависимости 
$r_k \sim Ak^{-\alpha}$, $0\hm < \alpha \hm<1$, $A\hm>0$. Как показано в~работе~[10], 
случай краткосрочной зависимости, т.\,е.\ когда $\sum\nolimits_{-\infty}^{+\infty} 
|r_k| \hm< \infty$, эквивалентен модели с~независимым шумом.

 Во многих случаях нельзя оценить функцию~$f$, просто применив к~данным обратный 
 оператор~$K^{-1}$, поскольку такой оператор либо не существует, либо не 
 является ограниченным. Статистические задачи такого рода называются \mbox{некорректно} 
 поставленными. 
 
 Для решения такого рода задач в~работе~[11] предложен метод так 
 называемого вейв\-лет-вейг\-лет-раз\-ло\-же\-ния, 
 который хорошо зарекомендовал себя при обращении линейных однородных операторов, т.\,е.\
  таких операторов~$K$, для которых выпол-\linebreak нено
  
 
  
  \noindent
\begin{equation}
K\left[f\left(a\left(x-x_0\right)\right)\right]=a^{-\beta}(Kf)\left[a\left(x-x_0\right)\right]\notag
\end{equation}
для любого~$x_0$ и~любого $a\hm>0$. Параметр~$\beta$ называется 
показателем однородности. Примерами линейных однородных операторов служат 
оператор интегрирования, преобразование Абеля и~операторы Рисса.

 Вейвлет-разложение функции $f\hm\in L^2(\mathbb{R})$ представляет собой ряд
\begin{align}
f=\sum\limits_{j,k\in Z}\langle f,\psi_{jk}\rangle\psi_{jk},
\label{Wavelet_Decomp}
\end{align}
где $\psi_{jk}(t)\hm=2^{j/2}\psi(2^jt-k)$, а $\psi(t)$~--- 
некоторая материнская вейв\-лет-функ\-ция (семейство $\{\psi_{jk}\}_{jk\in Z}$ 
образует ортонормированный базис в~$L^2(\mathbb{R})$). Индекс~$j$ 
в~\eqref{Wavelet_Decomp} называется масштабом, а~индекс~$k$~--- сдвигом.

В дискретной постановке задачи функция задана в~отсчетах на конечном отрезке. 
Дискретное вейв\-лет-пре\-обра\-зо\-ва\-ние представляет собой умножение вектора 
значений функции~$f$ на орто\-гональную матрицу~$W$, определяемую вейв\-лет-функ\-цией~$\psi$. 
При этом в~силу ортогональности матри\-цы дискретные вейв\-лет-ко\-эф\-фи\-ци\-ен\-ты 
\mbox{связаны} с~непрерывными следующим образом: 
$\mu_{jk}\hm\approx 2^{J/2}\langle f,\psi_{jk}\rangle$, где $2^J$~--- 
число отсчетов функции~$f$~[12]. Всюду далее предполагается, что используются 
вейвлеты Мейера~[12], преобразование\linebreak \mbox{Фурье} которых обладает необходимым количеством 
непрерывных производных.

Поскольку наблюдается не функция~$f$, а~$Kf$, коэффициенты разложения 
в~(\ref{Wavelet_Decomp}) вычислить
напрямую нельзя. Идея метода вейв\-лет-вейг\-лет-раз\-ло\-же\-ния 
заключается в~том, чтобы выразить коэффициенты разложения в~(\ref{Wavelet_Decomp}) 
через~$Kf$. Если \mbox{оператор}~$K$ однороден, то существует последовательность
 функций~$\xi_{jk}$ такая, что $\langle Kf, \xi_{jk}\rangle \hm= \langle f, 
 \psi_{jk}\rangle$. Нормированные функции $u_{jk}\hm=2^{-\beta j}\xi_{jk}$ 
 получили название\linebreak <<вейглеты>>. Своими свойствами они очень похо\-жи на вейвлеты 
 (если соответствующие вейвлеты\linebreak удовлетворяют определенным условиям глад\-кости~[11]), 
 за исключением свойства ортогональности. Таким образом, ряд~(\ref{Wavelet_Decomp}) 
 можно переписать в~виде
\begin{equation*}
f=\sum\limits_{j,k\in Z}2^{-\beta j}\langle Kf, u_{jk}\rangle\psi_{jk}\,,
%\label{Wavelet_Vague_Decomp}
\end{equation*}
которое и~представляет собой вейв\-лет-вейг\-лет-разложение.

При применении дискретного аналога этого разложения 
к~модели~(\ref{Data_Model_Operator}) по аналогии с~дискретным вейв\-лет-пре\-обра\-зо\-ва\-ни\-ем 
получается следующая модель дискретных вейг\-лет-ко\-эф\-фи\-ци\-ен\-тов~\cite{9-sh1}:

\columnbreak 

\noindent
\begin{multline*}
X_{jk} = \mu_{j,k} + 2^{J(1-\alpha)/2} 
\eps_{jk}\,, \\
j=0,\ldots,J-1;\enskip k=0,\ldots,2^j-1\,,
\end{multline*}
где $\mu_{j,k} = 2^{J/2}\langle Kf, \xi_{j,k}\rangle$, 
$\eps_{jk} \hm= \int \xi_{j,k}\, d \mathbf{B}_H$, а $\mathbf{B}_H(t)$~--- 
процесс дробного броуновского движения с~$H \hm= 1-\alpha/2$.
Дисперсии коэффициентов~$X_{jk}$ не зависят от~$k$ и~равны~\cite{10-sh1}
$$
\sigma^2_{j} = C 2^{(J - j)(1 - \alpha)} 2^{2\beta j}\,,
$$
где $C$~--- положительная константа, зависящая от параметров~$A$, $\alpha$ и~$\beta$.

\section{Оценка среднеквадратичной погрешности}

Для подавления шума в~методе вейв\-лет-вейг\-лет-раз\-ло\-же\-ния 
используется процедура пороговой\linebreak обработ\-ки коэффициентов, смысл которой
 заклю\-чается 
в~удалении достаточно маленьких коэффициентов, которые считаются шумом. 
В~данной работе рассматривается так называемая мягкая пороговая обработка. 
К~каждому ко\-эф\-фи\-ци\-ен\-ту применяется функция $\rho_{T}(x)\hm=
\textrm{sgn}\,(x)\left(\abs{x}-T\right)_{+}$, т.\,е.\ 
коэффициенты, которые по модулю меньше порога~$T$, 
обнуляются, а абсолютные величины остальных коэффициентов уменьшаются на 
величину порога.

Среднеквадратичная погрешность (или риск) мягкой пороговой обработки 
определяется следующим образом:
\begin{equation*}
R_J(f)=\sum\limits_{j=0}^{J-1}\sum\limits_{k=0}^{2^j-1}
\Expect\left(\mu_{jk}-\rho_{T}(X_{jk})\right)^2\,.
\end{equation*}

В~\cite{13-sh1} было предложено использовать порог $T_{j} \hm= 
\sigma_{j}\sqrt{2\ln 2^{j}}$ и~показано, что при таком пороге среднеквадратичная 
ошибка близка к~минимальной~\cite{12-sh1}. Этот порог получил название 
<<универсальный>>. В~дальнейшем будет использоваться именно такой вид порога.

Вычислить $R_J(f)$ в~явном виде нельзя, так как в~выражении присутствуют неизвестные 
<<чистые>> коэффициенты~$\mu_{jk}$. Однако его можно оценить с~по\-мощью 
следующей величины:
\begin{equation}
\widehat{R}_J(f)=\sum\limits_{j=0}^{J-1}\sum\limits_{k=0}^{2^j-1}F\left[X_{jk}^2,T\right]\,,
\label{Risk_Estimate}
\end{equation}
где $F[x,T]=(x-\sigma^2)\Ik(|x|\leqslant T^2)\hm+
(\sigma^2+T^2)\Ik(|x|\hm>T^2)$. В~работе~\cite{1-sh1} 
было показано, что~$\widehat{R}_J(f)$ является несмещенной оценкой~$R_J(f)$.

\section{Сильная состоятельность оценки среднеквадратичной погрешности}

В работе~\cite{9-sh1} показано, что~(\ref{Risk_Estimate}) при определенных 
условиях гладкости на функцию~$f$ является асимптотически нормальной и~состоятельной 
оценкой~$R_J(f)$. Оказывается, что эта оценка является также сильно состоятельной 
даже при более слабых ограничениях.

Для доказательства этого утверждения потребуется следующая лемма Боска~\cite{14-sh1}, 
в~которой оценивается вероятность отклонения суммы ограниченных слабозависимых 
случайных величин от ее математического ожидания.

\smallskip

\noindent
\textbf{Лемма.}\ \textit{Пусть $\{X_i,\;i\in Z\}$~--- 
последовательность случайных величин таких, что $\Expect X_i\hm=0$ 
и~$\abs{X_i}\leqslant b$ п.в.\ для всех $i\hm\in Z$, где $b\hm>0$~--- 
некоторая константа. Тогда для любого $q\hm\in[1,n/2]$ и~любого $\eps\hm>0$
\begin{multline}
{\sf P}\left(\abs{\sum\limits_{i=1}^n X_i}>n\eps\right)\leqslant 
4\exp\left\{-\fr{\eps^2}{8b^2}q\right\}+{}\\
{}+22\left(1+\fr{4b}{\eps}\right)^{1/2}
q\alpha\left(\left[\fr{n}{2q}\right]\right)\,,
\label{Bosq_inequality}
\end{multline}
где $\alpha(k)$~--- коэффициент $\alpha$-пе\-ре\-ме\-ши\-ва\-ния последовательности
 $\{X_i,\;i\hm\in Z\}$.}
 
 \smallskip

Докажем теперь сильную состоятельность оценки~(\ref{Risk_Estimate}).

\smallskip

\noindent

\textbf{Теорема.}\ \textit{Пусть $f\in  L^2(\mathbb{R})$ задана на 
конечном отрезке, а $K$~--- линейный однородный оператор с~показателем $\beta\hm>0$. 
Тогда имеет место сходимость
\begin{equation}
\label{R_Conv}
\fr{\widehat{R}_J(f)-R_J(f)}{2^{\lambda J}}\rightarrow 0 \;\mbox{ п.в.~при } 
J\rightarrow\infty
\end{equation}
при любом $\lambda>1/2\hm+2\beta$ в~случае $\alpha\hm+2\beta\hm\geqslant1/2$ 
и~любом $\lambda\hm>1\hm-\alpha$ в~случае $\alpha\hm+2\beta\hm<1/2$}.

\smallskip

\noindent
Д\,о\,к\,а\,з\,а\,т\,е\,л\,ь\,с\,т\,в\,о\,.\ \ 
Пусть $0\hm<p\hm<1$ некоторое число, которое будет выбрано позднее. 
Представим числитель~(\ref{R_Conv}) в~виде
$\widehat{R}_J(f)\hm-R_J(f)\hm=\widehat{R}_1\hm+\widehat{R}_2$, где
\begin{align*}
\widehat{R}_1&=\sum\limits_{j=0}^{[pJ]-1}\sum\limits_{k=0}^{2^j-1}\left(F
\left[X_{jk}^2,T\right]-\Expect F\left[X_{jk}^2,T\right]\right)\,;\\
\widehat{R}_2&=\sum\limits_{j=[pJ]}^{J-1}\sum\limits_{k=0}^{2^j-1}\left(F\left[X_{jk}^2,T\right]
-\Expect F\left[X_{jk}^2,T\right]\right)\,.
\end{align*}
Рассмотрим~$\widehat{R}_2$. Для произвольного $\delta\hm>0$ имеем:
\begin{multline}
p_J={\sf P}\left(\abs{\widehat{R}_2}>\delta2^{\lambda J}\right)\leqslant{}\\
{}\leqslant \sum\limits_{j=[pJ]}^{J-1}{\sf P}\left(
\abs{\sum\limits_{k=0}^{2^j-1}\left(F\left[X_{jk}^2,T\right]-
\Expect F\left[X_{jk}^2,T\right]\right)}>{}\right.\\
\left.{}>\delta J^{-1}2^{\lambda J}
\vphantom{\abs{\sum\limits_{k=0}^{2^j-1}\left(F\left[X_{jk}^2,T\right]-
\Expect F\left[X_{jk}^2,T\right]\right)}}
\right)\,.
\label{PJ_inequality}
\end{multline}
Из вида функции $F[x,T]$ следует, что $-\sigma^2_j\hm\leqslant F[X_{jk}^2,T_j]
\hm\leqslant\sigma^2_j\hm+T^2_j$. В~работе~\cite{10-sh1} 
показано, что в~силу свойств вейвлетов Мейера при каждом~$j$ слагаемые в~сумме 
под вероятностью в~(\ref{PJ_inequality}) удовлетворяют свойству $\rho$-пе\-ре\-ме\-ши\-ва\-ния 
с~коэффициентом $\rho(k)\hm\leqslant C k^{-M}$, где $C\hm>0$~--- 
некоторая константа, а~$M$ можно сделать достаточно большим, выбрав соответствующий 
вейвлет Мейера.

Известно~\cite{15-sh1}, что для коэффициентов $\alpha$-пе\-ре\-ме\-ши\-ва\-ния 
и~$\rho$-пе\-ре\-ме\-ши\-ва\-ния справедливо неравенство $4\alpha(k)\hm\leqslant\rho(k)$. 
Применяя неравенство~(\ref{Bosq_inequality}) с~$q\hm=2^{\theta j}$ ($\theta\hm<1$) 
для каждого~$j$ в~сумме~(\ref{PJ_inequality}) и~выбирая~$M$
 достаточно большим, получаем:
\begin{multline}
p_J\leqslant c_1 J\times{}\\
{}\times\max\limits_{[pJ]\leqslant j\leqslant J-1}
\left\{\exp\!\left[-c_2J^{-3} 2^{2(\lambda-1+\alpha)J+(\theta-2\alpha-4\beta)j}\right]
\!\right\}+{}\\
{}+o_J\,.
\label{P_inequality}
\end{multline}
Здесь $c_1$ и~$c_2$~--- некоторые положительные константы, а~$o_J$ убывает по~$J$ 
быстрее, чем~$2^{-M_0 pJ}$, где~$M_0$~--- некоторое положительное число, 
зависящее от~$M$.

Если $\alpha+2\beta\geqslant 1/2$, то $\theta\hm-2\alpha\hm-4\beta\hm<0$, и~при $j\hm=J$ 
правая часть~(\ref{P_inequality}) не превосходит $c_1 J\exp\left[-c_2J^{-3} 
2^{(2\lambda-2+\theta-4\beta)J}\right]$. Поскольку $\theta\hm<1$ 
можно выбрать произвольно, для того чтобы выполнялось неравенство 
$2\lambda\hm-2\hm+\theta\hm-4\beta\hm>0$,
 до\-статочно потребовать $\lambda\hm>1/2\hm+2\beta$. 
Если же\linebreak $\alpha\hm+2\beta\hm<1/2$, то можно выбрать $\theta\hm<1$ так, 
что $\theta\hm-2\alpha\hm-4\beta\hm>0$, и~правая часть~(\ref{P_inequality}) 
не превосходит $c_1 J\exp\left[-c_2J^{-3} 2^{2(\lambda-1+\alpha)J}\right]$. 
Следовательно, чтобы выполнялось неравенство $\lambda\hm-1\hm+\alpha\hm>0$, достаточно 
потребовать $\lambda\hm>1\hm-\alpha$. При таком выборе~$\lambda$
\begin{equation*}
\sum\limits_{J=1}^{\infty}p_J<\infty\,,
\end{equation*}
и в~силу леммы Бо\-ре\-ля--Кан\-тел\-ли для любого $\delta\hm>0$ 
событие $\left\{\abs{\widehat{R}_2}\hm>\delta2^{\lambda J}\right\}$ осуществляется лишь 
конечное число раз. Следовательно, $\widehat{R}_2 2^{-\lambda J}\rightarrow 0$ п.в.

В $\widehat{R}_1$ при каждом фиксированном~$j$ ($0\hm\leqslant j\hm\leqslant [pJ]\hm-1$) 
чис\-ло слагаемых равно~$2^j$, а каждое слагаемое не превосходит по
 модулю $B J2^{J(1-\alpha)}2^{j(2\beta+\alpha-1)}$, где $B\hm>0$~--- 
 некоторая константа. Следовательно, 
 $\abs{\widehat{R}_1}\hm\leqslant B_1J 2^{J(1-\alpha+p(2\beta+\alpha))}$, где 
 $B_1$~--- некоторая положительная константа. Если $2\beta\hm+\alpha\hm\geqslant1/2$, 
 то при $\lambda\hm>1/2\hm+2\beta$ всегда можно выбрать такое $0\hm<p\hm<1$, что 
 $\lambda\hm-(1\hm-\alpha\hm+p(2\beta\hm+\alpha))\hm>0$, и,~следовательно, 
 $\widehat{R}_1 2^{-\lambda J}\rightarrow 0$ п.в. 
 Если же $\alpha\hm+2\beta\hm<1/2$, то при $\lambda\hm>1\hm-\alpha$ 
 всегда можно выбрать такое $0\hm<p\hm<1$, что $\lambda\hm-(1\hm-\alpha\hm+p(2\beta
 \hm+\alpha))\hm>0$, и,~следовательно, $\widehat{R}_1 2^{-\lambda J}\rightarrow 0$ 
 п.в. Теорема до\-ка\-зана.


{\small\frenchspacing
 {%\baselineskip=10.8pt
 \addcontentsline{toc}{section}{References}
 \begin{thebibliography}{99}
\bibitem{1-sh1}
\Au{Donoho D., Johnstone~I.\,M.} Adapting to unknown smoothness via wavelet shrinkage~// 
J.~Amer. Stat. Assoc., 1995. Vol.~90. P.~1200--1224.

\bibitem{3-sh1} %2
\Au{Маркин А.\,В.} Предельное распределение оценки риска при пороговой обработке 
вейв\-лет-ко\-эф\-фи\-ци\-ен\-тов~// Информатика и~её применения, 2009. Т.~3. Вып.~4. С.~57--63.

\bibitem{4-sh1} %3
\Au{Маркин А.\,В., Шестаков~О.\,В.} О~со\-сто\-ятель\-ности оценки риска при пороговой 
обработке вейв\-лет-ко\-эф\-фи\-ци\-ен\-тов~// Вестн. Моск.
ун-та. Сер.~15: Вычисл. матем. и~киберн., 2010. №\,1. C.~26--34.

\bibitem{2-sh1} %4
\Au{Кудрявцев А.\,А., Шестаков~О.\,В.} Асимптотика оценки риска при вейг\-лет-вейв\-лет-раз\-ло\-же\-нии 
наблюдаемого сигнала~// T-Comm: Телекоммуникации и~транспорт, 2011. №\,2. С.~54--57.



\bibitem{5-sh1}
\Au{Шестаков О. \,В.}  Асимптотическая нормальность оценки риска 
пороговой обработки вейв\-лет-ко\-эф\-фи\-ци\-ен\-тов при выборе
адаптивного порога~// Докл. РАН, 2012. Т.~445. №\,5. С.~513--515.

\bibitem{6-sh1}
\Au{Шестаков О.\,В.} О~свойствах оценки среднеквадратичного риска при регуляризации 
обращения линейного однородного оператора с~помощью адаптивной пороговой 
обработки коэффициентов вейг\-лет-вейв\-лет раз\-ло\-же\-ния~// 
Вестн. ТвГУ. Серия: Прикладная математика, 2012. №\,8. С.~117--130.

\bibitem{9-sh1} %7
\Au{Ерошенко А.\,А., Шестаков~О.\,В.} Асимптотическая нормальность оценки риска 
при вейв\-лет-вейг\-лет-раз\-ло\-же\-нии функции сигнала в~модели с~коррелированным 
шумом~// Вестн. Моск. ун-та. Сер.~15: Вычисл. матем. и~киберн., 2014. №\,3. C.~110--117.

\bibitem{7-sh1} %8
\Au{Ерошенко А.\,А.} Состоятельность оценок риска при вейв\-лет-вейг\-лет 
и~вейг\-лет-вейв\-лет-раз\-ло\-же\-ни\-ях функции сигнала в~модели с~коррелированным 
шумом~// Вестн. ТвГУ. Серия: Прикладная математика, 2015. №\,1. С.~103--114.

\bibitem{8-sh1} %9
\Au{Ерошенко А.\,А., Кудрявцев~А.\,А., Шестаков~О.\,В.} 
Предельное распределение оценки риска метода вейг\-лет-вейв\-лет-раз\-ло\-же\-ния 
сигнала в~модели с~коррелированным шумом~// Вестн. Моск. ун-та. Сер.~15: 
Вычисл. матем. и~киберн., 2015. №\,1. C.~12--18.



\bibitem{10-sh1}
\Au{Johnstone I.\,M.} Wavelet shrinkage for correlated data and inverse problems: 
Adaptivity results~// Stat. Sinica, 1999. Vol.~9. No.\,1. P.~51--83.

\bibitem{11-sh1}
\Au{Donoho D.} Nonlinear solution of linear inverse problems by wavelet-vaguelette 
decomposition~// Appl. Comput. Harmon. Anal., 1995. Vol.~2. P.~101--126.

\bibitem{12-sh1}
\Au{Mallat S.} A~wavelet tour of signal processing.~--- New York, NY, USA:
Academic Press, 1999. 857~p.

\bibitem{13-sh1}
\Au{Kolaczyk E.\,D.} Wavelet methods for the inversion of certain homogeneous 
linear operators in the presence of noisy data.~--- Stanford, CA, USA:
Stanford University, 1994.
 PhD Thesis.

\bibitem{14-sh1}
\Au{Bosq D.} Nonparametric statistics for stochastic processes: Estimation and 
prediction.~--- New York, NY, USA: Springer-Verlag, 1996. 169~p.

\bibitem{15-sh1}
\Au{Bradley R.\,C.} Basic properties of strong mixing conditions. 
A~survey and some open questions~// Probab. Surveys, 2005. Vol.~2. P.~107--144.

 \end{thebibliography}

 }
 }

\end{multicols}

\vspace*{-3pt}

\hfill{\small\textit{Поступила в~редакцию 11.11.16}}

\vspace*{8pt}

%\newpage

%\vspace*{-24pt}

\hrule

\vspace*{2pt}

\hrule

%\vspace*{8pt}


\def\tit{STRONG CONSISTENCY OF~THE~MEAN SQUARE RISK ESTIMATE IN~THE~INVERSE STATISTICAL 
PROBLEMS}

\def\titkol{Strong consistency of~the~mean square risk estimate in~the~inverse statistical 
problems}

\def\aut{O.\,V.~Shestakov$^{1,2}$}

\def\autkol{O.\,V.~Shestakov}

\titel{\tit}{\aut}{\autkol}{\titkol}

\vspace*{-9pt}


\noindent
$^1$Department of Mathematical Statistics, Faculty of Computational Mathematics 
and Cybernetics, M.\,V.~Lo-\linebreak
$\hphantom{^1}$monosov Moscow State University, 1-52~Leninskiye Gory,
 GSP-1, Moscow 119991, Russian Federation
 
 \noindent
 $^2$Institute of Informatics Problems, Federal Research Center 
 ``Computer Science and Control'' of the Russian\linebreak
 $\hphantom{^1}$Academy of Sciences, 44-2~Vavilov Str., 
 Moscow 119333, Russian Federation



\def\leftfootline{\small{\textbf{\thepage}
\hfill INFORMATIKA I EE PRIMENENIYA~--- INFORMATICS AND
APPLICATIONS\ \ \ 2017\ \ \ volume~11\ \ \ issue\ 2}
}%
 \def\rightfootline{\small{INFORMATIKA I EE PRIMENENIYA~---
INFORMATICS AND APPLICATIONS\ \ \ 2017\ \ \ volume~11\ \ \ issue\ 2
\hfill \textbf{\thepage}}}

\vspace*{3pt}



\Abste{Nonlinear methods of digital signal processing based on 
thresholding of wavelet coefficients became a popular tool for solving 
the problems of signal de-noising and compression. This is explained by 
the fact that the wavelet methods allow much more effective analysis of 
nonstationary signals than traditional Fourier analysis,
thanks to the 
better adaptation to the functions with varying degrees of regularity. 
Wavelet thresholding risk
analysis is an important practical task, because 
it allows determining the quality of techniques themselves and the equipment\linebreak\vspace*{-12pt}}

\Abstend{which is being used. In some applications, the data are observed not directly 
but after applying a~linear transformation. The problem of inverting this 
transformation is usually set incorrectly, leading to an increase in the 
noise variance. In this paper, the asymptotic properties of the mean square 
error (MSE) estimate are studied when inverting linear homogeneous operators by means of wavelet 
vaguelette decomposition and thresholding. 
The strong consistency of this estimate has been proved under
mild conditions.}

\KWE{wavelets; thresholding; MSE risk estimate; correlated noise;
 asymptotic normality} 
 
\DOI{10.14357/19922264170213} 

\vspace*{-8pt}

\Ack
\noindent
The work was partly supported by the Russian Foundation
for Basic Research projects Nos.\,15-37-20611 and
16-07-00677).



\vspace*{3pt}

  \begin{multicols}{2}

\renewcommand{\bibname}{\protect\rmfamily References}
%\renewcommand{\bibname}{\large\protect\rm References}

{\small\frenchspacing
 {%\baselineskip=10.8pt
 \addcontentsline{toc}{section}{References}
 \begin{thebibliography}{99}

\bibitem{1-sh1-1}
\Aue{Donoho, D., and I.\,M.~Johnstone.} 1995. Adapting to unknown smoothness via
 wavelet shrinkage. \textit{J.~Amer. Stat. Assoc.} 90:1200--1224.
 
 \bibitem{3-sh1-1}
\Aue{Markin, A.\,V.} 2009. Predel'noe raspredelenie otsenki riska pri 
porogovoy obrabotke veyvlet-koeffitsientov [Limit distribution of risk estimate 
of wavelet coefficient thresholding]. \textit{Informatika i~ee Primeneniya~--- 
Inform. Appl.}  3(4):57--63.

\bibitem{4-sh1-1}
\Aue{Markin, A.\,V., and O.\,V.~Shestakov.} 2010. Consistency of risk estimation 
with thresholding of wavelet coefficients. \textit{Moscow Univ. Comput. Math. 
Cybern.} 34(1):22--30.


\bibitem{2-sh1-1} %4
\Aue{Kudryavtsev, A.\,A., and O.\,V.~Shestakov}. 2011. Asimptotika otsenki riska pri 
veyglet-veyvlet-razlozhenii nablyudaemogo signala 
[The asymptotic behavior of the risk estimate under wavelet-vaguelette 
decomposition of the observed signal]. \textit{T-Comm: Telekommunikatsii i Transport}
[T-Comm: Telecommunications and Transport] 2:54--57.



\bibitem{5-sh1-1}
\Aue{Shestakov, O.\,V.} 2012. Asymptotic normality of adaptive wavelet thresholding 
risk estimation. \textit{Dokl. Math.} 86(1):556--558.

\bibitem{6-sh1-1}
\Aue{Shestakov, O.\,V.} 2012. O~svoystvakh otsenki sred\-ne\-kvad\-ra\-ti\-ch\-no\-go riska pri 
regulyarizatsii obrashcheniya lineynogo odnorodnogo operatora s~pomoshch'yu adaptivnoy 
porogovoy obrabotki koeffitsientov veyglet-veyvlet razlozheniya 
[The properties of mean square error estimate when regularizing the inversion of the homogeneous 
linear operator using adaptive thresholding of wavelet-vaguelette decomposition 
coefficients]. \textit{Vestn. TvGU. Seriya: Prikladnaya matematika} 
[Herald of Tver State University. Series: Applied Mathematics] 8:117--130.

\bibitem{9-sh1-1} %7
\Aue{Eroshenko, A.\,A., and O.\,V.~Shestakov}. 2014. Asymptotic normality of 
estimating risk upon the wavelet-vaguelette decomposition of a~signal function 
in a~model with correlated noise. \textit{Moscow Univ. Comput. Math.  Cybern.} 
38(3):110--117.

\bibitem{7-sh1-1} %8
\Aue{Eroshenko, A.\,A.} 2015. Sostoyatel'nost' otsenok riska pri veyvlet-veyglet 
i~veyglet-veyvlet-razlozheniyakh funktsii signala v~modeli s~korrelirovannym shumom 
[Consistency of risk estimates for wavelet-vaguelette and vaguelette-wavelet 
decompositions of signal function in the model of data with correlated noise]. 
\textit{Vestn. TvGU. Seriya: Prikladnaya matematika} 
[Herald of Tver State University. Series: Applied Mathematics] 1:103--114.

\bibitem{8-sh1-1} %9
\Aue{Eroshenko, A.\,A., A.\,A.~Kudryavtsev,  and O.\,V.~Shestakov.} 2015. 
Limit distribution of a~risk estimate using the vaguelette-wavelet decomposition 
of signals in a model with correlated noise. 
\textit{Moscow Univ. Comput. Math. Cybern.} 39(1):6-13.



\bibitem{10-sh1-1}
\Aue{Johnstone, I.\,M.} 1999. Wavelet shrinkage for correlated data and inverse
 problems: Adaptivity results. \textit{Stat. Sinica} 9(1):51--83.

\bibitem{11-sh1-1}
\Aue{Donoho, D.} 1995. Nonlinear solution of linear inverse problems by 
wavelet-vaguelette decomposition. \textit{Appl. Comput. Harmon. Anal.}  
2:101--126.


\bibitem{12-sh1-1}
\Aue{Mallat, S.} 1999. \textit{A~wavelet tour of signal processing.} 
New York, NY: Academic Press. 857~p.

\bibitem{13-sh1-1}
\Aue{Kolaczyk, E.\,D.} 1994. Wavelet methods for the inversion of certain homogeneous 
linear operators in the presence of noisy data. Stanford, CA: Stanford University. 
PhD Thesis. 163~p.

\bibitem{14-sh1-1}
\Aue{Bosq, D.} 1996. \textit{Nonparametric statistics for stochastic processes: 
Estimation and prediction.} New York, NY: Springer-Verlag. 169~p.

\bibitem{15-sh1-1}
\Aue{Bradley, R.\,C.} 2005. Basic properties of strong mixing conditions. 
A~survey and some open questions. \textit{Probab. Surveys} 2:107--144.
\end{thebibliography}

 }
 }

\end{multicols}

\vspace*{-3pt}

\hfill{\small\textit{Received November 11, 2016}}

\vspace*{-12pt}

\Contrl

\noindent
\textbf{Shestakov Oleg V.} (b.\ 1976)~--- 
Doctor of Science in physics and mathematics, associate professor, 
Department of Mathematical Statistics, Faculty of Computational Mathematics 
and Cybernetics, M.\,V.~Lomonosov Moscow State University, 1-52~Leninskiye Gory,
 GSP-1, Moscow 119991, Russian Federation; senior scientist, 
 Institute of Informatics Problems, Federal Research Center 
 ``Computer Science and Control'' of the Russian Academy of Sciences, 44-2~Vavilov Str., 
 Moscow 119333, Russian Federation; \mbox{oshestakov@cs.msu.su}
\label{end\stat}


\renewcommand{\bibname}{\protect\rm Литература} 