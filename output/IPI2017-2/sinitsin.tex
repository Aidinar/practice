\def\raa{{\mathrm{ЭА}}}
\def\hXt{{\hat X_t}}
\def\ss2{\mathop {\sum\limits^{n^\Theta}\sum\limits^{n^\Theta}}}



\def\stat{sinitsin}

\def\tit{МОДИФИЦИРОВАННЫЕ ЭЛЛИПСОИДАЛЬНЫЕ 
УСЛОВНО-ОПТИМАЛЬНЫЕ ФИЛЬТРЫ ДЛЯ~НЕЛИНЕЙНЫХ СТОХАСТИЧЕСКИХ 
   СИСТЕМ НА~МНОГООБРАЗИЯХ$^*$}

\def\titkol{Модифицированные эллипсоидальные 
условно-оптимальные фильтры для~нелинейных МСтС} %стохастических     систем на~многообразиях}

\def\aut{И.\,Н.~Синицын$^1$, В.\,И.~Синицын$^2$, Э.\,Р.~Корепанов$^3$}

\def\autkol{И.\,Н.~Синицын, В.\,И.~Синицын, Э.\,Р.~Корепанов}

\titel{\tit}{\aut}{\autkol}{\titkol}

\index{Синицын И.\,Н.}
\index{Синицын В.\,И.}
\index{Корепанов Э.\,Р.}


{\renewcommand{\thefootnote}{\fnsymbol{footnote}} \footnotetext[1]
{Работа выполнена при поддержке РФФИ (проект 15-07-02244).}}


\renewcommand{\thefootnote}{\arabic{footnote}}
\footnotetext[1]{Институт проблем информатики Федерального исследовательского центра <<Информатика и~управление>> Российской академии наук,
\mbox{sinitsin@dol.ru}}
\footnotetext[2]{Институт проблем информатики Федерального исследовательского центра <<Информатика и~управление>> Российской академии наук,
\mbox{vsinitsin@ipiran.ru}}
\footnotetext[3]{Институт проблем информатики Федерального исследовательского центра <<Информатика и~управление>> Российской академии наук,
\mbox{ekorepanov@ipiran.ru}}
 
\vspace*{-10pt} 

\Abst{Представлена теория аналитического синтеза по критерию минимума средней 
квадратической ошибки (с.к.)\ модифицированных эллипсоидальных услов\-но-оп\-ти\-маль\-ных
 фильт\-ров 
(МЭУОФ) для нелинейных дифференциальных стохастических систем на  
гладких многообразиях (МСтС) 
на основе эллипсоидальной аппроксимации ненормированных апостериорных распределений. 
Рассмотрены случаи гауссовских и~негауссовских МСтС. Алгоритмы МЭУОФ по сравнению 
с~алгоритмами ЭУОФ обладают достаточной простотой. Алгоритмы МЭУОФ положены в~основу 
модуля инструментального программного обеспечения StS-Filter (version 2017).}

\KW{винеровский шум;
метод эллипсоидальной аппроксимации (МЭА);
метод эллипсоидальной линеаризации (МЭЛ);
 модифицированный эллипсоидальный СОФ (МЭСОФ);
ненормированная одномерная апостериорная характеристическая функция;
пуассоновский шум;
субоптимальный фильтр (СОФ);
условно-оптимальный фильтр (УОФ);
уравнения точности и~чувствительности;
эллипсоидальный УОФ (ЭУОФ)}

\vspace*{-6pt}

\DOI{10.14357/19922264170211} 


\vskip 10pt plus 9pt minus 6pt

\thispagestyle{headings}

\begin{multicols}{2}

\label{st\stat}

\section{Введение}

В~[1, 2] представлена теория УОФ на базе 
методов нормальной аппроксимации (МНА),
статистической линеаризации (МСЛ) и~ортогональных разложений (МОР) для 
МСтС с~винеровскими шумами 
в~уравнениях наблюдения и~винеровскими и~пуассоновскими шумами в~уравнениях со\-сто\-яния.
В~основу теории УОФ были положены точные нелинейные уравнения для апостериорного 
одномерного распределения.

В~[3] рассмотрено развитие~[1, 2] на случай, ко-\linebreak гда апостериорное одномерное 
распределение ошибки фильтрации допускает эллипсоидальную аппроксимацию~[4]. 
Получены точные фильтрационные уравнения, а~также уравнения точности\linebreak 
и~чувствительности на основе МОР, даны элементы эл\-лип\-соидального анализа 
распределений, выведены уравнения ЭУОФ по методам 
эллипсоидальной аппроксимации (МЭА) и~эллипсоидальной\linebreak линеаризации (МЭЛ).
В~\cite{6-s, 5-s} разработана теория аналитического синтеза МЭУОФ 
на основе приближенного решения по МЭА (МЭЛ) фильтрационного уравнения для 
ненормированной апостериорной характеристической функции.

\columnbreak

Рассмотрим обобщение~\cite{4-s, 5-s} на случай  ЭУОФ.  В~основу 
рассмотрения положим соответствующие  уравнения для ненормированных апостериорных 
распределений.


\section{Точные фильтрационные уравнения для~апостериорного распределения}

Следуя~[5--7], будем рассматривать задачу фильт\-ра\-ции состояния систем, 
моделями которых могут служить
стохастические дифференциальные  уравнения, понимаемые в~смысле Ито. При 
этом стохастические дифференциальные урав\-не\-ния модели изучаемой системы могут
иметь неизвестные па\-ра\-мет\-ры и,~как правило, всегда содержат
параметры, известные с~ограниченной точ\-ностью. Поэтому возникает
задача непрерывного оценивания неизвестных па\-ра\-мет\-ров сис\-те\-мы
(точнее, ее модели) по результатам непрерывных наблюдений.
Предположим, что правые части урав\-не\-ний зависят от конечного множества 
неизвестных па\-ра\-мет\-ров, которые
будем рассматривать как компоненты вектора параметров~$\theta$.
Одним из возможных подходов в~таких случаях является следующий
прием: неизвестный векторный параметр~$\theta$ считают стохастическим
процессом  $\Theta \hm=\Theta_t$, который определяется
дифференциальным уравнением $\dot\Theta_t \hm=0$, и~включают
компоненты этого векторного процесса в~вектор со\-сто\-яния системы
(<<расширяют>> вектор состояния путем включения в~него неизвестных
параметров в~качестве дополнительных компонент).
Таким образом, задача непрерывного оценивания неизвестных
параметров модели системы сводится к~задаче непрерывного
оценивания состояния системы с~расширенным вектором состояния.
От неизвестных параметров могут зависеть и~уравнения наблюдения. 
Эти параметры следует включить в~вектор ~$\Theta$ и,~следовательно, 
в~расширенный вектор состояния.

Итак, пусть векторный стохастический процесс (СтП) 
$\lk X_t^{\mathrm{T}} Y_t^{\mathrm{T}} \rk^{\mathrm{T}}$
определяется системой векторных стохастических дифференциальных
уравнений Ито~[1--3]:
    \begin{multline}
    dX_t =\varphi \left(X_t,Y_t,\Theta, t\right) dt + 
    \psi' \left(X_t,Y_t,\Theta, t\right) dW_0 +{}\\
{}+\iii_{R_0^q} \psi''
    \left(X_t,Y_t,\Theta, t,v\right) P^0 (dt, dv)\,,\\
     X(t_0) = X_0\,;
    \label{e2.1-s}
    \end{multline}
    
    \vspace*{-12pt}
    
    \noindent
    \begin{multline}
        dY_t =\varphi_1 \left(X_t,Y_t,\Theta, t\right) dt +
    \psi_1' \left(X_t,Y_t,\Theta, t\right) dW_0 + {}\\
{}+\int\limits_{R_0^q} \psi_1'' (X_t,Y_t,\Theta, t,v) P^0
    (dt,dv)\,,\\
     Y(t_0) = Y_0\,.
    \label{e2.2-s}
    \end{multline}
Здесь $Y_t=Y(t)$~--- $n_y$-мер\-ный наблюдаемый
СтП, $Y_t \hm\in \Delta^y$ ($\Delta^y$~--- гладкое многообразие наблюдений);
 $X_t \hm=X(t)$~--- $n_x$-мер\-ный ненаблюдаемый
СтП (вектор состояния), $X_t \hm\in \Delta^x$ ($\Delta^x$~--- 
гладкое многообразие состояний); $W_0\hm =W_0(t)$~--- $n_w$-мер\-ный
винеровский СтП $(n_w\hm\ge n_y)$ интенсивности  $\nu_0 \hm=\nu_0 (\Theta, t)$; 
$P^0(\Delta,A)\hm=P(\Delta,A)\hm-\mu_P (\Delta,A)$, $P(\Delta,A)$  представляет 
собой для любого множества~$A$ прос\-той пуассоновский СтП, а~$\mu_P (\Delta,A)$~--- 
его математическое ожидание, причем
    $$
    \mu_P (\Delta,A)=\mm P (\Delta,A)=\iii_\Delta \nu_P(\tau, A)\, d\tau\,;
    $$
$\nu_P (\Delta, A)$~--- интенсивность соответствующего пуассоновского потока событий, 
$\Delta \hm=(t_1,t_2]$; интегрирование по~$v$ распространяется на все 
пространство~$R^q$ с~выколотым началом координат; $\Theta$~--- 
вектор случайных параметров размерности~$n_\Theta$;
$\varphi\hm=\varphi(X_t,Y_t,\Theta, t)$, $\varphi_1\hm=\varphi_1(X_t,Y_t,\Theta, t)$,
 $\psi'\hm=\psi'(X_t,Y_t,\Theta, t)$ и~$\psi_1'\hm=\psi_1'(X_t,Y_t,\Theta, t)$~--- 
 известные функции, отображающие
$R^{n_x}\times R^{n_y}\times  R$ соответст\-венно в~$R^{n_x}$,
$R^{n_y}$, $R^{n_xn_w}$ и~$R^{n_yn_w}$; $\psi''\hm=\psi''(X_t,Y_t,\Theta, t,v)$ 
и~$\psi_1''(X_t,Y_t,\Theta, t,v)$~--- известные
функции, отображающие $R^{n_x}\times R^{n_y}\times R^q$ 
в~$R^{n_x}$ и~$R^{n_y}$. Требуется найти оценку~$\hat X_t$ СтП~$X_t$ 
в~каж\-дый момент времени~$t$ по результатам наблюдения
СтП~$Y(\tau)$ до момента~$t$, $Y_{t_0}^t \hm=  \{ Y(\tau) \,: t_0 \hm\le \tau\hm< t\}$.

Предположим, что выполнены условия~[1--3]:
\begin{itemize}
\item  уравнение состояния имеет вид~(\ref{e2.1-s});

\item уравнение наблюдения~(\ref{e2.2-s}), во-пер\-вых, не содержит
пуассоновского шума $(\psi_1'' \hm\equiv 0)$, а~во-вто\-рых, 
коэффициент при винеровском шуме~$\psi_1'$  в~уравнениях наблюдения не зависит 
от со\-сто\-яния $(\psi_1' (X_t, Y_t,\Theta, t)\hm =\psi_1'(Y_t,\Theta, t))$.
\end{itemize}

В этом случае уравнения задачи нелинейной фильтрации имеют следующий вид:
\begin{multline}
    dX_t =\varphi \left(X_t, Y_t,\Theta, t\right)\,dt+\psi'\left(X_t, Y_t,\Theta, t\right)\,dW_0+{}\\
{}+\int\limits_{R_0^q} \psi''\left(X_t, Y_t,\Theta, t,v\right) P^0 (dt, dv)\,,\\  
X\left(t_0\right) = X_0\,;
\label{e2.3-s}
\end{multline}

\vspace*{-12pt}

\noindent
\begin{multline}
    dY_t =\varphi_1 \left(X_t, Y_t,\Theta, t\right)\,dt +
    \psi_1 \left(Y_t,\Theta, t\right) dW_0\,,\\ 
    Y\left(t_0\right) = Y_0\,.
    \label{e2.4-s}
    \end{multline}
    

Предположим, что выполнены условия существования и~единственности СтП, 
определяемого~(\ref{e2.3-s}) и~(\ref{e2.4-s}) при соответствующих начальных 
условиях~[7--9].

Как известно~\cite{6-s, 9-s}, для любых СтП~$X_t$ и~$Y_t$ оптимальная оценка~$\hat X^t$, 
минимизирующая средний квадрат ошибки в~каждый момент времени~$t$, 
представляет собой апостериорное математиче-\linebreak ское ожидание СтП~$X_t$: 
$\hat X_t\hm = \mm \lk X_t \mid Y_{t_0}^t\rk$. Чтобы\linebreak найти
 это условное 
математическое ожидание, необходи\-мо знать $p_t \hm= p_t (x)$ и~$g_t\hm=g_t(\lambda)$~--- 
апостериорную одномерную плот\-ность и~характеристи\-ческую функ\-цию распределения 
СтП~$X_t$. Со\-от\-вет\-ст\-ву\-ющие точные уравнения линейной фильтрации приведены в~\cite{3-s}.

Введем ненормированные одномерные апостериорные плотность $\tilde p_t (x,\Theta)$  
и~характеристическую функцию $\tilde g_t (\lambda,\Theta)$ согласно формулам:
    \begin{align*}
    \tilde{p}_t (x,\Theta) &= \mu_t p_t (x,\Theta)\,; \\
    \tilde{g}_t (\lambda,\Theta) &= 
    \mm_{\Delta^x}^{p_t} \left[ e^{i\lambda^{\mathrm{T}} X_t} \mu_t\right] =
    \mu_t g_t (\lambda,\Theta)\,.
    %    \label{e2.5-s}
    \end{align*}
Тогда, обобщая~\cite{6-s} на случай уравнений~(\ref{e2.3-s}) и~(\ref{e2.4-s}), 
получим следующее точное уравнение с.к.\ 
оптимальной нелинейной фильт\-рации:

\noindent
    \begin{multline}
    d\tilde{g}_t (\lambda,\Theta) =
    \mm_{\Delta^x}^{\tilde{p}_t} \Big\{
     \Big[ 
i \lambda^{\mathrm{T}} \varphi 
    \left( X,Y_t,\Theta, t\right) -{}\\
    {}- \fr{1}{2} \left(\psi'\nu_0{\psi'}^{\mathrm{T}}\right) 
    \left( X,Y_t,\Theta, t\right)+{}\\
{}+ \int\limits_{R_0^q} \left[ e^{i\lambda^{\mathrm{T}} \psi'' 
\left( X,Y_t,\Theta, t,v\right)} -1- {}\right.\\
\left.{}-i\lambda^{\mathrm{T}} \psi'' 
\left( X,Y_t,\Theta, t,v\right)
\vphantom{e^{i\lambda^{\mathrm{T}} \psi'' 
\left( X,Y_t,\Theta, t,v\right)}}
\right] \nu_P (\Theta, t, dv)\Big] 
e^{i\lambda^{\mathrm{T}} X} \Big\}\,dt +{}\\
  {}+ \mm_{\Delta^x}^{\tilde{p}_t} \Big\{ \Big[ 
  \varphi_1 \left( X,Y_t,\Theta, t\right)^{\mathrm{T}} + {}\\
  {}+
  i\lambda^{\mathrm{T}} \left(\psi'\nu_0 {\psi'}^{\mathrm{T}}\right) 
  \left( X,Y_t,\Theta, t\right)\Big] 
  e^{i\lambda^{\mathrm{T}} X}\Big\} \times{}\\
  {}\times
  \left(\psi'\nu_0{\psi'}^{\mathrm{T}}\right)^{-1}
  \left( Y_t,\Theta, t\right)\, dY_t\,.
  \label{e2.6-s}
  \end{multline}

Если, следуя~\cite{8-s}, функция $\psi''$ в~(\ref{e2.3-s}) допускает представление
\begin{equation*}
\psi'' = \psi' \omega (\Theta, v)\,,
%\label{e2.7-s}
\end{equation*}
где  $P^0 (\Delta, A) \hm= P^0 ((0,t], dv)$, то уравнения~(\ref{e2.3-s}) 
и~(\ref{e2.4-s}) примут следующий вид:
\begin{multline}
\dot X_t =\varphi \left(X_t, Y_t, \Theta, t\right)+\psi' 
\left(X_t, Y_t, \Theta, t\right)V(\Theta, t)\,,\\ X\left(t_0\right)=X_0\,;
\label{e2.8-s}
\end{multline}

\vspace*{-12pt}

\noindent
\begin{multline}
\dot Y_t = \varphi\left(X_t, Y_t, \Theta, t\right)+\psi_1 
\left(Y_t, \Theta, t\right) V_0 (\Theta, t)\,,\\ 
Y\left(t_0\right) = Y_0\,.
\label{e2.9-s}
\end{multline}
Здесь
$V_0 (\Theta, t) \hm=\dot W_0 (\Theta, t)$; $V (\Theta, t)\hm=\dot{\bar W}  (\Theta, t)$,
\begin{equation*}
\bar W (\Theta, t) = W_0  (\Theta, t)+ \iii_{R_0^q} \omega 
(\Theta, v) P^0 ((0,t],dv)\,,
%\label{e2.10-s}
\end{equation*}
где $\nu_P  (\Theta, t,v) dv\hm= \lk \prt \mu  (\Theta, t,v)/\prt t\rk dv$~--- 
интенсивность пуассоновского потока скачков, равных $\omega (\Theta, t)$.
При этом логарифмические производные от одномерных характеристических
 функций определяются известными формулами:
\begin{align*}
\chi^{W_0} (\rho; \Theta, t)& = -\fr{1}{2}\, \rho^{\mathrm{T}} 
    \nu_0  (\Theta, t) \rho\,,
\\
    \chi^{\bar W} (\rho;\Theta,t) &= - \fr{1}{2}\, \rho^{\mathrm{T}}  
    (\Theta, t) \rho^{\mathrm{T}}+ {}\\
    &\hspace*{-35pt}{}+\iii_{R_0^q} \left[ e^{i\rho^{\mathrm{T}} \omega 
    (\Theta, v)} -1- i\rho^{\mathrm{T}} \omega 
    (\Theta, v)\right] \nu_P  (\Theta, t,v)\,dv\,.
    %\label{e2.11-s}
    \end{align*}
В таком случае интегральный член в~(\ref{e2.6-s}) допускает следующую запись:
\begin{multline}
    \gamma = \iii_{R_0^q} \Big[ e^{ i\lambda^{\mathrm{T}} \psi'' 
    \left(X_t, Y_t, \Theta, t\right)\omega(\Theta, v)} - 1- {}\\
    {}-i\lambda^{\mathrm{T}} 
    \psi'' \left(X_t, Y_t, \Theta, t\right)\omega 
    (\Theta, v)\Big]\nu_P  (\Theta, t,v)\,dv\,.
    \label{e2.12-s}
    \end{multline}

Очевидно, что для гауссовской МСтС $\gamma \hm\equiv 0$. Тогда приходим к~известным 
утверждениям~\cite{6-s, 8-s, 10-s}.

\smallskip

\noindent
\textbf{Теорема~1.}\ \textit{Пусть для негауссовской МСтС}~(\ref{e2.3-s}), 
(\ref{e2.4-s}) \textit{выполнены условия существования и~един\-ст\-вен\-ности.  Тогда 
уравнение с.к.\ оптимальной нелинейной фильт\-ра\-ции для ненормированной 
характеристической функции $\tilde{g}_t (\lambda, \Theta)$ имеет вид}~(\ref{e2.6-s}).


\smallskip

\noindent
\textbf{Теорема~2.}\ \textit{Пусть для гауссовской МСтС}~(\ref{e2.8-s}), 
(\ref{e2.9-s}) \textit{выполнены условия существования и~един\-ст\-вен\-ности.
 Тогда уравнение с.к.\ оптимальной нелинейной фильт\-ра\-ции для ненормированной 
 характеристической функции имеет вид}~(\ref{e2.6-s}) \textit{при условии}~(\ref{e2.12-s}).
 
 \smallskip


Как известно~\cite{6-s}, необходимость обработки результатов наблюдений в~реальном
 масштабе времени непосредственно в~процессе эксперимента\linebreak
  привела
 к~появлению ряда приближенных методов оптимальной нелинейной  фильтрации,
 на\-зы\-ва\-емых обычно методами  услов\-но-оп\-ти\-маль\-ной фильт\-ра\-ции. В~этом случае 
 для приближенного ре-\linebreak шения уравнения  для апостериорной
одномерной характеристической функции  $g_1(\lambda, \Theta)$ вектора~$X_t$ можно
использовать методы аналитического моделирования, основанные на параметризации
 одномерных  распределений СтП, опре\-де\-ля\-емо\-го стохастическим
 дифференциальным уравнением.  Эти\linebreak методы
 позволяют изучить  стохастические дифференциальные урав\-не\-ния для па\-ра\-мет\-ров
 апостериорного распределения. Простейшим таким\linebreak методом является
 МНА апостериорного распределения.
Исключительно важное практическое значение имеют квазилинейные
фильтры, получаемые с~помощью методов эквивалентной линеаризации~\cite{6-s}.
Эллипсоидальные УОФ, основанные на приближенном 
решении уравнений для нормированных апостериорных распределений, рассмотрены в~\cite{3-s}, 
а~для ненормированных~--- в~\cite{5-s}.

Наряду с~методами субоптимальной фильтрации  широкое применение нашли методы, 
средства и~информационные технологии, основанные на принципах УОФ  В.\,С.~Пугачёва~\cite{6-s}.

\vspace*{-3pt}

\section{Уравнения субоптимальной фильтрации}

\vspace*{-2pt}


Следуя~\cite{5-s}, аппроксимируем ненормированную
апостериорную плотность вероятности формулой
\begin{equation}
\tilde p_t (x) \approx p_t^* (u) =w(u) \left[ \mu+ \sss_{\nu=1}^N
    c_\nu p_\nu (u)\right]\,.
    \label{e3.1-s}
    \end{equation}
Имеем
\begin{equation}
c_\nu =\mu_t {\sf M}^\raa \left[ q_\nu (U_t)\right] =\left[ q_\nu (U_\lambda) \tilde
    g_t^\raa (\la) \right]_{\lambda=0}\,,
    \label{e3.2-s}
    \end{equation}
причем

\pagebreak

\noindent
    \begin{align*}
    u &= \left(x^{\mathrm{T}} -\hat X_t^{\mathrm{T}}\right) 
    C_t\left( x-\hat X_t\right)\,;\\
    U_t  &= \left(X_t^{\mathrm{T}} -\hat X_t^{\mathrm{T}}\right)
    C_t\left(X_t-\hat X_t\right)\,;\\
    U_\lambda &= \left(\fr{\partial^{\mathrm{T}}}{i\partial\lambda} -\hat X_t\right)
    C_t\left(\fr{\partial}{i\partial \lambda} - \hat X_t\right)\,.
    %    \label{e3.3-s}
    \end{align*}


Как известно~\cite{3-s, 5-s}, для того чтобы составить стохастические 
дифференциальные уравнения
для коэффициентов~$c_\nu$, надо найти
стохастический дифференциал Ито произведения $q_\chi (u) \tilde
g_t (\lambda)$, имея в~виду, что $u$ зависит от $\hat X_t\hm = m_t/\mu_t$
и~что~$m_t$ и~$\mu_t$ определяются стохастическими
дифференциальными уравнениями. Потом следует заменить переменные~$x$ 
и~$u$ операторами $\partial/(i\partial\lambda)$ и~$U_\lambda$,
выполнить дифференцирование и~после этого положить $\lambda\hm=0$.

Повторяя~\cite{5-s}, сначала получим уравнения для~$m_t$ и~$\mu_t$ 
с~функцией~$\hat \varphi_1$, определяемой формулой
\begin{equation*}
\hat \varphi_1 =\mm_{\Delta^x}^{p_t} \lk \varphi_1\rk\,;
%\label{e3.4-s}
\end{equation*}
c учетом обозначений
  \begin{equation*}
    \sigma_0 = \psi\nu_0 \psi^{\mathrm{T}}\,; \enskip
    \sigma_1 = \psi\nu_0 \psi_1^{\mathrm{T}}\,; \enskip 
    \sigma_2 = \psi_1\nu_0 \psi_1^{\mathrm{T}}
%    \label{e3.5-s}
    \end{equation*}
их можно представить в~виде:
\begin{multline}
    dm_t = \left[
    \vphantom{    \sss_{\nu=1}^N}
    \mu_t f_0 \left(Y_t,\hat X_t,\Theta, t\right) +{}\right.\\
\left.    {}+
    \sss_{\nu=1}^N \hspace*{-1.4261pt}c_\nu f_\nu \left(Y_t,\hat     X_t,\Theta, t\right)\right] dt+
\left[ \vphantom{    \sss_{\nu=1}^N}
    \mu_t h_0 \left(Y_t,\hat X_t,\Theta, t\right) +{}\right.\\
\left.    {}+\sss_{\nu=1}^N c_\nu
    h_\nu \left(Y_t,\hat X_t,\Theta, t\right) 
          \right] dY_t\,;
    \label{e3.6-s}
    \end{multline}
    
    \vspace*{-12pt}
    
    \noindent
    \begin{multline}
    d\mu_t =\left[
    \vphantom{\sum\limits^N_{\nu=1}}
    \mu_t b_0 \left(Y_t,\hat
    X_t,t\right)+{}\right.\\
\left.    {}+\sss_{\nu=1}^N c_\nu b_\nu \left(Y_t,\hat X_t,\Theta, t\right) \right]
\, dY_t\,,
\label{e3.7-s}
\end{multline}
где
   \begin{equation}
   \left.
   \begin{array}{rl}
   f_0&=f_0 \left(Y_t,\hat X_t,,\Theta, t\right) =
   \mm_{\Delta^x}^{w} \left[ \varphi\right]\,;\\[6pt]
    f_\nu&=f_\nu \left(Y_t,\hat X_t,,\Theta, t\right) =
    \mm_{\Delta^x}^{w p_\nu} \left[ \varphi\right]\,;
    \end{array}
    \right\}
    \label{e3.8-s}
    \end{equation}
      \begin{equation}
\left.
\begin{array}{rl}
    h_0&=h_0 \left(Y_t,\hat X_t,\Theta, t\right) ={}\\[6pt]
    &\hspace*{-6mm}{}=
    \mm_{\Delta^U}^{w} \!\lk \sigma_1 \left(Y_t,\Theta, t\right)+
    X\varphi_1 \left(X,Y_t,\Theta, t\right)^{\mathrm{T}}\rk\times{}\\[6pt]
    &\hspace*{35mm}{}\times \sigma_2 \left(Y_t,\Theta,t\right)^{-1}\,;\\[6pt]
    h_\nu&= h_\nu \left(Y_t,\hat X_t,\Theta, t\right)={}\\[6pt]
    &\hspace*{-9mm}{}=\mm_{\Delta^U}^{w p_\nu}  \!\lk \sigma_1 \left(X,Y_t,\Theta, t\right)+ X\varphi_1
   \left( X,Y_t,\Theta, t\right)^{\mathrm{T}}\rk\!\times\!{}\\[6pt]
   &\hspace*{35mm}{}\times  \sigma_2 \left(Y_t,\Theta,t\right)^{-1};
   \end{array}\!\!
   \right\}\!\!
   \label{e3.9-s}
   \end{equation}
   \begin{equation}
   \left.
   \begin{array}{rl}
    b_0&= b_0 \left(Y_t,\hat X_t,\Theta, t\right)
    ={}\\[6pt]
    &\hspace*{-5mm}{}=\mm_{\Delta^U}^{w} \lk \varphi_1 \left(X,Y_t,\Theta, t\right)^{\mathrm{T}} \rk 
    \sigma_2\left(Y_t,\Theta,t\right)^{-1}\,;\\[6pt]
    b_\nu&=b_\nu \left(Y_t,\hat X_t,\Theta, t\right)
={}\\[6pt]
&\hspace*{-5mm}{}=\mm_{\Delta^U}^{w p_\nu} \lk \varphi_1\left(X,Y_t,\Theta, t\right) \rk \sigma_2
\left(Y_t,\Theta,t\right)^{-1}\,.
\end{array}
\right\}
\label{e3.10-s}
\end{equation}

Далее запишем уравнения~(\ref{e3.6-s}) и~уравнение для одномерной ненормированной
характеристической функции в~виде:

\noindent
  \begin{gather*}
    dm_t = f dt + h\, dY_t\,;\quad d\mu_t = b\, dY_t\,; %\label{e3.11-s}
    \\
d\tilde g_t = A\, dt + B\,dY_t\,. %\label{e3.12-s}
\end{gather*}
Здесь обозначено:

\noindent
   \begin{gather*}
    f=\mu_t f_0
    +\sss_{\nu=1}^N c_\nu f_\nu\,;\quad h= \mu_t h_0 +\sss_{\nu=1}^N
    c_\nu h_\nu\,; \\
    b= \mu_t b_0 +\sss_{\nu=1}^N c_\nu b_\nu\,;
    \end{gather*}
    
    \vspace*{-14pt}
    
    \noindent
    \begin{multline*}
    A=\mm_{\Delta^x}^{\tilde{p}_t}  \biggl\{
    i\lambda^{\mathrm{T}} \varphi \left(X,Y_t,\Theta,t\right) -{}\\
    {}-\fr{1}{2}\,
    \lambda^{\mathrm{T}} \left(\psi' \nu_0 {\psi'}^{\mathrm{T}}\right) 
    \left(X,Y_t,\Theta,t\right) \lambda\times{}\\
{}\times \iii_{R_0^q} \bigg[ e^{i\lambda^{\mathrm{T}} \psi'' 
\left(X,Y_t,\Theta,t,v\right)} -1- {}\\
{}-i\lambda^{\mathrm{T}}
    \psi'' \left(X,Y_t,\Theta,t,v\right)\bigg] \nu_P (t,dv) 
    e^{i\lambda^{\mathrm{T}} X}\biggr\}\,;
    \end{multline*}
    
 \vspace*{-14pt}
    
    \noindent
    \begin{multline*}
    B=\mm_{\Delta^x}^{\tilde{p}_t}  \bigg[ 
    \varphi_1 \left(X,Y_t,\Theta,t\right)^{\mathrm{T}} + {}\\
    {}+
    i\lambda^{\mathrm{T}} \left(\psi'\nu_0 {\psi'}_1^{\mathrm{T}}\right) 
    \left(X,Y_t,\Theta,t\right)\bigg]\times{}\\
{}\times e^{i\lambda^{\mathrm{T}} X} \left(\psi_1^\prime \nu_0 {\psi'}_1^{\mathrm{T}}\right)^{-1} 
\left(X,Y_t,\Theta,t\right)\,.
%    \label{e3.13-s}
    \end{multline*}

Дифференциальные уравнения для коэффициентов МОР в~(\ref{e3.1-s}) и~(\ref{e3.2-s}) 
в~силу~\cite{5-s} имеют следующий вид:

\noindent
    \begin{multline*}
    \hspace*{-3.42464pt}\!\!dc_\chi =\mm_{\Delta^x}^{p^*} \!\left\{ 
    \vphantom{\fr{\mathrm{tr}\!
 \left[ \!\left(h+ \hat X_t b\right) \!
\sigma_1^{\mathrm{T}} C_t\!\left(\!X-\hat X_t\right)\!
    \left(\!X^{\mathrm{T}}-\hat X_t^{\mathrm{T}}\right)\! C_t\right]}{\mu_t}}\!
    q_\chi^\prime(u) \left(2\varphi^{\mathrm{T}} C_t\!\left(
    X-\hat X_t\right) + \mathrm{tr} \left[ C_t\sigma_0\right]\right) +{}\right.\\
{}+2q_\chi^{\prime\prime} (u) \left(X^{\mathrm{T}} -\hat X_t^{\mathrm{T}}\right) 
C_t\sigma_0 C_t\left(X-\hat X_t\right)+{}\\
\hspace*{1mm}{}+\iii_{R_0^q}\! \left[ q_\chi\left(\bar u\right) -q_\chi (u) - 2 q_\chi' (u) 
\left(X^{\mathrm{T}}-\hat X_t^{\mathrm{T}}\right) C_t\psi''\right] \times{}\\
\hspace*{1.5mm}{}\times \nu_P (t, dv) - 
q_\chi' (u) \left(X^{\mathrm{T}}-\hat X_t^{\mathrm{T}}\right) 
C_t\left(h+\hat X_t b\right) \fr{\varphi_1}{\mu_t} +{}\\
{}+q_\chi^\prime(u) \fr{\mathrm{tr}\, \left[ \left(h+
    \hat X_t b\right) \sigma_1^{\mathrm{T}} C_t\right]}{\mu_t} +
    2q_\chi^{\prime\prime}(u) \times{}
\end{multline*}

\noindent
\begin{multline}
\hspace*{-3pt}\left.\!\!\!\hspace*{-8.33337pt}{}\times \fr{\mathrm{tr}\!
 \left[ \!\left(h+ \hat X_t b\right) \!
\sigma_1^{\mathrm{T}} C_t\!\left(\!X-\hat X_t\right)\!
    \left(\!X^{\mathrm{T}}-\hat X_t^{\mathrm{T}}\right)\! C_t\right]}{\mu_t} \!\!\right\}dt +{}\\
{} + \biggl\{ \fr{1}{2n} \left(c_{\chi-1} + 2\chi
    c_\chi\right) \mathrm{tr} \left[ \dot C_t K_t\right]+
     \fr{c_{\chi-1}}{2n} \times{}\\
     {}\times\fr{\mathrm{tr} \left[ C_t h\sigma_2 h^{\!\mathrm{T}}\right]\! -\! 2
    \hat X_t^{\mathrm{T}} C_t h \sigma_2 b^{\mathrm{T}} 
+
    \hat X_t^{\mathrm{T}} C_t\hat X_t b \sigma_2
    b^{\mathrm{T}}}{\mu_t^2}\!\biggr\} dt+ {}\\
    {}+
    \mm_{\Delta^x}^{p^*}\left\{\left[
    \vphantom{    \left(X^{\mathrm{T}}-\hat X_t^{\mathrm{T}}\right)}
     q_\chi (u)
    \varphi_1^{\mathrm{T}} +{}\right.\right.\\
\left.\left.    {}+ q_\chi^\prime(u) 
    \left(X^{\mathrm{T}}-\hat X_t^{\mathrm{T}}\right) C_t\sigma_1 \right]  
    \sigma_2^{-1}\right\} dY_t\,.
    \label{e3.14-s}
    \end{multline}

Примем в~дополнение к~обозначениям~(\ref{e3.8-s})--(\ref{e3.10-s}) следующие:
\begin{multline*}
    \gamma_{\chi 0}=
    \gamma_{\chi 0} \left(Y_t,\hat X_t,\Theta, t\right) ={}\\
    {}=
    \mm_{\Delta^x}^{w} \biggl\{ q_\chi^\prime(u) \left(2\varphi 
    \left(X,Y_t,\Theta,t\right)^{\mathrm{T}} C_t\left(X-\hat X_t\right)+{}\right.\\
    \left.{}+\mathrm{tr} 
    \left[ C_t\sigma_0
    \left(X,Y_t,\Theta,t\right)\right] 
    \vphantom{\left(X,Y_t,\Theta,t\right)^{\mathrm{T}}}\!\right)+{}\\
{}+ 2q_\chi^{\prime\prime}(u)\!\left(\!X^{\mathrm{T}}-\hat X_t^{\mathrm{T}}\right)\! 
C_t\sigma_0 \left(X,Y_t,\Theta,t\right)  C_t\!\left(\!X-\hat X_t\right)+{}\\
{}+\iii_{R_0^q} \left[ q_\chi \left(\bar u\right) -q_\chi (u) - 2
    q_\chi^\prime(u) \left(X^{\mathrm{T}}-\hat X_t^{\mathrm{T}}\right) \times{}\right.\\
    \left.{}\times
    C_t\psi'' \left(X,Y_t,\Theta,t,v\right)
    \vphantom{\left(X^{\mathrm{T}}-\hat X_t^{\mathrm{T}}\right)}
    \right] \nu_P (t,dv)\biggr\} \,;
    \end{multline*}
    
\vspace*{-12pt}

    \noindent
    \begin{multline*}
        \gamma_{\chi \nu}=
    \gamma_{\chi \nu} \left(Y_t,\hat X_t,\Theta,t\right) ={}\\
    {}=\mm_{\Delta^x}^{w p_\nu} \biggl\{
    q_\chi'(u) \left(2\varphi \left(X,Y_t,\Theta,t\right)^{\mathrm{T}} 
    C_t\left(X-\hat X_t\right)+ {}\right.\\
\left.    {}+\mathrm{tr} \left[
    C_t\sigma_0 \left(X,Y_t,\Theta,t\right)\right] 
    \vphantom{\left(X-\hat X_t\right)}
    \!\right)+{}\\
{}+ 2q_\chi'' (u)\!\left(\!X^{\mathrm{T}}-\hat X_t^{\mathrm{T}}\right)\! C_t\sigma_0
\!    \left(\!X,Y_t,\Theta,t\right) C_t\!\left(\!X-\hat X_t\right)+{}\\
{}+\iii_{R_0^q} \left[ q_\chi \left(\bar u\right) -q_\chi(u) - 
2 q_\chi ' (u) \left(X^{\mathrm{T}}-\hat X_t^{\mathrm{T}}\right) \times{}\right.\\
\left.{}\times
C_t \psi'' \left(X,Y_t,\Theta,t,v\right)
    \vphantom{\left(X^{\mathrm{T}}-\hat X_t^{\mathrm{T}}\right)}
    \right] \nu_P (t,dv)\biggr\} \,;
%\label{e3.15-s}
\end{multline*}

\vspace*{-12pt}

    \noindent
    \begin{multline*}
    \varepsilon_{\chi 0}= \varepsilon_{\chi 0} \left(Y_t,\hat X_t,\Theta,t\right)
    ={}\\
    {}=\mm_{\Delta^x}^{w}\biggl\{ q_\chi '(u) \left[ \sigma_1 
    \left(X,Y_t,\Theta,t\right)^{\mathrm{T}}-{}\right.\\
\left.    {}
    -\varphi_1 \left(X,Y_t,\Theta,t\right) 
    \left(X^{\mathrm{T}}-\hat X_t^{\mathrm{T}}\right)\right]+{}\\
{}+ 2q_\chi'' (u)\sigma_1 \left(X,Y_t,\Theta,t\right)^{\mathrm{T}} 
C_t\!\left(\!X-\hat X_t\right)\!\left(X^{\mathrm{T}}-\hat X_t^{\mathrm{T}}\right)\!\biggr\}\,;
\hspace*{-4.18732pt}
\end{multline*}

\vspace*{-12pt}

    \noindent
    \begin{multline*}
    \varepsilon_{\chi \nu}= \varepsilon_{\chi \nu} \left(Y_t,\hat X_t,\Theta,t\right)
    ={}\\
    {}=\mm_{\Delta^x}^{w p_\nu} \biggl\{ q_\chi'(u) 
    \left[ \sigma_1 \left(X,Y_t,\Theta,t\right)^{\mathrm{T}}
    -{}\right.
    \end{multline*}
    
\noindent
    \begin{multline*}
\left.    {}-\varphi_1\left(X,Y_t,\Theta,t\right)\left(X^{\mathrm{T}}-\hat X_t^{\mathrm{T}}
    \right)\right]+{}\\
{}+ 2q_\chi'' (u)\sigma_1 \!\left(X,Y_t,\Theta,t\right)^{\mathrm{T}} C_t\!
\left(X-\hat X_t\right)\!\left(X^{\mathrm{T}}-\hat X_t^{\mathrm{T}}\right)\biggr\};\hspace*{-4.18732pt}
%\label{e3.16-s}
\end{multline*}


    \noindent
    \begin{multline*}
    \eta_{\chi 0} = \eta_{\chi 0} \left(Y_t,\hat X_t,\Theta,t\right)
    ={}\\[-2pt]
    {}=\mm_{\Delta^x}^{w}\biggl\{ q_\chi (u) \varphi_1 \left(X,Y_t,\Theta,t\right)^{\mathrm{T}} 
    + q_\chi' (u)\left(X^{\mathrm{T}}-\hat X_t^{\mathrm{T}}\right)\times{}\\[-2pt]
    {}\times C_t \sigma_1 \left(X,Y_t,\Theta,t\right)\biggr\}  
    \sigma_2 \left(Y_t,\Theta,t\right)^{-1}\,;
    \end{multline*}
    
    \vspace*{-12pt}
   
   \noindent
    \begin{multline*}
    \eta_{\chi \nu} =
    \eta_{\chi \nu} \left(Y_t,\hat X_t,\Theta,t\right) ={}\\[-2pt]
    {}=
    \mm_{\Delta^x}^{w p_\nu} \biggl\{ q_\chi(u) \varphi_1 
    \left(X,Y_t,\Theta,t\right)^{\mathrm{T}} + 
    q_\chi' (u)\left(X^{\mathrm{T}}-\hat X_t^{\mathrm{T}}\right)\times{}\hspace*{-1.52713pt}\\
{}\times C_t \sigma_1 \left(X,Y_t,\Theta,t\right)\biggr\} 
\sigma_2 \left(Y_t,\Theta,t\right)^{-1}\,.
%\label{e3.17-s}
\end{multline*}
Тогда можем переписать уравнения~(\ref{e3.14-s}) в~виде:
\begin{multline*}
d c_\chi =\left\{
\vphantom{\left(\fr{b_nu\left(Y_t, \hat{X}_t\right)^T}{\mu_t}\right)}
    \mu_t\gamma_{\chi 0} \left(Y_t,\hat X_t,\Theta,t\right) + {}\right.\\[-2pt]
    {}+
    \sss_{\nu=1}^N c_\nu \gamma_{\chi\nu} \left(Y_t,\hat X_t,\Theta,t\right)+{}\\[-2pt]
{}+ \mathrm{tr}\,\left[ 
\vphantom{\fr{ c_\nu \varepsilon_{\chi\nu}
    \left(Y_t,\hat X_t,\Theta,t\right)}{\mu_t}} 
\mu_t\left(h_0 \left(Y_t,\hat X_t,\Theta,t\right)+\hat X_t b_0 \left(Y_t,\hat
    X_t,\Theta,t\right)\right)+{}\right.\\[-2pt]
{}+\sss_{\nu=1}^N c_\nu \left(h_\nu \left(Y_t,\hat X_t,\Theta,t\right) +\hat X_t
    b_\nu \left(Y_t,\hat X_t,\Theta,t\right)\right)\times{}\\[-2pt]
{}\times \left\{ 
\vphantom{\fr{ c_\nu \varepsilon_{\chi\nu}
    \left(Y_t,\hat X_t,\Theta,t\right)}{\mu_t}}
    \varepsilon_{\chi 0}
    \left(Y_t,\hat X_t,\Theta,t\right) +{}\right.\\[-2pt]
\left.\left.    {}+
    \sss_{\nu=1}^N \fr{ c_\nu \varepsilon_{\chi\nu}
    \left(Y_t,\hat X_t,\Theta,t\right)}{\mu_t}\right\} C_t\right]+{}\\[-2pt]
    {}+ \fr{1}{2n} \left(c_{\chi-1} +2\chi c_\chi\right)\mathrm{tr} \left[ \dot C_t
    K_t\right]+{}\\[-2pt]
    {}+ \fr{c_{\chi-1}}{2n}\,\mathrm{tr} \left[ 
\vphantom{\fr{c_\nu h_\nu \left(Y_t,\hat
    X_t,\Theta,t\right)^{\mathrm{T}}}{\mu_t}}
C_t\left(
\vphantom{\fr{c_\nu h_\nu \left(Y_t,\hat
    X_t,t\right)}{\mu_t}}
h_0
    \left(Y_t,\hat X_t,\Theta,t\right)+{}\right.\right.\\[-2pt]
    \left. {}+\sss_{\nu=1}^N \fr{c_\nu h_\nu \left(Y_t,\hat
    X_t,t\right)}{\mu_t}\right) \times{}\\[-2pt]
    {}\times
     \sigma_2 \left(Y_t,\Theta,t\right)\left( h_0 \left(Y_t,\hat
    X_t,\Theta,t\right)^{\mathrm{T}}+{}\right.\\[-2pt]
    \left.\left.    {}+\sss_{\nu=1}^N \fr{c_\nu h_\nu \left(Y_t,\hat
    X_t,\Theta,t\right)^{\mathrm{T}}}{\mu_t}\right)\right]-{}
        \end{multline*}

\noindent
\begin{multline}
       {}-2 \hat X_t^{\mathrm{T}} C_t \left(
\vphantom{\fr{c_\nu h_\nu \left(Y_t,\hat X_t,\Theta,t\right)}{\mu_t}}
h_0 \left(Y_t,\hat X_t,\Theta,t\right)
    +{}\right.\\
\left.    {}+ \sss_{\nu=1}^N
    \fr{c_\nu h_\nu \left(Y_t,\hat X_t,\Theta,t\right)}{\mu_t}\right)
    \sigma_2 \left(Y_t,\Theta,t\right)\times{}\\
{}\times  \left( b_0 
\left(Y_t,\hat X_t,\Theta,t\right)^{\mathrm{T}} +\sss_{\nu=1}^N
   \fr{c_\nu b_\nu \left(Y_t,\hat X_t,\Theta,t\right)^{\mathrm{T}}}{\mu_t}\right)+{}\\
{}+\hat X_t^{\mathrm{T}} C_t \hat X_t \left( 
\vphantom{\fr{c_\nu h_\nu \left(Y_t,\hat X_t,\Theta,t\right)}{\mu_t}}
b_0 
\left(Y_t,\hat X_t,\Theta,t\right)+{}\right.\\
\left.{}+\sss_{\nu=1}^N c_\nu
   \fr{b_\nu \left(Y_t,\hat X_t,\Theta,t\right)}{\mu_t}\right)\sigma_2 \left(Y_t,\Theta,t\right)\times{}\\
\left.{}\times  \left( \!
b_0 \left(Y_t,\hat X_t,\Theta,t\right) +\!\sss_{\nu=1}^N \!\!c_\nu
\fr{b_\nu \left(Y_t,\hat X_t,\Theta,t\right)^{\!\mathrm{T}}}{\mu_t}\!\right)\!\! \right\} dt+{}\hspace*{-.57072pt}\\
{}+\left\{ \mu_t\eta_{\chi 0} \left(Y_t,\hat X_t,\Theta,t\right) +
 \sss_{\nu=1}^N \!c_\nu
    \eta_{\chi\nu} \left(Y_t,\hat X_t,\Theta,t\right)\!\right\}  dY_t\hspace*{-1.78561pt}\\
     (\chi =1\tr     N)\,.
    \label{e3.18-s}
    \end{multline}

Уравнения~(\ref{e3.6-s}), (\ref{e3.7-s}), (\ref{e3.18-s}) и~соотношение $\hat X_t
\hm= m_t/\mu_t$ при начальных условиях
\begin{equation}
\left.
\begin{array}{c}
m\left(t_0\right) =\mathrm{M}\left[ X_0\mid Y_0\right]\,;\enskip 
\mu\left(t_0\right) = 1\,;\\[6pt] 
c_\chi \left(t_0\right) =c_{\chi 0}\enskip (\chi =1\tr  N),
\end{array}
\right\}
\label{e3.19-s}
\end{equation}
где $c_{\chi 0}$ $(\chi\hm=1\tr N)$~--- коэффициенты разложения~(\ref{e3.1-s}) 
условной плотности вероятности $\tilde p_{t_0} (x) \hm= p_0 (x\mid Y_0)$ 
вектора~$X_0$ относительно~$Y_0$, определяют
МЭСОФ.

После решения уравнений~(\ref{e3.7-s}) и~(\ref{e3.18-s})  с.к.\ оптимальная оценка
вектора состояния и~ковариационная матрица ошибки фильтрации 
в~МЭСОФ определяют\-ся по следующим приближенным формулам:
    \begin{align}
    \hat X_t &= \fr{m_t}{\mu_t}\,;\label{e3.20-s}\\
    R_t &= \mm_{\Delta^x}^{w} \left[\left( X-\fr{m_t}{\mu_t}\right) 
    \left( X^{\mathrm{T}}-\fr{m_t^{\mathrm{T}}}{\mu_t}\right)\right] +{}\notag\\
   &\hspace*{-5mm}{}+ \sss_{\nu=1}^N \fr{c_\nu}{\mu_t} \mm_{\Delta^x}^{w p_\nu}\left[
    \left( X-\fr{m_t}{\mu_t}\right) \left( X^{\mathrm{T}}-
    \fr{m_t^{\mathrm{T}}}{\mu_t}\right) \right]\,.
    \label{e3.21-s}
    \end{align}

Порядок МЭСОФ особенно при большой размерности~$n$~вектора состояния
системы значительно ниже порядка других УОФ. 
Так, при учете в~разложении~(\ref{e3.7-s}) моментов до десятого
порядка, уже при $n\hm>3$,
 $N\hm=5$ имеем $n\hm+ N\hm+1 \hm\le n(n\hm+3)/2$.
При $n\hm>3$, $N\hm=5$ МЭУОФ имеет более низкий порядок, чем фильтры 
МНА, обобщенный фильтр Кал\-ма\-на--Бью\-си, фильтры
второго порядка и~гауссовский фильтр.

Таким образом, в~основе алгоритма модифицированной эллипсоидальной 
услов\-но-оп\-ти\-маль\-ной нелинейной фильтрации лежат следующие утверждения.

\smallskip

\noindent
\textbf{Теорема~3.}\
\textit{В~условиях теоремы~$1$, если МЭУОФ существует, то он 
определяется уравнениями}~(\ref{e3.6-s}), (\ref{e3.7-s}) и~(\ref{e3.18-s}) 
\textit{при условиях}~(\ref{e3.19-s})--(\ref{e3.21-s}).

\smallskip

\noindent
\textbf{Теорема~4.}\
\textit{В~условиях теоремы~$2$, если МЭУОФ существует, то он определяется 
уравнениями теоремы~$3$ при условиях}~(\ref{e2.12-s}).

\smallskip

Следуя~\cite{5-s} для приближенного анализа фильт\-ра\-ционных уравнений и~учитывая 
случайность параметров~$\Theta$, придем к~следующим уравнениям для функций 
чувствительности первого порядка~\cite{3-s}:
\begin{equation}
\left.
\begin{array}{rl}
\hspace*{-15mm}d\nabla^\Theta \hat X_s &= \nabla^\Theta A^{\hat X_s}\, dt + 
\nabla^\Theta B^{\hat X_s}\,dY_t\,,\\[6pt] 
&\hspace*{25mm}\nabla^\Theta B^{\hat X_s}(t_0) =0\,; \\[6pt]
\hspace*{-15mm}d\nabla^\Theta R_{sq} &= \nabla^\Theta A^{R_{sq}} \,dt + 
\nabla^\Theta B^{R_{sq}}\,dY_t\,, \\[6pt] 
&\hspace*{25mm}\nabla^\Theta R_{sq}(t_0) =0\,;\\[6pt]
\hspace*{-15mm}d\nabla^\Theta c_{\kappa} &= \nabla^\Theta A^{c_\kappa}\, dt + 
\nabla^\Theta B^{c_\kappa}\,dY_t\,,\\[6pt]
& \hspace*{25mm}\nabla^\Theta c_\kappa(t_0) =0\,.
    \end{array}
    \right\}\hspace*{-3mm}
    \label{e3.22-s}
    \end{equation}
Здесь процедура взятия производных осуществляется по всем входящим 
переменным, а~коэффициенты чувствительности вычисляются при\linebreak  $\Theta\hm=m^\Theta$. 
При этом предполагается малость дис\-пер\-сий по сравнению с~их математическими 
ожиданиями. Очевидно, что при дифференцировании по~$\Theta$ 
$(\nabla^\Theta \hm= \prt /\prt\Theta)$
порядок уравнений возрастает пропорционально числу производных. Аналогично 
составляются уравнения для элементов матриц вторых функций чувствительности.

Для оценки качества МЭСОФ, следуя~\cite{5-s}, при гауссовских~$\Theta$ 
с~математическим ожиданием~$m^\Theta$ и~ковариационной матрицей~$K^\Theta$ 
введем условную функцию потерь, допускающую квадратичную аппроксимацию:
\begin{multline}
\rho^{\hat X_s}=\rho^{\hat X_s}(\Theta) =
\rho \left(m^\Theta\right) +\sss_{ii=1}^{n^\Theta} \rho_i' 
\left(m^\Theta\right)\Theta_s^0+ {}\\
{}+\ss2\limits_{i,j=1} \rho_{ij}'' 
\left(m^\Theta\right)\Theta_i^0 \Theta_j^0\,,
\label{e3.23-s}
\end{multline}
а также показатель~$\varepsilon$
    \begin{equation}
    \varepsilon =\varepsilon_2^{1/4}\,,\label{e3.24-s}
    \end{equation}
где введено обозначение:

\noindent
  \begin{equation}
    \eps_2 = \mm^{\mathrm{ЭА}} \lk \rho (\Theta)^2\rk -\rho \left(m^\Theta\right)^2.
    \label{e3.25-s}
    \end{equation}
    Здесь
    \begin{multline*}
   \mm^{\mathrm{ЭА}} \lk \rho(\Theta)^2\rk = 
   \rho \left(m^\Theta\right)^2 +\rho' \left(m^\Theta\right)^{\mathrm{T}} K^\Theta 
   \rho'\left(m^\Theta\right)+ {}\\
   {}+2\rho\left(m^\Theta\right) 
   \mathrm{tr} \left[ \rho''\left(m^\Theta\right)K^\Theta\right]
+\left\{ \mathrm{tr} \left[ \rho'' \left(m^\Theta\right) 
K^\Theta\right] \right\}^2+{}\\
{}+2\mathrm{tr} 
\left[ \rho''\left(m^\Theta\right) K^\Theta\right]^2\,,
\end{multline*}
при этом функции $\rho'$ и~$\rho''$ по известным формулам определяются на
 основе первых и~вторых функций чувствительности.
Таким образом, в~основе оценки качества МЭСОФ, в~условиях теорем~3 и~4, 
лежат уравнения~(\ref{e3.22-s})--(\ref{e3.25-s}) при 
существовании соответствующих производных в~правых частях уравнений~(\ref{e3.22-s}) 
(\textbf{теорема~5}).

Изложенные выше методы синтеза МЭСОФ дают
принципиальную возможность получить фильтр, близкий к~оптимальному по
оценке с~любой степенью точности.
Чем выше максимальный порядок учитываемых моментов,
коэффициент ЭА, тем выше будет точность
приближения к~оптимальной оценке. Однако число уравнений,
опре\-де\-ля\-ющих параметры апостериорного одномерного эл\-лип\-со\-и\-даль\-но\-го 
распределения, быстро растет с~увеличением числа учитываемых па\-ра\-метров.


\vspace*{-4pt}

\section{Основные классы модифицированных эллипсоидальных условно-оптимальных
фильтров}

\vspace*{-2pt}


\textbf{4.1.}\ Рассмотрим задачу услов\-но-оп\-ти\-маль\-ной фильтрации, когда 
требуется найти оптимальную оценку~$\hat X_t$  процесса~$X_t$ в~момент  $t\hm>t_0$ 
по результатам наблюдения этого процесса до момента~$t$ (т.\,е.\ на интервале
$[t_0,t)$) в~классе допустимых оценок, определяемых формулой $\hat X_t \hm= A U_t$ 
и~стохастическим дифференциальным уравнением вида
  \begin{multline}
  dU_t ={}\\
  {}+\left[ \alpha_t\xi (Y_t,\hXt,t)+\gamma_t \right] dt +\beta_t\eta \left(Y_t,\hXt,t\right)
    dY_t\label{e4.1-s}
    \end{multline}
при заданных векторной и~матричной структурных функциях~$\xi$
и~$\eta$ и~при всех возможных функциях времени~$\alpha_t$, $\beta_t$
и~$\gamma_t$ ($\alpha_t$ и~$\beta_t$~--- матрицы; $\gamma_t$~---
вектор). В~качестве критерия
оптимальности примем критерий минимума с.к.\ ошибки
оценки~$\hXt$.
Как известно~\cite{6-s}, самым сложным вопросом в~практических
применениях теории услов\-но-оп\-ти\-маль\-ной фильт\-ра\-ции является вопрос
о~выборе класса до\-пус\-ти\-мых фильтров, опре\-де\-ля\-емо\-го заданием
структурных функций~$\xi$ и~$\eta$ в~уравнении~(\ref{e4.1-s}). 
В~принципе, их можно задать произвольно. При желании можно выбрать~$\xi$ 
и~$\eta$ так, чтобы класс допустимых фильтров содержал
произвольно заданный УОФ. В~этом случае
такой фильтр будет практически всегда точнее заданного
услов\-но-оп\-ти\-маль\-но\-го. В~то же время, выбрав в~качестве компонент
векторной функции~$\xi$ и~элементов матричной функции~$\eta$
конечный отрезок некоторого базиса в~соответствующем гильбертовом
пространстве~$L_2$, можно получить приближение с~любой степенью
точ\-ности к~неизвестным оптимальным функциям~$\xi$ и~$\eta$.
По-ви\-ди\-мо\-му, это наиболее рациональный способ выбора функций~$\xi$, $\eta$, 
основанный на уравнениях теории услов\-но-оп\-ти\-маль\-ной
фильтрации (см.\ разд.~3). При этом новые возможности
открывают уравнения УОФ, полученные из
уравнения для ненормированной апостериорной характеристической функции.

\textbf{4.2.} Для применения уравнений МЭСОФ, полученных из ненормированных
уравнений для апостериорного распределения, необходимо изменить
постановку задач услов\-но-оп\-ти\-маль\-ной фильтрации~\cite{6-s} так,
чтобы использовать уравнение для множителя~$\mu_t $.

С~этой целью воспользуемся для определения класса допустимых
МЭУОФ~(\ref{e4.1-s}) уравнениями:

\noindent
\begin{align}
d\mu_t &=\rho_t \chi \left(Y_t,\hXt,t\right) d Y_t\,;\label{e4.2-s}\\
    \hat X_t &= \fr{A \hXt }{\mu_t}\,,\notag %\label{e4.3-s}
    \end{align}
где $\chi (Y_t,\hXt,t)$~--- некоторая заданная матричная функция;
$\rho_t$~--- мат\-ри\-ца-стро\-ка коэффициентов, зависящих  
от~$t$ и~подлежащих оптимизации наряду с~коэффициентами~$\alpha_t$,
$\beta_t$ и~$\gamma_t$ в~уравнении фильт\-ра~(\ref{e4.1-s}).

Основываясь на результатах разд.~3, можно выделить два класса
допустимых МЭУОФ.

\textbf{4.3.\ Первый класс}. Этот важный класс допустимых МЭУОФ можно
получить, положив $U_t \hm=m_t$, $ A\hm=I_n$ и~определив функции~$\xi$,
$\eta$ и~$\chi$ в~(\ref{e4.1-s}) и~(\ref{e4.2-s}), руководствуясь уравнениями~(\ref{e3.6-s}) 
и~(\ref{e3.7-s}). Это дает следующие выражения для структурных
функций:


\noindent
 \begin{multline*}
 \xi=\xi \left(Y_t,\hXt,t\right) ={}\\[-2pt]
 {}=
  \left[ 
 \mu_t f_0 \left(Y_t,\fr{\hXt}{\mu_t},t\right)^{\mathrm{T}} f_1 
 \left(Y_t,\fr{\hXt}{\mu_t},t\right)^{\mathrm{T}}\cdots\right.\\[-6pt]
\left. \cdots   f_N \left(Y_t,\fr{\hXt}{\mu_t},t\right)^{\mathrm{T}}
\right]^{\mathrm{T}}\,; %\label{e4.4-s}
\end{multline*}


\noindent
\begin{multline*}
    \eta=\eta \left(Y_t,\hXt,t\right) ={}\\
    {}= \left[ 
    \mu_t h_0 \left(Y_t,\fr{\hXt}{\mu_t},t\right)^{\mathrm{T}} h_1 
    \left(Y_t,\fr{\hXt}{\mu_t},t\right)^{\mathrm{T}}\cdots\right.\\
\left.\cdots    h_N\left(Y_t,\fr{\hXt}{\mu_t},t\right)^{\mathrm{T}}\right]^{\mathrm{T}}
\,;
%\label{e4.5-s}
\end{multline*}

\vspace*{-12pt}

\noindent
\begin{multline*}
\chi=\chi \left(Y_t,\hXt,t\right) = {}\\
{}=\left[ \mu_t b_0 
\left(Y_t,\fr{\hXt}{\mu_t},t\right)^{\mathrm{T}} b_1 
\left(Y_t,\fr{\hXt}{\mu_t},t\right)^{\mathrm{T}}\cdots\right.
\\
\left.\cdots    b_N \left(Y_t,\fr{\hXt}{\mu_t},t\right)^{\mathrm{T}}
\right]^{\mathrm{T}}\,,
%\label{e4.6)-s}
\end{multline*}
при этом порядок МЭУОФ, определяемого уравнениями~(\ref{e4.1-s}) и~(\ref{e4.2-s}),
 будет равен~$n\hm+1$.

\textbf{4.4.\ Второй класс}. Для получения более широкого класса
допустимых МЭУОФ перепишем уравнение~(\ref{e3.18-s}) в~виде:
    \begin{multline}
    dc_\chi = \left\{ F_{\chi 0}\left(Y_t,\hat X_t,t\right) + \sss_{\nu=1}^N
    c_\nu F_{\chi\nu} \left(Y_t,\hat X_t,t\right)+{}\right.\\
{}+\sss_{\lambda,\nu=1}^N c_\lambda  c_\nu F_{\chi\lambda\nu} 
\left(Y_t,\hat X_t,t\right) +{}\\
\left.{}+ c_{\chi-1} \sss_{\lambda,\nu=1}^N
    c_\lambda c_\nu F_{\chi\lambda\nu}' \left(Y_t,\hat X_t,t\right)\right\} dt+{}\\
{}+\left\{ \vphantom{\sss_{\nu=1}^N}
\mu_t
    \eta_{\chi 0} \left(Y_t,\hat X_t,t\right) +{}\right.\\
\left.    {}+ \sss_{\nu=1}^N c_\nu \eta_{\chi \nu}
    \left(Y_t,\hat X_t,t\right)\right\} dY_t\,.\label{e4.7-s}
    \end{multline}
Здесь введены следующие обозначения:
\begin{multline*}
    F_{\chi 0} \left(Y_t,\hat X_t,t\right) = 
    \mu_t\gamma_{\chi 0} \left(Y_t,\hat X_t,t\right)+{}\\
    {}+
    \mu_t \mathrm{tr} \left[ \left(h_0 \left(Y_t,\hat X_t,t\right)+{}\right.\right.\\
\left.\left.{}+ \hat X_t b_0 \left(Y_t,\hat X_t,t\right)\right) \varepsilon_{\chi 0} 
\left(Y_t,\hat X_t,t\right)\right]\,;
%\label{e4.8-s}
\end{multline*}

\vspace*{-12pt}

\noindent
\begin{multline*}
F_{\chi \nu} \left(Y_t,\hat X_t,t\right) =
    \gamma_{\chi 0} \left(Y_t,\hat X_t,t\right)+ {}\\
    {}+\mathrm{tr} 
    \left[ \left(h_\nu \left(Y_t,\hat X_t,t\right)+{}\right.\right.\\
    \left.    {}+
    \hat X_t b_\nu \left(Y_t,\hat X_t,t\right)\right) 
\varepsilon_{\chi 0} \left(Y_t,\hat X_t,t\right)
+{}\\
\left.{}+ \left(h_0 \!\left(\!Y_t,\hat X_t,t\right)+ 
\hat X_t b_0 \!\left(\!Y_t,\hat X_t,t\right)\!\right)
    \varepsilon_{\chi \nu}\! \left(\!Y_t,\hat X_t,t\right)\!\right]+{}\hspace*{-6.10587pt}
    \end{multline*}
    
    \noindent
    \begin{multline*}
{}+\fr{1}{2n}\,
    \delta_{\chi-1,\nu} \left\{
    \mathrm{tr} \left[ 
        \vphantom{b_0 \left(Y_t,\hat X_t,t\right)^{\!\mathrm{T}}}
        b_\nu K_t+{}\right.\right.\\
\left.  {}+ C_th_0 \left(Y_t,\hat
    X_t,t\right)\sigma_2 \left(Y_t,t\right) h_0
    \left(Y_t,\hat X_t,t\right)^{\!\mathrm{T}}\right]-{}\\
{}-2\hat X_t^{\mathrm{T}} C_t h_0 \left(Y_t,\hat X_t,t\right)\sigma_2 
\left(Y_t,t\right) 
b_0\left(Y_t,\hat X_t,t\right)^{\!\mathrm{T}}+{}\\
\left.{}+\hat X_t^{\mathrm{T}} C_t\hat X_t b_0\left(Y_t,\hat X_t,t\right)
\sigma_2\left(Y_t,t\right) b_0 \left(Y_t,\hat X_t,t\right)^{\!\mathrm{T}}\right\}+{}\\
{}+
\fr{1}{n} \, \chi\delta_{\chi\nu} \mathrm{tr} \left[ \dot C_t K_t\right]\,;
%\label{e4.9-s}
\end{multline*}

\vspace*{-14pt}

\noindent
\begin{multline*}
F_{\chi \lambda \nu} = \fr{1}{\mu_t}\,\mathrm{tr} \left[ 
\vphantom{\hat X_t  b_\lambda \!\left(\!Y_t,\hat X_t,t\right)}
C_t\left(
h_\lambda \left(Y_t,\hat X_t,t\right) +{}\right.\right.\\
\left.\left.\!{}+
\hat X_t  b_\lambda \!\left(\!Y_t,\hat X_t,t\right)\!\right)\! \varepsilon_{\chi \nu}
    \left(Y_t,\hat X_t,t\right) \right]+\fr{1}{n}\,\delta_{\chi-1,\lambda} \fr{1}{\mu_t}\times{}\\
{}\times\left\{
    \mathrm{tr}\left[ C_t h_0 \left(Y_t,\hat X_t,t\right) 
    \sigma_2 \left(Y_t,t\right) h_\nu \left(Y_t,\hat X_t,t\right)^{\mathrm{T}}\right] -\right.{}\\
{}-\hat X_t^{\mathrm{T}} C_t
h_0 \left(Y_t,\hat X_t,t\right)\sigma_2 
\left(Y_t,t\right) b_\nu  \left(Y_t,\hat X_t,t\right)^{\mathrm{T}}+{}\\
    {}+ h_\nu \left(Y_t,\hat X_t,t\right)\sigma_2 \left(Y_t,t\right) b_0 \left(Y_t,\hat
    X_t,t\right)^{\mathrm{T}} +{}\\
\left.{}+\hat X_t^{\mathrm{T}} C_t\hat X_t b_0 \left(Y_t,\hat X_t,t\right) 
\sigma_2 \left(Y_t,t\right) 
b_\nu \left(Y_t,\hat X_t,t\right)^{\mathrm{T}}\right\};\hspace*{-2.70535pt}
%\label{e4.10-s}
\end{multline*}

\vspace*{-14pt}

\noindent
\begin{multline*}
 F_{\chi \la \nu}' =\fr{1}{2n}\,\fr{1}{\mu_t^2}\left\{ \mathrm{tr} 
 \left[ 
 \vphantom{\left(Y_t,\hat  X_t,t\right)^{\mathrm{T}}}
 C_t  h_\lambda \left(Y_t,\hat X_t,t\right) +{}\right.\right.\\
\left. {}+\sigma_2 \left(Y_t,t\right)
    h_\nu \left(Y_t,\hat X_t,t\right)^{\mathrm{T}}\right] -{}\\
{}-2\hat X_t^{\mathrm{T}} C_th_\lambda \left(Y_t,\hat X_t,t\right)\sigma_2
    \left(Y_t,t\right) b_\nu \left(Y_t,\hat X_t,t\right)^{\mathrm{T}}+{}\\
\left.{}+\hat X_t^{\mathrm{T}} C_t\hat X_t b_\lambda \left( Y_t,\hat X_t,t\right) 
\sigma_2 \left(Y_t,t\right) b_\nu \left(Y_t,\hat  X_t,t\right)^{\!\mathrm{T}}\right\}.\hspace*{-2.09656pt}
%\label{e4.11-s}
\end{multline*}

Взяв за основу для построения класса допустимых МЭУОФ уравнения~(\ref{e3.6-s}), 
(\ref{e3.7-s}) и~(\ref{e4.7-s}), следует принять за
компоненты вектора~$\hXt$ все компоненты вектора~$m_t$ и~коэффициенты $c_1\tr c_N$, 
так что $ \hXt\hm=\lk m_t^{\mathrm{T}} c_1\cdots
c_N\rk^{\mathrm{T}}$. Порядок всех допустимых фильтров при этом равен
$n\hm+N\hm+1$.

\textbf{4.5.}\ Для нахождения коэффициентов~$\alpha_t$, $\beta_t$ и~$\gamma_t$
уравнения МЭУОФ~(\ref{e4.1-s}) необходимо знать совместное одномерное
распределение случайных процессов~$X_t$ и~$\hat X_t$. Это
распределение находится  путем решения задачи анализа системы,
описываемой стохастическими дифференциальными уравнениями~(\ref{e4.1-s})
и~(\ref{e4.2-s}). Как всегда в~тео\-рии услов\-но-оп\-ти\-маль\-ной фильтрации,
все сложные вычисления, необходимые для нахождения оптимальных
коэффициентов уравнения МЭУОФ~(\ref{e4.1-s}) или~(\ref{e4.2-s}), основаны
только на априорных данных и~поэтому могут быть выполнены заранее
в~процессе проектирования МЭУОФ. При этом может быть определена
 и~точ\-ность фильтрации для каждого допустимо-\linebreak\vspace*{-12pt}

\pagebreak

\noindent
го МЭУОФ. Сам же процесс
фильтрации сводится к~решению дифференциального уравнения, что
дает возможность производить фильтрацию в~реальном времени.

Таким образом, приходим к~следующему результату.

\smallskip

\noindent
\textbf{Теорема~6.}\ \textit{В~условиях теоремы~$1$ 
уравнения \mbox{МЭУОФ} вида}~(\ref{e4.1-s}), (\ref{e4.2-s}) 
\textit{будут совпадать с~уравнениями \mbox{МЭСОФ} $($см.\ разд.~$3)$, если структурные функции УОФ 
выбрать из описанных выше классов. При этом качество МЭУОФ оценивается согласно 
тео\-ре\-ме~$5$}.

\section{Квазилинейные модифицированные эллипсоидальные условно-оптимальные фильтры}

Построим квазилинейный МЭУОФ на основе МЭЛ для МСтС~(\ref{e2.1-s}), 
(\ref{e2.2-s}) при $\psi'\hm=\psi'(\Theta,t)$, $\psi''\hm=\psi''(\Theta,t,v)$, 
$\psi_1'\hm=\psi_1'(\Theta,t)$ и~$\psi_1''\hm=\psi_1''(\Theta,t,v)$ (т.\,е.\
 с~аддитивными винеровскими и~пуассоновскими шумами). Уравнения ЭСОФ 
 проще получаются, если нелинейные функции~$\varphi$ и~$\varphi_1$ 
 на основе эл\-лип\-со\-и\-даль\-но\-го распределения с~известным~$c_\nu$ заменить 
 на статистически линеаризованные:
\begin{multline}
\varphi =\varphi \left( X_t, Y_t, \Theta, t\right) \approx {}\\
{}\approx
\varphi_0^{\mathrm{э}} + k_x^{\mathrm{э}\varphi} \left(X_t - m_t^x\right) +
k_y^{\mathrm{э}\varphi} \left(Y_t - m_t^y\right)\,;\label{e5.1-s}
\end{multline}

\vspace*{-12pt}

\noindent
\begin{multline}
\varphi_1 = \varphi_1\left( X_t, Y_t, \Theta, t\right) \approx{}\\
{}\approx 
\varphi_{10}^{\mathrm{э}}  + k_x^{\mathrm{э}\varphi_1}  \left(X_t - m_t^x\right) + 
k_y^{\mathrm{э}\varphi_1} \left(Y_t - m_t^y\right)\,,\label{e5.2-s}
\end{multline}
а затем использовать уравнения линейной фильт\-ра\-ции~\cite{6-s}. 
Входящие в~(\ref{e5.1-s}), (\ref{e5.2-s}) 
коэффициенты с~МЭЛ зависят от математических ожиданий, дисперсий и~ковариаций:
    $$
    Z_t =\begin{bmatrix}
     X_t\\ Y_t
     \end{bmatrix}\,; \enskip 
     m_t^z =\begin{bmatrix}
     m_t^x\\ m_t^y\end{bmatrix}\,;\enskip 
     K_t^z=\begin{bmatrix}
      K_t^x&K_t^{xy}\\
      K_t^{xy}&K_t^y\end{bmatrix}\,.
      $$
Они определяются из уравнений:
    \begin{align*}
    \dot Z_t &= A^z Z_t + A_0^z + B_0^z V \,,\enskip V= \dot W\,; %\label{e5.3-s}
    \\
    \dot m_t^z &= A^z m_t^z + A_0^z \,,\enskip m_{t_0}^Z = m_0^z\,;\\ %\label{e5.4-s}\\
    \dot K_t^z &= B^z K_t^z + K_t^z (B^z)^{\mathrm{T}} + B_0^z \nu^m 
    \left(B_0^z\right)^{\!\mathrm{T}}\!,\ K_{t_0}^z &= K_0^z.\hspace*{-0.111pt} %\label{e5.5-s}
    \end{align*}
Здесь введены следующие обозначения:
    $$
    A_0^z = \begin{bmatrix} 
    a_0\\ b_0\end{bmatrix}\,;\enskip 
    A^z =\begin{bmatrix} 
    a_1&a\\ 
    b_1&b\end{bmatrix}\,;\enskip 
    B_0^z =\begin{bmatrix} 
    \bar \psi\\ \bar\psi_1\end{bmatrix}\,;
    $$
   \begin{alignat*}{3}
    a&= k_y^{\mathrm{э}\varphi} \,;&\enskip 
    a_1 &= k_x^{\mathrm{э}\varphi} \,;&\enskip 
    a_0 &=\varphi_0^{\mathrm{э}}  - k_x^{\mathrm{э}\varphi}  m_t^x - 
    k_y^{\mathrm{э}\varphi}  m_t^y\,;
    \\
    b&= k_y^{\mathrm{э}\varphi_1} \,; &\enskip 
    b_1&=k_x^{\mathrm{э}\varphi_1}\,;&\enskip 
    b_0&=\varphi_0^{\mathrm{э}}  -k_x^{\mathrm{э}\varphi_1}  m_t^x -
    k_y^{\mathrm{э}\varphi_1} m_t^y\,; %\label{e5.6-s}
    \end{alignat*}
\begin{equation}
\left.
\begin{array}{rl}
\psi \,dW_0 + \iii_{R_0^q} \psi'' P^0 (dt, dv) &=\bar \psi \, dW\,;\\[6pt] 
\psi_1' \,dW_0 + \iii_{R_0^q} \psi_1'' P^0 (dt, dv)&= \bar \psi_1\, dW\,,
\end{array}
\right\}
\label{e5.7-s}
\end{equation}
где $\nu^W$~--- интенсивность СтП с~независимыми приращениями, состоящего
 из винеровской и~пуассоновской частей~(\ref{e5.7-s}). 
 Тогда уравнения квазилинейного ЭУОФ будут иметь вид:
   \begin{multline}
   \hat X_t = a Y_t + a_1\hat X_t + a_0 + {}\\
   {}+\beta_t \left[ 
    Z_t - \left(bY_t + b_1 \hat X_t + b_0\right)\right]\,.
    \label{e5.8-s}
 \end{multline}
 Здесь
 
 \noindent
\begin{equation}
    \beta_t = \left(R_t b_1^{\mathrm{T}} + \bar\psi \nu^W \bar\psi_1^{\mathrm{T}}\right) 
    \left(\bar\psi_1\nu^W\bar\psi_1^{\mathrm{T}}\right)^{-1}\,,
\label{e5.9-s}
 \end{equation}
 где
 
 \noindent
\begin{multline}
    \dot R_t = a_1 R_t + R_t a_1^{\mathrm{T}} + \bar\psi \nu^W \bar\psi^{\mathrm{T}} - {}\\
    {}-
    \left(R_t b_1^{\mathrm{T}} +\bar\psi \nu^W\bar\psi_1^{\mathrm{T}}\right)
    \left(\bar\psi_1 \nu^W\bar\psi_1^{\mathrm{T}}\right)^{-1} \times{}\\
    {}\times
    \left(b_1 R_t + \bar\psi_1 \nu^W\bar\psi^{\mathrm{T}}\right)\,.
\label{e5.10-s}
 \end{multline}
 
 \vspace*{-3pt}


\noindent
\textbf{Теорема~7.}\ \textit{Пусть МСтС}~(\ref{e2.1-s}), (\ref{e2.2-s}) 
\textit{содержит только аддитивные винеровские и~пуассоновские шумы и~допускает 
замену статистически линеаризованной, а~мат\-ри\-ца  $\sigma_1 \hm=
\bar\psi_1 \nu^W \bar\psi_1^{\mathrm{T}}$ не вырождена. Тогда 
в~основе алгоритма квазилинейного МЭУОФ лежат уравнения}~(\ref{e5.8-s})--(\ref{e5.10-s}) 
\textit{при соответствующих начальных условиях}.

\vspace*{-12pt}

\section{Заключение}

\vspace*{-2pt}

Разработана теория аналитического синтеза 
МЭУОФ для нелинейных дифференциальных гауссовских и~негауссовских МСтС~(\ref{e2.3-s}), 
(\ref{e2.4-s}) и~(\ref{e2.8-s}), (\ref{e2.9-s}) 
на основе МЭА и~МЭЛ.  Алгоритмы ЭУОФ положены в~основу разрабатываемого модуля 
экспериментального программного обеспечения  StS-Filter (version 2017).

Результаты допускают обобщение на случай дискретных  и~не\-пре\-рыв\-но-дис\-крет\-ных МСтС, 
а~также автокоррелированных МСтС.

Теоретический и~практический интерес представляет теория  МЭСОФ и~МЭУОФ для МСтС~(\ref{e2.1-s}) 
и~(\ref{e2.2-s}).

\vspace*{-12pt}

{\small\frenchspacing
 {%\baselineskip=10.8pt
 \addcontentsline{toc}{section}{References}
 \begin{thebibliography}{99}
 
 \vspace*{-2pt}
 
\bibitem{1-s}
\Au{Синицын И.\,Н.}
Ортогональные субоптимальные фильтры для нелинейных стохастических систем 
на многообразиях~// Информатика и~её применения, 2016. Т.~10. Вып.~1. С.~34--44.

\bibitem{2-s}
\Au{Синицын И.\,Н.}
Нормальные и~ортогональные субоптимальные фильтры для нелинейных стохастических 
сис\-тем на многообразиях~// Системы и~средства информатики, 2016. Т.~26. №\,1. С.~199--226.

\bibitem{3-s}
\Au{Синицын И.\,Н., Синицын~В.\,И., Корепанов~Э.\,Р.}
Эллипсоидальные субоптимальные фильтры для нелинейных 
стохастических систем на многообразиях~// Информатика и~её применения, 2016. Т.~10. Вып.~2. С.~24--35.

\bibitem{4-s}
\Au{Синицын И.\,Н., Синицын~В.\,И.}
Лекции по теории нормальной и~эллипсоидальной аппроксимации распределений 
в~стохастических системах.~--- М.: ТОРУС ПРЕСС, 2013. 488~с.



\bibitem{6-s}
\Au{Синицын И.\,Н.}
Фильтры Калмана и~Пугачева.~--- 2-е изд.~--- М.: Логос, 2007. 776~с.

\bibitem{5-s}
\Au{Синицын И.\,Н., Синицын~В.\,И., Корепанов~Э.\,Р.}
Модифицированные эллипсоидальные субоптимальные фильтры 
для нелинейных стохастических систем на многообразиях~// 
Системы и~средства информатики, 2016. Т.~26. №\,2. С.~79--97.

\bibitem{7-s}
\Au{Пугачёв В.\,С., Синицын И.\,Н.}
Теория стохастических систем.~--- М.: Логос, 2000; 2004. 1000~с.
%[Англ. пер. Stochastic Systems. Theory and  Applications. --
%Singapore: World Scientific, 2001. 908~p.].

\bibitem{8-s}
\Au{Wonham W.\,M.}
Some applications of stochastic differential equations to optimal
nonlinear filtering~// J.~Soc. Ind. Appl. Math. A, 1964.
Vol.~2. No.\,3. P.~347--369.
 
\bibitem{9-s}
Справочник по теории вероятностей и~математической статистике~/ Под
ред.\ В.\,С.~Королюка, Н.\,И.~Портенко, А.\,В.~Скорохода, А.\,Ф.~Турбина.~--- 
М.: Наука, 1985. 640~с.

\bibitem{10-s} 
\Au{Zakai M.}
On the optimal filtering of diffusion processes~// 
Ztschr. Wahrschein lichkeitstheor. Verm. Geb., 1969. Bd.~11. S.~230--243.
 \end{thebibliography}

 }
 }

\end{multicols}

\vspace*{-3pt}

\hfill{\small\textit{Поступила в~редакцию 08.02.17}}

\vspace*{12pt}

%\newpage

%\vspace*{-24pt}

\hrule

\vspace*{2pt}

\hrule

%\vspace*{8pt}


\def\tit{MODIFICATED ELLIPSOIDAL CONDITIONALLY OPTIMAL
FILTERS FOR NONLINEAR STOCHASTIC SYSTEMS ON~MANIFOLDS}

\def\titkol{Modificated ellipsoidal conditionally optimal
filters for nonlinear stochastic systems on~manifolds}

\def\aut{I.\,N.~Sinitsyn, V.\,I.~Sinitsyn, and~E.\,R.~Korepanov}

\def\autkol{I.\,N.~Sinitsyn, V.\,I.~Sinitsyn, and~E.\,R.~Korepanov}

\titel{\tit}{\aut}{\autkol}{\titkol}

\vspace*{-9pt}


\noindent
Institute of Informatics Problems, Federal Research Center 
``Computer Science and Control'' of the Russian
Academy of Sciences,  44-2~Vavilov Str., Moscow 119333, Russian Federation



\def\leftfootline{\small{\textbf{\thepage}
\hfill INFORMATIKA I EE PRIMENENIYA~--- INFORMATICS AND
APPLICATIONS\ \ \ 2017\ \ \ volume~11\ \ \ issue\ 2}
}%
 \def\rightfootline{\small{INFORMATIKA I EE PRIMENENIYA~---
INFORMATICS AND APPLICATIONS\ \ \ 2017\ \ \ volume~11\ \ \ issue\ 2
\hfill \textbf{\thepage}}}

\vspace*{3pt}


\Abste{The analytical synthesis theory for modificated ellipsoidal conditionally 
optimal filters (MECOF) for nonlinear stochastic systems on manifolds (MStS) 
based on the nonnormed a posteriori characteristic function is developed. Gaussian 
and non-Gaussian MStS are considered.  The MECOF algorithms are more simple than the ECOF 
algorithms. The MECOF algorithms are the basis of the software tool 
``StS-Filter''  (version 2017).}

\KWE{accuracy and sensitivity equations;
ellipsoidal approximation  and linearization methods (EAM \& ELM);
ellipsoidal conditionally optimal filter (ECOF);
modificated ellipsoidal conditionally optimal filter (MECOF);
nonnormed characteristic function;
Poisson noise;
conditionally optimal filter (COF);
Wiener noise}


\DOI{10.14357/19922264170211} 

\vspace*{-18pt}

\Ack
\noindent
The research was supported by the Russian Foundation for 
Basic Research (project 15-07-002244).



\vspace*{9pt}

  \begin{multicols}{2}

\renewcommand{\bibname}{\protect\rmfamily References}
%\renewcommand{\bibname}{\large\protect\rm References}

{\small\frenchspacing
 {%\baselineskip=10.8pt
 \addcontentsline{toc}{section}{References}
 \begin{thebibliography}{99}
\bibitem{1-s-1}
\Aue{Sinitsyn, I.\,N.} 2016.
Ortogonal'nye suboptimal'nye fil'try dlya nelineynykh stokhasticheskikh sistem 
na mno\-go\-ob\-ra\-zi\-yakh [Orthogonal suboptimal filters for nonlinear stochastic 
systems on manifolds]. \textit{Informatika i~ee Primeneniya~---
Inform. Appl.} 10(1):34--44.

\bibitem{2-s-1} 
\Aue{Sinitsyn, I.\,N.} 2016.
Normal'nye i~ortogonal'nye suboptimal'nye fil'try dlya nelineynykh stokhasticheskikh 
sistem na mnogoobraziyakh  [Normal and orthogonal conditionally optimal filters 
for nonlinear stochastic systems on manifolds].
\textit{Informatika i~ee Primeneniya~---
Inform. Appl.} 10(1):199--226.

\bibitem{3-s-1}
\Aue{Sinitsyn, I.\,N., V.\,I.~Sinitsyn, and E.\,R.~Korepanov.} 2016.
Ellipsoidal'nye suboptimal'nye fil'try dlya nelineynykh stokhasticheskikh 
sistem na mnogoobraziyakh [Ellipsoidal conditionally optimal filters for nonlinear 
stochastic systems on manifolds].
\textit{Informatika i~ee Primeneniya~---
Inform. Appl.} 10(1):24--35.


\bibitem{4-s-1}
\Aue{Sinitsyn, I.\,N., and V.\,I.~Sinitsyn.} 2013. 
Lektsii po teorii normal'noy i ellipsoidal'noy 
approkskimatsii raspredeleniy v~stokhasticheskikh sistemakh [Lectures on normal 
and ellipsoidal approximation of distributions in stochastic
systems].  Moscow: TORUS PRESS. 
488~p.

\bibitem{6-s-1}
\Aue{Sinitsyn, I.\,N.} 2007.
\textit{Fil'try Kalmana i~Pugacheva} [Kalman and Pugachev filters]. 
2nd ed. Moscow: Logos. 776~p.

\bibitem{5-s-1}
\Aue{Sinitsyn, I.\,N., V.\,I.~Sinitsyn, and E.\,R.~Korepanov}. 2016.
Modifitsirovannye ellipsoidal'nye suboptimal'nye fil'try dlya 
nelineynykh stokhasticheskikh sistem na mno\-go\-ob\-ra\-zi\-yakh [Modificated ellipsoidal 
conditionally optimal filters for nonlinear stochastic systems on manifolds].
\textit{Sistemy i~Sredstva Informatiki~--- Systems and Means of Informatics}
26(2):79--97.

%\columnbreak

\bibitem{7-s-1}
\Aue{Pugachev, V.\,S., and I.\,N.~Sinitsyn.} 2000, 2004.
Teoriya stokhasticheskikh sistem [Stochastic systems. Theory and  applications]. 
Moscow: Logos. 1000~p.  %[Angl. per. . -- Singapore: World Scientific, 2001].

\bibitem{8-s-1}
\Aue{Wonham, M.} 1965.
Some applications of stochastic differential equations to optimal nonlinear filtering.
\textit{J.~Soc. Ind. Appl. Math.~A} 2(3):347--369.

\bibitem{9-s-1}
Korolyuk, V.\,S., N.\,I.~Portenko, A.\,V.~Skorokhod, and A.\,F.~Turbin, eds. 1985.
\textit{Spravochnik po teorii veroyatnosti i~matematicheskoy statistike}
[Handbook: Probability theory and mathematical statistics]. Moscow: Nauka.  640~p.

\bibitem{10-s-1}
\Aue{Zakai, M.} 1969.
On the optimal filtering of diffusion processes. \textit{Ztschr. Wahrschein lichkeitstheor. 
Verm. Geb.} 11:230--243.

\end{thebibliography}

 }
 }

\end{multicols}

\vspace*{-3pt}

\hfill{\small\textit{Received February 8, 2017}}


\Contr

\noindent
\textbf{Sinitsyn Igor N.} (b.\ 1940)~--- 
Doctor of Science in technology, professor, Honored scientist of RF, 
principal scientist, Institute of Informatics Problems, Federal Research Center 
``Computer Science and Control'' of the Russian Academy of Sciences, 
44-2~Vavilov Str., Moscow 119333, Russian Federation; \mbox{sinitsin@dol.ru}

\vspace*{3pt}

\noindent
\textbf{Sinitsyn Vladimir I.} (b.\ 1968)~--- 
Doctor of Science in physics and mathematics, associate professor, 
Head of Department, Institute of Informatics Problems, Federal Research Center 
``Computer Science and Control'' of the Russian Academy of Sciences, 
44-2~Vavilov Str., Moscow 119333, Russian Federation; \mbox{VSinitsyn@ipiran.ru} 

\vspace*{3pt}

\noindent
\textbf{Korepanov Eduard R.} (b.\ 1966)~--- 
Candidate of Science (PhD) in technology, Head of Department, Institute of 
Informatics Problems, Federal Research Center ``Computer Science and Control'' 
of the Russian Academy of Sciences, 44-2~Vavilov Str., Moscow 119333, 
Russian Federation; \mbox{Ekorepanov@ipiran.ru}
\label{end\stat}


\renewcommand{\bibname}{\protect\rm Литература} 