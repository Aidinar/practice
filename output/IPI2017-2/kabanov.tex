\renewcommand{\figurename}{\protect\bf Figure}
\renewcommand{\tablename}{\protect\bf Table}

\def\stat{kabanov}


\def\tit{DYNAMIC MODELS OF SYSTEMIC RISK AND CONTAGION}

\def\titkol{Dynamic models of systemic risk and contagion}

\def\autkol{Kh.\ El Bitar,  Yu.~Kabanov, and~R.~Mokbel}

\def\aut{Kh.\ El Bitar$^1$,  Yu.~Kabanov$^2$, and~R.~Mokbel$^3$}

\titel{\tit}{\aut}{\autkol}{\titkol}

%{\renewcommand{\thefootnote}{\fnsymbol{footnote}}
%\footnotetext[1] {The 
%research of Yuri Kabanov was done under partial financial support   of the grant 
%of  RSF No.\,14-49-00079.}}

\renewcommand{\thefootnote}{\arabic{footnote}}
\footnotetext[1]{Laboratoire de Math$\acute{\mbox{e}}$matiques, Universit$\acute{\mbox{e}}$ de 
Franche-Comt$\acute{\mbox{e}}$, 16~Route de Gray, 25030 \mbox{Besan{\!\ptb{\c{c}}}on}, CEDEX, France, 
\mbox{khalilbitar\_aw@hotmail.com}}
\footnotetext[2]{Laboratoire de 
Math$\acute{\mbox{e}}$matiques, Universit$\acute{\mbox{e}}$ de
 Franche-Comt$\acute{\mbox{e}}$, 16~Route de Gray, 25030 
\mbox{Besan{\!\ptb{\c{c}}}on}, CEDEX, France; 
Institute of Informatics Problems, Federal Research 
Center ``Computer Science and Control'' of the Russian Academy of Sciences, 
44-2~Vavilov Str., Moscow 119333, Russian Federation; 
National Research University 
``MPEI,'' 14~Krasnokazarmennaya Str., Moscow 111250, Russian Federation, 
\mbox{Youri.Kabanov@univ-fcomte.fr}}
\footnotetext[3]{Laboratoire de 
Math$\acute{\mbox{e}}$matiques, Universit$\acute{\mbox{e}}$ de 
Franche-Comt$\acute{\mbox{e}}$, 
16~Route de Gray, 25030  \mbox{Besan{\!\ptb{\c{c}}}on}, CEDEX, France,
\mbox{ritamokbel@hotmail.com}}

\index{El Bitar Kh.}
\index{Kabanov Yu.}
\index{Mokbel R.}
\index{Эль Битар Х.}
\index{Кабанов Ю.}
\index{Мокбель Р.}


%\vspace*{-9pt}

\def\leftfootline{\small{\textbf{\thepage}
\hfill INFORMATIKA I EE PRIMENENIYA~--- INFORMATICS AND APPLICATIONS\ \ \ 2017\ \ \ volume~11\ \ \ issue\ 2}
}%
 \def\rightfootline{\small{INFORMATIKA I EE PRIMENENIYA~--- INFORMATICS AND APPLICATIONS\ \ \ 2017\ \ \ volume~11\ \ \ issue\ 2
\hfill \textbf{\thepage}}}




\Abste{Modern financial systems are complicated networks of interconnected 
financial institutions and default of any of them may have serious consequences 
for others. The recent crises have shown that complexity and interconnectedness 
are the major factors of systemic risk, which became the subject of intensive 
studies usually concentrated on static models. The authors develop 
a~dynamic model based on the so-called structural approach, where defaults are 
triggered by the exit of some stochastic process from a~domain. In the case
considered, this is 
a~process defined by the evolution of bank's portfolios values. At the exit time, 
a~bank defaults and a~cascade of defaults starts. The authors believe that the 
distribution of the exit time and the subsequent losses may serve as indicators 
allowing regulators to monitor the state of the system and take corrective actions 
in order to avoid contagion in a~financial system. The authors model the development 
of a~financial system as a~random graph using the preferable attachment algorithm and 
provide results of numerical experiments on simulated data.}


\KWE{systemic risk; contagion; scale free network; default}

\DOI{10.14357/19922264170201} 

%\vspace*{-12pt}


\vskip 10pt plus 9pt minus 6pt

      \thispagestyle{myheadings}

      \begin{multicols}{2}

                  \label{st\stat}




\section{Introduction} 

\noindent
In interbank market, systemic risk is a~risk arising from a~complexity of 
financial network and threatening the entire system by a~potential financial 
crisis,
resulting in high economical and social costs.  Controlling financial stability 
and assessing systemic  risk is a~major concern of central banks and financial 
regulators.
The rapid growth of financial innovation and integration as well as a~
complicated  network of claims and obligations  linking the balance sheets of 
banks raise the challenge of the systemic risk analysis.
This kind of risk is highly dynamic, slowly building up    during periods of 
stability and rapidly rising during crises and spreading through the network.  
On the other hand, the interconnections of banks have a~positive side since they   
enhance  liquidity and increase the risk sharing among the financial 
institutions. 

One of the aim of theoretical studies is to provide regulators comprehensive 
indicators allowing  to monitor  the risk of contagion, understood as  a~cascade 
of defaults that may lead to a~serious consequences and even to the collapse of 
the whole economy.  To the moment, there is a~substantial progress in 
understanding various phenomena  causing the contagion
 on the basis of modeling using random graphs.  Network models became the 
mainstream of current researches in the field (see the recent book by 
Hurd~\cite{Hurd}  and the references wherein).

Recent crisis revealed that the systemic risk might take various forms. One of 
them is an interbank contagion process when, due to the interconnectedness of 
banks through interbank loans, the default of one bank leads to losses and 
subsequent chain of  defaults of other banks. This kind of risk is usually 
combined with a~risk  related to a~correlation  between banks' portfolios which
consists in the phenomena that a~common shock, due to common asset holdings, 
affects many banks at once.

De Bandt {\it et al.}\ provided a~categorization of systemic 
risks, distinguishing between those understood in  a~broad and in a~narrow 
sense~\cite{Bandt}: contagion effects pose a~systemic risk in the narrow sense while in the 
broad sense, it is a~common shock that affects many nodes and once.
A~similar idea is developed  by Gai and Kapadia~\cite{GK},  who model 
two channels of contagion in financial system that can trigger further rounds of 
defaults: contagion due to the direct interbank claims and obligations and 
contagion due to common shocks on the asset side of the balance sheet, 
especially when the market for key financial system assets is illiquid.

Deposits also could affect the financial system stability:  a~large sudden 
withdrawal of funds  by depositors in panic  could lead to a~collapse of the 
system. However, in the present paper,  this is not considered as one of the 
major sources of system risk. in fact, its impact can be minimized and 
controlled by the central bank intervention imposing  an appropriate withdrawal 
limit.
{\looseness=1

}

A large part  of the literature has focused on the analysis of the contagion 
effect due to the interbank market while only few authors studied the impact 
of the correlated defaults which is of great importance, the magnitude 
of correlation between the banks balance sheets, the amount of external 
investments, and the appropriate assessment of the risk embedded in these 
external assets.
Acharya and Yorulmazer proved that banks are motivated to 
increase the correlation between their investments amplifying by such actions  
the risk of a~common shock~\cite{AY}. In their analysis, Elsinger {\it et al.}\ 
combine the two major sources  of systemic risk and find that the 
correlation in investments is far more important than financial linkages~[5, 6].

One can also consider a~subordinate source of risk due to the fire sale of 
external assets of defaulting banks which will lead to other banks default 
because of the price depreciation.  This is  why some banks have an interest to 
bailout other peers in order to minimize the default cost of the system and to 
prevent fire sale and the writing down of their own external assets.

While the interbank risk is concerned, Gai and Kapadia show  that the 
risk of systemic crises is reduced with increasing connectivity while
at the same time,
 the 
amplitude of the systemic crises is increasing~\cite{7a}. Higher 
connectivity simply creates more channels of contact through which default could 
spread, increasing the potential or probability for contagion. However, in the 
financial system setup, greater connectivity allows counterparties risk sharing 
as exposures are distributed over a~wider set of banks, especially in periods 
of stability. In times of crisis, however, the same interconnections can amplify 
shocks that spread through the system.

Allen and Gale demonstrated that the spread of contagion 
depends on the network structure of the financial system and  strongly 
interconnected banking systems are less affected by the systemic risk~\cite{Allen}. They also 
pointed out that the assumption that the agents have complete information on 
their environment is not realistic. Acharya and Bisin 
compared over-the-counter (OTC) and centralized clearing markets in a~general 
equilibrium model~\cite{AB}. They showed that the intransparency of OTC markets is ex-ante 
inefficient and will lead to underpricing of counterparty risk.

The counterparty risk makes it clear that the network structure of financial 
system plays an important role when assessing systemic risk.

Empirical analyses of the interbank network structure exist for a~number of 
countries. It shows that the interbank network has a~scale free topology. This 
means that there are a~few large banks with many interconnections and many small 
banks with  a~few connections. In contrast, other authors argue that the 
intransparency of real data makes the random network more valid to capture the 
hidden links. More formally, the terminology 
``scale free network'' means that at least, when the number of nodes increases to 
infinity, 
the number~$k$ of connections (``in'' or ``out'')  attributed to each node decays as 
$k^{-\gamma}$, $\gamma >1$.  

    

Georg proposed a~dynamic model of cascading banking defaults~\cite{Co}: 
at each stage of the cascade, each bank collects all his exposures, pays all his 
liabilities, adjust the price of its external assets, and, when remains solvent, 
it optimizes a~portfolio of risky and risk-free assets and initiates other 
interconnections within the banking system.

On the other hand, Gai and Kapadia highlighted that in normal times, 
developed country banks are robust and minor variations in their default 
probabilities do not affect lending decisions on the interbank market~\cite{7a}. But in 
crises, as illustrated by the sudden failures of Lehman Brothers, contagion may 
spread rapidly with banks having little time to alter their behavior before they 
are affected. Thus, the almost static behavior of the system during crisis is 
best captured by the static model as also applied in the present paper.

It seems that the majority of existing literature  deals with ``homogeneous''
models, like Erd\"os--Renyi model where the network graph is generated by 
a~matrix whose nondiagonal entries are identically distributed independent  
Bernoulli random variables (see~\cite{GK}), or even models where 
all nodes have the same number of connections~\cite{MA}.   Though such models 
are  convenient 
for theoretical studies, they look to be too far from the reality and in the 
present paper,  the behavior of the systemic risk indicator is investigated using 
networks with a~structure obtained by a~preference attachment algorithm leading 
to a~scale free network.  

Under the Basel II accord, improving the quality of default models  is the key 
risk-management priority. Many researchers have studied the loss or impact of 
the systemic risk once a~crisis or shock is in place. However, there is a~need 
to predict and prevent the defaults of banks before it happens. To the date, the 
major part of research papers concentrates on studies of static or stationary 
models. In this note, the authors suggest an approach influenced by the structural model 
of defaultable securities (see~\cite{Biel-Rut}). Namely, it is supposed that the 
cascade of default is triggered in a~natural way when the value of a~portfolio 
process of some bank falls below  a~certain level. Financial market  reacts 
negatively to such an event. Prices of the external assets drop down and 
contagion propagates not only to interconnected banks but also via correlation. 
Assuming that the matrix of exposures as well as the vector of the investments 
into external assets is known, the regulators, having a~model for the dynamic of 
the ``reference portfolio,''  can compute, with moving time horizons, two 
``alert indicators:''  the probability that the default  happens during 
the planning period and the total losses incurred when the default happens. The 
total losses are the aggregation of the losses due to the external asset price 
depreciation (correlation)  and the losses due to the interbank linkages 
(contagion). To simplify the calculation, it is assumed that there is a~single 
external risky asset common to all banks in the system and the difference is 
only in the size of portfolios. A~model where each bank has its own portfolio 
structure  can be treated in a~similar way.
The present approach is rather flexible and can be combined with existing methods of 
reconstructing of the exposure matrix.



Thus, the main novelty of this approach, in contrast to the majority of existing  
studies  concentrated on  static or stationary models, is in developing  a~
dynamic model of financial system before the crisis in combination with a~static 
contagion model for the crisis.  
The model is described by a~graph whose nodes are banks (or other 
financial institutions). The directed graph structure  arises from the matrix of 
liabilities/exposures. Each bank is characterized by a~stylized balance sheet. 
On the asset side, 
there are exposures (due to the  interbank lending) and liquid assets, risky 
(stocks) and nonrisky (cash).   
 The liability side is composed by the received interbank loans and the net 
worth, the quantity, equating both sides of the balance sheet.  The dynamic is 
introduced via random fluctuations of the value of the risky asset. Decreasing 
of its price means that the net worth is decreasing. The authors suppose that the risky 
asset is unique for all banks. One may think of this asset as a~``benchmark (or 
reference) portfolio.''  Taking into account that banks try to mimic behavior of 
each other (``herding effect''), we believe that this assumption may suit to our 
highly stylized model but for  practical applications, it can be relaxed.  Of 
course,  there is a~need to introduce dynamics in other parts of the balance 
sheet  but we prefer avoid this in the paper. 
 

The paper contains some numerical experiments. Unfortunately,  the liability 
matrix of a~financial system is not publicly available (with rare exceptions). 
By this reason, the  applicability of the model was tested on simulated data. In 
numerical experiments,  a~construction of the scale-free network is used with the
help of 
a~preferential attachment algorithm (see~\cite{BarAlb}).   
The present authors populate the model by balance sheets and compute the alert indicators.   
The carried out experiments show that the alert indicators can be used as a~tool to support 
regulator's decision to increase the stability of the financial system by 
withdrawal of the license of the 
bank  having low re\-liability.  
{\looseness=1

}



The structure of the article is as follows. In Section~2, 
the general network approach to contagion is described. 
Section~3  gives the model description and the definition of the alert 
indicators. 
Section~4 is devoted to simulation results.  
 

\section{Network Approach} 
%\label{network}

\subsection{General principles}

\noindent
The basic ideas are very simple and can be described as follows. The set 
$G=\{1,\ldots ,N\}$ stands for the banking system involving  $N$ financial 
institutions  described by an $N\times N$ matrix $L=(L^{ij})$ with
nonnegative entries vanishing on the diagonal ($L^{ii}=0$) and a~vector $C\in 
{\bf R}^N$ with nonnegative components. 

The entry $L^{ij}$ represents the {\it  liability} of the $i$th bank to the 
$j$th bank, i.\,e.,  the debts  of~$i$ to~$j$ or, in other words, the total amount 
of credit provided by~$j$ to~$i$. By the reciprocity, for the $i$th bank, the 
value~$L^{ji}$ is its {\it  exposure} to the bank~$j$.  By this reason, in the 
literature, the model  is described quite often by the matrix of the liabilities 
$X=(X^{ij})$, $X=L'$ where $'$ is used to denote the transpose.   Let 
$B^{ij}=I_{\{L^{ij}>0\}}$. The matrix~$B$ (whose entries are zeros and units) 
defines the directed graph structure on the set of~$N$ points in a~usual way (as 
is done in the theory of Markov chains): there is a~flesh $i\to j$ if 
$B^{ij}>0$, showing that  the $i$th bank is indebted  to the $j$th bank 
(attention: in some papers, the direction of fleshes can be opposite). With this 
observation, one can use the standard terminology of the network theory and 
identify banks with the nodes of the (weighted) oriented graph. 

The component $C^i$ of the vector~$C$ represents the proper capital  reserve the 
$i$th bank; it is {\it  solvent}
if the net worth 
\begin{equation*}
\mathrm{NW}^i:=\sum\limits_{j\in G}L^{ji}-\sum\limits_{j\in G}L^{ij}+  C^i\ge 0\,.
\end{equation*}
If the above solvency condition does not hold, the bank {\it defaults}. 

It is important to note that the definitions ``exposure,'' ``liability,'' and
``default'' 
appeal to a~common sense rather having a~precise meaning. Their understanding 
varies from paper to paper. 
In practice, the balance sheet of a~bank has a~much more complicated structure. 
For example, the exposure may include
overnight credits as well as long term loans,  the debts are of different 
seniority, and so on. The ``standard'' highly stylized balance sheet,  i.\,e., 
the equality $\mathrm{Assets}=\mathrm{Liabilities}$ presented as a~table, containing 
the interbank exposures (loans) and external assets (that can be split in 
liquid and illiquid, risky and nonrisky) on the assets 
sides  while on the liability side, there are 
interbank borrowings, deposits, and, to equate the both side, the net worth 
(called also capital reserve, or equity)~---  in the case that the bank is 
solvent.  

\subsection{Defaults}

\noindent
In the literature, the typical descriptions of the contagion process and 
defaults ``in cascade" can be found (e.\,g., in  \cite{Hurd}). Here, they are 
presented in a~succinct  
way  as follows.  Let us denote by $I_{\mathrm{out}}(i)$ (respectively, by $I_{\mathrm{in}}(i)$) 
the set  of banks to which the bank $i$
has a~liability (respectively, an exposure). That is,  $I_{\mathrm{out}}(i)$ is the set 
of  nodes terminal for the fleshes (arcs) outgoing from the node $i$ while 
$I_{\mathrm{in}}(i)$ is the set of nodes with fleshes ending at this node. Let us 
denote by 
$n_{\mathrm{out}}(i)$ and $n_{\mathrm{in}}(i)$ 
the cardinality of the corresponding sets, i.\,e., the 
numbers  of outgoing  and ingoing fleshes.  Clearly,  $n_{\mathrm{out}}(i)=\sum\nolimits_jB^{ij}$ 
and $n_{\mathrm{in}}(i)=\sum\nolimits_jB^{ji}$. 

The default of the bank~$i$ triggers the following procedure. The bank is 
excluded from the network.  
Debts are collected from debtors at liquidation. Creditors loose a~fraction $1-
R$  
of their exposures to~$i$, where the parameter~$R$ is referred to as {\it 
recovery rate}. Formally, one can think that the matrix $L$ is replaced 
by the matrix~$\bar L$ obtained by replacing the elements of the $i$th row and 
$i$th column by zeros. The transformed vector of capital reserves~$\bar C$ has
the components   $\bar C^j=C^j+RL^{ij}-L^{ji}$, $j\neq i$, $\bar C^i=0$. Put 
$D_0(i):=\{i\}$ and skip further the argument~$i$ here and in further 
definitions (depending also on~$R$). For some~$j$ (different from~$i$),  the 
solvency condition 
\begin{equation*}
\sum\limits_{k\in G\setminus D_0}\bar L^{kj}-\sum\limits_{k\in G\setminus D_0}\bar L^{jk}+  
\bar C^j\ge 0\,,  
\end{equation*}
which can be written also as 
\begin{equation*}
%\label{first}
\sum\limits_{k\in G}L^{kj}-\sum\limits_{k\in G} L^{jk}+  C^j- (1- R)L^{ij}\ge 0\,,  
\end{equation*}
may fail. Let us denote by $D_1:=D_1(i)$ the set of such indices, corresponding  to 
the first-order defaults in the cascade of the defaults 
triggered by  the default of~$i$. In the same way,  the contagion is propagated 
further, to the set of banks $D_2=D_2(i)$ which is a~subset of 
indices~$j$ outside of the union~$D_0^1$ of~$D_0$ and~$D_1$  and  such that the 
solvency condition becomes negative:
$$
\sum\limits_{k\in G}L^{kj}-\sum\limits_{k\in G} L^{jk}+  C^j- (1-R)\sum\limits_{k\in D_0^1}L^{kj}< 0\,.   
$$
Continue in the same way, for the set~$D_0^n$,  put  $D_0^{n+1}:=D_0^n\cup 
D_{n+1}$ where~$D_{n+1}$ is the set of indices~$j$ in the complement of~$D_0^n$ 
such that
$$
\sum\limits_{k\in G}^NL^{kj}-\sum\limits_{j\in G} L^{jk}+  C^j- (1-R)\sum\limits_{k\in D_0^n}L^{jk}< 
0\,. 
$$
The process stops if $D_{n+1}=\emptyset$. One can consider the value 
$$
L(i):=(1-R)\sum\limits_{n=0}^N\sum\limits_{j\in D_{n+1}}\sum\limits_{k\in D_0^n}L^{jk} 
$$ 
as the total losses incurred by the cascade of defaults triggered by the default 
of the $i$th bank. 

It is not difficult to extend the above definitions to obtain expressions for 
losses triggered by  simultaneous defaults
of a~group of banks.  

\vspace*{-4pt}


\subsection{Practical aspects and difficulties}

\vspace*{-2pt}

\noindent
On the first sight, the above formulae are of great help for the researchers in 
financial systemic risk providing them~$N$~functions of the recovery rate which 
allows  to  classify banks according to their systemic importance. The 
described procedure  can be also used to find the most vulnerable banks, 
sensitive to defaults of others. However, the practical implementation is not so 
straightforward. 
Indeed, in the majority of cases, the exposure matrix~$X$ (having one million 
entries for a~system with $N=1000$) is not publicly available
though a~certain  subset of its entries may be known.   
Usually, only the  sums of  elements along  each line and each column can be 
extracted from the balance sheets.  
If only this information is available, one cannot recover  the matrix~$L$ in 
a~unique way: one needs to solve the system of~$2N$~equations

\noindent
\begin{equation}
\label{systeq}
\sum\limits_{j\in G} L^{ji}=a^i\,; \enskip 
\sum\limits_{j\in G} L^{ij}=b^i\,, \enskip 1\le i,j\le N\,, 
\end{equation}
with $N^2-N$ unknown $L^{ij}\ge 0$, $i\neq j$,  and all $L^{ii}=0$.   

Obviously, system~(\ref{systeq}) has the nonnegative solution 
$x^{ij}=a^jb^i/\sum\nolimits_ib^i$ (note that $\sum\nolimits_ib^i=\sum\nolimits_ja^j)$. 
But this is not the needed solution since not all $x^{ii}=0$. In the literature 
(see, for example,~\cite{Mistrulli}), it is recommended to take as the matrix~$X$ the solution 
of the entropy minimization problem: 

\noindent
$$
\sum\limits_{ij}\ln\fr {L^{ji}}{x^{ji}}\to \min\,,
$$  
under constraints~(\ref{systeq}),  $L^{ij}\ge 0$ and  $L^{ii}=0$ for all~$i,j$. 

This approach is  criticized since it leads to a~matrix generating a~complete 
graph and the overestimation of stability 
of financial system. On the other hand, in some cases,  a~part of the matrix~$L$ 
is known, e.\,g., the  absence of  connections 
between some nodes can be a~plausible hypothesis. The entropy minimization 
method can be easily adapted to such cases leading  to a~rather realistic 
recovery of the exposure matrix.   

\vspace*{-4pt}

\subsection{Probabilistic modeling}

\vspace*{-2pt}

\noindent
Due to the lack  of the information on the real structure of the financial 
system, there is an 
interest to generate\linebreak\vspace*{-12pt}

\pagebreak

\noindent
 numerically models which have, at least, basic features of 
such models. 
 
Apparently, the liability  matrix~$L$ and the reserve vector~$R$ are random and 
evolve as stochastic processes. Due to the high dimensionality of the problem, 
their modeling is extremely complicated and simplifying assumptions are 
unavoidable.  The majority of available studies consider static models or 
stationary models and start  modeling with the description of the incidence 
matrix~$B$. 

The simplest model is based on the hypothesis that the nondiagonal elements of 
the incidence matrix~$B$ are independent identically  distributed Bernoulli 
random variables (see, e.\,g.,~\cite{NierYYA} where low-parameter models are 
suggested to evaluate the impact of various factors on the financial stability).  
In addition to~$N$ and $p=P(B^{ij}=1)$, there are three more parameters: 
the total value of assets~$A$, the value of external assets~$C$, and the net 
worth as the percentage of the total value of assets~$\gamma$. 
These parameters are used to generate the balance sheets. In our notations, the 
interbank exposures and liabilities for the $i$th bank are defined as follows: 
$$
a^i = (A-C)\fr{n_{\mathrm{in}}(i)}{|B|}\,; 
\quad b^i = (A-C)\fr{n_{\mathrm{out}}(i)}{|B|}
$$
 where $|B|:= \sum\nolimits_{ij}B^{ij}$. The value of external assets of the bank are 
defined by the formula:
 \begin{multline*}
C^i= \left(b^i - a^i\right)I_{\{a^i < b^i\}}\\
{}+ \fr {1}{N}
 \left(C  - \sum\limits_j \left(b^j - a^j\right)I_{\{a^j 
< b^j\}}\right)\,.  
 \end{multline*}    
If the second term is positive, then all banks in the system are solvent. Since~$a^j$ 
and~$b^j$ are random,  one should have a~sufficiently high  ratio $C/A$ 
(in~\cite{NierYYA}, it was always taken greater than~0.3).  The  quantity 
$\gamma (a^ i-b^i +C^i)$ models the net worth while $(1-\gamma) (a^ i-b^i +C^i)$ 
stands for the consumer deposits. 

%\section{Simulations}
%\label{sim}
%\subsection {Model}

%In this section we give a~description of a~model with 3 groups of banks with 
%different %levels of interconnections  within each group. 
 

\section{Dynamic Models and~Alert~Indicators}
%\label{alert}

\subsection {Structural model}

\noindent
The aim of the model is to provide regulators two functions on the current state 
of the system 
which can be used to calculate the alert indicators. 
The first one is the probability that the system will suffer  a~cascade of 
defaults before a~specified 
time horizon. The second  indicator 
is the total losses  incurred by the cascade of defaults, if it happens. 
 
Suppose that at time zero, the regulators dispose the liability matrix~$L$ or 
its transpose the exposure matrix $X=L'$ (in reality, this  information is  
public only in rare countries, like Brasil, but can be available to central 
banks) and the vector of capital reserves~$C$ which 
is decomposed into nonrisky reserve~$c$ (say, Treasury bonds) and investments~$y$ 
into a~single risky asset which can be  interpreted as a~market index, or 
a~market portfolio.   In the present very stylized model,  all these values are fixed up to 
the time horizon~$T$.  
Of course, in reality, the banks trade and portfolios are composed in many 
assets.
Nevertheless,  quite often banks mimic the behavior of each other and one may 
guess  that a~ typical portfolio value has the same evolution as a~certain 
reference 
portfolio. Let us describe its dynamics by a~geometric Brownian motion:
\begin{equation}
\fr{dS_t}{S_t}=\mu\, dt + \sigma\, dW_t\,, 
\label{e3a-kab}
\end{equation}
that is,  
$$
S_t=S_0e^{\sigma W_t+(\mu -\sigma^2/2)t}\,. 
$$
At time zero, all banks are supposed to be  solvent.  

The default cascade will be triggered at the instant when one of the solvency 
conditions is violated. 

The solvency condition for the $i$th bank has the form: 
\begin{equation*}
V_t  +y^i_0S_0e^{\sigma W_t+(\mu -\sigma^2/2)t}\ge 0\,.
\end{equation*}
Here, 
$$
V_t{:=}b^i_t-a^i_t +  c^i_t
$$
where
$$ 
b^i_t{:=} \sum\limits_{j\in G}L^{ji}_t\,;  
\qquad a^i_t:=\sum\limits_{j\in G}L^{ij}_t\,.
$$

\noindent
\textbf{Hypothesis:}\ $V_t=V$ for all $t\in [0,T]$. 

\smallskip

The above assumption allows one to provide the regulators some easily calculated  
indicators of the system stability. Without any doubts, in the present 
oversimplified form, they can be criticized. For example,  assume that the 
interbank operations to a~large extend are balanced by liquid assets.  In favor 
of this are evidences that interbank lending is not the main activity of banks.  
Let us also assume a~rigidity of the investment portfolio. 
Again, econometric studies confirm that banks have a~tendency to follow similar 
behavior. 
The benchmark portfolio process may have various dynamics and various 
theoretical and 
statistical models can be used for its description.  

\begin{table*}[b]\small 
%\vspace*{-12pt}
    \begin{center}
      \Caption{Summary of the parameters considered in the network construction 
of the model}
\vspace*{2ex}


    \begin{tabular}{clcc}
        \hline
        Parameter & \multicolumn{1}{c}{Description} &  Value & Range of variation \\ 
        \hline
        $n$  & Number of banks in the initial random graph & 10 & fixed \\ 
       % \hline
        $N$  & Total number of banks in the network & 250\hphantom{9} & fixed \\
        %\hline
        $m$  & Maximum number of connections in the random network & \hphantom{9}5 & 
fixed \\ 
        %\hline
        $m'$  & Maximum number of connections in the scale free network & \hphantom{9}3 & 
fixed \\
        \hline
    \end{tabular}
  \end{center}
\end{table*}

\smallskip 

Put 
$$
\lambda ^i:=\fr {1}{\sigma}\ln \fr{V^i}{y^iS_0}
$$
with a~convention that $\lambda ^i:=-\infty$, if $V^i\le 0$. 
Let~$i_0$ be the index corresponding to the largest of values of~$\lambda^i$.  
One may assume, with very minor loss of generality, that all  finite values 
of~$\lambda^i$ are different (the coincidence is not expected in the present 
context) and that~$\lambda^{i_0}$ is finite (otherwise, there will be no defaults). 

Let us introduce the stopping time  
$$
\tau:=\inf \left\{t\ge 0:\ w_t+\beta t\le \lambda ^{i_0} \right\}
$$
where $\beta :=\mu/\sigma-\sigma /2$. 
If  $\tau\le T$, the system will have a~default and it happens with the 
node~$i_0$; the price of the market portfolio at this date will 
be $S_0e^{\sigma\lambda ^{i_0} }$. 
The distribution of~$\tau$ is the well-known inverse Gaussian distribution (see, 
e.\,g.,~\cite{Bor-Sal}) and one has that 
$$
P(\tau\le T)=\Phi(h_1(T))+e^{2\beta \lambda ^{i_0}}\Phi(h_2(T))
$$
where $\Phi$ is the standard Gaussian distribution function; 
$$
h_1(T):= \fr{\lambda ^{i_0}- \beta T}{\sqrt{T}}\,; \qquad h_2(T):= \fr{\lambda 
^{i_0}+\beta T}{\sqrt{T}}\,. 
$$

The default of the bank~$i_0$ generates a~cascade of the defaults. 
It seems reasonable to suppose that the market reacts to such an event and the 
risky asset may loss a~certain percentage of its value.
With this assumption the set $D_1=D_1(i_0)$ of the first-order defaults of the banks 
correspond to the indices~$j$
such that 
\begin{equation*}
%\label{first-1}
\sum\limits_{k\in G}L^{kj}-\sum\limits_{j\in G} L^{jk}+  c^j+\alpha y^jS_0e^{\lambda ^{i_0} } \\
{}- (1- R)L^{ij_0}<0\,,   
\end{equation*}  
$D_0^1(i_0):=D_0\cup D_1$, etc. The parameter $\alpha \in ]0,1]$ represents the 
default impact on the price of the reference portfolio. 

The second alert indicator is the amount of total losses: 

\vspace*{-2pt}

\noindent
$$
L(i_0):=(1-R)\sum\limits_{n=0}^N\sum\limits_{j\in D_{n+1}}\sum\limits_{k\in D_0^n}L^{jk}\,. 
$$ 
In the considered setting, it can be augmented, e.\,g., 
by the losses of nondefaulted banks due to a~depreciation of their portfolios: 
$$
\tilde L(i_0):=(1-\alpha)\sum\limits_{j\in G\setminus D_0^N}y^jS_0e^{\lambda ^{i_0} }\,.  
$$ 

\vspace*{-12pt}

\subsection {Discussion}

\vspace*{-1pt}

\noindent
The model introduced above has an advantage of its simplicity. It combines 
structural approach to defaultable 
securities with ideas of modern theory of financial networks.  The alert 
indicators have a~simple and comprehensive meaning. They can be easily computed 
at the monitoring dates $t_m$ (when the new  balance sheets are communicated)  
for the moving time horizons $t_m+T$. This allows regulator to see dangerous 
trends in the evolution of the system. It is worth noting that the 
model combines two channels of contagions: via the network as well as via the 
correlation due to common source of randomness.

Surely, the model is highly stylized. How serious are the weak points and how 
the model can be improved? 

 It is assumed that the investment in the single risky asset are static 
though in reality, there is an intensive trading. 
For a~fixed input, there is only  the bank triggering the default is uniquely 
determined. 

To our mind, these objections should be examined carefully. Due to extreme 
complexity of  financial systems (recall that they may contain hundreds of 
banks) and complexity of individual  balance sheets, for more sophisticated 
models, one can have an accumulation of various factors: misspecification errors, 
calibration errors,  data aggregation errors, etc.   That is why, simplifying 
hypotheses seems to be inevitable. It seems that one  can accept  that banks 
investment portfolios are close to the most performant one.  
{\looseness=-1

}

Of course, the predetermined bank triggering  of the default cascade is not 
intuitive. However, as it is known from the literature, 
the matrix~$L$ is rarely known and should be reconstructed from the aggregated 
exposure of the banks. It is not difficult to  implement a~random reconstruction 
procedure;  for each realized reconstruction, one can compute conditional alert 
indicators and take the average. 
  


    \begin{figure*}[b] %fig1
    \vspace*{1pt}
    \begin{minipage}[t]{80mm}
\begin{center}
\mbox{%
\epsfxsize=77.973mm
\epsfbox{kab-1.eps}
}
\end{center}
\vspace*{-9pt}
         \Caption{Illustration of a~financial  network generated by the 
algorithm and limited to~40~banks  for clarity of the figure}
        \label{fig:1}
        \end{minipage}
        \hfill
            %\end{figure*}
%    \begin{figure*} %fig2
    \vspace*{1pt}
    \begin{minipage}[t]{80mm}
\begin{center}
\mbox{%
\epsfxsize=78.228mm
\epsfbox{kab-2.eps}
}
\end{center}
\vspace*{-9pt}
         \Caption{Representation of the scale-free topology}
        \label{fig:2}
                \end{minipage}
            \end{figure*}   
            
            \vspace*{-6pt}         

\section{Numerical Experiments}
%\label{simulations} 

            \vspace*{-6pt}


\subsection{Network construction}

            \vspace*{-4pt}

\noindent
Here, the  numerical simulations are presented to offer further insight into the role of 
the external assets in contributing to a~systemic risk in the financial system 
and to show an impact of  parameters  range to financial stability.
Table~1 summarizes the baseline simulations parameters. The system comprises~$N$ 
banks. As in~\cite{Eboli},  the  banking system is considered as 
a~network of nodes where each node represents a~bank (or any other financial 
institution) and each link represents a~directional lending relationship between 
two nodes (two banks). 
We believe that network reflects its ``genetic'' structure. The development of the 
system starts from relatively small kernel composed from a~few banks. A~newly 
created bank establishes relationships  preferably with more ``important" 
nodes of the network, namely, those that already have more connections  than 
others. Also, usually, well connecting bank 
 has better chances not to be eliminated from the system (``too 
connected to fail''). 

As an example, let us make a~look on  the development of the banking system in 
Russia. In the USSR, the number of banks were about a~dozen. The first commercial 
bank was registered in August~1988 (Cooperative bank ``Patent,'' Leningrad).  
Already to the end of 1989, the country had~43~commercial banks.   Afterwards,  
the number of banks in Russia evolved (approximately) as follows:  
1991~--- 1400;  1994~--- 2300; 1997~--- 2000; 2003~--- 1300; and  2011~---
1000. 

In   2016, the total number was  less than~700.  To compare:  in~2014, the USA had  
more than~6800~banks.  

Of course, the evolution of the banking system is a~rather complicated process 
but  statistically, it leads to a~network having common features with other 
types of networks like Internet connections.  In particular, interbank networks 
have a~scale-free topology with a~few large banks having many interconnections 
and many small banks with a~few connections.
By this reason, we generate our simulating financial system using the 
methodology introduced in~\cite{BarAlb}. We are 
based on the idea that a~larger and more connected bank is usually more trusted 
and, as a~consequence, other system banks tend to deal with it rather than with 
the less connected  one.

The algorithm starts with creating the initial network (the ``seed'') with 
a~small amount of nodes $n$.  The connections between them are taken randomly, and 
the maximal number of connections (``in'' and ``out'')  is limited by $m$ as well 
as the total number~$M$ of connections (in the present experiments, $n=10$, $m=5$, 
and $M=20$). 
  When drawing the network, we allow for the possibility that two banks can be 
linked to each other via both lending and borrowing links but at most one   in 
the same direction is possible between the~2~banks (Fig.~1).

\smallskip  

\noindent
\textbf{Remark.}\ In general, we expect that the structure of the initial  network 
has a~relatively small impact 
on the resulting network which may be in dozens or even hundred times larger 
than the ``seed.'' 
So, the initial network  involving a~few nodes can be created in various way, 
e.\,g., as Erd\"os--R$\acute{\mbox{e}}$nyi network with the matrix 
$B=(b^{ij})$ where  entries~$b^{i,j}$, $i\neq j$, form a~set of independent
 identically distributed  Bernoulli random variables with $P(b^{i,j}=1)=p$.      

\begin{table*}\small 
    \begin{center}
        \Caption{Summary of the system benchmark parameters}
        \vspace*{2ex}
    
    \begin{tabular}{clcc}
        \hline
         Parameter & \multicolumn{1}{c}{Description} &  Value & Range of variation \\ 
        \hline
        $A$  & Value of the total assets of the network  & 10,000,000,000  & 
fixed \\ 
       % \hline
        $\beta$  & Proportion of external assets from the total assets of a~
bank  & 0.5 & fixed \\
        %\hline
        $S_0$  &    Initial price of risky asset  & 10,000 & fixed \\ 
        %\hline
        $T$  & Time horizon & 40    & fixed \\
        \hline
    \end{tabular}
\end{center}
%\end{table*}
\vspace*{9pt}
%\begin{table*}\small %tabl3
\begin{center}
    \Caption{Summary of the balance sheet parameters}
    \vspace*{2ex}
    
    \begin{tabular}{clcc}
        \hline
       Parameter & \multicolumn{1}{c}{Description} &  Value & Range of variation \\ 
        \hline
        $\alpha$  & 
        \tabcolsep=0pt\begin{tabular}{l}Percentage of price degradation following the panic\\ in 
the market during the crisis\end{tabular}  & 1\hphantom{.9}  & 0 to 1 \\ 
        \hline
        $R$  & Recovery rate & 0.5 & 0 to 1 \\
        \hline
        $\theta$  & Percentage of riskless asset from total external asset  
& 0.4 & 0 to 1 \\ 
        \hline
        $\sigma$  & Volatility of the risky asset price & 0.4   & 0 to 1 \\
        \hline
        $\mu$  & Drift of the risky asset price & 0.2   & 0 to 1 \\
        \hline
    \end{tabular}
\end{center}
\vspace*{6pt}
\end{table*}


To generate a~scale free network, assuming that a~seed network of~$n$~banks is 
randomly generated as mentioned above, let us proceed recursively. At each step,  
add  $m'<m$ new nodes choosing each time a~partner~$i$ between the  existing 
nodes and selecting according to the  Connection Probability~$P(i)$ defined as 
follows: 
$$
P(i ) = \fr{\mbox{total\ number\ of\ connections\ of\ $i$}}{\mbox{total\ number\ of\ 
connections\ of\ the\ network}}
$$ 
where~$i$ is the existing node in the network. The nodes will be added each time 
until  a~network size limit  equal to~$N$ is reached. As shown in Fig.~\ref{fig:2},  
the 
distribution of a~number of nodes in the considered model is in a~conformity with that one 
can expect from   a~typical scale-free network topology where a~few nodes have 
a~high number of connections while the majority of nodes have a~small number of 
connections. In particular, 
on the realization of the algorithm depicted on the scatterplot, only~6~banks 
from~250 have each at least~25~connections while~185 have at most~5~connections.  




For any realization of the random graph, we populate the individual banks' 
balance sheets in a~manner consistent with bank level and aggregate balance 
sheet identities (Table~3).
 


Amongst assets, let us distinguish  external assets (investors' borrowing), denoted 
by~$C^i$, and interbank assets (other banks borrowing), denoted by~$a^i$. Thus, 
for the bank~$i$,  the asset part of the balance sheet  can be decomposed as  
$$
\mbox{Total\ assets} = C^i + a^i\,, \qquad i = 1,\ldots,N\,.
$$ 
Moreover, the external assets can be of two types: risky~$r^i$ and riskless 
(cash)~$l^i$, so that   $C^i=r^i+l^i$.  Introducing the parameter~$\theta$ as a~
proportion of cash holdings, the volume of the risky assets is $r^i = (1- 
\theta) C^i$.  
The liabilities of each bank  are composed of the net worth of a~bank, denoted 
by~$NW^i$, and the interbank borrowing, denoted by~$b^i$. Hence, for the bank~$i$, 
one has:  
$$
\mbox{Total\ liabilities}= NW^i + b^i\,, \qquad i = 1,\ldots,N\,.
$$ 
An example of a~bank balance sheet as generated by the simulator is also shown 
in Fig.~3.  

    

The balance sheets have been constructed for individual banks in a~sequence  of steps.
The entry parameter is the total value of all assets in the system denoted 
by~$A$ and the parameter~$\beta$ which defines the proportion of the external 
asset~$C$ representing the total loans made to ultimate investors and thus 
relating to the total size\linebreak\vspace*{-12pt}

\columnbreak

 { \begin{center}  %fig3
 \vspace*{-3pt}
 \mbox{%
\epsfxsize=78.001mm
\epsfbox{kab-3.eps}
}


%\vspace*{-3pt}


\noindent
{{\figurename~3}\ \ \small{Representation of the balance sheet}}
\end{center}
}

\vspace*{9pt}

\addtocounter{figure}{1}




\noindent
 of the flow of funds from savers to investors through 
the banking system. That is, $\beta = C/A$.
 The aggregate assets of the whole banking industry can be written as $A = C + 
I$ where $I=(1- \beta)A$ represents the aggregate volume of interbank 
exposures, i.\,e., $I:=\sum\nolimits_{i,j}L^{ij}$.  


Dividing the total interbank assets by the total number of nodes in the network, 
one arrives at the level of each bank.  So, weights of all links are equal banks 
borrow and lend by equal  portions $w=I/|B|$. Though this looks not very 
realistic,  such a~hypothesis is accepted to reduce the number of parameters.  
Hence, using~$w$ and the structure of the network, one
can calculate for each bank the volume of its liabilities $a^i=n_{\mathrm{in}}(i)w$ and  
exposures $b ^i=n_{\mathrm{out}}(i)w$.
For any bank to be able to operate, the value of its external 
assets is required to be not less than its net interbank borrowing, that is, 
one has:
 $C^i \ge 
b^i- a^i$. Let us fulfill this constraint by applying the following two-step 
algorithm.

First, for each bank, fill up the bank external assets part of the balance 
sheet in such a~way that its external assets plus interbank lending will 
equalize its interbank borrowing. That is, provide first the bank~$i$ the 
volume $\tilde C^i = (b_i-a_i)^-$ where $\tilde C^i$ is the fraction of the 
total volume $C$ reserved for the external assets.
At the second stage,  what is left in aggregated external assets is equally 
distributed among all banks. Note that the total of external assets is equal to~$C$. 
Hence, in the second step, distribute $\bar C =C- \sum^N_{i=1} \tilde 
C^i$ equally among all~$N$~banks. Hence, one has $C^i = \tilde C^i +\bar C/N $. 
The constraint can become difficult to meet if the percentage of external asset 
is too low. Since the distribution of links is stochastic, some banks may be 
assigned interbank borrowing much larger than interbank lending. When the total 
amount of external assets is low, there may be not enough assets to go round to 
close all balance sheet gaps opened up in this way. To avoid this difficulty,  
make sure that the total volume of external assets is at least~30\%~of the total 
volume of all assets.

%\smallskip

Although the model is applied to fully heterogeneous banks, for the purpose of 
illustration and simplicity, let us consider one common risky asset for all banks 
(full correlation among the banks). Further studies can be conducted for 
portfolios composed  of many  different risky assets. Furthermore, let us choose the 
risky asset evolving according to value of a~reference portfolio  whose dynamics 
follows a~geometric Brownian motion~(\ref{e3a-kab}).

Since $S_t = S_0$  at time $t = 0$, one can define~$y^i$ as the amount of the 
risky assets in the portfolio of each bank with $y^i =r^i/S_0$.

Hence, the asset side of the bank balance sheet is completed as well as interbank 
borrowing~$b$ on the liability side. The determination of the remaining 
component, the net worth (equity)~$\mathrm{NW}$ on the liability side, is relatively 
straightforward. The net worth is set as 
$\mathrm{NW} = a~+C -b$. This completes the construction of the banking system and of 
each constituent bank balance sheet. 

Now, let us calculate the probability of the first default. Then, let us
specify which 
bank will first default to check the price of the risky asset at the time of 
default~$S_\tau$ where~$\tau$ is the time of the first default (1st stopping 
time). This generates a~loss in the asset price from~$S_0$  to~$S_\tau$   on 
each bank~$i$ which will suffer a~loss of  $y^i  (S_0  - S_\tau )$. The sum of 
these losses  is equal to $\sum^N_{i=1} y^i  (S_0  - S_\tau )$ and is denoted   
as Corr\_loss (correlation loss).
In what follows, it is assumed that the first default will affect the neighborhood  
and will trigger a~cascade of default due to the loss transmission through the 
interbank connections. The sum of the total losses generated by the cascade of 
default is denoted as Con\_loss and is equal to $(1-R) \sum\nolimits^N_{n=0} 
\sum\nolimits_{i\in  D_{n+1}} \sum\nolimits_{ \in D_n}$. 
Having the probability as well as
$$
\mbox{Network\ total\ loss\;=\;Correlation\ loss\,+\,Contagion\ loss,}
$$ 
one can calculate  
$$
\mbox{
Probable\ loss\ Indicator\;=\;$P$\;$\cdot$\;Total\ loss\,.}
$$

In what follows, the parameters will be varied in the experiments.  

\subsection{Experiment~1: The influence of~the~external assets volume 
and~its~riskless proportion on~the~probability~and~losses}

\noindent
Given the balance sheet above we want to compute the probability of default when 
the proportion of the risky asset varies.
For the sake of more profit and wealth, banks have the choice to invest in 
either risky assets or riskless assets. It is known that the revenue in risky 
assets can be much larger than the riskless ones. Thus, banks have more tendency to 
invest there while keeping an eye on the risks and their liabilities towards 
other banks making sure they can settle when needed. Therefore, let us check 
the probability of the first default that can trigger a~cascade of defaults in the 
system based on the percentage of the risky assets. Regulators may impose 
a~minimum threshold on the level of riskless assets~$(\theta)$ held by banks as 
well as define a~certain proportion that a~bank can invest in  external assets. 
That is why, both parameters will be considered in this experience to check how 
both of them can affect the vulnerability of the system and what is the optimal 
requirement on the riskless assets thresholds for each volume of external 
investments.

As for the following figures, this experiment illustrates that with an increase 
of the riskless assets level, the probability of the first default decreases to 
become zero after a~certain threshold of~$\theta.$
Also, note that this threshold is smaller with an increasing volume of 
assets~$\beta$ (Fig.~4).

For example, for $\beta = 0.55 $, the probability of default is always less 
than~$0.5$ decreasing with~$\theta$ and becomes null after $\theta =0.6$. If the 
system is engaged with high level of external investments $\beta > 0.75 $, the 
probability of default is very low with a~very low rate of riskless assets 
$\theta = 0.55$. This means that the system is rather stable even if the level 
of risky assets is high. On the other hand, for a~high level of external assets 
investments as shown in Fig.~5, the probability of the first default 
remains low even with high level of risky assets.

   



In the following experiments,  $\beta=0.5 $  will be fixed as it is assumed that banks 
have the same probability to invest in external and internal assets. 

\pagebreak

\end{multicols}

 \begin{figure*} %fig4
    \vspace*{1pt}
\begin{center}
\mbox{%
\epsfxsize=162.513mm
\epsfbox{kab-4.eps}
}
\end{center}
\vspace*{-9pt}
       \Caption{Probability of default with~$\beta$ percentage of external 
assets and~$\theta$ proportion of riskless assets: \textit{1}~--- $\beta=0.35$;
\textit{2}~--- 0.45; \textit{3}~--- 0.55; \textit{4}~---0.65;
\textit{5}~--- 0.75; \textit{6}~--- 0.80; \textit{7}~--- 0.85;
\textit{8}~--- 0.90; and \textit{9}~--- $\beta=0.95$}
   % \end{figure*}
     %   \begin{figure*} %fig5
    \vspace*{6pt}
\begin{center}
\mbox{%
\epsfxsize=163.207mm
\epsfbox{kab-5.eps}
}
\end{center}
\vspace*{-9pt}
        \Caption{Probability of default with $\theta$ and  volatility of 
the asset price: \textit{1}~--- $\theta=0.1$;
\textit{2}~--- 0.2; \textit{3}~--- 0.3;
\textit{4}~--- 0.4;
\textit{5}~--- 0.5;
\textit{6}~--- 0.6;
\textit{7}~--- 0.7;
\textit{8}~--- 0.8;
\textit{9}~--- 0.9;
and \textit{10}~--- $\theta=1.0$}
        \label{fig:5}
       % \vspace*{6pt}
        \end{figure*}

\begin{multicols}{2}

\subsection{Experiment~2: The influence of~the~volatility and~drift of~the~risky~asset~price}

\noindent
Now, let us study how the volatility and drift of the asset price can also 
affect the probability taking into consideration also the level of the risky 
assets in banks balance sheet.  The volatility reflects the risk of changing the 
portfolio price due to  external and internal factors. 


Assuming that the portfolio of risky asset price has a~volatility that varies 
from~$0.1$ to~$1$, one can conclude that for the volatility less than~$0.25$, 
the probability of default is low for any~$\theta$ level. When volatility 
increases above~$0.25$, the probability of default increases with $\theta$ 
decreasing and volatility increasing. The pattern of every plot is changing with 
$\theta$: having the plot concave for high~$\theta$ and convex for low~$\theta$ 
shows that the behavior of the volatility influences more the probability of 
default since for higher~$\theta$, the increase in probability is faster than the 
lower one. So, summarizing, the volatility should be limited to a~certain extent 
in order to save the network from a~high probable default; otherwise, a~high 
impact will be affecting the asset price which, in its turn, will trigger a~higher 
indicator of default.




        
       
       \begin{figure*} %fig6
       \vspace*{1pt}
\begin{center}
\mbox{%
\epsfxsize=163.212mm
\epsfbox{kab-7.eps}
}
\end{center}
\vspace*{-9pt}
        \Caption{Probability of default with $\theta$  and the drift of 
the asset price: \textit{1}~--- $\theta=0.1$;
\textit{2}~--- 0.2; \textit{3}~--- 0.3;
\textit{4}~--- 0.4;
\textit{5}~--- 0.5;
\textit{6}~--- 0.6;
\textit{7}~--- 0.7;
\textit{8}~--- 0.8;
\textit{9}~--- 0.9;
and \textit{10}~--- $\theta=1.0$}
        \label{fig:7}
        \end{figure*}
        
         \begin{figure*}[b]  %fig7
 \vspace*{-3pt}
 \begin{minipage}[t]{80mm}
 \begin{center}
\mbox{%
\epsfxsize=78.822mm
\epsfbox{kab-9.eps}
}
\end{center}
\vspace*{-12pt}
\Caption{Contagion losses}
\end{minipage}
%\end{figure}
\hfill
%\begin{figure}  %fig8
 \vspace*{-3pt}
  \begin{minipage}[t]{80mm}
 \begin{center}
\mbox{%
\epsfxsize=78.336mm
\epsfbox{kab-10.eps}
}
\end{center}
\vspace*{-12pt}
\Caption{Fire sale losses}
\end{minipage}
\end{figure*}
        
           



Moreover, we also evaluate the effect of the asset price drift on the 
probability considering at the same time the level of the risky assets in banks' 
balance sheet. We conclude that a~portfolio price with high drift or high 
average of return will, for sure, lead to a~more stable financial system.  The 
optimal portfolio would be with high drift and low volatility.

Figure~6\textit{a} shows for each level~$\theta$ the variation of the probability 
relatively to the drift. In particular, one can see that for $\theta = 0.5$, the 
level of drift required to assure a~probability less than~0.2 is also~0.5 
but if $\theta = 0.6$, one can see that  in this case,
the required level of drift is~0.05.

Figure~6\textit{b} highlights the relation between the volatility and the drift affecting 
the probability of default. It is normal that the probability of default 
increases with increasing volatility and decreases with increasing drift. But 
from Fig.~6\textit{a}, one can 
also see that when volatility is very high, the influence of 
the increasing drift is reduced.
{\looseness=1

}




\subsection{Experiment 3: The~impact of recovery rate and~the~level of~the~riskless 
asset in~the~banks' balance sheet on~the~losses}

\noindent
In the present model, the system loss is another parameter that influences the indicator 
as probability of the first default can be low; however, when it happens, the loss in 
the system can be huge. Therefore, we analyze the behavior of the network when 
recovery rate is changing knowing that recovery rate is one of the main 
influencer of the losses due to the default cascade.
Figure~7 shows that the total loss in the system decreases when 
recovery rate is increasing  due to the fact that defaulted banks need to settle 
their due payment. One can clearly observe that there is a~linear relation 
between the recovery rate and the contagion losses.



 \subsection{Experiment~4: The impact of~the~fire~sale~on~the~losses} 

\noindent
Also, let us consider a~subordinate source of risk due to the fire sale of 
external assets of defaulting banks which
 will lead to other banks default 
because of the price depreciation. In bad times,  in order to compensate certain 
losses, distressed banks tend to sell assets in a~depressed price, a~situation 
called asset ``fire sale.'' Because of the correlation between the banks' balance 
sheets having common assets, the decrease in the asset price affects all banks 
holding these assets and, as a~consequence, a~cascade of losses created by others 
banks, too. In this experiment, we want to compute what will be the losses 
resulting from the ``fire sale'' mechanism that will respectively add to the 
total loss of the system (Fig.~8). It is expected that the fire sales loss will increase 
with the drop rate of the price but it is worth to note that the increase is 
very sharp and fast.




    \begin{figure*} %fig9
    \vspace*{1pt}
\begin{center}
\mbox{%
\epsfxsize=162.513mm
\epsfbox{kab-11.eps}
}
\end{center}
\vspace*{-9pt}
                \Caption{Comparison between before~(\textit{1}) 
                and after~(\textit{2}) removing the weak 
bank(s): (\textit{a})~probability of default; 
and (\textit{b})~total system loss}
        \end{figure*}


   


\subsection{Experiment~5:  Removing the~weakest bank to~make the~system more 
resilient} 

\noindent
The financial system is a~number of financial institutions, in the present paper 
considered as banks. Every bank has its investment strategy, priorities, 
relationship, and connections as a~consequence different influence, power, and 
risk level in the financial system.
Banks balance sheets are populated on the quarterly basis and can be available 
to regulators at any time. Thus, it is possible to understand the risk level of 
each bank to be defaulted. Since banks default can generate a~default cascade, 
it is worth to verify if it is better to exclude the risky bank from the 
financial network.  It is important to determine  the weakest bank that can 
default anytime either due to a~market price drop or liquidity shortage.

In this experiment, we create a~scale-free network of~250~nodes using the same 
methodology as above and determine the bank~$i$, having the highest probability 
to default or, in other words, the weakest bank to default. Also, we calculate, 
on the  basis of the above, the probable loss Indicator of this network. In some 
cases,\linebreak we may have more than one bank defaulting simultaneously.  


Once the bank is identified, let us check if removing the bank from the 
network is healthier for the financial system.  Of course, such an action has 
important  consequences to the owner, employees as well as debtors and creditors 
in this bank.  The modification on  the balance sheet of this bank 
and all banks connected and in relation to this bank will be described as follows:
\begin{itemize}
\item removed bank pays all its liabilities to all its creditors and, as 
a~consequence, updates its creditors' balance sheets~$a_j$ by adding the amount 
of money bank~$i$ has landed from bank~$j$ to the riskless assets;

\item removed bank collects all its exposures from all its debtors and, as 
a~consequence, updates its debtors balance sheets~$b_j$ by removing the amount of 
money bank~$i$ has credited to bank~$j$. Bank~$j$ is supposed to pay to bank~$i$ 
from their external assets (riskless assets and risky assets; so, either bank~$j$ 
has to pay from its liquid reserve or sell some external assets to pay its due).
\end{itemize}

We conduct this experiment and compute the probability, total loss, and indicator 
having the vulnerable bank (banks) in the system and after removing it (them) 
from the network.

Figure~9 confirms that when removing the risky bank, the probability of 
default will decrease. This means that the system becomes more resilient to 
default.



Now, let us check the total losses that may occur in the system due to a~default in 
the above two mentioned cases and note that, contrary to the probability, the 
total losses will be higher in the case where it has been decided to remove the 
vulnerable banks. The reason could be that the external assets prices that have 
decreased to~$S_\tau$ causing the first default in the first round has to 
decrease more to trigger the default after removing the set of banks that could 
defaulted on~$S_\tau$. 
In this case, though the probability is lower, the increase in the total losses 
from before to after removing the banks is due to the fact 
that the risky assets price should drop more and, accordingly, the correlation 
losses increase. On the other hand, since the shock on the net worth of every 
bank becomes higher, we expect that the contagion losses are larger now.
Having both correlations and contagion losses higher, this will lead to a~high 
total loss as shown in Fig.~9. It is worthy to note that for small 
recovery rate, the increase of the total loss after removing the weak banks is 
even higher.




\section{Concluding Remarks} 

\noindent
In this paper, a~financial network model of interbank interactions has been developed 
which incorporates a~dynamical behavior of banks portfolios and combines it with 
cascade defaults. It is assumed that the portfolios contain, 
together with riskless asset, a~unique risky asset, which can be interpreted as a~market index or a~benchmark portfolio and whose price evolution is described 
by a~geometric Brownian motion. This part of modeling follows the ideas of the 
so-called structural approach well known in the context of pricing defaultable 
securities. A~crisis starts  
when the net worth of a~bank hits zero triggering a~cascade of defaults. The 
time to default and the total losses calculated for the ``frozen'' parameters of 
the balance sheets  can constitute indicators of the ``health'' of financial 
system. They can be easily  monitored,  on a~regular basis, by the regulators. 
Usually, the detailed structure of the system is available only to regulators 
but it is  not public. By this reason,  the present study is completed with numerical 
experiments with 
simulated data. Due to complexity of financial systems this is a~nontrivial 
problem.  The network graph is built by a~version  of preferable attachment 
algorithm augmented by a~procedure of simulations of balance sheets.  The results of 
the experiments are presented by plots showing dependences of the indicators as 
functions   of parameters.    In particular, an experiment with 
removing the weakest  bank from the system is presented. We believe that developing 
this 
approach on the basis of practical data 
can provide regulator additional tools of monitoring vulnerability of banking 
system and measuring its stability. 
  

\Ack  
\noindent
This work was done 
under partial financial support   of the grant 
of  the Russian Science Foundation No.\,14-49-00079.

\renewcommand{\bibname}{\protect\rmfamily References}

\vspace*{-6pt}

{\small\frenchspacing
{%\baselineskip=10.8pt
\begin{thebibliography}{99}
\bibitem{Hurd} %1
\Aue{Hurd, T.} 
2016. \textit{Contagion! The spread of systemic risk in financial networks}.  Springer.
139~p.

\bibitem{Bandt} %2
\Aue{De Bandt, O., P.~Hartmann, and J.~Peydr$\acute{\mbox{o}}$}.  2009.
Systemic risk in banking: An update.  \textit{Oxford handbook of banking}.
Eds.\ A.~Berger, P.~Molyneux, and J.~Wilson. Oxford 
University Press. 994~p.

\bibitem{GK} %3
\Aue{Gai, P., and S.~Kapadia}. 2010.
Contagion in financial networks. Bank of England. Working 
Paper~383. 36~p.
\bibitem{AY} %4
\Aue{Acharya, V., and T.~Yorulmazer}. 2008.
Information contagion and bank herding. \textit{J.~Money Credit Bank}.   40(1):215--231.
\bibitem{ElsingerLS1} %5
\Aue{Elsinger, H., A.~Lehar,  and M.~Summer}. 2006. Risk assessment for banking systems. 
\textit{Manage. Sci.} 52(9):1301--1314.

\bibitem{ElsingerLS2} %6
\Aue{Elsinger, H., A.~Lehar,  and M.~Summer}. 2006.
Systemically important banks: An analysis 
for the European banking system. \textit{Int. Econ. Econ. Policy} 
1:73--89.3.
\bibitem{7a}
\Aue{Gai, P., and S.~Kapadia}. 
2010. Contagion in financial networks. \textit{Proc.~R. Soc.~A} 466:2401--2423.  



\bibitem{Allen} %7
\Aue{Allen, F., and D.~Gale}. 
2000. Financial contagion. \textit{J.~Polit. Econ.} 108(1):1--33.
\bibitem{AB} %8
\Aue{Acharya V., and A.~Bisin}. 2014. Counterparty risk externality: Centralized 
versus over-the-counter markets. \textit{J.~Econ. Theory} 149(C):153--182.

\bibitem{Co} %9
\Aue{Georg, C.\,P.} 2013.  The effect of the interbank network structure on contagion and 
common shocks.  \textit{J.~Bank. Financ.} 37(7).  
\bibitem{MA} %10
\Aue{May, R.\,M., and N.~Arinaminpathy}. 2010. 
Systemic risk: The dynamics of model banking systems. 
\textit{J.~R.~Soc. Interface} 7:823--838.   
\bibitem{Biel-Rut} %11
\Aue{Bielecki, T.\,R.,  and M.~Rutkowski}. 2002.
\textit{Credit risk: Modelling, valuation and hedging}. 
Berlin: Springer.  501~p.
\bibitem{BarAlb} %12
\Aue{Barab$\acute{\mbox{a}}$si,~A., and R.~Albert}. 
1999. Emergence of scaling in random networks. \textit{Science} 286:509--512.
\bibitem{Mistrulli} %13
\Aue{Mistrulli, P.\,E.} 2007.
  Assessing financial contagion in the interbank market: Maximum 
entropy versus observed interbank lending patterns. Rome, Italy: Bank of 
Italy. Working Paper~641.
 
\bibitem{NierYYA} %14
\Aue{Nier, E.,  J.~Yang, T.~Yorulmazer, and A.~Alentorn}. 2008.
 Network models and financial 
stability.  Bank of England. Working Paper~346.
% \bibitem{BIS}
%Bank of International Settlements.  1994. 
% Basel, Switzerland: BIS. 64th  Annual Report. 221~p.

%\bibitem{Bartho}
%\Aue{Bartholomew, P., L.~Mote, and G.~Whalen}. 1995. The definition of systemic risk. 
%\textit{Annual Meeting of the Western Economic Association}. San
%Diego, CA.


 
%\bibitem{Billio}
%\Aue{Billio, M., M.~Getmansky, A.~Lo,  and L.~Pelizzon}. 
%2010. Measuring systemic risk in the 
%finance and insurance sectors. MIT Sloan School. Working Paper 4774. 58~p.

%\bibitem{Bisias}
%\Aue{Bisias, D., M.~Flood, A.~A., and S.~Valavanis}. 2012.
% A~survey of systemic risk analytics. 
%Office of Financial Research. Working Paper. 160~p.

%\bibitem{Board} 
%Board of Governors of the Federal Reserve System. 2001. Policy statement on payment 
%systems risk. 66 FR 30199. 7~p.



%\bibitem{Cont-Moussa-Santos} 
%\Aue{Cont, R., A.~Moussa, and E.~Santos}.  2010.
%Network structure and systemic risk in banking 
%systems. Preprint. 41~p.
%Available at: {\sf http://papers.ssrn.com/sol3/id=1733528}  (accessed April~25, 2017).

%\bibitem{Dardac}
%\Aue{Dardac, N., and M.~Bogdan}. 2006. Assessing inter-bank contagion phenomenon using 
%quantitative methods. A~case study on Romania.
%Available at: {\sf http:// www.ase.ro/upcpr/profesori/756/Contagion\_2007.pdf} 
%(accessed April~25, 2017).

\bibitem{Bor-Sal} %15
\Aue{Borodin, A., and P.~Salminen}. 2002.
\textit{Handbook of Brownian motion: Facts and formulae}.  2nd 
ed.  Birkh$\ddot{\mbox{a}}$user. 465~p.


\bibitem{Eboli} %16
\Aue{Eboli, M.} 2004.
Systemic risk in financial networks: A~graph-theoretic approach. Italy: 
\mbox{Universit{\!\ptb{\`{a}}}} di Chieti Pescara. Preprint.  19~p.

%\bibitem{Eisenberg-Noe}
%\Aue{Eisenberg, L., and T.\,H.~Noe}. 2001.
%Systemic risk in financial systems. \textit{Manage. Sci.} 
%47(2):236--249.



%\bibitem{ECB}
%European Central Bank (ECB). 2010. Financial networks and financial stability. 
%\textit{Financ. Stability Rev.} 155--160.







%\bibitem{Jorda}
%\Aue{Jorda, O., M.~Schularick, and A.\,M.~Taylor}. 2014.
%Betting the house. Working paper ser. 28. 
%Federal Reserve Bank of San Francisco. 43~p.



%\bibitem{Mattig}
%\Aue{Mattig, A., and S.~Morkoetter}.  2011.
%Financial crises and agency of regulatory responses.
%\textit{Systemic risk, Basel III, Financial stability and 
%regulation}. 25~p.



%\bibitem{Mishkin}
%\Aue{Mishkin, F.} 2007. Systemic risk and
%the international lender of last resort.
%\textit{10th Annual International Banking Conference}.
%Chicago, IL. 
%Federal Reserve  Bank of Chicago. 
%Available at: {\sf https://www.federalreserve.gov/newsevents/speech/mishkin20070928a.htm}
%(accessed April~25, 2017).



%\bibitem{Moussa}
%\Aue{Moussa, A.} 2011. Contagion and systemic risk in financial networks. Columbia 
%University. D.Sc.\ Diss.
%Available at: {\sf http://hdl.handle.net/10022/AC:P:10249} (accessed April~25, 2017).



%\bibitem{Monetary} 
%The International Monetary Fund and the Financial Stability Board. 2009.  Fighting the 
%global crisis. Annual Report.

\end{thebibliography} } }

\end{multicols}

\vspace*{-6pt}

\hfill{\small\textit{Received November 14, 2016}}

%\vspace*{-18pt}

\Contr

%\vspace*{-3pt}

\noindent
\textbf{El Bitar  Khalil} (b.\ 1981)~--- 
PhD student, Laboratoire de Mathematiques, Universite de Franche-Comte, 
16~Route de Gray, 25030, \mbox{Besan{\!\ptb{\c{c}}}on}, CEDEX, France; 
\mbox{khalilbitar\_aw@hotmail.com}  

 \vspace*{1pt}
 
 \noindent
 \textbf{Kabanov Yuri M.} (b.\ 1948)~---
  professor, Laboratoire de Mathematiques, Universite de Franche-Comte, 
  16~Route de Gray, 25030, Besancon, CEDEX, France; leading scientist, 
  Institute of Informatics Problems, Federal Research Center 
  ``Computer Science and Control'' of the Russian Academy of Sciences,  
  44-2~Vavilov Str., Moscow 119333, Russian Federation; 
  National Research University ``MPEI,'' 14~Krasnokazarmennaya Str., 
  Moscow 111250, Russian Federation; \mbox{Youri.Kabanov@univ-fcomte.fr} 

\vspace*{1pt}
 
 \noindent
 \textbf{Mokbel Rita} (b.\ 1981)~--- 
 PhD student, Laboratoire de Mathematiques, Universite de Franche-Comte, 
 16~Route de Gray, 25030, Besancon, CEDEX, France; \mbox{ritamokbel@hotmail.com}




%\vspace*{8pt}

%\hrule

%\vspace*{2pt}

%\hrule

\newpage

\vspace*{-24pt}



\def\tit{ДИНАМИЧЕСКИЕ МОДЕЛИ СИСТЕМНOГО РИСКА  И~ЗАРАЖЕНИЯ$^*$}

\def\aut{Х.~Эль Битар$^1$, Ю.~Кабанов$^{1,2,3}$, Р.~Мокбель$^1$}


\def\titkol{Динамические модели системнoго риска  и~заражения}

\def\autkol{Х.~Эль Битар, Ю.~Кабанов, Р.~Мокбель}



{\renewcommand{\thefootnote}{\fnsymbol{footnote}}
\footnotetext[1]{Работа выполнена при частичной поддержке Российского научного 
фонда (грант №\,14-49-00079).}}



\titel{\tit}{\aut}{\autkol}{\titkol}

\vspace*{-12pt}

\noindent
$^1$Лаборатория математики Университета Франш-Кон\-те, г.~Безансон, Франция

\noindent
$^2$Институт проблем информатики Федерального исследовательского
центра <<Информатика и~управление>>\linebreak
$\hphantom{^1}$Российской академии наук, Российский
университет дружбы народов

\noindent
$^3$Национальный исследовательский университет <<МЭИ>>

\vspace*{6pt}

\def\leftfootline{\small{\textbf{\thepage}
\hfill ИНФОРМАТИКА И ЕЁ ПРИМЕНЕНИЯ\ \ \ том\ 11\ \ \ выпуск\ 2\ \ \ 2017}
}%
 \def\rightfootline{\small{ИНФОРМАТИКА И ЕЁ ПРИМЕНЕНИЯ\ \ \ том\ 11\ \ \ выпуск\ 2\ \ \ 2017
\hfill \textbf{\thepage}}}

\vspace*{-4pt}

\Abst{Современные финансовые системы являются сложными сетями взаимосвязанных 
финансовых институтов (банков, хедж-фон\-дов, 
страховых компаний, и~т.\,д.), и~дефолт одного из них может вызвать цепную реакцию 
дефолтов других институтов системы. После недавних финансовых кризисов важность 
системного риска вышла на первый план, и~теоретические исследования в~этой об\-ласти 
интенсифицировались. Б$\acute{\mbox{о}}$льшая часть известных результатов 
относится к~статическим моделям, которые посвящены процессам,  происходящим в~сис\-те\-ме, 
когда каскад дефолтов уже начался. Авторы предлагают 
динамическую модель так называемого структурного типа, когда дефолт 
начинается в~момент выхода некоторого стохастического процесса из области. 
Каскад инициируется в~момент достижения критического уровня процессом, описывающим 
портфели банков. Вероятность выхода и~суммарные издержки в~результате каскада 
дефолтов 
могут служить индикаторами, позволяющими регуляторам осуществлять мониторинг системы 
и~предпринимать упреждающие коррекции для понижения сис\-тем\-но\-го риска.  
Проводится численное моделирование сис\-те\-мы, которая строится на осно\-ве 
случайного графа, полученного при помощи алгоритма предпочтительного присоединения. 
Приводятся результаты численных экспериментов при различных значениях параметров.}  

\KW{системный риск; финансовые сети; заражение; дефолт}

\DOI{10.14357/19922264170201} 

%\vspace*{6pt}


 \begin{multicols}{2}

\renewcommand{\bibname}{\protect\rmfamily Литература}
%\renewcommand{\bibname}{\large\protect\rm References}

{\small\frenchspacing
{%\baselineskip=10.8pt
\begin{thebibliography}{99}
\bibitem{23-k} %1
\Au{Hurd T.}  Contagion! The spread of systemic risk in financial networks.~---
Springer, 2016. 139~p.
\bibitem{14-k} %2
\Au{De Bandt~O.,  Hartmann P., Peydro~J.}
 Systemic risk in banking: An update~// Oxford handbook of banking
 ~/ Eds. A.~Berger, P.~Molyneux, J.~Wilson.~--- 
 Oxford University Press, 2009. 994~p.
 \bibitem{20-k} %3
\Au{Gai P., Kapadia S.}
Contagion in financial networks.~--- Bank of England, 2010. Working Paper~383. 36~p.

\bibitem{2-k} %4
\Au{Acharya, V., Yorulmazer T.}  Information contagion and bank herding~//
J.~Money Credit Bank., 2008. Vol.~40. No.\,1. P.~215--231.
\bibitem{17-k} %5
\Au{Elsinger~H., Lehar~A., Summer~M}.  Risk assessment for banking systems~//
Manage. Sci., 2006. Vol.~52. No.\,9. P.~1301--1314.
\bibitem{18-k} %6
\Au{Elsinger H.,  Lehar A., Summer M.}  
Systemically important banks: An analysis for the European banking system~//
Int. Econ. Econ. Policy, 2006. Vol.~3. No.\,1. P.~73--89.
\bibitem{7a-1}
\Au{Gai P., Kapadia S.} 
 Contagion in financial networks~// Proc.~R. Soc.~A, 2010. Vol.~466. P.~2401--2423.  
\bibitem{3-k} %7
\Au{Allen F., Gale D.} Financial contagion~//
 J.~Polit. Econ., 2000. Vol.~108. No.\,1. P.~1--33.
 \bibitem{1-k} %8
\Au{Acharya, V., Bisin A.} Counterparty risk externality: Centralized versus
over-the-counter markets~// J.~Econ. Theory, 2014. Vol.~149(C). P.~153--182.
\bibitem{21-k} %9
\Au{Georg C.\,P.}  The effect of the interbank network structure on contagion and 
common shocks~// J.~Bank. Financ., 2013. Vol.~37. No.\,7. P.~2216--2228.
\bibitem{25-k} %10
\Au{May R.\,M., Arinaminpathy~n.} 
 Systemic risk: The dynamics of model banking systems~// J.~R.~Soc. Interface,
 2010. Vol.~7. P.~823--838.
\bibitem{7-k} %11
\Au{Bielecki, T.\,R., Rutkowski M.} 
Credit risk: Modelling, valuation and hedging.~---
Berlin:  Springer, 2002. 501~p.
\bibitem{4-k} %12
\Au{Barab$\acute{\mbox{a}}$si A., Albert R}. Emergence of scaling in random networks~//
Science, 1999. Vol.~286. P.~509--512.

\bibitem{27-k} %13
\Au{Mistrulli P.\,E.} Assessing financial contagion in the interbank market: Maximum 
entropy versus observed interbank lending patterns.~---  
Rome, Italy: Bank of Italy,  2007. Working Paper~641.
\bibitem{29-k} %14
\Au{Nier E., Yang J., Yorulmazer~T., Alentorn~A.} 
Network models and financial stability.~---  Bank of England, 2008. Working Paper 346.

\bibitem{11-k} %15
\Au{Borodin~A., Salminen P.} 
Handbook of Brownian motion: Facts and formulae.~--- 2nd ed.~--- Birkh$\ddot{\mbox{a}}$user,
2002. 465~p.

\bibitem{15-k} %16
\Au{Eboli M.}  Systemic risk in financial networks: 
A~graph-theoretic approach.~--- Italy: 
\mbox{Universit{\!\ptb{\`{a}}}} di Chieti Pescara,
2007. Preprint. 19~p.

%\bibitem{5-k}
%Bank of International Settlements.~---  
%Basel, Switzerland: BIZ, 1994.  64th Annual Report. 221~p.
%\bibitem{6-k}
%\Au{Bartholomew P., Mote L., Whalen~G.}  The definition of systemic risk~// 
%Annual Meeting of the Western Economic Association.~--- 
%San Diego, CA, USA. 1995.


%\bibitem{8-k}
%\Au{Billio M., Getmansky M., Lo A., Pelizzon~L.} 
% Measuring systemic risk in the finance and insurance sectors. 
% MIT Sloan School, 2010. Working Paper~4774. 58~p.
%\bibitem{9-k}
%\Au{Bisias D., Flood M., Lo A., Valavanis~S.}  
%A~survey of systemic risk analytics. Office of Financial Research, 2012. Working Paper.
% 160~p.
%\bibitem{10-k}
%Board of Governors of the Federal Reserve System. 
%Policy statement on payment systems risk. 2001.  66~FR 30199. 7~p.



%\bibitem{12-k}
%\Au{Cont R., Moussa A., Santos~E.} 
% Network structure and systemic risk in banking systems. Preprint, 2010.
% 41~p. {\sf http://papers.ssrn.com/sol3/id=1733528}.
%\bibitem{13-k}
%\Au{Dardac N., Bogdan M.}  Assessing inter-bank contagion phenomenon using 
%quantitative methods. A case study on Romania. 
%2006. {\sf http://www.ase.ro/upcpr/profesori/756/Contagion\_2007.pdf}. 
%Accessed on April 25, 2017.



%\bibitem{16-k}
%\Au{Eisenberg~L., Noe T.\,H.}  Systemic risk in financial systems~//
%Manage. Sci., 2001. Vol.~47. No.\,2. P.~236--249.

%\bibitem{19-k}
%European Central Bank (ECB). Financial networks and financial stability~// 
%Financ. Stability Rev., 2010. P.~155--160.


%\bibitem{22-k}
%\Au{Jorda O., Schularick~M., Taylor~A.\,M.} 
% Betting the house.~--- Working paper ser.~28.~--- Federal Reserve Bank of San Francisco,
% 2014. 43~p.

%\bibitem{24-k}
%\Au{Mattig A., Morkoetter~S}. 
% Financial crises and adequacy of regulatory responses~// Systemic risk, Basel~III, 
% financial stability and regulation, 2011. 25~p.

%\bibitem{26-k}
%\Au{Mishkin F.}
%Systemic risk and
%the international lender of last resort.~//  10th Annual International Banking Conference, 2007. 
%Federal Reserve Bank of Chicago. Chicago, IL, USA. Speech. 
%{\sf https://www.federalreserve.gov/newsevents/speech/mishkin20070928a.htm}.

%\bibitem{28-k}
%\Au{Moussa A.} Contagion and systemic risk in financial networks.~---  
%Columbia University, 
%2011. D.Sc.\ Diss. {\sf http://hdl.handle.net/10022/AC:P:10249}.

%\bibitem{30-k}
%Fighting the global crisis.
%The International Monetary Fund and the Financial Stability Board, 2009. 
% Annual Report.
\end{thebibliography}
} }

\end{multicols}

 \label{end\stat}

 \vspace*{-6pt}

\hfill{\small\textit{Поступила в редакцию  14.11.2016}}
%\renewcommand{\bibname}{\protect\rm Литература}
\renewcommand{\figurename}{\protect\bf Рис.}
\renewcommand{\tablename}{\protect\bf Таблица}