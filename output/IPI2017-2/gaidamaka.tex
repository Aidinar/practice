\def\stat{gaidamaka}

\def\tit{МОДЕЛИРОВАНИЕ ОТНОШЕНИЯ СИГНАЛ/ИНТЕРФЕРЕНЦИЯ В~МОБИЛЬНОЙ СЕТИ 
СО~СЛУЧАЙНЫМ БЛУЖДАНИЕМ ВЗАИМОДЕЙСТВУЮЩИХ УСТРОЙСТВ$^*$}

\def\titkol{Моделирование отношения сигнал/интерференция в~мобильной сети 
со случайным блужданием %взаимодействующих 
устройств}

\def\aut{Ю.\,В.~Гайдамака$^1$, Ю.\,Н.~Орлов$^2$, Д.\,А.~Молчанов$^3$, 
А.\,К.~Самуйлов$^4$}

\def\autkol{Ю.\,В.~Гайдамака, Ю.\,Н.~Орлов, Д.\,А.~Молчанов, 
А.\,К.~Самуйлов}

\titel{\tit}{\aut}{\autkol}{\titkol}

\index{Ю.\,В.~Гайдамака, Ю.\,Н.~Орлов, Д.\,А.~Молчанов, 
А.\,К.~Самуйлов}


{\renewcommand{\thefootnote}{\fnsymbol{footnote}} \footnotetext[1]
{Публикация подготовлена при финансовой поддержке Минобрнауки 
России (№\,2.3397.2017).}}


\renewcommand{\thefootnote}{\arabic{footnote}}
\footnotetext[1]{Российский университет дружбы народов; Институт проблем информатики Федерального исследовательского 
центра <<Информатика и~управ\-ле\-ние>> Российской академии наук, 
\mbox{gaydamaka\_yuv@rudn.university}}
\footnotetext[2]{Институт прикладной математики им. М.\,В.~Келдыша Российской академии наук, \mbox{yuno@kiam.ru}}
\footnotetext[3]{Российский университет дружбы народов, molchanov\_da@rudn.university}
\footnotetext[4]{Российский университет дружбы народов, samuylov\_ak@rudn.university}

\vspace*{-12pt}

\Abst{Целью исследования является анализ отношения сиг\-нал/ин\-тер\-фе\-рен\-ция (SIR,
signal-to-interference 
ratio) 
при прямом взаимодействии устройств в~мобильных сетях 5-го поколения (5G) с~учетом 
перемещения прие\-мо\-пе\-ре\-да\-ющих устройств в~зоне обслуживания. Величина SIR на 
приемнике ас\-со\-ци\-иро\-ван\-ной пары исследуется как изменяющийся во времени случайный 
процесс, а~математическая модель движения задана кинетическим уравнением с~учетом 
скорости перемещения устройств, их пространственной плотности и~максимально 
допустимого радиуса взаимодействия. Задача численного анализа решается методом 
имитационного моделирования. В~качестве функционала величины SIR исследуется 
плотность распределения случайной величины (СВ) длительности периода наличия 
(отсутствия) связи. Показано, что вероятность обрыва связи логарифмически растет 
с~увеличением как ско\-рости перемещения взаимодействующих устройств, так и~их числа 
в~заданной зоне обслуживания.}
 
\KW{беспроводная сеть; отношение сигнал/интерференция; взаимодействие устройств; 
стохастическая геометрия; модель движения; кинетическое уравнение; показатель 
эффективности; вероятность обрыва связи; длительность периода наличия связи}

\vspace*{-6pt}

\DOI{10.14357/19922264170206} 


\vskip 10pt plus 9pt minus 6pt

\thispagestyle{headings}

\begin{multicols}{2}

\label{st\stat}


\section{Введение}

  Методы анализа SIR 
  как ключевого показателя, характеризующего качество предоставления 
услуг в~беспроводной сети~[1, 2], разрабатывались начиная примерно с~2005~г. 
Хотя большинство исследователей используют для анализа этой 
характеристики методы имитационного моделирования, известны также 
точные~[3--6] и~приближенные~[7--9] аналитические методы расчета SIR, 
использующие аппарат стохастической геометрии и~предназначенные для 
анализа характеристик сети в~случае неподвижных абонентов~[10, 11].
%
 Авторам 
не известны публикации, где проводится анализ SIR, меняющегося 
во времени, а~именно в~этих предположениях следует исследовать 
интерференцию при анализе показателей эффективности беспроводных сетей 
с~перемещающимися абонентами. Первые шаги в~этом направлении 
предприняты в~[12]. 

В~данной статье отношение SIR на приемнике 
ассоциированной пары исследуется как случайный процесс, состояние которого 
зависит от ско\-рости движения при\-емо\-пе\-ре\-да\-ющих устройств, плот\-ности 
размещения передатчиков в~зоне об\-служивания и~расстояния 
в~ассоциированной паре при\-ем\-ник--пе\-ре\-дат\-чик. Для случая однород-\linebreak ных 
абонентов, перемещающихся на плоскости, получены распределения 
длительности периода\linebreak наличия и~периода отсутствия связи между 
устройствами пары. В~исследуемой модели задано кинетическое уравнение 
относительно плотности распределения случайных приращений координат для 
описания движения абонента, плотность абонентов, а~также максимальный 
допустимый радиус взаимодействия для прямых соединений. 
%
На основе 
построенной в~разд.~2 модели прямых соединений в~разд.~3 разработан метод 
расчета отношения сигнал/интерференция, а~в~разд.~4 определены показатели 
эффективности модели. 
%
С~использованием траекторий SIR во времени 
в~разд.~5 численно исследованы плотности распределения длительностей 
периода наличия и~периода отсутствия связи.

\vspace*{-6pt}

\section{Системная модель}

  Рассматривается сценарий, где в~ограниченной зоне обслуживания~$V$, 
представляющей собою область в~$M$-мер\-ном пространстве, перемещаются 
носители идентичных прие\-мо\-пе\-ре\-да\-ющих устройств. Такой сценарий 
пригоден, например, для описания хаотичного перемещения зрителей на 
городской площади, на стадионе во время массовых мероприятий ($M\hm=2$) 
или для описания движения покупателей в~многоуровневом торговом центре 
($M\hm=3$). Для исследования интерференции на приемнике произвольной 
ассоциированной пары взаимодействующих устройств, созда\-ва\-емой 
передатчиками других пар, рассмотрено~$N$~пар устройств, 
взаимодействующих напрямую (D2D, device-to-device) на основе одной из 
чувствительных к~интерференции (interference-limited) технологий, например 
LTE, Release~13 или Wi-Fi Direct Connect~[13, 14]. Перемещение передатчиков 
сочетает целеполагающее поступательное движение\linebreak и~хаотическое блуждание
 и~определяется кинетическим уравнением типа Фок\-ке\-ра--План\-ка 
с~заданной ско\-ростью сноса~$u$ и~коэффициентом диф\-фузии~$\alpha$. 
Ассоциированный приемник движется\linebreak вмес\-те с~передатчиком внут\-ри 
окруж\-ности максимально допустимого радиуса взаимодействия~$d$ с~центром 
в~точке расположения передатчика. Мощ\-ность сигнала на приемнике 
определяется из стандартной модели распространения сигнала $\phi(r)\hm= 
Ar^{-\gamma}$, где $r$~--- расстояние в~паре при\-ем\-ник--пе\-ре\-дат\-чик; 
$A$~--- константа, учитывающая излучаемую мощность и~коэффициенты 
усиления приемной и~передающей антенны; $\gamma$~--- коэффициент 
распространения сигнала. 
  
  Для оценки интерференции на приемнике используется SIR, 
вычисляемое по формуле:
  \begin{equation}
  \mathrm{SIR} =\fr{\phi(r_0)}{\sum\nolimits_{n=1}^{N-1} \phi(d_n)}\,,
  \label{e1-gai}
  \end{equation}
где $r_0$~--- расстояние между приемником и~передатчиком в~исследуемой 
паре; $d_n$~--- расстояние между приемником из исследуемой пары 
и~передатчиком из $n$-й интерферирующей пары. 

Также задан порог SIR$^*$, 
определяющий минимальное значение SIR, необходимое для 
поддержания соединения. При падении SIR на приемнике ниже 
порогового значения SIR$^*$ связь в~паре прерывается до момента, когда 
SIR вновь превысит данный порог. 
Таким образом, задачей 
исследования является нахождение плотностей распределения длительностей 
периодов наличия и~отсутствия связи. 

\vspace*{-6pt}

\section{Метод расчета отношения сигнал/интерференция}

\vspace*{-2pt}

  Моделирование траекторий проводится для случая нестационарного 
блуждания на задаваемом временн$\acute{\mbox{о}}$м горизонте $t\hm\in [1,T]$ в~дискретном 
времени с~единичным шагом по времени, следуя результатам~[15--17]. На 
каждом шаге $k\hm=1,2,\ldots, T$ для каждого устройства генерируется 
приращение координат в~соответствии с~решением уравнения  
Фок\-ке\-ра--План\-ка, которое для случая моделирования одномерного 
блуждания ($M\hm=1$) имеет вид:
  \begin{multline}
  \fr{\partial f(x,t)}{\partial t} +\fr{\partial}{\partial x}\left( u(x,t)f(x,t)\right) -
\fr{\alpha(t)}{2}\,\fr{\partial^2 f(x,t)}{\partial x^2}={}\\
{}=0\,.
  \label{e2-gai}
  \end{multline}
Здесь $f(x,t)$~--- плотность распределения приращений координат~$x$ 
положения устройств в~момент времени~$t$, а~параметры уравнения~--- 
скорость сноса $u(x,t)$ и~нестационарный в~общем случае неотрицательный 
коэффициент диффузии $\alpha(t)$~--- в~данной работе были построены по 
моделям типичных нестационарных процессов, описывающих изменения 
положений случайно блуждающих объектов, обсуждаемых в~\cite{17-gai}. Для 
полученного ансамбля траекторий расстояние~$r_{ij}(k)$ между точками~$i$ 
с~координатами~$R^i(k)$ и~$j$ с~координатами~$R^j(k)$, которые находятся 
на разных траекториях в~некоторой об\-ласти~$V$ в~$M$-мер\-ном пространстве 
в~один и~тот же момент времени $t\hm=k$, вычисляются по формуле:
$$
r_{ij}^2(k)= \sum\limits_{m=1}^M \left( R^i_m(k)-R_m^j(k)\right)^2\,.
  $$
  
  Введем функцию от расстояния между двумя точками $\phi_{ij}\hm= 
\phi(r_{ij})$, соответствующую определенной выше стандартной модели 
распространения сигнала $\phi(r)\hm= Ar^{-\gamma}$. Для произвольной пары 
точек, например $i\hm=1$ и~$j\hm=2$, функционал~(1) SIR от 
расстояния~$r_{12}(k)$ во введенных обозначениях в~момент времени 
$t\hm=k$ определяется формулой:
  \begin{equation}
  S\left( r_{12}(k),N\right) =\fr{\phi_{12}(k)}{\sum\nolimits_{j=3}^N 
\phi_{1j}(k)}\,.
  \label{e3-gai}
  \end{equation}
  
  С точки зрения анализа устойчивости соединения основной задачей является 
вычисление суммы мощностей сигналов в~знаменателе формулы~(\ref{e3-gai}), 
поскольку мощность сигнала в~числителе ограничивается максимально 
допустимым радиусом взаимодействия~$d$.
  
  Изложим далее общий сценарий численного моделирования величины SIR на 
одном из~$N$~взаимодействующих в~области~$V$ устройств, например на 
приемнике, расположенном в~точке $i\hm=1$.
  
  Сумма в~знаменателе формулы~(\ref{e3-gai}) за вычетом 
величины~$\phi_{12}$ представляет собой умноженное на $N\hm-1$ среднее по 
ансамблю значение функции связи с~первой точкой, определяемое как 
  $$
  U\left( r_{12}(k),k\right) =\int\limits_V \phi\left( \left\vert r_{12}(k)-
r^\prime\right\vert \right) f\left( r^\prime, k\right)\,dr^\prime\,,
  $$ 
где $f(r^\prime, k)$ есть плотность распределения~$N$~точек в~об\-ласти~$V$ 
в~момент времени $t\hm=k$ в~соответствии с~решением кинетического 
уравнения~(\ref{e2-gai}).

  Положим $r(k)=r_{12}(k)$ и~рассмотрим все остальные точки в~системе 
отсчета, связанной со второй точкой. Тогда функционал~(\ref{e3-gai}) примет 
вид:
  \begin{multline*}
 \hspace*{-4.2027pt} S(k)\equiv S(k;u,\alpha) = \fr{\phi(r(k))}{(N-1)U(r(k),k)-\phi(r(k))}= {}\\
  {}=
\fr{\phi(r(k))}{NU(r(k),k)}+o\left( \fr{1}{N}\right)\,.
  \end{multline*}
  
  Отметим, что зависимость функции $U(r(t),t)$ от времени двоякая:  
во-пер\-вых, эта зависимость неявно определяется тем, что положение точек 
получено генерацией их из распределения, построенного как решение 
кинетического уравнения~(\ref{e2-gai}) в~момент времени $t\hm=k$;  
во-вто\-рых, эта зависимость определяется и~самой плот\-ностью функ\-ции
рас\-пре\-де\-ле\-ния $f(r,t)$. Учитывая это, 
будем для краткости писать далее
  \begin{equation}
  U\equiv U(r,t)=\int\limits_V \phi\left(\left\vert r-r^\prime\right\vert \right) 
f\left(r^\prime, t\right)\,dr^\prime\,.
  \label{e4-gai}
  \end{equation}
  
  Заметим теперь, что выбранные выше точки~1 и~2, связь между которыми 
изучается с~помощью функционала~(\ref{e4-gai}), движутся по тому же 
статистическому закону, что и~прочие точки системы, т.\,е.\ в~каж\-дый момент 
времени они случайно перемещаются в~пространстве в~соответствии с~тем, 
какая конкретная выборочная траектория реализована для моделирования 
движения каждой из них. Поскольку выбранная пара точек произвольна, то 
качество связи между двумя точками определяется средним по ансамблю 
значением~$q(t)$ функционала~$S(r(t),t)$:
  \begin{equation}
  q(t)=\fr{1}{N}\int\limits_V \fr{\phi(r)}{U(r,t)}\,f(r,t)\,dr\,.
  \label{e5-gai}
  \end{equation}
  
  Выведем уравнение эволюции среднего значения функционала SIR, т.\,е.\ 
величины~(\ref{e5-gai}). Из~(\ref{e4-gai}) имеем:
  \begin{multline}
  N\fr{dq}{dt}=\int\limits_V \fr{\phi(r)}{U(r,t)}\,\fr{\partial f(r,t)}
  {\partial t}\,dr- {}\\
  {}-
\int\limits_V \fr{\phi(r)}{U^2(r,t)}\,\fr{\partial U(r,t)} {\partial t}\,f(r,t)\,dr\,.
  \label{e6-gai}
  \end{multline}
  
  Считая, что на границе области функция распределения обращается в~ноль, 
для производной $\partial U/\partial t$ получаем после подстановки производной 
$\partial f /\partial t$ из уравнения~(\ref{e2-gai}), интегрирования по частям 
и~замены возникающей после этого действия производной функции $\phi(\vert 
r\hm- r^\prime\vert)$ по~$r^\prime$ на производную по~$r$ следующее 
выражение:
  $$
  \fr{\partial U}{\partial t} =\fr{\alpha}{2}\,\Delta U-\mathrm{div}\, \mathbf{J}\,,
  $$
  где
  $$
  \mathbf{J}=\int\limits_V\phi\left( \left\vert r-r^\prime\right\vert \right) u\left( 
r^\prime, t\right) f\left( r^\prime, t\right)\,dr^\prime\,.
  $$
  
  Далее первое слагаемое в~(\ref{e6-gai}) с~использованием~(\ref{e2-gai}) 
преобразуется к~виду:
  \begin{multline*}
  \int\limits_V \fr{\phi(r)}{U(r,t)}\,\fr{\partial f(r,t)}{\partial t}\,dr={}\\
  {}= -\int\limits_V 
\fr{\phi(r)}{U(r,t)}\,\mathrm{div}\, (uf)\,dr+\fr{\alpha}{2} \int\limits_V \fr{\phi(r)} 
{U(r,t)}\,\Delta f\,dr\,,
\end{multline*}
где $\Delta$ обозначен оператор Лапласа. После интегрирования по частям 
получаем:
$$
\int\limits_V\fr{\phi}{U}\,\fr{\partial f}{\partial t}\,dr=\int\limits_V f\left( u\nabla 
+\fr{\alpha}{2}\,\Delta\right) \left( \fr{\phi}{U}\right)\,dr\,.
$$

\begin{figure*}[b] %fig1
 \vspace*{-3pt}
\begin{center}
\mbox{%
\epsfxsize=129.591mm
\epsfbox{gay-1.eps}
}
\end{center}
\vspace*{-9pt}
 \Caption{Фрагмент временн$\acute{\mbox{ы}}$х рядов SIR для различной плотности устройств: \textit{1}~--- 
$N\hm=100$; \textit{2}~--- $N\hm=10$}
 \end{figure*}
  
  В результате уравнение~(\ref{e6-gai}) принимает вид:
  \begin{multline*}
  N\fr{dq}{dt} =\int\limits_V \left(\left( u\nabla +
  \fr{\alpha}{2}\,\Delta\right) \left( 
\fr{\phi}{U} \right) -{}\right.\\
\left.{}-\fr{\phi}{U^2}\left( \fr{\alpha}{2}\,\Delta U+ 
\mathrm{div}\,\mathbf{J}\right)\right) f(r,t)\,dr\,.
  \end{multline*}
    Это уравнение сложным нелинейным образом зависит от функции 
распределения точек, т.\,е.\ от плотности ансамбля их выборочных траек-\linebreak торий. 

Следовательно, актуальной задачей дальнейших исследований является 
численное моделирование статистик, связанных с~распределением 
величины~$S(r(t),t)$ зависимости SIR от 
расстояния до точки~2, и~величины~$q(t)$, которая представляет собой среднее 
значение функционала~$S(r(t),t)$ по ан\-самблю траекторий. 
  
\section{Показатели эффективности}

\vspace*{2pt}

  Для определения показателей эффективности передачи данных в~канале 
между взаимодей\-ст\-ву\-ющи\-ми устройствами ассоциированной пары в~зоне 
обслуживания на рис.~1 приведены примеры чис\-лен\-но\-го моделирования SIR 
для некоторой пары устройств по описанной методике. 
  

 
  Для оценки вероятности обрыва связи между взаимодействующими 
устройствами следует учитывать количество выбросов SIR ниже порога SIR$^*$ 
на длительном интервале наблюдения. Кроме вероятности обрыва связи 
интересными для анализа исследуемой системы являются характеристики СВ 
$\tau^-_l$~--- длительности периода отсутствия и~СВ $\tau_l^+$~--- 
длительности периода наличия связи между устройствами, которые показаны 
на рис.~1 вместе с~необходимыми для их определения моментами~$t_l^-$ 
и~$t_l^+$ пересечения графиком величины SIR порогового значения отношения 
SIR$^*$. В~первую очередь интерес представляет плотность распределения  
СВ~$\tau_l^-$ и~$\tau_l^+$ в~предположении об их независимости 
в~совокупности и~одинаковом распределении. 
  
  В качестве примеров могут быть рассмотрены два типа приложений, 
использующих прямые соединения при предостав\-ле\-нии услуг~--- приложения 
в~реальном времени и~приложения с~кэшированием данных. К~приложениям 
в~реальном времени относятся, например, набирающие популярность игровые 
приложения и~голосовые приложения. Для таких приложений критична 
возможность непрерывного поддержания соединения в~ассоциированной паре, 
поэтому ключевым показателем качества предостав\-ле\-ния услуги является 
вероятность того, что SIR упадет ниже заданного порога SIR$^*$ на 
определенном отрезке времени (вероятность выброса величины SIR вниз). 
В~этом случае соединение в~такой ассоциированной паре прерывается 
и~планировщиком базовой станции сети LTE или контролирующим узлом D2D 
соединения для передачи данных в~паре при\-ем\-ник--пе\-ре\-дат\-чик должна 
быть выбрана новая радиочастота. Таким образом, для приложений в~реальном 
времени длительность интервала до прерывания соединения совпадает 
с~длительностью периода наличия связи~$\tau_1^+$ до первого обрыва связи 
в~момент~$t_1^-$. 

  \begin{table*}[b]\small
  \begin{center}
 
  
  \begin{tabular}{|l|c|}
\multicolumn{2}{c}{Параметры системной модели}\\
\multicolumn{2}{c}{\ }\\[-6pt]
  \hline
\multicolumn{1}{|c|}{Параметр}&Значение\\
\hline
Область обслуживания в~($M=2$)-мер\-ном пространстве, $V$&$50\times50$ 
кв.\ м\\
Число пар <<приемник--передатчик>>, $N$&10, 30, 50, 100\\
Средняя скорость сноса, $v$&1, 3, 5, 10, 40~м/c\\
Параметр диффузии, $\alpha$&2\\
Константа распространения, $A$&1\\
Коэффициент распространения, $\gamma$&3\\
Максимальное расстояние от приемника до передатчика в~ассоциированной 
паре, $d$&5~м\\
Расстояние от приемника до передатчика в~ассоциированной паре, $r_0$&$0<r_0\leq 
d$\\
Диффузия приемника в~ассоциированной паре&1 м/с\\
Уровень отсутствия связи, SIR$^*$&0,01\\
\hline
\end{tabular}
\end{center}
\end{table*}

  
  В отличие от приложений в~реальном времени, для кэшируемых 
приложений, к~которым в~первую очередь относится передача видео, обрыв 
связи не всегда приводит к~прерыванию соединения. Если качество соединения 
в~ассоциированной паре позволяет передавать данные на скорости, 
превышающей требуемую скорость воспроизведения, то при кратковременном 
отсутствии связи буферизация обеспечит непрерывное воспроизведение видео. 
Таким образом, для кэшируемых приложений падение SIR ниже заданного 
порога, приводящее к~обрыву связи, не является критичным для 
предо\-став\-ле\-ния услуги. Для таких приложений важны длительности периодов 
наличия и~отсутствия связи, поскольку новая радиочастота для поддержания 
соединения в~паре при\-ем\-ник--пе\-ре\-дат\-чик должна быть назначена, 
только если временное отсутствие связи не может быть компенсировано\linebreak 
механизмом буферизации. Для кэшируемых приложений длительность 
интервала времени до прерывания соединения определяется моментом обрыва 
связи~$t^-_{L(\tau^*)}$, $L(\tau^*)\hm\geq 1$, соответст\-ву\-ющий 
интервал~$\tau^-_{L(\tau^*)}$ отсутствия связи после которого превысит 
порог~$\tau^*$~--- задержку, которая может быть сглажена с~помощью 
механизма буферизации. В~этом случае длительность интервала времени до 
прерывания соединения равна $t^-_{L(\tau^*)}\hm+\tau^*$ и~определяется 
формулой $\sum\nolimits_{l=1}^{L(\tau^*)} (\tau_l^+\hm+\min (\tau_l^-,\tau^*))$, 
где $L(\tau^*)\hm= \mathrm{inf}\, \left\{ l:\ \tau^-_{L(\tau^*)} \hm\geq 
\tau^*\right\}$.
  
  Таким образом, основными метриками, определяющими качество 
функционирования кэширу\-емых приложений, являются характеристики 
СВ~$\tau_l^+$ ($\tau_l^-$) длительности периода наличия (отсутствия) связи 
между взаимодействующими устройствами. Обозначим $F_{\tau^+}(x)\hm= 
P\left\{ \tau_l^+<x\right\}$ $(F_{\tau^-}(x)\hm= P\{ \tau_l^-<x\})$ и~
$f_{\tau^+}(x)\hm= F^\prime_{\tau^+}(x)$ $(f_{\tau^-}(x)\hm= F^\prime_{\tau^-
}(x))$ функции и~плотности распределения этих СВ соответственно.
  
  Задачу численного анализа в~следующем разделе статьи будем решать 
изложенным выше методом с~использованием имитационного моделирования. 
Анализ вероятностных характеристик будем проводить в~зависимости от 
средней скорости движения устройств $v\hm= \int\nolimits_V u(x,t)f(x,t)\,dx$, 
предполагая для простоты эту скорость постоянной, а~также в~зависимости от 
числа~$N$ пар при\-емо\-пе\-ре\-да\-ющих устройств.

\vspace*{-8pt}

\section{Численный анализ}

\vspace*{-2pt}

  В~качестве примера численного анализа рассматривается сценарий 
перемещения носителей устройств внутри торгового центра (см.\ таблицу).
  

  Графики плотностей распределения длитель\-ности периодов наличия 
$f_{\tau^+}(x)$ и~отсутствия~$f_{\tau^-}(x)$ связи между устройствами 
в~зависимости от средней скорости~$v$ передвижения устройств и~числа~$N$ 
пар взаимодействующих устройств приведены на рис.~2 и~3.
  
  Рисунки~2,\,\textit{а} и~3,\,\textit{а} иллюстрируют плотность 
$f_{\tau^+}(x)$ распределения длительности периода наличия связи при уровне 
порога $\mathrm{SIR}^*\hm=0{,}01$ для различных значений средней скорости и~числа пар устройств. Следует отметить, что качественное поведение графика 
плотности остается неизменным для различных значений скоростей 
и~количества пар. Кроме того, статистика свидетельствует о~характерных для 
показательного распределения длительностях периодов наличия связи.
  


  Аналогично периодам наличия связи, плотность $f_{\tau^-}(x)$ 
распределения периода отсутствия связи (см.\ рис.~2,\,\textit{б} и~3,\,\textit{б}) 
имеет ярко выраженный показательный характер. Эти наблюдения хорошо 
согласуются с~теоретическими результатами, представленными в~[18], где 
показано, что выбросы траекторий стационарного нормального случайного 
процесса имеют показательный характер.


  
  Представленные результаты численного анализа плотностей распределения 
длительности периодов наличия и~отсутствия связи позволяют сделать важные 
практические выводы относительно качества обслуживания прямых 
соединений в~сетях беспроводной связи. Известно, что показательное %\linebreak 
распределение характеризуется отсутствием памяти. Таким образом, если 
можно утверждать о~достаточ\-но большом значении SIR в~момент 
установления прямого соединения, то сделать обоснованный количественный 
вывод о~продолжительности периода наличия устойчивой связи не 
представляется возможным вследствие характера плотности распределения 
периода наличия устойчивой связи. Аналогичные выводы можно сделать 
и~о~периоде\linebreak отсутствия связи. Наконец, необходимо отметить, что указанное 
поведение характерно для выбран-\linebreak\vspace*{-12pt}

\pagebreak

\end{multicols}

\begin{figure*} %fig2
\vspace*{1pt}
\begin{center}
\mbox{%
\epsfxsize=161.511mm
\epsfbox{gay-2.eps}
}
\end{center}
\vspace*{-9pt}
\Caption{Плотность распределения длительности периода наличия~(\textit{а})
и~отсутствия~(\textit{б}) связи при 
различных скоростях~$v$ ($N\hm=10$):
\textit{1}~--- $v\hm=1$; \textit{2}~--- 3; \textit{3}~--- 5; \textit{4}~--- 10; \textit{5}~--- 
$v\hm=40$}
%\end{figure*}
%  \begin{figure*} %fig3
  \vspace*{6pt}
\begin{center}
\mbox{%
\epsfxsize=161.632mm
\epsfbox{gay-3.eps}
}
\end{center}
\vspace*{-9pt}
  \Caption{Плотность распределения длительности периода наличия~(\textit{a})
  и~отсутствия~(\textit{б}) связи
для различного числа пар~$N$ ($v\hm=5$~м/с):
\textit{1}~--- $N\hm=10$; \textit{2}~--- 30; \textit{3}~--- 50; \textit{4}~--- $N\hm=100$}
\end{figure*}

\begin{multicols}{2}

\noindent
ной модели движения и~может отличаться для 
других моделей, что является предметом дальнейших исследований. 

\vspace*{-6pt}

\section{Заключение}

\vspace*{-2pt}

  В представленной работе построена модель для анализа характеристик 
беспроводных соединений с~технологией прямого взаимодействия устройств, 
которая является неотъемлемой частью сетей связи~5G. 

С~использованием аппарата кинетических уравнений движения получены 
характеристики, определяющие качество обслуживания для приложений 
в~реальном времени и~кэшируемых приложений. В~частности, исследованы 
плотности распределения длительности периодов наличия и~отсутствия связи 
как функции от скорости перемещения абонентов и~числа пар 
взаимодействующих устройств.
  
  Результаты численного анализа, представленные в~работе, показывают, что 
вероятность обрыва связи экспоненциально уменьшается при увеличении как 
числа пар взаимодействующих устройств, так и~скорости перемещения 
абонентов. Важным результатом для длительности периодов наличия 
и~отсутствия связи является показательный характер их распределений.
  
  Проведенный численный анализ позволяет сделать вывод о том, что 
поддержка приложений в~реальном времени в~сетях D2D возможна только в~тех 
областях, где средняя скорость устройств не превышает нескольких метров 
в~секунду, например внутри зданий, в~пешеходных зонах на открытой 
местности. Кроме того, при принятии решения об установлении нового 
прямого соединения должно учитываться количество уже установленных 
соединений. Выбор оптимального значения~$N$ для некоторого порога 
SIR$^*$ может быть сделан на основе предложенной в~статье методологии. Для 
кэшируемых приложений при дальнейших исследованиях интервал времени до 
прерывания соединения может быть представлен как сумма случайных 
величин, где число слагаемых имеет геометрическое распределение.
  
  В дальнейшем интересно провести качественный и~количественный анализ 
ве\-ро\-ят\-ност\-но-вре\-мен\-н$\acute{\mbox{ы}}$х характеристик периодов наличия 
и~отсутствия связи при различных типах случайного блужда\-ния абонентов. 
Отдельным направлением может также стать расширение сценария на случай 
наличия базовых станций. Кроме того, комментарий по дальнейшим 
исследованиям содержится в~заключении разд.~3 статьи.
  
  \smallskip
  
  В заключение авторы благодарят проф.\ К.\,Е.~Самуйлова за обсуждение 
постановки задачи исследований. 

{\small\frenchspacing
 {%\baselineskip=10.8pt
 \addcontentsline{toc}{section}{References}
 \begin{thebibliography}{99}
\bibitem{1-gai}
\Au{Rong~Z., Rappaport T.\,S.} Wireless communications: Principles and practice.~---1st ed.~---
 Upper Saddle River, NJ, USA: Prentice Hall, 1996. 641~p.
\bibitem{2-gai}
\Au{Andrews J.\,G., Claussen~H., Dohler~M., Rangan~S., Reed~M.\,C.} Femtocells: Past, present, 
and future~// IEEE J.~Sel. Area. Comm., 2012. Vol.~30. Iss.~3. P.~497--508. doi: 
10.1109/JSAC.2012.120401.

\bibitem{4-gai} %3
\Au{Samuylov A., Gaidamaka~Yu., Moltchanov~D., Andreev~S., Koucheryavy~Y.} Random 
triangle: A~baseline model for interference analysis in heterogeneous networks~// IEEE 
Trans. Veh. Technol., 2015. Vol.~65. Iss.~8. P.~6778--6782. doi: 
10.1109/TVT.2016.2596324.
\bibitem{5-gai} %4
\Au{Гайдамака Ю.\,В., Самуйлов~А.\,К.} Метод расчета характеристик интерференции двух 
взаимодей\-ст\-ву\-ющих устройств в~беспроводной гетерогенной сети~// Информатика и~её 
применения, 2015. Т.~9. Вып.~1. С.~9--14. doi:10.14357/19922264150102.

\bibitem{6-gai} %5
\Au{Гайдамака Ю.\,В., Андреев~С.\,Д., Сопин~Э.\,С., Самуйлов~К.\,Е., Шоргин~С.\,Я.} 
Анализ характеристик интерференции в~модели взаимодействия устройств с~учетом среды 
распространения сигнала~// Информатика и~её применения, 2016. Т.~10. Вып.~4. С.~2--10.
doi: 10.14357/19922264160401.

\bibitem{3-gai} %6
\Au{Samuylov A., Ometov~A., Begishev~V., Kovalchukov~R., Moltchanov~D., Gaidamaka~Yu., 
Samouylov~K., Andreev~S., Koucheryavy~Y.} Analytical performance estimation of 
network-assisted D2D communications in urban scenarios with rectangular cells~// Trans.
 Emerg. 
Telecommun. Technol., 2017. Vol.~28. No.\,2. P.~2999-1--2999-15. 
doi: 10.1002/ett.2999.
 (Version of record 
online: November~12,  2015.)

\bibitem{8-gai} %7
\Au{Gong Z., Haenggi~M.} Interference and outage in mobile random networks: Expectation, 
distribution, and correlation~// IEEE Trans. Mobile Comput., 2014. Vol.~13.  
P.~337--349. doi: 10.1109/TMC.2012.253.

\bibitem{7-gai} %8
\Au{Etezov Sh.\,A., Gaidamaka~Yu.\,V., Samouylov~K.\,E., Moltchanov~D.\,A., Samuylov~A.\,K., 
Andreev~S.\,D., Koucheryavy~E.\,A.} On distribution of SIR in dense D2D deployments~// 22nd 
European Wireless Conference Proceedings.~--- VDE, 2016. 
P.~333--337.

\bibitem{9-gai}
\Au{Petrov~V., Komarov~M., Moltchanov~D., Jornet~J.\,M., 
Koucheryavy~Y.}  Interference and SINR in millimeter
wave and terahertz communication systems with 
blocking and directional antennas~//
{IEEE Trans. Wireless Commun.}, 2017.
Vol.~16. P.~1791--1808. %ISSN 1536-1276.
%%2016 IEEE Global Communications Conference 
%(GLOBECOM), 2016. 7~p. doi: 10.1109/\mbox{GLOCOM}.2016.7841910.
\bibitem{10-gai}
\Au{Baccelli F., Blaszczyszyn~B.} Stochastic geometry and wireless networks~// Found. 
Trends Netw., 2010. Vol.~3. No.\,3-4. P.~249--449. 
doi:10.1561/1300000006; Vol.~4. No.\,1-2. P.~1--312. doi:10.1561/1300000026.
\bibitem{11-gai}
\Au{Haenggi M.} Stochastic geometry for wireless networks.~--- Cambridge: Cambridge University 
Press, 2012. 298~p.
\bibitem{12-gai}
\Au{Orlov Yu.\,N., Fedorov~S.\,L., Samuylov~A.\,K., Gaidamaka~Yu.\,V., Molchanov~D.\,A.} 
Simulation of devices mobility to estimate wireless channel quality metrics in 5G networks~// AIP 
Conference Proceedings: 12th Conference (International) of Numerical Analysis and Applied 
Mathematics.~--- New York, NY, USA: AIP Publishing, 
2017 (in press).
\bibitem{13-gai}
3GPP LTE Release 10 \& beyond (LTE-Advanced).~--- December 2009. {\sf 
ftp://www.3gpp.org/workshop/2009-12-17\_ITU-R\_IMT-Adv\_eval/docs/pdf/REV-090006.pdf}.
\bibitem{14-gai}
Wi-Fi Peer-to-Peer (P2P) Technical Specification v1.7.~--- Wi-Fi Alliance, 2010. {\sf 
https://www.wi-fi.\linebreak org/downloads-registered-guest/Wi-Fi\_P2P\_Technical\_\linebreak Specification\_v1.7.pdf/29559}.
\bibitem{15-gai}
\Au{Босов А.\,Д., Кальметьев~Р.\,Ш., Орлов~Ю.\,Н.} Моделирование нестационарного 
временного ряда с~заданными свойствами выборочного распределения~// Мат. 
моделирование, 2014. Т.~26. №\,3. С.~97--107.
\bibitem{16-gai}
\Au{Орлов Ю.\,Н., Федоров~С.\,Л.} Генерация нестационарных траекторий временного ряда 
на основе уравнения Фок\-ке\-ра--План\-ка~// Труды МФТИ, 2016. Т.~8. №\,2. С.~126--133.
\bibitem{17-gai}
\Au{Орлов Ю.\,Н., Федоров~С.\,Л.} Методы численного моделирования процессов 
нестационарного случайного блуждания.~--- М.: МФТИ, 2016. 112~с.
\bibitem{18-gai}
\Au{Тихонов В.\,И., Хименко~В.\,И.} Выбросы траекторий случайных процессов.~--- М.: 
Наука, 1987. 304~с.

 \end{thebibliography}

 }
 }

\end{multicols}

\vspace*{-3pt}

\hfill{\small\textit{Поступила в~редакцию 15.04.17}}

%\vspace*{8pt}

\newpage

\vspace*{-24pt}

%\hrule

%\vspace*{2pt}

%\hrule

%\vspace*{8pt}


\def\tit{MODELING THE SIGNAL-TO-INTERFERENCE RATIO\\ IN A~MOBILE NETWORK WITH MOVING 
DEVICES}

\def\titkol{Modeling the signal-to-interference ratio in a~mobile network with moving 
devices}

\def\aut{Yu.\,V.~Gaidamaka$^{1,2}$, Yu.\,N.~Orlov$^3$, D.\,A.~Molchanov$^1$,  
and~A.\,K.~Samuylov$^1$}

\def\autkol{Yu.\,V.~Gaidamaka, Yu.\,N.~Orlov, D.\,A.~Molchanov,  
and~A.\,K.~Samuylov}

\titel{\tit}{\aut}{\autkol}{\titkol}

\vspace*{-9pt}


\noindent
$^1$Peoples' Friendship University of Russia (RUDN University),  
6~Miklukho-Maklaya Str., Moscow 117198, Russian\linebreak
$\hphantom{^1}$Federation

\noindent
$^2$Institute of Informatics Problems, Federal Research Center ``Computer Science 
and Control'' of the Russian\linebreak
$\hphantom{^1}$Academy of Sciences, 44-2~Vavilov Str., Moscow 
119333, Russian Federation

\noindent
$^3$Keldysh Institute of Applied Mathematics of the Russian Academy of Sciences, 
4~Miusskaya Sq., Moscow 125047,\linebreak
$\hphantom{^1}$Russian Federation



\def\leftfootline{\small{\textbf{\thepage}
\hfill INFORMATIKA I EE PRIMENENIYA~--- INFORMATICS AND
APPLICATIONS\ \ \ 2017\ \ \ volume~11\ \ \ issue\ 2}
}%
 \def\rightfootline{\small{INFORMATIKA I EE PRIMENENIYA~---
INFORMATICS AND APPLICATIONS\ \ \ 2017\ \ \ volume~11\ \ \ issue\ 2
\hfill \textbf{\thepage}}}

\vspace*{3pt}



\Abste{The goal of the study is to analyze the signal-to-interference ratio (SIR) for device-to-device 
interaction of devices communication in the 5th generation mobile networks, taking into account 
the movement of the receiving and transmitting devices in the service area. The SIR value at the 
receiver of the associated pair is studied as a~time-varying random process, and the mathematical 
model of motion is given by a kinetic equation taking into account the given average speed of the 
devices, their spatial density, and the maximum allowable communication radius. The measures of 
performance quality were studied by numerical analysis using SIR simulation of a~key channel. The 
measures are the following: the signal interruption probability for the receiver--transmitter pair, the 
probability density function of the random variables for the duration of the availability period, and 
the period of absence of communication. It is shown that the signal interruption probability grows 
logarithmically as either the average device speed or the number of devices in the service area 
increases.}

\KWE{wireless network; signal-to-interference ratio; device-to-device; stochastic geometry; motion 
model; kinetic equation; performance measure; signal interruption probability}

\DOI{10.14357/19922264170206} 


\vspace*{-10pt}

\Ack
\noindent
The publication was supported by the Ministry of Education and Science of the Russian Federation (project 
No.\,2.3397.2017).



%\vspace*{3pt}

  \begin{multicols}{2}

\renewcommand{\bibname}{\protect\rmfamily References}
%\renewcommand{\bibname}{\large\protect\rm References}

{\small\frenchspacing
 {%\baselineskip=10.8pt
 \addcontentsline{toc}{section}{References}
 \begin{thebibliography}{99}
\bibitem{1-gai-1}
\Aue{Rong,~Z., and T.\,S.~Rappaport}. 1996. 
\textit{Wireless communications: Principles and 
practice}. 1st ed. Upper Saddle River, NJ: Prentice Hall. 641~p.
\bibitem{2-gai-1}
\Aue{Andrews, J.\,G., H.~Claussen, M.~Dohler, S.~Rangan, and M.\,C.~Reed.} 2012. Femtocells: 
Past, present, and future. \textit{IEEE J.~Sel. Area. Comm.} 30(3):497--508. 
doi: 10.1109/JSAC.2012.120401.

\bibitem{4-gai-1} %3
\Aue{Samuylov, A., Yu.~Gaidamaka, D.~Moltchanov, S.~Andreev, and Y.~Koucheryavy}. 2015. 
Random triangle: A~baseline model for interference analysis in heterogeneous networks. 
\textit{IEEE Trans. Veh. Technol.} 65(8):6778--6782. doi: 
10.1109/TVT.2016.2596324.
\bibitem{5-gai-1} %4
\Aue{Gaydamaka, Yu.\,V., and A.\,K.~Samuylov.} 2009. Metod rascheta kharakteristik 
interferentsii dvukh vzaimo\-dey\-st\-vu\-yushchikh ustroystv v~besprovodnoy geterogennoy seti [The 
method of calculation of the characteristics of the interference of two interacting devices in 
a~wireless heterogeneous network]. \textit{Informatika i~ee Primeneniya~--- Inform. Appl.}  
9(1):9--14. doi:10.14357/19922264150102.
\bibitem{6-gai-1} %5
\Aue{Gaydamaka, Yu.\,V., S.\,D.~Andreev, E.\,S.~Sopin, K.\,E.~Samouylov, and S.\,Ya.~Shorgin}. 
2016. Analiz kharakteristik interferentsii v~modeli vzaimodeystviya ustroystv s~uchetom sredy 
rasprostraneniya signala [Analysis of the characteristics of the interference in the model of 
interaction between devices taking into account the signal propagation environment]. 
\textit{Informatika i~ee Primeneniya~--- Inform. Appl.} 10(4):2--10. doi: 
10.14357/19922264160401.

\bibitem{3-gai-1} %6
\Aue{Samuylov, A., A.~Ometov, V.~Begishev, R.~Kovalchukov, D.~Moltchanov, 
Yu.~Gaidamaka, K.~Samouylov, S.~Andreev, and Y.~Koucheryavy}. 2015. Analytical 
performance estimation of network-assisted D2D communications in urban scenarios with 
rectangular cells. \textit{Trans. Emerg. Telecommun. Technol.} 28(2):2999-1--2999-15. 
doi: 10.1002/ett.2999.  (Version of record 
online: November~12,  2015.)

\bibitem{8-gai-1} %7
\Aue{Gong, Z., and M.~Haenggi}. 2014. Interference and outage in mobile random networks: 
Expectation, distribution, and correlation. \textit{IEEE Trans. Mobile Comput.}  
13:337--349. doi: 10.1109/TMC.2012.253.

\bibitem{7-gai-1} %8
\Aue{Etezov, Sh., Yu.~Gaidamaka, K.~Samouylov, D.~Mol\-tcha\-nov, A.~Samuylov, S.~Andreev, 
and E.~Koucheryavy.} 2016. On distribution of SIR in dense D2D deployments. \textit{22nd 
European Wireless Conference Proceedings.} VDE.
333--337.

\bibitem{9-gai-1}
\Aue{Petrov, V., M.~Komarov, D.~Moltchanov, J.\,M.~Jornet, 
and Y.~Koucheryavy.} 2017. Interference and SINR in millimeter
wave and terahertz communication systems with 
blocking and directional antennas. \textit{IEEE Trans. Wireless Commun.}
16:1791--1808. %ISSN 1536-1276.
%doi: 10.1109/\mbox{GLOCOM}.2016.7841910.
\bibitem{10-gai-1}
\Aue{Baccelli, F., and B.~Blaszczyszyn.} 2010. Stochastic geometry and wireless networks. 
\textit{Found. Trends Netw.} 3(3-4):249--449. 
doi:10.1561/1300000006;  4(1-2):1--312. 
doi:10.1561/1300000026.
\bibitem{11-gai-1}
\Aue{Haenggi, M.} 2012. \textit{Stochastic geometry for wireless networks}. Cambridge: 
Cambridge University Press. 298~p.
\bibitem{12-gai-1}
\Aue{Orlov, Yu.\,N., S.\,L.~Fedorov, A.\,K.~Samuylov, Yu.\,V.~Gai\-da\-ma\-ka, and 
D.\,A.~Molchanov}. 2017 (in press). Simulation of devices mobility to estimate wireless channel quality 
metrics in 5G networks. \textit{AIP Conference Proceedings: 12th Conference (International) of 
Numerical Analysis and Applied Mathematics}. New York, NY: 
AIP Publishing. 
\bibitem{13-gai-1}
3GPP LTE Release 10 \& beyond (LTE-Advanced).  December 2009. Available at: {\sf 
ftp://www.3gpp.org/\linebreak workshop/2009-12-17\_ITU-R\_IMT-Adv\_eval/docs/pdf/\linebreak REV-090006.pdf} (accessed 
April~20, 2017).
\bibitem{14-gai-1}
Wi-Fi Peer-to-Peer (P2P) Technical Specification v1.7.   Wi-Fi Alliance, 2010. Available at: {\sf 
https://www.wi-fi.\linebreak org/downloads-registered-guest/Wi-Fi\_P2P\_Technical\_\linebreak Specification\_v1.7.pdf/29559} 
(accessed April~20, 2017).
\bibitem{15-gai-1}
\Aue{Bosov, A.\,D., R.\,S.~Kalmetiev, and Yu.\,N.~Orlov.} 2014. Modelirovanie 
nestatsionarnogo vremennogo ryada s~zadannymi svoystvami vyborochnogo raspredeleniya 
[Sample distribution function construction for non-stationary
time-series forecasting]. 
\textit{Matematicheskoe mo\-de\-li\-ro\-va\-nie} [Mathematical Simulation] 26(3):97--107.
\bibitem{16-gai-1}
\Aue{Orlov, Yu.\,N., and S.\,L.~Fedorov.} 2016. Generatsiya ne\-sta\-tsi\-o\-nar\-nykh traektoriy 
vremennogo ryada na osnove uravneniya Fokkera--Planka [Generation of nonstationary 
trajectories of the time series based on the Fokker--Planck equation]. \textit{Trudy MFTI} 
[Proceedings of MIPT] 8(2):126--133.
\bibitem{17-gai-1}
\Aue{Orlov, Yu.\,N., and S.\,L.~Fedorov.} 2016. Metody chislennogo modelirovaniya protsessov 
nestatsionarnogo sluchaynogo bluzhdaniya [Methods of a numerical simulation of nonstationary 
random walk]. Moscow: MFTI. 112~p.
\bibitem{18-gai-1}
\Aue{Tikhonov, V.\,I., and V.\,I.~Khimenko}. 1987. \textit{Vybrosy traektoriy sluchaynykh 
protsessov} [Emissions of trajectories of random processes]. Moscow: Nauka. 304~p.

\end{thebibliography}

 }
 }

\end{multicols}

\vspace*{-3pt}

\hfill{\small\textit{Received April 15, 2017}}

\Contr

\noindent
\textbf{Gaidamaka Yuliya V.} (b.\ 1971)~--- Candidate of Science (PhD) in physics 
and mathematics, associate professor, Peoples' Friendship University of Russia 
(RUDN University), 6~Miklukho-Maklaya Str., Moscow 117198, Russian 
Federation; senior scientist, Institute of Informatics Problems, Federal Research 
Center ``Computer Science and Control'' of the Russian Academy of Sciences,  
44-2~Vavilov Str., Moscow 119333, Russian Federation; 
\mbox{gaydamaka\_yuv@rudn.university}

\vspace*{3pt}

\noindent
\textbf{Orlov Yurii N.} (b.\ 1964)~--- Doctor of Science in physics and 
mathematics, professor, head of sector, Keldysh Institute of Applied Mathematics of 
the Russian Academy of Sciences, 4~Miusskaya Sq., Moscow 125047, Russian 
Federation; \mbox{yuno@kiam.ru}

\vspace*{3pt}

\noindent
\textbf{Molchanov Dmitri A.} (b.\ 1978)~---  Candidate of Science (PhD) in 
technology, associate professor, Peoples' Friendship University of Russia (RUDN 
University), 6~Miklukho-Maklaya Str., Moscow 117198, Russian Federation; 
\mbox{molchanov\_da@rudn.university}

\vspace*{3pt}

\noindent
\textbf{Samuylov Andrey K.} (b.\ 1988)~--- Candidate of Science (PhD) in 
physics and mathematics, associate professor, Peoples' Friendship University of 
Russia (RUDN University), 6~Miklukho-Maklaya Str., Moscow 117198, Russian 
Federation; \mbox{samuylov\_ak@rudn.university}
\label{end\stat}


\renewcommand{\bibname}{\protect\rm Литература} 