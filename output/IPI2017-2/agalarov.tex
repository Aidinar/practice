\def\stat{agalarov}

\def\tit{МАКСИМИЗАЦИЯ СРЕДНЕГО СТАЦИОНАРНОГО ДОХОДА СИСТЕМЫ
МАССОВОГО ОБСЛУЖИВАНИЯ ТИПА 
$M/G/1$$^*$}

\def\titkol{Максимизация среднего стационарного дохода системы
массового обслуживания типа 
$M/G/1$}

\def\aut{Я.\,М.~Агаларов$^1$}

\def\autkol{Я.\,М.~Агаларов}

\titel{\tit}{\aut}{\autkol}{\titkol}

\index{Агаларов Я.\,М.}
\index{Agalarov Ya.\,M.}


{\renewcommand{\thefootnote}{\fnsymbol{footnote}} \footnotetext[1]
{Работа выполнена при частичной финансовой поддержке РФФИ (проект 15-07-03406).}}


\renewcommand{\thefootnote}{\arabic{footnote}}
\footnotetext[1]{Институт проблем информатики Федерального исследовательского центра <<Информатика и~управление>> Российской академии наук,
\mbox{agglar@yandex.ru}}

%\vspace*{-18pt}

    
  \Abst{Рассматривается задача оптимизации порогового значения длины очереди системы 
$M$/$G$/1 в~смысле максимизации предельного дохода, получаемого системой в~единицу 
времени. Доход определяется платой, получаемой за обслуживание заявок, затратами на 
техническое обслуживание прибора и~штрафами за задержку заявок в~очереди, за отказ 
заявке в~обслуживании, за простой системы. Сформулированы достаточные условия 
существования конечного порогового значения для системы $M/G/1$, доказаны утверждения 
о~необходимых и~достаточных условиях оптимальности порогового значения 
и~о~существовании конечного оптимального порога. Предложен алгоритм расчета 
оптимального порогового значения и~соответствующего значения максимального дохода. 
Приведены результаты вычислительных экспериментов, иллюстрирующие работу 
предложенного алгоритма.}
  
  \KW{система массового обслуживания; пороговое управление; оптимизация}
  
\DOI{10.14357/19922264170203} 


\vskip 10pt plus 9pt minus 6pt

\thispagestyle{headings}

\begin{multicols}{2}

\label{st\stat}

  
\section{Введение}

  В данной работе проводится дальнейшее ис\-следование проблемы 
оптимального управления очередью системы $M$/$G$/1, сформулированной 
в~рабо\-те~[1] в~виде задачи максимизации предельного дохо\-да 
системы$M$/$G$/1 в~единицу времени на множест\-ве пороговых стратегий. 
Доход, как и~в~работе~[1], определяется платой, получаемой за обслуживание 
заявок, затратами на техническое обслуживание прибора и~штрафами за 
задержку \mbox{заявок} в~очереди, за отказ заявке в~обслуживании, за прос\-той 
системы. Аналогичная постановка задачи для одноканальных СМО типа 
$M$/$G$/1 и~$G$/$M$/1 рас\-смот\-ре\-на автором данной статьи также 
в~работах~[2, 3]. Из других работ отечественных авторов близки по постановке 
задачи работы~[4, 5], в~которых данная задача сформулирована в~терминах 
теории управляемых цепей Маркова для сис\-те\-мы
массового обслуживания (СМО) типа $M$/$G$/1 
и~CBSMAP/$M$/$n$/$r$ и~предлагается численный метод ее решения. 
  
  В работе~[1] получены оценка снизу для оптимального 
  порогового значения и~алгоритм ее вы\-чис\-ле\-ния методом последовательного 
  приближения. 
В~качестве оценки предлагается решение\linebreak вспомогатель\-ной оптимизационной 
задачи, сформулированной как задача поиска порогового зна-\linebreak чения длины 
очереди (оптимальной стратегии управ\-ле\-ния очередью) системы $M$/$G$/1, 
максимизирующего предельный доход системы, усредненный по числу 
обслуженных заявок. В~работе~[1] сформулированы необходимые 
и~достаточные условия оптимальности порогового значения для 
вспомогательной задачи и~предложен алгоритм гарантированного поиска 
решения при условии выполнения достаточных условий. В~данной работе 
аналогичные результаты получены для задачи максимизации предельного 
дохода системы $M$/$G$/1 в~единицу времени и~доказано, что для этой 
системы достаточные условия существования оптимального порога 
выполняются всегда. Предложен алгоритм последовательного приближения, 
который для гарантированного поиска оптимального порога требует объема 
вычислений порядка двоичного логарифма значения этого порога. 

\vspace*{-6pt}
  
\section{Постановка задачи}

\vspace*{-2pt}

  Рассматривается СМО типа $M$/$G$/1 с~накопителем бесконечной емкости и~одним прибором обслуживания, на которую поступает пуассоновский поток 
заявок с~интенсивностью $\lambda\hm>0$, а время обслуживания каждой заявки 
распределено по произвольному закону~$H(t)$. Поступившая заявка 
допускается в~накопитель системы (занимает любое свободное место 
в~накопителе), если в~момент ее поступления число занятых мест в~накопителе 
меньше~$k$, где $k\hm>0$~---  некоторое заданное значение (тривиальный 
случай $k\hm=0$ не рассматривается), а~в~противном случае она отклоняется 
(не допус-\linebreak\vspace*{-12pt}

\pagebreak

\noindent
кается в~систему). В~дальнейшем обозначение~$k$ будем называть 
порогом, а его значение пороговым значением. Если заявка допущена 
в~накопитель, она занимает любое свободное место в~накопителе 
и~обслуживается на приборе в~порядке поступления. Заявка покидает систему 
только при завершении обслуживания, освободив одновременно прибор 
и~накопитель, а на освободившийся прибор поступает очередная заявка из 
накопителя (если таковая есть). Система получает доход, который определяется 
следующими составляющими:
  \begin{description}  
\item[\,]   $C_0 v\geq 0$~--- плата, получаемая системой, если поступившая заявка 
обслужена системой, где $v$~--- время занятия прибора заявкой; 
\item[\,]  $C_1\geq 0$~--- величина штрафа, если поступившая заявка отклонена;
\item[\,]   
  $C_2\geq 0$~--- величина штрафа за единицу времени ожидания заявки в~
очереди к~прибору;
  \item[\,] 
  $C_3\geq 0$~--- величина штрафа за единицу времени простоя прибора;
  \item[\,] 
  $C_4\geq 0$~--- затраты системы в~единицу времени на техническое 
обслуживание системы. 
  \end{description}
  
  Будем считать, что плату за обслуживание сис\-те\-ма получает в~момент 
завершения обслуживания каждой заявки в~зависимости от длины заявки 
(длительности занятия прибора).
  
  Отметим, что процесс обслуживания заявок в~данной системе описывается 
цепью Mаркова, где переходы цепи определяются моментами окончания 
обслуживания и~состояние системы есть число заявок, остающихся в~ней 
в~момент ухода с~прибора обслуженной заявки (см., например,~[6,~7]). 
Отметим также, что при пороге~$k$ указанная цепь Маркова имеет один 
положительный возвратный класс состояний $i\hm=0,\ldots, k-1$.
  
  Введем обозначения:
  \begin{description}
  \item[\,] $\pi_i^k$, $0\leq i\hm\leq k\hm-1$,~--- стационарное распределение 
вероятностей цепи при пороге~$k$ ($\pi_i^k$~--- вероятность того, что цепь 
находится в~состоянии~$i$);
  \item[\,] $g^k$~--- значение суммарного предельного дохода, усредненного по 
числу обслуженных заявок;
\item[\,] $q_i^k$~--- средний доход, получаемый системой в~состоянии~$i$ при 
пороге~$k$, $i\hm\geq 0$;
\item[\,] $\overline{v}=\int\nolimits_0^\infty t\,dH(t)$~--- среднее время 
пребывания сис\-те\-мы в~состоянии~$i$, $0\hm <\overline{v}\hm<\infty$;
\item[\,] $\rho=\lambda\overline{v}$~--- входная нагрузка.
\end{description}
  
  Значение предельного дохода, усредненного по числу обслуженных заявок, 
при пороге~$k$ равно~[1]:
  \begin{equation}
  g^k=\sum\limits_{i=0}^{k-1} \pi^k_i q_i^k\,.
  \label{e1-ag}
  \end{equation}
  
  Среднее значение предельного дохода системы в~единицу времени при 
пороге~$k$ составляет~[1]: 
  \begin{equation}
  Q^k = \lambda\left(1-\theta_k^k\right) g^k\,,
  \label{e2-ag}
  \end{equation}
где $\theta_k^k$~--- вероятность того, что поступившая заявка не
будет допущена в~систему. 

  Ставится задача: найти оптимальное пороговое значение $k^*\hm>0$, такое 
что 
  \begin{equation}
  \max\limits_{k>0} Q^k=Q^{k^*}\,.
  \label{e3-ag}
  \end{equation}
  
  Ниже всюду будем считать, что $k^0\hm>0$ и~$k^*\hm>0$~--- оптимальные 
пороговые значения в~случае функций~(1) и~(2) соответственно.
  
\section{Метод решения}

  Приводимый ниже метод решения подробно описан в~работе~[1] 
применительно к~задаче максимизации функции~$g^k$. Для наглядности 
изложения приведем кратко основные формулы и~утверждения из указанной 
работы, на которые ниже в~тексте сделаны ссылки. 
  
  Для стационарного распределения вероятностей состояний системы при 
пороге~$k$ получены рекуррентные формулы (см.~(4) и~(6) в~[1]):
  \begin{equation}
  \pi_j^k=\pi_0^k R_j\,,\enskip j=0,\ldots, k-1\,,
  \label{e4-ag}
  \end{equation}
где
\begin{equation*}
\pi_0^k= \left( \sum\limits_{i=0}^{k-1} R_i\right)^{-1}\,.
\end{equation*}
Здесь
\begin{gather*}
R_0=1\,;\quad
R_1=\fr{1-r_0}{r_0}\,;\\
R_{i+1}= \fr{1}{r_0}\!\left(\! R_i-r_i-\sum\limits_{j=1}^i\! R_j r_{i-j+1}\!\right),
\\  
\hspace*{40mm}i=1,\ldots, k-2\,,
   \end{gather*}
   где
   $$
r_l= \int\limits_0^\infty \fr{(\lambda v)^l}{l!}\,e^{-\lambda v}\,dH(v)\ \mbox{при } 
l\geq 0\,.
$$
  
  Для величины среднего дохода системы, получаемого за время пребывания 
в~состоянии $0\hm\leq i\hm\leq k\hm-1$, верна формула (см.~(10) в~[1]):

\noindent
  \begin{multline}
  q_i^k= C_0\overline{v}- C_1\sum\limits^\infty_{m=k-i+1} \left(m-k+i\right) r_m-{}\\
  {}-\fr{C_2}{\lambda}\left[ \fr{1}{2}\sum\limits_{m=2}^{k-i+1}\! (m-1) mr_m 
+(k-i) \sum\limits_{m=k-i+2}^\infty mr_m-{}\right.\\
  \left.{}- \fr{1}{2}\left( k-i\right) (k-i+1) \!\sum\limits_{m=k-i+2}^\infty\!\! \!\!r_m\right] 
-{}\\
{}-C_2(i-1)\overline{v}-C_4\overline{v}\,,\enskip 1\leq i\leq k-1\,;
  \\
  q_0^k=q_1^k-\fr{C_3+C_4}{\lambda}\,.
   \label{e5-ag}
  \end{multline}
  Данная формула и~формула, приведенная для~$q_i^k$ в~[1], отличаются 
только слагаемыми с~па\-ра\-мет\-ром~$C_0$. Это является следствием того, что 
в~рассматриваемой СМО плата за обслуживание заявки производится после 
завершения обслуживания и~в~размере~$C_0 {v}$.
  
  Справедливы следующие эквивалентные формулы (см.~(16) в~[1]):
  \begin{align}
  R_k&= \fr{1}{r_0}\left({\sum\limits_{j=1}^{k-1} R_j \sum\limits_{i=k-j+1}^\infty r_i 
+R_0\sum\limits^\infty_{i=k} r_i}\right)\,;\notag\\
  \pi_k^{k+1} &= \fr{1}{r_0}\left({\sum\limits_{j=1}^{k-1} \pi_j^{k+1}  
\sum\limits_{i=k-j+1}^\infty r_i +\pi_0^{k+1}\sum\limits^\infty_{i=k} r_i}\right)\,.\label{e6-ag}
  \end{align}
  
  С учетом изменений, внесенных в~формулу~(\ref{e5-ag}), равенство~(13) из 
работы~[1] принимает вид: 
  \begin{multline}
  g^k-g^{k+1}={}\\
  {}=\pi_k^{k+1}\left\{ g^k-\fr{1-\pi_k^{k+1}}{\pi_k^{k+1}} 
\sum\limits_{j=1}^{k-1} \pi_j^k \left\{ C_1\sum\limits_{m=k-j+1}^\infty r_m-
{}\right.\right.\\
  \left.{}- \fr{C_2}{\lambda}\sum\limits^\infty_{m=k-j+1}\left[ m-(k-j+1)\right] 
r_m\right\}-{}\\
  {}- \fr{1-\pi_k^{k+1}}{\pi_k^{k+1}}\,\pi_0^k\left[ C_1 
\sum\limits^\infty_{m=k} r_m -\fr{C_2}{\lambda} \sum\limits^\infty_{m=k} (m-k) 
r_m \right] -{}\\
\left.{}-q_k^{k+1}
\vphantom{\fr{1-\pi_k^{k+1}}{\pi_k^{k+1}} 
\sum\limits_{j=1}^{k-1} \pi_j^k}
\right\}\,.
  \label{e7-ag}
  \end{multline}
  
  Далее, использовав обозначения~(17) и~(20) из работы~[1], 
равенство~(\ref{e7-ag}) приведем к~виду (см.~(18) в~[1]):
  \begin{equation}
  g^k-g^{k+1} =\pi_k^{k+1}\left[ g^k-G(k)\right]\,.
  \label{e8-ag}
  \end{equation}
  Здесь
  $$
  G(k) = C_0 \overline{v} -C_1(\rho-1) -C_4\overline{v}  - \fr{C_2}{\lambda} 
\left[ r_0 F(k) -1+k\rho\right]\,,
  $$
где 

\vspace*{-2pt}

\noindent
 \begin{multline*}
  F(k) = \left(\sum\limits_{j=1}^{k-1} R_j \sum\limits^\infty_{m=k-j+1} [m-(k-
j)]r_m+{}\right.\\[-1pt]
\left.{}+R_0 \sum\limits^\infty_{m=k}(m-k+1)r_m
\vphantom{\sum\limits_{j=1}^{k-1}}
\right)\Bigg/
\left( \sum\limits_{j=1}^{k-1} 
R_j \sum\limits^\infty_{i=k-j+1}r_i +{}\right.\\[-1pt]
\left.{}+R_0 \sum\limits^\infty_{i=k}r_i
\vphantom{\sum\limits_{j=1}^{k-1}}\right) ={}\\[-1pt]
  {}= \left(\sum\limits_{j=1}^{k-1} \pi_j^{k+1} \sum\limits^\infty_{m=k-j+1} 
[m-(k-j)]r_m +{}\right.\\[-1pt]
\left.{}+\pi_0^{k+1}\!\! \sum\limits_{m=k}^\infty (m-k+1)r_m 
\vphantom{\sum\limits_{j=1}^{k-1}}\right)\!\!\Bigg/\!\! 
\left(\sum\limits_{j=1}^{k-1} \pi_j^{k+1}\!\!\!\!\! 
\sum\limits^\infty_{m=k-j+1}\!\!\!\! \!r_m 
+{}\right.\\[-1pt]
\left.{}+\pi_0^{k+1} \sum\limits^\infty_{m=k} r_m
\vphantom{\sum\limits_{j=1}^{k-1}}\right)\,.
  \end{multline*}
  
  \vspace*{-2pt}
     
  В~[1] доказаны следующие утверждения.
  

  
\vspace*{2pt}
  
  \noindent
  \textbf{Лемма~1}~[1]. \textit{Справедливы равенства}
  \begin{equation}
  \left.
  \begin{array}{l}
  q_j^{k+1} =\displaystyle q_j^k +C_1\sum\limits_{m=k-j+1}^\infty r_m-{}\\[4pt]
\!\! \displaystyle  {}- \fr{C_2}{\lambda} \left[ \sum\limits^\infty_{m=k-j+1}
 \!\!\!\! mr_m -(k-j+1) \!\!
\sum\limits^\infty_{m=k-j+1}\!\!\!\! r_m\right]\!,\\[4pt]
 \hspace*{40mm}j=1,\ldots, k-1\,,
 \\[4pt]
   \hspace*{32mm}q_0^{k+1} -q_0^k=q_1^{k+1}-q_1^k\,;
\\[4pt]
  \pi_j^{k+1} =\left(1-\pi_k^{k+1}\right) \pi_j^k\,,\enskip j=0,\ldots, k-1\,.
  \end{array}\!
  \right\}\!\!
  \label{e9-ag}
  \end{equation}
  
  \vspace*{-3pt}
  
  
  \noindent
  \textbf{Теорема~1}~[1]. \textit{Пусть $F(k)\hm- F(k+1)\hm< 
\lambda\overline{v}$, $0\hm<\lambda \hm<\infty$, $0\hm< \overline{v}\hm< 
\infty$, $k\hm>0$. Тогда при любых значениях параметров $0\hm\leq C_i \hm< 
\infty$, $i\hm=0,\ldots, 4$, $0\hm< C_2\hm< \infty$ cуществует стационарная 
стратегия $0\hm<k^0\hm<\infty$. При этом если $g^1\hm\geq G(1)$, то 
$k^0\hm=1$, а~если $C_2\hm=0$ и~$g^1\hm<G(1)$, то} $k^0\hm=\infty$.
  
  %\vspace*{2pt}
    
  \noindent
  \textbf{Следствие~1}~[1]. Пусть $F(k)$ удовлетворяет условию теоремы~1. 
Стратегия~$k^0$ удовлетворяет условию $\max\limits_{k>0} g^k\hm= g^{k^0}$ 
тогда и~только тогда, когда~$k^0$ удовлетворяет одному из трех условий:
  \begin{enumerate}[(1)]
\item $G(1)\leq g^1$, $k^0=1$;
\item $g(1)>g^1$, $k^0\hm=\min \{ k:\ G(k)\leq g^k\}$;
\item $g^{k^0-1}< g^{k^0}$, $g^{k^0+1}\hm> g^{k^0}$, $1\hm<k^0\hm<\infty$.
\end{enumerate}

  Докажем ниже, что аналогичные утверждения справедливы и~для 
задачи~(\ref{e3-ag}). Из~(\ref{e5-ag}), использовав соотношение 
$q_{i+1}^{k+1}\hm= q_i^k\hm-C_2\overline{v}$, находим:

\noindent
  \begin{multline*}
  q_k^{k+1}=q_1^2 -C_2(k-1)\overline{v} ={}\\[-1pt]
  {}=C_0\overline{v}-
C_1\left(\lambda\overline{v}-r_1-1+r_0+r_1\right)-{}
\end{multline*}

\pagebreak

\noindent
\begin{multline*}
  \!{}-\fr{C_2}{\lambda}\left[ r_2+\left( \lambda\overline{v}-r_1-2r_2\right) -\left(1-
r_0 -r_1-r_2\right)\right]-{}\\[-3pt]
{}-c_4\overline{v}-C_2(k-1)\overline{v}={}\\[-3pt]
  {}= C_0\overline{v}-C_1\left( \rho-1+r_0\right) +\fr{C_2}{\lambda}\left( 1-
r_0\right) -C_4\overline{v}-C_2 k\overline{v}\,.
  \end{multline*}
  
  Заменив в~правой части равенства~(\ref{e7-ag}) $\pi_k^{k+1}$ и~$\pi_j^k$, 
$j\hm=0,\ldots, k$, на их выражения в~(\ref{e6-ag}) и~(\ref{e4-ag}) 
и~$q_k^{k+1}$ на полученное выше для него выражение, получим после 
преобразований:

\noindent
  \begin{multline*}
  g_k-g^{k+1}=\pi_k^{k+1}\left\{ 
  \vphantom{\sum\limits_{j=1}^{k-1}}g^k-{}\right.\\[-2.5pt]
  {}-C_1\fr{1-\pi_k^{k+1}}{\pi_k^{k+1}}\left[ 
  \sum\limits_{j=1}^{k-1}\pi_j^k \sum\limits_{m=k-j+1}^\infty r_m+\pi_0^k 
\sum\limits_{m=k}^\infty r_m\right]+{}\\[-2.5pt]
  {}+
   \fr{C_2}{\lambda}\,\fr{1}{\pi_k^{k+1}}
   \left(\sum\limits_{j=1}^k \!\pi_j^{k+1}\!\sum\limits^\infty_{m=k-j+1}\![m-(k-j+1)]r_m 
+{}\right.\\[-2.5pt]
\left.{}+\pi_0^{k+1}\sum\limits^\infty_{m=k} (m-k)r_m
\vphantom{\sum\limits_{j=1}^k \!\pi_j^{k+1}\!\sum\limits^\infty_{m=k-j+1}}
\right)-{}\\[-2.5pt]
   \left.{}- \sum\limits_{m=1}^\infty (m-1)r_m -q_k^{k+1}
   \vphantom{g^k-C_1\fr{1-\pi_k^{k+1}}{\pi_k^{k+1}}\left[ 
  \sum\limits_{j=1}^{k-1}\pi_j^k \sum\limits_{m=k-j+1}^\infty r_m+\pi_o^k 
\sum\limits_{m=k}^\infty r_m\right]}
\right\} =\pi_k^{k+1}\Bigg\{ \vphantom{\fr{C_2}{\lambda}}
g^k-C_1r_0+{}\\[-2.5pt]
   {}+
  \fr{C_2}{\lambda}\,\fr{1}{\pi_l^{k+1}}\left(
  \sum\limits^k_{j=1} \!\pi_j^{k+1} \!\!
\sum\limits^\infty_{m=k-j+1}\![m-(k-j+1)]r_m +{}\right.\\[-2.5pt]
\left.{}+\pi_0^{k+1}\sum\limits^\infty_{m=k} (m-k)r_m
\vphantom{\sum\limits^k_{j=1} \!\pi_j^{k+1} \!\!
\sum\limits^\infty_{m=k-j+1}}\right) -{}\\[-2.5pt]
  {}-
  \fr{C_2}{\lambda}\left(\rho-1+r_0\right) -\left[
  \vphantom{\fr{C_1}{\lambda}} 
  C_0\overline{v} - C_1\left(\rho -
1+r_0\right) +{}\right.\\[-2.5pt]
\left.{}+\fr{C_2}{\lambda} \left(1-r_0\right) -C_4\overline{v}-C_2 k 
\overline{v}\right]
\Bigg\}={}\\[-2.5pt]
  {}=
   \pi_k^{k+1} \left\{ 
\vphantom{\sum\limits^k_{j=1}\pi_j^{k+1}}
 g^k -C_0\overline{v} +C_1(\rho-1) +C_4\overline{v}+{}\right.\\[-2.5pt]
   {}+\fr{C_2}{\lambda}\,\fr{1}{\pi_k^{k+1}}
\left[\left( \sum\limits^k_{j=1}\pi_j^{k+1} \!\!\!\!
\sum\limits^\infty_{m=k-j+1}  \!\!\!
[m-(k-j+1)]r_m +{}\right.\right.\\[-2.5pt]
\left.\left.\left.{}+\pi_0^{k+1} \sum\limits^\infty_{m=k} (m-k)r_m
\vphantom{\sum\limits^k_{j=1}\pi_j^{k+1}}\right)+
 (k-1)\rho\right]\right\}\,.
   \end{multline*}
  Обозначим 
  
  \noindent
  \begin{multline*}
 \hspace*{-7pt}\!\! \tilde{F}(k) =\fr{1}{\pi_k^{k+1}}\left(\sum\limits^k_{j=1\!} \pi_j^{k+1}
  \!\!\! \sum\limits^\infty_{m=k-
j+1}\!\! \!\!\!\![m-(k-j+1)]r_m+{}\right.\\[-3pt]
\left.{}+\pi_0^{k+1}\sum\limits^\infty_{m=k}(m-
k)r_m\right) = {}
\end{multline*}

\columnbreak

\noindent
\begin{multline*}
  {}=r_0\left(\sum\limits_{j=1}^{k-1} \pi_j^{k+1}\sum\limits^\infty_{m=k-
j+1} [m-(k-j)]r_m +{}\right.\\
\left.{}+\pi_0^{k+1}\! \sum\limits^\infty_{m=k}(m-k+1)r_m
\vphantom{\sum\limits^{k-1}_{j=1}}
\right)\!\!\Bigg/ \!\!
\left(\sum\limits^{k-1}_{j=1}\!\pi_j^{k+1} \!\!\!
\sum\limits^\infty_{i=k-j+1}\!\!\! r_i 
+{}\right.\\
\left.{}+\pi_0^{k+1} \sum\limits^\infty_{i=k}r_i
\vphantom{\sum\limits^{k-1}_{j=1}}
\right)+{}\\
  {}+ \sum\limits^\infty_{m=1} (m-1)r_m-r_0=r_0F(k)+\rho-1\,.
  \end{multline*}
  
  
  
  Тогда, как видно из последнего соотношения для разности $g^k\hm- 
g^{k+1}$, выражение для $G(k)$ в~(\ref{e8-ag}), если заменить $F(k)$ на его 
выражение через $\tilde{F}(k)$, примет вид: 
  $$
  G(k)=C_0\overline{v} -C_1(\rho-1) -C_4\overline{v} -\fr{C_2}{\lambda} \left[ 
\tilde{F}(k)+(k-1)\rho\right].
  $$
  
  Обратим внимание на то, что $\tilde{F}(k)$~--- среднее число заявок, 
отклоненных за время обслуживания одной заявки (на одном шаге 
соответствующей цепи Маркова) при стратегии $k+1$ в~стационарном режиме 
работы системы, если число поступивших заявок не меньше, чем число 
свободных мест в~накопителе.
  
  Заметим, что приведенное в~[1] доказательство теоремы~1 и~следствия~1 
полностью основывается на существовании для функции~$g^k$ соотношения 
$g^k\hm- g^{k+1}\hm= \alpha_k[g^k\hm- G(k)]$, где $0\hm<\alpha_k\hm<1$ при 
$k\hm>0$, $G(k)$ не возрастает по $k\hm>0$. То, что $G(k)$~--- 
не\-воз\-рас\-та\-ющая по $k\hm>0$ функция, следует из неравенства (условия 
теоремы~1):
  $$
  F(k)-F(k+1)<\rho\,,\enskip 0<\rho<\infty\,,\ k>0\,.
  $$

Заметим также, что $G(k)$ является не\-воз\-рас\-та\-ющей функцией по $k\hm>0$ 
и~при условии (см.~(21) в~[1])
\begin{equation}
\tilde{F}(k) -\tilde{F}(k+1) <\rho\,,\enskip 0<\rho<\infty\,,\ k>0\,,
\label{e10-ag}
\end{equation}
и~теорема~1 остается справедливой, если неравенство для $F(k)$ в~условии 
теоремы заменить на неравенство~(\ref{e10-ag}). 

  Так как в~момент перехода системы в~любое состояние в~накопителе всегда 
есть хотя бы одно свободное место, длительность интервала времени с~момента 
заполнения накопителя до момента перехода в~другое состояние всегда 
меньше~$v$ (длительности нахождения в~состоянии) и,~следовательно, 
справедливо неравенство $\vert \tilde{F}(k)\hm-\tilde{F}(k\hm+1)\vert <\rho$. Отсюда следует

\noindent
  \begin{multline*}
  \!\!\!G(k) -G(k+1) =-\fr{C_2}{\lambda}\left[ \tilde{F}(k) -\tilde{F}(k+1)\right] 
+C_2\overline{v}\geq{}\\
 \! {}\geq -\fr{C_2}{\lambda}\left\vert \tilde{F}(k) -\tilde{F}(k+1)\right\vert 
+C_2\overline{v} > -\fr{C_2}{\lambda}\,\rho +C_2\overline{v}=0,\hspace*{-4.64pt}
  \end{multline*}
т.\,е.\ в~рамках рассматриваемой задачи всегда $G(k)$~--- убывающая по 
$k\hm>0$ функция. 

  Положим 
  $$
  f^k=Q^k\hm= \lambda(1\hm- \theta_k^k)g^k\,,
  $$ 
  где $\theta_k^k$~--- 
вероятность того, что поступившая заявка будет допущена в~систему (см.~(2)), 
$\theta^k_k\hm= 1\hm- 1/(\pi_0^k+\rho)$~\cite{8-ag}.  
Использовав~(\ref{e9-ag}) и~приведенное выше равенство $g^k\hm- 
g^{k+1}\hm= \pi_k^{k+1}[g^k\hm- G(k)]$, находим:
  \begin{multline}
  f^k-f^{k+1}= \lambda\left[ \left(1-\theta^k_k\right) g^k -\left(1-
\theta_{k+1}^{k+1}\right) g^{k+1}\right]={}\\
  {}= \lambda\left[\left( \theta_{k+1}^{k+1} -\theta_k^k\right)g^k +\left(1-
\theta_{k+1}^{k+1}\right) \left( g^k-g^{k+1}\right)\right]={}\\
  {}=\lambda \left[ \left( \fr{1}{\pi_0^k+\rho}-\fr{1}{\pi_0^{k+1}+\rho}\right) 
g^k+{}\right.\\
\left.{}+ \pi_k^{k+1}\fr{1}{\pi_0^{k+1}+\rho}\left( g^k -G(k)\right)\right]={}\\
  {}=
  \lambda\left[ 
  \vphantom{\fr{\pi_0^k \pi_k^{k+1}}{(\pi_0^k+\rho) (\pi_0^{k+1}+\rho)}}
  \pi_k^{k+1}\fr{1}{\pi_0^{k+1}+\rho}\left( g^k -G(k)\right) -{}\right.\\
\left.  {}-
\fr{\pi_0^k \pi_k^{k+1}}{(\pi_0^k+\rho) (\pi_0^{k+1}+\rho)}\,g^k \right]={}\\
  {}=
  \lambda\fr{\pi_k^{k+1}}{\pi_0^{k+1}+\rho}\left[ g^k\left(1- 
\fr{\pi_0^k}{\pi_0^k+\rho}\right)-G(k)\right] ={}\\
{}= \lambda \fr{\pi_k^{k+1}} 
{\pi_0^{k+1}+\rho}\left[  g^k \fr{\rho}{\pi_0^k+\rho}-G(k)\right]={}\\
 {}= \fr{\rho \pi_k^{k+1}}{\pi_0^{k+1}+\rho}\left[g^k\fr{\lambda}{\pi_0^k+\rho}- 
\fr{\lambda G(k)}{\rho}\right] ={}\\
{}=\alpha_k \left[ f^k-\tilde{G}(k)\right]\,,
  \label{e11-ag}
  \end{multline}
где $\alpha_k=\rho\pi_k^{k+1}/(\pi_0^{k+1}+\rho)$; $\tilde{G}(k)\hm= 
G(k)/\overline{v}$. 
  
  Так как функция $G(k)$ убывает по $k\hm>0$, то $\tilde{G}(k)$ также 
убывает по $k\hm>0$. Кроме того, $0\hm< \alpha_k\hm<1$, $k\hm>0$. Как 
видим, функция $f^k\hm=Q^k$ обладает всеми свойствами функции~$g^k$ 
использованными при доказательстве теоремы~1. 
  
Следовательно, из доказательства теоремы~1 с~учетом того, что 
$\tilde{G}(k)$~--- строго убывающая функция по $k\hm>0$, следует 
справедливость следующего утверждения. 

\smallskip

\noindent
  \textbf{Утверждение~1.} При любых значениях па\-ра\-мет\-ров $0\hm\leq 
C_i\hm<\infty$, $i\hm=0,\ldots, 4$, $0\hm< C_2\hm<\infty$\linebreak существует 
оптимальный порог $0\hm< k^*\hm<\infty$, $k^*\hm=\min\{ k>0:\ 
\tilde{G}(k)\hm\leq Q^k\}$ При этом если $Q^1\hm\geq \tilde{G}(1)$, то 
$k^*\hm=1$, а если $C_2\hm=0$ и~$Q^1\hm<\tilde{G}(1)$, то $k^*\hm=\infty$.
  
  \smallskip
  
  Справедливо также следующее утверждение.
  
  \smallskip
  
  \noindent
  \textbf{Утверждение~2.} Пороговое значение~$k^*$ удовлетворяет 
условию $\max\limits_{k>0} Q^k\hm= Q^{k^*}$ тогда и~только тогда, 
когда~$k^*$ удовлетворяет одному из трех условий:
  \begin{enumerate}[(1)]
\item $\tilde{G}(1)\leq Q^1$, $k^*\hm=1$;
\item $\tilde{G}(1)>Q^1$, $k^*\hm= \min\{ k>0:\ \tilde{Q}(k)\hm\leq Q^k\}$;
\item $Q^{k^*-1}<Q^{k^*}$, $Q^{k^*+1}\hm< Q^{k^*}$, $1\hm< k^*\hm< 
\infty$.
\end{enumerate}
  
  \noindent
  Д\,о\,к\,а\,з\,а\,т\,е\,л\,ь\,с\,т\,в\,о\,.\ \ Необходимость указанных 
условий следует из утверждения~1, а их достаточность следует  
из~(\ref{e11-ag}) и~из того, что $\tilde{G}(k)$~--- убывающая функция: для всех 
$k\hm>0$, $\tilde{G}(k)\hm> Q^k$ тогда и~только тогда, когда $Q^k\hm< 
Q^{k+1}$, и~если $k^\prime \hm> 0$, $\tilde{G}(k^\prime) 
\hm\leq Q^{k^\prime}$, то  
$\tilde{G}(k) \hm< Q^k$ для всех $k\hm> k^\prime$.
  
  Далее приведем формулу, удобную для расчета функции $\tilde{F}(k)$ 
и~алгоритм поиска оптимального порогового значения. Преобразуем $\overline{F}(k)$:
  \begin{multline*}
 \!\! \overline{F}(k)= \fr{1}{\pi_k^{k+1}}\!\left(\sum\limits_{j=1}^{k-1} \pi_j^{k+1}\!\!\!\!\!\!
   \sum\limits^\infty_{m=k-j+1} \!\!\!\!
[m-(k-j+1)]r_m +{}\right.\hspace*{-1.89pt}\\
\left.{}+\pi_0^{k+1}\sum\limits^\infty_{m=k} (m-
k)r_m
\vphantom{\sum\limits_{j=1}^{k-1}}\right)={}\\
  {}=\fr{1}{\pi_k^{k+1}}\left(\sum\limits_{j=1}^{k-1}\pi_j^{k+1} \!\!\!\!
  \sum\limits^\infty_{m=k-j+1} \!\!\!\!
[m-(k-j)]r_m -{}\right.\\
\left.{}-\sum\limits^{k-1}_{j=1}\pi_j^{k+1}\sum\limits^\infty_{m=k-
j+1} r_m\right)+{}\\
  {}+
  \fr{\pi_0^{k+1}\sum\nolimits^\infty_{m=k} [m-(k-1)]r_m -
\pi_0^{k+1}\sum\nolimits^\infty_{m=k} r_m}{\pi_k^{k+1}}={}\\
  {}=\fr{\pi_{k-1}^{k+1} \sum\nolimits^\infty_{m=2}  (m-
1)r_m}{\pi_k^{k+1}}+{}\\
  {}+ \fr{1-\pi_k^{k+1}}{\pi_k^{k+1}}
   \left(\sum\limits^{k-2}_{j=1}\pi_j^k 
\sum\limits^\infty_{m=k-j}[m-(k-j)]r_m +{}\right.\\
\left.{}+\pi_0^k \sum\limits^\infty_{m=k-1} 
[m-(k-1)]r_m
\vphantom{\sum\limits_{j=1}^{k-2}}\right)-{}\\
  {}-
\fr{1-\pi_k^{k+1}}{\pi_k^{k+1}}\left( \sum\limits_{j=1}^{k-1} \pi_j^k 
\sum\limits^\infty_{m=k-j+1} r_m +\pi_0^k \sum\limits^\infty_{m=k} 
r_m\right)\,.
  \end{multline*}
  \begin{figure*} %fig1
 \vspace*{1pt}
 \begin{minipage}[t]{81mm}
\begin{center}
\mbox{%
\epsfxsize=76.495mm
\epsfbox{aga-1.eps}
}
\end{center}
\vspace*{-9pt}
\Caption{Зависимости функций~$Q^k$~(\textit{1}), $g^k$~(\textit{2}) и~$\tilde{G}(k)$ 
($G(k)$)~(\textit{3}) от порогового значения}
\end{minipage}
%\end{figure*}
\hfill
%\begin{figure*} %fig2
\vspace*{1pt}
 \begin{minipage}[t]{81mm}
\begin{center}
\mbox{%
\epsfxsize=76.495mm
\epsfbox{aga-2.eps}
}
\end{center}
\vspace*{-9pt}
\Caption{Зависимости функций $Q^k$~(\textit{1}), $g^k$~(\textit{2}), 
$\tilde{G}(k)$~(\textit{3}) и~$G(k)$~(\textit{4}) от порогового значения}
\end{minipage}
\vspace*{3pt}
\end{figure*}
  
  Использовав~(\ref{e6-ag}) и~(\ref{e9-ag}), перепишем последнее равенство 
в~виде:

\noindent
  \begin{multline*}
  \overline{F}(k)= \fr{\pi_{k-1}^{k+1}}{\pi_k^{k+1}}\sum\limits_{m=2}^\infty (m-1) r_m+{}\\
  {}+
  \fr{1-\pi_k^{k+1}}{\pi_k^{k+1}} \,\pi^k_{k-1}\overline{F}(k-1)-
   \fr{1-\pi_k^{k+1}}{\pi_k^{k+1}}\,\fr{r_0\pi_k^{k+1}}{1-\pi_k^{k+1}}= {}\\
   {}=
\fr{\pi_{k-1}^{k+1}}{\pi_k^{k+1}}\left( \rho+r_0 -1\right) +\fr{\pi_{k-
1}^{k+1}}{\pi_k^{k+1}}\,\overline{F}(k-1) - r_0={}\\
  {}=
  \fr{R_{k-1}}{R_k}\left[ \overline{F}(k-1)+a\right] -r_0\,,\enskip a=\rho+r_0-1\,,\\
   k\geq 1\,,\enskip \overline{F}(0)=1-r_0\,.
  \end{multline*}
  
  Применив последнюю формулу саму к~себе, получим: 
  \begin{multline}
  \tilde{F}(k) -a=\overline{F}(k)=\fr{R_0}{R_k}\,\overline{F}(0) 
 +a\sum\limits_{i=0}^{k-1}  \fr{R_i}{R_k}-{}\\
 {}-r_0 \left( 
1+\sum\limits_{i=1}^{k-1} \fr{R_i}{R_k}\right) = \fr{R_0}{R_k} +a\fr{R_0+\cdots+R_{k-1}}{R_k} -{}\\
  {}-r_0 
\fr{R_0+\cdots+R_k}{R_k} = \fr{1}{R_k} + \fr{\rho-1}{\pi_k^{k+1}}-a\,.
  \label{e12-ag}
  \end{multline}
  
  Предлагается наиболее простой алгоритм, основанный на условии~2 
утверждения~2.
  \begin{enumerate}[1.]
\item Положить $k=1$.\\[-14pt]
\item Вычислить $\tilde{G}(k)$ и~$Q^k$.\\[-14pt]
\item Если $C_2=0$ и~$Q^k\hm<\tilde{G}(k)$, то положить $k^*\hm=\infty$ 
и~перейти к~п.~8.\\[-14pt]
\item Если выполняется неравенство $\tilde{G}(k)\hm\leq Q^k$, то перейти 
к~п.~7.\\[-14pt]
\item Увеличить $k$ на единицу.\\[-14pt]
\item Вычислить $\tilde{G}(k)$, $Q^k$ и~перейти к~п.~4.\\[-14pt]

\item Положить $k^*\hm=k$.\\[-14pt]
\item Конец алгоритма.\\[-14pt]
\end{enumerate}
  
  Как видим, предложенный алгоритм гарантирует поиск оптимального 
порогового значения и~при этом число вычислений функций~$Q^k$ 
и~$\tilde{G}(k)$ не превышает значения оптимального порога. При расчете на 
очередном шаге алгоритма значений~$Q^{k+1}$ и~$\tilde{G}(k+1)$ 
используются рекуррентная формула~(\ref{e4-ag}), формулы~(\ref{e12-ag}) 
и~(\ref{e11-ag}).
  
  Заметим, что для поиска оптимального порога можно применить метод 
бинарного поиска, который потребует порядка $\log k^*$ вычислений функций 
$Q^{k+1}$ и~$\tilde{G}(k+1)$.
  
\section{Пример}

  В качестве примера рассмотрена СМО $M$/$H_n$/1 с~функцией 
распределения времени обслуживания $H_n(t)\hm= \sum\nolimits_{i=1}^n 
f_i(1\hm-e^{-\mu_it})$. На рис.~1 и~2 проиллюстрированы 
зависимости предельного максимального дохода СМО, усредненного по чис\-лу 
обслуженных заявок ($g^k$), и~предельного максимального дохода СМО 
в~единицу времени ($Q^k$) от порогового значения, а~также 
взаимозависимость функций~$Q^k$ $(g^k)$ и~$\tilde{G}(k)$ ($G(k)$)
при $C_0\hm= 
20$, $C_1\hm=10$, $C_2\hm= 0{,}5$, $C_3\hm= 0{,}01$, $C_4\hm= 0{,}01$,
$n\hm=2$ и~$\mu_1\hm=1$.
Графики рис.~1 построены при значениях параметров 
$\lambda \hm=2$, $f_1\hm=0{,}3$, $f_2\hm= 0{,}7$ 
и~$\mu_2\hm=1$, а~рис.~2~---
 при $\lambda\hm=1$, $f_1\hm=0{,}2$, $f_2\hm=0{,}8$ 
 и~$\mu_2\hm=2$. При этом в~случае рис.~1 $r_0\hm= 1/3$
и~$\rho\hm=2$, в~случае рис.~2 $r_0\hm=0{,}633$ и~$\rho\hm=0{,}6$.
  
\section{Заключение}

  Данная работа является непосредственным продолжением работы~[1], 
и~повторное исследование этой задачи позволило получить следующие новые 
результаты: 
  \begin{itemize}
\item доказано, что при $C_2\hm>0$ существует оптимальное пороговое 
значение $k\hm<\infty$, гарантиру\-ющее максимальный предельный доход 
сис\-те\-мы $M$/$G$/1 в~единицу времени;
\item сформулированы условия, при которых оптимальный порог равен 
бесконечности;
\item доказано утверждение о необходимых и~достаточных условиях 
оптимальности порогового значения; 
\item предложен алгоритм расчета оптимального порогового значения 
и~значения максимального дохода. 
\end{itemize}

  Отметим, что, в~отличие от постановки задачи, рассмотренной в~[1], 
в~данной работе плату за обслуживание заявки система получает в~момент 
завершения обслуживания и~величина платы прямо пропорциональна времени 
занятия прибора заявкой. Отметим также, что если константу~$C_0$ заменить 
на любую положительную функцию от времени обслуживания заявки 
с~конечным средним значением, то все приведенные выше утверждения 
останутся в~силе.
  
  Результаты работы могут быть использованы при исследовании и~разработке 
эффективных пороговых стратегий управления в~системах с~очередями.

\vspace*{-3pt}
  
{\small\frenchspacing
 {%\baselineskip=10.8pt
 \addcontentsline{toc}{section}{References}
 \begin{thebibliography}{9}
\bibitem{1-ag}
\Au{Агаларов Я.\,М.} Пороговая стратегия ограничения доступа к~ресурсам 
в~системе массового обслуживания\linebreak\vspace*{-10pt}

\columnbreak

\noindent
 $M/D/1$ с~функцией штрафов
 за несвоевременное обслуживание 
заявок~// Информатика и~её применения, 2015. Т.~9. Вып.~3. С.~56--65.
\bibitem{2-ag}
\Au{Агаларов Я.\,М., Агаларов~М.\,Я., Шоргин~В.\,С.} Об оптимальном 
пороговом значении длины очереди в~одной задаче максимизации дохода 
системы массового обслуживания 
типа $M/G/1$~// Информатика и~её применения, 2016. Т.~10. Вып.~2. С.~70--79.
\bibitem{3-ag}
\Au{Агаларов Я.\,М., Агаларов~М.\,Я., Шоргин~В.\,С.} Максимизация дохода 
системы массового обслуживания типа $G/M/1$ на множестве пороговых стратегий с~двумя точками 
переключения~// Системы и~средства информатики, 2016. Т.~26. Вып.~4. 
С.~74--88.
\bibitem{4-ag}
\Au{Каштанов В.\,А., Кондрашова~Е.\,В.} Исследование полумарковских 
систем массового обслуживания при управляемом входящем потоке.  
BSMAP-поток~// Управление большими системами, 2015. Вып.~57. 
С.~6--36. 
\bibitem{5-ag}
\Au{Гришунина Ю.\,Б.} Оптимальное управление очередью в~системе 
$M/G/1/\infty$ с~возможностью ограничения приема заявок~// Автоматика 
и~телемеханика, 2015. №\,3. С.~79--93. 
\bibitem{6-ag}
\Au{Карлин С.} Основы теории случайных процессов~/ Пер. с~англ.~--- М.: 
Мир, 1971. 536~с. (\Au{Karlin~S.} A~first course in stochastic processes.~--- New 
York\,--\,London: Academic Press, 1968. 502~p.). 
\bibitem{7-ag}
\Au{Бочаров П.\,П., Печинкин А.\,В.} Теория массового обслуживания.~--- М.: 
РУДН, 1995. 529~с.
\bibitem{8-ag}
\Au{Miyazawa M.} Complementary generating functions for the $M^X$/GI/1/$k$ and 
GI/$M^Y$/1/$k$ queues and their application to the comparison of loss probabilities~// 
J.~Appl. Probab., 1990. Vol.~27. P.~684--692.
 \end{thebibliography}

 }
 }

\end{multicols}

\vspace*{-3pt}

\hfill{\small\textit{Поступила в~редакцию 09.02.17}}

\vspace*{8pt}

%\newpage

%\vspace*{-24pt}

\hrule

\vspace*{2pt}

\hrule

%\vspace*{8pt}


\def\tit{MAXIMIZATION OF~AVERAGE STATIONARY PROFIT\\ IN~$M$/$G$/1 QUEUING SYSTEM}

\def\titkol{Maximization of~average stationary profit in~$M$/$G$/1 queuing system}

\def\aut{Ya.\,M.~Agalarov}

\def\autkol{Ya.\,M.~Agalarov}

\titel{\tit}{\aut}{\autkol}{\titkol}

\vspace*{-9pt}


\noindent
Institute of Informatics Problems, Federal Research Center 
``Computer Science and Control'' of the Russian
Academy of Sciences,  44-2~Vavilov Str., Moscow 119333, Russian Federation



\def\leftfootline{\small{\textbf{\thepage}
\hfill INFORMATIKA I EE PRIMENENIYA~--- INFORMATICS AND
APPLICATIONS\ \ \ 2017\ \ \ volume~11\ \ \ issue\ 2}
}%
 \def\rightfootline{\small{INFORMATIKA I EE PRIMENENIYA~---
INFORMATICS AND APPLICATIONS\ \ \ 2017\ \ \ volume~11\ \ \ issue\ 2
\hfill \textbf{\thepage}}}

\vspace*{3pt}
  
  
  
  
  \Abste{The problem of optimization of the queue length threshold in a~$M/G/1$ 
  system is considered in terms of maximizing the marginal return received by the 
  system per unit of time. The profit value consists of the following measures: 
  service fee; hardware maintenance fee; cost of service delay; 
  fine for unhandled 
  requests; and fine for system idle. The author formulates the necessary conditions of 
  existence of a finite threshold in an~$M/G/1$ system and prove the statements of 
  necessary and sufficient conditions for threshold optimality and existence of the 
  finite optimal threshold. The author proposes an algorithm for calculating the 
  optimal threshold value and the corresponding maximal profit. The author presents 
  the results of 
  computational experiments that illustrate the work of the proposed algorithm.}
  
  \KWE{queuing system; threshold management; optimization}
  
\DOI{10.14357/19922264170203} 

%\vspace*{-18pt}

\Ack
\noindent
The work was partly supported by the Russian Foundation
for Basic Research
(project 15-07-03406). 



%\vspace*{3pt}

  \begin{multicols}{2}

\renewcommand{\bibname}{\protect\rmfamily References}
%\renewcommand{\bibname}{\large\protect\rm References}

{\small\frenchspacing
 {%\baselineskip=10.8pt
 \addcontentsline{toc}{section}{References}
 \begin{thebibliography}{9}
\bibitem{1-ag-1}
\Aue{Agalarov, Ya.\,M.} 2015. Porogovaya strategiya ogranicheniya dostupa 
k~resursam v~sisteme massovogo ob\-slu\-zhi\-va\-niya 
$M/D/1$ s~funktsiey shtrafov za nesvoevremennoe 
obsluzhivanie zayavok [The threshold strategy for restricting access in the $M/D/1$ 
queuing system with penalty function for late service]. \textit{Informatika i~ee 
Primeneniya~--- Inform. Appl.} 9(3):56--65.
\bibitem{2-ag-1}
\Aue{Agalarov, Ya.\,M., M.\,Ya.~Agalarov, and V.\,S.~Shorgin.} 2016. Ob 
optimal'nom porogovom znachenii  dliny ocheredi v~odnoy zadache maksimizatsii 
dokhoda sistemy massovogo ob\-slu\-zhi\-va\-niya tipa $M/G/1$ [About the optimal threshold of queue length in 
particular problem of profit maximization in the $M/G/1$ queuing system]. 
\textit{Informatika i~ee Primeneniya~--- Inform. Appl.} 10(2):70--79.
\bibitem{3-ag-1}
\Aue{Agalarov, Ya.\,M., M.\,Ya.~Agalarov, and V.\,S.~Shorgin.} 2016. 
Maksimizatsiya dokhoda sistemy massovogo ob\-slu\-zhi\-va\-niya 
tipa $G/M/1$  na mnozhestve porogovykh strategiy 
s~dvumya tochkami pereklyucheniya [Profit maximization in $G/M/1$ queuing 
system on a~set of threshold strategies with two switch points]. \textit{Sistemy  i~Sredstva 
Informatiki~--- System and Means of Informatics} 26(4):74--88.
\bibitem{4-ag-1}
\Aue{Kashtanov, V.\,A., and E.\,V.~Kondrashova.} 2015. Issledovanie 
polumarkovskikh sistem massovogo obsluzhivaniya pri\ uprav\-lyaemom 
vkhodyashchem potoke.  BSMAP-potok [Research of semi-Markov queueing models 
using controlled input flow. BSMAP-flow]. \textit{Upravlenie bol'shimi sistemami}  
[Large-Scale Systems Control] 57:6--36.
\bibitem{5-ag-1}
\Aue{Grishunina, Yu.\,B.} 2015. Optimal control of 
queue in the $M/G/1/\infty$ system with possibility of customer admission 
restriction. \textit{Automat. Rem. Contr.} 
76(3):433--445.
\bibitem{6-ag-1}
\Aue{Karlin, S.} 1968. \textit{A~first course in stochastic processes}. 
New York\,--\,London: Academic Press. 502~p.
\bibitem{7-ag-1}
\Aue{Bocharov, P.\,P., and A.\,V.~Pechinkin.} 1995. \textit{Teoriya massovogo 
obsluzhivaniya} [Queueing theory]. Moscow: RUDN. 529~p.
\bibitem{8-ag-1}
\Aue{Miyazawa, M.} 1990. Complementary generating functions for the 
$M^X$/GI/1/$k$ and GI/$M^Y$/1/$k$ queues and their application to the comparison of 
loss probabilities. \textit{J.~Appl. Probab.} 27:684--692.
\end{thebibliography}

 }
 }

\end{multicols}

\vspace*{-3pt}

\hfill{\small\textit{Received February 9, 2017}}
  
  \Contrl
  
  \noindent
   \textbf{Agalarov Yaver M.} (b.\ 1952)~--- Candidate of Science in technology, associate professor, 
leading scientist, Institute of Informatics Problems, Federal Research Center ``Computer Science and 
Control'' of the Russian Academy of Sciences, 44-2~Vavilov Str., Moscow 119333, Russian Federation; 
\mbox{agglar@yandex.ru}
  
   
  
\label{end\stat}


\renewcommand{\bibname}{\protect\rm Литература} 
   