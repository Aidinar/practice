\def\stat{vasiliev}

\def\tit{ИНФОРМИРОВАННОСТЬ УЧАСТНИКОВ И~СУЩЕСТВОВАНИЕ РАВНОВЕСИЯ 
В~ПОЗИЦИОННЫХ МНОГОШАГОВЫХ ИГРАХ МНОГИХ ЛИЦ}

\def\titkol{Информированность участников и~существование равновесия 
в~позиционных многошаговых играх многих лиц}

\def\aut{Н.\,С.~Васильев$^1$}

\def\autkol{Н.\,С.~Васильев}

\titel{\tit}{\aut}{\autkol}{\titkol}

\index{Васильев Н.\,С.}
\index{Vasilyev N.\,S.}


%{\renewcommand{\thefootnote}{\fnsymbol{footnote}} \footnotetext[1]
%{Работа выполнена при частичной финансовой поддержке РФФИ (проект 15-07-03406).}}


\renewcommand{\thefootnote}{\arabic{footnote}}
\footnotetext[1]{Московский государственный технический 
университет им.\ Н.\,Э.~Баумана, \mbox{nik8519@yandex.ru}}

%\vspace*{-18pt}

 
    \Abst{В позиционных играх изучаются динамические модели принятия 
решений в~условиях конфликта интересов участников и~при информированности 
о~текущей позиции игры. Каждый игрок может оказывать некоторое допустимое 
управляющее воздействие на общую для всех динамическую систему. Выбираемая 
игроком стратегия управления~--- это функция, определенная на фазовом 
пространстве системы. Отслеживая траекторию движения, все стороны конфликта 
имеют неявное представление о~стратегиях партнеров. Принцип рационального 
поведения всех игроков состоит в~стремлении достичь ситуации равновесия Нэша. 
Доказано, что к~нему можно прийти в~результате коллективных усилий по выбору 
совместного программного управления системой. Устойчивость решения 
обеспечивается угрозой наказания, применяемого к~игроку, не выполняющего эту 
программу. Контроль за соблюдением согласованной траектории движения 
и~некоторая дополнительная информация позволяют игрокам идентифицировать 
нарушителя. Его наказание реализуется с~запаздыванием, связанным 
с~необходимостью обнаружения <<виновника>>. Теорема существования 
равновесия применена к~исследованию эко\-но\-ми\-ко-ма\-те\-ма\-ти\-че\-ской 
модели.}
    
    \KW{динамическая система; дифференциальная игра; позиционная 
многошаговая игра; программное управление; позиционная стратегия; 
контрстратегия; стратегия наказания; гарантирующая стратегия; ситуация 
равновесия Нэша; эффективность по Парето}

\DOI{10.14357/19922264170205} 


\vskip 10pt plus 9pt minus 6pt

\thispagestyle{headings}

\begin{multicols}{2}

\label{st\stat}

\section{Введение}

    Значение информированности игроков для принятия рациональных решений 
было выявлено в~исследованиях Ю.\,Б.~Гермейера, Н.\,Н.~Моисеева и~их учеников 
(В.\,А.~Горелика, А.\,Ф.~Кононенко,\linebreak Н.\,С.~Кукушкина, В.\,В.~Фёдорова  
и~др.)~[1--3]. Ак\-сиоматизация понятия стратегии формализует динамику принятия 
решений в~статических играх, моделиру\-ющую рефлексию игроков при обмене 
информацией~[4]. В~дифференциальных и~многошаговых играх добавляется другой 
тип динамики. Игроки принимают решения, управляя и~наблюдая за траекторией 
движения системы. В~каждой текущей позиции разыгрывается <<локальная>> 
статическая игра, участники которой получают неявное знание о стратегиях 
партнеров в~форме изменения позиций игры. Стратегии игроков являются 
воздействиями на управляемую систему, распределенными по ее фазовому 
пространству. 
    
    Начало изучению дифференциальных игр положили задачи преследования--убегания~[5]. 
    В~трудах Л.\,С.~Понтрягина и~его учеников эта теория развивалась 
в~рамках программного управления~[6, 7]. Позиционные стратегии (управ\-ле\-ния 
в~форме синтеза) потребовали обобщить понятие движения динамической сис\-те\-мы, 
которое, вообще говоря, со\-став\-ля\-ет ансамбль траекторий~[8]. В~классе 
произвольных позиционных стратегий было уста\-нов\-ле\-но существование седловой 
точки~[8]. Позиционные дифференциальные игры развивались в~трудах 
Н.\,Н.~Красовского, А.\,И.~Субботина~[8], В.\,А.~Горелика, А.\,Ф.~Кононенко, 
Ю.\,Е.~Чистякова~[9--13], В.\,И.~Жуковского~[14], Л.\,А.~Петросяна~[15] и~других 
авторов. При этом рассматривались игры как антагонистические~[5--8], так 
и~с~непротивоположными интересами  
игроков~\cite{1-vas, 9-vas, 10-vas, 11-vas, 12-vas, 13-vas, 14-vas, 15-vas}. 
Равновесие Нэша является одним из принципов рационального поведения 
игроков~[1]. Существование ситуации равновесия в~дифференциальных играх 
установлено в~работах~\cite{13-vas, 12-vas}. 
    
    Развитие сетевых технологий и~компьютерной техники стимулирует 
исследование моделей коллективного поведения, в~которых все участники игры 
заинтересованы в~рациональном разрешении конфликта интересов. Принятие 
решений существенно усложняется в~позиционной игре с~$n$, $n\hm\geq 2$, 
участниками, которые управляют движением динамической системы, описываемой 
дифференциальными или разностными уравнениями. Игроки преследуют, вообще 
говоря, несовпадающие цели, стремясь по возможности оптимизировать свои 
функции выигрыша. Выбираемые игроками стратегии определяют траекторию 
системы, а вместе с~нею и~результат игры. Всякая текущая позиция игры $(t,x_t)$ 
предполагается известной всем игрокам. Поэтому стратегиями являются 
отображения $\tilde{u}: \{ (t,x)\}\hm\to U$ фазового пространства системы\linebreak 
(совокупности позиций игры $(t,x)$) в~некоторое множество~$U$, которое 
ограничивает управляющие возможности игроков. 
    
    Подстановка в~дифференциальное уравнение произвольного управления 
в~форме синтеза $\tilde{u}\hm= u(t,x)$ создает проблему, связанную с~существованием 
и~единственностью решения задачи Коши. Неслучайно, что в~ранних 
исследованиях по дифференциальным играм классы допустимых стратегий игроков 
ограничивались, например, гладкими отоб\-ра\-же\-ни\-ями $\tilde{u}_t:\ x\hm\to u(t,x)$ 
или даже про\-грам\-мны\-ми управ\-ле\-ни\-ями $\tilde{u}:\ t\hm\to u(t)$. Для приложений 
предпочтительней сохранить единственность траектории движения системы и~не 
ограничивать игроков в~выборе своих стратегий. Поэтому далее рассматриваются 
игровые многошаговые позиционные задачи, в~которых, в~отличие от  
работ~\cite{9-vas, 15-vas}, не фиксируется порядок ходов игроков. Этот класс задач 
является дискретным аналогом дифференциальных игр, нуждающимся 
в~самостоятельном исследовании. В~этой работе использован подход 
Ю.\,Б.~Гермейера и~А.\,Ф.~Кононенко, применяемый для поиска равновесий Нэша 
в~дифференциальных играх двух лиц~\cite{12-vas, 10-vas}. Равновесие 
обеспечивается угрозой наказания~\cite{1-vas, 9-vas, 10-vas, 11-vas, 12-vas, 13-vas}. 
В~играх многих лиц реализация этого подхода опирается на ис\-поль\-зование 
разрешающего правила, которое служит обна\-ружению партнера по игре, 
нарушающего <<принимаемое>> всеми соглашение о совместном программном 
управ\-ле\-нии сис\-те\-мой. Предложенное в~статье правило опирается на сбор 
<<минимальной>> дополнительной информации о ходе игры.

\section{Многошаговая позиционная игра }

    \subsection{Постановка игровой задачи}
    
    Пусть в~игровой операции принимают участие игроки $i\hm= 1, 2,\ldots, n$. Они 
выбирают стратегии $\tilde{u}^i\hm= u_t^i(x)\hm= u(t,x)$, которые на промежутке 
времени $T\hm=\{t_0,t_0+1,\ldots , T-1\}$ определяют движение динамической 
системы, 
    \begin{equation}
    x_{t+1}= g\left(t,x_t,\tilde{u}\right),\ \tilde{u}=\left( \tilde{u}^1, \tilde{u}^2, \ldots 
, \tilde{u}^n\right),\ t\in T\,,
    \label{e2.1-vas}
    \end{equation}
стартующей из начальной позиции $(t_0, x_{t_0})\hm\in N\times R^m$. 
Позиционные стратегии игроков~--- это отображения $\tilde{u}^i:\ T\times R^m \hm\to 
U^i$, принимающие значения в~заданных множествах $U^i\hm= U^i(t,x)$. Такие 
управляющие воздействия игроков называются допустимыми. Интерес каждого 
участника игры состоит в~максимизации по возможности своего критерия 
эффективности
\begin{equation}
w^i=f^i(x_T)\,,\enskip i=1,2,\ldots, n\,.
\label{e2.2-vas}
\end{equation}
    
    Функции выигрыша~(\ref{e2.2-vas}), зависящие от терминального состояния 
системы~(\ref{e2.1-vas})~$x_T$, отвечающего моменту времени~$T$, не 
ограничивают общности задачи. Для платежей игроков~(\ref{e2.2-vas}) введем 
дополнительное обозначение $w^i\hm= F^i(\tilde{u})$, в~явном виде отражающее 
зависимость результатов игры от выбираемых стратегий. Если анализируется 
семейство игр~(\ref{e2.1-vas}), (\ref{e2.2-vas}), в~котором произвольно варьируется 
начальная позиция $(t,x)$, то для критерия~(\ref{e2.2-vas}), $f\hm= (f^1,\ldots , f^n)$, 
применяется более подробная запись $w^i\hm= F^i_{t,x}(\tilde{u})$, $i\hm=1,2,\ldots, 
n$. При этом вдоль траектории системы~(\ref{e2.1-vas}) $F^i_{t,x_t}(\tilde{u})\hm= 
F^i_{t+1, x_{t+1}}(\tilde{u})$.
    
    Согласно~(\ref{e2.1-vas}), (\ref{e2.2-vas}) выигрыш любого игрока 
определяется не только его действиями~$\tilde{u}^i$. Выбор 
параметров~$\tilde{u}^i$, $j\hm\not= i$, контролируется остальными участниками 
игры. Поэтому ни один из игроков не имеет возможности точно прогнозировать 
величину получаемого выигрыша и~вынужден выбирать свою стратегию в~условиях 
<<хаоса>> будущих результатов игры. Применяемые позиционные 
стратегии~$\tilde{u}^i$ могут зависеть от некоторой дополнительной 
информации~$I$, которой располагают игроки, и~тогда стратегии $\tilde{u}^i\hm= 
u_t^i(x;I)$~\cite{1-vas, 4-vas}. Коалиционное поведение игроков не рассматривается.
    
    Всякая ситуация игры~$\tilde{u}$ определяет некоторый  
результат~(\ref{e2.2-vas}). Наибольший интерес представляет собой нахождение 
ситуаций, отвечающих принципам оптимального поведения~\cite{1-vas}. Исследуем 
вопрос нахождения ситуация равновесия Нэша, реализующего принцип 
устойчивости. Напомним, что равновесная ситуация~$\tilde{u}^*$ обладает 
следующим свойством:
    \begin{equation}
    \left(\forall\ i\right) \left( \forall\ \tilde{u}^i\right) F^i
    \left( \tilde{u}^* \left\vert 
\tilde{u}^i\right. \right) \leq F^i\left( \tilde{u}^*\right)\,.
    \label{e2.3-vas}
    \end{equation}
Здесь $\tilde{u}^* \left\vert \tilde{u}^i\right. \hm= \left( \tilde{u}^{*1},\ldots, 
\tilde{u}^{*i-1}, \tilde{u}^i, \tilde{u}^{*i+1},\ldots, \tilde{u}^{*n}\right)$. Далее 
показано, что участники позиционных игр вполне могут рассчитывать на выбор 
эффективных по Парето равновесий, отвечающих и~принципу выгодности~\cite{1-vas, 4-vas}.

    \subsection{Основные определения и~допущения}
    
    Никто из игроков <<не согласится>> получить выигрыш, меньший того, 
который он может сам себе гарантировать. Игру~(\ref{e2.1-vas}), (\ref{e2.2-vas}) 
естественно рас\-смат\-ри\-вать в~рамках семейства задач управления 
системой~(\ref{e2.1-vas}), отличающихся лишь начальной позицией $(t,x)$. Если 
существует ситуация равновесия, то в~ней платежи игроков не могут быть меньше 
наилучших гарантированных результатов~\cite{1-vas, 4-vas}, равных
\begin{equation}
L^i (t,x) =\max\limits_{\tilde{u}^i\in \tilde{U}^i} 
\min\limits_{\tilde{u}^j\in \tilde{U}^j, j\not=i} 
F^i(\tilde{u})\,.
\label{e2.4-vas}
\end{equation}
Величина  $L=(L^i, i=1,2,\ldots, n)$ зависит от классов допустимых стратегий 
игроков~$\tilde{U}^i$, которые зафиксированы в~постановке задачи и~поэтому опущены 
в~обозначениях. Будем считать, что все ограничивающие множества~$U^i$, $i\hm=1,2,
\ldots, n$, 
конечны. Тогда в~определении~(\ref{e2.4-vas}) и~во всех последующих экстремальных задачах 
максимумы и~минимумы достигаются. Поэтому, например, игроки имеют \textit{гарантирующие} 
стратегии~$\tilde{u}^{\mathrm{Г}i}$, доставляющие максимумы в~задачах~(\ref{e2.4-vas}), $i\hm= 
1,2,\ldots, n$. Всех игроков объединяет общее стремление~--- достичь множества $X_T(L)\hm= 
\{x_T:\ f(x_T)\hm\geq L\}\not= \emptyset$~\cite{4-vas}. Более того, целесообразно попасть в~ту его 
часть~$P_T$, элементы которой $x_T^*\hm\in P_T$ являются эффективными решениями 
статической игры $\mathrm{Г}_T\hm= (f, X_T)$. Напомним~\cite{1-vas}, что векторы $x_T^*\hm\in P_T$, 
отвечающие принципу эффективности Парето, обладают свойством
\begin{multline*}
\lnot \left(\exists\ x_T\left( x_T\in X_T\right) \wedge 
\left( f(x_T)\geq f(x_T^*)\right) \wedge{}\right.\\
\left.{}\wedge \left( 
f(x_T)\not= f(x_T^*)\right)\right)\,.
\end{multline*}
В~позиционной игре~(\ref{e2.1-vas})--(\ref{e2.3-vas}) всякая ситуация~$\tilde{u}^*$ 
эффективна, если управление~$\tilde{u}^*$ задает траекторию~$x_t^*$, для которой 
$x_T^*\hm\in P_T$. Целесообразно дальнейшее сужение целевого терминального 
множества~$P_T$ вплоть до подмножества $X_T(M)\cap P_T$, где вектор $M\hm= 
M(t,x)$ имеет координаты $M(T,x)\hm= L(T,x)\eqdelta f(x)$ и
\begin{equation}
M^i(t,x) =\!\!\!\min\limits_{\tilde{u}^j\in \tilde{U}^j, j\not=i} \max\limits_{\tilde{u}^i\in 
\tilde{U}^i} F^i(\tilde{u})\,,\enskip i=1,2,\ldots, n.\!
\label{e2.5-vas}
\end{equation}
Напомним, что $M\geq L$~\cite{4-vas}. Решение задачи~(\ref{e2.5-vas}) 
$\tilde{u}^{\mathrm{н}j}(i)\hm\equiv u^{\mathrm{н}j}(t,x;i)$, $j\not=i$, называется 
\textit{стратегией наказания} $i$-го игрока~\cite{1-vas}. При исполнении наказания 
его оптимальная стратегия~$\tilde{u}^{\mathrm{а}i}$ доставляет максимум 
в~(\ref{e2.5-vas}). Для сравнения: гарантирующая стратегия игрока обеспечивает 
ему результат~$w^{\mathrm{Г}i}$, который при движении системы~(\ref{e2.1-vas}) 
располагается в~диапазоне значений (см.\ леммы~1 и~2)
\begin{multline*}
L^i(t-1,x_{t-1})\leq L^i(t,x_t) \leq w^{\mathrm{Г}i} \equiv{}\\
{}\equiv F^i_{t,x_t} \left( 
\tilde{u}^{\mathrm{Г}i}, \tilde{u}^{\mathrm{н}}(i)\right) \leq M^i(t-1,x_{t-1}) \leq 
M^i(t,x_t)\,.
\end{multline*}
     
    Зафиксируем произвольное программное управ\-ле\-ние $\overline{u}_t\hm= 
(\overline{u}_t^1, \overline{u}_t^2,\ldots , \overline{u}_t^n)$ и~рассмотрим множество 
позиций $D_\tau^i(\overline{u})$, $t_0\hm+1\hm\leq \tau\hm\leq T\hm-1$, 
$i\hm=1,2,\ldots, n$, достижимых системой~(\ref{e2.1-vas}) в~момент~$\tau$  
с~по\-мощью управ\-ле\-ний вида $u\hm= (\overline{u}\vert u_\tau^i)$, $u_\tau^i\hm\in U^i$, 
т.\,е.
    $$
    u^i(t,x)= \begin{cases}
    \overline{u}_t^j\,, &\ t\leq \tau-1\,,\ j\not= i\vee t\leq \tau-2\,;\\
    u^i\,, &\ t=\tau-1\,.
    \end{cases}
    $$
Управление системой~(\ref{e2.1-vas}) происходит на промежутке $t_0\hm\leq 
t\hm\leq \tau\hm-1$. Введем величины  $(i=1,2,\ldots, n)$:
\begin{equation}
\left.
\begin{array}{rl}
\overline{M}^i \left(\overline{u}\right) &=\max\limits_{(t,x)\in D^i(\overline{u})} M^i(t,x)\,;\\[6pt]
D^i\left(\overline{u}\right)&=\displaystyle 
\bigcup\limits_{\tau=t_0+1}^{T-1} D_\tau^i(\overline{u})\,.
\end{array}
\right\}
\label{e2.6-vas}
\end{equation}
Из них составим вектор 
$$
\overline{M}\left(\overline{u}\right)= \left( \overline{M}^1(\overline{u}), 
\overline{M}^2(\overline{u}),\ldots, \overline{M}^n(\overline{u})\right)\,.
$$

    Пусть множество $A\subset T\times R^m$. Рассмотрим движение $t\hm\to 
\overline{x}_t$ системы~(\ref{e2.1-vas}), $u\hm=\overline{u}$, для которого 
$(t,\overline{x}_t)\hm\in A$, $t\hm=t_0,\ldots, T\hm-2$. Допустим, что любой $i$-й игрок 
вместо~$\overline{u}^i$ в~момент времени~$t$ начинает использовать 
произвольную допустимую позиционную стратегию~$\tilde{u}^i$. Если 
у~остальных игроков имеется такая контрстратегия $\overset{\smallfrown}{u}(i)\hm= \left( 
\overset{\smallfrown}{u}^j(i), j\not= i\right)$, применяемая с~момента $t\hm+1$, что 
траектория~$\tilde{x}$ системы~(\ref{e2.1-vas}), $\tilde{u}\hm= \left( 
\overset{\smallfrown}{u}(i),\tilde{u}^i\right)$, не покидает множества~$A$, т.\,е.\ 
$(\tau,\tilde{x}_\tau)\hm\in A$, $\tau\hm=t+1,\ldots, T\hm-1$, то будем говорить 
о~\textit{стабильности} движения $t\hm\to \overline{x}_t$ системы~(\ref{e2.1-vas}) 
относительно множества $A\hm\subset T\times R^m$. Введенное определение является 
дискретным аналогом понятия стабильного моста, используемого 
в~дифференциальных играх~\cite{8-vas}. 
    
    Предположим, что игроки приняли соглашение применять программное 
управление~$u^*$, которому отвечает движение системы~$X^*$. Первое 
отклонение $\parallel x_t\hm- x_t^* \parallel\not=0$ системы~(\ref{e2.1-vas}) от 
траектории~$X^*$ становится известным всем игрокам и~служит сигналом 
нарушения в~момент времени $t\hm-1$ достигнутой договоренности. 
\textit{Разрешающим правилом} назовем такую <<процедуру>>, которая позволяет 
любому участнику конфликта идентифицировать иг\-ро\-ка-на\-ру\-ши\-те\-ля. Это 
правило должно опираться на использование ка\-ких-ли\-бо дополнительных 
сведений~$I$ о~ходе игры: информации о текущей позиции игры $(t,x_t)$, вообще 
говоря, недостаточно. Потребуем выполнения следующих допущений.
 \begin{description}
 \item[\,]   
    Д1. Пусть существует известное всем игрокам такое программное управ\-ле\-ние 
$u(t)\hm= u_t^*$ сис\-те\-мой~(\ref{e2.1-vas}), что $x_T^*\hm\in 
X_T(\overline{M}(u^*))$. 
  \item[\,]   
    Д2. У игроков имеется разрешающее правило, применяемое к~управ\-ле\-нию~$u^*$.
     \item[\,]
    Д3. Ситуация $u^*_{T-1}$~--- равновесие Нэша в~игре $\mathrm{Г}_{T-1}\hm= 
\left( f\circ g, U\right)$, $U\hm= U^1\times U^2\times\cdots\times U^n$.
\end{description}
    
    Рассмотрим произвольное программное управ\-ле\-ние~$\overline{u}_t$ 
и~связанное с~ним подмножество фазового пространства системы~(\ref{e2.1-vas}) 
вида
    $$
    S\left( \overline{u}\right) =\bigcup\limits_i \left\{ (t,x):\ M^i(t,x)\leq 
\overline{M}^i(\overline{u})\right\}\,.
    $$
    
    \noindent
    \textbf{Лемма~1.}\ \textit{В предположении}~Д2, \textit{применяемом 
к~$\overline{u}$, всякое движение $t\hm\to \overline{x}_t$ динамической 
системы}~(\ref{e2.1-vas}) \textit{стабильно относительно множества 
позиций}~$S(\overline{u})$.
    
    \smallskip
    
    \noindent
    Д\,о\,к\,а\,з\,а\,т\,е\,л\,ь\,с\,т\,в\,о\,.\ \ Во-пер\-вых, в~соответствии  
с~(\ref{e2.6-vas}) <<трубка>> $\bigcup\nolimits_i D^i(\overline{u})\hm\subset S(\overline{u})$. 
Во-вто\-рых, согласно предположению Д2, в~качестве 
контрстратегии~$\overset{\smallfrown}{u}(i)$ на стратегию 
$i$-го игрока $\tilde{u}^i\hm\in 
\tilde{U}^i$ можно взять стратегию наказания $\tilde{u}^{\mathrm{н}}(i)\hm= 
(u^{\mathrm{н}j}(t,x;i),\ j\not=i)\hm\in \prod\limits_{j\not=i} \tilde{U}^j$ 
(см.~(\ref{e2.5-vas})). Покажем, что она обеспечивает стабильность движения 
$t\hm\to \overline{x}_t$ относительно~$S$. 
    
    Применение наказания начинается в~той позиции игры $(t,\tilde{x}_t)\hm\in 
D^i_t(\overline{u})$, для которой впервые $\tilde{x}_t\not= \overline{x}_t$. Пусть 
$\tilde{x}_{t+1}$~--- следующее состояние сис\-те\-мы~(\ref{e2.1-vas}), отвечающее 
управлению~$\tilde{u}^i, \tilde{u}^{\mathrm{н}}(i)$. Тогда в~силу~(\ref{e2.2-vas}), 
(\ref{e2.5-vas}) и~(\ref{e2.6-vas}) 
    \begin{multline*}
    M^i(t+1,\tilde{x}_{t+1})\leq 
    \max\limits_{\substack{{\tilde{u}_\tau^i\in \tilde{U}^i,}\\ {\tau\geq t+1}} }
    F^i_{t,\tilde{x}} \left( 
\tilde{u}^i, \tilde{u}^{\mathrm{н}}(i)\right) \leq {}\\
{}\leq
\max\limits_{\substack{{\tilde{u}_\tau^i\in 
\tilde{U}^i,}\\ {\tau\geq t }}} F^i_{t,\overline{x}} ( \tilde{u}^i, \tilde{u}^{\mathrm{н}}(i)) 
\eqdelta M^i(t,\tilde{x}_t) \leq \overline{M}^i(\overline{u})\,.
    \end{multline*}
Неравенство можно продолжить далее для всех моментов времени $t\hm+2,\ldots , 
T\hm-1$ и~получить цепь соотношений:
\begin{multline*}
M^i(T-1,\tilde{x}_{T-1})\leq \cdots\leq M^i\left(t+1,\tilde{x}_{t+1}\right)\leq {}\\
{}\leq
M^i(t,\tilde{x}_t)\leq \overline{M}^i(\overline{u})\,.
\end{multline*}
Это доказывает, что движение системы $t\hm\to \tilde{x}_t$ проходит по 
множеству~$S(\overline{u})$. 
    
    \smallskip
    
    \noindent
    \textbf{Лемма~2.}\ $M^i(t,\overline{x}_t)\hm= M^i(t+1,\overline{x}_{t+1})$ 
\textit{вдоль траектории системы}~(\ref{e2.1-vas}), $\tilde{u}\hm= 
(\tilde{u}^{\mathrm{а}i}, \tilde{u}^{\mathrm{н}}(i))$.
    
    \smallskip
    
    Доказываемое равенство является следствием определения~(\ref{e2.5-vas}) 
стратегии~$\tilde{u}$. Если равновесие существует, то результат игры 
$w^*\hm=f(x^*_T)\hm\geq M(t_0,x_0)$. Предположение~Д3 необходимо для 
равновесия и~обусловлено <<краевым>> эффектом, связанным с~конечностью 
процесса управления системой~(\ref{e2.1-vas}), а~именно: предпоследней позиции 
$(T-1, x^*_{T-1})$ динамической игровой задачи~(\ref{e2.1-vas}), (\ref{e2.2-vas}) 
отвечает статическая игра~$\mathrm{Г}_{T-1}$, в~которой никому из игроков не 
удается повлиять на выбор партнеров. Угроза наказания не действует, так как 
о~нарушении соглашения~$u^*$ становится известно в~момент времени $t\hm=T$, 
когда игра уже закончилась и~уже ничего нельзя изменить. 
    
    Предположение~Д3 становится излишним, если следующим образом изменить 
классы стратегий игроков, а~именно: в~игре~$\mathrm{Г}_{T-1}$ все партнеры 
могут применять смешанные стратегии~\cite{1-vas}. От краевого эффекта можно 
избавиться, изменив правила игры, считая, что факт $u^i_{T-1}\hm \not= u_{T-1}^{*i}$ 
становится мгновенно известным всем игрокам.
    
    \subsection{Существование равновесия Нэша}
    
    В соответствии с~предположением~Д1 рас\-смот\-рим траекторию движения 
$X^*\hm= \{ (t,x_t^*), t\hm=t_0,\ldots, T\}$ системы~(\ref{e2.1-vas}) 
при $u(t,x)\hm=u_t^*$.\linebreak 
Через~$j(I)$ обозначим игрока-нарушителя программы~$u_t^*$, 
идентифицируемого всеми игроками $i\not= j$ согласно~Д3. 
    
    \smallskip
    
    \noindent
    \textbf{Теорема~1.}\ \textit{Пусть выполняются допущения}~Д1--Д3. 
\textit{Тогда в~позиционной игре многих лиц существует равновесие Нэша, равное}
    \begin{equation}
    \tilde{u}^{*i}=u_t^{*i}(x,I)=\begin{cases}
    u_t^*\,, &\ x=x_t^*\,;\\
    \tilde{u}_{t,x}^{\mathrm{н}i}(j(I))\,, &\ x\not= x_t^*\,,
        \end{cases}
    \label{e2.7-vas}
    \end{equation}
$t\in \{ t_0,\ldots, T-2\}$, \textit{для всех} $i\hm=1, 2,\ldots, n$.

\smallskip

\noindent
    Д\,о\,к\,а\,з\,а\,т\,е\,л\,ь\,с\,т\,в\,о\,.\ \ По условию теоремы 
соотношения~(\ref{e2.7-vas}) определяют допустимую ситуацию игры. Согласно 
лемме~1 траектория~$X^*$ стабильна относительно множества~$S(u^*)$. 
Придерживаясь программного управления~$u_t^*$, игроки получают выигрыши, 
равные $w^{*i}\hm= f^i(x^*_T)$, $i\hm=1, 2,\ldots , n$. Пусть теперь кто-то из них 
изменил свою стратегию и~вмес\-то~$\tilde{u}^{*j}$ с~момента времени~$t$, 
$t\hm\in \{t_0,\ldots\linebreak
\ldots , T\hm-2\}$, начал применять другую допустимую стратегию 
$\tilde{u}^j\hm\not= \tilde{u}^{*j}$. Зная позицию $(t+1, \tilde{x}_{t+1})\hm\in S$, 
все партнеры $i\not= j$ это обнаруживают и~согласно предположению~Д3 
идентифицируют нарушителя $j\hm=j(I)$. В~соответствии с~определением 
стратегий~$\tilde{u}^{*i}$ (см.~(\ref{e2.7-vas})) с~момента $t\hm+1$ игроки $i\not= 
j$ применяют контрстратегию~$\tilde{u}^{\mathrm{н}i}(j)$. Оценим выигрыш  
$j$-го игрока в~этой ситуации. Согласно предположению~Д1 и~последнему 
неравенству из доказательства леммы~1, примененному к~$\overline{u}\hm= u^*$, 
получаем:
   \begin{multline*}
    \tilde{w}^j =f^j(\tilde{x}_T)\leq M^i(T-1,\tilde{x}_{T-1})\leq{}\\
    {}\leq M^i(t,\tilde{x}_t) 
\leq \overline{M}^i(u^*)\leq f^j(x_T^*)=w^{*j}\,.
    \end{multline*}
     
    Осталось проанализировать случай, когда отход от программного 
управления~$u^*$ состоится в~предпоследний момент времени $T\hm-1$. Тогда 
разыгрывается игра~$\mathrm{Г}_{T-1}$, которая по предположению~Д3 имеет 
равновесие Нэша, совпадающее с~выбором управления~$\tilde{u}^*_{T-1}$. 
И~здесь $\tilde{w}^j\hm\leq \tilde{w}^{*j}$ по определению  
равновесия~(\ref{e2.3-vas}). Итак, доказано выполнение неравенств~(\ref{e2.3-vas}) 
при всех $j\hm= 1,2,\ldots ,n$ в~ситуации~(\ref{e2.7-vas}). Следовательно, она 
равновесна по Нэшу.
    
    \smallskip
    
    Применим теорему~1 к~следующей экономической модели, параметры 
которой, считаем, удовле\-тво\-ря\-ют условию:
    \begin{multline}
    \exp (C-1) +\exp (n(C-U))\leq \fr{1}{2}+\exp \fr{n}{2}\,,\\ U^i=U\,.
    \label{e2.8-vas}
    \end{multline}
     
     \noindent
    \textbf{Пример~1.}\ \textbf{Модель свободного сырьевого рынка}. 
    Имеются добывающие компании $i\hm=1,2,\ldots ,n$, которые в~моменты 
времени $t\hm= 0,1,\ldots, T\hm-1$ могут выходить на рынок с~предложением сырья. 
Объемы продаж~$u^i$ ограничены количеством добываемых ресурсов~$U^i$, 
$U^i\hm>1$. Реализация сырья происходит по цене~$c(u)$, зависящей от 
предложения товара $u\hm= \sum\nolimits_i u^i$ в~момент торгов. Каждая компания 
заинтересована в~максимизации своей суммарной прибыли. 
    
    Пусть $x_t^i$~--- прибыль компании $i\hm=1,2,\ldots, n$, извлеченная 
к~моменту времени $t\hm= 0,1,\ldots, T\hm-1$. Тогда эти величины изменяются 
согласно разностным уравнениям:
    $$
    x^i_{t+1}=x_t^i+c(u_t) u_t^i\,,\enskip
    x_0^i=0\,,\enskip 
    t=0,1,\ldots, T-1\,.
    $$
Каждая компания стремится максимизировать общую прибыль $f^i(\tilde{u})\hm= 
x^i_T$. Это игрок, вы\-би\-ра\-ющий свои допустимые позиционные стратегии 
$\tilde{u}^i\hm= u_t^i(x,I)$, $0\hm\leq u_t^i\hm\leq U^i$, где $x\hm=(x^1,\ldots , x^n)$. 
Пусть зависимость цены товара от предложения равна
$$
c(u)= \begin{cases} 1\,, &\ 0\leq u\leq v_0\,;\\
\exp \left(-n+v_0\right)\,, &\ u>v_0\,.
\end{cases}
$$
Объем предложения $u\hm\leq v_0$ характеризует сбалансированность спроса 
и~предложения на рынке, обеспечивающую стабильную цену сырья, равную 
единице. Будем считать, что константа $C\hm= v_0/n$ удовлетворяет неравенству 
$1/2\hm\leq C\hm<1$.
    
    Равновесие Нэша построим, исходя из результатов игроков, равных 
    $$
    x_T^{*i}\equiv \fr{v_0}{n}\left(T-1\right) +\exp\left( -n+v_0\right)\,,
    $$
достижимых на траектории системы, управляемой программой вида
$(i=1,2,\ldots, n)$:
$$
u_t^{*i}=\begin{cases}
\fr{v_0}{n}\,, &\ t=0,1,\ldots ,T-2\,;\\
1\,, &\ t=T-1\,.
\end{cases}
$$
В любой момент времени $t\hm=1,2,\ldots, T-1$ всякий игрок может самостоятельно 
обнаружить факт нарушения общего плана действий. Если $x_t\not= x_t^*$, то 
указанное событие случилось при $\tau\hm= t\hm-1$. За это следует наказать 
нарушителя~$j$. Начиная с~момента времени $\tau\hm=t$ участники игры $i\not= j$ 
будут использовать свои позиционные стратегии наказания, имеющие вид 
$u_\tau^{\mathrm{н}i}(x;j)\hm\equiv U^i$, $x\not= x^*$. Последние одинаковы для 
всех $j\not= i$, поэтому даже не нужно идентифицировать нарушителя! 
В~статической игре~$\mathrm{Г}_{T-1}$ имеется равновесие Нэша,  
равное $u_{T-1}^*\hm=1$. 
    
    Выполнены все условия теоремы~2.1, так что равновесие~$\tilde{u}^*$ 
в~позиционных стратегиях имеет вид~(\ref{e2.7-vas}). Справедливость сказанного 
обусловлена действенностью наказания, уменьшающего <<оговоренный>> 
выигрыш игрока~$x_T^{*j}$. Это происходит при соотношении~(\ref{e2.8-vas}) 
между параметрами модели. Сравним найденное позиционное решение игровой 
задачи с~программным равновесием $(\forall\ i) u_t^{\mathrm{р}i}\hm\equiv 1$, дающим 
результат игры $x_T^{\mathrm{р}i}\hm= T\exp (-n+v_0)$. Ввиду справедливости неравенств 
$(\forall\ i) x^{\mathrm{р}i}_T\hm< x_T^{*i}$, ситуация~$u^{\mathrm{р}}$ не эффективна по Парето. 
    
    \section{Разрешающее правило}

    Предположение Д2 излишне ($I\hm=\emptyset$) для игры двух лиц и~для игр 
специального вида, в~которых система~(\ref{e2.1-vas}) распадается на подсистемы: 
\begin{multline*}
    x^i_{t+1}=\overline{g}^i(t,x_t,\tilde{u}^{i})\,,\\
     i=1,2,\ldots, n\,,\enskip t=t_0, t_0+1,\ldots,  T-1\,.
\end{multline*}
Каждому игроку достаточно обнаружить выполнение неравенства $x_t^j\not= 
x_t^{*j}$ и~приступить к~наказанию $j$-го игрока. 
    
    В общем случае придется увеличить информированность игроков. 
Целесообразно расширить фазовое пространство и~вместо исходного многошагового 
уравнения~(\ref{e2.1-vas}) перейти к~управляемой системе, построенной с~помощью 
программы~$u_t^*$: 
\begin{multline}
z^i_{t+1}=g\left(t,z_t^i, u_t^*\vert \tilde{u}^i\right)\,,\\
i=1,2,\ldots, n\,,\enskip t=t_0, t_0+1,\ldots , T-1\,.
\label{e3.1-vas}
\end{multline}
    
    В полученной модели все действия игроков полностью контролируются. Они 
выбирают свои до\-пус\-ти\-мые стратегии в~классе отображений $\tilde{u}^i:\ (t,z)\hm\to 
u^i(t,z)$, так что $I\hm= \{(t,z)\}\not=\emptyset$. Нарушитель $j\hm=j(I)$ 
соглашения~$u_t^*$ идентифицируется с~по\-мощью усло\-вия $x_t^{*j}\not= z_t^j$. 
Если все игроки придерживаются выбора~$u_t^*$, то траектории движения всех 
сис\-тем~(\ref{e3.1-vas}), $i\hm=1,2,\ldots, n$, и~(\ref{e2.1-vas}) совпадают. При 
нарушении программы~$u_t^{*j}$ движение только одной из  
под\-сис\-тем~(\ref{e3.1-vas}) совпадает с~движением исходной сис\-темы.
    
    \textbf{Пример~2.} \textbf{Позиционные игры с~влиятельными игроками}.
    Пусть в~игре имеется~$n_1$ игроков, каждый из которых $k\hm=1,2,\ldots,n_1$ 
оказывает влияние на группу игроков~$L(k)$, непосредственно воздействуя на 
управляемые системы вида:
    \begin{multline}
    x^l_{t+1}=\overline{g}^l\left( t,x_t^l, \tilde{u}^l, \tilde{u}^k\right)\,,\\
    l\in L(k)\,,\enskip t=t_0, t_0+1,\ldots, T-1\,.
    \label{e3.2-vas}
    \end{multline}
У каждого игрока $k\hm=1,2,\ldots, n_1$ имеется свой объект управления
\begin{equation}
x^k_{t+1}=\overline{\overline{g}}^k\left( t,x_t,\tilde{u},\tilde{v}\right)\,,
\label{e3.3-vas}
\end{equation}
где
$$
\tilde{u}=\left( u^k, k=1,2,\ldots, n_1\right)\,;\enskip
\tilde{v}=\left( \tilde{u}^l, l\in L(k)\right)\,,
$$
с состояниями $x\hm= ((x^k, k=1,2,\ldots, n_1), (x^l, l\hm\in L(k)))$, зависящими от 
управляющих воздействий <<подчиненных>> игроков $l\hm\in L(k)$. 
    
    В игровой задаче~(\ref{e3.2-vas}), (\ref{e3.3-vas}) с~платежными функциями 
игроков~(\ref{e2.2-vas}) и~с~принципом оп\-ти\-маль\-ности~(\ref{e2.3-vas}) 
выделено~$n_1$ иерархических структур~\cite{2-vas}. Для реализации ситуации 
равновесия из теоремы~1 можно воспользоваться следующим разрешающим 
правилом, предполагающим наличие у~игроков лишь \textit{частичного} знания 
позиций игры. Так, каждый влиятельный игрок $k\hm= 1,2,\ldots, n_1$ должен иметь 
информацию о~текущих состояниях сис\-тем~(\ref{e3.3-vas}) всех влиятельных 
партнеров и~сис\-тем~(\ref{e3.2-vas}) всех <<подчиненных>> ему игроков $l\hm\in 
L(k)$. Все игроки $l\hm\in L(k)$ должны знать текущие состояния $(x^k, (x^l, l\in 
L(k)))$ своей иерархической под\-сис\-те\-мы. Равновесие~(\ref{e2.7-vas}) 
поддерживается идентификацией нарушителя~$j(x_t)$ программы~$u_t^*$ согласно 
правилу:

\noindent
    $$
    j(x_t) =\begin{cases}
    l, &\! \left(\exists_1 k\right) \left( \exists_1 l\in L(k)\right) x_t^l\not= x_t^{*l},\ 
x_t^k\not= x_t^{*k};\\
    k, &\! \left(\exists_1 k\right) \left( \exists_{\geq1} l\in L(k)\right) x_t^l\not= x_t^{*l},\ 
x_t^k\not= x_t^{*k}.
    \end{cases}
    $$
    
    При необходимости каждый влиятельный игрок $k\hm= 1,2,\ldots , n_1$ 
наказывает одного из подчиненных $l\hm\in L(k)$ либо участвует в~наказании 
одного из влиятельных участников конфликта $k^\prime\hm= j(x_t)$, $k^\prime\not= 
k$. Каждый подчиненный игрок $l\hm\in L(k)$ угрожает наказанием лишь игрокам 
из $k$-й иерархической подсистемы.
    
    \textbf{Пример~3.} \textbf{Позиционные игры с~различимыми воз-\linebreak действиями 
игроков на управляемую систему}. В~условиях теоремы~1 заменим 
допущение~Д2 сле\-ду\-ющим предположением. Для любых значений \linebreak
$u^k\hm\in 
U^k(t,x_t^*)\backslash \{u_t^{*k}\}$, $k\hm=i$, $j,t\hm\in T$,
     \begin{equation}
     g\left(t,x_t^*,\left.u_t^*\right\vert u^i\right) \not= g\left(t,x_t^*, \left.u_t^*\right\vert 
u^j\right)\,,\enskip i\not= j\,.
     \label{e3.4-vas}
     \end{equation}
     
     %\columnbreak
    
    Пусть один из игроков в~момент времени~$t$ не выполняет 
программу~$u_t^*$. Тогда система~(\ref{e2.1-vas}) может быть переведена лишь 
в~одну из позиций $(t,x)\hm\in \bigcup\nolimits_i D_t^i (u^*)$.  
Согласно~(\ref{e3.4-vas}),
$$
D_t^i\left(u^*\right)\cap D_t^j\left(u^*\right)=\emptyset,\enskip i\not= j\,.
$$
Это 
позволяет ввести отображение $I\hm= I_{t,x}:\ \{ 0,1,\ldots ,n\}\hm\to \{ 0,1,\ldots, 
n\}$, задающее информационное множество игрока в~зависимости от текущей 
позиции, в~которой пребывает система~(\ref{e2.1-vas}), и~от значения~$k$ этой 
функции, полученного в~предыдущей позиции: 
    \begin{align*}
    I_{t_0,x_0}(k)& \equiv 0\,;\\
    I_{t,x}(k)&=\begin{cases}
    0, &\ x=x_t^*\,;\\
    k, &\ x\not=x_t^*,\ k\not=0\,;\\
    j, &\ x\not=x_t^*,\ k=0,\ (t,x)\in D_t^j(u^*),
        \end{cases}\\
&\hspace*{50mm}t>t_0\,.
    \end{align*}
    
    Нарушитель соглашения~$u_t^*$ выделяется ненулевым номером~$I_{t,x}(k)$. 
Нулевое же значение этой функции отвечает случаю, когда все соблюдают 
договоренность, применяя программное управление~$u_t^*$. Итак, в~формуле 
равновесия~(\ref{e2.7-vas}) $j(I)\hm= I_{t,x}$.

\section{Заключение }

    По сравнению с~дифференциальными играми многошаговые динамические 
задачи допускают более простое исследование и~численное решение. В~теореме~1 
обосновано существование ситуации равновесия Нэша в~позиционной игре, 
в~которой угроза наказания реализуется при необходимости с~некоторым 
запаздыванием. Приведены примеры разрешающих правил идентификации 
нарушителя совместного программного поведения. Выделены специальные классы 
игр, в~которых это правило не предполагает сбора дополнительной информации 
о~ходе игры.


{\small\frenchspacing
 {%\baselineskip=10.8pt
 \addcontentsline{toc}{section}{References}
 \begin{thebibliography}{99}
 \bibitem{2-vas} %1
\Au{Моисеев Н.\,Н.} Элементы теории оптимальных сис\-тем.~--- М.: Наука, 1975. 
527~с.
\bibitem{1-vas} %2
\Au{Гермейер Ю.\,Б.} Игры с~непротивоположными интересами.~--- М.: Наука, 1976. 
326~с.

\bibitem{3-vas}
\Au{Кононенко А.\,Ф.} Структура оптимальной стратегии в~управляемых 
динамических сис\-те\-мах~// Ж.~вычисл. мат. мат. 
физ., 1980. Т.~20. №\,5. С.~1105--1116.
\bibitem{4-vas}
\Au{Васильев Н.\,С.} Коалиционно устойчивые эффективные равновесия в~моделях 
коллективного поведения с~обменом информацией~// Информатика и~её 
применения, 2015. Т.~9. Вып.~2. С.~2--13.
\bibitem{5-vas}
\Au{Айзекс Р.} Дифференциальные игры~/
Пер. с~англ.~--- М.: Мир, 1967. 480~с.
(\Au{Isaacs~R.}
Differntial games.~--- New York, NY, USA: John Wiley and Sons, 1965. 416~p.)
\bibitem{6-vas}
\Au{Понтрягин~Л.\,С.} Линейные дифференциальные игры преследования~// 
Мат. сб., 1980. Т.~112(154). Вып.~3(7). С.~307--330.
\bibitem{7-vas}
\Au{Никольский~М.\,С.} Первый метод Л.\,С.~Понтрягина в~дифференциальных 
играх.~--- М.: МГУ, 1984. 64~с.
\bibitem{8-vas}
\Au{Красовский~Н.\,Н., Субботин~А.\,И.} Позиционные дифференциальные  
игры.~--- М.: Наука, 1974. 458~с.
\bibitem{9-vas}
\Au{Кононенко~А.\,Ф.} О~многошаговых конфликтах с~обменом информацией~// 
Ж.~вычисл. мат. мат. физ., 1977. Т.~17. №\,4. 
С.~922--931.

\bibitem{13-vas} %10
\Au{Чистяков~Ю.\,Е.} Задача о ситуациях равновесия по Нэшу в~игре многих лиц 
с~памятью~// Прикладная математика и~механика, 1987. №\,2. С.~201--214.

\bibitem{12-vas} %11
\Au{Кононенко~А.\,Ф., Чистяков~Ю.\,Е.} О~равновесных позиционных стратегиях 
в~дифференциальных играх многих лиц~// ДАН СССР, 1988. Т.~299. №\,5.  
С.~1053--1056.

\bibitem{11-vas} %12
\Au{Горелик~В.\,А., Горелов~М.\,А., Кононенко~А.\,Ф.} Анализ конфликтных 
ситуаций в~сис\-те\-мах управ\-ле\-ния.~--- М.: Радио и~связь, 1991. 286~с. 
\bibitem{10-vas} %13
\Au{Горелов~М.\,А., Кононенко~А.\,Ф.} Динамические модели конфликтов~II. 
Равновесия~// Автоматика и~телемеханика, 2014. Т.~75. №\,12. С.~56--77.



\bibitem{14-vas}
\Au{Жуковский В.\,И.} Введение в~дифференциальные игры при неопределенности. 
Равновесие по Нэшу.~--- М.: URSS, 2010. 168~с.
\bibitem{15-vas}
\Au{Петросян~Л.\,А.} Теория игр.~--- СПб: БХВ-Петербург, 2012. 432~с.

 \end{thebibliography}

 }
 }

\end{multicols}

\vspace*{-3pt}

\hfill{\small\textit{Поступила в~редакцию 22.02.17}}

\vspace*{8pt}

%\newpage

%\vspace*{-24pt}

\hrule

\vspace*{2pt}

\hrule

%\vspace*{8pt}


\def\tit{PARTICIPANTS' INFORMATION AWARENESS 
AND~EXISTENCE OF~EQUILIBRIUM IN~POSITIONAL ITERATION GAMES\\ OF~MANY PLAYERS}

\def\titkol{Participants' information awareness and existence of equilibrium in positional iteration games of many players}

\def\aut{N.\,S.~Vasilyev}

\def\autkol{N.\,S.~Vasilyev}

\titel{\tit}{\aut}{\autkol}{\titkol}

\vspace*{-9pt}


\noindent
N.\,E.~Bauman Moscow State Technical University, 5 Baumanskaya 2nd Str.,Moscow 
105005, Russian Federation



\def\leftfootline{\small{\textbf{\thepage}
\hfill INFORMATIKA I EE PRIMENENIYA~--- INFORMATICS AND
APPLICATIONS\ \ \ 2017\ \ \ volume~11\ \ \ issue\ 2}
}%
 \def\rightfootline{\small{INFORMATIKA I EE PRIMENENIYA~---
INFORMATICS AND APPLICATIONS\ \ \ 2017\ \ \ volume~11\ \ \ issue\ 2
\hfill \textbf{\thepage}}}

\vspace*{3pt}

     
    \Abste{In positional games, dynamical decision-making models are studied 
    for the situation when there is a~conflict of interests and participants 
    know the current position of the game. Each player is able to control the 
    dynamical system partially. The control strategy chosen by a~player is 
    a~function defined on the system's phase space. Players check the system's 
    movement and obtain an implicit idea about strategies applied by their partners. 
    The principle of players' rational behavior consists in trying to achieve the 
    situation of Nash equilibrium. It is proved that an equilibrium can be reached 
    as a~result of collective efforts to choose the system's general program control. 
    Stability of the solution is reached by using the threat of punishment to those 
    who refuse to fulfill the program. Positions control and some additional 
    information give players the possibility to identify the guilty player. 
    Then, after a~delay,  he/she is punished by all other players. The theorem 
    of existence of an equilibrium is applied to economic and mathematical model.}
    
    \KWE{system; differential game; positional iteration game; 
    program control; positional strategy; counter strategy; punishment strategy; 
    guaranty strategy; Nash equilibrium situation; Pareto effectiveness}

\DOI{10.14357/19922264170205} 

%\vspace*{-18pt}

%\Ack
%\noindent




%\vspace*{3pt}

  \begin{multicols}{2}

\renewcommand{\bibname}{\protect\rmfamily References}
%\renewcommand{\bibname}{\large\protect\rm References}

{\small\frenchspacing
 {%\baselineskip=10.8pt
 \addcontentsline{toc}{section}{References}
 \begin{thebibliography}{99}
    

\bibitem{2-vas-1}
\Aue{Moiseev, N.\,N.} 1975. \textit{Elementy teorii optimal'nykh sistem} [Elements of 
the optimal systems theory]. Moscow: Nauka. 527~p.

\bibitem{1-vas-1}
\Aue{Germeyer, Yu.\,B.} 1976. \textit{Igry s~neprotivopolozhnymi interesami} 
[Games with nonantogonistic interests]. Moscow: Nauka. 326~p.
\bibitem{3-vas-1}
\Aue{Kononenko, A.\,F.} 1980. Struktura optimal'noy strategii v~upravlyaemykh 
dinamicheskikh sistemakh [Optimal strategy structure in control dynamic systems].  
\textit{Zh.~Vyichislitel'noy matematiki i~matematicheskoy fiziki} [Comput. Math. 
Math. Phys.] 20(5):1105--1116. 
\bibitem{4-vas-1}
\Aue{Vasilyev, N.\,S.} 2014. Koalitsionno ustoychivye effektivnye ravnovesiya 
v~modelyakh kollektivnogo povedeniya s~obmenom informatsiey [Coalitionally stable 
effective equilibria in collective behavior models with information exchange].  
\textit{Informatika i~ee Primeneniya~--- Inform. Appl.} 8(2):2--13. 
\bibitem{5-vas-1}
\Aue{Isaacs, R.} 1965. \textit{Differential games}. New York, NY: 
John Wiley and Sons. 416~p.
\bibitem{6-vas-1}
\Aue{Pontryagin, L.\,S.} 1980. Lineynye differentsial'nye igry presledovaniya [Linear 
differential games of pursuit]. \textit{Mat. sb.} [Mathematical Collection] 
112(3):307--330.
\bibitem{7-vas-1}
\Aue{Nikol'skiy, M.\,S.} 1984. Pervyy metod L.\,S.~Pontryagina v~differentsial'nykh 
igrakh [L.\,S.~Pontryagin's first method in differential games]. Moscow: MGU. 64~p. 
\bibitem{8-vas-1}
\Aue{Krasovskiy, N.\,N., and A.\,I.~Subbotin.} 1974. \textit{Pozitsionnye 
differentsial'nye igry} [Positional differential games]. Moscow: Nauka. 458~p.
\bibitem{9-vas-1}
\Aue{Kononenko, A.\,F.} 1977. O~mnogoshagovykh konfliktakh s~obmenom 
informatsiey [On many steps conflicts with information exchange]. \textit{Zh. 
Vychislitel'noy matematiki i~matematicheskoy fiziki} [Comput. Math. Math. Phys.] 
17(4):922--931.

\bibitem{13-vas-1} %10
\Aue{Chistyakov, Yu.\,E.} 1987. Zadacha o~situatsiyakh ravnovesiya po Neshu v~igre 
mnogikh lits s~pamyat'yu [Nash's equilibrium situation problem in the game of many 
players with memory]. \textit{Prikladnaya matematika i~mekhanika} [Appl. 
Math. Mech.] 2:201--214.

\bibitem{12-vas-1} %11
\Aue{Kononenko, A.\,F., and Yu.\,E.~Chistyakov}. 1988. O~ravnovesnykh 
pozitsionnykh stpategiyakh v~diffepentsial'nykh igpakh mnogikh lits [On equilibrium 
positional strategies in many players' differential games]. \textit{Dokl. USSR
Akad. Sci.} 299(5):1053--1056.

\bibitem{11-vas-1} %12
\Aue{Gorelik, V.\,A., Gorelov~M.\,A., and A.\,F.~Kononenko.} 1991. \textit{Analiz 
konfliktnykh situatsiy v~sistemakh upravleniya} [Analysis of conflict situations in 
control systems]. Moscow: Radio i~svyaz. 286~p.
\bibitem{10-vas-1} %13
\Aue{Gorelov, M.\,A., and A.\,F.~Kononenko}. 2014. Dynamic models of conflicts.
II.~Equilibria. 
\textit{Automat. Rem. Contr.} 75(12):2135--2151.

\bibitem{14-vas-1}
\Aue{Zhukovskiy, V.\,I.} 2010. \textit{Vvedenie v~differentsial'nye igry pri 
neopredelennosti. Ravnovesiya po Neshu} [Introduction in differential games under 
uncertainty. Nash's equilibria]. Moscow: URSS. 168~p.
\bibitem{15-vas-1}
\Aue{Petrosyan, L.\,A.} 2012. \textit{Teoriya igr} [Theory of games]. St.\ Peterburg: 
BHV-Peterburg. 432~p.
\end{thebibliography}

 }
 }

\end{multicols}

\vspace*{-3pt}

\hfill{\small\textit{Received February 22, 2017}}
    
    
    \Contrl
    
    \noindent
\textbf{Vasilyev Nikolai S.} (b.\ 1952)~--- Doctor of Science in physics and mathematics, professor, 
N.\,E.~Bauman Moscow State Technical University, 5~Baumanskaya 2nd  Str.,Moscow 105005, Russian 
Federation; \mbox{nik8519@yandex.ru}
    
\label{end\stat}


\renewcommand{\bibname}{\protect\rm Литература} 