\def\stat{kon-raz}

\def\tit{УПРАВЛЕНИЕ СЛУЧАЙНЫМ БЛУЖДАНИЕМ С~ЭТАЛОННЫМ СТАЦИОНАРНЫМ 
РАСПРЕДЕЛЕНИЕМ$^*$}

\def\titkol{Управление случайным блужданием с~эталонным стационарным 
распределением}

\def\aut{М.\,Г.~Коновалов$^1$, Р.\,В.~Разумчик$^2$}

\def\autkol{М.\,Г.~Коновалов, Р.\,В.~Разумчик}

\titel{\tit}{\aut}{\autkol}{\titkol}

\index{Коновалов М.\,Г.}
\index{Разумчик Р.\,В.}
\index{Konovalov M.\,G.}
\index{Razumchik R.\,V.}




{\renewcommand{\thefootnote}{\fnsymbol{footnote}} \footnotetext[1]
{Исследование выполнено при финансовой поддержке РФФИ в рамках
научного проекта  №\,18-07-00692.}}


\renewcommand{\thefootnote}{\arabic{footnote}}
\footnotetext[1]{Институт проблем информатики Федерального исследовательского центра 
<<Информатика 
и~управление>> Российской академии наук, \mbox{mkonovalov@ipiran.ru}}
\footnotetext[2]{Институт проблем информатики Федерального исследовательского центра 
<<Информатика и~управление>> Российской академии наук; Российский университет 
дружбы народов, \mbox{rrazumchik@ipiran.ru}}

%\vspace*{8pt}

 

\Abst{Рассматривается случайное блуждание на отрезке, до\-пус\-ка\-ющее управ\-ле\-ние в~форме 
выбора направления для очередного шага. Задано множество стратегий управления, 
параметризованных конечномерными векторами. Требуется найти из этого множества такую 
стратегию, при которой плотность стационарного распределения марковской цепи, описывающей 
блуждание, максимально приближена к~заданной эталонной плот\-ности распределения. Постановка 
задачи отличается от классической схемы марковского процесса принятия решений тем, что 
отсутствует одношаговый доход. Содержательная трактовка задачи появляется в~психологии, 
робототехнике, генетике. Предложен квазиградиентный алгоритм определения оптимальных 
значений па\-ра\-мет\-ров, основанный на оценках част\-ных производных целевой функции по 
наблюдениям за фазовой траекторией. Приведены чис\-лен\-ные результаты работы алгоритма 
в~примерах с~различными классами стратегий и~различными эталонными плотностями 
распределения.}

\KW{управление марковской цепью с~непрерывным множеством состояний; квазиградиентные 
алгоритмы; оценки производных по наблюдениям}

\DOI{10.14357/19922264180301}
  
\vspace*{3pt}


\vskip 10pt plus 9pt minus 6pt

\thispagestyle{headings}

\begin{multicols}{2}

\label{st\stat}

  \section{Введение}
  
  \vspace*{-3pt}
  
  В ряде прикладных областей возникает следующая модель управ\-ля\-емо\-го 
случайного блуж\-да\-ния. Пусть $n\hm=0, 1, 2,\ldots$  и~пусть $x_n\hm\in 
[0,1]$~--- положение блуж\-да\-юще\-го объекта в~момент~$n$. В~этом 
положении выбирается на\-прав\-ле\-ние движения, т.\,е.\ один из двух до\-ступ\-ных 
сег\-мен\-тов $[0, x_n]$ или $[x_n,1]$ соответственно с~вероятностями~$s(x_n)$ 
и~$1\hm- s(x_n)$. Затем, если был выбран отрезок~$[0,x_n]$, объект 
переходит в~точку $f(x_n)\hm\in [0,x_n]$, иначе~--- в~точку $g(x_n)\hm\in 
[x_n,1]$. Функции~$f(x_n)$ и~$g(x_n)$ могут быть как 
детерминированными, так и~рандомизированными, и~их вид обуслов\-лен 
спецификой задачи. Очевидно, процесс~$x_n$ является управ\-ля\-емой 
марковской цепью с~множеством со\-сто\-яний~$[0,1]$.
  
  По-видимому, впервые подобные цепи Маркова начали появляться в~связи с~математическим моделированием процессов, обладающих или наделяемых 
свойствами обуча\-емости. В~этой связи надо отметить основополагающую 
работу~[1], где для модели предлагается интерпретация из об\-ласти 
психологии: если предыдущий результат действия испытуемого есть~$x_n$, 
то $f(x_n)$ и~$g(x_n)$~--- результат сле\-ду\-юще\-го действия, вы\-би\-ра\-емо\-го из 
двух альтернатив с~вероятностями~$s(x_n)$ и~$1\hm-s(x_n)$.
  
  В~другой интерпретации~[2] значения~$x_n$ могут рас\-смат\-ри\-вать\-ся как 
текущий уровень интеллекта\linebreak
 испытуемого, значения~$g(x_n)$~--- как 
уровень интеллекта испытуемого после неправильного (правильного) ответа 
на очередной вопрос, а~$s(x_n)$~--- вероятность неправильного ответа при 
условии, что\linebreak текущий уровень интеллекта равен~$x_n$. Задачей является 
определение стационарного уровня интеллекта.
  
  В работах~[3, 4] описан пример из робототехники, в~котором изменение 
каждой из координат робота при механическом перемещении по двумерной 
области моделируется в~точ\-ности по схеме, описанной в~первом абзаце. 
Другие примеры применения рас\-смат\-ри\-ва\-емой модели случайного 
блуж\-да\-ния, в~том чис\-ле в~об\-ласти генетики, мож\-но найти в~[5--7].
  
  Основной вопрос, который возникает в~работах, ис\-поль\-зу\-ющих 
марковскую цепь $x_n$,~--- это условия существования стационарного 
распределения и~на\-хож\-де\-ние для него замкнутых формул или чис\-лен\-но\-го 
алгоритма расчета~\cite{3-kr, 4-kr, 8-kr, 9-kr, 10-kr}. Например, в~\cite{8-kr} 
показано, что если $s(x_n)\hm=1/2$, $f(x_n)\hm=1/x_n$, $g(x_n)\hm=1/(1\hm-
x_n)$, то цепь Маркова~$\{x_n\}$\footnote[3]{Эта цепь известна в~зарубежной 
литературе как цепь Диа\-ко\-ни\-са--Фрид\-ма\-на~\cite{8-kr}.} имеет единственное 
стационарное абсолютно непрерывное распределение  
c~арк\-си\-нус-плот\-ностью. Если же $s(x_n)\hm=1/2$ заменить на 
$s(x_n)=s$, $0\hm<s\hm<1$, то стационарное распределение также 
единственно и~имеет бе\-та-плот\-ность. Как показано в~\cite{4-kr},  
бе\-та-плот\-ность является предельной так\-же и~в более общем случае, когда 
длина <<шага влево>> $f(x_n)$ и~длина <<шага вправо>> $g(x_n)$ имеют  
бе\-та-рас\-пре\-де\-ле\-ния с~разными па\-ра\-мет\-ра\-ми\footnote[1]{Но один из 
параметров бета-распределения должен равняться единице.}, а~$s(x_n)$~--- линейная 
функция. Для произвольной функции~$s(x_n)$ предложен метод чис\-лен\-но\-го 
расчета стационарного распределения, а~так\-же доказываются достаточные 
условия для эргодичности цепи. Отдельно стоит отметить пример 
из~\cite{11-kr}, который показывает, что как только ве\-ро\-ят\-ность выбора 
на\-прав\-ле\-ния начинает зависеть от текущего со\-сто\-яния цепи, ситуация 
заметно услож\-ня\-ет\-ся: если $f(x_n)\hm= x_n/3$ и~$g(x_n)\hm= (x_n\hm+2)/3$, 
то существует стратегия~$s(x_n)$, $0\hm< s(x_n)\hm<1$, при которой 
стационарное распределение не единственно. 
%
Известно несколько простых 
достаточных условий эргодичности цепи, которые с~тео\-ре\-ти\-че\-ской точ\-ки 
зрения, быть может, и~являются ограничительными, но с~практической точки 
зрения предлагают удобное средство для исключения патологических 
ситуаций~\cite{3-kr, 4-kr}. К~примеру, если длины шагов влево и~вправо 
распределены равномерно, то достаточным условием является 
одновременное выполнение двух неравенств: $s(0)\hm<1$ и~$s(1)\hm>0$.
  
  Даже беглый обзор результатов свидетельствует о~том, что задача анализа, 
т.\,е.\ задача нахождения стационарного распределения цепи при 
фиксированных $s(x_n)$, $f(x_n)$ и~$g(x_n)$, изучена хорошо. 
Примечательно, что стационарное распределение цепи редко удается 
выписать в~явном виде. В~связи с~этим возникает вопрос: как решать задачу 
синтеза, т.\,е.\ задачу на\-хож\-де\-ния таких $s(x_n)$, $f(x_n)$ и~$g(x_n)$, 
которые приводят к~заданному стационарному распределению цепи? 

Прикладным мотивом к~рас\-смот\-ре\-нию по\-доб\-ной <<обратной задачи>> 
может являться упомянутый пример из~\cite{3-kr}. Пусть задачей робота 
является регулярное посещение <<каж\-до\-го участка>> некоторой об\-ласти, 
причем определенные заранее заданные час\-ти этой об\-ласти должны 
посещаться чаще, чем другие. Какая стратегия обеспечивает решение 
поставленной задачи? В~\cite{3-kr} предложено решение этой задачи 
в~случае, когда известен явный вид стационарного распределения вектора 
и~когда целевая плот\-ность распределения вектора уни\-мо\-дальная.
{\looseness=1

}
  
  Эта статья посвящена задаче синтеза управ\-ле\-ния случайным 
блуж\-да\-ни\-ем~$x_n$. В~разд.~2 формулируется постановка задачи 
и~конкретизируются случайные функции $f(x_n)$ и~$g(x_n)$. Цель 
управления~--- из за-\linebreak\vspace*{-12pt}

\columnbreak

\noindent
данного множества стратегий, па\-ра\-мет\-ри\-зо\-ван\-ных 
конечномерными наборами чис\-ло\-вых па\-ра\-мет\-ров, найти ту, при которой 
стационарное распределение цепи максимально приближено к~заданному 
эталонному виду. В~разд.~3 и~4 конструируется алгоритм решения, 
использующий идеи стохастической градиентной оптимизации на 
марковских цепях. В~разд.~5 пред\-став\-ле\-ны результаты чис\-лен\-ных 
экспериментов, которые позволяют оценить эффективность предложенного 
алгоритма.
  
  В~заключение этой вводной час\-ти сделаем еще одно замечание. 
  
  На 
сформулированную задачу приближения эталонного стационарного 
распределения на отрезке~$[0,1]$ мож\-но посмотреть и~выйдя за рамки 
случайного блуж\-да\-ния~$x_n$. Плодотворным в~этом случае является подход, 
основанный на сис\-те\-мах <<итерационных случайных  
функций>>~\cite{8-kr, 12-kr, 13-kr}. При таком подходе при\-бли\-же\-ние 
осуществляется с~по\-мощью специально подобранного набора случайных 
функций, каждая из которых, выбираемая с~некоторым вероятностным 
распределением, отображает отрезок~$[0,1]$ в~себя. Однако для цепи~$x_n$ 
такой метод не позволяет получать удовлетворительное решение из-за 
необходимости оперировать <<небогатой>> сис\-те\-мой случайных функций 
(лишь $f(x_n)$ и~$g(x_n)$).
{\looseness=1

}
  
  \section{Постановка задачи}
  
  Пусть управляемое случайное блуж\-да\-ние задается рекуррентным 
соотношением
  \begin{multline*}
  x_{n+1}=x_n+\fr{\xi_n}{\theta}\left[ -\left( 1-\sigma_n\right) x_n +\sigma_n 
\left( 1-x_n\right)\right]\,,\\
  n=0,1,2,\ldots
\end{multline*}
    В этой формуле $\sigma_n$~--- взаимно услов\-но-не\-за\-ви\-си\-мые 
бинарные случайные величины, принимающие значения~0 
с~вероятностью~$s(x_n)$, и~1 с~вероятностью $1\hm- s(x_n)$; $s(x)$~--- 
функция на отрезке $[0,1]$, $0\hm< s(x)\hm<1$; $\xi_n$~---  взаимно 
независимые случайные величины, равномерно распределенные на 
отрезке~$[0, 1]$; $0\hm< \theta\hm\leq 1$; $x_0\hm\in 
[0,1]$.
  
  Значения величины~$\sigma_n$ трактуются как выбор на\-прав\-ле\-ния 
движения на каждом шаге: 0~--- сдвиг влево; 1~--- сдвиг вправо. 
Подлежащую выбору функцию~$s(x)$ будем называть правилом управ\-ле\-ния. 
Размер сдвига на каждом шаге определяется значением случайных 
величин~$\xi_n$ и~чис\-ло\-вым па\-ра\-мет\-ром~$\theta$. Начальное значение 
процесса не существенно, но известно.
  
  Последовательность правил $\{ s_n(x),\ n\geq 0\}$ будем называть 
стратегией управ\-ле\-ния, если правило~$s_n(x)$ определяет выбор 
направления на шаге~$n$. Если все правила одинаковы, т.\,е.\ если 
$s_n(x)\hm= s(x)$ для всех~$n$, то стратегию будем называть однородной. 
Таким образом, имеется взаимно однозначное соответствие меж\-ду 
правилами и~однородными стратегиями.
  
  При фиксированной однородной стратегии~$s(x)$ и~фиксированном 
па\-ра\-мет\-ре~$\theta$ по\-сле\-до\-ва\-тель\-ность~$x_n$ является марковской цепью 
с~множеством со\-сто\-яний~$[0,1]$ и~переходной ве\-ро\-ят\-ностью с~плот\-ностью
  \begin{equation}
  p(x,y) =sq_0(x,y)+(1-s)q_1(x,y)\,,\enskip 0<x<1\,,
  \label{e1-kr}
  \end{equation}
где
\begin{align*}
q_0(x,y) &= \begin{cases}
\fr{1}{x\theta}\,, &\ y\in [x-x\theta, x]\,;\\
0\,, &\  y\notin [x-x\theta,x]\,;
\end{cases}\\
q_1(x,y)&= \begin{cases}
\fr{1}{(1-x)\theta}\,, & y\in [x,x+(1-x)\theta]\,;\\
0\,, &\ y\not\in [x, x+(1-x)\theta]\,.
\end{cases}
\end{align*}
В граничных точках $x\hm=0$ и~$x\hm=1$ переходная плот\-ность 
доопределяется по не\-пре\-рыв\-ности. Относительно этой цепи предположим, 
что она при любых~$s(x)$ и~$\theta$ имеет абсолютно непрерывное 
стационарное распределение с~плот\-ностью $\pi(x)\hm= \pi(x,s(x),\theta)$, 
$x\hm\in [0,1]$.

  Пусть $S=\{s(x,a),\ a\in A\subset \mathbb{R}^k\}$~--- некоторое заданное 
множество однородных стратегий (или, что эквивалентно, множество правил 
управления), па\-ра\-мет\-ри\-зо\-ван\-ных векторами $a\hm= \left( a^{(1)}, \ldots , 
a^{(k)}\right)$ из множества $A\hm= A(S)$. Стационарное 
распределение~$\pi(x)$, соответствующее стратегии~$s(x,a)$, зависит от 
набора па\-ра\-мет\-ров~$a$. Пусть также $\rho(x)$~--- заданная плот\-ность 
вероятностного распределения на отрезке~$[0,1]$, которую будем называть 
эталонной плот\-ностью.
  
  Цель управления заключается в~том, чтобы \mbox{найти} такой набор па\-ра\-мет\-ров 
$a\hm\in a(S)$, который минимизирует функцию
  \begin{equation}
  W=W(A)=\int\limits_0^{1} \left( p(x)-\rho(x)\right)^2dx\,.
  \label{e2-kr}
  \end{equation}
  
  Таким образом, требуется отыскать стратегию из заданного 
параметризованного множества~$S$, при которой стационарное 
распределение цепи~$x_n$ наиболее при\-бли\-же\-но в~смыс\-ле 
критерия~(\ref{e2-kr}) к~заданному распределению~$\rho$.
  
  \section{Производная целевой функции по~параметру}
  
  Плотность переходной ве\-ро\-ят\-ности марковской цепи за~$n$~шагов 
определяется как
  \begin{align*}
  p^{(n)}(x,y)&=\int\limits_0^1 p^{(n-1)}(x,z) p(z,y)\,dz\,,\enskip
  n=1,2,\ldots;\\
  p^{(0)}(x,z)&\equiv p_0(z)\,.
  \end{align*}
  
  Стационарная плот\-ность распределения~$\pi(x)$ удовлетворяет 
сле\-ду\-ющим условиям:
  \begin{gather}
  \pi(y) = \int\limits_0^1 \pi(x) p(x,y)\,dx\,;\label{e3-kr}\\
  \int\limits_0^1\pi(x)\,dx=1\,.\label{e4-kr}
  \end{gather}
  
  Выберем произвольный параметр из набора~$a$ и~будем обозначать 
дифференцирование по этому па\-ра\-мет\-ру штри\-хом. Продифференцируем 
функцию~(\ref{e2-kr}) и~равенство~(\ref{e3-kr}) в~предположении, что 
производные~$p^\prime$ и~$\pi^\prime$ существуют:
  \begin{align}
  W^\prime &= 2\int\limits_0^1 \pi^\prime(x) \left( \pi(x)-
\rho(x)\right)\,dx\,;\notag %\label{e5-kr}
\\
  \pi^\prime(y)&= \int\limits_0^1 \pi^\prime(x) p(x,y)\,dx +\int\limits_0^1 \pi(x) 
p^\prime(x,y)\,dx\,.\label{e6-kr}
  \end{align}
  
  Воспользуемся номенклатурой тео\-рии обобщенных функций~\cite{14-kr}. 
Согласно этой тео\-рии существует взаимно однозначное соответствие между 
локально суммируемыми функциями~$r(x)$ на отрезке~$[0,1]$ 
и~регулярными обобщенными функциями~$r$ (линейными непрерывными 
функционалами на пространстве функций, непрерывных на отрезке~$[0, 1]$). 
Интегралу $\int \varphi(x) r(x)\,dx$ соответствует функциональное 
обозначение $\langle\varphi, r\rangle$ для результата действия 
функционала~$r$ на функцию~$\varphi$.
  
  Обозначая обобщенные функции переходной плот\-ности $p(x,y)$, а~так\-же 
ее производной $p^\prime(x,y)$ соответственно через~$p_y$ и~$p_y^\prime$ 
и~используя обозначение~$\delta_y$ для обобщенной плот\-ности, 
сосредоточенной в~точ\-ке~$x$, равенство~(\ref{e6-kr}) перепишем в~виде:
  \begin{equation}
  \langle \pi^\prime, \delta_y-p_y\rangle =\langle \pi, p_y^\prime\rangle\,.
  \label{e7-kr}
  \end{equation}
Здесь использовано характеристическое свойство $\delta$-функ\-ции, 
согласно которому $\langle\pi^\prime,\delta_y\rangle \hm= \pi^\prime(y)$.

  Для фиксированного $N\hm>0$ определим функционал
  $$
  P_x(N)=\delta_x +p_x+p_x^{(2)}+\cdots + p_x^{(N)}=\sum\limits^N_{n=0} 
p_x^{(n)}\,,
  $$
где $p_x^{(n)}$~--- обобщенные функции, со\-от\-вет\-ст\-ву\-ющие переходным 
плотностям $p^{(n)}(y,x)$, $n\hm> 0$, $p^{(0)}(x,y)\hm= \delta(x,y)$. 
(Заметим, что из приведенных определений следует, что $p_x^{(n+1)}\hm= 
p_x^{(n)} p_x\hm= p_x p_x^{(n)}$.) Применим этот функционал к~обеим 
час\-тям равенства~(\ref{e7-kr}), рас\-смат\-ри\-вая их как функции~$y$:
\begin{multline*}
\left\langle \left\langle \pi^\prime, \delta_y-p_y\right\rangle, 
P_x(N)\right\rangle=\pi^\prime(x) -{}\\
{}-\left\langle \pi^\prime, p_x\right\rangle +\left\langle \pi^\prime, 
p_x\right\rangle-\left\langle \pi^\prime, p_x^{(2)}\right\rangle+ \left\langle 
\pi^\prime, p_x^{(2)}\right\rangle -\cdots{}\\
{}\cdots - \left\langle \pi^\prime, 
p_x^{N+1)}\right\rangle= \left\langle\left\langle \pi, p_y^\prime\right\rangle, P_x(N)\right\rangle\,.
\end{multline*}
    Отсюда получаем, что
  $$
  \pi^\prime(x)=\left\langle\left\langle \pi, p_y^\prime\right\rangle, 
P_x(N)\right\rangle +\left\langle \pi^\prime, p_x^{(N+1)}\right\rangle\,.
  $$
  
  Перейдем к~пределу при $N\hm\to \infty$. Второе сла\-га\-емое в~правой час\-ти 
стремится к~нулю:
  \begin{multline*}
  \lim\limits_{N\to\infty} \left\langle \pi^\prime, p_x^{(N)}\right\rangle = 
\lim\limits_{N\to\infty} \int\limits_0^1 \pi^\prime(y) p^{(N)}(y,x)\,dy={}\\
{}=
  \int\limits_0^1 \pi^\prime(y) \pi(x)\,dy=\pi(x) \int\limits_0^1 \pi^\prime(y)\,dy=0
  \end{multline*}
(последнее равенство следует из дифференцирования равенства~(\ref{e4-kr})), поэтому
$$
\pi^\prime(x)=\left\langle \left\langle \pi, p_y^\prime\right\rangle, 
P_x\right\rangle\,,
$$
где
\begin{equation}
P_x=\sum\limits^\infty_{m=0} p_x^{(m)}\,.
\label{e8-kr}
\end{equation}
  
  После подстановки полученного выражения для~$\pi^\prime(x)$ 
в~формулу~(\ref{e4-kr}) получим:
  \begin{equation}
  W^\prime= 2\left\langle\left\langle\left\langle \pi, p_y^\prime\right\rangle, 
P_x\right\rangle, \gamma\right\rangle\,,
  \label{e9-kr}
  \end{equation}
где через~$\gamma$ обозначена обобщенная функция, со\-от\-вет\-ст\-ву\-ющая 
раз\-ности функций $\pi(x)\hm- \rho(x)$.

  Дифференцирование плот\-ности~(\ref{e1-kr}) по выбранному па\-ра\-мет\-ру 
приводит к~выражению:
  $$
  p^\prime(x,y)=s^\prime(x)\left( q_0(x,y) -q_1(x,y)\right)\,,
  $$
которое подставим в~формулу~(\ref{e9-kr}). Правая часть~(\ref{e9-kr}) 
представляет собой ряд из-за наличия функционала~(\ref{e8-kr}). Рассмотрим 
вначале произвольное сла\-га\-емое с~$m\hm> 0$. Поскольку входящие в~него 
обобщенные функции регулярны, то можно эквивалентным образом перейти 
к~обычным функциям и~записать такое слагаемое в~виде:
%\begin{multline*}
$\int\nolimits_0^1\!
\int\nolimits_0^1\!
 \int\nolimits_0^1
  \pi(x)s^\prime(x) 
\left( q_0(x,y)-\right.$\linebreak %{}\\
%{}-
$\left.-q_1(x,y)\right) p^{(m)} (y,z) \gamma(z)\,dzdydx$.
%\end{multline*}
  
  Эта запись интерпретируется следующим образом. Положим
  \begin{equation}
  G_m^{(i)} =\lim\limits_{t\to \infty} {\sf M}\left(  s^\prime(x_t) {\sf 
M}_{x_t}^{(i)}\left( \gamma_{x_{t+m}}\right)\right)\,,
  \label{e10-kr}
  \end{equation}
где ${\sf M}$~--- безусловное математическое ожидание; ${\sf 
M}_{x_t}^{(i)}$~--- условное математическое ожидание при условии, что 
в~состоянии~$x_t$ был совершен шаг влево ($i\hm=0$) или вправо 
($i\hm=1$); $\gamma\hm= \pi(\cdot)\hm- \rho(\cdot)$. Тогда рас\-смат\-ри\-ва\-емое 
слагаемое записывается как
$G_{m+1}^{(0)} \hm-G^{(1)}_{m+1}.$
  
  Для слагаемого с~$m\hm=0$, где фигурирует сингулярная обобщенная 
функция~$\delta_x$, непосредственная интегральная запись неправомочна, 
однако интерпретация этого слагаемого совершенно аналогична. Поэтому 
окончательно
  \begin{equation*}
  W^\prime= \sum\limits^\infty_{m=0} \left( G_m^{(0)} -G_m^{(1)}\right)\,.
 % \label{e11-kr}
  \end{equation*}
  
  \section{Алгоритм оптимизации параметров стратегии}
  
  Пусть по-прежнему $S$~--- заданное множество стратегий, 
параметризованных $k$-мер\-ны\-ми векторами из некоторого множества 
$A\hm= A(S)\hm\subset \mathbb{R}^k$, и~пусть $a_0\hm\in A$~--- некоторый 
начальный набор па\-ра\-мет\-ров. Будем корректировать значения параметров 
для приближенной минимизации целевой функции~(\ref{e2-kr}), используя 
стохастический вариант алгоритма проекции градиента. Обозначим 
через~$a_n$ вектор па\-ра\-мет\-ров, который применяется для выбора 
на\-прав\-ле\-ния движения на $n$-м шаге случайного блуж\-да\-ния~$x_n$. 
Алгоритм коррекции имеет вид:
  \begin{equation}
  a_{n+1}=\prod\limits_A \left( a_n-\alpha_n w_n\right)\,,
  \label{e12-kr}
  \end{equation}
где $\prod_A$~--- оператор проектирования на множество~$A$;  
$\alpha_n$~--- подходящим образом подобранная чис\-ло\-вая 
по\-сле\-до\-ва\-тель\-ность; $w_n$~--- по\-сле\-до\-ва\-тель\-ность случайных величин, 
являющихся оценками градиента~$\nabla W(a_n)$. Управление случайным 
блуж\-да\-ни\-ем осуществляется таким образом, что на $n$-м шаге выбор 
направления <<сдвига>> происходит с~по\-мощью правила $s_n(x)\hm= 
s(x,a_n)\hm\in S$. (Заметим, что стратегия $\mathbf{s}\hm= \{ s_n(x),\ n\geq 0\}$ 
является неоднородной и~множеству~$S$ не принадлежит.) Алгоритмы 
типа~(\ref{e12-kr}) являются широко распространенным инструментом 
оптимизации и~предметом изучения в~огромном чис\-ле пуб\-ли\-каций.

  Реализация схемы~(\ref{e12-kr}) предполагает по\-стро\-ение оценок 
градиента целевой функции. Согласно~(\ref{e10-kr}) каждая частная 
производная функции~$W$ пред\-став\-ля\-ет собой ряд из сла\-га\-емых слож\-ной 
структуры в~виде предела повторного математического ожидания. При этом 
внешнее усреднение происходит по предельному распределению цепи, 
которое соответствует текущему значению набора параметров, за\-да\-ющих 
стратегию. Основная проб\-ле\-ма заключается в~том, чтобы совместить 
пошаговое изменение па\-ра\-мет\-ров, соответствующее изменению 
индекса~$n$, с~необходимостью <<зафиксировать>> значения па\-ра\-мет\-ров 
для оценки предельного математического ожидания. Для решения 
используется прием <<оценивания с~забыванием>>, который поясним на 
примере оценки величины~(\ref{e10-kr}).
  
  Пусть для определенности $i\hm=0$ и~пусть $\tau_1, \tau_2, \ldots ,  
\tau_l,\ldots$~--- последовательные моменты выбора действия~0 (<<сдвиг влево>>) 
при случайном блуж\-да\-нии, управ\-ля\-емом согласно стратегии~$\mathbf{s}$. Выберем 
чис\-ло\-вую последовательность $\beta_l\uparrow 1$, положим $z_0\hm=0$, 
$b_0\hm=1$ и~зададим рекуррентные соотношения:
  $$
  z_{l+1}=\beta_l z_l +s^\prime_{\tau_l} \left( x_{\tau_l}\right) 
\gamma_{\tau_l+m}\,,\enskip
  b_{l+1}=\beta_l b_l+1\,.
  $$
  
  Оценкой величины $G_m^{(0)}$ на $n$-м шаге случайного блуж\-да\-ния 
является отношение $g_{m,n}^{(0)}\hm= z_l/b_l$. Аналогично строятся 
оценки $g_{m,n}^{(1)}\hm= z_l/b_l$ величин~$G_m^{(1)}$.
  
  <<Скользящие>> суммы были использованы в~\cite{15-kr} для построения 
оценок градиента предельного среднего дохода в~задаче управ\-ле\-ния 
дискретной марковской цепью с~доходами с~по\-мощью алгоритма 
типа~(\ref{e12-kr}). Там же было доказано, что для схо\-ди\-мости алгоритма 
необходимо выполнение дополнительных условий, в~част\-ности на 
последовательности~$\alpha_n$ и~$\beta_l$. В~данной работе исследование 
алгоритма ограничивается численными экспериментами.
  
  \section{Экспериментальный анализ}
  
  В этом разделе приведены результаты экспериментов с~управляемым 
случайным блужданием~$x_n$ с~коэффициентом~$\theta$, задающим 
максимальный размер сдвига на одном шаге, равным~0,5. Была выбрана 
упрощенная модификация управ\-ля\-юще\-го алгоритма~(\ref{e12-kr}), для 
которой последовательности~$\alpha_n$ и~$\beta_l$ суть константы: 
$\alpha_n\hm\equiv \alpha$, $\beta_l\hm\equiv \beta$. В~качестве оценки 
част\-ных производных целевой функции на $n$-м шаге было взято выражение 
$\sum\nolimits_{m=0}^M \left( g_{m,n}^{(0)} \hm- g_{m,n}^{(1)}\right)$. 
Константы~$\alpha$, $\beta$ и~$M$ варьировались в~диапазонах $[10^{-8}; 
10^{-7}]$, $[0{,}999; 0{,}9999]$ и~$\{3,4,\ldots , 10\}$ соответственно.
  
  Определим четыре множества правил управ\-ле\-ния, образованных 
многочленами степеней от~0 до~3:
  \begin{multline*}
  S_k=\left\{ s(x,a) =\max \left( 0,\min\left(1, \sum\limits^k_{i=0} a^{(i)} 
x^i\right)\right), \right.\\ \left.a^{(i)}\in \mathbb{R}
\vphantom{\left( 0,\min\left(1, \sum\limits^k_{i=0} a^{(i)} 
x^i\right)\right)}
\right\}\,,\enskip
  k=0,1,2,3.
  \end{multline*}
Стратегии из класса~$S_k$ па\-ра\-мет\-ри\-зу\-ют\-ся набором коэффициентов 
$a\hm= \left( a^{(0)}, \ldots , a^{(k)}\right)\hm\in A_k \hm= \mathbb{R}^{k+1}$.

  Кроме того, определим множество правил управ\-ле\-ния синусоидального 
типа:
  \begin{multline*}
  S_t={}\\
  \hspace*{-1.35pt}{}=\left\{ s(x,a)=\max \left( 0,\min \left( 1, a_0 \sin\left( 
a_1+a_2x\right)+a_3\right)\right);\right.\\
 \left.a_0, a_1, a_2, a_3 \in \mathbb{R}\right\}\,,
\end{multline*}
которые параметризуются набором коэффициентов $\left( a^{(0)}, \ldots , 
a^{(3)}\right) \hm\in A_t\hm= \mathbb{R}^4$.
  
  Для выбранного класса правил управ\-ле\-ния~$S$ и~установленной 
эталонной плот\-ности~$c(x)$ эксперимент заключался в~имитации траектории 
случайного блуж\-да\-ния~$x_n$, управ\-ля\-емо\-го согласно стратегии $\mathbf{s}=\{ 
s_n(x)=s(x, a_n)\hm\in S,\ n\hm\geq 0\}$. При этом последо\-ва\-тель\-ность~$a_n$ 
порождалась алгоритмом~(\ref{e12-kr}), в~котором $A\hm= A(S)$, 
а~последовательности~$\sigma_n$ и~$w_n$ определены в~начале текущего 
раздела. Продолжительность эксперимента составляла $N\sim 10^7$~тактов. 
Результатами эксперимента стали финальное правило управ\-ле\-ния~$s_N(x)$, 
а~так\-же оценки значений целевой функции~$W_N$ и~стационарной 
плотности $p_N(x)$, соответствующих финальному правилу~$s_N(x)$.
  
  \smallskip
  
  \noindent
  \textbf{Пример~1.} Эталонная плот\-ность $c(x)$~--- линейная. Она 
приведена в~табл.~1. В~этой же таб\-ли\-це указано начальное правило 
управления~$s_0(x)$, а~так\-же полученные экспериментально значение 
целевой функции~$W_0$ и~график плот\-ности стационарного распределения 
$p_0(x)$, соответствующие правилу управ\-ле\-ния~$s_0(x)$. Результаты 
эксперимента приведены в~табл.~2. Они показывают, что алгоритм 
существенно улучшает начальное правило как по значениям целевой 
функции, так и~по форме графика стационарной плот\-ности. Заметим, 
впрочем, что при\-бли\-же\-ние графика стационарной
%\noindent
 плотности к~эталонному 
виду не ставилось задачей управ\-ле\-ния. Это побочный эффект оптимизации 
по мет\-ри\-ке~(\ref{e2-kr}). Дополнительно следует отметить, что качество 
оптимизации повышается с~рос\-том степени многочленов~--- <<базисных>> 
правил. %\linebreak\vspace*{-12pt}

\pagebreak

\end{multicols}

\begin{table*}
{\small %tabl1
   \begin{center}
   \Caption{Исходные данные для эксперимента с~линейной эталонной плот\-ностью}
   \vspace*{2ex}
   
   \begin{tabular}{|c|c|c|c|}
   \hline
  $c(x)=2(1-x)$&$s_0(x)\equiv 0{,}5$&$W_0$&$p_0(x)$\\ 
\hline 
&&&\\[-9pt]
 \mbox{%
 \epsfxsize=29.947mm 
 \epsfbox{kon-1t-1.eps}
 }
&
 \mbox{%
 \epsfxsize=32.821mm 
 \epsfbox{kon-1t-2.eps}
 }
&\raisebox{30pt}[0pt][0pt]{0,547}&
 \mbox{%
 \epsfxsize=32.106mm 
 \epsfbox{kon-1t-3.eps}
 }
\\ 
\hline 
\end{tabular} 
\end{center} }
%\end{table*}
%\begin{table*}
{\small %tabl2
\begin{center}
\Caption{Результаты эксперимента с~линейной эталонной плот\-ностью}
\vspace*{2ex}

\tabcolsep=10pt
\begin{tabular}{|c|c|c|c|}
\hline
$S$ &  $s_N(x)$ & $W_N$ & $p_N(x)$\\
\hline
&&&\\[-6pt]
\raisebox{24pt}[0pt][0pt]{$S_0$} & \mbox{%
 \epsfxsize=32.659mm 
 \epsfbox{kon-2t-1.eps}
 } & \raisebox{24pt}[0pt][0pt]{0,033}&  
 \mbox{%
 \epsfxsize=32.306mm 
 \epsfbox{kon-2t-4.eps}
 }\\
 & $\mathrm{const}=0{,}693$ &&\\
 \hline
  &&&\\[-6pt]
\raisebox{24pt}[0pt][0pt]{$S_1$}&   \mbox{%
 \epsfxsize=32.165mm 
 \epsfbox{kon-2t-2.eps}
 } & \raisebox{24pt}[0pt][0pt]{0,024}& 
  \mbox{%
 \epsfxsize=32.554mm 
 \epsfbox{kon-2t-5.eps}
 }\\
 &$0{,}703-0{,}047x$&&\\
 \hline
 &&&\\[-6pt]
\raisebox{24pt}[0pt][0pt]{$S_2$}& \mbox{%
 \epsfxsize=33.598mm 
 \epsfbox{kon-2t-3.eps}
 }& \raisebox{24pt}[0pt][0pt]{0,013} & \mbox{%
 \epsfxsize=32.454mm 
 \epsfbox{kon-2t-6.eps}
 }\\
 &$0{,}712-0{,}049x-0{,}120x^2$&&\\
 \hline
%$\mathrm{const}=0{,}693$& $0{,}703-0{,}047x$& $0{,}712-0{,}049x-0{,}120x^2$\\
\end{tabular}
\end{center}}
\vspace*{-6pt}
\end{table*}

\begin{multicols}{2}


  
   

  %\smallskip
  
  \noindent
  \textbf{Пример~2.}\ Эталонная плот\-ность~--- квад\-ра\-тич\-ная: $c(x) = 
2{,}727(1{,}4x \hm -x^2)$. Начальное правило такое же, как в~примере~1. Ему 
соответствует значение целевой функции $W_0\hm= 0{,}163$. Результаты 
эксперимента приведены в~табл.~3. Качественные выводы относительно 
результатов такие же, как в~примере~1. Плот\-ность~$c(x)$ для наглядного 
сравнения с~результирующей плот\-ностью  изображена на рис.~1.


\begin{table*}\small %tabl3
\begin{center}
\Caption{Результаты эксперимента с~квадратичной эталонной плот\-ностью}
\vspace*{2ex}

\begin{tabular}{|c|c|c|c|}
\hline
 $S$&$s_N(x)$ & $W_N$ & $p_N(x)$\\
\hline
&&&\\[-6pt]
\raisebox{24pt}[0pt][0pt]{$S_0$} &\mbox{%
 \epsfxsize=32.636mm 
 \epsfbox{kon-3t-1.eps}
 }& \raisebox{24pt}[0pt][0pt]{0,042}& \mbox{%
 \epsfxsize=31.915mm 
 \epsfbox{kon-3t-5.eps}
 }\\
& $\mathrm{const}=0{,}400$ &&\\
 \hline
 &&&\\[-6pt]
\raisebox{24pt}[0pt][0pt]{ $S_1$} & \mbox{%
 \epsfxsize=33.911mm 
 \epsfbox{kon-3t-2.eps}
 }& \raisebox{24pt}[0pt][0pt]{0,023} &\mbox{%
 \epsfxsize=32.071mm 
 \epsfbox{kon-3t-6.eps}}\\
& $0{,}455-0{,}092x$&&\\
 \hline
  &&&\\[-6pt]
  \raisebox{24pt}[0pt][0pt]{$S_2$} &\mbox{%
 \epsfxsize=33.528mm 
 \epsfbox{kon-3t-3.eps}
 }& \raisebox{24pt}[0pt][0pt]{0,010}& \mbox{%
 \epsfxsize=31.967mm 
 \epsfbox{kon-3t-7.eps}}\\
 &$0{,}472-0{,}069x -0{,}081x^2$&&\\
 \hline
  &&&\\[-6pt]
 \raisebox{24pt}[0pt][0pt]{ $S_3$}&\mbox{%
 \epsfxsize=33.94mm 
 \epsfbox{kon-3t-4.eps}
 } & \raisebox{24pt}[0pt][0pt]{0,004}&\mbox{%
 \epsfxsize=31.912mm 
 \epsfbox{kon-3t-8.eps}}\\
& $0{,}482- 0{,}64x -0{,}053x^2-0{,}070x^3$&&\\
\hline
\end{tabular}
\end{center}
\end{table*}

\begin{figure*} %fig1
\vspace*{1pt}
\begin{minipage}[t]{80mm}
 \begin{center}
 \mbox{%
 \epsfxsize=76.687mm 
 \epsfbox{kon-1.eps}
 }
 \end{center}
\vspace*{-9pt}
\Caption{Квадратичная эталонная плот\-ность}
\end{minipage}
%\end{figure*}
\hfill
  %   \begin{figure*} %fig2
   \vspace*{1pt}
   \begin{minipage}[t]{80mm}
 \begin{center}
 \mbox{%
 \epsfxsize=76.687mm 
 \epsfbox{kon-2.eps}
 }
 \end{center}
\vspace*{-9pt}
   \Caption{Кубическая эталонная плот\-ность}
   \end{minipage}
   \end{figure*}
   
   
   %\smallskip
   
   \noindent
   \textbf{Пример~3.}\ Эталонная плот\-ность~--- кубическая: $c(x)\hm= 
1{,}622(0{,}8\hm+0{,}4x\hm- 2{,}5x^2\hm+ 1{,}8x^3)$ (рис.~2). 
Начальное правило такое же, как в~предыду\-щих примерах. Ему соответствует 
значение целевой функции $W_0\hm=0{,}284$. Результаты эксперимента 
приведены в~табл.~4. Как и~в предыду\-щих примерах, оптимизация на любом 
из множеств~$S_k$ дает существенный выигрыш в~целевой функции по 
сравнению с~начальным при\-бли\-же\-ни\-ем. Вновь качество оптимизации 
повышается с~ростом значения~$k$, т.\,е.\ с~увеличением мощ\-ности 
множества <<базисных>> правил. В~отличие от первых двух примеров, 
форма стационарной плот\-ности становится похожей на эталонную плотность 
только при использовании правил управ\-ле\-ния~--- многочленов третьей 
сте\-пени.

\begin{table*}\small %tabl4
\begin{center}
\Caption{Результаты эксперимента с~кубической эталонной плот\-ностью}
\vspace*{2ex}

\begin{tabular}{|c|c|c|c|}
\hline
$S$ & $s_N(x)$ &  $W_N$& $p_N(x)$\\
\hline
&&&\\[-6pt]
\raisebox{24pt}[0pt][0pt]{$S_0$} & \mbox{%
 \epsfxsize=32.271mm 
 \epsfbox{kon-4t-1.eps}
 }& \raisebox{24pt}[0pt][0pt]{0,042}& \mbox{%
 \epsfxsize=32.064mm 
 \epsfbox{kon-4t-5.eps}
 }\\
 &  $\mathrm{const}=0{,}602$ &&\\
 \hline
 &&&\\[-6pt]
\raisebox{24pt}[0pt][0pt]{ $S_1$} & \mbox{%
 \epsfxsize=32.907mm 
 \epsfbox{kon-4t-2.eps}
 } & \raisebox{24pt}[0pt][0pt]{0,023}& \mbox{%
 \epsfxsize=32.017mm 
 \epsfbox{kon-4t-6.eps}
 }\\
 &$0{,}616-0{,}086x$&&\\
 \hline
 &&&\\[-6pt]
  \raisebox{24pt}[0pt][0pt]{$S_2$} & \mbox{%
 \epsfxsize=32.816mm 
 \epsfbox{kon-4t-3.eps}
 }& \raisebox{24pt}[0pt][0pt]{0,010}&\mbox{%
 \epsfxsize=32.35mm 
 \epsfbox{kon-4t-7.eps}
 }\\
& $0{,}622-0{,}126x -0{,}070x^2$&&\\
\hline
 &&&\\[-6pt]
 \raisebox{24pt}[0pt][0pt]{ $S_3$ }&\mbox{%
 \epsfxsize=32.608mm 
 \epsfbox{kon-4t-4.eps}
 }&\raisebox{24pt}[0pt][0pt]{0,004}& \mbox{%
 \epsfxsize=32.306mm 
 \epsfbox{kon-4t-8.eps}
 }\\
& $0{,}702-0{,}189x-0{,}095x^2-0{,}233x^3$&&\\
\hline
\end{tabular}
\end{center}
\end{table*}

\begin{table*}\small  %tabl5
   \begin{center}
   \Caption{Исходные данные для эксперимента с~синусоидальной эталонной плот\-ностью}
   \vspace*{2ex}
   
   \begin{tabular}{|c|c|c|c|}
   \hline
   $c(x)=0{,}798\sin (0{,}2+12x)+0{,}977$ & $s_0(x)$ & $W_0$ & $p_0(x)$\\
   \hline
   &&&\\[-6pt]
 \raisebox{-36pt}[0pt][0pt]{ \mbox{%
 \epsfxsize=31.205mm 
 \epsfbox{kon-5t-1.eps}
 }}& \raisebox{30pt}[0pt][0pt]{ 
 \tabcolsep=0pt\begin{tabular}{c}для базовых классов $S_0$--$S_2$:\\
$s_0(x)\equiv 0{,}5$\end{tabular}}& \raisebox{30pt}[0pt][0pt]{ 0,533}& \mbox{%
 \epsfxsize=32.728mm 
 \epsfbox{kon-5t-3.eps}
 }\\
\cline{2-4}
& для базового класса $S_t$: &&\\
& \mbox{%
 \epsfxsize=31.411mm 
 \epsfbox{kon-5t-2.eps}
 }&  \raisebox{30pt}[0pt][0pt]{ 143,9} &\mbox{%
 \epsfxsize=32.46mm 
 \epsfbox{kon-5t-4.eps}
 }\\
& $s(x)=\min(1,\sin 10x +1)$ &&\\
\hline
\end{tabular}
\end{center}
\end{table*}
  
  %\smallskip
  
  \noindent
  \textbf{Пример~4.}\ Эталонная плот\-ность~--- синусоидальная. Исходные 
данные указаны в~табл.~5. Результаты приведены в~табл.~6. Этот пример 
особенно подчеркивает важ\-ность выбора множества <<базисных>> правил. 
<<Тригонометрическое>> множество~$S_t$, несмотря на плохое начальное 
правило, позволяет добиться гораздо лучшего при\-бли\-же\-ния к~эталонной 
плот\-ности, чем множество полиномиальных правил.
   
   
\vspace*{-14pt}

  
  \section{Заключение}
  
\vspace*{-5pt}
  
  Сформулирована задача поиска стратегии управ\-ле\-ния случайным 
блужданием с~целью мини-\linebreak\vspace*{-12pt}

\pagebreak

\end{multicols}

\begin{table*}\small %tabl6
\begin{center}
\Caption{Результаты эксперимента с~синусоидальной эталонной плот\-ностью}
\vspace*{2ex}

\begin{tabular}{|c|c|c|c|}
\hline
$S$ & $s_N(x)$ & $W$ & $p_N(x)$\\
\hline
&&&\\[-6pt]
 \raisebox{24pt}[0pt][0pt]{$S_0$} & \mbox{%
 \epsfxsize=33.723mm 
 \epsfbox{kon-6t-1.eps}
 }& \raisebox{24pt}[0pt][0pt]{0,505}& \mbox{%
 \epsfxsize=31.925mm 
 \epsfbox{kon-6t-6.eps}
 }\\
 &$\mathrm{const}=0{,}533$&&\\
 \hline
 &&&\\[-6pt]
  \raisebox{24pt}[0pt][0pt]{$S_1$} & \mbox{%
 \epsfxsize=33.505mm 
 \epsfbox{kon-6t-2.eps}
 }& \raisebox{24pt}[0pt][0pt]{0,425}& \mbox{%
 \epsfxsize=31.884mm 
 \epsfbox{kon-6t-7.eps}
 }\\
 &$0{,}602-0{,}165x$&&\\
 \hline
 &&&\\[-6pt]
  \raisebox{24pt}[0pt][0pt]{$S_2$} & \mbox{%
 \epsfxsize=32.728mm 
 \epsfbox{kon-6t-3.eps}
 } & \raisebox{24pt}[0pt][0pt]{0,372}& \mbox{%
 \epsfxsize=32.112mm 
 \epsfbox{kon-6t-8.eps}
 }\\
& $0{,}686-0{,}074x -0{,}162x^2$&&\\
 \hline
 &&&\\[-6pt]
   \raisebox{24pt}[0pt][0pt]{$S_3$} & \mbox{%
 \epsfxsize=32.13mm 
 \epsfbox{kon-6t-4.eps}
 }& \raisebox{24pt}[0pt][0pt]{0,321}& \mbox{%
 \epsfxsize=31.761mm 
 \epsfbox{kon-6t-9.eps}
 }\\
 &$0{,}724 -0{,}093x--  0{,}182x^2-0{,}226x^3$\\
 \hline
 &&&\\[-6pt]
      \raisebox{24pt}[0pt][0pt]{$S_{\mathrm{т}}$} &
\mbox{%
 \epsfxsize=32.332mm 
 \epsfbox{kon-6t-5.eps}
 }& \raisebox{24pt}[0pt][0pt]{0,152}&\mbox{%
 \epsfxsize=31.981mm 
 \epsfbox{kon-6t-10.eps}
 }\\
  & $0{,}288\sin (-0{,}355+9{,}531x) + 0{,}566$&&\\
\hline
\end{tabular}
\end{center}
\vspace*{3pt}
\end{table*}

\begin{multicols}{2}

\noindent
мизировать функцию, оценивающую отклонение 
плотности стационарного распределения от заданной эталонной плот\-ности. 
Задача относится к~теории управ\-ле\-ния марковскими цепями с~недискретным 
множеством со\-сто\-яний, однако отличается от классической постановки 
марковского процесса принятия решений тем, что отсутствует одношаговый 
доход. Оптимальная стратегия ищется в~заданном множестве стратегий, 
параметризованных конечномерными наборами чис\-ло\-вых па\-ра\-мет\-ров. 

Предложенное при\-бли\-жен\-ное чис\-лен\-ное решение пред\-став\-ля\-ет собой 
градиентный алгоритм коррекции па\-ра\-мет\-ров стратегии, причем оценки\linebreak 
производных целевой функции строятся по результатам наблюдения за 
имитируемой траекторией процесса. Проведен экспериментальный анализ 
алгоритма для ряда па\-ра\-мет\-ри\-зо\-ван\-ных классов стратегий и~эталонных 
плотностей, который показал хорошие результаты в~плане минимизации 
целевой функции. Основной вывод на основании работы за\-клю\-ча\-ет\-ся 
в~констатации эффективности градиентного подхода к~оптимизации на 
марковских цепях с~непрерывным множеством со\-сто\-яний. 

Пред\-став\-ля\-ют\-ся 
интересными сле\-ду\-ющие на\-прав\-ле\-ния дальнейших исследований:
  %\begin{itemize}
%\item  
теоретический анализ сходимости алгоритма;
%\item  
распространение и~обоснование метода на произвольные\linebreak 
распределения без предположения о~существовании плотностей;
  %\item  
  использование различных мет\-рик для оценки бли\-зости 
распределений;
  %\item  
  изучение связи между исходным множеством заданных стратегий 
и~точ\-ностью при\-бли\-же\-ния эталонного распределения.
 % \end{itemize}
  
{\small\frenchspacing
 {%\baselineskip=10.8pt
 \addcontentsline{toc}{section}{References}
 \begin{thebibliography}{99}
  \bibitem{1-kr}
  \Au{Karlin S.} Some random walks arising in learning models.~I~// 
  Pac. J.~Math., 1953. 
Vol.~3. No.\,4. P.~725--756.
  \bibitem{2-kr}
  \Au{Kaijser T.} On a~theorem of Karlin~// Acta Appl. Math., 1994. Vol.~34. P.~51--69.
  \bibitem{3-kr}
  \Au{Ramli~M.\,A., Leng~G.} The stationary probability density of a~class of bounded Markov 
processes~// Adv. Appl. Probab., 2010. Vol.~42. P.~986--993.
  \bibitem{4-kr}
  \Au{McKinlay S., Borovkov~K.} On explicit form of the stationary distributions for a~class of 
bounded Markov chains~// J.~Appl. Probab., 2016. Vol.~53. Iss.~1. P.~231--243. 
  \bibitem{5-kr}
  \Au{Li~C.} Human genetics.~--- New York, NY, USA: McGraw-Hill, 1961. 218~p.
 
  \bibitem{7-kr}
  \Au{DeGroot M.\,H., Rao~M.\,M.} Stochastic give-and-take~// J.~Math. Anal. Appl., 1963. 
Vol.~7. P.~489--498.

 \bibitem{6-kr} %7
  \Au{McKinlay~S.} A~characterization of transient random walks on stochastic matrices with 
Dirichlet distributed limits~// J.~Appl. Probab., 2014. Vol.~51. P.~542--555.

 \bibitem{10-kr} %8
  \Au{Peign$\acute{\mbox{e}}$~M.} Iterated function systems and spectral decomposition of 
the associated Markov operator~// Publications math$\acute{\mbox{e}}$matiques et 
informatique de Rennes, 1993. No.\,2. P.~1--28.

  \bibitem{8-kr} %9
  \Au{Diaconis P., Freedman~D.} Iterated random functions~// SIAM Rev., 1999. Vol.~41. 
Iss.~1. P.~45--76.
  \bibitem{9-kr} %10
  \Au{Ladjimi F., Peign$\acute{\mbox{e}}$~M.} Iterated function systems with place 
dependent probabilities and application to the Diaconis--Friedman's chain on~$[0,1]$. {\sf 
https://hal.archives-ouvertes.fr/LMPT/hal-01567392v1}.
 
  \bibitem{11-kr}
  \Au{Stenflo $\ddot{\mbox{O}}$.} A~note on a~theorem of Karlin~// Stat. Probabil. 
Lett., 2001. Vol.~54. Iss.~2. P.~183--187.
  \bibitem{12-kr}
  \Au{Jacquin A.} A~fractal theory of iterated Markov operators with applications to digital 
image coding.~--- Atlanta, GA, USA: Georgia Institute of Technology, 1989. Ph.D. Thesis.
  \bibitem{13-kr}
  \Au{Forte B., Vrscay~E.\,R.} Solving the inverse problem for measures using iterated 
function systems: A~new approach~// Adv. Appl. Probab., 1995. 
Vol.~27. Iss.~3. P.~800--820.
  \bibitem{14-kr}
  \Au{Владимиров~В.\,С.} Обобщенные функции в~математической физике.~--- М.: 
Наука, 1976. 280~с.
  \bibitem{15-kr}
  \Au{Коновалов М.\,Г.} Методы адаптивной обработки информации и~их  
приложения.~--- М.: ИПИ РАН, 2007. 212~с.

 \end{thebibliography}

 }
 }

\end{multicols}

\vspace*{-6pt}

\hfill{\small\textit{Поступила в~редакцию 28.04.18}}

\vspace*{6pt}

%\newpage

%\vspace*{-24pt}

\hrule

\vspace*{2pt}

\hrule

\vspace*{-2pt}


\def\tit{FINDING CONTROL POLICY FOR~ONE~DISCRETE-TIME MARKOV 
CHAIN ON~$[0,1]$ WITH~A~GIVEN INVARIANT MEASURE}

\def\titkol{Finding control policy for~one~discrete-time MARKOV 
chain on~$[0,1]$ with~a~given invariant measure}

\def\aut{M.\,G.~Konovalov$^1$ and~R.\,V.~Razumchik$^{1,2}$}

\def\autkol{M.\,G.~Konovalov and~R.\,V.~Razumchik}

\titel{\tit}{\aut}{\autkol}{\titkol}

\vspace*{-11pt}


\noindent
$^1$Institute of Informatics Problems, 
Federal Research Center ``Computer Science and Control'' of the Russian\linebreak 
$\hphantom{^1}$Academy of Sciences, 44-2~Vavilov Str., Moscow 119333, Russian Federation

\noindent
$^2$Peoples' Friendship University of Russia (RUDN University),  
6~Miklukho-Maklaya Str., Moscow 117198, Russian\linebreak
$\hphantom{^1}$Federation


\def\leftfootline{\small{\textbf{\thepage}
\hfill INFORMATIKA I EE PRIMENENIYA~--- INFORMATICS AND
APPLICATIONS\ \ \ 2018\ \ \ volume~12\ \ \ issue\ 3}
}%
 \def\rightfootline{\small{INFORMATIKA I EE PRIMENENIYA~---
INFORMATICS AND APPLICATIONS\ \ \ 2018\ \ \ volume~12\ \ \ issue\ 3
\hfill \textbf{\thepage}}}

\vspace*{3pt}


\Abste{A~discrete-time Markov chain on the interval~$[0,1]$ with two possible transitions (left or right) at each 
step has been considerred. The probability of transition towards~0 (and towards~1) is a~function of the current value 
of the chain. Having chosen the direction, the chain moves to the randomly chosen point from the appropriate 
interval. The authors assume that the transition probabilities depend on the current value of the chain only through a~finite 
number of real-valued numbers. Under this assumption, they seek the transition probabilities, which guarantee 
the~$L_2$ distance between the stationary density of the Markov chain and the given invariant measure on $[0,1]$ 
is minimal. Since there is no reward function in this problem, it does not fit in the 
MDP (Markov decision process) framework. The authors follow the 
sensitivity-based approach and propose the gradient- and simulation-based method for estimating the parameters of 
the transition probabilities. Numerical results are presented which show the performance of the method for various 
transition probabilities and invariant measures on~$[0,1]$.}

\KWE{Markov chain; control; continuous state space; sensitivity-based approach; derivative estimation}


\DOI{10.14357/19922264180301}

%\vspace*{-14pt}

\Ack
\noindent
The reported study was funded by the Russian Foundation for
Basic Research according to the research 
project No.\,18-07-00692.



%\vspace*{6pt}

  \begin{multicols}{2}

\renewcommand{\bibname}{\protect\rmfamily References}
%\renewcommand{\bibname}{\large\protect\rm References}

{\small\frenchspacing
 {%\baselineskip=10.8pt
 \addcontentsline{toc}{section}{References}
 \begin{thebibliography}{99}
  \bibitem{1-kr-1}
  \Aue{Karlin, S.} 1953. Some random walks arising in learning models.~I.
  \textit{Pac. J.~Math.} 3(4):725--756.
  \bibitem{2-kr-1}
  \Aue{Kaijser, T.} 1994. On a~theorem of Karlin. \textit{Acta Appl. Math.} 34:51--69.
  \bibitem{3-kr-1}
  \Aue{Ramli, M.\,A., and G.~Leng}. 2010. The stationary probability density of a~class of 
bounded Markov processes. \textit{Adv. Appl. Probab.} 42:986--993.
  \bibitem{4-kr-1}
  \Aue{McKinlay, S., and K.~Borovkov}. 2016. On explicit form of the stationary distributions 
for a~class of bounded Markov chains. \textit{J.~Appl. Probab.} 53(1):231--243. 
  \bibitem{5-kr-1}
  \Aue{Li, C.} 1961. \textit{Human genetics}. New York, NY: McGraw-Hill. 218~p.
  
  \bibitem{7-kr-1}
  \Aue{DeGroot, M.\,H., and M.\,M.~Rao}. 1963. Stochastic give-and-take. \textit{J.~Math. 
Anal. Appl.} 7:489--498.

\bibitem{6-kr-1} %7
  \Aue{McKinlay, S.} 2014. A~characterization of transient random walks on stochastic 
matrices with Dirichlet distributed limits. \textit{J.~Appl. Probab.} 51:542--555.

\bibitem{10-kr-1} %8
  \Aue{Peign$\acute{\mbox{e}}$,~M.} 1993. Iterated function systems and spectral 
decomposition of the associated Markov operator. \textit{Publications 
math$\acute{\mbox{e}}$matiques et informatique de Rennes}. 2:1--28.

  \bibitem{8-kr-1} %9
  \Aue{Diaconis, P., and D.~Freedman}. 1999. Iterated random functions. \textit{SIAM 
Rev.} 41(1):45--76.
  \bibitem{9-kr-1} %10
  \Aue{Ladjimi, F., and M.~Peign$\acute{\mbox{e}}$.} Iterated function systems with place 
dependent probabilities and application to the Diaconis--Friedman's chain on $[0,1]$. Available 
at: {\sf https://hal.archives-ouvertes.fr/LMPT/hal-01567392v1/} (accessed April~4, 2018).
  
  \bibitem{11-kr-1}
  \Aue{Stenflo,~$\ddot{\mbox{O}}$.} 2001. A~note on a~theorem of Karlin. \textit{Stat. 
Probabil. Lett.} 54(2):183--187.
  \bibitem{12-kr-1}
  \Aue{Jacquin, A.} 1989. A~fractal theory of iterated Markov operators with applications to 
digital image coding. Atlanta, GA:  Georgia Institute of Technology. Ph.D. Thesis.
  \bibitem{13-kr-1}
  \Aue{Forte, B., and E.\,R.~Vrscay.} 1995. Solving the inverse problem for measures using 
iterated function systems: A~new approach. \textit{Adv. Appl.
Probab.} 27(3):800--820.
  \bibitem{14-kr-1}
  \Aue{Vladimirov, V.\,S.} 1976. \textit{Obobshchennye funktsii v~ma\-te\-ma\-ti\-che\-skoy fizike} 
[Generalized functions in mathematical physics]. Moscow: Nauka. 280~p.
  \bibitem{15-kr-1}
  \Aue{Konovalov, M.\,G.} 2007. \textit{Metody adaptivnoy obrabotki informatsii i~ikh 
prilozheniya} [Methods of adaptive information processing and their applications]. Moscow: IPI 
RAN. 212~p.
  \end{thebibliography}

 }
 }

\end{multicols}

\vspace*{-6pt}

\hfill{\small\textit{Received April 28, 2018}}

%\pagebreak

%\vspace*{-18pt}

\Contr

\noindent
\textbf{Konovalov Mikhail G.} (b.\ 1950)~--- Doctor of Science in technology, 
principal scientist, Institute of Informatics Problems, Federal Research 
Center ``Computer Science and Control'' of the Russian Academy of Sciences, 
44-2 Vavilov Str., Moscow 119333, Russian Federation; 
\mbox{mkonovalov@ipiran.ru}

\vspace*{3pt}

\noindent
\textbf{Razumchik Rostislav V.} (b.\ 1984)~--- Candidate of Science (PhD) in 
physics and mathematics, leading scientist, Institute of Informatics 
Problems, Federal Research Center ``Computer Science and Control'' of the 
Russian Academy of Sciences, 44-2~Vavilov Str., Moscow 119333, Russian 
Federation; associate professor, Peoples' Friendship University of Russia 
(RUDN University), 6~Miklukho-Maklaya Str., Moscow 117198, Russian 
Federation; \mbox{rrazumchik@ipiran.ru}

\label{end\stat}

\renewcommand{\bibname}{\protect\rm Литература}       