\def\stat{zatsman}

\def\tit{ИМПЛИЦИРОВАННЫЕ ЗНАНИЯ: ОСНОВАНИЯ И ТЕХНОЛОГИИ ИЗВЛЕЧЕНИЯ$^*$}

\def\titkol{Имплицированные знания: основания и~технологии извлечения}

\def\aut{И.\,М.~Зацман$^1$}

\def\autkol{И.\,М.~Зацман}

\titel{\tit}{\aut}{\autkol}{\titkol}

\index{Зацман И.\,М.}
\index{Zatsman I.\,M.}




{\renewcommand{\thefootnote}{\fnsymbol{footnote}} \footnotetext[1]
{Работа выполнена в~Институте проблем информатики ФИЦ ИУ РАН при поддержке РФФИ (проект  
18-07-00192).}}


\renewcommand{\thefootnote}{\arabic{footnote}}
\footnotetext[1]{Институт проблем информатики Федерального исследовательского 
центра <<Информатика и~управление>> Российской академии наук, 
\mbox{izatsman@yandex.ru}}

%\vspace*{8pt}

   \Abst{Дано описание теоретических оснований разработки информационных технологий, 
которые обеспечивают целенаправленность формирования лингвистических типологий как 
форм представления нового знания о~языке. Они формируются в~процессе контрастивного 
анализа параллельных выровненных текстов, служащих источниками нового знания. 
В~параллельных текстах встречаются случаи имплицирования субъективных знаний 
переводчиков, которые не представлены в~сис\-те\-ме современного знания о языке. Их 
экспликация возможна с~по\-мощью информационных технологий, позволяющих 
обрабатывать параллельные тексты и~целенаправленно извлекать имплицированные знания. 
Цель статьи состоит в~описании нового подхода к~разработке технологий, обеспечивающих 
целенаправленность формирования лингвистических типологий, и~его сопоставлении 
с~существующими подходами и~моделями процессов формирования (роста) знания. 
Сформулированы те условия, при которых технологически может быть обеспечен 
целенаправленный рост объективного знания (в~терминах К.\,Р.~Поппера). Предлагаемый 
подход иллюстрируется на примере задачи формирования типологии конструкций русского 
языка с~модальным значением, возникающих в~переводе немецких модальных конструкций.}
    
   \KW{параллельные тексты; корпусная лингвистика; имплицированные знания; 
извлечение новых знаний; эмерджентность; информационная технология; 
целенаправленность; формирование типологий}

\DOI{10.14357/19922264180311}
  
\vspace*{-4pt}


\vskip 10pt plus 9pt minus 6pt

\thispagestyle{headings}

\begin{multicols}{2}

\label{st\stat}
   
\section{Введение}
    
  Разработка информационной технологии, которая обеспечивает 
формирование лингвистических типологий как форм представления нового 
знания о языке, является одной из задач проекта по гранту РФФИ, который 
в~настоящее время выполняется в~Институте проблем информатики ФИЦ ИУ 
РАН. 
{\looseness=1

}

Проектируемая информационная технология формирования (ИТФ) 
лингвистических типологий\footnote[2]{В лингвистике типологии используются, 
как правило, для описания сходства и~различий между языками. В~статье этот 
термин используется в~другом значении: для описания сходства и~различий 
между языковыми единицами.} предназначена для использования в~процессе 
решения трех взаимосвязанных задач: 
\begin{enumerate}[(1)]
\item обнаружение лакун в~сис\-те\-ме 
современного знания о языке относительно исследуемых языковых единиц; 
\item целенаправленное извлечение новых знаний для заполнения лакун; 
\item формирование и~развитие лингвистических типологий на основе 
извлеченных знаний.
\end{enumerate}
  
  Цель статьи состоит в~описании подхода к~созданию ИТФ, обеспечивающей 
целенаправленность развития современного знания о~языке на основе 
экспликации субъективных знаний переводчиков, имплицированных 
в~параллельных текстах, за счет формирования новых или развития 
существующих типологий. 

Существует принципиальная разница между 
знаниями, представленными в~параллельных текстах и~имплицированными 
в~них. 

В~первом случае смысловое содержание оригинального и~переводного 
текстов непосредственно передается с~помощью естественных языков 
оригинала или перевода соответственно. 

Второй случай предполагает знание 
двух языков и~переводных соответствий языковых единиц оригинального 
и~переводного текс\-тов. Для понимания текс\-та перевода необязательно знать 
язык оригинала и~переводные соответствия, но для выполнения перевода 
знание переводных соответствий является необходимым. При этом в~процессе 
перевода возможны случаи использования концептов субъективных знаний 
переводчика, которые не представлены в~сис\-те\-ме современного знания о~языке. 
Такое их использование предлагается называть \textit{имплицированием} 
субъективных знаний в~параллельных текстах.
  
  Фрагменты параллельных текстов, в~которых имплицированы концепты 
субъективных знаний переводчиков, будем называть \textit{объектами 
интерпретации}, которые по определению являются двуязычными. 
Создаваемая ИТФ предназначена для поиска таких объектов в~параллельных 
текстах и~обеспечения целенаправленной экспликации имплицированных 
знаний в~процессе семантического анализа объектов интерпретации. Здесь 
важно сказать о~той точке зрения, которая широко распространена 
в~эпистемологии, что рост научного знания эмерджентен и~поэтому 
непредсказуем в~принципе. А~раз его рост является спонтанным, то и~нельзя 
говорить об обеспечении целенаправленности его роста. Поэтому первый 
вопрос, рассматриваемый в~статье, можно было бы сформулировать так: всегда 
ли рост знания является эмерджентным?
  
  Если ответ <<нет>>, то второй вопрос, являющийся актуальным для 
выполняемого проекта, можно\linebreak было бы сформулировать так: если современная 
сис\-те\-ма знания о~языке не удовлетворяет новым образовательным, 
технологическим (например, машинный перевод) или иным потребностям 
общества, то можно ли обеспечить целенаправленность его роста 
с~использованием средств информатики?

\vspace*{-6pt}
  
\section{Что говорит эпистемология?}

\vspace*{-2pt}
    
  Согласно К.\,Р.~Попперу, рост научного знания непредсказуем в~принципе: 
<<Мои доводы строятся на предположении, что существует такая вещь, как 
подлинный рост научного знания, или, выражаясь практически, что завтра или 
через год мы, возможно, выдвинем и~подвергнем проверке важные теории, 
о~которых до сих пор никто \textit{всерьез не думал} (курсив мой~--- ИЗ). Если 
существует рост научного знания в~этом смысле, то он не может быть 
предсказуем средствами науки. Ведь тот, кто смог бы сегодня средствами науки 
предсказать завтрашние открытия, мог бы сегодня их и~сделать, это означало 
бы, что рост научного знания закончился>>~\cite[с.~283]{1-zat}.
  
  В этой книге описывается четырехчленная итерационная схема спонтанного 
роста научного знания, представленного в~форме тео\-рий. Каждая итерация 
этой схемы имеет следующий вид: $\mathrm{Р}_1\hm\to \mathrm{ТТ} \hm\to 
\mathrm{ЕЕ}\hm\to \mathrm{Р}_2$, где $\mathrm{Р}_1$ означает <<проблема\linebreak 
в~начале итерации>>; $\mathrm{ТТ}$~--- <<пробная теория>>; 
$\mathrm{ЕЕ}$~--- <<устранение ошибок>>; $\mathrm{Р}_2$~---  
<<проб\-ле\-ма в~конце\linebreak  итерации>>. С~помощью этой схемы К.\,Р.~Поппер 
<<пытался показать, что результатом критики или устранения ошибок 
в~применении к~пробной теории, как правило, становится возникновение новой 
проблемы или даже нескольких новых проблем>>~\cite[с.~273--274]{1-zat}.
  
  Итерационный подход к~моделированию процессов роста знания 
используется и~в экономике. В~этой научной дисциплине первая итерационная 
модель была создана в~последнем десятилетии прошлого века: спиральная 
модель процессов формирования новых знаний. Она была предложена 
И.~Нонака и~впервые описана им в~работах~\cite{2-zat, 3-zat}. В~процессе ее 
построения рассматривались индивидуальные знания человека и~коллективные 
знания группы людей. Каждая из этих двух категорий знаний была разделена на 
эксплицированные (explicit) и~имплицитные (tacit) знания. Следовательно, 
спиральная модель включает в~рассмотрение следующие четыре категории 
знаний: индивидуальные имплицитные, коллективные имплицитные, 
коллективные и~индивидуальные эксплицированные знания.
  
  С использованием этих четырех категорий И.~Нонака определил понятие 
<<спираль формирования знаний>>. Было показано на примерах, что эта 
спираль может служить качественной моделью итерационного процесса 
формирования новых знаний во время проведения <<мозгового штурма>>. 
Развитие спиральной модели и~описание примеров ее использования дано 
в~работах~\cite{4-zat, 5-zat}.
  
  Обобщение и~принципиально новое развитие результатов И.~Нонака было 
предложено в~работах Й.~Накамори и~А.~Вежбицкого в~рамках создаваемой 
ими научной дисциплины, которую они называют <<Наука 
о~знаниях>>~\cite{6-zat, 7-zat, 8-zat}. 

В~этих работах знания разделены на 
индивидуальные знания человека, коллективные и~конвенциональные знания. 
Это деление Вежбицкий и~Накамори называют социальным аспектом или 
измерением, так как перечисленные знания отличаются тремя уровнями 
социализации (от первого индивидуального уровня до третьего 
конвенционального). С~учетом деления на эксплицированные и~имплицитные 
знания добавляются две новые категории знаний, которых нет в~спиральной 
модели: конвенциональные имплицитные и~конвенциональные 
эксплицированные знания. Развитие спиральной модели, включающее шесть 
категорий знаний, получило название <<креативное  
пространство>>~\cite{6-zat}.

  \begin{table*}\small %tabl1
  \begin{center}
  \Caption{Три подхода к~описанию роста знания}
  \vspace*{2ex}
  
  \begin{tabular}{|p{32mm}|p{46mm}|p{36mm}|p{35mm}|}
  \hline
\multicolumn{1}{|c|}{\tabcolsep=0pt\begin{tabular}{c}\textit{Позиции} \\
\textit{сопоставления}\end{tabular}}&
\multicolumn{1}{c|}{\tabcolsep=0pt\begin{tabular}{c}1.~Четырехчленная\\ схема~\cite{1-zat}\end{tabular}}&
\multicolumn{1}{c|}{\tabcolsep=0pt\begin{tabular}{c}2.~Спиральная\\ модель~\cite{2-zat,  3-zat, 4-zat, 5-zat}\end{tabular}}&
\multicolumn{1}{c|}{\tabcolsep=0pt\begin{tabular}{c}3.~Креативное\\ пространство~\cite{6-zat, 7-zat, 8-zat}\end{tabular}}\\
\hline
\textit{Социальное} \textit{измерение}\newline (включает три позиции номинативной шкалы: 
индивидуальное, коллективное, конвенциональное)&Индивидуальные (субъективные) 
знания мира~2 и~конвенциональные (объективные) знания мира~3&Индивидуальные 
и~коллективные знания&Индивидуальные, коллективные и~конвенциональные знания\\
\hline
\textit{Имплицитные/экс\-пли\-ци\-ро\-ван\-ные знания}&Имплицитные индивидуальные знания 
(мир~2), эксплицированные индивидуальные и~конвенциональные знания 
(мир~3)&Имплицитные и~эксплицированные знания (и~для индивидуальных, и~для 
коллективных)&Имплицитные и~эксплицированные знания (для индивидуальных, 
коллективных и~конвенциональных)\\
\hline
\textit{Источник роста знания}&Явно не указан (подразумеваются индивидуальные 
и~конвенциональные эксплицированные знания)&Явно не указан (подразумеваются 
индивидуальные и~коллективные эксплицированные знания)&Явно не указан 
(подразумеваются индивиду\-альные, коллективные и~конвенциональные эксплицированные 
знания)\\
\hline
\end{tabular}
\end{center}
\end{table*}
  
  Отметим, что в~четырехчленной схеме, спиральной модели и~креативном 
пространстве нет явно определенной оси времени. Это не дает воз\-мож\-ности 
фиксировать моменты времени создания пробной теории или генерации 
каждого нового концепта как структурного элемента создаваемой теории или 
развиваемой сис\-те\-мы знания. Также в~рассмотренных подходах явно не 
указываются информационные источники роста знания, но они 
подразумеваются.



  \begin{table*}[b]\small %tabl2
  \begin{center}
  \Caption{Семь значений слова \textit{face} по HTE~\cite{10-zat, 11-zat}}
  \vspace*{2ex}
  
  \begin{tabular}{|l|p{66mm}|l|}
  \hline
\multicolumn{1}{|c|}{Номер концепта 
в~HTE}&\multicolumn{1}{c|}{Концепт}&\multicolumn{1}{c|}{Слово и~год появления 
концепта}\\
\hline
01.02.03.08.01.04 n.&The body :: Face&Matching word(s): face (1290--)\\
\hline
01.02.03.08.01.04$\vert$04 n.&The body :: Face :: with reference to beauty&Matching word(s): 
face (1591--)\\
\hline
01.02.03.08.01.04.01 n.&The body :: Face with expression/expression&Matching word(s): face 
(1330--)\\
\hline
\multicolumn{1}{|l|}{\raisebox{-6pt}[0pt][0pt]{01.02.03.08.01.04.01$\vert$01 n.}}&The body :: Face with expression/expression :: 
grimace/distortion&\multicolumn{1}{l|}{\raisebox{-6pt}[0pt][0pt]{Matching word(s): face (1602--)}}\\
\hline
\multicolumn{1}{|l|}{\raisebox{-6pt}[0pt][0pt]{01.12.05.03.01$\vert$13 n.}}&Relative position :: Surface :: one of several surfaces of 
a~thing&\multicolumn{1}{l|}{\raisebox{-6pt}[0pt][0pt]{Matching word(s): face (1340--)}}\\
\hline
01.12.05.03.01$\vert$19 n.&Relative position :: Surface :: front surface&Matching word(s): face 
(1611\;+\;1820--)\\
\hline
\multicolumn{1}{|l|}{\raisebox{-6pt}[0pt][0pt]{01.12.05.03.01$\vert$19.01 n.}}&
Relative position :: Surface :: front surface :: specifically of 
a~coin/medal/seal/die, etc.&\multicolumn{1}{l|}{\raisebox{-6pt}[0pt][0pt]{Matching word(s): face (1515--)}}\\
\hline
\multicolumn{3}{p{162.5mm}}{\footnotesize \hspace*{3mm}\textbf{Примечание:}
если в~третьем столбце после <<-->> не указан год, то 
это говорит о том, что это значение встречается в~текстах до настоящего времени.}
   \end{tabular}
   \end{center}
   \end{table*}
   
  
  Сопоставим три рассмотренных подхода к~описанию роста знания, которые 
пронумеруем согласно первой строке табл.~1. Строго говоря, в~описании самой 
четырехчленной схемы не указаны категории знания, но они отмечены для 
первого подхода в~позиции  
<<\textit{Им\-пли\-цит\-ные/экс\-пли\-ци\-ро\-ван\-ные знания}>> в~этой 
таблице. Почему? В~той же самой работе К.\,Р.~Поппер проводит границу 
между субъективным и~объективным знанием, строит теорию познания, 
в~контексте которой и~дано описание рассматриваемой четырехчленной схемы. 
  
  

  В~процессе ее построения он описывает идею <<трех миров>> следующим 
образом: <<Примерами объективного знания являются теории, 
опубликованные в~журналах и~книгах и~хранящиеся в~биб\-лио\-те\-ках, обсуждения 
этих теорий, трудности или проблемы, на которые было указано в~связи 
с~такими теориями и~т.\,д. Мы можем назвать физический мир <<миром~1>>, 
мир наших осознанных переживаний~--- <<миром~2>>, а~мир логического 
содержания книг, библиотек, компьютерной памяти и~тому подобного~--- 
<<миром~3>>~\cite[с.~78]{1-zat}. Следовательно, если рассматривать 
четырехчленную схему в~контексте описания идеи <<трех миров>>, то это 
является основанием указать для нее три категории знания в~табл.~2: 
имплицитные индивидуальные, эксплицированные индивидуальные 
и~конвенциональные знания.
  
  Вернемся к~первому вопросу статьи: возможно ли именно целенаправленное 
выявление лакун и~их заполнение, обеспечивающее рост знания? Отметим, что 
во всех трех рассмотренных подходах категория \textit{имплицированных} 
знаний отсутствует, т.\,е.\ для них первый вопрос остается открытым. Во всех 
трех подходах можно найти неявные указания на источники нового знания. 
В~табл.~1 указаны по два источника для первого и~второго подходов, три~--- 
для третьего. Источники не указаны в~явном виде, но они подразумеваются 
в~описании связей между знаниями разных категорий. Например, зависимость 
имплицитного индивидуального знания от конвенционального в~первом 
подходе описана так: <<В~мире~3 мы можем открыть новые проблемы, 
которые были там до того, как их открыли, и~до того, как они были осознаны, 
то есть до того, как что-ли\-бо \textit{соответствующее им появилось 
в~мире~2}~$\langle\ldots\rangle$\ Основной тезис: наше \textit{осознанное 
субъективное знание (знание в~мире~2) зависит от мира~3, то есть от 
теорий, сформулированных (хотя бы виртуально) на определенном языке} 
(курсив мой~--- ИЗ)>>~\cite[с.~78]{1-zat}.
  
  Рассмотренные три теоретических подхода сопоставим с~практикой 
конкурсного финансирования фундаментальных исследований, 
ориентированных на рост научного знания.

\vspace*{-4pt}
  
\section{Практика формирования нового знания}

\vspace*{-2pt}
    
  Для примера возьмем проект, поддержанный Национальным научным 
фондом (ННФ) США, который посвящен моделированию процесса 
возникновения новых значений (=\;кон\-цеп\-тов) слов.\linebreak \textit{Целью} проекта 
ННФ является разработка рет\-ро\-спективной модели, описывающей временн$\acute{\mbox{у}}$ю\linebreak 
траекторию возникновения новых концептов. Исполнители проекта исходили 
из того, что в~естественном языке используется конечное число слов, 
обозначающих концепты, но при этом слова должны быть способны выражать 
значительно большее чис\-ло концептов, чем число слов в~языке. Из этого они 
сделали вывод, что в~процессе развития языка существующие слова 
приобретают все новые и~новые концепты согласно некоторой модели, что 
и~было ими показано в~результате выполнения проекта.
  
  Для достижения цели проекта его исполнители поставили \textit{задачу} 
разработать ретроспективную модель (\textit{ожидаемый результат 
проекта}), описывающую процесс появления в~прошлом концептов, исследуя 
эволюцию английского языка~\cite{9-zat}. В~этой модели есть ось времени, на 
которой указаны моменты появления новых значений у~исследуемого слова. 
Эти моменты определялись по Историческому тезаурусу английского языка 
(The Historical Thesaurus of English~--- HTE), в~котором содержится 
информация о годах появления новых концептов слов, начиная 
с~1000~г.~\cite{10-zat}.
  
  Ретроспективная модель описывает связи между концептами каждого 
исследуемого слова и~степень их семантической близости, которая определятся 
на основе иерархического номера, присвоенного каждому концепту 
в~HTE~\cite{11-zat}. В~качестве примера приведем семь концептов слова 
\textit{face} по HTE c указанием их номеров и~годов появления в~текстах 
Британского национального корпуса (British National Corpus~--- 
 BNC)~\cite{12-zat} (см.\ табл.~2). Отметим, что для всех семи концептов не 
указаны годы их исчезновения из текстов, т.\,е.\ они встречаются 
в~современных текстах BNC.
  

  
  В ретроспективной модели источниками нового знания о временн$\acute{\mbox{ы}}$х 
траекториях возникновения новых концептов являются HTE и~BNC. Для 
рассмотренного проекта три позиции сравнения, приведенных в~табл.~1, будут 
иметь следующий вид:
  \begin{enumerate}[(1)]
  \item конвенциональные знания, отраженные в~BNC, а~также 
индивидуальные и~коллективные знания исполнителей проекта;
  \item  эксплицированные конвенциональные знания, а~также 
эксплицированные и~имплицитные индивидуальные и~коллективные знания 
исполнителей проекта;
  \item  источником нового знания о~временн$\acute{\mbox{ы}}$х траекториях являются HTE 
и~BNC.
  \end{enumerate}
  
  Возникает закономерный вопрос, в~какой степени новое знание в~форме 
моделей и~временн$\acute{\mbox{ы}}$х траекторий возникновения новых концептов является 
спонтанным? Авторы проекта сформулировали его цель и~задачу, описали 
ожидаемый результат, доступные информационные ресурсы HTE и~BNC,
запланировали выполнение проекта, включая\linebreak проведение вычислительных 
экспериментов с~использованием методов корпусной лингвистики, получили 
ожидаемый результат и~опубликовали\linebreak  статью~\cite{9-zat}. При этом они 
сформировали новое знание в~форме моделей и~их действия в~целях его 
получения были \textit{целенаправленными}. Таким образом, этот пример 
говорит о возможности целенаправленного роста научного знания.

  \begin{table*}\small %tabl3
  \begin{center}
  \Caption{Примеры объектов интерпретации с~модальными конструкциями в~оригинале}
  \vspace*{2ex}
  
  \begin{tabular}{|p{75mm}|p{75mm}|}
  \hline
  \multicolumn{1}{|c|}{Оригинал} & \multicolumn{1}{c|}{Перевод}\\
  \hline
Sollte jetzt etwa eine Predigt stattfinden?\newline
[Franz Kafka. Der Prozess (1914)]&Неужели сейчас кто-то будет читать проповедь?\newline
[Франц Кафка. Процесс (Р.~Райт-Ковалева, 1965)]\\
\hline
Warum \mbox{mu{\!\ptb{\ss}}te} er diese Demoiselle St$\ddot{\mbox{u}}$wing heiraten und 
den$\ldots$ Laden$\ldots$\newline
[Thomas Mann. Buddenbrooks (1896--1900)]&Зачем ему понадобилось жениться на этой 
мадемуазель Штювинг с~ее\ $\ldots$\ лавкой?\newline
[Томас Манн. Будденброки (Н.~Ман, 1953)]\\
\hline
Er durfte nun eine Weile lang guten Gewissens ruhen.\newline
[Patrick S$\ddot{\mbox{u}}$skind. Das Parfum: Die Geschichte eines M$\ddot{\mbox{o}}$rders 
(1985)]&Теперь он имел право некоторое время отдыхать.\newline
[Патрик Зюскинд. Парфюмер: История одного убийцы (Э.~Венгерова, 1992)]\\
\hline
"Das Kind kann nichts daf$\ddot{\mbox{u}}$r", h$\ddot{\mbox{o}}$rte sie die Stimme sagen, 
``aber du, Kassiopeia~--- warum hast du das nur getan?''\newline
[Michael Ende. Momo (1973)]&-- Ребенок не виноват,~--- услышала она говорившего.~--- Но 
ты, Кассиопея, почему ты это сделала? \newline
[Михаэль Энде. Момо (Ю.\,И.~Коринец, 1982)]\\
\hline
``Mag sein,'' sagte der Advokat, ``wir wollen aber trotzdem nichts 
$\ddot{\mbox{u}}$bereilen.''\newline
[Franz Kafka. Der Prozess (1914)]&-- Возможно,~--- сказал адвокат,~--- и~все же не будем 
торопиться.\newline
[Франц Кафка. Процесс (Р.~Райт-Ковалева, 1965)]\\
\hline
\end{tabular}
\end{center}
\vspace*{-6pt}
\end{table*}
  
  Естественно, что возможны ситуации, когда не удается получить ожидаемые 
результаты проектов фундаментальных исследований и/или спонтанно 
возникают незапланированные новые результаты. Для положительного ответа 
на первый вопрос \mbox{статьи} достаточно было бы и~одного приведенного примера, 
но практика научных фондов позволяет утверждать, что это далеко не 
единичный случай. Таким образом, ответ на первый вопрос предлагается такой: 
многолетняя практика работы отечественных и~зарубежных фондов по 
финансированию проектов фундаментальных исследований говорит 
о~возможности целенаправленного формирования нового научного знания.
  
  Рассмотрим второй вопрос~--- о~воз\-мож\-ности обеспечить 
целенаправленность роста научного знания с~использованием средств 
информатики. При его рассмотрении ограничимся только сис\-те\-мой знания 
о~языке.

\vspace*{-6pt}

\section{Теоретические основания создания информационной технологии формирования}

\vspace*{-2pt}
    
  Разрабатываемая в~рамках проекта РФФИ концепция создания ИТФ 
предназначена в~первую очередь для решения трех задач, 
которые перечислены в~начале статьи. Первая основная идея пред\-ла\-га\-емой 
концепции и~ее отличие от 
трех подходов, приведенных в~табл.~1, заключается в~явном описании 
источника нового знания о~языке и~фрагментации его информации на объекты 
интерпретации (пять их примеров приведены в~табл.~3).
  

  
  В разрабатываемой концепции фрагментация информации источника знания 
на объекты интерпретации позиционируется как необходимое условие создания 
ИТФ. Проиллюстрируем это условие на примере задачи построения типологии 
конструк\-ций русского языка с~модальным значением, возникающих в~переводе 
конструкций немецкого языка с~модальными глаголами, которая является 
сегодня актуальной в~лингвистике~\cite{13-zat}. Для решения этой задачи 
естественно использовать параллельные тексты с~предложениями на немецком 
языке и~их переводами на русский язык (см.\ табл.~3).
  
  В настоящее время наиболее представительный массив не\-мец\-ко-рус\-ских 
текстов в~электронной форме объемом~2,6~млн словоупотреблений хранится 
в~Параллельном немецком корпусе (ПНК), который находится в~открытом 
доступе~\cite{14-zat}. Так как эти тексты являются выровненными 
(оригинальным предложениям поставлены в~соответствие их переводы), то 
необходимая фрагментация в~ПНК уже есть, при этом объект  
интерпретации~--- это пара немецкого и~русского предложений (см.\ табл.~3). Когда 
используются невыровненные тексты оригинала и~перевода, тогда операция 
фрагментации является необходимым компонентом ИТФ~\cite{15-zat}.



\setcounter{table}{4}
\begin{table*}\small %tabl5
\begin{center}
\Caption{Пример завершенной аннотации}
\vspace*{2ex}

\begin{tabular}{|p{40mm}| p{14mm}| p{40mm}| p{14mm}|}
\hline
\textit{Sie} \textbf{sollten} diese ungem$\ddot{\mbox{u}}$tliche Sache jetzt \textit{lieber sein 
lassen}.
&\textbf{sollen}\newline 
$\langle$+Inf I$\rangle$\newline 
$\langle$Sie$\rangle$\newline 
$\langle$Praet$\rangle$\newline 
$\langle$Konj II$\rangle$\newline 
$\langle$sollen-3$\rangle$
&-- \textbf{Давай} \textit{оставим} это дело \textit{в~покое}.
&$\langle$1pl$\rangle$\newline
$\langle$Imperat$\rangle$\newline
$\langle$2sg$\rangle$\newline
$\langle$давай$\rangle$\\
\hline 
\multicolumn{4}{p{121mm}}{\footnotesize \hspace*{3mm}\textbf{Примечания:} %\newline
$\langle$+Inf I$\rangle$~--- управляет глаголом в~форме инфинитива~I; %,\newline
$\langle$Sie$\rangle$~--- вежливая форма обращения ко 2-му лицу; %\newline
$\langle$Praet$\rangle$~--- Pr$\ddot{\mbox{a}}$teritum; %,\newline
$\langle$Konj II$\rangle$~--- Konjunktiv~II; %\newline
$\langle$sollen-3$\rangle$~--- соответствует 3-му значению в~словарной статье модального глагола 
\textit{sollen}; %\newline
$\langle$1pl$\rangle$~--- 1-е лицо, множественное число (глагола, управляемого формой \textit{давай}); %\newline
$\langle$Imperat$\rangle$~--- императив (характеристика формы \textit{давай}); %\newline
$\langle$2sg$\rangle$~--- характеристика формы \textit{давай}.}
\end{tabular}
\end{center}
\vspace*{-6pt}
\end{table*}


\begin{table*}\small %tabl6
\begin{center}
\Caption{Пример незавершенной аннотации}
\vspace*{2ex}

\begin{tabular}{|p{30mm}|p{19mm}|p{40mm}|p{32mm}|}
\hline
Gut, sagte ich, \textbf{soll} \textit{er} dich \textit{verehren}, aber soviel kostbare Blumen, das ist 
aufdringlich. &\textbf{Sollen}\newline
$\langle$+Inf I$\rangle$\newline $\langle$3sg$\rangle$\newline $\langle$Praes$\rangle$\newline 
$\langle$sollen-x$\rangle$\newline 
$\langle$Verb-Initial$\rangle$ 
&--~Очень мило,~--- сказал~я,~--- \textit{поклонник поклонником}, но дарить такой большой 
букет дорогих цветов~--- значит навязываться.& $\langle$xN-Nomin+xN-Instr$\rangle$\\
\hline
\multicolumn{4}{p{133mm}}{\footnotesize \hspace*{3mm}\textbf{Примечания:} %\newline
$\langle$sollen-x$\rangle$~--- означает, что значение модального глагола в~этом предложении не найдено 
в~не\-мец\-ко-рус\-ском словаре в~процессе лингвистического аннотирования; %\newline
$\langle$xN-Nomin+xN-Instr$\rangle$~--- конструкция <<существительное в~именительном падеже\;+\;то же 
существительное в~творительном>>.}
\end{tabular}
\end{center}
%\vspace*{6pt}
\end{table*}

  
  Для решения задачи построения типологии необходимо из корпуса 
выровненных текстов отобрать с~помощью поиска по леммам предложения 
с~исследуемыми немецкими модальными глаголами и~одновременно переводы 
этих предложений на русский язык. Таблица~4 содержит данные прове-\linebreak\vspace*{-12pt}

{\small %tabl4
    \vspace*{18pt}
    \noindent
 {{\tablename~4}\ \ \small{Число объектов интерпретации в~не\-мец\-ко-рус\-ских текстах ПНК}}

%\vspace*{4pt}

 \begin{center} 
  \tabcolsep=11pt
  \begin{tabular}{|l|c|}
  \hline
\multicolumn{1}{|c|}{Модальный глагол}&
\tabcolsep=0pt\begin{tabular}{c}Число\\ объектов\\ интерпретации\end{tabular}\\
\hline
\hspace*{17mm}D$\ddot{\mbox{u}}$rfen&\hphantom{9\,}758\\
\hspace*{17mm}K$\ddot{\mbox{o}}$nnen&5\,782\\
\hspace*{17mm}M$\ddot{\mbox{o}}$gen&\hphantom{9\,}937\\
\hspace*{17mm}M$\ddot{\mbox{u}}$ssen&3\,209\\
\hspace*{17mm}Sollen&2\,041\\
\hspace*{17mm}Wollen&3\,541\\
\hline
\textit{Всего объектов интерпретации}&16\,268\hphantom{9}\\
\hline
\end{tabular}
\vspace*{2pt}
\end{center}
}
%\end{table*}

\noindent
денного 
поиска: для рассматриваемой задачи исследователям доступны в~ПНК 
более~16~тыс.\ объектов интерпретации (в~таблице они распределены
 по шести немецким модальным глаголам).
  
  
  
  Вторая идея предлагаемой концепции заключается в~обеспечении с~помощью 
ИТФ лингвистического аннотирования отобранных объектов интерпретации, 
позволяющего фиксировать лакуны в~сис\-те\-ме современного знания о~языке. 
До начала аннотирования для рассматриваемой задачи был выбран  
не\-мец\-ко-рус\-ский словарь~\cite{16-zat}, который отражает современный 
уровень, и~он используется в~процессе аннотирования. Результатом 
аннотирования объектов интерпретации могут быть как завершенные (табл.~5), 
так и~незавершенные (табл.~6) двуязычные аннотации\footnote{Аннотации 
сформированы В.\,И.~Карповым (см.\ табл.~5) и~А.\,А.~Гончаровым (см.\ табл.~6). 
Примечания к~табл.~5 и~6 подготовлены Д.\,О.~Добровольским и~Анной 
А.~Зализняк.}.
  

  
  Массив отобранных завершенных аннотаций содержит структурированное 
описание конструкций немецкого языка с~модальными глаголами 
и~конструкций русского языка с~модальным значением. Исследование 
соотношения между количеством завершенных и~незавершенных аннотаций,\linebreak 
которое меняется в~процессе аннотирования, являет\-ся самостоятельной 
задачей. Есть первые результаты ее решения для переводов французских 
глаголов на русский язык~\cite{17-zat}. Был проведен эксперимент по 
аннотированию~2\,500~объектов\linebreak интерпретации, т.\,е.\ пар предложений, 
вклю\-ча\-ющих переводы французских глаголов, которые были выровнены так 
же, как и~немецкие предложения в~табл.~3. В~результате их аннотирования 
было сформировано~97,7\% завершенных аннотаций. При этом семантический 
анализ~2,3\% незавершенных аннотаций позволил лингвистам извлечь четыре 
низкочастотных модели перевода французских глаголов прошедшего времени 
несовершенного вида (imparfait) на русский язык, отсутствующих 
в~современных контрастивных грамматиках [18].
  
  Завершенность процесса аннотирования объекта интерпретации говорит 
о~том, что значение немецкого модального глагола, которое встретилось 
в~объекте интерпретации, присутствует в~словаре. Если этого значения 
в~словаре нет, то в~аннотации ставятся специальные теги, которые фиксируют 
ее незавершенность, причину незавершенности и~тем самым описывают 
потенциальные лакуны в~сис\-те\-ме современного знания о~языке. Таким образом, 
массив незавершенных аннотаций фиксирует некоторый спектр потенциальных 
лакун. Одной из целей создания ИТФ является обеспечение процесса 
аннотирования и~разметки незавершенных аннотаций с~помощью тегов, 
которые передаются на этап их семантического анализа, выполняемого  
линг\-ви\-ста\-ми-экс\-пер\-та\-ми. Результатом анализа является описание 
новых значений немецких модальных глаголов и~пополнение ими  
не\-мец\-ко-рус\-ско\-го словаря с~использованием переводов тех предложений, в~которых эти новые значения были обнаружены в~процессе аннотирования.
  
  Третья идея предлагаемой концепции заключается в~обеспечении с~помощью 
ИТФ многоаспектной кластеризации завершенных аннотаций. Выбор аспектов 
кластеризации зависит от решаемой лингвистической задачи и~исследуемых 
языковых единиц. Например, для формирования типологии конструкций 
русского языка с~модальным значением, возникающих в~переводе конструкций 
немецкого языка с~модальными глаголами, завершенные аннотации 
группируются по немецким модальным глаголам (см.\ табл.~4), но это является 
только одним из аспектов кластеризации.
  
\section{Заключение}
    
  Применение ИТФ в~процессе лингвистического аннотирования дает 
возможность лингвистам\linebreak
 сформировать массив незавершенных аннотаций,\linebreak 
отражающих потенциальные лакуны в~сис\-те\-ме совре\-менного знания о~языке. 
Последующий семантический анализ таких аннотаций позволяет выявить 
новые значения модальных глаголов и~сформировать искомую 
типологию~\cite{13-zat}. Сопоставим предлагаемые тео\-ре\-ти\-че\-ские основания 
создания ИТФ с~ретроспективной моделью возникновения новых 
концептов~\cite{9-zat, 10-zat, 11-zat}. Основное отличие заключается в~том, что 
в~первом случае ИТФ помогает лингвистам целенаправленно выявлять 
и~описывать новые концепты, имплицированные в~параллельных текс\-тах, а~во 
втором случае информационные ресурсы BNC используются для описания 
ретроспективной временн$\acute{\mbox{о}}$й траектории появления концептов, которые 
эксплицированы, так как они уже были описаны в~HTE до начала проекта ННФ. 
  
  В первом случае ИТФ обеспечивает извлечение имплицированных 
субъективных знаний переводчиков (мир~2) и~формирование на их основе 
новых эксплицированных знаний лингвистов в~мире~3 как объективного 
знания в~терминах К.\,Р.~Поппера. Во втором случае были построены 
временн$\acute{\mbox{ы}}$е траектории и~новые модели появления концептов с~использованием 
информационных ресурсов BNC и~HTE без извлечения из них 
имплицированных знаний. Из HTE копировалась информация о концептах 
и~годах их первого появления в~текстах BNC, на основе которой строились 
траектории и~модели появления концептов исследуемых слов.
  
  Необходимыми условиями создания ИТФ, обеспечивающих рост научного 
знания в~лингвистике за счет экспликации имплицированных знаний, являются 
следующие положения: 
  \begin{itemize}
  \item  наличие источника нового знания для обозначенной предметной 
области и~возможность фрагментации его информации на объекты 
интерпретации, в~которых имплицированы индивидуальные (субъективные) 
знания;
  \item  возможность отражения потенциальных лакун в~сис\-те\-ме современного 
знания с~помощью разметки тегами объектов интерпретации;
  \item возможность заполнения лакун результатами семантического анализа 
соответствующих объектов интерпретации в~процессе экспликации 
имплицированных знаний.
  \end{itemize}
  
  Для ряда задач контрастивной лингвистики эти три условия 
выполняются~\cite{17-zat, 18-zat}, и~они являются необходимыми 
тео\-ре\-ти\-че\-ски\-ми основаниями разработки ИТФ. В~настоящее время разработан 
ее первый компонент, ориентированный на обнаружение потенциальных лакун в~сис\-те\-ме современного знания о переводах конструкций с~немецкими 
модальными глаголами на русский язык. Первые результаты 
экспериментальной эксплуатации позволяют говорить о~воз\-мож\-ности 
обеспечения целенаправленности экспликации имплицированных знаний 
с~использованием средств информатики. Что касается других предметных 
областей, то вопрос применимости в~них аналогичных средств информатики, 
обеспечивающих целенаправленность роста знания, в~на\-сто\-ящее время 
является открытым, так как требуется отдельная проверка вы\-пол\-ни\-мости в~этих 
областях трех перечисленных условий.
  
{\small\frenchspacing
 {%\baselineskip=10.8pt
 \addcontentsline{toc}{section}{References}
 \begin{thebibliography}{99}
\bibitem{1-zat}
\Au{Поппер К.\,Р.} Объективное знание. Эволюционный подход~/ Пер. 
с~англ.~--- М.: Эдиториал УРСС, 2002. 384~с. (\Au{Popper~K.\,R.} Objective 
knowledge. An evolutionary approach.~--- Oxford: Clarendon Press, 1979. 395~p.)
\bibitem{2-zat}
\Au{Nonaka I.} The knowledge-creating company~// Harvard Bus. Rev., 
1991. Vol.~69. No.\,6. P.~96--104.
\bibitem{3-zat}
\Au{Nonaka I.} A~dynamic theory of organizational knowledge creation~// 
Organ. Sci., 1994. Vol.~5. No.\,1. P.~14--37.
\bibitem{4-zat}
\Au{Нонака И., Такеучи~Х.} Компания~--- создатель знания.~--- М.:  
Олимп-биз\-нес, 2003. 384~с. (\Au{Nonaka~I., Takeuchi~H.} The  
knowledge-creating company.~--- Oxford, NY, USA: Oxford University Press, 1995. 
284~p.)
\bibitem{5-zat}
\Au{Nonaka I., Toyama~R.} The knowledge-creating theory revisited: Knowledge 
creation as a synthesizing process~// Knowl. Man. Res. Pract., 
2003. Vol.~1. No.\,1. P.~2--10.
\bibitem{6-zat}
\Au{Wierzbicki A.\,P., Nakamori~Y.} Basic dimensions of creative space~// Creative 
space: Models of creative processes for knowledge civilization age~/ Eds. 
A.\,P.~Wierzbicki, Y.~Nakamori.~--- Berlin: Springer Verlag, 2006. P.~59--90.
\bibitem{7-zat}
\Au{Wierzbicki A.\,P., Nakamori~Y.} Knowledge sciences: Some new 
developments~// Z. Betriebswirt., 2007. Vol.~77. No.\,3.  
P.~271--295.
\bibitem{8-zat}
\Au{Nakamori Y.} Knowledge and systems science~--- enabling systemic knowledge 
synthesis.~--- London\,--\,New York: CRC Press, Taylor \& Francis Group, 2013. 
234~p.
\bibitem{10-zat}
\Au{Kay C., Roberts~J., Samuels~M., Wotherspoon~I., Alexander~M.} The historical 
thesaurus of English. Version~4.2.~--- Glasgow, U.K.: University of Glasgow, 2015. {\sf 
https://historicalthesaurus.arts.gla.ac.uk}.
\bibitem{11-zat}
\Au{Kay C., Roberts~J., Samuels~M., Wotherspoon~I., Alexander~M.} The historical 
thesaurus of English: Face.~--- Glasgow, U.K.: University of Glasgow, 2015.\linebreak
{\sf http://historicalthesaurus.arts.gla.ac.uk/category-selection/?qsearch=face}.
\bibitem{9-zat}
\Au{Ramiro C., Srinivasan~M., Malt~B.\,C., Xu~Y.} Algorithms in the historical 
emergence of word senses~// P.~Natl. Acad. Sci. USA, 
2018. Vol.~115. Nо.\,10. P.~2323--2328.
\bibitem{12-zat}
The British National Corpus (Oxford University Computing Services). {\sf 
www.natcorp.ox.ac.uk}.
\bibitem{13-zat}
\Au{Добровольский Д.\,О., Зализняк~Анна~А.} Немецкие конструкции 
с~модальными глаголами и~их русские соответствия: проект надкорпусной 
базы данных~//\linebreak Компьютерная лингвистика и~интеллектуальные технологии: 
По мат-лам Междунар. конф. <<Диалог>>.~--- М.: РГГУ, 2018. С.~172--184.
\bibitem{14-zat}
Параллельный немецкий корпус. {\sf http://www. ruscorpora.ru/search-para-de.html}.
\bibitem{15-zat}
\Au{Loiseau S., Sitchinava~D.\,V., Zalizniak~Anna~A., Zatsman~I.\,M.} Information 
technologies for creating the database of equivalent verbal forms in the  
Russian--French multivariant parallel corpus~// Информатика и~её применения, 
2013. Т.~7. Вып.~2. С.~100--109.
\bibitem{16-zat}
Немецко-русский словарь: актуальная лексика~/ Под ред. 
Д.\,О.~Добровольского.~--- М.: Лексрус, 2018 (в~печати).
\bibitem{17-zat}
\Au{Zatsman I., Buntman~N.} Outlining goals for discovering new knowledge and 
computerised tracing of emerging meanings discovery~// 16th European Conference 
on Knowledge Management Proceedings.~--- Reading, MA, USA: Academic Publishing 
International Ltd., 2015. P.~851--860.
\bibitem{18-zat}
\Au{Zatsman I., Buntman~N., Coldefy-Faucard~A., Nuriev~V.} WEB knowledge 
base for asynchronous brainstorming~// 17th European Conference on Knowledge 
Management Proceedings.~--- Reading: Academic Publishing International Ltd., 
2016. P.~976--983.

 \end{thebibliography}

 }
 }

\end{multicols}

\vspace*{-6pt}

\hfill{\small\textit{Поступила в~редакцию 10.06.18}}

\vspace*{6pt}

%\newpage

%\vspace*{-24pt}

\hrule

\vspace*{2pt}

\hrule

\vspace*{-2pt}


\def\tit{IMPLIED KNOWLEDGE: FOUNDATIONS AND~TECHNOLOGIES OF~EXPLICATION}



\def\titkol{Implied knowledge: Foundations and~technologies of~explication}

\def\aut{I.\,M.~Zatsman}

\def\autkol{I.\,M.~Zatsman}

\titel{\tit}{\aut}{\autkol}{\titkol}

\vspace*{-11pt}


\noindent
    Institute of Informatics Problems, Federal Research Center ``Computer Science 
and Control'' of the Russian Academy of Sciences, 44-2~Vavilov Str., Moscow 
119333, Russian Federation


\def\leftfootline{\small{\textbf{\thepage}
\hfill INFORMATIKA I EE PRIMENENIYA~--- INFORMATICS AND
APPLICATIONS\ \ \ 2018\ \ \ volume~12\ \ \ issue\ 3}
}%
 \def\rightfootline{\small{INFORMATIKA I EE PRIMENENIYA~---
INFORMATICS AND APPLICATIONS\ \ \ 2018\ \ \ volume~12\ \ \ issue\ 3
\hfill \textbf{\thepage}}}

\vspace*{3pt}
  
    
    
    
    \Abste{The theoretical foundations of the development of information technologies 
    that provide the goal-oriented creation of new linguistic typologies are described. 
    They are formed in the process of contrastive analysis of parallel aligned texts, 
    which are sources of new knowledge of the language. In parallel texts, there are 
    the implications of subjective knowledge of translators that are not represented 
    in the system of modern knowledge of language. Their explication is possible 
    with the help of information technologies, which allow processing parallel 
    texts and goal-oriented extraction of the implied knowledge. The aim of the 
    paper is to describe a~new approach to the development of technologies that 
    provide purposefulness and compare it with existing approaches and models. 
    The conditions under which the growth of objective knowledge (in terms of 
    K.\,R.~Popper) can be technologically ensured are formulated. 
    The proposed approach is illustrated by the example of the task of forming 
    a~typology of Russian language constructions with 
    a~modal value that arise in the translation of German modal constructions.}
    
    \KWE{parallel texts; corpus linguistics; implied knowledge; 
    extraction of new knowledge; emergence; 
    information technology; purposefulness; formation of typologies}
    
\DOI{10.14357/19922264180311} %

%\vspace*{-14pt}

\Ack
\noindent
The work was fulfilled at the Federal Research Center ``Computer Science 
and Control'' of the Russian Academy of Sciences and was supported by the
Russian Foundation for Basic Research (project  
18-07-00192).



%\vspace*{6pt}

  \begin{multicols}{2}

\renewcommand{\bibname}{\protect\rmfamily References}
%\renewcommand{\bibname}{\large\protect\rm References}

{\small\frenchspacing
 {%\baselineskip=10.8pt
 \addcontentsline{toc}{section}{References}
 \begin{thebibliography}{99}
  \bibitem{1-zat-1}
  \Aue{Popper, K.\,R.} 1979. \textit{Objective knowledge. An evolutionary 
approach}. Oxford: Clarendon Press. 395~p.
  \bibitem{2-zat-1}
  \Aue{Nonaka, I.} 1991. The knowledge-creating company. \textit{Harvard 
Bus. Rev.} 69(6):96--104.
  \bibitem{3-zat-1}
  \Aue{Nonaka, I.} 1994. A~dynamic theory of organizational knowledge creation. 
\textit{Organ. Sci.} 5(1):14--37.
  \bibitem{4-zat-1}
  \Aue{Nonaka, I., and H.~Takeuchi.} 1995. \textit{The knowledge-creating 
company.} Oxford, NY: Oxford University Press. 284~p.
  \bibitem{5-zat-1}
  \Aue{Nonaka, I., and R.~Toyama.} 2003. The knowledge-creating theory 
revisited: Knowledge creation as a~synthesizing process. \textit{Knowl. 
Man. Res.  Pract.} 1(1):2--10.
  \bibitem{6-zat-1}
  \Aue{Wierzbicki, A.\,P., and Y.~Nakamori.} 2006. Basic dimensions of creative 
space. \textit{Creative space: Models of creative processes for knowledge civilization 
age}. Eds. A.\,P.~Wierzbicki and Y.~Nakamori. Berlin: Springer Verlag. 59--90.
  \bibitem{7-zat-1}
  \Aue{Wierzbicki, A.\,P., and Y.~Nakamori.} 2007. Knowledge sciences: Some 
new developments. \textit{Z.~Betriebswirt.} 77(3):271--295.
  \bibitem{8-zat-1}
  \Aue{Nakamori, Y.} 2013. \textit{Knowledge and systems science~--- enabling 
systemic knowledge synthesis.} London\,--\,New York: CRC Press, Taylor \& Francis 
Group. 234~p.
  \bibitem{10-zat-1}
  \Aue{Kay, C., J.~Roberts, M.~Samuels, I.~Wotherspoon, and M.~Alexander.} 2015. The 
historical thesaurus of English. Version~4.2. Glasgow, U.K.: University of Glasgow. Available at: 
{\sf https://historicalthesaurus.arts.gla.ac.uk} (accessed May~16, 2018).
  \bibitem{11-zat-1}
  \Aue{Kay, C., J.~Roberts, M.~Samuels, I.~Wotherspoon, and M.~Alexander.} 2015. 
{The historical thesaurus of English: Face}. Glasgow, U.K.: University of Glasgow. 
Available at: {\sf https://historicalthesaurus.arts.gla.ac.uk/category-selection/?qsearch=face} 
(accessed May~5, 2018).
  \bibitem{9-zat-1}
  \Aue{Ramiro, C., M.~Srinivasan, B.\,C.~Malt, and Y.~Xu.} 2018. Algorithms in 
the historical emergence of word senses. \textit{P.~Natl. 
Acad. Sci. USA} 115(10):2323--2328.
  \bibitem{12-zat-1}
  The British National Corpus (Oxford University Computing Services). Available at: {\sf 
http://www.natcorp.ox.ac.uk} (accessed May~16, 2018).
  \bibitem{13-zat-1}
  \Aue{Dobrovol'skij, D.\,O., and Anna A.~Zalizniak.} 2018. Nemetskie konstruktsii 
s~modal'nymi glagolami i~ikh russkie sootvetstviya: proyekt nadkorpusnoy bazy dannykh 
[German constructions with modal verbs and their Russian correlates: 
A~supracorpora database 
project]. \textit{Computer Linguistics and Intellectual Technologies: 
Conference (International) ``Dialog'' Proceedings}. Moscow: RGGU.  
17(24):172--184.
  \bibitem{14-zat-1}
  Parallel'nyy nemetskiy korpus [Parallel German corpus]. Available at: {\sf 
http://www.ruscorpora.ru/search-para-de.html} (accessed May~19, 2018).
  \bibitem{15-zat-1}
  \Aue{Loiseau, S., D.\,V.~Sitchinava, Anna A.~Zalizniak, and I.\,M.~Zatsman.} 2013. 
Information technologies for creating the database of equivalent verbal forms in the Russian-French 
multivariant parallel corpus. \textit{Informatika i~ee Primeneniya~--- Inform. Appl.} 
7(2):100--109.
  \bibitem{16-zat-1}
  Dobrovol'skij, D.\,O., ed. 2018 (in press). \textit{Nemetsko-russkiy slovar': 
aktual'naya leksika }[German--Russian dictionary: Actual vocabulary]. Moscow: 
Leksrus.
  \bibitem{17-zat-1}
  \Aue{Zatsman, I., and N.~Buntman.} 2015. Outlining goals for discovering new 
knowledge and computerised tracing of emerging meanings discovery. \textit{16th 
European Conference on Knowledge Management Proceedings}. Reading, MA: Academic 
Publishing International Ltd. 851--860.
  \bibitem{18-zat-1}
  \Aue{Zatsman, I., N.~Buntman, A.~Coldefy-Faucard, and V.~Nuriev.} 2016. 
WEB knowledge base for asynchronous brainstorming. \textit{17th European 
Conference on Knowledge Management Proceedings}. Reading: Academic 
Publishing International Ltd. 976--983.
    
 \end{thebibliography}

 }
 }

\end{multicols}

\vspace*{-6pt}

\hfill{\small\textit{Received June 10, 2018}}

%\pagebreak

%\vspace*{-18pt}  
  
  \Contrl
  
\noindent
\textbf{Zatsman Igor M.} (b.\ 1952)~--- Doctor of Science in technology, 
Head of Department, Institute of Informatics Problems, Federal Research 
Center ``Computer Science and Control'' of the Russian Academy of 
Sciences, 44-2~Vavilov Str., Moscow 119333, Russian Federation; 
\mbox{izatsman@yandex.ru}
  
\label{end\stat}

\renewcommand{\bibname}{\protect\rm Литература}       

    
    