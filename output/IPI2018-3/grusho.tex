\def\stat{grusho}

\def\tit{ПАРАМЕТРИЗАЦИЯ В~ПРИКЛАДНЫХ ЗАДАЧАХ ПОИСКА 
ЭМПИРИЧЕСКИХ ПРИЧИН$^*$}

\def\titkol{Параметризация в~прикладных задачах поиска 
эмпирических причин}

\def\aut{А.\,А.~Грушо$^1$, Н.\,А.~Грушо$^2$, М.\,И.~Забежайло$^3$, 
Д.\,В.~Смирнов$^4$, Е.\,Е.~Тимонина$^5$}

\def\autkol{А.\,А.~Грушо, Н.\,А.~Грушо, М.\,И.~Забежайло и~др.}

\titel{\tit}{\aut}{\autkol}{\titkol}

\index{Грушо А.\,А.}
\index{Грушо Н.\,А.}
\index{Забежайло М.\,И.}
\index{Смирнов Д.\,В.}
\index{Тимонина Е.\,Е.}
\index{Grusho A.\,A.}
\index{Grusho N.\,A.}
\index{Zabezhailo M.\,I.}
\index{Smirnov D.\.V.}
\index{Timonina E.\,E.}




{\renewcommand{\thefootnote}{\fnsymbol{footnote}} 
\footnotetext[1]
{Работа выполнена при частичной финансовой поддержке 
РФФИ (проекты 18-07-00274-а, 15-29-07981-офи-м).}}


\renewcommand{\thefootnote}{\arabic{footnote}}
\footnotetext[1]{Институт проблем информатики Федерального исследовательского центра 
<<Информатика и~управление>> 
Российской академии наук, \mbox{grusho@yandex.ru}}
\footnotetext[2]{Институт проблем информатики Федерального исследовательского центра 
<<Информатика и~управление>> 
Российской академии наук, \mbox{info@itake.ru@ipiran.ru}}
\footnotetext[3]{Вычислительный центр им.\ А.\,А.~Дородницына Федерального 
исследовательского центра <<Информатика и~управление>> Российской академии 
наук, \mbox{m.zabezhailo@yandex.ru}}
\footnotetext[4]{ПАО Сбербанк России, dvlsmirnov@sberbank.ru}
\footnotetext[5]{Институт проблем информатики Федерального исследовательского центра 
<<Информатика и~управление>> 
Российской академии наук, \mbox{eltimon@yandex.ru}}

%\vspace*{8pt}
  
  
  
  \Abst{Представление конечного класса объектов в~форме множества характеристик 
(параметров) этих объектов назовем параметризацией рассматриваемого класса. Кроме 
идентификации объектов множествами характеристик существует задача выявления причины 
того, что некоторые объекты класса обладают свойством~$P$. Для решения этой задачи 
в~условиях появления новых объектов исходного множества характеристик может не хватить. 
В~этом случае необходимо изменять параметризацию. Работа посвящена построению методов 
изменения начальной параметризации в~задаче уточнения эмпирической причины появления 
свойства~$P$ при расширении исходных данных. Построенные методы продемонстрированы на 
практических примерах.}
  
  \KW{ДСМ-методы искусственного интеллекта; параметризация классов объектов; 
эмпирическая причина; аутентификация}

\DOI{10.14357/19922264180309}
  
%\vspace*{4pt}


\vskip 10pt plus 9pt minus 6pt

\thispagestyle{headings}

\begin{multicols}{2}

\label{st\stat}
  
\section{Введение}
  
  Работа посвящена исследованию следующей проблемы. Если в~классе объектов 
часть их обладает свойством~$P$, а~часть не обладает этим свойством, то 
возникает задача выявления причины появления свойства~$P$ у~части объектов. 
Объекты описываются множеством характеристик этих объектов, или 
параметров~--- по Эшби~[1]. Поэтому можно отож\-де\-ст\-влять объекты 
и~соответствующие множества их характеристик.
  
  Множество характеристик полно, если каждый объект однозначно выделяется 
по своему подмножеству характеристик. Однако при исследовании причины 
появления свойства~$P$ в~некоторых объектах можно прийти к~выводу, что 
полнота множества характеристик не гарантирует описания причин появления 
свойства~$P$. Тогда необходимо изменять параметризацию класса объектов 
(множество параметров описания объектов), уточняя ее в~такой степени, чтобы 
в~этом описании содержались причины появления свойства~$P$.
  
  В работе рассматривается несколько способов расширения множества 
характеристик данного класса объектов при сохранении свойства 
полноты.
  
\section{Эмпирические причины}

\vspace*{-12pt}
  
  Рассмотрим простейшую модель ДСМ-ме\-то\-да~[2] интеллектуального 
анализа данных (ИАД), построенную на языке теории множеств. Пусть $U\hm= 
\{u_1, \ldots , u_m\}$ является множеством ха\-рак\-те\-ристик наблюдаемых объектов 
$O_1,\ldots , O_n$, т.\,е.\linebreak
 объекты полностью описываются наборами 
характеристик из~$U$. Можно считать, что любой объект~$O$~--- это 
подмножество~$U$, а~множество всех возможных объектов~--- это множество 
всех подмножеств множества~$U$. Само множество~$U$ будем называть 
параметризацией наблюдаемых объектов. Кроме характеристик определим 
понятие свойства объекта. Свойство~$P$ объекта~$O$ отражает некоторую 
интегральную характеристику объекта~$O$, которая может присутствовать в~$O$ 
или отсутствовать в~$O$. Если объекты появляются последовательно, то вновь 
появившийся объект должен проверяться на наличие свойства~$P$.
  
  Предположим, что рассматриваемое свойство~$P$ удовлетворяет следующим 
условиям:\\[-15pt]
  \begin{itemize}
\item[(А)] если свойство~$P$ выявлено в~объектах~$O_1$ и~$O_2$, то оно есть 
в~объекте $O\hm=O_1\cap O_2$;
\item[(В)] если объект~$O$ обладает свойством~$P$, то для любого 
объекта~$O_1$, содержащего~$O$, объект~$O_1$ также обладает~$P$.
\end{itemize}

  Определим понятие эмпирической причины некоторого свойства~$P$ 
в~наблюдаемых данных. Основываясь на понимании причины Д.\,С.~Миллем~[3], 
в~рассматриваемой модели эмпирической причиной свойства~$P$ для множества 
объектов $O_1, O_2, \ldots ,O_l$ будем называть производный объект $O\hm= 
\bigcap\limits_i O_i$ по всем индексам~$i$, для которых~$O_i$ обладает 
свойством~$P$, но ни один из остальных~$O_j$ свойством~$P$ не обладает. 
Причина должна быть единственной, и~от нее ничего нельзя убавить или к~ней 
добавить~[4].
  
  \smallskip
  
  \noindent
  \textbf{Пример~1.}\ Пусть $U\hm=\{ \mathrm{а}, \mathrm{б}, \ldots , 
\mathrm{я}\}$~--- русский алфавит. Свойство~$P$ означает множество букв, из 
которых можно собрать слово, выражающее понятие <<дом>>. Пусть есть два 
объекта $O_1\hm= \{\mathrm{д}, \mathrm{о}, \mathrm{м}\}$ и~$O_2\hm = 
\{\mathrm{с}, \mathrm{у}, \mathrm{к}\}$. Ясно, что~$O_1$ является 
эмпирической причиной свойства~$P$. При этом~$O_2$ не обладает 
свойством~$P$.
  
\section{Различные параметризации}

  Однозначное представление объектов в~виде подмножеств множества 
характеристик не означает, что не существует другого множества характеристик 
для описания того же множества наблюдаемых объектов и~свойств.
  
  \smallskip
  
  \noindent
  \textbf{Пример~2.}\ Понятие <<дом>> можно выразить на английском языке 
словом \textit{home}. Тогда множество характеристик есть английский алфавит, 
а~аналогичное свойство примера~1 выражается другой причиной, выраженной 
другим множеством характеристик.
  
  \smallskip
  
  Представление объектов из множества характеристик можно строить 
с~помощью упорядоченных\linebreak
 наборов характеристик. Тогда множество 
описываемых объектов совпадает с~множеством слов ко\-нечной длины~$U^*$ из 
исходного множества характеристик. Это множество бесконечное 
и~$U^{**}\hm=U^*$.\linebreak Эмпирическая причина для~$U^*$ определяется словом, 
описывающим интегральное свойство~$P$ в~языке~$U^*$. При этом 
свойство~$P$ примера~1 выражается проще. Объект <<дом>> обладает 
свойством~$P$, а~объект <<мода>> свойством~$P$ уже не обладает. Однако 
в~данном представлении объектов может возникнуть ситуация, когда 
эмпирическая причина отсутствует.
  
  \smallskip
  
  
  \noindent
  \textbf{Пример~3.}\ Объект <<дом>> соответствует понятию, описанному 
словом \textit{дом}, и~является эмпирической причиной свойства~$P$, а~объект 
<<строение>> не является словом \textit{дом}, но соответствует смыслу 
свойства~$P$. Таким образом, возникает объект, не обладающий эмпирической 
причиной и~одновременно обладает свойством~$P$.
  
  \smallskip
  
  Чаще всего добавление новых объектов фальсифицирует эмпирическую 
причину, найденную по исходному набору объектов. Фальсификация происходит 
в~следующих формах:
  \begin{enumerate}[(1)]
\item найденная эмпирическая причина появляется в~новом объекте, который не 
обладает свойством~$P$;
\item найденная эмпирическая причина не появляется в~новом объекте, который, 
как предполагается, обладает свойством~$P$.
\end{enumerate}

  Рассмотрим некоторые другие примеры изменения параметризации, 
позволяющие тоньше идентифицировать причины их фальсификации.
  
\section{Схема аутентификации как~поиск~причины 
в~модифицированной параметризации}

  В данном разделе рассматривается задача изменения параметризации за счет 
добавления множеств характеристик из различных информационных пространств.
  
  Пусть $A$~--- субъект, который должен подтвердить субъекту~$W$ свое имя. 
Эта процедура называется аутентификацией~\cite{5-gr} и~может быть проведена 
следующим образом.
  
  Пусть $\Sigma_1$~--- это информационное пространство, содержащее объекты 
$O_1^{(1)}, O_2^{(1)},\ldots , O_{s_1}^{(1)}$. Каждый из этих объектов 
классифицируется двумя значениями~$t$ (truth) и~$f$ (false). Свойство~$P$ 
состоит в~знании для этих объектов правильного вектора $x^{(1)}\hm= \left( 
x_1^{(1)}, x_2^{(1)}, \ldots , x^{(1)}_{s_1}\right)$, где 
  $$
  x_i^{(1)}= \begin{cases}
  t\,;\\ f
  \end{cases}
  \mbox{для } i=1, \ldots , s_1\,. 
  $$
  
  Пусть вектор $x^{(1)}$ известен~$A$ и~$W$. Предположим, что существует 
субъект~$B$, который назвал себя именем~$A$ и~также пытается подтвердить это 
имя. Возможны два случая.
  \begin{enumerate}[1.]
  \item $B$ случайно выбирает значение вектора $x^{(1)}\hm= \left( x_1^{(1)}, 
x_2^{(1)}, \ldots , x_{s_1}^{(1)}\right)$.
\item Пространство~$\Sigma_1$ известно~$B$, и~он знает вектор~$x^{(1)}$.
\end{enumerate}

  Субъект~$W$ проверяет знание вектора~$x^{(1)}$, считая что он общается 
  с~некоторым субъектом~$\Phi$, который может принимать значения $\Phi\hm=A$ 
или~$B$. Субъект~$W$ предъявляет объекты $O_1^{(1)}, O_2^{(1)}, 
\ldots , O_{s_1}^{(1)}$ субъекту~$\Phi$. Тогда, получив вектор~$x^{(1)}$, 
субъект~$W$ может считать, что субъект~$\Phi$ знает эмпирическую причину 
свойства~$P$. При этом эту причину может знать как~$A$, так и~$B$. 
Причем~$B$ может случайно угадать вектор~$x^{(1)}$ (ложная аутентификация).
  
  Проверка имени~$A$ может быть продолжена, если к~информационному 
пространству~$\Sigma_1$ добавить информационные пространства 
$\Sigma_2,\ldots , \Sigma_m$. Из характеристик каждого из добавленных 
пространств можно сформировать объекты $O_1^{(i)}, O_2^{(i)}, \ldots , 
O^{(i)}_{s_i}$, $i\hm=2, \ldots ,m$, для которых~$W$ и~$A$ знают 
векторы~$x^{(i)}$, $i\hm=2,\ldots , m$. Таким образом, пара субъектов~$W$ 
и~$A$ расширяет параметризацию для определения причин свойства~$P$ до 
знания векторов $x^{(1)}, \ldots , x^{(m)}$.
  
  Если субъект~$B$ выбирает значение вектора~$x_j^{(i)}$ случайно, то 
с~вероятностью, как угодно близкой к~1, он на ка\-ком-то шаге ошибется. Это 
позволит субъекту~$W$ понять, что субъект~$\Phi$ не обладает свойством~$P$.
  
  Если субъект~$B$ скомпрометировал пространства $\Sigma_{i_1}, \ldots , 
\Sigma_{i_r}$, $r\hm<m$, то субъект~$W$ знает о~возможной компрометации  
ка\-ких-то информа\-ци\-онных пространств, но не знает каких. Тогда\linebreak субъект~$W$ 
предъявляет~$\Phi$~объекты, созданные\linebreak
 в~информационных пространствах 
$\Sigma_1, \Sigma_2, \ldots ,\Sigma_m$, попадает на объекты 
нескомпрометированных пространств и~определяет фальсификацию 
эмпирической причины~$P$. Если субъект~$\Phi$ определяет правильно все 
векторы $x^{(1)}, \ldots , x^{(m)}$, то причина~$P$ подтверждена 
и~аутентификация прошла успешно.
  
  Рассмотренный пример показывает, что расширение множества характеристик 
за счет привлечения дополнительной информации позволяет уточнять причину 
исследуемого свойства~$P$. Идея подтвержде\-ния эмпирической причины за счет 
расширения исходной параметризации с~помощью добавления характеристик 
рассматривалась в~работах~\cite{6-gr, 7-gr}.
  
\section{Изменение параметризации за~счет разбиения параметров}
   
   Пусть $U\hm=\{u_1, \ldots , u_m\}$~--- это хосты сети. Пусть свойство~$P$ 
соответствует обработке информационной технологии за время $\tau\hm> T_0$. 
Пусть объект~$O$~--- это множество хостов, выделяемое провайдером для 
реализации информационной технологии. При повторах технологии выделяется 
множество объектов $\{O^+\}$, на которых наблюдаются задержки $\tau\hm > 
T_0$ выполнения информационной технологии, и~множество объектов~$\{O^-\}$, 
для которых $\tau\hm\leq T_0$, т.\,е.\ в~сети возникает так называемая 
<<мерцающая>> ошибка~\cite{8-gr, 9-gr}.
   
   Предположим, что причина~$o$ свойства~$P$ ищется с~помощью пересечения 
объектов из множества~$\{O^+\}$. Эмпирическая причина является подобъектом~$o$, 
если $o$~не встречается в~множестве~$\{O^-\}$. Пусть к~множеству объектов 
добавляется новый объект~$O$ и~пусть $o\hm\subseteq O$, но информационная 
технология реализуется так, что $\tau\hm\leq T_0$. Таким образом, происходит 
фальсификация эмпирической при\-чины.
   
   При исследовании характеристик~$U$ оказалось, что~$u_1$~--- это 
локальная сеть из двух машин $u_1^{(1)}$ и~$u_1^{(2)}$ с~прок\-си-сер\-ве\-ром 
для выхода в~общую сеть. Для реализации информационной технологии любая из 
этих машин выбирается случайно. Оказывается, что~$u_1^{(2)}$ работает всегда 
быстро, а~$u_1^{(1)}$~--- всегда медленно. Рассмотрим новое пространство 
характеристик $U^\prime\hm= \left\{ u_1^{(1)}, u_1^{(2)}, u_2, \ldots , u_m\right\}$. 
Тогда в~новом множестве характеристик причина свойства $P\hm= (\tau\hm > 
T_0)$ определяется тем же методом, что и~ранее, но эмпирическая причина не 
фальсифицируется.

\vspace*{-6pt}
   
\section{Заключение}
  
  Выявление устойчивых при увеличении исходных данных эмпирических 
причин исследуемых свойств не всегда возможно в~условиях исходной 
параметризации. Изменение параметризации требует выполнения полноты 
описания известных и~вновь поступающих объектов.
  
  Возможно дополнение изменения параметризации требованием сохранения 
описания эмпирических причин других свойств, поэтому необходимо строить 
новую параметризацию путем уточнения старой параметризации. В~работе 
рассмотрены примеры построения таких уточняющих параметризаций. Возможны 
гибридные варианты применения изложенных методов.
  
  В дальнейшем предполагается исследовать возможность доказательства того, 
что других путей уточнения параметризации нет.

\vspace*{-6pt}
  
{\small\frenchspacing
 {%\baselineskip=10.8pt
 \addcontentsline{toc}{section}{References}
 \begin{thebibliography}{9}
\bibitem{1-gr}
\Au{Эшби У.\,Р.} Конструкция мозга~/ Пер. с~англ.~--- М.: Иностранная литература, 1962. 
397~с. (\Au{Ashby~W.\,R.} Design for a~brain.~--- New York, NY, USA: Wiley, 
1954. 260~p.)
\bibitem{2-gr}
\Au{Аншаков О.\,М., Фабрикантова~Е.\,Ф.} ДСМ-ме\-тод автоматического порождения гипотез: 
Логические и~эпи-\linebreak\vspace*{-12pt}

\pagebreak

\noindent
стемологические основания.~--- М.: Либроком, 2009. 432~с.
\bibitem{3-gr}
\Au{Милль Дж.\,С.} Система логики силлогической и~индуктивной: Изложение принципов 
доказательства в~связи с~методами научного исследования~/ Пер. с~англ.~--- 5-е изд., испр.  
и~доп.~--- М.: ЛЕНАНД, 2011. 832~с. (\Au{Mill~J.\,S.} A~system of logic ratiocinative and 
inductive, being a~connected view of the principles of evidence and the methods of scientific 
investigation.~--- 1st ed.~--- London: John W.~Parker, 1843. 622~p.)
\bibitem{4-gr}
\Au{Финн В.\,К.} Искусственный интеллект: Методология, применения, философия.~--- М.: 
Красанд, 2011. 448~с.
\bibitem{5-gr}
\Au{Грушо А.\,А., Применко~Э.\,А., Тимонина~Е.\,Е.} Теоретические основы компьютерной 
безопасности.~--- М.: Академия, 2009. 
272~с.
\bibitem{6-gr}
\Au{Грушо А.\,А., Забежайло~М.\,И., Смирнов~Д.\,В., Тимонина~Е.\,Е.} О~комплексной 
аутентификации~// Сис\-те\-мы и~средства информатики, 2017. Т.~27. №\,3.\linebreak
 С.~3--10.
\bibitem{7-gr}
\Au{Грушо А.\,А., Забежайло~М.\,И., Смирнов~Д.\,В., Тимонина~Е.\,Е.} Модель множества 
информационных пространств в~задаче поиска инсайдера~// Информатика и~её применения, 
2017. Т.~11. Вып.~4. С.~65--69.
\bibitem{8-gr}
\Au{Грушо А.\,А., Забежайло~М.\,И., Зацаринный~А.\,А., Николаев~А.\,В., Писковский~В.\,О., 
Тимонина~Е.\,Е.} Классификация ошибочных состояний в~распределенных вычислительных 
системах и~источники их возникновения~// Системы и~средства информатики, 2017. Т.~27. №\,2. 
С.~30--41.
\bibitem{9-gr}
\Au{Грушо А.\,А., Забежайло~М.\,И., Зацаринный~А.\,А., Николаев~А.\,В., Писковский~B.\,О., 
Сенчило~В.\,В., Судариков~И.\,В., Тимонина~Е.\,Е.} Об анализе ошибочных состояний 
в~распределенных вычислительных системах~// Системы и~средства информатики, 2018. Т.~28. 
№\,1. С.~99--109.
 \end{thebibliography}

 }
 }

\end{multicols}

\vspace*{-6pt}

\hfill{\small\textit{Поступила в~редакцию 10.06.18}}

\vspace*{8pt}

%\newpage

%\vspace*{-24pt}

\hrule

\vspace*{2pt}

\hrule

\vspace*{-2pt}


\def\tit{PARAMETRIZATION IN~APPLIED PROBLEMS OF~SEARCH OF~EMPIRICAL REASONS}

\def\titkol{Parametrization in~applied problems of search of 
empirical reasons}

\def\aut{A.\,A.~Grusho$^1$, N.\,A.~Grusho$^1$, M.\,I.~Zabezhailo$^2$, D.\,V.~Smirnov$^3$, 
and~E.\,E.~Timonina$^1$}

\def\autkol{A.\,A.~Grusho, N.\,A.~Grusho, M.\,I.~Zabezhailo, et al.}
%D.\,V.~Smirnov$^3$,  and~E.\,E.~Timonina$^1$}

\titel{\tit}{\aut}{\autkol}{\titkol}

\vspace*{-11pt}


\noindent
$^1$Institute of Informatics Problems, Federal Research Center ``Computer Sciences and 
Control'' of the Russian\linebreak
$\hphantom{^1}$Academy of Sciences, 44-2~Vavilov Str., Moscow 119333, 
Russian Federation

\noindent
$^2$A.\,A.~Dorodnicyn Computing Center, Federal Research Center ``Computer Sciences 
and Control'' of the Russian
\linebreak
$\hphantom{^1}$Academy of Sciences, 40~Vavilov Str., Moscow 119133, 
Russian Federation

\noindent
$^3$Sberbank of Russia, 19~Vavilov Str., Moscow 117999, Russian Federation


\def\leftfootline{\small{\textbf{\thepage}
\hfill INFORMATIKA I EE PRIMENENIYA~--- INFORMATICS AND
APPLICATIONS\ \ \ 2018\ \ \ volume~12\ \ \ issue\ 3}
}%
 \def\rightfootline{\small{INFORMATIKA I EE PRIMENENIYA~---
INFORMATICS AND APPLICATIONS\ \ \ 2018\ \ \ volume~12\ \ \ issue\ 3
\hfill \textbf{\thepage}}}

\vspace*{3pt}




\Abste{The authors define description of a~finite class of objects in the form 
of a~set of characteristics (parameters) of these objects as parametrization 
of the considered class. Besides identification of objects, the problem of 
the causality that some objects have property~$P$ exists. For the solution of this 
task, in the conditions of emergence of new objects, an initial set of 
characteristics cannot be enough. In this case, it is necessary to change 
parametrization. The paper is devoted to creation of methods of changes of 
the initial parametrization in the problem of specification of the empirical 
causality of emergence of the property~$P$ at expansion of 
basic data. The constructed methods are shown on practical examples.}

\KWE{JSM-methods of artificial intelligence; parametrization of classes 
of objects; empirical reason; authentication}
   


\DOI{10.14357/19922264180309}

\vspace*{-10pt}

 \Ack
\noindent
  The paper was partially supported by the Russian Foundation for Basic Research (projects  
18-07-00274-a and 15-29-07981-ofi-m).



\vspace*{4pt}

  \begin{multicols}{2}

\renewcommand{\bibname}{\protect\rmfamily References}
%\renewcommand{\bibname}{\large\protect\rm References}

{\small\frenchspacing
 {%\baselineskip=10.8pt
 \addcontentsline{toc}{section}{References}
 \begin{thebibliography}{9}
 
 \vspace*{-4pt}
 
\bibitem{1-gr-1}
\Aue{Ashby, W.\,R.} 1954. \textit{Design for a~brain}. New York, NY: Wiley. 260~p.
\bibitem{2-gr-1}
\Aue{Anshakov, O.\,M., and E.\,F.~Fabrikantova.} 
2009. \textit{DSM-metod avtomaticheskogo porozhdeniya gipotez: 
Logicheskie i~epistemologicheskie osnovaniya} [JSM-method of automatic hypothesis generation: 
Logical and epistemological]. Moscow: Librokom. 432~p.
\bibitem{3-gr-1}
\Aue{Mill, J.\,S.} 1843. \textit{A~system of logic ratiocinative and inductive, being a~connected 
view of the principles of evidence and the methods of scientific investigation.} 1st ed. London: John 
W.~Parker. 622~p.
\bibitem{4-gr-1}
\Aue{Finn, V.\,K.} 2011. \textit{Iskusstvennyy intellekt: Metodologiya, primeneniya, 
filosofiya} 
[Artificial intelligence: Methodology, applications, philosophy]. Moscow: Krasand.\linebreak
 448~p.
\bibitem{5-gr-1}
\Aue{Grusho, A., Ed.~Primenko, and E.~Timonina.} 2009. \textit{Teoreticheskie osnovy 
komp'yuternoy bezopasnosti} [Theoretical 
bases of computer security]. Moscow: 
Academy. 272~р.
\bibitem{6-gr-1}
\Aue{Grusho, A.\,A., M.\,I.~Zabezhailo, D.\,V.~Smirnov, and E.\,E.~Timonina.} 
2017. O~kom\-pleks\-noy autenti\-fi\-ka\-tsii [About complex authentication]. \textit{Sistemy i~Sredstva 
Informatiki~--- Systems and Means of Informatics} 27(3):\linebreak 3--10.
\bibitem{7-gr-1}
\Aue{Grusho, A.\,A., M.\,I.~Zabezhailo, D.\,V.~Smirnov, and E.\,E.~Timonina.} 2017. Model' 
mnozhestva informatsionnykh prostranstv v~zadache poiska insaydera [The model of the set of 
information spaces in the problem of insider detection]. \textit{Informatika i~ee Primeneniya~--- 
Inform. Appl.} 11(4):65--69.
\bibitem{8-gr-1}
\Aue{Grusho, A.\,A., M.\,I.~Zabezhailo, A.\,A.~Zatsarinnyy, A.\,V.~Nikolaev, V.\,O.~Piskovski, 
and E.\,E.~Timonina.} 2017. Klassifikatsiya oshibochnykh sostoyaniy v~raspredelennykh 
vychislitel'nykh sistemakh i~istochniki ikh vozniknoveniya [Erroneous states classification in 
distributed computing systems and sources of their occurrence]. \textit{Sistemy i~Sredstva 
Informatiki~--- Systems and Means of Informatics} 27(3):30--41.
\bibitem{9-gr-1}
\Aue{Grusho, A.\,A., M.\,I.~Zabezhailo, A.\,A.~Zatsarinnyy, A.\,V.~Nikolaev, V.\,O.~Piskovski, 
V.\,V.~Senchilo, I.\,V.~Sudarikov, and E.\,E.~Timonina.} 2018. Ob analize oshibochnykh 
sostoyaniy v~raspredelennykh vychislitel'nykh sistemakh [About the analysis of erratic statuses in 
the distributed computing systems]. \textit{Sistemy i~Sredstva Informatiki~--- Systems and Means 
of Informatics} 28(1):99--109.
\end{thebibliography}

 }
 }

\end{multicols}

\vspace*{-6pt}

\hfill{\small\textit{Received June 10, 2018}}

%\pagebreak

%\vspace*{-18pt}

\Contr

\noindent
\textbf{Grusho Alexander A.} (b.\ 1946)~--- Doctor of Science in physics and 
mathematics, professor; principal scientist, Institute of Informatics Problems, Federal 
Research Center ``Computer Sciences and Control'' of the Russian Academy of 
Sciences, 44-2~Vavilov Str., Moscow 119133, Russian Federation; 
\mbox{grusho@yandex.ru}

\vspace*{3pt}

\noindent
\textbf{Grusho Nikolai A.} (b.\ 1982)~--- Candidate of Science (PhD) in physics and 
mathematics, senior scientist, Institute of Informatics Problems, Federal Research 
Center ``Computer Sciences and Control'' of the Russian Academy of Sciences, 
44-2~Vavilov Str., Moscow 119133, Russian Federation; \mbox{info@itake.ru}

\vspace*{3pt}

\noindent
\textbf{Zabezhailo Michael I.} (b.\ 1956)~--- Doctor of Science in physics and 
mathematics, associate professor; principal scientist, A.\,A.~Dorodnicyn Computing 
Center, Federal Research Center ``Computer Sciences and Control'' of the Russian 
Academy of Sciences, 40~Vavilov Str., Moscow 119133, Russian Federation; 
\mbox{m.zabezhailo@yandex.ru}

\vspace*{3pt}

\noindent
\textbf{Smirnov Dmitry V.} (b.\ 1984)~--- business partner for IT security 
department of Sberbank of Russia, 19~Vavilov Str., Moscow 117999, Russian 
Federation; \mbox{dvlsmirnov@sberbank.ru}

\vspace*{3pt}

\noindent
\textbf{Timonina Elena E.} (b.\ 1952)~--- Doctor of Science in technology, 
professor; leading scientist, Institute of Informatics Problems, Federal Research 
Center ``Computer Sciences and Control'' of the Russian Academy of Sciences,  
44-2~Vavilov Str., Moscow 119133, Russian Federation; 
\mbox{eltimon@yandex.ru}

 

\label{end\stat}


\renewcommand{\bibname}{\protect\rm Литература} 