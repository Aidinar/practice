\documentclass[10pt]{book}
\usepackage[utf8]{inputenc}

\usepackage{latexsym,amssymb,amsfonts,amsmath,amsxtra,indentfirst,shapepar,%fleqn,%
picinpar,shadow,floatflt,enumerate,multicol,colortbl,moreverb,cite,ipi}

\usepackage{rotating}
\usepackage{mathrsfs}
\usepackage[noend]{algorithmic}
\usepackage{ulem}
\usepackage{graphicx}
%\usepackage{algorithm2e}
\usepackage[linesnumbered,boxed,ruled]{algorithm2e}
%\usepackage{xypic}
\usepackage{oldgerm}
\usepackage{epic}
\usepackage{eepic}


\SetAlgorithmName{Алгоритм}{алгоритм}{Список алгоритмов}

%из Дюковой

\newcommand{\algKeyword}[1]{{\bf #1}}
\newcommand{\Proc}[1]{\text{\tt #1}}
\def\CALL{\algKeyword{call}~}

\newenvironment{AlgProcedure}[1]
{
    \small
    \medskip
    %    \hrule
    \medskip
    \algKeyword{PROCEDURE} #1
    \begin{algorithmic}[1]}
    {\end{algorithmic}
    %    \hrule
    \bigskip
}

\def\CALL{\algKeyword{call}~}

%конец для Дюковой

%\RequirePackage[ruled]{algorithm}


\input{epsf}

%\nofiles

%\includeonly{avtor} %+pdf+
%\includeonly{obchak,avtor}
%\includeonly{pred}                 %+
%\includeonly{podgot-rus-site,podgot-eng-site}  
%\includeonly{ocherk} 
%\includeonly{nekrol} 
%\includeonly{ipi-ind} 
%\includeonly{toc-rus}
%\includeonly{toc-en} 


%\includeonly{kon-raz}            %+pdf
%\includeonly{shestakov}          %без рис+pdf
%\includeonly{kudr}               %без рис+pdf
%\includeonly{gorshenin}          %+pdfpdf
%\includeonly{nazarov}            %без рис+pdf 
%\includeonly{sopin}              %без рис+pdf??
%\includeonly{gorbunova}          %+рис+pdf
%\includeonly{krivenko}           %+рис+pdf
%\includeonly{grusho}             % без рис+ pdf
%\includeonly{stupnikov}          %+без рис+pdf
%\includeonly{zatsman}            %без рис+ pdf
%\includeonly{inkova}             %+pdf
%\includeonly{kozerenko}          %+pdf
%\includeonly{bosov}              %без рис+pdf
%\includeonly{bosov-naumov}       %+pdf???
%\includeonly{borisov}            %безрис16+ pdf





%\includeonly{toc-rus, toc-en}
%\includeonly{obchak} %,toc-en}
%\includeonly{rekl}
%\includeonly{rekl-1}
%\includeonly{reshal}  %
%\includeonly{eng-index}
%\includeonly{cover3}

\usepackage{acad}
%\usepackage{courier}
\usepackage{decor}
\usepackage{newton}
\usepackage{pragmatica}
\usepackage{zapfchan}
\usepackage{petrotex}
\usepackage{bm}                     % полужирные греческие буквы
\usepackage{upgreek}                % прямые греческие буквы
\usepackage{eufrak}
\usepackage{verbatim}

\renewcommand{\bottomfraction}{0.99}
\renewcommand{\topfraction}{0.99}
\renewcommand{\textfraction}{0.01}

\setcounter{secnumdepth}{1} %здесь - 3 + chapter = 4

\arraycolsep=1.5pt

%\usepackage[pdftex]{graphicx}

%\usepackage{oz}

%NEW COMMANDS


\renewcommand*{\hm}[1]{#1\nobreak\discretionary{}%
            {\hbox{$\mathsurround=0pt #1$}}{}} %% Дублирует знаки операций
                               %при переносе в формуле (перед знаком, который
                               %надо продублировать ставится команда \hm)

%\newcommand{\endproof}{\hfill$\Box$}
\renewcommand{\r}{\mathbb{R}}
%\newcommand{\I}{{\rm I\hspace{-0.7mm}I}}
%\newcommand{\Ikl}{{\tt{1}}\hspace*{-1.44mm}\mathtt{1}}
\newcommand{\Ik}{\mbox{{\small \tt {1}}\hspace{-1.3mm}{\tt 1}}}
\newcommand{\argmin}{\mathop{\mathrm{arg}\,\mathrm{min}}}
\newcommand{\argmax}{\mathop{\mathrm{arg}\,\mathrm{max}}}
%\newcommand{\capr}{\mathop{\cap\,}}
%\newcommand{\cupr}{\mathop{\cup\,}}
%\def\argmin{\mathop{arg\,min}}

\def\vrp{\varphi}
\def\prt{\partial}
\def\mm{{\sf M}}
\def\modnop#1{\mathop{#1}\limits_{n}}
\def\eam{\mathbin{{\mathop{=}\limits^{\mathrm{def}}}}}
\def\dey#1#2{#1 (#2)}
\def\deyc#1#2{#1 \cdot  #2}
\def\ra#1{\;\mathop{\to}\limits^{#1}\;}
\def\raz#1{\;\mathop{\longrightarrow}\limits^{\!\!\!#1}\;}
\def\ral#1{\;\mathop{\longrightarrow}\limits^{#1}\;}

\newcommand{\Nor}{\mathcal{N}}
\newcommand{\T}{\mathbb{T}}
\newcommand{\Z}{\mathbb{Z}}



\newcommand{\il}[2]{\int\limits_{#1}^{#2}}%интеграл с пределами #1 и #2

\def\sm2{\mathop {\sum\limits^{n^\Theta}\sum\limits^{n^\Theta}}}
\def\sss{\sum\limits}
\def\tr{,\,\ldots\,,\,}
\def\rk{\right]}
\def\lk{\left[}
\def\rf{\right\}}
\def\lf{\left\{}
\def\lv{\,\left\vert}
\def\rv{\right\vert\,}
\def\iii{\int\limits}
\def\iin{\int\limits_{-\infty}^\infty}
\def\rrv{\right\vert}


\def\ee{{\cal E}}
\def\ww{{\cal W}}
\def\yy{{\cal Y}}
\def\vv{{\cal V}}

\newcommand{\R}{\mathbb R}
\newcommand{\E}{\mathbb E}
\newcommand{\N}{\mathbb N}

\renewcommand{\P}{\mathbb{P}}

\newcommand{\h}{{\bf H}}
\newcommand{\p}{{\sf P}}  % вероятность

\newcommand{\e}{{\sf E}}  % мат. ожидание
\newcommand{\D}{{\sf D}}  % дисперсия
\newcommand{\eps}{\varepsilon}
\newcommand{\vp}{{\mathbf p}}
\newcommand{\vz}{{\mathbf z}}
\newcommand{\vx}{{\mathbf x}}
\newcommand{\vf}{{\mathbf f}}
\newcommand{\F}{{\mathcal F}}
\def\ap{{\mathrm{ЭР}}}
\newcommand{\ud}{\Delta_n} %uniform ditance
\newcommand{\nud}{\Delta_n(x)}
%\renewcommand{\Re}{\mathrm{Re}\,}

\newcommand{\abs}[1]{\left\vert#1\right\vert}

\newcommand{\norm}[1]{\left\Vert#1\right\Vert}
\def\da{(\Delta_t,A)}

\newcommand{\corr}{\mathrm{corr}}

\newcommand{\cov}{\mathrm{cov}}
\newcommand{\Expect}{\mathbb{E}}

\def\w{\omega}
\def\W{\Omega}

\def\inh{\int\limits_{nh}^{(n+1)h}}

\def\sumin{\sum_{i=1}^N}


\def\bxt{(Y,t)}
\def\xt{(y,t)}

\def\ovth{{\fr{\tau-nh}{h}}}
\def\ov{\overline}
\def\tm{\tilde m}
\def\tl{\tilde\lambda}
\def\tB{\widetilde B}
\def\tb{\tilde b}
\def\ld{\ldots}
\def\cd{\cdots}


\DeclareMathOperator{\sign}{sign}

%\newcommand{\gr}{{\geqslant}}


\newcommand{\g}{\mbox{\textit{g}}}

\renewcommand{\la}{\lambda}
\newcommand{\si}{\sigma}
\newcommand{\alp}{\alpha}

%\newcommand{\pto}{\stackrel{P}{\longrightarrow}} % сходимость по веpоятности

\newcommand{\eqd}{\stackrel{\mathrm{d}}{=}} % равенство по pаспpеделению
\newcommand{\eqdelta}{\stackrel{\triangle}{=}} % равенство по pаспpеделению

\def\be#1{\begin{equation}\label{#1}}
\def\ee{\end{equation}}
\def\re#1{(\ref{#1})}

\def\bn{\begin{enumerate}}
\def\en{\end{enumerate}}
\def\bi{\begin{itemize}}
\def\ei{\end{itemize}}
%\def\i{\item}

%\newcommand{\kp}{\kappa}
%\def\Q{{\cal Q}} \def\H{{\cal H}}
%\newcommand{\bet}{\beta_{2+\delta}}


%\newtheorem{definition}{Определение}
%\renewcommand{\thedefinition}{\arabic{definition}.}
%END NEW COMMANDS

%\renewcommand{\baselinestretch}{1.2}

%\pagestyle{myheadings}

\setlength{\textwidth}{167mm}      % 122mm
\setlength{\textheight}{658pt}
%\setlength{\textheight}{635.6pt}
\setlength{\columnsep}{4.5mm}

\setcounter{secnumdepth}{4}

%\addtolength{\headheight}{2pt}
%\addtolength{\headsep}{-2mm}

\addtolength{\topmargin}{-7mm}  % for printing


%\hoffset=-30mm  % From Yap
\hoffset=-23mm  % From Acrobat

%\voffset=0mm % From Yap
\voffset=-5mm   % From Acrobat

%\addtolength{\evensidemargin}{-2.5mm} % for printing
%\addtolength{\oddsidemargin}{2.5mm}  % for printing

\addtolength{\evensidemargin}{-12mm} % for printing
\addtolength{\oddsidemargin}{8mm}  % for printing

%\renewcommand{\thefootnote}{\fnsymbol{footnote}}
%\renewcommand{\thefootnote}{\arabic{footnote}}
\renewcommand{\figurename}{\protect\bf Рис.}
\renewcommand{\tablename}{\protect\bf Таблица}

\newcommand{\Caption}[1]{\caption{\protect\small %\baselineskip=2.5ex
#1}}

\renewcommand{\thefigure}{\arabic{figure}}
\renewcommand{\thetable}{\arabic{table}}
\renewcommand{\theequation}{\arabic{equation}}
\renewcommand{\thesection}{\arabic{section}}

\renewcommand{\contentsname}{СОДЕРЖАНИЕ}
\newcommand{\fr}[2]{\displaystyle\frac{\displaystyle #1\mathstrut}{\displaystyle #2\mathstrut}}

%\renewcommand{\thefootnote}{\fnsymbol{footnote}}
%\newcommand{\g}{\mbox{\textit{g}}}

%\newcommand{\Caption}[1]{\caption{\protect\small\baselineskip=2ex #1}}
\newcounter{razdel}
\setcounter{razdel}{0}


\newcommand{\titel}[4]{%
\

\vspace*{5pt}

\ifodd\therazdel {\raggedright\noindent\Large\textrm\textbf
 \lineskip .75em
  \baselineskip=3.2ex #1 \par}
\vskip 1em {\noindent\large\textrm\textbf #2 \par}
\addcontentsline{toc}{subsection}{{\textrm\textbf #1}\protect\newline #2}
\def\rightheadline{\underline{\noindent\hbox to \textwidth{\hfill\small\textrm{#4}
%\hfill \large\bf\thepage
}}}
\def\leftheadline{\underline{\noindent\parbox{\textwidth}{
%\raggedleft\large\bf\thepage \hfill
\small\textit{#3}\hfill}}}
\def\leftfootline{\small{\textbf{\thepage}
\hfill ИНФОРМАТИКА И ЕЁ ПРИМЕНЕНИЯ\ \ \ том~12\ \ \ выпуск 3\ \ \ 2018}
}%
 \def\rightfootline{\small{ИНФОРМАТИКА И ЕЁ ПРИМЕНЕНИЯ\ \ \ том~12\ \ \ выпуск~3\ \ \ 2018
\hfill \textbf{\thepage}}}
\vskip 2em \setcounter{figure}{0}
\setcounter{table}{0}
\setcounter{equation}{0}
\setcounter{section}{0}
\setcounter{subsection}{0}
\setcounter{subsubsection}{0}
\setcounter{footnote}{0}
\setcounter{razdel}{0}
%\end{flushleft}
\else {
 \raggedright\noindent\Large\textrm\textbf
 \lineskip .75em
\baselineskip=3.2ex #1 \par} \vskip 1em
%\begin{flushleft}
{\noindent\large\textrm\textbf #2 \par}
\addcontentsline{toc}{subsection}{{\textrm\textbf #1}\protect\newline #2}
\def\rightheadline{\underline{\noindent\hbox to \textwidth{\hfill\small\textrm{#4}
%\hfill \large\bf\thepage
}}}
\def\leftheadline{\underline{\noindent\parbox{\textwidth}{%\raggedleft\large\bf\thepage \hfill
\small\textit{#3}\hfill}}}
\def\leftfootline{\small{\textbf{\thepage}
\hfill ИНФОРМАТИКА И ЕЁ ПРИМЕНЕНИЯ\ \ \ том~12\ \ \ выпуск~3\ \ \ 2018}
}%
 \def\rightfootline{\small{ИНФОРМАТИКА И ЕЁ ПРИМЕНЕНИЯ\ \ \ том~12\ \ \ выпуск~3\ \ \ 2018
\hfill \textbf{\thepage}}} \vskip 2em \setcounter{figure}{0}
\setcounter{table}{0} \setcounter{equation}{0} \setcounter{section}{0}
\setcounter{subsection}{0} \setcounter{subsubsection}{0}
\setcounter{footnote}{0}
%\end{flushleft}
\fi}

\newcommand{\titelr}[2]{%
\

\vspace*{5pt}

\ifodd\therazdel {\raggedright\noindent%\Large\textrm\textbf
 \lineskip .75em
  \baselineskip=3.2ex #1 \par}
\vskip 1em {\noindent\normalsize\textrm\textbf #2 \par}
\else {
 \raggedright\noindent\Large\textrm\textbf
 \lineskip .75em
\baselineskip=3.2ex #1 \par} \vskip 1em
%\begin{flushleft}
{\noindent\large\textrm\textbf #2 \par
%\noindent\normalsize\textrm\textbf #2 \par
} \fi}

\newcommand{\titele}[5]{%
\

%\vspace*{5pt}

\ifodd\therazdel {\raggedright\noindent\large
\textrm\textbf
 \lineskip .75em
%  \baselineskip=3.2ex
#1 \par}
\vskip .5em {\noindent\large\textrm\textbf #2 \par}
\vskip .5em
 {\noindent\textrm #3 \par}
\addcontentsline{toc}{subsection}{{\textrm\textbf #1}\protect\newline #2}
\def\rightheadline{\underline{\noindent\hbox to \textwidth{\hfill\small\textrm{#4}
%\hfill \large\bf\thepage
}}}
\def\leftheadline{\underline{\noindent\parbox{\textwidth}{
%\raggedleft\large\bf\thepage \hfill
\small\textrm{#5}\hfill}}}
\def\leftfootline{\small{\textbf{\thepage}
\hfill ИНФОРМАТИКА И ЕЁ ПРИМЕНЕНИЯ\ \ \ том~12\ \ \ выпуск~3\ \ \ 2018}
}%
 \def\rightfootline{\small{ИНФОРМАТИКА И ЕЁ ПРИМЕНЕНИЯ\ \ \ том~12\ \ \ выпуск~3\ \ \ 2018
\hfill \textbf{\thepage}}} \vskip 1em \setcounter{figure}{0}
\setcounter{table}{0} \setcounter{equation}{0} \setcounter{section}{0}
\setcounter{subsection}{0} \setcounter{subsubsection}{0}
\setcounter{footnote}{0} \setcounter{razdel}{0}
%\end{flushleft}
\else {
 \raggedright\noindent\large
 \textrm\textbf
 \lineskip .75em
%\baselineskip=3.2ex
#1 \par} \vskip .5em
%\begin{flushleft}
{\noindent\large\textrm\textbf #2 \par} \vskip .5em
 {\noindent\textrm #3 \par}
\addcontentsline{toc}{subsection}{{\textrm\textbf #1}\protect\newline #2}
\def\rightheadline{\underline{\noindent\hbox to \textwidth{\hfill\small\textrm{#4}
%\hfill \large\bf\thepage
}}}
\def\leftheadline{\underline{\noindent\parbox{\textwidth}{%\raggedleft\large\bf\thepage \hfill
\small\textrm{#5}\hfill}}}
\def\leftfootline{\small{\textbf{\thepage}
\hfill ИНФОРМАТИКА И ЕЁ ПРИМЕНЕНИЯ\ \ \ том~12\ \ \ выпуск~3\ \ \ 2018}
}%
 \def\rightfootline{\small{ИНФОРМАТИКА И ЕЁ ПРИМЕНЕНИЯ\ \ \ том~12\ \ \ выпуск~3\ \ \ 2018
\hfill \textbf{\thepage}}} \vskip 1em \setcounter{figure}{0}
\setcounter{table}{0} \setcounter{equation}{0} \setcounter{section}{0}
\setcounter{subsection}{0} \setcounter{subsubsection}{0}
\setcounter{footnote}{0}
%\end{flushleft}
\fi}

\def\Abst#1{
\begin{center}\small\nwt
\parbox{150mm}{%\baselineskip=2.5ex
\textbf{Аннотация:}\ \
%\hspace*{\parindent}
#1}
\end{center}}
\def\Abste#1{
\begin{center}\small\nwt
\parbox{150mm}{%\baselineskip=2.5ex
\textbf{Abstract:}\ \
%\hspace*{\parindent}
#1}
\end{center}}

\def\DOI#1{
\begin{center}\small\nwt
\parbox{150mm}{%\baselineskip=2.5ex
\textbf{DOI:}\ \
%\hspace*{\parindent}
#1}
\end{center}}

\def\Abstend#1{
\begin{center}\small\nwt
\parbox{150mm}{%\baselineskip=2.5ex
%\hspace*{\parindent}
#1}
\end{center}}


\def\KW#1{
\begin{center}\small\nwt
\parbox{150mm}{%\baselineskip=2.5ex
\textbf{Ключевые слова:}\ \ #1}
\end{center}}

\def\KWE#1{
\begin{center}\small\nwt
\parbox{150mm}{%\baselineskip=2.5ex
\textbf{Keywords:}\ \ #1}
\end{center}}


\def\KWN#1{
%\begin{center}
%\small
%\parbox{150mm}\end{center}
}

\newcommand{\Avtors}[1]{%\smallskip
%\vspace*{.5pt}
\hangindent=23pt\noindent
%\nwt
{\bfseries#1}\
}


\renewcommand{\thesubsection}{\thesection.\arabic{subsection}\hspace*{-5pt}}
\renewcommand{\thesubsubsection}{\thesubsection\hspace*{5pt}.\arabic{subsubsection}\hspace*{-3pt}}

\newcommand{\Ack}{\section*{\protect\rmfamily Acknowledgments}\noindent}
\newcommand{\Contr}{\section*{\protect\rmfamily Contributors}\noindent}
\newcommand{\Contrl}{\section*{\protect\rmfamily Contributor}\noindent}

\makeindex


\begin{document}
\Rus

\nwt
%\ptb


%\renewcommand{\contentsname}{\protect\Large\bf Содержание}

\setcounter{tocdepth}{2}

%\tableofcontents

\renewcommand{\bibname}{\protect\rmfamily Литература}
  \def\Au#1{{\it #1}}
    \def\Aue#1{{#1}}

%\newcommand{\No}{№}
  \newcommand{\tg}{\,\mathrm{tg}\,}
    \newcommand{\ctg}{\,\mathrm{ctg}\,}
  \newcommand{\arctg}{\,\mathrm{arctg}\,}

\def\forallb{\mathop{\forall}}
\def\cupb{\mathop{\cup}}
\def\existsb{\mathop{\exists}}


\newpage
\addtocounter{razdel}{1}
%\def\razd{РЕГУЛИРУЕМЫЙ ЭЛЕКТРОПРИВОД ДЛЯ ЭЛЕКТРОЭНЕРГЕТИКИ}


\setcounter{page}{2}

%   { %\Large  
   { %\baselineskip=16.6pt
   
   \vspace*{-48pt}
   \begin{center}\LARGE
   \textit{Предисловие}
   \end{center}
   
   %\vspace*{2.5mm}
   
   \vspace*{25mm}
   
   \thispagestyle{empty}
   
   { %\small 

    
Вниманию читателей журнала <<Информатика и её применения>> предлагается 
очередной тематический выпуск <<Вероятностно-статистические методы и 
задачи информатики и информационных технологий>>. Предыдущие тематические 
выпуски журнала по данному направлению вышли в 2008~г.\ (т.~2, вып.~2), 
в 2009~г.\ (т.~3, вып.~3) и в 2010~г.\ (т.~4, вып.~2). 

Статьи, собранные в данном журнале, посвящены разработке новых вероятностно-статистических 
методов, ориентированных на применение к решению конкретных задач информатики и информационных 
технологий, а также~--- в ряде случаев~--- и других прикладных задач. Проблематика, охватываемая 
публикуемыми работами, развивается в рамках научного сотрудничества между Институтом проблем 
информатики Российской академии наук (ИПИ РАН) и Факультетом вычислительной математики и 
кибернетики Московского государственного университета им.\ М.\,В.~Ломоносова в ходе работ 
над совместными научными проектами (в том числе в рамках функционирования 
Научно-образовательного центра <<Вероятностно-статистические методы анализа рисков>>). 
Многие из авторов статей, включенных в данный номер журнала, являются активными участниками 
традиционного международного семинара по проблемам устойчивости стохастических моделей, 
руководимого В.\,М.~Золотаревым и В.\,Ю.~Королевым; регулярные сессии этого семинара 
проводятся под эгидой МГУ и ИПИ РАН (в 2011~г.\ указанный семинар проводится в октябре 
в Калининградской области РФ). 

Наряду с представителями ИПИ РАН и МГУ в число авторов данного выпуска журнала входят 
ученые из Научно-исследовательского института системных исследований РАН, Института 
проблем технологии микроэлектроники и особочистых материалов РАН, Института 
прикладных математических исследований Карельского НЦ РАН, Московского 
авиационного института, Вологодского государственного педагогического университета, 
НИИММ им.\ Н.\,Г.~Чеботарева, Казанского государственного университета, Дебреценского 
университета (Венгрия).

Несколько статей выпуска посвящено разработке и применению стохастических методов и 
информационных технологий для решения различных прикладных задач. В~работе В.\,Г.~Ушакова 
и О.\,В.~Шестакова рассмотрена задача определения вероятностных характеристик случайных 
функций по распределениям интегральных преобразований, возникающих в задачах эмиссионной 
томографии. В~статье Д.\,О.~Яковенко и М.\,А.~Целищева рассмотрены некоторые вопросы 
математической теории риска и предложен новый подход к диверсификации инвестиционных 
портфелей. Работа И.\,А.~Кудрявцевой и А.\,В.~Пантелеева посвящена построению и 
исследованию математической модели, описывающей динамику сильноионизованной плазмы. 
В~статье П.\,П.~Кольцова изучается качество работы ряда алгоритмов сегментации изображений. 
Статья А.\,Н.~Чупрунова и И.~Фазекаша посвящена вероятностному анализу числа без\-оши\-бочных 
блоков при помехоустойчивом кодировании; получены усиленные законы больших чисел для указанных 
величин.

В данном выпуске традиционно присутствует тематика, весьма активно разрабатываемая в течение 
многих лет специалистами ИПИ РАН и МГУ,~--- методы моделирования и управления для 
информационно-телекоммуникационных и вычислительных систем, в частности методы 
теории массового обслуживания. В~статье А.\,И.~Зейфмана с соавторами рассматриваются 
модели обслуживания, описываемые марковскими цепями с непрерывным временем в случае 
наличия катастроф. В~работе М.\,М.~Лери и И.\,А.~Чеплюковой рассматриваются случайные 
графы Интернет-типа, т.\,е.\ графы, степени вершин которых имеют степенные распределения; 
такие задачи находят применение при исследовании глобальных сетей передачи данных. 
Работа Р.\,В.~Разумчика посвящена исследованию систем массового обслуживания специального 
вида~--- с отрицательными заявками и хранением вытесненных заявок.

Ряд статей посвящен развитию перспективных теоретических 
вероятностно-статистических методов, которые находят широкое применение в различных 
задачах информатики и информационных технологий. В~работе В.\,Е.~Бенинга, А.\,К.~Горшенина 
и В.\,Ю.~Королева рассмотрена задача статистической проверки гипотез о числе компонент 
смеси вероятностных распределений, приводится конструкция асимптотически наиболее мощного 
критерия. Результаты этой работы найдут применение в ряде прикладных задач, использующих 
математическую модель смеси вероятностных распределений (в информатике, моделировании 
финансовых рынков, физике турбулентной плазмы и~т.\,д.). В~статье В.\,Ю.~Королева, 
И.\,Г.~Шевцовой и С.\,Я.~Шоргина строится новая, улучшенная оценка точности нормальной 
аппроксимации для пуассоновских случайных сумм; как известно, указанные случайные суммы 
широко используются в качестве моделей многих реальных объектов, в том числе в информатике, 
физике и других прикладных областях. Работа В.\,Г.~Ушакова и Н.\,Г.~Ушакова посвящена 
исследованию ядерной оценки плотности распределения; эти результаты могут применяться, 
в част\-ности, при анализе трафика в телекоммуникационных системах. Серьезные приложения 
в статистике могут получить результаты работы О.\,В.~Шестакова, в которой доказаны оценки 
скорости сходимости распределения выборочного абсолютного медианного отклонения к нормальному 
закону. 

\smallskip

Редакционная коллегия журнала выражает надежду, что данный тематический  выпуск 
будет интересен специалистам в области теории вероятностей и математической статистики 
и их применения к решению задач информатики и информационных технологий.
     
     %\vfill 
     \vspace*{20mm}
     \noindent
     Заместитель главного редактора журнала <<Информатика и её 
применения>>,\\
     директор ИПИ РАН, академик  \hfill
     \textit{И.\,А.~Соколов}\\
     
     \noindent
     Редактор-составитель тематического выпуска,\\
     профессор кафедры математической статистики факультета\\
      вычислительной математики и кибернетики МГУ им.\ М.\,В.~Ломоносова,\\
     ведущий научный сотрудник ИПИ РАН,\\ 
доктор физико-математических наук \hfill
      \textit{В.\,Ю.~Королев}
     
     } }
     }

\def\stat{kon-raz}

\def\tit{УПРАВЛЕНИЕ СЛУЧАЙНЫМ БЛУЖДАНИЕМ С~ЭТАЛОННЫМ СТАЦИОНАРНЫМ 
РАСПРЕДЕЛЕНИЕМ$^*$}

\def\titkol{Управление случайным блужданием с~эталонным стационарным 
распределением}

\def\aut{М.\,Г.~Коновалов$^1$, Р.\,В.~Разумчик$^2$}

\def\autkol{М.\,Г.~Коновалов, Р.\,В.~Разумчик}

\titel{\tit}{\aut}{\autkol}{\titkol}

\index{Коновалов М.\,Г.}
\index{Разумчик Р.\,В.}
\index{Konovalov M.\,G.}
\index{Razumchik R.\,V.}




{\renewcommand{\thefootnote}{\fnsymbol{footnote}} \footnotetext[1]
{Исследование выполнено при частичной финансовой поддержке РФФИ в рамках
научного проекта  №\,18-07-00692.}}


\renewcommand{\thefootnote}{\arabic{footnote}}
\footnotetext[1]{Институт проблем информатики Федерального исследовательского центра 
<<Информатика 
и~управление>> Российской академии наук, \mbox{mkonovalov@ipiran.ru}}
\footnotetext[2]{Институт проблем информатики Федерального исследовательского центра 
<<Информатика и~управление>> Российской академии наук; Российский университет 
дружбы народов, \mbox{rrazumchik@ipiran.ru}}

%\vspace*{8pt}

 

\Abst{Рассматривается случайное блуждание на отрезке, до\-пус\-ка\-ющее управ\-ле\-ние в~форме 
выбора направления для очередного шага. Задано множество стратегий управления, 
параметризованных конечномерными векторами. Требуется найти из этого множества такую 
стратегию, при которой плотность стационарного распределения марковской цепи, описывающей 
блуждание, максимально приближена к~заданной эталонной плот\-ности распределения. Постановка 
задачи отличается от классической схемы марковского процесса принятия решений тем, что 
отсутствует одношаговый доход. Содержательная трактовка задачи появляется в~психологии, 
робототехнике, генетике. Предложен квазиградиентный алгоритм определения оптимальных 
значений па\-ра\-мет\-ров, основанный на оценках част\-ных производных целевой функции по 
наблюдениям за фазовой траекторией. Приведены чис\-лен\-ные результаты работы алгоритма 
в~примерах с~различными классами стратегий и~различными эталонными плотностями 
распределения.}

\KW{управление марковской цепью с~непрерывным множеством состояний; квазиградиентные 
алгоритмы; оценки производных по наблюдениям}

\DOI{10.14357/19922264180301}
  
\vspace*{3pt}


\vskip 10pt plus 9pt minus 6pt

\thispagestyle{headings}

\begin{multicols}{2}

\label{st\stat}

  \section{Введение}
  
  \vspace*{-3pt}
  
  В ряде прикладных областей возникает следующая модель управ\-ля\-емо\-го 
случайного блуж\-да\-ния. Пусть $n\hm=0, 1, 2,\ldots$  и~пусть $x_n\hm\in 
[0,1]$~--- положение блуж\-да\-юще\-го объекта в~момент~$n$. В~этом 
положении выбирается на\-прав\-ле\-ние движения, т.\,е.\ один из двух до\-ступ\-ных 
сег\-мен\-тов $[0, x_n]$ или $[x_n,1]$ соответственно с~вероятностями~$s(x_n)$ 
и~$1\hm- s(x_n)$. Затем, если был выбран отрезок~$[0,x_n]$, объект 
переходит в~точку $f(x_n)\hm\in [0,x_n]$, иначе~--- в~точку $g(x_n)\hm\in 
[x_n,1]$. Функции~$f(x_n)$ и~$g(x_n)$ могут быть как 
детерминированными, так и~рандомизированными, и~их вид обуслов\-лен 
спецификой задачи. Очевидно, процесс~$x_n$ является управ\-ля\-емой 
марковской цепью с~множеством со\-сто\-яний~$[0,1]$.
  
  По-видимому, впервые подобные цепи Маркова начали появляться в~связи с~математическим моделированием процессов, обладающих или наделяемых 
свойствами обуча\-емости. В~этой связи надо отметить основополагающую 
работу~[1], где для модели предлагается интерпретация из об\-ласти 
психологии: если предыдущий результат действия испытуемого есть~$x_n$, 
то $f(x_n)$ и~$g(x_n)$~--- результат сле\-ду\-юще\-го действия, вы\-би\-ра\-емо\-го из 
двух альтернатив с~вероятностями~$s(x_n)$ и~$1\hm-s(x_n)$.
  
  В~другой интерпретации~[2] значения~$x_n$ могут рас\-смат\-ри\-вать\-ся как 
текущий уровень интеллекта\linebreak
 испытуемого, значения~$g(x_n)$~--- как 
уровень интеллекта испытуемого после неправильного (правильного) ответа 
на очередной вопрос, а~$s(x_n)$~--- вероятность неправильного ответа при 
условии, что\linebreak текущий уровень интеллекта равен~$x_n$. Задачей является 
определение стационарного уровня интеллекта.
  
  В работах~[3, 4] описан пример из робототехники, в~котором изменение 
каждой из координат робота при механическом перемещении по двумерной 
области моделируется в~точ\-ности по схеме, описанной в~первом абзаце. 
Другие примеры применения рас\-смат\-ри\-ва\-емой модели случайного 
блуж\-да\-ния, в~том чис\-ле в~об\-ласти генетики, мож\-но найти в~[5--7].
  
  Основной вопрос, который возникает в~работах, ис\-поль\-зу\-ющих 
марковскую цепь $x_n$,~--- это условия существования стационарного 
распределения и~на\-хож\-де\-ние для него замкнутых формул или чис\-лен\-но\-го 
алгоритма расчета~\cite{3-kr, 4-kr, 8-kr, 9-kr, 10-kr}. Например, в~\cite{8-kr} 
показано, что если $s(x_n)\hm=1/2$, $f(x_n)\hm=1/x_n$, $g(x_n)\hm=1/(1\hm-
x_n)$, то цепь Маркова~$\{x_n\}$\footnote[3]{Эта цепь известна в~зарубежной 
литературе как цепь Диа\-ко\-ни\-са--Фрид\-ма\-на~\cite{8-kr}.} имеет единственное 
стационарное абсолютно непрерывное распределение  
c~арк\-си\-нус-плот\-ностью. Если же $s(x_n)\hm=1/2$ заменить на 
$s(x_n)=s$, $0\hm<s\hm<1$, то стационарное распределение также 
единственно и~имеет бе\-та-плот\-ность. Как показано в~\cite{4-kr},  
бе\-та-плот\-ность является предельной так\-же и~в более общем случае, когда 
длина <<шага влево>> $f(x_n)$ и~длина <<шага вправо>> $g(x_n)$ имеют  
бе\-та-рас\-пре\-де\-ле\-ния с~разными па\-ра\-мет\-ра\-ми\footnote[1]{Но один из 
параметров бета-распределения должен равняться единице.}, а~$s(x_n)$~--- линейная 
функция. Для произвольной функции~$s(x_n)$ предложен метод чис\-лен\-но\-го 
расчета стационарного распределения, а~так\-же доказываются достаточные 
условия для эргодичности цепи. Отдельно стоит отметить пример 
из~\cite{11-kr}, который показывает, что как только ве\-ро\-ят\-ность выбора 
на\-прав\-ле\-ния начинает зависеть от текущего со\-сто\-яния цепи, ситуация 
заметно услож\-ня\-ет\-ся: если $f(x_n)\hm= x_n/3$ и~$g(x_n)\hm= (x_n\hm+2)/3$, 
то существует стратегия~$s(x_n)$, $0\hm< s(x_n)\hm<1$, при которой 
стационарное распределение не единственно. 
%
Известно несколько простых 
достаточных условий эргодичности цепи, которые с~тео\-ре\-ти\-че\-ской точ\-ки 
зрения, быть может, и~являются ограничительными, но с~практической точки 
зрения предлагают удобное средство для исключения патологических 
ситуаций~\cite{3-kr, 4-kr}. К~примеру, если длины шагов влево и~вправо 
распределены равномерно, то достаточным условием является 
одновременное выполнение двух неравенств: $s(0)\hm<1$ и~$s(1)\hm>0$.
  
  Даже беглый обзор результатов свидетельствует о~том, что задача анализа, 
т.\,е.\ задача нахождения стационарного распределения цепи при 
фиксированных $s(x_n)$, $f(x_n)$ и~$g(x_n)$, изучена хорошо. 
Примечательно, что стационарное распределение цепи редко удается 
выписать в~явном виде. В~связи с~этим возникает вопрос: как решать задачу 
синтеза, т.\,е.\ задачу на\-хож\-де\-ния таких $s(x_n)$, $f(x_n)$ и~$g(x_n)$, 
которые приводят к~заданному стационарному распределению цепи? 

Прикладным мотивом к~рас\-смот\-ре\-нию по\-доб\-ной <<обратной задачи>> 
может являться упомянутый пример из~\cite{3-kr}. Пусть задачей робота 
является регулярное посещение <<каж\-до\-го участка>> некоторой об\-ласти, 
причем определенные заранее заданные час\-ти этой об\-ласти должны 
посещаться чаще, чем другие. Какая стратегия обеспечивает решение 
поставленной задачи? В~\cite{3-kr} предложено решение этой задачи 
в~случае, когда известен явный вид стационарного распределения вектора 
и~когда целевая плот\-ность распределения вектора уни\-мо\-дальная.
{\looseness=1

}
  
  Эта статья посвящена задаче синтеза управ\-ле\-ния случайным 
блуж\-да\-ни\-ем~$x_n$. В~разд.~2 формулируется постановка задачи 
и~конкретизируются случайные функции $f(x_n)$ и~$g(x_n)$. Цель 
управления~--- из за-\linebreak\vspace*{-12pt}

\columnbreak

\noindent
данного множества стратегий, па\-ра\-мет\-ри\-зо\-ван\-ных 
конечномерными наборами чис\-ло\-вых па\-ра\-мет\-ров, найти ту, при которой 
стационарное распределение цепи максимально приближено к~заданному 
эталонному виду. В~разд.~3 и~4 конструируется алгоритм решения, 
использующий идеи стохастической градиентной оптимизации на 
марковских цепях. В~разд.~5 пред\-став\-ле\-ны результаты чис\-лен\-ных 
экспериментов, которые позволяют оценить эффективность предложенного 
алгоритма.
  
  В~заключение этой вводной час\-ти сделаем еще одно замечание. 
  
  На 
сформулированную задачу приближения эталонного стационарного 
распределения на отрезке~$[0,1]$ мож\-но посмотреть и~выйдя за рамки 
случайного блуж\-да\-ния~$x_n$. Плодотворным в~этом случае является подход, 
основанный на сис\-те\-мах <<итерационных случайных  
функций>>~\cite{8-kr, 12-kr, 13-kr}. При таком подходе при\-бли\-же\-ние 
осуществляется с~по\-мощью специально подобранного набора случайных 
функций, каждая из которых, выбираемая с~некоторым вероятностным 
распределением, отображает отрезок~$[0,1]$ в~себя. Однако для цепи~$x_n$ 
такой метод не позволяет получать удовлетворительное решение из-за 
необходимости оперировать <<небогатой>> сис\-те\-мой случайных функций 
(лишь $f(x_n)$ и~$g(x_n)$).
{\looseness=1

}
  
  \section{Постановка задачи}
  
  Пусть управляемое случайное блуж\-да\-ние задается рекуррентным 
соотношением
  \begin{multline*}
  x_{n+1}=x_n+\fr{\xi_n}{\theta}\left[ -\left( 1-\sigma_n\right) x_n +\sigma_n 
\left( 1-x_n\right)\right]\,,\\
  n=0,1,2,\ldots
\end{multline*}
    В этой формуле $\sigma_n$~--- взаимно услов\-но-не\-за\-ви\-си\-мые 
бинарные случайные величины, принима\-ющие значения~0 
с~вероятностью~$s(x_n)$, и~1 с~вероятностью $1\hm- s(x_n)$; $s(x)$~--- 
функция на отрезке $[0,1]$, $0\hm< s(x)\hm<1$; $\xi_n$~---  взаимно 
независимые случайные величины, равномерно распределенные на 
отрезке~$[0, 1]$; $0\hm< \theta\hm\leq 1$; $x_0\hm\in 
[0,1]$.
  
  Значения величины~$\sigma_n$ трактуются как выбор на\-прав\-ле\-ния 
движения на каждом шаге: 0~--- сдвиг влево; 1~--- сдвиг вправо. 
Подлежащую выбору функцию~$s(x)$ будем называть правилом управ\-ле\-ния. 
Размер сдвига на каждом шаге определяется значением случайных 
величин~$\xi_n$ и~чис\-ло\-вым па\-ра\-мет\-ром~$\theta$. Начальное значение 
процесса не существенно, но известно.
  
  Последовательность правил $\{ s_n(x),\ n\geq 0\}$ будем называть 
стратегией управ\-ле\-ния, если правило~$s_n(x)$ определяет выбор 
направления на шаге~$n$. Если все правила одинаковы, т.\,е.\ если 
$s_n(x)\hm= s(x)$ для всех~$n$, то стратегию будем называть однородной. 
Таким образом, имеется взаимно однозначное соответствие меж\-ду 
правилами и~однородными стратегиями.
  
  При фиксированной однородной стратегии~$s(x)$ и~фиксированном 
па\-ра\-мет\-ре~$\theta$ по\-сле\-до\-ва\-тель\-ность~$x_n$ является марковской цепью 
с~множеством со\-сто\-яний~$[0,1]$ и~переходной ве\-ро\-ят\-ностью с~плот\-ностью
  \begin{equation}
  p(x,y) =sq_0(x,y)+(1-s)q_1(x,y)\,,\enskip 0<x<1\,,
  \label{e1-kr}
  \end{equation}
где
\begin{align*}
q_0(x,y) &= \begin{cases}
\fr{1}{x\theta}\,, &\ y\in [x-x\theta, x]\,;\\
0\,, &\  y\notin [x-x\theta,x]\,;
\end{cases}\\
q_1(x,y)&= \begin{cases}
\fr{1}{(1-x)\theta}\,, & y\in [x,x+(1-x)\theta]\,;\\
0\,, &\ y\not\in [x, x+(1-x)\theta]\,.
\end{cases}
\end{align*}
В граничных точках $x\hm=0$ и~$x\hm=1$ переходная плот\-ность 
доопределяется по не\-пре\-рыв\-ности. Относительно этой цепи предположим, 
что она при любых~$s(x)$ и~$\theta$ имеет абсолютно непрерывное 
стационарное распределение с~плот\-ностью $\pi(x)\hm= \pi(x,s(x),\theta)$, 
$x\hm\in [0,1]$.

  Пусть $S=\{s(x,a),\ a\in A\subset \mathbb{R}^k\}$~--- некоторое заданное 
множество однородных стратегий (или, что эквивалентно, множество правил 
управления), па\-ра\-мет\-ри\-зо\-ван\-ных векторами $a\hm= \left( a^{(1)}, \ldots , 
a^{(k)}\right)$ из множества $A\hm= A(S)$. Стационарное 
распределение~$\pi(x)$, соответствующее стратегии~$s(x,a)$, зависит от 
набора па\-ра\-мет\-ров~$a$. Пусть также $\rho(x)$~--- заданная плот\-ность 
вероятностного распределения на отрезке~$[0,1]$, которую будем называть 
эталонной плот\-ностью.
  
  Цель управления заключается в~том, чтобы \mbox{найти} такой набор па\-ра\-мет\-ров 
$a\hm\in a(S)$, который минимизирует функцию
  \begin{equation}
  W=W(A)=\int\limits_0^{1} \left( p(x)-\rho(x)\right)^2dx\,.
  \label{e2-kr}
  \end{equation}
  
  Таким образом, требуется отыскать стратегию из заданного 
параметризованного множества~$S$, при которой стационарное 
распределение цепи~$x_n$ наиболее при\-бли\-же\-но в~смыс\-ле 
критерия~(\ref{e2-kr}) к~заданному распределению~$\rho$.
  
  \section{Производная целевой функции по~параметру}
  
  Плотность переходной ве\-ро\-ят\-ности марковской цепи за~$n$~шагов 
определяется как
  \begin{align*}
  p^{(n)}(x,y)&=\int\limits_0^1 p^{(n-1)}(x,z) p(z,y)\,dz\,,\enskip
  n=1,2,\ldots;\\
  p^{(0)}(x,z)&\equiv p_0(z)\,.
  \end{align*}
  
  Стационарная плот\-ность распределения~$\pi(x)$ удовлетворяет 
сле\-ду\-ющим условиям:
  \begin{gather}
  \pi(y) = \int\limits_0^1 \pi(x) p(x,y)\,dx\,;\label{e3-kr}\\
  \int\limits_0^1\pi(x)\,dx=1\,.\label{e4-kr}
  \end{gather}
  
  Выберем произвольный параметр из набора~$a$ и~будем обозначать 
дифференцирование по этому па\-ра\-мет\-ру штри\-хом. Продифференцируем 
функцию~(\ref{e2-kr}) и~равенство~(\ref{e3-kr}) в~предположении, что 
производные~$p^\prime$ и~$\pi^\prime$ существуют:
  \begin{align}
  W^\prime &= 2\int\limits_0^1 \pi^\prime(x) \left( \pi(x)-
\rho(x)\right)\,dx\,;\notag %\label{e5-kr}
\\
  \pi^\prime(y)&= \int\limits_0^1 \pi^\prime(x) p(x,y)\,dx +\int\limits_0^1 \pi(x) 
p^\prime(x,y)\,dx\,.\label{e6-kr}
  \end{align}
  
  Воспользуемся номенклатурой тео\-рии обобщенных функций~\cite{14-kr}. 
Согласно этой тео\-рии существует взаимно однозначное соответствие между 
локально суммируемыми функциями~$r(x)$ на отрезке~$[0,1]$ 
и~регулярными обобщенными функциями~$r$ (линейными непрерывными 
функционалами на пространстве функций, непрерывных на отрезке~$[0, 1]$). 
Интегралу $\int \varphi(x) r(x)\,dx$ соответствует функциональное 
обозначение $\langle\varphi, r\rangle$ для результата действия 
функционала~$r$ на функцию~$\varphi$.
  
  Обозначая обобщенные функции переходной плот\-ности $p(x,y)$, а~так\-же 
ее производной $p^\prime(x,y)$ соответственно через~$p_y$ и~$p_y^\prime$ 
и~используя обозначение~$\delta_y$ для обобщенной плот\-ности, 
сосредоточенной в~точ\-ке~$x$, равенство~(\ref{e6-kr}) перепишем в~виде:
  \begin{equation}
  \langle \pi^\prime, \delta_y-p_y\rangle =\langle \pi, p_y^\prime\rangle\,.
  \label{e7-kr}
  \end{equation}
Здесь использовано характеристическое свойство $\delta$-функ\-ции, 
согласно которому $\langle\pi^\prime,\delta_y\rangle \hm= \pi^\prime(y)$.

  Для фиксированного $N\hm>0$ определим функционал
  $$
  P_x(N)=\delta_x +p_x+p_x^{(2)}+\cdots + p_x^{(N)}=\sum\limits^N_{n=0} 
p_x^{(n)}\,,
  $$
где $p_x^{(n)}$~--- обобщенные функции, со\-от\-вет\-ст\-ву\-ющие переходным 
плотностям $p^{(n)}(y,x)$, $n\hm> 0$, $p^{(0)}(x,y)\hm= \delta(x,y)$. 
(Заметим, что из приведенных определений следует, что $p_x^{(n+1)}\hm= 
p_x^{(n)} p_x\hm= p_x p_x^{(n)}$.) Применим этот функционал к~обеим 
час\-тям равенства~(\ref{e7-kr}), рас\-смат\-ри\-вая их как функции~$y$:
\begin{multline*}
\left\langle \left\langle \pi^\prime, \delta_y-p_y\right\rangle, 
P_x(N)\right\rangle=\pi^\prime(x) -{}\\
{}-\left\langle \pi^\prime, p_x\right\rangle +\left\langle \pi^\prime, 
p_x\right\rangle-\left\langle \pi^\prime, p_x^{(2)}\right\rangle+ \left\langle 
\pi^\prime, p_x^{(2)}\right\rangle -\cdots{}\\
{}\cdots - \left\langle \pi^\prime, 
p_x^{N+1)}\right\rangle= \left\langle\left\langle \pi, p_y^\prime\right\rangle, P_x(N)\right\rangle\,.
\end{multline*}
    Отсюда получаем, что
  $$
  \pi^\prime(x)=\left\langle\left\langle \pi, p_y^\prime\right\rangle, 
P_x(N)\right\rangle +\left\langle \pi^\prime, p_x^{(N+1)}\right\rangle\,.
  $$
  
  Перейдем к~пределу при $N\hm\to \infty$. Второе сла\-га\-емое в~правой час\-ти 
стремится к~нулю:
  \begin{multline*}
  \lim\limits_{N\to\infty} \left\langle \pi^\prime, p_x^{(N)}\right\rangle = 
\lim\limits_{N\to\infty} \int\limits_0^1 \pi^\prime(y) p^{(N)}(y,x)\,dy={}\\
{}=
  \int\limits_0^1 \pi^\prime(y) \pi(x)\,dy=\pi(x) \int\limits_0^1 \pi^\prime(y)\,dy=0
  \end{multline*}
(последнее равенство следует из дифференцирования равенства~(\ref{e4-kr})), поэтому
$$
\pi^\prime(x)=\left\langle \left\langle \pi, p_y^\prime\right\rangle, 
P_x\right\rangle\,,
$$
где
\begin{equation}
P_x=\sum\limits^\infty_{m=0} p_x^{(m)}\,.
\label{e8-kr}
\end{equation}
  
  После подстановки полученного выражения для~$\pi^\prime(x)$ 
в~формулу~(\ref{e4-kr}) получим:
  \begin{equation}
  W^\prime= 2\left\langle\left\langle\left\langle \pi, p_y^\prime\right\rangle, 
P_x\right\rangle, \gamma\right\rangle\,,
  \label{e9-kr}
  \end{equation}
где через~$\gamma$ обозначена обобщенная функция, со\-от\-вет\-ст\-ву\-ющая 
раз\-ности функций $\pi(x)\hm- \rho(x)$.

  Дифференцирование плот\-ности~(\ref{e1-kr}) по выбранному па\-ра\-мет\-ру 
приводит к~выражению:
  $$
  p^\prime(x,y)=s^\prime(x)\left( q_0(x,y) -q_1(x,y)\right)\,,
  $$
которое подставим в~формулу~(\ref{e9-kr}). Правая часть~(\ref{e9-kr}) 
представляет собой ряд из-за наличия функционала~(\ref{e8-kr}). Рассмотрим 
вначале произвольное сла\-га\-емое с~$m\hm> 0$. Поскольку входящие в~него 
обобщенные функции регулярны, то можно эквивалентным образом перейти 
к~обычным функциям и~записать такое слагаемое в~виде:
%\begin{multline*}
$\int\nolimits_0^1\!
\int\nolimits_0^1\!
 \int\nolimits_0^1
  \pi(x)s^\prime(x) 
\left( q_0(x,y)-\right.$\linebreak %{}\\
%{}-
$\left.-q_1(x,y)\right) p^{(m)} (y,z) \gamma(z)\,dzdydx$.
%\end{multline*}
  
  Эта запись интерпретируется следующим образом. Положим
  \begin{equation}
  G_m^{(i)} =\lim\limits_{t\to \infty} {\sf M}\left(  s^\prime(x_t) {\sf 
M}_{x_t}^{(i)}\left( \gamma_{x_{t+m}}\right)\right)\,,
  \label{e10-kr}
  \end{equation}
где ${\sf M}$~--- безусловное математическое ожидание; ${\sf 
M}_{x_t}^{(i)}$~--- условное математическое ожидание при условии, что 
в~состоянии~$x_t$ был совершен шаг влево ($i\hm=0$) или вправо 
($i\hm=1$); $\gamma\hm= \pi(\cdot)\hm- \rho(\cdot)$. Тогда рас\-смат\-ри\-ва\-емое 
слагаемое записывается как
$G_{m+1}^{(0)} \hm-G^{(1)}_{m+1}.$
  
  Для слагаемого с~$m\hm=0$, где фигурирует сингулярная обобщенная 
функция~$\delta_x$, непосредственная интегральная запись неправомочна, 
однако интерпретация этого слагаемого совершенно аналогична. Поэтому 
окончательно
  \begin{equation*}
  W^\prime= \sum\limits^\infty_{m=0} \left( G_m^{(0)} -G_m^{(1)}\right)\,.
 % \label{e11-kr}
  \end{equation*}
  
  \section{Алгоритм оптимизации параметров стратегии}
  
  Пусть по-прежнему $S$~--- заданное множество стратегий, 
параметризованных $k$-мер\-ны\-ми векторами из некоторого множества 
$A\hm= A(S)\hm\subset \mathbb{R}^k$, и~пусть $a_0\hm\in A$~--- некоторый 
начальный набор па\-ра\-мет\-ров. Будем корректировать значения параметров 
для приближенной минимизации целевой функции~(\ref{e2-kr}), используя 
стохастический вариант алгоритма проекции градиента. Обозначим 
через~$a_n$ вектор па\-ра\-мет\-ров, который применяется для выбора 
на\-прав\-ле\-ния движения на $n$-м шаге случайного блуж\-да\-ния~$x_n$. 
Алгоритм коррекции имеет вид:
  \begin{equation}
  a_{n+1}=\prod\limits_A \left( a_n-\alpha_n w_n\right)\,,
  \label{e12-kr}
  \end{equation}
где $\prod_A$~--- оператор проектирования на множество~$A$;  
$\alpha_n$~--- подходящим образом подобранная чис\-ло\-вая 
по\-сле\-до\-ва\-тель\-ность; $w_n$~--- по\-сле\-до\-ва\-тель\-ность случайных величин, 
являющихся оценками градиента~$\nabla W(a_n)$. Управление случайным 
блуж\-да\-ни\-ем осуществляется таким образом, что на $n$-м шаге выбор 
направления <<сдвига>> происходит с~по\-мощью правила $s_n(x)\hm= 
s(x,a_n)\hm\in S$. (Заметим, что стратегия $\mathbf{s}\hm= \{ s_n(x),\ n\geq 0\}$ 
является неоднородной и~множеству~$S$ не принадлежит.) Алгоритмы 
типа~(\ref{e12-kr}) являются широко распространенным инструментом 
оптимизации и~предметом изучения в~огромном чис\-ле пуб\-ли\-каций.

  Реализация схемы~(\ref{e12-kr}) предполагает по\-стро\-ение оценок 
градиента целевой функции. Согласно~(\ref{e10-kr}) каждая частная 
производная функции~$W$ пред\-став\-ля\-ет собой ряд из сла\-га\-емых слож\-ной 
структуры в~виде предела повторного математического ожидания. При этом 
внешнее усреднение происходит по предельному распределению цепи, 
которое соответствует текущему значению набора параметров, за\-да\-ющих 
стратегию. Основная проб\-ле\-ма заключается в~том, чтобы совместить 
пошаговое изменение па\-ра\-мет\-ров, соответствующее изменению 
индекса~$n$, с~необходимостью <<зафиксировать>> значения па\-ра\-мет\-ров 
для оценки предельного математического ожидания. Для решения 
используется прием <<оценивания с~забыванием>>, который поясним на 
примере оценки величины~(\ref{e10-kr}).
  
  Пусть для определенности $i\hm=0$ и~пусть $\tau_1, \tau_2, \ldots ,  
\tau_l,\ldots$~--- последовательные моменты выбора действия~0 (<<сдвиг влево>>) 
при случайном блуж\-да\-нии, управ\-ля\-емом согласно стратегии~$\mathbf{s}$. Выберем 
чис\-ло\-вую последовательность $\beta_l\uparrow 1$, положим $z_0\hm=0$, 
$b_0\hm=1$ и~зададим рекуррентные соотношения:
  $$
  z_{l+1}=\beta_l z_l +s^\prime_{\tau_l} \left( x_{\tau_l}\right) 
\gamma_{\tau_l+m}\,,\enskip
  b_{l+1}=\beta_l b_l+1\,.
  $$
  
  Оценкой величины $G_m^{(0)}$ на $n$-м шаге случайного блуж\-да\-ния 
является отношение $g_{m,n}^{(0)}\hm= z_l/b_l$. Аналогично строятся 
оценки $g_{m,n}^{(1)}\hm= z_l/b_l$ величин~$G_m^{(1)}$.
  
  <<Скользящие>> суммы были использованы в~\cite{15-kr} для построения 
оценок градиента предельного среднего дохода в~задаче управ\-ле\-ния 
дискретной марковской цепью с~доходами с~по\-мощью алгоритма 
типа~(\ref{e12-kr}). Там же было доказано, что для схо\-ди\-мости алгоритма 
необходимо выполнение дополнительных условий, в~част\-ности на 
последовательности~$\alpha_n$ и~$\beta_l$. В~данной работе исследование 
алгоритма ограничивается численными экспериментами.
  
  \section{Экспериментальный анализ}
  
  В этом разделе приведены результаты экспериментов с~управляемым 
случайным блужданием~$x_n$ с~коэффициентом~$\theta$, задающим 
максимальный размер сдвига на одном шаге, равным~0,5. Была выбрана 
упрощенная модификация управ\-ля\-юще\-го алгоритма~(\ref{e12-kr}), для 
которой последовательности~$\alpha_n$ и~$\beta_l$ суть константы: 
$\alpha_n\hm\equiv \alpha$, $\beta_l\hm\equiv \beta$. В~качестве оценки 
част\-ных производных целевой функции на $n$-м шаге было взято выражение 
$\sum\nolimits_{m=0}^M \left( g_{m,n}^{(0)} \hm- g_{m,n}^{(1)}\right)$. 
Константы~$\alpha$, $\beta$ и~$M$ варьировались в~диапазонах $[10^{-8}; 
10^{-7}]$, $[0{,}999; 0{,}9999]$ и~$\{3,4,\ldots , 10\}$ соответственно.
  
  Определим четыре множества правил управ\-ле\-ния, образованных 
многочленами степеней от~0 до~3:
  \begin{multline*}
  S_k=\left\{ s(x,a) =\max \left( 0,\min\left(1, \sum\limits^k_{i=0} a^{(i)} 
x^i\right)\right), \right.\\ \left.a^{(i)}\in \mathbb{R}
\vphantom{\left( 0,\min\left(1, \sum\limits^k_{i=0} a^{(i)} 
x^i\right)\right)}
\right\}\,,\enskip
  k=0,1,2,3.
  \end{multline*}
Стратегии из класса~$S_k$ па\-ра\-мет\-ри\-зу\-ют\-ся набором коэффициентов 
$a\hm= \left( a^{(0)}, \ldots , a^{(k)}\right)\hm\in A_k \hm= \mathbb{R}^{k+1}$.

  Кроме того, определим множество правил управ\-ле\-ния синусоидального 
типа:
  \begin{multline*}
  S_t={}\\
  \hspace*{-1.35pt}{}=\left\{ s(x,a)=\max \left( 0,\min \left( 1, a_0 \sin\left( 
a_1+a_2x\right)+a_3\right)\right);\right.\\
 \left.a_0, a_1, a_2, a_3 \in \mathbb{R}\right\}\,,
\end{multline*}
которые параметризуются набором коэффициентов $\left( a^{(0)}, \ldots , 
a^{(3)}\right) \hm\in A_t\hm= \mathbb{R}^4$.
  
  Для выбранного класса правил управ\-ле\-ния~$S$ и~установленной 
эталонной плот\-ности~$c(x)$ эксперимент заключался в~имитации траектории 
случайного блуж\-да\-ния~$x_n$, управ\-ля\-емо\-го согласно стратегии $\mathbf{s}=\{ 
s_n(x)=s(x, a_n)\hm\in S,\ n\hm\geq 0\}$. При этом последо\-ва\-тель\-ность~$a_n$ 
порождалась алгоритмом~(\ref{e12-kr}), в~котором $A\hm= A(S)$, 
а~последовательности~$\sigma_n$ и~$w_n$ определены в~начале текущего 
раздела. Продолжительность эксперимента составляла $N\sim 10^7$~тактов. 
Результатами эксперимента стали финальное правило управ\-ле\-ния~$s_N(x)$, 
а~так\-же оценки значений целевой функции~$W_N$ и~стационарной 
плотности $p_N(x)$, соответствующих финальному правилу~$s_N(x)$.
  
  \smallskip
  
  \noindent
  \textbf{Пример~1.} Эталонная плот\-ность $c(x)$~--- линейная. Она 
приведена в~табл.~1. В~этой же таб\-ли\-це указано начальное правило 
управления~$s_0(x)$, а~так\-же полученные экспериментально значение 
целевой функции~$W_0$ и~график плот\-ности стационарного распределения 
$p_0(x)$, соответствующие правилу управ\-ле\-ния~$s_0(x)$. Результаты 
эксперимента приведены в~табл.~2. Они показывают, что алгоритм 
существенно улучшает начальное правило как по значениям целевой 
функции, так и~по форме графика стационарной плот\-ности. Заметим, 
впрочем, что при\-бли\-же\-ние графика стационарной
%\noindent
 плотности к~эталонному 
виду не ставилось задачей управ\-ле\-ния. Это побочный эффект оптимизации 
по мет\-ри\-ке~(\ref{e2-kr}). Дополнительно следует отметить,\linebreak\vspace*{-12pt}

\pagebreak

\end{multicols}

\begin{table*}
{\small %tabl1
   \begin{center}
   \Caption{Исходные данные для эксперимента с~линейной эталонной плот\-ностью}
   \vspace*{2ex}
   
   \begin{tabular}{|c|c|c|c|}
   \hline
  $c(x)=2(1-x)$&$s_0(x)\equiv 0{,}5$&$W_0$&$p_0(x)$\\ 
\hline 
&&&\\[-9pt]
 \mbox{%
 \epsfxsize=43mm 
 \epsfbox{kon-1t-1.eps}
 }
&
 \mbox{%
 \epsfxsize=46mm 
 \epsfbox{kon-1t-2.eps}
 }
&\raisebox{30pt}[0pt][0pt]{0,547}&
 \mbox{%
 \epsfxsize=46mm 
 \epsfbox{kon-1t-3.eps}
 }
\\ 
\hline 
\end{tabular} 
\end{center} }
\vspace*{6pt}
%\end{table*}
%\begin{table*}
{\small %tabl2
\begin{center}
\Caption{Результаты эксперимента с~линейной эталонной плот\-ностью}
\vspace*{2ex}

\tabcolsep=10pt
\begin{tabular}{|c|c|c|c|}
\hline
$S$ &  $s_N(x)$ & $W_N$ & $p_N(x)$\\
\hline
&&&\\[-6pt]
\raisebox{36pt}[0pt][0pt]{$S_0$} & \mbox{%
 \epsfxsize=46mm 
 \epsfbox{kon-2t-1.eps}
 } & \raisebox{24pt}[0pt][0pt]{0,033}&  
 \mbox{%
 \epsfxsize=46mm 
 \epsfbox{kon-2t-4.eps}
 }\\
 & $\mathrm{const}=0{,}693$ &&\\
 \hline
  &&&\\[-6pt]
\raisebox{36pt}[0pt][0pt]{$S_1$}&   \mbox{%
 \epsfxsize=46mm 
 \epsfbox{kon-2t-2.eps}
 } & \raisebox{24pt}[0pt][0pt]{0,024}& 
  \mbox{%
 \epsfxsize=46mm 
 \epsfbox{kon-2t-5.eps}
 }\\
 &$0{,}703-0{,}047x$&&\\
 \hline
 &&&\\[-6pt]
\raisebox{36pt}[0pt][0pt]{$S_2$}& \mbox{%
 \epsfxsize=46mm 
 \epsfbox{kon-2t-3.eps}
 }& \raisebox{24pt}[0pt][0pt]{0,013} & \mbox{%
 \epsfxsize=46mm 
 \epsfbox{kon-2t-6.eps}
 }\\
 &$0{,}712-0{,}049x-0{,}120x^2$&&\\
 \hline
%$\mathrm{const}=0{,}693$& $0{,}703-0{,}047x$& $0{,}712-0{,}049x-0{,}120x^2$\\
\end{tabular}
\end{center}}
\vspace*{18pt}
\end{table*}

%\end{multicols}
\begin{multicols}{2}



   
   
   %\smallskip
   
%\begin{multicols}{2}
  %\smallskip
  
   \noindent
   что качество 
оптимизации повышается с~рос\-том степени многочленов~--- <<базисных>> 
правил.

\smallskip
  
  \noindent
  \textbf{Пример~2.}\ Эталонная плот\-ность~--- квад\-ра\-тич\-ная: $c(x) = 
2{,}727(1{,}4x \hm -x^2)$. Начальное правило такое же, как в~примере~1. Ему 
соответствует значение целевой функции $W_0\hm= 0{,}163$. Результаты 
эксперимента приведены в~табл.~3. Качественные выводы относительно 
результатов такие же, как в~примере~1. Плот\-ность~$c(x)$ для наглядного 
сравнения с~результирующей плот\-ностью  изображена на рис.~1.

\smallskip

 \noindent
   \textbf{Пример~3.}\ Эталонная плот\-ность~--- кубическая: $c(x)\hm= 
1{,}622(0{,}8\hm+0{,}4x\hm- 2{,}5x^2\hm+ 1{,}8x^3)$ (рис.~2).
Начальное правило такое же, как в~предыду\-щих примерах. Ему соответствует 
значение целевой\linebreak\vspace*{-12pt} 



\begin{table*}\small %tabl3
\begin{center}
\Caption{Результаты эксперимента с~квадратичной эталонной плот\-ностью}
\vspace*{2ex}

\begin{tabular}{|c|c|c|c|}
\hline
 $S$&$s_N(x)$ & $W_N$ & $p_N(x)$\\
\hline
&&&\\[-6pt]
\raisebox{36pt}[0pt][0pt]{$S_0$} &\mbox{%
 \epsfxsize=46mm %32.636mm 
 \epsfbox{kon-3t-1.eps}
 }& \raisebox{36pt}[0pt][0pt]{0,042}& \mbox{%
 \epsfxsize=46mm %31.915mm 
 \epsfbox{kon-3t-5.eps}
 }\\
& $\mathrm{const}=0{,}400$ &&\\
 \hline
 &&&\\[-6pt]
\raisebox{36pt}[0pt][0pt]{ $S_1$} & \mbox{%
 \epsfxsize=46mm %33.911mm 
 \epsfbox{kon-3t-2.eps}
 }& \raisebox{36pt}[0pt][0pt]{0,023} &\mbox{%
 \epsfxsize=46mm %32.071mm 
 \epsfbox{kon-3t-6.eps}}\\
& $0{,}455-0{,}092x$&&\\
 \hline
  &&&\\[-6pt]
  \raisebox{36pt}[0pt][0pt]{$S_2$} &\mbox{%
 \epsfxsize=46mm %33.528mm 
 \epsfbox{kon-3t-3.eps}
 }& \raisebox{36pt}[0pt][0pt]{0,010}& \mbox{%
 \epsfxsize=46mm %31.967mm 
 \epsfbox{kon-3t-7.eps}}\\
 &$0{,}472-0{,}069x -0{,}081x^2$&&\\
 \hline
  &&&\\[-6pt]
 \raisebox{36pt}[0pt][0pt]{ $S_3$}&\mbox{%
 \epsfxsize=46mm %33.94mm 
 \epsfbox{kon-3t-4.eps}
 } & \raisebox{36pt}[0pt][0pt]{0,004}&\mbox{%
 \epsfxsize=46mm %31.912mm 
 \epsfbox{kon-3t-8.eps}}\\
& $0{,}482- 0{,}64x -0{,}053x^2-0{,}070x^3$&&\\
\hline
\end{tabular}
\end{center}

\renewcommand{\tablename}{\protect\bf Рис.}


\setcounter{table}{0}

%\begin{figure*} %fig1
\vspace*{20pt}
\begin{minipage}[t]{80mm}
 \begin{center}
 \mbox{%
 \epsfxsize=66.178mm 
 \epsfbox{kon-1.eps}
 }
 \end{center}
\vspace*{-9pt}
\Caption{Квадратичная эталонная плот\-ность}
\end{minipage}
%\end{figure*}
\hfill
  %   \begin{figure*} %fig2
   \vspace*{1pt}
   \begin{minipage}[t]{80mm}
 \begin{center}
 \mbox{%
 \epsfxsize=66.187mm 
 \epsfbox{kon-2.eps}
 }
 \end{center}
\vspace*{-9pt}
   \Caption{Кубическая эталонная плот\-ность}
   \end{minipage}
%   \end{figure*}
\end{table*}

\renewcommand{\figurename}{\protect\bf Рис.}
\renewcommand{\tablename}{\protect\bf Таблица}

\pagebreak

\end{multicols}


\setcounter{table}{3}
\begin{table*}\small %tabl4
\begin{center}
\Caption{Результаты эксперимента с~кубической эталонной плот\-ностью}
\vspace*{2ex}

\begin{tabular}{|c|c|c|c|}
\hline
$S$ & $s_N(x)$ &  $W_N$& $p_N(x)$\\
\hline
&&&\\[-6pt]
\raisebox{24pt}[0pt][0pt]{$S_0$} & \mbox{%
 \epsfxsize=46mm %32.271mm 
 \epsfbox{kon-4t-1.eps}
 }& \raisebox{36pt}[0pt][0pt]{0,042}& \mbox{%
 \epsfxsize=46mm %32.064mm 
 \epsfbox{kon-4t-5.eps}
 }\\
 &  $\mathrm{const}=0{,}602$ &&\\[3pt] 
 \hline
 &&&\\[-6pt]
\raisebox{36pt}[0pt][0pt]{ $S_1$} & \mbox{%
 \epsfxsize=46mm %32.907mm 
 \epsfbox{kon-4t-2.eps}
 } & \raisebox{36pt}[0pt][0pt]{0,023}& \mbox{%
 \epsfxsize=46mm %32.017mm 
 \epsfbox{kon-4t-6.eps}
 }\\
 &$0{,}616-0{,}086x$&&\\[3pt]
 \hline
 &&&\\[-6pt]
  \raisebox{36pt}[0pt][0pt]{$S_2$} & \mbox{%
 \epsfxsize=46mm %32.816mm 
 \epsfbox{kon-4t-3.eps}
 }& \raisebox{36pt}[0pt][0pt]{0,010}&\mbox{%
 \epsfxsize=46mm %32.35mm 
 \epsfbox{kon-4t-7.eps}
 }\\
& $0{,}622-0{,}126x -0{,}070x^2$&&\\[3pt]
\hline
 &&&\\[-6pt]
 \raisebox{36pt}[0pt][0pt]{ $S_3$ }&\mbox{%
 \epsfxsize=46mm %32.608mm 
 \epsfbox{kon-4t-4.eps}
 }&\raisebox{36pt}[0pt][0pt]{0,004}& \mbox{%
 \epsfxsize=46mm %32.306mm 
 \epsfbox{kon-4t-8.eps}
 }\\
& $0{,}702-0{,}189x-0{,}095x^2-0{,}233x^3$&&\\
\hline
\end{tabular}
\end{center}
\vspace*{12pt}
\end{table*}





\begin{multicols}{2}



\noindent
 функции $W_0\hm=0{,}284$. Результаты эксперимента 
приведены в~табл.~4. Как и~в предыду\-щих
 примерах, оптимизация на любом 
из множеств~$S_k$ дает существенный выигрыш в~целевой функции
 по 
сравнению с~начальным при\-бли\-же\-ни\-ем. Вновь качество оптимизации 
повышается с~ростом значения~$k$, т.\,е.\ с~увеличением мощ\-ности 
множества <<базисных>> правил. 

В~отличие от первых двух примеров, 
форма стационарной плот\-ности становится похожей на эталонную плотность 
только при использовании правил управ\-ле\-ния~--- многочленов третьей 
сте-\linebreak пени.



  
%\end{multicols}

  \smallskip
  
%\begin{multicols}{2}

  \noindent
  \textbf{Пример~4.}\ Эталонная плот\-ность~--- синусоидальная. Исходные 
данные указаны в~табл.~5. Результаты приведены в~табл.~6. 
%
Этот пример 
особенно подчеркивает важ\-ность выбора множества <<базисных>> правил. 
<<Тригонометрическое>> множество~$S_t$, несмотря на плохое начальное 
правило, позволяет добиться гораздо лучшего при\-бли\-же\-ния к~эталонной 
плот\-ности, чем множество полиномиальных правил.


   
\end{multicols}


\setcounter{table}{4}
\begin{table*}\small  %tabl5
   \begin{center}
   \Caption{Исходные данные для эксперимента с~синусоидальной эталонной плот\-ностью}
   \vspace*{2ex}
   
   \tabcolsep=5.5pt
   \begin{tabular}{|c|c|c|c|}
   \hline
   $c(x)=0{,}798\sin (0{,}2+12x)+0{,}977$ & $s_0(x)$ & $W_0$ & $p_0(x)$\\
   \hline
   &&&\\[-6pt]
 \raisebox{-56pt}[0pt][0pt]{ \mbox{%
 \epsfxsize=46mm %31.205mm 
 \epsfbox{kon-5t-1.eps}
 }}& \raisebox{40pt}[0pt][0pt]{ 
 \tabcolsep=0pt\begin{tabular}{c}для базовых классов $S_0$--$S_2$:\\
$s_0(x)\equiv 0{,}5$\end{tabular}}& \raisebox{40pt}[0pt][0pt]{ 0,533}& \mbox{%
 \epsfxsize=46mm %32.728mm 
 \epsfbox{kon-5t-3.eps}
 }\\
\cline{2-4}
& для базового класса $S_t$: &&\\
& \mbox{%
 \epsfxsize=46mm %31.411mm 
 \epsfbox{kon-5t-2.eps}
 }&  \raisebox{40pt}[0pt][0pt]{ 143,9} &\mbox{%
 \epsfxsize=46mm 
 \epsfbox{kon-5t-4.eps}
 }\\
& $s(x)=\min(1,\sin 10x +1)$ &&\\
\hline
\end{tabular}
\end{center}
%\vspace*{2cm}
\end{table*}


\begin{table*}\small %tabl6
\begin{center}
\Caption{Результаты эксперимента с~синусоидальной эталонной плот\-ностью}
\vspace*{2ex}

\begin{tabular}{|c|c|c|c|}
\hline
$S$ & $s_N(x)$ & $W$ & $p_N(x)$\\
\hline
&&&\\[-6pt]
 \raisebox{36pt}[0pt][0pt]{$S_0$} & \mbox{%
 \epsfxsize=46mm %33.723mm 
 \epsfbox{kon-6t-1.eps}
 }& \raisebox{36pt}[0pt][0pt]{0,505}& \mbox{%
 \epsfxsize=46mm %31.925mm 
 \epsfbox{kon-6t-6.eps}
 }\\
 &$\mathrm{const}=0{,}533$&&\\
 \hline
 &&&\\[-6pt]
  \raisebox{36pt}[0pt][0pt]{$S_1$} & \mbox{%
 \epsfxsize=46mm %33.505mm 
 \epsfbox{kon-6t-2.eps}
 }& \raisebox{36pt}[0pt][0pt]{0,425}& \mbox{%
 \epsfxsize=46mm %31.884mm 
 \epsfbox{kon-6t-7.eps}
 }\\
 &$0{,}602-0{,}165x$&&\\
 \hline
 &&&\\[-6pt]
  \raisebox{36pt}[0pt][0pt]{$S_2$} & \mbox{%
 \epsfxsize=46mm %32.728mm 
 \epsfbox{kon-6t-3.eps}
 } & \raisebox{36pt}[0pt][0pt]{0,372}& \mbox{%
 \epsfxsize=46mm %32.112mm 
 \epsfbox{kon-6t-8.eps}
 }\\
& $0{,}686-0{,}074x -0{,}162x^2$&&\\
 \hline
 &&&\\[-6pt]
   \raisebox{36pt}[0pt][0pt]{$S_3$} & \mbox{%
 \epsfxsize=46mm %32.13mm 
 \epsfbox{kon-6t-4.eps}
 }& \raisebox{36pt}[0pt][0pt]{0,321}& \mbox{%
 \epsfxsize=46mm %31.761mm 
 \epsfbox{kon-6t-9.eps}
 }\\
 &$0{,}724 -0{,}093x- 0{,}182x^2-0{,}226x^3$\\
 \hline
 &&&\\[-6pt]
      \raisebox{36pt}[0pt][0pt]{$S_{\mathrm{т}}$} &
\mbox{%
 \epsfxsize=46mm %32.332mm 
 \epsfbox{kon-6t-5.eps}
 }& \raisebox{24pt}[0pt][0pt]{0,152}&\mbox{%
 \epsfxsize=46mm %31.981mm 
 \epsfbox{kon-6t-10.eps}
 }\\
  & $0{,}288\sin (-0{,}355+9{,}531x) + 0{,}566$&&\\
\hline
\end{tabular}
\end{center}
%\vspace*{3pt}
%\vspace*{5cm}
\end{table*}

\begin{multicols}{2}

\section{Заключение}
  
  Сформулирована задача поиска стратегии управ\-ле\-ния случайным 
блужданием с~целью минимизировать функцию, оценивающую отклонение 
плотности стационарного распределения от заданной эталонной плот\-ности.
Задача относится к~теории управ\-ле\-ния марковскими цепями с~недискретным 
множеством со\-сто\-яний, однако отличается от классической постановки 
марковского процесса принятия решений тем, что отсутствует одношаговый 
доход. Оптимальная стратегия ищется в~заданном множестве стратегий, 
параметризованных конечномерными наборами чис\-ло\-вых па\-ра\-мет\-ров.
{\looseness=1

} 

Предложенное при\-бли\-жен\-ное чис\-лен\-ное решение пред\-став\-ля\-ет собой 
градиентный алгоритм коррекции па\-ра\-мет\-ров стратегии, причем оценки\linebreak 
производных целевой функции строятся по результатам наблюдения за 
имитируемой траекторией процесса. Проведен экспериментальный анализ 
алгоритма для ряда па\-ра\-мет\-ри\-зо\-ван\-ных классов стратегий и~эталонных 
плотностей, который показал хорошие результаты в~плане минимизации 
целевой функции. Основной вывод на основании работы за\-клю\-ча\-ет\-ся 
в~констатации эффективности градиентного подхода к~оптимизации на 
марковских цепях с~непрерывным множеством со\-сто\-яний. 

Пред\-став\-ля\-ют\-ся 
интересными сле\-ду\-ющие на\-прав\-ле\-ния дальнейших исследований:
% \begin{itemize}
%\item  
теоретический анализ сходимости алгоритма;
%\item  
распространение и~обоснование метода на произвольные\linebreak 
распределения без предположения о~существовании плотностей;
%\item  
  использование различных мет\-рик для оценки бли\-зости 
распределений;
%\item  
  изучение связи между исходным множеством заданных стратегий 
и~точ\-ностью при\-бли\-же\-ния эталонного распределения.
 %\end{itemize}
  
{\small\frenchspacing
 {%\baselineskip=10.8pt
 \addcontentsline{toc}{section}{References}
 \begin{thebibliography}{99}
  \bibitem{1-kr}
  \Au{Karlin S.} Some random walks arising in learning models.~I~// 
  Pac. J.~Math., 1953. 
Vol.~3. No.\,4. P.~725--756.
  \bibitem{2-kr}
  \Au{Kaijser T.} On a~theorem of Karlin~// Acta Appl. Math., 1994. Vol.~34. P.~51--69.
  \bibitem{3-kr}
  \Au{Ramli~M.\,A., Leng~G.} The stationary probability density of a~class of bounded Markov 
processes~// Adv. Appl. Probab., 2010. Vol.~42. P.~986--993.
  \bibitem{4-kr}
  \Au{McKinlay S., Borovkov~K.} On explicit form of the stationary distributions for a~class of 
bounded Markov chains~// J.~Appl. Probab., 2016. Vol.~53. Iss.~1. P.~231--243. 
  \bibitem{5-kr}
  \Au{Li~C.} Human genetics.~--- New York, NY, USA: McGraw-Hill, 1961. 218~p.
 
  \bibitem{7-kr}
  \Au{DeGroot M.\,H., Rao~M.\,M.} Stochastic give-and-take~// J.~Math. Anal. Appl., 1963. 
Vol.~7. P.~489--498.

 \bibitem{6-kr} %7
  \Au{McKinlay~S.} A~characterization of transient random walks on stochastic matrices with 
Dirichlet distributed limits~// J.~Appl. Probab., 2014. Vol.~51. P.~542--555.

 \bibitem{10-kr} %8
  \Au{Peign$\acute{\mbox{e}}$~M.} Iterated function systems and spectral decomposition of 
the associated Markov operator~// Publications math$\acute{\mbox{e}}$matiques et 
informatique de Rennes, 1993. No.\,2. P.~1--28.

\pagebreak

  \bibitem{8-kr} %9
  \Au{Diaconis P., Freedman~D.} Iterated random functions~// SIAM Rev., 1999. Vol.~41. 
Iss.~1. P.~45--76.



  \bibitem{9-kr} %10
  \Au{Ladjimi F., Peign$\acute{\mbox{e}}$~M.} Iterated function systems with place 
dependent probabilities and application to the Diaconis--Friedman's chain on~$[0,1]$. {\sf 
https://hal.archives-ouvertes.fr/LMPT/hal-01567392v1}.
 
  \bibitem{11-kr}
  \Au{Stenflo $\ddot{\mbox{O}}$.} A~note on a~theorem of Karlin~// Stat. Probabil. 
Lett., 2001. Vol.~54. Iss.~2. P.~183--187.
  \bibitem{12-kr}
  \Au{Jacquin A.} A~fractal theory of iterated Markov operators with applications to digital 
image coding.~--- Atlanta,
 GA, USA: Georgia Institute of Technology, 1989. Ph.D. Thesis.
 

 
  \bibitem{13-kr}
  \Au{Forte B., Vrscay~E.\,R.} Solving the inverse problem for measures using iterated 
function systems: A~new approach~// Adv. Appl. Probab., 1995. 
Vol.~27. Iss.~3. P.~800--820.
  \bibitem{14-kr}
  \Au{Владимиров~В.\,С.} Обобщенные функции в~математической физике.~--- М.: 
Наука, 1976. 280~с.
  \bibitem{15-kr}
  \Au{Коновалов М.\,Г.} Методы адаптивной обработки информации и~их  
приложения.~--- М.: ИПИ РАН, 2007. 212~с.

 \end{thebibliography}

 }
 }

\end{multicols}

\vspace*{-6pt}

\hfill{\small\textit{Поступила в~редакцию 28.04.18}}

\vspace*{6pt}

%\newpage

%\vspace*{-24pt}

\hrule

\vspace*{2pt}

\hrule

\vspace*{-2pt}


\def\tit{FINDING CONTROL POLICY FOR~ONE~DISCRETE-TIME MARKOV 
CHAIN ON~$[0,1]$ WITH~A~GIVEN INVARIANT MEASURE}

\def\titkol{Finding control policy for~one~discrete-time MARKOV 
chain on~$[0,1]$ with~a~given invariant measure}

\def\aut{M.\,G.~Konovalov$^1$ and~R.\,V.~Razumchik$^{1,2}$}

\def\autkol{M.\,G.~Konovalov and~R.\,V.~Razumchik}

\titel{\tit}{\aut}{\autkol}{\titkol}

\vspace*{-11pt}


\noindent
$^1$Institute of Informatics Problems, 
Federal Research Center ``Computer Science and Control'' of the Russian\linebreak 
$\hphantom{^1}$Academy of Sciences, 44-2~Vavilov Str., Moscow 119333, Russian Federation

\noindent
$^2$Peoples' Friendship University of Russia (RUDN University),  
6~Miklukho-Maklaya Str., Moscow 117198, Russian\linebreak
$\hphantom{^1}$Federation


\def\leftfootline{\small{\textbf{\thepage}
\hfill INFORMATIKA I EE PRIMENENIYA~--- INFORMATICS AND
APPLICATIONS\ \ \ 2018\ \ \ volume~12\ \ \ issue\ 3}
}%
 \def\rightfootline{\small{INFORMATIKA I EE PRIMENENIYA~---
INFORMATICS AND APPLICATIONS\ \ \ 2018\ \ \ volume~12\ \ \ issue\ 3
\hfill \textbf{\thepage}}}

\vspace*{3pt}


\Abste{A~discrete-time Markov chain on the interval~$[0,1]$ with two possible transitions (left or right) at each 
step has been considerred. The probability of transition towards~0 (and towards~1) is a~function of the current value 
of the chain. Having chosen the direction, the chain moves to the randomly chosen point from the appropriate 
interval. The authors assume that the transition probabilities depend on the current value of the chain only through a~finite 
number of real-valued numbers. Under this assumption, they seek the transition probabilities, which guarantee 
the~$L_2$ distance between the stationary density of the Markov chain and the given invariant measure on $[0,1]$ 
is minimal. Since there is no reward function in this problem, it does not fit in the 
MDP (Markov decision process) framework. The authors follow the 
sensitivity-based approach and propose the gradient- and simulation-based method for estimating the parameters of 
the transition probabilities. Numerical results are presented which show the performance of the method for various 
transition probabilities and invariant measures on~$[0,1]$.}

\KWE{Markov chain; control; continuous state space; sensitivity-based approach; derivative estimation}


\DOI{10.14357/19922264180301}

\vspace*{-10pt}

\Ack
\noindent
The reported study was partly supported by the Russian Foundation for
Basic Research according to the research 
project No.\,18-07-00692.



%\vspace*{6pt}

  \begin{multicols}{2}

\renewcommand{\bibname}{\protect\rmfamily References}
%\renewcommand{\bibname}{\large\protect\rm References}

{\small\frenchspacing
 {%\baselineskip=10.8pt
 \addcontentsline{toc}{section}{References}
 \begin{thebibliography}{99}
  \bibitem{1-kr-1}
  \Aue{Karlin, S.} 1953. Some random walks arising in learning models.~I.
  \textit{Pac. J.~Math.} 3(4):725--756.
  \bibitem{2-kr-1}
  \Aue{Kaijser, T.} 1994. On a~theorem of Karlin. \textit{Acta Appl. Math.} 34:51--69.
  \bibitem{3-kr-1}
  \Aue{Ramli, M.\,A., and G.~Leng}. 2010. The stationary probability density of a~class of 
bounded Markov processes. \textit{Adv. Appl. Probab.} 42:986--993.
  \bibitem{4-kr-1}
  \Aue{McKinlay, S., and K.~Borovkov}. 2016. On explicit form of the stationary distributions 
for a~class of bounded Markov chains. \textit{J.~Appl. Probab.} 53(1):231--243. 
  \bibitem{5-kr-1}
  \Aue{Li, C.} 1961. \textit{Human genetics}. New York, NY: McGraw-Hill. 218~p.
  
  \bibitem{7-kr-1}
  \Aue{DeGroot, M.\,H., and M.\,M.~Rao}. 1963. Stochastic give-and-take. \textit{J.~Math. 
Anal. Appl.} 7:489--498.

\bibitem{6-kr-1} %7
  \Aue{McKinlay, S.} 2014. A~characterization of transient random walks on stochastic 
matrices with Dirichlet distributed limits. \textit{J.~Appl. Probab.} 51:542--555.

\bibitem{10-kr-1} %8
  \Aue{Peign$\acute{\mbox{e}}$,~M.} 1993. Iterated function systems and spectral 
decomposition of the associated Markov operator. \textit{Publications 
math$\acute{\mbox{e}}$matiques et informatique de Rennes}. 2:\linebreak 1--28.



  \bibitem{8-kr-1} %9
  \Aue{Diaconis, P., and D.~Freedman}. 1999. Iterated random functions. \textit{SIAM 
Rev.} 41(1):45--76.
  \bibitem{9-kr-1} %10
  \Aue{Ladjimi, F., and M.~Peign$\acute{\mbox{e}}$.} Iterated function systems with place 
dependent probabilities and appli-\linebreak\vspace*{-12pt}

\pagebreak

\noindent
cation to the Diaconis--Friedman's chain on $[0,1]$. Available 
at: {\sf https://hal.archives-ouvertes.fr/LMPT/hal-01567392v1/} (accessed April~4, 2018).
  
  \bibitem{11-kr-1}
  \Aue{Stenflo,~$\ddot{\mbox{O}}$.} 2001. A~note on a~theorem of Karlin. \textit{Stat. 
Probabil. Lett.} 54(2):183--187.
  \bibitem{12-kr-1}
  \Aue{Jacquin, A.} 1989. A~fractal theory of iterated Markov operators with applications to 
digital image coding. Atlanta, GA:  Georgia Institute of Technology. Ph.D.\linebreak Thesis.

\columnbreak

  \bibitem{13-kr-1}
  \Aue{Forte, B., and E.\,R.~Vrscay.} 1995. Solving the inverse problem for measures using 
iterated function systems: A~new approach. \textit{Adv. Appl.
Probab.} 27(3):800--820.
  \bibitem{14-kr-1}
  \Aue{Vladimirov, V.\,S.} 1976. \textit{Obobshchennye funktsii v~ma\-te\-ma\-ti\-che\-skoy fizike} 
[Generalized functions in mathematical physics]. Moscow: Nauka. 280~p.
  \bibitem{15-kr-1}
  \Aue{Konovalov, M.\,G.} 2007. \textit{Metody adaptivnoy obrabotki informatsii i~ikh 
prilozheniya} [Methods of adaptive information processing and their applications]. Moscow: IPI 
RAN. 212~p.
  \end{thebibliography}

 }
 }

\end{multicols}

\vspace*{-6pt}

\hfill{\small\textit{Received April 28, 2018}}

%\pagebreak

%\vspace*{-18pt}

\Contr

\noindent
\textbf{Konovalov Mikhail G.} (b.\ 1950)~--- Doctor of Science in technology, 
principal scientist, Institute of Informatics Problems, Federal Research 
Center ``Computer Science and Control'' of the Russian Academy of Sciences, 
44-2 Vavilov Str., Moscow 119333, Russian Federation; 
\mbox{mkonovalov@ipiran.ru}

\vspace*{3pt}

\noindent
\textbf{Razumchik Rostislav V.} (b.\ 1984)~--- Candidate of Science (PhD) in 
physics and mathematics, leading scientist, Institute of Informatics 
Problems, Federal Research Center ``Computer Science and Control'' of the 
Russian Academy of Sciences, 44-2~Vavilov Str., Moscow 119333, Russian 
Federation; associate professor, Peoples' Friendship University of Russia 
(RUDN University), 6~Miklukho-Maklaya Str., Moscow 117198, Russian 
Federation; \mbox{rrazumchik@ipiran.ru}

\label{end\stat}

\renewcommand{\bibname}{\protect\rm Литература}          %1
\def\stat{shestakov+vor}

\def\tit{АСИМПТОТИЧЕСКАЯ НОРМАЛЬНОСТЬ И~СИЛЬНАЯ СОСТОЯТЕЛЬНОСТЬ ОЦЕНКИ РИСКА ПРИ~ИСПОЛЬЗОВАНИИ FDR-ПОРОГА В УСЛОВИЯХ СЛАБОЙ ЗАВИСИМОСТИ}

\def\titkol{Асимптотическая нормальность и~сильная состоятельность оценки риска при~использовании FDR-порога} % в~условиях слабой зависимости}

\def\aut{М.\,О.~Воронцов$^1$, О.\,В.~Шестаков$^2$}

\def\autkol{М.\,О.~Воронцов, О.\,В.~Шестаков}

\titel{\tit}{\aut}{\autkol}{\titkol}

\index{Воронцов М.\,О.}
\index{Шестаков О.\,В.}
\index{Vorontsov M.\,O.}
\index{Shestakov O.\,V.}


%{\renewcommand{\thefootnote}{\fnsymbol{footnote}} \footnotetext[1]
%{Работа 
%выполнена при поддержке Программы развития МГУ, проект №\,23-Ш03-03. При анализе 
%данных использовалась инфраструктура Центра коллективного пользования 
%<<Высокопроизводительные вычисления и~большие данные>> 
%(ЦКП <<Информатика>>) ФИЦ ИУ РАН (г.~Москва)}}


\renewcommand{\thefootnote}{\arabic{footnote}}
\footnotetext[1]{Московский государственный университет 
имени~М.\,В.~Ломоносова, факультет вычислительной математики и~кибернетики;  
Московский центр фундаментальной и~прикладной математики, \mbox{m.vtsov@mail.ru}}
\footnotetext[2]{Московский государственный университет 
имени М.\,В.~Ломоносова, факультет вычислительной математики и~кибернетики; 
Федеральный исследовательский центр <<Информатика и~управление>> Российской 
академии наук; Московский центр фундаментальной и~прикладной математики, 
\mbox{oshestakov@cs.msu.ru}}


\vspace*{-12pt}





\Abst{Рассматривается подход к~решению задачи удаления шума в~большом массиве 
разреженных данных, основанный на методе контроля средней доли ложных отклонений 
гипотез (False Discovery Rate, FDR). Данный подход эквивалентен процедурам 
пороговой обработки, обнуляющим компоненты массива, значения которых не 
превосходят некоторого заданного порога.  Наблюдения в~модели считаются слабо 
зависимыми. Для контроля степени зависимости используются ограничения на 
коэффициент сильного перемешивания и~максимальный коэффициент корреляции. 
В~качестве меры эффективности рассматриваемого подхода используется 
среднеквадратичный риск. Вычислить значение риска можно только на тестовых 
данных, поэтому в~работе рассматривается его статистическая оценка и~исследуются 
ее свойства. Показана асимптотическая нормальность и~сильная состоятельность 
оценки риска при использовании FDR-по\-ро\-га в~условиях слабой зависимости в~данных.}

\KW{пороговая обработка; множественная проверка гипотез; 
оценка риска}

\DOI{10.14357/19922264240309}{ZOQVTO}
  
%\vspace*{-6pt}


\vskip 10pt plus 9pt minus 6pt

\thispagestyle{headings}

\begin{multicols}{2}

\label{st\stat}



\section{Введение}

Во многих прикладных областях возникает задача обработки больших массивов 
зашумленных данных. Примерами служат задачи обработки изоб\-ра\-же\-ний с~высоким 
разрешением~\cite{FDRImage}, задачи множественной проверки гипотез, возникающие 
в~\mbox{исследованиях} в~об\-ласти генетики~\cite{MultipleTesting}, и~другие проб\-ле\-мы. 
В~связи с~этим рас\-смот\-рим модель
$$
x_i = \mu_i + z_i, \enskip i=\overline{1,n}\,,
$$
где $\mu_i\in\mathbb{R}$~--- <<полезные>> данные; $z_i \sim N(0,\sigma^2)$~--- 
шум. Задача заключается в~нахождении оценки неизвестного вектора $\mu \hm= 
(\mu_1,\ldots,\mu_n)$ как функции вектора $x \hm= (x_1,\ldots,x_n)$ и~может 
рассматриваться как задача множественной проверки гипотез о~равенстве нулю 
компонент вектора~$\mu$~\cite{AdaptingFDR}. При этом обычно предполагается, что 
вектор~$\mu$ имеет в~определенном смысле <<разреженную>> структуру, т.\,е.\ для 
<<полезных>> данных используется <<экономное>> представление.



В работе~\cite{AdaptingFDR} для решения рассматриваемой задачи в~условиях 
независимости компонент вектора~$x$ и~разреженности вектора~$\mu$ была 
предложена процедура построения оценки~$\hat{\mu}_F$ вектора~$\mu$, основанная 
на методе контроля средней доли ложных отклонений (FDR) 
гипотез при помощи алгоритма Бен\-жа\-ми\-ни--Хох\-бер\-га,
и~было проведено исследование асимптотики ее среднеквадратичного риска. 
В~работах~\cite{ZasShe17,Mathematics2020} была показана состоятельность 
и~асимптотическая нормальность оценки риска данной процедуры. Аналогичные 
результаты для других методов построения~$\hat{\mu}_F$ получены в~работах~\cite{Shestakov2021-1,Shestakov2021-2,Shestakov2022}.

В то же время в~определенных приложениях, например  при анализе полученных 
в~результате использования ДНК-мик\-ро\-чи\-пов данных~\cite{ResultsOnFDRUnderDependence}, исследовании геофизических процессов 
и~анализе помех\linebreak в~телекоммуникационных каналах, условие незави\-си\-мости компонент 
вектора $x$ может не выполняться. Ранее в~работах~\cite{VorontsovShestakov2023,Vorontsov2024} была \mbox{исследована} асимп\-то\-ти\-ка 
среднеквадратичного риска оценки~$\hat{\mu}_F$ \mbox{в~случае}, когда~$\mu$ принадлежит 
одному из классов разреженности
$$
l_0[\eta] = \left\{\mu\,:\, ||\mu||_0 \leq \eta n\right\}, \enskip \eta \in 
(0,1),
$$

\vspace*{-12pt}

\noindent
\begin{multline*}
m_p[\eta] \equiv{}\\
{}\equiv \left\{\mu \in \mathbb{R}^n : |\mu|_{(k)} \leq \eta n^{1/p} 
k^{-1/p},\ k=\overline{1,n}\right\}, \\
 p\in(0, 2),
\end{multline*}
а компоненты вектора~$x$ слабо зависимы~--- имеют достаточно быстро убывающий 
коэффициент сильного перемешивания~\cite{Bosq}

\noindent
\begin{multline*}
\alpha(k) = \sup\limits_{1\leq m\leq n}\alpha\left(\sigma(x_i, i\leq m), 
\sigma(x_i, i\geq m+k)\right), \\ 
k=\overline{1,n-1}\,,
\end{multline*}
где символом $\sigma(x_i, i\in I)$ обозначена сиг\-ма-ал\-геб\-ра, порожденная 
множеством случайных величин $\{x_i, i \hm\in I\}$, а~мера  $\alpha(\cdot, \cdot)$ 
близости двух сиг\-ма-ал\-гебр определяется как
$$
\alpha(\mathcal{B},\mathcal{C}) = \sup\limits_{B\in\mathcal{B}, 
C\in\mathcal{C}} \left|\p(BC)-\p(B)\p(C)\right|.
$$

В настоящей работе показана асимптотическая нормальность и~сильная 
состоятельность оценки риска при применении FDR-про\-це\-ду\-ры в~случае, когда 
компоненты вектора~$x$ слабо зависимы, а~$\mu$ принадлежит одному из классов 
раз\-ре\-жен\-ности: 
$l_0[\eta]$ или $m_p[\eta]$.


\section{Обработка вектора данных с~помощью FDR-процедуры}

Широким классом методов построения оценки~$\hat{\mu}$ стала пороговая обработка 
вектора~$x$ с~некоторым порогом~$T$. Различают жесткую пороговую обработку, при 
которой полагается
\begin{equation*}
\left(\hat{\mu}\right)_i  = p_H(x_i,T) \equiv
 \begin{cases}
   x_i, & |x_i| > T\,;\\
   0, & |x_i| \leq T\,,
 \end{cases}
\end{equation*}
и мягкую пороговую обработку, для которой
\begin{equation*}
(\hat{\mu})_i  = p_S(x_i,T) \equiv
 \begin{cases}
   x_i-T, & \hphantom{\vert\vert}x_i > T;\\
   x_i+T, & \hphantom{\vert\vert}x_i <- T;\\
   0, & |x_i| \leq T.
 \end{cases}
\end{equation*}
Среднеквадратичный риск подобных процедур определяется как
\begin{equation}
\label{riskDef}
R(T) = {\mathsf E} ||\hat{\mu}-\mu||^2 = \sum\limits_{i=1}^n {\mathsf E} \left((\hat{\mu})_i-
\mu_i\right)^2.
\end{equation}
Обозначим через~$T_m$ наилучшее значение порога:
$$
T_m : \, R(T_m) = \min\limits_{T} R(T).
$$

Предложенная в~\cite{AdaptingFDR} процедура заключается в~жесткой пороговой 
обработке компонент вектора~$x$ с~порогом $\hat{t}_F \hm= \hat{t}_F(x)$, и~ее 
результат~--- оценка $\hat{\mu}_F$ вектора~$\mu$ с~компонентами $(\hat{\mu}_F)_i  
\hm= p_H(x_i,\hat{t}_F)$, где
\begin{multline*}
\hat{t}_F = \sigma z\left(\fr{q \hat{k}_F}{2n}\right), \enskip
\hat{k}_F = \max 
\left\{k \, :\, |x|_{(k)} \geq t_k \right\}, \\
 t_k = \sigma z\left(\fr{q  k}{2n}\right);
\end{multline*}
$z(\alpha)$ --- квантиль уровня $(1\hm-\alpha)$ стандартного нормального 
распределения; $|x|_{(k)}$~--- $k$-й элемент вектора, получаемого в~результате 
упорядочения вектора~$|x|$ по невозрастанию:
$$
|x|_{(1)} \geq |x|_{(2)} \geq \cdots \geq |x|_{(n)};
$$
$q\in(0;1)$~--- управ\-ля\-ющий параметр FDR-ме\-то\-да.
Далее полагается, что $q\hm\equiv q_n$ зависит от~$n$. В~\cite{AdaptingFDR} 
показано, что эта процедура эквивалентна множественной проверке гипотез 
о~равенстве нулю компонент наблюдаемого вектора. Также показано, что с~помощью 
метода штрафных функций данную процедуру можно свести к~другим видам пороговой 
обработки, в~част\-ности к~мягкой пороговой обработке.

В работах~\cite{VorontsovShestakov2023, Vorontsov2024} была исследована 
асимптотика среднеквадратичного риска~$R(\hat{t}_F)$ описанной процедуры 
в~случае, когда компоненты вектора $x$ слабо зависимы, а $\mu$ принадлежит классу 
разреженности~$\Theta_n$, где~$\Theta_n$ есть~$l_0[\eta_n]$ или~$m_p[\eta_n]$. 
Было показано, что~$R(\hat{t}_F)$ асимптотически отличается от минимаксного 
риска
$\inf\nolimits_{\hat{\mu}\hm=\hat{\mu}(x)} \sup\nolimits_{\mu\in \Theta_n} {\mathsf E} 
||\hat{\mu}-\mu||^2$
на множитель не более чем логарифмического по\-рядка.

Отметим, что в~выражении для среднеквадратичного риска~(\ref{riskDef}) 
присутствуют неизвестные величины~$\mu_i$, а~потому вычислить~$R(T_m)$ и~$T_m$ 
не представляется возможным. На практике можно пользоваться, например, следующей 
оценкой среднеквадратичного риска~\cite{Mallat}:
$$
\hat{R}(T) = \sum\limits_{i=1}^n F[x_i, T],
$$
где  
\begin{multline*}
F[x_i, T] = {}\\[3pt]
{}=\!\begin{cases}
\left(x_i^2-\sigma^2\right) \Ik(|x_i|\leq T) + \sigma^2 \Ik\left(|x_i|>T\right) &\\[3pt]
&\hspace*{-53mm}\mbox{для\ жесткой\ пороговой\ обработки};\\[3pt]
\left(x_i^2-\sigma^2\right) \Ik\left(|x_i|\leq T\right) + (\sigma^2+T^2) 
\Ik \left(|x_i|>T\right) \hspace*{-11.21576pt}&\\[3pt]
&\hspace*{-51mm}\mbox{для\ мягкой\ пороговой\ обработки}.
\end{cases}\hspace*{-7.17859pt}
\end{multline*}


\noindent
\textbf{Замечание}.\ При пороговой обработке иногда также используется так 
называемый универсальный порог $T_U\hm = \sigma \sqrt{2\ln n}$, предложенный 
в~работе~\cite{spatialAdaptation}. Исследования в~\cite{AdaptingSURE, ExactRisk} 
показали, что порог~$T_U$ в~определенном смысле максимальный, и~рас\-смат\-ри\-вать 
пороги выше него не имеет смысла. Более того, нетрудно показать, что $t_k \hm< T_U$ 
для всех~$k$ и~всех достаточно больших~$n$, в~связи с~чем всюду далее полагаем, 
что порог~$\hat{t}_F$ выбирается на отрезке $[0; T_U]$.

\section{Вспомогательные утверждения}

Кроме коэффициента сильного перемешивания~$\alpha(\cdot)$ также понадобится 
следующее понятие~\cite{Bosq}.

\smallskip

\noindent
\textbf{Определение.} %\label{defRho}
Максимальным коэффициентом корреляции~$\rho(\cdot)$ компонент вектора~$x$ 
называется
\begin{multline*}
\rho (k) \equiv \rho_n (k) = {}\\
{}=\sup\limits_{1\leq m\leq n}\rho\left(\sigma(x_i, 
i\leq m), \sigma(x_i, i\geq m+k)\right), \\
 k=\overline{1,n-1}\,,
\end{multline*}
где мера $\rho(\cdot, \cdot)$ близости двух сиг\-ма-ал\-гебр определяется как
$$
\rho(\mathcal{B},\mathcal{C}) = \sup\limits_{\substack{\xi 
\in\mathcal{L}^2(\mathcal{B}) \\
 \eta \in\mathcal{L}^2(\mathcal{C})}} 
\left|\mathrm{corr}\,(\xi, \eta)\right|.
$$


Введем обозначения:
$$
T_1 = \sqrt{2\ln \eta_n^{-p}};  \,\gamma_n = \fr{1}{\ln\ln n}; \, \kappa_n 
= \fr{n \eta_n^p T_1^{-p}}{1 - q_n - \gamma_n}; 
$$
$$ 
\kappa_n^0 = \fr{[n \eta_n]}{1 - q_n - \gamma_n} ;\, \rho^\star (k) = 
\sup\limits_{n\geq k+1} \rho(k), k \in \mathbb{N} ;
$$
$$
t_{\kappa_n} = \sigma z\left(\fr{q_n \kappa_n }{2n}\right) , \,\, t_{\kappa_n^0} 
= \sigma z\left(\fr{q_n \kappa_n^0 }{2n}\right).
$$


Следующие два утверждения показывают, что случайный порог~$\hat{t}_F$ в~случае 
$\mu\hm\in m_p[\eta_n]$ (соответственно $\mu\hm\in l_0[\eta_n]$) с~большой 
вероятностью будет не меньше~$t_{\kappa_n}$ (соответственно~$ t_{\kappa_n^0}$). 
Их  доказательства приведены в~работах~\cite{VorontsovShestakov2023, Vorontsov2024}.

\smallskip

\noindent
%\begin{lem}\label{lem5}
\textbf{Лемма~1.}\ \textit{Пусть $n^{-\delta_1} \hm\leq \eta_n^p \hm\leq n^{-\delta_2}$, 
$0\hm<\delta_2\hm<\delta_1<1$, $\mathrm{lim\,inf} q_n \ln n \hm\geq C \hm> 0$, 
$m\hm\in[1;n/2]\cap\mathbb{N}$, а $\alpha(\cdot)$~--- коэффициент сильного 
перемешивания компонент вектора~$x$. Для некоторого $N\hm\in\mathbb{N}$ при $n \hm\geq 
N$ справедливо}
\begin{multline*}
\hspace*{-3pt}\sup\limits_{\mu\in m_p[\eta_n]} \p \left(\hat{k}_F \geq \kappa_n \right) \leq 
4 n \exp\left\{-\fr{m}{256n}  \kappa_n q_n \gamma_n^2    \right\}+{}\\
{}+ 22\left(1+\fr{8n}{\kappa_n q_n \gamma_n}\right)^{1/2} n m 
\alpha\left(\left[\fr{n}{2m}\right]\right).
\end{multline*}



\smallskip

\noindent
\textbf{Лемма 2.}\ 
%\label{lem1}
\textit{Пусть $\eta_n \hm\leq b\hm<1$, $m\in[1;n/2]\cap\mathbb{N}$, а~$\alpha(\cdot)$~--- 
коэффициент сильного перемешивания компонент вектора~$x$. Для некоторого 
$N\hm\in\mathbb{N}$ при $n \hm\geq N$ справедливо}
\begin{multline*}
\sup\limits_{\mu\in l_0[\eta_n]} \p \left(\hat{k}_F \geq \kappa_n^0 \right) 
\leq{}\\
{}\leq 4 n \exp\left\{-\fr{(1-b)m}{64n}\,  \kappa_n^0 q_n \gamma_n^2    
\right\}+{}\\
{}+ 22\left(1+\fr{4n}{(1-b)\kappa_n^0 q_n \gamma_n}\right)^{1/2} n m 
\alpha\left(\left[\fr{n}{2m}\right]\right).
\end{multline*}

Следующие два утверждения доказаны в~\cite{Bosq} и~представляют собой аналоги 
неравенств Хеффдинга и~Бернштейна для слабо зависимых случайных величин.


\smallskip

\noindent
\textbf{Лемма 3.}\
\textit{Пусть для набора действительных случайных величин $X_1, \ldots, X_n$ 
с~коэффициентом сильного перемешивания $\alpha(\cdot)$ выполняется ${\mathsf E} X_i \hm=0$, 
$|X_i|\hm\leq b$, $i\hm=\overline{1,n}$. Тогда для любого целого числа $m\hm\in[1; n/2]$ 
и~любого $\eps\hm>0$ справедливо}
\begin{multline*}
\p\left(\left|\sum\limits_{i=1}^n X_i\right| > n\eps \right) \leq 4 
\exp\left\{-\fr{\eps^2 m}{8 b^2}\right\}+ {}\\
{}+
22\left(1+\fr{4b}{\eps}\right)^{1/2} m\, 
\alpha\left(\left[\fr{n}{2m}\right]\right).
\end{multline*}


\smallskip

\noindent
\textbf{Лемма 4.}\
\textit{Пусть для набора действительных случайных величин $X_1, \ldots, X_k$ 
с~коэффициентом сильного перемешивания $\alpha(\cdot)$ выполняется ${\mathsf E} X_i \hm=0$, 
$|X_i|\hm\leq b$, $i\hm=\overline{1,k}$. Тогда для любого целого числа $m\hm\in[1; k/2]$ 
и~любого $\eps\hm>0$ справедливо}
\begin{multline*}
\p\left(\left|\sum\limits_{i=1}^k X_i\right| > \eps \right) \leq 4 
\exp\left\{-\fr{\eps^2 m}{8 v^2 k^2}\right\}+{}\\
{}+ 22\left(1+\fr{4bk}{\eps}\right)^{1/2} m\, 
\alpha\left(\left[\fr{k}{2m}\right]\right),
\end{multline*}
\textit{где $p = k/(2m)$}:
\begin{multline*}
v^2 =
 \fr{b \eps}{2k} + {}\\
 {}+\fr{2}{p^2} \,  \max\limits_{ j\in[0,\,2m-1]} 
{\mathsf E} \big( ([jp]+1-jp)X_{[jp]+1} + X_{[jp]+2}+{}\\
{}+ \cdots +  X_{[(j+1)p]} + ((j+1)p-[(j+1)p])X_{[(j+1)p+1]}\big)^2.
\end{multline*}

\noindent
\textbf{Замечание.}
Если существует такое число $S \hm> 0$, что сразу для всех $i\hm\in[1;k]$  выполняется 
${\mathsf E} X_i^2 \hm\leq S^2$, то в~качестве~$v^2$ можно взять
$$
v^2 = \fr{b \eps}{2k} + 8 S^2.
$$


Д\,о\,к\,а\,з\,а\,т\,е\,л\,ь\,с\,т\,в\,о\ \ сле\-ду\-юще\-го утверж\-де\-ния приведено в~работе~\cite{AdaptingFDR}.

\smallskip

\noindent
\textbf{Лемма 5.}\ 
\textit{Для $y\leq 0{,}01$ справедливы представления}
\begin{multline}
\label{lem1eq1}
z^2(y) = 2 \ln y^{-1} - \ln \ln y^{-1} - r_2(y), \\
 r_2(y) \in [1{,}8; 3];
\end{multline}

\noindent
\begin{equation}
\label{lem1eq2}
z(y) = \sqrt{2 \ln y^{-1}} - r_1(y), \, \, r_1(y) \in [0; 1{,}5].
\end{equation}


\section{Асимптотическая нормальность оценки риска при~применении FDR-процедуры в~условиях слабой зависимости}

Перейдем к~описанию достаточных условий для асимптотической нормальности оценки 
риска $\hat{R}(\hat{t}_F)$ в~случае $\mu \hm\in m_p[\eta_n]$.

\smallskip

\noindent
\textbf{Теорема~1.}\
\textit{Пусть $\mu \hm\in m_p[\eta_n],$ $\eta_n^p \hm\in[n^{-\delta_1}; n^{-\delta_2}],$ $1/2 \hm< 
\delta_2 \hm< \delta_1<1;$ имеются такие константы $c_1, c_2>0$, что для 
коэффициента сильного перемешивания $\alpha(\cdot)$ компонент вектора $x$ 
справедливо  $\alpha(k) \hm\leq c_1 k^{-1-(5/2)\delta_1/(1-\delta_1)-c_2},$ 
$k\hm=\overline{1,n-1};$ $q_n \hm< c_3 \hm< 1;$ $\mathrm{lim\,inf} q_n \ln n \hm= c_4 \hm> 0;$ и,~кроме того, 
для максимального коэффициента корреляции $\rho(\cdot)$ компонент вектора~$x$ 
справедливо}
$$
\sum\limits_{k = 1}^{\infty} \sup\limits_{n\geq k+1} \rho(k) \equiv 
\sum\limits_{k = 1}^{\infty}  \rho^\star (k) = c_5 < \infty. 
$$
\textit{Тогда при $n \to \infty$}
$$
\fr{\hat{R}(\hat{t}_F) - R(T_m)}{C_\rho \sqrt{2n}} \Rightarrow N(0, 1),
$$
\textit{где}
$$
C_\rho = \sigma^2\sqrt{1 +  \lim\limits_{n\to\infty} \fr{1}{n} \sum\limits_{j\neq i} \mathrm{corr}^2 (x_i, x_j)}.
$$

\noindent
Д\,о\,к\,а\,з\,а\,т\,е\,л\,ь\,с\,т\,в\,о\  \
 приводится для метода мягкой пороговой обработки; в~случае жесткой пороговой 
обработки доказательство аналогично. Обозначим
$$
U(T) = \hat{R}(T) -  \hat{R}(T_m) = \sum \limits_{i=1}^n H_i(T, T_m),
$$
где
$$
H_i(T, T_m) = F[x_i, T] - F[x_i, T_m].
$$
Имеем

\vspace*{-3pt}

\noindent
\begin{multline}
\label{D00}
\hat{R}(\hat{t}_F) - R(T_m) + \hat{R}(T_m) - \hat{R}(T_m) ={}\\
{}= \hat{R}(T_m) - 
R(T_m) + U(\hat{t}_F).
\end{multline}
Покажем, что
\begin{equation}
\label{D0}
\fr{\hat{R}(T_m) - R(T_m)}{C_\rho\sqrt{2n}} \Rightarrow N(0, 1).
\end{equation}


Повторяя рассуждения из~\cite{KuShe2016_1,KuShe2016_2,Jansen}, можно показать, 
что $T_m \hm\geq t_{\kappa_n}$. Учитывая также $T_m\hm \leq T_U$, имеем 
$$
C \sqrt{\ln n} \leq T_m \leq C^\prime \sqrt{\ln n}
$$ 
для некоторых положительных констант $C$ и~$C^\prime$.

\columnbreak

В случае мягкой пороговой обработки $\hat{R}(T_m)$ представляет собой 
несмещенную оценку~$R(T_m)$, а~при жесткой пороговой обработке и~выполнении 
условий теоремы смещение стремится к~нулю при делении на $\sqrt{n}$~\cite{Mallat}.

Для дисперсии числителя~(\ref{D0}) имеем:
\begin{multline*}
{\mathsf D} \left(\hat{R}(T_m) - R(T_m)\right) = \sum\limits_{i=1}^n {\mathsf D} F[x_i, T_m] + {}\\
{}+
\sum\limits_{i=1}^n\sum\limits_{\substack{j=1 \\  j\neq i}}^n \mathrm{cov}\left( F[x_i, T_m], F[x_j, 
T_m] \right).
\end{multline*}

Поскольку $\mu \in m_p[\eta_n]$,
\begin{equation}
\left.
\begin{array}{l}
 \displaystyle\sum\limits_{i: |\mu_i| > 1/T_1} {\mathsf D} F[x_i, T_m]  \leq{}\\
 \hspace*{15mm}{}\leq  4\left(\sigma^2 + T_m^2\right)^2 n \eta_n^p 
T_1^p = o(n);
\\[6pt]
\displaystyle \sum\limits_{\substack{{i,j: \max\{|\mu_i|, |\mu_j|\} > 1/T_1,}\\{j\neq i}}}  \hspace*{-12mm}\mathrm{cov}\,(F[x_i, 
T_m],F[x_j, T_m])  \leq{}\\
\hspace*{10mm}{}\leq 16\left(\sigma^2 + T_m^2\right)^2 n \eta_n^p T_1^p c_5 = o(n). 
\end{array}
\right\}    
\label{D2}
\end{equation}
Далее, учитывая что ${\mathsf D} x_i^2 \hm= 2\sigma^4 \hm+ 4\sigma^2 \mu_i^2$, нетрудно 
убедиться, что
\begin{multline}
\label{D3}
\sum\limits_{i: |\mu_i| \leq 1/T_1}\hspace*{-4mm} {\mathsf D} F[x_i, T_m] ={}\\
{}= \sum\limits_{i: |\mu_i| \leq 1/T_1} \hspace*{-4mm} {\mathsf D} 
x_i^2 + o(n) = 2\sigma^4 n + o(n).
\end{multline}


Введем обозначение 
$$
D_n = \left\{(i,j) : \max\left\{|\mu_i|, |\mu_j|\right\}  \leq \fr{1}{T_1}\,, \enskip j\hm\neq i\right\}.
$$
 Для суммы ковариаций аналогично~(\ref{D3}) получим
\begin{multline*}
\sum\limits_{(i,j)\in D_n} \hspace*{-2mm}\mathrm{cov}\left( F[x_i, T_m], F[x_j, T_m] \right) = {}\\
{}=
\sum\limits_{(i,j)\in D_n} \hspace*{-2mm}\mathrm{cov}\left( x_i^2, x_j^2 \right) + o(n).
\end{multline*}
Воспользуемся тождеством~\cite{Eroshenko}
$$
\mathrm{cov}\left (x_i^2, x_j^2\right) = 4 {\mathsf E} x_i {\mathsf E} x_j \mathrm{cov}\left(x_i, x_j\right) + 2 \mathrm{cov}^2 \left(x_i, x_j\right)
$$
для вектора $(x_i, x_j)$, имеющего двумерное нормальное распределение. Заметим, 
что
\begin{gather*}
 \sum\limits_{(i,j)\in D_n} 4 | {\mathsf E} x_i {\mathsf E} x_j \mathrm{cov}\left(x_i, x_j\right)| \leq 8 T_1^{-2} 
\sigma^2 n c_5 = o(n);
\\
\sum\limits_{(i,j)\in D_n} 2 \mathrm{cov}^2 (x_i, x_j)  = 2\sigma^4 \sum\limits_{(i,j)\in D_n} 
\mathrm{corr}^2 (x_i, x_j). 
\end{gather*}
Более того, поскольку  %< 4 \sigma^2 n c_5.$$
\begin{equation*}
\sum\limits_{\substack{{i,j: \max\{|\mu_i|, |\mu_j|\} > 1/T_1} \\ {j\neq i}}}
\hspace*{-10mm}\mathrm{corr}^2 (x_i, x_j)  
\leq  4 n \eta_n^p T_1^p c_5 =  o(n),
\end{equation*}
имеем
\begin{multline*}
\sum\limits_{(i,j)\in D_n} \mathrm{corr}^2 (x_i, x_j) ={}\\
{}= \sum\limits_{j\neq i} \mathrm{corr}^2 (x_i, x_j) 
+o(n)= c_6 n + o(n),
\end{multline*}
где
$$
c_6 = \lim\limits_{n\to\infty} \fr{1}{n} \sum\limits_{j\neq i} \mathrm{corr}^2 (x_i, x_j) 
\leq 2 c_5.
$$
Полагая $C_\rho \hm= \sigma^2\sqrt{1 + c_6}$, получим, наконец,
\begin{equation}
\label{D1}
{\mathsf D} \left(\hat{R}(T_m) - R(T_m)\right)  =  2 n C_\rho^2 + o(n).
\end{equation}
Заметим, что из~(\ref{D2}), (\ref{D3}) и~(\ref{D1}) следует, что
\begin{equation}
\label{D5}
\sup\limits_{n} \fr{\sum\nolimits_{i=1}^n {\mathsf D} F[x_i, T_m]}{V_n^2} < \infty\,,
\end{equation}
где 
$$
V_n^2 = {\mathsf D} \sum\limits_{i=1}^n \left(F[x_i, T_m] \hm- {\mathsf E} F[x_i, T_m]\right).
$$
Кроме того, поскольку $F[x_i, T_m]$ по модулю ограничены величиной $\sigma^2 \hm+ 
T_m^2$, выполнено условие Линдеберга: для любого $\eps\hm>0$ при $n \hm\to \infty$
\begin{multline}
\label{D6}
\!\!\!\fr{1}{V_n^2}\sum\limits_{i=1}^n {\mathsf E} \left( \!\left( F\left[x_i, T_m\right]\! -\! {\mathsf E} F\left[x_i, T_m\right]\right)^2 
\Ik \left(\vert F\left[x_i, T_m\right] -{}\right.\right.\hspace*{-2.69505pt}\\
\left.\left.{}- {\mathsf E} F\left[x_i, T_m\right]\vert >\eps V_n\right)\!
\vphantom{\left( F\left[x_i, T_m\right]\! -\! {\mathsf E} F\left[x_i, T_m\right]\right)^2}
\right) 
\to  0\,.
\end{multline}
Из~(\ref{D1})--(\ref{D6}), очевидного неравенства
$$ 
\lim\limits_{k\to\infty} \sup\limits_{n\geq k+1}\rho(k) \equiv 
\lim\limits_{k\to\infty} \rho^\star (k)  < 1
$$
 и~центральной предельной теоремы для сильно перемешанных случайных величин~\cite{Peligrad} следует~(\ref{D0}).

Перейдем к~доказательству того, что $U(\hat{t}_F) \, n^{-1/2} \overset{\, \p \, }{\to} 0$.
Всюду далее, не ограничивая общности, полагаем $\sigma=1$. 
Введем обозначения:

\noindent
\begin{align*}
S_1(T) &= \sum\limits_{i: |\mu_i| > 1/T_1} H_i(T, T_m); \\
S_2(T) &= \sum\limits_{i: |\mu_i| \leq 1/T_1} H_i(T, T_m); 
\\
N_1(a, b) &= \sum\limits_{i: |\mu_i| > 1/T_1} \Ik (a<|x_i|\leq b); \\ 
N_2(a, b) &= \sum\limits_{i: |\mu_i| \leq 1/T_1} \Ik (a<|x_i|\leq b);
\end{align*}

\noindent
\begin{align*}
Z_l(T) &= S_l(T) - {\mathsf E} S_l(T),\enskip l = 1,2\,; \\  
d_n &= \fr{T_U -  t_{\kappa_n}}{n};\\
T_j^{\prime} &= t_{\kappa_n}+j d_n,\enskip j = \overline{0,n-1}\,.
\end{align*} 

\vspace*{-3pt}

\noindent
Для произвольного $\eps>0$

\vspace*{-3pt}

\noindent
\begin{multline}
\p \left( \fr{|U(\hat{t}_F)|}{\sqrt{n}}> 4\eps \right) \leq 
\p\left(\hat{t}_F \leq t_{\kappa_n}\right) + {}\\
{}+\p \left(\fr{\sup\nolimits_{T\in 
[t_{\kappa_n}, T_U]} |U(T)|}{\sqrt{n}}>4\eps \right)\leq  {}\\
{}\leq \p\left(\hat{t}_F \leq t_{\kappa_n}\right) + \p\left(\fr{\sup\nolimits_{T\in 
[t_{\kappa_n}, T_U]} |{\mathsf E} U(T)|}{\sqrt{n}}>\eps\right)+{}\\
{}+ \p \left(\sup\limits_{T\in [t_{\kappa_n}, T_U]} |Z_1(T)| > 
\eps\sqrt{n}\right) +{}\\
{}+ \p \left(\sup\limits_{j \in [0, n-1]} |Z_2(T_j^{\prime})| > 
\eps\sqrt{n}\right) +{}\\
{}+ \p \left(\sup\limits_{\substack{j \in [0, n-1] \\
 T\in [T_j^{\prime},T_j^{\prime}+d_n]}} |Z_2(T)-Z_2(T_j^{\prime})| > \eps\sqrt{n}\right).
\label{M1}
\end{multline}
Заметим, что $\gamma_n\hm > \ln^{-1} n$, $\kappa_n\hm > n \eta_n^p \ln ^{-1} n \hm\geq 
n^{1-\delta_1} \ln ^{-1} n$ и~$q_n\hm > c_4 \ln ^{-1} n /2$ для всех достаточно 
больших~$n$.
Для первого слагаемого в~(\ref{M1}) по лемме~1 с~$m \hm= n^{\delta_1} \ln 
^7 n$ для  больших~$n$ имеем

\vspace*{-3pt}

\noindent
\begin{multline}
\label{M1next}
\p\left(\hat{t}_F \leq t_{\kappa_n}\right)  = \p \left(\hat{k}_F \geq \kappa_n 
\right) \leq 4 n e^{-\ln^2 n} + {}\\
{}+n^{1+(3/2)\,\delta_1} \ln^9 n \, 
\alpha\left(\left[\fr{n^{1-\delta_1}}{\ln^{7} n}\right]\right) = o(1)
\end{multline}
при $n\to\infty$. 
Для оценки второго слагаемого в~(\ref{M1}) заметим, что при $T \hm\in 
[t_{\kappa_n}, T_U]$ справедливо
\begin{equation}
\label{M2}
{\mathsf E} H_i(T, T_m) \leq T_U^2 + 1.
\end{equation}
Если же кроме $T \hm\in [t_{\kappa_n}, T_U]$ также выполнено $|\mu_i| \hm\leq T_1^{-1}$, то

\vspace*{-6pt}

\noindent
\begin{multline*}
|{\mathsf E} H_i (T, T_m)| \leq 2 T_U^2 \, \p \left(|x_i| > t_{\kappa_n}\right) \leq {}\\
{}\leq2 
T_U^2 \, \p \left(|x_i-\mu_i| > t_{\kappa_n}-T_1^{-1}\right) \leq{}\\
{}\leq 2 T_U^2  \exp\left\{ -\fr{1}{2} \left(t_{\kappa_n} - T_1^{-
1}\right)^2 \right\}  \leq{}\\
{}\leq
 4 (\ln n)  \exp\left\{ -\fr{1}{2} 
\left(z\left(\fr{q_n\kappa_n}{2n}\right)\right)^2 + t_{\kappa_n} T_1^{-
1}\right\},
\end{multline*}

\vspace*{-2pt}

\noindent
где использовано неравенство 

\noindent
$$
2(1-\Phi(x))\hm \leq \fr{e^{-x^2/2}}{x}
$$

\pagebreak


\noindent
 для $x\hm\geq 0$ 
($\Phi(x)$~--- функция распределения $N(0,1)$). Рас\-смот\-рим выражение 
в~экспоненте. Второе слагаемое не превышает $1\hm+o(1)$ при $n\hm\to\infty$, поскольку 
$t_{\kappa_n} \hm\leq T_1 (1+o(1))$ при $\sigma\hm=1$, что нетрудно получить из 
определения~$t_{\kappa_n}$, пред\-став\-ле\-ния~(\ref{lem1eq2}) и~ограничения на~$q_n$ 
из формулировки тео\-ре\-мы. Для первого слагаемого, используя пред\-став\-ле\-ние~(\ref{lem1eq1}) 
и~ограничения, наложенные на~$q_n$, при больших~$n$ получим
\begin{multline*}
-\fr{1}{2}\left(z\left(\fr{q_n \kappa_n}{2n}\right)\right)^2 \leq - \ln 
\fr{2n (1-q_n-\gamma_n)}{q_n n \eta_n^p T_1^{-p}} + {}\\
{}+\fr{1}{2} \ln 
\left((1+o(1)) \ln \eta_n^{-p}\right) + \fr{3}{2} \leq{}\\
{}\leq \ln \fr{c_3}{1-c_3} + \ln \eta_n^p + \ln T_1^{-p} + \ln T_1 + 
\fr{3}{2}+ o(1).
\end{multline*}
Из приведенных соотношений следует, что с~некоторой константой $c_7 = c_7(c_3, 
p, \delta_1, \delta_2, c_4)$
\begin{equation}\label{M3}
\sup\limits_{\substack{i: |\mu_i| \leq 1/T_1 \\ T\in [t_{\kappa_n}, T_U]}} |{\mathsf E} 
H_i (T, T_m)|  \leq c_7 (\ln n)^{(3-p)/2}\eta_n^p.
\end{equation}
Из (\ref{M2}) и~(\ref{M3}) с~учетом $\delta_2 \hm> 1/2$ следует
\begin{multline*}
\sup\limits_{T\in [t_{\kappa_n}, T_U]} |{\mathsf E} U(T)| \leq{}\\
{}\leq 
 n\eta_n^p T_1^p 
(T_U^2+1) + c_7 (\ln n)^{(3-p)/2} n \eta_n^p = o(\sqrt{n})
\end{multline*}
при $n\to\infty$, а следовательно, для любого $\eps\hm>0$ второе слагаемое в~(\ref{M1}) обращается в~ноль для всех достаточно больших~$n$.

Далее, поскольку при $T \hm\leq T_U$ и~$\sigma\hm=1$
$$
|H_i(T, T_m) - {\mathsf E} H_i(T, T_m)| \leq 2 (T_U^2 +2), \enskip i=\overline{1, n}\,,
$$
а число слагаемых в~$Z_1(T)$ не превосходит $n\eta_n^p T_1^p$, имеем
$$
\sup\limits_{T\in [t_{\kappa_n}, T_U]} |Z_1(T)|  \leq 2 n\eta_n^p T_1^p (T_U^2 
+2) = o(\sqrt{n})
$$
при $n\to\infty$, а следовательно, для любого $\eps\hm>0$ и~третье слагаемое в~(\ref{M1}) обращается в~ноль для всех достаточно больших~$n$.

Перейдем к~оценке четвертого слагаемого в~(\ref{M1}). Аналогично~(\ref{M3}) 
можно получить:
\begin{multline}
\label{M10}
\!\!\sup\limits_{\substack{i: |\mu_i| \leq 1/T_1 \\ T\in [t_{\kappa_n}, T_U]}} \!{\mathsf D} 
H_i (T, T_m)  \leq \!\sup\limits_{\substack{i: |\mu_i| \leq 1/T_1 \\ T\in 
[t_{\kappa_n}, T_U]}} \!{\mathsf E} \left(H_i (T, T_m)\right)^2  \leq{}\\
{}\leq 2 c_7 (\ln n)^{(5-p)/2} \eta_n^p.
\end{multline}
По лемме~4 с~$m \hm= \sqrt{n} (\ln n)^3$ и~$k \hm= n-[n\eta_n^p T_1^p]$ 
для четвертого слагаемого в~(\ref{M1}) имеем:

\noindent
\begin{multline}
\p \left(\sup\limits_{j \in [0, n-1]} |Z_2(T_j^\prime)| > \eps\sqrt{n}\right) 
\leq {}\\
{}\leq \sum\limits_{j \in [0, n-1]} \hspace*{-3mm}\p \left( |Z_2(T_j^\prime)| > \varepsilon\sqrt{n}\right)\leq{}\\
{}\leq 4 n \exp \left\{ - \fr{\eps^2 n^{3/2} (\ln n)^3}{n-[n\eta_n^p T_1^p]}\!\Bigg/\! \big( 8 (T_U^2+2)\eps\sqrt{n} +{}\right.\\
\left.{}+ 128 c_7 (\ln n)^{(5-p)/2} \eta_n^p  (n-
[n\eta_n^p T_1^p])\big) 
\vphantom{ \fr{\eps^2 n^{3/2} (\ln n)^3}{n-[n\eta_n^p T_1^p]}}
\right\} +{}\\
{}
+ 22 \left(1+\fr{8(T_U^2+2) (n-[n\eta_n^p T_1^p])}{\eps 
\sqrt{n}}\right)^{1/2}\times{}\\
{}\times n^{3/2} (\ln n)^3 \alpha\left(\left[\fr{n-[n\eta_n^p 
T_1^p]}{2 (\ln n)^3 \sqrt{n}}\right]\right).
\label{M5}
\end{multline}
Используя ограничения $n^{-\delta_1}\hm\leq \eta_n^p \leq n^{-\delta_2}$ 
и~$1/2\hm<\delta_2\hm<\delta_1\hm<1$, из~(\ref{M5}) получим для любого $\eps\hm>0$
$$
\p \left(\sup\limits_{j \in [0, n-1]} |Z_2(T_j^\prime)| > \eps\sqrt{n}\right) 
\to 0
$$
при $n \to \infty$.

Рассмотрим, наконец, пятое слагаемое в~(\ref{M1})). Заметим, что при $0\hm< a \hm< b$ 
справедливо
$$
|Z_2(b)-Z_2(a)| \leq 2 |N_2(a,b)-{\mathsf E} N_2(a,b)| + n (b^2-a^2).
$$
Полагая $a = T_j^\prime$, $b \hm= T \hm\in [T_j^\prime, T_j^\prime+d_n]$ для 
произвольного $j \hm\in [0, n-1]$ и~учитывая, что
$$
(T^2 - (T_j^\prime )^2) = (T - T_j^\prime)(T+ T_j^\prime ) \leq  2 d_n T_U < 2 
T_U^2 n^{-1}; 
$$

\vspace*{-12pt}

\noindent
\begin{multline*}
\p\left(T_j^\prime < |x_i| \leq T \right) \leq \p\left(T_j^\prime < |x_i| \leq 
T_j^\prime+d_n\right) <{}\\
{}< d_n < T_U n^{-1}, 
\end{multline*}
получим  оценку
$$
|Z_2(T)-Z_2(T_j^\prime)| \leq 2 N_2(T_j^\prime, T) +  3 T_U^2 .
$$
Далее, поскольку $N_2 (T_j^\prime, T) \hm\leq N_2 (T_j^\prime, T_j^\prime+d_n)$ и~${\mathsf E} N_2 (T_j^\prime, T_j^\prime+d_n) \hm< T_U^2$,
имеем
\begin{multline*}
\sup\limits_{T \in [T_j^\prime, T_j^\prime+d_n]} |Z_2(T)-Z_2(T_j^\prime)| \leq {}\\
{}\leq
2 \left|N_2 (T_j^\prime, T_j^\prime+d_n) - {\mathsf E} N_2 (T_j^\prime, 
T_j^\prime+d_n)\right| +  5 T_U^2 .
\end{multline*}
Аналогично~(\ref{M3}) показывается, что
\begin{multline}
\label{M11}
\sup\limits_{\substack{i : |\mu_i| \leq 1/T_1 \\ j \in [0, n-1]}} {\mathsf D} \Ik 
(T_j^\prime < |x_i| \leq T_j^\prime + d_n) <{}\\
{}< c_7 (\ln n)^{(1-p)/2} \eta_n^p.
\end{multline}
Пусть $n > N(\eps)$ настолько, что 
$$
\fr{\eps\sqrt{n} - 5 T_U^2}{2} > \fr{\eps \sqrt{n} }{4}\,.
$$
%
 Тогда для пятого слагаемого в~(\ref{M1}) по лемме~4 с~$m \hm= 
\sqrt{n} (\ln n)^2$ и~$k \hm= n\hm-[n\eta_n^p T_1^p]$ имеем
\begin{multline}
\p \left(\sup\limits_{\substack{j \in [0, n-1] \\ T\in 
[T_j^{\prime},T_j^{\prime}+d_n]}} |Z_2(T)-Z_2(T_j^{\prime})| > 
\eps\sqrt{n}\right) \leq{}\\
{}\leq  \sum\limits_{j \in [0, n-1]} \p \left(  \left|N_2 (T_j^\prime, 
T_j^\prime+d_n) -{}\right.\right.\\
\left.\left.{}- {\mathsf E} N_2 (T_j^\prime, T_j^\prime+d_n)\right| > \fr{\eps\sqrt{n}}{4} 
\right) \leq{}\\
{}\leq  4n \exp \left\{ -  \fr{\eps^2 n^{3/2} (\ln n)^2}{(n-[n\eta_n^p T_1^p])^{-1}}\Bigg/ 
\big( 16 \eps \sqrt{n} +{}\right.\\
\left.{}+ 64 c_7 (\ln n)^{(1-p)/2} \eta_n^p (n-[n\eta_n^p 
T_1^p]) \big) 
\vphantom{\fr{\eps^2 n^{3/2} (\ln n)^2}{(n-[n\eta_n^p T_1^p])^{-1}}}
\right\} +{}\\
{}+ 22 \left(1+\fr{16 (n-[n\eta_n^p T_1^p])}{\eps \sqrt{n}}\right)^{1/2}\times{}\\
{}\times 
n^{3/2} (\ln n)^2 \alpha\left(\left[\fr{n-[n\eta_n^p T_1^p]}{2 (\ln n)^2 
\sqrt{n}}\right]\right).
\label{M6}
\end{multline}
Используя ограничения $n^{-\delta_1}\hm\leq \eta_n^p\hm \leq n^{-\delta_2}$ 
и~$1/2\hm<\delta_2\hm<\delta_1<1$, из~(\ref{M6}) получим для любого $\eps\hm>0$
$$
\p \left(\sup\limits_{\substack{j \in [0, n-1] \\ T\in 
[T_j^{\prime},T_j^{\prime}+d_n]}} |Z_2(T)-Z_2(T_j^{\prime})| > 
\eps\sqrt{n}\right) \to 0
$$
при $n \to \infty$.

Таким образом, показано, что для любого $\eps>0$ все слагаемые в~(\ref{M1}) 
стремятся к~нулю при $n\to\infty$. Следовательно,
$$
\fr{|U(\hat{t}_F)|}{\sqrt{n}}  \overset{\, \p \, }{\to} 0 \,,
$$
что вместе с~(\ref{D0}) завершает доказательство тео\-ремы.~\hfill$\square$

\smallskip

Следующая теорема дает достаточные условия для асимптотической нормальности 
оценки риска $\hat{R}(\hat{t}_F)$ в~случае $\mu \hm\in l_0[\eta_n]$.

\smallskip

\noindent
\textbf{Теорема 2.}\ 
\textit{Пусть $\mu \hm\in l_0[\eta_n]$, $\eta_n\hm\in[n^{-\delta_1}, n^{-\delta_2}]$, $1/2\hm < 
\delta_2\hm < \delta_1\hm<1;$ имеются такие константы $c_1, c_2\hm>0$, что для 
коэффициента сильного перемешивания $\alpha(\cdot)$ компонент вектора~$x$ 
справедливо} 
\begin{gather*}
\alpha(k) \leq c_1 k^{-1-(5/2)\delta_1/(1\hm-\delta_1)\hm-c_2},\enskip 
k=\overline{1,n-1};\\
 q_n < c_3 < 1;\enskip \mathrm{lim\,inf} q_n \ln n = c_4 > 0;
\end{gather*}
\textit{для максимального коэффициента корреляции~$\rho(\cdot)$ компонент вектора~$x$ 
справедливо}
$$
\sum\limits_{k = 1}^{\infty} \sup\limits_{n\geq k+1} \rho(k) \equiv 
\sum\limits_{k = 1}^{\infty}  \rho^\star (k) = c_5 < \infty. 
$$
\textit{Тогда при $n \to \infty$}
$$
\fr{\hat{R}(\hat{t}_F) - R(T_m)}{C_\rho \sqrt{2n}} \Rightarrow N(0, 1),
$$
\textit{где}
$$
C_\rho = \sigma^2\sqrt{1 +   \lim\limits_{n\to\infty} \fr{1}{n} 
\sum\limits_{j\neq i} \mathrm{corr}^2 (x_i, x_j)}\,.
$$

\noindent
Д\,о\,к\,а\,з\,а\,т\,е\,л\,ь\,с\,т\,в\,о\  проводится аналогично доказательству теоремы~1. 
Переменная~$D_n$ теперь определяется как $D_n \hm= \{(i,j) : 
|\mu_i|\hm=|\mu_j|=0$, $j\hm\neq i\}$. Условия вида $|\mu_i|\hm<T_1^{-1}$ (вида 
$|\mu_i|\hm\geq T_1^{-1}$) заменяются условиями  $\mu_i\hm=0$ (соответственно 
$|\mu_i|\hm>0$).
Поскольку $\mu \hm\in l_0[\eta_n]$, количество~$i$ таких, что $|\mu_i|\hm>0$ 
(а~значит, и~число слагаемых в~$Z_1(T)$), не превышает~$[n \eta_n]$.

Для оценки первого слагаемого в~(\ref{M1}) используется лемма~2, 
в~которой можно взять, например, $b\hm=1/2$, а~для~$\kappa_n^0$ использовать оценку 
$\kappa_n^0 \hm> n \eta_n$. Формулы (\ref{M3}),  (\ref{M10}) и~(\ref{M11}) 
принимают вид соответственно
\begin{align*}
\sup\limits_{\substack{i: \mu_i =0 \\ T\in [t_{\kappa_n^0}, T_U]}} |{\mathsf E} H_i (T, 
T_m)| & \leq c_8 (\ln n)^{3/2} \eta_n ;
\\
\sup\limits_{\substack{i: \mu_i =0 \\ T\in [t_{\kappa_n^0}, T_U]}} {\mathsf D} H_i (T, 
T_m)  & \leq 2 c_8 (\ln n)^{5/2} \eta_n;
\\
\sup\limits_{\substack{i : \mu_i =0 \\ j \in [0, n-1]}} {\mathsf D} \Ik (T_j^\prime < 
|x_i| \leq T_j^\prime + d_n) &< c_8 (\ln n)^{1/2} \eta_n,
\end{align*}
где $c_8 = c_8(c_3,\delta_1, \delta_2, c_4)$. В~остальном доказательство 
аналогично.~\hfill$\square$

\section{Сильная состоятельность оценки риска при~применении FDR-процедуры 
в~условиях слабой зависимости}

Следующая теорема дает достаточные условия для сильной состоятельности оценки 
риска $\hat{R}(\hat{t}_F)$ в~случаях $\mu \hm\in m_p[\eta_n]$ и~$\mu \hm\in 
l_0[\eta_n]$.

\smallskip

\noindent
\textbf{Теорема 3.}
\textit{Пусть $\mu\hm \in m_p[\eta_n]$, $\eta_n^p\hm\in[n^{-\delta_1}, n^{-\delta_2}]$ либо 
$\mu \hm\in l_0[\eta_n]$, $\eta_n\hm\in[n^{-\delta_1}, n^{-\delta_2}]$; $0 \hm< \delta_2 
\hm< \delta_1<1$; имеются такие константы $c_1, c_2\hm>0$, что для коэффициента 
сильного перемешивания $\alpha(\cdot)$ компонент вектора~$x$ справедливо}  
$\alpha(k) \hm\leq c_1 k^{-2-(7/2)\delta_1/(1\hm-\delta_1)\hm-c_2}$, $k\hm=\overline{1,n-1}$; 
$q_n \hm< c_3 \hm< 1$; $\mathrm{lim\,inf} q_n \ln n \hm= c_4 \hm> 0$. \textit{Тогда при} $n \hm\to \infty$
$$
\fr{\hat{R}(\hat{t}_F) - R(T_m)}{n} \rightarrow 0 \, \, \,\textit{п.~в.}
$$


\noindent
Д\,о\,к\,а\,з\,а\,т\,е\,л\,ь\,с\,т\,в\,о\,.  Воспользуемся представлением~(\ref{D00}).

Покажем, что $(\hat{R}(T_m)-R(T_m))n^{-1}\hm \to 0$ п.~в.\ при $n\hm\to\infty$. 
При мягкой пороговой обработке ${\mathsf E} \hat{R}(T_m) \hm= R(T_m)$, а~при жесткой 
пороговой обработке
\begin{multline*}
\fr{\hat{R}(T_m)-R(T_m)}{n} = {}\\
{}=\fr{\hat{R}(T_m)-{\mathsf E} \hat{R}(T_m)}{n} 
+\fr{{\mathsf E}\hat{R}(T_m)-R(T_m)}{n}\,,
\end{multline*}
где второе слагаемое стремится к~нулю при $n\to\infty$ \cite{Mallat}. 
Следовательно, достаточно показать, что $(\hat{R}(T_m)\hm-{\mathsf E}\hat{R}(T_m))n^{-1} \hm\to 0$ п.~в.

Полагая в~лемме~3 $X_i \hm= F[x_i, T_m] \hm- {\mathsf E} F[x_i, T_m]$, $b \hm= 
2(\sigma^2\hm+T_m^2)$ и~$m \hm= n^{1/4}$ и~учитывая ограничения на $\alpha(\cdot)$ из 
условия, нетрудно убедиться, что для всех~$n$
$$
\p \left(\left| \fr{\hat{R}(T_m)-{\mathsf E} \hat{R}(T_m)}{n}\right| >\eps \right) 
\leq \fr{c_5}{n^{1+c_6}}\,, 
$$
где константы $c_5$, $c_6$ положительны. Отсюда
$$
\sum\limits_{n=1}^{\infty}\p \left(\left|\fr{\hat{R}(T_m)-{\mathsf E} 
\hat{R}(T_m)}{n}\right| >\eps \right) < \infty,
$$
и по теореме~1.3.4 из~\cite{Serfling2002} 
$$
\left(\hat{R}(T_m)-{\mathsf E}\hat{R}(T_m)\right)n^{-1} \to 0~\mbox{п.~в.}
$$



Покажем теперь, что  $U(\hat{t}_F) \, n^{-1}\hm \to 0$ п.~в. Доказательство 
проведено для $\mu \hm\in m_p[\eta_n]$, в~случае $\mu\hm \in l_0[\eta_n]$ 
доказательство аналогично.
Аналогично формуле~(\ref{M1}), для произвольного $\eps\hm>0$ в~терминах тео\-ре\-мы~1 имеем
\begin{multline*}
\p \left( \fr{|U(\hat{t}_F)|}{n}> 4\eps \right) \leq \p\left(\hat{t}_F 
\leq t_{\kappa_n}\right) +{}\\
{}+ \p\left(\fr{\sup\nolimits_{T\in [t_{\kappa_n}, T_U]} |{\mathsf E} 
U(T)|}{n}>\eps\right)+{}\\
{}+ \p \left(\sup\limits_{T\in [t_{\kappa_n}, T_U]} |Z_1(T)| > \eps n\right) +{}
\end{multline*}

\noindent
\begin{multline}
{}+ \p  \left(\sup\limits_{j \in [0, n-1]} |Z_2(T_j^{\prime})| > \eps n\right) +{}\\
{}+ \p \left(\sup\limits_{\substack{j \in [0, n-1] \\ T\in 
[T_j^{\prime},T_j^{\prime}+d_n]}} |Z_2(T)-Z_2(T_j^{\prime})| > \eps n\right).
\label{M1SC}
\end{multline}
Применяя рассуждения, аналогичные приведенным в~доказательстве теоремы~1, можно показать, что
$$
\sup\limits_{T\in [t_{\kappa_n}, T_U]} |{\mathsf E} U(T)| = o(n); \enskip
\sup\limits_{T\in [t_{\kappa_n}, T_U]} |Z_1(T)|  = o(n),
$$
откуда следует, что второе и~третье слагаемые в~(\ref{M1SC}) обращаются в~ноль 
для всех достаточно больших~$n$.

Для некоторых положительных констант  $c_7$ и~$c_8$ первое, четвертое и~пятое 
слагаемые  в~(\ref{M1SC}) не превышают $c_7 n^{-1-c_8}$ для всех достаточно 
боль\-ших~$n$, что можно показать с~помощью ограничения на $\alpha(\cdot)$ из 
условия и~рассуждений, аналогичных приведенным при выводе соответственно формул~(\ref{M1next}), (\ref{M5}) и~(\ref{M6}), с~тем отличием, что при применении 
леммы~4 полагается $m \hm= (\ln n)^3$.

Из доказанного следует, что
$$
\sum\limits_{n=1}^{\infty}\p \left( \fr{|U(\hat{t}_F)|}{n}> 4\eps \right) 
< \infty,
$$
и по теореме~1.3.4 из~\cite{Serfling2002} $U(\hat{t}_F) \, n^{-1} \to 0$ п.~в., 
что завершает доказательство теоремы.~\hfill$\square$



{\small\frenchspacing
 {\baselineskip=11.5pt
 %\addcontentsline{toc}{section}{References}
 \begin{thebibliography}{99}
\bibitem{FDRImage}
\Au{Krylov V.\,A., Moser~G., Serpico~S.\,B., Zerubia~J.}
False discovery rate approach to unsupervised image change detection~// IEEE 
T. Image Process., 2016. Vol.~25. No.\,10. P.~4704--4718. doi: 10.1109/TIP.2016.2593340.

\bibitem{MultipleTesting} %2
\Au{Menyhart~O., Weltz~B., Gyorffy~B.}
MultipleTesting.com: A~tool for life science researchers for multiple hypothesis 
testing correction~// PLoS One, 2021. Vol.~16. No.\,6. Art.~0245824. doi: 10.1371/journal.pone.0245824.

\bibitem{AdaptingFDR} %3
\Au{Abramovich~F., Benjamini~Y., Donoho~D., Johnstone~I.}
Adapting to unknown sparsity by controlling the false discovery rate~// Ann. Stat., 2006. Vol.~34. No.\,2. P.~584--653.
doi: 10.1214/009053606000000074.

\bibitem{ZasShe17} %4
\Au{Заспа~А.\,Ю., Шестаков~О.\,В.}
Состоятельность оценки риска при множественной проверке гипотез с~FDR-по\-ро\-гом~// 
Вестник ТвГУ. Сер. Прикладная математика, 2017. Вып.~1. С.~5--16.
doi: 10.26456/vtpmk119. EDN: YFYJXT.

\bibitem{Mathematics2020} %5
\Au{Palionnaya~S.\,I., Shestakov~O.\,V.}
Asymptotic properties of MSE estimate for the false discovery rate controlling 
procedures in multiple hypothesis testing // Mathematics, 2020. Vol.~8. No.~11. 
Art.~1913. 11~p. doi: 10.3390/ math8111913.

\bibitem{Shestakov2021-1} %6
\Au{Шестаков~О.\,В.}
Анализ несмещенной оценки среднеквадратичного риска метода блочной пороговой 
обработки~// Информатика и~её применения, 2021. Т.~15. Вып.~2. С.~30--35.
doi: 10.14357/19922264210205. EDN: DSQQAU.

\bibitem{Shestakov2021-2} %7
\Au{Шестаков~О.\,В.}
Пороговые функции в~методах подавления шума, основанных на вейв\-лет-раз\-ло\-же\-нии 
сигнала~// Информатика и~её применения, 2021. Т.~15. Вып.~3. С.~51--56.
doi: 10.14357/19922264210307. EDN: WSEAYG.

\bibitem{Shestakov2022} %8
\Au{Шестаков~О.\,В.}
Несмещенная оценка риска пороговой обработки с~двумя пороговыми значениями~// 
Информатика и~её применения, 2022. Т.~16. Вып.~4. С.~14--19.
doi: 10.14357/19922264220403. EDN: \mbox{DZBVLC}.

\bibitem{ResultsOnFDRUnderDependence} %9
\Au{Farcomeni~A.}
Some results on the control of the false discovery rate under dependence~// 
Scand. J. Stat., 2007. Vol.~34. No.\,2. P.~275--297.
doi: 10.1111/j.1467-9469.2006.00530.x.

\bibitem{VorontsovShestakov2023} %10
\Au{Воронцов~М.\,О., Шестаков~О.\,В.}
Среднеквадратичный риск FDR-про\-це\-ду\-ры в~условиях слабой за\-ви\-си\-мости~// 
Информатика и~её применения, 2023. Т.~17. Вып.~2. С.~34--40.
doi: 10.14357/19922264230205. EDN: AVJZDX.

\bibitem{Vorontsov2024} %11
\Au{Воронцов~М.\,О.}
Анализ среднеквадратичного риска при использовании методов множественной 
проверки гипотез для выбора параметров пороговой обработки в~условиях слабой 
зависимости~// Вестник Московского университета. Сер. 15: Вычислительная 
математика и~кибернетика, 2024. №\,2. С.~18--24.

\bibitem{Bosq} %12
\Au{Bosq~D.}
Nonparametric statistics for stochastic processes: Estimation and prediction.~--- 
Lecture notes in statistics ser.~--- New York, NY, USA: Springer, 1996. Vol.~110. 
188~p.

\bibitem{Mallat} %13
\Au{Mallat~S.}
A wavelet tour of signal processing.~--- New York, NY, USA: Academic Press, 1999. 
857~p.

\bibitem{spatialAdaptation} %14
\Au{Donoho~D., Johnstone~I.}
Ideal spatial adaptation via wavelet shrinkage~// Biometrika, 1994. Vol.~81. 
No.\,3. P.~425--455. doi: 10.1093/biomet/81.3.425.

\bibitem{AdaptingSURE} %15
\Au{Donoho D., Johnstone I.\,M.}
Adapting to unknown smoothness via wavelet shrinkage~// J.~Amer. Stat. Assoc., 
1995. Vol.~90. P.~1200--1224.

\bibitem{ExactRisk} %16
\Au{Marron J.\,S., Adak~S., Johnstone~I.\,M., Neumann~M.\,H., Patil~P.}
Exact risk analysis of wavelet regression~// J.~Comput. Graph. Stat., 1998. 
Vol.~7. P.~278--309. doi: 10.1080/ 10618600.1998.10474777.

\bibitem{Jansen} %17
\Au{Jansen~M.}
Noise reduction by wavelet thresholding.~-- Lecture notes in statistics ser.~--- 
New York, NY, USA: Springer, 2001. Vol.~161. 217~p.

\bibitem{KuShe2016_1} %18
\Au{Кудрявцев~А.\,А., Шестаков~О.\,В.}
Асимптотическое поведение порога, минимизирующего усредненную\linebreak вероятность ошибки 
вычисления вейв\-лет-ко\-эф\-фи\-ци\-ен\-тов~// Докл. Акад. наук, 2016. Т.~468. №\,5. 
С.~487--491.

\bibitem{KuShe2016_2} %19
\Au{Кудрявцев~А.\,А., Шестаков~О.\,В.}
Асимптотически оптимальная пороговая обработка вейв\-лет-ко\-эф\-фи\-ци\-ен\-тов в~моделях с~негауссовым распределением шума~// Докл. Акад. наук, 2016. Т.~471. №\,1. 
С.~11--15.



\bibitem{Eroshenko} %20
\Au{Ерошенко~А.\,А.}
Статистические свойства оценок сигналов и~изображений при пороговой обработке 
коэффициентов в~вейв\-лет-раз\-ло\-же\-ни\-ях: Дис.\ \ldots\ канд. физ.-мат. наук.~--- 
М.: МГУ, 2015. 82~с.

\bibitem{Peligrad} %21
\Au{Peligrad~M.}
On the asymptotic normality of sequences of weak dependent random variables~// 
J. Theor. Probab., 1996. Vol.~9. No.\,3. P.~703--715. doi: 10.1007/BF02214083.

\bibitem{Serfling2002} %22
\Au{Serfling~R.\,J.}
Approximation theorems of mathematical statistics.~--- New York, NY, USA: John Wiley \&~Sons, Inc., 2002. 371~p.

\end{thebibliography}

 }
 }

\end{multicols}

\vspace*{-6pt}

\hfill{\small\textit{Поступила в~редакцию 21.05.24}}

\vspace*{8pt}

%\pagebreak

%\newpage

%\vspace*{-28pt}

\hrule

\vspace*{2pt}

\hrule



\def\tit{ASYMPTOTIC NORMALITY AND STRONG CONSISTENCY\\ OF~RISK ESTIMATE WHEN USING THE~FDR THRESHOLD\\ UNDER WEAK DEPENDENCE CONDITION}


\def\titkol{Asymptotic normality and strong consistency of~risk estimate when using the~FDR threshold under weak dependence condition}


\def\aut{M.\,O.~Vorontsov$^{1,2}$ and~O.\,V.~Shestakov$^{1,2,3}$}

\def\autkol{M.\,O.~Vorontsov and~O.\,V.~Shestakov}

\titel{\tit}{\aut}{\autkol}{\titkol}

\vspace*{-13pt}


\noindent
$^{1}$Department of Mathematical Statistics, Faculty of Computational Mathematics and Cybernetics,
 M.\,V.~Lo\-mo-\linebreak
 $\hphantom{^1}$nosov Moscow State University, 1-52~Leninskie Gory, GSP-1, Moscow 119991, Russian Federation

\noindent
$^{2}$Moscow Center for Fundamental and Applied Mathematics, M.\,V.~Lomonosov Moscow State University,\linebreak
$\hphantom{^1}$1~Leninskie Gory, GSP-1, Moscow 119991, Russian Federation

\noindent
$^{3}$Federal Research Center ``Computer Science and Control'' of the Russian Academy of Sciences, 44-2~Vavilov\linebreak
$\hphantom{^1}$Str., Moscow 119333, Russian Federation


\def\leftfootline{\small{\textbf{\thepage}
\hfill INFORMATIKA I EE PRIMENENIYA~--- INFORMATICS AND
APPLICATIONS\ \ \ 2024\ \ \ volume~18\ \ \ issue\ 3}
}%
 \def\rightfootline{\small{INFORMATIKA I EE PRIMENENIYA~---
INFORMATICS AND APPLICATIONS\ \ \ 2024\ \ \ volume~18\ \ \ issue\ 3
\hfill \textbf{\thepage}}}

\vspace*{2pt}






\Abste{An approach to solving the problem of noise removal in a large array of sparse data is considered
 based on the method of controlling the average proportion of false hypothesis rejections (False Discovery Rate, FDR). 
 This approach is equivalent to threshold processing procedures that remove array components whose values do not exceed 
 some specified threshold. The observations in the model are considered weakly dependent. To control the\linebreak\vspace*{-12pt}}
 
 \Abstend{degree of dependence, 
 restrictions on the strong mixing coefficient and the maximum correlation coefficient are used. The mean-square risk is 
 used as a measure of the effectiveness of the considered approach. It is possible to calculate the risk value only on the test data;
  therefore, its statistical estimate is considered in the work and its properties are investigated. The asymptotic normality and
   strong consistency of the risk estimate are proved when using the FDR threshold under conditions of weak dependence in the data.}

\KWE{thresholding; multiple hypothesis testing; risk estimate}

\DOI{10.14357/19922264240309}{ZOQVTO}

%\vspace*{-12pt}


    
   %   \Ack

%\vspace*{-3pt}
%\noindent



  \begin{multicols}{2}

\renewcommand{\bibname}{\protect\rmfamily References}
%\renewcommand{\bibname}{\large\protect\rm References}

{\small\frenchspacing
 {\baselineskip=10.8pt
 \addcontentsline{toc}{section}{References}
 \begin{thebibliography}{99} 

%1
\bibitem{FDRImage-1}
\Aue{Krylov, V.\,A., G.~Moser, S.\,B.~Serpico, and J.~Zerubia.} 2016. 
False discovery rate approach to unsupervised image change detection. 
\textit{IEEE T. Image Process.} 25(10):4704--4718. doi: 10.1109/TIP.2016.2593340.

%2
\bibitem{MultipleTesting-1}
\Aue{Menyhart, O., B.~Weltz, and B.~Gyorffy.} 2021. 
MultipleTesting.com: A~tool for life science researchers for multiple hypothesis testing correction. 
\textit{PLoS One} 16(6):0245824. 
doi: 10.1371/journal.pone.0245824.

%3
\bibitem{AdaptingFDR-1}
\Aue{Abramovich, F., Y.~Benjamini, D.~Donoho, and I.\,M.~Johnstone.} 2006. 
Adapting to unknown sparsity by controlling the false discovery rate. 
\textit{Ann. Stat.} 34(2):584--653. 
doi: 10.1214/009053606000000074.


%4
\bibitem{ZasShe17-1}
\Aue{Zaspa, A.\,Yu., and O.\,V.~Shestakov.} 2017.
Sostoyatel'nost' otsenki riska pri mnozhestvennoy proverke gipotez s~FDR-porogom
 [Consistency of the risk estimate of the multiple hypothesis testing with the FDR threshold]. 
\textit{Vestnik TvGU. Ser.: Prikladnaya matematika} [Herald of Tver State University. Ser. Applied Mathematics] 1:5--16.
doi: 10.26456/vtpmk119. EDN: YFYJXT.

%5
\bibitem{Mathematics2020-1}
\Aue{Palionnaya, S.\,I., and O.\,V.~Shestakov.} 2020. 
Asymptotic properties of MSE estimate for the false discovery rate controlling procedures in multiple hypothesis testing. 
\textit{Mathematics} 8(11):1913. 11~p.
doi: 10.3390/math8111913.

%6
\bibitem{Shestakov2021-1-1}
\Aue{Shestakov, O.\,V.} 2021.
Analiz nesmeshchennoy otsenki srednekvadratichnogo riska metoda blochnoy po\-ro\-go\-voy obrabotki 
[Analysis of the unbiased mean-square risk estimate of the block thresholding method]. 
\textit{Informatika i~ee Primeneniya~--- Inform. Appl.} 15(2):30--35.
doi: 10.14357/19922264210205. EDN: DSQQAU.

%7
\bibitem{Shestakov2021-2-1}
\Aue{Shestakov, O.\,V.} 2021.
Porogovye funktsii v~metodakh podavleniya shuma, osnovannykh na veyvlet-razlozhenii signala 
[Thresholding functions in the noise suppression methods based on the wavelet expansion of the signal]. 
\textit{Informatika i~ee Primeneniya~--- Inform. Appl.} 15(3):51--56.
doi: 10.14357/19922264210307. EDN: WSEAYG.

%8
\bibitem{Shestakov2022-1}
\Aue{Shestakov, O.\,V.} 2022.
Nesmeshchennaya otsenka riska porogovoy obrabotki s dvumya porogovymi znacheniyami [Unbiased thresholding risk estimate with two threshold values]. 
\textit{Informatika i~ee Primeneniya~--- Inform. Appl.} 16(4):14--19.
doi: 10.14357/19922264220403. EDN: DZBVLC.

%9
\bibitem{ResultsOnFDRUnderDependence-1}
\Aue{Farcomeni, A.} 2007. Some results on the control of the false discovery rate under dependence. 
\textit{Scand. J. Stat.} 34(2):275--297. 
doi: 10.1111/j.1467-9469.2006.00530.x.

%10
\bibitem{VorontsovShestakov2023-1}
\Aue{Vorontsov, M.\,O., and O.\,V.~Shestakov.} 2023.
Sred\-ne\-kvad\-ra\-tich\-nyy risk FDR-protsedury v~usloviyakh slaboy za\-vi\-si\-mosti [Mean-square risk of the FDR procedure under weak dependence]. 
\textit{Informatika i~ee Primeneniya~--- Inform. Appl.} 17(2):34--40.
doi: 10.14357/19922264230205. EDN: AVJZDX.

%11
\bibitem{Vorontsov2024-1}
\Aue{Vorontsov, M.\,O.} 2024. 
RMS risk analysis when using multiple hypothesis testing select parameters of thresholding under conditions of weak dependence. 
\textit{Moscow University Computational Mathematics Cybernetics} 48:91--97. 
doi: 10.3103/S027864192470002X.

%12
\bibitem{Bosq-1}
\Aue{Bosq, D.} 1996. 
\textit{Nonparametric statistics for stochastic processes: Estimation and prediction}. 
Lecture notes in statistics ser. New York, NY: Springer Verlag. Vol.~110. 188~p.

%13
\bibitem{Mallat-1}
\Aue{Mallat, S.} 1999. 
\textit{A wavelet tour of signal processing}. New York, NY: Academic Press. 857~p.

%14
\bibitem{spatialAdaptation-1}
\Aue{Donoho, D., and I.\,M.~Johnstone.} 1994. 
Ideal spatial adaptation via wavelet shrinkage. 
\textit{Biometrika} 81(3):425--455. doi: 10.1093/biomet/81.3.425.

%15
\bibitem{AdaptingSURE-1}
\Aue{Donoho, D., and I.\,M.~Johnstone.} 1995. 
Adapting to unknown smoothness via wavelet shrinkage. 
\textit{J. Am. Stat. Assoc.} 90(432):1200--1224. doi: 10.1080/01621459. 1995.10476626.

%16
\bibitem{ExactRisk-1}
\Aue{Marron, J.\,S., S.~Adak, I.\,M.~Johnstone, M.\,H.~Neumann, and P.~Patil.} 1998. 
Exact risk analysis of wavelet regression. 
\textit{J.~Comput. Graph. Stat.} 7(3):278-309. doi: 10.1080/ 10618600.1998.10474777.

%17
\bibitem{Jansen-1}
\Aue{Jansen, M.} 2001. 
\textit{Noise reduction by wavelet thresholding}. Lecture notes in statistics ser. New York, NY: Springer Verlag. Vol.~161. 217~p.

%18
\bibitem{KuShe2016_1-1}
\Aue{Kudryavtsev, A.\,A., and O.\,V.~Shestakov.} 2016. 
Asymptotic behavior of the threshold minimizing the average probability of error in calculation of wavelet coefficients. 
\textit{Dokl. Math.} 93(3):295--299.
doi: 10.1134/S1064562416030212. EDN: WUMUEV. 

%19
\bibitem{KuShe2016_2-1}
\Aue{Kudryavtsev, A.\,A., and O.\,V.~Shestakov.} 2016. 
Asymptotically optimal wavelet thresholding in the models with non-Gaussian noise distributions. 
\textit{Dokl. Math.} 94(3):615--619.
doi: 10.1134/S1064562416060028. EDN: YUYVUP.




%20
\bibitem{Eroshenko-1}
\Aue{Eroshenko, A.\,A.} 2015. Statisticheskie svoystva otsenok signalov i~izobrazheniy pri porogovoy obrabotke ko\-ef\-fi\-tsi\-en\-tov 
v~veyvlet-razlozheniyakh 
[Statistical properties of signal and image estimates under thresholding of coefficients in wavelet decompositions]. Moscow: MSU. PhD Diss. 82~p.

%21
\bibitem{Peligrad-1}
\Aue{Peligrad, M.} 1996. 
On the asymptotic normality of sequences of weak dependent random variables. 
\textit{J. Theor. Probab.} 9(3):703--715. doi: 10.1007/BF02214083.

%22
\bibitem{Serfling2002-1}
\Aue{Serfling, R.\,J.} 2002. 
\textit{Approximation theorems of mathematical statistics}. New York, NY: John Wiley \&~Sons. 371~p.
\end{thebibliography}

 }
 }

\end{multicols}

\vspace*{-6pt}

\hfill{\small\textit{Received May 21, 2024}} 

%\vspace*{-18pt}

\Contr

\vspace*{-3pt}


\noindent
\textbf{Vorontsov Mikhail O.} (b.\ 1996)~--- PhD student, Department of Mathematical Statistics, 
Faculty of Computational Mathematics and Cybernetics, M.\,V.~Lomonosov Moscow State University, 1-52~Leninskie Gory, GSP-1, Moscow 119991, Russian Federation;  
mathematician, Moscow Center for Fundamental and Applied Mathematics, M.\,V.~Lomonosov Moscow State University, 1~Leninskie Gory, GSP-1, Moscow 119991, Russian Federation;
\mbox{m.vtsov@mail.ru}

\vspace*{6pt}

\noindent
\textbf{Shestakov Oleg V.} (b.\ 1976)~--- Doctor of Science in physics and mathematics, professor, Department of Mathematical Statistics,
 Faculty of Computational Mathematics and Cybernetics, M.\,V.~Lomonosov Moscow State University, 1-52~Leninskie Gory, GSP-1, Moscow 119991, Russian Federation; 
 senior scientist, Federal Research Center ``Computer Science and Control'' of the Russian Academy of Sciences, 44-2~Vavilov Str., Moscow 119333, 
 Russian Federation; leading scientist, Moscow Center for Fundamental and Applied Mathematics, M.\,V.~Lomonosov Moscow State University, 
 1~Leninskie Gory, GSP-1, Moscow 119991, Russian Federation; \mbox{oshestakov@cs.msu.su}


\label{end\stat}

\renewcommand{\bibname}{\protect\rm Литература}  %2 
\def\stat{kudr}

\def\tit{ПРИБЛИЖЕННЫЕ МЕТОДЫ РЕШЕНИЯ ЗАДАЧИ ДИАГНОСТИКИ ПЛОСКИМ 
ЗОНДОМ СИЛЬНОИОНИЗОВАННОЙ ПЛАЗМЫ С~УЧЕТОМ КУЛОНОВСКИХ 
СТОЛКНОВЕНИЙ}

\def\titkol{Приближенные методы решения задачи диагностики плоским 
зондом сильноионизованной плазмы} %с~учетом Кулоновских  столкновений}

\def\autkol{И.\,А.~Кудрявцева, А.\,В.~Пантелеев}
\def\aut{И.\,А.~Кудрявцева$^1$, А.\,В.~Пантелеев$^2$}

\titel{\tit}{\aut}{\autkol}{\titkol}

%{\renewcommand{\thefootnote}{\fnsymbol{footnote}}\footnotetext[1]
%{Работа поддержана Российским фондом фундаментальных исследований
%(проекты 11-01-00515а и 11-07-00112а), а также Министерством
%образования и науки РФ в рамках ФЦП <<Научные и
%научно-педагогические кадры инновационной России на 2009--2013~годы>>.}}


\renewcommand{\thefootnote}{\arabic{footnote}}
\footnotetext[1]{Московский авиационный институт, irina.home.mail@mail.ru}
\footnotetext[2]{Московский авиационный институт, avpanteleev@inbox.ru}

\vspace*{-2pt}

\Abst{Сформирована математическая модель, описывающая динамику сильноионизованной 
плазмы с учетом столкновений заряженных частиц вблизи плоского зонда. Модель включает уравнение 
Фоккера--Планка и уравнение Пуассона. Предложено два подхода к решению задачи: на основе метода 
статистических испытаний Мон\-те-Кар\-ло и на основе композиции метода крупных частиц и метода 
расщепления.} 

\vspace*{-2pt}

\KW{телекоммуникационные системы; метод Монте-Карло; метод крупных частиц; метод 
расщепления; зонд; уравнение Фоккера--Планка; уравнение Пуассона} 

\vspace*{-4pt}

 \vskip 8pt plus 9pt minus 6pt

      \thispagestyle{headings}

      \begin{multicols}{2}
      
            \label{st\stat}

\section{Введение}

В настоящее время в области телекоммуникаций все более востребованными становятся 
информационные технологии, основанные на использовании математических моделей и численных 
методов физики плазмы. Поэтому особенно актуальным является решение разнообразных задач анализа 
поведения плазмы, включающих в себя формирование новых моделей и методов их исследования. 
Помимо этого, в разработке телекоммуникационного оборудования эффективно используются 
собственно физические свойства плазмы. В~частности, изготовлена антенна, работа которой основана 
на газовом разряде низкотемпературной плазмы~[1], интенсивно ведутся разработки по созданию и 
усовершенствованию источников бесперебойного питания на основе плазменных элементов~[2, 3]. 
      
      Одним из наиболее перспективных направлений для построения систем оптической 
беспроводной связи является использование лазеров~\cite{4-k, 5-k}. В~этой связи большое внимание 
уделяется использованию плазмы при разработке импульсных сильноточных коммутаторов~\cite{6-k}, 
так как практическое применение подобных разработок требует повышения уровня надежности и 
быстродействия лазерных систем.
      
      Исследования низкотемпературной плазмы также связаны с разработками в области дальней 
космической связи, так как моделирование процессов взаимодействия заряженного тела с верхними 
слоями атмосферы позволяет предлагать способы улучшения существующих систем радиосвязи с 
космическими летательными аппаратами~\cite{7-k}. 
      
      Наряду с этим актуальными также являются задачи диагностики плазмы, поскольку перспективы 
ее использования в области телекоммуникаций после более полного изучения физических свойств 
могут значительно расшириться. 

Для диагностики плазмы применяют зондовые методы исследования~[8--11]. Эти методы относятся к 
классу контактных методов; как следствие, возникает сложность в исследовании пристеночной области 
вблизи зонда, которая характеризуется достаточно сложным распределением потенциала и функциями 
распределения, отличными от максвелловских. 

Данная работа посвящена исследованию переходного режима обтекания заряженного тела плазмой. Для 
переходного режима выполняется следующее условие: длина свободного пробега иона до столкновения 
с нейтральным атомом или другим ионом невелика по сравнению с характерными размерами тела. 
В~этом случае возникает необходимость учета столкновений заряженных частиц с нейтральными 
атомами и кулоновских столкновений. В~работах~\cite{10-k, 11-k} подробно рассмотрена модель с 
учетом столкновений заряженных частиц с нейтральными атомами. В~настоящей статье представлена 
теоретическая модель, описывающая влияния ион-ионных и ион-элек\-т\-рон\-ных столкновений на 
измеряемые характеристики плазмы, что ранее детально не исследовалось.
      
      В~рамках данной работы предлагается модель, описывающая динамику сильноионизованной 
плазмы с учетом кулоновских столкновений. Эта модель учитывает такие процессы взаимодействия, 
как перенос частиц и столкновения между заряженными частицами типа <<ион--ион>> и 
      <<ион--электрон>> под влиянием макроскопического электрического поля. Перечисленные 
процессы описываются самосогласованной системой уравнений, включающей уравнение 
      Фок\-ке\-ра--План\-ка и уравнение Пуассона~[12].
      
      Вычислительная модель задачи строится на основе двух методов: метода статистических 
испытаний Мон\-те-Кар\-ло и композиции метода крупных частиц и метода расщепления. Приведены 
результаты численного моделирования, полученные с использованием вышеперечисленных методов.

\vspace*{-4pt}

\section{Постановка задачи}

\vspace*{-2pt}

Рассматривается следующая физическая постановка зондовой задачи~[11]. В~невозмущенную 
бесконечно протяженную плазму, состоящую из электронов и однозарядных ионов, внесена большая\linebreak 
заряженная до потенциала $\varphi_p$ плоскость. Плоскость, расположенная поперек потока плазмы, 
является идеально поглощающей для электронов. Ионы при ударе о плоскость нейтрализуются. 
Предполагается, что частицы в плазме движутся под действием внешнего электрического поля, 
магнитное поле отсутствует. Концентрации ионов $n_{i\infty}$ и электронов $n_{e\infty}$, а также 
температуры данных час\-тиц~$T_{i\infty}$ 
и~$T_{e\infty}$ в невозмущенной плазме заданы. За начальные 
функции распределения обоих типов час\-тиц принимаются функции распределения Максвелла. 
      
      Требуется с учетом столкновений между заряженными частицами найти напряженность 
самосогласованного электрического поля $\vec{E}(\vec{r},t)$, функции распределения однозарядных 
ионов $f_i(\vec{r}, \vec{v}, t)$ и электронов $f_e(\vec{r}, \vec{v}, t)$, 
а также их моменты (плотности 
токов ионов и электронов  $j_i(\vec{r},t)\hm
=q\int f_i(\vec{r}, \vec{v}, t)\vec{v}\,d\vec{v}$, $j_e(\vec{r},t) 
\hm={\sf e}\int f_e(\vec{r},\vec{v},t)\vec{v}\,d\vec{v}$, где $q=Z_i{\sf e}$, $Z_i=1$~--- заряд иона, ${\sf 
e}$~--- заряд электрона; концентрации ионов и электронов $n_i(\vec{r},t)\hm=\int 
f_i(\vec{r},\vec{v},t)\,d\vec{v}$, $n_e(\vec{r},t)\hm=\int f_e(\vec{r},\vec{v}, t)\,d\vec{v}$). 
Поведение частиц во 
времени~$t$ характеризуется ра\-ди\-ус-век\-то\-ром~$\vec{r}$ и вектором скорости~$\vec{v}$.
      
      Математическая модель, соответствующая данной физической постановке задачи, имеет 
вид~\cite{11-k, 13-k}:

\noindent
      \begin{equation}
      \left.
      \begin{array}{c}
      \fr{\partial f_\alpha (\vec{r},\vec{v},t)}{\partial t}+
      \vec{v}\fr{\partial f_\alpha (\vec{r},\vec{v},t)}{ 
\partial \vec{r}}+
\fr{\vec{F}_\alpha(\vec{r},t)}{m_\alpha}\times{}\\[4pt]
{}\times\fr{\partial f_\alpha(\vec{r},\vec{v},t)}{ \partial 
\vec{v}}=
\left(\fr{\partial f_\alpha(\vec{r},\vec{v},t)}{ \partial t}\right)_{\mathrm{с}}+S_\alpha 
(\vec{r},\vec{v},t)\,;\\[6pt]
      \Delta\varphi(\vec{r},t)=-\fr{{\sf e}}{\varepsilon_0}\left( n_i(\vec{r},t)-n_e(\vec{r},t)\right)\,;\\[6pt]
      \vec{E}(\vec{r},t)=-\nabla \varphi(\vec{r},t)\,.
      \end{array}\!\!
      \right\}\!\!
      \label{e1-k}
      \end{equation}
Здесь первое уравнение~--- уравнение Фок\-ке\-ра--План\-ка для частиц сорта~$\alpha$ ($\alpha=i,e$), 
второе~--- уравнение Пуассона для самосогласованного электрического поля; 
$f_\alpha(\vec{r},\vec{v},t)$~--- функция\linebreak
распределения час\-тиц сорта~$\alpha$; $(\partial 
f_\alpha(\vec{r},\vec{v},t)/\partial t)_{\mathrm{с}}$~--- 
оператор столкновений Фок\-ке\-ра--План\-ка; 
функция~$S_\alpha(\vec{r},\vec{v},t)$ описывает источники или стоки\linebreak
 час\-тиц; 
$\vec{F}_\alpha(\vec{r},t)=q_\alpha\vec{E}(\vec{r},t)$, где $\vec{E}(\vec{r},t)$~--- напряженность 
самосогласованного электрического поля, 
$$
q_\alpha =
\begin{cases}
-{\sf e}\,, & \alpha=e\,,\\
{\sf e}\,, & \alpha=i\,;
\end{cases}
$$
$\varphi(\vec{r},t)$~--- потенциал самосогласованного электрического поля; $n_\alpha(\vec{r},t)$ ($\alpha 
\hm=i,e$)~--- концентрация частиц сорта~$\alpha$; $m_\alpha$~--- масса частицы сорта~$\alpha$; 
$\varepsilon_0$~--- электрическая постоянная. 

Оператор столкновений Фок\-ке\-ра--План\-ка имеет вид~\cite{13-k, 14-k}
\begin{multline*}
\fr{1}{\Gamma_\alpha}\left( \fr{\partial f_\alpha}{\partial t}\right)_{\mathrm{с}} 
=\fr{1}{2}\,\nabla_v\nabla_v:\left(f_\alpha\nabla_v\nabla_vg_\alpha(\vec{r},\vec{v},t)\right)-{}\\
{}-
\nabla_v\cdot\left(f_\alpha\nabla_v h_\alpha\right)\,,
\end{multline*}
где $\nabla_v\nabla_v g_\alpha(\vec{r},\vec{v},t)$~--- ковариантная тензорная производная второго ранга, 
знак двоеточия ($:$) обозначает операцию двойного суммирования:
\begin{gather*}
\Gamma_\alpha=\fr{Z_\alpha^4 {\sf e}^4}{4\pi \varepsilon_0^2 m^2_\alpha}\,\ln D_\alpha\,;
\\
D_\alpha =\fr{12\pi\varepsilon_0 kT_{\alpha\infty}}{Z_\alpha^2 {\sf e}^2}\left( \fr{\varepsilon_0 k 
T_{e\infty}}{n_{e\infty} {\sf e}^2}\right)^{1/2}\,;\\
g_\alpha (\vec{r},\vec{v},t)=\sum\limits_{b=i,e}\left( \fr{Z_b}{Z_\alpha}\right) \int f_b 
(\vec{r},{\vec{v}}^{\,\prime},t)\left\vert \vec{v}-{\vec{v}}^{\,\prime}\right\vert\,d\vec{v}^{\,\prime}\,;\\
h_\alpha (\vec{r},\vec{v},t)=\sum\limits_{b=i,e} \fr{m_\alpha+m_b}{m_b} 
\left(\fr{Z_b}{Z_\alpha}\right)
\int
\fr{f_b(\vec{r},{\vec{v}}^{\,\prime}, t)}{\vert \vec{v}-{\vec{v}}^{\,\prime}\vert}
\,d{\vec{v}}^{\,\prime}\,;\\
Z_\alpha =1\,, \quad \alpha=i,e\,.
\end{gather*}
 
К системе уравнений~(\ref{e1-k}) необходимо добавить начальные и краевые условия:
\begin{equation}
\!\left.
\begin{array}{rrl}
t=0:\ & f_\alpha(\vec{r},\vec{v},0)&=f_\alpha^{\mathrm{maksv}}\,,\enskip \alpha=i,e;\\[9pt]
\vec{r}\in \Omega_p:\ & f_\alpha(\vec{r},\vec{v},t)\big\vert_{\vec{r}\in\Omega_p}&=0\,,\enskip \alpha=i,e\,;\\[9pt]
&\varphi(\vec{r},t)\big\vert_{\vec{r}\in\Omega_p}&=\varphi_p\,;\\[9pt]
\vec{r}\in\Omega_\infty:\ & 
f_\alpha(\vec{r},\vec{v},t)\big\vert_{\vec{r}\in\Omega_\infty}&= %{}\\[9pt]
f_\alpha^{\mathrm{maksv}}\,,\enskip \alpha=i,e\,;\\[9pt]
&\varphi(\vec{r},t)\big\vert_{\vec{r}\in\Omega_\infty}&=0\,,
\end{array}\!\!
\right\}\!\!\!\!
\label{e2-k}
\end{equation}
    где 
    
    \noindent
    \begin{multline*}
    f_\alpha^{\mathrm{maksv}}=n_{\alpha\infty}\left(\fr{m_\alpha}{2k\pi T_{\alpha\infty}}\right)^{3/2}\times{}\\
    {}\times
    \exp\left( -
\fr{m_\alpha}{2kT_{\alpha\infty}}\left\vert\vec{v}-\vec{v}_\infty\right\vert^2\right)\,,
\enskip \alpha=i, e\,;
\end{multline*} 
$\Omega_p$ и $\Omega_\infty$~--- множество радиус-векторов час\-тиц, концы которых принадлежат плоскости зонда и 
границе возмущенной зоны соответственно.

Для решения поставленной задачи введем декартову систему координат таким образом, чтобы 
заряженная плоскость совпала с плоскостью~$0xz$. Тогда положение частицы в пространстве будет 
определяться координатами $x,y,z$, а скорость~--- координатами $v_x, v_y, v_z$. В~силу того что 
плоскость является бесконечно большой в сравнении с характерным размером задачи, функции 
распределения частиц будут зависеть только от переменных $y, v_y, t$.

Поставленную задачу предлагается решать независимо двумя методами. Первый метод основывается на 
методе статистических испытаний Мон\-те-Кар\-ло, второй метод является композицией метода 
расщепления и метода крупных частиц.

\section{Применение метода Монте-Карло}

Запишем самосогласованную систему уравнений~(\ref{e1-k}) и~(\ref{e2-k}) в декартовой системе 
координат с учетом сделанных предположений:
\begin{equation}
\left.
\begin{array}{l}
\fr{\partial f_\alpha}{\partial t}+
v_y\fr{\partial f_\alpha}{\partial y}+\fr{F_y^\alpha}{m_\alpha}\,\fr{\partial 
f_\alpha}{\partial v_y}=\fr{1}{2}\,\fr{\partial^2 }{\partial [v_y]^2}\times{}\\
{}\times \left( 
f_\alpha\fr{\partial^2 g_\alpha  }{\partial [v_y]^2}\right) -
\fr{\partial}{\partial v_y}\left( f_\alpha\fr{\partial h_\alpha}{\partial v_y}\right)\,,
\enskip \alpha=i,e\,;\\[6pt]
    \fr{\partial^2\varphi}{\partial y^2} =-\fr{{\sf e}}{\varepsilon_0}\left(n_i-n_e\right)\,;
    \enskip E_y=-
\fr{\partial\varphi}{\partial y}\,;\\[6pt]
\hspace*{3.1mm}    t=0:\  \hspace*{2.6mm}f_\alpha(y,v_y,0)=f_\alpha^{\mathrm{maksv}}\,,\ \alpha=i,e\,;\\[9pt]
\hspace*{2.9mm} y=0:\ \hspace*{2.8mm}f_\alpha(0,v_y,t)=0\,,\ \alpha=i,e\,;\\[9pt]
\hspace*{24.3mm}\varphi(0,t)=\varphi_p\,;\\[9pt]
y=y_\infty:\ f_\alpha(y_\infty, v_y, t)=f_\alpha^{\mathrm{maksv}}\,,\ \alpha=i,e\,;\\[9pt]
\hspace*{21.5mm}\varphi(y_\infty, t)=0\,.
\end{array}
\right \}
\label{e3-k}
\end{equation}

В полученной системе уравнений~(\ref{e3-k}) перейдем к безразмерным величинам, применив 
соотношение $X=M_X \hat{X}$, где $M_X$~--- масштаб размерной величины~$X$, $\hat{X}$~--- 
безразмерная величина~$X$. В~качестве используемых масштабов были взяты следующие: радиус 
Дебая, скорость теплового движения частиц, концентрация частиц в невозмущенной плазме, потенциал, 
возникающий при разделении зарядов в дебаевской сфере, и производные от них величины.

Система безразмерных уравнений имеет следующий вид:
%\noindent
\begin{equation}
\left.
\begin{array}{l}
\fr{\partial 
\hat{f}_\alpha}{\partial\hat{t}}+A_\alpha\fr{\partial\hat{f}_\alpha}{\partial\hat{y}}+
B_\alpha\hat{E}_y\fr{\partial\hat{f}_\alpha}{\partial \hat{v}_y}={}\\
\!{}=
\fr{\partial^2}{\partial[\hat{v}_y]^2}\left(D_\alpha 
\hat{f}_\alpha\right)-\fr{\partial}{\partial\hat{v}_y}\left(K_\alpha \hat{f}_\alpha\right),\enskip 
\alpha=i,e;\\[9pt]
\fr{\partial^2\hat{\varphi}}{\partial\hat{y}^2}=-\left(\hat{n}_i-\hat{n}_e\right)\,;\enskip \hat{e}_y=-
\fr{\partial\hat\varphi}{\partial\hat{y}}\,;\\[9pt]
\hspace*{3.1mm}\hat{t}=0:\ \hspace*{2.6mm}\hat{f}_\alpha(\hat{y},\hat{v}_y,0)=\hat{f}_\alpha^{\mathrm{maksv}}\,,\enskip \alpha-i,e\,;\\[9pt]
\hspace*{2.9mm}\hat{y}=0:\ \hspace*{2.8mm}\hat{f}_\alpha(0,\hat{v}_y,\hat{t})=0\,,\enskip \alpha=i,e\,;\\[9pt]
\hspace*{24.3mm}\hat\varphi(0,\hat{t})=\hat{\varphi}_p\,;\\[9pt]
\hat{y}=\hat{y}_\infty:\ \hat{f}_\alpha(\hat{y}_\infty, \hat{v}_y, \hat{t})=\hat{f}^{\mathrm{maksv}}_\alpha\,,\enskip 
\alpha=i,e\,;\\[9pt]
\hspace*{21.5mm}\hat\varphi(\hat{y}_\infty,\hat{t})=0\,.
\end{array}
\right\}
\label{e4-k}
\end{equation}
Здесь 

\vspace*{-2pt}

\noindent
\begin{gather*}
A_\alpha=\sqrt{\delta_\alpha }\,\hat{v}_y\,;\enskip 
B_\alpha=\sqrt{\delta_\alpha}\,\fr{z_\alpha}{2\varepsilon_\alpha}\,;\\
\delta_\alpha=\fr{\varepsilon_\alpha}{\mu_\alpha}\,;\enskip 
\varepsilon_\alpha=\fr{T_{\alpha\infty}}{T_{i\infty}}\,;\\
\mu_\alpha=\fr{m_\alpha}{m_i}\,;\enskip 
D_\alpha=A_g^\alpha\fr{\partial^2\hat{g}_\alpha}{\partial  [\hat{v}_y]^2}\,;\\
K_\alpha=A_h^\alpha \fr{\partial \hat{h}_\alpha}{\partial \hat{v}_y}\,,\enskip \alpha=i,e\,,
\end{gather*}
где $A_g^\alpha$ и $A_h^\alpha$~--- коэффициенты, определяемые характерными параметрами 
задачи~\cite{15-k}.

Поиск решения самосогласованной системы уравнений~(\ref{e4-k}) осуществляется по следующей 
схе-\linebreak ме. Вначале находятся значения напряженности\linebreak
 электрического поля по значениям потенциала, 
полученным из граничной задачи для уравнения Пуассона. Далее, используя найденные значения 
напряженности, решается уравнение Фок\-ке\-ра--План\-ка путем перехода к стохастическому 
дифференциальному уравнению (СДУ) Ито:

\noindent
\begin{multline*}
d\Theta_\alpha(\hat{t}) = a_\alpha \left(\hat{t},\Theta_\alpha(\hat{t})\right)+{}\\
{}+\sigma\left(
\hat{t},\Theta_\alpha(\hat{t})\right)\,dW(\hat{t})\,,\quad \alpha=i,e\,,
%\label{e5-k}
\end{multline*}
где 

\noindent
\begin{align*}
\Theta_\alpha(\hat{t})&=\begin{bmatrix}
\hat{y}(\hat{t})\\ \hat{v}_y(\hat{t})
\end{bmatrix}\,;\\
a_\alpha\left(\hat{t},\Theta_\alpha(\hat{t})\right)&=\begin{bmatrix}
-A_\alpha\\ -K_\alpha -B_\alpha \hat{E}_y
\end{bmatrix}\,;\\
\sigma_\alpha\left(\hat{t},\Theta_\alpha(\hat{t})\right)\sigma_\alpha^{\mathrm{T}}\left( 
\hat{t},\Theta_\alpha(\hat{t})\right)&=D_\alpha\,,\enskip \alpha=i,e\,;
\end{align*} 
$W(\hat{t})$~--- стандартный винеровский случайный процесс.
\pagebreak

Для нахождения значений вектора состояния~$\Theta_\alpha(\hat{t})$ применим явную разностную 
схему стохастического метода Эйлера~\cite{16-k}:
\begin{multline*}
\Theta_\alpha^{n+1}=\Theta_\alpha^n +h_\tau a_\alpha \left( \hat{t}_n, \Theta_\alpha^n\right)+\sigma_\alpha 
\left( \hat{t}_n, \Theta_\alpha^n\right)\Delta W_n\,,\\ 
n=0,\ldots , N\,,\ \alpha=i,e\,,
%\label{e6-k}
\end{multline*}
где $\Theta_\alpha^n$, $n=0,\ldots , N$,~--- приближенное значение вектора 
состояния~$\Theta_\alpha(\hat{t})$, $\alpha=i,e$, в момент времени $\hat{t}\hm=\hat{t}_n$, 
$\hat{t}_n\hm=n h_\tau$, $n=0,\ldots , N$; $h_\tau$~--- достаточно малый шаг интегрирования; $\Delta 
W_n$, $n=0,\ldots ,N$,~--- величина приращения винеровского процесса~$W(\hat{t})$ на отрезке $\left[ 
\hat{t}_n,\,\hat{t}_{n+1}\right]$, по определению независимая от~$\Theta_\alpha^0$, 
$\Delta W_0,\ldots , 
\Delta W_{n-1}$: $\Delta W_n\hm=W(\hat{t}_{n-1})\hm-W(\hat{t}_n)$; $\Delta W_n\hm\sim N(0,\,h_\tau)$, 
т.\,е.\ $\Delta W_n$ представляют собой гауссовские случайные величины с нулевыми математическими 
ожиданиями и дисперсиями, равными шагу интегрирования; $\Theta_\alpha^0$~--- значение вектора 
состояния $\Theta_\alpha(\hat{t})$, $\alpha\hm=i,e$, в момент времени $\hat{t}=0$, 
$\Theta_\alpha^0\hm\sim \hat{f}_\alpha^{\mathrm{maksv}}$. 

Частные производные $\partial^2\hat{g}_\alpha/\partial[\hat{v}_y]^2$ и $\partial \hat{h}_\alpha/\partial 
\hat{v}_y$, являющиеся составляющими матрицы $\sigma_\alpha (\hat{t}_n, 
\Theta_\alpha^n)\sigma_\alpha^{\mathrm{T}}(\hat{t}_n,\Theta_\alpha^n)$ и вектора $a_\alpha(\hat{t}_n, 
\Theta_\alpha^n)$ соответственно, аппроксимируются со вторым порядком точности на трехточечном 
шаблоне на основе значений~$\hat{g}_\alpha$ и~$\hat{h}_\alpha$~\cite{17-k}.
      
      В выражения для функций~$\hat{g}_\alpha$ и~$\hat{h}_\alpha$ входят интегралы, которые 
вычисляются методом Мон\-те-Кар\-ло с использованием набора значений скоростной компоненты 
вектора состояния~$\hat{v}_y$, полученных из решения СДУ Ито:
      \begin{equation*}
      \int \hat{f}_\alpha \left\vert \hat{v}_y-
\hat{v}_y^\prime\right\vert\,dv_y^\prime=M\left(\zeta\left(\hat{V}_y\right)\right)\,,
\end{equation*}
где
$$
      \zeta\left(\hat{V}_y\right)=\left\vert \hat{v}_y-\hat{V}_y\right\vert\,,\enskip \hat{V}_y\sim 
\hat{f}_\alpha\,.
  $$
      
      Для вычисления напряженности самосогласованного электрического поля $\hat{E}_y=-
\partial\hat{\varphi}/\partial\hat{y}$, входящей в вектор $a_\alpha(\hat{t}_n, \Theta_\alpha^n)$, необходимо 
аналогично аппроксимировать со вторым порядком точности производную 
$\partial\hat{\varphi}/\partial\hat{y}$ на трехточечном шаблоне с использованием значений 
потенциала~$\hat{\varphi}$~\cite{17-k}. Значения потенциала~$\hat\varphi$ находятся из решения 
уравнения Пуассона. 
      
      Граничную задачу для уравнения Пуассона 
      \begin{align*}
      \fr{\partial^2 \hat\varphi}{\partial \hat{y}^2} & = -\left(\hat{n}_i-\hat{n}_e\right)\,;\\
      \hat{\varphi}\big|_{\hat{y}=0} &=\hat{\varphi}_p\,;\\
      \hat{\varphi}\big|_{\hat{y}_\infty=0} &=0
      \end{align*}
    предлагается решать путем перехода к конечно-разностной системе с последующим ее решением 
методом прогонки~\cite{17-k}:

\noindent
\begin{gather*}
\hat{\varphi}^n_{l-1}+2\hat{\varphi}_l^n+\hat{\varphi}^n_{l+1}=
h_y\hat{\delta}_l^n\,,\enskip l=1,\ldots , 
N_y\,;\\
\hat{\delta}_l^n=-\left( \hat{n}^n_{i,l}-\hat{n}^n_{e,l}\right)\,;\enskip 
\hat{\varphi}_0=\hat{\varphi}_p\,;\enskip \hat{\varphi}_{N_y}=0\,,
\end{gather*}
где $N_y$~--- число шагов по переменной~$\hat{y}$, $h_y$~--- величина шагов разбиения по~$\hat{y}$. 
      
      Концентрации $\hat{n}_\alpha$, $\alpha=i,e$, и плотности токов частиц на зонд~$\hat{f}_\alpha$, 
$\alpha=i,e$, вычисляются согласно описанному выше методу Мон\-те-Карло.

\section{Применение метода расщепления и~метода крупных~частиц}

Решение задачи в данном случае предлагается начать с записи правой части уравнения 
Фок\-ке\-ра--План\-ка в декартовой системе координат в виде:
$$
\mathbf{Q} f_\alpha = \fr{1}{2}\,\fr{\partial^2 f_\alpha}{\partial [v_y]^2}\,\fr{\partial^2 g_\alpha}{\partial 
[v_y]^2}+\fr{\partial f_\alpha}{\partial v_y}\,\fr{\partial C_\alpha}{\partial v_y}+H_\alpha\,,\enskip 
\alpha=i,e\,,
$$  
где 
\begin{align*}
C_\alpha(\vec{r},\vec{v},t)&=
\begin{cases}
\fr{1-\gamma}{Z_i^2}\int\fr{f_e(\vec{r},{\vec{v}}^{\,\prime},t)}{|\vec{v}-{\vec{v}}^{\,\prime} |}\,d{\vec{v}}^{\,\prime}\,, 
&\alpha=i\,;\\[9pt]
\fr{Z_i^2(\gamma-1)}{\gamma}\int \fr{f_i(\vec{r},{\vec{v}}^{\,\prime}, t)}
{|\vec{v}-{\vec{v}}^{\,\prime} 
|}\,d{\vec{v}}^{\,\prime}\,, &\alpha=e\,;
\end{cases} 
\\
H_\alpha&=
\begin{cases}
4\pi \left( \fr{\gamma f_e}{Z_i^2}+f_i\right)f_i\,, & \alpha=i\,;\\[9pt]
4\pi\left(\fr{Z_i^2 f_i}{\gamma}+f_e\right)f_e\,, &\alpha=e\,.
\end{cases}
\end{align*}
Тогда при переходе к безразмерным величинам (см.\ разд.~3) система~(\ref{e1-k}) запишется 
следующим образом:
      \begin{equation}
      \left.
\!\!\begin{array}{l}
      \fr{\partial 
\hat{f}_\alpha}{\partial\hat{t}}+A_\alpha\fr{\partial\hat{f}_\alpha}{\partial\hat{y}}+
B_\alpha  \hat{E}_y
\fr{\partial\hat{f}_\alpha}{\partial\hat{v}_\alpha}=\tilde{\mathbf{Q}}\hat{f}_\alpha\,,\enskip 
\alpha=i,e;\\[9pt]
      \fr{\partial^2\hat{\varphi}}{\partial\hat{y}^2}=-\left( \hat{n}_i-\hat{n}_e\right)\,,\enskip \hat{E}_y=-
\fr{\partial\hat\varphi}{\partial\hat{y}}\,,\\[9pt]
\hspace*{3.1mm}\hat{t}=0:\ \hspace*{2.6mm}\hat{f}_\alpha(\hat{y},\hat{v}_y, 0)=\hat{f}_\alpha^{\mathrm{maksv}}\,,\enskip \alpha=i,e\,,\\[9pt]
\hspace*{2.9mm} \hat{y}=0:\ \hspace*{2.8mm}\hat{f}_\alpha(0,\hat{v}_y,\hat{t})=0\,,\enskip \alpha=i,e\,;\\[9pt]
\hspace*{24.3mm}\hat\varphi(0,\hat{t})=\hat{\varphi}_p\,;\\[9pt]
      \hat{y}=\hat{y}_\infty:\ \hat{f}_\alpha(\hat{y}_\infty, 
\hat{v}_y,\hat{t})=\hat{f}_\alpha^{\mathrm{maksv}}\,,\enskip \alpha=i,e\,;\\[9pt]
\hspace*{21.5mm}\hat{\varphi}(\hat{y}_\infty,\hat{t})=0\,,\\[9pt]
    \end{array}
\right\}\!\!
\label{e7-k}
\end{equation}
где 
\begin{gather*}
\tilde{\mathbf{Q}} \hat{f}_\alpha=D_\alpha\fr{\partial^2\hat{f}_\alpha}{\partial 
[\hat{v}_y]^2}+K_\alpha\fr{\partial\hat{f}_\alpha}{\partial\hat{v}_y}+H_\alpha\,;\\
D_\alpha=A_g^\alpha\fr{\partial^2\hat{g}_\alpha}{\partial [\hat{v}_y]^2}\,;\enskip 
K_\alpha=A_h^\alpha \fr{\partial \hat{h}_\alpha}{\partial\hat{v}_y}\,,\ \alpha=i,e\,.
\end{gather*}

Для решения системы уравнений~(\ref{e7-k}) применяется модификация метода 
расщепления~\cite{17-k}, согласно которой исходная задача разбивается на две вспомогательные. Такое 
разбиение можно осуществить, переписав уравнение Фок\-ке\-ра--План\-ка в следующем виде:
$$
\fr{\partial\hat{f}_\alpha}{\partial\hat{t}} =
\tilde{\mathbf{Q}}_1\hat{f}_\alpha+\tilde{\mathbf{Q}}_2\hat{f}_\alpha\,,
$$
где 
\begin{align*}
\tilde{\mathbf{Q}}_1\hat{f}_\alpha &=-
\left(A_\alpha\fr{\partial\hat{f}_\alpha}{\partial\hat{y}}+
B_\alpha\fr{\partial\hat{f}_\alpha}{\partial\hat{y}}
\right)\,;\\
\tilde{\mathbf{Q}}_2\hat{f}_\alpha 
&=\left(D_\alpha\fr{\partial^2\hat{f}_\alpha}{\partial[\hat{v}_y]^2}+K_\alpha\fr{\partial 
\hat{f}_\alpha}{\partial\hat{v}_y}+H_\alpha\right)\,.
\end{align*}

      Правая часть уравнения Фок\-ке\-ра--План\-ка представляет собой сумму двух операторов, 
первый из которых отвечает за перенос частиц, второй~--- за столкновения заряженных частиц. 
В~результате образуются следующие задачи, которые решаются последовательно:
      \begin{itemize}
\item первая задача:
\begin{align*}
&\fr{\partial w_\alpha(\hat{y},\hat{v}_y,\hat{t})}{\partial\hat{t}} =\mathbf{Q}_1 
w_\alpha(\hat{y},\hat{v}_y,\hat{t})\,,\enskip \alpha=i,e\,;\\[9pt]
&\fr{\partial^2\hat\varphi}{\partial\hat{y}^2}=-\left(\hat{n}_i-\hat{n}_e\right)\,;\enskip
\hat{E}_y=-
\fr{\partial\hat\varphi}{\partial\hat{y}}\,;\\[9pt]
&w_\alpha(\hat{y},\hat{v}_y,\hat{t}^n)=\hat{f}_\alpha(\hat{y},\hat{v}_y,\hat{t}^n)\,,\enskip n=0,\ldots ,N-
1\,;\\[9pt]
&\hspace{2.9mm}\hat{y}=0:\ \hspace*{2.9mm}w_\alpha(0,\hat{v}_y,\hat{t})=0\,,\enskip \alpha=i,e\,;\\[9pt]
&\hspace*{25.1mm}\hat\varphi(0,\hat{t})=\hat{\varphi}_p\,;\\[9pt]
&\hat{y}=\hat{y}_\infty:\ w_\alpha(\hat{y}_\infty, \hat{v}_y, \hat{t})=
\hat{f}_\alpha^{\mathrm{maksv}}\,,\enskip 
\alpha=i,e\,;\\[9pt]
&\hspace*{22.5mm}\hat\varphi(\hat{y}_\infty,\hat{t})=0\,;
\end{align*}
\item вторая задача:
\begin{align*}
\!\!\!\!\!\!\!\fr{\partial s_\alpha(\hat{y},\hat{v}_y,\hat{t})}{\partial \hat{t}} &=\mathbf{Q}_2 
s_\alpha(\hat{y},\hat{v}_y,\hat{t})\,, & \alpha&=i,e\,;\\
\!\!\!\!\!\!\!s_\alpha (\hat{y},\hat{v}_y,\hat{t}^n) &=w_\alpha (\hat{y},\hat{v}_y, \hat{t}^{n+1}),& n&=0,\ldots ,N-
1.
\end{align*}
\end{itemize}

Первая задача представляет собой систему безразмерных уравнений Вла\-со\-ва--Пуас\-со\-на. Для ее 
решения применяется метод крупных частиц~\cite{18-k}. Согласно этому методу решение задачи 
осуществляется путем расщепления на два этапа: на первом этапе не учитываются конвективные члены 
и решение получается обычным интегрированием на неподвижной эйлеровой сетке, а на втором этапе 
рассматривается система, которая описывает перенос частиц в лагранжевой системе координат. Кроме 
того, на первом этапе необходимо решить уравнение Пуассона для получения значений потенциала 
самосогласованного электрического поля. Для этого применяется метод, описанный в разд.~3. 

Вторая задача решается путем перехода к ко\-неч\-но-раз\-ност\-ной сис\-те\-ме. При этом частные 
производные $\partial^2\hat{g}_\alpha/\partial[\hat{v}_y]^2$ и $\partial\hat{h}_\alpha/\partial\hat{v}_y$ 
аппроксимируются со вторым порядком точности с использованием трехточечного шаблона, а 
производная $\partial s_\alpha/\partial\hat{t}$ аппроксимируется на двухточечном шаблоне с первым 
порядком точности~\cite{16-k}. К~полученной системе разностных уравнений предлагается применить 
один из классических методов решения систем линейных уравнений, например метод 
Гаусса~\cite{19-k}.
      
      Решением первой задачи является функция $w_\alpha(\hat{y}, \hat{v}_y, \hat{t}^n)$, 
$n\hm=0,\ldots ,N$, , которая дает начальное условие для второй задачи. Решая вторую задачу, находим 
функцию $s_\alpha(\hat{y},\hat{v}_y,\hat{t}^n)\hm=\hat{f}_\alpha(\hat{y},\hat{v}_y,\hat{t}^n)$, 
$n=1,\ldots ,N$, $\alpha=i,e$, которая определяет решение $\hat{f}_\alpha(\hat{y},\hat{v}_y,\hat{t}^n)$, 
$\alpha=i,e$, исходной системы~(\ref{e7-k}) для рассматриваемых моментов времени $n=1,\ldots ,N$.

Моменты функций распределения $\hat{f}_\alpha$, $\alpha=i,e$, находятся с помощью методов 
численного интегрирования, например метода трапеций~\cite{19-k}.

\section{Результаты численного моделирования}

Для двух описанных выше методов реализованы две отдельные программы в среде {Matlab~7.0}. 
Эти программы позволяют по заданным значениям концентраций и температур частиц $n_{i\infty}$, 
$n_{e\infty}$, $T_{i\infty}$ и~$T_{e\infty}$ в невозмущенной плазме, а также потенциала~$\varphi_p$, 
подаваемого на зонд, изучить эволюцию во времени плотностей тока частиц~$j_i$ и~$j_e$, концентраций 
частиц~$n_i$  и~$n_e$ в произвольной точке пространства в возмущенной зоне, а также динамику 
изменения напряженности~$E_y$ самосогласованного электрического поля во времени и пространстве.

С использованием разработанных программ проведены серии расчетных экспериментов, в которых 
значение концентраций варьировалось в пределах $n_{i\infty} \hm = n_{e\infty}\hm =10^{18}\div 
10^{22}$~м$^{-3}$. Значение температур было выбрано неизменным и равным $T_{i\infty}\hm = 
T_{e\infty}\hm=3000$~K, а значения потенциала, подаваемого на зонд, изменялись в пределах 
$\varphi_p\hm=0\div 2{,}6$~В.

На рис.~1  и~2 приведены графики изменения напряженности самосогласованного электрического
 поля (см.\ рис.~1) и плотности токов ионов (см.\linebreak\vspace*{-12pt}

\pagebreak

\end{multicols}

\begin{figure} %fig1
\vspace*{1pt}
\begin{center}
\mbox{%
\epsfxsize=162.594mm
\epsfbox{kud-1.eps}
}
\end{center}
\vspace*{-9pt}
\Caption{Динамика изменения плотности тока ионов во времени в фиксированной точке возмущенной 
зоны для значений потенциала: \textit{1}~--- $\varphi_p=-6$; 
\textit{2}~--- $\varphi_p=-16$; \textit{3}~--- $\varphi_p=- 30$ 
в случае применения методов Монте-Карло~(\textit{а}) 
и крупных частиц~(\textit{б})}
\end{figure}

\begin{figure} %fig2
\vspace*{1pt}
\begin{center}
\mbox{%
\epsfxsize=162.713mm
\epsfbox{kud-2.eps}
}
\end{center}
\vspace*{-9pt}
\Caption{Динамика изменения напряженности электрического поля во времени в фиксированной точке 
возмущенной зоны для значений потенциала: 
\textit{1}~--- $\varphi_p=-6$; \textit{2}~--- $\varphi_p=-16$; 
\textit{3}~--- $\varphi_p=-30$ в случае применения методов Монте-Карло~(\textit{а}) и
крупных частиц~(\textit{б})
}
\end{figure}

\begin{multicols}{2}

\noindent
 рис.~2) во времени в фиксированной точке пространства 
возмущенной зоны в случае применения обоих разработанных алгоритмов.


На основании полученных результатов можно отметить похожее поведение зависимостей 
напряженности электрического поля и плотности тока от времени в двух рассматриваемых случаях. 
Графики кривых сначала убывают, затем начинают возрастать, выходя в некоторый момент 
времени~$t^\prime$ (момент установления) на стационарные значения. 

Одинаковое поведение 
напряженности и плот\-ности тока можно объяснить из следующих соображений: плотность тока ионов в 
данной области пространства равна произведению концентрации ионов на их направленную скорость и 
на заряд иона. Скорость ионов, в свою очередь, зависит от заряда, массы и напряженности 
электрического поля. 
%\columnbreak

При внесении в плазму отрицательно заряженного зонда возникает электрическое поле, которое 
нарушает квазинейтральность плазмы. Для того чтобы компенсировать действие внешнего 
электрического поля, ионы устремляются к зонду, а электроны~--- от зонда. Это приводит к дисбалансу 
концентраций вблизи зонда и, как следствие, к увеличению разности потенциалов; график 
напряженности электрического поля убывает. Вскоре разделение зарядов компенсирует внешнее 
электрическое поле; график выходит на стационарное значение. 

Также можно отметить, что значения 
напряженности электрического поля и плотности тока частиц на зонд в момент установления для двух 
методов совпадают. 

Момент установления~$t^\prime$ зависит от при\-ме\-ня\-емо\-го метода решения. В~случае метода 
Мон\-те-Кар\-ло $t^\prime=3{,}5\div 4$~ед., а для метода крупных частиц совместно с методом 
расщепления $t^\prime\hm=5\div 5{,}5$~ед. Используя ко\-неч\-но-раз\-ност\-ный метод, можно 
получить динамику изменения функций распределения частиц~$f_\alpha$, $\alpha=i,e$, во времени и 
пространстве. Функции распределения позволяют наглядно представить влияние на картину 
распределения частиц вблизи зонда самой поверхности зонда и электрического поля.

\section{Заключение}
      
      В работе найдено решение задачи диагностики плоским зондом сильноионизованной плазмы с 
учетом столкновений заряженных частиц. Разработана математическая модель исследуемого явления, 
описываемая уравнениями Фок\-ке\-ра--План\-ка и Пуассона. Решение получено двумя методами:\linebreak 
статистическим и ко\-неч\-но-раз\-ност\-ным на основе\linebreak сформированных алгоритмов. Приведены 
резуль-\linebreak таты численного моделирования при различных\linebreak характерных параметрах задачи.
 Из  проведенных 
вычислительных экспериментов вытекает, что искомые величины: напряженность 
электрического поля, плотности токов частиц на зонд, концентрации частиц вблизи зонда~--- как по 
характеру зависимости, так и по числовым значениям совпадают. При применении метода 
      Мон\-те-Кар\-ло момент установления наступает быстрее по сравнению с конечно-разностным 
методом, однако конечно-разностный метод позволяет получить более наглядные результаты.

{\small\frenchspacing
{%\baselineskip=10.8pt
\addcontentsline{toc}{section}{Литература}
\begin{thebibliography}{99}

\bibitem{1-k}
\Au{Alexeff I., Anderson T.}
Experimental and theoretical results with plasma antenna~// IEEE Trans. Plasma Sci., 2006. Vol.~34. 
No.\,2. P.~166--172.

\bibitem{2-k}
\Au{Сысун В.\,И.}
Сильноионизованная низкотемпературная плазма в приборах электронной техники: Методы 
исследования, свойства, применение. Дисс. \ldots д-ра физ.-мат. наук в форме науч. докл.: 
01.04.08.~--- Пет\-ро\-за\-водск, 1996.

\bibitem{3-k}
\Au{Тухас В.\,А.}
Методология создания средств измерений и испытаний на устойчивость к кондуктивным помехам~// 
Мат-лы VI Междунар. симп. по электромагнитной совместимости и 
электромагнитной экологии.~--- СПб., 2005. С.~231--234.

\bibitem{4-k}
\Au{Гудзенко Л.\,И., Яковленко С.\,И.}
Плазменные лазеры.~--- М.: Атомиздат, 1978.  256~с.

\bibitem{5-k}
\Au{Звелто О.}
Принципы лазеров.~--- М.: Мир, 1990.  560~с.

\bibitem{6-k}
\Au{Сысун В.\,И., Хромой Ю.\,Д.}
Расширение канала мощного импульсного разряда в парах ртути~// Электронная техника, 1974. 
Сер.~4. Вып.~10. С.~80--85. 

\bibitem{7-k}
\Au{Винклер Дж.\,Р.}
Искусственные пучки частиц в космической плазме.~--- М.: Мир, 1985.  451~с.

\bibitem{8-k}
\Au{Bernstein I.\,B., Rabinowitz I.\,N.}
Theory of electrostatic probes in low-density plasma~// Phys. Fluids, 1959. Vol.~2. No.\,2. P.~112--121. 

\bibitem{9-k}
\Au{Альперт Я.\,Л., Гуревич А.\,В., Питаевский~Л.\,П.}
Искусственные спутники в разреженной плазме.~--- М.: Наука, 1964.  282~с.

\bibitem{10-k}
\Au{Чан П., Тэлбот Л., Турян~К.}
Электрические зонды в неподвижной и движущейся плазме.~--- М.: Мир, 1978.  202~с.

\bibitem{11-k}
\Au{Алексеев Б.\,В., Котельников В.\,А.}
Зондовый метод диагностики плазмы.~--- М.: Энергоатомиздат, 1989.  240~с.

\bibitem{12-k}
\Au{Пантелеев А.\,В., Кудрявцева И.\,А.}
Формирование математической модели двухкомпонентной плазмы с учетом столкновений 
заряженных частиц в случае плоского зонда~// Теоретические вопросы вычислительной техники и 
программного обеспечения: Межвузовский сб. научн. тр.~--- М.: МИРЭА, 2006. С.~11--21.

\bibitem{13-k}
\Au{Олдер Б.}
Вычислительные методы в физике плазмы.~--- М.: Мир, 1974.  111~с.

\bibitem{14-k}
\Au{Montgomery D.\,C., Tidman D.\,A.}
Plasma kinetic theory.~--- New York, 1964. 

\bibitem{15-k}
\Au{Кудрявцева И.\,А., Пантелеев А.\,В.}
Применение метода Мон\-те-Кар\-ло для анализа поведения двухкомпонентной плазмы с учетом 
столкновений между заряженными частицами~// Теоретические вопросы\linebreak
вычислительной техники и 
программного обеспечения: Межвузовский сб. научн. тр.~--- М.: МИРЭА, 2008. С.~122--128. 

\bibitem{16-k}
\Au{Семенов В.\,В., Пантелеев А.\,В., Руденко~Е.\,А., Бор\-та\-ков\-ский~А.\,С.}
Методы описания, анализа и синтеза нелинейных систем управления.~--- М.: МАИ, 1993.  312~с.

\bibitem{17-k}
\Au{Киреев В.\,И., Пантелеев А.\,В.}
Численные методы в примерах и задачах.~--- М.: Высшая школа, 2006.  480~с.

\bibitem{18-k}
\Au{Белоцерковский О.\,М., Давыдов~Ю.\,М.}
Метод крупных частиц в газовой динамике. Вычислительный эксперимент.~--- М.: Наука, 
Физматгиз, 1982.

\label{end\stat}

\bibitem{19-k}
\Au{Вержбицкий В.\,М.}
Основы численных методов.~--- М.: Высшая школа, 2002.  840~с.
 \end{thebibliography}
}
}


\end{multicols}        %3
\def\stat{gorshenin}

\def\tit{ЗАШУМЛЕНИЕ ДАННЫХ КОНЕЧНЫМИ СМЕСЯМИ НОРМАЛЬНЫХ 
И~ГАММА-РАСПРЕДЕЛЕНИЙ\\ С~ПРИМЕНЕНИЕМ К~ЗАДАЧЕ ОКРУГЛЕНИЯ НАБЛЮДЕНИЙ$^*$}

\def\titkol{Зашумление данных конечными смесями нормальных 
и~гамма-распределений с~применением к~задаче округления} % наблюдений}

\def\aut{А.\,К.~Горшенин$^1$}

\def\autkol{А.\,К.~Горшенин}

\titel{\tit}{\aut}{\autkol}{\titkol}

\index{Горшенин А.\,К.}
\index{Gorshenin A.\,K.}


{\renewcommand{\thefootnote}{\fnsymbol{footnote}} \footnotetext[1]
{Работа выполнена при поддержке РНФ (проект 18-71-00156).}}


\renewcommand{\thefootnote}{\arabic{footnote}}
\footnotetext[1]{Институт проблем информатики Федерального исследовательского центра 
<<Информатика и~управление>> Российской академии наук, \mbox{agorshenin@frccsc.ru}}

\vspace*{-12pt}




\Abst{Во многих реальных задачах проводится статистический анализ данных, 
содержащих дополнительные ошибки измерения, в~том числе в~виде округления, 
что в~ряде ситуаций может приводить к~достаточно существенным искажениям. 
В~настоящей статье для одной из возможных моделей округления получены оценки 
для неизвестного математического ожидания наблюдений в~предположении, что 
исходные данные дополнительно зашумлены с~по\-мощью случайных величин, 
име\-ющих распределения типа конечных смесей нормальных и~гам\-ма-за\-ко\-нов. 
Построены доверительные интервалы для неизвестного математического ожидания 
с~использованием уточненной оценки для дисперсии целой части случайной величины. 
Обсуждается алгоритм определения значения параметра для искусственного шума, 
добавление которого к~исходным данным способствует повышению качества работы 
метода скользящего разделения смесей.}

\KW{зашумленные данные; округленные наблюдения; конечные смеси нормальных 
распределений; конечные смеси гам\-ма-рас\-пре\-де\-ле\-ний; доверительные интервалы;  
метод скользящего разделения смесей}

\DOI{10.14357/19922264180304}
  
\vspace*{-4pt}


\vskip 10pt plus 9pt minus 6pt

\thispagestyle{headings}

\begin{multicols}{2}

\label{st\stat}


\section{Введение}

Во многих реальных задачах данные, являющиеся непрерывными по своей сути, 
регистрируются с~помощью инструментов, вносящих дополнительные ошибки 
измерения, в~том чис\-ле в~виде округления. Таким образом, статистический 
анализ проводится не для исходных, а для преобразованных некоторым 
случайным образом наблюдений, что в~ряде ситуаций может приводить к~достаточно
 существенным искажениям.

Для преодоления данной проблемы развивались различные подходы, в~том числе 
на основе смешанных моделей (см., например, статью~\cite{Wright2003}, в~которой 
различные компоненты  используются для пред\-став\-ле\-ния уровней округления). 
В~работе~\cite{Bai2009} приводятся результаты для моделей авторегрессии и~скользящего 
среднего для округленных данных, а~в~статье~\cite{Zhang2010} эти результаты 
развиваются и~исследуются их асимптотические свойства. 
В~статье~\cite{Zhao2012} исследован метод оценивания па\-ра\-мет\-ров конечных смесей 
вероятностных распределений (в~том чис\-ле, и~многомерных) 
на основе использования EM (expectation-maximization) 
алгоритма~\cite{Korolev2011-i} с~\mbox{целью} получения состоятельных 
и~асимптотически нормальных оценок.

В настоящей статье развиваются результаты для моделей округления, 
описанных в~работах~\cite{Ushakov2015,Ushakov2017a,Ushakov2017b}. 
В~их рамках будут получены оценки для неизве\-ст\-ного математического ожидания 
наблюдений в~предположении, что исходные данные зашумлены с~по\-мощью случайных 
величин, имеющих распределения типа конечных смесей нормальных и~гам\-ма-за\-ко\-нов. 
Это позволяет учесть большее количество случайных факторов, влия\-ющих на величину 
<<дополнительной>> ошибки. Также будут построены доверительные интервалы для 
неизвестного математического ожидания. Выражения для гам\-ма-рас\-пре\-де\-ле\-ний 
получены впервые. Также обсуждается алгоритм определения значения па\-ра\-мет\-ра для 
искусственного шума, добавление которого к~исходным данным способствует 
повышению качества работы метода скользящего разделения смесей~\cite{Gorshenin2016}.

\vspace*{-12pt}

\section{Предположения и~базовые отношения}

Для сокращения формулировок теорем в~сле\-ду\-ющих разделах сделаем ряд 
предположений, на которые будем ссылаться в~дальнейшем. Итак, пусть:
\begin{itemize}
\item[(A)] $X_1,X_2,\ldots$~--- независимые одинаково распределенные 
случайные величины с~неизвестным математическим ожиданием ${\sf E}_X\hm<+\infty$;
\item[(B)] $\varepsilon_1,\varepsilon_2,\ldots$~--- независимые одинаково 
распределенные случайные величины с~математическим ожиданием 
${\sf E}_\varepsilon\hm<+\infty$; %\label{B}
\item[(C)] $X_1,X_2,\ldots$ и~$\varepsilon_1,\varepsilon_2,\ldots$ 
являются независимыми;
\item[(D)] $Y_j=\left[X_j+\varepsilon_j+1/2\right]$ для всех $j\hm=1,2,\ldots$ 
представляют собой округление значения суммы случайных величин $X_j\hm+\varepsilon_j$ 
до ближайшего целого сверху (при этом запись~$[\cdot]$ соответствует целой 
части выражения).
\end{itemize}

В рамках данных предположений в~статье будут рассмотрены вопросы качества 
приближения неизвестного математического ожидания~${\sf E}_X$ для исходных данных 
в~ситуации, когда наблюдения для анализа получены с~аддитивной ошибкой c известными 
распределениями (см.\ предположение~(B)) и~дополнительно округляются до 
ближайшего целого (см.\ предположение~(D)).

Заметим, что в~силу усиленного закона больших чисел справедливы следующие выражения:
\begin{multline}
\fr{1}{n}\sum\limits_{j=1}^n Y_j\xrightarrow[n\to\infty]{\text{п.н.}}
{\sf E}_Y\equiv\mathbb{E}\left[X_1+\varepsilon_1+\fr{1}{2}\right]={}\\
{}=\mathbb{E}\left(X_j+\varepsilon_j+\fr{1}{2}\right)-\mathbb{E}
\left\{X_j+\varepsilon_j+\fr{1}{2}\right\}={}\\
{}={\sf E}_X+{\sf E}_\varepsilon+\fr{1}{2}-\mathbb{E}\left\{X_j+\varepsilon_j+\fr{1}{2}\right\}. 
\label{Law}
\end{multline}

Запись $\{\cdot\}$ в~формуле~\eqref{Law} соответствует дробной 
части выражения, а~п.н.\ обозначает сходимость в~смысле почти наверное.

Для доказательства результатов в~дальнейшем потребуется следующее 
представления для дробной части  абсолютно непрерывной случайной величины~$Z$ 
с~абсолютно  интегрируемой характеристической функцией~$\varphi_Z(t)$
 (см., например, Лемму~4 в~работе~\cite{Ushakov2017b}):
\begin{equation}
\label{Fract}
\mathbb{E}\{Z\}=\fr{1}{2}-\sum\limits_{n=1}^\infty 
\fr{\mathrm{Im}\left (\varphi_Z(2\pi n)\right)}{\pi n}\,.
\end{equation}

Через $\mathrm{Im}\,(\cdot)$ в~формуле~\eqref{Fract} обозначена мнимая часть 
соответствующей функции.

При построении доверительных интервалов в~дальнейшем будет 
использована следующая оценка, справедливая для любой случайной величины~$Z$:
\begin{equation}
\mathbb{D}[Z]\leqslant \left(\sqrt{\mathbb{D} Z}+\fr{1}{2}\right)^2.
\label{Var}
\end{equation}
Она может быть проверена непосредственно с~учетом представления 
$\mathbb{D} [Z]\hm=\mathbb{D}\left(Z\hm-\{Z\}\right)$, неравенства 
Ко\-ши--Бу\-ня\-ков\-ско\-го для ковариации и~соотношения 
 $\mathbb{D}\{Z\}\hm\leqslant 1/4$, справедливого для любой случайной величины~$Z$ 
 (см., например, статью~\cite{Ushakov2017b}). Отметим, что данная оценка 
 является более точной по сравнению с~использованным для аналогичных 
 целей в~работе~\cite{Ushakov2017b} соотношением 
 $\mathbb{D} [Z]\hm\leqslant 2\mathbb{D} Z\hm+1/2$. Действительно,
\begin{equation*}
2\mathbb{D} Z+\fr{1}{2}-\left(\sqrt{\mathbb{D} Z}+\fr{1}{2}\right)^2=
\left(\sqrt{\mathbb{D} Z}-\fr{1}{2}\right)^2\geqslant0\,,
\end{equation*}
причем для всех $\sqrt{\mathbb{D} Z}\hm\neq {1}/{2}$ 
данное неравенство является строгим.

\section{Конечные смеси нормальных законов}

Для случайной величины~$X$, имеющей распределение типа 
конечной смеси нормальных законов~\cite{Korolev2011-i} с~параметрами 
${\bf a}\hm=(a_1,\ldots, a_k)$, $a_j\hm\in \mathbb{R}$, 
$\boldsymbol{\sigma}\hm=(\sigma_1,\ldots, \sigma_k)$, 
$\sigma_j\hm>0$, ${\bf p}\hm=(p_1,\ldots, p_k)$, $p_j\hm\geqslant 0$, 
$\sum\nolimits_{j=1}^{k}p_j\hm=1$, плот\-ность которого задается выражением
\begin{equation}
f_X(x)=\sum\limits_{j=1}^{k}\fr{p_j}{\sigma_j\sqrt{2\pi}}\,e^{-(x-a_j)^2/(2\sigma_j^2)}\,,
\label{FinNormMixt}
\end{equation}
характеристическая функция имеет вид:
\begin{equation}
\varphi_X(t)=\int\limits_{-\infty}^{+\infty}\!\!e^{itx} f_X(x)\, dx = 
\sum\limits_{j=1}^{k}p_j e^{ita_j-\sigma_j^2 t^2/2}.
\label{ChiFinNormMixt}
\end{equation}

Абсолютная интегрируемость  $\varphi_X(t)$ вытекает из свойств 
характеристической функции нормального распределения. 
Заметим, что в~точке $t\hm=2\pi n$ выражение~\eqref{ChiFinNormMixt} принимает 
сле\-ду\-ющий вид:
\begin{equation}
\label{ChiFinNormMixt2npi}
\varphi_X(2\pi n)= \sum\limits_{j=1}^{k}p_j e^{-2\pi^2 \sigma_j^2 n^2}\,.
\end{equation}

Рассмотрим вопрос точности оценивания неизвестного математического ожидания~${\sf E}_X$ 
при до\-бав\-ле\-нии зашумления.

\smallskip

\noindent
\textbf{Теорема~1.}\ 
\textit{Пусть выполнены предположения}~(A)--(D), 
\textit{причем случайные величины~$\varepsilon_j$, $j\hm=1,2,\ldots$, 
имеют распределение типа конечной $k$-ком\-по\-нент\-ной смеси нормальных законов 
вида}~\eqref{FinNormMixt} \textit{с~па\-ра\-мет\-ра\-ми~${\bf a}$, $\boldsymbol{\sigma}$ 
и~${\bf p}$. Тогда}
\begin{equation}
\label{Th1Eq}
\left\lvert {\sf E}_Y-{\sf E}_X\right\rvert \leqslant 
A+\fr{1}{\pi}\left(1+\fr{1}{4\pi^2\sigma^2}\right)e^{-2\pi^2\sigma^2}\,, 
\end{equation}
\textit{где} $A=\max(|a_1|,\ldots,|a_k|)$, $\sigma\hm=\min(\sigma_1,\ldots,\sigma_k)$.

\smallskip


\noindent
Д\,о\,к\,а\,з\,а\,т\,е\,л\,ь\,с\,т\,в\,о\,.\ \
С~учетом пред\-став\-ле\-ний~\eqref{Law},~\eqref{Fract} и~\eqref{ChiFinNormMixt2npi}, 
ограниченности модуля характеристической функции, а~также не\-за\-ви\-си\-мости 
случайных величин~$X_j$ и~$\varepsilon_j$ имеем:
\begin{multline*}
\left\lvert {\sf E}_Y-{\sf E}_X\right\rvert =
\left\lvert {\sf E}_\varepsilon+\fr{1}{2}-\mathbb{E}\left\{X_j+
\varepsilon_j+\fr{1}{2}\right\}\right\rvert={}\\
{}=\left\lvert {\sf E}_\varepsilon+\sum\limits_{n=1}^\infty
\fr{\mathrm{Im} \left(\varphi_{X_j}(2\pi n)\varphi_{\varepsilon_j}(2\pi n)
\varphi_{1/2}(2\pi n)\right)}{\pi n}\right\rvert={}\\
=\left\lvert 
\vphantom{\fr{(-1)^n\sum\nolimits_{j=1}^{k}p_j e^{-2\pi^2 \sigma_j^2 n^2} 
\mathrm{Im} \left(\varphi_{X_j}(2\pi n)\right)}{\pi n}}
{\sf E}_\varepsilon+{}\right.\\
\left.{}+\sum\limits_{n=1}^\infty
\fr{\mathrm{Im} \left(\varphi_{X_j}(2\pi n) 
\sum\nolimits_{j=1}^{k}p_j e^{-2\pi^2 \sigma_j^2 n^2} 
e^{\pi n}\right)}{\pi n}\right\rvert={}\\
{}=\left\lvert 
\vphantom{\fr{(-1)^n\sum\nolimits_{j=1}^{k}p_j e^{-2\pi^2 \sigma_j^2 n^2} 
\mathrm{Im} \left(\varphi_{X_j}(2\pi n)\right)}{\pi n}}
{\sf E}_\varepsilon+{}\right.\\
\left.{}+\sum\limits_{n=1}^\infty
\fr{(-1)^n\sum\nolimits_{j=1}^{k}p_j e^{-2\pi^2 \sigma_j^2 n^2} 
\mathrm{Im} \left(\varphi_{X_j}(2\pi n)\right)}{\pi n}\right\rvert\leqslant{}\\
{}\leqslant \left\lvert {\sf E}_\varepsilon\right\rvert+\left\lvert\
\sum\limits_{j=1}^{k}p_j\sum\limits_{n=1}^\infty 
\fr{1}{\pi n} e^{-2\pi^2 \sigma_j^2 n^2}\right\rvert\leqslant {}\\
\\
{}\leqslant
\max\left(|a_1|,\ldots,|a_k|\right)+{}\\
{}+\sum\limits_{j=1}^{k} 
\fr{p_j}{\pi} \left(\!1+\fr{1}{4\pi^2\sigma_j^2}\!\right)\!e^{-2\pi^2\sigma_j^2}\leqslant{}\\
{}\leqslant
A+\fr{1}\pi\left(1+\fr{1}{4\pi^2\sigma^2}\right)e^{-2\pi^2\sigma^2}\,.
\end{multline*}

Справедливость использованной оценки 
\begin{equation*}
\sum\limits_{n=1}^\infty
\fr{e^{-2\pi^2 \sigma_j^2 n^2}}{n}\leqslant 
\left(1+\fr{1}{4\pi^2\sigma_j^2}\right)e^{-2\pi^2\sigma_j^2}
\end{equation*}
может быть проверена непосредственно (например, см.\ доказательство Теоремы~6 
в~статье~\cite{Ushakov2017b}).~\hfill$\square$

\smallskip

\noindent
\textbf{Замечание~1.}
В~случае, если зашумление производится нормально распределенными случайными 
величинами c нулевыми средними (т.\,е.\ в~формуле~\eqref{Th1Eq} необходимо считать 
$A\hm=0$, $k\hm=1$), то Тео\-ре\-ма~1 совпадает с~результатом, 
полученным в~работе~\cite{Ushakov2017b}.


\smallskip

Рассмотрим вопросы построения доверительного интервала для неизвестного 
математического ожидания~${\sf E}_X$ в~предположении, что случайные величины~$X_j$ не 
содержат ошибок измерения, а~все погрешности учтены исключительно в~за\-шум\-ля\-ющих 
элементах~$\varepsilon_j$.

\smallskip

\noindent
\textbf{Теорема~2.}\ 
\textit{Пусть выполнены предположения}~(A)--(D), 
\textit{причем случайные величины~$\varepsilon_j$, $j\hm=1,2,\ldots$, имеют 
распределение типа конечной $k$-ком\-по\-нент\-ной смеси нормальных законов 
вида}~\eqref{FinNormMixt} \textit{с~параметрами~${\bf a}$, $\boldsymbol{\sigma}$ 
и~${\bf p}$, а~случайные величины} $X_j\stackrel{\text{п.н.}}{=}{\sf E}_X$, $j\hm=1,2,\ldots$ 
\textit{Тогда доверительный интервал для~${\sf E}_X$ при условии $0\hm<\alpha\hm<1$ имеет вид}:
\begin{equation} 
\label{Th2Eq}
\hat{{\sf E}}_X - f({\bf a},\boldsymbol{\sigma},\alpha,n) 
\leqslant {\sf E}_X \leqslant  \hat{{\sf E}}_X + f({\bf a},\boldsymbol{\sigma},\alpha,n),
\end{equation}
\textit{где}

\vspace*{-2pt}

\noindent
\begin{align}
\hat{{\sf E}}_X&=\fr{1}{n} \sum\limits_{j=1}^{n} Y_j\,; \label{Th2hatE}\\
f({\bf a},\boldsymbol{\sigma},\alpha,n)&=
\fr{z_{1-{\alpha}/2}}{\sqrt{n}} \left(\sqrt{A^2+\Sigma^2}+\fr{1}{2}\right) +{}\notag\\
&{}+A+\fr{1}\pi\left(1+\fr{1}{4\pi^2\sigma^2}\right)e^{-2\pi^2\sigma^2}\,;
  \label{Th2f}
\end{align}
\textit{$z_{1-{\alpha}/2}$~--- $\left(1-{\alpha}/2\right)$-кван\-тиль 
стандартного нормального распределения; $A\hm=\max(|a_1|,\ldots,|a_k|)$; 
$\Sigma\hm=\max(\sigma_1,\ldots,\sigma_k)$; $\sigma\hm=\min(\sigma_1,\ldots,\sigma_k)$}. 


\smallskip

\noindent
\noindent
Д\,о\,к\,а\,з\,а\,т\,е\,л\,ь\,с\,т\,в\,о\,.\ \
Из центральной предельной тео\-ре\-мы с~учетом условия~(A) следует, 
что величина~$\hat{{\sf E}}_X$~\eqref{Th2hatE} асимптотически нормальна с~математическим 
ожиданием 
\begin{equation}
{\sf E}_Y\equiv \mathbb{E}\left[{\sf E}_X+\varepsilon_1+\fr{1}{2}\right] \label{EY}
\end{equation}
и дисперсией
\begin{equation}
\fr{1}{n} {\sf D}_Y\equiv \fr{1}{n}\mathbb{D}\left[{\sf E}_X+\varepsilon_1+
\fr{1}{2}\right]. \label{DY}
\end{equation}

Воспользовавшись оценкой~\eqref{Var}, получим:

\vspace*{-2pt}

\noindent
\begin{multline*}
{\sf D}_Y \leqslant  \left(\sqrt{\mathbb{D} \left({\sf E}_X+\varepsilon_1+\fr{1}{2}\right)}+
\fr{1}{2}\right)^2={}\\
{}=
\left(\sqrt{\mathbb{D}\varepsilon_1}+\fr{1}{2}\right)^2= {}\\
{}= \left(\sqrt{\sum\limits_{j=1}^{k}p_j\left(\left(a_j-\sum\limits_{t=1}^{k}
p_t a_t\right)^2+\sigma_j^2\right)}+\fr{1}{2}\right)^2\leqslant {}\\ 
{}\leqslant \left(\sqrt{A^2+\Sigma^2}+\fr{1}{2}\right)^2\,.
\end{multline*}
Тогда доверительный интервал уровня $1\hm-\alpha$ для математического ожидания~${\sf E}_Y$ 
имеет вид:
\begin{equation*}
\mathbb{P}\left(\left\lvert \hat{{\sf E}}_X-{\sf E}_Y\right\rvert \leqslant 
\fr{z_{1-{\alpha}/2}}{\sqrt{n}} 
\left(\sqrt{A^2+\Sigma^2}+\fr{1}{2}\right)\right)\geqslant 1-\alpha\,.
\end{equation*}

\begin{table*}[b]\small
\begin{center}

\begin{tabular}{|c|c|c|c|c|c|c|c|}
\multicolumn{7}{p{100mm}}{Численные решения уравнений~\eqref{f1} и~\eqref{f2} относительно 
параметра~$\sigma$ для некоторых значений~$n$ и~$\alpha$}\\
\multicolumn{7}{c}{\ }\\[-6pt]
\hline
\multicolumn{1}{|c|}{Размер}  & \multicolumn{2}{c|}{Уровень $\alpha=0{,}1$}& 
\multicolumn{2}{c|}{Уровень $\alpha=0{,}05$}& 
\multicolumn{2}{c|}{Уровень $\alpha=0{,}01$}\\
\cline{2-7}
\multicolumn{1}{|c|}{выборки $n$}&$\sigma_1$&$\sigma_2$&$\sigma_1$&$\sigma_2$&$\sigma_1$&$\sigma_2$\\
\hline
$\hphantom{000}100$&$0{,}4302$&$0{,}435$&$0{,}419$&$0{,}425$&$0{,}4002$&$0{,}408$\\
%\hline
$\hphantom{000}200$&$0{,}452$&$0{,}455$ &$0{,}441$&$0{,}445$&$0{,}424$&$0{,}429$\\
%\hline
$\hphantom{00}1000$&$0{,}499$&$0{,}499$ &$0{,}489$&$0{,}489$&$0{,}473$&$0{,}475$\\
%\hline
$\hphantom{0}10000$&$0{,}558$&$0{,}556$ &$0{,}549$&$0{,}547$&$0{,}536$&$0{,}534$\\
%\hline
$100000$&$0{,}611$&$0{,}607$ &$0{,}603$&$0{,}599$&$0{,}591$&$0{,}588$\\
\hline
\end{tabular}
\end{center}
\end{table*}


\noindent
Откуда следует справедливость соотношения~\eqref{Th2Eq} c~уче\-том 
очевидного неравенства

\pagebreak

\noindent
\begin{equation*}
\left\lvert \hat{{\sf E}}_X-{\sf E}_X\right\rvert \leqslant 
\left\lvert \hat{{\sf E}}_X-{\sf E}_Y\right\rvert +\left\lvert {\sf E}_Y-{\sf E}_X\right\rvert 
\end{equation*}
и оценки~\eqref{Th1Eq} из Теоремы~1.~\hfill$\square$

\smallskip

\noindent
\textbf{Замечание~2.}
В~работе~\cite{Gorshenin2016} было продемонстрировано повышение точ\-ности 
работы метода скользящего разделения конечных нормальных смесей за счет 
введения дополнительной компоненты, имеющей нормальное 
распределение $\mathcal{N}(0,\sigma^2)$ с~математическим ожиданием, равным~$0$, 
и~стандартным отклонением~$\sigma$. При этом была отмечена сложность выбора 
параметра~$\sigma$ для сохранения структуры выборки, близкой к~исходной. 
Результат Теоремы~2 может быть использован с~данной целью, если положить $k\hm=1$, 
$a_j\hm=0$ для всех $j\hm=1,2,\ldots$ и~выбирать величину~$\sigma$ как 
минимизирующую длину доверительного интервала~\eqref{Th2Eq}. Для 
этого необходимо найти производную функции $f(0,\sigma,\alpha,n)$~\eqref{Th2f} 
и~численно решить уравнение
\begin{multline}
f_\sigma'(0,\sigma,\alpha,n)\equiv \fr{z_{1-{\alpha}/2}}{\sqrt{n}} - {}\\
{}-
e^{-2\pi^2\sigma^2}\left(4\pi\sigma+\fr{1}{2\pi^3\sigma^3}+
\fr{1}{\pi\sigma}\right)=0
\label{f1}
\end{multline}
относительно неизвестного параметра~$\sigma$ при выбранных значениях величин~$n$ 
и~$\alpha$. В~качестве альтернативы можно использовать вид доверительного интервала 
из статьи~\cite{Ushakov2017b}, полученный с~помощью неравенства $\mathbb{D} [Z]
\hm\leqslant 2\mathbb{D} Z\hm+{1}/{2}$, и~искать решение уравнения вида:
\begin{multline}
\hspace*{-2.90578pt}\fr{2\sigma z_{1-{\alpha}/2}}{\sqrt{n (2\sigma^2+{1}/{2})}} -
 e^{-2\pi^2\sigma^2}\left(4\pi\sigma+\fr{1}{2\pi^3\sigma^3}+
 \fr{1}{\pi\sigma}\right)={}\\
 {}=0\,.\label{f2}
\end{multline}

Примеры найденных значений~$\sigma$ для типичных размеров выборок в~методе 
скользящего разделения смесей (учитываются как возможная ширина окна, 
так и~общее количество наблюдений в~анализируемом ряде) приведены в~таблице 
(использован метод оптимизации \verb"Trust-Region Dogleg" пакета \verb"MATLAB" 
c~настройками по умолчанию), в~которой через~$\sigma_1$ обозначено приближенное  
решение уравнения~\eqref{f1}, a~через $\sigma_2$~--- уравнения~\eqref{f2}.


Проверка практической эффективности данного подхода в~качестве 
критерия выбора параметров зашумляющего распределения для повышения 
точности работы метода скользящего разделения смесей может быть отмечена 
как задача для дальнейших исследований.


\section{Конечные смеси гамма-распределений}

Для случайной величины~$X$, имеющей распределение типа конечной смеси 
гам\-ма-рас\-пре\-де\-ле\-ний с~параметрами ${\bf r}\hm=(r_1,\ldots, r_k)$,
 $r_j\hm>0$, $\boldsymbol{\lambda}\hm=(\lambda_1,\ldots, \lambda_k)$, $\lambda_j\hm>0$, 
 ${\bf p}\hm=(p_1,\ldots, p_k)$, $p_j\hm\geqslant 0$, $\sum\nolimits_{j=1}^{k}p_j\hm=1$, 
 плот\-ность которого задается выражением
\begin{equation}
f_X(x)=\sum\limits_{j=1}^{k}p_j\fr{\lambda_j^{r_j} e^{-\lambda_j x}}
{\Gamma(r_j)}\,x^{r_j-1}\,,
\label{FinGammaMixt}
\end{equation}
характеристическая функция имеет следующий вид:
%характеристическая функция задается следующим выражением:
\begin{equation}
\varphi_X(t)=\!\int\limits_{-\infty}^{+\infty}\!\!\!e^{itx} f_X(x)\, dx = \!
\sum\limits_{j=1}^{k}p_j \left(\!1-\fr{it}{\lambda_j}\right)^{-r_j}\!.\!
\label{ChiFinGammaMixt}
\end{equation}

Отметим, что подобные модели зашумления разумно использовать в~случае, 
если известно, что данные сосредоточены на положительной полуоси, например 
при анализе различных информационных потоков (см., в~част\-ности, 
 работу~\cite{Gorshenin2013}). 

Проверим абсолютную интегрируемость функции $\varphi_X(t)$~\eqref{ChiFinGammaMixt}. 
Имеем:
\begin{multline*}
\int\limits_{-\infty}^{+\infty}\left\lvert\varphi_X(t)\right\rvert\, dt\leqslant 
\sum\limits_{j=1}^{k}p_j \int\limits_{-\infty}^{+\infty}\left\lvert \left(
1-\fr{it}{\lambda_j}\right)^{-r_j}\right\rvert \, dt={}\\
{}=\sum\limits_{j=1}^{k}p_j \int\limits_{-\infty}^{+\infty} \left\lvert\left(
\fr{\lambda_j(\lambda_j+it)}{\lambda_j^2+t^2}\right)^{r_j}\right\rvert\, dt \leqslant{}\\
{}\leqslant\sum\limits_{j=1}^{k}p_j \lambda_j \int\limits_{-\infty}^{+\infty}\left(
1+y^2\right)^{-{r_j}/{2}}\, dy\,.
\end{multline*}

Подынтегральное выражение при $r_j\hm\geqslant 2$ может быть оценено сверху 
функцией $1/({1+y^2})$, при этом соответствующий интеграл равен~$\pi$, что влечет 
абсолютную интегрируемость характеристической функции для конечной смеси 
гам\-ма-рас\-пре\-де\-ле\-ний. Поэтому в~дальнейшем будем предполагать,
 что $r_j\hm\geqslant 2$ для всех возможных значений $j\hm=1,2,\ldots$

Рассмотрим вопрос точ\-ности оценивания неизвестного математического ожидания ${\sf E}_X\hm>0$ 
при добавлении зашумления.

\smallskip

\noindent
\textbf{Теорема~3.}
\textit{Пусть выполнены предположения}~(A)--(D), 
\textit{причем случайные величины~$\varepsilon_j$, $j\hm=1,2,\ldots$, имеют 
распределение типа конечной $k$-ком\-по\-нент\-ной смеси 
гам\-ма-рас\-пре\-де\-ле\-ний вида}~\eqref{FinGammaMixt} 
\textit{с~па\-ра\-мет\-ра\-ми~${\bf r}$, $\boldsymbol{\lambda}$ и~${\bf p}$. Тогда}
\begin{equation}
\label{Th3Eq}
\left\lvert {\sf E}_Y-{\sf E}_X\right\rvert \leqslant \fr{R}{\lambda}+
\fr{\Lambda^{R}}{2^{r}\pi^{r+1}}\left(1+\frac1{r}\right)\,,
\end{equation}
\textit{где} $r=\min(r_1, \ldots,r_k)$; $R\hm=\max(r_1, \ldots,r_k)$; 
$\lambda\hm=\max(\lambda_1, \ldots,\lambda_k)$; 
$\Lambda\hm=\max(\lambda_1, \ldots,\lambda_k)$.

\smallskip

\noindent
Д\,о\,к\,а\,з\,а\,т\,е\,л\,ь\,с\,т\,в\,о\,.\ \
С~учетом пред\-став\-ле\-ний~\eqref{Law} и~\eqref{Fract}, ограниченности 
модуля характеристической функции, перехода от тригонометрической к~показательной 
записи комплексных чисел, а~также независимости случайных величин~$X_j$ 
и~$\varepsilon_j$ \mbox{имеем}:
\begin{multline*}
\left\lvert {\sf E}_Y-{\sf E}_X\right\rvert
\leqslant \left\lvert {\sf E}_\varepsilon\right\rvert+ {}\\
{}+\left\lvert\sum\limits_{n=1}^\infty
\left(
(-1)^n\mathrm{Im} \left(\sum\limits_{j=1}^{k}p_j \varphi_{X_j}(2\pi n)\left(
\vphantom{\fr{2\pi n}{\lambda_j}}
1-{}\right.\right.\right.\right.\\
\left.\left.\left.\left.{}-i\left(\fr{2\pi n}{\lambda_j}\right)\right)^{-r_j}\right)
\Bigg/ ({\pi n})
\vphantom{\sum\limits_{j=1}^{k}}
\right)\right\rvert={}\\
{}=\left\lvert {\sf E}_\varepsilon\right\rvert+ 
\left\lvert\sum\limits_{n=1}^\infty
\left(\!(-1)^n\mathrm{Im} \!\left(\sum\limits_{j=1}^{k}p_j \left(\!
1+\fr{4\pi^2 n^2}{\lambda_j^2}\right)^{- {r_j}/2}\!\times{}\right.\right.\right.\hspace*{-2.8663pt}\\
\left.\left.\left.{}\times \varphi_{X_j}(2\pi n)\,
e^{-ir_j\mathrm{arctan}\,({{t}/{\lambda_j}})}\right)
\Bigg/
({\pi n})
\vphantom{\left(
1+\fr{4\pi^2 n^2}{\lambda_j^2}\right)^{- {r_j}/2}}
\right)\right\rvert\leqslant{}\\
{}\leqslant \left\lvert {\sf E}_\varepsilon\right\rvert+\sum\limits_{j=1}^{k}
p_j\sum\limits_{n=1}^\infty\fr{1}{\pi n}\left(
1+\fr{4\pi^2 n^2}{\lambda_j^2}\right)^{-{r_j}/2}\leqslant{}\\
{}\leqslant  \fr{R}\lambda + \sum\limits_{j=1}^{k}p_j
\sum\limits_{n=1}^\infty\left(\fr{1}{\pi n}\,
\fr{\lambda_j^{r_j}}{(2\pi)^{r_j} n^{r_j}}\right)\leqslant {}
\\
{}\leqslant  \fr{R}{\lambda} + \sum\limits_{j=1}^{k}p_j 
\fr{\lambda_j^{r_j}}{2^{r_j}\pi^{r_j+1}}\left(1+\int\limits_{1}^{\infty}
\fr{1}{ x^{r_j+1}}\,dx\right)
\leqslant{}\\
{}\leqslant \fr{R}{\lambda}+\fr{\Lambda^{R}}{2^{r}\pi^{r+1}}\left(1+\fr{1}{r}\right).
\end{multline*}

При переходе от суммы к~интегралу используется факт убывания функции как переменной~$n$ 
(или~$x$).~\hfill$\square$


\smallskip

\noindent
\textbf{Замечание~3.}\
Теорема~3 описывает соответ\-ст\-ву\-ющий результат для гам\-ма-рас\-пре\-де\-лен\-ных 
за\-шум\-ля\-ющих случайных величин, если положить $k\hm=1$ в~выражении~\eqref{Th3Eq}. 
При этом, очевидно, $r\hm\equiv R$ и~$\lambda\hm\equiv \Lambda$.


\smallskip

Рассмотрим вопросы построения доверительного интервала для неизвестного 
математического ожидания ${\sf E}_X\hm>0$ в~предположении, что случайные величины~$X_j$ 
не содержат ошибок измерения, а все погрешности учтены исключительно в~за\-шум\-ля\-ющих 
элементах~$\varepsilon_j$.

\smallskip

\noindent
\textbf{Теорема~4.}
\textit{Пусть выполнены предположения}~(A)--(D), 
\textit{причем случайные величины~$\varepsilon_j$, $j\hm=1,2,\ldots$, имеют 
распределение типа конечной $k$-ком\-по\-нент\-ной смеси 
гам\-ма-рас\-пре\-де\-ле\-ний вида}~\eqref{FinGammaMixt} 
\textit{с~па\-ра\-мет\-ра\-ми~${\bf r}$, $\boldsymbol{\lambda}$ и~${\bf p}$, 
а~случайные величины} $X_j\stackrel{\text{п.н.}}{=}{\sf E}_X$, $j=1,2,\ldots$ 
\textit{Тогда доверительный интервал для~${\sf E}_X$ при условии $0\hm<\alpha\hm<1$ имеет вид}:
\begin{equation} 
\label{Th4Eq}
\left\lvert {\sf E}_X - \hat{{\sf E}}_X\right\rvert \leqslant  
f({\bf r},\boldsymbol{\lambda},\alpha,n),
\end{equation}
\textit{где}

\vspace*{-9pt}

\noindent
\begin{align}
\hat{{\sf E}}_X&=\fr{1}{n} \sum\limits_{j=1}^{n} Y_j\,; \label{Th4hatE}\\[-4pt]
f({\bf r}, \boldsymbol{\lambda},\alpha,n)&=\fr{z_{1-{\alpha}/2}}{\sqrt{n}} \left(
\sqrt{\fr{R(R+1)}{\lambda^2}-\fr{r^2}{\Lambda^2}}+\fr{1}{2}\right) +{}\notag\\[-1pt]
&\hspace*{20mm}{}+
\fr{R}{\lambda}+\fr{\Lambda^{R}}{2^{r}\pi^{r+1}}\left(1+\fr{1}{r}\right); \notag
\end{align}
\textit{$z_{1-{\alpha}/2}$~--- $\left(1-{\alpha}/2\right)$-кван\-тиль 
стандартного нормального распределения; $r\hm=\min(r_1, \ldots,r_k)$; 
$R\hm=\max(r_1, \ldots,r_k)$; $\lambda\hm=\max(\lambda_1, \ldots,\lambda_k)$; 
$\Lambda\hm=\max(\lambda_1, \ldots,\lambda_k)$}. 

\smallskip

\noindent
Д\,о\,к\,а\,з\,а\,т\,е\,л\,ь\,с\,т\,в\,о\,.\ \
Из центральной предельной теоремы с~учетом условия~(A) 
следует, что величина~$\hat{{\sf E}}_X$~\eqref{Th4hatE} асимптотически нормальна 
с~математическим ожиданием~${\sf E}_Y$~\eqref{EY} и~дисперсией $(1/n){\sf D}_Y$~\eqref{DY}. 
Пользуясь определением и~свойствами гам\-ма-функ\-ции, а~также оценкой~\eqref{Var} 
получим:

\noindent
\begin{multline*}
{\sf D}_Y \leqslant \left(\sqrt{\sum\limits_{j=1}^k p_j
\fr{\lambda_j^{r_j}}{\Gamma(r_j)} \int\limits_{0}^{+\infty} 
e^{\lambda_j x}x^{r_j+1}\, dx}+\fr{1}{2}\right)^2= {}\\[-0.5pt]
= \left(\sqrt{\sum\limits_{j=1}^{k}p_j
\fr{r_j(r_j+1)}{\lambda_j^2}-\left(\sum\limits_{j=1}^{k}p_j
\fr{r_j}{\lambda_j}\right) ^2}+\fr{1}{2}\right)^2\leqslant {}\\[-1.5pt]
{}\leqslant \left(\sqrt{\fr{R(R+1)}{\lambda^2}-\fr{r^2}{\Lambda^2}}+\fr{1}{2}\right)^2\,.
\end{multline*}

Аналогично доказательству Тео\-ре\-мы~2 с~учетом оценки~\eqref{Th3Eq} 
отсюда следует справедливость соотношения~\eqref{Th4Eq}.~\hfill$\square$

\vspace*{-12pt}

\section{Заключение}

Итак, в~работе получены оценки для математического ожидания наблюдений в~предположении 
зашумления конечными смесями нормальных\linebreak (Тео\-ре\-ма~1) 
и~гам\-ма-рас\-пре\-де\-ле\-ний (Тео\-ре\-ма~3). 
%
Построены доверительные интервалы 
для неизвестного математического ожидания в~этих случаях с~использованием 
уточненной оценки~\eqref{Var} 
(Тео\-ре\-мы~2 и~4 соответственно). Отметим, что соответствующие соотношения 
зависят только от <<экстремальных>> значений параметров смесей, но не от числа 
компонент и~весов в~распределении зашумляющих наблюдений. 
%
Замечание~2 
предлагает подход, который  может быть использован для определения неизвестного 
параметра искусственно добавляемого к~исходным данным шума для улучшения качества 
работы метода скользящего разделения смесей.

\smallskip
Автор выражает признательность доктору фи\-зи\-ко-ма\-те\-ма\-ти\-че\-ских наук, 
профессору Виктору Юрьевичу Королеву за идею использования оценки 
дисперсии вида~\eqref{Var} и~другие полезные обсуждения в~рамках 
работы над данной статьей.

\vspace*{-12pt}

{\small\frenchspacing
 {%\baselineskip=10.8pt
 \addcontentsline{toc}{section}{References}
 \begin{thebibliography}{99}
\bibitem{Wright2003} \Au{Wright~D.\,E., Bray~I.} 
A~mixture model for rounded data~// J.~Roy. Stat. Soc.~D 
Sta., 2003. Vol.~52. P.~3--13.

\columnbreak

\bibitem{Bai2009} \Au{Bai~Z., Zheng~S., Zhang~B., Hu~G.} 
Statistical analysis for rounded data~// J.~Stat. Plan.  Infer., 2009. 
Vol.~139. Iss.~8. P.~2526--2542.

\bibitem{Zhang2010} \Au{Zhang~B., Liu~T., Bai~Z.\,D.} 
Analysis of rounded data from dependent sequences~// 
Ann. I.~Stat. Math., 2010. Vol.~62. Iss.~6. P.~1143--1173.

\bibitem{Zhao2012} \Au{Zhao~N., Bai~Z.} 
Analysis of rounded data in mixture normal model~// Stat. Pap., 2012. 
Vol.~53. P.~895--914.

\bibitem{Korolev2011-i} \Au{Королев~В.\,Ю.} 
Ве\-ро\-ят\-но\-ст\-но-ста\-ти\-сти\-че\-ские методы декомпозиции волатильности 
хаотических процессов.~--- М.: Изд-во Моск. ун-та, 2011. 512~с.

\bibitem{Ushakov2015} \Au{Ушаков В.\,Г., Ушаков Н.\,Г.} 
Об усреднении округленных данных~// Информатика и~её применения, 2015. Т.~9. 
Вып.~4. С.~106--109.

\bibitem{Ushakov2017a} \Au{Ушаков~В.\,Г., Ушаков~Н.\,Г.} 
Границы точ\-ности восстановления информации, 
теряемой при округлении результатов наблюдений~// 
Вестник Московского университета. Серия~15: Вычислительная математика и~кибернетика, 
2017. №\,2. С.~26--30.

\bibitem{Ushakov2017b} \Au{Ushakov~N.\,G., Ushakov~V.\,G.} 
Statistical analysis of rounded data: Recovering of information lost due to rounding~// 
J.~Korean Stat. Soc., 2017.  Vol.~46. No.\,3. P.~426--437.

\bibitem{Gorshenin2016} \Au{Gorshenin~A.\,K., Korolev~V.\,Yu.} 
A~noising method for the identification of the stochastic structure of 
information flows~// Comm. Com. Inf. Sc., 2017. 
Vol.~678. P.~279--289.

\bibitem{Gorshenin2013} 
\Au{Gorshenin~A., Korolev~V.} Modelling of statistical
fluctuations of information flows by mixtures of gamma distributions~// 
27th European Conference on Modelling and Simulation Proceedings.~--- 
Dudweiler, Germany: Digitaldruck Pirrot GmbHP, 2013. P.~569--572.
 \end{thebibliography}

 }
 }

\end{multicols}

\vspace*{-6pt}

\hfill{\small\textit{Поступила в~редакцию 03.08.18}}

\vspace*{6pt}

%\newpage

%\vspace*{-24pt}

\hrule

\vspace*{2pt}

\hrule

\vspace*{-2pt}


\def\tit{DATA NOISING BY FINITE NORMAL AND~GAMMA MIXTURES WITH~APPLICATION 
TO~THE~PROBLEM OF~ROUNDED OBSERVATIONS}


\def\titkol{Data noising by finite normal and~gamma mixtures with~application 
to~the~problem of~rounded observations}



\def\aut{A.\,K.~Gorshenin}

\def\autkol{A.\,K.~Gorshenin}

\titel{\tit}{\aut}{\autkol}{\titkol}

\vspace*{-11pt}


\noindent
Institute of Informatics Problems, Federal Research Center ``Computer Science and
Control'' of the Russian Academy of Sciences, 44-2~Vavilov Str., Moscow 119333,
Russian Federation


\def\leftfootline{\small{\textbf{\thepage}
\hfill INFORMATIKA I EE PRIMENENIYA~--- INFORMATICS AND
APPLICATIONS\ \ \ 2018\ \ \ volume~12\ \ \ issue\ 3}
}%
 \def\rightfootline{\small{INFORMATIKA I EE PRIMENENIYA~---
INFORMATICS AND APPLICATIONS\ \ \ 2018\ \ \ volume~12\ \ \ issue\ 3
\hfill \textbf{\thepage}}}

\vspace*{3pt}



\Abste{In many real problems, statistical analysis of data containing additional 
measurement errors, including rounding, is performed, which in some situations can 
lead to sufficiently significant distortions. In this paper, estimates for an 
unknown expectation of observations are obtained for one of the possible 
rounding models under the assumption that the original data are additionally 
noised with random variables having distributions of the type of finite 
mixtures of normal and gamma laws. Confidence intervals for an 
unknown expectation are constructed using the refined estimate for 
the variance of the integer part of the random variable. An algorithm 
for determining the value of the parameter of artificial noise, which 
can be added to the initial data to improve the quality of the 
method of moving separation of mixtures, is discussed.}


\KWE{noisy data; rounded data; finite normal mixtures; finite gamma mixtures; 
confidence intervals; moving separation of mixtures}



\DOI{10.14357/19922264180304}

%\vspace*{-14pt}

\Ack
\noindent
The research was supported by the Russian Science Foundation (project 18-71-00156).



%\vspace*{6pt}

  \begin{multicols}{2}

\renewcommand{\bibname}{\protect\rmfamily References}
%\renewcommand{\bibname}{\large\protect\rm References}

{\small\frenchspacing
 {%\baselineskip=10.8pt
 \addcontentsline{toc}{section}{References}
 \begin{thebibliography}{99}
\bibitem{1-gor-1}
\Aue{Wright,~D.\,E., and I.~Bray.} 2003.
A~mixture model for rounded data.  \textit{J.~Roy. Stat. Soc.~D Sta.} 52:3--13.

\bibitem{2-gor-1}
\Aue{Bai,~Z., S.~Zheng, B.~Zhang, and G.~Hu.} 2009. 
Statistical analysis for rounded data. \textit{J.~Stat. Plan. 
Infer.} 139(8):2526--2542.

\bibitem{3-gor-1}
\Aue{Zhang,~B., T.~Liu, and Z.\,D.~Bai.} 2010. 
Analysis of rounded data from dependent sequences. 
\textit{Ann. I.~Stat. Math.} 62(6):1143--1173.

\bibitem{4-gor-1}
\Aue{Zhao,~N., and Z.~Bai.} 2012. Analysis of rounded data in mixture normal model. 
\textit{Stat. Pap.} 53:895--914.

\bibitem{5-gor-1}
\Aue{Korolev, V.\,Yu.} 2011. 
\textit{Veroyatnostno-statisticheskie metody dekompozitsii volatil'nosti 
khaoticheskikh protsessov} [Probabilistic and statistical methods of 
decomposition of volatility of chaotic processes]. 
Moscow: Moscow University Publishing House. 512~p.

\bibitem{6-gor-1}
\Aue{Ushakov, V.\,G., and N.\,G.~Ushakov.} 
2015. Ob usrednenii okruglennykh dannykh [On averaging of rounded data].
\textit{Informatika i~ee Primeneniya~--- Inform. Appl.} 9(4):106--109.

\bibitem{7-gor-1}
\Aue{Ushakov,~V.\,G., and N.\,G.~Ushakov.} 2017. 
Boundaries of the precision of restoring information lost after rounding
 the results from observations. 
 \textit{Moscow University Computational Math. Cybernetics} 41(2):76--80.

\bibitem{8-gor-1}
\Aue{Ushakov,~N.\,G., and  V.\,G.~Ushakov.} 2017. 
Statistical analysis of rounded data: Recovering of information lost due to rounding. 
\textit{J.~Korean Stat. Soc.} 46(3):426--437.

\bibitem{9-gor-1}
\Aue{Gorshenin,~A.\,K., and V.\,Yu.~Korolev.} 2016. 
A~noising method for the identification of the stochastic structure of information 
flows. \textit{Comm. Com. Inf. Sc.} 678:279--289.

\bibitem{10-gor-1}
\Aue{Gorshenin,~A., and V.~Korolev.} 2013.  Modelling of statistical fluctuations of
information flows by mixtures of gamma distributions. 
\textit{27th European Conference on Modelling and Simulation Proceedings}. 
Dudweiler, Germany: Digitaldruck Pirrot GmbHP. 569--572.

\end{thebibliography}

 }
 }

\end{multicols}

\vspace*{-6pt}

\hfill{\small\textit{Received August 3, 2018}}

%\pagebreak

%\vspace*{-18pt}

\Contrl

\noindent
\textbf{Gorshenin Andrey K.} (b.\ 1986)~--- Candidate of Science (PhD) in physics and
mathematics, associate professor, leading scientist, Institute of Informatics Problems,
Federal Research Center ``Computer Science and Control'' of the Russian Academy of
Sciences, 44-2 Vavilov Str., Moscow 119333, Russian Federation; 
\mbox{agorshenin@frccsc.ru}
\label{end\stat}

\renewcommand{\bibname}{\protect\rm Литература}       %4
\def\stat{nazarov}

\def\tit{ВЕРОЯТНОСТНАЯ МОДЕЛЬ ВЛИЯНИЯ КНИГИ ЗАКАЗОВ
НА~ПРОЦЕСС ЦЕНЫ}

\def\titkol{Вероятностная модель влияния книги заказов
на~процесс цены}

\def\aut{Е.\,В.~Быковец$^1$, В.\,В.~Лаврентьев$^2$,  Л.\,В.~Назаров$^3$}

\def\autkol{Е.\,В.~Быковец, В.\,В.~Лаврентьев,  Л.\,В.~Назаров}

\titel{\tit}{\aut}{\autkol}{\titkol}

\index{Быковец Е.\,В.}
\index{Лаврентьев В.\,В.}
\index{Назаров Л.\,В.}
\index{Nazarov L.\,V.}
\index{Lavrentyev V.\,V.}
\index{Bykovets E.\,V.}




%{\renewcommand{\thefootnote}{\fnsymbol{footnote}} \footnotetext[1]
%{Работа поддержана РНФ (проект 16-11-10227).}}


\renewcommand{\thefootnote}{\arabic{footnote}}
\footnotetext[1]{Московский государственный университет им.\ М.\,В.~Ломоносова, 
факультет вычислительной математики и~кибернетики, 
\mbox{eugene.bykovets@stud.cs.msu.su}}
\footnotetext[2]{Московский государственный университет им.\ М.\,В. Ломоносова, 
факультет вычислительной математики и~кибернетики, \mbox{lavrent@cs.msu.ru}}
\footnotetext[3]{Московский государственный университет им.\ М.\,В. Ломоносова, 
факультет вычислительной математики и~кибернетики, 
\mbox{nazarov@cs.msu.ru}}

\vspace*{-9pt}


   

\Abst{Рассматривается модель книги заказов, в~которой заказы на покупку 
и~продажу образуют два независимых процесса Кокса. Предложен механизм
        влияния поступающих заказов на цену актива на основе физической модели 
        абсолютно упругого соударения. В~этой модели цена представляет собой 
        материальную точку с~некоторой массой, движущуюся по прямой без трения. 
        Приходящие заказы на покупку и~уходящие заказы на продажу упруго 
        сталкиваются с~ней и~придают дополнительный импульс в~одном 
        направлении, а~приходящие заказы на продажу и~уходящие заказы на покупку~--- 
        в~противоположном. Получена функциональная предельная теорема для процесса 
        цены при высокой интенсивности входящего потока заказов, позволяющая 
        аппроксимировать его некоторым процессом Леви.}

\KW{лимитные заявки; абсолютно упругий удар; модель книги заказов; процесс цены; процесс Кокса; 
функциональная предельная тео\-рема}

\DOI{10.14357/19922264180205}
  
\vspace*{-3pt}


\vskip 10pt plus 9pt minus 6pt

\thispagestyle{headings}

\begin{multicols}{2}

\label{st\stat}

\section{Введение}

Рассмотрим некоторый торгуемый на бирже актив, в~отношении которого 
могут приходить два вида запросов: на покупку и~на продажу. 
Список таких запросов формирует книгу заказов для данного актива. 
Информация, содержащаяся в~книге, позволяет делать прогнозы относительно 
возможного движения цены рассматриваемого актива. 
Особенный интерес эта информация начала представлять с~развитием высокочастотной
 торговли.

В работе рассматривается модель, которая описывает влияние книги 
заказов на цену актива. Базовой моделью для исследования была выбрана модель 
книги заказов, близкая к~описанной в~\cite{first}. Основное отличие состоит
 в~следующем: на бирже торгуемый актив имеет цену, которая размещается в~узлах 
 сетки~$nh$, где $n$~--- некоторое целое число; $h$~--- тик, т.\,е.\ 
 минимальное изменение цены. Однако высокая частота узлов сетки позволяет считать, 
 что рассматриваемый актив может иметь произвольную цену, равно как и~заявки на 
 покупку и~продажу торгуемого актива. С~учетом этого рассматриваем следующую модель 
 влияния заявок на цену актива, используя физическую аналогию.  
 %
 Рассмотрим 
 материальную точку массой~$M$, которая может двигаться по прямой (числовой оси) 
 в~любом на\-прав\-ле\-нии без трения. При этом, связывая модель 
 физическую и~математическую, будем считать, что текущее положение на оси~--- 
 это текущая цена~$X(t)$.  Будем далее для краткости называть ценой и~саму 
 указанную материальную точку массой~$M$, т.\,е.\ 
 будем говорить о~ско\-рости цены, импульсе цены и~т.\,п. 
 Каждый заказ на продажу (поступающий по цене~$A_{i}$
не ниже, чем~$X(t)$) придает цене дополнительный импульс в~направлении от~$A_{i}$ 
к~$X(t)$. Заказы живут экспоненциальное время, после чего уходят из книги 
за счет исполнения или отмены. Уход заказа из книги  придает цене 
дополнительный импульс той же абсолютной величины, что и~при ее 
поступлении, но противоположного направления. С~заказами на продажу все аналогично, 
только приходят они с~ценой, не превосходящей~$X(t)$.

Цель данной работы состоит в~том, чтобы выяснить, какой процесс движения цены 
порождает такая сис\-те\-ма при интенсивном потоке приходящих заказов.
Близкая задача решалась авторами в~работе~\cite{second}, но там рас\-смат\-ри\-вал\-ся 
другой механизм влияния по\-сту\-па\-ющих заказов на цену. Здесь следует упомянуть 
и~работу~\cite{Korolev1}, в~которой также изуча\-ет\-ся связь механизма 
функционирования книги заказов на микроуровне с~процессом цены.

\vspace*{-12pt}

\section{Описание модели}

\vspace*{-4pt}

Будем рассматривать работу книги заказов на временн$\acute{\mbox{о}}$м интервале $t\hm\in[0,T]$. 
В~начальный момент времени $t\hm=0$ заказов в~книге нет. Считаем, что поток 
приходящих заказов является процессом Кокса следующего вида:
\begin{equation*}
\left\{N(\Lambda(t))=N_1(\Lambda(t)), t\geqslant0\right\}\,,
\end{equation*}
где
$N_1$~--- пуассоновский процесс, интенсивность которого равна~1; 
$\Lambda(t)$~--- стартующий из нуля случайный процесс, у~которого траектории 
являются неубывающими и~непрерывными справа функциями, а~также справедливо 
$\mathbb{P}(\Lambda(t)\hm<\infty)$.

Каждый заказ, приходящий в~книгу, находится в~ней некоторое случайное время. 
Более точно, время нахождения конкретного заказа в~книге является случайной 
величиной, распределенной по экспоненциальному закону.

Для каждого приходящего заказа определен набор параметров $(h_i, \gamma_i, \eta_i)$, 
где 
$h_i$~--- абсолютное значение разности между ценой заказа и~текущей ценой; 
$\gamma_i$~--- разность между ско\-ростью приходящего заказа и~текущей ско\-ростью цены; 
$\eta_i$~--- время пребывания заказа в~книге. 

Случайные величины $h_i$, $\gamma_i$ и~$\eta_i$, $i\hm=1, 2, \ldots,$ независимы 
в~совокупности и~не зависят от потока заявок, a~$\eta_i$ распределены 
экспоненциально с~параметром~$\mu$. Параметр~$\gamma_i$ фактически определяет 
тип заказа. Для заказов на покупку~$\gamma_i$ положительны, для заказов на 
продажу~--- отрицательны.

В качестве физической модели влияния заказа на цену возьмем модель абсолютно 
упругого удара.  Считаем, что $i$-й заказ, поступающий в~момент~$t$,~--- 
это материальная точка массой $m_0(h_i)\hm>0$, которая движется по той же прямой, 
что и~цена, и~имеет в~момент времени~$t$ скорость, равную $u_i \hm= v_{i-1}\hm+\gamma_i$, 
где~$v_{i-1}$~--- скорость цены до столкновения с~$i$-м заказом. В~момент~$t$ 
происходит их упругое соударение. На распределения~$h_i$ и~$\gamma_i$ 
наложим следующие ограничения:
\begin{equation}
\mathbb{E}\gamma_i = 0;\enskip
\mathbb{E}\gamma_i^2 = \overline{\gamma}<\infty;\enskip
\mathbb{E}m_0(h_i)^2 = \overline{m} < \infty\,.
\label{e1-naz}
\end{equation}
Смысл первого условия заключается в~том, что разности между скоростями 
приходящих заказов и~текущей скоростью актива для заказов на покупку и~продажу 
в~среднем равны. Остальные ограничения имеют технический характер и~лишь постулируют 
конечность соответствующих моментов.

Функция $m_0$ является убывающей на интервале~$(0, \infty)$, поскольку 
воздействие заказа на цену тем больше, чем ближе его цена к~текущей цене актива.
Это соответствует реальному положению дел на рынке, где заказы на уровнях, 
близких к~текущей цене, выставляются более ответственно, так как могут быть 
тут же удовлетворены. В~то же время заказы на более удаленных уровнях чаще 
ставятся для дезориентации других участников  рынка и~снимаются\linebreak\vspace*{-12pt}

\columnbreak

\noindent
 при приближении к~ним 
цены. Иными словами, они не отражают реальный спрос. 

\vspace*{-7pt}

\section{Процесс цены}

Рассмотрим точку на числовой прямой, которая представляет собой текущую цену 
актива. Положим 
$M$~--- масса точки; 
$v_{i}$~--- текущая скорость цены, полученная после удара $i$-й частицы, 
полагаем $v_ {0}=0$; 
$u_{i}$~--- ско\-рость $i$-го заказа до соударения; 
$m_0(h_i)$~--- масса $i$-го заказа. 
Как было сказано в~предыду\-щем разделе, данная точ\-ка (исследуемая цена) 
в~определенные моменты времени абсолютно упруго соударяется с~другими 
частицами (заказами). В~этом случае есть возможность выразить ско\-рость точки после 
удара $i$-й час\-ти\-цы через массу точ\-ки и~массу час\-ти\-цы, а~также их ско\-рости 
до столк\-но\-ве\-ния (это следует из закона сохранения импульса и~закона сохранения 
энергии, см.~\cite[гл.~4, \S\,28]{third}):
\begin{equation*}
v_i = -v_{i-1} +2\fr{Mv_{i-1}+ m_0(h_i)u_i}{M+ m_0(h_i)}\,.
\end{equation*}
Обозначим $\Delta v_i \hm= v_i - v_{i-1}$, тогда

\noindent
\begin{multline*}
\Delta v_i = -2v_{i-1} +2\fr{Mv_{i-1}+ m_0\left(h_i\right)u_i}{M+ m_0\left(h_i\right)} = {}\\
{}=
\fr{2 m_0\left(h_i\right)}{M+ m_0\left(h_i\right)}\left(u_{i}-v_{i-1}\right).
\end{multline*}
Фактически $\Delta v_i$ показывает изменение ско\-рости цены после соударения 
с~$i$-м заказом.

Пусть в~начальный момент книга заказов пус\-та, а~начальная скорость $v_0 \hm= 0$. 
Далее полагаем, что заказ с~номером~$i$ приходит в~момент времени~$\tau_{i0}$ и~уходит 
в~момент времени~$\tau_{i1}$. В~итоге получаем, что скорость цены является случайным 
процессом $\{V(t), t\hm\geqslant0\}$ с~ку\-соч\-но-по\-сто\-ян\-ны\-ми траекто\-риями:

\noindent
\begin{equation*}
V(t) = \sum\limits_{i=1}^{N_1(\Lambda(t))}\Delta v_i\mathbb{I}_{\{\tau_{i0}\le t\le 
\tau_{i1}\}}(t)\,.
\end{equation*}
Тогда изменение цены за время~$T$ будет иметь вид:
\begin{equation}
X(T) = \int\limits_{0}^{{T}} V(t)\, dt = \sum\limits_{i=1}^{N_1(\Lambda(T))}X_i(T)\,,
\label{e2-naz}
\end{equation}
где $X_i(T)$~--- изменение цены на интервале [0,T] за счет удара $i$-го заказа:

\vspace*{-2pt}

\noindent
\begin{multline*}
X_i(T) = \int\limits_{0}^{{T}} \Delta v_i\mathbb{I}_{\{\tau_{i0}\leqslant 
t\leqslant \tau_{i1}\}}(t)\,dt ={}\\
{}= \fr{2 m_0(h_i)}{M+ m_0(h_i)}
\int\limits_{0}^{{T}}(u_{i}-v_{i-1})\mathbb{I}_{\{\tau_{i0}\leqslant t
\leqslant \tau_{i1}\}}(t)\,dt\,.
\end{multline*}
Поскольку по определению $u_{i}\hm-v_{i-1} \hm= \gamma_i$ , то последнее 
выражение можем переписать в~виде ($a \wedge b\hm = \min(a,b)$):
\begin{multline*}
X_i(T)=\fr{2 m_0(h_i)\gamma_i}{M+ m_0(h_i)}\int\limits_{0}^{{T}}
\mathbb{I}_{\{\tau_{i0}\leqslant t\leqslant \tau_{i1}\}}(t)\,dt = {}\\
{}=
\fr{2 m_0(h_i)\gamma_i}{M+ m_0(h_i)}\left(T\wedge\tau_{i1}-
T\wedge\tau_{i0}\right) ={}\\
{}=\fr{2 m_0(h_i)\gamma_i}{M+ m_0(h_i)}
\left(T\wedge\left(\tau_{i0}+\eta_i\right)-T\wedge\tau_{i0}\right).
\end{multline*}
Строго говоря, случайные величины $\{X_i(T)$, $i\hm=1,2,\dots\}$ не являются 
независимыми, но суммы в~(\ref{e2-naz}) можно представить в~виде сумм 
независимых случайных величин. 

Рассмотрим распределение вектора 
моментов прихода заявок $\tau_0\hm=\{\tau_{10},\ldots ,\tau_{n0}\}$. По свойству 
пуассоновского потока при $N_1(\Lambda(T))\hm=n$ распределение~$\tau_0$ 
есть распределение вариационного ряда выборки из~$n$~независимых случайных 
величин,\linebreak равномерно распределенных на $[0, \Lambda(T)]$.
Поскольку значение конечной суммы при перестановке\linebreak слагаемых не меняется, 
далее будем считать, что в~каждой из сумм~(\ref{e2-naz})~$\tau_{i0}$ 
независимы и~равномерно распределены на $[0, \Lambda(T)]$, а~следовательно, 
случайные величины~$\{X_i(T)$, $i\hm=1,2,\dots\}$ также независимы.

Изучим асимптотические свойства моментов~$X_i(T)$.

\smallskip

\noindent
\textbf{Лемма~1.}\ \textit{ Пусть случайная величина~$\xi$ 
равномерно распределена на $[0,T]$, $\eta_0$ не зависит от~$\xi$ и~имеет 
экспоненциальное распределение с~параметром~$\mu$ и}
\begin{equation*}
s = T\wedge\left(\xi+\eta_0\right)-\xi\,.
\end{equation*}

\vspace*{-8pt}

\noindent
\textit{Тогда}
\begin{enumerate}[(1)]
\item $s\stackrel{d}=\xi\wedge\eta_0$;
\item \textit{моменты случайной величины s обладают следующими асимптотическими 
свойствами}: 
\begin{equation*}
\lim\limits_{\mu\rightarrow\infty}\mu\mathbb{E}s = 1;\
 \lim\limits_{\mu\rightarrow\infty}\mu^2\mathbb{E}s^2 = 2;\
  \lim\limits_{\mu\rightarrow\infty}\mu^2\mathbb{D}s = 1.
\end{equation*}
\end{enumerate}

\noindent
{Д\,о\,к\,а\,з\,а\,т\,е\,л\,ь\,с\,т\,в\,о\,.}\ \ 
Вычислим математическое ожидание случайной величины~$s$ с~учетом независимости~$\xi$ 
и~$\eta_0$. По определению
\begin{equation*}
s = T\wedge\left(\xi+\eta_0\right)-\xi = (T-\xi)\wedge\eta_0\,.
\end{equation*}
Справедливость первого утверждения леммы следует из независимости~$\xi$ и~$\eta_0$ 
и~одинаковой распределенности~$\xi$ и~$T\hm-\xi$. Таким образом, математическое 
ожидание~$s$ есть
\begin{equation*}
\mathbb{E}s  = \mathbb{E}\left(\xi\wedge\eta_0\right).
\end{equation*}
При вычислении моментов неоднократно будет требоваться значение интеграла
\begin{equation*}
\int\limits_{0}^{{T}}y^ne^{-\mu y}\,dy = \fr{n!}{\mu^{n+1}}\,F_{n+1}(T)\,,
\end{equation*}
где $F_{n+1}$~--- 
функция распределения Эрланга $(n+1)$-го порядка:
$$
F_{n+1}(x) = 1 - e^{-\mu x}\sum\limits_{i=1}^{n}\fr{\mu^iT^i}{i!}\,.
$$ 
Вычислим 
$\mathbb{E}(\xi\wedge\eta_0)$:
\begin{multline*}
\mathbb{E}\left(\xi\wedge\eta_0\right) =
\fr{\mu}{T}\int\limits_{0}^{\infty}\!e^{-\mu y}\,dy\int\limits_{0}^{T}(x\wedge y)\,dx 
 ={}\\
 {}=\fr{\mu}{T}\int\limits_{0}^{\infty}\!e^{-\mu y}\,dy\left\{
 \mathbb{I}_{\{y<T\}}\left[\int\limits_{0}^{y}x\,dx+
 \int\limits_{y}^{T}y\,dx\right] + {}\right.\\
\left. {}+
 \mathbb{I}_{\{y\geqslant T\}}\int\limits_{0}^{T}x\,dx\right\} = \fr{\mu}{T}\left[T\int\limits_{0}^{T}\!ye^{-\mu y}\,dy -{}\right.\\
\left.{}-
\fr{1}{2}\int\limits_{0}^{T}\!y^2e^{-\mu y}\,dy + 
\fr{T^2}{2}\int\limits_{T}^{\infty}\!e^{-\mu y}\,dy \right]={} \\
{}=\left[\fr{1}{\mu}-\fr{1}{\mu^2 T}\right]+\left[\fr{T}{2}+\fr{1}{\mu^2 T}\right]
e^{-\mu T}.
\end{multline*}
И,~соответственно,
\begin{multline*}
\lim\limits_{\mu \rightarrow \infty}\mu\mathbb{E}s ={}\\
{}= 
\lim\limits_{\mu \rightarrow \infty}\left\{\left[1-\fr{1}{\mu T}\right]+
\left[\fr{T}{2}+\fr{1}{\mu^2 T}\right]\mu e^{-\mu T}\right\}=1. 
\end{multline*}
Вычислим $\mathbb{E}s^2\hm=\mathbb{E}(\xi\wedge\eta_0)^2$:
\begin{multline*}
\mathbb{E}\left(\xi\wedge\eta_0\right)^2=
\fr{\mu}{T}\int\limits_{0}^{\infty}\!e^{-\mu y}\,dy
\int\limits_{0}^{T}\left(x\wedge y\right)^2\,dx = {}\\
{}=\fr{\mu}{T}\int\limits_{0}^{\infty}\!e^{-\mu y}\,dy\left\{
\mathbb{I}_{\{y<T\}}\left[{\int\limits_{0}^{y}\!x^2\,dx+
\int\limits_{y}^{T}\!y^2\,dx}\right] + {}\right.\\
\left.{}+
\mathbb{I}_{\{y\geqslant T\}}\int\limits_{0}^{T}\!x^2\,dx\right\}= 
\fr{\mu}{T}\left[T\int\limits_{0}^{T}\!y^2e^{-\mu y}\,dy -{}\right.\\
\left.{}-
\fr{2}{3}\int\limits_{0}^{T}\!y^3e^{-\mu y}dy + 
\fr{T^3}{3}\int\limits_{T}^{\infty}\!e^{-\mu y}dy \right]= {}\\
{}=\left[\fr{2}{\mu^2}-\fr{4}{\mu^3 T}\right]+\left[
\fr{2}{\mu^2}+\fr{4}{\mu^3 T}\right]e^{-\mu T}.
\end{multline*}
Отсюда получаем асимптотику второго момента
\begin{multline*}
\lim\limits_{\mu \rightarrow \infty}\mu^2\mathbb{E}s^2 = {}\\
{}=
\lim\limits_{\mu \rightarrow \infty}\left\{\left[2-\fr{4}{T\mu}\right]+
\left[\fr{2}{\mu^2}+\fr{4}{\mu^3 T}\right]\mu^2e^{-\mu T}\right\} = 2
\end{multline*}
и дисперсии
\begin{multline*}
\lim\limits_{\mu \rightarrow \infty}\mu^2\mathbb{D}s =  
\lim\limits_{\mu \rightarrow \infty}\mu^2\left[\mathbb{E}s^2-(\mathbb{E}s)^2\right] ={}\\
{}=
 \lim\limits_{\mu \rightarrow \infty}\mu^2\mathbb{E}s^2 - 
  \lim\limits_{\mu \rightarrow \infty}(\mu\mathbb{E}s)^2 =1\,.
\end{multline*}
Утверждение леммы доказано.

\smallskip

Рассмотрим следующую последовательность:
\begin{equation}
\left\{X_{n}(t) = \sum\limits_{i=1}^{N_1(\Lambda_n(t))}X_{ni}(t),\ 
 t \in [0,T]\right\}.
 \label{e3-naz}
\end{equation}
При этом каждому члену~$\{X_n\}$ соответствует процесс $\{\Lambda_n(t)$, 
$t\hm\in [0,T]\}$, параметр~$\mu_n$ и~функция массы ударяющей частицы (приходящего 
заказа)~$m_{n0}$. При увеличении~$n$ будем увеличивать интенсивность входящего 
потока заявок~$\Lambda_n(t)$ и~уменьшать время пребывания заказов в~книге 
посредством увеличения~$\mu_n$ ($\Lambda_n(t)\hm\Rightarrow\infty$, 
$\mu_n\hm\rightarrow\infty$ при $n \hm\rightarrow \infty$). Будем также 
уменьшать влияние отдельного заказа на цену:
\begin{equation*}
m_{n0} = \alpha_n m_0,\enskip
 \alpha_n >0\,,\enskip
  \alpha_n \rightarrow 0\,,\enskip
   n\rightarrow\infty\,.
\end{equation*}
Получим асимптотические свойства моментов случайных величин $X_{n1}(T)$ 
при установленных параметрических зависимостях. Аргумент~$T$ у~них одинаков и~для 
краткости будем его опускать.

\smallskip

\noindent
\textbf{Лемма~2.}\  \textit{Пусть $\mu_n\hm \rightarrow \infty$, 
$\alpha_n \hm\rightarrow 0$, $k_n \hm= {\mu_n^2}/{\alpha_n^2}$.  Тогда}
\begin{enumerate}[(1)]
\item $k_n\mathbb{E}X_{n1}\hm\rightarrow 0$, 
$k_n\mathbb{D}X_{n1}\hm\rightarrow {8\overline{m}\overline{\gamma}}/{M^2}$,
$n\hm\rightarrow\infty$;
\item \textit{Выполняется условие Линдеберга, т.\,е.\ для любого} $\varepsilon \hm>0$
\begin{equation*}
\lim\limits_{n\rightarrow\infty}k_n
\mathbb{E}\left[X_{n1}^2\mathbb{I}(|X_{n1}|>\varepsilon)\right]=0\,.
\end{equation*}
\end{enumerate}

\noindent
Д\,о\,к\,а\,з\,а\,т\,е\,л\,ь\,с\,т\,в\,о\,.\ \
Как было показано выше, 
$$
X_{n1} = \fr{2 m_{n0}(h_1)\gamma_{1}}{M+ m_{n0}(h_1)}s_n\,,
$$
 где $s_n\stackrel{d} =\xi\wedge\eta_{n0}$ и~$\eta_{n0}$ распределена 
 экспоненциально с~параметром~$\mu_n$, а~величины $h_1$, $\xi$, $\eta_{n0}$
и~$\gamma_{1}$ независимы. Так как в~соответствии с~(\ref{e1-naz}) 
 $\mathbb{E}\gamma_1 \hm= 0$, то $k_n\mathbb{E}X_{n1}\hm=0$ для любого~$n$. 
 Проверим соотношение для дисперсии:
 
 \noindent
\begin{multline*}
k_n\mathbb{D}X_{n1} = \fr{\mu_n^2}{\alpha_n^2}\, \mathbb{E} 
\left[\fr{2 m_{n0}(h_1)\gamma_{1}}{M+ m_{n0}(h_1)}\,s_n\right]^2={}\\
{}=4\mathbb{E}\gamma_{1}^2\mathbb{E}\left[
\fr{m_{n0}(h_1)}{\alpha_n}\,\fr{1}{M+m_{n0}(h_1)}\right]^2\mu_n^2\mathbb{E}s_n^2 = {}\\
{}=
4\overline{\gamma}\mathbb{E}\left[\fr{m_0(h_1)}{M+m_{n0}(h_1)}\right]^2\mu_n^2
\mathbb{E}s_n^2.
\end{multline*}

Последовательность случайных величин 
$\{[{m_0(h_1)}/({M+m_{n0}(h_1)})]^2\}$ мажорируется интегрируемой случайной 
величиной $[{m_0(h_1)}/{M}]^2$ и~поточечно сходится к~ней, так как $\alpha_n 
\hm\rightarrow 0$, $n\hm\rightarrow\infty$. Поэтому

\vspace*{-6pt}

\noindent
\begin{multline*}
\lim\limits_{n\rightarrow\infty}k_n\mathbb{D}X_{n1} =  
\lim\limits_{n\rightarrow\infty}4\overline{\gamma}\mathbb{E}
\left[\fr{m_0(h_1)}{M}\right]^2\mu_n^2\mathbb{E}s_n^2 ={}\\
{}=
\fr{4\overline{m}\overline{\gamma}}{M^2}
\lim\limits_{n\rightarrow\infty}\mu_n^2\mathbb{E}s_n^2 = 
\fr{8\overline{m}\overline{\gamma}}{M^2}\,.
%\label{e4-naz}
\end{multline*}

Докажем справедливость условия Линдеберга. Рассмотрим

\vspace*{-6pt}

\noindent
\begin{multline*}
\fr{\mu_n}{\alpha_n}\left\vert X_{n1}\right\vert = 
 \fr{\mu_n}{\alpha_n}\left\vert \fr{2 m_{n0}(h_1)\gamma_{1}}{M+ m_{n0}
 \left(h_1\right)}\,s_n\right\vert = {}\\
 {}=
 2\fr{m_{n0}(h_1)}{\alpha_n}\,\fr{1}{M+ m_{n0}(h_1)}\left\vert \gamma_1\right\vert
 \mu_ns_n \leqslant {}\\
{}\leqslant 2\fr{m_0(h_1)}{M}\left\vert \gamma_1\right\vert \mu_ns_n
\leqslant {}\\
{}\leqslant \fr{2m_0(h_1)|\gamma_1|\mu_n\eta_{n0}}{M} \stackrel{d}= 
\fr{2m_0(h_1)|\gamma_1|\hat{\eta}}{M}\,,
\end{multline*}
где $\hat{\eta}$ распределена экспоненциально с~па\-ра\-мет\-ром~1. 
Распределение случайной величины 
$$
Y_n =   \fr{{2m_0(h_1)|\gamma_1|\mu_n\eta_{n0}}}{M}
$$ 
не зависит от~$n$ и~согласно~(1) имеет конечный второй момент, поэтому

\noindent
\begin{multline*}
k_n\mathbb{E}\left[X_{n1}^2\mathbb{I}(|X_{n1}|>\varepsilon)\right] ={}\\
{}= 
k_n\mathbb{E}\left[X_{n1}^2\mathbb{I}(\sqrt{k_n}|X_{n1}|>
\sqrt{k_n}\varepsilon)\right] \leqslant {} \\
{}\leqslant \mathbb{E}\left[Y_{n}^2\mathbb{I}(|Y_{n}|>
\sqrt{k_n}\varepsilon)\right]
= \mathbb{E}\left[Y_{1}^2\mathbb{I}(|Y_{1}|>
\sqrt{k_n}\varepsilon)\right].\hspace*{-3.79228pt}
\end{multline*}
Последнее математическое ожидание стремится к~нулю при $n\hm\rightarrow\infty$ по 
тео\-ре\-ме Лебега о~ма\-жо\-ри\-ру\-емой сходимости. Утверждение леммы доказано.

\vspace*{2pt}

Сформулируем доказанную в~работе~\cite{fourth} функ\-циональную центральную 
предельную тео\-ре\-му, устанавливающую условия, при которых процессы вида~(\ref{e3-naz}) 
сходятся к~некоторому предельному процессу~$X$ в~про\-стран\-ст\-ве Скорохода 
$\mathcal{D}\hm = \mathit{(D[0,1], d_0)}$ (см.~\cite[гл.~3]{five}). 
Позднее были получены более сильные результаты, касающиеся схо\-ди\-мости обобщенных 
процессов Кокса~(\ref{e3-naz}) (см., на\-при\-мер,~\cite{Korolev_FLT}), 
но достаточно будет приводимого ниже утверж\-де\-ния.
{ %\looseness=1

}

\smallskip

\noindent
\textbf{Теорема}~\cite{fourth}.\ 
\textit{Пусть для некоторой неограниченно возрастающей последовательности 
чисел~$\{k_n\}_{n\geqslant1}$ выполнены условия}:
\begin{enumerate}[(1)]
\item \textit{существуют числа $a \hm\in \mathbb{R}$ и~$\sigma \hm> 0$ такие, что}
\begin{equation*}
k_n\mathbb{E}X_{n1} \rightarrow a;\enskip
 k_n\mathbb{D}X_{n1} \rightarrow \sigma^2 (n\rightarrow \infty);
\end{equation*}
\item \textit{условие Линдеберга, т.\,е.\ для любого} $\varepsilon\hm>0$
\begin{equation*}
\lim\limits_{n\rightarrow\infty}k_n\mathbb{E}\left[(X_{n1}-a_n)^2\mathbb{I}
\left(\left\vert X_{n1}-a_n\right\vert >\varepsilon\right)\right]=0\,,
\end{equation*}
\textit{где $\mathbb{I}(A)$~--- индикатор события}~$A$; $a_n \hm= \mathbb{E}X_{n1}$;
\item \textit{существует безгранично делимая случайная величина~$U$ такая, 
что $\mathbb{P}(U=0) \hm< 1$, $\mathbb{P}(U\geqslant0) \hm= 1$, 
$\mathbb{E}U^2 \hm< \infty$ и}
\begin{equation*}
k_n^{-1}\Lambda_n(1)\Rightarrow U,n\rightarrow\infty;
\end{equation*}
\item
$\displaystyle \sup\limits_nk_n^{-2}\mathbb{E}\Lambda_n(1)^2<\infty.
$
\end{enumerate}
\textit{Тогда обобщенные процессы Кокса $\{X_n\}$ 
слабо сходятся в~пространстве Скорохода~$\mathcal{D}$ к~процессу Леви~$X$ такому, что}
\begin{equation*}
X(1) \stackrel{d}=\sigma\sqrt{U}N(0,1)+aU\,,
\end{equation*}
\textit{где $N(0,1)$~--- случайная величина, имеющая стандартное нормальное 
распределение, при этом не зависящая от}~$U$.

Для последовательности $\{k_n = {\mu_n^2}/{\alpha_n^2}\}$ при $a\hm=0$ первые 
два условия теоремы выполняются по лемме~2. Таким образом, достаточно наложить 
определенные условия на ин\-тен\-сив\-ность входящего потока заявок, чтобы была 
справедлива сле\-ду\-ющая теорема.

\smallskip

\noindent
\textbf{Теорема 1.}  \textit{Пусть $\mu_n \hm\rightarrow \infty$, 
$\alpha_n \hm\rightarrow 0$, $k_n = {\mu_n^2}/{\alpha_n^2}$, 
$\sup_nk_n^{-2}\mathbb{E}\Lambda_n(1)^2\hm<\infty$ и~существует 
безгранично делимая случайная величина~$U$ такая, что}
\begin{equation*}
\mathbb{P}(U=0) < 1\,;\enskip
\mathbb{P}(U\geqslant0) = 1\,;\enskip
 \mathbb{E}U^2 < \infty
\end{equation*}
и

\noindent
\begin{equation*}
k_n^{-1}\Lambda_n(1)\Rightarrow U\,,\enskip n\rightarrow\infty\,.
\end{equation*}
\textit{Тогда обобщенные процессы Кокса~$\{X_n\}$ слабо сходятся в~пространстве 
Скорохода $\mathcal{D}$ к~процессу Леви~$X$ такому, что}
\begin{equation*}
X(1) \stackrel{d}=\sigma\sqrt{U}N(0,1)\,,
\end{equation*}

%\columnbreak

\noindent
\textit{где $\sigma = {8\overline{m}\overline{\gamma}}/{M^2}$, а $N(0,1)$~--- 
случайная величина со стандартным нормальным распределением, независимая от}~$U$. 

%\vspace*{-24pt}

\section{Заключение}

В настоящей работе была предложена модель механизма влияния по\-сту\-па\-ющих 
заказов на цену актива на основе физической модели абсолютно упругого 
соударения час\-тиц. 

Была установлена справедливость функциональной предельной 
тео\-ре\-мы, на основании результатов которой можно аппроксимировать процесс 
цены при интенсивном потоке приходящих заявок процессом Леви, приращения 
которого являются смесью нормальных законов и~поддаются более точному анализу. 
Такая аппроксимация дает также воз\-мож\-ность оценки риска динамических
 стратегий~\cite{six}.


%\vspace*{-48pt}

    {\small\frenchspacing
 {%\baselineskip=10.8pt
 \addcontentsline{toc}{section}{References}
 \begin{thebibliography}{9}
\bibitem{first} 
\Au{Kukanov A.} Stochastic models of limit order markets.~--- 
Columbia University, 2013. Ph.D. Thesis. 131~p.

\bibitem{second} 
\Au{Лаврентьев В.\,В., Назаров~Л.\,В.} 
Процесс движения цены, порожденный непрерывной моделью книги заказов~// 
Вестн. Тверского государственного ун-та. Сер. Прикладная математика, 2015. 
№~4. С.~55--63.

\bibitem{Korolev1}
\Au{Korolev V.\,Yu., Chertok~A.\,V., Korchagin~A.\,Yu, Zeifman~A.\,I.} 
Modeling high-frequency order flow imbalance by functional limit theorems for 
two-sided risk processes~// Appl. Math. Comput., 2015. Vol.~253. P.~224--241.

\bibitem{third} 
\Au{Сивухин Д.\,В.} Общий курс физики.~--- 
В~5 т.~--- Т.~1. Механика.~--- 4-е изд.~--- М.: МФТИ, 2005. 560~с.

\bibitem{fourth} 
\Au{Кащеев Д.\,Е.} Моделирование динамики финансовых временных рядов и~оценивание 
производных ценных бумаг: Дис.\ \ldots\ канд. физ.-мат. наук.~--- 
Тверь: ТвГУ, 2001. 191~c.

\bibitem{five} 
\Au{Биллингсли П.} Сходимость вероятностных мер~/
Пер. с~англ.~--- М.: Наука, 1977. 353~с.
(\Au{Billingsley~P.}  
{Convergence of probability measures}.~--- New York, NY, USA: John Wiley \& Sons, Inc., 
1977. 277~p.)

\bibitem{Korolev_FLT} 
\Au{Korolev V.\,Yu., Chertok~A.\,V., Korchagin~A.\,Yu, Kossova~E.\,V., Zeifman~A.\,I.} 
A~note on functional limit theorems for compound Cox processes~// 
J.~Math. Sci., 2016. Vol.~218. No.\,2. P.~182--194.

\bibitem{six} 
\Au{Balasanov~Y., Doynikov~A., Lavrent'ev~V., Nazarov~L.} 
Estimating risk of dynamic trading strategies from high frequency data flow~// 
Advances in data mining: Applications and theoretical aspects~/
 Ed.\ P.~Perner.~---
Lecture notes in computer science ser.~--- Springer, 2015.  
 Vol.~9165. P.~153--165.
 \end{thebibliography}

 }
 }

\end{multicols}

\vspace*{-3pt}

\hfill{\small\textit{Поступила в~редакцию 07.12.17}}

%\vspace*{6pt}

\newpage

\vspace*{-28pt}

%\hrule

%\vspace*{2pt}

%\hrule

%\vspace*{8pt}


\def\tit{A~PROBABILITY MODEL OF~THE~INFLUENCE\\ OF~THE~ORDER BOOK ON~THE~PRICE PROCESS}

\def\titkol{A probability model of the influence of the order book on the price process}

\def\aut{L.\,V.~Nazarov, V.\,V.~Lavrentyev, and~E.\,V.~Bykovets}

\def\autkol{L.\,V.~Nazarov, V.\,V.~Lavrentyev, and~E.\,V.~Bykovets}

\titel{\tit}{\aut}{\autkol}{\titkol}

\vspace*{-9pt}


\noindent
Faculty of Computational Mathematics and Cybernetics, 
M.\,V.~Lomonosov Moscow State University, 1-52~Leninskiye Gory, GSP-1, Moscow 119991, 
Russian Federation 


\def\leftfootline{\small{\textbf{\thepage}
\hfill INFORMATIKA I EE PRIMENENIYA~--- INFORMATICS AND
APPLICATIONS\ \ \ 2018\ \ \ volume~12\ \ \ issue\ 2}
}%
 \def\rightfootline{\small{INFORMATIKA I EE PRIMENENIYA~---
INFORMATICS AND APPLICATIONS\ \ \ 2018\ \ \ volume~12\ \ \ issue\ 2
\hfill \textbf{\thepage}}}

\vspace*{3pt} 
 


\Abste{The Limit Order Book model is considered, with buy and sell orders arriving 
as two independent Cox processes. It includes the price impact model built on the 
basis of a physical model of perfectly elastic collision. Price is treated as 
a~particle of some mass, moving along a~straight line without friction. The 
incoming buy orders and outgoing sell orders hit the price giving it additional 
momentum in one direction, while incoming sell orders and outgoing buy orders do 
the same in the opposite direction. A~functional limit theorem for the price 
process is obtained at a~high intensity 
of incoming order flow, which allows approximation by some L$\acute{\mbox{e}}$vy process}

\KWE{limit orders; perfectly elastic collision; limit order book model; 
price process; Cox process; functional limit theorem}

 
\DOI{10.14357/19922264180205} %

%\vspace*{-14pt}

  %\Ack
  % \noindent
  


%\vspace*{-3pt}

  \begin{multicols}{2}

\renewcommand{\bibname}{\protect\rmfamily References}
%\renewcommand{\bibname}{\large\protect\rm References}

{\small\frenchspacing
 {%\baselineskip=10.8pt
 \addcontentsline{toc}{section}{References}
 \begin{thebibliography}{9}

\bibitem{1-naz}
\Aue{Kukanov, A.} 2013. Stochastic models of limit order markets. 
Columbia University. Ph.D. Thesis.  131~p.

\bibitem{2-naz}
\Aue{Lavrent'ev, V.\,V., and L.\,V.~Nazarov.} 2015. Protsess dvizheniya tseny, 
porozhdennyy nepreryvnoy model'yu knigi zakazov 
[Price process, generated by the continuous model of the order book]. 
\textit{Vestnik Tverskogo gosudarstvennogo un-ta. Ser. 
Prikladnaya matematika} [Bull. of the Tverskoy State University. Ser. 
Appl. Math.] 4:55--63.

\bibitem{3-naz}
\Aue{Korolev, V.\,Yu., A.\,V.~Chertok, A.\,Yu.~Korchagin, and A.\,I.~Zeifman.} 
2015. Modeling high-frequency order flow imbalance by functional limit theorems
 for two-sided risk processes. \textit{Appl. Math. Comput.} 253:224--241.

\bibitem{4-naz}
\Aue{Sivukhin, D.\,V.} 2005. 
\textit{Obshchiy kurs fiziki. Mekhanika}
[General course of physics. Mechanics].
4~ed. Moscow: MIPT Publs. Vol.~1.  560~p. 

\bibitem{5-naz}
\Aue{Kashcheev, D.\,E.} 2001. Modelirovanie dinamiki finansovykh vremennykh ryadov
 i~otsenivanie proizvodnykh tsennykh bumag [Modeling of dynamics of financial time series and 
 estimation of derivative securities].  
 Tver'. PhD Thesis. 191~p.

\bibitem{6-naz}
\Aue{Billingsley, P.} 1977. 
\textit{Convergence of probability measures}. New York, NY: John Wiley \& Sons, Inc. 
277~p.

\bibitem{7-naz}
\Aue{Korolev, V.\,Yu., A.\,V.~Chertok, A.\,Yu.~Korchagin, E.\,V.~Kossova, 
and A.\,I.~Zeifman.} 2016. 
A~note on functional limit theorems for compound Cox processes. 
\textit{J.~Math. Sci.} 218(2):182--194. 

\bibitem{8-naz}
\Aue{Balasanov, Y., A.~Doynikov, V.~Lavrent'ev, and L.~Nazarov}. 
2015. Estimating risk of dynamic trading strategies from high frequency data flow.
\textit{Advances in data mining: Applications and theoretical aspects.} 
Ed.\ P.~Perner.  Lecture notes in computer science ser.  
Springer. 9165:153--165.
\end{thebibliography}

 }
 }

\end{multicols}

\vspace*{-3pt}

\hfill{\small\textit{Received December 7, 2017}}

%\vspace*{-24pt}




\Contr

\noindent
\textbf{Bykovets Eugene V.} (b.\ 1994)~--- MSc student,  
Faculty of Computational Mathematics and Cybernetics, M.\,V.~Lomonosov Moscow 
State University, 1-52~Leninskiye Gory, GSP-1, Moscow 119991, Russian Federation; 
\mbox{eugene.bykovets@stud.cs.msu.su}

\vspace*{3pt}

\noindent
\textbf{Lavrentyev Victor V.} (b.\ 1955)~---  
Candidate of Science (PhD) in physics and mathematics, scientist, 
Faculty of Computational Mathematics and Cybernetics, M.\,V.~Lomonosov Moscow 
State University, 1-52~Leninskiye Gory, GSP-1, Moscow 119991, Russian Federation; 
\mbox{lavrent@cs.msu.ru}

\vspace*{3pt}

\noindent
\textbf{Nazarov Leonid V.} (b.\ 1957)~--- 
Candidate of Science (PhD) in physics and mathematics, senior scientist, 
Faculty of Computational Mathematics and Cybernetics, M.\,V.~Lomonosov Moscow 
State University, 1-52~Leninskiye Gory, GSP-1, Moscow 119991, Russian Federation; 
\mbox{nazarov@cs.msu.ru}
\label{end\stat}


\renewcommand{\bibname}{\protect\rm Литература}  %5

\def\stat{sopin}

\def\tit{АНАЛИЗ МЕХАНИЗМОВ НАРЕЗКИ СЕТИ С УЧЕТОМ ГАРАНТИЙ ДЛЯ РАЗЛИЧНЫХ 
ТИПОВ ТРАФИКА$^*$}

\def\titkol{Анализ механизмов нарезки сети с~учетом гарантий для различных 
типов трафика}

\def\aut{К.\,А.~Агеев$^1$, Э.\,С.~Сопин$^2$, Н.\,В.~Яркина$^3$, 
К.\,Е.~Самуйлов$^4$, С.\,Я.~Шоргин$^5$}

\def\autkol{К.\,А.~Агеев, Э.\,С.~Сопин, Н.\,В.~Яркина и др.} 
%К.\,Е.~Самуйлов$^4$, С.\,Я.~Шоргин$^5$ и~др.}

\titel{\tit}{\aut}{\autkol}{\titkol}

\index{Агеев К.\,А.}
\index{Сопин Э.\,С.}
\index{Яркина Н.\,В.} 
\index{Самуйлов К.\,Е.}
\index{Шоргин С.\,Я.}
\index{Ageev K.\,A.}
\index{Sopin E.\,S.}
\index{Yarkina N.\,V.}
\index{Samouylov K.\,Е.}
\index{Shorgin S.\,Ya.}
 

{\renewcommand{\thefootnote}{\fnsymbol{footnote}} \footnotetext[1]
{Исследование выполнено при поддержке Программы РУДН <<5-100>> и~при частичной финансовой 
поддержке РФФИ в~рамках научных проектов №\,19-07-00933 и~№\,19-37-90147.}}


\renewcommand{\thefootnote}{\arabic{footnote}}
\footnotetext[1]{Российский университет дружбы народов, ageev-ka@rudn.ru}
\footnotetext[2]{Российский университет дружбы народов; Институт проблем информатики Федерального исследовательского 
центра <<Информатика и~управ\-ле\-ние>> Российской академии наук, \mbox{sopin-es@rudn.ru}}
\footnotetext[3]{Российский университет дружбы народов, yarkina-nv@rudn.ru}
\footnotetext[4]{Российский университет дружбы народов; Институт проб\-лем информатики Федерального 
исследовательского центра <<Информатика и~управ\-ле\-ние>> 
Российской академии наук,  
\mbox{samouylov-ke@rudn.university}}
\footnotetext[5]{Институт проблем информатики Федерального исследовательского центра <<Информатика  
и~управ\-ле\-ние>> Российской академии наук, \mbox{sshorgin@ipiran.ru}}

\vspace*{-5pt}

 
  
  \Abst{Нарезка радиоресурсов сети (network slicing)~--- это одна из ключевых 
возможностей современных сетей, позволяющая нескольким виртуальным мобильным 
операторам использовать ресурсы одной базовой станции. Это дает возможность 
операторам, владельцам ресурсов, предоставлять в~аренду и~управлять несколькими 
выделенными логическими сетями с~определенной функциональностью, реализуемой поверх 
общей инфраструктуры. Каждая из этих логических сетей называется слайсом сети и~может 
быть адаптирована для обеспечения определенного поведения системы, чтобы наилучшим 
образом поддерживать определенные показатели качества услуг. В~работе построена модель 
механизма нарезки радиоресурсов, распределяющего ресурс по слайсам, и~проведен анализ 
этой модели методом имитационного моделирования.}
  
  
  \KW{имитационное моделирование; система массового обслуживания; ограниченные 
ресурсы; нарезка сети}

\DOI{10.14357/19922264200314} 
 
%\vspace*{-6pt}


\vskip 10pt plus 9pt minus 6pt

\thispagestyle{headings}

\begin{multicols}{2}

\label{st\stat}
  
\section{Введение}

\vspace*{-2pt}

  Нарезка радиоресурсов сети (англ.\ \textit{network slicing}) дает возможность 
оператору мобильной связи предоставлять выделенные логические сети 
в~аренду виртуальным сетевым операторам в~виде сетевых слайсов 
с~функциями, специфичными для клиента. Слайс сети, который охватывает все 
сегменты сетевой инфраструктуры, может быть выделен для конкретных видов 
услуг нескольким виртуальным операторам, предоставляющим схожие услуги, 
либо отдельно для каждого виртуального оператора~[1,~2].
  
  Для каждого слайса сети выделяются ресурсы (например, 
виртуализированные сетевые функции, пропускная способность сети и~др.), 
и~ошибки или неисправность, возникающие в~одном слайсе, не влияют на 
обеспечение показателей качества обслуживания QoS (Quality of Service) 
в~других слайсах; иными словами, гарантируется изоляция слайса для 
обеспечения гарантированного качества обслуживания. При этом алгоритм 
нарезки радиоресурсов должен обеспечивать эффективное использование 
ресурсов соты с~учетом гарантированного объема ресурсов, выделенного для 
каждого слайса~[3].
  
  Данная тематика в~последнее время привлекает повышенное внимание 
исследователей. В~\cite{4-sop} представлена гибкая модель нарезки сети 
радиодоступа (Radio Access Network, RAN). Основные цели заключаются 
в~определении уровня изоляции производительности между операторами 
виртуальных сетей (Virtual Network Operator, VNO), которые выступают 
в~качестве арендаторов сети, с~тем чтобы гарантировать, что их соглашения об 
уровне обслуживания (Service Level Agreement, SLA) не будут затронуты 
изменением различных параметров сети, и~в~то же время оптимизировать 
использование инфраструктуры RAN путем динамического распределения 
радиоресурсов между различными сегментами справедливым образом. 
  
  В~\cite{5-sop} также рассматривается система управления виртуальными 
радиоресурсами (Virtual radio resource management, VRRM), которая 
обеспечивает оптимальное использование виртуализированных ресурсов 
поставщика инфраструктуры между несколькими операторами виртуальной 
сети. В~статье представлена архитектура инструмента моделирования VRRM 
в~терминах систем массового обслуживания (СМО). С~по\-мощью разработанного 
инструмента проводится анализ практического сценария с~тремя поставщиками 
и~различными типами SLA и~исследуются показатели производительности при 
изменении нагрузки на трафик и~SLA.
  
  В~работах~\cite{6-sop, 7-sop} рассматривается теоретическая основа для 
многооператорного планирования (Multi-Operator Scheduling, MOS). Благодаря 
динамической адаптации к~каналу и~нагрузке централизованный подход 
максимизирует спектральную эффективность для нескольких операторов 
с~полным контролем над гарантиями совместного использования. 
  
  В данной работе рассматривается сценарий функционирования одной соты 
беспроводной сети связи, в~которой активировано~$S$~слайсов, модуль 
нарезки делит между ними~$C$~единиц радиоресурсов. Каждый слайс 
предоставляет пользователям услугу связи, предполагающую непрерывную 
передачу данных с~определенным выделенным ресурсом, скоростью передачи, 
не менее $a_s\hm\geq 0$ и~не более $b_s\hm\geq a_s$, $s\hm\in S$. 
При этом скорость передачи является переменной: в~каждый момент времени 
она пересчитывается и~зависит от числа активных сессий в~каждом слайсе. 
Предполагается, что ресурс в~рамках одного слайса распределяется поровну 
между пользователями. В~работе описана модель в~виде 
СМО, предложен алгоритм разделения радиоресурсов, описана 
работа средства имитационного моделирования, проведен численный 
эксперимент и~анализ полученных результатов.

\section{Математическая модель}

  Пусть в~многолинейную СМО
поступает~$S$~потоков заявок, соответствующих запросам на передачу 
данных от пользователей~$S$~различных слайсов. Потоки являются 
пуассоновскими с~интенсивностями~$\lambda_s$, $s\hm\in S$. Объемы  
заявок~--- независимые случайные величины, распределенные по 
экспоненциальному закону с~параметрами~$1/\mu_s$, $s\hm\in S$, а~скорость 
обслуживания заявки определяется объемом выделенного ей ресурса. 
  
  Пусть общий объем ресурсов СМО для обслуживания заявок равен~$C$. 
Количество ресурсов, выделяемых заявке, принятой на обслуживание, зависит 
от состояния системы и~может варьироваться в~диапазоне $[a_s, b_s]$, $s\hm\in 
S$. При этом после каж\-до\-го поступления либо ухода заявки происходит 
перераспределение ресурсов между слайсами.
  
  Определим случайный процесс $X(t)\hm= \{ m_1(t), m_2(t), \ldots , m_S(t)\}$, 
где~$m_s$, $s\hm\in S$,~--- число заявок в~слайсе в~момент времени~$t$, 
причем 
  $$
  m_s\in \left\{ 0,1,\ldots , \left\lceil \fr{C}{a_s}\right\rceil \right\}\,,\enskip s\in S\,.
  $$
  %
  Тогда пространство возможных состояний процесса имеет вид:
  $$
  \mathrm{X}=\left\{ \left( m_1, m_2, \ldots , m_s\right)\,,\ \sum\limits^S_{s=1} 
m_s a_s\leq C\right\}\,.
  $$
  
  Обозначим через~$r_s$, $s\hm\in S$, количество выделенного ресурса одной 
заявке в~слайсе~$s$, $s\hm\in S$. Тогда интенсивности обслуживания заявок 
соответствующих слайсов определятся как~$r_s\mu_s$, $s\hm\in S$.
  
  Для обеспечения изоляции слайсов обозначим через~$\overline{R}_s$ объем 
ресурсов, который гарантированно выделен слайсу~$s$, $s\hm\in S$, а~через 
$\overline{M}_s\hm= \overline{R}_s/a_s$~--- число заявок, которое 
гарантированно может быть принято в~слайсе~$s$, $s\hm\in S$. Слайсы, число 
заявок в~которых превышает гарантированное значение, будем называть 
нарушителями. В~случае нехватки ресурсов и~наличия  
слай\-сов-на\-ру\-ши\-те\-лей поступившая заявка другого слайса может 
вытеснить одну или несколько заявок слай\-сов-на\-ру\-ши\-те\-лей. В~случае 
нехватки ресурсов и~отсутствия нарушителей, а~также в~случае когда все 
слайсы нарушают, поступающая заявка будет сброшена.
  
  Рассмотрим подробнее возможные события при поступлении сессий 
в~состоянии $(m_1, m_2, \ldots , m_s)\hm\in \mathrm{X}$. Пусть в~систему 
поступает заявка слайса~$s$, $s\hm\in S$. Тогда возможны следующие случаи:
  \begin{itemize}
  \item $(m_1, m_2, \ldots , m_s+1, \ldots , m_S)\hm\in \mathrm{X}$, т.\,е.\ 
предо\-став\-ле\-ние минимального количества ресурса для поступающей заявки 
возможно; в~этом случае заявка встает на обслуживание, а~случайный процесс 
переходит в~состояние  $(m_1, m_2, \ldots , m_s+1,\ldots , m_S)$;
  \item  $(m_1, m_2, \ldots , m_s+1, \ldots , m_S)\not= \mathrm{X}$, т.\,е.\ не 
гарантируется предоставление минимально тре\-бу\-емо\-го количества ресурса, 
тогда:
\begin{itemize}
  \item  если $m_s\hm< \overline{M}_s$ и~$m_k\hm> \overline{M}_k$, $k\hm\in S$, $k\not= 
s$, то выполняется освобождение ресурсов слай\-са-на\-ру\-ши\-те\-ля. 
Алгоритм сброса заявок приведен в~разд.~3;
  \item  в~остальных случаях поступающая заявка будет сброшена.
  \end{itemize}
  \end{itemize}
  
  Таким образом, множество состояний сброса заявок при поступлении:
  \begin{multline*}
  D_s=\left\{\vphantom{\left(\overline{M}_k\right)}\left (m_1, m_2, \ldots , m_S\right): {}\right.\\
{}:  \left( \left(m_1, m_2, \ldots , m_s+1, \ldots , m_S\right)\not= 
\mathrm{X}\right)\cap{}\\ 
\left.{}\cap\left( \left( m_s\geq \overline{M}_s\right) \cup \left( m_k\leq 
\overline{M}_k,\ k\in S,\ k\not=s\right)\right)
  \right\}\,.
\end{multline*}
  
  Множество состояний прерывания обслуживания:
\begin{multline*}
  B_k=\left\{ \vphantom{\left(\overline{M}_k\right)}
  \left( m_1, m_2, \ldots , m_S\right) :{}\right.\\
  {}: \left( \left( m_1, m_2, \ldots , m_s+1, \ldots , 
m_S\right) \not=\mathrm{X}\right) \cap {}\\
\left.{}\cap\left( \left( m_s<\overline{M}_s\right) \cup 
\left( m_k>\overline{M}_k,\ k\in S\,,\ k\not= s\right)\right)\right\}\,.
\end{multline*}

\section{Алгоритм выбора сбрасываемой заявки}

  Для выбора сбрасываемой заявки при наличии нескольких  
слай\-сов-на\-ру\-ши\-те\-лей введем понятие веса слайса, показывающего, 
насколько сильно слайс нарушает границы других слайсов. Ниже приведены 
три возможные формулы для вычисления весов: 
\begin{align*}
 w_s^{(1)}&=\begin{cases}
  1\,, & m_s\leq \overline{M}_s\,;\\
  \fr{1}{m_s-\overline{M}_s+1}\,, & m_s>\overline{M}_s\,;
  \end{cases}\\
  w_s^{(2)}&=\begin{cases}
  1\,, & m_s\leq \overline{M}_s\,;\\
  \fr{\overline{M}_s}{m_s}\,, & m_s>\overline{M}_s\,;
  \end{cases}\\
  w_s^{(3)}&=\begin{cases}
  1\,, & m_s\leq \overline{M}_s\,;\\
  w=const<1, & m_s>\overline{M}_s\,.
  \end{cases} %\label{e3-sop}
\end{align*}
 
  Освобождение ресурсов требуется в~случае, если заявка поступает в~слайс, 
который не является нарушителем, при этом она не может быть принята 
в~сис\-те\-му из-за нехватки свободного ресурса: 
\begin{multline*}
\left\{ \left( \left( m_1, m_2, \ldots , m_s+1, \ldots , m_S\right) \notin 
\mathrm{X}\right) :{}\right.\\
\left.{}: \left( m_s+1\right)a_s\leq R_s\right\}\,.
\end{multline*}

 \textbf{Шаг 1:} ищем слайс с~наименьшим весом, т.\,е.\ $s^*\hm\in S$: $w_{s^*}\hm= 
\min \{ w_r, r\hm\in S\}$.
  
  \textbf{Шаг 2:} сбрасываем заявку найденного слайса $(m_1, m_2, \ldots , m_{s^*}-1, 
\ldots , m_S)$.
  
  Шаги 1 и~2 выполняются, пока не будет удовле\-тво\-ре\-но условие $C\hm- 
\sum\nolimits^S_{i=1} m_i a_i\hm\geq a_s$. Отметим, что в~случае, когда 
у~нескольких слайсов вес минимален, выбор слайса для сброса заявки 
происходит случайным образом.
  
  
  \section{Распределение ресурсов}
  
  В состоянии избытка ресурсов 
  $$
  \mathrm{X}_0= \left(\! \left(m_1, m_2, \ldots , 
m_s+1, \ldots, m_S\right): \ \!\sum\limits^S_{s=1}\! m_s b_s \hm\leq C\!\right)
\hspace*{-0.94673pt}
$$ 
всем  заявкам во всех слайсах будет выделен максимальный объем ресурса~$b_s$.
  
  В состоянии ограниченных ресурсов $\mathrm{X}_1\hm= 
\mathrm{X}\backslash \mathrm{X}_0$ для справедливого и~эффективного 
распределения ресурсов решается задача оптимизации. Пусть скорости 
передачи данных в~слайсах имеют функцию полезности $U_s(r_s)\hm= \ln \left( 
r_s\right)$. Тогда задача выглядит следующим образом:
  \begin{gather}
  \sum\limits_{s\in S} w_s(m_s) m_s U_s(r_s)\to \max\,,\enskip s\in S\,;\notag\\
  \sum\limits_{s\in S} m_s r_s=C\,;\label{e5-sop}\\
  P=\left\{ \mathbf{r}\in \mathbf{R}^S:\ a_s\leq r_s\leq b_s\,,\ s\in S\right\}\,,
  \label{e6-sop}
  \end{gather}
где вектор~$\mathbf{r}$ имеет вид $\mathbf{r}\hm= (r_1, r_2, \ldots , r_S)$. 
Решение задачи выполняется с~по\-мощью при\-бли\-жен\-но\-го метода 
проецирования градиента, в~котором итеративная процедура поиска максимума 
описывается соотношениями:
\begin{gather*}
\mathbf{d}_k=\mathbf{P}\nabla f(\mathbf{x}_k)\,;\\
\mathbf{x}_{k+1} =\mathbf{x}_k+\tau_k \mathbf{d}_k\,;\\
\tau_k>0:\ \mathbf{x}_{k+1}\in P\,,
\end{gather*}
где вектор~$\mathbf{x}$ является решением, а~$\mathbf{P}$~--- матрица 
проецирования на гиперплоскость~(\ref{e5-sop}):
$$
\mathbf{P}=\mathbf{I}-\mathbf{m}\left(\mathbf{m}
\mathbf{m}^{\mathrm{T}}\right)^{-1}\mathbf{m} = \mathbf{I} -\fr{1}{\sum\nolimits_{s\in S} 
m_s^2}\,\mathbf{m}^{\mathrm{T}}\mathbf{m}\,.
$$
    Здесь вектор $\mathbf{m}\hm= (m_1, m_2, \ldots , m_S)$~--- текущее 
состояние системы. Градиент функции полезности представляет собой  
век\-тор-стол\-бец:
  $$
  \nabla f(\mathbf{x}_k)=\left( \fr{w_s m_s}{r_s}\right)_{s\in S}\,.
  $$
  
  Длину шага~$\tau_k$ выбираем таким образом, чтобы не выйти за пределы 
области~$P$, задаваемой прямыми ограничениями задачи~(\ref{e6-sop}). 
В~качестве начального приближения~$\mathbf{x}_0$ удобно взять точку 
пересечения диагонали координатного параллелограмма~$P$, соединяющей 
точки $(a_1, a_2, \ldots , a_S)$ и~$(b_1, b_2, \ldots , b_S)$ 
с~гиперплоскостью~(\ref{e5-sop}). Данная точка находится при решении 
системы линейных уравнений:
  \begin{equation*}
  \left.
  \begin{array}{c}
  (b_S-a_S)x_1-(b_1-a_1)x_S=a_1b_S-a_Sb_1\,;\\[6pt]
  (b_S-a_S)x_2-(b_2-a_2)x_S=a_2b_S-a_Sb_2\,;\\[6pt]
  \cdots\\[6pt]
  \hspace*{-8mm}(b_S-a_S)x_{S-1}-(b_{S-1}-a_{S-1})x_S={}\\[6pt]
  \hspace*{32mm}{}=a_{S-1}b_S - a_Sb_{S-1}\,;\\[6pt]
  \hspace*{-5mm}m_1x_1+\cdots + m_S x_S=C\,.
  \end{array}
  \right.
  \end{equation*}
  
  \begin{figure*}[b] %fig1
\vspace*{-4pt}
 \begin{center}
 \mbox{%
 \epsfxsize=159.488mm 
 \epsfbox{sop-1.eps}
 }
 \end{center}
   \vspace*{-9pt}
\Caption{Схема работы инструмента имитационного моделирования}
\end{figure*}
  
  Итеративная процедура обеспечивает движение по  
гиперплоскости~(\ref{e5-sop}) в~направлении возрастания функции полезности 
до границы области~$P$, где и~будет найдено решение.

\vspace*{-6pt}
  
  \section{Описание работы средства имитационного моделирования}
  
  \vspace*{-3pt}
  
  Общая схема работы имитатора изображена на рис.~1. На шаге 
инициализации входных па\-ра\-мет\-ров задаются начальные параметры: 
$\lambda_s$, $\mu_s$, $\overline{M}_s$, $a_s$, $b_s$, $s\hm\in S$, $C$, $\max$~--- 
максимальное число принятых заявок в~системе. Затем выполняется запуск 
имитационного моделирования. Далее определяется ближайшее событие: 

\begin{enumerate}[(1)]
\item если это поступление, то выполняется проверка на достаточность ресурсов:
\begin{enumerate}[({1}.1)] 
\item если ресурсов хватает, то добавляется заявка, пересчитываются ресурсы, 
и~определяется продолжение моделирования; 
\item если ресурсов не хватает, то 
проверяется, является ли слайс нарушителем: 
\begin{enumerate}[({1.2.}1)]
\item если является, то входящая 
заявка считается заблокированной; 
\item если не является, то запускается 
процесс поиска нарушителя и~освобождение ресурсов, после чего происходит 
добавление заявки, пересчитываются ресурсы и~определяется продолжение 
моделирования;
\end{enumerate}
\end{enumerate}
\item если ближайшее событие~--- это обслуживание, то 
выполняется освобождение и~перерасчет ресурсов.
\end{enumerate}

\vspace*{-8pt}

  
  \section{Численный эксперимент}
  
  \vspace*{-2pt}
  
  Для численного эксперимента рассматривается диапазон 10~МГц в~сети LTE
  (long-term evolution). 
Минимально допустимая скорость в~таком диапазон составляет 0,75~Мбит/с 
$(a_s\hm=0{,}75$, $s\hm\in S)$,\linebreak максимальная~--- 79,9~Мбит/с ($b_s\hm= 
79{,}9$, $s\hm\in S$). Максимальный объем ресурса, который может 
единовременно предоставлять базовая станция в~диапазоне 10~МГц,~--- 
450~Мбит/с, для расчетов используются несколько значений объема ресурсов: 
$$
C= [325,350,375,400,425,450]\,.
$$
 Данный ресурс будет разделен на трех 
($S\hm=3$) виртуальных операторов с~гарантированным выделенным ресурсом 
$\overline{R}_1\hm=225$, $\overline{R}_2\hm= 150$ и~$\overline{R}_3\hm= 75$. 
Интенсивности поступлений $\lambda_1\hm= \lambda_2\hm= \lambda_3\hm=30$, 
интенсивности обслуживания $\mu_1\hm= \mu_2\hm= \mu_3\hm= 30$ оставим 
равными для всех операторов.
  
  На рис.~2,\,\textit{а} видно, что когда ресурсов не хватает для обеспечения 
гарантированной скорости для всех операторов, то и~вероятность блокировки 
выше. Далее при увеличении ресурса видно, что для оператора, которому 
предоставлено больше гарантированных ресурсов, уменьшение вероятности 
блокировки происходит быстрее.
  

  На рис.~2,\,\textit{б} видно, что для оператора, которому предоставляется больший 
гарантированный ресурс, увеличивается средний размер слайса, в~то время как 
для оператора, которому гарантируется меньший ресурс, этот параметр 
меняется незначительно. Это связано со способом решения оптимального 
распределения ресурса, описанного в~данной работе.
  
  На рис.~2,\,\textit{в} отображен график, показывающий изменение доли времени, когда 
оператор является нарушителем. В~связи с~тем что общий размер \mbox{ресурса} 
растет, гарантированный объем остается неизменным, а~интенсивности 
поступления~--- фиксированные для каждого оператора, можно наблюдать 
резкое увеличение значений этого параметра для оператора~3.
  
   
\vspace*{-8pt}

  
  \section{Заключение}
  
  \vspace*{-2pt}
  
  В работе описана СМО с~ограниченными 
ресурсами с~распределением ресурсов в~зависимости от
веса слайса сети. 
Предложен и~реализован алгоритм освобождения ресурсов  
слай\-са\-ми-на\-ру\-ши\-те\-ля\-ми. Предложен и~реализован алгоритм 
распределения\linebreak\vspace*{-12pt}

{ \begin{center}  %fig2
 \vspace*{-3pt}
   \mbox{%
 \epsfxsize=79.374mm 
 \epsfbox{sop-2.eps}
 }

\end{center}

\noindent
{{\figurename~2}\ \ \small{
Вероятность блокировки~(\textit{а}),  
средний размер слайса~(\textit{б}) 
и~доля времени, когда слайс находится в~состоянии нарушителя~(\textit{в})
в~зависимости от объема ресурсов: \textit{1}~--- VNO~1; 
\textit{2}~--- VNO~2; \textit{3}~--- VNO~3
}}}

\vspace*{12pt}





\noindent
 ресурсов на основе весов слайсов. Разработано средство 
имитационного моделирования механизма распределения ресурсов. Проведен 
численный эксперимент для трех слайсов. В~дальнейшем планируется 
разработать модель, в~которой перераспределение слайсов происходит не при 
каждом изменении состояния процесса, а~при выполнении некоторого 
критерия.

\vspace*{-8pt}
  
{\small\frenchspacing
 {%\baselineskip=10.8pt
 \addcontentsline{toc}{section}{References}
 \begin{thebibliography}{9}
 
 \vspace*{-1pt}
 
 
\bibitem{1-sop}
3GPP TS 23.501 V15.4.0. System architecture for the 5G System, 2018. {\sf 
http://www.3gpp.org/ftp//Specs/ archive/23\_series/23.501/23501-f40.zip}.
\bibitem{2-sop}
ITU-T Rec. Y.3101. Requirements of the IMT-2020 network, 2018. {\sf 
https://www.itu.int/rec/dologin\_pub.asp?\linebreak lang=e\&id=T-REC-Y.3101-201801-I!!PDF-E\&type= items}.
\bibitem{3-sop}
\Au{P$\acute{\mbox{e}}$rez-Romero~J., 
Sallent~O., Ferr$\acute{\mbox{u}}$s~R., 
\mbox{Agust{\!\!\ptb{$\acute{\mbox{\i}}$}}}~R.} On 
the configuration of radio resource management in a~sliced RAN~// IEEE/IFIP 
Network Operations and Management Symposium.~--- IEEE, 2018. P.~1--6. doi: 
10.1109/ NOMS.2018.8406280.
\bibitem{4-sop} 
\Au{Rouzbehani B., Correia~L.\,M., Caeiro~L.} An SLA-based method for radio resource slicing 
and allocation in virtual %\linebreak\vspace*{-12pt}
%\columnbreak 
%\noindent
 RANs~// IRACON 7th MC and Technical Meeting.~--- 
Cartagena, Spain, 2018. Cost Action 15104 TD(18)07034. P.~1--7.
\bibitem{5-sop} 
\Au{Ageev K., Garibyan~A., Golskaya~A., Gaidamaka~Yu., Sopin~E., Samouylov~K., Correia~L.} 
Modelling of virtual radio resources slicing in 5G networks~// 
Information technologies and 
mathematical modelling. Queueing theory and applications~/
Eds. A.~Dudin, A.~Nazarov, A.~Moiseev.~--- Communications in 
computer and information science ser.~--- Cham: Springer, 2019.  
Vol.~1109. P.~150--161. doi: 10.1007/978-3-030-33388-1\_13.
\bibitem{6-sop} 
\Au{Malanchini I., Valentin~S., Aydin~O.} An analysis of\linebreak gen\-er\-al\-ized 
resource sharing for multiple 
operators in cel\-lu\-lar networks~// 25th Annual Symposium
 (In\-ter\-na\-tion\-al) on Personal, Indoor, 
and Mobile Radio Com\-mu\-ni\-ca\-tion.~--- IEEE, 2014. P.~1157--1162. doi: 
10.1109/\linebreak \mbox{PIMRC}.2014.7136342.
\bibitem{7-sop}
\Au{Malanchini I., Valentin~S., Aydin~O.} Wireless resource sharing for multiple operators: 
Generalization, fairness, and the value of prediction~// Comput. Netw., 2016. Vol.~100. 
P.~110--123. doi: 10.1016/j.comnet.2016.02.014.
\end{thebibliography}

 }
 }

\end{multicols}

\vspace*{-6pt}

\hfill{\small\textit{Поступила в~редакцию 15.07.20}}

\vspace*{8pt}

%\pagebreak

%\newpage

%\vspace*{-28pt}

\hrule

\vspace*{2pt}

\hrule

%\vspace*{-2pt}

\def\tit{ANALYSIS OF~THE~NETWORK SLICING MECHANISMS WITH~GUARANTEED ALLOCATED 
RESOURCES\\ FOR~VARIOUS TRAFFIC TYPES}


\def\titkol{Analysis of~the~network slicing mechanisms with~guaranteed allocated 
resources for~various traffic types}

\def\aut{K.\,A.~Ageev$^1$, E.\,S.~Sopin$^{1,2}$, N.\,V.~Yarkina$^1$, K.\,E.~Samouylov$^{1,2}$, 
and~S.\,Ya.~Shorgin$^2$}

\def\autkol{K.\,A.~Ageev, E.\,S.~Sopin, N.\,V.~Yarkina, 
%K.\,E.~Samouylov$^{1,2}$, and~S.\,Ya.~Shorgin$^2$ 
et al.}

\titel{\tit}{\aut}{\autkol}{\titkol}

\vspace*{-9pt}


\noindent
$^1$Peoples' Friendship University of Russia (RUDN University), 6~Miklukho-Maklaya Str., Moscow 
117198, Russian\linebreak
$\hphantom{^1}$Federation

\noindent
$^2$Institute of Informatics Problems, Federal Research Center ``Computer Sciences and Control'' of the 
Russian\linebreak
$\hphantom{^1}$Academy of Sciences; 44-2~Vavilov Str., Moscow 119133, Russian Federation

\def\leftfootline{\small{\textbf{\thepage}
\hfill INFORMATIKA I EE PRIMENENIYA~--- INFORMATICS AND
APPLICATIONS\ \ \ 2020\ \ \ volume~14\ \ \ issue\ 3}
}%
 \def\rightfootline{\small{INFORMATIKA I EE PRIMENENIYA~---
INFORMATICS AND APPLICATIONS\ \ \ 2020\ \ \ volume~14\ \ \ issue\ 3
\hfill \textbf{\thepage}}}

\vspace*{3pt} 




\Abste{Network slicing is one of the key capabilities of
 modern networks, allowing several virtual mobile operators 
 to use the physical resources of one base station. 
 This allows operators and resource owners (tenants) to 
 lease and manage several dedicated logical networks with 
 specific functionality working on top of a~common infrastructure. 
 Each of these logical networks is called a~network slice and can 
 be adapted to provide certain system behavior to maintain 
 a~specified level of quality of service indicators. The paper
  describes the developed mathematical framework of the network
   slicing mechanisms 
and analyzes it by means of extensive simulations.}

\KWE{simulation modeling; queuing system; limited resources; network slicing}

\DOI{10.14357/19922264200314} 

\vspace*{-20pt}

\Ack
\noindent
The reported study was funded by the ``RUDN University Program 5-100'' and in
part by RFBR, projects  
Nos.\,19-07-00933 and 19-37-90147.

\vspace*{4pt}

 \begin{multicols}{2}

\renewcommand{\bibname}{\protect\rmfamily References}
%\renewcommand{\bibname}{\large\protect\rm References}

{\small\frenchspacing
 {%\baselineskip=10.8pt
 \addcontentsline{toc}{section}{References}
 \begin{thebibliography}{9}
 
 \vspace*{-2pt}
 
\bibitem{1-sop-1}
3GPP TS 23.501 V15.4.0. 2018. System architecture for the 5G System. 
Available at: {\sf 
http://www.3gpp.org/ftp// Specs/archive/23\_series/23.501/23501-f40.zip} 
(accessed June~15, 2020).
\bibitem{2-sop-1}
ITU-T Rec. Y.3101. 2018. Requirements of the IMT-2020 network. Available at: {\sf 
https://www.itu.int/rec/ dologin\_pub.asp?lang=e\&id=T-REC-Y.3101-201801-I!!PDF-E\&type=items} 
(accessed June~15, 2020).
\bibitem{3-sop-1}
\Aue{P$\acute{\mbox{e}}$rez-Romero, J., O.~Sallent,
 R.~Ferr$\acute{\mbox{u}}$s, and 
R.~\mbox{Agust{\!\!\ptb{$\acute{\mbox{\i}}$}}}}. 2018. On the configuration of radio resource management in a sliced 
RAN. \textit{IEEE/IFIP Network Operations and Management Symposium}.
 IEEE. 1--6.  doi: 10.1109/ NOMS.2018.8406280.
\bibitem{4-sop-1}
\Aue{Rouzbehani, B., L.\,M.~Correia, and L.~Caeiro.} 2018. An SLA-based method for radio resource 
slicing and allocation in virtual RANs. \textit{IRACON 7th MC and Technical Meeting}. Cartagena. 
TD(18)07034. 1--7.
\bibitem{5-sop-1}
\Aue{Ageev, K., A.~Garibyan, A.~Golskaya, Yu.~Gaidamaka, E.~Sopin, K.~Samouylov, and L.~Correia.} 
2019. Modelling of virtual radio resources slicing in 5G networks. 
\textit{Information technologies and 
mathematical modelling. Queueing theory and applications}. Eds. 
A.~Dudin, A.~Nazarov, and A.~Moiseev. 
Communications in computer and information science ser. Cham:
Springer. 1109:150--161. 
\bibitem{6-sop-1}
\Aue{Malanchini, I., S.~Valentin, and O.~Aydin.} 2014. An analysis of generalized resource sharing for 
multiple operators in cellular networks. \textit{25th Annual Symposium (International)
on Personal,  Indoor, and Mobile Radio Communication}. IEEE. 1157--1162. 
doi:  10.1109/PIMRC.2014.7136342.
\bibitem{7-sop-1}
\Aue{Malanchini, I., S.~Valentin, and O.~Aydin.} 2016. Wireless resource sharing for multiple operators: 
Generalization, fairness, and the value of prediction. \textit{Comput. Netw.} 100:110--123.
doi: 10.1016/j.comnet.2016.02.014.

\end{thebibliography}

 }
 }

\end{multicols}

\vspace*{-6pt}

\hfill{\small\textit{Received July 15, 2020}}

%\pagebreak

%\vspace*{-24pt}



\Contr

\noindent
\textbf{Ageev Kirill A.} (b.\ 1993)~--- PhD student, Peoples' Friendship University of Russia (RUDN 
University), 6~Miklukho-Maklaya Str., Moscow 117198, Russian Federation; \mbox{ageev-ka@rudn.ru}

\vspace*{3pt}

\noindent
\textbf{Sopin Eduard S.} (b.\ 1987)~--- Candidate of Science in physics and mathematics, associate 
professor, Peoples' Friendship University of Russia (RUDN University), 6~Miklukho-Maklaya Str., 
Moscow 117198, Russian Federation; senior scientist, Institute of Informatics Problems, Federal Research 
Center ``Computer Sciences and Control'' of the Russian Academy of Sciences; 44-2 Vavilov Str., Moscow 
119133, Russian Federation; \mbox{sopin-es@rudn.ru}

\vspace*{3pt}

\noindent
\textbf{Yarkina Natalia V.} (b.\ 1979)~--- Candidate of Science in physics and mathematics, associate 
professor, Peoples' Friendship University of Russia (RUDN University), 6~Miklukho-Maklaya Str., 
Moscow 117198, Russian Federation; \mbox{yarkina-nv@rudn.ru}

\vspace*{3pt}

\noindent
\textbf{Samouylov Konstantin E.} (b.\ 1955)~--- Doctor of Science in technology, professor, Head of 
Department, Peoples' Friendship University of Russia (RUDN University), 6~Miklukho-Maklaya Str., 
Moscow 117198, Russian Federation; senior scientist, Institute of Informatics Problems, Federal Research 
Center ``Computer Sciences and Control'' of the Russian Academy of Sciences; 44-2~Vavilov Str., Moscow 
119133, Russian Federation; \mbox{samuylov\_ke@rudn.university}

\vspace*{3pt}

\noindent
\textbf{Shorgin Sergey Ya.} (b.\ 1952)~--- Doctor of Science in physics and mathematics, professor, 
principal scientist, Institute of Informatics Problems, Federal Research Center ``Computer Sciences and 
Control'' of the Russian Academy of Sciences, 44-2~Vavilov Str., Moscow 119333, Russian Federation; 
\mbox{sshorgin@ipiran.ru}
\label{end\stat}

\renewcommand{\bibname}{\protect\rm Литература}    %6
\def\stat{gorbunova}

\def\tit{РЕСУРСНЫЕ СИСТЕМЫ МАССОВОГО ОБСЛУЖИВАНИЯ КАК~МОДЕЛИ БЕСПРОВОДНЫХ СИСТЕМ 
СВЯЗИ$^*$}

\def\titkol{Ресурсные системы массового обслуживания как~модели беспроводных систем 
связи}

\def\aut{А.\,В.~Горбунова$^1$, В.\,А.~Наумов$^2$, Ю.\,В.~Гайдамака$^3$, К.\,Е.~Самуйлов$^4$}

\def\autkol{А.\,В.~Горбунова, В.\,А.~Наумов, Ю.\,В.~Гайдамака, К.\,Е.~Самуйлов}

\titel{\tit}{\aut}{\autkol}{\titkol}

\index{Горбунова А.\,В.}
\index{Наумов В.\,А.}
\index{Гайдамака Ю.\,В.}
\index{Самуйлов К.\,Е.}
\index{Gorbunova A.\,V.}
\index{Naumov V.\,A.}
\index{Gaidamaka Yu.\,V.}
\index{Samouylov K.\,E.}




{\renewcommand{\thefootnote}{\fnsymbol{footnote}} \footnotetext[1]
{Публикация подготовлена при финансовой поддержке Минобрнауки России 
(проект 2.882.2017/4.6).}}


\renewcommand{\thefootnote}{\arabic{footnote}}
\footnotetext[1]{Российский университета дружбы народов, 
\mbox{gorbunova\_av@rudn.university}}
\footnotetext[2]{Исследовательский институт инноваций, Хельсинки, 
Финляндия, \mbox{valeriy.naumov@pfu.fi}}
\footnotetext[3]{Российский университет дружбы народов; Институт 
проб\-лем информатики Федерального исследовательского центра <<Информатика 
и~управ\-ле\-ние>> Российской академии наук, \mbox{gaydamaka\_yuv@rudn.university}}
\footnotetext[4]{Российский университет дружбы народов; Институт 
проблем информатики Федерального исследовательского центра <<Информатика 
и~управ\-ле\-ние>> Российской академии наук, \mbox{samouylov\_ke@rudn.university}}

\vspace*{-5pt}



\Abst{Представлен обзор ресурсных систем массового обслуживания (СМО), используемых 
для моделирования широкого класса реальных систем, в~которых ресурсы являются 
заведомо ограниченными. Несмотря на объективную важность исследования подобных 
систем, работ, посвященных их анализу, до последнего времени существовало совсем 
немного, что было связано со сложностью построения случайного процесса, 
описывающего их функционирование, и,~соответственно, получения численных 
результатов. Однако за последние годы произошел существенный сдвиг в~изучении 
ресурсных систем, были предложены новые методы их анализа, позволяющие строить 
рекуррентные алгоритмы, пригодные для численных расчетов.
В~этой связи в~обзоре отражена только часть полученных результатов, а~именно:
рассмотрены ресурсные системы без мест для ожидания с~экспоненциальным временем 
обслуживания. Рассмотрены модели беспроводных систем связи, основанные на 
ресурсных СМО (РСМО), выражения для оценки основных 
ве\-ро\-ят\-но\-ст\-но-вре\-мен\-ных характеристик и~алгоритмы их вычисления.}


\KW{ресурсная система массового обслуживания; непрерывный 
ресурс; дискретный ресурс; ограниченный ресурс; рекуррентный алгоритм; 
гетерогенная сеть; стационарное распределение; полумарковский процесс; 
беспроводные системы связи}

\DOI{10.14357/19922264180307}
  
%\vspace*{4pt}


\vskip 10pt plus 9pt minus 6pt

\thispagestyle{headings}

\begin{multicols}{2}

\label{st\stat}

\section{Введение}

В классических СМО приборы и~места ожидания 
играют роль необходимых для обслуживания ресурсов. В~РСМО кроме 
приборов и~мест ожидания заявкам могут потребоваться различные дополнительные 
ресурсы. Это может быть некоторый случайный объем ресурса, занимаемого на время 
ожидания начала обслуживания, либо на время обслуживания, либо на все время 
пребывания заявки в~сис\-те\-ме. Если у~сис\-те\-мы нет достаточного числа свободных 
ресурсов, поступившая заявка теряется. В~дальнейшем будем использовать термин 
<<ресурс>> только для обозначения дополнительного ресурса, отличного от приборов 
или мест ожидания.

Интерес к~РСМО объясняется возможностью их применения для моделирования 
достаточно широкого спектра технических устройств 
и~в~целом ин\-фор\-ма\-ци\-он\-но-вы\-чис\-ли\-тель\-ных систем.
В~частности, если говорить о единственном типе ресурса ограниченного объема, то 
таким образом может моделироваться ограниченность памяти некоторого устройства 
или отдельной системы.
Таким образом, увеличивается реалистичность модели и,~соответственно, ее 
практическая цен\-ность.
%
Если же говорить о~множественных ресурсах, то стоит вспом\-нить услуги 
беспроводных сетей, таких, например, как Long Term Evolution (LTE)~\cite{Andrews}.
 Рост их популярности делает необходимым создание эффективных 
инструментов для оценки телекоммуникационными операторами работы 
радиоинтерфейсов~\cite{Galinina,Samuylov}. В~этих сетях каждая активная сессия 
занимает определенный объем радиоресурсов (например, ширину полосы пропускания 
спектра час\-тот, мощ\-ности передачи радиочастотного усилителя и~др.), которые 
являются заведомо ограниченными и~должны быть распределены при поступлении 
вызова пользователя и~освобождены по завершении сессии~\cite{Naumov_3_2016}.

Стоит отметить, что моделированию беспроводных систем связи с~по\-мощью СМО 
с~множественными ресурсами начиная с~\cite{Gimpelson} посвящено большое чис\-ло 
публикаций. Однако основной акцент в~них делается на анализ различных схем 
распределения ресурсов в~системах c детерминированными требованиями заявок 
к~ресурсам. Обзор этих работ можно найти в~\cite{Kelly,Ross,Basharin}.

Статья организована следующим образом: в~разд.~2 описываются основные типы 
РСМО без мест для ожидания, методы их исследования и~полученные результаты 
в~виде выражений для основных ве\-ро\-ят\-но\-ст\-но-вре\-мен\-н$\acute{\mbox{ы}}$х характеристик 
функционирования указанных сис\-тем. В~разд.~3 представлены подходы, позволяющие 
провести численные расчеты с~по\-мощью полученных соотношений. В~заключении кратко 
подведены итоги работы.

\section{Ресурсные системы массового обслуживания}

Более подробно остановимся на описании общей модели РСМО без мест для ожидания 
(рис.~1).
Система может располагать ограниченным или неограниченным объемом ресурсов как 
одного, так и~нескольких типов. Схему ее функционирования можно описать 
следующим образом:
\begin{enumerate}[(1)]
\item для обслуживания каждой заявки требуется один прибор и~некоторый объем 
ресурса каж\-до\-го типа;
\item поступившая заявка теряется, если в~момент поступления объем 
требуемого ей ресурса превышает объем свободного ресурса этого типа либо все 
приборы заняты;
\item в~момент начала обслуживания заявки суммарный объем занятого ресурса 
каждого типа увеличивается на величину ресурса, выделенного этой заявке;
\item в~момент окончания обслуживания заявки суммарный объем занятого 
ресурса каждого типа уменьшается на величину ресурса, выделенного этой заявке.
\end{enumerate}



В СМО может поступать один или несколько классов заявок, для которых
$A_l(t)$~--- функция распределения времени между поступлениями заявок класса~$l$,
$H_l(t,\mathbf{x})$~--- совместная функция распределения длительности 
обслуживания и~вектора объема ресурсов, необходимых поступившей заявке\linebreak\vspace*{-12pt}

{ \begin{center}  %fig1
 \vspace*{9pt}
  \mbox{%
 \epsfxsize=78.288mm 
 \epsfbox{gor-1.eps}
 }


\vspace*{6pt}


\noindent
{{\figurename~1}\ \ \small{Схема функционирования  РСМО общего вида}}
\end{center}
}

%\vspace*{9pt}

{ \begin{center}  %fig2
 \vspace*{-2pt}
  \mbox{%
 \epsfxsize=61.777mm 
 \epsfbox{gor-2.eps}
 }


\vspace*{9pt}

\noindent
{{\figurename~2}\ \ \small{Схема функционирования простейшей РСМО}}
\end{center}
}

\vspace*{9pt}





\noindent
 класса~$l$, 
$l\hm=\overline{1,L}$.
Для случая, когда случайные величины длительности обслуживания и~вектора объема 
необходимых ресурсов независимы, имеем 
$$
H_l(t,\mathbf{x})=B_l(t)F_l(\mathbf{x})\,,
$$
 где
$B_l(t)$~--- функция распределения времени обслуживания заявки класса~$l$;
$F_l(\mathbf{x})$~--- функция распределения вектора объема ресурсов, тре\-бу\-емых 
заявкам класса~$l$.

Первые статьи, посвященные анализу СМО с~выделением 
каждой поступающей заявке помимо прибора некоторого случайного объема ресурса 
единственного типа появились в~начале \mbox{1970-х~гг.}~\cite{Romm_21_1971,Kac}.
В~част\-ности, в~работе~\cite{Romm_21_1971} рассматривалась бесконечно линейная 
СМО с~пуассоновским входящим потоком 
и~экспоненциальным временем обслуживания (рис.~2). 
Величины требуемых 
ресурсов~--- независимые одинаково распределенные случайные величины с~функцией 
распределения~$F(x)$. В~качестве емкости системы, т.\,е.\ максимально допустимого 
объема ресурсов, выступает величина~$R$.

Система уравнений равновесия (СУР) для случайного процесса, описывающего 
систему, фактически представляет собой обобщение системы Эрланга. В~результате 
решения СУР были получены стационарные вероятности того, что в~сис\-те\-ме 
находится~$k$~заявок:
\begin{equation*}
p_k=\fr{({1}/{k!})({\lambda}/{\mu})^kF^{(k)}(R)}{\sum\nolimits_{i=0}^{\infty}
({1}/{i!})({\lambda}/{\mu})^iF^{(i)}(R)}\,,
\end{equation*}
где $F^{(k)}(x)$ является $k$-крат\-ной сверткой функции распределения~$F(x)$, 
$k=2,3,\ldots$,
$F^{(0)}(x)\hm=1$, $F^{(1)}(x)\hm=F(x)$.
В~условиях описанной модели потеря заявки или отказ в~обслуживании происходят 
только тогда, когда раз\-ность между величиной объема всей сис\-те\-мы и~суммарным 
объемом ресурсов, занятых находящимися в~сис\-те\-ме заявками, меньше, чем величина 
требуемого объема ресурсов у~вновь поступившей заявки. Таким образом, 
вероятность потери заявки равна
\begin{equation*}
B=1-
\fr{\sum\nolimits_{k=0}^{\infty}({1}/{k!})({\lambda}/{\mu})^kF^{(k+1)}(R)}
{\sum\nolimits_{k=0}^{\infty}({1}/{k!})({\lambda}/{\mu})^kF^{(k)}(R)}\,.
\end{equation*}

Для того чтобы более детально ознакомиться с~особенностями построения и~анализа 
РСМО, подробнее остановимся на статье~\cite{Naumov_3_2016}.
Здесь рас\-смат\-ри\-ва\-ет\-ся многолинейная СМО c $N\hm\leq \infty$ приборами. Поступающий 
поток является пуассоновским с~па\-ра\-мет\-ром~$\lambda$, длительности обслуживания 
заявок независимы между собой и~от поступающего потока и~имеют экспоненциальное 
распределение с~параметром~$\mu$. Система располагает ограниченным объемом 
ресурсов~$M$~типов.
Обозначим через~$R_m$ общий объем ресурса типа~$m$, $\mathbf{R}\hm=(R_1,\ldots,R_M)$, 
и~через $\mathbf{r}_j\hm=(r_{j1}, r_{j2},\ldots, r_{jM})$~--- вектор объемов 
ресурсов, необходимых $j$-й поступившей заявке, $j \hm= 1, 2,\ldots$
Будем считать, что случайные векторы~$\mathbf{r}_j$ не зависят от процессов 
поступления и~обслуживания заявок, независимы в~совокупности и~одинаково 
распределены с~функцией распределения $F(\mathbf{x}), \mathbf{x}\hm=(x_1,\ldots,x_M)$.
Состояние такой системы в~момент~$t$ можно описать полумарковским процессом 
$X(t)\hm=\{\xi(t),\boldsymbol{\Gamma}(t)\}$~\cite{Naumov_3_2016}. Здесь~$\xi(t)$~--- 
число заявок в~сис\-те\-ме, а~$\mathbf{\Gamma}(t)\hm=
(\boldsymbol{\gamma}_1(t),\boldsymbol{\gamma}_2(t),\ldots,\boldsymbol{\gamma}_{\xi(t)}
(t))$, где $\boldsymbol{\gamma}_i(t)$~--- вектор объемов 
всех типов ресурсов, занимаемых $i$-й обслуживаемой заявкой. Находящиеся на 
обслуживании заявки перенумеровываются в~порядке убывания остаточного времени 
обслуживания.
Рассмотрим предельное распределение процесса~$X(t):$
\begin{align*}
p_0&=\lim_{t\rightarrow \infty}P\{\xi(t)=0\}\,;
\\
P_k\left(\mathbf{x}_1,\mathbf{x}_2,\ldots,\mathbf{x}_k\right)&=\lim\limits_{t\rightarrow 
\infty} P
\left\{\xi(t)=k;\right.\\
&
 \hspace*{-20mm}\left.\boldsymbol{\gamma}_1(t)\leq 
\mathbf{x}_1,\enskip
\boldsymbol{\gamma}_2(t)\leq 
\mathbf{x}_2,\ldots,\boldsymbol{\gamma}_k(t)\leq \mathbf{x}_k\right\}\,.
\end{align*}
После решения соответствующей СУР получаем
\begin{align*}
%\label{eq:p_0}
p_0&=\left(1+\sum\limits_{i=1}^{N}F^{(k)}(\mathbf{R})\fr{\rho^k}{k!}     \right)^{-1}\,;\\
P_k(\mathbf{x}_1,\mathbf{x}_2,\ldots,\mathbf{x}_k)&=p_0F(\mathbf{x}_1)F(\mathbf{x}_2
)\cdots F(\mathbf{x}_k)\frac{\rho^k}{k!},\\
&\hspace*{-20mm}\mathbf{x}_1,\mathbf{x}_2,\ldots,\mathbf{x}_k \geq \mathbf{0}, \enskip 
\sum\limits_{i=1}^{k}\mathbf{x}_i\leq \mathbf{R}, \enskip 1\leq k \leq N,
\end{align*}
где $\rho=\lambda/\mu$; $F^{(k)}(\mathbf{x})$~--- $k$-крат\-ная свертка 
функции~$F(\mathbf{x})$; $\mathbf{x}_i=(x_{i1},\ldots,x_{iM})$, $i\hm=\overline{1,k}$.
Далее, если обозначить вектор суммарных объемов занятых ресурсов каждого типа 
$\boldsymbol{\delta}(t)\hm=\sum\nolimits_{i=1}^{\xi(t)}\boldsymbol{\gamma}_i(t)$, 
$\boldsymbol{\delta}(t)\hm=(\delta_1(t),\ldots,\delta_M(t))$, стационарное 
распределение~$Q_k(\mathbf{x})$ случайного процесса 
$X(t)\hm=(\xi(t);\boldsymbol{\delta}(t))$ примет вид:
\begin{multline*}
\label{eq:Q}
\hspace*{-6pt}Q_k(\mathbf{x})=\lim_{t\rightarrow \infty} P\{ \xi(t)=k; 
\boldsymbol{\delta}(t)\leq \mathbf{x}\}=p_0F^{(k)}(\mathbf{x}) 
\fr{\rho^{k}}{k!}\,,\\
\mathbf{0}\leq \mathbf{x} \leq \mathbf{R}\,,\enskip  1\leq k \leq N\,.
\end{multline*}

В~\cite{Naumov_6_2015} исследуется РСМО с~единственным типом ограниченного 
ресурса объема~$R$, но уже с~$L$ входящими пуассоновскими потоками 
с~интенсивностями $\lambda_1,\ldots,\lambda_L$ и~с~$N\hm\leq\infty$ приборами.\linebreak 
Длительности обслуживания заявок независимы между собой, от поступающих потоков и~экспоненциально распределены с~параметром~$\mu_l$ для заявок класса~$l$, 
$l\hm=\overline{1,L}$.
Предполагается, что объем %\linebreak 
ресурса, требуемого заявкам класса~$l$, является 
случайной величиной 
с~функцией распределения~$F_l(x)$, не зависящей от процессов поступления 
и~обслуживания заявок. Обслуживающимся заявкам присваивается номер, причем так, 
чтобы заявка с~номером $i$ имела $i$-е по величине остаточное время 
обслуживания. Этот номер следует отличать от порядкового номера заявки. При 
поступлении новой заявки все находящиеся на обслуживании заявки 
перенумеровываются.
Состояние системы в~момент~$t$ описывается полумарковским процессом 
$X(t)\hm=(\xi(t);\boldsymbol{\theta}(t);\boldsymbol{\gamma}(t))$. Здесь, как 
и~прежде, $\xi(t)$~--- число заявок в~сис\-те\-ме; 
$\boldsymbol{\theta}(t)\hm=(\theta_1(t),\theta_2(t),\ldots,\theta_{\xi(t)}(t))$; 
$\boldsymbol{\gamma}(t)\hm=(\gamma_1(t),\gamma_2(t),\ldots,\gamma_{\xi(t)}(t))$, где 
$\theta_i(t)$~--- класс $i$-й обслуживаемой заявки; $\gamma_i(t)$~--- объем 
занимаемого ею ресурса.

Введем стационарное распределение процесса~$X(t)$
\begin{align*}
p_0&=\lim_{t\rightarrow \infty}P\{\xi(t)=0\}\,;
\\
p^k_{l_1,\ldots,l_k}(x_1,\ldots,x_k)&=\lim\limits_{t\rightarrow \infty}P
\left\{\xi(t)=k; \right.\\
&\hspace*{-10mm}\theta_1(t)=l_1,\ldots,\theta_k(t)=l_k; \\
&\left.\gamma_1(t)\leq x_1,\ldots,\gamma_k(t)\leq x_k
\right\}.
\end{align*}
В~результате решения соответствующей СУР получены выражения для стационарных 
вероятностей описанной сис\-те\-мы.
Кроме того, в~\cite{Naumov_6_2015} показано, что стационарные вероятности того, 
что в~сис\-те\-ме находятся
$k_j$ заявок типа~$j$ и~суммарный объем занимаемого ими ресурсов не превосходит~$x_j$, 
$j\hm=\overline{1,L}$, имеют мультипликативный вид:
\begin{equation*}
P_{k_1,\ldots,k_L}(x_1,\ldots,x_k)=p_0
\prod\limits_{j=1}^{L}F_j^{(k_j)}(x_j)\fr{\rho_j^{k_j}}{k_j!}\,.
\end{equation*}

В~\cite{Naumov_10_2015} исследуются показатели эффективности сетей LTE. 
Ресурсная СМО, моделирующая сис\-те\-му, аналогична представленной 
в~\cite{Naumov_6_2015}, но уже с~$M$~типами ограниченных ресурсов, и~потому 
стационарные вероятности
\begin{align*}
p_0&=\lim\limits_{t\rightarrow \infty}P\{\xi(t)=0\}\,;
\\
p^k_{l_1,\ldots,l_k}(\mathbf{x}_1,\ldots,\mathbf{x}_k)&=
\lim\limits_{t\rightarrow \infty}P
\left\{\xi(t)=k;\right.\\
 &\theta_1(t)=l_1,\ldots,\theta_k(t)=l_k;\\
&\left.\boldsymbol{\gamma}_1(t)\leq \mathbf{x}_1,\ldots,\boldsymbol{\gamma}_k(t)\leq 
\mathbf{x}_k
\right\}
\end{align*}
после решения соответствующей СУР примут вид:
\begin{multline*}
\hspace*{70pt}p_0=\left( 
\vphantom{\sum\limits_{r=1}^{N}}
1+{}\right.\\
\left.{}+\sum\limits_{r=1}^{N}\sum\limits_{k_1+\cdots+k_r=r}\hspace*{-3mm}\left(
F_1^{(k_1)}*F_2^{(k_2)}*\cdots *F_r^{(k_r)}
\right)(\mathbf{R})\times{}\right.\\
\left.{}\times \fr{\rho_1^{k_1}}{k_1!}\,\fr{\rho_2^{k_2}}{k_2!}\cdots
\fr{\rho_1^{k_r}}{k_r!} \right)^{-1};
\end{multline*}

\vspace*{-12pt}

\noindent
\begin{multline*}
p^k_{l_1,\ldots,l_k}(\mathbf{x}_1,\ldots,\mathbf{x}_k)={}\\
{}=p_0 
F_{l_1}(\mathbf{x}_1)F_{l_2}(\mathbf{x}_2)\cdots F_{l_k}(\mathbf{x}_k)
\displaystyle\prod\limits_{n=1}^{k}\fr{\lambda_{l_n}}{\sum\nolimits_{i=1}^{n}\mu_{l_i}},\\
1\leq l_1,\ldots,l_k \leq L\,, \enskip 
\mathbf{x}_1,\mathbf{x}_2,\ldots,\mathbf{x}_k\geq \mathbf{0}\,, \\
\displaystyle \sum\limits_{i=1}^{k}\mathbf{x}_i\leq \mathbf{R}, \enskip
1\leq k \leq N,
\end{multline*}
где символ~$*$ означает свертку функции распределения.

В работе~\cite{Naumov_15_2017} моделируется ситуация, когда объем ресурсов, 
запрашиваемых пользователями, может быть не только положительным, но 
и~отрицательным. Запросы на отрицательный объем ресурса увеличивают объем 
доступного ресурса для пользователей, запрашивающих его положительные объемы. 
В~\cite{Naumov_15_2017} предполагается зависимость времени обслуживания 
и~интервалов между поступлениями заявок от числа заявок в~системе. В~результате 
анализа моделей получены формулы для расчета основных 
ве\-ро\-ят\-но\-ст\-но-вре\-мен\-ных характеристик.

В~\cite{ Sopin_12_2017,Sopin_13_2017} для анализа сетей LTE с~динамически 
меняющимися требованиями к~ресурсам рас\-смат\-ри\-ва\-ют\-ся РСМО с~добавлением 
пуассоновского потока сигналов, инициирующего перераспределение ресурсов для 
активных пользователей. Развитием работы~\cite{Sopin_13_2017} 
стали статьи~\cite{Naumov_14_2017, Dohler_2017}.
Были исследованы два сценария перераспределения ресурсов и~сопоставлены 
посредством численного анализа.

В серии работ~\cite{Sopin_13_2017,Sopin_4_2015,Sopin_5_2015,Sopin_7_2016,Sopin_8_2017,Vihrova_
9_2017,Sopin_11_2017,Sopin_17_2018} исследуются РСМО, в~которых объем выделяемых 
заявке ресурсов имеет дискретное распределение, т.\,е.\ для $i$-й поступившей 
в~систему заявки с~вероятностью $p_j\hm=P(r_i\hm=j)$ потребуется ресурс объема~$j$.
Так, в~\cite{Sopin_7_2016} анализируется РСМО с~$L$~входящими пуассоновскими 
потоками и~$M$~типами ресурсов. Получены выражения для стационарных 
вероятностей:
\begin{multline*}
q^k_{k_1,\ldots,k_L}(\mathbf{r}_1,\ldots,\mathbf{r}_L)={}\\
{}=q_0
\sum\limits_{k_1+\cdots+k_l=k}p^{(k_1)}_{1,\mathbf{r}_1}\cdots p^{(k_L)}_{1,\mathbf{r}_L}
\fr{\rho_1^{k_1}}{k_1!}\cdots \fr{\rho_L^{k_L}}{k_L!}\,;
\end{multline*}

\vspace*{-12pt}

\noindent
\begin{multline*}
\hspace*{76pt}q_0={}\\
\!{}=\left(\!
\vphantom{\sum\limits_{k=0}^{N}\sum\limits_{k_1+\cdots+k_L=k}
\sum\limits_{\mathbf{r}_1+\cdots+\mathbf{r}_L \leq 
\mathbf{R}} p^{(k_L)}_{1,\mathbf{r}_L}\fr{\rho_1^{k_1}}{k_1!}\cdots
\fr{\rho_L^{k_L}}{k_L!}}
1+ {}\right. 
\left.\!\!\!\sum\limits_{k=0}^{N}\sum\limits_{k_1+\cdots+k_L=k}
\sum\limits_{\mathbf{r}_1+\cdots+\mathbf{r}_L \leq 
\mathbf{R}} \hspace*{-8pt}p^{(k_L)}_{1,\mathbf{r}_L}\fr{\rho_1^{k_1}}{k_1!}\cdots
\fr{\rho_L^{k_L}}{k_L!}
\!\right)^{\!-1}\!\!\!,\hspace*{-8.1138pt}
\end{multline*}
где $q^k_{k_1,\ldots,k_L}(\mathbf{r}_1,\ldots,\mathbf{r}_L)$~--- это вероятность 
того, что в~системе находятся~$k$~заявок, из которых~$k_1$~--- класса~1, 
$k_2$~--- класса~2 и~т.\,д., а~суммарный объем ресурсов каждого типа, занятых заявками 
класса~1, равен~$\mathbf{r}_1$ и~т.\,д.

В статье~\cite{Vihrova_9_2017} при исследовании той же СМО, что 
и~в~\cite{Sopin_7_2016}, было получено распределение стационарных вероятностей~$q_k(\mathbf{r})$ 
с~объединенным входящим потоком и~средневзвешенным требованием 
$$
p_{\mathbf{r}}= \sum\limits_{l=1}^{L} \fr{\rho_l}{\rho} \,p_{l,\mathbf{r}},$$
 где 
$\rho\hm=\sum\nolimits_{l=1}^{L}\rho_l$:
\begin{equation*}
q_k(\mathbf{r})=q_0\fr{\rho^k}{k!}\,p_{\mathbf{r}}^{(k)}\,, \quad q_0=\left( 
\sum\limits_{k=0}^{N}\sum\limits_{\mathbf{r}=\mathbf{0}}^{\mathbf{R}} p_{\mathbf{r}}^{(k)} 
\right)^{-1},
\end{equation*}
что позволило выразить вероятность блокировки и~среднего объема занятых ресурсов 
в~аналитическом виде.

В случае рассматриваемой СМО, но с~заявками одного класса, 
в~\cite{Sopin_11_2017} представлены выражения для стационарных вероятностей 
состояний числа заявок в~системе и~суммарного объема занятых ресурсов, а~также 
формулы для вероятности блокировки и~среднего объема занятых ресурсов.


\section{Вычисление характеристик ресурсных систем массового обслуживания}

В работе~\cite{Naumov_1_2014} для системы с~одним типом ресурса показано, что 
в~предположении о~гам\-ма-рас\-пре\-де\-ле\-нии необходимого заявкам ресурса плотность 
распределения высвобождаемого заявкой ресурса при заданном числе заявок 
в~системе и~заданном векторе суммарных объемов занятых ресурсов совпадает 
с~бе\-та-рас\-пре\-де\-ле\-ни\-ем, позволяющим легко рассчитывать многократные свертки, к~которым 
приводит необходимость учитывать объемы всех заявок в~системе. В~остальных 
случаях наличие в~формулах большого числа сверток создает значительную 
вычислительную сложность при расчете стационарных характеристик РСМО.
Так, для расчета характеристик СМО из~\cite{Sopin_7_2016} необходимо для каж\-до\-го 
$k\hm\in \{0,\ldots,N\}$, а~также всех наборов векторов $\mathbf{r}\hm \leq \mathbf{R}$ 
хранить в~памяти компьютера значения сверток вероятностей~$p_{\mathbf{r}}$. 
А~при больших значениях~$N$ и~$\mathbf{R}$ вычисление вероятностей блокировок 
системы и~также объемов занятого ресурса по представленным формулам вообще 
нерационально. Поэтому задача получения действенных численных методов является 
крайне важной.
В~работе~\cite{Sopin_8_2017} для модели СМО из~\cite{Sopin_7_2016}, чтобы 
сократить вычисления, был предложен рекуррентный алгоритм вычисления 
нормировочной константы $G(N,\mathbf{R})\hm=q_0^{-1}$, основанный на алгоритме 
Бузена~\cite{Buzen}. Кроме того, на основе разработанного алгоритма были 
получены рекуррентные формулы для вычисления вероятностных характеристик 
сис\-те\-мы: вероятности блокировки сис\-те\-мы, среднего объема дисперсии занятых 
ресурсов.
Если обозначить
\begin{equation*}
G(n,\mathbf{r})\sum\limits_{k=0}^{n}\sum\limits_{\mathbf{j}=\mathbf{0}}^{\mathbf{r}}p_{\mathbf
{j}}^{(k)}\fr{\rho^k}{k!}\,, \enskip 
n\geq 0\,, \enskip \mathbf{r}\geq \mathbf{0}\,,
\end{equation*}
то функция $G(n,\mathbf{r})$ будет удовлетворять следующему рекуррентному 
соотношению:
\begin{multline*}
G(n,\mathbf{r})=G(n-1,\mathbf{r})+{}\\
{}+\fr{\rho}{n!}
\sum\limits_{\mathbf{j}=\mathbf{0}}^{\mathbf{r}}p_{\mathbf{j}}
\left( G(n-1,\mathbf{r}-\mathbf{j})- G(n-2,\mathbf{r}-\mathbf{j})   \right)
\end{multline*}
с начальными условиями
\begin{equation*}
G(0,\mathbf{r})=1,\enskip \mathbf{r}\geq 0\,; \quad
G(1,\mathbf{r})=1+\sum\limits_{\mathbf{j}=\mathbf{0}}^{\mathbf{r}}p_{\mathbf{j}}\,.
\end{equation*}
При анализе РСМО,  описывающих M2M (machine-to-machine) трафик в~сетях LTE, 
аналогичный рекуррент\-ный алгоритм для вычисления нормировочной константы был 
разработан в~\cite{Sopin_11_2017}. Мат\-рич\-ные методы анализа РСМО, применимые при 
моделировании соты сети LTE с~двумя типами трафика, M2M и~H2H (human-to-human), 
предложены в~работах~\cite{Vish_2017,Vish_2016}.

\section{Заключение}

В настоящем обзоре кратко представлены основные разновидности 
РСМО, существующие методы их анализа, выражения для оценки 
основных ве\-ро\-ят\-но\-ст\-но-вре\-мен\-н$\acute{\mbox{ы}}$х 
характеристик и~алгоритмы их вычисления.

{\small\frenchspacing
 {%\baselineskip=10.8pt
 \addcontentsline{toc}{section}{References}
 \begin{thebibliography}{99}
%1
\bibitem{Andrews} %1
\Au{Andrews J.\,G., Buzzi~S., Choi~W., Hanly~S.\,V., Lozano~A., Soong~A.\,C.\,K., 
Zhang~J.\,C.} What will 5G be?~// {IEEE J.~Sel. Area.  
Comm.}, 2014. Vol.~32. No.\,6. P.~1065--1082.



%3
\bibitem{Samuylov} %2
\Au{Buturlin I.\,A., Gaidamaka~Y.\,V., Samuylov~A.\,K.}
Utility function maximization problems for two cross-layer optimization 
algorithms in OFDM wireless networks~// {4th Congress (International) on Ultra 
Modern Telecommunications and Control Systems}, 2012.  P.~63--65.

%2
\bibitem{Galinina} %3
\Au{Galinina O., Andreev~S.\,D., Gerasimenko~M., Kou\-che\-rya\-vy~Y., Himayat~N., 
Yeh~S.\,P., Talwar~S.} Capturing spatial randomness of heterogeneous 
cellular/WLAN deployments with dynamic traffic~// {IEEE J.~Sel. 
Area. Comm.}, 2014. Vol.~32. No.\,6. P.~1083--1099.

%4
\bibitem{Naumov_3_2016}
\Au{Наумов В.\,А., Самуйлов~К.\,Е., Самуйлов~А.\,К.} О~суммарном объеме 
ресурсов, занимаемых обслуживаемыми заявками~// {Автоматика и~телемеханика}, 
2016. №.\,8. С.~125--135.

%5
\bibitem{Gimpelson}
\Au{Gimpelson L.\,A.}
Analysis of mixtures of wide- and narrow-band traffic~// {IEEE T.~Commun.
 Techn.}, 1968. Vol.~13. No.\,3. P.~258--266.

%6
\bibitem{Kelly}
\Au{Kelly F.\,P.}
Loss networks~// {Ann. Appl. Probab.}, 1991. No.\,1. P.~319--378.

%7
\bibitem{Ross}
\Au{Ross K.\,W.}
Multiservice loss models for broadband telecommunication networks.~--- {London: 
Springer-Verlag}, 1995. 343~p.

%8
\bibitem{Basharin}
\Au{Basharin G.\,P., Samouylov~K.\,E., Yarkina~N.\,V., Gudkova~I.\,A.}
A~new stage in mathematical teletraffic theory~// {Automat. Rem. 
Contr.}, 2009. Vol.~70. No.\,12. P.~1954--1964.

%9
\bibitem{Romm_21_1971}
\Au{Ромм Э.\,Л., Скитович~В.\,В.}
Об одном обобщении задачи Эрланга~// {Автоматика и~телемеханика}, 1971. №.\,6. 
С.~164--168.

%10
\bibitem{Kac}
\Au{Кац Б.\,А.}
Об обслуживании сообщений случайной длины~// {Теория массового обслуживания: 
Тр. 3-й Всесоюзн. шко\-лы-со\-ве\-ща\-ния по тео\-рии массового обслуживания}, 1976. 
С.~157--168.

%11
\bibitem{Naumov_6_2015}
\Au{Наумов В.\,А., Самуйлов~А.\,К.}
Модель выделения ресурсов беспроводной сети объемами случайной величины~// 
{Вестник РУДН. Серия: Математика, информатика, физика}, 2015. №\,2. С.~38--45.

%12
\bibitem{Naumov_10_2015}
\Au{Naumov~V., Samouylov~K., Yarkina~N., Sopin~E., Andreev~S., Samuylov~A.}
LTE performance analysis using queuing systems with finite resources and random 
requirements~// {7th Congress on Ultra Modern Telecommunications and Control 
Systems}.~--- IEEE, 2015. P.~100--103.

%13
\bibitem{Naumov_15_2017}
\Au{Naumov V., Samouylov~K.}
Analysis оf multi-resource loss system with state dependent arrival and service 
rates~// {Probab.  Eng. Inform. Sc.}, 2017. 
Vol.~31. No.\,4. P.~413--419.

%14
\bibitem{Sopin_12_2017}
\Au{Samouylov K., Sopin~E., Vikhrova~O.}
Analysis of queueing system with resources and signals~// {Comm.  
Com. Inf. Sc.}, 2017. Vol.~800. P.~358--369.

%15
\bibitem{Sopin_13_2017}
\Au{Sopin E., Vikhrova~O., Samouylov~K.}
LTE network model with signals and random resource requirement~// {9th 
Congress (International) on Ultra Modern Telecommunications and Control Systems 
and Workshops}.~--- IEEE, 2017. P.~101--106.



%17
\bibitem{Dohler_2017}
\Au{Petrov V., Solomitckii~D., Samuylov~A., Lema Maria~A., Gapeyenko~M., 
Moltchanov~D., Andreev~S., Naumov~V., Samouylov~K., Dohler~M., Koucheryavy~Ye.}
Dynamic multi-connectivity performance in ultra-dense urban mmWave deployments~// 
{IEEE J.~Sel. Area. Comm.}, 2017. Vol.~35. No.~9. 
P.~2038--2055.

%16
\bibitem{Naumov_14_2017}
\Au{Наумов В.\,А., Самуйлов~К.\,Е.}
Анализ сетей ресурсных систем массового обслуживания~// {Автоматика 
и~телемеханика}, 2018. №\,5. С.~59--68.

%18
\bibitem{Sopin_4_2015}
\Au{Samouylov K., Sopin~E., Vikhrova~O.}
Analyzing blocking probability in LTE wireless network via queuing system with 
finite amount of resources~// {Comm.  Com. Inf.
Sc.}, 2015. Vol.~564. P.~393--403.

%19
\bibitem{Sopin_5_2015}
\Au{Вихрова О.\,Г., Самуйлов~К.\,Е., Сопин~Э.\,С., Шоргин~С.\,Я.}
К~анализу показателей качества обслуживания в~современных беспроводных сетях~// 
{Информатика и~её применения}, 2015. Т.~9. Вып.~4. С.~48--55.

%20
\bibitem{Sopin_7_2016}
\Au{Sopin~E., Samouylov~K., Vikhrova~O., Kovalchukov~R., Moltchanov~D., 
Samuylov~A.}
Evaluating a case of downlink uplink decoupling using queuing system with random 
requirement~// 
Internet of Things, smart spaces, and
next generation
networks and systems~/
Eds. O.~Galinina, S.\,I.~Balandin, Y.~Koucheryavy.~---
{Lecture notes in computer science ser.}~--- Springer, 2016. Vol.~9870. P.~440--450.

%21
\bibitem{Sopin_8_2017}
\Au{Samouylov K., Sopin~E., Vikhrova~O., Shorgin~S.}
Convolution algorithm for normalization constant evaluation in queuing system 
with random requirements~// {AIP Conf. Proc.}, 2017. Vol.~1863. Art. 
No.\,090004. 4~p.

%22
\bibitem{Vihrova_9_2017}
\Au{Вихрова О.\,Г.}
К~вычислению вероятностных характеристик СМО ограниченной емкости со случайными 
требованиями к~ресурсам~// {Вестник РУДН. Серия: Математика, информатика, 
физика}, 2017. №\,3. С.~203--210.



%24
\bibitem{Sopin_17_2018}
\Au{Sopin E., Samouylov~K.}
On the analysis of the limited resources queuing system under MAP arrivals~// 
{Conference (International)  on Applied Mathematics, Computational Science and 
Systems Engineering}, 2018. Vol.~16. Art. No.\,01008. 4~p.

%23
\bibitem{Sopin_11_2017}
\Au{Sopin E., Gaidamaka~Yu., Markova~E., Vikhrova~O.}
Performance analysis of M2M traffic in LTE network using queuing systems with 
random resource requirements~// {Autom. Control Comp.~S.}, 2018 
(in press).

%25
\bibitem{Naumov_1_2014}
\Au{Наумов В.\,А., Самуйлов~К.\,Е.}
О~моделировании систем массового обслуживания с~множественными ресурсами~// 
{Вестник РУДН. Серия: Математика, информатика, физика}, 2014. №\,3. С.~60--64.

%26
\bibitem{Buzen}
\Au{Buzen J.\,P.}
Computational algorithms for closed queueing networks with exponential servers~// 
{Commun. ACM}, 1973. Vol.~16. P.~527--531.



%28
\bibitem{Vish_2016}
\Au{Вишневский В.\,М., Самуйлов~К.\,Е., Наумов~В.\,А., Яркина~Н.\,В.}
Модель соты LTE с~межмашинным трафиком в~виде мультисервисной системы массового 
обслуживания с~эластичными и~потоковыми заявками и~марковским входящим потоком~// 
{Вестник РУДН. Серия: Математика, информатика, физика}, 2016. №\,4. С.~26--36.

%27
\bibitem{Vish_2017}
\Au{Vishnevsky~V., Samouylov~K., Naumov~V., Krishnamoorty~A., Yarkina~N.}
Multiservice queieing system with map arrivals for modelling LTE cell with H2H 
and M2M communications and M2M aggregation~// {Comm. Com. 
Inf. Sc.}, 2017. Vol.~700. P.~63--74.

 \end{thebibliography}

 }
 }

\end{multicols}

\vspace*{-6pt}

\hfill{\small\textit{Поступила в~редакцию 16.06.18}}

\vspace*{6pt}

%\newpage

%\vspace*{-24pt}

\hrule

\vspace*{2pt}

\hrule

\vspace*{-2pt}


\def\tit{RESOURCE QUEUING SYSTEMS AS~MODELS OF~WIRELESS COMMUNICATION SYSTEMS}


\def\titkol{Resource queuing systems as~models of~wireless communication systems}

\def\aut{A.\,V.~Gorbunova$^1$, V.\,A.~Naumov$^2$, Yu.\,V.~Gaidamaka$^{1,3}$, and~K.\,E.~Samouylov$^{1,3}$}

\def\autkol{A.\,V.~Gorbunova, V.\,A.~Naumov, Yu.\,V.~Gaidamaka, and~K.\,E.~Samouylov}

\titel{\tit}{\aut}{\autkol}{\titkol}

\vspace*{-11pt}


\noindent
$^1$Peoples' Friendship University of Russia 
(RUDN University), 6~Miklukho-Maklaya Str., Moscow 117198, Russian\linebreak
$\hphantom{^1}$Federation

\noindent
$^2$Service Innovation Research Institute, 8A~Annankatu, Helsinki 
00120, Finland

\noindent
$^3$Institute of Informatics Problems, 
Federal Research Center ``Computer Science and Control'' 
of the Russian\linebreak
$\hphantom{^1}$Academy of Sciences, 44-2~Vavilov Str., Moscow 119333, 
Russian Federation


\def\leftfootline{\small{\textbf{\thepage}
\hfill INFORMATIKA I EE PRIMENENIYA~--- INFORMATICS AND
APPLICATIONS\ \ \ 2018\ \ \ volume~12\ \ \ issue\ 3}
}%
 \def\rightfootline{\small{INFORMATIKA I EE PRIMENENIYA~---
INFORMATICS AND APPLICATIONS\ \ \ 2018\ \ \ volume~12\ \ \ issue\ 3
\hfill \textbf{\thepage}}}

\vspace*{3pt}


 
\Abste{The article presents an overview of the resource queuing 
systems used for modeling of a~wide class of real systems with 
admittedly limited resources. Despite the objective importance of studying 
of such systems, there have been very few works devoted to their 
analysis until recently, which was due to the complexity of constructing
a~random process to describe their functioning and, accordingly, of obtaining 
the numerical results. However, in
recent years, there has been 
a~significant shift in the study of the resource systems~--- new 
methods for their analysis have been proposed, which made it possible to 
construct recursive algorithms suitable for the numerical calculations.
In this regard, the current review reflects only a part of the previously 
obtained results, namely, it considers\linebreak\vspace*{-12pt}}

\Abstend{the resource systems without waiting 
space with exponentially distributed service time. The authors consider the models 
of wireless communication systems based on resource queuing systems, expressions 
for estimating the main 
probabilistic, and temporal characteristics and algorithms for their calculation.}

\KWE{resource queueing systems; continuous resource; discrete resource; 
limited resource; recursive algorithm; heterogeneous network; 
stationary distribution; semi-Markov process; wireless communication systems}
 
\DOI{10.14357/19922264180307}

%\vspace*{-14pt}

\Ack
\noindent
The work was partly supported by the Russian Ministry of Education and
Science
(project 2.882.2017/4.6).



%\vspace*{6pt}

  \begin{multicols}{2}

\renewcommand{\bibname}{\protect\rmfamily References}
%\renewcommand{\bibname}{\large\protect\rm References}

{\small\frenchspacing
 {%\baselineskip=10.8pt
 \addcontentsline{toc}{section}{References}
 \begin{thebibliography}{99}
\bibitem{1-gor}
\Aue{Andrews, J.\,G., S.~Buzzi, W.~Choi, S.\,V.~Hanly, A.~Lozano, 
A.\,C.\,K.~Soong, and J.\,C.~Zhang.} 2014. What will 5G be? 
\textit{IEEE J.~Sel. Area. Comm.} 32(6):1065--1082.

\bibitem{3-gor}
\Aue{Buturlin, I.\,A., Y.\,V.~Gaidamaka, and A.\,K.~Samuylov.} 
2012. Utility function maximization problems for two cross-layer optimization 
algorithms in OFDM wireless networks. 
\textit{4th Congress (International) on Ultra Modern Telecommunications and 
Control Systems}. 63--65.

\bibitem{2-gor}
\Aue{Galinina, O.\,S., D.~Andreev, M.~Gerasimenko, Y.~Koucheryavy, N.~Himayat, 
S.\,P.~Yeh, and S.~Talwar.} 2014. Capturing spatial randomness of heterogeneous 
cellular/WLAN deployments with dynamic traffic. 
\textit{IEEE J.~Sel. Area. Comm.} 32(6):1083--1099.
\bibitem{4-gor}
\Aue{Naumov, V.\,A., K.\,E.~Samuilov, and A.\,K.~Samuilov.} 2016. 
On the total 
amount of resources occupied by serviced customers. 
\textit{Automat. Rem. Contr.} 77(8):1419--1427.
\bibitem{5-gor}
\Aue{Gimpelson, L.\,A.} 1968. Analysis of 
mixtures of wide- and narrow-band traffic. 
\textit{IEEE T.~Commun. Techn.} 13(3):258--266.
\bibitem{6-gor}
\Aue{Kelly, F.\,P.} 1991. Loss networks. \textit{Ann. Appl. Probab.} 1:319--378.
\bibitem{7-gor}
\Aue{Ross, K.\,W.} 1995. \textit{Multiservice loss models for broadband telecommunication 
networks}. London: Springer-Verlag. 343~p.
\bibitem{8-gor}
\Aue{Basharin, G.\,P., K.\,E.~Samouylov, N.\,V.~Yarkina, and I.\,A.~Gudkova.} 2009. 
A~new stage in mathematical teletraffic theory. 
\textit{Automat. Rem. Contr.} 70(12):1954--1964.
\bibitem{9-gor}
\Aue{Romm, E.\,L., and V.\,V.~Skitovich.} 1971. Ob odnom obobshchenii zadachi Erlanga 
[On a generalization of the Erlang problem]. 
\textit{Automat. Rem. Contr.} 6:164--168.
\bibitem{10-gor}
\Aue{Kats, B.\,A.} 1976. Ob obsluzhivanii soobshcheniy sluchaynoy dliny 
[On serving messages of random length]. 
\textit{Teoriya massovogo obsluzhivaniya. Tr. 3~Vsesoyuzn. 
shkoly-soveshchaniya po teorii massovogo obsluzhivaniya} 
[Queuing Theory: 3rd All-Union School-Seminar on Queuing Theory Proceedings]. 157--168.
\bibitem{11-gor}
\Aue{Naumov, V.\,A., and A.\,K.~Samuylov.} 
2015. Model' vydeleniya resursov besprovodnoy seti ob''emami sluchaynoy velichiny 
[Queuing system with resource allocation of the random volume]. 
\textit{RUDN J.~Math. 
Information Sci. Phys.} 2:38--45.
\bibitem{12-gor}
\Aue{Naumov, V., K.~Samouylov, N.~Yarkina, E.~Sopin, S.~Andreev, and A.~Samuylov.}
2015. LTE performance analysis using queuing systems with finite resources 
and random requirements. 
\textit{7th Congress on Ultra Modern Telecommunications and Control Systems}. 
IEEE. 100--103.
\bibitem{13-gor}
\Aue{Naumov, V., and K.~Samouylov.} 2017. Analysis оf multi-resource loss 
system with state dependent arrival and service rates. 
\textit{Probab. Eng. Inform. Sc.} 31(4):413--419.
\bibitem{14-gor}
\Aue{Samouylov, K., E.~Sopin, and O.~Vikhrova.} 2017. 
Analysis of queueing system with resources and signals. 
\textit{Comm. Com. Inf. Sc.} 800:358--369.
\bibitem{15-gor}
\Aue{Sopin, E., O.~Vikhrova, and K.~Samouylov.} 2017. 
LTE network model with signals and random resource requirement. 
\textit{9th Congress (International) on Ultra Modern Telecommunications and 
Control Systems and Workshops}. 101--106.

\bibitem{17-gor}
\Aue{Petrov, V., D.~Solomitckii, A.~Samuylov, A.~Maria Lema, M.~Gapeyenko, 
D.~Moltchanov, S.~Andreev, V.~Naumov, K.~Samouylov, M.~Dohler, and Ye.~Koucheryavy}. 
2017. Dynamic multi-connectivity performance in ultra-dense urban 
mmWave deployments. \textit{IEEE J.~Sel. Area. Comm.} 35(9):2038--2055.

\bibitem{16-gor}
\Aue{Naumov, V.\,A., and K.\,E.~Samuilov.} 2018. 
Analysis of networks of the resource queuing systems. 
\textit{Automat. Rem. Contr}. 79(5):822--829.
\bibitem{18-gor}
\Aue{Samouylov, K., E.~Sopin, and O.~Vikhrova.} 2015. 
Analyzing blocking probability in LTE wireless network via queuing system 
with finite amount of resources. 
\textit{Comm. Com. Inf. Sc.} 564:393--403.
\bibitem{19-gor}
\Aue{Vikhrova, O.\,G., K.\,E.~Samouylov, E.\,S.~Sopin, and S.\,Ya.~Shorgin.} 
2015. K~analizu pokazateley kachestva obsluzhivaniya 
v~sovremennykh besprovodnykh setyakh [On performance analysis of modern 
wireless networks]. \textit{Informatika i~ee Primeneniya~---
Inform. Appl.} 9(4):48--55.
\bibitem{20-gor}
\Aue{Sopin, E., K.~Samouylov, O.~Vikhrova, R.~Kovalchukov, D.~Moltchanov, and 
A.~Samuylov.} 2016. Evaluating a~case of downlink uplink decoupling 
using queuing system with random requirement. 
\textit{Internet of Things, smart spaces, and
next generation
networks and systems}.
Eds. O.~Galinina, S.\,I.~Balandin, and Y.~Koucheryavy.
{Lecture notes in computer science ser.} Springer. 9870:440--450.
\bibitem{21-gor}
\Aue{Samouylov, K., E.~Sopin, O.~Vikhrova, and S.~Shorgin.} 
2017. Convolution algorithm for normalization constant evaluation in 
queuing system with random requirements. \textit{AIP Conf. Proc}. 
1863:090004. 4~p.
\bibitem{22-gor}
\Aue{Vikhrova, O.\,G.} 2017. 
K~vychisleniyu veroyatnostnykh kharakteristik SMO ogranichennoy emkosti so 
sluchaynymi trebovaniyami k~resursam [About probability characteristics evaluation 
in queuing system with limited resources and random requirements]. 
\textit{RUDN J.~Math. Information Sci. Phys.} 25(3):203--210.

\bibitem{24-gor}
\Aue{Sopin, E., and K.~Samouylov.} 2018. On the analysis of the limited resources 
queuing system under MAP arrivals. 
\textit{Conference (International) 
on Applied Mathematics, Computational Science and Systems Engineering}. 16:01008. 4~p.

\bibitem{23-gor}
\Aue{Sopin, E., Yu.~Gaidamaka, E.~Markova, and O.~Vikhrova.} 2018 (in press).
 Performance analysis of M2M traffic in LTE network using queuing systems 
 with random resource requirements. 
 \textit{Autom. Control Comp.~S.}
\bibitem{25-gor}
\Aue{Naumov, V.\,A., and K.\,E.~Samuylov.} 2014. 
O~modelirovanii sistem massovogo obsluzhivaniya s~mnozhestvennymi resursami 
[On the modeling of queueing systems with multiple resources]. 
\textit{[RUDN J.~Math. Information Sci. Phys.} 3:60--64.
\bibitem{26-gor}
\Aue{Buzen, J.\,P.} 1973. Computational algorithms for closed queueing 
networks with exponential servers. \textit{Commun. ACM} 16:527--531.

\bibitem{28-gor}
\Aue{Vishnevsky, V.\,M., K.\,E.~Samouylov, V.\,A.~Naumov, and N.\,V.~Yarkina.}
2016. Model' soty LTE s~mezhmashinnym trafikom v~vide mul'tiservisnoy sistemy 
massovogo obsluzhivaniya s~elastichnymi i~potokovymi zayavkami 
i~markovskim vkhodyashchim potokom [Multiservice queuing system with
 elastic and streaming flows and markovian arrival process for modelling 
 LTE cell with M2M traffic]. 
 \textit{RUDN J.~Math. Information Sci. Phys.} 4:26--36.
 
 \bibitem{27-gor}
\Aue{Vishnevsky, V., K.~Samouylov, V.~Naumov, A.~Krishnamoorty, and N.~Yarkina.} 2017.
Multiservice queieing system with map arrivals for modelling LTE cell with H2H 
and M2M communications and M2M aggregation. 
\textit{Comm. Com. Inf. Sc.} 700:63--74.
 
 \end{thebibliography}

 }
 }

\end{multicols}

\vspace*{-6pt}

\hfill{\small\textit{Received June 16, 2018}}

%\pagebreak

%\vspace*{-18pt}

 
\Contr

\noindent
\textbf{Gorbunova Anastasiya V.} (b.\ 1986)~--- 
Candidate of Science (PhD) in physics and mathematics, 
assistant professor, Peoples' Friendship University of Russia 
(RUDN University), 6~Miklukho-Maklaya Str., 
Moscow 117198, Russian Federation; \mbox{gorbunova\_av@rudn.university}

\vspace*{3pt}

\noindent
\textbf{Naumov Valeriy A.} (b.\ 1950)~--- 
Candidate of Science (PhD) in physics and mathematics, 
Research Director, Service Innovation Research Institute, 8A~Annankatu, Helsinki 
00120, Finland; \mbox{valeriy.naumov@pfu.fi}

\vspace*{3pt}

\noindent
\textbf{Gaidamaka Yuliya V.} (b.\ 1971)~--- Doctor of Science in physics and mathematics, 
professor, Peoples' Friendship University of Russia 
(RUDN University), 6~Miklukho-Maklaya Str., 
Moscow 117198, Russian Federation;\linebreak senior scientist, 
Institute of Informatics Problems, 
Federal Research Center\ ``Computer Science and Control''\linebreak 
of the Russian Academy of Sciences, 44-2~Vavilov Str., Moscow 119333, 
Russian Federation; \mbox{gaydamaka\_yuv@rudn.university}

\vspace*{3pt}

\noindent
\textbf{Samuylov Konstantin E.} (b.\ 1955)~--- Doctor of Science in technology, 
professor, Head of Department, Peoples' Friendship University of Russia 
(RUDN University), 6~Miklukho-Maklaya Str., 
Moscow 117198, Russian Federation; senior scientist, 
Institute of Informatics Problems, Federal Research Center 
``Computer Science and Control'' of the Russian Academy of Sciences, 
44-2~Vavilov Str., Moscow 119333, Russian Federation; 
\mbox{samuylov\_ke@rudn.university}

\label{end\stat}

\renewcommand{\bibname}{\protect\rm Литература}        %7
\def\stat{krivenko}

\def\tit{МНОГОМЕРНЫЙ РЕФЕРЕНСНЫЙ РЕГИОН\\ ВЫСОКОЙ ПЛОТНОСТИ}

\def\titkol{Многомерный референсный регион высокой плотности}

\def\aut{М.\,П.~Кривенко$^1$}

\def\autkol{М.\,П.~Кривенко}

\titel{\tit}{\aut}{\autkol}{\titkol}

\index{Кривенко М.\,П.}
\index{Krivenko M.\,P.}


%{\renewcommand{\thefootnote}{\fnsymbol{footnote}} \footnotetext[1]
%{Работа выполнена при финансовой поддержке РФФИ (проекты 16-07-00677 
%и~15-37-20611-мол\_а\_вед).}}


\renewcommand{\thefootnote}{\arabic{footnote}}
\footnotetext[1]{Институт проблем информатики Федерального исследовательского центра <<Информатика и~управление>> Российской академии наук,
\mbox{mkrivenko@ipiran.ru}}

\vspace*{4pt}



\Abst{Рассматриваются принципы построения многомерных референсных регионов
(MRR~--- multivariate reference region). 
Предложен оригинальный метод построения региона на основе областей с~высокой 
плотностью точек и~аппроксимации распределения данных с~помощью смеси нормальных 
распределений. Для оценки порога для плотности распределения используется  
бут\-стреп-ме\-тод. В~качестве эксперимента рассмотрена задача построения 
и~использования эталонной области для прогнозирования типа мочевого камня. Обработка 
реальных данных продемонстрировала преимущества предлагаемых решений.}

\KW{многомерный референсный регион; область высокой плотности; бут\-стреп-ме\-тод; 
смесь многомерных нормальных распределений}

\vspace*{6pt}

\DOI{10.14357/19922264170207} 


\vskip 10pt plus 9pt minus 6pt

\thispagestyle{headings}

\begin{multicols}{2}

\label{st\stat}

\section{Введение}

     Многомерный референсный регион 
был предложен в~литературе по клинической химии в~начале 1970-х~гг.\ как 
альтернатива одномерным референсным интервалам~[1]. Там излагались 
преимущества предлагаемых множественных тестов, хоть и~имеющих 
упрощенный вид, но снижающих (по отношению к~одномерным вариантам) 
число ложных положительных результатов. Появление MRR оказалось 
особенно привлекательным для интерпретации результатов наборов 
медицинских тестов. Тем не менее возникали трудности в~построении 
и~использовании процедур многомерного анализа (см., например,~[2]), 
связанные, в~частности, с~быстрым увеличением числа параметров, которые 
должны быть оценены. Немногие лаборатории использовали MRR в~своей 
практике, причем в~экспериментальном режиме, и,~как следствие, на 
сегодняшний день имеется относительно малое количество соответствующих 
публикаций. 

\vspace*{-6pt}

\section{Многомерный референсный регион на основе расстояния Махалонобиса}

\vspace*{-2pt}

     Одномерный референсный интервал, полученный статистическим путем, 
использует центральную часть значений анализируемого показателя, обычно 
соответствующую~95\% некоторой популяции~--- совокупности особей 
определенного вида (например, здоровой части населения определенного пола 
из некоторого диапазона возрастов). Одномерные референсные интервалы 
применялись в~течение многих лет в~качестве стандартного приема 
интерпретации лабораторных данных. Они легко формируются, хранятся, 
извлекаются и~передаются в~лабораторных информационных системах, просты 
в~понимании, хорошо воспринимаются медицинским сообществом в~ходе 
длительного использования. Тем не менее одномерные референсные интервалы 
при классификации данных могут дать большое число ложно аномальных 
результатов. Этот далеко не единственный недостаток однофакторного 
референсного интервала может быть полностью или частично устранен 
с~помощью MRR.
     
     Простейшим и~весьма распространенным способом построения MRR 
является использование прямого произведения отдельных референсных 
интервалов в~предположении, что они статистически независимы. Пусть 
$(1\hm-\alpha)$~--- вероятность попадания в~MRR, а~$p_0$~--- вероятность 
попадания в~референсный интервал для любого из~$d$~признаков, тогда 
$p_0\hm= \sqrt[d]{1-\alpha}$. С~ростом размерности~$d$ значения~$p_0$ 
быстро приближаются к~1, что фактически лишает смысла применение MRR.
     
     Как и~в одномерном случае, отправной точкой для построения MRR 
может стать нормальное распределение. Идеи центрального расположения 
референсного региона и~заданной вероятности попадания в~него приводят для 
$d$-мер\-но\-го нормального распределения, имеющего плотность 
распределения
     \begin{multline*}
     \varphi(y,\mu,\Sigma) ={}\\
     {}=(2\pi)^{-d/2}\vert\Sigma\vert^{-1/2}\exp \left( -\fr{\left(y-
\mu\right)^{\mathrm{T}} \Sigma^{-1}(y-\mu)}{2}\right),
   \end{multline*}
где величина $(y-\mu)^{\mathrm{T}} \Sigma^{-1} (y-\mu)$ есть квадрат так 
называемого расстояния Махаланобиса между~$y$ и~$\mu$, к~использованию 
многомерного эллипсоида
\begin{multline*}
(2\pi)^{-d/2}\vert\Sigma\vert^{-1/2}\exp \left( -\fr{\left(y-\mu\right)^{\mathrm{T}}
\Sigma^{-1} 
(y-\mu)}{2}\right) ={}\\
{}=const
\end{multline*}
или, что то же самое, 
$$ 
(y-\mu)^{\mathrm{T}} \Sigma^{-1}(y-\mu)=const\,.
$$
Его называют эллипсоидом равной плотности распределения (или просто 
эллипсоидом равной вероятности). 
     
     Если задаться вероятностью $(1\hm-\alpha)$ попадания в~эллипсоид 
равной вероятности вида $(y\hm-\mu)^{\mathrm{T}}\Sigma^{-1} (y\hm-\mu)\hm= 
\rho$, то параметр~$\rho$ удовлетворяет уравнению $\mathrm{Pr}\left\{ 
\chi_d^2\leq \rho\right\} \hm=1\hm-\alpha$.
     
     Использование эллипсоида в~качестве MRR будет оправдано только 
тогда, когда исходное распределение данных есть многомерное нормаль-\linebreak ное. 
Поэтому становятся актуальными критерии\linebreak подгонки, а~также использование 
процедур норма\-ли\-зации распределения данных в~многомерном\linebreak случае.
 Если 
с~помощью тестов выявляется, что распределение не является нормальным, то 
Международная федерация клинической химии и~лабораторной медицины 
рекомендует, согласно~[3], использовать двухступенчатую процедуру 
нормализации. Следует обратить внимание, что многошаговость здесь 
относится не к~многомерности, а касается лишь покоординатного 
преобразования распределения данных к~нормальному.
     
     Первые же попытки применения MRR на основе расстояния 
Махалонобиса (фактически это означает принятие модели нормального 
распределения референсных значений) выявили ряд недостатков (более 
подробно смотри в~\cite[разд.~6.2]{4-kri}):
     \begin{itemize}
\item проявление <<проклятий>> размерности при механическом 
увеличении~$d$, в~особенности если игнорируется этап анализа состава 
признаков~[1, 5, 6];
\item из-за небольших объемов обучающей выборки невысокая устойчивость 
при применении, в~частности чувствительность к~увеличению неточностей 
измерений после того, как регион был установлен~\cite{5-kri, 7-kri}. 
\item предположение о нормальном распределении и~попытки <<подправить>> 
действительность с~помощью преобразований реальных данных для их 
нормализации при увеличении размерности данных становятся все более 
шаткими~\cite{5-kri};
\item представление и~интерпретация выводов на основе MRR трудно 
понимаемы не только для специалистов в~предметной области~[8].
\end{itemize}

\vspace*{-9pt}

\section{Многомерный референсный регион высокой плотности}

\vspace*{-2pt}

     Заметим, что в~случае нормального распределения референсных значений 
для точек внут\-ри построенного эллипсоида значения плотности\linebreak распределения 
больше, чем на границе, а~вне~--- меньше. Это замечание позволяет 
предложить другой подход к~построению MRR.
     
     \smallskip
     
     \noindent
     \textbf{Определение.}\ Eсли плотность распределения референсных 
значений есть $f(y)$, то MRR есть область $A_t\hm= \left\{ y\in 
\mathcal{R}^d\vert f(y)\hm\geq t\right\}$ для некоторого порогового 
значения~$t$. 
     
     \smallskip
     
     Для нормального распределения это уже упомянутый эллипсоид равной 
вероятности. Если задается вероятность $(1\hm-\alpha)$ попадания в~$A_t$, то 
пороговое значение~$t$ есть решение уравнения $\int\nolimits_{A_t} 
f(u)\,du\hm=1\hm-\alpha$, получить которое аналитически в~случае 
произвольной плотности распределения вряд ли возможно. Здесь присутствуют 
две проблемы: вычисление многомерного интеграла и~зависимость области 
интегрирования от неизвестного значения. Для решения их предлагается 
привлечь метод моделирования.
     
     Сгенерируем выборку из $f(y)$, которую обозначим как $Y^f\hm= \left\{ 
y_1^f, \ldots, y_m^f\right\}$. Для оценки $\int\nolimits_{A_t} f(u)\,du$ 
используем отношение:

\noindent
\begin{multline*}
     \fr{\left\vert \left\{ y_i^f\vert y_i^f\in A_t\right\}\right\vert }{m} =
      \fr{\left\vert\left\{ y_i^f\vert 
f\left(y_i^f\right) \geq t\right\}\right\vert }{m} ={}\\
{}= 1-\fr{\left\vert \left\{ y_i^f\vert f(y_i^f)<t\right\}\right\vert }{m}=1-
F_m(t)\,,
     \end{multline*}
где $F_m(t)$~--- эмпирическая функция распределения случайной 
величины~$f(y)$, т.\,е.\ случайной величины, являющейся результатом 
преобразования с~помощью функции~$f(\cdot)$ случайной величины, име\-ющей 
плотность распределения~$f(u)$.

     Таким образом, искомая оценка~$t^*$ должна удовле\-тво\-рять уравнению 
$F_m(t^*)\hm=\alpha$ и~может быть получена как непараметрическая оценка 
квантиля\linebreak\vspace*{-12pt}

\pagebreak

\noindent
 порядка~$\alpha$ из распределения $F_m(\cdot)$. Если обозначить 
$f_i\hm= f(y_i^f)$, то~$t^*$ есть~$f_{(r)}$, где
     $$
     r= \begin{cases}
     m\alpha, &\ m\alpha~\mbox{---~целое}\,;\\
     \lfloor m\alpha+1\rfloor\,, & m\alpha~\mbox{--- не целое}\,.
     \end{cases}
     $$
     Заметим, что для такой оценки можно указать доверительный интервал.
     
     Для построения MRR необходимо знать распределение данных. При 
реализации принципа точек высокой плотности в~первую очередь следует 
обратиться к~параметрическим моделям, в~част\-ности к~смеси нормальных 
распределений, име\-ющей плотность распределения
     $$
     f(u) =\sum\limits_{j=1}^k p_j \varphi\left (u,\mu_j, \Sigma_j\right)\,.
     $$
Если $\hat{f}(u)$~--- оценка смеси, то~$t^*$ строится сле\-ду\-ющим образом:
\begin{itemize}
\item генерируется выборка $\left\{ y_1^f,\ldots , y_m^f\right\}$ из $\hat{f}(u)$ и~
для каждого ее $i$-го элемента подсчитывается значение $\hat{f}\left( 
y_i^f\right)$;
\item в~качестве~$t^*$ берется непараметрическая оценка квантиля 
порядка~$\alpha$ (в случае необходимости дополнительно находится 
непараметрическая оценка доверительного интервала для~$t^*$, что 
может характеризовать правильность выбранного объема для 
генерируемой выборки).
\end{itemize}

     Пусть для $f(u)$ имеется~$A_t$, а также получена $\hat{f}(u)$ 
и~соответствующий MRR вида~$\hat{A}_t$. Качество аппроксимации~$A_t$ 
с~по\-мощью~$\hat{A}_t$ можно оценить с~по\-мощью вероятности совпадения 
этих областей, т.\,е. 
     $$
     P_c= \int\limits_{\{ u\in A_t\}\cup \{u\in \hat{A}_t\}} \hspace*{-6mm}
f(u)\,du+\int\limits_{\{u\not\in A_t\} \cup\{ u\not\in \hat{A}_t\}}\hspace*{-6mm} f(u)\,du\,.
     $$
     
     Для оценки  $P_c$ можно использовать величину
     \begin{multline*}
     \hat{P}_c= \fr{\left\vert \left\{ 
     y_i^f\vert y_i^f \in \left\{\left\{ y_i^f\in A_t\right\}\cup \left\{y_i^f\in 
\hat{A}_t\right\}\right\}\right\}\right\vert}{m}+{}\\
{}+\fr{\left\vert \left\{ y_i^f\vert y_i^f \in \left\{\left\{ y_i^f\not\in A_t\right\}\cup 
\left\{ y_i^f\not\in \hat{A}_t\right\}\right\}\right\}\right\vert}{m}\,.
     \end{multline*}
     
     Использование MRR высокой плотности для диагностирования сводится 
к~реализации так называемого слабого критерия значимости для наблюденного 
значения~$x$: нулевая гипотеза заключается в~том, что $x\hm\in A_t$, 
статистика критерия есть $\hat{f}(x)$ и~решение о~принадлежности 
критической об\-ласти~$A_t$ принимается при больших значениях~$\hat{f}(x)$.
     
     Для медицинской практики важна возможность использования 
референсного региона при интерпретации результатов обследования 
некоторого пациента с~вектором признаков~$x$. В~подобных случаях 
сложившейся практикой для слабых критериев значимости является 
использование критического уровня~$\alpha_{\mathrm{cr}}$ (более распространенным 
в~медицине является употребление термина $p$-зна\-че\-ние)  $\alpha_{\mathrm{cr}}\hm= 
\mathrm{Pr}\left\{ \hat{f}(y)\hm\leq \hat{f}(x)\right\}$, где $y$~--- случайная 
величина, имеющая плотность распределения~$\hat{f}(u)$, а $\hat{f}(x)$~--- 
значение плотности распределения~$\hat{f}(u)$ в~точке~$x$. Эта 
характеристика дает представление о~том, насколько сильно данное 
наблюденное значение~$x$ противоречит гипотезе (или подкрепляет ее) 
о~принадлежности данных MRR. При выбранном же заранее уровне 
значимости с~помощью~$\alpha_{\mathrm{cr}}$ сразу же можно принять конкретное 
решение. 

\vspace*{-9pt}

\section{Эксперименты}

\vspace*{-2pt}

     Для демонстрации возможностей MRR использовались данные по 
прогнозу химического состава мочевых камней по метаболическим 
показателям мочи и~сыворотки крови, а также антропологическим 
характеристикам пациентов~[9]. В качестве исходной классификации камней 
рассматривалась следующая: чисто оксалатные (далее обозначены как O), чисто 
уратные (U), чисто фосфатные (P), смесь только оксалатных и~уратных (OU), 
смесь только оксалатных и~фосфатных (OP), смесь только уратных 
и~фосфатных (UP), все остальные. Данная классификация была построена 
в~[10] на основе доминирующих частот встречаемости основных компонентов. 
В~качестве референсных значений рассматривались наборы метаболических 
и~антропологических показателей (их всего было~14), соответствующих 
определенному классу камней.

\begin{table*}\small
\begin{center}


\begin{tabular}{|c|c|c|c|c|c|c|}
\multicolumn{7}{c}{Качество классификации с~помощью MRR}\\
\multicolumn{7}{c}{\ }\\[-6pt]
\hline
\multicolumn{1}{|c|}{\raisebox{-6pt}[0pt][0pt]{\tabcolsep=0pt\begin{tabular}{c}Тип\\ камня\end{tabular}}}&
\multicolumn{1}{c|}{\raisebox{-6pt}[0pt][0pt]{$N$}}&$(1-\alpha)$, 
&\multicolumn{2}{c|}{MRR(5)}&\multicolumn{2}{c|}{MRR(1)}\\
\cline{4-7}
&&&&&&\\[-9pt]
&&\%&$(1-\hat{\alpha})$, \%&$\hat{\beta}$, \%&$(1-\hat{\alpha})$, \%&$\hat{\beta}$, \%\\
\hline
\multicolumn{1}{|c|}{\raisebox{-18pt}[0pt][0pt]{O}}&
\multicolumn{1}{c|}{\raisebox{-18pt}[0pt][0pt]{82}}
&95&100\hphantom{9}&71&90&24\\
&&85&96&78&89&36\\
&&75&91&85&77&44\\
&&65&76&88&74&50\\
\hline
\multicolumn{1}{|c|}{\raisebox{-18pt}[0pt][0pt]{U}}&
\multicolumn{1}{c|}{\raisebox{-18pt}[0pt][0pt]{76}}&95&100\hphantom{9}&75&91&24\\
&&85&99&85&80&35\\
&&75&82&89&74&48\\
&&65&71&91&68&56\\
\hline
\multicolumn{1}{|c|}{\raisebox{-18pt}[0pt][0pt]{P}}&
\multicolumn{1}{c|}{\raisebox{-18pt}[0pt][0pt]{83}}&95&100\hphantom{9}&66&87&25\\
&&85&94&78&86&33\\
&&75&86&82&82&41\\
&&65&77&87&75&47\\
\hline
\end{tabular}
\end{center}
\end{table*}
     
     
     Для каждого из основных классов O, U, P, OU, OP и~UP перед построением 
MRR проводилась селекция признаков и~принималось то значение размерности 
признакового пространства~$d$ и~соответствующий набор показателей, 
которые позволяли прогнозировать состав камней без потери качества 
(методика описана в~\cite{9-kri} и~привела к~значению $d\hm=9$). В~качестве 
модели данных в~первую очередь рассматривалась смесь многомерных 
нормальных распределений из пяти элементов (подбор числа элементов смеси 
проводился с~по\-мощью AIC~--- Akaike information criterion), для соответствующего региона было принято 
обозначение MRR(5). Для сравнения также использовалась модель 
нормального распределения, которой соответствовал MRR(1). Полученные 
результаты приводятся час\-тич\-но в~таблице, где $N$~--- объем 
классифицируемых данных; $\hat{\alpha}$~--- оценка для~$\alpha$; 
$\hat{\beta}$~--- оценка мощности критерия при определении типа камня на 
основании MRR.


     Одной из базовых характеристик является вероятность попадания в~MRR 
$(1\hm-\alpha)$ и~ее оценка $(1\hm-\hat{\alpha})$. Сравнение соответствующих 
столбцов с~учетом значений~$N$ и~ориентировочных значений разброса 
(стандартные отклонения на основе биномиального распределения) не 
позволило выявить явных отклонений. Необходимо, правда, отметить, что во 
всех проанализированных случаях для MRR(5) оказалось, что $1\hm-
\hat{\alpha}\hm\geq 1\hm-\alpha$.
     
     Назначение MRR, заключающееся в~сжатом представлении референсных 
значений, в~многомерном случае практически не проявляется. Для задания 
MRR(5) необходимо указать следующие величины: $1\hm-\alpha$, $t$, 
$p_1,\ldots, p_{k-1}$, $\mu_1, \Sigma_1,\ldots , \mu_k,\Sigma_k$, общее 
количество которых равно  $[2\hm+ (k\hm-1)\hm+ k(d\hm+ d(d\hm+1)/2)]$ 
и,~в~частности, в~рассматриваемых экспериментах~--- 276. Для MRR(1) это 
значение меньше и~равно~56. При этом для обрабатываемой обучающей 
выборки в~зависимости от класса камней речь идет о~порядка~10$^2$ векторах 
данных (см.\ столбец со значениями~$N$), что приблизительно 
дает~10$^3$~скалярных величин.
     
     Другое назначение MRR состоит в~его использовании для 
диагностирования (классификации). В~этой связи в~первую очередь 
проводился сравнительный анализ MRR(1) (фактически это означает, что 
построение региона осуществляется на основе расстояния Махаланобиса) 
и~MRR(5) (модель смеси нормальных распределений и~предложенный 
в~данной работе метод оценивания па\-ра\-мет\-ров региона). Показателем 
информативности метода построения многомерного региона выступала 
мощность соответствующего слабого критерия значимости, а~именно: 
вероятность не попасть в~MRR при условии, что данные берутся из дополнения 
к~классу, для которого построена MRR. Сравнение соответствующих столбцов 
говорит о~явном преимуществе двух предложенных моментов: усложнение 
модели данных путем перехода от нормального распределения к~смеси 
нормальных распределений и~построение региона высокой плотности.
     
     Использование критического уровня можно продемонстрировать  
с~по\-мощью зависимости результатов сравнения двух классов от того, какой 
класс взять за основу. Введем для возможных значений $p$-ве\-ли\-чи\-ны три 
интервала: $(-\infty, 1\%)$, $[1\%, 5\%)$, $[5\%, 100\%)$ с~соответствующей 
интерпретацией положения наблюденного набора показателей для пациента 
относительно построенного MRR: уверенное непопадание, неуверенное 
попадание, уверенное попадание. Если MRR построить для оксалатных камней, 
то результаты для анализа пациентов с~фосфатными камнями дадут следующий 
вектор относительных частот попадания $p$-ве\-ли\-чин в~указанные 
интервалы: $(60\%, 18\%, 22\%)$. Если же MRR строить для фосфатных 
камней, то получим $(71\%, 5\%, 24\%)$. Таким образом, для классификации 
указанных камней при приблизительно одинаковых частотах попадания в~MRR 
(22\% или~24\%) уверенный отказ от референсного региона происходит чаще, 
если принять за базовый MRR регион для фосфатных камней. Построение 
шкалы, подобной рассмотренной, является прерогативой специалистов 
в~предметной области, в~данной работе она использовалась только для 
иллюстрации. 

\vspace*{-6pt}

\section{Заключение}

\vspace*{-2pt}

     На настоящий момент имеется относительно мало примеров применения 
MRR в~клинической практике. Тому есть несколько причин. Математическое 
обеспечение, необходимое для получения и~применения MRR, не отвечает 
возможностям большинства клинических лабораторий. Лаборатории слабо 
оснащены программными средствами\linebreak для реализации достаточно сложного 
математического аппарата многомерного анализа, а~еще важнее, что 
отсутствуют методики, инструкции по\linebreak использованию соответствующих 
средств. Лишь немногие клинические применения демонстрируют 
преимущества MRR, хотя свидетельств неудачных попыток больше.
     
     Несмотря на сложности внедрения мно\-го\-мерно\-го анализа референсных 
значений, можно сформулировать некоторые рекомендации по иссле\-до\-ва\-нию 
и~разработке MRR. Во-пер\-вых, эффективная размерность в~MRR должна 
быть как можно меньше, чтобы избежать затенения диагностически полезной 
информации тестами, со\-зда\-ющи\-ми шум. Низкая размерность также должна 
уменьшить неблагоприятные последствия увеличения неточности результатов 
в~связи с~ростом числа анализируемых показателей. Во-вто\-рых, показатели 
(тес\-ты), включенные в~MRR, должны быть физиологически релевантными 
исследуемому кругу расстройств, чтобы максимизировать информацию, 
полученную от MRR. В-треть\-их, чтобы учесть эффекты долгосрочной 
лабораторной из\-мен\-чи\-вости, данные, используемые для получения MRR, 
долж\-ны быть собраны и~проанализированы в~течение достаточно большого 
периода времени (от нескольких недель до нескольких месяцев).  
В-чет\-вер\-тых, представление результатов лабораторных исследований 
следует осуществлять в~графическом виде, чтобы помочь врачам лучше понять 
MRR. Различные подходы к~уменьшению размерности помогут выполнить это 
требование.
     
     Необходима дальнейшая разработка пояснительных инструментов, 
способных воспринять результаты анализа MRR. При этом дополнительно 
необходима информация о~том, какие именно тес\-ты являются важнейшими 
факторами нарушения нормы. Надо признать, что соответствующий 
математический аппарат еще предстоит разработать. Решение перечисленных 
вопросов играет важную роль для обеспечения постоянного клинического 
применения MRR. 

\vspace*{-6pt}
     
{\small\frenchspacing
 {%\baselineskip=10.8pt
 \addcontentsline{toc}{section}{References}
 \begin{thebibliography}{99}
 
 \vspace*{-2pt}
 
\bibitem{1-kri}
\Au{Boyd J.\,C.} Reference regions of two or more dimensions~// Clin. Chem. Lab. 
Med., 2004. Vol.~42. No.\,7. P.~739--746.
\bibitem{2-kri}
\Au{Winkel P.} Patterns and clusters~--- multivariate approach for interpreting 
clinical chemistry results~// Clin. Chem., 1973. Vol.~19. No.\,12. P.~1329--1333.
\bibitem{3-kri}
IFCC. Expert panel on theory of reference values. Approved recommendation on the 
theory of reference values. Part~5. Statistical treatment of collected reference values. 
Determination of reference limits~// J.~Clin. Chem. Clin. Biochem., 1987. Vol.~25. 
No.\,9. P.~645--656.
\bibitem{4-kri}
\Au{Кривенко М.\,П.} Статистические методы представления и~предварительной 
обработки референсных значений.~--- М.: ФИЦ ИУ РАН, 2016. 160~с.
\bibitem{5-kri}
\Au{Boyd J.\,C., Lacher~D.\,A.} The multivariate reference range: An alternative 
interpretation of multi-test profiles~// Clin. Chem., 1982. Vol.~28. No.\,2.  
P.~259--265.
\bibitem{6-kri}
\Au{Albert A., Harris~E.\,K.} Multivariate interpretation of clinical laboratory  
data.~--- New York, NY, USA: CRC Press, 1987. 328~p.
\bibitem{7-kri}
\Au{Linnet K.} Influence of sampling variation and analytical errors on the 
performance of the multivariate reference region~// Meth. Inf. Med., 1988. Vol.~27. 
No.\,1. P.~37--42.
\bibitem{8-kri}
\Au{Durbridge T.\,C.} Clinical acceptance of a multi-test reference region for 
biochemical-panel results~// Clin. Chem., 1983. Vol.~29. No.\,10. P.~1724--1726.
\bibitem{9-kri}
\Au{Кривенко М.\,П.} Критерии значимости отбора признаков классификации~// 
Информатика и~её применения, 2016. Т.~10. Вып.~3. С.~32--40.
\bibitem{10-kri}
\Au{Кривенко М.\,П., Голованов~С.\,А., Сивков~А.\,В.} Анализ однородности 
данных о химическом составе камней при уролитиазе~// Информатика и~её 
применения, 2013. Т.~7. Вып.~4. С.~94--104.
 \end{thebibliography}

 }
 }

\end{multicols}

\vspace*{-10pt}

\hfill{\small\textit{Поступила в~редакцию 5.12.16}}

\vspace*{4pt}

%\newpage

%\vspace*{-24pt}

\hrule

\vspace*{2pt}

\hrule

\vspace*{-3pt}


\def\tit{HIGH-DENSITY MULTIVARIATE REFERENCE REGION\\[-5pt]}

\def\titkol{High-density multivariate reference region}

\def\aut{M.\,P.~Krivenko\\[-7pt]}

\def\autkol{M.\,P.~Krivenko}

\titel{\tit}{\aut}{\autkol}{\titkol}

\vspace*{-16pt}


\noindent
Institute of Informatics Problems, Federal Research Center 
``Computer Science and Control'' of the Russian
Academy of Sciences,  44-2~Vavilov Str., Moscow 119333, Russian Federation



\def\leftfootline{\small{\textbf{\thepage}
\hfill INFORMATIKA I EE PRIMENENIYA~--- INFORMATICS AND
APPLICATIONS\ \ \ 2017\ \ \ volume~11\ \ \ issue\ 2}
}%
 \def\rightfootline{\small{INFORMATIKA I EE PRIMENENIYA~---
INFORMATICS AND APPLICATIONS\ \ \ 2017\ \ \ volume~11\ \ \ issue\ 2
\hfill \textbf{\thepage}}}

\vspace*{2pt}




\Abste{The paper considers the principles of construction of multivariate 
reference regions. An original method of construction of 
a~region on the basis of areas of high density of points and approximation 
of data distribution with a~mixture of normal distributions is suggested. 
To estimate the threshold for the probability density, the bootstrap method is used. 
As an experiment, the paper considers the problem of description and use of 
the reference region for predicting the type of urinary stones. 
Real data treatment demonstrated the benefits of the proposed solutions.}

\KWE{multivariate reference region; high-density region; bootstrap method; 
multivariate normal mixture}

\DOI{10.14357/19922264170207} 

%\vspace*{-18pt}

%\Ack
%\noindent



%\vspace*{3pt}

  \begin{multicols}{2}

\renewcommand{\bibname}{\protect\rmfamily References}
%\renewcommand{\bibname}{\large\protect\rm References}

{\small\frenchspacing
 {%\baselineskip=10.8pt
 \addcontentsline{toc}{section}{References}
 \begin{thebibliography}{99}
\bibitem{1-kri-1}
\Aue{Boyd, J.\,C.} 2004. Reference regions of two or more dimensions. \textit{Clin. 
Chem. Lab. Med.} 42(7):739--746.

\bibitem{2-kri-1}
\Aue{Winkel, P.} 1973. Patterns and clusters~--- multivariate approach for interpreting 
clinical chemistry results. \textit{Clin. Chem.} 19(12):1329--1333.
\bibitem{3-kri-1}
IFCC. 1987. Expert panel on theory of reference values. Approved recommendation on the 
theory of reference values. Part~5. Statistical treatment of collected reference values. 
Determination of reference limits. \textit{J.~Clin. Chem. Clin. Biochem.} 
25(9):645--656.
\bibitem{4-kri-1}
\Aue{Krivenko, M.\,P.} 2016. \textit{Statisticheskie metody predstavleniya 
i~predvaritel'noy obrabotki referensnykh znacheniy}
[Statistical methods for representation and preliminary processing of
reference values]. Moscow: FRC CSC RAS. 160~p.

\bibitem{5-kri-1}
\Aue{Boyd, J.\,C., and D.\,A.~Lacher.} 1982. The multivariate reference range: An 
alternative interpretation of multi-test profiles. \textit{Clin. Chem.}  
28(2):259--265.
\bibitem{6-kri-1}
\Aue{Albert, A., and E.\,K.~Harris.} 1987. \textit{Multivariate interpretation of 
clinical laboratory data}. New York, NY: CRC Press. 328~p.
\bibitem{7-kri-1}
\Aue{Linnet, K.} 1988. Influence of sampling variation and analytical errors on the 
performance of the multivariate reference region. \textit{Meth. Inf. Med.}  
27(1):37--42.
\bibitem{8-kri-1}
\Aue{Durbridge, T.\,C.} 1983. Clinical acceptance of a multi-test reference region 
for biochemical-panel results. \textit{Clin. Chem.} 29(10):1724--1726.
\bibitem{9-kri-1}
\Aue{Krivenko, M.\,P.} 2016. Kriterii znachimosti otbora priznakov klassifikatsii
[Significance tests of feature selection for~classification]. \textit{Informatika i~ee 
Primeneniya~--- Inform. Appl.} 10(3):32--40.
\bibitem{10-kri-1}
\Aue{Krivenko, M.\,P., S.\,A.~Golovanov, and A.\,V.~Sivkov}. 2013. Analiz 
odnorodnosti dannykh o~khimicheskom sostave kamney pri urolitiaze
[Analysis of data homogeneity of~the~chemical compositions 
of~stones in~case of~urolithiasis]. \textit{Informatika i~ee Primeneniya~---
Inform Appl.} 7(4):94--104.
\end{thebibliography}

 }
 }

\end{multicols}

\vspace*{-3pt}

\hfill{\small\textit{Received December 5, 2016}}


\Contrl

\noindent
\textbf{Krivenko Michail P.} (b.\ 1946)~--- Doctor of Science in technology, 
professor, leading scientist, Institute of Informatics Problems, Federal Research 
Center ``Computer Science and Control'' of the Russian Academy of Sciences, 
\mbox{44-2}~Vavilov Str., Moscow 119333, Russian Federation; \mbox{mkrivenko@ipiran.ru}

\label{end\stat}


\renewcommand{\bibname}{\protect\rm Литература}   %8
\def\stat{grusho}

\def\tit{АРХИТЕКТУРНЫЕ РЕШЕНИЯ В~ЗАДАЧЕ ВЫЯВЛЕНИЯ МОШЕННИЧЕСТВА ПРИ~АНАЛИЗЕ 
ИНФОРМАЦИОННЫХ ПОТОКОВ В~ЦИФРОВОЙ ЭКОНОМИКЕ$^*$}

\def\titkol{Архитектурные решения в~задаче выявления мошенничества при~анализе 
информационных потоков в
%~цифровой 
экономике}

\def\aut{А.\,А.~Грушо$^1$, М.\,И.~Забежайло$^2$, Н.\,А.~Грушо$^3$, 
Е.\,Е.~Тимонина$^4$}

\def\autkol{А.\,А.~Грушо, М.\,И.~Забежайло, Н.\,А.~Грушо, 
Е.\,Е.~Тимонина}

\titel{\tit}{\aut}{\autkol}{\titkol}

\index{Грушо А.\,А.}
\index{Забежайло М.\,И.}
\index{Грушо Н.\,А.}
\index{Тимонина Е.\,Е.}
\index{Grusho A.\,A.}
\index{Zabezhailo M.\,I.}
\index{Grusho N.\,A.}
\index{Timonina E.\,E.}


{\renewcommand{\thefootnote}{\fnsymbol{footnote}} \footnotetext[1]
{Работа частично поддержана РФФИ (проекты 18-29-03081 и~18-07-00274).}}


\renewcommand{\thefootnote}{\arabic{footnote}}
\footnotetext[1]{Институт проблем информатики Федерального исследовательского центра <<Информатика и~управление>> 
Российской академии наук, grusho@yandex.ru}
\footnotetext[2]{Институт проблем информатики Федерального исследовательского центра <<Информатика и~управление>> 
Российской академии наук, m.zabezhailo@yandex.ru}
\footnotetext[3]{Институт проблем информатики Федерального исследовательского центра <<Информатика и~управление>> 
Российской академии наук, info@itake.ru}
\footnotetext[4]{Институт проблем информатики Федерального исследовательского центра <<Информатика и~управление>> 
Российской академии наук, eltimon@yandex.ru}

\vspace*{-12pt}
   

 
  
  \Abst{Cформулирован подход к~исследованию некоторых видов мошенничества в~цифровой 
экономике с~использованием причинно-следственных связей. Во всех видах рассматриваемых 
мошенничеств должно наблюдаться несоответствие между целями финансовых транзакций 
и~реальной стоимостью достижения этих целей. Данные о транзакциях можно собирать, 
наблюдая информационные потоки, в~которых отражаются эти транзакции. Архитектура сбора 
данных и~их анализа может быть организована с~помощью распределенных реестров 
с~централизованным консенсусом, что позволяет создать аналог электронной бухгалтерской 
книги, фиксирующей финансово-экономическую деятельность субъектов цифровой экономики в~регионе. 
  Рассматриваемые методы выявления мошенничества основаны на противоречиях 
между действиями, описанными в~транзакциях, и~информацией, содержащейся в~планах, 
стандартах, прецедентах и~др. Рассмотрен метод, основанный на некоторой упрощенной схеме 
реализации абстрактного проекта. Для выявления противоречий необходимо проводить анализ 
от следствия к~причине, т.\,е.\ искать аномалии в~информации, описывающей порождение 
наблюдаемых следствий. 
  Показано, как в~реализации проекта можно выделять простые <<необходимые условия>> 
нарушения при\-чин\-но-след\-ст\-вен\-ных связей, т.\,е.\ множество <<необходимых условий>>, 
нарушение которых свидетельствует о наличии мошенничества. Это множество <<необходимых 
условий>> можно назвать метаданными для контроля проекта на выявление мошенничества.} 
 
 
  \KW{цифровая экономика; информационные потоки; при\-чин\-но-след\-ст\-вен\-ные связи; 
выявление мошеннических схем} 

\DOI{10.14357/19922264190204}
  
\vspace*{-4pt}


\vskip 10pt plus 9pt minus 6pt

\thispagestyle{headings}

\begin{multicols}{2}

\label{st\stat}

\section{Введение}

\vspace*{3pt}

  В работе сформулирован подход к~исследованию некоторых видов 
мошенничества в~цифровой экономике с~использованием  
при\-чин\-но-след\-ст\-вен\-ных связей. Рассматриваются три вида мошенничества, 
а именно:
  \begin{enumerate}[(1)]
\item отмыв денег; 
\item обман при выполнении договорных обязательств при реализации 
технических проектов (строительные проекты и~др.); 
\item незаконный вывод денег. 
\end{enumerate}

  Названные виды мошенничества могут быть сведены к~решению одного типа 
задач. Для отмывания денег источник должен заключать фиктивные контракты, 
в~соответствии с~которыми будут переводиться средства за заведомо ненужную 
работу и~материалы. 
  
  Мошенничество, связанное с~невыполнением договорных обязательств, связано 
со снижением качества услуг, качества и~количества закупаемых 
материалов, выполнением работ с~ненадлежащим качеством. 
  
  Вывод денег связан с~переводом средств фир\-мам-од\-но\-днев\-кам, которые 
заведомо не могут выполнить обязательства по контрактам, за которые им 
переводятся средства. 
  
  Таким образом, во всех трех видах рассматриваемых мошенничеств должно 
наблюдаться несоответствие между целями финансовых транзакций и~реальной 
стоимостью достижения этих целей. Данные о транзакциях можно собирать, 
наблюдая информационные потоки, в~которых отражаются эти транзакции. 
  
  Однако для наблюдения таких информационных потоков необходимо создавать 
архитектуру\linebreak телекоммуникационной системы, позволяющей перехватывать 
и~собирать данные о всех транзакциях. Например, такая архитектура может быть 
организована с~помощью распределенных реестров с~централизованным 
консенсусом, т.\,е.\ все информационные потоки, сформированные в~цифровой 
экономике и~несущие информацию о транзакциях, проходят через некоторый 
центральный узел, запоминающий их в~форме распределенного реестра. Такие 
реестры могут дублироваться в~аналогичных центрах различных регионов, что 
позволяет создать аналог электронной бухгалтерской книги, фиксирующей 
фи\-нан\-со\-во-эко\-но\-ми\-че\-скую деятельность субъектов цифровой экономики. Такой 
подход предложено реализовать на базе системы ситуационных центров, что 
отражено в~работах~[1, 2].
  
  Собранная из информационных потоков информация о~транзакциях, т.\,е.\ 
о~контрактах, договорах, платежах, отчетах, закупленных материалах, 
характеристиках исполнителей работ и~др., собирается в~базе данных в~указанном 
центре. Согласно теории интеллектуальных сис\-тем~[3], эту базу данных можно 
называть базой фактов (БФ). Базу фактов можно представить как бинарную мат\-ри\-цу, 
строки которой описывают характеристики, входящие в~транзакции, а столбцы 
нумеруются характеристиками. Строки матрицы будем называть 
\textit{объектами}~[4, 5]. 
  
  Рассматриваемые в~работе методы выявления мошенничества будут основаны 
на противоречиях между действиями, описанными в~транзакциях, и~информацией, 
содержащейся в~планах, стандартах, прецедентах и~др. Для нахождения 
противоречий в~архитектуре центра предусмотрена другая база данных~--- база 
знаний (БЗ)~\cite{3-gr, 6-gr}, которая устроена так же, как БФ. 
  
  Информация в~БЗ собирается на основе положительного опыта или расчетов. 
Используя БЗ, можно выводить факты нарушения при\-чин\-но-след\-ст\-вен\-ных 
связей. Нарушения при\-чин\-но-след\-ст\-вен\-ных связей будем называть 
\textit{аномалиями}. 
  
  Для упрощения дальнейшее изложение будет вестись в~рамках поиска 
противоречий при выполнении некоторого абстрактного проекта. Выявление 
аномалий будет происходить на основе фактов из БФ с~помощью знаний из БЗ 
методами искусственного интеллекта и~интеллектуального анализа 
данных~\cite{6-gr}. 

\vspace*{-10pt}
  
  \section{Модели}
  
  \vspace*{-3pt}
  
  Наиболее сложная из рассмотренных выше задач~--- выявление противоречий, 
т.\,е.\ использование БЗ для получения новых знаний и~выявление аномалий из 
полученных фактов. 
  
  Все способы выявления противоречий основаны на определении 
  причинно-следственных связей. При этом противоречия в~параметрах транзакций по 
отношению к~требуемым в~БЗ составляют сущность аномалий. 
  
   Далее будет рассмотрен метод, основанный на некоторой упрощенной схеме 
реализации абстрактного проекта. 
  
  Каждый проект имеет цель: например, цель представляет собой построение 
некоторой системы. Воспользуемся структурным подходом, который позволяет 
строить проект на основе разбиения системы на подсистемы и~определения 
взаимодействий подсистем~\cite{7-gr}. При этом каждая подсистема также 
представима структурной моделью. 
  
  Как сама система, так и~каждая ее подсистема имеют свой функционал 
и~спецификацию, па\-ра\-мет\-ры настройки и~домены параметров настройки. Кроме 
этих характеристик существует множество характеристик, связанных 
с~<<жизненным циклом>> создания системы. Сюда входят работы, ресурсы, 
сроки выполнения работ по созданию подсистем и~самой системы, стоимости 
компонентов и~материалов, стоимости работ, схемы поставок, договорные 
обязательства и~др. Все характеристики связаны между собой, поэтому можно 
говорить о стоимости и~времени изготовления структурных компонентов системы. 
  
  Одной из важнейших характеристик является смета (система смет для 
подсистем). Смета сопоставляет каждому компоненту системы стоимость его 
изготовления и~настройки. 
  
  Схема построения системы может быть пред\-став\-ле\-на диаграммой, 
изображенной на рис.~1. 

{ \begin{center}  %fig1
 \vspace*{9pt}
   \mbox{%
 \epsfxsize=79mm 
 \epsfbox{gru-1.eps}
 }


\vspace*{9pt}


\noindent
{{\figurename~1}\ \ \small{Диаграмма достижения цели}}
\end{center}
}

\vspace*{9pt}

\addtocounter{figure}{1}
  
  


  Представленная на рис.~1 диаграмма позволяет описать основные классы 
возможных противоречий при достижении цели. Противоречия возникают, когда 
данные БФ не соответствуют требуемым характеристикам. 
  
  
  \section{Потенциальные классы аномалий при~достижении цели}
  
  Выделим четыре потенциальных класса противоречий, которые показывают, 
каким образом нужно искать эти противоречия.
  
 
  Противоречие цели и~проекта (рис.~2) возникает при отсутствии обоснования 
или в~случае логического противоречия между возможностями проектируемого 
функционала и~целью системы. Отметим, что в~проект входят сроки, перечень 
работ, материалы, настройки, которые описываются соответствующими 
параметрами и~допустимыми значениями этих параметров. Проект формируется 
на основе БЗ и~расчетов, исходя из информации, полученной по аналогии 
с~другими проектами и~решениями, которые считаются апробированными. 
  
  Отметим, что цель порождает проект и~в этом смысле является причиной 
проекта. Однако для анализа противоречий необходимо двигаться по штриховой 
стрелке диаграммы (см.\ рис.~2) от проекта к~цели. В~самом деле, любой компонент 
проекта направлен на теоретическое достижение цели. Цель~--- сложный объект, 
поэтому в~проекте могут возникнуть характеристики, противоречащие хотя бы 
некоторым характеристикам цели. Это делает проект противоречивым, но вывод 
об этом может быть сделан только на уровне описания цели. 
  

  Противоречия между проектом и~его реализацией, исключая настройки 
(рис.~3), могут возникать, например, при закупке исполнителем материалов более 
низкого качества по более низким ценам, при попытках достижения требуемых 
сроков работы за счет снижения качества выполнения работ, за счет нахождения 
<<объективных>> причин для увеличения сроков работы и,~следовательно, 
увеличения цены реализации проекта. 


  Для выявления указанных противоречий необходимо двигаться по диаграмме 
(см.\ рис.~3) в~обратную сторону в~соответствии со~штриховыми стрелками. 
Действительно, выявить противоречия между характеристиками закупленных 
материалов и~требуемыми по проекту можно только при обращении к~проекту 
и~его спецификациям. Манипуляции со сроками работы также можно выявить 
только при обращении к~соответствующим расчетам в~проекте. Задержки в~сроках 
работы, связанные с~поставками материалов, можно определить только на 
предыдущем этапе диаграммы (см.\ рис.~3) в~описании проекта. 


  


  Противоречия между реализацией проекта и~его настройкой (рис.~4) возникает, 
когда не удается добиться требуемых значений параметров функционала, не 
удается обеспечить необходимый уровень\linebreak\vspace*{-12pt}

{ \begin{center}  %fig2
 \vspace*{-6pt}
   \mbox{%
 \epsfxsize=16mm 
 \epsfbox{gru-2.eps}
 }


\vspace*{6pt}


\noindent
{{\figurename~2}\ \ \small{Противоречия цели и~проекта}}
\end{center}
}

%\vspace*{9pt}

\addtocounter{figure}{1}

{ \begin{center}  %fig3
 \vspace*{6pt}
    \mbox{%
 \epsfxsize=79mm 
 \epsfbox{gru-3.eps}
 }


\end{center}

\vspace*{-2pt}


\noindent
{{\figurename~3}\ \ \small{Противоречия проекта и~его реализации (без настройки)}}
}

\vspace*{6pt}

\addtocounter{figure}{1}

{ \begin{center}  %fig4
 \vspace*{1pt}
   \mbox{%
 \epsfxsize=54.5mm 
 \epsfbox{gru-4.eps}
 }


\end{center}


\noindent
{{\figurename~4}\ \ \small{Противоречия реализации проекта и~его на\-стройки}}
}

%\vspace*{9pt}

\addtocounter{figure}{1}

{ \begin{center}  %fig5
 \vspace*{5pt}
    \mbox{%
 \epsfxsize=79mm 
 \epsfbox{gru-5.eps}
 }


\end{center}



\noindent
{{\figurename~5}\ \ \small{Противоречия цели и~достигнутой реализации проекта}}
}

\vspace*{6pt}

\addtocounter{figure}{1}

\noindent
 качества реализации проекта. Для 
определения противоречия в~настройках надо опять же двигаться по диаграмме 
(см.\ рис.~4) в~обратную сторону по штриховым стрелкам, так как для выявления 
характеристик результатов работы, которые не дают возможности реализации 
определенного функционала, необходимо иметь информацию о результатах этой 
работы. 


  



  Противоречие между целью и~достигнутой реализацией проекта (рис.~5) 
возникает, когда реализованная система не позволяет достичь цели. В~этом случае 
опять противоречие нужно искать, двигаясь от цели к~реальному достигнутому 
функционалу по штриховой стрелке (см.\ рис.~5).
  
  Суммируя положения, изложенные в~данном разделе, приходим к~выводу, что 
для выявления противоречий необходимо проводить анализ от следствия 
к~причине, т.\,е.\ искать аномалии в~информации, описывающей порождение 
наблюдаемых следствий. 
  
  
  \section{Связь противоречий и~причин}
  
  Прежде чем построить связь между причинами и~противоречиями, кратко 
опишем простейшую модель связи этих понятий. Причины и~противоречия будут 
сформулированы для представления компонентов системы как объектов, 
обладающих наборами известных характеристик~\cite{4-gr, 5-gr}. 
  
  Пусть $U\hm=\{\alpha, \beta, \ldots\}$~--- совокупность характеристик 
(пространство характеристик). Согласно~\cite{4-gr} \textit{объектом}~$O$ 
называется любое подмножество характеристик $O\hm\subseteq U$. Рассмотрим 
последовательность объектов, возможно в~различных пространствах 
характеристик. 
  
  \smallskip
  
  \noindent
  \textbf{Определение~1.}\ Объект~$P$ с~числом характеристик, большим или 
равным~2, является \textit{причиной} объекта (\textit{свойства})~$B$ в~цепочке 
наблюдаемых объектов тогда и~только тогда, когда выполнены следующие 
условия:
  \begin{enumerate}[(1)]
\item для каждого объекта~$C$, если $P\hm\subseteq C$, то $C\mapsto B$, где 
$C\mapsto B$ означает, что объект~$B$ присутствует в~объекте, следующем за 
объектом~$C$;
\item объект~$P$ является минимальным объектом, удовлетворяющим 
условию~1, а~именно: $\forall \alpha\hm\in P$ объект~$P\backslash \{\alpha\}$ 
не является причиной, т.\,е.\ $\exists C:\ \alpha\not\in C$, $P\backslash 
\{\alpha\}\hm\subseteq C$ и~$C\not\mapsto B$, где $C\not\mapsto B$ означает, 
что~$B$ не может содержаться в~объекте, следующем за объектом~$C$. 
\end{enumerate}

  Приведенное определение причины является упрощением причин, 
возникающих в~реальном мире. Например, реальные причины могут возникать\linebreak 
как совокупность характеристик из разных пространств. Одно следствие может 
порождаться разными причинами или возникать из внешних\linebreak и~ненаблюдаемых 
характеристик. Однако пред\-став\-лен\-ная далее формализация позволяет доступно 
изложить при\-чин\-но-след\-ст\-вен\-ные истоки противоречий, которые 
инициируют в~дальнейшем глубокое исследование рассматриваемых процессов.
  
  Будем считать, что для любого интересующего нас свойства~$B$ существует 
причина. Тогда справедлива следующая теорема.
  
  \smallskip
  
  \noindent
  \textbf{Теорема~1.}\ \textit{Для любого свойства~$B$ существует 
единственная причина}. 
  
  \smallskip
  
  \noindent
  Д\,о\,к\,а\,з\,а\,т\,е\,л\,ь\,с\,т\,в\,о\,.\ \ Доказательство будем вести от противного, 
т.\,е.\ предположим, что существуют две причины свойства~$B$: $P$ 
и~$P^\prime$, $P\hm\not= P^\prime$. Тогда существует $\alpha\hm\in U$, которое 
удовлетворяет одному из двух условий:
  \begin{itemize}
\item[(а)] $\alpha\in P$, $\alpha\notin P^\prime$;
\item[(б)] $\alpha\notin P$, $\alpha \in P^\prime$.
\end{itemize}

  Пусть выполняется условие~(б). Тогда $P^\prime\backslash \{\alpha\}$ не 
является причиной по условию~2 определения~1, т.\,е.\ $\exists C$ такое, что 
$\alpha\notin C$, $P^\prime\backslash \{\alpha\}\hm\subseteq C$ и~$C\not\mapsto B$. 
Но если~$B$ произошло и~$P$ его причина, то $C\mapsto B$, что противоречит 
предположению. Теорема~1 доказана.
  
  \smallskip
  
  \noindent
  \textbf{Лемма.} \textit{Если $P$~--- причина появления свойства~$B$, то 
объект~$B$ определяет существование свойства~$P$ в~объекте, 
предшествующем~$B$. }
  
  \smallskip
  
  \noindent
  Д\,о\,к\,а\,з\,а\,т\,е\,л\,ь\,с\,т\,в\,о\,.\ \ Из предположения, что у~каж\-до\-го 
свойства~$B$ есть причина, и~условия, что~$P$ является причиной~$B$, следует, 
что при появлении в~данных свойства~$B$ объект~$C$, предшествующий 
появлению~$B$, содержит как часть объект~$P$. Это следует из теоремы~1 
и~определения причины. 
  
  Докажем принцип <<необходимого условия>>, который, несмотря на простоту 
доказательства, будет играть в~дальнейшем существенную роль.
  
  \smallskip
  
  \noindent
  \textbf{Теорема~2.} \textit{Если~$P$~--- причина появления свойства~$B$ 
и~$A\hm\subseteq P$, то объект~$B$ определяет наличие свойства~$A$ 
в~объекте, предшествующем~$B$}. 
  
  \smallskip
  
  \noindent
  Д\,о\,к\,а\,з\,а\,т\,е\,л\,ь\,с\,т\,в\,о\,.\ \ Пусть в~данных имеется объект~$B$ 
и~$P\mapsto B$, тогда в~силу существования и~единственности причины~$B$ 
в~данных должен существовать объект~$C$, предшествующий~$B$ 
и~содержащий причину~$P$. Поскольку $A\hm\subseteq P$ и~$B$ содержит 
причину~$P$, то $B\mapsto A$. С~учетом леммы теорема~2 доказана.
  
  \smallskip
  
  Пусть даны пространства $U_1, U_2,\ldots$ и~имеется последовательность 
данных (процесс выполнения этапов проекта в~соответствии с~рис.~1) $A, B, 
\ldots$, где каждый объект является подмножеством некоторого 
пространства~$U_i$, $i\hm=1,\ldots$ Тогда в~объекте~$A$ присутствует 
причина~$P$ появления интересующего нас свойства~$C$ в~объекте~$B$. Пусть 
$P\hm\subseteq A$, тогда по теореме~2 $\forall \alpha\hm\in P$:  
$C\mapsto \{\alpha\}$, т.\,е.\ из появления~$C$ следует появление 
характеристики~$\alpha$ в~предшествующем объекте. Это необходимое условие 
того, что~$C$ удовлетворяет причинно-следственным связям развития процесса 
выполнения проекта. Если для~$C$ нет характеристики~$\alpha$, которую можно 
отнести к~причине~$C$, то можно считать, что нарушена  
при\-чин\-но-след\-ст\-вен\-ная связь и~$C$~--- аномальный объект. 
  
  \smallskip
  
  \noindent
  \textbf{Пример.} Если объект~$C$ состоит в~получении суммы~$a$ 
фирмой~$K$, то согласно теореме~2 в~пред\-шест\-ву\-ющем объекте должна 
существовать причина перевода суммы~$a$ на фирму~$K$. Если эта причина 
в~проекте отсутствует, то это можно считать признаком мошеннической схемы. 
Все проекты по предположению собираются из <<кубиков>>, содержащихся в~БЗ. 
Тогда можно сравнить цену объекта~$C$, породившего получение суммы~$a$, 
и~сумму, присутствующую в~смете проекта. Если разница велика, то это либо 
ошибка проекта, либо признак мошеннической схемы.
  
  \section{Поиск противоречий на~основе~принципа <<необходимых~условий>>}
   
  Как было показано в~разд.~3, нахождение противоречий соответствуют 
движению от следствия к~причине. Для каждого объекта в~наблюдаемых данных 
выявление причин его появления является трудоемкой задачей. Кроме того, при 
реализации контроля соблюдения при\-чин\-но-след\-ст\-вен\-ных связей на 
большом множестве участников экономической деятельности задача анализа 
причин становится трудоемкой. Поэтому процедуру контроля необходимо разбить 
на два этапа, где первый этап состоит в~анализе простых <<необходимых 
условий>> проявления мошенничества, когда используется хотя бы одна 
известная характеристика причины. Второй этап (в~режиме офлайн) состоит 
в~выявлении причин, позволяющих провести анализ источников мошеннических 
схем. 
  
  Один из подходов к~выбору <<необходимых условий>> состоит в~построении 
множества подцелей исходной цели проекта (структурный метод построения 
проекта~\cite{7-gr}). Каждая подцель описывается диаграммой на рис.~1, 
и~реализации подцелей должны образовывать полный функционал цели. Это 
является необходимым, но не достаточным условием достижения цели, так как 
при таком подходе отсутствует компонент согласования всех подцелей в~единую 
систему. Однако такой подход значительно упрощает анализ выполнения проекта 
на предмет поиска мошенничества. Если признаки мошенничества будут 
обнаружены в~реализации хотя бы одной из подцелей, то это значит, что 
мошенничество присутствует в~реализации всего проекта. 
  
  Аналогично в~реализации каждого этапа в~любой из подцелей можно выделять 
простые <<необходимые условия>> нарушения при\-чин\-но-след\-ст\-венн\-ых 
связей. 
  
  Таким образом, получается множество <<необходимых условий>>, нарушение 
которых свидетельствует о наличии мошенничества. Это множество 
<<необходимых условий>> можно назвать метаданными~[8, 9] для контроля 
проекта на выявление мошенничества. 
  
  
  \section{Заключение }
  
  В поиске противоречий необходимо от транзакций, соответствующих 
следствиям при\-чин\-но-след\-ст\-вен\-ных связей, переходить к~анализу причин 
наблюдаемых следствий. Это сложная задача, которая связана с~описанием причин 
определенных свойств. 
  
  В работе представлена модель, позволяющая строить множество необходимых 
условий соответствия наблюдаемого следствия вызвавшей его причине. Этот 
подход делает поиск противоречий вполне вычислимой задачей, но не гарантирует 
успех. 
  
  {\small\frenchspacing
 {%\baselineskip=10.8pt
 \addcontentsline{toc}{section}{References}
 \begin{thebibliography}{9}
\bibitem{1-gr}
\Au{Грушо А.\,А., Зацаринный~А.\,А., Тимонина~Е.\,Е.} Блокчейны цифровой экономики на базе 
системы ситуационных центров и~централизованного консенсуса~// Радиолокация, навигация, 
связь: Мат-лы XXV Междунар. научн.-технич. конф.~---
Воронеж: Издательский дом ВГУ, 2019. Т.~6. С.~183--191. 
\bibitem{2-gr}
\Au{Grusho A., Zatsarinny~A., Timonina~E.} A~system approach to information security in 
distributed ledgers on the situational centers platform.~---
Lecture notes in computer science ser.~--- Springer, 2019 
(in press).
\bibitem{3-gr}
\Au{Финн В.\,К.} Искусственный интеллект: Методология, применения, философия.~--- М.: 
Красанд, 2011. 448~с.

\bibitem{5-gr} %4
\Au{Аншаков~О.\,М., Фабрикантова~Е.\,Ф.} ДСМ-ме\-тод автоматического порождения 
гипотез: Логические и~эпистемологические основания.~--- М.: Либроком, 2009. 432~с.

\bibitem{4-gr} %5
\Au{Poelmans J., Elzinga~P., Viaene~S., Dedene~G.} Formal concept analysis in knowledge 
discovery: A~survey~// Conceptual structures: From information to intelligence~/ Eds.\ M.~Croitoru, 
S.~Ferr$\acute{\mbox{e}}$, and D.~Lukose.~--- Lecture notes in computer science 
ser.~--- Berlin--Heidelberg: Springer, 2010. Vol.~6208.  P.~139--153.

\bibitem{6-gr}
\Au{Панкратова~Е.\,С., Финн~В.\,К.} Автоматическое по\-рож\-де\-ние гипотез в~интеллектуальных 
системах.~--- М.: Либроком, 2009. 528~с. 
\bibitem{7-gr}
\Au{Денисов А.\,А., Колесников~Д.\,Н.} Теория больших систем управления.~--- Л.: Энергоиздат, 1982. 488~с.

\bibitem{9-gr}
\Au{Грушо А.\,А., Грушо Н.\,А., Забежайло~М.\,И., Смирнов~Д.\,В., Тимонина~Е.\,Е.} 
Параметризация в~прикладных задачах поиска эмпирических причин~// Информатика и~её 
применения, 2018. Т.~12. Вып.~3. С.~62--66.

\bibitem{8-gr}
\Au{Грушо А.\,А., Грушо Н.\,А., Левыкин~М.\,В., Тимонина~Е.\,Е.} Методы идентификации 
захвата хоста в~распределенной ин\-фор\-ма\-ци\-он\-но-вы\-чис\-ли\-тель\-ной сис\-те\-ме, 
защищенной с~помощью метаданных~// Информатика и~её применения, 2018. Т.~12. Вып.~4. 
С.~41--45.

 \end{thebibliography}

 }
 }

\end{multicols}

\vspace*{-3pt}

\hfill{\small\textit{Поступила в~редакцию 03.04.19}}

%\vspace*{8pt}

%\pagebreak

\newpage

\vspace*{-28pt}

%\hrule

%\vspace*{2pt}

%\hrule

%\vspace*{-2pt}

\def\tit{ARCHITECTURAL DECISIONS IN~THE~PROBLEM 
OF~IDENTIFICATION OF~FRAUD IN~THE~ANALYSIS 
OF~INFORMATION FLOWS IN~DIGITAL ECONOMY\\[-5pt]}


\def\titkol{Architectural decisions in~the~problem 
of~identification of~fraud in~the~analysis 
of~information flows in~digital economy}

\def\aut{A.\,A.~Grusho, M.\,I.~Zabezhailo, N.\,A.~Grusho, and~E.\,E.~Timonina}

\def\autkol{A.\,A.~Grusho, M.\,I.~Zabezhailo, N.\,A.~Grusho, and~E.\,E.~Timonina}

\titel{\tit}{\aut}{\autkol}{\titkol}

\vspace*{-13pt}


 \noindent
   Institute of Informatics Problems, Federal Research Center ``Computer Sciences and 
Control'' of the Russian Academy of Sciences; 44-2~Vavilov Str., Moscow 119133, 
Russian Federation

\def\leftfootline{\small{\textbf{\thepage}
\hfill INFORMATIKA I EE PRIMENENIYA~--- INFORMATICS AND
APPLICATIONS\ \ \ 2019\ \ \ volume~13\ \ \ issue\ 2}
}%
 \def\rightfootline{\small{INFORMATIKA I EE PRIMENENIYA~---
INFORMATICS AND APPLICATIONS\ \ \ 2019\ \ \ volume~13\ \ \ issue\ 2
\hfill \textbf{\thepage}}}

\vspace*{3pt}


   
     
   \Abste{An approach to a~research of some types of fraud in digital economy with the usage of relationships of 
cause and effect is formulated. In all types of the considered frauds, the discrepancy between the 
purposes of financial transactions and actual cost of achievement of these purposes
has to be observed. Data on 
transactions can be collected by observing information flows in which these transactions are reflected. 
The architecture of data collection and their analysis can be organized by means of the distributed 
ledgers with the centralized consensus that allows creating an analog of the electronic account book 
fixing financial and economic activity of subjects of digital economy in the region. 
   The methods of fraud identification considered are based on the contradictions 
between actions described in transactions and information, which is contained in plans, standards, 
precedents, etc. 
   The method based on a~simplified scheme of implementation of the abstract project is considered. 
For identification of contradictions, it is necessary to carry out the analysis from the effect to the cause, 
i.\,e., to look for anomalies in information describing the generation of the observed effects. 
   It is shown how in implementation of the project it is possible to allocate simple ``necessary 
conditions'' of violation of cause and effect relationships, i.\,e., a~set of ``necessary conditions'' 
violation of which demonstrates fraud existence. It is possible to call this set of "necessary conditions" 
by metadata for control of the project for fraud identification.} 
   
   \KWE{digital economy; information flows; relationships of reason and effect; detection of 
fraudulent schemes}
   
  

 \DOI{10.14357/19922264190204}

\vspace*{-20pt}

 \Ack
   \noindent
   The work was partially supported by the Russian Foundation for Basic Research (projects  
18-29-03081 and 18-07-00274).



%\vspace*{6pt}

  \begin{multicols}{2}

\renewcommand{\bibname}{\protect\rmfamily References}
%\renewcommand{\bibname}{\large\protect\rm References}

{\small\frenchspacing
 {\baselineskip=10.5pt
 \addcontentsline{toc}{section}{References}
 \begin{thebibliography}{9}
\bibitem{1-gr-1}
\Aue{Grusho, A.\,A., A.\,A.~Zatsarinny, and E.\,E.~Timonina.} 2019. Blokcheyny tsifrovoy ekonomiki 
na baze sistemy situatsionnykh tsentrov i~tsentralizovannogo konsensusa [Blockchains of digital 
economy on the basis of the system of the situational centres and the centralized consensus]. 
\textit{25th Scientific and Technical Conference (International) ``Radar-Location, Navigation, 
Communication'' Proceedings}. Voronezh: VSU Publs. 6:183--191.
\bibitem{2-gr-1}
\Aue{Grusho, A., A.~Zatsarinny, and E.~Timonina.} 2019 (in press). 
A~system approach to information security 
in distributed ledgers on the situational centers platform. 
Lecture notes in computer science ser. Springer.
\bibitem{3-gr-1}
\Aue{Finn, V.\,K.} 2011. \textit{Iskusstvennyy intellekt: Metodologiya, primeneniya, filosofiya} 
[Artificial intelligence: Methodology, applications, philosophy]. Moscow: KRASAND. 448~p.

\bibitem{5-gr-1}
\Aue{Anshakov, O.\,M., and E.\,F.~Fabrikantova}. 2009. \textit{DSM-metod avtomaticheskogo porozhdeniya gipotez: Logicheskie 
i~epistemologicheskie osnovaniya} [JSM-method of automatic hypothesis generation: Logical and 
epistemological]. Moscow: KD LIBROKOM. 432~p.
\bibitem{4-gr-1} %5
\Aue{Poelmans, J., P.~Elzinga, S.~Viaene, and G.~Dedene.} 2010. Formal concept analysis in 
knowledge discovery: A~survey. \textit{Conceptual structures: From information to intelligence}. 
Eds.\ M.~Croitoru, S.~Ferr$\acute{\mbox{e}}$, and D.~Lukose. Lecture notes in 
computer science ser. Berlin--Heidelberg: Springer. 6208:139--153.

\bibitem{6-gr-1}
\Aue{Pankratov, E.\,S., and V.\,K.~Finn}. 
2009. \textit{Avtomaticheskoe porozhdenie gipotez v~intellektual'nykh 
sistemakh} [Automatic hypotheses generation in intelligent systems]. Moscow: KD 
\mbox{LIBROKOM}.  528~p. 
\bibitem{7-gr-1}
\Aue{Denisov, A.\,A., and D.\,N.~Kolesnikov.} 1982. \textit{Teoriya bol'shikh 
sistem upravleniya} [Theory of big control systems]. Leningrad: Energoizdat. 488~p.

\bibitem{9-gr-1}
\Aue{Grusho, A.\,A., N.\,A.~Grusho, M.\,I.~Zabezhailo, D.\,V.~Smirnov, and 
E.\,E.~Timonina.} 2018. 
Parametrizatsiya v~prikladnykh zadachakh poiska empiricheskikh prichin 
[Parametrization in applied 
problems of search of the empirical reasons]. 
\textit{Informatika i~ee Primeneniya~--- 
Inform. Appl.} 12(3):62--66.

\bibitem{8-gr-1}
\Aue{Grusho, A.\,A., N.\,A.~Grusho, M.\,V.~Levykin, and E.\,E.~Timonina.} 2018. Metody 
identifikatsii zakhvata khosta v~raspredelennoy informatsionno-vychislitel'noy sisteme, 
zashchishchennoy s~pomoshch'yu metadannykh [Methods of identification of host capture 
in the  distributed information system which is protected on the base of meta data].
\textit{Informatika i~ee 
Primeneniya~--- Inform. Appl.} 12(4):41--45.
{ %\looseness=1

}

\end{thebibliography}

 }
 }

\end{multicols}

\vspace*{-12pt}

\hfill{\small\textit{Received April 3, 2019}}

%\pagebreak

%\vspace*{-18pt}

\Contr

\noindent
\textbf{Grusho Alexander A.} (b.\ 1946)~--- Doctor of Science in physics and 
mathematics, professor, principal scientist, Institute of Informatics Problems, 
Federal Research Center ``Computer Sciences and Control'' of the Russian 
Academy of Sciences; 44-2~Vavilov Str., Moscow 119133, Russian Federation; 
\mbox{grusho@yandex.ru} 

\vspace*{3pt}

\noindent
\textbf{Zabezhailo Michael I.} (b.\ 1956)~--- Doctor of Science in physics and 
mathematics, principal scientist, Institute of Informatics Problems, Federal Research 
Center ``Computer Sciences and Control'' of the Russian Academy of Sciences;  
44-2~Vavilov Str., Moscow 119133, Russian Federation; 
\mbox{m.zabezhailo@yandex.ru} 

\vspace*{3pt}


\noindent
\textbf{Grusho Nikolai A.} (b.\ 1982)~--- Candidate of Science (PhD) in physics 
and mathematics, senior scientist, Institute of Informatics Problems, Federal 
Research Center ``Computer Sciences and Control'' of the Russian Academy of 
Sciences; 44-2~Vavilov Str., Moscow 119133, Russian Federation; 
\mbox{info@itake.ru} 

\vspace*{3pt}


\noindent
\textbf{Timonina Elena E.} (b.\ 1952)~--- Doctor of Science in technology, 
professor, leading scientist, Institute of Informatics Problems, Federal Research 
Center ``Computer Sciences and Control'' of the Russian Academy of Sciences;  
44-2~Vavilov Str., Moscow 119133, Russian Federation; 
\mbox{eltimon@yandex.ru} 

\label{end\stat}

\renewcommand{\bibname}{\protect\rm Литература}     %9
\def\stat{stupnikov}

\def\tit{ВЕРИФИЦИРУЕМОЕ ОТОБРАЖЕНИЕ МОДЕЛИ ДАННЫХ, ОСНОВАННОЙ НА~МНОГОМЕРНЫХ МАССИВАХ, 
В~ОБЪЕКТНУЮ~МОДЕЛЬ ДАННЫХ$^*$}

\def\titkol{Верифицируемое отображение модели данных, основанной на~многомерных массивах, 
в~объектную модель данных}

\def\autkol{С.\,А.~Ступников}

\def\aut{С.\,А.~Ступников$^1$}

\titel{\tit}{\aut}{\autkol}{\titkol}

{\renewcommand{\thefootnote}{\fnsymbol{footnote}}\footnotetext[1] {Работа 
выполнена при поддержке РФФИ (проект 11-07-00402-а). Статья рекомендована к 
публикации в журнале Программным комитетом конференции <<Электронные 
библиотеки: перспективные методы и технологии, электронные коллекции>> 
(RCDL-2012).}}

\renewcommand{\thefootnote}{\arabic{footnote}}
\footnotetext[1]{Институт проблем информатики Российской академии наук, 
ssa@ipi.ac.ru}

\vspace*{-6pt}       

\Abst{Рассматривается отображение модели данных, основанной на 
многомерных мас\-си\-вах (ММ-мо\-де\-ли), в объектную модель данных. Изложены 
общие принципы отображения ММ-мо\-де\-лей в объектные модели данных. 
Рассмотрено отображение конкретной модели~--- Array Data Model (ADM), 
использующейся в системе управления базами данных (СУБД) SciDB, в язык СИНТЕЗ, 
использующийся в качестве канонической модели данных в технологии предметных 
посредников. Проиллюстрирован метод верификации отображения~--- доказательства 
сохранения информации и семантики операций при отображении. Верификация 
осуществляется при помощи формального языка спецификаций AMN. Практической 
целью работы ставилось создание базы для виртуальной или материализованной 
интеграции ресурсов, основанных на многомерных массивах.}

\vspace*{-1pt}

\KW{многомерные массивы; объектная модель данных; отображение моделей 
данных; интеграция баз данных}

\vspace*{-6pt}

\vskip 14pt plus 9pt minus 6pt

      \thispagestyle{headings}

      \begin{multicols}{2}

            \label{st\stat}
            

\section{Введение}

        Развитие науки и промышленности, широкое распространение 
информационных технологий ведет к накоплению огромных объемов данных 
как в науке, так и в бизнесе. Данные могут быть как наблюдательными, 
экспериментальными, так и полученными в ходе компьютерного 
моделирования. Данные таких масштабов (часто измеряемых уже в петабайтах) 
называются <<большими данными>> (Big Data)~\cite{1-stu}. Они плохо 
поддаются обработке и анализу в рамках широко известных технологий баз 
данных, опирающихся в основном на реляционную модель данных.
        
        Именно поэтому развиваются различные модели данных, нацеленные на 
параллельную обработку и анализ данных в распределенных средах~--- гридах 
и облаках. Важными видами таких моделей являются модели данных, 
основанные на многомерных массивах (array-based data models, или ADM) 
и называемые далее ММ-мо\-де\-ля\-ми. Родственны данным моделям 
так называемые <<кубы данных>>, используемые в 
OLAP (online analytical processing) тех\-но\-ло\-гии~[2--4]. 
Исследования ММ-мо\-де\-лей начались достаточно 
давно~\cite{4-stu, 5-stu} и продолжают развиваться. В~данной статье 
рассматривается конкретная модель, а именно модель, используемая в СУБД 
SciDB~\cite{6-stu}.
        
        История SciDB начинается с 2007~г., когда на симпозиуме по 
экстремально большим базам данных (XLDB~--- extremely large data bases) 
представителями науки и 
промышленности был сделан вывод о том, что существующие СУБД не в 
состоянии манипулировать объемами данных, которые появятся в ближайшем 
будущем. Одним из примеров поставщиков таких данных служит строящийся 
телескоп LSST (Large Synoptic Survey Telescope)~\cite{7-stu}. Был также сделан 
вывод о необходимости разработки СУБД нового поколения, которая должна 
удовлетворять, в частности, следующим требованиям~\cite{8-stu}:
        \begin{itemize}
\item модель данных основывается на многомерных массивах, а не на 
кортежах;
\item модель хранения базируется на версионности, а не на обновлении 
значений;
\item масштабируемость до сотен петабайт и высокая отказоустойчивость;
\item СУБД является свободно распространяемым программным 
обеспечением.
\end{itemize}

        Некоторое время спустя был запущен международный проект под 
руководством Майкла Стоунбрейкера, целью которого стало создание новой 
СУБД, получившей название SciDB. В~настоящее 
время свободно распространяется очередная версия системы для операционных
сис\-тем (ОС) Ubuntu и  RedHat.
        
        Целью данной статьи является исследование вопроса о верифицируемом 
отображении ММ-мо\-де\-лей, и в частности ADM~\cite{9-stu}, 
использующейся в системе SciDB, в объектные 
модели данных для виртуальной или материализованной интеграции ресурсов 
при создании федеративных баз данных или хранилищ данных. 
        
        При материализованной интеграции предполагается создание 
хранилища данных (warehouse), в которое загружаются ресурсы, подлежащие 
интеграции. В~процессе загрузки происходит преобразование данных из схемы 
ресурса в общую схему хранилища.
        
        Виртуальная же интеграция рассматривается в статье применительно к 
предметным посредникам~\cite{10-stu}. Предметные посредники представляют 
собой специальный вид программного обеспечения (ПО), образующий 
промежуточный слой между пользователем (приложением) и неоднородными 
информационными ресурсами. При этом данные из ресурсов не 
материализуются в посреднике. Федеративная схема посредника, описывающая 
некоторую предметную область, создается независимо от существующих 
ресурсов. Ресурсы, релевантные предметной области, затем регистрируются в 
посреднике~--- их схемы связываются специальными семантическими 
отображениями с федеративной схемой. Исполнительная среда посредников 
предо\-став\-ля\-ет возможность пользователям (приложениям) задавать запросы 
(программы) к посреднику в терминах федеративной схемы. Эти запросы 
переписываются в частичные запросы над информационными ресурсами, а 
затем исполняются на ресурсах. Результаты частичных запросов объединяются 
и выдаются пользователю также в терминах федеративной схемы.
        
        Важным понятием технологии систем интеграции баз данных является 
каноническая модель, служащая общим языком, унифицирующим 
разнообразные модели ресурсов.
        
        Необходимым предусловием интеграции ресурсов, основанных на 
многомерных массивах, является построение отображения соответствующей\linebreak 
ММ-мо-де\-ли в каноническую модель данных, сохраняющего информацию и 
семантику операций языка манипулирования данными (ЯМД)~\cite{11-stu}. 
Это обусловлено тем, что семантические отображения, связывающие 
федеративную схему и схемы ресурсов, нужно проводить в единой 
(канонической) модели~\cite{12-stu}. Отображение должно быть 
верифицируемым~--- доказуемо правильным. 
        
        В качестве канонической модели в данной работе рассматривается язык 
СИНТЕЗ~\cite{13-stu}~--- комбинированная слабоструктурированная и 
объектная модель данных, нацеленная на разработку предметных посредников 
для решения задач в средах неоднородных ресурсов. Разработан прототип 
программных средств для поддержки среды предметных посредников с языком 
СИНТЕЗ в роли канонической модели~\cite{14-stu}.
        
        С точки зрения предметных посредников СУБД, основанные на 
многомерных массивах, пред\-став\-ля\-ют собой новый вид ресурсов, подлежащих 
интеграции в посредниках вместе с привычными ресурсами~--- реляционными 
и объектными СУБД, веб-сер\-ви\-са\-ми и~т.\,д. 
        
        Нужно отметить, что ADM подвергается некоторой критике со стороны 
исследователей, продолжающих развитие моделей, основанных на 
многомерных массивах. Так, авторы языка SciQL~\cite{15-stu} отмечают, что 
язык ADM является смесью SQL и деревьев алгебраических операций. По их 
мнению, язык для СУБД, основанных на многомерных массивах, должен быть 
интегрирован с синтаксисом и семантикой SQL:2003. Несмотря на эти 
замечания, модель ADM представляет несомненный практический интерес для 
интеграции баз данных. SciDB используется как в научных проектах, связанных 
с LSST (предполагается после запуска телескопа) и физикой высоких энергий, 
так и в коммерческих, связанных с генетикой, страхованием, финансами. 
Сравнительное тестирование SciDB с СУБД Postgres и статистическим ПО R 
показало преимущества SciDB по производительности и масштабируемости.
        
        Статья организована следующим образом. В~разд.~2 рассмотрены и 
проиллюстрированы основные принципы отобра\-же\-ния модели данных ADM в 
язык СИНТЕЗ. Принципы обобщены на случай моделей, отличных от ADM и 
СИНТЕЗ. В~разд.~3 рассмотрен метод доказательства сохранения информации 
и семантики операций при отоб\-ра\-же\-нии моделей с использованием 
формального языка спецификаций AMN~\cite{16-stu}. Метод 
проиллюстрирован на структурах данных и операциях ЯМД моделей SciDB и 
СИНТЕЗ. В~разд.~4 рассмотрены некоторые родственные исследования и 
направления дальнейшей работы.

\vspace*{6pt}

\section{Отображение модели ADM в~язык СИНТЕЗ}

\vspace*{2pt}

        SciDB поддерживает два языка для работы с массивами: AQL (Array 
Query Language) и AFL (Array\linebreak Functional Language). Язык AQL является 
        SQL (Structured Query Language)
        по\-доб\-ным декларативным языком, включающим как операции 
языка описания данных (ЯОД), так и операции ЯМД. Язык AFL представляет собой функциональный язык 
манипулирования массивами, операции которого можно объединять в 
композиции. Допускается использование операций AFL в запросах AQL.
        
        Операции языков и отображение будут иллюстрироваться на 
адаптированных примерах из сценария применения SciDB в области 
оптической астрономии~\cite{17-stu}, а также на простых примерах из 
документации SciDB~\cite{9-stu}.

\subsection{Отображение языка определения данных}

        Отображение ЯОД в данном разделе описывается независимо от вида 
интеграции~--- виртуальной или материализованной.
        
        Основной единицей определения данных в модели ADM является 
массив, имеющий конечное количество {измерений} $d_1, d_2, \ldots , 
d_n$~[9]. Длиной измерения называется количество упорядоченных значений в 
этом измерении. По умолчанию типом измерения являются 64-бит\-ные целые 
числа. Поддерживаются также нецелочисленные измерения, например строки 
или числа с плавающей точкой. Каждая комбинация значений измерений 
соответствует ячейке массива, которая может содержать конечное количество 
значений, называемых \textit{атрибутами}. Типом атрибута может быть один 
из встроенных типов ADM~\cite{9-stu}.
        
        Основная операция ЯОД ADM~--- создание массива~--- выглядит 
следующим образом:
        \begin{verbatim}
CREATE ARRAY source
< ampExposureId: int64 NULL, 
   filterId: int8,
   apMag: double >
[ ra(double), de(double), objectId=0:*];
\end{verbatim}

        Создается массив оптических источников {\sf source}, измерениями 
которого являются координаты {\sf ra} и {\sf de} типа {\sf double} и целочисленный 
идентификатор объекта. Для целочисленного измерения указаны его нижняя (0) 
и верхняя (<<*>>, обозна\-ча\-ющая бесконечность) границы. Ячейка массива 
состоит из трех атрибутов: {\sf ampExposureId}, {\sf filterId}, 
{\sf apMag}. Указано, что 
атрибут {\sf ampExposureId} может принимать неопределенное значение {\sf NULL}. 
В~данном примере приведены только некоторые из реально используемых 
атрибутов и измерений.
        
        В языке СИНТЕЗ создание массива представляется определением 
одноименного класса:
        \begin{verbatim}
{ source; in: class;
  instance_type:{
  double ra;
  ra2long: {in: function; 
            params: {-ret/long}; };
  double de;
  de2long: {in: function; 
            params: {-ret/long}; };
  long objectId; metaslot lower: 0;  
  higher: inf; end
  objectIdBounds: {in: invariant;
    {{all s(source(s) -> s.objectId >= 0)}}
  };
  long ampExposureId;
  short filterId;
  double apMag;
  key: { unique; { ra, de, objectId } };
  definiteness: {obligatory;
    { ra, de, objectId, filterId, apMag } };
  };
}
\end{verbatim}

        Как измерения, так и атрибуты, составляющие ячейку, представляются в 
языке СИНТЕЗ атрибутами типа экземпляров ({\sf instance\_type}) класса. Между 
встроенными типами ADM ({\sf int8}, {\sf int64}, {\sf double} и~др.)\ и встроенными 
типами языка \mbox{СИНТЕЗ} ({\sf short}, {\sf long}, {\sf double}) устанавливается взаимно 
однозначное соответствие. Совокупность атрибутов, со\-от\-вет\-ст\-ву\-ющих 
измерениям, объявляется уникальной (инвариант {\sf key}, выражаемый 
встроенным утверждением {\sf unique}). Объявляется также, что атрибуты, 
соответствующие измерениям и не-{\sf NULL} атрибутам ADM, должны быть 
определены у всех экземпляров класса (инвариант {\sf definiteness}, выражаемый 
встроенным утверждением {\sf obligatory}).
        
        Таким образом обеспечивается сохранение отличи\-тель\-ных свойств 
многомерных массивов (<<кубов данных>>), существенным образом 
раз\-ли\-ча\-ющих измерения и атрибуты, со\-став\-ля\-ющие \mbox{ячейку}:
        \begin{itemize}
\item по набору значений измерений однозначно определяется набор 
значений атрибутов ячейки (уникальность измерений);
\item ячейка массива всегда определяется полным набором значений 
измерений (определенность измерений).
\end{itemize}

        Заметим также, что отсутствие в коллекции объекта с некоторым 
набором значений измерений означает \textit{пустую ячейку} в массиве.
        
        Для нецелочисленных измерений {\sf ra} и {\sf de} в языке СИНТЕЗ кроме 
атрибутов определяются функции {\sf ra2long}, {\sf de2long}, преобразующие 
нецелочисленные значения в целочисленные. Необходимость при\-вне\-се\-ния этих 
функций вызвана следующим. При попытке описать операции, характерные для 
ММ-мо\-де\-лей, в объектной модели (в частности, в языке СИНТЕЗ) 
выясняется необходимость применения принципиально различных механизмов 
работы с целочисленными и нецелочисленными измерениями. Это вызвано 
различием типов измерений, возможной неравномерностью шага измерения 
и~т.\,д.\linebreak Для того чтобы обеспечить возможность единообразного описания 
операций над цело\-чис\-лен\-ными и нецелочисленными измерениями и 
необходимы функции, приводящие нецелочисленные\linebreak измерения к 
целочисленным.
        
        Ограничения, связанные с нижними и верхними границами 
целочисленных измерений, пред\-став\-ля\-ют\-ся в языке СИНТЕЗ, во-пер\-вых, 
мета\-слотом соответствующего атрибута (например,\linebreak {\sf objectId}). В~метаслоте 
хранится метаинформация, связанная с атрибутом как с отдельной сущностью 
языка. В~данном случае метаслот включает два слота {\sf lower} и {\sf higher}, 
отвечающих соответственно верхней и нижней границе измерения. 
        Во-вто\-рых, создается инвариант (например, {\sf objectIdBounds}), 
предикативная спецификация которого устанавливает ограничения на значения 
измерения для каждого из объектов класса, отвечающего массиву. 
Спецификация инварианта имеет вид формулы первого порядка, где {\sf all}~--- 
квантор существования, <<\verb -> >> --- импликация.
        
        Необходимо отметить, что массив представляется в объектной модели 
множеством объектов класса (фактически кортежей значений атрибутов). При 
этом наблюдается некоторое противоречие со стремлением создателей 
        ММ-мо\-де\-лей \mbox{отойти} от моделей, основанных на кортежах. Однако в 
контексте интеграции ресурсов ММ-мо\-де\-ли это лишь один класс из 
большого множества разнообразных классов моделей данных. Представление 
специфических ММ-мо\-де\-лей в объектной модели является методологически 
и технически гораздо более простым и естественным, нежели использование 
многомерных массивов в качестве канонической модели.
        
        Изложенные принципы отображения ЯОД могут быть обобщены на 
случай, когда канонической является объектная или 
        объ\-ект\-но-ре\-ля\-ци\-он\-ная модель, отличная от языка СИНТЕЗ. 
Также не принципиален выбор модели данных, основанной на многомерных 
массивах. В~общем виде принципы отображения ЯОД выглядят следующим 
образом:
        \begin{itemize}
\item массив отображается в коллекцию типизированных объектов (класс) 
объектной модели;
\item измерения и атрибуты, составляющие ячейку массива, отображаются в 
атрибуты типа экземпляров класса;
\item между встроенными типами модели, основанной на многомерных 
массивах, и встроенными типами объектной модели устанавливается 
взаимно однозначное соответствие;
\item совокупность атрибутов, соответствующих измерениям, объявляется 
уникальной (при помощи механизма ключей, утверждений или 
инвариантов);
\item атрибуты, соответствующие измерениям и не-{\sf NULL} атрибутам ячейки 
массива, объявляются определенными (при помощи утверждений или 
инвариантов);
\item для нецелочисленных измерений в типе экземпляров дополнительно 
определяются методы, преобразующие нецелочисленные значения в 
целочисленные;
\item ограничения, связанные с нижними и верхними границами 
целочисленных измерений, отображаются при помощи инвариантов или 
встроенных утверждений о кардинальности соответствующих атрибутов. 
В~случае использования инвариантов при отображении границы измерений 
представляются также метаданными атрибута.
\end{itemize}

\subsection{Отображение языка манипулирования данными}

        При интеграции баз данных для установления семантических 
соотношений между схемами ресурсов и федеративной схемой необходимо 
отображение ЯОД исходной модели в каноническую. Язык манипулирования данными канонической 
модели, напротив, необходимо отображать в ЯМД исходной модели. Это 
связано с тем, что запросы к посреднику в канонической модели необходимо 
отображать в запросы к ресурсам.
        
        Отметим отличие виртуальной и материализованной интеграции. При 
виртуальной интеграции отображение ЯМД обеспечивает возможность 
трансляции программ на языке посредника в запросы на языке ресурсов. 
        
        В случае материализованной интеграции данные извлекаются из ресурса 
и представляются в хранилище в канонической модели. При этом программы 
на языке канонической модели исполняются непосредственно на данных. 
Отоб\-ра\-же\-ние\linebreak ЯМД нужно лишь для того, чтобы убедиться, что отображение 
моделей сохраняет информацию и семантику операций. Семантически 
правильное\linebreak отоб\-ра\-же\-ние служит базой для процесса 
        <<из\-вле\-че\-ния--пре\-образо\-ва\-ния--за\-груз\-ки>> (ETL), 
формирующего из данных ресурса данные хранилища:\linebreak ETL-про\-цесс может 
быть выражен только в терминах канонической модели.
        
        \smallskip
        
        Язык запросов (программ) модели СИНТЕЗ представляет собой 
        Datalog-по\-доб\-ный язык в объектной среде. Программа представляет 
собой набор конъюнктивных запросов (правил) вида 

\noindent
\begin{multline*}
        q(x/T): - C_1(x_1/T_1),\ldots , C_n(x_n/T_n), (X_1,Y_1), 
\ldots \\
\ldots F_m(X_m,Y_m), B\,.
        \end{multline*}
        Тело запроса представляет собой конъюнкцию 
        пре\-ди\-ка\-тов-кол\-лек\-ций, функциональных предикатов и 
ограничения. Здесь $C_i$~--- имена коллекций (классов), $F_i$~--- имена 
функций, $x_i$~--- имена переменных, значения которых пробегают по 
классам, $T_i$~--- типы переменных, $X_j$ и $Y_j$~--- входные и выходные 
параметры функций, $B$~--- ограничение, налагаемое на $x_i$, $X_j$, $Y_j$. 
Предикаты, находящиеся в голове правил, могут быть использованы в телах 
других правил в качестве пре\-ди\-ка\-тов-кол\-лек\-ций. 
        
        В дальнейшем будет часто использоваться запись 
        пре\-ди\-ка\-та-кол\-лек\-ции вида {\sf source([ra, de])}. Неформально это 
означает, что представляют интерес не объекты класса {\sf source} целиком, а 
лишь их атрибуты {\sf ra} и {\sf de}. Формально запись означает сокращение от 
{\sf source(\_/source.inst[ra, de])}. Здесь знак <<{\sf \_}>> обозначает анонимную 
переменную, {\sf source.inst}~--- анонимный тип экземпляров (instance) класса 
{\sf source}, а {\sf ra} и {\sf de}~--- необходимые атрибуты типа экземпляров класса.
        
        Будет также использоваться запись {\sf source([i, j, val1/val])}, означающая 
переименование атрибута {\sf val} в {\sf val1}.
        
        \medskip
        
        При отображении ЯМД будут сначала рассмотрены основные 
конструкции языка программ СИНТЕЗ, соответствующие конструкциям языка 
AQL. Затем будут рассмотрены конструкции \mbox{СИНТЕЗ}, соответствующие 
конструкциям языка AFL.
        
        Начнем рассмотрение с конструкций языка СИНТЕЗ, соответствующих 
конструкциям языка AQL, связанных с {извлечением} данных.
        
%        \smallskip
        
\paragraph*{Предикаты-классы, условия, подзапросы.} Рас\-смот\-рим 
программу, извлекающую координаты ({\sf ra}, {\sf de}) и апертурную звездную 
величину ({\sf apMag}) астрономических источников из класса  {\sf source} с 
условием на фильтр ({\sf filterId}) и апертурную звездную величину, причем 
запрос~{\sf q} использует результаты запроса~{\sf r}:
        \begin{verbatim}
q([ra,de,apMag]) :- r([ra,de,apMag]),
   filterId= #filterId.
r([ra,de,apMag]) :- source([ra,de,apMag]),
   apMag >= #apMag.
\end{verbatim}
Здесь {\sf \#filterId} и {\sf \#apMag}~--- некоторые константы 
соответствующих типов.
        
        Такая программа представляется в AQL сле\-ду\-ющим запросом:
        \begin{verbatim}
SELECT apMag FROM 
  ( SELECT apMag FROM source
    WHERE apMag >= #apMag )
WHERE filterId = #filterId;
\end{verbatim}
        
        Простые условия отображаются в AQL без изменений, рекурсивные 
запросы представляются вложенными запросами. Заметим, что координаты 
{\sf ra} и {\sf de} не указываются в секции {\sf SELECT}~--- они являются измерениями и 
извлекаются по умолчанию.
        
\paragraph*{Соединение классов.} Соединение по определенным атрибутам 
(например, {\sf objectId})
        \begin{verbatim}
q2([ra, de, filterId, uMag]) :- 
    source([ra, de, objectId, fliterId]), 
    objectSummary([objectId, uMag]).
\end{verbatim}
представляется в AQL конструкцией {\sf JOIN-ON}:
\begin{verbatim}
SELECT filterId, uMag INTO q2
FROM source
JOIN objectSummary 
ON source.objectId = objectSummary.objectId;
\end{verbatim}
где массив {\sf objectSummary} имеет вид: 
\begin{verbatim}
CREATE ARRAY objectSummary
<uMag: float NULL,  gMag: float NULL>
[ objectId=0:* ];
\end{verbatim}
        
\paragraph*{Агрегация.} Рассмотрим запрос, возвращающий объекты с 
минимальной звездной величиной {\sf uMag}:
        \begin{verbatim}
q([objectId, uMag]) :-  
  objectSummary(obj/[objectId, uMag]), 
    uMag = min(obj.uMag).
\end{verbatim}

        Запрос представляется в AQL с использованием агрегирующей функции 
того же рода:
        \begin{verbatim}
SELECT uMag
FROM source, 
 (SELECT min(uMag) AS min FROM Source)
WHERE uMag = min;
\end{verbatim}
        
\paragraph*{Группирование.} Рассмотрим запрос, возвра\-ща\-ющий среднее 
значение звездной величины {\sf uMag}, вычисленное на группе по 
идентификатору объекта {\sf filterId}:
        \begin{verbatim}
q([objectId, avgMag]) :- 
    group_by({objectId}, 
       q2(obj/[ra,de,filterId, uMag])),
    avgMag = average(obj.uMag).
\end{verbatim}

        Здесь коллекция {\sf q2}, на которой производится группирование по 
атрибуту {\sf objectId}~--- результат соединения классов {\sf source} и 
{\sf objectSummary}, рассмотренных выше.
        
        Очевидно, в AQL запрос представляется при помощи конструкции 
GROUP BY:
        \begin{verbatim}
SELECT avg(uMag) AS avgMag
FROM q2 GROUP BY objectId;
\end{verbatim}
        
        Рассмотрим конструкции языка СИНТЕЗ, соответствующие 
конструкциям языка AQL и связанные с {изменением} данных.

        
\paragraph*{Обновление.} Рассмотрим запрос, изменяющий значения в 
квадратной матрице (см.\ предыдущий пример) на значения с обратным знаком 
в том случае, если модуль значения больше~5:
        \begin{verbatim}
source(x/[i, j, val]) :- 
    source(x/[i, j, val1/val]), 
       abs(val) > 5, val = -val1.
\end{verbatim}
        
        В AQL данный запрос представляется сле\-ду\-ющим образом:
        \begin{verbatim}
UPDATE source
SET val =  -val WHERE abs(val) > 5;
\end{verbatim}


        
\paragraph*{Удаление.} Рассмотрим программу, удаляющую из базы данных 
класс и все его содержимое:
        \begin{verbatim}
-source(x) :- source(x).; 
delete_frame(source).
\end{verbatim}

        В правилах со знаком минус в голове осуществляется удаление объектов 
из коллекции. В~данном случае из коллекции удаляются все объекты. Функция 
{\sf delete\_frame} удаляет коллекцию как объект из базы данных. Операция <<{\sf ;}>> 
обозначает последовательную композицию программ. В~AQL данный запрос 
представляется при помощи операции {\sf DROP}:
\begin{verbatim}
DROP ARRAY source;
\end{verbatim}

        Рассмотрим принципы отображения конструкций языка СИНТЕЗ, 
соответствующих конструкциям AFL, на примере {расширения элементов 
мас\-си\-ва в подмассивы}. Каждый элемент массива расширя\-ется в подмассив 
определенного размера. Значения всех ячеек подмассива дублируют значение 
оригинальной ячейки. Пример программы, расширяющей каждую ячейку 
матрицы $3\times3$ в подматрицу $2\times2$:
        \begin{verbatim}
q([i,j,val]) :- {x/[i,j,val] | exists y (
  source(y/[i1/i, j1/j, val]) & 
  ( i = i1*2 & j = j1*2 | i = i1*2 +1 & 
  j = j1*2 | i= i1*2 & 
  j= j1*2 + 1 | i= i1*2 +1 & j= j1*2 +1))}.
\end{verbatim}
    Здесь выражение $\{x/T \vert F(x)\}$, где $F$~--- формула со свободной 
переменной~$x$, обозначает конструкцию выделения множества; {\sf exists} 
обозначает квантор существования. 

\columnbreak
        
        В ADM запрос представляется с использованием операции {\sf xgrid}:
        \begin{verbatim}
SELECT * FROM xgrid(source, 2, 2);
\end{verbatim}
        
        Можно заметить, что операция AFL {\sf xgrid} имеет достаточно сложно 
устроенный прообраз в канонической модели (это справедливо и для многих 
других операций). Между тем эти операции являются естественными для 
массивов. Поэтому при интеграции ресурсов, основанных на многомерных 
массивах, в канонической модели возможно использование специального 
класса {\sf array}, инкапсулирующего специфические операции, характерные для 
многомерных массивов:
        \begin{verbatim}
{ array; in: class;
  instance_type: {
  xgrid: { in: function; 
    params: {
     +dimensions/{sequence; 
      type_of_element: string;},
     +scales/{sequence; 
      type_of_element: integer;}};
  };  };
}
\end{verbatim}
        В приведенном примере рассмотрена сигнатура единственной операции 
{\sf xgrid}, параметрами которой являются последовательность имен измерений\linebreak 
{\sf dimensions} и последовательность масштабов увеличения по каждому из 
измерений {\sf scales}. Па\-ра\-мет\-ром операции по умолчанию также считается 
класс\linebreak {\sf array} как коллекция объектов. При отображении ЯОД каждый класс~--- 
образ массива (например, класс {\sf source} из подразд.~2.1) становится подклассом 
класса {\sf array}:
        \begin{verbatim}
{ source; in: class; superclass: array;
  instance_type: { ... };
}
\end{verbatim}

        Аналогично {\sf xgrid}, операциями класса {\sf array} могут быть 
представлены такие операции AFL, как {\sf substitute}, {\sf sort}, 
{\sf multiply} и~т.\,д. 
        
        Заметим, что решение о представлении операций, характерных для 
многомерных массивов, в рамках специального класса канонической модели 
допускает обобщение на объектные канонические модели, отличные от языка 
СИНТЕЗ, и модели, основанные на многомерных массивах, отличные от ADM.
        
        \smallskip
        
        Разработанные отображения ЯОД и ЯМД были частично реализованы на 
языке ATL (ATLAS\linebreak Transformation Language)~\cite{18-stu}. ATL-программы 
пред\-став\-ля\-ют собой де\-кла\-ра\-тив\-но-им\-пе\-ра\-тив\-ные трансформации, 
реализующие отображения произвольных исходных моделей уровня М1 
(согласно классификации MOF~\cite{19-stu}), конформных исходной 
метамодели уровня М2, в целевые модели уровня М1, конформные целевой 
метамодели уровня М2. Модели уровня М1 являются схемами, 
представленными в канонической модели данных или модели ADM; модели 
уровня М2 есть описание абстрактного синтаксиса канонической модели или 
модели ADM. В~качестве метамодели уровня М3, которой конформны 
метамодели уровня M2, рассматривается модель Ecore~\cite{20-stu}. Cинтаксис 
ЯОД и ЯМД ядра канонической информационной модели (языка СИНТЕЗ) и 
модели ADM был представлен в метамодели Ecore. 
        
        Было осуществлено построение ATL-транс\-фор\-ма\-ций, реализующих 
отображения подмножества ЯОД модели ADM в ЯОД канонической модели и 
подмножества ЯМД канонической модели в ЯМД модели ADM. Подмножества 
ЯМД определялись конструкциями ЯОД и ЯМД канонической модели, 
поддерживаемыми в настоящее время в исполнительной среде предметных 
посредников. Поддерживаемый язык запросов канонической модели включает 
правила, в голове которых могут быть пре\-ди\-ка\-ты-кол\-лек\-ции, а в теле~--- 
конъюнкция пре\-ди\-ка\-тов-кол\-лек\-ций, условий соединения коллекций и 
других условий на значения атрибутов типов экземпляров коллекций. 
Условием соединения может быть только равенство атрибутов. 
Поддерживаются основные арифметические предикаты и функции, а также 
термы~--- вызовы функций. 

\section{Сохранение информации и~семантики операций языка манипулирования данными 
при~отображении}
        
        Метод доказательства сохранения информации и семантики операций 
при отображении моделей данных~\cite{21-stu} основывается на представлении 
семантики моделей в формальном языке спецификаций AMN~\cite{16-stu}. 
        
        Язык AMN представляет собой тео\-ре\-ти\-ко-мо\-дель\-ную нотацию, 
основанную на теории множеств и типизированном языке первого порядка. 
Спецификации AMN называются абстрактными машинами. Язык AMN позволяет 
интегрированно рас\-смат\-ри\-вать спецификацию пространства состояний и 
поведения машины (определенного операциями на состояниях). В~AMN 
формализуется специальное отношение между спецификациями~--- 
{уточнение}. Неформально спецификация~$B$ уточняет 
спецификацию~$A$, если пользователь может использовать $B$ вместо~$A$, 
не замечая факта замены~$A$ на~$B$. 
{\looseness=1

}
        
        Идея метода заключается в следующем. Рассмотрим исходную 
модель~$S$ и целевую модель~$T$. Построим отображение~$\theta$ 
модели~$S$ в модель~$T$ (подобно изложенному в предыдущем разделе). 
Выразим семантику моделей в виде абстрактных машин AMN, построив при 
этом машины $M_S$ и $M_T$ соответственно. При этом структуры данных 
моделей~--- классы, массивы~--- представляются переменными машин, 
различные свойства структур данных представляются инвариантами машин, 
характерные операции моделей данных представляются операциями машин. 
Рассматриваемые операции исходной и целевой модели должны быть связаны 
отображением ЯМД. Отображение ЯОД представляется в виде специального 
\textit{склеивающего инварианта}~--- замкнутой формулы, связывающей 
состояния машин~$M_S$ и~$M_T$.
        
        Будем считать отображение~$\theta$ сохраняющим инфор\-ма\-цию и 
семантику операций, если машина~$M_S$, соответствующая исходной модели, 
уточняет машину~$M_T$, соответствующую целевой модели. Уточнение 
доказывается интерактивно при помощи специальных программных 
средств~\cite{22-stu}.
        
        \smallskip
        
        В качестве иллюстрации основных принципов выражения семантики 
моделей ADM и СИНТЕЗ в AMN рассмотрим частичные 
        AMN-спе\-ци\-фи\-ка\-ции, соответствующие данным моделям.
        
        Cпецификация, выражающая семантику объектной модели языка 
СИНТЕЗ, представляется в языке AMN конструкцией {\sf REFINEMENT}:
\begin{verbatim}
REFINEMENT ObjectDM
\end{verbatim}

        Переменные, составляющие пространство состояний объектной модели, 
объявлены в разделе {\sf ABSTRACT\_VARIABLES} машины {\sf ObjectDM} и 
типизируются в разделе {\sf INVARIANT}:
\begin{verbatim}
ABSTRACT_VARIABLES
    typeNames, classNames, attributeNames,
    instanceType, typeAttributes, 
      attributeType,
    unique, obligatory,
    intAttributeLowerBound, 
      intAttributeHigherBound,
    objectIDs, objectType, objectsOfClass,
    integerAttributeValue,
    adtAttributeValue
INVARIANT ...
\end{verbatim}

        Раздел {\sf INVARIANT} содержит формулу, которая состоит из предикатов, 
типизирующих переменные состояния и налагающих различные совместные 
ограничения на переменные. Предикаты соединяются операцией конъюнкции.
        
        Так, имена типов и классов представлены переменными {\sf typeNames} и 
{\sf classNames}, тип которых~--- подмножество множества строк 
({\sf STRING\_Type}):
        \begin{verbatim}
typeNames: POW(STRING_Type) &
classNames: POW(STRING_Type)
\end{verbatim}
        
        \noindent
        Здесь {\sf POW}~--- конструктор множества подмножеств.
        
        Атрибуты (переменная {\sf attributeNames}) пред\-став\-ле\-ны частичной 
функцией (знак <<\verb +-> >>), ставящей в соответствие уникальному идентификатору 
атрибута (натуральному числу из множества {\sf NAT}) имя атрибута (строку):
        \begin{verbatim}
attributeNames: NAT +-> STRING_Type
\end{verbatim}

        Типы экземпляров классов (переменная\linebreak {\sf instanceType}) представлены 
тотальной функцией (знак \verb -> ) из множества имен классов в 
множество имен типов:
        \begin{verbatim}
instanceType: classNames --> typeNames
\end{verbatim}

        Принадлежность атрибутов типам (переменная {\sf typeAttributes}) 
выражена тотальной функцией из множества имен типов в множество 
подмножеств атрибутов:
        \begin{verbatim}
typeAttributes: 
  typeNames --> POW(dom(attributeNames))
\end{verbatim}
        Здесь {\sf dom}~--- операция, возвращающая область определения 
функции, а {\sf dom(attributeNames)}~--- множество имен атрибутов.
        
        Типы значений атрибутов (переменная\linebreak {\sf attributeType}) представлены 
функцией из множества атрибутов в множество идентификаторов встроенных 
типов данных {\sf BuiltInTypes}:
        \begin{verbatim}
attributeType: 
  dom(attributeNames) +-> BuiltInTypes
\end{verbatim}

        Множества уникальных атрибутов типов {\sf unique}\linebreak и множества 
определенных атрибутов типов\linebreak {\sf obligatory} представлены тотальными 
функциями из множества имен типов в множество подмножеств атрибутов:
\begin{verbatim}
unique: 
  typeNames --> POW(dom(attributeNames))&
obligatory: 
  typeNames --> POW(dom(attributeNames))
\end{verbatim}

        Нижние границы целочисленных атрибутов (переменная 
{\sf intAttributeLowerBound}) представлены час\-тич\-ной функцией из множества 
атрибутов в множество целых чисел:
\begin{verbatim}
intAttributeLowerBound: 
  dom(attributeNames) +-> INT
\end{verbatim}

        Аналогично представляются верхние границы.
        
        Идентификаторы объектов (переменная\linebreak {\sf objectIDs}) представлены 
подмножеством натуральных чисел:
        \begin{verbatim}
objectIDs: POW(NAT)
\end{verbatim}

        Типы объектов (переменная {\sf objectType}) представлены тотальной 
функцией из множества объектных идентификаторов в множество имен типов:
\begin{verbatim}
objectType: objectIDs --> typeNames
\end{verbatim}

        Состав классов (переменная {\sf objectsOfClass}) представлен тотальной 
функцией из множества имен классов в множество подмножеств 
идентификаторов объектов:
        \begin{verbatim}
objectsOfClass: 
  classNames --> POW(objectIDs)
\end{verbatim}
        
        Значения атрибутов объектов (переменные\linebreak {\sf integerAttributeValue}, 
{\sf adtAttributeValue} и~др.)\ пред\-став\-ле\-ны функциями из множества атрибутов\linebreak 
в функции из множества идентификаторов объектов в множества значений 
атрибутов. Для простоты рассмотрены лишь функции для целочисленных 
атрибутов и атрибутов со значениями АТД\linebreak (абстрактного типа данных) (объектами):
        \begin{verbatim}
integerAttributeValue: 
 dom(attributeNames) +-> (objectIDs+->INT)& 
adtAttributeValue: 
 dom(attributeNames) +-> (objectIDs+->NAT)
\end{verbatim}
        
        Дополнительные необходимые свойства переменных состояния 
представлены конъюнктивными компонентами инварианта. Так, каждый 
атрибут является атрибутом некоторого типа:
        \begin{verbatim}
        
UNION(tt).(tt:typeNames|typeAttributes(tt))=
    dom(attributeNames)
\end{verbatim}
        Здесь {\sf UNION}~--- родовая операция объединения, в данном случае 
объединяются множества атрибутов {\sf typeAttributes(tt)} по всем именам 
типов~{\sf tt} из множества {\sf typeNames}. 
        
        Никакой атрибут не принадлежит двум типам одновременно:
        \begin{verbatim}
!(t1, t2).(t1: typeNames & t2: typeNames =>
  (typeAttributes(t1) /\ typeAttributes(t2) 
    = {}))
\end{verbatim}
   Здесь <<\verb ! >>~--- знак квантора всеобщности, <<\verb => >>~--- логическая 
импликация, <<\verb /\ >>~--- символ пересечения множеств, <<\verb {} >>~--- пустое 
множество.
        
        Уникальные и определенные атрибуты типа выбираются из множества 
атрибутов типа:
        \begin{verbatim}
!(tt).(tt: dom(unique) => unique(tt) <: 
typeAttributes(tt)) &
!(tt).(tt: dom(obligatory) => 
    obligatory(tt) <: typeAttributes(tt))
\end{verbatim}
        Здесь <<\verb <: >>~--- символ отношения мно\-жес\-во--под\-мно\-жество.
        
        Нижние и верхние границы могут быть определены только для 
целочисленных атрибутов:
        \begin{verbatim}
!(attr).(attr: dom(intAttributeLowerBound)=> 
    attributeType(attr) = Integer) 
\end{verbatim}

        Тип объекта~--- экземпляра класса есть тип экземпляров этого класса:
        \begin{verbatim}
!(cc).(cc: classNames => 
    !(oo).(oo: objectsOfClass(cc) => 
       objectType(oo) = instanceType(cc))) 
\end{verbatim}

        Для каждого атрибута определена ровно одна функция значений:
        \begin{verbatim}
dom(adtAttributeValue) /\ 
  dom(integerAttributeValue) = {} &
dom(adtAttributeValue) \/ 
  dom(integerAttributeValue) = 
    dom(attributeNames)
\end{verbatim}
   Здесь <<\verb \/ >>~--- символ объединения множеств.
        
        Для любого объекта и любого определенного атрибута типа этого 
объекта функция значений атрибута определена на объекте:
        \begin{verbatim}
!(oo, aa).(oo: dom(objectType) & 
  aa: typeAttributes(objectType(oo)) & 
  aa: obligatory(objectType(oo)) =>
      (attributeType(aa) = Integer => 
       oo: dom(integerAttributeValue(aa))) &
      (attributeType(aa) = ADT =>
       oo: dom(adtAttributeValue(aa)))) 
\end{verbatim}

        Для любого объекта и любого целочисленного атрибута типа объекта, 
определенного на объекте и для которого определена нижняя (верхняя) 
граница, значение атрибута не меньше (не больше) нижней (верхней) границы:
        \begin{verbatim}
!(oo, aa).(oo: objectIDs & 
    aa: typeAttributes(objectType(oo)) &
    oo: dom(integerAttributeValue(aa) => 
    (aa: dom(intAttributeLowerBound) =>
        (integerAttributeValue(aa)(oo) >= 
         intAttributeLowerBound(aa))) ) 
\end{verbatim}

        Объект однозначно идентифицируется набором своих уникальных 
атрибутов:
        \begin{verbatim}
!(oo1, oo2).(oo1: objectIDs & 
  oo2: objectIDs &
    objectType(oo1) = objectType(oo2) & 
    unique(objectType(oo1)) /= {} &
    !(aa).(aa: unique(objectType(oo1)) => 
      (attributeType(aa) = Integer =>
        integerAttributeValue(aa)(oo1) =
         integerAttributeValue(aa)(oo2)) &
      (attributeType(aa) = ADT =>
         adtAttributeValue(aa)(oo1) =
          adtAttributeValue(aa)(oo2)) ) => 
    oo1 = oo2 )
\end{verbatim}

        Из всего ЯМД в спецификации рассмотрена единственная операция 
{\sf update} обновления значений атрибута в объектах класса:
        \begin{verbatim}
OPERATIONS
update(cls, attr, exp, cond) =
PRE cls: classNames & 
  attr: typeAttributes(instanceType(cls)) &
  attributeType(attr) = Integer &
  exp: INT --> INT & cond: NAT --> BOOL
THEN
 integerAttributeValue := 
 integerAttributeValue <+ 
 { xx | xx: (NAT*(NAT<->INT)) &
  #(oo, val).( oo: objectsOfClass(cls) & 
  val: INT &
    xx = attr |-> ({oo |-> val}) & 
  (cond(integerAttributeValue(attr)(oo)) 
  = TRUE =>
  val=exp(integerAttributeValue(attr)(oo)))&
  (cond(integerAttributeValue(attr)(oo)) 
  = FALSE => 
  val=integerAttributeValue(attr)(oo)))}
END
\end{verbatim}

        Параметрами операции являются имя класса {\sf cls}, имя целочисленного 
атрибута {\sf attr} типа экземпляров класса, функция {\sf exp}, отвечающая за 
преобразование атрибута, и функция {\sf cond}, отвечающая условию на значение 
атрибута. Пусть {\sf o}~--- некоторый объект класса {\sf cls}, для которого определено 
значение атрибута {\sf attr}, и это значение есть~{\sf v}. Тогда операция {\sf update} 
изменяет значение атрибута на {\sf exp(v)} в случае, если выражение {\sf cond(v)} 
принимает значение <<истина>>, и оставляет значение атрибута без изменений в 
противном случае. Очевидно, такая операция {\sf update} есть обобщение примера 
обновления, рассмотренного в подразд.~2.2. Действительно, для рассмотренного 
примера {\sf cls} есть {\sf source}, {\sf attr} есть {\sf val}, 
{\sf exp(v)}\;=\;-\,{\sf v}, {\sf cond(v)}\;=\;{\sf abs(v)}.
        
        Заметим, что в рассмотренной спецификации для простоты не 
рассмотрены некоторые черты объектной модели, например отношения 
        тип--под\-тип и класс--под\-класс.
        
        \smallskip
        
        Спецификация, выражающая семантику модели ADM, представляется в 
языке AMN конструкцией
        \begin{verbatim}
REFINEMENT ArrayDM
\end{verbatim}

        Переменные, составляющие пространство состояний объектной модели, 
объявлены в разделе {\sf ABSTRACT\_VARIABLES} машины {\sf ArrayDM}:
        \begin{verbatim}
ABSTRACT_VARIABLES
    arrayNames, dimensionNames, 
    cellAttributeNames,
    arrayDimensions, arrayCellAttributes,    
    cellAtrributeType, nullable, 
    dimLowerBound, dimHigherBound,
    cells, dimensionValue, 
    integerCellAttributeValue
\end{verbatim}

        Имена массивов представлены переменной\linebreak 
{\sf arrayNames}; имена измерений~--- переменной\linebreak 
{\sf  dimensionNames}; имена атрибутов ячеек массива~--- переменной 
\mbox{{\sf cellAttributeNames}}; принадлежность измерений массивам~--- переменной 
\mbox{{\sf arrayDimensions}}; принадлежность атрибутов ячеек~--- переменной 
\mbox{{\sf arrayCellAttributes}}; 
тип атрибута ячейки~--- переменной \mbox{{\sf cellAtrributeType}}; 
атрибуты ячеек массивов, которые могут принимать неопределенные 
значения,~--- переменной \mbox{{\sf nullable}}; верхние (нижние) границы измерений~--- 
переменной \mbox{{\sf dimLowerBound}} (\mbox{{\sf dimHigherBound}}); множества 
идентификаторов ячеек массивов~--- переменной 
\mbox{{\sf cells}}, значения измерений в 
ячейках~--- переменной \mbox{{\sf dimensionValue}}; значения атрибутов ячеек~--- 
переменной \mbox{{\sf integerCellAttributeValue}}. Переменные типизируются в разделе 
\mbox{{\sf INVARIANT}} при помощи частичных и тотальных функций аналогично 
переменным, использующимся для придания семантики объектной модели:
        \begin{verbatim}
INVARIANT
   arrayNames: POW(STRING_Type) &
   dimensionNames: NAT +-> STRING_Type &
   cellAttributeNames: NAT +-> STRING_Type &
   arrayDimensions: arrayNames --> 
   POW(dom(dimensionNames)) &
   arrayCellAttributes: arrayNames --> 
     POW(dom(cellAttributeNames)) &
   cellAtrributeType: 
     dom(cellAttributeNames) --> 
       BuiltInTypes &
   nullable: 
     dom(cellAttributeNames) --> BOOL &
   dimLowerBound: 
     dom(dimensionNames) --> INT &
   dimHigherBound: 
     dom(dimensionNames) +-> INT &
   cells: arrayNames --> POW(NAT) & 
   dimensionValue: 
     NAT*dom(dimensionNames) +-> INT  &
   integerCellAttributeValue: 
     NAT*dom(cellAttributeNames) +-> INT &
\end{verbatim}
        Здесь <<\verb * >>~--- знак декартова произведения множеств.
        
        Дополнительные необходимые свойства переменных состояния 
представлены конъюнктивными компонентами инварианта. Так, любая ячейка 
любого массива однозначно идентифицируется набором значений измерений:
        \begin{verbatim}
!(arr, cell1, cell2).(arr: arrayNames & 
  cell1: cells(arr) &  cell2: cells(arr) &
  !(dim).(dim: arrayDimensions(arr) =>
    dimensionValue(cell1, dim) = 
    dimensionValue(cell2, dim)) =>
    cell1 = cell2)
        \end{verbatim}
        
                \vspace*{-6pt}
        
        Для любой ячейки любого массива определены значения всех измерений 
и значение по крайней мере одного атрибута:
        \begin{verbatim}
!(arr, cell).(arr: arrayNames & 
 cell: cells(arr) =>
  !(dim).(dim: arrayDimensions(arr) => 
   (cell |-> dim): dom(dimensionValue)) &
   #(attr).(attr: arrayCellAttributes(arr) & 
    cellAtrributeType(attr) = Integer & 
    (cell, attr): 
      dom(integerCellAttributeValue)) )
        \end{verbatim}
        
        \vspace*{-6pt}
        
        Аналогично объектной модели рассмотрена единственная операция 
ЯМД~--- операция об\-нов\-ле\-ния {\sf update}:
        \begin{verbatim}
OPERATIONS
update(cls, attr, exp, cond) =
PRE cls: arrayNames & 
 attr: arrayCellAttributes(cls) &
  cellAtrributeType(attr) = Integer &
  exp: INT --> INT & cond: NAT --> BOOL
THEN
  integerCellAttributeValue := 
  integerCellAttributeValue <+
  { yy | yy: (NAT*NAT)*INT &
    #(cell, val).(cell: cells(cls) & 
     val: INT & 
    yy = ((cell |-> attr)|-> val) &
    (cond(integerCellAttributeValue(cell, 
     attr)) = TRUE =>
      val = 
       exp(integerCellAttributeValue(cell,
        attr))) &
      (cond(integerCellAttributeValue(cell, 
       attr))= FALSE  =>
    val = 
     integerCellAttributeValue(cell,attr)))}
END   
END
        \end{verbatim}
        
                \vspace*{-6pt}
        
        Сигнатура операции совпадает с сигнатурой операции объектной 
модели. Семантика операции также аналогична: значение~{\sf v} атрибута {\sf attr} 
массива {\sf cls} заменяется на {\sf exp(v)}, если значение {\sf cond(v)} есть 
<<истина>>, и не изменяется в противном случае. 
        
        Заметим, что в данной спецификации для прос\-то\-ты не рассмотрены 
некоторые черты ADM, например нецелочисленные измерения.
        
        \smallskip
        
        Для формального доказательства того, что машина {\sf ArrayDM} уточняет 
машину {\sf ObjectDM}, необходимо построить {инвариант уточнения}, 
связы\-вающий переменные машин, и добавить его к\linebreak инварианту уточняющей 
машины. 
        
        Инвариант формализует принципы отображения ЯОД, изложенные в 
подразд.~2.1, и объединяет их в одну конъюнкцию.
        
        Так, множество имен массивов совпадает с множеством имен классов:
        \begin{verbatim}
classNames = arrayNames
\end{verbatim}

%                \vspace*{-6pt}
        
        Множество идентификаторов и имен измерений и атрибутов ячеек 
совпадает с множеством идентификаторов и имен атрибутов типов экземпляров 
классов:
        \begin{verbatim}
attributeNames = 
  dimensionNames \/ cellAttributeNames
\end{verbatim}

%                \vspace*{-6pt}

        Любому измерению любого массива соответствует атрибут типа 
экземпляра класса, соответствующего этому массиву:
        \begin{verbatim}
!(arr, dim).(arr: arrayNames & 
  dim: arrayDimensions(arr) =>
    #(attr).(attr: 
     typeAttributes(instanceType(arr)) &
          attr = dim & 
          attributeType(attr) = Integer) )s
        \end{verbatim}
        
                        \vspace*{-6pt}
        
        Любому атрибуту ячейки любого массива соответствует атрибут типа 
экземпляра класса, соответствующего этому массиву, и типы атрибутов 
совпадают:
        \begin{verbatim}
!(arr, cattr).(arr: arrayNames & 
   cattr: arrayCellAttributes(arr) =>
    #(attr).(attr: 
      typeAttributes(instanceType(arr)) & 
         attr = cattr & 
         attributeType(attr) = 
           attributeType(cattr)))
        \end{verbatim}
        
                        \vspace*{-9pt}
        
        Атрибут ячейки массива, который может принимать неопределенные 
значения, соответствует определенному ({\sf obligatory}) атрибуту типа:
        \begin{verbatim}
!(arr, cattr).(arr: arrayNames & 
  cattr /: dom(nullable) &
    cattr: arrayCellAttributes(arr) => 
    cattr: obligatory(instanceType(arr)) )
        \end{verbatim}
        
\vspace*{-9pt}

           Здесь знак <<\verb /: >> обозначает отношение непринадлежности элемента 
множеству.
        
        Измерения соответствуют уникальным атрибутам типов:
        \begin{verbatim}
!(arr, dim).(arr: arrayNames & 
    dim: arrayDimensions(arr) => 
      dim: unique(instanceType(arr)) )
        \end{verbatim}
        
                        \vspace*{-6pt}
        
        Верхние (нижние) границы измерений равны верхним (нижним) 
границам соответствующих атрибутов типов:
        \begin{verbatim}
!(dim).(dim: dom(dimLowerBound) =>
    dim: dom(intAttributeLowerBound) & 
    dimLowerBound(dim) = 
      intAttributeLowerBound(dim))
        \end{verbatim}
        
                        \vspace*{-6pt}
        
        Непустые ячейки массивов соответствуют объектам классов:
        \begin{verbatim}
cells = objectsOfClass
\end{verbatim}

%                \vspace*{-6pt}

        Для любой ячейки значения ее измерений и определенных атрибутов 
совпадают со значениями соответствующих атрибутов объекта, 
соответствующего ячейке:
        \begin{verbatim}
!(cell, dim).(cell: NAT & dim: NAT & 
  (cell |-> dim): dom(dimensionValue) =>
  cell: dom(integerAttributeValue(dim)) &
  dimensionValue(cell, dim) = 
    integerAttributeValue(dim)(cell)) &
!(cell, cattr).(cell: NAT & cattr: NAT & 
   (cell |-> cattr): 
   dom(integerCellAttributeValue) =>
   cell: dom(integerAttributeValue(cattr)) &
   integerCellAttributeValue(cell, cattr) =
     integerAttributeValue(cattr)(cell) )
        \end{verbatim}
        
                        \vspace*{-6pt}
        
        Для указания того, что машина {\sf ArrayDM} уточняет машину 
{\sf ObjectDM}, в машину {\sf ArrayDM} была добавлена директива
        \begin{verbatim}
REFINES ObjectDM
\end{verbatim}

%                \vspace*{-6pt}

        Спецификации {\sf ObjectDM} и {\sf ArrayDM} вместе с инвариантом 
уточнения были загружены в инструментальное средство 
        Atelier~B~\cite{22-stu}. Автоматически были сгенерированы теоремы, 
выражающие уточнение спецификаций. В~частности, для операции {\sf update} 
были сгенерированы 10~тео\-рем. Три из них были доказаны автоматически, 
для доказательства остальных необходимо применять интерактивные средства 
доказательства.

\vspace*{-9pt}
  
\section{Родственные исследования и~направления дальнейшей 
работы}

\vspace*{-2pt}

        Родственными данной работе следует считать исследования, связанные с 
отображением моделей, основанных на многомерных массивах, в реляционную 
модель данных. Обычно они нацелены на реализацию многомерных массивов 
при помощи реляционных СУБД. Такие работы начались одновременно с 
исследованиями моделей, основанных на многомерных массивах~\cite{5-stu}, и 
продолжаются в настоящее время~\cite{23-stu}.
        
        Основные особенности данной работы состоят в следующем. 
В~качестве исходной модели при отображении используется специфическая 
модель, основанная на многомерных массивах СУБД SciDB, язык которой 
представляет собой комбинацию декларативного SQL-по\-доб\-но\-го языка и 
функционального языка, включающего специфические\linebreak операции над 
многомерными массивами. В~качестве целевой модели используется объектная 
модель с Datalog-по\-доб\-ным языком запросов (программ)~--- язык СИНТЕЗ. 
Для отображения\linebreak обеспечивается формальное доказательство сохранения 
информации и семантики операций ЯМД.
        
        Отметим, что результаты работы могут быть с легкостью обобщены и 
использованы при интеграции в системах, использующих каноническую 
модель, отличную от языка СИНТЕЗ, например другую объектную (ODMG) 
или объект\-но-ре\-ля\-ци\-он\-ную модель (SQL:2003). Результаты также могут 
быть использованы для интеграции ресурсов, представленных в модели, 
основанной на многомерных массивах, но отличной от ADM.
        
        Некоторые вопросы отображения требуют дальнейших исследований. 
Например, следует ли иметь в канонической модели при интеграции 
        масс\-сив-ори\-ен\-ти\-ро\-ван\-ных моделей данных операции, 
связанные с размером порции (chunk size) данных в БД~\cite{9-stu}?
        
        Дальнейшую работу можно разбить на два этапа:
        \begin{enumerate}[(1)]
\item расширение инструментальных средств поддержки предметных 
посредников для виртуальной интеграции SciDB-ресурсов: 
\begin{itemize}
\item[(а)] расширение средств регистрации ресурсов в посреднике~\cite{10-stu} 
трансформацией ЯОД\ ADM в каноническую модель; 
\item[(б)] создание 
SciDB-адап\-те\-ра~--- специального ПО, связывающего исполнительную 
среду посредников с SciDB-ресурсами (составной частью адаптера является 
разработанная трансформация ЯМД);
\end{itemize}
\item применение технологии предметных посредников для решения 
научных задач в некоторой предметной области над множеством\linebreak 
неоднородных ресурсов, включающим SciDB-ре\-сурсы.
\end{enumerate}

\bigskip
        Автор выражает благодарность Л.\,А.~Калиниченко, П.\,Е.~Велихову и 
А.\,Е.~Вовченко за полезные замечания, высказанные в ходе обсуждения 
данной работы на семинарах ИПИ РАН.

\vspace*{-6pt}

{\small\frenchspacing
{%\baselineskip=10.8pt
\addcontentsline{toc}{section}{Литература}
\begin{thebibliography}{99}

\vspace*{-2pt}

\bibitem{1-stu} %1
Challenges and opportunities with big data~// A~community white paper developed 
by leading researchers across the United States, 2012. {\sf http://cra.org/ccc/docs/ init/bigdatawhitepaper.pdf}. 

\bibitem{1-2-stu} %2
\Au{Abrial J.-R.} The B-Book: Assigning programs to 
meanings.~--- Cambridge: Cambridge University Press, 1996. 

\bibitem{2-stu} %3
\Au{Vassiliadis P., Sellis T.\,K.} A~survey of logical models for OLAP databases~// SIGMOD 
Record, 1999. Vol.~28. No.\,4. P.~64--69. 

\bibitem{3-stu}
\Au{Pedersen T.\,B., Jensen C.\,S.} Multidimensional database technology~// IEEE Computer, 
2001. Vol.~34. No.\,12. P.~40--46. 

\bibitem{4-stu} %5
\Au{Libkin L., Machlin R., Wong~L.} A~query language for multidimensional arrays: Design, 
implementation, and optimization techniques.~--- SIGMOD, 1996. P.~228--239. 
\bibitem{5-stu} %6
\Au{Baumann P.} A~database array algebra for spatio-temporal data and beyond~// Next 
generation information technologies and systems. Lectures notes in computer science ser.
Springer Verlag KG, 1999. Vol.~1649. P.~76--93.
\bibitem{6-stu} %7
Overview of SciDB: Large scale array storage, processing and analysis. The SciDB development 
team.~--- SIGMOD, 2010. 
\bibitem{7-stu}
Large synoptic survey telescope. {\sf http://www.lsst.org}. 
\bibitem{8-stu}
\Au{Becla J., Lim K.-T.} Report from the First Workshop on Extremely Large Databases~// Data 
Sci.~J., 2008. Vol.~7.
\bibitem{9-stu}
SciDB User's Guide. Version~12.3, 2012. {\sf http:// www.scidb.org}.
\bibitem{10-stu}
\Au{Kalinichenko L.\,A., Briukhov D.\,O., Martynov~D.\,O., Skvortsov~N.\,A., Stupnikov~S.\,A.} 
Mediation framework for enterprise information system infrastructures~// Volume Databases and 
Information Systems Integration: 9th Conference (International) on Enterprise Information 
Systems (ICEIS 2007) Proceedings ~--- Funchal, 2007. P.~246--251.
\bibitem{11-stu}
\Au{Захаров В.\,Н., Калиниченко Л.\,А., Соколов~И.\,А., Ступников~С.\,А.} Конструирование 
канонических информационных моделей для интегрированных информационных 
сис\-тем~// Информатика и её применения, 2007. Т.~1. Вып.~2. C.~15--38. 
\bibitem{12-stu}
\Au{Kalinichenko L.\,A., Stupnikov S.\,A.} Heterogeneous information model unification as a 
prerequisite to resource schema mapping~// Information Systems: People, Organizations, 
Institutions, and Technologies: 5th Conference of the Italian Chapter of Association for 
Information Systems itAIS Proceedings.~--- Berlin--Heidelberg: Springer Physica Verlag, 2010. 
P.~373--380. 
\bibitem{13-stu}
\Au{Kalinichenko L.\,A., Stupnikov S.\,A., Martynov~D.\,O.} SYNTHESIS: A~language for 
canonical information modeling and mediator definition for problem solving in heterogeneous 
information resource environments.~--- Moscow: IPI RAN, 2007. 171~p. 
\bibitem{14-stu}
\Au{Брюхов Д.\,О., Вовченко А.\,Е., Захаров~В.\,Н., Желенкова~О.\,П., Калиниченко~Л.\,А., 
Мартынов~Д.\,О., Скворцов~Н.\,А., Ступников~С.\,А.} Архитектура промежуточного слоя 
предметных посредников для решения \mbox{задач} над множеством интегрируемых 
неоднородных распределенных информационных ресурсов в гиб\-рид\-ной 
грид-ин\-фра\-струк\-ту\-ре виртуальных обсерваторий~// Информатика и её применения, 
2008. Т.~2. Вып.~1. С.~2--34. 

\bibitem{15-stu} %16
\Au{Kersten M.\,L., Zhang~Y., Ivanova~M., Nes~N.} SciQL, a query language for science 
applications~// EDBT/ICDT~--- Workshop on Array Databases 2011 Proceedings.~--- Uppsala, 
Sweden, 2011. P.~1--12.

\bibitem{16-stu} %17
\Au{Abrial J.-R.} The B-Book: Assigning programs to meanings.~--- Cambridge: Cambridge 
University Press, 1996.
\bibitem{17-stu} %18
Astronomy in ArrayDB. 
{\sf http://trac.scidb.org/\linebreak raw-attachment/wiki/UseCases/Astronomy\%20in\%20\linebreak
ArrayDB.pdf }
\bibitem{18-stu} %19
ATL Project. {\sf http://www.eclipse.org/m2m/atl}.
\bibitem{19-stu} %20
\Au{Budinsky F., Steinberg D., Ellersick~R., Grose~T.}
Eclipse modeling framework. Ch.~5: Ecore modeling concepts.~--- Addison Wesley 
Professional, 2004.
\bibitem{20-stu} %21
Meta Object Facility (MOF) 2.0 Core Specification, 2003. 
{\sf http://www.omg.org/cgi-bin/apps/doc?ptc/\linebreak 03-10-04.pdf}. 
\bibitem{21-stu} %22
\Au{Kalinichenko L.\,A.} Method for data models integration in the common paradigm~//  1st 
East-European Symposium on Advances in Databases and Information Systems \mbox{ADBIS'97} 
Proceedings.~--- St.-Petersburg: Nevsky Dialect, 1997. Vol.~1: Regular papers. P.~275--284.
\bibitem{22-stu}
Atelier~B: The industrial tool to efficiently deploy the B Method. 
{\sf http://www.atelierb.eu/index-en.php}.

\label{end\stat}

\bibitem{23-stu} %24
\Au{Van Ballegooij A.} RAM: Array database management through relational mapping~// SIKS 
Dissertation ser. No.\,2009-25. {\sf http://oai.cwi.nl/oai/asset/14074/ 14074D.pdf}.
         
\end{thebibliography}
} }

\end{multicols} %10
\def\stat{zatsman}

\def\tit{ТРАНСФОРМАЦИИ ОБЪЕКТОВ ПЕРВОГО И~ВТОРОГО ПОРЯДКА 
В~ЛЕКСИКОГРАФИЧЕСКОЙ ИНФОРМАЦИОННОЙ СИСТЕМЕ$^*$}

\def\titkol{Трансформации объектов первого и~второго порядка 
в~лексикографической информационной системе}

\def\aut{И.\,М.~Зацман$^1$}

\def\autkol{И.\,М.~Зацман}

\titel{\tit}{\aut}{\autkol}{\titkol}

\index{Зацман И.\,М.}
\index{Zatsman I.\,M.}


{\renewcommand{\thefootnote}{\fnsymbol{footnote}} \footnotetext[1]
{Исследование выполнено в~ФИЦ ИУ РАН за счет гранта Российского научного фонда №\,24-18-00155, {\sf 
https://rscf.ru/project/24-18-00155}. Работа выполнялась с~использованием инфраструктуры Центра 
коллективного пользования <<Высокопроизводительные вычисления и~большие данные>> (ЦКП 
<<Информатика>>) ФИЦ ИУ РАН (г.\ Москва).}}


\renewcommand{\thefootnote}{\arabic{footnote}}
\footnotetext[1]{ Федеральный исследовательский центр <<Информатика и~управление>> Российской академии наук, 
\mbox{izatsman@yandex.ru}}

\vspace*{-12pt}


  
  \Abst{Рассматриваются теоретические основания проектирования информационных 
технологий (ИТ) интеграции двуязычных словарей и~параллельных корпусов. Дано описание 
первых результатов создания третьего уровня классификации трансформаций объектов 
предметной области информатики, которую предполагается использовать при создании 
концепции лексикографической информационной системы, обеспечивающей интеграцию. 
Все сущности информатики в~статье разделены на два глобальных класса: объекты и~их 
трансформации. Для каждого такого класса конструируется своя классификация. Ранее были 
описаны два верхних уровня классификации трансформаций объектов предметной области. 
В~данной статье рассматривается третий уровень этой классификации. Основанием для 
построения самого верхнего ее уровня служило деление предметной области информатики 
на среды (ментальная, сенсорно воспринимаемая, цифровая и~ряд других сред), каждая из 
которых по определению включает объекты одной природы. Основанием для построения 
второго уровня классификации трансформаций объектов служила типология знаковых  
сис\-тем А.~Соломоника. Цель статьи состоит в~систематизации трансформаций первого 
и~второго порядка объектов предметной области на третьем уровне этой классификации. 
Основанием для систематизации служит средовая версия иерархии Акоффа.}
  
  \KW{объекты предметной области; трансформации объектов; классификация; данные; 
информация; знание; лексикографическая информационная сис\-тема}

\DOI{10.14357/19922264240211}{VZTGVV}
  
\vspace*{3pt}


\vskip 10pt plus 9pt minus 6pt

\thispagestyle{headings}

\begin{multicols}{2}

\label{st\stat}
  
\section{Введение}

\vspace*{-9pt}

  Возникновение параллельных корпусов, в~которых предложениям 
оригинального текста со\-по\-став\-ле\-ны предложения его перевода, обеспечило 
возможность контрастивного лингвистического\linebreak \mbox{анализа} на принципиально 
новом уровне полноты и~точности, недостижимом в~докорпусную эпоху. 
Пионерскими в~этой области стали работы \mbox{1990-х~гг}. Стига Йоханссона  
с~анг\-ло-нор\-веж\-ским корпусом~[1]. В России параллельные корпусы стали 
формироваться в~начале XXI~века в~рамках Национального корпуса русского 
языка~[2].
  
  Создатели двуязычных словарей используют параллельные корпусы для 
сбора материала и~эмпирической проверки своих гипотез, касающихся 
межъязы\-ко\-вой эквивалентности. Ценность параллельных корпусов 
определяется тем, что в~лингвистике этап сбора исходного материала считается 
наиболее трудоемким и~наименее творческим, а~параллельные корпусы 
позволяют значительно сэкономить время и~силы для творческого этапа 
создания словарей~[3].
 % 
  При этом двуязычные словари, создаваемые на основе исходного материала, 
извлеченного из параллельных корпусов, сейчас формируются без связей с~их 
текстами. Другими словами, онлайновые связи созданных словарей 
с~параллельными корпусами, которые служили источниками исходного 
материала, отсутствуют. 

Параллельные корпусы постоянно пополняются 
новыми текстами, в~предложениях которых можно обнаружить новые значения 
слов и~устойчивых словосочетаний. Однако при этом отсутствуют методы 
и~средства оперативного обновления словарей по корпусным данным. 
В~настоящее время проблема установления связей между двуязычными 
словарями и~параллельными корпусами (далее~--- проблема интеграции) 
находится на стадии поиска концептуальных подходов к~их интеграции на 
уровне значений.
  
  Подход к~решению проблемы интеграции, предлагаемый в~статье, учитывает 
  и~появление новых значений слов и~устойчивых словосочетаний, и~динамику 
смысловых значений, которая обусловлена развитием и~пополнением знания 
лингвистов, фиксирующих эти значения в~результате семантического анализа 
пополняемых корпусных данных. Проведенные эксперименты показали, что 
обнаружение нового лингвистического знания обусловливает и~формирование 
дефиниций новых значений, и~пересмотр уже существующих дефиниций~[4, 5].
  
  Например, в~проведенных экспериментах с~использованием ЦКП 
<<Информатика>> ФИЦ ИУ РАН фиксировалась эволюция значений немецких 
модальных глаголов, исходное состояние значений которых было описано 
в~не\-мец\-ко-рус\-ском словаре. В~экспериментальном массиве текстов как 
потенциальных источниках нового знания 16\,268 предложений содержали 
немецкие модальные глаголы и~в~2041 из них встречался глагол sollen. 
В~начале эксперимента в~словаре были описаны~12~значений этого модального 
глагола. По окончании эксперимента лингвисты обнаружили два новых его 
значения, согласовали их дефиниции и~описали эволюцию дефиниций~[6, 7].
  
  Таким образом, для решения проблемы интеграции требуется фиксировать 
новое знание, обнаруженное лингвистами в~текстовых данных параллельных 
корпусов, отслеживать эволюцию знания, представленного в~виде дефиниций 
значений слов и~устойчивых словосочетаний, и,~соответственно, 
актуализировать электронные двуязычные словари. Предлагаемый 
концептуальный подход к~интеграции, который планируется реализовать 
в~процессе проектирования лексикографической информационной сис\-те\-мы, 
фиксирующей эволюцию лингвистического знания, основан на решении 
следующих задач:\\[-14pt]
  \begin{itemize}
  \item категоризация трех базовых понятий информатики, включенных 
  в~иерархию Акоффа~[8] (данные, информация, знание), на объекты 
проектируемой сис\-те\-мы, которая необходима, чтобы фиксировать 
<<кванты>> нового знания и~отслеживать его эволюцию в~этой сис\-теме;\\[-15pt]
  \item  систематизация трансформаций объектов этой сис\-темы.\\[-14pt]
  \end{itemize}
  
  Цель статьи и~состоит в~решении двух задач: категоризации трех базовых 
понятий информатики на объекты лексикографической информационной  
сис\-те\-мы и~сис\-те\-ма\-ти\-за\-ции трансформаций первого и~второго порядка 
ее объектов.
  
  Трансформациями первого порядка, о которых сказано в~формулировке цели 
статьи, называются взаимные преобразования между двумя объектами  
сис\-те\-мы одной природы. Например, перевод в~сис\-те\-ме текста с~русского 
языка на английский относится к~ним. Трансформациями второго порядка 
и~выше называются взаимные преобразования между двумя и~более объектами 
разной природы. Например, кодирование символов текс\-та компьютерными 
кодами и~их декодирование относятся по определению к~трансформациям 
второго порядка.

%\vspace*{-9pt}
  
\section{Процессы трансформаций в~информатике}

%\vspace*{-3pt}

Процессы трансформаций, рассматриваемые в~статье, относятся к~теоретическому ядру информатики, а~не 
только к~проектированию лексикографической информационной сис\-те\-мы. Например, из трех основных 
подходов к~описанию предметной об\-ласти информатики\footnote{В статье предметная область информатики 
трактуется согласно концепции полиадического компьютинга Пола Розенблума~\cite{9-zac}.} (объектный, 
трансформационный и~синтетический) сис\-те\-ма\-ти\-за\-ция трансформаций ближе всего ко второму 
подходу. Примерами первого подхода, в~рамках которого основное внимание уделяется объектам предметной 
области информатики и~в~меньшей степени отношениям\linebreak между ними, могут служить  
работы~\cite{8-zac, 10-zac, 11-zac}; \mbox{примерами} второго подхода, в~рамках которого основное внимание 
уделяется трансформациям и~в~меньшей степени трансформируемым объектам,~---  
работы~\cite{12-zac, 13-zac}; примерами третьего, синтетического подхода, в~котором уделяется внимание 
и~объектам предметной об\-ласти информатики, и~отношениям между ними, могут служить работы~\cite{14-zac, 
15-zac, 16-zac, 17-zac, 18-zac}.

  Таким образом, для описания трансформаций объектов лексикографической 
информационной\linebreak системы предпочтительнее всего трансформационный 
подход, который упоминается и~в определениях информатики. Например, 
в~2009~г.\ П.~Деннинг и~П.~Розенблум сформулировали суть \mbox{информатики} как 
компьютинга следующим образом: <<$\ldots$информатика~--- это не просто 
алгоритмы и~структуры данных; это преобразования [трансформации] 
представлений>>~\cite{12-zac}. Чуть позже, в~контексте краткого описания 
парадигмы информатики как компьютинга, П.~Деннинг и~П.~Фриман изменили 
эту формулировку на такую: <<Центральный объект внимания в~информатике 
можно определить как информационные процессы~--- \textit{естественные или 
искусственные процессы, преобразующие информацию} (курсив мой~--- 
И.\,З.)>>~\cite{13-zac}. Согласно парадигме, предлагаемой авторами этой 
статьи, на начальном этапе проектирования автоматизированных систем 
базовыми элементами моделей их функционирования служат 
\textit{информационные про\-цессы}.
  
  Однако если 15~лет назад в~формулировке из работы~\cite{13-zac} шла речь 
о~процессах, преобразующих информацию, то в~последние~10~лет в~спектр 
процессов трансформаций все чаще стали включать процессы, преобразующие 
не только информацию, но также и~другие объекты автоматизированных 
систем, в~первую очередь данные и~знания~[19--21]. Например, Виктория 
Стодден, позиционируя науку о~данных как одну из дисциплин информатики, 
говорит, что центральный объект исследований в~науке о~данных~--- это 
<<изучение обобщаемого извлечения знания из данных>>~\cite{21-zac}. 
Увеличение и~чис\-ла объектов, и~спект\-ра процессов их трансформаций 
в~автоматизированных сис\-те\-мах обуслов\-ли\-ва\-ет не\-об\-хо\-ди\-мость 
систематизации и~объектов, и~процессов их трансформаций на начальном этапе 
проектирования сис\-тем.
  
  Для создания концепции лексикографической информационной сис\-те\-мы 
и~проектирования ИТ, обеспечивающих интеграцию 
двуязычных словарей и~параллельных корпусов, сначала выполним 
категоризацию на объекты этой сис\-те\-мы трех базовых понятий информатики 
(данные, информация, знание) в~контексте построения классификаций 
сущностей ее предметной об\-ласти.
  
  Необходимость использования классификаций информатики в~процессе 
создания концепции проиллюстрируем, используя иерархию  
Акоффа~\cite{8-zac}. Он использовал принцип их вертикального размещения 
в~иерархии снизу вверх: данные, информация и~знание. Еще в~ней есть термин 
<<мудрость>>, который в~статье не рассматривается. Такое размещение Акофф 
прокомментировал так: <<Каждое из пе\-ре\-чис\-лен\-ных понятий [кроме данных] 
содержит в~себе нижестоящие$\ldots$>>~\cite{8-zac}.
  
  Этому принципу размещения и~комментарию Акоффа свойственны 
недостатки, проанализированные, в~частности, в~работе~\cite{10-zac}. Главный 
вывод, к~которому пришла Роули после изучения иерархии Акоффа, 
заключается в~следующем: <<$\ldots$информация определяется в~терминах 
данных, знание~--- в~терминах информации$\ldots$ но существует меньше 
консенсуса в~описании трансформаций, которые преобразуют сущности, 
расположенные ниже в~иерархии, в~те, которые находятся над ними, что 
приводит к~их терминологической неопределенности>>~\cite{10-zac}. Причина 
этой неопределенности, скорее всего, в~том, что базовые понятия информатики 
включены в~иерархию Акоффа изолированно от общего контекста 
классификаций сущностей ее предметной об\-ласти.

%\vspace*{-9pt}
  
\section{Классификации сущностей информатики}


%\vspace*{-2pt}

  Все сущности предметной области информатики в~работах~[22, 23] 
разделены на два глобальных класса: ее объекты и~их трансформации. Для 
каждого такого класса была предложена своя классификация. 
В~работе~\cite{22-zac} дано описание классификации объектов предметной 
области информатики, первый уровень которой содержит базовые понятия ее 
предметной области (данные, информация, знания и~др.).  
В~работе~\cite{23-zac} дано описание двух верхних уровней классификации 
трансформаций объектов предметной об\-ласти (см.\ рисунок 
в~работе~\cite{23-zac}). Основанием для построения самого верхнего ее уровня послужило деление 
предметной области информатики на среды\footnote{В~работе~\cite{24-zac} дано описание пяти сред 
предметной области информатики (ментальная; сенсорно воспринимаемая, или информационная; 
цифровая; нейро- и~ДНК-среда), каждая из которых по определению включает объекты одной и~той же 
природы.} и~степень разнообразия природы объектов, вовлеченных в~трансформации:
\begin{itemize}
\item  первый класс верхнего уровня классификации включает 
трансформации объектов в~пределах среды только одной природы 
(трансформации первого порядка);
\item  второй класс включает трансформации объектов, относящихся 
к~двум средам разной природы (трансформации второго порядка);
\item третий и~последующие классы включают трансформации объектов, 
относящихся к~трем и~более средам разной природы (трансформации 
третьего и~более высоких порядков).
\end{itemize}

  В работе~\cite{23-zac} были приведены примеры для трех первых классов 
трансформаций, включая пример трансформаций объектов, относящихся 
к~двум средам разной природы (компьютерное кодирование символов текстов 
с~по\-мощью таб\-лиц Unicode).
  
Основанием для построения второго уровня классификации трансформаций объектов послужила типология 
знаковых сис\-тем А.~Соломоника~\cite[c.~131]{25-zac}: естественные знаковые сис\-те\-мы, образные,  
ес\-тест\-вен\-но-язы\-ко\-в$\acute{\mbox{ы}}$е,  
вер\-баль\-но-не\-сло\-вес\-ные сис\-те\-мы записи\footnote{Под системой записи понимается знаковая 
система, сочетающая вербальные знаки с~несловесными (языки нотной записи, карт, таблиц и~др.).} 
и~формализованные знаковые сис\-те\-мы, включая математические. Введем понятие обобщенного текста~--- 
это текст, который может быть создан в~любой из перечисленных знаковых систем. Тогда обобщенные тексты 
могут быть естественными, образными, ес\-тест\-вен\-но-язы\-ко\-в$\acute{\mbox{ы}}$\-ми,  
вер\-баль\-но-не\-сло\-вес\-ны\-ми и~формализованными. Второй уровень классификации трансформаций 
охватывает не все виды объектов предметной  
об\-ласти информатики, а~только перечисленные~5~видов текс\-тов и~их представления, вовлеченные 
в~процессы трансформаций в~одной или более средах вместе с~данными, знанием и~его концептами.

\begin{figure*}[b] %fig1
\vspace*{6pt}
      \begin{center}
     \mbox{%
\epsfxsize=121.191mm 
\epsfbox{zac-1.eps}
}
\end{center}
\vspace*{-6pt}
\Caption{Средовая версия иерархии Акоффа}
\end{figure*}

\section{Классификация трансформаций: построение~третьего 
уровня}

  Основанием для систематизации трансформаций первого и~второго порядка 
на третьем уровне этой классификации служит иерархия Акоффа~\cite{8-zac}, 
на основе которой и~была создана ее средов$\acute{\mbox{а}}$я версия~[26, 
27]. Для создания средов$\acute{\mbox{о}}$й версии была выполнена 
категоризация трех базовых понятий информатики (данные, информация, 
знания) на объекты лексикографической информационной сис\-те\-мы 
в~процессе создания ее концепции\linebreak (рис.~1).
  


  В отличие от классической иерархии Акоффа, в~ее 
средов$\acute{\mbox{о}}$й версии различаются три вида данных: сенсорно 
воспринимаемые, цифровые и~те данные, которые генерируются 
искусственными нейронными сетями (ИНС) в~системах искусственного интеллекта 
(далее~--- ИИ-дан\-ные). Последний вид данных необходим, например, для 
различения входа и~выхода процесса применения обученной 
ИНС в~цифровой модели генерации знания, описанию которой 
посвящена работа~\cite{27-zac}.
  
  Также предлагается различать два вида информации: сенсорно 
воспринимаемая и~цифровая. Кроме знания в~средов$\acute{\mbox{у}}$ю 
версию добавлены концепты и~ментальные образы сенсорно воспринимаемых 
данных. Последние служат промежуточной сущностью между сенсорно 
воспринимаемыми данными и~генерируемым знанием при описании процессов 
извлечения знания из текстовых данных лексикографической информационной 
системы. Описание объектов средов$\acute{\mbox{о}}$й версии иерархии 
Акоффа (см.\ рис.~1) и~отношений между ними дано в~работах~\cite{26-zac, 28-zac}.
  
  В средов$\acute{\mbox{о}}$й версии число объектов равно восьми. Если 
учитывать направления трансформаций, то между восемью объектами на 
рис.~1 она включает~16 их видов (трансформации на границе между сенсорно 
воспринимаемыми данными и~информацией, обозначенные символом~<<?>>, 
в~статье не рас\-смат\-ри\-ва\-ют\-ся). В~будущем число объектов 
в~средов$\acute{\mbox{о}}$й версии, которая выбрана как основание для 
сис\-те\-ма\-ти\-за\-ции трансформаций первого и~второго порядка, может быть 
увеличено. Для построения классификации трансформаций 
важ\-но не возможное увеличение числа объектов 
и~трансформаций между ними, а то, что их виды в~средов$\acute{\mbox{о}}$й 
версии распределены между трансформациями первого и~второго порядка. Из 
16~видов на рис.~1 шесть относятся к~трансформациям первого порядка, это\linebreak 
виды с~номерами~7, 8, 13--16 (далее~--- типология трансформаций первого 
порядка), а~десять~--- к~трансформациям второго порядка, это виды 
с~\mbox{номерами}~1--6 и~9--12 (далее~--- типология трансформаций второго 
порядка). Разместим обе типологии на третьем уровне классификации (см.\ ее 
схему на рис.~2). Перечислим виды трансформаций первой типологии, вводя 
в~скобках их краткие названия, используемые ниже на рис.~3:
  \begin{description}
  \item[\,] 7~--- членение знания на концепты с~помощью одной или нескольких 
знаковых систем (далее~--- членение знания);
  \item[\,] 8~--- формирование знания на основе концептов (формирование 
знания);
  \item[\,] 13~--- обучение ИНС;
  \end{description}
  
  \vspace*{-6pt}
  
  \pagebreak
  
  \end{multicols}
  
  \begin{figure*} %fig2
\vspace*{1pt}
      \begin{center}
     \mbox{%
\epsfxsize=127.513mm 
\epsfbox{zac-2.eps}
}
\end{center}
\vspace*{-9pt}
\Caption{Схема трех верхних уровней классификации трансформаций объектов (объединены 
по три слоя и~для второго, и~для третьего уровней этой классификации)}
\end{figure*}
  
  \begin{multicols}{2}
  
  \noindent
  \begin{description}
  \item[\,] 14~--- восстановление обучающей информации на основе 
содержания обученной ИНС (обращение ИНС);
  \item[\,] 15~--- использование обученной ИНС (использование ИНС);



  \item[\,] 16~--- восстановление исходных данных, соответствующих 
полученным результатам работы обучен\-ной ИНС (восстановление исходных данных 
по результатам ИНС).
  \end{description}
  
  
  Не все виды трансформаций 13--16 поддерживаются в~конкретных системах 
искусственного интеллекта, но с~теоретической точки зрения все их 
предлагается включить в~первую типологию для полноты спектра видов 
трансформаций.
  
  Перечислим виды трансформаций второй типологии:
  \begin{description}
  \item[\,] 1~--- декодирование цифровых данных в~компьютерных системах 
(декодирование данных);
  \item[\,]  2~--- кодирование сенсорно воспринимаемых данных (кодирование 
данных);
  \item[\,] 3~--- ментальное копирование сенсорно воспринимаемых данных 
(ментальное копирование);
  \item[\,] 4~--- восстановление сенсорно воспринимаемых данных по 
ментальным образам (восстановление по образам);
  \item[\,] 5~--- смысловая интерпретация без деления на концепты ментальных 
образов сенсорно воспринимаемых данных (смысловая интерпретация);
  \item[\,] 6~--- восстановление ментальных образов (восстановление образов);
  \item[\,] 9~--- представление концептов в~виде сенсорно воспринимаемой 
информации, например текс\-та\-ми, формулами, таблицами, рисунками и~т.\,д.\ 
(представление концептов);
  \item[\,] 10~--- понимание смысла сенсорно воспринимаемой информации 
(понимание смысла);
  \item[\,] 11~--- кодирование сенсорно воспринимаемой информации 
(кодирование информации);
\end{description}

\vspace*{-6pt}

\pagebreak

\end{multicols}

\begin{figure*} %fig3
\vspace*{1pt}
      \begin{center}
     \mbox{%
\epsfxsize=163mm 
\epsfbox{zac-3.eps}
}
\end{center}
\vspace*{-9pt}
\Caption{Схема частного случая классификации трансформаций объектов (трансформации 
пронумерованы согласно рис.~1)}
\end{figure*}

\begin{multicols}{2}

\noindent
\begin{description}

  \item[\,] 12~--- декодирование цифровой информации (декодирование 
информации).
  \end{description}
  
  Отметим, что в~существующих ИТ
  и~компьютерных системах наиболее часто используются виды 
трансформаций~13 и~15 типологии первого порядка и~1, 2, 11 и~12 типологии 
второго порядка. На рис.~2 в~первом слое третьего уровня классификации 
показаны типологии первого порядка без указания числа трансформаций в~них 
и~без детализации трансформируемых объектов.
  
  Во втором слое третьего уровня классификации условно (без названий) 
показаны типологии второго порядка. Также на рис.~2 в~третьем слое третьего 
уровня классификации условно (также без названий) показаны типологии 
третьего порядка, которые планируется рассмотреть в~отдельной статье. По 
определению они должны включать трансформации между тремя объектами 
разной природы, но средов$\acute{\mbox{а}}$я версия иерархии Акоффа 
включает трансформации только между двумя объектами разной природы. 
Поэтому потребуется другое основание для их систематизации (ранее были 
рассмотрены отдельные примеры трансформаций третьего 
порядка\footnote{Далеко не всегда трансформации третьего и~более высоких порядков можно 
рассматривать как последовательность трансформаций второго порядка. Примером этого могут 
служить трансформации в~процессе обучения пациента пользованию роботизированной рукой, 
охватывающие личностные концепты пациента, релевантные его намерениям, сигналы активности 
мозга как объекты нейросреды и~компьютерные коды~\cite{29-zac}.}~\cite{29-zac}).

\section{Классификация трансформаций: частный~случай}

  Выше было отмечено, что в~будущем число объектов 
в~средов$\acute{\mbox{о}}$й версии иерархии Акоффа может быть увеличено. 
Это означает, что увеличатся и~чис\-ло объектов, и~чис\-ло трансформаций между 
ними в~классификации трансформаций, так как эта средов$\acute{\mbox{а}}$я 
версия служит по определению основанием для систематизации 
трансформаций первого и~второго порядка. Поэтому на третьем уровне рис.~2 
указаны типологии без детализации объектов и~без указания числа 
трансформаций в~каждой из них. С~одной стороны, при таком подходе 
получаем достаточно общий вид этой классификации, так как она не зависит от 
числа объектов в~том или ином варианте средов$\acute{\mbox{о}}$й версии 
(и~это существенно упрощает рис.~2). С~другой стороны, на третьем уровне 
такой общей классификации подразумевается, но не эксплицируется природа 
трансформируемых объектов и~их возможные сочетания в~трансформациях. 

При проектировании лексикографической информационной системы важно 
эксплицировать природу трансформируемых объектов и~их возможные 
сочетания.
  %
  Поэтому в~парадигму информатики~\cite{30-zac} кроме общей 
классификации трансформаций предлагается включать и~ее частные случаи, 
эксплицирующие природу трансформируемых объектов. 

В~этом разделе 
рассмотрим один частный случай, когда используются только естественные 
знаковые сис\-те\-мы из типологии А.~Соломоника~\cite{25-zac} вместе 
с~данными, знанием и~его концептами. Чис\-ло естественных языков при этом не 
ограничено. И~этот частный случай классификации включает только три 
класса природных трансформаций (первого, второго и~третьего порядка, см.\ 
схему классификации на рис.~3).
  
  Первый и~второй уровни схемы общей классификации (см.\ рис.~2) можно 
объединить в~один уровень в~этом частном случае. Ниже этого уровня 
приведено содержание типологий первого и~второго порядка без содержания 
типологий третьего по\-рядка.




  Наполнение типологий первого и~второго порядка соответствует 
средов$\acute{\mbox{о}}$й версии иерархии Акоффа на рис.~1, содержащей 
6~видов трансформаций типологии первого порядка и~10~видов 
трансформаций типологии второго порядка (на рис.~3 стрелки указывают 
направления трансформаций согласно средов$\acute{\mbox{о}}$й версии на рис.~1).
  
  Таким образом, частный случай классификации содержит для этих двух 
типологий 16~теоретически возможных трансформаций, 6 из которых 
в~настоящее время в~существующих ИТ применяются наиболее часто: виды 
трансформаций~1, 2, 11 и~12 типологии второго порядка реализуются 
с~помощью тех или иных методов ко\-ди\-ро\-ва\-ния/де\-ко\-ди\-ро\-ва\-ния 
(например, с~использованием таблиц Unicode), а~виды трансформаций~13 и~15
 в~типологии первого порядка реализуются полностью с~по\-мощью процессов 
цифровой обработки компьютерами.
  
  Остальные виды трансформаций или применяются намного реже (это 
виды~3, 5, 7, 9 и~10), или находятся в~стадии поиска и~разработки (14 и~16) или 
в~настоящее время носят только теоретический характер, обеспечивая полноту 
первой и~второй типологий (4, 6 и~8). Знаком~<<?>> обозначены те виды 
трансформаций, которые по определению не существуют в~используемой 
парадигме информатики~\cite{30-zac}. Однако возможно, что в~других 
будущих подходах к~построению ее парадигмы эти виды трансформаций будут 
существовать.
  
\section{Заключение}

  На сегодняшний день процесс построения классификаций объектов 
предметной области информатики~\cite{22-zac} и~их  
трансформаций~\cite{23-zac} еще не завершен. Однако первые результаты их 
построения уже используются для создания концепции лексикографической 
информационной сис\-те\-мы, обеспечивающей интеграцию двуязычных 
словарей и~параллельных корпусов.
  
  \bigskip
  
  
  Автор признателен рецензентам за помощь в~улучшении статьи.
  
{\small\frenchspacing
 { %\baselineskip=10.6pt
 %\addcontentsline{toc}{section}{References}
 \begin{thebibliography}{99}
\bibitem{1-zac}
\Au{Aijmer K., Altenberg~B.} Advances in corpus-based contrastive linguistics. Studies in honour 
of Stig Johansson.~--- Amsterdam: John Benjamins, 2013. 295~p.  doi: 10.1075/scl.54.
\bibitem{2-zac}
\Au{Добровольский Д.\,О., Кретов~А.\, А., Шаров~С.\,А.} Корпус параллельных текстов~// 
Научная и~техническая информация. Сер.~2: Информационные процессы и~сис\-те\-мы, 2005. 
№\,6. С.~16--27.
\bibitem{3-zac}
\Au{Добровольский Д.\,О.} Корпус параллельных текстов и~сопоставительная 
лексикология~// Труды Института русского языка им.\ В.\,В.~Виноградова, 2015. №\,6. 
С.~413--449. EDN: VJQBHP.
\bibitem{4-zac}
\Au{Гончаров А.\,А., Зацман~И.\,М., Кружков~М.\,Г.} Эволюция классификаций 
в~надкорпусных базах данных~// Информатика и~её применения, 2020. Т.~14. Вып.~4. 
С.~108--116. doi: 10.14357/19922264200415.  
EDN: \mbox{GKWBZT}.
\bibitem{5-zac}
\Au{Гончаров А.\, А., Зацман И. \,М., Кружков~М.\, Г}. Представление новых 
лексикографических знаний в~динамических классификационных сис\-те\-мах~// 
Информатика и~её применения, 2021. Т.~15. Вып.~1. С.~86--93.  doi: 10.14357/19922264210112. EDN: OPEFXW.
\bibitem{6-zac}
\Au{Zatsman I.} Finding and filling lacunas in linguistic typologies~// 15th Forum (International) 
on Knowledge Asset Dynamics Proceedings.~--- Matera, Italy: Institute of Knowledge Asset 
Management, 2020. P.~780--793.
\bibitem{7-zac}
\Au{Zatsman I.} Three-dimensional encoding of emerging meanings in AI-systems~// 21st 
European Conference on Knowledge Management Proceedings.~--- Reading, U.K.: Academic 
Publishing International Ltd., 2020. P.~878--887.
\bibitem{8-zac}
\Au{Ackoff R.} From data to wisdom~// J.~Applied Systems Analysis, 1989. Vol.~16. No.\,1. P.~3--9.
\bibitem{9-zac}
\Au{Rosenbloom P.\,S.} On computing: The fourth great scientific domain.~--- Cambridge, MA, 
USA: MIT Press, 2013. 307~p.
\bibitem{10-zac}
\Au{Rowley J.} The wisdom hierarchy: Representations of the DIKW hierarchy~// J.~Inf. 
Sci., 2007. Vol.~33. Iss.~2. P.~163--180. doi: 10.1177/0165551506070706.
\bibitem{11-zac} 
\Au{Frick$\acute{\mbox{e}}$~M.\,H.} Data--Information--Knowledge--Wisdom (DIKW) pyramid, 
framework, continuum~// Encyclopedia of big data~/ Eds. L.~Schintler, C.~McNeely.~--- Cham: 
Springer, 2018. 4~p. doi: 10.1007/978-3-319-32001-4\_331-1.
\bibitem{12-zac}
\Au{Denning P., Rosenbloom~P.} Computing: The fourth great domain of science~// Commun. 
ACM, 2009. Vol.~52. Iss.~9. P.~27--29.
\bibitem{13-zac}
\Au{Denning P., Freeman~P.} Computing's paradigm~// Commun.  ACM, 2009. Vol.~52. 
Iss.~12. P.~28--30. doi: 10.1145/ 1610252.1610265.
\bibitem{17-zac} %14
\Au{Farradane J.} Knowledge, information, and information science~// J.~Inf. Sci., 
1980. Vol.~2. Iss.~2. P.~75--80. doi: 10.1177/01655515800020020.

\bibitem{15-zac}
\Au{Шрейдер Ю.\,А.} Информация и~знание~// Сис\-тем\-ная концепция информационных 
процессов.~--- М.: ВНИИСИ, 1988. С.~47--52.
\bibitem{16-zac}
\Au{Ingwersen P.} Information and information science~// Enclyclopaedie of library and 
information science~/ Eds. J.\,D.~McDonald, 
M.~Levine-Clark.~--- New York, NY, USA: Marcel Dekker Inc., 1992. Vol.~56. Sup.~19. 
P.~137--174.

\bibitem{14-zac} %17
Информатика как наука об информации: Информационный, документальный, 
технологический, экономический, социальный и~организационный аспекты~/ Под ред. 
Р.\,С.~Гиляревского.~--- М.: Фаир-Пресс, 2006. 592~с.

\bibitem{18-zac}
\Au{Hjorland B.} Library and information science: practice, theory, and philosophical basis~// 
Inform. Process. Manag., 2000. Vol.~36. Iss.~3. P.~501--531. doi:  
10.1016/S0306-\mbox{4573(99)00038-2}.
\bibitem{19-zac}
Deep shift~--- technology tipping points and societal impact.~--- Geneva: WE Forum, 2015. 44~p. 
{\sf http://www3.weforum.org/docs/WEF\_GAC15\_ Technological\_Tipping\_Points\_report\_2015.pdf}.
\bibitem{20-zac}
\Au{Berman F., Rutenbar~R., Hailpern~B., Christensen~H., Davidson~S., Estrin~D., 
Franklin~M., Martonosi~M., Raghavan~P., Stodden~V., Szalay~A.\,S.} Realizing the potential of 
data science~// Commun.  ACM, 2018. Vol.~61. Iss.~4. P.~67--72. doi: 10.1145/3188721.

\bibitem{21-zac}
\Au{Stodden V.} The data science life cycle: A~disciplined approach to advancing data science as 
a~science~// Commun.  ACM, 2020. Vol.~63. Iss.~7. P.~58--66. doi: 10.1145/ 3360646.


\bibitem{23-zac} %22
\Au{Зацман И.\,М.} Научная парадигма информатики: классификация трансформаций 
объектов предметной об\-ласти~// Системы и~средства информатики, 2023. Т.~33. №\,4. 
С.~126--138. doi: 10.14357/08696527230412. EDN: ZIKUWO.

\bibitem{22-zac} %23
\Au{Зацман И.\,М.} Научная парадигма информатики: классификация объектов предметной  
об\-ласти~// Информатика и~её применения, 2023. Т.~17. Вып.~4. С.~96--103. doi: 
10.14357/19922264230413. EDN: FIUQAT.

\bibitem{24-zac}
\Au{Зацман И.\,М.} О~научной парадигме информатики: верхний уровень классификации 
объектов ее предметной об\-ласти~// Информатика и~её применения, 2022. Т.~16. Вып.~4. 
С.~73--79. doi: 10.14357/ 19922264220411. EDN: XZNKVI.

\bibitem{25-zac}
\Au{Соломоник А.\,Б.} Философия знаковых систем и~язык.~--- М.: ЛКИ, 2011. 408~с.
\bibitem{26-zac}
\Au{Зацман И.\,М.} Трансформация иерархии Акоффа в~научной парадигме информатики~// 
Информатика и~её применения, 2023. Т.~17. Вып.~3. С.~107--113. doi: 
10.14357/19922264230315. EDN: UMVRRV.

\bibitem{27-zac}
\Au{Zatsman I.} Building digital spiral models of knowledge generation~// 19th Forum 
(International) on Knowledge Asset Dynamics Proceedings.~--- Matera, Italy: Arts for Business 
Institute, 2024. P.~2185--2196.
\bibitem{28-zac}
\Au{Zatsman I.} Digital spiral model of knowledge creation and encoding its dynamics~// 18th 
Forum (International) on Knowledge Asset Dynamics Proceedings.~--- Matera, Italy: Arts for 
Business Institute, 2023. P.~581--596.
\bibitem{29-zac}
\Au{Зацман И.\,М.} Интерфейсы третьего порядка в~информатике~// Информатика и~её 
применения, 2019. Т.~13. Вып.~3. С.~82--89. doi: 10.14357/19922264190312. EDN: 
EHRQLF.

\bibitem{30-zac}
\Au{Зацман И.\,М.} Научная парадигма информатики как третьей культуры~//  
На\-уч\-но-тех\-ни\-че\-ская информация. Сер.~1: Организация и~методика информационной 
работы, 2023. №\,11. С.~1--14.

\end{thebibliography}

 }
 }

\end{multicols}

\vspace*{-9pt}

\hfill{\small\textit{Поступила в~редакцию 14.04.24}}

\vspace*{4pt}

%\pagebreak

%\newpage

%\vspace*{-28pt}

\hrule

\vspace*{2pt}

\hrule



\def\tit{OBJECT TRANSFORMATIONS OF~THE~FIRST AND~SECOND ORDER
IN~A~LEXICOGRAPHIC INFORMATION SYSTEM\\[-5pt]}


\def\titkol{Object transformations of~the~first and~second order
in~a~lexicographic information system}


\def\aut{I.\,M.~Zatsman}

\def\autkol{I.\,M.~Zatsman}

\titel{\tit}{\aut}{\autkol}{\titkol}

\vspace*{-13pt}


\noindent
Federal Research Center ``Computer Science and Control'' of the Russian Academy of Sciences, 
44-2~Vavilov Str., Moscow 119133, Russian Federation


\def\leftfootline{\small{\textbf{\thepage}
\hfill INFORMATIKA I EE PRIMENENIYA~--- INFORMATICS AND
APPLICATIONS\ \ \ 2024\ \ \ volume~18\ \ \ issue\ 2}
}%
 \def\rightfootline{\small{INFORMATIKA I EE PRIMENENIYA~---
INFORMATICS AND APPLICATIONS\ \ \ 2024\ \ \ volume~18\ \ \ issue\ 2
\hfill \textbf{\thepage}}}

\vspace*{2pt}



\Abste{The theoretical foundations of the design of information technologies used for 
the integration of bilingual dictionaries and parallel corpora are considered. The 
description of the first outcomes of the creation of the third\linebreak\vspace*{-12pt}}

\Abstend{ level of object 
transformations classification in the subject domain of informatics, which is supposed 
to be used
in creating the lexicographic information system providing integration, is 
given. All the entities of informatics are divided into two global classes: objects and 
their transformations. For each such class, its own classification is constructed. 
Previously, the two upper levels of the object transformation classification in the subject 
domain have been described. The present paper discusses the third level of this classification. The 
basis for the construction of its highest level was the division of the subject domain of 
informatics into media (mental, sensory, digital, and a~number of other media), each 
of which by definition includes objects of the same nature. The Solomonick's 
typology of sign systems served as the basis for constructing the second level of the 
object transformation classification. The aim of the paper is to systematize object 
transformations of the first and second orders at the third level of this classification. 
The basis for systematization is the medium version of the Ackoff's hierarchy.}

\KWE{subject domain objects; object transformations; classification; data; 
information; knowledge; lexicographic information system}


\DOI{10.14357/19922264240211}{VZTGVV}

\vspace*{-12pt}

\Ack

\vspace*{-3pt}


\noindent
The reported study was funded by the Russian Science Foundation, project  
No.\,24-18-00155, {\sf 
https://rscf.ru/project/24-18-00155}. The research was carried out using the infrastructure of the Shared 
Research Facilities ``High Performance Computing and Big Data'' (CKP 
``Informatics'') of FRC CSC RAS (Moscow) .
   


  \begin{multicols}{2}

\renewcommand{\bibname}{\protect\rmfamily References}
%\renewcommand{\bibname}{\large\protect\rm References}

{\small\frenchspacing
 {%\baselineskip=10.8pt
 \addcontentsline{toc}{section}{References}
 \begin{thebibliography}{99} 
\bibitem{1-zac-1}
\Aue{Aijmer, K., and B.~Altenberg.} 2013. \textit{Advances in corpus-based 
contrastive linguistics. Studies in honour of Stig Johansson}. Amsterdam: John 
Benjamins. 295~p. doi: 10.1075/scl.54.
\bibitem{2-zac-1}
\Aue{Dobrovolskiy, D.\,O., A.\,A.~Kretov, and S.\,A.~Sharov.} 2005. Korpus 
parallel'nykh tekstov [Corpus of parallel texts]. \textit{Nauchnaya i~tekhnicheskaya 
informatsiya. Ser. 2. Informatsionnye protsessy i~sistemy} [Scientific and Technical 
Information. Ser.~2: Information Processes and Systems] 6:16--27.
\bibitem{3-zac-1}
\Aue{Dobrovolskiy, D.\,O.} 2015. Korpus parallel'nykh tekstov i~sopostavitel'naya 
leksikologiya [The corpus of parallel texts and contrastive lexicology]. \textit{Trudy 
Instituta russkogo yazyka im. V.\,V.~Vinogradova} [Proceedings of the 
V.\,V.~Vinogradov Russian Language Institute] 6:413--449. EDN: VJQBHP.
\bibitem{4-zac-1}
\Aue{Goncharov, A.\,A., I.\,M.~Zatsman, and M.\,G.~Kruzhkov.} 2020. Evolyutsiya 
klassifikatsiy v~nadkorpusnykh ba\-zakh dannykh [Evolution of classifications in 
supracorpora databases]. \textit{Informatika i~ee Primeneniya~--- Inform. \mbox{Appl.}}  
14(4):108--116. doi: 10.14357/19922264200415.  
EDN: GKWBZT.
\bibitem{5-zac-1}
\Aue{Goncharov, A.\,A., I.\,M.~Zatsman, and M.\,G.~Kruzhkov.} 2021. 
Predstavlenie novykh leksikograficheskikh znaniy v~dinamicheskikh 
klassifikatsionnykh sistemakh [Representation of new lexicographical knowledge in 
dynamic classification systems]. \textit{Informatika i~ee Primeneniya~--- Inform. 
Appl.} 15(1):86--93. doi: 10.14357/19922264210112. EDN: OPEFXW.
\bibitem{6-zac-1}
\Aue{Zatsman, I.} 2020. Finding and filling lacunas in linguistic typologies. 
\textit{15th Forum (International) on Knowledge Asset Dynamics Proceedings}. 
Matera, Italy: Institute of Knowledge Asset Management. 780--793.
\bibitem{7-zac-1}
\Aue{Zatsman, I.} 2020. Three-dimensional encoding of emerging meanings in  
AI-systems. \textit{21st European Conference on Knowledge Management 
Proceedings}. Reading, U.K.: Academic Publishing International Ltd. 878--887.
\bibitem{8-zac-1}
\Aue{Ackoff, R.} 1989. From data to wisdom. \textit{J.~Applied Systems Analysis} 
16(1):3--9.
\bibitem{9-zac-1}
\Aue{Rosenbloom, P.\,S.} 2013. \textit{On computing: The fourth great scientific 
domain}. Cambridge, MA: MIT Press. 307~p.
\bibitem{10-zac-1}
\Aue{Rowley, J.} 2007. The wisdom hierarchy: Representations of the DIKW 
hierarchy. \textit{J.~Inf. Sci.} 33(2):163--180. doi: 10.1177/0165551506070706.
\bibitem{11-zac-1}
\Aue{Frick$\acute{\mbox{e}}$, M.\,H.} 2018.  
Data-Information-Knowledge-Wisdom (DIKW) pyramid, framework, continuum. 
\textit{Encyclopedia of big data}. Eds. L.~Schintler and C.~McNeely. Cham: 
Springer. 4~p. doi: 10.1007/978-3-319-32001- 4\_331-1.
\bibitem{12-zac-1}
\Aue{Denning, P., and P.~Rosenbloom.} 2009. Computing: The fourth great domain 
of science. \textit{Commun. ACM} 52(9):27--29.
\bibitem{13-zac-1}
\Aue{Denning, P., and P.~Freeman.} 2009. Computing's paradigm. \textit{Commun. 
ACM} 52(12):28--30. doi: 10.1145/ 1610252.1610265.

\bibitem{17-zac-1} %14
\Aue{Farradane, J.} 1980. Knowledge, information, and information science. 
\textit{J.~Inf. Sci.} 2(2):75--80. doi: 10.1177/ 01655515800020020.

\bibitem{15-zac-1}
\Aue{Shreyder, Yu.\,A.} 1988. Informatsiya i~znanie [Information and knowledge]. 
\textit{Sistemnaya kontseptsiya in\-for\-ma\-tsi\-on\-nykh protsessov} [System concept of 
information processes]. Moscow: VNIISI. 47--52.
\bibitem{16-zac-1}
\Aue{Ingwersen, P.} 1995. Information and information science. 
\textit{Encyclopedia of library and information science}. Eds. J.\,D.~McDonald and 
M.~Levine-Clark. New York, NY: Marcel Dekker Inc. 56(19):137--174.

\bibitem{14-zac-1} %17
Gilyarevskiy, R.\,S., ed. 2006. \textit{Informatika kak nauka ob informatsii: 
informatsionnyy, dokumental'nyy, tekh\-no\-lo\-gi\-che\-skiy, ekonomicheskiy, sotsial'nyy 
i~organizatsionnyy aspekty} [Informatics as information science: Informational, 
documentary, technological, economic, social, and organizational dimensions]. 
Moscow: FAIR-PRESS. 592~p.

\bibitem{18-zac-1}
\Aue{Hjorland, B.} 2000. Library and information science: Practice, theory, and 
philosophical basis. \textit{Inform. Process. Manag.} 36(3):501--531. doi:  
10.1016/S0306-\mbox{4573(99)00038-2}.
\bibitem{19-zac-1}
Deep shift~--- technology tipping points and societal impact. 2015. \textit{World Economic 
Forum}. Geneva. 44~p. Available at: {\sf 
http://www3.weforum.org/docs/WEF\_ GAC15\_Technological\_Tipping\_Points\_report\_2015.pdf} (accessed May~20, 
2024).
\bibitem{20-zac-1}
\Aue{Berman, F., R.~Rutenbar, B.~Hailpern, H.~Christensen, S.~Davidson, 
D.~Estrin, M.~Franklin, M.~Martonosi, P.~Raghavan, V.~Stodden, and 
A.\,S.~Szalay.} 2018. Realizing the potential of data science. \textit{Commun. ACM} 
61(4):67--72. doi: 10.1145/3188721.
\bibitem{21-zac-1}
\Aue{Stodden, V.} 2020. The data science life cycle: A~disciplined approach to 
advancing data science as a~science. \textit{Commun. ACM} 
 63(7):58--66. doi: 10.1145/3360646.

\bibitem{23-zac-1} %22
\Aue{Zatsman, I.\,M.} 2023. Nauchnaya paradigma informatiki: klassifikatsiya 
transformatsiy ob''ektov predmetnoy oblasti [Scientific paradigm of informatics: 
Transformation classification of domain objects]. \textit{Sistemy i~Sredstva 
Informatiki~--- Systems and Means of Informatics} 33(4):126--138. doi: 
10.14357/08696527230412. EDN: ZIKUWO.

\bibitem{22-zac-1} %23
\Aue{Zatsman, I.\,M.} 2023. Nauchnaya paradigma informatiki: klassifikatsiya 
ob''ektov predmetnoy oblasti [Scientific paradigm of informatics: Classification of 
domain objects]. \textit{Informatika i~ee Primeneniya~--- Inform. Appl.} 
 17(4):96--103. doi: 10.14357/19922264230413. EDN: FIUQAT.
 
\bibitem{24-zac-1}
\Aue{   Zatsman, I.\,M.} 2022. O nauchnoy paradigme informatiki: verkhniy uroven' 
klassifikatsii ob''ektov ee predmetnoy oblasti [On the scientific paradigm of 
informatics: The classification high level of its objects]. \textit{Informatika i~ee 
Primeneniya~--- Inform. Appl.} 16(4):73--79. doi: 10.14357/19922264220411. EDN: 
XZNKVI.
\bibitem{25-zac-1}
\Aue{Solomonick, A.\,B.} 2011. \textit{Filosofiya znakovykh system i~yazyk} 
[Philosophy of sign systems and language]. Moscow: LKI. 408~p.
\bibitem{26-zac-1}
\Aue{Zatsman, I.\,M.} 2023. Transformatsiya ierarkhii Akoffa v~nauchnoy 
paradigme informatiki [Transformation of the Ackoff's hierarchy in the scientific 
paradigm of informatics]. \textit{Informatika i~ee Primeneniya~--- Inform. \mbox{Appl.}} 
17(3):107--113. doi: 10.14357/19922264230315. EDN: UMVRRV.
\bibitem{27-zac-1}
\Aue{Zatsman, I.} 2024. Building digital spiral models of knowledge 
generation. \textit{19th Forum (International) on Knowledge Asset Dynamics 
Proceedings}. Matera, Italy: Arts for Business Institute. 2185--2196.
\bibitem{28-zac-1}
\Aue{Zatsman, I.} 2023. Digital spiral model of knowledge creation and encoding its 
dynamics. \textit{18th Forum (International) on Knowledge Asset Dynamics 
Proceedings}. Matera, Italy: Arts for Business Institute. 581--596.
\bibitem{29-zac-1}
\Aue{Zatsman, I.\,M.} 2019. Interfeysy tret'ego poryadka v~informatike 
 [Third-order interfaces in informatics]. \textit{Informatika i~ee Primeneniya~--- 
Inform. Appl.} 13(3):82--89. doi: 10.14357/19922264190312. EDN: EHRQLF.
\bibitem{30-zac-1}
\Aue{Zatsman, I.} 2023. Scientific paradigm of informatics as a~third culture. 
\textit{Scientific Technical Information Processing} 50(4):246--258. doi: 
10.3103/S0147688223040111. EDN: CKHMYS.

\end{thebibliography}

 }
 }

\end{multicols}

\vspace*{-6pt}

\hfill{\small\textit{Received April 14, 2024}} 


\vspace*{-12pt}


\Contrl

\vspace*{-3pt}

\noindent
\textbf{Zatsman Igor M.} (b.\ 1952)~--- Doctor of Science in technology, head of 
department, Federal Research Center ``Computer Science and Control'' of the 
Russian Academy of Sciences, 44-2~Vavilov Str., Moscow 119333, Russian 
Federation; \mbox{izatsman@yandex.ru}





\label{end\stat}

\renewcommand{\bibname}{\protect\rm Литература}  %11
\def\stat{inkova}

\def\tit{СТЕПЕНЬ СЕМАНТИЧЕСКОЙ БЛИЗОСТИ ДИСКУРСИВНЫХ ОТНОШЕНИЙ: МЕТОДЫ И~ИНСТРУМЕНТЫ РАСЧЕТА$^*$}

\def\titkol{Степень семантической близости дискурсивных отношений: методы и~инструменты расчета}

\def\aut{О.\,Ю.~Инькова$^1$, М.\,Г.~Кружков$^2$}

\def\autkol{О.\,Ю.~Инькова, М.\,Г.~Кружков}

\titel{\tit}{\aut}{\autkol}{\titkol}

\index{Инькова О.\,Ю.}
\index{Кружков М.\,Г.}
\index{Inkova O.\,Yu.}
\index{Kruzhkov M.\,G.}


{\renewcommand{\thefootnote}{\fnsymbol{footnote}} \footnotetext[1]
{Работа выполнена в~Федеральном исследовательском центре <<Информатика и~управление>> Российской 
академии наук с~использованием ЦКП <<Информатика>> ФИЦ ИУ РАН.}}


\renewcommand{\thefootnote}{\arabic{footnote}}
\footnotetext[1]{Федеральный исследовательский центр <<Информатика и~управление>> Российской академии наук; 
Женевский университет, \mbox{olyainkova@yandex.ru}}
\footnotetext[2]{Федеральный исследовательский центр <<Информатика и~управление>> Российской 
академии наук, \mbox{magnit75@yandex.ru}}

%\vspace*{-14pt}


  
  \Abst{Рассматриваются методы оценки семантической близости дискурсивных 
отношений. Авторы предлагают несколько подходов к~решению этой проблемы с~применением двух информационных ресурсов: коллекции сформированных авторами 
структурированных определений ло\-ги\-ко-се\-ман\-ти\-че\-ских отношений (ЛСО) 
и~Надкорпусной базы данных коннекторов (НБДК), включающей в~себя аннотации переводных 
соответствий текстовых фрагментов с~маркерами ЛСО на русском, французском 
и~итальянском языках. Показано, что при оценке семантической близости ЛСО высокий 
приоритет будут иметь такие факторы, как принадлежность различительных признаков ЛСО к~одному семейству в~структурированных определениях отношений, соответствия между 
показателями различных ЛСО в~оригинальных и~переводных текстах, а также случаи, когда 
различные ЛСО выражаются одинаковыми показателями в~разных контекстах. Менее значим 
фактор сочетаемости различных ЛСО в~рамках одного и~того же контекста. Предполагается, 
что на основе сформулированных методов станет возможным более точно определить, какие 
различительные признаки ЛСО имеют наибольший вес при определении их семантической  
бли\-зости.}
  
  \KW{надкорпусная база данных; логико-семантические отношения; коннекторы; 
аннотирование; фасетная классификация}

  \DOI{10.14357/19922264230412}{FXTSPZ}
  
%\vspace*{-1pt}


\vskip 10pt plus 9pt minus 6pt

\thispagestyle{headings}

\begin{multicols}{2}

\label{st\stat}
  
\section{Степень семантической близости дискурсивных 
отношений}

%\vspace*{-4pt}

  Проблемы классификации дискурсивных отношений, обеспечивающих 
связность текста, занимают лингвистов и~специалистов по автоматической 
обработке текста не один десяток лет: первые исследования начались  
в~1970-х~гг.~[1, 2]. Были предложены их многочисленные классификации (ср.\ 
наиболее известные~[3--7]), однако никто, насколько известно авторам, не 
пытался определить степень семантической близости (ССБ) дискурсивных 
отношений. Это связано прежде всего с~тем, что классификации имеют, за 
редким исключением~\cite{7-in, 8-in, 9-in}, форму списка, и~этот вопрос просто 
не ставился. Однако его решение полезно не только для анализа текста, в~том 
числе автоматического, но и~для когнитивных наук и~переводоведения, 
поскольку позволяет выявить общие закономерности человеческого мышления.
  
  Кроме того, сами дискурсивные отношения определены во многом неточно 
или тавтологично\footnote[3]{См., например, определение отношения альтернативы 
(disjunction) в~теории риторической структуры: (а)~элемент пред\-став\-ля\-ет собой (не 
обязательно исключающую) альтернативу другому; (б)~слу\-ша\-ющий/чи\-та\-тель 
распознает, что связанные элементы альтернативны (см.\ {\sf http://www.sfu.ca/rst}).}, схожие 
или идентичные отношения носят даже в~англоязычных классификациях разные 
названия, а одинаковые названия описывают разную языковую реальность. 
Например, в~теории сегментированного представления дискурса (Segmented 
Discourse Representation Theory, SDRT~[10]) отношение contrast включает как 
отношения <<вопреки ожидаемому>>, так и~уступительные отношения. 
В~классификации Пенсильванского аннотированного корпуса им 
соответствуют два отношения (opposition и~contra-expectation)~\cite{7-in}, 
а~в~теории риторической структуры~--- contrast и~concession~[11] (подробнее 
см.~\cite[с.~37]{9-in}). 

\begin{table*}[b]\small %tabl1
\vspace*{-10pt}
\begin{center}
\Caption{Структурированные определения уступительных ЛСО и~ЛСО <<вопреки 
ожидаемому>>}
\vspace*{2ex}

\tabcolsep=3pt
\begin{tabular}{|l|p{40mm}|p{38mm}|p{57mm}|}
\hline
\multicolumn{1}{|c|}{\textbf{ЛСО}} & \multicolumn{1}{c|}{\tabcolsep=0pt\begin{tabular}{c}\textbf{Базовая семантическая}\\ \textbf{операция}\end{tabular}}&
\multicolumn{1}{c|}{\textbf{Уровень}} &
\multicolumn{1}{c|}{ \tabcolsep=0pt\begin{tabular}{c}\textbf{Дополнительные}\\ \textbf{характеристики}\end{tabular}}\\
\hline
&&&\\[-20pt]
\multicolumn{1}{|l|}{\raisebox{-26pt}[0pt][0pt]{\textbf{Уступительные}}}& 
%\begin{itemize}
\multicolumn{1}{l|}{\raisebox{-26pt}[0pt][0pt]{\ \ \ \  --\ \ операция импликации}}
%\end{itemize} 
& 
%\begin{itemize}
\multicolumn{1}{l|}{\raisebox{-26pt}[0pt][0pt]{\tabcolsep=0pt\begin{tabular}{l}\ \ \ \ --\ \ пропозициональный\\
\hphantom{\ \ \ \ --\ \ }уровень\end{tabular}}}
%\end{itemize}
&
\begin{itemize}
\item $p$ и~$q$~--- положения вещей;\vspace*{-3pt}
\item как правило, если имеет место $q$, то не имеет места~$p$\vspace*{-8pt}
   \end{itemize}
\\
\hline
&&&\\[-20pt]
\multicolumn{1}{|l|}{\raisebox{-48pt}[0pt][0pt]{\tabcolsep=0pt\begin{tabular}{l}\textbf{<<Вопреки}\\ \textbf{ожидаемому>>}\end{tabular} }}& 
%\begin{itemize}
\multicolumn{1}{l|}{\raisebox{-48pt}[0pt][0pt]{\tabcolsep=0pt\begin{tabular}{l}\ \ \ \  --\ \ операция сравнения,\\
 \hphantom{\ \ \ \ --\ \ }уста\-нав\-ли\-ва\-ющая не-\\
 \hphantom{\ \ \ \ --\ \ }сходство $p$ и~$q$\end{tabular}}}
%\end{itemize} 
&
%\begin{itemize}
\multicolumn{1}{l|}{\raisebox{-48pt}[0pt][0pt]{\tabcolsep=0pt\begin{tabular}{l}\ \ \ \  --\ \ пропозициональный\\ 
 \hphantom{\ \ \ \ --\ \ }уровень\end{tabular}}}
%\end{itemize} 
&
 \begin{itemize}
 \item $q$ имеет большую аргументативную\newline силу, чем~$p$;\vspace*{-3pt}
  \item положение вещей $p$ служит аргументом в~пользу ожи\-да\-емо\-го вывода~$r$;\vspace*{-3pt}
  \item положение вещей $q$ служит аргументом в~пользу ожи\-да\-емо\-го вывода не-$r$\vspace*{-8pt}
  \end{itemize}\\
\hline
\end{tabular}
\end{center}
\end{table*}
  
  В~этой связи были сделаны попытки сравнить\linebreak существующие 
классификации, чтобы понять, насколько соотносимы выделяемые в~них 
дискурсивные отношения~[12--14]. В~[14] для этого применяется 
набор различительных признаков. Этих\linebreak признаков, однако, недостаточно, чтобы 
сформулировать уникальное определение отношения, и~некоторые из них 
имеют одинаковый набор признаков. Это касается, например, четырех 
отношений (narration, precondition, background и~parallel) в~SDRT~\cite[с.~38]{14-in}. 
  
  В~работе~[15] были заложены основы для разработки структурированных 
определений дискурсивных, или в~терминологии автора  
ло\-ги\-ко-се\-ман\-ти\-че\-ских, отношений на основе применяемой 
в~НБДК классификации. Каждое 
ЛСО может быть описано набором различительных признаков (см.\ примеры 
в~\cite{16-in} и~\cite{17-in}). Некоторые признаки оказываются общими для 
нескольких ЛСО, другие~--- индивидуальны, т.\,е.\ свойственны только данному 
ЛСО. На момент написания статьи в~НБДК были описаны 26~ЛСО 
с~использованием~52~различительных признаков. Это позволяет дать каждому 
ЛСО уникальное определение (см.\ примеры в~разд.~2), а~также определить 
ССБ ЛСО. 

\vspace*{-6pt}
  
\section{Критерии, лежащие в~основе определения степени 
семантической близости логико-семантических отношений}

\vspace*{-3pt}

  В~предыдущей работе авторов~[17] показано, что не все различительные 
признаки имеют одинаковый вес при определении семантической близости 
ЛСО и~что, предположительно, наибольшее значение имеет принадлежность 
общих признаков к~одному семейству. 
  

  
  В~основе уступительных ЛСО и~ЛСО <<вопреки ожидаемому>> лежат 
разные базовые операции: импликация~--- для первого и~сравнение, 
уста\-нав\-ли\-ва\-ющее несходство $p$ и~$q$,~--- для второго (табл.~1). Это 
значит, что эти два ЛСО находятся в~разных семантических группах. Оба ЛСО 
при этом установлены на пропозициональном уровне, т.\,е.\ непосредственно 
между положениями дел $p$ и~$q$, которые они связывают, и~оба используют 
отрицательный коррелят одного из положений вещей. Иначе говоря, признаки 
<<как правило, если имеет место~$q$, то не имеет места $p$>> и~<<положение 
вещей~$q$ служит аргументом в~пользу ожидаемого вывода не-$r$>> 
принадлежат к~одному семейству. В~примере~(1) с~ЛСО <<вопреки 
ожидаемому>>: \textit{Ему [$\ldots$] очень неприятно было сталкиваться с~народом,} {\bfseries\textit{но}} \textit{он шел именно туда, где виднелось больше 
народу}. [Ф.\,М.~Достоевский. Преступление и~наказание], положение вещей 
$p$\;=\;<<ему очень неприятно было сталкиваться с~народом>> ориентирует в~пользу вывода $r$\;=\;<<он не должен был бы идти к~народу>>. Этот вывод 
опровергается непосредственно в~$q$ (=\;не-$r$)\;=\;<<он шел именно туда, где 
виднелось больше народу>>. Семантический механизм, лежащий в~основе 
уступительных отношений (их прототипическим показателем может считаться 
союз \textit{хотя}), совпадает с~этим семантическим механизмом, но 
в~зеркальном отражении: 
  \begin{gather*}
p\ \mbox{\textit{хотя}}\  q (q \to  \mbox{не-}p)\\
p \to r\ \mbox{но}\  q\ (q = \mbox{не-}r),\ \mbox{т.\,е.}\ p \to \mbox{не-}q\ 
\mbox{\textit{но}}\ q.
\end{gather*}
  %
  Отсюда необходимость при замене \textit{хотя} на \textit{но} и~наоборот 
изменить порядок следования фрагментов текста: \textit{Ему неприятно было 
сталкиваться с~народом}, {\bfseries\textit{но}} \textit{он шел туда, где виднелось 
больше народу} (ЛСО <<вопреки ожидаемому>>); \textit{Он шел туда, где 
виднелось больше народу}, {\bfseries\textit{хотя}} \textit{ему неприятно было 
сталкиваться с~народом} (ЛСО уступки)~\cite{18-in}. Это позволяет говорить 
о~семантической близости двух ЛСО и,~например, в~классификации~\cite{7-in} 
они объединены в~одну группу concession.

\begin{table*}[b]\small %tabl2
\vspace*{-6pt}
\begin{center}
\Caption{Логико-семантические отношения, соответствующие ЛСО <<вопреки ожидаемому>> в~оригинальных и~переводных текстах }
\vspace*{2ex}

\tabcolsep=4.3pt
\begin{tabular}{|c|l|c|c|c|c|c|c|}
\hline
\textbf{ЛСО1}&\multicolumn{1}{c|}{\textbf{ЛСО2}}&\textbf{1}\;+\;\textbf{2}&\textbf{1}&
\textbf{2}&\textbf{1}\;$\to$\;\textbf{2}&\textbf{2}\;$\to$\;\textbf{1}&\textbf{Сумма}\\
\hline
<<вопреки ожидаемому>>&уступительные&237\hphantom{9}&2140&853&11,07\%\hphantom{9}&27,78\%\hphantom{9}&38,86\%\hphantom{9}\\
<<вопреки ожидаемому>>&одновременность&139\hphantom{9}&2140&1268\hphantom{9}&6,50\%&10,96\%\hphantom{9}&17,46\%\hphantom{9}\\
<<вопреки ожидаемому>>&соединительные&149\hphantom{9}&2140&2088\hphantom{9}&6,96\%&7,14\%&14,10\%\hphantom{9}\\
<<вопреки ожидаемому>>&сопоставительные&78&2140&807&3,64\%&9,67\%&13,31\%\hphantom{9}\\
<<вопреки ожидаемому>>&пропозициональное 
сопутствование&39&2140&378&1,82\%&10,32\%\hphantom{9}&12,14\%\hphantom{9}\\
<<вопреки ожидаемому>>&исключение из 
рассмотрения&\hphantom{9}8&2140&\hphantom{9}90&0,37\%&8,89\%&9,26\%\\
<<вопреки ожидаемому>>&иллокутивное 
сопутствование&17&2140&471&0,79\%&3,61\%&4,40\%\\
<<вопреки ожидаемому>>&интенсиональная 
генерализация&\hphantom{9}8&2140&248&0,37\%&3,23\%&3,60\%\\
<<вопреки ожидаемому>>&замещение&\hphantom{9}7&2140&294&0,33\%&2,38\%&2,71\%\\
<<вопреки ожидаемому>>&пропозициональная 
коррекция&\hphantom{9}4&2140&165&0,19\%&2,42\%&2,61\%\\
<<вопреки ожидаемому>>&условные&12&2140&1075\hphantom{9}&0,56\%&1,12\%&1,68\%\\
<<вопреки ожидаемому>>&спецификация&11&2140&1608\hphantom{9}&0,51\%&0,68\%&1,20\%\\
<<вопреки ожидаемому>>&исключение&\hphantom{9}5&2140&615&0,23\%&0,81\%&1,05\%\\
<<вопреки ожидаемому>>&отрицательная 
альтернатива&\hphantom{9}2&2140&271&0,09\%&0,74\%&0,83\%\\
<<вопреки ожидаемому>>&оговорка&\hphantom{9}1&2140&150&0,05\%&0,67\%&0,71\%\\
<<вопреки ожидаемому>>&экстенсиональная 
генерализация&\hphantom{9}2&2140&588&0,09\%&0,34\%&0,43\%\\
<<вопреки ожидаемому>>&переформулирование&\hphantom{9}2&2140&1183\hphantom{9}&0,09\%&0,17\%&0,26\%\\
<<вопреки ожидаемому>>&пропозициональная 
альтернатива&\hphantom{9}1&2140&1238\hphantom{9}&0,05\%&0,08\%&0,13\%\\
\hline
\multicolumn{8}{p{163mm}}{\footnotesize \hspace*{3mm}Расшифровка названий столбцов: 
1\;+\;2~--- число переводных аннотаций, в~которых ЛСО1 в~тексте на одном языке 
соответствует ЛСО2 в~тексте на другом языке; 1~--- число аннотаций, в~которых в~любом из 
текстов проставлено ЛСО1; 2~--- число аннотаций, в~которых в~любом из текстов 
проставлено ЛСО2; 1\;$\to$\;2~--- процент соответствия для ЛСО1 с~ЛСО2; 2\;$\to$\;1~--- 
процент соответствия для ЛСО2 с~ЛСО1; сумма~--- сумма двух предыдущих показателей.}
\end{tabular}
\end{center}
\end{table*}

  
  
  Кроме того, сформулирована гипотеза, согласно которой при определении 
ССБ ЛСО могут учитываться также другие 
факторы:
\begin{enumerate}[(1)] 
\item соответствия ЛСО в~оригинальных и~переводных текстах; 
\item случаи, когда разные ЛСО выражаются одним и~тем же показателем; 
\item сочетаемость показателей ЛСО в~одном фрагменте текста.
\end{enumerate}
 В~НБДК для 
ЛСО, имеющих структурированные определения, были получены 
количественные данные по этим трем критериям.

  
  
\subsection{Соответствие логико-семантических отношений в~оригинальных и~переводных текстах}

  Соответствие ЛСО в~оригинальных и~переводных текстах означает, что 
некоторому ЛСО в~тексте оригинала, точнее, его показателю, соответствует 
показатель иного ЛСО в~тексте перевода. Так, если для перевода на 
французский язык коннектора \textit{но} в~примере~(1) был выбран коннектор 
\textit{mais}, также выражающий ЛСО <<вопреки ожидаемому>>: (2)~\textit{Il 
lui $\acute{\mbox{e}}$tait d$\acute{\mbox{e}}$sagr$\acute{\mbox{e}}$able, 
tr$\grave{\mbox{e}}$s d$\acute{\mbox{e}}$sagr$\acute{\mbox{e}}$able, de 
rencontrer du monde} {\bfseries\textit{mais}} \textit{il allait justement 
l$\grave{\mbox{a}}$ o$\grave{\mbox{u}}$ l'on en voyait le plus} [перевод 
$\acute{\mbox{E}}$lisabeth Guertik], то в~примере~(3) тот же коннектор 
переведен \textit{bien que}~--- показателем уступительных ЛСО: 
\textit{С~такой поправкой смысл телеграммы становился ясен,} 
{\bfseries\textit{но}}\textit{, конечно, трагичен}.~--- \textit{Ainsi 
corrig$\acute{\mbox{e}}$, le t$\acute{\mbox{e}}$l$\acute{\mbox{e}}$gramme 
prenait un sens parfaitement clair,} {\bfseries\textit{bien que}} \textit{tragique, 
naturellement}. [М.~Булгаков. Мастер и~Маргарита, перевод Claude Ligny].
  
  Количественные данные по ЛСО, соответствующим ЛСО <<вопреки 
ожидаемому>> в~оригинальных и~переводных текстах на русском, французском и~итальянском языках, приведены в~табл.~2.
  
  
  Для ЛСО <<вопреки ожидаемому>> в~НБДК сформирована 2141~двуязычная 
аннотация. В~237~случаях ему соответствует уступительное ЛСО. Это 
подтверждает важность критерия принадлежности \mbox{различительных} признаков к~одному семейству. 

Схожую картину можно наблюдать для других отношений 
(табл.~3): для сопоставительных и~соединительных ЛСО (основаны на 
общей базовой операции и~имеют общий различительный признак 
<<сходство~$p$ и~$q$ относительно некоторого ``общего\linebreak знаменателя''>>); для 
ЛСО оговорки и~пропозициональной альтернативы (они имеют общий 
различительный признак~--- <<$p$ и~$q$~--- положения вещей, име\-ющие 
статус гипотезы>>); для ЛСО \mbox{одновременности} и~со\-по\-став\-ле\-ния (их 
различительные при\-зна\-ки <<T$p$ включает в~себя T$q$>> и~<<$p$ и~$q$ 
актуальны для говорящего в~момент речи T$d$>> принадлежат к~семейству 
признаков <<Единство временного интервала>>); для ЛСО одновременности 
и~пропозиционального сопутствования (об\-щий признак <<T$p$ включает 
в~себя T$q$>>). 
  
\begin{table*}\small %tabl3
\begin{center}
\Caption{Соответствия других ЛСО }
\vspace*{2ex}

\begin{tabular}{|l|l|c|c|c|c|c|c|}
\hline
\multicolumn{1}{|c|}{\textbf{ЛСО1}}&\multicolumn{1}{c|}{\textbf{ЛСО2}}&\textbf{1}\;+\;\textbf{2}&\textbf{1}&\textbf{2}&\textbf{1}\;
$\to$\;\textbf{2}&\textbf{2}\;$\to$\;\textbf{1}&\textbf{Сумма}\\
\hline
соединительные&сопоставительные&272\hphantom{9}&2088&807&13,03\%&33,71\%&46,73\%\\
оговорка&пропозициональная альтернатива&40&\hphantom{9}150&1238\hphantom{9}&26,67\%&\hphantom{9}3,23\%&29,90\%\\
одновременность&сопоставление&180\hphantom{9}&1268&807&14,20\%&22,30\%&36,50\%\\
одновременность &пропозициональное 
сопутствование&43&1268&378&\hphantom{9}3,39\%&11,38\%&14,77\%\\
\hline
\end{tabular}
\end{center}
\vspace*{-4pt}
\end{table*}

\begin{table*}[b]\small %tabl4
\vspace*{-12pt}
\begin{center}
\Caption{Количественные данные по ЛСО, выражаемым одним показателем}
\vspace*{2ex}

\begin{tabular}{|c|l|l|c|}
\hline 
\textbf{Язык}&\multicolumn{1}{c|}{\textbf{Коннектор}}&\multicolumn{1}{c|}{\textbf{ЛСО}}&\textbf{Количество аннотаций}\\
\hline
\multicolumn{1}{|c|}{\raisebox{-11pt}[0pt][0pt]{RU}}&\multicolumn{1}{l|}{\raisebox{-11pt}[0pt][0pt]{а то}}&отрицательная альтернатива&125\hphantom{9}\\
&&пропозициональная альтернатива&12\\
&&исключение из рассмотрения&\hphantom{9}6\\
\hline
\multicolumn{1}{|c|}{\raisebox{-6pt}[0pt][0pt]{RU}}&\multicolumn{1}{l|}{\raisebox{-6pt}[0pt][0pt]{если$\|$то}}&условные&183\hphantom{9}\\
&&сопоставительные&13\\
\hline
\multicolumn{1}{|c|}{\raisebox{-6pt}[0pt][0pt]{RU}}&\multicolumn{1}{l|}{\raisebox{-6pt}[0pt][0pt]{когда}}&одновременность&13\\
&&условные&\hphantom{9}1\\
\hline
\multicolumn{1}{|c|}{\raisebox{-6pt}[0pt][0pt]{RU}}&\multicolumn{1}{l|}{\raisebox{-6pt}[0pt][0pt]{когда$\|$то}}&одновременность&38\\
&&условные&\hphantom{9}6\\
\hline
\multicolumn{1}{|c|}{\raisebox{-11pt}[0pt][0pt]{RU}}
&\multicolumn{1}{l|}{\raisebox{-11pt}[0pt][0pt]{между тем}}
&одновременность&126\hphantom{9}\\
&&<<вопреки ожидаемому>>&53\\
&&сопоставительные&11\\
\hline
\multicolumn{1}{|c|}{\raisebox{-6pt}[0pt][0pt]{RU}}&\multicolumn{1}{l|}{\raisebox{-6pt}[0pt][0pt]{между тем как}}&сопоставительные&29\\
&&одновременность&\hphantom{9}6\\
\hline
\multicolumn{1}{|c|}{\raisebox{-18pt}[0pt][0pt]{RU}}
&\multicolumn{1}{l|}{\raisebox{-18pt}[0pt][0pt]{разве}}
&оговорка&20\\
&&исключение&\hphantom{9}5\\
&&исключение из рассмотрения&\hphantom{9}4\\
&&условные&\hphantom{9}2\\
\hline
\multicolumn{1}{|c|}{\raisebox{-6pt}[0pt][0pt]{FR}}&\multicolumn{1}{l|}{\raisebox{-6pt}[0pt][0pt]{cependant}}&<<вопреки ожидаемому>>&100\hphantom{9}\\
&&одновременность&27\\
\hline
\multicolumn{1}{|c|}{\raisebox{-6pt}[0pt][0pt]{FR}}&\multicolumn{1}{l|}{\raisebox{-6pt}[0pt][0pt]{en m$\hat{\mbox{e}}$me temps}}&одновременность&29\\
&&сопоставительные&\hphantom{9}1\\
\hline
\multicolumn{1}{|c|}{\raisebox{-6pt}[0pt][0pt]{FR}}&\multicolumn{1}{l|}{\raisebox{-6pt}[0pt][0pt]{quand}}&одновременность&197\hphantom{9}\\
&&условные&10\\
\hline
\end{tabular}
\end{center}
\end{table*}

  
  Напротив, ЛСО, соответствующие ЛСО <<вопреки ожидаемому>> 
и~представленные менее чем в~1\% аннотаций (см.\ табл.~2), не имеют 
различительных признаков, принадлежащих к~одному семейству, и~выбор их 
показателей для перевода показателя ЛСО <<вопреки ожидаемому>> может 
быть квалифицирован как авторский и~контекстуальный.
  
\subsection{Разные логико-семантические отношения выражаются одним~и~тем~же~показателем}

  Известно, что коннекторы в~значительной своей части относятся 
к~многозначным языковым единицам, т.\,е.\ могут служить показателями более 
чем одного ЛСО. Так, для русского союза \textit{и} принято выделять пять 
значений: сочинительное, временного следования, добавления,  
ре\-зуль\-та\-тив\-но-след\-ст\-вен\-ное и~несоответствия; для союза 
\textit{когда}~--- два: одновременности и~условия; у~союза \textit{но} 
выделяются собственно противительное  
и~про\-ти\-ви\-тель\-но-усту\-пи\-тель\-ное значения, а~у~\textit{хотя}~--- 
уступительное и~усту\-пи\-тель\-но-про\-ти\-ви\-тель\-ное и~т.\,д.~[19--21]. Это 
отражают и~данные НБДК, причем с~указанием на частотность того или иного 
значения коннектора в~сформированных аннотациях. 

В~табл.~4 приведены 
выборочно данные для многозначных коннекторов русского и~французского 
языков.
  

  
  Приведенные данные подтверждают прежде всего положения теории 
грамматикализации, согласно которым семантическая эволюция языковых 
единиц имеет определенные закономерности.\linebreak Так, было показано, что на основе 
значения одновременности может развиваться семантика сопоставления и~противопоставления, а~также импликации~\cite{22-in}. Это хорошо видно на 
примере \mbox{коннекторов} \textit{когда}, \textit{между тем}, а~также французских 
\textit{cependant} `в~то же время, однако', \textit{en m$\hat{\mbox{e}}$me temps} 
`в~то же время' и~\textit{quand} `когда' (см.\ табл.~4). С~другой стороны, эти 
данные подтверждают гипотезу авторов о~том, что набор ЛСО, которые может 
маркировать один показатель, не случаен, а~включает семантически близкие 
ЛСО. Так, коннектор \textit{разве} зафиксирован в~НБДК как показатель ЛСО 
оговорки, исключения, исключения из рассмотрения и~условия. Эти ЛСО имеют 
общие различительные признаки. Ло\-ги\-ко-се\-ман\-ти\-че\-ские отношения оговорки и~условия~--- два признака: 
базовая операция импликации и~признаки из семейства гипотетичность; ЛСО 
условия и~исключения устанавливаются на пропозициональном уровне, а~ЛСО 
оговорки и~исключения из рас\-смот\-ре\-ния~--- на уров\-не вы\-ска\-зы\-ва\-ния; ЛСО 
оговорки, исключения и~исключения из рас\-смот\-ре\-ния обладают общими 
признаками на уровне семейства признаков (семантика исключения), а~ЛСО 
исключения и~исключения из рас\-смот\-ре\-ния осно\-ва\-ны на общей базовой 
операции (соотнесение элемента и~множества).
  
  Таким образом, данный критерий может быть полезен при определении CСБ 
ЛСО и~иметь достаточно высокий приоритет.
  
\subsection{Сочетаемость логико-семантических отношений в~рамках одного фрагмента текста}

  Третий критерий, который можно учитывать при определении ССБ ЛСО,~--- 
сочетаемость ЛСО, точнее их показателей. Здесь, однако, возникает ряд 
сложностей, связанных с~тем, что возможность сочетаемости показателей 
зависит в~первую очередь от морфологической природы показателя ЛСО. Как 
известно, коннекторы относятся к~разнообразным морфологическим классам: 
сочинительные со\-юзы (\textit{и}, \textit{а}, \textit{но}); подчинительные союзы 
(\textit{хотя}, \textit{потому что}, \textit{как}), так называемые 
<<конкретизаторы со\-юзов>>, перешедшие в~класс коннекторов, как правило, из 
наречных выражений (\textit{в~то же время}, \textit{однако}, \textit{впрочем}); 
предлоги (\textit{кроме}, \textit{после}). Союзы, например, как сочинительные, 
так и~подчинительные, не могут сочетаться между собой в~рамках единого 
фрагмента текста, и, наоборот, наибольшей легкостью в~сочетании именно с~союзами обладают <<конкретизаторы>> (\textit{но однако}, \textit{но впрочем}, 
\textit{а~между тем}, \textit{или например}, \textit{и~в~частности}). Если для 
показателей некоторых ЛСО можно выявить закономерности, то другие менее 
избирательны в~своих сочетаниях. Так, показатель ЛСО спецификации 
\textit{например} сочетается со всеми сочинительными союзами, а~показатель 
ЛСО <<вопреки ожидаемому>> \textit{впрочем} только с~союзами~\textit{а} 
и~\textit{но}, т.\,е.\ показателями близких ему (\textit{а}) или тех же (\textit{но}) 
ЛСО. Можно также учитывать двухместные реализации коннекторов, т.\,е.\ 
такие, где компоненты коннектора находятся в~каждом из соединяемых 
фрагментов текста, например \textit{хотя$\ldots$\ но}: \textit{Хотя он меня 
очень уговаривал, но я~не согласился}. Но такие сочетания возможны не для 
всех ЛСО и~сужают круг возможностей для получения адекватных 
количественных данных.
 
  В~связи с~вышесказанным при подсчете ССБ ЛСО этот критерий может 
использоваться лишь как дополнительный.
  
\section{Заключение}

  Из четырех рассмотренных критериев определения ССБ ЛСО: 
(1)~принадлежности различительных признаков ЛСО к~одному семейству, 
(2)~соответствия ЛСО в~оригинальных и~переводных \mbox{текс\-тах}, (3)~возможности 
одного показателя выражать разные ЛСО и~(4)~сочетаемости показателей ЛСО 
в~одном фрагменте текста~--- первые три могут иметь достаточно высокий 
приоритет. Четвертый признак обладает, напротив, наименьшим весом при 
определении ССБ ЛСО. 
  
  Степень детальности разметки, а следовательно, и~определений ЛСО не 
позволяет пока объяснить некоторые явления. Например, семантическую 
близость ЛСО условия и~одновременности, который подтверждается как их 
соответствиями в~оригинальных и~переводных текстах, так и~воз\-мож\-ностью 
выражаться одним показателем (\textit{когда}). Их общий признак <<T$p$ 
включает в~себя T$q$>> не входит в~определение условных ЛСО, так как 
соотношение временн$\acute{\mbox{ы}}$х планов положений вещей~$p$ и~$q$ может быть 
самым различным в~условном периоде. С~другой стороны, при ЛСО 
одновременности различным может быть их семантическое соотношение 
(семантическая независимость, противопоставленность, причина, следствие 
и~т.\,д.). Перевод показателя ЛСО одновременности показателем условных 
ЛСО наблюдается только при одновременной реализации положений 
вещей~$p$ и~$q$ и~при возможности установить между ними отношение 
импликации. Семантическая близость данных двух ЛСО может быть, 
следовательно, установлена на более низком иерархическом уровне, а~именно: 
при определении частных случаев его реализации. В~НБДК такая возможность 
предусмотрена, что позволит в~дальнейшем более детально описывать каждое 
ЛСО и~его виды, а~значит, более точно определить ССБ ЛСО.
{\looseness=1

}
  
{\small\frenchspacing
 {\baselineskip=10.6pt
 %\addcontentsline{toc}{section}{References}
 \begin{thebibliography}{99}
\bibitem{1-in}
\Au{Hobbs J.\,R.} A~computational approach to discourse analysis.~--- 
New York, NY, USA: Department of Computer Science, City College, City University of New 
York, 1976.  Research Report 76-2. P.~28--38.
\bibitem{2-in}
\Au{Hobbs J.\,R.} Why is discourse coherent?~--- Menlo Park, CA, 
USA: SRI International, 1978. SRI Technical Note 176. 44~p.
\bibitem{3-in}
\Au{Halliday M.\,A.\,K., Hasan~R.}  Cohesion in English.~--- London: Longman, 1976. 374~p.


\bibitem{5-in} %4
\Au{Mann W.\,C., Thompson~S.\,A.} Rhetorical structure theory: Towards a functional theory of 
text organization~// Text, 1988. Vol.~8. No.\,3. P.~243--281. doi: 10.1515/text.\linebreak  1.1988.8.3.243.

\bibitem{6-in} %5
\Au{Asher N.} Reference to abstract objects in discourse.~--- Dordrecht: Kluwer, 1993. 455~p.

\bibitem{4-in} %6
\Au{Halliday M.\,A.\,K.} An introduction to functional grammar.~--- 2nd ed.~--- London: 
Edward Arnold, 1994. 434~p.

\bibitem{7-in} %7
PDTB Research Group. The Penn Discourse Treebank 2.0 annotation manual.~--- Philadelphia, PA, USA: Institute for Research in Cognitive Science, University 
of Pennsylvania, 2007.  Technical Report 
IRCS-08-01. 104~p. {\sf https://www.cis.upenn.edu/$\sim$elenimi/\linebreak pdtb-manual.pdf}.
\bibitem{8-in}
\Au{Breindl E., Volodina~A., \mbox{Wa{\!\ptb{\!\ss}}\,ner}~U.\,H.} Handbuch der deutschen 
Konnektoren~2: Semantik der deutschen Satzverkn$\ddot{\mbox{u}}$pfer.~--- Berlin: Walter de Gruyter, 2014. 
1327~p.
\bibitem{9-in}
\Au{Инькова О.\,Ю.} Логико-се\-ман\-ти\-че\-ские отношения: проблемы 
классификации~// Связность текста: мереологические ло\-ги\-ко-се\-ман\-ти\-че\-ские 
отношения.~--- М.: ЯСК, 2019. С.~11--98.
\bibitem{10-in}
\Au{Asher N., Lascarides~A.} Logics of conversation.~--- Cambridge: Cambridge University 
Press, 2003. 526~p.
\bibitem{11-in}
\Au{Carlson L., Marcu D.} Discourse tagging reference manual.~--- Marina del Rey, CA, USA: Information Sciences Institute, University of Southern 
California, 2001.  Technical Report ISI-TR-545. 87~p.



\bibitem{13-in} %12
\Au{Chiarcos Ch.} Towards interoperable discourse annotation: Discourse features in the 
Ontologies of Linguistic Annotation~// 9th Conference (International) on Language Resources 
and Evaluation Proceedings~/ Eds.\ N.~Calzolari, K.~Choukri, T.~Declerck, \textit{et al.}~--- Reykjavik, Iceland: European Language Resources Association 
(ELRA), 2014. P.~4569--4577.

\bibitem{12-in} %13
\Au{Benamara F., Taboada~M.} Mapping different rhetorical relation annotations: A~proposal~// 
4th Joint Conference on Lexical and Computational Semantics  Proceedings~/ Eds.\ M.~Palmer, G.~Boleda, P.~Rosso.~--- Denver, CO, USA: 
Association for Computational Linguistics, 2015. Р.~147--152. doi: 10.18653/v1/S15-1016.

\bibitem{14-in}
\Au{Sanders T., Demberg~V., Hoek~J., Scholman~M., Asr~F.\,T., Zufferey~S., Evers-Vermeul~J.} 
Unifying dimensions in coherence relations: How various annotation frameworks are related~// 
Corpus Linguist. Ling., 2018. Vol.~17. No.\,1. P.~1--71. doi:  
10.1515/cllt-2016-0078.
\bibitem{15-in}
\Au{Инькова О.\,Ю.} Определения дискурсивных отношений: опыт Надкорпусной базы 
данных коннекторов~// Компьютерная лингвистика и~интеллектуальные технологии: По 
мат-лам ежегодной \mbox{Междунар.} конф. <<Диалог>>.~--- М.: РГГУ, 2021. Вып.~20(27). 
С.~328--338.
\bibitem{16-in}
\Au{Инькова О.\,Ю., Кружков М.\,Г.} Структурированные определения дискурсивных 
отношений в~Надкорпусной базе данных коннекторов~// Информатика и~её применения, 
2021. Т.~15. Вып.~4. С.~27--32. doi: 10.14357/19922264210404. EDN: EZJXVI.

\bibitem{17-in}
\Au{Инькова О.\,Ю., Кружков М.\,Г.} Критерии определения семантической близости 
дискурсивных отношений~// Информатика и~её применения, 2023. Т.~17. Вып.~3.  
С.~100--106. doi: 10.14357/19922264230314. EDN: UJZJZI.

\bibitem{18-in}
\Au{Инькова О.\,Ю., Нуриев В.\,А.} Насколько лингвоспецифичен союз \textit{хотя}?~// 
Компьютерная лингвистика и~интеллектуальные технологии: По мат-лам ежегодной 
Междунар. конф. <<Диалог>>.~--- М.: РГГУ, 2018. Вып.~17(24). С.~254--266.

\bibitem{20-in} %19
Словарь современного русского литературного языка: в~17~т.~/ Под ред. 
В.\,И.~Чернышева.~--- М., Л.: Изд-во Академии наук СССР~/ Наука, 1950--1965.

\bibitem{19-in} %20
Русская грамматика~/ Под ред. Н.\,Ю.~Шведовой.~--- М.: Наука, 1980.   Т.~2.
714~с.

\bibitem{21-in}
Словарь русского языка: в~4~т.~/ Под ред. А.\,П.~Ев\-гень\-евой.~--- М.: Русский язык, 
 1981--1984. 
\bibitem{22-in}
\Au{Heine B., Kuteva T.} World lexicon of grammaticalization.~--- Cambridge: Cambridge 
University Press, 2002. 387~p.
\end{thebibliography}

 }
 }

\end{multicols}

\vspace*{-10pt}

\hfill{\small\textit{Поступила в~редакцию 15.10.23}}

\vspace*{8pt}

%\pagebreak

%\newpage

%\vspace*{-28pt}

\hrule

\vspace*{2pt}

\hrule



\def\tit{EVALUATING THE DEGREE OF~DISCOURSE RELATIONS SEMANTIC AFFINITY: 
METHODS AND~INSTRUMENTS}


\def\titkol{Evaluating the degree of~discourse relations semantic affinity: 
Methods and instruments}


\def\aut{O.\,Yu.~Inkova$^{1,2}$ and~M.\,G.~Kruzhkov$^1$}

\def\autkol{O.\,Yu.~Inkova and~M.\,G.~Kruzhkov}

\titel{\tit}{\aut}{\autkol}{\titkol}

\vspace*{-14pt}


\noindent
$^1$Federal Research Center ``Computer Science and Control'' of the Russian Academy of Sciences, 
44-2~Vavilov\linebreak
$\hphantom{^1}$Str., Moscow 119333, Russian Federation

\noindent
$^2$University of Geneva, 22 Bd des Philosophes, CH-1205 Geneva 4, Switzerland


\def\leftfootline{\small{\textbf{\thepage}
\hfill INFORMATIKA I EE PRIMENENIYA~--- INFORMATICS AND
APPLICATIONS\ \ \ 2023\ \ \ volume~17\ \ \ issue\ 4}
}%
 \def\rightfootline{\small{INFORMATIKA I EE PRIMENENIYA~---
INFORMATICS AND APPLICATIONS\ \ \ 2023\ \ \ volume~17\ \ \ issue\ 4
\hfill \textbf{\thepage}}}

\vspace*{3pt}




\Abste{The methods for evaluating semantic affinity of discourse relations are examined. The 
authors propose several approaches to this problem using two information resources: 
a~collection of structured definitions of logical-semantic relations (LSRs) formed by the authors
and the Supracorpora 
Database of Connectives incorporating\linebreak\vspace*{-12pt}}

\Abstend{corpus-based annotations of translation correspondences 
that include text fragments with LSR markers in Russian,
French, and Italian. It is demonstrated that when it comes to 
assessing the semantic affinity of LSRs, the following factors will be of a~higher priority: affiliation of 
distinctive features of LSRs with the same family in the structured definitions of relations; correspondences 
between markers of different LSRs in the source and target texts; and cases when different LSRs are 
regularly expressed by the same markers in different contexts. Of a~lesser importance is the factor of 
compatibility of different LSRs within the same context. It is assumed that based on the proposed 
methods, it will become possible to specify more precisely which distinguishing features of LSRs 
have the greatest impact on their potential semantic affinity.}

\KWE{supracorpora database; logical-semantic relations; connectives; annotation; faceted 
classification}


  \DOI{10.14357/19922264230412}{FXTSPZ}

\vspace*{-16pt}

\Ack

\vspace*{-3pt}

\noindent
The research was carried out using the infrastructure of the Shared Research Facilities ``High 
Performance Computing and Big Data'' (CKP ``Informatics'') of FRC CSC RAS (Moscow).


\vspace*{6pt}

  \begin{multicols}{2}

\renewcommand{\bibname}{\protect\rmfamily References}
%\renewcommand{\bibname}{\large\protect\rm References}

{\small\frenchspacing
 {%\baselineskip=10.8pt
 \addcontentsline{toc}{section}{References}
 \begin{thebibliography}{99}
\bibitem{1-in-1}
\Aue{Hobbs, J.\,R.} 1976. A~computational approach to discourse analyses. New York, NY: 
Department of Computer Science, City College, City University of New York. Research Report  
76-2. 28--38.
\bibitem{2-in-1}
\Aue{Hobbs, J.\,R.} 1978. Why is discourse coherent? Menlo Park, CA: SRI International. SRI 
Technical Note 176. 44~p.
\bibitem{3-in-1}
\Aue{Halliday, M.\,A.\,K., and R.~Hasan.} 1976. \textit{Cohesion in English}. London: Longman. 
374~p.


\bibitem{5-in-1} %4
\Aue{Mann, W.\,C., and S.\,A.~Thompson.} 1988. Rhetorical structure theory: Towards 
a~functional theory of text organization. \textit{Text} 8(3):243--281. doi: 
10.1515/text.1.1988.8.3.243.
\bibitem{6-in-1} %5
\Aue{Asher, N.} 1993. \textit{Reference to abstract objects in discourse}. Dordrecht: Kluwer. 
455~p.
\bibitem{4-in-1} %6
\Aue{Halliday, M.\,A.\,K.} 1994. \textit{An introduction to functional grammar}. 2nd ed. London: 
Edward Arnold. 434~p.

\bibitem{7-in-1}
PDTB Research Group. 2007. The Penn Discourse Treebank 2.0 annotation manual. Philadelphia, 
PA: Institute for Research in Cognitive Science, University of Pennsylvania. Technical Report 
IRCS-08-01. 104~p. Available at: {\sf https://www.cis.upenn.edu/$\sim$elenimi/pdtb-manual.pdf} 
(accessed November~28, 2023).
\bibitem{8-in-1}
\Aue{Breindl, E., A.~Volodina, and U.\,H.~Wa{\!\ptb{\!\ss}}ner.} 2014. \textit{Handbuch der 
deutschen Konnektoren~2: Semantik der deutschen Satzverkn$\ddot{\mbox{u}}$pfer}. Berlin: Walter de Gruyter. 
1327~p.
\bibitem{9-in-1}
\Aue{Inkova, O.\,Yu.} 2019. Logiko-semanticheskie otnosheniya: problemy klassifikatsii  
[Logical-semantic relations: Classification problems]. \textit{Svyaznost' teksta: mereologicheskie 
logiko-semanticheskie otnosheniya} [Text coherence: Mereological logical semantic relations]. 
Moscow: LRC Publishing House. 11--98.
\bibitem{10-in-1}
\Aue{Asher, N., and A.~Lascarides.} 2003. \textit{Logics of conversation}. Cambridge: Cambridge 
University Press. 526~p.
\bibitem{11-in-1}
\Aue{Carlson, L., and D.~Marcu.} 2001. Discourse tagging reference manual.  Marina del Rey, CA: Information Sciences Institute, University of Southern 
California. Technical Report 
ISI-TR-545.  87~p. Available at: {\sf https://www.isi.edu/~marcu/discourse/tagging-ref-manual.pdf} 
(accessed November~28, 2023).

\bibitem{13-in-1} %12
\Aue{Chiarcos, Ch.} 2014. Towards interoperable discourse annotation: Discourse features in the 
Ontologies of Linguistic Annotation. \textit{9th Conference (International) on\linebreak Language Resources 
and Evaluation Proceedings}. Eds. N.~Calzolari, K.~Choukri, T.~Declerck, \textit{et al.} Reykjavik, Iceland: 
European Language Resources Association. 4569--4577.
{ %\looseness=1

}

\bibitem{12-in-1} %13
\Aue{Benamara, F., and M.~Taboada.} 2015. Mapping different rhetorical relation annotations: 
A~proposal. \textit{4th Joint Conference on Lexical and Computational Semantics}. Eds. 
M.~Palmer, G.~Boleda, and P.~Rosso. Denver, CO, USA: Association for Computational 
Linguistics. 147--152. doi: 10.18653/v1/S15-1016.

\bibitem{14-in-1}
\Aue{Sanders, T., V.~Demberg, J.~Hoek, M.~Scholman, F.\,T.~Asr, S.~Zufferey, and  
J.~Evers-Vermeul.} 2018. Unifying dimensions in coherence relations: How various annotation 
frameworks are related. \textit{Corpus Linguist. Ling.} 17(1):1--71. doi: 10.1515/cllt-2016-0078.
\bibitem{15-in-1}
\Aue{Inkova, O.\,Yu.} 2021. Opredeleniya diskursivnykh otnosheniy: opyt Nadkorpusnoy bazy 
dannykh konnektorov [Definition of discursive relations: The experience of the supracorpora 
database of connectors]. \textit{Komp'yuternaya lingvistika i~intellektual'nye Tekhnologii: Po 
mat-lam ezhegodnoy Mezhdunar.  konf. ``Dialog''} [Computational Linguistics 
and Intellectual Technologies: Papers from the Annual Conference (International) ``Dialogue'']. 
Moscow: RGGU. 20(27):328--338.
\bibitem{16-in-1}
\Aue{Inkova, O.\,Yu., and M.\,G.~Kruzhkov.} 2021. Strukturirovannye opredeleniya 
diskursivnykh otnosheniy v~Nadkorpusnoy baze dannykh konnektorov [Structured definitions of 
discourse relations in the Supracorpora Database of Connectives]. \textit{Informatika i~ee 
Primeneniya~--- Inform. Appl.} 15(4):27--32. doi: 10.14357/ 19922264210404. EDN: EZJXVI.
\bibitem{17-in-1}
\Aue{Inkova, O.\,Yu., and M.\,G.~Kruzhkov.} 2023. Kriterii opredeleniya semanticheskoy blizosti 
diskursivnykh otnosheniy [Evaluation criteria for discourse relations semantic affinity]. 
\textit{Informatika i~ee Primeneniya~--- Inform. Appl.} 17(3):100--106. doi: 
10.14357/19922264230314. EDN: UJZJZI.

\pagebreak


\bibitem{18-in-1}
\Aue{Inkova, O.\,Yu., and V.\,A.~Nuriev.} 2018. Naskol'ko lingvospetsifichen soyuz \textit{khotya}? [To 
what extent is the conjunction \textit{khotya} language-specific?]. \textit{Komp'yuternaya lingvistika 
i~intellektual'nye tekhnologii: Po mat-lam ezhegodnoy Mezhdunar. konf. ``Dialog''} 
[Computational Linguistics and Intellectual Technologies: Papers from the Annual Conference 
(International) ``Dialogue'']. Moscow: RGGU. 17(24):254--266. 

\bibitem{20-in-1} %19
Chernyshev, V.\,I., ed. 1950--1965. \textit{Slovar' sovremennogo russkogo literaturnogo yazyka} 
[Dictionary of modern Russian literary language]. In 17~vols. Moscow, Leningrad: USSR Academy 
of Sciences Publishing House/Nauka.

\bibitem{19-in-1} %20
Shvedova, N.\,Yu., ed. 1980. \textit{Russkaya grammatika} [Russian grammar]. Moscow: Nauka. Vol.~2. 714~p.

\bibitem{21-in-1} %21
Evgen'eva, A.\,P., ed. 1981--1984. \textit{Slovar' russkogo yazyka} [Dictionary of the Russian 
language].  Moscow: Russkiy yazyk. 4~vols.


\bibitem{22-in-1}
\Aue{Heine, B., and T.~Kuteva.} 2002. \textit{World lexicon of grammaticalization}. Cambridge: 
Cambridge University Press. 387~p.

\end{thebibliography}

 }
 }

\end{multicols}

\vspace*{-6pt}

\hfill{\small\textit{Received October 5, 2023}} 

%\vspace*{-18pt}

\Contr

\vspace*{-4pt}

\noindent
\textbf{Inkova Olga Yu.} (b.\ 1965)~--- Doctor of Science in philology, senior scientist, Federal 
Research Center ``Computer Science and Control'' of the Russian Academy of Sciences,  
44-2~Vavilov Str., Moscow 119333, Russian Federation; faculty member, University of Geneva, 
22~Bd des Philosophes, CH-1205 Geneva~4, Switzerland; \mbox{olyainkova@yandex.ru}

\vspace*{3pt}

\noindent
\textbf{Kruzhkov Mikhail G.} (b.\ 1975)~--- senior scientist, Federal Research Center ``Computer 
Science and Control'' of the Russian Academy of Sciences, 44-2~Vavilov Str., Moscow 119333, 
Russian Federation; \mbox{magnit75@yandex.ru}


\label{end\stat}

\renewcommand{\bibname}{\protect\rm Литература}  %12
\def\stat{kozerenko}

\def\tit{КОГНИТИВНО-ЛИНГВИСТИЧЕСКИЕ ПРЕДСТАВЛЕНИЯ 
В~СИСТЕМАХ ОБРАБОТКИ ТЕКСТОВ}

\def\titkol{Когнитивно-лингвистические представления 
в~системах обработки текстов}

\def\autkol{Е.\,Б.~Козеренко, И.\,П.~Кузнецов}
\def\aut{Е.\,Б.~Козеренко$^1$, И.\,П.~Кузнецов$^2$}

\titel{\tit}{\aut}{\autkol}{\titkol}

%{\renewcommand{\thefootnote}{\fnsymbol{footnote}}\footnotetext[1]
%{Работа выполнена при поддержке Российского фонда фундаментальных
%исследований, проект~10-01-00480. Статья написана на основе материалов доклада, 
%представленного на IV Международном семинаре <<Прикладные задачи теории вероятностей 
%и математической статистики, связанные с моделированием информационных систем>> 
%(зимняя сессия, Аоста, Италия, январь--февраль 2010 г.).}}

\renewcommand{\thefootnote}{\arabic{footnote}}
\footnotetext[1]{Институт проблем информатики Российской академии наук, kozerenko@mail.ru}
\footnotetext[2]{Институт проблем информатики Российской академии наук, igor-kuz@mtu-net.ru}


\Abst{Рассмотрены вопросы проектирования и развития 
семантико-синтаксических и лексико-семантических представлений в 
лингвистических процессорах ряда систем, основанных на аппарате расширенных 
семантических сетей (РСС). Системы этого класса создаются для извлечения знаний из 
текстов на естественных языках, отображения извлеченных сущностей и связей в 
структуры базы знаний (БЗ) и использования знаний для поддержки экспертных 
аналитических решений в различных сферах приложения. В~фокусе внимания 
находятся ин\-же\-нер\-но-линг\-ви\-сти\-че\-ские представления, позволяющие 
построить целостную работающую лингвистическую модель, которая 
модифицируется в зависимости от конкретной задачи: от <<тяжелой>> формы на 
основе детальных глубинных представлений до фокусных редуцированных 
оболочек, настроенных на узкую предметную область (ПО) и ограниченный язык 
общения. Особое внимание уделяется способам описания 
дис\-три\-бу\-тив\-но-транс\-фор\-ма\-ци\-он\-ных признаков языковых объектов.}

\KW{интеллектуальные системы; семантические представления; лингвистические 
процессоры; обработка естественного языка; извлечение знаний}

       \vskip 14pt plus 9pt minus 6pt

      \thispagestyle{headings}

      \begin{multicols}{2}

      \label{st\stat}

\section{Введение}

     Данная работа посвящена проблемам создания\linebreak 
     когни\-тив\-но-линг\-ви\-сти\-че\-ских моделей естественного языка для 
различных классов информационных систем и описанию опыта создания 
линг\-ви\-сти\-че\-ских представлений для интеллектуальных\linebreak технологий 
обработки текстов. Вопросы извлечения знаний из текстов и создания модели 
естественного языка рассматриваются в единстве. В центре внимания будут 
находиться лингвистические процессоры интеллектуальных систем, 
разработанных на основе аппарата \textit{расширенных семантических 
сетей}~[1--5]. %\cite{1koz}--\cite{3koz}, \cite{18koz}--\cite{19koz}. 
Будем 
называть их \textit{РСС-сис\-те\-мы}. Эти системы создавались коллективом 
разработчиков, включая авторов данной статьи в Институте проб\-лем 
информатики РАН на протяжении целого ряда лет в рамках 
исследовательских проектов и прикладных систем, ориентированных на 
конкретные ПО заказчиков. Можно выделить четыре 
поколения РСС-систем. Ко\-гни\-тив\-но-линг\-ви\-сти\-че\-ские 
представления, заложенные в основу систем этого класса, прошли 
определенный эволюционный путь. 
     
     Интеллектуальные РСС-сис\-те\-мы содержат развитые \textit{базы 
знаний}, при этом знания представлены в виде записей на языке 
РСС, называемых 
     \textit{РСС-струк\-ту\-ра\-ми}. Лингвистические знания, таким 
образом, являются частным случаем <<знаний>> и также представлены в 
виде записей на языке РСС. Основным 
конструктивным элементом РСС\linebreak является именованный $N$-мест\-ный 
предикат, на\-зы\-ва\-емый <<\textit{фрагментом}>>. Все множество языковых 
объектов задается в виде системы пре\-ди\-кат\-но-ак\-тант\-ных структур, при этом 
поддерживаются механизмы представления вложенных структур, что дает 
очень мощные изобразительные возможности для описания объектов 
различных языковых уровней. Очень важными факторами являются 
однородность и единообразие лингвистических представлений. 
     
     В процессе анализа и синтеза предложений естественного языка 
используется фор\-маль\-но-грам\-ма\-ти\-че\-ский аппарат, сходный с 
грамматиками зависимостей. При этом подходе опорными элементами 
служат слова и конструкции, выполняющие роль предикатов в предложении, 
и результатом анализа предложения должен стать один предикат, 
соответствующий сказуемому рассматриваемого предложения (т.\,е.\ 
основному глаголу в личной форме или другому основному предикатному 
выражению). Таким образом, в процессе анализа происходит выявление 
\textit{когнитивных опор} предложения: <<слов-дейст\-вий>> и 
     <<слов-от\-но\-ше\-ний>>, т.\,е.\ глаголов и других слов, имеющих 
синтактико-семантические валентности. Примером <<слов-от\-но\-ше\-ний>> 
могут служить, например, слова <<отец>>, <<друг>> и~т.\,п., т.\,е.\ в данном 
случае <<отношения>> (или \textit{функции}~--- в терминах языка логики 
предикатов 1-го порядка)~--- это слова, которые задают сильные, четко 
выраженные син\-так\-ти\-ко-се\-ман\-ти\-че\-ские ожидания. 
     
     Семантический анализ в ин\-же\-нер\-но-линг\-ви\-сти\-че\-ском 
понимании~--- это процесс перевода ес\-тест\-вен\-но-язы\-ко\-вых 
выражений во <<внутренние>> структуры БЗ, в 
рассматриваемой ситуации этими <<внутренними>> структурами являются 
записи на языке РСС. Таким образом, структуры БЗ~--- это код смысла в 
интеллектуальных информационных системах подобного рода. 
     
     В работе рассматриваются ин\-же\-нер\-но-линг\-ви\-сти\-че\-ские 
решения в системах с <<пол\-ным>> линг\-ви\-сти\-че\-ским анализом~--- это 
     сис\-те\-мы 1-го и 2-го поколения: ДИЕС1, ДИЕС2, 
     Логос-Д~\cite{2koz, 3koz}~--- и сис\-те\-мах с <<фактографическим>> 
подходом: интеллектуальных системах поддержки аналитических решений 
(ИСПАР)~\cite{18koz, 19koz}, где целью анализа является выделение 
сущностей и связей из текстов,~--- это системы 3-го и 4-го поколения. 

\section{Процесс концептуально-лингвистического моделирования 
в системах, основанных на аппарате расширенных семантических сетей}
     
\subsection{Центральные вопросы семантического моделирования} %2.1
     
     Концептуально-лингвистическое моделирование (КЛМ)~--- это 
процесс построения ес\-тест\-вен\-но-язы\-ко\-вой модели ПО (рис.~1), синтезирующий в себе подходы 
концептуального и лингвистического моделирования~[1--3]. 
По\-стро\-ение концептуально-лингвистической модели некоторой 
ПО подразделяется на следующие этапы:
     \begin{itemize}
     \item построение собственно концептуальной модели, т.\,е.\ вычленение 
базовых понятий, организация их в ро\-до-ви\-до\-вые деревья и определение 
связей между ними;
     \item разработка идеографического словаря ПО, т.\,е.\ 
лексическое наполнение концептуальной модели;
     \item ввод базовых правил, описывающих на естественном языке 
<<модель мира>>, релевантную данной ПО.
     \end{itemize}
     
     
     Методика КЛМ на 
основе аппарата РСС базируется на следующих принципах:
     \begin{itemize}
\item модель должна быть <<открытой>>, т.\,е.\ поддерживать эффективный 
механизм расширения и обновления информации;
\begin{center} %fig1
%\vspace*{3pt}
\hspace*{-10.7158pt}\mbox{%
\epsfxsize=77.871mm
\epsfbox{koz-1.eps}
}\hspace{10.7158pt}
%\end{center}
\vspace*{4pt}
%\begin{center}
{{\figurename~1}\ \ \small{Процесс КЛМ}}
\end{center}
\vspace*{3pt}

%\bigskip
\addtocounter{figure}{1}
\item модель представления <<смысла>> должна учитывать факты 
экстралингвистической реаль\-ности, которые в виде правил и отношений 
составляют некоторую базовую <<модель мира>>, достраиваемую 
конкретными моделями ПО;
\item модель должна быть практичной, т.\,е.\ не перегруженной детальными 
описаниями связей и отношений между понятиями, чтобы обеспечить 
возможность ее реализации, но в то же время отражать всю релевантную 
конкретной задаче информацию.
\end{itemize}

     \begin{figure*} %fig2
%     \begin{center}
\hspace*{23mm}\{(ВЫРАБАТЫВА895\_\_)(DICSEM)\\
\hspace*{23mm}COORD(PROGNOZ1,RUS,ВЫРАБАТЫВА895\_\_,S50\_31\_51\_20,\%)\\
\hspace*{23mm}SUB(UNIV,0+)~SUB(UNIV,1+)~SUB(UNIV,2+)\\
\hspace*{23mm}ВЫРАБАТЫВ(0-,1-,2-/3+)~INFI(3-)~ПРИДЕТСЯ(3-)~ПРИДЕТСЯ(3$-$/4+) \\
\hspace*{23mm}FUT1(4$-$)~SUB(СРЕД,5+)
%\end{center}
%\vspace*{2pt}
\Caption{Пример записи представления глагола <<вырабатывать>> в семантическом 
словаре
\label{f2koz}}
%\vspace*{6pt}
\end{figure*}

     Реалистичный подход к постановке задачи диктует необходимость 
ограничения моделируемого подмножества естественного языка. Суть 
ограничений сводится к следующему:
     \begin{enumerate}[(1)]
     \item анализируемые текстовые материалы содержат 
экспертные знания из конкретных ПО (в разработанных 
авторами системах это были такие ПО, как диагностика 
брака при изготовлении микросхем, социальное прогнозирование, 
криминалистика и другие);
     \item в целях максимально возможного устранения 
неоднозначности словарь строится по модульному принципу: есть некоторая 
наиболее общая часть (1--2~уровня), которая достраивается специальными 
словарями для каж\-дой отдельной~ПО.
     \end{enumerate}
     
     Предлагаемая модель лексической семантики основана на принципе 
<<ядерного>> значения, реализуемого в контексте данной 
ПО, с последующим индуктивным наращиванием других значений (если 
они актуализируются в рас\-смат\-ри\-ва\-емых контекстах). Также используется 
таксономия, которая реализуется в виде иерархических деревьев классов 
слов. 
     
     Общая <<модель мира>> системы является основой для моделей ПО. 
Элементами этой модели служат классы слов, которые подразделяются на 
понятия/имена, отношения, действия, свойства, характеристики действий, 
временные и пространственные характеристики.
     
     Самым общим понятием является \textit{концепт}, или 
\textit{универсальный класс}, который подразделяется на объект, ситуацию, 
процесс и~др. 
     
     Слова, относящиеся к классам действий и отношений, представлены 
как се\-ман\-ти\-ко-син\-так\-си\-че\-ские фреймы, задающие 
     пре\-ди\-кат\-но-ак\-тант\-ные структуры (модель управления). Однако 
в описываемом подходе (назовем его РСС-под\-хо\-дом) существенно 
расширена область значений актантов. Суть расширения состоит, во-первых, 
в том, что в роли актантов могут выступать не только простые объекты, 
соответствующие отдельным словам, но и структурные объекты, 
представляющие словосочетания и фразы, а во-вторых, в том, что понятие 
падежа включает в себя не только семантические, но и синтаксические 
признаки.
     
     Подход, основанный на РСС, позволяет отражать произвольный 
уровень вложенности структур за счет пропозициональных вершин 
семантической сети. Это обеспечивает представление\linebreak сложных 
синтаксических конструкций фраз\linebreak естественного языка, а также позволяет 
отразить\linebreak структурный характер лексической семантики,\linebreak которая в 
предлагаемой модели имеет иерар\-хи\-че\-ски-се\-те\-вую структуру. 
Линг\-ви\-сти\-че\-ские зна-\linebreak ния пред\-став\-ле\-ны в системном словаре и 
декла\-ра\-тивных модулях линг\-ви\-сти\-че\-ско\-го процессора.\linebreak В РСС-сис\-те\-мах 
так\-же реализована функция динамически форми\-ру\-емо\-го семантического 
словаря, который на основе исходной лингвистической информации 
достраивается системой автоматически в процессе об\-ра\-бот\-ки конкретных 
текстов. На рис.~\ref{f2koz} пред\-став\-ле\-но \mbox{такое} <<внутреннее>> описание 
глагола в семантическом словаре. Этот словарь автоматически генерируется 
РСС-системами ДИЕС2, ЛОГОС-Д, ИКС в процессе обработки 
     естест\-вен\-но-язы\-ко\-вых \mbox{текстов}. 
     {\looseness=1
     
     }
     
     
\subsection{Особенности применения аппарата расширенных семантических сетей 
в~когнитивно-лингвистическом моделировании} %2.2
     
     Дадим краткое описание аппарата РСС и  
обос\-ну\-ем выбор именно этого метода представления для моделирования 
естественного языка. Классическое понятие семантической сети сводится к 
следующему: задаются некоторые вершины, соответствующие объектам,  
вершины связываются дугами, которые помечаются именами отношений. 
Однако с помощью подобных сетей оказывается трудно представлять 
сложные виды информации, например, когда объекты, связанные 
отношениями, образуют агрегаты и когда отношения связываются между 
собой отношениями и~др. Поэтому в сети вводятся вершины, 
соответствующие именам отношений, а также специальный композиционный 
элемент, называемый вершиной связи. Вершина связи как бы <<разрывает>> 
дугу и подсоединяется одним ребром к вершине-отношению, а другими 
ребрами~--- к вершинам-объектам. Расширенная семантическая сеть является развитием такого сорта 
сетей в направлении повышения изобразительных возможностей при 
сохранении свойства однородности.
     
     Основой РСС является множество вершин ($V$), из которых 
составляются элементарные фрагменты (ЭФ) вида
     $
     V_0(V_1,V_2,\ldots ,V_k/V_{k+1})
     $, 
     где
$V_0, V_1, V_2,\ldots , V_k, V_{k+1}>0$.
     
     
     Такой фрагмент представляет $k$-местное отношение. Позиции 
вершин в ЭФ определяют их роли. 
Вершина~$V_0$ ставится в соответствие имени отношения, 
вершины~$V_1$, $V_2$, \ldots , $V_k$~--- объектам, участ\-ву\-ющим в 
отношении, а вершина~$V_{k+1}$, отделенная косой линией,~--- всей 
совокупности упомянутых объектов с учетом их отношения. В~дальнейшем 
будем $V_{k+1}$ называть $C$-вершиной ЭФ.\linebreak 
Множество ЭФ образует РСС. 
С~помощью РСС представляются наборы отношений, различные ситуации, 
сце\-нарии. Сильной стороной РСС-под\-хо\-да является возможность 
однородного пред\-став\-ле\-ния как предметной (концептуальной), так и 
лингвистической информации, что обеспечивает эффективную обработку 
знаний и поддержание непротиворечи\-вости~БЗ.
          \begin{figure*} %fig3
     \vspace*{1pt}
\begin{center}
\mbox{%
\epsfxsize=125.039mm
\epsfbox{koz-3.eps}
}
\end{center}
\vspace*{-9pt}
     \Caption{Семантико-синтаксический анализ без выявления глагольных 
словоформ
      \label{f3koz}}
\vspace*{12pt}
 %     \end{figure*}
%            \begin{figure*} %fig4
           \vspace*{1pt}
\begin{center}
\mbox{%
\epsfxsize=103.129mm
\epsfbox{koz-4.eps}
}
\end{center}
\vspace*{-9pt}
      \Caption{Целостная семантическая структура предложения
      \label{f4koz}}
      \end{figure*}

     
     Посредством РСС в БЗ представлены лингвистические  и 
предметные знания. Обработка этих знаний осуществляется 
продукциями языка ДЕКЛ, на котором реализованы сле\-ду\-ющие шесть 
блоков: морфологического анализа, семанти\-ческого анализа слов, 
син\-так\-ти\-ко-се\-ман\-ти\-че\-ско\-го анализа форм, 
прагматических функций, организации системной активности и 
обратный лингвистический процессор. С~помощью продукций 
осущест\-вля\-ет\-ся последовательное преобразование сети~--- РСС. При этом 
проходятся фазы, соответствующие уровню понимания входного текста. 
Рас\-смот\-рим~их.
     \begin{enumerate}[1.]
     \item На первом шаге анализа строится 
пространственная структура предложения с морфологической информацией 
для каждого слова.\linebreak Каж\-дый член предложения представляется вершиной 
семантической сети. Вместо слова генерируется код (если слово 
многозначно, т.\,е.\ принадлежит к нескольким классам,~--- то более одного 
кода). Основой кода служит корень слова. На этом этапе предложение 
представляется в виде набора фрагментов типа LRR (специальных меток 
результатов 1-го этапа анализа), объединяемых в целостную структуру 
посредством вершины связи. Результат 1-го этапа постоянно обращается к 
словарю: <<Что значит данное слово?>>
     \item На втором этапе каждой вершине сопоставляется семантический 
класс и присваивается новый код. За словами (т.\,е.\ конкретными вершинами 
РСС) система видит объекты, действия, свойства, т.\,е.\ строит 
классификации. Производится се\-ман\-ти\-ко-син\-так\-си\-че\-ский анализ 
без выявления глагольных словоформ, при этом предложение представляется 
в виде совокупности фрагментов типа SEM и SEMD~--- специальных меток 
результатов 2-го этапа анализа (рис.~\ref{f3koz}).
     \item На третьем этапе происходит частичное <<сворачивание>> 
синтаксических структур в более компактные (например, свойство объекта и 
сам объект) с присваиванием нового кода и строится фрагмент для объекта, 
обладающего этим свойством.
     \begin{figure*}[b] %fig5
          \vspace*{12pt}
\begin{center}
\mbox{%
\epsfxsize=147.485mm
\epsfbox{koz-5.eps}
}
\end{center}
\vspace*{-9pt}
     \Caption{Глубинная структура предложений
      \label{f5koz}}
      \end{figure*}      
     \item На четвертом этапе выявляются отношения и действия и 
производится анализ непосредственного контекста на соответствие заданным 
семантическим падежам. Система проверяет, подходят ли объекты 
(концепты, понятия) на аргументные места данного действия или отношения. 
При этом отглагольные существительные (<<делатель>>, т.\,е.\ агент 
действия, или <<делание>>~--- процесс~--- анализируются как слова с 
двойной природой: вначале как действия, а затем как объекты). Результатом 
этого этапа является целостная семантическая структура предложения, 
которая представляется фрагментом типа SEMSTR~--- метки результата 4-го 
этапа анализа (рис.~\ref{f4koz}).
     \item На пятом этапе происходит анализ прагматики: установление 
кореференциальных отношений, частичное восстановление эллиптических 
конструкций, система производит дальнейшие действия с построенными 
фрагментами.
     \end{enumerate}

     
Система ДИЕС допускает ввод полисемичных форм глаголов. Для этого следует 
воспользоваться формальной записью лингвистических знаний. 
     В~сис\-те\-мах, основанных на РСС, все функции реализованы на 
единой основе~--- в рамках языков РСС и ДЕКЛ, которые были разработаны 
с ориентацией на задачи обработки естественного языка.

%\vspace*{-6pt}

\section{Представление семантики глаголов, глубинные 
и~поверхностные структуры}
     
     В процессе анализа выявляются семантические вершины предложения: 
происходит выявление <<слов-дей\-ст\-вий>>, т.\,е.\ глаголов, и 
     <<слов-от\-но\-ше\-ний>>. Что же является конструктивной основой\linebreak 
задания семантических представлений предикатных слов и выражений? Как 
убедительно показано в работе~\cite{4koz}, семантика глагола 
определяется его дис\-три\-бу\-тив\-но-транс\-фор\-ма\-ци\-он\-ны\-ми\linebreak 
свойствами. Поэтому смысл предикатных выражений должен кодироваться с 
учетом их дистрибутивных и трансформационных признаков. 
     
     Выдвинутая рядом лингвистов (Хомский, Филлмор) гипотеза о том, что 
все предложения имеют глубинные и поверхностные 
     структуры~[7--10], явилась очень продуктивным 
источником проектных решений при создании первых РСС-сис\-тем и 
развивалась в дальнейшем. 

В~тео\-ре\-ти\-ко-линг\-ви\-сти\-че\-ском 
понимании глубинная структура~--- это абстракция, содержащая все 
элементы, необходимые для образования поверхностных структур 
предложений со сходной семантикой. 

     В~ин\-же\-нер\-но-линг\-ви\-сти\-че\-ском понимании\linebreak глубинная 
структура~--- это запись на языке БЗ, например на языке РСС, 
которая может быть представлена в <<поверхностном>> виде на одном из 
естественных языков в результате конечного числа определенных 
преобразований. Например, предложения

\noindent
\begin{align*}    
(1)\ &\mbox{\textit{The programmer writes the code}}\\
(2)\ &\mbox{\textit{The code is written by the programmer}}
\end{align*}
имеют истоком одну глубинную структуру:

\medskip

\noindent
     \begin{verbatim}
  Programmer <---- write ----> Code
      agent                   object,
\end{verbatim}

\medskip

\noindent
хотя и отличаются своими поверхностными структурами. В~каждом из них 
имеется агент (the programmer), объект (the code) и действие (write).\linebreak Согласно 
концепции \textit{падежной грамматики} Филлмора~\cite{5koz} глубинная 
структура для обоих предложений инвариантна. Эту структуру можно 
представить в виде скобочной записи $V(\mathrm{AGENT}, \mathrm{OBJECT})$. В~графическом 
виде глубинная структура предложения также может быть представлена 
диаграммой в виде дерева, где отражены инвариантные отношения 
зависимости между предикатной вершиной и актантами (рис.~\ref{f5koz}), 
причем в таком представлении явным образом разграничиваются 
\textit{модальность} (MOD) и \textit{пропозиция} (PROP).
     

     В исходном варианте~\cite{5koz} теория признавала шесть падежей: 
агентив, инструменталис, датив, объектив, локатив и фактитив. По мере 
развития теории~\cite{8koz} происходило увеличение числа падежей, однако 
<<умножение>> количества падежей утяжеляет первоначальную 
конфигурацию, поэтому при построении инженерных семантических 
представлений требуется некоторый <<компромиссный>> вариант, 
сочетающий в себе необходимую полноту, с одной стороны, и простоту и 
гибкость, с другой.

\begin{figure*}[b] %fig6
\vspace*{24pt}
\begin{center}
\mbox{%
\epsfxsize=156.873mm
\epsfbox{koz-6.eps}
}
\end{center}
%\vspace*{-9pt}
\Caption{Обобщенное функциональное представление систем ИСПАР
\label{f6koz}}
\end{figure*}
     
%\vspace*{-6pt}

\section{Некоторые базовые аспекты построения многоязычных 
систем}
     
     Одним из приоритетных направлений развития РСС-сис\-тем является 
обеспечение обработки текстов на нескольких языках, прежде всего для 
рус\-ско-анг\-лий\-ской языковой пары. В системах 2-го поколения~--- ДИЕС2, 
ИКС, ЛОГОС-Д были реализованы лингвистические процессоры и словари 
для русского и английского языка, позволявшие обрабатывать тексты для 
ряда ПО. При этом поддерживался как режим ввода 
лингвистических знаний линг\-вис\-том-ана\-ли\-ти\-ком, так и 
автоматический режим самообучения системы по вводимым \mbox{текстам}. 
{\looseness=1

}

Проводились также эксперименты с итальянским и французским языком. 
При создании многоязычных систем авторы обращались к европейским 
языкам. Очевидно, что европейские языки обладают большим числом общих 
правил, чем любой из них с языками других групп. Но при этом все 
естественные языки обладают общей структурой на самом глубинном 
уровне. На этом уровне располагаются главные элементы естественного 
языка: \textit{предложение}, \textit{модальность}, \textit{пропозиция}.
     
     Моделирование смысловых представлений~--- это процесс, 
развивающийся в направлении от поверхностных семантических структур к 
глубинным. Поиск такого внутреннего представления смысла в условиях 
многоязычной ситуации является на\-прав\-ле\-ни\-ем развития методов 
     КЛМ на базе  РСС. 
     
%     \vspace*{-48pt}
     
\section{Интеллектуальные системы поддержки аналитических 
решений}
     
Системы РСС 3-го и 4-го поколения на\-прав\-ле\-ны на извлечение знаний 
в виде \textit{объектов}, или \textit{сущностей}, и связей между ними из 
пред\-мет\-но-ориен\-ти\-ро\-ван\-ных текстов на русском и английском 
языке~\cite{18koz, 19koz}.

    
В настоящее время во всем мире активно ведутся работы по созданию 
систем извлечения фактов из текстов на естественных языках~[11--14], создаются развитые тезаурусы и 
онтологии~\cite{17koz}. Сис\-те\-мы РСС функционально шире, поскольку 
имеют возможность не только извлекать факты, но и поддерживать 
механизмы логического анализа и экспертного вывода на основе 
извлеченных знаний. Сис\-те\-ма\-ми такого рода являются ИСПАР. В~целом это 
направление исследований требует дальнейшей проработки 
     лек\-си\-ко-се\-ман\-ти\-че\-ских представлений, создания 
     пред\-мет\-но-ориен\-ти\-ро\-ван\-ных семантических словарей. 

Обобщенное функциональное представление систем ИСПАР дано на 
рис.~\ref{f6koz}. 
     
     В рамках ИСПАР на основе РСС 
(\mbox{ИСПАР}--РСС) были реализованы полномасштабные и\linebreak пилотные 
проекты для ряда ПО: криминалистики, управления 
кадрами, мониторинга финансово-экономического кризиса и 
др.~\cite{18koz, 19koz}.

\section{Применение аппарата расширенных семантических сетей в~лингвистических 
исследованиях}
     
     В настоящее время в рамках проектов, на\-прав\-лен\-ных на создание 
открытых лингвистических ресурсов~\cite{20koz} для 
     на\-уч\-но-прак\-ти\-че\-ских целей, ведутся работы по выравниванию 
параллельных текстов научных статей, патентов и 
     фи\-нан\-со\-во-эко\-но\-ми\-че\-ских текстов. В~качестве одного из 
методов выравнивания используется РСС-под\-ход, поскольку он позволяет 
отразить глу\-бин\-но-се\-ман\-ти\-че\-ский уровень языковых структур. 

На  рис.~7 представлен фрагмент первого этапа лингвистического 
анализа в многоязычных системах. Для <<идеальной>> ситуации, когда 
структуры исходного текста и текста перевода практически совпадают, такая 
ситуация имеет место в меньшинстве случаев. Основные трудности 
возникают при наличии переводческих трансформаций в параллельных 
текстах. Особое внимание следует уделять гла\-голь\-но-имен\-ным 
трансформациям, например явлению \textit{номинализации}, поскольку она 
очень продуктивна для всех исследовавшихся языков.

     
     Ключевой задачей при разработке методов сопоставления 
параллельных текстов является выявление и детальное описание тех 
языковых трансформаций, которые имеют место при переводе 
     естест\-вен\-но-язы\-ко\-вых конструкций с одного языка на 
другой~\cite{9koz}, потому что далеко не всегда некое содержание 
передается струк\-тур\-но-по\-доб\-ны\-ми средствами в текстах на разных 
языках. Сравнительное исследование употребления различных частей речи в 
параллельных текстах на разных языках создает основу для выявления и 
описания языковых транс-\linebreak

\begin{center} %fig7
\vspace*{3pt}
\mbox{%
\epsfxsize=79.726mm
\epsfbox{koz-7.eps}
}
\end{center}
\vspace*{4pt}
%\begin{center}
{{\figurename~7}\ \ \small{Первый этап анализа параллельных текстов ($W_n$
обозначает словоформу с номером~$n$, $1\leq n\geq 5$)}}
%\end{center}
%\vspace*{9pt}

%\bigskip
\addtocounter{figure}{1}
      

\noindent 
формаций, при этом центральной трансформацией
является \textit{номинализация}. Явление номинализации
было исследовано в 
ряде работ отечественных и зарубежных лингвистов~[17--20]. 
Ближе всего к правильному, по мнению авторов данной статьи, 
пониманию этого явления следующие определения номинализации: 
<<конструкции\ldots называются номинализованными~--- в том смысле, что 
их естественно рассматривать как результат номинализации конструкций с 
предикативным употреблением глаголов и прилагательных>>; 
<<номинализация~--- это синтаксический процесс, который соотносит 
предложения с именными группами>>~\cite{9koz, 10koz}. Выявление 
номинализованных конструкций в параллельных научных и патентных 
текстах на русском, английском, французском и немецком языках в научных 
и патентных текстах и сопоставительное описание гла\-голь\-но-имен\-ных 
межъязыковых трансформаций~--- одна из центральных задач 
     ин\-же\-нер\-но-линг\-ви\-сти\-че\-ских исследований. 
     
     Следующей базовой трансформацией в исследуемых текстах на 
нескольких европейских языках является адъек\-тив\-но-ад\-вер\-би\-аль\-ное 
преобразование. Это означает, что при переводе с одного языка на другой 
происходит синтаксическое преобразование имен прилагательных в наречия 
и обратное преобразование~--- наречий в прилагательные. Установление 
семантических соответствий между этими языковыми объектами также 
возможно осуществить посредством аппарата~РСС. 
     
     При семантическом выравнивании непараллельных текстов, имеющих 
одну и ту же денотативную составляющую, аппарат РСС позволяет выявить в 
текстах когнитивные опоры (слова с сильной валентностью~--- 
     <<сло\-ва-дейст\-вия>> и <<сло\-ва-от\-но\-ше\-ния>>) и установить 
между ними семантические соответствия.

\section{Заключение}

     В данной работе представлен опыт создания и развития 
     когни\-тив\-но-линг\-ви\-сти\-че\-ских пред\-став\-ле\-ний в 
интеллектуальных информационных сис\-те\-мах, разработанных на основе 
аппарата РСС. Аппарат РСС 
обеспечивает мощные изобразительные возможности для описания всех 
уровней естественного языка, включая уровень 
     глу\-бин\-но-се\-ман\-ти\-че\-ских представлений и межъязыковых 
соответствий. Конкретные лингвистические процессоры, которые были 
созданы на основе этого подхода, прошли определенный путь развития и 
позволили выработать проектные решения для основных задач текущего 
этапа~--- извлечения и обработки содержательных знаний из текстов на 
естественных языках и сопоставления языковых структур в текстах на 
различных языках с учетом базовых трансформаций.
     
     Проблема извлечения и обработки знаний открывает перспективы 
развития интеллектуальных направлений компьютерной лингвистики, 
поскольку ее основной акцент смещен в сторону\linebreak глубинных представлений 
языка, в которых используются как грамматические (морфологические и 
синтаксические), так и семантические атрибуты для описания языковых 
объектов. Проводи-\linebreak мые авторами исследования параллельных текстов 
направлены также на рассмотрение этой проблемы~\cite{20koz}. 
Центральное место в проводящихся линг\-ви\-сти\-че\-ских исследованиях 
занимает изучение и формализация процессов трансформации языковых 
структур, особенно все варианты глагольно-но\-ми\-на\-тив\-ных трансформаций, 
создание развитых дис\-три\-бу\-тив\-но-транс\-фор\-ма\-ци\-он\-ных 
описаний предикатых структур для рассматриваемых языков. 
     
     Для задач извлечения знаний и создания \mbox{ИСПАР} 
     дис\-три\-бу\-тив\-но-транс\-фор\-ма\-ци\-он\-ные описания имеют 
особое значение, поскольку таким образом задаются все возможные способы 
перевода языковых структур в пре\-ди\-кат\-но-ар\-гу\-мент\-ные 
представления, которые затем используются в процедурах обработки знаний.

{\small\frenchspacing
{%\baselineskip=10.8pt
%\addcontentsline{toc}{section}{Литература}
\begin{thebibliography}{99}

     \bibitem{1koz}
     \Au{Кузнецов~И.\,П.}
     Семантические представления.~--- М.: Наука, 1986. 290~с.
     
     \bibitem{2koz}
     \Au{Козеренко~Е.\,Б.}
     Кон\-цеп\-ту\-аль\-но-линг\-вис\-ти\-че\-ское моделирование в среде 
интеллектуального редактора знаний ИКС~// Проблемы проектирования и 
использования баз знаний.~--- Киев: Ин-т кибернетики им.\ В.\,М.~Глушкова, 
1992. C.~73--79.
     
     \bibitem{3koz}
     \Au{Kozerenko~E.\,B.}
     Multilingual processors: A unified approach to semantic and syntactic 
knowledge presentation~// Conference (International ) on Artificial Intelligence 
IC-AI'2001 Proceedings. Las Vegas, Nevada, USA. June 25--28, 2001.~--- Las 
Vegas: CSREA Press, 2001. P.~1277--1282.

     \bibitem{18koz} %4
     \Au{Kuznetsov~I.\,P., Efimov~D.\,A., Kozerenko~E.\,B.}
     Tools for tuning the semantic processor to application areas~// ICAI'09 
Proceedings, WORLDCOMP'09. July 13--16, 2009. Las Vegas, Nevada, USA. 
Vol.~I.~--- Las Vegas: CRSEA Press, 2009. P.~467--472.
     
     \bibitem{19koz} %5
     \Au{Kuznetsov~I.\,P., Kozerenko~E.\,B., Kuznetsov~K.\,I., 
Timonina~N.\,O.}
     Intelligent system for entities extraction (ISEE) from natural language 
texts~// Workshop (International) on Conceptual Structures for Extracting Natural 
Language Semantics (Sense'09) at the 17th Conference 
(International ) on Conceptual Structures (ICCS'09) Proceedings. University Higher School of 
Economics. Moscow, Russia, 2009. P.~17--25.
     
     \bibitem{4koz} %6
     \Au{Апресян~Ю.\,Д.}
     Экспериментальное исследование семантики русского глагола.~--- М.: 
Наука, 1967.  252~с.
     
     \bibitem{5koz} %7
     \Au{Филлмор~Ч.}
     Дело о падеже~// Новое в зарубежной линг\-вистике, 1968. Вып.~X. С.~369--495.
     
     \bibitem{6koz} %8
     \Au{Хомский~Н.}
     Аспекты теории синтаксиса.~--- М.: МГУ, 1972.
     
     \bibitem{7koz} %9
     \Au{Хомский Н.}
     Язык и мышление.~--- М.: МГУ, 1972.
     
     
     \bibitem{8koz} %10
     \Au{Fillmore~C.}
     The case for case reopened~// Syntax and Semantics. Vol.~8.~--- N.Y.: 
Academic Press, 1977. 
     

          \bibitem{15koz} %11
     FASTUS: A cascaded finite-state trasducer for extracting information from 
natural-language text~// AIC, SRI International, Menlo Park, California, 1996. 
     
     \bibitem{16koz} %12
     \Au{Han~J., Pei~Y., Mao~R.}
     Mining frequent patterns without candidate generation: A frequent-pattern 
tree approach~// Data Mining and Knowledge Discovery, 2004. Vol.~8. No.\,1. 
P.~53--87.
     
     
     \bibitem{13koz} %13
     \Au{Cunningham~H.}
     Automatic information extraction~// Encyclopedia of Language and 
Linguistics. 2nd ed.~--- Elsevier, 2005.
     
     \bibitem{14koz} %14
     \Au{Han~J., Kamber~M.}
     Data mining: Concepts and techniques.~--- Morgan Kaufmann, 2006.
     
     
     \bibitem{17koz} %15
     \Au{Добров~Б.\,В., Лукашевич~Н.\,В.}
     Онтологии для автоматической обработки текстов: Описание понятий 
и лексических значений~// Компьютерная лингвистика и интеллектуальные 
технологии: Тр. межд. конф. <<Диалог'06>>. Бекасово, 31~мая\,--\,4~июня 
2006. С.~138--142.

     \bibitem{20koz} %16
     \Au{Kozerenko~E.\,B.}
     INTERTEXT: A multilingual knowledge base for machine translation~// 
Conference (International) on Machine Learning, Models, Technologies and 
Applications Proceedings. June 25--28, 2007. Las Vegas, USA.~--- Las Vegas: 
CSREA Press, 2007. P.~238--243.

     \bibitem{9koz} %17
     \Au{Жолковский~А.\,К., Мельчук~И.\,А.}
     О семантическом синтезе~// Проблемы кибернетики, 1967. Вып.~19.
     
         
     \bibitem{11koz} %18
     \Au{Jacobs~R.\,A., Rosenbaum P.\,S.}
     English transformational grammar.~--- Blaisdell, 1968.
     

\label{end\stat}
     
          \bibitem{12koz} %19
     \Au{Балли~Ш.}
     Общая лингвистика и вопросы французского языка. 2-е изд.~--- М.: 
УРСС, 2001.

\bibitem{10koz} %20
     \Au{Падучева~Е.\,В.}
     О~семантике синтаксиса: Мат-лы к трансформационной 
грамматике русского языка. 2-е изд.~--- М: КомКнига, 2007.  296~с. 
     
 \end{thebibliography}
}
}


\end{multicols} %13
\def\stat{bosov+stef}

\def\tit{УПРАВЛЕНИЕ ВЫХОДОМ СТОХАСТИЧЕСКОЙ ДИФФЕРЕНЦИАЛЬНОЙ СИСТЕМЫ 
ПО~КВАДРАТИЧНОМУ КРИТЕРИЮ. I.~ОПТИМАЛЬНОЕ РЕШЕНИЕ МЕТОДОМ 
ДИНАМИЧЕСКОГО ПРОГРАММИРОВАНИЯ$^*$}

\def\titkol{Управление выходом стохастической дифференциальной системы 
по~квадратичному критерию. I}
%.~Оптимальное решение методом 
%динамического программирования}

\def\aut{А.\,В.~Босов$^1$, А.\,И.~Стефанович$^2$}

\def\autkol{А.\,В.~Босов, А.\,И.~Стефанович}

\titel{\tit}{\aut}{\autkol}{\titkol}

\index{Босов А.\,В.}
\index{Стефанович А.\,И.}
\index{Bosov A.\,V.}
\index{Stefanovich A.\,I.}




{\renewcommand{\thefootnote}{\fnsymbol{footnote}} \footnotetext[1]
{Работа выполнена при частичной поддержке РФФИ (проект 16-07-00677).}}


\renewcommand{\thefootnote}{\arabic{footnote}}
\footnotetext[1]{Институт проблем информатики Федерального исследовательского центра <<Информатика 
и~управление>> Российской академии наук, \mbox{AVBosov@ipiran.ru}}
\footnotetext[2]{Институт проблем информатики Федерального исследовательского центра <<Информатика 
и~управление>> Российской академии наук, \mbox{AStefanovich@frccsc.ru}}

%\vspace*{8pt}



  
  \Abst{Решается задача оптимального управления для диффузионного процесса 
Ито и~линейного управ\-ля\-емо\-го выхода. Рассматриваемая постановка близка 
к~классической ли\-ней\-но-квад\-ра\-тич\-ной гауссовской задаче управления 
(linear-quadratic Gaussian (LQG) control). Отличия состоят в~том, что состояние описывается нелинейным 
дифференциальным уравнение Ито $dy_t\hm= A_t(y_t) \,dt\hm+ \Sigma_t(y_t)\,dv_t$ 
и~не зависит от управ\-ле\-ния~$u_t$, оптимизации подлежит управ\-ля\-емый 
линейный выход $dz_t\hm= a_t y_t\,dt\hm+ b_t z_t \,dt\hm+ c_t u_t \,dt\hm+ \sigma_t\, 
dw_t$. Дополнительные обобщения внесены в~квад\-ра\-тич\-ный критерий качества 
с~целью воз\-мож\-ности постановки таких задач, как отслеживание выходом 
состояния или управ\-ле\-ни\-ем~--- линейной комбинации состояния и~выхода. Для 
решения используется метод динамического программирования. Функцию 
Беллмана позволяет найти предположение о~ее структуре вида $V_t(y,z)\hm= 
\alpha_t z^2\hm+ \beta_t(y)z \hm+\gamma_t(y)$. Решение дают три 
дифференциальных уравнения для коэффициентов~$\alpha_t$, $\beta_t(y)$ 
и~$\gamma_t(y)$. Эти уравнения со\-став\-ля\-ют оптимальное решение 
рас\-смат\-ри\-ва\-емой задачи.}
  
  \KW{стохастическое дифференциальное уравнение; оптимальное управ\-ле\-ние; 
динамическое программирование; функция Беллмана; уравнение Риккати; 
линейные уравнения параболического типа}

\DOI{10.14357/19922264180314}
  
%\vspace*{4pt}


\vskip 10pt plus 9pt minus 6pt

\thispagestyle{headings}

\begin{multicols}{2}

\label{st\stat}

\section{Введение}

     Ключевые результаты в~области оптимизации стохастических 
динамических систем, со\-став\-ля\-ющие классическую теорию управления, 
получены более~40~лет назад (такова работа~[1] в~отношении задачи 
управ\-ле\-ния ли\-ней\-но-гаус\-сов\-ски\-ми стохастическими сис\-те\-ма\-ми по 
квад\-ра\-тич\-но\-му критерию). К~классической тео\-рии следует относить 
линейные модели стохастических сис\-тем и~квадратичный критерий качества. 
Это исходный базис, на котором основано множество успешно 
исследованных и~решенных задач стохастического управ\-ле\-ния 
и~оптимизации. 

Дальнейшее развитие~--- это новые модели и~критерии, но 
прежде всего это новые методы: от тео\-рии линейных регуляторов, метода 
динамического программирования и~принципа максимума к~адаптивному 
и~минимаксному подходу, импульсному управ\-ле\-нию и~т.\,д. Множество 
инноваций как в~час\-ти моделей, так и~в~час\-ти математического аппарата, 
имевших мес\-то в~по\-сле\-ду\-ющие годы, существенно обогатили тео\-рию 
управ\-ле\-ния. Но и~до настоящего времени линейные модели и~квадратичный 
критерий, несмотря на всю справедливую критику в~отношении их 
аде\-кват\-ности и~гиб\-кости, сохраняют исследовательский интерес и~находят 
современные области приложения.
     
     Не претендуя на сколь\-ко-ни\-будь полное обосно\-ва\-ние последнего 
тезиса, приведем несколько примеров, показавшихся наиболее ин\-те\-рес\-ными. 

Так, в~[2] решается ли\-ней\-но-квад\-ра\-тич\-ная за\-да\-ча в~игровой 
постановке с~запаздыванием. В~близ\-кой по модели работе~[3] задача 
управ\-ле\-ния ставится в~терминах $H_\infty$-ро\-баст\-ности. Точнее \mbox{называть} 
эту тематику $H_2/H_\infty$-управ\-ле\-ни\-ем, и~работ по этой теме очень 
много. Аккуратности ради следует уточнить, что под линейными 
понимаются модели с~мультипликативными по состоянию воз\-му\-ще\-ниями. 

Совсем другой класс моделей, особо популярных в~по\-след\-ние годы, 
составляют скачкообразные процессы. Например, линейные уравнения 
в~сочетании с~пуассоновскими скачками в~[4] используются в~моделях, 
описывающих различные показатели функционирования сетевых протоколов 
передачи данных транспортного уровня. Телекоммуникации представляют 
в~последние годы самый популярный прикладной материал для 
исследований, работ по этой проб\-ле\-ма\-ти\-ке множество, математические 
техники привлекаются самые разные и~самые современные, но и~линейным 
моделям место находится. Еще один любопытный пример исследования 
скачкообразного процесса и~оптимизации на основе квад\-ра\-тич\-но\-го критерия 
можно найти в~[5] применительно к~задаче инвестирования на финансовом 
рынке. Наконец, упомянем еще работу~[6], подводящую итог исследований 
в~отношении классической детерминированной  
ли\-ней\-но-квад\-ра\-тич\-ной задачи с~использованием техники матричных 
неравенств.
     
     В данной работе также эксплуатируются привлекательные свойства 
линейных моделей и~квад\-ра\-тич\-но\-го критерия, причем в~стохастической 
постановке. На\-прав\-ле\-ни\-ем для обобщения \mbox{выбрана} модель динамики 
сис\-те\-мы: основные усилия на\-прав\-ле\-ны на то, чтобы сделать ее нелинейной. 
Кроме того, пред\-став\-лен\-ная постановка может рас\-смат\-ри\-вать\-ся и~как 
обобщение ранее решенной задачи в~дискретном времени~[7, 8] на время 
непрерывное. В~упомянутых работах помимо собственно модельной 
постановки важна еще и~привлекаемая прикладная об\-ласть~--- 
функционирование сложных программных сис\-тем. Результатов, 
ориентированных непосредственно на такие приложения, к~настоящему 
времени пренебрежимо мало, поэтому~[7, 8]~--- это еще и~прикладное 
обоснование рас\-смат\-ри\-ва\-емой далее задачи.
     
     Оптимизируемая динамическая сис\-те\-ма описывается двумя 
уравнениями. Состояние задается нелинейным стохастическим 
дифференциальным уравнением Ито, не содержащим управ\-ля\-емой 
переменной. Возмущение здесь описывается стандартным винеровским 
процессом, накладываются простые условия существования 
и~един\-ст\-вен\-ности решения. Поскольку состояние не управ\-ля\-ет\-ся, то уместно 
его интерпретировать как слож\-ное внешнее возмущение. Вторая 
переменная~--- управ\-ля\-емый выход~--- задается линейным стохастическим 
дифференциальным уравнением. Цель оптимизации выхода формируется 
квадратичным критерием общего вида. Формальная постановка задачи 
приведена в~сле\-ду\-ющем разделе.
     
     Для решения задачи используется метод динамического 
программирования, решается уравнение Беллмана~[9]. Соответственно, 
в~результате получаются аналитические выражения и~для оптимального 
управ\-ле\-ния, и~для значения функционала качества. Технически 
традиционный, стандартный подход к~задаче обременен, пожалуй, 
единственной проблемой~--- поиском верного пред\-став\-ле\-ния структуры 
функции Беллмана. Справиться с~этой проблемой в~большей степени удается 
за счет результата, полученного при решении дискретного по времени 
аналога рассматриваемой постановки~\cite{8-bos}. Конечные соотношения 
для оптимального решения, как и~во всех подобных задачах, включая 
классическую ли\-ней\-но-квад\-ра\-тич\-ную, содержат решения 
определенных дифференциальных уравнений (обыкновенных и~в~частных 
производных). Вывод этих уравнений и~со\-став\-ля\-ет содержание первой час\-ти 
данной работы. Во второй части будет обсуждаться их приближенное 
чис\-лен\-ное решение и~компьютерные эксперименты.
     
     Кратко обозначим основные положения, при\-вле\-ка\-емые далее 
к~решению задачи, следуя в~основном обозначениям 
и~терминологии~\cite{9-bos}, а~именно: будем рассматривать задачу 
оптимального управления в~стохастической динамической сис\-те\-ме по полной 
информации, применяя метод динамического программирования. В~качестве 
целевого функционала, опре\-де\-ля\-юще\-го качество управ\-ле\-ния $U_0^T\hm= \{ 
u_t,\ 0\leq t\leq T\}$, выступает
     \begin{equation}
     J\left(U_0^T\right)={\sf E}\left\{ \int\limits_0^T L_t \left(x_t, u_t\right)\,dt+ 
l\left(x_T\right)\right\}\,.
     \label{e1-bos}
     \end{equation}
Здесь ${\sf E}\{\cdot\}$~--- оператор математического ожидания; $x_t$~--- 
случайный процесс, описываемый стохастическим дифференциальным 
уравнением Ито
     \begin{equation}
     dx_t=m_t\left( x_t, u_t\right) dt+ \sigma_t\left( x_t\right)dW_t\,,\enskip 
x_0=X\,,
     \label{e2-bos}
     \end{equation}
где $W_t$~--- стандартный винеровский процесс подходящей раз\-мер\-ности; 
$X$~--- случайный вектор.

     $U_0^T$ будем выбирать из класса допустимых неупреждающих (по 
отношению к~$W_t$) управлений~\cite{9-bos}. Соответственно, 
относительно функций сноса и~диффузии~$m_t$ и~$\sigma_t$  
в~(\ref{e2-bos}) будем предполагать выполненными ка\-кие-ли\-бо условия 
существования сильного решения для заданного до\-пус\-ти\-мо\-го управ\-ле\-ния. 
Например, для управ\-ле\-ния с~обратной связью $u_t\hm= u_t(x_t)$ будем 
считать, что $m_t(x,u_t(x))$ и~$\sigma_t(x)$ удовлетворяют условию 
линейного рос\-та и~локальному условию Липшица по~$x$ равномерно 
по~$t$ (т.\,е.\ условиям Ито).
     
     Для поиска оптимального управления, минимизирующего $J(U_0^T)$, 
рас\-смат\-ри\-ва\-ет\-ся функция Беллмана
     \begin{equation}
     V_t(x)=\left.\mathop{\mathrm{inf}}\limits_{U_t^T} {\sf E} \left\{ \int\limits_t^T 
L_t \left( x_t, u_t\right)\,dt+l\left( x_T\right) \right\vert \mathcal{F}_t^x\right\}\,,
     \label{e3-bos}
     \end{equation}
где $\mathcal{F}_t^x$~--- $\sigma$-ал\-геб\-ра, по\-рож\-ден\-ная~$x_\tau$, 
$0\hm\leq \tau\hm\leq t$, ${\sf E}\{\cdot\vert \mathcal{F}\}$~--- оператор условного 
математического ожидания относительно~$\mathcal{F}$. Соответственно, 
в~качестве достаточного условия оп\-ти\-маль\-ности воспользуемся уравнением 
динамического программирования
\begin{multline}
\fr{\partial V_t(x)}{\partial t} +\fr{1}{2}\sum\limits^n_{i,j=1} \sigma^2_{t_{ij}}
\fr{\partial^2 V_t(x)}{\partial x_i \partial x_j}+{}\\
{}+\min\limits_u\left[  
\sum\limits^n_{i=1} m_{t_i} \fr{\partial V_t(x)}{\partial x_i} + L_t(x,u)\right] 
=0\,,\\
V_T(x)=l(x)\,,
\label{e4-bos}
\end{multline}
где $m_{t_i}$~--- $i$-й элемент век\-тор-функ\-ции~$m_t(x,u)$; 
$\sigma^2_{t_{ij}} \hm= \sum\nolimits^m_{k=1} 
\sigma_{t_{ik}}\sigma_{t_{ki}}$, $\sigma_{t_{ij}}$~--- $i$-й по строке, $j$-й 
по столб\-цу элемент мат\-рич\-ной функции~$\sigma_t(x)$; $n$ и~$m$~--- 
размерности~$x_t$ и~$W_t$ соответственно.

     Традиционно в~рамках применения метода динамического 
программирования будем предполагать, что функции~$L_t$, $l$, $m_t$ 
и~$\sigma_t$ обеспечивают существование хотя бы одного решения 
уравнения~(\ref{e4-bos}), а~следовательно, и~оптимального 
управления~$u_t^*$, $0\hm\leq t\hm\leq T$, до\-став\-ля\-юще\-го минимум 
целевому функционалу~(\ref{e1-bos}). Задача оптимизации далее получается 
путем указания конкретных выражений для~$L_t$, $l$, $m_t$ и~$\sigma_t$.

\section{Постановка задачи управления выходом}

     Рассматриваемые далее случайные функции будут предполагаться 
скалярными. Такое упрощение позволит разгрузить выкладки и~итоговые 
выражения от не самых существенных деталей.
     
     Рассмотрим стохастическую дифференциальную сис\-те\-му, со\-сто\-яние 
которой представляет диффузи\-он\-ный процесс~$y_t$, описываемый 
нелинейным стохастическим дифференциальным уравнением Ито
     \begin{equation}
     dy_t=A_t\left( y_t\right) dt +\Sigma_t \left( y_t\right) dv_t\,,\enskip 
y_0=Y\,,
     \label{e5-bos}
     \end{equation}
где $v_t$~--- стандартный (одномерный) винеровский процесс; $Y$~--- 
случайная величина с~конечным вторым моментом; функции~$A_t$ 
и~$\Sigma_t$ удовлетворяют условиям Ито:
\begin{equation*}
\left\vert A_t(y)\right\vert +\left\vert \Sigma_t(y)\right\vert \leq C(1+\vert y\vert )\ 
\mbox{для\ всех } 0\leq t\leq T\,;
\end{equation*}

\vspace*{-12pt}

\noindent
\begin{multline*}
\hspace*{-2.10051pt}\left\vert A_t\left(y_1\right) -A_t \left( y_2\right) \right\vert +\left\vert 
\Sigma_t\left( y_1\right) -\Sigma_t \left(y_2\right)\right\vert \leq
C\left\vert y_1-y_2\right\vert\\
 \mbox{для\ всех\ } 0\leq t\leq T\ \mbox{и } 
y_1,y_2\in \mathbb{R}^1\,,
\end{multline*}
обеспечивающим существование единственного сильного (потраекторного) 
решения уравнения.
     
     Будем считать, что~$y_t$ описывает состояние некоторой 
динамической системы. Соответственно, поведение этой сис\-те\-мы опишем 
выходом, линейно связанным с~со\-сто\-янием:
     \begin{equation}
     dz_t=a_t y_t \,dt+ b_t z_t \,dt+ c_t u_t \,dt+\sigma_t \,dw_t\,,\enskip
     z_0=Z\,.
     \label{e6-bos}
     \end{equation}
Здесь $w_t$~--- не зависящий от~$v_t$, $Y$ и~$Z$ стандартный (одномерный) 
винеровский процесс; $Z$~--- случайная величина с~конечным вторым 
моментом; $u_t$~--- допустимое неупреждающее управ\-ле\-ние, качество 
которого определяется целевым функционалом следующего вида:
\begin{multline}
\!\hspace*{-3.98538pt}J\left( U_0^T\right) ={\sf E}\left\{ \int\limits_0^T \!\left( S_t\left( s_ty_t-g_t z_t -h_t 
u_t\right)^2 +G_t z_t^2+{}\right.\right.\\
\left.\left.{}+ H_t u_t^2
\vphantom{S_t\left( s_ty_t-g_t z_t -h_t 
u_t\right)^2}
\right) dt+S_T\left( s_T y_T -g_T 
z_T\right)^2+G_T z_T^2
\vphantom{\int\limits_0^T}\right\}\,,
\label{e7-bos}
\end{multline}
где $S_t$, $G_t$ и~$H_t$~--- неотрицательные функции\linebreak
$0\hm\leq t\hm\leq T$. 
Такой критерий отражает физический смысл задачи распределения ресурсов 
со\-глас\-но аналогичной~(\ref{e5-bos})--(\ref{e7-bos}) задаче для дис\-крет\-но\-го 
времени, рас\-смот\-рен\-ной в~\cite{7-bos}. В~част\-ности,  
функци\-онал~(\ref{e7-bos}) поз\-во\-ля\-ет ставить задачи отслеживания
 выходом 
со\-сто\-яния сис\-те\-мы, используя сла\-га\-емое $(y_t\hm- z_t)^2$, или 
управлением~--- линейной комбинации со\-сто\-яния и~выхода, сла\-га\-емое типа\linebreak 
$(y_t\hm+ z_t\hm- u_t)^2$. Поскольку задача формулируется 
в~предположении наличия пол\-ной информации о~со\-сто\-янии~$y_t$ 
и~выходе~$z_t$ (соответствующую $\sigma$-ал\-геб\-ру 
обозначим~$\mathcal{F}_t^{y,z}$), то допустимое управ\-ле\-ние ищется 
в~классе~$\mathcal{F}_t^{y,z}$-из\-ме\-ри\-мых неупреждающих функций 
(и,~как будет показано далее, оказывается управ\-ле\-ни\-ем с~обратной связью).

     Функции~$a_t$, $b_t$, $c_t$ и~$\sigma_t$ будем предполагать 
ограниченными: $\vert a_t\vert \hm+ \vert b_t\vert \hm+\vert c_t\vert \hm+ \vert 
\sigma_t \vert \hm\leq C$ для всех $0\hm\leq t\hm\leq T$, процесс  
управления~--- допустимым не\-упреж\-да\-ющим~\cite{9-bos}, обеспечивая, 
таким образом, существование сильного решения урав\-не\-ния~(\ref{e6-bos}) 
для любого допустимого управ\-ления.
     
     Задачу составляет поиск~$u_t^*$~--- допустимого управ\-ле\-ния, 
доставляющего минимум квад\-ра\-тич\-но\-му функционалу~$J(U_0^T)$.
      
     Поставленная задача очевидным образом формулируется в~терминах 
введенных выше в~(\ref{e1-bos})--(\ref{e3-bos}) обозначений, а~именно: 
     требуется обозначить
     \begin{gather*}
      x_t=\begin{pmatrix}
     y_t\\ z_t\end{pmatrix};\quad  m_t(x_t, u_t)=\begin{pmatrix}
     A_t(y_t)\\ a_t y_t +b_t z_t +c_t u_t\end{pmatrix};\\
     \sigma_t(x_t)= \begin{pmatrix}
     \Sigma_t(y_t)& 0\\
     0& \sigma_t\end{pmatrix};\quad W_t=\begin{pmatrix}
     v_t \\ w_t\end{pmatrix}
     %     \label{e8-bos}
     \end{gather*}
для записи уравнения со\-сто\-яния типа~(\ref{e2-bos}) и
\begin{align*}
L_t(x,u)&= L_t(y,z,u) ={}\\
&\hspace*{3mm}{}=S_t\left( s_t y-g_t z -h_t u\right)^2 +G_t z^2 +H_t  u^2\,;\\
l(x)&= l(y,z) =S_T \left( S_T y-g_T z\right)^2 +G_T z^2
%\label{e9-bos}
\end{align*}
для записи целевого функционала в~виде~(\ref{e1-bos}).

     Функция Беллмана~(\ref{e3-bos}) принимает вид 
     $V_t(x)\hm= V_t(y,z)$. Для записи со\-от\-вет\-ст\-ву\-юще\-го~(\ref{e4-bos}) 
уравнения Беллмана для~$V_t(y,z)$ заметим, что
     $$
     \left( \sigma^2_{t_{ij}}\right)_{i,j=1,2}= \begin{pmatrix}
     \Sigma_t^2(y) & 0\\
     0 & \sigma_t^2\end{pmatrix}\,.
     $$
     
     С~учетом перечисленных обозначений урав\-не\-ние динамического 
программирования~(\ref{e4-bos}) принимает вид:
     \begin{multline}
     \fr{\partial V_t(y,z)}{\partial t} +\fr{1}{2}\left( \Sigma_t^2(y) \fr{\partial^2 
V_t(y,z)} {\partial y^2}+\sigma_t^2\fr{\partial^2 V_t(y,z)} {\partial 
z^2}\right)+{}\\
    {}+\min\limits_u\! \left[ A_t(y) \fr{\partial V_t(y,z)}{\partial y}+\left( a_t 
y+b_t z+c_t u\right) \fr{\partial V_t(y,z)}{\partial z} +{}\right.\hspace*{-3pt}\\
\left.{}+ S_t\left( s_t y-g_t z-h_t 
u\right)^2+G_t z^2+H_t u^2
     \vphantom{\fr{\partial V_t(y,z)}{\partial y}}\right] =0\,,\\
     V_T(y,z)=S_T\left( s_T y-g_T z\right)^2+G_T z^2\,.
     \label{e10-bos}
     \end{multline}
     Это и~есть то самое уравнение, которое требуется решить: 
существование решения данного урав\-не\-ния суть достаточное условие 
оптимальности; оптимальное управ\-ле\-ние при этом~--- точ\-ка минимума 
со\-от\-вет\-ст\-ву\-юще\-го сла\-га\-емого.
     
\section{Динамическое программирование и~оптимальное 
управление}

     В рассматриваемой постановке линейность\linebreak выхода и~квадратичность 
критерия дают те же преимущества, что и~в~классической  
ли\-ней\-но-квад\-ра\-тич\-ной задаче управ\-ле\-ния~\cite{1-bos}, а~именно: 
позволяют сразу определить вид оптимального управ\-ле\-ния и~фактические 
условия его существования. Действительно, со\-хра\-няя в~(\ref{e10-bos}) под 
знаком $\min\nolimits_u$ только члены, зависящие от~$u$, получаем
     \begin{multline*}
     \fr{\partial V_t(y,z)}{\partial t} +\fr{1}{2}\left( \Sigma_t^2(y) \fr{\partial^2 
V_t(y,z)} {\partial y^2}+\sigma_t^2\fr{\partial^2 V_t(y,z)} {\partial 
z^2}\right)+{}\\
     {}+A_t(y)\fr{\partial V_t(y,z)}{\partial y}+\left( a_t y+b_t z\right) 
\fr{\partial V_t(y,z)}{\partial z}+{}\\
{}+S_t\left( s_t y-g_t z\right)^2 +G_t z^2+{}
\end{multline*}

\noindent
\begin{multline*}
     {}+\min\limits_u \left[ \left( c_t \fr{\partial V_t(y,z)}{\partial z}-2S_t \left( 
s_t y-g_t z\right) h_t\right)u +{}\right.\\
\left.{}+\left( S_t h_t^2+H_t\right) u^2
\vphantom{\fr{\partial V_t(y,z)}{\partial z}}
\right]=0\,,
     %\label{e11-bos}
     \end{multline*}
откуда в~предположении $S_t h_t^2\hm+ H_t\hm>0$ следует, что существует 
оптимальное управ\-ле\-ние, которое определяется равенством
\begin{multline}
u_t^* = u_t^*(y,z)=-\fr{1}{2}\left( S_t h_t^2 +H_t\right)^{-1} \left( c_t 
\fr{\partial V_t(y,z)}{\partial z}-{}\right.\\
\left.{}-2S_t\left( s_t y-g_t z\right) h_t
\vphantom{\fr{\partial V_t(y,z)}{\partial z}}
\right)
\label{e12-bos}
\end{multline}
и доставляет минимум соответствующему сла\-га\-емо\-му в~урав\-не\-нии Беллмана, 
равный
$-\left( S_t h_t^2\hm+\right.$\linebreak
$\left.{}+H_t\right)^{-1} \left( c_t 
{\partial V_t(y,z)}/{\partial 
z}\hm-2S_t\left( s_t y \hm-g_t z\right) h_t \right)^2/4.
$ 
     
     Отметим, что, как и~в~классической ли\-ней\-но-квад\-ра\-тич\-ной 
задаче, управ\-ле\-ние из класса до\-пус\-ти\-мых не\-упреж\-да\-ющих получилось 
управ\-ле\-ни\-ем с~обратной связью.
     
     Таким образом, функция Беллмана описывается сле\-ду\-ющим 
дифференциальным уравнением:
     \begin{multline}
     \fr{\partial V_t(y,z)}{\partial t} +\fr{1}{2}\left( \Sigma_t^2(y) \fr{\partial^2 
V_t(y,z)} {\partial y^2}+\sigma_t^2\fr{\partial^2 V_t(y,z)} {\partial 
z^2}\right)+{}\\
     {}+ A_t(y) \fr{\partial V_t(y,z)}{\partial y}+\left( a_t y+b_t z\right) 
\fr{\partial V_t(y,z)}{\partial z}+{}\\
{}+ S_t \left( s_t y- g_t z\right)^2 +G_t z^2-
 \fr{1}{4}\left( S_t h_t^2+H_t\right)^{-1}\times{}\\
 {}\times \left( c_t \fr{\partial V_t(y,z)} 
{\partial z}-2S_t\left( s_t y -g_t z\right) h_t \right)^2=0\,.
     \label{e13-bos}
     \end{multline}
     
     Возводя в~квадрат по\-след\-нее сла\-га\-емое в~(\ref{e13-bos}), перепишем 
его в~виде:
     \begin{multline}
     \fr{\partial V_t(y,z)}{\partial t} +\fr{1}{2}\left( \Sigma_t^2(y) \fr{\partial^2 
V_t(y,z)} {\partial y^2}+\sigma_t^2\fr{\partial^2 V_t(y,z)} {\partial 
z^2}\!\right)+{}\\
{}+A_t(y) \fr{\partial V_t(y,z)}{\partial y}
+ \left( 
\vphantom{\left( S_t h_t^2 +H_t\right)^{-1}}
a_t y+b_t z+{}\right.\\
\left.{}+\left( S_t h_t^2 +H_t\right)^{-1}
 c_t S_t \left( s_t y-g_t z\right) h_t
\right) 
     \fr{\partial V_t(y,z)}{\partial z}+{}\\
     {}+\left( S_t-\left( S_t h_t^2 +H_t\right)^{-1} S_t^2 h_t^2\right)\left( s_t y -
g_t z\right)^2+{}\\
     \!\!{}+
     G_t z^2 -\fr{1}{4}\left( S_t h_t^2+H_t\right)^{-1}\! c_t^2
     \left(\fr{\partial V_t(y,z)}{\partial z}\right)^{\!2}=0\,.\!\!
     \label{e14-bos}
     \end{multline}
     
     Рассматривая полученное уравнение, заметим, что его решение может 
быть пред\-став\-ле\-но в~виде:
   \begin{equation}
     V_t(y,z)= \alpha_t z^2+\beta_t(y) z +\gamma_t(y)\,,
     \label{e15-bos}
     \end{equation}
т.\,е.\ будем искать решение при дополнительном предположении 
о~квад\-ра\-тич\-ности функции Белл\-ма\-на по переменной~$z$, и~сведем, таким 
образом, поиск оптимального решения к~уравнениям относительно функций 
$\alpha_t$, $\beta_t(y)$ и~$\gamma_t(y)$. Отметим сразу, что явный вид 
функции~$\gamma_t(y)$ для реализации оптимального управ\-ле\-ния не 
требуется, однако далее будет предложен вариант вы\-чис\-ле\-ния и~этой 
функции, что пред\-став\-ля\-ет\-ся небесполезным, поскольку позволит выполнять 
расчет минимума целевого функционала. Источником для 
предложения~(\ref{e15-bos}) является уже упоминавшаяся аналогичная 
задача для случая дис\-крет\-но\-го времени~\cite{7-bos, 8-bos}. В~той задаче 
выражение для функции Беллмана получается формально без 
дополнительных усилий. При этом форма~(\ref{e15-bos}) обнаруживается 
как свойство оптимального решения. В~рассматриваемом случае 
непрерывного времени~(\ref{e15-bos}) постулируется, а~пра\-виль\-ность 
постулата под\-тверж\-да\-ет\-ся далее ре\-зуль\-ти\-ру\-ющи\-ми уравнениями 
для~$\alpha_t$, $\beta_t(y)$ и~$\gamma_t(y)$ Кроме того, данное 
предположение пред\-став\-ля\-ет\-ся вы\-те\-ка\-ющим из линейной структуры задачи 
в~отношении переменной~$z$, в~част\-ности, тем фактом, что такой вид 
функции Беллмана обеспечивает линейность оптимального 
управ\-ле\-ния~(\ref{e12-bos}) по~$z$.

     Граничное условие при выбранном предположении~(\ref{e15-bos}) 
принимает вид:

\noindent
     \begin{multline*}
     V_T(y,z)= S_T\left( s_T y- g_T z\right)^2+G_T z^2 ={}\\[-0.5pt]
     {}=\alpha_T z^2 
+\beta_T(y) z +\gamma_T(y)\,,
    \end{multline*}
т.\,е.

\noindent
\begin{align*}
\alpha_T&= S_T g_T^2 +G_T\,;\\[-0.5pt]
\beta_T(y)&=-2S_T s_T g_T y\,;\\[-0.5pt]
\gamma_T(y)&=S_T s_T^2 y^2\,.
%\label{e16-bos}
\end{align*}
          При этом само оптимальное управ\-ле\-ние, определенное 
выражением~(\ref{e12-bos}), оказывается управ\-ле\-ни\-ем с~обратной связью 
по~$y_t$ и~$z_t$:

\noindent
     \begin{multline}
     u_t^*=u_t^*(y,z) ={}\\[-0.5pt]
     {}=
     -\fr{1}{2}\left( S_t h_t^2 +H_t\right)^{-1}
     \left( c_t \left( 2\alpha_t z +\beta_t(y)\right) +{}\right.\\[-0.5pt]
    \left. {}+2S_t\left( s_t y-g_t z\right) 
h_t\right)\,.
     \label{e17-bos}
     \end{multline}
          Подставляем $V_t(y,z)\hm= \alpha_t z^2 \hm+ \beta_t(y) 
z\hm+\gamma_t(y)$ в~(\ref{e14-bos}):

\noindent
     \begin{multline*}
     \fr{\partial \alpha_t}{\partial t}\, z^2 +
     \fr{\partial \beta_t(y)}{\partial t}\,z +
     \fr{\partial \gamma_t(y)}{\partial t}+{}\\[-0.5pt]
     {}+\fr{1}{2}\left( \Sigma_t^2(y) \left( 
\fr{\partial^2\beta_t(y)}{\partial y^2}\,z +\fr{\partial^2 \gamma_t(y)}{\partial 
y^2}\right) +2\sigma_t^2\alpha_t\right)+{}\\[-0.5pt]
 {}+A_t(y)\left(\fr{\partial \beta_t(y)}{\partial y}\,z + \fr{\partial 
\gamma_t(y)}{\partial y}\right) +{}\\[-0.5pt]
\hspace*{-0.22987pt}{}+\left( a_t y+b_t z+\left( S_t h_t^2 +H_t\right)^{-1} c_t S_t \left( s_t y-
g_t z\right) h_t\right)\times{}
\end{multline*}

\noindent
\begin{multline*}
         {}\times \left( 2\alpha_t z+\beta_t(y)\right)+{}\\
     {}+\left( S_t-\left( S_t h_t^2 +H_t\right)^{-1} S_t^2 h_t^2\right)\left( s_t y-
g_t z\right)^2+{}\\
     {}+ G_t z^2 -\fr{1}{4}\left( S_t h_t^2 +H_t\right)^{-1} c_t^2 \left( 
2\alpha_t z+\beta_t(y)\right)^2=0\,.
     \end{multline*}
          Далее выделяем слагаемые при~$z^2$, $z$ и~$z^0$
          
          \noindent
     \begin{multline*}
     \fr{\partial \alpha_t}{\partial t}\, z^2 +\fr{\partial \beta_t(y)}{\partial t}\,z +
     \fr{\partial \gamma_t(y)}{\partial 
t}+\fr{1}{2}\,\Sigma_t^2(y)\fr{\partial^2\beta_t(y)}{\partial y^2}\,z+ {}\\
{}+
\fr{1}{2}\,\Sigma_t^2(y)\fr{\partial^2\gamma_t(y)}{\partial 
y^2}+\sigma_t^2\alpha_t+A_t(y)\fr{\partial \beta_t(y)}{\partial y}\,z +{}\\
{}+A_t(y) \fr{\partial 
\gamma_t(y)}{\partial y}+{}\\
{}+ 2\alpha_t \left( b_t -\left( S_t h_t^2+H_t\right)^{-1} c_t 
S_t h_t g_t \right)z^2+{}\\
     {}+
     \left( 2\alpha_t\left( \alpha_t+\left( S_t h_t^2+H_t\right)^{-1} c_t S_t h_t 
s_t\right)y +{}\right.\\
\left.{}+\beta_t(y) \left( b_t-\left( S_t h_t^2+H_t\right)^{-1} c_t S_t h_t 
g_t\right) \right) z+{}\\
     {}+\beta_t(y)\left( a_t +\left( S_t h_t^2+H_t\right)^{-1} c_t S_t h_t s_t\right) 
y+{}\\
{}+ \left( S_t -\left( S_t h_t^2+H_t\right)^{-1} S_t^2 h_t^2\right) g_t^2 z^2-{}\\
     {}- 2\left( S_t -\left( S_t h_t^2+H_t\right)^{-1} S_t^2 h_t^2\right) s_t g_t yz 
+{}\\
{}+
     \left( S_t-\left( S_t h_t^2+H_t\right)^{-1} S_t^2 h_t^2\right) s_t^2 y^2+{}\\
     {}+G_t z^2 -\left( S_t h_t^2 +H_t\right)^{-1} c_t^2 \alpha_t^2 z^2 -{}\\
     {}-\left( 
S_t h_t^2+H_t\right)^{-1} c_t^2 \alpha_t \beta_t(y) z-{}\\
{}-
\fr{1}{4}\left( S_t h_t^2+H_t\right)^{-1}  c_t^2 \beta_t^2(y)=0\,,
     \end{multline*}
группируем их и~получаем сле\-ду\-ющие уравнения:
\begin{itemize}
\item  для~$\alpha_t$:

\noindent
\begin{multline}
\fr{\partial\alpha_t}{\partial t}+2\alpha_t\left( b_t-\left( S_t h_t^2+H_t\right)^{-1} c_t 
S_t h_t g_t\right)+{}\\
{}+ \left( S_t- \left( S_t h_t^2+H_t\right)^{-1} S_t^2 h_t^2\right) 
g_t^2+G_t-{}\\
\hspace*{-8mm}{}-\left( S_t h_t^2+H_t\right)^{-1} c_t^2 \alpha_t^2 =0\,,\enskip \alpha_T=S_T 
g_t^2+G_T\,;\!\!
\label{e18-bos}
\end{multline}
\item для $\beta_t$:

\noindent
\begin{multline}
\fr{\partial\beta_t(y)}{\partial 
t}+\fr{1}{2}\,\Sigma_t^2(y)\fr{\partial^2\beta_t(y)}{\partial y^2} 
+A_t(y)\fr{\partial \beta_t(y)}{\partial y}+{}\\
{}+ 2\alpha_t\left( a_t +\left( S_t h_t^2+H_t\right)^{-1} c_t S_t h_t s_t\right) y+{}\\
{}+
\beta_t(y)\left( b_t -\left( S_t h_t^2 +H_t\right)^{-1} c_t S_t h_t g_t\right)-{}\\
{}-2\left( S_t-\left( S_t h_t^2+H_t\right)^{-1} S_t^2 h_t^2\right) s_t g_t y-{}
\\
{}-
\left( S_t h_t^2+H_t\right)^{-1} c_t^2 \alpha_t \beta_t(y)=0\,,\\
\beta_T(y)=-2S_T s_T g_T y\,;
\label{e19-bos}
\end{multline}
\item  для $\gamma_t$:
\begin{multline}
\hspace*{-0.8pt}\fr{\partial \gamma_t(y)}{\partial t}+\fr{1}{2}\,\Sigma_t^2(y)
\fr{\partial^2 \gamma_t(y)}{\partial y^2} +\sigma_t^2 \alpha_t +A_t(y)
\fr{\partial \gamma_t(y)}{\partial y}+{}\\
{}+ \beta_t(y)\left( a_t +\left( S_t h_t^2+H_t\right)^{-1} c_t S_t h_t s_t\right) y+{}\\
{}+
\left( S_t-\left( S_t h_t^2+H_t\right)^{-1} S_t^2 h_t^2\right)  s_t^2 y^2-{}\\
{}-\fr{1}{4}\left( S_t h_t^2+H_t\right)^{-1} c_t^2 \beta_t^2(y) =0\,,\\
\gamma_T(y)=S_T s_T^2 y^2\,.
\label{e20-bos}
\end{multline}
\end{itemize}
     
     Уравнение~(\ref{e18-bos}), легко заметить, является уравнением 
Риккати, которое в~силу сформулированного выше условия   
имеет единственное неотрицательное решение для всех $0\hm\leq t\hm\leq T$. 
Этот факт требует дополнительного комментария. Для получения 
уравнения~(\ref{e18-bos}) рас\-смот\-рим исходную задачу при дополнительных 
условиях $a_t\hm=0$ и~$s_t\hm=0$ для всех $0\hm\leq t\hm\leq T$. Нетрудно 
видеть, что эти условия рассматриваемую по\-ста\-нов\-ку сводят фактически 
к~классической ли\-ней\-но-квад\-ра\-тич\-ной задаче. Имеющуюся 
в~рассматриваемой формулировке чуть более общую форму целевой 
функции (принципиального значения это обобщение, конечно, не имеет) 
сведем к~классической еще одним предположением: $S_t\hm=0$ для всех 
$0\hm\leq t\hm\leq T$. Теперь уравнение для~$\alpha_t$ принимает хорошо 
известный вид:
     \begin{equation}
     \fr{\partial \alpha_t}{\partial t}+2\alpha_t b_t +G_t- H_t^{-1} c_t^2 
\alpha_t^2=0\,,\enskip \alpha_T=G_T\,.
     \label{e21-bos}
     \end{equation}

     В таком случае, как известно~\cite{10-bos}, существует единственное 
оптимальное управление~--- линейное с~обратной связью по выходу~$z_t$, 
с~коэффициентом усиления, опи\-сы\-ва\-емым уравнением  
Риккати~(\ref{e21-bos}). Именно этот результат дают  
уравнения~(\ref{e18-bos})--(\ref{e20-bos}) и~описываемая ими функция 
Беллмана~(\ref{e15-bos}), так как из $a_t\hm=0$ и~$s_t\hm=0$ немедленно 
следует, что $\beta_t(y)\hm=0$, откуда, в~свою очередь, с~учетом 
не\-за\-ви\-си\-мости решения от~$y_t$ следует, что $\gamma_t(y)\hm=\gamma_t$, 
т.\,е.\ не зависит от~$y$ и~задается уравнением: 
     $$
     \fr{\partial \gamma_t(y)}{\partial t} +\sigma^2_t \alpha_t=0\,,\enskip 
\gamma_T=0\,.
     $$ 
     Оптимальное управ\-ле\-ние при этом 
     $$
     u_t^*= -H_t^{-1} c_t \alpha_t z_t\,,
     $$
      т.\,е.\ все полностью совпадает с~известным классическим решением.
     
     С уравнениями~(\ref{e19-bos}) и~(\ref{e20-bos}) ситуация, естественно, 
обстоит сложнее. Это линейные уравнения второго порядка параболического 
типа, поскольку\linebreak
 $\Sigma_t^2(y)\hm>0$. Фактически отсутствуют 
конструктивные условия, гарантирующие существование их\linebreak
 решений 
(требовать, чтобы все фигурирующие в~уравнениях коэффициенты были 
представлены аналитическими функциями на всем пространстве значений, 
вряд ли целесообразно), поэтому далее будем предполагать, что данные 
уравнения имеют на рас\-смат\-ри\-ва\-емом интервале $0\hm\leq t\hm\leq T$ хотя 
бы одно ограниченное решение и~именно эти условия будем рас\-смат\-ри\-вать 
как достаточные условия существования оптимального решения 
рассматриваемой задачи.
     
     Таким образом, доказана следующая тео\-рема.
     
     \smallskip
     
     \noindent
     \textbf{Теорема.}\ \textit{Пусть для диффузионного 
процесса}~(\ref{e5-bos}) \textit{выполнены условия Ито, для 
     процесса}~(\ref{e6-bos})~--- \textit{ограничены коэффициенты, 
уравнения}~(\ref{e18-bos})--(\ref{e20-bos}) \textit{имеют ограниченные 
решения для $0\hm\leq t\hm\leq T$. Тогда минимум  
функционалу}~(\ref{e7-bos}) \textit{доставляет оптимальное 
управ\-ле\-ние}~(\ref{e17-bos}), \textit{где} $y\hm= y_t$; $z\hm=z_t$.
     
\section{Заключение}

     Рассмотренная задача оптимизации в~целом близка и~по модели, и~по 
критерию к~классической ли\-ней\-но-квад\-ра\-тич\-ной постановке. 
Принципиальным отличием является нелинейная модель для описания 
со\-сто\-яния динамической сис\-те\-мы, в~которой отсутствует управ\-ля\-ющее 
воздействие.\linebreak
 Такую модель наряду с~традиционной интер\-пре\-тацией  
<<со\-сто\-яние--вы\-ход>> мож\-но понимать как\linebreak модель неконтролируемого 
слож\-но\-го внешнего воздействия. Небольшое дополнительное отличие дает 
предложенная форма квад\-ра\-тич\-но\-го критерия, поз\-во\-ля\-ющая, в~част\-ности, 
ставить такие задачи, как отслеживание выходом или управ\-ле\-ни\-ем со\-сто\-яния 
сис\-те\-мы или ее выхода.
     
     Поскольку обсуждать возможности точного решения уравнений, 
определяющих оптимальное управ\-ле\-ние, не имеет смыс\-ла, наиболее 
актуальной далее является задача их приближенного чис\-лен\-но\-го решения 
и~анализа воз\-мож\-ности практической реализации. Этому посвящена вторая 
часть данной работы, пла\-ни\-ру\-емая к~выходу в~ближайшее время.

{\small\frenchspacing
 {%\baselineskip=10.8pt
 \addcontentsline{toc}{section}{References}
 \begin{thebibliography}{99}
\bibitem{1-bos}
\Au{Athans M.} Editorial on the LQG problem~// IEEE~T. Automat. Contr., 1971. Vol.~16. 
No.\,6. P.~528--552. doi: 10.1109/TAC.1971.1099845.
\bibitem{2-bos}
\Au{Wu Z.} Forward-backward stochastic differential equations, linear quadratic stochastic 
optimal control and nonzero sum differential games~// J.~Syst. Sci. Complex., 2005. Vol.~18. 
No.\,2. P.~179--192.
\bibitem{3-bos}
\Au{Chen B.\,S., Zhang~W.} Stochastic H2/H1 control with state-dependent noise~// IEEE 
T.~Automat. Contr., 2004. Vol.~49. No.\,1. P.~45--56. doi: 10.1109/TAC.2003.821400.
\bibitem{4-bos}
\Au{Bohacek S.} A~stochastic model of TCP and fair video transmission~// IEEE 
INFOCOM, 2003. Vol.~2. P.~1134--1144. doi: 10.1109/INFCOM.2003.1208950.
\bibitem{5-bos}
\Au{Домбровский В.\,В., Объедко~Т.\,Ю.} Управление с~прогнозированием системами 
с~марковскими скачками при ограничениях и~применение к~оптимизации 
инвестиционного портфеля~// Автомат. телемех., 2011. №\,5. С.~96--112. doi: 
10.1134/S0005117911050079.
\bibitem{6-bos}
\Au{Баландин Д.\,В., Коган~М.\,М.} Оптимальное линейно-квад\-ра\-тич\-ное управление: от 
матричных уравнений к~линейным матричным неравенствам~// Автомат. телемех., 2011. 
№\,11. С.~60--69. doi: 10.1134/ S0005117911110038.
\bibitem{7-bos}
\Au{Босов А.\,В.} Обобщенная задача распределения ресурсов программной системы~// 
Информатика и~её применения, 2014. Т.~8. Вып.~2. С.~39--47. doi: 
10.14357/19922264140204.
\bibitem{8-bos}
\Au{Босов А.\,В.} Управление линейным выходом дискретной стохастической системы по 
квадратичному критерию~// Изв. РАН. Теория и~системы управления, 2016. №\,3.  
С.~19--35. doi: 10.1134/S1064230716030060.
\bibitem{9-bos}
\Au{Флеминг У., Ришел~Р.} Оптимальное управление детерминированными 
и~стохастическими системами~/ Пер. с~англ.~--- М.: Мир, 1978. 316~с. 
(\Au{Fleming~W.\,H., Rishel~R.\,W.} Deterministic and stochastic optimal control.~--- New 
York, NY, USA: Springer-Verlag, 1975. 222~p.)
\bibitem{10-bos}
\Au{Девис М.\,Х.\,А.} Линейное оценивание и~стохастическое управление~/ Пер. с~англ.~--- 
М.: Наука, 1984. 206~с. (\Au{Davis~M.\,H.\,A.} Linear estimation and stochastic control.~--- 
London: Chapman and Hall, 1977. 224~p.)

 \end{thebibliography}

 }
 }

\end{multicols}

\vspace*{-6pt}

\hfill{\small\textit{Поступила в~редакцию 30.03.18}}

\vspace*{4pt}

%\newpage

%\vspace*{-24pt}

\hrule

\vspace*{2pt}

\hrule

\vspace*{-2pt}


\def\tit{STOCHASTIC DIFFERENTIAL SYSTEM OUTPUT CONTROL 
BY~THE~QUADRATIC CRITERION.~I.~DYNAMIC\\ PROGRAMMING 
OPTIMAL SOLUTION}


\def\titkol{Stochastic differential system output control 
by~the~quadratic criterion. I.~Dynamic programming 
optimal solution}

\def\aut{A.\,V.~Bosov and~A.\,I.~Stefanovich}

\def\autkol{A.\,V.~Bosov and~A.\,I.~Stefanovich}

\titel{\tit}{\aut}{\autkol}{\titkol}

\vspace*{-11pt}


\noindent
Institute of Informatics Problems, Federal Research Center ``Computer Science 
and Control'' of the Russian Academy of Sciences, 44-2~Vavilov Str., Moscow 
119333, Russian Federation


\def\leftfootline{\small{\textbf{\thepage}
\hfill INFORMATIKA I EE PRIMENENIYA~--- INFORMATICS AND
APPLICATIONS\ \ \ 2018\ \ \ volume~12\ \ \ issue\ 3}
}%
 \def\rightfootline{\small{INFORMATIKA I EE PRIMENENIYA~---
INFORMATICS AND APPLICATIONS\ \ \ 2018\ \ \ volume~12\ \ \ issue\ 3
\hfill \textbf{\thepage}}}

\vspace*{3pt}



\Abste{The problem of optimal control for the Ito diffusion 
process and a~controlled linear output is solved. The considered 
statement is close to the classical linear-quadratic Gaussian 
control  (LQG control) problem. Differences consist in the fact 
that the state is described by the nonlinear differential Ito equation  $dy_y = A_t(y_t) 
\,dt+\Sigma_t(y_t)\,dv_t$ and does not depend on the control~$u_t$, 
optimization subject is controlled linear output 
 $dz_t=a_ty_t\,dt +b_tz_t\,dt +c_t u_t\,dt +\sigma_t \,dw_t$. 
Additional generalizations are included in the quadratic 
quality criterion for the purpose of statement such problems 
as state tracking by output or a linear combination of state 
and output tracking by control. The method of dynamic programming 
is used for the solution. 
The assumption about Bellman function in the form  $V_t(y,z)= \alpha_t 
z^2+\beta_t(y) z+\gamma_t(y)$ allows one to find it. 
Three differential equations for the coefficients $\alpha_t$,  $\beta_t(y)$,
and $\gamma_t(y)$ give the solution. 
These equations constitute the optimal solution of the problem under consideration.}

\KWE{stochastic differential equation; optimal control; dynamic programming; 
Bellman function; Riccati equation; linear differential equations of parabolic type}


\DOI{10.14357/19922264180314}

\vspace*{-12pt}

\Ack
\noindent
This work was partially supported by the Russian Science Foundation (grant  
16-07-00677).



%\vspace*{6pt}

  \begin{multicols}{2}

\renewcommand{\bibname}{\protect\rmfamily References}
%\renewcommand{\bibname}{\large\protect\rm References}

{\small\frenchspacing
 {%\baselineskip=10.8pt
 \addcontentsline{toc}{section}{References}
 \begin{thebibliography}{99}
\bibitem{1-bos-1}
\Aue{Athans, M.} 1971. Editorial on the LQG problem. \textit{IEEE~T. 
Automat. Contr.} 16(6):528--552. doi: 10.1109/ TAC.1971.1099845.
\bibitem{2-bos-1}
\Aue{Wu, Z.} 2005. Forward-backward stochastic differential equations, linear 
quadratic stochastic optimal control and\linebreak\vspace*{-12pt}

\columnbreak

\noindent
 nonzero sum differential games. 
\textit{J.~Syst. Sci. Complex.} 18(2):179--192.
\bibitem{3-bos-1}
\Aue{Chen, B.\,S. and W.~Zhang.} 2004. Stochastic H2/H1 control with  
state-dependent noise. \textit{IEEE~T. Automat. Contr.} 49(1):45--56.
doi: 10.1109/TAC.2003.821400.
\bibitem{4-bos-1}
\Aue{Bohacek, S.} 2003. A~stochastic model of TCP and fair video 
transmission. \textit{IEEE INFOCOM}. 2:1134--1144.
doi: 10.1109/INFCOM.2003.1208950.
\bibitem{5-bos-1}
\Aue{Dombrovskii, V.\,V., and T.\,Yu.~Ob''edko.} 2011. Predictive control of 
systems with Markovian jumps under constraints and its application to the 
investment portfolio optimization. \textit{Automat. Rem. Contr.}  
72(5):989--1003.
\bibitem{6-bos-1}
\Aue{Balandin, D.\,V., and M.\,M.~Kogan.} 2011. Optimal linear-quadratic 
control: From matrix equations to linear matrix inequalities. \textit{Automat. 
Rem. Contr.} 72(11):2276--2284.
\bibitem{7-bos-1}
\Aue{Bosov, A.\,V.} 2014. Obobshchennaya zadacha raspredeleniya resursov 
programmnoy sistemy [The generalized problem of software system resources 
distribution]. \textit{Informatika i~ee Primeneniya~--- Inform. Appl.}  
8(2):39--47. doi: 
10.14357/19922264140204.
\bibitem{8-bos-1}
\Aue{Bosov, A.\,V.} 2016. Discrete stochastic system linear output control 
with respect to a quadratic criterion. \textit{J.~Comput. Syst. Sc. 
Int.} 55(3):349--364.
\bibitem{9-bos-1}
\Aue{Fleming, W.\,H., and R.\,W.~Rishel.} 1975. \textit{Deterministic and 
stochastic optimal control.} New York, NY: Springer-Verlag. 222~p.
\bibitem{10-bos-1}
\Aue{Davis, M.\,H.\,A.} 1977. \textit{Linear estimation and stochastic 
control.} London: Chapman and Hall. 224~p.
\end{thebibliography}

 }
 }

\end{multicols}

\vspace*{-6pt}

\hfill{\small\textit{Received March 30, 2018}}

%\pagebreak

%\vspace*{-18pt}
     
     \Contr
     
       \noindent
       \textbf{Bosov Alexey V.} (b.\ 1969)~--- Doctor of Science in technology, 
principal scientist, Institute of Informatics Problems, Federal Research 
Center ``Computer Science and Control'' of the Russian Academy of Sciences, 
44-2~Vavilov Str., Moscow 119333, Russian Federation; 
\mbox{AVBosov@ipiran.ru}
       
       \vspace*{3pt}
       
       \noindent
       \textbf{Stefanovich Alexey I.} (b.\ 1983)~--- principal specialist, 
Institute of Informatics Problems, Federal Research Center ``Computer Science 
and Control'' of the Russian Academy of Sciences, 44-2~Vavilov Str., Moscow 
119333, Russian Federation; \mbox{AStefanovich@frccsc.ru}
\label{end\stat}

\renewcommand{\bibname}{\protect\rm Литература}       

       %14
\def\stat{bosov+naumov}

\def\tit{МОДЕЛЬ ПЕРЕДВИЖЕНИЯ ПОЕЗДОВ И~МАНЕВРОВЫХ ЛОКОМОТИВОВ 
НА~ЖЕЛЕЗНОДОРОЖНОЙ СТАНЦИИ В~ПРИЛОЖЕНИИ К~ОЦЕНКЕ И~АНАЛИЗУ 
ВЕРОЯТНОСТИ БОКОВОГО СТОЛКНОВЕНИЯ$^*$}

\def\titkol{Модель передвижения поездов и~маневровых локомотивов 
на~железнодорожной станции} % в~приложении к~оценке и~анализу  вероятности бокового столкновения}

\def\aut{А.\,В.~Босов$^1$, А.\,Н.~Игнатов$^2$, А.\,В.~Наумов$^3$}

\def\autkol{А.\,В.~Босов, А.\,Н.~Игнатов, А.\,В.~Наумов}

\titel{\tit}{\aut}{\autkol}{\titkol}

\index{Босов А.\,В.}
\index{Игнатов А.\,Н.}
\index{Наумов А.\,В.}
\index{Bosov A.\,V.}
\index{Ignatov A.\,N.}
\index{Naumov A.\,V.}




{\renewcommand{\thefootnote}{\fnsymbol{footnote}} \footnotetext[1]
{Работа выполнена при поддержке Российского научного фонда (проект №\,16-11-00062).}}


\renewcommand{\thefootnote}{\arabic{footnote}}
\footnotetext[1]{Институт проблем информатики Федерального исследовательского центра <<Информатика 
и~управление>> Российской академии наук, \mbox{AVBosov@ipiran.ru}}
\footnotetext[2]{Московский авиационный институт (национальный исследовательский университет), 
\mbox{alexei.ignatov1@gmail.com}}
\footnotetext[3]{Московский авиационный институт (национальный исследовательский университет), 
\mbox{naumovav@mail.ru}}

\vspace*{-6pt}

  

\Abst{Предложена математическая модель для решения задачи управления движением 
маневровых локомотивов на железнодорожной станции при заданном расписании движения 
пассажирских/грузовых поездов через станцию и~фиксированном графике маневровых работ, 
под которыми понимается отцепка и~прицепка вагонов, выпуск и~расформирование поездов. 
Модель используется для постановки и~решения задачи минимизации времени передвижения по 
станции маневрового локомотива для осуществления очередной маневровой работы с~учетом 
занятости некоторых путей для движения вследствие наличия на них пассажирских/грузовых 
поездов, а также с~учетом ограничений на время исполнения маневровых работ. Исходная 
постановка сводится к~задаче смешанного целочисленного линейного программирования. 
Представленная модель использована для оценки вероятности бокового столкновения на 
станции с~учетом возможных случайных задержек в~движении пассажирских поездов. 
Приведены результаты численных экспериментов.}

\KW{имитационная модель; расписание; интенсивность; смешанное целочисленное линейное 
программирование}

\DOI{10.14357/19922264180315}
  
%\vspace*{4pt}


\vskip 10pt plus 9pt minus 6pt

\thispagestyle{headings}

\begin{multicols}{2}

\label{st\stat}

\section{Введение}

      Основной задачей управления рисками на железнодорожном транспорте 
является достижение и~поддержание допустимого уровня риска различных 
неблагоприятных событий~[1]. К~таким событиям относятся, например, 
столкновения на железнодорожных переездах поездов с~автотранспортом, сход 
вагонов при поездной работе, пожары на локомотивах, излом рельса под поездом и~др. 

Изучение риска таких событий привлекало внимание как российских~[2--5], 
так и~западных~[6] исследователей. Работа~[6] в~этой связи выделяется тем, что 
в~ней исследовались вопросы оптимального положения вагонов с~опасными 
грузами в~поезде. Отдельный пласт задач связан с~оценкой и~анализом рисков 
происшествий, происходящих на железнодорожных станциях.
      
      На крупных станциях, где маневровые работы осуществляются при 
интенсивном движении поездов, в~качестве возможных неблагоприятных событий 
рассматриваются столкновения между маневровыми составами 
и~пассажирскими/грузовыми поездами, взрез стрелки (случайный перевод 
стрелки колесами подвижного состава) или сход с~рельсов маневрового состава. 
Для описания соответствующих рисков в~работе~[7] была предложена оценка 
вероятности хотя бы одного столкновения на станции за произвольный 
промежуток времени, а~в~[8] оценено число взрезов и~сходов с~рельсов. 
Вычисление этих величин основано на интенсивностях пересечения маневровыми 
составами стрелочных переводов, используемых для перевода подвижного состава с~одного пути на другой. 

Для уточнения приведенной в~[7] оценки 
интенсивностей можно провести длительные натурные наблюдения за работой 
конкретной станции, которые могут оказаться весьма дорогостоящими. При\linebreak
 этом 
результаты (оценки интенсивности) будут\linebreak получены только для одной этой 
станции и~не дадут ка\-ких-ли\-бо оценок для других. 

Другим способом уточнения 
интенсивностей является по\-стро\-ение имитационной модели передвижения 
поездов на железнодорожной станции, которую можно было бы использовать на 
различных станциях, задавая небольшой набор входных данных, таких как схема 
станции, график движения поездов и~маневровых работ, под которыми 
понимаются отцепка и~прицепка вагонов, выпуск и~расформирование поездов. 
Набор этих работ будет далее предполагаться фиксированным, поскольку он 
может быть получен из суточного плана работы станции. Кроме того, важно 
отметить, что сами маневровые работы могут быть отсортированы по времени 
начала исполнения. С~течением суток вследствие исполнения маневровых работ 
их общее число уменьшается. Для предотвращения нарушения суточного плана 
работы станции необходимо оптимальным образом прокладывать траекторию 
движения маневрового состава от момента исполнения последней работы 
к~моменту начала новой работы.
{\looseness=-1

}
      
      
      Применение имитационных моделей для задач на железнодорожном 
транспорте является обычной\linebreak практикой. Чаще всего моделируют назначение\linebreak 
локомотивов для составов с~целью уменьшения общего числа используемых  
локомотивов~[9--12], передвижение пассажиров по станции с~целью увеличения 
пропускной способности станции~[13]. Схожими по смыслу являются 
Комплексная автоматизированная система ведения технологических процессов 
работы железнодорожных станций (АС ВТП) и~Автоматизированная система 
разработки и~мониторинга выполнения Единых технологических процессов 
работы железнодорожных путей необщего пользования и~станций примыкания 
(АС ЕТП). В~них заложены описания, модели всех технологических процессов, 
происходящих на станции. Разрабатываемая в~данной работе модель 
интенсивностей пересечения стрелочных переводов маневровыми локомотивами 
строится с~учетом компромисса между учетом всех технологических процессов, 
происходящих на станции, и~наименьшим объемом входных данных с~целью 
быстрой,\linebreak но близкой к~точной оценки искомых интенсивностей. Построенная 
система моделирования\linebreak также позволит учитывать случайные задержки 
в~прибытии и~отправлении поездов на станции, позволяя проанализировать, что 
произойдет на станции в~случае неисполнения расписания.
      
      Далее в~статье сформулированы принципы и~описана структура 
имитационной модели передвижения поездов на железнодорожной станции. 
Станция представляется в~виде неориентированного нагруженного графа, 
вершинами которого являются стрелочные переводы, а также места перехода 
кривых участков пути в~прямые и~места перехода прямых участков пути в~кривые, 
а также точки входа/выхода со станции. Ребрами являются железнодорожные 
пути, связывающие вершины. Каждому ребру сопоставляется число, равное 
расстоянию от одной вершины до другой на плоскости, т.\,е.\ длина пути. На 
станции предполагается возможным движение транзитных пассажирских поездов, 
пассажирских поездов местного формирования, грузовых поездов, а также 
маневровых составов. Пассажирские и~грузовые поезда следуют по расписанию, 
которое в~силу различных происшествий на железной дороге может исполняться 
не полностью, т.\,е.\ предусматривается возможность задержек поездов. 
В~заключительной части статьи приводятся примеры расчета интенсивности 
пересечения стрелочных переводов маневровыми составами, выполненные на 
основе построенной имитационной модели.

\vspace*{-8pt}

\section{Постановка задачи}

\vspace*{-2pt}

      Пусть имеется определяющий модель станции неориентированный граф 
$G\hm= \langle V, E\rangle$, где $V$~--- множество вершин (стрелочных 
переводов, стыков между рельсами и~точек входа и~выхода со станции (границ 
станции)); $E$~--- множество ребер (железнодорожных путей), соединяющих 
данные вершины. Также задана функция $D:\ E\to {\sf R}_+$, характеризующая 
длину ребра. Пусть $\vert E\vert \hm=m$. Пронумеровав ребра графа~$G$ от~1 
до~$m$, составим новый граф $G^\prime\hm= \langle V^\prime, E^\prime\rangle$, 
множеством вершин~$V^\prime$ которого являются номера ребер графа~$G$, 
т.\,е.\ $V^\prime\hm= \{1,2,\ldots , m\}$. Множество ребер~$E^\prime$ включает 
в~себя ребра между вершинами из~$V^\prime$, если эти вершины являются 
смежными ребрами в~графе~$G$. На элементах множества~$V^\prime$ введем 
функцию $D^\prime:\ V^\prime \hm \to {\sf R}_+$, характеризующую <<вес>> 
вершин в~графе~$G^\prime$, т.\,е.\ длину соответствующих ребер в~графе~$G$.
      
      Предположим, что расписание движения пассажирских поездов и~набор 
маневровых работ являются корректными в~том смысле, что некоторым 
назначением траекторий движения маневровых локомотивов можно исполнить 
суточный план работы станции без его нарушения. Пусть максимальная скорость 
передвижения по станции маневрового локомотива равна~$v_{\max}$; номер 
ребра графа~$G$, на котором заканчивается предыдущая маневровая работа, равен 
$j_0\hm\in V^\prime$; номер ребра графа~$G$, на котором начинается следующая 
маневровая работа, равен $j_T\hm\in V^\prime$, где~$T$~--- время окончания 
передвижения по станции маневрового локомотива, назначенного для 
осуществления маневровой работы. С~учетом данных параметров поставим 
задачу по отысканию маршрута передвижения по станции маневрового 
локомотива для выполнения очередной еще не исполненной маневровой работы 
в~плане на сутки\linebreak с~учетом ряда физических ограничений, описыва\-емых ниже, 
с~целью минимизации времени~$T$. Будем отсчитывать время от момента 
окончания последней маневровой работы. Пусть $u(t)$~--- номер ребра, которое 
проходит маневровый локомотив в~момент времени~$t$ от окончания последней 
маневровой работы ($u(t)\hm\in V^\prime$).
      
      Поскольку на станции осуществляется движение пассажирских поездов 
и~других маневровых составов, часть ребер закрыта для проезда. В~связи с~этим 
введем функцию $F:\ V^\prime\times {\sf R}_+^1\hm\to \{0,1\}$ вида
 $$
 F(j,t) \stackrel{\mathrm{def}}{=}
 \begin{cases}
 0\,, &\ \mbox{ребро\ (графа\ $G$)\ с\ номером\ $j$}\\
 &\ \mbox{свободно\ в\ момент\ времени\  $t$}\,;\\
 1 &\ \mbox{иначе},
 \end{cases}
 $$
которая характеризует занятость ребра для движения маневрового локомотива 
в~момент времени~$t$ от начала выполнения маневровой работы.

      Перебор всех возможных путей (называемых в~дальнейшем маршрутами) 
в~графе~$G^\prime$ из вершины~$j_0$ в~вершину~$j_T$ невозможен в~силу их 
бесконечного количества, связанного с~наличием циклов в~графе~$G$. Поэтому 
выберем несколько таких маршрутов, общим числом равных~$L$. Составим из 
этих последовательностей множество~$J$. Для минимизации времени~$T$ 
необходимо найти минимальное время, за которое можно пройти каждый 
маршрут (каждую последовательность вершин) из множества~$J$, а~затем 
выбрать среди найденных времен минимальное.
      
      Сформулируем задачу по минимизации времени прохождения маневрового 
состава через станцию по маршруту, задаваемому произвольной 
последовательностью из множества~$J$. Для этого отметим, что произвольный 
элемент~$\overline{J}_l$ множества~$J$ имеет вид:
$$\overline{J}_l\hm= \{ j_{0,l}, 
j_{1,l},\ldots , j_{k,l},\ldots , l_{K_l,l}\}\,,$$
где $j_{k,l}\hm\in V^\prime$, причем $j_{0,l}\hm=j_0$ и~$j_{K_l,l}\hm=j_T$, 
а $k\hm=\overline{1,K_l}$, $l\hm= \overline{1,L}$. 

%o
Пусть~$t_{k,l}$~--- момент перехода с~вершины с~номером~$j_{k-1,l}$   
на вершину с~номером~$j_{k,l}$ графа~$G^\prime$, а~$t_{0,l}\hm= 0$, 
$t_{K_l+1,l}\hm= T$. Тогда для каждого элемента~$\overline{J}_l$ множества~$J$ 
условие физической реализуемости можно записать в~виде:
$$t_{k+1,l}\hm- t_{k,l} \hm\geq \fr{D^\prime(j_{k,l})}{v_{\max}}\,, 
k\hm= \overline{0,K_l}\,,$$ 
т.\,е.\ исключить нереализуемую возможность проехать любое ребро 
графа~$G$ за бесконечно малое время.
      
Условие на движение только по свободным ребрам графа~$G$ 
записывается в~виде:
$$\forall j_{k,l} \forall t\hm\in [t_{k,l}, t_{k+1,l}]\quad 
F(j_{k,l},t)=0\,, k= \overline{0,K_l}\,,$$ 
которое эквивалентно ограничениям: 
$$\int\limits_{t_{k,l}}^{t_{k+1,l}} F(j_{k,l},t)\,dt \hm= 0\,, 
k\hm= \overline{0,K_l}\,.$$ 

%o
Также имеет место ограничение вида: 
$$t_{\mathrm{мин}}\hm\leq t_{K_l+1,l}\hm\leq t_{\mathrm{макс}}\,,$$ 
которое гарантирует, что очередная 
маневровая работа не начнется позже момента времени~$t_{\mathrm{макс}}$ 
и/или раньше~$t_{\mathrm{мин}}$. Величины~$t_{\mathrm{мин}}$ и 
$t_{\mathrm{макс}}$ определяются исходя из графика маневровых работ и~связаны 
с~тем, что маневровый локомотив должен подъехать к~месту осуществления работ 
в~определенный промежуток времени $[t_{\mathrm{мин}}, t_{\mathrm{макс}}]$, 
так как в~иное время очередная работа может быть пропущена или еще не 
начаться согласно графику работ.
      
      Таким образом, задача по минимизации времени прохождения маневрового 
состава через станцию по маршруту, задаваемому произвольной 
последовательностью~$\overline{J}_l$ из множества~$J$, имеет вид:
      \begin{equation}
      t_{K_l+1,i}\to \min\limits_{t_{k,l},\ k=\overline{1,K_l+1}}
      \label{e1-nau}
      \end{equation}
при ограничениях 
\begin{equation}
\left.
\begin{array}{c}
t_{k+1,l}-t_{k,l}\geq \fr{D^\prime (j_{k,l})}{v_{\max}}\,,\enskip 
k=\overline{0,K_l}\,;\\[6pt]
\displaystyle\int\limits_{t_{k,l}}^{t_{k+1,l}} F(j_{k,l},t)\,dt=0\,,\enskip k=\overline{0,K_l}\,;\\[6pt]
t_{\mathrm{мин}}\leq t_{K_l+1,j}\leq t_{\mathrm{макс}}\,,\enskip t_{0,l}=0\,.
\end{array}
\right\}
\label{e2-nau}
\end{equation}

\vspace*{-8pt}

\section{Сведение задачи о~выборе маршрута движения к~задаче 
смешанного целочисленного линейного программирования}

\vspace*{-2pt}

Заметим, что используемые выше функции 
$$H(j_{k,l},t) 
\stackrel{\mathrm{def}}{=} \int\limits_0^t F(j_{k,l},y)\,dy\,, 
k\hm= \overline{1,K_l}\,,$$
являются ку\-соч\-но-ли\-ней\-ны\-ми, поэтому ограничения~(2) 
в~задаче~(1) являются нелинейными, что делает поиск решения в~задаче~(1) при 
ограничениях~(2) весьма затруднительным. Но указанное свойство 
функций~$H(j_{k,l},t)$, $k\hm= \overline{0,K_l}$, позволяет путем введения 
целочисленных переменных свести исходную задачу нелинейного 
программирования к~задаче смешанного целочисленного линейного 
программирования. Для этого сформируем множество~${\sf T}_{k,l}$, состоящее 
из левой и~правой границ интервалов времени, когда ребро с~номером~$j_{k,l}$ 
оказывалось свободным для движения маневрового состава, $k\hm= 
\overline{0,K_l}$.
\pagebreak
      
      С помощью множества~${\sf T}_{k,l}$ можно определить <<окна>> (т.\,е.\ 
интервалы времени, в~которые ребро с~номером~$j_{k,l}$ свободно). Упорядочив 
элементы множества~${\sf T}_{k,l}$ по возрастанию, составим из них 
вектор~$\tau_{k,l}$. Пусть $\mathrm{dim}\,\tau_{k,l}\hm= 2I_{k,l}$, где  
$I_{k,l}$~--- число <<окон>>. Ввведем новые переменные~$\delta^i_{k,l}$, 
равные единице, если движение по ребру с~номером~$j_{k,l}$ осуществляется 
в~промежуток времени между~$\tau_{k,l}^{2i-1}$ и~$\tau_{k,l}^{2i}$, 
и~равные нулю, если движение по ребру с~номером~$j_{k,l}$ в~промежуток 
времени между~$\tau_{k,l}^{2i-1}$ и~$\tau_{k,l}^{2i}$ не 
осуществляется, $k\hm= \overline{0,K_l}$, $i\hm= \overline{1, I_{k,l}}$. С~учетом 
новых переменных~$\delta^i_{k,l}$ задача~(1) при ограничениях~(2) 
эквивалентным образом сводится к~следу\-ющей задаче:
\begin{equation}
t_{K_l+1,l}\to \min\limits_{t_{k,l},\ t_{K_l+1,l},\ \delta^i_{0,l},\ 
\delta^i_{k,l},\ k=\overline{1,K_l},\ i=\overline{1,I_{k,l}}}
      \label{e3-nau}
      \end{equation}
при ограничениях 
\begin{equation}
\left.
\begin{array}{l}
t_{k+1,l}-t_{k,l}\geq \fr{D^\prime(j_{k,l})}{v_{\max}}\,,\enskip 
k=\overline{0,K_l}\,;\\[6pt]
\displaystyle\sum\limits_{i=1}^{I_{k,l}} \delta^i_{k,l}=1\,,\enskip k=\overline{0,K_l}\,;\\[6pt]
t_{k+1,l}\leq \delta^i_{k,l} \tau^{2i}_{k,l}+\left( 1-\delta^i_{k,l}\right) 
t_{\mathrm{макс}}\,,\\[6pt]
\hspace*{31mm}k=\overline{0,K_l},\ i=\overline{1,I_{k,l}}\,;\\[6pt]
t_{k,l}\geq \delta^i_{k,l} \tau^{2i-1}_{k,l}\,,\enskip k=\overline{0,K_l}\,,\ 
i=\overline{1,I_{k,l}}\,;\\[6pt]
t_{\mathrm{мин}}\leq t_{K_l+1,j}\leq t_{\mathrm{макс}}\,,\ \delta^i_{k,l}\in 
\{0,1\}\,,\\[6pt]
\hspace*{18mm}i=\overline{1,I_{k,l}}\,,\ k=\overline{0,K_l}\,,\ t_{0,l}=0\,.
\end{array}
\right\}
\label{e4-nau}
\end{equation}
      
Поскольку задача~(1) при ограничениях~(2) может не иметь решения, то 
введем новую величину:
      $$
      T_l \stackrel{\mathrm{def}}{=} \begin{cases}
      T_l^*\,, &\mbox{решение\ задачи\ (1)}\\
     & \mbox{при\ ограничениях\ (2)\ 
существует}\,;\\
      +\infty &\mbox{иначе},
      \end{cases}
      $$
$l\hm=\overline{1,L}$. Здесь~$T_l^*$~--- оптимальное значение критерия 
в~задаче~(3)--(4). С~использованием величин~$T^*_l$ заключаем, что можно 
оценить величину~$T^*$ сверху как $T^*\hm\leq \min\nolimits_{l=\overline{1,L}} 
T_l^*$, где $T^*$~--- минимальное время, за которое можно добраться из 
вершины~$j_0$ в~вершину~$j_T$ в~графе~$G^\prime$, не нарушая и~график 
маневровых работ, и~график движения пассажирских поездов.
      
      Таким образом, решение задачи, сформулированной выше, дает 
возможность оценить сверху время передвижения маневрового локомотива по 
станции до исполнения очередной маневровой работы.
      
      После передвижения по станции к~месту выполнения маневровой работы 
      и~ее исполнения маневровому локомотиву назначается любая маневровая работа из 
перечня оставшихся маневровых работ, такая что она может быть выполнена 
вовремя указанным маневровым локомотивом.
      
\vspace*{-8pt}

\section{Применение полученных результатов для~оценки интенсивности 
пересечения стрелочного перевода маневровым составом} %4
\vspace*{-2pt}

      Наряду с~максимальным числом пассажирских/грузовых поездов, 
пересекающих станцию, которое характеризует доход от функционирования 
станции, важной характеристикой является вероятность хотя бы одного бокового 
столкновения между пассажирскими и~грузовыми поездами и~маневровыми 
составами, которая характризует степень безопасности движения. Для увеличения 
максимального числа поездов, пересекающих станцию, вначале необходимо 
рассчитать вероятность хотя бы одного бокового столкновения, например за 
сутки. Для этого необходимо рассчитать вероятность столкновения произвольного 
пас\-са\-жир\-ско\-го/гру\-зо\-во\-го поезда с~маневровым составом на 
произвольной стрелке при пересечении пассажирским/грузовым поездом станции. 

%o
В~[7] была предложена следующая оценка вероятности бокового столкновения, 
получаемая на основе предположения о~том, что поток пересечений маневровыми 
составами стрелок~--- пуассоновский:
\vspace*{-6pt}

\noindent
\begin{multline}
\hspace*{-6pt}\p(A)=\left( \lambda_{\mathrm{м}} \left( \p_{\mathrm{м}} +\p_{\mathrm{м}} 
\p_{\mathrm{п/г}}+\p_{\mathrm{п/г}}\right) \left( 
\fr{l_{\mathrm{п/г}}}{v_{\mathrm{п/г}}}+\fr{l_{\mathrm{м}}}{v_{\mathrm{м}}} \right)+{}
\right.\\
\left.{}+ \lambda_{\mathrm{с}} \p_{\mathrm{п/г}} \tau_{\mathrm{с}} 
+\lambda_{\mathrm{м}} \p_{\mathrm{м}} \p_{\mathrm{пс/гс}} 
\tau_{\mathrm{пс/гс}}
\vphantom{\left(\fr{l_{\mathrm{п/г}}}{v_{\mathrm{п/г}}}+
\fr{\lambda_{\mathrm{м}}}{v_{\mathrm{м}}} \right)}
\right) \kappa_{\mathrm{с}}\,,\\
\lambda_{\mathrm{м}}=\displaystyle\fr{1}{d}\sum\limits^Q_{i=1} \fr{N_i}{N}\,.
\label{e5-nau}
\end{multline}
      
      \setcounter{figure}{1}
\begin{figure*}[b] %fig2
\vspace*{1pt}
 \begin{center}
 \mbox{%
 \epsfxsize=162.278mm 
 \epsfbox{bos-2.eps}
 }
 \end{center}
\vspace*{-9pt}
\Caption{Схема станции}
\end{figure*}

\noindent
Здесь $\lambda_{\mathrm{м}}$~--- интенсивность пересечения стрелки маневровыми 
составами в~направлении, при котором\linebreak возможно боковое столкновение между 
пассажирским/грузовым поездом и~маневровым составом\linebreak (рис.~1) [1/ч];  
$Q$~--- число маневровых локомотивов на станции; $N$~--- число стрелок на 
станции; $N_i$~--- число стрелок, пересекаемых в~час $i$-м маневровым 
составом; $d$~--- общее число на\-прав\-ле\-ний движения по стрелке маневрового\linebreak 
состава (на рис.~1 $d\hm=4$); $l_{\mathrm{п/г}}$~--- средняя длина\linebreak  
пасса\-жир\-ско\-го/гру\-зо\-во\-го поезда [км]; $l_{\mathrm{м}}$~--- средняя 
длина маневрового состава [км]; $v_{\mathrm{п/г}}$~--- сред-\linebreak няя скорость 
пас\-са\-жир\-ско\-го/гру\-зо\-во\-го поезда\linebreak\vspace*{-12pt}

{ \begin{center}  %fig1
 \vspace*{1pt}
  \mbox{%
 \epsfxsize=64.436mm 
 \epsfbox{bos-1.eps}
 }

\end{center}

\noindent
{{\figurename~1}\ \ \small{Направление следования (отмечено пунктиром) маневрового состава через 
стрелочный перевод (отмечен кружком), при котором возможно столкновение с~пассажирским 
составом}}
}

\vspace*{9pt}

\noindent
 [км/ч];
 $v_{\mathrm{м}}$~--- средняя ско\-рость 
маневрового со\-ста\-ва [км/ч]; $\tau_{\mathrm{с}}$~--- среднее время нахождения 
маневрового со\-ста\-ва на стрелочном переводе
при условии остановки на нем [ч]; 
$\tau_{\mathrm{пс/гс}}$~--- среднее время стоянки
пас\-са\-жир\-ско\-го/гру\-зо\-во\-го поезда
 на стрелочном переводе [ч]; 
$\p_{\mathrm{пс/гс}}$~--- вероятность остановки  
пас\-са\-жир\-ско\-го/гру\-зо\-во\-го поезда\linebreak на стрелочном переводе; 
$\p_{\mathrm{п/г}}$~--- вероятность\linebreak проезда на запрещающий сигнал %\linebreak 
светофора 
пас\-са\-жир\-ско\-го/гру\-зо\-во\-го поезда; $\p_{\mathrm{м}}$~--- вероятность 
проезда на запрещающий сигнал светофора маневрового состава; 
$\kappa_{\mathrm{с}}$~--- коэффициент, характеризующий неизолированность 
стрелочного перевода (возможность бокового столкновения), принимающий 
значение~1, если стрелочный перевод неизолированный, и~0, если 
изолированный.

Вероятности $\p_{\mathrm{пс/гс}}$, $\p_{\mathrm{п/г}}$ и~$\p_{\mathrm{м}}$ 
задаются согласно~\cite{7-nau, 14-nau}. Как видно из формулы~(\ref{e5-nau}), 
интенсивность~$\lambda_{\mathrm{м}}$ (при одинаковых~$d$) получается одной 
и~той же для стрелок и~с~интенсивным, и~с~неинтенсивным движением по ним 
маневровых составов. При этом сама величина~$\lambda_{\mathrm{м}}$ 
существенно влияет на вероятность $\p(A)$. Поэтому на основе составленного 
расписания движения маневрового локомотива по станции предложим иной 
способ вычисления интенсивностей~$\lambda_{\mathrm{м}}$.
     
     Наличие предложенной имитационной модели позволяет использовать для 
оценки интенсивности пересечения стрелочных переводов маневровыми 
локомотивами классическую оценку в~виде частоты. Пусть $w$~--- число 
пересечений маневровыми локомотивами стрелочного перевода в~направлении, 
при котором возможно боковое столкновение между пассажирским/грузовым 
поездом и~маневровым составом, в~течение суток. Тогда 
$\lambda_{\mathrm{м}}\hm= w/24$.
     
\vspace*{-8pt}

\section{Результаты численного эксперимента}

\vspace*{-2pt}

Рассмотрим некоторую станцию, схема части которой отображена на рис.~2.

 Общее число стрелок на данной станции равно~149. Пусть данную 
станцию в~некоторые сутки пересекают~84~пассажирских поезда (включая 
поездные локомотивы), общее число маневровых работ равно~76, а~максимальная 
скорость~$v_{\mathrm{м}}$ передвижения маневровых локомотивов по станции 
равна~10,8~км/ч. Очевидно, что чем больше маршрутов передвижения 
маневрового локомотива по станции, найденных по графу~$G^\prime$, к~месту 
выполнения очередной маневровой работы исследуется, тем большая доля 
маневровых работ может быть выполнен, а время передвижения по станции 
маневрового локомотива уменьшится. Однако в~этом случае также увеличится 
суммарное время, необходимое для нахождения решения задач~(\ref{e3-nau}) при 
ограничениях~(\ref{e4-nau}). 

%o
Проанализируем время счета и~доля выполненных 
маневровых работ в~зависимости от различного числа маршрутов передвижения 
маневрового локомотива по станции к~месту выполнения очередной маневровой 
работы.
      
\begin{table*}\small %tabl1
 \begin{center}
 \parbox{320pt}{\Caption{Время передвижения по станции для осуществления $i$-й маневровой 
работы [мин]\,/\,время поиска решения в~задачах~(3) при ограничениях~(4) [мин]}

}

\vspace*{2ex}
       
\begin{tabular}{|c|c|c|c|c|c|c|c|c|}
\hline
&\multicolumn{8}{c|}{$i$}\\
\cline{2-9}
\raisebox{6pt}[0pt][0pt]{$L$}&$\ldots$&14&15&16&17&18&19&$\ldots$\\
\hline
1&$\cdots$&32/0,003&25/0,01&4/0,002&3/0,005&6/0,009&6/0,01&$\cdots$\\
$\vdots$&$\vdots$&$\vdots$&$\vdots$&$\vdots$&$\vdots$&$\vdots$&$\vdots$&$\vdots$\\
7&$\cdots$&30/0,02&23/0,1&2/0,03&2/0,04&4/0,17&5/0,11&$\cdots$\\
\hline
\end{tabular}
\end{center}
\end{table*}
      
Расчеты проводились на персональном компьютере со следующими характеристиками: 
процессор Intel Core i7 2,8 ГГц, память 8~ГБ 1\,600~МГц DDR3.
      
Как следует из табл.~1 и~2, время счета ожидаемо растет при увеличении 
числа~$L$, однако рас\-тет и~доля выполненных маневровых работ. При этом 
для выполнения~100\% маневровых работ достаточно рассмотреть небольшое 
число маршрутов\linebreak
\vspace*{-12pt}

%\begin{table*}
{ %tabl2
      
 \noindent
{{\tablename~2}\ \ \small{Общее время счета и~доля выполненных маневровых работ}}

 \tabcolsep=13pt
 \begin{center}
\small
\begin{tabular}{|c|c|c|}
\hline
$L$&\tabcolsep=0pt\begin{tabular}{c}Общее время\\ счета, ч\end{tabular}&
\tabcolsep=0pt\begin{tabular}{c}Доля выполненных\\ 
маневровых работ, \%\end{tabular}\\
\hline
1&0,26&73\hphantom{,9}\\
3&2,71&83,8\\
5&2,95&97,3\\
7&3,03&100\hphantom{,99}\\
\hline
\end{tabular}
\end{center}
\vspace*{3pt}
}
%\end{table*}

\setcounter{figure}{2}

 {\begin{center}  %fig1
% \vspace*{2pt}
  \mbox{%
 \epsfxsize=64.436mm 
 \epsfbox{bos-3.eps}
 }

\end{center}

\noindent
{{\figurename~3}\ \ \small{Интенсивности пересечения стрелочного 
перевода №\,178 по формуле~(\ref{e5-nau}) по 
данным из~\cite{7-nau}\,/\,по предлагаемой имитационной модели}}
\vspace*{9pt}
}

\noindent 
пересечения маневровым локомотивом станции до места 
исполнения очередной маневровой работы.
      
Теперь отдельно рассмотрим стрелочный перевод №\,178 и~приведем число 
пересечений маневровыми локомотивами данного перевода в~различных 
направлениях.

  Как следует из рис.~3, интенсивности пересечения стрелочного перевода 
№\,178, полученные по предлагаемой имитационной модели и~по 
формуле~(\ref{e5-nau}), отличаются. Это связано с~тем, что интенсивности, 
полученные по формуле~(\ref{e5-nau}), назначаются одинаковыми каждому 
стрелочному переводу, а реальная загруженность того или иного стрелочного 
перевода (удаленность или близость конкретного стрелочного перевода от места 
проведения большинства маневровых работ) в~расчет не принимается.

\section{Заключение}
\vspace*{-2pt}

 В работе рассмотрена задача по составлению расписания движения по 
станции маневровых локомотивов с~учетом различных технических ограничений, 
суточного расписания движения поездов, а также набора маневровых работ. 

%o
Сформулирована задача нелинейного программирования, которая сведена к~задаче 
смешанного целочисленного линейного программирования. 

%o
На основе 
построенной имитационной модели получена модифицированная оценка 
интен\-сивности пересечения стрелочных переводов\linebreak в~различных направлениях 
маневровыми локомотивами, которая в~дальнейшем используется для уточнения 
оценки вероятности бокового столкновения между маневровым локомотивом 
и~пассажирским/грузовым поездом на стрелочном пере\-воде.

\vspace*{-8pt}
      
{\small\frenchspacing
 {%\baselineskip=10.8pt
 \addcontentsline{toc}{section}{References}
 \begin{thebibliography}{99}
\vspace*{-2pt}

\bibitem{1-nau}
ГОСТ 33433-2015. Безопасность функциональная. Управление рисками на железнодорожном 
транспорте.~--- М.: Стандартинформ, 2016. 34~c.

\bibitem{4-nau} %2
\Au{Шубинский И.\,Б., Проневич~О.\,Б., Данилова~А.\,Д.} Особенности оценки 
вероятности возникновения пожаров на тепловозах различных серий~// 
Надежность, 2016. T.~16. №\,4. С.~24--29. 
doi: 10.21683/1729-2646-2016-16-4-24-29.

\bibitem{5-nau} %3
\Au{Крутиков А.\,М.} Оценка надежности рельсов Р65 по ресурсу: экспериментальные 
исследования.~--- М.: Финансы и~статистика, 2016. 151~c.

\bibitem{3-nau} %4
\Au{Замышляев А.\,М., Игнатов~А.\,Н., Кибзун~А.\,И., Новожилов~Е.\,О.} 
Функциональная зависимость между количеством вагонов в~сходе из-за 
неисправностей вагонов или пути и~факторами движения~// 
Надежность, 2018. T.~18. №\,1. С.~53--60. 
doi: 10.21683/1729-2646-2018-18-1-53-60.

\bibitem{2-nau} %5
\Au{Кибзун А.\,И., Игнатов А.\,Н.} О~задаче распределения инвестиций 
в~установку средств, предотвращающих несанкционированный проезд 
автотранспортом железнодорожных переездов, для различных статистических 
критериев~// Надежность, 2018. T.~18. №\,2. С.~31--37. 
doi: 10.21683/1729-2646-2018-18-2-31-37.

\bibitem{6-nau}
\Au{Bagheri M., Saccomanno~F., Chenouri~S., Fu~L.} Reducing the threat 
of in-transit derailments involving dangerous goods through effective 
placement along the train consist~// Accident Anal. Prev., 2011. 
Vol.~43. Iss.~4. P.~613--620. doi: 10.1016/j.aap.2010.09.008.

\bibitem{7-nau}
\Au{Игнатов А.\,Н., Кибзун~А.\,И., Платонов~Е.\,Н.} Оценка вероятности 
столкновения железнодорожных составов на железнодорожных станциях на 
основе пуассоновской модели~// Автоматика и~телемеханика, 2016. №\,11. С.~43--59. 
doi: 10.1134/S0005117916110035.

\bibitem{8-nau}
\Au{Шубинский И.\,Б., Замышляев~А.\,М., Игнатов~А.\,Н., Кан~Ю.\,С., 
Кибзун~А.\,И., Платонов~Е.\,Н.} Оценка рисков, связанных с~проездом 
запрещающего сигнала светофора маневровым составом или пассажирским 
поездом~// Надежность, 2016. T.~16. №\,3. С.~39--46. 
doi: 10.21683/1729-2646-2016-16-3-39-46.

\bibitem{9-nau}
\Au{Иванов С.\,В., Кибзун~А.\,И., Осокин~А.\,В.} Оптимизационная 
стохастическая модель назначения локомотивов для перевозки грузовых составов~// 
Автоматика и~телемеханика, 2016. №\,11. С.~80--95. 
doi: 10.1134/S0005117916110059.

\bibitem{11-nau} %10
\Au{Лазарев А.\,А., Мусатова~Е.\,Г., Тарасов~И.\,А.} Решение задачи планирования 
двухстороннего движения на однопутном участке железной дороги с~разъездом~// 
Автоматика и~телемеханика, 2016. №\,11. С.~158--174. 
doi: 10.1134/S0005117916110047.

\bibitem{10-nau} %11
\Au{Гайнанов Д.\,Н., Рассказова~В.\,А.} Математическое моделирование в~задаче 
оптимального назначения и~перемещения локомотивов методами теории графов 
и~комбинаторной оптимизации~// Труды МАИ, 2017. №\,92. 24~с.

\bibitem{12-nau}
\Au{Буянов М.\,В., Иванов~С.\,В., Кибзун~А.\,И., Наумов~А.\,В.} Развитие 
математической модели управления грузоперевозками на участке железнодорожной 
сети с~учетом случайных факторов~// Информатика и~её применения, 2017. T.~11. 
Вып.~4. С.~85--93. doi: 10.14357/19922264170411.

\bibitem{13-nau}
\Au{Искаков Т.\,А.} Имитационное моделирование функционирования транспортного узла~// 
Интеллекту\-аль\-ные системы управления на железнодорожном\linebreak транспорте. 
Компьютерное 
и~математическое моделирование: Труды V~научно-технич. конф. 
с~международным участием.~--- М.: НИИАС, 2016. С.~221--225.
\bibitem{14-nau}
\Au{Шубинский И.\,Б.} Функциональная надежность информационных систем. Методы 
анализа.~--- Ульяновск: Печатный двор, 2012. 296~c.
 \end{thebibliography}

 }
 }

\end{multicols}

\vspace*{-4pt}

\hfill{\small\textit{Поступила в~редакцию 01.02.18}}

\vspace*{10pt}

%\newpage

%\vspace*{-24pt}

\hrule

\vspace*{2pt}

\hrule

\vspace*{-2pt}


\def\tit{MODEL OF TRANSPORTATION OF~TRAINS AND~SHUNTING LOCOMOTIVES 
AT~A~RAILWAY STATION FOR~EVALUATION AND~ANALYSIS OF~SIDE-COLLISION PROBABILITY}

\def\titkol{Model of transportation of~trains and~shunting locomotives 
at~a~railway station for~evaluation
of~side-collision probability}

\def\aut{A.\,V.~Bosov$^1$, A.\,N.~Ignatov$^2$, and~A.\,V.~Naumov$^2$}

\def\autkol{A.\,V.~Bosov, A.\,N.~Ignatov, and~A.\,V.~Naumov}

\titel{\tit}{\aut}{\autkol}{\titkol}

\vspace*{-11pt}

\noindent
$^1$Institute of Informatics Problems, Federal Research Center ``Computer Science and Control'' of the 
Russian\linebreak
$\hphantom{^1}$Academy of Sciences, 44-2~Vavilov Str., Moscow 119333, Russian Federation

\noindent
$^2$Moscow Aviation Institute (National Research University), 4~Volokolamskoe Shosse, Moscow 
125993, Russian\linebreak
$\hphantom{^1}$Federation


\def\leftfootline{\small{\textbf{\thepage}
\hfill INFORMATIKA I EE PRIMENENIYA~--- INFORMATICS AND
APPLICATIONS\ \ \ 2018\ \ \ volume~12\ \ \ issue\ 3}
}%
 \def\rightfootline{\small{INFORMATIKA I EE PRIMENENIYA~---
INFORMATICS AND APPLICATIONS\ \ \ 2018\ \ \ volume~12\ \ \ issue\ 3
\hfill \textbf{\thepage}}}

\vspace*{3pt}



\Abste{A mathematical model for solution of the shunting locomotives 
traffic control problem is proposed for a~fixed schedule of passenger/freight 
trains traffic across a~station and a~fixed time-table of shunting operations: 
set off and attaching of cars, output operation and deconsolidation of trains. 
The model is used for formulation and solution of the problem to minimize 
time of shunting locomotive transportation across the station to perform 
next scheduled operation with respect to busy condition of some tracks for 
transportation owing to presence of passenger/freight trains on them and 
with respect to restriction on shunting operation execution time. The 
original statement is reduced to mixed integer linear programming. 
The presented model was used for evaluation of side-collision probability 
at a~railway station with respect to possible random drag in passenger trains 
traffic.  The results of numerical experiments are presented.}

\KWE{simulation model; schedule; intensity; mixed integer linear programming}
      
      
\DOI{10.14357/19922264180315} %

%\vspace*{-14pt}

\Ack
\noindent
This work was supported by the Russian Science Foundation (grant No.\,16-11-00062).

\pagebreak


  \begin{multicols}{2}

\renewcommand{\bibname}{\protect\rmfamily References}
%\renewcommand{\bibname}{\large\protect\rm References}

{\small\frenchspacing
 {%\baselineskip=10.8pt
 \addcontentsline{toc}{section}{References}
 \begin{thebibliography}{99}
\bibitem{1-nau-1}
GOST 33433-2015. 2016. \textit{Bezopasnost' funktsional'naya. Upravlenie riskami na 
zheleznodorozhnom transporte} [Functional safety. Risk control in railroad transport]. 
Moscow: Standartinform. 34~p.

\bibitem{4-nau-1} %2
\Aue{Shubinsky, I.\,B., O.\,B.~Pronevich, and A.\,D.~Danilova.} 2016. Osobennosti 
otsenki veroyatnosti vozniknoveniya pozharov na teplovozakh razlichnykh seriy 
[Special aspects of estimating the probability of fire occurrence on diesel 
locomotives of various types]. \textit{Dependability} 16(4):24--29.
doi: 10.21683/1729-2646-2016-16-4-24-29.

\bibitem{5-nau-1} %3
\Aue{Krutikov, A.\,M.} 2016. \textit{Otsenka nadezhnosti rel'sov R65 po resursu: 
eksperimental'nye issledovaniya} [Evaluation of the dependability of rails R65 on the 
resource: Experimental researches]. Moscow: Finansy i~statistika. 151~p.

\bibitem{3-nau-1} %4
\Aue{Zamyshliaev, A.\,M., A.\,N.~Ignatov, A.\,I.~Kibzun, and E.\,O.~Novozhilov.} 
2018. Funktsional'naya zavisimost' mezhdu kolichestvom vagonov v~skhode iz-za 
neispravnostey vagonov ili puti i~faktorami dvizheniya [Functional dependency 
between the number of wagons derailed due to wagon or track defects and the traffic 
factors]. \textit{Dependability} 18(1):53--60.
doi: 10.21683/1729-2646-2018-18-1-53-60.

\bibitem{2-nau-1} %5
\Aue{Kibzun, A.\,I., and A.\,N.~Ignatov.} 2018. O~zadache raspredeleniya investitsiy 
v~ustanovku sredstv, pred\-ot\-vra\-shcha\-yushchikh nesanktsionirovannyy proezd 
avtotransportom zheleznodorozhnykh pereezdov, dlya razlichnykh statisticheskikh 
kriteriev [On the task of allocating investment to facilities preventing 
unauthorized movement of road vehicles across level crossings for various statistical 
criteria]. \textit{Dependability} 18(2):31--37.
doi: 10.21683/1729-2646-2018-18-2-31-37.

\bibitem{6-nau-1}
\Aue{Bagheri, M., F.~Saccomanno, S.~Chenouri, and L.~Fu.} 2011. Reducing the 
threat of in-transit derailments involving dangerous goods through effective placement 
along the train consist. \textit{Accident Anal. Prevent.} 43(3):613--620. 
doi: 10.1016/j.aap.2010.09.008.

\bibitem{7-nau-1}
\Aue{Ignatov, A.\,N., A.\,I.~Kibzun, and E.\,N.~Platonov.} 2016. Estimating collision 
probabilities for trains on railroad stations based on a~Poisson model. \textit{Automat.  
Rem. Contr.} 77(11):1914--1927.

\bibitem{8-nau-1}
\Aue{Shubinsky, I.\,B., A.\,M.~Zamyshlyaev, A.\,N.~Ignatov, Yu.\,S.~Kan, 
A.\,I.~Kibzun, and E.\,N.~Platonov}. 2016. Otsenka riskov, svyazannykh s~proezdom 
zapreshchayushchego signala svetofora, manevrovym sostavom ili passazhirskim 
poezdom [Estimation of risks related to stop signal passed by shunting loco 
or passenger train]. \textit{Dependability} 16(3):39--46.
doi: 10.21683/1729-2646-2016-16-3-39-46.

\bibitem{9-nau-1}
\Aue{Ivanov, S.\,V., A.\,I.~Kibzun, and A.\,V.~Osokin.} 2016. Stochastic optimization 
model of locomotive assignment to freight trains. \textit{Automat. Rem. Contr.} 
77(11):1944--1956.

\bibitem{11-nau-1} %10
\Aue{Lazarev, A.\,A., E.\,G.~Musatova, and I.\,A.~Tarasov.} 2016. Two-directional 
traffic scheduling problem solution for a~single-track railway with siding. \textit{Automat.  
Rem. Contr.} 77(12):2118--2131.

\bibitem{10-nau-1} %11
\Aue{Gainanov, D.\,N., and V.\,A.~Rasskazova.} 2017. Ma\-te\-ma\-ti\-che\-skoe 
modelirovanie v~zadache optimal'nogo na\-zna\-che\-niya i~peremeshcheniya lokomotivov 
metodami teorii grafov i~kombinatornoy optimizatsii [Mathematical modelling of 
locomotives' traffic problem by graph theory and combinatorial optimization methods]. 
\textit{Trudy MAI} 92. 24~p.

\bibitem{12-nau-1}
\Aue{Buyanov, M.\,V., S.\,V.~Ivanov, A.\,I.~Kibzun, and A.\,V.~Naumov.} 2017. 
Razvitie matematicheskoy modeli upravleniya gruzoperevozkami na uchastke 
zheleznodorozhnoy seti s~uchetom sluchaynykh faktorov [Development of the 
mathematical model of cargo transportation control on a railway network segment 
taking into account random factor]. \textit{Informatika i~ee Primeneniya~--- Inform. 
Appl.} 11(4):85--93. doi: 10.14357/19922264170411.

\bibitem{13-nau-1}
\Aue{Iskakov, T.\,A.} 2016. Imitatsionnoe modelirovanie funktsi\-oni\-ro\-va\-niya 
transportnogo uzla [Simulation modelling of functioning transport hub]. \textit{Trudy 
V nauchno-tekhnich. konf. s~mezhdunarodnym uchastiem ``Intellektual'nye sistemy 
upravleniya na zheleznodorozhnom transporte. Komp'yuternoe i~matematicheskoe 
modelirovanie''} [5th Science and Technological Conference 
(with International Participation) ``Intellectual Control Systems in Railroad Transport. Computer and 
Mathematical Modeling'' Proceedings]. Moscow. 221--225.
\bibitem{14-nau-1}
\Aue{Shubinskiy, I.\,B.} 2012. \textit{Funktsional'naya nadezhnost' informatsionnykh sistem. 
Metody analiza} [International functional dependability of information 
systems. Methods of analysis]. Ul'yanovsk: Pechatnyy dvor. 296~p.
\end{thebibliography}

 }
 }

\end{multicols}

\vspace*{-6pt}

\hfill{\small\textit{Received February 1, 2018}}

%\pagebreak

\vspace*{-12pt}

\Contr

\noindent
\textbf{Bosov Alexey V.} (b.\ 1969)~--- Doctor of Science in technology, principal scientist, Institute 
of Informatics Problems, Federal Research Center ``Computer Science and Control'' of the Russian 
Academy of Sciences, 44-2~Vavilov Str., Moscow 119333, Russian Federation; 
\mbox{AVBosov@ipiran.ru}

\vspace*{3pt}

\noindent
\textbf{Ignatov Alexey N.} (b.\ 1991)~--- Candidate of Science (PhD) in physics and mathematics, 
senior lecturer, Moscow Aviation Institute (National Research University), 4~Volokolamskoe Shosse, 
Moscow 125993, Russian Federation; \mbox{alexei.ignatov1@gmail.com}

\vspace*{3pt}

\noindent
\textbf{Naumov Andrey V.} (b.\ 1966)~--- Doctor of Science in physics and mathematics, associate 
professor, Department ``Probability Theory and Computational Modeling,'' Moscow Aviation Institute 
(National Research University), 4~Volokolamskoe Shosse, Moscow 125993, Russian Federation; 
\mbox{naumovav@mail.ru}

\label{end\stat}

\renewcommand{\bibname}{\protect\rm Литература}        %15
%\newcommand {\ff}{{\mathcal F}}
\newcommand {\ebd}{\triangleq}
\newcommand{\me}[2]{\mathbf{E}_{ #1 }\left\{ \mathop{#2} \right\} }



\def\stat{borisov}

\def\tit{ФИЛЬТРАЦИЯ СОСТОЯНИЙ МАРКОВСКИХ СКАЧКООБРАЗНЫХ ПРОЦЕССОВ 
ПО~ДИСКРЕТИЗОВАННЫМ НАБЛЮДЕНИЯМ$^*$}

\def\titkol{Фильтрация состояний марковских скачкообразных процессов 
по~дискретизованным наблюдениям}

\def\aut{А.\,В.~Борисов$^1$}

\def\autkol{А.\,В.~Борисов}

\titel{\tit}{\aut}{\autkol}{\titkol}

\index{Борисов А.\,В.}
\index{Borisov A.\,A.}




{\renewcommand{\thefootnote}{\fnsymbol{footnote}} \footnotetext[1]
{Работа выполнена при частичной поддержке РФФИ (проект 16-07-00677).}}


\renewcommand{\thefootnote}{\arabic{footnote}}
\footnotetext[1]{Институт проблем информатики Федерального исследовательского центра <<Информатика 
и~управление>> Российской академии наук,
\mbox{aborisov@frccsc.ru}}

%\vspace*{8pt}



\Abst{Статья посвящена решению задачи оптимальной 
фильтрации состояний однородного марковского скачкообразного процесса (МСП). 
Наблюдения представляют собой приращения случайных процессов~--- интегральных 
преобразований состояний, зашумленные винеровскими процессами, интенсивность 
которых также зависит от оцениваемого состояния. Оптимальная оценка в~моменты 
получения нового наблюдения вычисляется как функция предыдущей оценки и~новых 
наблюдений, а~между моментами наблюдений~--- простейшим прогнозом в~силу системы 
уравнений Колмогорова. Рекуррентная формула пересчета ресурсозатратна, так как 
содержит  интегралы~--- мас\-штаб\-но-сдви\-го\-вые смеси многомерных гауссиан, 
где в~качестве смешивающих выступают распределения времени пребывания 
состояния в~каждом из возможных значений. Предложены более простые аппроксимации, 
основанные на предположении об ограниченности числа скачков состояния за время между 
наблюдениями. Получены универсальные локальная и~глобальная характеристики точности 
аппроксимаций, зависящие от па\-ра\-мет\-ров оцениваемого процесса, величины 
временн$\acute{\mbox{о}}$го шага  между наблюдениями и~максимального числа учитываемых скачков.}

\KW{марковский скачкообразный процесс; оптимальная фильтрация; мультипликативные 
шумы в~наблюдениях; стохастическое дифференциальное уравнение; численная аппроксимация}

\DOI{10.14357/19922264180316}
  
%\vspace*{4pt}


\vskip 10pt plus 9pt minus 6pt

\thispagestyle{headings}

\begin{multicols}{2}

\label{st\stat}



 \section{Введение}
 
 Фильтр Вонэма~\cite{Won_65}~--- один из редких удачных случаев, когда 
 оценка оптимальной фильтрации состо\-яния стохастической системы наблюдения 
 выражается в~виде решения некоторой замк\-ну\-той\linebreak конечномерной сис\-те\-мы 
 стохастических дифференциальных уравнений. 
 
 Алгоритм данного фильт\-ра 
 позволяет вычислить оценку фильт\-ра\-ции со\-сто\-яния \textit{марковского скачкообразного 
 процесса} с~\mbox{конечным} множеством состояний по наблюдениям в~присутствии 
 аддитивных винеровских шумов. Теоретически оптимальная оценка со\-сто\-яния~--- 
 его условное распределение в~текущий момент времени~--- 
 обладает очевидными свойствами неотрицательности и~нормировки. 
 При чис\-лен\-ной реализации данного фильтра классическим методом 
 Эй\-ле\-ра--Ма\-ру\-ямы~\cite{KP_92} данные свойства могут не сохраняться и~процедура 
 вы\-чис\-ле\-ния становится неустойчивой.  В~связи с~этим обстоятельством разрабатывались 
 другие алгоритмы чис\-лен\-но\-го решения уравнения фильтра Вонэма, обладающие 
 требуемыми свойствами устойчивости (см.~\cite{YZL_04, PR_10} и~библиографию в~них). 
 В~час\-ти этих работ доказана лишь слабая сходимость пред\-ла\-га\-емых аппроксимационных 
 схем к~оценке фильт\-ра Вонэма, в~то время как ка\-кая-ли\-бо 
 характеризация точ\-ности этих приближений отсутствует.
 
 В~\cite{B_18} было представлено распространение фильт\-ра Вонэма на случай 
 наблюдений с~мультипликативными шумами. При этом уравнение обобщенного 
 фильт\-ра содержит в~правой части квадратическую характеристику шумов в~наблюдениях. 
 Данный процесс на практике никогда не наблюдается непосредственно, а~является лишь 
 некоторым нелинейным интегральным преобразованием наблюдений. Очевидно, что 
 имеющиеся в~настоящий момент времени алгоритмы приближенного вычисления оценки 
 фильтрации Вонэма для данной системы не подходят. 
 
 Целью предлагаемой работы является ис\-поль\-зование результатов оптимальной 
 фильтрации со\-стояний сис\-тем с~дискретным временем для аппроксимации решения 
 аналогичной задачи для\linebreak стохастических дифференциальных сис\-тем. 
 
 Статья организована следующим образом. Раздел~2 содержит формальную постановку 
 задачи фильт\-ра\-ции со\-сто\-яний однородного МСП с~конечным множеством со\-сто\-яний 
 по наблюдениям, полученным путем временн$\acute{\mbox{о}}$й дискретизации процессов с~непрерывным 
 временем~--- интегральных преобразований со\-сто\-яния сис\-те\-мы в~присутствии 
 мультипликативных винеровских шумов.\linebreak
  В~разд.~3 пред\-став\-ле\-но решение поставленной 
 задачи фильт\-ра\-ции: пересчет оценок со\-сто\-яний в~момент получения новых 
 дискретизованных наблюдений выполняется в~соответствии с~некоторыми\linebreak 
 рекуррентными интегральными соотношениями, в~то время как между 
 моментами наблюдений оценка корректируется в~соответствии с~прогнозом в~силу 
 сис\-те\-мы уравнений Колмогорова. Вы\-чис\-ли\-тель\-ная слож\-ность 
 упомянутых выше интегральных\linebreak 
 соотношений связана с~тем, что в~расчет принимается воз\-мож\-ность того, что между 
 моментами наблюдений оцениваемое со\-сто\-яние может совершить произвольное чис\-ло 
 скачков. В~разд.~4 пред\-став\-лен более простой алгоритм приближенного вы\-чис\-ле\-ния 
 оценки фильт\-ра\-ции, основанный на ограничении возможного числа учитываемых скачков 
 МСП. Доказана тео\-ре\-ма, опре\-де\-ля\-ющая как\linebreak
  локальную (одношаговую), так и~глобальную 
 (многошаговую) характеристики точ\-ности предложенного при\-бли\-же\-ния~--- 
 $\ell_1$-нор\-мы ошибки аппроксимации. Полученные характеристики являются\linebreak 
 универсальными, т.\,е.\ не асимптотическими по шагу дискретизации, и~зависят от характеристик 
 самого МСП, %\linebreak
  шага временн$\acute{\mbox{о}}$й дискретизации и~чис\-ла
  скачков со\-сто\-яния, учи\-ты\-ва\-емых 
 на шаге. Об\-суж\-де\-ние результатов и~заключительные комментарии пред\-став\-ле\-ны 
 в~разд.~5.
 
 \section{Постановка задачи фильтрации}
 
 На полном вероятностном пространстве с~фильт\-ра\-цией 
 $(\Omega,\mathcal{F},\mathcal{P},\{\mathcal{F}_{t}\}_{t \geqslant 0})$ рассматривается система наблюдений
\begin{equation}
 \left.
 \begin{array}{rl}
 \displaystyle X_t &=X_0 +  \displaystyle
 \int\limits_0^t \Lambda^{\top}X_{s}\,ds + \mu_s\,;  \\[6pt]
 \displaystyle Y_k &=  \displaystyle\int\limits_{t_{k-1}}^{t_k}fX_s\,ds+
 \int\limits_{t_{k-1}}^{t_k} 
 \sum\limits_{n=1}^NX_s^ng_n \,dW_s, \\[6pt]
 &\hspace*{10mm}\{t_k\}_{k \geqslant 0}: \; 0 = t_0 < t_1 < t_2\cdots,
 \end{array}
 \right\}
 \label{eq:obsys_1}
 \end{equation}
 где
  \begin{itemize}
  \item
  $X_t \ebd \mathrm{col}\left(X_t^1,\ldots,X_t^N\right) \hm\in \mathbb{S}^N$~--- 
  ненаблюда\-емое состояние системы, являющееся однородным МСП с~конечным 
  множеством состояний $ \mathbb{S}^N \ebd$\linebreak $\ebd \{e_1,\ldots,e_N\}$ ($\mathbb{S}^N$~--- 
  множество единичных векторов евклидова пространства~$\mathbb{R}^N$), 
  матрицей интенсивностей переходов~$\Lambda$ и~начальным распределением~$\pi$;
  \item
  $\mu_t \ebd \mathrm{col}\left(
  \mu_t^1,\ldots,\mu_t^N\right)\hm\in \mathbb{R}^N$~--- 
  ${\mathcal{F}}_t$-со\-гла\-со\-ван\-ный мартингал;
  \item
  $\{Y_k\}_{k \in \mathbb{N}}:\;  Y_k \ebd \mathrm{col}\left(Y_k^1,\ldots,Y_k^M\right) 
  \hm\in \mathbb{R}^M$~--- последовательность дискретизованных наблюдений, 
  доступных в~известные неслучайные  моменты времени~$\{t_k\}_{k \in \mathbb{N}}$,
в~которых $W_t \ebd$\linebreak $\ebd \mathrm{col}\left(W_t^1,\ldots,W_t^M\right) \hm\in \mathbb{R}^M$
 является ${\mathcal{F}}_t$-со\-гла\-со\-ван\-ным стандартным винеровским процессом, 
 определяющим шумы в~наблюдениях,\linebreak  $f$~--- $(M \times N)$-мер\-ная 
 мат\-ри\-ца плана наблюдений, а~набор мат\-риц~$\{g_n\}_{n=\overline{1,N}}$ 
 характеризует интенсивности шумов в~зависимости от текущего состояния~$X_t$.
  \end{itemize}
  
  Введем также в~рассмотрение неубывающие семейства $\sigma$-ал\-гебр 
  $\mathcal{O}_k \ebd \sigma\{ Y_{\ell}: \; 1 \hm\leqslant \ell \hm\leqslant k\}$ 
  и~$\mathcal{O}_t \ebd  \mathcal{O}_{k(t)}$, где 
  $k(t) \ebd \sum\nolimits_{j \in \mathbb{N}}\mathbf{I}(t-t_{j})$; 
  $\mathcal{O}_0 \ebd \{\varnothing,\; \Omega\}$.
  
   \textit{Задача оптимальной фильтрации состояния~$X$ по наблюдениям~$Y$} 
   заключается в~нахождении \textit{условного математического ожидания} (УМО)
  \begin{equation*}
  \widehat{X}_t \ebd {\sf E}\left\{X_t|\mathcal{O}_{t} \right\}\,.
 % \label{eq:fest_1}
  \end{equation*}
  
  Относительно системы~(\ref{eq:obsys_1})  сделаны следующие предположения:
   \begin{itemize}
 \item[(а)]
 ${\mathcal{F}}_t \equiv {\mathcal{F}}_{t}^X \bigvee 
 {\mathcal{F}}_{t}^W $ для любого $t \hm\geqslant 0$;
 \item[(б)]
 шумы в~наблюдениях равномерно невырожденные, т.\,е.\
  $g_ng_n^{\top} \hm\geqslant \alpha I \hm> 0$ для всех $n\hm=\overline{1,N}$ 
  и~некоторого $\alpha\hm>0$.
% \item
 % Верно неравенство
  %\begin{equation}
  %\min_{1\leqslant k \leqslant N}|\lambda_{kk}| > 0.
  %\label{eq:ineq_0}
  % \end{equation}
 %\item
 %Для любого $t \geqslant 0$ все компоненты вектора $p_t \ebd \me{}{X_t}$ строго %положительны. 
 \end{itemize} 

 \section{Уравнения оптимального фильтра} 
 
 Для получения уравнений оптимального фильт\-ра воспользуемся подходом, 
 применяемым для решения аналогичной задачи в~стохастических сис\-те\-мах 
 наблюдения с~дискретным временем~\cite{BSh_85}. 
 Воспользу\-ем\-ся методом математической индукции. 
 
 При $r=0$ 
 \begin{equation}
 \widehat{X}_{t_0}={\sf E}\{X_0|\mathcal{O}_0\}={\sf E}\{X_0\}=\pi\,.
 \label{eq:in_cond}
 \end{equation} 
 
 Пусть для некоторого $ r \hm\geqslant 0$ известна оценка оптимальной 
 фильтрации~$\widehat{X}_{t_r} \hm= {\sf E}{X_{t_r} |\mathcal{O}_r}$. 
 Определим оценку оптимальной фильтрации~$\widehat{X}_{t} $ для $t\hm \in (t_r,t_{r+1}]$. 
 
 Для произвольного момента $t \hm\in (t_r,t_{r+1})$ в~силу мартингального 
 разложения МСП~$X_t$ и~свойств УМО верна следующая цепочка равенств:
 \begin{multline*}
 \widehat{X}_{t} = {\sf E}\left\{X_t | \mathcal{O}_r\right\}={}\\
 {}=
 {\sf E}\left\{{\cal P}^{\top}(t_r,t)X_{t_r}+
 \int\limits_{t_r}^t{\cal P}^{\top}(t_r,s)\,dM_s\big\vert \mathcal{O}_r\right\} = {}
\end{multline*}

\noindent
   \begin{multline}
 \hspace*{-11.66pt}{}=\mathcal{P}^{\top}(t_r,t)\widehat{X}_{t_r} + {\sf E}\hspace*{-2pt}
 \left\{{\sf E}\hspace*{-2pt}\left\{\int\limits_{t_r}^t\hspace*{-2pt}\mathcal{P}^{\top}(t_r,s)\,dM_s |
 {\mathcal{F}}_{t_r}\right\}\!\big\vert 
 \mathcal{O}_r\!\right\} ={}\hspace*{-4.24124pt}\\
 {}=
  \mathcal{P}^{\top}(t_r,t)\widehat{X}_{t_r}\,,
 \label{eq:bw_obs}
 \end{multline}
 где $\mathcal{P}(s,t)$ $(s \hm\leqslant t)$~--- матрица переходной ве\-ро\-ят\-ности МСП 
 на промежутке $[s,t]$, являющаяся решением сис\-те\-мы дифференциальных 
 уравнений Колмогорова
 \begin{equation*}
 \mathcal{P}'_t(s,t) = \mathcal{P}(s,t) \Lambda, \enskip t > s, \enskip \mathcal{P}(s,s) = I.
 \end{equation*}
 В случае однородного МСП $\mathcal{P}(s,t) \hm= e^{(t-s)\Lambda}$.
 
 Далее необходимо определить совместное распределение $(X_{t_{r+1}},Y_{r+1})$ 
 относительно~$ \mathcal{O}_r$. Из модели наблюдений следует, что 
 распределение~$Y_{r+1}$ относительно 
 $\sigma$-ал\-геб\-ры~$\mathcal{F}^X_{t_{r+1}} \vee \mathcal{O}_r$~---
 гауссовское с~параметрами 
 \begin{align*}
{\sf E}\left\{Y_{r+1}|{\mathcal{F}}^X_{t_{r+1}}\right\}& = f \tau_{r+1}\,; \\[6pt]
 \mathrm{cov} \left(Y_{r+1},Y_{r+1}|{\mathcal{F}}^X_{t_{r+1}}\right) &= 
 \displaystyle\sum\limits_{n=1}^N \tau_{r+1}^n g_ng_n^{\top}\,,
% \label{eq:occup_1}
 \end{align*}
 где $\tau_{r+1} \hm= \tau_{r+1}(X(\omega))=
 \mathrm{col}\left(\tau_{r+1}^1,\ldots,\tau_{r+1}^N\right) \ebd$\linebreak
 $\ebd 
 \int\nolimits_{t_r}^{t_{r+1}}X_s\,ds$~--- случайный вектор, $n$-я 
 компонента которого равна времени пребывания процесса~$X$ в~со\-сто\-янии~$e_n$ 
 на  интервале времени $[t_r, t_{r+1}]$. 
 Обозначим через $\mathcal{D}_{r+1} \ebd \{u=\mathrm{col}\,(u^1,\ldots,u^N):\; 
 u_m \hm\geqslant 0,\; \sum\nolimits_{m=1}^Mu_m\hm= t_{r+1}-t_r\}$ $(M-1)$-мер\-ный 
 симплекс в~пространстве~$\mathbb{R}^M$, являющийся носителем распределения 
 вектора~$\tau_{r+1}$. Пусть $\rho^{k,\ell}_{r+1}(du)$~--- 
 распределение вектора $\tau_{r+1} X_{t_{r+1}}^{\ell}$ при условии $X_{t_r}\hm=e_k$, 
 т.\,е.\ 
 для любого $\mathcal{A} \hm\in \mathcal{B}(\mathbb{R}^M)$ верно тождество:
\begin{multline*}
 \mathbf{P}\left\{\omega: \; X_{t_{r+1}}(\omega)=e_{\ell},\right.\\
 \left. 
 \tau_{r+1}(X(\omega)) \in \mathcal{A}\;|\;X_{t_r}=e_k\right\} \equiv
   \rho^{k,\ell}_{r+1}(\mathcal{A})\,.
\end{multline*}
 
Обозначим через
\begin{multline*}
 \mathcal{N}(y,m,K) \ebd (2\pi)^{-M/2} \mathrm{ det}^{-1/2} K\times{}\\
 {}\times\exp
 \left\{ -\fr{1}{2}\left(y-m\right)^{\top}K^{-1}(y-m)\right\}
\end{multline*}
 $M$-мер\-ную плот\-ность гауссовского распределения с~математическим 
 ожиданием~$m$ и~ковариационной матрицей~$K$.
 
 Из марковского свойства  $\{X_{t_{r}},Y_{r})\}_{r \geqslant 0}$ 
 относительно~${\mathcal{F}}_{t_{r}}$~\cite{ZhSh_95} и~теоремы Фубини следует, что 
 для любого  множества $\mathcal{A} \hm\in \mathcal{B}(\mathbb{R}^M)$ 
 верна следующая цепочка равенств:
 \begin{multline*}
 {\sf E}\left\{X_{t_{r+1}}\mathbf{I}_{\mathcal{A}}
 \left(Y_{r+1}\right)\big|\mathcal{O}_r\right\}={}\\
 {}=
{\sf E}\left\{{\sf E}\left\{X_{t_{r+1}}\mathbf{I}_{\mathcal{A}}
\left(Y_{r+1}\right)\big|
\mathcal{F}^X_{t_{r+1}} \vee \mathcal{O}_r\right\}
 \big|\mathcal{O}_r\right\} = {}
\end{multline*}

\noindent
\begin{multline*}
 %{}=
% {\sf E}\left\{{\sf E}\left\{X_{t_{r+1}}\mathbf{I}_{\mathcal{A}}
% \left(Y_{r+1}\right)\vert X_{t_r}\right\}
% \vert\mathcal{O}_r\right\} = {}\\
% {}=
%{\sf E}\left\{\sum\limits_{k=1}^N {\sf E}\left\{X_{t_{r+1}}\mathbf{I}_{\mathcal{A}}
%\left(Y_{r+1}\right)  \big| X_{t_r}=e_k\right\}X_{t_r}^k
% \big|\mathcal{O}_r\right\} = {}\\ 
% {}=
% \sum\limits_{k=1}^N{\sf E}
% \left\{X_{t_{r+1}}\mathbf{I}_{\mathcal{A}}\left(Y_{r+1}\right)\bigl| X_{t_r}=e_k\right\} 
% \widehat{X}_{t_r}^k ={}\\
% {}=\!
% \sum\limits_{k=1}^N{\sf E}
% \left\{{\sf E}\left\{X_{t_{r+1}}\mathbf{I}_{\mathcal{A}}
% \left(Y_{r+1}\right)\!\bigl| {\mathcal{F}}_{t_{r+1}}\right\}\!\bigl| 
% X_{t_r}\!=e_k\right\} \widehat{X}_{t_r}^k ={}\\
% {}=
% \sum\limits_{k=1}^N {\sf E}\left\{
% \vphantom{\int\limits_A\left(\sum\limits_{p=1}^N\right)}
% X_{t_{r+1}} \times{}\right.\\
% {}\times\int\limits_{\mathcal{A}}  
% \mathcal{N}\left(y,f \tau_{r+1}(X),\sum\limits_{p=1}^N \tau_{r+1}^p(X) g_pg_p^{\top}\right)dy
% \Biggl| X_{t_r}={}\\
%\left. {}=e_k
% \vphantom{\int\limits_A\left(\sum\limits_{p=1}^N\right)}
%\right\} \widehat{X}_{t_r}^k = 
% \sum\limits_{k=1}^N \int\limits_{\mathcal{A}}{\sf E}\left\{ 
% \vphantom{\sum\limits_{p=1}^N}
% X_{t_{r+1}} \times{}\right.\\
% {}\times\mathcal{N}\left(y,f \tau_{r+1}(X),\sum\limits_{p=1}^N \tau_{r+1}^p(X) 
% g_p g_p^{\top}\right)
% \Biggl| X_{t_r}={}\\
%\left. {}=e_k
%\vphantom{\sum\limits^N_{p=1}}
%\right\} \widehat{X}_{t_r}^k\, dy
 %={}\\
 {}=
 \sum\limits_{\ell=1}^N e_{\ell} \int\limits_{\mathcal{A}} 
 \left[ \sum\limits_{k=1}^N 
 \int\limits_{\mathcal{D}_{r+1}} 
 \mathcal{N}\left(y,f u,\sum_{p=1}^N u^p g_pg_p^{\top}\right)\times{}\right.\\
\left. {}\times
 \rho^{k,\ell}_{r+1}(du)\widehat{X}_{t_r}^k
 \vphantom{\int\limits_A\sum\limits_{p=1}^N}
 \right] 
 dy,
 \end{multline*}
 из чего следует, что интегранд в~квадратных скобках в~последнем выражении 
 определяет искомое совместное распределение $(X_{t_{r+1}},Y_{r+1})$ 
 относительно~$ \mathcal{O}_r$. Оценка~$\widehat{X}_{t_{r+1}}$ покомпонентно 
 определяется~\cite{BSh_85} с~помощью обобщенного варианта формулы Байеса:
 \begin{multline}
 \widehat{X}_{t_{r+1}}^j = {}\\
 \hspace*{-1mm}{}=
 \fr{\int\nolimits_{\mathcal{D}_{r+1}}\hspace*{-6mm} 
 \mathcal{N}\left(Y_{r+1},f u,\sum\nolimits_{p=1}^N \hspace*{-2mm}
 u^p g_pg_p^{\top}\!\right)\hspace*{-1mm}
 \sum\nolimits_{k=1}^N \hspace*{-2mm}
 \widehat{X}_{t_r}^k
 \rho^{k,j}_{r+1}(du)
 }
 { \int\nolimits_{\mathcal{D}_{r+1}} \hspace*{-6mm}
 \mathcal{N}\left(Y_{r+1},f v,\sum\nolimits_{q=1}^N \hspace*{-2mm}
 v^q g_qg_q^{\top}\!\right)\hspace*{-1mm}
 \sum\nolimits_{i,\ell=1}^N \hspace*{-2mm}
 \widehat{X}_{t_r}^i
 \rho^{i,\ell}_{r+1}(dv)
  },  \\ 
  j = \overline{1,N}\,.
 \label{eq:filt_1}
 \end{multline}
 Таким образом, доказана следующая
 
 %\smallskip
 
 \noindent
 \textbf{Лемма~1.}
\textit{Если для системы наблюдения}~(\ref{eq:obsys_1}) 
\textit{верны условия~(а) и~(б), то оценка~$\widehat{X}_t$ оптимальной фильтрации 
определяется формулой}~(\ref{eq:in_cond}) 
\textit{при $t\hm=0$, рекуррентным соотношением}~(\ref{eq:filt_1})~---
\textit{в~моменты~$t_{r+1}$ получения наблюдений~$Y_{r+1}$ 
и~формулой}~(\ref{eq:bw_obs})~--- 
\textit{в~промежутках времени между моментами получения наблюдений}.


\smallskip
 

 
 Несмотря на компактную запись~(\ref{eq:filt_1}), их прямая численная реализация 
 ресурсозатратна. Во-пер\-вых, в~(\ref{eq:filt_1}) требуется вычислять 
 распределения мас\-штаб\-но-сдви\-го\-вых смесей многомерных нормальных 
 распределений, что является трудоемкой\linebreak процедурой. Во-вто\-рых, 
 распределения~$\rho^{k,j}_{r+1}$ вре-\linebreak мени пребывания представляют собой 
 сумму\linebreak бесконечного ряда, слагаемые которого вычис\-ляются с~помощью 
 некоторой рекуррентной про\-це\-дуры~\cite{S_00}. В-третьих, 
 распределения~$\rho^{k,j}_{r+1}$ не являются абсолютно непрерывными 
 относительно меры Ле\-бега.
 { %\looseness=1
 
 }
 
 Следующий раздел посвящен численной аппроксимации~(\ref{eq:filt_1}) и~исследованию 
 ее точностных характеристик.
 
 \section{Приближенное вычисление оценки фильтрации}
 
 Без ограничения общности будем считать, что сетка~$\{t_r\}_{r \geqslant 0}$ 
 является равномерной с~шагом~$\Delta$, т.\,е.\ $t_r \hm= r \Delta$ 
 и~$\mathcal{D}_r \hm\equiv \mathcal{D}$.
 Обозначим через~$N_{r+1}$ об-\linebreak\vspace*{-12pt}
 
 \pagebreak
 
 \noindent
 щее число скачков процесса~$X_t$, имевших место 
 на промежутке $(t_r,t_{r+1}]$. Тогда из формулы полной вероятности следует, 
 что~(\ref{eq:filt_1}) представима в~виде:
 \begin{multline}
 \widehat{X}_{t_{r+1}}^j =  \left(
 \int\limits_{\mathcal{D}} 
 \mathcal{N}\left(Y_{r+1},f u,\sum\limits_{p=1}^N u^p g_pg_p^{\top}\right)\times{}\right.\\
\left. {}\times
 \sum\limits_{h=0}^{\infty}\sum\limits_{k=1}^N \widehat{X}_{t_r}^k
 \rho^{k,j,h}_{r+1}(du)
 \right)\Bigg/ \\
 \left(
 \vphantom{\sum\limits_{m=0}^{\infty}
 \sum\limits_{i,\ell=1}^N \widehat{X}_{t_r}^i
 \rho^{i,\ell,m}_{r+1}(dv)}
 \int\limits_{\mathcal{D}} 
 \mathcal{N}\left(Y_{r+1},f v,\sum\limits_{q=1}^N v^q g_qg_q^{\top}\right)\times{}\right.\\
\left.{}\times \sum\limits_{m=0}^{\infty}
 \sum\limits_{i,\ell=1}^N \widehat{X}_{t_r}^i
 \rho^{i,\ell,m}_{r+1}(dv)
 \right)
  \,, \enskip j = \overline{1,N}\,,
  \label{eq:filt_1_1}
 \end{multline}
 где 
 $ \rho^{k,j,h}_{r+1}(du)$~--- распределение вектора 
 $\tau_{r+1}X_{t_{r+1}}^{j}\mathbf{I}_{\{h\}}(N_{r+1})$ при 
 условии $X_{t_r}\hm=e_k$, т.\,е.\ 
 для любого $\mathcal{A} \hm\in \mathcal{B}(\mathbb{R}^M)$ верно тождество
\begin{multline*}
 \mathbf{P}\left\{\omega: \; X_{t_{r+1}}(\omega)=e_{j}, \; N_{r+1} = h,\right.\\ 
\left. \tau_{r+1}(X(\omega)) \in \mathcal{A}\;|\;X_{t_r}=e_k\right\} \equiv
  \rho^{k,j,h}_{r+1}(\mathcal{A}).
\end{multline*}
В качестве аппроксимации оценок можно использовать  
 $\overline{X}_{t_{r+1}}^n \ebd 
 \mathrm{col}\,(\overline{X}_{t_{r+1}}^{n,1},\ldots,\overline{X}_{t_{r+1}}^{n,N})$, 
 полученные из~(\ref{eq:filt_1_1}) путем урезания сумм ряда в~числителе и~знаменателе:
 
 \noindent
 \begin{multline}
 \overline{X}_{t_{r+1}}^{n,j} = 
 \left(
 \int\limits_{\mathcal{D}} 
 \mathcal{N}\left(Y_{r+1},f u,\sum\limits_{p=1}^N u^p g_pg_p^{\top}\right)\times{}\right.\\[-1pt]
\left.{}\times \sum\limits_{h=0}^{n}\sum\limits_{k=1}^N \overline{X}_{t_r}^k
 \rho^{k,j,h}_{r+1}(du)
 \right)\Bigg/ \\[-1pt]
 \left(
 \int\limits_{\mathcal{D}} 
 \mathcal{N}\left(Y_{r+1},f v,\sum\limits_{q=1}^N v^q g_qg_q^{\top}\right)\times{}\right.\\[-1pt]
\left. {}\times
 \sum\limits_{m=0}^{n}
 \sum\limits_{i,\ell=1}^N \overline{X}_{t_r}^i
 \rho^{i,\ell,m}_{r+1}(dv)
  \right)\,, \enskip
   j = \overline{1,N}.
  \label{eq:filt_2}
 \end{multline}
 Ниже по формуле полной вероятности получены интегралы из~(\ref{eq:filt_2}) для 
 $h\hm=0,1,2$:
 
\vspace*{-3pt}

 \noindent
  \begin{multline*}
 \int\limits_{\mathcal{D}}  \mathcal{N}
 \left(Y_{r+1},f u,\sum\limits_{p=1}^N u^p g_pg_p^{\top}\right) 
 \rho^{k,j,0}_{r+1}(du) = {}\\[-1pt]
 {}=
 \delta_{kj}\mathcal{N}\left(Y_{r+1},\Delta f^j,\Delta g_jg_j^{\top}\right)
 e^{\lambda_{jj}\Delta};
 %\label{eq:h0}
\\[-1pt]
 \int\limits_{\mathcal{D}}  \mathcal{N}\left(
 Y_{r+1},f u,\sum\limits_{p=1}^N u^p g_pg_p^{\top}\right) 
 \rho^{k,j,1}_{r+1}(du) ={} 
 \end{multline*}
 
 \noindent
 \begin{multline}
 \hspace*{-6.7pt}{}=\left(1-\delta_{kj}\right)\lambda_{kj}e^{\lambda_{jj}\Delta}
\! \int\limits_0^{\Delta}\!
 e^{(\lambda_{kk}-\lambda_{jj})u^k}
 \mathcal{N}\left(Y_{r+1},u^kf^k +{}\right.\hspace*{-0.28818pt}\\[-1pt]
\hspace*{-3mm}\left. {}+ \left(\Delta - u^k\right)f^j, u^k g_kg_k^{\top}+
 \left(\Delta-u^k\right)g_jg_j^{\top}\right)\,du^k;
 \label{eq:h1}
 \end{multline}
 
 \vspace*{-12pt}
 
 \noindent
 \begin{multline}
 \int\limits_D \mathcal{N}\left( 
Y_{r+1},f u,\sum\limits_{p=1}^N u^p g_pg_p^{\top}\right)du ={}\\[-1pt]
{}=
\sum\limits_{\substack{{\ell:\ell \neq k,}\\ {\ell \neq j}}}
 \lambda_{k\ell}\lambda_{\ell j} e^{\lambda_{jj}\Delta}\times {}\\[-1pt] 
 {}\times
 \int\limits_0^{\Delta} \int\limits_0^{\Delta-u^k} \!
e^{(\lambda_{kk}-\lambda_{\ell\ell})u^k+(\lambda_{\ell\ell}-
 \lambda_{jj})u^{\ell}}\times{} \\[-1pt] 
{}  \times
 \mathcal{N}\left(Y_{r+1},u^k f^k+u^{\ell}f^{\ell}+\left(
 \Delta-u^k-u^{\ell} \right)f^j,\right.\\[-1pt]
 \hspace*{-1mm}\left.
 u^k g_kg_k^{\top}+u^{\ell}g_{\ell}g_{\ell}^{\top}+\left(
 \Delta-u^k-u^{\ell} \right)
 g_jg_j^{\top}
 \right) du^{\ell}du^{k}, \!\!
  \label{eq:h2}
 \end{multline} 
 
\vspace*{-2pt}
 
 \noindent
  где  $\delta_{ij}$~--- символ Кронекера. Интегралы для $h\hm>2$ также могут 
  быть получены в~явном виде, однако их сложность резко возрастает.
 

   Так как система~(\ref{eq:obsys_1}) является автономной, то в~качестве локальной 
   характеристики бли\-зости~$\{\overline{X}_{t_r}\}$ 
   к~$\{\widehat{X}_{t_r}\}$ может быть выбрана величина
   
\noindent
 \begin{multline*}
 \overline{\sigma}(\pi) \ebd {\sf E}\left\{
 \|\widehat{X}_{t_{1}}(\pi, Y_{1}) - \overline{X}_{t_{1}}
 \left(\pi,Y_{1}\right)\|_{1}\right\} = {}\\
 {}=
 \sum\limits_{j=1}^N{\sf E}
 \left\{\left\vert \widehat{X}^j_{t_{1}}\left(\pi, Y_{1}\right) - \overline{X}^{n,j}_{t_{1}}
 \left(\pi,Y_{1}\right)\right\vert\right\}.
 %\label{eq:prec_1}
 \end{multline*}
 При этом начальное распределение $\pi \hm\in \mathcal{D}_1 \ebd $\linebreak $\ebd
 \{\mathrm{col}\,(\pi^1,\ldots,\pi^N):\;\pi^j > 0$, 
 $\sum\nolimits_{j=1}^N\pi^j\hm=1\}$ является начальным условием применения 
 одного шага рекурсии~(\ref{eq:filt_1}) или~(\ref{eq:filt_2}) для вычисления 
 оценки~$\widehat{X}_{t_{1}}$
   или~$\overline{X}_{t_{1}}$ соответственно. Фактически, 
 характеристика~$\overline{\sigma}(\pi)$ определяет, насколько сильно 
 рекурсивные схемы~(\ref{eq:filt_1}) и~(\ref{eq:filt_2}) разойдутся за 
 один шаг, стартуя из общей точки~$\pi$.
 
 Рекуррентные схемы~(\ref{eq:filt_1}) и~(\ref{eq:filt_2}), примененные~$r$~раз, 
 позволяют вычислить оценки~$\widehat{X}_{t_r}$ и~$\overline{X}_{t_r}$ 
 в~точке~$t_r$. В~качестве характеристики точности глобальной аппроксимации в~этом 
 случае естественно рассмотреть величину
 
 \vspace*{-2pt}
 
 \noindent
 \begin{equation*}
 \overline{\Sigma}_{t_r}(\pi) \ebd {\sf E}
 \left\{\|\widehat{X}_{t_{r}} - \overline{X}_{t_{r}}\|_{1}\right\} = 
 \!\sum\limits_{j=1}^N\!{\sf E}
 \left\{\left\vert \widehat{X}^j_{t_{r}} - 
 \overline{X}^{n,j}_{t_{r}}\right\vert \right\}.
% \label{eq:prec_2}
 \end{equation*}
 
 Следующее утверждение определяет оценки локальной и~глобальной 
 точности схемы аппроксимации~(\ref{eq:filt_2}).
 
 %\smallskip
 
 \noindent
 \textbf{Теорема~1.}\
\textit{Выполняются неравенства} 

%\vspace*{-2pt}

\noindent
 \begin{equation}
 \sup_{\pi \in \mathcal{D}_1} \overline{\sigma}(\pi) 
 \leqslant 2 \fr{(\overline{\lambda}\Delta)^{n+1}}{(n+1)!}\,;
 \label{eq:prec_loc}
\end{equation}

\noindent
\begin{align}
  \sup\limits_{\pi \in \mathcal{D}_1} \overline{\Sigma}_{t_r}(\pi)
   &\leqslant 2r \fr{(\overline{\lambda}\Delta)^{n+1}}{(n+1)!} +{}\notag\\[-0.5pt]
   &\hspace*{-20mm}{}+
  r(r-1)\left(
  \fr{(\overline{\lambda}\Delta)^{n+1}}{(n+1)!}
  \right)^2
  \left(
  1-\fr{(\overline{\lambda}\Delta)^{n+1}}{(n+1)!}
  \right)^{r-2},
 \label{eq:prec_glob}
 \end{align}
 
 \vspace*{-2pt}
 
 \noindent
 \textit{где} $\overline{\lambda} \ebd \max_{1 \leqslant j \leqslant N}|\lambda_{jj}|$.


%\smallskip

 Доказательство теоремы~1 приведено в~приложении.
 
 Данное утверждение представляет полезные оценки точности. Во-пер\-вых, 
 они являются равномерными по начальному распределению $\pi \hm\in \mathcal{D}_1$. 
 Во-вто\-рых, оценки носят универсальный, а~не асимптотический характер. Это 
 существенно в~практических задачах оценивания по дискретизованным 
 наблюдениям с~физическими или алгоритмическими ограничениями на шаг 
 по времени. Например, в~случае наблюдаемого процесса восстановления в~силу 
 центральной предельной теоремы для процессов восстановления~\cite{B_80} его
  приращения можно рассматривать как гауссовские случайные величины. 
  Однако данная аппроксимация обладает удовлетворительной точностью 
  только в~случае, когда шаг дискретизации по времени достаточно большой. 
 %
 В-третьих, неравенство~(\ref{eq:prec_glob}) позволяет получить порядок 
 аппроксимации при $\Delta \hm\to 0$. Зафиксируем момент времени $t\hm=T$ и~рассмотрим 
 характеристику $\sup\nolimits_{\pi \in \mathcal{D}_1} 
 \overline{\Sigma}_{T}(\pi)$ при $r\hm={T}/{\Delta}$ и~$\Delta \hm\to 0$. 
 Как только~$\Delta$ становится настолько мало, что 
 $\max\left({(\overline{\lambda}\Delta)^{n+1}}/{(n+1)!}, 
 \Delta ({T\lambda^{n+1}}/{(n+1)!})\right)\hm< 1$, из~(\ref{eq:prec_glob}) 
 следует неравенство
  %\begin{equation}
  $\sup\nolimits_{\pi \in \mathcal{D}_1} \overline{\Sigma}_{T}(\pi) 
  \hm\leqslant  ({3\overline{\lambda}^{n+1}}/{(n+1)!}) T\Delta^n.$
 %\label{eq:prec_asympt}
 %\end{equation}
 Это значит, что с~ростом времени~$T$ 
 ошибка аппроксимации копится пропорционально~$T$ и~при этом порядок точности 
 по~$\Delta$ равен~$n$.
 
 %\vspace*{-7pt}
 
  \section{Заключение}
  
  \vspace*{-4pt}
 
  В работе решена задача оценивания состояния однородного МСП по 
  дискретизованным наблюдениям. Получено аналитическое решение и~его 
  чис\-лен\-ные аппроксимации. Локальные и~глобальные показатели точ\-ности этих 
  приближений в~статье так\-же пред\-став\-ле\-ны. Примечательно, что  част\-ный случай 
  аппроксимаций~(\ref{eq:filt_2}) при $n\hm=0$ и~$\Lambda\hm=0$ был ранее 
  пред\-став\-лен в~\cite{B_17_1,B_17_2} для решения задачи байесовской классификации 
  случайного вектора по непрерывным наблюдениям с~мультипликативными шумами. 
 % 
Алгоритм оптимальной фильт\-ра\-ции и~его субоптимальные версии могут 
рас\-смат\-ри\-вать\-ся в~качестве основы чис\-лен\-ной реализации обобщения фильт\-ра 
Вонэма для сис\-тем с~мультипликативными шумами в~наблюдениях. 
Однако для их непосредственного использования необходимо решить 
следующие проб\-ле\-мы. Во-пер\-вых, в~(\ref{eq:h1}) и~(\ref{eq:h2}) присутствуют
 многомерные интегралы. Следует выяснить, какую результирующую погрешность 
 будут вносить ошибки их вы\-чис\-ле\-ния. Во-вто\-рых, представляется интересным 
 определить характеристики точ\-ности оптимальной фильт\-ра\-ции по дискретизованным 
 наблюдениям по отношению к~оптимальной фильт\-ра\-ции по непрерывным наблюдениям: 
 каков порядок точ\-ности по шагу временной дискретизации~$\Delta$? Для случая 
 вы\-чис\-ле\-ния классического фильт\-ра Вонэма с~по\-мощью алгоритма Эй\-ле\-ра--Ма\-ру\-ямы 
 подобный результат известен: порядок глобальной ошибки равен~${1}/{2}$. 
 Перечисленные задачи являются предметом дальнейших исследований.
 
 
  \vspace*{-10pt}
 
{\small
\subsection*{\raggedleft Приложение} 

\vspace*{-2pt}


\noindent
Д\,о\,к\,а\,з\,а\,т\,е\,л\,ь\,с\,т\,в\,о\ \ теоремы~1.\ \ Введем следующие 
обозначения для случайных величин и~мат\-риц, составленных из них:
\begin{align*}
\xi^{ji}(\ell)&\ebd 
\sum\limits_{h=0}^n \int\limits_{\mathcal{D}} 
 \mathcal{N}\left(Y_{\ell},f u,\sum\limits_{p=1}^N u^p g_pg_p^{\top}\right)
 \rho^{j,i,h}_{1}(du)\,; \\
  \theta^{ji}(\ell)&\ebd 
\sum\limits_{h=n+1}^{\infty} \int\limits_{\mathcal{D}} 
 \mathcal{N}\left(Y_{\ell},f u,\sum\limits_{p=1}^N u^p g_pg_p^{\top}\right)
 \rho^{j,i,h}_{1}(du)\,;
\\
 \xi(\ell)&\ebd \|\xi^{ji}(\ell)\|_{j,i=\overline{1,N}}\,,\quad 
 \Xi(r) \ebd \xi(r) \xi(r-1)\cdots \xi(1)\,;
 \\
 \theta(\ell)&\ebd \|\theta^{ji}(\ell)\|_{j,i=\overline{1,N}}\,, \quad 
 \Theta(r) \ebd \theta(r) \theta(r-1)\cdots \theta(1)\,.
%\label{eq:not_1}
\end{align*}
 
 Рекуррентные формулы~(\ref{eq:filt_1}) и~(\ref{eq:filt_2}) можно записать в~явной 
 форме
 
 
\noindent
\begin{align*}
 \widehat{X}_{t_r}& = \left( \mathbf{1}\left(\Xi(r) + 
 \Theta(r)\right)\pi\right)^{-1} \left(\Xi(r) + \Theta(r)\right)\pi\,;
\\
 \overline{X}_{t_r} &= \left( \mathbf{1}\Xi(r)\pi\right)^{-1} \Xi(r) \pi,
\end{align*}

\vspace*{-2pt}

\noindent
где $\mathbf{1} \ebd (1,\ldots,1)$~--- век\-тор-стро\-ка 
подходящей раз\-мер\-ности, составленная из единиц.

%Далее для краткости записи зависимость от~$r$ в~обозначениях~$\Xi(r)$ 
%и~$\Theta(r)$ будет опущена. 
Верна следующая цепочка неравенств:

 \vspace*{-3pt}

\noindent
\begin{multline}
\overline{\Sigma}_{t_r}(\pi)=%
%\me{}{\left\| 
%\widehat{X}_{t_r}(\pi, Y_1,\ldots,Y_r) - \overline{X}_{t_r}(\pi, Y_1,\ldots,Y_r)
%\right\|_1} =\\=
{\sf E}\left\{\left\| 
\fr{1}{\mathbf{1}\left(\Xi(r) + \Theta(r)\right)\pi} \left(\Xi(r) +{}\right.\right.\right.\\[-1pt]
\left.\left.\left.{}+ \Theta(r)\right)\pi
- \fr{1}{\mathbf{1}\Xi(r)\pi}\,\Xi(r) \pi
\right\|_1\right\} ={} \\[-1pt]
{}=
{\sf E}\left\{\fr{1}{\mathbf{1}\left(\Xi(r) + \Theta(r)\right)\pi \mathbf{1}\Xi(r)\pi}
\left\|
 \mathbf{1}\Xi(r) \pi \Theta(r)\pi -{}\right.\right.\\[-1pt]
\left.\left. {}- \mathbf{1}\Theta(r)\pi \Xi(r) \pi
 \right\|_1
 \vphantom{\fr{1}{\mathbf{1}\left(\Xi(r) + \Theta(r)\right)\pi \mathbf{1}\Xi(r)\pi}}
\right\} \leqslant {}\\[-1pt]
{}\leqslant 
{\sf E}\left\{\fr{1}{\mathbf{1}\left(\Xi(r) + \Theta(r)\right)\pi \mathbf{1}\Xi(r)\pi}
\left(
\mathbf{1}\Xi(r)\pi \| \Theta(r)\pi \|_1 +{}\right.\right.\\[-1pt]
\left.\left.{}+ \mathbf{1}\Theta(r)\pi 
\|
\Xi(r) \pi
\|_1
\right)
 \vphantom{\fr{1}{\mathbf{1}\left(\Xi(r) + \Theta(r)\right)\pi \mathbf{1}\Xi(r)\pi}}
\right\} ={}\\[-1pt]
{}=
2\,{\sf E}\left\{\fr{1}{\mathbf{1}\left(\Xi(r) + \Theta(r)\right)\pi}\mathbf{1}\Theta(r)\pi 
\right\}.
\label{eq:ineq_1}
\end{multline}

 
 \noindent
 Рассмотрим случайные события $a_{\ell} \ebd \{\omega \in \Omega: 
 N_{\ell}(\omega) \hm\leqslant n\}$, $\ell \hm= \overline{1,r}$, и~$A_r \ebd \{
 \omega\hm \in \Omega: \max_{1 \leqslant {\ell} \leqslant r}N_{\ell}(\omega) 
 \hm\leqslant n
 \}\hm=\prod\nolimits_{\ell=1}^r a_{\ell}$ и~оценку 
 $
 \widetilde{X}_{t_r}(\pi, Y_1,\ldots,Y_r)\ebd$\linebreak $\ebd
 {\sf E}\left\{X_{t_r}(\omega)\mathbf{I}_{A_r}(\omega)|\mathcal{O}_r\right\}.
 $
 Используя введенные выше обозначе\-ния и~абстрактный вариант формулы Байеса, 
 получаем, что
 
 \noindent
\begin{align}
\widetilde{X}_{t_r}& = \fr{1}{{\mathbf{1}\left(\Xi(r) + 
 \Theta(r)\right)\pi}}\,\Xi(r)\pi\,;\notag
 \\
\widehat{X}_{t_r} - \widetilde{X}_{t_r} &=
{\sf E}\left\{X_{t_r}(\omega)\mathbf{I}_{\overline{A}_r}(\omega)|\mathcal{O}_r\right\} ={}\notag\\[-1pt]
&\hspace*{17mm}{}= 
\fr{1}{\mathbf{1}\left(\Xi(r) + \Theta(r)\right)\pi}\Theta(r)\pi\,. 
\label{eq:eq_2}
 \end{align}
 Из (\ref{eq:ineq_1}) и~(\ref{eq:eq_2}) для $r\hm=1$ следует, что
 
 \vspace*{-4pt}
 
 \noindent
 \begin{multline}
 \overline{\sigma}(\pi) \leqslant 2\,{\sf E}
 \left\{\|{\sf E}\left\{X_{t_1}(\omega)\mathbf{I}_{\overline{a}_1}(\omega)|\mathcal{O}_1
 \right\}\|_1
 \right\} ={}\\[-1.5pt]
 {}=
 2\,{\sf E}\left\{\sum\limits_{n=1}^N {\sf E}
 \left\{X^n_{t_1}(\omega)\mathbf{I}_{\overline{a}_1}
 (\omega)|\mathcal{O}_1\right\}\right\} ={} \\[-2pt] 
 {}=
  2\,{\sf E}\left\{{\sf E}\left\{\mathbf{I}_{\overline{a}_1}(\omega)|\mathcal{O}_1
  \right\}\right\} =
   2 \mathbf{P}\left\{\overline{a}_1(\omega)\right\}.
\label{eq:ineq_3}
\end{multline}

 \vspace*{-2pt}
 
 \noindent
 Процесс $N^X_t$ общего числа скачков состояния~$X_t$ является считающим, и~его
  квадратическая характеристика равна 
  
\vspace*{-2pt}
  
  \noindent
 $$
 \langle N^X, N^X\rangle_t = - \int\limits_0^t \sum\limits_{n=1}^N \lambda_{nn} X_s^n\,ds\,,
 $$
 поэтому искомая вероятность ограничена сверху:
 $$ 
 \mathbf{P}\left\{\overline{a}_1(\omega)\right\} \leqslant 
 e^{-\overline{\lambda}\Delta}\sum\limits_{k=n+1}^{\infty} 
 \fr{(\overline{\lambda}\Delta)^{k}}{k!} <
 \fr{(\overline{\lambda}\Delta)^{n+1}}{(n+1)!}.
 $$
 
  \vspace*{-2pt}
  
  \noindent
 Из последнего неравенства и~(\ref{eq:ineq_3}) следует, что  для любого 
 начального распределения~$\pi$ выполняется неравенство $\overline{\sigma}(\pi)  
 \hm< 2({(\overline{\lambda}\Delta)^{n+1}}/{(n+1)!})$, т.\,е.\ 
 локальная оценка~(\ref{eq:prec_loc}) верна.
 
 С помощью марковского свойства пары $(X_t, N^X_t)$ и~последнего 
 неравенства можно оценить сверху вероятность 
 $\mathbf{P}\left\{\overline{A}_r(\omega)\right\}$:
 
  \vspace*{-2pt}
 
 \noindent
 \begin{multline*}
 \mathbf{P}\left\{\overline{A}_r(\omega)\right\} \leqslant 1 - \left(
 1- \fr{(\overline{\lambda}\Delta)^{n+1}}{(n+1)!}
 \right)^r \leqslant r \fr{(\overline{\lambda}\Delta)^{n+1}}{(n+1)!} + {}\\[-1pt]
 {}+\left|
 \sum\limits_{k=2}^r C_r^k \left(-\fr{(\overline{\lambda}\Delta)^{n+1}}{(n+1)!}
 \right)^k
 \right| \leqslant
 r \fr{(\overline{\lambda}\Delta)^{n+1}}{(n+1)!} +{}\\[-1pt]
 {}+\fr{r(r-1)}{2}
 \left(
 \fr{(\overline{\lambda}\Delta)^{n+1}}{(n+1)!}
 \right)^2
 \left(
 1-\fr{(\overline{\lambda}\Delta)^{n+1}}{(n+1)!}
 \right)^{r-2},
 \end{multline*} 
 из чего следует истинность глобальной оценки~(\ref{eq:prec_glob}).
Теорема~1 доказана.

}

%\vspace*{-12pt}

{\small\frenchspacing
 {%\baselineskip=10.8pt
 \addcontentsline{toc}{section}{References}
 \begin{thebibliography}{99}

\bibitem{Won_65}
\Au{Wonham W.} 
Some applications of stochastic differential equations to optimal
  nonlinear filtering~//
SIAM~J.~Control, 1965. Vol.~2. P.~347--369. 

\bibitem{KP_92}
\Au{Kloeden P., Platen E.} Numerical solution of stochastic
differential equations.~--- Berlin: Springer, 1992.~636~p.

\bibitem{YZL_04}
\Au{Yin G., Zhang Q., Liu Y.} 
Discrete-time approximation of Wonham filters~//
J.~Control Theory Applications, 2004. Iss.~2. P.~1--10.

\bibitem{PR_10}
\Au{Platen E., Rendek R.}
Quasi-exact approximation of hidden Markov chain filters~//
Communicat.~Stoch.~Analys., 2010. Vol.~4. Iss.~1. P.~129--142.

\bibitem{B_18}
\Au{Борисов А.} Фильтрация Вонэма по наблюдениям с~мультипликативными шумами~// 
Автоматика и~телемеханика, 2018.
№~1. C.~52--65. 
 
  \bibitem{BSh_85} %6
\Au{Бертсекас Д., Шрив С.} Стохастическое оптимальное управление. 
Случай дискретного времени~/ Пер. с~англ.~--- М.: Наука, 1985.~280~c.
(\Au{Betsekas~D.\,P., Shreve~S.\,E.} Stochastic optimal control:
The discrete-time case.~--- Orlando, FL, USA:
Academic Press Inc., 1978. 323~p.)

  \bibitem{ZhSh_95} %7
\Au{Жакод Ж., Ширяев А.} Предельные теоремы для случайных процессов,~I.~/
Пер. с~англ.~--- 
М.: Физматлит, 1995.~544~c.
(\Au{Jacod~J., Shiryaev~A.} Limit theorems for stochastic processes.~---
Berlin: Springer, 2003. 664~p.)

\bibitem{S_00}
\Au{Sericola B.} Occupation times in Markov processes~//
Commun. Stat. Stochastic Models, 2000. Vol.~16. Iss.~5. P.~479--510. 

  \bibitem{B_80}
\Au{Боровков А.} Асимптотические методы в~тео\-рии массового обслуживания.~--- 
М.: Физматлит, 1995.~384~c.

  \bibitem{B_17_1}
\Au{Борисов А.} Классификация по непрерывным наблюдениям с~мультипликативными шумами.~I. 
Формулы байесовской оценки~// Информатика и~её применения, 2017. Т.~11. Вып.~1. C.~11--19.
doi: 10.14357/19922264170102.

  \bibitem{B_17_2}
\Au{Борисов А.} Классификация по непрерывным наблюдениям с~мультипликативными 
шумами.~II. Алгоритм численной реализации оценки~// Информатика и~её 
применения, 2017. Т.~11. Вып.~2. C.~33--41.
doi: 10.14357/19922264170204.

 \end{thebibliography}

 }
 }

\end{multicols}

\vspace*{-4pt}

\hfill{\small\textit{Поступила в~редакцию 10.07.18}}

\vspace*{6pt}

%\pagebreak

%\newpage

%\vspace*{-28pt}

\hrule

\vspace*{2pt}

\hrule

%\vspace*{-2pt}

\def\tit{FILTERING OF~MARKOV JUMP PROCESSES\\ BY~DISCRETIZED OBSERVATIONS}

\def\titkol{Filtering of Markov jump processes by discretized observations}

\def\aut{A.\,V.~Borisov}

\def\autkol{A.\,V.~Borisov}

\titel{\tit}{\aut}{\autkol}{\titkol}

\vspace*{-11pt}


\noindent
Institute of Informatics Problems, Federal Research Center ``Computer Science 
and Control'' of the Russian Academy of Sciences, 44-2~Vavilov Str., Moscow 
119333, Russian Federation


\def\leftfootline{\small{\textbf{\thepage}
\hfill INFORMATIKA I EE PRIMENENIYA~--- INFORMATICS AND
APPLICATIONS\ \ \ 2018\ \ \ volume~12\ \ \ issue\ 3}
}%
 \def\rightfootline{\small{INFORMATIKA I EE PRIMENENIYA~---
INFORMATICS AND APPLICATIONS\ \ \ 2018\ \ \ volume~12\ \ \ issue\ 3
\hfill \textbf{\thepage}}}

\vspace*{6pt}



\Abste{The article is devoted to a~solution of the optimal filtering problem 
of a~homogenous Markov
jump process state. The available observations represent 
time increments of the integral transformations of the Markov\linebreak\vspace*{-12pt}}

\Abstend{state corrupted by 
Wiener processes. The noise intensity is also state-dependent. At the instant of 
the consecutive
observation obtaining, the optimal estimate is calculated recursively 
as a~function of previous estimate and the new observation, meanwhile between 
observations the filtering estimate is a simple forecast by virtue of the Kolmogorov 
differential system. The recursion is rather expensive because of  need to calculate 
the integrals, which are the location-scale mixtures of Gaussians. The mixing 
distributions represent the occupation of the state in each of possible values 
during the mid-observation intervals. The paper contains numerically cheaper 
approximations, based on the restriction of the state transitions number between 
the observations. Both the local and global characteristics of approximation 
accuracy are obtained as functions of the dynamics parameters, mid-observation 
interval length, and upper bound of transitions number.}

\KWE{Markov jump process; optimal filtering; multiplicative observation noises; 
stochastic differential equation; numerical approximation}




\DOI{10.14357/19922264180316}

%\vspace*{-14pt}

\Ack
\noindent
The work was supported in part by the Russian Foundation
for Basic Research (Project No.\,16-07-00677).



%\vspace*{6pt}

  \begin{multicols}{2}

\renewcommand{\bibname}{\protect\rmfamily References}
%\renewcommand{\bibname}{\large\protect\rm References}

{\small\frenchspacing
 {%\baselineskip=10.8pt
 \addcontentsline{toc}{section}{References}
 \begin{thebibliography}{99}
\bibitem{Won_65-1}
\Aue{Wonham, W.} 1965.
Some applications of stochastic differential equations to optimal
  nonlinear filtering.
\textit{SIAM~J.~Control} 2:347--369. 

\bibitem{KP_92-1}
\Aue{Kloeden,~P., and E.~Platen.} 1992. \textit{Numerical solution of stochastic
differential equations.} Berlin: Springer. 636~p.

\bibitem{YZL_04-1}
\Aue{Yin,~G., Q.~Zhang, and Y.~Liu.} 2004.
Discrete-time approximation of Wonham filters.
\textit{J.~Control Theory Applications} 2:1--10.

\bibitem{PR_10-1}
\Aue{Platen, E., and R.~Rendek.} 2010.
Quasi-exact approximation of hidden Markov chain filters.
\textit{Communicat. Stoch. Analys.} 4(1):129--142.

\bibitem{B_18-1}
\Aue{Borisov, A.} 2018. Wonham filtering by observations
with multiplicative noises. \textit{Automat.~Rem.~Contr.} 79(1):39--50.  
doi: 10.1134/ S0005117918010046.
 
  \bibitem{BSh_85-1}
\Aue{Bertsekas, D., and S.~Shreve.} 1996.
\textit{Stochastic optimal control: The discrete-time case}.
Nashua, NH: Athena Scientific. 330~p.
  
  \bibitem{ZhSh_95-1}
  \Aue{Jacod,~J., and A.~Shiryaev.} 2003.
\textit{Limit theorems for stochastic processes.}
Berlin: Springer. 664~p.

\bibitem{S_00-1}
\Aue{Sericola, B.}
2000. Occupation times in Markov processes.
\textit{Commun. Stat.} 16(5):479--510. 

  \bibitem{B_80-1}
\Aue{Borovkov, A.} 1984.
 \textit{Asymptotic methods in queueing theory}. 
 Hoboken, NJ: Wiley-Blackwell.~304~p.

  \bibitem{B_17_1-1}
  \Aue{Borisov, A.} 2017. 
  Klassifikatsiya po ne\-pre\-ryv\-nym nablyu\-de\-miyam s~mul'tiplikativnymi shumami. I. 
  Formuly bayesov\-skoy otsenki [Classification by continuous-time observations
in multiplicative noise. I.~Formulae for Bayesian 
estimate]. \textit{Informatika i~ee Primeneniya~--- Inform.~Appl.}
11(1):11--19. doi: 10.14357/19922264170102.

  \bibitem{B_17_2-1}
\Aue{Borisov, A.} 2017. Klassifikatsiya po nepreryvnym nablyudemiyam 
s~mul'tiplikativnymi summami. II.~Formuly bayesovskoy otsenki 
[Classification by continuous-time observations
in multiplicative noise. II.~Numerical algorithm].
\textit{Informatika i~ee Primeneniya~--- Inform.~Appl.}
11(2):33--41. doi: 10.14357/19922264170204.

\end{thebibliography}

 }
 }

\end{multicols}

\vspace*{-6pt}

\hfill{\small\textit{Received July 10, 2018}}

%\pagebreak

%\vspace*{-18pt}

\Contrl

\noindent
\textbf{Borisov Andrey V.} (b.\ 1965)~--- 
Doctor of Science in physics and mathematics, principal scientist, Institute of
Informatics Problems, Federal Research Center ``Computer Science and Control''
 of the Russian Academy of
Sciences, 44-2 Vavilov Str., Moscow 119333, Russian Federation; 
\mbox{aborisov@frccsc.ru}
\label{end\stat}

\renewcommand{\bibname}{\protect\rm Литература}         %16






 


%%%%%%%%%%%%%%%%%%%%%%%%%%%%%%%%%%%%%%%%%%%%%%%

%\def\stat{rez}
{%\hrule\par
%\vskip 7pt % 7pt
\raggedleft\Large \bf%\baselineskip=3.2ex
Р\,Е\,Ц\,Е\,Н\,З\,И\,И \vskip 17pt
    \hrule
    \par
\vskip 6pt plus 6pt minus 3pt }

%\thispagestyle{headings} %с верхним колонтитулом
%\thispagestyle{myheadings} %с нижним колонтитулом, но в верхнем РЕЦЕНЗИИ

\def\tit{НОВАЯ КНИГА И.\,Н.~СИНИЦЫНА, А.\,С.~ШАЛАМОВА <<ЛЕКЦИИ ПО ТЕОРИИ 
ИНТЕГРИРОВАННОЙ ЛОГИСТИЧЕСКОЙ ПОДДЕРЖКИ>> (М.: ТОРУС ПРЕСС, 2012. 624~с.)}

%1
\def\aut{Д.ф.-м.н., профессор С.\,Я.~Шоргин}

\def\auf{\ }

\def\leftkol{\ % РЕЦЕНЗИИ
}

\def\rightkol{ \ } 

%\def\leftkol{\ } % ENGLISH ABSTRACTS}

%\def\rightkol{\ } %ENGLISH ABSTRACTS}

%\def\leftkol{РЕЦЕНЗИИ}

%\def\rightkol{РЕЦЕНЗИИ}

\titele{\tit}{\aut}{\auf}{\leftkol}{\rightkol}
\vspace*{-18pt}


     \label{st\stat}

     \begin{multicols}{2}
     {\small
     {\baselineskip=10.1pt
     

      В книге представлено системное изложение теоретических основ одного из новейших 
направлений в \mbox{об\-ласти} экономики послепродажного обслуживания изделий наукоемкой 
продукции (ИНП) длительного пользования~--- интегрированной логистической поддержки
(ИЛП). 
{\looseness=1

}

Приведены также результаты новых работ, выполненных в Институте проблем информатики 
Российской академии наук в рамках научного направления <<Информационные технологии и 
анализ сложных сис\-тем>>.
 {%\looseness=1

}
     
      Излагаемые в книге научные подходы позво\-ляют карди\-наль\-но реформировать 
существующие системы производства и эксплуатации ИНП путем создания и внед\-ре\-ния 
методов рационального и оптимального управ\-ле\-ния процессами расходования 
вре\-мен\-н$\acute{\mbox{ы}}$х, 
мате\-ри\-аль\-ных, трудовых и других ресурсов на всех стадиях жизненного цикла изделий (ЖЦИ) по 
критериям экономической целесообразности и эф\-фек\-тив\-ности.
  {\looseness=1

}
    
      В книге приведен краткий обзор причин возник\-новения и
      развития CALS-методологии как основы 
современных международных стандартов по созданию и функционированию глобальных 
ин\-фор\-ма\-ци\-он\-но-ком\-му\-ни\-ка\-ци\-он\-ных систем, ее ключевых возможностей и эффективности 
результатов ее использования. 
Авторы %\linebreak 
предлагают ряд научных обоснований для разработки 
единой теории проектирования и управления систем ИЛП для полноценного использования 
преимуществ %\linebreak
 суще\-ст\-ву\-ющей методологии, определяют \mbox{общую} структурную схему 
комплексной системы <<ИНП-СППО>> и необходимость разработки для ее описания 
гибридных стохастических моделей.
{%\looseness=1

}

%\columnbreak
      
      Книга состоит из пяти частей, где последовательно излагается материал по каждой из 
следующих тем: <<Интегрированная логистическая поддержка>>, <<Теория гибридных 
стохастических систем и компьютерная поддержка исследований и разработок>>, <<Основы 
математического моделирования, анализа и синтеза систем послепродажного обслуживания>>, 
<<Определение и анализ показателей экспортного потенциала ИНП при проектировании>>, 
<<Задачи управления поддержкой послепродажного обслуживания>>, а также 
<<Моделирование инвестиционных процессов ИЛП в условиях неравновесных финансовых 
рынков>>. 
   
      В конце каждой главы приведены выводы и даны вопросы и задания для 
самоконтроля. В~приложениях содержатся основные определения по программам работ по 
анализу ИЛП, логистическим базам данных и компьютерным решениям, эквивалентной статистической 
линеаризации нелинейных преобразований ИЛП, справочный материал, а также развернутые 
уравнения для вероятностных характеристик.


      \def\leftkol{РЕЦЕНЗИИ}

\def\rightkol{РЕЦЕНЗИИ} 

      
      Книга заинтересует широкий круг специалистов и может быть использована научными 
проектными организациями в сфере промышленного производства ИНП. Большое количество 
иллюстраций, примеров и вопросов, обращенных к читателю, позволяет использовать книгу 
также в качестве учебного пособия для студентов и аспирантов машиностроительных, 
транспортных и~других специальностей, а также для самостоятельного изучения. 
{%\looseness=-1

}

Книга 
представляет несомненный интерес для специалистов и студентов в области прикладной 
математики и информатики.
    

}

}
\end{multicols}

%\newpage

\def\stat{authorsrus}
{%\hrule\par
%\vskip 7pt % 7pt
\raggedleft\Large \bf%\baselineskip=3.2ex
О\,Б\ \ А\,В\,Т\,О\,Р\,А\,Х \vskip 17pt
    \hrule
    \par
\vskip 21pt plus 8pt minus 4pt }


\def\tit{\ }

\def\aut{\ }

\def\auf{\ }

\def\leftkol{\ } % ENGLISH ABSTRACTS}

\def\rightkol{ОБ АВТОРАХ} %ENGLISH ABSTRACTS}

\titele{\tit}{\aut}{\auf}{\leftkol}{\rightkol}
      
            \label{st\stat}



\vspace*{24pt}

\begin{multicols}{2}




\noindent
\textbf{Архипов Олег Петрович} (р.\ 1948)~---
кандидат технических наук, директор Орловского филиала Института проб\-лем информатики
Российской академии наук
%302025, г.Орел, Московское шоссе, д.137

\vspace*{3pt}

\noindent
\textbf{Бирюкова Татьяна Константиновна} (р.\ 1968)~---
кандидат фи\-зи\-ко-ма\-те\-ма\-ти\-че\-ских наук, старший научный сотрудник Института проб\-лем информатики
Российской академии наук

\vspace*{3pt}

\noindent 
\textbf{Бобков  Сергей Геннадьевич} (р.\ 1955)~---
доктор технических наук,  заведующий отделением На\-уч\-но-ис\-сле\-до\-ва\-тель\-ско\-го 
института системных исследований Российской академии наук
%117218, Москва, Нахимовский просп., 36, к.1 

\vspace*{3pt}

\noindent \textbf{Васильев Николай Семенович} (р.\ 1952)~--- доктор 
фи\-зи\-ко-ма\-те\-ма\-ти\-че\-ских наук, профессор, 
МГТУ им.\ Н.\,Э.~Баумана 
%, Москва 105005, 2-я Бауманская ул., д.~5,

\vspace*{3pt}

\noindent
\textbf{Гершкович Максим Михайлович} (р.\ 1968)~---
старший научный сотрудник Института проб\-лем информатики
Российской академии наук

\vspace*{3pt}

\noindent 
\textbf{Дьяченко Юрий Георгиевич} (р.\ 1958)~--- кандидат технических наук, 
старший научный сотрудник Института проб\-лем информатики
Российской академии наук

\vspace*{3pt}

\noindent 
\textbf{Ерошенко Александр Андреевич} (р.\ 1989)~--- аспирант кафедры 
математической статистики факультета вычисли\-тельной математики и кибернетики 
Московского государственного университета им.\ М.\,В.~Ломоносова
%119991, Москва ГСП-1, Ленинские горы, д.\ 1, стр. 52

\vspace*{3pt}
 
\noindent 
\textbf{Захаров Виктор Николаевич} (р.\ 1948)~--- 
доктор технических наук, доцент, ученый секретарь Института проб\-лем информатики
Российской академии наук

\vspace*{3pt}

\noindent
\textbf{Зейфман Александр Израилевич} (р.\ 1954)~---
доктор фи\-зи\-ко-ма\-те\-ма\-ти\-че\-ских наук, профессор, 
заведующий кафедрой Вологодского государственного университета; 
старший научный сотрудник Института проб\-лем информатики
Российской академии наук; главный научный сотрудник ИСЭРТ Российской академии наук

\vspace*{3pt}

\noindent
\textbf{Зыкин Сергей Владимирович} (р.\ 1959)~--- 
доктор технических наук, профессор, заведующий лабораторией Института математики 
им.\ С.\,Л.~Соболева Сибирского отделения Российской академии наук, Новосибирск 
%630090, пр.\ ак.\ Коптюга, 4 

\vspace*{4pt}

\noindent
\textbf{Киреев Владимир Иванович} (р.\ 1938)~---
доктор фи\-зи\-ко-ма\-те\-ма\-ти\-че\-ских наук, профессор Московского 
государственного горного университета
%Адрес: Россия, 119991, г. Москва, Ленинский проспект, д. 6

%\columnbreak

\vspace*{4pt}

\noindent
\textbf{Козеренко Елена Борисовна} (р.\ 1959)~---
кандидат филологических наук, заведующая лабораторией Института проб\-лем информатики
Российской академии наук

\vspace*{4pt}

\noindent
\textbf{Королев Виктор Юрьевич} (р.\ 1954)~--- доктор
фи\-зи\-ко-ма\-те\-ма\-ти\-че\-ских наук, профессор кафедры математической 
статистики факультета вычисли\-тельной математики и кибернетики 
Московского государственного университета; 
ведущий научный сотрудник Института проб\-лем информатики
Российской академии наук

\vspace*{4pt}

\noindent
\textbf{Коротышева Анна Владимировна} (р.\ 1988)~---
старший преподаватель Вологодского государственного университета

\vspace*{4pt}

\noindent 
\textbf{Кун Де Турк} (р.\ 1981)~--- научный сотрудник 
исследовательской группы SMACS факультета телекоммуникаций и обработки информации
Университета Гента, Бельгия
%В-9000 Гент, Бельгия

\vspace*{4pt}

\noindent
\textbf{Лупенцов Олег Сергеевич} (р.\ 1986)~---
аспирант Омского государственного института сервиса
%Омск 644043, ул.\ Певцова 13

\vspace*{4pt}

\noindent
\textbf{Лучко Олег Николаевич} (р.\ 1961)~---
кандидат педагогических наук, профессор, заведующий кафедрой 
Омского государственного института сервиса
%Омск 644043, ул.\ Певцова 13

\vspace*{4pt}

\noindent
\textbf{Малашенко Юрий Евгеньевич} (р.\ 1946)~---
доктор фи\-зи\-ко-ма\-те\-ма\-ти\-че\-ских наук, заведующий сектором 
Вычислительного центра им.\ А.\,А.~Дородницына Российской академии наук
%Адрес: 119333, Москва, ул. Вавилова, 40,

\vspace*{4pt}

\noindent
\textbf{Маньяков Юрий Анатольевич} (р.\ 1984)~---
кандидат технических наук, научный сотрудник Орловского филиала Института проб\-лем информатики
Российской академии наук
%302025, г.Орел, Московское шоссе, д.137

\vspace*{4pt}

\noindent
\textbf{Маренко Валентина Афанасьевна} (р.\ 1951)~---
кандидат технических наук, доцент, старший научный сотрудник 
Института математики им.\ С.\,Л.~Соболева Сибирского отделения Российской академии наук
%Новосибирск 630090, пр. ак. Коптюга, 4 

\vspace*{3pt}

\noindent 
\textbf{Морозов Евсей Викторович} (р.\ 1947)~--- доктор 
фи\-зи\-ко-ма\-те\-ма\-ти\-че\-ских, профессор, ведущий научный сотрудник 
Института прикладных математических исследований Карельского научного центра Российской
академии наук; 
%%185910 Россия, Республика Карелия, г.\ Петрозаводск, ул.\ Пушкинская, 11
профессор Петрозаводского государственного университета, Петрозаводск
%185910 Россия, Республика Карелия, г.\ Петрозаводск, пр.\ Ленина, 33

%\pagebreak

\vspace*{3pt}

\noindent
\textbf{Назарова Ирина Александровна} (р.\ 1966)~---
кандидат фи\-зи\-ко-ма\-те\-ма\-ти\-че\-ских наук, 
научный сотрудник Вычислительного центра им.\ А.\,А.~Дородницына Российской академии наук 
%Адрес: 119333, Москва, ул. Вавилова, 40

\vspace*{3pt}

\noindent
\textbf{Павлов Игорь Валерианович} (р.\ 1945)~--- 
доктор фи\-зи\-ко-ма\-те\-ма\-ти\-че\-ских наук, профессор МГТУ им.\ Н.\,Э.~Баумана 
%Москва 105005, 2-я Бауманская ул., д.~5 

%\pagebreak

\vspace*{3pt}

\noindent 
\textbf{Потахина Любовь Викторовна} (р.\ 1989)~--- аспирантка
Института прикладных математических исследований Карельского научного центра
Российской академии наук; 
%%185910 Россия, Республика Карелия, г.\ Петрозаводск, ул.\ Пушкинская, 11
инженер Петрозаводского государственного университета, Петрозаводск
%185910 Россия, Республика Карелия, г.\ Петрозаводск, пр.\ Ленина, 33

\vspace*{3pt}

\noindent 
\textbf{Рождественский Юрий Владимирович} (р.\ 1952)~--- 
кандидат технических наук, заведующий сектором Института проб\-лем информатики
Российской академии наук

\vspace*{3pt}

\noindent 
\textbf{Синицын Игорь Николаевич} (р.\ 1940)~--- доктор технических наук,
профессор, заслуженный деятель\linebreak\vspace*{-12pt}

\columnbreak

\noindent
 науки РФ, заведующий отделом Института проб\-лем информатики
Российской академии наук

\vspace*{7pt}


\noindent
\textbf{Сиротинин Денис Олегович} (р.\ 1984)~---
кандидат технических наук, научный сотрудник Орловского филиала Института проб\-лем информатики
Российской академии наук
%302025, г.Орел, Московское шоссе, д.137

\vspace*{7pt}

%\columnbreak

\noindent 
\textbf{Соколов  Игорь Анатольевич} (р.\ 1954)~--- академик (действительный член) Российской 
академии наук, доктор технических наук, директор Института проб\-лем информатики
Российской академии наук

\vspace*{7pt}

\noindent
\textbf{Степченков Юрий Афанасьевич} (р.\ 1951)~---
кандидат технических наук, заведующий отделом Института проб\-лем информатики
Российской академии наук

\vspace*{7pt}

\noindent
\textbf{Сурков Алексей Викторович} (р.\ 1978)~--- 
старший научный сотрудник На\-уч\-но-ис\-сле\-до\-ва\-тель\-ско\-го 
института системных исследований Российской академии наук
%117218, Москва, Нахимовский просп., 36, к.1 

\vspace*{7pt}

\noindent 
\textbf{Шестаков Олег Владимирович} (р.\ 1976)~--- доктор 
фи\-зи\-ко-ма\-те\-ма\-ти\-че\-ских, доцент кафедры математической статистики 
факультета вычисли\-тельной математики и кибернетики Московского 
государственного университета им.\ М.\,В.~Ломоносова; 
%119991, Москва ГСП-1, Ленинские горы, д.\ 1, стр. 52
старший научный сотрудник Института проб\-лем информатики
Российской академии наук
%, Москва 119333, ул. Вавилова, д.~44, корп.~2

\vspace*{7pt}

\noindent 
\textbf{Шоргин Сергей Яковлевич} (р.\ 1952.)~--- доктор
фи\-зи\-ко-ма\-те\-ма\-ти\-че\-ских наук, профессор, заместитель директора Института 
проб\-лем информатики Российской академии наук





%%%%%%%%%%%%%%%%%%%%%%%%%%%%%%%%%%%%%%%%%%%%%%%%%%%%%%%%%%%%%%%%%%%%%%%%%%%%%%%




%\def\rightkol{ОБ АВТОРАХ}
%\def\leftkol{ОБ АВТОРАХ}

 \label{end\stat}





%\def\leftfootline{\small{\textbf{\thepage}
%\hfill ИНФОРМАТИКА И ЕЁ ПРИМЕНЕНИЯ\ \ \ том~7\ \ \ выпуск~1\ \ \ 2013}
%}%
% \def\rightfootline{\small{ИНФОРМАТИКА И ЕЁ ПРИМЕНЕНИЯ\ \ \ том~7\ \ \ выпуск~1\ \ \ 2013
%\hfill \textbf{\thepage}}}


%\thispagestyle{myheadings}



\end{multicols}

\newpage

%\end{document}

%
\def\stat{rekl}
%\label{preobr}

%\def\tit{АКАДЕМИК ПУГАЧЁВ  ВЛАДИМИР СЕМЁНОВИЧ\\
%25.03.1911--25.03.1998}


%   \vspace*{-48pt}
%   \begin{center}\LARGE
%Академик Пугачёв  Владимир Семёнович\\ (25.03.1911--25.03.1998)
%   \end{center}

   %\vspace*{2.5mm}

   \begin{center}

{\prgsh\LARGE
ЮБИЛЕИ}

\end{center}
%\hrule

\vspace*{6pt}


   \vspace*{8mm}

   \thispagestyle{empty}


%\def\stat{emel}


\section*{К 70-летию заместителя директора ИПИ РАН,\\ члена редколлегии журнала
<<Информатика и её применения>>\\ доктора технических наук В.\,И.~Будзко}

\vspace*{18pt}




          \begin{multicols}{2}

%            \label{st\stat}

\begin{center}
\vspace*{1pt}
\mbox{%
\epsfxsize=78mm
\epsfbox{bud-1.eps}
}
\end{center}

\vspace*{12pt}

      14 августа 2014~г.\ исполнилось 70~лет за\-мес\-ти\-те\-лю директора ИПИ РАН по
научной работе доктору технических наук Владимиру Игоревичу Будзко.

      Владимир Игоревич Будзко родился в г.~Москве. Высшее образование получил на факультете
элект\-рон\-но-вы\-чис\-ли\-тель\-ных устройств в Московском
ин\-же\-нер\-но-фи\-зи\-че\-ском институте
(МИФИ), который он окончил в 1968~г., после чего был на\-прав\-лен для прохождения
службы в одну из войс\-ко\-вых частей, где прошел путь от инженера до первого заместителя
командира войсковой части.

      С приходом В.\,И.~Будзко в ИПИ РАН (2001~г.)\ в институте
сформировалось новое научное на\-прав\-ле\-ние теоретических исследований~--- <<Постро\-ение
ин\-фор\-ма\-ци\-он\-но-те\-ле\-ком\-му\-ни\-ка\-ци\-он\-ных\linebreak сис\-тем
высокой до\-ступ\-ности>>. В~рамках этого
направления выполнен широкий круг фундаментальных исследований по поиску подходов и
определению принципов построения средств обеспечения доступности, конфиденциальности
и целостности современных крупномасштабных
ин\-фор\-ма\-ци\-он\-но-те\-ле\-ком\-му\-ни\-ка\-ци\-он\-ных
сис\-тем (ИТС). Разработаны основные сис\-тем\-но-тех\-ни\-че\-ские принципы и базовые
архитектурные решения построения перспективных для условий России ИТС с
централизованной обработкой и хранением информации, сочетающих в себе свойства
высокой доступности, отказо- и катастрофоустойчивости, информационной защищенности.
Определены принципы, методы и математические основы рационального построения и
оптимизации средств восстановления функционирования центров обработки данных (ЦОД)
после возникновения отказов и катастроф, передачи и хранения данных, обеспечения
информационной безопасности при достижении минимальной совокупной стоимости
владения такими системами. Результаты нашли практическое воплощение при реализации
проектов в интересах ряда отечественных государственных и негосударственных
организаций, таких как Банк России (БР), Внешторгбанк, ОАО <<ГМК <<Норильский Никель>>,
<<Газпром>>, Минэкономразвития России, Правительство Москвы, а также ряд силовых
ведомств.

      Под руководством В.\,И.~Будзко начиная с 2001~г.\ выполнен комплекс
      на\-уч\-но-ис\-сле\-до\-ва\-тель\-ских и
      опыт\-но-кон\-ст\-рук\-тор\-ских работ (свыше 100~проектов),
направленных на развитие электронной информационной технологии БР.
Разработаны концепции развития ИТС БР сначала до 2008~г., а затем до 2013~г., которые
были приняты в качестве основы проведения технической политики. За реализацию проекта
<<Катастрофоустойчивая тер\-ри\-то\-ри\-аль\-но-рас\-пре\-де\-лен\-ная
      ин\-фор\-ма\-ци\-он\-но-те\-ле\-ком\-му\-ни\-ка\-ци\-он\-ная сис\-те\-ма централизованной
обработки банковской информации>> В.\,И.~Будзко удостоен Премии Правительства РФ в
области науки и техники за 2010~г.

      В.\,И.~Будзко возглавлял и возглавляет работы по ряду других прикладных проектов,
связанных с созданием, совершенствованием и развитием крупномасштабных ИТС.

      В.\,И.~Будзко~--- генерал-майор, доктор технических наук, член-кор\-рес\-пон\-дент
Академии криптографии РФ, известный ученый в области информатики и применения
информационных технологий при построении территориально распределенных ИТС
различного назначения. Является автором свыше 250~научных работ, опубликованных в
на\-уч\-но-тех\-ни\-че\-ских и специальных изданиях.

    \thispagestyle{empty}

      В.\,И.~Будзко уделяет большое внимание подготовке научных кадров. Под его
руководством защищено 6~диссертаций на соискание ученой степени кандидата
технических наук. Свыше 30~лет он читает лекции в ИКСИ Академии ФСБ, профессор
кафедры НИЯУ МИФИ. Является членом двух диссертационных советов, главным
редактором журнала <<Системы высокой доступности>> и членом редколлегии журнала
<<Информатика и её применения>>.

      \bigskip

      Редакционный совет и Редакционная коллегия журнала <<Информатика и её
применения>> сердечно поздравляют Владимира Игоревича Будзко с 70-ле\-ти\-ем и желают
крепкого здоровья и новых научных достижений.

\end{multicols}

%\def\stat{cont}
{%\hrule\par
%\vskip 7pt % 7pt
\raggedleft\Large \bf%\baselineskip=3.2ex
А\,В\,Т\,О\,Р\,С\,К\,И\,Й\ \ У\,К\,А\,З\,А\,Т\,Е\,Л\,Ь\ \ З\,А\ \ 2\,0\,1\,0 г. \vskip 17pt
    \hrule
    \par
\vskip 21pt plus 6pt minus 3pt }

\label{st\stat}

\def\tit{\ }

\def\aut{\ }
\def\auf{\ }

\def\leftkol{\ } % ENGLISH ABSTRACTS}

\def\rightkol{\ } %АВТОРСКИЙ УКАЗАТЕЛЬ ЗА 2010 г.} %ENGLISH ABSTRACTS}

\titele{\tit}{\aut}{\auf}{\leftkol}{\rightkol}

\vspace*{-12pt}

{\tabcolsep=3pt
\begin{tabular}{p{388pt}rr}
&\textbf{Выпуск} & \textbf{Стр.}\\[6pt]
\hangindent=23pt\noindent\textbf{Арутюнян~А.\,Р.} Моделирование влияния деформаций отпечатков пальцев на 
точность\linebreak
\vspace*{-12pt}\\
\hspace*{23pt}дактилоскопической идентификации$\dotfill$&1&51\\
\hangindent=23pt\noindent\textbf{Архипов~О.\,П., Зыкова~З.\,П.} Интеграция гетерогенной информации о цветных 
пикселях\linebreak
\vspace*{-12pt}\\
\hspace*{23pt}и их цветовосприятии$\dotfill$&4&15\\
\hangindent=23pt\noindent\textbf{Баранов~С.\,И., Френкель~С.\,Л., Захаров~В.\,Н.} Полуформальная верификация 
цифрового устройства с конвейером, основанная на использовании алгоритмических машин\linebreak
\vspace*{-12pt}\\
\hspace*{23pt}состояния$\dotfill$&4&49\\
\textbf{Бекетова~И.\,В.} см.~Каратеев~С.\,Л.&&\\
\textbf{Белоусов~В.\,В.} см.~Синицын~И.\,Н.&&\\
\hangindent=23pt\noindent\textbf{Бенинг~В.\,Е., Королев~Р.\,А.} О предельном поведении мощностей критериев в 
случае\linebreak
\vspace*{-12pt}\\
\hspace*{23pt}распределения Лапласа$\dotfill$&2&63\\
\hangindent=23pt\noindent\textbf{Бенинг~В.\,Е., Сипина~А.\,В.} Асимптотическое разложение для мощности 
критерия,\linebreak
\vspace*{-12pt}\\
\hspace*{23pt}основанного на выборочной медиане, в случае распределения Лапласа$\dotfill$&1&18\\
\textbf{Бондаренко~А.\,В.} см.~Каратеев~С.\,Л.&&\\
\hangindent=23pt\noindent\textbf{Бородина~А.\,В., Морозов~Е.\,В.} Об оценивании асимптотики вероятности 
большого\linebreak
\vspace*{-12pt}\\
\hspace*{23pt}уклонения стационарной регенеративной очереди с одним прибором$\dotfill$&3&29\\
\hangindent=23pt\noindent\textbf{Бунтман~Н.\,В., Минель~Ж.-Л., Ле~Пезан~Д., Зацман~И.\,М.} Типология и 
компьютерное\linebreak
\vspace*{-12pt}\\
\hspace*{23pt}моделирование трудностей перевода$\dotfill$&3&77\\
\textbf{Визильтер~Ю.\,В.} см.~Каратеев~С.\,Л.&&\\
\hangindent=23pt\noindent\textbf{Гавриленко~С.\,В.} Оценки скорости сходимости распределений случайных сумм с 
безгранично делимыми индексами к нормальному закону$\dotfill$&4&81\\
\hangindent=23pt\noindent\textbf{Григорьева~М.\,Е., Шевцова~И.\,Г.} Уточнение неравенства 
Каца--Берри--Эссеена$\dotfill$&2&75\\
\hangindent=23pt\noindent\textbf{Грушо~А.\,А., Грушо~Н.\,А., Тимонина~Е.\,Е.} Поиск конфликтов в политиках 
безопасности: модель случайных графов$\dotfill$&3&38\\
\textbf{Грушо~Н.\,А.} см.~Грушо~А.\,А.&&\\
\hangindent=23pt\noindent\textbf{Гудков~В.\,Ю.} Математические модели изображения отпечатка пальца на основе 
описания линий$\dotfill$&1&58\\
\textbf{Гуртов~А.\,В.} см.~Лукьяненко~А.\,С.&&\\
\textbf{Желтов~С.\,Ю.} см.~Каратеев~С.\,Л.&&\\
\hangindent=23pt\noindent\textbf{Захаров~А.\,А., Серебряков~В.\,А.} Система управления электронной библиотекой 
LibMeta$\dotfill$&4&2\\
\textbf{Захаров~В.\,Н.} см.~Баранов~С.\,И.&&\\
\textbf{Захарова~Т.\,В.} см.~Матвеева~С.\,С.&&\\
\hangindent=23pt\noindent\textbf{Зацаринный~А.\,А., Чупраков~К.\,Г.} Некоторые аспекты выбора технологии для 
постро-\linebreak
\vspace*{-12pt}\\
\hspace*{23pt}ения систем отображения информации ситуационного центра$\dotfill$&3&59\\
\textbf{Зацман~И.\,М.} см.~Бунтман~Н.\,В.&&\\
\hangindent=23pt\noindent\textbf{Зейфман~А.\,И., Коротышева~А.\,В., Сатин~Я.\,А., Шоргин~С.\,Я.} Об 
устойчивости нестаци-\linebreak
\vspace*{-12pt}\\
\hspace*{23pt}онарных систем обслуживания с катастрофами$\dotfill$&3&9\\
\textbf{Зыкова~З.\,П.} см.~Архипов~О.\,П.&&\\
\hangindent=23pt\noindent\textbf{Илюшин~Г.\,Я., Соколов~И.\,А.} Организация управляемого доступа пользователей 
к\linebreak
\vspace*{-12pt}\\
\hspace*{23pt}разнородным ведомственным информационным ресурсам$\dotfill$&1&24\\
\hangindent=23pt\noindent\textbf{Кавагучи~Ю., Ульянов~В.\,В., Фуджикоши~Я.} Приближения для статистик, 
описывающих\linebreak
\vspace*{-12pt}\\
\hspace*{23pt}геометрические свойства данных большой размерности, с оценками 
ошибок$\dotfill$&1&12\\
\hangindent=23pt\noindent\textbf{Каратеев~С.\,Л., Бекетова~И.\,В., Ососков~М.\,В., Князь~В.\,А., 
Визильтер~Ю.\,В., Бондаренко~А.\,В., Желтов~С.\,Ю.} Автоматизированный контроль 
качества цифровых\linebreak
\vspace*{-12pt}\\
\hspace*{23pt}изображений для персональных документов$\dotfill$&1&65\\
\end{tabular}
}

\pagebreak

\def\leftkol{АВТОРСКИЙ УКАЗАТЕЛЬ ЗА 2010 г.} % ENGLISH ABSTRACTS}

\def\rightkol{АВТОРСКИЙ УКАЗАТЕЛЬ ЗА 2010 г.} %ENGLISH ABSTRACTS}

{\tabcolsep=3pt
\begin{tabular}{p{388pt}rr}
&\textbf{Выпуск} & \textbf{Стр.}\\[3pt]
\hangindent=23pt\noindent\textbf{Козеренко~Е.\,Б.} Лингвистические фильтры в статистических моделях машинного\linebreak
\vspace*{-12pt}\\
\hspace*{23pt}перевода$\dotfill$&2&83\\
\hangindent=23pt\noindent\textbf{Козеренко~Е.\,Б., Кузнецов~И.\,П.} Когнитивно-лингвистические представления в 
систе-\linebreak
\vspace*{-12pt}\\
\hspace*{23pt}мах обработки текстов$\dotfill$&3&69\\
\textbf{Князь~В.\,А.} см.~Каратеев~С.\,Л.&&\\
\hangindent=23pt\noindent\textbf{Колесников~А.\,В., Солдатов~С.\,А.} Алгоритм координации для гибридной 
интеллектуальной системы решения сложной задачи оперативно-производственного\linebreak
\vspace*{-12pt}\\
\hspace*{23pt}планирования$\dotfill$&4&61\\
\hangindent=23pt\noindent\textbf{Коновалов~М.\,Г.} О планировании потоков в системах вычислительных 
ресурсов$\dotfill$&2&3\\
\textbf{Конушин~А.\,С.} см.~Конушин~В.\,С.&&\\
\hangindent=23pt\noindent\textbf{Конушин~В.\,С., Кривовязь~Г.\,Р., Конушин~А.\,С.} Алгоритм распознавания людей 
в видео-\linebreak
\vspace*{-12pt}\\
\hspace*{23pt}последовательности по одежде$\dotfill$&1&74\\
\textbf{Корепанов~Э.\, Р.} см.~Синицын~И.\,Н.&&\\
\textbf{Королев~В.\,Ю.} см.~Соколов~И.\,А.&&\\
\textbf{Королев~Р.\,А.} см.~Бенинг~В.\,Е.&&\\
\textbf{Коротышева~А.\,В.} см.~Зейфман~А.\,И.&&\\
\hangindent=23pt\noindent\textbf{Кривенко~М.\,П.} Непараметрическое оценивание элементов байесовского 
клас\-си-\linebreak
\vspace*{-12pt}\\
\hspace*{23pt}фикатора$\dotfill$&2&13\\
\textbf{Кривовязь~Г.\,Р.} см.~Конушин~В.\,С.&&\\
\textbf{Крылов~А.\,С.} см.~Павельева~Е.\,А.&&\\
\hangindent=23pt\noindent\textbf{Крылов~В.\,А.} Моделирование и классификация многоканальных дистанционных\linebreak
\vspace*{-12pt}\\
\hspace*{23pt}изображений с использованием копул$\dotfill$&4&34\\
\hangindent=23pt\noindent\textbf{Крючин~О.\,В.} Разработка параллельных эвристических алгоритмов подбора 
весовых\linebreak
\vspace*{-12pt}\\
\hspace*{23pt}коэффициентов искусственной нейтронной сети$\dotfill$&2&53\\
\hangindent=23pt\noindent\textbf{Кудрявцев~А.\,А., Шоргин~С.\,Я.} Байесовские модели массового обслуживания и 
надеж-\linebreak
\vspace*{-12pt}\\
\hspace*{23pt}ности: характеристики среднего числа заявок в системе $M\vert M \vert 1\vert 
\infty$$\dotfill$&3&16\\
\hangindent=23pt\noindent\textbf{Кузнецов~А.\,А.} Связь между временными и структурно-топологическими 
характери-\linebreak
\vspace*{-12pt}\\
\hspace*{23pt}стиками диаграмм ритма сердца здоровых людей$\dotfill$&4&39\\
\textbf{Кузнецов~И.\,П.} см.~Козеренко~Е.\,Б.&&\\
\textbf{Ле~Пезан~Д.} см.~Бунтман~Н.\,В.&&\\
\hangindent=23pt\noindent\textbf{Лукьяненко~А.\,С., Морозов~Е.\,В., Гуртов~А.\,В.} Анализ сетевого протокола с общей 
функ-\linebreak
\vspace*{-12pt}\\
\hspace*{23pt}цией расширения окна передачи сообщения при конфликтах$\dotfill$&2&46\\
\hangindent=23pt\noindent\textbf{Лямин~О.\,О.} О предельном поведении мощностей критериев в случае обобщенного\linebreak
\vspace*{-12pt}\\
\hspace*{23pt}распределения Лапласа$\dotfill$&3&47\\
\hangindent=23pt\noindent\textbf{Маркин~А.\,В., Шестаков~О.\,В.} Асимптотики оценки риска при пороговой 
обработке\linebreak
\vspace*{-12pt}\\
\hspace*{23pt}вейвлет-вейглет коэффициентов в задаче томографии$\dotfill$&2&36\\
\hangindent=23pt\noindent\textbf{Матвеева~С.\,С., Захарова~Т.\,В.} Сети массового обслуживания с наименьшей 
длиной\linebreak
\vspace*{-12pt}\\
\hspace*{23pt}очереди$\dotfill$&3&22\\
\hangindent=23pt\noindent\textbf{Матюшенко~С.\,И.} Стационарные характеристики двухканальной системы 
обслужива-\linebreak
\vspace*{-12pt}\\
\hspace*{23pt}ния с переупорядочиванием заявок и распределениями фазового типа$\dotfill$&4&68\\
\textbf{Минель~Ж.-Л.} см.~Бунтман~Н.\,В.&&\\
\textbf{Морозов~Е.\,В.} см.~Бородина~А.\,В.&&\\
\textbf{Морозов~Е.\,В.} см.~Лукьяненко~А.\,С.&&\\
\textbf{Ососков~М.\,В.} см.~Каратеев~С.\,Л.&&\\
\hangindent=23pt\noindent\textbf{Павельева~Е.\,А., Крылов~А.\,С.} Поиск и анализ ключевых точек радужной 
оболочки\linebreak
\vspace*{-12pt}\\
\hspace*{23pt}глаза методом преобразования Эрмита$\dotfill$&1&79\\
\textbf{Печинкин~А.\,В.} см.~Френкель~С.\,Л.,&&\\
\hangindent=23pt\noindent\textbf{Протасов~В.\,И.} Составление субъективного портрета с использованием 
эволюционно-\linebreak
\vspace*{-12pt}\\
\hspace*{23pt}го морфинга и квалиметрия метода$\dotfill$&1&83\\
\hangindent=23pt\noindent\textbf{Рудаков~К.\,В., Торшин~И.\,Ю.} Вопросы разрешимости задачи распознавания 
вторичной\linebreak
\vspace*{-12pt}\\
\hspace*{23pt}структуры белка$\dotfill$&2&25\\
\textbf{Сатин~Я.\,А.} см.~Зейфман~А.\,И.&&\\
\hangindent=23pt\noindent\textbf{Сейфуль-Мулюков~Р.\,Б.} Нефть как носитель информации о своем 
происхождении,\linebreak
\vspace*{-12pt}\\
\hspace*{23pt}структуре и эволюции$\dotfill$&1&41\\
\end{tabular}
}

{\tabcolsep=3pt
\begin{tabular}{p{388pt}rr}
&\textbf{Выпуск} & \textbf{Стр.}\\[6pt]
\textbf{Семендяев~Н.\,Н.} см.~Синицын~И.\,Н.&&\\
\textbf{Серебряков~В.\,А.} см.~Захаров~А.\,А.&&\\
\textbf{Синицын~В.\,И.} см.~Синицын~И.\,Н.&&\\
\hangindent=23pt\noindent\textbf{Синицын~И.\,Н., Синицын~В.\,И., Корепанов~Э.\, Р., Белоусов~В.\,В., 
Семендяев~Н.\,Н.} Оперативное построение информационных моделей движения полюса 
Земли\linebreak
\vspace*{-12pt}\\
\hspace*{23pt}методами линейных и линеаризованных фильтров$\dotfill$&1&2\\
\textbf{Сипина~А.\,В.} см.~Бенинг~В.\,Е.&&\\
\hangindent=23pt\noindent\textbf{Соколов~И.\,А.} О работах заслуженного деятеля науки Российской Федерации 
И.\,Н.~Синицына в области информационных технологий и автоматизации (к 70-летию\linebreak
\vspace*{-12pt}\\
\hspace*{23pt}со дня рождения)$\dotfill$&3&84\\
\textbf{Соколов~И.\,А.} см.~Илюшин~Г.\,Я.&&\\
\hangindent=23pt\noindent\textbf{Соколов~И.\,А., Королев~В.\,Ю.} Предисловие$\dotfill$&2&2\\
\textbf{Солдатов~С.\,А.} см.~Колесников~А.\,В.&&\\
\hangindent=23pt\noindent\textbf{Степанов~С.\,Ю.} Использование координатного метода фрагментации 
коммутаторной\linebreak
\vspace*{-12pt}\\
\hspace*{23pt}нейронной сети для сокращения трафика$\dotfill$&2&57\\
\textbf{Тимонина~Е.\,Е.} см.~Грушо~А.\,А.&&\\
\textbf{Торшин~И.\,Ю.} см.~Рудаков~К.\,В.&&\\
\textbf{Ульянов~В.\,В.} см.~Кавагучи~Ю.&&\\
\textbf{Фазекаш~И.} см.~Чупрунов~А.\,Н.&&\\
\textbf{Френкель~С.\,Л.} см.~Баранов~С.\,И.&&\\
\hangindent=23pt\noindent\textbf{Френкель~С.\,Л., Печинкин~А.\,В.} Оценка времени самовосстановления в 
цифровых\linebreak
\vspace*{-12pt}\\
\hspace*{23pt}системах после сбоев, вызываемых переходными помехами$\dotfill$&3&2\\
\textbf{Фуджикоши~Я.} см.~Кавагучи~Ю.&&\\
\hangindent=23pt\noindent\textbf{Цискаридзе~А.\,К.} Математическая модель и метод восстановления позы человека 
по\linebreak
\vspace*{-12pt}\\
\hspace*{23pt}стереопаре силуэтных изображений$\dotfill$&4&27\\
\hangindent=23pt\noindent\textbf{Чупраков~К.\,Г.} К вопросу о размещении коллективных средств отображения в 
ситуа-\linebreak
\vspace*{-12pt}\\
\hspace*{23pt}ционном зале с заданными параметрами$\dotfill$&4&89\\
\textbf{Чупраков~К.\,Г.} см.~Зацаринный~А.\,А.&&\\
\hangindent=23pt\noindent\textbf{Чупрунов~А.\,Н., Фазекаш~И.} Законы повторного логарифма для числа 
безошибочных\linebreak
\vspace*{-12pt}\\
\hspace*{23pt}блоков при помехоустойчивом кодировании$\dotfill$&3&42\\
\textbf{Шевцова~И.\,Г.} см.~Григорьева~М.\,Е.&&\\
\hangindent=23pt\noindent\textbf{Шестаков~О.\,В.} Аппроксимация распределения оценки риска пороговой 
обработки вейвлет-коэффициентов нормальным распределением при использовании 
выбо-\linebreak
\vspace*{-12pt}\\
\hspace*{23pt}рочной дисперсии$\dotfill$&4&73\\
\textbf{Шестаков~О.\,В.} см.~Маркин~А.\,В.&&\\
\textbf{Шоргин~С.\,Я.} см.~Зейфман~А.\,И.&&\\
\textbf{Шоргин~С.\,Я.} см.~Кудрявцев~А.\,А.&&\\
\end{tabular}
}

%\thispagestyle{myheadings}
\def\leftfootline{\small{\textbf{\thepage}
\hfill ИНФОРМАТИКА И ЕЁ ПРИМЕНЕНИЯ\ \ \ том~4\ \ \ выпуск~4\ \ \ 2010}
}%
 \def\rightfootline{\small{ИНФОРМАТИКА И ЕЁ ПРИМЕНЕНИЯ\ \ \ том~4\ \ \ выпуск~4\ \ \ 2010
 \hfill \textbf{\thepage}}}
 \label{end\stat}




%
%Том 10 Выпуск 1-4 Год 2016

\def\stat{cont-e}
{%\hrule\par
%\vskip 7pt % 7pt
\raggedleft\Large \bf%\baselineskip=3.2ex
2\,0\,1\,6\ \ A\,U\,T\,H\,O\,R\ \ I\,N\,D\,E\,X \vskip 17pt
 \hrule
 \par
\vskip 21pt plus 6pt minus 3pt }

\label{st\stat}

\def\tit{\ }

\def\aut{\ }
\def\auf{\ }

\def\leftkol{\ } %2016 AUTHOR INDEX} % ENGLISH ABSTRACTS}

\def\rightkol{\ } %2016 AUTHOR INDEX} %ENGLISH ABSTRACTS}

\titele{\tit}{\aut}{\auf}{\leftkol}{\rightkol}

\def\leftfootline{\small{\textbf{\thepage}
\hfill INFORMATIKA I EE PRIMENENIYA~--- INFORMATICS AND APPLICATIONS\ \ \ 2016\
\ \ volume~10\ \ \ issue\ 4}
}%
 \def\rightfootline{\small{INFORMATIKA I EE PRIMENENIYA~--- INFORMATICS AND APPLICATIONS\ \ \ 2016\ \ \ volume~10\ \ \ issue\ 4
\hfill \textbf{\thepage}}}

\vspace*{-12pt}
\vspace*{-18pt}

{\tabcolsep=2.8pt
\begin{tabular}{p{382pt}cc}
&\textbf{Issue} & \textbf{Page}\\[6pt]
\Avtors{Agalarov~M.\,Ya.} see~Agalarov~Ya.\,M.&&\\
\Avtors{Agalarov~Ya.\,M., Agalarov~M.\,Ya., and
Shorgin~V.\,S.} About the optimal threshold of queue\linebreak
\\[-12pt]
\hspace*{23pt}length in a~particular problem of profit maximization
in the $M/G/1$ queuing system&2&70--79\\
\Avtors{Alexeyevsky~D.\,A.} BioNLP ontology extraction from 
a~restricted language corpus with\linebreak
\\[-12pt]
\hspace*{23pt}context-free grammars&1&119--128\\
\Avtors{Andreev~S.\,D.} see~Gaidamaka~Yu.\,V.&&\\
\Avtors{Andreev~S.\,D.} see~Ometov~A.\,Ya.&&\\
\Avtors{Arkhipov~O.\,P., Arkhipov~P.\,O., and Sidorkin~I.\,I.} The
option to create a~local coordinate\linebreak
\\[-12pt]
\hspace*{23pt}system for synchronization of selected images&3&91--97\\
\Avtors{Arkhipov~P.\,O.} see~Arkhipov~O.\,P.&&\\
\Avtors{Belousov~V.\,V.} see~Shnurkov~P.\,V.&&\\
\Avtors{Belousov~V.\,V.} see~Shnurkov~P.\,V.&&\\
\Avtors{Bening~V.\,E.} Calculation of~the~asymptotic deficiency
of~some statistical procedures based\linebreak
\\[-12pt]
\hspace*{23pt}on~samples with~random sizes&4&34--45\\
\Avtors{Borisov~A.\,V., Bosov~A.\,V., and Miller~G.\,B.} Modeling and
monitoring of VoIP connection&2&\hphantom{1}2--13\\
\Avtors{Bosov~A.\,V.} see~Borisov~A.\,V.&&\\
\Avtors{Briukhov~D.\,O.} see~Stupnikov~S.\,A.&&\\
\Avtors{Callaos~N.\,K.\ and Seyful-Mulyukov~R.\,B.} Complexity and
its information content&1&129--139\\
\Avtors{Chertok~A.\,V., Kadaner~A.\,I., Khazeeva~G.\,T., and
Sokolov~I.\,A.} Regime switching detection\linebreak
\\[-12pt]
\hspace*{23pt}for~the~Levy driven
Ornstein--Uhlenbeck process using CUSUM methods&4&46--56\\
\Avtors{Chichagov~V.\,V.} Asymptotic expansions of mean absolute
error of uniformly minimum variance unbiased and maximum likelihood
estimators on the one-parameter exponential\linebreak
\\[-12pt]
\hspace*{23pt}family model of lattice distributions&3&66--76\\
\Avtors{Danishevsky~V.\,I.} see~Kolesnikov A.\,V.&&\\
\Avtors{Fazliev~A.\,Z.} see~Kalinichenko~L.\,A.&&\\
\Avtors{Fedoseev~A.\,A.} What is behind the concept of ``knowledge in
small packages''&3&105--110\\
\Avtors{Gaidamaka~Yu.\,V., Andreev~S.\,D., Sopin~E.\,S.,
Samouylov~K.\,E., and Shorgin~S.\,Ya.} Interference analysis
of~the~device-to-device communications model with~regard to~a~signal\linebreak
\\[-12pt]
\hspace*{23pt}propagation environment&4&\hphantom{1}2--10\\
\Avtors{Gasilov~A.\,V.} see~Yakovlev~O.\,A.&&\\
\Avtors{Goncharov~A.\,V.\ and Strijov~V.\,V.} Metric time series
classification using weighted dynamic\linebreak
\\[-12pt]
\hspace*{23pt}warping relative to centroids of classes&2&36--47\\
\Avtors{Gordov~E.\,P.} see~Kalinichenko~L.\,A.&&\\
\Avtors{Gorshenin~A.\,K.} Concept of online service for stochastic
modeling of real processes&1&72--81\\
\Avtors{Gorshenin~A.\,K.} see~Shnurkov~P.\,V.&&\\
\Avtors{Gorshenin~A.\,K.} see~Shnurkov~P.\,V.&&\\
\Avtors{Grusho~A.\,A., Grusho~N.\,A., Zabezhailo~M.\,I., and
Timonina~E.\,E.} Integration of statistical and\linebreak
\\[-12pt]
\hspace*{23pt}deterministic methods for
analysis of information security&3&2--8\\
\Avtors{Grusho~A.\,A., Zabezhailo~M.\,I., and Zatsarinny~A.\,A.} On
the advanced procedure to reduce\linebreak
\\[-12pt]
\hspace*{23pt}calculation of Galois closures&4&\hphantom{1}96--104\\
\Avtors{Grusho~N.\,A.} see~Grusho~A.\,A.&&\\
\Avtors{Havanskov~V.\,A.} see~Minin~V.\,A.&&\\
\Avtors{Inkova~O.\,Yu.} see~Zatsman~I.\,M.&&\\
\Avtors{Isachenko~R.\,V.\ and Strijov~V.\,V.} Metric learning in
multiclass time series classification\linebreak
\\[-12pt]
\hspace*{23pt}problem&2&48--57\\
\end{tabular}
}
\pagebreak

\def\leftfootline{\small{\textbf{\thepage}
\hfill INFORMATIKA I EE PRIMENENIYA~--- INFORMATICS AND APPLICATIONS\ \ \ 2016\
\ \ volume~10\ \ \ issue\ 4}
}%
 \def\rightfootline{\small{INFORMATIKA I EE PRIMENENIYA~---
INFORMATICS AND APPLICATIONS\ \ \ 2016\ \ \ volume~10\ \ \ issue\ 4
\hfill \textbf{\thepage}}}

\def\leftkol{2016 AUTHOR INDEX} % ENGLISH ABSTRACTS}

\def\rightkol{2016 AUTHOR INDEX} %ENGLISH ABSTRACTS}


{\tabcolsep=2.83pt
\begin{tabular}{p{382pt}cc}
&\textbf{Issue} & \textbf{Page}\\[6pt]
\Avtors{Kadaner~A.\,I.} see~Chertok~A.\,V.&&\\[.255pt]
\Avtors{Kalinichenko~L.\,A., Volnova~A.\,A., Gordov~E.\,P.,
Kiselyova~N.\,N., Kovaleva~D.\,A., Malkov~O.\,Yu., Okladnikov~I.\,G.,
Podkolodnyy~N.\,L., Pozanenko~A.\,S., Ponomareva~N.\,V.,
Stupnikov~S.\,A.,} \textbf{and Fazliev~A.\,Z.} Data access challenges for data
intensive\linebreak
\\[-12pt]
\hspace*{23pt}research in Russia&1& 2--22\\[.255pt]
\Avtors{Karasikov~M.\,E.\ and Strijov~V.\,V.} Feature-based
time-series classification&4&121--131\\[.255pt]
\Avtors{Khazeeva~G.\,T.} see~Chertok~A.\,V.&&\\[.255pt]
\Avtors{Khokhlov~Yu.\,S.} Multivariate fractional Levy motion and its
applications&2&\hphantom{1}98--106\\[.255pt]
\Avtors{Kirikov~I.\,A., Kolesnikov~A.\,V., Listopad~S.\,V., and
Rumovskaya~S.\,B.} Fine-grained hybrid\linebreak
\\[-12pt]
\hspace*{23pt}intelligent systems. Part 2:
Bidirectional hybridization&1&\hphantom{1}96--105\\[.255pt]
\Avtors{Kirikov~I.\,A., Kolesnikov~A.\,V., Listopad~S.\,V., and
Rumovskaya~S.\,B.} ``Virtual council''~---\linebreak
\\[-12pt]
\hspace*{23pt}source environment
supporting complex diagnostic decision making&3&81--90\\[.255pt]
\Avtors{Kiselyova~N.\,N.} see~Kalinichenko~L.\,A.&&\\[.255pt]
\Avtors{Kolesnikov A.\,V., Listopad~S.\,V., Rumovskaya~S.\,B., and
Danishevsky~V.\,I.} Informal axiomatic\linebreak
\\[-12pt]
\hspace*{23pt}theory of~the~role visual models&4&114--120\\[.255pt]
\Avtors{Kolesnikov~A.\,V.} see~Kirikov~I.\,A.&&\\[.255pt]
\Avtors{Kolesnikov~A.\,V.} see~Kirikov~I.\,A.&&\\[.255pt]
\Avtors{Kolin~K.\,K.} Humanitarian aspects of information
security&3&111--121\\[.255pt]
\Avtors{Konovalov~M.\,G.\ and Razumchik~R.\,V.} Dispatching
to~two parallel nonobservable queues using\linebreak
\\[-12pt]
\hspace*{23pt}only static
information&4&57--67\\[.255pt]
\Avtors{Korchagin~A.\,Yu.} see~Korolev~V.\,Yu.&&\\[.255pt]
\Avtors{Korchagin~A.\,Yu.} see~Korolev~V.\,Yu.&&\\[.255pt]
\Avtors{Korepanov~E.\,R.} see~Sinitsyn~I.\,N.&&\\[.255pt]
\Avtors{Korepanov~E.\,R.} see~Sinitsyn~I.\,N.&&\\[.255pt]
\Avtors{Korolev~V.\,Yu., Korchagin~A.\,Yu., and Zeifman~A.\,I.} The
Poisson theorem for Bernoulli trials\linebreak
\\[-12pt]
\hspace*{23pt}with~a~random probability
of~success and~a~discrete analog of~the~Weibull distribution&4&11--20\\[.255pt]
\Avtors{Korolev~V.\,Yu., Zeifman~A.\,I., and Korchagin~A.\,Yu.}
Asymmetric Linnik distributions as~limit\linebreak
\\[-12pt]
\hspace*{23pt}laws for~random sums
of~independent random variables with~finite variances&4&21--33\\[.255pt]
\Avtors{Koucheryavy~E.\,A.} see~Ometov~A.\,Ya.&&\\[.255pt]
\Avtors{Kovaleva~D.\,A.} see~Kalinichenko~L.\,A.&&\\[.255pt]
\Avtors{Kovalyov~S.\,P.} Metaprogramming to increase
manufacturability of large-scale software-\linebreak
\\[-12pt]
\hspace*{23pt}intensive systems&1&56--66\\[.255pt]
\Avtors{Krivenko~M.\,P.} Significance tests of feature selection for
classification&3&32--40\\[.255pt]
\Avtors{Kruzhkov~M.\,G.} see~Zalizniak~Anna~A.&&\\[.255pt]
\Avtors{Kruzhkov~M.\,G.} see~Zatsman~I.\,M.&&\\[.255pt]
\Avtors{Kudryavtsev~A.\,A.} Bayesian queueing and reliability models:
\textit{A~priori} distributions with\linebreak
\\[-12pt]
\hspace*{23pt}compact support&1&67--71\\[.255pt]
\Avtors{Kudryavtsev~A.\,A.} Characteristics dependent on the balance
coefficient in Bayesian models\linebreak
\\[-12pt]
\hspace*{23pt}with compact support of \textit{a priori}
distributions&3&77--80\\[.255pt]
\Avtors{Kudryavtsev~A.\,A.\ and Palionnaia~S.\,I.} Bayesian recurrent
model of reliability growth:\linebreak
\\[-12pt]
\hspace*{23pt}Parabolic distribution of parameters&2&80--83\\[.255pt]
\Avtors{Kudryavtsev~A.\,A.\ and Titova~A.\,I.} Bayesian queuing
and~reliability models: Degenerate-\linebreak
\\[-12pt]
\hspace*{23pt}Weibull case&4&68--71\\[.255pt]
\Avtors{Leontyev~N.\,D.\ and Ushakov~V.\,G.} Analysis of a queueing
system with autoregressive arrivals\linebreak
\\[-12pt]
\hspace*{23pt}and nonpreemptive priority&3&15--22\\[.255pt]
\Avtors{Listopad~S.\,V.} see~Kirikov~I.\,A.&&\\[.255pt]
\Avtors{Listopad~S.\,V.} see~Kirikov~I.\,A.&&\\[.255pt]
\Avtors{Listopad~S.\,V.} see~Kolesnikov A.\,V.&&\\[.255pt]
\Avtors{Malkov~O.\,Yu.} see~Kalinichenko~L.\,A.&&\\[.255pt]
\Avtors{Markov~A.\,S., Monakhov~M.\,M., and
Ulyanov~V.\,V.} Generalized Cornish--Fisher expansions\linebreak
\\[-12pt]
\hspace*{23pt}for distributions of statistics based on samples
of random size&2&84--91\\[.255pt]
\Avtors{Melnikov~A.\,K.\ and Ronzhin~A.\,F.} Generalized statistical
method of~text analysis based\linebreak
\\[-12pt]
\hspace*{23pt}on~calculation of~probability distributions
of~statistical values&4&89--95\\
\end{tabular}
}
\pagebreak

\def\leftfootline{\small{\textbf{\thepage}
\hfill INFORMATIKA I EE PRIMENENIYA~--- INFORMATICS AND APPLICATIONS\ \ \ 2016\
\ \ volume~10\ \ \ issue\ 4}
}%
 \def\rightfootline{\small{INFORMATIKA I EE PRIMENENIYA~---
INFORMATICS AND APPLICATIONS\ \ \ 2016\ \ \ volume~10\ \ \ issue\ 4
\hfill \textbf{\thepage}}}

\def\leftkol{2016 AUTHOR INDEX} % ENGLISH ABSTRACTS}

\def\rightkol{2016 AUTHOR INDEX} %ENGLISH ABSTRACTS}


{\tabcolsep=3pt
\begin{tabular}{p{381pt}cc}
&\textbf{Issue} & \textbf{Page}\\[6pt]
\Avtors{Meykhanadzhyan~L.\,A.} Stationary characteristics of the finite
capacity queueing system with\linebreak
\\[-12pt]
\hspace*{23pt}inverse service order and generalized
probabilistic priority&2&123--131\\[.23pt]
\Avtors{Miller~G.\,B.} see~Borisov~A.\,V.&&\\[.23pt]
\Avtors{Minin~V.\,A., Zatsman~I.\,M., Havanskov~V.\,A., and
Shubnikov~S.\,K.} Intensity of citation of scientific publications in
inventions on information and computer technologies patented\linebreak
\\[-12pt]
\hspace*{23pt}in Russia by domestic and foreign applicants&2&107--122\\[.23pt]
\Avtors{Monakhov~M.\,M.} see~Markov~A.\,S.&&\\[.23pt]
\Avtors{Naumov~V.\,A.\ and Samouylov~K.\,E.} On relationship
between queuing systems with resources\linebreak
\\[-12pt]
\hspace*{23pt}and Erlang networks&3&\hphantom{1}9--14\\[.23pt]
\Avtors{Okladnikov~I.\,G.} see~Kalinichenko~L.\,A.&&\\[.23pt]
\Avtors{Ometov~A.\,Ya., Andreev~S.\,D., Turlikov~A.\,M., and
Koucheryavy~E.\,A.} Performance analysis of\linebreak
\\[-12pt]
\hspace*{23pt}a wireless data
aggregation system with contention for contemporary sensor
networks&3&23--31\\[.23pt]
\Avtors{Palionnaia~S.\,I.} see~Kudryavtsev~A.\,A.&&\\[.23pt]
\Avtors{Podkolodnyy~N.\,L.} see~Kalinichenko~L.\,A.&&\\[.23pt]
\Avtors{Ponomareva~N.\,V.} see~Kalinichenko~L.\,A.&&\\[.23pt]
\Avtors{Popkova~N.\,A.} see~Zatsman~I.\,M.&&\\[.23pt]
\Avtors{Pozanenko~A.\,S.} see~Kalinichenko~L.\,A.&&\\[.23pt]
\Avtors{Razumchik~R.\,V.} see~Konovalov~M.\,G.&&\\[.23pt]
\Avtors{Ronzhin~A.\,F.} see~Melnikov~A.\,K.&&\\[.23pt]
\Avtors{Rumovskaya~S.\,B.} see~Kirikov~I.\,A.&&\\[.23pt]
\Avtors{Rumovskaya~S.\,B.} see~Kirikov~I.\,A.&&\\[.23pt]
\Avtors{Rumovskaya~S.\,B.} see~Kolesnikov A.\,V.&&\\[.23pt]
\Avtors{Samouylov~K.\,E.} see~Gaidamaka~Yu.\,V.&&\\[.23pt]
\Avtors{Samouylov~K.\,E.} see~Naumov~V.\,A.&&\\[.23pt]
\Avtors{Serebryanskii~S.\,M.} see~Tyrsin~A.\,N.&&\\[.23pt]
\Avtors{Seyful-Mulyukov~R.\,B.} see~Callaos~N.\,K.&&\\[.23pt]
\Avtors{Shestakov~O.\,V.} Statistical properties of the denoising method
based on the stabilized hard\linebreak
\\[-12pt]
\hspace*{23pt}thresholding&2&65--69\\[.23pt]
\Avtors{Shestakov~O.\,V.} The strong law of large numbers for the risk
estimate in the problem of\linebreak
\\[-12pt]
\hspace*{23pt}tomographic image reconstruction from
projections with a correlated noise&3&41--45\\[.23pt]
\Avtors{Shestakov~O.\,V.} see~Zakharova~T.\,V.&&\\[.23pt]
\Avtors{Shnurkov~P.\,V., Gorshenin~A.\,K., and Belousov~V.\,V.}
Analytical solution of~the~optimal control\linebreak
\\[-12pt]
\hspace*{23pt}task of~a~semi-Markov
process with~finite set of~states&4&72--88\\[.23pt]
\Avtors{Shnurkov~P.\,V., Zasypko~V.\,V., Belousov~V.\,V., and
Gorshenin~A.\,K.} Development of the algorithm of numerical solution
of the optimal investment control problem\linebreak
\\[-12pt]
\hspace*{23pt}in the closed dynamical model of three-sector economy&1&82--95\\[.23pt]
\Avtors{Shorgin~S.\,Ya.} see~Gaidamaka~Yu.\,V.&&\\[.23pt]
\Avtors{Shorgin~V.\,S.} see~Agalarov~Ya.\,M.&&\\[.23pt]
\Avtors{Shubnikov~S.\,K.} see~Minin~V.\,A.&&\\[.23pt]
\Avtors{Sidorkin~I.\,I.} see~Arkhipov~O.\,P.&&\\[.23pt]
\Avtors{Sinitsyn~I.\,N.} Analytical modeling of processes in stochastic
systems with complex fractional\linebreak
\\[-12pt]
\hspace*{23pt}order Bessel nonlinearities&3&55--65\\[.23pt]
\Avtors{Sinitsyn~I.\,N.} Orthogonal supoptimal filters for nonlinear
stochastic systems on manifolds&1&34--44\\[.23pt]
\Avtors{Sinitsyn~I.\,N.\ and Korepanov~E.\,R.} Normal Pugachev
conditionally-optimal filters and extra-\linebreak
\\[-12pt]
\hspace*{23pt}polators for state linear stochastic systems&2&14--23\\[.23pt]
\Avtors{Sinitsyn~I.\,N.\ and Sinitsyn~V.\,I.} Analytical modeling of
distributions in stochastic systems on\linebreak
\\[-12pt]
\hspace*{23pt}manifolds based on ellipsoidal approximation&1&45--55\\[.23pt]
\Avtors{Sinitsyn~I.\,N., Sinitsyn~V.\,I., and
Korepanov~E.\,R.} Ellipsoidal suboptimal filters for nonlinear\linebreak
\\[-12pt]
\hspace*{23pt}stochastic systems on manifolds&2&24--35\\[.23pt]
\Avtors{Sinitsyn~V.\,I.} see~Sinitsyn~I.\,N.&&\\[.23pt]
\Avtors{Sinitsyn~V.\,I.} see~Sinitsyn~I.\,N.&&\\[.23pt]
\Avtors{Skvortsov~N.\,A.} see~Stupnikov~S.\,A.&&\\[.23pt]
\Avtors{Sokolov~I.\,A.} see~Chertok~A.\,V.&&\\
\end{tabular}
}
\pagebreak

\def\leftfootline{\small{\textbf{\thepage}
\hfill INFORMATIKA I EE PRIMENENIYA~--- INFORMATICS AND APPLICATIONS\ \ \ 2016\
\ \ volume~10\ \ \ issue\ 4}
}%
 \def\rightfootline{\small{INFORMATIKA I EE PRIMENENIYA~---
INFORMATICS AND APPLICATIONS\ \ \ 2016\ \ \ volume~10\ \ \ issue\ 4
\hfill \textbf{\thepage}}}

\def\leftkol{2016 AUTHOR INDEX} % ENGLISH ABSTRACTS}

\def\rightkol{2016 AUTHOR INDEX} %ENGLISH ABSTRACTS}


{\tabcolsep=3pt
\begin{tabular}{p{382pt}cc}
&\textbf{Issue} & \textbf{Page}\\[6pt]
\Avtors{Sopin~E.\,S.} see~Gaidamaka~Yu.\,V.&&\\
\Avtors{Strijov~V.\,V.} see~Goncharov~A.\,V.&&\\
\Avtors{Strijov~V.\,V.} see~Isachenko~R.\,V.&&\\
\Avtors{Strijov~V.\,V.} see~Karasikov~M.\,E.&&\\
\Avtors{Stupnikov~S.\,A., Briukhov~D.\,O., and Skvortsov~N.\,A.}
Co-lending systemic risk analysis over\linebreak
\\[-12pt]
\hspace*{23pt}heterogeneous data collections&1&23--33\\
\Avtors{Stupnikov~S.\,A.} see~Kalinichenko~L.\,A.&&\\
\Avtors{Suchkov~A.\,P.} see~Zatsarinny~A.\,A.&&\\
\Avtors{Timonina~E.\,E.} see~Grusho~A.\,A.&&\\
\Avtors{Titova~A.\,I.} see~Kudryavtsev~A.\,A.&&\\
\Avtors{Turlikov~A.\,M.} see~Ometov~A.\,Ya.&&\\
\Avtors{Tyrsin~A.\,N.\ and Serebryanskii~S.\,M.} Recognition of
dependences on the basis of inverse\linebreak
\\[-12pt]
\hspace*{23pt}mapping&2&58--64\\
\Avtors{Ulyanov~V.\,V.} see~Markov~A.\,S.&&\\
\Avtors{Ushakov~V.\,G.} Queueing system with working vacations and
hyperexponential input stream&2&92--97\\
\Avtors{Ushakov~V.\,G.} see~Leontyev~N.\,D.&&\\
\Avtors{Volnova~A.\,A.} see~Kalinichenko~L.\,A.&&\\
\Avtors{Yakovlev~O.\,A.\ and Gasilov~A.\,V.} Speeded-up stereo
matching using geodesic support weights&3&\hphantom{1}98--104\\
\Avtors{Zabezhailo~M.\,I.} see~Grusho~A.\,A.&&\\
\Avtors{Zabezhailo~M.\,I.} see~Grusho~A.\,A.&&\\
\Avtors{Zakharova~T.\,V.\ and Shestakov~O.\,V.} Precision analysis of
wavelet processing of aerodynamic\linebreak
\\[-12pt]
\hspace*{23pt}flow patterns&3&46--54\\
\Avtors{Zalizniak~Anna~A.\ and Kruzhkov~M.\,G.} Database
of~Russian impersonal verbal constructions&4&132--141\\
\Avtors{Zasypko~V.\,V.} see~Shnurkov~P.\,V.&&\\
\Avtors{Zatsarinny~A.\,A.\ and Suchkov~A.\,P.} Systems engineering
approaches to~the~establishment of\linebreak
\\[-12pt]
\hspace*{23pt}a~system for~decision support based
on~situational analysis&4&105--113\\
\Avtors{Zatsarinny~A.\,A.} see~Grusho~A.\,A.&&\\
\Avtors{Zatsman~I.\,M., Inkova~O.\,Yu., Kruzhkov~M.\,G., and
Popkova~N.\,A.} Representation of cross-\linebreak
\\[-12pt]
\hspace*{23pt}lingual knowledge about
connectors in supracorpora databases&1&106--118\\
\Avtors{Zatsman~I.\,M.} see~Minin~V.\,A.&&\\
\Avtors{Zeifman~A.\,I.} see~Korolev~V.\,Yu.&&\\
\Avtors{Zeifman~A.\,I.} see~Korolev~V.\,Yu.&&\\
\end{tabular}
}

%\thispagestyle{myheadings}
\def\leftfootline{\small{\textbf{\thepage}
\hfill INFORMATIKA I EE PRIMENENIYA~--- INFORMATICS AND APPLICATIONS\ \ \ 2016\
\ \ volume~10\ \ \ issue\ 4}
}%
 \def\rightfootline{\small{INFORMATIKA I EE PRIMENENIYA~---
INFORMATICS AND APPLICATIONS\ \ \ 2016\ \ \ volume~10\ \ \ issue\ 4
\hfill \textbf{\thepage}}}

 \label{end\stat}

\newpage

%\def\stat{rekl}
%\label{preobr}

%\def\tit{АКАДЕМИК ПУГАЧЁВ  ВЛАДИМИР СЕМЁНОВИЧ\\
%25.03.1911--25.03.1998}


%   \vspace*{-48pt}
%   \begin{center}\LARGE
%Академик Пугачёв  Владимир Семёнович\\ (25.03.1911--25.03.1998)
%   \end{center}
   
   %\vspace*{2.5mm}
   
   \begin{center}

{\prgsh\LARGE
ОБЪЯВЛЕНИЯ О КОНФЕРЕНЦИЯХ}

\end{center}
%\hrule

\vspace*{6pt}

   
   \vspace*{10mm}
   
   \thispagestyle{empty}

\noindent
\begin{tabular}{cc}
%\begin{center}
\multicolumn{1}{c}{\raisebox{-40pt}[0pt][0pt]{\mbox{%
\epsfxsize=33mm
\epsfbox{vspu.eps}
}}}
%\end{center}
&
\tabcolsep=0pt\begin{tabular}{c}
{\prg{\Large\textbf{XII Всероссийское совещание}}}\\[6pt]
{\prg{\Large\textbf{по проблемам управления}}}\\[12pt]
{\prg{\large 16--19 июня 2014~г.}}\\[6pt] 
{\prg{\large Институт проблем управления имени В.\,А.~Трапезникова РАН}}\\[6pt]
{\prg{\large Москва, Россия}}
\end{tabular}
\end{tabular}

\vspace*{60pt}

     
 { %\large    
 XII Всероссийское совещание по проблемам управления (ВСПУ XII), посвященное 75-летию 
Института проблем управления (ИПУ) имени В.\,А.~Трапезникова РАН, проводится 16--19~июня 
2014~г.\ 
в ИПУ РАН (г.~Москва, Россия). ВСПУ XII организуется ИПУ РАН при поддержке РФФИ, Отделения 
энергетики, машиностроения, механики и процессов управления Российской академии наук, 
Российского 
национального комитета по автоматическому управлению, Академии навигации и управ\-ле\-ния 
движением, 
Научного совета РАН по комплексным проблемам управления и автоматизации, Совета по 
мехатронике и робототехнике РАН. Официальный язык Совещания~--- русский.

\vspace*{24pt}
     
     \textbf{Направления работы}
     \begin{enumerate}[1.]
\item Теория систем управления
\item Управление подвижными объектами и навигация
\item Интеллектуальные системы управления
\item Управление в промышленности, транспортом и логистикой
\item Управление системами междисциплинарной природы
\item Средства измерения, вычислений и контроля в управлении
\item Системный анализ и принятие решений в задачах управления
\item Информационные технологии в управлении
\item Проблемы образования в области управления: современное содержание и технологии обучения
\end{enumerate}

\vspace*{24pt}

     Подробная информация о Совещании находится на сайте {\sf http://vspu2014.ipu.ru}. Срок 
окончательной подачи докладов через систему подачи докладов на сайте~--- \textbf{30~ноября} 
2013~г.
}

%\include{rekl-1}

%\end{document}

%   \vspace*{-48pt}

\begin{center}
\vspace*{6pt}
\mbox{%
\epsfxsize=53.502mm
\epsfbox{foto-1.eps}
}
\end{center}

\vspace*{6pt} %Академик


   \begin{center}
\fbox{\Large\textbf{Профессор Игорь Алексеевич Ушаков}}\\[12pt]
\textbf{\large 22.01.1935--27.02.2015}
   \end{center}


   %\vspace*{2.5mm}

   \vspace*{5mm}

   \thispagestyle{empty}

%\

%\vspace*{-12pt}


Редакционный совет и редакционная коллегия журнала <<Информатика и~её применения>> с~глубоким прискорбием извещают, что 27~февраля 2015~г.\ после тяжелой
и~продолжительной болезни скончался Игорь Алексеевич Ушаков~--- доктор технических наук, профессор, член редколлегии журнала <<Информатика и ее применения>>.

Игорь Алексеевич Ушаков окончил Московский авиационный институт, в~1963~г.\ защитил кандидатскую, а~в~1968~г.~--- докторскую диссертацию. С~1958 по 1989~гг.\ работал в~ряде научно-исследовательских организаций СССР, в~том числе руководил отделами в~НИИ АА и~ВЦ АН СССР; с 1969 по 1989 гг. преподавал в~МФТИ (был профессором, а~затем заведующим кафедрой) и~в~МЭИ. С~1989~г.~---- в~США: являлся профессором университета Дж.\ Вашингтона, университета Дж.\ Мэйсона и~Калифорнийского университета, сотрудником компаний MCI, Qualcomm и Hughes.

И.\,А.~Ушаков с момента основания журнала <<Надежность и~контроль качества>> был заместителем ответственного редактора, а~затем на протяжении многих лет членом редколлегии. В~2006~г.\ основал электронный международный журнал ``Reliability: Theory \& Application'', главным редактором которого оставался до конца жизни.

Учебниками и справочниками по теории надежности, написанными И.\,А.~Ушаковым, пользовались и~пользуются несколько поколений ученых и~специалистов в~разных странах мира.

Игорь Алексеевич всегда уделял огромное внимание работе с~молодежью; более~50 его учеников защитили докторские и~кандидатские диссертации.

И.\,А.~Ушаков вел активную научно-про\-све\-ти\-тель\-скую деятельность. В~частности, он был одним из организаторов и~руководителей Московского кабинета качества и~надежности при Политехническом музее (целью этого Кабинета было оказание консультаций работникам промышленных предприятий и~чтение курсов лекций для инженеров, занимающихся проблемой надежности). Находясь в~США, И.\,А.~Ушаков создал международный ин\-тер\-нет-фо\-рум им.\ Б.\,В.~Гнеденко, объединивший около~400~видных специалистов по приложениям теории вероятностей и~математической статистики, преимущественно в~об\-ласти теории надежности и~анализа риска, из десятков стран мира; коллективным членов этого Форума является и~наш журнал. Цели Форума~--- содействие контактам между специалистами из разных стран, организация обмена профессиональными 
новостями и~информацией (новые публикации, предстоящие события и~др.). Также необходимо отметить большое число на\-уч\-но-по\-пу\-ляр\-ных работ, опубликованных И.\,А.~Ушаковым.

И.\,А.~Ушаков обладал большим личным обаянием, имел широкий круг интересов. Все знавшие И.\,А.~Ушакова всегда будут помнить его как замечательного ученого и~прекрасного человека.

\bigskip

Редакционный совет и редакционная коллегия журнала <<Информатика и~её применения>> 
выражают глубокие соболезнования родным и близким покойного, всем, кто его знал и~работал с~ним.



%\end{document}

%\include{IPPM-25}

\def\stat{cont-rus}
{%\hrule\par
%\vskip 7pt % 7pt
\vspace*{-24pt}
\raggedleft\Large \bf%\baselineskip=3.2ex
Правила подготовки рукописей  для публикации в журнале
<<Информатика~и~её~применения>> \vskip 8pt
    \hrule
    \par
\vskip 14pt plus 6pt minus 3pt }

\label{st\stat}

\def\tit{\ }

\def\aut{\ }
\def\auf{\ }

\def\leftkol{\ }
% Правила подготовки рукописей  для публикации в журнале
%<<Информатика и её применения>>

\def\rightkol{\ }
%Правила подготовки рукописей  для публикации в журнале
%<<Информатика и её применения>>}


\titele{\tit}{\aut}{\auf}{\leftkol}{\rightkol}


\vspace*{-60pt}
{ %\small

Журнал <<Информатика и её применения>>
публикует теоретические, обзорные и дискуссионные статьи,
посвященные научным исследованиям и разработкам в области
информатики и ее приложений.

Журнал издается на русском языке. По специальному решению
редколлегии отдельные статьи могут печататься на английском языке.

Тематика журнала охватывает следующие направления:
\begin{itemize}
\item теоретические основы информатики;\\[-15pt]
      \item
математические методы исследования сложных систем и процессов;\\[-15pt]
           \item
информационные системы и сети;\\[-15pt]
                \item
информационные технологии;\\[-15pt]
                     \item
архитектура и программное обеспечение вычислительных комплексов и сетей.\\[-15pt]
\end{itemize}


\noindent
\begin{enumerate}[1.]
\item В журнале печатаются статьи, содержащие результаты, ранее не опубликованные и
не предназначенные к одновременной публикации в других изданиях.

%Публикация не должна нарушать закон об авторских правах.
Публикация предоставленной автором(ами) рукописи не должна нарушать 
положений глав~69, 70 раздела~VII части~IV Гражданского кодекса, 
которые определяют права на результаты интеллектуальной деятельности 
и~средства индивидуализации, в~том числе авторские права, в~РФ.

Ответственность за нарушение авторских прав, в~случае предъявления претензий к~редакции журнала,  
несут авторы статей.



Направляя рукопись в редакцию, авторы сохраняют свои права на данную
рукопись и при этом передают учредителям и редколлегии журнала неисключительные права на
издание статьи на русском языке 
(или на языке статьи, если он отличен от рус\-ско\-го) и~на перевод ее на английский
язык, а~также на
ее распространение в России и за рубежом. 
Каждый автор должен представить в~редакцию подписанный 
с~его стороны <<Лицензионный договор о~передаче неисключительных прав 
на использование произведения>>, текст которого размещен по адресу 
{\sf http://www.ipiran.ru/publications/licence.doc}. 
Этот договор может быть пред\-став\-лен в~бумажном (в~2-х экз.)\ 
или в~электронном виде (отсканированная копия заполненного и~подписанного документа).




Редколлегия вправе запросить у авторов экспертное заключение о возможности
пуб\-ли\-ка\-ции пред\-став\-лен\-ной статьи в открытой печати.\\[-13.5pt]

\item К статье прилагаются данные автора (авторов) (см.\ п.~8). При наличии нескольких
авторов указывается фамилия автора, ответственного за переписку с редакцией.\\[-13.5pt]

\item Редакция журнала осуществляет экспертизу присланных статей в соответствии с
принятой в журнале процедурой рецензирования.

Возвращение рукописи на доработку не означает ее принятия к печати.

Доработанный вариант с ответом на замечания рецензента необходимо прислать в
редакцию.\\[-13.5pt]

\item Решение редколлегии о публикации статьи или ее отклонении сообщается авторам.

Редколлегия может также направить авторам текст рецензии на их статью. Дискуссия по
поводу отклоненных статей не ведется.\\[-13.5pt]

%\pagebreak

\item Редактура статей высылается авторам для просмотра. Замечания к редактуре должны
быть присланы авторами в кратчайшие сроки.\\[-13.5pt]

\item Рукопись предоставляется в электронном виде в форматах MS WORD (.doc или
.docx) или \LaTeX\  (.tex), дополнительно~--- в формате .pdf, на дискете, лазерном диске
или электронной почтой. Предоставление бумажной рукописи необязательно.\\[-13.5pt]

\item При подготовке рукописи в MS Word рекомендуется использовать следующие
настройки.

Параметры страницы:
формат~--- А4; ориентация~--- книжная; поля (см): внутри~--- 2,5, снаружи~--- 1,5,
сверху~--- 2, снизу~--- 2, от края до нижнего колонтитула~--- 1,3.

Основной текст: стиль~--- <<Обычный>>, шрифт~--- Times New Roman, размер~---
14~пунк\-тов, абзацный отступ~--- 0,5~см, 1,5~интервала, выравнивание~--- по ширине.

\pagebreak

\def\leftkol{Правила подготовки рукописей  для публикации в журнале
<<Информатика и её применения>>}

\def\rightkol{Правила подготовки рукописей  для публикации в журнале
<<Информатика и её применения>>}



Рекомендуемый объем рукописи~--- не свыше 10~страниц указанного формата.
При превышении указанного объема редколлегия вправе потребовать от 
автора сокращения объема рукописи.


Сокращения слов, помимо стандартных, не допускаются. Допускается минимальное
количество аббревиатур.


Все страницы рукописи нумеруются.

Шаблоны оформления представлены в интернете:

\noindent
 {\sf
http://www.ipiran.ru/journal/template\_iiep\_ssi\_2024.zip}\\[-14pt]

\item Статья должна содержать следующую информацию на {\bfseries\textit{русском и
английском языках}}:\\[-16pt]

\begin{itemize}
\item название статьи;\\[-15pt]
\item Ф.И.О.\ авторов, на английском можно только имя и фамилию;\\[-15pt]
\item место работы, с указанием почтового адреса организации и электронного адреса каждого
автора;\\[-15pt]
\item сведения об авторах, в соответствии с форматом, образцы которого
представлены на страницах:



\def\leftfootline{\small{\textbf{\thepage}
\hfill ИНФОРМАТИКА И ЕЁ ПРИМЕНЕНИЯ\ \ \ том\ 18\ \ \ выпуск\ 3\ \ \ 2024}
}%
 \def\rightfootline{\small{ИНФОРМАТИКА И ЕЁ ПРИМЕНЕНИЯ\ \ \ том\ 18\ \ \ выпуск\ 3\ \ \ 2024
\hfill \textbf{\thepage}}}



{\sf http://www.ipiran.ru/journal/issues/2013\_07\_01/authors.asp} и

{\sf http://www.ipiran.ru/journal/issues/2013\_07\_01\_eng/authors.asp};
\item аннотация (не менее 100~слов на каждом из языков). Аннотация~--- это краткое
резюме работы, которое может публиковаться отдельно. Она является основным
источником информации в~ин\-фор\-ма\-ци\-он\-ных системах и базах данных. Английская
аннотация должна быть оригинальной, может не быть дословным переводом русского
текста и должна быть написана хорошим английским языком. В~аннотации не должно
быть ссылок на литературу и, по возможности, формул;\\[-15pt]
\item ключевые слова~--- желательно из принятых в мировой
на\-уч\-но-тех\-ни\-че\-ской литературе тематических тезаурусов. Предложения не
могут быть ключевыми словами;\\[-15pt]
\item источники финансирования работы (ссылки на гранты, проекты,
поддерживающие организации и~т.\,п.).
\end{itemize}



%\pagebreak

\item  Требования к спискам литературы.\\[-14pt]

Ссылки на литературу в тексте статьи нумеруются (в квадратных скобках) и
располагаются в каждом из списков литературы в порядке  первых упоминаний. Если источник имеет DOI и/или EDN,
то их необходимо указывать.

Списки литературы представляются в двух вариантах:\\[-14pt]


\noindent
\begin{enumerate}[(1)]
\item \textbf{Список литературы к русскоязычной части}. Русские и английские
работы~---  на языке и в алфавите оригинала;\\[-14.5pt]
\item  \textbf{References}. Русские работы и работы на других языках~--- в латинской
транслитерации с переводом на английский язык; английские работы и работы на других
языках~--- на языке оригинала.
\end{enumerate}

Необходимо для составления списка ``References'' пользоваться размещенной на сайте
{\sf http://www. translit.net/ru/bgn/} бесплатной программой транслитерации русского
 текста в~латиницу. %, при этом в~за\-клад\-ке <<варианты\ldots>> следует выбратьопцию BGN.

Список литературы ``References'' приводится полностью отдельным блоком, повторяя все
позиции из списка литературы к русскоязычной части, независимо от того, имеются или
нет в нем иностранные источники. Если в списке литературы к русскоязычной части есть
ссылки на иностранные публикации, набранные латиницей, они полностью повторяются в
списке ``References''.

Ниже приведены примеры ссылок на различные виды публикаций в списке ``References''.

\def\leftfootline{\small{\textbf{\thepage}
\hfill ИНФОРМАТИКА И ЕЁ ПРИМЕНЕНИЯ\ \ \ том\ 18\ \ \ выпуск\ 3\ \ \ 2024}
}%
 \def\rightfootline{\small{ИНФОРМАТИКА И ЕЁ ПРИМЕНЕНИЯ\ \ \ том\ 18\ \ \ выпуск\ 3\ \ \ 2024
\hfill \textbf{\thepage}}}

{\small

\noindent
\textbf{Описание статьи из журнала:}

\Aue{Zagurenko, A.\,G., V.\,A.~Korotovskikh, A.\,A.~Kolesnikov, A.\,V.~Timonov, and D.\,V.~Kardymon}. 2008.
Tekhniko-ekonomicheskaya optimizatsiya dizayna gidrorazryva plasta [Technical and
economic optimization of the design
of hydraulic fracturing]. \textit{Neftyanoe hozyaystvo} [\textit{Oil Industry}] 11:54--57.

\Aue{Zhang, Z., and D.~Zhu}. 2008. Experimental research on the localized
electrochemical micromachining. \textit{Russ. J.~Electrochem.}  44(8):926--930.
{\sf doi:10.1134/S1023193508080077}.

\noindent
\textbf{Описание статьи из электронного журнала:}

\Aue{Swaminathan, V., E.~Lepkoswka-White, and B.\,P.~Rao}. 1999. Browsers or buyers in cyberspace? An
investigation of electronic factors influencing electronic exchange. \textit{JCMC}
5(2). Available at: {\sf http://www.ascusc.org/jcmc/vol5/issue2/} (accessed April~28, 2011).

\def\leftkol{Правила подготовки рукописей  для публикации в журнале
<<Информатика и её применения>>}

\def\rightkol{Правила подготовки рукописей  для публикации в журнале
<<Информатика и её применения>>}


\noindent
\textbf{Описание статьи из продолжающегося издания (сборника трудов):}

\Aue{Astakhov, M.\,V., and T.\,V.~Tagantsev}. 2006. Eksperimental'noe
issledovanie prochnosti soedineniy ``stal'--kompozit'' [Experimental study of
the strength of joints ``steel--composite'']. \textit{Trudy MGTU
``Matematicheskoe modelirovanie slozhnykh tekh\-ni\-che\-skikh sistem''}
[\textit{Bauman MSTU ``Mathematical Modeling of Complex Technical
Systems'' Proceedings}]. 593:125--130.


\pagebreak



\noindent
\textbf{Описание материалов конференций:}

\Aue{Usmanov, T.\,S., A.\,A.~Gusmanov, I.\,Z.~Mullagalin, R.\,Ju.~Muhametshina, A.\,N.~Chervyakova, and
A.\,V.~Sveshnikov}. 2007. Osobennosti proektirovaniya razrabotki mestorozhdeniy
s primeneniem gidrorazryva
plasta [Features of the design of field development with the use of hydraulic fracturing].
\textit{Trudy 6-go
Mezhdu\-na\-rod\-no\-go Simpoziuma ``Novye resursosberegayushchie tekhnologii nedropol'zovaniya i povysheniya
neftegazootdachi''} [\textit{6th  Symposium (International) ``New Energy Saving Subsoil Technologies and
the Increasing of the Oil and Gas Impact'' Proceedings}]. Moscow. 267--272.



\def\leftfootline{\small{\textbf{\thepage}
\hfill ИНФОРМАТИКА И ЕЁ ПРИМЕНЕНИЯ\ \ \ том\ 18\ \ \ выпуск\ 3\ \ \ 2024}
}%
 \def\rightfootline{\small{ИНФОРМАТИКА И ЕЁ ПРИМЕНЕНИЯ\ \ \ том\ 18\ \ \ выпуск\ 3\ \ \ 2024
\hfill \textbf{\thepage}}}



\noindent
\textbf{Описание книги (монографии, сборники):}



Lindorf, L.\,S., and L.\,G.~Mamikoniants, eds. 1972.
\textit{Ekspluatatsiya turbogeneratorov s neposredstvennym
okhlazhdeniem} [\textit{Operation of turbine generators with direct cooling}].
Moscow: Energy Publs. 352~p.


\Aue{Latyshev, V.\,N.} 2009. \textit{Tribologiya rezaniya. Kn.~1: Friktsionnye protsessy
pri rezanii metallov}
[\textit{Tribology of cutting. Vol.~1: Frictional processes in metal cutting}]. Ivanovo: Ivanovskii
State Univ. 108~p.

\def\leftkol{Правила подготовки рукописей  для публикации в журнале
<<Информатика и её применения>>}

\def\rightkol{Правила подготовки рукописей  для публикации в журнале
<<Информатика и её применения>>}

\noindent
\textbf{Описание переводной книги}
(в списке литературы к русскоязычной части необходимо указать:~/ Пер.\ с англ.~---
после названия книги, а в конце ссылки указать оригинал книги в круглых скобках):
\begin{enumerate}[1.]
\item  В русскоязычной части:

\def\leftfootline{\small{\textbf{\thepage}
\hfill ИНФОРМАТИКА И ЕЁ ПРИМЕНЕНИЯ\ \ \ том\ 18\ \ \ выпуск\ 3\ \ \ 2024}
}%
 \def\rightfootline{\small{ИНФОРМАТИКА И ЕЁ ПРИМЕНЕНИЯ\ \ \ том\ 18\ \ \ выпуск\ 3\ \ \ 2024
\hfill \textbf{\thepage}}}

\Au{Тимошенко С.\,П., Янг Д.\,Х., Уивер~У.}
Колебания в инженерном деле~/ Пер.\ с англ.~--- М.: Машиностроение, 1985. 472~с.
(\Au{Timoshenko~S.\,P., Young~D.\,H., Weaver~W.}
Vibration problems in engineering.~--- 4th ed.~--- New York, NY, USA: Wiley, 1974. 521~p.)\\[-13.5pt]
\item  В англоязычной части:

\Aue{Timoshenko, S.\,P., D.\,H.~Young, and W.~Weaver}.
1974. \textit{Vibration problems in engineering}. 4th ed. New York: 
Wiley. 521~p.
\end{enumerate}

\vspace*{-3pt}


\noindent
\textbf{Описание неопубликованного документа:}


\Aue{Latypov, A.\,R., M.\,M.~Khasanov, and V.\,A.~Baikov}.
2004 (unpubl.). Geologiya i~dobycha (NGT GiD) [Geology and production (NGT GiD)]. Certificate on official registration of the computer program
No.\,2004611198. 

\noindent
\textbf{Описание интернет-ресурса:}


Pravila tsitirovaniya istochnikov [Rules for the citing of sources]. Available at: {\sf
http://www.scribd.com/doc/1034528/} (accessed February~7, 2011).

%\pagebreak

\noindent
\textbf{Описание диссертации или автореферата диссертации:}

\Aue{Semenov, V.\,I.}
2003. Matematicheskoe modelirovanie plazmy v sisteme kompaktnyy tor [Mathematical
modeling of the plasma in the compact torus].  Moscow.  D.Sc.\ Diss. 272~p.

\Aue{Kozhunova, O.\,S.} 2009. Tekhnologiya razrabotki semanticheskogo
slovarya informatsionnogo monitoringa [Technology of development of
semantic dictionary of information monitoring system].  Moscow: IPI RAN. PhD Thesis. 23~p.


\noindent
\textbf{Описание ГОСТа:}

GOST 8.586.5-2005. 2007. Metodika vypolneniya izmereniy. Izmerenie raskhoda i~kolichestva zhidkostey i~gazov
s~pomoshch'yu standartnykh suzhayushchikh ustroystv [Method of measurement.
Measurement of flow rate and volume of liquids and gases by means of orifice devices]. Moscow:
Standardinform  Publs. 10~p.

\noindent
\textbf{Описание патента:}

\Aue{Bolshakov, M.\,V., A.\,V.~Kulakov, A.\,N.~Lavrenov, and M.\,V.~Palkin}.
2006. Sposob orientirovaniya po krenu letatel'nogo
apparata s opti\-che\-skoy golovkoy
samonavedeniya [The way to orient on the roll of aircraft with optical homing head].
Patent RF No.\,2280590.
}

\item Присланные в редакцию материалы авторам не возвращаются.\\[-13.5pt]

\item При отправке файлов по электронной почте просим придерживаться следующих
правил:
\begin{itemize}
\item указывать в поле subject (тема) название журнала и фамилию автора;\\[-13.5pt]
\item указывать в тексте письма название статьи, авторов и~журнал, в~который направляется статья;\\[-13.5pt]
\item использовать attach (присоединение);\\[-13.5pt]
\item в состав электронной версии статьи должны входить: файл, содержащий текст
статьи, и файл(ы), содержащий(е) иллюстрации.\\[-13.5pt]
\end{itemize}

\item Журнал <<Информатика и её применения>> является некоммерческим изданием.
Плата за публикацию не взимается, гонорар авторам не выплачивается.
\end{enumerate}



\def\leftfootline{\small{\textbf{\thepage}
\hfill ИНФОРМАТИКА И ЕЁ ПРИМЕНЕНИЯ\ \ \ том\ 18\ \ \ выпуск\ 3\ \ \ 2024}
}%
 \def\rightfootline{\small{ИНФОРМАТИКА И ЕЁ ПРИМЕНЕНИЯ\ \ \ том\ 18\ \ \ выпуск\ 3\ \ \ 2024
\hfill \textbf{\thepage}}}


\vspace*{-1mm}

\begin{center}

\textbf{Адрес редакции журнала <<Информатика и её применения>>:} \\




Москва 119333, ул.~Вавилова, д.~44, корп.~2, ФИЦ ИУ РАН\\[-10pt]

\

Тел.: +7\,(499)\,135-86-92\ \ Факс:  +7\,(495)\,930-45-05\\[-10pt]

 \

e-mail:   {\sf iiep@frccsc.ru} (Стригина Светлана Николаевна)\\[-10pt]

\

{\sf http://www.ipiran.ru/journal/issues/}
\end{center}
}


\def\leftkol{Правила подготовки рукописей  для публикации в журнале
<<Информатика и её применения>>}

\def\rightkol{Правила подготовки рукописей  для публикации в журнале
<<Информатика и её применения>>}


\def\leftfootline{\small{\textbf{\thepage}
\hfill ИНФОРМАТИКА И ЕЁ ПРИМЕНЕНИЯ\ \ \ том\ 18\ \ \ выпуск\ 3\ \ \ 2024}
}%
 \def\rightfootline{\small{ИНФОРМАТИКА И ЕЁ ПРИМЕНЕНИЯ\ \ \ том\ 18\ \ \ выпуск\ 3\ \ \ 2024
\hfill \textbf{\thepage}}} 
\def\stat{podg-e}
{%\hrule\par
%\vskip 7pt % 7pt
\vspace*{-24pt}
\raggedleft\Large \bf%\baselineskip=3.2ex
Requirements for manuscripts submitted to Journal
``Informatics~and~Applications'' \vskip 8pt
    \hrule
    \par
\vskip 21pt plus 6pt minus 3pt }

\label{st\stat}

\def\tit{\ }

\def\aut{\ }
\def\auf{\ }

\def\leftkol{\ }

\def\rightkol{\ }
%Requirements for manuscripts submitted to Journal
%``Informatics~and~Applications''}

\titele{\tit}{\aut}{\auf}{\leftkol}{\rightkol}

\def\leftfootline{\small{\textbf{\thepage}
\hfill INFORMATIKA I EE PRIMENENIYA~--- INFORMATICS AND APPLICATIONS\ \ \ 2019\
\ \ volume~13\ \ \ issue\ 4}
}%
 \def\rightfootline{\small{INFORMATIKA I EE PRIMENENIYA~--- INFORMATICS AND APPLICATIONS\ \ \ 2019\ \ \ volume~13\ \ \ issue\ 4
\hfill \textbf{\thepage}}}

\vspace*{-60pt}

{\small

\noindent
Journal ``Informatics and Applications'' (Inform.\ Appl.)
publishes theoretical, review, and discussion
articles on the research and development in the
field of informatics and its applications.

The journal is published in Russian.
By a special decision of the editorial
board, some articles can be published in English.


The topics covered include the following areas:
\begin{itemize}
               \item
     theoretical fundamentals of informatics; \\[-14pt]
\item
mathematical methods for studying complex systems and processes; \\[-14pt]
\item
information systems and networks;\\[-14pt]
\item
information technologies; and \\[-14pt]
\item
architecture and software of computational complexes and networks. \\[-14pt]
\end{itemize}

\noindent
\begin{enumerate}[1.]
\item The Journal publishes original articles which have not been published before and are not
intended for simultaneous publication in other editions. An article submitted to the Journal must not violate the
Copyright law. Sending the manuscript to the Editorial Board, the authors retain all rights of the
owners of the manuscript and transfer the nonexclusive rights to publish the article in Russian
(or the language of the article, if not Russian) and its distribution in Russia and abroad to the
Founders and the Editorial Board. Authors should submit a letter to the Editorial Board in the
following form:

{\bfseries\textit{Agreement on the transfer of rights to publish:}}

``\textit{We, the undersigned authors of the manuscript ``\ldots'', pass to the
Founder and the Editorial Board of the Journal ``Informatics and Applications''
the nonexclusive right to publish the manuscript of the article in Russian (or
in English) in both print and electronic versions of the Journal. We affirm
that this publication does not violate the Copyright of other persons or
organizations.}

\textit{Author(s) signature(s): (name(s), address(es), date).}

This agreement should be submitted in paper form or in the form of a scanned copy (signed by
the authors).


%The Editorial Board has the right to request from the authors an official expert conclusion that
%the submitted article has no secret data prohibited for publication. \\[-13.5pt]
\item
A submitted article should be attached with \textbf{the data on the author(s)} (see item~8). If
there are several authors, the contact person should be indicated who is responsible for
correspondence with the Editorial Board and other authors about revisions and final approval
of the proofs.\\[-13.5pt]

\item The Editorial Board of the Journal examines the article according to the established
reviewing procedure. If the authors receive their article for correction after reviewing, it does not
mean that the article is approved for publication. The corrected article should be sent to the
Editorial Board for the subsequent review and approval.\\[-13.5pt]

\item The decision on the article publication or its rejection is communicated to the authors. The
Editorial Board may also send the reviews on the submitted articles to the authors. Any
discussion upon the rejected articles is not possible.\\[-13.5pt]

\item The edited articles will be sent to the authors for proofread. The comments of the authors
to the edited text of the article should be sent to the Editorial Board as soon as possible.\\[-13.5pt]

\item The manuscript of the article should be presented electronically in the MS WORD (.doc or
.docx) or \LaTeX\ (.tex) formats, and additionally in the .pdf format. All documents
 may be sent
by e-mail or provided on a CD or diskette. A~hard copy submission is not necessary.\\[-13.5pt]

\item The recommended typesetting instructions for manuscript.

Pages parameters: format A4, portrait orientation, document margins (cm): left~--- 2.5, right~---
1.5, above~--- 2.0, below~--- 2.0, footer 1.3.

Text: font~---Times New Roman, font size~--- 14, paragraph indent~--- 0.5, line spacing~--- 1.5,
justified alignment.

The recommended manuscript size: not more than 15~pages of the specified format.
If the specified size exceeded, the editorial board is entitled to require the author
to reduce the manuscript.

Use only standard abbreviations. Avoid  abbreviations in the title and
abstract. The full term for which an abbreviation stands should precede
its first use in the text unless it is a standard unit of measurement.

All pages of the manuscript should be numbered.

The templates for the manuscript typesetting are presented on site: {\sf
http://www.ipiran.ru/journal/template.doc}.\\[-13.5pt]


%\def\leftkol{Requirements for manuscripts submitted to Journal
%``Informatics~and~Applications''}

\item The articles should enclose data both in \textbf{Russian and English}:
\begin{itemize}
\item title;\\[-13.5pt]
\item author's name and surname;\\[-13.5pt]
\item affiliation~--- organization, its address with ZIP code, city, country, and
official e-mail address;\\[-13.5pt]
\item data on authors according to the format: (see site)

{\sf http://www.ipiran.ru/journal/issues/2013\_07\_01/authors.asp}  and

{\sf  http://www.ipiran.ru/journal/issues/2013\_07\_01\_eng/authors.asp};\\[-13.5pt]

\pagebreak

\def\leftfootline{\small{\textbf{\thepage}
\hfill INFORMATIKA I EE PRIMENENIYA~--- INFORMATICS AND APPLICATIONS\ \ \ 2019\
\ \ volume~13\ \ \ issue\ 4}
}%
 \def\rightfootline{\small{INFORMATIKA I EE PRIMENENIYA~--- INFORMATICS AND APPLICATIONS\ \ \ 2019\ \ \ volume~13\ \ \ issue\ 4
\hfill \textbf{\thepage}}}


%\def\leftkol{Requirements for manuscripts submitted to Journal
%``Informatics~and~Applications''}

%\def\rightkol{Requirements for manuscripts submitted to Journal
%``Informatics~and~Applications''}



\item abstract (not less than 100 words) both in Russian and in English. Abstract is a short
summary of the article that can be published separately. The abstract is the
main source of information on the article and it could be included in leading information
systems and data bases. The abstract in English has to be an original text and should
not be an exact translation of the Russian one. Good English is required.
In abstracts, avoid references and formulae;\\[-13.5pt]
\item indexing is performed on the basis of keywords. The use of keywords from the
internationally accepted thematic Thesauri is recommended.

%\def\leftkol{Requirements for manuscripts submitted to Journal
%``Informatics~and~Applications''}

%\def\rightkol{Requirements for manuscripts submitted to Journal
%``Informatics~and~Applications''}

Important! Keywords must not be sentences;
\item Acknowledgments.
\end{itemize}

\item References. Russian references have to be presented both in English translation and Latin
transliteration (refer {\sf http://www.translit.net/ru/bgn/}).

Please take into account the following examples of Russian references appearance:

\noindent
\textbf{Article in journal:}

\Aue{Zhang, Z., and D.~Zhu}. 2008. Experimental research on the localized electrochemical
micromachining.
\textit{Rus. J.~Electrochem.}  44(8):926--930. {\sf doi:10.1134/S1023193508080077}.


\noindent
\textbf{Journal article in electronic format:}

\Aue{Swaminathan, V., E.~Lepkoswka-White, and B.\,P.~Rao}. 1999. Browsers or buyers in
cyberspace? An
investigation of electronic factors influencing electronic exchange. \textit{JCMC}
5(2). Available at: {\sf http://www.ascusc.org/jcmc/vol5/issue2/} (accessed April~28, 2011).




\noindent
\textbf{Article from the continuing publication (collection of works, proceedings):}

\Aue{Astakhov, M.\,V., and T.\,V.~Tagantsev}. 2006. Eksperimental'noe
issledovanie prochnosti soedineniy ``stal'--kompozit'' [Experimental study of
the strength of joints ``steel--composite'']. \textit{Trudy MGTU
``Matematicheskoe modelirovanie slozhnykh tekh\-ni\-che\-skikh sistem''}
[\textit{Bauman MSTU ``Mathematical Modeling of Complex Technical
Systems'' Proceedings}]. 593:125--130.

\def\leftfootline{\small{\textbf{\thepage}
\hfill INFORMATIKA I EE PRIMENENIYA~--- INFORMATICS AND APPLICATIONS\ \ \ 2019\
\ \ volume~13\ \ \ issue\ 4}
}%
 \def\rightfootline{\small{INFORMATIKA I EE PRIMENENIYA~--- INFORMATICS AND APPLICATIONS\ \ \ 2019\ \ \ volume~13\ \ \ issue\ 4
\hfill \textbf{\thepage}}}

\def\leftkol{Requirements for manuscripts submitted to Journal
``Informatics~and~Applications''}

\def\rightkol{Requirements for manuscripts submitted to Journal
``Informatics~and~Applications''}

\noindent
\textbf{Conference proceedings:}

\Aue{Usmanov, T.\,S., A.\,A.~Gusmanov, I.\,Z.~Mullagalin, R.\,Ju.~Muhametshina,
A.\,N.~Chervyakova, and
A.\,V.~Sveshnikov}. 2007. Osobennosti proektirovaniya razrabotki mestorozhdeniy
s primeneniem gidrorazryva
plasta [Features of the design of field development with the use of hydraulic fracturing].
\textit{Trudy 6-go
Mezhdu\-na\-rod\-no\-go Simpoziuma ``Novye resursosberegayushchie tekhnologii
nedropol'zovaniya i povysheniya
neftegazootdachi''} [\textit{6th  Symposium (International) ``New Energy Saving Subsoil
Technologies and
the Increasing of the Oil and Gas Impact'' Proceedings}]. Moscow. 267--272.


\noindent
\textbf{Books and other monographs:}




Lindorf, L.\,S., and L.\,G.~Mamikoniants, eds. 1972.
\textit{Ekspluatatsiya turbogeneratorov s neposredstvennym
okhlazhdeniem} [\textit{Operation of turbine generators with direct cooling}].
Moscow: Energy Publs. 352~p.


%\Aue{Latyshev, V.\,N.} 2009. \textit{Tribologiya rezaniya. Kn.~1: Frikcionnye prosessy
%pri rezanii metallov}
%[\textit{Tribology of cutting. Vol.~1: Frictional processes in metal cutting}]. Ivanovo: Ivanovskii
%State Univ. 108~p.


%\noindent
%\textbf{Unpublished material:}

%\Aue{Latypov, A.\,R., M.\,M.~Khasanov, and V.\,A.~Baikov}.
%2004. Geology and production (NGT GiD). Certificate on official registration of the computer
%program
%No.\,2004611198. (In Russian, unpubl.)

%\noindent
%\textbf{Internet-source:}

%APA Style. 2011. Available at: {\sf http://www.apastyle.org/apa-style-help.aspx} (accessed
%February~5, 2011).

%Pravila citirovaniya istochnikov [Rules for the citing of sources]. Available at: {\sf
%http://www.scribd.com/doc/1034528/} (accessed February~7, 2011).


\noindent
\textbf{Dissertation and Thesis:}

%\Aue{Semenov, V.\,I.}
%2003. Matematicheskoe modelirovanie plazmy v sisteme kompaktnyy tor. [Mathematical
%modeling of the plasma in the compact torus]. D.Sc.\ Diss. Moscow. 272~p.

\Aue{Kozhunova, O.\,S.} 2009. Tekhnologiya razrabotki semanticheskogo
slovarya informatsionnogo monitoringa [Technology of development of
semantic dictionary of information monitoring system]. PhD Thesis. Moscow: IPI RAN. 23~p.


\noindent
\textbf{State standards and patents:}

GOST 8.586.5-2005. 2007. Metodika vypolneniya izmereniy. Izmerenie raskhoda i~kolichestva
zhidkostey i gazov 
s~pomoshch'yu standartnykh suzhayushchikh ustroystv [Method of measurement.
Measurement of flow rate and volume of liquids and gases by means of orifice devices]. M.:
Standardinform
Publs. 10~p.

%\noindent
%\textbf{Patent:}

\Aue{Bolshakov, M.\,V., A.\,V.~Kulakov, A.\,N.~Lavrenov, and M.\,V.~Palkin}.
2006. Sposob orientirovaniya po krenu letatel'nogo
apparata s opti\-che\-skoy golovkoy
samonavedeniya [The way to orient on the roll of aircraft with optical homing head].
Patent RF No.\,2280590.

References in Latin transcription are presented in the original language.

References in the text are numbered according to the order of their
first appearance; the number is
placed in square brackets. All items from the reference list should be
cited.\\[-13.5pt]

\item Manuscripts and additional materials are not returned to Authors by the Editorial Board.\\[-13.5pt]

\item Submissions of files by e-mail must include:\\[-13.5pt]
\begin{itemize}
\item   the journal title and author's name in the ``Subject'' field; \\[-13.5pt]
\item   an article and additional materials have to be attached using the ``attach'' function;\\[-13.5pt]
\item   an electronic version of the article should contain the file with the text and a separate file
with figures.\\[-13.5pt]
\end{itemize}

\item ``Informatics and Applications'' journal is not a profit publication. There are no
charges for the authors as well as there are no royalties.\\[-13.5pt]
\end{enumerate}

\def\leftfootline{\small{\textbf{\thepage}
\hfill INFORMATIKA I EE PRIMENENIYA~--- INFORMATICS AND APPLICATIONS\ \ \ 2019\
\ \ volume~13\ \ \ issue\ 4}
}%
 \def\rightfootline{\small{INFORMATIKA I EE PRIMENENIYA~--- INFORMATICS AND APPLICATIONS\ \ \ 2019\ \ \ volume~13\ \ \ issue\ 4
\hfill \textbf{\thepage}}}

\def\leftkol{Requirements for manuscripts submitted to Journal
``Informatics~and~Applications''}

\def\rightkol{Requirements for manuscripts submitted to Journal
``Informatics~and~Applications''}


%\vspace*{5mm}


\begin{center}
\textbf{Editorial Board address:} \\

%ABOUT AUTHORS



FRC CSC RAS, 44, block~2, Vavilov Str., Moscow 119333, Russia\\[-10pt]

\

Ph.: +7\,(499)\,135\,86\,92,\ \ Fax: +7\,(495)\,930\,45\,05\\[-10pt]

\

 e-mail: {\sf rust@ipiran.ru} (to Prof.\ Rustem Seyful-Mulyukov)\\[-10pt]

\

 {\sf http://www.ipiran.ru/english/journal.asp}
\end{center}
 }
%\thispagestyle{myheadings}

\def\leftkol{Requirements for manuscripts submitted to Journal
``Informatics~and~Applications''}

\def\rightkol{Requirements for manuscripts submitted to Journal
``Informatics~and~Applications''}

\def\leftfootline{\small{\textbf{\thepage}
\hfill INFORMATIKA I EE PRIMENENIYA~--- INFORMATICS AND APPLICATIONS\ \ \ 2019\
\ \ volume~13\ \ \ issue\ 4}
}%
 \def\rightfootline{\small{INFORMATIKA I EE PRIMENENIYA~--- INFORMATICS AND APPLICATIONS\ \ \ 2019\ \ \ volume~13\ \ \ issue\ 4
\hfill \textbf{\thepage}}}

 \label{end\stat}

\newpage

%\vspace*{-60pt} {\small
{\baselineskip=9.1pt
\section*{Правила подготовки рукописей статей для публикации в журнале
<<Информатика и её применения>>}

\thispagestyle{empty}

 Журнал <<Информатика и её применения>> публикует
теоретические, обзорные и дискуссионные статьи, посвященные научным
исследованиям и разработкам в области информатики и ее приложений. Журнал
издается на русском языке. По специальному решению редколлегии отдельные статьи,
в виде исключения, могут печататься на английском языке.
Тематика журнала охватывает следующие направления:
\begin{itemize}
\item теоретические основы информатики; %\\[-13.5pt]
\item математические методы исследования сложных систем и процессов; %\\[-13.5pt]
\item информационные системы и сети; %\\[-13.5pt]
\item информационные технологии; %\\[-13.5pt]
\item архитектура и программное
обеспечение вычислительных комплексов и сетей.
\end{itemize}
\begin{enumerate}
\item В журнале печатаются результаты, ранее не
опубликованные и не предназначенные к одновременной публикации в других
изданиях. Публикация не должна нарушать закон об авторских правах. Направляя
свою рукопись в редакцию, авторы автоматически передают учредителям и
редколлегии неисключительные права на издание данной статьи на русском языке и
на ее распространение в России и за рубежом. При этом за авторами сохраняются
все права как собственников данной рукописи. В связи с этим авторами должно
быть представлено в редакцию письмо в следующей форме:
Соглашение о передаче права на публикацию:

\textit{<<Мы, нижеподписавшиеся, авторы рукописи <<$\qquad\qquad$>>, передаем
учредителям и редколлегии журнала <<Информатика и её применения>>
неисключительное право опубликовать данную рукопись статьи на русском языке как
в печатной, так и в электронной версиях журнала. Мы подтверждаем, что данная
публикация не нарушает авторского права других лиц или организаций. Подписи
авторов: (ф.\,и.\,о., дата, адрес)>>.}

Указанное соглашение может быть представлено 
как в бумажном виде, так и в виде отсканированной копии (с подписями авторов).


Редколлегия вправе запросить у авторов экспертное заключение о возможности
опубликования представленной статьи в открытой печати. %\\[-13.5pt]
\item Статья
подписывается всеми авторами. На отдельном листе представляются данные автора
(или всех авторов): фамилия, полные имя и отчество, телефон, факс, e-mail,
почтовый адрес. Если работа выполнена несколькими авторами, указывается фамилия
одного из них, ответственного за переписку с редакцией. %\\[-13.5pt]
\item Редакция журнала
осуществляет самостоятельную экспертизу присланных статей. Возвращение рукописи
на доработку не означает, что статья уже принята к печати. Доработанный вариант
с ответом на замечания рецензента необходимо прислать в редакцию. %\\[-13.5pt]
\item Решение
редакционной коллегии о принятии статьи к печати или ее отклонении сообщается
авторам. Редколлегия не обязуется направлять рецензию авторам отклоненной
статьи. %\\[-13.5pt]
\item Корректура статей высылается авторам для просмотра. Редакция
просит авторов присылать свои замечания в кратчайшие сроки. %\\[-13.5pt]
\item При
подготовке рукописи в MS Word рекомендуется использовать следующие настройки.
Параметры страницы: формат~--- А4; ориентация~--- книжная; поля (см): внутри~---
2,5, снаружи~--- 1,5, сверху~--- 2, снизу~--- 2, от края до нижнего
колонтитула~--- 1,3. Основной текст: стиль~--- <<Обычный>>: шрифт Times New
Roman, размер 14~пунктов, абзацный отступ~--- 0,5~см, 1,5 интервала,
выравнивание~--- по ширине. Рекомендуемый объем рукописи~--- не свыше
25~страниц указанного формата. Ознакомиться с шаблонами, содержащими примеры
оформления, можно по адресу в Интернете:
\textsf{http://www.ipiran.ru/journal/template.doc}.
\item К рукописи, предоставляемой в 2-х
экземплярах, обязательно прилагается электронная версия статьи (как правило, в
форматах MS WORD (.doc) или \LaTeX\ (.tex), а также~--- дополнительно~--- в
формате .pdf) на дискете, лазерном диске или по электронной почте. Сокращения
слов, кроме стандартных, не применяются. Все страницы рукописи должны быть
пронумерованы. %\\[-13.5pt]
\item Статья должна содержать следующую информацию на русском и
английском языках: название, Ф.И.О. авторов, места работы авторов и их
электронные адреса, подробные сведения об авторах, оформленные в соответствии с форматом, 
определяемым файлами {\sf http://www.ipiran.ru/journal/issues/2011\_05\_01/authors.asp} и 
{\sf http://www.ipiran.ru/journal/issues/2011\_01\_eng/authors.asp},
аннотация (не более 100~слов), ключевые слова. Ссылки на
литературу в тексте статьи нумеруются (в квадратных скобках) и располагаются в
порядке их первого упоминания. В~списке литературы не должно быть позиций, на которые нет ссылки в тексте статьи.
Все фамилии авторов, заглавия статей, названия
книг, конференций и~т.\,п.\ даются на языке оригинала, если этот язык
использует кириллический или латинский алфавит. %\\[-13.5pt]
\item Присланные в редакцию материалы авторам не возвращаются.
\item При отправке файлов по электронной
почте просим придерживаться следующих правил:
\begin{itemize}
\item указывать в поле subject (тема) название журнала и фамилию автора; %\\[-13.5pt]
\item использовать attach (присоединение); %\\[-13.5pt]
\item в случае больших объемов информации возможно
использование общеизвестных архиваторов (ZIP, RAR); %\\[-13.5pt]
\item в состав электронной версии статьи должны входить: файл, содержащий текст статьи, и файл(ы),
содержащий(е) иллюстрации. %\\[-13.5pt]
\end{itemize}
\item Журнал <<Информатика и её применения>> является некоммерческим изданием. 
Плата за публикацию с авторов не взимается, гонорар авторам не выплачивается.
\end{enumerate}
\thispagestyle{empty}
\textbf{Адрес редакции:} Москва 119333,
ул.~Вавилова, д.~44, корп.~2, ИПИ РАН\\
\hphantom{\textbf{Адрес редакции:} }Тел.: +7 (499) 135-86-92\ \
Факс:  +7 (495) 930-45-05\ \  E-mail:   rust@ipiran.ru }
}

%\include{ipi-ind}

%\tableofcontents

\end{document}


%\tableofcontents

%\end{document}





%\def\stat{cont}
{%\hrule\par
%\vskip 7pt % 7pt
\raggedleft\Large \bf%\baselineskip=3.2ex
А\,В\,Т\,О\,Р\,С\,К\,И\,Й\ \ У\,К\,А\,З\,А\,Т\,Е\,Л\,Ь\ \ З\,А\ \ 2\,0\,0\,7 г. \vskip 17pt
    \hrule
    \par
\vskip 21pt plus 6pt minus 3pt }

\label{st\stat}

\def\tit{\ }

\def\aut{\ }
\def\auf{\ }

\def\leftkol{\ } % ENGLISH ABSTRACTS}

\def\rightkol{\ } %ENGLISH ABSTRACTS}

\titele{\tit}{\aut}{\auf}{\leftkol}{\rightkol}


\contentsline {chapter}{\ }{Выпуск \quad Стр.} 
\contentsline {section}{\textbf{Батракова Д.\,А., Королев В.\,Ю., Шоргин С.\,Я.}\ \ Новый метод вероятностно-ста\-ти\-сти\-че\-ско\-го анализа информационных потоков в\nobreakspace {}телекоммуникационных сетях}{\qquad 1 \qquad 40} 
\contentsline {section}{\textbf{Борисов А.\,В.}\ \ Байесовское оценивание в системах наблюдения с\nobreakspace {}марковскими скачкообразными процессами: игровой подход}{\qquad 2 \qquad 65}
\contentsline {section}{\textbf{Босов А.\,В., Иванов А.\,В.}\ \ Программная инфраструктура информационного Web-пор\-тала}{\qquad 2 \qquad 50}
\contentsline {section}{\textbf{Захаров В.\,Н., Калиниченко Л.\,А., Соколов И.\,А., Ступников С.\,А.}\ \ Конструирование канонических информационных моделей для интегрированных информационных систем}{\qquad 2 \qquad 15}
\contentsline {section}{\textbf{Захаров В.\,Н., Козмидиади В.\,А.}\ \ Средства обеспечения отказоустойчивости при\-ло\-жений}{\qquad 1 \qquad 14} 
\contentsline {section}{\textbf{Иванов А.\,В.}\ \ см. Босов А.\,В.\hfill\hfill\hfill\hfill\hfill\hfill\hfill\hfill\hfill\hfill\hfill\hfill\hfill\hfill\hfill\hfill\hfill\hfill\hfill\hfill\hfill\hfill\hfill\hfill\hfill\hfill\hfill\hfill\hfill\hfill\hfill\hfill\hfill\hfill\hfill}{\ }
\contentsline {section}{\textbf{Ильин В.\,Д., Соколов И.\,А.}\ \ Символьная модель системы знаний информатики в\nobreakspace {}че\-ло\-ве\-ко-автоматной среде}{\qquad 1 \qquad 66} 
\contentsline {section}{\textbf{Калиниченко Л.\,А.}\ \ см. Захаров В.\,Н.\hfill\hfill\hfill\hfill\hfill\hfill\hfill\hfill\hfill\hfill\hfill\hfill\hfill\hfill\hfill\hfill\hfill\hfill\hfill\hfill\hfill\hfill\hfill\hfill\hfill\hfill\hfill\hfill\hfill\hfill\hfill\hfill\hfill\hfill\hfill}{\ }
\contentsline {section}{\textbf{Козеренко Е.\,Б.}\ \ Лингвистическое моделирование для систем машинного перевода и обработки знаний}{\qquad 1 \qquad 54} 
\contentsline {section}{\textbf{Козмидиади В.\,А.}\ \ см. Захаров В.\,Н.\hfill\hfill\hfill\hfill\hfill\hfill\hfill\hfill\hfill\hfill\hfill\hfill\hfill\hfill\hfill\hfill\hfill\hfill\hfill\hfill\hfill\hfill\hfill\hfill\hfill\hfill\hfill\hfill\hfill\hfill\hfill\hfill\hfill\hfill\hfill }{\ } 
\contentsline {section}{\textbf{Королев В.\,Ю.}\ \ см. Батракова Д.\,А.\hfill\hfill\hfill\hfill\hfill\hfill\hfill\hfill\hfill\hfill\hfill\hfill\hfill\hfill\hfill\hfill\hfill\hfill\hfill\hfill\hfill\hfill\hfill\hfill\hfill\hfill\hfill\hfill\hfill\hfill\hfill\hfill\hfill\hfill\hfill}{\ } 
\contentsline {section}{\textbf{Кудрявцев А.\,А., Шоргин С.\,Я.}\ \ Байесовский подход к\nobreakspace {}анализу систем массового обслуживания и\nobreakspace {}показателей надежности}{\qquad 2 \qquad 76}
\contentsline {section}{\textbf{Печинкин А.\,В., Соколов И.\,А., Чаплыгин В.\,В.}\ \ Многолинейная система массового обслуживания с конечным накопителем и ненадежными приборами}{\qquad 1 \qquad 27} 
\contentsline {section}{\textbf{Печинкин А.\,В., Соколов И.\,А., Чаплыгин В.\,В.}\ \ Стационарные характеристики многолинейной\nobreakspace {}системы массового обслуживания с\nobreakspace {}одновременными отказами приборов}{\qquad 2 \qquad 39}
\contentsline {section}{\textbf{Синицын И.\,Н.}\ \ Корреляционные методы построения аналитических информационных моделей флуктуаций полюса Земли по априорным данным}{\qquad 2 \qquad \hphantom{9}2}
\contentsline {section}{\textbf{Синицын И.\,Н.}\ \ Развитие теории фильтров Пугачева для оперативной обработки информации в стохастических системах}{{\qquad 1 \qquad \hphantom{9}3}} 
\contentsline {section}{\textbf{Соколов И.\,А.}\ \ см. Захаров В.\,Н.\hfill\hfill\hfill\hfill\hfill\hfill\hfill\hfill\hfill\hfill\hfill\hfill\hfill\hfill\hfill\hfill\hfill\hfill\hfill\hfill\hfill\hfill\hfill\hfill\hfill\hfill\hfill\hfill\hfill\hfill\hfill\hfill\hfill\hfill\hfill}{\ }
\contentsline {section}{\textbf{Соколов И.\,А.}\ \ см. Ильин В.\,Д.\hfill\hfill\hfill\hfill\hfill\hfill\hfill\hfill\hfill\hfill\hfill\hfill\hfill\hfill\hfill\hfill\hfill\hfill\hfill\hfill\hfill\hfill\hfill\hfill\hfill\hfill\hfill\hfill\hfill\hfill\hfill\hfill\hfill\hfill\hfill}{\ } 
\contentsline {section}{\textbf{Соколов И.\,А.}\ \ см. Печинкин А.\,В.\hfill\hfill\hfill\hfill\hfill\hfill\hfill\hfill\hfill\hfill\hfill\hfill\hfill\hfill\hfill\hfill\hfill\hfill\hfill\hfill\hfill\hfill\hfill\hfill\hfill\hfill\hfill\hfill\hfill\hfill\hfill\hfill\hfill\hfill\hfill}{\ } 
\contentsline {section}{\textbf{Соколов И.\,А.}\ \ см. Печинкин А.\,В.\hfill\hfill\hfill\hfill\hfill\hfill\hfill\hfill\hfill\hfill\hfill\hfill\hfill\hfill\hfill\hfill\hfill\hfill\hfill\hfill\hfill\hfill\hfill\hfill\hfill\hfill\hfill\hfill\hfill\hfill\hfill\hfill\hfill\hfill\hfill}{\ }
\contentsline {section}{\textbf{Ступников С.\,А.}\ \ см. Захаров В.\,Н.\hfill\hfill\hfill\hfill\hfill\hfill\hfill\hfill\hfill\hfill\hfill\hfill\hfill\hfill\hfill\hfill\hfill\hfill\hfill\hfill\hfill\hfill\hfill\hfill\hfill\hfill\hfill\hfill\hfill\hfill\hfill\hfill\hfill\hfill\hfill}{\ }
\contentsline {section}{\textbf{Чаплыгин В.\,В.}\ \ см. Печинкин А.\,В.\hfill\hfill\hfill\hfill\hfill\hfill\hfill\hfill\hfill\hfill\hfill\hfill\hfill\hfill\hfill\hfill\hfill\hfill\hfill\hfill\hfill\hfill\hfill\hfill\hfill\hfill\hfill\hfill\hfill\hfill\hfill\hfill\hfill\hfill\hfill}{\ } 
\contentsline {section}{\textbf{Чаплыгин В.\,В.}\ \ см. Печинкин А.\,В.\hfill\hfill\hfill\hfill\hfill\hfill\hfill\hfill\hfill\hfill\hfill\hfill\hfill\hfill\hfill\hfill\hfill\hfill\hfill\hfill\hfill\hfill\hfill\hfill\hfill\hfill\hfill\hfill\hfill\hfill\hfill\hfill\hfill\hfill\hfill}{\ }
\contentsline {section}{\textbf{Шоргин С.\,Я.}\ \ см. Батракова Д.\,А.\hfill\hfill\hfill\hfill\hfill\hfill\hfill\hfill\hfill\hfill\hfill\hfill\hfill\hfill\hfill\hfill\hfill\hfill\hfill\hfill\hfill\hfill\hfill\hfill\hfill\hfill\hfill\hfill\hfill\hfill\hfill\hfill\hfill\hfill\hfill}{\ } 
\contentsline {section}{\textbf{Шоргин С.\,Я.}\ \ см. Кудрявцев А.\,А.\hfill\hfill\hfill\hfill\hfill\hfill\hfill\hfill\hfill\hfill\hfill\hfill\hfill\hfill\hfill\hfill\hfill\hfill\hfill\hfill\hfill\hfill\hfill\hfill\hfill\hfill\hfill\hfill\hfill\hfill\hfill\hfill\hfill\hfill\hfill}{\ }
%\thispagestyle{myheadings}
\def\leftfootline{\small{\textbf{\thepage}
\hfill ИНФОРМАТИКА И ЕЁ ПРИМЕНЕНИЯ\ \ \ том~1\ \ \ выпуск~2\ \ \ 2007}
}%
 \def\rightfootline{\small{ИНФОРМАТИКА И ЕЁ ПРИМЕНЕНИЯ\ \ \ том~1\ \ \ выпуск~2\ \ \ 2007
 \hfill \textbf{\thepage}}}
 \label{end\stat}

%\def\stat{cont-e}
{%\hrule\par
%\vskip 7pt % 7pt
\raggedleft\Large \bf%\baselineskip=3.2ex
2\,0\,0\,7\ \ A\,U\,T\,H\,O\,R\ \ I\,N\,D\,E\,X \vskip 17pt
    \hrule
    \par
\vskip 21pt plus 6pt minus 3pt }

\label{st\stat}

\def\tit{\ }

\def\aut{\ }
\def\auf{\ }

\def\leftkol{\ } % ENGLISH ABSTRACTS}

\def\rightkol{\ } %ENGLISH ABSTRACTS}

\titele{\tit}{\aut}{\auf}{\leftkol}{\rightkol}


\contentsline {chapter}{\ }{Issue \quad Page} 
\contentsline {subsection}{\textbf{Batrakova D.\,A., Korolev V.\,Yu., Shorgin S.\,Ya.}\ \ A New Method for the Probabilistic and Statistical Analysis of Information Flows in Telecommunication Networks}{\qquad 1 \qquad 40} 
\contentsline {subsection}{\textbf{Borisov A.\,V.}\ \ Bayesian Estimation in\nobreakspace {}Observation Systems with\nobreakspace {}Markov Jump Processes: Game-Theoretic Approach}{\qquad 2 \qquad 65} 
\contentsline {subsection}{\textbf{Bosov A.\,V., Ivanov A.\,V.}\ \ Linguistic Simulation for Machine Translation and Knowledge Management Systems}{\qquad 2 \qquad 50} 
\contentsline {subsection}{\textbf{Chaplygin V.\,V.} see Pechinkin A.\,V.\hfill\hfill\hfill\hfill\hfill\hfill\hfill\hfill\hfill\hfill\hfill\hfill\hfill\hfill\hfill\hfill\hfill\hfill\hfill\hfill\hfill\hfill\hfill\hfill\hfill\hfill\hfill\hfill\hfill\hfill\hfill\hfill\hfill\hfill\hfill}{\ }
\contentsline {subsection}{\textbf{Chaplygin V.\,V.} see Pechinkin A.\,V.\hfill\hfill\hfill\hfill\hfill\hfill\hfill\hfill\hfill\hfill\hfill\hfill\hfill\hfill\hfill\hfill\hfill\hfill\hfill\hfill\hfill\hfill\hfill\hfill\hfill\hfill\hfill\hfill\hfill\hfill\hfill\hfill\hfill\hfill\hfill}{\ }
\contentsline {subsection}{\textbf{Ilyin V.\,D., Sokolov I.\,A.}\ \ The Symbol Model of Informatics Knowledge System in Human-Automaton Environment}{\qquad 1 \qquad 66} 
\contentsline {subsection}{\textbf{Ivanov A.\,V.} see Bosov A.\,V.\hfill\hfill\hfill\hfill\hfill\hfill\hfill\hfill\hfill\hfill\hfill\hfill\hfill\hfill\hfill\hfill\hfill\hfill\hfill\hfill\hfill\hfill\hfill\hfill\hfill\hfill\hfill\hfill\hfill\hfill\hfill\hfill\hfill\hfill\hfill}{\ }
\contentsline {subsection}{\textbf{Kalinichenko L.\,A.} see Zakharov V.\,N.\hfill\hfill\hfill\hfill\hfill\hfill\hfill\hfill\hfill\hfill\hfill\hfill\hfill\hfill\hfill\hfill\hfill\hfill\hfill\hfill\hfill\hfill\hfill\hfill\hfill\hfill\hfill\hfill\hfill\hfill\hfill\hfill\hfill\hfill\hfill}{\ }
\contentsline {subsection}{\textbf{Korolev V.\,Yu.} see Batrakova D.\,A.\hfill\hfill\hfill\hfill\hfill\hfill\hfill\hfill\hfill\hfill\hfill\hfill\hfill\hfill\hfill\hfill\hfill\hfill\hfill\hfill\hfill\hfill\hfill\hfill\hfill\hfill\hfill\hfill\hfill\hfill\hfill\hfill\hfill\hfill\hfill}{\ }
\contentsline {subsection}{\textbf{Kozerenko E.\,B.}\ \ Linguistic Simulation for Machine Translation and Knowledge Management Systems}{\qquad 1 \qquad 54} 
\contentsline {subsection}{\textbf{Kozmidiady V.\,A.} see Zakharov V.\,N.\hfill\hfill\hfill\hfill\hfill\hfill\hfill\hfill\hfill\hfill\hfill\hfill\hfill\hfill\hfill\hfill\hfill\hfill\hfill\hfill\hfill\hfill\hfill\hfill\hfill\hfill\hfill\hfill\hfill\hfill\hfill\hfill\hfill\hfill\hfill}{\ }
\contentsline {subsection}{\textbf{Kudryavtsev A.\,A., Shorgin S.\,Ya.}\ \ Bayesian Approach to Queueing Systems and Reliability Characteristics}{\qquad 2 \qquad 76} 
\contentsline {subsection}{\textbf{Pechinkin A.\,V., Sokolov I.\,A., Chaplygin V.\,V.}\ \ Multichannel Queuing System with Finite Buffer and Unreliable Servers}{\qquad 1 \qquad 27} 
\contentsline {subsection}{\textbf{Pechinkin A.\,V., Sokolov I.\,A., Chaplygin V.\,V.}\ \ Stationary Characteristics of a Multichannel Queueing System with\nobreakspace {}Simultaneous Refusals of Servers}{\qquad 2 \qquad 39} 
\contentsline {subsection}{\textbf{Shorgin S.\,Ya.} see Batrakova D.\,A.\hfill\hfill\hfill\hfill\hfill\hfill\hfill\hfill\hfill\hfill\hfill\hfill\hfill\hfill\hfill\hfill\hfill\hfill\hfill\hfill\hfill\hfill\hfill\hfill\hfill\hfill\hfill\hfill\hfill\hfill\hfill\hfill\hfill\hfill\hfill}{\ }
\contentsline {subsection}{\textbf{Shorgin S.\,Ya.} see Kudryavtsev A.\,A.\hfill\hfill\hfill\hfill\hfill\hfill\hfill\hfill\hfill\hfill\hfill\hfill\hfill\hfill\hfill\hfill\hfill\hfill\hfill\hfill\hfill\hfill\hfill\hfill\hfill\hfill\hfill\hfill\hfill\hfill\hfill\hfill\hfill\hfill\hfill}{\ }
\contentsline {subsection}{\textbf{Sinitsyn I.\,N.}\ \ Correlational Methods for Analytical Informational Models of the Earth Pole Fluctuations Design Based on a priori Data}{\qquad 2 \qquad \hphantom{9}2}
\contentsline {subsection}{\textbf{Sinitsyn I.\,N.}\ \ Development of Pugachev Filtering for Stochastic Systems}{\qquad 1 \qquad \hphantom{9}3}
\contentsline {subsection}{\textbf{Sokolov I.\,A.} see Ilyin V.\,D.\hfill\hfill\hfill\hfill\hfill\hfill\hfill\hfill\hfill\hfill\hfill\hfill\hfill\hfill\hfill\hfill\hfill\hfill\hfill\hfill\hfill\hfill\hfill\hfill\hfill\hfill\hfill\hfill\hfill\hfill\hfill\hfill\hfill\hfill\hfill}{\ }
\contentsline {subsection}{\textbf{Sokolov I.\,A.} see Pechinkin A.\,V.\hfill\hfill\hfill\hfill\hfill\hfill\hfill\hfill\hfill\hfill\hfill\hfill\hfill\hfill\hfill\hfill\hfill\hfill\hfill\hfill\hfill\hfill\hfill\hfill\hfill\hfill\hfill\hfill\hfill\hfill\hfill\hfill\hfill\hfill\hfill}{\ }
\contentsline {subsection}{\textbf{Sokolov I.\,A.} see Pechinkin A.\,V.\hfill\hfill\hfill\hfill\hfill\hfill\hfill\hfill\hfill\hfill\hfill\hfill\hfill\hfill\hfill\hfill\hfill\hfill\hfill\hfill\hfill\hfill\hfill\hfill\hfill\hfill\hfill\hfill\hfill\hfill\hfill\hfill\hfill\hfill\hfill}{\ }
\contentsline {subsection}{\textbf{Sokolov I.\,A.} see Zakharov V.\,N.\hfill\hfill\hfill\hfill\hfill\hfill\hfill\hfill\hfill\hfill\hfill\hfill\hfill\hfill\hfill\hfill\hfill\hfill\hfill\hfill\hfill\hfill\hfill\hfill\hfill\hfill\hfill\hfill\hfill\hfill\hfill\hfill\hfill\hfill\hfill}{\ }
\contentsline {subsection}{\textbf{Stupnikov S.\,A.} see Zakharov V.\,N.\hfill\hfill\hfill\hfill\hfill\hfill\hfill\hfill\hfill\hfill\hfill\hfill\hfill\hfill\hfill\hfill\hfill\hfill\hfill\hfill\hfill\hfill\hfill\hfill\hfill\hfill\hfill\hfill\hfill\hfill\hfill\hfill\hfill\hfill\hfill}{\ }
\contentsline {subsection}{\textbf{Zakharov V.\,N., Kalinichenko L.\,A., Sokolov I.\,A., Stupnikov S.\,A.}\ \ Development of Canonical Information Models for Integrated Information Systems}{\qquad 2 \qquad 15} 
\contentsline {subsection}{\textbf{Zakharov V.\,N., Kozmidiady V.\,A.}\ \ Means Providing Applications Fault Tolerance}{\qquad 1 \qquad 14} 
\def\leftfootline{\small{\textbf{\thepage}
\hfill ИНФОРМАТИКА И ЕЁ ПРИМЕНЕНИЯ\ \ \ том~1\ \ \ выпуск~2\ \ \ 2007}
}%
 \def\rightfootline{\small{ИНФОРМАТИКА И ЕЁ ПРИМЕНЕНИЯ\ \ \ том~1\ \ \ выпуск~2\ \ \ 2007
 \hfill \textbf{\thepage}}}
 \label{end\stat}


%\tableofcontents


\end{document}

\newcommand{\Ack}{\subsection*{\protect\large\bf Acknowledgments}}