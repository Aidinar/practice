\def \SS{{\frak S}}
\def \SK{{\frak K}}
\def \SL{{\frak L}}
\def\bs{\backslash}


\def\stat{nazarova}

\def\tit{АНАЛИЗ РАЗРЕЗНЫХ ПОВРЕЖДЕНИЙ В~МНОГОПОЛЮСНЫХ СЕТЯХ}

\def\titkol{Анализ разрезных повреждений в~многополюсных сетях}

\def\aut{Ю.\,Е.~Малашенко$^1$, И.\,А.~Назарова$^2$, Н.\,М.~Новикова$^3$}

\def\autkol{Ю.\,Е.~Малашенко, И.\,А.~Назарова, Н.\,М.~Новикова}

\titel{\tit}{\aut}{\autkol}{\titkol}

\index{Малашенко Ю.\,Е.}
\index{Назарова И.\,А.}
\index{Новикова Н.\,М.}
\index{Malashenko Yu.\,E.}
\index{Nazarova I.\,A.}
\index{Novikova N.\,M.}




%{\renewcommand{\thefootnote}{\fnsymbol{footnote}} \footnotetext[1]
%{Работа выполнена при частичной поддержке РФФИ (проект 16-07-00677).}}


\renewcommand{\thefootnote}{\arabic{footnote}}
\footnotetext[1]{Вычислительный центр им.\ А.\,А.~Дородницына Федерального исследовательского 
центра <<Информатика и~управление>> Российской академии наук, \mbox{malash09@ccas.ru}}
\footnotetext[2]{Вычислительный центр им.\ А.\,А.~Дородницына Федерального исследовательского центра 
<<Информатика и~управление>> Российской академии наук, \mbox{irina-nazar@yandex.ru}}
\footnotetext[3]{Вычислительный центр им.\ А.\,А.~Дородницына Федерального исследовательского центра 
<<Информатика и~управление>> Российской академии наук, \mbox{n\_novikova@umail.ru}}


\vspace*{-10pt}



\Abst{Предложен метод оценки изменения функциональных возможностей многополюсной 
потоковой сетевой системы после повреждающего  воздействия. Для каждой 
стоковой дуги вычисляется максимальный поток, не зависящий от величины 
потока по остальным стоковым дугам. Разрезным структурным по\-вреж\-де\-ни\-ем 
считается удаление из сети всех дуг, образующих минимальный разрез, 
соответствующий максимальному потоку по некоторой стоковой дуге. Среди 
найденных структурных по\-вреж\-де\-ний по введенному критерию выбираются критически опасные.
 Для каждой дуги, принадлежащей хотя бы одному разрезному структурному по\-вреж\-де\-нию, 
 проводится количественная оценка последствий ее разрушения. Описанный подход 
 предлагается использовать при исследовании уязвимости территориально распределенных 
 многопользовательских систем передачи одного вида продукта, имеющих сетевую 
 структуру связей.}

\KW{однопродуктовая потоковая сеть; структурная  уязвимость  сети; 
многополюсная потоковая модель}

\DOI{10.14357/19922264180305}
  
%\vspace*{4pt}


\vskip 10pt plus 9pt minus 6pt

\thispagestyle{headings}

\begin{multicols}{2}

\label{st\stat}


\section{Введение}

В~\cite{MalInf17} на примере многополюсной однопродуктовой модели изучались
 возможности передачи пользователям запрашиваемых величин потоков при полном 
 разрушении отдельных элементов исходной сетевой системы. Последовательно 
 определялись гарантированные оценки величин потоков, которые можно передать в~сети 
 после аварии. В~\cite{MalInf181} был описан  метод  вычисления  векторов ущерба 
 для пользователей сети при различных по\-вреж\-да\-ющих воздействиях. 
 Предлагалось одновременно анализировать уязвимость и~живучесть  сети, а~также 
 вычислять величины потоков, которые можно гарантированно доставить всем пользователям.

В~настоящей работе в~рамках того же формализма однопродуктового потока и~многосто\-ковой 
модели рассматриваются предельные функ\-циональные возможности сети и~их изменение\linebreak 
после \mbox{аварий}. Для оценки функциональных характеристик сис\-те\-мы  для каждой 
стоковой дуги вычисляется максимальный поток, не зависящий от величины потока 
по остальным стоковым дугам, и~определяется соответствующий ему минимальный 
разрез~\cite{ford}. Для сети вводится понятие критически опасного по\-вреж\-де\-ния 
как по\-вреж\-де\-ния, в~результате которого передача потока из источника хотя бы в~одну 
стоковую вершину оказывается невозможной. Разрезным структурным по\-вреж\-де\-ни\-ем 
считается удаление из сети всех дуг, соответствующих минимальному разрезу для 
некоторой стоковой вершины. Для каждой стоковой вершины выбирается свой 
минимальный разрез. После моделирования такого по\-вреж\-де\-ния сети определяются 
значения максимального потока по всем стоковым дугам и~подсчитывается 
число нулевых значений. На основании проведенных расчетов выявляются 
стоковые вершины, передача потоков для которых чаще других оказывается в~зоне 
критически опасных по\-вреж\-де\-ний, вычисляются показатели, позволяющие оценить 
уязвимость системы в~случае разрезных по\-вреж\-де\-ний, строится таб\-ли\-ца сравнительного 
<<вклада>> в~уязвимость сети каждой из ее дуг, относящихся к~множеству обра\-зу\-ющих 
критические разрезы.

\vspace*{-4pt}

\section{Многополюсная потоковая модель}

\vspace*{-2pt}

Сеть передачи единственного вида продукта в~многополюсной 
системе будем описывать ориентированным графом $\overline {\mathcal{G}} \hm= 
\langle \overline{\mathcal{V}}, \overline{\mathcal{L}} \rangle$ без петель, 
который определяется множеством вершин (узлов) $\overline{\mathcal{V}}
\hm = \{v_1,v_2,\ldots,v_ N\}$, где $|\overline{\mathcal{V}}| \hm= N$, 
$\mathcal{N}$~--- множество индексов вершин,
и~множеством направленных дуг
$\overline {\mathcal{L}}\hm =\{l_{ij} \ | \ i \in \mathcal{N}, \ j \in \mathcal{N}, 
i \not= j \}$,
соединяющих вершины. Здесь
$l_{ij} \hm= (v_i, v_j)$ ~---  дуга, ведущая из вершины~$v_i$ в~вершину~$v_j$, 
$|\overline{\mathcal{L}}| \hm= L$.

Обозначим через $\mathcal{V}_\SS$ и~$\mathcal{V}_\SK$  множества вершин 
графа $\overline{\mathcal{G}}$, являющихся соответственно источниками и~стоками 
для потока, который передается по многополюсной сети;
$\mathcal{N}_\SS$ и~$\mathcal{N}_\SK$~---  множества индексов вер\-шин-источ\-ни\-ков 
и~вер\-шин-сто\-ков:
$\mathcal{V}_\SS \hm= \{v_i | \ i \in \mathcal{N}_\SS \}$,   
$|\mathcal{V}_\SS| \hm= S$,  
$\mathcal{{V}}_\SS \hm\subset \mathcal{\overline{V}}$, 
$\mathcal{N}_\SS \hm\subset \mathcal{N}$,  $S \hm\geq 1,$
$\mathcal{V}_\SK \hm= \{v_i | \ i \in \mathcal{N}_\SK\}$,   
$|\mathcal{V}_\SK | \hm= K$, $\mathcal{V}_\SK \hm\subset \mathcal{\overline{V}}$,  
$\mathcal{N}_\SK \hm\subset \mathcal{N}$, $K \hm\geq 1, $
$\mathcal{V}_\SS \bigcap \mathcal{V}_\SK \hm= \emptyset. $

Считается, что на дугах графа $\overline{\mathcal{G}}$ заданы веса~--- 
значения~$d_{ij}$ пропускной спо\-соб\-ности дуг~$l_{ij}$. 
Вектор~$d$ определяет максимально до\-пус\-ти\-мую величину потока  по дугам,
$d \hm= \{d_{ij} \ |\  d_{ij}\hm \geq 0$, $l_{ij}\hm\in \overline{\mathcal{L}}\}$.
К~графу~$\overline{\mathcal{G}}$ добавим:
\begin{description}
\item[\,]
$v_0$~---  единственный источник потока бесконечной мощности  и~дуги 
$(v_0, v_j)$,   $j \hm\in \mathcal{N}_\SS$, соединяющие~$v_0$ с~каждым 
уз\-лом-источ\-ни\-ком. Для каждой дуги~$l_{0j}$ формально определим 
верхнее ограничение~$d_{0j}$, которое соответствует величине максимального 
потока из уз\-ла-источ\-ни\-ка~$v_j$ в~сис\-те\-му. Будем считать, что  
дуги $(v_0, v_j)$, $j \hm\in \mathcal{N}_\SS$, являются ду\-га\-ми-источ\-ни\-ка\-ми, 
и~обозначим их множество через
$ \hat{\mathcal{L}}\hm=\{l_{0 j} \ | \  j \hm\in \mathcal{N}_\SS \}$, 
$| \hat{\mathcal{L}}| \hm=  S; $
\item[\,]
$v_{N+1}$~---  единственный узел-сток бесконечного объема и~дуги 
$(v_i, v_{N+1})$,  $i \hm\in \mathcal{N}_\SK$, соединяющие каждый 
узел-сток с~$v_{N+1}$. Назовем направленные дуги $(v_i, v_{N+1})$, 
$i \in \mathcal{N}_\SK$,  ду\-га\-ми-сто\-ка\-ми, или стоковыми дугами, и~обозначим 
их множество через
$\tilde{\mathcal{L}}\hm=\{ l_{iN + 1}  |  i  \in \mathcal{N}_\SK\}$, 
$|\tilde{\mathcal{L}}| \hm=  K.$
Для каждой дуги~$l_{iN+1}$ формально определим верхнее ограничение~$d_{iN+1}$, 
которое соответствует величине максимального потока из системы в~$v_i$. 
Значение~$d_{iN+1}$ задает верхний предел для величины потока,  
покидающего систему через узел~$v_i$.
Ориентированный граф, который определяется множествами вершин $\mathcal{V}\hm = 
\overline{\mathcal{V}} \bigcup \{v_0, v_{N+1}\}$ и~дуг 
$\mathcal{L} \hm= \tilde{\mathcal{L}} \bigcup \overline{\mathcal{L}} 
\bigcup \hat{\mathcal{L}}$, обозначим $\mathcal{G} \hm= \langle {\mathcal{V}}, 
{\mathcal{L}} \rangle $.
\end{description}

Для графа $\mathcal{G}$ введем:
$x_{ij}$~--- поток по дуге~$l_{ij}$, $l_{ij}\hm\in {\mathcal{L}}$, 
протекающий в~соответствии с~ее направлением;
$\mathcal{N}^{-}_j$~--- множество индексов узлов, из которых исходят дуги, ведущие 
в~$j$-й узел, $\mathcal{N}^{-}_j \hm\subset \mathcal{N} \bigcup \{0\} $;
$\mathcal{N}^{+}_j$~--- множество индексов узлов, в~которые ведут дуги,
 исходящие из $j$-го узла, $\mathcal{N}^{+}_j \hm\subset \mathcal{N} \bigcup \{N+1\}$.
Вектор потоков
$x\hm= \langle x_{0j},\ldots, x_{ij},\ldots, x_{i(N+1)}\rangle$ по дугам 
$l_{ij}\hm\in {\mathcal{L}}$, где

\noindent
\begin{multline}
i \in \mathcal{N} \bigcup \{0\}, \ 
j \in \mathcal{N} \bigcup \{N+1\}, \ 
i \neq j, \ \  
l_{ij} \in {\mathcal{L}},\\
  \mbox{ и~если} \ 
i  = 0, \ \mbox{то} \ j \not = N+1\,,
\label{e1-naz}
\end{multline}
должен удовлетворять
условию сохранения потока в~транзитных узлах
\begin{equation}
\sum\limits_{i \in \mathcal{N}^{-}_j}^{}{x_{ij}}= 
\sum\limits_{i \in \mathcal{N}^{+}_j}^{}{x_{ji}},  \enskip
 j \in \mathcal{N}\,,
 \label{e2-naz}
 \end{equation}
и ограничению на пропускную способность соответствующих дуг
\begin{equation}
0 \le x_{ij} \le d_{ij}, \ 
l_{ij}\in {\mathcal{L}}, \ \mbox{для $i, j$ выполняется~(1)}.
\label{e3-naz}
\end{equation}
Множество допустимых потоков в~сети
$ \mathcal{X} \hm= \{x \ | \mbox{ выполняются~(1)--(3)}  \}$.


При анализе функциональных возможностей системы будем 
рассматривать потоки по стоковым дугам.
Перенумеруем стоковые дуги по некоторому правилу натуральными числами от~1 до~$K$, 
т.\,е. установим взаимно однозначное  соответствие $  l_{k} \hm= l _{j (N+1)}$, 
$k \hm= \overline{1, K}$, $j \hm\in \mathcal{N}_\SK$.
Пусть $\overline {x}_{k}$~--- величина потока по стоковой дуге~$l_{k}$,
$\overline {x}_{k} \hm= x _{j (N+1)}$, $x _{j (N+1)}\hm\geq 0$, 
$k \hm= \overline{1, K}$, $j \hm\in \mathcal{N}_\SK. $
Таким образом, вектор
$ \overline {x} \hm= \langle \overline {x}_{1}, \ldots, 
\overline {x}_{k}, \ldots, \overline {x}_{K}\rangle$, $k \hm= \overline{1, K},$
покомпонентно определяет величины потока, которые передаются по каждой 
стоковой дуге сети  в~соответствии с~некоторым допустимым потоком 
$x \hm\in \mathcal{X}$.
Обозначим через~$\overline{\mathcal{X}}$ множество всех допустимых 
векторов~$\overline{x}$:
\begin{multline}
\overline{\mathcal{X}} = \{ \overline {x} \ | \ \overline{x}_{k} = 
x _{j (N+1)}, \\
 k = \overline{1, K}, \ j \in \mathcal{N}_\SK, \ 
\ x \in \mathcal{X} \}.  \label{e4-naz}
\end{multline}

\vspace*{-4pt}

\section{Предельные функциональные характеристики}

\vspace*{-2pt}

Рассмотрим множество $\overline{\mathcal{X}}$ век\-то\-ров-по\-то\-ков 
по стоковым дугам~(\ref{e4-naz}).
Каждый допустимый вектор $\overline {x} \hm\in \overline{\mathcal{X}}$
 характеризует возможности системы  по передаче потоков, 
 покидающих сеть по всем стоковым дугам одновременно.
Множество~$\overline{\mathcal{X}}$ является выпуклым и~многогранным. 
Точки, лежащие на границе множества~$\overline{\mathcal{X}}$, будем называть 
предельными функциональными характеристиками системы (ПФ-ха\-рак\-те\-ри\-сти\-ками).

\smallskip

\noindent
\textbf{Определение~1.}\ 
Монопольным режимом  передачи потока из фиктивного уз\-ла-источ\-ни\-ка~$v_0$ 
в~фиктивный сток~$v_{N+1}$ по  стоковой дуге~$l_{a}$ будем называть такой 
режим функционирования системы, при котором потоки по всем остальным 
стоковым дугам полагаются равными нулю.

\smallskip

Для анализа множества~$\overline{\mathcal{X}}$ рассмотрим некоторую фиксированную 
стоковую дугу~$l_{a}$, по которой поток~$\overline {x}_a$ из вершины 
$v_a \hm\in \mathcal{V}_{\SK}$ покидает систему. Вычислим максимальную 
величину потока~$\overline {x}_a$ при работе системы в~монопольном режиме, т.\,е.\
 решим следующую задачу о~максимальном потоке.
 
 \smallskip

\noindent
\textbf{Задача~1.}\ Для выделенной стоковой вершины $v_a \hm\in \mathcal{V}_{\SK}$ 
найти
$$
 \overline {x}_a^* = \max\limits_{\overline {x} \in \overline{\mathcal{X}}} \overline {x}_a 
 $$
при дополнительном условии 
$\overline {x}_{i} \hm= 0$ {для всех} $i \not= a$,  $i \hm\in \mathcal{N}_{\SK}$.

\smallskip

Оптимальное решение задачи~1~--- максимально допустимый поток~$\overline {x}_a^*$ 
по дуге~$l_a$ из стоковой вершины 
$v_a \hm\in \mathcal{V}_{\SK}$ при передаче в~монопольном режиме  в~вершину~$v_{N+1}$.
Значение~$\overline {x}_a^*$ для стоковой вершины~$v_a$ будем называть максимальным 
потоком, передаваемым в~монопольном режиме, или МРМ-по\-то\-ком.
Вектор $\left\langle 0, 0, \ldots, \overline {x}_a^*, 0, 0, \ldots, 0\right\rangle $ 
является угловой точкой множества~$\overline{\mathcal{X}}$ и~лежит на соответствующей 
координатной оси. Величина МРМ-по\-то\-ка~$\overline {x}_a^*$ определяет 
значение $a$-й ПФ-ха\-рак\-те\-ри\-сти\-ки сети.

Последовательно решим задачу~1 и~вычислим МРМ-потоки  для каждой стоковой дуги~$l_k$,  
$k \hm = \overline{1, K}$.
С~по\-мощью полученных решений сформируем вектор МРМ-по\-то\-ков по стоковым дугам
${\overline{{x}}^*} \hm= \langle \overline {x}_{1}^*,\ldots, \overline {x}_{k}^*, 
\ldots, \overline {x}_{K}^* \rangle$,
значение  $k$-й компоненты которого равно величине потока по стоковой дуге~$l_k$ 
в~фиктивный узел-сток~$v_{N+1}$ при монопольном режиме, т.\,е.\
 без учета потоков по другим стоковым дугам. Для вычисления предельных 
 значений потоков при одновременной передаче их в~сети решим сле\-ду\-ющую задачу.
 
 \smallskip
 
 \noindent
 \textbf{Задача~2.}\ Найти
$$
\alpha^{*} = \max_{\alpha, \  \overline{{x}} \in \overline{\mathcal{X} }}\alpha
$$
при условиях 
$ \alpha \overline {x}_k^{*} \hm\le  \overline {x}_k$, 
$k \hm= \overline {1, K}$, $\alpha \hm\geq 0.$ 

Вектор потоков, отвечающий оптимальному решению  $\alpha^{*}\hm>0$ задачи~2, обозначим 
через~${f}^*$,
${f}^* \hm= \alpha^* \overline {x}^* \hm=  \langle \alpha^*\overline {x}_{1}^*,\ldots, 
\alpha^*\overline {x}_{j}^*,\ldots, \alpha^*\overline {x}_{K}^* \rangle.$

Точка с~координатами $(\alpha^*\overline {x}_{1}^*,\ldots, 
\alpha^*\overline {x}_{j}^*,\ldots$\linebreak $\ldots,  
\alpha^*\overline {x}_{K}^*)$ лежит на слейтеровской~\cite{germ, Pod} грани 
множества~$\overline{\mathcal{X}}$ по критерию максимизации вектора потоков 
по всем стоковым дугам.
В~рамках формализма модели можно утверждать, что вектор~${f}^*$ 
определяет передачу в~равных долях максимальных потоков~$\overline{x}_{j}^*$, 
достижимых только при монопольных режимах управления.  Такое равнодолевое 
распределение обеспечивает допустимый поток при одновременной передаче 
из фиктивного источника бесконечной мощности~$v_{\SS}$ по всем стоковым дугам 
в~фиктивный сток бесконечной емкости~$v_{\SK}$.
Поскольку хотя бы для одной стоковой дуги компонента вектора~${f}^*$ 
задает ее предельные функциональные возможности при одновременной пе\-редаче 
потока в~условиях равнодолевого распределения, далее вектор~${f}^*$ 
будем называть ПФР-ха\-рак\-те\-ри\-сти\-кой многополюсной потоковой сетевой\linebreak 
сис\-те\-мы (предельной функциональной равнодолевой характеристикой). 

\vspace*{-4pt}

\section{Критически опасные повреждения сети}

\vspace*{-2pt}

Пусть повреждение сети задается подмножеством~$w$ дуг исходной сети, 
пропускная способность которых становится равной нулю. Предполагается, 
что по\-вреж\-де\-ны могут быть любые дуги сети, кроме стоковых. Положим $\mathcal{L}' \hm= 
\overline{\mathcal{L}} \hm\bigcup \hat{\mathcal{L}}$, тогда
$w \hm\subseteq \mathcal{L}' $.
Обозначим  через~$d_{ij}(w)$ пропускную способность дуги~$l_{ij}$ по\-вреж\-ден\-ной сети 
и~положим
\begin{equation}
 d_{ij}(w) =
\begin{cases}
0 , & \mbox{если }  l_{ij} \in w\,;  \\
d_{ij}, & \mbox{если }  l_{ij} \in \mathcal{L}  \bs w\,.
\end{cases}
\label{e5-naz}
\end{equation}
Для допустимого после по\-вреж\-де\-ния сети потока
${x}(w)$, где распределение потоков по дугам описывается компонентами  вектора
${x} (w)\hm=\langle x_{0j}(w),\ \ldots$\linebreak $\ldots , x_{ij}(w)$,\ \ldots, $x_{i(N+1)}(w)\rangle, $
должно выполняться~(\ref{e1-naz}). 
Для любых допустимых потоков~${x}(w)$ в~по\-вреж\-ден\-ной сети выполняются 
стандартные ограничения
\begin{multline}
 0 \le x_{ij}(w) \le d_{ij}(w),\\
   \mbox{ где }
 d_{ij}(w) \mbox{ определяется~(5)}, \ \ 
 l_{ij}\in {\mathcal{L}}, 
 \label{e6-naz}
 \end{multline}
и закон сохранения потока в~транзитных узлах
\begin{equation}
 \sum\limits_{i \in \mathcal{N}^{-}_j}{x_{ij}(w)}= 
 \sum\limits_{i \in \mathcal{N}^{+}_j}{x_{ji}(w)},  \enskip
   j \in \mathcal{N}\,.
   \label{e7-naz}
   \end{equation}
Множество $\mathcal{X}(w)$  допустимых потоков  в~по\-вреж\-ден\-ной сети  
определяется условиями~(\ref{e1-naz}), (\ref{e6-naz}) и~(\ref{e7-naz}):
$\mathcal{X}(w) \hm= \{{x}(w) \ | \mbox{ выполняется (1), (6), (7)} \}.$

Пусть $\overline {x}_k(w)$~--- величина потока по $k$-й стоковой дуге после 
разрушающего воздействия, тогда
$\overline {x}(w)$~--- вектор величин потоков по всем стоковым дугам,
$\overline {{x}}(w) \hm= \langle \overline {x}_1(w), \ldots, \overline {x}_k(w),\ldots,  
\overline {x}_K(w)\rangle.$
Аналогично~(\ref{e4-naz}) множество
\begin{multline*}
 \overline{\mathcal{X}}(w)= \{ \overline {{x}}(w) \ | \ \overline {x}_{k}(w) 
 = x _{j (N+1)}(w)\,, \\
  k = \overline{1, K}\,,  \ 
 j \in \mathcal{N}_\SK\,, \ \ x(w) \in \mathcal{X} (w)\} 
 \end{multline*}
описывает достижимые потоки по стоковым дугам после по\-вреж\-дения.

\smallskip

\noindent
\textbf{Определение~2.}\
Критически опасным структурным по\-вреж\-де\-ни\-ем (КС-по\-вреж\-де\-ни\-ем)~$w$ 
будем называть разрушающее воздействие на дуги сети, при котором поток хотя 
бы к~одной стоковой вершине оказывается равным нулю. Формально в~случае 
КС-по\-вреж\-де\-ния сети~$w$ у~вектора потоков~$\overline {x}(w)$ 
всегда найдется компонента, равная нулю, т.\,е.
$$
\forall \ \overline {{x}}(w) \in \overline{\mathcal{X}}(w)\ \  \exists\ k:\ \
 \overline {x}_{k}(w) = 0.
 $$
 
 \smallskip

Рассмотрим влияние КС-по\-вреж\-де\-ний на ПФ- и~ПФР-ха\-рак\-те\-ри\-сти\-ки многополюсной 
потоковой сетевой системы.
Согласно классической тео\-рии\linebreak
 о~передаче однопродуктового  потока~\cite{ford} 
максимальному потоку в~сети соответствует минимальный разрез, 
пропускная способность которого\linebreak
 равна величине максимального потока. 
При разрушении дуг, определяющих минимальный разрез, передача потока в~сети 
становится невозможной. Таким образом,  по определению~2, все разрезы, 
соответствующие МРМ-по\-то\-кам, образуются КС-по\-вреж\-де\-ни\-ями.
Действительно, для того чтобы применить~\cite{ford} в~задаче~1, формально 
положим пропускную способность всех стоковых дуг, кроме~$l_a$, равной нулю и~решим 
задачу на максимум потока без дополнительного условия, имевшегося в~задаче~1. 
Получим МРМ-по\-ток для дуги~$l_a$.
Для МРМ-по\-то\-ка~$\overline {x}_a^*$ в~этой задаче уже найдем со\-от\-вет\-ст\-ву\-ющее  
минимальному разрезу (далее~--- МРМ-раз\-рез) подмножество $\mathcal{R}(a)$
дуг $l_{ij} \hm\in \mathcal{L}'$, суммарная пропускная способность которых 
равна~$\overline{x}_a^*$, и~такое, что в~случае по\-вреж\-де\-ния~$\mathcal{R}(a)$ 
при монопольном для~$l_a$ режиме функционирования передача потока из единственного 
источника~$v_0$ в~единственный узел-сток~$v_{N+1}$ через стоковую дугу~$l_a$ 
окажется невозможна, т.\,е.
$$
 \sum\limits_{l_{ij} \in \mathcal{R}(a)} d_{ij} = \overline {x}_a^*, \quad    
 \overline {x}_a = 0  \ \ \forall    
  \ {{x}}(\mathcal{R}(a)) \in \mathcal{X}(\mathcal{R}(a))\,.
  $$

Далее для сохранения простоты изложения будем считать, что 
каждому МРМ-по\-то\-ку соответствует единственный МРМ-раз\-рез и~единственное 
КС-по\-вреж\-де\-ние~$\mathcal{R}(a)$.
Все такие по\-вреж\-де\-ния объединим в~одно множество, т.\,е\
 введем  $\overline{\mathcal{R}} \hm= \{ \mathcal{R}(a)\ | \ a \hm\in 
 \mathcal{N}_{\SK}\}.$
Если МРМ-по\-то\-ку~$\overline {x}_a^*$ соответствует несколько КС-по\-вреж\-де\-ний, 
то любые (все) из них по\-мес\-тим в~$\overline{\mathcal{R}}$. Повреждения, за\-да\-ва\-емые 
МРМ-раз\-ре\-за\-ми, будем называть разрезными. Каждое  
$R(a) \in\overline{\mathcal{R}} $ является оптимальным в~игровой 
задаче за условного противника, разрушающего сеть по критерию отделения 
одной из стоковых вершин ($v_a$) с~минимальными затратами на 
уничтожение пропускной способности дуг. Когда же выбираемая 
<<разрушителем>> стоковая вершина не известна заранее (как в~модели <<оборона 
против нападения>>~\cite{ germ}), можно ожидать любого по\-вреж\-де\-ния 
из~$\overline{\mathcal{R}}$. Поэтому из большого набора КС-по\-вреж\-де\-ний 
для анализа уязвимости сети выберем именно разрезные и~получение оценок 
уязвимости проведем с~помощью множества~$\overline{\mathcal{R}}$.

Обозначим через $R$ общее число разрезных КС-по\-вреж\-де\-ний 
в~множестве~$\overline{\mathcal{R}}$.
Для каждой стоковой дуги~$l_k$ вычислим долю случаев~$\rho_k$, когда при 
по\-вреж\-де\-нии того или иного множества дуг~$\mathcal{R}(a)$, 
$a \hm\in \mathcal{N}_{\SK}$, поток по выбранной стоковой дуге~$l_k$ 
оказывается равным нулю. Для этого число разрезных по\-вреж\-де\-ний~$m(l_k)$, 
<<об\-ну\-ля\-ющих>> поток по выбранной стоковой дуге~$l_k$, разделим на общее чис\-ло~$R$ 
по\-вреж\-де\-ний в~множестве~$\overline{\mathcal{R}}$, т.\,е.\ положим
$$
 \rho_k = \fr{m(l_k)}{R}\,.
 $$

Передача потока через стоковые дуги~$l_k$, для которых величина~$\rho_k$ 
больше чем у~других, представляется более рискованной, так как пути в~соответствующие 
стоковые вершины~$v_k$ чаще проходят в~зоне критических по\-вреж\-де\-ний сети. 
Следовательно, можно считать, что пользователи, стоящие за подобными 
стоковыми вершинами~$v_k$, наиболее уязвимы в~случае разрезных по\-вреж\-де\-ний. 
Получили просто вычисляемый показатель относительной уязвимости стоковых 
вершин многополюсной сети. 

\vspace*{-4pt}

\section{Эффективные повреждения сети}

\vspace*{-2pt}

Теперь предложим способ ранжирования разрезных КС-по\-вреж\-де\-ний 
из~$\overline{\mathcal{R}}$. Обозначим разрушение всех дуг 
некоторого $\mathcal{R}(a)\hm\in\overline{\mathcal{R}}$ через~$w_a$. 
Напомним, что в~этом случае поток по стоковой дуге~$l_a$ оказывается равным нулю.
Для оценки изменений относительно исходной ПФР-ха\-рак\-те\-ри\-сти\-ки~$ f^{*}$ 
решим на сети после по\-вреж\-де\-ния $w_a\hm= \mathcal{R}(a)$ сле\-ду\-ющую задачу.

\smallskip

\noindent
\textbf{Задача~3.}\ Найти
$$
 \beta^{*}(w_a) = \min\limits_{{x}(w_a) \in {\mathcal{X}(w_a)}} 
 \sum\limits_{k = \overline{1, K}} \left( \fr{f_k^* - \overline{x}_k(w_a)}
 {f_k^*} \right)^2. 
 $$

Вектор, отвечающий решению задачи~3, обозначим через~${x}^*(w_a)$. 
Вектору~${x}^*(w_a)$ соответствует вектор потоков по стоковым 
дугам~$\overline{{x}}^*(w_a)$ при по\-вреж\-де\-нии~$w_a$: 
$ \overline{{x}}^*(w_a)\hm =  \left\langle \overline{x}_1^*(w_a), 
\overline{x}_2^*(w_a), \ldots, \overline{x}_K^*(w_a)  \right\rangle.$
Компоненты вектора~$\overline{{x}}^*(w_a)$ определяют максимальные (но не 
выше исходных значений у~ПФР-ха\-рак\-те\-ри\-сти\-ки) 
потоки по стоковым дугам при условии, что пропускная способность дуг из 
выбранного множества~$\mathcal{R}(a)$ равна нулю. Заметим, что $a$-я компонента 
вектора~$\overline{{x}}^*(w_a)$ гарантированно равна нулю.




Для решения $\overline{{x}}^*(w_a)$ задачи~3 введем множество номеров 
стоковых дуг, величина потока по которым равна нулю,
$\mathcal{K}^-(w_a) \hm= \{ k = \overline{1, K} \  | \ \overline{x}_k^*(w_a) \hm= 0\}$, 
и~обозначим через~$\mathcal{K}^+(w_a)$ множество номеров стоковых дуг, 
величина потока по которым больше нуля,
$\mathcal{K}^+(w_a) \hm= \{ k = \overline{1, K}\  | \ \overline{x}_k^*(w_a) \hm> 0\} $.
При этом
$ \mathcal{K}^-(w_a) \hm\bigcup \mathcal{K}^+(w_a)\hm = \hat{\mathcal{L}}$.
Вычислим долю стоковых дуг с~нулевым потоком и~обозначим ее
$$
\kappa(w_a) \hm= \fr{|\mathcal{K}^-(w_a)|}{K}\,.
$$
Найдем  значение суммарных относительных потерь для сохранившихся потоков 
по стоковым дугам:
$$
\Delta (w_a)= \sum\limits_{k\in\mathcal{K}^+(w_a)}
\fr{f_k^* - \overline {x}_k^*(w_a)}{f_k^*}\,.  
$$

Процедуру построения векторов~$\overline{{x}}^*(w_k)$ и~определения
величин~$\kappa(w_k)$ 
и~$\Delta (w_k)$ проведем для каж\-до\-го из КС-по\-вреж\-де\-ний~$\mathcal{R}(k)$, 
$k \hm\in \mathcal{N}_\SK$, входящих в~$\overline{\mathcal{R}}$. Выберем 
из~$\overline{\mathcal{R}}$ подмножество~$\overline{\mathcal{R}}^*$ 
разрезных по\-вреж\-де\-ний, недоминируемых сразу по двум показателям $(\kappa, \Delta)$:
\begin{multline*}
 \overline{\mathcal{R}}^* = \{ 
\mathcal{R}(k)\in\overline{\mathcal{R}} \ | \
\forall\ a \in \mathcal{N}_{\SK}: \ \ \mbox{если} \ \kappa(w_a) > \kappa(w_k),\\ 
\ \mbox{то} \  \Delta(w_k) > \Delta(w_a)  \}.
\end{multline*}
Критически опасные структурные по\-вреж\-де\-ния из~$\overline{\mathcal{R}}^*$ будем называть эффективными 
(с~точки зрения  разрушителя сети~--- 
по векторному критерию~$(\kappa, \Delta)$ на максимум). Они задают конфигурации 
дуг сети, образующие ее <<слабые мес\-та>>, защита которых требуется в~первую очередь. 
Остальные элементы~$\overline{\mathcal{R}}$~--- доминируемые по\-вреж\-де\-ния, 
но тоже могут быть полезны для целей аналитики.

Опираясь на полученные результаты, определим, полное разрушение 
каких дуг из КС-по\-вреж\-де\-ний, входящих в~множество~$ \overline{\mathcal{R}}$, 
оказывает  наибольшее влияние на уязвимость сети.
Для этого\linebreak построим таблицу, организованную следующим образом. 
В~первый стролбец таблицы внесем дуги сети~$l_{ij}$, которые могут быть 
подвержены разрушению, во второй~--- общее число разрезных по\-вреж\-де\-ний~$n(l_{ij})$, 
в~которые входит дуга~$l_{ij}$, в~третий~---\linebreak\vspace*{-12pt}

%\begin{table*}[b]
{\small
\vspace*{12pt}

\begin{center}


\begin{tabular}{|c|c|c|}
\multicolumn{3}{p{66mm}}{Формирование разрезных и~эффективных пов\-реж\-де\-ний}\\
\multicolumn{3}{c}{\ }\\[-6pt]
 \hline
\multicolumn{1}{|c|}{Дуги сети} &  
\tabcolsep=0pt\begin{tabular}{c}Разрезные\\ повреждения\end{tabular} &
      \tabcolsep=0pt\begin{tabular}{c}     Эффективные\\ повреждения\end{tabular}\\
\hline
$l_{0j}$& $n(l_{0j})$&   $n^*(l_{0j})$\\
$\vdots$&      $\vdots$&      $\vdots$\\
$l_{ij}$& $n(l_{ij})$ & $n^*(l_{ij})$\\
$\vdots$&      $\vdots$&      $\vdots$\\
$l_{iN}$ & $n(l_{iN})$ & $n^*(l_{iN})$ \\  
      \hline
\end{tabular}
\vspace*{3pt}
\end{center}}
%\end{table*}

\noindent 
общее число эффективных разрезных по\-вреж\-де\-ний~$n^*(l_{ij})$ с~дугой~$l_{ij}$.



Чем больше значение~$ n(l_{ij})$, тем важнее сохранить дугу~$l_{ij}$ от разрушения,
 поскольку удаление~$l_{ij}$ снижает пропускную способность минимального разреза 
 (и~МРМ-по\-то\-ки) для большего числа стоковых вершин. Заметим, что 
 ПФР-ха\-рак\-те\-ри\-сти\-ки КС-по\-вреж\-де\-ний для указанных вершин не обязательно 
 хуже, чем при удалении иной дуги. Для дуг с~большими значениями как~$n (l_{ij})$, 
 так и~$n^*(l_{ij})$ их
  влияние на уязвимость сети по отношению к~разрезным 
 по\-вреж\-де\-ни\-ям, по-ви\-ди\-мо\-му, еще существенней. 

\vspace*{-4pt}

\section{Заключение}

\vspace*{-2pt}

Сценарии повреждения дуг, соответствующих минимальным разрезам, не являются 
единственно возможными. При исследовании сетевых систем приходится обращаться к~теории 
графов и~методам комбинаторной оптимизации~\cite{Sigal, Harari}. Проблема 
состоит в~том, что попытки решения многих практических задач требуют применения 
ресурсоемких переборных методов анализа структуры сети с~последующим специальным 
изучением функциональных потоковых  характеристик~\cite{Mur13, ros}. 

Предложенная 
в~настоящей работе схема выделяет на дереве перебора содержательно важные 
точки ветвления~--- конфигурации дуг сети, образующие МРМ-раз\-ре\-зы. 
Подобным образом опытный шахматист или хорошая шахматная программа заранее 
тщательно рассматривает некоторые стандартные часто встречающиеся позиции, 
чтобы сэкономить время в~реальной партии и/или лучше понять внутреннюю логику 
возможных последующих ходов в~игре. 

Разработанный подход опирается на мето\-до\-логию 
исследования операций~\cite{germ}. Определение\linebreak допустимых потоков через все 
стоковые дуги позволяет анализировать сложные взаимосвязи, свойственные большим 
территориально распределенным сис\-те\-мам (см., например,~\cite{Koz17}). 
Изучение разрезных по\-вреж\-де\-ний помогает выявить наиболее уязвимые стоковые вершины. 
Следует подчеркнуть, что методы потокового программирования~\cite{ford}, положенные 
в~основу анализа, дают способ быст\-ро и~эффективно проводить необходимые 
вычисления и~получать значимые  результаты. 

\vspace*{-4pt}

{\small\frenchspacing
 {%\baselineskip=10.8pt
 \addcontentsline{toc}{section}{References}
 \begin{thebibliography}{99}
 
\vspace*{-2pt}
    
\bibitem{MalInf17} 
\Au{Малашенко Ю.\,Е., Назарова И.\,А., Новикова~Н.\,М.} 
Метод анализа функциональной уязвимости потоковых сетевых систем~// 
Информатика и~её применения, 2017. Т.~11. Вып.~4. С.~47--54.
\pagebreak

\bibitem{MalInf181} 
\Au{Малашенко Ю.\,Е., Назарова~И.\,А., Новикова~Н.\,М.} 
Диаграммы уязвимости потоковых сетевых систем~// Информатика и~её применения, 2018. 
Т.~12. Вып.~1. С.~11--18.

\bibitem{ford} 
\Au{Форд Л., Фалкерсон~Д.} Потоки в~сетях~/ Пер. с~англ.~--- 
М.: Мир, 1966. 277~с. (\Au{Ford~L.\,R.,   Fulkerson~D.\,R.} Flows in networks.~--- 
Princeton, NJ, USA: Princeton University Press, 1962. 332~p.)

\bibitem{germ}  
\Au{Гермейер Ю.\,Б.} Введение в~теорию исследования операций.~--- 
М.: Наука, 1971. 384~с.

\bibitem{Pod} %5
\Au{Подиновский В.\,В., Ногин~В.\,Д.} 
Па\-ре\-то-оп\-ти\-маль\-ные решения многокритериальных задач.~--- 2-е изд.~--- 
М.: Физматлит, 2007.  256~с.

\bibitem{Harari}  %6
\Au{Харари Ф.} Теория графов~/ Пер. с~англ.~--- М.: Мир, 1973. 302~с. 
(\Au{Harari~F.} Graph theory.~--- Boston, MA, USA: Addison-Wesley, 1969. 300~p.)
\columnbreak

\bibitem{Sigal}  %7
\Au{Сигал И.\,Х., Иванова~А.\,П.} Введение в~прикладное дискретное программирование.~--- 
М.: Физматлит, 2002. 240~с.

\bibitem{ros}   %8
\Au{ Rosas-Casals M., Valverde S., Sole~R.\,V.}
 Topological vulnerability of the European power grid under errors and attacks~// 
 Int. J.~Bifurcat. Chaos, 2007. Vol.~17. Iss.~7. 
 Р.~2465-2475.

\bibitem{Mur13}  %9
\Au{Murray A.\,T. } An overview of network vulnerability modeling approaches~// 
GeoJournal, 2013. Vol.~78. P.~209--221.

\bibitem{Koz17} 
\Au{Козлов М.\,В., Малашенко Ю.\,Е., Назарова~И.\,А. и~др.} 
Управление   топ\-лив\-но-энер\-ге\-ти\-че\-ской  сис\-те\-мой  
при  круп\-но\-масш\-таб\-ных по\-вреж\-де\-ни\-ях. I.~Сетевая  модель  и~программная реализация~// 
Изв. РАН. ТиСУ, 2017. №\,6. С.~50--73.

 \end{thebibliography}

 }
 }

\end{multicols}

\vspace*{-5pt}

\hfill{\small\textit{Поступила в~редакцию 28.06.18}}

\vspace*{6pt}

%\newpage

%\vspace*{-24pt}

\hrule

\vspace*{2pt}

\hrule

\vspace*{-6pt}


\def\tit{ANALYSIS OF CUTTING DAMAGES TO~MULTIPOLAR NETWORKS\\[-5pt]}

\def\titkol{Analysis of cutting damages to multipolar networks}

\def\aut{Yu.\,E.~Malashenko, I.\,A.~Nazarova, and~N.\,M.~Novikova\\[-5pt]}

\def\autkol{Yu.\,E.~Malashenko, I.\,A.~Nazarova, and~N.\,M.~Novikova}

\titel{\tit}{\aut}{\autkol}{\titkol}

\vspace*{-22pt}


\noindent
Dorodnicyn Computing Center, Federal Research Center ``Computer Science and Control'' 
of the Russian Academy of Sciences, 40~Vavilov Str., 
Moscow 119333, Russian Federation


\def\leftfootline{\small{\textbf{\thepage}
\hfill INFORMATIKA I EE PRIMENENIYA~--- INFORMATICS AND
APPLICATIONS\ \ \ 2018\ \ \ volume~12\ \ \ issue\ 3}
}%
 \def\rightfootline{\small{INFORMATIKA I EE PRIMENENIYA~---
INFORMATICS AND APPLICATIONS\ \ \ 2018\ \ \ volume~12\ \ \ issue\ 3
\hfill \textbf{\thepage}}}

\vspace*{2pt}




\Abste{The method of estimating changes in the functional 
capabilities of a~multipolar  flow network system after a~damage is proposed.  
For each sink arc, the maximal flow is calculated, independent of the flow 
value across the remaining sink arcs. The authors consider cutting structural 
damages that correspond to removing all arcs forming a~minimal cut. 
The capacity of the cut is equal to the maximal flow along some sink arc. 
Among the structural damages, the critically dangerous ones are selected with 
an introduced criterion. For each arc belonging to at least one cutting 
structural damage, a~quantitative characteristic is computed to estimate 
consequences of its destruction. The described approach is proposed to be used 
in studying vulnerability of  territorially distributed  multiuser systems 
with the network 
structure in the case of a~single-product transfer.} 

\KWE{single-product flow network; structural 
vulnerability of network; multipolar flow model }

\DOI{10.14357/19922264180305}

%\vspace*{-14pt}

%\Ack
%\noindent



%\vspace*{6pt}

  \begin{multicols}{2}

\renewcommand{\bibname}{\protect\rmfamily References}
%\renewcommand{\bibname}{\large\protect\rm References}

{\small\frenchspacing
 {%\baselineskip=10.8pt
 \addcontentsline{toc}{section}{References}
 \begin{thebibliography}{99}

\bibitem{1-n}
\Au{Malashenko, Yu.\,E., I.\,A.~Nazarova, and N.\,M.~Novikova.} 
2017. Metod analiza funktsional'noy uyazvimosti potokovykh setevykh system 
[Method of the analysis of the functional vulnerability of flow network systems]. 
\textit{Informatika i~ee Primeneniya~--- Inform. Appl.} 11(4):50--73.

\bibitem{2-n}
\Aue{Malashenko, Yu.\,E., I.\,A.~Nazarova, and N.\,M.~Novikova.} 2018. 
Diagrammy uyazvimosti potokovykh setevykh sistem [Diagram of the functional
 vulnerability of flow network systems]. 
 \textit{Informatika i~ee Primeneniya~--- Inform. Appl.} 12(1):11--18. 

\bibitem{3-n}
\Aue{Ford, L.\,R., and   D.\,R.~Fulkerson.} 1962. 
\textit{Flows in networks.}  Princeton, NJ: Princeton University Press. 332~p. 

\bibitem{4-n}
\Aue{Germeier, Yu.\,B.} 1971. \textit{Vvedenie v~teoriyu issledovaniya operatsiy} 
[An introduction to operations research theory]. Moscow: Nauka. 384~p. 

\bibitem{5-n}
\Aue{Podinovskij, V.\,V., and V.\,D.~Nogin.}  2007. 
\textit{Pareto-optimal'nye resheniya mnogokriterial'nykh zadach} 
[Pareto-optimal solutions of multicriteria tasks]. 2nd ed.  
Moscow: Fizmatlit. 256~p. 

 


\bibitem{7-n} %6
\Aue{Harari, F.} 1969.
\textit{Graph theory.}  Boston, MA: Addison-Wesley Publ. 300~p.

\bibitem{6-n} %7
\Aue{Sigal, I.\,H., and  A.\,P.~Ivanova.} 2002. 
\textit{Vvedenie v~prikladnoe diskretnoe programmirovanie}
[An introduction to applied discrete programming]. Moscow: Fizmatlit. 240~p. 

\bibitem{9-n} %8
\Aue{Rosas-Casals, M., S.~Valverde, and R.\,V.~Sole.}  2007. 
Topological vulnerability of the European power grid under errors and attacks. 
\textit{Int. J.~Bifurcat.  Chaos} 17(7):2465--2475. 

\bibitem{8-n} %9
\Aue{Murray, A.\,T.} 2013. An overview of network vulnerability modeling approaches. 
\textit{GeoJournal} 78:209--221.

\bibitem{10-n}
\Aue{Kozlov, M.\,V.,  Yu.\,E.~Malashenko, I.\,A.~Nazarova, \textit{et al.}} 
2017. Fuel and energy system control at large-scale damages. 
I.~Network model and software implementation.  
\textit{J.~Comput.  Sys. Sc. Int.}  56(6):945--968.    
\end{thebibliography}

 }
 }

\end{multicols}

\vspace*{-9pt}

\hfill{\small\textit{Received June 28, 2018}}

\pagebreak

%\vspace*{-18pt}

\Contr


\noindent
\textbf{Malashenko Yuri E.} (b.\ 1946)~--- 
Doctor of Science in physics and mathematics, principal scienist, Dorodnicyn 
Computing Center, Federal Research Center ``Computer Science and Control'' 
of the Russian Academy of Sciences, 40~Vavilov Str., 
Moscow 119333, Russian Federation; \mbox{malash09@ccas.ru}

\vspace*{5pt}

\noindent
\textbf{Nazarova Irina A.} (b.\ 1966)~--- 
Candidate of Science (PhD) in physics and mathematics, scientist, 
Dorodnicyn Computing Center, Federal Research Center 
``Computer Science and Control'' of the Russian Academy of Sciences, 
40~Vavilov Str., Moscow 119333, Russian Federation; \mbox{irina-nazar@yandex.ru}

\vspace*{5pt}

\noindent
\textbf{Novikova Natalya M.} (b.\ 1953)~--- 
Doctor of Science in physics and mathematics,  professor,  leading scientist, 
Dorodnicyn Computing Center, Federal Research Center 
``Computer Science and Control'' of the Russian Academy of Sciences, 
40~Vavilov Str., Moscow 119333, Russian Federation; \mbox{n\_novikova@umail.ru}
\label{end\stat}

\renewcommand{\bibname}{\protect\rm Литература}       