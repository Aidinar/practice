\def\stat{gorbunova}

\def\tit{РЕСУРСНЫЕ СИСТЕМЫ МАССОВОГО ОБСЛУЖИВАНИЯ КАК~МОДЕЛИ БЕСПРОВОДНЫХ СИСТЕМ 
СВЯЗИ$^*$}

\def\titkol{Ресурсные системы массового обслуживания как~модели беспроводных систем 
связи}

\def\aut{А.\,В.~Горбунова$^1$, В.\,А.~Наумов$^2$, Ю.\,В.~Гайдамака$^3$, К.\,Е.~Самуйлов$^4$}

\def\autkol{А.\,В.~Горбунова, В.\,А.~Наумов, Ю.\,В.~Гайдамака, К.\,Е.~Самуйлов}

\titel{\tit}{\aut}{\autkol}{\titkol}

\index{Горбунова А.\,В.}
\index{Наумов В.\,А.}
\index{Гайдамака Ю.\,В.}
\index{Самуйлов К.\,Е.}
\index{Gorbunova A.\,V.}
\index{Naumov V.\,A.}
\index{Gaidamaka Yu.\,V.}
\index{Samouylov K.\,E.}




{\renewcommand{\thefootnote}{\fnsymbol{footnote}} \footnotetext[1]
{Публикация подготовлена при финансовой поддержке Минобрнауки России 
(проект 2.882.2017/4.6).}}


\renewcommand{\thefootnote}{\arabic{footnote}}
\footnotetext[1]{Российский университета дружбы народов, 
\mbox{gorbunova\_av@rudn.university}}
\footnotetext[2]{Исследовательский институт инноваций, Хельсинки, 
Финляндия, \mbox{valeriy.naumov@pfu.fi}}
\footnotetext[3]{Российский университет дружбы народов; Институт 
проб\-лем информатики Федерального исследовательского центра <<Информатика 
и~управ\-ле\-ние>> Российской академии наук, \mbox{gaydamaka\_yuv@rudn.university}}
\footnotetext[4]{Российский университет дружбы народов; Институт 
проблем информатики Федерального исследовательского центра <<Информатика 
и~управ\-ле\-ние>> Российской академии наук, \mbox{samouylov\_ke@rudn.university}}

\vspace*{-5pt}



\Abst{Представлен обзор ресурсных систем массового обслуживания (СМО), используемых 
для моделирования широкого класса реальных систем, в~которых ресурсы являются 
заведомо ограниченными. Несмотря на объективную важность исследования подобных 
систем, работ, посвященных их анализу, до последнего времени существовало совсем 
немного, что было связано со сложностью построения случайного процесса, 
описывающего их функционирование, и,~соответственно, получения численных 
результатов. Однако за последние годы произошел существенный сдвиг в~изучении 
ресурсных систем, были предложены новые методы их анализа, позволяющие строить 
рекуррентные алгоритмы, пригодные для численных расчетов.
В~этой связи в~обзоре отражена только часть полученных результатов, а~именно:
рассмотрены ресурсные системы без мест для ожидания с~экспоненциальным временем 
обслуживания. Рассмотрены модели беспроводных систем связи, основанные на 
ресурсных СМО (РСМО), выражения для оценки основных 
ве\-ро\-ят\-но\-ст\-но-вре\-мен\-ных характеристик и~алгоритмы их вычисления.}


\KW{ресурсная система массового обслуживания; непрерывный 
ресурс; дискретный ресурс; ограниченный ресурс; рекуррентный алгоритм; 
гетерогенная сеть; стационарное распределение; полумарковский процесс; 
беспроводные системы связи}

\DOI{10.14357/19922264180307}
  
%\vspace*{4pt}


\vskip 10pt plus 9pt minus 6pt

\thispagestyle{headings}

\begin{multicols}{2}

\label{st\stat}

\section{Введение}

В классических СМО приборы и~места ожидания 
играют роль необходимых для обслуживания ресурсов. В~РСМО кроме 
приборов и~мест ожидания заявкам могут потребоваться различные дополнительные 
ресурсы. Это может быть некоторый случайный объем ресурса, занимаемого на время 
ожидания начала обслуживания, либо на время обслуживания, либо на все время 
пребывания заявки в~сис\-те\-ме. Если у~сис\-те\-мы нет достаточного числа свободных 
ресурсов, поступившая заявка теряется. В~дальнейшем будем использовать термин 
<<ресурс>> только для обозначения дополнительного ресурса, отличного от приборов 
или мест ожидания.

Интерес к~РСМО объясняется возможностью их применения для моделирования 
достаточно широкого спектра технических устройств 
и~в~целом ин\-фор\-ма\-ци\-он\-но-вы\-чис\-ли\-тель\-ных систем.
В~частности, если говорить о единственном типе ресурса ограниченного объема, то 
таким образом может моделироваться ограниченность памяти некоторого устройства 
или отдельной системы.
Таким образом, увеличивается реалистичность модели и,~соответственно, ее 
практическая цен\-ность.
%
Если же говорить о~множественных ресурсах, то стоит вспом\-нить услуги 
беспроводных сетей, таких, например, как Long Term Evolution (LTE)~\cite{Andrews}.
 Рост их популярности делает необходимым создание эффективных 
инструментов для оценки телекоммуникационными операторами работы 
радиоинтерфейсов~\cite{Galinina,Samuylov}. В~этих сетях каждая активная сессия 
занимает определенный объем радиоресурсов (например, ширину полосы пропускания 
спектра час\-тот, мощ\-ности передачи радиочастотного усилителя и~др.), которые 
являются заведомо ограниченными и~должны быть распределены при поступлении 
вызова пользователя и~освобождены по завершении сессии~\cite{Naumov_3_2016}.

Стоит отметить, что моделированию беспроводных систем связи с~по\-мощью СМО 
с~множественными ресурсами начиная с~\cite{Gimpelson} посвящено большое чис\-ло 
публикаций. Однако основной акцент в~них делается на анализ различных схем 
распределения ресурсов в~системах c детерминированными требованиями заявок 
к~ресурсам. Обзор этих работ можно найти в~\cite{Kelly,Ross,Basharin}.

Статья организована следующим образом: в~разд.~2 описываются основные типы 
РСМО без мест для ожидания, методы их исследования и~полученные результаты 
в~виде выражений для основных ве\-ро\-ят\-но\-ст\-но-вре\-мен\-н$\acute{\mbox{ы}}$х характеристик 
функционирования указанных сис\-тем. В~разд.~3 представлены подходы, позволяющие 
провести численные расчеты с~по\-мощью полученных соотношений. В~заключении кратко 
подведены итоги работы.

\section{Ресурсные системы массового обслуживания}

Более подробно остановимся на описании общей модели РСМО без мест для ожидания 
(рис.~1).
Система может располагать ограниченным или неограниченным объемом ресурсов как 
одного, так и~нескольких типов. Схему ее функционирования можно описать 
следующим образом:
\begin{enumerate}[(1)]
\item для обслуживания каждой заявки требуется один прибор и~некоторый объем 
ресурса каж\-до\-го типа;
\item поступившая заявка теряется, если в~момент поступления объем 
требуемого ей ресурса превышает объем свободного ресурса этого типа либо все 
приборы заняты;
\item в~момент начала обслуживания заявки суммарный объем занятого ресурса 
каждого типа увеличивается на величину ресурса, выделенного этой заявке;
\item в~момент окончания обслуживания заявки суммарный объем занятого 
ресурса каждого типа уменьшается на величину ресурса, выделенного этой заявке.
\end{enumerate}



В СМО может поступать один или несколько классов заявок, для которых
$A_l(t)$~--- функция распределения времени между поступлениями заявок класса~$l$,
$H_l(t,\mathbf{x})$~--- совместная функция распределения длительности 
обслуживания и~вектора объема ресурсов, необходимых поступившей заявке\linebreak\vspace*{-12pt}

{ \begin{center}  %fig1
 \vspace*{9pt}
  \mbox{%
 \epsfxsize=78.288mm 
 \epsfbox{gor-1.eps}
 }


\vspace*{6pt}


\noindent
{{\figurename~1}\ \ \small{Схема функционирования  РСМО общего вида}}
\end{center}
}

%\vspace*{9pt}

{ \begin{center}  %fig2
 \vspace*{-2pt}
  \mbox{%
 \epsfxsize=61.777mm 
 \epsfbox{gor-2.eps}
 }


\vspace*{9pt}

\noindent
{{\figurename~2}\ \ \small{Схема функционирования простейшей РСМО}}
\end{center}
}

\vspace*{9pt}





\noindent
 класса~$l$, 
$l\hm=\overline{1,L}$.
Для случая, когда случайные величины длительности обслуживания и~вектора объема 
необходимых ресурсов независимы, имеем 
$$
H_l(t,\mathbf{x})=B_l(t)F_l(\mathbf{x})\,,
$$
 где
$B_l(t)$~--- функция распределения времени обслуживания заявки класса~$l$;
$F_l(\mathbf{x})$~--- функция распределения вектора объема ресурсов, тре\-бу\-емых 
заявкам класса~$l$.

Первые статьи, посвященные анализу СМО с~выделением 
каждой поступающей заявке помимо прибора некоторого случайного объема ресурса 
единственного типа появились в~начале \mbox{1970-х~гг.}~\cite{Romm_21_1971,Kac}.
В~част\-ности, в~работе~\cite{Romm_21_1971} рассматривалась бесконечно линейная 
СМО с~пуассоновским входящим потоком 
и~экспоненциальным временем обслуживания (рис.~2). 
Величины требуемых 
ресурсов~--- независимые одинаково распределенные случайные величины с~функцией 
распределения~$F(x)$. В~качестве емкости системы, т.\,е.\ максимально допустимого 
объема ресурсов, выступает величина~$R$.

Система уравнений равновесия (СУР) для случайного процесса, описывающего 
систему, фактически представляет собой обобщение системы Эрланга. В~результате 
решения СУР были получены стационарные вероятности того, что в~сис\-те\-ме 
находится~$k$~заявок:
\begin{equation*}
p_k=\fr{({1}/{k!})({\lambda}/{\mu})^kF^{(k)}(R)}{\sum\nolimits_{i=0}^{\infty}
({1}/{i!})({\lambda}/{\mu})^iF^{(i)}(R)}\,,
\end{equation*}
где $F^{(k)}(x)$ является $k$-крат\-ной сверткой функции распределения~$F(x)$, 
$k=2,3,\ldots$,
$F^{(0)}(x)\hm=1$, $F^{(1)}(x)\hm=F(x)$.
В~условиях описанной модели потеря заявки или отказ в~обслуживании происходят 
только тогда, когда раз\-ность между величиной объема всей сис\-те\-мы и~суммарным 
объемом ресурсов, занятых находящимися в~сис\-те\-ме заявками, меньше, чем величина 
требуемого объема ресурсов у~вновь поступившей заявки. Таким образом, 
вероятность потери заявки равна
\begin{equation*}
B=1-
\fr{\sum\nolimits_{k=0}^{\infty}({1}/{k!})({\lambda}/{\mu})^kF^{(k+1)}(R)}
{\sum\nolimits_{k=0}^{\infty}({1}/{k!})({\lambda}/{\mu})^kF^{(k)}(R)}\,.
\end{equation*}

Для того чтобы более детально ознакомиться с~особенностями построения и~анализа 
РСМО, подробнее остановимся на статье~\cite{Naumov_3_2016}.
Здесь рас\-смат\-ри\-ва\-ет\-ся многолинейная СМО c $N\hm\leq \infty$ приборами. Поступающий 
поток является пуассоновским с~па\-ра\-мет\-ром~$\lambda$, длительности обслуживания 
заявок независимы между собой и~от поступающего потока и~имеют экспоненциальное 
распределение с~параметром~$\mu$. Система располагает ограниченным объемом 
ресурсов~$M$~типов.
Обозначим через~$R_m$ общий объем ресурса типа~$m$, $\mathbf{R}\hm=(R_1,\ldots,R_M)$, 
и~через $\mathbf{r}_j\hm=(r_{j1}, r_{j2},\ldots, r_{jM})$~--- вектор объемов 
ресурсов, необходимых $j$-й поступившей заявке, $j \hm= 1, 2,\ldots$
Будем считать, что случайные векторы~$\mathbf{r}_j$ не зависят от процессов 
поступления и~обслуживания заявок, независимы в~совокупности и~одинаково 
распределены с~функцией распределения $F(\mathbf{x}), \mathbf{x}\hm=(x_1,\ldots,x_M)$.
Состояние такой системы в~момент~$t$ можно описать полумарковским процессом 
$X(t)\hm=\{\xi(t),\boldsymbol{\Gamma}(t)\}$~\cite{Naumov_3_2016}. Здесь~$\xi(t)$~--- 
число заявок в~сис\-те\-ме, а~$\mathbf{\Gamma}(t)\hm=
(\boldsymbol{\gamma}_1(t),\boldsymbol{\gamma}_2(t),\ldots,\boldsymbol{\gamma}_{\xi(t)}
(t))$, где $\boldsymbol{\gamma}_i(t)$~--- вектор объемов 
всех типов ресурсов, занимаемых $i$-й обслуживаемой заявкой. Находящиеся на 
обслуживании заявки перенумеровываются в~порядке убывания остаточного времени 
обслуживания.
Рассмотрим предельное распределение процесса~$X(t):$
\begin{align*}
p_0&=\lim_{t\rightarrow \infty}P\{\xi(t)=0\}\,;
\\
P_k\left(\mathbf{x}_1,\mathbf{x}_2,\ldots,\mathbf{x}_k\right)&=\lim\limits_{t\rightarrow 
\infty} P
\left\{\xi(t)=k;\right.\\
&
 \hspace*{-20mm}\left.\boldsymbol{\gamma}_1(t)\leq 
\mathbf{x}_1,\enskip
\boldsymbol{\gamma}_2(t)\leq 
\mathbf{x}_2,\ldots,\boldsymbol{\gamma}_k(t)\leq \mathbf{x}_k\right\}\,.
\end{align*}
После решения соответствующей СУР получаем
\begin{align*}
%\label{eq:p_0}
p_0&=\left(1+\sum\limits_{i=1}^{N}F^{(k)}(\mathbf{R})\fr{\rho^k}{k!}     \right)^{-1}\,;\\
P_k(\mathbf{x}_1,\mathbf{x}_2,\ldots,\mathbf{x}_k)&=p_0F(\mathbf{x}_1)F(\mathbf{x}_2
)\cdots F(\mathbf{x}_k)\frac{\rho^k}{k!},\\
&\hspace*{-20mm}\mathbf{x}_1,\mathbf{x}_2,\ldots,\mathbf{x}_k \geq \mathbf{0}, \enskip 
\sum\limits_{i=1}^{k}\mathbf{x}_i\leq \mathbf{R}, \enskip 1\leq k \leq N,
\end{align*}
где $\rho=\lambda/\mu$; $F^{(k)}(\mathbf{x})$~--- $k$-крат\-ная свертка 
функции~$F(\mathbf{x})$; $\mathbf{x}_i=(x_{i1},\ldots,x_{iM})$, $i\hm=\overline{1,k}$.
Далее, если обозначить вектор суммарных объемов занятых ресурсов каждого типа 
$\boldsymbol{\delta}(t)\hm=\sum\nolimits_{i=1}^{\xi(t)}\boldsymbol{\gamma}_i(t)$, 
$\boldsymbol{\delta}(t)\hm=(\delta_1(t),\ldots,\delta_M(t))$, стационарное 
распределение~$Q_k(\mathbf{x})$ случайного процесса 
$X(t)\hm=(\xi(t);\boldsymbol{\delta}(t))$ примет вид:
\begin{multline*}
\label{eq:Q}
\hspace*{-6pt}Q_k(\mathbf{x})=\lim_{t\rightarrow \infty} P\{ \xi(t)=k; 
\boldsymbol{\delta}(t)\leq \mathbf{x}\}=p_0F^{(k)}(\mathbf{x}) 
\fr{\rho^{k}}{k!}\,,\\
\mathbf{0}\leq \mathbf{x} \leq \mathbf{R}\,,\enskip  1\leq k \leq N\,.
\end{multline*}

В~\cite{Naumov_6_2015} исследуется РСМО с~единственным типом ограниченного 
ресурса объема~$R$, но уже с~$L$ входящими пуассоновскими потоками 
с~интенсивностями $\lambda_1,\ldots,\lambda_L$ и~с~$N\hm\leq\infty$ приборами.\linebreak 
Длительности обслуживания заявок независимы между собой, от поступающих потоков и~экспоненциально распределены с~параметром~$\mu_l$ для заявок класса~$l$, 
$l\hm=\overline{1,L}$.
Предполагается, что объем %\linebreak 
ресурса, требуемого заявкам класса~$l$, является 
случайной величиной 
с~функцией распределения~$F_l(x)$, не зависящей от процессов поступления 
и~обслуживания заявок. Обслуживающимся заявкам присваивается номер, причем так, 
чтобы заявка с~номером $i$ имела $i$-е по величине остаточное время 
обслуживания. Этот номер следует отличать от порядкового номера заявки. При 
поступлении новой заявки все находящиеся на обслуживании заявки 
перенумеровываются.
Состояние системы в~момент~$t$ описывается полумарковским процессом 
$X(t)\hm=(\xi(t);\boldsymbol{\theta}(t);\boldsymbol{\gamma}(t))$. Здесь, как 
и~прежде, $\xi(t)$~--- число заявок в~сис\-те\-ме; 
$\boldsymbol{\theta}(t)\hm=(\theta_1(t),\theta_2(t),\ldots,\theta_{\xi(t)}(t))$; 
$\boldsymbol{\gamma}(t)\hm=(\gamma_1(t),\gamma_2(t),\ldots,\gamma_{\xi(t)}(t))$, где 
$\theta_i(t)$~--- класс $i$-й обслуживаемой заявки; $\gamma_i(t)$~--- объем 
занимаемого ею ресурса.

Введем стационарное распределение процесса~$X(t)$
\begin{align*}
p_0&=\lim_{t\rightarrow \infty}P\{\xi(t)=0\}\,;
\\
p^k_{l_1,\ldots,l_k}(x_1,\ldots,x_k)&=\lim\limits_{t\rightarrow \infty}P
\left\{\xi(t)=k; \right.\\
&\hspace*{-10mm}\theta_1(t)=l_1,\ldots,\theta_k(t)=l_k; \\
&\left.\gamma_1(t)\leq x_1,\ldots,\gamma_k(t)\leq x_k
\right\}.
\end{align*}
В~результате решения соответствующей СУР получены выражения для стационарных 
вероятностей описанной сис\-те\-мы.
Кроме того, в~\cite{Naumov_6_2015} показано, что стационарные вероятности того, 
что в~сис\-те\-ме находятся
$k_j$ заявок типа~$j$ и~суммарный объем занимаемого ими ресурсов не превосходит~$x_j$, 
$j\hm=\overline{1,L}$, имеют мультипликативный вид:
\begin{equation*}
P_{k_1,\ldots,k_L}(x_1,\ldots,x_k)=p_0
\prod\limits_{j=1}^{L}F_j^{(k_j)}(x_j)\fr{\rho_j^{k_j}}{k_j!}\,.
\end{equation*}

В~\cite{Naumov_10_2015} исследуются показатели эффективности сетей LTE. 
Ресурсная СМО, моделирующая сис\-те\-му, аналогична представленной 
в~\cite{Naumov_6_2015}, но уже с~$M$~типами ограниченных ресурсов, и~потому 
стационарные вероятности
\begin{align*}
p_0&=\lim\limits_{t\rightarrow \infty}P\{\xi(t)=0\}\,;
\\
p^k_{l_1,\ldots,l_k}(\mathbf{x}_1,\ldots,\mathbf{x}_k)&=
\lim\limits_{t\rightarrow \infty}P
\left\{\xi(t)=k;\right.\\
 &\theta_1(t)=l_1,\ldots,\theta_k(t)=l_k;\\
&\left.\boldsymbol{\gamma}_1(t)\leq \mathbf{x}_1,\ldots,\boldsymbol{\gamma}_k(t)\leq 
\mathbf{x}_k
\right\}
\end{align*}
после решения соответствующей СУР примут вид:
\begin{multline*}
\hspace*{70pt}p_0=\left( 
\vphantom{\sum\limits_{r=1}^{N}}
1+{}\right.\\
\left.{}+\sum\limits_{r=1}^{N}\sum\limits_{k_1+\cdots+k_r=r}\hspace*{-3mm}\left(
F_1^{(k_1)}*F_2^{(k_2)}*\cdots *F_r^{(k_r)}
\right)(\mathbf{R})\times{}\right.\\
\left.{}\times \fr{\rho_1^{k_1}}{k_1!}\,\fr{\rho_2^{k_2}}{k_2!}\cdots
\fr{\rho_1^{k_r}}{k_r!} \right)^{-1};
\end{multline*}

\vspace*{-12pt}

\noindent
\begin{multline*}
p^k_{l_1,\ldots,l_k}(\mathbf{x}_1,\ldots,\mathbf{x}_k)={}\\
{}=p_0 
F_{l_1}(\mathbf{x}_1)F_{l_2}(\mathbf{x}_2)\cdots F_{l_k}(\mathbf{x}_k)
\displaystyle\prod\limits_{n=1}^{k}\fr{\lambda_{l_n}}{\sum\nolimits_{i=1}^{n}\mu_{l_i}},\\
1\leq l_1,\ldots,l_k \leq L\,, \enskip 
\mathbf{x}_1,\mathbf{x}_2,\ldots,\mathbf{x}_k\geq \mathbf{0}\,, \\
\displaystyle \sum\limits_{i=1}^{k}\mathbf{x}_i\leq \mathbf{R}, \enskip
1\leq k \leq N,
\end{multline*}
где символ~$*$ означает свертку функции распределения.

В работе~\cite{Naumov_15_2017} моделируется ситуация, когда объем ресурсов, 
запрашиваемых пользователями, может быть не только положительным, но 
и~отрицательным. Запросы на отрицательный объем ресурса увеличивают объем 
доступного ресурса для пользователей, запрашивающих его положительные объемы. 
В~\cite{Naumov_15_2017} предполагается зависимость времени обслуживания 
и~интервалов между поступлениями заявок от числа заявок в~системе. В~результате 
анализа моделей получены формулы для расчета основных 
ве\-ро\-ят\-но\-ст\-но-вре\-мен\-ных характеристик.

В~\cite{ Sopin_12_2017,Sopin_13_2017} для анализа сетей LTE с~динамически 
меняющимися требованиями к~ресурсам рас\-смат\-ри\-ва\-ют\-ся РСМО с~добавлением 
пуассоновского потока сигналов, инициирующего перераспределение ресурсов для 
активных пользователей. Развитием работы~\cite{Sopin_13_2017} 
стали статьи~\cite{Naumov_14_2017, Dohler_2017}.
Были исследованы два сценария перераспределения ресурсов и~сопоставлены 
посредством численного анализа.

В серии работ~\cite{Sopin_13_2017,Sopin_4_2015,Sopin_5_2015,Sopin_7_2016,Sopin_8_2017,Vihrova_
9_2017,Sopin_11_2017,Sopin_17_2018} исследуются РСМО, в~которых объем выделяемых 
заявке ресурсов имеет дискретное распределение, т.\,е.\ для $i$-й поступившей 
в~систему заявки с~вероятностью $p_j\hm=P(r_i\hm=j)$ потребуется ресурс объема~$j$.
Так, в~\cite{Sopin_7_2016} анализируется РСМО с~$L$~входящими пуассоновскими 
потоками и~$M$~типами ресурсов. Получены выражения для стационарных 
вероятностей:
\begin{multline*}
q^k_{k_1,\ldots,k_L}(\mathbf{r}_1,\ldots,\mathbf{r}_L)={}\\
{}=q_0
\sum\limits_{k_1+\cdots+k_l=k}p^{(k_1)}_{1,\mathbf{r}_1}\cdots p^{(k_L)}_{1,\mathbf{r}_L}
\fr{\rho_1^{k_1}}{k_1!}\cdots \fr{\rho_L^{k_L}}{k_L!}\,;
\end{multline*}

\vspace*{-12pt}

\noindent
\begin{multline*}
\hspace*{76pt}q_0={}\\
\!{}=\left(\!
\vphantom{\sum\limits_{k=0}^{N}\sum\limits_{k_1+\cdots+k_L=k}
\sum\limits_{\mathbf{r}_1+\cdots+\mathbf{r}_L \leq 
\mathbf{R}} p^{(k_L)}_{1,\mathbf{r}_L}\fr{\rho_1^{k_1}}{k_1!}\cdots
\fr{\rho_L^{k_L}}{k_L!}}
1+ {}\right. 
\left.\!\!\!\sum\limits_{k=0}^{N}\sum\limits_{k_1+\cdots+k_L=k}
\sum\limits_{\mathbf{r}_1+\cdots+\mathbf{r}_L \leq 
\mathbf{R}} \hspace*{-8pt}p^{(k_L)}_{1,\mathbf{r}_L}\fr{\rho_1^{k_1}}{k_1!}\cdots
\fr{\rho_L^{k_L}}{k_L!}
\!\right)^{\!-1}\!\!\!,\hspace*{-8.1138pt}
\end{multline*}
где $q^k_{k_1,\ldots,k_L}(\mathbf{r}_1,\ldots,\mathbf{r}_L)$~--- это вероятность 
того, что в~системе находятся~$k$~заявок, из которых~$k_1$~--- класса~1, 
$k_2$~--- класса~2 и~т.\,д., а~суммарный объем ресурсов каждого типа, занятых заявками 
класса~1, равен~$\mathbf{r}_1$ и~т.\,д.

В статье~\cite{Vihrova_9_2017} при исследовании той же СМО, что 
и~в~\cite{Sopin_7_2016}, было получено распределение стационарных вероятностей~$q_k(\mathbf{r})$ 
с~объединенным входящим потоком и~средневзвешенным требованием 
$$
p_{\mathbf{r}}= \sum\limits_{l=1}^{L} \fr{\rho_l}{\rho} \,p_{l,\mathbf{r}},$$
 где 
$\rho\hm=\sum\nolimits_{l=1}^{L}\rho_l$:
\begin{equation*}
q_k(\mathbf{r})=q_0\fr{\rho^k}{k!}\,p_{\mathbf{r}}^{(k)}\,, \quad q_0=\left( 
\sum\limits_{k=0}^{N}\sum\limits_{\mathbf{r}=\mathbf{0}}^{\mathbf{R}} p_{\mathbf{r}}^{(k)} 
\right)^{-1},
\end{equation*}
что позволило выразить вероятность блокировки и~среднего объема занятых ресурсов 
в~аналитическом виде.

В случае рассматриваемой СМО, но с~заявками одного класса, 
в~\cite{Sopin_11_2017} представлены выражения для стационарных вероятностей 
состояний числа заявок в~системе и~суммарного объема занятых ресурсов, а~также 
формулы для вероятности блокировки и~среднего объема занятых ресурсов.


\section{Вычисление характеристик ресурсных систем массового обслуживания}

В работе~\cite{Naumov_1_2014} для системы с~одним типом ресурса показано, что 
в~предположении о~гам\-ма-рас\-пре\-де\-ле\-нии необходимого заявкам ресурса плотность 
распределения высвобождаемого заявкой ресурса при заданном числе заявок 
в~системе и~заданном векторе суммарных объемов занятых ресурсов совпадает 
с~бе\-та-рас\-пре\-де\-ле\-ни\-ем, позволяющим легко рассчитывать многократные свертки, к~которым 
приводит необходимость учитывать объемы всех заявок в~системе. В~остальных 
случаях наличие в~формулах большого числа сверток создает значительную 
вычислительную сложность при расчете стационарных характеристик РСМО.
Так, для расчета характеристик СМО из~\cite{Sopin_7_2016} необходимо для каж\-до\-го 
$k\hm\in \{0,\ldots,N\}$, а~также всех наборов векторов $\mathbf{r}\hm \leq \mathbf{R}$ 
хранить в~памяти компьютера значения сверток вероятностей~$p_{\mathbf{r}}$. 
А~при больших значениях~$N$ и~$\mathbf{R}$ вычисление вероятностей блокировок 
системы и~также объемов занятого ресурса по представленным формулам вообще 
нерационально. Поэтому задача получения действенных численных методов является 
крайне важной.
В~работе~\cite{Sopin_8_2017} для модели СМО из~\cite{Sopin_7_2016}, чтобы 
сократить вычисления, был предложен рекуррентный алгоритм вычисления 
нормировочной константы $G(N,\mathbf{R})\hm=q_0^{-1}$, основанный на алгоритме 
Бузена~\cite{Buzen}. Кроме того, на основе разработанного алгоритма были 
получены рекуррентные формулы для вычисления вероятностных характеристик 
сис\-те\-мы: вероятности блокировки сис\-те\-мы, среднего объема дисперсии занятых 
ресурсов.
Если обозначить
\begin{equation*}
G(n,\mathbf{r})\sum\limits_{k=0}^{n}\sum\limits_{\mathbf{j}=\mathbf{0}}^{\mathbf{r}}p_{\mathbf
{j}}^{(k)}\fr{\rho^k}{k!}\,, \enskip 
n\geq 0\,, \enskip \mathbf{r}\geq \mathbf{0}\,,
\end{equation*}
то функция $G(n,\mathbf{r})$ будет удовлетворять следующему рекуррентному 
соотношению:
\begin{multline*}
G(n,\mathbf{r})=G(n-1,\mathbf{r})+{}\\
{}+\fr{\rho}{n!}
\sum\limits_{\mathbf{j}=\mathbf{0}}^{\mathbf{r}}p_{\mathbf{j}}
\left( G(n-1,\mathbf{r}-\mathbf{j})- G(n-2,\mathbf{r}-\mathbf{j})   \right)
\end{multline*}
с начальными условиями
\begin{equation*}
G(0,\mathbf{r})=1,\enskip \mathbf{r}\geq 0\,; \quad
G(1,\mathbf{r})=1+\sum\limits_{\mathbf{j}=\mathbf{0}}^{\mathbf{r}}p_{\mathbf{j}}\,.
\end{equation*}
При анализе РСМО,  описывающих M2M (machine-to-machine) трафик в~сетях LTE, 
аналогичный рекуррент\-ный алгоритм для вычисления нормировочной константы был 
разработан в~\cite{Sopin_11_2017}. Мат\-рич\-ные методы анализа РСМО, применимые при 
моделировании соты сети LTE с~двумя типами трафика, M2M и~H2H (human-to-human), 
предложены в~работах~\cite{Vish_2017,Vish_2016}.

\section{Заключение}

В настоящем обзоре кратко представлены основные разновидности 
РСМО, существующие методы их анализа, выражения для оценки 
основных ве\-ро\-ят\-но\-ст\-но-вре\-мен\-н$\acute{\mbox{ы}}$х 
характеристик и~алгоритмы их вычисления.

{\small\frenchspacing
 {%\baselineskip=10.8pt
 \addcontentsline{toc}{section}{References}
 \begin{thebibliography}{99}
%1
\bibitem{Andrews} %1
\Au{Andrews J.\,G., Buzzi~S., Choi~W., Hanly~S.\,V., Lozano~A., Soong~A.\,C.\,K., 
Zhang~J.\,C.} What will 5G be?~// {IEEE J.~Sel. Area.  
Comm.}, 2014. Vol.~32. No.\,6. P.~1065--1082.



%3
\bibitem{Samuylov} %2
\Au{Buturlin I.\,A., Gaidamaka~Y.\,V., Samuylov~A.\,K.}
Utility function maximization problems for two cross-layer optimization 
algorithms in OFDM wireless networks~// {4th Congress (International) on Ultra 
Modern Telecommunications and Control Systems}, 2012.  P.~63--65.

%2
\bibitem{Galinina} %3
\Au{Galinina O., Andreev~S.\,D., Gerasimenko~M., Kou\-che\-rya\-vy~Y., Himayat~N., 
Yeh~S.\,P., Talwar~S.} Capturing spatial randomness of heterogeneous 
cellular/WLAN deployments with dynamic traffic~// {IEEE J.~Sel. 
Area. Comm.}, 2014. Vol.~32. No.\,6. P.~1083--1099.

%4
\bibitem{Naumov_3_2016}
\Au{Наумов В.\,А., Самуйлов~К.\,Е., Самуйлов~А.\,К.} О~суммарном объеме 
ресурсов, занимаемых обслуживаемыми заявками~// {Автоматика и~телемеханика}, 
2016. №.\,8. С.~125--135.

%5
\bibitem{Gimpelson}
\Au{Gimpelson L.\,A.}
Analysis of mixtures of wide- and narrow-band traffic~// {IEEE T.~Commun.
 Techn.}, 1968. Vol.~13. No.\,3. P.~258--266.

%6
\bibitem{Kelly}
\Au{Kelly F.\,P.}
Loss networks~// {Ann. Appl. Probab.}, 1991. No.\,1. P.~319--378.

%7
\bibitem{Ross}
\Au{Ross K.\,W.}
Multiservice loss models for broadband telecommunication networks.~--- {London: 
Springer-Verlag}, 1995. 343~p.

%8
\bibitem{Basharin}
\Au{Basharin G.\,P., Samouylov~K.\,E., Yarkina~N.\,V., Gudkova~I.\,A.}
A~new stage in mathematical teletraffic theory~// {Automat. Rem. 
Contr.}, 2009. Vol.~70. No.\,12. P.~1954--1964.

%9
\bibitem{Romm_21_1971}
\Au{Ромм Э.\,Л., Скитович~В.\,В.}
Об одном обобщении задачи Эрланга~// {Автоматика и~телемеханика}, 1971. №.\,6. 
С.~164--168.

%10
\bibitem{Kac}
\Au{Кац Б.\,А.}
Об обслуживании сообщений случайной длины~// {Теория массового обслуживания: 
Тр. 3-й Всесоюзн. шко\-лы-со\-ве\-ща\-ния по тео\-рии массового обслуживания}, 1976. 
С.~157--168.

%11
\bibitem{Naumov_6_2015}
\Au{Наумов В.\,А., Самуйлов~А.\,К.}
Модель выделения ресурсов беспроводной сети объемами случайной величины~// 
{Вестник РУДН. Серия: Математика, информатика, физика}, 2015. №\,2. С.~38--45.

%12
\bibitem{Naumov_10_2015}
\Au{Naumov~V., Samouylov~K., Yarkina~N., Sopin~E., Andreev~S., Samuylov~A.}
LTE performance analysis using queuing systems with finite resources and random 
requirements~// {7th Congress on Ultra Modern Telecommunications and Control 
Systems}.~--- IEEE, 2015. P.~100--103.

%13
\bibitem{Naumov_15_2017}
\Au{Naumov V., Samouylov~K.}
Analysis оf multi-resource loss system with state dependent arrival and service 
rates~// {Probab.  Eng. Inform. Sc.}, 2017. 
Vol.~31. No.\,4. P.~413--419.

%14
\bibitem{Sopin_12_2017}
\Au{Samouylov K., Sopin~E., Vikhrova~O.}
Analysis of queueing system with resources and signals~// {Comm.  
Com. Inf. Sc.}, 2017. Vol.~800. P.~358--369.

%15
\bibitem{Sopin_13_2017}
\Au{Sopin E., Vikhrova~O., Samouylov~K.}
LTE network model with signals and random resource requirement~// {9th 
Congress (International) on Ultra Modern Telecommunications and Control Systems 
and Workshops}.~--- IEEE, 2017. P.~101--106.



%17
\bibitem{Dohler_2017}
\Au{Petrov V., Solomitckii~D., Samuylov~A., Lema Maria~A., Gapeyenko~M., 
Moltchanov~D., Andreev~S., Naumov~V., Samouylov~K., Dohler~M., Koucheryavy~Ye.}
Dynamic multi-connectivity performance in ultra-dense urban mmWave deployments~// 
{IEEE J.~Sel. Area. Comm.}, 2017. Vol.~35. No.~9. 
P.~2038--2055.

%16
\bibitem{Naumov_14_2017}
\Au{Наумов В.\,А., Самуйлов~К.\,Е.}
Анализ сетей ресурсных систем массового обслуживания~// {Автоматика 
и~телемеханика}, 2018. №\,5. С.~59--68.

%18
\bibitem{Sopin_4_2015}
\Au{Samouylov K., Sopin~E., Vikhrova~O.}
Analyzing blocking probability in LTE wireless network via queuing system with 
finite amount of resources~// {Comm.  Com. Inf.
Sc.}, 2015. Vol.~564. P.~393--403.

%19
\bibitem{Sopin_5_2015}
\Au{Вихрова О.\,Г., Самуйлов~К.\,Е., Сопин~Э.\,С., Шоргин~С.\,Я.}
К~анализу показателей качества обслуживания в~современных беспроводных сетях~// 
{Информатика и~её применения}, 2015. Т.~9. Вып.~4. С.~48--55.

%20
\bibitem{Sopin_7_2016}
\Au{Sopin~E., Samouylov~K., Vikhrova~O., Kovalchukov~R., Moltchanov~D., 
Samuylov~A.}
Evaluating a case of downlink uplink decoupling using queuing system with random 
requirement~// 
Internet of Things, smart spaces, and
next generation
networks and systems~/
Eds. O.~Galinina, S.\,I.~Balandin, Y.~Koucheryavy.~---
{Lecture notes in computer science ser.}~--- Springer, 2016. Vol.~9870. P.~440--450.

%21
\bibitem{Sopin_8_2017}
\Au{Samouylov K., Sopin~E., Vikhrova~O., Shorgin~S.}
Convolution algorithm for normalization constant evaluation in queuing system 
with random requirements~// {AIP Conf. Proc.}, 2017. Vol.~1863. Art. 
No.\,090004. 4~p.

%22
\bibitem{Vihrova_9_2017}
\Au{Вихрова О.\,Г.}
К~вычислению вероятностных характеристик СМО ограниченной емкости со случайными 
требованиями к~ресурсам~// {Вестник РУДН. Серия: Математика, информатика, 
физика}, 2017. №\,3. С.~203--210.



%24
\bibitem{Sopin_17_2018}
\Au{Sopin E., Samouylov~K.}
On the analysis of the limited resources queuing system under MAP arrivals~// 
{Conference (International)  on Applied Mathematics, Computational Science and 
Systems Engineering}, 2018. Vol.~16. Art. No.\,01008. 4~p.

%23
\bibitem{Sopin_11_2017}
\Au{Sopin E., Gaidamaka~Yu., Markova~E., Vikhrova~O.}
Performance analysis of M2M traffic in LTE network using queuing systems with 
random resource requirements~// {Autom. Control Comp.~S.}, 2018 
(in press).

%25
\bibitem{Naumov_1_2014}
\Au{Наумов В.\,А., Самуйлов~К.\,Е.}
О~моделировании систем массового обслуживания с~множественными ресурсами~// 
{Вестник РУДН. Серия: Математика, информатика, физика}, 2014. №\,3. С.~60--64.

%26
\bibitem{Buzen}
\Au{Buzen J.\,P.}
Computational algorithms for closed queueing networks with exponential servers~// 
{Commun. ACM}, 1973. Vol.~16. P.~527--531.



%28
\bibitem{Vish_2016}
\Au{Вишневский В.\,М., Самуйлов~К.\,Е., Наумов~В.\,А., Яркина~Н.\,В.}
Модель соты LTE с~межмашинным трафиком в~виде мультисервисной системы массового 
обслуживания с~эластичными и~потоковыми заявками и~марковским входящим потоком~// 
{Вестник РУДН. Серия: Математика, информатика, физика}, 2016. №\,4. С.~26--36.

%27
\bibitem{Vish_2017}
\Au{Vishnevsky~V., Samouylov~K., Naumov~V., Krishnamoorty~A., Yarkina~N.}
Multiservice queieing system with map arrivals for modelling LTE cell with H2H 
and M2M communications and M2M aggregation~// {Comm. Com. 
Inf. Sc.}, 2017. Vol.~700. P.~63--74.

 \end{thebibliography}

 }
 }

\end{multicols}

\vspace*{-6pt}

\hfill{\small\textit{Поступила в~редакцию 16.06.18}}

\vspace*{6pt}

%\newpage

%\vspace*{-24pt}

\hrule

\vspace*{2pt}

\hrule

\vspace*{-2pt}


\def\tit{RESOURCE QUEUING SYSTEMS AS~MODELS OF~WIRELESS COMMUNICATION SYSTEMS}


\def\titkol{Resource queuing systems as~models of~wireless communication systems}

\def\aut{A.\,V.~Gorbunova$^1$, V.\,A.~Naumov$^2$, Yu.\,V.~Gaidamaka$^{1,3}$, and~K.\,E.~Samouylov$^{1,3}$}

\def\autkol{A.\,V.~Gorbunova, V.\,A.~Naumov, Yu.\,V.~Gaidamaka, and~K.\,E.~Samouylov}

\titel{\tit}{\aut}{\autkol}{\titkol}

\vspace*{-11pt}


\noindent
$^1$Peoples' Friendship University of Russia 
(RUDN University), 6~Miklukho-Maklaya Str., Moscow 117198, Russian\linebreak
$\hphantom{^1}$Federation

\noindent
$^2$Service Innovation Research Institute, 8A~Annankatu, Helsinki 
00120, Finland

\noindent
$^3$Institute of Informatics Problems, 
Federal Research Center ``Computer Science and Control'' 
of the Russian\linebreak
$\hphantom{^1}$Academy of Sciences, 44-2~Vavilov Str., Moscow 119333, 
Russian Federation


\def\leftfootline{\small{\textbf{\thepage}
\hfill INFORMATIKA I EE PRIMENENIYA~--- INFORMATICS AND
APPLICATIONS\ \ \ 2018\ \ \ volume~12\ \ \ issue\ 3}
}%
 \def\rightfootline{\small{INFORMATIKA I EE PRIMENENIYA~---
INFORMATICS AND APPLICATIONS\ \ \ 2018\ \ \ volume~12\ \ \ issue\ 3
\hfill \textbf{\thepage}}}

\vspace*{3pt}


 
\Abste{The article presents an overview of the resource queuing 
systems used for modeling of a~wide class of real systems with 
admittedly limited resources. Despite the objective importance of studying 
of such systems, there have been very few works devoted to their 
analysis until recently, which was due to the complexity of constructing
a~random process to describe their functioning and, accordingly, of obtaining 
the numerical results. However, in
recent years, there has been 
a~significant shift in the study of the resource systems~--- new 
methods for their analysis have been proposed, which made it possible to 
construct recursive algorithms suitable for the numerical calculations.
In this regard, the current review reflects only a part of the previously 
obtained results, namely, it considers\linebreak\vspace*{-12pt}}

\Abstend{the resource systems without waiting 
space with exponentially distributed service time. The authors consider the models 
of wireless communication systems based on resource queuing systems, expressions 
for estimating the main 
probabilistic, and temporal characteristics and algorithms for their calculation.}

\KWE{resource queueing systems; continuous resource; discrete resource; 
limited resource; recursive algorithm; heterogeneous network; 
stationary distribution; semi-Markov process; wireless communication systems}
 
\DOI{10.14357/19922264180307}

%\vspace*{-14pt}

\Ack
\noindent
The work was partly supported by the Russian Ministry of Education and
Science
(project 2.882.2017/4.6).



%\vspace*{6pt}

  \begin{multicols}{2}

\renewcommand{\bibname}{\protect\rmfamily References}
%\renewcommand{\bibname}{\large\protect\rm References}

{\small\frenchspacing
 {%\baselineskip=10.8pt
 \addcontentsline{toc}{section}{References}
 \begin{thebibliography}{99}
\bibitem{1-gor}
\Aue{Andrews, J.\,G., S.~Buzzi, W.~Choi, S.\,V.~Hanly, A.~Lozano, 
A.\,C.\,K.~Soong, and J.\,C.~Zhang.} 2014. What will 5G be? 
\textit{IEEE J.~Sel. Area. Comm.} 32(6):1065--1082.

\bibitem{3-gor}
\Aue{Buturlin, I.\,A., Y.\,V.~Gaidamaka, and A.\,K.~Samuylov.} 
2012. Utility function maximization problems for two cross-layer optimization 
algorithms in OFDM wireless networks. 
\textit{4th Congress (International) on Ultra Modern Telecommunications and 
Control Systems}. 63--65.

\bibitem{2-gor}
\Aue{Galinina, O.\,S., D.~Andreev, M.~Gerasimenko, Y.~Koucheryavy, N.~Himayat, 
S.\,P.~Yeh, and S.~Talwar.} 2014. Capturing spatial randomness of heterogeneous 
cellular/WLAN deployments with dynamic traffic. 
\textit{IEEE J.~Sel. Area. Comm.} 32(6):1083--1099.
\bibitem{4-gor}
\Aue{Naumov, V.\,A., K.\,E.~Samuilov, and A.\,K.~Samuilov.} 2016. 
On the total 
amount of resources occupied by serviced customers. 
\textit{Automat. Rem. Contr.} 77(8):1419--1427.
\bibitem{5-gor}
\Aue{Gimpelson, L.\,A.} 1968. Analysis of 
mixtures of wide- and narrow-band traffic. 
\textit{IEEE T.~Commun. Techn.} 13(3):258--266.
\bibitem{6-gor}
\Aue{Kelly, F.\,P.} 1991. Loss networks. \textit{Ann. Appl. Probab.} 1:319--378.
\bibitem{7-gor}
\Aue{Ross, K.\,W.} 1995. \textit{Multiservice loss models for broadband telecommunication 
networks}. London: Springer-Verlag. 343~p.
\bibitem{8-gor}
\Aue{Basharin, G.\,P., K.\,E.~Samouylov, N.\,V.~Yarkina, and I.\,A.~Gudkova.} 2009. 
A~new stage in mathematical teletraffic theory. 
\textit{Automat. Rem. Contr.} 70(12):1954--1964.
\bibitem{9-gor}
\Aue{Romm, E.\,L., and V.\,V.~Skitovich.} 1971. Ob odnom obobshchenii zadachi Erlanga 
[On a generalization of the Erlang problem]. 
\textit{Automat. Rem. Contr.} 6:164--168.
\bibitem{10-gor}
\Aue{Kats, B.\,A.} 1976. Ob obsluzhivanii soobshcheniy sluchaynoy dliny 
[On serving messages of random length]. 
\textit{Teoriya massovogo obsluzhivaniya. Tr. 3~Vsesoyuzn. 
shkoly-soveshchaniya po teorii massovogo obsluzhivaniya} 
[Queuing Theory: 3rd All-Union School-Seminar on Queuing Theory Proceedings]. 157--168.
\bibitem{11-gor}
\Aue{Naumov, V.\,A., and A.\,K.~Samuylov.} 
2015. Model' vydeleniya resursov besprovodnoy seti ob''emami sluchaynoy velichiny 
[Queuing system with resource allocation of the random volume]. 
\textit{RUDN J.~Math. 
Information Sci. Phys.} 2:38--45.
\bibitem{12-gor}
\Aue{Naumov, V., K.~Samouylov, N.~Yarkina, E.~Sopin, S.~Andreev, and A.~Samuylov.}
2015. LTE performance analysis using queuing systems with finite resources 
and random requirements. 
\textit{7th Congress on Ultra Modern Telecommunications and Control Systems}. 
IEEE. 100--103.
\bibitem{13-gor}
\Aue{Naumov, V., and K.~Samouylov.} 2017. Analysis оf multi-resource loss 
system with state dependent arrival and service rates. 
\textit{Probab. Eng. Inform. Sc.} 31(4):413--419.
\bibitem{14-gor}
\Aue{Samouylov, K., E.~Sopin, and O.~Vikhrova.} 2017. 
Analysis of queueing system with resources and signals. 
\textit{Comm. Com. Inf. Sc.} 800:358--369.
\bibitem{15-gor}
\Aue{Sopin, E., O.~Vikhrova, and K.~Samouylov.} 2017. 
LTE network model with signals and random resource requirement. 
\textit{9th Congress (International) on Ultra Modern Telecommunications and 
Control Systems and Workshops}. 101--106.

\bibitem{17-gor}
\Aue{Petrov, V., D.~Solomitckii, A.~Samuylov, A.~Maria Lema, M.~Gapeyenko, 
D.~Moltchanov, S.~Andreev, V.~Naumov, K.~Samouylov, M.~Dohler, and Ye.~Koucheryavy}. 
2017. Dynamic multi-connectivity performance in ultra-dense urban 
mmWave deployments. \textit{IEEE J.~Sel. Area. Comm.} 35(9):2038--2055.

\bibitem{16-gor}
\Aue{Naumov, V.\,A., and K.\,E.~Samuilov.} 2018. 
Analysis of networks of the resource queuing systems. 
\textit{Automat. Rem. Contr}. 79(5):822--829.
\bibitem{18-gor}
\Aue{Samouylov, K., E.~Sopin, and O.~Vikhrova.} 2015. 
Analyzing blocking probability in LTE wireless network via queuing system 
with finite amount of resources. 
\textit{Comm. Com. Inf. Sc.} 564:393--403.
\bibitem{19-gor}
\Aue{Vikhrova, O.\,G., K.\,E.~Samouylov, E.\,S.~Sopin, and S.\,Ya.~Shorgin.} 
2015. K~analizu pokazateley kachestva obsluzhivaniya 
v~sovremennykh besprovodnykh setyakh [On performance analysis of modern 
wireless networks]. \textit{Informatika i~ee Primeneniya~---
Inform. Appl.} 9(4):48--55.
\bibitem{20-gor}
\Aue{Sopin, E., K.~Samouylov, O.~Vikhrova, R.~Kovalchukov, D.~Moltchanov, and 
A.~Samuylov.} 2016. Evaluating a~case of downlink uplink decoupling 
using queuing system with random requirement. 
\textit{Internet of Things, smart spaces, and
next generation
networks and systems}.
Eds. O.~Galinina, S.\,I.~Balandin, and Y.~Koucheryavy.
{Lecture notes in computer science ser.} Springer. 9870:440--450.
\bibitem{21-gor}
\Aue{Samouylov, K., E.~Sopin, O.~Vikhrova, and S.~Shorgin.} 
2017. Convolution algorithm for normalization constant evaluation in 
queuing system with random requirements. \textit{AIP Conf. Proc}. 
1863:090004. 4~p.
\bibitem{22-gor}
\Aue{Vikhrova, O.\,G.} 2017. 
K~vychisleniyu veroyatnostnykh kharakteristik SMO ogranichennoy emkosti so 
sluchaynymi trebovaniyami k~resursam [About probability characteristics evaluation 
in queuing system with limited resources and random requirements]. 
\textit{RUDN J.~Math. Information Sci. Phys.} 25(3):203--210.

\bibitem{24-gor}
\Aue{Sopin, E., and K.~Samouylov.} 2018. On the analysis of the limited resources 
queuing system under MAP arrivals. 
\textit{Conference (International) 
on Applied Mathematics, Computational Science and Systems Engineering}. 16:01008. 4~p.

\bibitem{23-gor}
\Aue{Sopin, E., Yu.~Gaidamaka, E.~Markova, and O.~Vikhrova.} 2018 (in press).
 Performance analysis of M2M traffic in LTE network using queuing systems 
 with random resource requirements. 
 \textit{Autom. Control Comp.~S.}
\bibitem{25-gor}
\Aue{Naumov, V.\,A., and K.\,E.~Samuylov.} 2014. 
O~modelirovanii sistem massovogo obsluzhivaniya s~mnozhestvennymi resursami 
[On the modeling of queueing systems with multiple resources]. 
\textit{[RUDN J.~Math. Information Sci. Phys.} 3:60--64.
\bibitem{26-gor}
\Aue{Buzen, J.\,P.} 1973. Computational algorithms for closed queueing 
networks with exponential servers. \textit{Commun. ACM} 16:527--531.

\bibitem{28-gor}
\Aue{Vishnevsky, V.\,M., K.\,E.~Samouylov, V.\,A.~Naumov, and N.\,V.~Yarkina.}
2016. Model' soty LTE s~mezhmashinnym trafikom v~vide mul'tiservisnoy sistemy 
massovogo obsluzhivaniya s~elastichnymi i~potokovymi zayavkami 
i~markovskim vkhodyashchim potokom [Multiservice queuing system with
 elastic and streaming flows and markovian arrival process for modelling 
 LTE cell with M2M traffic]. 
 \textit{RUDN J.~Math. Information Sci. Phys.} 4:26--36.
 
 \bibitem{27-gor}
\Aue{Vishnevsky, V., K.~Samouylov, V.~Naumov, A.~Krishnamoorty, and N.~Yarkina.} 2017.
Multiservice queieing system with map arrivals for modelling LTE cell with H2H 
and M2M communications and M2M aggregation. 
\textit{Comm. Com. Inf. Sc.} 700:63--74.
 
 \end{thebibliography}

 }
 }

\end{multicols}

\vspace*{-6pt}

\hfill{\small\textit{Received June 16, 2018}}

%\pagebreak

%\vspace*{-18pt}

 
\Contr

\noindent
\textbf{Gorbunova Anastasiya V.} (b.\ 1986)~--- 
Candidate of Science (PhD) in physics and mathematics, 
assistant professor, Peoples' Friendship University of Russia 
(RUDN University), 6~Miklukho-Maklaya Str., 
Moscow 117198, Russian Federation; \mbox{gorbunova\_av@rudn.university}

\vspace*{3pt}

\noindent
\textbf{Naumov Valeriy A.} (b.\ 1950)~--- 
Candidate of Science (PhD) in physics and mathematics, 
Research Director, Service Innovation Research Institute, 8A~Annankatu, Helsinki 
00120, Finland; \mbox{valeriy.naumov@pfu.fi}

\vspace*{3pt}

\noindent
\textbf{Gaidamaka Yuliya V.} (b.\ 1971)~--- Doctor of Science in physics and mathematics, 
professor, Peoples' Friendship University of Russia 
(RUDN University), 6~Miklukho-Maklaya Str., 
Moscow 117198, Russian Federation;\linebreak senior scientist, 
Institute of Informatics Problems, 
Federal Research Center\ ``Computer Science and Control''\linebreak 
of the Russian Academy of Sciences, 44-2~Vavilov Str., Moscow 119333, 
Russian Federation; \mbox{gaydamaka\_yuv@rudn.university}

\vspace*{3pt}

\noindent
\textbf{Samuylov Konstantin E.} (b.\ 1955)~--- Doctor of Science in technology, 
professor, Head of Department, Peoples' Friendship University of Russia 
(RUDN University), 6~Miklukho-Maklaya Str., 
Moscow 117198, Russian Federation; senior scientist, 
Institute of Informatics Problems, Federal Research Center 
``Computer Science and Control'' of the Russian Academy of Sciences, 
44-2~Vavilov Str., Moscow 119333, Russian Federation; 
\mbox{samuylov\_ke@rudn.university}

\label{end\stat}

\renewcommand{\bibname}{\protect\rm Литература}       