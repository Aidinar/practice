\def\stat{bosov+naumov}

\def\tit{МОДЕЛЬ ПЕРЕДВИЖЕНИЯ ПОЕЗДОВ И~МАНЕВРОВЫХ ЛОКОМОТИВОВ 
НА~ЖЕЛЕЗНОДОРОЖНОЙ СТАНЦИИ В~ПРИЛОЖЕНИИ К~ОЦЕНКЕ И~АНАЛИЗУ 
ВЕРОЯТНОСТИ БОКОВОГО СТОЛКНОВЕНИЯ$^*$}

\def\titkol{Модель передвижения поездов и~маневровых локомотивов 
на~железнодорожной станции} % в~приложении к~оценке и~анализу  вероятности бокового столкновения}

\def\aut{А.\,В.~Босов$^1$, А.\,Н.~Игнатов$^2$, А.\,В.~Наумов$^3$}

\def\autkol{А.\,В.~Босов, А.\,Н.~Игнатов, А.\,В.~Наумов}

\titel{\tit}{\aut}{\autkol}{\titkol}

\index{Босов А.\,В.}
\index{Игнатов А.\,Н.}
\index{Наумов А.\,В.}
\index{Bosov A.\,V.}
\index{Ignatov A.\,N.}
\index{Naumov A.\,V.}




{\renewcommand{\thefootnote}{\fnsymbol{footnote}} \footnotetext[1]
{Работа выполнена при поддержке Российского научного фонда (проект №\,16-11-00062).}}


\renewcommand{\thefootnote}{\arabic{footnote}}
\footnotetext[1]{Институт проблем информатики Федерального исследовательского центра <<Информатика 
и~управление>> Российской академии наук, \mbox{AVBosov@ipiran.ru}}
\footnotetext[2]{Московский авиационный институт (национальный исследовательский университет), 
\mbox{alexei.ignatov1@gmail.com}}
\footnotetext[3]{Московский авиационный институт (национальный исследовательский университет), 
\mbox{naumovav@mail.ru}}

\vspace*{-6pt}

  

\Abst{Предложена математическая модель для решения задачи управления движением 
маневровых локомотивов на железнодорожной станции при заданном расписании движения 
пассажирских/грузовых поездов через станцию и~фиксированном графике маневровых работ, 
под которыми понимается отцепка и~прицепка вагонов, выпуск и~расформирование поездов. 
Модель используется для постановки и~решения задачи минимизации времени передвижения по 
станции маневрового локомотива для осуществления очередной маневровой работы с~учетом 
занятости некоторых путей для движения вследствие наличия на них пассажирских/грузовых 
поездов, а также с~учетом ограничений на время исполнения маневровых работ. Исходная 
постановка сводится к~задаче смешанного целочисленного линейного программирования. 
Представленная модель использована для оценки вероятности бокового столкновения на 
станции с~учетом возможных случайных задержек в~движении пассажирских поездов. 
Приведены результаты численных экспериментов.}

\KW{имитационная модель; расписание; интенсивность; смешанное целочисленное линейное 
программирование}

\DOI{10.14357/19922264180315}
  
%\vspace*{4pt}


\vskip 10pt plus 9pt minus 6pt

\thispagestyle{headings}

\begin{multicols}{2}

\label{st\stat}

\section{Введение}

      Основной задачей управления рисками на железнодорожном транспорте 
является достижение и~поддержание допустимого уровня риска различных 
неблагоприятных событий~[1]. К~таким событиям относятся, например, 
столкновения на железнодорожных переездах поездов с~автотранспортом, сход 
вагонов при поездной работе, пожары на локомотивах, излом рельса под поездом и~др. 

Изучение риска таких событий привлекало внимание как российских~[2--5], 
так и~западных~[6] исследователей. Работа~[6] в~этой связи выделяется тем, что 
в~ней исследовались вопросы оптимального положения вагонов с~опасными 
грузами в~поезде. Отдельный пласт задач связан с~оценкой и~анализом рисков 
происшествий, происходящих на железнодорожных станциях.
      
      На крупных станциях, где маневровые работы осуществляются при 
интенсивном движении поездов, в~качестве возможных неблагоприятных событий 
рассматриваются столкновения между маневровыми составами 
и~пассажирскими/грузовыми поездами, взрез стрелки (случайный перевод 
стрелки колесами подвижного состава) или сход с~рельсов маневрового состава. 
Для описания соответствующих рисков в~работе~[7] была предложена оценка 
вероятности хотя бы одного столкновения на станции за произвольный 
промежуток времени, а~в~[8] оценено число взрезов и~сходов с~рельсов. 
Вычисление этих величин основано на интенсивностях пересечения маневровыми 
составами стрелочных переводов, используемых для перевода подвижного состава с~одного пути на другой. 

Для уточнения приведенной в~[7] оценки 
интенсивностей можно провести длительные натурные наблюдения за работой 
конкретной станции, которые могут оказаться весьма дорогостоящими. При\linebreak
 этом 
результаты (оценки интенсивности) будут\linebreak получены только для одной этой 
станции и~не дадут ка\-ких-ли\-бо оценок для других. 

Другим способом уточнения 
интенсивностей является по\-стро\-ение имитационной модели передвижения 
поездов на железнодорожной станции, которую можно было бы использовать на 
различных станциях, задавая небольшой набор входных данных, таких как схема 
станции, график движения поездов и~маневровых работ, под которыми 
понимаются отцепка и~прицепка вагонов, выпуск и~расформирование поездов. 
Набор этих работ будет далее предполагаться фиксированным, поскольку он 
может быть получен из суточного плана работы станции. Кроме того, важно 
отметить, что сами маневровые работы могут быть отсортированы по времени 
начала исполнения. С~течением суток вследствие исполнения маневровых работ 
их общее число уменьшается. Для предотвращения нарушения суточного плана 
работы станции необходимо оптимальным образом прокладывать траекторию 
движения маневрового состава от момента исполнения последней работы 
к~моменту начала новой работы.
{\looseness=-1

}
      
      
      Применение имитационных моделей для задач на железнодорожном 
транспорте является обычной\linebreak практикой. Чаще всего моделируют назначение\linebreak 
локомотивов для составов с~целью уменьшения общего числа используемых  
локомотивов~[9--12], передвижение пассажиров по станции с~целью увеличения 
пропускной способности станции~[13]. Схожими по смыслу являются 
Комплексная автоматизированная система ведения технологических процессов 
работы железнодорожных станций (АС ВТП) и~Автоматизированная система 
разработки и~мониторинга выполнения Единых технологических процессов 
работы железнодорожных путей необщего пользования и~станций примыкания 
(АС ЕТП). В~них заложены описания, модели всех технологических процессов, 
происходящих на станции. Разрабатываемая в~данной работе модель 
интенсивностей пересечения стрелочных переводов маневровыми локомотивами 
строится с~учетом компромисса между учетом всех технологических процессов, 
происходящих на станции, и~наименьшим объемом входных данных с~целью 
быстрой,\linebreak но близкой к~точной оценки искомых интенсивностей. Построенная 
система моделирования\linebreak также позволит учитывать случайные задержки 
в~прибытии и~отправлении поездов на станции, позволяя проанализировать, что 
произойдет на станции в~случае неисполнения расписания.
      
      Далее в~статье сформулированы принципы и~описана структура 
имитационной модели передвижения поездов на железнодорожной станции. 
Станция представляется в~виде неориентированного нагруженного графа, 
вершинами которого являются стрелочные переводы, а также места перехода 
кривых участков пути в~прямые и~места перехода прямых участков пути в~кривые, 
а также точки входа/выхода со станции. Ребрами являются железнодорожные 
пути, связывающие вершины. Каждому ребру сопоставляется число, равное 
расстоянию от одной вершины до другой на плоскости, т.\,е.\ длина пути. На 
станции предполагается возможным движение транзитных пассажирских поездов, 
пассажирских поездов местного формирования, грузовых поездов, а также 
маневровых составов. Пассажирские и~грузовые поезда следуют по расписанию, 
которое в~силу различных происшествий на железной дороге может исполняться 
не полностью, т.\,е.\ предусматривается возможность задержек поездов. 
В~заключительной части статьи приводятся примеры расчета интенсивности 
пересечения стрелочных переводов маневровыми составами, выполненные на 
основе построенной имитационной модели.

\vspace*{-8pt}

\section{Постановка задачи}

\vspace*{-2pt}

      Пусть имеется определяющий модель станции неориентированный граф 
$G\hm= \langle V, E\rangle$, где $V$~--- множество вершин (стрелочных 
переводов, стыков между рельсами и~точек входа и~выхода со станции (границ 
станции)); $E$~--- множество ребер (железнодорожных путей), соединяющих 
данные вершины. Также задана функция $D:\ E\to {\sf R}_+$, характеризующая 
длину ребра. Пусть $\vert E\vert \hm=m$. Пронумеровав ребра графа~$G$ от~1 
до~$m$, составим новый граф $G^\prime\hm= \langle V^\prime, E^\prime\rangle$, 
множеством вершин~$V^\prime$ которого являются номера ребер графа~$G$, 
т.\,е.\ $V^\prime\hm= \{1,2,\ldots , m\}$. Множество ребер~$E^\prime$ включает 
в~себя ребра между вершинами из~$V^\prime$, если эти вершины являются 
смежными ребрами в~графе~$G$. На элементах множества~$V^\prime$ введем 
функцию $D^\prime:\ V^\prime \hm \to {\sf R}_+$, характеризующую <<вес>> 
вершин в~графе~$G^\prime$, т.\,е.\ длину соответствующих ребер в~графе~$G$.
      
      Предположим, что расписание движения пассажирских поездов и~набор 
маневровых работ являются корректными в~том смысле, что некоторым 
назначением траекторий движения маневровых локомотивов можно исполнить 
суточный план работы станции без его нарушения. Пусть максимальная скорость 
передвижения по станции маневрового локомотива равна~$v_{\max}$; номер 
ребра графа~$G$, на котором заканчивается предыдущая маневровая работа, равен 
$j_0\hm\in V^\prime$; номер ребра графа~$G$, на котором начинается следующая 
маневровая работа, равен $j_T\hm\in V^\prime$, где~$T$~--- время окончания 
передвижения по станции маневрового локомотива, назначенного для 
осуществления маневровой работы. С~учетом данных параметров поставим 
задачу по отысканию маршрута передвижения по станции маневрового 
локомотива для выполнения очередной еще не исполненной маневровой работы 
в~плане на сутки\linebreak с~учетом ряда физических ограничений, описыва\-емых ниже, 
с~целью минимизации времени~$T$. Будем отсчитывать время от момента 
окончания последней маневровой работы. Пусть $u(t)$~--- номер ребра, которое 
проходит маневровый локомотив в~момент времени~$t$ от окончания последней 
маневровой работы ($u(t)\hm\in V^\prime$).
      
      Поскольку на станции осуществляется движение пассажирских поездов 
и~других маневровых составов, часть ребер закрыта для проезда. В~связи с~этим 
введем функцию $F:\ V^\prime\times {\sf R}_+^1\hm\to \{0,1\}$ вида
 $$
 F(j,t) \stackrel{\mathrm{def}}{=}
 \begin{cases}
 0\,, &\ \mbox{ребро\ (графа\ $G$)\ с\ номером\ $j$}\\
 &\ \mbox{свободно\ в\ момент\ времени\  $t$}\,;\\
 1 &\ \mbox{иначе},
 \end{cases}
 $$
которая характеризует занятость ребра для движения маневрового локомотива 
в~момент времени~$t$ от начала выполнения маневровой работы.

      Перебор всех возможных путей (называемых в~дальнейшем маршрутами) 
в~графе~$G^\prime$ из вершины~$j_0$ в~вершину~$j_T$ невозможен в~силу их 
бесконечного количества, связанного с~наличием циклов в~графе~$G$. Поэтому 
выберем несколько таких маршрутов, общим числом равных~$L$. Составим из 
этих последовательностей множество~$J$. Для минимизации времени~$T$ 
необходимо найти минимальное время, за которое можно пройти каждый 
маршрут (каждую последовательность вершин) из множества~$J$, а~затем 
выбрать среди найденных времен минимальное.
      
      Сформулируем задачу по минимизации времени прохождения маневрового 
состава через станцию по маршруту, задаваемому произвольной 
последовательностью из множества~$J$. Для этого отметим, что произвольный 
элемент~$\overline{J}_l$ множества~$J$ имеет вид:
$$\overline{J}_l\hm= \{ j_{0,l}, 
j_{1,l},\ldots , j_{k,l},\ldots , l_{K_l,l}\}\,,$$
где $j_{k,l}\hm\in V^\prime$, причем $j_{0,l}\hm=j_0$ и~$j_{K_l,l}\hm=j_T$, 
а $k\hm=\overline{1,K_l}$, $l\hm= \overline{1,L}$. 

%o
Пусть~$t_{k,l}$~--- момент перехода с~вершины с~номером~$j_{k-1,l}$   
на вершину с~номером~$j_{k,l}$ графа~$G^\prime$, а~$t_{0,l}\hm= 0$, 
$t_{K_l+1,l}\hm= T$. Тогда для каждого элемента~$\overline{J}_l$ множества~$J$ 
условие физической реализуемости можно записать в~виде:
$$t_{k+1,l}\hm- t_{k,l} \hm\geq \fr{D^\prime(j_{k,l})}{v_{\max}}\,, 
k\hm= \overline{0,K_l}\,,$$ 
т.\,е.\ исключить нереализуемую возможность проехать любое ребро 
графа~$G$ за бесконечно малое время.
      
Условие на движение только по свободным ребрам графа~$G$ 
записывается в~виде:
$$\forall j_{k,l} \forall t\hm\in [t_{k,l}, t_{k+1,l}]\quad 
F(j_{k,l},t)=0\,, k= \overline{0,K_l}\,,$$ 
которое эквивалентно ограничениям: 
$$\int\limits_{t_{k,l}}^{t_{k+1,l}} F(j_{k,l},t)\,dt \hm= 0\,, 
k\hm= \overline{0,K_l}\,.$$ 

%o
Также имеет место ограничение вида: 
$$t_{\mathrm{мин}}\hm\leq t_{K_l+1,l}\hm\leq t_{\mathrm{макс}}\,,$$ 
которое гарантирует, что очередная 
маневровая работа не начнется позже момента времени~$t_{\mathrm{макс}}$ 
и/или раньше~$t_{\mathrm{мин}}$. Величины~$t_{\mathrm{мин}}$ и 
$t_{\mathrm{макс}}$ определяются исходя из графика маневровых работ и~связаны 
с~тем, что маневровый локомотив должен подъехать к~месту осуществления работ 
в~определенный промежуток времени $[t_{\mathrm{мин}}, t_{\mathrm{макс}}]$, 
так как в~иное время очередная работа может быть пропущена или еще не 
начаться согласно графику работ.
      
      Таким образом, задача по минимизации времени прохождения маневрового 
состава через станцию по маршруту, задаваемому произвольной 
последовательностью~$\overline{J}_l$ из множества~$J$, имеет вид:
      \begin{equation}
      t_{K_l+1,i}\to \min\limits_{t_{k,l},\ k=\overline{1,K_l+1}}
      \label{e1-nau}
      \end{equation}
при ограничениях 
\begin{equation}
\left.
\begin{array}{c}
t_{k+1,l}-t_{k,l}\geq \fr{D^\prime (j_{k,l})}{v_{\max}}\,,\enskip 
k=\overline{0,K_l}\,;\\[6pt]
\displaystyle\int\limits_{t_{k,l}}^{t_{k+1,l}} F(j_{k,l},t)\,dt=0\,,\enskip k=\overline{0,K_l}\,;\\[6pt]
t_{\mathrm{мин}}\leq t_{K_l+1,j}\leq t_{\mathrm{макс}}\,,\enskip t_{0,l}=0\,.
\end{array}
\right\}
\label{e2-nau}
\end{equation}

\vspace*{-8pt}

\section{Сведение задачи о~выборе маршрута движения к~задаче 
смешанного целочисленного линейного программирования}

\vspace*{-2pt}

Заметим, что используемые выше функции 
$$H(j_{k,l},t) 
\stackrel{\mathrm{def}}{=} \int\limits_0^t F(j_{k,l},y)\,dy\,, 
k\hm= \overline{1,K_l}\,,$$
являются ку\-соч\-но-ли\-ней\-ны\-ми, поэтому ограничения~(2) 
в~задаче~(1) являются нелинейными, что делает поиск решения в~задаче~(1) при 
ограничениях~(2) весьма затруднительным. Но указанное свойство 
функций~$H(j_{k,l},t)$, $k\hm= \overline{0,K_l}$, позволяет путем введения 
целочисленных переменных свести исходную задачу нелинейного 
программирования к~задаче смешанного целочисленного линейного 
программирования. Для этого сформируем множество~${\sf T}_{k,l}$, состоящее 
из левой и~правой границ интервалов времени, когда ребро с~номером~$j_{k,l}$ 
оказывалось свободным для движения маневрового состава, $k\hm= 
\overline{0,K_l}$.
\pagebreak
      
      С помощью множества~${\sf T}_{k,l}$ можно определить <<окна>> (т.\,е.\ 
интервалы времени, в~которые ребро с~номером~$j_{k,l}$ свободно). Упорядочив 
элементы множества~${\sf T}_{k,l}$ по возрастанию, составим из них 
вектор~$\tau_{k,l}$. Пусть $\mathrm{dim}\,\tau_{k,l}\hm= 2I_{k,l}$, где  
$I_{k,l}$~--- число <<окон>>. Ввведем новые переменные~$\delta^i_{k,l}$, 
равные единице, если движение по ребру с~номером~$j_{k,l}$ осуществляется 
в~промежуток времени между~$\tau_{k,l}^{2i-1}$ и~$\tau_{k,l}^{2i}$, 
и~равные нулю, если движение по ребру с~номером~$j_{k,l}$ в~промежуток 
времени между~$\tau_{k,l}^{2i-1}$ и~$\tau_{k,l}^{2i}$ не 
осуществляется, $k\hm= \overline{0,K_l}$, $i\hm= \overline{1, I_{k,l}}$. С~учетом 
новых переменных~$\delta^i_{k,l}$ задача~(1) при ограничениях~(2) 
эквивалентным образом сводится к~следу\-ющей задаче:
\begin{equation}
t_{K_l+1,l}\to \min\limits_{t_{k,l},\ t_{K_l+1,l},\ \delta^i_{0,l},\ 
\delta^i_{k,l},\ k=\overline{1,K_l},\ i=\overline{1,I_{k,l}}}
      \label{e3-nau}
      \end{equation}
при ограничениях 
\begin{equation}
\left.
\begin{array}{l}
t_{k+1,l}-t_{k,l}\geq \fr{D^\prime(j_{k,l})}{v_{\max}}\,,\enskip 
k=\overline{0,K_l}\,;\\[6pt]
\displaystyle\sum\limits_{i=1}^{I_{k,l}} \delta^i_{k,l}=1\,,\enskip k=\overline{0,K_l}\,;\\[6pt]
t_{k+1,l}\leq \delta^i_{k,l} \tau^{2i}_{k,l}+\left( 1-\delta^i_{k,l}\right) 
t_{\mathrm{макс}}\,,\\[6pt]
\hspace*{31mm}k=\overline{0,K_l},\ i=\overline{1,I_{k,l}}\,;\\[6pt]
t_{k,l}\geq \delta^i_{k,l} \tau^{2i-1}_{k,l}\,,\enskip k=\overline{0,K_l}\,,\ 
i=\overline{1,I_{k,l}}\,;\\[6pt]
t_{\mathrm{мин}}\leq t_{K_l+1,j}\leq t_{\mathrm{макс}}\,,\ \delta^i_{k,l}\in 
\{0,1\}\,,\\[6pt]
\hspace*{18mm}i=\overline{1,I_{k,l}}\,,\ k=\overline{0,K_l}\,,\ t_{0,l}=0\,.
\end{array}
\right\}
\label{e4-nau}
\end{equation}
      
Поскольку задача~(1) при ограничениях~(2) может не иметь решения, то 
введем новую величину:
      $$
      T_l \stackrel{\mathrm{def}}{=} \begin{cases}
      T_l^*\,, &\mbox{решение\ задачи\ (1)}\\
     & \mbox{при\ ограничениях\ (2)\ 
существует}\,;\\
      +\infty &\mbox{иначе},
      \end{cases}
      $$
$l\hm=\overline{1,L}$. Здесь~$T_l^*$~--- оптимальное значение критерия 
в~задаче~(3)--(4). С~использованием величин~$T^*_l$ заключаем, что можно 
оценить величину~$T^*$ сверху как $T^*\hm\leq \min\nolimits_{l=\overline{1,L}} 
T_l^*$, где $T^*$~--- минимальное время, за которое можно добраться из 
вершины~$j_0$ в~вершину~$j_T$ в~графе~$G^\prime$, не нарушая и~график 
маневровых работ, и~график движения пассажирских поездов.
      
      Таким образом, решение задачи, сформулированной выше, дает 
возможность оценить сверху время передвижения маневрового локомотива по 
станции до исполнения очередной маневровой работы.
      
      После передвижения по станции к~месту выполнения маневровой работы 
      и~ее исполнения маневровому локомотиву назначается любая маневровая работа из 
перечня оставшихся маневровых работ, такая что она может быть выполнена 
вовремя указанным маневровым локомотивом.
      
\vspace*{-8pt}

\section{Применение полученных результатов для~оценки интенсивности 
пересечения стрелочного перевода маневровым составом} %4
\vspace*{-2pt}

      Наряду с~максимальным числом пассажирских/грузовых поездов, 
пересекающих станцию, которое характеризует доход от функционирования 
станции, важной характеристикой является вероятность хотя бы одного бокового 
столкновения между пассажирскими и~грузовыми поездами и~маневровыми 
составами, которая характризует степень безопасности движения. Для увеличения 
максимального числа поездов, пересекающих станцию, вначале необходимо 
рассчитать вероятность хотя бы одного бокового столкновения, например за 
сутки. Для этого необходимо рассчитать вероятность столкновения произвольного 
пас\-са\-жир\-ско\-го/гру\-зо\-во\-го поезда с~маневровым составом на 
произвольной стрелке при пересечении пассажирским/грузовым поездом станции. 

%o
В~[7] была предложена следующая оценка вероятности бокового столкновения, 
получаемая на основе предположения о~том, что поток пересечений маневровыми 
составами стрелок~--- пуассоновский:
\vspace*{-6pt}

\noindent
\begin{multline}
\hspace*{-6pt}\p(A)=\left( \lambda_{\mathrm{м}} \left( \p_{\mathrm{м}} +\p_{\mathrm{м}} 
\p_{\mathrm{п/г}}+\p_{\mathrm{п/г}}\right) \left( 
\fr{l_{\mathrm{п/г}}}{v_{\mathrm{п/г}}}+\fr{l_{\mathrm{м}}}{v_{\mathrm{м}}} \right)+{}
\right.\\
\left.{}+ \lambda_{\mathrm{с}} \p_{\mathrm{п/г}} \tau_{\mathrm{с}} 
+\lambda_{\mathrm{м}} \p_{\mathrm{м}} \p_{\mathrm{пс/гс}} 
\tau_{\mathrm{пс/гс}}
\vphantom{\left(\fr{l_{\mathrm{п/г}}}{v_{\mathrm{п/г}}}+
\fr{\lambda_{\mathrm{м}}}{v_{\mathrm{м}}} \right)}
\right) \kappa_{\mathrm{с}}\,,\\
\lambda_{\mathrm{м}}=\displaystyle\fr{1}{d}\sum\limits^Q_{i=1} \fr{N_i}{N}\,.
\label{e5-nau}
\end{multline}
      
      \setcounter{figure}{1}
\begin{figure*}[b] %fig2
\vspace*{1pt}
 \begin{center}
 \mbox{%
 \epsfxsize=162.278mm 
 \epsfbox{bos-2.eps}
 }
 \end{center}
\vspace*{-9pt}
\Caption{Схема станции}
\end{figure*}

\noindent
Здесь $\lambda_{\mathrm{м}}$~--- интенсивность пересечения стрелки маневровыми 
составами в~направлении, при котором\linebreak возможно боковое столкновение между 
пассажирским/грузовым поездом и~маневровым составом\linebreak (рис.~1) [1/ч];  
$Q$~--- число маневровых локомотивов на станции; $N$~--- число стрелок на 
станции; $N_i$~--- число стрелок, пересекаемых в~час $i$-м маневровым 
составом; $d$~--- общее число на\-прав\-ле\-ний движения по стрелке маневрового\linebreak 
состава (на рис.~1 $d\hm=4$); $l_{\mathrm{п/г}}$~--- средняя длина\linebreak  
пасса\-жир\-ско\-го/гру\-зо\-во\-го поезда [км]; $l_{\mathrm{м}}$~--- средняя 
длина маневрового состава [км]; $v_{\mathrm{п/г}}$~--- сред-\linebreak няя скорость 
пас\-са\-жир\-ско\-го/гру\-зо\-во\-го поезда\linebreak\vspace*{-12pt}

{ \begin{center}  %fig1
 \vspace*{1pt}
  \mbox{%
 \epsfxsize=64.436mm 
 \epsfbox{bos-1.eps}
 }

\end{center}

\noindent
{{\figurename~1}\ \ \small{Направление следования (отмечено пунктиром) маневрового состава через 
стрелочный перевод (отмечен кружком), при котором возможно столкновение с~пассажирским 
составом}}
}

\vspace*{9pt}

\noindent
 [км/ч];
 $v_{\mathrm{м}}$~--- средняя ско\-рость 
маневрового со\-ста\-ва [км/ч]; $\tau_{\mathrm{с}}$~--- среднее время нахождения 
маневрового со\-ста\-ва на стрелочном переводе
при условии остановки на нем [ч]; 
$\tau_{\mathrm{пс/гс}}$~--- среднее время стоянки
пас\-са\-жир\-ско\-го/гру\-зо\-во\-го поезда
 на стрелочном переводе [ч]; 
$\p_{\mathrm{пс/гс}}$~--- вероятность остановки  
пас\-са\-жир\-ско\-го/гру\-зо\-во\-го поезда\linebreak на стрелочном переводе; 
$\p_{\mathrm{п/г}}$~--- вероятность\linebreak проезда на запрещающий сигнал %\linebreak 
светофора 
пас\-са\-жир\-ско\-го/гру\-зо\-во\-го поезда; $\p_{\mathrm{м}}$~--- вероятность 
проезда на запрещающий сигнал светофора маневрового состава; 
$\kappa_{\mathrm{с}}$~--- коэффициент, характеризующий неизолированность 
стрелочного перевода (возможность бокового столкновения), принимающий 
значение~1, если стрелочный перевод неизолированный, и~0, если 
изолированный.

Вероятности $\p_{\mathrm{пс/гс}}$, $\p_{\mathrm{п/г}}$ и~$\p_{\mathrm{м}}$ 
задаются согласно~\cite{7-nau, 14-nau}. Как видно из формулы~(\ref{e5-nau}), 
интенсивность~$\lambda_{\mathrm{м}}$ (при одинаковых~$d$) получается одной 
и~той же для стрелок и~с~интенсивным, и~с~неинтенсивным движением по ним 
маневровых составов. При этом сама величина~$\lambda_{\mathrm{м}}$ 
существенно влияет на вероятность $\p(A)$. Поэтому на основе составленного 
расписания движения маневрового локомотива по станции предложим иной 
способ вычисления интенсивностей~$\lambda_{\mathrm{м}}$.
     
     Наличие предложенной имитационной модели позволяет использовать для 
оценки интенсивности пересечения стрелочных переводов маневровыми 
локомотивами классическую оценку в~виде частоты. Пусть $w$~--- число 
пересечений маневровыми локомотивами стрелочного перевода в~направлении, 
при котором возможно боковое столкновение между пассажирским/грузовым 
поездом и~маневровым составом, в~течение суток. Тогда 
$\lambda_{\mathrm{м}}\hm= w/24$.
     
\vspace*{-8pt}

\section{Результаты численного эксперимента}

\vspace*{-2pt}

Рассмотрим некоторую станцию, схема части которой отображена на рис.~2.

 Общее число стрелок на данной станции равно~149. Пусть данную 
станцию в~некоторые сутки пересекают~84~пассажирских поезда (включая 
поездные локомотивы), общее число маневровых работ равно~76, а~максимальная 
скорость~$v_{\mathrm{м}}$ передвижения маневровых локомотивов по станции 
равна~10,8~км/ч. Очевидно, что чем больше маршрутов передвижения 
маневрового локомотива по станции, найденных по графу~$G^\prime$, к~месту 
выполнения очередной маневровой работы исследуется, тем большая доля 
маневровых работ может быть выполнен, а время передвижения по станции 
маневрового локомотива уменьшится. Однако в~этом случае также увеличится 
суммарное время, необходимое для нахождения решения задач~(\ref{e3-nau}) при 
ограничениях~(\ref{e4-nau}). 

%o
Проанализируем время счета и~доля выполненных 
маневровых работ в~зависимости от различного числа маршрутов передвижения 
маневрового локомотива по станции к~месту выполнения очередной маневровой 
работы.
      
\begin{table*}\small %tabl1
 \begin{center}
 \parbox{320pt}{\Caption{Время передвижения по станции для осуществления $i$-й маневровой 
работы [мин]\,/\,время поиска решения в~задачах~(3) при ограничениях~(4) [мин]}

}

\vspace*{2ex}
       
\begin{tabular}{|c|c|c|c|c|c|c|c|c|}
\hline
&\multicolumn{8}{c|}{$i$}\\
\cline{2-9}
\raisebox{6pt}[0pt][0pt]{$L$}&$\ldots$&14&15&16&17&18&19&$\ldots$\\
\hline
1&$\cdots$&32/0,003&25/0,01&4/0,002&3/0,005&6/0,009&6/0,01&$\cdots$\\
$\vdots$&$\vdots$&$\vdots$&$\vdots$&$\vdots$&$\vdots$&$\vdots$&$\vdots$&$\vdots$\\
7&$\cdots$&30/0,02&23/0,1&2/0,03&2/0,04&4/0,17&5/0,11&$\cdots$\\
\hline
\end{tabular}
\end{center}
\end{table*}
      
Расчеты проводились на персональном компьютере со следующими характеристиками: 
процессор Intel Core i7 2,8 ГГц, память 8~ГБ 1\,600~МГц DDR3.
      
Как следует из табл.~1 и~2, время счета ожидаемо растет при увеличении 
числа~$L$, однако рас\-тет и~доля выполненных маневровых работ. При этом 
для выполнения~100\% маневровых работ достаточно рассмотреть небольшое 
число маршрутов\linebreak
\vspace*{-12pt}

%\begin{table*}
{ %tabl2
      
 \noindent
{{\tablename~2}\ \ \small{Общее время счета и~доля выполненных маневровых работ}}

 \tabcolsep=13pt
 \begin{center}
\small
\begin{tabular}{|c|c|c|}
\hline
$L$&\tabcolsep=0pt\begin{tabular}{c}Общее время\\ счета, ч\end{tabular}&
\tabcolsep=0pt\begin{tabular}{c}Доля выполненных\\ 
маневровых работ, \%\end{tabular}\\
\hline
1&0,26&73\hphantom{,9}\\
3&2,71&83,8\\
5&2,95&97,3\\
7&3,03&100\hphantom{,99}\\
\hline
\end{tabular}
\end{center}
\vspace*{3pt}
}
%\end{table*}

\setcounter{figure}{2}

 {\begin{center}  %fig1
% \vspace*{2pt}
  \mbox{%
 \epsfxsize=64.436mm 
 \epsfbox{bos-3.eps}
 }

\end{center}

\noindent
{{\figurename~3}\ \ \small{Интенсивности пересечения стрелочного 
перевода №\,178 по формуле~(\ref{e5-nau}) по 
данным из~\cite{7-nau}\,/\,по предлагаемой имитационной модели}}
\vspace*{9pt}
}

\noindent 
пересечения маневровым локомотивом станции до места 
исполнения очередной маневровой работы.
      
Теперь отдельно рассмотрим стрелочный перевод №\,178 и~приведем число 
пересечений маневровыми локомотивами данного перевода в~различных 
направлениях.

  Как следует из рис.~3, интенсивности пересечения стрелочного перевода 
№\,178, полученные по предлагаемой имитационной модели и~по 
формуле~(\ref{e5-nau}), отличаются. Это связано с~тем, что интенсивности, 
полученные по формуле~(\ref{e5-nau}), назначаются одинаковыми каждому 
стрелочному переводу, а реальная загруженность того или иного стрелочного 
перевода (удаленность или близость конкретного стрелочного перевода от места 
проведения большинства маневровых работ) в~расчет не принимается.

\section{Заключение}
\vspace*{-2pt}

 В работе рассмотрена задача по составлению расписания движения по 
станции маневровых локомотивов с~учетом различных технических ограничений, 
суточного расписания движения поездов, а также набора маневровых работ. 

%o
Сформулирована задача нелинейного программирования, которая сведена к~задаче 
смешанного целочисленного линейного программирования. 

%o
На основе 
построенной имитационной модели получена модифицированная оценка 
интен\-сивности пересечения стрелочных переводов\linebreak в~различных направлениях 
маневровыми локомотивами, которая в~дальнейшем используется для уточнения 
оценки вероятности бокового столкновения между маневровым локомотивом 
и~пассажирским/грузовым поездом на стрелочном пере\-воде.

\vspace*{-8pt}
      
{\small\frenchspacing
 {%\baselineskip=10.8pt
 \addcontentsline{toc}{section}{References}
 \begin{thebibliography}{99}
\vspace*{-2pt}

\bibitem{1-nau}
ГОСТ 33433-2015. Безопасность функциональная. Управление рисками на железнодорожном 
транспорте.~--- М.: Стандартинформ, 2016. 34~c.

\bibitem{4-nau} %2
\Au{Шубинский И.\,Б., Проневич~О.\,Б., Данилова~А.\,Д.} Особенности оценки 
вероятности возникновения пожаров на тепловозах различных серий~// 
Надежность, 2016. T.~16. №\,4. С.~24--29. 
doi: 10.21683/1729-2646-2016-16-4-24-29.

\bibitem{5-nau} %3
\Au{Крутиков А.\,М.} Оценка надежности рельсов Р65 по ресурсу: экспериментальные 
исследования.~--- М.: Финансы и~статистика, 2016. 151~c.

\bibitem{3-nau} %4
\Au{Замышляев А.\,М., Игнатов~А.\,Н., Кибзун~А.\,И., Новожилов~Е.\,О.} 
Функциональная зависимость между количеством вагонов в~сходе из-за 
неисправностей вагонов или пути и~факторами движения~// 
Надежность, 2018. T.~18. №\,1. С.~53--60. 
doi: 10.21683/1729-2646-2018-18-1-53-60.

\bibitem{2-nau} %5
\Au{Кибзун А.\,И., Игнатов А.\,Н.} О~задаче распределения инвестиций 
в~установку средств, предотвращающих несанкционированный проезд 
автотранспортом железнодорожных переездов, для различных статистических 
критериев~// Надежность, 2018. T.~18. №\,2. С.~31--37. 
doi: 10.21683/1729-2646-2018-18-2-31-37.

\bibitem{6-nau}
\Au{Bagheri M., Saccomanno~F., Chenouri~S., Fu~L.} Reducing the threat 
of in-transit derailments involving dangerous goods through effective 
placement along the train consist~// Accident Anal. Prev., 2011. 
Vol.~43. Iss.~4. P.~613--620. doi: 10.1016/j.aap.2010.09.008.

\bibitem{7-nau}
\Au{Игнатов А.\,Н., Кибзун~А.\,И., Платонов~Е.\,Н.} Оценка вероятности 
столкновения железнодорожных составов на железнодорожных станциях на 
основе пуассоновской модели~// Автоматика и~телемеханика, 2016. №\,11. С.~43--59. 
doi: 10.1134/S0005117916110035.

\bibitem{8-nau}
\Au{Шубинский И.\,Б., Замышляев~А.\,М., Игнатов~А.\,Н., Кан~Ю.\,С., 
Кибзун~А.\,И., Платонов~Е.\,Н.} Оценка рисков, связанных с~проездом 
запрещающего сигнала светофора маневровым составом или пассажирским 
поездом~// Надежность, 2016. T.~16. №\,3. С.~39--46. 
doi: 10.21683/1729-2646-2016-16-3-39-46.

\bibitem{9-nau}
\Au{Иванов С.\,В., Кибзун~А.\,И., Осокин~А.\,В.} Оптимизационная 
стохастическая модель назначения локомотивов для перевозки грузовых составов~// 
Автоматика и~телемеханика, 2016. №\,11. С.~80--95. 
doi: 10.1134/S0005117916110059.

\bibitem{11-nau} %10
\Au{Лазарев А.\,А., Мусатова~Е.\,Г., Тарасов~И.\,А.} Решение задачи планирования 
двухстороннего движения на однопутном участке железной дороги с~разъездом~// 
Автоматика и~телемеханика, 2016. №\,11. С.~158--174. 
doi: 10.1134/S0005117916110047.

\bibitem{10-nau} %11
\Au{Гайнанов Д.\,Н., Рассказова~В.\,А.} Математическое моделирование в~задаче 
оптимального назначения и~перемещения локомотивов методами теории графов 
и~комбинаторной оптимизации~// Труды МАИ, 2017. №\,92. 24~с.

\bibitem{12-nau}
\Au{Буянов М.\,В., Иванов~С.\,В., Кибзун~А.\,И., Наумов~А.\,В.} Развитие 
математической модели управления грузоперевозками на участке железнодорожной 
сети с~учетом случайных факторов~// Информатика и~её применения, 2017. T.~11. 
Вып.~4. С.~85--93. doi: 10.14357/19922264170411.

\bibitem{13-nau}
\Au{Искаков Т.\,А.} Имитационное моделирование функционирования транспортного узла~// 
Интеллекту\-аль\-ные системы управления на железнодорожном\linebreak транспорте. 
Компьютерное 
и~математическое моделирование: Труды V~научно-технич. конф. 
с~международным участием.~--- М.: НИИАС, 2016. С.~221--225.
\bibitem{14-nau}
\Au{Шубинский И.\,Б.} Функциональная надежность информационных систем. Методы 
анализа.~--- Ульяновск: Печатный двор, 2012. 296~c.
 \end{thebibliography}

 }
 }

\end{multicols}

\vspace*{-4pt}

\hfill{\small\textit{Поступила в~редакцию 01.02.18}}

\vspace*{10pt}

%\newpage

%\vspace*{-24pt}

\hrule

\vspace*{2pt}

\hrule

\vspace*{-2pt}


\def\tit{MODEL OF TRANSPORTATION OF~TRAINS AND~SHUNTING LOCOMOTIVES 
AT~A~RAILWAY STATION FOR~EVALUATION AND~ANALYSIS OF~SIDE-COLLISION PROBABILITY}

\def\titkol{Model of transportation of~trains and~shunting locomotives 
at~a~railway station for~evaluation
of~side-collision probability}

\def\aut{A.\,V.~Bosov$^1$, A.\,N.~Ignatov$^2$, and~A.\,V.~Naumov$^2$}

\def\autkol{A.\,V.~Bosov, A.\,N.~Ignatov, and~A.\,V.~Naumov}

\titel{\tit}{\aut}{\autkol}{\titkol}

\vspace*{-11pt}

\noindent
$^1$Institute of Informatics Problems, Federal Research Center ``Computer Science and Control'' of the 
Russian\linebreak
$\hphantom{^1}$Academy of Sciences, 44-2~Vavilov Str., Moscow 119333, Russian Federation

\noindent
$^2$Moscow Aviation Institute (National Research University), 4~Volokolamskoe Shosse, Moscow 
125993, Russian\linebreak
$\hphantom{^1}$Federation


\def\leftfootline{\small{\textbf{\thepage}
\hfill INFORMATIKA I EE PRIMENENIYA~--- INFORMATICS AND
APPLICATIONS\ \ \ 2018\ \ \ volume~12\ \ \ issue\ 3}
}%
 \def\rightfootline{\small{INFORMATIKA I EE PRIMENENIYA~---
INFORMATICS AND APPLICATIONS\ \ \ 2018\ \ \ volume~12\ \ \ issue\ 3
\hfill \textbf{\thepage}}}

\vspace*{3pt}



\Abste{A mathematical model for solution of the shunting locomotives 
traffic control problem is proposed for a~fixed schedule of passenger/freight 
trains traffic across a~station and a~fixed time-table of shunting operations: 
set off and attaching of cars, output operation and deconsolidation of trains. 
The model is used for formulation and solution of the problem to minimize 
time of shunting locomotive transportation across the station to perform 
next scheduled operation with respect to busy condition of some tracks for 
transportation owing to presence of passenger/freight trains on them and 
with respect to restriction on shunting operation execution time. The 
original statement is reduced to mixed integer linear programming. 
The presented model was used for evaluation of side-collision probability 
at a~railway station with respect to possible random drag in passenger trains 
traffic.  The results of numerical experiments are presented.}

\KWE{simulation model; schedule; intensity; mixed integer linear programming}
      
      
\DOI{10.14357/19922264180315} %

%\vspace*{-14pt}

\Ack
\noindent
This work was supported by the Russian Science Foundation (grant No.\,16-11-00062).

\pagebreak


  \begin{multicols}{2}

\renewcommand{\bibname}{\protect\rmfamily References}
%\renewcommand{\bibname}{\large\protect\rm References}

{\small\frenchspacing
 {%\baselineskip=10.8pt
 \addcontentsline{toc}{section}{References}
 \begin{thebibliography}{99}
\bibitem{1-nau-1}
GOST 33433-2015. 2016. \textit{Bezopasnost' funktsional'naya. Upravlenie riskami na 
zheleznodorozhnom transporte} [Functional safety. Risk control in railroad transport]. 
Moscow: Standartinform. 34~p.

\bibitem{4-nau-1} %2
\Aue{Shubinsky, I.\,B., O.\,B.~Pronevich, and A.\,D.~Danilova.} 2016. Osobennosti 
otsenki veroyatnosti vozniknoveniya pozharov na teplovozakh razlichnykh seriy 
[Special aspects of estimating the probability of fire occurrence on diesel 
locomotives of various types]. \textit{Dependability} 16(4):24--29.
doi: 10.21683/1729-2646-2016-16-4-24-29.

\bibitem{5-nau-1} %3
\Aue{Krutikov, A.\,M.} 2016. \textit{Otsenka nadezhnosti rel'sov R65 po resursu: 
eksperimental'nye issledovaniya} [Evaluation of the dependability of rails R65 on the 
resource: Experimental researches]. Moscow: Finansy i~statistika. 151~p.

\bibitem{3-nau-1} %4
\Aue{Zamyshliaev, A.\,M., A.\,N.~Ignatov, A.\,I.~Kibzun, and E.\,O.~Novozhilov.} 
2018. Funktsional'naya zavisimost' mezhdu kolichestvom vagonov v~skhode iz-za 
neispravnostey vagonov ili puti i~faktorami dvizheniya [Functional dependency 
between the number of wagons derailed due to wagon or track defects and the traffic 
factors]. \textit{Dependability} 18(1):53--60.
doi: 10.21683/1729-2646-2018-18-1-53-60.

\bibitem{2-nau-1} %5
\Aue{Kibzun, A.\,I., and A.\,N.~Ignatov.} 2018. O~zadache raspredeleniya investitsiy 
v~ustanovku sredstv, pred\-ot\-vra\-shcha\-yushchikh nesanktsionirovannyy proezd 
avtotransportom zheleznodorozhnykh pereezdov, dlya razlichnykh statisticheskikh 
kriteriev [On the task of allocating investment to facilities preventing 
unauthorized movement of road vehicles across level crossings for various statistical 
criteria]. \textit{Dependability} 18(2):31--37.
doi: 10.21683/1729-2646-2018-18-2-31-37.

\bibitem{6-nau-1}
\Aue{Bagheri, M., F.~Saccomanno, S.~Chenouri, and L.~Fu.} 2011. Reducing the 
threat of in-transit derailments involving dangerous goods through effective placement 
along the train consist. \textit{Accident Anal. Prevent.} 43(3):613--620. 
doi: 10.1016/j.aap.2010.09.008.

\bibitem{7-nau-1}
\Aue{Ignatov, A.\,N., A.\,I.~Kibzun, and E.\,N.~Platonov.} 2016. Estimating collision 
probabilities for trains on railroad stations based on a~Poisson model. \textit{Automat.  
Rem. Contr.} 77(11):1914--1927.

\bibitem{8-nau-1}
\Aue{Shubinsky, I.\,B., A.\,M.~Zamyshlyaev, A.\,N.~Ignatov, Yu.\,S.~Kan, 
A.\,I.~Kibzun, and E.\,N.~Platonov}. 2016. Otsenka riskov, svyazannykh s~proezdom 
zapreshchayushchego signala svetofora, manevrovym sostavom ili passazhirskim 
poezdom [Estimation of risks related to stop signal passed by shunting loco 
or passenger train]. \textit{Dependability} 16(3):39--46.
doi: 10.21683/1729-2646-2016-16-3-39-46.

\bibitem{9-nau-1}
\Aue{Ivanov, S.\,V., A.\,I.~Kibzun, and A.\,V.~Osokin.} 2016. Stochastic optimization 
model of locomotive assignment to freight trains. \textit{Automat. Rem. Contr.} 
77(11):1944--1956.

\bibitem{11-nau-1} %10
\Aue{Lazarev, A.\,A., E.\,G.~Musatova, and I.\,A.~Tarasov.} 2016. Two-directional 
traffic scheduling problem solution for a~single-track railway with siding. \textit{Automat.  
Rem. Contr.} 77(12):2118--2131.

\bibitem{10-nau-1} %11
\Aue{Gainanov, D.\,N., and V.\,A.~Rasskazova.} 2017. Ma\-te\-ma\-ti\-che\-skoe 
modelirovanie v~zadache optimal'nogo na\-zna\-che\-niya i~peremeshcheniya lokomotivov 
metodami teorii grafov i~kombinatornoy optimizatsii [Mathematical modelling of 
locomotives' traffic problem by graph theory and combinatorial optimization methods]. 
\textit{Trudy MAI} 92. 24~p.

\bibitem{12-nau-1}
\Aue{Buyanov, M.\,V., S.\,V.~Ivanov, A.\,I.~Kibzun, and A.\,V.~Naumov.} 2017. 
Razvitie matematicheskoy modeli upravleniya gruzoperevozkami na uchastke 
zheleznodorozhnoy seti s~uchetom sluchaynykh faktorov [Development of the 
mathematical model of cargo transportation control on a railway network segment 
taking into account random factor]. \textit{Informatika i~ee Primeneniya~--- Inform. 
Appl.} 11(4):85--93. doi: 10.14357/19922264170411.

\bibitem{13-nau-1}
\Aue{Iskakov, T.\,A.} 2016. Imitatsionnoe modelirovanie funktsi\-oni\-ro\-va\-niya 
transportnogo uzla [Simulation modelling of functioning transport hub]. \textit{Trudy 
V nauchno-tekhnich. konf. s~mezhdunarodnym uchastiem ``Intellektual'nye sistemy 
upravleniya na zheleznodorozhnom transporte. Komp'yuternoe i~matematicheskoe 
modelirovanie''} [5th Science and Technological Conference 
(with International Participation) ``Intellectual Control Systems in Railroad Transport. Computer and 
Mathematical Modeling'' Proceedings]. Moscow. 221--225.
\bibitem{14-nau-1}
\Aue{Shubinskiy, I.\,B.} 2012. \textit{Funktsional'naya nadezhnost' informatsionnykh sistem. 
Metody analiza} [International functional dependability of information 
systems. Methods of analysis]. Ul'yanovsk: Pechatnyy dvor. 296~p.
\end{thebibliography}

 }
 }

\end{multicols}

\vspace*{-6pt}

\hfill{\small\textit{Received February 1, 2018}}

%\pagebreak

\vspace*{-12pt}

\Contr

\noindent
\textbf{Bosov Alexey V.} (b.\ 1969)~--- Doctor of Science in technology, principal scientist, Institute 
of Informatics Problems, Federal Research Center ``Computer Science and Control'' of the Russian 
Academy of Sciences, 44-2~Vavilov Str., Moscow 119333, Russian Federation; 
\mbox{AVBosov@ipiran.ru}

\vspace*{3pt}

\noindent
\textbf{Ignatov Alexey N.} (b.\ 1991)~--- Candidate of Science (PhD) in physics and mathematics, 
senior lecturer, Moscow Aviation Institute (National Research University), 4~Volokolamskoe Shosse, 
Moscow 125993, Russian Federation; \mbox{alexei.ignatov1@gmail.com}

\vspace*{3pt}

\noindent
\textbf{Naumov Andrey V.} (b.\ 1966)~--- Doctor of Science in physics and mathematics, associate 
professor, Department ``Probability Theory and Computational Modeling,'' Moscow Aviation Institute 
(National Research University), 4~Volokolamskoe Shosse, Moscow 125993, Russian Federation; 
\mbox{naumovav@mail.ru}

\label{end\stat}

\renewcommand{\bibname}{\protect\rm Литература}       