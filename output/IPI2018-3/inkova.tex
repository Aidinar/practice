\renewcommand{\figurename}{\protect\bf Figure}
\renewcommand{\tablename}{\protect\bf Table}

\def\stat{inkova}


\def\tit{STATISTICAL ANALYSIS OF~LANGUAGE SPECIFICITY 
OF~CONNECTIVES BASED ON~PARALLEL TEXTS}

\def\titkol{Statistical analysis of~language specificity 
of~connectives based on~parallel texts}

\def\autkol{O.\,Yu.~Inkova and M.\,G.~Kruzhkov}

\def\aut{O.\,Yu.~Inkova$^1$ and M.\,G.~Kruzhkov$^1$}

\titel{\tit}{\aut}{\autkol}{\titkol}

%{\renewcommand{\thefootnote}{\fnsymbol{footnote}}
%\footnotetext[1] {The 
%research of Yuri Kabanov was done under partial financial support   of the grant 
%of  RSF No.\,14-49-00079.}}

\renewcommand{\thefootnote}{\arabic{footnote}}
\footnotetext[1]{Institute of Informatics Problems, Federal Research Center ``Computer Science and Control''
 of the Russian 
Academy of Sciences, 44-2 Vavilov Str., Moscow 119333, Russian Federation}
%; \mbox{olyainkova@yandex.ru}}
%\footnotetext[2]{Institute of Informatics Problems, Federal Research Center "Computer Science and Control" of the Russian 
%Academy of Sciences, 44-2 Vavilov Str., Moscow 119333, Russian Federation; \mbox{magnit75@yandex.ru}}


\index{Kruzhkov M.\,G.}
\index{Inkova О.\,Yu.}
\index{Кружков М.\,Г.}
\index{Инькова О.\,Ю.}

\def\leftfootline{\small{\textbf{\thepage}
\hfill INFORMATIKA I EE PRIMENENIYA~--- INFORMATICS AND
APPLICATIONS\ \ \ 2018\ \ \ volume~12\ \ \ issue\ 3}
}%
 \def\rightfootline{\small{INFORMATIKA I EE PRIMENENIYA~---
INFORMATICS AND APPLICATIONS\ \ \ 2018\ \ \ volume~12\ \ \ issue\ 3
\hfill \textbf{\thepage}}}

  
  \Abste{In recent decades, problems of language specificity in the Russian language 
attract considerable attention of researchers, although until recently, they have not 
been thoroughly examined using corpus-based methods. This paper presents a~new 
method of investigating language specificity of Russian connectives based on 
statistical analysis of annotated parallel texts. Russian--French and French--Russian 
parallel texts are processed with the help of the Supracorpora Database (SCDB) of 
Connectives designed specifically for annotation of translation correspondences 
(TCs) found in parallel texts. Each TC includes annotations of a~Russian connective 
and its translation equivalent (TE), which enables one to obtain statistical data on 
various translation models (TMs) based on several proposed parameters of language 
specificity of connectives. As an example, in this work, language specificity of two 
Russian connectives will be examined: \textit{или} and \textit{а~то}. Based on the 
proposed statistical parameters, it will be demonstrated that \textit{или} has a~very 
low degree of language specificity in the context of the Russian--French language 
pair, while \textit{а~то} is a~highly language-specific connective. The results of this 
research are applicable to informatics (machine translation and statistical analysis of 
textual data) and comparative study of languages, such as lexical typology, 
lexicography, and theory and practice of translation.}
  
  \KWE{supracorpora databases; statistical analysis; contrastive corpus analysis; 
language specificity; parallel corpora; linguistic information resources; connectives; 
discourse relations; semantics}

\DOI{10.14357/19922264180312} 


%\vspace*{-4pt}


\vskip 12pt plus 9pt minus 6pt

      \thispagestyle{myheadings}

      \begin{multicols}{2}

                  \label{st\stat}
  
  \section{Introduction}
  
  \noindent
  Language-specific phenomena (the term introduced in~[1]) are examined in 
Russian linguistics since 1990s (see, e.\,g.,~[2, 3]). According to~[4],  
a~language-specific lexical unit ``includes \textbf{a~unique conceptual 
configuration}, meaning that all of its existing translation equivalents either lack 
a~certain semantic component, or include an irremovable superfluous component.'' 
Until recently, language-specific phenomena had been studied based almost 
exclusively on comparative semantic analysis. By now, semantic analysis has been 
supplemented by quantitative statistical analysis of annotated parallel texts. 
  
  The goal of this paper is to describe new approaches to identifying the degree of 
language specificity of connectives based on data from parallel corpora. The 
language specificity of Russian connectives is examined in comparison with French 
TEs that appear in parallel texts. 
  
  The SCDB of Connectives is a~new information resource that contains aligned 
texts of parallel corpora from the Russian National Corpus (RNC) and makes it 
possible to create and annotate Russian connectives together with their 
TEs in parallel texts. Such annotated pairs are called translation 
correspondences. Annotations of connectives and their TEs include relevant 
context properties of Russian connectives and their parallels, such as type of relation, 
position, structural features, etc. The concept of the SCDBs and their use are 
described in~[5, 6], and the annotation system and statistical features~--- in~[7]. 
  
  The quantitative analysis requires adequate interpretation and should be 
accompanied by semantic and functional analysis. This analysis makes it possible to 
determine the type and degree of language specificity of various Russian connectives, 
as will be demonstrated by the example of~210~annotations of the Russian 
connective \textit{или} with rather obvious semantics and~210~annotations of the 
polysemic Russian connective \textit{а~то}.

\vspace*{-4pt}
  
  \section{Data and Information Resources}
  
\vspace*{-2pt}
  
  \noindent
  The primary source of data for this research are parallel texts of Russian--French 
corpus of the RNC. These texts include morphological annotation for individual 
words (such as part of speech, tense, number, etc.)\ and they are aligned to mark 
correspondences between large text segments of Russian and French texts ranging 
from one to several sentences. An example fragment of an aligned Russian--French 
parallel text is presented in\linebreak
 Table~1.
  
  \begin{table*}\small %tabl1
  \begin{center}
  \Caption{Example fragment of an aligned Russian--French parallel text$^*$}
  \vspace*{2ex}
  
  \begin{tabular}{p{50mm}p{50mm}p{50mm}}
  \hline
\multicolumn{1}{c}{Original Russian Text}&\multicolumn{1}{c}{ [English 
translation]}&\multicolumn{1}{c}{French Translation Text}\\
\hline
--- Ну что ты, Алеша$\ldots$ лежи, молчи$\ldots$&[--- Come, come, Alyosha$\ldots$ lie down, 
don't talk$\ldots$]&--- Voyons, Aliocha$\ldots$ Allons, reste couch$\acute{\mbox{e}}$ 
et tais-toi$\ldots$\\
\hline
Пальто-то этой дамы у нас пока будет?&[Should we keep the coat of this lady for now?]&Le manteau de 
cette dame, on le garde ici, pour l'instant?\\
\hline
--- Да, да. Чтобы Николка не вздумал тащить его.&[--- Yes, yes. Don't let Nikolka even dream of 
bringing it back.]&--- Oui, oui. Et que Nikolka ne s'avise pas de le porter.\\
\hline
А то на улице$\ldots$ Слышишь? &[Because on the street$\ldots$ Do you hear?]&Sinon, dans la 
rue$\ldots$ Tu comprends? \\
\hline
Вообще, ради бога, не пускай его никуда.&[For god's sake, don't let him go anywhere at all.]&En 
g$\acute{\mbox{e}}$n$\acute{\mbox{e}}$ral, pour l'amour du ciel, ne le laisse aller nulle part.\\
  \hline
  \multicolumn{3}{p{156mm}}{\footnotesize \hspace*{2mm}$^*$Mikhail Boulgakov. 
The White Guard. 1926. French translation by Claude Ligny, 1970. Literal English translation of Russian 
texts here and hereafter is provided by the authors of the paper.}
  \end{tabular}
  \end{center}
  \vspace*{12pt}
  \end{table*}
  
  \begin{figure*}[b] %fig1
  \vspace*{6pt}
 \begin{center}
 \mbox{%
 \epsfxsize=128.842mm 
 \epsfbox{ink-1.eps}
 }
 \end{center}
\vspace*{3pt}

\noindent
{\small The framework of the SCDB of Connectives and an example of an 
annotated TC (the TC presented in this figure is based on the fragment of the parallel text 
presented in Table~1)}
  \end{figure*}
  
  
  For many tasks of contrastive analysis, such as analy\-sis of language specificity of 
Russian connectives in\linebreak\vspace*{-12pt}

\pagebreak

\noindent
 contrast with French, an additional layer of annotation is 
needed. The goal of this additional layer is to enable annotation of TCs between 
specific language units (LUs), in this case~--- Russian connectives, and their TEs in 
French. Note that Russian connectives may include more than one word while in 
contemporary corpora, only individual words are usually annotated. To support this 
type of annotation, a~new information resource was created~--- the SCDB of 
Connectives.
  
  The SCDB of Connectives includes information about TCs that experts record and 
annotate for Russian connectives using customizable sets of features (tags). All LUs 
in the SCDB of Connectives (both Russian connectives and their corresponding TEs) 
are annotated with the following information:
  \begin{itemize}
\item word composition and relevant context of the LU;
  \item
  one main feature (tag) that specifies the type of the LU. A combination of main 
features of a~Russian connective and its French TE defines one of the possible 
TMs for that connective; and
  \item several additional features (tags) that describe context properties of the LU. 
The most notable among these are the tags that specify the meaning of each 
connective; in other words, discourse relation between the corresponding text 
fragments that this connective marks.
  \end{itemize}
  
  In addition, some features apply to the TC as a~whole and not to any of the LUs 
composing it. For example, translation type of a~TC is tagged \textit{congruent} if 
a~Russian connective corresponds to a~French LU of the same category (i.\,e., a~French 
connective), and \textit{divergent} if it corresponds to a~French LU of a~different 
category (see~\cite[p.~23--25]{8-in}). The figure presents an example of an annotated 
TC in the framework of the SCDB of Connectives.
  
  
  
  \section{Method of~Calculation of~Language Specificity Parameters}
  
  \noindent
  In~[9], the following features are proposed as parameters that may signal language 
specificity of a~Russian connective.
  
  In Russian--French parallel texts:
  \begin{enumerate}[1.]
\item A large number of different TMs.\\[-14pt]
\item A high proportion of divergent TEs (i.\,e., units other 
than connectives in translation).\\[-14pt]
\item A high proportion of zero-equivalents (connectives are not translated).
\end{enumerate}

  In French--Russian parallel texts: 
  \begin{enumerate}[1.]
  \setcounter{enumi}{3}
\item A large number of different translation stimuli (TSs).\\[-14pt]
\item A high proportion of divergent TSs.\\[-14pt]
\item A high proportion of zero-stimuli (no stimuli for the connective is found 
in the original text).
\end{enumerate}


%\begin{table*}
{\small %tabl2
 \vspace*{9pt} 
  
  \noindent
{{\tablename~2}\ \ \small{Normalization of the language specificity parameter~1 per~100~TCs 
based on the assumption of linear growth}}

\begin{center}

   \tabcolsep=8pt
   \begin{tabular}{cccc}
         \hline
Connective &\tabcolsep=0pt\begin{tabular}{c}Number\\ of TCs\end{tabular}&
\tabcolsep=0pt\begin{tabular}{c}Number\\ of TMs\end{tabular}&
\tabcolsep=0pt\begin{tabular}{c}Estimated\\ number of TMs\\ per 100 TCs\end{tabular}\\
\hline
$K_1$&300&100&33.3\\
$K_2$&\hphantom{9}50&\hphantom{9}20&40\hphantom{.9}\\
\hline
\end{tabular}
\end{center}
\vspace*{9pt}
}
%\end{table*}

  Parameters~2--3 and 5--6 are calculated as percentages from the total number of 
TCs for each connective. Evaluation of parameters~1 and~4 is less obvious. It is 
obvious that as the number of annotated TCs for a~certain connective grows in the 
SCDB, the number of identified TMs for that connective will also grow. For instance, 
let us suppose that in the Russian--French parallel corpus of the SCDB~50~TCs were 
annotated for connective~$K_1$ and 300~TCs~--- for connective~$K_2$. In addition, 
20~TMs were identified for connective~$K_1$ and 100~TMs~--- for 
connective~$K_2$. Then, based on the assumption of linear growth, one would arrive 
at the estimate shown in Table~2. 


  
  
  
  Based on this estimate, it could be concluded that connective~$K_2$ generates new 
translation models slightly more often than connective~$K_1$. This is, however, not 
true, because increase of the number of TMs relative to the number of TCs is not 
linear. It is intuitively clear that as more and more annotated TCs for a~certain 
connective are registered in the database, growth of the number of new, previously 
unidentified TMs will be gradually slowing down. This assumption is supported by 
the data. 





  
  This means that in order to compare language specificity of these two connectives 
based on the parameter~1, it is necessary to approximate the ratio of increase in the 
number of TMs relative to the total number of TCs for a~given connective (the same 
logic applies to parameter~4, which is calculated based on reverse French--Russian 
translations).
  
  Analysis of available data showed that the growth of the number of TMs relative to 
number of TCs can be approximately described by the following 
formula\footnote{The formula was proposed and tested by M. Kruzhkov.}:
  $$
  n_i=R_i  m_i^{0.65}
  $$
  where $m_i$ is the number of TCs created for connective~$K_i$ in the SCDB; 
$n_i$ is the number of TMs identified for connective~$K_i$ in these TCs; and~$R_i$ 
is the \textit{TM-factor} of language specificity of the connective~K$_i$. The higher 
is the TM-factor, the higher is the likelihood for a~certain connective to generate new 
TMs with each new TC created in the SCDB. Based on this formula, the TM-factor is 
calculated as follows:

\noindent
  $$
  R_i=\fr {n_i}{m_i^{0.65}}\,.
  $$
  
  The proposed method for generating more accurate language specificity 
parameters is based on this formula. The updated estimate for the above speculative 
example (Table~3) clearly shows that Russian connective~$K_1$ has 
a~significantly higher TM-factor of language specificity and, thus, generates new 
TMs more often compared to connective~$K_2$.
  
  \setcounter{table}{2}
  \begin{table*}\small %tabl3
  \begin{center}
  \Caption{Language specificity parameters based on the formula 
%  $n_i = R_i * (m_i)^{0.65}$}
  $n_i = R_i  m_i^{0.65}$}
   \vspace*{2ex}
   
   \tabcolsep=9pt
   \begin{tabular}{ccccc}
   \hline
Connective&\tabcolsep=0pt\begin{tabular}{c}Number\\ of TCs, $m_i$\end{tabular}&
\tabcolsep=0pt\begin{tabular}{c}Number\\ of TMs, $n_i$\end{tabular}&
TM-factor, $R_i$&
\tabcolsep=0pt\begin{tabular}{c}Estimated\\ number of TMs\\ per 100 TCs\end{tabular} \\
\hline 
$K_1$&300&100&2.45&49.0\\
$K_2$&\hphantom{9}50&\hphantom{9}20&1.57&31.4\\
\hline
\end{tabular}
\end{center}
%\end{table*}
%
 %\begin{table*}\small %tabl4
  \begin{center}
\Caption{Language specificity parameters of \textit{а~то} and \textit{или} based on 
the SCDB data}
\vspace*{2ex}

\begin{tabular}{ccccccccc}
\hline
&&&&&&&&\\[-9pt]
Russian connctives  
&Total TCs&Zero$^1$&Zero, \%&Dvrg$^2$&Dvrg, \%&TMs&TM-factor&TMs per 100 \\
\hline
а то&210&54&25,71&6&2,86&33&1.02&20.37\\
или&210&20&9,52&1&0,48&18&0.56&11.11\\
\hline
\multicolumn{9}{p{100mm}}{\footnotesize $^1$Number of zero-equivalents (see 
language specificity parameter~3).\newline
$^2$Number of divergent TEs (see language specificity parameter~2).}
\end{tabular}
\end{center}
\end{table*}


  In order to calculate language specificity parameters based on the number of 
different TSs (parameter~4), more experimental data should be gathered and 
processed to test and adopt the above formula to correctly reflect the growth of the 
number of TSs relative to number of TCs.

%\vspace*{-18pt}
  
  \section{Examination of~Language Specificity Parameters 
  for~\textit{а~то}~and~\textit{или}}
  
  %\vspace*{-2pt}
  
  \noindent
  Table~4 shows the language specificity parameters calculated based 
on~210~annotations of \textit{а~то} and~210~annotations of \textit{или} in the 
SCDB's Russian--French corpus.
  
 


  All the parameters differ significantly for the two connectives, their values for 
\textit{а~то} are at least twice as high as for \textit{или}. Of course, further semantic 
analysis is required to clarify the nature of these differences.

%\vspace*{-48pt}
  
  \subsection{Connective `или'}
  
  \noindent 
  Connective \textit{или} expresses the relation of alternative~(1) and in~70.48\% of 
TCs in the SCDB, its French functional equivalent is connective \textit{ou}, which 
has similar semantics.
  \begin{enumerate}[(1)]
  \item Закон, устанавливающий \textit{или} отягчающий ответственность, обратной силы не 
имеет.\footnote{The law that establishes or aggravates responsibility does not have a~retroactive effect.}
    
  La loi $\acute{\mbox{e}}$tablissant \textit{ou} aggravant la responsabilit$\acute{\mbox{e}}$ 
d'une personne n'a pas d'effet r$\acute{\mbox{e}}$troactif. [Constitution de la 
F$\acute{\mbox{e}}$d$\acute{\mbox{e}}$ration de Russie (M.: Lesage, 2000)]
  \end{enumerate}
  
  Nevertheless, in different contexts the opposition of disjuncts can be more or less 
strong. For example, it is minimal in~(2), where alternative is perceived almost as 
a~connective relation, and maximal in~(3), where alternative is interpreted as one 
that excludes one of the disjuncts.
  \begin{enumerate}[(1)]
  \setcounter{enumi}{1}
\item Здесь чужой акцент только намечается, но он уже порождает оговорку 
\textit{или} заминку в~речи.\footnote{Here, the foreign accent is 
barely hinted at, but it 
already entails a~fault \textit{or} a~halt in discourse.}   
 
Ici l'accent ``$\acute{\mbox{e}}$tranger'' est seulement 
esquiss$\acute{\mbox{e}}$, mais cela suffit $\Grave{\mbox{a}}$~faire 
na$\hat{\mbox{\ptb{\i}}}$tre dans le discours des r$\acute{\mbox{e}}$serves 
\textit{et} des h$\acute{\mbox{e}}$sitations. 
[{\mbox{Mikha{$\ddot{\mbox{\ptb{\!\i}}}$l}}} Bakhtine. La 
po$\acute{\mbox{e}}$tique de \mbox{Dosto{$\ddot{\mbox{\ptb{\hspace*{-0.3pt}\!\i}}}$}evski} 
(Isabelle Kolitcheff, 1970)]

\item Но правило для всех одно: <<Прогибайся \textit{или} 
прогибай>>.\footnote{But the rule is the same for everybody: ``Yield, \textit{or} force others to 
yield.''}   

Mais la r$\Grave{\mbox{e}}$gle est la m$\hat{\mbox{e}}$me pour tout le 
monde: \textit{ou} tu te fais $\acute{\mbox{e}}$craser, \textit{ou} tu 
$\acute{\mbox{e}}$crases les autres. [Svetlana Alexievitch. La fin de l'homme 
rouge ou le temps du d$\acute{\mbox{e}}$senchantement (Sophie Benech, 
2013)]
\end{enumerate}

  These subtleties of meaning can be made explicit by translators. When opposition 
between disjuncts weakens, translators choose copulative \textit{et}~(2), \textit{ainsi 
que}, and in the presence of negation of disjuncts~--- \textit{ni}. Translation models 
with copulative semantics account for~7.62\% of TCs with \textit{или}. Some of the 
zero-equivalents (9.52\%) can also be attributed to weakening the opposition: in such 
cases, disjuncts in translation are separated by a~comma. When the opposition is 
strong, translators choose duplicate conjunctions \textit{ou}$\|$\textit{ou}~(3), 
\textit{soit}$\|$\textit{soit}, single or duplicate \textit{ou bien}. They account 
for~6.68\% of the total TCs (5.24\% of which account for single \textit{ou bien}).
  
  The remaining TMs (mostly, metalinguistic alternatives, such as 
\textit{ou plut$\hat{\mbox{o}}$t}, \textit{autrement dit},  
\textit{c'est-$\Grave{\mbox{a}}$-dire}) account for~8.58\%. These factors explain the 
number of translation models~(18) and the TM-factor~(0.56). However, they are all 
rather homogeneous in their semantics and could be replaced by the most frequent 
TM \textit{ou}. Finally, the relation of alternative is extremely rarely 
expressed by other means than connectives: a~divergent translation (a~temporal 
alteration adverbial \textit{de temps en temps}) was encountered only once (0.48\%). 
This usage can be viewed as a~context-driven translator's choice.
  
  \subsection{Connective `а то'}
  
  \noindent 
  As opposed to \textit{или}, \textit{а~то} is a~polysemic connective; researchers 
distinguish up to 6 meanings for it (i.\,e., discourse relations that can be expressed 
by it, see~\cite{10-in}). It is important to note that language specificity parameters of 
\textit{а~то} vary considerably for different meanings of this connective (Table~5).
  
  
  \begin{table*}\small %tabl5
  \begin{center}
  \Caption{Language specificity parameters of \textit{а~то} combined with relations 
expressed by it}
  \vspace*{2ex}
  
  \begin{tabular}{lccccccccc}
  \hline
\multicolumn{1}{c}{Relation}&TCs&TCs, \%&Zero&Zero, \%&Dvrg&Dvrg, 
\%&TMs&TM-factor&TMs per 100 \\
\hline
negative 
alternative&116\hphantom{9}&56&11&\hphantom{9}9.48&2&1.72&19\hphantom
{9}&0.86&17.25\\
cause&52&25&33&63.46&0&0.00&9&0.69&13.77\\
discrepancy&15&\hphantom{9}7&\hphantom{9}3&20.00&1&6.67&7&1.20&24.04
\\
alternative&13&\hphantom{9}6&\hphantom{9}3&23.08&1&7.69&8&1.51&30.11\\
complementarity&\hphantom{9}9&\hphantom{9}4&\hphantom{9}3&33.33&0&0.00&6&
\hphantom{(!)}1.44(!)&\hphantom{(!)$^*$}28.71(!)$^*$\\
exception&\hphantom{9}2&\hphantom{9}1&\hphantom{9}0&\hphantom{9}0.00&
2&100.00\hphantom{99}&2&---&---\\
  \hline
  \multicolumn{10}{l}{\footnotesize 
  \hspace*{2mm}$^*$The estimate is approximate since there is not enough data.}
   \end{tabular}
   \end{center}
   \end{table*}
  
  The most frequent meaning of \textit{а~то} is the negative alternative relation~--- 
it accounts for~56\% of all its uses~(4).
  \begin{enumerate}[(1)]
  \setcounter{enumi}{3}
  \item --- Говори, \textit{а~то} я~заплачу.\footnote[1]{Speak, \textit{or else} I will cry.}  
   
--- Parle, \textit{sinon} je pleure. [\mbox{Sergue{$\ddot{\mbox{\ptb{\!\i}}}$}} Dovlatov. 
L'$\acute{\mbox{E}}$trang$\Grave{\mbox{e}}$re (Jacques Michaut-Paterno, 
2001)]
  \end{enumerate}

  When combined with this relation, the language specificity parameters of 
\textit{а~то} are relatively low. Zero-equivalents are less than~10\%, divergent 
translations are also rare (1,72\%), and TM-factor is also relatively low (0.86). The most 
frequent TM is \textit{sinon} (63.79\%), which can be combined with 
\textit{mais} and \textit{et} (1.72\%); there are also a~few synonyms of \textit{sinon}: 
\textit{sans cela}, \textit{sans quoi}, \textit{sans {\ptb{\c{c}}}a}, \textit{autrement}, 
and \textit{ou} (12.07\%). Translation models from this semantic group account 
for~77.58\% of all TCs for negative alternative \textit{а~то} in the SCDB, and if we 
include zero-equivalents, the total reaches~87.06\%. 
  
  The second most frequent meaning of \textit{а~то} is the one that expresses cause 
(25\%):
  \begin{enumerate}[(1)]
  \setcounter{enumi}{4}
\item По крайней мере позвольте объяснить господину Вольдемару, в~чем 
дело, ($\ldots$) \textit{а~то} он совсем растерялся.\footnote[2]{At least, let me explain 
to Mr.\ Voldemar what is the matter, ($\ldots$) \textit{since} he is completely at a~loss.}

Permettez au moins que nous expliquions le jeu 
$\Grave{\mbox{a}}$~M.~Vold$\acute{\mbox{e}}$mar, ($\ldots$). \textit{Car} 
il a~compl$\Grave{\mbox{e}}$tement perdu le nord$\ldots$ [Ivan Tourgueniev. 
Premier Amour (Michel-Rostislav Hofmann, 1974)]
\end{enumerate}

  The most frequent equivalent here is the zero-equivalent, which is not unusual for 
causal relations (see~\cite{11-in, 12-in}), but the actual percentage (63.46\%) surely 
attracts attention. In cases where a~TE can be identified, equivalents are usually 
markers of casual relations: \textit{car} (17.31\%), \textit{parce que} (3.85\%), 
\textit{$\Grave{\mbox{a}}$~cause de} (1.92\%). One should also note that in some 
cases, \textit{а~то} with negative alternative relation is translated into French by 
causal conjunctions (1.74\%) and at the same time, casual \textit{а~то} in some 
cases (7.72\%, even more often) is translated into French by units that express the 
negative alternative relation (\textit{sinon}, \textit{sans quoi}). Compare~(6), where 
in the original text connective \textit{а~то} expresses cause, while in the translation, 
the present tense is substituted by the future and, as a~result, the causal motivation 
is transformed into motivation through possible negative consequences of failure to 
implement the initial proposition.
  \begin{enumerate}[(1)]
  \setcounter{enumi}{5}
  \item Нужно физическое движенье, \textit{а~то} мой характер решительно 
портится.\footnote[3]{Physical movement is necessary, \textit{since} my character clearly deteriorates.}
  
  J'ai besoin d'un exercice violent, \textit{sinon} mon caract$\Grave{\mbox{e}}$re deviendra 
intraitable [L$\acute{\mbox{e}}$on Tolsto{$\ddot{\mbox{\ptb{\!\i}}}$}. Anna 
Kar$\acute{\mbox{e}}$nine (Henri Mongault, 1952)]
  \end{enumerate}

  Nevertheless, the TM-factor for causal relation is lower than for negative 
alternative relation: 0.69 which is evidently caused by extremely high percentage of 
zero-equivalents.
  
  Next in terms of frequency are relations of discrepancy (7\%) and alternative 
(6\%). The TM-factors for these relations are quite high: 1.2 and~1.51, respectively. 
This may have to do with the fact that \textit{а~то} belongs to an informal style of 
discourse abundant with fixed expressions for which it is often difficult to find an 
exact equivalent; so, the choice of a~TMs is often dependent on the 
context. However, as in the above cases, all TMs belong to the same 
semantic class.
  
  The most frequent TM for \textit{а~то} with discrepancy meaning is 
\textit{mais} (26.67\%, see~(7)); next in frequency are \textit{tandis que} and  
zero-equivalent (20\% each).
  \begin{enumerate}[(1)]
  \setcounter{enumi}{6}
\item Ну, захотел помочь~--- дай пятнадцать, дай двадцать, ну да хоть три 
целковых себе оставь, \textit{а~то} все двадцать пять так 
и~отвалил!\footnote[4]{Well, if you want to help, then give 15, give 20, but keep at least three rubles 
for yourself, \textit{but} he just likes that gave away all~25.} 

 
Ma foi, si tu voulais l'aider, tu n'avais qu'$\Grave{\mbox{a}}$ donner quinze, 
vingt roubles, et en garder pour toi ne serait-ce que trois, \textit{mais} il les 
a~l$\hat{\mbox{a}}$ch$\acute{\mbox{e}}$s d'un coup tous les vingt-cinq! 
[F$\acute{\mbox{e}}$dor Dosto{$\ddot{\mbox{\ptb{\!\i}}}$}evski. Crime et 
ch$\hat{\mbox{a}}$timent ($\acute{\mbox{E}}$lisabeth Guertik, 1947)]
\end{enumerate}

  Another possible equivalent is conjunction \textit{et} (13.33\%), which is also used 
here in an oppositional context~(8) and, finally, there is \textit{au lieu que} (6.67\%), 
so that the total amounts to~86.66\%.
  \begin{enumerate}[(1)]
  \setcounter{enumi}{7}
\item --- Он женится! Хочешь об заклад, что не женится?~--- возразил он.~--- 
Да ему Захар и~спать-то помогает, \textit{а~то} жениться!\footnote{``He's 
getting married! Want a~bet that he will not marry?'' he retorted. ``Zakhar even helps him to sleep, 
\textit{and} [you talk of] marriage!''}

 
--- Lui, se marier! je parie qu'il ne se mariera pas!  
r$\acute{\mbox{e}}$pliqua-t-il. Il a~besoin de Zakhar m$\hat{\mbox{e}}$me 
pour dormir, \textit{et} tu veux qu'il se marie! [Ivan Gontcharov. Oblomov (Luba 
Jurgenson, 1988)]
\end{enumerate}

  High TM-factor of \textit{а~то} coupled with relation of alternative~(1.51) has to 
do with the fact that different TMs express different shades of this 
relation: temporal alternation (\textit{tant$\hat{\mbox{o}}$t}, \textit{de temps en 
temps}), gradation (\textit{m$\hat{\mbox{e}}$me}), or intensification of alternative 
(\textit{ou encore}, \textit{ou bien}). Nevertheless, in more than~50\% of cases, we 
encounter either TMs belonging to different semantic classes 
(30.76\%), or zero-equivalents (23.8\%).
  
  Unfortunately, there are insufficient data at this point to obtain representative 
statistics and make reliable conclusions about the two remaining relations~--- 
complementarity (4.21\%) and exception (0.53\%).

\vspace*{-6pt}
  
  \section{Concluding Remarks}
  
  \noindent
  The new information resource, the SCDB of Connectives, and the associated new 
method of analyzing corpus data offer researchers valuable insights that can produce 
tangible evidence that \textit{или} is not a~language specific connector in the context 
of Russian--French language pair. In 70.48\% of cases, its TE is 
conjunction \textit{ou} that expresses the same relation; in most other cases,
 the TMs selected by translators may be replaced by \textit{ou} without distortion of 
meaning. 
  
  On the other hand, analysis of the SCDB data suggests that Russian connective 
\textit{а~то} has a~relatively high degree of language specificity, especially for some 
relations that can be expressed by it. This is due to the fact that no connective with 
similar semantic configuration exists in French. Its most frequent TE
is \textit{sinon}, but it is found in only~38\% of TCs, and what's more, it 
is associated with a~different range of meanings (\textit{sinon} is translated into 
Russian by \textit{а~то} in~44.75\% of cases, mainly expressing the relation of 
negative alternative, which is common for both connectives and has one of the lowest 
TM-factors). These findings allow us to attribute \textit{а~то} to language 
specificity Type~4, where ``the connective $K_A$ of language~$A$ has in 
language~$B$ a~systemic equivalent~$K_B$ which is the most frequent in 
translation, but does not reproduce all the values and uses of $K_A$''~\cite{9-in}. In 
addition, statistical analysis of the corpus data allowed us to arrange~6~meanings 
of the Russian connective \textit{а~то} according to their frequencies.
{\looseness=1

}
  
  Thus, the proposed method of statistical analysis can lend greater objectivity to 
linguistic research by supplementing semantic analysis with quantitative data that can 
either support or question it. The results of this research are applicable to many areas 
related to informatics (machine translation) and comparative study of languages, such 
as lexical typology, lexicography, and theory and practice of translation.

\vspace*{-12pt}

\Ack
  \noindent
   The work was carried out at the Institute of Informatics Problems (FRC CSC RAS) and funded by Russian 
Science Foundation according to the research project No.\,16-18-10004.
  
 \renewcommand{\bibname}{\protect\rmfamily References}

%\vspace*{-6pt}

\vspace*{-6pt}

{\small\frenchspacing
{\baselineskip=10.65pt
\begin{thebibliography}{99}

\bibitem{1-in}
  \Aue{Wierzbicka, A.} 1992. \textit{Semantics, culture, and cognition. Universal 
human concepts in culture-specific configurations}. Oxford: Oxford University Press. 
496~p.
\bibitem{2-in}
\Aue{Zaliznyak, Anna A., and I.\,B.~Levontina}. 1996. Otrazhenie natsional'nogo 
kharaktera v~leksike russkogo yazyka [The reflection of the national character in the 
vocabulary of the Russian language]. \textit{Russ. Linguist.} 20:237--264.
  \bibitem{3-in}
  \Aue{Zaliznyak, Anna A., I.\,B.~Levontina, and A.\,D.~Shmelev.} 2012. 
\textit{Konstanty i~peremennye russkoy yazykovoy kartiny mira} [Constants and 
variables of the Russian language picture of the world]. Moscow: Yazyki 
Slavyanskikh Kul'tur [Languages of Slavic Cultures]. 696~p.
  \bibitem{4-in}
  \Aue{Zaliznyak, Anna~A.} 2015. Lingvospetsifichnye edinitsy russkogo yazyka 
v~svete kontrastivnogo lingvisticheskogo analiza [Russian language-specific words 
as an object of contrastive corpus analysis]. \textit{Computational Linguistics and 
Intellectual Technologies: Conference (International) ``Dialogue 2015'' 
Proceedings}. Moscow: RGGU. 14(21):683--695.
  \bibitem{5-in}
  \Aue{Kruzhkov, M.} 2016. Supracorpora Databases as corpus-based 
superstructure for manual annotation of parallel corpora. \textit{8th Conference 
(International) on Corpus %\linebreak 
Linguistics}. EPiC ser. in language and linguistics.  
1:236--248. Available at: {\sf https://easychair.org/publications/\linebreak paper/270289} (accessed 
May~29, 2017).
  \bibitem{6-in}
  \Aue{ Inkova, O., and M.~Kruzhkov.} 2016. Nadkorpusnye russko-frantsuzskie 
bazy dannykh glagol'nykh form i~konnektorov [Supracorpora databases of Russian 
and French verbal forms and connectors]. \textit{Lingue slave a~confronto}. 
Bergamo: Bergamo University Press.  365--392.
  \bibitem{7-in}
  \Aue{Inkova, O., and N.~Popkova.} 2017. Statistical data as information source 
for linguistic analysis of Russian connectors. \textit{Informatika i~ee  
Primeneniya~--- Inform. Appl.} 11(3):123--131.
  \bibitem{8-in}
  \Aue{Johansson, S.} 2007. \textit{Seeing through Multilingual Corpora}. 
Amsterdam: John Benjamins B.V. 355~p.
  \bibitem{9-in}
  \Aue{Inkova, O.} 2017. Printsipy opredeleniya stepeni lingvospetsifichnosti 
konnektorov [Principles of how to determine the degree of language-specificity of 
connec-\linebreak \vspace*{-10pt}
\pagebreak

\noindent
tives]. \textit{Computational Linguistics and Intellectual Technologies: 
Conference (International) ``Dialogue 2017'' Proceedings}. Moscow: RGGU. 
16(23):139--149.
  \bibitem{10-in}
  \Aue{Inkova-Manzotti, O.} 2005. Encore sur la conjonction russe a~to. 
  \textit{Revue des $\acute{\mbox{e}}$tudes 
slaves} 76(4):485--497.
%\columnbreak


  \bibitem{11-in}
  \Aue{Sanders, T.\,J.\,M.} 2005. Coherence, causality and cognitive complexity in 
discourse. \textit{1st Symposium (International) on the Exploration and Modelling of 
Meaning Proceedings.} Toulouse: University of Toulouse-le-Mirail. %\linebreak  
105--114.
  \bibitem{12-in}
  \Aue{Hoek, J., J.~Evers-Vermeul, and T.~Sanders.} 2015. The role of 
expectedness in the implicitation and explicitation of discourse relations. 
\textit{Discourse in Machine Translation\linebreak (DiscoMT 2015): 2nd Workshop 
Proceedings}. Lisbon, Portugal: Association for Computational 
Linguistics. 41--46.
\end{thebibliography} } }

\end{multicols}

\vspace*{-6pt}

\hfill{\small\textit{Received July 6, 2018}}

\vspace*{-21pt}
  
  \Contr
  
  \noindent
  \textbf{Inkova Olga Yu.} (b.\ 1965)~--- Doctor of Science (PhD) in philology, 
senior scientist, Institute of Informatics Problems, Federal Research Center 
``Computer Science and Control'' of the Russian Academy of Sciences, 44-2~Vavilov 
Str., Moscow 119333, Russian Federation; \mbox{olyainkova@yandex.ru}
  
  \vspace*{1pt}
  
  \noindent
  \textbf{Kruzhkov Mikhail G.} (b.\ 1975)~--- senior scientist, Institute of 
Informatics Problems, Federal Research Center ``Computer Science and Control'' of 
the Russian Academy of Sciences, 44-2~Vavilov Str., Moscow 119333, Russian 
Federation; \mbox{magnit75@yandex.ru}

\vspace*{8pt}

\hrule

\vspace*{2pt}

\hrule

\vspace*{-2pt}

%\newpage

\def\tit{СТАТИСТИЧЕСКИЙ АНАЛИЗ ЛИНГВОСПЕЦИФИЧНОСТИ КОННЕКТОРОВ
(НА~МАТЕРИАЛЕ~ПАРАЛЛЕЛЬНЫХ КОРПУСОВ)$^*$}

\def\titkol{Статистический анализ лингвоспецифичности коннекторов
(на материале параллельных корпусов)}

\def\aut{О.\,Ю.~Инькова, М.\,Г.~Кружков}

\def\autkol{О.\,Ю.~Инькова, М.\,Г.~Кружков}

{\renewcommand{\thefootnote}{\fnsymbol{footnote}} \footnotetext[1]
{Работа выполнена при финансовой поддержке РНФ (проект 16-18-10004).}}



\titel{\tit}{\aut}{\autkol}{\titkol}

\vspace*{-11pt}

\noindent
Институт проблем информатики Федерального исследовательского центра 
<<Информатика и~управление>> 
Российской академии наук

\vspace*{5pt}

\def\leftfootline{\small{\textbf{\thepage}
\hfill ИНФОРМАТИКА И ЕЁ ПРИМЕНЕНИЯ\ \ \ том\ 12\ \ \ выпуск\ 3\ \ \ 2018}
}%
 \def\rightfootline{\small{ИНФОРМАТИКА И ЕЁ ПРИМЕНЕНИЯ\ \ \ том\ 12\ \ \ выпуск\ 3\ \ \ 2018
\hfill \textbf{\thepage}}}

\vspace*{-3pt}

  \Abst{В~последние десятилетия проблемы лингвоспецифичности в~русском 
языке привлекают пристальное внимание исследователей, хотя до последнего 
времени они рассматривались без привлечения корпусных методов. В~этой 
работе описываются новые методы исследования лингвоспецифичности 
русских коннекторов с~привлечением статистического анализа 
аннотированных параллельных корпусов. Рус\-ско-фран\-цуз\-ские  
и~фран\-ко-рус\-ские параллельные тексты обрабатываются с~помощью 
Надкорпусной базы данных коннекторов, разработанной специально для 
аннотирования переводных соответствий, выявляемых в~параллельных 
корпусах. Каждое переводное соответствие включает в~себя аннотацию 
русского коннектора и~его переводного эквивалента, что позволяет 
генерировать статистические данные по различным типам моделей  
перевода~--- на основе предлагаемых для коннекторов параметров 
лингвоспецифичности. В~качестве примера в~данной работе рассматриваются 
два русских коннектора: \textit{или} и~\textit{а~то}. На основе предлагаемых 
статистических параметров будет показано, что \textit{или} имеет очень низкий 
уровень лингвоспецифичности в~сопоставлении с~французским языком, в~то 
время как коннектор \textit{а~то} обладает высокой лингвоспецифичностью. 
Результаты данного исследования могут быть востребованы в~информатике 
(машинный перевод, статистический анализ текстовых данных), а~также 
в~различных областях, связанных с~контрастивными лингвистическими 
исследованиями, таких как лексическая типология, лексикография 
и~переводоведение.}
  
  \KW{надкорпусные базы данных; статистический анализ; контрастивный 
корпусный анализ; лингвоспецифичность; параллельные корпуса; 
лингвистические информационные ресурсы; коннекторы; дискурсивные 
отношения; семантика}
  
\DOI{10.14357/19922264180312}



\vspace*{-3pt}


 \begin{multicols}{2}

\renewcommand{\bibname}{\protect\rmfamily Литература}
%\renewcommand{\bibname}{\large\protect\rm References}

{\small\frenchspacing
{%\baselineskip=10.8pt
\begin{thebibliography}{99}
\vspace*{-3pt}

  \bibitem{1-in-1}
  \Au{Wierzbicka A.} Semantics, culture, and cognition. Universal human concepts 
in culture-specific configurations.~--- Oxford: Oxford University Press, 1992. 496~p.
  \bibitem{2-in-1}
  \Au{Зализняк Анна~А., Левонтина~И.\,Б.} Отражение национального 
характера в~лексике русского языка~// Russ. Linguist., 1996. Вып.~20. 
С.~237--264.
  \bibitem{3-in-1}
  \Au{Зализняк Анна~А., Левонтина~И.\,Б., Шмелев~А.\,Д.} Константы 
и~переменные русской языковой картины мира.~--- М.: Языки славянских 
культур, 2012. 696~c.
  \bibitem{4-in-1}
  \Au{Зализняк Анна~А.} Лингвоспецифичные единицы русско\-го языка в~свете 
контрастивного корпусного анализа~// Компьютерная лингвистика 
и~интеллектуальные технологии: По мат-лам ежегодной Между-\linebreak
\vspace*{-12pt}

\pagebreak

\noindent
нар. конф. 
<<Диалог>>.~--- М.: РГГУ, 2015. Вып.~14(21). С.~683--695.
  \bibitem{5-in-1}
  \Au{Kruzhkov M.} Supracorpora Databases as corpus-based superstructure for 
manual annotation of parallel corpora~// 8th Conference (International) on Corpus 
Linguistics.~--- EPiC ser. in language and linguistics.~--- 2016. Vol.~1.  
P.~236--248. {\sf https://easychair.org/\linebreak publications/paper/270289}.
  \bibitem{6-in-1}
  \Au{Инькова О., Кружков~М.} Надкорпусные рус\-ско-фран\-цуз\-ские базы 
данных глагольных форм и~коннекторов~// Lingue slave a~confronto.~--- 
Bergamo: Bergamo University Press, 2016. С.~365--392.
  \bibitem{7-in-1}
  \Au{Inkova O., Popkova~N.} Statistical data as information source for linguistic 
analysis of Russian connectors~// Информатика и~её применения, 2017. Т.~11. 
Вып.~3. С.~123--131.
  \bibitem{8-in-1}
  \Au{Johansson S.} Seeing through Multilingual Corpora.~--- Amsterdam: John 
Benjamins B.V., 2007. 355~p.
  \bibitem{9-in-1}
  \Au{Инькова О.} Принципы определения степени лингвоспецифичности 
коннекторов~// Компьютерная лингвистика и~интеллектуальные технологии: 
По мат-лам ежегодной Междунар. конф. <<Диалог>>.~--- М.: РГГУ, 2017. 
Вып.~16(23). С.~139--149.
  \bibitem{10-in-1}
  \Au{Inkova-Manzotti O.} Encore sur la conjonction russe a~to~// Revue des 
$\acute{\mbox{e}}$tudes slaves, 2005. Vol.~76. No.\,4. P.~485--497.
  \bibitem{11-in-1}
  \Au{Sanders T.\,J.\,M.} Coherence, causality and cognitive complexity in 
discourse~// 1st Symposium (International) on the Exploration and Modelling of 
Meaning Proceedings.~--- Toulouse: University of Toulouse-le-Mirail, 2005. 
P.~105--114.
  \bibitem{12-in-1}
  \Au{Hoek J., Evers-Vermeul~J., Sanders~T.} The role of expectedness in the 
implicitation and explicitation of discourse relations~// Discourse in Machine 
Translation (DiscoMT 2015): 2nd Workshop Proceedings.~--- Lisbon, Portugal: 
Association for Computational Linguistics, 2015. P.~41--46.
  
\end{thebibliography}
} }

\end{multicols}

 \label{end\stat}

 \vspace*{-3pt}

\hfill{\small\textit{Поступила в~редакцию  06.07.2018}}


%\renewcommand{\bibname}{\protect\rm Литература}
\renewcommand{\figurename}{\protect\bf Рис.}
\renewcommand{\tablename}{\protect\bf Таблица}