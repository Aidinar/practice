
\newcommand{\mujk}{\mu_{j,k}}
\newcommand{\sumk}{\sum_{k=0}^{2^j-1}}
%\newcommand{\abs}[1]{\left|#1\right|}
%\newcommand{\p}{{\sf P}}

%\newcommand{\I}{\mathbf{1}}


\def\stat{shest}

\def\tit{СРЕДНЕКВАДРАТИЧНЫЙ РИСК ПОРОГОВОЙ ОБРАБОТКИ
ПРИ~СЛУЧАЙНОМ ОБЪЕМЕ ВЫБОРКИ$^*$}

\def\titkol{Среднеквадратичный риск пороговой обработки
при~случайном объеме выборки}

\def\aut{О.\,В.~Шестаков$^1$}

\def\autkol{О.\,В.~Шестаков}

\titel{\tit}{\aut}{\autkol}{\titkol}

\index{Шестаков О.\,В.}
\index{Shestakov O.\,V.}




{\renewcommand{\thefootnote}{\fnsymbol{footnote}} \footnotetext[1]
{Работа выполнена при финансовой поддержке Российского научного фонда (проект 18-11-00155).}}


\renewcommand{\thefootnote}{\arabic{footnote}}
\footnotetext[1]{Московский государственный университет им.\ М.\,В.~Ломоносова, 
кафедра математической статистики факультета вычислительной математики и~кибернетики; 
Институт проб\-лем информатики Федерального исследовательского центра 
<<Информатика и~управ\-ле\-ние>> Российской академии наук, \mbox{oshestakov@cs.msu.su}}

\vspace*{12pt}



\Abst{Нелинейные методы удаления шума из сигналов, основанные на пороговой 
обработке вейв\-лет-ко\-эф\-фи\-ци\-ен\-тов, широко используются в~различных 
прикладных областях. Свою популярность эти методы приобрели за счет быстроты 
алгоритмов построения оценок и~возможности лучшей, чем линейные методы, 
адаптации к~функциям, принадлежащим различным классам ре\-гу\-ляр\-ности. 
При использовании методов пороговой обработки обычно предполагается, 
что число вейв\-лет-ко\-эф\-фи\-ци\-ен\-тов фиксировано, а~распределение шума 
является гауссовым. Эта модель хорошо изучена в~литературе, и~для разных классов 
сигналов вы\-чис\-ле\-ны оптимальные значения порогов. Однако в~некоторых ситуациях 
объем выборки заранее не известен и~моделируется случайной величиной. 
В~данной работе рас\-смат\-ри\-ва\-ет\-ся модель со случайным чис\-лом наблюдений, содержащих 
гауссов шум и~оценивается порядок среднеквадратичного риска при растущем объеме 
выборки.}

\KW{пороговая обработка; случайный объем выборки; среднеквадратичный риск}

\DOI{10.14357/19922264180302}
  
%\vspace*{4pt}


\vskip 10pt plus 9pt minus 6pt

\thispagestyle{headings}

\begin{multicols}{2}

\label{st\stat}


\section{Введение}

Во многих областях, таких как физика плазмы, медицина, геофизика, 
астрономия, радиолокация, сис\-те\-мы связи и~передачи информации, компьютерная 
графика и~т.\,д., возникает по\-треб\-ность в~анализе и~обработке сигналов самых 
разных видов и~происхождения. В~некоторых случаях объем данных, доступных для 
анализа (объем выборки), заранее не известен. Такие ситуации могут возникать, 
например, в~случае пропуска данных (по техническим причинам), огра\-ни\-чен\-ности 
времени сбора данных при случайных временах регистрации наблюдений или недостатке 
информации о~характеристиках используемого оборудования (например, ис\-поль\-зу\-емый 
прибор может принадлежать некоторой партии, внут\-ри которой определенные 
технические характеристики могут быть жест\-ко не фиксированы~--- это может приводить 
к~тому, что <<разрешение>> сигнала заранее не известно). 
В~таком случае предполагается, что объем выборки данных пред\-став\-ля\-ет 
собой случайную величину с~некоторым распределением вероятностей.

На одном из первых этапов обработки данных обычно осуществляется 
преобразование, приводящее к~их <<экономному>> (разреженному) пред\-став\-ле\-нию, т.\,е.\
 фактически разложение функции, опи\-сы\-ва\-ющей наблюдения (сигнал) по некоторому\linebreak 
 базису, зависящему от конкретного класса рас\-смат\-ри\-ва\-емых сигналов. 
 Примерами подобных базисов могут служить различные классы вейв\-ле\-тов. 
 В~данной работе рас\-смат\-ри\-ва\-ют\-ся именно такие вейв\-лет-ба\-зи\-сы. 
 Затем осуществляется пороговая обработка коэффициентов; в~результате 
часть коэффициентов разложения, которая считается несущественной, 
 обнуляется. Такие процедуры, во-пер\-вых, позволяют подавить б$\acute{\mbox{о}}$льшую часть 
 шума, возникающего из-за дискретизации исходной информации, несовершенства 
 оборудования, случайных помех, наличия фонового излучения и~других причин. 
 Во-вто\-рых, таким образом осуществляется сжатие данных с~незначительной 
 потерей информации, что позволяет более экономно хранить информацию и~быст\-рее 
 передавать ее по каналам цифровой связи. 
 
 В~моделях с~фиксированным объемом выборки 
 статистические свойства данных процедур хорошо изучены и~получены 
 выражения для <<оптимальных>> порогов, ориентированных на различные 
 классы функций сигналов и~различные распределения шума (см., например,~\cite{DJ94, DJ95, DJ98, Jan01, Jan06, SH17}). 
 
 В~данной работе рассматривается модель со случайным чис\-лом 
 вейв\-лет-ко\-эф\-фи\-ци\-ен\-тов 
 функции сигнала, <<загрязненных>> белым гауссовым шумом, и~оценивается 
 порядок сред\-не\-квад\-ра\-тич\-но\-го риска пороговой обработки.


\section{Пороговая обработка вейвлет-коэффициентов}

Пусть функция наблюдаемого сигнала принадлежит классу 
функций, регулярных по Липшицу с~показателем $\gamma\hm>0$. 
После вейв\-лет-пре\-об\-ра\-зо\-ва\-ния сигнала получается набор 
эмпирических вейв\-лет-ко\-эф\-фи\-ци\-ен\-тов, 
име\-ющих сле\-ду\-ющий вид:
\begin{multline}
\label{data_model}
Y_{j,k}=\mujk+z_{j,k}\,, \\
 j=0,\ldots,J-1\,,\enskip k=0,\ldots,2^j-1\,,
\end{multline}
где $\mujk$ --- вейв\-лет-ко\-эф\-фи\-ци\-ен\-ты <<чис\-то\-го>> 
сигнала, а~$z_{j,k}$~--- <<шумовые>> коэффициенты, относительно которых 
предполагается, что они независимы и~имеют нормальное распределение с~нулевым 
средним и~дис\-пер\-си\-ей~$\sigma^2$. Если базисные вейв\-лет-функ\-ции
 удовлетворяют некоторым дополнительным условиям глад\-кости~\cite{Mal99}, то, 
 поскольку функция сигнала регулярна по Липшицу, существует такая положительная 
 константа~$A$, что
 \begin{equation}
\label{Wavelet_CoeffDecacy}
\abs{\mujk}\leqslant\fr{A \cdot2^{J/2}}{2^{j\left(\gamma+1/2\right)}}\,.
\end{equation}
Через $\mathrm{Lip}\,(A,\gamma)$ обозначим класс регулярных по Липшицу функций, 
вейв\-лет-ко\-эф\-фи\-ци\-ен\-ты которых удовлетворяют~\eqref{Wavelet_CoeffDecacy}.

Популярным методом удаления шума является пороговая обработка эмпирических 
вейв\-лет-ко\-эф\-фи\-ци\-ен\-тов, смысл которой заключается в~обнулении коэффициентов, 
чьи абсолютные значения не превышают заданного порога. Оценка~$\hat{Y}_{j,k}$ 
вы\-чис\-ля\-ет\-ся с~по\-мощью пороговой функции~$\rho_T(Y_{j,k})$ с~порогом~$T$.
 Наиболее распространены функ\-ция жест\-кой пороговой обработки $\rho_T^{(h)}(x)\hm=
 x \mathbf{1} (|x|\hm>T)$ и~мягкой пороговой обработки $\rho_T^{(s)}(x)\hm=
 {\mathrm{sign}}\,(x)(|x|\hm-T)_+$. Смысл такой обработки заключается в~том, что,
  поскольку большинство <<чис\-тых>> коэффициентов малы по абсолютному\linebreak значению, 
  обнуление эмпирических коэффициентов должно удалить шум, не сильно за\-тро\-нув
   полезный сигнал.

Основным критерием качества методов пороговой обработки является 
среднеквадратичный риск
\begin{equation}\label{Risk_Definition}
r_J(f)=\fr{1}{2^J}\sum\limits_{j=0}^{J-1}\sumk\limits 
{\sf E}\left(\hat{Y}_{j,k}-\mujk\right)^2.
\end{equation}

В ситуации, когда число эмпирических вейв\-лет-ко\-эф\-фи\-ци\-ен\-тов 
не случайно, вы\-чис\-ле\-ны оптимальные значения порогов и~оценен порядок 
сред\-не\-квад\-ра\-тич\-но\-го рис\-ка для различных классов функций сигналов. 
В~част\-ности, в~работе~\cite{DJ95} доказано сле\-ду\-ющее утверж\-де\-ние, 
оценивающее минимаксный 
порядок~\eqref{Risk_Definition}.

\smallskip

\noindent
\textbf{Теорема~1.}\ \textit{При выборе асимптотически оптимального 
порога для жест\-кой и~мяг\-кой пороговой обработки справедливо соотношение}:
$$
\sup\limits_{f\in \mathrm{Lip}\left(A,\gamma\right)}{r_J(f)}\leqslant 
C_M2^{-({2\gamma}/({2\gamma+1}))J}
J^{{2\gamma}/({2\gamma+1})}\,,
$$
\textit{где $C_M$~--- некоторая положительная константа}.

\smallskip

Асимптотически оптимальный порог в~тео\-ре\-ме~1 при $J\hm\to\infty$ удовлетворяет 
соотношению~\cite{Jan01}:
$$
T\simeq\sigma\sqrt{\fr{4\gamma}{2\gamma+1}\ln2^J}\,.
$$

% Таким образом, с~точностью до множителя, имеющего не более чем логарифмический порядок от числа вейвлет-коэффициентов, порядок функции потерь равен $O(2^{-\frac{2\gamma}{2\gamma+1}J})$.

В следующем разделе оценивается порядок сред\-не\-квад\-ра\-тич\-но\-го риска 
пороговой обработки в~модели со случайным чис\-лом эмпирических вейв\-лет-ко\-эф\-фи\-ци\-ен\-тов.

\vspace*{-6pt}

\section{Случайное число вейвлет-коэффициентов}

Пусть $M$~--- положительная целочисленная случайная величина и~$N\hm=2^M$. 
Тогда сред\-не\-квад\-ра\-тич\-ный риск для 
модели~\eqref{data_model} принимает вид:

\noindent
\begin{multline}
\label{Risk_Definition_Rand}
r(f)={}\\
{}=\sum\limits_{J=0}^{\infty}{\sf P}\left(N=2^J\right)
\fr{1}{2^J}\sum\limits_{j=0}^{J-1}\sumk\limits 
{\sf E}\left(\hat{Y}_{j,k}-\mujk\right)^2
\end{multline}
и его асимптотический порядок в~значительной степени 
зависит от распределения~$N$. Чтобы получить осмысленные оцен\-ки порядка 
рис\-ка~\eqref{Risk_Definition_Rand}, величина~$N$ долж\-на быть <<большой>>. 
Рас\-смот\-рим по\-сле\-до\-ва\-тель\-ность~$N_n$, $n\hm=1, 2,\ldots$,  и~предположим, 
что существует неслучайная возрастающая по\-сле\-до\-ва\-тель\-ность 
натуральных чисел~$J_n$, $n\hm=1, 2,\ldots$, такая, что~$N_n/2^{J_n}$ 
имеет некоторый предел (в~смыс\-ле равномерной схо\-ди\-мости по распределению) при 
$n\hm\to\infty$, т.\,е.

\noindent
\begin{equation}
\label{converge}
\sup\limits_{x\geqslant0}\abs{H_n(x)-H(x)}<\fr{\eps_n}{2}\to0\,, \enskip  n\to\infty\,,
\end{equation}
где

\noindent
\begin{equation*}
H_n(x)={\sf P}\left(\fr{N_n}{2^{J_n}}<x\right)\,,
\end{equation*}
а $H(x)$~--- предельная функция распределения. Предположим, что~$H(x)$ 
не имеет атома в~нуле и~исследуем поведение
\begin{equation*}
%\label{Risk_Definition_Rand2}
r_n(f)=\sum\limits_{J=0}^{\infty}{\sf P}\left(N_n=2^J\right)\fr{1}{2^J}
\sum\limits_{j=0}^{J-1}\sumk\limits 
{\sf E}\left(\hat{Y}_{j,k}-\mujk\right)^2 
\end{equation*}
при $n\hm\to\infty$.

\pagebreak

Пусть $\delta_n\to0$, $\alpha_n\hm\to0$ при $n\hm\to\infty$ так, 
что $J_n\hm+\log_2\delta_n\hm\to\infty$ и~$H(\delta_n)\hm+1\hm-H(\delta^{-1}_n)
\hm<\alpha_n$ при всех $n\hm=1, 2,\ldots$
% Условие $J_n+\log_2\delta_n\to\infty$ гарантирует, что \log_2\delta_n=o(J_n) или O(J_n) и~не повлияет на порядок главных множителей в~средней сумме. Можно было бы взять и~другое поведение $H(x)$ на бесконечности, а не завязывать на поведение в~нуле. Было бы более громоздко, но принципиально не отличалось бы.
Тогда

\noindent
\begin{multline*}
r_n(f)={}\\
\!{}=\sum\limits_{J=0}^{[J_n+\log_2\delta_n]}\hspace*{-5mm}
{\sf P}\left(N_n=2^J\right)\fr{1}{2^J}\sum\limits_{j=0}^{J-1}\sumk\limits 
{\sf E}\left(\hat{Y}_{j,k}-\mujk\right)^2+{}\\
{}+\hspace*{-2.1mm}\sum\limits_{J=[J_n+\log_2\delta_n]+1}^{[J_n-\log_2\delta_n]}\hspace*{-8mm}
{\sf P}\left(N_n=2^J\right)\fr{1}{2^J}\!\sum\limits_{j=0}^{J-1}\sumk\limits 
{\sf E}\left(\hat{Y}_{j,k}-\mujk\right)^{\!2}+{}\\
+\hspace*{-2.1mm}\sum\limits_{J=[J_n-\log_2\delta_n]+1}^{\infty}\hspace*{-8.1mm}{\sf P}
\left(N_n=2^J\right)\fr{1}{2^J}\!\sum\limits_{j=0}^{J-1}\sumk\limits 
{\sf E}\left(\hat{Y}_{j,k}-\mujk\right)^{\!2}\!\equiv{}\hspace*{-0.3115pt}\\
{}\equiv S_1+S_2+S_3\,.
\end{multline*}
Учитывая~\eqref{converge}, для $S_1\hm+S_3$ имеем
\begin{equation*}
S_1+S_3\leqslant H_n\left(\delta_n\right)+1-H_n\left(\delta^{-1}_n\right)
\leqslant\alpha_n+\eps_n\,. %\notag
\end{equation*}
Для $S_2$ с~помощью тео\-ре\-мы~1 можно получить \mbox{оценку}:

\noindent
\begin{multline*}
S_2\leqslant{}\\
\!{}\leqslant C_1 2^{-({2\gamma}/({2\gamma+1}))(J_n+\log_2\delta_n)}
\left(J_n+\log_2\delta_n\right)^{{2\gamma}/({2\gamma+1})},\hspace*{-7.95976pt}
\end{multline*} 
% В S_2 вид оптимального порога в~пределах суммирования не влияет на порядок логарифмического множителя, и~начиная с~некоторого n максимальным будет первое слагаемое в~S_2. Слагаемые в~сумме риска не превосходят $C min(T^2,с\mu^2+\rho(0))$ (\rho(0) слагаемое риска с~нулевым \mu). Эта сумма разбивается на две, в~первой (S_21) оцениваются T^2, во второй \rho(0) (Cai со случайными отсчетами, DJ94 формулы 37-39), и~в пределах суммирования в~S_2 вид T^2 влияет только на константу в~S_21. Порог надо брать как можно меньше, т.к. порядок риска не ухудшается, а вероятность задеть полезный сигнал ниже. Итак, в~S_2 заменяем все слагаемые на первое, но есть еще p_n и~их сумма <1. Таким образом вся S_2 имеет меньший (или равный) порядок, как ее первое слагаемое. См. Kushe5 и~Cai со случайными отсчетами.
справедливую для всех $f\hm\in \mathrm{Lip}\,(A,\gamma)$ (здесь $C_1$~--- 
некоторая положительная константа). Таким образом, справедливо сле\-ду\-ющее утверж\-дение.

\smallskip


\noindent
\textbf{Теорема~2.}\ \textit{В~модели со случайным чис\-лом вейв\-лет-ко\-эф\-фи\-ци\-ен\-тов 
при выборе асимптотически оптимального порога начиная с~некоторого~$n$ 
справедлива оценка}:

\noindent
\begin{multline*}
\sup_{f\in Lip(A,\gamma)}{r_n(f)}\leqslant 
\alpha_n+\eps_n+{}\\
{}+C_1 2^{-({2\gamma}/({2\gamma+1}))
(J_n+\log_2\delta_n)}(J_n+\log_2\delta_n)^{{2\gamma}/({2\gamma+1})}.
\end{multline*}
\textit{Сам асимптотически оптимальный порог при $n\hm\to\infty$ 
удовлетворяет соотношению}:
$$
T_n\simeq\sigma\sqrt{\fr{4\gamma}{2\gamma+1}\ln 2^{J_n+\log_2\delta_n}}\,.
$$
% этот порог минимальный для S_2. рассуждения см. выше.

{Вид~$\alpha_n$, $\eps_n$ и~$\delta_n$ в~тео\-ре\-ме~$2$ существенно зависит от поведения 
по\-сле\-до\-ва\-тель\-ности~$N_n/2^{J_n}$ и~предельной функции распределения~$H(x)$. 
Так, $\eps_n$ характеризует ско\-рость сходимости~$H_n(x)$ к~$H(x)$, 
а~$\alpha_n$ и~$\delta_n$ зависят от поведения~$H(x)$ в~окрест\-ности 
нуля и~бес\-ко\-неч\-ности}.

\smallskip

\noindent
\textbf{Следствие~1.}\ Если предельное распределение~$N_n/2^{J_n}$ вы\-рож\-ден\-но: 
$N_n/2^{J_n}\xrightarrow{{\sf P}}1$ при $n\hm\to\infty$, тогда начиная с~некоторого~$n$

\noindent
\begin{equation*}
\sup_{f\in Lip(A,\gamma)}{r_n(f)}
\leqslant \eps_n+C_2 2^{-({2\gamma}/({2\gamma+1}))J_n}
J_n^{{2\gamma}/({2\gamma+1})}\,,
\end{equation*}
где $C_2$~--- некоторая положительная константа.

Если $\eps_n$ убывает достаточно быст\-ро, то эта оценка сов\-па\-да\-ет с~оценкой 
для сред\-не\-квад\-ра\-тич\-но\-го рис\-ка в~модели с~неслучайным чис\-лом 
вейв\-лет-ко\-эф\-фи\-ци\-ен\-тов.

\smallskip

\noindent
\textbf{Следствие~2.}\ Пусть~$H(x)$ дифференцируема в~окрест\-ности нуля и~для 
некоторых положительных констант~$b$ и~$B$ в~этой окрест\-ности 
выполнено $b\hm\leqslant H'(x)\hm\leqslant B$. Пусть
$\delta_n=2^{-({2\gamma}/({4\gamma+1}))J_n}.$ Тогда  начиная с~некоторого~$n$ 
справедлива оценка:
\begin{multline}
\label{hard_col1}
\sup_{f\in Lip(A,\gamma)} r_n(f)\leqslant {}\\
{}\leqslant
\eps_n+C_3 2^{-({2\gamma}/({4\gamma+1}))J_n}
J_n^{{2\gamma}/({2\gamma+1})},
\end{multline}
где $C_3$~--- некоторая положительная константа.
% В условиях следствия 2 степень $-\frac{2\gamma}{4\gamma+1}$ неулучшаема.

Заметим, что ${2\gamma}/({4\gamma+1})<{1}/{2}$ 
при $\gamma\hm>0$, и~если $\eps_n\hm=O(2^{-J_n/2})$ (что 
является распространенной оценкой для ско\-рости схо\-ди\-мости 
по распределению), то второе сла\-га\-емое в~\eqref{hard_col1} определяет 
ско\-рость убывания сред\-не\-квад\-ра\-тич\-но\-го рис\-ка. Таким образом, среднеквадратичный 
риск для случайного числа вейв\-лет-ко\-эф\-фи\-ци\-ен\-тов может стремиться к~нулю 
значительно медленнее, чем сред\-не\-квад\-ра\-тич\-ный риск для неслучайного чис\-ла 
вейв\-лет-ко\-эф\-фи\-ци\-ен\-тов.

\vspace*{-9pt}

{\small\frenchspacing
 {%\baselineskip=10.8pt
 \addcontentsline{toc}{section}{References}
 \begin{thebibliography}{9}
 
\bibitem{DJ94}
\Au{Donoho D., Johnstone~I.\,M.} 
Ideal spatial adaptation via wavelet shrinkage~// Biometrika, 1994. 
Vol.~81. No.\,3. P.~425--455.

\bibitem{DJ95}
\Au{Donoho D., Johnstone~I.\,M., Kerkyacharian~G., Picard~D.} 
Wavelet shrinkage: Asymptopia?~// J.~Roy. Stat. Soc.~B, 1995. 
Vol.~57. No.\,2. P.~301--369.

\bibitem{DJ98}
\Au{Donoho D., Johnstone~I.\,M.} Minimax estimation via wavelet shrinkage~// 
Ann. Stat., 1998. Vol.~26. No.\,3. P.~879--921.

\bibitem{Jan01}
\Au{Jansen M.} Noise reduction by wavelet thresholding.~--- Lecture notes in statistics
ser.~---
New York, NY, USA: Springer Verlag, 2001.  Vol.~161. 196~p.

\bibitem{Jan06}
\Au{Jansen M.} Minimum risk thresholds for data with heavy noise~// 
IEEE Signal Proc. Let., 2006. Vol.~13. No.\,5. P.~296--299.

\bibitem{SH17}
\Au{Шестаков О.\,В.} Минимаксный среднеквадратичный риск пороговой обработки 
в~моделях с~негауссовым распределением шума~// Вестн. Моск. ун-та. Сер.~15: 
Вычисл. матем. и~киберн., 2017. №\,4. C.~35--40.

\bibitem{Mal99}
\Au{Mallat S.} A~wavelet tour of signal processing.~--- New York, NY, USA:
Academic Press, 1999. 857~p.
 \end{thebibliography}

 }
 }

\end{multicols}

\vspace*{-6pt}

\hfill{\small\textit{Поступила в~редакцию 10.05.18}}

%\vspace*{6pt}

\newpage

\vspace*{-24pt}

%\hrule

%\vspace*{2pt}

%\hrule

\vspace*{-2pt}


\def\tit{MEAN-SQUARE THRESHOLDING RISK\\ WITH~A~RANDOM~SAMPLE~SIZE}

\def\titkol{Mean-square thresholding risk with~a~random sample size}

\def\aut{O.\,V.~Shestakov$^{1,2}$}

\def\autkol{O.\,V.~Shestakov}

\titel{\tit}{\aut}{\autkol}{\titkol}

\vspace*{-11pt}


\noindent
$^1$Department of Mathematical Statistics, Faculty of Computational Mathematics 
and Cybernetics, M.\,V.~Lo-\linebreak
$\hphantom{^1}$monosov Moscow State University, 1-52~Leninskiye Gory, 
GSP-1, Moscow 119991, Russian Federation

\noindent
$^2$Institute of Informatics Problems, Federal Research Center ``Computer 
Science and Control'' of the Russian\linebreak
$\hphantom{^1}$Academy of Sciences, 44-2~Vavilov Str., 
Moscow 119333, Russian Federation


\def\leftfootline{\small{\textbf{\thepage}
\hfill INFORMATIKA I EE PRIMENENIYA~--- INFORMATICS AND
APPLICATIONS\ \ \ 2018\ \ \ volume~12\ \ \ issue\ 3}
}%
 \def\rightfootline{\small{INFORMATIKA I EE PRIMENENIYA~---
INFORMATICS AND APPLICATIONS\ \ \ 2018\ \ \ volume~12\ \ \ issue\ 3
\hfill \textbf{\thepage}}}

\vspace*{3pt}




\Abste{Nonlinear methods of signal de-noising based on the threshold 
processing of wavelet coefficients are widely used in various application areas. 
These methods have gained their popularity due to the speed of the algorithms 
for constructing estimates and the possibility of adapting to functions belonging 
to different classes of regularity better than linear methods. When applying 
thresholding techniques, it is usually assumed that the number of wavelet 
coefficients is fixed and the noise distribution is Gaussian. This model 
has been well studied in the literature, and optimal threshold values have 
been calculated for different classes of signals. However, in some situations, 
the sample size is not known in advance and is modeled by a random variable. 
The present author considers a~model with a random number of 
observations containing Gaussian noise and estimates the order of the mean-square 
risk with increasing sample size.}

\KWE{thresholding; random sample size; mean-square risk}




\DOI{10.14357/19922264180302}

%\vspace*{-14pt}

 \Ack
\noindent
This research is supported by the Russian Science Foundation (project No.\,18-11-00155).



%\vspace*{6pt}

  \begin{multicols}{2}

\renewcommand{\bibname}{\protect\rmfamily References}
%\renewcommand{\bibname}{\large\protect\rm References}

{\small\frenchspacing
 {%\baselineskip=10.8pt
 \addcontentsline{toc}{section}{References}
 \begin{thebibliography}{9}

\bibitem{1-s}
\Aue{Donoho,~D., and I.\,M.~Johnstone.} 1994. 
Ideal spatial adaptation via wavelet shrinkage. \textit{Biometrika} 81(3):425--455.

\bibitem{2-s}
\Aue{Donoho, D., I.\,M.~Johnstone, G.~Kerkyacharian, and D.~Picard.} 1995. 
Wavelet shrinkage: Asymptopia? \textit{J.~Roy. Stat. Soc.~B} 57(3):301--369.

\bibitem{3-s}
\Aue{Donoho, D., and I.\,M.~Johnstone.} 1998. 
Minimax estimation via wavelet shrinkage. \textit{Ann. Stat.}  26(3):879--921.

\bibitem{4-s}
\Aue{Jansen, M.} 2001. \textit{Noise reduction by wavelet thresholding.}
 Lecture notes in statistics ser. New York, NY: Springer Verlag.  Vol.~161. 196~p.

\bibitem{5-s}
\Aue{Jansen, M.} 2006. Minimum risk thresholds for data with heavy noise. 
\textit{IEEE Signal Proc. Let.} 13(5):296--299.

\bibitem{6-s}
\Aue{Shestakov, O.\,V.} 2017. 
Minimax mean-square thresholding risk in models with non-Gaussian noise distribution. 
\textit{Mosc. Univ. Comput. Math. Cybern.} 41(4):187--192.

\bibitem{7-s}
\Aue{Mallat, S.} 1999. 
\textit{A~wavelet tour of signal processing}. New York, NY: Academic Press. 857~p.

\end{thebibliography}

 }
 }

\end{multicols}

\vspace*{-6pt}

\hfill{\small\textit{Received May 10, 2018}}

%\pagebreak

\vspace*{-18pt}


\Contrl


\noindent
\textbf{Shestakov Oleg V.} (b.\ 1976)~--- 
Doctor of Science in physics and mathematics, associate professor, 
Department of Mathematical Statistics, Faculty of Computational Mathematics 
and Cybernetics, M.\,V.~Lomonosov Moscow State University, 1-52~Leninskiye Gory, GSP-1, 
Moscow 119991, Russian Federation; senior scientist, 
Institute of Informatics Problems, Federal Research Center 
``Computer Science and Control'' of the Russian Academy of Sciences, 
44-2~Vavilov Str., Moscow 119333, Russian Federation; \mbox{oshestakov@cs.msu.su}
\label{end\stat}


\renewcommand{\bibname}{\protect\rm Литература} 