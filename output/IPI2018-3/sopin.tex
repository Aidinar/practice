\def\stat{sopin}

\def\tit{ОБ ИНВАРИАНТНОСТИ СТАЦИОНАРНОГО РАСПРЕДЕЛЕНИЯ СИСТЕМЫ МАССОВОГО ОБСЛУЖИВАНИЯ 
С~ОГРАНИЧЕННЫМИ РЕСУРСАМИ И~С~ИНТЕНСИВНОСТЯМИ ПОСТУПЛЕНИЯ  
И~ОБСЛУЖИВАНИЯ, ЗАВИСЯЩИМИ ОТ~СОСТОЯНИЯ СИСТЕМЫ$^*$}

\def\titkol{Об инвариантности стационарного распределения СМО %системы массового обслуживания 
с~ограниченными ресурсами} % и~с~интенсивностями поступления   и~обслуживания, зависящими от~состояния системы}

\def\aut{Э.\,С. Сопин$^1$, В.\,А.~Наумов$^2$, К.\,Е.~Самуйлов$^3$}

\def\autkol{Э.\,С. Сопин, В.\,А.~Наумов, К.\,Е.~Самуйлов}

\titel{\tit}{\aut}{\autkol}{\titkol}

\index{Сопин Э.\,С.}
\index{Наумов В.\,А.}
\index{Самуйлов К.\,Е.}
\index{Sopin E.\,S.}
\index{Naumov V.\,A.} 
\index{Samouylov K.\,E.}




{\renewcommand{\thefootnote}{\fnsymbol{footnote}} 
\footnotetext[1]
{Публикация подготовлена при финансовой 
поддержке Минобрнауки России (проект 2.882.2017/4.6).}}


\renewcommand{\thefootnote}{\arabic{footnote}}
\footnotetext[1]{Российский университет дружбы народов, Институт проблем информатики Федерального 
исследовательского центра <<Информатика и~управ\-ле\-ние>> Российской академии наук, 
\mbox{sopin\_es@rudn.university}}
\footnotetext[2]{Исследовательский институт инноваций, г.~Хельсинки, Финляндия, \mbox{valeriy.naumov@pfu.fi}}
\footnotetext[3]{Российский университет дружбы народов; Институт проблем 
информатики Федерального исследовательского центра <<Информатика 
и~управ\-ле\-ние>> Российской академии наук, 
\mbox{samouylov\_ke@rudn.university}}

\vspace*{-12pt}
  
     
     
  
  \Abst{Рассматривается дальнейшее обобщение сис\-тем массового обслуживания (СМО), 
в~которых заявкам для обслуживания требуется не только прибор, но и~некоторый объем 
ресурсов, суммарный объем которых ограничен. В~рассматриваемой сис\-те\-ме 
интенсивности поступления и~обслуживания заявок зависят от состояния сис\-те\-мы, при 
этом объемы работ, приносимых заявками, имеют произвольное распределение 
с~конечным средним. Сформулирована и~доказана тео\-ре\-ма 
о~мультипликативном виде стационарного распределения вероятностей при 
пуассоновском входящем потоке, зависящем от числа заявок в~сис\-те\-ме. 
Показано, что стационарные вероятности числа заявок в~сис\-те\-ме и~объемов занятых ими 
ресурсов, аналогично сис\-те\-ме Эрланга с~потерями, не зависят от вида функции 
распределения (ФР) объема работ, а~зависят только от математического ожидания.}
  
  \KW{ограниченные ресурсы; система массового обслуживания; инвариантность; время 
обслуживания}

\DOI{10.14357/19922264180306}
  
\vspace*{-2pt}


\vskip 10pt plus 9pt minus 6pt

\thispagestyle{headings}

\begin{multicols}{2}

\label{st\stat}
  

\section{Введение}

\vspace*{-2pt}

  Обобщение многолинейной модели Эрланга с~потерями в~виде СМО, 
  в~которых заявка на все время 
обслуживания занимает прибор и~случайный объем ограниченных ресурсов, 
все чаще применяется для анализа современных вычислительных 
и~телекоммуникационных систем. К~примеру, при построении и~анализе 
моделей перспективных беспроводных сетей основная сложность 
заключается в~том, что объем используемых час\-тот\-но-вре\-мен\-н$\acute{\mbox{ы}}$х 
ресурсов одной пользовательской сессией зависит не только от типа услуги, 
но и~массы других факторов, влияющих на распространение радиосигнала, 
таких как расстояние между мобильным устройством и~базовой станцией, 
наличие дополнительных препятствий между ними, используемая  
ко\-до\-во-мо\-ду\-ля\-ци\-он\-ная схема и~др. Сис\-те\-мы массового обслуживания с~ограниченными 
ресурсами позволяют учесть все эти факторы и~описывают 
функционирование таких сис\-тем с~высокой степенью адекватности~[1]. 
Первые результаты для сис\-тем с~ограниченными ресурсами были получены 
в~работе~[2]. Рассматриваемая в~настоящей статье ресурсная СМО, 
в~которой интенсивности поступления и~обслуживания заявок зависят от 
состояния сис\-те\-мы, была изучена в~[3] в~случае экспоненциального 
распределения времени обслуживания и~пуассоновского входящего потока.
  
  Впервые свойство инвариантности было доказано для сис\-те\-мы Эрланга 
с~потерями~[4, 5]. В~дальнейшем было показано, что свойство 
инвариант\-ности справедливо для класса <<симметричных>> СМО, 
примерами которых служат однолинейная СМО с~дисциплиной разделения 
процессора и~бес\-ко\-неч\-но-ли\-ней\-ная СМО. Для более полного обзора 
инвариантных сис\-тем читателю предлагается ознакомиться 
с~работой~\cite{7-sop}.
  
  Для СМО с~ограниченными ресурсами и~постоянными интенсивностями 
поступления и~обслуживания заявок свойство инвариантности было доказано 
в~\cite{6-sop}. В~данной работе доказывается свойство инвариантности 
сис\-те\-мы из~\cite{3-sop} относительно вида распределения времени 
обслуживания. За основу доказательства свойства инвариантности более 
общей сис\-те\-мы была взята идея доказательства из~\cite{7-sop}.

\vspace*{-6pt}

\section{Модель системы в~случае экспоненциального 
распределения времени обслуживания}

  Рассмотрим многолинейную СМО
с~$N\hm<\infty$ приборами. Предположим, что поступающий поток является 
пуассоновским с~параметром~$\lambda_k$, зависящим от чис\-ла~$k$~заявок 
в~сис\-те\-ме, а~объемы работ, которые необходимо выполнить для 
обслуживания заявок, независимы между собой и~от поступающего потока 
и~экспоненциально распределены с~параметром $\mu\hm=1/b$. 
Предположим также, что каждый прибор обслуживает заявки с~постоянной, 
зависящей от числа~$k$ заявок в~сис\-те\-ме ско\-ростью~$\sigma_k$. Сис\-те\-ма 
располагает ограниченным объемом ресурсов~$M$~типов и~функционирует 
сле\-ду\-ющим образом:
  \begin{enumerate}[1.]
\item Каждой находящейся в~сис\-те\-ме заявке требуется один прибор 
и~некоторый объем ресурса каждого типа.
  \item Поступившая заявка теряется, если в~момент поступления объем 
требуемого ей ресурса превышает объем свободного ресурса этого типа либо 
в~сис\-те\-ме нет свободных приборов. 
  \item В момент поступления заявки объем свободного ресурса каждого 
типа уменьшается на величину ресурса, выделенного этой заявке.
  \item В момент ухода заявки объем свободного ресурса каждого типа 
увеличивается на величину ресурса, выделенного этой заявке.
  \end{enumerate}
  
  Обозначим~$R_m$ общий объем ресурса типа~$m$ и~$\mathbf{r}_j\hm= 
\left( r_{j1}, r_{j2},\ldots , r_{jM}\right)$~--- вектор объемов ресурсов, 
необходимых $j$-й поступившей заявке, $j\hm\geq 1$. Будем считать, что 
случайные векторы~$\mathbf{r}_j$ не зависят от процесса поступления 
и~обслуживания заявок, независимы в~совокупности и~одинаково 
распределены с~ФР~$F(\mathbf{x})$.
  
  Будем считать, что поступившие заявки располагаются в~очереди 
в~порядке поступления и~опишем состояние сис\-те\-мы в~момент~$t$ 
процессом $X(t)\hm= (\xi(t), \gamma(t))$. Здесь $\xi(t)$~--- число заявок 
в~сис\-те\-ме и~$\gamma(t)\hm= \left( \gamma_1(t), \gamma_2(t), \ldots, 
\gamma_{\xi(t)}(t)\right)$, где $\gamma_i(t)$~--- вектор объемов ресурсов, 
занимаемых $i$-й заявкой. Со\-сто\-яние сис\-те\-мы может измениться только 
в~моменты~$t_i$, когда либо в~сис\-те\-му поступает, либо ее покидает заявка. 
  
  Введем обозначения для стационарных вероятностей процесса ~$X(t)$:
  
  \noindent
   \begin{equation}
   p_0 = \lim\limits_{t\to\infty} {\sf P}(\xi(t)=0)\,;\label{e1-sop}
   \end{equation}
   
   \noindent
   \begin{multline}
   p_k\left(\mathbf{r}_1, \ldots, \mathbf{r}_k\right) =
    \lim\limits_{t\to\infty} {\sf P}\left( \xi(t)=k, \right.\\
\left.\gamma_1(t)\leq \mathbf{r}_1,\ldots, \gamma_k(t)\leq \mathbf{r}_k\right)\,,\enskip 
1\leq k\leq 
N\,.\label{e2-sop}
   \end{multline}
  
  Выражения для стационарного распределения~(1), (2) имеют сле\-ду\-ющий 
вид:

\noindent
  \begin{equation*}
  p_0=\left( 1+\sum\limits^N_{k=1} F^{(k)} (\mathbf{R}) 
\fr{b^k}{k!}\prod\limits^k_{i=1} \fr{\lambda_{i-1}}{\sigma_i} \right)^{-
1}\,;
%\label{e3-sop}
  \end{equation*}
  
  \vspace*{-16pt}
  
  \noindent
  \begin{multline}
  p_k\left( \mathbf{r}_1, \mathbf{r}_2, \ldots, \mathbf{r}_k\right) ={}\\
  {}=p_0 F\left( 
\mathbf{r}_1\right) F\left( \mathbf{r}_2\right) \cdots F\left( \mathbf{r}_k\right) 
\fr{b^k}{k!}\prod\limits^k_{i=1} \fr{\lambda_{i-1}}{\sigma_i}\,,\\
  1\leq k\leq N\,,\ \mathbf{r}_1\geq \mathbf{0}\,,\ldots, \mathbf{r}_k\geq \mathbf{0}\,,\enskip 
\sum\limits^k_{i=1} \mathbf{r}_i\leq \mathbf{R}\,,
  \label{e4-sop}
  \end{multline}
  где $F^{(k)}(\mathbf{x})$ есть $k$-крат\-ная свертка 
ФР~$F(\mathbf{x})$.

\vspace*{-9pt}
  
\section{Свойство инвариантности относительно вида 
распределения времени обслуживания}

\vspace*{-2pt}

  Рассмотрим поведение системы в~случае, когда распределение объемов 
работ не является экспоненциальным, а распределено в~соответствии 
с~ФР~$B(x)$ с~таким же математическим ожиданием~$b$. Будем считать, 
что распределение времени обслуживания имеет плотность~$b(x)$ на 
полуоси $[0,\infty)$, однако заметим, что опираясь на свойства слабой 
непрерывности, как в~\cite{8-sop}, полученные результаты можно доказать 
и~в~более общем случае. Состояние системы в~момент~$t$ описывается 
случайным процессом $\tilde{X}(t)\hm= (\xi(t), \gamma(t), \beta(t))$. Здесь, как 
и~прежде, $\xi(t)$~--- число заявок в~сис\-те\-ме, $\gamma(t)$ описывает объемы 
ресурсов, занимаемых каждой заявкой, а~\mbox{третья} компонента $\beta(t)\hm= 
\left( \beta_1(t), \beta_2(t),\ldots , \beta_{\xi(t)}(t)\right)$, где~$\beta_i(t)$~--- 
объем обслуженной работы $i$-й заявки. При поступлении заявки с~вектором 
требований ресурсов~$\mathbf{r}$ система переходит из состояния $\left( k, 
\left( \mathbf{r}_1,\ldots\right.\right.$\linebreak
$\left.\left.\ldots, \mathbf{r}_k\right), \left( x_1, \ldots , x_k\right)\right)$ 
в~одно из состояний $\left( k+1,\right.$\linebreak
$\left. \left( \mathbf{r}_1,\, \ldots\, , \mathbf{r}_{i-1}, 
\mathbf{r}, \mathbf{r}_i,\, \ldots\, , \mathbf{r}_k\right), \left( x_1,\, \ldots,\, x_{i-1}, 0, 
 x_i,\  \ldots\ \right.\right.$\linebreak $\left.\left.\ldots, x_k\right)\right)$, $0\hm\leq i\hm\leq k$, с~вероятностью 
$1/(k\hm+1)$. Обозначим 

\noindent
  \begin{equation}
  q_0(t)={\sf P}(\xi(t)=0)\,;\label{e5-sop}
  \end{equation}
  
  \vspace*{-12pt}
  
  \noindent
  \begin{multline}
  Q_k(\mathbf{r}_1,\ldots, \mathbf{r}_k, x_1, \ldots , x_k;t) ={}\\
  {}={\sf P}\left( \xi(t)=k,\ 
\gamma_1(t)\leq \mathbf{r}_1,\ldots ,\ \gamma_k(t)\leq \mathbf{r}_k,\right.\\
 \left.
  \beta_1(t)<x_1,\ldots ,\ \beta_k(t)<x_k\right)\,,\enskip
  1\leq k\leq N\,,
  \label{e6-sop}
  \end{multline}
вероятности состояний системы в~момент времени~$t$ 
и~$q_k(\mathbf{r}_1,\ldots, \mathbf{r}_k, x_1,\ldots , x_k;t)$~--- плотности 
вероятностей $Q_k(\mathbf{r}_1,\ldots, \mathbf{r}_k, x_1, \ldots , x_k;t)$, 
$k\hm>0$.

  Выпишем уравнения изменения $k$-мер\-ных плотностей 
$q_k(\mathbf{r}_1,\ldots, \mathbf{r}_k, x_1,\ldots , x_k;t)$,  в~точке 
$\mathbf{r}_1, \ldots, \mathbf{r}_k, x_1, \ldots , x_k$ за малое время~$\Delta t$. 
Для вероятностей пустой системы справедливо равенство:

\noindent
  \begin{multline}
  q_0(t+\Delta t) =q_0(t) \left( 1-\lambda_0 F(\mathbf{R}) \Delta t\right) +{}\\
  {}+\int\limits_{0\leq \mathbf{r}\leq \mathbf{R}} \int\limits^\infty_0 
q_1(d\mathbf{r},x;t) \fr{b(x) \sigma_1\Delta t}{1-B(x)}\,dx +o(\Delta t)\,.
  \label{e7-sop}
  \end{multline}
Первое слагаемое в~правой части уравнения~(\ref{e7-sop})\linebreak
 соответствует 
случаю, при котором в~интервале $(t, t\hm+\Delta t)$ не было поступлений 
заявок, а~второе~--- окончанию обслуживания заявки на данном интервале. 
Множитель $b(x)\sigma_1\Delta t/(1\hm-B(x))$ имеет здесь смысл вероятности 
того, что на интервале $(t, t\hm+\Delta t)$ заявка будет обслужена при 
условии, что объем работы, принесенной заявкой, превышает~$x$. Выведем 
теперь уравнения для случая, когда в~сис\-те\-ме находится $0\hm< n\hm< N$ 
заявок:

\vspace*{-3pt}

\noindent
\begin{multline}
q_k\left( \mathbf{r}_1,\ldots,\mathbf{r}_k, x_1,\ldots ,x_k; t+\Delta t\right)={}\\[-1pt]
{}=\hspace*{-3mm}\int\limits_{\substack{{\mathbf{0}\leq \mathbf{s}_i\leq \mathbf{r}_i,}\\[-1pt]  {i=1,2,\ldots,k}}} 
\hspace*{-4mm}\hspace*{-2.7pt}q_k\left( d\mathbf{s}_1, \ldots, d\mathbf{s}_k, x_1-\sigma_k \Delta t,\ldots, x_k-\sigma_k\Delta t; 
t\right) \times{}\\[-1pt]
{}\times\left[ 1-\left( \lambda_k F \left( \mathbf{R} -\sum\limits^k_{i=1} \mathbf{s}_i\right) 
+\sum\limits^k_{i=1} \fr{b(x_i) \sigma_k}{1-B(x_i)}\right)\Delta t\right]+{}\\[-1pt]
{}+\sum\limits_{j=1}^{k+1} \int\limits_{\substack{{\mathbf{0}\leq \mathbf{s}_i\leq \mathbf{r}_i,}\\[-1pt]  
{i=1,2,\ldots,k,}\\ {\mathbf{s}\leq \mathbf{R}-\mathbf{s}_1-\cdots -\mathbf{s}_k}}}\!\!
\int\limits_0^\infty q_{k+1}\left( d\mathbf{s}_1, \ldots, d\mathbf{s}_{j-1}, d\mathbf{s}, d\mathbf{s}_j,\ldots\right.\\[-1pt]
\ldots, d\mathbf{s}_k, x_1-
\sigma_{k+1}\Delta t, \ldots, x_{j-1}-\sigma_{k+1}\Delta t, x,\\[-1pt]
\left. x_j-\sigma_{k+1} 
\Delta t, x_k-\sigma_{k+1}\Delta t; t\right)
 \fr{b(x) \sigma_{k+1}\Delta t}{1-B(x)}\,dx +{}\\[-3pt]
 {}+o(\Delta t)\,,\enskip 
\sum\limits^k_{i=1} \mathbf{r}_i \leq \mathbf{R}\,,\ 1\leq k\leq N-1\,.
\label{e8-sop}
\end{multline}
  
  Аналогично уравнению~(\ref{e7-sop}) первое слагаемое в~правой 
части~(\ref{e8-sop}) соответствует случаю, в~котором за интервал $(t, 
t\hm+\Delta t)$ не происходит поступлений новых заявок и~ни одна заявка не 
покидает систему, а~второе~--- окончанию обслуживания одной заявки. Если 
в~системе уже находится~$N$~заявок, то новых поступлений быть не может 
и~невозможен переход из состояний с~$N\hm+1$ заявками. Тогда уравнение 
упрощается:

\vspace*{-3pt}

\noindent
  \begin{multline}
  q_N\left( \mathbf{r}_1, \mathbf{r}_N, x_1,\ldots, x_N; t+\Delta t\right)={}\\
  {}=q_N\left( \mathbf{r}_1,\ldots, \mathbf{r}_N, x_1-\sigma_N \Delta t, \ldots, 
x_N-\sigma_N \Delta t;t\right)\times{}\\[-3pt]
\!\!{}\times\left[ 1-\sum\limits^N_{i=1} \fr{b(x_i)\sigma_k}{1-
B(x_i)}\,\Delta t\right] +o(\Delta t)\,,\enskip \sum\limits^N_{i=1} \mathbf{r}_i\leq 
\mathbf{R}\,.\!\!
  \label{e9-sop}
  \end{multline}
    Поделив левые и~правые части уравнений~(\ref{e7-sop})--(\ref{e9-sop}) 
на~$\Delta t$ и~устремив~$\Delta t$ к~нулю, получим:

\columnbreak

\noindent
  \begin{multline}
  \fr{\partial q_0(t)}{\partial t} +\lambda_0 q_0(t) F(\mathbf{R}) ={}\\[-1pt]
  {}=\sigma_1 
\int\limits_{0\leq \mathbf{r}\leq \mathbf{R}} \int\limits_0^\infty 
q_1(d\mathbf{r}, x; t) \fr{b(x)}{1-B(x)}\,dx\,;\label{e10-sop}
  \end{multline}

\vspace*{-12pt}

\noindent
  \begin{multline}
  \fr{\partial q_k(\mathbf{r}_1,\ldots, \mathbf{r}_k, x_1, \ldots, x_k;t)}{\partial 
t} +{}\\[-1pt]
{}+\sigma_k\sum\limits^k_{i=1}\fr{\partial q_k(\mathbf{r}_1,\ldots, 
\mathbf{r}_k, x_1, \ldots, x_k;t)}{\partial x_i}={}\\[-1pt]
  {}= - \!\!\!\int\limits_{\substack{{\mathbf{0}\leq 
  \mathbf{s}_i\leq \mathbf{r}_i,}\\  {i=1,2,\ldots,k}}} 
\!\!\!q_k(d\mathbf{s}_1,\ldots, d\mathbf{s}_k, x_1,\ldots, x_k;t)\times{}\\[-1pt]
{}\times\left( \lambda_k F\left( \mathbf{R}-
\sum\limits^k_{i=1} \mathbf{s}_i\right) +\sum\limits^k_{i=1} \fr{b(x_i) \sigma_k}{1-
B(x_i)}\right)+{}\\[-1pt]
  {}+\sum\limits_{j=1}^{k+1}\!
  \int\limits_{\substack{{\mathbf{0}\leq \mathbf{s}_i\leq \mathbf{r}_i,}\\[-1pt]  
  {i=1,2,\ldots,k,}\\ {\mathbf{s}\leq 
\mathbf{R}-\mathbf{s}_1-\cdots -\mathbf{s}_k}}} \!\!\!\int\limits_0^\infty q_{k+1}
  \left( d\mathbf{s}_1, \ldots, d\mathbf{s}_{j-1}, d\mathbf{s}, d\mathbf{s}_j,\ldots\right.\\[-1pt]
 \left. \ldots, d\mathbf{s}_k, x_1,\ldots, x_{j-1}, x, x_j, 
x_k;t\right) \fr{b(x)\sigma_{k+1}}{1-B(x)}\,dx\,,\\[-1pt]
  \sum\limits^k_{i=1} \mathbf{r}_i\leq \mathbf{R}\,,\enskip 1\leq k\leq N-1\,;
  \label{e11-sop}
  \end{multline}

\vspace*{-12pt}

\noindent
\begin{multline}
\fr{\partial q_N (\mathbf{r}_1,\ldots, \mathbf{r}_N, x_1, \ldots, x_N;t)}{\partial 
t} +{}\\[-1pt]
{}+\sigma_N \sum\limits^N_{i=1} \fr{\partial q_N(\mathbf{r}_1, \ldots, 
\mathbf{r}_N, x_1, \ldots, x_N;t)}{\partial x_i} ={}\\[-1pt]
{}=
-q_N\left( \mathbf{r}_1,\ldots, \mathbf{r}_N, x_1,\ldots, x_N;t\right) 
\sum\limits^N_{i=1} \fr{b(x_i)\sigma_N}{1-B(x_i)}\,,\\[-1pt] 
\sum\limits^N_{i=1} 
\mathbf{r}_i\leq \mathbf{R}\,.
\label{e12-sop}
\end{multline}
  
  Для вывода граничных условий рассмотрим поведение системы в~моменты 
поступления заявок:

\vspace*{-3pt}

\noindent
  \begin{multline*}
  \int\limits_0^{\sigma_k \Delta t} q_k\left( \mathbf{r}_1, \ldots,  
\mathbf{r}_{i-1}, \mathbf{r}, \mathbf{r}_j,\ldots , \mathbf{r}_k, x_1+\sigma_k 
\Delta t,\ldots\right.\\[-1pt]
\ldots, x_{i-1}+\sigma_k \Delta t, x, x_i+\sigma_k\Delta t,\ldots,  
x_{k-1}+\sigma_k \Delta t;\\
\left. t+\Delta t\right)dx=\fr{\lambda_k\Delta t}{k}F(\mathbf{r})\times{}\\[-1pt]
{}\times q_{k-1}\left( \mathbf{r}_1, \ldots, 
\mathbf{r}_{k-1}, x_1, \ldots, x_{k-1};t\right)+o(\Delta t)\,,\\[-1pt]
  \sum\limits^k_{i=1} \mathbf{r}_i\leq \mathbf{R}-\mathbf{r}\,,\enskip
   0<k\leq 
N\,.
  %\label{e13-sop}
  \end{multline*}

Аналогично, поделив на~$\Delta t$ и~устремив~$\Delta t$ к~нулю, получим:

\pagebreak

\noindent
\begin{multline}
\!\!\sigma_k q_k\left( \mathbf{r}_1, \ldots, \mathbf{r}_{i-1}, \mathbf{r}, 
\mathbf{r}_i, \ldots \mathbf{r}_k, x_1, \ldots, x_{i-1},0, x_i, \ldots\right.\\
\left.\ldots,  
x_{k-1};t\right)={}\\
{}= \fr{\lambda_k}{k}\,F(\mathbf{r})q_{k-1} \left(\mathbf{r}_1, \ldots,  
\mathbf{r}_{k-1}, x_1,\ldots, x_{k-1};t\right)\,,\\
\sum\limits^k_{i=1}\mathbf{r}_i\leq \mathbf{R}-\mathbf{r}\,,\enskip 0<k\leq N\,.
\label{e14-sop}
\end{multline}
  
  Чтобы получить уравнения для стационарного режима, приравняем к~нулю 
производные функций~(\ref{e5-sop})--(\ref{e6-sop}) по времени~$t$. 
В~результате уравнения~(\ref{e10-sop})--(\ref{e12-sop}) и~граничное 
условие~(\ref{e14-sop}) примут\linebreak вид:
  \begin{equation}
  \hspace*{-3mm}\lambda_0 q_0 F(\mathbf{R}) =\sigma_1 \hspace*{-2mm}\int\limits_{0\leq \mathbf{r}\leq 
\mathbf{R}} \int\limits_0^\infty q_1(d\mathbf{r},x)\fr{b(x)}{1-B(x)}\,dx\,;\!\!
  \label{e15-sop}
  \end{equation}
  
\vspace*{-12pt}

\noindent
\begin{multline}
  \sigma_k \sum\limits^k_{i=1} \fr{\partial q_k(\mathbf{r}_1, \ldots, 
\mathbf{r}_k, x_1,\ldots, x_k)}{\partial x_i}={}\\[2pt]
  {}= - \hspace*{-2mm}\int\limits_{\substack{{\mathbf{0}\leq 
  \mathbf{s}_i\leq \mathbf{r}_i,}\\  {i=1,2,\ldots,k}}} 
  \hspace*{-2mm}q_k (d\mathbf{s}_1,\ldots, d\mathbf{s}_k, x_1,\ldots, x_k) \times{}\\[2pt]
  {}\times\left( \lambda_k F\left( \mathbf{R}-
\sum\limits^k_{i=1} \mathbf{s}_i\right) +\sum\limits^k_{i=1} \fr{b(x_i)\sigma_k}{1-
B(x_i)}\right)+{}\\[2pt]
  {}+ \sum\limits^{k+1}_{j=1}
  \int\limits_{\substack{{\mathbf{0}\leq \mathbf{s}_i\leq \mathbf{r}_i,}\\[2pt]  
  {i=1,2,\ldots,k,}\\ {\mathbf{s}\leq 
\mathbf{R}-\mathbf{s}_1-\cdots -\mathbf{s}_k}}} \!\!\!\int\limits_0^\infty  q_{k+1}
  \left(d\mathbf{s}_1, \ldots, d\mathbf{s}_{j-1}, d\mathbf{s}, d\mathbf{s}_j,\ldots\right.\\[2pt]
\left.  \ldots, d\mathbf{s}_k, x_1,\ldots, x_{j-1},x,x_j, x_k\right)
  \fr{b(x)\sigma_{k+1}}{1-B(x)}\,dx\,,\\[2pt]
  \sum\limits^k_{i=1}\mathbf{r}_i\leq \mathbf{R}\,,\quad 1\leq k\leq N-1\,;
  \label{e16-sop}
  \end{multline}
  
%\vspace*{-12pt}

\noindent
  \begin{multline}
  \sigma_N \sum\limits^N_{i=1} \fr{\partial q_N(\mathbf{r}_1,\ldots, 
\mathbf{r}_N, x_1,\ldots, x_N)}{\partial x_i}={}\\
  {}= -q_N(\mathbf{r}_1,\ldots, \mathbf{r}_N, x_1,\ldots, x_N) 
\sum\limits^N_{i=1} \fr{b(x_i) \sigma_N}{1-B(x_i)}\,,\\ 
\sum\limits^N_{i=1} \mathbf{r}_i\leq \mathbf{R}\,;
  \label{e17-sop}
  \end{multline}
  
\vspace*{-12pt}

\noindent
  \begin{multline}
  \!\!\!\sigma_k q_k\left( \mathbf{r}_1,\ldots \mathbf{r}_{i-1}, \mathbf{r}, 
\mathbf{r}_i, \ldots, \mathbf{r}_k, x_1,\ldots, x_{i-1},0, x_i, \ldots\right.\\
\left.\ldots, 
x_{k-1}\right)={}\\
  {}=\fr{\lambda_k}{k}\,F(\mathbf{r}) q_{k-1}(\mathbf{r}_1, \ldots, 
\mathbf{r}_{k-1}, x_1,\ldots, x_{k-1})\,,\\
  \sum\limits^k_{i=1} \mathbf{r}_i \leq \mathbf{R} -\mathbf{r}\,,\enskip 
0<k\leq N\,.
  \label{e18-sop}
  \end{multline}
    Решение системы~(\ref{e15-sop})--(\ref{e18-sop}) задается выражениями:
    
    \columnbreak
    
    \noindent
  \begin{multline}
  q_k\left( \mathbf{r}_1, \ldots, \mathbf{r}_k, x_1, \ldots, x_k\right) ={}\\[-1pt]
  {}=q_0 
F(\mathbf{x}_1) \cdots F(\mathbf{x}_k) \fr{1}{k!} \prod\limits^k_{i=1} 
\fr{\lambda_{i-1}}{\sigma_i} \left( 1-B(x_i)\right)\,,\\[-1pt]
  0\leq k\leq N\,,\ \mathbf{x}_1\geq \mathbf{0},\ldots , \mathbf{x}_k\geq \mathbf{0},\ 
\sum\limits^k_{i=1} \mathbf{x}_i\leq \mathbf{R}\,.
  \label{e19-sop}
  \end{multline}
    Это следует из того, что для любой непрерывной функции~$B(x)$
    
    \noindent
  $$
  \fr{d(1-B(x))}{dx}=-b(x)\,,
  $$ 
и~тогда каждое $i$-е слагаемое в~левой части уравнений~(\ref{e16-sop})
и~(\ref{e17-sop}) равно $i$-му слагаемому в~первой сумме правой части. 
Кроме того, так как 

\noindent
$$ 
\int\limits_0^\infty b(x)\,dx=1\,,
$$
то

\vspace*{-9pt}

\noindent
\begin{multline*}
\hspace*{-1mm}\sum\limits^{k+1}_{j=1}\int\limits_{\substack{{\mathbf{0}\leq 
\mathbf{s}_i\leq \mathbf{r}_i,}\\[-1pt]  
{i=1,2,\ldots,k,}\\ {\mathbf{s}\leq \mathbf{R}-\mathbf{s}_1-\cdots -\mathbf{s}_k}}}\!\!
\int\limits_0^\infty  
q_{k+1}\left( d\mathbf{s}_1,\ldots, d\mathbf{s}_{j-1}, d\mathbf{s}, d\mathbf{s}_j,\ldots\right.\\[-1pt]
\left.\ldots, d\mathbf{s}_k, x_1,\ldots, x_{j-1}, x, 
x_j, x_k\right)\fr{b(x)\sigma_{k+1}}{1-B(x)}\,dx-{}\\[-1pt]
{}-\lambda_k F\left( \mathbf{R}-\sum\limits^k_{i=1} \mathbf{s}_i\right)\times{}\\[-1pt]
{}\times  
\int\limits_{\substack{{\mathbf{0}\leq \mathbf{s}_i\leq \mathbf{r}_i,}\\[-1pt]  
{i=1,2,\ldots,k}}}\hspace*{-4mm}
q_k\left( d\mathbf{s}_1,\ldots, d\mathbf{s}_k, x_1,\ldots, x_k\right)=0\,.
\end{multline*}
Таким образом, выражение~(\ref{e19-sop}) является решением 
системы~(\ref{e15-sop})--(\ref{e18-sop}). Доказательство того, что 
сис\-те\-ма~(\ref{e15-sop})--(\ref{e18-sop}) имеет единственное вероятностное 
решение, можно провести аналогично работе~\cite{9-sop}. Таким образом, 
доказана следующая теорема.

%\smallskip

\noindent
\textbf{Теорема.}\ \textit{Если объемы работ имеют ФР~$B(x)$ и~конечное 
математическое ожидание~$b$, то стационарное  
распределение}~(\ref{e6-sop}) \textit{процесса~$\tilde{X}(t)$ имеет вид}:

\vspace*{-6pt}

\noindent
\begin{multline*}
Q_k\left( \mathbf{r}_1,\ldots, \mathbf{r}_k, x_1,\ldots, x_k\right) ={}\\[-1pt]
{}=q_0 
F(\mathbf{r}_1)\cdots F(\mathbf{r}_k) \fr{b^k}{k!}\prod\limits^k_{i=1} 
\fr{\lambda_{i-1}}{\sigma_i} \,\tilde{B}(x_i)\,,\\[-1pt]
1\leq k\leq N\,, \ \mathbf{r}_1\geq \mathbf{0},\ldots , \mathbf{r}_k\geq \mathbf{0}\,,\ 
\sum\limits^k_{i=1} \mathbf{r}_i \leq \mathbf{R}\,,
%\label{e20-sop}
\end{multline*}
\textit{где $\tilde{B}(x)$ есть ФР остаточного объема 
работы}:

%\vspace*{-3pt}

\noindent
\begin{equation*}
\tilde{B}(x) =\fr{1}{b}\int\limits_0^x (1-B(y))\,dy\,.
%\label{e21-sop}
\end{equation*}

\noindent
    Отсюда, в~частности, следует, что формула для маргинальных 
вероятностей $Q_k(\mathbf{r}_1, \ldots, \mathbf{r}_k, \infty, \ldots, \infty)$ 
идентична выражению~(\ref{e4-sop}) для случая экспоненциального 
распределения объема работы. Таким образом, доказано, что стационарное 
распределение вероятностей СМО с~ограниченными ресурсами 
и~с~интенсивностями поступления и~обслуживания заявок, зависящими от 
состояния сис\-те\-мы, не зависит от вида распределения объема работы заявки, 
а~зависит только от математического ожидания.
  
  \section{Заключение}
  
  В работе был проведен анализ стационарного распределения СМО
   с~ограниченными ресурсами и~интенсивностями 
поступления и~обслуживания, зависящими от со\-сто\-яния сис\-те\-мы, в~случае 
произвольного распределения объема работы, приносимого заявкой. Была 
доказана теорема о~мультипликативном виде стационарного распределения 
системы. Кроме того, было показано, что стационарное распределение чис\-ла 
заявок в~системе и~объема занимаемых ими ресурсов не зависит от вида 
распределения объема работы заявки, а~зависит только от математического 
ожидания.
  
{\small\frenchspacing
 {%\baselineskip=10.8pt
 \addcontentsline{toc}{section}{References}
 \begin{thebibliography}{9}
\bibitem{1-sop}
\Au{Naumov V., Samouylov~K., Yarkina~N., Sopin~E., And\-re\-ev~S., Samuylov~A.} 
LTE performance analysis using queuing systems with finite resources and random 
requirements~//  7th Congress (International) on Ultra Modern 
Telecommunications and Control Systems and Workshops.~--- IEEE, 
2015. P.~100--103. doi: 10.1109/ICUMT.2015.7382412.
\bibitem{2-sop}
\Au{Тихоненко О.\,М., Климович~К.\,Г.} Анализ систем обслуживания 
требований случайной длины при ограниченном суммарном объеме~// 
Проблемы передачи информации, 2001. Т.~37. Вып.~1. С.~78--88.
\bibitem{3-sop}
\Au{Naumov V., Samouylov~K.} Analysis of multi-resource loss system with state 
dependent arrival and service rates~// Probab. Eng. Inform. 
Sc., 2017. Vol.~31. Iss.~4.  
P.~413--419. doi: 10.1017/S0269964817000079.
\bibitem{4-sop}
\Au{Севастьянов Б.\,А.} Эргодическая теорема для марковских процессов 
и~ее приложение к~телефонным системам с~отказами~// Теория вероятностей 
и~ее применения, 1957. Т.~2. Вып.~1. С.~106--116.
\bibitem{5-sop}
\Au{Гнеденко Б.\,В., Коваленко~И.\,Н.} Введение в~теорию массового 
обслуживания.~--- М.: Наука, 1966. 432~с.

\bibitem{7-sop} %6
\Au{Taylor P.\,G.} Insensitivity in stochastic models~//  Queueing 
networks~/ Eds. R.~Boucherie, N.~van Dijk.~--- International ser. in 
operations research \& management science.~--- Springer, 2011.
Vol.~154. P.~121--140. doi:  
10.1007/978-1-4419-6472-4.

\bibitem{6-sop} %7
\Au{Samouylov K.\,E., Gaidamaka~Y.\,V., Sopin~E.\,S.} Simplified analysis of 
queueing systems with random requirements~//   Statistics and simulation~/ 
Eds. J.~Pilz, D.~Rasch, V.~Melas, K.~Moder.~--- Springer proceedings in 
mathematics \& statistics ser.~--- Springer, 2018.Vol.~231. P.~381--390. doi:  
10.1007/978-3-319-76035-3\_27.

\bibitem{8-sop}
\Au{Whitt W.} Continuity of generalized semi-Markov processes~// Math. 
Oper. Res., 1980. Vol.~5. Iss.~4. P.~494--501. doi: 
10.1287/moor.5.4.494.
\bibitem{9-sop}
\Au{Miyazawa M., Yamazaki~G.} The basic equations for a~supplemented GSMP 
and its applications to queues~// J.~\mbox{Appl}. Probab., 1988. Vol.~25. Iss.~3. 
P.~565--578. doi: 10.2307/3213985.
 \end{thebibliography}

 }
 }

\end{multicols}

\vspace*{-3pt}

\hfill{\small\textit{Поступила в~редакцию 15.06.18}}

\vspace*{8pt}

%\newpage

%\vspace*{-24pt}

\hrule

\vspace*{2pt}

\hrule

%\vspace*{-4pt}


\def\tit{ON THE INSENSITIVITY OF THE STATIONARY DISTRIBUTION OF~THE~LIMITED 
RESOURCES QUEUING SYSTEM WITH~STATE-DEPENDENT ARRIVAL AND~SERVICE RATES}

\def\titkol{On the insensitivity of the stationary distribution of~the~limited 
resources queuing system with~state-dependent arrival} % and~service rates}

\def\aut{E.\,S.~Sopin$^{1,2}$, V.\,A.~Naumov$^3$, 
and~K.\,Е.~Samouylov$^{1,2}$}

\def\autkol{E.\,S.~Sopin, V.\,A.~Naumov, 
and~K.\,Е.~Samouylov}

\titel{\tit}{\aut}{\autkol}{\titkol}

\vspace*{-11pt}


\noindent
$^1$Peoples' Friendship University of Russia (RUDN University), 6~Miklukho-Maklaya Str., 
Moscow 117198, Russian\linebreak
$\hphantom{^1}$Federation

\noindent
$^2$Institute of Informatics Problems, Federal Research Center ``Computer Sciences and 
Control'' of the Russian\linebreak
$\hphantom{^1}$Academy of Sciences, 44-2~Vavilov Str., Moscow 119333, Russian 
Federation

\noindent
$^3$Service Innovation Research Institute (PIKE), 8A~Annankatu, Helsinki 00120, 
Finland


\def\leftfootline{\small{\textbf{\thepage}
\hfill INFORMATIKA I EE PRIMENENIYA~--- INFORMATICS AND
APPLICATIONS\ \ \ 2018\ \ \ volume~12\ \ \ issue\ 3}
}%
 \def\rightfootline{\small{INFORMATIKA I EE PRIMENENIYA~---
INFORMATICS AND APPLICATIONS\ \ \ 2018\ \ \ volume~12\ \ \ issue\ 3
\hfill \textbf{\thepage}}}

\vspace*{3pt}


  


\Abste{The authors consider further generalization of the queuing systems, in which 
customers require not only a~server but also a certain amount of limited resources. In the 
considered queuing system, arrival and serving intensities depend on the statе of the system.  The 
authors assume an arbitrary distribution of the service time. The authors prove that the stationary 
distribution of the system has product form in the case of Poisson arrivals.\linebreak\vspace*{-12pt}}

\Abstend{Moreover, it was 
shown that the steady-state probability distribution of number of customers in the system and 
volumes of occupied resources depends on the service time distribution only through its mean.}

\KWE{queueing system; limited resources; insensitivity; service time}
  

  
\DOI{10.14357/19922264180306}

%\vspace*{-14pt}

 \Ack
\noindent
The publication was supported by the Ministry of Education and Science of the Russian Federation 
(project No.\,2.882.2017/4.6).



%\vspace*{6pt}

  \begin{multicols}{2}

\renewcommand{\bibname}{\protect\rmfamily References}
%\renewcommand{\bibname}{\large\protect\rm References}

{\small\frenchspacing
 {%\baselineskip=10.8pt
 \addcontentsline{toc}{section}{References}
 \begin{thebibliography}{9}
\bibitem{1-sop-1}
\Aue{Naumov, V., K.~Samouylov, N.~Yarkina, E.~Sopin, S.~And\-re\-ev, and 
A.~Samuylov.} 2015. LTE performance analysis using queuing systems with 
finite resources and random requirements. \textit{7th Congress (International) on 
Ultra Modern Telecommunications and Control Systems 
Proceedings}. IEEE. 100--103. doi: 
10.1109/ICUMT.2015.7382412.
\bibitem{2-sop-1}
\Aue{Tikhonenko, O.\,M., and K.\,G.~Klimovich.} 2001. 
Analysis of queuing systems for random-length arrivals with limited 
cumulative volume. \textit{Probl. Inform. 
Transm.} 37(1):70--79.
\bibitem{3-sop-1}
\Aue{Naumov, V., and K.~Samouylov.} 2017. Analysis of multi-resource loss 
system with state dependent arrival and service rates. \textit{Probab. 
Eng. Inform. Sc.} 31(4):413--419. doi: 10.1017/S0269964817000079.
\bibitem{4-sop-1}
\Aue{Sevast'yanov, B.\,А.} 1957. An ergodic 
theorem for Markov processes and its application to telephone systems
with refusals. 
\textit{Theor. Probab. Appl.} 2(1):104--112.
\bibitem{5-sop-1}
\Aue{Gnedenko, B.\,V., and I.\,N.~Kovalenko.} 1966. \textit{Vvedenie 
v~teoriyu massovogo oblsluzhivaniya} [Introduction to queuing theory]. 
Moscow: Nauka. 432~p.

\bibitem{7-sop-1}
\Aue{Taylor, P.\,G.} 2011. Insensitivity in stochastic models. \textit{Queueing 
networks.} Eds. R.~Boucherie and  N.~van Dijk. 
International ser. in operations research \& management 
science.  Springer. 154:121--140. doi:  
10.1007/978-1-4419-6472-4.

\bibitem{6-sop-1}
\Aue{Samouylov, K.\,E., Y.\,V.~Gaidamaka, and E.\,S.~Sopin.} 2018. 
Simplified analysis of queueing systems with random requirements. 
\textit{Statistics and simulation}. 
Eds. J.~Pilz, D.~Rasch, V.~Melas, and K.~Moder.
Springer proceedings in 
mathematics \& statistics ser.   Springer.
231:381--390. doi: 10.1007/978-3-319-76035-3\_27.
\bibitem{8-sop-1}
\Aue{Whitt, W.} 1980. Continuity of generalized semi-Markov processes. 
\textit{Math. Oper. Res.} 5(4):494--501. doi: 
10.1287/moor.5.4.494.
\bibitem{9-sop-1}
\Aue{Miyazawa, M., and G.~Yamazaki.} 1988. The basic equations for 
a~supplemented GSMP and its applications to queues. \textit{J.~Appl. 
Probab.} 25(3):565--578. doi: 10.2307/3213985.
\end{thebibliography}

 }
 }

\end{multicols}

\vspace*{-6pt}

\hfill{\small\textit{Received June 15, 2018}}

%\pagebreak

%\vspace*{-18pt}

\Contr

\noindent
\textbf{Sopin Eduard S.} (b.\ 1986)~--- Candidate of Science (PhD) in physics 
and mathematics, associate professor, Peoples' Friendship University of Russia 
(RUDN University), 6~Miklukho-Maklaya Str., Moscow 117198, Russian 
Federation; senior scientist, Institute of Informatics Problems, Federal Research 
Center ``Computer Science and Control'' of the Russian Academy of Sciences,  
44-2~Vavilov Str., Moscow 119333, Russian Federation; 
\mbox{sopin\_es@rudn.university}

\vspace*{3pt}

\noindent
\textbf{Naumov Valeriy A.} (b.\ 1950)~--- Candidate of Science (PhD) in physics 
and mathematics, Research Director, Service Innovation Research Institute (PIKE), 
8A~Annankatu, Helsinki 00120, Finland; \mbox{valeriy.naumov@pfu.fi}

\vspace*{3pt}

\noindent
\textbf{Samouylov Konstantin E.} (b.\ 1955)~--- Doctor of Science in 
technology, professor, Head of Department, Peoples' Friendship University of 
Russia (RUDN University), 6~Miklukho-Maklaya Str., Moscow 117198, Russian 
Federation; senior scientist, Institute of Informatics Problems, Federal Research 
Center ``Computer Science and Control'' of the Russian Academy of Sciences,  
44-2~Vavilov Str., Moscow 119333, Russian Federation, 
\mbox{samuylov\_ke@rudn.university}
  
\label{end\stat}


\renewcommand{\bibname}{\protect\rm Литература} 