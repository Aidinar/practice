\def\stat{gorshenin}

\def\tit{ЗАШУМЛЕНИЕ ДАННЫХ КОНЕЧНЫМИ СМЕСЯМИ НОРМАЛЬНЫХ 
И~ГАММА-РАСПРЕДЕЛЕНИЙ\\ С~ПРИМЕНЕНИЕМ К~ЗАДАЧЕ ОКРУГЛЕНИЯ НАБЛЮДЕНИЙ$^*$}

\def\titkol{Зашумление данных конечными смесями нормальных 
и~гамма-распределений с~применением к~задаче округления} % наблюдений}

\def\aut{А.\,К.~Горшенин$^1$}

\def\autkol{А.\,К.~Горшенин}

\titel{\tit}{\aut}{\autkol}{\titkol}

\index{Горшенин А.\,К.}
\index{Gorshenin A.\,K.}


{\renewcommand{\thefootnote}{\fnsymbol{footnote}} \footnotetext[1]
{Работа выполнена при поддержке РНФ (проект 18-71-00156).}}


\renewcommand{\thefootnote}{\arabic{footnote}}
\footnotetext[1]{Институт проблем информатики Федерального исследовательского центра 
<<Информатика и~управление>> Российской академии наук, \mbox{agorshenin@frccsc.ru}}

\vspace*{-12pt}




\Abst{Во многих реальных задачах проводится статистический анализ данных, 
содержащих дополнительные ошибки измерения, в~том числе в~виде округления, 
что в~ряде ситуаций может приводить к~достаточно существенным искажениям. 
В~настоящей статье для одной из возможных моделей округления получены оценки 
для неизвестного математического ожидания наблюдений в~предположении, что 
исходные данные дополнительно зашумлены с~по\-мощью случайных величин, 
име\-ющих распределения типа конечных смесей нормальных и~гам\-ма-за\-ко\-нов. 
Построены доверительные интервалы для неизвестного математического ожидания 
с~использованием уточненной оценки для дисперсии целой части случайной величины. 
Обсуждается алгоритм определения значения параметра для искусственного шума, 
добавление которого к~исходным данным способствует повышению качества работы 
метода скользящего разделения смесей.}

\KW{зашумленные данные; округленные наблюдения; конечные смеси нормальных 
распределений; конечные смеси гам\-ма-рас\-пре\-де\-ле\-ний; доверительные интервалы;  
метод скользящего разделения смесей}

\DOI{10.14357/19922264180304}
  
\vspace*{-4pt}


\vskip 10pt plus 9pt minus 6pt

\thispagestyle{headings}

\begin{multicols}{2}

\label{st\stat}


\section{Введение}

Во многих реальных задачах данные, являющиеся непрерывными по своей сути, 
регистрируются с~помощью инструментов, вносящих дополнительные ошибки 
измерения, в~том чис\-ле в~виде округления. Таким образом, статистический 
анализ проводится не для исходных, а для преобразованных некоторым 
случайным образом наблюдений, что в~ряде ситуаций может приводить к~достаточно
 существенным искажениям.

Для преодоления данной проблемы развивались различные подходы, в~том числе 
на основе смешанных моделей (см., например, статью~\cite{Wright2003}, в~которой 
различные компоненты  используются для пред\-став\-ле\-ния уровней округления). 
В~работе~\cite{Bai2009} приводятся результаты для моделей авторегрессии и~скользящего 
среднего для округленных данных, а~в~статье~\cite{Zhang2010} эти результаты 
развиваются и~исследуются их асимптотические свойства. 
В~статье~\cite{Zhao2012} исследован метод оценивания па\-ра\-мет\-ров конечных смесей 
вероятностных распределений (в~том чис\-ле, и~многомерных) 
на основе использования EM (expectation-maximization) 
алгоритма~\cite{Korolev2011-i} с~\mbox{целью} получения состоятельных 
и~асимптотически нормальных оценок.

В настоящей статье развиваются результаты для моделей округления, 
описанных в~работах~\cite{Ushakov2015,Ushakov2017a,Ushakov2017b}. 
В~их рамках будут получены оценки для неизве\-ст\-ного математического ожидания 
наблюдений в~предположении, что исходные данные зашумлены с~по\-мощью случайных 
величин, имеющих распределения типа конечных смесей нормальных и~гам\-ма-за\-ко\-нов. 
Это позволяет учесть большее количество случайных факторов, влия\-ющих на величину 
<<дополнительной>> ошибки. Также будут построены доверительные интервалы для 
неизвестного математического ожидания. Выражения для гам\-ма-рас\-пре\-де\-ле\-ний 
получены впервые. Также обсуждается алгоритм определения значения па\-ра\-мет\-ра для 
искусственного шума, добавление которого к~исходным данным способствует 
повышению качества работы метода скользящего разделения смесей~\cite{Gorshenin2016}.

\vspace*{-12pt}

\section{Предположения и~базовые отношения}

Для сокращения формулировок теорем в~сле\-ду\-ющих разделах сделаем ряд 
предположений, на которые будем ссылаться в~дальнейшем. Итак, пусть:
\begin{itemize}
\item[(A)] $X_1,X_2,\ldots$~--- независимые одинаково распределенные 
случайные величины с~неизвестным математическим ожиданием ${\sf E}_X\hm<+\infty$;
\item[(B)] $\varepsilon_1,\varepsilon_2,\ldots$~--- независимые одинаково 
распределенные случайные величины с~математическим ожиданием 
${\sf E}_\varepsilon\hm<+\infty$; %\label{B}
\item[(C)] $X_1,X_2,\ldots$ и~$\varepsilon_1,\varepsilon_2,\ldots$ 
являются независимыми;
\item[(D)] $Y_j=\left[X_j+\varepsilon_j+1/2\right]$ для всех $j\hm=1,2,\ldots$ 
представляют собой округление значения суммы случайных величин $X_j\hm+\varepsilon_j$ 
до ближайшего целого сверху (при этом запись~$[\cdot]$ соответствует целой 
части выражения).
\end{itemize}

В рамках данных предположений в~статье будут рассмотрены вопросы качества 
приближения неизвестного математического ожидания~${\sf E}_X$ для исходных данных 
в~ситуации, когда наблюдения для анализа получены с~аддитивной ошибкой c известными 
распределениями (см.\ предположение~(B)) и~дополнительно округляются до 
ближайшего целого (см.\ предположение~(D)).

Заметим, что в~силу усиленного закона больших чисел справедливы следующие выражения:
\begin{multline}
\fr{1}{n}\sum\limits_{j=1}^n Y_j\xrightarrow[n\to\infty]{\text{п.н.}}
{\sf E}_Y\equiv\mathbb{E}\left[X_1+\varepsilon_1+\fr{1}{2}\right]={}\\
{}=\mathbb{E}\left(X_j+\varepsilon_j+\fr{1}{2}\right)-\mathbb{E}
\left\{X_j+\varepsilon_j+\fr{1}{2}\right\}={}\\
{}={\sf E}_X+{\sf E}_\varepsilon+\fr{1}{2}-\mathbb{E}\left\{X_j+\varepsilon_j+\fr{1}{2}\right\}. 
\label{Law}
\end{multline}

Запись $\{\cdot\}$ в~формуле~\eqref{Law} соответствует дробной 
части выражения, а~п.н.\ обозначает сходимость в~смысле почти наверное.

Для доказательства результатов в~дальнейшем потребуется следующее 
представления для дробной части  абсолютно непрерывной случайной величины~$Z$ 
с~абсолютно  интегрируемой характеристической функцией~$\varphi_Z(t)$
 (см., например, Лемму~4 в~работе~\cite{Ushakov2017b}):
\begin{equation}
\label{Fract}
\mathbb{E}\{Z\}=\fr{1}{2}-\sum\limits_{n=1}^\infty 
\fr{\mathrm{Im}\left (\varphi_Z(2\pi n)\right)}{\pi n}\,.
\end{equation}

Через $\mathrm{Im}\,(\cdot)$ в~формуле~\eqref{Fract} обозначена мнимая часть 
соответствующей функции.

При построении доверительных интервалов в~дальнейшем будет 
использована следующая оценка, справедливая для любой случайной величины~$Z$:
\begin{equation}
\mathbb{D}[Z]\leqslant \left(\sqrt{\mathbb{D} Z}+\fr{1}{2}\right)^2.
\label{Var}
\end{equation}
Она может быть проверена непосредственно с~учетом представления 
$\mathbb{D} [Z]\hm=\mathbb{D}\left(Z\hm-\{Z\}\right)$, неравенства 
Ко\-ши--Бу\-ня\-ков\-ско\-го для ковариации и~соотношения 
 $\mathbb{D}\{Z\}\hm\leqslant 1/4$, справедливого для любой случайной величины~$Z$ 
 (см., например, статью~\cite{Ushakov2017b}). Отметим, что данная оценка 
 является более точной по сравнению с~использованным для аналогичных 
 целей в~работе~\cite{Ushakov2017b} соотношением 
 $\mathbb{D} [Z]\hm\leqslant 2\mathbb{D} Z\hm+1/2$. Действительно,
\begin{equation*}
2\mathbb{D} Z+\fr{1}{2}-\left(\sqrt{\mathbb{D} Z}+\fr{1}{2}\right)^2=
\left(\sqrt{\mathbb{D} Z}-\fr{1}{2}\right)^2\geqslant0\,,
\end{equation*}
причем для всех $\sqrt{\mathbb{D} Z}\hm\neq {1}/{2}$ 
данное неравенство является строгим.

\section{Конечные смеси нормальных законов}

Для случайной величины~$X$, имеющей распределение типа 
конечной смеси нормальных законов~\cite{Korolev2011-i} с~параметрами 
${\bf a}\hm=(a_1,\ldots, a_k)$, $a_j\hm\in \mathbb{R}$, 
$\boldsymbol{\sigma}\hm=(\sigma_1,\ldots, \sigma_k)$, 
$\sigma_j\hm>0$, ${\bf p}\hm=(p_1,\ldots, p_k)$, $p_j\hm\geqslant 0$, 
$\sum\nolimits_{j=1}^{k}p_j\hm=1$, плот\-ность которого задается выражением
\begin{equation}
f_X(x)=\sum\limits_{j=1}^{k}\fr{p_j}{\sigma_j\sqrt{2\pi}}\,e^{-(x-a_j)^2/(2\sigma_j^2)}\,,
\label{FinNormMixt}
\end{equation}
характеристическая функция имеет вид:
\begin{equation}
\varphi_X(t)=\int\limits_{-\infty}^{+\infty}\!\!e^{itx} f_X(x)\, dx = 
\sum\limits_{j=1}^{k}p_j e^{ita_j-\sigma_j^2 t^2/2}.
\label{ChiFinNormMixt}
\end{equation}

Абсолютная интегрируемость  $\varphi_X(t)$ вытекает из свойств 
характеристической функции нормального распределения. 
Заметим, что в~точке $t\hm=2\pi n$ выражение~\eqref{ChiFinNormMixt} принимает 
сле\-ду\-ющий вид:
\begin{equation}
\label{ChiFinNormMixt2npi}
\varphi_X(2\pi n)= \sum\limits_{j=1}^{k}p_j e^{-2\pi^2 \sigma_j^2 n^2}\,.
\end{equation}

Рассмотрим вопрос точности оценивания неизвестного математического ожидания~${\sf E}_X$ 
при до\-бав\-ле\-нии зашумления.

\smallskip

\noindent
\textbf{Теорема~1.}\ 
\textit{Пусть выполнены предположения}~(A)--(D), 
\textit{причем случайные величины~$\varepsilon_j$, $j\hm=1,2,\ldots$, 
имеют распределение типа конечной $k$-ком\-по\-нент\-ной смеси нормальных законов 
вида}~\eqref{FinNormMixt} \textit{с~па\-ра\-мет\-ра\-ми~${\bf a}$, $\boldsymbol{\sigma}$ 
и~${\bf p}$. Тогда}
\begin{equation}
\label{Th1Eq}
\left\lvert {\sf E}_Y-{\sf E}_X\right\rvert \leqslant 
A+\fr{1}{\pi}\left(1+\fr{1}{4\pi^2\sigma^2}\right)e^{-2\pi^2\sigma^2}\,, 
\end{equation}
\textit{где} $A=\max(|a_1|,\ldots,|a_k|)$, $\sigma\hm=\min(\sigma_1,\ldots,\sigma_k)$.

\smallskip


\noindent
Д\,о\,к\,а\,з\,а\,т\,е\,л\,ь\,с\,т\,в\,о\,.\ \
С~учетом пред\-став\-ле\-ний~\eqref{Law},~\eqref{Fract} и~\eqref{ChiFinNormMixt2npi}, 
ограниченности модуля характеристической функции, а~также не\-за\-ви\-си\-мости 
случайных величин~$X_j$ и~$\varepsilon_j$ имеем:
\begin{multline*}
\left\lvert {\sf E}_Y-{\sf E}_X\right\rvert =
\left\lvert {\sf E}_\varepsilon+\fr{1}{2}-\mathbb{E}\left\{X_j+
\varepsilon_j+\fr{1}{2}\right\}\right\rvert={}\\
{}=\left\lvert {\sf E}_\varepsilon+\sum\limits_{n=1}^\infty
\fr{\mathrm{Im} \left(\varphi_{X_j}(2\pi n)\varphi_{\varepsilon_j}(2\pi n)
\varphi_{1/2}(2\pi n)\right)}{\pi n}\right\rvert={}\\
=\left\lvert 
\vphantom{\fr{(-1)^n\sum\nolimits_{j=1}^{k}p_j e^{-2\pi^2 \sigma_j^2 n^2} 
\mathrm{Im} \left(\varphi_{X_j}(2\pi n)\right)}{\pi n}}
{\sf E}_\varepsilon+{}\right.\\
\left.{}+\sum\limits_{n=1}^\infty
\fr{\mathrm{Im} \left(\varphi_{X_j}(2\pi n) 
\sum\nolimits_{j=1}^{k}p_j e^{-2\pi^2 \sigma_j^2 n^2} 
e^{\pi n}\right)}{\pi n}\right\rvert={}\\
{}=\left\lvert 
\vphantom{\fr{(-1)^n\sum\nolimits_{j=1}^{k}p_j e^{-2\pi^2 \sigma_j^2 n^2} 
\mathrm{Im} \left(\varphi_{X_j}(2\pi n)\right)}{\pi n}}
{\sf E}_\varepsilon+{}\right.\\
\left.{}+\sum\limits_{n=1}^\infty
\fr{(-1)^n\sum\nolimits_{j=1}^{k}p_j e^{-2\pi^2 \sigma_j^2 n^2} 
\mathrm{Im} \left(\varphi_{X_j}(2\pi n)\right)}{\pi n}\right\rvert\leqslant{}\\
{}\leqslant \left\lvert {\sf E}_\varepsilon\right\rvert+\left\lvert\
\sum\limits_{j=1}^{k}p_j\sum\limits_{n=1}^\infty 
\fr{1}{\pi n} e^{-2\pi^2 \sigma_j^2 n^2}\right\rvert\leqslant {}\\
\\
{}\leqslant
\max\left(|a_1|,\ldots,|a_k|\right)+{}\\
{}+\sum\limits_{j=1}^{k} 
\fr{p_j}{\pi} \left(\!1+\fr{1}{4\pi^2\sigma_j^2}\!\right)\!e^{-2\pi^2\sigma_j^2}\leqslant{}\\
{}\leqslant
A+\fr{1}\pi\left(1+\fr{1}{4\pi^2\sigma^2}\right)e^{-2\pi^2\sigma^2}\,.
\end{multline*}

Справедливость использованной оценки 
\begin{equation*}
\sum\limits_{n=1}^\infty
\fr{e^{-2\pi^2 \sigma_j^2 n^2}}{n}\leqslant 
\left(1+\fr{1}{4\pi^2\sigma_j^2}\right)e^{-2\pi^2\sigma_j^2}
\end{equation*}
может быть проверена непосредственно (например, см.\ доказательство Теоремы~6 
в~статье~\cite{Ushakov2017b}).~\hfill$\square$

\smallskip

\noindent
\textbf{Замечание~1.}
В~случае, если зашумление производится нормально распределенными случайными 
величинами c нулевыми средними (т.\,е.\ в~формуле~\eqref{Th1Eq} необходимо считать 
$A\hm=0$, $k\hm=1$), то Тео\-ре\-ма~1 совпадает с~результатом, 
полученным в~работе~\cite{Ushakov2017b}.


\smallskip

Рассмотрим вопросы построения доверительного интервала для неизвестного 
математического ожидания~${\sf E}_X$ в~предположении, что случайные величины~$X_j$ не 
содержат ошибок измерения, а~все погрешности учтены исключительно в~за\-шум\-ля\-ющих 
элементах~$\varepsilon_j$.

\smallskip

\noindent
\textbf{Теорема~2.}\ 
\textit{Пусть выполнены предположения}~(A)--(D), 
\textit{причем случайные величины~$\varepsilon_j$, $j\hm=1,2,\ldots$, имеют 
распределение типа конечной $k$-ком\-по\-нент\-ной смеси нормальных законов 
вида}~\eqref{FinNormMixt} \textit{с~параметрами~${\bf a}$, $\boldsymbol{\sigma}$ 
и~${\bf p}$, а~случайные величины} $X_j\stackrel{\text{п.н.}}{=}{\sf E}_X$, $j\hm=1,2,\ldots$ 
\textit{Тогда доверительный интервал для~${\sf E}_X$ при условии $0\hm<\alpha\hm<1$ имеет вид}:
\begin{equation} 
\label{Th2Eq}
\hat{{\sf E}}_X - f({\bf a},\boldsymbol{\sigma},\alpha,n) 
\leqslant {\sf E}_X \leqslant  \hat{{\sf E}}_X + f({\bf a},\boldsymbol{\sigma},\alpha,n),
\end{equation}
\textit{где}

\vspace*{-2pt}

\noindent
\begin{align}
\hat{{\sf E}}_X&=\fr{1}{n} \sum\limits_{j=1}^{n} Y_j\,; \label{Th2hatE}\\
f({\bf a},\boldsymbol{\sigma},\alpha,n)&=
\fr{z_{1-{\alpha}/2}}{\sqrt{n}} \left(\sqrt{A^2+\Sigma^2}+\fr{1}{2}\right) +{}\notag\\
&{}+A+\fr{1}\pi\left(1+\fr{1}{4\pi^2\sigma^2}\right)e^{-2\pi^2\sigma^2}\,;
  \label{Th2f}
\end{align}
\textit{$z_{1-{\alpha}/2}$~--- $\left(1-{\alpha}/2\right)$-кван\-тиль 
стандартного нормального распределения; $A\hm=\max(|a_1|,\ldots,|a_k|)$; 
$\Sigma\hm=\max(\sigma_1,\ldots,\sigma_k)$; $\sigma\hm=\min(\sigma_1,\ldots,\sigma_k)$}. 


\smallskip

\noindent
\noindent
Д\,о\,к\,а\,з\,а\,т\,е\,л\,ь\,с\,т\,в\,о\,.\ \
Из центральной предельной тео\-ре\-мы с~учетом условия~(A) следует, 
что величина~$\hat{{\sf E}}_X$~\eqref{Th2hatE} асимптотически нормальна с~математическим 
ожиданием 
\begin{equation}
{\sf E}_Y\equiv \mathbb{E}\left[{\sf E}_X+\varepsilon_1+\fr{1}{2}\right] \label{EY}
\end{equation}
и дисперсией
\begin{equation}
\fr{1}{n} {\sf D}_Y\equiv \fr{1}{n}\mathbb{D}\left[{\sf E}_X+\varepsilon_1+
\fr{1}{2}\right]. \label{DY}
\end{equation}

Воспользовавшись оценкой~\eqref{Var}, получим:

\vspace*{-2pt}

\noindent
\begin{multline*}
{\sf D}_Y \leqslant  \left(\sqrt{\mathbb{D} \left({\sf E}_X+\varepsilon_1+\fr{1}{2}\right)}+
\fr{1}{2}\right)^2={}\\
{}=
\left(\sqrt{\mathbb{D}\varepsilon_1}+\fr{1}{2}\right)^2= {}\\
{}= \left(\sqrt{\sum\limits_{j=1}^{k}p_j\left(\left(a_j-\sum\limits_{t=1}^{k}
p_t a_t\right)^2+\sigma_j^2\right)}+\fr{1}{2}\right)^2\leqslant {}\\ 
{}\leqslant \left(\sqrt{A^2+\Sigma^2}+\fr{1}{2}\right)^2\,.
\end{multline*}
Тогда доверительный интервал уровня $1\hm-\alpha$ для математического ожидания~${\sf E}_Y$ 
имеет вид:
\begin{equation*}
\mathbb{P}\left(\left\lvert \hat{{\sf E}}_X-{\sf E}_Y\right\rvert \leqslant 
\fr{z_{1-{\alpha}/2}}{\sqrt{n}} 
\left(\sqrt{A^2+\Sigma^2}+\fr{1}{2}\right)\right)\geqslant 1-\alpha\,.
\end{equation*}

\begin{table*}[b]\small
\begin{center}

\begin{tabular}{|c|c|c|c|c|c|c|c|}
\multicolumn{7}{p{100mm}}{Численные решения уравнений~\eqref{f1} и~\eqref{f2} относительно 
параметра~$\sigma$ для некоторых значений~$n$ и~$\alpha$}\\
\multicolumn{7}{c}{\ }\\[-6pt]
\hline
\multicolumn{1}{|c|}{Размер}  & \multicolumn{2}{c|}{Уровень $\alpha=0{,}1$}& 
\multicolumn{2}{c|}{Уровень $\alpha=0{,}05$}& 
\multicolumn{2}{c|}{Уровень $\alpha=0{,}01$}\\
\cline{2-7}
\multicolumn{1}{|c|}{выборки $n$}&$\sigma_1$&$\sigma_2$&$\sigma_1$&$\sigma_2$&$\sigma_1$&$\sigma_2$\\
\hline
$\hphantom{000}100$&$0{,}4302$&$0{,}435$&$0{,}419$&$0{,}425$&$0{,}4002$&$0{,}408$\\
%\hline
$\hphantom{000}200$&$0{,}452$&$0{,}455$ &$0{,}441$&$0{,}445$&$0{,}424$&$0{,}429$\\
%\hline
$\hphantom{00}1000$&$0{,}499$&$0{,}499$ &$0{,}489$&$0{,}489$&$0{,}473$&$0{,}475$\\
%\hline
$\hphantom{0}10000$&$0{,}558$&$0{,}556$ &$0{,}549$&$0{,}547$&$0{,}536$&$0{,}534$\\
%\hline
$100000$&$0{,}611$&$0{,}607$ &$0{,}603$&$0{,}599$&$0{,}591$&$0{,}588$\\
\hline
\end{tabular}
\end{center}
\end{table*}


\noindent
Откуда следует справедливость соотношения~\eqref{Th2Eq} c~уче\-том 
очевидного неравенства

\pagebreak

\noindent
\begin{equation*}
\left\lvert \hat{{\sf E}}_X-{\sf E}_X\right\rvert \leqslant 
\left\lvert \hat{{\sf E}}_X-{\sf E}_Y\right\rvert +\left\lvert {\sf E}_Y-{\sf E}_X\right\rvert 
\end{equation*}
и оценки~\eqref{Th1Eq} из Теоремы~1.~\hfill$\square$

\smallskip

\noindent
\textbf{Замечание~2.}
В~работе~\cite{Gorshenin2016} было продемонстрировано повышение точ\-ности 
работы метода скользящего разделения конечных нормальных смесей за счет 
введения дополнительной компоненты, имеющей нормальное 
распределение $\mathcal{N}(0,\sigma^2)$ с~математическим ожиданием, равным~$0$, 
и~стандартным отклонением~$\sigma$. При этом была отмечена сложность выбора 
параметра~$\sigma$ для сохранения структуры выборки, близкой к~исходной. 
Результат Теоремы~2 может быть использован с~данной целью, если положить $k\hm=1$, 
$a_j\hm=0$ для всех $j\hm=1,2,\ldots$ и~выбирать величину~$\sigma$ как 
минимизирующую длину доверительного интервала~\eqref{Th2Eq}. Для 
этого необходимо найти производную функции $f(0,\sigma,\alpha,n)$~\eqref{Th2f} 
и~численно решить уравнение
\begin{multline}
f_\sigma'(0,\sigma,\alpha,n)\equiv \fr{z_{1-{\alpha}/2}}{\sqrt{n}} - {}\\
{}-
e^{-2\pi^2\sigma^2}\left(4\pi\sigma+\fr{1}{2\pi^3\sigma^3}+
\fr{1}{\pi\sigma}\right)=0
\label{f1}
\end{multline}
относительно неизвестного параметра~$\sigma$ при выбранных значениях величин~$n$ 
и~$\alpha$. В~качестве альтернативы можно использовать вид доверительного интервала 
из статьи~\cite{Ushakov2017b}, полученный с~помощью неравенства $\mathbb{D} [Z]
\hm\leqslant 2\mathbb{D} Z\hm+{1}/{2}$, и~искать решение уравнения вида:
\begin{multline}
\hspace*{-2.90578pt}\fr{2\sigma z_{1-{\alpha}/2}}{\sqrt{n (2\sigma^2+{1}/{2})}} -
 e^{-2\pi^2\sigma^2}\left(4\pi\sigma+\fr{1}{2\pi^3\sigma^3}+
 \fr{1}{\pi\sigma}\right)={}\\
 {}=0\,.\label{f2}
\end{multline}

Примеры найденных значений~$\sigma$ для типичных размеров выборок в~методе 
скользящего разделения смесей (учитываются как возможная ширина окна, 
так и~общее количество наблюдений в~анализируемом ряде) приведены в~таблице 
(использован метод оптимизации \verb"Trust-Region Dogleg" пакета \verb"MATLAB" 
c~настройками по умолчанию), в~которой через~$\sigma_1$ обозначено приближенное  
решение уравнения~\eqref{f1}, a~через $\sigma_2$~--- уравнения~\eqref{f2}.


Проверка практической эффективности данного подхода в~качестве 
критерия выбора параметров зашумляющего распределения для повышения 
точности работы метода скользящего разделения смесей может быть отмечена 
как задача для дальнейших исследований.


\section{Конечные смеси гамма-распределений}

Для случайной величины~$X$, имеющей распределение типа конечной смеси 
гам\-ма-рас\-пре\-де\-ле\-ний с~параметрами ${\bf r}\hm=(r_1,\ldots, r_k)$,
 $r_j\hm>0$, $\boldsymbol{\lambda}\hm=(\lambda_1,\ldots, \lambda_k)$, $\lambda_j\hm>0$, 
 ${\bf p}\hm=(p_1,\ldots, p_k)$, $p_j\hm\geqslant 0$, $\sum\nolimits_{j=1}^{k}p_j\hm=1$, 
 плот\-ность которого задается выражением
\begin{equation}
f_X(x)=\sum\limits_{j=1}^{k}p_j\fr{\lambda_j^{r_j} e^{-\lambda_j x}}
{\Gamma(r_j)}\,x^{r_j-1}\,,
\label{FinGammaMixt}
\end{equation}
характеристическая функция имеет следующий вид:
%характеристическая функция задается следующим выражением:
\begin{equation}
\varphi_X(t)=\!\int\limits_{-\infty}^{+\infty}\!\!\!e^{itx} f_X(x)\, dx = \!
\sum\limits_{j=1}^{k}p_j \left(\!1-\fr{it}{\lambda_j}\right)^{-r_j}\!.\!
\label{ChiFinGammaMixt}
\end{equation}

Отметим, что подобные модели зашумления разумно использовать в~случае, 
если известно, что данные сосредоточены на положительной полуоси, например 
при анализе различных информационных потоков (см., в~част\-ности, 
 работу~\cite{Gorshenin2013}). 

Проверим абсолютную интегрируемость функции $\varphi_X(t)$~\eqref{ChiFinGammaMixt}. 
Имеем:
\begin{multline*}
\int\limits_{-\infty}^{+\infty}\left\lvert\varphi_X(t)\right\rvert\, dt\leqslant 
\sum\limits_{j=1}^{k}p_j \int\limits_{-\infty}^{+\infty}\left\lvert \left(
1-\fr{it}{\lambda_j}\right)^{-r_j}\right\rvert \, dt={}\\
{}=\sum\limits_{j=1}^{k}p_j \int\limits_{-\infty}^{+\infty} \left\lvert\left(
\fr{\lambda_j(\lambda_j+it)}{\lambda_j^2+t^2}\right)^{r_j}\right\rvert\, dt \leqslant{}\\
{}\leqslant\sum\limits_{j=1}^{k}p_j \lambda_j \int\limits_{-\infty}^{+\infty}\left(
1+y^2\right)^{-{r_j}/{2}}\, dy\,.
\end{multline*}

Подынтегральное выражение при $r_j\hm\geqslant 2$ может быть оценено сверху 
функцией $1/({1+y^2})$, при этом соответствующий интеграл равен~$\pi$, что влечет 
абсолютную интегрируемость характеристической функции для конечной смеси 
гам\-ма-рас\-пре\-де\-ле\-ний. Поэтому в~дальнейшем будем предполагать,
 что $r_j\hm\geqslant 2$ для всех возможных значений $j\hm=1,2,\ldots$

Рассмотрим вопрос точ\-ности оценивания неизвестного математического ожидания ${\sf E}_X\hm>0$ 
при добавлении зашумления.

\smallskip

\noindent
\textbf{Теорема~3.}
\textit{Пусть выполнены предположения}~(A)--(D), 
\textit{причем случайные величины~$\varepsilon_j$, $j\hm=1,2,\ldots$, имеют 
распределение типа конечной $k$-ком\-по\-нент\-ной смеси 
гам\-ма-рас\-пре\-де\-ле\-ний вида}~\eqref{FinGammaMixt} 
\textit{с~па\-ра\-мет\-ра\-ми~${\bf r}$, $\boldsymbol{\lambda}$ и~${\bf p}$. Тогда}
\begin{equation}
\label{Th3Eq}
\left\lvert {\sf E}_Y-{\sf E}_X\right\rvert \leqslant \fr{R}{\lambda}+
\fr{\Lambda^{R}}{2^{r}\pi^{r+1}}\left(1+\frac1{r}\right)\,,
\end{equation}
\textit{где} $r=\min(r_1, \ldots,r_k)$; $R\hm=\max(r_1, \ldots,r_k)$; 
$\lambda\hm=\max(\lambda_1, \ldots,\lambda_k)$; 
$\Lambda\hm=\max(\lambda_1, \ldots,\lambda_k)$.

\smallskip

\noindent
Д\,о\,к\,а\,з\,а\,т\,е\,л\,ь\,с\,т\,в\,о\,.\ \
С~учетом пред\-став\-ле\-ний~\eqref{Law} и~\eqref{Fract}, ограниченности 
модуля характеристической функции, перехода от тригонометрической к~показательной 
записи комплексных чисел, а~также независимости случайных величин~$X_j$ 
и~$\varepsilon_j$ \mbox{имеем}:
\begin{multline*}
\left\lvert {\sf E}_Y-{\sf E}_X\right\rvert
\leqslant \left\lvert {\sf E}_\varepsilon\right\rvert+ {}\\
{}+\left\lvert\sum\limits_{n=1}^\infty
\left(
(-1)^n\mathrm{Im} \left(\sum\limits_{j=1}^{k}p_j \varphi_{X_j}(2\pi n)\left(
\vphantom{\fr{2\pi n}{\lambda_j}}
1-{}\right.\right.\right.\right.\\
\left.\left.\left.\left.{}-i\left(\fr{2\pi n}{\lambda_j}\right)\right)^{-r_j}\right)
\Bigg/ ({\pi n})
\vphantom{\sum\limits_{j=1}^{k}}
\right)\right\rvert={}\\
{}=\left\lvert {\sf E}_\varepsilon\right\rvert+ 
\left\lvert\sum\limits_{n=1}^\infty
\left(\!(-1)^n\mathrm{Im} \!\left(\sum\limits_{j=1}^{k}p_j \left(\!
1+\fr{4\pi^2 n^2}{\lambda_j^2}\right)^{- {r_j}/2}\!\times{}\right.\right.\right.\hspace*{-2.8663pt}\\
\left.\left.\left.{}\times \varphi_{X_j}(2\pi n)\,
e^{-ir_j\mathrm{arctan}\,({{t}/{\lambda_j}})}\right)
\Bigg/
({\pi n})
\vphantom{\left(
1+\fr{4\pi^2 n^2}{\lambda_j^2}\right)^{- {r_j}/2}}
\right)\right\rvert\leqslant{}\\
{}\leqslant \left\lvert {\sf E}_\varepsilon\right\rvert+\sum\limits_{j=1}^{k}
p_j\sum\limits_{n=1}^\infty\fr{1}{\pi n}\left(
1+\fr{4\pi^2 n^2}{\lambda_j^2}\right)^{-{r_j}/2}\leqslant{}\\
{}\leqslant  \fr{R}\lambda + \sum\limits_{j=1}^{k}p_j
\sum\limits_{n=1}^\infty\left(\fr{1}{\pi n}\,
\fr{\lambda_j^{r_j}}{(2\pi)^{r_j} n^{r_j}}\right)\leqslant {}
\\
{}\leqslant  \fr{R}{\lambda} + \sum\limits_{j=1}^{k}p_j 
\fr{\lambda_j^{r_j}}{2^{r_j}\pi^{r_j+1}}\left(1+\int\limits_{1}^{\infty}
\fr{1}{ x^{r_j+1}}\,dx\right)
\leqslant{}\\
{}\leqslant \fr{R}{\lambda}+\fr{\Lambda^{R}}{2^{r}\pi^{r+1}}\left(1+\fr{1}{r}\right).
\end{multline*}

При переходе от суммы к~интегралу используется факт убывания функции как переменной~$n$ 
(или~$x$).~\hfill$\square$


\smallskip

\noindent
\textbf{Замечание~3.}\
Теорема~3 описывает соответ\-ст\-ву\-ющий результат для гам\-ма-рас\-пре\-де\-лен\-ных 
за\-шум\-ля\-ющих случайных величин, если положить $k\hm=1$ в~выражении~\eqref{Th3Eq}. 
При этом, очевидно, $r\hm\equiv R$ и~$\lambda\hm\equiv \Lambda$.


\smallskip

Рассмотрим вопросы построения доверительного интервала для неизвестного 
математического ожидания ${\sf E}_X\hm>0$ в~предположении, что случайные величины~$X_j$ 
не содержат ошибок измерения, а все погрешности учтены исключительно в~за\-шум\-ля\-ющих 
элементах~$\varepsilon_j$.

\smallskip

\noindent
\textbf{Теорема~4.}
\textit{Пусть выполнены предположения}~(A)--(D), 
\textit{причем случайные величины~$\varepsilon_j$, $j\hm=1,2,\ldots$, имеют 
распределение типа конечной $k$-ком\-по\-нент\-ной смеси 
гам\-ма-рас\-пре\-де\-ле\-ний вида}~\eqref{FinGammaMixt} 
\textit{с~па\-ра\-мет\-ра\-ми~${\bf r}$, $\boldsymbol{\lambda}$ и~${\bf p}$, 
а~случайные величины} $X_j\stackrel{\text{п.н.}}{=}{\sf E}_X$, $j=1,2,\ldots$ 
\textit{Тогда доверительный интервал для~${\sf E}_X$ при условии $0\hm<\alpha\hm<1$ имеет вид}:
\begin{equation} 
\label{Th4Eq}
\left\lvert {\sf E}_X - \hat{{\sf E}}_X\right\rvert \leqslant  
f({\bf r},\boldsymbol{\lambda},\alpha,n),
\end{equation}
\textit{где}

\vspace*{-9pt}

\noindent
\begin{align}
\hat{{\sf E}}_X&=\fr{1}{n} \sum\limits_{j=1}^{n} Y_j\,; \label{Th4hatE}\\[-4pt]
f({\bf r}, \boldsymbol{\lambda},\alpha,n)&=\fr{z_{1-{\alpha}/2}}{\sqrt{n}} \left(
\sqrt{\fr{R(R+1)}{\lambda^2}-\fr{r^2}{\Lambda^2}}+\fr{1}{2}\right) +{}\notag\\[-1pt]
&\hspace*{20mm}{}+
\fr{R}{\lambda}+\fr{\Lambda^{R}}{2^{r}\pi^{r+1}}\left(1+\fr{1}{r}\right); \notag
\end{align}
\textit{$z_{1-{\alpha}/2}$~--- $\left(1-{\alpha}/2\right)$-кван\-тиль 
стандартного нормального распределения; $r\hm=\min(r_1, \ldots,r_k)$; 
$R\hm=\max(r_1, \ldots,r_k)$; $\lambda\hm=\max(\lambda_1, \ldots,\lambda_k)$; 
$\Lambda\hm=\max(\lambda_1, \ldots,\lambda_k)$}. 

\smallskip

\noindent
Д\,о\,к\,а\,з\,а\,т\,е\,л\,ь\,с\,т\,в\,о\,.\ \
Из центральной предельной теоремы с~учетом условия~(A) 
следует, что величина~$\hat{{\sf E}}_X$~\eqref{Th4hatE} асимптотически нормальна 
с~математическим ожиданием~${\sf E}_Y$~\eqref{EY} и~дисперсией $(1/n){\sf D}_Y$~\eqref{DY}. 
Пользуясь определением и~свойствами гам\-ма-функ\-ции, а~также оценкой~\eqref{Var} 
получим:

\noindent
\begin{multline*}
{\sf D}_Y \leqslant \left(\sqrt{\sum\limits_{j=1}^k p_j
\fr{\lambda_j^{r_j}}{\Gamma(r_j)} \int\limits_{0}^{+\infty} 
e^{\lambda_j x}x^{r_j+1}\, dx}+\fr{1}{2}\right)^2= {}\\[-0.5pt]
= \left(\sqrt{\sum\limits_{j=1}^{k}p_j
\fr{r_j(r_j+1)}{\lambda_j^2}-\left(\sum\limits_{j=1}^{k}p_j
\fr{r_j}{\lambda_j}\right) ^2}+\fr{1}{2}\right)^2\leqslant {}\\[-1.5pt]
{}\leqslant \left(\sqrt{\fr{R(R+1)}{\lambda^2}-\fr{r^2}{\Lambda^2}}+\fr{1}{2}\right)^2\,.
\end{multline*}

Аналогично доказательству Тео\-ре\-мы~2 с~учетом оценки~\eqref{Th3Eq} 
отсюда следует справедливость соотношения~\eqref{Th4Eq}.~\hfill$\square$

\vspace*{-12pt}

\section{Заключение}

Итак, в~работе получены оценки для математического ожидания наблюдений в~предположении 
зашумления конечными смесями нормальных\linebreak (Тео\-ре\-ма~1) 
и~гам\-ма-рас\-пре\-де\-ле\-ний (Тео\-ре\-ма~3). 
%
Построены доверительные интервалы 
для неизвестного математического ожидания в~этих случаях с~использованием 
уточненной оценки~\eqref{Var} 
(Тео\-ре\-мы~2 и~4 соответственно). Отметим, что соответствующие соотношения 
зависят только от <<экстремальных>> значений параметров смесей, но не от числа 
компонент и~весов в~распределении зашумляющих наблюдений. 
%
Замечание~2 
предлагает подход, который  может быть использован для определения неизвестного 
параметра искусственно добавляемого к~исходным данным шума для улучшения качества 
работы метода скользящего разделения смесей.

\smallskip
Автор выражает признательность доктору фи\-зи\-ко-ма\-те\-ма\-ти\-че\-ских наук, 
профессору Виктору Юрьевичу Королеву за идею использования оценки 
дисперсии вида~\eqref{Var} и~другие полезные обсуждения в~рамках 
работы над данной статьей.

\vspace*{-12pt}

{\small\frenchspacing
 {%\baselineskip=10.8pt
 \addcontentsline{toc}{section}{References}
 \begin{thebibliography}{99}
\bibitem{Wright2003} \Au{Wright~D.\,E., Bray~I.} 
A~mixture model for rounded data~// J.~Roy. Stat. Soc.~D 
Sta., 2003. Vol.~52. P.~3--13.

\columnbreak

\bibitem{Bai2009} \Au{Bai~Z., Zheng~S., Zhang~B., Hu~G.} 
Statistical analysis for rounded data~// J.~Stat. Plan.  Infer., 2009. 
Vol.~139. Iss.~8. P.~2526--2542.

\bibitem{Zhang2010} \Au{Zhang~B., Liu~T., Bai~Z.\,D.} 
Analysis of rounded data from dependent sequences~// 
Ann. I.~Stat. Math., 2010. Vol.~62. Iss.~6. P.~1143--1173.

\bibitem{Zhao2012} \Au{Zhao~N., Bai~Z.} 
Analysis of rounded data in mixture normal model~// Stat. Pap., 2012. 
Vol.~53. P.~895--914.

\bibitem{Korolev2011-i} \Au{Королев~В.\,Ю.} 
Ве\-ро\-ят\-но\-ст\-но-ста\-ти\-сти\-че\-ские методы декомпозиции волатильности 
хаотических процессов.~--- М.: Изд-во Моск. ун-та, 2011. 512~с.

\bibitem{Ushakov2015} \Au{Ушаков В.\,Г., Ушаков Н.\,Г.} 
Об усреднении округленных данных~// Информатика и~её применения, 2015. Т.~9. 
Вып.~4. С.~106--109.

\bibitem{Ushakov2017a} \Au{Ушаков~В.\,Г., Ушаков~Н.\,Г.} 
Границы точ\-ности восстановления информации, 
теряемой при округлении результатов наблюдений~// 
Вестник Московского университета. Серия~15: Вычислительная математика и~кибернетика, 
2017. №\,2. С.~26--30.

\bibitem{Ushakov2017b} \Au{Ushakov~N.\,G., Ushakov~V.\,G.} 
Statistical analysis of rounded data: Recovering of information lost due to rounding~// 
J.~Korean Stat. Soc., 2017.  Vol.~46. No.\,3. P.~426--437.

\bibitem{Gorshenin2016} \Au{Gorshenin~A.\,K., Korolev~V.\,Yu.} 
A~noising method for the identification of the stochastic structure of 
information flows~// Comm. Com. Inf. Sc., 2017. 
Vol.~678. P.~279--289.

\bibitem{Gorshenin2013} 
\Au{Gorshenin~A., Korolev~V.} Modelling of statistical
fluctuations of information flows by mixtures of gamma distributions~// 
27th European Conference on Modelling and Simulation Proceedings.~--- 
Dudweiler, Germany: Digitaldruck Pirrot GmbHP, 2013. P.~569--572.
 \end{thebibliography}

 }
 }

\end{multicols}

\vspace*{-6pt}

\hfill{\small\textit{Поступила в~редакцию 03.08.18}}

\vspace*{6pt}

%\newpage

%\vspace*{-24pt}

\hrule

\vspace*{2pt}

\hrule

\vspace*{-2pt}


\def\tit{DATA NOISING BY FINITE NORMAL AND~GAMMA MIXTURES WITH~APPLICATION 
TO~THE~PROBLEM OF~ROUNDED OBSERVATIONS}


\def\titkol{Data noising by finite normal and~gamma mixtures with~application 
to~the~problem of~rounded observations}



\def\aut{A.\,K.~Gorshenin}

\def\autkol{A.\,K.~Gorshenin}

\titel{\tit}{\aut}{\autkol}{\titkol}

\vspace*{-11pt}


\noindent
Institute of Informatics Problems, Federal Research Center ``Computer Science and
Control'' of the Russian Academy of Sciences, 44-2~Vavilov Str., Moscow 119333,
Russian Federation


\def\leftfootline{\small{\textbf{\thepage}
\hfill INFORMATIKA I EE PRIMENENIYA~--- INFORMATICS AND
APPLICATIONS\ \ \ 2018\ \ \ volume~12\ \ \ issue\ 3}
}%
 \def\rightfootline{\small{INFORMATIKA I EE PRIMENENIYA~---
INFORMATICS AND APPLICATIONS\ \ \ 2018\ \ \ volume~12\ \ \ issue\ 3
\hfill \textbf{\thepage}}}

\vspace*{3pt}



\Abste{In many real problems, statistical analysis of data containing additional 
measurement errors, including rounding, is performed, which in some situations can 
lead to sufficiently significant distortions. In this paper, estimates for an 
unknown expectation of observations are obtained for one of the possible 
rounding models under the assumption that the original data are additionally 
noised with random variables having distributions of the type of finite 
mixtures of normal and gamma laws. Confidence intervals for an 
unknown expectation are constructed using the refined estimate for 
the variance of the integer part of the random variable. An algorithm 
for determining the value of the parameter of artificial noise, which 
can be added to the initial data to improve the quality of the 
method of moving separation of mixtures, is discussed.}


\KWE{noisy data; rounded data; finite normal mixtures; finite gamma mixtures; 
confidence intervals; moving separation of mixtures}



\DOI{10.14357/19922264180304}

%\vspace*{-14pt}

\Ack
\noindent
The research was supported by the Russian Science Foundation (project 18-71-00156).



%\vspace*{6pt}

  \begin{multicols}{2}

\renewcommand{\bibname}{\protect\rmfamily References}
%\renewcommand{\bibname}{\large\protect\rm References}

{\small\frenchspacing
 {%\baselineskip=10.8pt
 \addcontentsline{toc}{section}{References}
 \begin{thebibliography}{99}
\bibitem{1-gor-1}
\Aue{Wright,~D.\,E., and I.~Bray.} 2003.
A~mixture model for rounded data.  \textit{J.~Roy. Stat. Soc.~D Sta.} 52:3--13.

\bibitem{2-gor-1}
\Aue{Bai,~Z., S.~Zheng, B.~Zhang, and G.~Hu.} 2009. 
Statistical analysis for rounded data. \textit{J.~Stat. Plan. 
Infer.} 139(8):2526--2542.

\bibitem{3-gor-1}
\Aue{Zhang,~B., T.~Liu, and Z.\,D.~Bai.} 2010. 
Analysis of rounded data from dependent sequences. 
\textit{Ann. I.~Stat. Math.} 62(6):1143--1173.

\bibitem{4-gor-1}
\Aue{Zhao,~N., and Z.~Bai.} 2012. Analysis of rounded data in mixture normal model. 
\textit{Stat. Pap.} 53:895--914.

\bibitem{5-gor-1}
\Aue{Korolev, V.\,Yu.} 2011. 
\textit{Veroyatnostno-statisticheskie metody dekompozitsii volatil'nosti 
khaoticheskikh protsessov} [Probabilistic and statistical methods of 
decomposition of volatility of chaotic processes]. 
Moscow: Moscow University Publishing House. 512~p.

\bibitem{6-gor-1}
\Aue{Ushakov, V.\,G., and N.\,G.~Ushakov.} 
2015. Ob usrednenii okruglennykh dannykh [On averaging of rounded data].
\textit{Informatika i~ee Primeneniya~--- Inform. Appl.} 9(4):106--109.

\bibitem{7-gor-1}
\Aue{Ushakov,~V.\,G., and N.\,G.~Ushakov.} 2017. 
Boundaries of the precision of restoring information lost after rounding
 the results from observations. 
 \textit{Moscow University Computational Math. Cybernetics} 41(2):76--80.

\bibitem{8-gor-1}
\Aue{Ushakov,~N.\,G., and  V.\,G.~Ushakov.} 2017. 
Statistical analysis of rounded data: Recovering of information lost due to rounding. 
\textit{J.~Korean Stat. Soc.} 46(3):426--437.

\bibitem{9-gor-1}
\Aue{Gorshenin,~A.\,K., and V.\,Yu.~Korolev.} 2016. 
A~noising method for the identification of the stochastic structure of information 
flows. \textit{Comm. Com. Inf. Sc.} 678:279--289.

\bibitem{10-gor-1}
\Aue{Gorshenin,~A., and V.~Korolev.} 2013.  Modelling of statistical fluctuations of
information flows by mixtures of gamma distributions. 
\textit{27th European Conference on Modelling and Simulation Proceedings}. 
Dudweiler, Germany: Digitaldruck Pirrot GmbHP. 569--572.

\end{thebibliography}

 }
 }

\end{multicols}

\vspace*{-6pt}

\hfill{\small\textit{Received August 3, 2018}}

%\pagebreak

%\vspace*{-18pt}

\Contrl

\noindent
\textbf{Gorshenin Andrey K.} (b.\ 1986)~--- Candidate of Science (PhD) in physics and
mathematics, associate professor, leading scientist, Institute of Informatics Problems,
Federal Research Center ``Computer Science and Control'' of the Russian Academy of
Sciences, 44-2 Vavilov Str., Moscow 119333, Russian Federation; 
\mbox{agorshenin@frccsc.ru}
\label{end\stat}

\renewcommand{\bibname}{\protect\rm Литература}      