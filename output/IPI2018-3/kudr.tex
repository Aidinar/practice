\def\stat{kudr}

\def\tit{БАЙЕСОВСКИЕ МОДЕЛИ БАЛАНСА$^*$}

\def\titkol{Байесовские модели баланса}

\def\aut{А.\,А.~Кудрявцев$^1$}

\def\autkol{А.\,А.~Кудрявцев}

\titel{\tit}{\aut}{\autkol}{\titkol}

\index{Кудрявцев А.\,А.}
\index{Kudryavtsev A.\,A.}




{\renewcommand{\thefootnote}{\fnsymbol{footnote}} \footnotetext[1]
{Работа выполнена при частичной финансовой поддержке РФФИ (проект 17-07-00577).}}


\renewcommand{\thefootnote}{\arabic{footnote}}
\footnotetext[1]{Московский государственный университет им.~М.\,В.~Ломоносова, 
факультет вычислительной математики и~кибернетики, \mbox{nubigena@mail.ru}}



\Abst{Ряд предыдущих работ автора был посвящен применению байесовского 
подхода к~задачам массового обслуживания и~надежности. 
В~данной статье метод распространяется на широкий круг задач из различных областей 
знания: демографии, физики, политологии, моделирования чрезвычайных ситуаций, 
медицины и~др. В~основе метода лежит разделение факторов, влияющих на исследуемую 
систему, на способствующие функционированию (позитивные, или p-фак\-то\-ры) и~препятствующие функционированию (негативные, или n-фак\-то\-ры). Рассматривается 
индекс баланса сис\-те\-мы, равный отношению n-фак\-то\-ра к~p-фак\-то\-ру, и~индекс преимущества, 
равный отношению p-фак\-то\-ра к~сумме n- и~p-фак\-то\-ров. Предполагается, что факторы, 
влияющие на систему, меняются со временем, причем точные значения факторов 
невозможно определить ввиду несовершенства измерительного оборудования, 
излишне высокой стоимости досконального изучения, нехватки временн$\acute{\mbox{ы}}$х и~материальных 
ресурсов и~т.~п. Такие предпосылки обусловливают применение к~описанным задачам 
байесовского метода, заключающегося в~рандомизации исходных параметров (факторов) 
и,~как следствие, индексов баланса и~преимущества. Основной целью исследования 
является изучение вероятностных характеристик индексов 
баланса и~преимущества в~предположении, что априорные распределения 
факторов известны. В~случае, когда n- и~p-фак\-то\-ры являются независимыми 
случайными величинами, задача сводится к~исследованию свойств смесей распределения. 
В~отличие от популярных в~настоящее время смесей нормальных законов в~байесовских 
моделях баланса смешиваемые распределения имеют положительные носители. 
Особое внимание уделяется априорным распределениям гам\-ма-ти\-па, поскольку 
эти распределения являются адекватными асимптотическими аппроксимациями 
широкого класса вероятностных распределений. Ранее рассматривались смеси 
показательного, эрланговского и~вейбулловского априорных распределений. 
В~данной статье особое внимание уделено случаю, когда n- и~p-факторы имеют 
m-рас\-пре\-де\-ле\-ние Накагами и~его частные виды (распределение Рэлея, 
Макс\-вел\-ла--Больц\-ма\-на, хи-рас\-пре\-де\-ле\-ние и~др.). 
Получены явные виды плот\-ности, функции распределения и~моментов 
индекса баланса для различных комбинаций описанных априорных распределений. 
Результаты статьи могут применяться в~задачах исследования разного рода индексов, 
рейтингов и~показателей.}

\KW{байесовский метод; смешанные распределения; индекс баланса; индекс преимущества; 
процесс баланса; m-рас\-пре\-де\-ле\-ние Накагами}

\DOI{10.14357/19922264180303}
  
%\vspace*{-4pt}


\vskip 10pt plus 9pt minus 6pt

\thispagestyle{headings}

\begin{multicols}{2}

\label{st\stat}


\section{Введение}

\vspace*{-17pt}

Подавляющее большинство аспектов современной жизни~--- 
от бытовых приборов до государственного управ\-ле\-ния~--- усложнилось настолько, 
что определение критериев эффективности путем детерминированного анализа 
стало практически невозможным. По этой причине все чаще можно встретить 
разного рода индексы и~рейтинги, дающие возможность быст\-ро принимать 
решения в~ситуациях, на исследование которых могли бы уйти годы и~значительные 
финансовые и~материальные ресурсы. 

Об эффективности и~адекватности рейтингов 
как инструмента анализа можно спорить~\cite{Yurasova2017}, однако сложно 
представить современный мир без их использования.


\section{Примеры моделей баланса}

\vspace*{-17pt}

В основе построения рейтингов и~индексов обычно лежит разделение параметров 
модели на два класса. 
Первый класс включает па\-ра\-мет\-ры, способствующие 
функционированию целевого объекта и~позитивно влияющие на исследуемый 
процесс\linebreak (\textit{p-фак\-то\-ры}); второй класс включает параметры, препятствующие 
и~негативно влияющие ({\it n-фак\-то\-ры}). Такое разделение достаточно условно.
 Так, при катализе в~зависимости от постановки задачи ингибитор может выступать 
 и~в~роли n-фак\-то\-ра (если требуется увеличить ско\-рость реакции), 
 и~в~роли \mbox{p-фак}\-то\-ра (при замедлении нежелательных реакций).

Разделение параметров на негативные и~позитивные факторы свойственно моделям 
из очень %\linebreak\vspace*{-12pt}
%\pagebreak
%\noindent
  разных областей знания. Так, в~модели Ричардсона гонки 
вооружений~\cite{ShiCh2004} для достижения баланса сил необходимо рассматривать 
коэффициент рос\-та вооружений и~<<прошлые обиды>> как условные p-фак\-то\-ры 
и~бремя расходов и~коэффициент доб\-рой воли как условные n-фак\-то\-ры; 
при определении степени пожарной опас\-ности принято использовать комплексный 
показатель Нестерова~\cite{ShaRaSha2010}, в~котором p-фак\-то\-ром 
можно считать температуру точки росы, а~n-фак\-то\-ром~--- 
температуру воздуха; математическая модель инфекционного заболевания строится 
на основе соотношений баланса~\cite{BoLu2009} для каждой из зависимых переменных: 
концентрация размножающихся антигенов, доля разрушенных антигеном клеток (n-фак\-то\-ры), 
концентрация носителей антител, концентрация антител (p-фак\-то\-ры); 
при анализе экономического развития государства легальные и~теневые базовые 
параметры рынка можно рассматривать и~как p-фак\-то\-ры, 
и~как n-фак\-то\-ры~\cite{Peskova2006}. Можно привести множество других примеров.

Вполне естественно, что функционирование исследуемой сис\-те\-мы в~итоге
 зависит не столько от значений n- и~p-фак\-то\-ров, сколько от их соотношения. 
 При этом большое расхождение между величинами факторов обычно свидетельствует 
 либо о~чрезмерных затратах на <<борьбу с~негативным влиянием>>, либо о~недооценке 
 негативного воздействия. Таким образом, для того чтобы система была\linebreak сбалансированной, 
 имеет смысл стремиться приблизить к~единице отношение n-фак\-то\-ра к~\mbox{p-фак}\-то\-ру. 
 
 Однако существуют постановки задач, в~которых преобладания p-фак\-то\-ра 
 над n-фак\-то\-ром %\linebreak 
 имеет смысл добиваться, невзирая на <<цену вопроса>>. Такие постановки имеют 
 место, когда речь идет, например, о~безопас\-ности или на\-деж\-ности. 
 В~этом случае отношение негативного к~позитивному фактору стремится к~нулю 
 и~для лучшего понимания близости к~решению поставленной задачи рассматривается 
 отношение p-фак\-то\-ра к~сумме\linebreak \mbox{p-} и~n-фак\-то\-ров и~его бли\-зость к~единице.
Наилучшим образом различие между двумя постановками задач можно проиллюстрировать 
современными геополитическими реалиями, в~которых наряду с~<<международными отношениями, 
построенными на балансе взаимодействия и~конкуренции>> присутствует 
идея глобального доминирования, при которой <<свои интересы продавливаются 
любой ценой>>~\cite{Put2017}.

Обозначим через $\lambda$ и~$\mu$ соответственно n- и~\mbox{p-фак}\-то\-ры модели. 
Рассмотрим {\it индекс баланса} $\rho\hm=\lambda/\mu$ и~{\it индекс преимущества}
 $\pi\hm=\mu/(\mu\hm+\lambda)\hm=1/(1\hm+\rho)$. Заметим, что зачастую в~качестве 
 индекса баланса удобно рассматривать величину~$\rho^{-1}$, а наряду с~индексом 
 преимущества рассматривать {\it индекс недостатка} $1\hm-\pi$.

Приведем ряд примеров индексов  баланса и~преимущества и~соответствующих им n- 
и~p-фак\-то\-ров из различных областей знания. Так, в~демографии уровень 
младенческой смертности определяется как отношение числа умерших в~возрасте 
до года к~числу родившихся за период~\cite{Borisov2001}, а~индекс разводимости~--- 
как отношение коэффициента суммарной разводимости к~коэффициенту суммарной 
брач\-ности~\cite{Rybako2005}; в~физике коэффициент трансформации~--- 
это отношение выходного напряжения к~входному, а~универсальная функция Кирхгофа~--- 
отношение излучательной способности тела к~поглощательной~\cite{KuzRog2012}; 
в~тео\-рии массового обслуживания отношение интенсивности входящего потока\linebreak 
к~ин\-тен\-сив\-ности об\-слу\-жи\-ва\-ния определяет коэффициент загрузки 
сис\-те\-мы~\cite{BoPe1995}; при моделировании чрезвычайных ситуаций 
пожароопасность\linebreak
 объекта определяется отношением угрозы возникновения 
пожара к~фактору пожарозащиты~\cite{ShaRaSha2010};\linebreak в~тео\-рии на\-деж\-ности 
ожидаемое время безотказной работы представимо в~виде отношения среднего 
времени безотказной работы к~среднему времени вос\-ста\-нов\-ле\-ния~\cite{ZdRo2004}.

Примерами индексов преимущества служат: в~физике 
коэффициент полезного действия источника тока (n-фак\-тор~--- 
сопротивление источника тока, p-фак\-тор~--- сопротивление нагрузки)~\cite{KuzRog2012}; 
в~политологии предельная доля мобилизованного населения (n-фак\-тор~--- 
коэффициент выбытия, p-фак\-тор~--- коэффициент агитируемости)~\cite{ShiCh2004}; 
в~тео\-рии на\-деж\-ности коэффициент го\-тов\-ности, вероятность пребывания 
в~работоспособном со\-сто\-янии (\mbox{n-фак}\-тор~--- 
среднее время восстановления, \mbox{p-фак}\-тор~--- средняя наработка на 
отказ)~\cite{Kozlov1970,Gost27,ZdRo2004} и~предельная на\-деж\-ность 
(\mbox{n-фак}\-тор~--- сред\-ний параметр дефективности, \mbox{p-фак}\-тор~--- 
средний параметр эффективности)~\cite{KS2006}; 
в~тео\-рии массового обслуживания вероятность того, что вызов не будет потерян 
(n-фак\-тор~--- интенсивность входящего потока, \mbox{p-фак}\-тор~--- 
интенсивность обслуживания)~\cite{IKK1982}.
Индексы баланса и~преимущества также\linebreak можно использовать при анализе рынка ценных\linebreak 
бумаг для исследования процесса дисбаланса интенсивности~\cite{KChKZ2015}, спроса 
при помощи функций Торнквиста и~оптимальных запасов по формуле 
Харриса~\cite{ShiCh2004}. 
%
Ряд примеров можно продолжить.

\vspace*{-12pt}

\section{Байесовские процессы баланса}

\vspace*{-2pt}

С течением времени n- и~p-фак\-то\-ры, а~сле\-довательно, и~индексы ба\-лан\-са/пре\-иму\-щест\-ва 
претер\-пе\-ва\-ют изменения. Это связано с~неустойчивостью среды, в~которой происходит 
функционирование,~--- изменяются экономическое развитие, политическая система, 
технологии производства, пристрастия населения и~т.\,д. По этой причине имеет 
смысл рассматривать не только мгновенные значения факторов и~индексов, 
но и~со\-от\-вет\-ст\-ву\-ющие функции от времени: {\it n-про\-цесс} $\lambda(t)$, 
{\it p-про\-цесс} $\mu(t)$, {\it процесс баланса} $\rho(t)\hm=
\lambda(t)/\mu(t)$ и~{\it процесс преимущества} $\pi(t)\hm=\mu(t)/(\mu(t)\hm+
\lambda(t))$. Кроме того, в~подавляющем большинстве случаев перечисленные 
процессы отслеживаются не постоянно, а~в~некоторые отсчеты времени 
$t_1\hm<t_2<\cdots$. Моменты времени~$t_i$ могут распределяться не однородно 
(минуты, дни, годы), а~быть привязаны к~некоторым внешним событиям 
(пуб\-ли\-ка\-ция доклада, внеплановая проверка, стихийное бедствие и~т.\,п.). 
Обозначим мгновенное значение фактора или индекса через $\xi_i\hm=\xi(t_i)$, 
где $\xi(t)$~--- один из рассматриваемых процессов.

Заметим, что с~течением времени изменяются <<состояния природы>>, которые 
в~тео\-рии вероятностей принято называть элементарными исходами. 
Таким образом, невозможно предугадать частное значение фактора 
из-за невозможности досконального изучения <<состояния природы>>. 
Измерения\linebreak при помощи приборов также не могут дать абсолютно точного 
значения из-за неизбежно вносимых погрешностей, имеющих в~большинстве 
случа\-ев изменчивый (случайный) характер: <<наблюдения в~различные моменты 
времени учитывают влияние изменения условий окружающей среды и~пе\-ре\-ка\-либ\-ров\-ки 
оборудования между наблюдениями>>~\cite{Gost5725}. Это дает предпосылки для 
рассмотрения факторов, а~следовательно, и~индексов как случайных величин. 
При этом стоит учитывать, что глобальные изменения окружающей среды 
происходят достаточно редко, поэтому законы, влияющие на значения факторов, 
можно считать (в~рамках конкретной модели) неизменными. Из этого следует, 
что распределения рассматриваемых случайных величин надо полагать заданными 
априорно.


Приведенные выше рассуждения обусловливают применение к~моделям баланса 
байесовского метода.

В настоящее время в~прикладной статистике, медицине, социологии, экономике 
и~других науках активно применяется байесовский 
подход~\cite{Congdon2006,HWRM2008,CL2008,Albert2009},\linebreak основанный на вычислении 
апостериорных вероятностей и~формуле Байеса~\cite{Laplas1840}; метод, исследовавшийся 
Байесом~\cite{Bayes1763} и~основанный на рандомизации параметров при помощи 
известных \mbox{априорных} распределений, было предложено применять к~задачам 
массового обслуживания и~надежности~\cite{Shorgin05,BKSSh2007,KuSh2015}.

Везде далее будем предполагать, что даны два случайных процесса~$\lambda(t)$ 
и~$\mu(t)$ с~известными априорными распределениями. 
Основной задачей исследования является вычисление (или оценивание) 
распределения процесса баланса~$\rho(t)$ и~процесса\linebreak
 преимущества~$\pi(t)$. 
Зная эти распределения, можно делать выводы о стабильности системы, 
строить прогнозы, исследовать средние и~экстремальные значения и~т.\,п. 
При вычислении таких \mbox{важных} характеристик распределений процессов, 
как моментные функции и~ковариационные матрицы,\linebreak необходимо делать 
предположения о~взаимной зави\-си\-мости проекций процесса. При этом следует учитывать, 
что зачастую \mbox{n-про}\-цесс зависит от \mbox{p-про}\-цес\-са и~наоборот. Так, при подготовке 
об\-нов\-ле\-ния программного обеспечения, необходимого для устранения сбоев, 
улучшения показателей производительности и/или других характеристик 
продукта~\cite{Orlik2005}, p-фак\-тор меняется в~зависимости от n-фак\-то\-ра, 
а~мутацию вируса гриппа под влиянием вакцинации и~эффект первородного
 антигенного греха~\cite{Amantonio,KSCJ,Boni} можно рассматривать 
 как влияние p-фак\-то\-ра на n-фак\-тор.

Строить предположения о зависимости факторов следует исходя из конкретной модели. 
Однако, если взять за основу за\-ви\-си\-мость~$\lambda_i$ только от~$\mu_{i-1}$, 
а~$\mu_i$~--- только от~$\lambda_{i-1}$, $i\hm=1,2,\ldots$, а~так\-же 
не\-за\-ви\-си\-мость~$\lambda_0$ и~$\mu_0$ в~начальный момент времени, 
получаем для каждого~$i$ независимость пары случайных величин~$(\lambda_i,\, \mu_i)$.

Исследование процессов баланса и~преимущества и~их конечномерных распределений,
 таким образом, имеет смысл осуществлять при помощи метода анализа 
 масштабных смесей априорных распределений негативных и~позитивных факторов. 
 При этом в~байесовских моделях баланса, в~отличие от популярных моделей, 
 описываемых в~терминах сдвиговых и~масштабных смесей нормальных 
 законов~\cite{Korolev2011}, оба смешиваемых распределения в~большинстве 
 случаев имеют положительные носители.
 
 \vspace*{-6pt}

\section{Примеры вычисления одномерных распределений процессов баланса}

 \vspace*{-2pt}

Основная сложность при исследовании проекций процессов баланса заключается 
в~том, что характеристики распределений индексов баланса и~преимущества 
зачастую выражаются в~терминах специальных функций: интегральной показательной 
функции Эйлера~\cite{KuShSh2009}, обобщенной гипергеометрической функции 
Гаусса~\cite{ZhaKuSh2014}, преобразований Лапласа специального вида~\cite{KuTi2016}, 
гам\-ма-экс\-по\-нен\-ци\-аль\-ной функции~\cite{KuTi2017}, бета- и~гам\-ма-функ\-ций и~др. 
Данное обстоятельство не только вынуждает применять компьютерные средства для 
вычисления характеристик модели, но и~искать новые аналитические
 подходы~\cite{KuTi2017}.

Рассмотрим ряд утверждений, обоснования которых приводятся в~следующем разделе, 
с~примерами характеристик баланса.

Введем следующие обозначения. Пусть~$f_\lambda(x)$ и~$F_\lambda(x)$~--- 
соответственно плот\-ность и~функция распределения случайной величины~$\lambda$. 
Через~$B(q,p)$ и~$\Gamma(q)$, $q,p\hm>0$, будем соответственно обозначать 
бета- и~гам\-ма-функ\-ции. Пусть
$$
\left(\alpha\right)_j=\alpha (\alpha+1)\cdots (\alpha+j-1)\,,\enskip
 (\alpha)_0=1.$$
Рассмотрим вырожденную гипергеометрическую функцию
\begin{multline*}
G_0(\alpha,\beta;x)=\sum\limits_{j=0}^\infty
\fr{(\alpha)_jx^j}{(\beta)_j \,j!}\,,\\
\alpha\in\mathbb{R}\,,\ \beta\neq 0, -1,\ldots,\ \ x\in\mathbb{R}\,,
\end{multline*}
неполную гамма-функцию
$$
\gamma(p,y)=\int\limits_0^y x^{p-1} e^{-x}\, dx\,,\enskip p>0\,,\enskip y>0\,,
$$
и неполную бета-функцию
$$
\beta(q,p,y)=\int\limits_0^y x^{q-1}(1-x)^{p-1} \,dx\,, \enskip q,p>0\,,\enskip y>0\,.
$$


В дальнейшем изложении речь будет идти о случайных величинах~$\lambda$, 
имеющих бе\-та-рас\-пре\-де\-ле\-ние:
$$
f_\lambda(x)=\fr{x^{q-1}(1-x)^{\theta-1}}{B(q,\theta)}\,,\enskip
q,\theta>0\,, \enskip x\in(0,\,1)\,,
$$
гамма-распределение:
$$
f_\lambda(x)=\fr{\theta^q x^{q-1}e^{-\theta x}}{\Gamma(q)}\,,\enskip 
q,\theta>0\,, \enskip x>0\,,
$$
m-распределение Накагами $m(q,\theta)$~\cite{Naka1960}:
\begin{multline}
\label{Density_lambda_m}
f_\lambda(x)=\fr{2 q^q x^{2q-1}}{\theta^q\Gamma(q)}
\exp\left\{- \fr{q x^2}{\theta}\right\}\,,\\ 
q\ge1/2\,,\ \theta>0\,, \  x>0\,,
\end{multline}
распределение максимума процесса броуновского движения~$\mathrm{Bm}(\theta)$ \cite{Kruglov2016}:
\begin{equation}
\label{Density_lambda_Bm}
f_\lambda(x)=\fr{2 }{\sqrt{2\pi\theta}}
\exp\left\{- \fr{x^2}{2\theta}\right\},\enskip \theta>0\,, \enskip x>0\,,
\end{equation}
распределение Рэлея $\mathrm{Ray}\,(\theta)$~\cite{Siddiqui1964}:
\begin{equation}
\label{Density_lambda_Ray}
f_\lambda(x)=\fr{2 x}{\theta^2}\exp\left\{-  
\fr{x^2}{\theta^2}\right\}\,,\enskip \theta>0\,, \enskip x>0\,,
\end{equation}
хи-распределение $\chi(q)$:
\begin{equation}
\label{Density_lambda_chi}
f_\lambda(x)=\fr{x^{q-1}e^{- x^2/2}}{2^{q/2-1}\Gamma(q/2)}\,,\enskip
q>0\,, \enskip x>0\,,
\end{equation}
распределение Максвелла--Больц\-ма\-на $\mathrm{MB}\,(\theta)$~\cite{Mandl2008}:
\begin{equation}
\label{Density_lambda_MB}
f_\lambda(x)=\sqrt{\fr{2}{\pi}}\,
\fr{x^2}{\theta^{3/2}}\exp\left\{-  \fr{x^2}{2\theta}\right\}\,,\enskip
\theta>0\,, \enskip x>0\,.
\end{equation}
Плотности $f_\mu(x)$ случайных величин~$\mu$ будут иметь аналогичный вид с~заменой 
па\-ра\-мет\-ров~$(q,\theta)$ парой па\-ра\-мет\-ров~$(p,\alpha)$.

\smallskip

\noindent
\textbf{Теорема~1.}\ 
\textit{Пусть негативный фактор~$\lambda$ имеет гам\-ма-рас\-пре\-де\-ле\-ние 
с~параметрами~$q$ и~$\theta$, а~позитивный фактор~$\mu$ имеет бе\-та-рас\-пре\-де\-ле\-ние 
с~параметрами~$p$ и~$\alpha$, причем~$\lambda$ и~$\mu$ независимы. 
Тогда индекс баланса $\rho\hm=\lambda/\mu$ имеет плот\-ность}

\noindent
\begin{multline}
\label{Density_rho_gamma_beta}
f_\rho(x)=\fr{\theta^q x^{q-1} B(q+p,\alpha)}{\Gamma(q)B(p,\alpha)}\,
 G_0(q+p,q+p+\alpha;-\theta x)\,,\\
   x>0\,,
\end{multline}
\textit{и функцию распределения}

\noindent
\begin{multline*}
F_\rho(x)=\fr{\theta^q B(q+p,\alpha)}{\Gamma(q)B(p,\alpha)}
\sum\limits_{j=0}^{\infty}\fr{(q+p)_j(-\theta)^j}{(q+p+\alpha)_jj!}\,
\fr{x^{q+j}}{q+j}\,,\\ x>0\,.
\end{multline*}

\smallskip

Выражение плотности индекса баланса~$\rho$ через специальную функцию~$G_0$ 
затрудняет исследование свойств распределения~$\rho$ стандартными методами 
тео\-рии вероятностей и~математической статистики. Однако для приложений 
существенно большую роль играют моментные характеристики~$\rho$.

Следующее утверждение и~предположение о независимости n- 
и~p-фак\-то\-ров дает возможность исследовать средние значения индекса баланса.

\smallskip

\noindent
\textbf{Теорема~2.}\ 
\textit{Пусть негативный фактор~$\lambda$ имеет вырожденное в~$\theta\hm>0$ 
распределение, а~позитивный фактор~$\mu$ имеет бе\-та-рас\-пре\-де\-ле\-ние 
с~па\-ра\-мет\-ра\-ми~$p$ и~$\alpha$. 
Тогда вероятностные характеристики индекса баланса~$\rho$ имеют вид}:
\begin{align*}
f_\rho(x)&=\fr{x^{-1}}{B(p,\alpha)}
\left(\fr{\theta}{x}\right)^p\left(1-\fr{\theta}{x}\right)^{\alpha-1}\,,\enskip
x>\theta;\\
{\sf E}\,\rho^l&=\fr{\theta^{l}B(p-l,\alpha)}{B(p,\alpha)}\,,\enskip l<p\,.
\end{align*}


\noindent
\textbf{Следствие~1.}\
Пусть негативный фактор~$\lambda$ имеет гам\-ма-рас\-пре\-де\-ле\-ние 
с~параметрами~$q$ и~$\theta$, а~позитив-\linebreak\vspace*{-12pt}

\pagebreak

\noindent
ный фактор~$\mu$ 
имеет бе\-та-рас\-пре\-де\-ле\-ние с~па\-ра\-мет\-ра\-ми~$p$ и~$\alpha$, 
причем~$\lambda$ и~$\mu$ независимы. Тогда для моментов индекса баланса~$\rho$ 
справедливо
$$
{\sf E}\,\rho^l=\fr{\Gamma(q+l)B(p-l,\alpha)}{\theta^l\Gamma(q)B(p,\alpha)}\,,\enskip
l<p\,.
$$


Отдельное место при исследовании индексов баланса занимают модели, в~которых
 распределения n- и~p-фак\-то\-ров являются различными обобщениями гам\-ма-рас\-пре\-де\-ле\-ния. 
 Это обусловлено тем, что распределения гам\-ма-ти\-па встречаются во многих 
 прикладных задачах. Так, m-рас\-пре\-де\-ле\-ние 
 Накагами используется для моделирования замираний сигналов в~беспроводных каналах 
 связи, распределение Рэлея было предложено для исследования гармонических колебаний 
 со случайными фазами и~применяется для описания амплитудных флуктуаций радиосигнала, 
 распределение Макс\-вел\-ла--Больц\-ма\-на лежит в~основании кинетической тео\-рии 
 газов и~описывает электронные процессы переноса и~т.\,д. Кроме того, распределения 
 гам\-ма-ти\-па выступают в~роли адекватных асимптотических аппроксимаций 
 широкого круга вероятностных распределений~\cite{ZaKo2013}. В~свою очередь, 
 это дает возможность рассматривать факторы с~распределениями гам\-ма-ти\-па 
 в~качестве агрегирующих факторов.

Ранее рассматривались модели, в~которых факторы имели показательное~\cite{KuSh2009},
 эрланговское~\cite{KuShSh2009}  и~вейбулловское распределения~\cite{KuTi2016,KuTi2017}. 
 Приведем ряд примеров применения аналогичных методов получения характеристик 
 индекса баланса для распределения Накагами и~некоторых его частных случаев.
 
 \smallskip
 
 \noindent
 \textbf{Теорема~3.}
\textit{Пусть негативный и~позитивный факторы~$\lambda$ и~$\mu$ 
имеют m-рас\-пре\-де\-ле\-ние Накагами с~па\-ра\-мет\-ра\-ми $(q, \theta)$
 и~$(p, \alpha)$ соответственно, причем~$\lambda$ и~$\mu$ независимы. 
 Тогда индекс баланса~$\rho$ имеет плот\-ность распределения}
\begin{equation}
\label{Density_rho_m_m}
f_\rho(x)=\fr{2(q/\theta)^q(p/\alpha)^p  x^{2q-1}}
{B(q,p) (q x^2/\theta+p/\alpha)^{q+p}}\,,\enskip x>0\,,
\end{equation}
\textit{и функцию распределения}
%\begin{equation}\label{Distrib_rho_m_m}
$$
F_\rho(x)=\fr{\beta\left(q,p,\alpha qx^2/(\alpha qx^2+\theta p)\right)}{B(q,p)}\,,\enskip 
x>0\,.
$$


Для нахождения моментов, соответствующих плотности~(\ref{Density_rho_m_m}), 
воспользуемся следующим утверж\-де\-нием.

\smallskip

\noindent
\textbf{Теорема~4.}
\textit{Пусть негативный фактор~$\lambda$ имеет вырожденное в~$\theta\hm>0$ 
распределение, а позитивный фактор~$\mu$ имеет распределение Накагами 
с~па\-ра\-мет\-ра\-ми~$p$ и~$\alpha$. Тогда вероятностные характеристики индекса баланса~$\rho$ 
имеют вид}:

\noindent
\begin{alignat*}{2}
F_\rho(x)&=1-\fr{\gamma\left(p, \theta^2p/\left(\alpha x^2\right)\right)}{\Gamma(p)}\,,&\  x&>0;\\
f_\rho(x)&=\fr{2\theta^{2p}(p/\alpha)^p}{\Gamma(p) x^{2p+1}}\exp\left\{-
\fr{\theta^2 p}{\alpha x^2}\right\}\,,&\  x&>0;\\
{\sf E}\rho^l&=\fr{\theta^l(p/\alpha)^{l/2}}{\Gamma(p)}\Gamma\left(p-\fr{l}{2}\right)\,,&\  l&<2p\,.
\end{alignat*}


\noindent
\textbf{Следствие~2.}\
Пусть негативный и~позитивный факторы~$\lambda$ и~$\mu$ имеют распределение 
Накагами с~параметрами $(q, \theta)$ и~$(p, \alpha)$ соответственно, причем~$\lambda$ 
и~$\mu$ независимы. Тогда для моментов индекса баланса~$\rho$ справедливо
$$
{\sf E}\rho^l=\fr{(p/\alpha)^{l/2}\Gamma(q+l/2)}
{(q/\theta)^{l/2}\Gamma(q)\Gamma(p)}\Gamma\left(p-\fr{l}{2}\right)\,,\enskip l<2p\,.
$$


\smallskip

Заметим, что плотность распределения Накагами~(\ref{Density_lambda_m}) обобщает 
плот\-ности распределений~(\ref{Density_lambda_Bm})--(\ref{Density_lambda_MB}), 
что дает возможность сформулировать ряд следствий для всевозможных комбинаций 
априорных распределений~(\ref{Density_lambda_m})--(\ref{Density_lambda_MB}) 
n- и~p-фак\-то\-ров.

\smallskip

\noindent
\textbf{Следствие~3.}\
Пусть негативный и~позитивный факторы~$\lambda$ и~$\mu$ 
имеют распределения, определяемые 
плотностями~(\ref{Density_lambda_m})--(\ref{Density_lambda_MB}) 
соответственно с~параметрами $(q, \theta)$ и~$(p, \alpha)$, 
причем~$\lambda$ и~$\mu$ независимы. Тогда вероятностные характеристики 
индекса баланса~$\rho$ имеют вид:
\begin{alignat*}{2}
f_\rho(x)&=\fr{2R^QS^P x^{2Q-1}}{B(Q,P) (R x^2+S)^{Q+P}}\,,&\ x&>0\,;\\
F_\rho(x)&=\fr{\beta\left(Q,P,Rx^2/\left(Rx^2+S\right)\right)}{B(Q,P)}\,,&\  x&>0\,;\\
{\sf E}\rho^l&=\fr{S^{l/2}\Gamma(Q+l/2)}{R^{l/2}\Gamma(Q)\Gamma(P)}\Gamma
\left(P-\fr{l}{2}\right)\,,&\ 
l&<2P\,,
\end{alignat*}
где вектор констант $(Q,R,P,S)$ определяется из~таб\-лицы.


\begin{table*}\small
\begin{center}


%\tabcolsep=5pt
\begin{tabular}{|c|c|c|c|c|c|}
\multicolumn{6}{c}{Значения констант $(Q,R,P,S)$ для частных случаев m-распределения Накагами}\\
\multicolumn{6}{c}{\ }\\[-6pt]
\hline
&\multicolumn{5}{c|}{$\mu$}\\
\cline{2-6}
\multicolumn{1}{|c|}{\raisebox{6pt}[0pt][0pt]{$\lambda$}}
& $m\left(p,\alpha\right)$&$\mathrm{Bm}\left(\alpha\right)$& 
$\mathrm{Ray}\left(\alpha\right)$& $\chi\left(p\right)$& $\mathrm{MB}\left(\alpha\right)$\\
\hline
&&&&&\\[-9pt]
 $m\left(q,\theta\right)$& $\left(q,\fr{q}{\theta},p,\fr{p}{\alpha}\right)$& $\left(q,\fr{q}{\theta},\fr{1}{2},\fr{1}{2\alpha}\right)$& $\left(q,\fr{q}{\theta},1,\fr{1}{\alpha^2}\right)$& $\left(q,\fr{q}{\theta},\fr{p}{2},\fr{1}{2}\right)$& 
 $\left(q,\fr{q}{\theta},\fr{3}{2},\fr{1}{2\alpha}\right)$\\
 $\mathrm{Bm}\left(\theta\right)$& $\left(\fr{1}{2},\fr{1}{2\theta},p,\fr{p}{\alpha}\right)$& $\left(\fr{1}{2},\fr{1}{2\theta},\fr{1}{2},\fr{1}{2\alpha}\right)$& $\left(\fr{1}{2},\fr{1}{2\theta},1,\fr{1}{\alpha^2}\right)$& $\left(\fr{1}{2},\fr{1}{2\theta},\fr{p}{2},\fr{1}{2}\right)$& $\left(\fr{1}{2},\fr{1}{2\theta},\fr{3}{2},\fr{1}{2\alpha}\right)$\\
 $\mathrm{Ray}\left(\theta\right)$& $\left(1,\fr{1}{\theta^2},p,\fr{p}{\alpha}\right)$& $\left(1,\fr{1}{\theta^2},\fr{1}{2},\fr{1}{2\alpha}\right)$& $\left(1,\fr{1}{\theta^2},1,\fr{1}{\alpha^2}\right)$& $\left(1,\fr{1}{\theta^2},\fr{p}{2},\fr{1}{2}\right)$& $\left(1,\fr{1}{\theta^2},\fr{3}{2},\fr{1}{2\alpha}\right)$\\
 $\chi\left(q\right)$& $\left(\fr{q}{2},\fr{1}{2},p,\fr{p}{\alpha}\right)$& $\left(\fr{q}{2},\fr{1}{2},\fr{1}{2},\fr{1}{2\alpha}\right)$& $\left(\fr{q}{2},\fr{1}{2},1,\fr{1}{\alpha^2}\right)$& $\left(\fr{q}{2},\fr{1}{2},\fr{p}{2},\fr{1}{2}\right)$& $\left(\fr{q}{2},\fr{1}{2},\fr{3}{2},\fr{1}{2\alpha}\right)$\\
 $\mathrm{MB}\left(\theta\right)$& $\left(\fr{3}{2},\fr{1}{2\theta},p,\fr{p}{\alpha}\right)$& $\left(\fr{3}{2},\fr{1}{2\theta},\fr{1}{2},\fr{1}{2\alpha}\right)$& $\left(\fr{3}{2},\fr{1}{2\theta},1,\fr{1}{\alpha^2}\right)$& $\left(\fr{3}{2},\fr{1}{2\theta},\fr{p}{2},\fr{1}{2}\right)$& $\left(\fr{3}{2},\fr{1}{2\theta},\fr{3}{2},\fr{1}{2\alpha}\right)$\\
\hline
\end{tabular}
\end{center}
\end{table*}

\noindent
\textbf{Замечание~1.}
Условие $l\hm<2P$ в~следствии~3 ограничивает число конечных моментов. 
Например, в~случае, когда  \mbox{p-фак}\-тор имеет распределение Рэлея, 
у~индекса баланса~$\rho$ не существует дисперсии.


\smallskip

\noindent
\textbf{Замечание~2.}
Для получения вероятностных характеристик индекса преимущества $\pi\hm=1/(1\hm+\rho)$ 
можно использовать соотношения:
\begin{multline*}
f_\pi(x)=\fr{1}{x^2}f_\rho\left(\fr{1-x}{x}\right)\,,\\ 
F_\pi(x)=1-F_\rho\left(\fr{1-x}{x}\right)\,,\enskip
x\in(0,1).
\end{multline*}
Однако вычисление моментов ${\sf E}\pi^l$ представляет отдельную задачу, 
в~том числе и~по причине представления плотности~$\rho$ в~терминах специальных функций.

\section{Доказательства утверждений}

Приведем обоснования утверждений из разд.~4.

\noindent
Д\,о\,к\,а\,з\,а\,т\,е\,л\,ь\,с\,т\,в\,о\ \ {\bf теоремы~1}. 
Поскольку при $x\hm>0$
\begin{multline*}
f_\rho(x)=\int\limits_0^1yf_\lambda(xy)f_\mu(y)\,dy={}\\
{}=\fr{\theta^qx^{q-1}}{\Gamma(q)B(p,\alpha)}
\int\limits_0^1y^{q+p-1}(1-y)^{\alpha-1}e^{-\theta xy}\,dy\,,
\end{multline*}
применив~\cite[формула 3.383]{GR1971}, получаем~(\ref{Density_rho_gamma_beta}). 
Выражение для функции распределения~$\rho$ получаем, 
проинтегрировав~(\ref{Density_rho_gamma_beta}).

\smallskip

\noindent
Д\,о\,к\,а\,з\,а\,т\,е\,л\,ь\,с\,т\,в\,о\ \ {\bf теоремы~2} 
непосредственно вытекает из равенства
$$
f_\rho(x)=\fr{\theta}{x^2}\,f_\mu\left(\fr{\theta}{x}\right)\,, \enskip x>\theta\,.
$$

\noindent
Д\,о\,к\,а\,з\,а\,т\,е\,л\,ь\,с\,т\,в\,о\ \ {\bf теоремы~3}. 
Для плотности индекса баланса при $x\hm>0$ имеем
\begin{multline*}
f_\rho(x)={}\\
\hspace*{-2pt}\!{}=\fr{4(q/\theta)^q(p/\alpha)^p}{\Gamma(q)\Gamma(p)}\,x^{2q-1}
\!\!\!\int\limits_0^\infty \!y^{2q+2p-1}e^{-(qx^2/\theta+p/\alpha)y^2}\,dy={}\hspace*{-3.94243pt}\\
{}=\fr{2(q/\theta)^q(p/\alpha)^p}{\Gamma(q)\,\Gamma(p)}x^{2q-1}
\int\limits_0^\infty t^{q+p-1}e^{-(qx^2/\theta+p/\alpha)t}\,dt={}\\
{}=
\fr{2(q/\theta)^q(p/\alpha)^p x^{2q-1}}{B(q,p) (q x^2/\theta+p/\alpha)^{q+p}}\,,
\end{multline*}
откуда получаем:
\begin{multline*}
F_\rho(x)=\fr{2(q/\theta)^q(p/\alpha)^p}{B(q,p)}
\int\limits_0^x \fr{u^{2q-1}\,du}{(q u^2/\theta+p/\alpha)^{q+p}}={}\\
{}=\fr{(p/\alpha)^p}{B(q,p)}
\int\limits_{p/\alpha}^{qx^2/\theta+p/\alpha}\fr{(t-p/\alpha)^{q-1}}{t^{q+p}}\,dt={}\\
{}=
\fr{1}{B(q,p)}\int\limits_0^{qx^2/(qx^2+\theta p/\alpha)}z^{q-1}(1-z)^{p-1}\,dz\,.
\end{multline*}

\noindent
Д\,о\,к\,а\,з\,а\,т\,е\,л\,ь\,с\,т\,в\,о\ \ {\bf теоремы~4}. Поскольку при $x\hm>0$
$$
F_\rho(x)=1-\fr{(p/\alpha)^p}{\Gamma(p)}\int\limits_0^{\theta^2/x^2} y^{p-1}
e^{-py/\alpha}\,dy\,,
$$
продифференцировав по~$x$, получаем $f_\rho(x)$ и~выражение для моментов при $l\hm<2p$:
\begin{multline*}
{\sf E}\rho^l=\fr{\theta^{2p}(p/\alpha)^p}{\Gamma(p)}
\int\limits_0^\infty y^{p-l/2-1}e^{-\theta^2py/\alpha}\,dy={}\\
{}=
\fr{\theta^l(p/\alpha)^{l/2}}{\Gamma(p)}\,\Gamma\left(p-\fr{l}{2}\right)\,.
\end{multline*}
Последнее равенство наряду с~представлением моментов распределения Накагами
$${\sf E}\lambda^l=\fr{\Gamma(q+l/2)}{(q/\theta)^{l/2}\Gamma(q)}
$$
обосновывает следствие~2.

\section{Заключение}

В данной работе было существенно расширено множество возможных областей
 применения байесовских постановок задач определения значений индексов 
 баланса и~преимущества, целесообразность которых ранее обусловливалась 
 только применительно к~теории массового обслуживания 
 и~на\-деж\-ности~\cite{Shorgin05,BKSSh2007,KuSh2015}. Решение задачи нахождения 
 вероятностных характеристик частного двух случайных факторов имеет не 
 только практическую значимость, но и~представляет тео\-ре\-ти\-че\-ский интерес.



{\small\frenchspacing
 {%\baselineskip=10.8pt
 \addcontentsline{toc}{section}{References}
 \begin{thebibliography}{99}

\bibitem{Yurasova2017}
\Au{Юрасова~М.\,В.}
Рейтинг как инструмент измерения успеха: <<за>> и~<<против>>~// 
Вестник Московского ун-та. Сер. 18: Социология и~политология, 2017. Т.~23. Вып.~2. С.~137--164.

\bibitem{ShiCh2004}
\Au{Шикин~Е.\,В., Чхартишвили~А.\,Г.}
Математические методы и~модели в~управлении.~--- 3-е изд.~---
М.: Дело, 2004. 440~с.

\bibitem{ShaRaSha2010}
\Au{Шаптала~В.\,Г., Радоуцкий~В.\,Ю., Шаптала~В.\,В.}
Основы моделирования чрезвычайных ситуаций.
%/ Под общ. ред. В.\,Г.~Шапталы.
--- Белгород: БГТУ, 2010. 166~с.

\bibitem{BoLu2009}
\Au{Болодурина~И.\,П., Луговскова~Ю.\,П.}
Оптимальное управление динамикой взаимодействия иммунной системы человека 
с~инфекционными заболеваниями~// Вестник СамГУ. Естественнонаучная сер., 2009. 
№\,8(74). С.~138--153.

\bibitem{Peskova2006}
\Au{Пескова~Д.\,Р.}
Теневой сектор: ингибитор или катализатор экономического развития?~// 
Вестник Башкирского ун-та. Раздел экономика, 2006. Т.~11. Вып.~3. С.~141--143.

\bibitem{Put2017}
%\textit{\/}
Мир будущего: через столкновение к~гармонии: 
Итоговая пленарная сессия XIV~ежегодного заседания Международного 
дискуссионного клуба <<Валдай>>.~--- Сочи, 2017. 
{\sf http://kremlin.ru/events/president/news/\linebreak 55882}.

\bibitem{Borisov2001}
\Au{Борисов~В.\,А.}
Демография.~--- М.: NOTABENE, 2001. 272~с.

\bibitem{Rybako2005}
\Au{Волгин~Н.\,А., Рыбаковский~Л.\,Л., Калмыкова~Н.\,М., Ар\-хан\-гель\-ский~В.\,Н., 
Иванова~Е.\,И., Захарова~О.\,Д., Иванова~А.\,Е., Денисенко~М.\,Б., 
Тихомиров~Н.\,П., Тихомирова~Т.\,М.}
Демография~/ Под ред. Н.\,А.~Волгина, Л.\,Л.~Рыбаковского.~--- М.: Логос, 2005. 280~с.

\bibitem{KuzRog2012}
\Au{Кузнецов~С.\, И., Рогозин~К.\, И.}
Справочник по физике.~--- Томск: ТПУ, 2012. 224~с.

\bibitem{BoPe1995}
\Au{Бочаров~П.\,П., Печинкин~А.\,В.}
Теория массового обслуживания.~--- М.: РУДН, 1995. 529~с.

\bibitem{ZdRo2004}
\Au{Здоровцов~И.\,А., Королев~В.\,Ю.}
Основы теории надежности во\-ло\-кон\-но-оп\-ти\-че\-ских 
линий передачи железнодорожного транспорта.~--- М.: МАКС Пресс, 2004. 308~с.

\bibitem{Kozlov1970}
\Au{Kozlov~B.\,A., Ushakov~I.\,A.}
Reliability handbook.~---  New York, NY, USA: Holt, Rinehart \& Winston. 1970. 391~p.

\bibitem{Gost27}
ГОСТ 27.002-89 Надежность в~технике. Основные понятия. Термины и~определения.~---
М.: Госкомстандарт, 1989. 24~с.

\bibitem{KS2006}
\Au{Королев~В.\,Ю., Соколов~И.\,А.}
Основы математической теории надежности модифицируемых сис\-тем.~--- М.: ИПИ РАН, 2006. 102~с.

\bibitem{IKK1982}
\Au{Ивченко~Г.\,И., Каштанов~В.\,А., Коваленко~И.\,Н.}
Теория массового обслуживания.~--- 
М.: Высшая школа, 1982. 256~с.

\bibitem{KChKZ2015}
\Au{Korolev~V.\,Yu., Chertok~A.\,V., Korchagin~A.\,Yu, Zeifman~A.\,I.}
Modeling high-frequency order flow imbalance by functional limit theorems 
for two-sided risk processes~// Appl. Math. Comput., 2015. Vol.~253. P.~224--241.
%Applied Mathematics and Computation (New York), издательство Elsevier BV (Netherlands), том 253, с. 224-241, 2015

\bibitem{Gost5725}
ГОСТ Р ИСО 5725-1-2002 Точность (правильность и~прецизионность) 
методов и~результатов измерений. Часть~1. Основные положения 
и~определения.~--- М.: Стандартинформ, 2009. 24~с.

\bibitem{Congdon2006}
\Au{Congdon~P.\/}
Bayesian statistical modelling.~--- 2nd ed.~--- Chichester, U.K.: John Wiley \& Sons, 2006. 596~p.

\bibitem{HWRM2008}
\Au{Hamada~M.\,S., Wilson~A., Reese~C. S., Martz~H.}
Bayesian reliability.~--- New York, NY, USA: Springer, 2008. 436~p.

\bibitem{CL2008}
\Au{Carlin~B.\,P., Louis~T.\,A.}
Bayesian methods for data analysis.~--- 3rd ed.~--- 
New York, NY, USA: Chapman \& Hall, 2008. 552~p.

\bibitem{Albert2009}
\Au{Albert~J.}
Bayesian computation with~$R$.~--- New York, NY, USA: Springer, 2009. 300~p.

\bibitem{Laplas1840}
\Au{Laplace~P.-S.}
A~philosophical essay on probabilities~/ 
Transl. from the French by F.\,W.~Truscott, F.\,L.~Emory.~--- 
New York, NY, USA: John Wiley \& Sons, 1902. 223~p.

\bibitem{Bayes1763}
\Au{Bayes~T., Price~R.}
An essay towards solving a~problem in the doctrine of chances~// 
Phil. Trans., 1763. Vol.~53. P.~370--418.

\bibitem{Shorgin05}
\Au{Шоргин~С.\,Я.}
О~байесовских моделях массового обслуживания~// 
II Научная сессия Института проб\-лем информатики РАН: Тезисы докладов.~--- 
М.: ИПИ РАН, 2005. С.~120--121.

\bibitem{BKSSh2007}
\Au{Бенинг~В.\,Е., Королев~В.\,Ю., Соколов~И.\,А., Шоргин~С.\,Я.}
Рандомизационные модели и~методы тео\-рии на\-деж\-ности информационных 
и~технических сис\-тем.~--- М.: ТОРУС ПРЕСС, 2007. 256~с.

\bibitem{KuSh2015}
\Au{Кудрявцев~А.\,А., Шоргин~С.\,Я.\/}
Байесовские модели в~тео\-рии массового обслуживания и~надежности.~--- 
М.: ФИЦ ИУ РАН, 2015. 76~с.

\bibitem{Orlik2005} %27
\Au{Орлик~C.}
Введение в~программную инженерию и~управ\-ле\-ние жизненным циклом ПО. 
Программная инженерия. Сопровождение программного обеспечения, 2004--2005.
{\sf http://www.\linebreak software-testing.ru/files/se/3-5-software\_engineering\_\linebreak maintenance.pdf}.

\bibitem{Boni} %28
\Au{Boni~M.\,F.}
Vaccination and antigenic drift in influenza~// Vaccine, 2008. Vol.~26. Suppl.~3. 
P.~8--14.

\bibitem{KSCJ} %29
\Au{Kim~J.\,H., Skountzou~I., Compans~R., Jacob~J.}
Original antigenic sin responses to influenza viruses~// 
J.~Immunology, 2009.
Vol.~183. Iss.~5. P.~3294--3301.

\bibitem{Amantonio} %30
\Au{Amantonio.} Разбираемся с~прививками. Часть~20. 
Грипп~(1)~// Livejournal, 09.11.2017. 
{\sf https:// amantonio. livejournal.com/29621.html}. Продолжение: Часть~21. 
Грипп~(2)~// Livejournal, 10.11.2017. 
{\sf https:// amantonio. livejournal.com/29886.html}.





\bibitem{Korolev2011}
\Au{Королев~В.\,Ю.}
Ве\-ро\-ят\-но\-ст\-но-ста\-ти\-сти\-че\-ские 
методы декомпозиции волатильности хаотических процессов.~--- 
М.: Изд-во Московского ун-та, 2011. 510~с.

\bibitem{KuShSh2009}
\Au{Кудрявцев~А.\,А., Шоргин~В.\,С., Шоргин~С.\,Я.}
Байесовские модели массового обслуживания и~надежности: общий эрланговский случай~// 
Информатика и~её применения, 2009. Т.~3. Вып.~4. С.~30--34.

\bibitem{ZhaKuSh2014}
\Au{Жаворонкова~Ю.\,В., Кудрявцев~А.\,А., Шоргин~С.\,Я.}
Байесовская рекуррентная модель роста на\-деж\-ности: бе\-та-рас\-пре\-де\-ле\-ние параметров~// 
Информатика и~её применения, 2014. Т.~8. Вып.~2. С.~48--54.

\bibitem{KuTi2016}
\Au{Кудрявцев А.\,А., Титова А.\,И.}
Байесовские модели массового обслуживания и~надежности: 
вы\-рож\-ден\-но-вей\-бул\-лов\-ский случай~// Информатика и~её применения, 2016. Т.~10. Вып.~4. С.~68--71.

\bibitem{KuTi2017}
\Au{Кудрявцев~А.\,А., Титова~А.\,И.}
Гам\-ма-экс\-по\-нен\-ци\-аль\-ная функция в~байесовских моделях массового обслуживания~// 
Информатика и~её применения, 2017. Т.~11. Вып.~4. С.~104--108.

\bibitem{Naka1960}
\Au{Nakagami~M.}
The m-distribution, a general formula of intensity of rapid fading~// 
Statistical Methods in Radio Wave Propagation 
Symposium Proceedings~/ Ed. W.\,C.~Hoffman.~--- 
New York, NY, USA: Pergamon Press, 1960. P.~3--36.
%M. Nakagami. <The m-Distribution, a general formula of intensity of rapid fading>. In William C. Hoffman, editor, Statistical %Methods in Radio Wave Propagation: Proceedings of a Symposium held June 18-20, 1958, pp 3-36. Pergamon Press, 1960.
%M. Nakagami, "The m-distribution: A general formula of intensity distribution of rapid fading", in W. C. Hoffman (ed.), %Statistical Methods in Radio Wave Propagation, Pergamon Press, New York, pp. 3-36, 1960.

\bibitem{Kruglov2016}
\Au{Круглов~В.\,М.}
Случайные процессы. Ч.~1. Основы общей теории.~--- 2-е изд., перераб. и~доп.~--- 
М.: Юрайт, 2016. 276~с.

\bibitem{Siddiqui1964}
\Au{Siddiqui~M.\,M.}
Statistical inference for Rayleigh distributions~// 
J.~Res. NBS~D Rad. Sci., 1964. 
Vol.~68D. No.~9. P.~1005--1010.
% Siddiqui, M. M. (1964) "Statistical inference for Rayleigh distributions", The Journal of Research of the National Bureau of %Standards, Sec. D: Radio Science, Vol. 68D, No. 9, p. 1005-1010.
%Publisher National Bureau of Standards
%Cite J. Res. Natl. Bur. Stand., Sec. D: Radio Sci., Vol. 68D, No. 9, p. 1005

\bibitem{Mandl2008}
\Au{Mandl~F.}
Statistical physics.~--- 2nd ed.~--- Chichester, U.K.: John Wiley \& Sons, 1988. 385~p.

\bibitem{ZaKo2013}
\Au{Закс~Л.\,М., Королев~В.\,Ю.}
Обобщенные дисперсионные гам\-ма-рас\-пре\-де\-ле\-ния как предельные для случайных сумм~// 
Информатика и~её применения, 2013. Т.~7. Вып.~1. С.~105--115.

\bibitem{KuSh2009}
\Au{Кудрявцев~А.\,А., Шоргин~С.\,Я.\/}
Байесовские модели массового обслуживания и~надежности: 
экс\-по\-нен\-ци\-аль\-но-эр\-лан\-гов\-ский случай~// Информатика и~её 
применения, 2009. Т.~3. Вып.~1. С.~44--48.

\bibitem{GR1971}
\Au{Градштейн~И.\,С., Рыжик~И.\,М.}
Таблицы интегралов, сумм, рядов и~произведений.~--- М.: Наука, 1971. 1108~с.
 \end{thebibliography}

 }
 }

\end{multicols}

\vspace*{-6pt}

\hfill{\small\textit{Поступила в~редакцию 03.02.18}}

\vspace*{6pt}

%\newpage

%\vspace*{-24pt}

\hrule

\vspace*{2pt}

\hrule

\vspace*{-2pt}


\def\tit{BAYESIAN BALANCE MODELS}

\def\titkol{Bayesian balance models}

\def\aut{A.\,A.~Kudryavtsev}

\def\autkol{A.\,A.~Kudryavtsev}

\titel{\tit}{\aut}{\autkol}{\titkol}

\vspace*{-11pt}


\noindent
Department of Mathematical Statistics, Faculty of Computational 
Mathematics and Cybernetics, M.\,V.~Lomonosov Moscow State University, 
1-52~Leninskiye Gory, GSP-1, Moscow 119991, Russian Federation


\def\leftfootline{\small{\textbf{\thepage}
\hfill INFORMATIKA I EE PRIMENENIYA~--- INFORMATICS AND
APPLICATIONS\ \ \ 2018\ \ \ volume~12\ \ \ issue\ 3}
}%
 \def\rightfootline{\small{INFORMATIKA I EE PRIMENENIYA~---
INFORMATICS AND APPLICATIONS\ \ \ 2018\ \ \ volume~12\ \ \ issue\ 3
\hfill \textbf{\thepage}}}

\vspace*{3pt}




\Abste{A number of previous author's works were devoted to the 
Bayesian approach to queuing and reliability. In this paper, 
the method application is extended to a wide range of problems, 
such as demography, physics, political science, modeling of emergencies, 
medicine, etc. The method is based on separation of system factors into 
two groups: those that support functioning of the system (positive, or p-factors) 
and those that inhibit system's functioning (negative, or n-factors). 
In the paper, system's balance index, which equals to the ratio of n- and p-factors, 
and the advantage index, which equals to the ratio of p-factor to the sum of n- 
and p-factors, are considered. It is assumed that the factors, which affect 
the system, change over time, and besides their exact values are impossible 
to determine due to the measuring equipment's imperfections, excessively 
high expenses on thorough research, lack of time and resources, and so on. 
Such prerequisites lead to usage of the Bayesian method in application 
to the problems described. The method implies randomization of the initial 
parameters (factors) and, as a~consequence, randomization of the balance 
and advantage indices. The main goal of the research is to study probabilistic 
characteristics of the balance and advantage indices assuming that the 
apriori distributions of the system's factors are known. In the case 
of independently distributed n- and p-factors, which are random variables,
 the problem is reduced to studying properties of the distributions' mixtures. 
 As opposed to popular normal mixtures, in Bayesian balance models, the distribution 
 being mixed has a positive support. Special attention is paid to apriori gamma-type 
 distributions, since these distributions are adequate asymptotic approximations 
 of a~wide range of probability distributions. The mixtures of exponential, 
 Erlang, and Weibull apriori distributions were considered earlier. 
 In this paper, special attention is paid to the case of Nakagami m-distribution 
 of n- and p-factors (with its particular cases of Rayleigh, Maxwell--Boltzmann, 
 chi-, and other distributions). The explicit formulas for density, distribution 
 functions, and moments of the balance index for different combinations
  of distributions are obtained. The results provided in this paper 
can be applied to many different tasks conserning indices, 
ratings, and indicators.}

\KWE{Bayesian method; mixed distributions; balance index; advantage index; 
balance process; Nakagami m-distribution}




\DOI{10.14357/19922264180303}

%\vspace*{-14pt}

\Ack
\noindent
The work was partly supported by the Russian Foundation for Basic Research (project 
No.\,17-07-00577).



%\vspace*{6pt}

  \begin{multicols}{2}

\renewcommand{\bibname}{\protect\rmfamily References}
%\renewcommand{\bibname}{\large\protect\rm References}

{\small\frenchspacing
 {%\baselineskip=10.8pt
 \addcontentsline{toc}{section}{References}
 \begin{thebibliography}{99}


\bibitem{1-ku}
\Aue{Yurasova, M.\,V.} 2017. 
Reyting kak instrument izmereniya uspekha: ``za'' i~``protiv'' 
[Rating as a tool for measuring success: ``Pro'' and ``contra'']. 
\textit{Moscow State University Bull. 
Ser.~18. Sociology Political Sci.} 23(2):137--164.

\bibitem{2-ku}
\Aue{Shikin, E.\,V., and A.\,G.~Chkhartishvili.} 
2004. \textit{Ma\-te\-ma\-ti\-che\-skie metody i~modeli v~upravlenii} 
[Mathematical methods and models in management]. 3rd ed.  Moscow: Delo. 440~p.

\bibitem{3-ku}
\Aue{Shaptala, V.\,G., V.\,Yu.~Radoutskiy, and V.\,V.~Shaptala.} 2010. 
\textit{Osnovy modelirovaniya chrezvychaynykh situatsiy} 
[Basics of modeling of emergency situations]. Belgorod: BGTU. 166~p.

\bibitem{4-ku}
\Aue{Bolodurina, I.\,P., and Yu.\,P.~Lugovskova.} 2009. 
Optimal'noe upravlenie dinamikoy vzaimodeystviya immunnoy sistemy cheloveka 
s~infektsionnymi zabolevaniyami [Optimum control of dynamics of interaction of 
the human immune system with infectious diseases]. \textit{Vestnik SamGU}
 74(8):138--153.

\bibitem{5-ku}
\Aue{Peskova, D.\,R.} 2006. Tenevoy sektor: ingibitor ili 
katalizator ekonomicheskogo razvitiya? [The informal sector: An 
inhibitor or catalyst for economic development?]. 
\textit{Vestnik Bashkirskogo universiteta} 
[Bull. Bashkir University. Section of the economy] 11(3):141--143.

\bibitem{6-ku}
 Mezhdunarodnyy diskussionnyy klub ``Valday''
[International 
 Discussion Club ``Valdai'']. 2017.
 \textit{Mir budushchego: cherez stolknovenie k~garmonii. 
Itogovaya plenarnaya sessiya XIV ezhegodnogo zasedaniya} 
 [The world of the future: Through a~clash to harmony: 
 The Final Plenary Session of the 14th Annual Meeting].  Sochi. Available at: 
 {\sf http://kremlin.ru/events/president/news/55882} (accessed February~13, 2018).

\bibitem{7-ku}
\Aue{Borisov, V.\,A.} 2001. \textit{Demografiya} [Demography]. 
Moscow: NOTABENE. 272~p.

\bibitem{8-ku}
\Aue{Volgin, N.\,A., L.\,L.~Rybakovskiy, N.\,M.~Kalmykova, \textit{et al.}} 2005. 
\textit{Demografiya} [Demography]. Eds. N.\,A.~Volgin and
L.\,L.~Rybakovskiy. Moscow: Logos. 280~p.

\bibitem{9-ku}
\Aue{Kuznetsov, S.\,I., and K.\,I.~Rogozin.} 2012. 
\textit{Spravochnik po fizike} [Handbook of physics]. Tomsk: TPU. 224~p.

\bibitem{10-ku}
\Aue{Bocharov, P.\,P., and A.\,V.~Pechinkin.} 1995. 
\textit{Teoriya massovogo obsluzhivaniya} [Queueing theory]. Moscow: RUDN. 529~p.

\bibitem{11-ku}
\Aue{Zdorovtsov, I.\,A., and V.\,Yu.~Korolev.} 2004. 
\textit{Osnovy teorii nadezhnosti volokonno-opticheskikh liniy peredachi 
zheleznodorozhnogo transporta} [Fundamentals of reliability theory 
of fiber optic transmission lines for railway transport]. Moscow: MAKS Press. 308~p.

\bibitem{12-ku}
\Aue{Kozlov, B.\,A., and I.\,A.~Ushakov.} 1970. 
\textit{Reliability handbook}. New York, NY: Holt, Rinehart \& Winston.  391~p.

\columnbreak

\bibitem{13-ku}
GOST 27.002-89. 1989. 
\textit{Nadezhnost' v tekhnike. Osnovnye ponyatiya. Terminy i~opredeleniya} 
[Industrial product dependability. General concepts. 
Terms and definitions]. Moscow. 24~p.

\bibitem{14-ku}
\Aue{Korolev, V.\,Yu., and I.\,A.~Sokolov}. 2006. 
\textit{Osnovy ma\-te\-ma\-ti\-che\-skoy teorii nadezhnosti modifitsiruemykh sistem} 
[Fundamentals of mathematical theory of modified systems reliability]. 
Moscow: IPI RAN. 102~p.

\bibitem{15-ku}
\Aue{Ivchenko, G.\,I., V.\,A.~Kashtanov, and I.\,N.~Kovalenko.} 1982. 
\textit{Teoriya massovogo obsluzhivaniya} [Queueing theory]. Moscow: Higher school. 256~p.

\bibitem{16-ku}
\Aue{Korolev, V.\,Yu., A.\,V.~Chertok, A.\,Yu.~Korchagin, and A.\,I.~Zeifman.} 2015. 
Modeling high-frequency order flow imbalance by functional limit theorems 
for two-sided risk processes. \textit{Appl. Math. Comput.} 253:224--241.

\bibitem{17-ku}
GOST R ISO 5725-1-2002. 2009. \textit{Tochnost' (pravil'nost' i~pretsizionnost') 
metodov i~rezul'tatov izmereniy. Chast'~1. Osnovnye polozheniya i~opredeleniya} 
[Accuracy (trueness and precision) of measurement methods and results. Part~1. 
General principles and definitions]. Moscow: Standardinform Publs. 24~p.

\bibitem{18-ku}
\Aue{Congdon, P.} 2006. \textit{Bayesian statistical modelling}. 2nd ed. 
 Chichester, U.K.: John Wiley \& Sons. 596~p.

\bibitem{19-ku}
\Aue{Hamada, M.\,S., A.~Wilson, C.\,S.~Reese, and H.~Martz.} 2008. 
\textit{Bayesian reliability}. New York, NY: Springer. 436~p.

\bibitem{20-ku}
\Aue{Carlin, B.\,P., and T.\,A.~Louis.} 2008. \textit{Bayesian methods for data analysis}. 
3rd ed. New York,  NY: Chapman \& Hall. 552~p.

\bibitem{21-ku}
\Aue{Albert, J.} 2009. \textit{Bayesian computation with~R.}
New York,  NY: Springer. 300~p.

\bibitem{22-ku}
\Aue{Laplace, P.-S.} 1902. 
\textit{A~philosophical essay on probabilities}. 
Transl. from the French by F.\,W.~Truscott and F.\,L.~Emory. 
New York, NY: John Wiley \& Sons. 223~p.

\bibitem{23-ku}
\Aue{Bayes, T., and R.~Price.} 1763. 
An essay towards solving a~problem in the doctrine of chances. 
\textit{Phil. Trans.} 53:370--418.

\bibitem{24-ku}
\Aue{Shorgin, S.\,Ya.} 2005. 
O~bayesovskikh modelyakh massovogo obsluzhivaniya [On Bayesian queuing models]. 
\textit{II~Nauchnaya Sessiya instituta problem informatiki RAN: Tezisy dokladov} 
[2nd Scientific Session of the Institute of Informatics Problems of the 
Russian Academy of Sciences: Abstracts]. Moscow: IPI RAN. 120--121.

\bibitem{25-ku}
\Aue{Bening, V.\,E., V.\,Yu. Korolev, I.\,A.~Sokolov, and S.\,Ya.~Shorgin.} 2007. 
\textit{Randomizatsionnye modeli i~metody teorii nadezhnosti informatsionnykh 
i~tekhnicheskikh sistem} [Randomization models and methods of reliability theory 
for information and technical systems]. Moscow: TORUS PRESS. 256~p.
\bibitem{26-ku}
\Aue{Kudryavtsev, A.\,A., and S.\,Ya.~Shorgin.} 2015. 
\textit{Bayesovskie modeli v~teorii massovogo obsluzhivaniya i~nadezhnosti} 
[Bayesian models in mass service and reliability theories]. Moscow: FIC IU RAN. 76 p.

\bibitem{27-ku}
\Aue{Orlik, S.} 2004--2005.
\textit{Vvedenie v~programmnuyu inzheneriyu i~upravlenie zhiznennym tsiklom PO. 
Programmnaya inzheneriya. Soprovozhdenie programmnogo obespecheniya} 
[Introduction to software engineering and software lifecycle management. 
Software engineering. Maintenance of software]. 
Available at: 
{\sf http://www.\linebreak software-testing.ru/files/se/3-5-software\_engineering\_\linebreak maintenance.pdf}
 (accessed February~13, 2018).





\bibitem{30-ku} %28
\Aue{Boni, M.\,F.} 2008. Vaccination and antigenic drift in influenza. 
\textit{Vaccine} 26(Suppl.~3):8--14.

\bibitem{29-ku} %29
\Aue{Kim, J.\,H., I.~Skountzou, R.~Compans, and J.~Jacob.} 2009. 
Original antigenic sin responses to influenza viruses. \textit{J.~Immunology} 183(5):3294--3301.

\bibitem{28-ku} %30
\Aue{Amantonio}.  09.11.2017. Razbiraemsya s~privivkami. Chast'~20. Gripp~(1) 
[We deal with vaccinations. Part~20. Influenza~(1)]. \textit{Livejournal}. 
Available at: {\sf https://\linebreak amantonio.livejournal.com/29621.html} 
(accessed February~13, 2018). 10.11.2017.
Prodolzhenie: Chast'~21. Gripp~(2) [Continuation: Part~21. Influenza~(2)]. 
\textit{Livejournal}.  Available at: 
{\sf https://amantonio.livejournal. com/29886.html} (accessed February~13, 2018).

\bibitem{31-ku}
\Aue{Korolev, V.\,Yu.} 2011. \textit{Veroyatnostno-statisticheskie metody dekompozitsii 
volatil'nosti khaoticheskikh protsessov} [Probabilistic and statistical methods 
of decomposition of volatility of chaotic processes]. Moscow: MSU Publs. 510~p.

\bibitem{32-ku}
\Aue{Kudryavtsev, A.\,A., V.\,S.~Shorgin, and S.\,Ya.~Shorgin.} 2009. 
Bayesovskie modeli massovogo obsluzhivaniya i~nadezhnosti: obshchiy erlangovskiy 
sluchay [Bayesian queueing and reliability models: General Erlang case]. 
\textit{Informatika i~ee Primeneniya~--- Inform. Appl.} 3(4):30--34.

\bibitem{33-ku}
\Aue{Zhavoronkova, Iu.\,V., A.\,A.~Kudryavtsev, and S.\,Ya.~Shor\-gin.} 2014. 
Bayesovskaya rekurrentnaya mo\-del' rosta nadezhnosti: beta-raspredelenie parametrov 
[Bayesian recurrent model of reliability growth: Beta-distribution of parameters]. 
\textit{Informatika i~ee Primeneniya~--- Inform. Appl.} 8(2):48--54.

\bibitem{34-ku}
\Aue{Kudryavtsev, A.\,A., and A.\,I.~Titova.} 2016. Bayesovskie modeli 
massovogo obsluzhivaniya i~nadezhnosti: vyrozhdenno-veybullovskiy sluchay  
[Bayesian queuing and reliability models: Degenerate-Weibull case]. 
\textit{Informatika i~ee Primeneniya~--- Inform. Appl.} 10(4):68--71.

\bibitem{35-ku}
\Aue{Kudryavtsev, A.\,A., and A.\,I.~Titova.}
 2017. Gamma-eksponentsial'naya funktsiya v~bayesovskikh modelyakh 
 massovogo obsluzhivaniya [Gamma-exponential function in Bayesian queuing models]. 
 \textit{Informatika i~ee Primeneniya~--- Inform. Appl.} 11(4):104--108.

\bibitem{36-ku}
\Aue{Nakagami, M.} 1960. The m-distribution, a~general formula of intensity of 
rapid fading. \textit{Statistical Methods in Radio Wave Propagation 
Symposium Proceedings}. Ed. W.\,C.~Hoffman. 
New York, NY: Pergamon Press. 3--36.

\bibitem{37-ku}
\Aue{Kruglov, V.\,M.} 2016. 
\textit{Sluchaynye protsessy. Ch.~1. Osnovy obshchey teorii} 
[Stochastic processes. Part~1. Bases of general theory].   
2nd ed.  Moscow: Yurayt. 276 p.

\bibitem{38-ku}
\Aue{Siddiqui, M.\,M.} 1964. 
Statistical inference for Rayleigh distributions. 
\textit{J.~Res. NBS~D Rad. Sci.}
68D(9):1005--1010.

\bibitem{39-ku}
\Aue{Mandl, F.} 1988. 
\textit{Statistical physics.} 2nd ed.  Chichester, U.K.: John Wiley \& Sons. 385~p.

\bibitem{40-ku}
\Aue{Zaks, L.\,M., and V.\,Yu.~Korolev.} 2013. 
Obobshchennye dispersionnye gamma-raspredeleniya kak predel'nye 
dlya sluchaynykh sum [Generalized dispersion gamma distributions 
as limiting for random sums]. 
\textit{Informatika i~ee Primeneniya~--- Inform. Appl.} 7(1):105--115.

\bibitem{41-ku}
\Aue{Kudryavtsev, A.\,A., and S.\,Ya.~Shorgin.} 2009. 
Bayesovskie modeli massovogo obsluzhivaniya i~nadezhnosti: 
eksponentsial'no-erlangovskiy sluchay [Bayesian queuing and reliability models:
An exponential-Erlang case]. 
\textit{Informatika i~ee Primeneniya~--- Inform. Appl.} 3(1):44--48.

\bibitem{42-ku}
\Aue{Gradshteyn, I.\,S., and I.\,M.~Ryzhik.} 1971. 
\textit{Tablitsy integralov, summ, ryadov i~proizvedeniy} 
[Tables of integrals, sums, series, and products]. Moscow: Nauka. 1108~p.
 \end{thebibliography}

 }
 }

\end{multicols}

\vspace*{-6pt}

\hfill{\small\textit{Received February 3, 2018}}

%\pagebreak

%\vspace*{-18pt}

\Contrl

\noindent
\textbf{Kudryavtsev Alexey A.} (b.\ 1978)~--- 
Candidate of Science (PhD) in physics and mathematics, associate professor, 
Department of Mathematical Statistics, Faculty of Computational Mathematics 
and Cybernetics, M.\,V.~Lomonosov Moscow State University, 1-52~Leninskiye Gory, 
GSP-1, Moscow 119991, Russian Federation; \mbox{nubigena@mail.ru}
\label{end\stat}

\renewcommand{\bibname}{\protect\rm Литература}       