
\def\stat{kozerenko}

\def\tit{СЕМАНТИЧЕСКАЯ ОБРАБОТКА НЕСТРУКТУРИРОВАННЫХ ТЕКСТОВЫХ 
ДАННЫХ НА~ОСНОВЕ ЛИНГВИСТИЧЕСКОГО ПРОЦЕССОРА PullEnti}

\def\titkol{Семантическая обработка неструктурированных текстовых 
данных на~основе лингвистического процессора PullEnti}

\def\aut{Е.\,Б.~Козеренко$^1$, К.\,И.~Кузнецов$^2$, Д.\,А.~Романов$^3$}

\def\autkol{Е.\,Б.~Козеренко, К.\,И.~Кузнецов, Д.\,А.~Романов}

\titel{\tit}{\aut}{\autkol}{\titkol}

\index{Козеренко Е.\,Б.}
\index{Кузнецов К.\,И.}
\index{Романов Д.\,А.}
\index{Kozerenko E.\,B.}
\index{Kuznetsov K.\,I.}
\index{Romanov D.\,A.}




%{\renewcommand{\thefootnote}{\fnsymbol{footnote}} \footnotetext[1]
%{Работа выполнена при частичной поддержке РФФИ (проект 16-07-00677).}}


\renewcommand{\thefootnote}{\arabic{footnote}}
\footnotetext[1]{Институт проблем информатики Федерального исследовательского центра <<Информатика и~управ\-ле\-
ние>> Российской академии наук, \mbox{kozerenko@mail.ru}}
\footnotetext[2]{Институт проблем информатики Федерального исследовательского центра <<Информатика 
и~управление>> Российской академии наук, \mbox{k.smith@mail.ru}}
\footnotetext[3]{Национальный исследовательский университет <<Высшая школа экономики>>, DRomanov@it.ru}

%\vspace*{8pt}
  

  
   
     \Abst{Представлена методика создания систем извлечения 
знаний, основанная на подходе, главным инструментом которого является 
программный пакет PullEnti, включающий алгоритмы морфологического 
и~се\-ман\-ти\-ко-син\-так\-си\-че\-ско\-го анализа для выделения сущностей 
определенных типов из текстов естественного языка 
(персоны, организации, локации и~другие целевые семантические объекты). 
В~сис\-те\-ме PullEnti используются динамически подключаемые компоненты 
(плагины), что позволяет без перекомпилирования активировать различные 
функциональные воз\-мож\-ности. Именно таким образом запускается блок 
семантического анализа. В~процессе анализа выделяются семантические 
единицы (токены), которые представляют собой типизированные фразы: 
текстовые, чис\-ло\-вые и~др. Приводятся примеры реализованных проектов 
для различных предметных областей.}
     
     \KW{семантическое моделирование; извлечение именованных 
сущностей; области с~интенсивным использованием данных; 
автоматизированные сис\-те\-мы извлечения знаний; семантический поиск; 
интеллектуальные ин\-тер\-нет-тех\-но\-логии}

\DOI{10.14357/19922264180313}
  
\vspace*{-1pt}


\vskip 10pt plus 9pt minus 6pt

\thispagestyle{headings}

\begin{multicols}{2}

\label{st\stat}
     
     \section{Введение}
     
     \vspace*{-4pt}
     
     Задача автоматического анализа текстовой информации, 
представленной в~Интернете, является актуальной во всем мире. В~данной 
статье пред\-ставле\-ны результаты исследований и~разработок, на\-прав\-лен\-ных 
на решение научной проб\-ле\-мы со\-зда\-ния оптимальной методики  
ло\-ги\-ко-ста\-ти\-сти\-че\-ско\-го моделирования механизмов целевого 
семантического анализа в~информационных сис\-те\-мах с~интенсивным 
использованием знаний, выполняющих функции извлечения знаний, 
поддержки аналитических решений, в~том чис\-ле в~среде нескольких 
естественных языков. 

Для решения полного спектра задач обработки 
естественного языка создан се\-ман\-ти\-че\-ски-ори\-ен\-ти\-ро\-ван\-ный 
лингвистический процессор (СОЛП). Центральным компонентом СОЛП 
является инструментальный пакет (SDK-мо\-дуль) PullEnti. Этот процессор 
в~рамках проводимых соревнований конференции <<Диа\-лог-2016>> занял 
несколько первых мест при анализе текс\-тов в~рамках решения задач 
извлечения именованных сущностей. Разработчик PullEnti~--- Кузнецов 
Константин Игоревич. В~сис\-те\-ме PullEnti используются динамически 
подключаемые компоненты (плагины), что позволяет без 
перекомпилирования запускать различные функциональные возможности. 
Именно таким образом активируется блок семантического анализа. 
     
     В процессе анализа выделяются семантические единицы (токены), 
которые представляют собой типизированные фразы, такие как текс\-то\-вые, 
чис\-ло\-вые и~др. Например, в~результате анализа фразы <<В~2017~году>> 
будут выделены три токена: <<В>>~--- текс\-то\-вый; <<году>>~--- текстовый; 
<<2017>>~--- чис\-ло\-вой. Такие токены можно назвать прос\-ты\-ми. Кроме того, 
выделяются \textit{метатокены}~--- слож\-ные токены, которые объединяют 
несколько прос\-тых токенов, например существительные с~определителями, 
скобки, кавычки и~т.\,п.
     
     В системе существует пополняемый статический словарь терминов. 
В~него можно добавлять термины и~затем проверять их наличие в~тексте. 
Кроме того, в~сис\-те\-ме можно формировать динамически подобные словари 
на основе анализа текста.
     
     При анализе текста создается аналитический контейнер, в~который 
помещаются вы\-де\-ля\-емые сущности, токены в~определенной 
последовательности, статистические данные и~др.
     
    \section{Лингвистическое моделирование в~системах 
обработки знаний в~многоязычной среде}
     
     Способы представления информации, знаний многообразны. Огромный 
объем данных пред\-став\-лен в~виде текс\-тов естественного языка, что делает 
задачу извлечения и~структурирования информации из текстов весьма 
важной. Это относится к~различным предметным областям. Для 
оперирования данными на компьютере необходимо выделить из текста 
объекты, их атрибуты, связи между объектами, процессы, в~которых эти 
объекты задействованы, другую важ\-ную информацию, которая бы позволяла 
не только описать ситуацию, но и~строить выводы, характерные для 
конкретной предметной об\-ласти, прогнозировать развитие ситуации.
     
     Для решения поставленных задач проведены эксперименты 
с~различными грамматическими формализмами, в~том чис\-ле с~грамматикой 
категориального типа~[1]. Проведено сравнительное\linebreak
 исследование методов 
классификации применительно к~лингвистическим задачам; выработан\linebreak 
эффективный метод отображения вектора ес\-те\-ст\-вен\-но-язы\-ко\-вых 
структур в~расширенное пространство признаков для классификации новых 
языковых объектов и~структур; сформирована фокусная выборка 
параллельных текстов деловых и~научных документов на русском 
и~английском языках по различным отраслям науки и~техники;\linebreak 
сформирована расширенная система новых категорий для повышения 
изобразительных возможностей двуязычной грамматики; выработаны пути\linebreak 
расширения базовых представлений на основе аппарата расширенных 
семантических сетей~[2]\linebreak и~результатов применения метода векторных 
пространств, направленного на разрешение не\-од\-но\-знач\-ности языковых 
структур для синтаксического разбора при распознавании текста в~процессе 
извлечения знаний из текстов на разных естественных языках. Разработаны 
алгоритмы автоматического выравнивания параллельных текстов для 
развития грамматических компонент сис\-тем обработки знаний 
в~многоязычном режиме. 

%o
Основной результат исследований~--- модель 
лингвистической со\-став\-ля\-ющей интеллектуальных информационных сис\-тем, 
работающих в~многоязычном пространстве для поиска информации, 
обеспечения оптимальных аналитических и~управ\-лен\-че\-ских решений 
в~сферах деятельности с~интенсивным использованием данных. Результаты 
исследований применяются в~ло\-ги\-ко-се\-ман\-ти\-че\-ских 
и~статистических процедурах обработки слабоструктурированной текс\-то\-вой 
информации, при разработке технологии и~инструментальных средств 
построения лингвистических компонент интеллектуальных сис\-тем и~сис\-тем 
машинного перевода.
     
    \section{Представление лингвистических знаний на~основе 
векторных пространств }
    
     Процедуры анализа и~синтеза ес\-те\-ст\-вен\-но-язы\-ко\-вых высказываний 
отражают динамический характер языка как деятельности; соответственно, 
в~модели, которая кладется в~основу проекта сис\-те\-мы обработки  
ес\-те\-ст\-вен\-но-язы\-ко\-вых высказываний, дол\-жен быть заложен 
механизм, позволяющий строить пред\-став\-ле\-ния движения. 
     
     Методы машинного обучения на основе векторных моделей 
развиваются и~используются в~различных областях знаний, применительно 
к~лингвистическим задачам эти методы вполне эффективны для разрешения 
лексической мно\-го\-знач\-ности~[3--8]. 
     
     Более сложной задачей и~новым направлением исследований 
возможности применения векторных моделей для пред\-став\-ле\-ния и~обработки 
лингвистических данных является моделирование грамматических 
преобразований на основе векторных пространств и~тензоров. Тензор (от 
лат.\ \textit{tensus}, напряженный)~--- объект линейной алгебры, пре\-об\-ра\-зу\-ющий 
элементы одного линейного пространства в~элементы другого. Часто тензор 
представляют как многомерную таб\-ли\-цу, заполненную чис\-ла\-ми~--- 
компонентами тензора $d \cdot d \cdots d$, где $d$~--- раз\-мер\-ность, 
над которой задан тензор, а~чис\-ло сомножителей совпадает с~так называемой 
валентностью, или рангом тензора. Важно, что такое представление (кроме 
скаляров, т.\,е.\ тензоров валентности ноль) возможно только после выбора 
базиса (или системы координат): при смене базиса компоненты тензора 
меняются определенным образом. Сам тензор как <<геометрическая 
сущность>> от выбора базиса не зависит, компоненты вектора меняются при 
смене координатных осей, но сам вектор, образом которого может быть 
прос\-то нарисованная стрелка, от этого не изменяется. Тензор обычно 
обозначают некоторой буквой с~совокупностью верх\-них (контрвариантных) и~ниж\-них (ковариантных) индексов: $X_{j_1 j_2\ldots j_s}^{i_1i_2\ldots i_r}$. 
При смене базиса ковариантные компоненты меняются так же, как и~базис 
(с~по\-мощью того же преобразования), а~контрвариантные~--- обратно 
изменению базиса (обратным преобразованием). Тензор является сущностью 
любой системы реального мира и~сохраняется, несмотря на происходящие 
изменения в~этой системе~\cite{9-koz}. Эта особенность тензора чрезвычайно 
актуальна для моделирования языковых преобразований в~лингвистических 
процессорах, когда необходимо выявлять сходные значения, выраженные 
многочисленными способами, сис\-те\-мой разнородных языковых средств. 
     
     В работе, представленной в~данной статье, используются два основных 
подхода к~пред\-став\-ле\-нию смысла в~вы\-чис\-ли\-тель\-ной лингвистике: 
символьный подход~\cite{10-koz, 11-koz} и~подход на основе 
дистрибутивной семантики~\cite{5-koz, 7-koz, 8-koz, 9-koz}; решение 
заключается\linebreak в~сочетании методов компьютерной лингвистики и~когнитивной 
науки, в~котором символьное и~<<\textit{коннекционистское}>> (от англ.\ 
\textit{connectionist}, т.\,е.\ основанное на нейронных сетях как модели 
машинного обучения) представления объединяются с~по\-мощью тензорных 
произведений. Исследованы возможные применения данного метода для 
обработки синтаксических структур и~контекстов в~русском и~английском 
языках и~проведены межъязыковые сопоставления.
\vspace*{-3pt}
     
\section{Семантико-ориентированный лингвистический процессор}
     
     Методы, описанные выше, используются в~процедурах семантической 
обработки текстовых знаний в СОЛП, который решает задачу извлечения 
структурированной информации из текс\-тов на русском и~английском языках. 
Ядром СОЛП является программный пакет PullEnti, вклю\-ча\-ющий алгоритмы 
морфологического и~синтаксического анализа, который позволяет выделять 
сущности определенных типов из текс\-тов естественного языка (персоны, 
организации, локации и~другие семантические объекты). <<Именованная 
сущность>>~--- это объект, содержащий набор значений атрибутов, 
отличающий его от других объектов этого же типа. В~тексте находятся 
именованные сущности и~устанавливаются семантические связи между 
ними, при этом учитывается воз\-мож\-ность обозначения одной сущ\-ности 
несколькими способами (синонимия). Все множество сущностей, 
выделенных из текста или нескольких текс\-тов, представляет собой 
ориентированный граф.
     
     Предварительный этап обработки текстов включает в~себя 
морфологический и~синтаксический анализ. При морфологическом анализе 
текст разбивается на словоформы, так\-же называемые токенами (от англ.\ 
\textit{token}~--- пример использования лингвистической единицы в~тексте). 
Основными наследными классами базового класса Token являются TextToken и~MetaToken. 
TextToken~--- это исходный фрагмент текста, содержащий результат 
морфологического анализа. TextToken ссылается\linebreak на MorphToken, 
содержащий все морфологические варианты разбора. MetaToken~--- это 
группа токенов, соответствующих одной синтаксической или семантической 
группе. К~классу метатокенов относятся NumberToken, пред\-став\-ля\-ющий 
чис\-ло, и~\mbox{ReferentToken}, представляющий сущ\-ность.
     
     Приведем пример морфологического анализа предложения.
     
     Исходный текст: 
     
     <<По словам директора департамента экономического развития 
автономного округа Павла Сидорова, на эти цели планируется при\-влечь 
200~млн рублей из федерального бюджета и~еще 450~млн 
рублей из внебюджетных источников>>.
     
     В результате морфологического разбора текст был разбит на 
словоформы, для каждой словоформы указана начальная форма, часть речи 
и~морфологические характеристики. В~случае мно\-го\-знач\-ности или 
омонимии указываются все варианты морфологического разбора. 
     
     Также были выделены сле\-ду\-ющие метатокены:
     
      Павел Сидоров~--- текстовый фрагмент <<директора департамента 
экономического развития автономного округа Павла Сидорова>>
      
      200.000.000 RUB~--- текстовый фрагмент <<200~миллионов рублей>>
      
      450.000.000 RUB~--- текстовый фрагмент <<450~миллионов рублей>>
     
     Вслед за морфологическим анализом проводится выделение 
именованных сущностей различных типов. Се\-ман\-ти\-ко-ори\-ен\-ти\-ро\-ван\-ный 
лингвистический процессор извлекает из текстов 
объекты следующих типов: дата, временной период, территориальное 
образование, денежная сумма, телефон, URL, адрес, организация, транспорт, 
свойство персоны, персона, декрет, часть декрета. Каждому типу 
соответствуют свои свойства и~связи с~объектами других типов. 

\begin{figure*}[b] %fig1
  \vspace*{6pt}
 \begin{center}
 \mbox{%
 \epsfxsize=163mm 
 \epsfbox{koz-1.eps}
 }
 \end{center}
\vspace*{-9pt}
\Caption{Результаты работы программы <<Доктор Ватсон>>. Выделение сущностей}
\end{figure*}
     
     Базовым классом для сущностей является класс Referent. Тип 
сущностей задается классом Referent Class, на\-след\-ным от Referent, который 
содержит набор атрибутов. Значение может быть как прос\-тым (строка, 
число), так и~ссылкой на другую сущ\-ность. Помимо значений атрибутов 
сущность содержит список ссылок на участки исходного текс\-та, в~которых 
эта сущность располагается. Для задач, в~которых требуется обрабатывать 
множество текстов и~хранить по\-лу\-ча\-емые сущности, в~СОЛП используется 
базовый класс Repository Base, об\-лег\-ча\-ющий реализацию хранилища 
сущностей. Мес\-то хранения сериализованных данных от сущностей 
определяется в~наследном классе (например, это может быть реляционная 
система управления базами данных или файловая сис\-те\-ма). 
Repository Base берет на себя функции 
отож\-де\-ст\-вле\-ния новых данных со старыми данными и~поддержки 
не\-про\-ти\-во\-ре\-чи\-вости семантической сети.
     
Извлекаемая из текс\-та информация должна быть адресной, поэтому из 
одного и~того же текста можно извлекать совершенно различные виды 
информации, характерные для конкретной предметной об\-ласти. 
В~результате анализа текстовой информации выделяются типизированные 
объекты предметной об\-ласти. Программа PullEnti стала основой для 
построения множества сис\-тем, таких как программа <<Доктор Ватсон>>, 
система поиска экспертов, процессор BRef и~др. Программа <<Доктор 
Ватсон>> предназначена для исследования массивов текс\-то\-вой информации с~целью выявления сущностей и~связей между ними. При этом пользователь 
может добавить недостающие сущности и~связи (которые не были выделены 
программой), настроить выдаваемую информацию, сформировать отчет 
о~результатах работы программы. Данная программа может использоваться в~таких сферах деятельности, как криминалистика, конкурентная разведка, 
маркетинг, реклама, безопас\-ность. Результат работы программы~--- отчет об 
исследуемом объекте, диаграммы сущностей и~связей~--- пред\-став\-лен на 
рис.~1. Из текущего текста выделены организации, персоны, их связи.
   
   На рис.~2 представлены выделенные объекты, их связи. Для каждой связи 
выделяется тип связи и~название (например, тип связи~--- <<родственные>>, 
заголовок связи~--- <<отец>>; тип связи~--- <<владение>>, заголовок  
связи~--- <<особняк в~центре Вашингтона>> и~т.\,д.), определяются попарно 
объекты-участники связи. Для более полного определения ситуации 
выделяется не только время, характеризующее текущую ситуацию, но 
и~интервалы времени. Дополнительные параметры поз\-во\-ля\-ют выяснить, 
является ли связь симметричной для данной пары выделенных объектов 
(субъектов). Также для каж\-до\-го выделенного объекта (субъекта) выделяются 
атрибуты. Например, для типа объекта <<персона>> выделяются имя, 
фамилия, отчество, дата рождения и~др. 


   Результаты работы программы могут быть пред\-став\-ле\-ны в~виде графа 
(вкладка <<Диаграммы>>) (см.\ рис.~3). В~отчете выводятся обнаруженные 
объекты (персоны, организации, локации, атрибуты), их связи в~удоб\-ном для 
анализа виде.


   
   Программа <<Логика ECM. Правовая экспертиза>> предназначена для 
автоматизации процесса\linebreak проведения экспертизы проектов  
нор\-ма\-тив\-но-пра\-во\-вых актов,  
ор\-га\-ни\-за\-ци\-он\-но-рас\-по\-ря\-ди\-тель\-ных документов, договоров 
и~других документов. Сис\-те\-ма значительно упрощает процесс проведения 
правовой экспертизы и~сокращает его сроки, выполняя рутинные операции 
и~кардинально снижая за\-тра\-ты рабочего времени квалифицированных 
юристов. Сис\-те\-ма <<Логика ECM. Правовая экспер-\linebreak\vspace*{-12pt}

\pagebreak

\end{multicols}

\begin{figure*} %fig2
\vspace*{1pt}
 \begin{center}
 \mbox{%
 \epsfxsize=163mm 
 \epsfbox{koz-2.eps}
 }
 \end{center}
\vspace*{-11pt}
\Caption{Выделение связей, периодов}
%\end{figure*}
%\begin{figure*} %fig3
\vspace*{6pt}
 \begin{center}
 \mbox{%
 \epsfxsize=163mm 
 \epsfbox{koz-3.eps}
 }
 \end{center}
\vspace*{-11pt}
\Caption{Графическое представление результатов работы программы <<Доктор 
Ватсон>>}
\vspace*{-2pt}
\end{figure*}
   

\begin{multicols}{2}

\noindent
тиза>> автоматически за 
несколько секунд поможет, например, установить:
   \begin{itemize}
   \item не содержатся ли в~проверяемом документе ссылки на нормативные 
правовые акты, которые утратили силу;
   \item нет ли в~проверяемом документе фрагментов других документов, не 
возникает ли избыточное дублирование нормативной документации;
   \item соответствует ли оформление и~структура документа уста\-нов\-лен\-ным 
   в~организации правилам;\\[-13pt]
   \item нет ли ошибок в~оформлении цифровой информации в~договоре, 
соответствуют ли друг другу суммы, указанные цифрами и~про\-писью, 
правильно ли рассчитан налог на добавленную стоимость и~т.\,п. На основе лингвистического процессора 
PullEnti был реализован процессор обработки ссылок и~списка литературы 
BREF,\linebreak
\vspace*{-12pt}

\pagebreak

\noindent
который позволяет по выделенной информации по\-строить \textit{Граф 
Цитирования} и~\textit{Граф Соавторов}, отражающие формальные связи 
в~коллекции документов.
   \end{itemize}
    
   Лингвистический процессор PullEnti под псевдонимом Pink на 
соревновании FactRuEval конференции <<Диа\-лог-2016>> занял первые места 
на большинстве дорожек~\cite{12-koz}.
   
   Соревнование проводилось на следующих дорожках:
   \begin{itemize}
   \item определение в~тексте границ именованных сущностей, таких как 
персона, организация, локация;
   \item выделение именованных сущностей с~определением атрибутов 
в~нормализованном виде. Для персон это фамилия, имя и~отчество. Для 
организаций и~локаций~--- нормализованное название;
   \item извлечение фактов (например: <<встреча>>, <<покупка>>, $\ldots$) 
и~наборов строковых полей (например: <<участник встречи~1>>, 
<<участник встречи~2>>, <<место встречи>>, <<да\-та/вре\-мя начала 
встречи>>, $\ldots$).
   \end{itemize}
   
   \vspace*{-13pt}
   
  \section{Заключение}
  \vspace*{-3pt}
  
   Лингвистические процессоры на основе программы PullEnti могут быть 
использованы в~различных областях, в~которых информация пред\-став\-ле\-на 
в~текс\-то\-вом виде. Особенно это важно в~тех случаях, когда необходимо 
выделять важ\-ную информацию из большого потока документов на 
естественном языке. Очень хорошо данная технология работает в~задачах 
кластеризации текстов по определенным признакам. При этом существует 
воз\-мож\-ность автоматической настройки программы на требования 
пользователя.
   
   Описанные выше системы, созданные на основе технологии PullEnti, 
доказывают ее эффективность в~самых различных областях. Сле\-ду\-ющи\-ми 
шагами исследований станут: методы уточнения границ мо\-де\-ли\-ру\-емых 
предметных областей за счет построения семантического ядра каж\-дой 
области (в~том чис\-ле с~использованием методов вероятностного 
тематического моделирования); выделение массива неявных ссылок 
(упоминаний персон и~идей, выраженных ключевыми фра\-за\-ми/зна\-чи\-мы\-ми 
словосочетаниями); расчет корреляции между явными и~неявными ссылками 
в~рамках созданной коллекции, формирование  
функ\-цио\-на\-ль\-но-грам\-ма\-ти\-че\-ских модулей естественных языков, 
включаемых в~лингвистический процессор.

    
{\small\frenchspacing
 {%\baselineskip=10.8pt
 \addcontentsline{toc}{section}{References}
 \begin{thebibliography}{99}

 \vspace*{-4pt}

\bibitem{1-koz}
\Au{Shaumyan S.} Categorial grammar and semiotic universal grammar~// 
Conference (International) on Artificial Intelligence Proceedings.~--- 
Las Vegas, NV, USA: CSREA Press, 2003. 
P.~623--629.

\bibitem{2-koz}
\Au{Kuznetsov I.\,P., Kozerenko~E.\,B., Matskevich~A.\,G.} 
Intelligent extraction of knowledge structures from natural language texts~// 
IEEE/WIC/ACM Joint Conferences (International) on Web Intelligence and 
Intelligent Agent Technology Proceedings.~---
Washington, DC, USA: IEEE Computer Society, 2011. Vol.~3. P.~269--272.

\bibitem{3-koz}
\Au{Dempster A.\,P., Laird N.\,M., Rubin~D.\,B.} 
Maximum likelihood from incomplete data via the EM algorithm~// 
J.~Roy. Stat. Soc.~B, 1977. Vol.~39. Iss.~1. P.~1--22.

\bibitem{7-koz} %4
\Au{Lund K., Burgess~C.} Producing high-dimensional semantic spaces from 
lexical co-occurrence~// 
Behav. Res. Meth. Ins. C., 1996. Vol.~28. Iss.~2. 
P.~203--208.

\bibitem{5-koz} %5
\Au{Curran J.\,R.} From distributional to semantic similarity.~--- 
Edinburgh: University of 
Edinburgh, 2004. PhD Thesis. 177~p.
{\sf https://www.inf.ed.ac.uk/publications/thesis/ online/IP030023.pdf}

\bibitem{8-koz} %6
\Au{McCarthy D., Koeling R., Weeds~J., Carroll~J.} Finding predominant 
senses in untagged text~// 42nd Annual Meeting of the Association for 
Computational Linguistics Proceedings.~---
Stroudsburg, PA, USA: Association for 
Computational Linguistics, 2004. P.~280--287. doi: 10.3115/1218955.1218991.

\bibitem{4-koz} %7
\Au{Clark S., Pulman S.} Combining symbolic and distributional models of 
meaning~// AAAI Spring Symposium on Quantum Interaction Proceedings.~--- 
Palo Alto, CA, USA: AAAI Press, 2007. 4~p. {\sf 
http://www.cl.cam.ac.uk/ $\sim$sc609/pubs/aaai07.pdf.}

\bibitem{6-koz} %8
\Au{Kozerenko E.\,B.} Parallel texts alignment strategies~// 
\textit{Conference (International) on Artificial Intelligence Proceedings}.~--- 
Las Vegas, NV, USA: CSREA Press, 
2012. Vol.~2. P.~945--951.

\bibitem{9-koz}
\Au{Danielson D.\,A.} Vectors and tensors in engineering and physics.~--- 
2nd ed.~--- Boulder, CO, USA: Westview (Perseus), 2003. 287~p.

\bibitem{11-koz} %10
\Au{Montague R.} Universal grammar~// Theoria, 1970. Vol.~36. P.~373--398. 
(Reprinted in: Formal philosophy: Selected 
papers of Richard Montague~/ 
Ed. R.\,H.~Thomason.~--- 
New Haven, CT, USA: Yale University Press, 1974. P.~7--27.)

\bibitem{10-koz}
\Au{Pang B., Knight K., Marcu~D.} Syntax-based alignment of multiple translations: 
Extracting paraphrases and generating new sentences~// 
Conference of the North 
American Chapter of the Association for Computational Linguistics on Human Language 
Technology Proceedings.~--- 
Stroudsburg, PA, USA: Association for Computational Linguistics.
2003. Vol.~1. P.~102--109. doi: 10.3115/1073445.1073469.

\bibitem{12-koz}
FACRUEVAL. Evaluation of named entity recognition and fact extraction systems for 
Russian, 2016. {\sf http:// ww.dialog-21.ru/media/3430/starostinaetal.pdf.}
 \end{thebibliography}

 }
 }

\end{multicols}

%\vspace*{-12pt}

\hfill{\small\textit{Поступила в~редакцию 13.07.18}}

%\vspace*{-36pt}

\pagebreak

\vspace*{-36pt}

%\hrule

%\vspace*{2pt}

%\hrule

\vspace*{-2pt}


\def\tit{SEMANTIC PROCESSING OF~UNSTRUCTURED TEXTUAL DATA BASED 
ON~THE~LINGUISTIC PROCESSOR PullEnti}


\def\titkol{Semantic processing of~unstructured textual data based 
on~the~linguistic processor PullEnti}


\def\aut{E.\,B.~Kozerenko$^1$, K.\,I.~Kuznetsov$^1$, and~D.\,A.~Romanov$^2$}

\def\autkol{E.\,B.~Kozerenko, K.\,I.~Kuznetsov, and~D.\,A.~Romanov}

\titel{\tit}{\aut}{\autkol}{\titkol}

\vspace*{-11pt}


\noindent
$^1$Institute of Informatics Problems, Federal Research Center ``Computer Science 
and Control'' of the Russian 
$\hphantom{^1}$Academy of Sciences,  44-2~Vavilov Str., Moscow 119333, 
Russian Federation

\noindent
$^2$National Research University ``Higher School of Economics,'' 
20~Myasnitskaya Str., Moscow 101000, Russian 
$\hphantom{^1}$Federation


\def\leftfootline{\small{\textbf{\thepage}
\hfill INFORMATIKA I EE PRIMENENIYA~--- INFORMATICS AND
APPLICATIONS\ \ \ 2018\ \ \ volume~12\ \ \ issue\ 3}
}%
 \def\rightfootline{\small{INFORMATIKA I EE PRIMENENIYA~---
INFORMATICS AND APPLICATIONS\ \ \ 2018\ \ \ volume~12\ \ \ issue\ 3
\hfill \textbf{\thepage}}}

\vspace*{3pt}



\Abste{The paper presents the method for creation of knowledge extraction 
systems based on the approach employing the software tool system 
PullEnti comprising the algorithms for morphological and semantic-syntactical 
analysis which makes it possible to extract entities of certain types 
from natural language texts (persons, organizations, locations, and other 
target semantic objects). The PullEnti system uses dynamically connected 
components (plugins) which makes it possible to activate various functions 
without recompiling. This is how the semantic analysis unit is incorporated. 
During the analysis, the semantic units (tokens) are established, which are 
typed phrases: text, numerical data, etc. 
Examples of implemented projects for different subject areas are given.}

\KWE{semantic modeling; named entities recognition, data intensive domains; 
automated systems of knowledge extraction; semantic search; intelligent Internet 
technologies}




\DOI{10.14357/19922264180313} %

%\vspace*{-14pt}

%\Ack
%\noindent



%\vspace*{6pt}

  \begin{multicols}{2}

\renewcommand{\bibname}{\protect\rmfamily References}
%\renewcommand{\bibname}{\large\protect\rm References}

{\small\frenchspacing
 {%\baselineskip=10.8pt
 \addcontentsline{toc}{section}{References}
 \begin{thebibliography}{99}
     
\bibitem{1-koz-1}
 \Aue{Shaumyan, S.} 2003. Categorial grammar and semiotic universal grammar.  
\textit{Conference (International) on Artificial Intelligence Proceedings}. 
Las Vegas, NV: CSREA Press. 623--629.

\bibitem{2-koz-1}
\Aue{Kuznetsov, I.\,P., E.\,B.~Kozerenko, and A.\,G.~Matskevich.} 
2011. Intelligent extraction of knowledge structures from natural language texts. 
\textit{IEEE/WIC/ACM Conferences (International) on Web Intelligence and 
Intelligent Agent Technology Proceeding}. 
Washington, DC: IEEE Computer Society. 3:269--272. doi: 10.1109/WI-IAT.2011.235.

\bibitem{3-koz-1}
\Aue{Dempster, A.\,P., N.\,M.~Laird, and D.\,B.~Rubin.} 1977. Maximum likelihood 
from incomplete data via the EM algorithm. 
\textit{J.~Roy. Stat. Soc.~B} 39(1):1--22.

 \bibitem{7-koz-1} %4
\Aue{Lund, K., and C.~Burgess.} 1996. Producing high-dimensional semantic spaces 
from lexical co-occurrence. \textit{Behav. Res. Meth. Ins. C.}  
28(2):203--208.

\bibitem{5-koz-1} %5
\Aue{Curran, J.\,R.} 2004. From distributional to semantic similarity. Edinburgh: 
University of Edinburgh. PhD Thesis. 177~p.  Available at: 
{\sf https://www.inf.ed.ac.uk/\linebreak publications/thesis/online/IP030023.pdf} 
(accessed July~19, 2018).

 \bibitem{8-koz-1} %6
\Aue{McCarthy, D., R.~Koeling, J.~Weeds, and J.~Carroll.} 2004. 
Finding predominant senses in untagged text. 
\textit{42nd Annual Meeting of Association for Computational Linguistics Proceedings}. 
Stroudsburg, PA: Association for 
Computational Linguistics. 280--287. doi: 10.3115/1218955.1218991.

\bibitem{4-koz-1}%7
\Aue{Clark, S., and S.~Pulman.} 2007. Combining symbolic and distributional 
models of meaning. \textit{AAAI Spring Symposium on Quantum Interaction Proceedings}. 
Palo Alto, CA: AAAI Press. 4~p. Available at: 
{\sf http://www.cl.cam.ac.uk/ $\sim$sc609/pubs/aaai07.pdf} (accessed July~19, 2018).

 \bibitem{6-koz-1} %8
 \Aue{Kozerenko, E.\,B.} 2012. Parallel texts alignment strategies. 
\textit{Conference (International) on Artificial Intelligence Proceedings}. 
Las Vegas, NV: CSREA Press. 2:945--951.

\bibitem{9-koz-1}
\Aue{Danielson, D.\,A.} 2003. \textit{Vectors and tensors in engineering and 
physics.} 2nd ed. Boulder, CO: Westview Press. 287~p.

\bibitem{11-koz-} %10
\Aue{Montague, R.} 1970. Universal grammar. \textit{Theoria} 36:373--398. 
(Reprinted in: 1974.
\textit{Formal philosophy: Selected papers of Richard Montague}. 
Ed. R.\,H.~Thomason. New Haven, CT: Yale University Press. 7--27.)

\bibitem{10-koz-1} %11
\Aue{Pang, B., K.~Knight, and D.~Marcu.} 2003. Syntax-based alignment of 
multiple translations: Extracting paraphrases and generating new sentences. 
\textit{Conference of the North American Chapter of the Association for 
Computational Linguistics on Human Language Technology Proceedings}. 
Stroudsburg, PA: Association 
for Computational Linguistics. 1:102--109. doi: 10.3115/1073445.1073469.

\bibitem{12-koz-1}
FACRUEVAL. 2016. Evaluation of named entity recognition and fact extraction 
systems for Russian. Available at: {\sf http://www.dialog-21.ru/media/3430/\linebreak starostinaetal.pdf} 
(accessed July~19, 2018).

\end{thebibliography}

 }
 }

\end{multicols}

\vspace*{-6pt}

\hfill{\small\textit{Received July 13, 2018}}

\pagebreak

%\vspace*{-18pt}
     
     \Contr
     
     \noindent
     \textbf{Kozerenko Elena B.} (b.\ 1959)~--- Candidate of Science (PhD) in linguistics, leading scientist, 
Institute of Informatics Problems, Federal Research Center ``Computer Science and Control'' of the Russian 
Academy of Sciences,  44-2 Vavilov Str., Moscow 119333, Russian Federation; \mbox{kozerenko@mail.ru} 
      
       \vspace*{6pt}
      
     \noindent
       \textbf{Kuznetsov Konstantin I.} (b.\ 1968)~--- leading engineer, Institute 
of Informatics Problems, Federal Research Center ``Computer Science and 
Control'' of the Russian Academy of Sciences,  44-2~Vavilov Str., Moscow 
119333, Russian Federation; \mbox{k.smith@mail.ru} 
       
       \vspace*{6pt}
       
     \noindent
       \textbf{Romanov Dmitri A.} (b.\ 1967)~--- Candidate of Science (PhD) in 
technology, associate professor, National Research University ``Higher School of 
Economics,'' 20~Myasnitskaya Str., Moscow 101000, Russian Federation; 
\mbox{DRomanov@it.ru} 

\label{end\stat}

\renewcommand{\bibname}{\protect\rm Литература}       
       