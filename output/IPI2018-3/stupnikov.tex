\def\stat{shanin}

\def\tit{МЕТОДЫ И~СРЕДСТВА ОБНАРУЖЕНИЯ НЕШТАТНЫХ СИТУАЦИЙ, ВОЗНИКАЮЩИХ 
НА~ЭЛЕМЕНТАХ ЖИЛИЩНО-КОММУНАЛЬНОЙ ИНФРАСТРУКТУРЫ$^*$}

\def\titkol{Методы и~средства обнаружения нештатных ситуаций %, возникающих 
на~элементах жилищно-коммунальной инфраструктуры}

\def\aut{И.\,А.~Шанин$^1$, С.\,А.~Ступников$^2$, В.\,Н.~Захаров$^3$}

\def\autkol{И.\,А.~Шанин, С.\,А.~Ступников, В.\,Н.~Захаров}

\titel{\tit}{\aut}{\autkol}{\titkol}

\index{Шанин И.\,А.}
\index{Ступников С.\,А.}
\index{Захаров В.\,Н.}
\index{Shanin I.\,A.}
\index{Stupnikov S.\,A.}
\index{Zakharov V.\,N.}




{\renewcommand{\thefootnote}{\fnsymbol{footnote}} \footnotetext[1]
{Работа выполнена при финансовой поддержке Министерства образования и~науки Российской Федерации (уникальный идентификатор 
проекта RFMEFI60717X0176).}}


\renewcommand{\thefootnote}{\arabic{footnote}}
\footnotetext[1]{Институт проблем информатики Федерального исследовательского центра <<Информатика и~управление>> Российской 
академии наук, \mbox{v08shanin@gmail.com}}
\footnotetext[2]{Институт проблем информатики Федерального исследовательского центра <<Информатика и~управление>> Российской 
академии наук, \mbox{sstupnikov@ipiran.ru}}
\footnotetext[3]{Институт проблем информатики Федерального исследовательского центра <<Информатика и~управление>> Российской 
академии наук, \mbox{vzakharov@ipiran.ru}}

%\vspace*{8pt}
        

\Abst{Работа относится к~области разработки специализированных 
информационных систем на основе технологии Интернета вещей. 
Рассматривается подход к~программной реализации компонентов модуля 
обнаружения нештатных ситуаций как составной части информационной 
системы, обес\-пе\-чи\-ва\-ющей поддержку технического обслуживания элементов 
жилищной инфраструктуры в~целях оперативного контроля их состояния, 
предиктивного ремонта и~оповещения о возникающих нештатных ситуациях 
и~регулярных событиях. Описаны алгоритмы и~особенности реализации 
компонентов, осуществляющих построение моделей функционирования 
элементов жилищной инфраструктуры и~обнаружение нештатных ситуаций. 
Приводятся экспериментальные результаты применения подхода для 
обнаружения нештатных ситуаций на модельных наборах данных.}

\KW{Интернет вещей; анализ данных; обнаружение нештатных ситуаций; 
жи\-лищ\-но-ком\-му\-наль\-ная инфраструктура}

\DOI{10.14357/19922264180310}
  
%\vspace*{4pt}


\vskip 10pt plus 9pt minus 6pt

\thispagestyle{headings}

\begin{multicols}{2}

\label{st\stat}

\section{Введение}

      Активное развитие технологии Интернета вещей, предполагающей, 
в~частности, оснащение физических предметов средствами взаимодействия 
друг с~другом или с~внешней средой, открывает перспективы появления 
и~совершенствования новых классов информационных систем для решения 
насущных задач различного рода. Например, в~области  
жи\-лищ\-но-ком\-му\-наль\-ной инфраструктуры оснащение элементов 
инфраструктуры (таких как трансформаторы, насосы, кондиционеры и~др.) 
разнообразными датчиками~--- температуры, тока, напряжения, звука~--- 
позволяет собирать с~них данные и~осуществлять их анализ с~целью поддержки 
технического обслуживания. 

Данная работа проводится в~рамках проекта,\linebreak 
нацеленного на создание комплекса про\-грам\-мно-тех\-ни\-че\-ских решений по 
созданию информаци\-онной системы, обеспечивающей поддержку 
тех\-нического обслуживания элементов жилищной\linebreak инфраструктуры в~целях 
оперативного контроля их состояния, предиктивного ремонта и~оповещения 
о~возникающих нештатных ситуациях и~регулярных событиях. К~нештатным 
ситуациям относится, в~частности, отказ элементов жилищной инфраструктуры в~эксплуатационный период, заявленный производителем; к~регулярным 
событиям~--- техническое обслуживание и~замена элементов по истечении 
эксплуатационного периода. В~рамках проекта рассматриваются такие 
элементы жилищной инфраструктуры, как трансформаторы, насосы, 
вентиляционные установки и~кондиционеры, газовые котлы, трубчатые электронагреватели
(\mbox{ТЭНы}), 
электрические лампы.
      
      В работе~[1] авторами была предложена архитектура информационной 
системы, включающая модули сбора данных, хранения данных, обнаружения 
и~предсказания нештатных ситуаций, информирования пользователя 
о~произошедших и~потенциально возможных нештатных ситуациях. Одной из 
основных составляющих архитектуры является модуль обнаружения 
и~предсказания нештатных ситуаций, включающий, в~част\-ности, компоненты 
построения моделей функционирования элементов жилищной инфраструктуры 
и~обнаружения нештатных ситуаций~[1]. В~данной работе рас\-смат\-ри\-ва\-ет\-ся 
подход к~программной реализации упомянутых компонентов модуля. Описаны 
алгоритмы функционирования и~особенности реализации компонентов модуля 
(разд.~2). Приводятся экспериментальные результаты применения подхода для 
обнаружения нештатных ситуаций на модельных наборах данных (разд.~3).
      
\section{Алгоритмы функционирования компонентов модуля 
обнаружения нештатных ситуаций}

\subsection{Построение моделей функционирования элементов жилищной 
инфраструктуры}

      Компонент реализует следующие подходы к~анализу временных рядов: 
сезонная интегрированная модель авторегрессии~--- скользящего среднего 
(SARIMA~--- seasonal autoregressive integrated moving average) 
и~скрытые марковские модели (HMM~--- hidden Markov models).
      
      \textit{Модель SARIMA.} Являясь обобщением классической модели 
авторегрессии~--- скользящего среднего (ARMA~--- autoregressive
moving average)~[2], данная модель находит 
массу применений в~анализе временн$\acute{\mbox{ы}}$х рядов, в~част\-ности в~задаче 
прогнозирования и~интерполяции значений. Модель\linebreak использует понятие  
тренд-се\-зон\-но\-го разложения временн$\acute{\mbox{о}}$го ряда~[3], представляя 
наблюдаемые значения в~виде суммы (или произведения) трех компонент: 
\textit{тренда}, \textit{сезонной со\-став\-ля\-ющей} и~\textit{остатков}. Трендовая 
составляющая отражает\linebreak долгосрочное изменение показателей, сезонная 
составляющая отражает периодические колебания временн$\acute{\mbox{о}}$го ряда. Остатки не 
имеют ни сезонных, ни трендовых особенностей. Для обучения модели 
необходимо оценить гиперпараметры $p$, $d$, $q$, $P$, $D$, $Q$ и~$s$:
      \begin{itemize}
\item целочисленные параметры $p$ и~$q$ соответствуют порядкам моделей 
авторегрессии (AR~--- autoregression) и~скользящего среднего (MA~---
moving average), их значение отображает 
число элементов временн$\acute{\mbox{о}}$го ряда, участ\-ву\-ющих в~по\-стро\-ении линейных 
моделей AR и~MA;
\item параметр $d$ соответствует порядку дифференцирования, 
необходимому для удаления несезонных трендов и~приведения ряда 
к~стационарному виду;
\item целочисленный параметр~$s$ соответствует периоду сезонности~--- 
периодических колебаний временн$\acute{\mbox{о}}$го ряда;
\item параметры $P$, $D$ и~$Q$ имеют тот же смысл, что и~$p$, $d$ и~$q$, но 
относятся к~сезонной со\-став\-ля\-ющей ряда.
\end{itemize}
      
      Параметры $p$, $d$, $q$, $P$, $D$ и~$Q$ принимают значения из 
множества $\{0, 1, 2, 3\}$. Для выбранных наборов значений гиперпараметров 
оцениваются коэффициенты моделей AR и~MA. 
      
      Входные данные ({сlean\_data}) при построении модели 
представляют собой набор кортежей вида $\langle${timestamp}: 
{time}, {value}: {real}$\rangle$~--- временной ряд значений 
некоторого датчика, установленного на элементе жилищной инфраструктуры, 
соответствующий штатному режиму работы элемента. Построение модели 
включает следующие шаги:
      \begin{itemize}
\item вывести на экран графики функций автокорреляции (ACF~--- autocorrelation
function) и~час\-тич\-ной 
автокорреляции (PACF~--- partial ACF) данных. По свойствам этих графиков можно сделать 
вывод о возможных значениях гиперпараметров модели, тем самым 
существенно ускорив время обучения;
\item считать введенные пользователем множества возможных значений 
гиперпараметров {p\_range}, {d\_range}, {q\_range}, 
{P\_range}, {D\_range} и~{Q\_range}, являющиеся 
подмножествами множества $\{0, 1, 2, 3\}$ и~значение~$s^*$ параметра~$s$;
\item для всех возможных наборов значений гиперпараметров $p$, $d$, $q$, 
$P,$ $D$ и~$Q$ и~входных данных {сlean\_data} вычисляется значение 
информационного критерия Акаике (AIC~--- Akaike information
criterion), выбираются конкретные значения 
гиперпараметров $p^*$, $d^*$, $q^*$, $P^*$, $D^*$, $Q^*$ и~$s^*$, 
соответствующие минимальному значению AIC;
\item обучить модель SARIMA на основании {сlean\_data} 
и~значений~$p^*$, $d^*$, $q^*$, $P^*$, $D^*$, $Q^*$ и~$s^*$,  при этом 
определить AR$^*$ и~MA$^*$~--- кортежи вещественных коэффициентов 
линейных моделей AR и~MA;
\item вычислить остатки построенной модели на входных данных 
{сlean\_data};
\item проверить статистическую гипотезу о~нестационарности остатков 
временн$\acute{\mbox{о}}$го ряда с~использованием расширенного теста Ди\-ки--Фул\-ле\-ра 
(ADF~--- augmented Dickey--Fuller test). 
В~случае если $p$-зна\-че\-ние теста больше~0,05, вернуть 
предупреждение о~неприменимости SARIMA, иначе вернуть набор 
параметров модели в~виде кортежа $\langle p^*, d^*, q^*, P^*, D^*, Q^*, 
\mathrm{AR}^*, \mathrm{MA}^*\rangle$.
      \end{itemize}
      
      Для реализации указанных шагов использовались функции пакета $p^*$, 
{statsmodels}\footnote{{\sf 
http://www.statsmodels.org/dev/generated/statsmodels.tsa.statespace.sarimax.SARIMAX.html}.} языка Python.
      
      \textit{Модель HMM.} Данная модель~[4] основывается на гипотезе 
о~том, что у анализируемой сис\-те\-мы имеется конечное множество 
<<скрытых>> состо\-яний. Таким образом, помимо последовательности\linebreak 
значений анализируемого временн$\acute{\mbox{о}}$го ряда существует скрытая 
последовательность состояний (сис\-те\-ма может принимать ровно одно 
со\-сто\-яние в~каждый момент времени). Скрытая марковская модель 
характеризуется набором па\-ра\-мет\-ров $\langle T, N, P, M, A\rangle$, где $T$~--- 
множество возможных скрытых со\-сто\-яний; $N$~--- множество возможных 
наблюдаемых значений; $P$~--- множество условных вероятностных 
распределений $p(x\vert t_i)$ для каждого скрытого состояния~$t_i$; $M$~---  
мат\-ри\-ца размера $\vert T\vert \times \vert T\vert$, элементы~$M_{ij}$ которой 
являются вероятностями перехода системы из состояния~$t_i$ 
в~состояние~$t_j$; $A$~--- вероятностное распределение первого состояния 
скрытой марковской модели (априорное распределение).
      
      В данном подходе метки классов неисправностей в~каждый момент 
времени будут напрямую соответствовать скрытым состояниям модели. 
Используются два принципиально разных способа применения скрытых 
марковских моделей: в~первом случае активно используется обучающая 
выборка, в~которой сигнал уже разделен на сегменты, соответствующие 
исправным и~неисправным режимам работы; разметка используется на этапе 
обучения параметров модели по методу максимума правдоподобия. Второй 
способ не предполагает наличие размеченной выборки, в~этом случае для 
обуче\-ния модели применяется алгоритм Бау\-ма--Уэл\-ша (частный случай  
EM (expectation-maximization) ал\-го\-рит\-ма)~[5]. Главным ограничением такого подхода является 
необходимость заранее оценить число типов различных нештатных ситуаций, 
а~следовательно, применимость HMM заметно 
снижается в~случае отсутствия размеченной обучающей выборки. На практике 
также оказывается эффективным смешанный подход: проводить 
инициализацию параметров модели с~помощью обучающей выборки по методу 
максимума правдоподобия, чтобы потом улучшить параметры с~помощью 
алгоритма Бау\-ма--Уэлша.
      
      Входные данные ({training\_data}) при по\-стро\-ении модели 
представляют собой набор кортежей\linebreak
 вида $\langle${value}: {real}, 
{hidden\_state}: {int}$\rangle$, где {value}~--- элемент 
последовательности значений анализируемого временного ряда, 
а~{hidden\_state}~--- разметка элементов последовательности по классам\linebreak 
аномалий. При реализации используется класс {HiddenMarkovModel} 
пакета {pomegranate}\footnote[2]{{\sf  
https://pomegranate.readthedocs.io/en/latest/}.} языка Python; для обучения на 
неразмеченных данных используется вызов метода {fit} на входных 
данных с~параметром {algorithm}\;=\;`{baum-welch}'. Для обучения 
на размеченных данных используется\linebreak вызов метода {fit} на входных 
данных с~параметрами {labels}\;=\;{training\_data.hidden\_state}, 
{algorithm}\;=\;{labeled}'; а~затем~--- вызов метода {fit} на 
входных данных с~параметром {algorithm}\;=`{baum-welch}'.
      
      В результате обучения формируется модель, содержащая конкретные 
значения вышеупомянутых параметров $\langle T, N, P, M, A\rangle$.

\subsection{Обнаружение нештатных ситуаций}

      Компонент получает на вход параметры построенных моделей SARIMA 
и~HMM и~анализируемые временные ряды показаний датчиков, прошедшие 
предобработку. Цель работы данного компонента~--- бинарная классификация 
показаний датчиков на показания, соответствующие штатной работе 
исправного оборудования, и~показания, соответствующие различным 
нештатным ситуациям (аномалиям во временн$\acute{\mbox{о}}$м ряду). 
      
      
      \textit{Обнаружение нештатных ситуаций при помощи модели 
SARIMA.} Модель SARIMA позволяет прогнозировать следующее значение 
временн$\acute{\mbox{о}}$го ря-\linebreak да, зная несколько предыдущих значений. 
Таким\linebreak образом, для 
обнаружения аномалий требуется сравнить реальные значения временн$\acute{\mbox{о}}$го 
ряда с~границами доверительного интервала SARIMA. Входными данными при 
обнаружении аномалий являются кортеж {model}\;=\;$\langle p, d, q, P, D, 
Q, \mathrm{AR}, \mathrm{MA}\rangle$\linebreak
 па\-ра\-мет\-ров модели и~набор {data} кортежей вида 
$\langle${timestamp}: {time}, {value}: {real}$\rangle$. 
В~качестве выходных данных выступает бинарный массив {labels}, по 
числу элементов совпадающий с~{data}, элементы которого принимают 
значения из множества~$\{0, 1\}$, где~0 означает штатность соответствующего 
элемента\linebreak {data}, а~1~--- его аномальность. Процедура обнаружения 
аномалий включает следующие шаги: 
      \begin{itemize}
\item инициализировать переменную {filtered\_data} значением 
{data};
\item для каждого~$i$ начиная с~$\max(p, q)$  
до~$\vert \mathrm{data}\vert\hm-1$ на основании первых~$i$ элементов 
временного ряда {filtered\_data} предсказать ($i+1$)-й элемент ряда 
с~использованием модели {model}, вычислить предсказанное значение 
{forecast\_value} и~доверительный интервал {confidence\_interval};
\item проверить вхождение значения {data}[$i$].{value} 
в~интервал {confidence\_interval}. В~случае вхож\-дения присвоить~0 
элементу {labels}[$i$], иначе\linebreak
 присвоить~1~элементу {labels}[$i$] 
и~присвоить значение $\langle${data}[$i$].{timestamp}, 
{forecast\_value}$\rangle$ элементу {filtered\_data}[$i$] для 
обеспечения воз\-мож\-ности дальнейшего предсказания: фактически, при этом 
во временн$\acute{\mbox{о}}$м ряду аномальное значение заменяется на штатное, 
предсказанное мо\-делью.
\end{itemize}
      
      Для реализации указанных шагов использовались функции пакета 
{statsmodels} языка Python.
      
      \textit{Обнаружение нештатных ситуаций при помощи модели HMM.} 
Здесь задача детектирования нештатных режимов работы оборудования 
рассматривается как задача сегментации сигнала: по известному набору 
параметров HMM требуется восстановить наиболее вероятную 
последовательность скрытых состояний системы. Данная задача решается при 
помощи алгоритма Витерби~--- классического алгоритма динамического 
программирования.
      
      На входе процедура обнаружения аномалий получает обученную 
скрытую марковскую модель\linebreak {model} и~анализируемые данные 
{data}~--- набор кортежей вида $\langle${timestamp}: {time}, 
{value}: {real}$\rangle$, соответствующие реальной (возможно, 
нештатной) работе оборудования~--- источника данных. При\linebreak реализации 
используется вызов метода {predict} класса \mbox{HiddenMarkovModel} из 
пакета {pomegranate}\linebreak языка Pyhon на наборе {data} с~параметром 
{algorithm}\;=\;`{viterbi}'. Результатом работы является
целочисленный массив {labels}, по числу элементов\linebreak совпадающий 
с~{data}, значения которого соответствуют скрытым состояниям 
(режимам работы) сис\-те\-мы. Значение~0 соответствует штатному режиму 
работы, а~все остальные~--- аномальным режимам.

\section{Экспериментальные результаты}

      В рамках проекта RFMEFI60717X0176 для проведения экспериментов 
создан исследователь-\linebreak ский стенд, развернутый на нескольких объектах 
жилищной инфраструктуры, включающий ап\-па\-рат\-но-про\-грам\-мные 
контроллеры и~датчики (рас\-хо\-до\-ме\-ры, датчики вход\-но\-го/вы\-ход\-но\-го 
то\-ка/на\-пря\-же\-ния, температуры, звука, осве\-щен\-ности,\linebreak
 дав\-ле\-ния), установленные 
на трансформаторах, водяных насосах, вентиляционных установках, 
кондиционерах, газовых котлах, ТЭНах, электрических лампах. В~настоящее 
время продолжается сбор данных с~компонентов исследовательского стенда для 
их последующего анализа. В~рамках же данной статьи эксперименты 
проводятся на двух готовых наборах данных: {Intel Lab Data} 
и~{LUCE}. Краткое описание наборов данных приведено \mbox{ниже}.
      
      \textit{Данные исследовательской лаборатории Intel в~Беркли.} Набор 
данных состоит из показаний бытовых домашних метеостанций, размещенных в~помещении лаборатории: измерений датчиков освещенности, влажности, 
температуры и~напряжения~\cite{6-st}. Измерения снимались раз в~31~секунду 
на протяжении~36~дней, таким образом было собрано 2,3~млн записей. 
Данные представлены в~виде набора кортежей $\langle${date}:  
{yyyy-mm-dd}, {time}: {hh:mm:ss.xxx}, {temperature}: 
{real}, {light}: {real}$\rangle$. Для анализа использовались 
данные по температуре и~освещенности. Одной из важных особенностей этого 
набора данных является ярко выраженная сезонность измерений 
(периодические колебания) с~периодом~24~ч, связанная с~суточным 
изменением температуры и~освещенности.
      
      \textit{Данные эксперимента LUCE проекта SensorScope.} Данный 
набор данных также представляет собой показания метеостанций, но датчики 
расположены не в~помещении, а~на улице (использовались уличные 
метеостанции)~\cite{7-st}. Измерения считывались каждые~120~с, 
в~набор данных вошли измерения за~44~дня. Среди измеренных значений 
присутствуют данные об окружающей среде, такие как температура, 
относительная влажность, солнечная радиация, скорость ветра и~др. Для 
анализа была выбрана температурная составляющая измерений. Таким 
образом, кортеж анализируемых данных имеет вид $\langle${date}:  
{yyyy-mm-dd}, {time}: \textit{hh:mm:ss.xxx}, {temperature}: 
{real}$\rangle$. В~данных также ярко выражена сезонная составляющая 
с~периодом~24~ч.
      
      Классификация возможных аномалий в~наборах данных Intel Lab 
и~SensorScope предложена в~\cite{8-st}. Позднее в~\cite{9-st} было 
сформулировано сле\-ду\-ющее определение. Пусть $r\_i$~--- значение в~момент 
времени~$i$, предсказываемое моделью, а~$r\_i + \mathrm{eps}\_i$~--- наблюдаемое 
значение. Тогда предлагается рас\-смат\-ри\-вать сле\-ду\-ющие типы аномалий: 
      \begin{itemize}
\item \textit{случайные аномалии}: одиночно встречающиеся зна\-че\-ния-выбросы;
\item \textit{неисправности}: частое наличие в~данных не\-вер\-ных значений 
таких, что $\mathrm{eps}\_i \hm> t$ (где~$t$~--- неко-\linebreak торое пороговое значение); 
в~последовательности неверных значений не предполагается наличие 
закономерности.
\end{itemize}
      
      В работе~\cite{10-st} разработаны алгоритмы моделирования данных 
классов аномалий, в~част\-ности построен контрольный набор данных, в~который\linebreak 
включена размеченная выборка, включающая в~себя смоделированные 
аномалии. Результаты генерации неисправностей опубликованы в~виде 
контрольно\-го набора данных~\cite{11-st}, в~котором содержатся данные 
Intel Lab и~SensorScope, включающие синтезированные неисправности. Кроме того, для 
каж\-до\-го измерения добавлено контрольное значение, соответствующее классу 
нештатной ситуации либо ее отсутствию в~конкретный момент. Классы 
размечены следующим образом: <<0>>~--- нет аномалий; <<1>>~--- случайные 
аномалии; <<2>>~--- неисправности; <<4>>~--- смещение; <<8>>, <<16>>, 
<<32>>~--- полиномиальные искажения, сгенерированы разными алгоритмами. 
Эти данные были выбраны для контроля качества детектирования аномалий 
разрабатываемой сис\-темы. 
      
      На рассматриваемых данных были применены методы обнаружения 
аномалий, основанные на моделях SARIMA и~HMM. Из набора данных 
{Intel Lab Data} были использованы данные по датчикам света, из набора 
данных {LUCE}~--- данные по датчикам температуры. 
      
      В~результате настройки моделей SARIMA наиболее часто встречались 
следующие наборы гиперпараметров $\langle p, d, q, P, D, Q\rangle:\ \langle 0, 1, 
1, 1, 1, 1\rangle$ и~$\langle 0, 1, 1, 0, 1, 1\rangle$, при сезонном параметре $s 
\hm= 24$~ч (сутки). Эксперименты показали, что модели \mbox{SARIMA} лучше 
подходят для обнаружения краткосрочных случайных аномалий, так как 
в~случае долгосрочных аномалий доверительные интервалы ее предсказаний 
становятся шире, а~следовательно, слабее.
      
      Метод HMM показал себя более успешным на продолжительных 
аномалиях (неисправностях), при этом заметно уступая методу SARIMA 
в~краткосрочных аномалиях (случайных выбросах). В~работе~\cite{12-st} 
также для долгосрочных аномалий отдается предпочтение скрытым 
марковским моделям.
      
      Так, на наборе {Intel Lab Data} средняя доля обнаруженных 
случайных аномалий составила~92,6\% для SARIMA и~29,3\% для HMM, 
а~доля обнаруженных неисправностей составила~43,2\% для \mbox{SARIMA} 
и~69,1\% для HMM. На наборе {LUCE} средняя доля обнаруженных 
случайных аномалий составила~89,2\% для \mbox{SARIMA} и~32,7\% для HMM, 
а~доля обнаруженных неисправностей составила~40,1\% для SARIMA 
и~71,9\% для HMM. Доля ложных срабатываний при этом не превышает~2\% 
от общего числа измерений.
      
      В работе~\cite{8-st} метод SARIMA был применен к~тем же наборам 
данных, но моделирование аномалий проводилось иначе. Кроме того, 
параметры модели SARIMA использовались фиксированные: $\langle 0, 1, 1, 0, 
1, 1\rangle$. Результаты на случайных аномалиях оказались схожими~--- 
более~96\% ARIMA (других сравнимых типов аномалий в~этой работе не 
было).
    
{\small\frenchspacing
 {%\baselineskip=10.8pt
 \addcontentsline{toc}{section}{References}
 \begin{thebibliography}{99}
\bibitem{1-st}
\Au{Kovalev D., Shanin I., Stupnikov~S., Zakharov~V.} Data mining methods and techniques for 
fault detection and predictive maintenance in housing and utility infrastructure~// Conference 
(International) on Engineering Technologies and Computer Science.~--- IEEE, 2018.
doi: 10.1109/EnT.2018.00016.
\bibitem{2-st}
\Au{Box G.\,E.\,P., Jenkins G.\,M., Reinsen~G.\,C., Ljung~G.\,M.} Time series analysis: 
Forecasting and control.~--- 5th ed.~--- Wiley, 2015. 712~p.
\bibitem{3-st}
\Au{Cleveland R.\,B., Cleveland~W.\,S., McRae~J.\,E., Terpenning~I.} STL: A~seasonal-trend 
decomposition~// J.~Off. Stat., 1990. Vol.~6. Iss.~1. P.~3--73.
\bibitem{4-st}
\Au{Bengio Y., Frasconi~P.} An input output HMM architecture~//
NIPS Proceedings.~--- MIT Press, 1995. P.~427--434.

\bibitem{5-st}
\Au{Rabiner L.\,R.} A~tutorial on hidden Markov models and selected applications in speech 
recognition~// Readings in speech recognition~/
Eds. A.~Waibel, K.-F.~Lee.~--- San Francisco, CA, USA:
Morgan Kaufmann, 1990. P.~267--296.
\bibitem{6-st}
\Au{Bodik P., Hong~W., Guestrin~C., Madden~S., Paskin~M., Thibaux~R.} Intel Lab Data. Intel 
Berkeley Research lab, 2004. {\sf http://db.csail.mit.edu/labdata/labdata.html}.
\bibitem{7-st}
Sensorscope: Sensor networks for environmental monitoring. Lausanne Urban Canopy Experiment 
(LUCE).~--- EPFL, 2006. {\sf https://lcav.epfl.ch/page-145180-en.html}.
\bibitem{8-st}
\Au{Sharma A.\,B., Golubchik~L., Govindan~R.} Sensor faults: Detection methods and prevalence 
in real-world datasets~// ACM Trans. Sens. Netw., 2010. Vol.~6. Iss.~3. 
P.~23. 
\bibitem{9-st}
\Au{Baljak V., Tei~K., Honiden~S.} Fault classification and model learning from sensory 
Readings~---
Framework for fault tolerance in wireless sensor networks~// 
8th Conference (International) on 
Intelligent Sensors, 
Sensor Networks and Information Processing.~--- IEEE, 
2013. P.~408--413.
{\looseness=1

}
\bibitem{10-st}
\Au{De Bruijn B., Nguyen~T.\,A., Bucur~D., Tei~K.} Benchmark datasets for fault detection and 
classification in sensor data~// 5th Conference (International) on Sensor Networks  
Proceedings.~--- SCITEPRESS, 2016. P.~185--195.
\bibitem{11-st}
\Au{De Bruijn B., Nguyen T.\,A., Bucur~D., Tei~K.} Benchmark datasets for fault detection and 
classification in sensor data, 2015. {\sf http://tuananh.io/datasets}.
\bibitem{12-st}
\Au{Warriach E.\,U., Aiello~M., Tei~K.} A~machine learning approach for identifying and 
classifying faults in wireless sensor network~// 
 15th Conference (International) on
Computational Science and Engineering.~--- IEEE, 2012. P.~618--625.

 \end{thebibliography}

 }
 }

\end{multicols}

\vspace*{-6pt}

\hfill{\small\textit{Поступила в~редакцию 15.07.18}}

%\vspace*{6pt}

\newpage

\vspace*{-28pt}

%\hrule

%\vspace*{2pt}

%\hrule

%\vspace*{-2pt}


\def\tit{METHODS AND TOOLS FOR~FAULT DETECTION 
ON~ELEMENTS OF~HOUSING AND~UTILITY INFRASTRUCTURE}


\def\titkol{Methods and tools for~fault detection on~elements of~housing and~utility 
infrastructure}

\def\aut{I.\,A.~Shanin, S.\,A.~Stupnikov, and~V.\,N.~Zakharov}

\def\autkol{I.\,A.~Shanin, S.\,A.~Stupnikov, and~V.\,N.~Zakharov}

\titel{\tit}{\aut}{\autkol}{\titkol}

\vspace*{-9pt}

\noindent
Institute of Informatics Problems, Federal Research Center ``Computer Science 
and Control'' of the Russian Academy of Sciences, 44-2~Vavilov Str., Moscow 
119333, Russian Federation


\def\leftfootline{\small{\textbf{\thepage}
\hfill INFORMATIKA I EE PRIMENENIYA~--- INFORMATICS AND
APPLICATIONS\ \ \ 2018\ \ \ volume~12\ \ \ issue\ 3}
}%
 \def\rightfootline{\small{INFORMATIKA I EE PRIMENENIYA~---
INFORMATICS AND APPLICATIONS\ \ \ 2018\ \ \ volume~12\ \ \ issue\ 3
\hfill \textbf{\thepage}}}

\vspace*{5pt}
      
      
      
      \Abste{The work belongs to the area of development of specific 
      information systems based on the Internet of Things technology. 
      An approach for program implementation of a~module intended for detection 
      of faults on elements of housing and utility infrastructure is proposed. 
      The module is considered as a~part of an information system aimed at 
      technical maintenance of mentioned elements: condition monitoring, 
      predictive maintenance, fault detection, and reporting. 
      Operation algorithms of module components are described: building 
      of operation models for housing and utility infrastructure elements and 
      fault detection. The approach is applied on 
      a~couple of datasets for fault detection, experimental results are presented.}
      
      \KWE{Internet of Things; data analysis; fault detection; 
      housing and utility infrastructure}
      
      
      
      
\DOI{10.14357/19922264180310} %

%\vspace*{-14pt}

\Ack
      \noindent
      The research is supported by Ministry of Education and Science of the 
Russian Federation (project's unique identifier RFMEFI60717X0176).



\vspace*{6pt}

  \begin{multicols}{2}

\renewcommand{\bibname}{\protect\rmfamily References}
%\renewcommand{\bibname}{\large\protect\rm References}

{\small\frenchspacing
 {%\baselineskip=10.8pt
 \addcontentsline{toc}{section}{References}
 \begin{thebibliography}{99}
\bibitem{1-st-1}
\Aue{Kovalev, D., I.~Shanin, S.~Stupnikov, and V.~Zakharov.} 2018. Data mining 
methods and techniques for fault detection and predictive maintenance in housing and utility 
infrastructure. \textit{Conference (International) on Engineering Technologies and Computer 
Science}. IEEE. doi: 10.1109/EnT.2018.00016.
\vspace*{2pt}

\bibitem{2-st-1}
\Aue{Box, G.\,E.\,P., G.\,M.~Jenkins, G.\,C.~Reinsen, and G.\,M.~Ljung.} 2015. 
\textit{Time 
series analysis: Forecasting and control.} 5th ed. Wiley. 712~p.
\vspace*{2pt}

\bibitem{3-st-1}
\Aue{Cleveland, R.\,B., W.\,S.~Cleveland, J.\,E.~McRae, and I.~Terpenning.} 1990. STL: 
A~seasonal-trend decomposition. \textit{J.~Off. Stat.} 6(1):3--73.
\vspace*{2pt}

\bibitem{4-st-1}
\Aue{Bengio, Y., and P.~Frasconi.} 1995. An input output HMM architecture. 
\textit{NIPS Proceedings}. MIT Press. 427--434.
\vspace*{2pt}

\bibitem{5-st-1}
\Aue{Rabiner, L.\,R.} 2004. A~tutorial on hidden Markov models and selected applications in 
speech recognition. \textit{Readings in speech recognition}. 
Eds. A.~Waibel and K.-F.~Lee. San Francisco, CA: Morgan Kaufmann. 267--296.
\vspace*{2pt}

\bibitem{6-st-1}
\Aue{Bodik, P., W.~Hong, C.~Guestrin, S.~Madden, M.~Paskin, and R.~Thibaux.} 2004. Intel Lab 
Data. Intel Berkeley Research lab. Available at: 
{\sf http://db.csail.mit.edu/\linebreak labdata/labdata.html} 
(accessed July~16, 2018).
\bibitem{7-st-1}
Sensorscope. 2006. Sensorscope: Sensor networks for environmental monitoring. Lausanne Urban 
Canopy Experiment (LUCE). EPFL. Available at: {\sf  
https://lcav. epfl.ch/page-145180-en.html} (accessed July~16, 2018).
\vspace*{2pt}

\bibitem{8-st-1}
\Aue{Sharma, A.\,B., L.~Golubchik, and R.~Govindan.} 2010. Sensor faults: Detection methods 
and prevalence in real-world datasets. \textit{ACM Trans. Sens. Netw.} 
6(3):23. 
\vspace*{2pt}

\bibitem{10-st-1} %9
\Aue{Baljak, V., K.~Tei, and S.~Honiden.} 2013. Fault classification and model learning from 
sensory Readings~--- Framework for fault tolerance in wireless sensor networks. 
\textit{8th Conference 
(International) on Intelligent Sensors, Sensor Networks and Information Processing.} 
IEEE. 408--413.
\vspace*{2pt}

\bibitem{9-st-1} %10
\Aue{De Bruijn, B., T.\,A.~Nguyen, D.~Bucur, and K.~Tei.} 2016. Benchmark datasets for fault 
detection and classification in sensor data. \textit{5th Conference (International) on Sensor 
Networks Proceedings}. SCITEPRESS. 185--195.
\vspace*{2pt}

\bibitem{11-st-1}
\Aue{De Bruijn, B., T.\,A.~Nguyen, D.~Bucur, and K.~Tei.} 2015. Benchmark datasets for fault 
detection and classification in sensor data. Available at: {\sf http://tuananh.io/datasets/} (accessed 
July~16, 2018).
\vspace*{2pt}

\bibitem{12-st-1}
\Aue{Warriach, E.\,U., M.~Aiello, and K.~Tei.} 2012. A machine learning approach for identifying 
and classifying faults in wireless sensor network. 
\textit{15th Conference (International) on Computational Science and Engineering}. 
 IEEE. 618--625.
\end{thebibliography}

 }
 }

\end{multicols}

%\vspace*{-6pt}

\hfill{\small\textit{Received July 15, 2018}}

\pagebreak

%\vspace*{-18pt}
      
      \Contr
      
      \noindent
      \textbf{Shanin Ivan A.} (b.\ 1991)~--- junior scientist, Institute of Informatics 
Problems, Federal Research Center ``Computer Science and Control'' of the Russian 
Academy of Sciences, 44-2~Vavilov Str., Moscow 119333, Russian Federation; 
\mbox{v08shanin@gmail.com}

\vspace*{5pt}
      
      \noindent
      \textbf{Stupnikov Sergey A.} (b.\ 1978)~--- Candidate of Science (PhD) in 
technology,, senior scientist, Institute of Informatics Problems, Federal Research 
Center ``Computer Science and Control'' of the Russian Academy of Sciences,  
44-2~Vavilov Str., Moscow 119333, Russian Federation; 
\mbox{sstupnikov@ipiran.ru}

\vspace*{5pt}
      
      \noindent
      \textbf{Zakharov Victor N.} (b.\ 1948)~--- Doctor of Science in technology, 
associate professor; Scientific Secretary, Federal Research Center ``Computer 
Science and Control'' of the Russian Academy of Sciences, 44-2~Vavilov Str., 
Moscow 119333, Russian Federation; \mbox{vzakharov@ipiran.ru}
\label{end\stat}

\renewcommand{\bibname}{\protect\rm Литература}       