\documentclass[10pt,a4paper]{article}
\usepackage[utf8]{inputenc}
\usepackage{graphicx}
\usepackage[english]{babel}
\textwidth=17cm
\oddsidemargin=1.18pt
\topmargin=0pt
\textheight=23cm
\newdimen\stdindent


\begin{document}





\begin{center}
\Large{\bf Search of the conflicts in security policies: a model of random graphs 
\\[25pt]}
\end{center}

\begin{center}
{\bf A. Grusho, N. Grusho, E. Timonina  \\[5pt]}
\end{center}

\begin{abstract}
Risk analysis requires at least a rough estimate of the probability of problematic situations in the implementation of security policy. Such estimates can be obtained in the analysis of conflicts in security policies using models of random graphs. In this paper we consider two examples of conflicts in security policy and construct a model of random graphs for their research. The results of analysis were used to evaluate the influence of certain parameters of random graphs on the probability of the existence of conflicts and complexity of search algorithms in a real conflict security policies. 

\end{abstract}

\end{document}




%%%%%%%%%%%%%%%%%%%%%%%%%%%%%%%%%%%%%%%%%%%%%%%%%%%%%%%