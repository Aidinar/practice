

%%%%%%%%%%%%%%%%%%%%%%%%%%%%%%%%%%%%%%%%%%%%%%%%%%%%%%%%%%%%%%%%%
\def\P{{\bf P}}                       
\def\E{{\bf E}}                       
\def\DDD{{\bf D}}
\def\N{{\bf N}}                    
\def\AD{{\cal A}}                       



\def\stat{faz}

\def\tit{ЗАКОНЫ ПОВТОРНОГО ЛОГАРИФМА ДЛЯ ЧИСЛА БЕЗОШИБОЧНЫХ
БЛОКОВ ПРИ ПОМЕХОУСТОЙЧИВОМ КОДИРОВАНИИ}

\def\titkol{Законы повторного логарифма для~числа безошибочных
блоков при~помехоустойчивом кодировании}

\def\autkol{А.\,Н.~Чупрунов, И.~Фазекаш}
\def\aut{А.\,Н.~Чупрунов$^1$, И.~Фазекаш$^2$}

\titel{\tit}{\aut}{\autkol}{\titkol}

%{\renewcommand{\thefootnote}{\fnsymbol{footnote}}\footnotetext[1]
%{Исследование поддержано грантами РФФИ 08-07-00152 и 09-07-12032.
%Статья написана на основе материалов доклада, представленного на IV 
%Международном семинаре  <<Прикладные задачи теории вероятностей и математической статистики, 
%связанные с моделированием информационных систем>> (зимняя сессия, Аоста, Италия, январь--февраль 2010~г.).}}

\renewcommand{\thefootnote}{\arabic{footnote}}
\footnotetext[1]{Научно-исследовательский институт математики и механики
им.\ Н.\,Г.~Чеботарева, achuprunov@mail.ru}
\footnotetext[2]{Дебреценский университет, fazekas.istvan@inf.unideb.hu}

\vspace{-8pt}


\Abst{Рассматриваются  сообщения, состоящие из блоков.
Каждый блок кодируется помехоустойчивым кодом, который может
исправить не более $r$~ошибок. При этом предполагается, что число
ошибок в блоке~--- неотрицательная целочисленная случайная величина.
Эти случайные величины независимы и одинаково распределены. Кроме
того, предполагается, что число ошибок в сообщении принадлежит
некоторому конечному подмножеству множества неотрицательных чисел. 
Получены аналоги закона повторного логарифма для случайной
величины~--- числа безошибочных блоков в сообщении.}

\vspace{-2pt}

\KW{обобщенная схема размещения; условная
вероятность; условное математическое ожидание; экспоненциальное
неравенство; закон повторного логарифма; код БЧХ}

\vspace{-1pt}

       \vskip 14pt plus 9pt minus 6pt

      \thispagestyle{headings}

      \begin{multicols}{2}

      \label{st\stat}

\section{Введение и основные результаты}

 Будем рассматривать код, который позволяет исправить не
более   $r$~ошибок типа замещения. Частным случаем такого кода
является код Боу\-за--Чоуд\-ху\-ри--Хок\-вин\-гхе\-ма (БЧХ) (см.\ о кодах БЧХ, например, в~[1]). Работа
посвящена изучению асимптотического поведения  случайной величины~$S_N$~--- 
числа безошибочных блоков  в сообщении, состоящем из $N$~блоков, 
причем каждый блок подвергается помехоустойчивому
кодированию, а число ошибок в сообщении принадлежит некоторому
конечному множеству. Будем предполагать, что все рассматриваемые
случайные величины определены на вероятностном пространстве
$(\Omega,
\mathfrak{A}, {\mathbf P})$.


 Рассмотрим  сообщение, состоящее из $N$~блоков. Пусть случайная величина~$\xi_{Nj}$~---
 число ошибок в $j$-м блоке. Будем предполагать, что  $\xi_{Nj}$, $1\le
j\le N$,~--- независимые неотрицательные целочисленные случайные
величины распределенные так же, как случайная величина~ $\xi$.   Кроме того,  будем считать,
 что число ошибок в сообщении  принадлежит некоторому конечному
 подмножеству~$M$ множества неотрицательных целых  чисел.
  Тогда число безошибочных блоков в  сообщении~--- случайная величина

\noindent
$$
S_{MN}=\sum_{i=1}^NI_{MNi}\,,
$$
где $I_{MNi}$~--- индикатор события~$A_{MNi}$, состоящего в том, что
$i$-й блок  сообщения имеет не более $r$~ошибок. Заметим, что
событие

\noindent
\begin{multline*}
A_{MNi}=\left \{\xi_{Ni}\le r\,\,\, \vert |\,\,\, \xi_{N1}+\xi_{N2}+ \dots {}\right.\\
\left.{}\dots +\xi_{NN}\in
M\, \right \}=\{\xi_{Ni}\le r\,\,\, \vert |\,\,\, A\,\ \}\,,
\end{multline*}
где событие

\noindent
\begin{multline*}
A= A_{MN}=\left\{\xi_{N1}+\xi_{N2}+ \dots +\xi_{NN}\in
M\, \right\}={}\\
{}=\cup_{k\in M}A_k\,,
\end{multline*}
а события
%\noindent
$$ A_{k}=A_{kN}=\left\{ \xi_{N1}+\xi_{N2}+ \dots +\xi_{NN}=k\right\}\,.
$$

Будем предполагать, что распределение случайной величины~$\xi$
зависит от параметра~$\theta$. Пусть существует последовательность
неотрицательных чисел~ $b_0$, $b_1, \dots$ такая, что радиус
сходимости~$R$ ряда
$$ 
B(\theta)=\sum_{k=0}^{\infty}\fr{b_k\theta^k}{k!}
$$
положителен. Тогда  случайная величина $\xi=$\linebreak $=\xi(\theta)$, $
0<\theta<R$ распределена по следующему закону:
$$
p_k=p_k(\theta)={\mathbf
P}\{\xi=k\}=\fr{b_k\theta^k}{k!B(\theta)}\,,\enskip k=0,1,2,\dots
$$
Будем  предполагать, что выполняется условие~$A_1$, т.\,е.\
 $b_0>0$, $b_1>0$.

Если множество~$M$ состоит из одного элемента, то события~$A_{MN}$
являются событиями обобщенной схемы размещения и их вероятности не
зависят от~$\theta$. Обобщенная схема размещения была введена В.\,Ф.~Колчиным в~[2] (см.\ также монографию~[3]).

 Условие $A_1$ и случайные величины~ $\xi(\theta)$ были введены в~[4]. В~[4--6] 
 получены предельные теоремы для сумм независимых случайных
величин~$\xi_{Ni}(\theta)$. В частности, в~[4] показано, что
математическое ожидание
$$
m=m(\theta)=  {\E}\xi=  \fr{\theta B'\left(\theta\right)}{B(\theta)}
$$
и дисперсия
$$
 \sigma^2=\sigma^2(\theta)=  {\DDD}^2\xi=
\fr{\theta^2 B^{\prime\prime} (\theta)}{B(\theta)}+ \fr{\theta
B'(\theta)}{B(\theta)}- \fr{\theta^2
(B'(\theta))^2}{(B(\theta))^2}\,.
$$
Поэтому дисперсия
\begin{equation}
\sigma^2(\theta)=\theta m'(\theta)\,. \label{e1.1faz}
\end{equation}


Пусть $0<\theta'<\theta^{\prime\prime}<R$. Если $\sigma^2(\theta)=0$ для
некоторого  $\theta\in [\theta', \theta^{\prime\prime}]$, то случайная
величина $\xi(\theta)$~--- константа. Но так как $b_0>0$, $b_1>0$,
случайная величина~$\xi(\theta)$ константой не является. Поэтому
$\sigma^2(\theta)$, $\theta\in [\theta', \theta^{\prime\prime}]$~---
положительная непрерывная функция. Следовательно,
\begin{multline}
0<C_1 =\inf_{\theta\in [\theta', \theta^{\prime\prime}]} \sigma^2(\theta) \le
\sup_{\theta\in [\theta', \theta^{\prime\prime}]}\sigma^2(\theta)={}\\
{}= C_2<\infty\,.
\label{e1.2faz}
\end{multline}
Условия~(\ref{e1.1faz}) и~(\ref{e1.2faz}) влекут
$$
0<\fr{C_1}{\theta^{\prime\prime}}=\inf_{\theta\in [\theta',
\theta^{\prime\prime}]}m'(\theta) \le \sup_{\theta\in [\theta',
\theta^{\prime\prime}]}m'(\theta)= \fr{C_2}{\theta'}<\infty\,.
$$
Таким образом, $m(\theta)$, $\theta\in [\theta', \theta^{\prime\prime}]$,~---
положительная непрерывная строго возрастающая функция. Обозначим
через~$m^{-1}$ ее обратную функцию, $m(R)=\lim_{\theta\to
R-0}m(\theta)$. Будем обозначать через $\xi_1(\theta), \dots ,
\xi_N(\theta)$ независимые копии случайной величины
$\xi=\xi(\theta)$.

Обозначим $n'_M=\sup\{n: n\in M\}$, $\alpha_{MN}=$\linebreak $={n'_M}/{N}$. Будем
предполагать, что $\alpha_{MN}<m(R)$. Пусть
$\theta_{MN}=m^{-1}(\alpha_{MN})$. Тогда случайная величина
$T=(1/\sqrt{N})\sum\limits_{i=1}^NI_{\{\xi_{Ni}(\theta_{NM})\le r\}}$
имеет дисперсию $\sigma^2_{MN}=p_{\le r}(\theta_{MN})(1-p_{\le
r}(\theta_{MN}))$, где $p_{\le
r}(\theta_{MN})=\sum\limits_{k=0}^rp_{k}(\theta_{MN})$. Будем использовать
следующую оценку для дисперсии случайной величины~$T$. Пусть
$\theta'\le\theta_{MN}\le\theta^{\prime\prime}$. Тогда
\begin{equation}
\sigma^2_{MN}\ge \fr{\sum_{k=0}^rb_k(\theta')^k}{B(\theta^{\prime\prime})}
\,
\fr{\sum_{k=r+1}^{\infty}b_k(\theta')^k}{B(\theta^{\prime\prime})}=C_3>0\,.
\label{e1.3faz}
\end{equation}

Основными результатами статьи являются следующие аналоги закона
повторного логарифма для числа безошибочных блоков~--- случайной
величины~$S_{MN}(\theta_{MN})$.

\medskip

\noindent

\textbf{Теорема 1.} {\it  Пусть  $0<\alpha'< \alpha^{\prime\prime}<m(R)$. Пусть
$M_N$~--- такая последовательность конечных подмножеств множества
неотрицательных чисел, что $\alpha'\le\alpha_{M_NN}\le\alpha^{\prime\prime}$.
Обозначим $S_{N}=S_{M_NN}(\theta_{M_NN})$, $\sigma^2_N=
\sigma^2_{M_NN}$. Тогда
\begin{equation}
\limsup_{N\to\infty}\fr{|S_N-{\E}S_N|}{\sqrt{N\ln(N)}\sigma_N}\le
2\sqrt{6} 
\label{e1.4faz}
\end{equation}
почти наверное. }

\medskip

\noindent

\textbf{Теорема 2.} {\it Пусть  $0<\alpha'< \alpha^{\prime\prime}<m(R)$. Пусть
$M_n$~---  последовательность конечных подмножеств множества
неотрицательных чисел.  Обозначим $S_{nN}=S_{M_nN}(\theta_{M_nN})$,
$\sigma^2_{nN}= \sigma^2_{M_nN}$.  Тогда
\begin{equation}
\limsup_{N,n\to\infty,\alpha'<\alpha_{nN}< \alpha^{\prime\prime}
 }
\fr{|S_{nN}-{\E}S_{nN} |}{\sqrt{N\ln(N)}\sigma_{nN}}
 \le 2\sqrt{10}
\label{e1.5faz}
\end{equation}
почти наверное. }


%\medskip

\section{Леммы}

Пусть  $A\in\AD$~--- такое фиксированное событие, что $\P(A)>0$.
Напомним, что условная вероятность~${\P^A}$ определяется формулой
 $$
 \P^A(B)=\fr{\P(B\cap A)}{\P(A)}\,,\enskip  B \in\AD\,.
 $$
Будем обозначать через ${\E}^A$  математическое ожидание
относительно вероятности $\P^A$.

Легко видеть, что для любой случайной величины $S$ и  для любого
 $p>0$ справедливо неравенство ${\E}^A|S|^p\le (1/\P(A)){\E}|S|^p$.
 Следующая лемма показывает, что аналогичное неравенство верно для
центрированных абсолютных моментов случайной величины $S$.

\medskip

\noindent
\textbf{Лемма 2.1.} {\it Пусть $1\le  p<\infty$. Тогда
\begin{equation}
 {\E}^A|S-{\E}^AS|^p   \le  2^p\fr{{\E}|S-{\E}S|^p}{{\P}(A)}\,.
\label{e2.1faz}
\end{equation}}

\medskip

Лемма~2.1 доказана в~[7].

\medskip

Будем обозначать через $r_i$, $i\in\N$,  функции Радемахера, $\E^r$~--- 
математическое ожидание относительно $\sigma$-алгебры,
определенной случайными величинами~$r_i$,$\gamma$~--- гауссовская
случайная величина с нулевым средним и единичной дисперсией.
Воспользуемся  неравенством Хинчина, в котором константа имеет
наиболее точный вид.

\medskip

\noindent
\textbf{Лемма 2.2.} {\it Пусть $1\le p<\infty$. Пусть $c_i\in{\bf R}$,
$1\le i\le n$. Тогда
\begin{multline}
{\E_r}\left|\sum\limits_{i=1}^n c_ir_i\right|^p \le{}\\
{}\le
\sqrt{2}\left(1+O\left(\fr{1}{p}\right)\right)e^{-p/2}
p^{p/2} \left(\sum\limits_{i=1}^n(c_i)^2\right)^{p/2}\,.
\label{e2.2faz}
\end{multline}
}

\medskip

\noindent
Д\,о\,к\,а\,з\,а\,т\,е\,л\,ь\,с\,т\,в\,о\,.\ В~[8] доказано неравенство Хинчина с
неулучшаемой константой
\begin{equation}
{\E_r}\left|\sum\limits_{i=1}^n c_ir_i\right|^p \le {\mathbf
E}|\gamma|^p\left(\sum\limits_{i=1}^n(c_i)^2\right)^{p/2}\,.
\label{e2.3faz}
\end{equation}
Применяя оценку для гамма-функции~$\Gamma(p)$, полученную в~[9],
имеем
\begin{multline}
 {\mathbf
E}|\gamma|^p=\fr{2^{p/2}}{\sqrt{\pi}}\,\Gamma\left(\fr{p}{2}+\fr{1}{2}\right)={}\\[2pt]
{}=
\fr{2^{p/2}}{\sqrt{2\pi}}\,e^{-p/2-1/2}\left(\fr{p}{2}+\fr{1}{2}\right)^{p/2+1/2}\times{}\\[2pt]
{}\times
\sqrt{\fr{2\pi}{\left(p/2+1/2\right)}}\left(1+O\left(\fr{1}{p}\right)\right)={}\\[2pt]
{}=
\sqrt{2}2^{p/2}e^{-p/2-1/2}\left(\fr{p}{2}+\fr{1}{2}\right)^{p/2}
\left(1+O\left(\fr{1}{p}\right)\right)={}\\[2pt]
{}=\sqrt{2}e^{-p/2}e^{-1/2}\left(p+1\right)^{{p}/{2}}
\left(1+O\left(\fr{1}{p}\right)\right)={}\\[2pt]
{}=
\sqrt{2}e^{-p/2}e^{-1/2}\left(\fr{p+1}{p}\right)^{p/2}p^{p/2}
\left(1+O\left(\fr{1}{p}\right)\right)\le{}\\[2pt]
{}\le
\sqrt{2}e^{-{p}/{2}}e^{-{1}/{2}}e^{{1}/{2}}p^{p/2}
\left(1+O\left(\fr{1}{p}\right)\right)\,.
\label{e2.4faz}
\end{multline}
Используя~(\ref{e2.4faz}) в~(\ref{e2.3faz}), получаем~(\ref{e2.2faz}). Лемма доказана.

\medskip

\noindent

\textbf{Лемма 2.3.} {\it  Пусть $\eta_i$, $1\le i\le n$,~---
независимые случайные величины с математическими ожиданиями~$a_i$ и
дисперсиями~$\sigma_i^2$ такие, что $0\le\eta_i\le 1$. Обозначим
$$
a=\fr{1}{n} \sum_{i=1}^na_i\,;\quad \sigma^2=\fr{1}{n}\sum_{i=1}^n\sigma^2_i\,.
$$
Пусть $p\ge 2$. Тогда
\begin{equation}
a^p\le {\E}\left(\frac{ \sum_{i=1}^n\eta_i}{n}\right)^p\le a^p(1+B)\,,
\label{e2.5faz}
\end{equation}
где
$$
B= \fr{\sigma^2}{2}f_2\left(\fr{p}{a\sqrt{n}}\right)\,,
$$
а функция
\begin{multline*}
f_2(x)=x^2\left(2{\mathbf
E}\left(\gamma^2\exp{(x|\gamma|)}\right)-1\right)={}\\
{}=
4x^2\left(e^{x^2/2}(x^3+3x)\Phi(x)+\fr{x}{\sqrt{2\pi}}\right)\,,
\end{multline*}
где $\Phi$~--- функция распределения случайной величины~$\gamma$.
}

\medskip

\noindent
Д\,о\,к\,а\,з\,а\,т\,е\,л\,ь\,с\,т\,в\,о\,.\  Левое неравенство в~(\ref{e2.5faz}) следует из
неравенства Иенсена. Пусть семейство $\{\eta'_i, 1\le i\le n\}$~---
независимая копия семейства $\{\eta_i, 1\le i\le n\}$.  Функция
$g(x)=|a+x|^p$  имеет непрерывную вторую производную. Поэтому по
формуле Тейлора имеем
$$
g(x)=a^p+\fr{pa^{p-1}}{1!}x+\frac{p(p-1)|a+\theta
x|^{p-2}}{2!}x^2\,,
$$
где $\theta\in (-1, 1)$. Отсюда при $x={
\sum\limits_{i=1}^n(\eta_i-a_i)}/{n}$ получается равенство:
\begin{multline*}
{\E}\left|\frac{ \sum_{i=1}^n\eta_i}{n}\right|^p=a^p +
\fr{p(p-1)}{2}{\E}\times{}\\
{}\times \left|
a'+\theta\frac{
\sum_{i=1}^n(\eta_i-a_i)}{n}\right|^{p-2}\left(\frac{
\sum_{i=1}^n(\eta_i-a_i)}{n}\right)^2={}\\
{}=a^p(1+B')\,,
\end{multline*}
где $\theta=\theta(\omega)$, $-1\le\theta\le 1$. Так как $1+x\le
e^x$, $0\le x<\infty$, и
\begin{multline*}
{\E}\left|\frac{
\sum_{i=1}^n(\eta_i-\eta'_i)^2}{n}\right|^{(k+2)/2}\!\!\le{}\\
{}\le {\E}\left(\frac{
\sum_{i=1}^n(\eta_i-\eta'_i)^2}{n}\right)=2\sigma^2,
\end{multline*}
используя~(\ref{e2.2faz}), получаем
\begin{multline*}
B'\le\fr{p(p-1)}{2}\,{\E}\left(1+\left|\frac{
\sum_{i=1}^n(\eta_i-a_i)}{na}\right|\right)^{p-2}\times{}\\
{}\times 
\left(\frac{
\sum_{i=1}^n(\eta_i-a_i)}{na}\right)^2\le{}\\
{}\le\fr{p(p-1)}{2}{\E}\exp\left((p-2)\left|\frac{
\sum_{i=1}^n(\eta_i-a_i)}{na}\right|\right)\times{}
\end{multline*}
\begin{multline*}
{}\times\left(\frac{
\sum_{i=1}^n(\eta_i-a_i)}{na}\right)^2\le{}\\[2pt]
{}
\le\fr{p(p-1)}{2}\sum_{k=0}^{\infty}\fr{1}{k!}(p-2)^k{\E}\left|\frac{
\sum_{i=1}^n(\eta_i-a_i)}{na}\right|^{k+2}\le{}\\[2pt]
{}
\le\fr{1}{2}\,\fr{p(p-1)}{a^2n}\sum_{k=0}^{\infty}\fr{1}{k!}\left(\fr{p-2}{a\sqrt{n}}
\right)^k\times{}\\[2pt]
{}\times {\E}\left|\frac{
\sum_{i=1}^n(\eta_i-a_i)}{\sqrt{n}}\right|^{k+2}\le{}\\[2pt]
{}
\le\fr{1}{2}\,\fr{p(p-1)}{a^2n}\left(
\vphantom{\frac{
\sum_{i=1}^n(\eta_i-\eta'_i)^2}{n}}
\sigma^2+\sum_{k=1}^{\infty}\fr{1}{k!}\left(\fr{p-2}{a\sqrt{n}}
\right)^k\right.\times {}\\[2pt]
{}\left.{}\times {\E}|\gamma|^{k+2}{\E}\left|\frac{
\sum_{i=1}^n(\eta_i-\eta'_i)^2}{n}\right|^{(k+2)/2}\right)\le{}\\[2pt]
{}\le
\fr{\sigma^2}{2}\,\fr{p(p-1)}{a^2n}\left(1+2\sum_{k=1}^{\infty}\fr{1}{k!}\left(\fr{p-2}{a\sqrt{n}}
\right)^k{\E}|\gamma|^{k+2}\right)={}\\[9pt]
{}=
\fr{\sigma^2}{2}\,\fr{p(p-1)}{a^2n}\times{}\\[2pt]
{}\times \left(1+2\left({\E}\left(\gamma^2\exp
\left(\fr{p-2}{a\sqrt{n}}|\gamma|\right)\right)-1 \right)
\right) \le B\,.
\end{multline*}
Доказательство закончено.

\bigskip

Пусть $A_i$, $1\le i\le n$~--- независимые события, $p_i=\P(A_i)$, $I_i$~--- индикаторы событий $A_i$, $\sigma^2_i =
 p_i(1-p_i)$~--- дисперсии индикатора~$I_i$. Будем предполагать, что $0<p_i<1$, $1\le i\le n$. 
 Рас\-смот\-рим случайную величину
$$
\mu=\sum_{i=1}^nI_i\,.
$$
Тогда $\sigma^2=(1/n)\sum\limits_{i=1}^{n}\sigma_i^2$ -- дисперсия
случайной величины $(1/\sqrt{n})\mu$.

Основной результат этого параграфа~--- сле\-ду\-ющее экспоненциальное
неравенство.

\medskip

\noindent
\textbf{Лемма 2.4.} {\it Пусть $\varepsilon>0$. Тогда
\begin{multline}
\P^A\left\{ \fr{|\mu-{\E^A}\mu |}{\sqrt{n}\sigma} \ge
\varepsilon\right\} \le
\fr{\sqrt{2}}{\P(A)}\left(1+O\left(\fr{1}{\varepsilon^2}\right)
\right)\times{}\\[9pt]
{}\times
\left(1+\fr{1}{8}f_2\left(\fr{\varepsilon^2}{32\sigma^2\sqrt{n}}\right)\right)
e^{-{\varepsilon^2}/16 }\,. 
\label{e2.6faz}
\end{multline}
}

%\medskip

\noindent
Д\,о\,к\,а\,з\,а\,т\,е\,л\,ь\,с\,т\,в\,о\,.\ Пусть  семейство $\{I'_i, 1\le i\le$\linebreak $\le n\}$~---
независимая копия семейства $\{I_i, 1\le$\linebreak $\le i\le n\}$. Будем
предполагать, что $\{I_i, 1\le$\linebreak $\le i\le n\}$, $\{I'_i, 1\le i\le n\}$ и
$\{r_i, 1\le i\le n\}$~--- независимые семейства. Так как
$0<\sigma^2\le 1/4$, то
$\sigma^2(1-2\sigma^2)\le {1}/{2}$. Поэтому, используя леммы~2.1--2.3 и неравенство Иенсена,
 получаем
\begin{multline}
\P^A\left\{\fr{|\mu-{\E}^A\mu|}{\sqrt{n}\sigma} \ge
\varepsilon\right)\le
\fr{1}{\varepsilon^p}{\E^A}\left|\fr{\mu-{\E}^A\mu}{\sqrt{n}\sigma}\right|^p
\le{}\\[9pt]
{}\le \fr{2^p}{\varepsilon^p\P(A)}{\E}\left|\frac{\mu-{\E}\mu}{\sqrt{n}{\sigma}}\right|^p
\le{}\\[9pt]
{}
\le\fr{2^p}{\varepsilon^p\P(A)}{\E}\left|\frac{\sum_{i=1}^n(I_i-p_i)}{\sqrt{n}{\sigma}}\right|^p \le{}\\[9pt]
{}
\le\fr{2^p}{\varepsilon^p\P(A)}{\E}\left|\frac{\sum_{i=1}^n(I_i-I'_i)}{\sqrt{n}{\sigma}}\right|^p={}\\
{}
=\fr{2^p}{\varepsilon^p\P(A)}{\E_r}{\E}\left|\frac{\sum_{i=1}^nr_i
(I_i-I'_i)}{\sqrt{n}{\sigma}}\right|^p
\le\fr{\sqrt{2}2^p}{\varepsilon^p\P(A)\sigma^p
}\times{}\\
{}\times \left(1+O\left(\fr{1}{p}\right)\right)e^{-p/2}
p^{p/2}{\E}\left(\frac{\sum_{i=1}^n
(I_i-I'_i)^2}{n}\right)^{p/2}\!\!\!\le{}\\
{}
\le\fr{\sqrt{2}2^p}{\varepsilon^p\P(A)\sigma^p
}\left(1+O\left(\fr{1}{p}\right)\right)e^{-{p}/{2}} p^{
{p}/{2}} 2^{{p}/{2}}\sigma^p{\E}\times{}\\
{}\times
\left(1+\sigma^2(1-2\sigma^2)
f_2\left(\fr{p}{4\sigma^2\sqrt {n}}\right)\right)\le{}\\
{}
\le\fr{\sqrt{2}2^p}{\varepsilon^p\P(A)\sigma^p
}\left(1+O\left(\fr{1}{p}\right)\right)e^{-{p}/{2}} 
p^{{p}/{2}}2^{{p}/{2}}\sigma^p{\E}\times{}\\
{}\times
\left(1+\fr{1}{8}
f_2\left(\fr{p}{4\sigma^2\sqrt {n}}\right)\right)\,. 
\label{e2.7faz}
\end{multline}

При $p={\varepsilon^2}/{8}$ из~(\ref{e2.7faz}) следует~(\ref{e2.6faz}).

 Пусть\ \, $ \theta'=m^{-1}( \alpha')$, $
\theta^{\prime\prime}=m^{-1}( \alpha^{\prime\prime})$. Заметим, что $ \theta'\le \theta
\le \theta^{\prime\prime}$ тогда и только тогда, когда $ \alpha'\le \alpha\le
\alpha^{\prime\prime}$.


\medskip

\noindent
\textbf{Лемма 2.5.} {\it Пусть $0< \alpha'< \alpha^{\prime\prime}<m(R)$. Пусть
$\alpha = n'_M/N$, $\theta=m^{-1}(\alpha)$. Существует $N_0\in{\N}$
со свойством: если $n, N\in {\N}$ такие, что $N>N_0$, а $
\alpha'\le\alpha\le \alpha^{\prime\prime}$, то
\begin{equation}
 \P(A_{MN}) > \fr{1}{4\sqrt{C_1}\sqrt{N}}\,.
\label{e2.8faz}
\end{equation}
}

\medskip

\noindent
Д\,о\,к\,а\,з\,а\,т\,е\,л\,ь\,с\,т\,в\,о\,.\ В~силу теоремы~4 из~[4] существует $N_0\in
{\N}$ такое, что если
 $N>N_0$ и $\alpha'\le\alpha\le \alpha^{\prime\prime}$, то
\begin{multline*}
 \sigma(\theta)\sqrt{N} \P(A_{n'_MN})-{}\\
{}-\fr{1}{\sqrt{2\pi}}\exp\left\{-\fr{(n-m(\theta)N)^2}{2\sigma^2(\theta)N}\right\}
>\fr{1}{4}-\fr{1}{ \sqrt{2\pi}}\,.
\end{multline*}
Так как $m(\theta)=\alpha$, имеем
$$
\fr{1}{\sqrt{2\pi}}\,\exp\left\{-\fr{(n-m(\theta)N)^2}{2\sigma^2(\theta)N}\right\}=\fr{1}{\sqrt{2\pi}}\,.
$$
Поэтому
\begin{equation}
\sigma(\theta)\sqrt{N} \P(A_{nMN})\ge \sigma(\theta)\sqrt{N}
\P(A_{n'_MN})
>\fr{1}{4}\,.
\label{e2.9faz}
\end{equation}
Из~(\ref{e2.9faz}) и~(\ref{e1.2faz}) следует~(\ref{e2.8faz}). Лемма доказана.

\medskip

\section{Доказательства теорем}

\noindent
Д\,о\,к\,а\,з\,а\,т\,е\,л\,ь\,с\,т\,в\,о\,\ теоремы~1.  Выберем $N_0\ge 2$ такое, что
справедливо утверждение  леммы~2.5.
 Пусть $t>2\sqrt{6}$. Тогда
${t^2}/{16}-{1}/{2}>1$. Поэтому, используя лемму~2.4, лемму~2.5 и~(\ref{e1.3faz}), получаем
\begin{multline*}
 \sum_{k=N_0+1}^{\infty} \P\left\{\fr{|S_k-{\E}S_k|}{\sqrt{k\ln(k)}\sigma_k} \ge t\right\} ={}\\
 {}=\sum_{k=N_0+1}^{\infty}
\P\left\{\fr{|S_{k}-{\E}S_{k}|} {\sqrt{k}\sigma_k}\ge
\sqrt{\ln(k)}t\right\}\le{}
\\
{}\le\sqrt{2}\sum_{k=N_0+1}^{\infty}4\sqrt{C_1}\sqrt{k}\left(1+O\left(\fr{1}{t^2\ln(k)}\right)
\right)\times{}\\
{}\times \left(1+f_2\left(\fr{t^2\ln(k)}{32C_3
\sqrt{k}}\right)\right)e^{-{(\sqrt{\ln(k)}t)^2}/{16}}\le{}\\
{}
\le 4\sqrt{2}\sqrt{C_1}
\sum_{k=N_0+1}^{\infty}\left(1+O\left(\fr{1}{t^2\ln(k)}\right)
\right)\times{}\\
{}\times
\left(1+f_2\left(\fr{t^2\ln(k)}{32C_3\sqrt{k}}\right)\right)
k^{-t^2/16 +1/2}<\infty\,.
\end{multline*}
Следовательно, для всех $t>2\sqrt{6}$  по лемме Бореля--Кантелли
\begin{equation}
\limsup_{k\to\infty}\fr{|S_{k}-{\E}S_{k}|}{\sqrt{k\ln(k)}\sigma_k}\le t
\label{e3.1faz}
\end{equation} 
почти наверное. Из~(\ref{e3.1faz}) следует~(\ref{e1.4faz}). Теорема доказана.
%

\medskip

\noindent

Д\,о\,к\,а\,з\,а\,т\,е\,л\,ь\,с\,т\,в\,о\,\ теоремы~2. Выберем $N_0\ge$\linebreak $\ge 2$ такое, что
справедливо утверждение  леммы~2.5.
 Пусть $t>2\sqrt{10}$. Тогда
${t^2}/{16}-1-{1}/{2}>1$. Поэтому, используя  леммы~2.4 и~2.5 и~(\ref{e1.3faz}), получаем

\begin{multline*}
\sum\limits_{N=N_0+1}^{\infty}\sum\limits_{N \alpha'\le n\le
\alpha^{\prime\prime}N}
\!\!\!\!\!\P\left\{\fr{|S_{nN}-{\E}S_{nN}|}{\sqrt{N\ln(N)}\sigma_{nN}}\ge
t\right\} = {}\\
{}=
\sum\limits_{N=N_0+1}^{\infty}\sum\limits_{N \alpha'\le
n\le \alpha^{\prime\prime}N}
\!\!\!\!\!\P\left\{\fr{|S_{nN}-{\E}S_{nN}|}{\sqrt{N}\sigma_{nN}}\ge\right.{}\\
\left.{}\ge
 \sqrt{\ln(N)}t
\vphantom{\fr{|S_{nN}-{\E}S_{nN}|}{\sqrt{N}\sigma_{nN}}}
\right\}\le
\sum_{N=N_0+1}^{\infty}\sum\limits_{N \alpha'\le n\le
\alpha^{\prime\prime}N}
\!\!\!\!\!4\sqrt{2}\sqrt{C_1}\sqrt{N}\times{}
\end{multline*}
\begin{multline*}
{}\times \left(1+O\left(\fr{1}{\varepsilon^2\ln(N)}\right)
\right)\times{}\\[2pt]
{}\times
\left(1+f_2\left(\fr{\varepsilon^2\ln{N}}{32C_3\sqrt{N}}\right)\right)e^{-{(\sqrt{\ln(N)}t)^2}/616}\le{}\\[2pt]
{}
\le 4\sqrt{2}\sqrt{C_1}\! \!\sum_{N=N_0+1}^{\infty}\!\!\!( \alpha^{\prime\prime}-
\alpha')N\left(1+O\left(\fr{1}{\varepsilon^2\ln(N)}\right) \right)\times{}\\[2pt]
{}\times
\left(1+f_2\left(\fr{\varepsilon^2\ln{N}}{32C_3\sqrt{N}}\right)\right)
N^{-t^2/16+{1}/2}<\infty\,.
 \end{multline*}
 
 \noindent
Следовательно, для всех $t>2\sqrt{10}$  по лемме Бо\-ре\-ля--Кан\-тел\-ли
\begin{equation}
\limsup_{n,N\to\infty\,,\,\alpha'<\alpha<
\alpha^{\prime\prime}}\fr{|S_{nN}-{\E}S_{nN}|}{\sqrt{N\ln(N)\sigma_{nN}}}\le t
\label{e3.2faz}
\end{equation}
почти наверное. Из~(\ref{e3.2faz}) следует~(\ref{e1.5faz}). Теорема доказана.


{\small\frenchspacing
{%\baselineskip=10.8pt
\addcontentsline{toc}{section}{Литература}
\begin{thebibliography}{9}

\bibitem{1faz}
\Au{Питерсон У., Уэлдон Э.} 
Коды, исправляющие ошибки.~--- М.: Мир, 1976. 596~с.

\bibitem{2faz}
\Au{Колчин В.\,Ф.} 
Один класс предельных теорем для условных распределений~// Литовск. матем. сб., 1968. Т.~8. №\,1. С.~53--63.

\bibitem{3faz}
\Au{Колчни В.\,Ф.} Случайные графы.~--- М.: Физматгиз, 2000.

\bibitem{4faz}
\Au{Колчин А.\,В.} 
Предельные теоремы для обобщенной схемы размещения~// Дискрет. матем., 2003. Т.~15. №\,4. С.~143--157.

\bibitem{5faz}
\Au{Колчин А.\,В., Колчин В.\,Ф.} 
О переходе распределений сумм
независимых одинаково распределенных случайных величин с одной
решетки на другую в обобщенной схеме размещения~// Дискрет. матем.,
2006. Т.~18. №\,4. С.~113--127.

\bibitem{6faz}
\Au{Колчин А.\,В., Колчин В.\,Ф.} Переход с одной решетки на
другую распределений сумм случайных величин, встречающихся в
обобщенной схеме размещения~// Дискрет. матем., 2007. Т.~19. №\,3. С.~15--21.

\bibitem{7faz}
\Au{M$\acute{\mbox{o}}$ri T.} 
Sharp inequalities between centered moments~// 
J. Inequalities in Pure and Applied Mathematics, 2009.
Vol.~10. Is.\,4. Art.~99.

\bibitem{8faz}
\Au{Haagerup U.} 
The best constant in Khinchin inequality~// Stud. Math., 1982. Vol.~70. P.~231--283.

\label{end\stat}

\bibitem{9faz}
\Au{Nemes G.} 
New asymptotic expansion for the  $\Gamma(z)$ function~// Stan's Library, 2007. Vol.~II. P.~31.
 \end{thebibliography}
}
}


\end{multicols}