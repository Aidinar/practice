\renewcommand{\figurename}{\protect\bf Figure}
\renewcommand{\tablename}{\protect\bf Table}
\renewcommand{\bibname}{\protect\rmfamily References}

\def\stat{fren}

\def\u{\overline u}

\def\tit{ESTIMATION OF SELF-HEALING TIME FOR DIGITAL
SYSTEMS UNDER TRANSIENT FAULTS$^*$}

\def\titkol{Estimation of self-healing time for digital
systems under transient faults}

\def\autkol{S.\,L.~Frenkel and A.\,V.~Pechinkin}
\def\aut{S.\,L.~Frenkel$^1$ and A.\,V.~Pechinkin$^2$}

\titel{\tit}{\aut}{\autkol}{\titkol}

{\renewcommand{\thefootnote}{\fnsymbol{footnote}}\footnotetext[1]
{The
work was performed within the Russian Academy of
Sciences Presidium basic research program~3
``Fundamental problems of system programming.''}}

\renewcommand{\thefootnote}{\arabic{footnote}}
\footnotetext[1]{Institute
of Informatics Problems of the Russian Academy of Sciences,
Slf-ipiran@mtu-net.ru}
\footnotetext[2]{Institute
of Informatics Problems of the Russian Academy of Sciences,
apechinkin@ipiran.ru}

\vspace*{8pt}

\Abste{This paper suggests a new approach to self-healing property  prediction for digital systems.  
Self-healing refers to the system's ability to continue operating properly in the case of the 
failure of some of its components. This phenomenon is very considerable aspect of high-reliable 
systems design. The self-healing time characteristics are analyzed during design process, and 
the computation of probability distribution function of self-healing time needs for fair prediction 
of real time systems reliability. 
This paper considers the possible ways of estimation of time to self-healing under transient faults using 
a Markov model of  a design behavior for reliability analysis of digital systems with some fault-tolerant 
properties, modeled by the well-known Finite State Machine formalism. 
}

\vspace*{4pt}

\KWE{fault-tolerant computer; self-healing
fault-tolerance; transient faults; finite state machine; Markov chains}

\vspace*{8pt}

       \vskip 14pt plus 9pt minus 6pt

      \thispagestyle{headings}

      \begin{multicols}{2}

      \label{st\stat}



\section{Introduction}

\noindent
Designers of nanotechnology-based digital systems
for safety-critical applications need to take into
account that the systems can be affected by radiation-induced
faults, e.\,g,\ Single Event Upsets
(SEU)~[1].
Thus, computing systems for safety-critical
applications must be \textit{fault-tolerant} to be able to
continue properly functioning despite these transient
failures of their hardware or software~[2--4].

{\it Self-healing} refers to the system's ability to continue
operating properly on the event of the failure of some of its
components, that is, as its ability to maintain operation over a
period of time~$t$. In other words, a system is ``self-healing'' if
it is capable to recover from errors while remaining valid
system outputs. Moreover, systems designed to be self-healing must be
able to heal themselves at runtime in response to changing
environmental or operational status, in order to fix errors
in its components without any external interaction~\cite{4fr}.

In spite of plethora of the self-healing specific mechanisms, one
of the most usable model of the systems of interest is the FSM.
Therefore,  the self-healing in automata models of
systems used at early design stages, is considered. For example, a concept of
a partially monotonic FSM, where the transitions are computed by
partially monotonic Boolean functions, is used to provide
self-healing properties~\cite{6fr}. In particular, if  a
self-checking digital circuit design is considered, the different properties of
logical functions may provide self-healing properties of the
circuit~\cite{6fr}. The architecture that supports the self-healing
property of the FSM is a well-known self-checking architecture~\cite{6fr} that uses output of self-checking checker.
Thus, further,  only FSM representations of digital
systems will be considered.

Since, in general, both input data and faults appear
randomly, the time before a system healing is some
random values.

This paper describes a probabilistic approach to
analysis of self-healing properties of FSM modeled
systems under some transient faults.
The model of system that allows to compare correct
behavior with erroneous one is a Markov chain (MC)
defined on direct product of state spaces of two FSMs,
describing behavior of both fault-free and faulty
design~\cite{7fr, 8fr}.
The goal of this modeling is to estimate the time
to return to the correct behavior after hitting in
some erroneous one.

%2
\section{Some Concepts of Fault-Tolerant Design}

\noindent
Let us outline some principal definitions of the
fault-tolerant systems research area.
A {\it fault} is a physical cause of incorrect behavior,
e.\,g., a defect in a memory cell.
The most popular fault model in the area of digital
system testing is so-called stuck-at-zero (a variable~$u$
has constant logical zero which designated as
$u\equiv 0$), and stuck-at-one (correspondingly,
$u\equiv 1$)~\cite{2fr}.

The faults may be both \textit{permanent} (that remain in
existence indefinitely if no corrective action is
taken) and \textit{transient} ones (that appear and disappear
quickly).

An \textit{error} is an undesired state or condition in
a component of a target system, understood as a
discrepancy between an observed or measured value
or condition and a specified theoretically correct
value or condition.
Error is a fault consequence.
Faults may or may not cause one or few errors.
Errors induce \textit{failures}.
A \textit{failure} occurs when a system is no longer able to
satisfy its specification, e.\,g., an incorrect word
is formed on its output.

Correspondingly, one should differ between
manifestation latency for the faults and the
errors.
It is considered that designers have an \textit{oracle} that check
the correctness of its output (result of
computations).

The aim of a fault-tolerant design is to avoid
manifestation of the fault/error at the system
designed output in order to prevent the failure
behavior.
Note that in accordance with~\cite{6fr}, when a transient
fault occurs, a system may transit from a fault-free
behavior (``mode'') either to the erroneous mode
(the output is erroneous one) or in a mode where
the faulty behavior will be ``silent,'' that is, an
inner state will be incorrect while its output stays
in a correct mode.
If the system is able to return from the fault-free
mode after its functioning in the silent mode, this
is, obviously, the self-healing.
In other words, in this case, future of a system
(either it will ``recover'' or ``die'') depends of its
behavior during several clocks after moving to the
silent mode.
This parameter of number of clocks looks promising
and motivating for the self-healing characterization.

%3
\section{Some Architectures Which Enable Self-Healing}

\noindent
Let us outline some architectures of digital systems that can
provide their self-healing under some transient faults. There are
two main types of the faults that are being considered, the SEU
and the Single Event Transient (SET). The SEU occurs when radiation affects the transistors, for example,
that are part of the look up table logic of the FPGA (Field-Programmable Gate Array). Radiation
can hit areas on the device (FPGA, in particular) and causes an
incorrect bit value at the input or output of some blocks. The
SET affects current processing of data in the
circuit. Some authors remark that the SET is
less damaging than the SEU mainly because it does
not affect the hardware makeup of the current FPGA design~\cite{5fr}.
A~transient fault induced by the SEU may or may not be latched by a
storage cell, but in case of the fault occurrence, a correct
operation of the corrupted module can be restored and the current
state of the circuit can be reset. It can be achieved either due
to the monotonic properties of Boolean function describing the
transitions of FSM representing the given system~\cite{5fr} or a reconfiguration~\cite{2fr}.
Recall that a system of logical functions~$\psi$
is partially monotonic in~$u'$ input variables if for any pair of
Boolean $m$-tuples $A$, $B$, the condition $\psi(A)\le \psi(B)$ is
satisfied for $A\le B$~\cite{6fr}.

Another example of a system with self-healing, based on some
reconfigurations of a system under some errors provoked by a
fault has been considered in~\cite{5fr}. As it mentioned above, this
reconfiguration architecture functioning can be also described as
a FSM~\cite{10fr}. Partial Reconfiguration and Triple Module Redundancy (TMR)
are used together to create a true self-healing system design~\cite{5fr}.
In order to do this, the full TMR implementation must be extended
to provide additional status outputs to the partial
reconfiguration controller.

%4
\section{Problem Definition and Model Description}

\noindent
Let $A$ be an automaton (FSM) modeling an electronic
design implemented as a device (e.\,g., FPGA).
It is considered that this design should operate in the
presence of some radiation which hits at an physical
area of the device causing an incorrect bit
value to the input or output of   its modules
but does not affect immediately any primary outputs.

Let us assume that a transient fault during the clock~$t_0$ has
changed a next state~$x_0$, where fault-free FSM must
transit under an input vector~$\vec u$, for the state~$x_1$.
We say that in this moment,  the change of a
{\it trajectory} of the FSM transitions, corresponding
to its transition table, took a place.
All future transitions will be performed according to
the FSM transition functions (in accordance with
Table~1 in our example).
If at one of the steps after this clock ($t_0$) the
primary output is changed relatively to the
fault-free outputs and given transitions have still
not come back to the fault-free trajectory, then
either alarm or correction mechanism will be started,
and the FSM will be stopped or moved to a
restoration mode.
There is no any self-healing in this case.



Otherwise, if the FSM under the transient fault
will return to the normal (fault-free) trajectory
before the output is changed, it means that the self-healing took place (at least, before the next
external induced transient fault).

It is considered that a self-healing system can
recover from some transient faults within a finite
time and can provide that no further faults occur before
the system is healing again.
On the other hand, systems that are not self-healing
might function ate in incorrect states arbitrary long time, even
if no further faults occur.

\pagebreak

\noindent
\begin{center} %tabl1
%\vspace*{4pt}
{{\tablename~1}\ \ \small{Transition functions of FSM}}
\vspace*{2ex}

{\small
\tabcolsep=9pt
\begin{tabular}{cccc}
\hline
 $a_t$  & $a_s$  & $\vec{u}(a_t,a_s)$ & $\vec v(a_t,a_s)$     \\
\hline
 $a_1$  &  $a_1$  & $\u_1$        & $v_2v_4$         \\
%\hline
            & $a_1$  & $u_1\u_2$     & $v_2v_3$         \\
%\hline
            &  $a_2$  & $u_1u_2$     & $v_2v_4$         \\
%\hline
\hline
 $a_2$  & $a_1$  & $\u_2$        & $v_2v_4$         \\
%\hline
            &  $a_1$  &\ $u_2\u_3$     & $v_1v_4$         \\
%\hline
            &  $a_2$  & $u_1u_3$     & $v_2v_4$         \\
\hline
\end{tabular}
}
\end{center}

\bigskip


The authors' goal is to  express  the  probability
distribution function of the  time which can be spent
to  self-healing  under transient faults  using the
Markov chains (FSMs) product model~\cite{7fr, 8fr}.
Let us consider some examples of the self-healing
phenomenon.

Let us describe an FSM in Table~1 where columns~$a_t$
and~$a_s$  are the current and the next states;
$\vec u$ and $\vec v$ are the input and output signals in cubic
form, that is, it has free components with all
possible combinations of ``0'' and ``1''
(in the 3-input FSM in Table~1, the input variables
absent in the corresponding cells may take any
values).


Let the initial state of the FSM be~$a_1$, and let an
input sequence be $u_1u_2\to u_1u_2u_3$, which generates
the transitions $a_1\to a_2(v_2v_4)\to a_2(v_2v_4)$.
Let the variable~$u_2$ was corrupted by an external
fault effect (e.\,g., SEU) so that the sequence
$u_1\u_2\to u_1u_2u_3$ appeared instead of previous
one, which generate wrong transition
$a_1\to a_1(v_2v_3)\to a_2(v_2v_4)$, that is, the FSM
transmits to the state~$a_2$ instead~$a_1$.
However, because of discrepancy of the outputs in
two cases (that is, the FSM either under the transient
error or not), it will be  either alarm or correction
mechanism  started, and the FSM will be either
stopped or moved in a restoration mode, that is, the
self-healing does not take place.
Now, let the input sequence be
$\u_1u_2\to \u_1u_2u_3\to u_1u_2u_3$ generating the
transitions
$a_1\to a_1(v_2v_4)\to a_1(v_2v_4)\to a_2(v_2v_4)$,
which is corrupted as
$u_1u_2\to \u_1u_2u_3\to u_1u_2u_3$ with
transitions
$a_1\to a_2(v_2v_4)\to a_2(v_2v_4)\to a_2(v_2v_4)$.
Since the outputs are equal, the FSM under this
transient fault comes to the right state~$a_2$
if the external fault effect be disappeared to
the next clock.

%5
\section{Self-Healing Model Based on~Finite State Machine}

\noindent
Let an FSM, subject to external radiation mentioned
above during a clock (e.\,g., hit of some radioactive
particles), changes its correct trajectory to an
incorrect one, but it returns to the correct trajectory
after a number of clocks when the radiation effect is
stopped.
Following~\cite{7fr}, this phenomenon will be modeled by an
MC defined on the transition space of
two FSMs product: one of these FSM is the original one
(called as ``fault-free''), but another (called as
``faulty'') has the transition table of original
FSM but its trajectory is corrupted during the
clock-under-noise.

Let FSM be a Mealy machine, with the state set
$S=$\linebreak $=\{a_1,\ldots,a_n\}$,
the input set $U=\{\vec u\}=\{u_1,\ldots, u_m\}$
and the output set $V=\{\vec v\}=\{v_1,\ldots, v_k\}$.
Functions~$\delta$ and~$\lambda$ are the multiple-output Boolean
functions which are a relation between the ({\it input state,\,
present state})\linebreak pairs and the next states ($\delta$),
and between the (\textit{input state, present state}) pairs
and the output states ($\lambda$).
Let the input words of the FSM be a randomly generated
input vectors.
Obviously that the probability of the self-healing property
fulfillment depends on the distribution of input vectors
(for example, see Table~1).
Let us consider self-healing time
computation by the FSM product model.

Let $\{X_t,\ \ t\ge0\}$ and $\{Y_t,\ \  t\ge0\}$ are the MCs
describing the target behavior of a fault-free system  (in
accordance with the previous section), that is, functioning without
effect of any transient faults ($X_t$), and in the presence of
some faults ($Y_t$),\ \ $S$ is the set of states of the MCs.

Let $Z_t=\{(X_t,Y_t), \ t\ge0\}$ be an MC corresponding to
behavior of the MCs pairs that is an MC with space
$S^2=S\times S$ of pairs $(a_i,a_j)$, $a_i,a_j\in S$.

Let the states~$S$ are integers.
Fixing  the states numeration in  some way,
for example, as $s_1=(a_1,a_1)$,
where the ``ones'' are the numbers of states of
$X_t,Y_t$,  $s_2=(a_1,a_2),\ldots,$  $s_n=(a_1,a_n)$,
$s_{n+1}=(a_2,a_1)$, $s_{n^2}=(a_n,a_n)$,
the vector of the state transition probabilities of
the MC after $t$ steps may be defined as
$$
\vec p^{\,*}(t)=(p_{1}^*(t),\ldots,p_{n^2}^*(t))\,,
\enskip t\ge 0\,,
$$
where the state probability vector of the MC~$Z_t$
after $t$~clocks.
 This vector can be computed using the state
probabilities of~$X_t$ and~$Y_t$,
\begin{align*}
\vec p(t)&=(p_{1}(t),\ldots,p_{n}(t))\,,
\quad  t\ge 0\,;
\\
\vec p^{\,F}(t)&=(p_{1}^F(t),\ldots,p_{n}^F(t))\,,
\quad t\ge 0\,,
\end{align*}
with initial  states probabilities of the MCs,
where index~``$F$'' corresponds to the  FSM under
a transient fault (``faulty'' FSM):
$$
p_{i}(0)
=
\begin{cases}
1\,,      &a_i=X_0\,,   \\
0\,,      &a_i\ne X_0\,,
\end{cases}
\ \ i=1,\ldots ,n\,;
$$
$$
p^F_{j}(0)
=
\begin{cases}
1\,,      &a_j=Y_0\,,   \\
0\,,      &a_j\ne Y_0\,,
\end{cases}
\ \ j=1,\ldots ,n\,.
$$
Then,  the event of coincidence of the
trajectories of~$X_t$ and~$Y_t$ (self-healing)
may be denoted as hit
of the MC~$Z_t$ in a subset $D \subseteq S^2$ of all
states $(a_i,a_i)$,\ \ $i=1,\ldots,n$, with the same
right and left elements.

In general, the self-healing time  can be expressed as
\begin{equation}
T =
\min\{t:\ X_t=Y_t\}=\min\{t:\ Z_t\in D\}\,.
\label{e1fr}
\end{equation}
Let us characterize the self-healing time ~$T$ by
a functional~$Q(T)$ of the time.
It can be interpreted as a speed of returning to
a correct mode.

Equation~(\ref{e1fr}) can be rewritten in different ways in
dependence of the functional that will be used for the time to healing characterization (and
it can be interpreted as a ``loss function'').

Most simple type of the~$Q_1(t)$ is the mean time
to healing, that is:
$$
Q = Q_1 = {\mathbf E}(T)\,.
$$

More interesting measure is the probability~$Q_2(t_0)$
that time of returning of the FSM to its correct
trajectory
after fault effect disappearance  is more than~$t_0$.
Let us define:
$$
Q = Q_2 = Q_2(t_0) = {\mathbf Pr}\{T\ge t_0\}\,.
$$
Note that relationship between~$Q_1$ and~$Q_2$ is
$$
Q_1 =
\sum_{t_0=1}^\infty {\mathbf Pr}\{T\ge t_0\}
=
\sum_{t_0=1}^\infty Q_2(t_0)\,.
$$

Both these times to self-healing measures are computed
under assumptions that the initial states~$X_0$ and~$Y_0$ of both MCs are known.
Therefore, the function~$Q_2$ can be expressed as
$$
Q_2
=
1 - {\mathbf Pr}\{Z_{t_0-1}\in D\}
=
1 - \sum_{z\in D} p_{z}^*(t_0-1)\,.
$$
Also, it can be easily shown   that
$$
Q_2
\ge
\max\limits_V
\left|{\mathbf Pr}\{Y_{t_0-1}\in V\}
-
{\mathbf Pr}\{X_{t_0-1}\in V\}\right|
$$
where $V\subseteq S$.

Note that the functionals~$Q_1$ and $Q_2$ have been built
without explicit consideration of equality of  output
values of both FSMs.
Below,  more general case of FSM under
transient fault self-healing characterization will be considered, namely,
the probability distribution function of the time to
return to the FSM fault-free behavior before its output
mismatching with its copy, which functionates without any
transient faults.

%6
\section{Computation of   Self-Healing Time Probability
Distribution Function}

\noindent
Below, the time to self-healing probability
distribution function will be computed as a distribution of  number
of steps  to absorbing state of~$Z_t$, corresponding
to the event ``the automaton under transient fault
has returned to the fault-free trajectory before
its output mismatching.''

Let us define all specific states of the MC~$Z_t$ needed
to find out accurately the  self-healing time. Again, let  the
states set of both MCs $S_1$ and~$S_2$,\  $|S_1|=|S_2|= n$, are the
same and their initial states~$X_0 $ and~$Y_0$,\  $Y_0 \ne X_0$,
are known. Let us define the set of $Z_t$ states as $S^* =
\{(i,j),\ \ i,j = 1,\ldots,n,\ \ j \ne i\} \cup A_0\cup A_1$
where:
\begin{itemize}
\item
$(i,j)$ means that fault-free automaton is in the state~$i$,
but the automaton under a transient fault is in the state~$j$,
$j\ne i$;
\item
$A_0$ is an absorbing state of~$Z_t$, corresponding
to the event ``the automaton under transient fault has returned
to the fault-free trajectory before its output would be corrupted
by the transient fault effect;'' and
\item
$A_1$ is another absorbing state corresponding to the event
``faulty output appeared  before the trajectory restoration.''
\end{itemize}

The number of the~$Z_t$ states is $n(n-1)+2$.

Let us consider the way of MC~$Z_t$, transition
probability matrix~$P^*$ computation.
Let~$Z_t$  be in the state $(i_1,j_1)$ when the input signal
$\vec u$ is applied which transfers the fault-free automat
from the state~$i_1$ to the state~$i_2$, whereas the
automaton under a transient fault
will transfer to the state~$j_2$, and the output vectors
of the automata are~$\vec v_0$ and~$\vec v_1$, correspondingly.
Then the following situations are possible:
\begin{itemize}
\item $Z_t$ is in the state $A_1$ if $\vec v_0 \ne \vec v_1$;
\item
$Z_t$ is in the state $A_0$ if $i_2=j_2$ and $\vec v_0 =\vec v_1$; and
\item
$Z_t$ transits to the state $(i_2,j_2)$ if the outputs
coincide ($\vec v_0=\vec v_1$), but the states do not ($i_2\ne j_2$).
\end{itemize}

The transition probability matrix of the MC~$Z_t$ is computed by the known probabilities of input
vectors and transition tables of the FSM~\cite{7fr}, considering
the signals with the same input probabilities
distributions $\mathrm{Pr}\{u_i=1\}$, $i=1,\ldots,m$.

Let us denote:
\begin{itemize}
\item $p^*_{(1)}(t)$ is the probability that a faulty output behavior
of the automaton can appear right up to the $t$th step before the
automaton behavior will be restored;
\item
$p^*_{(0)}(t)$ is the probability that a faulty behavior of the
automaton can appear right up to the $t$th step and the output
both automata will coincide by this moment; and
\item
$p^*_{i,j}(t)$ is the probability that the MC $Z_t$ will get
in the state $(i,j)$\  ($j \ne i$) thereby by the moment~$t$, but the outputs will coincide.
\end{itemize}

Then, let us express the probability that~$Z_t$ will get
in the state~$(i,j)$.

Initial state $\vec p^{\,*}(0)$ is determined by the distribution
of initial states of both automata.
If the fault-free automaton is at the initial moment 0 in the
state $i_0=X_0$,whereas the faulty automaton will be in state
$j_0=Y_0$,\  $j_0 \ne i_0$, then $p^*_{i_0,j_0}(0) = 1$
and other components of the vectors are zero.

Matrix~$P^*$ is computed using input probabilities of Boolean
functions $u_l=u_l(a_t,a_s)$, describing  transitions from
the current state~$a_t$ to the next state~$a_s$, takes the
Boolean value.
This computation  is performed under assumption that input
variables of the FSM are independent random vectors from one
time step to another and that their probability distribution
is fixed so that, at any given time step, an input takes
place with probability $\mathrm{Pr}\{u_i=1\}=p_k$,\  $i=1,\ldots,m$.
For example, the FSM of Table~1 stays in the state~$a_2$ with
probability $p_1 p_3$ since this is a Boolean
conjunction of the Boolean variables~$u_1$ and~$u_3$.

The distribution $Z_t$ in the moment~$t$
\begin{equation}
\vec p^{\,*}(t)
=
\vec p^{\,*}(t-1) P^* = \vec p^{\,*}(0)(P^*)^t             \label{e2fr}
\end{equation}
and the vector component, corresponding to the transition
to an absorbing state~$A_0$, represents the probability
distribution of the self-healing time.

Let us denote:
\begin{itemize}
\item $\pi_0$ is the probability that exists an instant~$t$ when
outputs of both (faulty and fault-free automata) are
different (the vector $\vec v_0 \ne \vec v_1$);
\item
$\pi_1$ is the probability of the automaton ``healing''
that is the return to the transitions trajectory of
normal functioning after an external fault effect will
be disappeared and $\vec v_0=\vec v_1$ for any~$t$; and
\item
$\pi$ is the probability that the automaton never more
come back to the normal transitions trajectory after
the external fault effect will disappear.
\end{itemize}

The relationships between these probabilities are:
$$
\pi_0 =
\lim_{t\to\infty} p_{(0)}(t)\,;
\quad
\pi_1 =
\lim_{t\to\infty} p_{(1)}(t)\,.
$$
Therefore,
\begin{equation}
\pi=1-\pi_0-\pi_1\,.
\label{e3fr}
\end{equation}

Let us demonstrate an example of the time to probabilities
computation.

\medskip

\noindent
\textbf{Example.}
Let us consider the automaton with two states in Table~1.

Let the transient fault be the erroneous transition to the
state~$a_2$ instead of~$a_1$.

Then corresponding MC~$Z_t$ that corresponds
to the MC describing the product of two automata has
four states: (1,2), (2,1), $A_0$, and $A_1$ where  state~(1,2)
describes the case when fault-free and SEU-affected
automata are correspondingly in the states~$a_1$ and~$a_2$,
(2,1)  means that they  are in states~$a_2$ and~$a_1$, and~$A_0$ and
$A_1$ are the  absorbing states mentioned above.
\columnbreak

In order to define the MC~$Z_t$ completely,
let us compute its state transition probabilities
(see Table~1).


First, the probabilities that the~$Z_t$ are in the
absorbing states~$A_0$ and~$A_1$ are equal to~1.

From state (1,2) to~$A_0$, it is possible to get only
if the input vector $\u_1u_2u_3$ is applied;
thereby, the probability of this transition is
$q_1p_2p_3$
where $p_l=\mathrm{Pr}\{x_l =1\}$ and $q_l=1-p_l$,
$l=1,2,3$.
Then both automata transit to the state~$a_2$ and
their outputs equal to~$v_2v_4$.

The only way to stay in the state~(1,2) is to apply
the input vector $u_1u_2u_3$ (with probability
$p_1p_2p_3$).
Then the fault-free automaton remains in the state~$a_1$,
whereas the automaton subjected the SEU hit will be
in the state~$a_2$, and both inputs vectors are~$v_2v_4$.

The transition from~(1,2) to~(2,1) is possible with
probability $q_1q_2$ if the signal $\u_1\u_2$ will
be applied, whereas the outputs are~$v_2v_4$.

Finally, the transition from~(1,2) to $A_1$ is possible
under input signals $u_1\u_2$ (with probability~$p_1q_2$),
$u_1u_2\u_3$ (with probability~$p_1p_2q_3$), and
$\u_1u_2\u_3$(with probability~$q_1p_2q_3$).

Indeed, under all of possible combinations of input
signals, there is a discrepancy between outputs vectors
of both automata, which means the state~$A_1$.
All possible transitions (determined by corresponding
input vectors) and there probabilities are given in
the row ``(1,2)'' of Table~2.

\bigskip

\begin{center} %tabl2
%\vspace*{-6pt}
{{\tablename~2}\ \ \small{Possible transitions and probabilities of input vectors}}
\vspace*{2ex}

{\small
\tabcolsep=8.6pt
\begin{tabular}{ccccc}
\hline
       &   $A_0$  &    (1,2)   &   (2,1)   &   $A_1$    \\
\hline
$A_0$  &     1    &      0   &     0    &     0     \\
%\hline
(1,2)  & $q_1p_2p_3$ & $p_1p_2p_3$ & $q_1q_2$ & $p_1q_2+p_2q_3$      \\
%\hline
(2,1)  & $q_1p_2p_3$ & $q_1q_2$ & $p_1p_2p_3$ & $p_1q_2+p_2q_3$      \\
%\hline
$A_1$  &    0      &    0      &    0      &    1         \\
\hline
\end{tabular}
}
\end{center}

\vspace*{6pt}

\bigskip
\setcounter{table}{2}


Analogously can be computed all transition probabilities
from state~(2,1).

Therefore, the probability transition matrix~$P^*$ of
the MC~$Z_t$ is:
$$
P^*
=
\begin{pmatrix}
1         &   0       &   0       &   0              \\
q_1p_2p_3 & p_1p_2p_3 & q_1q_2    &  p_1q_2+p_2q_3   \\
q_1p_2p_3 & q_1q_2    & p_1p_2p_3 &  p_1q_2+p_2q_3   \\
0         &   0       & 0         &   1
\end{pmatrix}
\,.
$$

Note that in the matrix~$P^*$, the following
transition probabilities are equal:
\begin{itemize}
\item from the states $(i,j)$ and $(j,i)$ to~$A_0$;
\item
from the states $(i,j)$ and $(j,i)$ to~$A_1$; and
\item
from the states $(i,j)$ to $(i,j)$
and from $(j,i)$ to~$(j,i)$.
\end{itemize}

%\noindent


\begin{table*}\small %tabl3
\begin{center}
\Caption{Possibilities of different states
}
\vspace*{2ex}

\begin{tabular}{lccccccc}
\hline
       & $t=0$ & $t=1$ & $t=2$ & $t=3$ & $t=4$ & $\ldots$ & $t=28$  \\
\hline
$p_{(0)}(t)$
       & 0,000000 & 0,080000 & 0,120000 & 0,140000 & 0,150000
                      & $\ldots$ & 0,160000 \\
%\hline
$p_{(1)}(t)$
       & 0,000000 & 0,420000 & 0,630000 & 0,735000 & 0,787500
                      & $\ldots$ & 0,840000 \\
%\hline
$\hat p(t)$
       & 1,000000 & 0,500000 & 0,250000 & 0,125000 & 0,062500
                      & $\ldots$ & 0,000000 \\
\hline
\end{tabular}
\end{center}
\end{table*}

Let suppose the following probabilities of Boolean~1
in the input vector bits:
$$
p_1=0{{,}}2\,;\quad
p_2=0{,}4\,;\quad
p_3=0{,}25
$$
and the initial distribution of the MC:
$$
\vec p^{\,*}(0) = (0,1,0,0)\,.
$$
Then
$$
P^*
=
\begin{pmatrix}
1      &   0    &   0    &   0       \\
0{,}08 & 0{,}02 & 0{,}48 &  0{,}42   \\
0{,}08 & 0{,}48 & 0{,}02 &  0{,}42   \\
0      &   0    & 0      &   1
\end{pmatrix}
$$
and
\begin{multline*}
\vec p^{\,*}(1)
=
\vec p^{\,*}(0) P^*
={}\\
{}=
(0,1,0,0)
\begin{pmatrix}
1      &   0    &   0    &   0       \\
0{,}08 & 0{,}02 & 0{,}48 &  0{,}42   \\
0{,}08 & 0{,}48 & 0{,}02 &  0{,}42   \\
0      &   0    & 0      &   1
\end{pmatrix}={}\\
{}= (0{,}08,\,0{,}02,\,0{,}48,\,0{,}42)\,;
\end{multline*}

\vspace*{-6pt}

\noindent
\begin{multline*}
\vec p^{\,*}(2)
=
\vec p^{\,*}(1) P^*
={}\\
{}=
(0{,}1200,\,0{,}2308,\,0{,}0192,\,0{,}6300)\,.
\end{multline*}

The  first row of Table~3 contains  probability
$p_{(0)}(t)$ that the automaton (see Table~1) will restore the
correct behavior before the fault effect will appear
at its output variables.
The second row contains the probability $p_{(1)}(t)$
where the output variables are corrupted by the
$t$th step and the third row contains the probability
$\hat p(t)=1-p_{(0)}(t)-p_{(1)}(t)$
that the state of the automaton was changed during a clock
(before the moment~$t$) by a transient fault,
but the output will remain correct.
{\looseness=1

}


Note that for this example, both conditional expected times to reach the state~$A_0$
and conditional expected time to reach the state~$A_1$  are equal to~2.

%7
\section{Concluding Remarks}

\noindent
In this paper, the self-healing phenomenon in
digital systems in terms of product of MC
describing faulty and fault-free FSMs under independent
random input binary signals was  analyzed.
Since we deal with the FSM models, it is possible to use
this self-healing probability in analysis of  reliability
of digital and computer systems at FSM level of their
modeling that is in rather early design stages. In other words,
these models can be used in  reliability analysis of a
target system in hierarchical system design.
They can be a base for a tool of fault-tolerant
systems design, dealing with such fault tolerance aspects
as fault detection latency (for permanent faults)~\cite{8fr}
and self-healing in presence of some transient faults
(Soft Upset Errors, in particular).
Now, there are some preliminary results of using this
approach to self-healing probability estimation for a
self-checking design~\cite{6fr}.
{\looseness=1

}


The authors' model (Eqs.~(1)--(3)) does not take into account probability
distribution of the transient faults during FSM functioning.
Therefore, this aspect can be indicated as a very important issue
of the future work.

{\small\frenchspacing
{%\baselineskip=10.8pt
%\addcontentsline{toc}{section}{Литература}
\begin{thebibliography}{99}

\bibitem{1fr}
\Au{Baumann R.}
Soft errors in advanced computer systems~//
IEEE Design and Test, May-June, 2005. P.~258--266.

% 2.
\bibitem{2fr}
\Au{Lala P.}
Self-checking and fault-tolerant digital  design.~---
Morgan Kaufmann Publs., 2000.

% 3.
\bibitem{3fr}
\Au{Lala P.\,K., Kumar B.\,K.}
An Architecture for self-healing digital systems~//
J.\ Electronic Testing: Theory and Applications, 2003.
Vol.~19. P.~523--535.

% 4.
\bibitem{9fr}
\Au{Hawthorne M., Perry D.}
Architectural styles for adaptable self-healing dependable systems~//
ICSEТ05 Proceedings, May 15--21, 2005. St.\ Louis, Missouri, USA.

% 5.
\bibitem{4fr}
\Au{Park J., Jung J., Piao Sh., Lee E.}
Self-healing mechanism for reliable computing~//
Int.\ J.\ Multimedia Ubiquitous Engineering,
2008. Vol.~3. No.\,1. P.~75--76.

% 6.
\bibitem{6fr}
\Au{Levin I., Matrosova A., Ostanin S.}
Survivable self-checking sequential circuits~//
DFTТ01 Proceedings, 2001. P.~395.

% 7.
\bibitem{7fr}
{\it Shedletsky J., McCluskey E.}
The error latency of fault in a sequential digital circuit~//
IEEE Transaction on Computers, 1976. Vol.~25, No.\,6. P.~655--659.

% 8.
\bibitem{8fr}
{\it Frenkel S., Pechinkin A., Chaplygin V., Levin I.}
A mathematical tool for support of fault-tolerant
embedded systems design~//
ERCIM/DECOS Dependable Smart Systems: Research,
Industrial Applications, Standardization,
Certification and Education. Workshop on
``Dependable Embedded Systems.'' L$\ddot{\text u}$beck, Germany, 2007.

% 9.
\bibitem{5fr}
{\it Custodio E., Marsland B.}
Self-healing partial reconfiguration of an FPGA.
Project Report, Project Number: MQP-BYK-GD07
Worcester Polytechnic Institute, April 26, 2007.

% 10.
\bibitem{10fr}
{\it Lala P.\,K., Kumar B.\,K.}
On self-healing digital system design~//
J.\ Microelectronics, 2006.
Vol.~37. P.~353--362.

% 11.
%\bibitem{11}
%{\it Feller W.}
%Probability Theory and its Applications.
 \end{thebibliography}
}
}


\end{multicols}

%\hrule

\vspace*{-24pt}

\def\tit{ОЦЕНКА ВРЕМЕНИ САМОВОССТАНОВЛЕНИЯ В ЦИФРОВЫХ СИСТЕМАХ ПОСЛЕ СБОЕВ,
ВЫЗЫВАЕМЫХ ПЕРЕХОДНЫМИ ПОМЕХАМИ}

\def\aut{С.\,Л.~Френкель$^1$, А.\,В.~Печинкин$^2$}

\titelr{\tit}{\aut}

\vspace*{12pt}

\noindent
$^1$Институт проблем информатики Российской академии наук,
Slf-ipiran@mtu-net.ru\\
\noindent
$^2$Институт проблем информатики Российской академии наук, apechinkin@ipiran.ru\\

%\vspace*{10mm}
\medskip


\Abst{Рассмотрен новый подход к оценке свойств
самовосстановления  в циф\-ро\-вых системах. Данное свойство характеризует способность системы
продолжать выполнять свои функции в случае сбоя в работе тех или иных компонентов, его учет может быть
полезен при расчете надежности.
Оценивается  время  до возвращения проектируемой системы в режим нормального функционирования после
проявления действия внешней помехи как функция распределения вероятности (ФРВ) этого времени.
Предложены возможные пути расчета ФРВ времени до самовосстановления на основе использовании
марковской модели поведения   цифровой системы, представленной  как конечный автомат  и
подверженной действию переходных помех.
}


\KW{отказоустойчивые компьютеры; самовосстановление и отказоустойчивость;
переходные неисправности; конечные машины состояний; цепи Маркова}

\label{end\stat}


\renewcommand{\figurename}{\protect\bf Рис.}
\renewcommand{\tablename}{\protect\bf Таблица}
\renewcommand{\bibname}{\protect\rmfamily Литература}