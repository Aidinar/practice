\def\stat{grusho}

\def\tit{ПОИСК КОНФЛИКТОВ В ПОЛИТИКАХ БЕЗОПАСНОСТИ: МОДЕЛЬ СЛУЧАЙНЫХ ГРАФОВ$^*$}

\def\titkol{Поиск конфликтов в политиках безопасности: модель случайных графов}

\def\autkol{А.\,А.~Грушо,  Н.\,А.~Грушо, Е.\,Е.~Тимонина}
\def\aut{А.\,А.~Грушо$^1$,  Н.\,А.~Грушо$^2$, Е.\,Е.~Тимонина$^3$}

\titel{\tit}{\aut}{\autkol}{\titkol}

{\renewcommand{\thefootnote}{\fnsymbol{footnote}}\footnotetext[1]
{Работа выполнена при поддержке Российского фонда фундаментальных
исследований, проект~10-01-00480. Статья написана на основе материалов доклада, 
представленного на IV Международном семинаре <<Прикладные задачи теории вероятностей 
и математической статистики, связанные с моделированием информационных систем>> 
(зимняя сессия, Аоста, Италия, январь--февраль 2010 г.).}}

\renewcommand{\thefootnote}{\arabic{footnote}}
\footnotetext[1]{Российский государственный гуманитарный
университет, Институт информационных наук и технологий безопасности;
Московский государственный университет им.\ М.\,В.~Ломоносова,
факультет вычислительной математики и кибернетики, grusho@yandex.ru}
\footnotetext[2]{Российский государственный
гуманитарный университет, Институт информационных наук и технологий
безопасности, grusho@yandex.ru}
\footnotetext[3]{Российский
государственный гуманитарный университет, Институт информационных
наук и технологий безопасности,\linebreak eltimon@yandex.ru}


\Abst{Анализ рисков требует хотя бы приблизительной
оценки вероятностей проблемных ситуаций при внедрении политики
безопасности (ПБ). Такие оценки можно получить при анализе конфликтов в
ПБ с по\-мощью моделей случайных графов. Рас\-смот\-ре\-ны два примера конфликтов в 
ПБ и по\-стро\-ены модели случайных графов для их исследования.
Результаты анализа позволили оценить влияние некоторых параметров
случайных графов на вероятность существования конфликтов и на
сложность алгоритмов поиска конфликтов в реальных политиках
безопасности.}

\KW{политика безопасности; угрозы информационной
безопасности; математические модели анализа политики безопасности}

       \vskip 14pt plus 9pt minus 6pt

      \thispagestyle{headings}

      \begin{multicols}{2}

      \label{st\stat}

\section{Введение}

Защита компьютерных систем (КС), как известно, строится на основе
ПБ~[1--4]. Значительная часть
ПБ~--- это правила ограничения доступа субъектов к объектам~\cite{b6}. 
Дискреционная ПБ предполагает, что у каждого объекта
существует собственник, который определяет права доступа к своему
объекту. Права субъектов~$S$ к объектам~$O$ записываются в виде
матрицы контроля доступов (ACL~--- Access Control List)~\cite{b2, b11}. Матрица  ACL может
быть преобразована в ходе работы компьютерной системы~\cite{b2, b11}. 
Например, предоставление собственником права читать (grant
\textit{read}) позволяет создать копию, которой можно распоряжаться
как собственностью. Отсюда возникла возможность непрямого разрешения~\cite{b9, b4} 
или не\-конт\-ро\-ли\-ру\-емо\-го распространения прав доступа, т.\,е.\
появилась брешь в информационной безопасности, хотя правила ПБ
соблюдаются. О таких ситуациях можно говорить как о конфликтах в ПБ.
Поиск конфликтов предполагает исследование множества всех
потенциальных доступов во всех множествах состояний КС. Необходима
интегральная картина доступности любых субъектов к любым объектам.
Такая картина в реальности очень сложна, но определенное
представление о ней можно получить, заменив реальные доступы
стохастическим процессом.
{\looseness=1

}

\section{Конфликты неконтролируемого распространения прав доступа}

Анализу моделей дискреционной ПБ посвящено много работ~[7--10]. 
Для того чтобы проанализировать проблемы
поиска конфликтов в дискреционной ПБ, авторы обратились к одной из
самых первых моделей. В~\cite{b2} приведено описание известной ранее
математической модели распространения прав доступа с помощью
операций \textit{take} (\textit{t}), \textit{grant} (\textit{g}) и
\textit{create} в КС, где все объекты являются субъектами. В~этой
классической модели под безопасностью понимается невозможность для
произвольного фиксированного субъекта~$P$ получить доступ~$\alpha$ к
произвольному фиксированному объекту~$X$ путем преобразования
текущей матрицы доступа~$A$ некоторой последовательностью команд в
матрицу~$A'$, где указанный доступ разрешен. Если в матрице доступа~$A$ 
оставить только команды преобразования (например, \textit{t} и/или
\textit{g}), то полученной матрице~$G$ можно поставить в соответствие
ориентированный граф, где матрица смежности получается из~$G$
заменой допустимых команд на~1, причем эти команды можно
рассматривать как метки на дугах графа.

В предположениях этой модели доказаны необходимые и достаточные
условия, когда субъект~$y$, не имея прав доступа~$\alpha$ к субъекту~$x$, 
может их получить, используя правила \textit{t} и \textit{g}.
Незаконное получение прав доступа субъектов к объектам в КС~---
наиболее известный конфликт в дискреционной~ПБ.

В работе~\cite{b7} для проблемы поиска конфликтов при распространении прав доступа к файлам и 
операционной сис\-те\-ме 
доказаны неразрешимость и в определенных условиях~--- NP-полнота. В~настоящее время эта проблема 
актуальна в распределенных КС~\cite{b4,b12}.

В более общем случае следует рассматривать наборы множеств субъектов
$A_1, \ldots ,A_k$ и $B_1, \ldots ,B_r$ таких, что некоторая~(0,\,1) матрица~$D$ размера $k\times r$ 
определяет запреты на доступы субъектов из
множеств $A_1,\ldots ,A_k$ к субъектам из множеств $B_1,\ldots ,B_r$. 
В~первоначальной матрице ACL таких доступов нет, но это не значит, что
не может возникнуть конфликтов в ходе работы КС. Поскольку наборы
таких классов субъектов могут быть произвольными, то возникает
задача о классификации всех субъектов по доступности друг к другу.
Например, как устроено множество субъектов, доступных для любых
других, при каких условиях такое множество может существовать и
какова его мощность? Как устроены множества субъектов, из которых,
используя допустимые правила, можно достигнуть заданного субъекта? В~этих 
общих задачах не всегда нужна конкретика, только общие
соображения о том, чего можно ожидать в самых общих предположениях.

Представляют интерес различные приближенные решения задачи поиска
конфликтов при распространении прав доступов в условиях, когда чис\-ло
субъектов велико и возможны асимптотические оценки.

В~данном разделе рассматривается модель случайного распределения
прав изменения матриц ACL. Задача поиска конфликтов в этом случае
сводится к поиску путей в случайном графе, который определяется
исходя из допущений о непрямом разрешении доступов. Иногда такой
подход позволяет оценить трудоемкость поиска конфликтов при
использовании различных алгоритмов и указать условия эффективности
такого поиска.

Пусть время дискретно и в каждый момент задана матрица ACL.
Рассмотрим три способа распространения прав доступов. Предположим,
что некоторые субъекты могут брать у других их права. Оставим в
матрице ACL только субъекты и из всех прав оставим только~\textit{t}.
Заменим~\textit{t} на~1, получим матрицу смежности ориентированного
графа. Таким образом получили последовательность \textit{t}-гра\-фов.
Аналогично строится ориентированный граф, если допустить
существование только правила~\textit{g}, т.\,е.\ возможность любого
субъекта передавать свои права. Если использовать оба правила~(\textit{t}, 
\textit{g}), то, как показано ранее (см., например,~\cite{b2}), 
распространение прав связано с существованием
неориентированных путей в графе, получаемом заменой~\textit{t} или
\textit{g} на~1 в матрице ACL и выбрасыванием остальных прав.
Необходимо заметить, что подобные графовые модели строятся для
других задач компьютерной безопасности. Например, при анализе
скрытых каналов или при анализе потенциальных деревьев атак.

Рассмотрим графы с~$N$ вершинами и проблему распространения прав
доступов с помощью команды \textit{take}. Для удобства расчетов будем
считать, что число исходящих дуг вершины $x$ равно~$d(x)$.
Рассмотрим задачу при простейшем ограничении $d(x)=m\geq 2$. Это
значит, что каждая вершина может взять права в точности у~$m$ других
вершин. Получение~$x$ запрещенных прав у некоторой вершины~$y$
означает существование ориентированного пути из~$x$ в~$y$.
Предположим, что каждая дуга попадает в любую вершину с вероятностью\linebreak
$1/N$ независимо от других дуг (петли допускаются
исключительно для удобства использования\linebreak известных результатов).
Таким образом определенные случайные ориентированные графы
анализировались в работе~\cite{b1}. Оказывается, что такой случайный
ориентированный граф при $N\rightarrow \infty$, $m=const\geq 2$ с
вероятностью, стремящейся к~1, имеет единственную компоненту сильной
связ\-ности размера $(\theta_0/m)N+o(N)$, где $\theta_0$~---
положительный корень уравнения
$$
1-\fr{\theta}{m}-e^{-\theta}=0\,,\quad \fr{\theta_0}{m}> \fr{1}{2}\,,\quad
m\geq 2\,.
$$

Любая вершина орграфа имеет ориентированный путь в эту компоненту,
длина которого меньше~$f_N$, где $f_N$~--- произвольная положительная
функция, $f_N\overset {N\rightarrow \infty}{\longrightarrow} \infty$. Отсюда
можно сделать следующие выводы.
\begin{enumerate}
    \item Если вершина, у которой необходимо взять нужные права, находится в компоненте 
    сильной связности, то для любой вершины это будет возможно с вероятностью, стремящейся к~1.
    \item Некоторые вершины могут быть недоступны сразу для всех других. Число таких вершин с 
    вероятностью, стремящейся к~1, имеет поря-\linebreak док~ $(1-(\theta_0/m))N +o(N)$. Можно показать, что 
    максимальная длина подхода к этим вершинам имеет порядок не более чем $c_0\ln{N}$, где $c_0>1$. 
    Отсюда следует, что алгоритм проверки защищенности объекта должен вычислять субъекты, способные 
    достичь данный, за $k c_0\ln{N}$ шагов с вероятностью, стремящейся к~1. Тогда можно\linebreak принять состоятельное 
    (т.\,е.\ вероятность ошибки стремится к~0) решение о недоступности\linebreak объекта из некоторого множества 
    опасных с вероятностью, стремящейся к~1. Такой подход представляет упрощение решения с использованием 
    вероятностных характеристик, т.\,е.\ он предлагает упрощение алгоритма поиска всех вершин, имеющих 
    ориентированный путь в данную вершину (с вероятностью, стремящейся к~1). Если это множество строится 
    за более чем $(1-\theta_0/m)N$ шагов, то оно (с ве\-ро\-ят\-ностью, стремящейся к~1) совпадает с множеством всех объектов. 
    Таким образом, сложность алгоритма поиска конфликта существенно меньше сложности тотального алгоритма построения всех 
    путей в данную вершину.
\end{enumerate}

Для $g$-графа и $m\geq 2$ результаты, относящиеся к структуре
случайного графа, интерпретируются аналогично. Вершина может быть
доступной для всех других (если ее выбор случаен, то вероятность
этого события ${\theta_0}/{m}$). Или же она доступна для не
более чем $~ (1-{\theta_0}/{m})N$ вершин с вероятностью 
$(1-{\theta_0}/{m})$. Причем эта доступность (длина пути) не
превосходит $c_0N$~шагов.

Картина резко меняется при $m=1$. В~этой ситуации случайный граф
$t$-до\-сту\-па или $g$-до\-сту\-па представляет собой случайное отображение
множества всех субъектов в себя. Структуру таких случайных графов
исследовал В.\,Е.~Степанов~\cite{b3}. В~таких графах порядка
$({1}/{2})\ln{n}$ компонент, в каждой из которых ровно один
ориентированный цикл (компонента сильной связности) и из всех других
вершин компоненты связности существуют ориентированные пути,
максимальная длина которых имеет порядок~$\sqrt{n}$, в точки цикла
этой компоненты. Размер максимальных компонент имеет порядок
$\epsilon n$, $\epsilon < 1,$ а размер минимальных~--- константа. Всего
число точек, принадлежащих циклам, имеет порядок~$\sqrt{n}$.

В интерпретации поиска конфликтов в~\textit{t}- или \textit{g}-гра\-фах
получается следующая картина. Если субъекты, пытающиеся получить
доступ, и субъекты, у которых доступ первых запрещен, находятся в
разных компонентах связности, то конфликта ПБ нет в~$t$, $g$ или
$(t, g)$ доступах. В одной компоненте для всех субъектов доступны
субъекты, соответствующие циклическим точкам. Число циклических
точек в любой компоненте ограничено величиной порядка~$\sqrt{n}$.
Для максимальных компонент это пренебрежимо меньше числа всех точек
в компоненте. Значит, упрощенного алгоритма поиска запрета на доступ
в данном случае нет. Вместе с тем вероятность, что одна случайно
выбранная точка будет доступна из другой случайно выбранной точки,
стремится к~0.

Рассмотрим случай, когда передача прав возможна, но маловероятна. Построим следующую 
вероятностную модель случайного графа, описывающего этот случай. Для всех вершин~$x$ исходящие степени~$d(x)$ 
определяются следующим вероятностным законом
$$
P(d(x)=0)=p\,.
$$

Все вершины можно считать упорядоченными. Тогда вероятность дуги~$(xy)$ 
равна $(1-p)({1}/{N-1})$ независимо от появления остальных
дуг, а для любого~$k$ вероятность существования фиксированной
ориентированной цепочки длины~$k$ равна
$((1-p)/(N-1))^{k-1}$.

Применяя неравенство Маркова, получаем оценку вероятности
существования в графе хотя бы одной ориентированной цепочки длины~$k$
$$
N(N-1)\cdots(N-k+1)((1-p)/(N-1))^{k-1}\,.
$$
При $1-p=o({1}/{N})$ вероятность появления любой цепочки длины
$k\geq 2$ стремится к~0, т.\,е.\ с ве\-ро\-ят\-ностью, стремящейся к~1,
распространения прав не будет.

\section{Конфликты неэффективности аудита}

Рассмотрим еще один класс конфликтов в политиках безопасности.
Начиная с класса С$_2$ стандарта TCSEC (Trusted Computer System Evaluation Criteria)~\cite{b6}, в ПБ требуется
отслеживание действий пользователей в КС. По результатам
отслеживания определяются нарушения ПБ. Несмотря на очевидность
процедуры отслеживания, быстро выясняется ее неадекватность в
некоторых ситуациях.

Приведем следующий пример. Пусть два пользователя~$U_1$ и~$U_2$
имеют легальный доступ к конфиденциальному содержанию объекта~$O$.
Предположим, что некто опубликовал конфиденциальное содержание
объекта~$O$. По данным отслеживания к объекту легально имеют доступ
оба пользователя~$U_1$ и~$U_2$. Однако достоверного выявления
виновного эти данные не обеспечивают. Возможность однозначно
определить нарушителя безопасности основана на предположении, что
только один (случайный) пользователь из двух легальных реально имел
доступ к~$O$.

Обобщим этот пример и построим для него модель случайных графов.
Пусть~$\textbf{S}$~--- множество субъектов,
$\left|\mathbf{S}\right|=M$, $\mathbf{O}$~--- множество объектов,
$\left|\mathbf{O}\right|=N$, дуга от субъекта~$S$ к объекту~$O$
означает доступ заданного вида, внесенный в журнал отслеживания
действий пользователей. Распределение на дугах случайного графа
следующее. Каждый субъект может не обращаться к объектам с
вероятностью~$p$ и может обращаться к любому объекту с вероятностью
$(1-p)/N$ независимо от других обращений (дуг). Очевидно, что
конфликтная ситуация, состоящая в отсутствии однозначно определенной
ответственности для данного объекта, наступает с вероятностью
$$
P=\left(\left(1-p\right)\fr{1}{N}\right)^{2}\,.
$$

Отсюда следует, что при $N\rightarrow \infty$ с вероятностью,
стремящейся к~1, в системе не возникает конфликтов с ПБ для данных
двух субъектов относительно любого объекта. Конфликта не следует
ожидать для любой пары субъектов, если
$$
Q=\fr{(1-p)^{2}}{N}\left(^{M}_{2}\right) \longrightarrow 0\,,\ \   N\rightarrow \infty\,,\ \  M\rightarrow \infty\,.
$$
Эта величина является оценкой вероятности конфликта.

Отсюда легко построить упрощенный критерий проверки возможности появления конфликта в данных аудита в течение 
времени~$T$. Если суммарное число субъектов, не обратившихся к объектам за время~$T$, равно~$\nu$, то оценка 
$p={\nu}/{M}$. Тогда при малых~$Q$ возникновение конфликта от\-вет\-ст\-вен\-ности маловероятно. 
В~противном случае необходимо находить дополнительные механизмы обеспечения ответственности.

\section{Заключение}

Некоторые задачи анализа ПБ являются очень трудоемкими. В~связи с
этим на практике такие задачи просто не рассматриваются. Вместе с
тем, анализ рисков требует хотя бы приблизительной оценки
вероятностей проблемных ситуаций при внедрении ПБ. Такие оценки
можно получить при анализе конфликтов в ПБ с помощью моделей
случайных графов. Более того, с помощью моделей случайных графов
удается определить значения наблюдаемых параметров, при которых
вероятности появления конфликтов малы или трудоемкость обнаружения
конфликтов является маленькой. В~этой работе рассмотрены два примера
конфликтов в ПБ и построены модели случайных графов для их
исследования. Результаты анализа позволили оценить влияние некоторых
параметров случайных графов на вероятность существования конфликтов
и на сложность алгоритмов поиска конфликтов в реальных~ПБ.


{\small\frenchspacing
{%\baselineskip=10.8pt
%\addcontentsline{toc}{section}{Литература}
\begin{thebibliography}{99}

\bibitem{b6} %1
Department of Defence Trusted Computer System Evaluation Criteria.~--- DoD, 1985.

\bibitem{b2} %2
\Au{Грушо А.\,А., Тимонина~Е.\,Е.} 
Теоретические основы компьютерной безопасности.~--- М.: Яхтсмен, 1996.

\bibitem{b11} %3
\Au{Pieprzyk G., Hardjono~Th., Seberry~J.} 
Fundamentals of computer security.~--- Springer, 2003.

\bibitem{b13} %4
\Au{Whitman M.\,E., Mattotd H.\,J.} 
Management  of information security. 2nd ed.~--- Thomson Course Technology, 2008.

\bibitem{b9} %5
\Au{Grahom G., Denning P.} Protection: Principles and practicies~//
AFIPS Spring Joint Computer Conference Proceedings, 1972. P.~417--429.

\bibitem{b4} %6
\Au{Щербаков А.\,Ю.} 
Введение в теорию и практику компьютерной безопасности.~--- М.: Издатель Молгачева~С.\,В., 2003.

\bibitem{b10} %7
\Au{Moffet G.\,D., Sloman~M.\,S.} 
The representation of polycies as system objects~// ACM Conference on
Organizational Computing Systems Proceedings. Atlanta, GA. November 1991. P.~171--184.

\bibitem{b5} %8
\Au{Abadi M., Barrows~M., Lampson~B., Plotkin~G.} 
A calculus for
access control in distributed systems~// ACM Transactions on
Programming Languages and Systems, 1993. Vol.~15. No.\,4. P.~706--734.

\bibitem{b8} %9
\Au{Gail-Joon Ahn, Sandhu R.} 
The RSL99 language for role-based
separation of duty constraints~// ACM Workshop on Role-Based Access
Control, 1999. P.~43--54.

\bibitem{b12} %10
\Au{Ryutov T., Neuman~C.} 
The set and function approach to modeling
autorization in distributed systems~//
Workshop (International) on Methods, Models, and Architectures for Network Security  Proceedings,
MMM-ACNS 2001, St.~Petersburg // Information Assurance in Computer
Networks. LNCS 2052.~--- Springer, 2001. P.~189--206.

\bibitem{b7} %11
\Au{Harrison M., Ruzzo~W., Ullman~J.} 
Protection in operating
systems~// Communications of the ACM, 1976. Vol.~19. No.\,8. P.~461--471.

\bibitem{b1} %12
\Au{Грушо А.\,А.} 
О предельных распределениях некоторых
характеристик случайных автоматных графов~// Математические заметки, 1973. 
Т.~14. Вып.~1. С.~133--141.

\label{end\stat}

\bibitem{b3} %13
\Au{Степанов В.\,Е.} 
Предельные распределения некоторых
характеристик случайных отображений~// Теория вероятностей и ее
применения, 1969. Т.~14. Вып.~4. С.~639--653.


 \end{thebibliography}
}
}


\end{multicols}