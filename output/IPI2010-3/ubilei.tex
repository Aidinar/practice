\def\stat{sokolov}

\def\tit{О РАБОТАХ ЗАСЛУЖЕННОГО ДЕЯТЕЛЯ НАУКИ 
РОССИЙСКОЙ ФЕДЕРАЦИИ И.\,Н.~СИНИЦЫНА В ОБЛАСТИ 
ИНФОРМАЦИОННЫХ ТЕХНОЛОГИЙ И АВТОМАТИЗАЦИИ\\
(к 70-летию со дня рождения)}

\def\titkol{О работах заслуженного деятеля науки 
РФ И.\,Н.~Синицына в области 
информационных технологий и автоматизации
%(к семидесятилетию со дня рождения)
}

\def\autkol{И.\,А.~Соколов}
\def\aut{И.\,А.~Соколов$^1$}

\titel{\tit}{\aut}{\autkol}{\titkol}

%{\renewcommand{\thefootnote}{\fnsymbol{footnote}}\footnotetext[1]
%{Работа поддерживается РФФИ, грант  10-07-00017.}}

\renewcommand{\thefootnote}{\arabic{footnote}}
\footnotetext[1]{Институт проблем информатики Российской академии наук, isokolov@ipiran.ru}


\bigskip



%\bigskip

       \vskip 14pt plus 9pt minus 6pt

      \thispagestyle{headings}

      \begin{multicols}{2}

      \label{st\stat}
      
 \begin{center}
\mbox{%
\epsfxsize=50mm
\epsfbox{sok-1.eps}
}
\end{center}
\vspace*{4pt}
%\begin{center}

\bigskip



      14~августа 2010~г.\ исполнилось 70~лет Игорю Николаевичу Синицыну~--- члену 
редколлегии журнала <<Информатика и её применения>>, крупному ученому в области 
прикладной механики и управ\-ле\-ния, прикладной математики и информатики, основателю 
научной школы в области стохастических информационных технологий.
      
      И.\,Н.~Синицын родился в Москве. Высшее образование получил в МВТУ им.\ 
Н.\,Э.~Баумана и МГУ им.\ М.\,В.~Ломоносова. Одновременно с учебой в МГУ начал работать 
в известном ра\-кет\-но-кос\-ми\-че\-ском НИИ, ныне Институте прикладной механики им.\ 
В.\,И.~Кузнецова (НИИПМ). Инженерную и научную деятельность в НИИПМ в области 
разработки и испытаний гироскопических командных приборов и 
      ин\-фор\-ма\-ци\-он\-но-из\-ме\-ри\-тель\-ных систем (1960--1983~гг.), он совмещал с 
преподавательской работой сначала в МВТУ им.\ Н.\,Э.~Баумана, затем в 
      Воен\-но-воз\-душ\-ной инженерной академии им.\ профессора Н.\,Е.~Жуковского 
(ВВИА).
      
      Начиная с 1974~г.\ И.\,Н.~Синицын работал на факультете авиационного вооружения 
ВВИА. Занимался подготовкой авиационных инженеров, принимал участие в разработке и 
испытаниях\linebreak
 специальной техники, участвовал в подготовке первых космонавтов СССР.
      
      Для организации работ в области специальных применений ЭВМ новых поколений 
И.\,Н.~Синицын в 1984~г.\ переводится в только что организованный Институт проблем 
информатики АН СССР (ныне ИПИ РАН).
      
      В настоящее время И.\,Н.~Синицын работает заведующим отделом стохастических 
проблем информатики и управления ИПИ РАН, много внимания уделяет подготовке научных 
кадров. Он руководит специальной секцией ученого совета ИПИ РАН, комиссией Минобрнауки 
по информатике в военных вузах, является членом экспертного совета РФФИ, заместителем 
главных редакторов журналов <<Наукоемкие технологии>> и <<Системы высокой 
доступности>>, членом редколлегий журналов <<Pattern Recognition and Image Analysis>>, 
<<Информатика и её применения>>. С~1987~г.\ И.\,Н.~Синицын~--- профессор МАИ, читает 
лекции по теории и практике информационных технологий в инженерном деле. 
     
     В разные годы И.\,Н.~Синицын был заместителем генерального конструктора и главным 
конструктором ряда автоматизированных и информационных систем специального назначения. 
     
     В 2001~г.\ И.\,Н.~Синицыну присвоено почетное звание Заслуженного деятеля науки 
Российской Федерации.
     
     И.\,Н.~Синицын имеет большой опыт работы в промышленности и высших технических 
учебных заведениях. Он автор более 500~научных трудов, свыше 50 книг, монографий и 
30~изобретений. Его основные научные труды относятся к следующим областям:
     \begin{itemize}
     \item статистическая теория информационных технологий и автоматизированных систем;
     \item прецизионные ин\-фор\-ма\-ци\-он\-но-из\-ме\-ри\-тель\-ные технологии и системы для научных 
исследований и специального назначения;
     \item
      информационно-аналитические технологии и системы поддержки принятия решений для 
информатизации высших органов государственной власти РФ, федеральных ведомств и~др.
     \end{itemize}
     
     И.\,Н.~Синицыну принадлежат фундаментальные результаты по теории канонических 
представлений случайных функций в сложных стохастических системах (СтС), в том числе СтС 
с\linebreak
 распределенными параметрами и случайной структурой. Методы теории СтС им 
распространены на\linebreak
 СтС, описываемые дифференциальными уравнениями со случайными 
функциями состояния,\linebreak уравнениями в гильбертовых и банаховых пространствах. Им 
разработаны эффективные вычислительные методы нахождения распределений, основанные на 
параметризации, позволяющие радикально сократить число уравнений для па\-ра\-мет\-ров 
распределений, а также новые вычислительные методы статистического анализа и синтеза,\linebreak 
допускающие эффективное оценивание точности и ориентированные на параллельные 
статистические вычисления. 
     
     И.\,Н.~Синицын разработал методы нахождения точных выражений для распределений с 
инвариантной мерой, обнаружил ряд новых классов точных распределений. Им получены 
фундаментальные результаты в области нелинейной условно оптимальной и субоптимальной 
фильтрации в реальном масштабе времени. Важные результаты получены И.\,Н.~Синицыным в 
области тео\-ре\-ти\-ко-груп\-по\-вых методов анализа и синтеза автоматизированных систем. 
Им разработана статистическая теория катастрофоустойчивости автоматизированных сис\-тем 
высокой точности и доступности. 
     
     И.\,Н.~Синицын~--- основоположник стохастических информационных технологий 
оперативной обработки информации, контроля и мониторинга автоматизированных систем, а 
также\linebreak стохастического управления информационными\linebreak активами, моделирования и синтеза 
систем: проб\-лем\-но-ориен\-ти\-ро\-ван\-ных диалоговых систем и\linebreak библиотек 
     <<СтС-Ана\-лиз>>, <<СтС-Фильтр>>, <<СтС-Мо\-дель>>, Nailb, <<TransStatLib>>, 
<<Безопасность и надежность>>, <<Здоровье РФ>> и~др. В~последние годы им разработаны 
эффективные символьные методы анализа и синтеза СтС. Создано и внедрено 
специализированное программное обеспечение СтС-СМА и СтС-\mbox{ИТКР}. 
     
     Его книги~--- <<Стохастические дифференциальные системы. Анализ и фильтрация>>, 
<<Лекции по функциональному анализу и его приложениям>>, <<Теория стохастических 
систем>> (совместно с В.\,С.~Пугачевым), а также <<Фильтры Калмана и Пугачева>> и 
<<Канонические представления случайных функций и их применение в задачах компьютерной 
поддержки научных исследований>>~--- широко известны в России и за рубежом.
     
     И.\,Н.~Синицыным впервые разработана теория ряда 
     ин\-фор\-ма\-ци\-он\-но-из\-ме\-ри\-тель\-ных систем в условиях случайных динамических 
возмущений, открыт ряд новых статистических динамических эффектов (выбросы разных 
типов, флуктуационные уходы, накопление возмущений и~др.). Ему принадлежат первые 
работы по статистической динамике командно-измерительных гироскопических приборов, 
акселерометров, градиентометров и метрологических систем высочайшей точности, 
информационной теории и методам измерений, калибровок, ускоренных испытаний в 
экстремальных условиях, а также статистического и полунатурного моделирования. Под его 
руководством и при его непосредственном участии разработано и внедрено несколько 
поколений серийных систем, обла\-да\-ющих уникальными характеристиками. И.\,Н.~Синицын 
принимал непосредственное участие в определении технической политики в области новой 
специальной техники. 
     
     С именем И.\,Н.~Синицына связано создание концепций автоматизации научных 
исследований в РФ, в первую очередь основанных на средствах массовой вычислительной 
техники. Под его руководством и при участии сформулированы принципы создания 
микровидеосистем, создан ряд базовых персональных микровидеосистем. Они внедрены в МВД 
и Минздраве РФ. Достигнутые результаты получили развитие в автоматизированных системах 
метрологического обеспечения, видеоконтроля и биометрических системах. 
     
     В последние годы под руководством И.\,Н.~Синицына разработаны принципы построения 
и архитектуры вычислительных систем командных пунктов, а также новые методы и алгоритмы 
быстрой обработки изображений, обладающих сильной про\-стран\-ст\-вен\-но-вре\-мен\-н\'{о}й 
деформацией. В~целях\linebreak автоматизации астрометрических научных исследований по 
фундаментальной проблеме <<Статистическая динамика вращения Земли>> был создан 
комплекс моделей, алгоритмов и специального\linebreak программного обеспечения и информационных 
ресурсов для нестандартной интегрированной обработки параметров вращения Земли. 
И.\,Н.~Синицын впервые обнаружил ряд новых эффектов: автоколебания полюса Земли на 
чандлеровской частоте, параметрическую стабилизацию чандлеровских колебаний, нелинейные 
флуктуационные дрейфы нестабильности вращения Земли и~др.
     
     В области информационно-аналитических технологий и автоматизированных систем 
поддержки принятия решений для информатизации высших органов государственной власти, 
федеральных ведомств и~др.\ под руководством И.\,Н.~Синицына был разработан и внедрен 
ряд базовых информационных технологий (обработка информации от независимых источников, 
формирование и хранение больших баз электронных образов, управления информационными 
активами и~др.). Сформулированы принципы и разработаны базовые системотехнические 
решения для ряда крупномасштабных автоматизированных информационных и 
     ин\-фор\-ма\-ци\-он\-но-управ\-ля\-ющих систем специального назначения, высокой 
точности и доступности. 
     
     И.\,Н.~Синицын пользуется широким международным авторитетом: его книги и работы 
изданы на английском, французском, испанском и китайском языках; в 1990--1994~гг.\ он был 
директором Рос\-сий\-ско-фран\-цуз\-ско\-го центра <<Эвклид>>, с~1983~г.~--- членом 
программных комитетов многих международных конференций; в период 1985--2006~гг.\ был 
экспертом фонда INTAS.
     
     И.\,Н.~Синицын ведет активную работу по подготовке научно-педагогических кадров. 
Под его руководством выполнено свыше 25~кандидатских и докторских диссертаций. Он 
состоит членом ряда специализированных диссертационных советов; в течение многих лет был 
членом экспертного совета ВАК России. И.\,Н.~Синицын~--- член редколлегии нашего 
журнала, он не толко публикуется в нем, но и ведет большую редакционную работу.

\label{end\stat}
     
     \bigskip
     Редколлегия журнала сердечно поздравляет И.\,Н.~Синицына с юбилеем и желает ему 
здо\-ровья, счастья, новых творческих успехов.


\end{multicols}