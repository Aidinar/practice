
\def\stat{shorgin}

\def\tit{БАЙЕСОВСКИЕ МОДЕЛИ МАССОВОГО  ОБСЛУЖИВАНИЯ И~НАДЕЖНОСТИ:
ХАРАКТЕРИСТИКИ СРЕДНЕГО ЧИСЛА ЗАЯВОК  В~СИСТЕМЕ $M|M|1|\infty$$^*$}

\def\titkol{Байесовские модели массового  обслуживания и надежности}
%: характеристики среднего числа заявок  в системе $M|M|1|\infty$}

\def\autkol{А.\,А.~Кудрявцев, С.\,Я.~Шоргин}
\def\aut{А.\,А.~Кудрявцев$^1$, С.\,Я.~Шоргин$^2$}

\titel{\tit}{\aut}{\autkol}{\titkol}

{\renewcommand{\thefootnote}{\fnsymbol{footnote}}\footnotetext[1]
{Работа выполнена при поддержке РФФИ, проекты 08-07-00152-а, 08-01-00567-а 
и 09-07-12032-офи-м. Статья написана на основе материалов доклада, 
представленного на IV Международном семинаре  
<<Прикладные задачи теории вероятностей и математической статистики, 
связанные с моделированием информационных систем>> 
(зимняя сессия, Аоста, Италия, январь--февраль 2010~г.).}}

\renewcommand{\thefootnote}{\arabic{footnote}}
\footnotetext[1]{Московский
государственный университет им.\ М.\,В.~Ломоносова, факультет
вычислительной математики и кибернетики, nubigena@hotmail.com}
\footnotetext[2]{Институт проблем информатики Российской академии наук, sshorgin@ipiran.ru}

\vspace*{-6pt}

\Abst{Данная работа продолжает ряд статей, посвященных байесовским моделям массового обслуживания и
надежности. В работе рассматриваются вероятностные характеристики среднего числа заявок в системе $M|M|1|\infty$
в условиях рандомизации параметров входящего потока и обслуживания. Обсуждается интерпретация получаемых результатов с
учетом возможной несобственности распределения среднего числа заявок.}

\KW{байесовский подход; системы массового обслуживания; надежность; смешанные
распределения; моделирование; несобственное распределение; <<дефектное>> распределение}

\vskip 14pt plus 9pt minus 6pt

      \thispagestyle{headings}

      \begin{multicols}{2}

      \label{st\stat}


\section{Основные предположения и~обозначения}

Подробное изложение основ байесовского подхода к моделированию систем массового обслуживания (СМО) и ненадежных
восстанавливаемых систем, а также результаты вычисления основных вероятностных характеристик коэффициентов
загрузки и готовности системы $M|M|1|0$, входные параметры которой не известны исследователю в точности (известно лишь их
априорное распределение), можно найти в работах~[1--6].

Основным предположением в рамках данного подхода для моделей $M|M|1$ является рандомизация интенсивностей входящего
потока~$\lambda$ и обслуживания~$\mu$. При этом, естественно, становится случайной и загрузка рассматриваемой системы
$\rho=\lambda/\mu$, от значения которой, в частности, зависит наличие стационарного режима у рассматриваемой системы.
Кроме того, величина~$\rho$ входит во многие формулы, описывающие характеристики разнообразных СМО.
В статье рассматривается одна из таких характеристик, а именно среднее число заявок в системе  $M|M|1|\infty$
$$
N=\fr{\rho}{1-\rho}\,.
$$

В дальнейшем изложении будем предполагать, что входные параметры системы стохастически независимы и имеют вырожденное
($D$), равномерное~($R$), экспоненциальное~($M$), Эрланга~($E$) распределение. В~скобках будем указывать
соответствующие параметры распределения, например обозначение~$D(a)$ будет соответствовать вырожденному в точке~$a$
распределению.

\section{Функция распределения и~плотность}

Будем обозначать функции распределения случайных величин~$\rho$ и~$N$ соответственно
\begin{equation}
\left.
\begin{array}{rl}
F_\rho(x)&=\p(\rho<x)\,;\\[6pt]
 F_N(x)&=\p(N<x)=F_\rho\left( \fr{x}{1+x}\right)\,,\enskip x>0\,.
 \end{array}
 \right \}
\label{e1sh}
\end{equation}
Обозначим через $f_\rho(x)$ и~$f_N(x)$ соответствующие плотности. Очевидно,
\begin{equation}
f_N(x)=\fr{1}{(1+x)^2}f_\rho\left(\fr{x}{1+x}\right)\,,\enskip x>0\,.
\label{e2sh}
\end{equation}

\begin{table*}\small
\begin{center}
\Caption{Функция распределения коэффициента загрузки $\rho$
\label{t1sh}}
\vspace*{2ex}

\begin{tabular}{|c|c|c|}
\hline
$\lambda$&$\mu$&$F_\rho(x),\ x>0$\\
\hline
&&\\[-9pt]
$D(\lambda)$&$M(\alpha)$&$e^{-\alpha\lambda/x}$\\
\hline
&&\\[-9pt]
$D(\lambda)$&$E(n,\alpha)$&$e^{-\alpha\lambda/x}\displaystyle\sum\limits_{m=0}^{n-1}\fr{(\alpha\lambda)^m}{x^mm!}$\\
&&\\[-9pt]
\hline
&&\\[-9pt]
$M(\theta)$&$D(\mu)$&$1-e^{-\mu\theta x}$\\
&&\\[-9pt]
\hline
&&\\[-9pt]
$M(\theta)$&$M(\alpha)$&$\displaystyle\fr{\theta x}{\alpha+\theta x}$\\
&&\\[-9pt]
\hline
&&\\[-9pt]
$M(\theta)$&$E(n,\alpha)$&$1-\left(\displaystyle\fr{\alpha}{\alpha+\theta x}\right)^n$\\
&&\\[-9pt]
\hline
&&\\[-9pt]
$E(k,\theta)$&$D(\mu)$&$1-e^{-\mu\theta x}\displaystyle\sum\limits_{m=0}^{k-1}\displaystyle\fr{(\mu \theta x)^m}{m!}$\\
&&\\[-9pt]
\hline
&&\\[-9pt]
$E(k,\theta)$&$M(\alpha)$&$\left(\displaystyle\fr{\theta x}{\alpha+\theta x}\right)^k$\\
&&\\[-9pt]
\hline
&&\\[-9pt]
$E(k,\theta),\ \ k\ge2$&$E(n,\alpha),\ \ n\ge2$&$\left(\displaystyle\fr{\theta x}{\alpha+\theta x}\right)^{n+k-1}
\displaystyle\sum\limits_{m=0}^{n-1}\left(\displaystyle\fr{\alpha}{\theta x}\right)^m$\\[9pt]
\hline
\end{tabular}
\end{center}
%\vspace*{9pt}
\end{table*}

Для удобства дальнейшего изложения приведем таблицу функций
распределения коэффициента загрузки~$\rho$ при различных
распределениях интенсивностей входящего потока $\lambda$ и
обслуживания~$\mu$. Представленные результаты опубликованы авторами
в работах~\cite{KuSh07, KuSh09a, KuSh09b}. (Для представления
функции распределения Эрланга с параметрами~$k$ и~$\mu\theta$ в
случае, когда~$\lambda$ имеет распределение $E(k,\theta)$, а~$\mu$~--- 
$D(\mu)$, использована формула~3.351.1 из~\cite{GR71}.)


Как известно (см., например,~\cite{BoPe95}), при классической постановке задачи величину~$N$ имеет смысл рассматривать
только в случае $\rho<1$ (в противном случае $N=\infty$). При байесовском подходе нельзя однозначно сказать, какое
значение примет величина~$\rho$, а следовательно, априори нельзя сделать и точных выводов о том, будет ли система
переполняться. Вполне естественно, что основную роль в прогнозах, касающихся среднего числа заявок~$N$, будет играть
величина
\begin{equation}
\delta=\p(\rho\ge1)=1-F_\rho(1)\,.
\label{e3sh}
\end{equation}
При этом, если не выполнено условие $\p(\lambda<\mu)=1$, величина~$\delta$ будет положительной. В~этом случае функция
распределения рандомизированного среднего числа заявок в системе будет отделена от единицы на величину $\delta$.
Другими словами,
\begin{equation}
\lim_{x\to\infty}F_N(x)=1-\delta\,.
\label{e4sh}
\end{equation}

При рассмотрении распределений, обладающих свойством~(\ref{e4sh}), возможны два подхода. Согласно первому (см., 
например,~\cite{Shiryaev80}) предполагается, что случайная величина~$N$ может принимать бесконечное значение с положительной
вероятностью. В~этом случае~$N$ называется расширенной случайной величиной. Согласно второму подходу 
(см.~\cite{Feller67}) делается предположение, что <<масса>> распределения случайной величины~$N$ строго 
меньше единицы. При
этом само распределение называется несобственным или <<дефектным>>, но предположений о возможном бесконечном значении
соответствующей случайной величины, по сути, не делается. Заметим, что для исследователя зачастую более удобен второй
подход к данной проблеме, поскольку, например, математическое ожидание расширенной случайной величины всегда равняется
бесконечности, в то время как математическое ожидание <<дефектной>> случайной величины, рассматриваемое как координата
центра масс системы, масса которой меньше единицы, может быть конечным числом.

Следуя терминологии~\cite{Feller67}, назовем величину~$\delta$, определенную в~(\ref{e3sh}), 
\textit{<<дефектом>> системы}.

Приведем значения величины <<дефекта>>~$\delta$ для байесовских СМО при различных предположениях об априорных
распределениях интенсивностей~$\lambda$ и~$\mu$. 
Применяя формулу~(\ref{e3sh}) к выражениям из табл.~\ref{t1sh}, 
получим результаты, показанные в табл.~\ref{t2sh}.

\begin{table*}\small %tabl2
\begin{center}
\Caption{Значения величины <<дефекта>> $\delta$
\label{t2sh}}
\vspace*{2ex}

\begin{tabular}{|c|c|c|}
\hline
$\lambda$&$\mu$&$\delta$\\
\hline
&&\\[-9pt]
$D(\lambda)$&$M(\alpha)$&$1-e^{-\alpha\lambda}$\\
&&\\[-9pt]
\hline
&&\\[-9pt]
$D(\lambda)$&$E(n,\alpha)$&$1-e^{-\alpha\lambda}\sum\limits_{m=0}^{n-1}\displaystyle\fr{(\alpha\lambda)^m}{m!}$\\
&&\\[-9pt]
\hline
&&\\[-9pt]
$M(\theta)$&$D(\mu)$&$e^{-\mu\theta}$\\
&&\\[-9pt]
\hline
&&\\[-9pt]
$M(\theta)$&$M(\alpha)$&$\displaystyle\fr{\alpha}{\alpha+\theta}$\\
&&\\[-9pt]
\hline
&&\\[-9pt]
$M(\theta)$&$E(n,\alpha)$&$\left(\displaystyle\fr{\alpha}{\alpha+\theta}\right)^n$\\
&&\\[-9pt]
\hline
&&\\[-9pt]
$E(k,\theta)$&$D(\mu)$&$e^{-\mu\theta}\displaystyle\sum\limits_{m=0}^{k-1}\displaystyle\fr{(\mu\theta)^m}{m!}$\\
&&\\[-9pt]
\hline
&&\\[-9pt]
$E(k,\theta)$&$M(\alpha)$&$1-\left(\displaystyle\fr{\theta}{\alpha+\theta}\right)^k$\\
&&\\[-9pt]
\hline
&&\\[-9pt]
$E(k,\theta),\ \ k\ge2$&$E(n,\alpha),\ \ n\ge2$&$1-\displaystyle\fr{\theta^{k+1}(\theta^{n-1}-\alpha^{n-1})}{(\theta-\alpha)
(\theta+\alpha)^{n+k-1}},\ \alpha\neq\theta$\\
&&$\displaystyle1-\fr{n}{2^{n+k-1}},\  \alpha=\theta$\\
\hline
\end{tabular}
\end{center}
%\vspace*{-6pt}
\end{table*}

Аналогично, зная вид функции распределения коэффициента загрузки~$\rho$ и воспользовавшись формулами~(\ref{e1sh}) и~(\ref{e2sh}),
несложно получить выражения для функции распределения и плотности случайной величины~$N$. Не будем останавливаться
здесь на результатах, полученных для распределений~$\lambda$ и~$\mu$, заданных в табл.~\ref{t1sh}, поскольку они не
существенны для дальнейшего изложения и получаются тривиальной заменой переменных.

Далее приведем важный пример системы с нулевым <<дефектом>>. В~силу постулируемой независимости случайных 
величин~$\lambda$ и~$\mu$ обеспечить выполнение условия $\p(\lambda<\mu)=1$ может лишь определенное взаимное расположение
носителей соответствующих распределений.

Пусть случайные величины $\lambda$ и~$\mu$ имеют равномерные распределения $R(a_\lambda,b_\lambda)$ и $R(a_\mu,b_\mu)$,
соответственно, причем $0<a_\lambda<b_\lambda<a_\mu<b_\mu$. Очевидно, что <<дефект>>~$\delta$ такой системы равняется
нулю.

В работе~\cite{KuSh07} были приведены формулы для функции распределения и плотности коэффициента загрузки~$\rho$ для
случая $a_\lambda/a_\mu<b_\lambda/b_\mu$. Воспользуемся этими результатами и формулами~(\ref{e1sh}) и~(\ref{e2sh}). Обозначим

\noindent
$$
c_\lambda=(b_\lambda-a_\lambda)^{-1}\enskip \mbox{и}\enskip   c_\mu=(b_\mu-a_\mu)^{-1}\,.
$$
После несложных арифметических преобразований получаем
$$
F_N(x)=0\,,\ \  \mbox{если}\ \  x<a_\lambda/(b_\mu-a_\lambda)\,;
$$

\vspace*{-12pt}

\noindent
\begin{multline*}
F_N(x)=\fr{c_\lambda c_\mu}{2}\left(b_\mu\sqrt{\fr{x}{1+x}}-a_\lambda\sqrt{\fr{1+x}{x}}\right)^2\,\\
\mbox{если}\ 
\fr{a_\lambda}{b_\mu-a_\lambda}\le x\le\fr{a_\lambda}{a_\mu-a_\lambda}\,;
\end{multline*}

\vspace*{-13pt}

\noindent
\begin{multline*}
F_N(x)=c_\lambda\left(\fr{(b_\mu+a_\mu)x}{2(1+x)}-a_\lambda\right)\,, \\
\mbox{если}\  \fr{a_\lambda}{a_\mu-a_\lambda}\le x\le\fr{b_\lambda}{b_\mu-b_\lambda}\,;
\end{multline*}

\vspace*{-12pt}

\noindent
\begin{multline*}
F_N(x)=c_\lambda c_\mu\left(\fr{b_\lambda(1+x)}{x}-a_\mu\right)\times{}\\
{}\times\left(\fr{b_\lambda}{2}+
\fr{a_\mu x}{2(1+x)}-a_\lambda\right)+
c_\mu\left(b_\mu-\fr{b_\lambda(1+x)}{x}\right)\,,\\
 \mbox{если}\ \fr{b_\lambda}{b_\mu-b_\lambda}\le x\le\fr{b_\lambda}{a_\mu-b_\lambda}\,;
  \end{multline*}
  
%  \vspace*{-1pt}
  
  \noindent
  $$
F_N(x)=1\,,\   \mbox{если} \  x>b_\lambda/(a_\mu-b_\lambda)\,.
$$


Соответствующая плотность будет иметь вид:
$$
f_N(x)=0\,,\ \mbox{если}\  x<a_\lambda/(b_\mu-a_\lambda)
\  \mbox{и}\ 
x>b_\lambda/(a_\mu-b_\lambda)\,;
$$

\vspace*{-12pt}

\noindent
\begin{multline*}
f_N(x)=\fr{c_\lambda c_\mu}{2}\left(\fr{b_\mu^2}{(1+x)^2}-\fr{a_\lambda^2}{x^2}\right)\,,\\
  \mbox{если}\ 
\fr{a_\lambda}{b_\mu-a_\lambda}\le x\le\fr{a_\lambda}{a_\mu-a_\lambda}\,;
\end{multline*}

\vspace*{-12pt}

\noindent
\begin{multline*}
f_N(x)=\fr{c_\lambda (b_\mu+a_\mu)}{2(1+x)^2}\,,\\
 \mbox{если}\ 
\fr{a_\lambda}{a_\mu-a_\lambda}\le x\le\fr{b_\lambda}{b_\mu-b_\lambda}\,;
\end{multline*}

\vspace*{-12pt}

\noindent
\begin{multline*}
f_N(x)=\fr{c_\lambda c_\mu(2b_\lambda a_\lambda-b_\lambda^2)}{2x^2}-\fr{c_\lambda c_\mu a_\mu^2}{2(1+x)^2}
+\fr{c_\mu b_\lambda}{x^2}\,,\\
 \mbox{если}\ \fr{b_\lambda}{b_\mu-b_\lambda}\le x\le\fr{b_\lambda}{a_\mu-b_\lambda}\,.
\end{multline*}

Аналогичные результаты можно получить для случая $a_\lambda/a_\mu\ge b_\lambda/b_\mu$.

%\vspace*{-6pt}
\section{Моментные характеристики}

%\vspace*{-6pt}

Как упоминалось в разд.~2, вопрос о существовании математического ожидания среднего числа заявок~$N$ существенным
образом зависит от величины <<дефекта>> системы~$\delta$. В~случае $\delta>0$ математическое ожидание, понимаемое в
классическом смысле, всегда равняется бесконечности.

Однако можно рассмотреть аналог математического ожидания~--- координату центра масс системы, масса которой равняется
$1-\delta$. Обозначим эту характеристику $\hat{\sf E}N$. Формально имеет место следующее определение:
$$
\hat{\sf E} N=\fr{1}{F_N(\infty)}\int\limits_{0}^{\infty}x\, dF_N(x)\,.
$$

Заметим, что в случае нулевого <<дефекта>> $\hat{\sf E}N$ совпадает с~$\e N$ и, по сути, является условным
математическим ожиданием $\hat{\sf E}N=\e\left(N\ |\ \rho<1\right)$.

Соотношения~(\ref{e1sh}), (\ref{e3sh}) и~(\ref{e4sh}) 
дают возможность привести цепочку тождеств, позволяющих записать $\hat{\sf E}N$ при
помощи различных характеристик системы:
\begin{multline}
\hat{\sf E}N=\fr{1}{F_N(\infty)}\il{0}{\infty}(F_N(\infty)-F_N(x))\, dx={}\\
{}=\fr{1}{1-\delta}\il{0}{\infty}(1-\delta-F_N(x))\, dx={}\\
{}=\fr{1}{F_\rho(1)}\il{0}{\infty}\left(F_\rho(1)-F_\rho\left(\fr{x}{1+x}\right)\right)\, dx={}\\
{}
=\fr{1}{F_\rho(1)}\il{0}{1} \fr{F_\rho(1)-F_\rho(x)}{1-x}\, \fr{dx}{1-x}\,.
\label{e5sh}
\end{multline}

Последнее выражение в~(\ref{e5sh}) показывает, что в абсолютно непрерывном случае для существования $\hat{\sf E}N$
необходимо, чтобы плотность~$f_\rho(x)$ в некоторой окрестности $(1-\varepsilon,\,1]$ убывала к нулю при $y\to 1$
быстрее, чем $(1-y)^p$ для некоторого $p\in(0,\,1)$.
Приведем пример такого распределения с положительным <<дефектом>>.

Пусть
$$
f_\rho(x)=\fr{3}{4}\left\vert1-x\right\vert^{{1}/{2}}\,,\enskip x\in[0,\,2]\,.
$$
Тогда
\begin{multline*}
\hat{\sf E}N=\fr{1}{F_\rho(1)}\il{0}{\infty}x\, dF_\rho\left(\fr{x}{1+x}\right)
={}\\
{}=\fr{1}{F_\rho(1)}\il{0}{1}\fr{yf_\rho(y)}{1-y}\,dy=
\fr{3}{2}\il{0}{1}\fr{y\,dy}{\sqrt{1-y}}={}\\
{}=\fr{3}{2}\il{0}{1}\fr{dy}{\sqrt{1-y}}-
\fr{3}{2}\il{0}{1}\sqrt{1-y}\,dy=2\,.
\end{multline*}

Таким образом, в данном случае получена некоторая конечная моментная характеристика распределения случайной величины~$N$, 
при этом математическое ожидание <<дефектного>> распределения~$N$ бесконечно.

Несмотря на то что подобный подход может давать дополнительную
информацию о распределении среднего числа заявок~$N$, ограничение на
поведение плотности коэффициента загрузки~$\rho$ в окрестности
единицы является столь существенным, что для всех распределений
интенсивностей\linebreak входящего потока~$\lambda$ и обслуживания~$\mu$,
приведенных в табл.~\ref{t1sh}, как несложно убедиться, будут получаться
бесконечные значения для~$\hat{\sf E}N$. По этой\linebreak причине
необходимо привлечение иных методов изучения интересующего нас
распределения. Об~одном из таких методов речь пойдет в сле\-ду\-ющем
разделе.

В~заключение данного раздела рассмотрим пример <<недефектного>> распределения, для которого существуют классические
моментные характеристики.

\begin{table*}[b]\small
\begin{center}
\Caption{Значения квантилей $x_q$ распределения $N$
\label{t3sh}}
\vspace*{2ex}

\tabcolsep=7pt
\begin{tabular}{|c|c|c|}
\hline
$\lambda$&$\mu$&$x_q,\ 0<q<1-\delta$\\
\hline
%\hline
&&\\[-9pt]
$D(\lambda)$&$M(\alpha)$&$-\displaystyle\fr{\alpha\lambda}{\ln\,q+\alpha\lambda}$\\
&&\\[-9pt]
\hline
&&\\[-9pt]
$D(\lambda)$&$E(n,\alpha)$&$\displaystyle\fr{\alpha\lambda}{y(q,n)-\alpha\lambda}$ (см.~(6))\\
&&\\[-9pt]
\hline
&&\\[-9pt]
$M(\theta)$&$D(\mu)$&$\displaystyle-\fr{\ln (1-q)}{\mu\theta+\ln (1-q)}$\\
&&\\[-9pt]
\hline
&&\\[-9pt]
$M(\theta)$&$M(\alpha)$&$\displaystyle\fr{\alpha q}{\theta-(\theta+\alpha)q}$\\
&&\\[-9pt]
\hline
&&\\[-9pt]
$M(\theta)$&$E(n,\alpha)$&$\displaystyle\fr{\alpha((1-q)^{-1/n}-1)}{\theta-\alpha((1-q)^{-1/n}-1)}$\\
&&\\[-9pt]
\hline
&&\\[-9pt]
$E(k,\theta)$&$D(\mu)$&$\displaystyle\fr{y(1-q,k)}{\mu\theta y(1-q,k)}$ (см.~(6))\\
&&\\[-9pt]
\hline
&&\\[-9pt]
$E(k,\theta)$&$M(\alpha)$&$\displaystyle\fr{\alpha}{\theta q^{-1/k}-\theta-\alpha}$\\
&&\\[-9pt]
\hline
&&\\[-9pt]
$E(k,\theta),\  k\ge2$&$E(n,\alpha),\  n\ge2$&$\fr{\alpha}{\theta z(q,k,n)-\alpha}$ (см. (6))\\
\hline
\end{tabular}
\end{center}
\end{table*}

Пусть случайные величины $\lambda$ и~$\mu$ имеют распределения $R(a_\lambda,b_\lambda)$ и $R(a_\mu,b_\mu)$
соответственно ($0<a_\lambda<b_\lambda<a_\mu<b_\mu$, $a_\lambda/a_\mu<b_\lambda/b_\mu$). Воспользовавшись полученной
в разд.~2 формулой для плотности~$f_N(x)$ и соотношениями
\begin{gather*}
\il{A}{B}\fr{x\,dx}{(1+x)^2}=\ln\fr{1+B}{1+A}\,;\\
\il{A}{B}\fr{x^2\,dx}{(1+x)^2}=B+\fr{1}{1+B}-A-\fr{1}{1+A}-2\ln\fr{1+B}{1+A}\,;
\end{gather*}
получим
\begin{multline*}
\e N=\il{0}{\infty}xf_N(x)\,dx=\fr{c_\lambda c_\mu b_\mu^2}{2}\,\ln\fr{a_\mu(b_\mu-
a_\lambda)}{b_\mu(a_\mu-a_\lambda)}-{}\\
{}-\fr{c_\lambda c_\mu a_\lambda^2}{2}\,\ln
\fr{b_\mu- a_\lambda}{a_\mu-a_\lambda}+{}
\end{multline*}
\begin{multline*}
{}+\fr{c_\lambda c_\mu (b_\mu^2-a_\mu^2)}{2}\,\ln
\fr{b_\mu(a_\mu-a_\lambda)}{a_\mu(b_\mu-b_\lambda)}+{}\\
%\end{multline*}
{}+\fr{c_\lambda c_\mu }{2}
\left(2b_\lambda a_\lambda-b_\lambda^2+2\fr{b_\lambda}{c_\lambda}\right)\,\ln
\fr{b_\mu-b_\lambda}{a_\mu-a_\lambda}-{}\\
{}-\fr{c_\lambda c_\mu a_\mu^2}{2}\,\ln
\fr{a_\mu(b_\mu-b_\lambda)}{b_\mu(a_\mu-b_\lambda)}={}\\
{}=\fr{c_\lambda c_\mu}{2}\left[(b_\mu^2-a_\lambda^2)\,\ln(b_\mu-a_\lambda)-{}\right.\\
{}-
(a_\mu^2-a_\lambda^2)\ln\left(a_\mu-a_\lambda\right)+\left(a_\mu^2-b_\lambda^2\right)\ln
\left(a_\mu-b_\lambda\right)-{}\\
\left.{}-\left(b_\mu^2-b_\lambda^2\right)\ln\left(b_\mu-b_\lambda\right)\right]\,.
\end{multline*}

Аналогично получаем выражение для второго момента
\begin{multline*}
\e N^2=\il{0}{\infty}x^2f_N(x)\,dx={}\\
{}=c_\lambda c_\mu\left(a_\mu^2\ln\fr{a_\mu-a_\lambda}
{a_\mu-b_\lambda}-b_\mu^2\ln\fr{b_\mu-a_\lambda}{b_\mu-b_\lambda}\right)\,.
\end{multline*}

Таким образом, в данном примере существуют не только математическое ожидание и дисперсия, но и моменты любого
порядка.

\section{Квантильные характеристики}

Вполне традиционным методом <<борьбы>> с несуществующим математическим ожиданием является рассмотрение такой
характеристики центра распределения, как медианы. Заметим, что при изучении систем с положительным <<дефектом>> опять
сталкиваемся с возможной неопреде\-лен\-ностью этого объекта. Действительно, для собственной функции распределения
квантиль любого порядка существует. Однако для <<дефектных>> функций
распределения существенным является значение величины $\delta$, поскольку определены лишь квантили порядка 
$q\in(0,\,1-\delta)$. Соответствующее замечание относится и к квантильному аналогу дисперсии~--- интерквартильному 
размаху.

В табл.~\ref{t3sh} приведены значения квантилей~$x_q$ порядка $q\in(0,\, 1)$ функций распределения среднего числа
заявок~$N$ для распределений интенсивностей входящего потока~$\lambda$ и обслуживания~$\mu$, рассмотренных
авторами в работах~\cite{KuSh07, KuSh09a, KuSh09b} и отображенных в табл.~\ref{t1sh}.

При нахождении медианы $x_{1/2}$ необходимо выполнение условия $\delta<1/2$ (см.\ табл.~\ref{t2sh}), 
а интерквартильный размах
$x_{3/4}-x_{1/4}$ определен лишь при $\delta<1/4$.

Обозначим через $y(q,n)$ и $z(q,k,n)$ соответственно решения уравнений
\begin{equation}
\sum\limits_{m=0}^{n-1}\fr{y^m}{m!}=qe^y\,;\quad \sum\limits_{m=0}^{n-1}z^m=q(z+1)^{n+k-1}\,.
\label{e6sh}
\end{equation}

Так же как и в случае с моментными характеристиками, не составляет труда найти квантили любого порядка для распределения
с нулевым <<дефектом>>. Например, если входные параметры~$\lambda$ и~$\mu$ имеют равномерные распределения,
заданные в разд.~2, квантиль порядка~$q$ для распределения~$N$ находится из решения уравнения $F_N(x)=q$. Так, при
$$
\fr{a_\lambda(b_\mu-a_\mu)}{2a_\mu(b_\lambda-a_\lambda)}\le q\le\fr{b_\mu b_\lambda+a_\mu b_\lambda-2a_\lambda
b_\mu}{2b_\mu(b_\lambda-a_\lambda)}
$$
квантиль порядка~$q$ равняется
$$
x_q=\fr{2a_\lambda(1-q)+2b_\lambda q}{b_\mu+a_\mu-2a_\lambda(1-q)-2b_\lambda q}\,.
$$
Аналогично находятся квантили при
$$
0<q<\fr{a_\lambda(b_\mu-a_\mu)}{2a_\mu(b_\lambda-a_\lambda)}
$$
и
$$
\fr{b_\mu b_\lambda+a_\mu b_\lambda-2a_\lambda b_\mu}{2b_\mu(b_\lambda-a_\lambda)}<q<1\,.
$$

{\small\frenchspacing
{%\baselineskip=10.8pt
\addcontentsline{toc}{section}{Литература}
\begin{thebibliography}{99}

\bibitem{Shorgin05}
\Au{Шоргин С.\,Я.}
О байесовских моделях массового обслуживания~// 
II Научная сессия Института проблем информатики РАН: Тез. докл.~--- М.: ИПИ РАН, 2005. С.~120--121.

\bibitem{Apice06}
\Au{D'Apice C., Manzo~R., Shorgin S.}
Some Bayesian queueing and reliability models~// Electronic J. Reliability: Theory \& Applications, 
2006. Vol.~1. No.\,4.

\bibitem{KuSh07}
\Au{Кудрявцев А.\,А., Шоргин С.\,Я.}
Байесовский подход к анализу систем массового обслуживания и показателей надежности~// 
Информатика и её применения, 2007. Т.~1. Вып.~2. С.~76--82.

\bibitem{KuShShCh08}
\Au{Kudryavtsev A., Shorgin S., Shorgin~V., Chentsov~V.}
Bayesian queueing and reliability models~// Systems and means of informatics: Spec. issue: 
Mathematical and computer modeling in applied problems.~--- M.: IPI
RAN, 2008. P.~40--53.

\bibitem{KuSh09a}
\Au{Кудрявцев А.\,А., Шоргин С.\,Я.}
Байесовские модели массового обслуживания и надежности: экспоненциально-эрланговский случай~// Информатика и её 
применения, 2009. Т.~3. Вып.~1. С.~44--48.

\bibitem{KuSh09b}
\Au{Кудрявцев А.\,А., Шоргин В.\,С., Шоргин С.\,Я.}
Байесовские модели массового обслуживания и надежности: общий эрланговский случай~// 
Информатика и её применения, 2009. Т.~3. Вып.~4. С.~30--34.

\bibitem{GR71}
\Au{Градштейн И.\,С., Рыжик И.\,М.}
Таблицы интегралов, сумм, рядов и произведений.~--- М.: Наука, 1971. 1108~с.

\bibitem{BoPe95}
\Au{Бочаров П.\,П., Печинкин А.\,В.}
Теория массового обслуживания.~---
М.: РУДН, 1995. 529~с.

\bibitem{Shiryaev80}
\Au{Ширяев А.\,Н.}
Вероятность.~--- М.: Наука, 1980. 576~с.

\label{end\stat}


\bibitem{Feller67}
\Au{Феллер В.}
Введение в теорию вероятностей и ее приложения.~--- М.: Мир, 1967. Т.~2. 752~с.
 \end{thebibliography}
}
}

\end{multicols}