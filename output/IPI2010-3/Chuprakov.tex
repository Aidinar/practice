\def\stat{chupr}

\def\tit{НЕКОТОРЫЕ АСПЕКТЫ ВЫБОРА ТЕХНОЛОГИИ 
ДЛЯ~ПОСТРОЕНИЯ СИСТЕМ ОТОБРАЖЕНИЯ\\ ИНФОРМАЦИИ  СИТУАЦИОННОГО ЦЕНТРА}

\def\titkol{Некоторые аспекты выбора технологии для 
построения систем отображения информации 
ситуационного центра}

\def\autkol{А.\,А.~Зацаринный, К.\,Г.~Чупраков}
\def\aut{А.\,А.~Зацаринный$^1$, К.\,Г.~Чупраков$^2$}

\titel{\tit}{\aut}{\autkol}{\titkol}

%{\renewcommand{\thefootnote}{\fnsymbol{footnote}}\footnotetext[1]
%{Работа выполнена при поддержке РФФИ, проекты 08--07--00152-а, 08--01--00567-а 
%и 09--07--12032-офи-м. Статья написана на основе материалов доклада, 
%представленного на IV Международном семинаре  
%<<Прикладные задачи теории вероятностей и математической статистики, 
%связанные с моделированием информационных систем>> 
%(зимняя сессия, Аоста, Италия, январь--февраль 2010~г.).}}

\renewcommand{\thefootnote}{\arabic{footnote}}
\footnotetext[1]{Институт проблем информатики Российской академии наук, AZatsarinny@ipiran.ru}
\footnotetext[2]{Институт проблем информатики Российской академии наук, chkos@rambler.ru}

\Abst{Рассмотрены основные этапы выбора систем отображения 
информации ситуационных центров (СЦ), среди которых выделен этап выбора 
технологии. Рассмотрены технологии систем отображения информации (СОИ), 
используемые в СЦ. Предложены основные параметры 
СОИ и методический подход к выбору 
технологии на основании метода исключения по параметрам.}

\KW{системы отображения информации; ситуационный центр; технологии 
отображения информации; проблема выбора, метод исключения}

\vspace*{12pt}

       \vskip 14pt plus 9pt minus 6pt

      \thispagestyle{headings}

      \begin{multicols}{2}

      \label{st\stat}
      
\section{Введение}

В настоящее время резко возрос интерес к 
ситуационному подходу в различных сферах человеческой деятельности. 
Создаются СЦ государственных органов управления 
и силовых ведомств. На крупных предприятиях создаются специальные 
центры\linebreak для анализа работы подразделений с использо-\linebreak ванием методов 
ситуационного моделирования\linebreak для прогнозирования событий. 
В~образовательных учреждениях внедряются методы ситуационного 
обучения~[1--3]. 
     
     При этом основная задача современного СЦ~--- поддержка процессов 
принятия решений полномочным должностным лицом или органом на 
основе создания наглядных представлений (образов) возникающих ситуаций 
в подконтрольной сфере и визуализации результатов их анализа в наиболее 
удобном для принятия решения виде~[4]. Кроме того, СЦ должен обеспечивать 
предоставление руководителю не только результатов анализа текущей 
ситуации, но и тенденций ее развития, позволя\-ющих  
прогнозирование дальнейшего развития ситуации~[5, 6].
     
     В структурно-функциональном плане СЦ представляется как 
взаимоувязанная совокупность трех компонентов: ин\-фор\-ма\-ци\-он\-но-ана\-ли\-ти\-че\-ско\-го, 
ин\-фор\-ма\-ци\-он\-но-тех\-но\-ло\-ги\-че\-ско\-го и 
технического~\cite{1chu}. Первый компонент определяет перечень типовых 
функциональных задач, решаемых СЦ в\linebreak рамках заданной предметной 
области, второй~--- технологии их решения и соответствующие прог\-рам\-мные 
комплексы, третий~--- аппаратно-прог\-рам\-мные комплексы, 
непосредственно реали\-зу\-ющие задачи СЦ. 
    
    Одним из наиболее сложных элементов технического компонента 
является~СОИ. 
    
    Средства отображения визуальной информации~--- мощный и 
эффективный инструмент, позволяющий обеспечить виртуальное 
погружение оперативного штаба СЦ в условия реальной ситуации. 
И~эффективность принятого решения во многом зависит от того, насколько 
качественно произошло взаимодействие между информационной системой и 
лицом, принимающим решение. Поэтому правильный выбор средств 
отображения информации является одной из ключевых задач при реализации~СЦ. 
    
    Проблема выбора СОИ для СЦ 
пред\-став\-ля\-ет собой задачу группового выбора~\cite{7chu}, в которой 
учитываются и свойства самого оборудования, и вопросы организационного 
взаимодействия, и, конечно, полная стоимость решения. 
   
   Решение задачи выбора средств отображения визуальной информации 
предлагается осуществлять в три этапа:
   \begin{enumerate}[(1)]
\item постановка задачи и обоснование направлений ее решения;
\item выбор технологии для создания СОИ;
\item выбор технических средств, обеспечивающих создание 
СОИ в соответствии с заданными требованиями.
\end{enumerate}

   Выделение выбора технологии в отдельный этап связано с тем, что 
технология, на базе которой реализовано средство отображения, в 
значительной степени определяет его характеристики и условия 
эксплуатации. При этом появляется возможность взглянуть на суть 
технологии и выявить ее возможные слабые места во время реализации и 
эксплуатации системы отображения. Кроме того, такой подход позволяет 
значительно сузить круг потенциальных решений задачи выбора. 
Технология, признанная неприемлемой, позволяет исключить из 
рассмотрения все возможные решения, которые основаны на этой 
технологии. 
     
     В статье рассматриваются общие методические подходы к выбору 
технологии средства отображения, а также приводятся результаты 
сравнительного анализа технологий. 

\vspace*{-6pt}
   
\section{Постановка задачи и общий методический подход к выбору 
технологии}

\vspace*{-2pt}

   Создание любого программно-аппаратного комплекса должно начинаться 
с корректной постановки системотехнической задачи, включающей несколько 
шагов: 
   \begin{itemize}
\item осознание необходимости в реализации программно-аппаратного 
комплекса;
\item формирование причин, обоснование по\-лу\-ча\-емых преимуществ, 
выявление потенциальных рисков и слабых мест; 
\item оценку материальных, организационных и человеческих ресурсов, 
имеющихся в распоряжении и необходимых для решения задачи;
\item формирование общего замысла системотехнической задачи с 
определением целей, задач и критериев;
\item формирование условия задачи: требования, параметры, критерии. 
\end{itemize}

   При этом постановка задачи должна опираться на основополагающие 
принципы, описанные в~\cite{8chu, 9chu}.
   
   Общий методологический подход включает: 
\begin{itemize}
\item обоснование перечня требований и па\-ра\-мет\-ров, выдвигаемых со 
стороны трех компонентов программно-аппаратного комплекса: 
технологического, функционального, человеческого: $R = \{r_i \vert i = 1, 
\ldots , p\}$; 
\item определение совокупности исходных условий $M = \{m_j \vert j = 1, 
\ldots , q\}$, включая нормативно-правовые, особенности проектирования, 
возможности и тенденции технологий систем отоб\-ра\-жения;
\item формирование общих альтернатив $S = \{s_n \vert n =$\linebreak $= 1, \ldots , \nu\}$ 
построения и функционирования системы отображения, представляющей 
собой сложную систему из нескольких компонентов, описываемую 
множеством параметров $X_n = \{x_k^n\vert k=1, \ldots p+q\}$. Таким 
образом, существует инъекция~$\varphi$: из $S$ в~$p+q$-мерное 
пространство над полем вещественных чисел и каждая 
альтернатива~$s_n$ представляется в виде вектора $X_n = \varphi (s_n)$ из 
евклидова пространства размерности $p+q$ над полем вещественных чисел; 
\item выбор основного критерия~--- показателя эффективности~$E$ на 
множестве альтернатив: сопоставление каждой альтернативе из~$S$ 
некоторого вещественного числа~$E(s)$. Основной крите\-рий должен 
соответствовать обоснованному значению ценности той или иной 
альтернативы в данной конкретной задаче. 
\end{itemize}

    Таким образом, в общем виде постановка задачи заключается 
определении того номера~$z$, который максимизирует критерий 
ценности~$E$, соблюдая при этом все необходимые условия задачи~$M$: 
    \begin{equation*}
    E(X_z) =E(\varphi(s_z))=\underset{n}{\max} E(X_n)\,.
    \end{equation*}
    
    Сформулированная задача носит оптимизационный характер и потому с 
точки зрения теории исследования операций набор~$X_z$ является 
оптимальным набором параметров. 
    
    Второй особенностью поставленной задачи является ее сложность ввиду 
большого объема информации, который необходимо получить, 
формализовать и проанализировать. Для упрощения можно воспользоваться 
принципом маргиналь\-ности~\cite{8chu}, предусматривающим: 
    \begin{itemize}
    \item декомпозицию общей задачи на ряд частных;
\item оптимизацию одного из параметров при задании ограничения на 
остальные параметры;
\item обоснованное применение сразу нескольких критериев 
эффективности~$E$ c учетом особенностей решаемой задачи. 
\end{itemize}

В основу общего подхода к выбору средства отоб\-ра\-же\-ния информации и 
его технологии ложится совокупность требований, обусловленных 
нормативными документами, правовыми актами, техническими 
особенностями и экономической оправданностью~\cite{10chu} 
(рис.~\ref{f1chu}).

     Фактически выбор технологии осуществляется в три этапа.


     На первом~--- составляется полный перечень технологий, на базе 
которых возможна реализа-\linebreak\vspace*{-12pt}
\pagebreak

\end{multicols}
   

\begin{figure} %fig1
\vspace*{1pt}
\begin{center}
\mbox{%
\epsfxsize=164.754mm
\epsfbox{zac-1.eps}
}
\end{center}
\vspace*{-6pt}
\Caption{Общая схема методического подхода к выбору технологии системы 
отображения информации ситуационного центра
\label{f1chu}}
\vspace*{6pt}
\end{figure}

    
     \begin{multicols}{2}
     
\noindent
ция аппаратно-программного комплекса. При этом 
необходимо учитывать существующие требования конкретной задачи, 
которым результирующая сис\-те\-ма должна соответствовать безусловно. 
     
     На втором~--- производится выбор показателей для сравнительной 
оценки отобранных технологий. В~данной статье предложен основной набор 
таких показателей. 
     
     На третьем~--- реализуется некоторая методика, предлагающая
алгоритм рационального выбора решения. 

\section{Особенности современных технологий систем 
отображения}

   Современные системы отображения информации ситуационных центров 
строятся с использованием следующих технологий: 
   \begin{itemize}
\item плазменной; 
\item жидкокристаллической; 
\item проекционной; 
\item светодиодной.
\end{itemize}

   В этот перечень не вошли технологии печатания на бумажных носителях, 
которые носят достаточно специфичный характер и требуют, вообще говоря, 
иного подхода к их выбору, а также технология CRT, или ЭЛТ (cathode ray 
tube, электронно-лучевая трубка), которая не используется при реализации 
ситуационных центров. 

\vspace*{-6pt}
    
\subsection{Плазменная технология} %3.1

\vspace*{-2pt}



     К преимуществам плазменной технологии можно отнести почти 
мгновенную реакцию на изменения изображения, практически полные углы 
обзора, качественную цветопередачу. Кроме того, плазмен\-ная технология 
способна давать яркость более 1000~кд/м$^2$. 
     
     К основным недостаткам плазменной технологии можно отнести 
существенное энер\-го\-по\-треб-\linebreak\vspace*{-12pt}
\pagebreak

\noindent
ле\-ние и высокое тепловыделение. Поэтому любая 
плазменная панель снабжается эффективной системой охлаждения и имеет 
достаточно низкий КПД. 
     
     Другим серьезным минусом плазменной технологии является 
относительно недолгий срок службы, особенно в тех случаях, когда на 
дисплей на длительное время выводится статичное изображение. Причиной 
тому~--- ускорение выгорания люминофоров. 
     
     Серьезным ограничением для производителей плазменных дисплеев 
является минимальный размер пиксела, который не может быть уменьшен 
ниже некоторого предела в силу физико-химических взаимодействий, 
происходящих в матрице. По этой причине сегодня не существует 
плазменных дисплеев с диагональю менее 32~дюймов. Зато для больших 
диагоналей (65~дюймов и более) плазменные дисплеи оказались 
сравнительно недорогими и надежными.     

\subsection{Жидкокристаллическая технология} %3.2

     Жидкокристаллическая технология прошла один из самых длинных 
путей в своем развитии и потому успела устранить большинство своих 
недостатков, предоставляя взамен множество уникальных преимуществ~[11]. 
Технология позволяет изготавливать средства отображения практически 
любого размера, сохраняя при этом наибольшую плотность отображения 
информации. Ввиду относительно небольшого тепловыделения и 
энер\-го\-по\-треб\-ле\-ния дисплеев на базе ЖК-тех\-но\-ло\-гии именно они 
используются для работы в тяжелых условиях окружающей среды. 
     
     Преимуществом ЖК-технологии перед плазменной является более 
устойчивая работа со статичным изображением. Тем не менее и она имеет 
ограничения по этому параметру~--- со статичным изображением 
рекомендуется работать не более 20~ч в сутки. 
     
     К другим преимуществам ЖК-технологии стоит отнести высокий 
уровень яркости и контраст\-ности, а также большие углы обзора, что 
позволяет работать с дисплеями даже при высокой яркости внешнего 
освещения. 
     
     К недостаткам технологии относится неравномерность подсветки, 
резкое увеличение стоимости при производстве матриц большой площади из-
за повышения риска возникновения битых пикселов. Так же, как и при 
плазменной технологии, контакты подходят к матрице сбоку, поэтому при 
построении на базе ЖК-технологии полиэкранов образуется заметный шов.

\subsection{Проекционная технология} %3.3

     Проекционная технология сегодня реализуется в двух вариантах: 
     \begin{itemize}
\item составная~--- проектор и экран;
\item видеокубы.
\end{itemize}
     
     При этом система проектор--экран может быть фронтальной и задней 
проекции. Разница в том, что в первом случае наблюдатель и проектор 
находятся в одном полупространстве относительно плоскости экрана и 
используется отражающий экран. В~другом случае наблюдатель и проектор 
находятся по разные стороны от просветного экрана. 
     
     Видеокубы представляют собой системы обратной проекции, 
служащие как настроенные готовые дисплеи, использующие DLP-проекторы 
внутри и оборудованные сложной электроникой и механизмами, 
необходимыми для удобного и эффективного использования. 
   
   Технология формирующей матрицы видеопроектора бывает следующих 
видов~[12]: 
   \begin{itemize}
\item CRT (Cathode Ray Tube);
\item DLP (Digital Light Processing);
\item 3LCD (Liquid Cristal Display);
\item D-ILA (Image Lighting Amplifier). 
\end{itemize}
   
   Наиболее часто используемой является технология DLP, поскольку она 
имеет ряд уникальных преимуществ, необходимых для приложений 
критичной важности и круглосуточной эксплуатации. Сегодня наблюдается 
плавный переход в DLP-тех\-но\-ло\-гии от использования UHP-ламп к 
   LED-под\-свет\-ке, что обеспечивает значительно более глубокие цвета, отказ 
от подвижных компонентов, а также не требует обслуживания в течение не 
менее 5~лет использования. 
 
\subsection{Светодиодная технология} %3.4

     К плюсам светодиодной технологии можно отнести высокую 
надежность, долгий срок службы, устойчивость к низким и очень низким 
температурам, высокую прочность. Однако дисплеи, созданные на базе 
матрицы светодиодов, обладает крайне низкой информативной емкостью по 
сравнению с предыдущими технологиями. Кроме того, светодиоды быстро 
теряют свои свойства при высоких температурах и потому требуют особого 
внимания к системам охлаждения. 
     
     Светодиодные матрицы сегодня используются в проекционной и 
жидкокристаллической технологии в качестве модулей подсветки вместо 
обычных ламп. Это позволяет значительно улучшить качество изображения, 
а в случае с проекционной технологией обеспечивает улучшение 
эксплуатационных характеристик устройств. 
    
\section{Сравнительная оценка технологий}

\subsection{Основные критерии сравнения средств отображения 
информации} %4.1

     В качестве главных критериев для сравнительной оценки технологий 
рассмотрим основные характеристики дисплеев, важные с точки зрения 
установки и использования.

\textbf{Размер средства отображения}~--- один из наиболее существенных 
параметров, который определяет проектирование средства отображения. Под 
ним понимается геометрический размер активной (информативной) части 
поверхности СОИ. Возможно построение СОИ практически любых размеров. 

\textbf{Тип использования: }
\begin{itemize}
\item индивидуальное (1 наблюдатель);
\item коллективное (2 и более наблюдателей).
\end{itemize}

Данный параметр определяет сценарий использования СОИ относительно 
числа одновременных наблюдателей. 
     
     После того как число пользователей определено, необходимо 
обеспечить соответствие СОИ правилам и стандартам, прописанным в 
ГОСТах~\cite{13chu, 14chu}. 

\textbf{Характер использования:}
\begin{itemize}
\item профессиональное;
\item бытовое;
\item смешанное.
\end{itemize}
   
   Под профессиональными средствами отображения понимаются 
устройства, необходимые для реализации программно-аппаратных 
комплексов в рамках задач, решаемых организационными структурами. 
   
   Бытовые устройства используются для личных целей, как правило, в 
домашних условиях. 
   
   Пример смешанного характера использования~--- настольные 
   ЖК-мониторы.
   
   Характер использования влечет за собой наличие необходимых 
интерфейсов подключения, требования к цветопередаче и стоимости.
     
     Под \textbf{форматом дисплея} понимается соотношение высоты и 
ширины активной поверхности средства отображения. На сегодняшний день 
существует достаточно ограниченное число используемых форматов, 
которые задаются производителями той или иной технологии. 
     
     Другим определяющим формат параметром является формат 
отображаемого сигнала. Можно выделить наиболее популярные форматы: 
     \begin{itemize}
\item 16:9;
\item 4:3;
\item 1:1;
\item 5:4;
\item 16:10.
\end{itemize}

Преобладают первые два формата как основные форматы изображения и 
видеосигнала и наблюдается тенденция перехода к широким форматам ввиду 
того, что большинство сигналов высокого\linebreak разрешения имеют формат 16:9, а 
также ввиду попыток найти соотношение сторон, со\-от\-вет\-ст\-ву\-ющее 
усредненным параметрам человеческого восприятия. 

     \textbf{Тип интерфейса.} Подключение устройств к средствам 
отображения накладывают требования на сов\-мес\-ти\-мость интерфейсов как по 
физическому набору контактов и форм-факторов, так и по типу 
передаваемого сигнала. По второму параметру интерфейсы делятся на два 
вида:
     \begin{itemize}
\item аналоговые (S-Video, D-Sub (15~pin), композитный, компонентный, 
DVI-I); 
\item цифровые (DVI-D, HDMI, Ethernet).
 \end{itemize}

     \textbf{Характер входного сигнала} определяет динамику 
изображения:
     \begin{itemize}
\item статичное;
\item динамичное;
\item любое.
\end{itemize}

Существуют примеры, когда средства отображения не могут показывать 
динамично меняющуюся картинку без заметных искажений изображения. 
С~другой стороны, далеко не все дисплеи могут работать со статичным 
изображением. 

     \textbf{Число источников} может меняться от одного до нескольких 
сотен. В зависимости от него средство отображения информации обрастает 
вычислительными мощностями, интерфейсными блоками, которые 
обеспечивают одновременную работу с необходимым набором входных 
сигналов. 
     
     \textbf{Число отображаемых окон (источников)} характеризует 
функциональность дисплея и определяется возможностью одновременного 
вывода нескольких различных источников в нескольких отдельных окнах. 
Число выводимых окон может меняться от одного до нескольких сотен. 

\textbf{Цветопередача.} Средства отображения информации могут быть:
\begin{itemize}
\item цветные;
\item черно-белые.
\end{itemize}

Согласно~\cite{15chu, 16chu}, кодирование информации в диспетчерских с 
помощью цвета не должно использовать более 11~цветов, а при 
необходимости быстрого поиска~--- более~6. С~другой стороны, 
обеспечение правильной цветопередачи позволяет адекватно оценивать 
объект наблюдения и уменьшает нагрузку на глаза.

     \textbf{Информационная емкость} средства отображения\linebreak 
определяется числом его <<атомарных>> единиц~--- пикселов или точек, 
помноженным на его способность к изменению информации во времени. 
Теоретически возможно построение СОИ со сколь\linebreak угодно большой 
разрешающей спо\-соб\-ностью. Спо\-соб\-ность изменения изображения во времени 
разумно ограничить числом кадров в секунду, воспринимаемых человеком. 
     
     \textbf{Уровень яркости} может изменяться от нуля до нескольких 
тысяч кд/ м$^2$. Этот параметр достаточно сильно влияет на восприятие и 
объем передаваемой информации (в том числе из-за того, что яркостное 
кодирование~--- преобладающая форма передачи информации). Большая 
яркость не всегда является плюсом~--- при длительном наблюдении 
необходимо соблюдать баланс уровня яркости между излучением СОИ и 
освещением помещения. 

     \textbf{Угол обзора}~--- характеристика, говорящая о возможности 
использовать средство отображения при наблюдении с ненулевого угла 
относительно перпендикуляра к поверхности экрана в точке наблюдения. 
Требования к данному параметру содержатся в~\cite{15chu, 16chu}. 
   
   По этому параметру устройства можно разделить на две группы:
   \begin{itemize}
\item прямого наблюдения~--- предназначенные для работы в рамках 
узкого конуса;
\item произвольного наблюдения~--- предназначенные для работы с 
любого угла наблюдения. 
\end{itemize}

     \textbf{Время отклика} характеризует способность дисплея к 
изменению изображения после изменения подаваемого сигнала. По этому 
параметру средства отображения можно разделить на три группы: 
     \begin{itemize}
\item статичные~--- несколько секунд;
\item условно-динамичные~--- несколько миллисекунд;
\item динамичные~--- незаметное для наблюдателя время изменения 
любого контента.
\end{itemize}

   \textbf{Интерактивность} говорит о возможности работать с 
отображаемой информацией на уровне естественных движений, позволяя 
наблюдающим\linebreak видеть ее в динамике. Интерактивность~--- это не свойство 
самих дисплеев, а методика, включающая три компонента: средство 
отображения, сенсорную матрицу, совмещенную с активной 
визуализирующей поверхностью средства отображения, и вычислительные 
мощности, привязанные к отображаемой информации. Вычислительные 
мощности нужны для того, чтобы производить сопоставление каждой точки 
сенсорной матрицы с точкой в отображаемом контенте, а также для того, 
чтобы изменять контент в режиме реального времени.

   \textbf{Срок эксплуатации} ограничивается возможностями технологии 
отображения и может изменяться от нескольких месяцев до нескольких лет. 
Помимо срока эксплуатации, существует еще и срок моральной 
актуальности, который для средств отображения коллективного 
использования приближается к 5~годам. За это время технология устройств 
уходит далеко вперед, открывая новые возможности для решения задач 
визуализации. 

   \textbf{Тип монтажа} определяет точку опоры для средства отображения. 
Это затрагивает вопросы механической нагрузки, а также заставляет решать 
проб\-ле\-му установки. Выделенным способом монтажа считается встроенный 
дизайн средства отоб\-ра\-же\-ния, когда оно фактически является частью другого 
устройства большего размера. По типу монтажа СОИ бывают:
   \begin{itemize}
\item встроенные;
\item настенные;
\item напольные;
\item настольные;
\item потолочные/подвесные.
\end{itemize}

   \textbf{Требуемое пространство инсталляции} характеризует 
необходимость со стороны дисплея в за\-ни\-ма\-емом пространстве. 
Встраиваемые средства отображения не требуют дополнительного 
пространства, поскольку включены в некоторый корпус другого устройства. 
Существуют средства отображения, наоборот, требующие дополнительного 
пространства, которое необходимо для обеспечения <<комфортной>> среды 
эксплуатации и сервисных нужд. Технологическое пространство 
необходимо полиэкранным системам с большим тепловыделением. Итак, 
возможны три варианта:
   \begin{itemize}
\item СОИ не требует пространства;
\item СОИ требует пространства, равного собственному объему;
\item СОИ требует дополнительного технологического пространства.
\end{itemize}

   По \textbf{обслуживанию} все СОИ делятся на три группы в зависимости 
от необходимости обеспечивать заданный уровень всех параметров 
устройства:
   \begin{enumerate}[(1)]
\item необслуживаемые;
\item обслуживаемые при необходимости;
\item требующие периодического обслуживания.
\end{enumerate}

     К последним относятся системы обратной проекции, которые требуют 
практически ежегодной замены ламп. Внедрение LED-тех\-но\-ло\-гии в 
     DLP-ку\-бы позволяет этим устройствам обходиться без обслуживания в 
течение нескольких лет. 

\textbf{Ремонтопригодность:}

\noindent
\begin{itemize}
\item ремонтопригодные;
\item неремонтопригодные.
\end{itemize}

При проектировании средств визуализации необходимо учитывать 
возможность восстановления и ремонта при возникновении
неисправности. В~случае выхода из строя плазменной панели ее с большой 
долей вероятности придется заменять новой, в то время как видеокубы, 
состоящие из большого числа отдельных компонентов, могут быть 
отремонтированы только заменой неисправных деталей. 

\textbf{Сложность.} СОИ может быть двух типов:
\begin{itemize}
\item простое;
\item составное.
\end{itemize}
   
   Под сложностью средства визуализации понимается число отдельных 
самостоятельных компонентов, входящих в его состав. Например, полиэкран 
является сложным СОИ. 
   
\textbf{Защищенность} связана с различными условиями эксплуатации 
средств отоб\-ра\-же\-ния. Основными ограничениями для эксплуатации могут 
стать температура, влажность, пыль, а также действия недобро\-желательно 
настроенных лиц. По этому параметру различают следующие виды средств 
отоб\-ра\-жения:
\begin{itemize}
\item незащищенные;
\item влагозащищенные;
\item температурно-защищенные;
\item пылезащищенные;
\item взрывозащищенные;
\item антивандальные;
\item противоударные.
\end{itemize}
   
   К незащищенным СОИ относятся устройства, пригодные к эксплуатации 
только в теплых помещениях с невысокой влажностью. Защищенные от 
какого-либо фактора СОИ обычно либо обладают этим свойством 
изначально, либо снабжаются специальными оболочками, которые 
предупреждают то или иное воздействие. 
   
   Особо стоит группа взрывозащищенных СОИ~--- устройств, которые не 
могут стать причиной взрыва/возгорания из-за электрических искр в корпусе. 
Эти устройства необходимы в горнодобывающих шахтах, на нефтяных 
вышках, на предприятиях химической и перерабатывающей 
промышленности. 

   \textbf{Энергопотребление} определяет сразу несколько параметров. 
   Во-первых, оно фигурирует при расчете стоимости владения, во-вторых, 
свидетельствует об уровне тепловыделения при прочих равных условиях. 
   
   \textbf{Непрерывность эксплуатации} характеризует возможности 
средства отображения по длительности активной работы: 
   \begin{itemize}
\item абсолютно непрерывная эксплуатация;
\item непрерывная эксплуатация;
\item условно непрерывная эксплуатация;
\item прерывистая эксплуатация.
\end{itemize}
   
   Абсолютно непрерывная эксплуатация~--- это возможность непрерывно 
использовать устройство от момента первого включения до окончания срока 
службы. 
   
   Непрерывная эксплуатация~--- возможность использовать непрерывно 
между необходимыми технологическими процессами, связанными с СОИ 
(например, замена ламп). 
   
   Условно-непрерывная~--- возможность непрерывно использовать 
устройство при соблюдении ряда условий эксплуатации. Например, тип 
контента. 
   
   Прерывистая эксплуатация~--- необходимость обеспечения периодов 
<<отдыха>> между сеансами эксплуатации. 

\subsection{Некоторые результаты сравнительной оценки технологий 
средств отображения информации} %4.2

     В табл.~1 сведены основные параметры для каж\-дой технологии, что 
позволяет сделать первичную оценку возможности использования той или 
иной технологии в конкретной задаче. 


     Данная таблица может использоваться для решения задачи выбора 
технологии по следующему алгоритму.
     \begin{enumerate}[1.]
\item Задать порядок строк согласно важности того или иного параметра для 
конкретной задачи. 
\item Выделить безусловные параметры~--- те, которые должны обязательно 
выполняться для решения поставленной задачи. 
\item Проходя по таблице сверху вниз по безусловным параметрам, удалить из 
начального списка технологий те, которые не удовлетворяют требованиям 
задачи.
\end{enumerate}

\end{multicols}

%\begin{table}

\begin{center}
%\Caption{
\addtocounter{table}{1}
{\tablename~\thetable}\ \ {\small Соответствие основных критериев и технологий}
%\label{t1chu}
%}
\vspace*{2ex}

{\small
\tabcolsep=4.05pt
\begin{tabular}{|l|c|c|c|c|}
\hline
\multicolumn{1}{|c|}{Параметр}&\multicolumn{4}{c|}{Технология}\\ 
\cline{2-5} 
&Плазменная&ЖК (LCD)&Проекционная (DLP)&Светодиодная (LED)\\ 
\hline
\tabcolsep=0pt\begin{tabular}{l}Размер (одного\\ модуля/устройства)\end{tabular}&
32$^{\prime\prime}$--150$^{\prime\prime}$&0$^{\prime\prime}$--108$^{\prime\prime}$
&\tabcolsep=0pt\begin{tabular}{c}50$^{\prime\prime}$--100$^{\prime\prime}$ (для систем\\ обратной проекции)\end{tabular}&Без ограничений\\ 
\hline
Тип&Коллективное&
\tabcolsep=0pt\begin{tabular}{c}Коллективное,  \\индивидуальное\end{tabular}&Коллективное&Коллективное\\ 
\hline
Класс&
\tabcolsep=0pt\begin{tabular}{c}Профессиональные,\\ бытовые\end{tabular}&
\tabcolsep=0pt\begin{tabular}{c}Профессиональные,\\ бытовые,\\ смешанные\end{tabular}&
\tabcolsep=0pt\begin{tabular}{c}Профессиональные, \\ бытовые\end{tabular}&Профессиональные\\ 
\hline
Формат&16:9&4:3/16:9/16:10&4:3/16:9&1:1\\ 
\hline
\tabcolsep=0pt\begin{tabular}{l}Тип\\ интерфейса\end{tabular}&
\tabcolsep=0pt\begin{tabular}{c}Аналоговый,  \\ цифровой\end{tabular}&
\tabcolsep=0pt\begin{tabular}{c}Аналоговый,  \\ цифровой\end{tabular}&
\tabcolsep=0pt\begin{tabular}{c}Аналоговый,  \\ цифровой\end{tabular}&
\tabcolsep=0pt\begin{tabular}{c}Аналоговый,  \\ цифровой\end{tabular}\\ 
\hline
\tabcolsep=0pt\begin{tabular}{l}Характер\\ входного\\ сигнала\end{tabular}&Динамичный&Нестатичный&
\tabcolsep=0pt\begin{tabular}{c}Динамичный,  \\ статичный\end{tabular}&
\tabcolsep=0pt\begin{tabular}{c}Динамичный,  \\ статичный\end{tabular}\\ 
\hline
Цветопередача&1--64 млрд цветов&16,7 млн цветов&16,7 млн цветов&До 281 трлн цветов\\ 
\hline
\tabcolsep=0pt\begin{tabular}{l}Информационная\\ емкость,\\ млн пикселов/м$^2$\end{tabular}&
До 4,25 &До 18,75&До 3 &До 0,01\\ 
\hline
\tabcolsep=0pt\begin{tabular}{l}Уровень\\ яркости\end{tabular}&
1000--1500 кд/м$^2$&250--700 кд/м$^2$&200--700 кд/м$^2$&2000--4000 кд/м$^2$\\ 
\hline
Угол обзора&180/180&178/178&180/180&160/160\\ 
\hline
Время реакции&$< 1$~мс&4--16~мс&$< 1$~мс&$< 1$~мс\\ 
\hline
Интерактивность&Возможна&Возможна&Возможна&Невозможна\\ 
\hline
\tabcolsep=0pt\begin{tabular}{l}Срок\\ эксплуатации\end{tabular}&
50--60 тыс.\ ч&50--60 тыс. ч&80--100 тыс. ч&50--100 тыс. ч\\ 
\hline
Тип монтажа&
\tabcolsep=0pt\begin{tabular}{c}Настенный,\\ напольный,\\ потолочный,\\ настольный\end{tabular}&
\tabcolsep=0pt\begin{tabular}{c}Настенный,\\ напольный,\\ потолочный, \\настольный\end{tabular}&
\tabcolsep=0pt\begin{tabular}{c}Напольный,\\ настенный,\\ потолочный\end{tabular}&
\tabcolsep=0pt\begin{tabular}{c}Напольный,  \\ подвесной\end{tabular}\\ 
\hline
\tabcolsep=0pt\begin{tabular}{l}Требуемое\\ пространство\\ инсталляции\end{tabular}&
\tabcolsep=0pt\begin{tabular}{c}Без/с\\ технологическим\\ пространством\end{tabular}&
\tabcolsep=0pt\begin{tabular}{c}Без/с\\ технологическим \\пространством\end{tabular}&
\tabcolsep=0pt\begin{tabular}{c}Без/с\\ технологическим\\ пространством\end{tabular}&
\tabcolsep=0pt\begin{tabular}{c}Без/с\\ технологическим\\ пространством\end{tabular}\\ 
\hline
Обслуживание&\tabcolsep=0pt\begin{tabular}{c}Необслуживаемые,\\ обслуживаемые\\ при необходимости\end{tabular}&
\tabcolsep=0pt\begin{tabular}{c}Необслуживаемые, \\обслуживаемые\\ при необходимости\end{tabular}&
\tabcolsep=0pt\begin{tabular}{c}Периодически\\ обслуживаемые,\\ 
обслуживаемые\\ при необходимости\end{tabular}&
Необслуживаемые\\ 
\hline
Ремонтопригодность&Неремонтопригодны&Неремонтопригодны&Ремонтопригодны&Неремонтопригодны \\
\hline
Сложность&Простой/сложный&Простой/сложный&Простой/сложный&Простой/сложный\\ 
\hline
Защищенность&
\tabcolsep=0pt\begin{tabular}{c}Кроме\\ взрывозащищенных\end{tabular}&
\tabcolsep=0pt\begin{tabular}{c}Любой\\ уровень\\ защищенности\end{tabular}&Незащищенные&
\tabcolsep=0pt\begin{tabular}{c}Кроме \\ взрывозащищенных\end{tabular}\\ 
\hline
Энергопотребление&100--1500 Вт&50--1000 Вт&50--200 Вт&300--1000 Вт\\ 
\hline
\tabcolsep=0pt\begin{tabular}{l}Непрерывность\\ эксплуатации\end{tabular}&
\tabcolsep=0pt\begin{tabular}{c}Условно\\ непрерывная\end{tabular}&
\tabcolsep=0pt\begin{tabular}{c}Условно\\ непрерывная \end{tabular}&Непрерывная &
\tabcolsep=0pt\begin{tabular}{c}Абсолютно \\ непрерывная\end{tabular}\\ 
\hline 
\end{tabular} 
}
\end{center} 

\vspace*{18pt}
%\end{table} 

\begin{multicols}{2}

\noindent
\begin{enumerate}
\setcounter{enumi}{3}
\item Получить в итоге множество технологий, которые удовлетворяют 
условиям задачи. 
\item В случае если конечное множество технологий окажется пустым, 
пересмотреть условие задачи. 
\item При необходимости множество параметров может быть расширено на 
основании достоверной информации о свойствах технологий. 
\end{enumerate}
   
   Положительной стороной такого метода является основание выбора на 
анализе слабых мест, а не на поиске сильных. На практике это обеспечит 
значительное уменьшение вероятности выхода из строя и повышение общей 
надежности. 
   
   После решения проблемы выбора технологии и его обоснования следует 
приступить к третьему шагу~--- выбору конкретных решений, основанных на 
базе множества технологий, полученных в результате использованного 
метода. Выбор конкретных сис\-те\-мо\-тех\-ни\-че\-ских решений может быть 
реализован на основании предложенного в статье метода исключений, где в 
качестве таблицы должна быть использована собственная таб\-ли\-ца 
па\-ра\-мет\-ров. Другим путем решения проблемы выбора может послужить 
метод анализа иерархий, пример использования которого для выбора 
оборудования приведен в~\cite{18chu}.

\section{Заключение}

     При решении проблемы группового выбора обору\-до\-ва\-ния для 
реализации системы отображения информации одной из подзадач является 
выбор технологии этой системы. В~настоящее время существует ряд 
технологий, которые используются в ситуационных центрах,~--- плазменная, 
жид\-ко\-крис\-тал\-ли\-че\-ская, проекционная, светодиодная. Каж\-дая из них 
предлагает свой способ создания подвижного изображения для демонстрации 
поданного на нее сигнала. Каждая из технологий обладает своими 
достоинствами и недостатками, поэтому потенциально является решением 
некоторой задачи группового выбора. 
     
     Как показали исследования, наиболее предпочтительными являются 
жидкокристаллическая и проекционная технология. Первая из них более 
удобна для реализации средств отображении на базе одного дисплея и 
оборудования рабочих мест операторов и руководителей. 
Жидкокристаллическая технология является наиболее универсальной~--- на 
ее основе возможно решение практически любой задачи и с любыми 
ограничениями по безопас\-ности. 
     
     Проекционная технология является незаменимой для круглосуточных 
приложений, где требуется отображение статичной информации. Такое 
требование, как правило, предъявляется к системам отображения 
ситуационных центров диспетчерского типа, функционирующих в 
непрерывном\linebreak режиме. Кроме того, проекционная технология~--- 
единственная, которая позволяет создавать полиэкранные системы высокой 
информационной емкости без заметных рамок между отдельными\linebreak 
дисплеями. 
     
     Плазменная и светодиодная технологии пока в меньшей мере 
востребованы в рамках приложений ситуационного центра, однако 
существующие тенденции развития технологий~\cite{17chu} указывают на 
то, что возможны смещения в предпочтениях техно\-логий. 

{\small\frenchspacing
{%\baselineskip=10.8pt
\addcontentsline{toc}{section}{Литература}
\begin{thebibliography}{99}
     
\bibitem{1chu}
\Au{Зацаринный А.\,А., Сучков А.\,В., Босов~А.\,В.}
Ситу\-ационные центры в современных информа\-ци\-он\-но-те\-ле\-ком\-му\-ни\-ка\-ци\-он\-ных 
системах специального назначения~// ВКСС 
Connect! (Ведомственные\linebreak корпоративные сети и системы), 2007. №\,5(44). 
С.~64--76.

\bibitem{2chu}
\Au{Ильин Н.\,И.}
Основные направления развития ситуационных центров органов 
государственной власти~// ВКСС Connect! (Ведомственные 
корпоративные сети и системы), 2007. № 6(45). С.~2--9.

\bibitem{3chu}
\Au{Зацаринный А.\,А.}
Тенденции развития ситуационных центров как компонентов 
информационно-телекоммуникационных систем в условиях глобальной 
информатизации общества~// Материалы XXXV междунар. 
конф. <<Информационные технологии в науке, образовании, 
телекоммуникации и бизнесе>> (IT\;+\;S\&E`08), Украина, Ялта-Гурзуф, 
20--30~мая 2008~г.

\bibitem{4chu}
\Au{Бадалов А.\,Ю., Баранов А.\,В.}
Некоторые подходы к созданию ситуационных центров~// Системы 
высокой доступности, 2006. Т.~2. №\,3--4. С.~73--76.

\bibitem{5chu}
Ситуационные центры (СЦ) и их история. {\sf  
http://\linebreak ta.interrussoft.com/s\_centre.html}.

\bibitem{6chu}
\Au{Филиппович А.\,Ю.}
Ситуационная система~--- что это такое?~// PCWeek/RE, 2003. №\,26. 

\bibitem{7chu}
\Au{Миркин Б.\,Г.}
Проблема группового выбора.~--- М.: Наука, 1974.

\bibitem{8chu}
\Au{Зацаринный А.\,А., Ионенков Ю.\,С., Кондрашев~В.\,А.}
Об одном подходе к выбору системотехнических решений построения 
ин\-фор\-ма\-ци\-он\-но-те\-ле\-ком\-му\-ни\-ка\-ци\-он\-ных систем~// Системы и средства 
информатики. Вып.~16.~--- М.: Наука, 2006. С.~65--72.

\bibitem{9chu}
\Au{Зацаринный А.\,А.}
Организационные принципы сис\-тем\-но\-го подхода к разработке, 
проектированию и внедрению современных информационно-те\-ле\-ком\-му\-ни\-ка\-ци\-он\-ных сетей~// 
ВКСС Connect! (Ведомственные 
корпоративные сети и системы), 2007. №\,1(40). С.~60--67.

\bibitem{10chu}
\Au{Зацаринный А.\,А., Ионенков Ю.\,С.}
Некоторые аспекты выбора технологии построения 
информационно-телекоммуникационных сетей~// Системы и средства 
информатики. Вып.~17.~--- М.: Наука, 2007. С.~5--16.

\bibitem{11chu}
\Au{Рыбаков М.}
Поле аукнется~--- точка откликнется~// Hard'n'Soft, 2006. №\,2.

\bibitem{12chu}
\Au{Королев О., Чупраков К.}
Мастера яркого цвета~// Hard'n'Soft, 2006. №\,11.

\bibitem{13chu}
ГОСТ Р 50948-2001 Средства отображения информации индивидуального 
пользования. Общие эргономические требования и требования 
безопасности. Переиздание, авг.\ 2006.~--- М.: Стандартинформ, 2006.

\bibitem{14chu}
ГОСТ Р 52870-2007 Средства отображения информации коллективного 
пользования. Требования к визу-\linebreak\vspace*{-12pt}
\pagebreak

\noindent
альному отображению информации и 
способы измерения.~--- М.: Стандартинформ, 2008.

\bibitem{15chu}
ГОСТ 12.2.032-78 Система стандартов безопасности труда. Рабочее место 
при выполнении работ сидя. Общие эргономические требования. Утв.\ 
по\-ста\-нов\-ле\-ни\-ем Государственного комитета стандартов Совета 
Министров СССР от 26~апреля 1978~г. №\,1102. 



\bibitem{16chu}
ГОСТ 12.2.033-78 Система стандартов безопасности труда. Рабочее место 
при выполнении работ стоя. Общие эргономические требования. Утв. 
по\-ста\-нов\-ле\-ни\-ем Государственного комитета стандартов Совета 
Министров СССР от 26~апреля 1978~г. №\,1100.


\bibitem{18chu}
\Au{Зацаринный А.\,А., Ионенков Ю.\,С.}
Методика выбора технических средств для построения 
телекоммуникационных сетей~// Системы и средства информатики, 2009. 
Вып.~19. №\,2. С.~4--14.

\label{end\stat}

\bibitem{17chu}
\Au{Чупраков К.}
Технологический анализ коллективных средств отображения информации 
в ситуационном центре~// Автоматизация в промышленности, 2009. 
№\,11. С.~27--30.
 \end{thebibliography}
}
}

\end{multicols}