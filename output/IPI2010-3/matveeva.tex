
\def\NCAL{{\cal N}}

\newcommand{\suml}[2]{\sum\limits_{#1}^{#2}}   %сумма
\def\sumi{\sum\limits_{i=1}^n}   %сумма по i=1 до n
\newcommand{\maxl}[1]{\max\limits_{#1}}
\newcommand{\minl}[1]{\min\limits_{#1}}
\def\limn{\lim\limits_{n\to\infty}}
\newcommand{\liml}[1]{\lim\limits_{#1}}
\newcommand{\limi}[1]{\liminf\limits_{#1}}
\newcommand{\lims}[1]{\limsup\limits_{#1}}
\newcommand{\cupl}[2]{\bigcup\limits_{#1}^{#2}}


%\newcommand{\rr}{{\rm I\hspace{-0.7mm}\rm R}^3}     % трехмерное пространство
\newcommand{\rr}{{\mathbb{R}}^3}     % трехмерное пространство
%\def\P{\mathop{\bf P}}
\def\limn{\lim\limits_{n\to\infty}}
%\newcommand{\naj}{\hbox{\aj N}}     % поле натуpальных чисел
\newcommand{\rn}{\mathbb{R}^{\mathbb{N}}}     % N-мерное пространство

%\newcommand{\proofbegin}{{\sc Доказательство.}}
%\newcommand{\proofend}{{\hfill$\Box$}}
\newcommand{\sqq}{\hbox{\vrule\vbox{\hrule\phantom{o}\hrule}\vrule}}

\def\stat{matv}

\def\tit{СЕТИ МАССОВОГО ОБСЛУЖИВАНИЯ С~НАИМЕНЬШЕЙ ДЛИНОЙ ОЧЕРЕДИ}

\def\titkol{Сети массового обслуживания с~наименьшей длиной очереди}

\def\autkol{C.\,C.~Матвеева,  Т.\,В.~Захарова}
\def\aut{C.\,C.~Матвеева$^1$,  Т.\,В.~Захарова$^2$}

\titel{\tit}{\aut}{\autkol}{\titkol}

%{\renewcommand{\thefootnote}{\fnsymbol{footnote}}\footnotetext[1]
%{Исследование поддержано грантами РФФИ 08-07-00152 и 09-07-12032.
%Статья написана на основе материалов доклада, представленного на IV 
%Международном семинаре  <<Прикладные задачи теории вероятностей и математической статистики, 
%связанные с моделированием информационных систем>> (зимняя сессия, Аоста, Италия, январь--февраль 2010~г.).}}

\renewcommand{\thefootnote}{\arabic{footnote}}
\footnotetext[1]{Московский государственный университет им.\ М.\,В.~Ломоносова, 
кафедра математической статистики факультета вычислительной математики и кибернетики, petkin@mail.ru}
\footnotetext[2]{Московский государственный университет имени М.\,В.~Ломоносова, 
кафедра математической статистики факультета вычислительной математики и кибернетики, lsa@cs.msu.su}


\Abst{Статья посвящена исследованию свойств
оптимальных размещений по критерию средней суммарной длины очереди в
пространстве для систем с дисциплиной обслуживания FIFO.
Рассматривается поток однородных требований, различающихся лишь
моментами поступления в систему. Станции представляют собой системы
массового обслуживания типа $M|G|1$. В статье дается описание
свойств оптимальных размещений, показаны алгоритмы построения
асимптотически оптимальных размещений, минимизирующих критерий
оптимальности.}

\KW{асимптотически оптимальное размещение;
средняя суммарная длина очереди; критерий оптимальности}

       \vskip 14pt plus 9pt minus 6pt

      \thispagestyle{headings}

      \begin{multicols}{2}

      \label{st\stat}

\section{Введение}

В статье рассмотрена сеть массового обслуживания, для которой ищется размещение станций обслуживания, 
минимизирующее среднюю суммарную длину очереди. Обслуживание заявок в ней производится территориально 
распределенными объектами. В~связи с этим возникают задачи нахождения оптимального или близкого к 
оптимальному размещения станций обслуживания. На практике задачи подобного рода довольно актуальны. 
Среди них задача размещения автозаправочных станций для увеличения прибыльности их функционирования, 
задачи минимизации транспортных расходов или времени. Спецификой изучаемого класса систем является 
необходимость использования информации о положении обслуживающих приборов, положении и числе поступающих 
вызовов, а при некоторых дисциплинах обслуживания -- и других характеристик. В данном случае рассматривается 
модель, в которой станции обслуживания могут быть расположены в произвольных точках некоторого пространства, 
а вызовы являются реализацией некоторой случайной величины, у которой известна плотность распределения. 
Расстояние между точками пространства определяется чебышевской нормой. Сами станции функционируют как 
независимые системы массового обслуживания типа $M|G|1$ с дисциплиной обслуживания FIFO.

Точные формулы для решений задач подобного рода удается получить лишь в исключительных ситуациях. 
Однако часто, применяя различные асимптотические методы, можно получить удовлетворительное для практики 
асимптотическое решение задачи. В~статье приводятся алгоритмы построения асимптотически оптимальных размещений, 
минимизирующих критерий оптимальности, а также исследуются свойства оптимальных размещений по критерию средней 
суммарной длины очереди.

В работе~\cite{3mat} была изучена аналогичная постановка задачи, но с вызовами, распределенными на плоскости. 
В~настоящей статье решается задача с вызовами, распределенными в пространстве~$\mathbb{R}$, 
которая обобщается на случай пространства произвольной размерности.

\section{Постановка задачи}

В пространстве~$\rr$ возникают требования в случайных точках $\xi_1,\xi_2,\dots,$ независимых и
одинаково распределенных с плотностью распределения~$p$. Для обслуживания этих требований
имеется $n$~станций. Моменты поступления требований образуют пуассоновский поток с параметром
$\lambda$. Интенсивность входящего потока~$\lambda$ изменяется с ростом чис\-ла станций~$n$. 
В~случае, когда нужно подчеркнуть эту зависимость, параметр входящего потока будем обозначать~$\lambda(n)$.

\medskip

\noindent
\textbf{Определение 1.} \textit{Размещением $n$~станций обслуживания в пространстве~$\rr$ назовем
множество точек пространства $\{x_1,\ldots,x_n\}$, в которых они расположены.}

\medskip

Обозначать размещение станций будем символом~$x$, т.\,е.\ $x=\{x_1,\ldots,x_n\}$. Станцию
обслуживания и точку пространства, где она расположена, будем обозначать одним и тем же
символом.

\smallskip

\noindent
\textbf{Определение 2.} \textit{Зоной влияния станции~$x_i$ назовем множество~$C_i$ тех точек
пространства, для которых эта станция является ближайшей:
$$
C_i=\{v\in\rr:\|v-x_i\|\leqslant\|v-x_j\|,~~ j=1,2,\ldots,n\}\,.
$$}

\vspace*{-3pt}

Расстояние $\|u-v\|$ между точками~$u$ и~$v$ пространства~$\rr$ задается чебышевской нормой:

\noindent
\begin{multline*}
\|u-v\|=\maxl{1\leqslant i\leqslant 3}\left|u_i-v_i\right|\,, \\
 u=(u_1,u_2,u_3), \quad v=(v_1,v_2,v_3)\,.
\end{multline*}

Станции обслуживают заявки только из своих зон влияния. Обслуживание осуществляется прибором,
двигающимся только по прямой и с постоянной скоростью. При поступлении заявки прибор со
станции перемещается в точку вызова, заявка обслуживается некоторое случайное время~$\eta$, 
затем прибор возвращается обратно на станцию. Дисциплина обслуживания
следующая: если в момент поступления вызова прибор занят, то поступающий вызов ставится в
очередь. После освобождения прибора на обслуживание поступает первая заявка из очереди.

Обозначим через~$\lambda_i$  интенсивность потока вызовов, поступающих на станцию~$x_i$,
$\beta_{i1}$, $\beta_{i2}$~--- соответственно первый и второй моменты времени обслуживания на~$x_i$.

Предполагается, что станции функционируют как независимые системы массового обслуживания типа
$M|G|1$, тогда их средняя суммарная длина очереди~$L(x)$ при размещении~$x$ и условии, что
загрузка каждой станции меньше единицы, т.\,е.\ $\maxl{1\leqslant i\leqslant n} \lambda_i
\beta_{i1}< 1 $, определяется по формуле
%\noindent
$$
L(x)=\suml{i=1}{n}\fr{\lambda_i^2}{2} \,\fr{\beta_{i2}} {1-\lambda_i \beta_{i1}}\,.
$$
\smallskip

Задача заключается в нахождении размещений, минимизирующих введенный критерий оптимальности~$L(x)$.

\smallskip

\noindent
\textbf{Определение~3.} \textit{Размещение~$x^*$ назовем оптимальным, если $L(x^*)\leqslant L(x)$ для
любого размещения~$x$ такого, что $|x|=|x^*|$. Через~$|x|$ здесь обозначено число элементов
размещения~$x$.}

\smallskip

\noindent
\textbf{Определение~4.} \textit{Размещение~$x$ такое, что при $|x|=|x^*|$ выполняется равенство
$$
\limn \fr{L(x)}{L(x^*)}=1\,,
$$
назовем асимптотически оптимальным.}
\columnbreak

%\smallskip

Введем еще ряд обозначений:
\begin{align*}
\e\eta&=\beta_1\,,\quad \e\eta^2=\beta_2\,;\\
C_1&=\int\!\!\!\il{O}\!\!\!\!\int \|u\|\,du_1du_2du_3\,;\\
 C_2&=\int\!\!\!\il{O}\!\!\!\!\int\|u\|^2\,du_1du_2du_3\,,
\end{align*}
где $u=(u_1,u_2,u_3)$, $O$~--- шар единичного объема с центром в нуле.
Можно вычислить точные значения констант: $C_1=3/8$; $C_2=3/20$.

Норму плотности распределения вызовов определим как
$$
|p|_m=\left(\int\!\!\!\int\!\!\!\int p^m(u)\,du_1du_2du_3\right)^{1/m}\,.
$$

Число станций оптимального размещения, попадающих в множество~$A$, обозначим
$
x_A^*=x^*\cap A$ и $\{x\}$~--- последовательность размещений.

\section{Свойства оптимальных размещений}

В первой теореме описываются асимптотические свойства оптимальных размещений для исходной
модели. Для краткости далее везде под интегралом понимается многомерный интеграл Лебега в соответствующем пространстве.

\medskip
\noindent
\textbf{Теоpема 1.}  \textit{Если плотность~$p$~--- ограниченная, интегрируемая по Лебегу
функция, $\e|\xi|^2<\infty$  и интенсивность входящего потока $\lambda(n)=o(n)$,
то для всякой последовательности оптимальных размещений $\{x^*\}$ справедливы
равенства:}
\begin{align*}
1)\ \ \ & \liml{n\rightarrow \infty}\fr{n}{\lambda^2(n)}L(x^*)=0{,}5\beta_2\,;\\
2)\ \ \ &\liml{n\rightarrow\infty}\fr{\left|x_A^*\right|}{n}=\il{A}{}p(u)\,du\,,
\end{align*}
\textit{где $A$~--- измеримое по Лебегу множество пространства~$\rr$.}

\medskip

Рассмотрим модель, в которой обслуживание состоит только лишь в перемещении прибора со станции
до точки, где возникло требование, и обратно. В~этом случае оптимальные размещения обладают
другими свойствами.

\medskip

\noindent
\textbf{Теоpема~2.}  \textit{Если плотность~$p$ ограничена, $p^{3/4}$ интегрируема по Лебегу,
${\e}|\xi|^2<\infty$  и интенсивность входящего потока требований изменяется так, что}
$$
\limn\fr{\lambda(n)}{n^{4/3}}=0\,,
$$
\textit{то для всякой последовательности оптимальных размещений $\{x^*\}$}
\begin{align*}
1)\ \ \ &\limn\fr{n^{5/3}}{\lambda^2(n)}{L(x^*)}=2C_2|p|_{3/4}^2\,;\\
2)\ \ \ &\limn\fr{\left|x_A^*\right|}{n}=|p|_{3/4}^{-3/4}\il{A}{}p^{3/4}(u)\,du
\end{align*}
\textit{для любого измеримого по Лебегу множества~$A$ пространства~$\rr$}.

\medskip

Результаты теоремы~1 обобщаются на пространство~$\rn$.

\medskip

\noindent
\textbf{Теоpема 3.}  \textit{Если плотность~$p$~--- ограниченная, интегрируемая по Лебегу
функция, $\e|\xi|^2<\infty$ и интенсивность входящего потока $\lambda(n)=o(n)$,
то для всякой последовательности оптимальных размещений $\{x^*\}$ справедливы
равенства}
\begin{align*}
1)\ \ \ &\liml{n\rightarrow \infty}\fr{n}{\lambda^2(n)}L(x^*)=0{,}5\beta_2\,;\\
2)\ \ \ &\liml{n\rightarrow\infty}\fr{\left|x_A^*\right|}{n}=\il{A}{}p(u)\,du\,,
\end{align*}
\textit{где $A$~--- измеримое по Лебегу множество пространства~$\rn$}.

В случае, когда $\e\eta=0$, результаты в пространстве~$\rn$ выглядят следующим образом.

\medskip

\noindent
\textbf{Теоpема~4.}  \textit{Если плотность~$p$ ограничена, $p^{N/(N+1)}$ интегрируема по Лебегу,
${\e}|\xi|^2<\infty$  и интенсивность входящего потока требований изменяется так, что
$$
\limn\fr{\lambda(n)}{n^{(N+1)/N}}=0\,,$$
то для всякой последовательности оптимальных размещений $\{x^*\}$}
\begin{align*}
1)\ \ &\limn\fr{n^{(N+2)/N}}{\lambda^2(n)}{L(x^*)}=\fr{0{,}5 N}{N+2}\, |p|_{N/(N+1)}^2\,;\\
2)\ \ &\limn\fr{\left|x_A^*\right|}{n}=|p|_{N/(N+1)}^{-N/(N+1)}\il{A}{}p^{N/(N+1)}(u)\,du
\end{align*}
для любого измеримого по Лебегу множества~$A$ пространства~$\rn$.

%\bigskip

\section{Доказательства для~пространства $\rr$}

В работе~\cite{4mat} представлено доказательство следующих двух лемм.

\medskip

\noindent
\textbf{Лемма~1.} \textit{Если $Q$~--- измеримое по Лебегу подмножество метрического пространства, 
$S$~--- шар с центром в точке $v$ из того же пространства, а меры Лебега множеств $Q$ и $S$ равны,
то
$$
\il{Q}{} a(\|u-v\|)\,du \geqslant \il{S}{} a(\|u-v\|)\,du
$$
для любой неубывающей на $[0,\infty)$ действительной функции~$a(u)$.}

\smallskip

Следующая лемма является аналогом результатов Л.\,Ф.~Тота~\cite{6mat}.

\smallskip

\noindent
\textbf{Лемма 2.} \textit{Пусть  $S_i$,  $i=1,2,\ldots,n$, обозначает шар с центром в нуле,
$\sigma_n$~--- шар с центром в нуле и мерой $\sigma_n=(1/n) \sumi S_i$. Справедливо
следующее нера\-венство:}
$$
\sumi \il{S_i}{} a(\|u\|)\,du \geqslant n \il{\sigma_n}{} a(\|u\|)\,du
$$
\textit{для любой неубывающей на $[0,\infty)$ действительной функции $a(u)$.}
\smallskip

Следующую лемму нам достаточно доказать для случая, когда плотность~$p$~--- простая функция, т.\,е.\ $p=\suml{j=1}{r}p_j
\textbf{1}_{K_j}$~, где $K_j$ -- измеримые по Лебегу непересекающиеся множества.

\smallskip

\noindent
\textbf{Лемма 3.} \textit{Если} ${\e} \eta=0,~ \lambda(n)=o(n^{4/3})$, \textit{то для любой
последовательности оптимальных размещений} $\{x^*\}$
$$
\liml{n\rightarrow\infty}\fr{n}{\lambda^2(n)}L(x^*)=0\,.
$$

%\medskip

\noindent
Д\,о\,к\,а\,з\,а\,т\,е\,л\,ь\,с\,т\,в\,о\,.\ 
Очевидно, что $\lambda^{-2} L(x^*) \geqslant 0$. Оценим сверху~$L(x^*)$. 
Построим размещение~$x^0$ следующим образом. Для~$K_j$ выберем при\-бли\-жа\-ющее его с точностью~$\varepsilon$ 
элементарное множество~$L_j$. Затем~$L_j$ заместим конгруэнтными кубами
объемом $\sigma_j=K_j/m_j$, где
$$
m_j=m(1-\delta) \fr{K_j}{\suml{i=1}{r}K_i}\,,
$$
$m$~--- некоторое натуральное число, $0<\delta<1$.
\pagebreak

Если $\mu(\sigma_j \bigcap K_j)>0$, то в центр $\sigma_j$ помещается
станция обслуживания. Число таких~$\sigma_j$ обозначим через~$n_j$.
$([m\delta]+1)$ станций разместим равномерно на множествах
$(K_j\backslash L_j)\cap K$, где $K$~--- наименьший куб с центром в
нуле, содержащий в себе носитель плотности~$p$. Полученное
размещение обозначим через $x^0=\{x_1,\ldots,x_n \}$.

Для оптимального размещения~$x^*$ справедливы неравенства
\begin{multline*}
0\leqslant\fr{1}{\lambda^2}L(x^*)\leqslant \fr{1}{\lambda^2} L(x^0)\leqslant {}\\
{}\leq
\suml{j=1}{r}n_j \fr{p_j^2 \sigma_j^2 \cdot
2 C_2 \sigma_j^{2/3}}{1-\lambda p_j \sigma_j \cdot 2 C_1 \sigma_j^{1/3}} + o(n^{-5/3})\,,
\end{multline*}
$$
\fr{1}{\lambda^2}L(x^*)\leqslant O\left(n^{-5/3}\right)\,,
$$
так как $\lambda p_j \sigma_j^{4/3}=o(1)$ и
\begin{multline*}
\suml{j=1}{r}n_j p_j^2 \sigma_j^{8/3}=\suml{j=1}{r}p_j^2 K_j \sigma_j^{5/3}(1+o(1))={}\\
{}=\suml{j=1}{r}p_j^2 K_j \cdot
K_j^{5/3} m_j^{-5/3}(1+o(1))={}\\
{}
=(1+o(1))\left(\suml{j=1}{r}p_j^2 K_j\right)\left(\suml{j=1}{r} K_j\right)^{5/3}
n^{-5/3}\,.
\end{multline*}
Следовательно,
$$
\limn\fr{n}{\lambda^2(n)}L(x^*)=0\,. \qquad\qquad\qquad\qquad  \sqq
$$

\smallskip


Пусть $G$~--- некоторый компакт на носителе плотности~$p$,
$D_n(x)=\maxl{1\leqslant i\leqslant n}diam A_i$, $A_i=\{u\in$\linebreak $\in A_i^{'} : p(u)>0 \}$, $A_i^{'}$~--- 
зона влияния~$x_i$ на компакте~$G$.

\medskip

\noindent
\textbf{Лемма~4.} \textit{Если для размещения $\{x\}$}
$$
\liml{n\rightarrow\infty} \fr{n}{\lambda^2(n)}\,L(x)=0\,,
$$ 
\textit{то}
$$
\liml{n\rightarrow\infty} D_n(x)=0\,.
$$

\smallskip

\noindent
Д\,о\,к\,а\,з\,а\,т\,е\,л\,ь\,с\,т\,в\,о\,.\ 
Предположим, что $\lims{n \to \infty~~~} D_n(x)=d>0$. Выберем подпоследовательность размещений такую, 
что для некоторой станции~$y_n$ из $n$-го
размещения этой подпоследовательности $\limn \|u_n-v_n\|=d$, где $u_n, v_n \in C_n$;
$C_n=\{u\in C_n' : p(u)>0 \}$; $C_n'$~--- зона влияния станции~$y_n$ на~$G$.

Ввиду компактности~$G$ без ограничения общ\-ности можно считать, что $y_n \to y_0$, $u_n \to u_0$,
$v_n \to v_0 $.

Пусть $\|u_0 - y_0\|\geqslant d/2$. Это означает, что лишь конечное число размещений из
выбранной подпоследовательности имеет станции в $d/4$-окрестности точки~$u_0$. Через~$R$
обозначим $d/8$-окрестность точки~$u_0$.
%
Поскольку $u_0$ является предельной точкой~$C_0$ и~$C_0$~--- подмножество носителя плотности,
то для некоторого $\varepsilon>0$ 
$$
\il{R}{}p(u)\,du > \varepsilon\,.
$$

Оценим снизу значение критерия~$L$ на размещениях, не имеющих станций в $d/4$-окрест\-ности
точки~$u_0$.
%
Поскольку в нашем случае $\beta_1=\beta_2=0$, то второй момент времени обслуживания $i$-й заявки 
$$
\beta_{i2}=\fr{4}{{\p} (C_i)}\il{C_i}{}\|u-x_i\|^2 p(u)\,du\,.
$$

Так как загрузка на каждой станции обслуживания меньше~1, то
\begin{multline*}
L(x)=\suml{i=1}{n}\fr{\lambda_i^2}{2} \,\fr{\beta_{i2}} {1-\lambda_i \beta_{i1}}\geqslant
0{,}5\lambda^2\suml{i=1}{n} {\p}^2(C_i)\beta_{i2}={}\\
{}=2\lambda^2\suml{i=1}{n} {\p}(C_i)\il{C_i}{}\|u-x_i\|^2 p(u)\,du\,.
\end{multline*}

Учитывая, что на множестве~$R$ $\|u-x_i\|>d/8$ для любого~$i$ на выбранном размещении, 
а также выпуклость функции $f(u,v)=u v$, имеем
\begin{multline*}
L(x)\geqslant 2\lambda^2\suml{i=1}{n} {\p}(C_i)\cdot \fr{1}{n}\suml{i=1}{n}\il{C_i}{}\|u-x_i\|^2 p(u)~du \geqslant{}\\
{}\geq 2\lambda^2 \fr{1}{n}\left(\fr{d}{8}\right)^2 \varepsilon\,.
\end{multline*}
Поэтому
$$
\fr{n}{\lambda^2(n)}L(x)\geqslant \fr{\varepsilon d^2}{32}\,.
$$
Отсюда
$$
\liml{n\to\infty}\fr{n}{\lambda^2(n)}L(x)> 0\,,
$$
что противоречит условию леммы.

Поэтому $d=0$. \hfill\sqq

Последняя лемма показывает, что на носителе плотности распределения вызовов диаметры зон влияния на 
любом компакте стремятся к нулю на последовательности оптимальных размещений.

Для случая ${\e}\eta>0$ также справедливы соответствующие леммы, доказательство 
которых сходно с доказательством лемм~3 и~4.

\medskip

\noindent
\textbf{Лемма 5.} \textit{Если ${\e} \eta>0,~ \lambda(n)=o(n)$, то для любой
последовательности оптимальных размещений} $\{x^*\}$
$$
\liml{n\rightarrow\infty}\fr{1}{\lambda^2(n)}L(x^*)=0\,.
$$

\medskip

\noindent
\textbf{Лемма 6.} \textit{Если для последовательности размеще\-ний~$\{x\}$}
$\liml{n\rightarrow\infty} \fr{1}{\lambda^2(n)}\,L(x)=0\,,$ 
\textit{то}
$$
\liml{n\rightarrow\infty} D_n(x)=0\,.
$$

\smallskip

\noindent
Д\,о\,к\,а\,з\,а\,т\,е\,л\,ь\,с\,т\,в\,о\ теоремы~1.
Для любой станции~$x_i$ с зоной влияния~$C_i$ некоторого размещения~$x$ первые два момента
времени обслуживания оцениваются как
\begin{align*}
\beta_{i1}&={\e} ((2\|\xi-x_i\|+\eta)|\xi\in C_i)\geqslant \e\eta=\beta_1\,;
\\
\beta_{i2}&={\e} ((2\|\xi-x_i\|+\eta)^2|\xi\in C_i)\geqslant \e\eta^2=\beta_2\,,
\end{align*}
а интенсивность потока поступающих на нее требований есть $\lambda_i=\lambda(n) {\p} (C_i)$.

С учетом этого, а также выпуклости функции $f(u,v)=u^2(1-v)^{-1}$~, оценим снизу
критерий~$L(x)$.
\begin{multline*}
L(x)=\suml{i=1}{n}\fr{\lambda_i^2}{2} \fr{\beta_{i2}} {1-\lambda_i \beta_{i1}}\geqslant{}\\
{}\ge
0{,}5\lambda^2\beta_2\suml{i=1}{n} \fr{{\p}^2(C_i)} {1-\lambda \beta_1 {\p}(C_i)}\geqslant{}\\
{}
\!\geqslant 0{,}5\lambda^2\beta_2 \fr{1}{n}\left(\suml{i=1}{n}{\p}(C_i)\right)^{\!2}\!\! \left(1-\lambda
\beta_1 \fr{1}{n}\,\suml{i=1}{n}{\p}(C_i)\right)^{\!\!-1}\!\!\!,\hspace*{-0.8pt}
\end{multline*}
т.\,е.
$$
\fr{n}{\lambda^2(n)}L(x^*)\geqslant 0{,}5\beta_2\left(1-\fr{\lambda \beta_1}{n}\right)^{-1}\,.
$$
Устремляя~$n$ к бесконечности, получим
$$
\limi{n \to \infty}\fr{n}{\lambda^2(n)}\,L(x^*)\geqslant 0{,}5\beta_2\,.
$$
\smallskip

Получим теперь верхнюю оценку. Предположим сначала, что плотность~$p$~--- простая функция, $p=\suml{j=1}{r}p_j
\textbf{1}_{K_j}$, где $K_j$~--- измеримые по Лебегу непересекающиеся множества.

Для такой плотности построим асимптотически оптимальное размещение~$x$, т.\,е.\ такое, что при
$|x|=|x^*|$ выполняется равенство
$$
\limn \fr{L(x)}{L(x^*)}=1\,.
$$

Для этого каждое множество~$K_j$ заменим
элементарным множеством~$L_j$ таким, что $\mu(K_j \Delta L_j)<$\linebreak $<\varepsilon$, затем~$L_j$
заместим конгруэнтными кубами, пересекающимися лишь по границе, и объемом
$\sigma_j=K_j/m_j$, где
$m_j=m (1-\delta) p_jK_j$; $m$~--- некоторое натуральное число и $0<\delta<1$.

Если $\mu(\sigma_j \bigcap L_j)>0$, то в центр куба~$\sigma_j$ помещается станция. Пусть
$n_j$~--- число таких станций. Поскольку размещение станций реализуется на компакте, можно построить наименьший 
куб~$K$, который содержит в себе полностью носитель плотности~$p$. $([m\delta]+1)$~станций 
разместим равномерно на множествах $(K_j\backslash L_j)\cap K$. Тем самым
получим некоторое размещение $x=\{x_1,\ldots,x_n\}$~, для которого
\begin{multline*}
L(x)\leqslant{}\\
{}\le \suml{j=1}{r}n_j\fr{\lambda^2 p_j^2 \sigma_j^2}{2}\, \fr{\beta_2+4\beta_1
C_1\sigma_j^{1/3}+4C_2\sigma_j^{2/3}} {1-\lambda \beta_1 p_j \sigma_j-2\lambda C_1p_j
\sigma_j^{4/3}}={}\\
{}= \fr{\lambda^2}{2}\,\suml{j=1}{r} \fr{\beta_2 p_j K_j
n^{-1}+o(n^{-1})}{1-\lambda \beta_1 n^{-1}+o(n^{-1})}\,.
\end{multline*}
Устремляя $m$, а тем самым и~$n$ к бесконечности, получаем
$$
\lims{n \to \infty}\fr{n}{\lambda^2(n)}L(x)\leqslant 0{,}5\beta_2\,.
$$
Так как всегда $L(x^*)\leqslant L(x)$, то с учетом нижней оценки для~$L(x^*)$ получаем, что
$x$~--- аcимптотически оптимальное размещение и
$$
\liml{n \to \infty}\fr{n}{\lambda^2(n)}L(x^*) = 0{,}5\beta_2\,.
$$
Пусть $p$~--- произвольная функция, удовле\-тво\-ря\-ющая условиям доказываемой теоремы. Введем
прос\-тые функции~$\bar{p}_k(u)$ по правилу
$\bar{p}_k(u)=$\linebreak $=(m+1)/k$, если $m/k<p(u)\leqslant (m+4)/k$ для $k\in \NCAL, m=0,1,\ldots$

Очевидно, что $p(u)\leqslant \bar{p}_{k}(u)$, а для простых функций уже была получена
предельная оценка сверху, поэтому справедливо следующее неравенство:
$$
\lims{n \to \infty}\fr{n}{\lambda^2(n)}\,L(x^*)\leqslant 0{,}5\beta_2 \left|\bar{p}_k \right|_1\,.
$$
При $k \to \infty$\ $\left|\bar{p}_k \right|_0 \to \left|p\right|_1=1$. И с учетом оценки
снизу для~$L(x^*)$ получаем, что
$$
\liml{n \to \infty}\fr{n}{\lambda^2(n)}\,L(x^*) = 0{,}5\beta_2\,.
$$
Докажем второй пункт теоремы~1.
\pagebreak

Рассмотрим размещение $x_A=x\bigcap A$. Пусть $k=\left|x_A\right|$~--- число станций,
попадающих в множество~$A$ при размещении~$x$. Определим~$L(x_A)$ как~$L(x)$ для~$p\mathbf{1}_A$.
\begin{multline*}
L(x_A)=\suml{i=1}{k}\fr{\lambda_i^2}{2}\,\fr{\beta_{i2}} {1-\lambda_i \beta_{i1}}\geqslant{}\\
{}\geq
0{,}5\lambda^2\beta_2\suml{i=1}{k} \frac{{\p}^2(C_i\bigcap A)} {1-\lambda \beta_1
{\p}(C_i\bigcap A)}\geqslant{}\\
{}\geqslant 0{,}5\lambda^2\beta_2 \fr{1}{k}\left(\suml{i=1}{k}{\p}(C_i\bigcap A)\right)^2
\left(
\vphantom{\suml{i=1}{k}}
1-{}\right.\\
\left.{}-\lambda \beta_1 \fr{1}{k}\,\suml{i=1}{k}{\p}(C_i\bigcap A)\right)^{-1}\,,
\end{multline*}
значит
$$
\fr{k}{\lambda^2(n)}L(x_A)\geqslant 0{,}5\beta_2{\p}^2(A)\left(1-\fr{\lambda \beta_1
{\p}(A)}{k}\right)^{-1}\,.
$$
Следовательно,
$$
\limi{n \to \infty}\fr{k}{\lambda^2(n)}\,L(x_A)\geqslant 0{,}5\beta_2{\p}^2(A)\,.
$$
Пусть $\gamma_1$~--- предельная точка последовательности $\left\{\left|x_A\right|n^{-1}
\right\}$.
\smallskip

Пусть теперь~$x$~--- асимптотически оптимальное размещение для критерия~$L$, тогда
$$
\lims{n \to \infty}\fr{n}{\lambda^2(n)}\,L(x_A) \leqslant 0{,}5\beta_2{\p}(A)\,.
$$
Из двух последних неравенств следует, что
$$
\gamma_1 \geqslant{\p}(A)\,.
$$
Пусть $\gamma_2$~--- предельная точка последовательности $\left\{\left|x_B\right|n^{-1}
\right\}$~, где $B=A^c$.

Для $\gamma_2$ аналогично доказывается соответству\-ющее неравенство
$$
\gamma_2 \geqslant {\p}(B)\,.
$$

Так как
$$
1=\gamma_1+\gamma_2\geqslant {\p}(A)+{\p}(B)=1\,,
$$
то в неравенстве достигнуто равенство. Это возможно только, если
$$
\gamma_1={\p}(A)\,, \qquad \gamma_2={\p}(B)\,.
$$
Тем самым для любого асимптотически оптимального размещения
$$
\liml{n\rightarrow\infty}\fr{\left|x_A\right|}{n}=\il{A}{}p(u)\,du\,,
$$
а, значит, это равенство верно и для оптимального размещения. Отсюда следует утверждение
второго пункта теоремы~1.

Доказательство теоремы~2 проводится аналогично доказательству теоремы~1.

Сначала оценивается значение критерия~$L(x)$ снизу
$$
\limi{n \to \infty} \fr{n^{5/3}}{\lambda^2(n)}L(x^*)\geqslant 2C_2|p|_{3/4}^2\,.
$$

Для оценки сверху~$L(x^*)$ строится асимптотически оптимальное размещение~$x$, алгоритм
по\-стро\-ения которого несколько иной.

Выберем последовательность вложенных расширяющихся кубов~$K$ с центром в нуле таких, что
${\e}|\xi|^2 \mathbf{1}_{K^c}=o(m^{-2})$.

Предположим сначала, что плотность $p$~--- прос\-тая функция, определенная так же, как и ранее.

Для каждого $K_j$ выберем элементарное множество~$L_j$, чтобы
$$
\mu(K_j \Delta L_j)<\varepsilon/K\,,\enskip \forall j\,.
$$

Каждое множество~$L_j$ покрываем правильной решеткой объемом
$\sigma_j=K_j/m_j$, где
$$
m_j=m(1-\delta) \fr{p_j^{3/4}K_j}{\suml{i}{} p_i^{3/4}K_i}\,,
$$
$m$~--- некоторое натуральное число и $0<\delta<1$.

Если $\mu(\sigma_j\cap L_j)>0~,$ то в центры таких~$\sigma_j$ помещаем станцию
обслуживания. Число таким образом размещенных станций обозначим через~$n_j$. 
$([m\delta]+1)$~станций разместим равномерно на множествах $(K_j\backslash L_j)\cap K$.

Общее число размещенных станций обозначим через~$n$. Тем самым получим некоторое размещение
$x=\{x_1,\ldots,x_n\}$, для которого с учетом леммы~4 справедливы следующие оценки:
\begin{multline*}
L(x)\leqslant \fr{\lambda^2}{2} \suml{j=1}{r} n_j\, \fr{p_j^2 \sigma_j^{8/3} 4C_2}
{1-2\lambda C_1 p_j \sigma_j^{4/3}}+{}\\[6pt]
{}+ p_j^2 (m \delta)^{-5/3} \varepsilon +o(m^{-5/3}) ={}\\[6pt]
{}
=\lambda^2~ 2~ C_2(1+o(1))\suml{j=1}{r} \fr{p_j^{3/4} K_j }{1+o(1)}~
\left(\suml{i=1}{r} p_i^{3/4} K_i\right)^{5/3}\times{}\\[6pt]
{}\times \left(n(1-\delta)\right)^{-5/3}+O(\varepsilon n^{-5/3})\,.
\end{multline*}
Отсюда следует, что
$$
\lims{n \to \infty}\fr{n^{5/3}}{\lambda^2(n)}L(x)\leqslant 2C_2|p|_{3/4}^2\,.
$$
Следовательно,
$$
\limn\fr{n^{5/3}}{\lambda^2(n)}L(x^*)=2C_2|p|_{3/4}^2\,.
$$

Используя теорему Лебега о предельном переходе под знаком интеграла, 
полученный результат можно обобщить на случай произвольной (в рамках 
ограничений доказываемой теоремы) плотности~$p$.

Доказательство второго пункта проводится так же, как и в теореме~1.

\section{Заключение}

В предлагаемой работе исследованы асимптотические свойства оптимальных 
размещений для исходной модели, описано поведение критерия на последовательности 
оптимальных размещений, найдена предельная оптимальная плотность. Для наглядности 
подробные доказательства приведены для случая трехмерного пространства, одна-\linebreak ко 
полученные результаты можно обобщить на\linebreak случай пространства произвольной конечной 
раз\-мер\-ности. Интересно отметить, что при $\beta_1 >0$ предельная плотность 
размещений не зависит от размерности пространства и эквивалентна, в некотором 
смысле, плотности распределения вызовов.

Полученные результаты носят не только теоретический, но и практический характер и могут быть 
применены для изучения реальных систем. Найденные алгоритмы построения асимптотически оптимальных 
размещений допускают реализацию на программном уровне.

{\small\frenchspacing
{%\baselineskip=10.8pt
\addcontentsline{toc}{section}{Литература}
\begin{thebibliography}{9}


\bibitem{3mat} %1
\Au{Захарова Т.\,В.} 
Размещение систем массового обслуживания, минимизирующее среднюю длину очереди~// Информатика и её применения, 2008. 
Т.~2. Вып.~1. С.~63--66.

\bibitem{4mat} %2
\Au{Захарова Т.\,В.} 
Оптимизация расположения станций обслуживания в пространстве~// Информатика и её применения, 2008. Т.~2. Вып.~2. С.~41--46.


\label{end\stat}

\bibitem{6mat} %3
\Au{Тот Ф.\,Л.} 
Расположение на плоскости, на сфере и в пространстве.~--- М.: ГИФМЛ, 1958.

%\bibitem{1mat} 
%\Au{Назаров Л.\,В., Смирнов С.\,Н.} 
%Обслуживание вызовов, распределенных в пространстве~// Изв. АН СССР. Техн. кибернет., 1982. №\,1. С.~95--99.

%\bibitem{2mat}
%\Au{Захарова Т.\,В.} 
%Оптимальные размещения систем массового обслуживания с дисциплиной обслуживания FIFO~// Вест. Моск. ун-та. Cер.~15. 
%Вычисл. матем. и киберн., 2007. №\,4. С.~32--37.

%\bibitem{5mat} 
%\Au{Ивченко Г.\,И., Каштанов В.\,А., Коваленко~И.\,Н.} 
%Теория массового обслуживания.~--- М.: Высшая школа, 1982.

 \end{thebibliography}
}
}


\end{multicols}