

\newcommand{\vpi}{{\mathbf p}}
\newcommand{\A}{{\mathbf A}}

\def\stat{zeif}

\def\tit{ОБ УСТОЙЧИВОСТИ НЕСТАЦИОНАРНЫХ СИСТЕМ ОБСЛУЖИВАНИЯ С КАТАСТРОФАМИ$^*$}

\def\titkol{Об устойчивости нестационарных систем обслуживания с катастрофами}

\def\autkol{А.\,И.~Зейфман,  А.\,В.~Коротышева, Я.\,А.~Сатин, С.\,Я.~Шоргин}
\def\aut{А.\,И.~Зейфман$^1$,  А.\,В.~Коротышева$^2$, Я.\,А.~Сатин$^3$, С.\,Я.~Шоргин$^4$}

\titel{\tit}{\aut}{\autkol}{\titkol}

{\renewcommand{\thefootnote}{\fnsymbol{footnote}}\footnotetext[1]
{Исследование поддержано грантами РФФИ 08-07-00152 и 09-07-12032.
Статья написана на основе материалов доклада, представленного на IV 
Международном семинаре  <<Прикладные задачи теории вероятностей и математической статистики, 
связанные с моделированием информационных систем>> (зимняя сессия, Аоста, Италия, январь--февраль 2010~г.).}}

\renewcommand{\thefootnote}{\arabic{footnote}}
\footnotetext[1]{Вологодский государственный
педагогический университет,  Институт проблем информатики Российской академии наук и\linebreak
$\hphantom{^1}$ИСЭРТ Российской академии наук, a\_zeifman@mail.ru}
\footnotetext[2]{Вологодский государственный педагогический университет, a\_korotysheva@mail.ru}
\footnotetext[3]{Вологодский государственный педагогический университет, yacovi@mail.ru}
\footnotetext[4]{Институт проблем
информатики Российской академии наук, SShorgin@ipiran.ru}


\Abst{Рассмотрены модели обслуживания, описываемые
процессами рождения и гибели (ПРГ) с катастрофами. Получены оценки
устойчивости различных характеристик таких систем. Рассмотрен пример
конкретной системы обслуживания.}

\KW{нестационарные системы обслуживания;
марковские модели с катастрофами; устойчивость; оценки; предельные
характеристики; аппроксимация}

       \vskip 14pt plus 9pt minus 6pt

      \thispagestyle{headings}

      \begin{multicols}{2}

      \label{st\stat}

\section{Введение}

Системы массового обслуживания с катастрофами изучались во многих
работах современных авторов  (см., например,~[1--9]). Вопросы, связанные с
устойчивостью неоднородных марковских цепей  с непрерывным временем,
впервые исследованы одним из авторов в~\cite{z85} и затем более
детально для нестационарных ПРГ в
работах~\cite{z98, ae}. В~настоящей статье будет исследована
устойчивость одного класса моделей, описываемых нестационарными ПРГ
с катастрофами. Причем будет рассмотрен случай, когда интенсивность
катастрофы не зависит от числа требований в системе и является
существенной. Случай, когда устойчивость гарантируется за счет
достаточно больших интенсивностей обслуживания, исследуется так же,
как в работах~\cite{z98, ae}, а ситуация, в которой интенсивности
катастроф зависят от числа требований, будет изучаться отдельно.


Пусть $X=X(t)$, $t\geq 0$,~--- ПРГ с катастрофами, а~$\lambda_n(t)$,
$\mu_n(t)$ и~$\xi (t)$~--- интенсивности рождения, гибели и
катастрофы  соответственно.

Обозначим через $p_{ij}(s,t)=\mathrm{Pr}\left\{ X(t)=\right.$\linebreak $\left.=j\left| X(s)=i\right.
\right\}$, $i,j \ge 0, \;0\leq s\leq t$, переходные вероятности
процесса $X=X(t)$, а через  $p_i(t)=\mathrm{Pr}\left\{ X(t) =i \right\}$~---
его вероятности состояний.

При выполнении естественных дополнительных условий (см., например,~\cite{z08b}) 
прямую систему Колмогорова для вероятностей состояний

\noindent
\begin{equation}
\left.
\begin{array}{rl}
%\begin{cases}
\fr{dp_0}{dt}& = -\left(\lambda_0 (t) + \xi(t)\right)p_0 +\mu_1(t)p_1 + \xi(t) \,; \\[9pt]
\fr{dp_k}{dt} &= \lambda_{k-1} (t)p_{k-1} -\left(\lambda_{k} (t) +
 \mu_{k}(t)+{}\right.\\[9pt]
&\hspace*{5mm}\left. {}+
\xi(t)\right)p_k +\mu_{k+1}(t)p_{k+1}\,,\enskip  k \ge 1\,,
\end{array}
\right \}
\label{eq111}
\end{equation}
можно записать в виде дифференциального уравнения
\begin{equation}
\fr{d\vpi}{dt}=\A\left( t\right) \vpi  +{\bf g} (t)\,, \quad t\ge 0
\label{eq112}
\end{equation}
в пространстве последовательностей~$l_1$, где ${\bf g}(t)=\left(\xi (t),0,0, \dots\right)^{\mathrm{T}}$,
$\vpi(t)=\left(p_0(t),p_1(t),\dots\right)^{\mathrm{T}}$~---
вектор-столбец вероятностей состояний; 
$\A(t)=$\linebreak $=\left\{a_{ij}(t),\, t\geq 0\right\}$~--- матрица, порождаемая
системой (\ref{eq111}), при этом ее элементы
определяются по формулам
\begin{equation*}
a_{ij} (t) = 
\begin{cases}
\lambda _{i-1}\left( t\right)\,, & \!\!\!\mbox {если } \quad j=i-1\,; \\
\mu _{i+1}\left( t\right)\,, & \!\!\!\mbox {если } \quad j=i+1\,;
\\
-\left( \lambda _i\left( t\right) +\mu _i\left( t\right) +{} \right.&\\
\hspace*{10mm}\left. {}+\xi(t) \right)\,, & \!\!\!\mbox {если } \quad j=i\,; \\
0 &\!\!\! \mbox{в остальных случаях}\,.
\end{cases}
%\label{eq111,5}
\end{equation*}
Далее предполагаем, что
\begin{equation*}
\lambda _n\left( t\right) =\nu_n \lambda \left( t\right)\,,\ \mu
_n\left( t\right) =\eta_n \mu \left( t\right)\,, \quad t\ge 0, \quad n\in E\,,
% \label{eq101}
\end{equation*}
где  $ 0 \le \eta_n \leq M,\  0 \leq \nu_n \leq M $,  а <<базисные>> функции~$\lambda(t)$,
$\mu(t)$ и~$\xi(t)$ локально интегрируемы на $[0;\infty)$ и (для простоты вычислений) ограничены, т.\,е.
\begin{equation*}
\lambda(t) + \mu(t) + \xi(t) \le L < \infty 
%\label{eq112b}
\end{equation*}
почти при всех $t \ge 0$ (см.\ подробное рассмотрение в~[14--16]).

Обозначим через $\Omega=\left\{{\bf x}: \: {\bf x}\geq 0,\: \|{\bf
x}\|_1=1\right\}$ множество всех стохастических векторов.


Тогда
\begin{equation*}
\|A(t)\|_1  = \sup_{j}\sum_i|a_{ij}(t)| \le 2M L  
%\label{eq112bb}
\end{equation*}
почти при всех $t \ge 0$, а значит, задача Коши для
уравнения~(\ref{eq112}) с начальным условием~$\vpi(0)$ имеет
единственное решение
\begin{equation*}
{\bf p}(t) =  U(t){\bf p}(0) + \int\limits_0^t U(t,\tau){\bf
g}(\tau)\, d\tau \,, 
%\label{eq112bc}
\end{equation*}
где $U(t,s)$~--- оператор Коши уравнения~(\ref{eq112}). При
этом если ${\bf p}(s) \in \Omega$, то и ${\bf p}(t) \in \Omega$ при
любом  $t \ge s$.

Рассмотрим теперь <<возмущенный>> ПРГ с катастрофами  $\bar{X}=\bar{X}(t)$, $t\geq 0$, обозначая через  $\bar{\lambda}_n(t)$,
$\bar{\mu}_n(t)$ и~$\bar{\xi} (t)$ его  интенсивности рождения, гибели и катастрофы  соответственно.

Введем обозначения:
\begin{align*}
\hat{\lambda}_n(t)&=\bar{\lambda}_n(t)-\lambda_n(t)\,;\\
\hat{\mu}_n(t)&=\bar{\mu}_n(t)-\mu_n(t)\,;\\
\hat{\xi}(t)&=\bar{\xi}(t)-\xi(t)\,.
\end{align*}

Для простоты записи оценок будем предполагать, что возмущения <<равномерно малы>>, 
т.\,е.\ при всех $n$ и почти всех $t \ge 0$ выполняются неравенства
\begin{equation}
\left|\hat{\lambda}_n(t)\right| \le \varepsilon_1\,, \enskip
\left|\hat{\mu}_n(t)\right| \le \varepsilon_2\,, \enskip 
\left|\hat{\xi}(t)\right| \le \varepsilon_3.
\label{p1}
\end{equation}

\section{Устойчивость вектора состояний}

Выпишем прямую систему Колмогорова, соответствующую возмущенному процессу,
\begin{equation}
\frac{d\bar{\bf p}}{dt}=\bar{A}(t)\bar{\bf p}(t)+\bar{\bf g}(t).
\label{eq112p}
\end{equation}

Введем обозначения:
$$
\hat{A}(t)=A(t)-\bar{A}(t)\,,\quad \hat{\bf g}(t)={\bf g}(t)-\bar{\bf g}(t)\,.
$$

Перепишем систему~(\ref{eq112p}) в виде
\begin{equation*}
\fr{d\bar{\bf p}}{dt}=A(t)\bar{\bf p}(t) + {\bf g}(t)-\hat{A}(t)\bar{\bf p}(t)-\hat{ \bf g}(t)\,.
%\label{112p1}
\end{equation*}

Получаем
\begin{equation*}
\vpi(t)=U(t)\vpi(0)+\int\limits_0^t U(t,\tau){\mathbf g}(\tau) \, d\tau\,;
\end{equation*}
\begin{multline*}
\bar{\vpi}(t)=U(t)\bar{\vpi}(0)+\int\limits_0^t U(t,\tau){\mathbf g}(\tau) \, d\tau-{}\\
{}-\int\limits_0^t U(t,\tau)
\left(\hat{A}(\tau)\bar{\mathbf p}(\tau) + \hat{\mathbf g}(\tau)\right)\, d\tau\,.
\end{multline*}

В этом параграфе возможно рассмотрение непосредственно по норме~$l_1$.
Пусть $\vpi(0)=\bar{\vpi}(0)$, тогда

\vspace*{-6pt}

\noindent
\begin{multline*}
\left\|\vpi(t)-\bar{\vpi}(t)\right\|\le \int\limits_0^t \|U(t,\tau)\|
\left(\|\hat{A}(\tau)\|\|\bar{\bf p}(\tau)\| +{}\right.\\
{}\left.+ \|\hat{\bf g}(\tau)\|\right)\, d\tau\,.
\end{multline*}

Имеем теперь
\begin{equation}
\|\bar{\bf p}(\tau)\| =1, \quad \tau \ge 0,
\label{p0}
\end{equation}
и, значит,
\begin{multline*}
\!\!\left(\|\hat{A}(\tau)\|\|\bar{\bf p}(\tau)\| + \|\bar{\bf g}(\tau)\|\right)\le
(2(\varepsilon_1+\varepsilon_2)+\varepsilon_3)+\varepsilon_3 ={}\\
{}= 2(\varepsilon_1+\varepsilon_2+\varepsilon_3)\,.
\end{multline*}

Далее, оценивая логарифмическую норму оператора~$A(t)$ в пространстве~$l_1$ (см.~\cite{z09a}), получаем
\begin{equation*}
\gamma \left(A(t)\right)_{1} = \sup_i \left(a_{ii}(t) + \sum_{j\neq
i} |a_{ji}(t)|\right) = -\xi(t)\,.
%\label{cat03}
\end{equation*}
Тогда
\begin{equation*}
\|U(t,s)\| \le  e^{-\int\limits_s^t \xi(\tau)\, d\tau}
%\label{cat04}
\end{equation*}
для всех $0 \le s \le t$ и формулируем следующее утверждение.

\bigskip

\noindent
\textbf{Теорема~1.} \textit{При совпадении начальных условий для исходного и возмущенного 
ПРГ с катастрофами для всех $t \ge 0$ справедлива следующая оценка:}
\begin{multline}
\left\|\vpi(t)-\bar{\vpi}(t)\right\|\le{}\\
{}\le
2(\varepsilon_1+\varepsilon_2+\varepsilon_3)\int_0^t e^{-\int\limits_{s_1}^t \xi(\tau)\,d\tau} \, d{s_1}\,.
\label{stab1}
\end{multline}

\bigskip


\noindent
\textbf{Следствие 1.} \textit{Пусть $\xi(t) \ge \xi >0$ почти при всех $t \ge 0$. Тогда вместо}~(\ref{stab1}) 
\textit{получаем}
\begin{equation*}
\left\|\vpi(t)-\bar{\vpi}(t)\right\|\le \fr{2(\varepsilon_1+\varepsilon_2+\varepsilon_3)}{\xi}\,.
%\label{stab2}
\end{equation*}

\bigskip

\noindent
\textbf{Замечание 1.} Более точные оценки отклонения можно получить, ослабив условия малости 
возмущений и потребовав, чтобы 
вместо~(\ref{p1})  выполнялись при всех~$n$ и почти всех $t \ge 0$  неравенства
\pagebreak

\noindent
\begin{equation}
\left.
\begin{array}{rl}
\left|\hat{\lambda}_n(t)\right| &\le \varepsilon_1\xi(t)\,; \\[9pt]
\left|\hat{\mu}_n(t)\right| &\le \varepsilon_2\xi(t)\,;\\[9pt] 
 \left|\hat{\xi}(t)\right| &\le \varepsilon_3\xi(t)\,,
 \end{array}
 \right \}
\label{p1'}
\end{equation}
а кроме того, чтобы было выполнено естественное условие существенности катастроф
(см.\ подробнее~\cite{z98})

\noindent
\begin{equation}
\int\limits_0^\infty  \xi(\tau)\,d\tau = +\infty\,.
\label{ek1}
\end{equation}

%\vspace*{-12pt}

\section{Оценка для среднего}

Обозначим через $E_k(t) = E\left\{X(t)\left|X(0)=k\right.\right\}$ математическое ожидание процесса в момент~$t$ 
при условии, что в нулевой момент времени он находится в состоянии~$k$. Иногда будет встречаться также несколько 
более общее выражение $E_{\bf p}(t)$~--- это математическое ожидание процесса в момент~$t$ 
при начальном распределении вероятностей состояний ${\mathbf p}(0) = {\mathbf p}$.

Соответствующие выражения для возмущенного процесса будем обозначать через  
$\bar{E}_k(t) =$\linebreak $= E\left\{\bar{X}(t)\left|\bar{X}(0)=k\right.\right\}$ и~$\bar{E}_{\bf \bar{p}}(t)$.


Легко видеть, что тогда $\left|E_{\bf p}(t)- \bar{E}_{\bf \bar{p}}(t)\right| \le$\linebreak
$\le\;\sum_k k|p_k(t)-\bar{p}_k(t)|$. 
Поэтому в качестве основного в этом пункте выберем пространство последовательностей
 $l_{1E}=\left\{{\bf z} =(p_0,p_1,p_2,\ldots)\right\} \in l_1$ таких, что 
 $\|{\bf z}\|_{1E}=\sum_k k|p_k| <\infty$. К~сожалению, в этой и связанных с ней нормах оценку типа~(\ref{p0}) 
 непосредственно получить не удается, поэтому приходится выбирать другой способ дальнейших рассуждений.

Для получения более простых оценок будем предполагать, что найдутся  $\rho \in (0;1)$ и натуральное  $k$ такие, 
что для всех~$n,t$ выполнено условие  $(1/k) \lambda_n(t) \le (1-\rho)\xi(t)$.

Перепишем исходную систему~(\ref{eq112}) для невозмущенного процесса в следующем виде:
\begin{equation*}
\fr{d{\bf p}}{dt}=\bar{A}(t){\bf p}(t) + \bar{\bf g}(t)+\hat{A}(t){\bf p}(t)+\hat{\bf g}(t)\,.
%\label{eq112-n}
\end{equation*}
Тогда
\begin{align*}
\vpi(t)&=\bar{U}(t)\vpi(0)+\int\limits_0^t \bar{U}(t,\tau)\bar{\bf g}(\tau) \, d\tau+{}\\[6pt]
&\hspace{10mm}{}+\int\limits_0^t \bar{U}(t,\tau)
\left(\hat{A}(\tau){\bf p}(\tau) + \hat{\bf g}(\tau)\right)\, d\tau\,;\\[6pt]
\bar{\vpi}(t)&=\bar{U}(t)\bar{\vpi}(0)+\int\limits_0^t \bar{U}(t,\tau)\bar{\bf g}(\tau) \, d\tau \,.
\end{align*}
Это означает, что
в {\it любой} норме при одинаковых начальных условиях справедлива оценка
\begin{multline}
\left\|\vpi(t)-\bar{\vpi}(t)\right\|\le{}\\
{}\le \int_0^t \|\bar{U}(t,\tau)\|
\left(\|\hat{A}(\tau)\|\|{\bf p}(\tau)\| + \|\hat{\mathbf g}(\tau)\|\right)\, d\tau\,.
\label{3000}
\end{multline}

Рассмотрим теперь матрицу
\begin{equation*}
D_k=diag\left( \underbrace{k, k, k, \dots , k}_{k+1 раз}, k+1, k+2, \dots \right)  
%\label{cat52}
\end{equation*}
и соответствующее пространство последовательностей $l_{1k}=\left\{{\bf z} =(p_0,p_1,p_2,\ldots)\right\}$ таких, 
что $\|{\mathbf z}\|_{1k}=$\linebreak $=\;\|D_k {\bf z}\|_1 <\infty$. Тогда, очевидно, при любом целом неотрицательном~$k$ 
имеем $\|{\mathbf z}\|_{1E} \le \|{\bf z}\|_{1k}$.

Оценивая логарифмическую норму~$\gamma(A(t))_{1k}$, получаем
\begin{multline*}
\gamma(A(t))_{1k} = \gamma(D_kA(t)D_k^{-1})_{1} \le
 \fr {M}{k}\,\lambda(t) - \xi(t) \le{}\\
 {}\le - \rho \xi(t)\,,
%\label{3001}
\end{multline*}
поскольку сумма по каждому из столбцов с номерами $0,1, \dots, k-1$ равна 
$- \xi(t)$, а если номер столбца $n \ge k$, то сумма по этому столбцу есть
$ ((n+1)/n) \lambda_n(t)  -\left(\lambda_n(t)+\mu_n(t)+\xi(t)\right) + 
((n\;-$\linebreak $-\;1)/n)\mu_n(t) \le (1/n) \lambda_n(t)  - \xi(t) \le (M/k) \lambda(t)  - \xi(t)$.

Далее
\begin{multline*}
\|\hat{A}(t)\|_{1k}=\|D_k\hat{A}(t)D_k^{-1}\|_{1} \le {}\\
{}\le \fr{2k+1}{k}\varepsilon_1 +
\fr{2k-1}{k}\varepsilon_2 +  \varepsilon_3 \le 3\varepsilon_1 +2\varepsilon_2 +  \varepsilon_3\,.
%\label{3002}
\end{multline*}
А тогда
\begin{multline*}
\gamma(\bar{A}(t))_{1k} \le \gamma(D_kA(t)D_k^{-1})_{1}+\|\hat{A}(t)\|_{1k}  \le{}\\
{}\le - \rho \xi(t) +
3\varepsilon_1 +2\varepsilon_2 +  \varepsilon_3 \,.
% \label{3003}
\end{multline*}

Оценим теперь
 \begin{multline*}
\|{\bf p}(t)\|_{1k} \le
\|U(t){\bf p}(0) \|_{1k} +
 \int\limits_0^t \| U(t,\tau){\bf g}(\tau)\, d\tau \|_{1k } \le {}\\
{}\le e^{\int\limits_{0}^t \left(-\rho \xi(u)\right)\, du}\|{\bf p}(0) \|_{1k} +  k\int\limits_0^t \xi(\tau)
e^{\int\limits_{\tau}^t \left(-\rho \xi(u)\right) \, du} \, d\tau \le {}\\
{}\le \|{\bf p}(0) \|_{1k} + \fr{k}{\rho} < \infty
%\label{cat56}
\end{multline*}
при любом ${\bf p}(0)$, поскольку $\|{\bf g}(\tau)\|_{1k} = k\xi(\tau)$.
Теперь с учетом~(\ref{3000}) имеем

\noindent
\begin{multline}
\hspace*{-9pt}\left|E_{\bf p}(t)- \bar{E}_{\bf p}(t)\right| \le \left\|\vpi(t)-\bar{\vpi}(t)\right\|_{1k}  \le \phantom{\left|E_{\bf p}(t)- \bar{E}_{\bf p}(t)\right|}{}\\
{}\le \left(\left(3\varepsilon_1 +2\varepsilon_2 +  \varepsilon_3\right)\left(\|{\bf p}(0) \|_{1k} +
 \fr{k}{\rho}\right) +{}\right.\\
{}+\left. k\varepsilon_3\right) \int\limits_0^t e^{-\int_{\tau}^t\left( \rho \xi(u) -
3\varepsilon_1 -2\varepsilon_2 -  \varepsilon_3 \right)\,du}
\, d\tau\,.
\label{3004}
\end{multline}

\medskip

\noindent
\textbf{Теорема~2.} 
\textit{При совпадении начальных условий для исходного и возмущенного ПРГ с катастрофами для всех $t \ge 0$ 
справедлива оценка устойчивости среднего}~(\ref{3004}).

\medskip

\noindent
\textbf{Следствие 2.}
\textit{Пусть $\xi(t) \ge \xi >0$ почти при всех $t \ge 0$, а $X(0)=0$. Тогда вместо}~(\ref{3004}) \textit{получаем}
\begin{multline*}
\left|E_{0}(t)- \bar{E}_{0}(t)\right| \le
\left(\left(3\varepsilon_1 +2\varepsilon_2 +  \varepsilon_3\right) 
\fr{k}{\rho} + k\varepsilon_3\right)\times{}\\
{}\times \left( \rho \xi -
3\varepsilon_1 -2\varepsilon_2 -  \varepsilon_3 \right)^{-1}\,.
%\label{3005}
\end{multline*}

\medskip

\noindent
\textbf{Замечание 2.} Более точные оценки отклонения можно получить, ослабив условия малости возмущений 
(см.~(\ref{p1'}) и~(\ref{ek1})).

\section{Пример}

В современных моделях, связанных с финансовой математикой, рассматриваются марковские цепи и их устойчивость 
(см.~[17--19]).

Рассмотрим здесь простейшую нестационарную модель, описывающую число клиентов страховой компании.

А именно, пусть  вначале 
\begin{align*}
\lambda_{n-1}(t) &= \lambda(t) = 5+5\sin 2\pi t \,; \\
\mu_n(t) &= \mu(t) =5 + 5\cos 2\pi t \;,\\
\xi(t)&=1 + \sin 2 \pi t\,,\enskip n \ge 1\,,
\end{align*} 
а все $\varepsilon_i = \varepsilon$.

Применяя методы, описанные в наших предыду\-щих работах, и строя последовательность 
аппроксимирующих процессов так, как это предложено в~\cite{zAT}, получаем следующее утверждение.

\medskip

\noindent
\textbf{Теорема~3.} \textit{Справедлива следующая оценка:}

\textit{при} $X(0) = X_n (0) = 0$
\begin{equation*}
\|{\bf \pi}(t) - {\bf p}_N (t)\|_{1} \le 2 e^{-t}+ \fr{42 e^{8 t}}{2^{N+3}}
\end{equation*}
\textit{при всех $t \ge 0$ и любом~$N$}.

\medskip


\noindent
Д\,о\,к\,а\,з\,а\,т\,е\,л\,ь\,с\,т\,в\,о\,.\ \ 
Отметим прежде всего, что 
$$
\|{\mathbf \pi}(t) - {\mathbf p}_N (t)\|_{1} \le \|{\bf \pi} (t) - {\mathbf p} (t)\|_{1} + \|{\mathbf p} (t) - 
{\mathbf p}_N (t)\|_{1}\,.
$$

Воспользуемся вначале неравенством
\begin{multline}
\|{\mathbf p}^*(t)-{\mathbf p}^{**}(t)\|_{1} \le{}\\
{}\le  e^{\int\limits_0^t
\gamma\left(A(\tau)\right)_{1} d\tau}\|{\mathbf p}^*(0)-{\mathbf p}^{**}(0)\|_{1} \le {} \\
{}\le  e^{-\int\limits_0^t {(1+\sin{2\pi t})} d\tau}
 \|{\mathbf p}^*(0)-{\mathbf p}^{**}(0)\|_{1} \le{}\\
 {}\le e^{-t} \|{\mathbf p}^*(0)-{\mathbf p}^{**}(0)\|_{1}. \nonumber
\end{multline}

Теперь, выбирая ${\mathbf p}^*(0) = {\bf \pi}(0)$, ${\mathbf p}^{**}(0) = {\mathbf p}(0)=$\linebreak
$= {\mathbf e}_0$, получаем первое слагаемое правой части.
Далее рассмотрим $\left\|{\mathbf p} (t) - {\mathbf p}_N (t)\right\|_{1}$ для получения второго слагаемого.


Будем отождествлять векторы
$\left(x_1,\dots,x_N,0,\right.$\linebreak $\left.0,\dots\right)^{\mathrm{T}}$ и $\left(x_1,\dots,x_N
\right)^T$. Рассмотрим прямую сис\-те\-му Колмогорова  для
исходного процесса  в следующей форме:
\begin{equation*}
\fr{d\mathbf{p}}{dt}=A_N(t) \mathbf{p} + {\bf g}(t) +\left(A(t) -
A_N(t) \right) \mathbf{p}\,,
\end{equation*}
а также соответствующую систему
\begin{equation*}
\fr{d\mathbf{p_N}}{dt}=A_N(t) \mathbf{p_N} +  {\bf g}(t)
\end{equation*}
для усеченного процесса.


Имеем
\begin{equation*}
{\mathbf p}_N (t)= U_N(t){\mathbf p} (0) + \int\limits_0^t U_N(t,\tau){\mathbf
g}(\tau)\, d\tau
\end{equation*}
при ${\mathbf p} (0) = {\mathbf p}_N (0)$ и
\begin{multline*}
{\mathbf p} (t)= U_N \left(t\right) {\mathbf p} (0) + \int\limits_0^t
U_N(t,\tau){\bf g}(\tau)\, d\tau + {}\\
{}+\int\limits_0^t U_N \left(t,
\tau\right) \left(A(\tau) - A_N(\tau) \right) {\bf p} (\tau)\,
d\tau\,.
\end{multline*}
Тогда получаем
\begin{multline*}
\left\|{\mathbf p} (t) - {\mathbf p}_N (t)\right\| \le {}\\
{}\le \int\limits_0^t
\|U_N (t, \tau)\| \|(A(\tau) - A_N(\tau)) {\mathbf p} (\tau)\| d\tau \,.
\end{multline*}
Далее  $\|U_N (t, \tau)\|_1=1$, а
\begin{multline*}
\left(A -A_N\right) {\mathbf p} = \left(0,\dots,0,-\lambda_Np_N +
\mu_{N+1}p_{N+1},\right.\\
 \lambda_Np_N - \left(\lambda_{N+1} + \mu_{N+1}+\xi
\right)p_{N+1}{}\\
\left.{}+ + \mu_{N+2}p_{N+2},\dots \right)^{\mathrm{T}}\,;
\end{multline*}

\noindent
\begin{multline*}
\|(A -A_N) {\mathbf p}\| \le \|A\| \sum_{n \ge N} p_k={}\\
{}=(21+11\sin{2\pi t}+10\cos{2\pi t})\sum_{n \ge N} p_n \le 
42 \sum_{n \ge N} p_n\,.
\end{multline*}
С другой стороны,
\begin{multline*}
\sum_{n\ge {0}} \fr{d(2^n p_n)}{dt} = {}\\
{}= \sum_{n \ge {0}} {2^n\left(2\lambda_n - (\lambda_n + \mu_n + \xi) + 
\fr{\mu_n}{2}\right)}p_n \le{}\\
{}\le \sum_{n \ge {0}} {2^n(1{,}5 + 4 \sin{2\pi t}-2{,}5 \cos{2\pi t}})p_n
\end{multline*}
и, значит,
\begin{equation*}
\fr{d\left(\sum_{n\ge {0}}{2^n p_n}\right)}{dt}\le 8\sum_{n\ge {0}}{2^n p_n}\,.
\end{equation*}
Отсюда
\begin{align*}
\sum_{n\ge {0}}{2^{n} p_n(t)}&\le e^{8 t}\sum_{n\ge {0}}{2^n p_n (0)}
= e^{8 t}\,;\\
2^{N}\sum_{n\ge {N}}{p_n}&\le \sum_{n\ge {N}}{2^n p_n}\le e^{8 t}\,,
\end{align*}
а тогда
\begin{equation*}
\sum_{n \ge {N}}p_{n} \le \fr{e^{8 t}}{2^{N}}\,.
\end{equation*}
Таким образом,
\begin{equation*}
\left\|{\bf p} (t) - {\mathbf p}_N (t)\right\| \le \int\limits_0^t
\left(\fr{42 e^{8 t}}{2^N}\right)\, d\tau \le \fr{42 e^{8 t}}{2^{N+3}}\,,
\end{equation*}
откуда получается второе слагаемое.

\smallskip
Теперь несложно проверить, что для построения предельного режима для невозмущенного процесса с точностью~$10^{-8}$
достаточно выбрать
$N=273$, $t\in [20,21]$.

\medskip

\noindent

\textbf{Теорема 4.}
\textit{При $X(0) = X_N (0) = 0$ справедлива оценка}
\begin{equation*}
\left|\phi(t)- E_{0,N}(t)\right| \le  \fr{k e^{-\rho t}}{\rho}+\fr{22 e^{8 t}}{2^{N}}
\end{equation*}
\textit{при всех $t \ge 0$ и любом $N\ge k$.}

\medskip

\noindent
Д\,о\,к\,а\,з\,а\,т\,е\,л\,ь\,с\,т\,в\,о\,.\ 
Первое слагаемое правой час\-ти вытекает из неравенства
\begin{multline*}
\left|E_k(t) - E_0(t)\right| \le \|{\mathbf p}(t)-{\bf \pi}(t)\|_{1E} \le {}\\
{}\le \|{\mathbf p}(t)-{\bf \pi}(t)\|_{1k} \le 
\|\pi(0)\|_{1k} e^{-\rho t}\,,
\end{multline*}
если предварительно выбрать ${\mathbf p}^*(0) = {\bf \pi}(0)$, ${\mathbf p}^{**}(0) = {\mathbf p}(0)= {\mathbf e}_0$ 
и воспользоваться рассуждениями теоремы~2.

Оценим $\|{\mathbf \pi} (0)\|_{1k}$. Имеем с учетом 1-пе\-ри\-о\-дич\-ности
\begin{multline*}
 \|{\mathbf \pi} (0)\|_{1k} \le \overline{\lim_{t \to \infty}}
\|{\bf \pi} (t)\|_{1k} \le{}\\
{}\le \overline{\lim_{t \to \infty}} \left( \|{\bf \pi} (0)\|_{1k}
e^{-\rho \int_0^t \xi (u)\, du}+{}\right.\\
\left.{}+
\int\limits_0^t {\|g(\tau)\|_{1k} e^{-\rho \int\limits_{\tau}^t \xi (u)\, du}\, d\tau} \right) \le{}\\
{}\le \overline{\lim_{t \to \infty}} \left( \|{\mathbf \pi} (0)\|_{1k} e^{-\rho t} + 
k\int\limits_0^t {\xi(\tau) e^{-\rho \int\limits_{\tau}^t \xi (u)\, du} d\tau} \right) \le{}\\
{}\le  \overline{\lim_{t \to \infty}} \left(\fr{k}{\rho} (1-e^{-\rho \int_0^t \xi (u)\, du}) \right) \le  \fr{k}{\rho}\,.
\end{multline*}

\medskip
Теперь рассмотрим $\left|E_0(t)- E_{0,N}(t)\right|$ для получения второго слагаемого.
В любой норме выполняется
\begin{multline*}
\left\|{\mathbf p} (t) - {\mathbf p}_N (t)\right\| = {}\\
{}=\left\|\int\limits_0^t
U_N \left(t,
 \tau\right) \left(A(\tau) - A_N(\tau) \right) {\mathbf p} (\tau)\, d\tau \right\|\,.
\end{multline*}

Рассмотрим матрицу Коши
\begin{equation*}
U_N =\begin{pmatrix}
  u_{00}^N & . & . & u_{0N}^N  & 0 & 0 & \cdots \\
u_{10}^N & . & . & u_{1N}^N  & 0 & 0 & \cdots \\
\cdots \\
u_{N0}^N & . & . & u_{NN}^N  & 0 & 0 & \cdots \\
0 & . & . & 0 & 1 & 0 & \cdots \\
0 & . & . & 0 & 0 & 1 & \cdots \\
\cdots
\end{pmatrix}\,.
\end{equation*}
\begin{figure*}[b] %fig1
\vspace*{1pt}
\begin{minipage}[t]{80mm}
\begin{center}
\mbox{%
\epsfxsize=78.547mm %79.852mm
\epsfbox{zei-1.eps}
}
\end{center}
\vspace*{-9pt}
\Caption{Приближенное значение предельной величины $J_{0}(t)=
\mathrm{Pr}\left(\bar{X}(t)  = 0\right)$ для случая~1. Среднее значение по
периоду приближенно равно 0,319
\label{f1zei}}
\end{minipage}
\hfill
%\end{figure*}
%\begin{figure*} %fig2
\vspace*{1pt}
\begin{minipage}[t]{80mm}
\begin{center}
\mbox{%
\epsfxsize=77.25mm %78.555mm
\epsfbox{zei-2.eps}
}
\end{center}
\vspace*{-9pt}
\Caption{Приближенное значение предельной величины $J_{0}(t)=
\mathrm{Pr}\left(\bar{X}(t)  = 0\right)$ для случая~2. Среднее значение по
периоду приближенно равно 0,433
\label{f2zei}}
\end{minipage}
\end{figure*}

\noindent
Тогда
\begin{multline*}
\left(A -A_N\right) {\mathbf p} = \left(0,\dots,0,-\lambda_Np_N +
\mu_{N+1}p_{N+1}\,,\right.\\
\left.\lambda_Np_N - \left(\lambda_{N+1} + \mu_{N+1}+\xi
\right)p_{N+1} + {}\right.\\
{}\left. +\mu_{N+2}p_{N+2},\dots \right)^{\mathrm{T}}
\end{multline*}
и, следовательно,

\noindent
{\small
\begin{multline*}
U_N\left(A -A_N\right) {\mathbf p} ={}\\
{}=
\begin{pmatrix}
u_{0N}^N\left(-\lambda_Np_N +
\mu_{N+1}p_{N+1}\right)\\
u_{1N}^N\left(-\lambda_Np_N +
\mu_{N+1}p_{N+1}\right) \\ \vdots \\
u_{NN}^N\left(-\lambda_Np_N +
 \mu_{N+1}p_{N+1}\right) \\ \lambda_Np_N -\left(\lambda_{N+1} +
\mu_{N+1}+\xi\right)p_{N+1} + \mu_{N+2}p_{N+2} \\ \vdots
\end{pmatrix}.\hspace*{-0.75854pt}
\end{multline*}
}
С учетом неравенств $ u_{ij}^N (t,\tau) \ge 0$ (при всех $i,j,
t,\tau $) и равенств $\sum_i u_{ij}^N (t,\tau) = 1$ (при всех $j,
t,\tau $) получаем оценку
\begin{multline*}
\|U_N (A-A_N)p\|_{1E} ={}\\
{}=
|-\lambda_Np_N + \mu_{N+1}p_{N+1}|\sum_{n\leq N} n
u_{nN}^N +{}\\
{}+ \sum_{n \ge N} (n+1) | \lambda_n p_n
 - (\lambda_{n+1} + \mu_{n+1} + \xi)p_{n+1} +{}\\
 {}+
\mu_{n+2}p_{n+2} | \le %{}\le {}
(2\lambda + 2\mu + \xi)(\sum_{n \ge N}np_n +\sum_{n \ge N}p_n)\,.
\end{multline*}

Первое слагаемое уже оценено: $ \sum_{n \ge {N}}p_{n} \le$\linebreak $\le e^{8 t}/2^{N}$.

Рассмотрим первое слагаемое и оценим его аналогичным способом, учитывая, что $2\lambda(t)-\mu(t) \le$\linebreak $\le 25$,
\begin{multline*}
\fr{d(\sum\limits_{n \ge {1}}np_{n})}{dt}\le{}\\
\le \lambda(t) p_0 + (\lambda(t) - 
\mu(t))\sum\limits_{n \ge {1}}p_{n} -\xi(t) \sum\limits_{n \ge {1}}np_{n} \le{}\\
{}\le (\lambda(t) - \mu(t))\fr{e^{8 t}}{2^{N}}+\lambda(t) \fr{e^{8 t}}{2^{N}} \le \fr{25 e^{8 t}}{2^{N}}\,.
\end{multline*}
Тогда
\begin{multline*}
\|U_N (A-A_N)p\|_{1E}  \le{}\\
{}\le (2\lambda(t) + 2\mu(t)+ \xi(t))\fr{e^{8 t}}{2^{N}}\left(\fr{25}{8}  + 1\right)
\end{multline*}
и получаем
\begin{equation*}
|E_0(t)- E_{0,n}(t)| \le \fr{174}{2^N} \int\limits_0^t e^{8 \tau}\, d\tau \le \fr{22 e^{8 t}}{2^{N}}\,.
\end{equation*}

\medskip
Для построения предельного среднего невозмущенного процесса с точностью~$10^{-8}$ 
выбираем $N=341$, $t\in [26,27]$, положив при этом
$\rho = 0{,}9$; $k = 50$.

\medskip
Далее, в оценках устойчивости~(\ref{stab1}) и~(\ref{3004}) теорем~1 и~2 получаем, 
выбирая $\varepsilon_1 =\varepsilon_2 =\varepsilon_3=\varepsilon$,
\begin{equation*}
\left\|\vpi(t)-\bar{\vpi}(t)\right\|\le
6\varepsilon \int\limits_0^t {e^{-\int_{\tau}^t \xi (u)\, du}\, d\tau}  \le 6\varepsilon e^{1/\pi}
\end{equation*}
и
\begin{multline*}
\left|E_{\mathbf p}(t)- \bar{E}_{\mathbf p}(t)\right| \le{}\\
{}\le\left( \fr{6\varepsilon k}{\rho}+k\varepsilon \right)
\int\limits_0^t {e^{-\int_{\tau}^t {(\rho \xi(u)-6\varepsilon)})\, du}\, d\tau} \le {}\\
{}\le
\fr{\varepsilon k (\rho +6)}{\rho(\rho -6\varepsilon)}\, e^{\rho/\pi}
\end{multline*}
соответственно.


\bigskip

Пусть  теперь при тех же интенсивностях поступления и обслуживания клиентов  
$\lambda_{n-1}(t) = \lambda(t) =$\linebreak $= 5+5\sin 2\pi t $, $\mu_n(t) = \mu(t)=5 + 5\cos 2\pi t $ 
интенсивность катастрофы  $\xi(t)=2 + 2\sin 2 \pi t$ вдвое больше.

\begin{figure*} %fig3
\vspace*{1pt}
\begin{minipage}[t]{80mm}
\begin{center}
\mbox{%
\epsfxsize=78.755mm %80.061mm
\epsfbox{zei-3.eps}
}
\end{center}
\vspace*{-12pt}
\Caption{Приближенное значение предельного среднего для возмущенного
процесса (случай~1). Среднее значение по периоду~--- оно же двойное
среднее~--- приближенно равно 2,100
\label{f3zei}}
%\end{figure*}
\end{minipage}
\hfill
%\begin{figure*} %fig4
\vspace*{1pt}
\begin{minipage}[t]{80mm}
\begin{center}
\mbox{%
\epsfxsize=78.755mm %80.835mm
\epsfbox{zei-4.eps}
}
\end{center}
\vspace*{-12pt}
\Caption{Приближенное значение предельного среднего для возмущенного
процесса (случай~2). Среднее значение по периоду~--- оно же двойное
среднее~--- приближенно равно 1,357
\label{f4zei}}
\end{minipage}
%\vspace*{-12pt}
\end{figure*}

Тогда тем же образом проверяется, что для получения требуемой точности предельного режима достаточно 
взять $N=127$, $t\in [10,11]$, а для построения предельного среднего~--- $N=155$, $t\in [13,14]$. 
При этом выписанные оценки устойчивости заведомо также выполнены.

Интересно сравнить построенные ниже предельные характеристики рассмотренных примеров
(рис.~1--4).

%\vspace*{-6pt}

{\small\frenchspacing
{ %\baselineskip=10.8pt
\addcontentsline{toc}{section}{Литература}
\begin{thebibliography}{99}

\bibitem{du1} %1
\Au{Dudin~A., Nishimura~S.}  
A BMAP/SM/1 queueing system with Markovian arrival input of disasters~//
J. Appl. Probab., 1999. Vol.~36. P.~868--881.

\bibitem{KK} %2
\Au{Krishna~Kumar~B., Arivudainambi~D.} 
Transient solution of an $M/M/1$ queue with catastrophes~// Comput. Math. Appl., 2000. Vol.~40. P.~1233--1240.

\bibitem{du2} %3
\Au{Dudin~A., Karolik~A.} 
BMAP/SM/1 queue with Markovian input of disasters and non-instantaneous recovery~//
Perform. Eval., 2001. Vol.~45. P.~19--32.

\bibitem{Di} %4
\Au{Di~Crescenzo~A., Giorno~V., Nobile~A.\,G., Ricciardi~L. M.}  
On the M/M/1 queue with catastrophes and its continuous
approximation~// Queueing Syst., 2003. Vol.~43. P.~329--347.

\bibitem{Di08} %5
\Au{Di~Crescenzo~A., Giorno~V., Nobile~A.\,G., Ricciardi~L.\,M.}
A~note on birth-death processes with catastrophes~//  Statist.
Probab. Lett., 2008. Vol.~78.  P.~2248--2257.

\bibitem{z08} %6
\Au{Zeifman~A., Satin~Ya., Chegodaev~A., Bening~V., Shorgin~V.}
Some bounds for $M(t)/M(t)/S$ queue with catastrophes~// 4th 
Conference (International) on Performance Evaluation Methodologies and Tools Proceedings.  Athens, Greece, October~20--24, 2008.

\bibitem{z09a} %7
\Au{Зейфман А.\,И., Сатин Я.\,А., Чегодаев~А.\,В.} 
О нестационарных системах обслуживания с катастрофами~// Информатика и её применения, 2009. Т.~3. Вып.~1. С.~47--54.

\bibitem{z09b} %8
\Au{Зейфман А.\,И., Сатин~Я.\,А., Коротышева~А.\,В., Терешина~Н.\,А.} 
О предельных характеристиках системы обслуживания $M(t)/M(t)/S$ с катастрофами~// 
Информатика и её применения, 2009. Т.~3. Вып.~3. С.~16--22.

\bibitem{z09c} %9
\Au{Zeifman A., Satin~Ya., Shorgin~S., Bening~V.} 
On $M_n(t)/M_n(t)/S$ queues with catastrophes~//  4th 
Conference (International) on Performance Evaluation Methodologies and Tools Proceedings.  Pisa, Italy, October~19--23, 2009.

\bibitem{z85}  %10
\Au{Zeifman A.\,I.} 
Stability for contionuous-time nonhomogeneous Markov chains~// Lect. Notes Math.,  1985. Vol.~1155. P.~401--414.

\bibitem{z98} %11
\Au{Zeifman A.} 
Stability of birth and death processes~// J.~Mathematical Sciences, 1998. Vol.~91. P.~3023--3031.


\bibitem{ae} %12
\Au{Андреев Д., Елесин~М., Кузнецов~А., Крылов~Е., Зейфман~А.} 
Эргодичность и устойчивость нестационарных систем обслуживания~// 
Теория вероятностей и математическая статистика, 2003. Т.~68. С.~1--11.

\bibitem{z08b} %13
\Au{Зейфман А.\,И., Бенинг В.\,Е., Соколов~И.\,А.} 
Марковские цепи и модели с непрерывным временем.~--- М.: Элекс-КМ, 2008.

\bibitem{gz00} %14
\Au{Granovsky~B., Zeifman~A.} 
The N-limit of spectral gap of a class of birth-death Markov chains~// Appl. Stoch. Models in Business and
Industry, 2000. Vol.~16. P.~235--248.

\bibitem{gz04} %15
\Au{Granovsky~B., Zeifman~A.} 
Nonstationary queues: Estimation of the rate of convergence~// Queueing Syst., 2004.  Vol.~46.  P.~363--388.

\bibitem{z06} %16
\Au{Zeifman~A., Leorato~S., Orsingher~E., Satin~Ya., Shilova~G.}
Some universal limits for nonhomogeneous birth and death
processes~// Queueing Syst., 2006. Vol.~52. P.~139--151.

\bibitem{e} %17
\Au{Enikeeva F., Kalashnikov V., Rusaitite D.} 
Continuity estimates for ruin probabilities~//  J. Scand. Actuarial, 2001. No.~1. P.~18--39.

\bibitem{i}  %18
\Au{Islam M.\,A.} 
A birth-death process approach to constructing multistate life tables~// Bull. Malaysian Math. Sc. Soc. (Second Series), 
2003. Vol.~26. P.~101--108.

\bibitem{cn}  %19
\Au{Ching W.-K., Ng M.\,K.} 
Markov chains: Models, algorithms and applications. International ser. in operations research \& management
science.~--- N.Y.: Springer,
2006. 

\label{end\stat}

\bibitem{zAT} %20
\Au{Зейфман А.\,И.} 
О нестационарной модели Эрланга~// Автоматика и телемеханика, 2009. Вып.~12. С.~71--80.


 \end{thebibliography}
}
}

\end{multicols}