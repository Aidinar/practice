\documentclass[10pt]{book}
\usepackage[utf8]{inputenc}

\usepackage{latexsym,amssymb,amsfonts,amsmath,indentfirst,shapepar,%fleqn,%
picinpar,shadow,floatflt,enumerate,multicol,colortbl,ipi}

\usepackage{rotating}
\input{epsf}

%\nofiles

%\includeonly{avtor,avtor-eng}
%\includeonly{avtor-eng}
%\includeonly{pred}  %+
%\includeonly{podgot-2str}  %+


%\includeonly{kudr+shorgin}  %+pdf %3
%\includeonly{chuprakov}  %9 pdf
%\includeonly{zeifman} %2pdf
%\includeonly{chuprunov} %7pdf
%\includeonly{matveeva}  %4pdf
%\includeonly{zatsman} %11 pdf
%\includeonly{frenkel} %1pdf
%\includeonly{frcorr} %1pdf
%\includeonly{frcorr-2} %1pdf
%\includeonly{grusho}  %6pdf
%\includeonly{kozerenko} %10pdf
%\includeonly{borodina} %5pdf
%\includeonly{ubilei}  %pdf
%\includeonly{lyamin} %8

%\includeonly{toc-rus,toc-en}
%\includeonly{toc-en}

%\includeonly{obchak}
%\includeonly{reshal}
%\includeonly{eng-index}
%\includeonly{cover3}

\usepackage{acad}
\usepackage{courier}
\usepackage{decor}
\usepackage{newton}
\usepackage{pragmatica}
\usepackage{zapfchan}
\usepackage{petrotex}
\usepackage{bm}                     % полужирные греческие буквы
\usepackage{upgreek}                % прямые греческие буквы
%\usepackage{verbatim}

\renewcommand{\bottomfraction}{0.99}
\renewcommand{\topfraction}{0.99}
\renewcommand{\textfraction}{0.01}

\setcounter{secnumdepth}{1} %здесь - 3 + chapter = 4

\arraycolsep=1.5pt

%\usepackage[pdftex]{graphicx}

%\usepackage{oz}

%NEW COMMANDS



%\renewcommand{\r}{{\rm I\hspace{-0.7mm}\rm R}}
\renewcommand{\r}{\mathbb{R}}
\newcommand{\I}{{\rm I\hspace{-0.7mm}I}}
\newcommand{\Ik}{\mbox{{\small \tt {1}}\hspace{-1.5mm}{\tt 1}}}
%\newcommand{\Ikl}{{\small \tt{1}}\hspace*{-0.4mm}\mathtt{1}}

%\mathrm{I}\hspace*{-0.7mm}\mathrm{R}

\newcommand{\il}[2]{\int\limits_{#1}^{#2}}%интеграл с пределами #1 и #2

%\newcommand{\il}[0]{\int\limits_{#1}^{#2}}%интеграл с пределами #1 и #2

\newcommand{\h}{{\bf H}}
\newcommand{\p}{{\sf P}}  % вероятность
\newcommand{\e}{{\sf E}}  % мат. ожидание
\newcommand{\D}{{\sf D}}  % дисперсия
\newcommand{\eps}{\varepsilon}
\newcommand{\vp}{\mathrm{v.p.}}
\newcommand{\F}{{\mathcal F}}
%\def\iint{\int\limits_{-\infty}^{\infty}}

%\newcommand{\gr}{{\geqslant}}

\newcommand{\g}{\mbox{\textit{g}}}

%\renewcommand{\la}{\lambda}
\newcommand{\si}{\sigma}
%\renewcommand{\a}{\alpha}

%\newcommand{\pto}{\stackrel{P}{\longrightarrow}} % сходимость по веpоятности

%\newcommand{\eqd}{\stackrel{d}{=}} % равенство по pаспpеделению

%\newcommand{\kp}{\kappa}
%\def\Q{{\cal Q}} \def\H{{\cal H}}
%\newcommand{\bet}{\beta_{2+\delta}}


%\newtheorem{definition}{Определение}
%\renewcommand{\thedefinition}{\arabic{definition}.}
%END NEW COMMANDS

%\renewcommand{\baselinestretch}{1.2}

%\pagestyle{myheadings}

\setlength{\textwidth}{167mm}      % 122mm
\setlength{\textheight}{658pt}
%\setlength{\textheight}{635.6pt}
\setlength{\columnsep}{4.5mm}

\setcounter{secnumdepth}{4}

%\addtolength{\headheight}{2pt}
%\addtolength{\headsep}{-2mm}

%\addtolength{\topmargin}{-20mm}  % for printing


\hoffset=-30mm  % From Yap
%\hoffset=-20mm  % From Acrobat

%\voffset=0mm % From Yap
%\voffset=-15mm   % From Acrobat

\addtolength{\evensidemargin}{-9.5mm} % for printing
\addtolength{\oddsidemargin}{9.5mm}  % for printing

%\renewcommand{\thefootnote}{\fnsymbol{footnote}}
%\renewcommand{\thefootnote}{\arabic{footnote}}
\renewcommand{\figurename}{\protect\bf Рис.}
\renewcommand{\tablename}{\protect\bf Таблица}

\newcommand{\Caption}[1]{\caption{\protect\small %\baselineskip=2.5ex
#1}}

\renewcommand{\thefigure}{\arabic{figure}}
\renewcommand{\thetable}{\arabic{table}}
\renewcommand{\theequation}{\arabic{equation}}
\renewcommand{\thesection}{\arabic{section}}

\renewcommand{\contentsname}{СОДЕРЖАНИЕ}
\newcommand{\fr}[2]{\displaystyle\frac{\displaystyle #1\mathstrut}{\displaystyle #2\mathstrut}}

%\renewcommand{\thefootnote}{\fnsymbol{footnote}}
%\newcommand{\g}{\mbox{\textit{g}}}

%\newcommand{\Caption}[1]{\caption{\protect\small\baselineskip=2ex #1}}
\newcounter{razdel}
\setcounter{razdel}{0}


\newcommand{\titel}[4]{%
\

\vspace*{5pt}

\ifodd\therazdel {\raggedright\noindent\Large\textrm\textbf
 \lineskip .75em
  \baselineskip=3.2ex #1 \par}
\vskip 1em {\noindent\large\textrm\textbf #2 \par}
\addcontentsline{toc}{subsection}{{\textrm\textbf #3}\protect\newline #1}
\def\rightheadline{\underline{\noindent\hbox to \textwidth{\hfill\small\textrm{#4}
%\hfill \large\bf\thepage
}}}
\def\leftheadline{\underline{\noindent\parbox{\textwidth}{
%\raggedleft\large\bf\thepage \hfill
\small\textit{#3}\hfill}}}
\def\leftfootline{\small{\textbf{\thepage}
\hfill ИНФОРМАТИКА И ЕЁ ПРИМЕНЕНИЯ\ \ \ том~4\ \ \ выпуск 3\ \ \ 2010}
}%
 \def\rightfootline{\small{ИНФОРМАТИКА И ЕЁ ПРИМЕНЕНИЯ\ \ \ том~4\ \ \ выпуск~3\ \ \ 2010
\hfill \textbf{\thepage}}} \vskip 2em \setcounter{figure}{0}
\setcounter{table}{0} \setcounter{equation}{0} \setcounter{section}{0}
\setcounter{subsection}{0} \setcounter{subsubsection}{0}
\setcounter{footnote}{0} \setcounter{razdel}{0}
%\end{flushleft}
\else {
 \raggedright\noindent\Large\textrm\textbf
 \lineskip .75em
\baselineskip=3.2ex #1 \par} \vskip 1em
%\begin{flushleft}
{\noindent\large\textrm\textbf #2 \par}
\addcontentsline{toc}{subsection}{{\textrm\textbf #3}\protect\newline #1}
\def\rightheadline{\underline{\noindent\hbox to \textwidth{\hfill\small\textrm{#4}
%\hfill \large\bf\thepage
}}}
\def\leftheadline{\underline{\noindent\parbox{\textwidth}{%\raggedleft\large\bf\thepage \hfill
\small\textit{#3}\hfill}}}
\def\leftfootline{\small{\textbf{\thepage}
\hfill ИНФОРМАТИКА И ЕЁ ПРИМЕНЕНИЯ\ \ \ том~4\ \ \ выпуск~3\ \ \ 2010}
}%
 \def\rightfootline{\small{ИНФОРМАТИКА И ЕЁ ПРИМЕНЕНИЯ\ \ \ том~4\ \ \ выпуск~3\ \ \ 2010
\hfill \textbf{\thepage}}} \vskip 2em \setcounter{figure}{0}
\setcounter{table}{0} \setcounter{equation}{0} \setcounter{section}{0}
\setcounter{subsection}{0} \setcounter{subsubsection}{0}
\setcounter{footnote}{0}
%\end{flushleft}
\fi}

\newcommand{\titelr}[2]{%
\

\vspace*{5pt}

\ifodd\therazdel {\raggedright\noindent\large\textrm\textbf
 \lineskip .75em
  \baselineskip=3.2ex #1 \par}
\vskip 1em {\noindent\normalsize\textrm\textbf #2 \par}
\else {
 \raggedright\noindent\large\textrm\textbf
 \lineskip .75em
\baselineskip=3.2ex #1 \par} \vskip 1em
%\begin{flushleft}
{\noindent\normalsize\textrm\textbf #2 \par}
\fi}

\newcommand{\titele}[5]{%
\

%\vspace*{5pt}

\ifodd\therazdel {\raggedright\noindent%\large
\textrm\textbf
 \lineskip .75em
%  \baselineskip=3.2ex
#1 \par}
\vskip .5em {\noindent\large\textrm\textbf #2 \par}
\vskip .5em
 {\noindent\textrm #3 \par}
\addcontentsline{toc}{subsection}{{\textrm\textbf #1}\protect\newline #2}
\def\rightheadline{\underline{\noindent\hbox to \textwidth{\hfill\small\textrm{#4}
%\hfill \large\bf\thepage
}}}
\def\leftheadline{\underline{\noindent\parbox{\textwidth}{
%\raggedleft\large\bf\thepage \hfill
\small\textrm{#5}\hfill}}}
\def\leftfootline{\small{\textbf{\thepage}
\hfill ИНФОРМАТИКА И ЕЁ ПРИМЕНЕНИЯ\ \ \ том~4\ \ \ выпуск~3\ \ \ 2010}
}%
 \def\rightfootline{\small{ИНФОРМАТИКА И ЕЁ ПРИМЕНЕНИЯ\ \ \ том~4\ \ \ выпуск~3\ \ \ 2010
\hfill \textbf{\thepage}}} \vskip 1em \setcounter{figure}{0}
\setcounter{table}{0} \setcounter{equation}{0} \setcounter{section}{0}
\setcounter{subsection}{0} \setcounter{subsubsection}{0}
\setcounter{footnote}{0} \setcounter{razdel}{0}
%\end{flushleft}
\else {
 \raggedright\noindent%\large
 \textrm\textbf
 \lineskip .75em
%\baselineskip=3.2ex
#1 \par} \vskip .5em
%\begin{flushleft}
{\noindent\large\textrm\textbf #2 \par} \vskip .5em
 {\noindent\textrm #3 \par}
\addcontentsline{toc}{subsection}{{\textrm\textbf #1}\protect\newline #2}
\def\rightheadline{\underline{\noindent\hbox to \textwidth{\hfill\small\textrm{#4}
%\hfill \large\bf\thepage
}}}
\def\leftheadline{\underline{\noindent\parbox{\textwidth}{%\raggedleft\large\bf\thepage \hfill
\small\textrm{#5}\hfill}}}
\def\leftfootline{\small{\textbf{\thepage}
\hfill ИНФОРМАТИКА И ЕЁ ПРИМЕНЕНИЯ\ \ \ том~4\ \ \ выпуск~3\ \ \ 2010}
}%
 \def\rightfootline{\small{ИНФОРМАТИКА И ЕЁ ПРИМЕНЕНИЯ\ \ \ том~4\ \ \ выпуск~3\ \ \ 2010
\hfill \textbf{\thepage}}} \vskip 1em \setcounter{figure}{0}
\setcounter{table}{0} \setcounter{equation}{0} \setcounter{section}{0}
\setcounter{subsection}{0} \setcounter{subsubsection}{0}
\setcounter{footnote}{0}
%\end{flushleft}
\fi}

\def\Abst#1{
\begin{center}\small\nwt
\parbox{150mm}{%\baselineskip=2.5ex
\textbf{Аннотация:}\ \
%\hspace*{\parindent}
#1}
\end{center}}
\def\Abste#1{
\begin{center}\small\nwt
\parbox{150mm}{%\baselineskip=2.5ex
\textbf{Abstract:}\ \
%\hspace*{\parindent}
#1}
\end{center}}

\def\KW#1{
\begin{center}\small\nwt
\parbox{150mm}{%\baselineskip=2.5ex
\textbf{Ключевые слова:}\ \ #1}
\end{center}}

\def\KWE#1{
\begin{center}\small\nwt
\parbox{150mm}{%\baselineskip=2.5ex
\textbf{Keywords:}\ \ #1}
\end{center}}


\def\KWN#1{
%\begin{center}
%\small
%\parbox{150mm}\end{center}
}

\renewcommand{\thesubsection}{\thesection.\arabic{subsection}\hspace*{-5pt}}
\renewcommand{\thesubsubsection}{\thesubsection\hspace*{5pt}.\arabic{subsubsection}\hspace*{-3pt}}

\begin{document}
\Rus

\nwt
%\ptb

%\renewcommand{\contentsname}{\protect\Large\bf Содержание}

\setcounter{tocdepth}{2}

%\tableofcontents

\renewcommand{\bibname}{\protect\rmfamily Литература}
  \def\Au#1{{\it #1}}

%\newcommand{\No}{№}
  \newcommand{\tg}{\,\mathrm{tg}\,}
    \newcommand{\ctg}{\,\mathrm{ctg}\,}
  \newcommand{\arctg}{\,\mathrm{arctg}\,}
  
\def\forallb{\mathop{\forall}}
\def\existsb{\mathop{\exists}}

\setcounter{page}{1}

\newpage
\addtocounter{razdel}{1}
%\def\razd{РЕГУЛИРУЕМЫЙ ЭЛЕКТРОПРИВОД ДЛЯ ЭЛЕКТРОЭНЕРГЕТИКИ}
%\newpage
%\def\stat{zakh}
\def\tit{СРЕДСТВА ОБЕСПЕЧЕНИЯ ОТКАЗОУСТОЙЧИВОСТИ ПРИЛОЖЕНИЙ}
\def\titkol{Средства обеспечения отказоустойчивости приложений}

\def\aut{В.\,Н.~Захаров$^1$, В.\,А.~Козмидиади$^2$}
\titel{\razd}{\tit}{\aut}{\titkol}


\Abst{Рассмотрены проблемы построения отказоустойчивых серверов, возникающие в связи с недетерминированностью поведения приложений. Предложена формальная модель, описывающая поведение приложения, основными объектами которой являются ресурсы и события. Предложены алгоритмы протоколирования работы приложения на резервном узле кластера, а также восстановления и продолжения его работы при отказе основного узла. При этом для клиентов сбой остается незаметным, за исключением некоторого увеличения времени обслуживания.}

\KW{сервер приложений $\bullet$ прозрачная отказоустойчивость $\diamond$
 процесс $\diamond$ ресурс $\diamond$ событие $\diamond$ контрольная точка
$\bullet$ детерминированность}

\vskip 12pt plus 6pt minus 3pt

\begin{multicols}{2}

\section*{ВВЕДЕНИЕ}

Средства вычислительной техники стали использоваться в областях,
требующих безотказной работы систем в течение многих лет (24 часа
в сутки, 365 дней в году).

\label{st\stat}

\footnotetext{$^1$ФГУП Центральный институт авиационного моторостроения
им. П.И. Баранова, Москва, Россия}
\footnotetext{$^2$ФГУП Центральный институт авиационного моторостроения
им. П.И. Баранова, Москва, Россия}

К таким областям относятся, например, центры хранения и обработки данных  в сетях (системы резервирования билетов, биллинговые,  банковские и т.д.), массированные распределенные вычисления (GRID-вычисления) и другие.

\thispagestyle{headings}

Обычно в подобных системах применяются частные решения, ориентированные в основном на обеспечение надежного хранения данных (например, файловые серверы, использующие для хранения RAID-контроллеры) и корректного их состояния при отказах (серверы баз данных с транзакционным выполнением запросов). Однако большинство приложений не гарантируют, что не произойдет потери части данных при отказе системы. Обычно предполагается, что клиентские средства должны повторять запросы после восстановления серверов, для того, чтобы данные не были потеряны, или что можно сделать возврат по времени на некоторое время назад и повторить работу с этого места. Однако далеко не все клиентские средства и условия применения приложений допускают это.

Отказоустойчивые системы для критически важных приложений, корректно решающие проблемы восстановления после сбоев,   предлагаемые ведущими производителями, как правило, дороги. Кроме того, они включают специфические серверные и клиентские приложения, не совместимые со стандартными приложениями, не обеспечивающими отказоустойчивость. Примером такого подхода к решению проблемы отказоустойчивости  хранения данных являются системы NetApp FAS компании Network Appliance, работающие на базе специализированной операционной системы Data ONTAP [1].

Построение отказоустойчивых систем, использующих серверы со стандартными приложениями, в свете вышесказанного, является актуальной проблемой, вызывающей значительный интерес. Рассмотрение методов достижения прозрачной отказоустойчивости таких систем и является предметом статьи.
\begin{figure*} %fig1
\vspace*{1pt}
\begin{center}
\mbox{%
\epsfxsize=1.6in
\epsfxsize=100mm
\epsfbox{BbR-1.eps}
}
\end{center}
\vspace*{-9pt}
\Caption{Базовый вариант трубы с разными выходными устройствами
(цилиндрическое, расширяющееся и сужающееся сопло)
\label{f1bab}}
\vspace*{-3pt}
\end{figure*}


\section{ОСНОВНЫЕ ПОНЯТИЯ И ПОДХОДЫ}

Под сервером в данной работе понимается вычислительный центр
(отдельный компьютер или кластер) в сети, предоставляющий клиентам
(пользователям, клиентским компьютерам) определенные услуги, разделяя
между ними свои ресурсы. Подобные серверы названы серверами приложений.
Широко распространенным примером сервера такого типа является файловый сервер, обеспечивающий удаленный коллективный доступ к файловой системе. Часто используются вычислительные серверы, предоставляющие клиентам возможность выполнять на них свои программы (например, в центрах коллективного пользования).


Обычно приложение представляет собой программу или группу программ, работающих в операционной среде, создаваемой операционной системой (в другой терминологии - один или несколько взаимодействующих процессов или потоков (threads)), которые реализуют функциональность сервера. Для построения отказоустойчивых серверов приложений широко используется кластерная технология. Следуя [2], кластером, названа разновидность параллельной или распределенной системы, которая:
\begin{itemize}
\item состоит из нескольких компьютеров (узлов кластера), связанных как минимум необходимыми коммуникационными каналами;
\item используется как единый, унифицированный компьютерный ресурс.
\end{itemize}

Прозрачная отказоустойчивость (Transparent Fault Tolerance, TFT) сервера приложений - это такое его поведение при возникновении аппаратных или программных отказов либо отказов в сети, при котором:
\begin{itemize}
\item отказ не вызывает потери или искажения данных, находящихся в базе данных сервера;
\item сервер продолжает нормально функционировать, несмотря на имевшие место отказы.
\end{itemize}

Клиенты сервера "не замечают" произошедших отказов. Единственным\footnote{допустимым
отклонением сервера от нормального поведения с точки зрения клиента является
некоторое увеличение времени обслуживания} (на несколько секунд или десятков секунд).

Обычно приложения, работающие на серверах приложений, не ориентированы на прозрачную отказоустойчивость. Они "заботятся" лишь о собственной целостности (например, состояния файловой системы или базы данных). Восстановление работоспособности сервера приводит к разрыву соединений с клиентами и потере их запросов. Это замечают клиенты - запросы следует повторять, на что клиентские приложения далеко не всегда рассчитаны. В данной работе предполагается, что приложения (прикладные программные средства), выполняемые на сервере, являются стандартными, то есть не имеют специальных средств, обеспечивающих отказоустойчивость.
\begin{figure*}[b] %fig1
\vspace*{1pt}
\begin{center}
\mbox{%
\epsfxsize=1.6in
\epsfxsize=100mm
\epsfbox{BbR-1.eps}
}
\end{center}
\vspace*{-9pt}
\Caption{Базовый вариант трубы с разными выходными устройствами
(цилиндрическое, расширяющееся и сужающееся сопло)
\label{f1bab}}
\vspace*{-3pt}
\end{figure*}

Серьезные исследования в области обеспечения отказоустойчивости серверов были развернуты после создания вычислительных серверов, предназначенных для решения задач, требующих больших вычислительных ресурсов. Решение этих задач выполняется на суперкомпьютерах, обеспечивающих массово-параллельные вычисления и представляющих собой кластеры из сотен и тысяч узлов (процессоров). Однако даже на этих "монстрах" решение может требовать десятков или сотен часов, и одиночный сбой, если не предприняты специальные меры, может привести к необходимости начинать работу сначала. Обычно решение вычислительной задачи в таких случаях осуществляется в модели относительно редко взаимодействующих между собой процессов, выполняемых на разных узлах кластера. Эти взаимодействия нужны для координации работы процессов, в частности, для обмена данными и промежуточными результатами. Взаимодействия опираются на специальный протокол, называемый MPI (Message-Passing Interface) и представляющий собой стандарт "de facto" [3].

Для преодоления последствий сбоя достаточно давно была разработана и широко применяется технология, опирающаяся на механизм контрольных точек (checkpoints) [4-6]. По этой технологии система должна иметь стабильную память, которая не меняется при отказах. Соответствующие программные средства периодически сохраняют информацию о состоянии процессов приложения в стабильной памяти. Все процессы также имеют доступ к устройству стабильной памяти.  В случае отказа или сбоя, записанная в стабильную память информация используется для повторения вычисления с момента, когда была записана эта информация, то есть выполняется откат назад по времени. Данные, сохранение которых позволяет выполнить откат, называются контрольной точкой. В качестве устройства стабильной памяти может использоваться дисковый том, энергонезависимая оперативная память, память другого узла или узлов кластера. В последнем случае узел, которому требуется сохранить информацию, пересылает ее через быстрый канал связи на другой узел. Стабильная память после отказа одного из узлов должна быть доступной узлу, на котором делается повтор.

Однако решение, опирающееся только на контрольные точки, не является прозрачным, поскольку не скрывает от клиентов факт отказа системы и требует от них выполнения определенных действий. Так как при работе процессы обмениваются сообщениями, возможны два варианта решения проблемы. Первый - все процессы выполняют записи контрольных точек одновременно, что затруднительно. Второй вариант, при несоблюдении синхронности, - возврат в каждом процессе к такому скоординированному набору контрольных точек, при котором невозможна противоречивая ситуация. Такая ситуация возникает, когда один процесс вернулся к контрольной точке, после которой он должен получить сообщение от другого процесса, а этот другой процесс вернулся к точке, которая следует за выдачей этого сообщения. Однако при повторе ожидаемое первым процессом сообщение не поступит. В этом случае  возможен эффект домино, в результате процессы оказываются отброшены как угодно далеко назад.

В этом состоит первая проблема, которую необходимо преодолеть.

Если нужно, чтобы последствия отказа узла не были видны клиенту,  это означает:
\begin{itemize}
\item клиент не должен терять и потом восстанавливать соединения с сервером;
\item клиент не должен повторять свои запросы;
\item клиент не должен повторно получать сообщения, которые он уже получил.
\end{itemize}

Вторая проблема, которую надо решать, связана с недетерминированностью поведения сервера приложений. Приведем пример.  Пусть имеется система продажи билетов на самолеты. Два клиента одновременно обратились к системе с запросом билета на один и тот же рейс. Клиентам безразлично, какие места им зарезервирует система. Система выполняет запросы клиентов параллельно, поэтому в какой-то момент между процессами, обрабатывающими эти запросы, может возникнуть конкуренция за ресурс - в данном случае, скажем, рейс. Один из процессов захватывает ресурс первым, резервирует место и освобождает ресурс. Потом второй процесс проделывает то же самое.

Порядок, в котором в этом примере процессы захватили ресурс, зависит от многих факторов и, в конечном счете, случаен. Однако  это не мешает правильному функционированию системы, поскольку клиентам важно одно - получить билеты, причем на разные места. Однако отсутствие детерминизма в поведении приложения приводит к тому, что при повторном выполнении могут быть получены другие результаты: например, клиенту уже сообщено, что ему зарезервировано место №5, а при повторе может получиться, что зарезервировано место №6. Система должна устранить это несоответствие и сделать его невидимым для клиента.
\begin{figure*} %fig1
\vspace*{1pt}
\begin{center}
\mbox{%
\epsfxsize=1.6in
\epsfxsize=100mm
\epsfbox{BbR-1.eps}
}
\end{center}
\vspace*{-9pt}
\Caption{Базовый вариант трубы с разными выходными устройствами
(цилиндрическое, расширяющееся и сужающееся сопло)
\label{f1bab}}
\vspace*{-3pt}
\end{figure*}

Недетерминированность поведения системы это следствие, по крайней мере, двух обстоятельств. Во-первых, это присущая системам с разделением времени неопределенность в порядке выполнения процессов. Во-вторых, это конкуренция процессов за общие ресурсы. Перечислим некоторые причины недетерминированного поведения приложений:
\begin{itemize}
\item синхронизация процессов с помощью семафоров или атомарных операций над операндами в общей памяти процессов;
\item зависимость от порядка получения клиентских запросов;
\item время, затраченное процессом на обработку полученного запроса;
\item генераторы случайных чисел;
\item системное управление процессами и потоками;
\item локальные таймеры;
\item доступ к реальному времени.
\end{itemize}

По различным  причинам время, которое тратится на выполнение вычислительной задачи с одними и теми же исходными данными, не является константой, то есть повторное выполнение может дать другое время. Процессы используют общие ресурсы, обращение к которым требует организации очередности выполнения (сериализации) - первым пришел, первым захватил. И, наконец,  результат работы процесса может зависеть от состояния ресурса, а это состояние может изменить другой процесс, ранее захвативший ресурс. Все это создает значительные трудности при попытках воспроизведения поведения процессов с сохраненной контрольной точки.

Прозрачная отказоустойчивость серверов приложений обычно осуществляется переносом приложения на другой узел кластера, идентичный первому по конфигурации аппаратных средств и операционной среды. Это делается методом, называемым snapshot/restore. На основном узле (оригинале)  периодически фиксируется состояние приложения на этом узле кластера (так называемый снимок или snapshot). После отказа оригинала на резервном узле (копии) делается восстановление (restore), то есть восстанавливается последнее зафиксированное состояние приложения. Операционная среда при этом приводится в состояние, которое соответствует моменту изготовления снимка. После этого узел-копия продолжает работу с зафиксированного места. Сравнение метода  snapshot/restore с другими подходами приведено в [7].

Ниже рассматриваются информационные  технологии, позволяющие решить ряд принципиальных вопросов, связанных с реализацией прозрачной отказоустойчивости серверов приложений. Ими являются:
\begin{itemize}
\item виртуализация операционной среды, в которой работает серверное приложение;
\item отказоустойчивая реализация протокола TCP;
\item создание контрольных точек состояния приложения и файловой системы, которые делаются внешним по отношению к приложению образом;
\item восстановление серверного приложения на основании контрольной точки.
\end{itemize}
\begin{figure*} %fig1
\vspace*{1pt}
\begin{center}
\mbox{%
\epsfxsize=1.6in
\epsfxsize=100mm
\epsfbox{BbR-1.eps}
}
\end{center}
\vspace*{-9pt}
\Caption{Базовый вариант трубы с разными выходными устройствами
(цилиндрическое, расширяющееся и сужающееся сопло)
\label{f1bab}}
\vspace*{-3pt}
\end{figure*}

\section{МОДЕЛЬ ОПИСАНИЯ ПОВЕДЕНИЯ ПРИЛОЖЕНИЯ}

Предлагаемый подход опирается на построение модели вычислений, связанной с использованием понятия времени в многопроцессных приложениях. Впервые подобные проблемы были изучены в классической работе Л. Лампорта [8].

Многопроцессными приложения называются потому, что в них параллельно работают несколько процессов. Процесс ведет себя детерминированно, пока в предписанном кодом порядке выполняет процессорные инструкции. Конечно, его работа может быть прервана практически в любой момент и процессор передан другому процессу или ядру. Поэтому абсолютное время, которое затрачивает процесс на выполнение определенной работы, не  константа, а случайная  величина. То же  относится к относительному времени, то есть времени, которое процесс занимал процессор,  поскольку одни и те же обращения к операционной среде могут вызвать работы разной длительности, а значит потребовать разное время на свое выполнение.

Кэшированность инструкций и данных, а также длина хэш-списков влияют на действительное время пребывания в операционной среде. Утрачивает смысл понятие одновременность действий, поскольку  нельзя установить, выполнили ли два разных процесса какие-либо действия одновременно или одно из них предшествовало другому. Таким образом, с процессом можно связать только его локальное время, которое линейно упорядочивает события,  происходившие в этом процессе.  Глобальное время, линейно упорядочивающее действия во всех процессах, отсутствует. Расстояние (в этом качестве используется время) между действиями оказывается случайной величиной.

Эти соображения важны, поскольку процессы в интересующих нас приложениях взаимодействуют и используют общие ресурсы. Для взаимодействия они используют средства синхронизации, предоставляемые операционной средой - например, наборы семафоров SVR4 (System V Release 4), POSIX-семафоры, бинарные семафоры и другие примитивы взаимного исключения (POSIX- mutual exclusion locks) и т.д. Подобные средства операционной среды, которые позволяют процессам синхронизировать свою деятельность друг с другом или сериализовать обращения к совместно используемым объектам,  будут ниже  называться ресурсами.

С каждым ресурсом связано свое локальное время, линейно упорядочивающее события в жизни ресурса. Например, в случае двоичных семафоров это создание семафора, а также его захват и освобождение процессом. Заметим, что событие - это не намерение процесса (например, захватить бинарный семафор), а сам факт захвата семафора процессом (т.е. успешное выполнение намерения). От изъявления намерения до его осуществления может многое произойти. Например, семафор, который хочет захватить рассматриваемый процесс, принадлежал другому процессу, потом тот процесс его освободил, но семафор был сначала передан операционной средой третьему процессу, который также на него претендовал, и т.д. Поведение рассматриваемого процесса в это время нас не интересует - он ресурсом еще не овладел, а только его захват определяет его дальнейшее поведение. По причинам,  изложенным выше, расстояние между двумя событиями - случайная величина. Однако, события замечательны тем, что они одновременно присутствуют и в локальном времени процесса, и в локальном времени ресурса. Поэтому все, что произошло в истории процесса или/и ресурса до этого события, предшествует ему. Далее  будет считаться, что истории и ресурсов и процессов состоят только из событий, причем между двумя последовательными событиями в жизни процесса последний ведет себя детерминированно.

Это означает, что на  поведении процесса сказывается только его предыдущая история, то есть состояние ресурсов, с которыми он взаимодействовал. Это свойство процессов ниже будет называться локальной детерминированностью. Этим свойством не обладают ресурсы, поскольку - следующее событие в истории ресурса не определяется однозначно по его предыдущей истории. Утверждение, касающееся детерминированного поведения процессов, неявно опирается на предположение,  что учтены все ресурсы, которые могут привести к  недетерминированности процессов.

Таким образом, описанное нами очень неформально время в многопроцессном комплексе представляет собой отношение частичного порядка, введенное на множестве событий. Зная полное состояние комплекса в некоторый момент времени,  нельзя однозначно определить, какое событие в истории ресурса наступит следующим. Можно говорить только о вероятности наступления того или иного события. Недетерминированность поведения есть следствие двух обстоятельств. Во-первых, это неопределенность времени, которое тратит процесс на переход от одного события к другому. Во-вторых, конкуренция процессов за общие ресурсы.

Выполнение приложения, на множестве событий которого введена частичная упорядоченность, можно описать направленным ациклическим графом выполнения. Вершинами этого графа являются события, с каждым  из которых связаны две входящие в него дуги. Одна дуга начинается в событии, которое непосредственно предшествует данному событию в истории процесса, другая - в истории ресурса.

Построение средств обеспечения прозрачной отказоустойчивости приложений опирается на следующее утверждение: для восстановления работы приложения после отказа достаточно располагать:
\begin{itemize}
\item контрольной точкой, которая отражает на некоторый момент времени состояния процессов и других ресурсов, образующих приложение;
\item графом выполнения приложения, который описывает работу приложения, начинающуюся с контрольной точки и заканчивающуюся отказом. Данные, которые нужны для построения графа выполнения, далее называются протоколом.
\end{itemize}
\begin{figure*} %fig1
\vspace*{1pt}
\begin{center}
\mbox{%
\epsfxsize=1.6in
\epsfxsize=100mm
\epsfbox{BbR-1.eps}
}
\end{center}
\vspace*{-9pt}
\Caption{Базовый вариант трубы с разными выходными устройствами
(цилиндрическое, расширяющееся и сужающееся сопло)
\label{f1bab}}
\vspace*{-3pt}
\end{figure*}

Вся эта информация должна находиться в стабильной памяти, не разрушающейся при отказе.

Ниже неформально описан алгоритм восстановления работы приложения после отказа, который опирается на наличие контрольной точки и графа выполнения. Будем считать, что в распоряжении имеются средства, позволяющие остановить процесс в тот момент, когда он намерен совершить некоторую операцию над ресурсом. Заметим, что событие в графе выполнения соответствует не изъявлению намерения, а его удовлетворению, то есть завершению выполнения операции.

Предварительно сделаем следующее:
\begin{itemize}
\item используя контрольную точку, приведем приложение в состояние, соответствующее этой контрольной точке;
\item в графе выполнения пометим все вершины (события) как "не наступившие". У некоторых вершин графа отсутствуют им непосредственно предшествующие; соответствующие события наступили сразу же после создания контрольной точки. Для каждой такой вершины включим в граф дополнительную вершину, ей предшествующую в истории процесса, и отметим эту дополнительную вершину как "наступившую";
\item разрешим процессам приложения выполняться.
\end{itemize}

Пусть некоторый процесс проявляет намерение выполнить операцию над каким-либо ресурсом. Отыщем для этого процесса в его истории последнее наступившее событие. Следующее в его истории событие - это то, которое соответствует требуемой операции. Посмотрим, наступило ли событие в истории ресурса, которое ему предшествует. Если нет, переведем процесс в состояния ожидания, отметив в предшествующем событии, что данный процесс ожидает его наступления. Если да, разрешим процессу выполняться, то есть выполнить операцию над ресурсом.

Пусть некоторый процесс объявляет, что он выполнил операцию над каким-либо ресурсом (это соответствует моменту протоколирования при оригинальном выполнении). Отыщем для этого процесса в его истории последнее наступившее событие и перейдем к следующему событию в его истории. Это опять то событие, которое мы рассматриваем. Отметим его как "наступившее". Если наступления этого события ожидал какой-нибудь процесс, выведем этот процесс из состояния ожидания. Наконец, разрешим процессу, выполнившему операцию, продолжаться дальше.

Когда выясняется, что наступили все события графа выполнения, повторное выполнение считается законченным.

Естественным следствием из сказанного является следующее утверждение: для того, чтобы размер протокола не рос неограниченно, нужно периодически создавать контрольные точки, очищая при этом протокол.

\section{ФОРМАЛЬНОЕ ОПИСАНИЕ МОДЕЛИ ПОВЕДЕНИЯ МНОГОПРОЦЕССНОГО ПРИЛОЖЕНИЯ}
\begin{figure*} %fig1
\vspace*{1pt}
\begin{center}
\mbox{%
\epsfxsize=1.6in
\epsfxsize=100mm
\epsfbox{BbR-1.eps}
}
\end{center}
\vspace*{-9pt}
\Caption{Базовый вариант трубы с разными выходными устройствами
(цилиндрическое, расширяющееся и сужающееся сопло)
\label{f1bab}}
\vspace*{-3pt}
\end{figure*}

Опишем формально поведение приложения, неформальное описание которого было приведено выше. Рассматриваются два типа объектов:
\begin{itemize}
\item ресурсы (r), например, наборы семафоров (POSIX- или SVR4-семафоры), бинарные семафоры (POSIX-mutex's), таймер реального времени, сокеты (sockets), то есть двусторонние виртуальные соединения с внешним миром;
\item процессы (p), например, процессы или потоки (threads) пользователя.
\end{itemize}

\end{multicols}

\label{end\stat}

%\def\stat{batr}

\def\tit{НОВЫЙ МЕТОД ВЕРОЯТНОСТНО-СТАТИСТИЧЕСКОГО\newline
АНАЛИЗА ИНФОРМАЦИОННЫХ ПОТОКОВ
В~ТЕЛЕКОММУНИКАЦИОННЫХ СЕТЯХ$^*$}
\def\titkol{Новый метод вероятностно-статистического
анализа информационных потоков
в~телекоммуникационных сетях}
\def\autkol{Д.\,А.~Батракова, В.\,Ю.~Королев, С.\,Я.~Шоргин}
\def\aut{Д.\,А.~Батракова$^1$, В.\,Ю.~Королев$^2$, С.\,Я.~Шоргин$^3$}

\titel{\tit}{\aut}{\autkol}{\titkol}

{\renewcommand{\thefootnote}{\fnsymbol{footnote}}\footnotetext[1]{Работа 
выполнена при поддержке РФФИ, проекты №№\,04-01-00671, 05-07-90103.} 
\renewcommand{\thefootnote}{\arabic{footnote}}}
 \footnotetext[1]{ИПИ РАН, 
daria.batrakova@gmail.com} \footnotetext[2]{Факультет вычислительной математики 
и кибернетики МГУ им.~М.\,В.~Ломоносова, ИПИ РАН, vkorolev@comtv.ru} 
\footnotetext[3]{ИПИ РАН, sshorgin@ipiran.ru}



\Abst{В данной работе предлагается метод исследования стохастической структуры
хаотических информационных потоков в сложных телекоммуникационных
сетях. Предлагаемый метод основан на стохастической модели
телекоммуникационной сети, в рамках которой она представляется в виде
суперпозиции некоторых простых последовательно-параллельных структур.
Эта модель естественно порождает смеси гамма-распределений для времени
выполнения (обработки) запроса сетью. Параметры получаемой смеси
гамма-распределений характеризуют стохастическую структуру
информационных потоков в сети. Для решения задачи статистического
оценивания параметров смесей экспоненциальных и гамма-распределений
(задачи разделения смесей) используется ЕМ-алгоритм. Чтобы проследить
изменение стохастической структуры информационных потоков во времени,
ЕМ-алгоритм применяется в режиме скользящего окна. Описывается
программный инструментарий для применения полученного решения к
реальным статистическим данным. Приводится интерпретация результатов.}

\KW{телекоммуникационные сети; информационные потоки;
разделение смесей  распределений;
метод скользящего окна;  программа для разделения смесей}

\vskip 24pt plus 9pt minus 6pt

\thispagestyle{headings}

\begin{multicols}{2}


\label{st\stat}

\section{Введение}

Развитие телекоммуникационных сетей, их усложнение поставило перед
инженерами важную прикладную задачу исследования характеристик
информационных потоков, возникающих в этих сетях. Здесь под
информационным потоком мы будем понимать упорядоченное движение
любого вида информации по сети.

Если на заре эры телекоммуникаций, в эпоху первых телефонных линий и
телеграфа эта проблема не была столь насущной, то со временем, при
постепенном охвате мирового пространства сетями возникла необходимость в
построении и исследовании адекватных моделей сетей и процессов,
происходящих в них.

\thispagestyle{headings}


Современные сети~--- это \textit{конвергентные} сети, т.е.\ совокупность крайне
разнородных как по топологии, так и по физической архитектуре сетей, которые
предлагают конечному пользователю самые разнообразные сервисы. Это~--- огромное
виртуальное и физическое пространство, состоящее из миллионов процессоров,
операционных платформ, линий передачи данных и стыковочных узлов.
%
Существует множество классификаций телекоммуникационных сетей по различным
признакам:
\begin{itemize}
\item масштабу (локальные сети~--- LAN, масштаба города~---
MAN, широкого масштаба~--- WAN);
\item топологии, или логической организации (<<звезда>>,
<<кольцо>>, <<шина>>);
\item физической организации (оптические сети, радио);
\item предлагаемым услугам (сотовые сети, для доступа в
Интернет);
\item назначению (военные, гражданские) и~др.
\end{itemize}


Конвергентная сеть входит во все эти классы, причем, как правило,
обладает всеми этими признаками. Поэтому построение модели для ее анализа
является и очень важной, и очень сложной задачей.

Существуют достаточно многочисленные математические методы, ориентированные на
моделирование и анализ телекоммуникационных сетей. В~большинстве своем они
основываются на теории массового обслуживания, то есть разделе теории
вероятностей, посвященном описанию функционирования сложных систем обслуживания
(в том чис\-ле телекоммуникационных сетей и систем) с помощью стохастических
процессов особого вида и анализу таких процессов. Указанная теория является
весьма развитой и широко применяется на практике. Тем не менее, ее применимость
ограничена~--- во-первых, все возрастающей сложностью структур и дисциплин
обслуживания в рас\-смат\-ри\-ва\-емых реальных сетях. Эта сложность во многих
случаях принципиально не может найти адекватного отображения в моделях
массового обслуживания, даже несмотря на постоянно растущую сложность самих
этих моделей. В результате даже модели, допускающие точный математический
анализ, дают возможность расчета всего лишь приближенных значений характеристик
реальных сетей, ибо предположения, принимаемые при построении моделей, во
многих случаях не соответствуют практике. Во-вторых, для описания
телекоммуникационной сети в виде сети массового обслуживания исследователь
должен располагать детальным описанием структуры сети, что далеко не всегда
имеет мес\-то на практике. В-третьих, разработано крайне мало моделей массового
обслуживания, в которых используется в качестве входной информация о
наблюдаемых (статистических) показателях функционирования сети; в то же время,
такая информация очень часто доступна исследователю, и ее использование при
анализе сети весьма целесообразно.

В данной работе предлагается в определенной степени альтернативный подход к
моделированию сложных телекоммуникационных сетей. Строится и исследуется
вероятностная модель сложной телекоммуникационной сети как суперпозиции
достаточно простых структур. При этом практически никакая априорная информация
о структуре исследуемой сети не используется~--- наоборот, в результате
исследования модели исследователь получает приближенное представление об этой
структуре. Характеристики типовых простых структур, составляющих в совокупности
модель сети, и сети в целом при этом подходе описываются
гам\-ма-рас\-пре\-де\-ле\-ни\-я\-ми. Ставится задача оценки параметров модели
на основе статистических данных о функционировании сети, а также предлагается
математическое решение этой задачи. В статье описан также созданный на основе
разработанных математических методов программный инструментарий и приведены
результаты расчетов для реального трафика. {\looseness=-1

}

\section{Математическая модель и~постановка задачи}

\subsection{Логическая модель сети}
 %1.1

Рассмотрим абстрактную \textit{конвергентную телекоммуникационную
сеть}. Это может быть как крупномасштабная транспортная сеть (WAN), сеть
отдельного оператора масштаба города (MAN) с различными сервисами, так и
локальная сеть (LAN).

Любой из этих случаев можно описать как ($p,\,q$)-\textit{сеть}.

\medskip
\textbf{Определение 1.} В теории графов и сетей под ($p,\,q)$-сетью понимается
набор вида $S =$\linebreak $=(G,\,V^\prime ,\,V^{\prime\prime})$, где $G$~---
граф, а $V^\prime$ и $V^{\prime\prime}$~--- выборки из множества $V(G)$ (вершин
графа) длины~$p$ и $q$ соответственно. При этом выборка $V^\prime$
($V^{\prime\prime}$) считается \textit{входной} (\textit{выходной}) выборкой, а
ее $i$-я вершина называется $i$-\textit{м} \textit{входным} (\textit{выходным})
\textit{полюсом} или, иначе, $i$-\textit{м} \textit{входом} (\textit{выходом})
сети~$S$. Вершины, не участвующие во входной и выходной выборках сети,
считаются ее внутренними вершинами~\cite{1bat}.

Сеть $S$ (рис.~\ref{f1bat}) имеет $p$ точек входа~--- точек соединения
с внешней средой (это могут быть точки стыковки разнородных сетей, сетей
различных операторов, физические подключения к интерфейсам
маршрутизаторов и~т.п.). Под \textit{внешней средой} будем понимать другие
сети, которые передают данные в сеть~$S$. Отдельные <<единицы>> данных
(кадры, сообщения, датаграммы, пакеты) поступают на входы сети~$S$,
обрабатываются и подаются на каждый из $q$ выходов, которые могут быть
соединены как с конечными пользователями, так и с другими сетями.
\begin{figure*} %fig1
\vspace*{1pt}
\begin{center}
\mbox{%
\epsfxsize=139.7mm \epsfbox{bat-1.eps}
%\epsfxsize=139.698mm
%\epsfbox{bek-3.eps}
}
\end{center}
\vspace*{-9pt} \Caption{Абстрактная телекоммуникационная сеть \label{f1bat}}
\end{figure*}

Следует отметить, что структура сложных телекоммуникационных сетей обладает
свойством некоторого самоподобия, т.е.\ на каком бы уровне сетевой архитектуры
мы ни рассматривали поведение информационных потоков, мы можем выделить
некоторые базовые структуры, субпотоки, суперпозицией которых мы можем получить
модель конкретной сети, какой бы уровень <<детализации>> сегментов сети мы ни
взяли. Так, например, физические подключения к интерфейсам оптического
кросс-коннекта в этом смысле подобны <<виртуальным>> подключениям к портам TCP
на сервере приложений.

Итак, независимо от уровня сетевой архитектуры мы можем
рассматривать некоторую величину, характеризующую количество каких-либо
ресурсов сети~$S$, занимаемых в процессе передачи и обработки данных.  Это
могут быть ресурсы, относящиеся как к <<объему>> (памяти сетевого
устройства, количеству занятых линий, размеру пакета), так и ко <<времени>>
(времени обслуживания заявки, времени простоя). Эта величина случайна, т.к.\
мы не можем абсолютно точно сказать для сложной телекоммуникационной
сети, какое сообщение на какой из входов поступит и какого размера оно будет.
Таким образом, случайный характер данной величины определяется
случайностью поведения внешней среды.

Пусть $R$~--- это описанная выше случайная величина, $R>0$. Далее, не
ограничивая общности, будем подразумевать под ней время, необходимое для
какой-либо операции сети (обработки <<единицы>> данных), предполагая, что
время обработки прямо зависит от объема сообщения.

\subsection{Вероятностная модель сети} %1.2.

Даже не зная реальной топологии сети, мы можем описать
функционирование некоторых ее участков как процесс выполнения операций
(задач сети) в последовательном  порядке (например, если доступен только
один канал данных) или как процесс одновременного выполнения субопераций
(когда доступно более одного пути выполнения). Это значит, что мы можем
представить функционирование сложной телекоммуникационной сети как
\textit{суперпозицию} таких <<последовательных>> и <<параллельных>>
блоков.

Для построения вероятностной модели распределения~$R$ используется
комбинация асимптотического подхода, основанного на предельных теоремах
теории вероятностей, и принципа максимальной неопределенности (энтропии).

Рассмотрим следующую модель. Предположим, что мы можем разделить
сеть~$S$ на несколько сегментов $S_i$. Пусть $T$~--- случайная величина,
время выполнения операции в отдельно взятом блоке $S_i$ (сегменте сети).

Если операции выполняются \textit{параллельно}, то время, необходимое
для их выполнения~--- это максимальное время, затрачиваемое на какую-либо
субоперацию:
$$
T = \underset{i}{\max}\, T_i\,,
$$
где $T_i$~--- случайные величины для со\-от\-вет\-ст\-ву\-ющих субопераций.
Модель такого выполнения пред\-став\-ле\-на на рис.~\ref{f2bat}.

\begin{figure*} %fig2
\vspace*{1pt}
\begin{center}
\mbox{%
\epsfxsize=117.271mm
\epsfbox{bat-2.eps}
}
\end{center}
\vspace*{-9pt}
\Caption{Параллельное выполнение
\label{f2bat}}
\end{figure*}

Известно, что предельное распределение экстремальных значений для
выборок ~--- это экспоненциальное распределение с плотностью~\cite{2bat}
$$
f(x) =
\begin{cases}
\lambda e^{-\lambda x}\,, & x>0\,,\\
0\,, & x\leq 0\,,
\end{cases}
$$
где $\lambda >0$~--- параметр масштаба.

 Учитывая также энтропийный подход, естественно будет считать
распределение $T$ экспоненциальным, т.к.\ экспоненциальное распределение
обладает наибольшей энтропией среди всех распределений с $x>0$.

Если же операции сети выполняются \textit{последовательно}, то величина
$T$~--- это сумма времен $T_i$, необходимых для выполнения каждой
субоперации:
$$
T = \sum\limits_i T_i\,,
$$
где $T_i$~--- случайные величины для со\-от\-вет\-ст\-ву\-ющих субопераций.
%
Такая модель представлена на рис.~\ref{f3bat}.

\begin{figure*} %fig3
\vspace*{1pt}
\begin{center}
\mbox{%
\epsfxsize=139.592mm
\epsfbox{bat-3.eps}
}
\end{center}
\vspace*{-9pt}
\Caption{Последовательное  выполнение
\label{f3bat}}
\end{figure*}

Это значит, что распределение общей длительности $T$ выполнения
блока~--- это свертка распределений <<элементарных>> времен $T_i$
(экспоненциально распределенных).

Известно, что результатом свертки экспоненциальных распределений
является гамма-распределение, определяемое плотностью
$$
\g_{\lambda , \alpha} (x) =
\begin{cases}
\fr{\lambda_0^{\alpha_0}}{\Gamma (\alpha_0 )}\,x^{\alpha_0-1}
e^{\lambda_0 x}\,, & x>0\,,\\
0,\, & x\leq 0\,,
\end{cases}
$$
где $\alpha >0$~--- параметр формы,  $\lambda >0$  параметр масштаба, а
$\Gamma (z)$~--- гамма-функция Эйлера:
$$
\Gamma (z) = \int\limits_0^\infty x^{z-1} e^{-x}\,dx\,.
$$

\begin{figure*} %fig4
\vspace*{1pt}
\begin{center}
\mbox{%
\epsfxsize=120.831mm
\epsfbox{bat-4.eps}
}
\end{center}
\vspace*{-9pt}
\Caption{Модель пути  обработки сообщения сетью~$S$
\label{f4bat}}
\end{figure*}

Известно~\cite{2bat}, что класс гамма-распределений замкнут над операцией
свертки, поэтому ре\-зуль\-ти\-ру\-ющее распределение случайной величины~$R$
будет также гамма-распределением
$$
\g_{\lambda , \alpha} (x) =
\begin{cases}
\fr{\lambda^{\alpha}}{\Gamma (\alpha )}\,x^{\alpha -1} e^{-\lambda x}\,, &
x>0\,,\\
0,\, & x\leq 0\,.
\end{cases}
$$

В силу случайного характера ввода данных в сеть~$S$ из внешней среды маршрут
передачи данных становится случайным, что представлено на рис.~\ref{f4bat}. Это
означает, что параметры ре\-зуль\-ти\-ру\-юще\-го распределения~$R$ тоже
случайны. Отсюда имеем следующую модель \textit{смеси
гам\-ма-рас\-пре\-де\-ле\-ний}, определяемой плотностью

\begin{equation} %1
p(x) = \iint \g_{\lambda , \alpha}(x)\,dH (\lambda ,\,\alpha )\,,
\end{equation}
где $H(\lambda , \alpha )$~--- смешивающая функция, функция распределения
входных параметров.

Поясним понятие \textit{смеси распределений}.

\medskip
\textbf{Определение~2.} Пусть имеется двух\-па\-ра\-мет\-ри\-че\-ское
семейство $n$-мерных плотностей  распределения
\begin{equation}
F = \{ f_\omega (x;\, \theta (\omega ))\}\,,
\end{equation}
где одномерный (целочисленный или непрерывный) параметр $\omega$ в
качестве нижнего индекса функции $f$ определяет специфику общего вида
каж\-до\-го компонента~--- распределения смеси, а в качестве аргумента при
многомерном, вообще говоря, параметре $\theta$ определяет зависимость
значений хотя бы части компонентов этого параметра от того, в каком именно
составляющем распределении $f_\omega$ он присутствует. Кроме того, пусть
$P = \{P(\omega )\}$~--- \textit{семейство смешивающих функций}
распределения.

Функция плотности распределения
\begin{equation}
f(x) = \int f_\omega (x;\,\theta(\omega ))\,dP (\omega )
\end{equation}
называется $P$-\textit{смесью} (или просто \textit{смесью})
\textit{распределений} семейства~$F$,  интеграл в~(3) понимается в смысле
Лебега--Стильтьеса~\cite{3bat}.

\medskip
\textbf{Определение 3.} Семейство смесей~(3) называется
\textit{идентифицируемым} (\textit{различимым}), если из равенства
$$
\int f_\omega (x;\,\theta(\omega ))\,dP (\omega ) =\int f_\omega
(x,\,\theta(\omega )) dP^*(\omega )
$$
следует, что $P(\omega ) \equiv P^*(\omega )$ для всех $P \in P(\omega
)$~\cite{3bat}.

\subsection{Постановка задачи} %1.3.

Перед нами встает задача \textit{разделения} такой смеси. Вообще говоря,
задача разделения смесей распределений со смешивающими функциями
общего вида является \textit{некорректно поставленной}, т.к.\ она допускает
существование нескольких решений. Поэтому будем искать решение в классе
\textit{конечных идентифицируемых смесей распределений}, где смешивающая
функция дискретна.

Для этого сузим данное выше определение и будем рассматривать в дальнейшем лишь 
случай конечного числа $k$ возможных значений па\-ра\-мет\-ра~$\omega$, что 
соответствует конечному числу скачков смешивающих функций $P(\omega )$.  
Величины этих скачков как раз и будут играть роль \textit{удельных весов} 
(\textit{априорных вероятностей}) $p_j$ компонентов смеси. Более того, в нашем 
случае мы постулируем также однотипность компонентов распределений $f_j$, т.е.\ 
принадлежность всех $f_j$ к одному общему па\-ра\-мет\-ри\-че\-ско\-му 
семейству $\{ f(X;\,\theta )\}$, где $\theta$~--- многомерный, вообще говоря, 
параметр. Так что~(3) в этом случае может быть записано в виде
\begin{equation} %4
p(x) = \sum\limits^k_{j=1} p_j f_j (x;\,\theta_j )\,.
\end{equation}

Переформулируем понятие идентифицируемости (различимости) смесей
специально применительно к такому виду смесей.

\medskip
\textbf{Определение 4.} \textit{Конечная смесь}~(3) называется
\textit{идентифицируемой} (\textit{различимой}), если из равенства
$$
\sum\limits_{j=1}^k p_j f_j (x;\,\theta_j ) = \sum\limits_{l=1}^{k^*} p_l^* f_l
(x;\,\theta_l^* )
$$
следует, что $k=k^*$ и для любого $j$ ($1\leq j \leq k$) найдется такое $l$ 
($1\leq l \leq k^*$), что $p_j = p_l^*$ и $f_j (x;\,\theta_j ) = f_l 
(x;\,\theta_l^* )$~\cite{3bat}.

Решить эту задачу в выборочном варианте~--- значит по выборке
классифицируемых наблюдений
$X_1,\ldots , X_n, $ извлеченной из генеральной совокупности, яв\-ля\-ющей\-ся смесью~(3)
генеральных совокупностей типа~(2) (при заданном общем виде составляющих
смесь функций $f_j (x;\,\theta_j )$), построить статистические оценки для числа
компонентов смеси~$k$, их удельных весов $p_j$ и, главное, для каждого из
компонентов %f_j (x;\,\theta_j )$ анализируемой смеси. Далее будет считать, что
функции $f_j$ однозначно определяются своими параметрами $\theta_j$: $f_j
=f(x;\,\theta_j)$.

Однако не следует ставить знак тождества между задачей разделения смеси
и задачей статистического оценивания параметров в модели~(4) по выборке $
X_1,\ldots , X_n$, поскольку задача разделения сохраняет смысл и
применительно к генеральным совокупностям, т.е.\ в теоретическом
варианте~\cite{3bat}.

Итак, для статистического анализа на основе реальных данных мы
аппроксимируем нашу общую модель~(1) следующей:
$$
p(x) \approx \hat{p}(x) = \sum\limits_{j=1}^k p_j \g_{\lambda_j , \alpha_j}
(x)\,,
$$
где $p_j$~--- дискретные смешивающие параметры, $\g_{\lambda_j , \alpha_j}
(x)$~--- плотности гамма-распределений.

Такая аппроксимация не только позволяет решить поставленную статистическую
задачу, но и полу\-чить наглядную визуализацию результатов. Существуют
достаточно эффективные методики разделения смесей распределений, среди них~---
семейство так называемых \textit{ЕМ-алгоритмов}
(\textit{Expectation-Maximization Algorithms}).

Полученные результаты могут быть достаточно просто интерпретированы. Число
компонентов смеси символизирует число типичных параллельных или
последовательных структур. Значения параметров составляющих смесь
гам\-ма-рас\-пре\-де\-ле\-ний показывают <<степень параллелизма>>
соответствующей структуры: чем ближе параметр формы к~1, тем выше эта
<<степень>>. И наоборот, чем дальше значение параметра формы от~1, тем больше
последовательных операций выполняется в соответствующем блоке.

Веса компонентов характеризуют примерную долю использования
ресурсов для сообщений, соответствующих каждому распределению входных
данных.

Итак, предложенный подход позволяет получить представление о
стохастической структуре телекоммуникационной сети.

\section{ЕМ-алгоритм разделения смесей распределений}

\subsection{Описание алгоритма} %2.1.

Определяемый ниже итерационный алгоритм решения поставленной в
предыдущем разделе задачи относится к процедурам, базирующимся на
\textit{методе максимального правдоподобия}.

Этот алгоритм позволяет находить максимум логарифмической функции
правдоподобия по параметрам $p_1,\,p_2,\ldots ,\,p_k$, $\theta_1 ,\,\theta_2,\ldots ,\,
\theta_k$ при фиксированном $k$ по выборке $X_1, \ldots , X_n$, т.е.\ решение
оптимизационной задачи вида

\begin{equation} \sum\limits_{i=1}^n \ln \left ( \sum\limits_{j=1}^k p_j f_j
(X_i;\,\theta_j )\right ) \rightarrow \underset{p_j,\,\theta_j}{\max}\,.
\end{equation}

Конкретные алгоритмы, построенные по этой схеме, часто называют
\textit{алгоритмами типа ЕМ}, поскольку в каждом из них можно выделить два
этапа, находящихся по отношению друг к другу в последовательности
итерационного взаимодействия: \textit{оценивание} (\textit{Estimation}) и
\textit{максимизация} (\textit{Maximization})~\cite{4bat}.

Введем в рассмотрение так называемые апостериорные вероятности
$\g_{ij}$ принадлежности наблюдения $X_i$ к $j$-му классу:
\begin{equation} %6
\g_{ij} = \fr{p_j f(X_i;\,\theta_j )}{\sum\limits_{l=1}^k p_l f(X_i;\,\theta_l 
)} \ (i=1,\ldots , n;\ j=1,\ldots ,k)\,.\!\!\end{equation} 
Очевидно, что для 
всех $i=1,\ldots ,n$ и $j=1,\ldots ,k$
$$
\g_{ij} \geq 0,\quad \sum_{j=1}^k \g_{ij} =1\,.
$$


Далее обозначим $\Theta = (p_1,\ldots p_k,\,\theta_1,\ldots ,\theta_k )$ и
представим анализируемую логарифмическую функцию правдоподобия
$$
\ln L(\Theta ) = \sum\limits_{i=1}^n \ln \left (\sum\limits_{j=1}^k p_j f_j
(X_i;\,\theta_j )\right )
$$
в виде
\begin{multline}
\ln L (\Theta ) = \sum\limits_{j=1}^k\sum\limits_{i=1}^n \g_{ij} \ln p_j+{}\\
{}+\sum\limits_{j=1}^k\sum\limits_{i=1}^n \g_{ij} f(X_i;\,\theta_j)-
\sum\limits_{j=1}^k\sum\limits_{i=1}^n \g_{ij} \ln \g_{ij}\,.
\end{multline}

Справедливость этого тождества легко проверяется с учетом
$$
\sum\limits_{j=1}^k \g_{ij} =1\,.
$$

Далее идея построения итерационного алгоритма вычисления оценок
$\hat{\Theta} = (\hat{p}_1,\ldots , \hat{p}_k,\
\hat{\theta}_1,\ldots , \hat{\theta}_k)$
для параметров $\Theta = (p_1,\ldots , p_k,\ \theta_1,\ldots , \theta_k)$ состоит в
следующем:
\begin{enumerate}[1.]
\item Выбирается некоторое \textit{начальное приближение}~$\hat{\Theta}^0$.
\item \textbf{E-step:} вычисляются по формулам~(6) начальные приближения
$\g_{ij}^0$ для апостериорных вероятностей $\g_{ij}$~--- \textit{этап
оценивания}.
\item \textbf{M-step:} затем, возвращаясь к~(7), при вычисленных значениях
$\g^0_{ij}$ следует определить значения $\hat{\Theta}^1$ из условия
максимизации отдельно каждого из первых двух слагаемых правой
части~(7), поскольку первое слагаемое
$$
\sum_{j=1}^k \sum_{i=1}^n \g_{ij} \ln p_j
$$
зависит только от параметров $p_j$, а второе слагаемое
$$
\sum_{j=1}^k \sum_{i=1}^n \g_{ij} f(X_i;\,\theta_j )
$$
зависит только от параметров $\theta_j$~--- \textit{этап максимизации}.
\item Проверяется \textit{условие останова}:
$$
\parallel \Theta^{(t)} - \Theta^{t-1}\parallel <\varepsilon\,,
$$
где $t$~--- номер итерации, а
$\parallel\bullet\parallel$~--- евклидова норма, для некоторого $\varepsilon
>0$.
\end{enumerate}

Очевидно, решение оптимизационной задачи
$$
\sum\limits_{j=1}^k\sum\limits_{i=1}^n \g_{ij}^{(t)}\ln p_j \rightarrow
\underset{p_j}{\max}
$$
дается выражением (с учетом $\sum_{j=1}^k p_j =1$):
$$
p_{ij}^{(t+1)} =\fr{1}{n}\,\sum\limits_{i=1}^n \g_{ij}^{(t)}\,,
$$
где $t$~--- номер итерации, $t = 0$, 1, 2,\,\ldots

Решение оптимизационной задачи
$$
\sum\limits_{j=1}^k \sum\limits_{i=1}^n \g_{ij}^{(t)} f(X_i;\,\theta_j )
\rightarrow \underset{\theta_j}{\max}
$$
получить намного проще решения задачи~(5): выражение для $\theta_j$
записывается с учетом знания конкретного вида функций
$f(X,\,\theta)$~\cite{3bat}.

\subsection{О сходимости алгоритма} %2.2.

В работе М.\,И.~Шлезингера~\cite{5bat}, где эта схема (позднее названная
ЕМ-схемой) впервые предложена, установлены и основные свойства
реа\-ли\-зу\-ющих ее алгоритмов. В частности, было доказано, что при достаточно
широких предположениях \textit{предельные точки} всякой последовательности,
порожденной итерациями ЕМ-алгоритма, являются стационарными точками
оптимизируемой логарифмической функции правдоподобия $\ln L(\Theta )$ и что
найдется неподвижная точка алгоритма, к которой будет сходиться каждая из таких
последовательностей. Если дополнительно потребовать положительной
определенности информационной мат\-ри\-цы Фишера для $\ln L(\Theta )$ при
истинных зна\-че\-ни\-ях па\-ра\-мет\-ра $\Theta$, то можно показать, что
асимптотически по $n\rightarrow\infty$ (т.е.\ при больших выборках) существует
единственное сходящееся (по веро\-ят\-но\-сти) решение $\hat{\Theta}(n)$
уравнений метода максимального правдоподобия и, кроме того, существует в
пространстве параметров $\Theta$ норма, в которой последовательность
$\Theta^{(t)}(n)$, порожденная ЕМ-ал\-го\-рит\-мом, сходится к $\hat{\Theta}
(n)$, если только начальная аппроксимация $\hat{\Theta}^0$ не была слишком
далека от~$\hat{\Theta} (n)$. {%\looseness=1

}

Таким образом, результаты исследования свойств ЕМ-алгоритмов метода
максимального правдоподобия разделения смеси и их практическое
использование показали, что они являются достаточно работоспособными (при
известном чис\-ле компонентов смеси) даже при большом чис\-ле $k$ компонентов и
при высоких размерностях анализируемого признака~$X$~\cite{3bat}.

\subsection{Уравнения для смеси экспоненциальных распределений}
%2.3.

Применим описанный выше алгоритм к разделению смеси
экспоненциальных распределений:
$$
p(x) = \sum\limits_{j=1}^k p_j \lambda_j e^{-\lambda_j x}\,.
$$
Получаем следующие итерационные уравнения:
\begin{align*}
\g_{ij}^{(t+1)} & = \fr{p_j^{(t)}\lambda_j^{(t)}e^{-
\lambda_j^{(t)}X_i}}{\sum\limits_{l=1}^k p_l^{(t)}\lambda_l^{(t)}
e^{-\lambda_l^{(t)}X_i}}\,,\\
p_j^{(t+1)} & = \fr{1}{n}\,\sum\limits_{i=1}^n \g_{ij}^{(t)}\,.
\end{align*}

Чтобы найти  оценки $\lambda_j$, подсчитаем первую производную функции
$$\sum_{j=1}^k\sum_{i=1}^n \g_{ij}^{(t)} \ln (\lambda_j e^{-\lambda_j X_i}):$$
\vspace*{-8pt}
\begin{multline*}
\left ( \sum\limits_{j=1}^k \sum\limits_{i=1}^n
\g_{ij}^{(t)}\ln \left ( \lambda_j
e^{-\lambda_j X_i} \right ) \right )^\prime \lambda_j =\\[-3pt]
{}= \left (
\sum\limits_{j=1}^k\sum\limits_{i=1}^n \g_{ij}^{(t)}\ln (\lambda_j -\lambda_j X_i )
\right )^\prime \lambda_j =\\[-3pt]
{}= \sum\limits_{i=1}^n \g_{ij}^{(t)}\left (
\fr{1}{\lambda_j} - X_i \right )\,.
\end{multline*}

Разрешая уравнение
$$
\sum\limits_{i=1}^n \g_{ij}^{(t)}\left ( \fr{1}{\lambda_j} -X_i\right ) =0
$$
относительно $\lambda_j$, получаем следующее итерационное уравнение:
$$
\lambda_j^{(t+1)} = \fr{\sum\limits_{i=1}^n
\g_{ij}^{(t)}}{\sum\limits_{i=1}^n \g_{ij}^{(t)} X_i}\,.
$$

\subsection{Уравнения для смеси гамма-распределений } %2.4.

Применим теперь ЕМ-алгоритм к смеси гам\-ма-рас\-пре\-де\-ле\-ний вида
$$
p(x) = \sum\limits_{j=1}^k p_j \fr{\alpha_j^{\alpha_j} x^{\alpha_j -
1}}{\lambda_j^{\alpha_j} \Gamma (\alpha_j )}\,e^{-(\alpha_j / \lambda_j)x}\,.
$$

Такая параметризация удобна для нахождения
оценок~$\alpha_j$~\cite{6bat}.

Аналогичным способом выписываются итерационные уравнения:
\begin{align*}
\g_{ij}^{(t+1)} & =   \fr{p_j^{(t)}\fr{(\alpha_j^{\alpha_j} )^{(t)}
x^{\alpha_j - 1}}{(\lambda_j^{\alpha_j} )^{(t)}\Gamma (\alpha_j)}\,
e^{-(\alpha_j /\gamma_j)^{(t)}x}}{\sum\limits_{l=1}^k
p_l^{(t)}\fr{(\alpha_l^{\alpha_l})^{(t)} x^{\alpha_l -
1}}{(\lambda_l^{\alpha_l})^{(t)}\Gamma (\alpha_l )}\,
e^{-(\alpha_l /\lambda_l)^{(t)} x}}\,,\\
p_j^{(t+1)} & = \fr{1}{n}\,\sum\limits_{i=1}^n \g_{ij}^{(t)}\,.
\end{align*}

Далее найдем оценки $\lambda_j$ для данного случая, приравнивая
производную
\begin{equation} %8
\sum\limits_{j=1}^k \sum\limits_{i=1}^n \g_{ij}^{(t)} \ln \left (
\fr{\alpha_j^{\alpha_j} x^{\alpha_j -1}}{\lambda_j^{\alpha_j}\Gamma
(\alpha_j)}\,e^{-(\alpha_j /\lambda_j) x}\right )
\end{equation}
по $\lambda_j$ к нулю и разрешая относительно~$\lambda_j$ уравнение:
$$
\sum\limits_{i=1}^n \g_{ij}^{(t+1)}\left ( \fr{\alpha_j^{(t)}}{\lambda_j^{(t)}}
- \fr{\alpha_j^{(t)}X_i}{\left ( \lambda_j^{(t)}\right )^2}\right ) =0 \,.
$$
Получаем
$$
\lambda_j^{(t+1)} = \fr{\sum\limits_{i=1}^n \g_{ij}^{(t)}
X_i}{\sum\limits_{i=1}^n \g_{ij}^{(t)}}\,.
$$

Для того чтобы получить итерационные уравнения для $\alpha_j$, найдем
первую производную~(8):
\begin{multline*}
\left ( \sum\limits_{j=1}^k\sum\limits_{i=1}^n \g_{ij}^{(t)}\ln \left (
\fr{\alpha_j^{\alpha_j} x^{\alpha_j -1}}{\lambda_j^{\alpha_j}\Gamma (\alpha_j
)}\,e^{-(\alpha_j /\lambda_j ) x} \right ) \right )^\prime \alpha_j ={}\\[-3pt]
{}=\left ( \sum\limits_{j=1}^k\sum\limits_{i=1}^n \g_{ij}^{(t)}\ln \left (
\fr{\alpha_j^{\alpha_j}}{\lambda_j^{\alpha_j}}\right ) - \ln \Gamma (\alpha_j )+{} \right.\\[-3pt]
{}+\left.
(\alpha_j -1 )\ln X_i - \fr{\alpha_j}{\lambda_j}\,X_i \right )^\prime \alpha_j =\\[-3pt]
{}=\sum\limits_{i=1}^n \g_{ij}^{(t)} \left ( \ln \alpha_j +1-\ln \lambda_j -
\fr{\Gamma^\prime (\alpha_j )}{\Gamma (\alpha_j)}\right.+\\[-3pt]
{}+\left. \ln X_i - \fr{X_i}{\lambda_j}\right )\,;
\end{multline*}
\begin{multline*}
\sum\limits_{i=1}^n \g_{ij}^{(t)} \left(  \ln \alpha_j +1 -\ln \lambda_j -{}\right. \\[-3pt]
\left. {}-\fr{\Gamma^\prime (\alpha_j )}{\Gamma (\alpha_j )}+\ln X_i 
-\fr{X_i}{\lambda_j} \right) =0\,;
\end{multline*}
\begin{multline}
\fr{\Gamma^\prime (\alpha_j )}{\Gamma (\alpha_j )} ={}\\[-3pt]
{}= \fr{\sum\limits_{i=1}^n \g_{ij}^{(t)} \left ( \ln \alpha_j +1-\ln\lambda_j 
+\ln X_i -\fr{X_i}{\lambda_j} \right )}{\sum\limits_{i=1}^n \g_{ij}^{(t)}}\,.
\end{multline}
%
Здесь $\Gamma^\prime (\alpha_j ) / \Gamma (\alpha_j )$~--- это
\textit{логарифмическая производная гамма-функции}. Для нее существует так
называемое \textit{разложение Абрамовитца}--\textit{Стигана}~\cite{4bat}:
$$
\fr{\Gamma^\prime (\alpha ) }{ \Gamma (\alpha )} = \mathrm{log}\,\alpha -
\fr{1}{2\alpha }-\fr{1}{12\alpha^2 }+\ldots
$$

Подставим первые три члена разложения в~(9) и разрешим это уравнение
относительно~$\alpha_j$:
$$
\alpha_{ij}^{(t+1)} = \fr{\sum\limits_{i=1}^n
\g_{ij}^{(t+1)}}{2\sum\limits_{i=1}^n \g_{ij}^{(t +1)}\left ( \fr{X_i}{\lambda_j^{(t)}} -
\ln \fr{X_i}{\lambda_j^{(t)}} -1\right )}\,.
$$
В итоге получаем итерационные уравнения для ~$\alpha_j$.

\section{Описание программного обеспечения (программа~ЕМ)}

\subsection{Назначение программы} %3.1.

Разработанная авторами статьи программа ЕМ предназначена для решения задачи
разделения смесей экспоненциальных и гамма-распределений, поставленной в
разд.~2, с использованием ЕМ-ал\-го\-рит\-ма и формул, описанных в разд.~3.

\subsection{Инструменты разработки} %3.2.

Для создания программы была использована среда разработки Microsoft
Visual Studio .NET 2005 и объектно-ориентированный язык C\#. Для
визуализации результатов была использована свободно распространяемая
графическая библиотека ZedGraph~\cite{7bat}.


\subsection{Возможности  программы} %3.3.

\noindent
\begin{itemize}
\item Загрузка выборочных данных из текстового файла
\item Оценивание по выборке параметров смеси экспоненциальных
распределений
\item Оценивание по выборке параметров смеси гамма-распределений
\item Отслеживание изменений параметров смесей распределений во
времени в режиме <<скользящего окна>>
\item Построение гистограммы по выборке
\end{itemize}

\subsection{Входные и выходные данные. Функционирование
программы} %3.4.

В качестве \textit{входных данных} программа ЕМ получает:
\begin{itemize}
\item выборочные данные из текстового файла;
\item число компонентов смеси;
\item размер <<скользящего окна>>;
\item размер класса гистограммы.
\end{itemize}

На \textit{выходе} мы получаем:
\begin{itemize}
\item точечные оценки параметров смеси экспоненциальных
распределений;
\item точечные оценки параметров смеси гамма-распределений;
\item графическое изображение результирующей смеси распределения;
\item графическое изображение компонентов каж\-дой смеси;
\item графическое изображение того, как меняются параметры смесей
распределений с течением времени в режиме <<скользящего окна>>;
\item гистограмма, построенная по выборке;
\item значение статистического теста.
\end{itemize}

Выборочные данные загружаются из текстового файла в память программы и подаются
на вход двум независимо работающим реализациям ЕМ-алгоритма~--- для
идентификации смеси экспоненциальных распределений и для идентификации смеси
гамма-распределений. Результатом их работы являются наборы значений оцениваемых
параметров модели, предложенной в разд.~2. Кроме того, результирующие
распределения визуализируются в виде графиков. В программе можно запустить
режим <<скользящего окна>>, который для всех подвыборок заданного
размера с помощью ЕМ-алгоритма оценивает параметры смесей распределений этих
подвыборок. Все действия программы документируются в окне информации.

\section{Описание тестовых расчетов}

С использованием разработанной программы были проведены тестовые
расчеты на выборочных данных реального сетевого трафика.

На вход программы EM были поданы выборки трафика:
\begin{enumerate}[I]
\item Между лабораторией Lawrence Berkeley (Berkeley, California) и
внешним миром размера примерно 7000~\cite{8bat}~--- \textit{выборка~1}.
\item
Сети радиодоступа ЗАО <<Синтерра>> размера примерно 1000~\cite{9bat}~---
 \textit{выборка~2}.
\end{enumerate}

\subsection{Выборка 1 ``Berkeley''} %5.1.

При числе компонентов смеси~5 и случайном начальном приближении
были получены результаты, представленные в табл.~\ref{t1bat}.


Данные результаты иллюстрирует рис.~\ref{f5bat}.

Гистограмма  на рис.~\ref{f6bat} показывает статистическую значимость
полученных результатов.

Данная выборка обладает той особенностью, что она собиралась в течение
достаточно длительного времени и в ней агрегирован самый разнородный
трафик. Поэтому в ней присутствует не только большое количество
<<коротких>> сообщений (что обычно для выборок из телетрафика), но и
некоторый массив сообщений средней длины, а также определенный
<<выброс>> больших сообщений. Это свидетельствует о \textit{пиковости}
телетрафика на довольно больших промежутках времени.

Как мы видим, ЕМ-алгоритм удачно справился с задачей идентификации
смеси.

\subsection{Выборка~2 ``Synterra''} %5.2.

Результаты применения ЕМ-алгоритма к выборке ``Synterra''
представлены в табл.~\ref{t2bat}.
\begin{table*}\small
\begin{minipage}[t]{76mm}
\begin{center}
\Caption{Результаты применения ЕМ-алго\-рит\-ма к выборке~1 ``Berkeley'' 
\label{t1bat}} \vspace*{2ex}

\tabcolsep=8.7pt
\begin{tabular}{|c|c|c|}
\hline
№&Начальное приближение&Результат\\
\hline
\multicolumn{3}{|c|}{$P$}\\
\hline
0&0,2&0,1896\\
1&0,2&0,1858\\
2&0,2&0,1830\\
3&0,2&0,2259\\
4&0,2&0,2154\\
\hline
\multicolumn{3}{|c|}{$\alpha$}\\
\hline
0&2,7028&10,9783\hphantom{9}\\
1&3,6273&5,8621 \\
2&5,7598&2,7092\\
3&0,2315&1,0235\\
4&0,9110&0,4772\\
\hline
\multicolumn{3}{|c|}{$\lambda$}\\
\hline
0&85,2066&137,1714  \\
1&23,9592&136,7349\\
2&63,8425&132,6482\\
3&\hphantom{9}1,8026&116,7317\\
4&98,3882&102,5278\\
\hline
\end{tabular}
\end{center}
\end{minipage}\hfill
\begin{minipage}[t]{76mm}
%\end{table*}
%\begin{table*}\small
\begin{center}
\Caption{Результаты применения ЕМ-алго\-рит\-ма к выборке~2 ``Synterra'' 
\label{t2bat}} \vspace*{2ex}

\tabcolsep=8.7pt
\begin{tabular}{|c|c|c|}
\hline
№&Начальное приближение&Результат\\
\hline
\multicolumn{3}{|c|}{$P$}\\
\hline
0&0,2&$0{,}3815\hphantom{{}\cdot 10^{-9}}$\\
1&0,2&$0{,}3594\hphantom{{}\cdot 10^{-9}}$\\
2&0,2&$0{,}2589\hphantom{{}\cdot 10^{-9}}$\\
3&0,2&$0{,}4401\cdot 10^{-9}$\\
4&0,2&$0{,}0\hphantom{{}\cdot 10^{-9}999}$\\
\hline
\multicolumn{3}{|c|}{$\alpha$}\\
\hline
0&6,0804&1,5833\\
1&3,1838&0,8554\\
2&1,4886&0,4557\\
3&4,6407&0,2278\\
4&3,7843&0,1139\\
\hline
\multicolumn{3}{|c|}{$\lambda$}\\
\hline
0&17,3387&15,8682\\
1&47,8294&16,9150\\
2&54,1984&19,2866\\
3&\hphantom{1}8,6254&19,2866\\
4&\hphantom{1}5,7252&19,2866\\
\hline
\end{tabular}
\end{center}
\end{minipage}
\end{table*}


Данные результаты иллюстрируют рис.~\ref{f7bat}.


Эти результаты также отражают действительную картину, как показано на
рис.~\ref{f8bat}.


Этот трафик был снят с базовой станции <<Лукойл-Юго-Запад>> сети
широкополосного радиодоступа ЗАО <<Синтерра>>. Сеть радиодоступа
является реализацией так называемой <<последней мили>>, переносящей два
разных вида трафика: данные (Ethernet пакеты) и голос (IP-телефония, VoIP).
Поэтому здесь присутствуют в качестве основной массы короткие, но
интенсивные сообщения (пакеты SIP и голосовые фреймы), а также длинные
сообщения, содержащие данные.

Как мы видим, программная реализация ЕМ-ал\-го\-рит\-ма успешно справилась с
задачей разделения смесей распределений для этих двух выборок, что делает
данную программу удобным инструментом построения стохастической картины
конкретной сети. По полученным данным, используя метод интерпретации,
предложенный в разд.~2, можно получить представление о количестве
последовательных и параллельных структур вероятностной модели сети.

\subsection{Режим <<скользящего окна>>} %5.3.

Результаты для выборки
``Berkeley'' в режиме <<скользящего окна>>  представлены
на рис.~\ref{f9bat}.


Данные графики показывают изменение параметров распределений подвыборок выборки 
``Berkeley''. Видно, что параметры распределений подвыборок не остаются 
неизменными во времени, наоборот, они имеют внешне случайный характер. На 
рис.~\ref{f9bat},\,\textit{в} видна даже своеобразная пульсация первой 
компоненты.
%
На основании расчетов можно сделать вывод о том, что пиковость трафика
обусловливается как формой, так и интенсивностью сообщений.

\section{Заключение}

В данной работе исследована вероятностная модель  информационных потоков,
возникающих в сложных телекоммуникационных конвергентных сетях, построенная с
помощью асимптотического и энтропийного подходов. Эта модель предполагает, что
функционирование сложной телекоммуникационной сети можно представить в виде
суперпозиции довольно простых стохастических структур~--- последовательных и
параллельных, которые по\-рож\-да\-ют смеси гамма-распределений для случайной
величины времени обработки и передачи сообщений в сети. Предложена простая
интерпретация параметров данной модели.
\begin{figure*} %fig5
\vspace*{1pt}
\begin{center}
\mbox{%
\epsfxsize=130mm %145.109mm 
\epsfbox{bat-5.eps} }
\end{center}
\vspace*{-13pt} \Caption{Компоненты смеси начального приближения~(\textit{а}) и 
результата~(\textit{б}) для выборки~1 ``Berkeley'' \label{f5bat}}
%\end{figure*}
%\begin{figure*} %fig6
\vspace*{12pt}
\begin{center}
\mbox{%
\epsfxsize=130mm %148.256mm 
\epsfbox{bat-7.eps} }
\end{center}
\vspace*{-13pt} \Caption{График смеси распределений~(\textit{1}) и гистограмма 
для выборки~1 ``Berkeley''~(\textit{2}) \label{f6bat}}
\end{figure*}



\begin{figure*} %fig7
\vspace*{1pt}
\begin{center}
\mbox{%
\epsfxsize=130mm %144.283mm 
\epsfbox{bat-8.eps} }
\end{center}
\vspace*{-16pt} \Caption{Компоненты смеси начального приближения~(\textit{а}) и 
результата~(\textit{б}) для выборки~2 ``Synterra'' \label{f7bat}}
%\end{figure*}
%\begin{figure*} %fig8
\vspace*{12pt}
\begin{center}
\mbox{%
\epsfxsize=130mm %148.256mm 
\epsfbox{bat-10.eps} }
\end{center}
\vspace*{-11pt} \Caption{График смеси распределений~(\textit{1}) и гистограмма
для выборки~2 ``Synterra''~(\textit{2}) \label{f8bat}}
\end{figure*}

\begin{figure*} %fig9
\vspace*{1pt}
\begin{center}
\mbox{%
\epsfxsize=119.041mm
\epsfbox{bat-11.eps} }
\end{center}
\vspace*{-9pt} \Caption{Изменение  смешивающих параметров~(\textit{а}), 
параметров формы~(\textit{б}) и параметров масштаба~(\textit{в}) во времени для 
выборки~1 ``Berkeley'' \label{f9bat}}
\end{figure*}

Для решения вытекающей из модели задачи предложен итерационный алгоритм,
базирующийся на методе максимального правдоподобия~--- ЕМ-ал\-го\-ритм, для
которого получены формулы для конкретного вида смесей~--- экспоненциальных и
гамма-распределений.
%
Кроме того, разработан программный инструментарий для оценки параметров 
предложенной модели на выборках из реальных трафиковых данных. Проведены 
исследования, которые подтвердили предположения вероятностной модели. 


Получение информации о стохастической структуре
телекоммуникационных сетей и наличие программных инструментов для
выявления более или менее стабильных структур позволит понять причины
возникновения неожиданных больших нагрузок, предотвратить такие нагрузки,
а также поможет в будущем в проектировании надежных, оптимальных по
стоимости и уровню сервиса телекоммуникационных сетей нового поколения.

%\vspace*{-15pt} 
{\small\frenchspacing
{%\baselineskip=10.8pt
\addcontentsline{toc}{section}{Литература}
\begin{thebibliography}{9}
\bibitem{1bat}
Teletraffic Engeneering Handbook. International Telecommunication Union, 
Geneva, 2005 {\sf http://www.itu.int}. \vspace*{5pt} 
\bibitem{2bat}
\Au{Севастьянов~Б.\,А.} Курс теории вероятностей и математической статистики. 
М., 2004. \vspace*{5pt} 
\bibitem{3bat}
\Au{Айвазян~C.\,А., Бухштабер~В.\,М., Енюков~И.\,С, Мешалкин~Л.\,Д.} Прикладная 
статистика. Классификация и снижение размерности~// Финансы и статистика. М., 
1989. \vspace*{5pt} 
\bibitem{4bat}
\Au{Bilmes~J.\,A.} A gentle tutorial of the EM algorithm and its application to 
parameter estimation for Gaussian mixture and hidden Markov models. Berkeley, 
CA, USA: International Computer Science Institute,  1998. \vspace*{5pt} 
\bibitem{5bat}
\Au{Шлезингер~М.\,И.} О самопроизвольном различении образов~// Шлезингер~М.\,И. 
Читающие. автоматы. Киев: Наукова думка, 1965. С.~38--45. \vspace*{5pt} 
\bibitem{6bat}
\Au{Hsiao~I.-T., Rangarajan~A., Gindi~G.}. Joint-MAP 
reconstruction/segmentation for transmission tomography using mixture-models as 
priors. Yale University, 1998. \vspace*{5pt} 
\bibitem{7bat}
{\sf http://zedgraph.org}. \vspace*{4pt} 
\bibitem{8bat}
{\sf http://ita.ee.lbl.gov/html/contrib/LBL-PKT.html}. \vspace*{5pt} 
\bibitem{9bat}
{\sf http://www.synterra.ru}.
\end{thebibliography}

} } \label{end\stat}
\end{multicols}


%\addtocounter{razdel}{1}
%\def\razd{НЕРЕГУЛИРУЕМЫЙ ЭЛЕКТРОПРИВОД ДЛЯ ЭЛЕКТРОЭНЕРГЕТИКИ}

\setcounter{page}{2}

%   { %\Large  
   { %\baselineskip=16.6pt
   
   \vspace*{-48pt}
   \begin{center}\LARGE
   \textit{Предисловие}
   \end{center}
   
   %\vspace*{2.5mm}
   
   \vspace*{25mm}
   
   \thispagestyle{empty}
   
   { %\small 

    
Вниманию читателей журнала <<Информатика и её применения>> предлагается 
очередной тематический выпуск <<Вероятностно-статистические методы и 
задачи информатики и информационных технологий>>. Предыдущие тематические 
выпуски журнала по данному направлению вышли в 2008~г.\ (т.~2, вып.~2), 
в 2009~г.\ (т.~3, вып.~3) и в 2010~г.\ (т.~4, вып.~2). 

Статьи, собранные в данном журнале, посвящены разработке новых вероятностно-статистических 
методов, ориентированных на применение к решению конкретных задач информатики и информационных 
технологий, а также~--- в ряде случаев~--- и других прикладных задач. Проблематика, охватываемая 
публикуемыми работами, развивается в рамках научного сотрудничества между Институтом проблем 
информатики Российской академии наук (ИПИ РАН) и Факультетом вычислительной математики и 
кибернетики Московского государственного университета им.\ М.\,В.~Ломоносова в ходе работ 
над совместными научными проектами (в том числе в рамках функционирования 
Научно-образовательного центра <<Вероятностно-статистические методы анализа рисков>>). 
Многие из авторов статей, включенных в данный номер журнала, являются активными участниками 
традиционного международного семинара по проблемам устойчивости стохастических моделей, 
руководимого В.\,М.~Золотаревым и В.\,Ю.~Королевым; регулярные сессии этого семинара 
проводятся под эгидой МГУ и ИПИ РАН (в 2011~г.\ указанный семинар проводится в октябре 
в Калининградской области РФ). 

Наряду с представителями ИПИ РАН и МГУ в число авторов данного выпуска журнала входят 
ученые из Научно-исследовательского института системных исследований РАН, Института 
проблем технологии микроэлектроники и особочистых материалов РАН, Института 
прикладных математических исследований Карельского НЦ РАН, Московского 
авиационного института, Вологодского государственного педагогического университета, 
НИИММ им.\ Н.\,Г.~Чеботарева, Казанского государственного университета, Дебреценского 
университета (Венгрия).

Несколько статей выпуска посвящено разработке и применению стохастических методов и 
информационных технологий для решения различных прикладных задач. В~работе В.\,Г.~Ушакова 
и О.\,В.~Шестакова рассмотрена задача определения вероятностных характеристик случайных 
функций по распределениям интегральных преобразований, возникающих в задачах эмиссионной 
томографии. В~статье Д.\,О.~Яковенко и М.\,А.~Целищева рассмотрены некоторые вопросы 
математической теории риска и предложен новый подход к диверсификации инвестиционных 
портфелей. Работа И.\,А.~Кудрявцевой и А.\,В.~Пантелеева посвящена построению и 
исследованию математической модели, описывающей динамику сильноионизованной плазмы. 
В~статье П.\,П.~Кольцова изучается качество работы ряда алгоритмов сегментации изображений. 
Статья А.\,Н.~Чупрунова и И.~Фазекаша посвящена вероятностному анализу числа без\-оши\-бочных 
блоков при помехоустойчивом кодировании; получены усиленные законы больших чисел для указанных 
величин.

В данном выпуске традиционно присутствует тематика, весьма активно разрабатываемая в течение 
многих лет специалистами ИПИ РАН и МГУ,~--- методы моделирования и управления для 
информационно-телекоммуникационных и вычислительных систем, в частности методы 
теории массового обслуживания. В~статье А.\,И.~Зейфмана с соавторами рассматриваются 
модели обслуживания, описываемые марковскими цепями с непрерывным временем в случае 
наличия катастроф. В~работе М.\,М.~Лери и И.\,А.~Чеплюковой рассматриваются случайные 
графы Интернет-типа, т.\,е.\ графы, степени вершин которых имеют степенные распределения; 
такие задачи находят применение при исследовании глобальных сетей передачи данных. 
Работа Р.\,В.~Разумчика посвящена исследованию систем массового обслуживания специального 
вида~--- с отрицательными заявками и хранением вытесненных заявок.

Ряд статей посвящен развитию перспективных теоретических 
вероятностно-статистических методов, которые находят широкое применение в различных 
задачах информатики и информационных технологий. В~работе В.\,Е.~Бенинга, А.\,К.~Горшенина 
и В.\,Ю.~Королева рассмотрена задача статистической проверки гипотез о числе компонент 
смеси вероятностных распределений, приводится конструкция асимптотически наиболее мощного 
критерия. Результаты этой работы найдут применение в ряде прикладных задач, использующих 
математическую модель смеси вероятностных распределений (в информатике, моделировании 
финансовых рынков, физике турбулентной плазмы и~т.\,д.). В~статье В.\,Ю.~Королева, 
И.\,Г.~Шевцовой и С.\,Я.~Шоргина строится новая, улучшенная оценка точности нормальной 
аппроксимации для пуассоновских случайных сумм; как известно, указанные случайные суммы 
широко используются в качестве моделей многих реальных объектов, в том числе в информатике, 
физике и других прикладных областях. Работа В.\,Г.~Ушакова и Н.\,Г.~Ушакова посвящена 
исследованию ядерной оценки плотности распределения; эти результаты могут применяться, 
в част\-ности, при анализе трафика в телекоммуникационных системах. Серьезные приложения 
в статистике могут получить результаты работы О.\,В.~Шестакова, в которой доказаны оценки 
скорости сходимости распределения выборочного абсолютного медианного отклонения к нормальному 
закону. 

\smallskip

Редакционная коллегия журнала выражает надежду, что данный тематический  выпуск 
будет интересен специалистам в области теории вероятностей и математической статистики 
и их применения к решению задач информатики и информационных технологий.
     
     %\vfill 
     \vspace*{20mm}
     \noindent
     Заместитель главного редактора журнала <<Информатика и её 
применения>>,\\
     директор ИПИ РАН, академик  \hfill
     \textit{И.\,А.~Соколов}\\
     
     \noindent
     Редактор-составитель тематического выпуска,\\
     профессор кафедры математической статистики факультета\\
      вычислительной математики и кибернетики МГУ им.\ М.\,В.~Ломоносова,\\
     ведущий научный сотрудник ИПИ РАН,\\ 
доктор физико-математических наук \hfill
      \textit{В.\,Ю.~Королев}
     
     } }
     }



%   { %\Large  
   { %\baselineskip=16.6pt
   
   \vspace*{-48pt}
   \begin{center}\LARGE
   \textit{Предисловие}
   \end{center}
   
   %\vspace*{2.5mm}
   
   \vspace*{25mm}
   
   \thispagestyle{empty}
   
   { %\small 

    
Вниманию читателей журнала <<Информатика и её применения>> предлагается 
очередной тематический выпуск <<Вероятностно-статистические методы и 
задачи информатики и информационных технологий>>. Предыдущие тематические 
выпуски журнала по данному направлению вышли в 2008~г.\ (т.~2, вып.~2), 
в 2009~г.\ (т.~3, вып.~3) и в 2010~г.\ (т.~4, вып.~2). 

Статьи, собранные в данном журнале, посвящены разработке новых вероятностно-статистических 
методов, ориентированных на применение к решению конкретных задач информатики и информационных 
технологий, а также~--- в ряде случаев~--- и других прикладных задач. Проблематика, охватываемая 
публикуемыми работами, развивается в рамках научного сотрудничества между Институтом проблем 
информатики Российской академии наук (ИПИ РАН) и Факультетом вычислительной математики и 
кибернетики Московского государственного университета им.\ М.\,В.~Ломоносова в ходе работ 
над совместными научными проектами (в том числе в рамках функционирования 
Научно-образовательного центра <<Вероятностно-статистические методы анализа рисков>>). 
Многие из авторов статей, включенных в данный номер журнала, являются активными участниками 
традиционного международного семинара по проблемам устойчивости стохастических моделей, 
руководимого В.\,М.~Золотаревым и В.\,Ю.~Королевым; регулярные сессии этого семинара 
проводятся под эгидой МГУ и ИПИ РАН (в 2011~г.\ указанный семинар проводится в октябре 
в Калининградской области РФ). 

Наряду с представителями ИПИ РАН и МГУ в число авторов данного выпуска журнала входят 
ученые из Научно-исследовательского института системных исследований РАН, Института 
проблем технологии микроэлектроники и особочистых материалов РАН, Института 
прикладных математических исследований Карельского НЦ РАН, Московского 
авиационного института, Вологодского государственного педагогического университета, 
НИИММ им.\ Н.\,Г.~Чеботарева, Казанского государственного университета, Дебреценского 
университета (Венгрия).

Несколько статей выпуска посвящено разработке и применению стохастических методов и 
информационных технологий для решения различных прикладных задач. В~работе В.\,Г.~Ушакова 
и О.\,В.~Шестакова рассмотрена задача определения вероятностных характеристик случайных 
функций по распределениям интегральных преобразований, возникающих в задачах эмиссионной 
томографии. В~статье Д.\,О.~Яковенко и М.\,А.~Целищева рассмотрены некоторые вопросы 
математической теории риска и предложен новый подход к диверсификации инвестиционных 
портфелей. Работа И.\,А.~Кудрявцевой и А.\,В.~Пантелеева посвящена построению и 
исследованию математической модели, описывающей динамику сильноионизованной плазмы. 
В~статье П.\,П.~Кольцова изучается качество работы ряда алгоритмов сегментации изображений. 
Статья А.\,Н.~Чупрунова и И.~Фазекаша посвящена вероятностному анализу числа без\-оши\-бочных 
блоков при помехоустойчивом кодировании; получены усиленные законы больших чисел для указанных 
величин.

В данном выпуске традиционно присутствует тематика, весьма активно разрабатываемая в течение 
многих лет специалистами ИПИ РАН и МГУ,~--- методы моделирования и управления для 
информационно-телекоммуникационных и вычислительных систем, в частности методы 
теории массового обслуживания. В~статье А.\,И.~Зейфмана с соавторами рассматриваются 
модели обслуживания, описываемые марковскими цепями с непрерывным временем в случае 
наличия катастроф. В~работе М.\,М.~Лери и И.\,А.~Чеплюковой рассматриваются случайные 
графы Интернет-типа, т.\,е.\ графы, степени вершин которых имеют степенные распределения; 
такие задачи находят применение при исследовании глобальных сетей передачи данных. 
Работа Р.\,В.~Разумчика посвящена исследованию систем массового обслуживания специального 
вида~--- с отрицательными заявками и хранением вытесненных заявок.

Ряд статей посвящен развитию перспективных теоретических 
вероятностно-статистических методов, которые находят широкое применение в различных 
задачах информатики и информационных технологий. В~работе В.\,Е.~Бенинга, А.\,К.~Горшенина 
и В.\,Ю.~Королева рассмотрена задача статистической проверки гипотез о числе компонент 
смеси вероятностных распределений, приводится конструкция асимптотически наиболее мощного 
критерия. Результаты этой работы найдут применение в ряде прикладных задач, использующих 
математическую модель смеси вероятностных распределений (в информатике, моделировании 
финансовых рынков, физике турбулентной плазмы и~т.\,д.). В~статье В.\,Ю.~Королева, 
И.\,Г.~Шевцовой и С.\,Я.~Шоргина строится новая, улучшенная оценка точности нормальной 
аппроксимации для пуассоновских случайных сумм; как известно, указанные случайные суммы 
широко используются в качестве моделей многих реальных объектов, в том числе в информатике, 
физике и других прикладных областях. Работа В.\,Г.~Ушакова и Н.\,Г.~Ушакова посвящена 
исследованию ядерной оценки плотности распределения; эти результаты могут применяться, 
в част\-ности, при анализе трафика в телекоммуникационных системах. Серьезные приложения 
в статистике могут получить результаты работы О.\,В.~Шестакова, в которой доказаны оценки 
скорости сходимости распределения выборочного абсолютного медианного отклонения к нормальному 
закону. 

\smallskip

Редакционная коллегия журнала выражает надежду, что данный тематический  выпуск 
будет интересен специалистам в области теории вероятностей и математической статистики 
и их применения к решению задач информатики и информационных технологий.
     
     %\vfill 
     \vspace*{20mm}
     \noindent
     Заместитель главного редактора журнала <<Информатика и её 
применения>>,\\
     директор ИПИ РАН, академик  \hfill
     \textit{И.\,А.~Соколов}\\
     
     \noindent
     Редактор-составитель тематического выпуска,\\
     профессор кафедры математической статистики факультета\\
      вычислительной математики и кибернетики МГУ им.\ М.\,В.~Ломоносова,\\
     ведущий научный сотрудник ИПИ РАН,\\ 
доктор физико-математических наук \hfill
      \textit{В.\,Ю.~Королев}
     
     } }
     }


%%\newcommand{\eol}{\end{enumerate}\setlength{\itemsep}{-\parsep}}
%\newcommand{\ang}[1]{\langle{#1}\rangle}
%\newcommand{\infinity}{\infty}
%\newcommand{\mess}[1]{\mbox{\tt #1}}
%\newcommand{\var}[1]{\mbox{\it #1}}
%\newcommand{\order}[1]{\stackrel{#1}\fa}
%\newcommand{\orderr}[1]{\stackrel{#1}\Longrightarrow}
%\newcommand{\infrel}[1]{\stackrel{#1}\Longrightarrow}
%\newcommand{\prog}{\mbox{\tt Prog}}
%\newcommand{\comment}[1]{}
%\newcommand{\set}[1]{\{#1\}}
%\newcommand{\pair}[2]{\langle #1,#2 \rangle}
%\newcommand{\remove}[1]{}
%\renewcommand{\qed}{\hfill\rule{2mm}{2mm}}
%\newcommand{\bull}[1]{\begin{itemize}\item{#1}\end{itemize}}
%\newcommand{\marg}[1]{\marginpar{\small #1}}


\renewcommand{\figurename}{\protect\bf Figure}
\renewcommand{\tablename}{\protect\bf Table}

\def\stat{frenkel}


\def\tit{SEAMLESS ROUTE UPDATES IN SOFTWARE-DEFINED NETWORKING 
VIA QUALITY OF~SERVICE COMPLIANCE VERIFICATION}

\def\titkol{Seamless route updates in software-defined networking via 
quality of service compliance verification}

\def\autkol{S.\,L.~Frenkel and~D.~Khankin}

\def\aut{S.\,L.~Frenkel$^1$ and~D.~Khankin$^2$}

\titel{\tit}{\aut}{\autkol}{\titkol}

%{\renewcommand{\thefootnote}{\fnsymbol{footnote}}
%\footnotetext[1] {The 
%research of Yuri Kabanov was done under partial financial support of the grant 
%of RSF No.\,14-49-00079.}}

\renewcommand{\thefootnote}{\arabic{footnote}}
\footnotetext[1]{Institute of Informatics Problems, Federal Research 
Center ``Computer Science and Control'' of the Russian Academy of Sciences,
 44-2~Vavilov Str., Moscow 119333, Russian Federation, \mbox{fsergei51@gmail.com}}
\footnotetext[2]{Computer Science Department, Ben-Gurion University of the Negev, 
Beer-Sheva 84105, Israel, \mbox{danielkh@post.bgu.ac.il}}


\index{Frenkel S.\,L.}
\index{Khankin D.}
\index{Френкель С.}
\index{Ханкин Д.}

\def\leftfootline{\small{\textbf{\thepage}
\hfill INFORMATIKA I EE PRIMENENIYA~--- INFORMATICS AND
APPLICATIONS\ \ \ 2018\ \ \ volume~12\ \ \ issue\ 4}
}%
 \def\rightfootline{\small{INFORMATIKA I EE PRIMENENIYA~---
INFORMATICS AND APPLICATIONS\ \ \ 2018\ \ \ volume~12\ \ \ issue\ 4
\hfill \textbf{\thepage}}}

\vspace*{4pt}

\Abste{In software-defined networking (SDN), the control plane and the data 
plane are decoupled. This allows high flexibility by providing abstractions 
for network management applications and being directly programmable. 
However, reconfiguration and updates of a~network are sometimes inevitable due 
to topology changes, maintenance, or failures. In the scenario,  
a~current route~$C$ and a set of possible new routes~$\{N_i\}$, where one of the 
new routes is required to replace the current route, are given. There is a chance that 
a~new route $N_i$ is longer than a~different new route $N_j$, but $N_i$ is 
a~more reliable one and it will update faster or perform better after the update 
in terms of quality of service (QoS) demands. 
Taking into account the random nature of the network functioning, 
the present authors supplement the recently proposed algorithm by Delaet
\textit{et al}.\ for route updates with 
a~technique based on Markov chains (MCs). As such, an enhanced algorithm 
for complying QoS demands during route updates is proposed
in a~seamless fashion. First, 
an extension to the update algorithm of Delaet \textit{et al}.\ 
that describes the transmission of packets through a~chosen route and compares 
the update process for all possible alternative routes is suggested. Second, several 
methods for choosing a~combination of preferred subparts of new routes, resulting 
in an optimal, in the sense of QoS compliance, new route is provided.} 

\KWE{software-defined networking; Markov chains; quality of service}

\DOI{10.14357/19922264180408}


\vspace*{8pt}


\vskip 12pt plus 9pt minus 6pt

 \thispagestyle{myheadings}

 \begin{multicols}{2}

 \label{st\stat}

\section{Introduction}
\label{s:Intro}

\noindent
Software-defined networking is an emerging network paradigm, in which the 
control plane is decoupled from the data plane enabling centralized control 
logic. Such a~dynamic network may require frequent modifications and updates to 
the network topology and configuration. 
Also, the network topology is available to the centralized control entity, there, 
due to this flexibility, it is possible to perform offline optimized calculations.

Network functions virtualization (NFV) allows replacing traditional network 
devices with software that is running on commodity servers. This software 
implements the functionality that was previously provided by dedicated hardware. 
Network functions virtualization
 allows services to be composed of virtual network functions (VNF) hosted on 
different data centers. Software-defined networking, 
when applied to NFV, helps in addressing challenges 
of dynamic resource management and intelligent service 
orchestration~\cite{rao_sdn_2014}. Sometimes, traffic is often required to pass 
through and be processed by an ordered sequence of possibly remote 
VNFs~\cite{ghaznavi_service_2016}. For example, traffic may be required to pass 
through intrusion detection system, proxy, load balancer, or a~firewall. 
Such concatenation of services is called \textit{service function chaining} 
(SFC).

Consider, for example, two communicating parties in a~network featuring complex 
network topology (e.\,g., Small-world network), and the communication flow is 
passed over a~series of VNFs. It may be the case that the network operator is 
required to move the communicating flow to a~different path due to QoS 
requirements or other possible cost considerations. We are interested 
to model the anticipated expected number of steps until the update is complete 
given a~possible new route following the required QoS demands, e.\,g., 
delay, communication rounds, cost, etc. 

%Aforesaid dynamic networking requires frequent modifications and updates to the network. 
Let us consider a pair $(C, \{N_i\})$ where a~current route~$C$ from~$s$ to~$d$ 
is scheduled to be replaced by a new route from the set~$\{N_i\}$, each from~$s$ 
to~$d$ either. Let us model each route as an ordered list of network elements, such 
as VNFs (SFCs) or generally saying routers. Each new route~$N_i$ is constructed 
during the update process, and thus, certain delays may be introduced due to
 initial packet processing or due to possible losses. 
 %There, the eventual arrival of packets along the new route during the update process is critical for successful route update. Another possible example is when the routes are SFCs, and the requirement is to update a current chain to a new one, different service chains may exhibit different delays. 

The design goals must be achieved by constructing effective algorithms for 
efficient packet QoS routing in NFV/SDN computer network. Depending on the 
QoS metric, the lower (e.\,g., for reliability) or upper (e.\,g., for a~delay) 
constraints represent the desired bounds that the orchestration must meet. 
Since different configurations could meet these bounds, the designer should also 
optimize against a~specific metric by using these both ends of the extreme. 

Methods based on integer linear programming (ILP) were proposed in several works 
(see section~\ref{sec:related_work}). The difficulty of using tools based on ILP 
 in the operational work of an administrator is that in view of the possible 
 infeasibility of the resulting solution, it may take not a~few resources (time, efforts) 
 until acceptable QoS values can be ensured.

We consider the use of ``design via verification'' approach, suggesting a~method 
for complying QoS demands. The method is based on augmenting the update algorithm with
a~verification logic. Namely, we suggest the use of 
\textit{Probabilistic real-time Computation Tree Logic} 
(PCTL)~\cite{hansson_logic_1994} for expressing real-time and probability in systems. 
Using PCTL, we can express the probability for a~process to complete after 
a~certain number of steps along an execution path and verify the selected route 
for the update.


%Assume that packets are sent from a source node $s$ to a destination node $d$ along the current route. After the update process is finished, packets will be forwarded from $s$ to $d$ along the new route. 
Delaet \textit{et al.}\ proposed a~multicast-based scheme for seamlessly updating 
a~current route to a~new one~\cite{delaet_seamless_2015}. 
According to the multicast scheme, the controller instructs 
a~router to temporarily have two $(s,d)$ entries in the routing table. When 
a~router $r \neq d$ receives a~packet from~$s$ to~$d$, it sends the packet 
according to the forwarding instructions of all of its $(s,d)$ routing 
table entries. When two identical copies of a~packet that was multicasted 
over the current and new portion of a~route arrive, the controller can dismantle 
the current route, as the new route is ready. During the update process, packets 
should not be lost, no cycles should be formed, and communication should not 
be disrupted.

%Taking into account the random nature of the network functioning, we supplement the algorithm for route updates introduced by Delaet et al. in \cite{delaet_seamless_2015}, with a technique based on Markov chains. In our extension of the algorithm, we describe the transmission of packets through a chosen route and compare the update process for all the possible alternative routes that are candidates for replacement. 

Our contribution is a model for a successful route update, including its 
intermediate steps, as MC states, each with 
a~given probability. With our model, we are able to characterize the quality of 
an update by expected number of steps in the~MC. 
%We use Markov chains to characterize the quality of the update service, and represent the expected number of steps in the Markov chain as the quality of a successful update. While, the probability for an update event 

We suggest an enhanced update method for the network administrator to augment 
his decision regarding QoS demands in terms of various network parameters and 
possible failure of the update process. Moreover, in contrast to other works, 
we are able to provide a~version of an algorithm that can perform real-time QoS
 assessment during a~route update, for each subpart of a~route. At last, using 
 our method, it is possible that the active new route will be comprised of subparts 
 of different new routes, providing optimal route update service in regard of 
 required network QoS. 

%We assume that each new route is legal. 
%However, mixing subroutes belonging to different routes may result in inconsistent state or a cycle formed in the network. We use different 
%
%
%
%We model the update process as a service, namely as a VNF, and we use Markov chains to characterize the quality of the update service. Using the expected number of steps in the Markov chain representing the update, we abstract the quality of the update service. We calculate for each possible new (sub-)route the expected number of steps required to update an old (sub-)route successfully. Subsequently, the old route is updated to the new route which requires less number of steps with high probability. We supplement the seamless update algorithm proposed by the authors of \cite{delaet_seamless_2015} with the model in this work.

%The virtualized service implementing the update algorithm will provide a recommendation for an optimal choice of a route, based on the performed calculations. Fundamentally, we create a QoS VNF for seamlessly updating a route, regarding network parameters, and taking into consideration the complexity and possible failures of updating a route. In case there exist several alternatives for a route update, there is a chance that one of the possible new routes is much longer, however, a more reliable one, and as such will update faster. 
%
%
%One of the important requirements to modification process is that the update process should not form congestion in the network, nor result in time delays, and not lose any packets. 
%
%
%Additionally, we provide an enhanced version of an algorithm that can perform the quality of service assessment during the update process, for each subpart of the new route. 
%
%We propose a directed graph $G=(V,E)$, for representing the possible legal combinations of sub-routes. The set of common nodes to $(C, \{N_i\})$ subdivides the old route and each of the new routes to sub-routes. For two sub-routes represented by the nodes $u,v \in V$, the sub-route $v$ can be launched after $u$ if and only if there exists a directed edge $(u,v) \in E$. Otherwise, the launch of $v$ after $u$ is forbidden and can result in a cycle formed in the network.


%The results of this work helped to develop an operating strategy for a network administrator, supporting both, seamlessly updating a route, and providing QoS requirements. 

Extended abstract of this work appeared as a conference paper 
in~\cite{frenkel_predicting_2017} which presented preliminary results. 
In this work, we describe in detail the system settings and bring new results 
by providing two additional algorithms.
{\looseness=1

}

In the following section, we overview the related work. Next, we provide 
the required definitions and the system settings and describe the MC 
characterization of the network. Further, we describe different update setting, 
accordingly accompanying algorithms and data structures, used for QoS assessment 
during route updates.

\vspace*{-9pt}

\section{Related Work}
\label{sec:related_work}

\vspace*{-2pt}
%The design goals must be achieved by constructing effective algorithms for efficient packet QoS routing in NFV/SDN computer network. %These algorithms, which must enable an administrator to orchestrate the existing services exported by remote providers, were considered in \cite{martins_clickos_2014, zaalouk_orchsec:_2014}. Likewise, the functional behavior (e.g., services being deprecated by their providers), as well as changes in the non-functional behavior of the orchestrated services (e.g., an increased execution time) were also considered.

%Depending on the QoS metric, the lower (e.g., for reliability) or upper (e.g., for delay) constraints represent the desired bounds that the orchestration must meet. Since different configurations could meet these bounds, the designer must also optimize against a specific metric by using these both ends of extreme.

\noindent
Quality of service routing using multipath was proposed in~\cite{devi_approach_2015}. 
The routing algorithm, initially, eliminates all links that do not meet the 
bandwidth requirements. Then, it finds disjoint shortest paths based on 
the residual network graph in each iteration.

The work~\cite{egilmez_distributed_2012} proposed a~QoS optimized routing 
over multidomain OpenFlow networks managed by a~distributed control plane, 
where each controller performs optimal routing within its domain. 
The QoS routing problem was posed as a~constrained shortest path (CSP) problem, 
and the proposed solution computes a~near-optimal route, based on LARAC 
(Lagrange relaxation based aggregated cost)
algorithm~\cite{juttner_lagrange_2001}. The proposed algorithm is an approximation 
algorithm; it always gives a~suboptimal solution.

For traditional network architecture, a~routing strategy approach based on 
ILP was introduced in~\cite{yu_efficient_2013}.
 The main disadvantage of using ILP is that the problem is NP-hard. 
 Additionally, ILP cannot be applied to probabilistic values. 
 Using linear programming (not limited to integers) rounded to integer solutions 
 will not yield an optimal solution.
 

Route updates are extensively researched in SDN~\cite{foerster_survey_2016}, 
standing on the work by Reitblatt \textit{et al.}\ where requirements for SDN 
updates were examined. This work focused on per-packet consistency property, 
stating that packets have to be forwarded either using the initial configuration 
or the final configuration but never a~mixture of them, throughout the update 
process~\cite{reitblatt_consistent_2011}. The authors proposed 
a~2-phase commit technique which relies on packets tagging so that either of 
the rules is applied. However, such technique wastes critical network resources 
and complications are formed due to packet tagging~\cite{foerster_survey_2016}. 
Further, Delaet \textit{et al.}\ showed in~\cite{delaet_seamless_2015} 
that using a~careful multicast during route updates provides 
a~better working solution.

Hogan and Esposito propose in~\cite{hogan_stochastic_2017} the use of
 Bayesian networks for delay estimation as a~traffic engineering tool and model 
 the path selection problem using a~risk minimization technique. 
 However, the authors state that the accuracy of their model is limited by its 
 ability to correctly identify dependencies in the data. In our work, 
 we suggest a~general tool for probabilistic verification of any network parameter, 
 which does not depend on variance within the dataset.
 
 

In~\cite{mcgeer_safe_2012}, an update protocol proposed where packets are 
sent to the controller during updates; such approach adds 
a~significant cost to the control plane bandwidth~\cite{delaet_seamless_2015}. 
In~\cite{mcgeer_correct_2013}, an algorithm to find 
a~safe update sequence expressed as a~logic circuit has been proposed. 
However, the algorithm 
requires a~dedicated protocol which is not currently 
supported~\cite{foerster_survey_2016}. The authors 
of~\cite{katta_incremental_2013} propose to perform the 2-phase update 
scheme from~\cite{reitblatt_consistent_2011} incrementally, making the update longer. 
%For a thorough review of route updates, the reader is referred to \cite{foerster_survey_2016}.






Software-defined networking allows the involvement of the network administrator into the network 
management during route udpdates and, in particular, during packet transmission. 
Thus, it would be highly desirable to support the decision making process 
with the right tools. Our novelty is exactly such tool, for augmenting 
online decision making of the network administrator during network management 
in a~stochastic environment.
%In this work, we propose a technique to optimize the update process by selecting the preferred (sub-)route in order to reduce the update time. We use the expected number of steps for successfully completing the update as a QoS metric, and extend the algorithm by Delaet~et~al. with Discrete Time Markov Chains (DTMC) for finding (sub-)routes which are preferred in terms of QoS. % As such, we propose to use the route updates algorithm from \cite{delaet_seamless_2015} as a virtual service for network updates per QoS requirements.

%The interaction of software components have a greater weight in NFV context, which may lead to stochastic-like behavior 

%At present, certain routing algorithms (including $k$ Edge-Disjoint) are based on the shortest path (SP) problem solution \cite{wood_toward_2015}. However, the method proposed by Wood et al. is generic and valuable only in the case of request arrival, and do not consider certain additional important requirements, such as removal or priorities of requests. 

%Several approaches for efficient SP-based QoS routing have been recently proposed in \cite{buchbinder_improved_2006}, where the authors introduce and analyze a centralized algorithm for an online scheduling and routing of arbitrary sequence of communication requests. 

%Unsplittable (single-path) assignment for each request of QoS routing is probably competitive with the best possible splittable (multipath assignment).

The work by Delaet \textit{et al.}~[4] introduced the Make\&Activate-Before-Break 
approach for seamless
route update in SDN. The authors described in a~high-level the multicasting-based 
update, which we
employ in this work. Also, they introduced a~controller-based method for 
verifying the correctness
of a~new route before the traffic redirection. Dinitz \textit{et al.}~[16] 
extended the work~[4] to the general
case of several dependent (via shared links) routes pairs. The routes update 
problem was proved to
be NP-hard~\cite{17-aaa}. The authors of~[16] suggested the use of 
artificial intelligence (AI) methods for 
solving the problem. As a~basis for AI-based solutions, Dinitz 
\textit{et al.}\ proposed a dependence graph model describing the current
state of the problem instance at any replacement stage. 
In addition, route readiness verification similar
to that in~[4] was implemented in~[16] as a high-level network protocol.

In this work, we investigate a different problem; we consider the route updates 
problem from a~QoS
perspective and describe in high-level both the prediction and the update processes.

\vspace*{-9pt}

\section{Preliminaries and Definitions}

\vspace*{-2pt}

\noindent
The basic system settings are as follows. 
For a~(route) sequence~$X$, we denote by~$x_i$ the $i$th element in it.
In a~(directed) communication network, 
we are given a~route~$C$ from source~$s$ to destination~$d$. 
Additionally, we are given a~set of different new routes~$N_i$, each going from~$s$ 
to~$d$. We model each route as an ordered set of network nodes connected by network 
links. We assume that neither of the routes contains cycles. 
Each router in a~route matches a~packet from~$s$ to~$d$ 
and forwards the packet to the next router in order. After the update 
is complete, each router in the new route should forward the packets from~$s$ 
to~$d$ to the next router in order along the new route. 

In our work, we consider the route replacement problem as a~sequence of 
subroutes replacements.
The routes replacement subsystem was in great detail described by Dinitz 
\textit{et al.} in~\cite{dinitz_dependence_2017}. We borrow
from~[16] the relevant parts which we briefly describe here.

\smallskip

\noindent
\textbf{Definition~1.} We  define a~subset from $a\in X$ to $b\in X$ of an ordered
set~$X$, when $a$ precedes~$b$, as~a~subroute from~$a$ to~$b$, and denote such subroute by
$[a,b]$.

\smallskip

 

\textbf{Subroutes.} The current route~$C$ subdivides each new route 
to~$k$~common subroutes (a~subroute may consist of one router in the simplest case) 
and $k-1$ noncommon subroutes. 
For illustration, see Fig.~1.
In Fig.~1 and figures below, the current route is depicted
in a~light grey color full nodes, connected with
solid edges. The new route is depicted in white colored nodes, connected with
dashed edges. The common nodes are depicted as shaded. 
If there are several new
routes, the nodes of each route are filled with a~designating pattern. 
Additionally, for easier reading,
when it is possible, we denote subroutes of some route~$X$ as~$X^\prime$, $X^{\prime\prime}$, 
etc. In other cases, a~subroute~$j$
of a~new (current) route~$i$ is denoted as $N_j^i (C_i^j)$. 
Similarly, routers of some route~$X$ are denoted by~$r^\prime$,
$r^{\prime\prime}$, etc.

 { \begin{center}  %fig1
\vspace*{1pt}
 \mbox{%
 \epsfxsize=78.631mm 
 \epsfbox{fre-1.eps}
 }


\vspace*{3pt}


\noindent
{{\figurename~1}\ \ \small{Route $C$ with two possible new routes sharing a~link}}
\end{center}
}

\vspace*{6pt}






In the example in Fig.~1, 
noncommon new subroutes 
of route~$N_1$ are denoted by~$N^1_1=[s,r_2]$ and~$N^2_1=[r_2,d]$, while the noncommon new 
subroutes of~$N_2$ are denoted by~$N^1_2=[s,r_1]$, $N^2_2=[r_1,r_3]$, 
$N^3_2=[r_3,r_2]$, and~$N^4_2=[r_2,d]$. 

Note that in general, the order of common subroutes along~$C$ and along~$N$ 
can be different. See, for example, the common subroutes of~$C$ and~$N_2$ in 
%Figure \ref{fig:two_routes}.
Fig.~1.

\smallskip

\noindent
\textbf{Definition~2.} A~new noncommon subroute of~$N$ from router~$a$
to router~$b$ is legitimate for update only if~$a$ precedes~$b$ on the route~$C$.

\smallskip

Definition~2 guides us on which subroutes can be launched without creating routing cycles in the
network system. (See~[4] for details.)


When an update of a~subroute~$N^\prime$ from router~$r$ to~$r^\prime$ is finished, 
the update flow goes along~$C$ from~$s$ to~$r$, continues along~$N^\prime$ up to~$r^\prime$, 
and finishes along~$C$ from~$r^\prime$
 to~$d$. 
For illustration, see the result of launching~$N^4_2$ in Fig.~2.

 { \begin{center}  %fig2
\vspace*{-1pt}
 \mbox{%
 \epsfxsize=78.631mm 
 \epsfbox{fre-2.eps}
 }


\vspace*{3pt}


\noindent
{{\figurename~2}\ \ \small{$N^4_2$ was launched}}
\end{center}
}

\vspace*{4pt}


 

 Note that launching a~currently nonlegitimate new subroute, for example,~$N^3_2$ 
 in Fig.~1, is forbidden since it will form a~cycle 
 resulting in packets circulating and overwhelming the network. 

\textbf{Dynamics of the system.}
%\label{sec:dynamics} 
Dinitz \textit{et al.}\ performed a~detailed analysis on the dynamics of a~subroutes
system. After an update of a~subroute is complete, the set of current subroutes~$C$ 
and the set
of new subroutes~$N$ are recalculated. This may result in different system of subroutes. For example,
see Fig.~2 where after the launch of $N^4_2$ from the example in Fig.~1, 
the sets of subroutes are
recalculated. As a~result, we obtain different subroutes (for clarity, the previous labels are kept). See
also~[16] for details and extensive analysis.

\vspace*{-4pt}

\subsection{Markov chain characterization of~the~network~states}

\noindent
We characterize execution of some (sub)route in the network by 
a~packet delay time between the (sub)route's common sender and common destination 
routers as well the probability of a~packet drop. Let us for now define our 
network routing model (conceptual model) informally in the following terms. 
Delay of a~packet is obtained using a~physical delay and the total processing 
time in the router. We consider that transmission of packets in 
a~network can have a~random behavior, caused by the random character of both, 
the input, and possible loss of packets. There we are interested in 
a~probabilistic model, namely, a~Markov model. In order to fully characterize 
the network as an~MC, the internal state of each router 
(and, in particular, the buffer occupancies), as well as the characteristics
 of all flows, need to be expressed as states in the chain. 

However, such approach would result in an enormous and intractable number of states. 
Therefore, to simplify these computations, let us characterize the delay time as 
an abstract variable~$t$. This abstract variable can be interpreted in different ways, 
e.\,g., the current processing queue length and a~packet transmission rate of the link, 
or possibly a~fixed value, such as an interval between the beginning of 
a~packet transmission after being processed in some node and the end of processing 
at the next node. 

We describe the functioning of the network in the transmission of packets 
as transitions of a~discrete-time MC (DTMC). The state space corresponds to the set 
of nodes such that 
the transmission of a~packet from a~node that has finished processing the packet 
to the next node corresponds to the transition of the chain to the next state.


Discrete-time MC is defined as a~tuple $D\linebreak =(S, s_0, P)$. In the tuple, $S$ is 
the finite set of states, $s_0\in S$ is the initial
state, $P:S \times S \rightarrow [0, 1]$ is the transition probability matrix in 
which $\forall s\in S$, $\sum\nolimits_{s' \in S} P(s,s') = 1$. 
For any two states $s, s' \in S$, if $P(s,s')>0$, then~$s'$ is the successor of~$s$. 
For a~subset of states $T \subseteq S$, the probability of moving from a~state~$s$ 
to any state $t \in T$ in a~single step is denoted by $P(s, T)$ and is given by 
$P(s,T)=\sum\nolimits_{t \in T} P(s, t)$. 
%The row $P(s,:)$, in the transition matrix $P$, contains the probabilities of moving from $s$ to its successors, while the column $P(:, s)$ contains the probabilities of entering the state $s$ from any other state.

\vspace*{-6pt}

\subsection{Verification syntax}

\noindent
For implementation of our PCTL-based model, we use PRISM~--- 
probabilistic model checker~\cite{kwiatkowska_prism_2011}. There, we follow 
PRISM property specification language. Here, we briefly describe the essential 
syntax while more details can be found in~\cite{noauthor_prism_nodate}.

Given a property~$\Psi$, we say that~$\Psi$ is true with probability~$p$ 
and write that as
$P_p [ \Psi ]$. If the probability~$p$ is unknown, PRISM allows, for DTMC, 
writing properties queries of the form $P_{=?}[ \Psi ]$, meaning 
``what is the probability that~$\Psi$ is true?''. Additionally, it is possible 
to use a~time bound and write properties queries such as 
$P_{=?}[F^{\leq T} \Psi]$, meaning ``what is the probability that~$\Psi$ 
is true after less than~$T$~steps?''. At last, it is possible to compute 
properties such as expected time or expected number of steps. 
For example, $R_{=?}[F \Psi]$, meaning ``what is the expected number of 
steps until $\Psi$ is true?''. 
%\section{Model Settings}
%, and a subroute of route $X$ from router $a$ to router $b$ is specified by $[a,b]_X$

%When a new subroute of $N$ that is scheduled to update a current sub-route of $C_i$ is launched, the route $C$ is updated such that the updated sub-route is replaced by launched sub-route, and the new sub-route is now part of the current route $C$.

\setcounter{figure}{3}
\begin{figure*}[b] %fig4
\vspace*{-6pt}
 \begin{center}
 \mbox{%
 \epsfxsize=149.177mm 
 \epsfbox{fre-3.eps}
 }
 \end{center}
\vspace*{-9pt}

 \Caption{New routes~$N_1$~(\textit{a}) and $N_2$~(\textit{b}) and
 MC states for~$N_1$~(\textit{c}) 
and~$N_2$~(\textit{d})}
 \label{fig:routes_dtmc_example}
\end{figure*}



\vspace*{-6pt}

\section{Prediction of Preferred Update}
%\section{Prediction of Preferred Update}
\label{sec:dtmc}

\noindent
The states of a~DTMC describe the nodes in the new route and the transition 
probabilities in the chain represent the possible delay or 
a~packet loss in the routers along the new route. The
states are defined as 
$\{s_1, \ldots , s_n\}$ where~$n$ is the number
  of nodes in the new route. 
The network achieves the state~$s_i$ if a packet has reached the $i$th node. 
For example, in Fig.~3, the self-transition 
edge represents the probability for a~delay due to packet loss, rules installation 
at the router, or congestion on the router-controller link, while the 
forward transition edge represents the probability for 
a~successful transition to the next state. These probabilities can be estimated 
from network statistics (see, for example,~\cite{hogan_stochastic_2017}). 
The labels on edges are the probability values, when edge has no label
 means probability~1.
 
 The initial probability distribution of states is given by the vector~$P_0$ of size~$n$. 
We can determine the prob-\linebreak\vspace*{-12pt}
 
 %\linebreak\vspace*{-12pt}

{ \begin{center}  %fig3
\vspace*{-0.5pt}
  \mbox{%
 \epsfxsize=77.518mm 
 \epsfbox{fre-4.eps}
 }


\end{center}

\vspace*{-3pt}

\noindent
{{\figurename~3}\ \ \small{Probability as a~function of number of steps to update routes~$N_1$~(\textit{1})
 and~$N_2$~(\textit{2})}}
}

\vspace*{12pt}



\noindent
ability that a~particular route delays the update process 
by~$k$, that is, the number of steps required for a~successful update is given by 
$p(k)=P_0 P^k$. Using this characteristic, which is, in fact, the 
probability distribution of the number of steps $P(k < x)$, one can 
calculate various properties like average delay time for the new route, 
maximum or minimum number of steps to update, etc.
 
 Consider the example illustrated in Fig.~4. 
Figure~4\textit{a} illustrates the current route~$C$ and a candidate new route~$N_1$. 
Figure~4\textit{b} shows the same current route~$C$ with another candidate 
new route~$N_2$. 
Figures~4\textit{c} and~4\textit{d} 
show the MCs for new routes~$N_1$ and~$N_2$, accordingly, with given transition 
probabilities.

During the update process, packets are sent along the current and the new routes. 
Since the new route is\linebreak\vspace*{-9.5pt}

\columnbreak

\noindent
 not operational yet, packets can be delayed due to 
congestion on certain nodes or due to switch configurations. 
%
For example, if routing rules have not yet been installed in some switch, then an 
arriving packet is sent to the controller~\cite{onf_openflow_2015}. The controller 
then decides reactively on further actions whether to install an appropriate rule 
for the packet. Also, the controller may be busy with other work and not respond 
immediately. Those packet processing actions may delay the update process. 
In the case buffer becomes full, for example, if the network is being congested, 
packets may be dropped. There, the transition to the next state during the 
update process depends on the likelihood of a~delay or a~loss of a~packet in the 
current state. 

In the example, the number of steps required for launching~$N_2$ is smaller than 
the number of steps required for launching~$N_1$. However, due to a higher likelihood 
of delays along the route~$N_2$, it is possible that~$N_1$ is preferred having 
a~higher probability for a~successful update. The network administrator may ask 
which new route is recommended for the update process, considering the expected 
number of steps required for the update. 
%
That is, updating paths requires the operator to decide 
on the possible choice of a~subroute for the next step. 
One should consider the possibility of including a~decision tool augmenting the 
controller during route updates. 

There were many attempts to use the LP/ILP 
approach, as it was already mentioned above (see, e.\,g.,~\cite{juttner_lagrange_2001}), 
but they have encountered the same difficulties, especially when taking 
into account online implementation. We show that it is possible to describe 
the routing process as DTMC. Thus, taking into consideration~$O(n^3)$ worst case 
computation complexity, we consider using the ``design via verification'' 
mentioned above based on PCTL verification, similar to the one used in 
PRISM~\cite{kwiatkowska_prism_2011}.


We have calculated the probability for a~successful update as a~function of 
number of steps for routes~$N_1$ and~$N_2$ from the example in 
Fig.~\ref{fig:routes_dtmc_example}. See Fig.~3 
where this function is shown. Curve~\textit{1}
represents the plot for~$N_1$ and curve~\textit{2} represents
 the plot for~$N_2$. 

Observe that after~20~steps, both new routes will be launched with probability~1 
which can be written as 
$$
P_{1}\left[F^{>20}N_1\right]=P_{1}\left[F^{>20}N_2\right]=1\,.
$$
The expected number of steps required for~$N_1$ is smaller than the required for~$N_2$:
$$
R \left[F~N_1\right] < R \left[F~N_2\right]\,.
$$
However, the probability for successfully updating in less than~15~steps 
is higher for route~$N_2$ ($0.55 \pm 0.040$ for~$N_1$ and 
$0.717 \pm 0.036$ for~$N_2$, based on~99\% confidence level):
$P_{0.717 \pm 0.036}\left[F^{\leq 15} N_2 \right].$

\vspace*{-6pt}


\section{Route Updates per~Quality~of~Service}
\label{sec:updates_qos}

\vspace*{-2pt}

\noindent
In this section, we show algorithm that we propose for various settings. 
First, we show an enhancement for the sequential update algorithm 
from~\cite{delaet_seamless_2015}, which during the update process decides on 
preferred subroute from the set of possible subroutes as part of QoS requirements. 
In the multicast-based update, several methods were proposed 
in~\cite{delaet_seamless_2015} for eliminating duplicated packets. 
In the case the common destination router is not able to immediately eliminate 
duplicated packets, the algorithm begins the update from the end, 
ensuring a~correct update process~[4].



\begin{algorithm*} %alg1
 \setlength{\algowidth}{100mm}
 \setlength{\hsize}{\algowidth}
 \caption{Update per QoS Algorithm}
 \label{alg:update_per_qos}

%\hrule
%\vspace*{2pt}
%\centerline
%{\textbf{Algorithm~1:} Update per QoS Algorithm}\par

%\vspace*{2pt}

%\hrule
 \small
 
 %\Input
 {directed graph $G$} 
 
 \BlankLine
 \tcc{$A$ is a collection of nodes} $A \leftarrow$ choose nodes from $G$ with in-degree $0$ \\
 
 \Repeat {out-degree of node $N^t_i > 0$}
 {
 \ForEach{$v \in A$ \label{alg:inner_loop}}
 {
 calculate $R[F~v]$ \\
% calculate the expected QoS for this node as described in Section \ref{sec:updates_qos} \\
 }\label{alg:end_inner_loop}
 
% $N^t_i \leftarrow$ choose the node that maximizes QoS \label{alg:choose_qos}\\ 
 $N^t_i \leftarrow \argmax_{v} (R[F~v])$ \label{alg:choose_qos} \\
 launch $N^t_i$ \\
 update $C$ accordingly \\
 merge any new and common subroutes as described in section~3 \\ 
 $A \leftarrow$ choose nodes neighboring to $N^t_i$ \\ 
 }
 
 \BlankLine 
 
\end{algorithm*}





 
%The algorithm starts from any node with in-degree 0 since it means that such node has no precedence dependence. Updating is completed when the algorithm arrives to a node with out-degree zero, which would be the last subroute to launch.


After that, we show an algorithm that chooses the subroutes for update arbitrary, 
assuming that the common destination node will not leak duplicated packets. 
However, the packets sending rate along the new subroute need to be temporarily limited~[4].

At last, we present a supplementing algorithm that suggests which subroutes can 
be updated in parallel.

%The set of common nodes for each pair of routes subdivides the routes to sub-routes relatively to each other. 

\vspace*{12pt}

\subsection{Sequential update}

\noindent
Let us begin the update from the end, namely, from the last alternative 
subroute of any new route. Provably, this prevents the formation of 
cycles~\cite{delaet_seamless_2015}. In order to represent all possible choices 
of a~path from a current state of the update process to the end of the update process, 
we propose to use a directed graph which nodes are the new, legitimate for launching, 
subroutes of the network. The edges of the graph represent a~legal order of launching 
new subroutes. Each path in this graph from a~current node to the last node in 
the path represents a~legal combination of chosen subroutes. The update process is 
continued as long as there is a~possible node to transition to. 

Let us examine the two possible new routes~$N_1$ and~$N_2$ that can replace the 
current route~$C$ from the example depicted in Fig.~1. 
The new route~$N_1$ is composed of~$N^1_1$ and~$N^2_1$, while the new route~$N_2$ 
composed of~$N^1_2$, $N^2_2$, $N^3_2$, and~$N^4_2$. Starting from the end, the only 
new subroutes that are allowable to launch are~$N^2_1$ and~$N^4_2$. 
Assume that based on the DTMC calculations performed as described in section~4, 
the subroute~$N^4_2$ is chosen for update. After the update of the subroute is 
complete, the current route~$C$ is composed of not updated yet part of the old 
route and~$N^4_2$. See Fig.~2 where the change in~$C$ 
is depicted.

After the subroute~$N^4_2$ is launched, we arrive at a~smaller problem in which 
less subroutes are left to update. Due to dynamics of the system 
(see section~3), some new subroutes can merge into a~single new subroute.
See Fig.~2 where after~$N^4_2$ was launched, the 
new subroutes~$N^3_2$ and~$N^2_2$ are merged into a~single subroute. Now, one 
can launch either~$N^1_1$ or~$N^2_2$ merged with~$N^3_2$. Assume that we choose to 
launch~$N^1_1$, which launch
 finishes the update. The route~$C$ updated to~$N^1_1$ 
and~$N^4_2$. See Fig.~5 illustrating that.


Figure~6 shows the directed graph that represents 
the possible update sequences. Initially, the subroutes that %\linebreak\vspace*{-12pt}
 are legal 
for launch are~$N^2_1$ and~$N^4_2$. As such, these are
the only subroutes that
 have in-degree~0. Launching~$N^3_2$
 is forbidden; hence, there is no node in the 
 graph~$G$ that represents this subroute. After launching~$N^4_2$, we\linebreak\vspace*{-12pt}
 
 \setcounter{figure}{4}

{ \begin{center}  %fig5
\vspace*{12pt}
 \mbox{%
 \epsfxsize=78.631mm 
 \epsfbox{fre-5.eps}
 }


\vspace*{3pt}


\noindent
{{\figurename~5}\ \ \small{$N^1_1$ was launched}}
\end{center}
}

\vspace*{6pt}

{ \begin{center}  %fig6
\vspace*{1pt}
 \mbox{%
 \epsfxsize=36.428mm 
 \epsfbox{fre-6.eps}
 }


\end{center}


\noindent
{{\figurename~6}\ \ \small{Graph 
representation for possible update paths for routes update example from Fig.~1}}

}

%\vspace*{6pt}

\noindent
  can 
 proceed by launching~$N^1_1$ or~$N^2_2$. However, if~$N^2_1$ was launched first, 
 it would be forbidden to launch~$N^2_2$ since it shares a~common edge with~$N^2_1$. 
 This is reflected in the graph~$G$ by not having a~directed edge from the
  node~$N^2_1$ to the node~$N^2_2$. We finish the update process
 by arriving either 
 to~$N^1_1$ or to~$N^1_2$. Notably, these nodes have out-degree~0.

 Algorithm~1 updates subroutes according to calculated QoS for each new subroute, by
 choosing at each step the new subroute that maximizes QoS.


The algorithm starts by selecting the initial set of subroute nodes. 
These are nodes with in-degree~0. The algorithm continues traversing the graph up 
to arrival at a node with out-degree~0 which would be the last subroute to launch. 
The inner loop at lines~\ref{alg:inner_loop}--\ref{alg:end_inner_loop} 
calculates the QoS for each neighboring node. Afterward, at 
line~\ref{alg:choose_qos}, the algorithm chooses the node that maximizes QoS. 
Then launches this node and updates the route~$C$, accordingly (see 
Figs.~1--5 for illustration). 
Afterward, the algorithm selects the next neighboring nodes.

After execution of Algorithm~1, the resulting new route maximally complies QoS 
requirements.

%\vspace*{12pt}

\subsection{Arbitrary subroutes selection} 
%\label{sec:arbitrary}

%\vspace*{-12pt}

\noindent
In this subsection, we assume that immediate duplicate packets elimination is possible. 
It may be that some of the subroutes are not ready for an update yet. 
Thus, meanwhile, the administrator may want to proceed with the update process 
to other subroutes or see possible variations of the update. 
For such scenario, we provide an algorithm which can select a~subroute for 
update arbitrary and continue the update process from there. 
We create a~forest graph of all possible update combinations from which the 
desired update sequence can be chosen. 
{\looseness=1

}
 


Figure~7 shows all possible combinations from example 
in Fig.~1. Noticeable, as mentioned earlier, some\linebreak\vspace*{-12pt}

{ \begin{center}  %fig7
\vspace*{1pt}
  \mbox{%
 \epsfxsize=71.694mm 
 \epsfbox{fre-7.eps}
 }


\end{center}


\noindent
{{\figurename~7}\ \ \small{Forest graph representing execution combinations for example from 
 Fig.~1}}
}

\vspace*{12pt}


\noindent
 combinations 
exhibit fewer steps, though possible that its QoS compliance is worse than others.



Algorithm~2 starts by iterating over all roots of the forest graph and 
calculating QoS using Algorithm~1 each tree. Afterward, launch the update 
of the tree that maximizes QoS.

\begin{algorithm*} %alg2
\setlength{\algowidth}{100mm}
 \setlength{\hsize}{\algowidth}
 \caption{Arbitrary Selection Update}
 \label{alg:arbitrary_update}
 \small
 
% \Input
{directed graph $G$} 
 
 %\BlankLine
 
 $A_0 \leftarrow$ choose nodes from $G$ with in-degree $0$ \\
 $Q \leftarrow \{\}$ \\
 
 \BlankLine
 \tcc{iterate over all roots of trees in the forest $G$}
 \ForEach{$v_r \in A_0$}
 {
 $q \leftarrow$ get the expected QoS using Algorithm~1 for $v_r$ \\
 $Q \leftarrow Q \cup \{q \rightarrow \mathrm{root} \}$ \\
 }

 \BlankLine
 $q_{\max} \leftarrow \max_{\mathrm{QoS}}(Q)$ \\
 launch maximizing QoS update order in $\mathrm{root}=Q[q_{\max}]$ \\ 
 
 
\end{algorithm*}


%\columnbreak

\vspace*{12pt}





\subsection{Parallel update}

\noindent
In certain cases, it is possible to update in parallel several subroutes 
and, as such, decrease update time. However, launching subroutes in parallel 
is not always possible
 since subroute may share a~link and, thus, leads to congestion 
during the update process, close a~cycle, or lead to an inconsistent state of the 
system. In~\cite{delaet_seamless_2015}, it was shown that two new subroutes~$N'$ 
from~$a$ to~$b$ and~$N''$ from~$c$ to~$d$ can be launched in parallel only if~$c$ 
succeeds~$b$ or~$a$ succeeds~$d$.



%\begin{proposition}
% Let $N'$ from $a$ to $b$ and $N''$ from $c$ to $d$ be two legitimate new subroutes. $N'$ and $N''$ can be launched in parallel only if $c$ succeeds $b$ or $a$ succeeds $d$.
%%Two subroutes that are each legitimate can be launched in parallel only if they share at most one common subroute.
%\end{proposition}
%\begin{proof}
% \textbf{Direction}: $\Rightarrow$ Let $N'$ from router $a$ to $b$ and $N''$ from router $c$ to $d$, be two new legitimate sub-routes. The only way for them to share more than one common sub-route is if $b$ succeeds $c$ on $C$. In such case, launching $N'$ will eliminate the part of $C$ from $c$ to $b$ with no proper connection from $b$ to $c$, which leaves the system in an inconsistent state. The same occurs if $N''$ is launched. \\
% \textbf{Direction}: $\Leftarrow$ Let $N'$ from router $a$ to $b$ and $N''$ from router $c$ to $d$, be two new sub-routes, not necessary part of the same new route, such that $b$ precedes $c$ or $b=c$. If $a$ precedes $b$, than $N'$ is legal for launching independently of $N''$. Similarly, if $c$ precedes $b$, than $N''$ is legal for launching independently of $N'$. Thus, since $N'$ can be launched independently from $N''$, they can be launched in parallel. Symmetric considerations lead to same result in case $a$ succeeds $d$.
% 
%\noindent Generalization to more than two sub-routes is trivial.
%\end{proof}



\begin{algorithm*}[b] %[t] %alg3
\setlength{\algowidth}{100mm}
 \setlength{\hsize}{\algowidth}
 \caption{Parallel Update}
 \label{alg:parallel_update}
 \small
 
 %\Input
 {weighted graph $G_S$} 
 
 \BlankLine
 
 \While{there are still current subroutes to update}
 {
 $A \leftarrow$ find maximum-weight independent set in $G_S$ \\
 
 \BlankLine 
 \tcc{do in parallel} 
 \ForEach{$N^t_i \in A$} 
 { 
 launch $N^t_i$ \\
 }
 }
 
 \vspace*{6pt}
 
\end{algorithm*}

We create a supplementary graph~$G_S$, in which nodes are the new legitimate 
for launching subroutes, and edges represent restrictions on parallel 
launching of subroutes. See Fig.~8 for illustration, 
depicting subroutes from example in Fig.~1 and their parallel 
restrictions. For example, $N^4_2$ and~$N^1_2$ can be launched in parallel since 
there is no edge connecting them.

Clearly, any independent set of subroutes from the supplementary 
graph contains subroutes that can be launched in parallel. 
This can be further enhanced by setting QoS calculated values as weights 
on nodes of the graph and finding the subroutes that can be launched 
in parallel by finding a~maximum-weight independent set of the graph~$G_S$. 
Since~$G_S$ has few
 number of nodes (several tens), it is possible to find 
the
 maximum-weight independent set even by enumerating
 all possible independent 
sets~\cite{wu_review_2015} and comparing their total weights.
{\looseness=-1



{ \begin{center}  %fig8
\vspace*{12pt}
  \mbox{%
 \epsfxsize=36.666mm 
 \epsfbox{fre-8.eps}
 }


\end{center}


\noindent
{{\figurename~8}\ \ \small{Supplementary graph of the example in 
 Fig.~1, showing which subroutes cannot be run in parallel}
}}

%\vspace*{12pt}



} 



Important, the parallel method should not be launched on its own. 
For example, assume that at the first iteration of Algorithm~3, 
the independent sets of nodes are~$A_1$ and~$A_2$. Let us assume that~$A_1$ complies 
better to QoS demands than~$A_2$ and, thus, $A_1$ will be selected. 
Also, let us assume that~$B_1$ is the next independent set in the graph 
if~$A_1$ was selected and~$B_2$ if~$A_2$ was selected. 
Also, let us assume that~$B_1$ is
the next independent set in the graph if~$A_1$ was selected and~$B_2$ if~$A_2$ 
was selected.
It is possible that due to the dynamics of the system (see section~3), 
we could obtain overall higher QoS results if we initially launched the 
subroutes from the sets~$A_2$ and~$B_2$ afterwards than from the sets~$A_1$ and~$B_1$.
 

Therefore, the graph that we create in this section for parallelization constraints 
is a~supplementary graph which must be used in conjunction with the graphs from 
previous sections. Optimal results will be obtained when used in conjunction with 
the forest graph from subsection~5.2.

It is also important to note that, in the worst case, when there are 
no disjoint subroutes, the parallel method is reduced to the sequential 
method thought with a higher running time.

\vspace*{-12pt} 


\section{Implementation}

\noindent
We implemented the update algorithms from~\cite{delaet_seamless_2015} as 
services for our QoS verification module. The update algorithm itself 
was not modified. In other words, we treated the update itself as 
an atomic action. The route updates
 algorithms are implemented as 
applications interacting with the northbound interface of an SDN controller. 
We used POX~\cite{kaur_network_2014} as a~platform for controller development and 
Mininet~\cite{lantz_network_2010} for network topology emulation. 
Figure~9 depicts the schematic arrangement of the 
functional elements. 



We created networks with topology of random graph and small-world features. 
During each simulation trial, a~pair of common source and destination nodes $(s,d)$ 
were selected. A~path connecting~$s$ and~$d$ was selected as a~current route and 
a~set of~4~new routes connecting $(s,d)$, to replace the current route, were 
selected, possibly with shared links among themselves and the current route. 

We considered latency due to the formed congestion as QoS demands for the update, 
implemented by forming congestion on randomly selected subroutes. Route 
update was executed by the update algorithm from~\cite{delaet_seamless_2015} for 
each pair of current and new routes. Further, one of the enhanced versions 
was executed, updating to the
 preferred combination of subroutes, by identifying 
the congested subroutes (e.\,g., by estimating latency).

{ \begin{center}  %fig9
\vspace*{8pt}
  \mbox{%
 \epsfxsize=58.544mm 
 \epsfbox{fre-9.eps}
 }

\vspace*{3pt}


\noindent
{{\figurename~9}\ \ \small{Description of the system}
}
\end{center}}

%\vspace*{12pt}



%\vspace*{-45pt}

\section{Concluding Remarks}

\noindent
The study in this paper illustrates a~feasibility of modeling and 
designing the route update process via verification using DTMC. The goal was to 
strengthen the network administrator involvement in management and decision 
making during route update. In the present model, the network administrator is able 
to consider network parameters such as packet losses, delay, communication 
rounds, flow table updates, congestion, and other inherent unreliabilities of 
the network. 

We extended the updating algorithm with the ability to compute QoS as the 
MC characteristics, where the MC corresponds to the states 
of the update process. Using this MC computation ability, it is 
possible to predict the expected number of steps (delay time) required to 
complete the update process. These prediction results allow the administrator 
to make a~decision whether a~new route can satisfy the user requirements per QoS 
or a~more reliable route will be selected.

We provided sequential update algorithm and an arbitrary order algorithm 
when for the later, it is assumed that immediate duplicate packets elimination 
is possible. Further, we suggest a supplementary graph and algorithm for launching 
updates in parallel when it is possible.

This paper proposes a~conceptual approach. In future research, we will focus 
on optimization of predictions supplementing the network administrator with 
a~powerful tool which will be able to enhance the update process 
with fine grained analysis of the network.

\vspace*{-12pt}


\Ack
\noindent
The first author has partially been supported by the 
Russian Foundation for Basic Research under grants RFBR 18-07-00669 and 18-29-03100. 
The second author has partially been supported by the Rita Altura Trust Chair in
Computer Sciences; The Lynne and William Frankel Center for Computer
Science.

%\bigskip


The authors thank Prof.\ Shlomi Dolev 
for his valuable input and Prof.\ Yefim Dinitz for his comments.
 
\renewcommand{\bibname}{\protect\rmfamily References}

%\vspace*{-6pt}

\vspace*{-6pt}

{\small\frenchspacing
{\baselineskip=10.35pt
\begin{thebibliography}{99}



\bibitem{rao_sdn_2014}  %1
\Aue{Rao, S.\,K.} 2014. SDN and its use-cases~--- NV and NFV:
A~state-of-the-art survey. NEC Technologies India Ltd. 25~p.

\bibitem{ghaznavi_service_2016}  %2
\Aue{Ghaznavi, M., N.~Shahriar, R.~Ahmed, and R.~Boutaba}. 2016. 
Service function chaining simplified. {arXiv.org}. arXiv:1601.00751.

\bibitem{hansson_logic_1994}  %3
\Aue{Hansson, H., and B.~Jonsson}. 
1994. A~logic for reasoning about time and reliability. 
\textit{Form. Asp. Comput.} 6(5):512--535.

\bibitem{delaet_seamless_2015}  %4
\Aue{Delaet, S., S.~Dolev, D.~Khankin, S.~Tzur-David, and T.~Godinger}. 
2015. Seamless SDN route updates. \textit{IEEE 14th Symposium (International)
on Network Computing and Applications}. IEEE. 120--125.

\bibitem{frenkel_predicting_2017} 
\Aue{Frenkel, S., D.~Khankin, and A.~Kutsyy}. 
2017. Predicting and choosing alternatives of route updates per QoS VNF in SDN. 
\textit{IEEE 16th Symposium (International) on Network Computing and Applications}. 
IEEE. 1--6. 

\bibitem{devi_approach_2015} 
\Aue{Devi, G., and S.~Upadhyaya}. 2015. 
An approach to distributed multi-path QoS routing. 
\textit{Indian J.~Sci. Technol.} 8(20):1--14. 
doi: 10.17485/ijst/2015/v8i20/49253.

\bibitem{egilmez_distributed_2012} 
\Aue{Egilmez, H.\,E., S.~Civanlar, and A.\,M.~Tekalp}. 2012. 
A~distributed QoS routing architecture for scalable video streaming over multi-domain 
OpenFlow networks. \textit{19th IEEE Conference (International) on Image Processing}.
IEEE. 2237--2240.

\bibitem{juttner_lagrange_2001} 
\Aue{Juttner, A., B.~Szviatovski, I.~Mecs, and Z.~Rajko}. 2001. 
Lagrange relaxation based method
for the QoS routing problem. \textit{IEEE Conference on Computer Communications. 
20th Annual Joint Conference of the IEEE Computer and Communications Society
 Proceedings}. IEEE. 2:859--868.

\bibitem{yu_efficient_2013} %9
\Aue{Yu, Z., F.~Ma, J.~Liu, B.~Hu, and Z.~Zhang}. 2013. 
An efficient approximate algorithm for disjoint QoS routing.
\textit{Math. Probl. Eng.} 2013:489149. 9~p. 
doi: 10.1155/2013/489149.

\bibitem{foerster_survey_2016} 
\Aue{Foerster, K.-T., S.~Schmid, and S.~Vissicchio} 2016. 
A~survey of consistent network updates. \mbox{Arxiv.org}. \mbox{arXiv}:\linebreak 1609.02305.

\bibitem{reitblatt_consistent_2011} 
\Aue{Reitblatt, M., N.~Foster, J.~Rexford, and D.~Walker}. 
2011. Consistent updates for software-defined networks: Change you can believe in! 
\textit{10th ACM Workshop on Hot Topics in Networks Proceedings}.
New York, NY: ACM. Art.\ No.\,7. doi: 10.1145/2070562.2070569.

\bibitem{hogan_stochastic_2017} 
\Aue{Hogan, M., and F.~Esposito}. 
2017. Stochastic delay forecasts for edge traffic engineering via Bayesian networks. 
\textit{IEEE 16th Symposium (International) on Network Computing and Applications}. 
IEEE. 1--4.

\bibitem{mcgeer_safe_2012} %15
\Aue{McGeer, R.} 2012. A~safe, efficient Update Protocol for Openflow Networks. 
\textit{1st Workshop on Hot Topics in Software Defined Networks Proceedings}. 
New York, NY: ACM. 12:61--66.
\bibitem{mcgeer_correct_2013} 
\Aue{McGeer, R.} 2013. A~correct, zero-overhead protocol for network updates. 
\textit{2nd ACM SIGCOMM Workshop on Hot Topics in Software Defined Networking
Proceedings}. New York, NY: ACM. 13:161--162.
\bibitem{katta_incremental_2013} 
\Aue{Katta, N.\,P., J.~Rexford, and D.~Walker}. 
2013. Incremental consistent updates. \textit{2nd ACM SIGCOMM Workshop on Hot Topics 
in Software Defined Networking Proceedings}.
New York, NY: ACM. 13:49--54.

\bibitem{dinitz_dependence_2017}  %16
\Aue{Dinitz, Y., S.~Dolev, and D.~Khankin}. 
2017. Dependence graph and master switch for seamless dependent routes 
replacement in SDN. \textit{IEEE 16th Symposium 
(International) on Network Computing and Applications}. IEEE. 1--7.

\bibitem{17-aaa}
\Aue{Amiri, S.\,A., S.~Dudycz, S.~Schmid, and S.~Wiederrecht}.
2016. Congestion-free rerouting of flows
on DAGs. \mbox{ArXiv}.org. arXiv:1611.09296.
% [cs, math], Nov. 2016, arXiv: 1611.09296. [Online]. Available:
%http://arxiv.org/abs/1611.09296

\bibitem{kwiatkowska_prism_2011}  %17
\Aue{Kwiatkowska, M., G.~Norman, and D.~Parker}. 2011. 
PRISM~4.0: Verification of probabilistic real-time systems. 
\textit{Computer aided verification}.
Eds. G.~Gopalakrishnan and S.~Qadeer.
Lecture notes in computer science ser. Springer.
6806:585--591.

\bibitem{noauthor_prism_nodate}  %18
\Aue{Kwiatkowska, M., G.~Norman, and D.~Parker}. 2018. 
{PRISM manual}. Available at:
{\sf http://www.\linebreak prismmodelchecker.org/manual/}
(accessed December~10, 2018).

\bibitem{onf_openflow_2015} %19
{Open Networking Foundation}. 2015. 
OpenFlow Switch Specification Ver~1.5.1. 


\bibitem{wu_review_2015}  %20
\Aue{Wu, Q., and J.-K.~Hao}. 2015. 
A~review on algorithms for maximum clique problems. 
\textit{Eur. J.~Oper. Res.} 242(3):693--709.

\bibitem{kaur_network_2014}  %21
\Aue{Kaur, S., J.~Singh, and N.\,S.~Ghumman}. 2014. 
Network programmability using POX controller. 
\textit{Conference (International) on Communication, Computing and Systems}.
138.

\bibitem{lantz_network_2010}  %22
\Aue{Lantz, B., B.~Heller, and N.~McKeown}. 2010. 
A~network in a~laptop: Rapid prototyping for software-defined networks. 
\textit{9th ACM SIGCOMM Workshop on Hot Topics in Networks Proceedings}. 
New York, NY: ACM.  Art.\ No.\,19. doi: 10.1145/1868447.1868466.
\end{thebibliography} } }

\end{multicols}

\vspace*{-9pt}

\hfill{\small\textit{Received October 9, 2018}}

\vspace*{-22pt}

\Contr

\vspace*{-3pt}

\noindent
\textbf{Frenkel Sergey L.} (b.\ 1951)~--- 
Candidate of Science (PhD) in technology, associate professor, 
senior scientist, Institute of Informatics Problems, Federal Research Center 
``Computer Sciences and Control'' of the Russian Academy of Sciences, 
44-2~Vavilov Str., Moscow 119333, Russian Federation; \mbox{fsergei51@gmail.com}

\vspace*{1pt}

\noindent
\textbf{Khankin D.} (b.\ 1983)~--- MSc, doctorate student, Department of Computer 
Science, Ben-Gurion University of the Negev, Beer-Sheva 84105, Israel; 
\mbox{danielkh@post.bgu.ac.il}

\vspace*{4pt}

\hrule

\vspace*{2pt}

\hrule

\vspace*{-7pt}

%\newpage

%\vspace*{-28pt}

\def\tit{НЕПРЕРЫВНЫЕ ОБНОВЛЕНИЯ МАРШРУТА В~SDN С~ИСПОЛЬЗОВАНИЕМ ПРОВЕРКИ СООТВЕТСТВИЯ 
КАЧЕСТВУ~ОБСЛУЖИВАНИЯ$^*$\\[-7pt]}

\def\titkol{Непрерывные обновления маршрута в~SDN с~использованием проверки соответствия 
качеству обслуживания}

\def\aut{С.\,Л.~Френкель$^1$, Д.~Ханкин$^2$\\[-7pt]}

\def\autkol{С.\,Л.~Френкель, Д.~Ханкин}

{\renewcommand{\thefootnote}{\fnsymbol{footnote}} \footnotetext[1]
{Работа была частично поддержана РФФИ (гранты 18-07~00669 и~18-29-03100), 
а~также Rita Altura Trust Chair in
Computer Sciences; The Lynne and William Frankel Center for Computer
Science.}}



\titel{\tit}{\aut}{\autkol}{\titkol}

\vspace*{-22pt}

\noindent
$^1$Институт проблем информатики Федерального исследовательского центра 
<<Информатика и~управление>>\linebreak
$\hphantom{^1}$Российской академии наук
%, fsergei51@gmail.com 

\noindent
$^2$Университет им.\ Бен-Гуриона в Негеве, Беэр-Шева, Израиль
%, danielkh@post.bgu.ac.il 

\vspace*{1pt}

\def\leftfootline{\small{\textbf{\thepage}
\hfill ИНФОРМАТИКА И ЕЁ ПРИМЕНЕНИЯ\ \ \ том\ 12\ \ \ выпуск\ 4\ \ \ 2018}
}%
 \def\rightfootline{\small{ИНФОРМАТИКА И ЕЁ ПРИМЕНЕНИЯ\ \ \ том\ 12\ \ \ выпуск\ 4\ \ \ 2018
\hfill \textbf{\thepage}}}

\vspace*{-1pt}


 
\Abst{В программно-определяемой сети (SDN~--- software-defined networking) 
уровень управ\-ле\-ния 
и~уровень данных разделены. Это обеспечивает высокую гибкость эксплуатации, 
предоставляя абстракции для управления сетью приложений 
и~возможность непосредственного программирования маршрутов.
Однако из-за изменений топологии, процедуры обслуживания или происходящих 
сбоев иногда необходима реконфигурация и~обновление сети. 
В~предлагаемом сценарии рассматривается текущий маршрут~$C$
и~набор возможных новых маршрутов~~$\{N_i\}$, где для замены текущего 
маршрута требуется 
один из\linebreak\vspace*{-12pt}}

\Abstend{новых маршрутов. Существует вероятность того, что новый маршрут~$N_i$ 
окажется длиннее некоторого другого нового маршрута~$N_j$, но при этом~$N_i$ 
будет более надежным и~он будет обновляться быстрее или работать лучше 
после обновления с~точки зрения требований качества обслуживания (QoS~---
quality of service). Принимая 
во внимание случайный характер функционирования сети, авторы дополнили недавно 
предложенный алгоритм обновления маршрута Delaet с~соавт.\ методом оценки соблюдения 
требований QoS во время непрерывного обновления маршрута, основанным на 
использовании цепей Маркова. При этом, во-пер\-вых, предлагается расширить 
алгоритм передачи пакетов по выбранному маршруту, сравнивая процесс обновления 
для возможных альтернатив маршрута. Во-вто\-рых, предлагается несколько 
способов выбора комбинаций предпочтительных отрезков путей новых маршрутов, 
что приводит к оптимальному в~смысле соответствия QoS маршруту.}


\KW{программно-определяемые сети; цепи Маркова; качество обслуживания}

\DOI{10.14357/19922264180408}



%\vspace*{-3pt}


 \begin{multicols}{2}

\renewcommand{\bibname}{\protect\rmfamily Литература}
%\renewcommand{\bibname}{\large\protect\rm References}

{\small\frenchspacing
{\baselineskip=10.5pt
\begin{thebibliography}{99}
%\vspace*{-3pt}


\bibitem{2-fr-1}
\Au{Rao S.\,K.} SDN and its use-cases~--- NV and NFV: A~state-of-the-art survey.~--- 
NEC Technologies India Ltd., 2014. 25~p.
\bibitem{3-fr-1}
\Au{Ghaznavi M., Shahriar~N., Ahmed~R., Boutaba~R.} 
Service function chaining simplified~// Arxiv.org, 2016. \mbox{arXiv}:1601.00751cs.
\bibitem{4-fr-1}
\Au{Hansson H., Jonsson~B.} A~logic for reasoning about time and reliability~// 
Form. Asp. Comput., 1994. Vol.~6. No.\,5. P.~512--535.

\bibitem{1-fr-1} %4
\Au{Delaet S., Dolev~S., Khankin~D., Tzur-David~S., Godinger~T.}
Seamless SDN route updates~// IEEE 14th Symposium (International)
 on Network Computing and Applications.~--- IEEE, 2015. P.~120--125.
 
 
\bibitem{5-fr-1}
\Au{Frenkel S., Khankin D., Kutsyy~A.} Predicting and choosing alternatives 
of route updates per QoS VNF in SDN~// IEEE 16th Symposium (International)
on Network Computing and Applications.~--- IEEE, 2017. P.~1--6.
\bibitem{6-fr-1}
\Au{Devi G., Upadhyaya~S.} An approach to distributed multi-path QoS routing~// 
Indian J.~Sci. Technol., 2015. Vol.~8. Iss.~20. P.~1--14. 
doi: 10.17485/ijst/2015/v8i20/49253.
\bibitem{7-fr-1}
\Au{Egilmez H.\,E., Civanlar S., Tekalp~A.\,M.} 
A~distributed QoS routing architecture for scalable video streaming over multi-domain 
OpenFlow networks~// 19th IEEE Conference (International)
on Image Processing.~--- IEEE, 2012. P.~2237--2240.
\bibitem{8-fr-1}
\Au{Juttner A., Szviatovski B., Mecs~I., Rajko~Z.}
Lagrange relaxation based method for the QoS routing problem~// 
IEEE INFOCOM 2001 Conference on Computer Communications. 20th 
Annual Joint Conference of the IEEE Computer and Communications Society
Proceedings.~--- IEEE, 2001. Vol.~2. P.~859--868.
\bibitem{9-fr-1}
\Au{Yu Z., Ma F., Liu~J., Hu~B., Zhang~Z.}
An efficient approximate algorithm for disjoint QoS routing~// 
Math. Probl. Eng., 2013. Vol.~2013. Art.\ No.\,489149. 9~p. 
doi: 10.1155/2013/489149.
\bibitem{10-fr-1}
\Au{Foerster K.-T., Schmid S., Vissicchio~S.}
A~survey of consistent network updates~// Arxiv.org, 2016. arXiv:1609.02305.
\bibitem{11-fr-1}
\Au{Reitblatt M., Foster N., Rexford J., Walker~D.} 
Consistent updates for software-defined networks: Change you can believe in!~// 
10th ACM Workshop on Hot Topics in Networks Proceedings.~--- New York, NY, USA: ACM, 
2011. Art.\ No.\,7. doi: 10.1145/2070562.2070569.
\bibitem{12-fr-1}
\Au{Hogan M., Esposito F.} Stochastic delay forecasts for edge traffic engineering 
via Bayesian Networks~// IEEE 16th Symposium (International)
on Network Computing and Applications.~--- IEEE, 2017. P.~1--4.
\bibitem{13-fr-1}
\Au{McGeer R.} A~safe, efficient Update Protocol for Openflow Networks~// 
1st Workshop on Hot Topics in Software Defined Networks Proceedings.~--- 
New York, NY, USA: ACM, 2012. Vol.~12. P.~61--66.
\bibitem{14-fr-1}
\Au{McGeer R.} 2013. A~correct, zero-overhead protocol for network updates~// 
2nd Workshop on Hot Topics in Software Defined Networking Proceedings.~--- 
New York, NY, USA: ACM, 2013. Vol.~13. P.~161--162.
\bibitem{15-fr-1}
\Au{Katta N.\,P., Rexford J., Walker~D.} Incremental consistent updates~// 
2nd Workshop on Hot Topics in Software Defined Networking Proceedings.~--- 
New York, NY, USA: ACM, 2013. Vol.~13. P.~49--54.
\bibitem{16-fr-1}
\Au{Dinitz Y., Dolev S., Khankin~D.}
 Dependence graph and master switch for seamless dependent 
 routes replacement in SDN~// IEEE 16th Symposium 
 (International) on Network Computing and Applications.~--- IEEE, 2017. P.~1--7.
 \bibitem{17-aaa-1}
\Au{Amiri~S.\,A., Dudycz~S., Schmid~S., Wiederrecht~S}.
 Congestion-free rerouting of flows
on DAGs~// ArXiv.org, 2016. arXiv:1611.09296.
% [cs, math], Nov. 2016, arXiv: 1611.09296. [Online]. Available:
%http://arxiv.org/abs/1611.09296

\bibitem{17-fr-1}
\Au{Kwiatkowska M., Norman~G., Parker~D.}
 PRISM~4.0: Verification of probabilistic real-time systems~//
 Computer aided verification~/
 Eds. G.~Gopalakrishnan, S.~Qadeer.~---
Lecture notes in computer science ser.~--- Springer, 2011. 
 Vol.~6806. P.~585--591.
\bibitem{18-fr-1}
\Au{Kwiatkowska M., Norman G., Parker~D.}
 PRISM manual, 2018. 
{\sf http://www.prismmodelchecker.org/manual}.
\bibitem{19-fr-1}
Open Networking Foundation. OpenFlow Switch Specification Ver~1.5.1, 2015. 

\bibitem{21-fr-1}
\Au{Wu Q., Hao J.-K.} A~review on algorithms for maximum clique problems~// 
Eur. J.~Oper. Res., 2015. Vol.~242. No.\,3. P.~693--709.

\bibitem{20-fr-1}
\Au{Kaur S., Singh J., Ghumman~N.\,S.}
 Network programmability using POX controller~// Conference
 (International) on Communication, Computing and Systems, 2014. P.~138.
\bibitem{22-fr-1}
\Au{Lantz B., Heller B., McKeown~N.} 
A~network in a~laptop: Rapid prototyping for software-defined networks~// 
9th ACM SIGCOMM Workshop on Hot Topics in Networks Proceedings.~--- 
New York, NY, USA: ACM, 2010. Art.\ No.\,19. doi: 10.1145/1868447.1868466.
\end{thebibliography}
} }

\end{multicols}

 \label{end\stat}

 \vspace*{-9pt}

\hfill{\small\textit{Поступила в~редакцию 09.10.2018}}


%\renewcommand{\bibname}{\protect\rm Литература}
\renewcommand{\figurename}{\protect\bf Рис.}
\renewcommand{\tablename}{\protect\bf Таблица} %1
%\include{frcorr} %1
\renewcommand{\figurename}{\protect\bf Figure}
\renewcommand{\tablename}{\protect\bf Table}
\renewcommand{\bibname}{\protect\rmfamily References}

\def\stat{fren}

\def\u{\overline u}

\def\tit{ESTIMATION OF SELF-HEALING TIME FOR DIGITAL
SYSTEMS UNDER TRANSIENT FAULTS$^*$}

\def\titkol{Estimation of self-healing time for digital
systems under transient faults}

\def\autkol{S.\,L.~Frenkel and A.\,V.~Pechinkin}
\def\aut{S.\,L.~Frenkel$^1$ and A.\,V.~Pechinkin$^2$}

\titel{\tit}{\aut}{\autkol}{\titkol}

{\renewcommand{\thefootnote}{\fnsymbol{footnote}}\footnotetext[1]
{The
work was performed within the Russian Academy of
Sciences Presidium basic research program~3
``Fundamental problems of system programming.''}}

\renewcommand{\thefootnote}{\arabic{footnote}}
\footnotetext[1]{Institute
of Informatics Problems of the Russian Academy of Sciences,
Slf-ipiran@mtu-net.ru}
\footnotetext[2]{Institute
of Informatics Problems of the Russian Academy of Sciences,
apechinkin@ipiran.ru}

\vspace*{8pt}

\Abste{This paper suggests a new approach to self-healing property  prediction for digital systems.  
Self-healing refers to the system's ability to continue operating properly in the case of the 
failure of some of its components. This phenomenon is very considerable aspect of high-reliable 
systems design. The self-healing time characteristics are analyzed during design process, and 
the computation of probability distribution function of self-healing time needs for fair prediction 
of real time systems reliability. 
This paper considers the possible ways of estimation of time to self-healing under transient faults using 
a Markov model of  a design behavior for reliability analysis of digital systems with some fault-tolerant 
properties, modeled by the well-known Finite State Machine formalism. 
}

\vspace*{4pt}

\KWE{fault-tolerant computer; self-healing
fault-tolerance; transient faults; finite state machine; Markov chains}

\vspace*{8pt}

       \vskip 14pt plus 9pt minus 6pt

      \thispagestyle{headings}

      \begin{multicols}{2}

      \label{st\stat}



\section{Introduction}

\noindent
Designers of nanotechnology-based digital systems
for safety-critical applications need to take into
account that the systems can be affected by radiation-induced
faults, e.\,g,\ Single Event Upsets
(SEU)~[1].
Thus, computing systems for safety-critical
applications must be \textit{fault-tolerant} to be able to
continue properly functioning despite these transient
failures of their hardware or software~[2--4].

{\it Self-healing} refers to the system's ability to continue
operating properly on the event of the failure of some of its
components, that is, as its ability to maintain operation over a
period of time~$t$. In other words, a system is ``self-healing'' if
it is capable to recover from errors while remaining valid
system outputs. Moreover, systems designed to be self-healing must be
able to heal themselves at runtime in response to changing
environmental or operational status, in order to fix errors
in its components without any external interaction~\cite{4fr}.

In spite of plethora of the self-healing specific mechanisms, one
of the most usable model of the systems of interest is the FSM.
Therefore,  the self-healing in automata models of
systems used at early design stages, is considered. For example, a concept of
a partially monotonic FSM, where the transitions are computed by
partially monotonic Boolean functions, is used to provide
self-healing properties~\cite{6fr}. In particular, if  a
self-checking digital circuit design is considered, the different properties of
logical functions may provide self-healing properties of the
circuit~\cite{6fr}. The architecture that supports the self-healing
property of the FSM is a well-known self-checking architecture~\cite{6fr} that uses output of self-checking checker.
Thus, further,  only FSM representations of digital
systems will be considered.

Since, in general, both input data and faults appear
randomly, the time before a system healing is some
random values.

This paper describes a probabilistic approach to
analysis of self-healing properties of FSM modeled
systems under some transient faults.
The model of system that allows to compare correct
behavior with erroneous one is a Markov chain (MC)
defined on direct product of state spaces of two FSMs,
describing behavior of both fault-free and faulty
design~\cite{7fr, 8fr}.
The goal of this modeling is to estimate the time
to return to the correct behavior after hitting in
some erroneous one.

%2
\section{Some Concepts of Fault-Tolerant Design}

\noindent
Let us outline some principal definitions of the
fault-tolerant systems research area.
A {\it fault} is a physical cause of incorrect behavior,
e.\,g., a defect in a memory cell.
The most popular fault model in the area of digital
system testing is so-called stuck-at-zero (a variable~$u$
has constant logical zero which designated as
$u\equiv 0$), and stuck-at-one (correspondingly,
$u\equiv 1$)~\cite{2fr}.

The faults may be both \textit{permanent} (that remain in
existence indefinitely if no corrective action is
taken) and \textit{transient} ones (that appear and disappear
quickly).

An \textit{error} is an undesired state or condition in
a component of a target system, understood as a
discrepancy between an observed or measured value
or condition and a specified theoretically correct
value or condition.
Error is a fault consequence.
Faults may or may not cause one or few errors.
Errors induce \textit{failures}.
A \textit{failure} occurs when a system is no longer able to
satisfy its specification, e.\,g., an incorrect word
is formed on its output.

Correspondingly, one should differ between
manifestation latency for the faults and the
errors.
It is considered that designers have an \textit{oracle} that check
the correctness of its output (result of
computations).

The aim of a fault-tolerant design is to avoid
manifestation of the fault/error at the system
designed output in order to prevent the failure
behavior.
Note that in accordance with~\cite{6fr}, when a transient
fault occurs, a system may transit from a fault-free
behavior (``mode'') either to the erroneous mode
(the output is erroneous one) or in a mode where
the faulty behavior will be ``silent,'' that is, an
inner state will be incorrect while its output stays
in a correct mode.
If the system is able to return from the fault-free
mode after its functioning in the silent mode, this
is, obviously, the self-healing.
In other words, in this case, future of a system
(either it will ``recover'' or ``die'') depends of its
behavior during several clocks after moving to the
silent mode.
This parameter of number of clocks looks promising
and motivating for the self-healing characterization.

%3
\section{Some Architectures Which Enable Self-Healing}

\noindent
Let us outline some architectures of digital systems that can
provide their self-healing under some transient faults. There are
two main types of the faults that are being considered, the SEU
and the Single Event Transient (SET). The SEU occurs when radiation affects the transistors, for example,
that are part of the look up table logic of the FPGA (Field-Programmable Gate Array). Radiation
can hit areas on the device (FPGA, in particular) and causes an
incorrect bit value at the input or output of some blocks. The
SET affects current processing of data in the
circuit. Some authors remark that the SET is
less damaging than the SEU mainly because it does
not affect the hardware makeup of the current FPGA design~\cite{5fr}.
A~transient fault induced by the SEU may or may not be latched by a
storage cell, but in case of the fault occurrence, a correct
operation of the corrupted module can be restored and the current
state of the circuit can be reset. It can be achieved either due
to the monotonic properties of Boolean function describing the
transitions of FSM representing the given system~\cite{5fr} or a reconfiguration~\cite{2fr}.
Recall that a system of logical functions~$\psi$
is partially monotonic in~$u'$ input variables if for any pair of
Boolean $m$-tuples $A$, $B$, the condition $\psi(A)\le \psi(B)$ is
satisfied for $A\le B$~\cite{6fr}.

Another example of a system with self-healing, based on some
reconfigurations of a system under some errors provoked by a
fault has been considered in~\cite{5fr}. As it mentioned above, this
reconfiguration architecture functioning can be also described as
a FSM~\cite{10fr}. Partial Reconfiguration and Triple Module Redundancy (TMR)
are used together to create a true self-healing system design~\cite{5fr}.
In order to do this, the full TMR implementation must be extended
to provide additional status outputs to the partial
reconfiguration controller.

%4
\section{Problem Definition and Model Description}

\noindent
Let $A$ be an automaton (FSM) modeling an electronic
design implemented as a device (e.\,g., FPGA).
It is considered that this design should operate in the
presence of some radiation which hits at an physical
area of the device causing an incorrect bit
value to the input or output of   its modules
but does not affect immediately any primary outputs.

Let us assume that a transient fault during the clock~$t_0$ has
changed a next state~$x_0$, where fault-free FSM must
transit under an input vector~$\vec u$, for the state~$x_1$.
We say that in this moment,  the change of a
{\it trajectory} of the FSM transitions, corresponding
to its transition table, took a place.
All future transitions will be performed according to
the FSM transition functions (in accordance with
Table~1 in our example).
If at one of the steps after this clock ($t_0$) the
primary output is changed relatively to the
fault-free outputs and given transitions have still
not come back to the fault-free trajectory, then
either alarm or correction mechanism will be started,
and the FSM will be stopped or moved to a
restoration mode.
There is no any self-healing in this case.



Otherwise, if the FSM under the transient fault
will return to the normal (fault-free) trajectory
before the output is changed, it means that the self-healing took place (at least, before the next
external induced transient fault).

It is considered that a self-healing system can
recover from some transient faults within a finite
time and can provide that no further faults occur before
the system is healing again.
On the other hand, systems that are not self-healing
might function ate in incorrect states arbitrary long time, even
if no further faults occur.

\pagebreak

\noindent
\begin{center} %tabl1
%\vspace*{4pt}
{{\tablename~1}\ \ \small{Transition functions of FSM}}
\vspace*{2ex}

{\small
\tabcolsep=9pt
\begin{tabular}{cccc}
\hline
 $a_t$  & $a_s$  & $\vec{u}(a_t,a_s)$ & $\vec v(a_t,a_s)$     \\
\hline
 $a_1$  &  $a_1$  & $\u_1$        & $v_2v_4$         \\
%\hline
            & $a_1$  & $u_1\u_2$     & $v_2v_3$         \\
%\hline
            &  $a_2$  & $u_1u_2$     & $v_2v_4$         \\
%\hline
\hline
 $a_2$  & $a_1$  & $\u_2$        & $v_2v_4$         \\
%\hline
            &  $a_1$  &\ $u_2\u_3$     & $v_1v_4$         \\
%\hline
            &  $a_2$  & $u_1u_3$     & $v_2v_4$         \\
\hline
\end{tabular}
}
\end{center}

\bigskip


The authors' goal is to  express  the  probability
distribution function of the  time which can be spent
to  self-healing  under transient faults  using the
Markov chains (FSMs) product model~\cite{7fr, 8fr}.
Let us consider some examples of the self-healing
phenomenon.

Let us describe an FSM in Table~1 where columns~$a_t$
and~$a_s$  are the current and the next states;
$\vec u$ and $\vec v$ are the input and output signals in cubic
form, that is, it has free components with all
possible combinations of ``0'' and ``1''
(in the 3-input FSM in Table~1, the input variables
absent in the corresponding cells may take any
values).


Let the initial state of the FSM be~$a_1$, and let an
input sequence be $u_1u_2\to u_1u_2u_3$, which generates
the transitions $a_1\to a_2(v_2v_4)\to a_2(v_2v_4)$.
Let the variable~$u_2$ was corrupted by an external
fault effect (e.\,g., SEU) so that the sequence
$u_1\u_2\to u_1u_2u_3$ appeared instead of previous
one, which generate wrong transition
$a_1\to a_1(v_2v_3)\to a_2(v_2v_4)$, that is, the FSM
transmits to the state~$a_2$ instead~$a_1$.
However, because of discrepancy of the outputs in
two cases (that is, the FSM either under the transient
error or not), it will be  either alarm or correction
mechanism  started, and the FSM will be either
stopped or moved in a restoration mode, that is, the
self-healing does not take place.
Now, let the input sequence be
$\u_1u_2\to \u_1u_2u_3\to u_1u_2u_3$ generating the
transitions
$a_1\to a_1(v_2v_4)\to a_1(v_2v_4)\to a_2(v_2v_4)$,
which is corrupted as
$u_1u_2\to \u_1u_2u_3\to u_1u_2u_3$ with
transitions
$a_1\to a_2(v_2v_4)\to a_2(v_2v_4)\to a_2(v_2v_4)$.
Since the outputs are equal, the FSM under this
transient fault comes to the right state~$a_2$
if the external fault effect be disappeared to
the next clock.

%5
\section{Self-Healing Model Based on~Finite State Machine}

\noindent
Let an FSM, subject to external radiation mentioned
above during a clock (e.\,g., hit of some radioactive
particles), changes its correct trajectory to an
incorrect one, but it returns to the correct trajectory
after a number of clocks when the radiation effect is
stopped.
Following~\cite{7fr}, this phenomenon will be modeled by an
MC defined on the transition space of
two FSMs product: one of these FSM is the original one
(called as ``fault-free''), but another (called as
``faulty'') has the transition table of original
FSM but its trajectory is corrupted during the
clock-under-noise.

Let FSM be a Mealy machine, with the state set
$S=$\linebreak $=\{a_1,\ldots,a_n\}$,
the input set $U=\{\vec u\}=\{u_1,\ldots, u_m\}$
and the output set $V=\{\vec v\}=\{v_1,\ldots, v_k\}$.
Functions~$\delta$ and~$\lambda$ are the multiple-output Boolean
functions which are a relation between the ({\it input state,\,
present state})\linebreak pairs and the next states ($\delta$),
and between the (\textit{input state, present state}) pairs
and the output states ($\lambda$).
Let the input words of the FSM be a randomly generated
input vectors.
Obviously that the probability of the self-healing property
fulfillment depends on the distribution of input vectors
(for example, see Table~1).
Let us consider self-healing time
computation by the FSM product model.

Let $\{X_t,\ \ t\ge0\}$ and $\{Y_t,\ \  t\ge0\}$ are the MCs
describing the target behavior of a fault-free system  (in
accordance with the previous section), that is, functioning without
effect of any transient faults ($X_t$), and in the presence of
some faults ($Y_t$),\ \ $S$ is the set of states of the MCs.

Let $Z_t=\{(X_t,Y_t), \ t\ge0\}$ be an MC corresponding to
behavior of the MCs pairs that is an MC with space
$S^2=S\times S$ of pairs $(a_i,a_j)$, $a_i,a_j\in S$.

Let the states~$S$ are integers.
Fixing  the states numeration in  some way,
for example, as $s_1=(a_1,a_1)$,
where the ``ones'' are the numbers of states of
$X_t,Y_t$,  $s_2=(a_1,a_2),\ldots,$  $s_n=(a_1,a_n)$,
$s_{n+1}=(a_2,a_1)$, $s_{n^2}=(a_n,a_n)$,
the vector of the state transition probabilities of
the MC after $t$ steps may be defined as
$$
\vec p^{\,*}(t)=(p_{1}^*(t),\ldots,p_{n^2}^*(t))\,,
\enskip t\ge 0\,,
$$
where the state probability vector of the MC~$Z_t$
after $t$~clocks.
 This vector can be computed using the state
probabilities of~$X_t$ and~$Y_t$,
\begin{align*}
\vec p(t)&=(p_{1}(t),\ldots,p_{n}(t))\,,
\quad  t\ge 0\,;
\\
\vec p^{\,F}(t)&=(p_{1}^F(t),\ldots,p_{n}^F(t))\,,
\quad t\ge 0\,,
\end{align*}
with initial  states probabilities of the MCs,
where index~``$F$'' corresponds to the  FSM under
a transient fault (``faulty'' FSM):
$$
p_{i}(0)
=
\begin{cases}
1\,,      &a_i=X_0\,,   \\
0\,,      &a_i\ne X_0\,,
\end{cases}
\ \ i=1,\ldots ,n\,;
$$
$$
p^F_{j}(0)
=
\begin{cases}
1\,,      &a_j=Y_0\,,   \\
0\,,      &a_j\ne Y_0\,,
\end{cases}
\ \ j=1,\ldots ,n\,.
$$
Then,  the event of coincidence of the
trajectories of~$X_t$ and~$Y_t$ (self-healing)
may be denoted as hit
of the MC~$Z_t$ in a subset $D \subseteq S^2$ of all
states $(a_i,a_i)$,\ \ $i=1,\ldots,n$, with the same
right and left elements.

In general, the self-healing time  can be expressed as
\begin{equation}
T =
\min\{t:\ X_t=Y_t\}=\min\{t:\ Z_t\in D\}\,.
\label{e1fr}
\end{equation}
Let us characterize the self-healing time ~$T$ by
a functional~$Q(T)$ of the time.
It can be interpreted as a speed of returning to
a correct mode.

Equation~(\ref{e1fr}) can be rewritten in different ways in
dependence of the functional that will be used for the time to healing characterization (and
it can be interpreted as a ``loss function'').

Most simple type of the~$Q_1(t)$ is the mean time
to healing, that is:
$$
Q = Q_1 = {\mathbf E}(T)\,.
$$

More interesting measure is the probability~$Q_2(t_0)$
that time of returning of the FSM to its correct
trajectory
after fault effect disappearance  is more than~$t_0$.
Let us define:
$$
Q = Q_2 = Q_2(t_0) = {\mathbf Pr}\{T\ge t_0\}\,.
$$
Note that relationship between~$Q_1$ and~$Q_2$ is
$$
Q_1 =
\sum_{t_0=1}^\infty {\mathbf Pr}\{T\ge t_0\}
=
\sum_{t_0=1}^\infty Q_2(t_0)\,.
$$

Both these times to self-healing measures are computed
under assumptions that the initial states~$X_0$ and~$Y_0$ of both MCs are known.
Therefore, the function~$Q_2$ can be expressed as
$$
Q_2
=
1 - {\mathbf Pr}\{Z_{t_0-1}\in D\}
=
1 - \sum_{z\in D} p_{z}^*(t_0-1)\,.
$$
Also, it can be easily shown   that
$$
Q_2
\ge
\max\limits_V
\left|{\mathbf Pr}\{Y_{t_0-1}\in V\}
-
{\mathbf Pr}\{X_{t_0-1}\in V\}\right|
$$
where $V\subseteq S$.

Note that the functionals~$Q_1$ and $Q_2$ have been built
without explicit consideration of equality of  output
values of both FSMs.
Below,  more general case of FSM under
transient fault self-healing characterization will be considered, namely,
the probability distribution function of the time to
return to the FSM fault-free behavior before its output
mismatching with its copy, which functionates without any
transient faults.

%6
\section{Computation of   Self-Healing Time Probability
Distribution Function}

\noindent
Below, the time to self-healing probability
distribution function will be computed as a distribution of  number
of steps  to absorbing state of~$Z_t$, corresponding
to the event ``the automaton under transient fault
has returned to the fault-free trajectory before
its output mismatching.''

Let us define all specific states of the MC~$Z_t$ needed
to find out accurately the  self-healing time. Again, let  the
states set of both MCs $S_1$ and~$S_2$,\  $|S_1|=|S_2|= n$, are the
same and their initial states~$X_0 $ and~$Y_0$,\  $Y_0 \ne X_0$,
are known. Let us define the set of $Z_t$ states as $S^* =
\{(i,j),\ \ i,j = 1,\ldots,n,\ \ j \ne i\} \cup A_0\cup A_1$
where:
\begin{itemize}
\item
$(i,j)$ means that fault-free automaton is in the state~$i$,
but the automaton under a transient fault is in the state~$j$,
$j\ne i$;
\item
$A_0$ is an absorbing state of~$Z_t$, corresponding
to the event ``the automaton under transient fault has returned
to the fault-free trajectory before its output would be corrupted
by the transient fault effect;'' and
\item
$A_1$ is another absorbing state corresponding to the event
``faulty output appeared  before the trajectory restoration.''
\end{itemize}

The number of the~$Z_t$ states is $n(n-1)+2$.

Let us consider the way of MC~$Z_t$, transition
probability matrix~$P^*$ computation.
Let~$Z_t$  be in the state $(i_1,j_1)$ when the input signal
$\vec u$ is applied which transfers the fault-free automat
from the state~$i_1$ to the state~$i_2$, whereas the
automaton under a transient fault
will transfer to the state~$j_2$, and the output vectors
of the automata are~$\vec v_0$ and~$\vec v_1$, correspondingly.
Then the following situations are possible:
\begin{itemize}
\item $Z_t$ is in the state $A_1$ if $\vec v_0 \ne \vec v_1$;
\item
$Z_t$ is in the state $A_0$ if $i_2=j_2$ and $\vec v_0 =\vec v_1$; and
\item
$Z_t$ transits to the state $(i_2,j_2)$ if the outputs
coincide ($\vec v_0=\vec v_1$), but the states do not ($i_2\ne j_2$).
\end{itemize}

The transition probability matrix of the MC~$Z_t$ is computed by the known probabilities of input
vectors and transition tables of the FSM~\cite{7fr}, considering
the signals with the same input probabilities
distributions $\mathrm{Pr}\{u_i=1\}$, $i=1,\ldots,m$.

Let us denote:
\begin{itemize}
\item $p^*_{(1)}(t)$ is the probability that a faulty output behavior
of the automaton can appear right up to the $t$th step before the
automaton behavior will be restored;
\item
$p^*_{(0)}(t)$ is the probability that a faulty behavior of the
automaton can appear right up to the $t$th step and the output
both automata will coincide by this moment; and
\item
$p^*_{i,j}(t)$ is the probability that the MC $Z_t$ will get
in the state $(i,j)$\  ($j \ne i$) thereby by the moment~$t$, but the outputs will coincide.
\end{itemize}

Then, let us express the probability that~$Z_t$ will get
in the state~$(i,j)$.

Initial state $\vec p^{\,*}(0)$ is determined by the distribution
of initial states of both automata.
If the fault-free automaton is at the initial moment 0 in the
state $i_0=X_0$,whereas the faulty automaton will be in state
$j_0=Y_0$,\  $j_0 \ne i_0$, then $p^*_{i_0,j_0}(0) = 1$
and other components of the vectors are zero.

Matrix~$P^*$ is computed using input probabilities of Boolean
functions $u_l=u_l(a_t,a_s)$, describing  transitions from
the current state~$a_t$ to the next state~$a_s$, takes the
Boolean value.
This computation  is performed under assumption that input
variables of the FSM are independent random vectors from one
time step to another and that their probability distribution
is fixed so that, at any given time step, an input takes
place with probability $\mathrm{Pr}\{u_i=1\}=p_k$,\  $i=1,\ldots,m$.
For example, the FSM of Table~1 stays in the state~$a_2$ with
probability $p_1 p_3$ since this is a Boolean
conjunction of the Boolean variables~$u_1$ and~$u_3$.

The distribution $Z_t$ in the moment~$t$
\begin{equation}
\vec p^{\,*}(t)
=
\vec p^{\,*}(t-1) P^* = \vec p^{\,*}(0)(P^*)^t             \label{e2fr}
\end{equation}
and the vector component, corresponding to the transition
to an absorbing state~$A_0$, represents the probability
distribution of the self-healing time.

Let us denote:
\begin{itemize}
\item $\pi_0$ is the probability that exists an instant~$t$ when
outputs of both (faulty and fault-free automata) are
different (the vector $\vec v_0 \ne \vec v_1$);
\item
$\pi_1$ is the probability of the automaton ``healing''
that is the return to the transitions trajectory of
normal functioning after an external fault effect will
be disappeared and $\vec v_0=\vec v_1$ for any~$t$; and
\item
$\pi$ is the probability that the automaton never more
come back to the normal transitions trajectory after
the external fault effect will disappear.
\end{itemize}

The relationships between these probabilities are:
$$
\pi_0 =
\lim_{t\to\infty} p_{(0)}(t)\,;
\quad
\pi_1 =
\lim_{t\to\infty} p_{(1)}(t)\,.
$$
Therefore,
\begin{equation}
\pi=1-\pi_0-\pi_1\,.
\label{e3fr}
\end{equation}

Let us demonstrate an example of the time to probabilities
computation.

\medskip

\noindent
\textbf{Example.}
Let us consider the automaton with two states in Table~1.

Let the transient fault be the erroneous transition to the
state~$a_2$ instead of~$a_1$.

Then corresponding MC~$Z_t$ that corresponds
to the MC describing the product of two automata has
four states: (1,2), (2,1), $A_0$, and $A_1$ where  state~(1,2)
describes the case when fault-free and SEU-affected
automata are correspondingly in the states~$a_1$ and~$a_2$,
(2,1)  means that they  are in states~$a_2$ and~$a_1$, and~$A_0$ and
$A_1$ are the  absorbing states mentioned above.
\columnbreak

In order to define the MC~$Z_t$ completely,
let us compute its state transition probabilities
(see Table~1).


First, the probabilities that the~$Z_t$ are in the
absorbing states~$A_0$ and~$A_1$ are equal to~1.

From state (1,2) to~$A_0$, it is possible to get only
if the input vector $\u_1u_2u_3$ is applied;
thereby, the probability of this transition is
$q_1p_2p_3$
where $p_l=\mathrm{Pr}\{x_l =1\}$ and $q_l=1-p_l$,
$l=1,2,3$.
Then both automata transit to the state~$a_2$ and
their outputs equal to~$v_2v_4$.

The only way to stay in the state~(1,2) is to apply
the input vector $u_1u_2u_3$ (with probability
$p_1p_2p_3$).
Then the fault-free automaton remains in the state~$a_1$,
whereas the automaton subjected the SEU hit will be
in the state~$a_2$, and both inputs vectors are~$v_2v_4$.

The transition from~(1,2) to~(2,1) is possible with
probability $q_1q_2$ if the signal $\u_1\u_2$ will
be applied, whereas the outputs are~$v_2v_4$.

Finally, the transition from~(1,2) to $A_1$ is possible
under input signals $u_1\u_2$ (with probability~$p_1q_2$),
$u_1u_2\u_3$ (with probability~$p_1p_2q_3$), and
$\u_1u_2\u_3$(with probability~$q_1p_2q_3$).

Indeed, under all of possible combinations of input
signals, there is a discrepancy between outputs vectors
of both automata, which means the state~$A_1$.
All possible transitions (determined by corresponding
input vectors) and there probabilities are given in
the row ``(1,2)'' of Table~2.

\bigskip

\begin{center} %tabl2
%\vspace*{-6pt}
{{\tablename~2}\ \ \small{Possible transitions and probabilities of input vectors}}
\vspace*{2ex}

{\small
\tabcolsep=8.6pt
\begin{tabular}{ccccc}
\hline
       &   $A_0$  &    (1,2)   &   (2,1)   &   $A_1$    \\
\hline
$A_0$  &     1    &      0   &     0    &     0     \\
%\hline
(1,2)  & $q_1p_2p_3$ & $p_1p_2p_3$ & $q_1q_2$ & $p_1q_2+p_2q_3$      \\
%\hline
(2,1)  & $q_1p_2p_3$ & $q_1q_2$ & $p_1p_2p_3$ & $p_1q_2+p_2q_3$      \\
%\hline
$A_1$  &    0      &    0      &    0      &    1         \\
\hline
\end{tabular}
}
\end{center}

\vspace*{6pt}

\bigskip
\setcounter{table}{2}


Analogously can be computed all transition probabilities
from state~(2,1).

Therefore, the probability transition matrix~$P^*$ of
the MC~$Z_t$ is:
$$
P^*
=
\begin{pmatrix}
1         &   0       &   0       &   0              \\
q_1p_2p_3 & p_1p_2p_3 & q_1q_2    &  p_1q_2+p_2q_3   \\
q_1p_2p_3 & q_1q_2    & p_1p_2p_3 &  p_1q_2+p_2q_3   \\
0         &   0       & 0         &   1
\end{pmatrix}
\,.
$$

Note that in the matrix~$P^*$, the following
transition probabilities are equal:
\begin{itemize}
\item from the states $(i,j)$ and $(j,i)$ to~$A_0$;
\item
from the states $(i,j)$ and $(j,i)$ to~$A_1$; and
\item
from the states $(i,j)$ to $(i,j)$
and from $(j,i)$ to~$(j,i)$.
\end{itemize}

%\noindent


\begin{table*}\small %tabl3
\begin{center}
\Caption{Possibilities of different states
}
\vspace*{2ex}

\begin{tabular}{lccccccc}
\hline
       & $t=0$ & $t=1$ & $t=2$ & $t=3$ & $t=4$ & $\ldots$ & $t=28$  \\
\hline
$p_{(0)}(t)$
       & 0,000000 & 0,080000 & 0,120000 & 0,140000 & 0,150000
                      & $\ldots$ & 0,160000 \\
%\hline
$p_{(1)}(t)$
       & 0,000000 & 0,420000 & 0,630000 & 0,735000 & 0,787500
                      & $\ldots$ & 0,840000 \\
%\hline
$\hat p(t)$
       & 1,000000 & 0,500000 & 0,250000 & 0,125000 & 0,062500
                      & $\ldots$ & 0,000000 \\
\hline
\end{tabular}
\end{center}
\end{table*}

Let suppose the following probabilities of Boolean~1
in the input vector bits:
$$
p_1=0{{,}}2\,;\quad
p_2=0{,}4\,;\quad
p_3=0{,}25
$$
and the initial distribution of the MC:
$$
\vec p^{\,*}(0) = (0,1,0,0)\,.
$$
Then
$$
P^*
=
\begin{pmatrix}
1      &   0    &   0    &   0       \\
0{,}08 & 0{,}02 & 0{,}48 &  0{,}42   \\
0{,}08 & 0{,}48 & 0{,}02 &  0{,}42   \\
0      &   0    & 0      &   1
\end{pmatrix}
$$
and
\begin{multline*}
\vec p^{\,*}(1)
=
\vec p^{\,*}(0) P^*
={}\\
{}=
(0,1,0,0)
\begin{pmatrix}
1      &   0    &   0    &   0       \\
0{,}08 & 0{,}02 & 0{,}48 &  0{,}42   \\
0{,}08 & 0{,}48 & 0{,}02 &  0{,}42   \\
0      &   0    & 0      &   1
\end{pmatrix}={}\\
{}= (0{,}08,\,0{,}02,\,0{,}48,\,0{,}42)\,;
\end{multline*}

\vspace*{-6pt}

\noindent
\begin{multline*}
\vec p^{\,*}(2)
=
\vec p^{\,*}(1) P^*
={}\\
{}=
(0{,}1200,\,0{,}2308,\,0{,}0192,\,0{,}6300)\,.
\end{multline*}

The  first row of Table~3 contains  probability
$p_{(0)}(t)$ that the automaton (see Table~1) will restore the
correct behavior before the fault effect will appear
at its output variables.
The second row contains the probability $p_{(1)}(t)$
where the output variables are corrupted by the
$t$th step and the third row contains the probability
$\hat p(t)=1-p_{(0)}(t)-p_{(1)}(t)$
that the state of the automaton was changed during a clock
(before the moment~$t$) by a transient fault,
but the output will remain correct.
{\looseness=1

}


Note that for this example, both conditional expected times to reach the state~$A_0$
and conditional expected time to reach the state~$A_1$  are equal to~2.

%7
\section{Concluding Remarks}

\noindent
In this paper, the self-healing phenomenon in
digital systems in terms of product of MC
describing faulty and fault-free FSMs under independent
random input binary signals was  analyzed.
Since we deal with the FSM models, it is possible to use
this self-healing probability in analysis of  reliability
of digital and computer systems at FSM level of their
modeling that is in rather early design stages. In other words,
these models can be used in  reliability analysis of a
target system in hierarchical system design.
They can be a base for a tool of fault-tolerant
systems design, dealing with such fault tolerance aspects
as fault detection latency (for permanent faults)~\cite{8fr}
and self-healing in presence of some transient faults
(Soft Upset Errors, in particular).
Now, there are some preliminary results of using this
approach to self-healing probability estimation for a
self-checking design~\cite{6fr}.
{\looseness=1

}


The authors' model (Eqs.~(1)--(3)) does not take into account probability
distribution of the transient faults during FSM functioning.
Therefore, this aspect can be indicated as a very important issue
of the future work.

{\small\frenchspacing
{%\baselineskip=10.8pt
%\addcontentsline{toc}{section}{Литература}
\begin{thebibliography}{99}

\bibitem{1fr}
\Au{Baumann R.}
Soft errors in advanced computer systems~//
IEEE Design and Test, May-June, 2005. P.~258--266.

% 2.
\bibitem{2fr}
\Au{Lala P.}
Self-checking and fault-tolerant digital  design.~---
Morgan Kaufmann Publs., 2000.

% 3.
\bibitem{3fr}
\Au{Lala P.\,K., Kumar B.\,K.}
An Architecture for self-healing digital systems~//
J.\ Electronic Testing: Theory and Applications, 2003.
Vol.~19. P.~523--535.

% 4.
\bibitem{9fr}
\Au{Hawthorne M., Perry D.}
Architectural styles for adaptable self-healing dependable systems~//
ICSEТ05 Proceedings, May 15--21, 2005. St.\ Louis, Missouri, USA.

% 5.
\bibitem{4fr}
\Au{Park J., Jung J., Piao Sh., Lee E.}
Self-healing mechanism for reliable computing~//
Int.\ J.\ Multimedia Ubiquitous Engineering,
2008. Vol.~3. No.\,1. P.~75--76.

% 6.
\bibitem{6fr}
\Au{Levin I., Matrosova A., Ostanin S.}
Survivable self-checking sequential circuits~//
DFTТ01 Proceedings, 2001. P.~395.

% 7.
\bibitem{7fr}
{\it Shedletsky J., McCluskey E.}
The error latency of fault in a sequential digital circuit~//
IEEE Transaction on Computers, 1976. Vol.~25, No.\,6. P.~655--659.

% 8.
\bibitem{8fr}
{\it Frenkel S., Pechinkin A., Chaplygin V., Levin I.}
A mathematical tool for support of fault-tolerant
embedded systems design~//
ERCIM/DECOS Dependable Smart Systems: Research,
Industrial Applications, Standardization,
Certification and Education. Workshop on
``Dependable Embedded Systems.'' L$\ddot{\text u}$beck, Germany, 2007.

% 9.
\bibitem{5fr}
{\it Custodio E., Marsland B.}
Self-healing partial reconfiguration of an FPGA.
Project Report, Project Number: MQP-BYK-GD07
Worcester Polytechnic Institute, April 26, 2007.

% 10.
\bibitem{10fr}
{\it Lala P.\,K., Kumar B.\,K.}
On self-healing digital system design~//
J.\ Microelectronics, 2006.
Vol.~37. P.~353--362.

% 11.
%\bibitem{11}
%{\it Feller W.}
%Probability Theory and its Applications.
 \end{thebibliography}
}
}


\end{multicols}

%\hrule

\vspace*{-24pt}

\def\tit{ОЦЕНКА ВРЕМЕНИ САМОВОССТАНОВЛЕНИЯ В ЦИФРОВЫХ СИСТЕМАХ ПОСЛЕ СБОЕВ,
ВЫЗЫВАЕМЫХ ПЕРЕХОДНЫМИ ПОМЕХАМИ}

\def\aut{С.\,Л.~Френкель$^1$, А.\,В.~Печинкин$^2$}

\titelr{\tit}{\aut}

\vspace*{12pt}

\noindent
$^1$Институт проблем информатики Российской академии наук,
Slf-ipiran@mtu-net.ru\\
\noindent
$^2$Институт проблем информатики Российской академии наук, apechinkin@ipiran.ru\\

%\vspace*{10mm}
\medskip


\Abst{Рассмотрен новый подход к оценке свойств
самовосстановления  в циф\-ро\-вых системах. Данное свойство характеризует способность системы
продолжать выполнять свои функции в случае сбоя в работе тех или иных компонентов, его учет может быть
полезен при расчете надежности.
Оценивается  время  до возвращения проектируемой системы в режим нормального функционирования после
проявления действия внешней помехи как функция распределения вероятности (ФРВ) этого времени.
Предложены возможные пути расчета ФРВ времени до самовосстановления на основе использовании
марковской модели поведения   цифровой системы, представленной  как конечный автомат  и
подверженной действию переходных помех.
}


\KW{отказоустойчивые компьютеры; самовосстановление и отказоустойчивость;
переходные неисправности; конечные машины состояний; цепи Маркова}

\label{end\stat}


\renewcommand{\figurename}{\protect\bf Рис.}
\renewcommand{\tablename}{\protect\bf Таблица}
\renewcommand{\bibname}{\protect\rmfamily Литература} %1

%\newcommand{\A}{{\mathbf A}}
%\newcommand{\B}{{\mathbf B}}
%\newcommand{\la}{{\lambda}}
%\newcommand{\be}{\begin{equation}}
%\newcommand{\ee}{\end{equation}}
%\newcommand{\ber}{\begin{eqnarray}}
%\newcommand{\eer}{\end{eqnarray}}

%\newcommand{\nin}{\noindent}
%\newcommand{\non}{\nonumber}
%\newcommand{\half}{\frac{1}{2}}
%\newcommand{\quarter}{\frac{1}{4}}

\def\stat{zeifman}

\def\tit{ОБ ОДНОМ КЛАССЕ МАРКОВСКИХ СИСТЕМ ОБСЛУЖИВАНИЯ$^*$}

\def\titkol{Об одном классе марковских систем обслуживания}

\def\autkol{Я.\,А.~Сатин, А.\,И.~Зейфман, А.\,В.~Коротышева, С.\,Я.~Шоргин}
\def\aut{Я.\,А.~Сатин$^1$, А.\,И.~Зейфман$^2$, А.\,В.~Коротышева$^3$, С.\,Я.~Шоргин$^4$}

\titel{\tit}{\aut}{\autkol}{\titkol}

{\renewcommand{\thefootnote}{\fnsymbol{footnote}}\footnotetext[1]
{Исследование поддержано РФФИ, гранты 11-07-00112-а и 11-01-12026-офи-м.}}


\renewcommand{\thefootnote}{\arabic{footnote}}
\footnotetext[1]{Вологодский государственный педагогический
университет, yacovi@mail.ru}
\footnotetext[2]{Вологодский государственный педагогический университет;  
Институт проблем информатики Российской академии наук; 
Институт социально-экономического развития территорий Российской академии наук,  a\_zeifman@mail.ru}
\footnotetext[3]{Вологодский государственный педагогический
университет,  a\_korotysheva@mail.ru}
\footnotetext[4]{Институт проблем информатики Российской академии наук, SShorgin@ipiran.ru}


\Abst{Рассматриваются модели обслуживания, описываемые конечными марковскими 
цепями с непрерывным временем. При этом предполагается,  что интенсивности 
поступления и обслуживания требований не зависят от числа требований в сис\-те\-ме. 
Получены оценки скорости сходимости и устойчивости различных характеристик таких сис\-тем.}

\KW{нестационарные марковские системы
обслуживания; скорость сходимости; устойчивость; оценки}

 \vskip 14pt plus 9pt minus 6pt

      \thispagestyle{headings}

      \begin{multicols}{2}
      
            \label{st\stat}

\section{Введение}

Классы систем массового обслуживания, описываемых процессами
рождения и гибели (стационарными и нестационарными, с катастрофами)
изучались начиная с 1970-х~гг.\ многими авторами
(см., например,~[1--6]). С~помощью методов,
разработанных одним из авторов настоящей \mbox{статьи}\linebreak (подробное изложение
этих методов приведено в~[7--9]), для таких сис\-тем
удалось получить точные оценки скорости сходимости и устойчивости.

Оказывается, этот же подход можно применить и к существенно более 
общему классу систем обслуживания.

Рассмотрим систему массового обслуживания, число требований в которой 
описывается нестационарной марковской цепью с непрерывным временем и 
конечным пространством состояний, причем требования могут поступать и 
обслуживаться группами.

Пусть $X=X(t)$, $t\geq 0$,~--- число требований в системе обслуживания ($0 \hm\le X(t) \hm\le r$).

Обозначим через 
\begin{gather*}
p_{ij}(s,t)=\mathrm{Pr}\left\{ X(t)=j\left| X(s)=i\right.
\right\}\,,\\
i,j \ge 0\,,\ 0\leq s\leq t\,,
\end{gather*}
переходные вероятности
процесса $X\hm=X(t)$, а через  $p_i(t)\hm=\mathrm{Pr}\left\{ X(t) \hm=i \right\}$~---
его вероятности состояний.

Будем предполагать, что интенсивности поступления и обслуживания $k$ требований в 
момент~$t$ в сис\-те\-ме об\-слу\-жи\-ва\-ния ($\lambda_{k}(t)$ и  $\mu_{k}(t)$ соответственно)  
не зависят от числа требований, находящихся в системе в момент~$t$, являются локально 
интегрируемыми на $[0,\infty)$ функциями времени~$t$ и, кроме того, 
$\lambda_{k+1}(t) \hm\le \lambda_{k}(t)$ и  $\mu_{k+1}(t) \hm\le \mu_{k}(t)$ при всех~$k$ 
и почти при всех $t \hm\ge 0$.

Тогда для описания вероятностной динамики процесса получаем прямую систему Колмогорова в виде
\begin{equation} 
\fr{d\vp}{dt}=A(t)\vp(t)\,,
\label{ur_1}
\end{equation}
 где
 {\footnotesize
\begin{multline*}
A(t)={}\\
{}=
\begin{pmatrix}
a_{00}(t) & \mu_1(t)  & \mu_2(t)   & \mu_3(t)  & \mu_4(t) & \cdots & \mu_r(t) \\
\la_1(t)   & a_{11}(t)  & \mu_1(t)  & \mu_2(t)   & \mu_3(t)  & \cdots & \mu_{r-1}(t) \\
\la_2(t)  & \la_1(t)    & a_{22}(t)& \mu_1(t)  & \mu2(t)    &  \cdots & \mu_{r-2}(t) \\
\cdots&\cdots&\cdots&\cdots&\cdots&\cdots&\cdots \\
\la_r(t) & \la_{r-1}(t) & \la_{r-2}(t) & \cdots & \la_2(t)  & \la_1 (t)   &  a_{rr}(t)
\end{pmatrix}\,,
\end{multline*}}
причем  
$$
a_{ii}(t)=-\sum\limits_{k=1}^{i}\mu_k(t) - \sum\limits_{k=1}^{r-i} \la_{r-k}(t)\,.
$$

Далее будем обозначать через $\|\bullet\|$  $l_1$-нор\-му, т.\,е.\ 
$\|{\vx}\|\hm=\sum|x_i|$, а $\|B\| \hm= \max\limits_j \sum\limits_i |b_{ij}|$, 
если $B \hm= (b_{ij})_{i,j=0}^{r}$.
%
Тогда, в частности, имеем 
$$
\|A(t)\| \le 2\sum\limits_{k=1}^{r}(\la_{k}(t)+ \mu_k(t))
$$ 
при  всех $t \hm\ge 0$.

Через 
$$
E(t,k) = E\left\{X(t)\left|X(0)\hm=k\right.\right\}
$$ 
будем далее обозначать математическое ожидание процесса (среднее число требований) в момент~$t$ 
при условии, что в нулевой момент времени он находится в состоянии~$k$, 
а через $E_{\bf p}(t)$ обозначим математическое ожидание процесса в момент~$t$ 
при начальном распределении вероятностей состояний $\mathbf{p}(0) \hm= \mathbf{p}$.

\section{Оценки скорости сходимости}

Рассмотрим вспомогательную последовательность положительных чисел $\{d_i\}$, $i\hm=1, \dots,r$.

Положим
\begin{equation*}
d=\min\limits_{1 \le i \le r} d_i\,; \enskip 
G=\sum\limits_{i=1}^r d_i\,; \enskip W=\min\limits_k \fr{d_k}{k}\,.
%\label{2.01}
\end{equation*}

Рассмотрим величины
\begin{multline*}
\alpha_i(t)= -a_{ii}(t)+\la_{r-i+1}(t)-\sum\limits_{k=1}^{i-1}(\mu_{i-k}(t)-{}\\
{}-
\mu_i(t))\fr{d_k}{d_i}-\sum\limits_{k=1}^{r-i}(\la_k(t)-\la_{i+r-1}(t))\fr{d_{k+i}}{d_i}\,,
%\label{2.02}
\end{multline*}

\noindent
\begin{equation*}
\alpha(t)=\min\limits_{1 \le i \le r}\alpha_i(t)\,.
%\label{2.03}
\end{equation*}

\smallskip

\noindent
\textbf{Теорема~1.} \textit{Пусть существует последовательность положительных 
чисел  $\{d_j\}$ такая, что}
\begin{equation}
\int\limits_0^{\infty} \alpha(t)\, dt = + \infty\,.
\label{2.031}
\end{equation}
\textit{Тогда $X(t)$ слабо эргодичен, при
любых начальных условиях} $\mathbf{p}^*(s)$, $\mathbf{p}^{**}(s)$ 
\textit{и любых $s$, $t$, $0\le s\le t$, справедлива оценка
\begin{equation} 
\label{2.04}
\|\vp^*(t)-\vp^{**}(t)\| \le \fr{8G}{d}\,e^{-\int\limits_s^t {\alpha(u)\,du}}\,.
\end{equation}
Кроме того,  $X(t)$ имеет предельное среднее $\phi(t)$ и при любых~$k$ и $t \hm\ge 0$ справедливо неравенство}:
\begin{equation}
\label{2.05}
|E(t,k)-\phi(t)|\le \fr{4G}{W}\,e^{-\int\limits_0^t {\alpha(u)\,du}}\,.
\end{equation}


\smallskip


\noindent
Д\,о\,к\,а\,з\,а\,т\,е\,л\,ь\,с\,т\,в\,о\,.\

Пользуясь предложенным в предыдущих работах способом, 
выразим 
$$
p_0=1-\sum\limits_{1\le i \le r}{p_i}\,.
$$

Тогда получим неоднородное уравнение:
\begin{equation} 
\label{ur_per}
\fr{d\vz}{dt}= B(t)\vz(t)+\vf(t)\,, 
%\label{2.06}
\end{equation}
\noindent
где $\vf(t)=\left(\la_1, \la_2,\cdots,\la_r \right)^{\mathrm{T}}$;

\end{multicols}


\hrule

\vspace*{6pt}

\begin{equation*}
B = \left(
\begin{array}{cccccccc}
a_{11}- \la_1   & \mu_1 - \la_1   & \mu_2 - \la_1   & \mu_3 -\la_1   & \cdots& \cdots & \mu_{r-1}- \la_1  \\
\la_1 -\la_2    & a_{22} -\la_2  & \mu_1-\la_2   & \mu_2 -\la_2     & \cdots&  \cdots & \mu_{r-2} -\la_2 \\
\la_2 -\la_3    & \la_1 -\la_3   & a_{33} -\la_3  & \mu_1-\la_2   & \cdots&  \cdots & \mu_{r-3} -\la_3 \\
\cdots&\cdots&\cdots&\cdots&\cdots&\cdots&\cdots \\
\la_{r-1} -\la_r  &\la_{r-2} -\la_r & \cdots & \cdots & \la_2 -\la_r   & \la_1 -\la_r     &  a_{rr} -\la_r
\end{array}
\right)\,.
%\label{2.07}
\end{equation*}

Рассмотрим треугольную матрицу
\begin{equation*}
D=\begin{pmatrix}
d_1   & d_1 & d_1 & \cdots & d_1 \\
0   & d_2  & d_2  &   \cdots & d_2 \\
\cdots&\cdots&\cdots&\cdots&\cdots \\
0  & 0 & \cdots & 0 &  d_r
\end{pmatrix}
%\label{2.08}
\end{equation*}
и соответствующую норму $\|{\bf z}\|_{D}\hm=\|D {\bf z}\|_1$.

Тогда имеем:
\begin{equation*}
 D BD^{-1}=\left(
\begin{array}{ccccccc}
a_{11}-\la_r  &  (\mu_1-\mu_2) \fr{d_1}{d_2}  & (\mu_2-\mu_3)\fr{d_1}{d_3}  & \cdots &  (\mu_{r-1}-\mu_r)\fr{d_1}{d_r} \\
(\la_1-\la_r) \fr{d_2}{d_1} &  a_{22}-\la_{r-1}  &(\mu_1-\mu_3)\fr{d_2}{d_3}  & \cdots &  (\mu_{r-2}-\mu_r)\fr{d_2}{d_r} \\
(\la_2-\la_r) \fr{d_3}{d_1} &  (\la_1-\la_{r-1})\fr{d_3}{d_2}   &a_{33}-\la_{r-2}   & \cdots &  (\mu_{r-3}-\mu_r)
\fr{d_3}{d_r}  \\
\cdots&\cdots&\cdots&\cdots&\cdots \\
(\la_{r-1} -\la_r) \fr{d_r}{d_1} & (\la_{r-2} -\la_{r-1}) \fr{d_r}{d_2}  & (\la_{r-3} -\la_{r-2}) \fr{d_r}{d_3}  & \cdots & a_{rr}-\la_1 \\
\end{array}
\right)\,.
%\label{2.09}
\end{equation*}


\begin{multicols}{2}


Далее, оценивая логарифмическую норму оператора~$B(t)$ (см., например, 
подробное рассмотрение в~[8--10]), получаем
\begin{multline*}
\gamma \left(B(t)\right)_{1D} = \gamma \left(DB(t)D^{-1}\right)_{1}={}\\
{}=
\max \left(\vphantom{\sum\limits_{k=1}^{i-1}}
a_{ii}(t) - \la_{r-i+1}(t) + \sum\limits_{k=1}^{i-1}\left(\mu_{i-k}(t)-{}\right.\right.\\
\left.\left.{}-\mu_i(t)\right)
\fr{d_k}{d_i} +
\sum\limits_{k=1}^{r-i}(\la_k(t)-\la_{i+r-1}(t))\fr{d_{k+i}}{d_i}\right) ={}\\
{}=
 - \min \alpha_i(t) = - \alpha(t)\,.
% \label{2.10}
\end{multline*}
Тогда\\[-7.9pt]
\begin{equation*}
\|\vz^*(t)-\vz^{**}(t)\|_{1D}\le  e^{-\int\limits_s^t {\alpha(u)du}}\|\vz^*(s)-\vz^{**}(s)\|_{1D}
%\label{2.11}
\end{equation*}
для всех $0 \le s \le t$ и любых начальных условий $\vz^*(s)$, $\vz^{**}(s)$.

Теперь, учитывая оценки для сравнения норм (см., например,~\cite{z08b}), получаем:
\begin{multline*}
\|\vp^*(t)-\vp^{**}(t)\| \le 2\|\vz^*(t)-\vz^{**}(t)\| \le{}\\
{}\le  \fr{4}{d}\|\vz^*(t)-\vz^{**}(t)\|_{1D}\le{} \\
{} \le \fr{4}{d}\,e^{-\int\limits_s^t {\alpha(u)\,du}}\|\vz^*(s)-\vz^{**}(s)\|_{1D} 
\le{}\\
{}\le
 \fr{4G}{d}\,e^{-\int\limits_s^t {\alpha(u)\,du}}\|\vz^*(s)-\vz^{**}(s)\| \le{} \\
{} \le  \fr{4G}{d}\,e^{-\int\limits_s^t {\alpha(u)\,du}}\|\vp^*(s)-\vp^{**}(s)\| \le 
\fr{8G}{d}\,e^{-\int\limits_s^t {\alpha(u)\,du}} 
%\label{2.11-a}
\end{multline*}
для любых начальных условий ${\bf p^*}(s)$, ${\bf p^{**}}(s)$ и любых $s,t$, $0\hm\le s\hm\le t$.

Из слабой эргодичности процесса с конечным пространством состояний 
вытекает существование предельного среднего, начальные условия для которого можно 
в общем случае выбрать произвольно.
Для оценки средних воспользуемся неравенством, приведенным в параграфе~2.3 из~\cite{z08b}:
\begin{multline*}
\|{\bf z}\|_{1D} = d_0 \left|\sum\limits_{i=1}^{\infty} p_i \right|
+ d_1 \left|\sum\limits_{i=2}^{\infty} p_i \right| + \dots \ge{}\\
{}\ge 
 W \sum\limits_{k \ge 1} k \left|\sum\limits_{i \ge k} p_i\right| \ge \fr{W}{2}
\sum\limits_{k \ge 1} k \left|p_k\right|\,.  
%\label{2.12}
\end{multline*}
Получаем теперь
\begin{multline*}
|E(t,k)-\phi(t)|\le \fr{2}{W}\,\|\vp^*(t)-\vp^{**}(t)\|_{1D}\le {} \\
{}\le\fr{2}{W}\,e^{-\int\limits_0^t {\alpha(u)\,du}}\|{\bf e}_k -
\vp^{**}(0)\|_{1D} \le \frac{4G}{W}e^{-\int\limits_0^t
{\alpha(u)\,du}}\,,
%\label{2.13}
\end{multline*}
что и требовалось доказать.
\columnbreak

%\smallskip

\noindent
\textbf{Замечание~1.} {Положим в условиях теоремы~1 
$$
\beta(t)=\max\limits_{1 \le i \le r}\alpha_i(t)\,.
$$ 
Тогда, пользуясь внедиагональной неотрицательностью матрицы $DB(t)D^{-1}$ 
с помощью методики, описанной в~\cite{z08b, z95b}, получаем справедливость неравенства

\noindent
\begin{equation*} 
%\label{2.14}
\|\vp^*(t)-\vp^{**}(t)\| \ge \fr{d}{8G}\,e^{-\int\limits_s^t {\beta(u)\,du}}
\end{equation*}
при любых $s$, $t$, $0\le s\le t$ и уже не при любых начальных условиях~${\bf p^*}(s)$, 
${\bf p^{**}}(s)$, а таких, что  $D\left({\bf p^*}(s) \hm-{\bf p^{**}}(s)\right) \hm\ge 0.$ 
Следовательно, оценки тео\-ре\-мы~1 будут заведомо иметь точный по времени порядок, если удастся 
выбрать вспомогательную последовательность $\{d_i\}$ так, что $\alpha(t)\hm=\beta(t)$, т.\,е.\ 
все $\alpha_i(t)$ одинаковы (не зависят от индекса~$i$)}.



\smallskip

Введем теперь в рассмотрение величины

\vspace*{-1pt}

\noindent
\begin{multline*}
\zeta_i(t)= -a_{ii}(t)+\la_{r-i+1}(t)+{}\\
{}+\sum\limits_{k=1}^{i-1}\left(\mu_{i-k}(t)-
\mu_i(t)\right) \fr{d_k}{d_i}+{}\\
{}+\sum\limits_{k=1}^{r-i}\left(\la_k(t)-\la_{i+r-1}(t)\right)\fr{d_{k+i}}{d_i}\,;
%\label{2.0211}
\end{multline*}
\begin{equation*}
\chi(t)=\max\limits_{1 \le i \le r}\zeta_i(t)\,.
%\label{2.0311}
\end{equation*}

\noindent
\textbf{Замечание 2.} {В условиях теоремы~1 при любых начальных условиях 
${\bf p^*}(s)$, ${\bf p^{**}}(s)$ и любых $s,t$,  $0\le s\le t$, 
справедлива следующая двухсторонняя оценка скорости сходимости:

\vspace*{-1pt}

\noindent
\begin{multline*} 
%\label{2.041}
\!\!\!\fr{d}{4G}\,e^{-\int\limits_s^t {\chi(u)\,du}}\|\vp^*(s)-\vp^{**}(s)\| \le
 \|\vp^*(t)-\vp^{**}(t)\| \le {}\\
 {}\le\fr{4G}{d}\,e^{-\int\limits_s^t {\alpha(u)\,du}}\|\vp^*(s)-\vp^{**}(s)\|.
\end{multline*}
Таким образом, можно оценить и сверху и снизу время  вхождения 
сис\-те\-мы обслуживания в предельный режим. Более подробно о получении 
нижних оценок см., например, в~\cite{z95b, gz05}.}

\smallskip

Рассмотрим два частных случая теоремы.

\smallskip

\noindent
\textbf{Следствие 1}. \textit{Пусть при выполнении остальных условий теоремы~1 
вместо}~(\ref{2.031}) \textit{выполняется условие $\alpha(t) \hm\ge \alpha \hm> 0$ 
почти при всех $t \hm\ge 0$. Тогда вместо}~(\ref{2.04}) \textit{и}~(\ref{2.05}) 
\textit{справедливы оценки}:

\vspace*{-1pt}

\noindent
\begin{align*} 
%\label{2.15}
\|\vp^*(t)-\vp^{**}(t)\| &\le \fr{8G}{d}\,e^{-\alpha \left(t-s\right)}\,;
\\
%\label{2.16}
|E(t,k)-\phi(t)|&\le \fr{4G}{W}\,e^{- \alpha t}\,.
\end{align*}

\pagebreak

%\smallskip

Положим 
\begin{gather*}
M_0=\max\limits_{|t-s|\le 1}\int\limits_s^t \alpha(u)\,du;\\
\alpha^* = \int\limits_0^1 \alpha(t)\, dt\,; \quad
M=e^{M_0+\alpha^*}\,.
\end{gather*}
С учетом неравенства 
$$
e^{-\int\limits_s^t {\alpha(u)\,du}} \hm\le M e^{-\alpha^* (t-s)}
$$ 
получаем следующее утверждение.

\smallskip

\noindent
\textbf{Следствие~2.} \textit{Пусть все $\lambda_k(t)$ и $\mu_k(t)$ 1-пе\-ри\-одич\-ны,  
а при выполнении остальных условий теоремы~1 вместо}~(\ref{2.031}) 
\textit{выполняется условие  $\alpha^* \hm> 0$.  Тогда предельный режим (скажем, $\vp^*(t)$) 
и соответствующее ему предельное среднее $\phi^*(t)$ можно выбрать 
1-пе\-ри\-оди\-че\-ски\-ми, а вместо}~(\ref{2.04}) \textit{и}~(\ref{2.05}) 
\textit{справедливы оценки}:
\begin{equation*} 
%\label{2.17}
\|\vp(t) - \vp^*(t)\| \le \fr{8GM}{d}\,e^{-\alpha^*t}
\end{equation*}
\textit{и, кроме того,}
\begin{equation*}
|E(t,k)-\phi^*(t)|\le \fr{4GM}{W}\,e^{-\alpha^*t}
%\label{2.18}
\end{equation*}
\textit{при любом $k$ и $t \ge 0$}.



\section{Устойчивость}

Рассмотрим также <<возмущенный>> процесс обслуживания $\bar{X}\hm=\bar{X}(t)$, $t\hm\geq 0$, 
в котором интенсивности поступления и обслуживания требований также не зависят от чис\-ла 
требований в системе, обозначая его соответствующие характеристики теми же буквами с 
чертой сверху. Для прос\-то\-ты записи оценок будем предполагать, что возмущения 
<<равномерно малы>>, т.\,е.\ выполняется неравенство $\| A(t)-\bar{A}(t)\| \hm\le \varepsilon$. 
Первые результаты для нестационарных цепей с непрерывным временем получены в~\cite{z85}, 
а детальное рассмотрение для более общего случая неравномерных оценок можно без труда 
провести так же, как это сделано в~\cite{z98, ae}. Для получения требуемых равномерных 
оценок устойчивости необходима экспоненциальная эргодичность соответствующего процесса, 
т.\,е.\ существование положительных констант $N$, $a$ таких, что  для правой части~(\ref{2.04}) 
справедливо неравенство:
\begin{equation}
e^{-\int\limits_s^t {\alpha(u)\,du}} \le Ne^{-a\left(t-s\right)}\,.
\label{3.01}
\end{equation}
Оценка~(\ref{3.01}) заведомо имеет место, в частности, если выполнены условия одного из следствий 
предыду\-ще\-го параграфа.

\smallskip

\noindent
\textbf{Теорема~2.}
\textit{Пусть выполнены условия теоремы~1 и}~(\ref{3.01}). \textit{Тогда при
 любых начальных условиях ${\bf p}(s)$ и ${\bar{\bf p}}(s)$ для процессов~$X(t)$ 
 и $\bar{X}(t)$ соответственно справедливы следующие оценки устойчивости:}
\begin{align*} 
%\label{3.02}
\limsup_{t \to \infty}  \|{\bf p}(t)- \bar{\bf p}(t)\| &\le
\fr{\varepsilon(1+\ln(4GN/d))}{a}\,;
\\
% \label{3.03}
\limsup\limits_{t \to \infty}   |E_{\bf p}(t)- \bar{E}_{\bar{\bf p}(t)}|&\le 
\fr{r \varepsilon(1+\ln(4GN/d))}{a}\,.
\end{align*}


\smallskip

\noindent
Д\,о\,к\,а\,з\,а\,т\,е\,л\,ь\,с\,т\,в\,о\ основано на подходе, 
введенном для стационарных процессов в~\cite{mit03} и описанном для нестационарной 
ситуации в~\cite{z11}.
Если  при любых начальных условиях для исходного процесса справедлива оценка
\begin{equation*} 
%\label{3.04}
\|\vp(t) - \vp^*(t)\| \le ce^{-b\left(t-s\right)}\,,
\end{equation*}
то, полагая
\begin{multline*}
\beta (t, s)=\sup\limits_{ \| {\bf v} \| =1, \sum {v_i}=0}
{\|V(t,s){\bf v}(t,s)\|} ={}\\
{}= \fr{1}{2} \max_{i,j} \sum\limits_k {|p_{ik}(t,
s)-p_{jk}(t, s)|}\,, 
\end{multline*}
где $V(t, s)$~--- матрица Коши
уравнения~(\ref{ur_1}), получаем в итоге следующее неравенство:
\begin{equation*}
\|{\bf p}(t)-\bar{\bf p}(t)\| \le{}
\begin{cases}
\|{\bf p}(s)-{\bf \bar{p}}(s)\|+ (t-s)\varepsilon \,, &\\
&\hspace*{-35mm} 0<t< b^{-1} \ln \left(\fr{c}{2}\right)\,; \\
b^{-1}\left(\ln \fr{c}{2} +1-\fr{c}{2}\,e^{-b(t-s)}\right)\varepsilon +{}&\\
{}+
\fr{c}{2}\,e^{-b(t-s)} \|{\bf p}(s)-{\bf \bar{p}}(s)\|\,, &\\
&\hspace*{-30mm}t\ge b^{-1}\ln \left(\fr{c}{2}\right)
\end{cases}
%\label{3.05}
\end{equation*}
для любых начальных условий ${\bf p}(s)$ и $\bar{\bf p}(s)$.
Из неравенств~(\ref{2.04}) и~(\ref{3.01}) вытекает, что $b=a$, $c={8GN}/{d}$.  
Устремив $t \hm\to \infty$ и взяв $s\hm=0$, получаем требуемые оценки.


\smallskip

\noindent
\textbf{Замечание~3.} 
В полученную оценку устойчивости для математического ожидания процесса 
в качестве множителя входит размерность~$r$, поэтому иногда лучший результат 
удается получить при помощи другого подхода, описанного в работе~\cite{z11}.

\smallskip

Положим 
$$
S=\max\limits_{{1 \le i, j \le r}} \fr{d_i}{d_j}\,,
$$ 
и пусть числа $K, L$ таковы, что 

\noindent
$$
d_1\la_1(t) + (d_1+d_2)\la_2(t) + \dots + 
\left(\sum\limits_{1 \le i \le r}d_i\right) \la_r(t) \le K\,,
$$ 
а 

\noindent
\begin{multline*}
d_1(\la_1(t)-\bar{\la}_1(t)) + (d_1+d_2)(\la_2(t)-\bar{\la}_2(t)) + \dots\\
\dots + 
\left(\sum\limits_{1 \le i \le r}d_i\right) (\la_r(t)-\bar{\la}_r(t)) \le 
L\varepsilon
\end{multline*} 
почти при всех $t \ge 0.$

\smallskip

\noindent
\textbf{Теорема~3.}
\textit{Пусть  выполнены условия теоремы~2 и, кроме того, при всех~$k$ 
и почти всех $t \hm\ge 0$ $\la_k(t) \hm< \infty$. Тогда при любых начальных условиях 
${\bf p}(s)$ и ${\bar{\bf p}}(s)$ для процессов $X(t)$ и $\bar{X}(t)$ 
соответственно справедливо неравенство}

\noindent
\begin{equation*}
\limsup\limits_{t \to \infty}   |E_{\bf p}(t)- \bar{E}_{\bar{\bf p}(t)}|\le 
\fr{ N\varepsilon\left(L a+ 2KNS\right)}{W a \left(a-2\varepsilon S\right)}\,.
\end{equation*}


\smallskip

\noindent
Д\,о\,к\,а\,з\,а\,т\,е\,л\,ь\,с\,т\,в\,о.\
 Перепишем исходную систему~(\ref{ur_per}) для невозмущенного процесса в следующем виде:
 \noindent
 
\begin{equation*}
\fr{d\vp}{dt}=\bar{B}(t)\vp(t) + {\bf f}(t)+\left(B(t)-\bar{B}(t)\right)\vp(t)\,.
%\label{eq112-n}
\end{equation*}
Тогда

\noindent
\begin{multline*}
\vp(t)=\bar{U}(t,0)\vp(0)+\int\limits_0^t \bar{U}(t,\tau){\bf{f}}(\tau) \, d\tau+{}\\
{}+\int\limits_0^t \bar{U}(t,\tau) \left(B(\tau)-\bar{B}(\tau)\right)\vp(\tau)\, d\tau\,;
\end{multline*}

\vspace*{-9pt}

\begin{equation*}
\hspace*{-15mm}\bar{\vp}(t)=\bar{U}(t,0)\bar{\vp}(0)+\int\limits_0^t \bar{U}(t,\tau){\bf{f}}(\tau) \, d\tau,
\end{equation*}
где $U(t,s)$~--- матрица Коши для уравнения~(\ref{ur_per}).
В любой норме при одинаковых начальных условиях получаем следующую оценку:
%\noindent
\begin{multline}
 \label{3000}
\!\!\!\!\!\!\left\|\vp(t)-\bar{\vp}(t)\right\|\le \!\!\int\limits_0^t \!\!\|\bar{U}(t,\tau)\|
\left(\| B(\tau)-\bar{B}(\tau)\| \|\vp(\tau)\| +\right.\\
\left.{}+ \| \vf(\tau)-\bar{\vf}(\tau)\|\right)\,d\tau\,.\!
\end{multline}
Имеем почти при всех $t \ge 0$:
\begin{equation*}
\|B(t)-\bar{B}(t)\|_{1D}=\|D(B(t)-\bar{B}(t))D^{-1}\| \le 2S\varepsilon\,;
%\label{3002}
\end{equation*}
%
%\vspace*{-14pt}
%
%\noindent
\begin{multline*}
\|{\bf f}(t)\|_{1D} \le d_1\la_1(t) + (d_1+d_2)\la_2(t) + \dots + {}\\
{}+
\left(\sum\limits_{1 \le i \le r}d_i\right) \la_r(t) \le K\,, 
\quad \|\vf(\tau)-\bar{\vf}(\tau)\|_{1D} \le L\varepsilon\,.
%\label{3002-a}
\end{multline*}
А тогда
\begin{multline*}
\gamma(\bar{B}(t))_{1D} \le \gamma(DB(t)D^{-1})+\|B(t)-\bar{B}(t)\|_{1D} \le  {}\\
{}\le -
\alpha(t)+2S \varepsilon \,.
% \label{3003}
\end{multline*}

Оценим теперь
\begin{multline*} 
%\label{8402}
\!\|{\bf p}(t)\|_{1D} \le
\|U(t){\bf p}(0) \|_{1D} +
 \int\limits_0^t \!\!\| U(t,\tau){\bf f}(\tau)\, d\tau \|_{1D} \le {}\\
 {}\le
 N e^{-a t} \| \vp(0)\|_{1D}  + \fr{K N}{a}.
\end{multline*}

 Тогда с учетом~(\ref{3000}) получаем:
\begin{multline*} 
%\label{3004}
\left\|\vp(t)-\bar{\vp}(t)\right\|_{1D}\le N\int\limits_0^t e^{-(a - 2\varepsilon S)(t-\tau)}\times{}\\
{}\times
\left(2S\varepsilon (N e^{-a \tau} \| \vp(0)\|_{1D}  + \fr{K N}{a}) +  L\varepsilon \right)\, d\tau  \le {} \\
{}\le  o(1)+\fr{ N\varepsilon(L+{2KNS}/{a})}{a-2\varepsilon S}\,. 
\end{multline*}

\vspace*{-9pt}

\section{Примеры}

\noindent
\textbf{Пример 1.}

Рассмотрим исходный процесс обслуживания с интенсивностями 
$\la_1(t)\hm=\la_2(t)\hm=\la_3(t)\hm=\la(t) \hm= 3\hm+\sin{2\pi t}$, 
$\mu_1(t)\hm=\mu_2(t)\hm=  \mu(t) \hm= 2\hm+\cos{2\pi t}$, 
$\la_4(t)=\ldots=\la_r(t)\hm=\mu_3(t)=\ldots=\mu_r(t)\hm=0$. Выберем последовательность  
$d_k\hm=h^k$, где $0{,}82 \hm< h \hm<1$. Тогда имеем
$$
d=h^r\,; \quad G \le \fr{h}{1-h}\,; \quad W=\fr{h^r}{r}\,.
$$

Будем предполагать, что возмущенный процесс имеет такую же структуру 
мат\-ри\-цы интенсивностей, причем $|\la(t)\hm-\bar{\la}(t)| \hm\le \varepsilon$ 
и  $|\mu(t)\hm-\bar{\mu}(t)| \hm\le \varepsilon$ почти при всех $t \hm\ge 0$. 
Отметим кстати, что при этом $\| A(t)\hm-\bar{A}(t)\| \hm\le 10 \varepsilon$ почти при 
всех $t \hm\ge 0$. Рассмотрим дальнейшие оценки:
$$
S=\fr{1}{h^2}\,; \ K=4 \left(3h+2h^2+h^3\right)\,; \ L=3h+2h^2+h^3\,;
$$
$$
\alpha(t) \ge \la(t)\left(3 - h - h^2 -h^3\right)-\mu(t)\left(\fr{1}{h^2}+\fr{1}{h}-2\right)\,;
$$
$$
\alpha^*= 3\left(3 - h - h^2 -h^3\right)-2\left(\fr{1}{h^2}+\fr{1}{h}-2\right)\,;
$$


\noindent
\begin{multline*}
M_0 \le \int\limits_0^1 |\alpha(t)|\, dt \le 4\left(3 - h - h^2 -h^3\right)+{}\\
{}+
3\left(\fr{1}{h^2}+\fr{1}{h}-2\right)\,;
\end{multline*}

\vspace*{-9pt}

\noindent
$$
M=e^{\alpha^*+M_0}\,.
$$

Если, например, взять 
$h\hm=0{,}9$, то $\alpha^*\hm=0{,}992$, $M_0\hm=3{,}281$, $M\hm=71{,}737$.

Тогда получаем следующие оценки.

По следствию~2
\begin{align*}
 \|{\bf p}(t)- {\bf p^{*}}(t)\| &\le \fr{8Me^{-\alpha^*t}}{h^{r-1}(1-h)}\,;\\
|E_{\bf p}(t)-\phi^*(t)| &\le  \fr{4Mre^{-\alpha^*t}}{h^{r-1}(1-h)}\,.
\end{align*}

По теореме~2 ($N=M$, $a=\alpha^*$) с использованием оценок следствия~2
\begin{align*}
\limsup\limits_{t \to \infty} \|{\bf p}(t)- \bar{\bf p}(t)\| &\le{} \notag\\
&\hspace*{-15mm}{}\le \fr{\varepsilon(1+\ln({4M}/({h^{r-1}(1-h)})))}{\alpha^*}\,;\\
\limsup\limits_{t \to \infty}   |E_{\bf p}(t)- \bar{E}_{\bar{\bf p}(t)}| &\le \notag\\
&\hspace*{-15mm}{}\le\fr{r\varepsilon(1+\ln(4M/(h^{r-1}(1-h))))}{\alpha^*}\,.
\end{align*}

По теореме~3 с использованием оценок следствия~2
\begin{multline*}
\limsup\limits_{t \to \infty}   |E_{\bf p}(t)- \bar{E}_{\bar{\bf p}(t)}| \le {}\\
{}\le
\fr{rM\varepsilon(3h+2h^2+h^3)(\alpha^* h^2+8M)}{h^r\alpha^*(\alpha^* h^2-2\varepsilon)}\,.
\end{multline*}

\noindent

\textbf{Пример 2.}

Рассмотрим процесс с интенсивностями 
$\la_1(t)\hm=\la_2(t)\hm=\ldots=\la_r(t) \hm= \la(t) \hm= 3\hm+\sin{2\pi t}$; 
$\mu_1(t)\hm=\mu_2(t)\hm= \mu(t) \hm= 2+\cos{2\pi t}$;
$\mu_3(t)=\ldots=\mu_r(t)=0$.

Будем предполагать, что возмущенный процесс имеет такую же структуру 
мат\-ри\-цы интен\-сив\-ностей, причем $|\la(t)-\bar{\la}(t)| \hm\le \varepsilon$ и  
$|\mu(t)-\bar{\mu}(t)| \hm\le \varepsilon$ почти при всех $t \hm\ge 0$. 
При этом будем иметь $\| A(t)\hm-\bar{A}(t)\| \hm\le 2r \varepsilon$ почти при всех $t \hm\ge 0$.

Выберем последовательность $d_k\hm=1$. Тогда  
\begin{gather*}
d=1\,; \enskip G=r\,; \enskip W=\fr{1}{r}\,; \enskip S=1\,; \\
K=\fr{4r(1+r)}{2}\,; \quad L=\fr{r(1+r)}{2}\,;
\\
\alpha(t)=\la(t)\,; \ \alpha=2\,; \ \alpha^*=3\,; M_0 \le 4\,; \ M \le  e^{7}\,.
\end{gather*}

И получаем следующие оценки.

\columnbreak

По следствию~1
\begin{align*}
 \|{\bf p^*}(t)- {\bf p^{**}}(t)\| &\le 8re^{-2t}\,;\\
|E_{\bf p}(t)- \phi(t)|&\le  4r^2 e^{-2t}\,.
\end{align*}

По следствию~2
\begin{align*}
\|{\bf p}(t)- {\bf p^{*}}(t)\| &\le 8re^{7-3t}\,;
\\[6pt]
|E_{\bf p}(t)- \phi^*(t)| &\le 4r^2 e^{7-3t}\,.
\end{align*}

По теореме~2 ($N=1$, $a=\alpha$) с учетом оценок следствия~1
\begin{align*}
\limsup\limits_{t \to \infty} \|{\bf p}(t)- \bar{\bf p}(t)\| &\le 
\fr{\varepsilon(1+\ln{4r})}{2}\,;
\\[6pt]
\limsup\limits_{t \to \infty}   |E_{\bf p}(t)- \bar{E}_{\bar{\bf p}(t)}|
&\le \fr{r\varepsilon(1+\ln{4r})}{2}\,.
\end{align*}

По теореме~2 ($N=M$, $a=\alpha^*$) с учетом оценок следствия~2
\begin{align*}
\limsup\limits_{t \to \infty} \|{\bf p}(t)- \bar{\bf p}(t)\| &\le 
\fr{\varepsilon(8+\ln{4r})}{3}\,;
\\
\limsup\limits_{t \to \infty}   \left|E_{\bf p}(t)- \bar{E}_{\bar{\bf p}(t)}\right| &\le 
\fr{r\varepsilon(8 + \ln{4r})}{3}\,.
\end{align*}

По теореме~3 с учетом оценок следствия~1
\begin{equation*}
\limsup\limits_{t \to \infty}   \left|E_{\bf p}(t)- \bar{E}_{\bar{\bf p}(t)}\right| \le 
\fr{5 \varepsilon r^2 (1+r)}{4(1- \varepsilon)}\,.
\end{equation*}

По теореме~3 с учетом оценок следствия~2
\begin{equation*}
\limsup\limits_{t \to \infty}   \left|E_{\bf p}(t)- \bar{E}_{\bar{\bf p}(t)}\right| \le 
\fr{\varepsilon e^{7} r^2 (1+r) (3+8e^{7})}{6(3-2\varepsilon)}\,.
\end{equation*}

{\small\frenchspacing
{%\baselineskip=10.8pt
\addcontentsline{toc}{section}{Литература}
\begin{thebibliography}{99}

 \bibitem{b} %1
\Au{Баруча-Рид~А.\,Т.} Элементы теории марковских процессов и их
приложения.~--- М.: Наука, 1969.

\bibitem{gm}  %2
\Au{Гнеденко~Б.\,В., Макаров~И.\,П.} Свойства решений задачи с потерями
в случае периодических интенсивностей~// Дифф. уравнения, 1971.
Вып.~7. С.~1696--1698.

\bibitem{g1}   %3
\Au{Gnedenko~D.\,B.} On a generalization of Erlang formulae~// 
Zastosow. Mat., 1971. Vol.~12. P.~239--242.

\bibitem{S}  %4
\Au{Саати~Т.\,Л.} Элементы теории массового обслуживания
 и ее приложения.~--- М.: Сов. радио, 1971.

\bibitem{g}  %5
\Au{Gnedenko~B., Soloviev~A.} On the conditions of the
existence of final probabilities for a Markov process~// Math.
Operations. Stat., 1973. P.~379--390.

\bibitem{gk} %6
\Au{Гнеденко~Б.\,В., Коваленко~И.\,Н.} Введение в теорию массового
обслуживания.~--- М.: Наука, 1987.
\pagebreak

\bibitem{gz00}   %7
\Au{Granovsky~B.\,L., Zeifman~A.\,I.}  The N-limit of spectral gap of 
a class of birth-death Markov chains~//
 Appl. Stoch. Models Business Ind., 2000. Vol.~16. P.~235--248.

\bibitem{z08b}  %8
\Au{Зейфман~А.\,И., Бенинг~В.\,Е., Соколов~И.\,А.} 
Марковские цепи и модели с непрерывным временем.~--- М.: Элекс-КМ, 2008.

\bibitem{dzp} %9
\Au{Van Doorn~E.\,A., Zeifman~A.\,I., Panfilova~T.\,L.}  
Bounds and asymptotics for the rate of convergence of birth-death processes~//  
Th. Prob. Appl., 2010. Vol.~54. P.~97--113.

\bibitem{z95b}   %10
\Au{Zeifman~A.\,I.} Upper and lower bounds on the rate of
convergence for nonhomogeneous birth and death processes~//  Stoch.
Proc. Appl., 1995. Vol.~59. P.~157--173.

\bibitem{gz05}  %11
\Au{Granovsky~B.\,L., Zeifman~A.\,I.} On the lower bound of the spectrum
 of some mean-field models~// Theory Prob. Appl., 2005. Vol.~49. P.~148--155.
 
\bibitem{z85}  %12
\Au{Zeifman~A.\,I.} Stability for contionuous-time
nonhomogeneous Markov chains~// Lect. Notes Math.,  1985. Vol.~1155.
P.~401--414.

\bibitem{z98} %13
\Au{Zeifman~A.} Stability of birth and death processes~// 
J.~Math. Sci., 1998. Vol.~91. P.~3023--3031.

\bibitem{ae} %14
\Au{Андреев~Д., Елесин~М., Кузнецов~А., Крылов~Е., Зейфман~А.}
Эргодичность и устойчивость нестационарных систем обслуживания~//
Теория вероятностей и математическая статистика, 2003. Т.~68.
С.~1--11.

\bibitem{mit03} %15
\Au{Mitrophanov~A.\,Yu.} Stability and exponential convergence of continuous-time 
Markov chains~//  J. Appl. Prob., 2003. Vol.~40. P.~970--979.

\label{end\stat} 

\bibitem{z11} %16
\Au{Зейфман~А.\,И., Коротышева~А.\,В., Панфилова~Т.\,Л., Шоргин~С.\,Я.} 
Оценки устойчивости  для некоторых систем обслуживания с катастрофами~//  
Информатика и её применения, 2011. Т.~5. Вып.~3. С.~27--33.
 \end{thebibliography}
}
}


\end{multicols}         %2


\def\stat{shorgin}

\def\tit{БАЙЕСОВСКИЕ МОДЕЛИ МАССОВОГО  ОБСЛУЖИВАНИЯ И~НАДЕЖНОСТИ:
ХАРАКТЕРИСТИКИ СРЕДНЕГО ЧИСЛА ЗАЯВОК  В~СИСТЕМЕ $M|M|1|\infty$$^*$}

\def\titkol{Байесовские модели массового  обслуживания и надежности}
%: характеристики среднего числа заявок  в системе $M|M|1|\infty$}

\def\autkol{А.\,А.~Кудрявцев, С.\,Я.~Шоргин}
\def\aut{А.\,А.~Кудрявцев$^1$, С.\,Я.~Шоргин$^2$}

\titel{\tit}{\aut}{\autkol}{\titkol}

{\renewcommand{\thefootnote}{\fnsymbol{footnote}}\footnotetext[1]
{Работа выполнена при поддержке РФФИ, проекты 08-07-00152-а, 08-01-00567-а 
и 09-07-12032-офи-м. Статья написана на основе материалов доклада, 
представленного на IV Международном семинаре  
<<Прикладные задачи теории вероятностей и математической статистики, 
связанные с моделированием информационных систем>> 
(зимняя сессия, Аоста, Италия, январь--февраль 2010~г.).}}

\renewcommand{\thefootnote}{\arabic{footnote}}
\footnotetext[1]{Московский
государственный университет им.\ М.\,В.~Ломоносова, факультет
вычислительной математики и кибернетики, nubigena@hotmail.com}
\footnotetext[2]{Институт проблем информатики Российской академии наук, sshorgin@ipiran.ru}

\vspace*{-6pt}

\Abst{Данная работа продолжает ряд статей, посвященных байесовским моделям массового обслуживания и
надежности. В работе рассматриваются вероятностные характеристики среднего числа заявок в системе $M|M|1|\infty$
в условиях рандомизации параметров входящего потока и обслуживания. Обсуждается интерпретация получаемых результатов с
учетом возможной несобственности распределения среднего числа заявок.}

\KW{байесовский подход; системы массового обслуживания; надежность; смешанные
распределения; моделирование; несобственное распределение; <<дефектное>> распределение}

\vskip 14pt plus 9pt minus 6pt

      \thispagestyle{headings}

      \begin{multicols}{2}

      \label{st\stat}


\section{Основные предположения и~обозначения}

Подробное изложение основ байесовского подхода к моделированию систем массового обслуживания (СМО) и ненадежных
восстанавливаемых систем, а также результаты вычисления основных вероятностных характеристик коэффициентов
загрузки и готовности системы $M|M|1|0$, входные параметры которой не известны исследователю в точности (известно лишь их
априорное распределение), можно найти в работах~[1--6].

Основным предположением в рамках данного подхода для моделей $M|M|1$ является рандомизация интенсивностей входящего
потока~$\lambda$ и обслуживания~$\mu$. При этом, естественно, становится случайной и загрузка рассматриваемой системы
$\rho=\lambda/\mu$, от значения которой, в частности, зависит наличие стационарного режима у рассматриваемой системы.
Кроме того, величина~$\rho$ входит во многие формулы, описывающие характеристики разнообразных СМО.
В статье рассматривается одна из таких характеристик, а именно среднее число заявок в системе  $M|M|1|\infty$
$$
N=\fr{\rho}{1-\rho}\,.
$$

В дальнейшем изложении будем предполагать, что входные параметры системы стохастически независимы и имеют вырожденное
($D$), равномерное~($R$), экспоненциальное~($M$), Эрланга~($E$) распределение. В~скобках будем указывать
соответствующие параметры распределения, например обозначение~$D(a)$ будет соответствовать вырожденному в точке~$a$
распределению.

\section{Функция распределения и~плотность}

Будем обозначать функции распределения случайных величин~$\rho$ и~$N$ соответственно
\begin{equation}
\left.
\begin{array}{rl}
F_\rho(x)&=\p(\rho<x)\,;\\[6pt]
 F_N(x)&=\p(N<x)=F_\rho\left( \fr{x}{1+x}\right)\,,\enskip x>0\,.
 \end{array}
 \right \}
\label{e1sh}
\end{equation}
Обозначим через $f_\rho(x)$ и~$f_N(x)$ соответствующие плотности. Очевидно,
\begin{equation}
f_N(x)=\fr{1}{(1+x)^2}f_\rho\left(\fr{x}{1+x}\right)\,,\enskip x>0\,.
\label{e2sh}
\end{equation}

\begin{table*}\small
\begin{center}
\Caption{Функция распределения коэффициента загрузки $\rho$
\label{t1sh}}
\vspace*{2ex}

\begin{tabular}{|c|c|c|}
\hline
$\lambda$&$\mu$&$F_\rho(x),\ x>0$\\
\hline
&&\\[-9pt]
$D(\lambda)$&$M(\alpha)$&$e^{-\alpha\lambda/x}$\\
\hline
&&\\[-9pt]
$D(\lambda)$&$E(n,\alpha)$&$e^{-\alpha\lambda/x}\displaystyle\sum\limits_{m=0}^{n-1}\fr{(\alpha\lambda)^m}{x^mm!}$\\
&&\\[-9pt]
\hline
&&\\[-9pt]
$M(\theta)$&$D(\mu)$&$1-e^{-\mu\theta x}$\\
&&\\[-9pt]
\hline
&&\\[-9pt]
$M(\theta)$&$M(\alpha)$&$\displaystyle\fr{\theta x}{\alpha+\theta x}$\\
&&\\[-9pt]
\hline
&&\\[-9pt]
$M(\theta)$&$E(n,\alpha)$&$1-\left(\displaystyle\fr{\alpha}{\alpha+\theta x}\right)^n$\\
&&\\[-9pt]
\hline
&&\\[-9pt]
$E(k,\theta)$&$D(\mu)$&$1-e^{-\mu\theta x}\displaystyle\sum\limits_{m=0}^{k-1}\displaystyle\fr{(\mu \theta x)^m}{m!}$\\
&&\\[-9pt]
\hline
&&\\[-9pt]
$E(k,\theta)$&$M(\alpha)$&$\left(\displaystyle\fr{\theta x}{\alpha+\theta x}\right)^k$\\
&&\\[-9pt]
\hline
&&\\[-9pt]
$E(k,\theta),\ \ k\ge2$&$E(n,\alpha),\ \ n\ge2$&$\left(\displaystyle\fr{\theta x}{\alpha+\theta x}\right)^{n+k-1}
\displaystyle\sum\limits_{m=0}^{n-1}\left(\displaystyle\fr{\alpha}{\theta x}\right)^m$\\[9pt]
\hline
\end{tabular}
\end{center}
%\vspace*{9pt}
\end{table*}

Для удобства дальнейшего изложения приведем таблицу функций
распределения коэффициента загрузки~$\rho$ при различных
распределениях интенсивностей входящего потока $\lambda$ и
обслуживания~$\mu$. Представленные результаты опубликованы авторами
в работах~\cite{KuSh07, KuSh09a, KuSh09b}. (Для представления
функции распределения Эрланга с параметрами~$k$ и~$\mu\theta$ в
случае, когда~$\lambda$ имеет распределение $E(k,\theta)$, а~$\mu$~--- 
$D(\mu)$, использована формула~3.351.1 из~\cite{GR71}.)


Как известно (см., например,~\cite{BoPe95}), при классической постановке задачи величину~$N$ имеет смысл рассматривать
только в случае $\rho<1$ (в противном случае $N=\infty$). При байесовском подходе нельзя однозначно сказать, какое
значение примет величина~$\rho$, а следовательно, априори нельзя сделать и точных выводов о том, будет ли система
переполняться. Вполне естественно, что основную роль в прогнозах, касающихся среднего числа заявок~$N$, будет играть
величина
\begin{equation}
\delta=\p(\rho\ge1)=1-F_\rho(1)\,.
\label{e3sh}
\end{equation}
При этом, если не выполнено условие $\p(\lambda<\mu)=1$, величина~$\delta$ будет положительной. В~этом случае функция
распределения рандомизированного среднего числа заявок в системе будет отделена от единицы на величину $\delta$.
Другими словами,
\begin{equation}
\lim_{x\to\infty}F_N(x)=1-\delta\,.
\label{e4sh}
\end{equation}

При рассмотрении распределений, обладающих свойством~(\ref{e4sh}), возможны два подхода. Согласно первому (см., 
например,~\cite{Shiryaev80}) предполагается, что случайная величина~$N$ может принимать бесконечное значение с положительной
вероятностью. В~этом случае~$N$ называется расширенной случайной величиной. Согласно второму подходу 
(см.~\cite{Feller67}) делается предположение, что <<масса>> распределения случайной величины~$N$ строго 
меньше единицы. При
этом само распределение называется несобственным или <<дефектным>>, но предположений о возможном бесконечном значении
соответствующей случайной величины, по сути, не делается. Заметим, что для исследователя зачастую более удобен второй
подход к данной проблеме, поскольку, например, математическое ожидание расширенной случайной величины всегда равняется
бесконечности, в то время как математическое ожидание <<дефектной>> случайной величины, рассматриваемое как координата
центра масс системы, масса которой меньше единицы, может быть конечным числом.

Следуя терминологии~\cite{Feller67}, назовем величину~$\delta$, определенную в~(\ref{e3sh}), 
\textit{<<дефектом>> системы}.

Приведем значения величины <<дефекта>>~$\delta$ для байесовских СМО при различных предположениях об априорных
распределениях интенсивностей~$\lambda$ и~$\mu$. 
Применяя формулу~(\ref{e3sh}) к выражениям из табл.~\ref{t1sh}, 
получим результаты, показанные в табл.~\ref{t2sh}.

\begin{table*}\small %tabl2
\begin{center}
\Caption{Значения величины <<дефекта>> $\delta$
\label{t2sh}}
\vspace*{2ex}

\begin{tabular}{|c|c|c|}
\hline
$\lambda$&$\mu$&$\delta$\\
\hline
&&\\[-9pt]
$D(\lambda)$&$M(\alpha)$&$1-e^{-\alpha\lambda}$\\
&&\\[-9pt]
\hline
&&\\[-9pt]
$D(\lambda)$&$E(n,\alpha)$&$1-e^{-\alpha\lambda}\sum\limits_{m=0}^{n-1}\displaystyle\fr{(\alpha\lambda)^m}{m!}$\\
&&\\[-9pt]
\hline
&&\\[-9pt]
$M(\theta)$&$D(\mu)$&$e^{-\mu\theta}$\\
&&\\[-9pt]
\hline
&&\\[-9pt]
$M(\theta)$&$M(\alpha)$&$\displaystyle\fr{\alpha}{\alpha+\theta}$\\
&&\\[-9pt]
\hline
&&\\[-9pt]
$M(\theta)$&$E(n,\alpha)$&$\left(\displaystyle\fr{\alpha}{\alpha+\theta}\right)^n$\\
&&\\[-9pt]
\hline
&&\\[-9pt]
$E(k,\theta)$&$D(\mu)$&$e^{-\mu\theta}\displaystyle\sum\limits_{m=0}^{k-1}\displaystyle\fr{(\mu\theta)^m}{m!}$\\
&&\\[-9pt]
\hline
&&\\[-9pt]
$E(k,\theta)$&$M(\alpha)$&$1-\left(\displaystyle\fr{\theta}{\alpha+\theta}\right)^k$\\
&&\\[-9pt]
\hline
&&\\[-9pt]
$E(k,\theta),\ \ k\ge2$&$E(n,\alpha),\ \ n\ge2$&$1-\displaystyle\fr{\theta^{k+1}(\theta^{n-1}-\alpha^{n-1})}{(\theta-\alpha)
(\theta+\alpha)^{n+k-1}},\ \alpha\neq\theta$\\
&&$\displaystyle1-\fr{n}{2^{n+k-1}},\  \alpha=\theta$\\
\hline
\end{tabular}
\end{center}
%\vspace*{-6pt}
\end{table*}

Аналогично, зная вид функции распределения коэффициента загрузки~$\rho$ и воспользовавшись формулами~(\ref{e1sh}) и~(\ref{e2sh}),
несложно получить выражения для функции распределения и плотности случайной величины~$N$. Не будем останавливаться
здесь на результатах, полученных для распределений~$\lambda$ и~$\mu$, заданных в табл.~\ref{t1sh}, поскольку они не
существенны для дальнейшего изложения и получаются тривиальной заменой переменных.

Далее приведем важный пример системы с нулевым <<дефектом>>. В~силу постулируемой независимости случайных 
величин~$\lambda$ и~$\mu$ обеспечить выполнение условия $\p(\lambda<\mu)=1$ может лишь определенное взаимное расположение
носителей соответствующих распределений.

Пусть случайные величины $\lambda$ и~$\mu$ имеют равномерные распределения $R(a_\lambda,b_\lambda)$ и $R(a_\mu,b_\mu)$,
соответственно, причем $0<a_\lambda<b_\lambda<a_\mu<b_\mu$. Очевидно, что <<дефект>>~$\delta$ такой системы равняется
нулю.

В работе~\cite{KuSh07} были приведены формулы для функции распределения и плотности коэффициента загрузки~$\rho$ для
случая $a_\lambda/a_\mu<b_\lambda/b_\mu$. Воспользуемся этими результатами и формулами~(\ref{e1sh}) и~(\ref{e2sh}). Обозначим

\noindent
$$
c_\lambda=(b_\lambda-a_\lambda)^{-1}\enskip \mbox{и}\enskip   c_\mu=(b_\mu-a_\mu)^{-1}\,.
$$
После несложных арифметических преобразований получаем
$$
F_N(x)=0\,,\ \  \mbox{если}\ \  x<a_\lambda/(b_\mu-a_\lambda)\,;
$$

\vspace*{-12pt}

\noindent
\begin{multline*}
F_N(x)=\fr{c_\lambda c_\mu}{2}\left(b_\mu\sqrt{\fr{x}{1+x}}-a_\lambda\sqrt{\fr{1+x}{x}}\right)^2\,\\
\mbox{если}\ 
\fr{a_\lambda}{b_\mu-a_\lambda}\le x\le\fr{a_\lambda}{a_\mu-a_\lambda}\,;
\end{multline*}

\vspace*{-13pt}

\noindent
\begin{multline*}
F_N(x)=c_\lambda\left(\fr{(b_\mu+a_\mu)x}{2(1+x)}-a_\lambda\right)\,, \\
\mbox{если}\  \fr{a_\lambda}{a_\mu-a_\lambda}\le x\le\fr{b_\lambda}{b_\mu-b_\lambda}\,;
\end{multline*}

\vspace*{-12pt}

\noindent
\begin{multline*}
F_N(x)=c_\lambda c_\mu\left(\fr{b_\lambda(1+x)}{x}-a_\mu\right)\times{}\\
{}\times\left(\fr{b_\lambda}{2}+
\fr{a_\mu x}{2(1+x)}-a_\lambda\right)+
c_\mu\left(b_\mu-\fr{b_\lambda(1+x)}{x}\right)\,,\\
 \mbox{если}\ \fr{b_\lambda}{b_\mu-b_\lambda}\le x\le\fr{b_\lambda}{a_\mu-b_\lambda}\,;
  \end{multline*}
  
%  \vspace*{-1pt}
  
  \noindent
  $$
F_N(x)=1\,,\   \mbox{если} \  x>b_\lambda/(a_\mu-b_\lambda)\,.
$$


Соответствующая плотность будет иметь вид:
$$
f_N(x)=0\,,\ \mbox{если}\  x<a_\lambda/(b_\mu-a_\lambda)
\  \mbox{и}\ 
x>b_\lambda/(a_\mu-b_\lambda)\,;
$$

\vspace*{-12pt}

\noindent
\begin{multline*}
f_N(x)=\fr{c_\lambda c_\mu}{2}\left(\fr{b_\mu^2}{(1+x)^2}-\fr{a_\lambda^2}{x^2}\right)\,,\\
  \mbox{если}\ 
\fr{a_\lambda}{b_\mu-a_\lambda}\le x\le\fr{a_\lambda}{a_\mu-a_\lambda}\,;
\end{multline*}

\vspace*{-12pt}

\noindent
\begin{multline*}
f_N(x)=\fr{c_\lambda (b_\mu+a_\mu)}{2(1+x)^2}\,,\\
 \mbox{если}\ 
\fr{a_\lambda}{a_\mu-a_\lambda}\le x\le\fr{b_\lambda}{b_\mu-b_\lambda}\,;
\end{multline*}

\vspace*{-12pt}

\noindent
\begin{multline*}
f_N(x)=\fr{c_\lambda c_\mu(2b_\lambda a_\lambda-b_\lambda^2)}{2x^2}-\fr{c_\lambda c_\mu a_\mu^2}{2(1+x)^2}
+\fr{c_\mu b_\lambda}{x^2}\,,\\
 \mbox{если}\ \fr{b_\lambda}{b_\mu-b_\lambda}\le x\le\fr{b_\lambda}{a_\mu-b_\lambda}\,.
\end{multline*}

Аналогичные результаты можно получить для случая $a_\lambda/a_\mu\ge b_\lambda/b_\mu$.

%\vspace*{-6pt}
\section{Моментные характеристики}

%\vspace*{-6pt}

Как упоминалось в разд.~2, вопрос о существовании математического ожидания среднего числа заявок~$N$ существенным
образом зависит от величины <<дефекта>> системы~$\delta$. В~случае $\delta>0$ математическое ожидание, понимаемое в
классическом смысле, всегда равняется бесконечности.

Однако можно рассмотреть аналог математического ожидания~--- координату центра масс системы, масса которой равняется
$1-\delta$. Обозначим эту характеристику $\hat{\sf E}N$. Формально имеет место следующее определение:
$$
\hat{\sf E} N=\fr{1}{F_N(\infty)}\int\limits_{0}^{\infty}x\, dF_N(x)\,.
$$

Заметим, что в случае нулевого <<дефекта>> $\hat{\sf E}N$ совпадает с~$\e N$ и, по сути, является условным
математическим ожиданием $\hat{\sf E}N=\e\left(N\ |\ \rho<1\right)$.

Соотношения~(\ref{e1sh}), (\ref{e3sh}) и~(\ref{e4sh}) 
дают возможность привести цепочку тождеств, позволяющих записать $\hat{\sf E}N$ при
помощи различных характеристик системы:
\begin{multline}
\hat{\sf E}N=\fr{1}{F_N(\infty)}\il{0}{\infty}(F_N(\infty)-F_N(x))\, dx={}\\
{}=\fr{1}{1-\delta}\il{0}{\infty}(1-\delta-F_N(x))\, dx={}\\
{}=\fr{1}{F_\rho(1)}\il{0}{\infty}\left(F_\rho(1)-F_\rho\left(\fr{x}{1+x}\right)\right)\, dx={}\\
{}
=\fr{1}{F_\rho(1)}\il{0}{1} \fr{F_\rho(1)-F_\rho(x)}{1-x}\, \fr{dx}{1-x}\,.
\label{e5sh}
\end{multline}

Последнее выражение в~(\ref{e5sh}) показывает, что в абсолютно непрерывном случае для существования $\hat{\sf E}N$
необходимо, чтобы плотность~$f_\rho(x)$ в некоторой окрестности $(1-\varepsilon,\,1]$ убывала к нулю при $y\to 1$
быстрее, чем $(1-y)^p$ для некоторого $p\in(0,\,1)$.
Приведем пример такого распределения с положительным <<дефектом>>.

Пусть
$$
f_\rho(x)=\fr{3}{4}\left\vert1-x\right\vert^{{1}/{2}}\,,\enskip x\in[0,\,2]\,.
$$
Тогда
\begin{multline*}
\hat{\sf E}N=\fr{1}{F_\rho(1)}\il{0}{\infty}x\, dF_\rho\left(\fr{x}{1+x}\right)
={}\\
{}=\fr{1}{F_\rho(1)}\il{0}{1}\fr{yf_\rho(y)}{1-y}\,dy=
\fr{3}{2}\il{0}{1}\fr{y\,dy}{\sqrt{1-y}}={}\\
{}=\fr{3}{2}\il{0}{1}\fr{dy}{\sqrt{1-y}}-
\fr{3}{2}\il{0}{1}\sqrt{1-y}\,dy=2\,.
\end{multline*}

Таким образом, в данном случае получена некоторая конечная моментная характеристика распределения случайной величины~$N$, 
при этом математическое ожидание <<дефектного>> распределения~$N$ бесконечно.

Несмотря на то что подобный подход может давать дополнительную
информацию о распределении среднего числа заявок~$N$, ограничение на
поведение плотности коэффициента загрузки~$\rho$ в окрестности
единицы является столь существенным, что для всех распределений
интенсивностей\linebreak входящего потока~$\lambda$ и обслуживания~$\mu$,
приведенных в табл.~\ref{t1sh}, как несложно убедиться, будут получаться
бесконечные значения для~$\hat{\sf E}N$. По этой\linebreak причине
необходимо привлечение иных методов изучения интересующего нас
распределения. Об~одном из таких методов речь пойдет в сле\-ду\-ющем
разделе.

В~заключение данного раздела рассмотрим пример <<недефектного>> распределения, для которого существуют классические
моментные характеристики.

\begin{table*}[b]\small
\begin{center}
\Caption{Значения квантилей $x_q$ распределения $N$
\label{t3sh}}
\vspace*{2ex}

\tabcolsep=7pt
\begin{tabular}{|c|c|c|}
\hline
$\lambda$&$\mu$&$x_q,\ 0<q<1-\delta$\\
\hline
%\hline
&&\\[-9pt]
$D(\lambda)$&$M(\alpha)$&$-\displaystyle\fr{\alpha\lambda}{\ln\,q+\alpha\lambda}$\\
&&\\[-9pt]
\hline
&&\\[-9pt]
$D(\lambda)$&$E(n,\alpha)$&$\displaystyle\fr{\alpha\lambda}{y(q,n)-\alpha\lambda}$ (см.~(6))\\
&&\\[-9pt]
\hline
&&\\[-9pt]
$M(\theta)$&$D(\mu)$&$\displaystyle-\fr{\ln (1-q)}{\mu\theta+\ln (1-q)}$\\
&&\\[-9pt]
\hline
&&\\[-9pt]
$M(\theta)$&$M(\alpha)$&$\displaystyle\fr{\alpha q}{\theta-(\theta+\alpha)q}$\\
&&\\[-9pt]
\hline
&&\\[-9pt]
$M(\theta)$&$E(n,\alpha)$&$\displaystyle\fr{\alpha((1-q)^{-1/n}-1)}{\theta-\alpha((1-q)^{-1/n}-1)}$\\
&&\\[-9pt]
\hline
&&\\[-9pt]
$E(k,\theta)$&$D(\mu)$&$\displaystyle\fr{y(1-q,k)}{\mu\theta y(1-q,k)}$ (см.~(6))\\
&&\\[-9pt]
\hline
&&\\[-9pt]
$E(k,\theta)$&$M(\alpha)$&$\displaystyle\fr{\alpha}{\theta q^{-1/k}-\theta-\alpha}$\\
&&\\[-9pt]
\hline
&&\\[-9pt]
$E(k,\theta),\  k\ge2$&$E(n,\alpha),\  n\ge2$&$\fr{\alpha}{\theta z(q,k,n)-\alpha}$ (см. (6))\\
\hline
\end{tabular}
\end{center}
\end{table*}

Пусть случайные величины $\lambda$ и~$\mu$ имеют распределения $R(a_\lambda,b_\lambda)$ и $R(a_\mu,b_\mu)$
соответственно ($0<a_\lambda<b_\lambda<a_\mu<b_\mu$, $a_\lambda/a_\mu<b_\lambda/b_\mu$). Воспользовавшись полученной
в разд.~2 формулой для плотности~$f_N(x)$ и соотношениями
\begin{gather*}
\il{A}{B}\fr{x\,dx}{(1+x)^2}=\ln\fr{1+B}{1+A}\,;\\
\il{A}{B}\fr{x^2\,dx}{(1+x)^2}=B+\fr{1}{1+B}-A-\fr{1}{1+A}-2\ln\fr{1+B}{1+A}\,;
\end{gather*}
получим
\begin{multline*}
\e N=\il{0}{\infty}xf_N(x)\,dx=\fr{c_\lambda c_\mu b_\mu^2}{2}\,\ln\fr{a_\mu(b_\mu-
a_\lambda)}{b_\mu(a_\mu-a_\lambda)}-{}\\
{}-\fr{c_\lambda c_\mu a_\lambda^2}{2}\,\ln
\fr{b_\mu- a_\lambda}{a_\mu-a_\lambda}+{}
\end{multline*}
\begin{multline*}
{}+\fr{c_\lambda c_\mu (b_\mu^2-a_\mu^2)}{2}\,\ln
\fr{b_\mu(a_\mu-a_\lambda)}{a_\mu(b_\mu-b_\lambda)}+{}\\
%\end{multline*}
{}+\fr{c_\lambda c_\mu }{2}
\left(2b_\lambda a_\lambda-b_\lambda^2+2\fr{b_\lambda}{c_\lambda}\right)\,\ln
\fr{b_\mu-b_\lambda}{a_\mu-a_\lambda}-{}\\
{}-\fr{c_\lambda c_\mu a_\mu^2}{2}\,\ln
\fr{a_\mu(b_\mu-b_\lambda)}{b_\mu(a_\mu-b_\lambda)}={}\\
{}=\fr{c_\lambda c_\mu}{2}\left[(b_\mu^2-a_\lambda^2)\,\ln(b_\mu-a_\lambda)-{}\right.\\
{}-
(a_\mu^2-a_\lambda^2)\ln\left(a_\mu-a_\lambda\right)+\left(a_\mu^2-b_\lambda^2\right)\ln
\left(a_\mu-b_\lambda\right)-{}\\
\left.{}-\left(b_\mu^2-b_\lambda^2\right)\ln\left(b_\mu-b_\lambda\right)\right]\,.
\end{multline*}

Аналогично получаем выражение для второго момента
\begin{multline*}
\e N^2=\il{0}{\infty}x^2f_N(x)\,dx={}\\
{}=c_\lambda c_\mu\left(a_\mu^2\ln\fr{a_\mu-a_\lambda}
{a_\mu-b_\lambda}-b_\mu^2\ln\fr{b_\mu-a_\lambda}{b_\mu-b_\lambda}\right)\,.
\end{multline*}

Таким образом, в данном примере существуют не только математическое ожидание и дисперсия, но и моменты любого
порядка.

\section{Квантильные характеристики}

Вполне традиционным методом <<борьбы>> с несуществующим математическим ожиданием является рассмотрение такой
характеристики центра распределения, как медианы. Заметим, что при изучении систем с положительным <<дефектом>> опять
сталкиваемся с возможной неопреде\-лен\-ностью этого объекта. Действительно, для собственной функции распределения
квантиль любого порядка существует. Однако для <<дефектных>> функций
распределения существенным является значение величины $\delta$, поскольку определены лишь квантили порядка 
$q\in(0,\,1-\delta)$. Соответствующее замечание относится и к квантильному аналогу дисперсии~--- интерквартильному 
размаху.

В табл.~\ref{t3sh} приведены значения квантилей~$x_q$ порядка $q\in(0,\, 1)$ функций распределения среднего числа
заявок~$N$ для распределений интенсивностей входящего потока~$\lambda$ и обслуживания~$\mu$, рассмотренных
авторами в работах~\cite{KuSh07, KuSh09a, KuSh09b} и отображенных в табл.~\ref{t1sh}.

При нахождении медианы $x_{1/2}$ необходимо выполнение условия $\delta<1/2$ (см.\ табл.~\ref{t2sh}), 
а интерквартильный размах
$x_{3/4}-x_{1/4}$ определен лишь при $\delta<1/4$.

Обозначим через $y(q,n)$ и $z(q,k,n)$ соответственно решения уравнений
\begin{equation}
\sum\limits_{m=0}^{n-1}\fr{y^m}{m!}=qe^y\,;\quad \sum\limits_{m=0}^{n-1}z^m=q(z+1)^{n+k-1}\,.
\label{e6sh}
\end{equation}

Так же как и в случае с моментными характеристиками, не составляет труда найти квантили любого порядка для распределения
с нулевым <<дефектом>>. Например, если входные параметры~$\lambda$ и~$\mu$ имеют равномерные распределения,
заданные в разд.~2, квантиль порядка~$q$ для распределения~$N$ находится из решения уравнения $F_N(x)=q$. Так, при
$$
\fr{a_\lambda(b_\mu-a_\mu)}{2a_\mu(b_\lambda-a_\lambda)}\le q\le\fr{b_\mu b_\lambda+a_\mu b_\lambda-2a_\lambda
b_\mu}{2b_\mu(b_\lambda-a_\lambda)}
$$
квантиль порядка~$q$ равняется
$$
x_q=\fr{2a_\lambda(1-q)+2b_\lambda q}{b_\mu+a_\mu-2a_\lambda(1-q)-2b_\lambda q}\,.
$$
Аналогично находятся квантили при
$$
0<q<\fr{a_\lambda(b_\mu-a_\mu)}{2a_\mu(b_\lambda-a_\lambda)}
$$
и
$$
\fr{b_\mu b_\lambda+a_\mu b_\lambda-2a_\lambda b_\mu}{2b_\mu(b_\lambda-a_\lambda)}<q<1\,.
$$

{\small\frenchspacing
{%\baselineskip=10.8pt
\addcontentsline{toc}{section}{Литература}
\begin{thebibliography}{99}

\bibitem{Shorgin05}
\Au{Шоргин С.\,Я.}
О байесовских моделях массового обслуживания~// 
II Научная сессия Института проблем информатики РАН: Тез. докл.~--- М.: ИПИ РАН, 2005. С.~120--121.

\bibitem{Apice06}
\Au{D'Apice C., Manzo~R., Shorgin S.}
Some Bayesian queueing and reliability models~// Electronic J. Reliability: Theory \& Applications, 
2006. Vol.~1. No.\,4.

\bibitem{KuSh07}
\Au{Кудрявцев А.\,А., Шоргин С.\,Я.}
Байесовский подход к анализу систем массового обслуживания и показателей надежности~// 
Информатика и её применения, 2007. Т.~1. Вып.~2. С.~76--82.

\bibitem{KuShShCh08}
\Au{Kudryavtsev A., Shorgin S., Shorgin~V., Chentsov~V.}
Bayesian queueing and reliability models~// Systems and means of informatics: Spec. issue: 
Mathematical and computer modeling in applied problems.~--- M.: IPI
RAN, 2008. P.~40--53.

\bibitem{KuSh09a}
\Au{Кудрявцев А.\,А., Шоргин С.\,Я.}
Байесовские модели массового обслуживания и надежности: экспоненциально-эрланговский случай~// Информатика и её 
применения, 2009. Т.~3. Вып.~1. С.~44--48.

\bibitem{KuSh09b}
\Au{Кудрявцев А.\,А., Шоргин В.\,С., Шоргин С.\,Я.}
Байесовские модели массового обслуживания и надежности: общий эрланговский случай~// 
Информатика и её применения, 2009. Т.~3. Вып.~4. С.~30--34.

\bibitem{GR71}
\Au{Градштейн И.\,С., Рыжик И.\,М.}
Таблицы интегралов, сумм, рядов и произведений.~--- М.: Наука, 1971. 1108~с.

\bibitem{BoPe95}
\Au{Бочаров П.\,П., Печинкин А.\,В.}
Теория массового обслуживания.~---
М.: РУДН, 1995. 529~с.

\bibitem{Shiryaev80}
\Au{Ширяев А.\,Н.}
Вероятность.~--- М.: Наука, 1980. 576~с.

\label{end\stat}


\bibitem{Feller67}
\Au{Феллер В.}
Введение в теорию вероятностей и ее приложения.~--- М.: Мир, 1967. Т.~2. 752~с.
 \end{thebibliography}
}
}

\end{multicols}   %3


\def\NCAL{{\cal N}}

\newcommand{\suml}[2]{\sum\limits_{#1}^{#2}}   %сумма
\def\sumi{\sum\limits_{i=1}^n}   %сумма по i=1 до n
\newcommand{\maxl}[1]{\max\limits_{#1}}
\newcommand{\minl}[1]{\min\limits_{#1}}
\def\limn{\lim\limits_{n\to\infty}}
\newcommand{\liml}[1]{\lim\limits_{#1}}
\newcommand{\limi}[1]{\liminf\limits_{#1}}
\newcommand{\lims}[1]{\limsup\limits_{#1}}
\newcommand{\cupl}[2]{\bigcup\limits_{#1}^{#2}}


%\newcommand{\rr}{{\rm I\hspace{-0.7mm}\rm R}^3}     % трехмерное пространство
\newcommand{\rr}{{\mathbb{R}}^3}     % трехмерное пространство
%\def\P{\mathop{\bf P}}
\def\limn{\lim\limits_{n\to\infty}}
%\newcommand{\naj}{\hbox{\aj N}}     % поле натуpальных чисел
\newcommand{\rn}{\mathbb{R}^{\mathbb{N}}}     % N-мерное пространство

%\newcommand{\proofbegin}{{\sc Доказательство.}}
%\newcommand{\proofend}{{\hfill$\Box$}}
\newcommand{\sqq}{\hbox{\vrule\vbox{\hrule\phantom{o}\hrule}\vrule}}

\def\stat{matv}

\def\tit{СЕТИ МАССОВОГО ОБСЛУЖИВАНИЯ С~НАИМЕНЬШЕЙ ДЛИНОЙ ОЧЕРЕДИ}

\def\titkol{Сети массового обслуживания с~наименьшей длиной очереди}

\def\autkol{C.\,C.~Матвеева,  Т.\,В.~Захарова}
\def\aut{C.\,C.~Матвеева$^1$,  Т.\,В.~Захарова$^2$}

\titel{\tit}{\aut}{\autkol}{\titkol}

%{\renewcommand{\thefootnote}{\fnsymbol{footnote}}\footnotetext[1]
%{Исследование поддержано грантами РФФИ 08-07-00152 и 09-07-12032.
%Статья написана на основе материалов доклада, представленного на IV 
%Международном семинаре  <<Прикладные задачи теории вероятностей и математической статистики, 
%связанные с моделированием информационных систем>> (зимняя сессия, Аоста, Италия, январь--февраль 2010~г.).}}

\renewcommand{\thefootnote}{\arabic{footnote}}
\footnotetext[1]{Московский государственный университет им.\ М.\,В.~Ломоносова, 
кафедра математической статистики факультета вычислительной математики и кибернетики, petkin@mail.ru}
\footnotetext[2]{Московский государственный университет имени М.\,В.~Ломоносова, 
кафедра математической статистики факультета вычислительной математики и кибернетики, lsa@cs.msu.su}


\Abst{Статья посвящена исследованию свойств
оптимальных размещений по критерию средней суммарной длины очереди в
пространстве для систем с дисциплиной обслуживания FIFO.
Рассматривается поток однородных требований, различающихся лишь
моментами поступления в систему. Станции представляют собой системы
массового обслуживания типа $M|G|1$. В статье дается описание
свойств оптимальных размещений, показаны алгоритмы построения
асимптотически оптимальных размещений, минимизирующих критерий
оптимальности.}

\KW{асимптотически оптимальное размещение;
средняя суммарная длина очереди; критерий оптимальности}

       \vskip 14pt plus 9pt minus 6pt

      \thispagestyle{headings}

      \begin{multicols}{2}

      \label{st\stat}

\section{Введение}

В статье рассмотрена сеть массового обслуживания, для которой ищется размещение станций обслуживания, 
минимизирующее среднюю суммарную длину очереди. Обслуживание заявок в ней производится территориально 
распределенными объектами. В~связи с этим возникают задачи нахождения оптимального или близкого к 
оптимальному размещения станций обслуживания. На практике задачи подобного рода довольно актуальны. 
Среди них задача размещения автозаправочных станций для увеличения прибыльности их функционирования, 
задачи минимизации транспортных расходов или времени. Спецификой изучаемого класса систем является 
необходимость использования информации о положении обслуживающих приборов, положении и числе поступающих 
вызовов, а при некоторых дисциплинах обслуживания -- и других характеристик. В данном случае рассматривается 
модель, в которой станции обслуживания могут быть расположены в произвольных точках некоторого пространства, 
а вызовы являются реализацией некоторой случайной величины, у которой известна плотность распределения. 
Расстояние между точками пространства определяется чебышевской нормой. Сами станции функционируют как 
независимые системы массового обслуживания типа $M|G|1$ с дисциплиной обслуживания FIFO.

Точные формулы для решений задач подобного рода удается получить лишь в исключительных ситуациях. 
Однако часто, применяя различные асимптотические методы, можно получить удовлетворительное для практики 
асимптотическое решение задачи. В~статье приводятся алгоритмы построения асимптотически оптимальных размещений, 
минимизирующих критерий оптимальности, а также исследуются свойства оптимальных размещений по критерию средней 
суммарной длины очереди.

В работе~\cite{3mat} была изучена аналогичная постановка задачи, но с вызовами, распределенными на плоскости. 
В~настоящей статье решается задача с вызовами, распределенными в пространстве~$\mathbb{R}$, 
которая обобщается на случай пространства произвольной размерности.

\section{Постановка задачи}

В пространстве~$\rr$ возникают требования в случайных точках $\xi_1,\xi_2,\dots,$ независимых и
одинаково распределенных с плотностью распределения~$p$. Для обслуживания этих требований
имеется $n$~станций. Моменты поступления требований образуют пуассоновский поток с параметром
$\lambda$. Интенсивность входящего потока~$\lambda$ изменяется с ростом чис\-ла станций~$n$. 
В~случае, когда нужно подчеркнуть эту зависимость, параметр входящего потока будем обозначать~$\lambda(n)$.

\medskip

\noindent
\textbf{Определение 1.} \textit{Размещением $n$~станций обслуживания в пространстве~$\rr$ назовем
множество точек пространства $\{x_1,\ldots,x_n\}$, в которых они расположены.}

\medskip

Обозначать размещение станций будем символом~$x$, т.\,е.\ $x=\{x_1,\ldots,x_n\}$. Станцию
обслуживания и точку пространства, где она расположена, будем обозначать одним и тем же
символом.

\smallskip

\noindent
\textbf{Определение 2.} \textit{Зоной влияния станции~$x_i$ назовем множество~$C_i$ тех точек
пространства, для которых эта станция является ближайшей:
$$
C_i=\{v\in\rr:\|v-x_i\|\leqslant\|v-x_j\|,~~ j=1,2,\ldots,n\}\,.
$$}

\vspace*{-3pt}

Расстояние $\|u-v\|$ между точками~$u$ и~$v$ пространства~$\rr$ задается чебышевской нормой:

\noindent
\begin{multline*}
\|u-v\|=\maxl{1\leqslant i\leqslant 3}\left|u_i-v_i\right|\,, \\
 u=(u_1,u_2,u_3), \quad v=(v_1,v_2,v_3)\,.
\end{multline*}

Станции обслуживают заявки только из своих зон влияния. Обслуживание осуществляется прибором,
двигающимся только по прямой и с постоянной скоростью. При поступлении заявки прибор со
станции перемещается в точку вызова, заявка обслуживается некоторое случайное время~$\eta$, 
затем прибор возвращается обратно на станцию. Дисциплина обслуживания
следующая: если в момент поступления вызова прибор занят, то поступающий вызов ставится в
очередь. После освобождения прибора на обслуживание поступает первая заявка из очереди.

Обозначим через~$\lambda_i$  интенсивность потока вызовов, поступающих на станцию~$x_i$,
$\beta_{i1}$, $\beta_{i2}$~--- соответственно первый и второй моменты времени обслуживания на~$x_i$.

Предполагается, что станции функционируют как независимые системы массового обслуживания типа
$M|G|1$, тогда их средняя суммарная длина очереди~$L(x)$ при размещении~$x$ и условии, что
загрузка каждой станции меньше единицы, т.\,е.\ $\maxl{1\leqslant i\leqslant n} \lambda_i
\beta_{i1}< 1 $, определяется по формуле
%\noindent
$$
L(x)=\suml{i=1}{n}\fr{\lambda_i^2}{2} \,\fr{\beta_{i2}} {1-\lambda_i \beta_{i1}}\,.
$$
\smallskip

Задача заключается в нахождении размещений, минимизирующих введенный критерий оптимальности~$L(x)$.

\smallskip

\noindent
\textbf{Определение~3.} \textit{Размещение~$x^*$ назовем оптимальным, если $L(x^*)\leqslant L(x)$ для
любого размещения~$x$ такого, что $|x|=|x^*|$. Через~$|x|$ здесь обозначено число элементов
размещения~$x$.}

\smallskip

\noindent
\textbf{Определение~4.} \textit{Размещение~$x$ такое, что при $|x|=|x^*|$ выполняется равенство
$$
\limn \fr{L(x)}{L(x^*)}=1\,,
$$
назовем асимптотически оптимальным.}
\columnbreak

%\smallskip

Введем еще ряд обозначений:
\begin{align*}
\e\eta&=\beta_1\,,\quad \e\eta^2=\beta_2\,;\\
C_1&=\int\!\!\!\il{O}\!\!\!\!\int \|u\|\,du_1du_2du_3\,;\\
 C_2&=\int\!\!\!\il{O}\!\!\!\!\int\|u\|^2\,du_1du_2du_3\,,
\end{align*}
где $u=(u_1,u_2,u_3)$, $O$~--- шар единичного объема с центром в нуле.
Можно вычислить точные значения констант: $C_1=3/8$; $C_2=3/20$.

Норму плотности распределения вызовов определим как
$$
|p|_m=\left(\int\!\!\!\int\!\!\!\int p^m(u)\,du_1du_2du_3\right)^{1/m}\,.
$$

Число станций оптимального размещения, попадающих в множество~$A$, обозначим
$
x_A^*=x^*\cap A$ и $\{x\}$~--- последовательность размещений.

\section{Свойства оптимальных размещений}

В первой теореме описываются асимптотические свойства оптимальных размещений для исходной
модели. Для краткости далее везде под интегралом понимается многомерный интеграл Лебега в соответствующем пространстве.

\medskip
\noindent
\textbf{Теоpема 1.}  \textit{Если плотность~$p$~--- ограниченная, интегрируемая по Лебегу
функция, $\e|\xi|^2<\infty$  и интенсивность входящего потока $\lambda(n)=o(n)$,
то для всякой последовательности оптимальных размещений $\{x^*\}$ справедливы
равенства:}
\begin{align*}
1)\ \ \ & \liml{n\rightarrow \infty}\fr{n}{\lambda^2(n)}L(x^*)=0{,}5\beta_2\,;\\
2)\ \ \ &\liml{n\rightarrow\infty}\fr{\left|x_A^*\right|}{n}=\il{A}{}p(u)\,du\,,
\end{align*}
\textit{где $A$~--- измеримое по Лебегу множество пространства~$\rr$.}

\medskip

Рассмотрим модель, в которой обслуживание состоит только лишь в перемещении прибора со станции
до точки, где возникло требование, и обратно. В~этом случае оптимальные размещения обладают
другими свойствами.

\medskip

\noindent
\textbf{Теоpема~2.}  \textit{Если плотность~$p$ ограничена, $p^{3/4}$ интегрируема по Лебегу,
${\e}|\xi|^2<\infty$  и интенсивность входящего потока требований изменяется так, что}
$$
\limn\fr{\lambda(n)}{n^{4/3}}=0\,,
$$
\textit{то для всякой последовательности оптимальных размещений $\{x^*\}$}
\begin{align*}
1)\ \ \ &\limn\fr{n^{5/3}}{\lambda^2(n)}{L(x^*)}=2C_2|p|_{3/4}^2\,;\\
2)\ \ \ &\limn\fr{\left|x_A^*\right|}{n}=|p|_{3/4}^{-3/4}\il{A}{}p^{3/4}(u)\,du
\end{align*}
\textit{для любого измеримого по Лебегу множества~$A$ пространства~$\rr$}.

\medskip

Результаты теоремы~1 обобщаются на пространство~$\rn$.

\medskip

\noindent
\textbf{Теоpема 3.}  \textit{Если плотность~$p$~--- ограниченная, интегрируемая по Лебегу
функция, $\e|\xi|^2<\infty$ и интенсивность входящего потока $\lambda(n)=o(n)$,
то для всякой последовательности оптимальных размещений $\{x^*\}$ справедливы
равенства}
\begin{align*}
1)\ \ \ &\liml{n\rightarrow \infty}\fr{n}{\lambda^2(n)}L(x^*)=0{,}5\beta_2\,;\\
2)\ \ \ &\liml{n\rightarrow\infty}\fr{\left|x_A^*\right|}{n}=\il{A}{}p(u)\,du\,,
\end{align*}
\textit{где $A$~--- измеримое по Лебегу множество пространства~$\rn$}.

В случае, когда $\e\eta=0$, результаты в пространстве~$\rn$ выглядят следующим образом.

\medskip

\noindent
\textbf{Теоpема~4.}  \textit{Если плотность~$p$ ограничена, $p^{N/(N+1)}$ интегрируема по Лебегу,
${\e}|\xi|^2<\infty$  и интенсивность входящего потока требований изменяется так, что
$$
\limn\fr{\lambda(n)}{n^{(N+1)/N}}=0\,,$$
то для всякой последовательности оптимальных размещений $\{x^*\}$}
\begin{align*}
1)\ \ &\limn\fr{n^{(N+2)/N}}{\lambda^2(n)}{L(x^*)}=\fr{0{,}5 N}{N+2}\, |p|_{N/(N+1)}^2\,;\\
2)\ \ &\limn\fr{\left|x_A^*\right|}{n}=|p|_{N/(N+1)}^{-N/(N+1)}\il{A}{}p^{N/(N+1)}(u)\,du
\end{align*}
для любого измеримого по Лебегу множества~$A$ пространства~$\rn$.

%\bigskip

\section{Доказательства для~пространства $\rr$}

В работе~\cite{4mat} представлено доказательство следующих двух лемм.

\medskip

\noindent
\textbf{Лемма~1.} \textit{Если $Q$~--- измеримое по Лебегу подмножество метрического пространства, 
$S$~--- шар с центром в точке $v$ из того же пространства, а меры Лебега множеств $Q$ и $S$ равны,
то
$$
\il{Q}{} a(\|u-v\|)\,du \geqslant \il{S}{} a(\|u-v\|)\,du
$$
для любой неубывающей на $[0,\infty)$ действительной функции~$a(u)$.}

\smallskip

Следующая лемма является аналогом результатов Л.\,Ф.~Тота~\cite{6mat}.

\smallskip

\noindent
\textbf{Лемма 2.} \textit{Пусть  $S_i$,  $i=1,2,\ldots,n$, обозначает шар с центром в нуле,
$\sigma_n$~--- шар с центром в нуле и мерой $\sigma_n=(1/n) \sumi S_i$. Справедливо
следующее нера\-венство:}
$$
\sumi \il{S_i}{} a(\|u\|)\,du \geqslant n \il{\sigma_n}{} a(\|u\|)\,du
$$
\textit{для любой неубывающей на $[0,\infty)$ действительной функции $a(u)$.}
\smallskip

Следующую лемму нам достаточно доказать для случая, когда плотность~$p$~--- простая функция, т.\,е.\ $p=\suml{j=1}{r}p_j
\textbf{1}_{K_j}$~, где $K_j$ -- измеримые по Лебегу непересекающиеся множества.

\smallskip

\noindent
\textbf{Лемма 3.} \textit{Если} ${\e} \eta=0,~ \lambda(n)=o(n^{4/3})$, \textit{то для любой
последовательности оптимальных размещений} $\{x^*\}$
$$
\liml{n\rightarrow\infty}\fr{n}{\lambda^2(n)}L(x^*)=0\,.
$$

%\medskip

\noindent
Д\,о\,к\,а\,з\,а\,т\,е\,л\,ь\,с\,т\,в\,о\,.\ 
Очевидно, что $\lambda^{-2} L(x^*) \geqslant 0$. Оценим сверху~$L(x^*)$. 
Построим размещение~$x^0$ следующим образом. Для~$K_j$ выберем при\-бли\-жа\-ющее его с точностью~$\varepsilon$ 
элементарное множество~$L_j$. Затем~$L_j$ заместим конгруэнтными кубами
объемом $\sigma_j=K_j/m_j$, где
$$
m_j=m(1-\delta) \fr{K_j}{\suml{i=1}{r}K_i}\,,
$$
$m$~--- некоторое натуральное число, $0<\delta<1$.
\pagebreak

Если $\mu(\sigma_j \bigcap K_j)>0$, то в центр $\sigma_j$ помещается
станция обслуживания. Число таких~$\sigma_j$ обозначим через~$n_j$.
$([m\delta]+1)$ станций разместим равномерно на множествах
$(K_j\backslash L_j)\cap K$, где $K$~--- наименьший куб с центром в
нуле, содержащий в себе носитель плотности~$p$. Полученное
размещение обозначим через $x^0=\{x_1,\ldots,x_n \}$.

Для оптимального размещения~$x^*$ справедливы неравенства
\begin{multline*}
0\leqslant\fr{1}{\lambda^2}L(x^*)\leqslant \fr{1}{\lambda^2} L(x^0)\leqslant {}\\
{}\leq
\suml{j=1}{r}n_j \fr{p_j^2 \sigma_j^2 \cdot
2 C_2 \sigma_j^{2/3}}{1-\lambda p_j \sigma_j \cdot 2 C_1 \sigma_j^{1/3}} + o(n^{-5/3})\,,
\end{multline*}
$$
\fr{1}{\lambda^2}L(x^*)\leqslant O\left(n^{-5/3}\right)\,,
$$
так как $\lambda p_j \sigma_j^{4/3}=o(1)$ и
\begin{multline*}
\suml{j=1}{r}n_j p_j^2 \sigma_j^{8/3}=\suml{j=1}{r}p_j^2 K_j \sigma_j^{5/3}(1+o(1))={}\\
{}=\suml{j=1}{r}p_j^2 K_j \cdot
K_j^{5/3} m_j^{-5/3}(1+o(1))={}\\
{}
=(1+o(1))\left(\suml{j=1}{r}p_j^2 K_j\right)\left(\suml{j=1}{r} K_j\right)^{5/3}
n^{-5/3}\,.
\end{multline*}
Следовательно,
$$
\limn\fr{n}{\lambda^2(n)}L(x^*)=0\,. \qquad\qquad\qquad\qquad  \sqq
$$

\smallskip


Пусть $G$~--- некоторый компакт на носителе плотности~$p$,
$D_n(x)=\maxl{1\leqslant i\leqslant n}diam A_i$, $A_i=\{u\in$\linebreak $\in A_i^{'} : p(u)>0 \}$, $A_i^{'}$~--- 
зона влияния~$x_i$ на компакте~$G$.

\medskip

\noindent
\textbf{Лемма~4.} \textit{Если для размещения $\{x\}$}
$$
\liml{n\rightarrow\infty} \fr{n}{\lambda^2(n)}\,L(x)=0\,,
$$ 
\textit{то}
$$
\liml{n\rightarrow\infty} D_n(x)=0\,.
$$

\smallskip

\noindent
Д\,о\,к\,а\,з\,а\,т\,е\,л\,ь\,с\,т\,в\,о\,.\ 
Предположим, что $\lims{n \to \infty~~~} D_n(x)=d>0$. Выберем подпоследовательность размещений такую, 
что для некоторой станции~$y_n$ из $n$-го
размещения этой подпоследовательности $\limn \|u_n-v_n\|=d$, где $u_n, v_n \in C_n$;
$C_n=\{u\in C_n' : p(u)>0 \}$; $C_n'$~--- зона влияния станции~$y_n$ на~$G$.

Ввиду компактности~$G$ без ограничения общ\-ности можно считать, что $y_n \to y_0$, $u_n \to u_0$,
$v_n \to v_0 $.

Пусть $\|u_0 - y_0\|\geqslant d/2$. Это означает, что лишь конечное число размещений из
выбранной подпоследовательности имеет станции в $d/4$-окрестности точки~$u_0$. Через~$R$
обозначим $d/8$-окрестность точки~$u_0$.
%
Поскольку $u_0$ является предельной точкой~$C_0$ и~$C_0$~--- подмножество носителя плотности,
то для некоторого $\varepsilon>0$ 
$$
\il{R}{}p(u)\,du > \varepsilon\,.
$$

Оценим снизу значение критерия~$L$ на размещениях, не имеющих станций в $d/4$-окрест\-ности
точки~$u_0$.
%
Поскольку в нашем случае $\beta_1=\beta_2=0$, то второй момент времени обслуживания $i$-й заявки 
$$
\beta_{i2}=\fr{4}{{\p} (C_i)}\il{C_i}{}\|u-x_i\|^2 p(u)\,du\,.
$$

Так как загрузка на каждой станции обслуживания меньше~1, то
\begin{multline*}
L(x)=\suml{i=1}{n}\fr{\lambda_i^2}{2} \,\fr{\beta_{i2}} {1-\lambda_i \beta_{i1}}\geqslant
0{,}5\lambda^2\suml{i=1}{n} {\p}^2(C_i)\beta_{i2}={}\\
{}=2\lambda^2\suml{i=1}{n} {\p}(C_i)\il{C_i}{}\|u-x_i\|^2 p(u)\,du\,.
\end{multline*}

Учитывая, что на множестве~$R$ $\|u-x_i\|>d/8$ для любого~$i$ на выбранном размещении, 
а также выпуклость функции $f(u,v)=u v$, имеем
\begin{multline*}
L(x)\geqslant 2\lambda^2\suml{i=1}{n} {\p}(C_i)\cdot \fr{1}{n}\suml{i=1}{n}\il{C_i}{}\|u-x_i\|^2 p(u)~du \geqslant{}\\
{}\geq 2\lambda^2 \fr{1}{n}\left(\fr{d}{8}\right)^2 \varepsilon\,.
\end{multline*}
Поэтому
$$
\fr{n}{\lambda^2(n)}L(x)\geqslant \fr{\varepsilon d^2}{32}\,.
$$
Отсюда
$$
\liml{n\to\infty}\fr{n}{\lambda^2(n)}L(x)> 0\,,
$$
что противоречит условию леммы.

Поэтому $d=0$. \hfill\sqq

Последняя лемма показывает, что на носителе плотности распределения вызовов диаметры зон влияния на 
любом компакте стремятся к нулю на последовательности оптимальных размещений.

Для случая ${\e}\eta>0$ также справедливы соответствующие леммы, доказательство 
которых сходно с доказательством лемм~3 и~4.

\medskip

\noindent
\textbf{Лемма 5.} \textit{Если ${\e} \eta>0,~ \lambda(n)=o(n)$, то для любой
последовательности оптимальных размещений} $\{x^*\}$
$$
\liml{n\rightarrow\infty}\fr{1}{\lambda^2(n)}L(x^*)=0\,.
$$

\medskip

\noindent
\textbf{Лемма 6.} \textit{Если для последовательности размеще\-ний~$\{x\}$}
$\liml{n\rightarrow\infty} \fr{1}{\lambda^2(n)}\,L(x)=0\,,$ 
\textit{то}
$$
\liml{n\rightarrow\infty} D_n(x)=0\,.
$$

\smallskip

\noindent
Д\,о\,к\,а\,з\,а\,т\,е\,л\,ь\,с\,т\,в\,о\ теоремы~1.
Для любой станции~$x_i$ с зоной влияния~$C_i$ некоторого размещения~$x$ первые два момента
времени обслуживания оцениваются как
\begin{align*}
\beta_{i1}&={\e} ((2\|\xi-x_i\|+\eta)|\xi\in C_i)\geqslant \e\eta=\beta_1\,;
\\
\beta_{i2}&={\e} ((2\|\xi-x_i\|+\eta)^2|\xi\in C_i)\geqslant \e\eta^2=\beta_2\,,
\end{align*}
а интенсивность потока поступающих на нее требований есть $\lambda_i=\lambda(n) {\p} (C_i)$.

С учетом этого, а также выпуклости функции $f(u,v)=u^2(1-v)^{-1}$~, оценим снизу
критерий~$L(x)$.
\begin{multline*}
L(x)=\suml{i=1}{n}\fr{\lambda_i^2}{2} \fr{\beta_{i2}} {1-\lambda_i \beta_{i1}}\geqslant{}\\
{}\ge
0{,}5\lambda^2\beta_2\suml{i=1}{n} \fr{{\p}^2(C_i)} {1-\lambda \beta_1 {\p}(C_i)}\geqslant{}\\
{}
\!\geqslant 0{,}5\lambda^2\beta_2 \fr{1}{n}\left(\suml{i=1}{n}{\p}(C_i)\right)^{\!2}\!\! \left(1-\lambda
\beta_1 \fr{1}{n}\,\suml{i=1}{n}{\p}(C_i)\right)^{\!\!-1}\!\!\!,\hspace*{-0.8pt}
\end{multline*}
т.\,е.
$$
\fr{n}{\lambda^2(n)}L(x^*)\geqslant 0{,}5\beta_2\left(1-\fr{\lambda \beta_1}{n}\right)^{-1}\,.
$$
Устремляя~$n$ к бесконечности, получим
$$
\limi{n \to \infty}\fr{n}{\lambda^2(n)}\,L(x^*)\geqslant 0{,}5\beta_2\,.
$$
\smallskip

Получим теперь верхнюю оценку. Предположим сначала, что плотность~$p$~--- простая функция, $p=\suml{j=1}{r}p_j
\textbf{1}_{K_j}$, где $K_j$~--- измеримые по Лебегу непересекающиеся множества.

Для такой плотности построим асимптотически оптимальное размещение~$x$, т.\,е.\ такое, что при
$|x|=|x^*|$ выполняется равенство
$$
\limn \fr{L(x)}{L(x^*)}=1\,.
$$

Для этого каждое множество~$K_j$ заменим
элементарным множеством~$L_j$ таким, что $\mu(K_j \Delta L_j)<$\linebreak $<\varepsilon$, затем~$L_j$
заместим конгруэнтными кубами, пересекающимися лишь по границе, и объемом
$\sigma_j=K_j/m_j$, где
$m_j=m (1-\delta) p_jK_j$; $m$~--- некоторое натуральное число и $0<\delta<1$.

Если $\mu(\sigma_j \bigcap L_j)>0$, то в центр куба~$\sigma_j$ помещается станция. Пусть
$n_j$~--- число таких станций. Поскольку размещение станций реализуется на компакте, можно построить наименьший 
куб~$K$, который содержит в себе полностью носитель плотности~$p$. $([m\delta]+1)$~станций 
разместим равномерно на множествах $(K_j\backslash L_j)\cap K$. Тем самым
получим некоторое размещение $x=\{x_1,\ldots,x_n\}$~, для которого
\begin{multline*}
L(x)\leqslant{}\\
{}\le \suml{j=1}{r}n_j\fr{\lambda^2 p_j^2 \sigma_j^2}{2}\, \fr{\beta_2+4\beta_1
C_1\sigma_j^{1/3}+4C_2\sigma_j^{2/3}} {1-\lambda \beta_1 p_j \sigma_j-2\lambda C_1p_j
\sigma_j^{4/3}}={}\\
{}= \fr{\lambda^2}{2}\,\suml{j=1}{r} \fr{\beta_2 p_j K_j
n^{-1}+o(n^{-1})}{1-\lambda \beta_1 n^{-1}+o(n^{-1})}\,.
\end{multline*}
Устремляя $m$, а тем самым и~$n$ к бесконечности, получаем
$$
\lims{n \to \infty}\fr{n}{\lambda^2(n)}L(x)\leqslant 0{,}5\beta_2\,.
$$
Так как всегда $L(x^*)\leqslant L(x)$, то с учетом нижней оценки для~$L(x^*)$ получаем, что
$x$~--- аcимптотически оптимальное размещение и
$$
\liml{n \to \infty}\fr{n}{\lambda^2(n)}L(x^*) = 0{,}5\beta_2\,.
$$
Пусть $p$~--- произвольная функция, удовле\-тво\-ря\-ющая условиям доказываемой теоремы. Введем
прос\-тые функции~$\bar{p}_k(u)$ по правилу
$\bar{p}_k(u)=$\linebreak $=(m+1)/k$, если $m/k<p(u)\leqslant (m+4)/k$ для $k\in \NCAL, m=0,1,\ldots$

Очевидно, что $p(u)\leqslant \bar{p}_{k}(u)$, а для простых функций уже была получена
предельная оценка сверху, поэтому справедливо следующее неравенство:
$$
\lims{n \to \infty}\fr{n}{\lambda^2(n)}\,L(x^*)\leqslant 0{,}5\beta_2 \left|\bar{p}_k \right|_1\,.
$$
При $k \to \infty$\ $\left|\bar{p}_k \right|_0 \to \left|p\right|_1=1$. И с учетом оценки
снизу для~$L(x^*)$ получаем, что
$$
\liml{n \to \infty}\fr{n}{\lambda^2(n)}\,L(x^*) = 0{,}5\beta_2\,.
$$
Докажем второй пункт теоремы~1.
\pagebreak

Рассмотрим размещение $x_A=x\bigcap A$. Пусть $k=\left|x_A\right|$~--- число станций,
попадающих в множество~$A$ при размещении~$x$. Определим~$L(x_A)$ как~$L(x)$ для~$p\mathbf{1}_A$.
\begin{multline*}
L(x_A)=\suml{i=1}{k}\fr{\lambda_i^2}{2}\,\fr{\beta_{i2}} {1-\lambda_i \beta_{i1}}\geqslant{}\\
{}\geq
0{,}5\lambda^2\beta_2\suml{i=1}{k} \frac{{\p}^2(C_i\bigcap A)} {1-\lambda \beta_1
{\p}(C_i\bigcap A)}\geqslant{}\\
{}\geqslant 0{,}5\lambda^2\beta_2 \fr{1}{k}\left(\suml{i=1}{k}{\p}(C_i\bigcap A)\right)^2
\left(
\vphantom{\suml{i=1}{k}}
1-{}\right.\\
\left.{}-\lambda \beta_1 \fr{1}{k}\,\suml{i=1}{k}{\p}(C_i\bigcap A)\right)^{-1}\,,
\end{multline*}
значит
$$
\fr{k}{\lambda^2(n)}L(x_A)\geqslant 0{,}5\beta_2{\p}^2(A)\left(1-\fr{\lambda \beta_1
{\p}(A)}{k}\right)^{-1}\,.
$$
Следовательно,
$$
\limi{n \to \infty}\fr{k}{\lambda^2(n)}\,L(x_A)\geqslant 0{,}5\beta_2{\p}^2(A)\,.
$$
Пусть $\gamma_1$~--- предельная точка последовательности $\left\{\left|x_A\right|n^{-1}
\right\}$.
\smallskip

Пусть теперь~$x$~--- асимптотически оптимальное размещение для критерия~$L$, тогда
$$
\lims{n \to \infty}\fr{n}{\lambda^2(n)}\,L(x_A) \leqslant 0{,}5\beta_2{\p}(A)\,.
$$
Из двух последних неравенств следует, что
$$
\gamma_1 \geqslant{\p}(A)\,.
$$
Пусть $\gamma_2$~--- предельная точка последовательности $\left\{\left|x_B\right|n^{-1}
\right\}$~, где $B=A^c$.

Для $\gamma_2$ аналогично доказывается соответству\-ющее неравенство
$$
\gamma_2 \geqslant {\p}(B)\,.
$$

Так как
$$
1=\gamma_1+\gamma_2\geqslant {\p}(A)+{\p}(B)=1\,,
$$
то в неравенстве достигнуто равенство. Это возможно только, если
$$
\gamma_1={\p}(A)\,, \qquad \gamma_2={\p}(B)\,.
$$
Тем самым для любого асимптотически оптимального размещения
$$
\liml{n\rightarrow\infty}\fr{\left|x_A\right|}{n}=\il{A}{}p(u)\,du\,,
$$
а, значит, это равенство верно и для оптимального размещения. Отсюда следует утверждение
второго пункта теоремы~1.

Доказательство теоремы~2 проводится аналогично доказательству теоремы~1.

Сначала оценивается значение критерия~$L(x)$ снизу
$$
\limi{n \to \infty} \fr{n^{5/3}}{\lambda^2(n)}L(x^*)\geqslant 2C_2|p|_{3/4}^2\,.
$$

Для оценки сверху~$L(x^*)$ строится асимптотически оптимальное размещение~$x$, алгоритм
по\-стро\-ения которого несколько иной.

Выберем последовательность вложенных расширяющихся кубов~$K$ с центром в нуле таких, что
${\e}|\xi|^2 \mathbf{1}_{K^c}=o(m^{-2})$.

Предположим сначала, что плотность $p$~--- прос\-тая функция, определенная так же, как и ранее.

Для каждого $K_j$ выберем элементарное множество~$L_j$, чтобы
$$
\mu(K_j \Delta L_j)<\varepsilon/K\,,\enskip \forall j\,.
$$

Каждое множество~$L_j$ покрываем правильной решеткой объемом
$\sigma_j=K_j/m_j$, где
$$
m_j=m(1-\delta) \fr{p_j^{3/4}K_j}{\suml{i}{} p_i^{3/4}K_i}\,,
$$
$m$~--- некоторое натуральное число и $0<\delta<1$.

Если $\mu(\sigma_j\cap L_j)>0~,$ то в центры таких~$\sigma_j$ помещаем станцию
обслуживания. Число таким образом размещенных станций обозначим через~$n_j$. 
$([m\delta]+1)$~станций разместим равномерно на множествах $(K_j\backslash L_j)\cap K$.

Общее число размещенных станций обозначим через~$n$. Тем самым получим некоторое размещение
$x=\{x_1,\ldots,x_n\}$, для которого с учетом леммы~4 справедливы следующие оценки:
\begin{multline*}
L(x)\leqslant \fr{\lambda^2}{2} \suml{j=1}{r} n_j\, \fr{p_j^2 \sigma_j^{8/3} 4C_2}
{1-2\lambda C_1 p_j \sigma_j^{4/3}}+{}\\[6pt]
{}+ p_j^2 (m \delta)^{-5/3} \varepsilon +o(m^{-5/3}) ={}\\[6pt]
{}
=\lambda^2~ 2~ C_2(1+o(1))\suml{j=1}{r} \fr{p_j^{3/4} K_j }{1+o(1)}~
\left(\suml{i=1}{r} p_i^{3/4} K_i\right)^{5/3}\times{}\\[6pt]
{}\times \left(n(1-\delta)\right)^{-5/3}+O(\varepsilon n^{-5/3})\,.
\end{multline*}
Отсюда следует, что
$$
\lims{n \to \infty}\fr{n^{5/3}}{\lambda^2(n)}L(x)\leqslant 2C_2|p|_{3/4}^2\,.
$$
Следовательно,
$$
\limn\fr{n^{5/3}}{\lambda^2(n)}L(x^*)=2C_2|p|_{3/4}^2\,.
$$

Используя теорему Лебега о предельном переходе под знаком интеграла, 
полученный результат можно обобщить на случай произвольной (в рамках 
ограничений доказываемой теоремы) плотности~$p$.

Доказательство второго пункта проводится так же, как и в теореме~1.

\section{Заключение}

В предлагаемой работе исследованы асимптотические свойства оптимальных 
размещений для исходной модели, описано поведение критерия на последовательности 
оптимальных размещений, найдена предельная оптимальная плотность. Для наглядности 
подробные доказательства приведены для случая трехмерного пространства, одна-\linebreak ко 
полученные результаты можно обобщить на\linebreak случай пространства произвольной конечной 
раз\-мер\-ности. Интересно отметить, что при $\beta_1 >0$ предельная плотность 
размещений не зависит от размерности пространства и эквивалентна, в некотором 
смысле, плотности распределения вызовов.

Полученные результаты носят не только теоретический, но и практический характер и могут быть 
применены для изучения реальных систем. Найденные алгоритмы построения асимптотически оптимальных 
размещений допускают реализацию на программном уровне.

{\small\frenchspacing
{%\baselineskip=10.8pt
\addcontentsline{toc}{section}{Литература}
\begin{thebibliography}{9}


\bibitem{3mat} %1
\Au{Захарова Т.\,В.} 
Размещение систем массового обслуживания, минимизирующее среднюю длину очереди~// Информатика и её применения, 2008. 
Т.~2. Вып.~1. С.~63--66.

\bibitem{4mat} %2
\Au{Захарова Т.\,В.} 
Оптимизация расположения станций обслуживания в пространстве~// Информатика и её применения, 2008. Т.~2. Вып.~2. С.~41--46.


\label{end\stat}

\bibitem{6mat} %3
\Au{Тот Ф.\,Л.} 
Расположение на плоскости, на сфере и в пространстве.~--- М.: ГИФМЛ, 1958.

%\bibitem{1mat} 
%\Au{Назаров Л.\,В., Смирнов С.\,Н.} 
%Обслуживание вызовов, распределенных в пространстве~// Изв. АН СССР. Техн. кибернет., 1982. №\,1. С.~95--99.

%\bibitem{2mat}
%\Au{Захарова Т.\,В.} 
%Оптимальные размещения систем массового обслуживания с дисциплиной обслуживания FIFO~// Вест. Моск. ун-та. Cер.~15. 
%Вычисл. матем. и киберн., 2007. №\,4. С.~32--37.

%\bibitem{5mat} 
%\Au{Ивченко Г.\,И., Каштанов В.\,А., Коваленко~И.\,Н.} 
%Теория массового обслуживания.~--- М.: Высшая школа, 1982.

 \end{thebibliography}
}
}


\end{multicols} %4

\def\stat{borodina}

\def\tit{ОБ  ОЦЕНИВАНИИ ЭФФЕКТИВНОЙ ПРОПУСКНОЙ СПОСОБНОСТИ
 СИСТЕМЫ С РЕГЕНЕРАТИВНЫМ\\ ВХОДНЫМ ПРОЦЕССОМ$^*$}

\def\titkol{Об  оценивании эффективной пропускной способности
 системы с регенеративным входным процессом}

\def\autkol{А.\,В.~Бородина,  Е.\,В.~Морозов}

\def\aut{А.\,В.~Бородина$^1$,  Е.\,В.~Морозов$^2$}

\titel{\tit}{\aut}{\autkol}{\titkol}

{\renewcommand{\thefootnote}{\fnsymbol{footnote}}\footnotetext[1]
{Работа выполнена при финансовой поддержке Программы
стратегического развития ПетрГУ   в рамках реализации комплекса
мероприятий  по развитию научно-исследовательской деятельности.}}

\renewcommand{\thefootnote}{\arabic{footnote}}
\footnotetext[1]{Институт прикладных
математических исследований Карельского научного центра Российской
академии наук; Петрозаводский государственный университет, borodina@krc.karelia.ru}
\footnotetext[2]{Институт прикладных математических исследований
Карельского научного центра Российской академии наук; Петрозаводский
государственный университет, emorozov@karelia.ru}


\Abst{Рассматривается  понятие эффективной
пропускной способности (ЭПС) коммуникационного узла, которая
гарантирует, что вероятность потери или превышения стационарной
нагрузкой некоторого уровня ограничена заданной (малой) величиной.
Показано, как вычисляется ЭПС в    жидкостной системе обслуживания в
случае входного процесса  с независимыми приращениями.  Далее
рассматривается жидкостная система с регенеративным входным
процессом. Для вычисления ЭПС ключевым является нахождение
предельной логарифмической экспоненциальной функции моментов
входного процесса. С использованием эвристических соображений
получена аппроксимация этого предела, которая  выражена в терминах
моментов длины цикла регенерации и величины работы, поступающей  в
систему в течение цикла.
Результаты численного моделирования ряда систем с
 регенеративным входным процессом показывают вполне удовлетворительную точность
оценивания вероятности потери в случае, когда  в системе
используется оценка ЭПС, получаемая на основе найденной
аппроксимации.}

\KW{система с постоянной скоростью обслуживания; эффективная пропускная
 способность; регенеративный процесс; регенеративная оценка; стационарный процесс нагрузки; вероятность потери}
 
 \vskip 14pt plus 9pt minus 6pt

      \thispagestyle{headings}

      \begin{multicols}{2}

            \label{st\stat}

\section{Введение}

В современных коммуникационных системах  одним из важнейших
показателей  качества  обслуживания (QoS) является вероятность
превышения стационарным процессом нагрузки~$W$ (незавершенной
работы) некоторого (большого) уровня~$b$. Для системы с конечным
буфером~$b$ указанная  вероятность является вероятностью потери.

Интерес к  системам с регенеративным входным процессом обусловлен
тем, что такие процессы сохраняют  свойство регенерации при
прохождении  через узлы коммуникационной сети~\cite{MorozovTechCyb87, MorozovOutput}. 
В~то же время пуассоновский
процесс   и даже общий процесс восстановления не обладают таким
свойством. (Исключением являются пуассоновские потоки в стационарной
сети Джексона без циклов, со\-сто\-ящей из узлов вида $M/M/1$.)
 Различные вопросы, связанные
с вычислением ЭПС в системе с регенеративным входным процессом,
рассматривались  в работах~[3--7], в
которых также предложено оценивать ЭПС на основе {\it
регенеративной} оценки, опирающейся на группировку данных по циклам
регенерации. Отметим  также близкую по тематике предшествующую
работу~\cite{Crosby}.   Важной работой в
 области исследования ЭПС является обзорная статья~\cite{Kelly}.

Если уровень $b$ задан, а  мощность   обслуживающего устройства~$C$
(величина работы, которую прибор может сделать за единицу времени)
можно изменять, то естественная задача QoS состоит в   выборе такого
значения~$C$, которое гарантирует, что  стационарная загрузка не
превысит уровня~$b$ с заданной (малой) вероятностью~$\Gamma$, т.\,е.\
\begin{equation}
\p(W>b)\le \Gamma\,.\label{1-bor}
\end{equation}
Минимальная  величина мощности~$C$, удовлетворяющая этому  условию,
и называется  \textit{эффективной пропускной способностью} сис\-темы.

Покажем, как решается  задача вычисления  ЭПС для системы с одним
обслуживающим устройством и неограниченным  буфером  для ожидающих
заявок. Удобно считать, что
  обслуженная работа поступает в систему и покидает ее
 в  (целочисленные) моменты~$t$   и что   величина $W(t)$
 равна  незавершенной работе в момент $t-1$ c учетом работы,
 поступившей в момент $t-1$,  и за вычетом  работы,
 покинувшей систему в момент $t-1,\,t=0,\,1,\ldots$~\cite{Lewis}.

 Предполагается, что объем уходящей работы в каждый  момент (дискретного) времени
равен~$C$, что согласуется с <<объемом жидкости>>, вытекшей из
системы в течение  интервала времени длины~1.
 Пусть $v_i$~--- величина работы,
поступившей в сис\-те\-му в момент~$i$, и тогда
величина $V(t):=\sum\limits_{i=0}^{t-1} v_i$ есть суммарная работа,
поступившая в систему в интервале  $[0,\,t-1]$. Обозначим через
 $W(t)$  незавершенную работу по
обслуживанию заявок, находящихся в системе в момент  времени
$t=0,1,\ldots$, полагая  $W(0)\hm=0$.
 Предположим, что $\{v_i,\,i\ge 0\}$~--- независимые, одинаково
распределенные случайные величины (н.о.р.с.в.), причем,  вообще
говоря, $\p(v_i=0)\hm>0$. Очевидно, имеет место такая рекурсия Линдли
(в дискретном времени):
\begin{equation}
W(t+1)=[W(t)+v_{t}-C]^+\,,\enskip t\ge0\,,
\label{2-bor}
\end{equation}
откуда следует, что процесс $\{W(t)\}$ образует марковскую цепь (с
общим пространством состояний). Обозначим $X_i\hm=v_i\hm-C$ и введем
случайное  блуждание
\begin{multline}
Z(t):=\sum\limits_{i=0}^{t-1} X_i=\sum\limits_{i=0}^{t-1}(v_{i}-C)={}\\{}=V(t)-C\,
t\,,\enskip t\ge1\,,\label{3-bor}
\end{multline}
где положено   $Z(0)\hm=0$. Обозначим  типичный шаг этого блуждания
через $X\hm=v\hm-C$. (Здесь и далее типичный элемент последовательности
н.о.р.с.в.\ обозначается без соответствующего индекса.) Так как
буфер для ожидания не ограничен, то предполагается  выполненным
следующее условие отрицательного сноса у случайного блуждания~$Z$:
\begin{equation}
\e X=\e v- C:=\lambda-C<0\,. \label{4-bor}
\end{equation}
Заметим, что~(\ref{4-bor}) является условием стационарности
процесса (незавершенной) нагрузки $\{W(t),\ t\hm\ge 0\}$.
Действительно, опираясь на рекурсию~(\ref{2-bor}), определим приращение
этого процесса $ \Delta(t)\hm=W(t+1)\hm-W(t)$
 между моментами  $t-1$ и $t$. Предположим, что марковская цепь
\begin{equation}
W(t)\stackrel{d}{\to} \infty\,,\enskip t\to   
\infty\,,
\label{5b-bor}
\end{equation} 
где $\stackrel{d}{\to}$  означает сходимость по вероятности.
 Поскольку $\Delta(t)\hm\le  v_t$ и $\e v\hm=\lambda\hm<\infty$, то  из~(\ref{2-bor}) 
 легко следует, что
$ \e \Delta (t)\hm\to \lambda\hm-C\hm<0$, $t\hm\to \infty. $ Этот результат,
как легко проверить, противоречит сходимости~(\ref{5b-bor}) и  поясняет
термин <<отрицательный снос>>. Для широкого класса цепей Маркова
последний результат, в свою очередь, влечет существование
стационарного процесса $W(t)\hm\Rightarrow W$~\cite{Asmus}. (Знак
$\Rightarrow $ обозначает сходимость по распределению.)

Данная статья является  продолжением работы~\cite{KRC}, в которой
основное внимание было уделено сравнению оценки по методу группового
среднего (batch mean) с регенеративной оценкой. Здесь представ\-ле\-на
более подробная мотивировка эвристических соображений, позволяющих
получить искомую аппроксимацию ЭПС   в жидкостной сис\-те\-ме  с
регенеративным входным процессом. Кроме того, представлены
результаты  численных экспериментов, которые подтверждают, что
полученная аппроксимация действительно может быть эффективно
использована для оценивания ЭПС в рассматриваемых сис\-те\=мах
обслуживания. В~разд.~2 показано как вычисляется  ЭПС,
удовлетворяющая условию~(\ref{1-bor}), в случае н.о.р.c.в.\
$\{v_i\}$.

\section{Вероятность большого уклонения и~эффективная пропускная способность}

Рассмотрим  асимптотику вероятности большого уклонения стационарной
нагрузки для рас\-смот\-рен\-ной выше системы с н.о.р.\ $\{v_i\}$ и с
{\it заданной скоростью обслуживания}~$C$. Напомним известный
результат~\cite{Asmus}:
\begin{multline}
W(n)=
 % V(t) -Ct-\inf_{u\le t}(V(u) -Cu)=
\sup\limits_{0\le t< n}[V(t)-C\,t]^+={}\\
{}=\sup\limits_{0\le t<n}Z(t)\,,\enskip
n=1,\ldots\,,
\label{4a-bor}
\end{multline}
связывающий  величину незавершенной работы с максимумом
 случайного блуждания~(\ref{3-bor}).  (Поскольку $Z(0)\hm=0$, то
$\sup[\cdot]=\sup[\cdot]^+$ в  соотношении~(\ref{4a-bor}), что
согласуется с~(\ref{2-bor}) и гарантирует $\inf_nW(n)\hm\ge 0$.)
 Отметим, что марковская цепь $\{W(t)\}$ с дискретным временем
регенерирует в последовательные моменты  опустошения системы, т.\,е.\ в моменты
$$
\beta_{n+1}=\min\{k>\beta_n: W(k)=0\}\,,\enskip n\ge 0\,,\ \beta_0:=0\,.
$$
Длина (типичного) цикла регенерации процесса $\alpha:=\beta_1$ есть
непериодическая с.в., поскольку ввиду~(\ref{4-bor}) выполнено условие:
\begin{multline}
\p(\alpha=1)=\p(W(t+1)=0|W(t)=0)={}\\{}=\p(v<C)>0\,. \label{6a-bor}
\end{multline}
Как хорошо известно, из~(\ref{4-bor}) и~(\ref{6a-bor}) следует, что $\e
\alpha\hm<\infty$, т.\,е.\ процесс $\{v_i\}$ является {\it положительно
возвратным} и, кроме того, стационарный процесс нагрузки~$W$
существует и является пределом  по распределению:
\begin{equation*}
W(t)\Rightarrow W=_{\mathrm{st}}\sup_{n\ge 0} Z(n)\,, %\label{11a-bor}
\end{equation*}
где $=_{\mathrm{st}}$  обозначает стохастическое равенство. % \cite{Asmus}.
 Предположим, что в некоторой положительной
 окрестности  $(0,\,\theta_0)$ параметра~$\theta$  выполнено условие:
\begin{equation}
\e e^{\theta v}<\infty\,, \label{7-bor}
\end{equation}
и приведем ряд  необходимых далее   понятий и результатов теории
 больших уклонений~\cite{Ganesh}.
Рассмотрим (нормированную)  {\it логарифмическую производящую
функцию моментов} случайного блуждания $Z(n)$
\begin{equation*}
\Lambda_n(\theta):=\fr{1}{n}\ln\e e^{\theta Z(n)}\,,\enskip n\ge1\,,
\end{equation*}
которая ввиду независимости
 слагаемых $\{X_i\}$, формирующих случайное блуждание $Z(n)$, может
 быть записана как
\begin{equation*}
\Lambda_n(\theta):=\Lambda(\theta)=\ln \e e^{\theta v}-\theta C\,,\enskip
n\ge 1\,. %\label{5a-bor}
\end{equation*}
Поскольку  функция $\psi(\theta):=\ln \e e^{\theta v}$  выпуклая,
$\psi(0)\hm=0$ и ввиду   условия~(\ref{4-bor})  $\psi^{\prime}(0)\hm=\e v\hm<C$, то
существует   единственный корень $\theta^*\hm>0$ уравнения
\begin{equation}
\ln \e e^{\theta v}=\theta C\,. \label{6-bor}
\end{equation}
   Будем предполагать, что с.в.~$v$ такова, что
 $\theta^*\hm\in (0,\,\theta_0)$.
Из работы~\cite{Glynn} следует, что при выполнении  условий~(\ref{4-bor}), (\ref{7-bor})
 стационарный процесс~$W$
 удовлетворяет   следующему асимптотическому соотношению~\cite{Ganesh, Glynn}:
\begin{equation}
\lim\limits_ {x\to \infty}\fr{1}{x}\ln \p(W> x)=-\theta^*\,. \label{10b-bor}
\end{equation}
Это  влечет такую экспоненциальную аппроксимацию  вероятности
большого уклонения стационарного процесса нагрузки:
%\begin{eqnarray}
 $\p(W> x)=e^{-\theta^*x+o(x)}$, $x\hm\to \infty.$
 Ввиду~(\ref{6-bor}) ЭПС равна
\begin{equation}
C=\fr{\ln\e e ^{\theta^* v}}{\theta^*} \label{12b-bor}
\end{equation}
и удовлетворяет условию~(\ref{4-bor}) (см.\ лемму~9.1.5 в работе~\cite{Chang}).

Отметим, что в  случае, когда с.в.\ $\{v_i\}$ являются зависимыми,
 асимптотика~(\ref{10b-bor}) также  может быть доказана в некоторых 
 случаях~[12--14]. В~частности, это верно, если
последовательность $\{v_i\}$~---  стационарная с перемешиванием~\cite{Lewis}.

Ключевыми для справедливости  асимптотики~(\ref{10b-bor}) являются
свойства   логарифмической производящей функции моментов {\it
входного процесса}
\begin{equation*}
%
\Lambda_V^{(n)}(\theta):=\fr{1}{n} \ln \e e ^{\theta V(n)}\,.
%\label{10-bor}
\end{equation*}
В первую очередь требуется   существование   {\it предела
Гарт\-не\-ра--Эл\-лиса}~\cite {Glynn, Chang}
\begin{equation*}
\lim\limits_{n\to \infty} \Lambda_V^{(n)}(\theta)=\Lambda _V
(\theta)\,,\enskip n\to \infty\,, %\label{10a-bor}
\end{equation*}
а также  условие  его конечности либо для всех $\theta\hm\in
(0,\,\theta_0)$~\cite{Ganesh, Glynn}, либо для всех $\theta\hm>0$~\cite{Chang}.

Когда скорость   $C$  задана,
 параметр $\theta^*$, найденный из условия~(\ref{6-bor}),
дает  скорость (экспоненциального) убывания хвоста вероятности
$\p(W>b)$. Рассмотрим теперь обратную задачу: скорость~$C$ (т.\,е.\
ЭПС) должна быть выбрана так, чтобы   обеспечить  требование~(\ref{1-bor}). 
ЭПС должна обеспечивать  условие стационарности~(\ref{4-bor})
и поэтому можно опираться на экспоненциальную асимптотику~(\ref{10b-bor}). 
Это дает такое (приближенное) уравнение:
 \begin{equation}
 \p(W>b)=\Gamma=e^{-\theta b}\,,\label{16a-bor}
 \end{equation}
решение которого имеет вид
\begin{equation}
 \theta^*=-\fr{\ln \Gamma}{b}>0\,. \label{12-bor}
\end{equation}
Таким образом, соотношения~(\ref{12b-bor}), (\ref{12-bor})  позволяют
(приближенно) определить ЭПС как
\begin{equation*}
C=\fr{\Lambda_V (\theta^*)}{\theta^*}=\fr{\Lambda_V(-\ln
\Gamma/b)b}{-\ln \Gamma}\,. %\label{12a-bor}
\end{equation*}
В случае н.о.р.\ $\{v_i\}$ получаем, см.~(\ref{6-bor}),
\begin{equation*}
C=-\fr{b}{\ln \Gamma}\ln \e e^{-{v\ln \Gamma
}/b}\,. %\label{13a-bor}
\end{equation*}
Это соотношение позволяет в ряде случаев получить явное  значение~$C$.

Приведенный  анализ
 верен также и для системы с конечным буфером
(большого) размера~$b$, т.\,е.\ для системы с потерями, поскольку
вероятность потери  $ \p(W> b)$  в такой системе ведет себя
асимптотически (при $b\hm\to \infty$) так же, как и в системе с
неограниченным буфером~\cite{Ganesh}.

\section{Вычисление  эффективной пропускной способности  для~регенеративного входного процесса}

Как было сказано выше,  интерес к  системам с регенеративным входным
потоком обусловлен сохранением свойства регенерации   при
прохождении потоков между узлами коммуникационной сети. Предлагаемый
ниже подход в значительной мере опирается на эвристические
соображения и в первую очередь мотивирован тем, что широко
используемая  оценка по методу группового среднего (batch mean) не
учитывает  зависимости между данными входного процесса. Как
отмечено, например, в работах~\cite{KRC, PPM09}, используемое в этой
оценке разбиение данных на блоки фиксированной длины, как правило,
приводит к недооценке доли потерь в системах  с (большим) конечным
буфером.  Поэтому использование такой оценки для вычисления ЭПС в
случае входного процесса с зависимыми данными может привести к
нарушению требуемых гарантий QoS, что неприемлемо в
системах высокой надежности.

 Рассмотрим вновь дискретную рекурсию Линдли~(\ref{2-bor}), и пусть теперь входная
 последовательность $\{v_n\}$ является регенерирующей с моментами
регенерации $\{\beta_k\}$ и периодами регенерации
$\alpha_k\hm=\beta_{k+1}\hm-\beta_k$. Таким образом,  значения~$v_i$
внутри каждого цикла регенерации могут быть зависимыми, но значения
$v_i$ и $v_j$, принадлежащие разным циклам, являются независимыми.
Поэтому  суммарная работа, поступающая на циклах регенерации
входного процесса,
\begin{equation}
   V_k:=\sum\limits_{i=\beta_k}^{\beta_{k+1}-1}v_i\,,\enskip k\ge
  0\,,\ \beta_0=0\,,
  \label{vor-eq9}
\end{equation}
образует последовательность н.о.р.\ {\it блоков} (c типичным
элементом~$V$). Будем предполагать, что для некоторого $\theta_0\hm>0$
при всех $\theta\hm\in (0,\,\theta_0)$
 выполнено  условие конечности экспоненциальных моментов величины блока
\begin{equation}
\ln \e e^{\theta V}<\infty\,.\label{5c-bor}
\end{equation}
Заметим, что поскольку величина блока~$V$ и длина цикла~$\alpha$
связаны очевидным образом
\begin{equation}
V=_{\mathrm{st}}\sum\limits_{i=0}^{\alpha-1}v_i\,, \label{24-bor}
\end{equation}
то их  моментные свойства также тесно связаны.  Скажем, в
специальном случае, когда
длина цик\-ла~$\alpha$ не зависит от последовательности н.о.р.с.в.\
$\{v_n\}$,
получаем
$
\e e^{\theta V}= \e [\e e^{\theta v}]^{\alpha}.$ 
В~дальнейшем потребуется лишь положительная возвратность
последовательности $\{v_n\}$, т.\,е.\ условие  $\e\alpha<\infty$.
Заметим, что если  длина цикла~$\alpha$ является моментом остановки
относительно последовательности н.о.р.с.в.\ $\{v_n\}$,  то по
неравенству Иенсена и тождеству Вальда $ \e e^{\theta V}\hm\ge \theta
\e V\hm=\theta\e \alpha \e v $ и положительная возвратность следует из
условия~(\ref{5c-bor}). В~этом случае положительную  возвратность можно
также получить  из разложения функции $\phi(\theta):=\e e^{\theta
V}\hm=1\hm+\theta \e V\hm+o(\theta)$ в ряд Тейлора в окрестности $\theta\hm=0$.

 Обозначим через $ k(n):=\max(k\ge0: \beta_k \le n)$ число циклов
регенерации  среди
заявок с номерами $0,\,1,\ldots,n$, так что $k(0)\hm=0$.  Заметим, что
$k(i)=0,\,i\le \beta_1-1$ и $k(\beta_i)=i,\,i\ge 1$. Напомним
обозначение: 
$$
V(n)\hm=\sum\limits_{i=0}^{n-1}v_i\,,\enskip V(0)=0\,.
$$
 В~теории
регенерации процесс $\{V(n)\}$ называется процессом накопления.
Заметим, что
\begin{multline*}
\ln \e e^{\theta \sum_{i=0}^{k(n)-1}V_i}\le \ln \e e ^{\theta
V(n)}\le {}\\
{}\le\ln \e e^{\theta \sum_{i=0}^{k(n)} V_i}\,,\enskip n\ge 0
\ \left(\sum\limits_\emptyset=0\right)\,. %\label{38a-bor}
\end{multline*}
Далее основное допущение состоит в   том, что {\it блоки
$V_i,\,i\le k(n),$ и величина $k(n)$ независимы при больших~$n$}.
(Вообще говоря, эти величины зависимы, поскольку  $k(n)$ зависит от
длин циклов, полученных к моменту~$n$.)
Обозначим $a:= \e e^{\theta V}$ и, используя свойство условного
математического ожидания, получим (при больших~$n$)
\begin{equation}
\fr{1}{n} \ln \e a ^{k(n)} \le \fr{1}{n} \ln \e e ^{\theta
V(n)}\le \fr{1}{n} \ln \e a ^{k(n)+1}\,.
  \label{27-bor}
\end{equation}
Следующее допущение состоит в том, что {\it нижняя и верхняя границы
в неравенстве~$(\ref{27-bor})$ асимптотически близки при больших~$n$}.
Приведем некоторые соображения в пользу   этого предположения.
Рассмотрим величину работы, поступающей в систему с момента~$n$ до
конца текущего  цикла регенерации, т.\,е.\
\begin{equation*}
U(n)=\sum\limits_{i=0}^{k(n)} V_i -V(n) =\sum\limits_{i=n}^{\beta_{k(n)}-1}v_i\,.
%\label{26a-bor}
\end{equation*}
Как показано в~\cite{Asmus}, при условии~(\ref{5c-bor}) для некоторого
$\varepsilon\hm>0$
$$
\e e^{\varepsilon U(n)}\to D\,,\enskip n\to \infty\,,
$$
где постоянная  $D<\infty$. Отметим также, что с.в.\
$Z(n):=V(n)\hm-\sum\limits_{i=0}^{k(n)-1} V_i$, равная величине  работы,
которая уже поступила в систему на текущем в момент $n$ цикле
регенерации, асимптотически (при $n\hm\to \infty$) распределена так же,
как и величина $U(n)$, и поэтому имеет (в пределе) такие же
моментные свойства. (С~учетом  сделанных  замечаний можно легко
доказать сближение нижней и верхней границы в~(\ref{27-bor})  в
предположении {\it независимости} величин $V(n)$, $U(n)$ и~$Z(n)$.)
Далее,  по неравенству Иенсена
\begin{equation*}
\fr{1}{n} \ln \e a ^{k(n)+1}\ge \fr{\e(k(n)+1)}{n}\ln a\,,
% \label{28-bor}
\end{equation*}
а по элементарной теореме восстановления $\e (k(n)\hm+1)/n\hm\to
1/\e\alpha$.  Приведенные выше  соображения позволяют заключить, что
при больших $n$ аппроксимация
\begin{equation}
\Lambda_V(\theta)= \fr{\ln \e e^{\theta V}}{\e \alpha}
  \label{29-bor}
\end{equation}
может дать   удовлетворительное  для практических целей значение
функции $\Lambda_V(\theta)$. Это, в свою очередь, является
основанием для аппроксимации искомой ЭПС с помощью формулы
\begin{equation}
C= \fr{\ln \e e^{\theta^* V}}{\theta^*\,\e \alpha}\,.
  \label{30-bor}
\end{equation}
Подчеркнем, что при получении приведенной выше аппроксимации ЭПС
используется  приближение~(\ref{16a-bor}), а также {\it предположение},
что в данной сис\-те\-ме верна экспоненциальная асимптотика~(\ref{10b-bor}).\linebreak
Кроме того, хотя это не оговаривалось ранее, в основе асимптотики~(\ref{10b-bor}) 
лежит также предположение о том, что входная
последовательность $\{v_n\}$ является {\it стационарной}.
(Конструкция стационарного регенерирующего процесса описана  в~\cite{Thorrison1}.)    
Наконец, отметим, что результат~(\ref{29-bor})
формально можно  получить из~(\ref{27-bor})
 заменой случайного чис\-ла цик\-лов  $k(n)$ его  математическим
ожиданием и переходом к переделу при $n\hm\to \infty$.

 Как и в работе~\cite{KRC}, ниже рассмотрены
 два следующих варианта получения  регенеративного
входного процесса.

В первом варианте   рассматривается  двухузловая  сеть, в которой на
вход  узла~1 в соответствии с процессом восстановления поступают
заявки c н.о.р.\ временами обслуживания $\{S_n^{(1)}\}$ и с
коэффициентом загрузки $\rho \hm<1$. В~этом случае выходной процесс из
узла~1 является положительно возвратным регенерирующим процессом со
средней длиной цикла $\e \alpha\hm<\infty$. Будем считать длину цикла~$\alpha$ 
равной числу заявок, поступивших в узел~1  на цикле
регенерации. (Это обычный подход при рассмотрении процессов,
вложенных в моменты прихода заявок. Разумеется, можно рассмотреть
длину цикла и в непрерывном времени.) Моментные свойства длины цикла $\alpha$ 
можно получать исходя из свойств  с.в.\ $S^{(1)}$.  В~этой
связи оказывается полезным следующий результат.
 Пусть~$\phi$ есть некоторая измеримая функция и $\rho\hm<1$.  Тогда (см.~\cite{Thorisson, Wolff})
\begin{equation}
\e \phi(S^{(1)})<\infty\ \mbox{влечет}\ 
\e\phi(\alpha)<\infty\,. \label{31-bor}
\end{equation}
 В частности, если $\e e^{\theta S^{(1)}}<\infty$, то $\e e^{\theta \alpha}<\infty$.  (Для
получения условий конечности моментов длины цикла регенерации узла~1
в непрерывном времени соответствующее условие надо наложить и на
длину интервала входного потока в узел~1, см.~\cite{Thorisson}.
Однако эта конструкция не используется в данной работе.) Далее
н.о.р.\ длины циклов регенерации $\{\alpha_n\}$ узла~1
 используются  в качестве  длин  циклов входного процесса
 в {\it жидкостной узел~2 с искомой скоростью~$C$},  где процесс
 загрузки
описывается  рекурсией~(\ref{2-bor}). Иными словами,  $\alpha$ равно
длине цикла регенерации (числу интервалов единичной длины) процесса
нагрузки $\{v_i\}$, поступающего в узел~2 и описываемого рекурсией~(\ref{2-bor}).
 Мотивировку такой модели, где  оба узла связаны
не напрямую, можно найти в работе~\cite{KRC}.

Во втором варианте входной регенеративный  процесс можно  считать
заданным и  наложить требуемые моментные условия на длины циклов  и
на объем поступающей на цикле работы.

Точность полученной  аппроксимации~(\ref{29-bor}) для обоих случаев
иллюстрируется в следующем разделе  с помощью имитационного
моделирования  ряда систем  с регенеративным входным процессом.

\medskip

\noindent
\textbf{Замечание~1.} С~учетом разложения функции
$\phi(\theta):=\Lambda_V(\theta)$ в ряд Тейлора в окрестности
$\theta\hm=0$ соотношение~(\ref{30-bor}) можно записать как
$$
C=\fr{1}{\e \alpha}\left(\e V +\fr{\e
(V^2)\theta^*}{2}+o(\theta^*) \right)\,,\enskip \theta^*\to 0\,,
$$
что в ряде случаев может быть использовано  для вычисления~$C$  на
основе лишь первых двух моментов  с.в.~$V$ и~$\alpha$, см.~(\ref{24-bor}).

\medskip

\noindent
\textbf{Замечание~2.}  Используя  неравенство Иенсена, из теории регенерации  получим
\begin{equation}
\fr{1}{n}\ln \e e^{\theta\, V(n)}\ge \theta\fr {\e V(n)}{n}\to
\theta \,\fr{\e V}{\e\alpha}\,,\ n\to \infty\,. \label{32a-bor}
\end{equation}
При этом величина  $ \e V/\e \alpha $ является ин\-тен\-сив\-ностью
входного регенерирующего процесса, и если существует стационарный
предел $v_n\hm\Rightarrow v$, то $\e v\hm=\e V/\e \alpha$~\cite{Asmus}.
 Если  использовать  нижнюю
границу~(\ref{32a-bor}) вместо  предела $\Lambda_V(\theta)$, то
 уравнение~(\ref{6-bor})
дает   $\e V/\e \alpha\hm=C$, что  влечет нестационарность сис\-те\-мы с
регенеративным входным процессом~\cite{Questa04}.


\medskip

\noindent
\textbf{Замечание~3.}  В~работах~[20--22]
приводится (в разных формах) сильный принцип инвариантности,
позволяющий аппроксимировать с вероятностью~1 процесс восстановления
$\{k(n)\}$, а также  исходный процесс накопления $\{V(n)\}$, с
помощью винеровского процесса и некоторой (случайной) {\it функции
уклонения}  $f(n)\hm=o( n)$. Однако  применение этих результатов для
получения асимптотики $\Lambda_V^{(n)}(\theta)$ опирается на
некоторые допущения о независимости и приводит в ряде случаев к
значению ЭПС, при котором   вероятность потери превышает величину~$\Gamma$.

\begin{table*}[b]\small
\vspace*{-12pt}
\begin{center}
\Caption{Регенеративная оценка ЭПС для двухузловой сети
}
\vspace*{2ex}


\tabcolsep=8pt
\begin{tabular}{|c|c|c|c|c|c|}
\hline
$\#$ & $\Gamma$ & $\theta^*$ & $\hat{C}(k)$ & $\hat\Gamma $  & $\Delta/\Gamma$ \\
\hline 
&&&&&\\[-9pt]
1 & $10^{-3}$ & 0,230259 & 0,264602 & $8{,}15\cdot 10^{-4}$ & 0,15\hphantom{9}\\
%\hline
%&&&&&\\[-9pt]
2 &  $10^{-4}$ & 0,307011 &  0,290134 & $2{,}05\cdot 10^{-5}$  & 0,75\hphantom{9}\\
%\hline
%&&&&&\\[-9pt]
3 &  $10^{-5}$ & 0,383764 & 0,348517 & $1{,}84\cdot 10^{-6}$ &  0,816\\
%\hline
%&&&&&\\[-9pt]
4 &  $10^{-6}$ & 0,460517 & 0,527721 &  $2{,}97\cdot 10^{-8}$ & 0,97\hphantom{9}\\
%\hline
%&&&&&\\[-9pt]
5 &  $10^{-7}$ & 0,53727\hphantom{9} & 0,661887  &  $0{,}45\cdot 10^{-8}$ &  0,955\\
%\hline
%&&&&&\\[-9pt]
6 &  $10^{-8}$ & 0,614023 & 0,986111  &  \hphantom{$^0$}$8{,}67\cdot 10^{-10}$ & 0,913\\
\hline 
\end{tabular}
\end{center}
\end{table*}

\section{Результаты численного моделирования}

В этом разделе рассматривается процесс незавершенной работы на
интервалах единичной длины  (слотах), удовлетворяющий рекурсии~(\ref{2-bor}), 
с положительно возвратной входной по\-сле\-до\-ва\-тель\-ностью
$\{v_i\}$, моментами регенерации $\{\beta_k\}$ и н.о.р.\ длинами
циклов $\{\alpha_k\}$. Ниже  исследуется точность оценивания
скорости обслуживания $C$ (искомой ЭПС)    на основе полученной в
предыдущем разделе аппроксимации
\begin{equation*}
C=\fr{\Lambda_V(\theta^*)}
%\ln \e e^{\theta^* V}}
{\theta^*},% \e \alpha}
% +\frac{\ln\e [\e
%e^{\theta^*V}]^{ N(0,C_1^2)}}{\theta^*},
%\label{37a-bor}
\end{equation*}
где
$$
\Lambda_V(\theta^*)=\fr{1}{\e \alpha} \ln \e e^{\theta^*
V}:=\Lambda_{\mathrm{REG}}(\theta^*)\,,
$$
а параметр $\theta^*\hm=-\ln\Gamma/b$. Поскольку $\beta_k/k\hm\to \e
\alpha$, то выборочная оценка
\begin{equation*}
\hat \Lambda_{\mathrm{REG}}^{(k)} (\theta^*):=\fr{\ln
(1/k)\sum\limits_{i=1}^k e^{\theta^* V_i}}{(1/k)\sum\limits_{i=1}^k
\alpha_i}=  \fr{k}{\beta_k}
  \ln\fr{1}{k} \sum\limits_{i=1}^{k} e^{\theta^*  V_i}
%\label{49-bor}
\end{equation*}
величины $\Lambda_{\mathrm{REG}}(\theta^*)$, полученная по $k$ регенеративным
блокам входного потока, является сильно состоятельной:
$
\hat \Lambda_{\mathrm{REG}}^{(k)} (\theta^*)\to
\Lambda_{\mathrm{REG}}(\theta^*)$, $k\hm\to \infty$ с вероятностью~1.
Поэтому ниже соответствующая регенеративная
оценка ЭПС вычисляется по формуле
\begin{equation}
\hat C(k):=\fr{\hat \Lambda_{\mathrm{REG}}^{(k)}(\theta^*)}{\theta^*}
\label{40-bor}
\end{equation}
для (достаточно большого) числа $k$ циклов регенерации, полученных в
процессе имитационного моделирования.

%\medskip

{\it Эксперимент~1.} Рассмотрена  двухузловая   сеть, где узел~1
есть система вида $M/M/1$ с показательным  временем обслуживания~$S$
с интенсивностью $\mu\hm= 1$ и с интенсивностью  пуассоновского
входного процесса $\lambda:=\rho\hm<1$. Поскольку время обслуживания
имеет конечные экспоненциальные моменты, $\e e^{\theta S}\hm<\infty$
для всех $\theta\hm<1$, то ввиду~(\ref{31-bor}) получаем также $\e
e^{\alpha \theta }\hm<\infty $ при $\theta\hm<1$. Подчеркнем, что значения
параметра $\theta\hm=\theta^*$ в табл.~1 выбраны с учетом этого
условия. (В~этой связи отметим, что в случае н.о.р.\ $\{v_i\}$ с
легким хвостом распределение хвоста суммы~$V$ в соотношении~(\ref{24-bor}) 
может иметь тяжелый хвост,  если индекс суммирования~$\alpha$ имеет тяжелый хвост~\cite{HT}.) 
Как было упомянуто выше,
длины цик\-лов регенерации  узла~1 используются  как длины циклов
регенерации в дискретном времени входного потока в узел~2, на основе
которых получаются регенеративные блоки~(14). В~узле~2
надо определить скорость~$C$, гарантирующую  непревышение заданного
уровня потерь~$\Gamma$.  В~данной модели коэффициент загрузки $\rho$
узла~1 существенно влияет на моментные свойства длины цикла~$\alpha$
и, кроме того,  значения с.в.\ $\{v_i\}$ на одном цикле являются
зависимыми. Точнее говоря, рассматриваются  независимые с.в.\
$\{\eta_k\}$, имеющие распределение Вейбулла (с легким хвостом):
\begin{equation*}
F_\eta (x) = 1 - e^{-\gamma x^c}\,, \ \gamma > 0\,, \ c \ge 1\,.
%\label{weibull}
\end{equation*}

\begin{table*}\small
\begin{center}
\Caption{Регенеративная оценка ЭПС в случае ограничения объема работы на цикле}
\vspace*{2ex}

\begin{tabular}{|c|c|c|c|c|c|c|c|c|}
\hline
$\#$ & $\Gamma$ & $\theta^*$ & $d$& $\hat\alpha$ &$\hat{C}(k)$ & $Var \hat{C}(k)$ & $\hat\Gamma$ & $\Delta/\Gamma$\\
\hline 
&&&&&&&&\\[-9pt]
1 &  $10^{-4}$ &  0,153506 & 50 & \hphantom{9}89,1 & 0,560441 & $5{,}23\cdot 10^{-6}$  & $0{,}3433 \cdot 10^{-5}$ & 0,6567\\
%&&&&&&&&\\[-9pt]
2 &  $10^{-5}$ & 0,191882 & 50 & \hphantom{9}89,2 & 0,560947 & $7{,}73\cdot 10^{-6}$ & $0{,}4153 \cdot 10^{-5}$ & 0,5847\\
%\hline
%&&&&&&&&\\[-9pt]
3 &  $10^{-6}$ & 0,230259 & 70 & 124,9& 0,561252 &  $2{,}64\cdot 10^{-6}$ & $0{,}8698 \cdot 10^{-6}$ & 0,1302\\
%\hline
%&&&&&&&&\\[-9pt]
4 &  $10^{-7}$ & 0,268635 & 70 & 124,8 & 0,562472 &  $4{,}23\cdot 10^{-6}$ & $0{,}8871 \cdot 10^{-7}$ & 0,1129\\
%\hline
%&&&&&&&&\\[-9pt]
5 &  $10^{-8}$ &  0,307011 & 70 & 124,5 & 0,563537  &  $6{,}98\cdot 10^{-6}$ & $0{,}2116 \cdot 10^{-8}$ & 0,7884\\
\hline
\end{tabular}
\end{center}
\vspace*{-6pt}
\end{table*}

\noindent
Тогда  величина работы, поступающей в узел~2 в момент~$j$ (любого)
текущего цикла регенерации входного процесса, задается соотношением:

\noindent
\begin{equation}
v_j = \fr{\sum\nolimits_{k=1}^{j}\eta_k}{j}\,,\enskip 1 \le j \le \alpha\,,
\label{dependence}
\end{equation}
где    $\alpha$  есть (типичная) длина цикла. Далее по формуле~(\ref{40-bor})  
вычисляется  значение  оценки $\hat{C}(k)$, которая
используется  в качестве скорости обслуживания~$C$ в узле~2.  Затем
(в предположении достижения
 стационарного режима в узле~2) с помощью выборочного среднего   оценивается
 вероятность $\p(W>b)$ в узле~2.  Заметим, что стандартный способ  получения стационарного
 значения~$W$  при имитационном моделировании состоит
  в пропуске начального, так называемого {\it  переходного},  периода.
 Кроме того, в рассматриваемой системе  пропуск  переходного  периода служит для   получения
 стационарного режима входного процесса.

 В табл.~1 представлены результаты оценивания
ЭПС, а также вероятности потери  при следующих условиях: параметры
распределения Вейбулла $\gamma \hm= 3$, $c\hm=4$; интенсивность трафика
$\rho\hm=0{,}4$ в узле~1; число циклов  $k\hm=40\,000$, уровень превышения
(буфер) $b\hm=30$. При этом разность  $\Delta: = \Gamma \hm-
\hat\Gamma\hm>0$, т.\,е.\ регенеративная оценка ЭПС~(\ref{40-bor})
обеспечивает требуемый  уровень надежности~$\Gamma$.

%\medskip

{\it Эксперимент~2.} Рассматривается  система с постоянной искомой
скоростью обслуживания $C$, а   цикл регенерации входного процесса
завершается, когда   величина работы, поступающей в систему на
цикле, достигает заданного объема~$d$. Как и выше,  в качестве
скорости $C$ использована ее оценка $\hat{C}(k)$. В табл.~2
представлены результаты моделирования при  $b\hm=60$, а зависимость
между с.в.\ $\{v_i\}$ внутри цик\-ла регенерации определяется
соотношением~(\ref{dependence}).




 Подчеркнем, что во всех
случаях $\Delta \hm> 0$, т.\,е.\ используемая оценка $\hat{C}(k)$
обеспечивает требуемую гарантию~$\Gamma$.
 В~этой связи важно отметить, что оценивание ЭПС по методу
группового среднего 
в ряде случаев приводит к значению  $\Delta \hm< 0$, т.\,е.\  к нарушению
гарантии~$\Gamma$ (см.~\cite{KRC}). В~то же время величина
$\Delta/\Gamma$ показывает, что при использовании оценки
$\hat{C}(k)$ уровень~$\Gamma$ обеспечивается с определенным запасом,
и это  связано с высокой чувствительностью доли потерь к изменению
скорости обслуживания. С~другой стороны, оценка ЭПС имеет очень
небольшую дисперсию и мало  чувствительна к изменению величины~$\Gamma$.  
Это особенно  хорошо заметно в табл.~2.
Например, изменение величины $\Gamma$ с  $10^{-4}$ до $10^{-5}$
влечет изменение $\hat{C}(k)$ с 0,5604 до 0,5609. Это несомненно
вызвано  тем, что  нагрузка на цикле ограничена постоянной величиной~$d$.

Отметим, что для вычисления оценки $\hat\Gamma$ вероятности
превышения стационарным процессом нагрузки (высокого) уровня $b\hm=60$
использовались ресурсы кластера КарНЦ РАН, а также метод ускоренного
регенеративного имитационного моделирования, разработанный  для
оценивания вероятностей редких событий~\cite{PPM09}.

\vspace*{-4pt}

\section{Заключение}

\vspace*{-2pt}

В статье обсуждается понятие ЭПС жидкостной сис\-те\-мы обслуживания в дискретном времени.
 Рас\-смат\-ри\-ва\-ет\-ся  входной процесс с независимыми приращениями, но  основное внимание
уделено  аппроксимации и  оцениванию  ЭПС  в случае входного
регенеративного процесса.
 Представлены  эвристические соображения, на основе
которых получена аппроксимация  ЭПС. Результаты численных
экспериментов   показывают, что оценка~(\ref{40-bor}), полученная на
основе аппроксимации~(\ref{30-bor}), гарантирует  уровень надежности
$\Gamma$. Таким образом,  аппроксимация~(\ref{30-bor}) может считаться
эффективной альтернативой оценке ЭПС по методу группового среднего
для системы с регенеративным входным процессом. Дальнейшие
исследования   в данном направлении  должны включать как
теоретическое обоснование предложенной аппроксимации, так и
расширение численных экспериментов по проверке ее точ\-ности.

\bigskip

Авторы благодарят М.\,А.~Лифшица за  ценные замечания.

{\small\frenchspacing
{%\baselineskip=10.8pt
\addcontentsline{toc}{section}{Литература}
\begin{thebibliography}{99}

\bibitem{MorozovTechCyb87} 
\Au{Морозов Е.} Критерий стационарности одного класса непуассоновских
сетей обслуживания~// Изв. АН СССР. Техн. кибернетика, 1988. №\,1. C.~129--133.

\bibitem{MorozovOutput}
\Au{Morozov E.} Stability of Jackson type network output~//
Queueing Syst., 2002. Vol.~40. P.~383--406.

\bibitem{Irina07} %3
\Au{Vorobieva I., Morozov E., Pagano~M., Procissi~G.} A~new
regenerative estimator for effective bandwidth prediction~// 
AMICT 2007 Proceedings.~--- Petrozavodsk, 2008. P.~175--187.

\bibitem{Rennes} %4
\Au{Morozov E., Dyudenko~I., Pagano~M.}  Regenerative estimator of the overflow probability in a
 tandem network~// 7th  Workshop (International)
 on Rare Event Simulation Proceedings.~---  Rennes, France, 2008. P.~283--287.

\bibitem{Minsk} %5
\Au{Dyudenko I., Morozov E., Pagano~M.}  Regenerative estimator for
effective bandwidth~// Mathematical  methods for analysis and
optimization of information telecommunication networks: Proceedings
of  the International Conference.~--- Minsk: Belarusian State
University, 2009. P.~58--60.

\bibitem{SP2009} %6
\Au{Dyudenko I., Morozov~E.,  Pagano~M., Sandmann~W.} Comparative
study of effective bandwidth estimators: Batch means and
regenerative cycles~// 6th St. Petersburg
Workshop on Simulation Proceedings.~--- St-Petersburg, 2009.
 Vol.~II. P.~1003--1007.
 
 \bibitem{KRC} %7
\Au{Бородина А.\,В., Морозов~Е.\,В.} Сравнение двух оценок
эффективной пропускной способности системы обслуживания~// Тр.
Карельского научного центра РАН, 2012. №\,5. C.~8--17.

\bibitem{Crosby} %8
\Au{Crosby S., Leslie I., Huggard~M., Lewis J.\,T., McGurk~B., Russel~R.} 
Predicting bandwidth requirements of ATM
 and Ethernet traffic~// IEE UK Teletraffic Symposium Proceedings.~--- Glasgow, U.K., 1996.

\bibitem{Kelly} %9
\Au{Kelly F.} Notes on   effective bandwiths~// Stochastic
networks: Theory and applications~/ Eds. F.\,P.~Kelly, S.~Zachary, 
I.~B. Ziedins.~--- Roy. Stat. Soc. Lecture Notes ser., 4.~---
Oxford University Press, 1996. P.~141--168.


\bibitem{Lewis}
\Au{Lewis J.\,T., Russell R.} An introduction to large deviation for
teletraffic engineers, 1997.
{\sf https://\linebreak engineering.purdue.edu/ece647/notes.html}.


\bibitem{Asmus}
\Au{Asmussen S.}  Applied probability and queues.~-- 2nd ed.~--- NY: Springer, 2003.

\bibitem{Ganesh}
\Au{Ganesh A., O'Connell N., Wischik~D.} Big queues.~--- Berlin:  Springer-Verlag, 2004.

\bibitem{Glynn} %13
\Au{Glynn P.\,W., Whitt W.} Logarithmic asymptotics for steady-state
tail probabilities in a single-server queue~// JAP, 1994. Vol.~31.
P.~131--156.

\bibitem{Chang} %14
\Au{Cheng-Shang Chang}.  Performance guarantees in communication
networks.~--- Springer, 2000.


\bibitem{PPM09} %15
\Au{Бородина А., Дюденко И., Морозов~Е.} Ускоренное оценивание
вероятности переполнения  регенеративных систем обслуживания~//
ОПиПМ, 2009. Т.~16. №\,4. С.~577--593.




\bibitem {Thorrison1} %16
\Au{Thorisson H.}  Coupling, stationarity, and regeneration.~--- NY: Springer, 2000.

\bibitem{Thorisson} %17
\Au{Thorisson H.} The queue $GI/GI/k$:  Finite moments of the cycle
variables and uniform rates of convergence~// Commun.
Stat.-Stochastic Models, 1985. Vol.~192. P.~221--238.

\bibitem {Wolff} %18
\Au{Wolff R.\,W.}   Stochastic modeling and the theory of queues.~---
Prentice-Hall, 1989.

\bibitem{Questa04} %19
\Au{Morozov E.} Weak regeneration in modeling of queueing processes~// 
Queueing  Syst., 2004.  Vol.~46. No.\,3--4. P.~295--315.

\bibitem{Csorgo} %20
\Au{Cs\"org\H{o} M., Horvath L., Steinebach~J.} Invariance
principles for renewal processes~//  Ann. Prob., 1987.
Vol.~15. No.\,4. P.~1441--1460.

\bibitem{Damerdji} %21
\Au{Damerdji H.} Strong consistency of the variance estimator in
steady-state simulation output analysis~// Math. Oper.
Res., 1994. Vol.~19. No.\,2. P.~494--512.

\bibitem{Sharma} %22
\Au{Sharma V.} Reliable estimation via simulation~// Queueing
Syst., 1995. Vol.~19. P.~169--192.


\label{end\stat}


\bibitem {HT} %23
 \Au{Robert C.\,Y.,  Segers~J.}  Tails of random sums of a
heavy-tailed number of light-tailed terms~// Insurance: Mathematics
and Economics, 2008. Vol.~43. P.~85--92.
\end{thebibliography}
}
}

\end{multicols} %5

\def\stat{grusho}

\def\tit{АРХИТЕКТУРНЫЕ РЕШЕНИЯ В~ЗАДАЧЕ ВЫЯВЛЕНИЯ МОШЕННИЧЕСТВА ПРИ~АНАЛИЗЕ 
ИНФОРМАЦИОННЫХ ПОТОКОВ В~ЦИФРОВОЙ ЭКОНОМИКЕ$^*$}

\def\titkol{Архитектурные решения в~задаче выявления мошенничества при~анализе 
информационных потоков в
%~цифровой 
экономике}

\def\aut{А.\,А.~Грушо$^1$, М.\,И.~Забежайло$^2$, Н.\,А.~Грушо$^3$, 
Е.\,Е.~Тимонина$^4$}

\def\autkol{А.\,А.~Грушо, М.\,И.~Забежайло, Н.\,А.~Грушо, 
Е.\,Е.~Тимонина}

\titel{\tit}{\aut}{\autkol}{\titkol}

\index{Грушо А.\,А.}
\index{Забежайло М.\,И.}
\index{Грушо Н.\,А.}
\index{Тимонина Е.\,Е.}
\index{Grusho A.\,A.}
\index{Zabezhailo M.\,I.}
\index{Grusho N.\,A.}
\index{Timonina E.\,E.}


{\renewcommand{\thefootnote}{\fnsymbol{footnote}} \footnotetext[1]
{Работа частично поддержана РФФИ (проекты 18-29-03081 и~18-07-00274).}}


\renewcommand{\thefootnote}{\arabic{footnote}}
\footnotetext[1]{Институт проблем информатики Федерального исследовательского центра <<Информатика и~управление>> 
Российской академии наук, grusho@yandex.ru}
\footnotetext[2]{Институт проблем информатики Федерального исследовательского центра <<Информатика и~управление>> 
Российской академии наук, m.zabezhailo@yandex.ru}
\footnotetext[3]{Институт проблем информатики Федерального исследовательского центра <<Информатика и~управление>> 
Российской академии наук, info@itake.ru}
\footnotetext[4]{Институт проблем информатики Федерального исследовательского центра <<Информатика и~управление>> 
Российской академии наук, eltimon@yandex.ru}

\vspace*{-12pt}
   

 
  
  \Abst{Cформулирован подход к~исследованию некоторых видов мошенничества в~цифровой 
экономике с~использованием причинно-следственных связей. Во всех видах рассматриваемых 
мошенничеств должно наблюдаться несоответствие между целями финансовых транзакций 
и~реальной стоимостью достижения этих целей. Данные о транзакциях можно собирать, 
наблюдая информационные потоки, в~которых отражаются эти транзакции. Архитектура сбора 
данных и~их анализа может быть организована с~помощью распределенных реестров 
с~централизованным консенсусом, что позволяет создать аналог электронной бухгалтерской 
книги, фиксирующей финансово-экономическую деятельность субъектов цифровой экономики в~регионе. 
  Рассматриваемые методы выявления мошенничества основаны на противоречиях 
между действиями, описанными в~транзакциях, и~информацией, содержащейся в~планах, 
стандартах, прецедентах и~др. Рассмотрен метод, основанный на некоторой упрощенной схеме 
реализации абстрактного проекта. Для выявления противоречий необходимо проводить анализ 
от следствия к~причине, т.\,е.\ искать аномалии в~информации, описывающей порождение 
наблюдаемых следствий. 
  Показано, как в~реализации проекта можно выделять простые <<необходимые условия>> 
нарушения при\-чин\-но-след\-ст\-вен\-ных связей, т.\,е.\ множество <<необходимых условий>>, 
нарушение которых свидетельствует о наличии мошенничества. Это множество <<необходимых 
условий>> можно назвать метаданными для контроля проекта на выявление мошенничества.} 
 
 
  \KW{цифровая экономика; информационные потоки; при\-чин\-но-след\-ст\-вен\-ные связи; 
выявление мошеннических схем} 

\DOI{10.14357/19922264190204}
  
\vspace*{-4pt}


\vskip 10pt plus 9pt minus 6pt

\thispagestyle{headings}

\begin{multicols}{2}

\label{st\stat}

\section{Введение}

\vspace*{3pt}

  В работе сформулирован подход к~исследованию некоторых видов 
мошенничества в~цифровой экономике с~использованием  
при\-чин\-но-след\-ст\-вен\-ных связей. Рассматриваются три вида мошенничества, 
а именно:
  \begin{enumerate}[(1)]
\item отмыв денег; 
\item обман при выполнении договорных обязательств при реализации 
технических проектов (строительные проекты и~др.); 
\item незаконный вывод денег. 
\end{enumerate}

  Названные виды мошенничества могут быть сведены к~решению одного типа 
задач. Для отмывания денег источник должен заключать фиктивные контракты, 
в~соответствии с~которыми будут переводиться средства за заведомо ненужную 
работу и~материалы. 
  
  Мошенничество, связанное с~невыполнением договорных обязательств, связано 
со снижением качества услуг, качества и~количества закупаемых 
материалов, выполнением работ с~ненадлежащим качеством. 
  
  Вывод денег связан с~переводом средств фир\-мам-од\-но\-днев\-кам, которые 
заведомо не могут выполнить обязательства по контрактам, за которые им 
переводятся средства. 
  
  Таким образом, во всех трех видах рассматриваемых мошенничеств должно 
наблюдаться несоответствие между целями финансовых транзакций и~реальной 
стоимостью достижения этих целей. Данные о транзакциях можно собирать, 
наблюдая информационные потоки, в~которых отражаются эти транзакции. 
  
  Однако для наблюдения таких информационных потоков необходимо создавать 
архитектуру\linebreak телекоммуникационной системы, позволяющей перехватывать 
и~собирать данные о всех транзакциях. Например, такая архитектура может быть 
организована с~помощью распределенных реестров с~централизованным 
консенсусом, т.\,е.\ все информационные потоки, сформированные в~цифровой 
экономике и~несущие информацию о транзакциях, проходят через некоторый 
центральный узел, запоминающий их в~форме распределенного реестра. Такие 
реестры могут дублироваться в~аналогичных центрах различных регионов, что 
позволяет создать аналог электронной бухгалтерской книги, фиксирующей 
фи\-нан\-со\-во-эко\-но\-ми\-че\-скую деятельность субъектов цифровой экономики. Такой 
подход предложено реализовать на базе системы ситуационных центров, что 
отражено в~работах~[1, 2].
  
  Собранная из информационных потоков информация о~транзакциях, т.\,е.\ 
о~контрактах, договорах, платежах, отчетах, закупленных материалах, 
характеристиках исполнителей работ и~др., собирается в~базе данных в~указанном 
центре. Согласно теории интеллектуальных сис\-тем~[3], эту базу данных можно 
называть базой фактов (БФ). Базу фактов можно представить как бинарную мат\-ри\-цу, 
строки которой описывают характеристики, входящие в~транзакции, а столбцы 
нумеруются характеристиками. Строки матрицы будем называть 
\textit{объектами}~[4, 5]. 
  
  Рассматриваемые в~работе методы выявления мошенничества будут основаны 
на противоречиях между действиями, описанными в~транзакциях, и~информацией, 
содержащейся в~планах, стандартах, прецедентах и~др. Для нахождения 
противоречий в~архитектуре центра предусмотрена другая база данных~--- база 
знаний (БЗ)~\cite{3-gr, 6-gr}, которая устроена так же, как БФ. 
  
  Информация в~БЗ собирается на основе положительного опыта или расчетов. 
Используя БЗ, можно выводить факты нарушения при\-чин\-но-след\-ст\-вен\-ных 
связей. Нарушения при\-чин\-но-след\-ст\-вен\-ных связей будем называть 
\textit{аномалиями}. 
  
  Для упрощения дальнейшее изложение будет вестись в~рамках поиска 
противоречий при выполнении некоторого абстрактного проекта. Выявление 
аномалий будет происходить на основе фактов из БФ с~помощью знаний из БЗ 
методами искусственного интеллекта и~интеллектуального анализа 
данных~\cite{6-gr}. 

\vspace*{-10pt}
  
  \section{Модели}
  
  \vspace*{-3pt}
  
  Наиболее сложная из рассмотренных выше задач~--- выявление противоречий, 
т.\,е.\ использование БЗ для получения новых знаний и~выявление аномалий из 
полученных фактов. 
  
  Все способы выявления противоречий основаны на определении 
  причинно-следственных связей. При этом противоречия в~параметрах транзакций по 
отношению к~требуемым в~БЗ составляют сущность аномалий. 
  
   Далее будет рассмотрен метод, основанный на некоторой упрощенной схеме 
реализации абстрактного проекта. 
  
  Каждый проект имеет цель: например, цель представляет собой построение 
некоторой системы. Воспользуемся структурным подходом, который позволяет 
строить проект на основе разбиения системы на подсистемы и~определения 
взаимодействий подсистем~\cite{7-gr}. При этом каждая подсистема также 
представима структурной моделью. 
  
  Как сама система, так и~каждая ее подсистема имеют свой функционал 
и~спецификацию, па\-ра\-мет\-ры настройки и~домены параметров настройки. Кроме 
этих характеристик существует множество характеристик, связанных 
с~<<жизненным циклом>> создания системы. Сюда входят работы, ресурсы, 
сроки выполнения работ по созданию подсистем и~самой системы, стоимости 
компонентов и~материалов, стоимости работ, схемы поставок, договорные 
обязательства и~др. Все характеристики связаны между собой, поэтому можно 
говорить о стоимости и~времени изготовления структурных компонентов системы. 
  
  Одной из важнейших характеристик является смета (система смет для 
подсистем). Смета сопоставляет каждому компоненту системы стоимость его 
изготовления и~настройки. 
  
  Схема построения системы может быть пред\-став\-ле\-на диаграммой, 
изображенной на рис.~1. 

{ \begin{center}  %fig1
 \vspace*{9pt}
   \mbox{%
 \epsfxsize=79mm 
 \epsfbox{gru-1.eps}
 }


\vspace*{9pt}


\noindent
{{\figurename~1}\ \ \small{Диаграмма достижения цели}}
\end{center}
}

\vspace*{9pt}

\addtocounter{figure}{1}
  
  


  Представленная на рис.~1 диаграмма позволяет описать основные классы 
возможных противоречий при достижении цели. Противоречия возникают, когда 
данные БФ не соответствуют требуемым характеристикам. 
  
  
  \section{Потенциальные классы аномалий при~достижении цели}
  
  Выделим четыре потенциальных класса противоречий, которые показывают, 
каким образом нужно искать эти противоречия.
  
 
  Противоречие цели и~проекта (рис.~2) возникает при отсутствии обоснования 
или в~случае логического противоречия между возможностями проектируемого 
функционала и~целью системы. Отметим, что в~проект входят сроки, перечень 
работ, материалы, настройки, которые описываются соответствующими 
параметрами и~допустимыми значениями этих параметров. Проект формируется 
на основе БЗ и~расчетов, исходя из информации, полученной по аналогии 
с~другими проектами и~решениями, которые считаются апробированными. 
  
  Отметим, что цель порождает проект и~в этом смысле является причиной 
проекта. Однако для анализа противоречий необходимо двигаться по штриховой 
стрелке диаграммы (см.\ рис.~2) от проекта к~цели. В~самом деле, любой компонент 
проекта направлен на теоретическое достижение цели. Цель~--- сложный объект, 
поэтому в~проекте могут возникнуть характеристики, противоречащие хотя бы 
некоторым характеристикам цели. Это делает проект противоречивым, но вывод 
об этом может быть сделан только на уровне описания цели. 
  

  Противоречия между проектом и~его реализацией, исключая настройки 
(рис.~3), могут возникать, например, при закупке исполнителем материалов более 
низкого качества по более низким ценам, при попытках достижения требуемых 
сроков работы за счет снижения качества выполнения работ, за счет нахождения 
<<объективных>> причин для увеличения сроков работы и,~следовательно, 
увеличения цены реализации проекта. 


  Для выявления указанных противоречий необходимо двигаться по диаграмме 
(см.\ рис.~3) в~обратную сторону в~соответствии со~штриховыми стрелками. 
Действительно, выявить противоречия между характеристиками закупленных 
материалов и~требуемыми по проекту можно только при обращении к~проекту 
и~его спецификациям. Манипуляции со сроками работы также можно выявить 
только при обращении к~соответствующим расчетам в~проекте. Задержки в~сроках 
работы, связанные с~поставками материалов, можно определить только на 
предыдущем этапе диаграммы (см.\ рис.~3) в~описании проекта. 


  


  Противоречия между реализацией проекта и~его настройкой (рис.~4) возникает, 
когда не удается добиться требуемых значений параметров функционала, не 
удается обеспечить необходимый уровень\linebreak\vspace*{-12pt}

{ \begin{center}  %fig2
 \vspace*{-6pt}
   \mbox{%
 \epsfxsize=16mm 
 \epsfbox{gru-2.eps}
 }


\vspace*{6pt}


\noindent
{{\figurename~2}\ \ \small{Противоречия цели и~проекта}}
\end{center}
}

%\vspace*{9pt}

\addtocounter{figure}{1}

{ \begin{center}  %fig3
 \vspace*{6pt}
    \mbox{%
 \epsfxsize=79mm 
 \epsfbox{gru-3.eps}
 }


\end{center}

\vspace*{-2pt}


\noindent
{{\figurename~3}\ \ \small{Противоречия проекта и~его реализации (без настройки)}}
}

\vspace*{6pt}

\addtocounter{figure}{1}

{ \begin{center}  %fig4
 \vspace*{1pt}
   \mbox{%
 \epsfxsize=54.5mm 
 \epsfbox{gru-4.eps}
 }


\end{center}


\noindent
{{\figurename~4}\ \ \small{Противоречия реализации проекта и~его на\-стройки}}
}

%\vspace*{9pt}

\addtocounter{figure}{1}

{ \begin{center}  %fig5
 \vspace*{5pt}
    \mbox{%
 \epsfxsize=79mm 
 \epsfbox{gru-5.eps}
 }


\end{center}



\noindent
{{\figurename~5}\ \ \small{Противоречия цели и~достигнутой реализации проекта}}
}

\vspace*{6pt}

\addtocounter{figure}{1}

\noindent
 качества реализации проекта. Для 
определения противоречия в~настройках надо опять же двигаться по диаграмме 
(см.\ рис.~4) в~обратную сторону по штриховым стрелкам, так как для выявления 
характеристик результатов работы, которые не дают возможности реализации 
определенного функционала, необходимо иметь информацию о результатах этой 
работы. 


  



  Противоречие между целью и~достигнутой реализацией проекта (рис.~5) 
возникает, когда реализованная система не позволяет достичь цели. В~этом случае 
опять противоречие нужно искать, двигаясь от цели к~реальному достигнутому 
функционалу по штриховой стрелке (см.\ рис.~5).
  
  Суммируя положения, изложенные в~данном разделе, приходим к~выводу, что 
для выявления противоречий необходимо проводить анализ от следствия 
к~причине, т.\,е.\ искать аномалии в~информации, описывающей порождение 
наблюдаемых следствий. 
  
  
  \section{Связь противоречий и~причин}
  
  Прежде чем построить связь между причинами и~противоречиями, кратко 
опишем простейшую модель связи этих понятий. Причины и~противоречия будут 
сформулированы для представления компонентов системы как объектов, 
обладающих наборами известных характеристик~\cite{4-gr, 5-gr}. 
  
  Пусть $U\hm=\{\alpha, \beta, \ldots\}$~--- совокупность характеристик 
(пространство характеристик). Согласно~\cite{4-gr} \textit{объектом}~$O$ 
называется любое подмножество характеристик $O\hm\subseteq U$. Рассмотрим 
последовательность объектов, возможно в~различных пространствах 
характеристик. 
  
  \smallskip
  
  \noindent
  \textbf{Определение~1.}\ Объект~$P$ с~числом характеристик, большим или 
равным~2, является \textit{причиной} объекта (\textit{свойства})~$B$ в~цепочке 
наблюдаемых объектов тогда и~только тогда, когда выполнены следующие 
условия:
  \begin{enumerate}[(1)]
\item для каждого объекта~$C$, если $P\hm\subseteq C$, то $C\mapsto B$, где 
$C\mapsto B$ означает, что объект~$B$ присутствует в~объекте, следующем за 
объектом~$C$;
\item объект~$P$ является минимальным объектом, удовлетворяющим 
условию~1, а~именно: $\forall \alpha\hm\in P$ объект~$P\backslash \{\alpha\}$ 
не является причиной, т.\,е.\ $\exists C:\ \alpha\not\in C$, $P\backslash 
\{\alpha\}\hm\subseteq C$ и~$C\not\mapsto B$, где $C\not\mapsto B$ означает, 
что~$B$ не может содержаться в~объекте, следующем за объектом~$C$. 
\end{enumerate}

  Приведенное определение причины является упрощением причин, 
возникающих в~реальном мире. Например, реальные причины могут возникать\linebreak 
как совокупность характеристик из разных пространств. Одно следствие может 
порождаться разными причинами или возникать из внешних\linebreak и~ненаблюдаемых 
характеристик. Однако пред\-став\-лен\-ная далее формализация позволяет доступно 
изложить при\-чин\-но-след\-ст\-вен\-ные истоки противоречий, которые 
инициируют в~дальнейшем глубокое исследование рассматриваемых процессов.
  
  Будем считать, что для любого интересующего нас свойства~$B$ существует 
причина. Тогда справедлива следующая теорема.
  
  \smallskip
  
  \noindent
  \textbf{Теорема~1.}\ \textit{Для любого свойства~$B$ существует 
единственная причина}. 
  
  \smallskip
  
  \noindent
  Д\,о\,к\,а\,з\,а\,т\,е\,л\,ь\,с\,т\,в\,о\,.\ \ Доказательство будем вести от противного, 
т.\,е.\ предположим, что существуют две причины свойства~$B$: $P$ 
и~$P^\prime$, $P\hm\not= P^\prime$. Тогда существует $\alpha\hm\in U$, которое 
удовлетворяет одному из двух условий:
  \begin{itemize}
\item[(а)] $\alpha\in P$, $\alpha\notin P^\prime$;
\item[(б)] $\alpha\notin P$, $\alpha \in P^\prime$.
\end{itemize}

  Пусть выполняется условие~(б). Тогда $P^\prime\backslash \{\alpha\}$ не 
является причиной по условию~2 определения~1, т.\,е.\ $\exists C$ такое, что 
$\alpha\notin C$, $P^\prime\backslash \{\alpha\}\hm\subseteq C$ и~$C\not\mapsto B$. 
Но если~$B$ произошло и~$P$ его причина, то $C\mapsto B$, что противоречит 
предположению. Теорема~1 доказана.
  
  \smallskip
  
  \noindent
  \textbf{Лемма.} \textit{Если $P$~--- причина появления свойства~$B$, то 
объект~$B$ определяет существование свойства~$P$ в~объекте, 
предшествующем~$B$. }
  
  \smallskip
  
  \noindent
  Д\,о\,к\,а\,з\,а\,т\,е\,л\,ь\,с\,т\,в\,о\,.\ \ Из предположения, что у~каж\-до\-го 
свойства~$B$ есть причина, и~условия, что~$P$ является причиной~$B$, следует, 
что при появлении в~данных свойства~$B$ объект~$C$, предшествующий 
появлению~$B$, содержит как часть объект~$P$. Это следует из теоремы~1 
и~определения причины. 
  
  Докажем принцип <<необходимого условия>>, который, несмотря на простоту 
доказательства, будет играть в~дальнейшем существенную роль.
  
  \smallskip
  
  \noindent
  \textbf{Теорема~2.} \textit{Если~$P$~--- причина появления свойства~$B$ 
и~$A\hm\subseteq P$, то объект~$B$ определяет наличие свойства~$A$ 
в~объекте, предшествующем~$B$}. 
  
  \smallskip
  
  \noindent
  Д\,о\,к\,а\,з\,а\,т\,е\,л\,ь\,с\,т\,в\,о\,.\ \ Пусть в~данных имеется объект~$B$ 
и~$P\mapsto B$, тогда в~силу существования и~единственности причины~$B$ 
в~данных должен существовать объект~$C$, предшествующий~$B$ 
и~содержащий причину~$P$. Поскольку $A\hm\subseteq P$ и~$B$ содержит 
причину~$P$, то $B\mapsto A$. С~учетом леммы теорема~2 доказана.
  
  \smallskip
  
  Пусть даны пространства $U_1, U_2,\ldots$ и~имеется последовательность 
данных (процесс выполнения этапов проекта в~соответствии с~рис.~1) $A, B, 
\ldots$, где каждый объект является подмножеством некоторого 
пространства~$U_i$, $i\hm=1,\ldots$ Тогда в~объекте~$A$ присутствует 
причина~$P$ появления интересующего нас свойства~$C$ в~объекте~$B$. Пусть 
$P\hm\subseteq A$, тогда по теореме~2 $\forall \alpha\hm\in P$:  
$C\mapsto \{\alpha\}$, т.\,е.\ из появления~$C$ следует появление 
характеристики~$\alpha$ в~предшествующем объекте. Это необходимое условие 
того, что~$C$ удовлетворяет причинно-следственным связям развития процесса 
выполнения проекта. Если для~$C$ нет характеристики~$\alpha$, которую можно 
отнести к~причине~$C$, то можно считать, что нарушена  
при\-чин\-но-след\-ст\-вен\-ная связь и~$C$~--- аномальный объект. 
  
  \smallskip
  
  \noindent
  \textbf{Пример.} Если объект~$C$ состоит в~получении суммы~$a$ 
фирмой~$K$, то согласно теореме~2 в~пред\-шест\-ву\-ющем объекте должна 
существовать причина перевода суммы~$a$ на фирму~$K$. Если эта причина 
в~проекте отсутствует, то это можно считать признаком мошеннической схемы. 
Все проекты по предположению собираются из <<кубиков>>, содержащихся в~БЗ. 
Тогда можно сравнить цену объекта~$C$, породившего получение суммы~$a$, 
и~сумму, присутствующую в~смете проекта. Если разница велика, то это либо 
ошибка проекта, либо признак мошеннической схемы.
  
  \section{Поиск противоречий на~основе~принципа <<необходимых~условий>>}
   
  Как было показано в~разд.~3, нахождение противоречий соответствуют 
движению от следствия к~причине. Для каждого объекта в~наблюдаемых данных 
выявление причин его появления является трудоемкой задачей. Кроме того, при 
реализации контроля соблюдения при\-чин\-но-след\-ст\-вен\-ных связей на 
большом множестве участников экономической деятельности задача анализа 
причин становится трудоемкой. Поэтому процедуру контроля необходимо разбить 
на два этапа, где первый этап состоит в~анализе простых <<необходимых 
условий>> проявления мошенничества, когда используется хотя бы одна 
известная характеристика причины. Второй этап (в~режиме офлайн) состоит 
в~выявлении причин, позволяющих провести анализ источников мошеннических 
схем. 
  
  Один из подходов к~выбору <<необходимых условий>> состоит в~построении 
множества подцелей исходной цели проекта (структурный метод построения 
проекта~\cite{7-gr}). Каждая подцель описывается диаграммой на рис.~1, 
и~реализации подцелей должны образовывать полный функционал цели. Это 
является необходимым, но не достаточным условием достижения цели, так как 
при таком подходе отсутствует компонент согласования всех подцелей в~единую 
систему. Однако такой подход значительно упрощает анализ выполнения проекта 
на предмет поиска мошенничества. Если признаки мошенничества будут 
обнаружены в~реализации хотя бы одной из подцелей, то это значит, что 
мошенничество присутствует в~реализации всего проекта. 
  
  Аналогично в~реализации каждого этапа в~любой из подцелей можно выделять 
простые <<необходимые условия>> нарушения при\-чин\-но-след\-ст\-венн\-ых 
связей. 
  
  Таким образом, получается множество <<необходимых условий>>, нарушение 
которых свидетельствует о наличии мошенничества. Это множество 
<<необходимых условий>> можно назвать метаданными~[8, 9] для контроля 
проекта на выявление мошенничества. 
  
  
  \section{Заключение }
  
  В поиске противоречий необходимо от транзакций, соответствующих 
следствиям при\-чин\-но-след\-ст\-вен\-ных связей, переходить к~анализу причин 
наблюдаемых следствий. Это сложная задача, которая связана с~описанием причин 
определенных свойств. 
  
  В работе представлена модель, позволяющая строить множество необходимых 
условий соответствия наблюдаемого следствия вызвавшей его причине. Этот 
подход делает поиск противоречий вполне вычислимой задачей, но не гарантирует 
успех. 
  
  {\small\frenchspacing
 {%\baselineskip=10.8pt
 \addcontentsline{toc}{section}{References}
 \begin{thebibliography}{9}
\bibitem{1-gr}
\Au{Грушо А.\,А., Зацаринный~А.\,А., Тимонина~Е.\,Е.} Блокчейны цифровой экономики на базе 
системы ситуационных центров и~централизованного консенсуса~// Радиолокация, навигация, 
связь: Мат-лы XXV Междунар. научн.-технич. конф.~---
Воронеж: Издательский дом ВГУ, 2019. Т.~6. С.~183--191. 
\bibitem{2-gr}
\Au{Grusho A., Zatsarinny~A., Timonina~E.} A~system approach to information security in 
distributed ledgers on the situational centers platform.~---
Lecture notes in computer science ser.~--- Springer, 2019 
(in press).
\bibitem{3-gr}
\Au{Финн В.\,К.} Искусственный интеллект: Методология, применения, философия.~--- М.: 
Красанд, 2011. 448~с.

\bibitem{5-gr} %4
\Au{Аншаков~О.\,М., Фабрикантова~Е.\,Ф.} ДСМ-ме\-тод автоматического порождения 
гипотез: Логические и~эпистемологические основания.~--- М.: Либроком, 2009. 432~с.

\bibitem{4-gr} %5
\Au{Poelmans J., Elzinga~P., Viaene~S., Dedene~G.} Formal concept analysis in knowledge 
discovery: A~survey~// Conceptual structures: From information to intelligence~/ Eds.\ M.~Croitoru, 
S.~Ferr$\acute{\mbox{e}}$, and D.~Lukose.~--- Lecture notes in computer science 
ser.~--- Berlin--Heidelberg: Springer, 2010. Vol.~6208.  P.~139--153.

\bibitem{6-gr}
\Au{Панкратова~Е.\,С., Финн~В.\,К.} Автоматическое по\-рож\-де\-ние гипотез в~интеллектуальных 
системах.~--- М.: Либроком, 2009. 528~с. 
\bibitem{7-gr}
\Au{Денисов А.\,А., Колесников~Д.\,Н.} Теория больших систем управления.~--- Л.: Энергоиздат, 1982. 488~с.

\bibitem{9-gr}
\Au{Грушо А.\,А., Грушо Н.\,А., Забежайло~М.\,И., Смирнов~Д.\,В., Тимонина~Е.\,Е.} 
Параметризация в~прикладных задачах поиска эмпирических причин~// Информатика и~её 
применения, 2018. Т.~12. Вып.~3. С.~62--66.

\bibitem{8-gr}
\Au{Грушо А.\,А., Грушо Н.\,А., Левыкин~М.\,В., Тимонина~Е.\,Е.} Методы идентификации 
захвата хоста в~распределенной ин\-фор\-ма\-ци\-он\-но-вы\-чис\-ли\-тель\-ной сис\-те\-ме, 
защищенной с~помощью метаданных~// Информатика и~её применения, 2018. Т.~12. Вып.~4. 
С.~41--45.

 \end{thebibliography}

 }
 }

\end{multicols}

\vspace*{-3pt}

\hfill{\small\textit{Поступила в~редакцию 03.04.19}}

%\vspace*{8pt}

%\pagebreak

\newpage

\vspace*{-28pt}

%\hrule

%\vspace*{2pt}

%\hrule

%\vspace*{-2pt}

\def\tit{ARCHITECTURAL DECISIONS IN~THE~PROBLEM 
OF~IDENTIFICATION OF~FRAUD IN~THE~ANALYSIS 
OF~INFORMATION FLOWS IN~DIGITAL ECONOMY\\[-5pt]}


\def\titkol{Architectural decisions in~the~problem 
of~identification of~fraud in~the~analysis 
of~information flows in~digital economy}

\def\aut{A.\,A.~Grusho, M.\,I.~Zabezhailo, N.\,A.~Grusho, and~E.\,E.~Timonina}

\def\autkol{A.\,A.~Grusho, M.\,I.~Zabezhailo, N.\,A.~Grusho, and~E.\,E.~Timonina}

\titel{\tit}{\aut}{\autkol}{\titkol}

\vspace*{-13pt}


 \noindent
   Institute of Informatics Problems, Federal Research Center ``Computer Sciences and 
Control'' of the Russian Academy of Sciences; 44-2~Vavilov Str., Moscow 119133, 
Russian Federation

\def\leftfootline{\small{\textbf{\thepage}
\hfill INFORMATIKA I EE PRIMENENIYA~--- INFORMATICS AND
APPLICATIONS\ \ \ 2019\ \ \ volume~13\ \ \ issue\ 2}
}%
 \def\rightfootline{\small{INFORMATIKA I EE PRIMENENIYA~---
INFORMATICS AND APPLICATIONS\ \ \ 2019\ \ \ volume~13\ \ \ issue\ 2
\hfill \textbf{\thepage}}}

\vspace*{3pt}


   
     
   \Abste{An approach to a~research of some types of fraud in digital economy with the usage of relationships of 
cause and effect is formulated. In all types of the considered frauds, the discrepancy between the 
purposes of financial transactions and actual cost of achievement of these purposes
has to be observed. Data on 
transactions can be collected by observing information flows in which these transactions are reflected. 
The architecture of data collection and their analysis can be organized by means of the distributed 
ledgers with the centralized consensus that allows creating an analog of the electronic account book 
fixing financial and economic activity of subjects of digital economy in the region. 
   The methods of fraud identification considered are based on the contradictions 
between actions described in transactions and information, which is contained in plans, standards, 
precedents, etc. 
   The method based on a~simplified scheme of implementation of the abstract project is considered. 
For identification of contradictions, it is necessary to carry out the analysis from the effect to the cause, 
i.\,e., to look for anomalies in information describing the generation of the observed effects. 
   It is shown how in implementation of the project it is possible to allocate simple ``necessary 
conditions'' of violation of cause and effect relationships, i.\,e., a~set of ``necessary conditions'' 
violation of which demonstrates fraud existence. It is possible to call this set of "necessary conditions" 
by metadata for control of the project for fraud identification.} 
   
   \KWE{digital economy; information flows; relationships of reason and effect; detection of 
fraudulent schemes}
   
  

 \DOI{10.14357/19922264190204}

\vspace*{-20pt}

 \Ack
   \noindent
   The work was partially supported by the Russian Foundation for Basic Research (projects  
18-29-03081 and 18-07-00274).



%\vspace*{6pt}

  \begin{multicols}{2}

\renewcommand{\bibname}{\protect\rmfamily References}
%\renewcommand{\bibname}{\large\protect\rm References}

{\small\frenchspacing
 {\baselineskip=10.5pt
 \addcontentsline{toc}{section}{References}
 \begin{thebibliography}{9}
\bibitem{1-gr-1}
\Aue{Grusho, A.\,A., A.\,A.~Zatsarinny, and E.\,E.~Timonina.} 2019. Blokcheyny tsifrovoy ekonomiki 
na baze sistemy situatsionnykh tsentrov i~tsentralizovannogo konsensusa [Blockchains of digital 
economy on the basis of the system of the situational centres and the centralized consensus]. 
\textit{25th Scientific and Technical Conference (International) ``Radar-Location, Navigation, 
Communication'' Proceedings}. Voronezh: VSU Publs. 6:183--191.
\bibitem{2-gr-1}
\Aue{Grusho, A., A.~Zatsarinny, and E.~Timonina.} 2019 (in press). 
A~system approach to information security 
in distributed ledgers on the situational centers platform. 
Lecture notes in computer science ser. Springer.
\bibitem{3-gr-1}
\Aue{Finn, V.\,K.} 2011. \textit{Iskusstvennyy intellekt: Metodologiya, primeneniya, filosofiya} 
[Artificial intelligence: Methodology, applications, philosophy]. Moscow: KRASAND. 448~p.

\bibitem{5-gr-1}
\Aue{Anshakov, O.\,M., and E.\,F.~Fabrikantova}. 2009. \textit{DSM-metod avtomaticheskogo porozhdeniya gipotez: Logicheskie 
i~epistemologicheskie osnovaniya} [JSM-method of automatic hypothesis generation: Logical and 
epistemological]. Moscow: KD LIBROKOM. 432~p.
\bibitem{4-gr-1} %5
\Aue{Poelmans, J., P.~Elzinga, S.~Viaene, and G.~Dedene.} 2010. Formal concept analysis in 
knowledge discovery: A~survey. \textit{Conceptual structures: From information to intelligence}. 
Eds.\ M.~Croitoru, S.~Ferr$\acute{\mbox{e}}$, and D.~Lukose. Lecture notes in 
computer science ser. Berlin--Heidelberg: Springer. 6208:139--153.

\bibitem{6-gr-1}
\Aue{Pankratov, E.\,S., and V.\,K.~Finn}. 
2009. \textit{Avtomaticheskoe porozhdenie gipotez v~intellektual'nykh 
sistemakh} [Automatic hypotheses generation in intelligent systems]. Moscow: KD 
\mbox{LIBROKOM}.  528~p. 
\bibitem{7-gr-1}
\Aue{Denisov, A.\,A., and D.\,N.~Kolesnikov.} 1982. \textit{Teoriya bol'shikh 
sistem upravleniya} [Theory of big control systems]. Leningrad: Energoizdat. 488~p.

\bibitem{9-gr-1}
\Aue{Grusho, A.\,A., N.\,A.~Grusho, M.\,I.~Zabezhailo, D.\,V.~Smirnov, and 
E.\,E.~Timonina.} 2018. 
Parametrizatsiya v~prikladnykh zadachakh poiska empiricheskikh prichin 
[Parametrization in applied 
problems of search of the empirical reasons]. 
\textit{Informatika i~ee Primeneniya~--- 
Inform. Appl.} 12(3):62--66.

\bibitem{8-gr-1}
\Aue{Grusho, A.\,A., N.\,A.~Grusho, M.\,V.~Levykin, and E.\,E.~Timonina.} 2018. Metody 
identifikatsii zakhvata khosta v~raspredelennoy informatsionno-vychislitel'noy sisteme, 
zashchishchennoy s~pomoshch'yu metadannykh [Methods of identification of host capture 
in the  distributed information system which is protected on the base of meta data].
\textit{Informatika i~ee 
Primeneniya~--- Inform. Appl.} 12(4):41--45.
{ %\looseness=1

}

\end{thebibliography}

 }
 }

\end{multicols}

\vspace*{-12pt}

\hfill{\small\textit{Received April 3, 2019}}

%\pagebreak

%\vspace*{-18pt}

\Contr

\noindent
\textbf{Grusho Alexander A.} (b.\ 1946)~--- Doctor of Science in physics and 
mathematics, professor, principal scientist, Institute of Informatics Problems, 
Federal Research Center ``Computer Sciences and Control'' of the Russian 
Academy of Sciences; 44-2~Vavilov Str., Moscow 119133, Russian Federation; 
\mbox{grusho@yandex.ru} 

\vspace*{3pt}

\noindent
\textbf{Zabezhailo Michael I.} (b.\ 1956)~--- Doctor of Science in physics and 
mathematics, principal scientist, Institute of Informatics Problems, Federal Research 
Center ``Computer Sciences and Control'' of the Russian Academy of Sciences;  
44-2~Vavilov Str., Moscow 119133, Russian Federation; 
\mbox{m.zabezhailo@yandex.ru} 

\vspace*{3pt}


\noindent
\textbf{Grusho Nikolai A.} (b.\ 1982)~--- Candidate of Science (PhD) in physics 
and mathematics, senior scientist, Institute of Informatics Problems, Federal 
Research Center ``Computer Sciences and Control'' of the Russian Academy of 
Sciences; 44-2~Vavilov Str., Moscow 119133, Russian Federation; 
\mbox{info@itake.ru} 

\vspace*{3pt}


\noindent
\textbf{Timonina Elena E.} (b.\ 1952)~--- Doctor of Science in technology, 
professor, leading scientist, Institute of Informatics Problems, Federal Research 
Center ``Computer Sciences and Control'' of the Russian Academy of Sciences;  
44-2~Vavilov Str., Moscow 119133, Russian Federation; 
\mbox{eltimon@yandex.ru} 

\label{end\stat}

\renewcommand{\bibname}{\protect\rm Литература}    %6



%%%%%%%%%%%%%%%%%%%%%%%%%%%%%%%%%%%%%%%%%%%%%%%%%%%%%%%%%%%%%%%%%
\def\P{{\bf P}}                       
\def\E{{\bf E}}                       
\def\DDD{{\bf D}}
\def\N{{\bf N}}                    
\def\AD{{\cal A}}                       



\def\stat{faz}

\def\tit{ЗАКОНЫ ПОВТОРНОГО ЛОГАРИФМА ДЛЯ ЧИСЛА БЕЗОШИБОЧНЫХ
БЛОКОВ ПРИ ПОМЕХОУСТОЙЧИВОМ КОДИРОВАНИИ}

\def\titkol{Законы повторного логарифма для~числа безошибочных
блоков при~помехоустойчивом кодировании}

\def\autkol{А.\,Н.~Чупрунов, И.~Фазекаш}
\def\aut{А.\,Н.~Чупрунов$^1$, И.~Фазекаш$^2$}

\titel{\tit}{\aut}{\autkol}{\titkol}

%{\renewcommand{\thefootnote}{\fnsymbol{footnote}}\footnotetext[1]
%{Исследование поддержано грантами РФФИ 08-07-00152 и 09-07-12032.
%Статья написана на основе материалов доклада, представленного на IV 
%Международном семинаре  <<Прикладные задачи теории вероятностей и математической статистики, 
%связанные с моделированием информационных систем>> (зимняя сессия, Аоста, Италия, январь--февраль 2010~г.).}}

\renewcommand{\thefootnote}{\arabic{footnote}}
\footnotetext[1]{Научно-исследовательский институт математики и механики
им.\ Н.\,Г.~Чеботарева, achuprunov@mail.ru}
\footnotetext[2]{Дебреценский университет, fazekas.istvan@inf.unideb.hu}

\vspace{-8pt}


\Abst{Рассматриваются  сообщения, состоящие из блоков.
Каждый блок кодируется помехоустойчивым кодом, который может
исправить не более $r$~ошибок. При этом предполагается, что число
ошибок в блоке~--- неотрицательная целочисленная случайная величина.
Эти случайные величины независимы и одинаково распределены. Кроме
того, предполагается, что число ошибок в сообщении принадлежит
некоторому конечному подмножеству множества неотрицательных чисел. 
Получены аналоги закона повторного логарифма для случайной
величины~--- числа безошибочных блоков в сообщении.}

\vspace{-2pt}

\KW{обобщенная схема размещения; условная
вероятность; условное математическое ожидание; экспоненциальное
неравенство; закон повторного логарифма; код БЧХ}

\vspace{-1pt}

       \vskip 14pt plus 9pt minus 6pt

      \thispagestyle{headings}

      \begin{multicols}{2}

      \label{st\stat}

\section{Введение и основные результаты}

 Будем рассматривать код, который позволяет исправить не
более   $r$~ошибок типа замещения. Частным случаем такого кода
является код Боу\-за--Чоуд\-ху\-ри--Хок\-вин\-гхе\-ма (БЧХ) (см.\ о кодах БЧХ, например, в~[1]). Работа
посвящена изучению асимптотического поведения  случайной величины~$S_N$~--- 
числа безошибочных блоков  в сообщении, состоящем из $N$~блоков, 
причем каждый блок подвергается помехоустойчивому
кодированию, а число ошибок в сообщении принадлежит некоторому
конечному множеству. Будем предполагать, что все рассматриваемые
случайные величины определены на вероятностном пространстве
$(\Omega,
\mathfrak{A}, {\mathbf P})$.


 Рассмотрим  сообщение, состоящее из $N$~блоков. Пусть случайная величина~$\xi_{Nj}$~---
 число ошибок в $j$-м блоке. Будем предполагать, что  $\xi_{Nj}$, $1\le
j\le N$,~--- независимые неотрицательные целочисленные случайные
величины распределенные так же, как случайная величина~ $\xi$.   Кроме того,  будем считать,
 что число ошибок в сообщении  принадлежит некоторому конечному
 подмножеству~$M$ множества неотрицательных целых  чисел.
  Тогда число безошибочных блоков в  сообщении~--- случайная величина

\noindent
$$
S_{MN}=\sum_{i=1}^NI_{MNi}\,,
$$
где $I_{MNi}$~--- индикатор события~$A_{MNi}$, состоящего в том, что
$i$-й блок  сообщения имеет не более $r$~ошибок. Заметим, что
событие

\noindent
\begin{multline*}
A_{MNi}=\left \{\xi_{Ni}\le r\,\,\, \vert |\,\,\, \xi_{N1}+\xi_{N2}+ \dots {}\right.\\
\left.{}\dots +\xi_{NN}\in
M\, \right \}=\{\xi_{Ni}\le r\,\,\, \vert |\,\,\, A\,\ \}\,,
\end{multline*}
где событие

\noindent
\begin{multline*}
A= A_{MN}=\left\{\xi_{N1}+\xi_{N2}+ \dots +\xi_{NN}\in
M\, \right\}={}\\
{}=\cup_{k\in M}A_k\,,
\end{multline*}
а события
%\noindent
$$ A_{k}=A_{kN}=\left\{ \xi_{N1}+\xi_{N2}+ \dots +\xi_{NN}=k\right\}\,.
$$

Будем предполагать, что распределение случайной величины~$\xi$
зависит от параметра~$\theta$. Пусть существует последовательность
неотрицательных чисел~ $b_0$, $b_1, \dots$ такая, что радиус
сходимости~$R$ ряда
$$ 
B(\theta)=\sum_{k=0}^{\infty}\fr{b_k\theta^k}{k!}
$$
положителен. Тогда  случайная величина $\xi=$\linebreak $=\xi(\theta)$, $
0<\theta<R$ распределена по следующему закону:
$$
p_k=p_k(\theta)={\mathbf
P}\{\xi=k\}=\fr{b_k\theta^k}{k!B(\theta)}\,,\enskip k=0,1,2,\dots
$$
Будем  предполагать, что выполняется условие~$A_1$, т.\,е.\
 $b_0>0$, $b_1>0$.

Если множество~$M$ состоит из одного элемента, то события~$A_{MN}$
являются событиями обобщенной схемы размещения и их вероятности не
зависят от~$\theta$. Обобщенная схема размещения была введена В.\,Ф.~Колчиным в~[2] (см.\ также монографию~[3]).

 Условие $A_1$ и случайные величины~ $\xi(\theta)$ были введены в~[4]. В~[4--6] 
 получены предельные теоремы для сумм независимых случайных
величин~$\xi_{Ni}(\theta)$. В частности, в~[4] показано, что
математическое ожидание
$$
m=m(\theta)=  {\E}\xi=  \fr{\theta B'\left(\theta\right)}{B(\theta)}
$$
и дисперсия
$$
 \sigma^2=\sigma^2(\theta)=  {\DDD}^2\xi=
\fr{\theta^2 B^{\prime\prime} (\theta)}{B(\theta)}+ \fr{\theta
B'(\theta)}{B(\theta)}- \fr{\theta^2
(B'(\theta))^2}{(B(\theta))^2}\,.
$$
Поэтому дисперсия
\begin{equation}
\sigma^2(\theta)=\theta m'(\theta)\,. \label{e1.1faz}
\end{equation}


Пусть $0<\theta'<\theta^{\prime\prime}<R$. Если $\sigma^2(\theta)=0$ для
некоторого  $\theta\in [\theta', \theta^{\prime\prime}]$, то случайная
величина $\xi(\theta)$~--- константа. Но так как $b_0>0$, $b_1>0$,
случайная величина~$\xi(\theta)$ константой не является. Поэтому
$\sigma^2(\theta)$, $\theta\in [\theta', \theta^{\prime\prime}]$~---
положительная непрерывная функция. Следовательно,
\begin{multline}
0<C_1 =\inf_{\theta\in [\theta', \theta^{\prime\prime}]} \sigma^2(\theta) \le
\sup_{\theta\in [\theta', \theta^{\prime\prime}]}\sigma^2(\theta)={}\\
{}= C_2<\infty\,.
\label{e1.2faz}
\end{multline}
Условия~(\ref{e1.1faz}) и~(\ref{e1.2faz}) влекут
$$
0<\fr{C_1}{\theta^{\prime\prime}}=\inf_{\theta\in [\theta',
\theta^{\prime\prime}]}m'(\theta) \le \sup_{\theta\in [\theta',
\theta^{\prime\prime}]}m'(\theta)= \fr{C_2}{\theta'}<\infty\,.
$$
Таким образом, $m(\theta)$, $\theta\in [\theta', \theta^{\prime\prime}]$,~---
положительная непрерывная строго возрастающая функция. Обозначим
через~$m^{-1}$ ее обратную функцию, $m(R)=\lim_{\theta\to
R-0}m(\theta)$. Будем обозначать через $\xi_1(\theta), \dots ,
\xi_N(\theta)$ независимые копии случайной величины
$\xi=\xi(\theta)$.

Обозначим $n'_M=\sup\{n: n\in M\}$, $\alpha_{MN}=$\linebreak $={n'_M}/{N}$. Будем
предполагать, что $\alpha_{MN}<m(R)$. Пусть
$\theta_{MN}=m^{-1}(\alpha_{MN})$. Тогда случайная величина
$T=(1/\sqrt{N})\sum\limits_{i=1}^NI_{\{\xi_{Ni}(\theta_{NM})\le r\}}$
имеет дисперсию $\sigma^2_{MN}=p_{\le r}(\theta_{MN})(1-p_{\le
r}(\theta_{MN}))$, где $p_{\le
r}(\theta_{MN})=\sum\limits_{k=0}^rp_{k}(\theta_{MN})$. Будем использовать
следующую оценку для дисперсии случайной величины~$T$. Пусть
$\theta'\le\theta_{MN}\le\theta^{\prime\prime}$. Тогда
\begin{equation}
\sigma^2_{MN}\ge \fr{\sum_{k=0}^rb_k(\theta')^k}{B(\theta^{\prime\prime})}
\,
\fr{\sum_{k=r+1}^{\infty}b_k(\theta')^k}{B(\theta^{\prime\prime})}=C_3>0\,.
\label{e1.3faz}
\end{equation}

Основными результатами статьи являются следующие аналоги закона
повторного логарифма для числа безошибочных блоков~--- случайной
величины~$S_{MN}(\theta_{MN})$.

\medskip

\noindent

\textbf{Теорема 1.} {\it  Пусть  $0<\alpha'< \alpha^{\prime\prime}<m(R)$. Пусть
$M_N$~--- такая последовательность конечных подмножеств множества
неотрицательных чисел, что $\alpha'\le\alpha_{M_NN}\le\alpha^{\prime\prime}$.
Обозначим $S_{N}=S_{M_NN}(\theta_{M_NN})$, $\sigma^2_N=
\sigma^2_{M_NN}$. Тогда
\begin{equation}
\limsup_{N\to\infty}\fr{|S_N-{\E}S_N|}{\sqrt{N\ln(N)}\sigma_N}\le
2\sqrt{6} 
\label{e1.4faz}
\end{equation}
почти наверное. }

\medskip

\noindent

\textbf{Теорема 2.} {\it Пусть  $0<\alpha'< \alpha^{\prime\prime}<m(R)$. Пусть
$M_n$~---  последовательность конечных подмножеств множества
неотрицательных чисел.  Обозначим $S_{nN}=S_{M_nN}(\theta_{M_nN})$,
$\sigma^2_{nN}= \sigma^2_{M_nN}$.  Тогда
\begin{equation}
\limsup_{N,n\to\infty,\alpha'<\alpha_{nN}< \alpha^{\prime\prime}
 }
\fr{|S_{nN}-{\E}S_{nN} |}{\sqrt{N\ln(N)}\sigma_{nN}}
 \le 2\sqrt{10}
\label{e1.5faz}
\end{equation}
почти наверное. }


%\medskip

\section{Леммы}

Пусть  $A\in\AD$~--- такое фиксированное событие, что $\P(A)>0$.
Напомним, что условная вероятность~${\P^A}$ определяется формулой
 $$
 \P^A(B)=\fr{\P(B\cap A)}{\P(A)}\,,\enskip  B \in\AD\,.
 $$
Будем обозначать через ${\E}^A$  математическое ожидание
относительно вероятности $\P^A$.

Легко видеть, что для любой случайной величины $S$ и  для любого
 $p>0$ справедливо неравенство ${\E}^A|S|^p\le (1/\P(A)){\E}|S|^p$.
 Следующая лемма показывает, что аналогичное неравенство верно для
центрированных абсолютных моментов случайной величины $S$.

\medskip

\noindent
\textbf{Лемма 2.1.} {\it Пусть $1\le  p<\infty$. Тогда
\begin{equation}
 {\E}^A|S-{\E}^AS|^p   \le  2^p\fr{{\E}|S-{\E}S|^p}{{\P}(A)}\,.
\label{e2.1faz}
\end{equation}}

\medskip

Лемма~2.1 доказана в~[7].

\medskip

Будем обозначать через $r_i$, $i\in\N$,  функции Радемахера, $\E^r$~--- 
математическое ожидание относительно $\sigma$-алгебры,
определенной случайными величинами~$r_i$,$\gamma$~--- гауссовская
случайная величина с нулевым средним и единичной дисперсией.
Воспользуемся  неравенством Хинчина, в котором константа имеет
наиболее точный вид.

\medskip

\noindent
\textbf{Лемма 2.2.} {\it Пусть $1\le p<\infty$. Пусть $c_i\in{\bf R}$,
$1\le i\le n$. Тогда
\begin{multline}
{\E_r}\left|\sum\limits_{i=1}^n c_ir_i\right|^p \le{}\\
{}\le
\sqrt{2}\left(1+O\left(\fr{1}{p}\right)\right)e^{-p/2}
p^{p/2} \left(\sum\limits_{i=1}^n(c_i)^2\right)^{p/2}\,.
\label{e2.2faz}
\end{multline}
}

\medskip

\noindent
Д\,о\,к\,а\,з\,а\,т\,е\,л\,ь\,с\,т\,в\,о\,.\ В~[8] доказано неравенство Хинчина с
неулучшаемой константой
\begin{equation}
{\E_r}\left|\sum\limits_{i=1}^n c_ir_i\right|^p \le {\mathbf
E}|\gamma|^p\left(\sum\limits_{i=1}^n(c_i)^2\right)^{p/2}\,.
\label{e2.3faz}
\end{equation}
Применяя оценку для гамма-функции~$\Gamma(p)$, полученную в~[9],
имеем
\begin{multline}
 {\mathbf
E}|\gamma|^p=\fr{2^{p/2}}{\sqrt{\pi}}\,\Gamma\left(\fr{p}{2}+\fr{1}{2}\right)={}\\[2pt]
{}=
\fr{2^{p/2}}{\sqrt{2\pi}}\,e^{-p/2-1/2}\left(\fr{p}{2}+\fr{1}{2}\right)^{p/2+1/2}\times{}\\[2pt]
{}\times
\sqrt{\fr{2\pi}{\left(p/2+1/2\right)}}\left(1+O\left(\fr{1}{p}\right)\right)={}\\[2pt]
{}=
\sqrt{2}2^{p/2}e^{-p/2-1/2}\left(\fr{p}{2}+\fr{1}{2}\right)^{p/2}
\left(1+O\left(\fr{1}{p}\right)\right)={}\\[2pt]
{}=\sqrt{2}e^{-p/2}e^{-1/2}\left(p+1\right)^{{p}/{2}}
\left(1+O\left(\fr{1}{p}\right)\right)={}\\[2pt]
{}=
\sqrt{2}e^{-p/2}e^{-1/2}\left(\fr{p+1}{p}\right)^{p/2}p^{p/2}
\left(1+O\left(\fr{1}{p}\right)\right)\le{}\\[2pt]
{}\le
\sqrt{2}e^{-{p}/{2}}e^{-{1}/{2}}e^{{1}/{2}}p^{p/2}
\left(1+O\left(\fr{1}{p}\right)\right)\,.
\label{e2.4faz}
\end{multline}
Используя~(\ref{e2.4faz}) в~(\ref{e2.3faz}), получаем~(\ref{e2.2faz}). Лемма доказана.

\medskip

\noindent

\textbf{Лемма 2.3.} {\it  Пусть $\eta_i$, $1\le i\le n$,~---
независимые случайные величины с математическими ожиданиями~$a_i$ и
дисперсиями~$\sigma_i^2$ такие, что $0\le\eta_i\le 1$. Обозначим
$$
a=\fr{1}{n} \sum_{i=1}^na_i\,;\quad \sigma^2=\fr{1}{n}\sum_{i=1}^n\sigma^2_i\,.
$$
Пусть $p\ge 2$. Тогда
\begin{equation}
a^p\le {\E}\left(\frac{ \sum_{i=1}^n\eta_i}{n}\right)^p\le a^p(1+B)\,,
\label{e2.5faz}
\end{equation}
где
$$
B= \fr{\sigma^2}{2}f_2\left(\fr{p}{a\sqrt{n}}\right)\,,
$$
а функция
\begin{multline*}
f_2(x)=x^2\left(2{\mathbf
E}\left(\gamma^2\exp{(x|\gamma|)}\right)-1\right)={}\\
{}=
4x^2\left(e^{x^2/2}(x^3+3x)\Phi(x)+\fr{x}{\sqrt{2\pi}}\right)\,,
\end{multline*}
где $\Phi$~--- функция распределения случайной величины~$\gamma$.
}

\medskip

\noindent
Д\,о\,к\,а\,з\,а\,т\,е\,л\,ь\,с\,т\,в\,о\,.\  Левое неравенство в~(\ref{e2.5faz}) следует из
неравенства Иенсена. Пусть семейство $\{\eta'_i, 1\le i\le n\}$~---
независимая копия семейства $\{\eta_i, 1\le i\le n\}$.  Функция
$g(x)=|a+x|^p$  имеет непрерывную вторую производную. Поэтому по
формуле Тейлора имеем
$$
g(x)=a^p+\fr{pa^{p-1}}{1!}x+\frac{p(p-1)|a+\theta
x|^{p-2}}{2!}x^2\,,
$$
где $\theta\in (-1, 1)$. Отсюда при $x={
\sum\limits_{i=1}^n(\eta_i-a_i)}/{n}$ получается равенство:
\begin{multline*}
{\E}\left|\frac{ \sum_{i=1}^n\eta_i}{n}\right|^p=a^p +
\fr{p(p-1)}{2}{\E}\times{}\\
{}\times \left|
a'+\theta\frac{
\sum_{i=1}^n(\eta_i-a_i)}{n}\right|^{p-2}\left(\frac{
\sum_{i=1}^n(\eta_i-a_i)}{n}\right)^2={}\\
{}=a^p(1+B')\,,
\end{multline*}
где $\theta=\theta(\omega)$, $-1\le\theta\le 1$. Так как $1+x\le
e^x$, $0\le x<\infty$, и
\begin{multline*}
{\E}\left|\frac{
\sum_{i=1}^n(\eta_i-\eta'_i)^2}{n}\right|^{(k+2)/2}\!\!\le{}\\
{}\le {\E}\left(\frac{
\sum_{i=1}^n(\eta_i-\eta'_i)^2}{n}\right)=2\sigma^2,
\end{multline*}
используя~(\ref{e2.2faz}), получаем
\begin{multline*}
B'\le\fr{p(p-1)}{2}\,{\E}\left(1+\left|\frac{
\sum_{i=1}^n(\eta_i-a_i)}{na}\right|\right)^{p-2}\times{}\\
{}\times 
\left(\frac{
\sum_{i=1}^n(\eta_i-a_i)}{na}\right)^2\le{}\\
{}\le\fr{p(p-1)}{2}{\E}\exp\left((p-2)\left|\frac{
\sum_{i=1}^n(\eta_i-a_i)}{na}\right|\right)\times{}
\end{multline*}
\begin{multline*}
{}\times\left(\frac{
\sum_{i=1}^n(\eta_i-a_i)}{na}\right)^2\le{}\\[2pt]
{}
\le\fr{p(p-1)}{2}\sum_{k=0}^{\infty}\fr{1}{k!}(p-2)^k{\E}\left|\frac{
\sum_{i=1}^n(\eta_i-a_i)}{na}\right|^{k+2}\le{}\\[2pt]
{}
\le\fr{1}{2}\,\fr{p(p-1)}{a^2n}\sum_{k=0}^{\infty}\fr{1}{k!}\left(\fr{p-2}{a\sqrt{n}}
\right)^k\times{}\\[2pt]
{}\times {\E}\left|\frac{
\sum_{i=1}^n(\eta_i-a_i)}{\sqrt{n}}\right|^{k+2}\le{}\\[2pt]
{}
\le\fr{1}{2}\,\fr{p(p-1)}{a^2n}\left(
\vphantom{\frac{
\sum_{i=1}^n(\eta_i-\eta'_i)^2}{n}}
\sigma^2+\sum_{k=1}^{\infty}\fr{1}{k!}\left(\fr{p-2}{a\sqrt{n}}
\right)^k\right.\times {}\\[2pt]
{}\left.{}\times {\E}|\gamma|^{k+2}{\E}\left|\frac{
\sum_{i=1}^n(\eta_i-\eta'_i)^2}{n}\right|^{(k+2)/2}\right)\le{}\\[2pt]
{}\le
\fr{\sigma^2}{2}\,\fr{p(p-1)}{a^2n}\left(1+2\sum_{k=1}^{\infty}\fr{1}{k!}\left(\fr{p-2}{a\sqrt{n}}
\right)^k{\E}|\gamma|^{k+2}\right)={}\\[9pt]
{}=
\fr{\sigma^2}{2}\,\fr{p(p-1)}{a^2n}\times{}\\[2pt]
{}\times \left(1+2\left({\E}\left(\gamma^2\exp
\left(\fr{p-2}{a\sqrt{n}}|\gamma|\right)\right)-1 \right)
\right) \le B\,.
\end{multline*}
Доказательство закончено.

\bigskip

Пусть $A_i$, $1\le i\le n$~--- независимые события, $p_i=\P(A_i)$, $I_i$~--- индикаторы событий $A_i$, $\sigma^2_i =
 p_i(1-p_i)$~--- дисперсии индикатора~$I_i$. Будем предполагать, что $0<p_i<1$, $1\le i\le n$. 
 Рас\-смот\-рим случайную величину
$$
\mu=\sum_{i=1}^nI_i\,.
$$
Тогда $\sigma^2=(1/n)\sum\limits_{i=1}^{n}\sigma_i^2$ -- дисперсия
случайной величины $(1/\sqrt{n})\mu$.

Основной результат этого параграфа~--- сле\-ду\-ющее экспоненциальное
неравенство.

\medskip

\noindent
\textbf{Лемма 2.4.} {\it Пусть $\varepsilon>0$. Тогда
\begin{multline}
\P^A\left\{ \fr{|\mu-{\E^A}\mu |}{\sqrt{n}\sigma} \ge
\varepsilon\right\} \le
\fr{\sqrt{2}}{\P(A)}\left(1+O\left(\fr{1}{\varepsilon^2}\right)
\right)\times{}\\[9pt]
{}\times
\left(1+\fr{1}{8}f_2\left(\fr{\varepsilon^2}{32\sigma^2\sqrt{n}}\right)\right)
e^{-{\varepsilon^2}/16 }\,. 
\label{e2.6faz}
\end{multline}
}

%\medskip

\noindent
Д\,о\,к\,а\,з\,а\,т\,е\,л\,ь\,с\,т\,в\,о\,.\ Пусть  семейство $\{I'_i, 1\le i\le$\linebreak $\le n\}$~---
независимая копия семейства $\{I_i, 1\le$\linebreak $\le i\le n\}$. Будем
предполагать, что $\{I_i, 1\le$\linebreak $\le i\le n\}$, $\{I'_i, 1\le i\le n\}$ и
$\{r_i, 1\le i\le n\}$~--- независимые семейства. Так как
$0<\sigma^2\le 1/4$, то
$\sigma^2(1-2\sigma^2)\le {1}/{2}$. Поэтому, используя леммы~2.1--2.3 и неравенство Иенсена,
 получаем
\begin{multline}
\P^A\left\{\fr{|\mu-{\E}^A\mu|}{\sqrt{n}\sigma} \ge
\varepsilon\right)\le
\fr{1}{\varepsilon^p}{\E^A}\left|\fr{\mu-{\E}^A\mu}{\sqrt{n}\sigma}\right|^p
\le{}\\[9pt]
{}\le \fr{2^p}{\varepsilon^p\P(A)}{\E}\left|\frac{\mu-{\E}\mu}{\sqrt{n}{\sigma}}\right|^p
\le{}\\[9pt]
{}
\le\fr{2^p}{\varepsilon^p\P(A)}{\E}\left|\frac{\sum_{i=1}^n(I_i-p_i)}{\sqrt{n}{\sigma}}\right|^p \le{}\\[9pt]
{}
\le\fr{2^p}{\varepsilon^p\P(A)}{\E}\left|\frac{\sum_{i=1}^n(I_i-I'_i)}{\sqrt{n}{\sigma}}\right|^p={}\\
{}
=\fr{2^p}{\varepsilon^p\P(A)}{\E_r}{\E}\left|\frac{\sum_{i=1}^nr_i
(I_i-I'_i)}{\sqrt{n}{\sigma}}\right|^p
\le\fr{\sqrt{2}2^p}{\varepsilon^p\P(A)\sigma^p
}\times{}\\
{}\times \left(1+O\left(\fr{1}{p}\right)\right)e^{-p/2}
p^{p/2}{\E}\left(\frac{\sum_{i=1}^n
(I_i-I'_i)^2}{n}\right)^{p/2}\!\!\!\le{}\\
{}
\le\fr{\sqrt{2}2^p}{\varepsilon^p\P(A)\sigma^p
}\left(1+O\left(\fr{1}{p}\right)\right)e^{-{p}/{2}} p^{
{p}/{2}} 2^{{p}/{2}}\sigma^p{\E}\times{}\\
{}\times
\left(1+\sigma^2(1-2\sigma^2)
f_2\left(\fr{p}{4\sigma^2\sqrt {n}}\right)\right)\le{}\\
{}
\le\fr{\sqrt{2}2^p}{\varepsilon^p\P(A)\sigma^p
}\left(1+O\left(\fr{1}{p}\right)\right)e^{-{p}/{2}} 
p^{{p}/{2}}2^{{p}/{2}}\sigma^p{\E}\times{}\\
{}\times
\left(1+\fr{1}{8}
f_2\left(\fr{p}{4\sigma^2\sqrt {n}}\right)\right)\,. 
\label{e2.7faz}
\end{multline}

При $p={\varepsilon^2}/{8}$ из~(\ref{e2.7faz}) следует~(\ref{e2.6faz}).

 Пусть\ \, $ \theta'=m^{-1}( \alpha')$, $
\theta^{\prime\prime}=m^{-1}( \alpha^{\prime\prime})$. Заметим, что $ \theta'\le \theta
\le \theta^{\prime\prime}$ тогда и только тогда, когда $ \alpha'\le \alpha\le
\alpha^{\prime\prime}$.


\medskip

\noindent
\textbf{Лемма 2.5.} {\it Пусть $0< \alpha'< \alpha^{\prime\prime}<m(R)$. Пусть
$\alpha = n'_M/N$, $\theta=m^{-1}(\alpha)$. Существует $N_0\in{\N}$
со свойством: если $n, N\in {\N}$ такие, что $N>N_0$, а $
\alpha'\le\alpha\le \alpha^{\prime\prime}$, то
\begin{equation}
 \P(A_{MN}) > \fr{1}{4\sqrt{C_1}\sqrt{N}}\,.
\label{e2.8faz}
\end{equation}
}

\medskip

\noindent
Д\,о\,к\,а\,з\,а\,т\,е\,л\,ь\,с\,т\,в\,о\,.\ В~силу теоремы~4 из~[4] существует $N_0\in
{\N}$ такое, что если
 $N>N_0$ и $\alpha'\le\alpha\le \alpha^{\prime\prime}$, то
\begin{multline*}
 \sigma(\theta)\sqrt{N} \P(A_{n'_MN})-{}\\
{}-\fr{1}{\sqrt{2\pi}}\exp\left\{-\fr{(n-m(\theta)N)^2}{2\sigma^2(\theta)N}\right\}
>\fr{1}{4}-\fr{1}{ \sqrt{2\pi}}\,.
\end{multline*}
Так как $m(\theta)=\alpha$, имеем
$$
\fr{1}{\sqrt{2\pi}}\,\exp\left\{-\fr{(n-m(\theta)N)^2}{2\sigma^2(\theta)N}\right\}=\fr{1}{\sqrt{2\pi}}\,.
$$
Поэтому
\begin{equation}
\sigma(\theta)\sqrt{N} \P(A_{nMN})\ge \sigma(\theta)\sqrt{N}
\P(A_{n'_MN})
>\fr{1}{4}\,.
\label{e2.9faz}
\end{equation}
Из~(\ref{e2.9faz}) и~(\ref{e1.2faz}) следует~(\ref{e2.8faz}). Лемма доказана.

\medskip

\section{Доказательства теорем}

\noindent
Д\,о\,к\,а\,з\,а\,т\,е\,л\,ь\,с\,т\,в\,о\,\ теоремы~1.  Выберем $N_0\ge 2$ такое, что
справедливо утверждение  леммы~2.5.
 Пусть $t>2\sqrt{6}$. Тогда
${t^2}/{16}-{1}/{2}>1$. Поэтому, используя лемму~2.4, лемму~2.5 и~(\ref{e1.3faz}), получаем
\begin{multline*}
 \sum_{k=N_0+1}^{\infty} \P\left\{\fr{|S_k-{\E}S_k|}{\sqrt{k\ln(k)}\sigma_k} \ge t\right\} ={}\\
 {}=\sum_{k=N_0+1}^{\infty}
\P\left\{\fr{|S_{k}-{\E}S_{k}|} {\sqrt{k}\sigma_k}\ge
\sqrt{\ln(k)}t\right\}\le{}
\\
{}\le\sqrt{2}\sum_{k=N_0+1}^{\infty}4\sqrt{C_1}\sqrt{k}\left(1+O\left(\fr{1}{t^2\ln(k)}\right)
\right)\times{}\\
{}\times \left(1+f_2\left(\fr{t^2\ln(k)}{32C_3
\sqrt{k}}\right)\right)e^{-{(\sqrt{\ln(k)}t)^2}/{16}}\le{}\\
{}
\le 4\sqrt{2}\sqrt{C_1}
\sum_{k=N_0+1}^{\infty}\left(1+O\left(\fr{1}{t^2\ln(k)}\right)
\right)\times{}\\
{}\times
\left(1+f_2\left(\fr{t^2\ln(k)}{32C_3\sqrt{k}}\right)\right)
k^{-t^2/16 +1/2}<\infty\,.
\end{multline*}
Следовательно, для всех $t>2\sqrt{6}$  по лемме Бореля--Кантелли
\begin{equation}
\limsup_{k\to\infty}\fr{|S_{k}-{\E}S_{k}|}{\sqrt{k\ln(k)}\sigma_k}\le t
\label{e3.1faz}
\end{equation} 
почти наверное. Из~(\ref{e3.1faz}) следует~(\ref{e1.4faz}). Теорема доказана.
%

\medskip

\noindent

Д\,о\,к\,а\,з\,а\,т\,е\,л\,ь\,с\,т\,в\,о\,\ теоремы~2. Выберем $N_0\ge$\linebreak $\ge 2$ такое, что
справедливо утверждение  леммы~2.5.
 Пусть $t>2\sqrt{10}$. Тогда
${t^2}/{16}-1-{1}/{2}>1$. Поэтому, используя  леммы~2.4 и~2.5 и~(\ref{e1.3faz}), получаем

\begin{multline*}
\sum\limits_{N=N_0+1}^{\infty}\sum\limits_{N \alpha'\le n\le
\alpha^{\prime\prime}N}
\!\!\!\!\!\P\left\{\fr{|S_{nN}-{\E}S_{nN}|}{\sqrt{N\ln(N)}\sigma_{nN}}\ge
t\right\} = {}\\
{}=
\sum\limits_{N=N_0+1}^{\infty}\sum\limits_{N \alpha'\le
n\le \alpha^{\prime\prime}N}
\!\!\!\!\!\P\left\{\fr{|S_{nN}-{\E}S_{nN}|}{\sqrt{N}\sigma_{nN}}\ge\right.{}\\
\left.{}\ge
 \sqrt{\ln(N)}t
\vphantom{\fr{|S_{nN}-{\E}S_{nN}|}{\sqrt{N}\sigma_{nN}}}
\right\}\le
\sum_{N=N_0+1}^{\infty}\sum\limits_{N \alpha'\le n\le
\alpha^{\prime\prime}N}
\!\!\!\!\!4\sqrt{2}\sqrt{C_1}\sqrt{N}\times{}
\end{multline*}
\begin{multline*}
{}\times \left(1+O\left(\fr{1}{\varepsilon^2\ln(N)}\right)
\right)\times{}\\[2pt]
{}\times
\left(1+f_2\left(\fr{\varepsilon^2\ln{N}}{32C_3\sqrt{N}}\right)\right)e^{-{(\sqrt{\ln(N)}t)^2}/616}\le{}\\[2pt]
{}
\le 4\sqrt{2}\sqrt{C_1}\! \!\sum_{N=N_0+1}^{\infty}\!\!\!( \alpha^{\prime\prime}-
\alpha')N\left(1+O\left(\fr{1}{\varepsilon^2\ln(N)}\right) \right)\times{}\\[2pt]
{}\times
\left(1+f_2\left(\fr{\varepsilon^2\ln{N}}{32C_3\sqrt{N}}\right)\right)
N^{-t^2/16+{1}/2}<\infty\,.
 \end{multline*}
 
 \noindent
Следовательно, для всех $t>2\sqrt{10}$  по лемме Бо\-ре\-ля--Кан\-тел\-ли
\begin{equation}
\limsup_{n,N\to\infty\,,\,\alpha'<\alpha<
\alpha^{\prime\prime}}\fr{|S_{nN}-{\E}S_{nN}|}{\sqrt{N\ln(N)\sigma_{nN}}}\le t
\label{e3.2faz}
\end{equation}
почти наверное. Из~(\ref{e3.2faz}) следует~(\ref{e1.5faz}). Теорема доказана.


{\small\frenchspacing
{%\baselineskip=10.8pt
\addcontentsline{toc}{section}{Литература}
\begin{thebibliography}{9}

\bibitem{1faz}
\Au{Питерсон У., Уэлдон Э.} 
Коды, исправляющие ошибки.~--- М.: Мир, 1976. 596~с.

\bibitem{2faz}
\Au{Колчин В.\,Ф.} 
Один класс предельных теорем для условных распределений~// Литовск. матем. сб., 1968. Т.~8. №\,1. С.~53--63.

\bibitem{3faz}
\Au{Колчни В.\,Ф.} Случайные графы.~--- М.: Физматгиз, 2000.

\bibitem{4faz}
\Au{Колчин А.\,В.} 
Предельные теоремы для обобщенной схемы размещения~// Дискрет. матем., 2003. Т.~15. №\,4. С.~143--157.

\bibitem{5faz}
\Au{Колчин А.\,В., Колчин В.\,Ф.} 
О переходе распределений сумм
независимых одинаково распределенных случайных величин с одной
решетки на другую в обобщенной схеме размещения~// Дискрет. матем.,
2006. Т.~18. №\,4. С.~113--127.

\bibitem{6faz}
\Au{Колчин А.\,В., Колчин В.\,Ф.} Переход с одной решетки на
другую распределений сумм случайных величин, встречающихся в
обобщенной схеме размещения~// Дискрет. матем., 2007. Т.~19. №\,3. С.~15--21.

\bibitem{7faz}
\Au{M$\acute{\mbox{o}}$ri T.} 
Sharp inequalities between centered moments~// 
J. Inequalities in Pure and Applied Mathematics, 2009.
Vol.~10. Is.\,4. Art.~99.

\bibitem{8faz}
\Au{Haagerup U.} 
The best constant in Khinchin inequality~// Stud. Math., 1982. Vol.~70. P.~231--283.

\label{end\stat}

\bibitem{9faz}
\Au{Nemes G.} 
New asymptotic expansion for the  $\Gamma(z)$ function~// Stan's Library, 2007. Vol.~II. P.~31.
 \end{thebibliography}
}
}


\end{multicols} %7


\newcommand{\be}{\beta}


\newcommand{\III}{{\bf 1}}
\newcommand{\n}{{\cal N}}  % каллиграфические буквы

\newcommand{\ercx}[1]{\:{\sf erfc}\left(#1\right)}
\newcommand{\ercf}{{\ercx{\frac{b}{2\sqrt{a}}}}}
\newcommand{\erx}[1]{\:{\sf erf}\left(#1\right)}
\newcommand{\cab}{C_{a,b}}
%\newcommand{\cabf}{\fr{\sqrt{a}}{\sqrt{\pi}\ercf\exp\left(\frac{b^2}{4a}\right)}}
\newcommand{\iab}{I_{a,b}}
\newcommand{\sign}[1]{\:{\sf sign}\left(#1\right)}

\def\stat{lyamin}

\def\tit{О ПРЕДЕЛЬНОМ ПОВЕДЕНИИ МОЩНОСТЕЙ КРИТЕРИЕВ В~СЛУЧАЕ ОБОБЩЕННОГО РАСПРЕДЕЛЕНИЯ ЛАПЛАСА}

\def\titkol{О предельном поведении мощностей критериев в~случае обобщенного распределения Лапласа}

\def\autkol{О.\,О.~Лямин}
\def\aut{О.\,О.~Лямин$^1$}

\titel{\tit}{\aut}{\autkol}{\titkol}

%{\renewcommand{\thefootnote}{\fnsymbol{footnote}}\footnotetext[1]
%{Исследование поддержано грантами РФФИ 08-07-00152 и 09-07-12032.
%Статья написана на основе материалов доклада, представленного на IV 
%Международном семинаре  <<Прикладные задачи теории вероятностей и математической статистики, 
%связанные с моделированием информационных систем>> (зимняя сессия, Аоста, Италия, январь--февраль 2010~г.).}}

\renewcommand{\thefootnote}{\arabic{footnote}}
\footnotetext[1]{Московский государственный
университет им. М.\,В. Ломоносова, факультет вычислительной математики и кибернетики,
oleg.lyamin@gmail.com}

\Abst{В работе~\cite{article} на эвристическом уровне была получена формула для предела 
отклонения мощности асимптотически наиболее мощного критерия от мощности наилучшего критерия 
в случае обобщенного распределения Лапласа. В~данной работе приводится формальное доказательство этой формулы.}

\KW{обобщенное распределение Лапласа; функция
мощности; асимптотически наиболее мощный критерий; асимптотическое
разложение}

       \vskip 14pt plus 9pt minus 6pt

      \thispagestyle{headings}

      \begin{multicols}{2}

      \label{st\stat}

\section{Введение}

Распределение Лапласа находит широкое применение в задачах
моделирования больших рисков, выделения сигналов на фоне помех и
других задачах математической статистики (см., например,~[2--4]). Данная работа продолжает
исследования, начатые в~\cite{article}, и содержит строгое
доказательство результатов, полученных в указанной работе на
эвристическом уровне. Здесь используются те же обозначения, что и в~\cite{article}.

В работе~\cite{article} была рассмотрена задача проверки гипотезы
$$
{\sf H}_0 \::\: \theta = 0
$$
против последовательности близких альтернатив вида
$$
{\sf H}_{n,1} \::\: \theta = \fr{t}{\sqrt{n}}\,, \quad 0 < t \leq C\,, \enskip C > 0\,,
$$
на основе выборки $(X_1, \dots, X_n)$~--- независимых одинаково распределенных наблюдений, 
имеющих распределение с плотностью вида
\begin{equation}
\label{l1}
p(x, \theta) = \cab e^{-a(x - \theta)^2- b|x - \theta|}\,,\quad x, \theta \in {\mathbb{R}}\,,
\end{equation}
где $\cab$~--- константа нормировки. Предположим, что $a>0$, $b>0$. В~этом случае константа нормировки 
$$
\cab = \fr{\sqrt{a}}{\sqrt{\pi}\;{\sf erfc}\left (b/(2\sqrt{a})\right )\exp \left(b^2/(4a)\right)}\,,
$$
где
$$
\ercx{x} = \fr{2}{\sqrt{\pi}}\int\limits_x^{\infty} e^{-z^2}\,dz\,.
$$
Для каждого фиксированного $t\in(0,C]$ обозначим через~$\beta_n^*(t)$ мощность наилучшего критерия 
уровня $\alpha \in (0,1)$. Такой критерий всегда существует согласно фундаментальной лемме 
Неймана--Пир\-со\-на и основан на логарифме отношения правдоподобия
\begin{multline*}
\Lambda_n(t) = \sum_{i = 1}^n\left(atn^{-1/2}\left(2X_i - tn^{-1/2}\right) +{}\right.\\
\left.{}+ b\left(|X_i| - |X_i - tn^{-1/2}|\right)\right)\,.
%\label{l2}
\end{multline*}
Для проверки ${\sf H}_0$ против~${\sf H}_{n,1}$ существуют критерии, основанные на отличных от~$\Lambda_n(t)$ 
статистиках и имеющие ту же предельную мощность, что и~$\beta_n^*(t)$. Такие критерии называются 
асимптотически наиболее мощными (АНМ), причем эти критерии не зависят от~$t$. Предположим, что статистику~$T_n$ 
некоторого АНМ-кри\-те\-рия можно монотонным преобразованием (не меняющим мощности критерия) преобразовать в 
статистику~$S_n(t)$ такую, что величина $\Delta_n(t) = S_n(t) - 
\Lambda_n(t)$ допускает при гипотезе~${\sf H}_0$ асимптотическое разложение вида
\begin{equation}
\Delta_n(t) = n^{-1/2} L_n(t) +  o (n^{-1/2})\,,
\label{l3}
\end{equation}
где $L_n(t)$~--- некоторая статистика. Было обнаружено (см.~\cite{bening}), что в этом случае 
для широкого класса АНМ-кри\-те\-ри\-ев мощность~$\beta_n(t)$ критерия, основанного на~$S_n(t)$ (или на~$T_n$), 
отличается от~$\beta_n^*(t)$ на величину порядка~$1/n$.

Для распределения~(\ref{l1}) рассмотрим АНМ-кри\-те\-рий, основанный на статистике
\begin{equation*}
T_n = \fr{1}{\sqrt{n}} \sum_{i = 1}^n\left(2aX_i + b\sign{X_i}\right)\,.
%\label{l4}
\end{equation*}
Обозначим через~$\beta_n(t)$ мощность этого критерия уровня $\alpha \in (0,1)$. 
В~работе \cite{article} было показано, что для этого критерия величина~$\Delta_n(t)$ допускает при 
гипотезе~${\sf H}_0$ асимптотическое разложение несколько иного, чем в~(\ref{l3}), вида, а именно
$$
\Delta_n(t) = n^{-1/4} L_n(t) +  o (n^{-1/4})\,,
$$
в связи с чем мощность критерия~$\beta_n(t)$ отличается от~$\beta_n^*(t)$ на величину порядка $1/\sqrt{n}$, 
а не~$1/n$, как следовало ожидать. Там же на эвристическом уровне была получена формула
\begin{multline}
r(t) \equiv \lim_{n \to \infty} \sqrt{n}\left(\beta_n^*(t) - \be_n(t)\right) = {}\\
{}=
\fr{2b^2\cab t^2}{3\sqrt{\iab}} \varphi\left(u_{\alpha} - t\sqrt{\iab}\right)\,,
\label{l5}
\end{multline}
где $\Phi(x)$, $\varphi(x)$~--- соответственно функция распределения и плотность стандартного нормального закона, 
$\Phi(u_{\alpha}) = 1 - \alpha$, $\alpha \in (0,\,1)$; $\iab$~--- фишеровская информация.

В этой работе для распределения~(\ref{l1}) будут проверены условия теоремы~3.2.1 из~\cite{bening}, т.\,е.\ 
получено формальное доказательство формулы~(\ref{l5}).

\section{Основные обозначения}

В соответствии с определениями работы~\cite{article} введем следующие обозначения:
\begin{align}
S_n(t) &= tT_n - \fr{\iab t^2}{2}\,;\notag\\
\Delta_n(t) &= S_n(t) - \Lambda_n(t) = -b\cab t^2 -{}\notag\\
&{}-2b\sum\limits_{i=1}^n\left(X_i-\fr{t}{\sqrt{n}}\right)
{\III}_{\left[0,t/\sqrt{n}\right]}(X_i)\,,
\label{l7}
\end{align}
где ${\III}_A(x)$~--- индикатор множества~$A$. В~дальнейшем будем опускать аргумент~$t$ 
и писать просто $\Lambda_n, S_n$ и~$\Delta_n$. Также положим
$$
\erx{x} = 1 - \ercx{x} = \fr{2}{\sqrt{\pi}} \int\limits_0^x e^{-z^2}\,dz\,.
$$
Для $\theta > 0$ и произвольного $x \in {\mathbb{R}}$ определим функции
\begin{equation}
g_{\theta}(x) = \cab \theta^2 + 2(x-\theta){\III}_{[0,\theta]}(x)\,;
\label{l8}
\end{equation}
\vspace*{-6pt}

\noindent
\begin{multline}
m_{\theta}(x) = a(2x - \theta) + b
\left[ \vphantom{\fr{2x}{\theta}}
-{\III}_{(-\infty,\,0)}(x) +{}\right.\\
{}\left.+ \left(\fr{2x}{\theta} - 1\right){\III}_{[0,\,\theta]}(x) + 
{\III}_{(\theta,\,\infty)}(x)\right]\,;
\label{l9}
\end{multline}

\vspace*{-6pt}

\noindent
\begin{equation}
M_{\theta}(x) = \theta m_{\theta}(x)\,;
\label{l10}
\end{equation}

\vspace*{-6pt}

\noindent
\begin{equation*}
d_{\theta}(x) = -\fr{\iab \theta}{2} + 2ax+b\sign{x}\,;
%\label{l11}
\end{equation*}
\begin{equation*}
D_{\theta}(x) = \theta d_{\theta}(x)\,.
%\label{l12}
\end{equation*}
Обозначим $g_n(x)$, $m_n(x)$, $M_n(x)$, $d_n(x)$ и $D_n(x)$ соответственно функции~$g_{\theta}(x)$, 
$m_{\theta}(x)$,  $M_{\theta}(x)$, $d_{\theta}(x)$ и~$D_{\theta}(x)$ при $\theta=tn^{-1/2}$.

Перепишем статистики в обозначениях введенных функций:
\begin{align}
\Lambda_n &= tn^{-1/2} \sum\limits_{i=1}^n m_n(X_i) = \sum\limits_{i=1}^n M_n(X_i)\,; \label{l13}\\
S_n &= tn^{-1/2} \sum\limits_{i=1}^n d_n(X_i) = \sum\limits_{i=1}^n D_n(X_i)\,; \label{l14}\\
\Delta_n &= -b\sum\limits_{i=1}^n g_n(X_i)\,. \label{l15}
\end{align}
Договоримся обозначать $f_{S_n}(s)$, $f_{\Lambda_n}(s)$, $f_{m_{\theta}}(s)$, 
$f_{M_{\theta}}(s)$, $f_{d_{\theta}}(s)$, $f_{D_{\theta}}(s)$ характеристические
 функции соответственно случайных величин~$S_n$, 
$\Lambda_n$, $m_{\theta}(X_1)$, $M_{\theta}(X_1)$, $d_{\theta}(X_1)$, $D_{\theta}(X_1)$. 
Смена индекса~$\theta$ на~$n$ будет обозначать, что эти функции рассматриваются при $\theta=tn^{-1/2}$. Заметим, что
\begin{multline*}
f_{M_{\theta}}(s) = {\e}_0 \exp\left\{isM_{\theta}(X_1)\right\} ={}\\
{}= {\e}_0 \exp\left\{is\theta m_{\theta}(X_1)\right\} = f_{m_{\theta}}(\theta s)\,,
\end{multline*}
$$
f_{D_{\theta}}(s) = f_{d_{\theta}}(\theta s)\,.
$$
Аналогично
\begin{align}
f_{M_n}(s) & = f_{m_n}(tn^{-1/2}s)\,, \label{l16}\\
f_{D_n}(s)&= f_{d_n}(tn^{-1/2}s)\,. \label{l17}
\end{align}
Также будем пользоваться показанным в~\cite{article} соотношением между константой нормировки и фишеровской информацией
$$
\iab = 2(b\cab+a)\,.
$$

\section{Вспомогательные результаты}

\noindent
\textbf{Лемма~3.1.}
{\it Для любых $0 < t_1 < t_2$ и распределения}~(\ref{l1}) \textit{справедливы равенства}
\begin{multline*}
\sup_{t\in[t_1,t_2]} \sup_{x \in {\mathbb{R}}} 
 \left| \vphantom{\fr{f_2(x)}{\sqrt{n}}}
 {\p}_{n, 0}\left(\Lambda_n < x\right) -{}\right.\\
\left. {}- f_1(x,t)
- \fr{f_2(x,t)}{\sqrt{n}} - \fr{f_3(x,t)}{n}\right| = o(n^{-1})\,,
\end{multline*}
\textit{где}
$$
f_1(x,t) = {\Phi}\left(\fr{x}{t\sqrt{I_{a,b}}} + \fr{\sqrt{I_{a,b}}t}{2}\right)\,;
$$

\noindent
\begin{multline*}
f_2(x,t) ={}\\
{}= \fr{b^2C_{a,b}t}{3\sqrt{I_{a,b}}}\left(\fr{x}{tI_{a,b}} - 
\fr{t}{2}\right) \varphi\left(\fr{x}{t\sqrt{I_{a,b}}} + \fr{\sqrt{I_{a,b}}t}{2}\right)\,;
\end{multline*}

\noindent
\begin{multline*}
f_3(x,t) = \fr{bC_{a,b}}{3I_{a,b}}\, \varphi\left(\fr{x}{t\sqrt{I_{a,b}}} + 
\fr{\sqrt{I_{a,b}}t}{2}\right) \times{}\\
{}\times \left\{\fr{t^2}{2\sqrt{I^3_{a,b}}}\left(
\fr{M_1(a, b, C_{a,b})x}{t} + \fr{2M_2(a, b, C_{a,b})t}{3I_{a,b}}\right) -{}\right.\\
{}- \fr{b^3C_{a,b}t^2}{6}\left(\fr{x}{tI_{a,b}} - \fr{t}{2}\right)^2\left(
\fr{x}{t\sqrt{I_{a,b}}} + \fr{\sqrt{I_{a,b}}t}{2}\right)- {}\\
{}-
\fr{M_3(a,b)t}{2\sqrt{I_{a,b}}}\left(\fr{x}{t\sqrt{I_{a,b}}}+ 
\fr{\sqrt{I_{a,b}}t}{2}\right)^2 +{}\\
{}
+ \fr{M_3(a,b)t}{2\sqrt{\iab}}+ \fr{M_3(a,b)}{4I_{a,b}}\left(
\fr{x}{t\sqrt{I_{a,b}}} + \fr{\sqrt{I_{a,b}}t}{2}\right)^3 - {}\\
\left.{}-
\fr{3M_3(a,b)}{4I_{a,b}}\left(\fr{x}{t\sqrt{I_{a,b}}} + \fr{\sqrt{I_{a,b}}t}{2}\right)
\vphantom{\fr{t^2}{2\sqrt{I^3_{a,b}}}}
\right\}\,;
\end{multline*}

\noindent
$$
M_1(a, b, C_{a,b}) = 6b^2\cab^2 + b(14a - b^2)\cab + 2a(4a - b^2)\,;
$$

\noindent
\begin{multline*}
M_2(a, b, C_{a,b}) = 18b^4\cab^4 + b^3(66a - b^2)\cab^3 +{}\\
{}
+ 2ab^2(45a-b^2)\cab^2 + a^2b(54a - b^2)\cab + 12a^4\,;
\end{multline*}

\noindent
$$
M_3(a,b) = 3I_{a,b} - b^2\,.
$$

\medskip

\noindent
Д\,о\,к\,а\,з\,а\,т\,е\,л\,ь\,с\,т\,в\,о\,. \
Для доказательства утверждения этой леммы применим предложение~1.1 из работы~\cite{chibisov} для $r = 4$.

Покажем равномерную интегрируемость семейства случайных величин $m_{\theta}^4(X_1)$ относительно~${\p}_0$. 
По определению равномерной интегрируемости необходимо показать, что
$$
\sup_{\theta} {\e}_0\left|m_{\theta}^4(X_1)\right| {\III}_{(c, \infty)}(\left|m_{\theta}^4(X_1)\right|) \to 0
\mbox{ при } c\to \infty\,.
$$
Используя~(\ref{l9}), представим ${\III}_{(c,\infty)}\left(|m_{\theta}^4(X_1)|\right)$ в следующем виде:
\begin{multline}
{\III}_{(c,\,\infty)}(|m_{\theta}^4(x)|) =
{\III}_{(\sqrt[4]{c},\,\infty)}(|m_{\theta}(x)|) ={}\\
{}
= {\III}_{\left(-\infty,\,x_1(c)\right) \cup \left(x_2(c),\,\infty\right)}(x)\cdot{\III}_{\left((a\theta + 
b)^4,\,\infty\right)}(c) +{}\\
{}+ {\III}_{\left(-\infty,\,x_3(c)\right) \cup \left( x_4(c),\,\infty\right)}(x)\cdot{\III}_{\left(0,\,(a\theta + 
b)^4\right)}(c)\,,
\label{l18}
\end{multline}
где
\begin{align*}
x_1(c) &= -\fr{\sqrt[4]{c}- a\theta - b}{2a}\,; &
x_2(c) &= \fr{\sqrt[4]{c} + a\theta - b}{2a}\,;\\
x_3(c) &= -\fr{\sqrt[4]{c} - a\theta - b}{2\left(a + {b}/{\theta}\right)}\,; &
x_4(c) &= \fr{\sqrt[4]{c} + a\theta + b}{2\left(a + b/\theta\right)}\,.
\end{align*}
При больших~$c$ второе слагаемое в~(\ref{l18}) равно нулю. Тогда для всякого $\theta>0$ 
непосредственным интегрированием получим
\begin{multline*}
{\e}_0 \left\vert m_{\theta}^4(X_1)\right\vert {\III}_{(c,\,\infty)}
(|m_{\theta}^4(X_1)|) ={}\\
{}= {\e}_0\left|m_{\theta}^4(X_1)\right|{\III}_{\left(-\infty,\,x_1(c)\right) \cup 
\left(x_2(c),\,\infty\right)}(X_1) ={}\\
{}= P_1(x_1(c)) \exp\left\{-\fr{(b + 2ax_1(c))^2}{4a}\right\} +{}\\
{}+ C_1 \ercx{\fr{b + 2ax_1(c)}{2\sqrt{a}}} +{}
\\
{}+ P_2(x_2(c)) \exp\left\{-\fr{(b + 2ax_2(c))^2}{4a}\right\} + {}\\
{}+
C_2 \ercx{\fr{b + 2ax_2(c)}{2\sqrt{a}}} \to 0 \mbox{ при } c \to \infty\,,
\end{multline*}
где $C_1$, $C_2$~--- константы, зависящие от~$a$, $b$ и~$\theta$; 
$P_1(y)$ и~$P_2(y)$~--- многочлены третьей степени с коэффициентами, зависящими от~$a,\,b$ и~$\theta$. 
Первое условие предложения~1.1 из~\cite{chibisov} доказано.

\smallskip

Для выполнения второго условия достаточно показать выполнение условия АС, для которого, 
в свою очередь, достаточно проверить условия теоремы~1.5 из~\cite{chibisov}.

\smallskip

Выберем $a_1$ и $a_2$ так, чтобы $a_2 > a_1 > t > 0$. 
Тогда условия~I, II и~III теоремы~1.5 очевидно выполнены. Условие~IV следует из того, что для 
каж\-дого~$\theta$ плотность $p_{\theta}(x)$ абсолютно непрерывна на любом отрезке, не содержащем 
точку $x = \theta$, а значит для любого $\theta \in [0, t]$ плотность~$p_{\theta}(x)$ абсолютно непрерывна 
на $x \in [a_1, a_2]$, так как $a_2 > a_1 > t$.

\smallskip

Обратимся к условию~V. Для каждого $\theta \in [0,\,t]$ функция~$p'_{\theta}(x)$ на отрезке 
$a_2 > x > a_1 > t \geq \theta$ определена как
\begin{multline*}
p'_{\theta}(x) = \fr{\sqrt{a} e^{-{b^2}/(4a)}}{\sqrt{\pi}\;{\sf erfc}\left(b/(2\sqrt{a})\right)}\times{}\\
{}\times e^{-b(x - \theta) - a(x - 
\theta)^2}\left(-b - 2a(x - \theta)\right)\,.
\end{multline*}
Тогда
\begin{multline*}
\int_{a_1}^{a_2} \left|p'_{\theta}(x)\right|^{2 + \delta}\,dx =
 \left(\fr{\sqrt{a} e^{-{b^2}/(4a)}}{\sqrt{\pi}\;{\sf erfc}\left(b/(2\sqrt{a})\right)}\right)^{2+\delta}\times{}\\
\!{}\times
\int\limits_{a_1}^{a_2}\! e^{-(a(x-\theta)^2+b(x-\theta))(2+\delta)} 
\left(b+2a(x-\theta)\right)^{2+\delta}\, dx \leq\hspace*{-0.78734pt}\\
\leq \left(\fr{\sqrt{a} e^{-{b^2}/(4a)}}{\sqrt{\pi}\;{\sf erfc}\left(b/(2\sqrt{a})\right)}\right)^{2+\delta} \times{}\\
{}\times
\int\limits_{a_1}^{a_2} \left(b + 2a(x - \theta)\right)^{2 + \delta}\, dx \leq\hspace*{10mm}
 \end{multline*}
 \begin{multline*}
{}
\leq \left(\fr{\sqrt{a} e^{-{b^2}/(4a)}}{\sqrt{\pi}\;{\sf erfc}\left(b/(2\sqrt{a})\right)}\right)^{2+\delta}\times{}\\
{}\times\left(b + 2a(a_2 - \theta)\right)^{2 + 
\delta}\left(a_2 - a_1\right) \leq{}\\
{}
\leq \left(\fr{\sqrt{a} e^{-{b^2}/(4a)}}{\sqrt{\pi}\;{\sf erfc}\left(b/(2\sqrt{a})\right)}\right)^{2+\delta}\,
\!\!\!\!\!\!\left(b + 2aa_2\right)^{2 + \delta}\,(a_2 - a_1)\,.\hspace*{-1.38pt}
\end{multline*}
Все условия теоремы~1.5 из~\cite{chibisov} выполнены, что дает возможность применить предложение~1.1, 
из которого непосредственно получается утверждение леммы. \hfill $\Box$


\smallskip

\noindent
\textbf{Лемма 3.2.}
\textit{При $\theta \to 0$ для функции $g_{\theta}(x)$ $($см.}\ (\ref{l8})) \textit{справедливы следующие соотношения:}
\begin{align*}
{\e}_{0} g_{\theta}(X_1) &= {\cal O}(\theta^3)\,;\\
{\e}_{0} g_{\theta}^2(X_1) &= {\cal O}(\theta^3)\,;\\
{\e}_{0} g_{\theta}^3(X_1) &= {\cal O}(\theta^4)\,.
\end{align*}

%\smallskip

\noindent
Д\,о\,к\,а\,з\,а\,т\,е\,л\,ь\,с\,т\,в\,о\,.\
В~работе~\cite{article} была получена характеристическая функция для случайной величины 
$(X_1-\theta){\III}_{[0,\theta]}(X_1)$. Из нее выводится характеристическая функция для $g_{\theta}(X_1)$, 
после чего задача вычисления моментов становится тривиальной. \hfill $\Box$


\smallskip

\noindent
\textbf{Лемма 3.3.}
\textit{Для статистики $\Delta_n$ $($см.} (\ref{l7})) \textit{при $n \to \infty$ справедливо соотношение}
$$
{\e}_{n,0} |\Delta_n|^k = {\cal O}\left(n^{-k/4}\right)\,,
$$
\textit{где $k$ --- произвольное натуральное число.}


\smallskip

\noindent
Д\,о\,к\,а\,з\,а\,т\,е\,л\,ь\,с\,т\,в\,о\,.\ Смотри доказательство следствия~3.1 в работе~\cite{article}.
\hfill $\Box$


\smallskip

\noindent
\textbf{Лемма 3.4.}
\textit{Справедливы следующие представления:}
\begin{multline*}
{\e}_{n,0}\exp\{is\Lambda_n\}\Delta_n ={}\\[6pt]
{}= -b n f^{n-1}_{M_n}(s) {\e}_0 \exp\left\{isM_n(X_1)\right\} g_n(X_1)\,;
\end{multline*}
\vspace*{-18pt}

\noindent
\begin{multline*}
{\e}_{n,0}\exp\{is\Lambda_n\}\Delta_n^2 ={}\\[6pt]
{}= b^2 n  f_{M_n}^{n-1}(s)  {\e}_0 \exp\left\{is M_n(X_1)\right\} g_n^2(X_1) +{}\\[6pt]
{}
+ b^2n(n-1) f_{M_n}^{n-2}(s)  \left[{\e}_0 \exp\left\{is M_n(X_1)\right\} g_n(X_1)\right]^2\,;
\end{multline*}
\vspace*{-18pt}

\noindent
\begin{multline*}
{\e}_{n,0}\exp\{isS_n\}\Delta_n ={}\\[6pt]
{}= -b n f^{n-1}_{D_n}(s) {\e}_0 \exp\left\{isD_n(X_1)\right\} g_n(X_1)\,.
\end{multline*}


\medskip

\noindent
Д\,о\,к\,а\,з\,а\,т\,е\,л\,ь\,с\,т\,в\,о\,.\
 Пользуясь~(\ref{l13}), (\ref{l15}), а также независимостью случайных величин $X_1, \dots, X_n$, 
 получаем первое утверждение
 
 \noindent
\begin{multline*}
{\e}_{n,0}\exp\{is\Lambda_n\}\Delta_n ={}\\[4pt]
{}
= -b  {\e}_{n,0}\sum_{i=1}^n \exp\left\{is\sum_{j=1}^n M_n(X_j)\right\} g_n(X_i)={}\\[4pt]
{}
= -b \sum\limits_{i=1}^n {\e}_{n,0} \exp\left\{is\sum_{j=1}^n M_n(X_j)\right\} g_n(X_i) ={}\\[4pt]
{}
= -b \sum\limits_{i=1}^n {\e}_{n,0}\left(
\vphantom{\prod^n_{j=1}}
\exp\left\{isM_n(X_i)\right\} g_n(X_i) \times{}\right.\\[4pt]
\left.{}\times \prod\limits_{j=1,  j\ne i}^n \exp\left\{isM_n(X_j)\right\} \right) ={}\\[4pt]
{}
= -b \sum\limits_{i=1}^n \left(
\vphantom{\prod^n_{j=1}}
{\e}_0 \exp\left\{isM_n(X_i)\right\} g_n(X_i) \times {}\right.\\[4pt]
\left.{}\times
\prod\limits_{j=1,j\ne i}^n {\e}_0 \exp\left\{isM_n(X_j)\right\}\right) ={}\\[4pt]
{}
= -b  f_{M_n}^{n-1}(s)  \sum\limits_{i=1}^n {\e}_0 \exp\left\{isM_n(X_i)\right\} g_n(X_i) ={}\\[4pt]
{}
= -b n f^{n-1}_{M_n}(s) {\e}_0 \exp\left\{isM_n(X_1)\right\} g_n(X_1)\,.
\end{multline*}
Второе утверждение доказывается по аналогии.

%\medskip

Чтобы доказать третье утверждение, необходимо воспользоваться представлением~(\ref{l14}) и 
пол\-ностью повторить доказательство первого утверж\-дения с заменой~$M_n$ на~$D_n$. Лемма доказана.
\hfill $\Box$

\medskip


\noindent
\textbf{Лемма 3.5.}
\textit{При $\theta \to 0$ для функций $M_{\theta}(x)$ $($см.}~(\ref{l10})) \textit{и~$g_{\theta}(x)$ 
$($см.}~(\ref{l8})) 
\textit{справедливо следующее соотношение:}
$$
{\e}_0 \left|M_{\theta}(X_1) g_{\theta}(X_1)\right| = {\cal O}(\theta^{5/2})\,.
$$


\medskip

\noindent
Д\,о\,к\,а\,з\,а\,т\,е\,л\,ь\,с\,т\,в\,о\,.\
По неравенству Коши--Бу\-ня\-ков\-ского
$$
{\e}_0 \left|M_{\theta}(X_1) g_{\theta}(X_1)\right| \leq 
\sqrt{{\e}_0 M_{\theta}^2(X_1)  {\e}_0 g_{\theta}^2(X_1)}\,.
$$
Используя полученную в работе~\cite{article} характеристическую функцию случайной величины~$M_{\theta}(X_1)$, 
вычислим второй момент $M_{\theta}(X_1)$. Очевидно,
$$
{\e}_0 M_{\theta}^2(X_1) = {\cal O}(\theta^2)\,.
$$
Используя лемму~3.2 для оценки ${\e}_0 g_{\theta}^2(X_1)$, получим утверждение леммы. \hfill $\Box$

\medskip


\noindent
\textbf{Лемма 3.6.}
\textit{При $\theta \to 0$ для функций $M_{\theta}(x)$ и $g_{\theta}(x)$ справедливы следующие соотношения:}
$$
{\e}_0 \exp\left\{isM_{\theta}(X_1)\right\} = 1 + \fr{is(is-1)\iab \theta^2}{2} +  o(\theta^2)\,;
$$

\noindent
\begin{align*}
\hspace*{-0.2pt}{\e}_0 \exp\left\{isM_{\theta}(X_1)\right\} g_{\theta}(X_1) &= \fr{b\cab(1+is)\theta^3}{3} +  o(\theta^3);
\\
\hspace*{-0.2pt}{\e}_0 \exp\left\{isM_{\theta}(X_1)\right\} g_{\theta}^2(X_1) &= \fr{4\cab \theta^3}{3} +  o(\theta^3).
\end{align*}

\medskip

\noindent
Д\,о\,к\,а\,з\,а\,т\,е\,л\,ь\,с\,т\,в\,о\,.\ 
Первое утверждение~--- разложение характеристической функции случайной величины $M_{\theta}(X_1)$ 
в ряд~--- было получено в работе~\cite{article}. Второе и третье утверждения доказываются 
непосредственным взятием соответствующих интегралов и разложением результатов интегрирования в ряд в 
точке $\theta=0$. \hfill $\Box$

\medskip

\noindent
\textbf{Лемма 3.7.}
\textit{Для любого $x\in{\mathbb{R}}$ и произвольного действительного $y\ne0$}
$$
\int\limits_x^{\infty} e^{-z^2}|\cos yz|\,dz < \int\limits_x^{\infty} e^{-z^2}\,dz\,.
$$


\medskip

\noindent
Д\,о\,к\,а\,з\,а\,т\,е\,л\,ь\,с\,т\,в\,о\,.\  Фиксируем произвольное $y\ne$\linebreak $\ne 0$ и рассмотрим функцию
$$
u(z) = e^{-z^2}\left(1-\left\vert \cos yz\right\vert \right)\,.
$$
Функция~$u(z)$ обращается в нуль в точках
$$
z_k=\fr{\pi k}{y}\,, \enskip k=0,\pm1,\pm2,\dots
$$
Обозначим через~$\bar{z}_1$ и~$\bar{z}_2$ произвольные точки серии~$z_k$, лежащие правее~$x$ и следующие 
в серии друг за другом. Выберем $\epsilon > 0$ так, чтобы $\epsilon < (\bar{z}_2-\bar{z}_1)/2$. 
Так как $u(z)\geq0$ для любого~$z$, то по свойствам интегралов
\begin{multline*}
\int\limits_x^{\infty} e^{-z^2}\,dz - \int\limits_x^{\infty} e^{-z^2}|\cos yz|\,dz ={}\\
{}= \int\limits_x^{\infty}u(z)dz \geq \int\limits_{\bar{z}_1+\epsilon}^{\bar{z}_2-\epsilon}u(z)\,dz > 0\,,
\end{multline*}
так как на $z \in (\bar{z}_1+\epsilon,\bar{z}_2-\epsilon)$ выполнено $|\cos yz|<1$, а значит и $u(z) > 0$.
Лемма доказана. \hfill $\Box$

\noindent

\textbf{Следствие.} \textit{Для любого $x\in{\r}$ и произвольного действительного $y\ne0$}
$$
\fr{2}{\sqrt{\pi}}\int\limits_x^{\infty} e^{-z^2}|\cos yz|\,dz < \ercx{x}\,.
$$


\medskip

\noindent
\textbf{Лемма 3.8.}
\textit{Для модуля характеристической функции $f_{D_n}(s)$ справедлива следующая оценка:}
\begin{align*}
|f_{D_n}(s)| &\leq \exp\left\{-\fr{\epsilon_1 s^2}{n}\right\}\,,   & |s|&\leq\sigma_1 \sqrt{n}\,;\\ 
|f_{D_n}(s)| &\leq q_d < 1\,,  & |s|&\geq\sigma_1 \sqrt{n}\,, 
\end{align*}
\textit{где $\sigma_1$, $\epsilon_1$~--- положительные константы; $q_d$~--- положительная константа, 
определенная с точностью до~$a,\,b$.
}

\medskip

\noindent
Д\,о\,к\,а\,з\,а\,т\,е\,л\,ь\,с\,т\,в\,о\,.\  
Рассмотрим характеристическую функцию случайной величины~$d_n(X_1)$. По теореме~1.2 
из~\cite{petrov} существуют положительные константы~$A_1$ и~$B_1$ такие, что
$$
|f_{d_n}(s)| \leq 1 - B_1 s^2
$$
для всех $|s| \leq A_1$. Пользуясь хорошо известной оценкой $1-z\leq e^{-z}$, справедливой для любого~$z$, 
получим, что для всех $|s| \leq A_1$
$$
|f_{d_n}(s)| \leq \exp\left\{-B_1s^2\right\}\,.
$$
Тогда из~(\ref{l17}) имеем
$$
\left|f_{D_n}(s)\right| = \left|f_{d_n}(tn^{-1/2}s)\right| \leq \exp\left\{-\fr{B_1 t^2 s^2}{n}\right\}
$$
для всех $|s| \leq A_1 \sqrt{n}/t$. Полагая $\epsilon_1 = B_1 t^2$ и $\sigma_1 = A_1/t$, 
приходим к первому утверждению леммы. Покажем второе утверждение.

Для любого $\theta>0$ непосредственным интегрированием получаем, что
\begin{multline*}
f_{d_{\theta}}(s) ={}\\
= \fr{\ercx{{b}/(2\sqrt{a})- i\sqrt{a}s} + 
\ercx{{b}/({2\sqrt{a}}) + i\sqrt{a}s}}{2\;{\sf erfc}\left (b/(2\sqrt{a})\right) \exp\left\{{is\iab \theta}/2 + as^2\right\}}.\hspace*{-7.37994pt}
\end{multline*}
Рассмотрим выражение, стоящее в числителе:
\begin{multline*}
\ercx{\fr{b}{2\sqrt{a}} - i\sqrt{a}s} + \ercx{\fr{b}{2\sqrt{a}} + i\sqrt{a}s} \equiv{}\\
{}
\equiv \fr{2}{\sqrt{\pi}}\int\limits_{{b}/(2\sqrt{a}) - i\sqrt{a}s}^{\infty} e^{-z^2}\,dz +{}\\
{}+
 \fr{2}{\sqrt{\pi}}\int\limits_{{b}/(2\sqrt{a}) + i\sqrt{a}s}^{\infty} e^{-z^2}\,dz ={}\\
= \fr{2}{\sqrt{\pi}}\int\limits_{{b}/(2\sqrt{a})}^{\infty}e^{-\left(w^2 - 2i\sqrt{a}sw - as^2\right)}\,dw+{}\\
{}+
 \fr{2}{\sqrt{\pi}}\int\limits_{{b}/(2\sqrt{a})}^{\infty}e^{-\left(w^2 + 2i\sqrt{a}sw - as^2\right)}\,dw ={}\\
{}
= \fr{4e^{as^2}}{\sqrt{\pi}} \int\limits_{{b}/(2\sqrt{a})}^{\infty}e^{-w^2} \cos 2\sqrt{a}\,sw\, dw\,.
\end{multline*}
Подставляя полученное выражение в формулу для $f_{d_{\theta}}(s)$, получим
\begin{equation}
\label{l22}
\left|f_{d_{\theta}}(s)\right| = J(s) \equiv \fr{2\left|I(s)\right|}{\sqrt{\pi}\;{\sf erfc}\left(
b/(2\sqrt{a})\right)}\,,
\end{equation}
где
$$
I(s) = \int\limits_{{b}/(2\sqrt{a})}^{\infty} e^{-w^2} \cos 2\sqrt{a}\,s w\, dw\,.
$$
Обозначим подынтегральную функцию интеграла~$I(s)$ как~$z(w,s)$. По признаку Вейерштрасса (теорема~7.8 
из~\cite{matan}) интеграл в выражении для~$I(s)$\linebreak
сходится равномерно на множестве всех действительных~$s$. 
Кроме того, для любого $w_0 \geq$\linebreak $\geq {b}/(2\sqrt{a})$ функция~$z(w,s)$ равномерно на множестве ${b}/(2\sqrt{a}) \leq 
w \leq w_0$ стремится к~$z(w,s_0)$ при $s\to s_0$, а значит по теореме~7.9 из~\cite{matan} функция~$I(s)$ 
непрерывна на множестве всех действительных~$s$. По признаку Вейерштрасса интеграл
$$
\il{{b}/(2\sqrt{a})}{\infty} we^{-w^2} \sin 2\sqrt{a}\,sw\,dw
$$
сходится равномерно на множестве всех действительных~$s$. Отсюда согласно теореме~7.14 
из~\cite{matan} следует возможность дифференцирования интеграла~$I(s)$ по параметру~$s$ в любой точке~$s$. 
Таким образом, для любого~$s$ взятием интеграла по частям имеем
\begin{multline}
I'(s) = -2\sqrt{a}\il{{b}/(2\sqrt{a})}{\infty} we^{-w^2} \sin 2\sqrt{a}\,sw\,dw ={}\\
{}= \sqrt{a} \exp\left\{-\fr{b^2}{4a}\right\} \sin bs + 2asI(s)\,.
\label{l19}
\end{multline}
Пусть $s \ne 0$. Тогда можно записать
\begin{align}
I(s) &= \fr{I'(s)}{2as} - \fr{\sin bs}{2\sqrt{a}\exp\left\{{b^2}/(4a)\right\} s}\,;\notag\\
|I(s)| &\leq \fr{|I'(s)|}{2a|s|} + \fr{1}{2\sqrt{a}\exp\left\{{b^2}/(4a)\right\}|s|}\,.
\label{l20}
\end{align}
Оценим $|I'(s)|$. По~(\ref{l19}) имеем
$$
\left|I'(s)\right| \leq 2\sqrt{a}\il{{b}/(2\sqrt{a})}{\infty} we^{-w^2}\,dw = 
\sqrt{a} \exp\left\{-\fr{b^2}{4a}\right\}\,.
$$
Подставляя в~(\ref{l20}), получим
$$
\left|I(s)\right| \leq \fr{1}{\sqrt{a} \exp\left\{{b^2}/(4a)\right\}} \, \fr{1}{|s|}\,.
$$
Таким образом, можно утверждать, что
\begin{equation}
\lim_{|s|\to\infty} I(s) = 0\,.
\label{l21}
\end{equation}
Вернемся к~(\ref{l19}). Очевидно, что вышеописанные свойства функции~$I(s)$ целиком переносятся на функцию~$J(s)$. 
Кроме того, $J(s) > 0$ для любого~$s$. Так как функция~$J(s)$ чeтная, последнюю часть доказательства 
проведeм только для положительных~$s$.

Положим
$$
q_d = \sup\limits_{s \geq A_1} J(s)\,,
$$
где $A_1$~--- положительная константа, определенная при доказательстве первого утверждения леммы. 
Обозначим $J_1 = J(A_1)$. Фиксируем некоторое $0< \epsilon < J_1$. По~(\ref{l21}) найдется такое $\delta>0$, 
что для всех $s>\delta$ можно утверждать, что $J(s)<\epsilon$. Тогда очевидно, что
$$
\sup\limits_{A_1 \leq s \leq \delta} J(s) = q_d\,.
$$
Из непрерывности~$J(s)$ следует, что эта точная верхняя грань
достигается, т.\,е.\ найдётся такое~$s_1$, что $A_1 \leq |s_1| \leq
\delta$ и $J(s_1) = q_d$. Тогда по следствию к лемме~3.7
$q_d < 1$. Тем самым доказано, что $J(s) \leq q_d < 1$ для всех $s
\geq A_1$.

Заметим, что в выражении (\ref{l22}) $|f_{d_{\theta}}(s)|$ можно заменить на~$|f_{d_n}(s)|$. 
При этом останется справедливой оценка
$$
|f_{d_n}(s)| \leq q_d < 1 \mbox{ для всех } |s| \geq A_1\,.
$$
Используя~(\ref{l17}), перейдём к функции~$f_{D_n}(s)$. Тогда можно
утверждать, что
$$
|f_{D_n}(s)| = |f_{d_n}(tn^{-1/2}s)| \leq q_d < 1
$$
для всех $|s| \geq \sigma_1 \sqrt{n}$  (где $\sigma_1 = A_1/t$). Лемма доказана.
\hfill $\Box$

\medskip

\noindent
\textbf{Лемма 3.9.}
\textit{Для модуля характеристической функции $f_{M_n}(s)$ справедлива следующая оценка:}
\begin{align*}
 |f_{M_n}(s)| &\leq \exp\left\{-{\epsilon_2 s^2}/n\right\}\,,  & |s| &\leq\sigma_2 \sqrt{n}\,; \\ 
|f_{M_n}(s)| &\leq q_h < 1\,,  & |s|&\geq\sigma_2 \sqrt{n}\,, 
\end{align*}
\textit{где $\sigma_2$, $\epsilon_2$~--- положительные константы; $q_h$~--- 
положительная константа, определенная с точностью до неизвестных~$a,b,t$.
}

\medskip

\noindent
Д\,о\,к\,а\,з\,а\,т\,е\,л\,ь\,с\,т\,в\,о\,.\  
Доказательство первого утверждения леммы аналогично
доказательству первого утверждения леммы~3.8. Заметим, что
в ходе доказательства вводятся некоторые положительные константы~$A_2$ и~$B_2$, 
которые идентичны по смыслу (но не по значению)
константам~$A_1$ и~$B_1$ из леммы~3.8. Перейдем к
доказательству второго утверждения.

Для любого $\theta > 0$ непосредственным интегрированием получаем
\begin{multline*}
f_{m_{\theta}}(s) = \fr{1}{2\;{\sf erfc}\left(b/(2\sqrt{a})\right)}\times{}\\
{}\times
\left[\left(f_1 + f_2\right)e^{-({s(s + i\theta)(b + a\theta)^2})/({a\theta^2})}\;+ \right.\\
\left.{}+\left(f_3 + f_4\right)e^{-as(s+ i\theta)}\right]\,,
\end{multline*}
где
\begin{align*}
f_1 &= \erx{\fr{b}{{2\sqrt{a}}}+ \sqrt{a}\theta - is\left(\fr{b}{\sqrt{a}\theta} + \sqrt{a}\right)}\,;
\\
f_2 &= \erx{-\fr{b}{{2\sqrt{a}}} + is\left(\fr{b}{\sqrt{a}\theta} + \sqrt{a}\right)}\,;\\
f_3 &= \ercx{\fr{b}{2\sqrt{a}} + \sqrt{a}\theta - is\sqrt{a}}\,;\\
f_4 &= \ercx{\fr{b}{2\sqrt{a}} + is\sqrt{a}}\,.
\end{align*}
Преобразуем сумму $f_1 + f_2$ следующим образом:
\begin{multline*}
f_1 + f_2 ={}\\
{}= \fr{2}{\sqrt{\pi}} \int\limits_0^{{b}/({2\sqrt{a}}) + 
\sqrt{a}\theta - is\left({b}/({\sqrt{a}\theta}) + \sqrt{a}\right)} e^{-z^2}\,dz +{}\\
{}+
\fr{2}{\sqrt{\pi}}\int\limits_0^{-b/(2\sqrt{a}) + is\left({b}/({\sqrt{a}\theta}) + 
\sqrt{a}\right)} e^{-z^2}\,dz ={}\\
{}
= \fr{2}{\sqrt{\pi}} \int\limits_{{b}/({2\sqrt{a}}) - is\left({b}/({\sqrt{a}\theta}) + 
\sqrt{a}\right)}^{{b}/({2\sqrt{a}}) + \sqrt{a}\theta - is\left({b}/({\sqrt{a}\theta}) + \sqrt{a}\right)} 
e^{-z^2}\,dz ={}\\
{}= \fr{2}{\sqrt{\pi}} \int\limits_{{b}/({2\sqrt{a}})}^{{b}/({2\sqrt{a}}) + 
\sqrt{a}\theta} e^{-\left(w - is\left({b}/({\sqrt{a}\theta}) + \sqrt{a}\right)\right)^2}\,dw ={}\\
{}
= \fr{2}{\sqrt{\pi}} e^{({s^2(a\theta + b)^2})/({a\theta^2})} \times{}\\
{}\times \int\limits_{{b}/({2\sqrt{a}})}^{{b}/({2\sqrt{a}}) + \sqrt{a}\theta}
e^{-w^2} e^{2is\left(\sqrt{a}+ {b}/({\sqrt{a}\theta})\right)w}\, dw\,.
\end{multline*}
Преобразуем сумму $f_3 + f_4$ следующим образом:

\noindent
\begin{multline*}
f_3 + f_4 = \fr{2}{\sqrt{\pi}}\int\limits_{{b}/({2\sqrt{a}}) + \sqrt{a}\theta - is\sqrt{a}}^{\infty}
e^{-z^2}\,dz +{}\\
{}+ \fr{2}{\sqrt{\pi}}\int\limits_{{b}/({2\sqrt{a}}) + is\sqrt{a}}^{\infty} e^{-z^2}\,dz ={}\\
{}=
 \fr{4e^{as^2}}{\sqrt{\pi}}\int\limits_{{b}/({2\sqrt{a}}) + \sqrt{a}\theta}^{\infty} 
 e^{-w^2} \cos 2s\sqrt{a} w\,dw +{}\\
 {}+
  \fr{2e^{as^2}}{\sqrt{\pi}}\int\limits_{{b}/({2\sqrt{a}})}^{{b}/({2\sqrt{a}}) + \sqrt{a}\theta} e^{-w^2} 
  \left(\cos 2s\sqrt{a} w -{}\right.\\
  {}\left. - i\sin 2s\sqrt{a} w\right)\,dw\,.
\end{multline*}
Подставляя эти выражения в формулу для~$f_{m_{\theta}}(s)$ и
пользуясь тем, что $\left|\int f(z)dz\right| \leq \int |f(z)|\,dz$,
$|a + b| \leq$\linebreak $\leq |a| + |b|$, получаем
\begin{multline*}
\!|f_{m_{\theta}}(s)| \leq \fr{2}{\sqrt{\pi}\;{\sf erfc}\left(b/(2\sqrt{a})\right)} 
\int\limits_{{b}/({2\sqrt{a}})}^{{b}/({2\sqrt{a}}) + 
\sqrt{a}\theta}\!\!\!\!\! e^{-w^2}\,dw +{}\\
{}+ \fr{2}{\sqrt{\pi}\;{\sf erfc}\left(b/(2\sqrt{a})\right)}\times{}\\
{}\times \left|\,\int\limits_{{b}/({2\sqrt{a}}) + \sqrt{a}\theta}^{\infty} e^{-w^2} \cos 2s\sqrt{a}\, 
w\,dw\right| \leq{}\\
{}
\leq J(s) \equiv \fr{2}{\sqrt{\pi}\;{\sf erfc}\left(b/(2\sqrt{a})\right)} \int\limits_{{b}/({2\sqrt{a}})}^{{b}/({2\sqrt{a}}) + 
\sqrt{a}t} \!\!\!\!\!\!e^{-w^2}\,dw +{}\\
{}+ \fr{2}{\sqrt{\pi}\;{\sf erfc}\left(b/(2\sqrt{a})\right)}
\left|\,\int\limits_{{b}/({2\sqrt{a}}) + \sqrt{a}t}^{\infty}\!\!\!\!\!\!\!\!\!\!\! e^{-w^2} \cos 2s\sqrt{a} w\,dw\right|.
\end{multline*}
Аналогично предыдущей лемме можно показать, что
$$
\lim\limits_{|s|\to\infty} J(s) = J_1 \equiv 
\fr{({2}/{\sqrt{\pi}})\il{{b}/({2\sqrt{a}})}{{b}/({2\sqrt{a}})+\sqrt{a}t}
\!\!\!\!\!e^{-w^2}\,dw}{{\sf erfc}\left(b/(2\sqrt{a})\right)}\,.
$$
При этом, так как~$t$ ограничено, то $J_1<1$. Очевидно также, что $J(s)>0$ для всех~$s$. 
Так как функция~$J(s)$ четная, то в дальнейшем будем рассматривать только область $s > 0$.

Положим
$$
q_h = \sup_{s \geq A_2} J(s)\,,
$$
где $A_2$~--- положительная константа, определенная при доказательстве первого утверждения леммы. 
Аналогично лемме~3.8 можно показать, что $q_h < 1$.

Заметим, что в оценке модуля характеристической функции очевидна возможность перехода 
от $f_{m_{\theta}}(s)$ к~$f_{m_n}(s)$ и
$$
|f_{m_n}(s)| \leq q_h < 1 \mbox { для всех } |s| \geq A_2\,.
$$
Используя~(\ref{l16}), перейдем к функции~$f_{M_n}(s)$. Тогда можно утверждать, что
\begin{multline*}
|f_{M_n}(s)| = |f_{m_n}(tn^{-1/2}s)| \leq q_h < 1 \\
\mbox{ для всех } |s| \geq \sigma_2\sqrt{n}\,,
\end{multline*}
где $\sigma_2 = A_2/t$. Лемма доказана. \hfill $\Box$

Перейдем к доказательству основных результатов.

\section{Проверка условий теоремы~3.2.1}

В этом разделе проверим условия теоремы~3.2.1 из~\cite{bening} и тем самым покажем 
справедливость установленной на эвристическом уровне формулы~(\ref{l5}).

\subsection{Условие~1}

Положим $\tau_n=n^{-1/4}$. Покажем, что условие~1 выполняется с
\begin{equation*}
\Phi_1(x) = \Phi\left(\fr{x}{t\sqrt{I_{a,b}}} + \fr{\sqrt{I_{a,b}}t}{2}\right)\,,\enskip  \Phi_2(x) = 0\,.
%\label{l28}
\end{equation*}
%$$
%\Phi_2(x) = 0\,.
%$$
При этом
$$
p(x) = \Phi_1'(x) = \fr{1}{t\sqrt{I_{a,b}}} \, \varphi\left(\fr{x}{t\sqrt{I_{a,b}}} + \fr{\sqrt{I_{a,b}}t}{2}\right)\,.
$$


\noindent
\textbf{Пункт (i).} Так как $\tau_n = n^{-1/4}$, выполнение условия~1 с 
такими~$\Phi_1(x)$ и~$\Phi_2(x)$ следует автоматически из леммы~3.1.

\medskip

\noindent
\textbf{Пункт (ii).} Выберем~$\beta$ произвольным образом, но так, чтобы $0 < \beta < 2$. 
Фиксируем произвольное $x_0 \in {\r}$. Заметим, что для любого множества $A \in {\cal B}({\r}^n)$ 
справедливо следующее представление:
\begin{equation}
{\p}_{n,\,1}(A) = {\e}_{n,\,0} \exp\{\Lambda_n\} {\III}(A)\,,
\label{l23}
\end{equation}
где ${\cal B}({\r}^n)$~--- борелевская $\sigma$-алгебра множеств в~${\r}^n$.

Обозначим через~$x_1$ такое $x \leq x_0$, на котором функция ${\p}_{n,\,0} \left(x \leq \Lambda_n \leq x + 
\tau_n^{2 + \beta}\right)$ достигает своего супремума. Очевидно, что
\begin{equation}
\sup\limits_{x\in{\sf X}} f(x)g(x) \leq \sup\limits_{x\in{\sf X}}f(x) \sup\limits_{x\in{\sf X}}g(x)\,,
\label{l24}
\end{equation}
где $f(x), g(x) \geq 0$~---- произвольные функции, а ${\sf X}$~--- некоторое множество. 
Тогда для любого $x \leq x_0$ в силу представления~(\ref{l23}), замечания~(\ref{l24}) и 
непрерывности распределения случайной величины~$\Lambda_n$ при любой гипотезе по лемме~3.1 имеем
\begin{multline*}
\tau_n^{-2}{\p}_{n,\,1}\left\{x \leq \Lambda_n \leq x + \tau_n^{2 + \beta}\right\} ={}\\
{}
= \tau_n^{-2}{\e}_{n,\,0} \exp\left\{\Lambda_n\right\} {\III}\left\{x \leq \Lambda_n \leq x + 
\tau_n^{2+\beta}\right\} \leq{}\\
{}
\leq\tau_n^{-2}\exp\left\{x + \tau_n^{2 + \beta}\right\}\times{}\\
{}\times {\e}_{n,\,0} {\III}\left\{x \leq \Lambda_n \leq x + 
\tau_n^{2+\beta}\right\} \leq{}\\
{}
\leq \tau_n^{-2}\exp\left\{x_0 + \tau_n^{2 + \beta}\right\}\times{}\\
{}\times {\p}_{n,\,0} \left\{x_1 \leq \Lambda_n \leq x_1 + 
\tau_n^{2 + \beta}\right\} ={}\\
{}
= \tau_n^{\beta}  \exp\left\{x_0 + \tau_n^{2 + \beta}\right\}\times{}\\
{}\times \fr{{\p}_{n,0}(\Lambda_n < x_1 + \tau_n^{2 + 
\beta}) - {\p}_{n,0}(\Lambda_n < x_1)}{\tau_n^{2 + \beta}} ={}\\
{}
= \tau_n^{\beta}  \exp\left\{x_0 + \tau_n^{2 + \beta}\right\} \times{}\\
{}
\times \left\{\left[{\Phi}\left(\fr{x_1 + \tau_n^{2 + \beta}}{t\sqrt{I_{a,b}}} +
 \fr{\sqrt{I_{a,b}}t}{2}\right) -{}\right.\right.\\
 {}-\left. {\Phi}\left(\fr{x_1}{t\sqrt{I_{a,b}}} + 
 \fr{\sqrt{I_{a,b}}t}{2}\right)\right]\tau_n^{-2 - \beta} + {}\\
\!{}
+ \fr{b^2C_{a,b}t}{3\sqrt{I_{a,b}}}\left(\fr{x_1}{tI_{a,b}} - \fr{t}{2}\right)\left[ 
\varphi\left(\fr{x_1 + \tau_n^{2 + \beta}}{t\sqrt{I_{a,b}}} + \fr{\sqrt{I_{a,b}}t}{2}\right) - {}\right.\\
\left.{}-\varphi\left(\fr{x_1}{t\sqrt{I_{a,b}}} + \fr{\sqrt{I_{a,b}}t}{2}\right)\right]\tau_n^{-\beta} +{}\\
\left.{}
+  o(1)
\vphantom{\fr{\sqrt{I_a}}{\sqrt{I_a}}}
\right\} = \tau_n^{\beta}  \exp\left\{x_0 + \tau_n^{2 + \beta}\right\} \times{}\\
{}\times
\left\{\fr{1}{t\sqrt{I_{a,b}}}\varphi\left(\fr{x_1}{t\sqrt{I_{a,b}}} + 
\fr{\sqrt{I_{a,b}}t}{2}\right) +  o(1)\right\} \to 0\,.
\end{multline*}
Итак, условие выполнено.

\subsection{Условие 2}

В силу непрерывности распределения случайной величины~$\Lambda_n$ при обеих гипотезах в качестве~$D_{n,i}$ 
можно выбрать~${\r}^n$. Тогда все случайные величины <<с волной>>  
тождественно равны соответствующим случайным величинам <<без волны>> и фигурирующие в записи~$S_n$ последовательности~$a_n$ 
и~$b_n$ равны тождественно~$t$ и $-\iab t^2/2$ соответственно.

Положим $\gamma_n = n^{-\nu}$,  $0<\nu<1/8$.

\medskip

\noindent\textbf{Пункт (i).} Очевиден при выбранных~$D_{n,i}$.

\medskip

\noindent\textbf{Пункт (ii).} Для индикатора ${\III}_{(\gamma_n,\infty)}(|\Delta_n|)$ справедлива следующая оценка:
\begin{equation}
{\III}_{(\gamma_n,\infty)}(|\Delta_n|) \leq \left(\fr{|\Delta_n|}{\gamma_n}\right)^m\,,
\label{l25}
\end{equation}
где $m$~--- произвольное неотрицательное число. Положим $m = 1$. Тогда, используя лемму~3.3, имеем
\begin{multline*}
\tau_n^{-2} \gamma_n^{-1}  {\e}_{n,0} \Delta_n^2 {\III}_{(\gamma_n,\infty)}(|\Delta_n|) \leq{}\\
{}\leq \tau_n^{-2} \gamma_n^{-2}  {\e}_{n,0} |\Delta_n|^3 \leq{}\\
{}
\leq C_1 n^{2\nu+1/2-3/4} = C_1 n^{2\nu-1/4} \to 0\\
 \mbox{ при } n \to \infty \mbox{ и } 0<\nu<\fr{1}{8}\,,
\end{multline*}
где $C_1$~--- положительная константа.

\medskip

\noindent\textbf{Пункт (iii).} Полагая в~(\ref{l25}) $m = 2$, по лемме~3.3 имеем
\begin{multline*}
\tau_n^{-2}  {\e}_{n,0} \exp\{\Lambda_n\} |\Delta_n| {\III}_{(\gamma_n,\infty)}(|\Delta_n|) \leq{}\\[2pt]
{}\leq \tau_n^{-2} \gamma_n^{-2}  {\e}_{n,0} \exp\{\Lambda_n\} |\Delta_n|^3 \leq{}\\[2pt]
{}
\leq \tau_n^{-2}  \gamma_n^{-2} \sqrt{{\e}_{n,0} \exp\{2\Lambda_n\}  {\e}_{n,0} |\Delta_n|^6} \leq{}\\[2pt]
{}\leq n^{2\nu+1/2} \sqrt{C_2n^{-3/2}} ={}\\[2pt]
{}
= \sqrt{C_2}\,  n^{2\nu+1/2}  n^{-3/4} = \sqrt{C_2} \, n^{2\nu-1/4} \to 0\\[2pt]
\mbox{ при } n \to \infty \mbox{ и } 0<\nu<\fr{1}{8}\,,
\end{multline*}
где $C_2$~--- положительная константа. Здесь использовано легко проверяемое соотношение
$$
{\e}_{n,0} \exp\left\{2\Lambda_n\right\} = {\cal O}(1)\,.
$$

\subsection {Условие~3}

Выберем последовательности $\Psi(n)$ и $\bar{\Psi}(n)$ та\-кими:
\begin{align}
\Psi(n) &= n^{\nu/4}\,; \label{l26}\\
\bar{\Psi}(n) &= n^{\beta+2}\,, \label{l27}
\end{align}
где $\nu$~--- константа, фигурирующая в показателе степени последовательности~$\gamma_n$ условия~2, 
а $\beta$~--- из условия~1. Такие последовательности будут удовлетворять всем требованиям, обозначенным 
в формулировке условия~3. По результатам работы~\cite{article} положим
\begin{align*}
\Pi &\sim \n \left(0,\, \fr{4b^2 \cab t^3}{3}\right)\,;\\
\Lambda &\sim \n \left(-\fr{t^2\iab}{2},\, \fr{t^2\iab}{2}\right)\,.
\end{align*}
В работе~\cite{article} также было показано, что случайные величины~$\Pi$ и~$\Lambda$ независимы.

\medskip

\noindent
\textbf{Пункт (i).} Преобразуем~$q_{n,l}$ следующим образом:
$$
q_{n,l} = \bar{q}_{n,l}(s) + R_l(s)\,, \enskip l = 0, 1\,,
$$
где

\noindent
\begin{align*}
\bar{q}_{n,l}(s) &\equiv -\fr{1}{l+1}\,{\e}_{n,0}\exp\left\{is\Lambda_n\right\}(\tau_n^{-1} \Delta_n)^{l+1}\,;
\\
R_l(s) &\equiv {\e}_{n,0}\exp\left\{is\Lambda_n\right\}\int\limits_{\tau_n^{-1} \Delta_n}^0z^l(e^{is\tau_n z}-1)\,dz\,.
\end{align*}
Для $R_l(s)$ справедлива следующая оценка:
$$
|R_l(s)| \leq |s|\tau_n {\e}_{n,0}\left |\tau_n^{-1}\Delta_n\right|^{l+2}\,.
$$
При $l=0$ по лемме~3.3 имеем
$$
|R_0(s)| \leq |s|\tau_n^{-1}{\e}_{n,0}|\Delta_n|^2 \leq C_3 |s| n^{-1/4}\,,
$$
где $C_3$~--- положительная константа. При $l=1$ по лемме~3.3 имеем
$$
|R_1(s)| \leq |s| \tau_n^{-2} {\e}_{n,0}|\Delta_n|^3 \leq C_4 |s| n^{-1/4}\,,
$$
где $C_4$~--- положительная константа.

Так как $\Psi^2(n) n^{-1/4} \to 0$, то
$$
\int\limits_{|s|\leq\Psi(n)}|R_l(s)|ds \to 0\,,\enskip l=0,1\,.
$$
Тогда для доказательства справедливости утверждений пункта~(i) достаточно будет установить, что
$$
\left|\int_{|s|\leq\Psi(n)}(\bar{q}_{n,l}(s) - q_l(s))ds\right| \to 0\,,  l=0,1\,.
$$
Для этого по теореме о мажорируемой сходимости достаточно показать поточечную 
сходимость~$\bar{q}_{n,l}(s)$ к~$q_l(s)$, а также существование интегрируемой 
на $|s|\leq \Psi(n)$ функции~$r_l(s)$ такой, что $\bar{q}_{n,l}(s)\leq r_l(s)$.

\medskip

\noindent
\textbf{Случай {\boldmath{$l=0$}}.} По лемме~3.4 можно записать
\begin{multline*}
{\e}_{n,0}\exp\{is\Lambda_n\}(\tau_n^{-1}\Delta_n) ={}\\
{}= -b n^{5/4} f^{n-1}_{M_n}(s) {\e}_0 \exp\left\{is M_n(X_1)\right\} g_n(X_1) ={}\\
{}
= -b n^{5/4} f^{n-1}_{M_n}(s) {\e}_0 (1+\delta_1(s,X_1)) g_n(X_1)\,,
\end{multline*}
где
$$
|\delta_1(s,X_1)| \leq |s| |M_n(X_1)|\,.
$$
По лемме~3.2 для $|s| \leq \Psi(n)$ и при достаточно больших~$n$
\begin{equation*}
\left|f_{M_n}^{n-1}(s)\right| \leq \exp\left\{-\fr{(n-1)\epsilon_2 s^2}{n}\right\} \leq
\exp\left\{-\frac{\epsilon_2 s^2}{2}\right\}\,.
\end{equation*}
Тогда по леммам~3.2 и~3.5
%\noindent
\begin{multline*}
\left|{\e}_{n,0}\exp\{is\Lambda_n\}(\tau_n^{-1}\Delta_n)\right| \leq{}\\
{}
\leq b n^{5/4}\exp\left\{-\fr{\epsilon_2 s^2}{2}\right\}  \left(
\left|{\e}_0g_n(X_1)\right| +{}\right.
\end{multline*}

\noindent
\begin{multline*}
\left.{}+ |s|{\e}_0 |M_n(X_1) g_n(X_1)|\right) \leq{}\\
{}
\leq b n^{5/4} \exp\left\{-\fr{\epsilon_2 s^2}{2}\right\} \left(C_5n^{-3/2} + C_6|s|n^{-5/4}\right) ={}\\
{}
= r_0(s) \equiv b \exp\left\{-\fr{\epsilon_2 s^2}{2}\right\} \left(C_5n^{-1/4} + C_6|s|\right)\,,
\end{multline*}
где $C_5$, $C_6$~--- положительные константы. Так как функция~$r_0(s)$ интегрируема на $|s|\leq\Psi(n)$, 
то требование ограниченности $\bar{q}_{n,0}(s)$ интегрируемой функцией выполнено. 
Покажем поточечную сходимость $\bar{q}_{n,0}(s)$ к~$q_0(s)$, т.\,е.
$$
{\e}_{n,0} \exp\left\{is\Lambda_n\right\} \left(\tau_n^{-1} \Delta_n\right) \to 
{\e} \exp\left\{is\Lambda\right\} \Pi\,.
$$
Имеем
$$
{\e} \exp\left\{is\Lambda\right\} \Pi = {\e} \exp\left\{is\Lambda\right\} {\e} \Pi = 0\,,
$$
поэтому достаточно показать, что для любого $s\in{\r}$
$$
\tau_n^{-1} {\e}_{n,0} \exp\left\{is\Lambda_n\right\} \Delta_n \to 0\,.
$$
По леммам~3.4 и~3.6 для любого $s\in{\r}$ имеем
\begin{multline*}
{\e}_{n,0}\exp\{is\Lambda_n\}(\tau_n^{-1}\Delta_n) ={}\\
{}= -b n^{5/4} f^{n-1}_{M_n}(s) {\e}_0 \exp\left\{isM_n(X_1)\right\} g_n(X_1) ={}\\
{}
= -b n^{5/4} \left(1+\fr{is(is-1)\iab t^2}{2n}+ o(n^{-1})\right)^{n-1} \times{}\\
{}\times
\left(\fr{b \cab (1+is)t^3}{3n^{3/2}}+  o(n^{-3/2})\right) \to 0\,.
\end{multline*}

\medskip

\noindent
\textbf{Случай {\boldmath{$l=1$}}.} По лемме~3.4 запишем
\begin{multline*}
{\e}_{n,0}\exp\{is\Lambda_n\}(\tau_n^{-1}\Delta_n)^2 = b^2 n^{3/2}\:f_{M_n}^{n-2}(s) \times{}\\
{}
\times \left(f_{M_n}(s) {\e}_0 \exp\left\{isM_n(X_1)\right\} g_n^2(X_1) + {}\right.{}\\
\left.{}+
(n-1) \left[{\e}_0 (1+\delta_2(s,X_1)) g_n(X_1)\right]^2\right)\,,
\end{multline*}
где
$$
|\delta_2(s,X_1)| \leq |s| |M_n(X_1)|\,.
$$
По лемме~3.9 для $|s| \leq \Psi(n)$ и при достаточно больших~$n$
$$
\left|f_{M_n}^{n-2}(s)\right| \leq \exp\left\{
-\fr{(n-2)\epsilon_2 s^2}{n}\right\} 
\leq \exp\left\{-\frac{\epsilon_2 s^2}{3}\right\}\,.
$$
Тогда по леммам~3.2 и~3.5
\begin{multline*}
\left|{\e}_{n,0}\exp\{is\Lambda_n\}(\tau_n^{-1}\Delta_n)^2\right| \leq{}\\
{}\leq b^2 n^{3/2} 
\exp\left\{-\fr{\epsilon_2 s^2}{3}\right\} \times{}\hspace*{10mm}
\end{multline*}
\begin{multline*}
{}\times \left({\e}_0 g_n^2(X_1) + n\left|{\e}_0g_n(X_1)\right|^2 +{}\right.\\[6pt]
{}
+ 2 n |s|  |{\e}_0 g_n(X_1)|  {\e}_0 |M_n(X_1) g_n(X_1)| +{}\\[6pt]
\left.{}+ns^2 \left({\e}_0 |M_n(X_1) g_n(X_1)|\right)^2\right) \leq
{}\\
{}
\leq b^2 n^{3/2}\exp\left\{-\fr{\epsilon_2 s^2}{3}\right\} \times{}\\[6pt]
{}\times
\left(C_7n^{-3/2} + C_8n^{-2} + C_9|s|n^{-7/4} + C_{10}s^2n^{-3/2}\right) ={}
\\
= r_1(s) \equiv b^2 \exp\left\{-\fr{\epsilon_2 s^2}{3}\right\} \times{}\\[6pt]
{}\times
\left(C_7 + C_8n^{-1/2} + C_9|s|n^{-1/4} + C_{10}s^2\right)\,,
\end{multline*}
где $C_7$, $C_8$, $C_9$, $C_{10}$~--- положительные константы. Так как~$r_1(s)$ 
интегрируема на $|s| \leq \Psi(n)$, то требование
ограниченности $\bar{q}_{n,1}(s)$ интегрируемой функцией выполнено.
Покажем поточечную сходимость $\bar{q}_{n,1}(s)$ к $q_1(s)$, т.\,е.
$$
{\e}_{n,0} \exp\left\{is\Lambda_n\right\} (\tau_n^{-1} \Delta_n)^2 \to {\e} \exp\left\{is\Lambda\right\} \Pi^2\,.
$$
Имеем
\begin{multline*}
{\e} \exp\left\{is\Lambda\right\} \Pi^2 = {\e} \exp\left\{is\Lambda\right\} {\e} \Pi^2 ={}\\[6pt]
{}= \fr{4b^2\cab t^3}{3} \exp\left\{-\fr{s^2t^2\iab}{2}-\fr{ist^2\iab}{2}\right\}\,.
\end{multline*}
По леммам~3.4 и~3.6 для любого $s\in{\r}$ получаем
\begin{multline*}
{\e}_{n,0}\exp\{is\Lambda_n\}(\tau_n^{-1}\Delta_n)^2 = b^2 n^{3/2}  f_{M_n}^{n-2}(s) \times{}\\[6pt]
{}\times
 \left(
 \vphantom{\left[X_1\right]^2}
 f_{M_n}(s) {\e}_0 \exp\left\{isM_n(X_1)\right\} g_n^2(X_1) +{}\right.\\[6pt]
\left. {}+ (n-1) \left[
 {\e}_0 \exp\left\{isM_n(X_1)\right\} g_n(X_1)\right]^2\right) ={}\\[6pt]
{}= b^2 n^{3/2} \left(1+\fr{is(is-1)\iab t^2}{2n}+  o(n^{-1})\right)^{n-2} \times{}\\[6pt]
{}
\times \left[
\vphantom{\left(\fr{b^2\cab^2 (1+is)^2t^6}{9n^3} +  o(n^{-3})\right)}
\left(1+\fr{is(is-1)\iab t^2}{2n}+  o(n^{-1})\right)\right.\times{}\\[6pt]
{}\times\left.
\left(\fr{4\cab t^3}{3n^{3/2}}+  o (n^{-3/2})\right) +\right.{}\\[6pt]
{}
\left.+ (n-1)\left(\fr{b^2\cab^2 (1+is)^2t^6}{9n^3} +  o(n^{-3})\right)\right] \to{}\\[6pt]
{}
\to \exp\left\{-\fr{s^2t^2\iab}{2}-\fr{ist^2\iab}{2}\right\} \times{}
\end{multline*}
\begin{multline*}
{}\times
\left[1 \times \fr{4b^2\cab t^3}{3} + 0\right] ={}\\[6pt]
{}
= \fr{4b^2\cab t^3}{3} \exp\left\{-\fr{s^2t^2\iab}{2}-\fr{ist^2\iab}{2}\right\}\,.
\end{multline*}

\medskip

\noindent\textbf{Пункт (ii).} Используя~(\ref{l26}) и~(\ref{l27}), получим
$$
\tilde{\Psi}(n) = \log \bar{\Psi}(n) \Psi^{-1}(n) = \left(\beta+2-\fr{\nu}{4}\right) \log n\,.
$$

\smallskip

\noindent\textbf{Третий предел.} В соответствии с~(\ref{l14}) можно записать
$$
f_{S_n}(s) = (f_{D_n}(s))^n\,.
$$
Очевидно, что $\tau_n^{-1} > \tilde{\Psi}(n)$ при больших~$n$. Тогда по лемме~3.8
\begin{multline*}
\!\!\tau_n^{-1} \max(\tilde{\Psi}(n),\tau_n^{-1})\! \sup_{\Psi(n)\leq|s|\leq\bar{\Psi}(n)}\!\!|{\e}_{n,0}\exp\{isS_n\}| ={}\\[6pt]
{}
= \sqrt{n}  \max \left\{\left[
\sup_{n^{\nu/4}\leq|s|\leq\sigma_1\sqrt{n}} |f_{D_n}(s)|\right]^n,\,\right.\\[6pt]
\left. 
\left[\sup_{\sigma_1\sqrt{n}\leq|s|\leq n^{\beta+2}} |f_{D_n}(s)|\right]^n\right\} ={}\\[6pt]
\!\!{}
= \sqrt{n}  \max\left\{\exp\left\{-\epsilon_1 n^{\nu/2}\right\},\, q_d^n\right\} \to 0 \mbox{ при } n \to \infty.
\end{multline*}

\smallskip

\noindent\textbf{Первый предел.} В~соответствии с леммой~3.4 и неравенством Коши--Буняковского имеет место следующая оценка:
\begin{multline*}
\!\!\!\left|{\e}_{n,0} \exp\{isS_n\} \Delta_n\right| \leq bn\left|f_{D_n}(s)\right|^{n-1} {\e}_0\left|g_n(X_1)\right| \leq{}\\[6pt]
{}\leq C_{11} \sqrt[4]{n} \left|f_{D_n}(s)\right|^{n-1}\,,
\end{multline*}
где $C_{11}$~--- положительная константа. По лемме~3.8 для всех $|s|\leq \sigma_2\sqrt{n}$ при $n\geq2$
$$
|f_{D_n}(s)|^{n-1} \leq \exp\left\{\!-\fr{(n-1)\epsilon_1 s^2}{n}\!\right\} \leq 
\exp\left\{-\fr{\epsilon_1 s^2}{2}\right\}.
$$
Тогда аналогично случаю третьего предела
\begin{multline*}
\tau_n^{-2} \tilde{\Psi}(n) \sup_{\Psi(n)\leq|s|\leq\bar{\Psi}(n)}|{\e}_{n,0}\exp\{isS_n\}\Delta_n| \leq{}\\
{}
\leq C_{12} n^{3/4} \log n \max\left\{\exp\left\{-\fr{\epsilon_1 n^{\nu/2}}{2}\right\},\, q_d^{n-1}\right\} \to{}\\
{}\to  0 
\mbox{ при } n \to \infty\,,
\end{multline*}
где $C_{12}$~--- некоторая положительная константа.

\smallskip

\noindent\textbf{Второй предел.} В соответствии с~(\ref{l13}) можно записать
$$
f_{\Lambda_n}(s) = (f_{M_n}(s))^n\,.
$$
Тогда по лемме~3.9, повторяя схему доказательства для первого предела, имеем
\begin{multline*}
\tau_n^{-1} \max\left(\tilde{\Psi}(n),(\Psi^{-1}(n)-\bar{\Psi}^{-1}(n))\tau_n^{-1}\right) \times{}\\
{}\times
\sup_{\Psi(n)\leq|s|\leq\bar{\Psi}(n)}|{\e}_{n,0}\exp\{is\Lambda_n\}| ={}\\
{}
= \sqrt{n}\left(\fr{1}{n^{\nu/4}}-\fr{1}{n^{\beta+2}}\right) \times{}\\
{}\times
\max\left\{\exp\left\{-\epsilon_2 n^{\nu/2}\right\},\, q_h^n\right\} \to 0 \mbox{ при } n \to \infty\,.
\end{multline*}
Доказательство равенства этого предела нулю завершает проверку условий теоремы.

\section{Заключение}

Теперь можно применить теорему~3.2.1 из работы~\cite{bening}, а значит,
$$
\lim\limits_{n \to \infty} \sqrt{n} \left(\beta_n^{*}-\beta_n\right) = \fr{1}{2}\, 
e^{\tilde{u}_{\alpha}} {\sf D}\left(\Pi | \Lambda = \tilde{u}_{\alpha}\right) p\left(\tilde{u}_{\alpha}\right)\,,
$$
где $\tilde{u}_{\alpha} = \Phi_1^{-1}(1-\alpha)$. Из условия~1 (см.\ предыдущий раздел)
$$
\tilde{u}_{\alpha} = t\sqrt{\iab}u_{\alpha} - \fr{\iab t^2}{2}\,,
$$
где $u_{\alpha}=\Phi^{-1}(1-\alpha)$. Тогда
$$
p(\tilde{u}_{\alpha}) = \Phi_1'(x)\biggr|_{x=\tilde{u}_{\alpha}} = \fr{1}{\sqrt{\iab}t} \, \varphi({u_{\alpha}})\,.
$$
Так как случайные величины~$\Pi$ и~$\Lambda$ независимы, то 
${\sf D}(\Pi|\Lambda=\tilde{u}_{\alpha}) = 4b^2\cab t^3/3$. 
После очевидных преобразований окончательно получаем, что
$$
\fr{1}{2} \, e^{\tilde{u}_{\alpha}} {\sf D}\left(\Pi | \Lambda \!=\! 
\tilde{u}_{\alpha}\right) p(\tilde{u}_{\alpha}) = \fr{2b^2\cab t^2}{3\sqrt{\iab}} \, \varphi(u_{\alpha}-t\sqrt{\iab})\,.
$$
Таким образом, получено формальное доказательство формулы~(\ref{l5}).


{\small\frenchspacing
{%\baselineskip=10.8pt
\addcontentsline{toc}{section}{Литература}
\begin{thebibliography}{9}

\bibitem{article} %1
\Au{Бенинг В.\,Е., Лямин О.\,О.} 
О мощности критериев в случае обобщенного распределения Лапласа~// Информатика и её применения, 2009. Т.~3. Вып.~3. 
С.~79--85.

\bibitem{kotz} %2
\Au{Kotz S., Kozubowski~T.\,J., Podgorski~K.} 
The Laplace distribution and generalizations: A revisit with applications to communications, economics, engineering and 
finance.~--- Boston: Birkhauser, 2001.

\bibitem{korolev}  %3
\Au{Бенинг В.\,Е., Королёв~В.\,Ю.} 
Некоторые статистические задачи, связанные с распределением Лапласа~// Информатика и её применения, 2008. Т.~2. Вып.~2. 
С.~19--34.

\bibitem{korolev_ra}  %4
\Au{Бенинг В.\,Е., Королёв~Р.\,А.} О предельном поведении мощностей критериев в случае распределения Лапласа~// 
Информатика и её применения, 2010. Т.~4. Вып.~2. С.~66--77.

\bibitem{bening} %5
\Au{Bening V.\,E.} Asymptotic theory of testing statistical hypotheses.~--- Utrecht: VSP, 2000.

\bibitem{chibisov} %6
\Au{Chibisov D.\,M., van Zwet~W.\,R.} On the Edgeworth expansion for the logarithm of the likelihood ratio. I~// 
Теория вероятностей и ее применения, 1984. Т.~29. №\,3. C.~417--439.



\bibitem{petrov} %7
\Au{Petrov V.\,V.} Sums of independent random variables.~--- Berlin: Springer-Verlag, 1975.

\label{end\stat}

\bibitem{matan} %8
\Au{Ильин В.\,А., Садовничий В.\,А., Сендов~Бл.\,X.} Математический анализ: Продолжение курса.~--- М.: МГУ,\linebreak 1987.
 \end{thebibliography}
 }}\end{multicols} %8

\def\stat{chuprakov}

\def\tit{К ВОПРОСУ О РАЗМЕЩЕНИИ КОЛЛЕКТИВНЫХ СРЕДСТВ 
ОТОБРАЖЕНИЯ В СИТУАЦИОННОМ ЗАЛЕ С~ЗАДАННЫМИ 
ПАРАМЕТРАМИ}

\def\titkol{К вопросу о размещении коллективных средств 
отображения в ситуационном зале с~заданными 
параметрами}

\def\autkol{К.\,Г.~Чупраков}
\def\aut{К.\,Г.~Чупраков$^1$}

\titel{\tit}{\aut}{\autkol}{\titkol}

%{\renewcommand{\thefootnote}{\fnsymbol{footnote}}\footnotetext[1]
%{Исследование поддержано грантами РФФИ 08-07-00152 и 09-07-12032.
%Статья написана на основе материалов доклада, представленного на IV 
%Международном семинаре  <<Прикладные задачи теории вероятностей и математической статистики, 
%связанные с моделированием информационных систем>> (зимняя сессия, Аоста, Италия, январь--февраль 2010~г.).}}

\renewcommand{\thefootnote}{\arabic{footnote}}
\footnotetext[1]{Институт проблем информатики Российской академии наук, chkos@rambler.ru}

\vspace*{6pt}

\Abst{Установлены некоторые зависимости между основными параметрами 
ситуационного зала: его размерами, информативностью отображаемого контента, числом 
одновременных наблюдателей и размерами экрана. Основой для таких зависимостей 
стали рекомендации, закрепленные в ГОСТах, и простые геометрические соображения.}

\vspace*{2pt}

\KW{системы отображения информации; ситуационный зал; область наилучшего 
наблюдения; аналитические зависимости}

\vspace*{4pt}

       \vskip 14pt plus 9pt minus 6pt

      \thispagestyle{headings}

      \begin{multicols}{2}

      \label{st\stat}



\section{Введение}
    
    Использование ситуационных центров доказало их практическую 
значимость для решения задач управления крупными и сложными 
объектами, в числе которых государственные учреждения~[1], а также 
крупные корпорации и предприятия~[2]. Однако создание ситуационных 
центров выявило ряд проблем, возникающих и в момент их разработки, и в 
процессе эксплуатации~[1, 3--7].

Ситуационный зал является одним из приложений ситуационного центра для 
решения задач ситуационного управления с помощью двух основных 
методик: обсуждение возможных решений экспертной группой и доклад 
(презентация) некоторого материала с помощью средств отображения 
информации~\cite{8chu, 9chu}. В~ситуационном зале могут анализироваться 
и разрабатываться различные варианты решения стратегического характера. 
Поэтому создание ситуационного зала~--- это не просто оснащение комнаты 
презентационным оборудованием с обеспечением некоторого 
респектабельного облика помещения, но и создание максимального уровня 
комфорта, где присутствие даже самой незначительной детали должно быть 
обосновано на уровне технического задания.

Для создания ситуационного зала в первую очередь необходимо оценить 
целый ряд параметров: размеры помещения, максимальное число человек, 
участвующих в переговорах, необходимую производительность средств 
отображений информации с точки зрения статической информативности 
экрана, а также геометрические размеры экранов этих средств отображения.

В статье предложен подход к определению взаимосвязей между основными 
параметрами ситуационного зала и оценке относительного расположения 
рабочих мест сотрудников и коллективного экрана.    

\section{Общий подход. Термины и~определения}

Создание и оборудование ситуационного зала среди прочих требований 
должно опираться на существующие стандарты по эргономике, действующие 
на территории РФ. В табл.~1 перечислены основные термины и понятия, 
которые будут использованы в рамках статьи~[10--12].

\begin{table*}\small
\begin{center}
\Caption{Основные термины и понятия
}
\vspace*{2ex}

\tabcolsep=5.5pt
\begin{tabular}{|l|c|l|}
\hline
\multicolumn{1}{|c|}{Термин}&\tabcolsep=0pt\begin{tabular}{c}Обозна-\\ чение\end{tabular}&
\multicolumn{1}{c|}{Определение}\\
\hline
Активная часть экрана&&Часть экрана, ограниченная пикселами\\
\hline
\tabcolsep=0pt\begin{tabular}{l}Стягиваемый угол\\ (угловой размер)\end{tabular}&
$\psi$&\tabcolsep=0pt\begin{tabular}{l}Размер визуального объекта при данном конкретном 
расстоянии наблю-\\ дения:
$\psi=2\arctg(h/(2D))$, где $h$~--- высота объекта; $D$~--- расстояние\\ наблюдения\end{tabular}\\
\hline
Высота знака&$h$&Линейная высота знака, м\\
\hline
Ширина знака&$w$&Линейная ширина знака, м\\
\hline
Формат знака&&
\tabcolsep=0pt\begin{tabular}{l}Число пикселов по горизонтали и вертикали в матрице, используемой для\\ построения 
символа\end{tabular}\\
\hline
\tabcolsep=0pt\begin{tabular}{l}Проектное расстояние\\ наблюдения\end{tabular}&
$D$&\tabcolsep=0pt\begin{tabular}{l}Расстояние или диапазон расстояний между экраном и глазами 
наблюдате-\\ля, при котором изображение соответствует требованиям разборчивости и\\ удобочитаемости\end{tabular} \\
\hline
\tabcolsep=0pt\begin{tabular}{l}Рабочая площадь\\ помещения\end{tabular}&
&\tabcolsep=0pt\begin{tabular}{l}Множество положений в помещении, в которых сохраняется 
разборчивость,\\ удобочитаемость материала, отображенного экраном\end{tabular} \\
\hline
Угол обзора человека&$\gamma$&
\tabcolsep=0pt\begin{tabular}{l}Угол, стягивающий точки пространства, которые человек способен 
наблю-\\ дать без напряжения для глазных мышц и поворотов головы \end{tabular}\\
\hline
\end{tabular}
\end{center}
\end{table*}
\begin{figure*}[b] %fig1
%\vspace*{-24pt}
\begin{center}
\mbox{%
\epsfxsize=149.148mm
\epsfbox{chu-1.eps}
}
\end{center}
\vspace*{-6pt}
\Caption{Рекомендуемый угол обзора человека в зависимости от интенсивности 
наблюдения
\label{f1chu}}
\end{figure*}

     Опираясь на эти понятия и термины, можно сформулировать и 
проанализировать требования и зависимости, описанные в государственных 
стандартах.
     
     В качестве одного из основных будет использовано понятие 
\textbf{области наилучшего наблюдения}, под которой понимается область 
в трехмерном или двумерном пространстве, удовлетворяющая 
эргономическим требованиям ГОСТов.
    

\subsection{Угол обзора человека} %2.1

Согласно~\cite{13chu, 14chu}:
\begin{itemize}
\item очень часто используемые средства отображения информации, 
требующие точного и быст\-ро\-го считывания показаний, следует 
располагать так, чтобы в вертикальном сечении они были видны под 
углом $\pm 15^\circ$ от нормальной линии взгляда и в горизонтальном 
сечении~--- под углом $\pm 15^\circ$ от плоскости симметрии 
человеческого тела (рис.~1);
\item часто используемые средства отображения информации, 
требующие менее точного и быстрого считывания показаний,~--- 
$\pm30^\circ$ по вертикали и горизонтали;
\item редко используемые средства отображения информации~--- $\pm 
60^\circ$ по вертикали и горизонтали.
\end{itemize}



\subsection{Угловой размер знака} %2.2

     Стягиваемый угол определяется согласно~\cite{10chu, 15chu}. 
Рекомендуемые показатели для обеспечения разборчивости (для латинского алфавита):
минимальный $\psi_{\min} =16^\prime$; предпочтительный $\psi_{\mathrm{предп}}=$\linebreak $=20^\prime\mbox{--}22^\prime$.

     Из определения угла $\psi$ следует, что
     \begin{equation}
     \fr{h}{D}=2\tg\fr{\psi}{2}\approx \psi\,.
     \label{e1chu}
     \end{equation}
     
     Формула~(\ref{e1chu}) верна, так как угол~$\psi$, измеряемый здесь и 
далее в радианах, очень мал. Данное соотношение позволяет оценить 
минимальное и рекомендуемое отношение высоты знака и проектного 
расстояния:
минимальное $(h/D)_{\min}=\psi_{\min}=$\linebreak $= 4{,}7\cdot 10^{-3}$;
рекомендуемое $(h/D)_{\mathrm{рек}}=\psi_{\mathrm{рек}}=$\linebreak $= 5{,}8\mbox{--}6{,}4\cdot 10^{-3}$.

\subsection{Формат букв} %2.3

     Отношение линейных параметров ширины знака к его высоте должно 
соответствовать следующим значениям (для латинского алфавита)~[12]:
допустимый диапазон~--- от 0,5:1 до 1:1,  
предпочтительный диапазон~--- от 0,6:1 до 0,9:1.




\begin{table*}\small %tabl2
\begin{center}
\Caption{ Основные параметры, используемые в зависимостях
}
\vspace*{2ex}
\begin{tabular}{|l|c|l|}
\hline
\multicolumn{1}{|c|}{Термины}&Обозначение&\multicolumn{1}{c|}{Определение}\\
\hline
Информативность&$I$&Максимальное число знаков в одном кадре контента\\
\hline
Число рабочих мест&$N$&Число рабочих мест для одного коллективного экрана\\
\hline
\tabcolsep=0pt\begin{tabular}{l}Диаметр\\ помещения\end{tabular}&$P$&
\tabcolsep=0pt\begin{tabular}{l}Максимальное расстояние между двумя точками поме-\\щения, в 
том числе обусловленное его геометрией, м\end{tabular}\\
\hline
Ширина экрана&$W$&Геометрическая ширина активной части экрана, м\\
\hline
Высота экрана&$H$&Геометрическая высота активной части экрана, м\\
\hline
\end{tabular}
\end{center}
\vspace*{9pt}
\end{table*}



\section{Формирование зависимостей между основными 
параметрами ситуационного зала}

    На основании рекомендаций, обозначенных в п.~2, можно приступить 
к формированию взаимосвязей между основными параметрами системы 
<<помещение--экран--наблюдатели>>. Этими основными параметрами 
являются: информативность, число рабочих мест, диаметр помещения, 
ширина и высота экрана (табл.~2).


     Стоит отметить разницу между проектным\linebreak расстоянием 
наблюдения~$D$ и диаметром помещения~$P$. Первый параметр 
характеризуется свойствами системы <<экран--наблюдатели>>, а 
параметр~$P$~--- исключительно свойствами помещения.

\subsection{Область наилучшего наблюдения для~коллективного экрана} 
%3.1
    
    Пусть $\varphi$~--- угол направления обзора элемента экрана 
относительно нормали к поверхности экрана из центра элемента. Тогда 
размер элемента при наблюдении под углом~$\varphi$ уменьшится и может 
быть приближенно вычислен как
    \begin{equation}
    w(\varphi)=w\cos\varphi\,,
    \label{e2chu}
    \end{equation}
    где $w$~--- линейный размер элемента, а $w(\varphi)$~--- его линейный 
размер при наблюдении под углом~$\varphi$. Поэтому из сохранения 
углового размера символа следует соотношение:
    \begin{equation*}
    D(\varphi)=D\cos\varphi\,.
%    \label{e3chu}
    \end{equation*}

Следовательно, область наилучшего наблюдения элемента экрана (в плоском 
горизонтальном сечении)~--- круг, касающийся экрана в точке, 
соответствующей этому элементу и диаметром, равным~$D$~--- проектному 
расстоянию (рис.~2).




    Рассматривая экран как множество активных точек~\cite{16chu}, для 
каждой из которых строится область\linebreak\vspace*{-12pt}

\begin{center} %fig2
%\vspace*{6pt}
\mbox{%
\epsfxsize=43.302mm
\epsfbox{chu-2.eps}
}
\end{center}
\vspace*{4pt}
\begin{center}
{{\figurename~2}\ \ \small{Область наилучшего наблюдения для точечного элемента экрана}}
\end{center}
%\vspace*{9pt}

\medskip
\addtocounter{figure}{1}

\noindent
 наилучшего наблюдения, можно 
построить область наилучшего наблюдения для всего экрана~--- пересечение 
множества построенных кругов (рис.~3). Далее рассматривается случай, 
когда экран плоский и рабочие места располагаются в горизонтальной 
плоскости.



В случае, когда экран плоский, областью наилучшего наблюдения будет 
пересечение двух наиболее далеких друг от друга кругов.

     Другим существенным ограничением, которое необходимо наложить 
на область наилучшего наблюдения, являются пропорции наблюдаемых 
символов, которые, согласно данным п.~2.3, не могут меняться более чем в 2~раза, 
так как максимальное и минимальное допустимые значения пропорций 
отличаются друг от друга ровно в  2~раза (максимальная ширина равна 
одной высоте знака, а минимальная~--- ее половине). Значит, в силу 
формулы~(\ref{e2chu}) максимальный угол отклонения от нормали не 
должен превышать~60$^\circ$.
     
     Пусть $\alpha$~--- максимальный угол наблюдения для любого малого 
элемента поверхности экрана из некоторой точки пространства, 
расположенной со стороны активной поверхности экрана. Ясно, что этот 
максимум в горизонтальной плоскости будет достигаться возле правого или 
левого края экрана. Ввиду вышесказанного угол~$\alpha$ не должен 
превосходить~60$^\circ$.

\end{multicols}

\begin{figure} %fig3
\vspace*{1pt}
\begin{center}
\mbox{%
\epsfxsize=97.554mm
\epsfbox{chu-3.eps}
}
\end{center}
\vspace*{-6pt}
\Caption{Область наилучшего наблюдения экрана
\label{f3chu}}
\end{figure}

\begin{figure} %fig4
\vspace*{1pt}
\begin{center}
\mbox{%
\epsfxsize=97.554mm
\epsfbox{chu-4.eps}
}
\end{center}
\vspace*{-6pt}
\Caption{Геометрическая модель области наилучшего наблюдения
\label{f4chu}}
\vspace*{9pt}
\end{figure}

\begin{multicols}{2}

%\begin{center} %fig4
%%\vspace*{6pt}
%\mbox{%
%\epsfxsize=80mm
%\epsfbox{chu-4.eps}
%}
%\end{center}
%\vspace*{4pt}
%\begin{center}
%{{\figurename~4}\ \ \small{Область наилучшего наблюдения экрана}}
%\end{center}
%%\vspace*{9pt}

%\bigskip
%\addtocounter{figure}{1}
     
     Получается, что область наилучшего наблюдения заключена внутри 
угла пересечения двух лучей, проведенных из крайних точек экрана под 
углом~$\alpha$ к нормали. С~учетом всех ограничений область наилучшего 
наблюдения будет являться фигурой пересечения двух уже построенных 
областей (рис.~4).



    Точки~$A$, $B$, $C$ и~$E$ имеют следующие коорди\-наты:
    \begin{align*}
&A\left( 0;\,\fr{W\ctg\alpha}{2}\right)\,;
\end{align*}

\noindent
\begin{align*}
&B\left(\fr{D\sin 2\alpha-
W}{2};\,D\cos^2\alpha\right)\,;\\
& C\left(0;\,\fr{D+\sqrt{D^2- W^2}}{2}\right)\,; \\
&  E\left( \fr{-D\sin 2\alpha +W}{2};\,D\cos^2\alpha\right)\,.
\end{align*}

Площадь области $ABCE$ можно оценить снизу с помощью 
треугольников~$ABC$ и~$ACE$. Площадь каждого из них может быть 
посчитана по формуле

\noindent
\begin{multline*}
S_{\triangle}=\fr{1}{2}AC\cdot BH ={}\\
{}=\fr{1}{8}\left( D\sin 2\alpha -
W\right)\left(D-W\ctg\alpha+\sqrt{D^2-W^2}\right),
%\label{e4chu}
\end{multline*}
где $BH$~--- высота треугольника~$ABC$, опущенная из вершины~$B$ на 
сторону~$AC$. С~учетом того, что таких треугольника два, получаем оценку 
площади области наилучшего наблюдения:
\begin{multline}
S_{\mathrm{о.н.н.}}=\fr{D^2}{4}\left(\sin2\alpha-C\right)\times{}\\
{}\times\left(1-
C\ctg\alpha+\sqrt{1-C^2}\right) ={}\\
{}=
\fr{W^2}{4C^2}\left(\sin2\alpha-C\right)
\left(1-C\ctg\alpha+\sqrt{1-C^2}\right)\,,
\label{e5chu}
\end{multline}
где
$C=W/D$~--- константа системы, которая, как будет видно далее, зависит от 
информативности контента~$I$, отношения~$k$ высоты экрана к ширине, 
отношения~$p$ ширины знака к его высоте и угла~$\psi$, стягиваемого 
одним символом.

\subsection{Число рабочих мест в~области наилучшего наблюдения} %3.2

Пусть $D$ и~$W$ зафиксированы. Тогда определена область наилучшего 
наблюдения и можно оценить максимальное число человек~$N$, которые 
смогут находиться в области наилучшего наблюдения, используя один 
коллективный экран. Главным условием комфортной работы условимся 
считать отсутствие помех со стороны других пользователей на расстоянии 
вытянутой руки. Таким образом, каждому пользователю сопоставим круг 
радиусом, равным половине маховой сажени (маховая сажень $\approx 
1{,}8$~м). Наиболее плотной расстановкой этих кругов на плоскости будет 
расположение, когда каж\-дый из кругов касается других шести. При этом 
центры кругов образуют сетку из равносторонних треугольников со 
стороной, равной одной маховой сажени, или 1,8~м (рис.~5).


Для оценки числа точек равномерной треугольной сетки, которые могут 
попасть внутрь области
 наилучшего наблюдения, воспользуемся формулой
Пика, связывающей число узлов квадратной сетки
 с шагом~1, попавших 
внутрь и на границу многоугольника, с площадью этого 
многоугольника~\cite{17chu}:
\begin{equation*}
S=B+\fr{\Gamma}{2}-1\,,
%\label{e7chu}
\end{equation*}
где $S$~--- целочисленная площадь многоугольника, $B$~--- число узлов 
квадратной сетки с шагом~1, попавших внутрь многоугольника; 
$\Gamma$~--- число узлов этой сетки, которые попали на границу.
\begin{center} %fig5
%\vspace*{6pt}
\mbox{%
\epsfxsize=73.141mm
\epsfbox{chu-5.eps}
}
\end{center}
\vspace*{4pt}
\begin{center}
{{\figurename~5}\ \ \small{Сетка из центров кругов}}
\end{center}
%\vspace*{9pt}

%\bigskip
\addtocounter{figure}{1}



Треугольную сетку с шагом~1,8 можно рас\-смат\-ри\-вать как квадратную сетку 
с шагом~1, которая претерпела два преобразования:
\begin{itemize}
\item сжатие по направлению вектора, параллельного диагонали квадрата 
сетки с коэффициентом~$\sqrt{2}$;
\item растяжение по двум любым взаимно перпендикулярным осям с 
коэффициентом~1,8.
\end{itemize}

Поэтому для применения формулы Пика необходимо рассматривать не 
исходное значение\linebreak площади, а ее преобразование, обратное преобразованиям 
сетки. То есть в случае равномерной треугольной сетки с шагом~1 будет 
выполнено следующее соотношение:
\begin{equation}
\fr{\sqrt{2}}{1{,}8^2}\,S=B+\fr{\Gamma}{2}-1\,.
\label{e8chu}
\end{equation}
Число рабочих мест в области наилучшего наблюдения при этом составляет
\begin{equation}
N=B+\Gamma=\fr{\sqrt{2}}{1{,}8^2}\,S+1+\fr{\Gamma}{2}\,.
\label{e9chu}
\end{equation}
Таким образом, для получения оценки числа рабочих мест в области 
наилучшего наблюдения необходимо оценить максимальное 
значение~$\Gamma$. Важно понимать, что формула Пика описывает 
площадь многоугольника с вершинами в узлах сетки. Ясно, что в случае 
области наилучшего наблюдения такое обеспечить не всегда возможно. 
Поэтому формула~(\ref{e9chu}) представляет собой верхнюю оценку  
числа рабочих мест.

Из оценки периметра области наилучшего наблюдения получаем, что
\begin{multline*}
\Gamma\leq \fr{1}{1{,}8}\left(2\vert AB\vert 
+2\widehat{CB}\right)={}\\
{}=\fr{D}{1{,}8}\left(\fr{\sin2\alpha-C}{\sin\alpha}+2\alpha-
\arcsin C\right)\,.
%\label{e10chu}
\end{multline*}
Таким образом, число рабочих мест в области наилучшего наблюдения 
может быть оценено сверху следующей величиной:
\begin{multline}
N=B+\Gamma=0{,}44S+{}\\
{}+\fr{D}{3{,}6}\left( \fr{\sin2\alpha-C}{\sin\alpha}+2\alpha-
\arcsin C\right)+1\,.
\label{e11chu}
\end{multline}

\subsection{Оценка максимальной площади области наилучшего 
наблюдения и~максимального числа рабочих мест~в~этой области}

Пусть~$I$ определено. Тогда исходя из пространственных ограничений, 
характеризуемых диа\-мет\-ром~$P$, можно оценить максимальную площадь 
области наилучшего наблюдения при некоторых значениях проектного 
расстояния~$D$ и ширины экрана~$W$.

Ввиду того, что соотношение сторон экрана у большинства производителей 
на настоящий момент составляет от~16:9 до~16:12, можно исключить из 
рассмотрения параметр высоты экрана. Дальнейшие оценки получаются в 
результате подсчета с двух сторон площади активной части экрана
\begin{equation*}
S_{\mathrm{display}}=WH=Iwh\,.
%\label{e12chu}
\end{equation*}
Пусть
\begin{equation*}
H=kW\,; %\label{e13chu}
\quad w=ph\,, %\label{e14chu}
\end{equation*}
где $k\in [9/16;\,12/16]$, $p\in [0{,}6;\,0{,}9]$ (см.\ п.~2.3). %Значение~$p$ соответствует  табл.~2.
Тогда
\begin{equation}
kW^2=Iph^2\ \ \mbox{или}\ \ W=h\sqrt{\fr{Ip}{k}}\,.
\label{e15chu}
\end{equation}
Из формулы~(\ref{e1chu}) следует, что
\begin{equation}
h=D\psi\,.
\label{e16chu}
\end{equation}

Из формул~(\ref{e15chu}) и~(\ref{e16chu}) получается, что при 
фиксированных значениях информативности~$I$, отношения~$k$ высоты 
экрана к ширине, отношения $p$ ширины знака к его высоте и угла~$\psi$, 
стягиваемого одним символом, существует константа~$C$, 
удовлетворяющая следующим соотношениям:
\begin{equation}
C=\fr{W}{D}=\psi\sqrt{\fr{Ip}{k}}\,.
\label{e17chu}
\end{equation}
Из формулы~(\ref{e17chu}) следует, что
\begin{equation}
C_{\min} 
=\psi_{\min}\sqrt{\fr{p_{\min}}{k_{\max}}}\,\sqrt{I}=\fr{\sqrt{I}}{193}\,.
\label{e18chu}
\end{equation}
     
     Из-за ограниченности помещения проектное расстояние~$D$ не может 
превышать диаметра помещения~$P$, поэтому ввиду оценок~(\ref{e5chu}) 
и~(\ref{e18chu})
     \begin{multline}
     S_{\max} ={}\\
     {}=\fr{D^2}{4}\left(\sin2\alpha-C\right)\left(1-C\ctg\alpha+\sqrt{1-
C^2}\right)\leq{}\\
     {}\leq \fr{P^2}{2}\left(\sin2\alpha-\fr{\sqrt{I}}{193}\right)\times{}\\
     {}\times\left(1-
\fr{\ctg \alpha\cdot \sqrt{I}}{193}+\sqrt{1-\fr{I}{193^2}}\right)\,.
     \label{e19chu}
     \end{multline}
     
     Отметим, что информативность не должна превосходить 
$(193\sin2\alpha)^2$~знаков. В~противном случае область наилучшего 
наблюдения окажется пустым множеством.

На основании формулы~(\ref{e19chu}) можно оценить число рабочих мест, 
которые можно разместить в области наилучшего наблюдения с учетом 
возможностей помещения, ограниченных его диа\-мет\-ром~$P$. Согласно 
оценкам~(\ref{e11chu}) и~(\ref{e19chu})
\begin{multline}
N_{\max}=0{,}44S_{\max}+\fr{P}{3{,}6}\left(\fr{\sin2\alpha-\sqrt{I}/193}{\sin\alpha}+{}\right.\\
\left.{}+2\alpha-
\arcsin\fr{\sqrt{I}}{193}\right)+1\,.
\label{e20chu}
\end{multline}

\subsection{Оценка минимальной ширины активной поверхности 
экрана}

Чем меньше размер экрана, тем меньше его стои\-мость при прочих равных 
условиях. Поэтому размеры экрана, в том числе и его ширина, должны быть 
по возможности минимизированы. Пусть известна информативность 
контента~$I$, угол наблюдения~$\alpha$ и диаметр помещения~$P$. 
Рассмотрим два принципиально разных случая:
\begin{enumerate}[1.]
\item Число наблюдателей неизвестно, необходимо оценить размеры 
экрана (его ширину), позволяющие эффективно использовать 
пространство помещения.
\item Число наблюдателей~$N$ известно, но оно меньше, чем полученное 
по формуле~(\ref{e20chu}) для известных параметров~$P$, $I$ и~$\alpha$. 
Необходимо оценить минимальные размеры экрана (его ширину), 
позволяющие обеспечить расположение всех наблюдателей в области 
наилучшего наблюдения с учетом отсутствия взаимных помех.
\end{enumerate}

\smallskip

\noindent
\textbf{Случай 1.} Из соотношения~(\ref{e17chu}) следует, что
\begin{equation*}
W= PC\geq PC_{\min}=\fr{P\sqrt{I}}{193}\,.
%\label{e21chu}
\end{equation*}

\smallskip

\noindent
\textbf{Случай 2.} Оценим минимальную площадь, на которой могут 
разместиться $N$~наблюдателей. Для этого воспользуемся тем 
соображением, что минимальная площадь равна количеству внутренних 
рабочих мест (тех, которые не попали на границу), умноженному на площадь 
двух треугольников, ограниченных сеткой. Такое соображение следует из 
того, что каждой внутренней точке можно сопоставить 6~треугольников 
сетки, а каждому треугольнику сетки~--- 3~точки. Это значит, что каждой 
точке сопоставляется по 2~треугольника.

Из формул~(\ref{e8chu}) и~(\ref{e9chu}) следует, что
\begin{equation*}
B=\fr{2\sqrt{2}}{1{,}8^2}\,S+2-N\,.
%\label{e22chu}
\end{equation*}
С другой стороны, как было замечено,
\begin{equation*}
S=B\cdot 2S_{\triangle}\,.
%\label{e23chu}
\end{equation*}
Поэтому
\begin{equation*}
S_{\min} =(N-2)\fr{2\cdot 1{,}8^2\cdot1{,}4}{4\sqrt{2}\cdot1{,}4-
1{,}8^2}=1{,}94(N-2)\,.
%\label{e24chu}
\end{equation*}

Из формулы~(\ref{e5chu}) получаем:
\begin{multline*}
W_{\min} =2C_{\min}
\left(
{S_{\min}}\Big /\left(\vphantom{\sqrt{1-C^2_{\min}}}(
\sin2\alpha-C_{\min})\times{}\right.\right.\\
\left.\left.{}\times(1-
C_{\min}\ctg\alpha+\sqrt{1-C^2_{\min}})\right)\right)^{1/2}={}\\
{}=2
\left(
{1{,}94I(N-2)}\Big /\left(\vphantom{\sqrt{193^2-I}}
(193\sin2\alpha-\sqrt{I})\times{}\right.\right.\\
\left.\left.{}\times (193-
\sqrt{I}\ctg\alpha+\sqrt{193^2-I})\right)\right)^{1/2}\,.
%\label{e25chu}
\end{multline*}


\section{Заключение}

В статье сформулированы основные определения и термины, используемые 
при проектировании средств отображения информации, а также описаны 
рекомендации к ним, действующие в рамках государственных стандартов.

На основании этих требований и рекомендаций, а также на основании 
простых геометрических соображений получены оценки основных 
параметров для проектирования средства отображения и рабочих мест. 
В~качестве основной определяющей величины выступает информативность 
контента, которая формируется на основании задач ситуационного зала и 
ситуационных моделей отображения.
     
     В качестве основного использован параметр отношения ширины экрана 
к проектному расстоянию наблюдения. Часто этот параметр для удобства 
считается фиксированным, но в общем случае показано, что это отношение 
прямо пропорционально корню квадратному от величины информативности 
контента.
    
Если, помимо информативности контента, известен диаметр помещения, в 
котором располагается ситуационный зал, то можно определить 
максимальную площадь области наилучшего наблюдения и оценить число 
рабочих мест, которые могут быть расположены в области наилучшего 
наблюдения с помощью формулы Пика и аффинных преобразований.

В качестве обратной задачи, в которой известно количество человек, 
одновременно работающих с экраном, получена зависимость для 
определения минимально необходимой ширина экрана.

{\small\frenchspacing
{%\baselineskip=10.8pt
\addcontentsline{toc}{section}{Литература}
\begin{thebibliography}{99}


\bibitem{1chu}
\Au{Ильин Н.\,И.}
Основные направления развития ситуационных центров органов государственной 
власти~// ВКСС Connect! (Ведомственные корпоративные сети и системы), 2007. 
№\,6(45). С.~2--9.

\bibitem{2chu}
\Au{Лисица К.\,В.}
Опыт создания и применения Автоматизированной системы стратегического 
управления в ОАО <<Российские железные дороги>>~// Ситуационные центры: 
модели, технологии, опыт практической реализации: Мат-лы науч.-практич. 
конф.~--- М.: РАГС, 2007.

\bibitem{4chu} %3
\Au{Филиппович А.\,Ю.}
Ситуационная система~--- что это такое?~// PCWeek/RE, 2003. No.\,26.

\bibitem{6chu} %4
\Au{Зацаринный А.\,А., Ионенков Ю.\,С., Кондрашев~В.\,А.}
Об одном подходе к выбору системотехнических решений построения 
информационно-те\-ле\-ком\-му\-никационных систем~// Системы и средства информатики. 
Вып.~16.~--- М.: Наука, 2006. С.~65--72.


\bibitem{3chu} %5
\Au{Зацаринный А.\,А., Сучков А.\,В., Босов~А.\,В.}
Ситуа\-ционные центры в современных информационно-те\-ле\-ком\-му\-ни\-кационных 
системах специального\linebreak назначения~// ВКСС Connect! (Ведомственные корпоративные 
сети и системы), 2007. №\,5(44). С.~64--76.

\bibitem{7chu} %6
\Au{Зацаринный А.\,А., Ионенков Ю.\,С.}
Некоторые аспекты выбора технологии построения информационно-те\-ле\-ком\-муникационных сетей~// 
Системы и средства информатики. Вып.~17.~--- М.: 
Наука, 2007. С.~5--16.


\bibitem{5chu} %7
\Au{Зацаринный А.\,А.}
Тенденции развития ситуационных центров как компонентов информационно-те\-ле\-ком\-му\-ни\-ка\-ционных 
систем в условиях глобальной информатизации общества~// 
Докл. XXXV\linebreak междунар. конф. <<Информационные технологии в науке, 
образовании, телекоммуникации и бизнесе (IT\;+\;S\&E'08)>>. Ялта--Гурзуф, Украина. 
2008.


\bibitem{8chu}
Ситуационные центры (СЦ) и их история. {\sf http://\linebreak ta.interrussoft.com/s\_centre.html}.

\bibitem{9chu}
\Au{Зацаринный А.\,А.}
Организационные принципы сис\-тем\-но\-го подхода к разработке, проектированию и 
внедрению современных информационно-те\-ле\-ком\-му\-никационных сетей~// ВКСС 
Connect! (Ведомственные корпоративные сети и системы), 2007. №\,1(40). С.~60--67.

\bibitem{11chu} %10
ГОСТ 26387-84 Система <<Человек--машина>>. Термины и определения.~--- М.: 
Стандартинформ, 2006.

\bibitem{12chu} %11
ГОСТ 27833-88 Средства отображения информации. Термины и определения.~--- М.: 
Стандартинформ, 2005.

\bibitem{10chu} %12
ГОСТ Р~52324-2005 (ИСО 13406-2:2001) Эргономические требования к работе с 
визуальными дисплеями, основанными на плоских панелях.~--- М.: Стандартинформ, 
2005.

\bibitem{13chu}
ГОСТ 12.2.032-78 Система стандартов безопасности труда. Рабочее место при 
выполнении работ сидя. Общие эргономические требования.~--- М.: Изд-во 
стандартов, 2001.

\bibitem{14chu}
ГОСТ 12.2.033-78 Система стандартов безопасности труда. Рабочее место при 
выполнении работ стоя. Общие эргономические требования.~--- М.: Изд-во 
стандартов, 2001.

\bibitem{15chu}
ГОСТ Р ИСО 9241-3 Эргономические требования при выполнении офисных работ с 
использованием видеодисплейных терминалов.~--- М.: Изд-во стандартов, 2003.

\bibitem{16chu}
ГОСТ 21958-76 Зал и кабины операторов. Взаимное расположение рабочих мест.~--- 
М.: Изд-во стандартов, 1976.

\label{end\stat}

\bibitem{17chu}
\Au{Прасолов В.\,В.}
Задачи по планиметрии.~--- М.: \mbox{МЦНМО}, 2001.


 \end{thebibliography}
}
}


\end{multicols}  %9

\def\stat{kozerenko}

\def\tit{КОГНИТИВНО-ЛИНГВИСТИЧЕСКИЕ ПРЕДСТАВЛЕНИЯ 
В~СИСТЕМАХ ОБРАБОТКИ ТЕКСТОВ}

\def\titkol{Когнитивно-лингвистические представления 
в~системах обработки текстов}

\def\autkol{Е.\,Б.~Козеренко, И.\,П.~Кузнецов}
\def\aut{Е.\,Б.~Козеренко$^1$, И.\,П.~Кузнецов$^2$}

\titel{\tit}{\aut}{\autkol}{\titkol}

%{\renewcommand{\thefootnote}{\fnsymbol{footnote}}\footnotetext[1]
%{Работа выполнена при поддержке Российского фонда фундаментальных
%исследований, проект~10-01-00480. Статья написана на основе материалов доклада, 
%представленного на IV Международном семинаре <<Прикладные задачи теории вероятностей 
%и математической статистики, связанные с моделированием информационных систем>> 
%(зимняя сессия, Аоста, Италия, январь--февраль 2010 г.).}}

\renewcommand{\thefootnote}{\arabic{footnote}}
\footnotetext[1]{Институт проблем информатики Российской академии наук, kozerenko@mail.ru}
\footnotetext[2]{Институт проблем информатики Российской академии наук, igor-kuz@mtu-net.ru}


\Abst{Рассмотрены вопросы проектирования и развития 
семантико-синтаксических и лексико-семантических представлений в 
лингвистических процессорах ряда систем, основанных на аппарате расширенных 
семантических сетей (РСС). Системы этого класса создаются для извлечения знаний из 
текстов на естественных языках, отображения извлеченных сущностей и связей в 
структуры базы знаний (БЗ) и использования знаний для поддержки экспертных 
аналитических решений в различных сферах приложения. В~фокусе внимания 
находятся ин\-же\-нер\-но-линг\-ви\-сти\-че\-ские представления, позволяющие 
построить целостную работающую лингвистическую модель, которая 
модифицируется в зависимости от конкретной задачи: от <<тяжелой>> формы на 
основе детальных глубинных представлений до фокусных редуцированных 
оболочек, настроенных на узкую предметную область (ПО) и ограниченный язык 
общения. Особое внимание уделяется способам описания 
дис\-три\-бу\-тив\-но-транс\-фор\-ма\-ци\-он\-ных признаков языковых объектов.}

\KW{интеллектуальные системы; семантические представления; лингвистические 
процессоры; обработка естественного языка; извлечение знаний}

       \vskip 14pt plus 9pt minus 6pt

      \thispagestyle{headings}

      \begin{multicols}{2}

      \label{st\stat}

\section{Введение}

     Данная работа посвящена проблемам создания\linebreak 
     когни\-тив\-но-линг\-ви\-сти\-че\-ских моделей естественного языка для 
различных классов информационных систем и описанию опыта создания 
линг\-ви\-сти\-че\-ских представлений для интеллектуальных\linebreak технологий 
обработки текстов. Вопросы извлечения знаний из текстов и создания модели 
естественного языка рассматриваются в единстве. В центре внимания будут 
находиться лингвистические процессоры интеллектуальных систем, 
разработанных на основе аппарата \textit{расширенных семантических 
сетей}~[1--5]. %\cite{1koz}--\cite{3koz}, \cite{18koz}--\cite{19koz}. 
Будем 
называть их \textit{РСС-сис\-те\-мы}. Эти системы создавались коллективом 
разработчиков, включая авторов данной статьи в Институте проб\-лем 
информатики РАН на протяжении целого ряда лет в рамках 
исследовательских проектов и прикладных систем, ориентированных на 
конкретные ПО заказчиков. Можно выделить четыре 
поколения РСС-систем. Ко\-гни\-тив\-но-линг\-ви\-сти\-че\-ские 
представления, заложенные в основу систем этого класса, прошли 
определенный эволюционный путь. 
     
     Интеллектуальные РСС-сис\-те\-мы содержат развитые \textit{базы 
знаний}, при этом знания представлены в виде записей на языке 
РСС, называемых 
     \textit{РСС-струк\-ту\-ра\-ми}. Лингвистические знания, таким 
образом, являются частным случаем <<знаний>> и также представлены в 
виде записей на языке РСС. Основным 
конструктивным элементом РСС\linebreak является именованный $N$-мест\-ный 
предикат, на\-зы\-ва\-емый <<\textit{фрагментом}>>. Все множество языковых 
объектов задается в виде системы пре\-ди\-кат\-но-ак\-тант\-ных структур, при этом 
поддерживаются механизмы представления вложенных структур, что дает 
очень мощные изобразительные возможности для описания объектов 
различных языковых уровней. Очень важными факторами являются 
однородность и единообразие лингвистических представлений. 
     
     В процессе анализа и синтеза предложений естественного языка 
используется фор\-маль\-но-грам\-ма\-ти\-че\-ский аппарат, сходный с 
грамматиками зависимостей. При этом подходе опорными элементами 
служат слова и конструкции, выполняющие роль предикатов в предложении, 
и результатом анализа предложения должен стать один предикат, 
соответствующий сказуемому рассматриваемого предложения (т.\,е.\ 
основному глаголу в личной форме или другому основному предикатному 
выражению). Таким образом, в процессе анализа происходит выявление 
\textit{когнитивных опор} предложения: <<слов-дейст\-вий>> и 
     <<слов-от\-но\-ше\-ний>>, т.\,е.\ глаголов и других слов, имеющих 
синтактико-семантические валентности. Примером <<слов-от\-но\-ше\-ний>> 
могут служить, например, слова <<отец>>, <<друг>> и~т.\,п., т.\,е.\ в данном 
случае <<отношения>> (или \textit{функции}~--- в терминах языка логики 
предикатов 1-го порядка)~--- это слова, которые задают сильные, четко 
выраженные син\-так\-ти\-ко-се\-ман\-ти\-че\-ские ожидания. 
     
     Семантический анализ в ин\-же\-нер\-но-линг\-ви\-сти\-че\-ском 
понимании~--- это процесс перевода ес\-тест\-вен\-но-язы\-ко\-вых 
выражений во <<внутренние>> структуры БЗ, в 
рассматриваемой ситуации этими <<внутренними>> структурами являются 
записи на языке РСС. Таким образом, структуры БЗ~--- это код смысла в 
интеллектуальных информационных системах подобного рода. 
     
     В работе рассматриваются ин\-же\-нер\-но-линг\-ви\-сти\-че\-ские 
решения в системах с <<пол\-ным>> линг\-ви\-сти\-че\-ским анализом~--- это 
     сис\-те\-мы 1-го и 2-го поколения: ДИЕС1, ДИЕС2, 
     Логос-Д~\cite{2koz, 3koz}~--- и сис\-те\-мах с <<фактографическим>> 
подходом: интеллектуальных системах поддержки аналитических решений 
(ИСПАР)~\cite{18koz, 19koz}, где целью анализа является выделение 
сущностей и связей из текстов,~--- это системы 3-го и 4-го поколения. 

\section{Процесс концептуально-лингвистического моделирования 
в системах, основанных на аппарате расширенных семантических сетей}
     
\subsection{Центральные вопросы семантического моделирования} %2.1
     
     Концептуально-лингвистическое моделирование (КЛМ)~--- это 
процесс построения ес\-тест\-вен\-но-язы\-ко\-вой модели ПО (рис.~1), синтезирующий в себе подходы 
концептуального и лингвистического моделирования~[1--3]. 
По\-стро\-ение концептуально-лингвистической модели некоторой 
ПО подразделяется на следующие этапы:
     \begin{itemize}
     \item построение собственно концептуальной модели, т.\,е.\ вычленение 
базовых понятий, организация их в ро\-до-ви\-до\-вые деревья и определение 
связей между ними;
     \item разработка идеографического словаря ПО, т.\,е.\ 
лексическое наполнение концептуальной модели;
     \item ввод базовых правил, описывающих на естественном языке 
<<модель мира>>, релевантную данной ПО.
     \end{itemize}
     
     
     Методика КЛМ на 
основе аппарата РСС базируется на следующих принципах:
     \begin{itemize}
\item модель должна быть <<открытой>>, т.\,е.\ поддерживать эффективный 
механизм расширения и обновления информации;
\begin{center} %fig1
%\vspace*{3pt}
\hspace*{-10.7158pt}\mbox{%
\epsfxsize=77.871mm
\epsfbox{koz-1.eps}
}\hspace{10.7158pt}
%\end{center}
\vspace*{4pt}
%\begin{center}
{{\figurename~1}\ \ \small{Процесс КЛМ}}
\end{center}
\vspace*{3pt}

%\bigskip
\addtocounter{figure}{1}
\item модель представления <<смысла>> должна учитывать факты 
экстралингвистической реаль\-ности, которые в виде правил и отношений 
составляют некоторую базовую <<модель мира>>, достраиваемую 
конкретными моделями ПО;
\item модель должна быть практичной, т.\,е.\ не перегруженной детальными 
описаниями связей и отношений между понятиями, чтобы обеспечить 
возможность ее реализации, но в то же время отражать всю релевантную 
конкретной задаче информацию.
\end{itemize}

     \begin{figure*} %fig2
%     \begin{center}
\hspace*{23mm}\{(ВЫРАБАТЫВА895\_\_)(DICSEM)\\
\hspace*{23mm}COORD(PROGNOZ1,RUS,ВЫРАБАТЫВА895\_\_,S50\_31\_51\_20,\%)\\
\hspace*{23mm}SUB(UNIV,0+)~SUB(UNIV,1+)~SUB(UNIV,2+)\\
\hspace*{23mm}ВЫРАБАТЫВ(0-,1-,2-/3+)~INFI(3-)~ПРИДЕТСЯ(3-)~ПРИДЕТСЯ(3$-$/4+) \\
\hspace*{23mm}FUT1(4$-$)~SUB(СРЕД,5+)
%\end{center}
%\vspace*{2pt}
\Caption{Пример записи представления глагола <<вырабатывать>> в семантическом 
словаре
\label{f2koz}}
%\vspace*{6pt}
\end{figure*}

     Реалистичный подход к постановке задачи диктует необходимость 
ограничения моделируемого подмножества естественного языка. Суть 
ограничений сводится к следующему:
     \begin{enumerate}[(1)]
     \item анализируемые текстовые материалы содержат 
экспертные знания из конкретных ПО (в разработанных 
авторами системах это были такие ПО, как диагностика 
брака при изготовлении микросхем, социальное прогнозирование, 
криминалистика и другие);
     \item в целях максимально возможного устранения 
неоднозначности словарь строится по модульному принципу: есть некоторая 
наиболее общая часть (1--2~уровня), которая достраивается специальными 
словарями для каж\-дой отдельной~ПО.
     \end{enumerate}
     
     Предлагаемая модель лексической семантики основана на принципе 
<<ядерного>> значения, реализуемого в контексте данной 
ПО, с последующим индуктивным наращиванием других значений (если 
они актуализируются в рас\-смат\-ри\-ва\-емых контекстах). Также используется 
таксономия, которая реализуется в виде иерархических деревьев классов 
слов. 
     
     Общая <<модель мира>> системы является основой для моделей ПО. 
Элементами этой модели служат классы слов, которые подразделяются на 
понятия/имена, отношения, действия, свойства, характеристики действий, 
временные и пространственные характеристики.
     
     Самым общим понятием является \textit{концепт}, или 
\textit{универсальный класс}, который подразделяется на объект, ситуацию, 
процесс и~др. 
     
     Слова, относящиеся к классам действий и отношений, представлены 
как се\-ман\-ти\-ко-син\-так\-си\-че\-ские фреймы, задающие 
     пре\-ди\-кат\-но-ак\-тант\-ные структуры (модель управления). Однако 
в описываемом подходе (назовем его РСС-под\-хо\-дом) существенно 
расширена область значений актантов. Суть расширения состоит, во-первых, 
в том, что в роли актантов могут выступать не только простые объекты, 
соответствующие отдельным словам, но и структурные объекты, 
представляющие словосочетания и фразы, а во-вторых, в том, что понятие 
падежа включает в себя не только семантические, но и синтаксические 
признаки.
     
     Подход, основанный на РСС, позволяет отражать произвольный 
уровень вложенности структур за счет пропозициональных вершин 
семантической сети. Это обеспечивает представление\linebreak сложных 
синтаксических конструкций фраз\linebreak естественного языка, а также позволяет 
отразить\linebreak структурный характер лексической семантики,\linebreak которая в 
предлагаемой модели имеет иерар\-хи\-че\-ски-се\-те\-вую структуру. 
Линг\-ви\-сти\-че\-ские зна-\linebreak ния пред\-став\-ле\-ны в системном словаре и 
декла\-ра\-тивных модулях линг\-ви\-сти\-че\-ско\-го процессора.\linebreak В РСС-сис\-те\-мах 
так\-же реализована функция динамически форми\-ру\-емо\-го семантического 
словаря, который на основе исходной лингвистической информации 
достраивается системой автоматически в процессе об\-ра\-бот\-ки конкретных 
текстов. На рис.~\ref{f2koz} пред\-став\-ле\-но \mbox{такое} <<внутреннее>> описание 
глагола в семантическом словаре. Этот словарь автоматически генерируется 
РСС-системами ДИЕС2, ЛОГОС-Д, ИКС в процессе обработки 
     естест\-вен\-но-язы\-ко\-вых \mbox{текстов}. 
     {\looseness=1
     
     }
     
     
\subsection{Особенности применения аппарата расширенных семантических сетей 
в~когнитивно-лингвистическом моделировании} %2.2
     
     Дадим краткое описание аппарата РСС и  
обос\-ну\-ем выбор именно этого метода представления для моделирования 
естественного языка. Классическое понятие семантической сети сводится к 
следующему: задаются некоторые вершины, соответствующие объектам,  
вершины связываются дугами, которые помечаются именами отношений. 
Однако с помощью подобных сетей оказывается трудно представлять 
сложные виды информации, например, когда объекты, связанные 
отношениями, образуют агрегаты и когда отношения связываются между 
собой отношениями и~др. Поэтому в сети вводятся вершины, 
соответствующие именам отношений, а также специальный композиционный 
элемент, называемый вершиной связи. Вершина связи как бы <<разрывает>> 
дугу и подсоединяется одним ребром к вершине-отношению, а другими 
ребрами~--- к вершинам-объектам. Расширенная семантическая сеть является развитием такого сорта 
сетей в направлении повышения изобразительных возможностей при 
сохранении свойства однородности.
     
     Основой РСС является множество вершин ($V$), из которых 
составляются элементарные фрагменты (ЭФ) вида
     $
     V_0(V_1,V_2,\ldots ,V_k/V_{k+1})
     $, 
     где
$V_0, V_1, V_2,\ldots , V_k, V_{k+1}>0$.
     
     
     Такой фрагмент представляет $k$-местное отношение. Позиции 
вершин в ЭФ определяют их роли. 
Вершина~$V_0$ ставится в соответствие имени отношения, 
вершины~$V_1$, $V_2$, \ldots , $V_k$~--- объектам, участ\-ву\-ющим в 
отношении, а вершина~$V_{k+1}$, отделенная косой линией,~--- всей 
совокупности упомянутых объектов с учетом их отношения. В~дальнейшем 
будем $V_{k+1}$ называть $C$-вершиной ЭФ.\linebreak 
Множество ЭФ образует РСС. 
С~помощью РСС представляются наборы отношений, различные ситуации, 
сце\-нарии. Сильной стороной РСС-под\-хо\-да является возможность 
однородного пред\-став\-ле\-ния как предметной (концептуальной), так и 
лингвистической информации, что обеспечивает эффективную обработку 
знаний и поддержание непротиворечи\-вости~БЗ.
          \begin{figure*} %fig3
     \vspace*{1pt}
\begin{center}
\mbox{%
\epsfxsize=125.039mm
\epsfbox{koz-3.eps}
}
\end{center}
\vspace*{-9pt}
     \Caption{Семантико-синтаксический анализ без выявления глагольных 
словоформ
      \label{f3koz}}
\vspace*{12pt}
 %     \end{figure*}
%            \begin{figure*} %fig4
           \vspace*{1pt}
\begin{center}
\mbox{%
\epsfxsize=103.129mm
\epsfbox{koz-4.eps}
}
\end{center}
\vspace*{-9pt}
      \Caption{Целостная семантическая структура предложения
      \label{f4koz}}
      \end{figure*}

     
     Посредством РСС в БЗ представлены лингвистические  и 
предметные знания. Обработка этих знаний осуществляется 
продукциями языка ДЕКЛ, на котором реализованы сле\-ду\-ющие шесть 
блоков: морфологического анализа, семанти\-ческого анализа слов, 
син\-так\-ти\-ко-се\-ман\-ти\-че\-ско\-го анализа форм, 
прагматических функций, организации системной активности и 
обратный лингвистический процессор. С~помощью продукций 
осущест\-вля\-ет\-ся последовательное преобразование сети~--- РСС. При этом 
проходятся фазы, соответствующие уровню понимания входного текста. 
Рас\-смот\-рим~их.
     \begin{enumerate}[1.]
     \item На первом шаге анализа строится 
пространственная структура предложения с морфологической информацией 
для каждого слова.\linebreak Каж\-дый член предложения представляется вершиной 
семантической сети. Вместо слова генерируется код (если слово 
многозначно, т.\,е.\ принадлежит к нескольким классам,~--- то более одного 
кода). Основой кода служит корень слова. На этом этапе предложение 
представляется в виде набора фрагментов типа LRR (специальных меток 
результатов 1-го этапа анализа), объединяемых в целостную структуру 
посредством вершины связи. Результат 1-го этапа постоянно обращается к 
словарю: <<Что значит данное слово?>>
     \item На втором этапе каждой вершине сопоставляется семантический 
класс и присваивается новый код. За словами (т.\,е.\ конкретными вершинами 
РСС) система видит объекты, действия, свойства, т.\,е.\ строит 
классификации. Производится се\-ман\-ти\-ко-син\-так\-си\-че\-ский анализ 
без выявления глагольных словоформ, при этом предложение представляется 
в виде совокупности фрагментов типа SEM и SEMD~--- специальных меток 
результатов 2-го этапа анализа (рис.~\ref{f3koz}).
     \item На третьем этапе происходит частичное <<сворачивание>> 
синтаксических структур в более компактные (например, свойство объекта и 
сам объект) с присваиванием нового кода и строится фрагмент для объекта, 
обладающего этим свойством.
     \begin{figure*}[b] %fig5
          \vspace*{12pt}
\begin{center}
\mbox{%
\epsfxsize=147.485mm
\epsfbox{koz-5.eps}
}
\end{center}
\vspace*{-9pt}
     \Caption{Глубинная структура предложений
      \label{f5koz}}
      \end{figure*}      
     \item На четвертом этапе выявляются отношения и действия и 
производится анализ непосредственного контекста на соответствие заданным 
семантическим падежам. Система проверяет, подходят ли объекты 
(концепты, понятия) на аргументные места данного действия или отношения. 
При этом отглагольные существительные (<<делатель>>, т.\,е.\ агент 
действия, или <<делание>>~--- процесс~--- анализируются как слова с 
двойной природой: вначале как действия, а затем как объекты). Результатом 
этого этапа является целостная семантическая структура предложения, 
которая представляется фрагментом типа SEMSTR~--- метки результата 4-го 
этапа анализа (рис.~\ref{f4koz}).
     \item На пятом этапе происходит анализ прагматики: установление 
кореференциальных отношений, частичное восстановление эллиптических 
конструкций, система производит дальнейшие действия с построенными 
фрагментами.
     \end{enumerate}

     
Система ДИЕС допускает ввод полисемичных форм глаголов. Для этого следует 
воспользоваться формальной записью лингвистических знаний. 
     В~сис\-те\-мах, основанных на РСС, все функции реализованы на 
единой основе~--- в рамках языков РСС и ДЕКЛ, которые были разработаны 
с ориентацией на задачи обработки естественного языка.

%\vspace*{-6pt}

\section{Представление семантики глаголов, глубинные 
и~поверхностные структуры}
     
     В процессе анализа выявляются семантические вершины предложения: 
происходит выявление <<слов-дей\-ст\-вий>>, т.\,е.\ глаголов, и 
     <<слов-от\-но\-ше\-ний>>. Что же является конструктивной основой\linebreak 
задания семантических представлений предикатных слов и выражений? Как 
убедительно показано в работе~\cite{4koz}, семантика глагола 
определяется его дис\-три\-бу\-тив\-но-транс\-фор\-ма\-ци\-он\-ны\-ми\linebreak 
свойствами. Поэтому смысл предикатных выражений должен кодироваться с 
учетом их дистрибутивных и трансформационных признаков. 
     
     Выдвинутая рядом лингвистов (Хомский, Филлмор) гипотеза о том, что 
все предложения имеют глубинные и поверхностные 
     структуры~[7--10], явилась очень продуктивным 
источником проектных решений при создании первых РСС-сис\-тем и 
развивалась в дальнейшем. 

В~тео\-ре\-ти\-ко-линг\-ви\-сти\-че\-ском 
понимании глубинная структура~--- это абстракция, содержащая все 
элементы, необходимые для образования поверхностных структур 
предложений со сходной семантикой. 

     В~ин\-же\-нер\-но-линг\-ви\-сти\-че\-ском понимании\linebreak глубинная 
структура~--- это запись на языке БЗ, например на языке РСС, 
которая может быть представлена в <<поверхностном>> виде на одном из 
естественных языков в результате конечного числа определенных 
преобразований. Например, предложения

\noindent
\begin{align*}    
(1)\ &\mbox{\textit{The programmer writes the code}}\\
(2)\ &\mbox{\textit{The code is written by the programmer}}
\end{align*}
имеют истоком одну глубинную структуру:

\medskip

\noindent
     \begin{verbatim}
  Programmer <---- write ----> Code
      agent                   object,
\end{verbatim}

\medskip

\noindent
хотя и отличаются своими поверхностными структурами. В~каждом из них 
имеется агент (the programmer), объект (the code) и действие (write).\linebreak Согласно 
концепции \textit{падежной грамматики} Филлмора~\cite{5koz} глубинная 
структура для обоих предложений инвариантна. Эту структуру можно 
представить в виде скобочной записи $V(\mathrm{AGENT}, \mathrm{OBJECT})$. В~графическом 
виде глубинная структура предложения также может быть представлена 
диаграммой в виде дерева, где отражены инвариантные отношения 
зависимости между предикатной вершиной и актантами (рис.~\ref{f5koz}), 
причем в таком представлении явным образом разграничиваются 
\textit{модальность} (MOD) и \textit{пропозиция} (PROP).
     

     В исходном варианте~\cite{5koz} теория признавала шесть падежей: 
агентив, инструменталис, датив, объектив, локатив и фактитив. По мере 
развития теории~\cite{8koz} происходило увеличение числа падежей, однако 
<<умножение>> количества падежей утяжеляет первоначальную 
конфигурацию, поэтому при построении инженерных семантических 
представлений требуется некоторый <<компромиссный>> вариант, 
сочетающий в себе необходимую полноту, с одной стороны, и простоту и 
гибкость, с другой.

\begin{figure*}[b] %fig6
\vspace*{24pt}
\begin{center}
\mbox{%
\epsfxsize=156.873mm
\epsfbox{koz-6.eps}
}
\end{center}
%\vspace*{-9pt}
\Caption{Обобщенное функциональное представление систем ИСПАР
\label{f6koz}}
\end{figure*}
     
%\vspace*{-6pt}

\section{Некоторые базовые аспекты построения многоязычных 
систем}
     
     Одним из приоритетных направлений развития РСС-сис\-тем является 
обеспечение обработки текстов на нескольких языках, прежде всего для 
рус\-ско-анг\-лий\-ской языковой пары. В системах 2-го поколения~--- ДИЕС2, 
ИКС, ЛОГОС-Д были реализованы лингвистические процессоры и словари 
для русского и английского языка, позволявшие обрабатывать тексты для 
ряда ПО. При этом поддерживался как режим ввода 
лингвистических знаний линг\-вис\-том-ана\-ли\-ти\-ком, так и 
автоматический режим самообучения системы по вводимым \mbox{текстам}. 
{\looseness=1

}

Проводились также эксперименты с итальянским и французским языком. 
При создании многоязычных систем авторы обращались к европейским 
языкам. Очевидно, что европейские языки обладают большим числом общих 
правил, чем любой из них с языками других групп. Но при этом все 
естественные языки обладают общей структурой на самом глубинном 
уровне. На этом уровне располагаются главные элементы естественного 
языка: \textit{предложение}, \textit{модальность}, \textit{пропозиция}.
     
     Моделирование смысловых представлений~--- это процесс, 
развивающийся в направлении от поверхностных семантических структур к 
глубинным. Поиск такого внутреннего представления смысла в условиях 
многоязычной ситуации является на\-прав\-ле\-ни\-ем развития методов 
     КЛМ на базе  РСС. 
     
%     \vspace*{-48pt}
     
\section{Интеллектуальные системы поддержки аналитических 
решений}
     
Системы РСС 3-го и 4-го поколения на\-прав\-ле\-ны на извлечение знаний 
в виде \textit{объектов}, или \textit{сущностей}, и связей между ними из 
пред\-мет\-но-ориен\-ти\-ро\-ван\-ных текстов на русском и английском 
языке~\cite{18koz, 19koz}.

    
В настоящее время во всем мире активно ведутся работы по созданию 
систем извлечения фактов из текстов на естественных языках~[11--14], создаются развитые тезаурусы и 
онтологии~\cite{17koz}. Сис\-те\-мы РСС функционально шире, поскольку 
имеют возможность не только извлекать факты, но и поддерживать 
механизмы логического анализа и экспертного вывода на основе 
извлеченных знаний. Сис\-те\-ма\-ми такого рода являются ИСПАР. В~целом это 
направление исследований требует дальнейшей проработки 
     лек\-си\-ко-се\-ман\-ти\-че\-ских представлений, создания 
     пред\-мет\-но-ориен\-ти\-ро\-ван\-ных семантических словарей. 

Обобщенное функциональное представление систем ИСПАР дано на 
рис.~\ref{f6koz}. 
     
     В рамках ИСПАР на основе РСС 
(\mbox{ИСПАР}--РСС) были реализованы полномасштабные и\linebreak пилотные 
проекты для ряда ПО: криминалистики, управления 
кадрами, мониторинга финансово-экономического кризиса и 
др.~\cite{18koz, 19koz}.

\section{Применение аппарата расширенных семантических сетей в~лингвистических 
исследованиях}
     
     В настоящее время в рамках проектов, на\-прав\-лен\-ных на создание 
открытых лингвистических ресурсов~\cite{20koz} для 
     на\-уч\-но-прак\-ти\-че\-ских целей, ведутся работы по выравниванию 
параллельных текстов научных статей, патентов и 
     фи\-нан\-со\-во-эко\-но\-ми\-че\-ских текстов. В~качестве одного из 
методов выравнивания используется РСС-под\-ход, поскольку он позволяет 
отразить глу\-бин\-но-се\-ман\-ти\-че\-ский уровень языковых структур. 

На  рис.~7 представлен фрагмент первого этапа лингвистического 
анализа в многоязычных системах. Для <<идеальной>> ситуации, когда 
структуры исходного текста и текста перевода практически совпадают, такая 
ситуация имеет место в меньшинстве случаев. Основные трудности 
возникают при наличии переводческих трансформаций в параллельных 
текстах. Особое внимание следует уделять гла\-голь\-но-имен\-ным 
трансформациям, например явлению \textit{номинализации}, поскольку она 
очень продуктивна для всех исследовавшихся языков.

     
     Ключевой задачей при разработке методов сопоставления 
параллельных текстов является выявление и детальное описание тех 
языковых трансформаций, которые имеют место при переводе 
     естест\-вен\-но-язы\-ко\-вых конструкций с одного языка на 
другой~\cite{9koz}, потому что далеко не всегда некое содержание 
передается струк\-тур\-но-по\-доб\-ны\-ми средствами в текстах на разных 
языках. Сравнительное исследование употребления различных частей речи в 
параллельных текстах на разных языках создает основу для выявления и 
описания языковых транс-\linebreak

\begin{center} %fig7
\vspace*{3pt}
\mbox{%
\epsfxsize=79.726mm
\epsfbox{koz-7.eps}
}
\end{center}
\vspace*{4pt}
%\begin{center}
{{\figurename~7}\ \ \small{Первый этап анализа параллельных текстов ($W_n$
обозначает словоформу с номером~$n$, $1\leq n\geq 5$)}}
%\end{center}
%\vspace*{9pt}

%\bigskip
\addtocounter{figure}{1}
      

\noindent 
формаций, при этом центральной трансформацией
является \textit{номинализация}. Явление номинализации
было исследовано в 
ряде работ отечественных и зарубежных лингвистов~[17--20]. 
Ближе всего к правильному, по мнению авторов данной статьи, 
пониманию этого явления следующие определения номинализации: 
<<конструкции\ldots называются номинализованными~--- в том смысле, что 
их естественно рассматривать как результат номинализации конструкций с 
предикативным употреблением глаголов и прилагательных>>; 
<<номинализация~--- это синтаксический процесс, который соотносит 
предложения с именными группами>>~\cite{9koz, 10koz}. Выявление 
номинализованных конструкций в параллельных научных и патентных 
текстах на русском, английском, французском и немецком языках в научных 
и патентных текстах и сопоставительное описание гла\-голь\-но-имен\-ных 
межъязыковых трансформаций~--- одна из центральных задач 
     ин\-же\-нер\-но-линг\-ви\-сти\-че\-ских исследований. 
     
     Следующей базовой трансформацией в исследуемых текстах на 
нескольких европейских языках является адъек\-тив\-но-ад\-вер\-би\-аль\-ное 
преобразование. Это означает, что при переводе с одного языка на другой 
происходит синтаксическое преобразование имен прилагательных в наречия 
и обратное преобразование~--- наречий в прилагательные. Установление 
семантических соответствий между этими языковыми объектами также 
возможно осуществить посредством аппарата~РСС. 
     
     При семантическом выравнивании непараллельных текстов, имеющих 
одну и ту же денотативную составляющую, аппарат РСС позволяет выявить в 
текстах когнитивные опоры (слова с сильной валентностью~--- 
     <<сло\-ва-дейст\-вия>> и <<сло\-ва-от\-но\-ше\-ния>>) и установить 
между ними семантические соответствия.

\section{Заключение}

     В данной работе представлен опыт создания и развития 
     когни\-тив\-но-линг\-ви\-сти\-че\-ских пред\-став\-ле\-ний в 
интеллектуальных информационных сис\-те\-мах, разработанных на основе 
аппарата РСС. Аппарат РСС 
обеспечивает мощные изобразительные возможности для описания всех 
уровней естественного языка, включая уровень 
     глу\-бин\-но-се\-ман\-ти\-че\-ских представлений и межъязыковых 
соответствий. Конкретные лингвистические процессоры, которые были 
созданы на основе этого подхода, прошли определенный путь развития и 
позволили выработать проектные решения для основных задач текущего 
этапа~--- извлечения и обработки содержательных знаний из текстов на 
естественных языках и сопоставления языковых структур в текстах на 
различных языках с учетом базовых трансформаций.
     
     Проблема извлечения и обработки знаний открывает перспективы 
развития интеллектуальных направлений компьютерной лингвистики, 
поскольку ее основной акцент смещен в сторону\linebreak глубинных представлений 
языка, в которых используются как грамматические (морфологические и 
синтаксические), так и семантические атрибуты для описания языковых 
объектов. Проводи-\linebreak мые авторами исследования параллельных текстов 
направлены также на рассмотрение этой проблемы~\cite{20koz}. 
Центральное место в проводящихся линг\-ви\-сти\-че\-ских исследованиях 
занимает изучение и формализация процессов трансформации языковых 
структур, особенно все варианты глагольно-но\-ми\-на\-тив\-ных трансформаций, 
создание развитых дис\-три\-бу\-тив\-но-транс\-фор\-ма\-ци\-он\-ных 
описаний предикатых структур для рассматриваемых языков. 
     
     Для задач извлечения знаний и создания \mbox{ИСПАР} 
     дис\-три\-бу\-тив\-но-транс\-фор\-ма\-ци\-он\-ные описания имеют 
особое значение, поскольку таким образом задаются все возможные способы 
перевода языковых структур в пре\-ди\-кат\-но-ар\-гу\-мент\-ные 
представления, которые затем используются в процедурах обработки знаний.

{\small\frenchspacing
{%\baselineskip=10.8pt
%\addcontentsline{toc}{section}{Литература}
\begin{thebibliography}{99}

     \bibitem{1koz}
     \Au{Кузнецов~И.\,П.}
     Семантические представления.~--- М.: Наука, 1986. 290~с.
     
     \bibitem{2koz}
     \Au{Козеренко~Е.\,Б.}
     Кон\-цеп\-ту\-аль\-но-линг\-вис\-ти\-че\-ское моделирование в среде 
интеллектуального редактора знаний ИКС~// Проблемы проектирования и 
использования баз знаний.~--- Киев: Ин-т кибернетики им.\ В.\,М.~Глушкова, 
1992. C.~73--79.
     
     \bibitem{3koz}
     \Au{Kozerenko~E.\,B.}
     Multilingual processors: A unified approach to semantic and syntactic 
knowledge presentation~// Conference (International ) on Artificial Intelligence 
IC-AI'2001 Proceedings. Las Vegas, Nevada, USA. June 25--28, 2001.~--- Las 
Vegas: CSREA Press, 2001. P.~1277--1282.

     \bibitem{18koz} %4
     \Au{Kuznetsov~I.\,P., Efimov~D.\,A., Kozerenko~E.\,B.}
     Tools for tuning the semantic processor to application areas~// ICAI'09 
Proceedings, WORLDCOMP'09. July 13--16, 2009. Las Vegas, Nevada, USA. 
Vol.~I.~--- Las Vegas: CRSEA Press, 2009. P.~467--472.
     
     \bibitem{19koz} %5
     \Au{Kuznetsov~I.\,P., Kozerenko~E.\,B., Kuznetsov~K.\,I., 
Timonina~N.\,O.}
     Intelligent system for entities extraction (ISEE) from natural language 
texts~// Workshop (International) on Conceptual Structures for Extracting Natural 
Language Semantics (Sense'09) at the 17th Conference 
(International ) on Conceptual Structures (ICCS'09) Proceedings. University Higher School of 
Economics. Moscow, Russia, 2009. P.~17--25.
     
     \bibitem{4koz} %6
     \Au{Апресян~Ю.\,Д.}
     Экспериментальное исследование семантики русского глагола.~--- М.: 
Наука, 1967.  252~с.
     
     \bibitem{5koz} %7
     \Au{Филлмор~Ч.}
     Дело о падеже~// Новое в зарубежной линг\-вистике, 1968. Вып.~X. С.~369--495.
     
     \bibitem{6koz} %8
     \Au{Хомский~Н.}
     Аспекты теории синтаксиса.~--- М.: МГУ, 1972.
     
     \bibitem{7koz} %9
     \Au{Хомский Н.}
     Язык и мышление.~--- М.: МГУ, 1972.
     
     
     \bibitem{8koz} %10
     \Au{Fillmore~C.}
     The case for case reopened~// Syntax and Semantics. Vol.~8.~--- N.Y.: 
Academic Press, 1977. 
     

          \bibitem{15koz} %11
     FASTUS: A cascaded finite-state trasducer for extracting information from 
natural-language text~// AIC, SRI International, Menlo Park, California, 1996. 
     
     \bibitem{16koz} %12
     \Au{Han~J., Pei~Y., Mao~R.}
     Mining frequent patterns without candidate generation: A frequent-pattern 
tree approach~// Data Mining and Knowledge Discovery, 2004. Vol.~8. No.\,1. 
P.~53--87.
     
     
     \bibitem{13koz} %13
     \Au{Cunningham~H.}
     Automatic information extraction~// Encyclopedia of Language and 
Linguistics. 2nd ed.~--- Elsevier, 2005.
     
     \bibitem{14koz} %14
     \Au{Han~J., Kamber~M.}
     Data mining: Concepts and techniques.~--- Morgan Kaufmann, 2006.
     
     
     \bibitem{17koz} %15
     \Au{Добров~Б.\,В., Лукашевич~Н.\,В.}
     Онтологии для автоматической обработки текстов: Описание понятий 
и лексических значений~// Компьютерная лингвистика и интеллектуальные 
технологии: Тр. межд. конф. <<Диалог'06>>. Бекасово, 31~мая\,--\,4~июня 
2006. С.~138--142.

     \bibitem{20koz} %16
     \Au{Kozerenko~E.\,B.}
     INTERTEXT: A multilingual knowledge base for machine translation~// 
Conference (International) on Machine Learning, Models, Technologies and 
Applications Proceedings. June 25--28, 2007. Las Vegas, USA.~--- Las Vegas: 
CSREA Press, 2007. P.~238--243.

     \bibitem{9koz} %17
     \Au{Жолковский~А.\,К., Мельчук~И.\,А.}
     О семантическом синтезе~// Проблемы кибернетики, 1967. Вып.~19.
     
         
     \bibitem{11koz} %18
     \Au{Jacobs~R.\,A., Rosenbaum P.\,S.}
     English transformational grammar.~--- Blaisdell, 1968.
     

\label{end\stat}
     
          \bibitem{12koz} %19
     \Au{Балли~Ш.}
     Общая лингвистика и вопросы французского языка. 2-е изд.~--- М.: 
УРСС, 2001.

\bibitem{10koz} %20
     \Au{Падучева~Е.\,В.}
     О~семантике синтаксиса: Мат-лы к трансформационной 
грамматике русского языка. 2-е изд.~--- М: КомКнига, 2007.  296~с. 
     
 \end{thebibliography}
}
}


\end{multicols} %10

\def\stat{zatsman}

\def\tit{ТРАНСФОРМАЦИИ ОБЪЕКТОВ ПЕРВОГО И~ВТОРОГО ПОРЯДКА 
В~ЛЕКСИКОГРАФИЧЕСКОЙ ИНФОРМАЦИОННОЙ СИСТЕМЕ$^*$}

\def\titkol{Трансформации объектов первого и~второго порядка 
в~лексикографической информационной системе}

\def\aut{И.\,М.~Зацман$^1$}

\def\autkol{И.\,М.~Зацман}

\titel{\tit}{\aut}{\autkol}{\titkol}

\index{Зацман И.\,М.}
\index{Zatsman I.\,M.}


{\renewcommand{\thefootnote}{\fnsymbol{footnote}} \footnotetext[1]
{Исследование выполнено в~ФИЦ ИУ РАН за счет гранта Российского научного фонда №\,24-18-00155, {\sf 
https://rscf.ru/project/24-18-00155}. Работа выполнялась с~использованием инфраструктуры Центра 
коллективного пользования <<Высокопроизводительные вычисления и~большие данные>> (ЦКП 
<<Информатика>>) ФИЦ ИУ РАН (г.\ Москва).}}


\renewcommand{\thefootnote}{\arabic{footnote}}
\footnotetext[1]{ Федеральный исследовательский центр <<Информатика и~управление>> Российской академии наук, 
\mbox{izatsman@yandex.ru}}

\vspace*{-12pt}


  
  \Abst{Рассматриваются теоретические основания проектирования информационных 
технологий (ИТ) интеграции двуязычных словарей и~параллельных корпусов. Дано описание 
первых результатов создания третьего уровня классификации трансформаций объектов 
предметной области информатики, которую предполагается использовать при создании 
концепции лексикографической информационной системы, обеспечивающей интеграцию. 
Все сущности информатики в~статье разделены на два глобальных класса: объекты и~их 
трансформации. Для каждого такого класса конструируется своя классификация. Ранее были 
описаны два верхних уровня классификации трансформаций объектов предметной области. 
В~данной статье рассматривается третий уровень этой классификации. Основанием для 
построения самого верхнего ее уровня служило деление предметной области информатики 
на среды (ментальная, сенсорно воспринимаемая, цифровая и~ряд других сред), каждая из 
которых по определению включает объекты одной природы. Основанием для построения 
второго уровня классификации трансформаций объектов служила типология знаковых  
сис\-тем А.~Соломоника. Цель статьи состоит в~систематизации трансформаций первого 
и~второго порядка объектов предметной области на третьем уровне этой классификации. 
Основанием для систематизации служит средовая версия иерархии Акоффа.}
  
  \KW{объекты предметной области; трансформации объектов; классификация; данные; 
информация; знание; лексикографическая информационная сис\-тема}

\DOI{10.14357/19922264240211}{VZTGVV}
  
\vspace*{3pt}


\vskip 10pt plus 9pt minus 6pt

\thispagestyle{headings}

\begin{multicols}{2}

\label{st\stat}
  
\section{Введение}

\vspace*{-9pt}

  Возникновение параллельных корпусов, в~которых предложениям 
оригинального текста со\-по\-став\-ле\-ны предложения его перевода, обеспечило 
возможность контрастивного лингвистического\linebreak \mbox{анализа} на принципиально 
новом уровне полноты и~точности, недостижимом в~докорпусную эпоху. 
Пионерскими в~этой области стали работы \mbox{1990-х~гг}. Стига Йоханссона  
с~анг\-ло-нор\-веж\-ским корпусом~[1]. В России параллельные корпусы стали 
формироваться в~начале XXI~века в~рамках Национального корпуса русского 
языка~[2].
  
  Создатели двуязычных словарей используют параллельные корпусы для 
сбора материала и~эмпирической проверки своих гипотез, касающихся 
межъязы\-ко\-вой эквивалентности. Ценность параллельных корпусов 
определяется тем, что в~лингвистике этап сбора исходного материала считается 
наиболее трудоемким и~наименее творческим, а~параллельные корпусы 
позволяют значительно сэкономить время и~силы для творческого этапа 
создания словарей~[3].
 % 
  При этом двуязычные словари, создаваемые на основе исходного материала, 
извлеченного из параллельных корпусов, сейчас формируются без связей с~их 
текстами. Другими словами, онлайновые связи созданных словарей 
с~параллельными корпусами, которые служили источниками исходного 
материала, отсутствуют. 

Параллельные корпусы постоянно пополняются 
новыми текстами, в~предложениях которых можно обнаружить новые значения 
слов и~устойчивых словосочетаний. Однако при этом отсутствуют методы 
и~средства оперативного обновления словарей по корпусным данным. 
В~настоящее время проблема установления связей между двуязычными 
словарями и~параллельными корпусами (далее~--- проблема интеграции) 
находится на стадии поиска концептуальных подходов к~их интеграции на 
уровне значений.
  
  Подход к~решению проблемы интеграции, предлагаемый в~статье, учитывает 
  и~появление новых значений слов и~устойчивых словосочетаний, и~динамику 
смысловых значений, которая обусловлена развитием и~пополнением знания 
лингвистов, фиксирующих эти значения в~результате семантического анализа 
пополняемых корпусных данных. Проведенные эксперименты показали, что 
обнаружение нового лингвистического знания обусловливает и~формирование 
дефиниций новых значений, и~пересмотр уже существующих дефиниций~[4, 5].
  
  Например, в~проведенных экспериментах с~использованием ЦКП 
<<Информатика>> ФИЦ ИУ РАН фиксировалась эволюция значений немецких 
модальных глаголов, исходное состояние значений которых было описано 
в~не\-мец\-ко-рус\-ском словаре. В~экспериментальном массиве текстов как 
потенциальных источниках нового знания 16\,268 предложений содержали 
немецкие модальные глаголы и~в~2041 из них встречался глагол sollen. 
В~начале эксперимента в~словаре были описаны~12~значений этого модального 
глагола. По окончании эксперимента лингвисты обнаружили два новых его 
значения, согласовали их дефиниции и~описали эволюцию дефиниций~[6, 7].
  
  Таким образом, для решения проблемы интеграции требуется фиксировать 
новое знание, обнаруженное лингвистами в~текстовых данных параллельных 
корпусов, отслеживать эволюцию знания, представленного в~виде дефиниций 
значений слов и~устойчивых словосочетаний, и,~соответственно, 
актуализировать электронные двуязычные словари. Предлагаемый 
концептуальный подход к~интеграции, который планируется реализовать 
в~процессе проектирования лексикографической информационной сис\-те\-мы, 
фиксирующей эволюцию лингвистического знания, основан на решении 
следующих задач:\\[-14pt]
  \begin{itemize}
  \item категоризация трех базовых понятий информатики, включенных 
  в~иерархию Акоффа~[8] (данные, информация, знание), на объекты 
проектируемой сис\-те\-мы, которая необходима, чтобы фиксировать 
<<кванты>> нового знания и~отслеживать его эволюцию в~этой сис\-теме;\\[-15pt]
  \item  систематизация трансформаций объектов этой сис\-темы.\\[-14pt]
  \end{itemize}
  
  Цель статьи и~состоит в~решении двух задач: категоризации трех базовых 
понятий информатики на объекты лексикографической информационной  
сис\-те\-мы и~сис\-те\-ма\-ти\-за\-ции трансформаций первого и~второго порядка 
ее объектов.
  
  Трансформациями первого порядка, о которых сказано в~формулировке цели 
статьи, называются взаимные преобразования между двумя объектами  
сис\-те\-мы одной природы. Например, перевод в~сис\-те\-ме текста с~русского 
языка на английский относится к~ним. Трансформациями второго порядка 
и~выше называются взаимные преобразования между двумя и~более объектами 
разной природы. Например, кодирование символов текс\-та компьютерными 
кодами и~их декодирование относятся по определению к~трансформациям 
второго порядка.

%\vspace*{-9pt}
  
\section{Процессы трансформаций в~информатике}

%\vspace*{-3pt}

Процессы трансформаций, рассматриваемые в~статье, относятся к~теоретическому ядру информатики, а~не 
только к~проектированию лексикографической информационной сис\-те\-мы. Например, из трех основных 
подходов к~описанию предметной об\-ласти информатики\footnote{В статье предметная область информатики 
трактуется согласно концепции полиадического компьютинга Пола Розенблума~\cite{9-zac}.} (объектный, 
трансформационный и~синтетический) сис\-те\-ма\-ти\-за\-ция трансформаций ближе всего ко второму 
подходу. Примерами первого подхода, в~рамках которого основное внимание уделяется объектам предметной 
области информатики и~в~меньшей степени отношениям\linebreak между ними, могут служить  
работы~\cite{8-zac, 10-zac, 11-zac}; \mbox{примерами} второго подхода, в~рамках которого основное внимание 
уделяется трансформациям и~в~меньшей степени трансформируемым объектам,~---  
работы~\cite{12-zac, 13-zac}; примерами третьего, синтетического подхода, в~котором уделяется внимание 
и~объектам предметной об\-ласти информатики, и~отношениям между ними, могут служить работы~\cite{14-zac, 
15-zac, 16-zac, 17-zac, 18-zac}.

  Таким образом, для описания трансформаций объектов лексикографической 
информационной\linebreak системы предпочтительнее всего трансформационный 
подход, который упоминается и~в определениях информатики. Например, 
в~2009~г.\ П.~Деннинг и~П.~Розенблум сформулировали суть \mbox{информатики} как 
компьютинга следующим образом: <<$\ldots$информатика~--- это не просто 
алгоритмы и~структуры данных; это преобразования [трансформации] 
представлений>>~\cite{12-zac}. Чуть позже, в~контексте краткого описания 
парадигмы информатики как компьютинга, П.~Деннинг и~П.~Фриман изменили 
эту формулировку на такую: <<Центральный объект внимания в~информатике 
можно определить как информационные процессы~--- \textit{естественные или 
искусственные процессы, преобразующие информацию} (курсив мой~--- 
И.\,З.)>>~\cite{13-zac}. Согласно парадигме, предлагаемой авторами этой 
статьи, на начальном этапе проектирования автоматизированных систем 
базовыми элементами моделей их функционирования служат 
\textit{информационные про\-цессы}.
  
  Однако если 15~лет назад в~формулировке из работы~\cite{13-zac} шла речь 
о~процессах, преобразующих информацию, то в~последние~10~лет в~спектр 
процессов трансформаций все чаще стали включать процессы, преобразующие 
не только информацию, но также и~другие объекты автоматизированных 
систем, в~первую очередь данные и~знания~[19--21]. Например, Виктория 
Стодден, позиционируя науку о~данных как одну из дисциплин информатики, 
говорит, что центральный объект исследований в~науке о~данных~--- это 
<<изучение обобщаемого извлечения знания из данных>>~\cite{21-zac}. 
Увеличение и~чис\-ла объектов, и~спект\-ра процессов их трансформаций 
в~автоматизированных сис\-те\-мах обуслов\-ли\-ва\-ет не\-об\-хо\-ди\-мость 
систематизации и~объектов, и~процессов их трансформаций на начальном этапе 
проектирования сис\-тем.
  
  Для создания концепции лексикографической информационной сис\-те\-мы 
и~проектирования ИТ, обеспечивающих интеграцию 
двуязычных словарей и~параллельных корпусов, сначала выполним 
категоризацию на объекты этой сис\-те\-мы трех базовых понятий информатики 
(данные, информация, знание) в~контексте построения классификаций 
сущностей ее предметной об\-ласти.
  
  Необходимость использования классификаций информатики в~процессе 
создания концепции проиллюстрируем, используя иерархию  
Акоффа~\cite{8-zac}. Он использовал принцип их вертикального размещения 
в~иерархии снизу вверх: данные, информация и~знание. Еще в~ней есть термин 
<<мудрость>>, который в~статье не рассматривается. Такое размещение Акофф 
прокомментировал так: <<Каждое из пе\-ре\-чис\-лен\-ных понятий [кроме данных] 
содержит в~себе нижестоящие$\ldots$>>~\cite{8-zac}.
  
  Этому принципу размещения и~комментарию Акоффа свойственны 
недостатки, проанализированные, в~частности, в~работе~\cite{10-zac}. Главный 
вывод, к~которому пришла Роули после изучения иерархии Акоффа, 
заключается в~следующем: <<$\ldots$информация определяется в~терминах 
данных, знание~--- в~терминах информации$\ldots$ но существует меньше 
консенсуса в~описании трансформаций, которые преобразуют сущности, 
расположенные ниже в~иерархии, в~те, которые находятся над ними, что 
приводит к~их терминологической неопределенности>>~\cite{10-zac}. Причина 
этой неопределенности, скорее всего, в~том, что базовые понятия информатики 
включены в~иерархию Акоффа изолированно от общего контекста 
классификаций сущностей ее предметной об\-ласти.

%\vspace*{-9pt}
  
\section{Классификации сущностей информатики}


%\vspace*{-2pt}

  Все сущности предметной области информатики в~работах~[22, 23] 
разделены на два глобальных класса: ее объекты и~их трансформации. Для 
каждого такого класса была предложена своя классификация. 
В~работе~\cite{22-zac} дано описание классификации объектов предметной 
области информатики, первый уровень которой содержит базовые понятия ее 
предметной области (данные, информация, знания и~др.).  
В~работе~\cite{23-zac} дано описание двух верхних уровней классификации 
трансформаций объектов предметной об\-ласти (см.\ рисунок 
в~работе~\cite{23-zac}). Основанием для построения самого верхнего ее уровня послужило деление 
предметной области информатики на среды\footnote{В~работе~\cite{24-zac} дано описание пяти сред 
предметной области информатики (ментальная; сенсорно воспринимаемая, или информационная; 
цифровая; нейро- и~ДНК-среда), каждая из которых по определению включает объекты одной и~той же 
природы.} и~степень разнообразия природы объектов, вовлеченных в~трансформации:
\begin{itemize}
\item  первый класс верхнего уровня классификации включает 
трансформации объектов в~пределах среды только одной природы 
(трансформации первого порядка);
\item  второй класс включает трансформации объектов, относящихся 
к~двум средам разной природы (трансформации второго порядка);
\item третий и~последующие классы включают трансформации объектов, 
относящихся к~трем и~более средам разной природы (трансформации 
третьего и~более высоких порядков).
\end{itemize}

  В работе~\cite{23-zac} были приведены примеры для трех первых классов 
трансформаций, включая пример трансформаций объектов, относящихся 
к~двум средам разной природы (компьютерное кодирование символов текстов 
с~по\-мощью таб\-лиц Unicode).
  
Основанием для построения второго уровня классификации трансформаций объектов послужила типология 
знаковых сис\-тем А.~Соломоника~\cite[c.~131]{25-zac}: естественные знаковые сис\-те\-мы, образные,  
ес\-тест\-вен\-но-язы\-ко\-в$\acute{\mbox{ы}}$е,  
вер\-баль\-но-не\-сло\-вес\-ные сис\-те\-мы записи\footnote{Под системой записи понимается знаковая 
система, сочетающая вербальные знаки с~несловесными (языки нотной записи, карт, таблиц и~др.).} 
и~формализованные знаковые сис\-те\-мы, включая математические. Введем понятие обобщенного текста~--- 
это текст, который может быть создан в~любой из перечисленных знаковых систем. Тогда обобщенные тексты 
могут быть естественными, образными, ес\-тест\-вен\-но-язы\-ко\-в$\acute{\mbox{ы}}$\-ми,  
вер\-баль\-но-не\-сло\-вес\-ны\-ми и~формализованными. Второй уровень классификации трансформаций 
охватывает не все виды объектов предметной  
об\-ласти информатики, а~только перечисленные~5~видов текс\-тов и~их представления, вовлеченные 
в~процессы трансформаций в~одной или более средах вместе с~данными, знанием и~его концептами.

\begin{figure*}[b] %fig1
\vspace*{6pt}
      \begin{center}
     \mbox{%
\epsfxsize=121.191mm 
\epsfbox{zac-1.eps}
}
\end{center}
\vspace*{-6pt}
\Caption{Средовая версия иерархии Акоффа}
\end{figure*}

\section{Классификация трансформаций: построение~третьего 
уровня}

  Основанием для систематизации трансформаций первого и~второго порядка 
на третьем уровне этой классификации служит иерархия Акоффа~\cite{8-zac}, 
на основе которой и~была создана ее средов$\acute{\mbox{а}}$я версия~[26, 
27]. Для создания средов$\acute{\mbox{о}}$й версии была выполнена 
категоризация трех базовых понятий информатики (данные, информация, 
знания) на объекты лексикографической информационной сис\-те\-мы 
в~процессе создания ее концепции\linebreak (рис.~1).
  


  В отличие от классической иерархии Акоффа, в~ее 
средов$\acute{\mbox{о}}$й версии различаются три вида данных: сенсорно 
воспринимаемые, цифровые и~те данные, которые генерируются 
искусственными нейронными сетями (ИНС) в~системах искусственного интеллекта 
(далее~--- ИИ-дан\-ные). Последний вид данных необходим, например, для 
различения входа и~выхода процесса применения обученной 
ИНС в~цифровой модели генерации знания, описанию которой 
посвящена работа~\cite{27-zac}.
  
  Также предлагается различать два вида информации: сенсорно 
воспринимаемая и~цифровая. Кроме знания в~средов$\acute{\mbox{у}}$ю 
версию добавлены концепты и~ментальные образы сенсорно воспринимаемых 
данных. Последние служат промежуточной сущностью между сенсорно 
воспринимаемыми данными и~генерируемым знанием при описании процессов 
извлечения знания из текстовых данных лексикографической информационной 
системы. Описание объектов средов$\acute{\mbox{о}}$й версии иерархии 
Акоффа (см.\ рис.~1) и~отношений между ними дано в~работах~\cite{26-zac, 28-zac}.
  
  В средов$\acute{\mbox{о}}$й версии число объектов равно восьми. Если 
учитывать направления трансформаций, то между восемью объектами на 
рис.~1 она включает~16 их видов (трансформации на границе между сенсорно 
воспринимаемыми данными и~информацией, обозначенные символом~<<?>>, 
в~статье не рас\-смат\-ри\-ва\-ют\-ся). В~будущем число объектов 
в~средов$\acute{\mbox{о}}$й версии, которая выбрана как основание для 
сис\-те\-ма\-ти\-за\-ции трансформаций первого и~второго порядка, может быть 
увеличено. Для построения классификации трансформаций 
важ\-но не возможное увеличение числа объектов 
и~трансформаций между ними, а то, что их виды в~средов$\acute{\mbox{о}}$й 
версии распределены между трансформациями первого и~второго порядка. Из 
16~видов на рис.~1 шесть относятся к~трансформациям первого порядка, это\linebreak 
виды с~номерами~7, 8, 13--16 (далее~--- типология трансформаций первого 
порядка), а~десять~--- к~трансформациям второго порядка, это виды 
с~\mbox{номерами}~1--6 и~9--12 (далее~--- типология трансформаций второго 
порядка). Разместим обе типологии на третьем уровне классификации (см.\ ее 
схему на рис.~2). Перечислим виды трансформаций первой типологии, вводя 
в~скобках их краткие названия, используемые ниже на рис.~3:
  \begin{description}
  \item[\,] 7~--- членение знания на концепты с~помощью одной или нескольких 
знаковых систем (далее~--- членение знания);
  \item[\,] 8~--- формирование знания на основе концептов (формирование 
знания);
  \item[\,] 13~--- обучение ИНС;
  \end{description}
  
  \vspace*{-6pt}
  
  \pagebreak
  
  \end{multicols}
  
  \begin{figure*} %fig2
\vspace*{1pt}
      \begin{center}
     \mbox{%
\epsfxsize=127.513mm 
\epsfbox{zac-2.eps}
}
\end{center}
\vspace*{-9pt}
\Caption{Схема трех верхних уровней классификации трансформаций объектов (объединены 
по три слоя и~для второго, и~для третьего уровней этой классификации)}
\end{figure*}
  
  \begin{multicols}{2}
  
  \noindent
  \begin{description}
  \item[\,] 14~--- восстановление обучающей информации на основе 
содержания обученной ИНС (обращение ИНС);
  \item[\,] 15~--- использование обученной ИНС (использование ИНС);



  \item[\,] 16~--- восстановление исходных данных, соответствующих 
полученным результатам работы обучен\-ной ИНС (восстановление исходных данных 
по результатам ИНС).
  \end{description}
  
  
  Не все виды трансформаций 13--16 поддерживаются в~конкретных системах 
искусственного интеллекта, но с~теоретической точки зрения все их 
предлагается включить в~первую типологию для полноты спектра видов 
трансформаций.
  
  Перечислим виды трансформаций второй типологии:
  \begin{description}
  \item[\,] 1~--- декодирование цифровых данных в~компьютерных системах 
(декодирование данных);
  \item[\,]  2~--- кодирование сенсорно воспринимаемых данных (кодирование 
данных);
  \item[\,] 3~--- ментальное копирование сенсорно воспринимаемых данных 
(ментальное копирование);
  \item[\,] 4~--- восстановление сенсорно воспринимаемых данных по 
ментальным образам (восстановление по образам);
  \item[\,] 5~--- смысловая интерпретация без деления на концепты ментальных 
образов сенсорно воспринимаемых данных (смысловая интерпретация);
  \item[\,] 6~--- восстановление ментальных образов (восстановление образов);
  \item[\,] 9~--- представление концептов в~виде сенсорно воспринимаемой 
информации, например текс\-та\-ми, формулами, таблицами, рисунками и~т.\,д.\ 
(представление концептов);
  \item[\,] 10~--- понимание смысла сенсорно воспринимаемой информации 
(понимание смысла);
  \item[\,] 11~--- кодирование сенсорно воспринимаемой информации 
(кодирование информации);
\end{description}

\vspace*{-6pt}

\pagebreak

\end{multicols}

\begin{figure*} %fig3
\vspace*{1pt}
      \begin{center}
     \mbox{%
\epsfxsize=163mm 
\epsfbox{zac-3.eps}
}
\end{center}
\vspace*{-9pt}
\Caption{Схема частного случая классификации трансформаций объектов (трансформации 
пронумерованы согласно рис.~1)}
\end{figure*}

\begin{multicols}{2}

\noindent
\begin{description}

  \item[\,] 12~--- декодирование цифровой информации (декодирование 
информации).
  \end{description}
  
  Отметим, что в~существующих ИТ
  и~компьютерных системах наиболее часто используются виды 
трансформаций~13 и~15 типологии первого порядка и~1, 2, 11 и~12 типологии 
второго порядка. На рис.~2 в~первом слое третьего уровня классификации 
показаны типологии первого порядка без указания числа трансформаций в~них 
и~без детализации трансформируемых объектов.
  
  Во втором слое третьего уровня классификации условно (без названий) 
показаны типологии второго порядка. Также на рис.~2 в~третьем слое третьего 
уровня классификации условно (также без названий) показаны типологии 
третьего порядка, которые планируется рассмотреть в~отдельной статье. По 
определению они должны включать трансформации между тремя объектами 
разной природы, но средов$\acute{\mbox{а}}$я версия иерархии Акоффа 
включает трансформации только между двумя объектами разной природы. 
Поэтому потребуется другое основание для их систематизации (ранее были 
рассмотрены отдельные примеры трансформаций третьего 
порядка\footnote{Далеко не всегда трансформации третьего и~более высоких порядков можно 
рассматривать как последовательность трансформаций второго порядка. Примером этого могут 
служить трансформации в~процессе обучения пациента пользованию роботизированной рукой, 
охватывающие личностные концепты пациента, релевантные его намерениям, сигналы активности 
мозга как объекты нейросреды и~компьютерные коды~\cite{29-zac}.}~\cite{29-zac}).

\section{Классификация трансформаций: частный~случай}

  Выше было отмечено, что в~будущем число объектов 
в~средов$\acute{\mbox{о}}$й версии иерархии Акоффа может быть увеличено. 
Это означает, что увеличатся и~чис\-ло объектов, и~чис\-ло трансформаций между 
ними в~классификации трансформаций, так как эта средов$\acute{\mbox{а}}$я 
версия служит по определению основанием для систематизации 
трансформаций первого и~второго порядка. Поэтому на третьем уровне рис.~2 
указаны типологии без детализации объектов и~без указания числа 
трансформаций в~каждой из них. С~одной стороны, при таком подходе 
получаем достаточно общий вид этой классификации, так как она не зависит от 
числа объектов в~том или ином варианте средов$\acute{\mbox{о}}$й версии 
(и~это существенно упрощает рис.~2). С~другой стороны, на третьем уровне 
такой общей классификации подразумевается, но не эксплицируется природа 
трансформируемых объектов и~их возможные сочетания в~трансформациях. 

При проектировании лексикографической информационной системы важно 
эксплицировать природу трансформируемых объектов и~их возможные 
сочетания.
  %
  Поэтому в~парадигму информатики~\cite{30-zac} кроме общей 
классификации трансформаций предлагается включать и~ее частные случаи, 
эксплицирующие природу трансформируемых объектов. 

В~этом разделе 
рассмотрим один частный случай, когда используются только естественные 
знаковые сис\-те\-мы из типологии А.~Соломоника~\cite{25-zac} вместе 
с~данными, знанием и~его концептами. Чис\-ло естественных языков при этом не 
ограничено. И~этот частный случай классификации включает только три 
класса природных трансформаций (первого, второго и~третьего порядка, см.\ 
схему классификации на рис.~3).
  
  Первый и~второй уровни схемы общей классификации (см.\ рис.~2) можно 
объединить в~один уровень в~этом частном случае. Ниже этого уровня 
приведено содержание типологий первого и~второго порядка без содержания 
типологий третьего по\-рядка.




  Наполнение типологий первого и~второго порядка соответствует 
средов$\acute{\mbox{о}}$й версии иерархии Акоффа на рис.~1, содержащей 
6~видов трансформаций типологии первого порядка и~10~видов 
трансформаций типологии второго порядка (на рис.~3 стрелки указывают 
направления трансформаций согласно средов$\acute{\mbox{о}}$й версии на рис.~1).
  
  Таким образом, частный случай классификации содержит для этих двух 
типологий 16~теоретически возможных трансформаций, 6 из которых 
в~настоящее время в~существующих ИТ применяются наиболее часто: виды 
трансформаций~1, 2, 11 и~12 типологии второго порядка реализуются 
с~помощью тех или иных методов ко\-ди\-ро\-ва\-ния/де\-ко\-ди\-ро\-ва\-ния 
(например, с~использованием таблиц Unicode), а~виды трансформаций~13 и~15
 в~типологии первого порядка реализуются полностью с~по\-мощью процессов 
цифровой обработки компьютерами.
  
  Остальные виды трансформаций или применяются намного реже (это 
виды~3, 5, 7, 9 и~10), или находятся в~стадии поиска и~разработки (14 и~16) или 
в~настоящее время носят только теоретический характер, обеспечивая полноту 
первой и~второй типологий (4, 6 и~8). Знаком~<<?>> обозначены те виды 
трансформаций, которые по определению не существуют в~используемой 
парадигме информатики~\cite{30-zac}. Однако возможно, что в~других 
будущих подходах к~построению ее парадигмы эти виды трансформаций будут 
существовать.
  
\section{Заключение}

  На сегодняшний день процесс построения классификаций объектов 
предметной области информатики~\cite{22-zac} и~их  
трансформаций~\cite{23-zac} еще не завершен. Однако первые результаты их 
построения уже используются для создания концепции лексикографической 
информационной сис\-те\-мы, обеспечивающей интеграцию двуязычных 
словарей и~параллельных корпусов.
  
  \bigskip
  
  
  Автор признателен рецензентам за помощь в~улучшении статьи.
  
{\small\frenchspacing
 { %\baselineskip=10.6pt
 %\addcontentsline{toc}{section}{References}
 \begin{thebibliography}{99}
\bibitem{1-zac}
\Au{Aijmer K., Altenberg~B.} Advances in corpus-based contrastive linguistics. Studies in honour 
of Stig Johansson.~--- Amsterdam: John Benjamins, 2013. 295~p.  doi: 10.1075/scl.54.
\bibitem{2-zac}
\Au{Добровольский Д.\,О., Кретов~А.\, А., Шаров~С.\,А.} Корпус параллельных текстов~// 
Научная и~техническая информация. Сер.~2: Информационные процессы и~сис\-те\-мы, 2005. 
№\,6. С.~16--27.
\bibitem{3-zac}
\Au{Добровольский Д.\,О.} Корпус параллельных текстов и~сопоставительная 
лексикология~// Труды Института русского языка им.\ В.\,В.~Виноградова, 2015. №\,6. 
С.~413--449. EDN: VJQBHP.
\bibitem{4-zac}
\Au{Гончаров А.\,А., Зацман~И.\,М., Кружков~М.\,Г.} Эволюция классификаций 
в~надкорпусных базах данных~// Информатика и~её применения, 2020. Т.~14. Вып.~4. 
С.~108--116. doi: 10.14357/19922264200415.  
EDN: \mbox{GKWBZT}.
\bibitem{5-zac}
\Au{Гончаров А.\, А., Зацман И. \,М., Кружков~М.\, Г}. Представление новых 
лексикографических знаний в~динамических классификационных сис\-те\-мах~// 
Информатика и~её применения, 2021. Т.~15. Вып.~1. С.~86--93.  doi: 10.14357/19922264210112. EDN: OPEFXW.
\bibitem{6-zac}
\Au{Zatsman I.} Finding and filling lacunas in linguistic typologies~// 15th Forum (International) 
on Knowledge Asset Dynamics Proceedings.~--- Matera, Italy: Institute of Knowledge Asset 
Management, 2020. P.~780--793.
\bibitem{7-zac}
\Au{Zatsman I.} Three-dimensional encoding of emerging meanings in AI-systems~// 21st 
European Conference on Knowledge Management Proceedings.~--- Reading, U.K.: Academic 
Publishing International Ltd., 2020. P.~878--887.
\bibitem{8-zac}
\Au{Ackoff R.} From data to wisdom~// J.~Applied Systems Analysis, 1989. Vol.~16. No.\,1. P.~3--9.
\bibitem{9-zac}
\Au{Rosenbloom P.\,S.} On computing: The fourth great scientific domain.~--- Cambridge, MA, 
USA: MIT Press, 2013. 307~p.
\bibitem{10-zac}
\Au{Rowley J.} The wisdom hierarchy: Representations of the DIKW hierarchy~// J.~Inf. 
Sci., 2007. Vol.~33. Iss.~2. P.~163--180. doi: 10.1177/0165551506070706.
\bibitem{11-zac} 
\Au{Frick$\acute{\mbox{e}}$~M.\,H.} Data--Information--Knowledge--Wisdom (DIKW) pyramid, 
framework, continuum~// Encyclopedia of big data~/ Eds. L.~Schintler, C.~McNeely.~--- Cham: 
Springer, 2018. 4~p. doi: 10.1007/978-3-319-32001-4\_331-1.
\bibitem{12-zac}
\Au{Denning P., Rosenbloom~P.} Computing: The fourth great domain of science~// Commun. 
ACM, 2009. Vol.~52. Iss.~9. P.~27--29.
\bibitem{13-zac}
\Au{Denning P., Freeman~P.} Computing's paradigm~// Commun.  ACM, 2009. Vol.~52. 
Iss.~12. P.~28--30. doi: 10.1145/ 1610252.1610265.
\bibitem{17-zac} %14
\Au{Farradane J.} Knowledge, information, and information science~// J.~Inf. Sci., 
1980. Vol.~2. Iss.~2. P.~75--80. doi: 10.1177/01655515800020020.

\bibitem{15-zac}
\Au{Шрейдер Ю.\,А.} Информация и~знание~// Сис\-тем\-ная концепция информационных 
процессов.~--- М.: ВНИИСИ, 1988. С.~47--52.
\bibitem{16-zac}
\Au{Ingwersen P.} Information and information science~// Enclyclopaedie of library and 
information science~/ Eds. J.\,D.~McDonald, 
M.~Levine-Clark.~--- New York, NY, USA: Marcel Dekker Inc., 1992. Vol.~56. Sup.~19. 
P.~137--174.

\bibitem{14-zac} %17
Информатика как наука об информации: Информационный, документальный, 
технологический, экономический, социальный и~организационный аспекты~/ Под ред. 
Р.\,С.~Гиляревского.~--- М.: Фаир-Пресс, 2006. 592~с.

\bibitem{18-zac}
\Au{Hjorland B.} Library and information science: practice, theory, and philosophical basis~// 
Inform. Process. Manag., 2000. Vol.~36. Iss.~3. P.~501--531. doi:  
10.1016/S0306-\mbox{4573(99)00038-2}.
\bibitem{19-zac}
Deep shift~--- technology tipping points and societal impact.~--- Geneva: WE Forum, 2015. 44~p. 
{\sf http://www3.weforum.org/docs/WEF\_GAC15\_ Technological\_Tipping\_Points\_report\_2015.pdf}.
\bibitem{20-zac}
\Au{Berman F., Rutenbar~R., Hailpern~B., Christensen~H., Davidson~S., Estrin~D., 
Franklin~M., Martonosi~M., Raghavan~P., Stodden~V., Szalay~A.\,S.} Realizing the potential of 
data science~// Commun.  ACM, 2018. Vol.~61. Iss.~4. P.~67--72. doi: 10.1145/3188721.

\bibitem{21-zac}
\Au{Stodden V.} The data science life cycle: A~disciplined approach to advancing data science as 
a~science~// Commun.  ACM, 2020. Vol.~63. Iss.~7. P.~58--66. doi: 10.1145/ 3360646.


\bibitem{23-zac} %22
\Au{Зацман И.\,М.} Научная парадигма информатики: классификация трансформаций 
объектов предметной об\-ласти~// Системы и~средства информатики, 2023. Т.~33. №\,4. 
С.~126--138. doi: 10.14357/08696527230412. EDN: ZIKUWO.

\bibitem{22-zac} %23
\Au{Зацман И.\,М.} Научная парадигма информатики: классификация объектов предметной  
об\-ласти~// Информатика и~её применения, 2023. Т.~17. Вып.~4. С.~96--103. doi: 
10.14357/19922264230413. EDN: FIUQAT.

\bibitem{24-zac}
\Au{Зацман И.\,М.} О~научной парадигме информатики: верхний уровень классификации 
объектов ее предметной об\-ласти~// Информатика и~её применения, 2022. Т.~16. Вып.~4. 
С.~73--79. doi: 10.14357/ 19922264220411. EDN: XZNKVI.

\bibitem{25-zac}
\Au{Соломоник А.\,Б.} Философия знаковых систем и~язык.~--- М.: ЛКИ, 2011. 408~с.
\bibitem{26-zac}
\Au{Зацман И.\,М.} Трансформация иерархии Акоффа в~научной парадигме информатики~// 
Информатика и~её применения, 2023. Т.~17. Вып.~3. С.~107--113. doi: 
10.14357/19922264230315. EDN: UMVRRV.

\bibitem{27-zac}
\Au{Zatsman I.} Building digital spiral models of knowledge generation~// 19th Forum 
(International) on Knowledge Asset Dynamics Proceedings.~--- Matera, Italy: Arts for Business 
Institute, 2024. P.~2185--2196.
\bibitem{28-zac}
\Au{Zatsman I.} Digital spiral model of knowledge creation and encoding its dynamics~// 18th 
Forum (International) on Knowledge Asset Dynamics Proceedings.~--- Matera, Italy: Arts for 
Business Institute, 2023. P.~581--596.
\bibitem{29-zac}
\Au{Зацман И.\,М.} Интерфейсы третьего порядка в~информатике~// Информатика и~её 
применения, 2019. Т.~13. Вып.~3. С.~82--89. doi: 10.14357/19922264190312. EDN: 
EHRQLF.

\bibitem{30-zac}
\Au{Зацман И.\,М.} Научная парадигма информатики как третьей культуры~//  
На\-уч\-но-тех\-ни\-че\-ская информация. Сер.~1: Организация и~методика информационной 
работы, 2023. №\,11. С.~1--14.

\end{thebibliography}

 }
 }

\end{multicols}

\vspace*{-9pt}

\hfill{\small\textit{Поступила в~редакцию 14.04.24}}

\vspace*{4pt}

%\pagebreak

%\newpage

%\vspace*{-28pt}

\hrule

\vspace*{2pt}

\hrule



\def\tit{OBJECT TRANSFORMATIONS OF~THE~FIRST AND~SECOND ORDER
IN~A~LEXICOGRAPHIC INFORMATION SYSTEM\\[-5pt]}


\def\titkol{Object transformations of~the~first and~second order
in~a~lexicographic information system}


\def\aut{I.\,M.~Zatsman}

\def\autkol{I.\,M.~Zatsman}

\titel{\tit}{\aut}{\autkol}{\titkol}

\vspace*{-13pt}


\noindent
Federal Research Center ``Computer Science and Control'' of the Russian Academy of Sciences, 
44-2~Vavilov Str., Moscow 119133, Russian Federation


\def\leftfootline{\small{\textbf{\thepage}
\hfill INFORMATIKA I EE PRIMENENIYA~--- INFORMATICS AND
APPLICATIONS\ \ \ 2024\ \ \ volume~18\ \ \ issue\ 2}
}%
 \def\rightfootline{\small{INFORMATIKA I EE PRIMENENIYA~---
INFORMATICS AND APPLICATIONS\ \ \ 2024\ \ \ volume~18\ \ \ issue\ 2
\hfill \textbf{\thepage}}}

\vspace*{2pt}



\Abste{The theoretical foundations of the design of information technologies used for 
the integration of bilingual dictionaries and parallel corpora are considered. The 
description of the first outcomes of the creation of the third\linebreak\vspace*{-12pt}}

\Abstend{ level of object 
transformations classification in the subject domain of informatics, which is supposed 
to be used
in creating the lexicographic information system providing integration, is 
given. All the entities of informatics are divided into two global classes: objects and 
their transformations. For each such class, its own classification is constructed. 
Previously, the two upper levels of the object transformation classification in the subject 
domain have been described. The present paper discusses the third level of this classification. The 
basis for the construction of its highest level was the division of the subject domain of 
informatics into media (mental, sensory, digital, and a~number of other media), each 
of which by definition includes objects of the same nature. The Solomonick's 
typology of sign systems served as the basis for constructing the second level of the 
object transformation classification. The aim of the paper is to systematize object 
transformations of the first and second orders at the third level of this classification. 
The basis for systematization is the medium version of the Ackoff's hierarchy.}

\KWE{subject domain objects; object transformations; classification; data; 
information; knowledge; lexicographic information system}


\DOI{10.14357/19922264240211}{VZTGVV}

\vspace*{-12pt}

\Ack

\vspace*{-3pt}


\noindent
The reported study was funded by the Russian Science Foundation, project  
No.\,24-18-00155, {\sf 
https://rscf.ru/project/24-18-00155}. The research was carried out using the infrastructure of the Shared 
Research Facilities ``High Performance Computing and Big Data'' (CKP 
``Informatics'') of FRC CSC RAS (Moscow) .
   


  \begin{multicols}{2}

\renewcommand{\bibname}{\protect\rmfamily References}
%\renewcommand{\bibname}{\large\protect\rm References}

{\small\frenchspacing
 {%\baselineskip=10.8pt
 \addcontentsline{toc}{section}{References}
 \begin{thebibliography}{99} 
\bibitem{1-zac-1}
\Aue{Aijmer, K., and B.~Altenberg.} 2013. \textit{Advances in corpus-based 
contrastive linguistics. Studies in honour of Stig Johansson}. Amsterdam: John 
Benjamins. 295~p. doi: 10.1075/scl.54.
\bibitem{2-zac-1}
\Aue{Dobrovolskiy, D.\,O., A.\,A.~Kretov, and S.\,A.~Sharov.} 2005. Korpus 
parallel'nykh tekstov [Corpus of parallel texts]. \textit{Nauchnaya i~tekhnicheskaya 
informatsiya. Ser. 2. Informatsionnye protsessy i~sistemy} [Scientific and Technical 
Information. Ser.~2: Information Processes and Systems] 6:16--27.
\bibitem{3-zac-1}
\Aue{Dobrovolskiy, D.\,O.} 2015. Korpus parallel'nykh tekstov i~sopostavitel'naya 
leksikologiya [The corpus of parallel texts and contrastive lexicology]. \textit{Trudy 
Instituta russkogo yazyka im. V.\,V.~Vinogradova} [Proceedings of the 
V.\,V.~Vinogradov Russian Language Institute] 6:413--449. EDN: VJQBHP.
\bibitem{4-zac-1}
\Aue{Goncharov, A.\,A., I.\,M.~Zatsman, and M.\,G.~Kruzhkov.} 2020. Evolyutsiya 
klassifikatsiy v~nadkorpusnykh ba\-zakh dannykh [Evolution of classifications in 
supracorpora databases]. \textit{Informatika i~ee Primeneniya~--- Inform. \mbox{Appl.}}  
14(4):108--116. doi: 10.14357/19922264200415.  
EDN: GKWBZT.
\bibitem{5-zac-1}
\Aue{Goncharov, A.\,A., I.\,M.~Zatsman, and M.\,G.~Kruzhkov.} 2021. 
Predstavlenie novykh leksikograficheskikh znaniy v~dinamicheskikh 
klassifikatsionnykh sistemakh [Representation of new lexicographical knowledge in 
dynamic classification systems]. \textit{Informatika i~ee Primeneniya~--- Inform. 
Appl.} 15(1):86--93. doi: 10.14357/19922264210112. EDN: OPEFXW.
\bibitem{6-zac-1}
\Aue{Zatsman, I.} 2020. Finding and filling lacunas in linguistic typologies. 
\textit{15th Forum (International) on Knowledge Asset Dynamics Proceedings}. 
Matera, Italy: Institute of Knowledge Asset Management. 780--793.
\bibitem{7-zac-1}
\Aue{Zatsman, I.} 2020. Three-dimensional encoding of emerging meanings in  
AI-systems. \textit{21st European Conference on Knowledge Management 
Proceedings}. Reading, U.K.: Academic Publishing International Ltd. 878--887.
\bibitem{8-zac-1}
\Aue{Ackoff, R.} 1989. From data to wisdom. \textit{J.~Applied Systems Analysis} 
16(1):3--9.
\bibitem{9-zac-1}
\Aue{Rosenbloom, P.\,S.} 2013. \textit{On computing: The fourth great scientific 
domain}. Cambridge, MA: MIT Press. 307~p.
\bibitem{10-zac-1}
\Aue{Rowley, J.} 2007. The wisdom hierarchy: Representations of the DIKW 
hierarchy. \textit{J.~Inf. Sci.} 33(2):163--180. doi: 10.1177/0165551506070706.
\bibitem{11-zac-1}
\Aue{Frick$\acute{\mbox{e}}$, M.\,H.} 2018.  
Data-Information-Knowledge-Wisdom (DIKW) pyramid, framework, continuum. 
\textit{Encyclopedia of big data}. Eds. L.~Schintler and C.~McNeely. Cham: 
Springer. 4~p. doi: 10.1007/978-3-319-32001- 4\_331-1.
\bibitem{12-zac-1}
\Aue{Denning, P., and P.~Rosenbloom.} 2009. Computing: The fourth great domain 
of science. \textit{Commun. ACM} 52(9):27--29.
\bibitem{13-zac-1}
\Aue{Denning, P., and P.~Freeman.} 2009. Computing's paradigm. \textit{Commun. 
ACM} 52(12):28--30. doi: 10.1145/ 1610252.1610265.

\bibitem{17-zac-1} %14
\Aue{Farradane, J.} 1980. Knowledge, information, and information science. 
\textit{J.~Inf. Sci.} 2(2):75--80. doi: 10.1177/ 01655515800020020.

\bibitem{15-zac-1}
\Aue{Shreyder, Yu.\,A.} 1988. Informatsiya i~znanie [Information and knowledge]. 
\textit{Sistemnaya kontseptsiya in\-for\-ma\-tsi\-on\-nykh protsessov} [System concept of 
information processes]. Moscow: VNIISI. 47--52.
\bibitem{16-zac-1}
\Aue{Ingwersen, P.} 1995. Information and information science. 
\textit{Encyclopedia of library and information science}. Eds. J.\,D.~McDonald and 
M.~Levine-Clark. New York, NY: Marcel Dekker Inc. 56(19):137--174.

\bibitem{14-zac-1} %17
Gilyarevskiy, R.\,S., ed. 2006. \textit{Informatika kak nauka ob informatsii: 
informatsionnyy, dokumental'nyy, tekh\-no\-lo\-gi\-che\-skiy, ekonomicheskiy, sotsial'nyy 
i~organizatsionnyy aspekty} [Informatics as information science: Informational, 
documentary, technological, economic, social, and organizational dimensions]. 
Moscow: FAIR-PRESS. 592~p.

\bibitem{18-zac-1}
\Aue{Hjorland, B.} 2000. Library and information science: Practice, theory, and 
philosophical basis. \textit{Inform. Process. Manag.} 36(3):501--531. doi:  
10.1016/S0306-\mbox{4573(99)00038-2}.
\bibitem{19-zac-1}
Deep shift~--- technology tipping points and societal impact. 2015. \textit{World Economic 
Forum}. Geneva. 44~p. Available at: {\sf 
http://www3.weforum.org/docs/WEF\_ GAC15\_Technological\_Tipping\_Points\_report\_2015.pdf} (accessed May~20, 
2024).
\bibitem{20-zac-1}
\Aue{Berman, F., R.~Rutenbar, B.~Hailpern, H.~Christensen, S.~Davidson, 
D.~Estrin, M.~Franklin, M.~Martonosi, P.~Raghavan, V.~Stodden, and 
A.\,S.~Szalay.} 2018. Realizing the potential of data science. \textit{Commun. ACM} 
61(4):67--72. doi: 10.1145/3188721.
\bibitem{21-zac-1}
\Aue{Stodden, V.} 2020. The data science life cycle: A~disciplined approach to 
advancing data science as a~science. \textit{Commun. ACM} 
 63(7):58--66. doi: 10.1145/3360646.

\bibitem{23-zac-1} %22
\Aue{Zatsman, I.\,M.} 2023. Nauchnaya paradigma informatiki: klassifikatsiya 
transformatsiy ob''ektov predmetnoy oblasti [Scientific paradigm of informatics: 
Transformation classification of domain objects]. \textit{Sistemy i~Sredstva 
Informatiki~--- Systems and Means of Informatics} 33(4):126--138. doi: 
10.14357/08696527230412. EDN: ZIKUWO.

\bibitem{22-zac-1} %23
\Aue{Zatsman, I.\,M.} 2023. Nauchnaya paradigma informatiki: klassifikatsiya 
ob''ektov predmetnoy oblasti [Scientific paradigm of informatics: Classification of 
domain objects]. \textit{Informatika i~ee Primeneniya~--- Inform. Appl.} 
 17(4):96--103. doi: 10.14357/19922264230413. EDN: FIUQAT.
 
\bibitem{24-zac-1}
\Aue{   Zatsman, I.\,M.} 2022. O nauchnoy paradigme informatiki: verkhniy uroven' 
klassifikatsii ob''ektov ee predmetnoy oblasti [On the scientific paradigm of 
informatics: The classification high level of its objects]. \textit{Informatika i~ee 
Primeneniya~--- Inform. Appl.} 16(4):73--79. doi: 10.14357/19922264220411. EDN: 
XZNKVI.
\bibitem{25-zac-1}
\Aue{Solomonick, A.\,B.} 2011. \textit{Filosofiya znakovykh system i~yazyk} 
[Philosophy of sign systems and language]. Moscow: LKI. 408~p.
\bibitem{26-zac-1}
\Aue{Zatsman, I.\,M.} 2023. Transformatsiya ierarkhii Akoffa v~nauchnoy 
paradigme informatiki [Transformation of the Ackoff's hierarchy in the scientific 
paradigm of informatics]. \textit{Informatika i~ee Primeneniya~--- Inform. \mbox{Appl.}} 
17(3):107--113. doi: 10.14357/19922264230315. EDN: UMVRRV.
\bibitem{27-zac-1}
\Aue{Zatsman, I.} 2024. Building digital spiral models of knowledge 
generation. \textit{19th Forum (International) on Knowledge Asset Dynamics 
Proceedings}. Matera, Italy: Arts for Business Institute. 2185--2196.
\bibitem{28-zac-1}
\Aue{Zatsman, I.} 2023. Digital spiral model of knowledge creation and encoding its 
dynamics. \textit{18th Forum (International) on Knowledge Asset Dynamics 
Proceedings}. Matera, Italy: Arts for Business Institute. 581--596.
\bibitem{29-zac-1}
\Aue{Zatsman, I.\,M.} 2019. Interfeysy tret'ego poryadka v~informatike 
 [Third-order interfaces in informatics]. \textit{Informatika i~ee Primeneniya~--- 
Inform. Appl.} 13(3):82--89. doi: 10.14357/19922264190312. EDN: EHRQLF.
\bibitem{30-zac-1}
\Aue{Zatsman, I.} 2023. Scientific paradigm of informatics as a~third culture. 
\textit{Scientific Technical Information Processing} 50(4):246--258. doi: 
10.3103/S0147688223040111. EDN: CKHMYS.

\end{thebibliography}

 }
 }

\end{multicols}

\vspace*{-6pt}

\hfill{\small\textit{Received April 14, 2024}} 


\vspace*{-12pt}


\Contrl

\vspace*{-3pt}

\noindent
\textbf{Zatsman Igor M.} (b.\ 1952)~--- Doctor of Science in technology, head of 
department, Federal Research Center ``Computer Science and Control'' of the 
Russian Academy of Sciences, 44-2~Vavilov Str., Moscow 119333, Russian 
Federation; \mbox{izatsman@yandex.ru}





\label{end\stat}

\renewcommand{\bibname}{\protect\rm Литература}  %11

\def\stat{sokolov}

\def\tit{О РАБОТАХ ЗАСЛУЖЕННОГО ДЕЯТЕЛЯ НАУКИ 
РОССИЙСКОЙ ФЕДЕРАЦИИ И.\,Н.~СИНИЦЫНА В ОБЛАСТИ 
ИНФОРМАЦИОННЫХ ТЕХНОЛОГИЙ И АВТОМАТИЗАЦИИ\\
(к 70-летию со дня рождения)}

\def\titkol{О работах заслуженного деятеля науки 
РФ И.\,Н.~Синицына в области 
информационных технологий и автоматизации
%(к семидесятилетию со дня рождения)
}

\def\autkol{И.\,А.~Соколов}
\def\aut{И.\,А.~Соколов$^1$}

\titel{\tit}{\aut}{\autkol}{\titkol}

%{\renewcommand{\thefootnote}{\fnsymbol{footnote}}\footnotetext[1]
%{Работа поддерживается РФФИ, грант  10-07-00017.}}

\renewcommand{\thefootnote}{\arabic{footnote}}
\footnotetext[1]{Институт проблем информатики Российской академии наук, isokolov@ipiran.ru}


\bigskip



%\bigskip

       \vskip 14pt plus 9pt minus 6pt

      \thispagestyle{headings}

      \begin{multicols}{2}

      \label{st\stat}
      
 \begin{center}
\mbox{%
\epsfxsize=50mm
\epsfbox{sok-1.eps}
}
\end{center}
\vspace*{4pt}
%\begin{center}

\bigskip



      14~августа 2010~г.\ исполнилось 70~лет Игорю Николаевичу Синицыну~--- члену 
редколлегии журнала <<Информатика и её применения>>, крупному ученому в области 
прикладной механики и управ\-ле\-ния, прикладной математики и информатики, основателю 
научной школы в области стохастических информационных технологий.
      
      И.\,Н.~Синицын родился в Москве. Высшее образование получил в МВТУ им.\ 
Н.\,Э.~Баумана и МГУ им.\ М.\,В.~Ломоносова. Одновременно с учебой в МГУ начал работать 
в известном ра\-кет\-но-кос\-ми\-че\-ском НИИ, ныне Институте прикладной механики им.\ 
В.\,И.~Кузнецова (НИИПМ). Инженерную и научную деятельность в НИИПМ в области 
разработки и испытаний гироскопических командных приборов и 
      ин\-фор\-ма\-ци\-он\-но-из\-ме\-ри\-тель\-ных систем (1960--1983~гг.), он совмещал с 
преподавательской работой сначала в МВТУ им.\ Н.\,Э.~Баумана, затем в 
      Воен\-но-воз\-душ\-ной инженерной академии им.\ профессора Н.\,Е.~Жуковского 
(ВВИА).
      
      Начиная с 1974~г.\ И.\,Н.~Синицын работал на факультете авиационного вооружения 
ВВИА. Занимался подготовкой авиационных инженеров, принимал участие в разработке и 
испытаниях\linebreak
 специальной техники, участвовал в подготовке первых космонавтов СССР.
      
      Для организации работ в области специальных применений ЭВМ новых поколений 
И.\,Н.~Синицын в 1984~г.\ переводится в только что организованный Институт проблем 
информатики АН СССР (ныне ИПИ РАН).
      
      В настоящее время И.\,Н.~Синицын работает заведующим отделом стохастических 
проблем информатики и управления ИПИ РАН, много внимания уделяет подготовке научных 
кадров. Он руководит специальной секцией ученого совета ИПИ РАН, комиссией Минобрнауки 
по информатике в военных вузах, является членом экспертного совета РФФИ, заместителем 
главных редакторов журналов <<Наукоемкие технологии>> и <<Системы высокой 
доступности>>, членом редколлегий журналов <<Pattern Recognition and Image Analysis>>, 
<<Информатика и её применения>>. С~1987~г.\ И.\,Н.~Синицын~--- профессор МАИ, читает 
лекции по теории и практике информационных технологий в инженерном деле. 
     
     В разные годы И.\,Н.~Синицын был заместителем генерального конструктора и главным 
конструктором ряда автоматизированных и информационных систем специального назначения. 
     
     В 2001~г.\ И.\,Н.~Синицыну присвоено почетное звание Заслуженного деятеля науки 
Российской Федерации.
     
     И.\,Н.~Синицын имеет большой опыт работы в промышленности и высших технических 
учебных заведениях. Он автор более 500~научных трудов, свыше 50 книг, монографий и 
30~изобретений. Его основные научные труды относятся к следующим областям:
     \begin{itemize}
     \item статистическая теория информационных технологий и автоматизированных систем;
     \item прецизионные ин\-фор\-ма\-ци\-он\-но-из\-ме\-ри\-тель\-ные технологии и системы для научных 
исследований и специального назначения;
     \item
      информационно-аналитические технологии и системы поддержки принятия решений для 
информатизации высших органов государственной власти РФ, федеральных ведомств и~др.
     \end{itemize}
     
     И.\,Н.~Синицыну принадлежат фундаментальные результаты по теории канонических 
представлений случайных функций в сложных стохастических системах (СтС), в том числе СтС 
с\linebreak
 распределенными параметрами и случайной структурой. Методы теории СтС им 
распространены на\linebreak
 СтС, описываемые дифференциальными уравнениями со случайными 
функциями состояния,\linebreak уравнениями в гильбертовых и банаховых пространствах. Им 
разработаны эффективные вычислительные методы нахождения распределений, основанные на 
параметризации, позволяющие радикально сократить число уравнений для па\-ра\-мет\-ров 
распределений, а также новые вычислительные методы статистического анализа и синтеза,\linebreak 
допускающие эффективное оценивание точности и ориентированные на параллельные 
статистические вычисления. 
     
     И.\,Н.~Синицын разработал методы нахождения точных выражений для распределений с 
инвариантной мерой, обнаружил ряд новых классов точных распределений. Им получены 
фундаментальные результаты в области нелинейной условно оптимальной и субоптимальной 
фильтрации в реальном масштабе времени. Важные результаты получены И.\,Н.~Синицыным в 
области тео\-ре\-ти\-ко-груп\-по\-вых методов анализа и синтеза автоматизированных систем. 
Им разработана статистическая теория катастрофоустойчивости автоматизированных сис\-тем 
высокой точности и доступности. 
     
     И.\,Н.~Синицын~--- основоположник стохастических информационных технологий 
оперативной обработки информации, контроля и мониторинга автоматизированных систем, а 
также\linebreak стохастического управления информационными\linebreak активами, моделирования и синтеза 
систем: проб\-лем\-но-ориен\-ти\-ро\-ван\-ных диалоговых систем и\linebreak библиотек 
     <<СтС-Ана\-лиз>>, <<СтС-Фильтр>>, <<СтС-Мо\-дель>>, Nailb, <<TransStatLib>>, 
<<Безопасность и надежность>>, <<Здоровье РФ>> и~др. В~последние годы им разработаны 
эффективные символьные методы анализа и синтеза СтС. Создано и внедрено 
специализированное программное обеспечение СтС-СМА и СтС-\mbox{ИТКР}. 
     
     Его книги~--- <<Стохастические дифференциальные системы. Анализ и фильтрация>>, 
<<Лекции по функциональному анализу и его приложениям>>, <<Теория стохастических 
систем>> (совместно с В.\,С.~Пугачевым), а также <<Фильтры Калмана и Пугачева>> и 
<<Канонические представления случайных функций и их применение в задачах компьютерной 
поддержки научных исследований>>~--- широко известны в России и за рубежом.
     
     И.\,Н.~Синицыным впервые разработана теория ряда 
     ин\-фор\-ма\-ци\-он\-но-из\-ме\-ри\-тель\-ных систем в условиях случайных динамических 
возмущений, открыт ряд новых статистических динамических эффектов (выбросы разных 
типов, флуктуационные уходы, накопление возмущений и~др.). Ему принадлежат первые 
работы по статистической динамике командно-измерительных гироскопических приборов, 
акселерометров, градиентометров и метрологических систем высочайшей точности, 
информационной теории и методам измерений, калибровок, ускоренных испытаний в 
экстремальных условиях, а также статистического и полунатурного моделирования. Под его 
руководством и при его непосредственном участии разработано и внедрено несколько 
поколений серийных систем, обла\-да\-ющих уникальными характеристиками. И.\,Н.~Синицын 
принимал непосредственное участие в определении технической политики в области новой 
специальной техники. 
     
     С именем И.\,Н.~Синицына связано создание концепций автоматизации научных 
исследований в РФ, в первую очередь основанных на средствах массовой вычислительной 
техники. Под его руководством и при участии сформулированы принципы создания 
микровидеосистем, создан ряд базовых персональных микровидеосистем. Они внедрены в МВД 
и Минздраве РФ. Достигнутые результаты получили развитие в автоматизированных системах 
метрологического обеспечения, видеоконтроля и биометрических системах. 
     
     В последние годы под руководством И.\,Н.~Синицына разработаны принципы построения 
и архитектуры вычислительных систем командных пунктов, а также новые методы и алгоритмы 
быстрой обработки изображений, обладающих сильной про\-стран\-ст\-вен\-но-вре\-мен\-н\'{о}й 
деформацией. В~целях\linebreak автоматизации астрометрических научных исследований по 
фундаментальной проблеме <<Статистическая динамика вращения Земли>> был создан 
комплекс моделей, алгоритмов и специального\linebreak программного обеспечения и информационных 
ресурсов для нестандартной интегрированной обработки параметров вращения Земли. 
И.\,Н.~Синицын впервые обнаружил ряд новых эффектов: автоколебания полюса Земли на 
чандлеровской частоте, параметрическую стабилизацию чандлеровских колебаний, нелинейные 
флуктуационные дрейфы нестабильности вращения Земли и~др.
     
     В области информационно-аналитических технологий и автоматизированных систем 
поддержки принятия решений для информатизации высших органов государственной власти, 
федеральных ведомств и~др.\ под руководством И.\,Н.~Синицына был разработан и внедрен 
ряд базовых информационных технологий (обработка информации от независимых источников, 
формирование и хранение больших баз электронных образов, управления информационными 
активами и~др.). Сформулированы принципы и разработаны базовые системотехнические 
решения для ряда крупномасштабных автоматизированных информационных и 
     ин\-фор\-ма\-ци\-он\-но-управ\-ля\-ющих систем специального назначения, высокой 
точности и доступности. 
     
     И.\,Н.~Синицын пользуется широким международным авторитетом: его книги и работы 
изданы на английском, французском, испанском и китайском языках; в 1990--1994~гг.\ он был 
директором Рос\-сий\-ско-фран\-цуз\-ско\-го центра <<Эвклид>>, с~1983~г.~--- членом 
программных комитетов многих международных конференций; в период 1985--2006~гг.\ был 
экспертом фонда INTAS.
     
     И.\,Н.~Синицын ведет активную работу по подготовке научно-педагогических кадров. 
Под его руководством выполнено свыше 25~кандидатских и докторских диссертаций. Он 
состоит членом ряда специализированных диссертационных советов; в течение многих лет был 
членом экспертного совета ВАК России. И.\,Н.~Синицын~--- член редколлегии нашего 
журнала, он не толко публикуется в нем, но и ведет большую редакционную работу.

\label{end\stat}
     
     \bigskip
     Редколлегия журнала сердечно поздравляет И.\,Н.~Синицына с юбилеем и желает ему 
здо\-ровья, счастья, новых творческих успехов.


\end{multicols}




%   { %\Large  
   { %\baselineskip=16.6pt
   
   \vspace*{-48pt}
   \begin{center}\LARGE
   \textit{Предисловие}
   \end{center}
   
   %\vspace*{2.5mm}
   
   \vspace*{25mm}
   
   \thispagestyle{empty}
   
   { %\small 

    
Вниманию читателей журнала <<Информатика и её применения>> предлагается 
очередной тематический выпуск <<Вероятностно-статистические методы и 
задачи информатики и информационных технологий>>. Предыдущие тематические 
выпуски журнала по данному направлению вышли в 2008~г.\ (т.~2, вып.~2), 
в 2009~г.\ (т.~3, вып.~3) и в 2010~г.\ (т.~4, вып.~2). 

Статьи, собранные в данном журнале, посвящены разработке новых вероятностно-статистических 
методов, ориентированных на применение к решению конкретных задач информатики и информационных 
технологий, а также~--- в ряде случаев~--- и других прикладных задач. Проблематика, охватываемая 
публикуемыми работами, развивается в рамках научного сотрудничества между Институтом проблем 
информатики Российской академии наук (ИПИ РАН) и Факультетом вычислительной математики и 
кибернетики Московского государственного университета им.\ М.\,В.~Ломоносова в ходе работ 
над совместными научными проектами (в том числе в рамках функционирования 
Научно-образовательного центра <<Вероятностно-статистические методы анализа рисков>>). 
Многие из авторов статей, включенных в данный номер журнала, являются активными участниками 
традиционного международного семинара по проблемам устойчивости стохастических моделей, 
руководимого В.\,М.~Золотаревым и В.\,Ю.~Королевым; регулярные сессии этого семинара 
проводятся под эгидой МГУ и ИПИ РАН (в 2011~г.\ указанный семинар проводится в октябре 
в Калининградской области РФ). 

Наряду с представителями ИПИ РАН и МГУ в число авторов данного выпуска журнала входят 
ученые из Научно-исследовательского института системных исследований РАН, Института 
проблем технологии микроэлектроники и особочистых материалов РАН, Института 
прикладных математических исследований Карельского НЦ РАН, Московского 
авиационного института, Вологодского государственного педагогического университета, 
НИИММ им.\ Н.\,Г.~Чеботарева, Казанского государственного университета, Дебреценского 
университета (Венгрия).

Несколько статей выпуска посвящено разработке и применению стохастических методов и 
информационных технологий для решения различных прикладных задач. В~работе В.\,Г.~Ушакова 
и О.\,В.~Шестакова рассмотрена задача определения вероятностных характеристик случайных 
функций по распределениям интегральных преобразований, возникающих в задачах эмиссионной 
томографии. В~статье Д.\,О.~Яковенко и М.\,А.~Целищева рассмотрены некоторые вопросы 
математической теории риска и предложен новый подход к диверсификации инвестиционных 
портфелей. Работа И.\,А.~Кудрявцевой и А.\,В.~Пантелеева посвящена построению и 
исследованию математической модели, описывающей динамику сильноионизованной плазмы. 
В~статье П.\,П.~Кольцова изучается качество работы ряда алгоритмов сегментации изображений. 
Статья А.\,Н.~Чупрунова и И.~Фазекаша посвящена вероятностному анализу числа без\-оши\-бочных 
блоков при помехоустойчивом кодировании; получены усиленные законы больших чисел для указанных 
величин.

В данном выпуске традиционно присутствует тематика, весьма активно разрабатываемая в течение 
многих лет специалистами ИПИ РАН и МГУ,~--- методы моделирования и управления для 
информационно-телекоммуникационных и вычислительных систем, в частности методы 
теории массового обслуживания. В~статье А.\,И.~Зейфмана с соавторами рассматриваются 
модели обслуживания, описываемые марковскими цепями с непрерывным временем в случае 
наличия катастроф. В~работе М.\,М.~Лери и И.\,А.~Чеплюковой рассматриваются случайные 
графы Интернет-типа, т.\,е.\ графы, степени вершин которых имеют степенные распределения; 
такие задачи находят применение при исследовании глобальных сетей передачи данных. 
Работа Р.\,В.~Разумчика посвящена исследованию систем массового обслуживания специального 
вида~--- с отрицательными заявками и хранением вытесненных заявок.

Ряд статей посвящен развитию перспективных теоретических 
вероятностно-статистических методов, которые находят широкое применение в различных 
задачах информатики и информационных технологий. В~работе В.\,Е.~Бенинга, А.\,К.~Горшенина 
и В.\,Ю.~Королева рассмотрена задача статистической проверки гипотез о числе компонент 
смеси вероятностных распределений, приводится конструкция асимптотически наиболее мощного 
критерия. Результаты этой работы найдут применение в ряде прикладных задач, использующих 
математическую модель смеси вероятностных распределений (в информатике, моделировании 
финансовых рынков, физике турбулентной плазмы и~т.\,д.). В~статье В.\,Ю.~Королева, 
И.\,Г.~Шевцовой и С.\,Я.~Шоргина строится новая, улучшенная оценка точности нормальной 
аппроксимации для пуассоновских случайных сумм; как известно, указанные случайные суммы 
широко используются в качестве моделей многих реальных объектов, в том числе в информатике, 
физике и других прикладных областях. Работа В.\,Г.~Ушакова и Н.\,Г.~Ушакова посвящена 
исследованию ядерной оценки плотности распределения; эти результаты могут применяться, 
в част\-ности, при анализе трафика в телекоммуникационных системах. Серьезные приложения 
в статистике могут получить результаты работы О.\,В.~Шестакова, в которой доказаны оценки 
скорости сходимости распределения выборочного абсолютного медианного отклонения к нормальному 
закону. 

\smallskip

Редакционная коллегия журнала выражает надежду, что данный тематический  выпуск 
будет интересен специалистам в области теории вероятностей и математической статистики 
и их применения к решению задач информатики и информационных технологий.
     
     %\vfill 
     \vspace*{20mm}
     \noindent
     Заместитель главного редактора журнала <<Информатика и её 
применения>>,\\
     директор ИПИ РАН, академик  \hfill
     \textit{И.\,А.~Соколов}\\
     
     \noindent
     Редактор-составитель тематического выпуска,\\
     профессор кафедры математической статистики факультета\\
      вычислительной математики и кибернетики МГУ им.\ М.\,В.~Ломоносова,\\
     ведущий научный сотрудник ИПИ РАН,\\ 
доктор физико-математических наук \hfill
      \textit{В.\,Ю.~Королев}
     
     } }
     }

%%%%%%%%%%%%%%%%%%%%%%%%%%%%%%%%%%%%%%%%%%%%%%%


                       
%\end{document}

%\def\stat{rez}
{%\hrule\par
%\vskip 7pt % 7pt
\raggedleft\Large \bf%\baselineskip=3.2ex
Р\,Е\,Ц\,Е\,Н\,З\,И\,И \vskip 17pt
    \hrule
    \par
\vskip 6pt plus 6pt minus 3pt }

%\thispagestyle{headings} %с верхним колонтитулом
%\thispagestyle{myheadings} %с нижним колонтитулом, но в верхнем РЕЦЕНЗИИ

\def\tit{НОВАЯ КНИГА И.\,Н.~СИНИЦЫНА, А.\,С.~ШАЛАМОВА <<ЛЕКЦИИ ПО ТЕОРИИ 
ИНТЕГРИРОВАННОЙ ЛОГИСТИЧЕСКОЙ ПОДДЕРЖКИ>> (М.: ТОРУС ПРЕСС, 2012. 624~с.)}

%1
\def\aut{Д.ф.-м.н., профессор С.\,Я.~Шоргин}

\def\auf{\ }

\def\leftkol{\ % РЕЦЕНЗИИ
}

\def\rightkol{ \ } 

%\def\leftkol{\ } % ENGLISH ABSTRACTS}

%\def\rightkol{\ } %ENGLISH ABSTRACTS}

%\def\leftkol{РЕЦЕНЗИИ}

%\def\rightkol{РЕЦЕНЗИИ}

\titele{\tit}{\aut}{\auf}{\leftkol}{\rightkol}
\vspace*{-18pt}


     \label{st\stat}

     \begin{multicols}{2}
     {\small
     {\baselineskip=10.1pt
     

      В книге представлено системное изложение теоретических основ одного из новейших 
направлений в \mbox{об\-ласти} экономики послепродажного обслуживания изделий наукоемкой 
продукции (ИНП) длительного пользования~--- интегрированной логистической поддержки
(ИЛП). 
{\looseness=1

}

Приведены также результаты новых работ, выполненных в Институте проблем информатики 
Российской академии наук в рамках научного направления <<Информационные технологии и 
анализ сложных сис\-тем>>.
 {%\looseness=1

}
     
      Излагаемые в книге научные подходы позво\-ляют карди\-наль\-но реформировать 
существующие системы производства и эксплуатации ИНП путем создания и внед\-ре\-ния 
методов рационального и оптимального управ\-ле\-ния процессами расходования 
вре\-мен\-н$\acute{\mbox{ы}}$х, 
мате\-ри\-аль\-ных, трудовых и других ресурсов на всех стадиях жизненного цикла изделий (ЖЦИ) по 
критериям экономической целесообразности и эф\-фек\-тив\-ности.
  {\looseness=1

}
    
      В книге приведен краткий обзор причин возник\-новения и
      развития CALS-методологии как основы 
современных международных стандартов по созданию и функционированию глобальных 
ин\-фор\-ма\-ци\-он\-но-ком\-му\-ни\-ка\-ци\-он\-ных систем, ее ключевых возможностей и эффективности 
результатов ее использования. 
Авторы %\linebreak 
предлагают ряд научных обоснований для разработки 
единой теории проектирования и управления систем ИЛП для полноценного использования 
преимуществ %\linebreak
 суще\-ст\-ву\-ющей методологии, определяют \mbox{общую} структурную схему 
комплексной системы <<ИНП-СППО>> и необходимость разработки для ее описания 
гибридных стохастических моделей.
{%\looseness=1

}

%\columnbreak
      
      Книга состоит из пяти частей, где последовательно излагается материал по каждой из 
следующих тем: <<Интегрированная логистическая поддержка>>, <<Теория гибридных 
стохастических систем и компьютерная поддержка исследований и разработок>>, <<Основы 
математического моделирования, анализа и синтеза систем послепродажного обслуживания>>, 
<<Определение и анализ показателей экспортного потенциала ИНП при проектировании>>, 
<<Задачи управления поддержкой послепродажного обслуживания>>, а также 
<<Моделирование инвестиционных процессов ИЛП в условиях неравновесных финансовых 
рынков>>. 
   
      В конце каждой главы приведены выводы и даны вопросы и задания для 
самоконтроля. В~приложениях содержатся основные определения по программам работ по 
анализу ИЛП, логистическим базам данных и компьютерным решениям, эквивалентной статистической 
линеаризации нелинейных преобразований ИЛП, справочный материал, а также развернутые 
уравнения для вероятностных характеристик.


      \def\leftkol{РЕЦЕНЗИИ}

\def\rightkol{РЕЦЕНЗИИ} 

      
      Книга заинтересует широкий круг специалистов и может быть использована научными 
проектными организациями в сфере промышленного производства ИНП. Большое количество 
иллюстраций, примеров и вопросов, обращенных к читателю, позволяет использовать книгу 
также в качестве учебного пособия для студентов и аспирантов машиностроительных, 
транспортных и~других специальностей, а также для самостоятельного изучения. 
{%\looseness=-1

}

Книга 
представляет несомненный интерес для специалистов и студентов в области прикладной 
математики и информатики.
    

}

}
\end{multicols}

%\newpage

\include{obchak}


\def\stat{authorsrus}
{%\hrule\par
%\vskip 7pt % 7pt
\raggedleft\Large \bf%\baselineskip=3.2ex
О\,Б\ \ А\,В\,Т\,О\,Р\,А\,Х \vskip 17pt
    \hrule
    \par
\vskip 21pt plus 8pt minus 4pt }


\def\tit{\ }

\def\aut{\ }

\def\auf{\ }

\def\leftkol{\ } % ENGLISH ABSTRACTS}

\def\rightkol{ОБ АВТОРАХ} %ENGLISH ABSTRACTS}

\titele{\tit}{\aut}{\auf}{\leftkol}{\rightkol}
      
            \label{st\stat}



\vspace*{24pt}

\begin{multicols}{2}




\noindent
\textbf{Архипов Олег Петрович} (р.\ 1948)~---
кандидат технических наук, директор Орловского филиала Института проб\-лем информатики
Российской академии наук
%302025, г.Орел, Московское шоссе, д.137

\vspace*{3pt}

\noindent
\textbf{Бирюкова Татьяна Константиновна} (р.\ 1968)~---
кандидат фи\-зи\-ко-ма\-те\-ма\-ти\-че\-ских наук, старший научный сотрудник Института проб\-лем информатики
Российской академии наук

\vspace*{3pt}

\noindent 
\textbf{Бобков  Сергей Геннадьевич} (р.\ 1955)~---
доктор технических наук,  заведующий отделением На\-уч\-но-ис\-сле\-до\-ва\-тель\-ско\-го 
института системных исследований Российской академии наук
%117218, Москва, Нахимовский просп., 36, к.1 

\vspace*{3pt}

\noindent \textbf{Васильев Николай Семенович} (р.\ 1952)~--- доктор 
фи\-зи\-ко-ма\-те\-ма\-ти\-че\-ских наук, профессор, 
МГТУ им.\ Н.\,Э.~Баумана 
%, Москва 105005, 2-я Бауманская ул., д.~5,

\vspace*{3pt}

\noindent
\textbf{Гершкович Максим Михайлович} (р.\ 1968)~---
старший научный сотрудник Института проб\-лем информатики
Российской академии наук

\vspace*{3pt}

\noindent 
\textbf{Дьяченко Юрий Георгиевич} (р.\ 1958)~--- кандидат технических наук, 
старший научный сотрудник Института проб\-лем информатики
Российской академии наук

\vspace*{3pt}

\noindent 
\textbf{Ерошенко Александр Андреевич} (р.\ 1989)~--- аспирант кафедры 
математической статистики факультета вычисли\-тельной математики и кибернетики 
Московского государственного университета им.\ М.\,В.~Ломоносова
%119991, Москва ГСП-1, Ленинские горы, д.\ 1, стр. 52

\vspace*{3pt}
 
\noindent 
\textbf{Захаров Виктор Николаевич} (р.\ 1948)~--- 
доктор технических наук, доцент, ученый секретарь Института проб\-лем информатики
Российской академии наук

\vspace*{3pt}

\noindent
\textbf{Зейфман Александр Израилевич} (р.\ 1954)~---
доктор фи\-зи\-ко-ма\-те\-ма\-ти\-че\-ских наук, профессор, 
заведующий кафедрой Вологодского государственного университета; 
старший научный сотрудник Института проб\-лем информатики
Российской академии наук; главный научный сотрудник ИСЭРТ Российской академии наук

\vspace*{3pt}

\noindent
\textbf{Зыкин Сергей Владимирович} (р.\ 1959)~--- 
доктор технических наук, профессор, заведующий лабораторией Института математики 
им.\ С.\,Л.~Соболева Сибирского отделения Российской академии наук, Новосибирск 
%630090, пр.\ ак.\ Коптюга, 4 

\vspace*{4pt}

\noindent
\textbf{Киреев Владимир Иванович} (р.\ 1938)~---
доктор фи\-зи\-ко-ма\-те\-ма\-ти\-че\-ских наук, профессор Московского 
государственного горного университета
%Адрес: Россия, 119991, г. Москва, Ленинский проспект, д. 6

%\columnbreak

\vspace*{4pt}

\noindent
\textbf{Козеренко Елена Борисовна} (р.\ 1959)~---
кандидат филологических наук, заведующая лабораторией Института проб\-лем информатики
Российской академии наук

\vspace*{4pt}

\noindent
\textbf{Королев Виктор Юрьевич} (р.\ 1954)~--- доктор
фи\-зи\-ко-ма\-те\-ма\-ти\-че\-ских наук, профессор кафедры математической 
статистики факультета вычисли\-тельной математики и кибернетики 
Московского государственного университета; 
ведущий научный сотрудник Института проб\-лем информатики
Российской академии наук

\vspace*{4pt}

\noindent
\textbf{Коротышева Анна Владимировна} (р.\ 1988)~---
старший преподаватель Вологодского государственного университета

\vspace*{4pt}

\noindent 
\textbf{Кун Де Турк} (р.\ 1981)~--- научный сотрудник 
исследовательской группы SMACS факультета телекоммуникаций и обработки информации
Университета Гента, Бельгия
%В-9000 Гент, Бельгия

\vspace*{4pt}

\noindent
\textbf{Лупенцов Олег Сергеевич} (р.\ 1986)~---
аспирант Омского государственного института сервиса
%Омск 644043, ул.\ Певцова 13

\vspace*{4pt}

\noindent
\textbf{Лучко Олег Николаевич} (р.\ 1961)~---
кандидат педагогических наук, профессор, заведующий кафедрой 
Омского государственного института сервиса
%Омск 644043, ул.\ Певцова 13

\vspace*{4pt}

\noindent
\textbf{Малашенко Юрий Евгеньевич} (р.\ 1946)~---
доктор фи\-зи\-ко-ма\-те\-ма\-ти\-че\-ских наук, заведующий сектором 
Вычислительного центра им.\ А.\,А.~Дородницына Российской академии наук
%Адрес: 119333, Москва, ул. Вавилова, 40,

\vspace*{4pt}

\noindent
\textbf{Маньяков Юрий Анатольевич} (р.\ 1984)~---
кандидат технических наук, научный сотрудник Орловского филиала Института проб\-лем информатики
Российской академии наук
%302025, г.Орел, Московское шоссе, д.137

\vspace*{4pt}

\noindent
\textbf{Маренко Валентина Афанасьевна} (р.\ 1951)~---
кандидат технических наук, доцент, старший научный сотрудник 
Института математики им.\ С.\,Л.~Соболева Сибирского отделения Российской академии наук
%Новосибирск 630090, пр. ак. Коптюга, 4 

\vspace*{3pt}

\noindent 
\textbf{Морозов Евсей Викторович} (р.\ 1947)~--- доктор 
фи\-зи\-ко-ма\-те\-ма\-ти\-че\-ских, профессор, ведущий научный сотрудник 
Института прикладных математических исследований Карельского научного центра Российской
академии наук; 
%%185910 Россия, Республика Карелия, г.\ Петрозаводск, ул.\ Пушкинская, 11
профессор Петрозаводского государственного университета, Петрозаводск
%185910 Россия, Республика Карелия, г.\ Петрозаводск, пр.\ Ленина, 33

%\pagebreak

\vspace*{3pt}

\noindent
\textbf{Назарова Ирина Александровна} (р.\ 1966)~---
кандидат фи\-зи\-ко-ма\-те\-ма\-ти\-че\-ских наук, 
научный сотрудник Вычислительного центра им.\ А.\,А.~Дородницына Российской академии наук 
%Адрес: 119333, Москва, ул. Вавилова, 40

\vspace*{3pt}

\noindent
\textbf{Павлов Игорь Валерианович} (р.\ 1945)~--- 
доктор фи\-зи\-ко-ма\-те\-ма\-ти\-че\-ских наук, профессор МГТУ им.\ Н.\,Э.~Баумана 
%Москва 105005, 2-я Бауманская ул., д.~5 

%\pagebreak

\vspace*{3pt}

\noindent 
\textbf{Потахина Любовь Викторовна} (р.\ 1989)~--- аспирантка
Института прикладных математических исследований Карельского научного центра
Российской академии наук; 
%%185910 Россия, Республика Карелия, г.\ Петрозаводск, ул.\ Пушкинская, 11
инженер Петрозаводского государственного университета, Петрозаводск
%185910 Россия, Республика Карелия, г.\ Петрозаводск, пр.\ Ленина, 33

\vspace*{3pt}

\noindent 
\textbf{Рождественский Юрий Владимирович} (р.\ 1952)~--- 
кандидат технических наук, заведующий сектором Института проб\-лем информатики
Российской академии наук

\vspace*{3pt}

\noindent 
\textbf{Синицын Игорь Николаевич} (р.\ 1940)~--- доктор технических наук,
профессор, заслуженный деятель\linebreak\vspace*{-12pt}

\columnbreak

\noindent
 науки РФ, заведующий отделом Института проб\-лем информатики
Российской академии наук

\vspace*{7pt}


\noindent
\textbf{Сиротинин Денис Олегович} (р.\ 1984)~---
кандидат технических наук, научный сотрудник Орловского филиала Института проб\-лем информатики
Российской академии наук
%302025, г.Орел, Московское шоссе, д.137

\vspace*{7pt}

%\columnbreak

\noindent 
\textbf{Соколов  Игорь Анатольевич} (р.\ 1954)~--- академик (действительный член) Российской 
академии наук, доктор технических наук, директор Института проб\-лем информатики
Российской академии наук

\vspace*{7pt}

\noindent
\textbf{Степченков Юрий Афанасьевич} (р.\ 1951)~---
кандидат технических наук, заведующий отделом Института проб\-лем информатики
Российской академии наук

\vspace*{7pt}

\noindent
\textbf{Сурков Алексей Викторович} (р.\ 1978)~--- 
старший научный сотрудник На\-уч\-но-ис\-сле\-до\-ва\-тель\-ско\-го 
института системных исследований Российской академии наук
%117218, Москва, Нахимовский просп., 36, к.1 

\vspace*{7pt}

\noindent 
\textbf{Шестаков Олег Владимирович} (р.\ 1976)~--- доктор 
фи\-зи\-ко-ма\-те\-ма\-ти\-че\-ских, доцент кафедры математической статистики 
факультета вычисли\-тельной математики и кибернетики Московского 
государственного университета им.\ М.\,В.~Ломоносова; 
%119991, Москва ГСП-1, Ленинские горы, д.\ 1, стр. 52
старший научный сотрудник Института проб\-лем информатики
Российской академии наук
%, Москва 119333, ул. Вавилова, д.~44, корп.~2

\vspace*{7pt}

\noindent 
\textbf{Шоргин Сергей Яковлевич} (р.\ 1952.)~--- доктор
фи\-зи\-ко-ма\-те\-ма\-ти\-че\-ских наук, профессор, заместитель директора Института 
проб\-лем информатики Российской академии наук





%%%%%%%%%%%%%%%%%%%%%%%%%%%%%%%%%%%%%%%%%%%%%%%%%%%%%%%%%%%%%%%%%%%%%%%%%%%%%%%




%\def\rightkol{ОБ АВТОРАХ}
%\def\leftkol{ОБ АВТОРАХ}

 \label{end\stat}





%\def\leftfootline{\small{\textbf{\thepage}
%\hfill ИНФОРМАТИКА И ЕЁ ПРИМЕНЕНИЯ\ \ \ том~7\ \ \ выпуск~1\ \ \ 2013}
%}%
% \def\rightfootline{\small{ИНФОРМАТИКА И ЕЁ ПРИМЕНЕНИЯ\ \ \ том~7\ \ \ выпуск~1\ \ \ 2013
%\hfill \textbf{\thepage}}}


%\thispagestyle{myheadings}



\end{multicols}

\newpage


%\vspace*{-48pt}
\begin{center}\LARGE
\textit{About Authors}
\end{center}

\thispagestyle{empty}
\def\tit{\ }

\def\aut{\ }

\def\auf{\ }


\def\leftkol{ABOUT AUTHORS}

\def\rightkol{ABOUT AUTHORS}

\vspace*{-18pt}

\titele{\tit}{\aut}{\auf}{\leftkol}{\rightkol}

%\vspace*{36pt}

\def\rightmark{{\noindent\hbox to \textwidth{\hfill\small ABOUT AUTHORS
%\hfill \large\bf\thepage
}}}
\def\leftmark{{\noindent\parbox{\textwidth}{
%\raggedleft\large\bf\thepage \hfill
\small\textrm{ABOUT AUTHORS}\hfill}}}


\def\leftfootline{\small{\textbf{\thepage}
\hfill ИНФОРМАТИКА И ЕЁ ПРИМЕНЕНИЯ\ \ \ том~6\ \ \ выпуск~2\ \ \ 2012}
}%
 \def\rightfootline{\small{ИНФОРМАТИКА И ЕЁ ПРИМЕНЕНИЯ\ \ \ том~6\ \ \ выпуск~2\ \ \ 2012
\hfill \textbf{\thepage}}}


\begin{multicols}{2}

\noindent
\textbf{Agalarov Yaver M.} (b.\ 1952)~--- Candidate of Science (PhD)
in technology, 
leading scientist, Institute of Informatics Problems, Russian Academy of Sciences

\vspace*{5pt}


  \noindent
\textbf{Bosov Alexey V.} (b.\ 1969)~--- Doctor of Science in technology, Head of
Laboratory, Institute of Informatics Problems, Russian Academy of Sciences

\vspace*{5pt}


\noindent
\textbf{Dulin Sergey K.} (b.\ 1950)~--- Doctor of Science in technology, 
professor, senior scientist, Institute of Informatics Problems, Russian Academy of Sciences

\vspace*{5pt}

\noindent
\textbf{Gorshenin Andrey K.}~--- (b.\ 1986)~--- Candidate of Science (PhD)
in physics and mathematics,
senior scientist, Institute of Informatics Problems, Russian Academy of Sciences

\vspace*{5pt}

\noindent
\textbf{Kalenov Nikolay E.}  (b.\ 1945)~--- Doctor of Science in technology,
professor, Director, Library for Natural Sciences,  Russian Academy of Sciences 

\vspace*{5pt}

\noindent
\textbf{Kalinichenko Leonid A.} (b.\ 1937)~--- Doctor of Science in physics and mathematics, 
professor, Honored scientist of RF, 
Head of Laboratory, Institute of Informatics Problems, Russian Academy of Sciences 

\vspace*{5pt}

\noindent
\textbf{Karpov Alexey A.} (b.\ 1978)~--- Candidate of Science (PhD) in technology, 
senior scientist, St.\ Petersburg Institute for
Informatics and Automation,  Russian Academy of Sciences

\vspace*{5pt}

\noindent
\textbf{Kuznetsov Igor P.} (b.\ 1938)~--- Doctor of Science in technology, 
professor, principal scientist, Institute of Informatics Problems, Russian Academy of Sciences

\vspace*{5pt}


\noindent
\textbf{Markova Natalia A.} (b.\ 1950)~--- Candidate of Science (PhD) in
physics and mathematics, leading scientist,  
Institute of Informatics Problems, Russian Academy of Sciences

\vspace*{5pt}

\noindent
\textbf{Nikolaev Andrey V.} (b.\ 1985)~--- Candidate of Science (PhD) in technology, 
senior lecturer, Tchaikovsky Technological Institute, Branch of the Izhevsk State Technical 
University

\vspace*{6pt}

\noindent
\textbf{Pavlov Igor V.} (b.\ 1945)~---  Doctor of Science in physics and mathematics,
professor, Bauman Moscow State Technical University

\vspace*{6pt}

%\columnbreak

\noindent
\textbf{Rozenberg Igor N.} (b.\ 1965)~--- Doctor of Science in technology, 
First Deputy Director General, Research \& Design Institute for Information 
Technology, Signalling and Telecommunications on Railway Transport (JSC NIIAS)

\vspace*{6pt}


\noindent
\textbf{Semenov Konstantin K.} (b.\ 1986)~--- MPhil, 
associate professor, St.\ Petersburg State Polytechnical University

\vspace*{6pt}

\noindent
\textbf{Sharnin Mikhail M.} (b.\ 1959)~--- Candidate of Science (PhD) 
in technology, senior scientist, Institute of Informatics Problems, Russian Academy of Sciences

\vspace*{6pt}

\noindent 
\textbf{Shestakov Oleg V.} (b.\ 1976)~--- Candidate of Science (PhD) in physics and mathematics,
associate professor, Department of Mathematical Statistics, Faculty of Computational Mathematics and Cybernetics,
M.\,V.~Lomonosov Moscow State University; senior scientist, Institute of Informatics Problems, 
Russian Academy of Sciences

\vspace*{6pt}

\noindent
\textbf{Stupnikov Sergey A.} (b.\ 1978)~--- Candidate of Science (PhD) in technology, 
senior scientist, Institute of Informatics Problems, Russian Academy of Sciences 

\vspace*{6pt}

\noindent
\textbf{Umansky Vladimir I.} (b.\ 1954)~--- Candidate of Science (PhD) in technology, 
Director General, ``IntechGeoTrans'' Closed Joint Stock Company

\vspace*{6pt}

\noindent
\textbf{Zhevnerchuk Dmitry V.} (b.\ 1978)~--- Candidate of Science (PhD) in technology, 
associate professor, Tchaikovsky Technological Institute, Branch of the Izhevsk State 
Technical University

%\vspace*{6pt}

\def\leftfootline{\small{\textbf{\thepage}
\hfill ИНФОРМАТИКА И ЕЁ ПРИМЕНЕНИЯ\ \ \ том~6\ \ \ выпуск~2\ \ \ 2012}
}%
 \def\rightfootline{\small{ИНФОРМАТИКА И ЕЁ ПРИМЕНЕНИЯ\ \ \ том~6\ \ \ выпуск~2\ \ \ 2012
\hfill \textbf{\thepage}}}



%\thispagestyle{myheadings}

\end{multicols}
\newpage


\vspace*{-60pt} {%\small 
{%\baselineskip=10.65pt
\section*{Правила подготовки рукописей статей для публикации в журнале
<<Информатика и её применения>>}

\thispagestyle{empty}

 Журнал <<Информатика и её применения>> публикует
теоретические, обзорные и дискуссионные статьи, посвященные научным
исследованиям и разработкам в области информатики и ее приложений. Журнал
издается на русском языке. По специальному решению редколлегии отдельные статьи,
в виде исключения, могут печататься на английском языке.
Тематика журнала охватывает следующие направления:
\begin{itemize}
\item теоретические основы информатики;
\item математические методы исследования сложных систем и процессов;
\item информационные системы и сети;
\item информационные технологии;
\item архитектура и программное
обеспечение вычислительных комплексов и сетей.
\end{itemize}
\begin{enumerate}
\item В журнале печатаются результаты, ранее не
опубликованные и не предназначенные к одновременной публикации в других
изданиях. Публикация не должна нарушать закон об авторских правах. Направляя
свою рукопись в редакцию, авторы автоматически передают учредителям и
редколлегии неисключительные права на издание данной статьи на русском языке и
на ее распространение в России и за рубежом. При этом за авторами сохраняются
все права как собственников данной рукописи. В связи с этим авторами должно
быть представлено в редакцию письмо в следующей форме:
Соглашение о передаче права на публикацию:

\textit{<<Мы, нижеподписавшиеся, авторы рукописи <<$\qquad\qquad$>>, передаем
учредителям и редколлегии журнала <<Информатика и её применения>>
неисключительное право опубликовать данную рукопись статьи на русском языке как
в печатной, так и в электронной версиях журнала. Мы подтверждаем, что данная
публикация не нарушает авторского права других лиц или организаций. Подписи
авторов: (ф.\,и.\,о., дата, адрес)>>.}

Указанное соглашение может быть представлено как в бумажном виде, так и в виде 
отсканированной копии (с подписями авторов).

Редколлегия вправе запросить у авторов экспертное заключение о возможности
опубликования представленной статьи в открытой печати.
\item Статья
подписывается всеми авторами. На отдельном листе представляются данные автора
(или всех авторов): фамилия, полные имя и отчество, телефон, факс, e-mail,
почтовый адрес. Если работа выполнена несколькими авторами, указывается фамилия
одного из них, ответственного за переписку с редакцией.
\item Редакция журнала
осуществляет самостоятельную экспертизу присланных статей. Возвращение рукописи
на доработку не означает, что статья уже принята к печати. Доработанный вариант
с ответом на замечания рецензента необходимо прислать в редакцию.
\item Решение
редакционной коллегии о принятии статьи к печати или ее отклонении сообщается
авторам. Редколлегия не обязуется направлять рецензию авторам отклоненной
статьи.
\item Корректура статей высылается авторам для просмотра. Редакция
просит авторов присылать свои замечания в кратчайшие сроки.
\item При
подготовке рукописи в MS Word рекомендуется использовать следующие настройки.
Параметры страницы: формат~--- А4; ориентация~--- книжная; поля (см): внутри~---
2,5, снаружи~--- 1,5, сверху~--- 2, снизу~--- 2, от края до нижнего
колонтитула~--- 1,3. Основной текст: стиль~--- <<Обычный>>: шрифт Times New
Roman, размер 14~пунктов, абзацный отступ~--- 0,5~см, 1,5 интервала,
выравнивание~--- по ширине. Рекомендуемый объем рукописи~--- не свыше
25~страниц указанного формата. Ознакомиться с шаблонами, содержащими примеры
оформления, можно по адресу в Интернете:
\textsf{http://www.ipiran.ru/journal/template.doc}.
\item К рукописи, предоставляемой в 2-х
экземплярах, обязательно прилагается электронная версия статьи (как правило, в
форматах MS WORD (.doc) или \LaTeX\  (.tex), а также~--- дополнительно~--- в
формате .pdf) на дискете, лазерном диске или по электронной почте. Сокращения
слов, кроме стандартных, не применяются. Все страницы рукописи должны быть
пронумерованы.
\item Статья должна содержать следующую информацию на русском и
английском языках: название, Ф.И.О.\ авторов, места работы авторов и их
электронные адреса,
подробные сведения об авторах, оформленные в соответствии с форматом, определяемым файлами

\noindent
{\sf http://www.ipiran.ru/journal/issues/2011\_05\_01/authors.asp} и 

\noindent
{\sf http://www.ipiran.ru/journal/issues/2011\_01\_eng/authors.asp},

\noindent
аннотация (не более 100~слов), ключевые слова. Ссылки на
литературу в тексте статьи нумеруются (в квадратных скобках) и располагаются в
порядке их первого упоминания. В~списке литературы не должно быть позиций, на 
которые нет ссылки в тексте статьи.
Все фамилии авторов, заглавия статей, названия
книг, конференций и~т.\,п.\ даются на языке оригинала, если этот язык
использует кириллический или латинский алфавит.
\item Присланные в редакцию
материалы авторам не возвращаются.
\item При отправке файлов по электронной
почте просим придерживаться следующих правил:
\begin{itemize}
\item указывать в поле subject (тема) название журнала и фамилию автора;
\item использовать attach (присоединение);
\item в случае больших объемов информации возможно
использование общеизвестных архиваторов (ZIP, RAR);
\item в состав электронной версии статьи должны входить: файл, содержащий текст статьи, и файл(ы),
содержащий(е) иллюстрации.
\end{itemize}
\item Журнал <<Информатика и её применения>> является некоммерческим изданием. 
Плата за публикацию с авторов не взимается, гонорар авторам не выплачивается.
\end{enumerate}
\thispagestyle{empty}

\medskip
\noindent
\textbf{Адрес редакции:} Москва 119333,
ул.~Вавилова, д.~44, корп.~2, ИПИ РАН\\
\hphantom{\textbf{Адрес редакции:} }Тел.: +7 (499) 135-86-92\ \
Факс:  +7 (495) 930-45-05\ \  E-mail:   rust@ipiran.ru }

\vfill
\begin{center}


Технический редактор Л. Кокушкина\\
Выпускающий редактор Т. Торжкова\\
Художественный редактор М. Седакова\\
Сдано в набор 11.01.12. Подписано в печать 02.03.12. Формат 60 х 84 / 8\\
Бумага офсетная. Печать цифровая. Усл.-печ. л. 19,0. Уч.-изд. л. 24,2. Тираж 100 экз.\\
\ \\
Заказ №\,279\\
\ \\
Издательство <<ТОРУС ПРЕСС>>, Москва 119991, ул. Косыгина, д.~4\\
torus@torus-press.ru; http://www.torus-press.ru\\
\ \\
Отпечатано в Академиздатцентре <<Наука>> РАН с готовых файлов\\
Москва 121099, Шубинский пер., д.~6\\
\end{center}

\end{document}


%\tableofcontents

%\end{document}

\def\stat{cont}
{%\hrule\par
%\vskip 7pt % 7pt
\raggedleft\Large \bf%\baselineskip=3.2ex
А\,В\,Т\,О\,Р\,С\,К\,И\,Й\ \ У\,К\,А\,З\,А\,Т\,Е\,Л\,Ь\ \ З\,А\ \ 2\,0\,1\,0 г. \vskip 17pt
    \hrule
    \par
\vskip 21pt plus 6pt minus 3pt }

\label{st\stat}

\def\tit{\ }

\def\aut{\ }
\def\auf{\ }

\def\leftkol{\ } % ENGLISH ABSTRACTS}

\def\rightkol{\ } %АВТОРСКИЙ УКАЗАТЕЛЬ ЗА 2010 г.} %ENGLISH ABSTRACTS}

\titele{\tit}{\aut}{\auf}{\leftkol}{\rightkol}

\vspace*{-12pt}

{\tabcolsep=3pt
\begin{tabular}{p{388pt}rr}
&\textbf{Выпуск} & \textbf{Стр.}\\[6pt]
\hangindent=23pt\noindent\textbf{Арутюнян~А.\,Р.} Моделирование влияния деформаций отпечатков пальцев на 
точность\linebreak
\vspace*{-12pt}\\
\hspace*{23pt}дактилоскопической идентификации$\dotfill$&1&51\\
\hangindent=23pt\noindent\textbf{Архипов~О.\,П., Зыкова~З.\,П.} Интеграция гетерогенной информации о цветных 
пикселях\linebreak
\vspace*{-12pt}\\
\hspace*{23pt}и их цветовосприятии$\dotfill$&4&15\\
\hangindent=23pt\noindent\textbf{Баранов~С.\,И., Френкель~С.\,Л., Захаров~В.\,Н.} Полуформальная верификация 
цифрового устройства с конвейером, основанная на использовании алгоритмических машин\linebreak
\vspace*{-12pt}\\
\hspace*{23pt}состояния$\dotfill$&4&49\\
\textbf{Бекетова~И.\,В.} см.~Каратеев~С.\,Л.&&\\
\textbf{Белоусов~В.\,В.} см.~Синицын~И.\,Н.&&\\
\hangindent=23pt\noindent\textbf{Бенинг~В.\,Е., Королев~Р.\,А.} О предельном поведении мощностей критериев в 
случае\linebreak
\vspace*{-12pt}\\
\hspace*{23pt}распределения Лапласа$\dotfill$&2&63\\
\hangindent=23pt\noindent\textbf{Бенинг~В.\,Е., Сипина~А.\,В.} Асимптотическое разложение для мощности 
критерия,\linebreak
\vspace*{-12pt}\\
\hspace*{23pt}основанного на выборочной медиане, в случае распределения Лапласа$\dotfill$&1&18\\
\textbf{Бондаренко~А.\,В.} см.~Каратеев~С.\,Л.&&\\
\hangindent=23pt\noindent\textbf{Бородина~А.\,В., Морозов~Е.\,В.} Об оценивании асимптотики вероятности 
большого\linebreak
\vspace*{-12pt}\\
\hspace*{23pt}уклонения стационарной регенеративной очереди с одним прибором$\dotfill$&3&29\\
\hangindent=23pt\noindent\textbf{Бунтман~Н.\,В., Минель~Ж.-Л., Ле~Пезан~Д., Зацман~И.\,М.} Типология и 
компьютерное\linebreak
\vspace*{-12pt}\\
\hspace*{23pt}моделирование трудностей перевода$\dotfill$&3&77\\
\textbf{Визильтер~Ю.\,В.} см.~Каратеев~С.\,Л.&&\\
\hangindent=23pt\noindent\textbf{Гавриленко~С.\,В.} Оценки скорости сходимости распределений случайных сумм с 
безгранично делимыми индексами к нормальному закону$\dotfill$&4&81\\
\hangindent=23pt\noindent\textbf{Григорьева~М.\,Е., Шевцова~И.\,Г.} Уточнение неравенства 
Каца--Берри--Эссеена$\dotfill$&2&75\\
\hangindent=23pt\noindent\textbf{Грушо~А.\,А., Грушо~Н.\,А., Тимонина~Е.\,Е.} Поиск конфликтов в политиках 
безопасности: модель случайных графов$\dotfill$&3&38\\
\textbf{Грушо~Н.\,А.} см.~Грушо~А.\,А.&&\\
\hangindent=23pt\noindent\textbf{Гудков~В.\,Ю.} Математические модели изображения отпечатка пальца на основе 
описания линий$\dotfill$&1&58\\
\textbf{Гуртов~А.\,В.} см.~Лукьяненко~А.\,С.&&\\
\textbf{Желтов~С.\,Ю.} см.~Каратеев~С.\,Л.&&\\
\hangindent=23pt\noindent\textbf{Захаров~А.\,А., Серебряков~В.\,А.} Система управления электронной библиотекой 
LibMeta$\dotfill$&4&2\\
\textbf{Захаров~В.\,Н.} см.~Баранов~С.\,И.&&\\
\textbf{Захарова~Т.\,В.} см.~Матвеева~С.\,С.&&\\
\hangindent=23pt\noindent\textbf{Зацаринный~А.\,А., Чупраков~К.\,Г.} Некоторые аспекты выбора технологии для 
постро-\linebreak
\vspace*{-12pt}\\
\hspace*{23pt}ения систем отображения информации ситуационного центра$\dotfill$&3&59\\
\textbf{Зацман~И.\,М.} см.~Бунтман~Н.\,В.&&\\
\hangindent=23pt\noindent\textbf{Зейфман~А.\,И., Коротышева~А.\,В., Сатин~Я.\,А., Шоргин~С.\,Я.} Об 
устойчивости нестаци-\linebreak
\vspace*{-12pt}\\
\hspace*{23pt}онарных систем обслуживания с катастрофами$\dotfill$&3&9\\
\textbf{Зыкова~З.\,П.} см.~Архипов~О.\,П.&&\\
\hangindent=23pt\noindent\textbf{Илюшин~Г.\,Я., Соколов~И.\,А.} Организация управляемого доступа пользователей 
к\linebreak
\vspace*{-12pt}\\
\hspace*{23pt}разнородным ведомственным информационным ресурсам$\dotfill$&1&24\\
\hangindent=23pt\noindent\textbf{Кавагучи~Ю., Ульянов~В.\,В., Фуджикоши~Я.} Приближения для статистик, 
описывающих\linebreak
\vspace*{-12pt}\\
\hspace*{23pt}геометрические свойства данных большой размерности, с оценками 
ошибок$\dotfill$&1&12\\
\hangindent=23pt\noindent\textbf{Каратеев~С.\,Л., Бекетова~И.\,В., Ососков~М.\,В., Князь~В.\,А., 
Визильтер~Ю.\,В., Бондаренко~А.\,В., Желтов~С.\,Ю.} Автоматизированный контроль 
качества цифровых\linebreak
\vspace*{-12pt}\\
\hspace*{23pt}изображений для персональных документов$\dotfill$&1&65\\
\end{tabular}
}

\pagebreak

\def\leftkol{АВТОРСКИЙ УКАЗАТЕЛЬ ЗА 2010 г.} % ENGLISH ABSTRACTS}

\def\rightkol{АВТОРСКИЙ УКАЗАТЕЛЬ ЗА 2010 г.} %ENGLISH ABSTRACTS}

{\tabcolsep=3pt
\begin{tabular}{p{388pt}rr}
&\textbf{Выпуск} & \textbf{Стр.}\\[3pt]
\hangindent=23pt\noindent\textbf{Козеренко~Е.\,Б.} Лингвистические фильтры в статистических моделях машинного\linebreak
\vspace*{-12pt}\\
\hspace*{23pt}перевода$\dotfill$&2&83\\
\hangindent=23pt\noindent\textbf{Козеренко~Е.\,Б., Кузнецов~И.\,П.} Когнитивно-лингвистические представления в 
систе-\linebreak
\vspace*{-12pt}\\
\hspace*{23pt}мах обработки текстов$\dotfill$&3&69\\
\textbf{Князь~В.\,А.} см.~Каратеев~С.\,Л.&&\\
\hangindent=23pt\noindent\textbf{Колесников~А.\,В., Солдатов~С.\,А.} Алгоритм координации для гибридной 
интеллектуальной системы решения сложной задачи оперативно-производственного\linebreak
\vspace*{-12pt}\\
\hspace*{23pt}планирования$\dotfill$&4&61\\
\hangindent=23pt\noindent\textbf{Коновалов~М.\,Г.} О планировании потоков в системах вычислительных 
ресурсов$\dotfill$&2&3\\
\textbf{Конушин~А.\,С.} см.~Конушин~В.\,С.&&\\
\hangindent=23pt\noindent\textbf{Конушин~В.\,С., Кривовязь~Г.\,Р., Конушин~А.\,С.} Алгоритм распознавания людей 
в видео-\linebreak
\vspace*{-12pt}\\
\hspace*{23pt}последовательности по одежде$\dotfill$&1&74\\
\textbf{Корепанов~Э.\, Р.} см.~Синицын~И.\,Н.&&\\
\textbf{Королев~В.\,Ю.} см.~Соколов~И.\,А.&&\\
\textbf{Королев~Р.\,А.} см.~Бенинг~В.\,Е.&&\\
\textbf{Коротышева~А.\,В.} см.~Зейфман~А.\,И.&&\\
\hangindent=23pt\noindent\textbf{Кривенко~М.\,П.} Непараметрическое оценивание элементов байесовского 
клас\-си-\linebreak
\vspace*{-12pt}\\
\hspace*{23pt}фикатора$\dotfill$&2&13\\
\textbf{Кривовязь~Г.\,Р.} см.~Конушин~В.\,С.&&\\
\textbf{Крылов~А.\,С.} см.~Павельева~Е.\,А.&&\\
\hangindent=23pt\noindent\textbf{Крылов~В.\,А.} Моделирование и классификация многоканальных дистанционных\linebreak
\vspace*{-12pt}\\
\hspace*{23pt}изображений с использованием копул$\dotfill$&4&34\\
\hangindent=23pt\noindent\textbf{Крючин~О.\,В.} Разработка параллельных эвристических алгоритмов подбора 
весовых\linebreak
\vspace*{-12pt}\\
\hspace*{23pt}коэффициентов искусственной нейтронной сети$\dotfill$&2&53\\
\hangindent=23pt\noindent\textbf{Кудрявцев~А.\,А., Шоргин~С.\,Я.} Байесовские модели массового обслуживания и 
надеж-\linebreak
\vspace*{-12pt}\\
\hspace*{23pt}ности: характеристики среднего числа заявок в системе $M\vert M \vert 1\vert 
\infty$$\dotfill$&3&16\\
\hangindent=23pt\noindent\textbf{Кузнецов~А.\,А.} Связь между временными и структурно-топологическими 
характери-\linebreak
\vspace*{-12pt}\\
\hspace*{23pt}стиками диаграмм ритма сердца здоровых людей$\dotfill$&4&39\\
\textbf{Кузнецов~И.\,П.} см.~Козеренко~Е.\,Б.&&\\
\textbf{Ле~Пезан~Д.} см.~Бунтман~Н.\,В.&&\\
\hangindent=23pt\noindent\textbf{Лукьяненко~А.\,С., Морозов~Е.\,В., Гуртов~А.\,В.} Анализ сетевого протокола с общей 
функ-\linebreak
\vspace*{-12pt}\\
\hspace*{23pt}цией расширения окна передачи сообщения при конфликтах$\dotfill$&2&46\\
\hangindent=23pt\noindent\textbf{Лямин~О.\,О.} О предельном поведении мощностей критериев в случае обобщенного\linebreak
\vspace*{-12pt}\\
\hspace*{23pt}распределения Лапласа$\dotfill$&3&47\\
\hangindent=23pt\noindent\textbf{Маркин~А.\,В., Шестаков~О.\,В.} Асимптотики оценки риска при пороговой 
обработке\linebreak
\vspace*{-12pt}\\
\hspace*{23pt}вейвлет-вейглет коэффициентов в задаче томографии$\dotfill$&2&36\\
\hangindent=23pt\noindent\textbf{Матвеева~С.\,С., Захарова~Т.\,В.} Сети массового обслуживания с наименьшей 
длиной\linebreak
\vspace*{-12pt}\\
\hspace*{23pt}очереди$\dotfill$&3&22\\
\hangindent=23pt\noindent\textbf{Матюшенко~С.\,И.} Стационарные характеристики двухканальной системы 
обслужива-\linebreak
\vspace*{-12pt}\\
\hspace*{23pt}ния с переупорядочиванием заявок и распределениями фазового типа$\dotfill$&4&68\\
\textbf{Минель~Ж.-Л.} см.~Бунтман~Н.\,В.&&\\
\textbf{Морозов~Е.\,В.} см.~Бородина~А.\,В.&&\\
\textbf{Морозов~Е.\,В.} см.~Лукьяненко~А.\,С.&&\\
\textbf{Ососков~М.\,В.} см.~Каратеев~С.\,Л.&&\\
\hangindent=23pt\noindent\textbf{Павельева~Е.\,А., Крылов~А.\,С.} Поиск и анализ ключевых точек радужной 
оболочки\linebreak
\vspace*{-12pt}\\
\hspace*{23pt}глаза методом преобразования Эрмита$\dotfill$&1&79\\
\textbf{Печинкин~А.\,В.} см.~Френкель~С.\,Л.,&&\\
\hangindent=23pt\noindent\textbf{Протасов~В.\,И.} Составление субъективного портрета с использованием 
эволюционно-\linebreak
\vspace*{-12pt}\\
\hspace*{23pt}го морфинга и квалиметрия метода$\dotfill$&1&83\\
\hangindent=23pt\noindent\textbf{Рудаков~К.\,В., Торшин~И.\,Ю.} Вопросы разрешимости задачи распознавания 
вторичной\linebreak
\vspace*{-12pt}\\
\hspace*{23pt}структуры белка$\dotfill$&2&25\\
\textbf{Сатин~Я.\,А.} см.~Зейфман~А.\,И.&&\\
\hangindent=23pt\noindent\textbf{Сейфуль-Мулюков~Р.\,Б.} Нефть как носитель информации о своем 
происхождении,\linebreak
\vspace*{-12pt}\\
\hspace*{23pt}структуре и эволюции$\dotfill$&1&41\\
\end{tabular}
}

{\tabcolsep=3pt
\begin{tabular}{p{388pt}rr}
&\textbf{Выпуск} & \textbf{Стр.}\\[6pt]
\textbf{Семендяев~Н.\,Н.} см.~Синицын~И.\,Н.&&\\
\textbf{Серебряков~В.\,А.} см.~Захаров~А.\,А.&&\\
\textbf{Синицын~В.\,И.} см.~Синицын~И.\,Н.&&\\
\hangindent=23pt\noindent\textbf{Синицын~И.\,Н., Синицын~В.\,И., Корепанов~Э.\, Р., Белоусов~В.\,В., 
Семендяев~Н.\,Н.} Оперативное построение информационных моделей движения полюса 
Земли\linebreak
\vspace*{-12pt}\\
\hspace*{23pt}методами линейных и линеаризованных фильтров$\dotfill$&1&2\\
\textbf{Сипина~А.\,В.} см.~Бенинг~В.\,Е.&&\\
\hangindent=23pt\noindent\textbf{Соколов~И.\,А.} О работах заслуженного деятеля науки Российской Федерации 
И.\,Н.~Синицына в области информационных технологий и автоматизации (к 70-летию\linebreak
\vspace*{-12pt}\\
\hspace*{23pt}со дня рождения)$\dotfill$&3&84\\
\textbf{Соколов~И.\,А.} см.~Илюшин~Г.\,Я.&&\\
\hangindent=23pt\noindent\textbf{Соколов~И.\,А., Королев~В.\,Ю.} Предисловие$\dotfill$&2&2\\
\textbf{Солдатов~С.\,А.} см.~Колесников~А.\,В.&&\\
\hangindent=23pt\noindent\textbf{Степанов~С.\,Ю.} Использование координатного метода фрагментации 
коммутаторной\linebreak
\vspace*{-12pt}\\
\hspace*{23pt}нейронной сети для сокращения трафика$\dotfill$&2&57\\
\textbf{Тимонина~Е.\,Е.} см.~Грушо~А.\,А.&&\\
\textbf{Торшин~И.\,Ю.} см.~Рудаков~К.\,В.&&\\
\textbf{Ульянов~В.\,В.} см.~Кавагучи~Ю.&&\\
\textbf{Фазекаш~И.} см.~Чупрунов~А.\,Н.&&\\
\textbf{Френкель~С.\,Л.} см.~Баранов~С.\,И.&&\\
\hangindent=23pt\noindent\textbf{Френкель~С.\,Л., Печинкин~А.\,В.} Оценка времени самовосстановления в 
цифровых\linebreak
\vspace*{-12pt}\\
\hspace*{23pt}системах после сбоев, вызываемых переходными помехами$\dotfill$&3&2\\
\textbf{Фуджикоши~Я.} см.~Кавагучи~Ю.&&\\
\hangindent=23pt\noindent\textbf{Цискаридзе~А.\,К.} Математическая модель и метод восстановления позы человека 
по\linebreak
\vspace*{-12pt}\\
\hspace*{23pt}стереопаре силуэтных изображений$\dotfill$&4&27\\
\hangindent=23pt\noindent\textbf{Чупраков~К.\,Г.} К вопросу о размещении коллективных средств отображения в 
ситуа-\linebreak
\vspace*{-12pt}\\
\hspace*{23pt}ционном зале с заданными параметрами$\dotfill$&4&89\\
\textbf{Чупраков~К.\,Г.} см.~Зацаринный~А.\,А.&&\\
\hangindent=23pt\noindent\textbf{Чупрунов~А.\,Н., Фазекаш~И.} Законы повторного логарифма для числа 
безошибочных\linebreak
\vspace*{-12pt}\\
\hspace*{23pt}блоков при помехоустойчивом кодировании$\dotfill$&3&42\\
\textbf{Шевцова~И.\,Г.} см.~Григорьева~М.\,Е.&&\\
\hangindent=23pt\noindent\textbf{Шестаков~О.\,В.} Аппроксимация распределения оценки риска пороговой 
обработки вейвлет-коэффициентов нормальным распределением при использовании 
выбо-\linebreak
\vspace*{-12pt}\\
\hspace*{23pt}рочной дисперсии$\dotfill$&4&73\\
\textbf{Шестаков~О.\,В.} см.~Маркин~А.\,В.&&\\
\textbf{Шоргин~С.\,Я.} см.~Зейфман~А.\,И.&&\\
\textbf{Шоргин~С.\,Я.} см.~Кудрявцев~А.\,А.&&\\
\end{tabular}
}

%\thispagestyle{myheadings}
\def\leftfootline{\small{\textbf{\thepage}
\hfill ИНФОРМАТИКА И ЕЁ ПРИМЕНЕНИЯ\ \ \ том~4\ \ \ выпуск~4\ \ \ 2010}
}%
 \def\rightfootline{\small{ИНФОРМАТИКА И ЕЁ ПРИМЕНЕНИЯ\ \ \ том~4\ \ \ выпуск~4\ \ \ 2010
 \hfill \textbf{\thepage}}}
 \label{end\stat}


%Том 10 Выпуск 1-4 Год 2016

\def\stat{cont-e}
{%\hrule\par
%\vskip 7pt % 7pt
\raggedleft\Large \bf%\baselineskip=3.2ex
2\,0\,1\,6\ \ A\,U\,T\,H\,O\,R\ \ I\,N\,D\,E\,X \vskip 17pt
 \hrule
 \par
\vskip 21pt plus 6pt minus 3pt }

\label{st\stat}

\def\tit{\ }

\def\aut{\ }
\def\auf{\ }

\def\leftkol{\ } %2016 AUTHOR INDEX} % ENGLISH ABSTRACTS}

\def\rightkol{\ } %2016 AUTHOR INDEX} %ENGLISH ABSTRACTS}

\titele{\tit}{\aut}{\auf}{\leftkol}{\rightkol}

\def\leftfootline{\small{\textbf{\thepage}
\hfill INFORMATIKA I EE PRIMENENIYA~--- INFORMATICS AND APPLICATIONS\ \ \ 2016\
\ \ volume~10\ \ \ issue\ 4}
}%
 \def\rightfootline{\small{INFORMATIKA I EE PRIMENENIYA~--- INFORMATICS AND APPLICATIONS\ \ \ 2016\ \ \ volume~10\ \ \ issue\ 4
\hfill \textbf{\thepage}}}

\vspace*{-12pt}
\vspace*{-18pt}

{\tabcolsep=2.8pt
\begin{tabular}{p{382pt}cc}
&\textbf{Issue} & \textbf{Page}\\[6pt]
\Avtors{Agalarov~M.\,Ya.} see~Agalarov~Ya.\,M.&&\\
\Avtors{Agalarov~Ya.\,M., Agalarov~M.\,Ya., and
Shorgin~V.\,S.} About the optimal threshold of queue\linebreak
\\[-12pt]
\hspace*{23pt}length in a~particular problem of profit maximization
in the $M/G/1$ queuing system&2&70--79\\
\Avtors{Alexeyevsky~D.\,A.} BioNLP ontology extraction from 
a~restricted language corpus with\linebreak
\\[-12pt]
\hspace*{23pt}context-free grammars&1&119--128\\
\Avtors{Andreev~S.\,D.} see~Gaidamaka~Yu.\,V.&&\\
\Avtors{Andreev~S.\,D.} see~Ometov~A.\,Ya.&&\\
\Avtors{Arkhipov~O.\,P., Arkhipov~P.\,O., and Sidorkin~I.\,I.} The
option to create a~local coordinate\linebreak
\\[-12pt]
\hspace*{23pt}system for synchronization of selected images&3&91--97\\
\Avtors{Arkhipov~P.\,O.} see~Arkhipov~O.\,P.&&\\
\Avtors{Belousov~V.\,V.} see~Shnurkov~P.\,V.&&\\
\Avtors{Belousov~V.\,V.} see~Shnurkov~P.\,V.&&\\
\Avtors{Bening~V.\,E.} Calculation of~the~asymptotic deficiency
of~some statistical procedures based\linebreak
\\[-12pt]
\hspace*{23pt}on~samples with~random sizes&4&34--45\\
\Avtors{Borisov~A.\,V., Bosov~A.\,V., and Miller~G.\,B.} Modeling and
monitoring of VoIP connection&2&\hphantom{1}2--13\\
\Avtors{Bosov~A.\,V.} see~Borisov~A.\,V.&&\\
\Avtors{Briukhov~D.\,O.} see~Stupnikov~S.\,A.&&\\
\Avtors{Callaos~N.\,K.\ and Seyful-Mulyukov~R.\,B.} Complexity and
its information content&1&129--139\\
\Avtors{Chertok~A.\,V., Kadaner~A.\,I., Khazeeva~G.\,T., and
Sokolov~I.\,A.} Regime switching detection\linebreak
\\[-12pt]
\hspace*{23pt}for~the~Levy driven
Ornstein--Uhlenbeck process using CUSUM methods&4&46--56\\
\Avtors{Chichagov~V.\,V.} Asymptotic expansions of mean absolute
error of uniformly minimum variance unbiased and maximum likelihood
estimators on the one-parameter exponential\linebreak
\\[-12pt]
\hspace*{23pt}family model of lattice distributions&3&66--76\\
\Avtors{Danishevsky~V.\,I.} see~Kolesnikov A.\,V.&&\\
\Avtors{Fazliev~A.\,Z.} see~Kalinichenko~L.\,A.&&\\
\Avtors{Fedoseev~A.\,A.} What is behind the concept of ``knowledge in
small packages''&3&105--110\\
\Avtors{Gaidamaka~Yu.\,V., Andreev~S.\,D., Sopin~E.\,S.,
Samouylov~K.\,E., and Shorgin~S.\,Ya.} Interference analysis
of~the~device-to-device communications model with~regard to~a~signal\linebreak
\\[-12pt]
\hspace*{23pt}propagation environment&4&\hphantom{1}2--10\\
\Avtors{Gasilov~A.\,V.} see~Yakovlev~O.\,A.&&\\
\Avtors{Goncharov~A.\,V.\ and Strijov~V.\,V.} Metric time series
classification using weighted dynamic\linebreak
\\[-12pt]
\hspace*{23pt}warping relative to centroids of classes&2&36--47\\
\Avtors{Gordov~E.\,P.} see~Kalinichenko~L.\,A.&&\\
\Avtors{Gorshenin~A.\,K.} Concept of online service for stochastic
modeling of real processes&1&72--81\\
\Avtors{Gorshenin~A.\,K.} see~Shnurkov~P.\,V.&&\\
\Avtors{Gorshenin~A.\,K.} see~Shnurkov~P.\,V.&&\\
\Avtors{Grusho~A.\,A., Grusho~N.\,A., Zabezhailo~M.\,I., and
Timonina~E.\,E.} Integration of statistical and\linebreak
\\[-12pt]
\hspace*{23pt}deterministic methods for
analysis of information security&3&2--8\\
\Avtors{Grusho~A.\,A., Zabezhailo~M.\,I., and Zatsarinny~A.\,A.} On
the advanced procedure to reduce\linebreak
\\[-12pt]
\hspace*{23pt}calculation of Galois closures&4&\hphantom{1}96--104\\
\Avtors{Grusho~N.\,A.} see~Grusho~A.\,A.&&\\
\Avtors{Havanskov~V.\,A.} see~Minin~V.\,A.&&\\
\Avtors{Inkova~O.\,Yu.} see~Zatsman~I.\,M.&&\\
\Avtors{Isachenko~R.\,V.\ and Strijov~V.\,V.} Metric learning in
multiclass time series classification\linebreak
\\[-12pt]
\hspace*{23pt}problem&2&48--57\\
\end{tabular}
}
\pagebreak

\def\leftfootline{\small{\textbf{\thepage}
\hfill INFORMATIKA I EE PRIMENENIYA~--- INFORMATICS AND APPLICATIONS\ \ \ 2016\
\ \ volume~10\ \ \ issue\ 4}
}%
 \def\rightfootline{\small{INFORMATIKA I EE PRIMENENIYA~---
INFORMATICS AND APPLICATIONS\ \ \ 2016\ \ \ volume~10\ \ \ issue\ 4
\hfill \textbf{\thepage}}}

\def\leftkol{2016 AUTHOR INDEX} % ENGLISH ABSTRACTS}

\def\rightkol{2016 AUTHOR INDEX} %ENGLISH ABSTRACTS}


{\tabcolsep=2.83pt
\begin{tabular}{p{382pt}cc}
&\textbf{Issue} & \textbf{Page}\\[6pt]
\Avtors{Kadaner~A.\,I.} see~Chertok~A.\,V.&&\\[.255pt]
\Avtors{Kalinichenko~L.\,A., Volnova~A.\,A., Gordov~E.\,P.,
Kiselyova~N.\,N., Kovaleva~D.\,A., Malkov~O.\,Yu., Okladnikov~I.\,G.,
Podkolodnyy~N.\,L., Pozanenko~A.\,S., Ponomareva~N.\,V.,
Stupnikov~S.\,A.,} \textbf{and Fazliev~A.\,Z.} Data access challenges for data
intensive\linebreak
\\[-12pt]
\hspace*{23pt}research in Russia&1& 2--22\\[.255pt]
\Avtors{Karasikov~M.\,E.\ and Strijov~V.\,V.} Feature-based
time-series classification&4&121--131\\[.255pt]
\Avtors{Khazeeva~G.\,T.} see~Chertok~A.\,V.&&\\[.255pt]
\Avtors{Khokhlov~Yu.\,S.} Multivariate fractional Levy motion and its
applications&2&\hphantom{1}98--106\\[.255pt]
\Avtors{Kirikov~I.\,A., Kolesnikov~A.\,V., Listopad~S.\,V., and
Rumovskaya~S.\,B.} Fine-grained hybrid\linebreak
\\[-12pt]
\hspace*{23pt}intelligent systems. Part 2:
Bidirectional hybridization&1&\hphantom{1}96--105\\[.255pt]
\Avtors{Kirikov~I.\,A., Kolesnikov~A.\,V., Listopad~S.\,V., and
Rumovskaya~S.\,B.} ``Virtual council''~---\linebreak
\\[-12pt]
\hspace*{23pt}source environment
supporting complex diagnostic decision making&3&81--90\\[.255pt]
\Avtors{Kiselyova~N.\,N.} see~Kalinichenko~L.\,A.&&\\[.255pt]
\Avtors{Kolesnikov A.\,V., Listopad~S.\,V., Rumovskaya~S.\,B., and
Danishevsky~V.\,I.} Informal axiomatic\linebreak
\\[-12pt]
\hspace*{23pt}theory of~the~role visual models&4&114--120\\[.255pt]
\Avtors{Kolesnikov~A.\,V.} see~Kirikov~I.\,A.&&\\[.255pt]
\Avtors{Kolesnikov~A.\,V.} see~Kirikov~I.\,A.&&\\[.255pt]
\Avtors{Kolin~K.\,K.} Humanitarian aspects of information
security&3&111--121\\[.255pt]
\Avtors{Konovalov~M.\,G.\ and Razumchik~R.\,V.} Dispatching
to~two parallel nonobservable queues using\linebreak
\\[-12pt]
\hspace*{23pt}only static
information&4&57--67\\[.255pt]
\Avtors{Korchagin~A.\,Yu.} see~Korolev~V.\,Yu.&&\\[.255pt]
\Avtors{Korchagin~A.\,Yu.} see~Korolev~V.\,Yu.&&\\[.255pt]
\Avtors{Korepanov~E.\,R.} see~Sinitsyn~I.\,N.&&\\[.255pt]
\Avtors{Korepanov~E.\,R.} see~Sinitsyn~I.\,N.&&\\[.255pt]
\Avtors{Korolev~V.\,Yu., Korchagin~A.\,Yu., and Zeifman~A.\,I.} The
Poisson theorem for Bernoulli trials\linebreak
\\[-12pt]
\hspace*{23pt}with~a~random probability
of~success and~a~discrete analog of~the~Weibull distribution&4&11--20\\[.255pt]
\Avtors{Korolev~V.\,Yu., Zeifman~A.\,I., and Korchagin~A.\,Yu.}
Asymmetric Linnik distributions as~limit\linebreak
\\[-12pt]
\hspace*{23pt}laws for~random sums
of~independent random variables with~finite variances&4&21--33\\[.255pt]
\Avtors{Koucheryavy~E.\,A.} see~Ometov~A.\,Ya.&&\\[.255pt]
\Avtors{Kovaleva~D.\,A.} see~Kalinichenko~L.\,A.&&\\[.255pt]
\Avtors{Kovalyov~S.\,P.} Metaprogramming to increase
manufacturability of large-scale software-\linebreak
\\[-12pt]
\hspace*{23pt}intensive systems&1&56--66\\[.255pt]
\Avtors{Krivenko~M.\,P.} Significance tests of feature selection for
classification&3&32--40\\[.255pt]
\Avtors{Kruzhkov~M.\,G.} see~Zalizniak~Anna~A.&&\\[.255pt]
\Avtors{Kruzhkov~M.\,G.} see~Zatsman~I.\,M.&&\\[.255pt]
\Avtors{Kudryavtsev~A.\,A.} Bayesian queueing and reliability models:
\textit{A~priori} distributions with\linebreak
\\[-12pt]
\hspace*{23pt}compact support&1&67--71\\[.255pt]
\Avtors{Kudryavtsev~A.\,A.} Characteristics dependent on the balance
coefficient in Bayesian models\linebreak
\\[-12pt]
\hspace*{23pt}with compact support of \textit{a priori}
distributions&3&77--80\\[.255pt]
\Avtors{Kudryavtsev~A.\,A.\ and Palionnaia~S.\,I.} Bayesian recurrent
model of reliability growth:\linebreak
\\[-12pt]
\hspace*{23pt}Parabolic distribution of parameters&2&80--83\\[.255pt]
\Avtors{Kudryavtsev~A.\,A.\ and Titova~A.\,I.} Bayesian queuing
and~reliability models: Degenerate-\linebreak
\\[-12pt]
\hspace*{23pt}Weibull case&4&68--71\\[.255pt]
\Avtors{Leontyev~N.\,D.\ and Ushakov~V.\,G.} Analysis of a queueing
system with autoregressive arrivals\linebreak
\\[-12pt]
\hspace*{23pt}and nonpreemptive priority&3&15--22\\[.255pt]
\Avtors{Listopad~S.\,V.} see~Kirikov~I.\,A.&&\\[.255pt]
\Avtors{Listopad~S.\,V.} see~Kirikov~I.\,A.&&\\[.255pt]
\Avtors{Listopad~S.\,V.} see~Kolesnikov A.\,V.&&\\[.255pt]
\Avtors{Malkov~O.\,Yu.} see~Kalinichenko~L.\,A.&&\\[.255pt]
\Avtors{Markov~A.\,S., Monakhov~M.\,M., and
Ulyanov~V.\,V.} Generalized Cornish--Fisher expansions\linebreak
\\[-12pt]
\hspace*{23pt}for distributions of statistics based on samples
of random size&2&84--91\\[.255pt]
\Avtors{Melnikov~A.\,K.\ and Ronzhin~A.\,F.} Generalized statistical
method of~text analysis based\linebreak
\\[-12pt]
\hspace*{23pt}on~calculation of~probability distributions
of~statistical values&4&89--95\\
\end{tabular}
}
\pagebreak

\def\leftfootline{\small{\textbf{\thepage}
\hfill INFORMATIKA I EE PRIMENENIYA~--- INFORMATICS AND APPLICATIONS\ \ \ 2016\
\ \ volume~10\ \ \ issue\ 4}
}%
 \def\rightfootline{\small{INFORMATIKA I EE PRIMENENIYA~---
INFORMATICS AND APPLICATIONS\ \ \ 2016\ \ \ volume~10\ \ \ issue\ 4
\hfill \textbf{\thepage}}}

\def\leftkol{2016 AUTHOR INDEX} % ENGLISH ABSTRACTS}

\def\rightkol{2016 AUTHOR INDEX} %ENGLISH ABSTRACTS}


{\tabcolsep=3pt
\begin{tabular}{p{381pt}cc}
&\textbf{Issue} & \textbf{Page}\\[6pt]
\Avtors{Meykhanadzhyan~L.\,A.} Stationary characteristics of the finite
capacity queueing system with\linebreak
\\[-12pt]
\hspace*{23pt}inverse service order and generalized
probabilistic priority&2&123--131\\[.23pt]
\Avtors{Miller~G.\,B.} see~Borisov~A.\,V.&&\\[.23pt]
\Avtors{Minin~V.\,A., Zatsman~I.\,M., Havanskov~V.\,A., and
Shubnikov~S.\,K.} Intensity of citation of scientific publications in
inventions on information and computer technologies patented\linebreak
\\[-12pt]
\hspace*{23pt}in Russia by domestic and foreign applicants&2&107--122\\[.23pt]
\Avtors{Monakhov~M.\,M.} see~Markov~A.\,S.&&\\[.23pt]
\Avtors{Naumov~V.\,A.\ and Samouylov~K.\,E.} On relationship
between queuing systems with resources\linebreak
\\[-12pt]
\hspace*{23pt}and Erlang networks&3&\hphantom{1}9--14\\[.23pt]
\Avtors{Okladnikov~I.\,G.} see~Kalinichenko~L.\,A.&&\\[.23pt]
\Avtors{Ometov~A.\,Ya., Andreev~S.\,D., Turlikov~A.\,M., and
Koucheryavy~E.\,A.} Performance analysis of\linebreak
\\[-12pt]
\hspace*{23pt}a wireless data
aggregation system with contention for contemporary sensor
networks&3&23--31\\[.23pt]
\Avtors{Palionnaia~S.\,I.} see~Kudryavtsev~A.\,A.&&\\[.23pt]
\Avtors{Podkolodnyy~N.\,L.} see~Kalinichenko~L.\,A.&&\\[.23pt]
\Avtors{Ponomareva~N.\,V.} see~Kalinichenko~L.\,A.&&\\[.23pt]
\Avtors{Popkova~N.\,A.} see~Zatsman~I.\,M.&&\\[.23pt]
\Avtors{Pozanenko~A.\,S.} see~Kalinichenko~L.\,A.&&\\[.23pt]
\Avtors{Razumchik~R.\,V.} see~Konovalov~M.\,G.&&\\[.23pt]
\Avtors{Ronzhin~A.\,F.} see~Melnikov~A.\,K.&&\\[.23pt]
\Avtors{Rumovskaya~S.\,B.} see~Kirikov~I.\,A.&&\\[.23pt]
\Avtors{Rumovskaya~S.\,B.} see~Kirikov~I.\,A.&&\\[.23pt]
\Avtors{Rumovskaya~S.\,B.} see~Kolesnikov A.\,V.&&\\[.23pt]
\Avtors{Samouylov~K.\,E.} see~Gaidamaka~Yu.\,V.&&\\[.23pt]
\Avtors{Samouylov~K.\,E.} see~Naumov~V.\,A.&&\\[.23pt]
\Avtors{Serebryanskii~S.\,M.} see~Tyrsin~A.\,N.&&\\[.23pt]
\Avtors{Seyful-Mulyukov~R.\,B.} see~Callaos~N.\,K.&&\\[.23pt]
\Avtors{Shestakov~O.\,V.} Statistical properties of the denoising method
based on the stabilized hard\linebreak
\\[-12pt]
\hspace*{23pt}thresholding&2&65--69\\[.23pt]
\Avtors{Shestakov~O.\,V.} The strong law of large numbers for the risk
estimate in the problem of\linebreak
\\[-12pt]
\hspace*{23pt}tomographic image reconstruction from
projections with a correlated noise&3&41--45\\[.23pt]
\Avtors{Shestakov~O.\,V.} see~Zakharova~T.\,V.&&\\[.23pt]
\Avtors{Shnurkov~P.\,V., Gorshenin~A.\,K., and Belousov~V.\,V.}
Analytical solution of~the~optimal control\linebreak
\\[-12pt]
\hspace*{23pt}task of~a~semi-Markov
process with~finite set of~states&4&72--88\\[.23pt]
\Avtors{Shnurkov~P.\,V., Zasypko~V.\,V., Belousov~V.\,V., and
Gorshenin~A.\,K.} Development of the algorithm of numerical solution
of the optimal investment control problem\linebreak
\\[-12pt]
\hspace*{23pt}in the closed dynamical model of three-sector economy&1&82--95\\[.23pt]
\Avtors{Shorgin~S.\,Ya.} see~Gaidamaka~Yu.\,V.&&\\[.23pt]
\Avtors{Shorgin~V.\,S.} see~Agalarov~Ya.\,M.&&\\[.23pt]
\Avtors{Shubnikov~S.\,K.} see~Minin~V.\,A.&&\\[.23pt]
\Avtors{Sidorkin~I.\,I.} see~Arkhipov~O.\,P.&&\\[.23pt]
\Avtors{Sinitsyn~I.\,N.} Analytical modeling of processes in stochastic
systems with complex fractional\linebreak
\\[-12pt]
\hspace*{23pt}order Bessel nonlinearities&3&55--65\\[.23pt]
\Avtors{Sinitsyn~I.\,N.} Orthogonal supoptimal filters for nonlinear
stochastic systems on manifolds&1&34--44\\[.23pt]
\Avtors{Sinitsyn~I.\,N.\ and Korepanov~E.\,R.} Normal Pugachev
conditionally-optimal filters and extra-\linebreak
\\[-12pt]
\hspace*{23pt}polators for state linear stochastic systems&2&14--23\\[.23pt]
\Avtors{Sinitsyn~I.\,N.\ and Sinitsyn~V.\,I.} Analytical modeling of
distributions in stochastic systems on\linebreak
\\[-12pt]
\hspace*{23pt}manifolds based on ellipsoidal approximation&1&45--55\\[.23pt]
\Avtors{Sinitsyn~I.\,N., Sinitsyn~V.\,I., and
Korepanov~E.\,R.} Ellipsoidal suboptimal filters for nonlinear\linebreak
\\[-12pt]
\hspace*{23pt}stochastic systems on manifolds&2&24--35\\[.23pt]
\Avtors{Sinitsyn~V.\,I.} see~Sinitsyn~I.\,N.&&\\[.23pt]
\Avtors{Sinitsyn~V.\,I.} see~Sinitsyn~I.\,N.&&\\[.23pt]
\Avtors{Skvortsov~N.\,A.} see~Stupnikov~S.\,A.&&\\[.23pt]
\Avtors{Sokolov~I.\,A.} see~Chertok~A.\,V.&&\\
\end{tabular}
}
\pagebreak

\def\leftfootline{\small{\textbf{\thepage}
\hfill INFORMATIKA I EE PRIMENENIYA~--- INFORMATICS AND APPLICATIONS\ \ \ 2016\
\ \ volume~10\ \ \ issue\ 4}
}%
 \def\rightfootline{\small{INFORMATIKA I EE PRIMENENIYA~---
INFORMATICS AND APPLICATIONS\ \ \ 2016\ \ \ volume~10\ \ \ issue\ 4
\hfill \textbf{\thepage}}}

\def\leftkol{2016 AUTHOR INDEX} % ENGLISH ABSTRACTS}

\def\rightkol{2016 AUTHOR INDEX} %ENGLISH ABSTRACTS}


{\tabcolsep=3pt
\begin{tabular}{p{382pt}cc}
&\textbf{Issue} & \textbf{Page}\\[6pt]
\Avtors{Sopin~E.\,S.} see~Gaidamaka~Yu.\,V.&&\\
\Avtors{Strijov~V.\,V.} see~Goncharov~A.\,V.&&\\
\Avtors{Strijov~V.\,V.} see~Isachenko~R.\,V.&&\\
\Avtors{Strijov~V.\,V.} see~Karasikov~M.\,E.&&\\
\Avtors{Stupnikov~S.\,A., Briukhov~D.\,O., and Skvortsov~N.\,A.}
Co-lending systemic risk analysis over\linebreak
\\[-12pt]
\hspace*{23pt}heterogeneous data collections&1&23--33\\
\Avtors{Stupnikov~S.\,A.} see~Kalinichenko~L.\,A.&&\\
\Avtors{Suchkov~A.\,P.} see~Zatsarinny~A.\,A.&&\\
\Avtors{Timonina~E.\,E.} see~Grusho~A.\,A.&&\\
\Avtors{Titova~A.\,I.} see~Kudryavtsev~A.\,A.&&\\
\Avtors{Turlikov~A.\,M.} see~Ometov~A.\,Ya.&&\\
\Avtors{Tyrsin~A.\,N.\ and Serebryanskii~S.\,M.} Recognition of
dependences on the basis of inverse\linebreak
\\[-12pt]
\hspace*{23pt}mapping&2&58--64\\
\Avtors{Ulyanov~V.\,V.} see~Markov~A.\,S.&&\\
\Avtors{Ushakov~V.\,G.} Queueing system with working vacations and
hyperexponential input stream&2&92--97\\
\Avtors{Ushakov~V.\,G.} see~Leontyev~N.\,D.&&\\
\Avtors{Volnova~A.\,A.} see~Kalinichenko~L.\,A.&&\\
\Avtors{Yakovlev~O.\,A.\ and Gasilov~A.\,V.} Speeded-up stereo
matching using geodesic support weights&3&\hphantom{1}98--104\\
\Avtors{Zabezhailo~M.\,I.} see~Grusho~A.\,A.&&\\
\Avtors{Zabezhailo~M.\,I.} see~Grusho~A.\,A.&&\\
\Avtors{Zakharova~T.\,V.\ and Shestakov~O.\,V.} Precision analysis of
wavelet processing of aerodynamic\linebreak
\\[-12pt]
\hspace*{23pt}flow patterns&3&46--54\\
\Avtors{Zalizniak~Anna~A.\ and Kruzhkov~M.\,G.} Database
of~Russian impersonal verbal constructions&4&132--141\\
\Avtors{Zasypko~V.\,V.} see~Shnurkov~P.\,V.&&\\
\Avtors{Zatsarinny~A.\,A.\ and Suchkov~A.\,P.} Systems engineering
approaches to~the~establishment of\linebreak
\\[-12pt]
\hspace*{23pt}a~system for~decision support based
on~situational analysis&4&105--113\\
\Avtors{Zatsarinny~A.\,A.} see~Grusho~A.\,A.&&\\
\Avtors{Zatsman~I.\,M., Inkova~O.\,Yu., Kruzhkov~M.\,G., and
Popkova~N.\,A.} Representation of cross-\linebreak
\\[-12pt]
\hspace*{23pt}lingual knowledge about
connectors in supracorpora databases&1&106--118\\
\Avtors{Zatsman~I.\,M.} see~Minin~V.\,A.&&\\
\Avtors{Zeifman~A.\,I.} see~Korolev~V.\,Yu.&&\\
\Avtors{Zeifman~A.\,I.} see~Korolev~V.\,Yu.&&\\
\end{tabular}
}

%\thispagestyle{myheadings}
\def\leftfootline{\small{\textbf{\thepage}
\hfill INFORMATIKA I EE PRIMENENIYA~--- INFORMATICS AND APPLICATIONS\ \ \ 2016\
\ \ volume~10\ \ \ issue\ 4}
}%
 \def\rightfootline{\small{INFORMATIKA I EE PRIMENENIYA~---
INFORMATICS AND APPLICATIONS\ \ \ 2016\ \ \ volume~10\ \ \ issue\ 4
\hfill \textbf{\thepage}}}

 \label{end\stat}

\newpage



%\def\stat{cont}
{%\hrule\par
%\vskip 7pt % 7pt
\raggedleft\Large \bf%\baselineskip=3.2ex
А\,В\,Т\,О\,Р\,С\,К\,И\,Й\ \ У\,К\,А\,З\,А\,Т\,Е\,Л\,Ь\ \ З\,А\ \ 2\,0\,0\,7 г. \vskip 17pt
    \hrule
    \par
\vskip 21pt plus 6pt minus 3pt }

\label{st\stat}

\def\tit{\ }

\def\aut{\ }
\def\auf{\ }

\def\leftkol{\ } % ENGLISH ABSTRACTS}

\def\rightkol{\ } %ENGLISH ABSTRACTS}

\titele{\tit}{\aut}{\auf}{\leftkol}{\rightkol}


\contentsline {chapter}{\ }{Выпуск \quad Стр.} 
\contentsline {section}{\textbf{Батракова Д.\,А., Королев В.\,Ю., Шоргин С.\,Я.}\ \ Новый метод вероятностно-ста\-ти\-сти\-че\-ско\-го анализа информационных потоков в\nobreakspace {}телекоммуникационных сетях}{\qquad 1 \qquad 40} 
\contentsline {section}{\textbf{Борисов А.\,В.}\ \ Байесовское оценивание в системах наблюдения с\nobreakspace {}марковскими скачкообразными процессами: игровой подход}{\qquad 2 \qquad 65}
\contentsline {section}{\textbf{Босов А.\,В., Иванов А.\,В.}\ \ Программная инфраструктура информационного Web-пор\-тала}{\qquad 2 \qquad 50}
\contentsline {section}{\textbf{Захаров В.\,Н., Калиниченко Л.\,А., Соколов И.\,А., Ступников С.\,А.}\ \ Конструирование канонических информационных моделей для интегрированных информационных систем}{\qquad 2 \qquad 15}
\contentsline {section}{\textbf{Захаров В.\,Н., Козмидиади В.\,А.}\ \ Средства обеспечения отказоустойчивости при\-ло\-жений}{\qquad 1 \qquad 14} 
\contentsline {section}{\textbf{Иванов А.\,В.}\ \ см. Босов А.\,В.\hfill\hfill\hfill\hfill\hfill\hfill\hfill\hfill\hfill\hfill\hfill\hfill\hfill\hfill\hfill\hfill\hfill\hfill\hfill\hfill\hfill\hfill\hfill\hfill\hfill\hfill\hfill\hfill\hfill\hfill\hfill\hfill\hfill\hfill\hfill}{\ }
\contentsline {section}{\textbf{Ильин В.\,Д., Соколов И.\,А.}\ \ Символьная модель системы знаний информатики в\nobreakspace {}че\-ло\-ве\-ко-автоматной среде}{\qquad 1 \qquad 66} 
\contentsline {section}{\textbf{Калиниченко Л.\,А.}\ \ см. Захаров В.\,Н.\hfill\hfill\hfill\hfill\hfill\hfill\hfill\hfill\hfill\hfill\hfill\hfill\hfill\hfill\hfill\hfill\hfill\hfill\hfill\hfill\hfill\hfill\hfill\hfill\hfill\hfill\hfill\hfill\hfill\hfill\hfill\hfill\hfill\hfill\hfill}{\ }
\contentsline {section}{\textbf{Козеренко Е.\,Б.}\ \ Лингвистическое моделирование для систем машинного перевода и обработки знаний}{\qquad 1 \qquad 54} 
\contentsline {section}{\textbf{Козмидиади В.\,А.}\ \ см. Захаров В.\,Н.\hfill\hfill\hfill\hfill\hfill\hfill\hfill\hfill\hfill\hfill\hfill\hfill\hfill\hfill\hfill\hfill\hfill\hfill\hfill\hfill\hfill\hfill\hfill\hfill\hfill\hfill\hfill\hfill\hfill\hfill\hfill\hfill\hfill\hfill\hfill }{\ } 
\contentsline {section}{\textbf{Королев В.\,Ю.}\ \ см. Батракова Д.\,А.\hfill\hfill\hfill\hfill\hfill\hfill\hfill\hfill\hfill\hfill\hfill\hfill\hfill\hfill\hfill\hfill\hfill\hfill\hfill\hfill\hfill\hfill\hfill\hfill\hfill\hfill\hfill\hfill\hfill\hfill\hfill\hfill\hfill\hfill\hfill}{\ } 
\contentsline {section}{\textbf{Кудрявцев А.\,А., Шоргин С.\,Я.}\ \ Байесовский подход к\nobreakspace {}анализу систем массового обслуживания и\nobreakspace {}показателей надежности}{\qquad 2 \qquad 76}
\contentsline {section}{\textbf{Печинкин А.\,В., Соколов И.\,А., Чаплыгин В.\,В.}\ \ Многолинейная система массового обслуживания с конечным накопителем и ненадежными приборами}{\qquad 1 \qquad 27} 
\contentsline {section}{\textbf{Печинкин А.\,В., Соколов И.\,А., Чаплыгин В.\,В.}\ \ Стационарные характеристики многолинейной\nobreakspace {}системы массового обслуживания с\nobreakspace {}одновременными отказами приборов}{\qquad 2 \qquad 39}
\contentsline {section}{\textbf{Синицын И.\,Н.}\ \ Корреляционные методы построения аналитических информационных моделей флуктуаций полюса Земли по априорным данным}{\qquad 2 \qquad \hphantom{9}2}
\contentsline {section}{\textbf{Синицын И.\,Н.}\ \ Развитие теории фильтров Пугачева для оперативной обработки информации в стохастических системах}{{\qquad 1 \qquad \hphantom{9}3}} 
\contentsline {section}{\textbf{Соколов И.\,А.}\ \ см. Захаров В.\,Н.\hfill\hfill\hfill\hfill\hfill\hfill\hfill\hfill\hfill\hfill\hfill\hfill\hfill\hfill\hfill\hfill\hfill\hfill\hfill\hfill\hfill\hfill\hfill\hfill\hfill\hfill\hfill\hfill\hfill\hfill\hfill\hfill\hfill\hfill\hfill}{\ }
\contentsline {section}{\textbf{Соколов И.\,А.}\ \ см. Ильин В.\,Д.\hfill\hfill\hfill\hfill\hfill\hfill\hfill\hfill\hfill\hfill\hfill\hfill\hfill\hfill\hfill\hfill\hfill\hfill\hfill\hfill\hfill\hfill\hfill\hfill\hfill\hfill\hfill\hfill\hfill\hfill\hfill\hfill\hfill\hfill\hfill}{\ } 
\contentsline {section}{\textbf{Соколов И.\,А.}\ \ см. Печинкин А.\,В.\hfill\hfill\hfill\hfill\hfill\hfill\hfill\hfill\hfill\hfill\hfill\hfill\hfill\hfill\hfill\hfill\hfill\hfill\hfill\hfill\hfill\hfill\hfill\hfill\hfill\hfill\hfill\hfill\hfill\hfill\hfill\hfill\hfill\hfill\hfill}{\ } 
\contentsline {section}{\textbf{Соколов И.\,А.}\ \ см. Печинкин А.\,В.\hfill\hfill\hfill\hfill\hfill\hfill\hfill\hfill\hfill\hfill\hfill\hfill\hfill\hfill\hfill\hfill\hfill\hfill\hfill\hfill\hfill\hfill\hfill\hfill\hfill\hfill\hfill\hfill\hfill\hfill\hfill\hfill\hfill\hfill\hfill}{\ }
\contentsline {section}{\textbf{Ступников С.\,А.}\ \ см. Захаров В.\,Н.\hfill\hfill\hfill\hfill\hfill\hfill\hfill\hfill\hfill\hfill\hfill\hfill\hfill\hfill\hfill\hfill\hfill\hfill\hfill\hfill\hfill\hfill\hfill\hfill\hfill\hfill\hfill\hfill\hfill\hfill\hfill\hfill\hfill\hfill\hfill}{\ }
\contentsline {section}{\textbf{Чаплыгин В.\,В.}\ \ см. Печинкин А.\,В.\hfill\hfill\hfill\hfill\hfill\hfill\hfill\hfill\hfill\hfill\hfill\hfill\hfill\hfill\hfill\hfill\hfill\hfill\hfill\hfill\hfill\hfill\hfill\hfill\hfill\hfill\hfill\hfill\hfill\hfill\hfill\hfill\hfill\hfill\hfill}{\ } 
\contentsline {section}{\textbf{Чаплыгин В.\,В.}\ \ см. Печинкин А.\,В.\hfill\hfill\hfill\hfill\hfill\hfill\hfill\hfill\hfill\hfill\hfill\hfill\hfill\hfill\hfill\hfill\hfill\hfill\hfill\hfill\hfill\hfill\hfill\hfill\hfill\hfill\hfill\hfill\hfill\hfill\hfill\hfill\hfill\hfill\hfill}{\ }
\contentsline {section}{\textbf{Шоргин С.\,Я.}\ \ см. Батракова Д.\,А.\hfill\hfill\hfill\hfill\hfill\hfill\hfill\hfill\hfill\hfill\hfill\hfill\hfill\hfill\hfill\hfill\hfill\hfill\hfill\hfill\hfill\hfill\hfill\hfill\hfill\hfill\hfill\hfill\hfill\hfill\hfill\hfill\hfill\hfill\hfill}{\ } 
\contentsline {section}{\textbf{Шоргин С.\,Я.}\ \ см. Кудрявцев А.\,А.\hfill\hfill\hfill\hfill\hfill\hfill\hfill\hfill\hfill\hfill\hfill\hfill\hfill\hfill\hfill\hfill\hfill\hfill\hfill\hfill\hfill\hfill\hfill\hfill\hfill\hfill\hfill\hfill\hfill\hfill\hfill\hfill\hfill\hfill\hfill}{\ }
%\thispagestyle{myheadings}
\def\leftfootline{\small{\textbf{\thepage}
\hfill ИНФОРМАТИКА И ЕЁ ПРИМЕНЕНИЯ\ \ \ том~1\ \ \ выпуск~2\ \ \ 2007}
}%
 \def\rightfootline{\small{ИНФОРМАТИКА И ЕЁ ПРИМЕНЕНИЯ\ \ \ том~1\ \ \ выпуск~2\ \ \ 2007
 \hfill \textbf{\thepage}}}
 \label{end\stat}

%\def\stat{cont-e}
{%\hrule\par
%\vskip 7pt % 7pt
\raggedleft\Large \bf%\baselineskip=3.2ex
2\,0\,0\,7\ \ A\,U\,T\,H\,O\,R\ \ I\,N\,D\,E\,X \vskip 17pt
    \hrule
    \par
\vskip 21pt plus 6pt minus 3pt }

\label{st\stat}

\def\tit{\ }

\def\aut{\ }
\def\auf{\ }

\def\leftkol{\ } % ENGLISH ABSTRACTS}

\def\rightkol{\ } %ENGLISH ABSTRACTS}

\titele{\tit}{\aut}{\auf}{\leftkol}{\rightkol}


\contentsline {chapter}{\ }{Issue \quad Page} 
\contentsline {subsection}{\textbf{Batrakova D.\,A., Korolev V.\,Yu., Shorgin S.\,Ya.}\ \ A New Method for the Probabilistic and Statistical Analysis of Information Flows in Telecommunication Networks}{\qquad 1 \qquad 40} 
\contentsline {subsection}{\textbf{Borisov A.\,V.}\ \ Bayesian Estimation in\nobreakspace {}Observation Systems with\nobreakspace {}Markov Jump Processes: Game-Theoretic Approach}{\qquad 2 \qquad 65} 
\contentsline {subsection}{\textbf{Bosov A.\,V., Ivanov A.\,V.}\ \ Linguistic Simulation for Machine Translation and Knowledge Management Systems}{\qquad 2 \qquad 50} 
\contentsline {subsection}{\textbf{Chaplygin V.\,V.} see Pechinkin A.\,V.\hfill\hfill\hfill\hfill\hfill\hfill\hfill\hfill\hfill\hfill\hfill\hfill\hfill\hfill\hfill\hfill\hfill\hfill\hfill\hfill\hfill\hfill\hfill\hfill\hfill\hfill\hfill\hfill\hfill\hfill\hfill\hfill\hfill\hfill\hfill}{\ }
\contentsline {subsection}{\textbf{Chaplygin V.\,V.} see Pechinkin A.\,V.\hfill\hfill\hfill\hfill\hfill\hfill\hfill\hfill\hfill\hfill\hfill\hfill\hfill\hfill\hfill\hfill\hfill\hfill\hfill\hfill\hfill\hfill\hfill\hfill\hfill\hfill\hfill\hfill\hfill\hfill\hfill\hfill\hfill\hfill\hfill}{\ }
\contentsline {subsection}{\textbf{Ilyin V.\,D., Sokolov I.\,A.}\ \ The Symbol Model of Informatics Knowledge System in Human-Automaton Environment}{\qquad 1 \qquad 66} 
\contentsline {subsection}{\textbf{Ivanov A.\,V.} see Bosov A.\,V.\hfill\hfill\hfill\hfill\hfill\hfill\hfill\hfill\hfill\hfill\hfill\hfill\hfill\hfill\hfill\hfill\hfill\hfill\hfill\hfill\hfill\hfill\hfill\hfill\hfill\hfill\hfill\hfill\hfill\hfill\hfill\hfill\hfill\hfill\hfill}{\ }
\contentsline {subsection}{\textbf{Kalinichenko L.\,A.} see Zakharov V.\,N.\hfill\hfill\hfill\hfill\hfill\hfill\hfill\hfill\hfill\hfill\hfill\hfill\hfill\hfill\hfill\hfill\hfill\hfill\hfill\hfill\hfill\hfill\hfill\hfill\hfill\hfill\hfill\hfill\hfill\hfill\hfill\hfill\hfill\hfill\hfill}{\ }
\contentsline {subsection}{\textbf{Korolev V.\,Yu.} see Batrakova D.\,A.\hfill\hfill\hfill\hfill\hfill\hfill\hfill\hfill\hfill\hfill\hfill\hfill\hfill\hfill\hfill\hfill\hfill\hfill\hfill\hfill\hfill\hfill\hfill\hfill\hfill\hfill\hfill\hfill\hfill\hfill\hfill\hfill\hfill\hfill\hfill}{\ }
\contentsline {subsection}{\textbf{Kozerenko E.\,B.}\ \ Linguistic Simulation for Machine Translation and Knowledge Management Systems}{\qquad 1 \qquad 54} 
\contentsline {subsection}{\textbf{Kozmidiady V.\,A.} see Zakharov V.\,N.\hfill\hfill\hfill\hfill\hfill\hfill\hfill\hfill\hfill\hfill\hfill\hfill\hfill\hfill\hfill\hfill\hfill\hfill\hfill\hfill\hfill\hfill\hfill\hfill\hfill\hfill\hfill\hfill\hfill\hfill\hfill\hfill\hfill\hfill\hfill}{\ }
\contentsline {subsection}{\textbf{Kudryavtsev A.\,A., Shorgin S.\,Ya.}\ \ Bayesian Approach to Queueing Systems and Reliability Characteristics}{\qquad 2 \qquad 76} 
\contentsline {subsection}{\textbf{Pechinkin A.\,V., Sokolov I.\,A., Chaplygin V.\,V.}\ \ Multichannel Queuing System with Finite Buffer and Unreliable Servers}{\qquad 1 \qquad 27} 
\contentsline {subsection}{\textbf{Pechinkin A.\,V., Sokolov I.\,A., Chaplygin V.\,V.}\ \ Stationary Characteristics of a Multichannel Queueing System with\nobreakspace {}Simultaneous Refusals of Servers}{\qquad 2 \qquad 39} 
\contentsline {subsection}{\textbf{Shorgin S.\,Ya.} see Batrakova D.\,A.\hfill\hfill\hfill\hfill\hfill\hfill\hfill\hfill\hfill\hfill\hfill\hfill\hfill\hfill\hfill\hfill\hfill\hfill\hfill\hfill\hfill\hfill\hfill\hfill\hfill\hfill\hfill\hfill\hfill\hfill\hfill\hfill\hfill\hfill\hfill}{\ }
\contentsline {subsection}{\textbf{Shorgin S.\,Ya.} see Kudryavtsev A.\,A.\hfill\hfill\hfill\hfill\hfill\hfill\hfill\hfill\hfill\hfill\hfill\hfill\hfill\hfill\hfill\hfill\hfill\hfill\hfill\hfill\hfill\hfill\hfill\hfill\hfill\hfill\hfill\hfill\hfill\hfill\hfill\hfill\hfill\hfill\hfill}{\ }
\contentsline {subsection}{\textbf{Sinitsyn I.\,N.}\ \ Correlational Methods for Analytical Informational Models of the Earth Pole Fluctuations Design Based on a priori Data}{\qquad 2 \qquad \hphantom{9}2}
\contentsline {subsection}{\textbf{Sinitsyn I.\,N.}\ \ Development of Pugachev Filtering for Stochastic Systems}{\qquad 1 \qquad \hphantom{9}3}
\contentsline {subsection}{\textbf{Sokolov I.\,A.} see Ilyin V.\,D.\hfill\hfill\hfill\hfill\hfill\hfill\hfill\hfill\hfill\hfill\hfill\hfill\hfill\hfill\hfill\hfill\hfill\hfill\hfill\hfill\hfill\hfill\hfill\hfill\hfill\hfill\hfill\hfill\hfill\hfill\hfill\hfill\hfill\hfill\hfill}{\ }
\contentsline {subsection}{\textbf{Sokolov I.\,A.} see Pechinkin A.\,V.\hfill\hfill\hfill\hfill\hfill\hfill\hfill\hfill\hfill\hfill\hfill\hfill\hfill\hfill\hfill\hfill\hfill\hfill\hfill\hfill\hfill\hfill\hfill\hfill\hfill\hfill\hfill\hfill\hfill\hfill\hfill\hfill\hfill\hfill\hfill}{\ }
\contentsline {subsection}{\textbf{Sokolov I.\,A.} see Pechinkin A.\,V.\hfill\hfill\hfill\hfill\hfill\hfill\hfill\hfill\hfill\hfill\hfill\hfill\hfill\hfill\hfill\hfill\hfill\hfill\hfill\hfill\hfill\hfill\hfill\hfill\hfill\hfill\hfill\hfill\hfill\hfill\hfill\hfill\hfill\hfill\hfill}{\ }
\contentsline {subsection}{\textbf{Sokolov I.\,A.} see Zakharov V.\,N.\hfill\hfill\hfill\hfill\hfill\hfill\hfill\hfill\hfill\hfill\hfill\hfill\hfill\hfill\hfill\hfill\hfill\hfill\hfill\hfill\hfill\hfill\hfill\hfill\hfill\hfill\hfill\hfill\hfill\hfill\hfill\hfill\hfill\hfill\hfill}{\ }
\contentsline {subsection}{\textbf{Stupnikov S.\,A.} see Zakharov V.\,N.\hfill\hfill\hfill\hfill\hfill\hfill\hfill\hfill\hfill\hfill\hfill\hfill\hfill\hfill\hfill\hfill\hfill\hfill\hfill\hfill\hfill\hfill\hfill\hfill\hfill\hfill\hfill\hfill\hfill\hfill\hfill\hfill\hfill\hfill\hfill}{\ }
\contentsline {subsection}{\textbf{Zakharov V.\,N., Kalinichenko L.\,A., Sokolov I.\,A., Stupnikov S.\,A.}\ \ Development of Canonical Information Models for Integrated Information Systems}{\qquad 2 \qquad 15} 
\contentsline {subsection}{\textbf{Zakharov V.\,N., Kozmidiady V.\,A.}\ \ Means Providing Applications Fault Tolerance}{\qquad 1 \qquad 14} 
\def\leftfootline{\small{\textbf{\thepage}
\hfill ИНФОРМАТИКА И ЕЁ ПРИМЕНЕНИЯ\ \ \ том~1\ \ \ выпуск~2\ \ \ 2007}
}%
 \def\rightfootline{\small{ИНФОРМАТИКА И ЕЁ ПРИМЕНЕНИЯ\ \ \ том~1\ \ \ выпуск~2\ \ \ 2007
 \hfill \textbf{\thepage}}}
 \label{end\stat}


%\tableofcontents


\end{document}