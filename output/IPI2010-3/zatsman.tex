\renewcommand{\figurename}{\protect\bf Figure}
\renewcommand{\tablename}{\protect\bf Table}
\renewcommand{\bibname}{\protect\rmfamily References}

\def\stat{bun}

\def\tit{TYPOLOGY AND COMPUTER MODELING OF TRANSLATION DIFFICULTIES$^*$}

\def\titkol{Typology and computer modeling of translation difficulties}

\def\autkol{N.~Buntman, J.-L.~Minel, D.~Le~Pesant, and~I.~Zatsman}
\def\aut{N.~Buntman$^1$, J.-L.~Minel$^2$, D.~Le~Pesant$^3$, and~I.~Zatsman$^4$}

\titel{\tit}{\aut}{\autkol}{\titkol}

{\renewcommand{\thefootnote}{\fnsymbol{footnote}}\footnotetext[1]
{This research is partially funded by RFBR Grant No. 09-07-00156.}}

\renewcommand{\thefootnote}{\arabic{footnote}}
\footnotetext[1]{Faculty of Foreign Languages and Area Studies at Moscow State University, nabunt@hotmail.com}
\footnotetext[2]{Universit$\acute{\mbox{e}}$ Paris Ouest Nanterre La D$\acute{\mbox{e}}$fense, 
jean-luc.minel@u-paris10.fr}
\footnotetext[3]{Universit$\acute{\mbox{e}}$ Paris Ouest Nanterre La D$\acute{\mbox{e}}$fense, 
denis.lepesant@wanadoo.fr}
\footnotetext[4]{Institute of Informatics Problems of the Russian Academy of Sciences, iz\_ipi@a170.ipi.ac.ru}

\vspace*{2pt}

\Abste{The problem of generating of goal-oriented knowledge systems in 
linguistics is reviewed. The problem belongs to a new research area, which is called ``cognitive informatics.'' 
The article focuses on computer coding of goal-oriented knowledge systems using as an example the 
typology of Russian-to-French translation difficulties.}

\KWE{time-dependent semiotic model; goal-oriented knowledge systems in linguistics; denotata; 
concepts; information objects; computer codes}

\vspace*{12pt}

       \vskip 14pt plus 9pt minus 6pt

      \thispagestyle{headings}

      \begin{multicols}{2}

      \label{st\stat}

\section{Introduction}

\noindent
The papers~[1--3] describe a new area of the research that aims to develop the theoretical 
foundations for creation of information and communication \mbox{technologies} (ICT) providing 
   goal-oriented development of new knowledge systems. It was proposed to name such 
systems as goal-oriented~\cite{3bun, 1bun}. Goal-oriented knowledge systems (GOKS) can be 
constructed and used practically in any area of science~\cite{2bun}. The problematics of 
\mbox{knowledge} system generation belongs to the informatics as the \mbox{computer} and information 
science~\cite{4bun}. The reason for this is that a description of knowledge system \mbox{generation} 
processes covers the mental sphere (knowledge), the material sphere of physical objects and 
phenomena, the social communication sphere (information), and the digital sphere (computer 
codes)~\cite{3bun, 1bun}. Within the bounds of informatics, this new area of research relates to 
cognitive informatics\footnote[5]{Cognitive informatics is the direction in informatics as computer and 
information science, which is the study of computational processes and the development of information and 
computer systems using methods of cognitive science that studies the mental processes (cognitive and creative) and 
mental objects (concepts), and the study of forms of presentation of concepts, their evolution, cognitive and creative 
processes using methods of computer and information science~[2, 5--7].}~[5--7].
   
   The purpose of developing the theoretical foundations for creation of ICT that should provide 
\mbox{generation} of a GOKS in linguistics and other fields and \mbox{areas} 
of application is the solution for at least three \mbox{pressing} \mbox{problems}: ($i$)~identification of GOKS 
concepts; ($ii$)~evaluation of the relevance of a generated GOKS to cultural, educational, 
economical, technological, and other needs; and ($iii$)~directed generation of a GOKS.
   
   To solve these problems, the stationary semiotic model of computer coding was built 
in~\cite{3bun}. 
The model is designed for computer coding of GOKS concepts, information 
objects, and denotata of the physical nature (physical objects) or denotata of the digital sphere, 
based on the fact that they all do not change over time.
   
Then in~\cite{1bun}, the time-dependent semiotic \mbox{model} of computer coding of GOKS 
concepts was built, \mbox{taking} into account that concepts, information objects, and \mbox{denotata}
\textit{can be changed over time} within GOKS \mbox{generation} processes. 

The model was illustrated 
with an example of a GOKS about indicators in science. In the \mbox{example}, each denotatum is a 
combination of programs for \mbox{computing} values of an indicator and database \mbox{information}, used by 
those programs. Along with modification and \mbox{interpretation} of denotata, i.\,e., 
programs and \mbox{information},  expert users are forming GOKS concepts, reflecting the evolution of indicator 
meanings. Each stage of indicators evolution has been fixed by an expert as a descriptor of the 
semantic dictionary (\mbox{thesaurus}) of an information system (IS), where one of the fields contains 
the name of the indicator assigned by the expert as its author. The model treats an indicator's 
name as an information object. This model gives the users of the IS the ability of computer 
coding of GOKS generation processes in the \mbox{thesaurus}.
   
   The model defines a new multidimensional space, which is called Frege's space~\cite{1bun}. 
Each point of the space fixes at a given time moment the state of evolution of three entities:
   \begin{enumerate}[(1)]
   \item
GOKS concept, i.\,e., meaning of an indicator at that moment;
\item information object, corresponding to the concept (in the example, it is a name of 
an indicator); and
\item denotatum, corresponding to the concept (in the example, any denotatum is a combination of \mbox{programs} and 
information).
\end{enumerate}

   A set of points, where each point fixes the state of evolution of three entities above at a given 
moment, is a network trace route of evolution of a GOKS over time~\cite{1bun}.
   
   The main goal of this paper is to describe the problem of GOKS generation in linguistics. 
The key task of the problem is computer coding of GOKS generation processes in the \mbox{thesaurus}. 
To solve the key task of the problem, it is suggested to use the time-dependent semiotic model 
from~\cite{1bun}. The paper discusses a GOKS in order to build the corpus-oriented 
\textit{typology} of Russian-to-French \textit{translation difficulties} (TTD)~\cite{8bun}. 
Development of a method of computer coding and of the software is the initial stage of a project 
of constructing TTD as a GOKS in linguistics.
   
   In this paper, the three key terms of the time-dependent semiotic model are used: ($i$)~denotatum; 
($ii$)~concept; and ($iii$)~information object. In the field of the TTD, these terms are treated as follows: 
denotatum is a pair of fragments of parallel texts in Russian and French, fixing a specific 
translation difficulty; concept is a meaning of the definition of that difficulty assigned by 
linguists-experts, as well as its position in the typology; and information object is a verbal 
designation of the concept, i.\,e., the name of a difficulty, consistently adopted by 
   linguists-experts. The TTD as a GOKS example in linguistics is interesting because the 
denotatum of each translation difficulty is a pair of fragments of parallel texts in Russian and 
French, so it belongs to the social communication sphere.

\section{Creation of Knowledge Systems: Review and Definitions}

\noindent
In 2004, a workshop of experts involved in the preparation of the 7th Framework Programme 
of the European Union was devoted to a discussion of the problematics of knowledge system 
generation, which was organized by the European Commission~\cite{2bun}. The workshop was 
named ``Knowledge Anywhere Anytime: The Social Life of Knowledge.'' Experts from 
different countries were invited to the workshop. Research on creation of knowledge systems 
took place before, but socio-economic approaches and models dominated, such as 
   SECI-model~\cite{9bun}. SECI means socialization, externalization, combination, and 
internalization. The SECI-model uses the term \textit{tacit knowledge}, thus emphasizing the 
personal experience of a subject and the subjective understanding. The tacit knowledge is 
opposed to the \textit{explicit knowledge}.
   
   By this opposition, the SECI-model is built dealing with relationship between tacit and 
explicit knowledge. The term \textit{knowledge} by itself (with neither \textit{tacit} nor 
\textit{explicit} adjective) is used for personal, collective, and conventional knowledge, the 
bearer of which can only be a human being. It is emphasized that a characteristic feature of the 
explicit knowledge is its expressibility in the form of information artifacts, such as books, 
articles, or reports.
   
   Following the review of the workshop ``Knowledge Anywhere Anytime: The Social Life 
of Knowledge,'' which took place in Brussels, the invited experts have prepared documents 
describing actual directions in the problematics of knowledge systems generation and evolution, 
including GOKS~\cite{2bun}.
   
   After 2004, the main results in this domain have been associated with structuring and more 
detailed statement of the problematics, as reflected in the 7th Framework Programme of the 
European Union~\cite{10bun}. The document contains formulations of a number of new 
directions and challenges related to the problems of generation, evolution, and representation of 
knowledge systems.
   
   Among recent publications that detail the problematics, the most significant for the 
establishment of the TTD as a GOKS are works of Wierzbicki and Nakamori, devoted to the models 
of generation of new knowledge systems, which distinguish between personal, collective 
(coordinated within the group), and conventional knowledge~\cite{11bun, 12bun}.
   
   Developing the idea of Wierzbicki and Nakamori on the division of personal, collective, and 
conventional knowledge, in~\cite{3bun, 1bun}, knowledge systems and their constituent 
concepts, as well as their corresponding \mbox{thesaurus} descriptors were divided into three categories 
(Table~1). This division became the basis for the definitions of personal and collective signs, 
personal and collective concepts.
   
   According to the definition from~\cite{13bun, 14bun}, the \textit{personal sign} differs from 
the traditional semiotic sign in that two of its sides~--- the form and meaning of the 
   sign~--- may be in a relation of a temporary connection, mediated by the consciousness of 
one person, may compose an unstable unity. In a period of GOKS generation 
time, the form, personally perceived, represents the meaning, personally
assigned to this sign.
   
   \textit{Collective sign} differs from the personal sign that the temporary relationship between 
its form and meaning is mediated by the consciousnesses of several persons, and \textit{is 
coordinated} among them. Consequently, there are at least two persons who use and understand 
coordinately their collective sign.
   
\begin{table*}\small
\begin{center}
\Caption{Categories of knowledge systems, concepts and corresponding descriptors
\label{t1bun}}
\vspace*{2ex}
   
   \begin{tabular}{p{35mm}p{35mm}p{35mm}p{35mm}}
   \hline
Categories of knowledge and descriptors   &Personal&Collective&Conventional\\
\hline
Knowledge systems and concepts&Personal knowledge systems and concepts&Collective 
knowledge systems and concepts&Conventional knowledge systems and concepts\\
\hline
Descriptors of IS thesaurus&Personal descriptors&Collective descriptors&Normatively 
аpproved descriptors\\
\hline
\end{tabular}
\end{center}
\end{table*}
   
   Using the definitions of personal and collective signs from~\cite{3bun, 1bun}, the following 
approach to the categorizations of concepts was suggested. \textit{Elementary personal concept} 
was defined as meaning of a personal sign, and \textit{personal concept}~--- as meaning of an 
expression of a natural language or of another sign system, if this expression contains at least one 
personal sign, or if this expression has no personal signs, but has a new meaning, explicitly 
defined by its author and registered in IS.
   
   \textit{Elementary collective concept} was defined as meaning of a collective sign, and 
\textit{collective concept}~--- as meaning of an expression of a natural language or of another 
sign system, if this expression contains at least one collective sign, or if this expression has no 
collective signs, but has a new meaning, explicitly defined, registered in IS, and is interpreted 
equally by at least two participants of GOKS generation process.
   
   The above definitions are the terminology basis for the description of the problem of GOKS 
generation in linguistics and other fields of knowledge and applications, as well as the creation 
of the TTD as a GOKS.

\section{Typology of Translation Difficulties as a Goal-Oriented Knowledge System}
   
\noindent
While preparing the project of constructing the TTD, one of the goals is the clarification of the 
description of GOKS generation with regard to linguistics. The project will be 
developed to help experts to create purposefully new 
systems of knowledge.
   
   Development of an information technology to be useful to the experts needs the preliminary study 
and solving of the following tasks:
   \begin{itemize}
\item analysis of cultural, educational, and scientific needs, 
whose satisfaction requires GOKS generation, for example, because of the 
incompleteness of existing knowledge systems;
\item developing methods for presentation and computer coding of a GOKS, taking 
into account its evolution over time;
\item creating tools for formal description of generation stages of a GOKS;
\item creating methods, information technologies and systems, supporting generation 
of a new GOKS; and
\item creating methods, information technologies and systems for evaluation of 
relevancy of a generated GOKS to socially important needs.
\end{itemize}

   One of the main tasks of the project is development of a method for computer coding of a GOKS 
and its usage in the process of generating the TTD. 
It is planned to implement the project by using the corpus of parallel texts of 
translations of classic Russian literature works. The proposed method for computer coding will 
give linguists-experts the possibility of modeling the process of TTD generation as a GOKS 
generation process taking into account the change of the typology over time. To demonstrate the 
feasibility of the proposed method, it is planned to generate the TTD as a new GOKS and to 
develop a computer model of TTD, reflecting the process of its development and evolution.
   
   The main task of the project is actual for both linguistics and informatics. In the field of 
linguistics, a new TTD will be developed, reflecting difficulties of Russian-to-French translation, 
registered in the corpus of parallel texts. In the field of informatics, a method of computer 
coding of a GOKS will be developed, and its feasibility will be shown with an example of 
constructing the TTD. So, the project has both computer information and linguistic components.
   
   The \textit{first (computer information) task} of the project is the development of a method 
for computer coding of a GOKS having the TTD as an example. This method for computer 
coding of a GOKS makes it possible to construct a semiotic model reflecting the design and 
evolution of the TTD. The construction of this model will solve a number of problems, such as 
fixing the changes of the TTD and evaluating its completeness concerning the corpus of parallel 
texts, as well as the level of consistency of the TTD as a whole and of its separate elements 
(classified translation difficulties) among linguists-experts implementing the project.
   
   This model is a formal description of translation difficulties, registered in the corpus of 
parallel texts, from three points of view: ($i$)~as denotata, i.\,e., pairs of fragments of parallel texts 
in Russian and French, fixing Russian-to-French translation difficulties; ($ii$)~as \mbox{concepts} of a 
generated GOKS, corresponding to these denotata; and ($iii$)~as verbal designations of these \mbox{concepts}, 
i.\,e., names of difficulties.
   
   The \textit{second (linguistic) task} of generation of the TTD as a GOKS is the study of 
asymmetry of French and Russian. As source of the study, it is suggested to use the corpus of 
texts of Russian classic fiction literature works, and their translations into French. In these texts, the 
fragments are marked where the asymmetry of French and Russian is most pronounced. The 
corpus, containing about 70~works of prose and drama, is being formed during last 19~years 
(1992--2010). The object of study for the second task will be a set of pairs of fragments from 
parallel texts in Russian and French, causing difficulties for the French translators. Note that 
native French speakers were involved for the selection of these pairs of fragments.
   
   The TTD created as a result of corpus-oriented study and analysis will include a systematized 
description of complicated cases connected with languages asymmetry, cultural differences, 
which caused difficulties of Russian-to-French translation of texts of the corpus. In the course of 
the project realization, it is expected to develop a prototype of the Russian--French TTD corpus, 
formed with an experimental array of pairs of fragments of parallel texts in Russian and French, 
fixing Russian-to-French translation difficulties. It is supposed that the prototype can contain several 
translations of the same works made by different translators. Each classified translation difficulty in 
the TTD will have some hypertext links to corresponding pairs of fragments in Russian and 
French, included in the prototype, as well as back links from those pairs to difficulty 
descriptions.
   
   Besides of constructing the TTD as a GOKS and computer modeling of the TTD, it is 
planned to formalize the markup of texts of the prototype. In order to implement this, it is 
supposed to develop a special language for markup of pairs of fragments of parallel texts in 
Russian and French, causing translation difficulties. This formal language is supposed to be used 
to markup pairs as denotata of translation difficulties. The development of this language and rules 
of the parallel texts markup is the task of the planned project at the junction of its computer 
information and linguistic components.
   
   In order to solve the tasks above, it is supposed to use the Web-technologies and sofware 
products developed in the Institute of Informatics Problems of the Russian Academy of Sciences. 
This will help linguists-experts to develop and coordinate the TTD using Internet as a tool of 
access to the description and the computer model of the TTD, stored at the server of the Institute; 
moreover, when the project is done, this will help future users to access to the description of the 
TTD with standard Web-browsers.

\section{Methods of Solving Tasks of~Constructing the~Typology of~Translation Difficulties}

\noindent
In the process of implementing the project, it is supposed to use the following three methods:
   \begin{enumerate}[(1)]
\item method for computer coding of the TTD as a GOKS;
\item method for classifying translation difficulties, firstly on the basic of their 
genus-species relations; and
\item method for tag-based markup of pairs of fragments of parallel texts causing 
translation difficulties.
\end{enumerate}

   The main idea of the method for computer coding of the TTD as a GOKS is that each 
classified translation difficulty is analyzed and fixed from three points of view: ($i$)~as a denotatum; 
($ii$)~as one of the concepts of the GOKS; and ($iii$)~as a verbal designation of the concept, i.\,e., a name 
of the translation difficulty. This method takes into account that during the process of 
development of any GOKS, all concepts and forms of their presentation can change, because the 
process of develoment of the TTD belongs to the stage of generating a new GOKS. At that stage, 
the level of variability of the TTD as a GOKS can be rather high. This is usual for the processes 
of generating a new GOKS.
   
   Using of the method for computer coding suggests that, at the stage of generating the TTD, 
denotata of translation difficulties are more stable compared with their concepts and verbal 
designations. The reason for the stability of the denotata is that, at the beginning of the process of 
creating a new GOKS, concepts can still be absent or just starting to take shape, but the object of 
study (the corpus of parallel texts with marked fragments of texts) has to be determined. It is 
possible to change the number of denotata (the corpus can be expanded, or the number of marked 
fragments of texts can be increased), but this kind of change can be eliminated by fixing the 
corpus and number of fragments. In contrast, the variability of concepts is the essential property 
of any generation process of a new GOKS.
   
   In addition to developing and applying the method of computer coding, it is planned to use 
the method of describing and classifying the difficulties of translation primarily on the basis of 
their genus-species relations. The application of the method for classifying difficulties can be illustrated with the 
following example. In ``The Overcoat'' of Nikolai Gogol, the protagonist cannot clearly 
articulate his thoughts, the meaning of his words sometimes cannot be understood, the word 
parasite \textit{того} (which is formally a case-form of the demonstrative pronoun) creates the 
effect of speech disorders (aphasia). In the French text, the word parasite is translated as \textit{n'est-ce} 
pas, a formula that serves to maintain the communication; its function is completely different~--- 
contact-establishing (fatic).
   
   Analysis of this example provides a basis to introduce a separate class of translation 
difficulties, which refers to all the fragments of texts attributed to a violation of connectedness of 
speech regardless of the reasons for its violation. Semantic failure may occur at the level of a 
lexeme, a syntagma, or a superphrase unity. Allocation of these levels may be one of the reasons 
of dividing of this class of difficulties into subclasses. In the example, the semantic failure occurs 
at the level of a lexeme.
   
   It is also supposed to use in the project the method for tag-based markup of pairs of 
fragments of parallel texts, which caused translation difficulties. The usage of the method 
implies the development of a set of tags and rules of markup with those tags. The markup is 
\mbox{necessary} because only when it is done, the formal borders for denotata as pairs of text fragments 
can be established. Methods for computer coding and classifying of 
\mbox{translation} difficulties are discussed in detail below.

\subsection{Method for computer coding}

\noindent
The method for computer coding defines a procedure of assignment to each classified 
difficulty three following computer codes: ($i$)~the code of its denotatum; ($ii$)~the code of its 
concept; and ($iii$)~the code of the verbal \mbox{designation} of this concept, i.\,e., the name of this 
translation difficulty. These codes are assigned \mbox{automatically} by tools of computer description 
and modeling of a GOKS (modeling program). Codes are assigned at the moment when a 
linguist-expert describes, using the modeling program, a classified difficulty as an element of the 
TTD. Each element of the TTD includes:
   \begin{enumerate}[(1)]
\item definition of a classified translation difficulty;
\item its internal links in the typology, i.\,e., links to other difficulties already 
classified;
\item its external links, i.\,e., links between a classified difficulty and corresponding 
fragments of parallel texts in Russian and French;
\item name of a classified translation difficulty (optionally);
\item linguistic notes (type of difficulty, level of context, etc.) and linguistic 
comments for the difficulty; and
\item variants of translation supposed by linguists-experts.
\end{enumerate}

   The specification of two last items of an element of the TTD can be postponed, because only 
the first four items are necessary for a code assignment by the modeling program. The name of 
any difficulty is optional. If the name is absent, the null code is assigned.
   
   Note that the modification of any element of the TTD, including the change of any target of 
an external link, leads to a creation of a new version of this element of the TTD, which is 
assigned with its own codes. In other words, any subsequent change of denotatum (target of an 
external link), concept (definition), or name of a difficulty leads to the generation of a new triad 
of codes at the time of generation of a new element of the TTD.
   
   Any change in denotatum, such as changing the formal boundaries and/or markup fragments 
of parallel texts in Russian and French, entails filling out a formalized questionnaire. Its filling 
by linguists-experts is planned to be implemented with a dialog program (the computer asks, the 
linguist answers). The goal of the dialog is the creation of a formal description of a semantic 
interpretation of a modified denotatum. On the basis of answers, the description of a new concept 
is formed and, optionally, a new name is assigned.
   
   Consequently, if a linguist-expert changes a denotatum, the result of the dialog program is 
either a new version of description of an existing classified difficulty, or a new classified 
difficulty of translation is coded by a linguist, as a new element of the TTD. In both cases, the 
information about a participant of the project (linguist) is incorporated automatically into the 
description. If other participants of the project agree (disagree) with proposed semantic 
interpretation of the modified denotatum, they may mark their attitude at special fields of 
description of the classified difficulty. Besides, they may suggest their variants of interpretation, 
describing a new variant of that element of the TTD.
   
   An essential element of the novelty of the method for computer coding is the ability to 
perform the dialog program when the denotatum remains unchanged, but a participant of the 
project wants to change the interpretation of the previously described denotatum and/or to offer a 
new variant of description of an existing element of the TTD.
   
   In the process of using the proposed method for computer coding, it is planned to assign triads 
of codes in order to build a multidimensional Frege's space~\cite{1bun}. Each element of the 
TTD corresponds to a single point of Frege's space. The appearance of a new element of the 
TTD or a new version of an element is fixed with a new point in Frege's space, including three 
codes and a time-stamp; that is, to describe the generation and evolution of the TTD, it is 
proposed to use four-dimensional space, which includes the time axis and three axes of computer 
codes for denotata, concepts, and their verbal names.
   
   The Frege's space is the basis for constructing the computer model of the TTD, because it is 
the \mbox{definitional} domain of functions which are suggested to be used for fixing the changes of the 
TTD and for evaluation of its completeness concerning the corpus, and for evaluation of the level 
of consistency of the TTD, as well as levels of coordination of intepretation of its different 
elements by linguists-experts.

\subsection{Classification of translation difficulties}

\noindent
Depending on the level of organization of the source text fragment in Russian, the TTD 
distinguishes at least six different levels of translation difficulties:
   \begin{enumerate}[(1)]
\item  superphrase;
      \item superphrase/phrase;
      \item phrase;
      \item syntactic;
      \item lexeme; and
      \item morpheme.
      \end{enumerate}
      
   Each difficulty is correlated to one of the following levels of the context:
   \begin{itemize}
      \item  microcontext (paragraph level);
      \item macrocontext (total text level);
      \item hypercontext (level of all works of an author and his/her idiolect); and
      \item cultural and historical context.
      \end{itemize}
      
   As mentioned in subsection~4.1, the description of each difficulty includes six components: 
definition of the difficulty, its internal links in the typology, its external links, name, linguistic 
notes, and variants of \mbox{translation}. These six components describing a translation \mbox{difficulty} 
compose a list of the main structural elements of a difficulty descriptor of the IS \mbox{thesaurus}. At 
the time of generation of each element of the TTD (classified \mbox{difficulty}) as a descriptor of the IS 
\mbox{thesaurus}, three computer codes are assigned to the element:
   \begin{enumerate}[(1)]
\item computer code of a difficulty definition and its internal links in the typology 
(concept computer code);
\item computer code of name of a classified translation difficulty (information object 
computer code); and
\item computer code of a target of external links between a classified difficulty and 
corresponding fragments of parallel texts in Russian and French (denotatum computer 
code).
\end{enumerate}

   A set of computer code triads for classified translation difficulties characterize the process of 
generation and evolution of the TTD in the Frege's space. Each triad fixes, at a given time 
moment, the evolution state of a single classified translation difficulty, including its concept, 
information object (name), and denotatum, corresponding to the concept. The set of triads is a 
network trace route of evolution of the TTD as a GOKS over time in the Frege's 
space~\cite{1bun}.

\section{Concluding Remarks}

\noindent
The suggested description of the problem of \mbox{generation} and evolution of GOKS has a 
fundamental difference from the approaches of Nonaka and Takeuchi~[9], Wierzbicky and 
Nakamori~\cite{11bun, 12bun}. It is in that the used model of generation and evolution of 
a GOKS is time-dependent. Using of the time-dependent semiotic model gives possibility to 
build in the Frege's space a network trace route of evolution of the TTD~\cite{1bun}.
   
   In the process of generation of the TTD, different linguists-experts may treat differently same 
translation difficulties; using of the time-dependent semiotic model makes it possible to build 
trace routes of evolution of personal concepts of different experts concerning the same pair of 
parallel texts.
   
   The article deals with the new approach to computer coding of the generation processes of 
GOKS. As for the TTD, a list of the main structural elements of the descriptors is suggested, 
code triads of which reflect the processes of generation of the TTD as a GOKS in the 
Frege'space.
   
   The need for further development of the time-dependent model is evident. In particular, it is 
necessary to construct the Frege's space where a semantic metrics is defined, which is suggested 
to be called \textit{semantic-metrical Frege's space} and to be used for setting, studying, and 
solving the problems of evaluation of relevancy and directed development of the TTD as a GOKS.
   
{\small\frenchspacing
{%\baselineskip=10.8pt
%\addcontentsline{toc}{section}{Литература}
\begin{thebibliography}{99}

    
     \bibitem{3bun} %1
     \Au{Zatsman I.\,M.}
     A semiotic model of correlations between concepts, information objects and computer 
codes~// Informatics and Its Applications, 2009. Vol.~3. No.\,2.  P.~65--81. (In Russian.)

\bibitem{1bun} %2
    \Au{Zatsman I.\,M.}
     A time-dependent semiotic model of \mbox{computer} coding of concepts, information objects and 
denotata~// Informatics and its Applications, 2009. Vol.~3. No.\,4. P.~87--101. (In Russian.)
     
     \bibitem{2bun} %3
     FP7 Exploratory Workshop 4 ``Knowledge Anywhere Anytime.'' {\sf 
http://cordis.europa.eu/ist/directorate\_f/\linebreak f\_ws4.htm}.
     
     \bibitem{4bun}
     \Au{Gorn S.}
     Informatics (computer and information science): Its ideology, methodology, and 
sociology~// The studies of information: Interdisciplinary messages~/ Eds. F.~Machlup and 
U.~Mansfield.~--- New York: John Wiley and Sons, Inc., 1983. P.~121--140.
     
     \bibitem{5bun}
     \Au{Wang Y.}
     Cognitive informatics: A new transdisciplinary research field~// Brain and Mind, 2003. 
Vol.~4. No.\,2. P.~115--127.
     
     \bibitem{6bun}
     \Au{Wang Y.}
     On cognitive informatics~// Brain and Mind, 2003. Vol.~4. No.\,2.  P.~151--167.
     
     \bibitem{7bun}
     \Au{Bryant A.}
     Cognitive informatics, distributed representation and embodiment~// Brain and Mind, 
2003. Vol.~4. No.\,2.  P.~215--228.
     
     \bibitem{8bun}
     \Au{Buntman N.\,V., Zatsman I.\,M.}
     Computer resource for typology of translation difficulties~// International Conference 
``Marginalia 2010: Borders of culture and text.'' Kargopol, September 25--26, 2010. Theses.  {\sf 
http://uni-persona.srcc.msu.ru/site/conf/ marginalii-2010/thesis.htm}. (In Russian.)
     
     \bibitem{9bun}
     \Au{Nonaka I., Takeuchi H.}
     The knowledge-creating company.~--- N.Y.: Oxford University Press, 1995.
     
     \bibitem{10bun}
    ICT FP7 Work Programme.  
     {\sf ftp://ftp.cordis.europa.eu/ pub/fp7/ict/docs/ict-wp-2007-08\_en.pdf}.
     
     \bibitem{11bun}
     \Au{Wierzbicki A.\,P., Nakamori~Y.}
     Basic dimensions of creative space~// In: Creative space: Models of creative processes for 
knowledge civilization age~/ Eds. A.\,P.~Wierzbicki and Y.~Nakamori.~--- Berlin--Heidelberg: 
Springer Verlag, 2006. P.~59--90.
     
     \bibitem{12bun}
     \Au{Wierzbicki A.\,P., Nakamori Y.}
     Knowledge sciences: Some new developments~// Zeitschrift f$\ddot{\mbox{u}}$r Betriebswirtschaft, 2007.  
Vol.~77.  No.\,3. P.~271--295.
     
     \bibitem{13bun}
     \Au{Zatsman I.\,M.}
     Conceptualization of data for scientometric investigations in scientific libraries~// 
10th Russian \mbox{Scientific} Conference ``Digital Libraries: Advanced Methods and 
Technologies, Digital Collections'' Proceedings. Dubna: Joint Institute for Nuclear Research, 2008. P.~45--54.
(In Russian.)

\bibitem{14bun}
     \Au{Zatsman I.\,M., Kosarik V.\,V., Kurchavova~O.\,A.}
     Problems of presentation of personal and collective concepts in the digital environment~// 
Informatics and Its Applications, 2008. Vol.~2. No.\,3.  P.~54--69. (In Russian.)
 \end{thebibliography}
}
}


\end{multicols}

\hrule

\smallskip

\def\tit{ТИПОЛОГИЯ И КОМПЬЮТЕРНОЕ МОДЕЛИРОВАНИЕ ТРУДНОСТЕЙ ПЕРЕВОДА}

\def\aut{Н.\,В.~Бунтман$^1$, Ж.-Л.~Минель$^2$, Д.~Ле Пезан$^3$, И.\,М. Зацман$^4$}

\titelr{\tit}{\aut}

\vspace*{12pt}

\noindent
$^1$Факультет иностранных языков и регионоведения, МГУ им.\ М.\,В.~Ломоносова, 
nabunt@hotmail.com\\
\noindent
$^2$Университет Париж--Нантер,  лаборатория UMR 7114 ``MoDyCo'' 
Национального центра научных ис-\linebreak
$\hphantom{^1}$следований Франции (CNRS), jean-luc.minel@u-paris10.fr\\
\noindent
$^3$Университет Париж--Нантер, лаборатория UMR 7114 ``MoDyCo'' 
Национального центра научных ис-\linebreak
$\hphantom{^1}$следований Франции (CNRS), denis.lepesant@wanadoo.fr\\
\noindent
$^4$Институт проблем информатики Российской академии наук, iz\_ipi@a170.ipi.ac.ru

%\vspace*{10mm}
\medskip


\Abst{Рассмотрена проблема формирования целевых систем знаний в 
лингвистике, которая относится к новому направлению исследований~--- когнитивной информатике. 
Исследованы вопросы компьютерного кодирования процессов генерации и эволюции целевых систем знаний 
на примере кор\-пус\-но-ориен\-ти\-ро\-ван\-ной типологии трудностей перевода с русского языка на французский.}


\KW{нестационарная семиотическая модель; целевые системы знаний в лингвистике; 
денотаты социально-коммуникативной среды; концепты; информационные объекты; компьютерные коды}

\label{end\stat}


\renewcommand{\figurename}{\protect\bf Рис.}
\renewcommand{\tablename}{\protect\bf Таблица}
\renewcommand{\bibname}{\protect\rmfamily Литература}