\def\stat{borodina}

\def\tit{ОБ ОЦЕНИВАНИИ АСИМПТОТИКИ ВЕРОЯТНОСТИ БОЛЬШОГО
УКЛОНЕНИЯ СТАЦИОНАРНОЙ  РЕГЕНЕРАТИВНОЙ     ОЧЕРЕДИ   С~ОДНИМ
ПРИБОРОМ$^*$}

\def\titkol{Об оценивании асимптотики вероятности большого
уклонения стационарной  регенеративной     очереди   %с~одним
%прибором
}

\def\autkol{А.\,В.~Бородина,  Е.\,В.~Морозов}
\def\aut{А.\,В.~Бородина$^1$,  Е.\,В.~Морозов$^2$}

\titel{\tit}{\aut}{\autkol}{\titkol}

{\renewcommand{\thefootnote}{\fnsymbol{footnote}}\footnotetext[1]
{Работа поддерживается РФФИ, грант  10-07-00017.}}

\renewcommand{\thefootnote}{\arabic{footnote}}
\footnotetext[1]{Институт прикладных
математических исследований  КарНЦ РАН, borodina@krc.karelia.ru}
\footnotetext[2]{Институт прикладных
математических исследований  КарНЦ РАН, emorozov@krc.karelia.ru}

\vspace*{6pt}

\Abst{Оценивание классическими методами имитационного
моделирования (малых) вероятностей таких нежелательных событий, как
потеря/разрушение данных, переполнение буфера,  конфликт сообщений в
современных телекоммуникационных сетях, требует неприемлемо больших
затрат времени и  вычислительных ресурсов. Однако точные
аналитические результаты   известны лишь для сравнительно узкого
класса систем и сетей обслуживания.  Это вызывает необходимость
развития  как асимптотических методов анализа, так и ускоренных
методов имитационного моделирования для оценивания ве\-ро\-ят\-ностей
событий  указанного выше типа.
  В данной статье  недавно развитый авторами метод ускоренного имитационного
  моделирования, основанный на расщеплении траекторий регенеративного процесса,
  применяется для  оценивания  вероятностей  превышения стационарным  процессом очереди/нагрузки высокого уровня.
Это позволяет существенно упростить и ускорить построение  оценки
показателя экспоненты в асимптотическом пред\-став\-ле\-нии вероятности
большого уклонения  в случае, когда  время обслуживания имеет
конечную логарифмическую функцию моментов (так называемый легкий
хвост). Приведены результаты численного моделирования.}

\vspace*{2pt}

\KW{асимптотика больших уклонений; одноканальная
система обслуживания; стационарное время ожидания; метод
расщепления; ускоренное оценивание}

\vspace*{12pt}

       \vskip 14pt plus 9pt minus 6pt

      \thispagestyle{headings}

      \begin{multicols}{2}

      \label{st\stat}

\section{Введение}

Проблема получения эффективных статистических оценок \textit{вероятностей редких событий}
  хорошо известна~\cite {Heidelberg, Asmus}.
 В контексте данной статьи \textit{задача оценивания малых вероятностей}
(порядка~$10^{-9}$ и меньше)  связана в первую очередь  с областью
телекоммуникационных технологий, систем и сетей обслуживания, где
такими  событиями являются,  например, отказ устройства,  потеря
данных из-за\linebreak переполнения буфера,   разрушение данных  при попытке
передачи из-за конфликта, связанного с разделением  передающего
канала, и~т.\,п. Однако\linebreak аналогичная проблема не менее актуальна и для
многих других областей. Для оценивания малых вероятностей требуется
специальный подход, поскольку аналитические результаты можно
получить, как правило, лишь для ограниченного класса простых
моделей, а применение классических методов имитационного
моделирования (прямой метод Монте-Карло (МК) и его модификации)
неэффективно. Действительно,  для построения оценки~$\hat\gamma$
(выборочного среднего) малой вероятности~$\gamma$ с заданной
точностью с помощью метода МК требуется неприемлемо большое число
наблюдений~$N$, а значит, и время моделирования. Это связано с тем,
что при фиксированном~$N$ относительная ошибка оценивания $\Lambda
(\hat\gamma)$ неограниченно растет с уменьшением оцениваемой
вероятности~$\gamma$
\begin{equation*}
\Lambda (\hat\gamma) := \fr{\sqrt {\D \hat\gamma }}{\e \hat\gamma
} = \fr{\sqrt{\gamma - \gamma^2}}{\gamma \sqrt{N}}\sim
\fr{1}{\sqrt{\gamma N}} \to \infty\ \mbox{при}\ \gamma \to
0,
\end{equation*}
где $a \sim b$  означает  $a/b \to 1$ и учтено, что $\e\hat
\gamma=\gamma$. В~описанной ситуации  альтернативой классическим
методам служат методы ускоренного имитационного моделирования.
 Такие  методы позволяют
построить искомую  оценку, как правило, на\linebreak несколько порядков
быстрее, чем классический метод МК~\cite{Heidelberg, GlassHeid96}.

В данной статье рассматривается ускоренное оценивание параметров
асимптотического представления вероятности большого уклонения\linebreak
процесса очереди/нагрузки в стационарной одноканальной системе в
случае, когда время обслуживания имеет так называемый \textit{легкий
хвост}. Актуальность такого подхода отмечена в работе~\cite{Morozov-asymptotics}. 
Основная цель данной статьи состоит в том,
чтобы привлечь внимание  специалистов по анализу и проектированию
телекоммуникационных и вычислительных сетей к возможностям
использования ускоренных методов для оценивания па\-ра\-мет\-ра
асимптотического представления вероятности перегрузки системы
обслуживания. Более точно, в работе используется недавно развитый
авторами метод регенеративного расщепления, который\linebreak является
модификацией классического метода\linebreak расщепления~[5--7]. 
Предлагаемый в данный работе подход опирается на прямое
оценивание искомой  вероятности  без привлечения сложной техники
тео\-рии больших уклонений (ТБУ). Кроме того, в случае регенерирующего
процесса обслуживания получаемые оценки являются состоятельными и
асимптотически нормальными~\cite{diss_2008, AG}.

Статья организована следующим образом. В~разд.~2 кратко
рассматриваются два основных подхода к ускоренному оцениванию:
изменение меры (метод существенной выборки) и многоуровневое
расщепление траектории моделируемого процесса. Последний подход
включает собственно\linebreak  метод расщепления, метод имитационного моделирования RESTART
(REpetitive Simulation
Trials\linebreak After Reaching Thresholds)~\cite{Alt91},
и модификацию\linebreak метода расщепления~--- метод регенеративного
расщепления с рандомизированными порогами, который подробно изложен
в~\cite{diss_2008}. В~разд.~3 рассматривается асимптотика
ве\-ро\-ят\-ности больших уклонений стационарного  процесса
нагрузки/очереди в случае, когда распределение  времени обслуживания
имеет  легкий хвост. В~частности, приведен анализ системы
Pareto$/M/1$, где интервалы входного потока имеют распределение
Парето. Наконец, в разд.~4 приведены численные результаты
применения метода расщепления к оцениванию показателя  в
асимптотическом пред\-став\-ле\-нии вероятности большого уклонения
стационарного процесса загрузки в системах $M/M/1$ и Pareto$/M/1$.

\section{Ускоренные методы моделирования }

В данном разделе кратко обсуждаются основные методы ускоренного
моделирования, применяемые для оценивания вероятностей редких
событий как эффективная альтернатива  прямому методу МК.

\subsection{Метод существенной выборки}

\textit{Метод существенной выборки} основан на идее сделать редкое
событие $\{X\in A\}$, где~$A$ есть некоторое фиксированное
множество, более вероятным, и тем самым ускорить процесс оценивания
вероятности $\gamma:=\p(X\in A)$ за счет соответствующего изменения
исходного распределения $F$ случайной величины (с.в.)~$X$~\cite{Asmus, Melas}. 
В~предположении существования плотности
$f(x)=dF(x)/dx$ имеем

\noindent
\begin{multline*}
\gamma=\int I(x \in A)f(x)\,dx ={}\\
{}=  \int I(x \in
A)\fr{f(x)}{\tilde{f}(x)}\tilde{f}(x)\,dx = E_{\tilde{f}}(I(X \in A)L(X))\,,
\end{multline*}
где $I$~--- индикатор, $L=f/\tilde f$~--- \textit{отношение
правдоподобия}, а  математическое ожидание~$\e_{\tilde{f}}$ берется
по  новой плотности~$\tilde{f}$.  Эта плотность выбирается с целью
минимизации дисперсии оценки (построенной по последовательности
независимых с.в.\  $X_n$, распределенных, как~$X$)
$$
\hat\gamma := \hat\gamma_N (\tilde{f}) =
\fr{1}{N}\sum\limits_{n=1}^{N}I(X_n \in A)L(X_n)
$$
и так, чтобы $\tilde{f}(x) > 0$ для всех $x\in A$, для которых
$f(x)>0$. Новая оценка является несмещенной,  $E_{\tilde{f}}(I(X_n
\in A)L(X_n))= \gamma$, строго состоятельной, а ее дисперсия равна
$$
D (\hat\gamma_N(\tilde{f})) = \fr{E_{\tilde{f}}(I(X \in A)L(X)^2)
- \gamma^2}{N}\,.
$$
Эффективность метода (сокращение дисперсии оценки и времени
оценивания) существенно зависит от  выбора~$\tilde{f}$~\cite{Heidelberg}. Заметим, что 
выбор $\tilde{f}(x) = f(x)/\gamma$
при $x\in A$ (и $\tilde{f}(x)=0$, если $x\not \in A$) был бы
оптимальным, однако он  на практике не применим, так как параметр
$\gamma$ является искомым. Отметим, что найти \textit{хорошую}
плотность~$\tilde{f}$,  как правило, весьма сложно и поэтому
использование данного метода не всегда приводит к уменьшению
дисперсии.

\subsection{Стандартный метод  расщепления}

Впервые метод расщепления был использован   в работе~\cite{KahnHarris51} 
для  моделирования физических процессов деления
частиц, а затем в работах~\cite{Bayes70, Bayes72} при анализе
некоторой простой  цепи Маркова.

Проиллюстрируем суть метода на примере одномерного (целочисленного)
случайного процесса $X=\{X_n\}$, определенного в пространстве
$R_+=$\linebreak $=[0,\,+\infty)$ (что достаточно для  приложений, рассматриваемых
в данной работе). Пространство~$R_+$ делится на $M+1$ вложенных
подмножеств, границы которых обозначаются   $L_i,\,i=0,\ldots,M,$ и
называются \textit{порогами (уровнями)}. Траектория~$X$ (обычно
стартующая с состояния $X_0=0$) расщепляется при достижении
некоторого  уровня~$L_i$ на несколько траекторий-потомков. При
достижении следующего уровня~$L_{i+1}$ каждая из этих траекторий
также расщепляется и~т.\,д. Увеличение числа  траекторий приводит к
повышению частоты достижения искомого  \textit{редкого множества}
$A=\{x:x\ge L_M:=L\}$.  Правило остановки траектории зависит от
того, какую вероятность нужно оценить. В~стандартном методе
расщепления каждая траектория: (1)~либо продолжается до достижения
заданного уровня~$L$ или до возвращения в~0; (2)~либо обрывается при
завершении процесса моделирования.

Если  процесс~$X$ является регенеративным, то алгоритм метода
расщепления позволяет построить оценку вероятности достижения
процессом уровня~ $L$ на цикле регенерации
\begin{equation}
\gamma_c= \p (\max_{1 \le n < \beta} X_n \ge L )\,,
 \label{gamma_c}
\end{equation}
где $\beta$~--- длина цикла в предположении, что $\e \beta < \infty$.
Если~$X$ является процессом очереди, изменения которого (в моменты
ухода/прихода) равны $\pm 1$, то в системе с неограниченным буфером
потенциально возможно  достижение любого (целочисленного) уровня~$L_i$.
 Заметим, что для процессов обслуживания  типичным моментом
регенерации является достижение состояния~$0$.

Существуют модификации  метода расщепления, в которых траектория
может быть прервана при попадании на несколько уровней ниже того, с
которого она стартовала. Если вероятность того, что траектория
вернется на исходный уровень, упав на несколько уровней ниже, очень
мала, то такое \textit{отсечение} траектории несущественно влияет на
итоговую оценку~\cite{Melas, GlassHeidProc96}. Эффективность метода
сильно зависит от выбора как самих уровней,  так и от числа
расщеплений на каждом уровне, что является основной проблемой,
которую надо решать при использовании метода. 
В~работах~\cite{GlassHeid96, GlassHeidProc96, Garvels2000} получены условия
оптимального выбора па\-ра\-мет\-ров, однако они опираются на искомое
значение оцениваемой вероятности.  Преимуществом метода расщепления
является существенное сокращение времени получения оценки по
сравнению с методом МК. Более того,  в~\cite{GlassHeidProc96,
Garvels2000} для марковского процесса показана несмещенность оценки~$\hat\gamma$, 
а в работе~\cite{Cerou} доказаны состоятельность и
асимптотическая нормальность оценки~$\hat\gamma$ для специального
класса марковских процессов, не связанных с сис\-те\-ма\-ми обслуживания.
В  монографии~\cite{AG} также обсуждаются  общие условия, при
которых оценки, полученные методом расщепления,  удовлетворяют
центральной предельной теореме.

\subsection{Метод RESTART}

Метод RESTART, предложенный   в~\cite{Alt91},
предназначен  для оценивания  вероятности
\begin{equation}
\gamma_s= \p (X \ge L )
\label{gamma_s}
\end{equation}
  превышения стационарной   очередью~$X$ в системе обслуживания некоторого
фиксированного уровня~$L$.  Основное различие между алгоритмами
метода RESTART и метода расщепления состоит в том, что в первом
случае все траектории, кроме одной, обрываются при пересечении
(сверху вниз) уровня, с которого они стартовали при расщеплении.
Далее моделируется только одна, последняя траектория
(предполагается, что все траектории пронумерованы), которая
расщепляется при пересечении (снизу вверх) любого уровня. В~методе
расщепления траектории не обрываются, однако расщепление возможно
только при пересечении более высоких уровней (по сравнению с уровнем
старта).  Как и в   методе расщепления, свойства  оценок в методе
RESTART при моделировании немарковских процессов изучены
недостаточно.

Необходимо отметить, что  метод расщепления был разработан для
оценивания вероятности вида~(\ref{gamma_c}) и его использование в
анализе систем обслуживания ограничивалось  исключительно
(цело\-чис\-лен\-ным) процессом очереди.

\subsection{Метод многоуровнего расщепления с~рандомизированными
порогами} 

В работах~[5, 6, 17--22] предложена модификация метода расщепления, которая
позволяет применить   результаты теории регенерации при обос\-но\-ва\-нии
состоятельности и асимптотической\linebreak нормальности получаемых оценок.
При этом была выявлена зависимость между  циклами регенерации,
получаемыми путем расщепления,  и исследовано ее влияние на ширину
доверительного интервала. Регенеративная структура траекторий также
позволяет применить стандартный метод расщепления для моделирования
процесса незавершенной нагрузки, что приводит, однако,  к
рандомизации уровней расщепления. Более того, моделирование по
циклам регенерации позволяет применить метод расщепления для
оценивания стационарной вероятности вида~(\ref{gamma_s}) как для
процесса очереди, так и для процесса нагрузки.

Данный метод  проиллюстрирован ниже на примере одноканальной системы
обслуживания $GI/G/1$ с независимыми одинаково распределенными (н.о.р.) временами обслуживания~$\{S_n\}$,  
н.о.р.\ интервалами
входного потока~$\{\tau_n\}$ и в предположении стационарности
$\rho:=\e S/E\tau<1$. (Здесь и далее
 опущен индекс при  обозначении типичного элемента последовательности н.о.р.\ с.в.) 
 Рас\-смот\-рим  процесс $\{W_n,\; n \ge 1\}$, где $W_n$~---
время ожидания в очереди заявки~$n$. Последовательность~$\{W_n\}$
является марковской цепью, которая удовлетворяет рекурсии Линдли:
\begin{equation}
W_{n+1} = (W_n + X_n)^+\,, \enskip X_n = S_n-\tau_n\,,\enskip n \ge 1\,.
\label{lindly}
\end{equation}
При $\rho<1$ существует  стационарное время ожидания~$W$, т.\,е.\
слабый предел $W_n \Rightarrow W$. Используя технику расщепления,
покажем, как построить оценку стационарной вероятности превышения
(фиксированного) уровня $x$, т.\,е.\
\begin{equation*}
\gamma_s = \p (W > x)\,.
%\label{gamma_s_W}
\end{equation*}
В отличие от (целочисленного) процесса очереди (со скачками равными
$\pm 1$), процесс~$W$  изменяется в момент прихода заявки~$n$ на
величину времени обслуживания~$S_n$. При моделировании марковской
цепи~(\ref{lindly}) расщепление траектории можно  проводить лишь  в
моменты поступления требований в систему. Таким образом, траектория
процесса за один скачок (вверх) может пересечь сразу несколько
заданных заранее уровней. Поэтому каждая траектория получает свою
собственную систему уровней расщепления, что и означает их
рандомизацию. Следовательно, способ построения оценки~$\gamma_s$,
используемый  в стандартном методе,  в данном случае неприменим.

В работах~\cite{avb7, avb9}  предложена  модификация метода
расщепления для моделирования \textit{процесса очереди} в системе
$M/G/1$ с использованием \textit{вложенной цепи Маркова}, когда
расщепление происходит в моменты ухода заявок, что снова приводит к
рандомизации уровней. При этом общее число цик\-лов регенерации
остается неизменным и равным $N = R_0 \cdots R_{M}$, где $R_i$~---
число расщеплений на уровне~$L_i$.  Для моделирования процесса
нагрузки~$\{W_n\}$ можно применить аналогичный подход, используя
следующее условие расщепления.

\medskip

\noindent
\textbf{Условие расщепления.} Если в момент прихода заявки
траектория, стартующая из области $G_i=[L_i, L_{i+1})$, пересекла
некоторый уровень $L_{i+k}$, то она расщепляется на
$\prod\limits_{j=1}^{k}R_{i+j}$ траекторий, $k \ge 1,\,i+k \le M$.

Если при этом условии  некоторая траектория  между двумя
последовательными уходами  пересекает (снизу вверх) несколько
заранее заданных уровней, то число потерянных траекторий
компенсируется, оставляя общее количество расщеплений (циклов
регенерации) неизменным и равным~$N$~\cite{avb7, avb6,  avb9}. 
В~\cite{diss_2008} доказана состоятельность и асимптотическая
нормальность построенной таким образом  оценки стационарной
вероятности~$\gamma_s$.

\section{Асимптотики для времени обслуживания с легким хвостом}

Одной из наиболее актуальных  задач при исследовании современных
телекоммуникационных  сис\-тем является оценивание  качества сервиса
(QoS), что часто  сводится  к оцениванию исключительно малых
вероятностей таких нежелательных событий, как потеря или разрушение
данных, отказ пе\-ре\-да\-юще\-го канала, переполнение буфера  и~т.\,д.
Ключевую роль при этом играют  процессы очереди  и времени ожидания
(незавершенной нагрузки).  В~данном разделе приводятся основные
асимптотические соотношения для указанных процессов в одноканальной
системе, в которой время обслуживания имеет легкий хвост.

\subsection{Основные асимптотические соотношения}

Хорошо известно, что  если в стационарной  системе $GI/G/1$ время
обслуживания~$S$ обладает  легким хвостом, т.\,е.\ если производящая
функция моментов $\e e^{\theta S}<\infty$
 в некоторой положительной окрестности параметра $\theta=0$, то
имеет место следующее асимптотическое представление для хвоста
распределения стационарной нагрузки~$W$ (см., например,~\cite{Glynn_Witt_94}):
\begin{equation}
\lim\limits_{x \to \infty}\fr{1}{x}\log \p(W > x) = - \delta\,,
\label{tail_asympt}
\end{equation}
где постоянная $\delta>0$. Таким образом, асимптотика хвоста
распределения~$W$ имеет \textit{экспоненциальную форму}.

Для определения искомого параметра~$\delta$ используется  ТБУ и
рассматривается предельная нормированная {\it логарифмическая
производящая функция моментов}
\begin{equation}
\Lambda (\theta) = \lim_{n \to \infty}\frac{1}{n}\log \e e^{\theta
\sum_{i=1}^n X_i}
\label{limit}
\end{equation}
в предположении, что предел существует и конечен в некоторой
(положительной) окрестности  $\theta=0$. Тогда параметр~$\delta$
определяется как
\begin{equation*}
\delta = \sup (\theta >0:\; \Lambda (\theta) \le 0)
%\label{ner_theta}
\end{equation*}
и является единственным положительным решением  уравнения
$\Lambda (\theta)=0$. % \label{eq_theta}
Таким образом,  проблема нахождения функции $\p(W>x)$   в рамках ТБУ
сводится к вычислению параметра~$\delta$  с предварительным
нахождением предела~(\ref{limit}).  Кроме весьма специальных случаев
(скажем, систем $M/M/1,\,M/M/m$),  данная  задача  не имеет
аналитического решения.  Это вынуждает строить  оценку параметра~$\delta$ 
на основе оценивания функции~$\Lambda(\theta)$. Разумеется,
 последняя проблема также является достаточно сложной.

  Стандартный метод расщепления и его регенеративный аналог  первоначально были предназначены (и
использованы) для расчета стационарной\linebreak вероятности переполнения
системы на цикле регенерации, чему способствовала  конструкция
расщепления и правило  остановки траекторий. В~связи со сказанным
уместно привести результаты\linebreak  работы~\cite {Sadowsky}, где
исследована асимптотика ве\-ро\-ят\-ности
  стационарной  очереди  в $m$-ка\-наль\-ной системе $GI/G/m$ {\it на цикле
  регенерации} в предположении, что
интервал входного потока~$\tau$ и время   обслуживания~$S$
удовлетворяют условиям
\begin{equation*}
\rho:=\fr{\e S}{\e \tau}<m\,,\quad \p(\tau>S)>0
%\label{condit}
\end{equation*}
и что функция $\Lambda_S(\theta):= \log \e e^{\theta S}<\infty$
 в некоторой  (положительной)
 окрестности $\theta=0$ (т.\,е.\ $S$ имеет  легкий хвост).
Пусть~$\nu_n$  означает  число заявок в (стационарной) очереди в
момент прихода заявки~$n$ и пусть $\beta $~---  длина цикла
регенерации системы. Обозначим через
$$
\gamma_c(k)=\p\left(\max\limits_{1\le l< \beta} \nu_l\ge k\right)
$$
вероятность превышения (стационарной) очередью  уровня~$k$ на цикле
регенерации.
 Пусть $\Lambda_\tau(-\theta m)=$\linebreak $=\log \e e^{-\theta m\tau}$. Обозначим
$\Lambda(\theta)=\log\e e^{\theta(S- m\tau)}$.\linebreak  Тогда, очевидно,
$$
\Lambda(\theta)=\Lambda_\tau(-\theta m)+\Lambda_S(\theta)\,.
$$
В~\cite{Sadowsky} показано, что  существует единственное решение $\delta>0$
уравнения
 $
 \Lambda(\theta)
=0 $ и что
\begin{equation*}
\lim\limits_{k\to \infty}\fr{1}{k}\log\gamma_c(k)=\Lambda_\tau(-\delta
m)\,. 
%\label{sad}
\end{equation*}
За исключением простейших случаев (например, системы $M/M/m$, см.~\cite{Morozov-asymptotics}), 
нахождение явного решения~$\delta$
уравнения $\Lambda(\theta)=0$, а тем более функции
$\Lambda_\tau(-m\delta )$, является неразрешимой задачей.
 Поэтому
    прямое {\it ускоренное  оценивание} вероятностей~$\gamma_c(k)$ 
    на основе регенеративного расщепления с последующим
оцениванием параметра~$\delta$ представляется весьма эффективным подходом.

 Наблюдения за  трафиком современных сетей показывают~\cite{Will}, что время
обслуживания~$S$ час\-то  имеет распределение с {\it тяжелым хвостом}
(например, Парето, логнормальное, Вейбулла).\linebreak Следовательно,  $S$
имеет неограниченную производящую функцию моментов $ \e e^{\theta S}
= \infty$ для любого  $\theta
>0$.  Заметим, что  в  стационарной системе $GI/G/1$ коэффициент загрузки
$\rho= \e S/\e \tau<1$, поэтому $\p(\tau>0)>0$. Таким образом, $\e
e^{-\theta \tau}>0$, поэтому  $\log\e e^{-\theta \tau}>-\infty$.
Следовательно, в  стационарной системе $GI/G/1$, где время
обслуживания имеет тяжелый хвост,
\begin{equation*}
\Lambda(\theta) = \log\e e^{\theta S} + \log \e e^{-\theta \tau} =
\infty 
%\label{moment}
\end{equation*}
 и поэтому
использование  ТБУ невозможно.  В то же время система вида $GI/M/1$
с распределением  входного потока, обладающим тяжелым хвостом, может
анализироваться на основе асимптотики  ТБУ. Этот вопрос будет
обсужден в следующем разделе.

\medskip

\noindent
\textbf{Замечание 1.} \textit{В работе}~\cite{avb4} \textit{рассматривается
проблема оценивания методами ТБУ так называемой {\it эффективной
пропускной способности}, тесно связанной с параметром~$\delta$, в
тандемной сети, обладающей свойством регенерации.}


\subsection{Анализ системы Pareto$/M/1$}

В этом разделе рассматривается  проблема вы\-чис\-ле\-ния параметра
асимптотики вероятности большого уклонения стационарной очереди в
сис\-те\-ме Pareto$/M/1$, где {\it интервал входного потока имеет
распределение Парето}  (с тяжелым хвостом), а время обслуживания~---
экспоненциально, т.\,е.\ имеет легкий хвост.

Пусть  скорость  обслуживания равна $\mu=1/\e S$, а н.о.р.\
интервалы входного потока распределены по закону Парето
\begin{multline}
1-A(x):=\p(\tau>x)=x^{-\alpha}\,\enskip x\ge 1\,,\enskip\\
\alpha>0\enskip (A(x)=0\,,\enskip x<1)\,.
\label{pareto}
\end{multline}
 Отметим, что
\begin{align*}
\e\tau &= \fr{\alpha}{\alpha
-1}<\infty\,,\enskip \mbox{если} \enskip \alpha>1\,;\\
D\tau  &=
\fr{ \alpha}{(\alpha - 1)^2(\alpha -
2)}<\infty\,,\enskip \mbox{если}\enskip \alpha>2\,.
\end{align*}
Распределение Парето  возникает  при моделировании трафиков
современных телекоммуникационных  систем, причем значение параметра~$\alpha$ 
часто оказывается  в диапазоне
 $1 < \alpha < 2$, когда второй момент соответствующей характеристики является
  бесконечным~\cite{Crovella&Bestavros97}.
Рассмотрим вложенную марковскую цепь $\{\nu_n\}$~---
последовательность чисел заявок в системе в моменты, непосредственно
пред\-шест\-ву\-ющие {\it приходу  заявок в систему}. Более точно, эта
цепь удовлетворяет рекурсии
\begin{equation}
\nu_{n+1}=(\nu_n+1 -\Delta_n)^+\,,\enskip n\ge 1\,,
\label{recursion}
\end{equation}
где $\Delta_n$ есть число заявок, обслуженных в течение $n$-го
интервала входного потока  в предположении, что {\it в системе
находится неограниченное число заявок}.  Данное предположение, не
меняя динамики процесса очереди, позволяет рассматривать
$\{\Delta_n\}$ как н.о.р.\ с.в. В~соответствии с методами ТБУ для
вычисления параметра~$\delta$  в асимптотическом пред\-став\-ле\-нии
необходимо найти значение производящей логарифмической функции
моментов (типичного) приращения $1-\Delta$ случайного блуждания,
формирующего рекурсию~(\ref{recursion}), т.\,е.\  функцию
\begin{equation*}
\Lambda(\theta)=\log \e e^{\theta(1-\Delta)}=\theta+\log \e
e^{-\theta \Delta}\,,
\end{equation*}
и затем решить уравнение $\Lambda(\theta)=0$. Очевидно, что
\begin{equation*}
\e  e^{-\theta \Delta}=\sum_{k=0}^\infty e^{-\theta k}a_k\,,
\end{equation*}
где
\begin{equation*}
a_k=\int_{0^-}^\infty e^{-\mu x}\fr{(\mu x)^k}{k!} dA(x)\,,\enskip k\ge
0\,.
\end{equation*}
Таким образом,
\begin{equation*}
\e  e^{-\theta \Delta}=\int\limits_{0^-}^\infty
e^{-\mu(1+e^{-\theta})x}\,dA(x)= \e ^{-\theta_{1} \tau}\,,
\end{equation*}
причем параметр  $\theta_1=\mu(1+e^{-\theta})$.
 Поэтому окончательно  получаем
\begin{equation*}
\Lambda(\theta)=\theta+\log \e ^{-\theta_{1} \tau}\,.
\end{equation*}
  Таким образом,  нахождение параметра асимптотики включает
 вычисление преобразования Лапласа--Стилтьеса  распределения Парето интервала входного потока. В~этой
связи обсудим трудности вычисления данного преобразования, следуя
работам~\cite{Starobinski&Sidi2000, Nadarajah&Kotz2006}. Так
  как плотность распределения Парето равна $f(x)=\alpha
x^{-\alpha-1}$, $\alpha>1$, $x\ge 1$, $(f(x)=0,\ x<1)$,  то
\begin{multline}
\e e^{-\theta \tau}  =  \alpha \int\limits_{1}^{\infty} x^{-\alpha - 1}
e^{-\theta x}\,dx ={}\\
{}=  \alpha \theta^{\alpha}
\int\limits_{1}^{\infty}x^{-\alpha - 1}e^{-x}dx = \alpha
             \theta^{\alpha} \Gamma(-\alpha)\,,
\label{gamma_vir}
\end{multline}
где $\Gamma(-\alpha)$~--- полная гамма-функция.


\medskip

\noindent
\textbf{Замечание 2.} {\it В работе}~\cite{Nadarajah&Kotz2006} {\it на самом
деле рассматривается плотность Парето, имеющая сдвиг переменной~$x$
на значение параметра положения~$x_0$ (в нашем случае $x_0=1$), т.\,е.\
 плотность $f(x)=\alpha / (x+x_0)^{\alpha +1}$. В~этом случае
выражение}~(\ref{gamma_vir})  {\it содержит  вместо гамма-функции функцию
Уитаккера
\begin{multline*}
W_{\lambda,\, \mu} (x) ={}\\
{}=
\fr{x^{\mu+1/2}e^{-x/2}}{\Gamma(\mu-\alpha
+1/2)}\!\int\limits_{0}^{\infty}\!t^{\mu - \lambda
-1/2}(1+t)^{\mu+\lambda-1/2}e^{-xt}\,dt\,.\hspace*{-1.3917pt}
\end{multline*}
После замен  $\lambda = -(\alpha+1)/2$, $\mu =
-\alpha/2$, $x=\theta$ и $x_0=1$ соотношение}~(\ref{gamma_vir})
{\it принимает вид
\begin{equation*}
\e e^{-\theta\tau} =  \alpha \theta^{(\alpha
-1)/2}e^{\theta/2}W_{-(\alpha+1)/2, -\alpha/2}(\theta)\,.
%\label{vitakker}
\end{equation*}
}


Таким образом, преобразование Лапласа--Стил\-тье\-са распределения
 Парето выражается через известные функции, для которых существуют  подробные таблицы,
например~\cite{Gradshteyn&Ryzhik2000}.
Отметим \mbox{также}, что вычисление
этих функций возможно с по\-мощью  пакетов  Maple и Mathematica.
Полученный результат дает возможность  в принципе находить численное
значение преобразования Лап\-ла\-са--Стилтьеса распределения Парето,
однако не позволяет решить обсуждавшуюся выше задачу вычисления
параметра~$\delta$. Таким образом, проведенный анализ показывает,
что аналитическое на\-хож\-де\-ние параметра асимптотики для системы
обслуживания с входным потоком Парето невозможно. Это подтверждает
важность прямого оценивания параметра асимптотики через
 ускоренное оценивание вероятности большого уклонения исходного
процесса.

\subsection{Оценивание параметра асимптотики методом  расщепления}

Поскольку найти в явном виде параметр~$\delta$ в  асимптотике~(\ref{tail_asympt}) 
с использованием логарифмической производящей
функции моментов возможно лишь в простейших случаях (системы\linebreak
$M/M/1,\,M/M/m$), а классический метод МК  является неэффективным
для оценивания стационарной ве\-ро\-ят\-ности $\gamma_s=\p(W
> b)$ при больших\linebreak значениях~$b$, то представляется естественным
применить ускоренный метод  расщепления
 для
вы\-чис\-ле\-ния показателя экспоненты в асимптотике~(\ref{tail_asympt}),
опираясь на прямое вычисление самой ве\-ро\-ят\-ности~$\gamma_s$.
Возможность такого подхода была отмечена   в работе~\cite{BDM} в
связи с трудностями, возникшими при  построении оценки эффективной
пропускной способности на основе использования так называемого
метода batch-mean~\cite{natalie}.  Более точно,  можно использовать
метод регенеративного расщепления для построения состоятельной и
асмптотически нормальной оценки $ \hat\gamma$ вероятности $\gamma_s$
при различных значениях порогов $b_1 <  \cdots < b_n$. Обозначим
через~$\hat\gamma_i$ оценку вероятности $\p (W > b_i)$. В~результате
получаем последовательность
 оценок~$\hat\delta_i$
\begin{equation}
\hat\delta_i = - \fr{\log \hat\gamma_i}{b_i} = \delta + C(b_i)\,,\enskip
i= 1,\ldots, n\,, 
\label{hat_delta_i}
\end{equation}
где слагаемое $C(b_i)\to 0$ при $b_i \to \infty$. Поэтому при
заданной точности оценивания $\epsilon > 0$
искомую  оценку~$\hat\delta$  показателя~$\delta$ можно определить,
например, как
\begin{equation}
\hat\delta = \min_{k > 1} \{\hat\delta_k: \; |\hat\delta_{k-1} -
\hat\delta_k| < \epsilon \}\,. 
\label{hat_delta}
\end{equation}
Определение оптимальной последовательности значений~$b_i$ (например,
для минимизации  времени моделирования) остается открытой проблемой.
Скажем, в примере из следующего раздела  последовательность~$b_i$
определяется  как арифметическая прогрессия с шагом $d=10$.
Поскольку  $\hat \delta_i-\delta=C(b_i)\to 0$, то (в зависимости  от
конкретной задачи) оценивание достаточно, видимо, провести  лишь
{\it для нескольких больших значений порога  или даже для одного из
них}.

Подчеркнем, что предложенный метод опирается на логарифмическую
асимптотику~(\ref{tail_asympt}) и  поэтому   скорость сходимости
$C(b_i)\to 0$ существенно зависит  от рассматриваемой модели. 
(В~этой связи см.~\cite{Morozov-asymptotics}.)

\section{Результаты численного моделирования}

В этом  разделе приведены  результаты численного моделирования и
оценивания  параметра~$\delta$ при анализе асимптотики большого
уклонения величины стационарной нагрузки  в системах $M/M/1$ и
$Pareto/M/1$. Изучено влияние слагаемого~$C(b_i)$ в~(\ref{hat_delta_i}) на оценку~$\hat\delta_i$  параметра~$\delta$,
которая, в свою очередь, вычислялась по формуле~(\ref{hat_delta}).

Рассмотрим вначале систему $M/M/1$ с коэффициентом загрузки $\rho =
\lambda / \mu < 1$.
Хвост распределения стационарного процесса нагрузки~$W$ имеет вид
\begin{equation}
\p (W > x) = \rho e^{-(\mu - \lambda )x}\,, \enskip x \ge 0\,, 
\label{prob_W}
\end{equation}
и поэтому искомый параметр  $\delta = \mu - \lambda $, а величина
$C(b_i) = -\log \rho/b_i$.  Отметим, что величину~$C(b)$ можно
трактовать как {\it систематическую ошибку оце-}
\begin{center} %fig1
\vspace*{3pt}
\mbox{%
\epsfxsize=72.615mm
\epsfbox{bor-1.eps}
}
\end{center}
\vspace*{4pt}
%\begin{center}
{{\figurename~1}\ \ \small{Зависимость  оценки параметра $\delta$ от величины порога в
системе $M/M/1$
($\rho=0{,}6$; $\lambda=0{,}6$; $\mu=1$): \textit{1}~--- $\delta=0{,}4$; \textit{2}~--- $\delta_i$;
\textit{3}~--- RS}}
%\end{center}
\vspace*{9pt}

\bigskip
\addtocounter{figure}{1}

\noindent
\textit{нивания} параметра~$\delta$, 
которая стремится к~0 с ростом величины порога. Из~(\ref{hat_delta_i}) следует, что
%для каждого $b_i$
$$
\delta_i =-\fr{\log \rho}{b_i} + \mu - \lambda\,,\enskip i=1, \ldots,n\,.
$$



\noindent
На рис.~1 приведены  результаты  вычисления оценок
параметра~$\delta$, полученные путем  подсчета   ве\-ро\-ят\-ности $\p
(W > b_i)$ по формуле~(\ref{prob_W})  (обозначены~$\delta_i$), а
также путем прямого  оценивания  этой вероятности методом
регенеративного расщепления с рандомизированными порогами (RS).
Параметры модели выбраны  $\mu=1$,  $\rho=0{,}6$. Полученные оценки
также сравниваются  с известным значением $\delta=0{,}4$. В~обоих
методах использованы значения  порогов $ b_1 = 30$, $b_i =
b_{i-1}+10$,  $i = 1, \dots, 11$. 
 Входными данными для алгоритма метода являются:
параметры~$\lambda$, $\mu$, заданное число редких событий, уровень~$b_i$, 
промежуточные уровни~$L_j$ (для данного уровня~$b_i$), чис\-ло
расщеплений~$R_j$ на каждом уровне~$L_j$. Условием завершения
моделирования считается достижение заданного  числа редких событий,
а также завершение текущего  цикла регенерации. Как отмечено в
разд.~2.4, траектория процесса нагрузки расщепляется лишь в
моменты прихода требований в систему. Поэтому  (в отличие от
процесса очереди) попадание на заранее заданные уровни~$L_j$  не
гарантировано и эти уровни  служат лишь границами зон, внутри
которых оказываются действительные (рандомизированные) уровни
расщепления траектории. Подробнее см.~\cite {avb4, diss_2008}.

Оценка $\hat\delta_i$ строилась по формуле~(\ref{hat_delta_i}), где
оценка~$\hat\gamma_i$ вероятности $\p (W > b_i)$ вычислялась как
выборочное среднее из 100~значений, полученных методом
регенеративного расщепления с условием достижения не менее 1000~редких событий. 
Например,\ для порога $b_9=110$ оценка $\hat\gamma_9 = 3{,}9497\times 10^{-20}$, 
среднее время моделирования одного
значения выборки составило 122~с (на процессоре Pentium(R)~D
CPU 2,8~ГГц).
 Отметим, что
для  уровня $b_{11}= 130$  оценка вероятности~$\hat\gamma_{11}$
имеет порядок $10^{-23}$ в  обоих случаях (см.\ рис.~1). Метод
регенеративного расщепления дает $\hat\delta_{11} = 0{,}40517$, при
этом $|\hat\delta_{10} - \hat\delta_{11}| = 0{,}0004$. В то же время
по  формуле~(\ref{prob_W}) получаем  $\delta_{11}=0{,}40393$,
$|\delta_{10}-\delta_{11}| = 0{,}0003$.  Полученные результаты можно
считать вполне удовлетворительными.

Методом регенеративного расщепления также проведено  оценивание
параметра~ $\delta$ в асимптотическом пред\-став\-ле\-нии вероятности $\p
(W > b)$ в системе
Pareto$/M/1$ в предположении, что  в формуле~(\ref{pareto}) $\alpha
= 4$, а коэффициент загрузки  системы $\rho = 0{,}75$. Результаты
численного моделирования приведены в табл.~1.

\bigskip

\noindent
{{\tablename~1}\ \ \small{Оценки  вероятности $\gamma_s$ и показателя $\delta$   в системе Pareto$/M/1$}}
%\vspace*{2ex}

\begin{center} %fig1
\tabcolsep=7.2pt
\begin{tabular}{|r|c|c|c|c|}
\hline
i & $b$ & $\hat\gamma_i $ & $\hat\delta_i$ & $|\hat\delta_{i-1}-\hat\delta_{i}|$ \\
\hline
1 &30 &  $5{,}4394\cdot 10^6$\hphantom{$^1$}& 0,4041 &  --- \\
%\hline
2&40 &  $2{,}9336\cdot 10^7$\hphantom{$^1$} & 0,3760 &  0,028\\
%\hline
3&50 & $5{,}7565\cdot 10^9$\hphantom{$^1$} & 0,3795 &      0,003\\
%\hline
4&60 & $2{,}8354\cdot 10^{11}$ &  0,4048 &  0,025\\
%\hline
5&70 & $6{,}3413\cdot 10^{13}$ & 0,4012 &      0,004\\
%\hline
6&80 & $6{,}9826\cdot 10^{15}$ & 0,4074 &  0,006\\
%\hline
7&90 & $3{,}4447\cdot 10^{16}$ &  0,3956&  0,012\\
%\hline
8&100\hphantom{9} &$2{,}9986\cdot 10^{18}$  & 0,4035 &  0,008\\
%\hline
9&110\hphantom{9} & $4{,}0204 \cdot 10^{20}$  & 0,4060 &  0,003\\
%\hline
10&120\hphantom{9} & $9{,}2678\cdot 10^{22}$  & 0,4036 &  0,002\\
%\hline
11&130\hphantom{9} & $2{,}0886\cdot 10^{23}$ & 0,4017 &     0,002\\
\hline
\end{tabular}
\end{center}
\vspace*{9pt}
%\begin{center}
%\end{center}
%\vspace*{9pt}

%\bigskip
%\addtocounter{figure}{1}



Значения оценки $\hat\gamma_i$ снова получены по выборке размера~100. 
Процедура моделирования останавливается после завершения
текущего цикла регенерации и при достижении не менее 1000~искомых
событий (достижение заданного порога~$b_i$). Например, для получения
одного значения в выборке при вычислении  $\hat\gamma_{11} =
2{,}0886\times 10^{-23}$ для порога $b_{11}=130$ в среднем была
затрачена 301~с. Как видно из табл.~1, для вычисления оценки~$\hat\delta$, скажем  с 
точностью $\epsilon = 0{,}002$,  можно
использовать значение порога $b_{10}=120$. Отметим, что, как и в
системе $M/M/1$, $\hat\delta_i-\hat\delta_{i-1}\to 0$  c ростом
величины порога~$b_i$, что согласуется с основным теоретическим
результатом~(\ref{tail_asympt}).


\section{Заключение}

В  работе предложен метод прямого оценивания\linebreak параметра
асимптотического представления вероятности большого значения
стационарного процесса нагрузки и стационарной очереди  на основе
развитого ранее авторами ускоренного метода\linebreak регенеративного
расщепления.  Такой  подход может служить  альтернативой более
сложному методу оценивания на основе теории больших уклонений
(использующему предельную производящую функцию моментов) и
принципиально  более эффективен, чем прямой метод Монте-Карло (с
точки зрения необходимого времени моделирования).  Как было
установлено ранее, оценки, получаемые данным методом для
регенеративных систем, являются состоятельными и асимптотически
нормальными.  Применение метода иллюстрируется численным
моделированием систем $M/M/1$  и Pareto$/M/1$.  Полученные
предварительные результаты оценивания дают хорошее согласие с
известным аналитическим результатом для  системы $M/M/1$, однако
представляется весьма важным дальнейшее обосно\-ва\-ние предложенного
метода.

{\small\frenchspacing
{%\baselineskip=10.8pt
%\addcontentsline{toc}{section}{Литература}
\begin{thebibliography}{99}


\bibitem{Heidelberg} %1
\Au{Heidelberger P.} Fast simulation of rare events in queuieng and
relaibility models~// Performance Evaluation of Computers and
Communications Systems.  Lecture Notes in Computer Sci., 1993. Vol.~729. P.~165--202.

\bibitem{Asmus} %2
\Au{Asmussen S.}  Applied probability and queues. 2nd ed.~--- NY: Springer,
2003.

\bibitem{GlassHeid96}
\Au{Glasserman P., Heidelberger~P., Shahabuddin~P.,  Zajic~T.} 
A~look at multilevel splitting~// Monte Carlo and Quasi Monte Carlo
Methods. Lecture Notes in Statistics, 1996. Vol.~127. P.~99--108.

\bibitem{Morozov-asymptotics}  
\Au{Морозов Е.\,В.} Асимптотики вероятности
больших уклонений стационарной очереди~// Информатика и её
применения, 2009. Т.~3. Вып.~3. С.~23--34.

\bibitem{avb4} \Au{Бородина А.\,В., Морозов Е.\,В.}
 Ускоренное регенеративное моделирование вероятности перегрузки односерверной
    очереди~// Обозрение прикладной и\linebreak промышленной математики, 2007. Т.~14. Вып.~3. С.~385--397.

\bibitem{avb7}     
\Au{Бородина А.\,В., Морозов Е.\,В.} Ускоренное состоятельное оценивание
    вероятности большой загрузки в сис\-те\-мах $M/G/1, GI/G/1$~//
     Статистические методы оценивания и проверки гипотез.~---
      Пермь: Пермский университет, 2007. С.~124--140.

\bibitem{diss_2008} 
\Au{Бородина А.\,В.} Регенеративная модификация метода расщепления
для оценивания вероятности перегрузки в системах обслуживания:
Дис. \ldots канд.\ \mbox{ф.-м. н.}~--- ПетрГУ, 2008.

\bibitem{AG}
{\it Asmessen S., Glynn P.} Stochastic simulation: Algorithms and
analysis.~--- N.Y.: Springer, 2007.

\bibitem{Alt91}
\Au{Villen-Altamirano M.,  Villen-Altamirano~J.} 
RESTART: A~method
for accelerating rare event simulations~// Queueing, Performance and
Control in ATM: 13th  Teletraffic
Congress (International) Proceedings, 1991.\linebreak P.~71--76.

\bibitem{Melas} 
\Au{Ермаков С.\,М., Мелас В.\,Б.}  Математический эксперимент
с  моделями сложных стохастических систем.~--- СПб.:  СПбГУ, 1993. 270~с.

\bibitem{KahnHarris51}
\Au{Kahn H., Harris~T.\,E.} Estimation of particle transmission by
random sampling~// National Bureau of Standards Applied Mathematics
Ser., 1951.

\bibitem{Bayes70}
\Au{Bayes A.\,J.} Statistical techniques for simulation models~//
The Australian Computer J., 1970. Vol.~2. No.~4.\linebreak P.~180--184.

\bibitem{Bayes72}
\Au{Bayes A.\,J.} A minimum variance technique for simulation models~// J. 
Association for Computing Machinery, 1972. Vol.~19. P.~734--741.


\bibitem{GlassHeidProc96}
{\it Glasserman P.,  Heidelberger P.,  Shahabuddin~P.,  Zajic~T.}
Splitting for rare event simulation: analysis of simple cases~// 
1996 Winter Simulation Conference Proceedings, 1996. P.~302--308.

\bibitem{Garvels2000}
\Au{Garvels M.}
The splitting method in rare event simulation. PhD Thesis.~--- The
University of Twente, The Netherlands, 2000.

\bibitem{Cerou}
\Au{Cerou T.,  Guyader A.}  Adaptive multilevel splitting for rare
event analysis~// INRIA, Oct. 2005. Research Report No.\,5710.

\bibitem{avb1}  
\Au{Бородина А.\,В., Морозов Е.\,В.} Доверительное оценивание вероятности переполнения буфера на основе ускоренного
    регенеративного моделирования системы $M/M/1$~// Тр. ИПМИ КарНЦ РАН, 2006. Т.~7. С.~125--135.

\bibitem{avb2} 
\Au{Borodina A., Morozov E.}
Simulation of rare events with speed-up techniques: Splitting and
RESTART~//  Finnish Data Processing Week at
   the Petrozavodsk State University (FDPW'2005) Proceedings, 2006. Vol.~7. P.~152--173.

\bibitem{avb3}  
\Au{Borodina A.} Rare events regenerativ estimation of queues based on splitting~//  
Distributed Computer and Communication Networks: Theory and Application
    (\mbox{DCCN'2007}): International
    Workshop Proceedings.~--- Moscow: IITP RAS, 2007. Vol.~1. P.~50--55.

\bibitem{avb5}  
\Au{Бородина А.\,В.} Влияние зависимости циклов, полученных методом расщепления,
    при доверительном оценивании вероятности перегрузки в системе $M/G/1$~// Тр.\
     ИПМИ КарНЦ РАН, 2007. Т.~8.\linebreak С.~76--83.

\bibitem{avb6}  
\Au{Borodina A.\,V.,  Morozov E.\,V.}
Speed-up consistent estimation of a high workload probability in
$M/G/1$
    queue~// Trans. of XXVI  Seminar ((International) ``Stability
     Problems for Stochastic Models.'' Nahariya, Israel, 2007. Vol.~I. P.~36--42.

\bibitem{avb9}    
\Au{Borodina A., Morozov E.} 
A regenerative modification of the multilevel splitting~//
7th  Workshop (International) on Rare Event Simulation (RESIM 2008) Proceedings. September 24--26.
Renn, 2008. P.~276--282.

\bibitem{Glynn_Witt_94} 
\Au{Glynn  P.,  Witt W.}  Logarithmic asymptotics for steady-state
tail probabilities in a single-server queue~// J. Appl. Probab., 1994. P.~131--156.

\bibitem {Sadowsky} 
\Au{Sadowsky J.} Large   deviations theory and
efficient simulation of excessive backlogs in a
\emph{GI}/\emph{GI}/\emph{m} queue~// IEEE Trans. on Automatic Control, 1991. Vol.~36. No.~12. P.~1383--1394.

\bibitem{Will}
\Au{Willinger W., Taqqu M.,  Sherman~R.,   Wilson~D.}
Self-similarity through high-variability: Statistical analysis of
Ethernet LAN traffic at the source level~// IEEE/ACM Trans. on
Networking, 1997. Vol.~5. No.~1. P.~71--86.

\bibitem{Crovella&Bestavros97} 
\Au{Crovella M.\,E., Bestavros A. } 
Self-similarity in World Wide Web traffic: evidence and possible causes~// IEEE/ACM
Trans. on Networking, 1997. Vol.~5. No.~6. P.~835--847.

\bibitem{Starobinski&Sidi2000} \Au{Starobinski D.,  Sidi M.} Modeling and analysis
of power-tail distributions via classical teletraffic methods~//
Queueing Systems, 2000. Vol.~36. P.~243--267.

\bibitem{Nadarajah&Kotz2006} 
\Au{Nadarajah S.,  Kotz S.} On the Laplace transform
of the Pareto distribution~// Queueing Systems, 2006.  Vol.~54. P.~243--244.

\bibitem{Gradshteyn&Ryzhik2000}\Au{Gradshteyn I.\,S.,  Ryzhik I.\,M.} 
Table of integrals,
series, and products. 6th ed.~--- San Diego: Academic Press, 2000.

\bibitem{BDM}  \Au{Бородина А., Дюденко  И., Морозов Е.}  Ускоренное
оценивание вероятности переполнения  регенеративной очереди~ // 
Обозрение прикладной и промышленной математики, 2009.  Вып.~4. Т.~16. С.~577--593.

\label{end\stat}


\bibitem{natalie} \Au{\it Steiger N.\,M., Wilson J.\,R.} An improved
batch means procedure for simulation output analysis~// Management
Science, INFORMS, 2002. Vol.~48. No.~12. P.~1569--1586.
 \end{thebibliography}
}
}


\end{multicols}