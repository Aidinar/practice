\def\stat{abstr}
{%\hrule\par
%\vskip 7pt % 7pt
\raggedleft\Large \bf%\baselineskip=3.2ex
A\,B\,S\,T\,R\,A\,C\,T\,S \vskip 17pt
    \hrule
    \par
\vskip 21pt plus 6pt minus 3pt }


\def\tit{ESTIMATION OF SELF-HEALING TIME FOR DIGITAL 
SYSTEMS UNDER TRANSIENT FAULTS}

%1
\def\aut{S.\,L.~Frenkel$^1$ and A.\,V.~Pechinkin$^1$}

\def\auf{$^1$IPI RAN, Slf-ipiran@mtu-net.ru\\[1pt]
$^2$IPI RAN, apechinkin@ipiran.ru}

\def\leftkol{\ } % ENGLISH ABSTRACTS}

\def\rightkol{\ } %ENGLISH ABSTRACTS}

\titele{\tit}{\aut}{\auf}{\leftkol}{\rightkol}

%\vspace*{-2pt}

\noindent 
This paper suggests a new approach to self-healing property  prediction for digital systems.  
Self-healing refers to the system's ability to continue operating properly in the case of the 
failure of some of its components. This phenomenon is very considerable aspect of high-reliable 
systems design. The self-healing time characteristics are analyzed during design process, and 
the computation of probability distribution function of self-healing time needs for fair prediction 
of real time systems reliability. 
This paper considers the possible ways of estimation of time to self-healing under transient faults using 
a Markov model of  a design behavior for reliability analysis of digital systems with some fault-tolerant 
properties, modeled by the well-known Finite State Machine formalism. 


\label{st\stat}

%\vspace*{-5pt}

\KWN{fault-tolerant computer; self-healing
fault-tolerance; transient faults; finite state machine; Markov chains
}


\vskip 14pt plus 6pt minus 3pt

%\vfil



%2
\def\tit{ON STABILITY FOR NONSTATIONARY QUEUEING SYSTEMS WITH CATASTROPHES}

\def\aut{A.\,I.~Zeifman$^1$,  A.\,V.~Korotysheva$^2$, Ya.\,A.~Satin$^3$, and~S.\,Ya.~Shorgin$^4$}

\def\auf{$^1$Vologda State Pedagogical University;
IPI RAN; VSCC CEMI RAS, a\_zeifman@mail.ru\\[1pt]
$^2$Vologda State Pedagogical University, a\_korotysheva@mail.ru\\[1pt]
$^3$Vologda State Pedagogical University, yacovi@mail.ru\\[1pt]
$^4$IPI RAN, SShorgin@ipiran.ru}


\def\leftkol{\ } % ENGLISH ABSTRACTS}

\def\rightkol{\ } %ENGLISH ABSTRACTS}

\titele{\tit}{\aut}{\auf}{\leftkol}{\rightkol}

%\vspace*{-2pt}

\noindent
Nonstationary birth and death processes (BDPs) with
catastrophes are considered. The bounds of stability for some characteristics of such systems are obtained. 
Also, a queueing example is considered.

%\vspace*{-5pt}

\KWN{nonstationary queues; Markovian models with catastrophes; stability; bounds; limiting characteristics;
approximations}
%\pagebreak

\vskip 14pt plus 6pt minus 3pt
%\vskip 14pt plus 9pt minus 6pt

%3
\def\tit{BAYESIAN QUEUEING AND RELIABILITY MODELS: CHARACTERISTICS OF MEAN NUMBER 
OF~CLAIMS IN THE SYSTEM $M|M|1|\infty$}

\def\aut{A.\,A.~Kudriavtsev$^1$ and S.\,Ya.~Shorgin$^2$}
\def\auf{$^1$Faculty of Computational Mathematics and Cybernetics, 
M.\,V.~Lomonosov Moscow State University,\\
$\hphantom{^1}$nubigena@hotmail.com\\[1pt]
$^2$IPI RAN, sshorgin@ipiran.ru}

\def\leftkol{\ } % ENGLISH ABSTRACTS}

\def\rightkol{\ } %ENGLISH ABSTRACTS}

%\def\leftkol{ENGLISH ABSTRACTS}

%\def\rightkol{ENGLISH ABSTRACTS}

\titele{\tit}{\aut}{\auf}{\leftkol}{\rightkol}

%\vspace*{-2pt}
\noindent
The paper is a continuation of the series of papers devoted to Bayesian 
queuing and reliability models investigation. The probability characteristics 
of mean number of claims in the system $M|M|1|\infty$ under conditions of input flow 
and service parameters randomization are considered. The paper includes the 
discussion of the obtained results interpretation taking into 
account that the mean number of claims distribution may be improper.  

%\vspace*{-5pt}

\KWN{Bayesian approach; queueing systems; reliability; mixed distributions; modeling; improper distribution; 
``defective'' distribution}
\pagebreak


%\vfil
%\vskip 8pt plus 6pt minus 3pt

%4
\def\tit{QUEUEING SYSTEMS WITH MINIMIZED AVERAGE WAITING LINE}


\def\aut{S.\,S.~Matveeva$^1$ and T.\,V.~Zakharova$^2$}
\def\auf{$^1$Department of Mathematical Statistics, Faculty of
Computational Mathematics and Cybernetics,\\  
\hphantom{$^1$}M.\,V.~Lomonosov Moscow State University, petkin@mail.ru\\[1pt]
$^2$Department of Mathematical Statistics, Faculty of
Computational Mathematics and Cybernetics,\\  
\hphantom{$^1$}M.\,V.~Lomonosov Moscow State University,
lsa@cs.msu.su}

\def\leftkol{\ } % ENGLISH ABSTRACTS}

\def\rightkol{\ } %ENGLISH ABSTRACTS}

\titele{\tit}{\aut}{\auf}{\leftkol}{\rightkol}

\vspace*{-2pt}

\noindent
A research of the properties of optimal arrangements is made in accordance 
with the average waiting line criteria in space for systems with FIFO service discipline. 
A stream of the homogeneous requirements differing only by the moments of receipt 
in system is considered. The service centers are being like $M|G|1$ systems. The properties of 
optimal arrangements are described and the algorithms for constructing asymptotically optimal 
arrangements, minimizing the optimal criteria are presented.

%\label{st\stat}

\vspace*{-5pt}

\KWN{asymptotically optimal arrangements; average waiting line; criterion of optimality}

%\pagebreak

% \thispagestyle{headings}



%\vful

 %\vskip 14pt plus 6pt minus 3pt

% \vskip 24pt plus 9pt minus 6pt
\vskip 4pt plus 3pt minus 3pt


%5
\def\tit{ON ESTIMATION OF THE LARGE DEVIATION ASYMPTOTIC OF A SINGLE SERVER REGENERATIVE STATIONARY QUEUE}


\def\aut{A.\,V.~Borodina$^1$ and E.\,V.~Morozov$^2$}

\def\auf{$^1$Institute of Applied Mathematical  Research, Karelian Research Centre RAS, borodina@krc.karelia.ru\\[1pt]
$^2$Institute of Applied Mathematical  Research, Karelian Research Centre RAS, emorozov@krc.karelia.ru\\[1pt]
}


\def\leftkol{ENGLISH ABSTRACTS}

\def\rightkol{ENGLISH ABSTRACTS}

\titele{\tit}{\aut}{\auf}{\leftkol}{\rightkol}

\vspace*{-2pt}

\noindent
The  (small) probabilities estimation of  such  undesirable  events  like   
loss/collapse   of data,  buffer overflow,  collision of  packets  in the modern telecommunication systems 
by classical methods  requires  unacceptable   large  time and computational efforts. 
However, exact analytical   results are known  only  for    a narrow class of   queues 
and queueing networks.  It   calls  a necessity   to  develop    both asymptotic methods 
of analysis and  speed up  simulation  to estimate the  probabilities of this type.
In this paper,   a speed-up  simulation  method   based on the splitting of   the 
trajectories of a regenerative process   developed by the authors  is applied to  
estimation of the   overflow probability  for  a  stationary workload/queue-size process.  
It allows to simplify and accelerate  considerably  the estimation   of the exponent  in 
the asymptotic  representation  of the large deviation probability  provided  that   
service time has a finite moment generating   function (the so-called light tail).  
Numerical simulation results are  presented.


\vspace*{-6pt}

\KWN{large deviation asymptotic; single-server system; stationary waiting time;  
splitting method;  accelerated simulation}

%\vskip 18pt plus 6pt minus 3pt

 \vskip 4pt plus 6pt minus 3pt

% \pagebreak

%6
\def\tit{SEARCH OF THE CONFLICTS IN SECURITY POLICIES: A MODEL OF RANDOM GRAPHS}

\def\aut{A.\,A.~Grusho$^1$, N.\,A.~Grusho$^2$, and~E.\,E.~Timonina$^3$}
\def\auf{$^1$Department of Computer Security, Faculty of Information Protection, 
Institute of Information Sciences and\\
$\hphantom{^1}$Security Technologies, Russian State Humanitarian University; 
Department of Mathematical Statistics, Faculty
$\hphantom{^1}$of Computational Mathematics and Cybernetics, 
M.\,V.~Lomonosov Moscow State University, grusho@yandex.ru\\[1pt]
$^2$Department of Computer Security, Faculty of Information Protection, 
Institute of Information Sciences and\\
$\hphantom{^1}$Security Technologies, Russian State Humanitarian University, grusho@yandex.ru\\[1pt]
$^3$Department of Fundamental and Applied Mathematics, 
Institute of Information Sciences and Security\\
$\hphantom{^1}$Technologies, 
Russian State Humanitarian University, eltimon@yandex.ru}

\titele{\tit}{\aut}{\auf}{\leftkol}{\rightkol}

\vspace*{-6pt}

\noindent
Risk analysis requires at least a rough estimate of the 
probability of problematic situations in the implementation of security policy. 
Such estimates can be obtained in the analysis of conflicts in security policies 
using models of random graphs. In this paper, two examples of conflicts 
in security policy are considered and a model of random graphs for their research is constructed. 
The results of analysis were used to evaluate the influence of certain parameters 
of random graphs on the probability of the existence of conflicts and complexity of 
search algorithms in real conflict security policies.


\vspace*{-2pt}

\KWN{security policy; threats to information security; mathematical models of security policy}
%\pagebreak

\vskip 8pt plus 6pt minus 3pt

%7
\def\tit{LAWS OF ITERATED LOGARITHM FOR~NUMBERS OF~NONERROR
BLOCKS  UNDER ERROR CORRECTED CODING}

\def\aut{A.\,N.~Chuprunov$^1$ and I.~Fazekas$^2$}
\def\auf{$^1$Department of Mathematical Statistics and Probability, Chebotarev Institute of
Mathematics and Mechanics, $\hphantom{^1}$Kazan State
University, achuprunov@mail.ru\\[1pt]
$^2$Faculty of Informatics, University of Debrecen, Hungary, fazekasi@inf.unideb.hu}


\titele{\tit}{\aut}{\auf}{\leftkol}{\rightkol}

%\vspace*{-2pt}

\noindent
Messages  containing
blocks are considered. Each block is encoded with some antinoise coding method,
which can correct not more than $r$ mistakes. It is assumed that the
number of mistakes  in a block  is a nonnegative integer valued
random variable. The random variables are independent  and
identically distributed. Also, it is assumed  that the number of
mistakes in a message belongs to a some finite set of integer
numbers. In the paper, the analogs of the Law of Iterated
Logarithm for a number of nonerror blocks in a message is proved.

%\vspace*{-5pt}

\KWN{generalized allocation scheme; conditional probability;
conditional expectation; exponential inequality; Law of Iterated
Logarithm; Hamming code}
%\pagebreak

 \vskip 8pt plus 6pt minus 3pt

%8
\def\tit{ON AN ASYMPTOTIC BEHAVIOR OF POWER OF TESTS IN CASE OF GENERALIZED LAPLACE DISTRIBUTION
}

\def\aut{O.\,O.~Lyamin$^1$}
\def\auf{$^1$Faculty of Computational Mathematics and Cybernetics, 
M.\,V.~Lomonosov Moscow State University,\\
$\hphantom{^1}$oleg.lyamin@gmail.com}

\titele{\tit}{\aut}{\auf}{\leftkol}{\rightkol}

%\vspace*{-2pt}

\noindent
In paper by Bening and Lyamin (Informatics and Its Applications, 2009,
vol.~3, issue~3, pp.~79--88),  a formula for a 
limit of deviation of power of the asymptotically most powerful 
test from power of the most powerful test in case of generalized 
Laplace distribution was heuristically obtained. In the present paper, a formal proof of this formula
is derived.


%\vspace*{-5pt}


\KWN{generalized Laplace distribution; power function; 
asymptotically most powerful test; asymptotic expansion}
%\pagebreak

 \vskip 8pt plus 6pt minus 3pt

%9

\def\tit{SOME ISSUES OF~TECHNOLOGY SELECTION TO~BUILD VISUALIZATION SYSTEM IN~SITUATIONAL CENTER}

\def\aut{A.\,A.~Zatsarinny$^1$ and K.\,G.~Chuprakov$^2$}
\def\auf{$^1$IPI RAN, AZatsarinny@ipiran.ru\\[1pt]
$^2$IPI RAN, chkos@rambler.ru}

\titele{\tit}{\aut}{\auf}{\leftkol}{\rightkol}

\noindent
The main steps of visualization systems selection in situational centers are shown.
An accent is made on a step devoted to describing technology selection. 
Displaying technologies used in situational centers are considered.
The authors suggest the main parameters of visualization system and a methodological approach to
technology selection using approach of exclusion by parameters.

\KWN{visualization systems; situational center; displaying technology; selection problem;
exclusion approach}
%\pagebreak



\vskip 8pt plus 6pt minus 3pt

%10
\def\tit{COGNITIVE LINGUISTIC PRESENTATIONS IN THE TEXT PROCESSING SYSTEMS 
}
\def\aut{E.\,B.~Kozerenko$^1$ and I.\,P.~Kuznetsov$^2$}


\def\auf{$^1$IPI RAN, kozerenko@mail.ru\\[1pt]
$^2$IPI RAN, igor-kuz@mtu-net.ru}


%\def\leftkol{ENGLISH ABSTRACTS}

%\def\rightkol{ENGLISH ABSTRACTS}

\titele{\tit}{\aut}{\auf}{\leftkol}{\rightkol}

% \label{end\stat}

\noindent
The paper deals with the issues of design and development of syntactic semantic and lexical 
semantic presentations in linguistic processors of the systems based on the Extended Semantic 
Networks mechanism. The systems of this class are developed for knowledge extraction from natural 
language texts and mapping the extracted entities and relations into the knowledge base structures 
for further use by experts in application areas. This paper focuses on the cognitive linguistic 
solutions employed for constructing an integral linguistic model which can be modified depending 
on the specific task, and which range from the ``heavy'' form based on the specific deep presentations 
to the reduced shells focused on a particular subject area and the controlled language. Special attention 
is given to the techniques of describing the distributional and transformational features of language objects.


\KWN{intelligent systems; semantic presentations; linguistic processors; natural language processing; 
knowledge mining}

\vskip 8pt plus 6pt minus 3pt

%11
\def\tit{TYPOLOGY AND COMPUTER MODELING OF TRANSLATION DIFFICULTIES 
}
\def\aut{N.~Buntman$^1$, J.-L.~Minel$^2$, D.\,Le~Pesant$^3$, and~I.~Zatsman$^4$}


\def\auf{$^1$Faculty of Foreign Languages and Area Studies at Moscow State University, nabunt@hotmail.com\\[1pt]
$^2$Universit$\acute{\mbox{e}}$ Paris Ouest Nanterre La D$\acute{\mbox{e}}$fense, 
jean-luc.minel@u-paris10.fr\\[1pt]
$^3$Universit$\acute{\mbox{e}}$ Paris Ouest Nanterre La D$\acute{\mbox{e}}$fense, 
denis.lepesant@wanadoo.fr\\[1pt]
$^4$IPI RAN, iz\_ipi@a170.ipi.ac.ru}


%\def\leftkol{ENGLISH ABSTRACTS}

%\def\rightkol{ENGLISH ABSTRACTS}

\titele{\tit}{\aut}{\auf}{\leftkol}{\rightkol}

 \label{end\stat}

\noindent
The problem of generating of goal-oriented knowledge systems in 
linguistics is reviewed. The problem belongs to a new research area, which is called ``cognitive informatics.'' 
The article focuses on computer coding of goal-oriented knowledge systems using as an example the 
typology of Russian-to-French translation difficulties.

\KWN{time-dependent semiotic model; goal-oriented knowledge systems in linguistics; denotata; 
concepts; information objects; computer codes}

%\pagebreak

 