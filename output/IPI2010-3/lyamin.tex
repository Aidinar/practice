
\newcommand{\be}{\beta}


\newcommand{\III}{{\bf 1}}
\newcommand{\n}{{\cal N}}  % каллиграфические буквы

\newcommand{\ercx}[1]{\:{\sf erfc}\left(#1\right)}
\newcommand{\ercf}{{\ercx{\frac{b}{2\sqrt{a}}}}}
\newcommand{\erx}[1]{\:{\sf erf}\left(#1\right)}
\newcommand{\cab}{C_{a,b}}
%\newcommand{\cabf}{\fr{\sqrt{a}}{\sqrt{\pi}\ercf\exp\left(\frac{b^2}{4a}\right)}}
\newcommand{\iab}{I_{a,b}}
\newcommand{\sign}[1]{\:{\sf sign}\left(#1\right)}

\def\stat{lyamin}

\def\tit{О ПРЕДЕЛЬНОМ ПОВЕДЕНИИ МОЩНОСТЕЙ КРИТЕРИЕВ В~СЛУЧАЕ ОБОБЩЕННОГО РАСПРЕДЕЛЕНИЯ ЛАПЛАСА}

\def\titkol{О предельном поведении мощностей критериев в~случае обобщенного распределения Лапласа}

\def\autkol{О.\,О.~Лямин}
\def\aut{О.\,О.~Лямин$^1$}

\titel{\tit}{\aut}{\autkol}{\titkol}

%{\renewcommand{\thefootnote}{\fnsymbol{footnote}}\footnotetext[1]
%{Исследование поддержано грантами РФФИ 08-07-00152 и 09-07-12032.
%Статья написана на основе материалов доклада, представленного на IV 
%Международном семинаре  <<Прикладные задачи теории вероятностей и математической статистики, 
%связанные с моделированием информационных систем>> (зимняя сессия, Аоста, Италия, январь--февраль 2010~г.).}}

\renewcommand{\thefootnote}{\arabic{footnote}}
\footnotetext[1]{Московский государственный
университет им. М.\,В. Ломоносова, факультет вычислительной математики и кибернетики,
oleg.lyamin@gmail.com}

\Abst{В работе~\cite{article} на эвристическом уровне была получена формула для предела 
отклонения мощности асимптотически наиболее мощного критерия от мощности наилучшего критерия 
в случае обобщенного распределения Лапласа. В~данной работе приводится формальное доказательство этой формулы.}

\KW{обобщенное распределение Лапласа; функция
мощности; асимптотически наиболее мощный критерий; асимптотическое
разложение}

       \vskip 14pt plus 9pt minus 6pt

      \thispagestyle{headings}

      \begin{multicols}{2}

      \label{st\stat}

\section{Введение}

Распределение Лапласа находит широкое применение в задачах
моделирования больших рисков, выделения сигналов на фоне помех и
других задачах математической статистики (см., например,~[2--4]). Данная работа продолжает
исследования, начатые в~\cite{article}, и содержит строгое
доказательство результатов, полученных в указанной работе на
эвристическом уровне. Здесь используются те же обозначения, что и в~\cite{article}.

В работе~\cite{article} была рассмотрена задача проверки гипотезы
$$
{\sf H}_0 \::\: \theta = 0
$$
против последовательности близких альтернатив вида
$$
{\sf H}_{n,1} \::\: \theta = \fr{t}{\sqrt{n}}\,, \quad 0 < t \leq C\,, \enskip C > 0\,,
$$
на основе выборки $(X_1, \dots, X_n)$~--- независимых одинаково распределенных наблюдений, 
имеющих распределение с плотностью вида
\begin{equation}
\label{l1}
p(x, \theta) = \cab e^{-a(x - \theta)^2- b|x - \theta|}\,,\quad x, \theta \in {\mathbb{R}}\,,
\end{equation}
где $\cab$~--- константа нормировки. Предположим, что $a>0$, $b>0$. В~этом случае константа нормировки 
$$
\cab = \fr{\sqrt{a}}{\sqrt{\pi}\;{\sf erfc}\left (b/(2\sqrt{a})\right )\exp \left(b^2/(4a)\right)}\,,
$$
где
$$
\ercx{x} = \fr{2}{\sqrt{\pi}}\int\limits_x^{\infty} e^{-z^2}\,dz\,.
$$
Для каждого фиксированного $t\in(0,C]$ обозначим через~$\beta_n^*(t)$ мощность наилучшего критерия 
уровня $\alpha \in (0,1)$. Такой критерий всегда существует согласно фундаментальной лемме 
Неймана--Пир\-со\-на и основан на логарифме отношения правдоподобия
\begin{multline*}
\Lambda_n(t) = \sum_{i = 1}^n\left(atn^{-1/2}\left(2X_i - tn^{-1/2}\right) +{}\right.\\
\left.{}+ b\left(|X_i| - |X_i - tn^{-1/2}|\right)\right)\,.
%\label{l2}
\end{multline*}
Для проверки ${\sf H}_0$ против~${\sf H}_{n,1}$ существуют критерии, основанные на отличных от~$\Lambda_n(t)$ 
статистиках и имеющие ту же предельную мощность, что и~$\beta_n^*(t)$. Такие критерии называются 
асимптотически наиболее мощными (АНМ), причем эти критерии не зависят от~$t$. Предположим, что статистику~$T_n$ 
некоторого АНМ-кри\-те\-рия можно монотонным преобразованием (не меняющим мощности критерия) преобразовать в 
статистику~$S_n(t)$ такую, что величина $\Delta_n(t) = S_n(t) - 
\Lambda_n(t)$ допускает при гипотезе~${\sf H}_0$ асимптотическое разложение вида
\begin{equation}
\Delta_n(t) = n^{-1/2} L_n(t) +  o (n^{-1/2})\,,
\label{l3}
\end{equation}
где $L_n(t)$~--- некоторая статистика. Было обнаружено (см.~\cite{bening}), что в этом случае 
для широкого класса АНМ-кри\-те\-ри\-ев мощность~$\beta_n(t)$ критерия, основанного на~$S_n(t)$ (или на~$T_n$), 
отличается от~$\beta_n^*(t)$ на величину порядка~$1/n$.

Для распределения~(\ref{l1}) рассмотрим АНМ-кри\-те\-рий, основанный на статистике
\begin{equation*}
T_n = \fr{1}{\sqrt{n}} \sum_{i = 1}^n\left(2aX_i + b\sign{X_i}\right)\,.
%\label{l4}
\end{equation*}
Обозначим через~$\beta_n(t)$ мощность этого критерия уровня $\alpha \in (0,1)$. 
В~работе \cite{article} было показано, что для этого критерия величина~$\Delta_n(t)$ допускает при 
гипотезе~${\sf H}_0$ асимптотическое разложение несколько иного, чем в~(\ref{l3}), вида, а именно
$$
\Delta_n(t) = n^{-1/4} L_n(t) +  o (n^{-1/4})\,,
$$
в связи с чем мощность критерия~$\beta_n(t)$ отличается от~$\beta_n^*(t)$ на величину порядка $1/\sqrt{n}$, 
а не~$1/n$, как следовало ожидать. Там же на эвристическом уровне была получена формула
\begin{multline}
r(t) \equiv \lim_{n \to \infty} \sqrt{n}\left(\beta_n^*(t) - \be_n(t)\right) = {}\\
{}=
\fr{2b^2\cab t^2}{3\sqrt{\iab}} \varphi\left(u_{\alpha} - t\sqrt{\iab}\right)\,,
\label{l5}
\end{multline}
где $\Phi(x)$, $\varphi(x)$~--- соответственно функция распределения и плотность стандартного нормального закона, 
$\Phi(u_{\alpha}) = 1 - \alpha$, $\alpha \in (0,\,1)$; $\iab$~--- фишеровская информация.

В этой работе для распределения~(\ref{l1}) будут проверены условия теоремы~3.2.1 из~\cite{bening}, т.\,е.\ 
получено формальное доказательство формулы~(\ref{l5}).

\section{Основные обозначения}

В соответствии с определениями работы~\cite{article} введем следующие обозначения:
\begin{align}
S_n(t) &= tT_n - \fr{\iab t^2}{2}\,;\notag\\
\Delta_n(t) &= S_n(t) - \Lambda_n(t) = -b\cab t^2 -{}\notag\\
&{}-2b\sum\limits_{i=1}^n\left(X_i-\fr{t}{\sqrt{n}}\right)
{\III}_{\left[0,t/\sqrt{n}\right]}(X_i)\,,
\label{l7}
\end{align}
где ${\III}_A(x)$~--- индикатор множества~$A$. В~дальнейшем будем опускать аргумент~$t$ 
и писать просто $\Lambda_n, S_n$ и~$\Delta_n$. Также положим
$$
\erx{x} = 1 - \ercx{x} = \fr{2}{\sqrt{\pi}} \int\limits_0^x e^{-z^2}\,dz\,.
$$
Для $\theta > 0$ и произвольного $x \in {\mathbb{R}}$ определим функции
\begin{equation}
g_{\theta}(x) = \cab \theta^2 + 2(x-\theta){\III}_{[0,\theta]}(x)\,;
\label{l8}
\end{equation}
\vspace*{-6pt}

\noindent
\begin{multline}
m_{\theta}(x) = a(2x - \theta) + b
\left[ \vphantom{\fr{2x}{\theta}}
-{\III}_{(-\infty,\,0)}(x) +{}\right.\\
{}\left.+ \left(\fr{2x}{\theta} - 1\right){\III}_{[0,\,\theta]}(x) + 
{\III}_{(\theta,\,\infty)}(x)\right]\,;
\label{l9}
\end{multline}

\vspace*{-6pt}

\noindent
\begin{equation}
M_{\theta}(x) = \theta m_{\theta}(x)\,;
\label{l10}
\end{equation}

\vspace*{-6pt}

\noindent
\begin{equation*}
d_{\theta}(x) = -\fr{\iab \theta}{2} + 2ax+b\sign{x}\,;
%\label{l11}
\end{equation*}
\begin{equation*}
D_{\theta}(x) = \theta d_{\theta}(x)\,.
%\label{l12}
\end{equation*}
Обозначим $g_n(x)$, $m_n(x)$, $M_n(x)$, $d_n(x)$ и $D_n(x)$ соответственно функции~$g_{\theta}(x)$, 
$m_{\theta}(x)$,  $M_{\theta}(x)$, $d_{\theta}(x)$ и~$D_{\theta}(x)$ при $\theta=tn^{-1/2}$.

Перепишем статистики в обозначениях введенных функций:
\begin{align}
\Lambda_n &= tn^{-1/2} \sum\limits_{i=1}^n m_n(X_i) = \sum\limits_{i=1}^n M_n(X_i)\,; \label{l13}\\
S_n &= tn^{-1/2} \sum\limits_{i=1}^n d_n(X_i) = \sum\limits_{i=1}^n D_n(X_i)\,; \label{l14}\\
\Delta_n &= -b\sum\limits_{i=1}^n g_n(X_i)\,. \label{l15}
\end{align}
Договоримся обозначать $f_{S_n}(s)$, $f_{\Lambda_n}(s)$, $f_{m_{\theta}}(s)$, 
$f_{M_{\theta}}(s)$, $f_{d_{\theta}}(s)$, $f_{D_{\theta}}(s)$ характеристические
 функции соответственно случайных величин~$S_n$, 
$\Lambda_n$, $m_{\theta}(X_1)$, $M_{\theta}(X_1)$, $d_{\theta}(X_1)$, $D_{\theta}(X_1)$. 
Смена индекса~$\theta$ на~$n$ будет обозначать, что эти функции рассматриваются при $\theta=tn^{-1/2}$. Заметим, что
\begin{multline*}
f_{M_{\theta}}(s) = {\e}_0 \exp\left\{isM_{\theta}(X_1)\right\} ={}\\
{}= {\e}_0 \exp\left\{is\theta m_{\theta}(X_1)\right\} = f_{m_{\theta}}(\theta s)\,,
\end{multline*}
$$
f_{D_{\theta}}(s) = f_{d_{\theta}}(\theta s)\,.
$$
Аналогично
\begin{align}
f_{M_n}(s) & = f_{m_n}(tn^{-1/2}s)\,, \label{l16}\\
f_{D_n}(s)&= f_{d_n}(tn^{-1/2}s)\,. \label{l17}
\end{align}
Также будем пользоваться показанным в~\cite{article} соотношением между константой нормировки и фишеровской информацией
$$
\iab = 2(b\cab+a)\,.
$$

\section{Вспомогательные результаты}

\noindent
\textbf{Лемма~3.1.}
{\it Для любых $0 < t_1 < t_2$ и распределения}~(\ref{l1}) \textit{справедливы равенства}
\begin{multline*}
\sup_{t\in[t_1,t_2]} \sup_{x \in {\mathbb{R}}} 
 \left| \vphantom{\fr{f_2(x)}{\sqrt{n}}}
 {\p}_{n, 0}\left(\Lambda_n < x\right) -{}\right.\\
\left. {}- f_1(x,t)
- \fr{f_2(x,t)}{\sqrt{n}} - \fr{f_3(x,t)}{n}\right| = o(n^{-1})\,,
\end{multline*}
\textit{где}
$$
f_1(x,t) = {\Phi}\left(\fr{x}{t\sqrt{I_{a,b}}} + \fr{\sqrt{I_{a,b}}t}{2}\right)\,;
$$

\noindent
\begin{multline*}
f_2(x,t) ={}\\
{}= \fr{b^2C_{a,b}t}{3\sqrt{I_{a,b}}}\left(\fr{x}{tI_{a,b}} - 
\fr{t}{2}\right) \varphi\left(\fr{x}{t\sqrt{I_{a,b}}} + \fr{\sqrt{I_{a,b}}t}{2}\right)\,;
\end{multline*}

\noindent
\begin{multline*}
f_3(x,t) = \fr{bC_{a,b}}{3I_{a,b}}\, \varphi\left(\fr{x}{t\sqrt{I_{a,b}}} + 
\fr{\sqrt{I_{a,b}}t}{2}\right) \times{}\\
{}\times \left\{\fr{t^2}{2\sqrt{I^3_{a,b}}}\left(
\fr{M_1(a, b, C_{a,b})x}{t} + \fr{2M_2(a, b, C_{a,b})t}{3I_{a,b}}\right) -{}\right.\\
{}- \fr{b^3C_{a,b}t^2}{6}\left(\fr{x}{tI_{a,b}} - \fr{t}{2}\right)^2\left(
\fr{x}{t\sqrt{I_{a,b}}} + \fr{\sqrt{I_{a,b}}t}{2}\right)- {}\\
{}-
\fr{M_3(a,b)t}{2\sqrt{I_{a,b}}}\left(\fr{x}{t\sqrt{I_{a,b}}}+ 
\fr{\sqrt{I_{a,b}}t}{2}\right)^2 +{}\\
{}
+ \fr{M_3(a,b)t}{2\sqrt{\iab}}+ \fr{M_3(a,b)}{4I_{a,b}}\left(
\fr{x}{t\sqrt{I_{a,b}}} + \fr{\sqrt{I_{a,b}}t}{2}\right)^3 - {}\\
\left.{}-
\fr{3M_3(a,b)}{4I_{a,b}}\left(\fr{x}{t\sqrt{I_{a,b}}} + \fr{\sqrt{I_{a,b}}t}{2}\right)
\vphantom{\fr{t^2}{2\sqrt{I^3_{a,b}}}}
\right\}\,;
\end{multline*}

\noindent
$$
M_1(a, b, C_{a,b}) = 6b^2\cab^2 + b(14a - b^2)\cab + 2a(4a - b^2)\,;
$$

\noindent
\begin{multline*}
M_2(a, b, C_{a,b}) = 18b^4\cab^4 + b^3(66a - b^2)\cab^3 +{}\\
{}
+ 2ab^2(45a-b^2)\cab^2 + a^2b(54a - b^2)\cab + 12a^4\,;
\end{multline*}

\noindent
$$
M_3(a,b) = 3I_{a,b} - b^2\,.
$$

\medskip

\noindent
Д\,о\,к\,а\,з\,а\,т\,е\,л\,ь\,с\,т\,в\,о\,. \
Для доказательства утверждения этой леммы применим предложение~1.1 из работы~\cite{chibisov} для $r = 4$.

Покажем равномерную интегрируемость семейства случайных величин $m_{\theta}^4(X_1)$ относительно~${\p}_0$. 
По определению равномерной интегрируемости необходимо показать, что
$$
\sup_{\theta} {\e}_0\left|m_{\theta}^4(X_1)\right| {\III}_{(c, \infty)}(\left|m_{\theta}^4(X_1)\right|) \to 0
\mbox{ при } c\to \infty\,.
$$
Используя~(\ref{l9}), представим ${\III}_{(c,\infty)}\left(|m_{\theta}^4(X_1)|\right)$ в следующем виде:
\begin{multline}
{\III}_{(c,\,\infty)}(|m_{\theta}^4(x)|) =
{\III}_{(\sqrt[4]{c},\,\infty)}(|m_{\theta}(x)|) ={}\\
{}
= {\III}_{\left(-\infty,\,x_1(c)\right) \cup \left(x_2(c),\,\infty\right)}(x)\cdot{\III}_{\left((a\theta + 
b)^4,\,\infty\right)}(c) +{}\\
{}+ {\III}_{\left(-\infty,\,x_3(c)\right) \cup \left( x_4(c),\,\infty\right)}(x)\cdot{\III}_{\left(0,\,(a\theta + 
b)^4\right)}(c)\,,
\label{l18}
\end{multline}
где
\begin{align*}
x_1(c) &= -\fr{\sqrt[4]{c}- a\theta - b}{2a}\,; &
x_2(c) &= \fr{\sqrt[4]{c} + a\theta - b}{2a}\,;\\
x_3(c) &= -\fr{\sqrt[4]{c} - a\theta - b}{2\left(a + {b}/{\theta}\right)}\,; &
x_4(c) &= \fr{\sqrt[4]{c} + a\theta + b}{2\left(a + b/\theta\right)}\,.
\end{align*}
При больших~$c$ второе слагаемое в~(\ref{l18}) равно нулю. Тогда для всякого $\theta>0$ 
непосредственным интегрированием получим
\begin{multline*}
{\e}_0 \left\vert m_{\theta}^4(X_1)\right\vert {\III}_{(c,\,\infty)}
(|m_{\theta}^4(X_1)|) ={}\\
{}= {\e}_0\left|m_{\theta}^4(X_1)\right|{\III}_{\left(-\infty,\,x_1(c)\right) \cup 
\left(x_2(c),\,\infty\right)}(X_1) ={}\\
{}= P_1(x_1(c)) \exp\left\{-\fr{(b + 2ax_1(c))^2}{4a}\right\} +{}\\
{}+ C_1 \ercx{\fr{b + 2ax_1(c)}{2\sqrt{a}}} +{}
\\
{}+ P_2(x_2(c)) \exp\left\{-\fr{(b + 2ax_2(c))^2}{4a}\right\} + {}\\
{}+
C_2 \ercx{\fr{b + 2ax_2(c)}{2\sqrt{a}}} \to 0 \mbox{ при } c \to \infty\,,
\end{multline*}
где $C_1$, $C_2$~--- константы, зависящие от~$a$, $b$ и~$\theta$; 
$P_1(y)$ и~$P_2(y)$~--- многочлены третьей степени с коэффициентами, зависящими от~$a,\,b$ и~$\theta$. 
Первое условие предложения~1.1 из~\cite{chibisov} доказано.

\smallskip

Для выполнения второго условия достаточно показать выполнение условия АС, для которого, 
в свою очередь, достаточно проверить условия теоремы~1.5 из~\cite{chibisov}.

\smallskip

Выберем $a_1$ и $a_2$ так, чтобы $a_2 > a_1 > t > 0$. 
Тогда условия~I, II и~III теоремы~1.5 очевидно выполнены. Условие~IV следует из того, что для 
каж\-дого~$\theta$ плотность $p_{\theta}(x)$ абсолютно непрерывна на любом отрезке, не содержащем 
точку $x = \theta$, а значит для любого $\theta \in [0, t]$ плотность~$p_{\theta}(x)$ абсолютно непрерывна 
на $x \in [a_1, a_2]$, так как $a_2 > a_1 > t$.

\smallskip

Обратимся к условию~V. Для каждого $\theta \in [0,\,t]$ функция~$p'_{\theta}(x)$ на отрезке 
$a_2 > x > a_1 > t \geq \theta$ определена как
\begin{multline*}
p'_{\theta}(x) = \fr{\sqrt{a} e^{-{b^2}/(4a)}}{\sqrt{\pi}\;{\sf erfc}\left(b/(2\sqrt{a})\right)}\times{}\\
{}\times e^{-b(x - \theta) - a(x - 
\theta)^2}\left(-b - 2a(x - \theta)\right)\,.
\end{multline*}
Тогда
\begin{multline*}
\int_{a_1}^{a_2} \left|p'_{\theta}(x)\right|^{2 + \delta}\,dx =
 \left(\fr{\sqrt{a} e^{-{b^2}/(4a)}}{\sqrt{\pi}\;{\sf erfc}\left(b/(2\sqrt{a})\right)}\right)^{2+\delta}\times{}\\
\!{}\times
\int\limits_{a_1}^{a_2}\! e^{-(a(x-\theta)^2+b(x-\theta))(2+\delta)} 
\left(b+2a(x-\theta)\right)^{2+\delta}\, dx \leq\hspace*{-0.78734pt}\\
\leq \left(\fr{\sqrt{a} e^{-{b^2}/(4a)}}{\sqrt{\pi}\;{\sf erfc}\left(b/(2\sqrt{a})\right)}\right)^{2+\delta} \times{}\\
{}\times
\int\limits_{a_1}^{a_2} \left(b + 2a(x - \theta)\right)^{2 + \delta}\, dx \leq\hspace*{10mm}
 \end{multline*}
 \begin{multline*}
{}
\leq \left(\fr{\sqrt{a} e^{-{b^2}/(4a)}}{\sqrt{\pi}\;{\sf erfc}\left(b/(2\sqrt{a})\right)}\right)^{2+\delta}\times{}\\
{}\times\left(b + 2a(a_2 - \theta)\right)^{2 + 
\delta}\left(a_2 - a_1\right) \leq{}\\
{}
\leq \left(\fr{\sqrt{a} e^{-{b^2}/(4a)}}{\sqrt{\pi}\;{\sf erfc}\left(b/(2\sqrt{a})\right)}\right)^{2+\delta}\,
\!\!\!\!\!\!\left(b + 2aa_2\right)^{2 + \delta}\,(a_2 - a_1)\,.\hspace*{-1.38pt}
\end{multline*}
Все условия теоремы~1.5 из~\cite{chibisov} выполнены, что дает возможность применить предложение~1.1, 
из которого непосредственно получается утверждение леммы. \hfill $\Box$


\smallskip

\noindent
\textbf{Лемма 3.2.}
\textit{При $\theta \to 0$ для функции $g_{\theta}(x)$ $($см.}\ (\ref{l8})) \textit{справедливы следующие соотношения:}
\begin{align*}
{\e}_{0} g_{\theta}(X_1) &= {\cal O}(\theta^3)\,;\\
{\e}_{0} g_{\theta}^2(X_1) &= {\cal O}(\theta^3)\,;\\
{\e}_{0} g_{\theta}^3(X_1) &= {\cal O}(\theta^4)\,.
\end{align*}

%\smallskip

\noindent
Д\,о\,к\,а\,з\,а\,т\,е\,л\,ь\,с\,т\,в\,о\,.\
В~работе~\cite{article} была получена характеристическая функция для случайной величины 
$(X_1-\theta){\III}_{[0,\theta]}(X_1)$. Из нее выводится характеристическая функция для $g_{\theta}(X_1)$, 
после чего задача вычисления моментов становится тривиальной. \hfill $\Box$


\smallskip

\noindent
\textbf{Лемма 3.3.}
\textit{Для статистики $\Delta_n$ $($см.} (\ref{l7})) \textit{при $n \to \infty$ справедливо соотношение}
$$
{\e}_{n,0} |\Delta_n|^k = {\cal O}\left(n^{-k/4}\right)\,,
$$
\textit{где $k$ --- произвольное натуральное число.}


\smallskip

\noindent
Д\,о\,к\,а\,з\,а\,т\,е\,л\,ь\,с\,т\,в\,о\,.\ Смотри доказательство следствия~3.1 в работе~\cite{article}.
\hfill $\Box$


\smallskip

\noindent
\textbf{Лемма 3.4.}
\textit{Справедливы следующие представления:}
\begin{multline*}
{\e}_{n,0}\exp\{is\Lambda_n\}\Delta_n ={}\\[6pt]
{}= -b n f^{n-1}_{M_n}(s) {\e}_0 \exp\left\{isM_n(X_1)\right\} g_n(X_1)\,;
\end{multline*}
\vspace*{-18pt}

\noindent
\begin{multline*}
{\e}_{n,0}\exp\{is\Lambda_n\}\Delta_n^2 ={}\\[6pt]
{}= b^2 n  f_{M_n}^{n-1}(s)  {\e}_0 \exp\left\{is M_n(X_1)\right\} g_n^2(X_1) +{}\\[6pt]
{}
+ b^2n(n-1) f_{M_n}^{n-2}(s)  \left[{\e}_0 \exp\left\{is M_n(X_1)\right\} g_n(X_1)\right]^2\,;
\end{multline*}
\vspace*{-18pt}

\noindent
\begin{multline*}
{\e}_{n,0}\exp\{isS_n\}\Delta_n ={}\\[6pt]
{}= -b n f^{n-1}_{D_n}(s) {\e}_0 \exp\left\{isD_n(X_1)\right\} g_n(X_1)\,.
\end{multline*}


\medskip

\noindent
Д\,о\,к\,а\,з\,а\,т\,е\,л\,ь\,с\,т\,в\,о\,.\
 Пользуясь~(\ref{l13}), (\ref{l15}), а также независимостью случайных величин $X_1, \dots, X_n$, 
 получаем первое утверждение
 
 \noindent
\begin{multline*}
{\e}_{n,0}\exp\{is\Lambda_n\}\Delta_n ={}\\[4pt]
{}
= -b  {\e}_{n,0}\sum_{i=1}^n \exp\left\{is\sum_{j=1}^n M_n(X_j)\right\} g_n(X_i)={}\\[4pt]
{}
= -b \sum\limits_{i=1}^n {\e}_{n,0} \exp\left\{is\sum_{j=1}^n M_n(X_j)\right\} g_n(X_i) ={}\\[4pt]
{}
= -b \sum\limits_{i=1}^n {\e}_{n,0}\left(
\vphantom{\prod^n_{j=1}}
\exp\left\{isM_n(X_i)\right\} g_n(X_i) \times{}\right.\\[4pt]
\left.{}\times \prod\limits_{j=1,  j\ne i}^n \exp\left\{isM_n(X_j)\right\} \right) ={}\\[4pt]
{}
= -b \sum\limits_{i=1}^n \left(
\vphantom{\prod^n_{j=1}}
{\e}_0 \exp\left\{isM_n(X_i)\right\} g_n(X_i) \times {}\right.\\[4pt]
\left.{}\times
\prod\limits_{j=1,j\ne i}^n {\e}_0 \exp\left\{isM_n(X_j)\right\}\right) ={}\\[4pt]
{}
= -b  f_{M_n}^{n-1}(s)  \sum\limits_{i=1}^n {\e}_0 \exp\left\{isM_n(X_i)\right\} g_n(X_i) ={}\\[4pt]
{}
= -b n f^{n-1}_{M_n}(s) {\e}_0 \exp\left\{isM_n(X_1)\right\} g_n(X_1)\,.
\end{multline*}
Второе утверждение доказывается по аналогии.

%\medskip

Чтобы доказать третье утверждение, необходимо воспользоваться представлением~(\ref{l14}) и 
пол\-ностью повторить доказательство первого утверж\-дения с заменой~$M_n$ на~$D_n$. Лемма доказана.
\hfill $\Box$

\medskip


\noindent
\textbf{Лемма 3.5.}
\textit{При $\theta \to 0$ для функций $M_{\theta}(x)$ $($см.}~(\ref{l10})) \textit{и~$g_{\theta}(x)$ 
$($см.}~(\ref{l8})) 
\textit{справедливо следующее соотношение:}
$$
{\e}_0 \left|M_{\theta}(X_1) g_{\theta}(X_1)\right| = {\cal O}(\theta^{5/2})\,.
$$


\medskip

\noindent
Д\,о\,к\,а\,з\,а\,т\,е\,л\,ь\,с\,т\,в\,о\,.\
По неравенству Коши--Бу\-ня\-ков\-ского
$$
{\e}_0 \left|M_{\theta}(X_1) g_{\theta}(X_1)\right| \leq 
\sqrt{{\e}_0 M_{\theta}^2(X_1)  {\e}_0 g_{\theta}^2(X_1)}\,.
$$
Используя полученную в работе~\cite{article} характеристическую функцию случайной величины~$M_{\theta}(X_1)$, 
вычислим второй момент $M_{\theta}(X_1)$. Очевидно,
$$
{\e}_0 M_{\theta}^2(X_1) = {\cal O}(\theta^2)\,.
$$
Используя лемму~3.2 для оценки ${\e}_0 g_{\theta}^2(X_1)$, получим утверждение леммы. \hfill $\Box$

\medskip


\noindent
\textbf{Лемма 3.6.}
\textit{При $\theta \to 0$ для функций $M_{\theta}(x)$ и $g_{\theta}(x)$ справедливы следующие соотношения:}
$$
{\e}_0 \exp\left\{isM_{\theta}(X_1)\right\} = 1 + \fr{is(is-1)\iab \theta^2}{2} +  o(\theta^2)\,;
$$

\noindent
\begin{align*}
\hspace*{-0.2pt}{\e}_0 \exp\left\{isM_{\theta}(X_1)\right\} g_{\theta}(X_1) &= \fr{b\cab(1+is)\theta^3}{3} +  o(\theta^3);
\\
\hspace*{-0.2pt}{\e}_0 \exp\left\{isM_{\theta}(X_1)\right\} g_{\theta}^2(X_1) &= \fr{4\cab \theta^3}{3} +  o(\theta^3).
\end{align*}

\medskip

\noindent
Д\,о\,к\,а\,з\,а\,т\,е\,л\,ь\,с\,т\,в\,о\,.\ 
Первое утверждение~--- разложение характеристической функции случайной величины $M_{\theta}(X_1)$ 
в ряд~--- было получено в работе~\cite{article}. Второе и третье утверждения доказываются 
непосредственным взятием соответствующих интегралов и разложением результатов интегрирования в ряд в 
точке $\theta=0$. \hfill $\Box$

\medskip

\noindent
\textbf{Лемма 3.7.}
\textit{Для любого $x\in{\mathbb{R}}$ и произвольного действительного $y\ne0$}
$$
\int\limits_x^{\infty} e^{-z^2}|\cos yz|\,dz < \int\limits_x^{\infty} e^{-z^2}\,dz\,.
$$


\medskip

\noindent
Д\,о\,к\,а\,з\,а\,т\,е\,л\,ь\,с\,т\,в\,о\,.\  Фиксируем произвольное $y\ne$\linebreak $\ne 0$ и рассмотрим функцию
$$
u(z) = e^{-z^2}\left(1-\left\vert \cos yz\right\vert \right)\,.
$$
Функция~$u(z)$ обращается в нуль в точках
$$
z_k=\fr{\pi k}{y}\,, \enskip k=0,\pm1,\pm2,\dots
$$
Обозначим через~$\bar{z}_1$ и~$\bar{z}_2$ произвольные точки серии~$z_k$, лежащие правее~$x$ и следующие 
в серии друг за другом. Выберем $\epsilon > 0$ так, чтобы $\epsilon < (\bar{z}_2-\bar{z}_1)/2$. 
Так как $u(z)\geq0$ для любого~$z$, то по свойствам интегралов
\begin{multline*}
\int\limits_x^{\infty} e^{-z^2}\,dz - \int\limits_x^{\infty} e^{-z^2}|\cos yz|\,dz ={}\\
{}= \int\limits_x^{\infty}u(z)dz \geq \int\limits_{\bar{z}_1+\epsilon}^{\bar{z}_2-\epsilon}u(z)\,dz > 0\,,
\end{multline*}
так как на $z \in (\bar{z}_1+\epsilon,\bar{z}_2-\epsilon)$ выполнено $|\cos yz|<1$, а значит и $u(z) > 0$.
Лемма доказана. \hfill $\Box$

\noindent

\textbf{Следствие.} \textit{Для любого $x\in{\r}$ и произвольного действительного $y\ne0$}
$$
\fr{2}{\sqrt{\pi}}\int\limits_x^{\infty} e^{-z^2}|\cos yz|\,dz < \ercx{x}\,.
$$


\medskip

\noindent
\textbf{Лемма 3.8.}
\textit{Для модуля характеристической функции $f_{D_n}(s)$ справедлива следующая оценка:}
\begin{align*}
|f_{D_n}(s)| &\leq \exp\left\{-\fr{\epsilon_1 s^2}{n}\right\}\,,   & |s|&\leq\sigma_1 \sqrt{n}\,;\\ 
|f_{D_n}(s)| &\leq q_d < 1\,,  & |s|&\geq\sigma_1 \sqrt{n}\,, 
\end{align*}
\textit{где $\sigma_1$, $\epsilon_1$~--- положительные константы; $q_d$~--- положительная константа, 
определенная с точностью до~$a,\,b$.
}

\medskip

\noindent
Д\,о\,к\,а\,з\,а\,т\,е\,л\,ь\,с\,т\,в\,о\,.\  
Рассмотрим характеристическую функцию случайной величины~$d_n(X_1)$. По теореме~1.2 
из~\cite{petrov} существуют положительные константы~$A_1$ и~$B_1$ такие, что
$$
|f_{d_n}(s)| \leq 1 - B_1 s^2
$$
для всех $|s| \leq A_1$. Пользуясь хорошо известной оценкой $1-z\leq e^{-z}$, справедливой для любого~$z$, 
получим, что для всех $|s| \leq A_1$
$$
|f_{d_n}(s)| \leq \exp\left\{-B_1s^2\right\}\,.
$$
Тогда из~(\ref{l17}) имеем
$$
\left|f_{D_n}(s)\right| = \left|f_{d_n}(tn^{-1/2}s)\right| \leq \exp\left\{-\fr{B_1 t^2 s^2}{n}\right\}
$$
для всех $|s| \leq A_1 \sqrt{n}/t$. Полагая $\epsilon_1 = B_1 t^2$ и $\sigma_1 = A_1/t$, 
приходим к первому утверждению леммы. Покажем второе утверждение.

Для любого $\theta>0$ непосредственным интегрированием получаем, что
\begin{multline*}
f_{d_{\theta}}(s) ={}\\
= \fr{\ercx{{b}/(2\sqrt{a})- i\sqrt{a}s} + 
\ercx{{b}/({2\sqrt{a}}) + i\sqrt{a}s}}{2\;{\sf erfc}\left (b/(2\sqrt{a})\right) \exp\left\{{is\iab \theta}/2 + as^2\right\}}.\hspace*{-7.37994pt}
\end{multline*}
Рассмотрим выражение, стоящее в числителе:
\begin{multline*}
\ercx{\fr{b}{2\sqrt{a}} - i\sqrt{a}s} + \ercx{\fr{b}{2\sqrt{a}} + i\sqrt{a}s} \equiv{}\\
{}
\equiv \fr{2}{\sqrt{\pi}}\int\limits_{{b}/(2\sqrt{a}) - i\sqrt{a}s}^{\infty} e^{-z^2}\,dz +{}\\
{}+
 \fr{2}{\sqrt{\pi}}\int\limits_{{b}/(2\sqrt{a}) + i\sqrt{a}s}^{\infty} e^{-z^2}\,dz ={}\\
= \fr{2}{\sqrt{\pi}}\int\limits_{{b}/(2\sqrt{a})}^{\infty}e^{-\left(w^2 - 2i\sqrt{a}sw - as^2\right)}\,dw+{}\\
{}+
 \fr{2}{\sqrt{\pi}}\int\limits_{{b}/(2\sqrt{a})}^{\infty}e^{-\left(w^2 + 2i\sqrt{a}sw - as^2\right)}\,dw ={}\\
{}
= \fr{4e^{as^2}}{\sqrt{\pi}} \int\limits_{{b}/(2\sqrt{a})}^{\infty}e^{-w^2} \cos 2\sqrt{a}\,sw\, dw\,.
\end{multline*}
Подставляя полученное выражение в формулу для $f_{d_{\theta}}(s)$, получим
\begin{equation}
\label{l22}
\left|f_{d_{\theta}}(s)\right| = J(s) \equiv \fr{2\left|I(s)\right|}{\sqrt{\pi}\;{\sf erfc}\left(
b/(2\sqrt{a})\right)}\,,
\end{equation}
где
$$
I(s) = \int\limits_{{b}/(2\sqrt{a})}^{\infty} e^{-w^2} \cos 2\sqrt{a}\,s w\, dw\,.
$$
Обозначим подынтегральную функцию интеграла~$I(s)$ как~$z(w,s)$. По признаку Вейерштрасса (теорема~7.8 
из~\cite{matan}) интеграл в выражении для~$I(s)$\linebreak
сходится равномерно на множестве всех действительных~$s$. 
Кроме того, для любого $w_0 \geq$\linebreak $\geq {b}/(2\sqrt{a})$ функция~$z(w,s)$ равномерно на множестве ${b}/(2\sqrt{a}) \leq 
w \leq w_0$ стремится к~$z(w,s_0)$ при $s\to s_0$, а значит по теореме~7.9 из~\cite{matan} функция~$I(s)$ 
непрерывна на множестве всех действительных~$s$. По признаку Вейерштрасса интеграл
$$
\il{{b}/(2\sqrt{a})}{\infty} we^{-w^2} \sin 2\sqrt{a}\,sw\,dw
$$
сходится равномерно на множестве всех действительных~$s$. Отсюда согласно теореме~7.14 
из~\cite{matan} следует возможность дифференцирования интеграла~$I(s)$ по параметру~$s$ в любой точке~$s$. 
Таким образом, для любого~$s$ взятием интеграла по частям имеем
\begin{multline}
I'(s) = -2\sqrt{a}\il{{b}/(2\sqrt{a})}{\infty} we^{-w^2} \sin 2\sqrt{a}\,sw\,dw ={}\\
{}= \sqrt{a} \exp\left\{-\fr{b^2}{4a}\right\} \sin bs + 2asI(s)\,.
\label{l19}
\end{multline}
Пусть $s \ne 0$. Тогда можно записать
\begin{align}
I(s) &= \fr{I'(s)}{2as} - \fr{\sin bs}{2\sqrt{a}\exp\left\{{b^2}/(4a)\right\} s}\,;\notag\\
|I(s)| &\leq \fr{|I'(s)|}{2a|s|} + \fr{1}{2\sqrt{a}\exp\left\{{b^2}/(4a)\right\}|s|}\,.
\label{l20}
\end{align}
Оценим $|I'(s)|$. По~(\ref{l19}) имеем
$$
\left|I'(s)\right| \leq 2\sqrt{a}\il{{b}/(2\sqrt{a})}{\infty} we^{-w^2}\,dw = 
\sqrt{a} \exp\left\{-\fr{b^2}{4a}\right\}\,.
$$
Подставляя в~(\ref{l20}), получим
$$
\left|I(s)\right| \leq \fr{1}{\sqrt{a} \exp\left\{{b^2}/(4a)\right\}} \, \fr{1}{|s|}\,.
$$
Таким образом, можно утверждать, что
\begin{equation}
\lim_{|s|\to\infty} I(s) = 0\,.
\label{l21}
\end{equation}
Вернемся к~(\ref{l19}). Очевидно, что вышеописанные свойства функции~$I(s)$ целиком переносятся на функцию~$J(s)$. 
Кроме того, $J(s) > 0$ для любого~$s$. Так как функция~$J(s)$ чeтная, последнюю часть доказательства 
проведeм только для положительных~$s$.

Положим
$$
q_d = \sup\limits_{s \geq A_1} J(s)\,,
$$
где $A_1$~--- положительная константа, определенная при доказательстве первого утверждения леммы. 
Обозначим $J_1 = J(A_1)$. Фиксируем некоторое $0< \epsilon < J_1$. По~(\ref{l21}) найдется такое $\delta>0$, 
что для всех $s>\delta$ можно утверждать, что $J(s)<\epsilon$. Тогда очевидно, что
$$
\sup\limits_{A_1 \leq s \leq \delta} J(s) = q_d\,.
$$
Из непрерывности~$J(s)$ следует, что эта точная верхняя грань
достигается, т.\,е.\ найдётся такое~$s_1$, что $A_1 \leq |s_1| \leq
\delta$ и $J(s_1) = q_d$. Тогда по следствию к лемме~3.7
$q_d < 1$. Тем самым доказано, что $J(s) \leq q_d < 1$ для всех $s
\geq A_1$.

Заметим, что в выражении (\ref{l22}) $|f_{d_{\theta}}(s)|$ можно заменить на~$|f_{d_n}(s)|$. 
При этом останется справедливой оценка
$$
|f_{d_n}(s)| \leq q_d < 1 \mbox{ для всех } |s| \geq A_1\,.
$$
Используя~(\ref{l17}), перейдём к функции~$f_{D_n}(s)$. Тогда можно
утверждать, что
$$
|f_{D_n}(s)| = |f_{d_n}(tn^{-1/2}s)| \leq q_d < 1
$$
для всех $|s| \geq \sigma_1 \sqrt{n}$  (где $\sigma_1 = A_1/t$). Лемма доказана.
\hfill $\Box$

\medskip

\noindent
\textbf{Лемма 3.9.}
\textit{Для модуля характеристической функции $f_{M_n}(s)$ справедлива следующая оценка:}
\begin{align*}
 |f_{M_n}(s)| &\leq \exp\left\{-{\epsilon_2 s^2}/n\right\}\,,  & |s| &\leq\sigma_2 \sqrt{n}\,; \\ 
|f_{M_n}(s)| &\leq q_h < 1\,,  & |s|&\geq\sigma_2 \sqrt{n}\,, 
\end{align*}
\textit{где $\sigma_2$, $\epsilon_2$~--- положительные константы; $q_h$~--- 
положительная константа, определенная с точностью до неизвестных~$a,b,t$.
}

\medskip

\noindent
Д\,о\,к\,а\,з\,а\,т\,е\,л\,ь\,с\,т\,в\,о\,.\  
Доказательство первого утверждения леммы аналогично
доказательству первого утверждения леммы~3.8. Заметим, что
в ходе доказательства вводятся некоторые положительные константы~$A_2$ и~$B_2$, 
которые идентичны по смыслу (но не по значению)
константам~$A_1$ и~$B_1$ из леммы~3.8. Перейдем к
доказательству второго утверждения.

Для любого $\theta > 0$ непосредственным интегрированием получаем
\begin{multline*}
f_{m_{\theta}}(s) = \fr{1}{2\;{\sf erfc}\left(b/(2\sqrt{a})\right)}\times{}\\
{}\times
\left[\left(f_1 + f_2\right)e^{-({s(s + i\theta)(b + a\theta)^2})/({a\theta^2})}\;+ \right.\\
\left.{}+\left(f_3 + f_4\right)e^{-as(s+ i\theta)}\right]\,,
\end{multline*}
где
\begin{align*}
f_1 &= \erx{\fr{b}{{2\sqrt{a}}}+ \sqrt{a}\theta - is\left(\fr{b}{\sqrt{a}\theta} + \sqrt{a}\right)}\,;
\\
f_2 &= \erx{-\fr{b}{{2\sqrt{a}}} + is\left(\fr{b}{\sqrt{a}\theta} + \sqrt{a}\right)}\,;\\
f_3 &= \ercx{\fr{b}{2\sqrt{a}} + \sqrt{a}\theta - is\sqrt{a}}\,;\\
f_4 &= \ercx{\fr{b}{2\sqrt{a}} + is\sqrt{a}}\,.
\end{align*}
Преобразуем сумму $f_1 + f_2$ следующим образом:
\begin{multline*}
f_1 + f_2 ={}\\
{}= \fr{2}{\sqrt{\pi}} \int\limits_0^{{b}/({2\sqrt{a}}) + 
\sqrt{a}\theta - is\left({b}/({\sqrt{a}\theta}) + \sqrt{a}\right)} e^{-z^2}\,dz +{}\\
{}+
\fr{2}{\sqrt{\pi}}\int\limits_0^{-b/(2\sqrt{a}) + is\left({b}/({\sqrt{a}\theta}) + 
\sqrt{a}\right)} e^{-z^2}\,dz ={}\\
{}
= \fr{2}{\sqrt{\pi}} \int\limits_{{b}/({2\sqrt{a}}) - is\left({b}/({\sqrt{a}\theta}) + 
\sqrt{a}\right)}^{{b}/({2\sqrt{a}}) + \sqrt{a}\theta - is\left({b}/({\sqrt{a}\theta}) + \sqrt{a}\right)} 
e^{-z^2}\,dz ={}\\
{}= \fr{2}{\sqrt{\pi}} \int\limits_{{b}/({2\sqrt{a}})}^{{b}/({2\sqrt{a}}) + 
\sqrt{a}\theta} e^{-\left(w - is\left({b}/({\sqrt{a}\theta}) + \sqrt{a}\right)\right)^2}\,dw ={}\\
{}
= \fr{2}{\sqrt{\pi}} e^{({s^2(a\theta + b)^2})/({a\theta^2})} \times{}\\
{}\times \int\limits_{{b}/({2\sqrt{a}})}^{{b}/({2\sqrt{a}}) + \sqrt{a}\theta}
e^{-w^2} e^{2is\left(\sqrt{a}+ {b}/({\sqrt{a}\theta})\right)w}\, dw\,.
\end{multline*}
Преобразуем сумму $f_3 + f_4$ следующим образом:

\noindent
\begin{multline*}
f_3 + f_4 = \fr{2}{\sqrt{\pi}}\int\limits_{{b}/({2\sqrt{a}}) + \sqrt{a}\theta - is\sqrt{a}}^{\infty}
e^{-z^2}\,dz +{}\\
{}+ \fr{2}{\sqrt{\pi}}\int\limits_{{b}/({2\sqrt{a}}) + is\sqrt{a}}^{\infty} e^{-z^2}\,dz ={}\\
{}=
 \fr{4e^{as^2}}{\sqrt{\pi}}\int\limits_{{b}/({2\sqrt{a}}) + \sqrt{a}\theta}^{\infty} 
 e^{-w^2} \cos 2s\sqrt{a} w\,dw +{}\\
 {}+
  \fr{2e^{as^2}}{\sqrt{\pi}}\int\limits_{{b}/({2\sqrt{a}})}^{{b}/({2\sqrt{a}}) + \sqrt{a}\theta} e^{-w^2} 
  \left(\cos 2s\sqrt{a} w -{}\right.\\
  {}\left. - i\sin 2s\sqrt{a} w\right)\,dw\,.
\end{multline*}
Подставляя эти выражения в формулу для~$f_{m_{\theta}}(s)$ и
пользуясь тем, что $\left|\int f(z)dz\right| \leq \int |f(z)|\,dz$,
$|a + b| \leq$\linebreak $\leq |a| + |b|$, получаем
\begin{multline*}
\!|f_{m_{\theta}}(s)| \leq \fr{2}{\sqrt{\pi}\;{\sf erfc}\left(b/(2\sqrt{a})\right)} 
\int\limits_{{b}/({2\sqrt{a}})}^{{b}/({2\sqrt{a}}) + 
\sqrt{a}\theta}\!\!\!\!\! e^{-w^2}\,dw +{}\\
{}+ \fr{2}{\sqrt{\pi}\;{\sf erfc}\left(b/(2\sqrt{a})\right)}\times{}\\
{}\times \left|\,\int\limits_{{b}/({2\sqrt{a}}) + \sqrt{a}\theta}^{\infty} e^{-w^2} \cos 2s\sqrt{a}\, 
w\,dw\right| \leq{}\\
{}
\leq J(s) \equiv \fr{2}{\sqrt{\pi}\;{\sf erfc}\left(b/(2\sqrt{a})\right)} \int\limits_{{b}/({2\sqrt{a}})}^{{b}/({2\sqrt{a}}) + 
\sqrt{a}t} \!\!\!\!\!\!e^{-w^2}\,dw +{}\\
{}+ \fr{2}{\sqrt{\pi}\;{\sf erfc}\left(b/(2\sqrt{a})\right)}
\left|\,\int\limits_{{b}/({2\sqrt{a}}) + \sqrt{a}t}^{\infty}\!\!\!\!\!\!\!\!\!\!\! e^{-w^2} \cos 2s\sqrt{a} w\,dw\right|.
\end{multline*}
Аналогично предыдущей лемме можно показать, что
$$
\lim\limits_{|s|\to\infty} J(s) = J_1 \equiv 
\fr{({2}/{\sqrt{\pi}})\il{{b}/({2\sqrt{a}})}{{b}/({2\sqrt{a}})+\sqrt{a}t}
\!\!\!\!\!e^{-w^2}\,dw}{{\sf erfc}\left(b/(2\sqrt{a})\right)}\,.
$$
При этом, так как~$t$ ограничено, то $J_1<1$. Очевидно также, что $J(s)>0$ для всех~$s$. 
Так как функция~$J(s)$ четная, то в дальнейшем будем рассматривать только область $s > 0$.

Положим
$$
q_h = \sup_{s \geq A_2} J(s)\,,
$$
где $A_2$~--- положительная константа, определенная при доказательстве первого утверждения леммы. 
Аналогично лемме~3.8 можно показать, что $q_h < 1$.

Заметим, что в оценке модуля характеристической функции очевидна возможность перехода 
от $f_{m_{\theta}}(s)$ к~$f_{m_n}(s)$ и
$$
|f_{m_n}(s)| \leq q_h < 1 \mbox { для всех } |s| \geq A_2\,.
$$
Используя~(\ref{l16}), перейдем к функции~$f_{M_n}(s)$. Тогда можно утверждать, что
\begin{multline*}
|f_{M_n}(s)| = |f_{m_n}(tn^{-1/2}s)| \leq q_h < 1 \\
\mbox{ для всех } |s| \geq \sigma_2\sqrt{n}\,,
\end{multline*}
где $\sigma_2 = A_2/t$. Лемма доказана. \hfill $\Box$

Перейдем к доказательству основных результатов.

\section{Проверка условий теоремы~3.2.1}

В этом разделе проверим условия теоремы~3.2.1 из~\cite{bening} и тем самым покажем 
справедливость установленной на эвристическом уровне формулы~(\ref{l5}).

\subsection{Условие~1}

Положим $\tau_n=n^{-1/4}$. Покажем, что условие~1 выполняется с
\begin{equation*}
\Phi_1(x) = \Phi\left(\fr{x}{t\sqrt{I_{a,b}}} + \fr{\sqrt{I_{a,b}}t}{2}\right)\,,\enskip  \Phi_2(x) = 0\,.
%\label{l28}
\end{equation*}
%$$
%\Phi_2(x) = 0\,.
%$$
При этом
$$
p(x) = \Phi_1'(x) = \fr{1}{t\sqrt{I_{a,b}}} \, \varphi\left(\fr{x}{t\sqrt{I_{a,b}}} + \fr{\sqrt{I_{a,b}}t}{2}\right)\,.
$$


\noindent
\textbf{Пункт (i).} Так как $\tau_n = n^{-1/4}$, выполнение условия~1 с 
такими~$\Phi_1(x)$ и~$\Phi_2(x)$ следует автоматически из леммы~3.1.

\medskip

\noindent
\textbf{Пункт (ii).} Выберем~$\beta$ произвольным образом, но так, чтобы $0 < \beta < 2$. 
Фиксируем произвольное $x_0 \in {\r}$. Заметим, что для любого множества $A \in {\cal B}({\r}^n)$ 
справедливо следующее представление:
\begin{equation}
{\p}_{n,\,1}(A) = {\e}_{n,\,0} \exp\{\Lambda_n\} {\III}(A)\,,
\label{l23}
\end{equation}
где ${\cal B}({\r}^n)$~--- борелевская $\sigma$-алгебра множеств в~${\r}^n$.

Обозначим через~$x_1$ такое $x \leq x_0$, на котором функция ${\p}_{n,\,0} \left(x \leq \Lambda_n \leq x + 
\tau_n^{2 + \beta}\right)$ достигает своего супремума. Очевидно, что
\begin{equation}
\sup\limits_{x\in{\sf X}} f(x)g(x) \leq \sup\limits_{x\in{\sf X}}f(x) \sup\limits_{x\in{\sf X}}g(x)\,,
\label{l24}
\end{equation}
где $f(x), g(x) \geq 0$~---- произвольные функции, а ${\sf X}$~--- некоторое множество. 
Тогда для любого $x \leq x_0$ в силу представления~(\ref{l23}), замечания~(\ref{l24}) и 
непрерывности распределения случайной величины~$\Lambda_n$ при любой гипотезе по лемме~3.1 имеем
\begin{multline*}
\tau_n^{-2}{\p}_{n,\,1}\left\{x \leq \Lambda_n \leq x + \tau_n^{2 + \beta}\right\} ={}\\
{}
= \tau_n^{-2}{\e}_{n,\,0} \exp\left\{\Lambda_n\right\} {\III}\left\{x \leq \Lambda_n \leq x + 
\tau_n^{2+\beta}\right\} \leq{}\\
{}
\leq\tau_n^{-2}\exp\left\{x + \tau_n^{2 + \beta}\right\}\times{}\\
{}\times {\e}_{n,\,0} {\III}\left\{x \leq \Lambda_n \leq x + 
\tau_n^{2+\beta}\right\} \leq{}\\
{}
\leq \tau_n^{-2}\exp\left\{x_0 + \tau_n^{2 + \beta}\right\}\times{}\\
{}\times {\p}_{n,\,0} \left\{x_1 \leq \Lambda_n \leq x_1 + 
\tau_n^{2 + \beta}\right\} ={}\\
{}
= \tau_n^{\beta}  \exp\left\{x_0 + \tau_n^{2 + \beta}\right\}\times{}\\
{}\times \fr{{\p}_{n,0}(\Lambda_n < x_1 + \tau_n^{2 + 
\beta}) - {\p}_{n,0}(\Lambda_n < x_1)}{\tau_n^{2 + \beta}} ={}\\
{}
= \tau_n^{\beta}  \exp\left\{x_0 + \tau_n^{2 + \beta}\right\} \times{}\\
{}
\times \left\{\left[{\Phi}\left(\fr{x_1 + \tau_n^{2 + \beta}}{t\sqrt{I_{a,b}}} +
 \fr{\sqrt{I_{a,b}}t}{2}\right) -{}\right.\right.\\
 {}-\left. {\Phi}\left(\fr{x_1}{t\sqrt{I_{a,b}}} + 
 \fr{\sqrt{I_{a,b}}t}{2}\right)\right]\tau_n^{-2 - \beta} + {}\\
\!{}
+ \fr{b^2C_{a,b}t}{3\sqrt{I_{a,b}}}\left(\fr{x_1}{tI_{a,b}} - \fr{t}{2}\right)\left[ 
\varphi\left(\fr{x_1 + \tau_n^{2 + \beta}}{t\sqrt{I_{a,b}}} + \fr{\sqrt{I_{a,b}}t}{2}\right) - {}\right.\\
\left.{}-\varphi\left(\fr{x_1}{t\sqrt{I_{a,b}}} + \fr{\sqrt{I_{a,b}}t}{2}\right)\right]\tau_n^{-\beta} +{}\\
\left.{}
+  o(1)
\vphantom{\fr{\sqrt{I_a}}{\sqrt{I_a}}}
\right\} = \tau_n^{\beta}  \exp\left\{x_0 + \tau_n^{2 + \beta}\right\} \times{}\\
{}\times
\left\{\fr{1}{t\sqrt{I_{a,b}}}\varphi\left(\fr{x_1}{t\sqrt{I_{a,b}}} + 
\fr{\sqrt{I_{a,b}}t}{2}\right) +  o(1)\right\} \to 0\,.
\end{multline*}
Итак, условие выполнено.

\subsection{Условие 2}

В силу непрерывности распределения случайной величины~$\Lambda_n$ при обеих гипотезах в качестве~$D_{n,i}$ 
можно выбрать~${\r}^n$. Тогда все случайные величины <<с волной>>  
тождественно равны соответствующим случайным величинам <<без волны>> и фигурирующие в записи~$S_n$ последовательности~$a_n$ 
и~$b_n$ равны тождественно~$t$ и $-\iab t^2/2$ соответственно.

Положим $\gamma_n = n^{-\nu}$,  $0<\nu<1/8$.

\medskip

\noindent\textbf{Пункт (i).} Очевиден при выбранных~$D_{n,i}$.

\medskip

\noindent\textbf{Пункт (ii).} Для индикатора ${\III}_{(\gamma_n,\infty)}(|\Delta_n|)$ справедлива следующая оценка:
\begin{equation}
{\III}_{(\gamma_n,\infty)}(|\Delta_n|) \leq \left(\fr{|\Delta_n|}{\gamma_n}\right)^m\,,
\label{l25}
\end{equation}
где $m$~--- произвольное неотрицательное число. Положим $m = 1$. Тогда, используя лемму~3.3, имеем
\begin{multline*}
\tau_n^{-2} \gamma_n^{-1}  {\e}_{n,0} \Delta_n^2 {\III}_{(\gamma_n,\infty)}(|\Delta_n|) \leq{}\\
{}\leq \tau_n^{-2} \gamma_n^{-2}  {\e}_{n,0} |\Delta_n|^3 \leq{}\\
{}
\leq C_1 n^{2\nu+1/2-3/4} = C_1 n^{2\nu-1/4} \to 0\\
 \mbox{ при } n \to \infty \mbox{ и } 0<\nu<\fr{1}{8}\,,
\end{multline*}
где $C_1$~--- положительная константа.

\medskip

\noindent\textbf{Пункт (iii).} Полагая в~(\ref{l25}) $m = 2$, по лемме~3.3 имеем
\begin{multline*}
\tau_n^{-2}  {\e}_{n,0} \exp\{\Lambda_n\} |\Delta_n| {\III}_{(\gamma_n,\infty)}(|\Delta_n|) \leq{}\\[2pt]
{}\leq \tau_n^{-2} \gamma_n^{-2}  {\e}_{n,0} \exp\{\Lambda_n\} |\Delta_n|^3 \leq{}\\[2pt]
{}
\leq \tau_n^{-2}  \gamma_n^{-2} \sqrt{{\e}_{n,0} \exp\{2\Lambda_n\}  {\e}_{n,0} |\Delta_n|^6} \leq{}\\[2pt]
{}\leq n^{2\nu+1/2} \sqrt{C_2n^{-3/2}} ={}\\[2pt]
{}
= \sqrt{C_2}\,  n^{2\nu+1/2}  n^{-3/4} = \sqrt{C_2} \, n^{2\nu-1/4} \to 0\\[2pt]
\mbox{ при } n \to \infty \mbox{ и } 0<\nu<\fr{1}{8}\,,
\end{multline*}
где $C_2$~--- положительная константа. Здесь использовано легко проверяемое соотношение
$$
{\e}_{n,0} \exp\left\{2\Lambda_n\right\} = {\cal O}(1)\,.
$$

\subsection {Условие~3}

Выберем последовательности $\Psi(n)$ и $\bar{\Psi}(n)$ та\-кими:
\begin{align}
\Psi(n) &= n^{\nu/4}\,; \label{l26}\\
\bar{\Psi}(n) &= n^{\beta+2}\,, \label{l27}
\end{align}
где $\nu$~--- константа, фигурирующая в показателе степени последовательности~$\gamma_n$ условия~2, 
а $\beta$~--- из условия~1. Такие последовательности будут удовлетворять всем требованиям, обозначенным 
в формулировке условия~3. По результатам работы~\cite{article} положим
\begin{align*}
\Pi &\sim \n \left(0,\, \fr{4b^2 \cab t^3}{3}\right)\,;\\
\Lambda &\sim \n \left(-\fr{t^2\iab}{2},\, \fr{t^2\iab}{2}\right)\,.
\end{align*}
В работе~\cite{article} также было показано, что случайные величины~$\Pi$ и~$\Lambda$ независимы.

\medskip

\noindent
\textbf{Пункт (i).} Преобразуем~$q_{n,l}$ следующим образом:
$$
q_{n,l} = \bar{q}_{n,l}(s) + R_l(s)\,, \enskip l = 0, 1\,,
$$
где

\noindent
\begin{align*}
\bar{q}_{n,l}(s) &\equiv -\fr{1}{l+1}\,{\e}_{n,0}\exp\left\{is\Lambda_n\right\}(\tau_n^{-1} \Delta_n)^{l+1}\,;
\\
R_l(s) &\equiv {\e}_{n,0}\exp\left\{is\Lambda_n\right\}\int\limits_{\tau_n^{-1} \Delta_n}^0z^l(e^{is\tau_n z}-1)\,dz\,.
\end{align*}
Для $R_l(s)$ справедлива следующая оценка:
$$
|R_l(s)| \leq |s|\tau_n {\e}_{n,0}\left |\tau_n^{-1}\Delta_n\right|^{l+2}\,.
$$
При $l=0$ по лемме~3.3 имеем
$$
|R_0(s)| \leq |s|\tau_n^{-1}{\e}_{n,0}|\Delta_n|^2 \leq C_3 |s| n^{-1/4}\,,
$$
где $C_3$~--- положительная константа. При $l=1$ по лемме~3.3 имеем
$$
|R_1(s)| \leq |s| \tau_n^{-2} {\e}_{n,0}|\Delta_n|^3 \leq C_4 |s| n^{-1/4}\,,
$$
где $C_4$~--- положительная константа.

Так как $\Psi^2(n) n^{-1/4} \to 0$, то
$$
\int\limits_{|s|\leq\Psi(n)}|R_l(s)|ds \to 0\,,\enskip l=0,1\,.
$$
Тогда для доказательства справедливости утверждений пункта~(i) достаточно будет установить, что
$$
\left|\int_{|s|\leq\Psi(n)}(\bar{q}_{n,l}(s) - q_l(s))ds\right| \to 0\,,  l=0,1\,.
$$
Для этого по теореме о мажорируемой сходимости достаточно показать поточечную 
сходимость~$\bar{q}_{n,l}(s)$ к~$q_l(s)$, а также существование интегрируемой 
на $|s|\leq \Psi(n)$ функции~$r_l(s)$ такой, что $\bar{q}_{n,l}(s)\leq r_l(s)$.

\medskip

\noindent
\textbf{Случай {\boldmath{$l=0$}}.} По лемме~3.4 можно записать
\begin{multline*}
{\e}_{n,0}\exp\{is\Lambda_n\}(\tau_n^{-1}\Delta_n) ={}\\
{}= -b n^{5/4} f^{n-1}_{M_n}(s) {\e}_0 \exp\left\{is M_n(X_1)\right\} g_n(X_1) ={}\\
{}
= -b n^{5/4} f^{n-1}_{M_n}(s) {\e}_0 (1+\delta_1(s,X_1)) g_n(X_1)\,,
\end{multline*}
где
$$
|\delta_1(s,X_1)| \leq |s| |M_n(X_1)|\,.
$$
По лемме~3.2 для $|s| \leq \Psi(n)$ и при достаточно больших~$n$
\begin{equation*}
\left|f_{M_n}^{n-1}(s)\right| \leq \exp\left\{-\fr{(n-1)\epsilon_2 s^2}{n}\right\} \leq
\exp\left\{-\frac{\epsilon_2 s^2}{2}\right\}\,.
\end{equation*}
Тогда по леммам~3.2 и~3.5
%\noindent
\begin{multline*}
\left|{\e}_{n,0}\exp\{is\Lambda_n\}(\tau_n^{-1}\Delta_n)\right| \leq{}\\
{}
\leq b n^{5/4}\exp\left\{-\fr{\epsilon_2 s^2}{2}\right\}  \left(
\left|{\e}_0g_n(X_1)\right| +{}\right.
\end{multline*}

\noindent
\begin{multline*}
\left.{}+ |s|{\e}_0 |M_n(X_1) g_n(X_1)|\right) \leq{}\\
{}
\leq b n^{5/4} \exp\left\{-\fr{\epsilon_2 s^2}{2}\right\} \left(C_5n^{-3/2} + C_6|s|n^{-5/4}\right) ={}\\
{}
= r_0(s) \equiv b \exp\left\{-\fr{\epsilon_2 s^2}{2}\right\} \left(C_5n^{-1/4} + C_6|s|\right)\,,
\end{multline*}
где $C_5$, $C_6$~--- положительные константы. Так как функция~$r_0(s)$ интегрируема на $|s|\leq\Psi(n)$, 
то требование ограниченности $\bar{q}_{n,0}(s)$ интегрируемой функцией выполнено. 
Покажем поточечную сходимость $\bar{q}_{n,0}(s)$ к~$q_0(s)$, т.\,е.
$$
{\e}_{n,0} \exp\left\{is\Lambda_n\right\} \left(\tau_n^{-1} \Delta_n\right) \to 
{\e} \exp\left\{is\Lambda\right\} \Pi\,.
$$
Имеем
$$
{\e} \exp\left\{is\Lambda\right\} \Pi = {\e} \exp\left\{is\Lambda\right\} {\e} \Pi = 0\,,
$$
поэтому достаточно показать, что для любого $s\in{\r}$
$$
\tau_n^{-1} {\e}_{n,0} \exp\left\{is\Lambda_n\right\} \Delta_n \to 0\,.
$$
По леммам~3.4 и~3.6 для любого $s\in{\r}$ имеем
\begin{multline*}
{\e}_{n,0}\exp\{is\Lambda_n\}(\tau_n^{-1}\Delta_n) ={}\\
{}= -b n^{5/4} f^{n-1}_{M_n}(s) {\e}_0 \exp\left\{isM_n(X_1)\right\} g_n(X_1) ={}\\
{}
= -b n^{5/4} \left(1+\fr{is(is-1)\iab t^2}{2n}+ o(n^{-1})\right)^{n-1} \times{}\\
{}\times
\left(\fr{b \cab (1+is)t^3}{3n^{3/2}}+  o(n^{-3/2})\right) \to 0\,.
\end{multline*}

\medskip

\noindent
\textbf{Случай {\boldmath{$l=1$}}.} По лемме~3.4 запишем
\begin{multline*}
{\e}_{n,0}\exp\{is\Lambda_n\}(\tau_n^{-1}\Delta_n)^2 = b^2 n^{3/2}\:f_{M_n}^{n-2}(s) \times{}\\
{}
\times \left(f_{M_n}(s) {\e}_0 \exp\left\{isM_n(X_1)\right\} g_n^2(X_1) + {}\right.{}\\
\left.{}+
(n-1) \left[{\e}_0 (1+\delta_2(s,X_1)) g_n(X_1)\right]^2\right)\,,
\end{multline*}
где
$$
|\delta_2(s,X_1)| \leq |s| |M_n(X_1)|\,.
$$
По лемме~3.9 для $|s| \leq \Psi(n)$ и при достаточно больших~$n$
$$
\left|f_{M_n}^{n-2}(s)\right| \leq \exp\left\{
-\fr{(n-2)\epsilon_2 s^2}{n}\right\} 
\leq \exp\left\{-\frac{\epsilon_2 s^2}{3}\right\}\,.
$$
Тогда по леммам~3.2 и~3.5
\begin{multline*}
\left|{\e}_{n,0}\exp\{is\Lambda_n\}(\tau_n^{-1}\Delta_n)^2\right| \leq{}\\
{}\leq b^2 n^{3/2} 
\exp\left\{-\fr{\epsilon_2 s^2}{3}\right\} \times{}\hspace*{10mm}
\end{multline*}
\begin{multline*}
{}\times \left({\e}_0 g_n^2(X_1) + n\left|{\e}_0g_n(X_1)\right|^2 +{}\right.\\[6pt]
{}
+ 2 n |s|  |{\e}_0 g_n(X_1)|  {\e}_0 |M_n(X_1) g_n(X_1)| +{}\\[6pt]
\left.{}+ns^2 \left({\e}_0 |M_n(X_1) g_n(X_1)|\right)^2\right) \leq
{}\\
{}
\leq b^2 n^{3/2}\exp\left\{-\fr{\epsilon_2 s^2}{3}\right\} \times{}\\[6pt]
{}\times
\left(C_7n^{-3/2} + C_8n^{-2} + C_9|s|n^{-7/4} + C_{10}s^2n^{-3/2}\right) ={}
\\
= r_1(s) \equiv b^2 \exp\left\{-\fr{\epsilon_2 s^2}{3}\right\} \times{}\\[6pt]
{}\times
\left(C_7 + C_8n^{-1/2} + C_9|s|n^{-1/4} + C_{10}s^2\right)\,,
\end{multline*}
где $C_7$, $C_8$, $C_9$, $C_{10}$~--- положительные константы. Так как~$r_1(s)$ 
интегрируема на $|s| \leq \Psi(n)$, то требование
ограниченности $\bar{q}_{n,1}(s)$ интегрируемой функцией выполнено.
Покажем поточечную сходимость $\bar{q}_{n,1}(s)$ к $q_1(s)$, т.\,е.
$$
{\e}_{n,0} \exp\left\{is\Lambda_n\right\} (\tau_n^{-1} \Delta_n)^2 \to {\e} \exp\left\{is\Lambda\right\} \Pi^2\,.
$$
Имеем
\begin{multline*}
{\e} \exp\left\{is\Lambda\right\} \Pi^2 = {\e} \exp\left\{is\Lambda\right\} {\e} \Pi^2 ={}\\[6pt]
{}= \fr{4b^2\cab t^3}{3} \exp\left\{-\fr{s^2t^2\iab}{2}-\fr{ist^2\iab}{2}\right\}\,.
\end{multline*}
По леммам~3.4 и~3.6 для любого $s\in{\r}$ получаем
\begin{multline*}
{\e}_{n,0}\exp\{is\Lambda_n\}(\tau_n^{-1}\Delta_n)^2 = b^2 n^{3/2}  f_{M_n}^{n-2}(s) \times{}\\[6pt]
{}\times
 \left(
 \vphantom{\left[X_1\right]^2}
 f_{M_n}(s) {\e}_0 \exp\left\{isM_n(X_1)\right\} g_n^2(X_1) +{}\right.\\[6pt]
\left. {}+ (n-1) \left[
 {\e}_0 \exp\left\{isM_n(X_1)\right\} g_n(X_1)\right]^2\right) ={}\\[6pt]
{}= b^2 n^{3/2} \left(1+\fr{is(is-1)\iab t^2}{2n}+  o(n^{-1})\right)^{n-2} \times{}\\[6pt]
{}
\times \left[
\vphantom{\left(\fr{b^2\cab^2 (1+is)^2t^6}{9n^3} +  o(n^{-3})\right)}
\left(1+\fr{is(is-1)\iab t^2}{2n}+  o(n^{-1})\right)\right.\times{}\\[6pt]
{}\times\left.
\left(\fr{4\cab t^3}{3n^{3/2}}+  o (n^{-3/2})\right) +\right.{}\\[6pt]
{}
\left.+ (n-1)\left(\fr{b^2\cab^2 (1+is)^2t^6}{9n^3} +  o(n^{-3})\right)\right] \to{}\\[6pt]
{}
\to \exp\left\{-\fr{s^2t^2\iab}{2}-\fr{ist^2\iab}{2}\right\} \times{}
\end{multline*}
\begin{multline*}
{}\times
\left[1 \times \fr{4b^2\cab t^3}{3} + 0\right] ={}\\[6pt]
{}
= \fr{4b^2\cab t^3}{3} \exp\left\{-\fr{s^2t^2\iab}{2}-\fr{ist^2\iab}{2}\right\}\,.
\end{multline*}

\medskip

\noindent\textbf{Пункт (ii).} Используя~(\ref{l26}) и~(\ref{l27}), получим
$$
\tilde{\Psi}(n) = \log \bar{\Psi}(n) \Psi^{-1}(n) = \left(\beta+2-\fr{\nu}{4}\right) \log n\,.
$$

\smallskip

\noindent\textbf{Третий предел.} В соответствии с~(\ref{l14}) можно записать
$$
f_{S_n}(s) = (f_{D_n}(s))^n\,.
$$
Очевидно, что $\tau_n^{-1} > \tilde{\Psi}(n)$ при больших~$n$. Тогда по лемме~3.8
\begin{multline*}
\!\!\tau_n^{-1} \max(\tilde{\Psi}(n),\tau_n^{-1})\! \sup_{\Psi(n)\leq|s|\leq\bar{\Psi}(n)}\!\!|{\e}_{n,0}\exp\{isS_n\}| ={}\\[6pt]
{}
= \sqrt{n}  \max \left\{\left[
\sup_{n^{\nu/4}\leq|s|\leq\sigma_1\sqrt{n}} |f_{D_n}(s)|\right]^n,\,\right.\\[6pt]
\left. 
\left[\sup_{\sigma_1\sqrt{n}\leq|s|\leq n^{\beta+2}} |f_{D_n}(s)|\right]^n\right\} ={}\\[6pt]
\!\!{}
= \sqrt{n}  \max\left\{\exp\left\{-\epsilon_1 n^{\nu/2}\right\},\, q_d^n\right\} \to 0 \mbox{ при } n \to \infty.
\end{multline*}

\smallskip

\noindent\textbf{Первый предел.} В~соответствии с леммой~3.4 и неравенством Коши--Буняковского имеет место следующая оценка:
\begin{multline*}
\!\!\!\left|{\e}_{n,0} \exp\{isS_n\} \Delta_n\right| \leq bn\left|f_{D_n}(s)\right|^{n-1} {\e}_0\left|g_n(X_1)\right| \leq{}\\[6pt]
{}\leq C_{11} \sqrt[4]{n} \left|f_{D_n}(s)\right|^{n-1}\,,
\end{multline*}
где $C_{11}$~--- положительная константа. По лемме~3.8 для всех $|s|\leq \sigma_2\sqrt{n}$ при $n\geq2$
$$
|f_{D_n}(s)|^{n-1} \leq \exp\left\{\!-\fr{(n-1)\epsilon_1 s^2}{n}\!\right\} \leq 
\exp\left\{-\fr{\epsilon_1 s^2}{2}\right\}.
$$
Тогда аналогично случаю третьего предела
\begin{multline*}
\tau_n^{-2} \tilde{\Psi}(n) \sup_{\Psi(n)\leq|s|\leq\bar{\Psi}(n)}|{\e}_{n,0}\exp\{isS_n\}\Delta_n| \leq{}\\
{}
\leq C_{12} n^{3/4} \log n \max\left\{\exp\left\{-\fr{\epsilon_1 n^{\nu/2}}{2}\right\},\, q_d^{n-1}\right\} \to{}\\
{}\to  0 
\mbox{ при } n \to \infty\,,
\end{multline*}
где $C_{12}$~--- некоторая положительная константа.

\smallskip

\noindent\textbf{Второй предел.} В соответствии с~(\ref{l13}) можно записать
$$
f_{\Lambda_n}(s) = (f_{M_n}(s))^n\,.
$$
Тогда по лемме~3.9, повторяя схему доказательства для первого предела, имеем
\begin{multline*}
\tau_n^{-1} \max\left(\tilde{\Psi}(n),(\Psi^{-1}(n)-\bar{\Psi}^{-1}(n))\tau_n^{-1}\right) \times{}\\
{}\times
\sup_{\Psi(n)\leq|s|\leq\bar{\Psi}(n)}|{\e}_{n,0}\exp\{is\Lambda_n\}| ={}\\
{}
= \sqrt{n}\left(\fr{1}{n^{\nu/4}}-\fr{1}{n^{\beta+2}}\right) \times{}\\
{}\times
\max\left\{\exp\left\{-\epsilon_2 n^{\nu/2}\right\},\, q_h^n\right\} \to 0 \mbox{ при } n \to \infty\,.
\end{multline*}
Доказательство равенства этого предела нулю завершает проверку условий теоремы.

\section{Заключение}

Теперь можно применить теорему~3.2.1 из работы~\cite{bening}, а значит,
$$
\lim\limits_{n \to \infty} \sqrt{n} \left(\beta_n^{*}-\beta_n\right) = \fr{1}{2}\, 
e^{\tilde{u}_{\alpha}} {\sf D}\left(\Pi | \Lambda = \tilde{u}_{\alpha}\right) p\left(\tilde{u}_{\alpha}\right)\,,
$$
где $\tilde{u}_{\alpha} = \Phi_1^{-1}(1-\alpha)$. Из условия~1 (см.\ предыдущий раздел)
$$
\tilde{u}_{\alpha} = t\sqrt{\iab}u_{\alpha} - \fr{\iab t^2}{2}\,,
$$
где $u_{\alpha}=\Phi^{-1}(1-\alpha)$. Тогда
$$
p(\tilde{u}_{\alpha}) = \Phi_1'(x)\biggr|_{x=\tilde{u}_{\alpha}} = \fr{1}{\sqrt{\iab}t} \, \varphi({u_{\alpha}})\,.
$$
Так как случайные величины~$\Pi$ и~$\Lambda$ независимы, то 
${\sf D}(\Pi|\Lambda=\tilde{u}_{\alpha}) = 4b^2\cab t^3/3$. 
После очевидных преобразований окончательно получаем, что
$$
\fr{1}{2} \, e^{\tilde{u}_{\alpha}} {\sf D}\left(\Pi | \Lambda \!=\! 
\tilde{u}_{\alpha}\right) p(\tilde{u}_{\alpha}) = \fr{2b^2\cab t^2}{3\sqrt{\iab}} \, \varphi(u_{\alpha}-t\sqrt{\iab})\,.
$$
Таким образом, получено формальное доказательство формулы~(\ref{l5}).


{\small\frenchspacing
{%\baselineskip=10.8pt
\addcontentsline{toc}{section}{Литература}
\begin{thebibliography}{9}

\bibitem{article} %1
\Au{Бенинг В.\,Е., Лямин О.\,О.} 
О мощности критериев в случае обобщенного распределения Лапласа~// Информатика и её применения, 2009. Т.~3. Вып.~3. 
С.~79--85.

\bibitem{kotz} %2
\Au{Kotz S., Kozubowski~T.\,J., Podgorski~K.} 
The Laplace distribution and generalizations: A revisit with applications to communications, economics, engineering and 
finance.~--- Boston: Birkhauser, 2001.

\bibitem{korolev}  %3
\Au{Бенинг В.\,Е., Королёв~В.\,Ю.} 
Некоторые статистические задачи, связанные с распределением Лапласа~// Информатика и её применения, 2008. Т.~2. Вып.~2. 
С.~19--34.

\bibitem{korolev_ra}  %4
\Au{Бенинг В.\,Е., Королёв~Р.\,А.} О предельном поведении мощностей критериев в случае распределения Лапласа~// 
Информатика и её применения, 2010. Т.~4. Вып.~2. С.~66--77.

\bibitem{bening} %5
\Au{Bening V.\,E.} Asymptotic theory of testing statistical hypotheses.~--- Utrecht: VSP, 2000.

\bibitem{chibisov} %6
\Au{Chibisov D.\,M., van Zwet~W.\,R.} On the Edgeworth expansion for the logarithm of the likelihood ratio. I~// 
Теория вероятностей и ее применения, 1984. Т.~29. №\,3. C.~417--439.



\bibitem{petrov} %7
\Au{Petrov V.\,V.} Sums of independent random variables.~--- Berlin: Springer-Verlag, 1975.

\label{end\stat}

\bibitem{matan} %8
\Au{Ильин В.\,А., Садовничий В.\,А., Сендов~Бл.\,X.} Математический анализ: Продолжение курса.~--- М.: МГУ,\linebreak 1987.
 \end{thebibliography}
 }}\end{multicols}