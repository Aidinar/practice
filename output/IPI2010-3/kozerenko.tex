\def\stat{kozerenko}

\def\tit{КОГНИТИВНО-ЛИНГВИСТИЧЕСКИЕ ПРЕДСТАВЛЕНИЯ 
В~СИСТЕМАХ ОБРАБОТКИ ТЕКСТОВ}

\def\titkol{Когнитивно-лингвистические представления 
в~системах обработки текстов}

\def\autkol{Е.\,Б.~Козеренко, И.\,П.~Кузнецов}
\def\aut{Е.\,Б.~Козеренко$^1$, И.\,П.~Кузнецов$^2$}

\titel{\tit}{\aut}{\autkol}{\titkol}

%{\renewcommand{\thefootnote}{\fnsymbol{footnote}}\footnotetext[1]
%{Работа выполнена при поддержке Российского фонда фундаментальных
%исследований, проект~10-01-00480. Статья написана на основе материалов доклада, 
%представленного на IV Международном семинаре <<Прикладные задачи теории вероятностей 
%и математической статистики, связанные с моделированием информационных систем>> 
%(зимняя сессия, Аоста, Италия, январь--февраль 2010 г.).}}

\renewcommand{\thefootnote}{\arabic{footnote}}
\footnotetext[1]{Институт проблем информатики Российской академии наук, kozerenko@mail.ru}
\footnotetext[2]{Институт проблем информатики Российской академии наук, igor-kuz@mtu-net.ru}


\Abst{Рассмотрены вопросы проектирования и развития 
семантико-синтаксических и лексико-семантических представлений в 
лингвистических процессорах ряда систем, основанных на аппарате расширенных 
семантических сетей (РСС). Системы этого класса создаются для извлечения знаний из 
текстов на естественных языках, отображения извлеченных сущностей и связей в 
структуры базы знаний (БЗ) и использования знаний для поддержки экспертных 
аналитических решений в различных сферах приложения. В~фокусе внимания 
находятся ин\-же\-нер\-но-линг\-ви\-сти\-че\-ские представления, позволяющие 
построить целостную работающую лингвистическую модель, которая 
модифицируется в зависимости от конкретной задачи: от <<тяжелой>> формы на 
основе детальных глубинных представлений до фокусных редуцированных 
оболочек, настроенных на узкую предметную область (ПО) и ограниченный язык 
общения. Особое внимание уделяется способам описания 
дис\-три\-бу\-тив\-но-транс\-фор\-ма\-ци\-он\-ных признаков языковых объектов.}

\KW{интеллектуальные системы; семантические представления; лингвистические 
процессоры; обработка естественного языка; извлечение знаний}

       \vskip 14pt plus 9pt minus 6pt

      \thispagestyle{headings}

      \begin{multicols}{2}

      \label{st\stat}

\section{Введение}

     Данная работа посвящена проблемам создания\linebreak 
     когни\-тив\-но-линг\-ви\-сти\-че\-ских моделей естественного языка для 
различных классов информационных систем и описанию опыта создания 
линг\-ви\-сти\-че\-ских представлений для интеллектуальных\linebreak технологий 
обработки текстов. Вопросы извлечения знаний из текстов и создания модели 
естественного языка рассматриваются в единстве. В центре внимания будут 
находиться лингвистические процессоры интеллектуальных систем, 
разработанных на основе аппарата \textit{расширенных семантических 
сетей}~[1--5]. %\cite{1koz}--\cite{3koz}, \cite{18koz}--\cite{19koz}. 
Будем 
называть их \textit{РСС-сис\-те\-мы}. Эти системы создавались коллективом 
разработчиков, включая авторов данной статьи в Институте проб\-лем 
информатики РАН на протяжении целого ряда лет в рамках 
исследовательских проектов и прикладных систем, ориентированных на 
конкретные ПО заказчиков. Можно выделить четыре 
поколения РСС-систем. Ко\-гни\-тив\-но-линг\-ви\-сти\-че\-ские 
представления, заложенные в основу систем этого класса, прошли 
определенный эволюционный путь. 
     
     Интеллектуальные РСС-сис\-те\-мы содержат развитые \textit{базы 
знаний}, при этом знания представлены в виде записей на языке 
РСС, называемых 
     \textit{РСС-струк\-ту\-ра\-ми}. Лингвистические знания, таким 
образом, являются частным случаем <<знаний>> и также представлены в 
виде записей на языке РСС. Основным 
конструктивным элементом РСС\linebreak является именованный $N$-мест\-ный 
предикат, на\-зы\-ва\-емый <<\textit{фрагментом}>>. Все множество языковых 
объектов задается в виде системы пре\-ди\-кат\-но-ак\-тант\-ных структур, при этом 
поддерживаются механизмы представления вложенных структур, что дает 
очень мощные изобразительные возможности для описания объектов 
различных языковых уровней. Очень важными факторами являются 
однородность и единообразие лингвистических представлений. 
     
     В процессе анализа и синтеза предложений естественного языка 
используется фор\-маль\-но-грам\-ма\-ти\-че\-ский аппарат, сходный с 
грамматиками зависимостей. При этом подходе опорными элементами 
служат слова и конструкции, выполняющие роль предикатов в предложении, 
и результатом анализа предложения должен стать один предикат, 
соответствующий сказуемому рассматриваемого предложения (т.\,е.\ 
основному глаголу в личной форме или другому основному предикатному 
выражению). Таким образом, в процессе анализа происходит выявление 
\textit{когнитивных опор} предложения: <<слов-дейст\-вий>> и 
     <<слов-от\-но\-ше\-ний>>, т.\,е.\ глаголов и других слов, имеющих 
синтактико-семантические валентности. Примером <<слов-от\-но\-ше\-ний>> 
могут служить, например, слова <<отец>>, <<друг>> и~т.\,п., т.\,е.\ в данном 
случае <<отношения>> (или \textit{функции}~--- в терминах языка логики 
предикатов 1-го порядка)~--- это слова, которые задают сильные, четко 
выраженные син\-так\-ти\-ко-се\-ман\-ти\-че\-ские ожидания. 
     
     Семантический анализ в ин\-же\-нер\-но-линг\-ви\-сти\-че\-ском 
понимании~--- это процесс перевода ес\-тест\-вен\-но-язы\-ко\-вых 
выражений во <<внутренние>> структуры БЗ, в 
рассматриваемой ситуации этими <<внутренними>> структурами являются 
записи на языке РСС. Таким образом, структуры БЗ~--- это код смысла в 
интеллектуальных информационных системах подобного рода. 
     
     В работе рассматриваются ин\-же\-нер\-но-линг\-ви\-сти\-че\-ские 
решения в системах с <<пол\-ным>> линг\-ви\-сти\-че\-ским анализом~--- это 
     сис\-те\-мы 1-го и 2-го поколения: ДИЕС1, ДИЕС2, 
     Логос-Д~\cite{2koz, 3koz}~--- и сис\-те\-мах с <<фактографическим>> 
подходом: интеллектуальных системах поддержки аналитических решений 
(ИСПАР)~\cite{18koz, 19koz}, где целью анализа является выделение 
сущностей и связей из текстов,~--- это системы 3-го и 4-го поколения. 

\section{Процесс концептуально-лингвистического моделирования 
в системах, основанных на аппарате расширенных семантических сетей}
     
\subsection{Центральные вопросы семантического моделирования} %2.1
     
     Концептуально-лингвистическое моделирование (КЛМ)~--- это 
процесс построения ес\-тест\-вен\-но-язы\-ко\-вой модели ПО (рис.~1), синтезирующий в себе подходы 
концептуального и лингвистического моделирования~[1--3]. 
По\-стро\-ение концептуально-лингвистической модели некоторой 
ПО подразделяется на следующие этапы:
     \begin{itemize}
     \item построение собственно концептуальной модели, т.\,е.\ вычленение 
базовых понятий, организация их в ро\-до-ви\-до\-вые деревья и определение 
связей между ними;
     \item разработка идеографического словаря ПО, т.\,е.\ 
лексическое наполнение концептуальной модели;
     \item ввод базовых правил, описывающих на естественном языке 
<<модель мира>>, релевантную данной ПО.
     \end{itemize}
     
     
     Методика КЛМ на 
основе аппарата РСС базируется на следующих принципах:
     \begin{itemize}
\item модель должна быть <<открытой>>, т.\,е.\ поддерживать эффективный 
механизм расширения и обновления информации;
\begin{center} %fig1
%\vspace*{3pt}
\hspace*{-10.7158pt}\mbox{%
\epsfxsize=77.871mm
\epsfbox{koz-1.eps}
}\hspace{10.7158pt}
%\end{center}
\vspace*{4pt}
%\begin{center}
{{\figurename~1}\ \ \small{Процесс КЛМ}}
\end{center}
\vspace*{3pt}

%\bigskip
\addtocounter{figure}{1}
\item модель представления <<смысла>> должна учитывать факты 
экстралингвистической реаль\-ности, которые в виде правил и отношений 
составляют некоторую базовую <<модель мира>>, достраиваемую 
конкретными моделями ПО;
\item модель должна быть практичной, т.\,е.\ не перегруженной детальными 
описаниями связей и отношений между понятиями, чтобы обеспечить 
возможность ее реализации, но в то же время отражать всю релевантную 
конкретной задаче информацию.
\end{itemize}

     \begin{figure*} %fig2
%     \begin{center}
\hspace*{23mm}\{(ВЫРАБАТЫВА895\_\_)(DICSEM)\\
\hspace*{23mm}COORD(PROGNOZ1,RUS,ВЫРАБАТЫВА895\_\_,S50\_31\_51\_20,\%)\\
\hspace*{23mm}SUB(UNIV,0+)~SUB(UNIV,1+)~SUB(UNIV,2+)\\
\hspace*{23mm}ВЫРАБАТЫВ(0-,1-,2-/3+)~INFI(3-)~ПРИДЕТСЯ(3-)~ПРИДЕТСЯ(3$-$/4+) \\
\hspace*{23mm}FUT1(4$-$)~SUB(СРЕД,5+)
%\end{center}
%\vspace*{2pt}
\Caption{Пример записи представления глагола <<вырабатывать>> в семантическом 
словаре
\label{f2koz}}
%\vspace*{6pt}
\end{figure*}

     Реалистичный подход к постановке задачи диктует необходимость 
ограничения моделируемого подмножества естественного языка. Суть 
ограничений сводится к следующему:
     \begin{enumerate}[(1)]
     \item анализируемые текстовые материалы содержат 
экспертные знания из конкретных ПО (в разработанных 
авторами системах это были такие ПО, как диагностика 
брака при изготовлении микросхем, социальное прогнозирование, 
криминалистика и другие);
     \item в целях максимально возможного устранения 
неоднозначности словарь строится по модульному принципу: есть некоторая 
наиболее общая часть (1--2~уровня), которая достраивается специальными 
словарями для каж\-дой отдельной~ПО.
     \end{enumerate}
     
     Предлагаемая модель лексической семантики основана на принципе 
<<ядерного>> значения, реализуемого в контексте данной 
ПО, с последующим индуктивным наращиванием других значений (если 
они актуализируются в рас\-смат\-ри\-ва\-емых контекстах). Также используется 
таксономия, которая реализуется в виде иерархических деревьев классов 
слов. 
     
     Общая <<модель мира>> системы является основой для моделей ПО. 
Элементами этой модели служат классы слов, которые подразделяются на 
понятия/имена, отношения, действия, свойства, характеристики действий, 
временные и пространственные характеристики.
     
     Самым общим понятием является \textit{концепт}, или 
\textit{универсальный класс}, который подразделяется на объект, ситуацию, 
процесс и~др. 
     
     Слова, относящиеся к классам действий и отношений, представлены 
как се\-ман\-ти\-ко-син\-так\-си\-че\-ские фреймы, задающие 
     пре\-ди\-кат\-но-ак\-тант\-ные структуры (модель управления). Однако 
в описываемом подходе (назовем его РСС-под\-хо\-дом) существенно 
расширена область значений актантов. Суть расширения состоит, во-первых, 
в том, что в роли актантов могут выступать не только простые объекты, 
соответствующие отдельным словам, но и структурные объекты, 
представляющие словосочетания и фразы, а во-вторых, в том, что понятие 
падежа включает в себя не только семантические, но и синтаксические 
признаки.
     
     Подход, основанный на РСС, позволяет отражать произвольный 
уровень вложенности структур за счет пропозициональных вершин 
семантической сети. Это обеспечивает представление\linebreak сложных 
синтаксических конструкций фраз\linebreak естественного языка, а также позволяет 
отразить\linebreak структурный характер лексической семантики,\linebreak которая в 
предлагаемой модели имеет иерар\-хи\-че\-ски-се\-те\-вую структуру. 
Линг\-ви\-сти\-че\-ские зна-\linebreak ния пред\-став\-ле\-ны в системном словаре и 
декла\-ра\-тивных модулях линг\-ви\-сти\-че\-ско\-го процессора.\linebreak В РСС-сис\-те\-мах 
так\-же реализована функция динамически форми\-ру\-емо\-го семантического 
словаря, который на основе исходной лингвистической информации 
достраивается системой автоматически в процессе об\-ра\-бот\-ки конкретных 
текстов. На рис.~\ref{f2koz} пред\-став\-ле\-но \mbox{такое} <<внутреннее>> описание 
глагола в семантическом словаре. Этот словарь автоматически генерируется 
РСС-системами ДИЕС2, ЛОГОС-Д, ИКС в процессе обработки 
     естест\-вен\-но-язы\-ко\-вых \mbox{текстов}. 
     {\looseness=1
     
     }
     
     
\subsection{Особенности применения аппарата расширенных семантических сетей 
в~когнитивно-лингвистическом моделировании} %2.2
     
     Дадим краткое описание аппарата РСС и  
обос\-ну\-ем выбор именно этого метода представления для моделирования 
естественного языка. Классическое понятие семантической сети сводится к 
следующему: задаются некоторые вершины, соответствующие объектам,  
вершины связываются дугами, которые помечаются именами отношений. 
Однако с помощью подобных сетей оказывается трудно представлять 
сложные виды информации, например, когда объекты, связанные 
отношениями, образуют агрегаты и когда отношения связываются между 
собой отношениями и~др. Поэтому в сети вводятся вершины, 
соответствующие именам отношений, а также специальный композиционный 
элемент, называемый вершиной связи. Вершина связи как бы <<разрывает>> 
дугу и подсоединяется одним ребром к вершине-отношению, а другими 
ребрами~--- к вершинам-объектам. Расширенная семантическая сеть является развитием такого сорта 
сетей в направлении повышения изобразительных возможностей при 
сохранении свойства однородности.
     
     Основой РСС является множество вершин ($V$), из которых 
составляются элементарные фрагменты (ЭФ) вида
     $
     V_0(V_1,V_2,\ldots ,V_k/V_{k+1})
     $, 
     где
$V_0, V_1, V_2,\ldots , V_k, V_{k+1}>0$.
     
     
     Такой фрагмент представляет $k$-местное отношение. Позиции 
вершин в ЭФ определяют их роли. 
Вершина~$V_0$ ставится в соответствие имени отношения, 
вершины~$V_1$, $V_2$, \ldots , $V_k$~--- объектам, участ\-ву\-ющим в 
отношении, а вершина~$V_{k+1}$, отделенная косой линией,~--- всей 
совокупности упомянутых объектов с учетом их отношения. В~дальнейшем 
будем $V_{k+1}$ называть $C$-вершиной ЭФ.\linebreak 
Множество ЭФ образует РСС. 
С~помощью РСС представляются наборы отношений, различные ситуации, 
сце\-нарии. Сильной стороной РСС-под\-хо\-да является возможность 
однородного пред\-став\-ле\-ния как предметной (концептуальной), так и 
лингвистической информации, что обеспечивает эффективную обработку 
знаний и поддержание непротиворечи\-вости~БЗ.
          \begin{figure*} %fig3
     \vspace*{1pt}
\begin{center}
\mbox{%
\epsfxsize=125.039mm
\epsfbox{koz-3.eps}
}
\end{center}
\vspace*{-9pt}
     \Caption{Семантико-синтаксический анализ без выявления глагольных 
словоформ
      \label{f3koz}}
\vspace*{12pt}
 %     \end{figure*}
%            \begin{figure*} %fig4
           \vspace*{1pt}
\begin{center}
\mbox{%
\epsfxsize=103.129mm
\epsfbox{koz-4.eps}
}
\end{center}
\vspace*{-9pt}
      \Caption{Целостная семантическая структура предложения
      \label{f4koz}}
      \end{figure*}

     
     Посредством РСС в БЗ представлены лингвистические  и 
предметные знания. Обработка этих знаний осуществляется 
продукциями языка ДЕКЛ, на котором реализованы сле\-ду\-ющие шесть 
блоков: морфологического анализа, семанти\-ческого анализа слов, 
син\-так\-ти\-ко-се\-ман\-ти\-че\-ско\-го анализа форм, 
прагматических функций, организации системной активности и 
обратный лингвистический процессор. С~помощью продукций 
осущест\-вля\-ет\-ся последовательное преобразование сети~--- РСС. При этом 
проходятся фазы, соответствующие уровню понимания входного текста. 
Рас\-смот\-рим~их.
     \begin{enumerate}[1.]
     \item На первом шаге анализа строится 
пространственная структура предложения с морфологической информацией 
для каждого слова.\linebreak Каж\-дый член предложения представляется вершиной 
семантической сети. Вместо слова генерируется код (если слово 
многозначно, т.\,е.\ принадлежит к нескольким классам,~--- то более одного 
кода). Основой кода служит корень слова. На этом этапе предложение 
представляется в виде набора фрагментов типа LRR (специальных меток 
результатов 1-го этапа анализа), объединяемых в целостную структуру 
посредством вершины связи. Результат 1-го этапа постоянно обращается к 
словарю: <<Что значит данное слово?>>
     \item На втором этапе каждой вершине сопоставляется семантический 
класс и присваивается новый код. За словами (т.\,е.\ конкретными вершинами 
РСС) система видит объекты, действия, свойства, т.\,е.\ строит 
классификации. Производится се\-ман\-ти\-ко-син\-так\-си\-че\-ский анализ 
без выявления глагольных словоформ, при этом предложение представляется 
в виде совокупности фрагментов типа SEM и SEMD~--- специальных меток 
результатов 2-го этапа анализа (рис.~\ref{f3koz}).
     \item На третьем этапе происходит частичное <<сворачивание>> 
синтаксических структур в более компактные (например, свойство объекта и 
сам объект) с присваиванием нового кода и строится фрагмент для объекта, 
обладающего этим свойством.
     \begin{figure*}[b] %fig5
          \vspace*{12pt}
\begin{center}
\mbox{%
\epsfxsize=147.485mm
\epsfbox{koz-5.eps}
}
\end{center}
\vspace*{-9pt}
     \Caption{Глубинная структура предложений
      \label{f5koz}}
      \end{figure*}      
     \item На четвертом этапе выявляются отношения и действия и 
производится анализ непосредственного контекста на соответствие заданным 
семантическим падежам. Система проверяет, подходят ли объекты 
(концепты, понятия) на аргументные места данного действия или отношения. 
При этом отглагольные существительные (<<делатель>>, т.\,е.\ агент 
действия, или <<делание>>~--- процесс~--- анализируются как слова с 
двойной природой: вначале как действия, а затем как объекты). Результатом 
этого этапа является целостная семантическая структура предложения, 
которая представляется фрагментом типа SEMSTR~--- метки результата 4-го 
этапа анализа (рис.~\ref{f4koz}).
     \item На пятом этапе происходит анализ прагматики: установление 
кореференциальных отношений, частичное восстановление эллиптических 
конструкций, система производит дальнейшие действия с построенными 
фрагментами.
     \end{enumerate}

     
Система ДИЕС допускает ввод полисемичных форм глаголов. Для этого следует 
воспользоваться формальной записью лингвистических знаний. 
     В~сис\-те\-мах, основанных на РСС, все функции реализованы на 
единой основе~--- в рамках языков РСС и ДЕКЛ, которые были разработаны 
с ориентацией на задачи обработки естественного языка.

%\vspace*{-6pt}

\section{Представление семантики глаголов, глубинные 
и~поверхностные структуры}
     
     В процессе анализа выявляются семантические вершины предложения: 
происходит выявление <<слов-дей\-ст\-вий>>, т.\,е.\ глаголов, и 
     <<слов-от\-но\-ше\-ний>>. Что же является конструктивной основой\linebreak 
задания семантических представлений предикатных слов и выражений? Как 
убедительно показано в работе~\cite{4koz}, семантика глагола 
определяется его дис\-три\-бу\-тив\-но-транс\-фор\-ма\-ци\-он\-ны\-ми\linebreak 
свойствами. Поэтому смысл предикатных выражений должен кодироваться с 
учетом их дистрибутивных и трансформационных признаков. 
     
     Выдвинутая рядом лингвистов (Хомский, Филлмор) гипотеза о том, что 
все предложения имеют глубинные и поверхностные 
     структуры~[7--10], явилась очень продуктивным 
источником проектных решений при создании первых РСС-сис\-тем и 
развивалась в дальнейшем. 

В~тео\-ре\-ти\-ко-линг\-ви\-сти\-че\-ском 
понимании глубинная структура~--- это абстракция, содержащая все 
элементы, необходимые для образования поверхностных структур 
предложений со сходной семантикой. 

     В~ин\-же\-нер\-но-линг\-ви\-сти\-че\-ском понимании\linebreak глубинная 
структура~--- это запись на языке БЗ, например на языке РСС, 
которая может быть представлена в <<поверхностном>> виде на одном из 
естественных языков в результате конечного числа определенных 
преобразований. Например, предложения

\noindent
\begin{align*}    
(1)\ &\mbox{\textit{The programmer writes the code}}\\
(2)\ &\mbox{\textit{The code is written by the programmer}}
\end{align*}
имеют истоком одну глубинную структуру:

\medskip

\noindent
     \begin{verbatim}
  Programmer <---- write ----> Code
      agent                   object,
\end{verbatim}

\medskip

\noindent
хотя и отличаются своими поверхностными структурами. В~каждом из них 
имеется агент (the programmer), объект (the code) и действие (write).\linebreak Согласно 
концепции \textit{падежной грамматики} Филлмора~\cite{5koz} глубинная 
структура для обоих предложений инвариантна. Эту структуру можно 
представить в виде скобочной записи $V(\mathrm{AGENT}, \mathrm{OBJECT})$. В~графическом 
виде глубинная структура предложения также может быть представлена 
диаграммой в виде дерева, где отражены инвариантные отношения 
зависимости между предикатной вершиной и актантами (рис.~\ref{f5koz}), 
причем в таком представлении явным образом разграничиваются 
\textit{модальность} (MOD) и \textit{пропозиция} (PROP).
     

     В исходном варианте~\cite{5koz} теория признавала шесть падежей: 
агентив, инструменталис, датив, объектив, локатив и фактитив. По мере 
развития теории~\cite{8koz} происходило увеличение числа падежей, однако 
<<умножение>> количества падежей утяжеляет первоначальную 
конфигурацию, поэтому при построении инженерных семантических 
представлений требуется некоторый <<компромиссный>> вариант, 
сочетающий в себе необходимую полноту, с одной стороны, и простоту и 
гибкость, с другой.

\begin{figure*}[b] %fig6
\vspace*{24pt}
\begin{center}
\mbox{%
\epsfxsize=156.873mm
\epsfbox{koz-6.eps}
}
\end{center}
%\vspace*{-9pt}
\Caption{Обобщенное функциональное представление систем ИСПАР
\label{f6koz}}
\end{figure*}
     
%\vspace*{-6pt}

\section{Некоторые базовые аспекты построения многоязычных 
систем}
     
     Одним из приоритетных направлений развития РСС-сис\-тем является 
обеспечение обработки текстов на нескольких языках, прежде всего для 
рус\-ско-анг\-лий\-ской языковой пары. В системах 2-го поколения~--- ДИЕС2, 
ИКС, ЛОГОС-Д были реализованы лингвистические процессоры и словари 
для русского и английского языка, позволявшие обрабатывать тексты для 
ряда ПО. При этом поддерживался как режим ввода 
лингвистических знаний линг\-вис\-том-ана\-ли\-ти\-ком, так и 
автоматический режим самообучения системы по вводимым \mbox{текстам}. 
{\looseness=1

}

Проводились также эксперименты с итальянским и французским языком. 
При создании многоязычных систем авторы обращались к европейским 
языкам. Очевидно, что европейские языки обладают большим числом общих 
правил, чем любой из них с языками других групп. Но при этом все 
естественные языки обладают общей структурой на самом глубинном 
уровне. На этом уровне располагаются главные элементы естественного 
языка: \textit{предложение}, \textit{модальность}, \textit{пропозиция}.
     
     Моделирование смысловых представлений~--- это процесс, 
развивающийся в направлении от поверхностных семантических структур к 
глубинным. Поиск такого внутреннего представления смысла в условиях 
многоязычной ситуации является на\-прав\-ле\-ни\-ем развития методов 
     КЛМ на базе  РСС. 
     
%     \vspace*{-48pt}
     
\section{Интеллектуальные системы поддержки аналитических 
решений}
     
Системы РСС 3-го и 4-го поколения на\-прав\-ле\-ны на извлечение знаний 
в виде \textit{объектов}, или \textit{сущностей}, и связей между ними из 
пред\-мет\-но-ориен\-ти\-ро\-ван\-ных текстов на русском и английском 
языке~\cite{18koz, 19koz}.

    
В настоящее время во всем мире активно ведутся работы по созданию 
систем извлечения фактов из текстов на естественных языках~[11--14], создаются развитые тезаурусы и 
онтологии~\cite{17koz}. Сис\-те\-мы РСС функционально шире, поскольку 
имеют возможность не только извлекать факты, но и поддерживать 
механизмы логического анализа и экспертного вывода на основе 
извлеченных знаний. Сис\-те\-ма\-ми такого рода являются ИСПАР. В~целом это 
направление исследований требует дальнейшей проработки 
     лек\-си\-ко-се\-ман\-ти\-че\-ских представлений, создания 
     пред\-мет\-но-ориен\-ти\-ро\-ван\-ных семантических словарей. 

Обобщенное функциональное представление систем ИСПАР дано на 
рис.~\ref{f6koz}. 
     
     В рамках ИСПАР на основе РСС 
(\mbox{ИСПАР}--РСС) были реализованы полномасштабные и\linebreak пилотные 
проекты для ряда ПО: криминалистики, управления 
кадрами, мониторинга финансово-экономического кризиса и 
др.~\cite{18koz, 19koz}.

\section{Применение аппарата расширенных семантических сетей в~лингвистических 
исследованиях}
     
     В настоящее время в рамках проектов, на\-прав\-лен\-ных на создание 
открытых лингвистических ресурсов~\cite{20koz} для 
     на\-уч\-но-прак\-ти\-че\-ских целей, ведутся работы по выравниванию 
параллельных текстов научных статей, патентов и 
     фи\-нан\-со\-во-эко\-но\-ми\-че\-ских текстов. В~качестве одного из 
методов выравнивания используется РСС-под\-ход, поскольку он позволяет 
отразить глу\-бин\-но-се\-ман\-ти\-че\-ский уровень языковых структур. 

На  рис.~7 представлен фрагмент первого этапа лингвистического 
анализа в многоязычных системах. Для <<идеальной>> ситуации, когда 
структуры исходного текста и текста перевода практически совпадают, такая 
ситуация имеет место в меньшинстве случаев. Основные трудности 
возникают при наличии переводческих трансформаций в параллельных 
текстах. Особое внимание следует уделять гла\-голь\-но-имен\-ным 
трансформациям, например явлению \textit{номинализации}, поскольку она 
очень продуктивна для всех исследовавшихся языков.

     
     Ключевой задачей при разработке методов сопоставления 
параллельных текстов является выявление и детальное описание тех 
языковых трансформаций, которые имеют место при переводе 
     естест\-вен\-но-язы\-ко\-вых конструкций с одного языка на 
другой~\cite{9koz}, потому что далеко не всегда некое содержание 
передается струк\-тур\-но-по\-доб\-ны\-ми средствами в текстах на разных 
языках. Сравнительное исследование употребления различных частей речи в 
параллельных текстах на разных языках создает основу для выявления и 
описания языковых транс-\linebreak

\begin{center} %fig7
\vspace*{3pt}
\mbox{%
\epsfxsize=79.726mm
\epsfbox{koz-7.eps}
}
\end{center}
\vspace*{4pt}
%\begin{center}
{{\figurename~7}\ \ \small{Первый этап анализа параллельных текстов ($W_n$
обозначает словоформу с номером~$n$, $1\leq n\geq 5$)}}
%\end{center}
%\vspace*{9pt}

%\bigskip
\addtocounter{figure}{1}
      

\noindent 
формаций, при этом центральной трансформацией
является \textit{номинализация}. Явление номинализации
было исследовано в 
ряде работ отечественных и зарубежных лингвистов~[17--20]. 
Ближе всего к правильному, по мнению авторов данной статьи, 
пониманию этого явления следующие определения номинализации: 
<<конструкции\ldots называются номинализованными~--- в том смысле, что 
их естественно рассматривать как результат номинализации конструкций с 
предикативным употреблением глаголов и прилагательных>>; 
<<номинализация~--- это синтаксический процесс, который соотносит 
предложения с именными группами>>~\cite{9koz, 10koz}. Выявление 
номинализованных конструкций в параллельных научных и патентных 
текстах на русском, английском, французском и немецком языках в научных 
и патентных текстах и сопоставительное описание гла\-голь\-но-имен\-ных 
межъязыковых трансформаций~--- одна из центральных задач 
     ин\-же\-нер\-но-линг\-ви\-сти\-че\-ских исследований. 
     
     Следующей базовой трансформацией в исследуемых текстах на 
нескольких европейских языках является адъек\-тив\-но-ад\-вер\-би\-аль\-ное 
преобразование. Это означает, что при переводе с одного языка на другой 
происходит синтаксическое преобразование имен прилагательных в наречия 
и обратное преобразование~--- наречий в прилагательные. Установление 
семантических соответствий между этими языковыми объектами также 
возможно осуществить посредством аппарата~РСС. 
     
     При семантическом выравнивании непараллельных текстов, имеющих 
одну и ту же денотативную составляющую, аппарат РСС позволяет выявить в 
текстах когнитивные опоры (слова с сильной валентностью~--- 
     <<сло\-ва-дейст\-вия>> и <<сло\-ва-от\-но\-ше\-ния>>) и установить 
между ними семантические соответствия.

\section{Заключение}

     В данной работе представлен опыт создания и развития 
     когни\-тив\-но-линг\-ви\-сти\-че\-ских пред\-став\-ле\-ний в 
интеллектуальных информационных сис\-те\-мах, разработанных на основе 
аппарата РСС. Аппарат РСС 
обеспечивает мощные изобразительные возможности для описания всех 
уровней естественного языка, включая уровень 
     глу\-бин\-но-се\-ман\-ти\-че\-ских представлений и межъязыковых 
соответствий. Конкретные лингвистические процессоры, которые были 
созданы на основе этого подхода, прошли определенный путь развития и 
позволили выработать проектные решения для основных задач текущего 
этапа~--- извлечения и обработки содержательных знаний из текстов на 
естественных языках и сопоставления языковых структур в текстах на 
различных языках с учетом базовых трансформаций.
     
     Проблема извлечения и обработки знаний открывает перспективы 
развития интеллектуальных направлений компьютерной лингвистики, 
поскольку ее основной акцент смещен в сторону\linebreak глубинных представлений 
языка, в которых используются как грамматические (морфологические и 
синтаксические), так и семантические атрибуты для описания языковых 
объектов. Проводи-\linebreak мые авторами исследования параллельных текстов 
направлены также на рассмотрение этой проблемы~\cite{20koz}. 
Центральное место в проводящихся линг\-ви\-сти\-че\-ских исследованиях 
занимает изучение и формализация процессов трансформации языковых 
структур, особенно все варианты глагольно-но\-ми\-на\-тив\-ных трансформаций, 
создание развитых дис\-три\-бу\-тив\-но-транс\-фор\-ма\-ци\-он\-ных 
описаний предикатых структур для рассматриваемых языков. 
     
     Для задач извлечения знаний и создания \mbox{ИСПАР} 
     дис\-три\-бу\-тив\-но-транс\-фор\-ма\-ци\-он\-ные описания имеют 
особое значение, поскольку таким образом задаются все возможные способы 
перевода языковых структур в пре\-ди\-кат\-но-ар\-гу\-мент\-ные 
представления, которые затем используются в процедурах обработки знаний.

{\small\frenchspacing
{%\baselineskip=10.8pt
%\addcontentsline{toc}{section}{Литература}
\begin{thebibliography}{99}

     \bibitem{1koz}
     \Au{Кузнецов~И.\,П.}
     Семантические представления.~--- М.: Наука, 1986. 290~с.
     
     \bibitem{2koz}
     \Au{Козеренко~Е.\,Б.}
     Кон\-цеп\-ту\-аль\-но-линг\-вис\-ти\-че\-ское моделирование в среде 
интеллектуального редактора знаний ИКС~// Проблемы проектирования и 
использования баз знаний.~--- Киев: Ин-т кибернетики им.\ В.\,М.~Глушкова, 
1992. C.~73--79.
     
     \bibitem{3koz}
     \Au{Kozerenko~E.\,B.}
     Multilingual processors: A unified approach to semantic and syntactic 
knowledge presentation~// Conference (International ) on Artificial Intelligence 
IC-AI'2001 Proceedings. Las Vegas, Nevada, USA. June 25--28, 2001.~--- Las 
Vegas: CSREA Press, 2001. P.~1277--1282.

     \bibitem{18koz} %4
     \Au{Kuznetsov~I.\,P., Efimov~D.\,A., Kozerenko~E.\,B.}
     Tools for tuning the semantic processor to application areas~// ICAI'09 
Proceedings, WORLDCOMP'09. July 13--16, 2009. Las Vegas, Nevada, USA. 
Vol.~I.~--- Las Vegas: CRSEA Press, 2009. P.~467--472.
     
     \bibitem{19koz} %5
     \Au{Kuznetsov~I.\,P., Kozerenko~E.\,B., Kuznetsov~K.\,I., 
Timonina~N.\,O.}
     Intelligent system for entities extraction (ISEE) from natural language 
texts~// Workshop (International) on Conceptual Structures for Extracting Natural 
Language Semantics (Sense'09) at the 17th Conference 
(International ) on Conceptual Structures (ICCS'09) Proceedings. University Higher School of 
Economics. Moscow, Russia, 2009. P.~17--25.
     
     \bibitem{4koz} %6
     \Au{Апресян~Ю.\,Д.}
     Экспериментальное исследование семантики русского глагола.~--- М.: 
Наука, 1967.  252~с.
     
     \bibitem{5koz} %7
     \Au{Филлмор~Ч.}
     Дело о падеже~// Новое в зарубежной линг\-вистике, 1968. Вып.~X. С.~369--495.
     
     \bibitem{6koz} %8
     \Au{Хомский~Н.}
     Аспекты теории синтаксиса.~--- М.: МГУ, 1972.
     
     \bibitem{7koz} %9
     \Au{Хомский Н.}
     Язык и мышление.~--- М.: МГУ, 1972.
     
     
     \bibitem{8koz} %10
     \Au{Fillmore~C.}
     The case for case reopened~// Syntax and Semantics. Vol.~8.~--- N.Y.: 
Academic Press, 1977. 
     

          \bibitem{15koz} %11
     FASTUS: A cascaded finite-state trasducer for extracting information from 
natural-language text~// AIC, SRI International, Menlo Park, California, 1996. 
     
     \bibitem{16koz} %12
     \Au{Han~J., Pei~Y., Mao~R.}
     Mining frequent patterns without candidate generation: A frequent-pattern 
tree approach~// Data Mining and Knowledge Discovery, 2004. Vol.~8. No.\,1. 
P.~53--87.
     
     
     \bibitem{13koz} %13
     \Au{Cunningham~H.}
     Automatic information extraction~// Encyclopedia of Language and 
Linguistics. 2nd ed.~--- Elsevier, 2005.
     
     \bibitem{14koz} %14
     \Au{Han~J., Kamber~M.}
     Data mining: Concepts and techniques.~--- Morgan Kaufmann, 2006.
     
     
     \bibitem{17koz} %15
     \Au{Добров~Б.\,В., Лукашевич~Н.\,В.}
     Онтологии для автоматической обработки текстов: Описание понятий 
и лексических значений~// Компьютерная лингвистика и интеллектуальные 
технологии: Тр. межд. конф. <<Диалог'06>>. Бекасово, 31~мая\,--\,4~июня 
2006. С.~138--142.

     \bibitem{20koz} %16
     \Au{Kozerenko~E.\,B.}
     INTERTEXT: A multilingual knowledge base for machine translation~// 
Conference (International) on Machine Learning, Models, Technologies and 
Applications Proceedings. June 25--28, 2007. Las Vegas, USA.~--- Las Vegas: 
CSREA Press, 2007. P.~238--243.

     \bibitem{9koz} %17
     \Au{Жолковский~А.\,К., Мельчук~И.\,А.}
     О семантическом синтезе~// Проблемы кибернетики, 1967. Вып.~19.
     
         
     \bibitem{11koz} %18
     \Au{Jacobs~R.\,A., Rosenbaum P.\,S.}
     English transformational grammar.~--- Blaisdell, 1968.
     

\label{end\stat}
     
          \bibitem{12koz} %19
     \Au{Балли~Ш.}
     Общая лингвистика и вопросы французского языка. 2-е изд.~--- М.: 
УРСС, 2001.

\bibitem{10koz} %20
     \Au{Падучева~Е.\,В.}
     О~семантике синтаксиса: Мат-лы к трансформационной 
грамматике русского языка. 2-е изд.~--- М: КомКнига, 2007.  296~с. 
     
 \end{thebibliography}
}
}


\end{multicols}