\def\stat{rekl}

   
   \begin{center}

{\prgsh\LARGE
ОБЪЯВЛЕНИЯ О КОНФЕРЕНЦИЯХ}

\end{center}
%\hrule

   \thispagestyle{empty}
   
   \vspace*{4mm}
   
\hfill   {\large{\sf http://www.ipiran.ru/conference/stabil2014/} }

\vspace*{4mm}
   


\noindent
\begin{tabular}{cc}
%\begin{center}
\multicolumn{1}{c}{\raisebox{-7pt}[0pt][0pt]{\mbox{%
\epsfxsize=30.5mm
\epsfbox{logo.eps}
}}}
%\end{center}
&
\tabcolsep=0pt\begin{tabular}{c}
{\prg{\Large\textbf{XXXII Международный семинар по проблемам}}}\\[6pt]
{\prg{\Large\textbf{ устойчивости стохастических моделей}}}\\[12pt]
{\prg{\Large\textbf{(XXXII International Seminar on Stability Problems}}}\\[6pt]
{\prg{\Large\textbf{for Stochastic Models)}}}\\[12pt]
{\prg{\large 16--24 июня 2014~г.}}\\[6pt]
{\prg{\large Норвежский университет науки и технологии}}\\[6pt]
{\prg{\large  Тронхейм, Норвегия}}
\end{tabular}
\end{tabular}


\vspace*{6mm}

XXXII Международный семинар по проблемам устойчивости стохастических моделей 
будет организован Норвежским университетом науки и технологии (Тронхейм) (НТНУ),  
МГУ им. М.\,В.~Ломоносова и Институтом проблем информатики Российской академии 
наук (ИПИ РАН). Семинар будет проведен в НТНУ. 

\vspace*{6pt}

Проведение традиционного Международного семинара по проблемам устойчивости 
стохастических моделей имеет давнюю традицию, начинающуюся с 70-х гг.\ XX~в. 
Основателем семинара был профессором В.\,М.~Золотарев. Семинар проходил во многих 
странах; в XXI~в.~--- в Венгрии, Болгарии, Испании, Латвии, Италии, Израиле, 
Румынии, Польше и России.

\vspace*{12pt}

{\centerline{\prg{\large\textbf{Главные темы семинара}}}}
\begin{itemize}
\item  Предельные теоремы теории вероятностей
\item Асимптотическая теория случайных процессов
\item Устойчивые распределения и процессы 
\item Асимптотические методы математической статистики
\item Теория риска
\item Теория вероятностных метрик
\item Характеризация вероятностных распределений
\item Дискретные вероятностные модели
\item  Актуарная и финансовая математика
\item Теория массового обслуживания и моделирование информационных систем
\end{itemize}

\vspace*{6pt}


{\centerline{\prg{\large\textbf{Международный программный и организационный 
 комитет}}}}
 
 \vspace*{12pt}

В.\,М. Золотарев (Россия)~--- почетный председатель

В.\,Ю. Королев (Россия / МГУ им.\ М.\,В.~Ломоносова, ИПИ РАН)~--- председатель

Н.\,Г. Ушаков (Норвегия)~--- зам. председателя

И.\,Г. Шевцова~--- (Россия / МГУ им.\ М.\,В.~Ломоносова, ИПИ РАН)~--- генеральный секретарь

Ш.~Баран (Венгрия)

В.\,Е. Бенинг (Россия / МГУ им.\ М.\,В.~Ломоносова, ИПИ РАН)

А.\,В. Булинский (Россия / МГУ им.\ М.\,В.~Ломоносова)

А. И. Зейфман (Россия / Вологодский ГУ, ИПИ РАН)

И. Мисевич (Польша)

Ю.\,С. Нефедова (Россия / МГУ им.\ М.\,В.~Ломоносова, ИПИ РАН)

Э. Омей (Бельгия)

Д. Пап (Венгрия)

Ю.\,С. Хохлов (Россия / РУДН)

С.\,Я. Шоргин (Россия / ИПИ РАН)

\vspace*{12pt}

   \thispagestyle{empty}


{\centerline{\prg{\large\textbf{Важные даты}}}}

\vspace*{5pt}

\textbf{1 декабря 2013 г.}~--- начало регистрации

\textbf{1 марта 2014 г.}~--- крайний срок подачи тезисов

\textbf{15 марта 2014~г.}~--- извещение о включении доклада в программу конференции 

\vspace*{11pt}

{\centerline{\prg{\large\textbf{Публикации}}}}

\vspace*{5pt}

Тезисы докладов XXXII Международного семинара по проблемам устойчивости стохастических 
моделей будут опубликованы к началу семинара.  Избранные труды семинара будут в 
дальнейшем опубликованы в журналах ``Journal of Mathematical Sciences'' (издательство 
``Springer 
Science+Business Media,'' ISSN: 1072-3374, индексируется в системе Scopus), <<Информатика и её 
применения>> (издательство ТОРУС ПРЕСС, ISSN: 1992-2264) и <<Системы и средства информатики>> (издательство ТОРУС 
ПРЕСС, ISSN: 0869-6527).


%\vspace*{12pt}

%\begin{center}
%{\prg{\large\textbf{Информация о семинаре и форма для регистрации~--- на сайте}}}\\[6pt]
% {\large{\sf http://www.ipiran.ru/conference/stabil2014/} }
% \end{center}
 
% \vspace*{12pt}
 
% \hrule
 
% \vspace*{2pt}
 
% \hrule
 
% \vspace*{11pt}

\newpage
 
\hfill {\large{\sf http://www.scs-europe.net/conf/ecms2014/index.html}}


\vspace*{3mm}
   


\noindent
\begin{tabular}{cc}
%\begin{center}
\multicolumn{1}{c}{\raisebox{5pt}[0pt][0pt]{\mbox{%
\epsfxsize=40mm
\epsfbox{logo2.eps}
}}}
%\end{center}
&
\tabcolsep=0pt\begin{tabular}{c}
%\begin{center}
{\prg{\Large\textbf{Специальная сессия}}}\\[5pt]
{\prg{\Large\textbf{<<Вероятностные и статистические методы}}}\\[5pt]
{\prg{\Large\textbf{математического и имитационного}}}\\[5pt]
{\prg{\Large\textbf{моделирования информационных систем}}}\\[5pt]
{\prg{\Large\textbf{высокой производительности>>}}}\\[10pt]
{\prg{\large Май 2014 г.}}\\[5pt]
{\prg{\large Брешия, Италия}}
%\end{center}
\end{tabular}
\end{tabular}


\vspace*{2mm}

В рамках 28-й Европейской конференции по математическому и имитационному 
моделированию (28th European Conference on Modelling and Simulation~--- ECMS~2014), 
которая состоится в Брешии (Италия) с 27 по 30~мая 2014~г., будет проведена 
специальная сессия <<Вероятностные и статистические методы математического и 
имитационного моделирования информационных систем высокой производительности>> 
(Probability and Statistical Methods for Modelling and Simulation of High 
Performance Information Systems), организуемая с участием Института проблем 
информатики Российской академии наук (ИПИ РАН) и Российского университета дружбы народов 
(РУДН).

Сессия проводится во второй раз. В~2013~г.\ она была проведена в рамках 
конференции ECMS~2013 в Олесунне (Норвегия).

\vspace*{12pt}

{\centerline{\prg{\large\textbf{Сопредседатели сессии}}}}

\vspace*{5pt}

А.\,И. Зейфман (Вологодский государственный университет, ИПИ РАН)

П.\,О. Абаев (РУДН)

Р.\,В. Разумчик (ИПИ РАН, РУДН)

\vspace*{12pt}

{\centerline{\prg{\large\textbf{Члены программного комитета}}}}

\vspace*{5pt}

А.\,А. Грушо (ИПИ РАН, МГУ им.\ М.\,В.~Ломоносова)

В.\,Ю. Королев (МГУ им.\ М.\,В.~Ломоносова, ИПИ РАН)

А.\,В. Печинкин (ИПИ РАН, РУДН)

К.\,Е. Самуйлов (РУДН)

С.\,Я. Шоргин (ИПИ РАН)

%\pagebreak

\vspace*{12pt}

{\centerline{\prg{\large\textbf{Тематика специальной сессии}}}}
\begin{itemize}
\item  Математические и имитационные модели систем массового обслуживания
\item  Моделирование и анализ производительности информационных и телекоммуникациных систем
\item  Моделирование и анализ информационных потоков
%\pagebreak
\item Математическое и имитационное моделирование перегрузок и управления потоками 
\item Оценка эффективности, надежности и устойчивости информационных систем высокой производительности
\end{itemize}

\vspace*{7pt}

\textbf{Труды конференции публикуются в изданиях, индексируемых в 
системах Web of Sciences и Scopus.}

\vspace*{14pt}

{\centerline{\prg{\large\textbf{Важные даты}}}}

\vspace*{7pt}

\textbf{13 февраля 2014~г.}~--- представление полного доклада

\textbf{14 марта 2014~г.}~--- извещение о принятии доклада

\textbf{9~апреля 2014~г.}~--- представление окончательной версии доклада, регистрация и оплата взносов
   \thispagestyle{empty}