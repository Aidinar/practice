\def\stat{shestakov}

\def\tit{ОБРАЩЕНИЕ СФЕРИЧЕСКОГО ПРЕОБРАЗОВАНИЯ РАДОНА В~КЛАССЕ ДИСКРЕТНЫХ СЛУЧАЙНЫХ 
ФУНКЦИЙ$^*$}

\def\titkol{Обращение сферического преобразования радона в классе дискретных случайных 
функций}

\def\autkol{О.\,В.~Шестаков, М.\,Г.~Кузнецова, И.\,А.~Садовой}

\def\aut{О.\,В.~Шестаков$^1$, М.\,Г.~Кузнецова$^2$, И.\,А.~Садовой$^3$}

\titel{\tit}{\aut}{\autkol}{\titkol}

{\renewcommand{\thefootnote}{\fnsymbol{footnote}}\footnotetext[1] {Работа выполнена при финансовой поддержке Министерства образования и науки РФ 
(государственный контракт №\,14.740.11.0996).}}

\renewcommand{\thefootnote}{\arabic{footnote}}
\footnotetext[1]{Московский государственный университет им.\ М.\,В.~Ломоносова,
факультет ВМК; Институт проблем информатики Российской академии наук,
oshestakov@cs.msu.su} 
\footnotetext[2]{Московский государственный университет им.\ М.\,В.~Ломоносова,
факультет ВМК, m.g.kuznetsova@gmail.com} 
\footnotetext[3]{Московский государственный университет им.\ М.\,В.~Ломоносова,
факультет ВМК, isadovoy@gmail.com} 

\Abst{Рассмотрена задача
восстановления вероятностных распределений случайных функций по
распределениям сферических проекций, описывающих данные в некоторых
видах томографических экспериментов, включая термоакустическую
томографию, оптоакустическую томографию и радиолокацию. Задачи
подобного рода возникают в ситуациях, когда исследуемый объект может
случайным образом менять свою структуру в процессе регистрации
проекционных данных. Это приводит к невозможности использования
традиционных методов вычислительной томографии. 
Предполагается, что случайный объект может иметь не более счетного
числа структурных состояний, каждое из которых описывается
интегрируемой функцией с компактным носителем. Для такого
дискретного класса случайных функций доказывается взаимно
однозначное соответствие между распределением случайной функции и
распределениями ее проекций и разрабатывается метод восстановления,
основанный на свойствах так называемых моментов проекций. Также
исследована устойчивость разработанного метода к погрешностям и
показано, что метод дает адекватные результаты в случаях, когда
проекционные данные содержат шум.}

\KW{случайные функции; сферическое преобразование Радона; стохастическая томография
}

\DOI{10.14357/19922264130408}

\vskip 14pt plus 9pt minus 6pt

      \thispagestyle{headings}

      \begin{multicols}{2}

            \label{st\stat}


\section{Введение}

Сферическое преобразование Радона возникает в различных приложениях,
включая термоакустическую томографию, оптоакустическую томографию,
обработку сигналов, получаемых радарами и сонарами, и~т.\,п.~[1--3].
Оно также используется в различных областях теории аппроксимации и
математической физики~[4].

Пусть $f(x,y)$~--- непрерывная функция с компактным носителем.
Далее, не ограничивая общности, будем считать что носителем функции
$f(x,y)$ является круг~$U$ единичного радиуса с центром в начале
координат. Определим сферическое преобразование Радона на плоскости
следующим образом:
\begin{equation}
Rf(p,r)=\int\limits_{S(p,r)}f(x,y)\,ds\,.\label{e1-she}
\end{equation}
Здесь $p\in \mathbf{R}^2$; $r\hm\in(0,\infty)$, $S(p,r)$~--- окружность
с центром в точке~$p$ и радиусом~$r$;  $ds$~--- мера на $S(p,r)$. 
В~данной работе будет рассмотрен важный для приложений (в частности,
томографических) случай, когда центры окружностей $S(p,r)$
располагаются на границе носителя $f(x,y)$, т.\,е.\
$p\hm=(\cos\theta,\sin\theta)$, $\theta\hm\in[0,2\pi)$. По аналогии с
классическим преобразованием Радона интегральные преобразования вида~(\ref{e1-she}) 
называются сферическими проекциями, т.\,е.\ сферическая проекция
представляет собой функцию от $r\hm\in(0,\infty)$ при фиксированном~$p$
(или в данном случае~$\theta$. Такие сферические проекции будем
обозначать через $Rf(\theta,r)$).

В ряде томографических приложений (см., например,~[5]) функцию
$f(x,y)$, описывающую изоб\-ра\-же\-ние изучаемого объекта, необходимо
считать случайной. При этом основной особенностью является то
обстоятельство, что состояния (реализации) функции меняются
случайным образом во время процесса получения проекций. Это приводит
к тому, что восстановление даже одной реализации случайной функции
обычными томографическими методами невозможно. Основной интерес в
такого рода задачах представляют вероятностные характеристики
функции $f(x,y)$.

В работах~[6, 7] рассмотрена задача определения вероятностных
характеристик двумерных случайных функций в модели классического
преобразования Радона. Показано, что в общем случае эта задача
характеризуется сильной неоднозначностью и содержательные результаты
удается получить лишь в том случае, когда случайная функция имеет не
более чем счетное число состояний. В~работе~[7] для класса таких
функций разрабатывается метод восстановления распределений двумерных
случайных функций.

В данной работе для класса случайных функций, имеющих не более чем
счетное число состояний, разрабатывается метод восстановления
распределений случайной функции по распределениям ее сферических
проекций.

\section{Постановка задачи}

Постановка задачи, рассматриваемой в настоящей работе, следующая.
Имеется двумерная случайная функция $\xi(x,y)$~--- стохастический
объект вида $\xi(x,y) \hm= f_{\nu}(x,y)$, где
$f_{1}(x,y),f_{2}(x,y),\ldots$~---\linebreak последовательность интегрируемых
функций, определенных в единичном круге~$U$, а $\nu$~--- случайная
величина, принимающая целые положительные значения. Вероятностная
структура $\xi(x,y)$ полностью определяется набором
$(f_{1}(x,y),f_{2}(x,y),\ldots;p_{1},p_{2},\ldots)$, где
$p_{i}\hm=P(\xi(x,y) \hm= f_{i}(x,y)),$ $i \hm= 1,2,\ldots ,$ 
$\sum\limits_{i=1}^{\infty}p_{i} \hm= 1$. Распределение $\xi(x,y)$ будем
обозначать через $P_{\xi}$, а распределения сферических проекций
(при фиксированном $\theta\hm\in [0,2\pi)$)~--- через $P_{\xi^{(\theta)}}$.
Требуется по распределениям $P_{\xi^{(\theta)}}$ для
$\theta\in\Lambda$, где $\Lambda$~--- некоторое подмножество
$[0,2\pi)$, определить распределение~$P_{\xi}$.

\section{Теорема единственности}

Проблема описания множества точек $p$, для которого задача обращения
преобразования~(\ref{e1-she}) имеет единственное решение, рассматривалась
многими авторами (см., например,~[1, 2, 4]). В~$\mathbf{R}^2$ эта
проб\-ле\-ма полностью решена в работе~[4]. В~частности, в этой работе
доказана следующая теорема.

\medskip

\noindent
\textbf{Теорема 1.} \textit{Пусть $P$~--- множество всех точек~$p$, для
которых известны сферические проекции вида~$(\ref{e1-she})$. Функцию $f(x,y)$ с
компактным носителем невозможно восстановить единственным образом по
преобразованиям $Rf(p,r)$, $p\hm\in P$, $r\hm\in(0,\infty)$, тогда и
только тогда, когда
$$P\subset Х{\sf M}\left(\Sigma_N\right)\bigcup\Phi.$$
Здесь ${\sf M}$ -- оператор движения (перемещения и/или поворота),;$\Phi$~--- 
конечное множество; $\Sigma_N$~--- сис\-те\-ма прямых Коксетера,
определяемая следующим образом: для любого натурального~$N$ сис\-те\-ма~$\Sigma_N$ 
со\-сто\-ит из прямых~$L_k$ ($k\hm=0,\ldots,N-1$), проходящих
через начало координат под углом $\pi k/N$ к оси}~$OX$.

\medskip

Кроме того, в работе~[2] также доказана еще одна теорема единственности.

\medskip

\noindent
\textbf{Теорема 2.} \textit{Пусть $D$~--- ограниченное открытое множество в
$\mathbf{R}^2$ с гладкой границей~$\partial D$, и пусть~$\overline{D}$ 
(замыкание множества~$D$) представляет собой строго
выпуклое множество. Обозначим через~$\Lambda$ любое открытое
подмножество~$\partial D$. Если носитель гладкой функции $f(x,y)$
содержится в~$\overline{D}$ и $Rf(p,r)\hm=0$ для всех $p\hm\in\Lambda$ и
всех~$r$, тогда $f(x,y)\equiv0$.}

\medskip


Из этих теорем следует, что в рамках рассматриваемой модели функцию
$f(x,y)$ можно однозначно восстановить по значениям сферических
проекций $Rf(\theta,r)$, заданным при $\theta\in\Lambda$, где
$\Lambda$~--- любое подмножество $[0,2\pi)$, имеющее положительную
меру Лебега.

Используя этот факт, покажем, что распределение двумерной случайной
функции полностью определяется распределениями проекций, а именно
имеет место следующая теорема.

\medskip

\noindent
\textbf{Теорема 3.} \textit{Пусть случайные функции $\xi(x,y)$ и $\eta(x,y)$ 
имеют описанный выше вид и
$$
P_{\xi^{(\theta)}}=P_{\eta^{(\theta)}}
$$ 
для всех $\theta\hm\in\Lambda$, где $\Lambda$~--- 
любое подмножество $[0,2\pi)$ положительной меры.  Тогда} $$
P_\xi=P_\eta\,.$$

\medskip

\noindent
Д\,о\,к\,а\,з\,а\,т\,е\,л\,ь\,с\,т\,в\,о\,.\ \ Идея доказательства в основном повторяет идеи
аналогичной теоремы в работе~[6]. Предположим, что
$P_\xi\not=P_\eta$. Это означает, что существует функция $f(x,y)$ с
носителем в круге~$U$ такая, что
$$
P(\xi(x,y)=f(x,y))\not=P(\eta(x,y)=f(x,y))\,.
$$
Обозначим через $f_1(x,y)$, $f_2(x,y),\ldots$ значения случайной функции $\xi(x,y)$,
отличные от $f(x,y)$, а через
$g_1(x,y),g_2(x,y),\ldots$ обозначим аналогичные значения $\eta(x,y)$ (таким образом, 
$f(x,y)\not\equiv f_i(x,y)$ и
$f(x,y)\not\equiv g_i(x,y)$, $i\hm=1,2,\ldots$). Для каждого фиксированного
$i\hm=1,2,\ldots$ пусть~$A_i$ обозначает множество всех
$\theta\hm\in\Lambda$, для которых
$$
Rf(\theta,r)\equiv Rf_i(\theta,r)
$$
и, соответственно, $B_i$~--- множество всех $\theta\hm\in\Lambda$, для которых
$$
Rf(\theta,r)\equiv Rg_i(\theta,r)\,.
$$
Каждое из множеств $A_i$ и $B_i$ имеет меру нуль. Если для некоторого~$i$
это было бы не так, то $Rf(\theta,r)$ совпадала бы с $Rf_i(\theta,r)$ или 
с $Rg_i(\theta,r)$ на множестве
положительной меры и в силу приведенных выше утверждений это влекло бы за
собой совпадение функции $f(x,y)$ с функцией $f_i(x,y)$ или с функцией $g_i(x,y)$.

Таким образом, множество
$$
C=\bigcup\limits_{i=1}^{\infty}\left(A_i\bigcup\limits B_i\right)
$$
имеет меру нуль. Следовательно, множество $\Lambda\setminus C$
непусто. Возьмем произвольное $\theta\hm\in\Lambda\setminus C$.
Поскольку для этого~$\theta$ выполнено $Rf(\theta,r)\not\equiv
Rf_i(\theta,r),\;i\hm=1,2,\ldots$,
$$
P(R\xi(\theta,r)=Rf(\theta,r))=P(\xi(x,y)=f(x,y))
$$ 
и аналогично
$$
P(R\eta(\theta,r)=Rf(\theta,r))=P(\eta(x,y)=f(x,y))\,.
$$
Полученное противоречие доказывает теорему.

\section{Группировка проекций}

Как следует из предыдущего раздела, в рамках описанной модели
возможно восстановить распределение двумерной случайной функции,
зная распределения ее сферических проекций на множестве~$\Lambda$,
имеющем положительную меру. В~этом разде\-ле будет предложен метод,
позволяющий разделить множество зарегистрированных сферических
проекций на группы, соответствующие различным состояниям случайной
функции. Для \mbox{удобства} будем полагать, что~$\Lambda$ совпадает с
$[0,2\pi)$.


Для простоты изложения в данной работе будет рассматриваться класс
функций, имеющих всего два состояния. Обобщение на любое конечное
число состояний очевидно, а для случая счетного числа состояний
можно произвести процедуру <<усечения>> распределений сферических
проекций, так же как это делается в работе~[7].

Итак, пусть случайная функция
$\xi(x,y)$ принимает значения $f_{1}(x,y)$ и $f_{2}(x,y)$ с
вероятностями~$p_{1}$ и~$p_{2}$. Предполагается, что известны
распределения сферических проекций для всех $\theta\hm\in [0,2\pi)$, т.\,е.\ для каждого
$\theta\in [0,2\pi)$ известны функции $Rf_{i}(\theta,r)$, $i\hm=1,2,$
являющиеся проекциями функций $f_{i}(x,y)$, $i\hm=1,2,$ и
реализующиеся с вероятностями~$p_{1}$ и~$p_{2}$ соответственно.
Причем, вообще говоря, заранее не известно, какому состоянию случайной функции соответствует
состояние проекции, т.\,е.\ может оказаться так, что $Rf_{1}(\theta,r)$ является проекцией $f_{2}(x,y)$, а
$Rf_{2}(\theta,r)$~--- проекцией $f_{1}(x,y)$. Необходимо разделить
функции $Rf_{i}(\theta,r)$, $i\hm=1,2,$ для всех $\theta\hm\in [0,2\pi)$ на
группы так, чтобы каждая группа состояний проекций относилась к
одному состоянию случайной функции.

Если $p_{1}\neq p_{2}$, то такое разделение можно произвести по вероятностям состояний
сферических проекций, т.\,е.\ для всех $\theta\hm\in [0,2\pi)$ то значение
$Rf_{i}(\theta,r)$, которое реализуется с вероятностью~$p_{1}$, 
относится к первой группе, а значение $Rf_{i}(\theta,r)$, которое
реализуется с вероятностью~$p_{2}$,~--- ко второй.

В случае, когда $p_{1}=p_{2}={1}/{2}$, метод группировки основан на 
использовании некоторых свойств сферических проекций.

\medskip

\noindent
\textbf{Лемма.} \textit{Пусть $Rf(\theta,r)$~--- сферическая проекция 
функции $f(x,y)$, имеющей носитель в круге~$U$. Тогда интеграл
\begin{equation}
J^{(k)}(\theta)=\int\limits_{0}^{\infty}r^{2k}Rf(\theta,r)\,dr\,,\enskip 
k=0,1\ldots\,,\label{e2-she}
\end{equation}
который называется $2k$-м моментом сферической проекции
$Rf(\theta,r)$, представляет собой тригоно\-мет\-ри\-че\-ский многочлен от~$\theta$ 
степени не выше чем~$k$.}

\medskip

\noindent
Д\,о\,к\,а\,з\,а\,т\,е\,л\,ь\,с\,т\,в\,о\,.\ \
\begin{multline*}
\int\limits_{0}^{\infty}r^{2k}Rf(\theta,r)\,dr={}\\
{}=
\int\limits_{0}^{\infty}r^{2k}\!\int\limits_{0}^{2\pi}f
(\cos\theta+r\cos\varphi,\sin\theta+r\sin\varphi)r\,d\varphi dr={}\\
{}=\int\limits_{-\infty}^{\infty}\int\limits_{-\infty}^{\infty}\!\!((x-\cos\theta)^2+(y-\sin\theta)^2)^kf(x,y)\,dxdy={}\\
{}=\!\int\limits_{-\infty}^{\infty}\int\limits_{-\infty}^{\infty}\!\!(x^2+y^2-2x\cos\theta-2y\sin\theta+1)^kf(x,y)\,dxdy\,.\hspace*{-5.9pt}
\end{multline*}
Указанные интегралы существуют, так как функция $f(x,y)$ имеет
носитель в круге~$U$. Последний интеграл, очевидно, представляет
собой тригонометрический многочлен от $\cos\theta$ и $\sin\theta$
степени не выше чем~$k$. Лемма доказана.

\medskip

\noindent
\textbf{Замечание.} Нетрудно убедиться, что при данном $\theta\hm\in[0,2\pi)$
$$
Rf(\theta,r)\equiv0
$$ 
при $r\in(0,\infty)$ тогда и только тогда, когда 
$$
\int\limits_{0}^{\infty}r^{2k}Rf(\theta,r)\,dr=0
$$
для всех $k=0,1,\ldots$ (интеграл на самом деле имеет конечные
пределы по $r\hm\in(0,2)$ в силу того, что функция $f(x,y)$ имеет
носитель в круге~$U$). Действительно, условие $Rf(\theta,r)\hm\equiv0$
эквивалентно тому, что
$$
\int\limits_{0}^{\infty}g(r^2)Rf(\theta,r)\,dr=0
$$
для любой непрерывной функции $g(r)$, имеющей компактный носитель. 
В~свою очередь, любую такую функцию можно равномерно приблизить
многочленом.

Используя приведенные свойства сферических проекций, можно построить метод группировки.

Сначала вычисляются интегралы от $Rf_{i}(\theta,r)$, $i\hm=1,2$, по~$r$ 
для некоторого $\theta\hm\in[0,2\pi)$. Если эти интегралы отличны
друг от друга, то, поскольку их значения не зависят от~$\theta$ (они
представляют собой тригонометрические многочлены нулевой степени, т.\,е.\ 
константы), можно произвести группировку, основываясь на этих
значениях. Для этого, вычисляя для каждого $\theta\hm\in[0,2\pi)$
интегралы по~$r$ от $Rf_{i}(\theta,r)$, $i\hm=1,2$, сферические
проекции $Rf_{i}(\theta,r)$ следует относить к той или иной группе в
зависимости от того, чему равен вычисленный интеграл. В~результате в
каждой группе окажутся сферические проекции $Rf_{i}(\theta,r)$,
$\theta\hm\in[0,2\pi)$, интегралы от которых по~$r$ равны одному и тому
же зна\-чению.
{\looseness=1

}

В случае, когда интегралы от $Rf_{i}(\theta,r)$, $i\hm=1,2$, по~$r$
совпадают, рассматриваются моменты $J^{(k)}_i(\theta)$. В~силу
сделанного выше замечания, если все моменты $J^{(k)}_i(\theta)$ двух
функций совпадают между собой, то эти функции эквивалентны. Значит,
если функции $Rf_{i}(\theta,r)$, $i\hm=1,2$, различны, то найдется
номер~$m$, для которого моменты сферических проекций,
$J^{(m)}_{i}(\theta)$, $i\hm=1,2$, различаются. Поскольку моменты
являются тригонометрическими многочленами от~$\theta$, они либо
тождественно равны, либо пересекаются в конечном числе точек. Пусть
$\delta$ задано так, что найдется точка~$\theta$, в окрестности
которой разница между функциями $m$-х моментов сферических проекций
$Rf_{i}(\theta,r)$, $i\hm=1,2$, по модулю больше~$\delta$ (т.\,е.\ в
этой окрестности одна функция момента превышает другую больше чем на~$\delta$). 
Посчитав значения $J^{(m)}_{i}(\theta_{j})$, $i\hm=1,2$, в
$2m+1$ близких точках~$\theta_{j}$ из этой окрестности, можно
разделить эти значения на группы, соответствующие каждому состоянию
случайной функции (если разности
$J^{(m)}_{1}(\theta_{j})\hm-J^{(m)}_{2}(\theta_{j})$ и
$J^{(m)}_{1}(\theta_{j+1})\hm-J^{(m)}_{2}(\theta_{j+1})$ имеют разный
знак, то значения $J^{(m)}_{1}(\theta_{j})$ и
$J^{(m)}_{1}(\theta_{j+1})$ соответствуют разным состояниям
случайной функции и следует поменять местами $Rf_1(\theta_{j+1},r)$
и $Rf_2(\theta_{j+1},r)$), а затем найти явный вид функций моментов
сферических проекций $J^{(m)}_{i}(\theta)$,  $i\hm=1,2$, от переменной~$\theta$. 
Для этого нужно решить следующие две линейные сис\-те\-мы уравнений:
\begin{multline}
a_{m,0}^i+\sum\limits_{n=1}^{m}a_{m,n}^{i}\cos(n\theta_{j})+
\sum\limits_{n=1}^{m}b_{m,n}^{i}\sin(n\theta_{j})={}\\
{}=J^{(m)}_{i}(\theta_{j})\,,\enskip
j=0,\ldots ,2m\,, \enskip  i=1,2\,,\label{e3-she}
\end{multline}
и найти коэффициенты
$a_{m,n}^{i}$ и $b_{m,n}^{i}$ тригоно\-мет\-ри\-че\-ских многочленов, 
которыми представляются функции $J^{(m)}_{i}(\theta)$.

Вывод явного вида функций $J^{(m)}_{i}(\theta)$ аналогичен выводу интерполяционного 
многочлена Лагранжа (подробности можно найти в~[8]).
Решая~(\ref{e3-she}), имеем
\begin{multline}
J^{(m)}_{i}(\theta)=\sum\limits_{j=0}^{2m}J^{(m)}_{i}(\theta_{j})\prod\limits_{k=0, k\neq
j}^{2m}\fr{\sin(({\theta-\theta_k})/2)}{\sin({(\theta_j-\theta_k)}/2)}\,,\\
 i=1,2\,.\label{e4-she}
 \end{multline}

Далее при вычислении для каждого~$\theta$ значения
$J^{(m)}_{i}(\theta)$ по формуле~(\ref{e2-she}) сферическая проекция относится
к той или иной группе в зависимости от того, со значением какой из
найденных функций~(\ref{e4-she}) в точке~$\theta$ совпадает это вычисленное значение.

После того как сферические проекции распределены по группам, можно восстановить
каждое состояние случайной функции, а значит, и ее распределение,
с помощью обычных формул обращения (см., например,~[9--11]).

\section{Устойчивый метод группировки при~наличии погрешностей}

На практике имеется конечный набор сферических проекций 
для $\theta_1,\ldots,\theta_N$. Причем, как правило, проекции
задаются не точно, а с некоторой погрешностью. Погрешности 
возникают вследствие несовершенства оборудования, регистрирующего
проекции, случайных помех при измерении, ошибок интерполяции и других причин.

Предположим, что сферические проекции каж\-до\-го состояния случайной функции 
$\xi(x,y)$ заданы с погрешностью, не превышающей заданного уровня $\eps\hm>0$:
\begin{multline*}
\abs{Rf_{i}(\theta_l,r)-Rf^{e}_{i}(\theta_l,r)}<\eps, \enskip
r\in[0,2]\,,\\
 i=1,2\,,\enskip  l=1,\ldots,N\,,
\end{multline*}
где $Rf^{e}_{i}(\theta_l,r)$~--- проекции, измеренные с ошибкой.

Тогда значения моментов сферических проекций заданы с погрешностью

\noindent
\begin{multline}
\abs{\int\limits\limits^{\infty}_{0}r^{2m}(Rf_{i}(\theta_l,r)-
Rf^{e}_{i}(\theta_l,r))\,dr}={}\\
{}=\abs{\int\limits\limits^{2}_{0}r^{2m}(Rf_{i}(\theta_l,r)-
Rf^{e}_{i}(\theta_l,r))\,dr}\leq{}\\
{}\leq\int\limits\limits^{2}_{0}r^{2m}\abs{Rf_{i}(\theta_l,r)-Rf^{e}_{i}(\theta_l,r)}dr<{}\\
{}<
\int\limits\limits^{2}_{0}\eps\abs{r^{2m}}dr<\fr{2^{2m+1}}{2m+1}\eps\,.\label{e5-she}
\end{multline}

Следовательно, можно считать, что интегралы от $Rf_{i}(\theta_l,r)$,
$i\hm=1,2$, $l\hm=1,\ldots,N,$ не совпадают,
и производить группировку проекций на основании значений этих интегралов, если выполнено условие
$$
\abs{\int\limits\limits^{2}_{0}(Rf_{i}(\theta_j,r)-Rf^{e}_{i}(\theta_j,r))\,dr}>4\eps
$$
для некоторого $1\leq j\leq N$. Если же это условие не выполнено, то можно 
считать, что интегралы совпадают и разница между ними
возникает за счет погрешностей.

Для оценки погрешности, с которой вы\-чис\-ля\-ют\-ся 
функции моментов сферических проекций, воспользуемся известной оценкой по\-греш\-ности
интерполяции тригонометрическими многочле\-на\-ми~[8]. 
Вычислим по формуле~(\ref{e2-she}) моменты $J^{(m)}_i(\theta_n)$ в
точках $\theta_n\hm={2n\pi}/(2m+1)$, $i\hm=1,2,$ $n\hm=0,\ldots,2m$. 
Всего существует $2^{2m+1}$ способов распределить
значения $J^{(m)}_l(\theta_n),$ $l\hm=1,2,$ по двум группам (на практике число $2m\hm+1$, 
как правило, невелико). Обозначим через $H$ множество
всех возможных распределений. Решим системы уравнений
\begin{multline*}
a_{m,0}^h+\sum\limits_{j=1}^{m}a_{m,j}^{h}\cos(j\theta_{n})+
\sum\limits_{j=1}^{m}b_{m,j}^{h}\sin(j\theta_{n})={}\\
{}=J^{(m)}_{i^h_n}(\theta_{n})\,,
\enskip
i^h_n=1\ \mbox{ или }\ 2\ \mbox{ в зависимости от }\ h\,,\\
n=0,\ldots, 2m\,,
\end{multline*}
для всех возможных распределений~$h$ из~$H$. В~результате получится
$2^{2m+1}$ функций (обозначим их через $I^{(m)}_h(\theta)$, $h\hm=1,\ldots,2^{2m+1}$), 
претендующих на роль функций моментов
$J^{(m)}_i(\theta)$, $i\hm=1,2$. Если предположить отсутствие
погрешностей, то достаточно, перебирая $I^{(m)}_h(\theta)$,
проверять, равно ли значение $I^{(m)}_h(\theta^*)$ ка\-ко\-му-ли\-бо из
значений $J^{(m)}_i(\theta^*),$ $i\hm=1,2$, вычисленных по формуле~(\ref{e2-she}) 
в произвольно выбранной
точке~$\theta^*$, отличной от точек~$\theta_n$. При том выборе~$h^*$,
при котором равенство имеет мес\-то, функция $I^{(m)}_{h^*}(\theta)$
совпадает с одной из $J^{(m)}_i(\theta),$ $i\hm=1,2,$ для всех
$\theta\hm\in[0,2\pi)$, поскольку тригонометрические многочлены степени~$m$, совпадающие
более чем в $2m+1$ точках, тождественно равны.

В случае же наличия погрешностей для функции $I^{(m)}_{h^*}(\theta)$, претендующей на роль функции
момента $J^{(m)}_i(\theta),$ $i\hm=1$ или 2, в силу~(\ref{e5-she}) 
должно выполняться
$$
\abs{I^{(m)}_{h^*}(\theta_l)-J^{(m)}_1(\theta_l)}\leq\eps
\fr{2^{2m+1}}{2m+1}\left(8+\fr{4}{\pi}\ln m\right)
$$
или
$$
\abs{I^{(m)}_{h^*}(\theta_l)-J^{(m)}_2(\theta_l)}\leq\eps
\fr{2^{2m+1}}{2m+1}\left(8+\fr{4}{\pi}\ln m\right)
$$
для всех $l=1,\ldots,N$ (см., например,~[8]). Поэтому практический алгоритм поиска такой функции
$I^{(m)}_{h^*}(\theta_i)$ описывается следующим образом.
\begin{enumerate}[1.]
\item Выбираем распределение $h$ из множества возможных распределений~$H$ и 
находим функцию $I^{(m)}_{h}(\theta_l),$ $l\hm=1,\ldots,N$.
\item Полагаем $l=1$.
\item Проверяем, выполняется ли условие
$$\abs{I^{(m)}_{h}(\theta_l)-J^{(m)}_1(\theta_l)}\leq\eps
\fr{2^{2m+1}}{2m+1}\left(8+\fr{4}{\pi}\ln m\right)
$$
или
$$
\abs{I^{(m)}_{h}(\theta_l)-J^{(m)}_2(\theta_l)}\leq\eps
\fr{2^{2m+1}}{2m+1}\left(8+\fr{4}{\pi}\ln m\right),$$
где $J^{(m)}_1(\theta_l)$ и $J^{(m)}_2(\theta_l)$ вычисляются по формуле~(\ref{e2-she}).
\item Если условие не выполнено, то исключаем распределение~$h$
из множества возможных распределений $H$ и переходим к шагу~1.
Если условие выполнено и $l\hm\neq N$, то полагаем $l\hm=l+1$ и
переходим к шагу~3. Если же условие выполнено и $l\hm=N$, то алгоритм
завершает работу и полагаем
$I^{(m)}_{h^*}(\theta_l)\hm=I^{(m)}_{h}(\theta_l),$ $l=1,\ldots,N$.
При этом распределение~$h^*$ и определяет группы сферических проекций, 
соответствующие каждому состоянию случайной функции.
\end{enumerate}

Описанный метод является более трудоемким, чем метод, предложенный в
предыдущем разделе, но при этом в случае наличия погрешностей он дает
более точное приближение функций моментов, что позволяет надеяться
на более точное восстановление состояний случайной функции.

{\small\frenchspacing
{%\baselineskip=10.8pt
\addcontentsline{toc}{section}{Литература}
\begin{thebibliography}{99}

\bibitem{3-she} %1
\Au{Louis A.\,K., Quinto E.\,T.} Local tomographic methods in
Sonar~// Surveys on solution methods for inverse problems.~---
Vienna: Springer, 2000. P.~147--154.

\bibitem{2-she}
\Au{Finch  D., Patch S., Rakesh}. Determining a function from
its mean values over a family of spheres~// SIAM J.~Math. Anal.,
2004. Vol.~35. No.\,5. P.~1213--1240.

\bibitem{1-she} %3
\Au{Ambartsoumian G., Kuchment P.} On the injectivity of the
circular Radon transform arising in thermoacoustic tomography~//
Inverse Probl., 2005. Vol.~21. P.~473--485.
{\looseness=1

}



\bibitem{4-she}
\Au{Agranovsky M.\,L., Quinto E.\,T.} Injectivity sets for the
Radon transform over circles and complete systems of radial
functions~// J.~Funct. Anal., 1996. Vol.~139. P.~383--413.

\bibitem{5-she}
\Au{Liu W., Frank J.} Estimation of variance distribution in
three-dimensional reconstruction. I.~Theory~// J.~Opt. Soc. Am. A,
1995. Vol.~12. P.~2615--2627.

\bibitem{6-she}
\Au{Ушаков В.\,Г., Ушаков Н.\,Г.} Восстановление вероятностных
характеристик многомерных случайных функций по проекциям~// Вестн.
Моск. ун-та. Сер. 15: Вычисл. матем. и киберн., 2001. №\,4.
C.~32--39.
{\looseness=1

}

\bibitem{7-she}
\Au{Shestakov O.\,V.} An algorithm to reconstruct
probabilistic distributions of multivariate random functions from
the distributions of their projections~// J.~Math. Sci., 2002.
Vol.~112. No.\,2. P.~4198--4204.

\bibitem{8-she}
\Au{Натансон И.\,П.} Конструктивная теория функций.~--- М.-Л.:~ГИТТЛ, 1949. 684~с.

\bibitem{9-she}
\Au{Norton S.\,J.} Reconstruction of a two-dimensional
reflecting medium over a circular domain: Exact solution~// 
J.~Acoust. Soc. Amer., 1980. Vol.~67. P.~1266--1273.

\bibitem{10-she}
\Au{Kunyansky L.} Explicit inversion formulas for the
spherical mean Radon transform~// Inverse Probl., 2007. Vol.~23.
P.~373--383.

\bibitem{11-she}
\Au{Finch D., Haltmeier M., Rakesh}. Inversion of spherical
means and the wave equation in even dimensions~// SIAM J.~Appl.
Math., 2007. Vol.~68. No.\,2. P.~392--412.
\end{thebibliography} } }



\end{multicols}

\hfill{\small\textit{Поступила в редакцию 03.03.13}}


\vspace*{18pt}

\hrule

\vspace*{2pt}

\hrule

\def\tit{INVERSION OF SPHERICAL RADON TRANSFORM IN THE CLASS OF DISCRETE RANDOM FUNCTIONS}

\def\aut{O.\,V.~Shestakov$^{1,2}$, M.\,G.~Kuznetsova$^1$, and~I.\,A.~Sadovoy$^1$}

\def\titkol{Inversion of spherical radon transform in the class of discrete random functions}

\def\autkol{O.\,V.~Shestakov, M.\,G.~Kuznetsova, and I.\,A.~Sadovoy}


\titel{\tit}{\aut}{\autkol}{\titkol}

\vspace*{-9pt}

\noindent
$^1$Department of Mathematical Statistics, Faculty of Computational Mathematics and
Cybernetics,\\
$\hphantom{^1}$M.\,V.~Lomonosov Moscow State University, Moscow 119991, Russian
Federation\\
\noindent$^2$Institute of Informatics 
Problems, Russian Academy of Sciences, Moscow 119333, Russian Federation\\

\vspace*{12pt}

\def\leftfootline{\small{\textbf{\thepage}
\hfill INFORMATIKA I EE PRIMENENIYA~--- INFORMATICS AND APPLICATIONS\ \ \ 2013\ \ \ volume~7\ \ \ issue\ 4}
}%
 \def\rightfootline{\small{INFORMATIKA I EE PRIMENENIYA~--- INFORMATICS AND APPLICATIONS\ \ \ 2013\ \ \ volume~7\ \ \ issue\ 4
\hfill \textbf{\thepage}}}


\Abste{The paper deals with the problem of reconstructing the probabilistic 
distributions of random functions from distribution of spherical projections that
describe the images in certain types of tomographic experiments, including 
optoacoustic tomography, thermoacoustic tomography, and radiolocation. 
The problems of this kind arise when the object under study may randomly change 
its structure during the registration of the projection data and the time within 
which its structure changes radically is considerably smaller than the time of 
registration of a required number of projections. In such cases, the conventional 
tomographic approach cannot be used directly. The authors assume that a random object may 
have at most countable set of structural states which are described by integrable 
functions with compact support. For such discrete class of random functions, the 
uniqueness of solution is proved and the reconstruction method is developed which is 
based on the properties of the so-called moments of projections. It is shown that the 
developed method is stable and gives adequate results when the projection data are 
corrupted by noise.}

\KWE{random functions; spherical Radon transform; stochastic tomography}

\DOI{10.14357/19922264130408}

\Ack
\noindent
The research was financially supported by the Ministry of Education
and Science of the Russian Federation (State Contract No.\,14.740.11.0996).

%\pagebreak

  \begin{multicols}{2}

\renewcommand{\bibname}{\protect\rmfamily References}
%\renewcommand{\bibname}{\large\protect\rm References}

{\small\frenchspacing
{%\baselineskip=10.8pt
\addcontentsline{toc}{section}{References}
\begin{thebibliography}{99}

\bibitem{3-she-1} %1
\Aue{Louis, A.\,K., and E.\,T.~Quinto.} 2000.
Local tomographic methods in Sonar. \textit{Surveys on solution methods for inverse problems}.
Vienna: Springer. 147--154.



\bibitem{2-she-1}
\Aue{Finch,  D., S.~Patch, Rakesh}. 2004. Determining a function from
its mean values over a family of spheres. \textit{SIAM J. Math. Anal.}
35(5):1213--1240.

\bibitem{1-she-1} %3
\Aue{Ambartsoumian, G., and P.~Kuchment} 
2005. On the injectivity of the circular Radon transform arising in thermoacoustic 
tomography. \textit{Inverse Probl.} 21:473--485.




\bibitem{4-she-1}
\Aue{Agranovsky, M.\,L., and E.\,T.~Quinto.} 1996.
Injectivity sets for the Radon transform over circles and complete 
systems of radial functions. \textit{J.~Funct. Anal.} 139:383--413.

\bibitem{5-she-1}
\Aue{Liu, W., and J.~Frank.} 1995. 
Estimation of variance distribution in three-dimensional reconstruction. I.~Theory.
\textit{J.~Opt. Soc. Am. A} 12:2615--2627.

\bibitem{6-she-1}
\Aue{Ushakov, V.\,G., and N.\,G.~Ushakov.} 
2001.
Vosstanovlenie veroyatnostnykh kharakteristik mnogomernykh slu\-chay\-nykh funktsiy po 
proekciyam 
[Reconstruction of probabilistic characteristics of multivariate random functions from 
projections]. \textit{Vestn. Mosk. un-ta. Ser.~15. Vychisl. matem. i kibern}. 
[\textit{Bulletin of Moscow University. Computational mathematics and cybernetics
ser.}] 4:32--39.


\bibitem{7-she-1}
\Aue{Shestakov, O.\,V.} 2002.
An algorithm to reconstruct probabilistic distributions of multivariate random functions from the
distributions of their projections. \textit{J.~Math. Sci.} 112(2):4198--4204.

\bibitem{8-she-1}
\Aue{Natanson, I.\,P.} 1949.
\textit{Konstruktivnaya teoriya funktsiy} 
[\textit{Constructive theory of functions}]. Moscow--Le\-nin\-grad: GITTL. 684~p.

\bibitem{9-she-1}
\Aue{Norton, S.\,J.}  1980.
Reconstruction of a two-dimensional reflecting medium over a circular domain: Exact solution.
\textit{J.~Acoust. Soc. Amer.} 67:1266--1273.

\bibitem{10-she-1}
\Aue{Kunyansky, L.}  2007.
Explicit inversion formulas for the spherical mean Radon transform.
\textit{Inverse Probl.} 23:373--383.



\bibitem{11-she-1}
\Aue{Finch, D., M.~Haltmeier, Rakesh}. 2007. Inversion of spherical
means and the wave equation in even dimensions. \textit{SIAM J.~Appl.
Math.} 68(2):392--412.
\end{thebibliography}
} }


\end{multicols}

\hfill{\small\textit{Received March 3, 2013}}

\Contr

\noindent
\textbf{Shestakov Oleg V.} (b.\ 1976)~--- Doctor of Science in physics and mathematics, 
assistant professor, Department of Mathematical Statistics, Faculty of 
Computational Mathematics and Cybernetics, M.\,V.~Lomonosov Moscow State 
University, Moscow 119991, Russian Federation; senior scientist, Institute of Informatics Problems, Russian 
Academy of Sciences, Moscow 119333, Russian Federation; oshestakov@cs.msu.su

\vspace*{3pt}

\noindent
\textbf{Kuznetsova Maria G.} (b.\ 1991)~--- student, Department of Mathematical Statistics, 
Faculty of Computational Mathematics and Cybernetics, M.\,V.~Lomonosov Moscow 
State University, Moscow 119991, Russian Federation; m.g.kuznetsova@gmail.com

\vspace*{3pt}

\noindent
\textbf{Sadovoy Ivan A.} (b.\ 1990)~--- student, Department of Mathematical Statistics, 
Faculty of Computational Mathematics and Cybernetics, M.\,V.~Lomonosov Moscow 
State University, Moscow 119991, Russian Federation; isadovoy@gmail.com

 \label{end\stat}


\renewcommand{\bibname}{\protect\rm Литература}