\vspace*{-60pt} %{ %small
{ %\baselineskip=9.1pt
\section*{Правила подготовки рукописей  для публикации в журнале
<<Информатика и её применения>>}

\vspace*{12pt}

\thispagestyle{empty}

Журнал <<Информатика и её применения>> 
публикует теоретические, обзорные и дискуссионные статьи, 
посвященные научным исследованиям и разработкам в области 
информатики и ее приложений. 

Журнал издается на русском языке. По специальному решению 
редколлегии отдельные статьи могут печататься на английском языке. 

Тематика журнала охватывает следующие направления: 
\begin{itemize}
\item теоретические основы информатики;\\[-13.5pt] 
      \item
математические методы исследования сложных систем и процессов;\\[-13.5pt] 
           \item
информационные системы и сети;\\[-13.5pt] 
                \item
информационные технологии;\\[-13.5pt] 
                     \item
архитектура и программное обеспечение вычислительных комплексов и сетей.\\[-13.5pt] 
\end{itemize}


\noindent
\begin{enumerate}[1.]
\item В журнале печатаются статьи, содержащие результаты, ранее не опубликованные и 
не предназначенные к одновременной публикации в других изданиях.  

Публикация не должна нарушать закон об авторских правах.  

Направляя рукопись в редакцию, авторы сохраняют все права собственников данной 
рукописи и при этом передают учредителям и редколлегии неисключительные права на 
издание статьи на русском языке (или на языке статьи, если он отличен от рус\-ско\-го) и на 
ее распространение в России и за рубежом. Авторы должны пред\-ста\-вить в редакцию 
письмо в следующей форме: 

{\bfseries\textit{Соглашение о передаче права на публикацию:}}

\noindent
<<\textit{Мы, нижеподписавшиеся, авторы рукописи <<\ldots>>, передаем учредителям 
и редколлегии журнала <<Информатика и её применения>> неисключительное 
право опубликовать данную рукопись статьи на русском языке как в печатной, так и в 
электронной версиях журнала. Мы подтверждаем, что данная публикация не нарушает 
авторского права других лиц или организаций.} 
 
\textit{Подписи авторов: (ф. и. о., дата, адрес)>>.}
  
 
Это соглашение может быть представлено в бумажном виде или в виде отсканированной 
копии (с подписями авторов).  
 

Редколлегия вправе запросить у авторов экспертное заключение о возможности 
пуб\-ли\-ка\-ции пред\-став\-лен\-ной статьи в открытой печати.\\[-13.5pt]  

\item К статье прилагаются данные автора (авторов) (см.\ п.~8). При наличии нескольких 
авторов указывается фамилия автора, ответственного за переписку с редакцией.\\[-13.5pt]  

\item Редакция журнала осуществляет экспертизу присланных статей в соответствии с 
принятой в журнале процедурой рецензирования.

Возвращение рукописи на доработку не означает ее принятия к печати.  

Доработанный вариант с ответом на замечания рецензента необходимо прислать в 
редакцию.\\[-13.5pt]  

\item Решение редколлегии о публикации статьи или ее отклонении сообщается авторам.  

Редколлегия может также направить авторам текст рецензии на их статью. Дискуссия по 
поводу отклоненных статей не ведется.\\[-13.5pt]  

%\pagebreak

\item Редактура статей высылается авторам для просмотра. Замечания к редактуре должны 
быть присланы авторами в кратчайшие сроки.\\[-13.5pt]  

\item Рукопись предоставляется в электронном виде в форматах MS WORD (.doc или 
.docx) или \LaTeX\  (.tex), дополнительно~--- в формате .pdf, на дискете, лазерном диске 
или электронной почтой. Предоставление бумажной рукописи необязательно.\\[-13.5pt] 

\item При подготовке рукописи в MS Word рекомендуется использовать следующие 
настройки.

Параметры страницы:  
формат~--- А4; ориентация~--- книжная; поля (см): внутри~--- 2,5, снаружи~--- 1,5, 
сверху~--- 2, снизу~--- 2, от края до нижнего колонтитула~--- 1,3.  

Основной текст: стиль~--- <<Обычный>>, шрифт~--- Times New Roman, размер~--- 
14~пунк\-тов, абзацный отступ~--- 0,5~см, 1,5~интервала, выравнивание~--- по ширине.  
 
 
 \thispagestyle{empty}

Рекомендуемый объем рукописи~--- не свыше 20~страниц указанного формата.  

Сокращения слов, помимо стандартных, не допускаются. Допускается минимальное 
количество аббревиатур. 

Все страницы рукописи нумеруются. 

Шаблоны примеров оформления представлены в Интернете: {\sf 
http://www.ipiran.ru/journal/\linebreak template.doc}\\[-13.5pt]  

\item Статья должна содержать следующую информацию на {\bfseries\textit{русском и 
английском языках}}: 
\begin{itemize}
\item название статьи;
\item Ф.И.О.\ авторов, на английском можно только имя и фамилию;\\[-13.5pt] 
\item место работы, с указанием почтового адреса организации и электронного адреса каждого 
автора;\\[-13.5pt]  
\item сведения об авторах, в соответствии с форматом, образцы которого 
представлены на страницах: 

\thispagestyle{empty}

{\sf http://www.ipiran.ru/journal/issues/2013\_07\_01\_rus/authors.asp} и 

{\sf http://www.ipiran.ru/journal/issues/2013\_07\_01\_eng/authors.asp}; 
\item аннотация (не менее 100~слов на каждом из языков). Аннотация~--- это краткое 
резюме работы, которое может публиковаться отдельно. Она является основным 
источником информации в информационных системах и базах данных. Английская 
аннотация должна быть оригинальной, может не быть дословным переводом русского 
текста и должна быть написана хорошим английским языком. В~аннотации не должно 
быть ссылок на литературу и, по возможности, формул;
\item ключевые слова~--- желательно из принятых в мировой 
на\-уч\-но-тех\-ни\-че\-ской литературе тематических тезаурусов. Предложения не 
могут быть ключевыми словами.
\end{itemize}

%\pagebreak

\item  Требования к спискам литературы.\\[-13.5pt]  

Ссылки на литературу в тексте статьи нумеруются (в квадратных скобках) и 
располагаются в каждом из списков литературы в порядке  первых упоминаний.

Списки литературы представляются в двух вариантах: 
\begin{enumerate}[(1)]
\item \textbf{Список литературы к русскоязычной части}. Русские и английские 
работы~---  на языке и в алфавите оригинала;\\[-13.5pt]  
\item  \textbf{References}. Русские работы и работы на других языках~--- в латинской 
транслитерации с переводом на английский язык; английские работы и работы на других 
языках~--- на языке оригинала. 
\end{enumerate}

Рекомендуется для составления списка ``References'' пользоваться размещенной на сайте 
{\sf http:// translit.ru/} (опция BGN) бесплатной программой транслитерации русского текста в 
латиницу. 

Список литературы ``References'' приводится полностью отдельным блоком, повторяя все 
позиции из списка литературы к русскоязычной части, независимо от того, имеются или 
нет в нем иностранные источники. Если в списке литературы к русскоязычной части есть 
ссылки на иностранные публикации, набранные латиницей, они полностью повторяются в 
списке ``References''.

Ниже приведены примеры ссылок на различные виды публикаций в списке ``References''. 

\thispagestyle{empty}

{\small

\noindent
\textbf{Описание статьи из журнала:}

\Aue{Zagurenko, A.\,G., V.\,A.~Korotovskikh, A.\,A.~Kolesnikov, A.\,V.~Timonov, and D.\,V.~Kardymon}. 2008. 
Tekhniko-ekonomicheskaya optimizatsiya dizayna gidrorazryva plasta [Technical and
economic optimization of the design 
of hydraulic fracturing]. \textit{Neftyanoe hozyaystvo} [\textit{Oil Industry}] 11:54--57.

\Aue{Zhang, Z., and D.~Zhu}. 2008. Experimental research on the localized 
electrochemical micromachining. \textit{Rus. J.~Electrochem.}  44(8):926--930. 
{\sf doi:10.1134/S1023193508080077}.

\noindent
\textbf{Описание статьи из электронного журнала:}

\Aue{Swaminathan, V., E.~Lepkoswka-White, and B.\,P.~Rao}. 1999. Browsers or buyers in cyberspace? An 
investigation of electronic factors influencing electronic exchange. \textit{JCMC} 
5(2). Available at: {\sf http://www.ascusc.org/jcmc/vol5/issue2/} (accessed April~28, 2011).




\noindent
\textbf{Описание статьи из продолжающегося издания (сборника трудов):}

\Aue{Astakhov, M.\,V., and T.\,V.~Tagantsev}. 2006. Eksperimental'noe 
issledovanie prochnosti soedineniy ``stal'--kompozit'' [Experimental study of 
the strength of joints ``steel--composite'']. \textit{Trudy MGTU 
``Matematicheskoe modelirovanie slozhnykh tekh\-ni\-che\-skikh sistem''} 
[\textit{Bauman MSTU ``Mathematical Modeling of Complex Technical 
Systems'' Proceedings}]. 593:125--130.

\pagebreak

\noindent
\textbf{Описание материалов конференций:}

\Aue{Usmanov, T.\,S., A.\,A.~Gusmanov, I.\,Z.~Mullagalin, R.\,Ju.~Muhametshina, A.\,N.~Chervyakova, and 
A.\,V.~Sveshnikov}. 2007. Osobennosti proektirovaniya razrabotki mestorozhdeniy 
s primeneniem gidrorazryva 
plasta [Features of the design of field development with the use of hydraulic fracturing]. 
\textit{Trudy 6-go 
Mezhdu\-na\-rod\-no\-go Simpoziuma ``Novye resursosberegayushchie tekhnologii nedropol'zovaniya i povysheniya 
neftegazootdachi''} [\textit{6th  Symposium (International) ``New Energy Saving Subsoil Technologies and 
the Increasing of the Oil and Gas Impact'' Proceedings}]. Moscow. 267--272.

\thispagestyle{empty}



\noindent
\textbf{Описание книги (монографии, сборники):}



Lindorf, L.\,S., and L.\,G.~Mamikoniants, eds. 1972. 
\textit{Ekspluatatsiya turbogeneratorov s neposredstvennym 
okhlazhdeniem} [\textit{Operation of turbine generators with direct cooling}]. 
Moscow: Energy Publs. 352~p.


\Aue{Latyshev, V.\,N.} 2009. \textit{Tribologiya rezaniya. Kn.~1: Friktsionnye protsessy 
pri rezanii metallov} 
[\textit{Tribology of cutting. Vol.~1: Frictional processes in metal cutting}]. Ivanovo: Ivanovskii 
State Univ. 108~p.

\noindent
\textbf{Описание переводной книги}
(в списке литературы к русскоязычной части необходимо указать:~/ Пер.\ с англ.~--- 
после названия книги, а в конце ссылки указать оригинал книги в круглых скобках):
\begin{enumerate}[1.]
\item  В русскоязычной части:

\thispagestyle{empty}

\Au{Тимошенко С.\,П., Янг Д.\,Х., Уивер~У.} 
Колебания в инженерном деле~/ Пер.\ с англ.~--- М.: Машиностроение, 1985. 472~с. 
(\Au{Timoshenko~S.\,P., Young~D.\,H., Weaver~W.} 
Vibration problems in engineering.~--- 4th ed.~--- N.Y.: Wiley, 1974. 521~p.) 
\item  В англоязычной части:

\Aue{Timoshenko, S.\,P., D.\,H.~Young, and W.~Weaver}. 
1974. \textit{Vibration problems in engineering}. 4th ed. N.Y.: Wiley. 521~p. 
\end{enumerate}


\noindent
\textbf{Описание неопубликованного документа:}

\Aue{Latypov, A.\,R., M.\,M.~Khasanov, and V.\,A.~Baikov}. 
2004. Geology and production (NGT GiD). Certificate on official registration of the computer program 
No.\,2004611198. (In Russian, unpubl.)

\noindent
\textbf{Описание интернет-ресурса:}


Pravila tsitirovaniya istochnikov [Rules for the citing of sources]. Available at: {\sf 
http://www.scribd.com/doc/1034528/} (accessed February~7, 2011).

%\pagebreak

\noindent
\textbf{Описание диссертации или автореферата диссертации:}

\Aue{Semenov, V.\,I.}
2003. Matematicheskoe modelirovanie plazmy v sisteme kompaktnyy tor [Mathematical 
modeling of the plasma in the compact torus]. D.Sc.\ Diss. Moscow. 272~p.

\Aue{Kozhunova, O.\,S.} 2009. Tekhnologiya razrabotki semanticheskogo 
slovarya informatsionnogo monitoringa [Technology of development of 
semantic dictionary of information monitoring system]. PhD Thesis. Moscow: IPI RAN. 23~p.


\noindent
\textbf{Описание ГОСТа:}

GOST 8.586.5-2005. 2007. Metodika vypolneniya izmereniy. Izmerenie raskhoda i kolichestva zhidkostey i gazov s 
pomoshch'yu standartnykh suzhayushchikh ustroystv [Method of measurement. 
Measurement of flow rate and volume of liquids and gases by means of orifice devices]. Moscow: 
Standardinform  Publs. 10~p.

\noindent
\textbf{Описание патента:}

\Aue{Bolshakov, M.\,V., A.\,V.~Kulakov, A.\,N.~Lavrenov, and M.\,V.~Palkin}. 
2006. Sposob orientirovaniya po krenu letatel'nogo 
apparata s opti\-che\-skoy golovkoy 
samonavedeniya [The way to orient on the roll of aircraft with optical homing head]. 
Patent RF No.\,2280590.
}

\item Присланные в редакцию материалы авторам не возвращаются.\\[-13.5pt]  

\item При отправке файлов по электронной почте просим придерживаться следующих 
правил: 
\begin{itemize}
\item указывать в поле subject (тема) название журнала и фамилию автора;\\[-13.5pt] 
\item использовать attach (присоединение);\\[-13.5pt] 
\item в состав электронной версии статьи должны входить: файл, содержащий текст 
статьи, и файл(ы), содержащий(е) иллюстрации.\\[-13.5pt] 
\end{itemize}

\item Журнал <<Информатика и её применения>> является некоммерческим изданием. 
Плата за публикацию не взимается, гонорар авторам не выплачивается. 
\end{enumerate}

\thispagestyle{empty}


%\vspace*{10mm}

\begin{center}

\textbf{Адрес редакции журнала <<Информатика и её применения>>:} \\




Москва 119333, ул.~Вавилова, д.~44, корп.~2, ИПИ РАН\\[-10pt]

\

Тел.: +7\,(499)\,135-86-92\ \ Факс:  +7\,(495)\,930-45-05\\[-10pt]
 
 \

e-mail:   {\sf rust@ipiran.ru} (Сейфуль-Мулюков Рустем Бадриевич)\\[-10pt]

\

{\sf http://www.ipiran.ru/journal/issues/}
\end{center}
}