\renewcommand{\bibname}{\protect\rmfamily References}
\renewcommand{\figurename}{\protect\bf Figure}
%\renewcommand{\tablename}{\protect\bf Table}

\def\stat{kalin}

\def\tit{CONCEPTUAL DECLARATIVE PROBLEM SPECIFICATION AND~SOLVING IN~DATA INTENSIVE
DOMAINS$^*$}

\def\titkol{Conceptual declarative problem specification and~solving in~data intensive
domains}

\def\autkol{L.~Kalinichenko, S.~Stupnikov, A.~Vovchenko,
and~D.~Kovalev}

\def\aut{L.~Kalinichenko$^1$, S.~Stupnikov$^1$, A.~Vovchenko$^1$,
and~D.~Kovalev$^1$}

\titel{\tit}{\aut}{\autkol}{\titkol}

{\renewcommand{\thefootnote}{\fnsymbol{footnote}}
\footnotetext[1] {This research has been done under the support of the RFBR (project 11-07-00402-а) and the
Program for Basic Research of the Presidium of the Russian Academy of Sciences.}}

\renewcommand{\thefootnote}{\arabic{footnote}}
\footnotetext[1]{Institute of Informatics Problems, Russian Academy of Sciences, Moscow 119333, 
Russian Federation} 


%\vspace*{6pt}

\def\leftfootline{\small{\textbf{\thepage}
\hfill INFORMATIKA I EE PRIMENENIYA~--- INFORMATICS AND APPLICATIONS\ \ \ 2013\ \ \ volume~7\ \ \ issue\ 4}
}%
 \def\rightfootline{\small{INFORMATIKA I EE PRIMENENIYA~--- INFORMATICS AND APPLICATIONS\ \ \ 2013\ \ \ volume~7\ \ \ issue\ 4
\hfill \textbf{\thepage}}}



\Abste{Various notations aimed at defining the semantics of a computation in
terms of the application domains have been experienced for conceptual
modeling. For example, entity-relationship (ER) approach and UML (Unified Modeling Language)
diagrams allow one to
specify the semantics informally. Ontology languages based on description
logic (DL) have been developed to formalize the semantics of data. However, it is
now generally acknowledged that data semantics alone are insufficient and still
representation of data analysis algorithms is necessary to specify data and
behavior semantics in one paradigm. Moreover, the curse of ever increasing
diversity of multistructured data models gave rise to a need for their unified,
integrated abstraction to make specifications independent of real data in data
intensive domains (DID).
To overcome these disadvantages, a novel approach for applying a combination
of the semantically different declarative rule-based languages (dialects) for
interoperable conceptual specifications over various rule-based systems (RSs)
relying on the logic program transformation technique recommended by the
W3C (World Wide Web Consortium) Rule Interchange Format (RIF) has been investigated. Such approach is
coherently combined with the specification facilities aimed at the semantic
rule-based mediation intended for the heterogeneous data base integration. The
infrastructure implementing the multidialect conceptual specifications by the
interoperable RSs and mediating systems (MSs) is introduced. The
proof-of-concept prototype of the infrastructure based on the SYNTHESIS
MS and RIF standard is presented. The approach for multidialect
conceptualization of a problem domain, rule delegation, rule-based programs
and mediators interoperability is explained in detail and illustrated on a real
nondeterministic polynomial time (NP) complete use-case in the finance domain. The research results are promising
for the usability of the  approach and of the infrastructure for conceptual,
declarative, resource independent and reusable data analysis in various
application domains.}

\KWE{conceptual specification; W3C RIF; logic rule languages; SYNTHESIS;
database integration; mediators; RIF-BLD; RIF-CASPD; multidialect
infrastructure; rule delegation}

\DOI{10.14357/19922264130412}


\vskip 16pt plus 9pt minus 6pt

      \thispagestyle{myheadings}

      \begin{multicols}{2}

            \label{st\stat}

\section{Introduction}

  \noindent
For decades, informatics is investigating proper facilities for conceptualization of the diverse
application domains, descriptive and predictive data analysis to extract information from
data, decision making, and generally, for the problem solving based on such insight. Many
times a point was reached when the level of maturity of the conceptual modeling techniques
could have been evaluated as satisfactory from the theoretical and practical points of view.
But every time new technological developments and research results disproved such
optimistic opinions. Now, in the epoch of the data intensive research\footnote[2]{Data Intensive
Research (DIR) has emerged as a new paradigm for inferring information from data (known
as the 4th paradigm for scientific domains~\cite{2-kal}) emphasizing the increasing value of
observational, experimental, and computer-generated data in virtually all domains. Similarly,
``Big Data'' is developing as a recognition of the increasing significance of massive
multistructured data in almost each area of the human activity.} and system development, a lack
of the proper declarative conceptual modeling facilities is felt again. For instance, the call
for such facilities has been expressed in the ACM (Association for Computing Machinery)
publication ``Challenges and
Opportunities with Big Data''~\cite{1-kal}. In particular, the call considers important to examine a
combination including
  \begin{itemize}
\item facilities for support of variety of data (heterogeneity of data types,
representation, and semantic interpretation) and semantic data integration; and\\[-16pt]
\item declarative specifications required to meet the programmability and composition
of complex analytical pipelines in an understandable form applying appropriate
high-level languages to express the analytics.
\end{itemize}

  This paper is related to such statements investigating a novel methodology and
infrastructure supporting conceptually-driven problem specifications and solving in
the DID. This research is motivated by aiming at the
specification reusability in various domains over different sets of information
resources (IRs), widely diverse data, and knowledge semantic integration capability as well
as at the accumulation of reproducible data analysis and problem solving methods,
related specifications, and experience in various DID.

  One of the intended outcomes of the work is to expose current practically
reachable limit of declarative conceptual specification construction applying the
wealth of various available facilities of logic programming, knowledge representation,
semantic Web, and heterogeneous database mediation in a coherent, cooperative way.

  Conceptual specifications should be expressed in terms of an application domain
independently of design and implementation concerns. Usually, the domain
specification understandable and acceptable in the respective community includes a
set of coherently tied ontologies together with a conceptual schema of the domain
defined on the basis of such ontologies. Conceptual specifications of the applications
in this domain are defined in terms of the conceptual schema to express abstract
representations of information structures and algorithms of their analysis for the
application problem solving.

  This research reflects an intention to define and implement facilities for
conceptually-driven problems specification and solving in DID to create the
perspectives ensuring the following capabilities for expressing the specifications:
  \begin{itemize}
\item declarative nature of the language facilities involved;
\item an ability to provide complete and precise specification of the abstract structure
and behavior of the domain entities, their consistency, relationship, and interaction;
\item well-grounded diversity of semantics of the modeling facilities providing for
proper expressiveness, compactness, and precision of the definition of behavior of the
domain entities and problem solving algorithm specifications;
\item arrangements for the probable extensions of the modeling facilities for
conceptual specifications satisfying the changing technological and practical needs;
\item provisions for specification independence from implementation languages and
systems;
\item provisions for specification independence from specific IRs
combined with facilities for their semantic integration and interoperability coping
with the increasing diversity of the techniques used for implementation of programs,
databases, services, ontologies, and other resources;
\item built-in methodologies for creation of unifying specification languages
providing for construction of semantic preserving mappings of conceptual
specifications into their implementations in specific languages and systems; and
\item combination of ease of a human perception with a sufficient efficiency and
expressiveness of the implementation for publication in communities intended for
specification reuse.
\end{itemize}

  The brief vision of the state-of-the-art in conceptual modeling (not claimed to be
exhaustive) is as follows. Various notations aimed at defining the semantics of a
computation in terms of the application problems to be solved have been experienced
for conceptual modeling for many years. Structured and object-oriented specification
notations for conceptual modeling in graphical (diagram) form dominate.
  Entity-relationship approach models the domain in terms of entities, attributes,
and relationships, extended sometimes with cardinality constraints, aggregation, sets,
temporality, taxonomic structures. Many parts of UML are the diagrams for specific
purposes (e.\,g., class diagrams, activity diagrams, use cases, state charts, object
interaction diagrams) used for the domain modeling and the design phase in software
engineering. Such notations often allow one to specify the semantics informally, in
terms of natural language; or the semantics is hard-coded using the tools supporting
the notation~\cite{3-kal}. Various logic specification languages, such as ontology languages
based on DL, have been developed to describe the semantics of data.
With certain limitations, the ontology languages allow one to define formal models of
domain concepts, their taxonomic relations, and constraints over them. They facilitate
automated reasoning (inference) over taxonomic structures. However, it is now
generally acknowledged that data semantics alone are insufficient and representation
of behavior is necessary to specify data and behavior semantics, algorithms for data
analysis in one paradigm.
{\looseness=1

}

  This is just what paper is focused on, emphasizing the importance of declarative
nature of the specification facilities. In particular, the approach proposed is aimed at
conceptual modeling related to DID applying rule-based declarative languages
possessing different, complementary semantics and capabilities combined with the
methods for heterogeneous data mediation and integration.

  In the work presented, the issues of interoperability of declarative programs defined
in the languages having different semantics combined with an ability to integrate
various IRs (such as data and knowledge bases, software services,
ontologies) for problem solving are investigated on the basis of two fundamental
techniques:\begin{enumerate}[(1)]
\item constructing of the unifying extensible language providing for
semantic preserving representation in it of various IR
specification languages (e.\,g., such as data definition (DDL) and manipulation (DML)
languages for databases);  and
\item creation of
the unified extensible family of rule-based languages (dialects) and a model of
interoperability of the programs expressed in such dialects with different semantics.
\end{enumerate}

  The first technique is based on the experience obtained in course of the
SYNTHESIS language development~\cite{4-kal, 5-kal} accompanied by the methods and
facilities for constructing its extensions. The kernel of the SYNTHESIS language is
based on the object-frame data model used together with the declarative rule-based
facilities in the logic language similar to a stratified Datalog with functions and
negation. The extensions of the SYNTHESIS kernel are constructed in such a way
that each extension together with the kernel is a result of semantic preserving
mapping of some IR language (mainly, a data model language considered to be a
combination of the respective DDL and DML) into the SYNTHESIS~[6, 7]. The
canonical information model (CIM) is constructed as a union of the kernel with such
extensions constructed for various IR languages. Canonical information model
is used for development of
mediators positioned between the users, conceptually formulating problems in terms
of the mediators, and various distributed IRs (such as databases and services) needed
for a specific application. A~schema of a subject mediator for a class of problems
includes the specification of the domain concepts defined by the respective
ontologies.

Another, multidialect technique for rule-based programs interoperability applied in the current
work is based on the RIF  standard~[8] of W3C. Rule interchange format introduces a
unified family of rule-based languages (dialects) together with a methodology for
constructing semantic preserving mappings of specific languages used in various
 RSs\footnote{Examples of RSs include SILK, OntoBroker, DLV, IBM ILOG,
IRIS, etc. More examples can be found at {\sf
http://www.w3.org/ 2005/rules/wiki/Implementations}.} into such dialects. Besides,
a methodology for extension of the family with the definition of new dialects is also defined
as a part of RIF. From the RIF point of view, an IR is a program developed in a specific
language of some RS. For the logic dialects, the RIF family should include declarative,
high-level languages reflecting in a systematic way the basic semantics of rule-based languages
of various existing logic-based RSs. These RSs and their declarative rule-based languages
form now a mature technology with decades of theoretical development, practical and
commercial use based on logic programming and nonmonotonic reasoning.

  The approach proposed here is built on the techniques briefly introduced above
aiming at construction of declarative specifications containing definitions of the
concepts and their relationships in the DID area (conceptual schema), conceptual
definitions of problems, and their respective algorithms in declarative form
simplifying their reuse. The \textit{conceptually-driven} problem specifications of an
application can be defined applying advanced declarative languages, modularization,
and composition facilities. Thus, the problem specifications can be considered also as
domain-driven, that is, they are defined in terms of the domain independently of
IRs (databases, services, workflows, mediators) and
implementation notations.

  The paper presents the results obtained including the description of an approach
and an infrastructure supporting:
  \begin{itemize}
 \item application domain conceptual specification and problem solving
algorithms\footnote{In the approach proposed, the logic programming model of
computation is assumed.} definitions based on the combination of the
heterogeneous database mediation technique and rule-based multidialect facilities;
\item interoperability of distributed multidialect rule-based programs and mediators
integrating heterogeneous databases; and
\item rule delegation approach for the peer interactions in the multidialect
environment.
\end{itemize}

  The proof-of-concept prototype of the infrastructure based on the SYNTHESIS
environment and RIF standards has been implemented. The approach for multidialect
conceptualization of a problem domain, rule delegation, rule-based programs, and
mediators interoperability is explained in detail and illustrated on a real NP-complete
use-case in the finance domain. For the conceptual definition of the use-case problem,
Web Ontology Language (OWL) is used for the domain concepts definition and programs in two dialects
  RIF-BLD (Basic Logic Dialect) and RIF-CASPD
  (Core Answer Set Programming Dialect~\cite{10-kal}) mapped into the SYNTHESIS formula language and
answer set programming (ASP) based DLV~\cite{9-kal} language, respectively.

  The paper\footnote{The approach presented in this paper has been reported for the
first time at the ADBIS'2013 conference and annotated in a short
publication~\cite{10-kal}.} is structured as follows. After introduction, the
sections containing an overview of the approach and an infrastructure for distributed
multidialect rule-based programs support are given. In section~4, the lessons
learned during the use-case specification and implementation are summarized.
A~related work section, the future plans section that outlines the intentions for the
future research, and the conclusion which summarizes the results end the paper. The
detailed description of the use-case and its implementation is given in an appendix.
Such detailed exposition of the results is justified by the novelty of the approach and a
necessity to show and make readable all the steps required for the specification and
implementation of the use-case applying the languages and the infrastructure
proposed.

%\vspace*{-4pt}

    \section{An Approach for~Conceptual Problem Specification
and Solving Based on~the~Techniques for~Heterogeneous
Database Mediation and~Multidialect Program
Interoperability}

%\vspace*{-2pt}

  According to the above, the problems of integration of various IRs and
interoperability of rule-based programs arising during application development are
proposed to be resolved on the combination of two techniques:
  \begin{enumerate}[(1)]
\item construction of the unifying extensible language providing for representation
in it of various IR languages preserving their semantics and for creation of
mediators for integration of heterogeneous IRs;  and
\item creation of the unified family of languages (dialects) for expressing the
problem solving algorithms and the model of interoperability of programs represented
in such dialects.
\end{enumerate}

\vspace*{-8pt}

\subsection{Heterogeneous Database Mediation}

\vspace*{-2pt}

  \noindent
  In this research, the first technique is based on the results of the SYNTHESIS
language project~\cite{4-kal} including the methods and means for the development
of this language extensions and for its use for database integration.

  The \textit{Unifier of information models}~\cite{11-kal} is the main facility
created for the development of the SYNTHESIS language extensions and the
mappings of specific data models into such extensions. Preserving semantics of a
specific data model by its mapping into SYNTHESIS in the Unifier is based on the
specification \textit{refinement} principle (it is said that specification~$A$ refines
specification~$B$ if it is possible to use~$A$ instead of~$B$ so that a user of~$B$
does not notice such substitution) developed by Abrial~\cite{12-kal}. Thus, it is
possible to construct provably correct extensions of the SYNTHESIS language and
mappings of specific languages into them based on the proof of refinement. Such
methodology has been applied for mapping into SYNTHESIS of diverse database
languages, service and process specification languages, ontological
languages~[5--7, 11]. Now, a possibility of mapping into
SYNTHESIS of various multistructured data models typical for the Big Data epoch
(such as, e.\,g., various NoSQL models~\cite{13-kal}, graph-based
  models~\cite{14-kal}, triplet data model~\cite{15-kal}, and array data
  model~\cite{16-kal}) is studied.

The main characteristic properties of the SYNTHESIS language use as the kernel of the
extensible language are as follows:
  \begin{enumerate}[(1)]
  \item  the merge of the kernel extensions constructed for diverse data models is
treated as a single language known as the CIM; and
  \item CIM is considered as a basis for the development of mediators positioned
between the programs providing for problem solving in terms of mediators (the
application domain terms) and specific distributed IRs (such as database, services)
selected for the problem solving. The mediator specification for a class of problems
includes definitions of the application domain concepts expressed by the respective
declarative specifications, specifications of the classes of the domain entities,
specifications of types of the instances of the classes mentioned and of their methods, and
specification of functions.
  \end{enumerate}

  A mediator in SYNTHESIS can be defined as a triple ($T$, $S$, $M$) where $T$
is the mediator (target) schema; $S$ is the IR schema; and $M$ is the set of assertions
(rules) relating elements of the target schema with elements of the resource schema. It
is assumed that IR data model (that is, the respective DDL and DML) has been
mapped into the canonical one; thus, both resource and target schemas are defined in
the canonical data model. Both resource and canonical data models may allow for the
expression of various constraints~$\Sigma$. In this generality, a data integration
setting is ($S$, $T$, $\Sigma_{r}$, $\Sigma_{{rt}}$, $\Sigma_{t}$) in which $S$ and
$\Sigma_r$ constitute the IR schema, $T$ and $\Sigma_t$ form the target schema, and
the resource-to-target dependencies included into $\Sigma_{{rt}}$ are the assertions of
the schema mappings in the mediator. $\Sigma_{{rt}}$ should be defined as a set of
assertions of weakly-acyclic class~\cite{7-kal, 17-kal}. Note that $\Sigma_r$ includes
not only data dependencies corresponding to the IR database in the original data
model, but also the dependencies generated in accordance with the extension of the
kernel of the canonical data model determining an information-preserving mapping of
the resource data model into the canonical one (some categories of such dependencies
are shown in~\cite{6-kal, 7-kal}).

  Specifications of mediators are expressed in CIM. Generally, a specification
of~$M$ in the approach oriented on the definition of the mediator schema~$T$ in
terms of the application domain independently of schemas of IRs involved is based on
the GLAV (Global and Local As View)~\cite{18-kal, 34-kal} schema mapping technique. In GLAV, two schema
mapping techniques are combined: LAV (Local as View) according to which a
schema of IR is defined as a view over the mediator schema and GAV (Global as
View) according to which the mediator schema is defined  as a global view over the
IRs schemas. In LAV, the schemas of IR classes are considered as materialized views
over the virtual classes of the mediator. Usually, GAV views are used for resolving
various conflicts between the specifications of IRs and of the mediator. In such cases,
the GAVs provide the rules for transformation of the results of a query over IRs
to their representation in the mediator. The GLAV technique provides for stability of the
mediator specification in time of the specific IRs change or the change of the actual
IR presence (remove of an IR, adding of new resources, and so on). Also, the GLAV
technique provides for scalability of mediators with respect to the number of IRs
involved in them. It worth mentioning that the mediators, in their turn, can play the role of
IRs towards the mediators of the higher level.

  The following characterization of the GLAV setting is applied in the SYNTHESIS
project. The sound GLAVs are assumed~\cite{18-kal}. The GLAV mapping is
constructed as a composition of two schema mappings~--- GAV mapping $M_{{rg}} =
(S_r, S_{g}, \Sigma_{{rg}})$ (from resource schemas to GAV schema) and LAV mapping
$M_{{gt}} = (S_{g}, T_{{rt}}, \Sigma_{{gt}})$ (from GAV schema to mediator (target)
schema). GLAV is formed by a definition of the composition $M_{{rg}}\circ M_{{gt}}$ by
means of the GAV and LAV dependencies ($\Sigma_{{rg}}$ and $\Sigma_{{gt}}$,
respectively). The GLAV mappings are formed showing how two parts of the schema
(resource and target) are linked together by the~$\Sigma_{{rt}}$ dependencies.

  Queries and programs over the mediator are expressed declaratively in the
SYNTHESIS formula language (resembling Datalog) in terms of the mediator
schema~$T$~\cite{4-kal}. An example of the mediator definition can be found in
appendix (see subsections~A3.1 and~A3.2).

\subsection{Diverse Rule-Based Programs Interoperability}

  \noindent
  Another fundamental principle used in this research is based on a multidialect technique
intended for providing diverse rule-based programs interoperability and known as the
W3C RIF standard~\cite{8-kal}. The range of intended
applications of the RIF standard (besides the Semantic Web) includes also
development of the intelligent information systems as well as knowledge
representation in various application domains. The family of dialects proposed in RIF
has a common core (the \textit{Core dialect}) and an ensemble of the dialects
extending the core and forming the directed acyclic graph. An arc of the graph is
directed from the dialect being extended to the dialect extending it.

  According to RIF, any specific RS can act in two independent roles~--- the role of
\textit{provider} and the role of \textit{consumer}. The first one indicates that RS is
capable to transform its native rule-based programs into the programs in the
respective dialect; the second one indicates that RS is capable to accept a program
defined in a proper dialect and transform it into the program in the native RS
language. Thus, to include any rule-based language into the family of interoperable
dialects, it is required to provide the respective RS with two semantic preserving
transformers~--- from its own language into the proper dialect (the role of provider) and
from the proper dialect into its own language (the role of consumer). In contrast to the
development of the SYNTHESIS language extensions, each dialect extending other
dialects may have its own, different of others, semantics. For example, for logic rule-based
languages, each dialect can use its own semantics (e.\,g., the classic first-order
semantics), well-founded semantics, stable model semantics, etc.~\cite{14-kal}.
Unlike the SYNTHESIS language, in which for development of the language
extensions and respective semantic preserving language mappings, it is required to
establish the refinement relation between the specifications, in RIF for the logic
rule-based languages, it is required to establish the \textit{entailment} relation between the
specifications. The entailment relation between formulas $f$ and $g$, $f\vert =g$ ($f$ entails~$g$)
is valid if and only if $g$ is true for all models
where~$f$ is true.

  In the RIF standard, only the basic, simple dialects have been fixed. Thus,
  RIF-BLD~\cite{20-kal} corresponds to Horn logic with
various syntactic and semantic extensions. RIF-PRD (Production Rules
Dialect)~\cite{21-kal} specifies a production rules dialect to enable the interchange of
production rules. RIF-PRD captures the main aspects of various production rule
systems. Production rules are defined using \textit{ad hoc} computational mechanisms, which
are not based on a logic. The condition language of RIF-PRD is defined as a common
subset of RIF-BLD and RIF-PRD. RIF-Core (Core Dialect) is a subset of both
  RIF-BLD and RIF-PRD thus enabling limited rule exchange between logic rule
dialects and production rules.

  Significant contribution of RIF consists in inclusion into the standard of the
Framework for Logic Dialects RIF-FLD~\cite{19-kal}. RIF-FLD represents quite
general logic language including the broad set of syntactic and semantic constructs
frequently used in logic. On the basis of the framework, concrete dialects having
specific semantics can be defined. Definitions of all dialects in the standard are
strictly formalized. Rule interchange format defines the general conception of the unified language
construction as a family of dialects intended for provision of the interoperability of
the numerous RSs. On the basis of the RIF-FLD framework, several
dialects have been designed in the world. They need still to pass through the
procedure of approval by W3C. Among these dialects are the following. RIF-CLPWD
(Core Logic Programming Well-founded Dialect~\cite{23-kal}) uses WFS~\cite{25-kal}
with the default negation and functions.
  RIF-CASPD~\cite{22-kal}
uses the semantics of ASP, known also as the stable model
semantics~\cite{26-kal}. Difference of semantics and application areas of WFS and
ASP is a typical example showing different capabilities of the rule-based programs
expressible in the languages with such semantics. The WFS and ASP language classes
correspond to two philosophically different approaches in logic programming: WFS
keeps the idea of defining a single (well-founded) model for a program while ASP provides
for the computation of the set of preferred models that is more than query answering.
Thus, the ASP-based systems are specifically oriented on solving the complex,
combinatorial (NP-complete) problems. The ASP application areas include: diagnosis,
information integration, constraint satisfaction, reasoning about actions (including
planning), routing, and scheduling, health care, biomedicine and biology, text mining
and classification, question answering, etc. In contrast to the ASP systems, the
WFS-based systems incorporate three-valued logic are computationally complete and
can be used as the universal logic programming facilities. The joint usage of programs
with different semantics for problem solving is obviously reasonable.

The   RIF standard has also defined the necessary concepts to ensure compatibility of
RIF with resource description framework (RDF)
and OWL~\cite{27-kal}, in spite of dissimilarity of their syntaxes and
semantics. Rule interchange format uses its frame syntax to communicate with RDF/OWL. These frames
are mapped onto RDF triples and a joint semantics is defined for the combination.
The basic idea is that rules and OWL will view each other as ``black boxes'' with
well-defined interfaces through exported predicates. The OWL-based ontologies will
export some of their classes and properties, while rule-based knowledge bases will
export some of the predicates that they define. Each type of the knowledge base will
be able to refer to the predicates defined in the other knowledge bases and treat them
extensionally as collections of facts. Ontologies are assumed to provide commonly
shared conceptualization of a domain. There may be different application-specific rule
programs for different applications in the domain.

  In the use-case considered in the paper, the following languages for conceptual
specifications are used: OWL~2, RIF-BLD and RIF-CASPD (the examples of
specifications in these languages can be found in the appendix, section~A2).

\subsection{Basic Ideas of~the~Conceptual Problem Solving Approach Built
on~the~Symbiosis of~the~Above}

  \noindent
  The idea of the conceptual problem solving approach considered in the paper
consists in developing a modular infrastructure in which alongside with the
mediators integrating data and services in the SYNTHESIS-like setting, the modules
representing knowledge and declarative rule-based programs over various resources
will be introduced. The infrastructure is based on the following principles:
  \begin{itemize}
\item \textit{the multidialect construction of the conceptually-driven problem
specifications   combining the SYNTHESIS and RIF approaches.} The specifications
are represented as a functional composition of declarative modules, each based on its
own language (dialect) with an appropriate semantics. Semantics of a conceptual
definition in such setting becomes a multidialect one. To the SYNTHESIS language,
one of the existing RIF dialects (or a new, specific dialect) should be put into
correspondence  (currently, OWL is used for the conceptual schema specification
together with the RIF-BLD for the mediator rules to be interpreted in SYNTHESIS);
\item \textit{the specification modules are treated as peers.} Rule-based program
modules are included into the specification alongside with the mediators that can be
supported by various MSs. Interoperability of such combination of modules is based
on P2P (peer-to-peer) and W3C RIF interoperability techniques;
\item \textit{combination of integration and interoperabilit}y. The
IR integration can be provided in the scope of individual mediator modules.
Interoperability is provided between the modules supporting different dialects
according to the RIF methodology; and
\item \textit{rule-based specifications on different levels of the infrastructure}.
Rule-based, inference providing modules can be used for declarative programming
over the mediators. In this case, the mediators support schema mapping for semantic
integration of the IRs supported by various MSs.
\end{itemize}




  In the paper, the ideas of construction of the infrastructure providing for conceptual
declarative programming and heterogeneous IR mediation
interoperating in the multidialect environment are presented. 

The infrastructure
generally outlined in Fig.~1 provides for implementation of conceptual specifications
on the basis of the CIM-like kernel and its extensions used in the mediators combined
with declarative languages of the rule-based programming systems possessing
diverse semantics and capabilities. Such environment provides for\linebreak\vspace*{-12pt}

\pagebreak

\end{multicols}

\begin{figure} %fig1
   \vspace*{1pt}
 \begin{center}
 \mbox{%
 \epsfxsize=92.944mm
 \epsfbox{kali-1.eps}
 }
 \end{center}
 \vspace*{-3pt}
\Caption{Infrastructure for conceptual problem solving}
%\vspace*{6pt}
  \end{figure}

  \begin{multicols}{2}

  \noindent
   programming over
heterogeneous resources (e.\,g., over databases and services) accessed through the
mapping of specific resource schemas into the target schemas of the mediators.
Multidialect program is implemented by means of transformation of conceptual
specifications into modular, component-based P2P program represented on the
languages of the mediation and rule-based systems. Interaction of
components of such distributed program is carried out by means of the delegation
technique (see subsection~3.3). 

An approach for transformation of conceptual specifications
into the distributed mediators and modules interpreted in RSs with
different semantics interoperating in P2P style by the rule delegation in the
  multidialect W3C RIF-based environment is presented in section~3.
  {\looseness=1
  
  }

  Practical value of the results obtained is demonstrated with a use-case in the
finance domain. The important result of the research conducted consists also in
accumulation of the experience of usage of the multidialect specification facilities for
\textit{conceptually-driven problem specifications} in specific domains and
implementation of such specifications in the prototype that has been developed.

\section{Infrastructure of~the~Multidialect Environment
for~Distributed Rule-based Programs and~Mediators
Interoperability}

\subsection{Conceptual Programming over~the~Conceptual Schema}

  \noindent
  The aim of the novel infrastructure proposed is a conceptual programming of
problems in the RIF dialects and an implementation of conceptual specifications using
declarative languages of the RSs and MSs. Conceptual
  multidialect logic programs specify the algorithms for problem solving in a
subject domain. They are implemented using their transformation into RS or MS
programs.

  \textit{Conceptual specification of a problem} (class of problems) is defined in the
context of a subject domain and consists of a set of \textit{RIF-documents} (document
is a specification unit of RIF). Each document contains the groups of rules. Conceptual
specification of a problem is an abstract program of a problem solving.

  The subject domain conceptualization is performed using OWL~2~\cite{28-kal}
ontologies containing entities of the domain and their relationships (Fig.~2,
  right-hand part). Thus, the \textit{conceptual schema} of the domain can be formed.
Conceptual specification is defined over conceptual schema: names of the entities (classes and attribites)
and only they are used in the rules of the RIF-documents as extensional predicates. 
  
  Ontologies are imported into the RIF-documents specifying an
import profile, for instance, \textit{OWL Direct}. A~profile defines a semantics of an
OWL. The profiles are formally defined in~\cite{27-kal}.

  Modular construction of the conceptual specification is based on the techniques of
document import and link. Each RIF-document is defined using a dialect which is a
\textit{specialization} of the RIF Framework for Logic Dialects
  (RIF-FLD)~\cite{19-kal}. Documents \textit{import} other documents having the
same semantics (the \textit{Import} directive) or \textit{link} documents defined
using other dialects and having different semantics (remote module directive
\textit{Module}). 

Documents refer to predicates from imported documents (using the
name of an imported module \textit{module:predicate}) as well as to predicates from
remote modules using \textit{remote terms f@r}~\cite{19-kal}. Here, $f$ means a
term and $r$ means a reference to a remote module.

  Retrieving results of problem solving is performed by querying the virtual
knowledge/database formed by combined infrastructure of mediators and rule-based
systems. 

Queries are formulated in terms of the conceptual schema and the intentional
predicates of the conceptual specification of a problem. The result is a Boolean value
(if the formula is closed) or a collection of tuples of values of free variables of the
formula.



\subsection{Mapping of the Peer Schemas into~the~Conceptual Specification}

  \noindent
  In the proposed infrastructure, (1)~the logic programs implementing
  RIF-documents of the conceptual schema in specific RSs and (2)~the mediators
supporting virtual collections of facts as the result of heterogeneous databases
integration are considered as peers.

  A schema $S_R$ of a peer~$R$ is a set of entities (classes or relations and their
attributes) corresponding to extensional and intensional predicates of the resources.

  The RS or the MS of each peer~$R$ should be a
\textit{conformant} $D_R$ \textit{consumer} where $D_R$ is the respective RIF dialect
(Fig.~2, left-hand part). Conformance is formally defined using formula entailment
and language mappings~\cite{19-kal}.

  The peer~$R$ is \textit{relevant} to a RIF-document~$d$ of a conceptual
specification of a problem (see Fig.~2, right-hand part) if:
  \begin{itemize}
\item $D_R$ is a subdialect of the document~$d$ dialect; and
\item entities of the peer schema~$S_R$ (if they exist) are \textit{ontologically
relevant}\footnote{In this paper,  the methods for ontological matching
of the schemas~\cite{15-kal} are not considered.} to entities of the conceptual schema the names of
which are used in $d$ for extensional predicates.
\end{itemize}

  The schema of a relevant peer is mapped into the conceptual schema. The mapping
establishes the correspondence of the conceptual entities referenced in the
document~$d$  to their expressions in terms of entities of the schema~$S_R$ using
logic rules of the $D_R$  dialect. These schema mapping rules constitute separate
RIF-document (Fig.~2, middle part). An example of mapping is given in appendix
(see subsection~A3.1).

\end{multicols}

\begin{figure}[h] %fig2
   \vspace*{1pt}
 \begin{center}
 \mbox{%
 \epsfxsize=162.976mm
 \epsfbox{kali-2.eps}
 }
 \end{center}
 \vspace*{-6pt}
\Caption{Conceptual and peer specifications}
\end{figure}



%\noindent
\begin{figure}[b] %fig3
   \vspace*{1pt}
 \begin{center}
 \mbox{%
 \epsfxsize=165.582mm
 \epsfbox{kali-3.eps}
 }
 \end{center}
 \vspace*{-6pt}
\Caption{Peer-to-peer multidialect architecture}
  \end{figure}

\begin{multicols}{2}

\subsection{Implementation of the Conceptual Specification}

\noindent
  Programs defined in the conceptual specification are implemented in P2P
environment formed by relevant peers which are related to conceptual specification
by mapping rules (see Fig.~2, middle part).

  Peers communicate using a technique for distributed execution of the logic
programs. The basic notion of the technique is \textit{delegation}~--- transferring
facts and rules from one peer to another. A~peer is installed on a \textit{node} of the
multidialect infrastructure. A~node is a combination of a wrapper, an RS or an MS,
and a peer (Fig.~3). A~wrapper transforms programs and facts from the specific RIF
dialect into the language of the RS or MS and \textit{vice versa}. A~wrapper also implements
the delegation mechanism. A~definition of the delegation is given in the latter part of
this subsection. Transferring facts and rules among peers is performed in RIF dialects.
The wrappers implement an interface \textit{RIFNodeWrapper} (see Fig.~3).

  A special component (\textit{Supervisor}) of the proposed architecture stores
shared information of the environment, i.\,e., conceptual specifications related to the
domain and to the problem, a list of the relevant resources, RIF-documents combining
logic rules for the conceptual specification, and a resource schema mapping.



Implementation of the conceptual specification includes the following steps:
\begin{itemize}
\item rewriting the conceptual documents into the RIF-programs of the peers
performed by the \textit{Supervisor}. A~rewriting includes also (1)~replacing the
document identifiers (used to mark predicates) by peer identifiers; and (2)~adding
schema mapping rules to programs (see Fig.~2, middle part);
\item a transfer of the rewritten programs to nodes containing peers relevant to the
respective conceptual documents. The transfer is performed by the \textit{Supervisor}
by calling the method \textit{loadRules} of the respective node wrappers; 
\item a transformation of the RIF-programs into the concrete RS or MS languages.
The transformation is performed by the \textit{NodeWrapper} or by the RS or MS
itself (if the RS or MS supports the respective RIF dialect); and
\item an execution of the produced programs in P2P environment.
  \end{itemize}

  During the process of rewriting the conceptual schema into the resource
programs, the relationships between RIF-documents of the conceptual schema defined
by remote or imported terms are replaced by relationships between peers also defined
by remote or imported terms. To implement remote and imported terms, a \textit{rule
delegation} mechanism is used~--- transferring facts and rules from one peer to
another. The details of rule delegation approach implemented in the current prototype
of the multidialect infrastructure are provided below.

  In general case, the programs transferred to some peer may include
\textit{nonlocal} rules. These rules contain remote or imported terms. On the
contrary, \textit{local} rules do not contain such terms. For simplicity, only remote
terms are mentioned in the latter part of this subsection. To make possible an execution
of a program on a node, the program should be \textit{normalized}, i.\,e., transformed
into an equivalent program including only local rules and \textit{delegation rules}.
There are two kinds of delegation rules for a peer~$n$:
  \begin{enumerate}[(1)]
\item \textit{fact delegation rule} like $p@m(X) :\mbox{-}\ q@n(X)$ where $p$ and~$q$
are the predicate names; $X$ is the variable list; and $m$ is the
peer different from~$n$. The rule
means that all the facts turning~$q$ into \textit{true} have to be transferred to the
peer~$m$ as facts turning~$p$ into \textit{true}. A~fact is a term $p(v_1,\ldots , v_n)$
such that $p(v_1,\ldots , v_n) = true$ where $v_1,\ldots , v_n$ are the values of
variables~$X = x_1, \ldots , x_n$; and
\item \textit{delegation rule} like $q@n(X) :\mbox{-} p@m(X)$. The rule has to be
transferred to the peer~$m$ where it becomes a fact delegation rule.
\end{enumerate}

  The procedure of \textit{normalization} (Algorithm~1) of a program~$pr$ on a
peer~$n$ is as follows. For each occurrence of a remote term in a rule~$r$, a fact
delegation rule is added to~$pr$ aimed at importing facts from another peer to the
peer~$n$ or at exporting facts from~$n$ to another peer. The rule includes a new
local predicate created for storing imported or exported facts. Each occurrence of a
remote term in a rule~$r$ is replaced by an occurrence of the respective local
predicate. In that way, the rule~$r$ is transformed into a local rule.

  \smallskip

\noindent
  \textbf{Algorithm 1.} $Normalization(pr, n)$

\noindent
\textbf{Input:} $pr$ is a program~--- a set of rules; $n$ is a peer name

\noindent
\textbf{Output:} normalized program $pr$

\noindent
\textbf{for each} rule $r \in pr$ {like} $head(r):\!\mbox{-}\ body(r)$.
\textbf{do}

  \hspace*{1mm}\textbf{if} the $head(r)$ of $r$ is a remote term $p@m(X)$
\textbf{then}

  \hspace*{3mm}// remove $r$ from $pr$

  \hspace*{3mm}$pr \leftarrow pr \backslash\{r\}$

  \hspace*{3mm}// add new fact delegation rule,

  \hspace{3mm}// here, $p\_m$ is a new local predicate

  \hspace*{3mm}$pr \leftarrow pr \cup \{ p@m(X):\!\mbox{-}\ p_m(X).\}$

  \hspace*{3mm}// add new local rule

  \hspace*{3mm}$pr \leftarrow pr \cup \{ p\_m(X) :\!\mbox{-}\ body(r)\}$

  \hspace{1mm}\textbf{elseif}\ the $head(r)$ is a local predicate \textbf{and}

       \hspace*{3mm}the $body(r)$ contains an occurrence
       
       \hspace*{3mm}of a remote  term $q@l(Y)$

  \hspace*{1mm}\textbf{then}

  \hspace*{2mm}\textbf{for each} remote term $q@l(Y)$ 

  \hspace*{14mm}contained in the $body(r)$ 
  
  \hspace*{2mm}\textbf{do}

  \hspace*{6mm}// add new fact delegation rule

\hspace*{6mm}// here, $q\_l$ is a new local predicate

  \hspace*{6mm}$pr \leftarrow pr \cup \{ q\_l(Y):\!\mbox{-}\ q@l(Y).\}$

  \hspace*{6mm}// replace the occurrence of the remote term in~$r$

\hspace*{6mm}// by a local term $q\_l(Y)$

  \hspace*{6mm}$r \leftarrow head(r) :\!\mbox{-}\ body(r)[q@l(Y)\rightarrow
q\_l(Y)].$

  \hspace*{3mm}\textbf{end for}

  \hspace*{1mm}\textbf{endif}


\noindent
\textbf{end for}

\noindent
  \textbf{return} $pr$

  \smallskip

  The algorithm of execution of a program on a node containing a peer~$n$ looks as
follows (Algorithm~2). The program stored on the peer is normalized. Delegation
rules produced during normalization are transferred to the respective peers. After that,
the algorithm waits for the facts from all the peers to which the delegation rules were
transferred. During waiting, some delegation rules can be obtained from other peers
(no mutual recursion between programs sent to different peers is presumed). When all
facts are got, they become a part of the peer. Then, local RIF-rules of the peer are
transformed into the RS or MS rule language and executed in the RS or MS. The last
step consists in transferring the required facts to other peers in accordance with the
fact delegation rules.

\noindent
  \textbf{Algorithm 2.} $Execute(n)$


\noindent
\textbf{Input:} $n$ is a peer name;

$prog(n)$ is a program stored at $n$ at the moment

(additional rules can be loaded into $prog(n)$

during execution of $Execute$)

  \noindent
  \textbf{Output:} $result$~--- a set of tuples which are the result
  
  \hspace*{1mm}of the executing program

\noindent
\textbf{Local variables}

$prn$~--- a variable to store a normalized program

$pre$~--- a variable to store local normalized rules

$pre1$~--- a variable to store a program transformed 

\hspace*{5mm}into the RS or MS rule language

\noindent
 $prn \leftarrow Normalization(prog(n), n)$

  \noindent
  \textbf{for each} fact delegation rule $r\in prn$ 
  
  like $q\_l(Y):\!\mbox{-}\ q@l(Y)$
  
  \noindent 
  \textbf{do}

// transfer rule $q\_l@n(Y):\!\mbox{-}\ q(Y)$ to peer~$l$

$l.loadRule(\mbox{``}q\_l@n(Y):\!\mbox{-}\ q(Y)\mbox{''})$

$pre \leftarrow  pre \backslash \{ r \}$

\noindent
  \textbf{end for}

 \noindent
  \textbf{wait until} for all fact delegation rules, $r\in  prn$ 
  
  \hspace*{4mm}like $q\_l(Y):\!\mbox{-}\ q@l(Y)$

\hspace*{4mm}a respective peer~$l$ has transferred facts

\hspace*{4mm}like $q(v_1, \ldots v_n)$ to peer~$n$ as facts 

\hspace*{4mm}like  $q\_l(v_1, \ldots v_n)$ 

  \noindent
  \textbf{for each} fact delegation rule $r \in prn$ 
  
like $q\_l(Y):\!\mbox{-}\ q@l(Y)$
  
  \noindent
  \textbf{do}

$load$ facts from predicate $q\_l$ into the peer~$n$

  \noindent
  \textbf{end for}

  \noindent
  // put all normalized local rules into $pre$

  \noindent
  $pre \leftarrow \{ r \vert r \in  prn \vee r\ \mbox{is local} \}$

  \noindent
  $transform$ the program $pre$ into the resource
  
  rule language program $pre1$

  \noindent
  $result \leftarrow execute$ the program $pre1$ on the resource 
  
  (mediator or RS)   connected to $n$

  \noindent
  \textbf{for each} fact delegation rule $r \in  prog(n)$
  
   like   $p@m(X):\!\mbox{-}\ q(X)$
   
   \noindent
   \textbf{do}

  \hspace*{1mm}// transfer facts from local predicate~$q$ into remote
  
  
\hspace*{1mm}// predicate~$p$ at the peer~$m$

  \hspace*{1mm}$m.loadFacts(\{ p(x) \vert x \in  q(x)\})$

\noindent
  \textbf{end for}

\noindent
  \textbf{return} $result$

  \smallskip

  A rule delegation mechanism described in this section resembles one proposed in
WebdamLog~\cite{30-kal}.

\section{Lessons Learned During the~Use Case Implementation }

\noindent
  To provide a proof of the multidialect infrastructure concept, a use case in the
financial domain has been implemented. The problem to be solved in the use case is
called \textit{investment portfolio diversification problem}. The detailed description
of the use case is included in the appendix. In this section, the important lessons
learned during the implementation of the use case are summarized.

  During the implementation of the \textit{investment portfolio} use case in the
multidialect infrastructure, the conceptual specifications were defined: a conceptual
schema of the financial domain (using OWL~2) and conceptual specifications of the
problem (using RIF dialects BLD and CASPD). Conceptual specification was defined over conceptual schema:
names of the entities (classes and attributes) of conceptual schema and only they were used in the
rules of the conceptual specification as extensional predicates.
Conceptual specifications were
implemented using a multidialect infrastructure which includes a mediation system
integrating the financial data and an ASP~\cite{26-kal} RS.

  Application specification in the multidialect infrastructure is started with the
informal definition of a problem (see appendix, section~A1). Then, a conceptual schema of a
subject domain is defined using OWL~2 (see section~A2). Due to the
fact that OWL~2 does not provide facilities to specify functions,
it is necessary to model the
functions related to the collections of facts of the domain as OWL classes (e.\,g.,
\textit{correlation}):
  \begin{verbatim}
Class(correlation)
  DataProperty(series1  domain(correlation)
    range(DatedValue))
  DataProperty(series2  domain(correlation)
    range(DatedValue))
  DataProperty(corr  domain(correlation)
    range(xsd:double))
\end{verbatim}

  Input and output parameters of the functions are modeled as data or object
properties of the respective classes (for instance, {\sf series1},
{\sf  series2},   and {\sf corr}). The types of the parameters are modeled as {\sf  range}
of the respective properties.

  \medskip

  \noindent
  \textbf{Lesson~1.} Functions related to the collections of facts in conceptual
schema of the domain have to be modeled as OWL classes.

The classes of the conceptual schema are used as predicates in rules which constitute
documents of the conceptual specification of the problem. Logic rules over taxonomic
structures of the schema may require existential quantifier in their heads
(see section~A2):
\columnbreak

\noindent
  \begin{verbatim}
Exists ?ts( And(?ts#gex:tickers
?ts[symbol->?symbol]) ):-
  And(?t#srt:stockRates  ?t[ticker->?symbol]
      ?t[isInTop500->?inTop500]
      External(pred:boolean-equal(?inTop500
      true))
  )
\end{verbatim}

  The rule states that for each object {\sf ?t} of {\sf stockRates} class there exists an
object {\sf ?ts} of a class {\sf tickers}. A~predicate {\sf e\#s} denotes that the
element~{\sf e} belongs to the set~{\sf s}.

  Built-in predicates (like Boolean equality {\sf pred: boolean-equal}) and functions
are used as \textit{external terms} of RIF. The symbol~{\sf External} indicates that
an atomic formula or a function term is defined externally (neither in the current
  RIF-document nor in the remote RIF-document). Built-in predicates are defined in
a special part of the RIF standard.

  \medskip

  \noindent
  \textbf{Lesson 2.} Conceptual specification of a problem using RIF logic rules over
taxonomic structures may require existential quantifier in the rule heads. The
  RIF-BLD dialect intended for serving as the basic logic dialect does not provide
such facility. Respective extension of RIF-BLD is suggested.

  \medskip

  \noindent
  \textbf{Lesson 3.} Conceptual specification of a problem often requires usage of
built-in predicates and functions as external terms.

  During the application of the RIF-CASPD dialect~\cite{24-kal} for conceptual
specification, it was found out that the ASP programs require a specific facility~---
so-called \textit{weak constraints}~\cite{9-kal}. Such constraints should be satisfied if it
is possible, but their violation does not invalidate the models. For instance, a weak
constraint used in the specification of the portfolio problem looks as follows
(see section~A2):
  \begin{verbatim}
    Forall ?X( :~ prt:nonPortfolio(?X))
\end{verbatim}

  The rule states that only such stable models that minimize the truth set of the
predicate {\sf nonPortfolio} are considered as a result of the logic program (that
includes the mentioned rule). The difference of the rule with a {\sf strong constraint}
like
  \begin{verbatim}
    Forall ?X( :- prt:nonPortfolio(?X))
\end{verbatim}
consists in that in a program with the strong constraint the predicate {\sf
nonPortfolio(?X)} has to be turned to \textit{false} for all~{\sf ?X}. Weak constraint
requires the set of all possible values of~{\sf ?X} that turns {\sf nonPortfolio(?X)}
into \textit{true} to be of minimum size.

  \medskip

  \noindent
  \textbf{Lesson 4.} Conceptual specification using ASP-like dialects requires a
specific kind of logic rules~--- \textit{weak constraint}. The RIF-CASPD dialect
intended for serving as the dialect supporting ASP does not provide such facility.
Respective extension of RIF-CASPD is sug\-gested.

  To implement the conceptual schema, a mediator intended to provide the collection
of facts implied by the schema was created. The mediator integrates the respective
financial data. The schema of the mediator (see subsection~A3.2) was created by mapping
 the conceptual schema into the CIM (the SYNTHESIS language~\cite{40-kal}).
The classes of the conceptual schema were mapped into types and/or classes of the
mediator schema, including their attributes. It appeared that some of the attributes
(e.\,g., \textit{isInTop500}) of the conceptual schema taking into account
implementation reasons better to interpret by the methods of the mediator schema. It
appeared also that the definition of \textit{isInTop500} as a method on the mediator
layer made it simpler to establish semantic mappings (GLAV views) between the
mediator and the resources.

  Another peculiarity of the conceptual schema implementation worth mentioning
here is that the classes of the conceptual schema (namely, {\sf correlation},
mentioned earlier in this section) can be mapped into a function of the mediator
schema:
  \begin{verbatim}
{ correlation; in: function;
    params: {+s1/{set; type_of_element:
    DatedValue;},
          +s2/{set; type_of_element:
          DatedValue;}, -corr/real };
};
  \end{verbatim}
  
%  \vspace*{-6pt}

  \noindent
  \textbf{Lesson~5.} The attributes of the conceptual schema can be mapped into
attributes or functions of the schemas of the mediators implementing the conceptual
specifications. The classes of the conceptual schema can be mapped into functions of the
schemas of the mediators.
{\looseness=1

}

  The mapping of conceptual schema into the mediator schema is formally specified
using schema mapping rules constituting a separate RIF-document as shown in
Fig.~3, middle part. These rules are similar to GAV views which represent classes of
the conceptual schema as views over the mediator schema. This allows one to consider the
schema mapping rules just as a consistent part of the logic program that have to be
executed in a peer (e.\,g., in a mediator).

  The document containing schema mapping rules refers to the conceptual schema
and to the mediator schema. Both schemas are identified by URIs:
  \begin{verbatim}
Import(<http://synthesis.ipi.ac.ru/
        optimalSecurityPortfolio>
        <http://www.w3.org/ns/entailment/
         OWL-Direct>)
Prefix(srt <http://synthesis.ipi.ac.ru/
            optimalSecurityPortfolio#>)
Prefix(fs <http://synthesis.ipi.ac.ru/
            FinanceServices#>)
  \end{verbatim}
  
\vspace*{-9pt}

%\columnbreak


  Entities of the mediator schema are referred in the schema mapping rules using
external terms of the RIF, for instance, {\sf External(?t\#fs:stockRates)}
(see subsection~A3.1). This means that the mediator schema is neither OWL-ontology or
RDF-document nor RIF-document and possesses its own semantics.

  Mapping of the \textit{correlation} class of the conceptual schema into the
correlation function of the mediator schema is defined using a rule with a frame
predicate in the head and functional predicate in the body (see subsection~A3.1):
  \begin{verbatim}
And(?c#correlation
   ?c[?series1->series1 ?series2->series2
   ?corr->corr]) :-
External(pred:numeric-equal(
   fs:correlation(?series1 ?series2) ?corr))
\end{verbatim}

\vspace*{-2pt}

\noindent
  \textbf{Lesson~6.} Entities of the peer (mediator) schemas have to be referred in
schema mapping rules using external terms of the RIF.

  \smallskip

  \noindent
  \textbf{Lesson 7.} Logic programs defined in the native languages of RS or MS
look more elegant and concise than the respective programs defined using
  RIF-dialects. The present authors consider this as a price paid for providing the interoperability of
multidialect programs with different semantics. More elegant notation to apply RIF
for conceptual specifications should be investigated.

   It appeared that the mediation system used does not support all the features of the
RIF-BLD (terms with named arguments, conjunctions of atoms in the head of a rule,
and some others). DLV system also does not support all the features of the CASPD
dialect (terms with named arguments, conjunctions of atoms in the head of a rule,
frame terms, subclass terms, membership terms, and some others). These systems can
be the consumers only for the subsets of the respective dialects.

  \smallskip

  \noindent
  \textbf{Lesson 8.} According to RIF, one could have expected that 
  the rule-based programs represented in a specific RIF-dialect should be 
  independent on the RS (or MS) consumer languages at least in the class of RSs (or MSs) 
  conforming to this dialect. For instance, any consumer in the class of the ASP 
  systems should be considered as conformant to the CASPD dialect. Noncompliance with 
  such condition makes the conceptual modeling in the RIF dialects difficult: 
  it is required to choose a specific RS (or MS) in advance and to adjust the dialect 
  to be used appropriately (as it was shown in Lessons~2 and~4 or in the observation 
  above showing that it is unlikely that an RS or MS would support all the features 
  of some standard dialect). Practically, it means that at least the RIF-dialect 
  standards should include instructions for the dialect adjustment making possible 
  to produce appropriate subdialects or superdialects at the early stage of the 
  conceptual design.

%\vspace*{-6pt}
\section{Related Work}

%\vspace*{-2pt}

  Among various future research issues predicted in~\cite{36-kal} to obtain
reasonable solutions in the next decades, the conceptual modeling has been positioned
as the basis for interoperability and for shareable information services. The solution
proposed in the present paper is tightly related to the issues listed in~\cite{36-kal}.

  Ambiguity and incompleteness of structured and object-oriented notations for
conceptual modeling in graphical form causes the unceasing process of generating
various contributions suggesting formal meaning to the diverse graphical
constructions, enhancing their expressiveness, e.\,g., as in several recent
papers~[34--36]. The present paper suggests more radical formalization
decisions.

  Diverse ontology centered conceptual modeling tools have been developed. Just a
couple of them to mention.

KAON2\footnote{{\sf http://kaon2.semanticweb.org/}.} is an infrastructure for managing
OWL-DL, SWRL (Semantic Web Rule Language), and \mbox{F-Logic} ontologies. Its main feature is its own inference engine.
KAON2 supports the SHIQ(D) subset of OWL-DL, DL-safe subset of the SWRL,
making reasoning decidable~\cite{40-kal}, as well as the
function-free subset of F-Logic. This approach is related to the multidialect architecture considered
here due to a provision of using several different languages treated as ontological notations.
  ICOM~3.0 tool~\cite{41-kal} for conceptual modeling is aimed at the support of
the design of multiple extended ER or UML diagrams with inter- and intramodel
constraints. Reasoning support via DL to help validate the models is
provided. The system is helpful for validating integrated models (support for the
integration of models organized in several ontologies is provided). But it is difficult to
improve such notations as ER or UML completely; they allow only to specify the
semantics informally, in terms of natural language; or the semantics are hard-coded
using the tools supporting the notation.

  Rule-based modeling is applied in various DIS. For example, in biology, rule-based
modeling has increasingly attracted attention due to enabling a concise and compact
description of biochemical systems. Proteins, individual cells, and cell populations
denote different levels of an organizational hierarchy, each of them with its own
dynamics. Multilevel modeling is concerned with describing a system at these
different levels and relating their dynamics. The approach proposed facilitates
developing and maintaining multilevel models that, for instance, interrelate
intracellular and intercellular dynamics~\cite{42-kal}.

  The idea used in the infrastructure prototype for ontology-based access (using
OWL~2) to the mediators integrating heterogeneous databases resembles the idea of
the MASTRO-I system~\cite{43-kal} in which QuOnto engine supporting inference
in the DL-Lite ontologies and conjunctive query answering is positioned above the
federated database. The latter is the result of relational databases integration.

  The rule exchange using RIF-PRD is
discussed in~\cite{44-kal, 45-kal}. In such case, the production rule systems share the
same operational semantics opposed to the present approach studying the problem of rule
exchange between systems with different semantics.

  Several papers on the use of database query languages for specifying declarative
distributed programs and managing data in distributed environment have been
published~[30, 42--44]. In contrast to multidialect approach,
a single declarative language is used in each of the proposed systems. Usually, it is a
conventional Datalog extended with the notion of localization and possibly other
  nondatalog constructs~\cite{48-kal}. In the multidialect approach, localization is
specified with the RIF remote and imported terms.

  Conceptual notion of \textit{delegation} applied in the present approach is similar to the
notion of delegation in Webdamlog defined as the ``possibility of installing a rule at
another peer. In its simplest form, delegation is essentially a remote materialized
view. In its general form, it allows peers to exchange rules, i.\,e., knowledge beyond
simple facts, and thereby provides the means for a peer to delegate work to other
peers''~\cite{16-kal}. Actually, the current implementation supports delegation as a
remote materialized view. Extending the approach for delegation of knowledge is a
future work. The idea of program normalization is similar to the rule localization
rewriting the step described in~\cite{48-kal}.

\vspace*{-6pt}

    \section{Future Plans}
    
    \vspace*{-2pt}

  A number of related issues remain open. Some of them that are planned to be
investigated are listed below.
{ %\looseness=1

}

  \textit{Conceptual specification of processes.} How to define compositions of
algorithmic modules in a process structure (like Petri nets, UML activity diagrams,
the Business Process Modeling) can be found in~\cite{49-kal} where process
modeling semantics, service discovery, mediation, and composition are considered.
Such notations as WSDL-S, SAWSDL, OWL-S, and WSMO
(Web Service Modeling
Ontology) can be applied for that.
For example, OWL-S is a semantic web service description language that borrows
ontological operators from OWL and provides a service upper ontology to describe
web services and service processes in a standard manner.
The WSMO and the combination of UML and
Object Constraints Language (OCL) are viewed as promising alternatives to OWL-S.
Notation WSMO, like OWL, is also based on the first-order logic but employs an object-oriented
style of modeling based on frames. Unlike OWL-S, WSMO includes service
and ontology specification formalisms with well-defined formal semantics, whereas
OWL-S relies on OWL for ontology specifications and on languages such as SWRL
and KIF (Knowledge Interchange Format) for formal semantics. The
present authors plan to investigate an approach for reusable
process specifications in conceptual terms compliant with the infrastructure proposed.
{ %\looseness=1

}

  \textit{Multiple layers of specifications.} Metamodeling as a kind of modeling
across multiple layers uses an ``instance-of'' relationships between layers.
UML/MOF/MDA by OMG is an example of a strict metamodeling hierarchy in
which metamodels introduce higher levels of abstractions for specification and
language concepts. Similar hierarchy can be introduced also in ontologies (e.\,g.,
upper ontologies above the domain ontologies in one and the same language). The
models can be used in a descriptive or prescriptive manner~\cite{50-kal}.
In this paper, the metamodeling issues are not considered
though in future plans, the authors are closer to the
Ontology-based software Development Environment (ODE) approach
  of~\cite{51-kal}.

  \textit{Hybrid specifications.} Conceptual modeling should likely explicitly
involve hybrid aspects of information (i.\,e., the coexistence of formal reasoning and
``informal'' human interactions). Correspondingly, ``hybrid ontologies'' will need to
be modeled and deployed within their supporting social implementation
environments~\cite{52-kal}.

  \textit{Modeling hypotheses in the DIS experiments.} Evaluating the conceptual
model and infrastructure proposed in the paper, the following far-reaching DIS related
ideas are planned to be taken into account~\cite{53-kal}. As the DIS research
evolves, references to previous results and specifications are needed as a source of
provenance data. \textit{In-silico}\footnote{\textit{In silico} is an expression used to
mean ``performed on computer or via computer simulation.'' The phrase was coined
in 1989 as an analogy to the Latin phrases \textit{in vivo}, \textit{in vitro}, and
\textit{in situ}, which are commonly used in biology and refer to experiments done in
living organisms, outside of living organisms, and where they are found in nature,
respectively (refer: Differences between \textit{in vitro}.
\textit{in vivo}, and \textit{in silico} studies. 2013.
Available at: {\sf http://mpkb.org/home/patients/assesing\_literature/in\_vitro\_studies} 
(accessed December~2, 2013)).} experiments must be
supported by a hypotheses model that describes the elements involved in a scientific
exploration and supports hypotheses assessment. Adopting a conceptual specification
perspective to represent hypotheses would allow high-level references to experiments
and provides support for hypotheses evolution. Such hypotheses model would support
scientists in describing, running simulations and interpreting their
  results~\cite{53-kal}. In conjunction with an axiom set specified as rules that
model known facts over the same universe and experimental data, the knowledge
base may contradict or validate some of the sentences in the hypotheses. In the case of
contradictions, the rules that caused the problems must be identified and eliminated
from the theory formed by the hypotheses~\cite{54-kal}. The conceptual specification
model proposed in the present paper looks suitable to such intention providing for a
possibility of hypotheses representation as a set of first-order predicate calculus
sentences applying the multiplicity of the dialects required.

\vspace*{-6pt}

\section{Concluding Remarks}

\vspace*{-2pt}

  The approach presented is an attempt of introducing the multidialect interoperable
conceptual programming over various semantically different rule-based programming
systems relying on the logic program transformation technique recommended by
W3C RIF. It is also shown how to coherently combine such approach with the
heterogeneous data bases integration applying the semantic mediation. Thus, the data
independence of conceptual specifications is provided. The results obtained so far are
quite encouraging for future work aimed at reaching the conceptual specifications
reusability in various applications over different sets of data, as well as for sharing
and accumulation of reproducible data analysis and problem solving methods and
experience in various DID.
%
At the same time, the RIF-dialect standards are suggested to be made more 
flexible including into them the instructions for the dialect adjustment. 
This makes possible to produce appropriate subdialects or superdialects at 
the early stage of the conceptual design (see section~4, Lesson~8).

\vspace*{18pt}

{{\hfill \textbf{APPENDIX A}}}

\vspace*{-6pt}
{\small 
%\section*{Appendix A}
\subsection*{The Use-case for~the~Multidialect
Infrastructure}
\subsection*{A1\ \ Investment Portfolio Diversification\\
\hspace*{20pt}Problem}

  \noindent
  The capabilities of the multidialect architecture are illustrated with the solution of
the \textit{investment portfolio diversification problem}~\cite{31-kal}. The portfolio
is a collection of securities (such as equities or bonds) of companies, and its size is the
number of securities in the portfolio. The task is to build a diversified portfolio of
maximum size. Diversification means that the prices of the securities in portfolio
should be almost independent of each other. If the price of one security falls, it will
not significantly affect the prices of other. Thus, the risk of a portfolio sharp decrease
is significantly reduced.

  The input data for the problem is a set of securities and respective time series of
indicators of the security price for each security. Time series for each security is a set
of pairs $(d, v)$ where $d$ is a date and $v$ is an indicator of the security price (for
instance, closing price).

For a pair of time series $X =\{(d_1, x_1), \ldots , (d_n, x_n)\}$ and
$Y = \{(d_1, y_1), \ldots  ,
(d_n, y_n)\}$, their \textit{correlation} (a measure of similarity) can be calculated. 
The Pearson
correlation\footnote{Refer: \Aue{De Smith,~M.\,J.} 2013. \textit{Statistical analysis handbook}.
Avalable at: {\sf
http://www.statsref.com/HTML/pearson\_product\_\linebreak moment\_correla.html} (accessed December~2, 2013).}  $r_{XY}$
was used between $X$ and~$Y$
$$
r_{XY} = \fr{\sum\limits_{i=1}^n \left( x_i-\overline{x}\right) \left( y_i-
\overline{y}\right)}{\sqrt{\sum\limits_{i=1}^n \left(  x_i-\overline{x}\right)^2
\sum\limits_{i=1}^n \left( y_i-\overline{y}\right)^2}}
  $$
  where $\overline{x}$ and $\overline{y}$ are the means of $x_1, \ldots ,
x_n$ and $y_1, \ldots , y_n$.

  For the diversified portfolio, the securities which time series are noncorrelated
should be used. Noncorrelation of the time series means that their correlation is less
than some predetermined price correlation value. This predetermined value is a
parameter of the portfolio problem. The lower is the correlation~--- the lower is the
risk measure of a sharp decrease of the portfolio value. 

So, the predetermined
correlation serves as the maximum risk measure. In practice, specified correlation
may differ for various types of securities.

  The output data for the problem is a subset of securities of the maximum size, for
which the pairwise correlation will be less than the predetermined one.

%\pagebreak

  The problem is divided into the following tasks:
  \begin{enumerate}[(1)]
\item computation of the security pairwise correlations (for specified dates); and
\item calculation of the maximum satisfying subset of securities.
\end{enumerate}

  To solve the first task, the financial services \textit{Google Finance}\footnote{{\sf
https://www.google.com/finance}} and \textit{Yahoo! Finance}\footnote{{\sf
http://finance.yahoo.com/}} are considered, both of which provide current and
historical information about various securities: stock prices, currencies, bonds, stock
indexes, etc. In particular, the information provided includes various indicators of the
security price for all trading days of last decades.  The mediation system is used to
solve the problem of resource integration~\cite{5-kal}.

 The second task can be formulated as follows. Let~$G$ be a graph where the vertices
are the securities, and an edge between two securities exists if absolute value of their
correlation is less than a specified number. So, any two securities connected by an
edge are considered as noncorrelated. In that case, the problem of finding the
portfolio of the maximum size is exactly the problem of finding a maximum clique in
an undirected graph. Indeed, a clique in a graph is a subset of its vertices such that
every two vertices in the subset are connected by an edge. All the vertices (securities)
in a clique are connected by edges (noncorrelated) and any clique is a candidate for a
portfolio. A~maximal clique is a maximal portfolio.

  Finding a maximum clique in an undirected graph is a well-known NP-complete
problem\footnote{Refer: \Aue{Cormen,~T.\,H., T.~Leisercon, R.\,L.~Rivest, and
C.~Stein}. 2009. NP-complete problems.
\textit{Introduction to algorithms}. 3rd ed. Cambridge: The MIT Press. 1292~p.
}. %A~clique in a
%graph is a subgraph in which each pair of vertices is connected by an edge.
%A~maximum clique is a clique of the maximum size.

  DLV~\cite{9-kal} is one of the ASP logic programming systems well-suited for
solving such problems~\cite{24-kal}. Answer set programming is based on the stable model (answer set)
semantics of logic programming~\cite{26-kal}. In ASP, the search problems are
reduced to computing stable models, and \textit{answer set solvers}~--- programs for
generating stable models~--- are used to perform search. With regard to solving the maximum
clique problem, ASP methods are applicable nowadays for graphs with thousands of
vertices.

\subsection*{A2\ \ Conceptual Specification of~the~Application Domain
and~the~Problem}

  \noindent
  Conceptual schema (ontology) of the application domain of historical prices of
securities is written in the simplified\footnote{To save space, ``Declaration'' keyword
is omitted; property, domain and range declarations are combined.} OWL functional
syntax~\cite{28-kal}:
}
\end{multicols}

\hrule

{\small

\noindent
\begin{verbatim}
Ontology(<http://synthesis.ipi.ac.ru/optimalSecurityPortfolio>
    Class(stockRates)
        DataProperty(ticker  domain(stockRates) range(xsd:string))
        DataExactCardinality(1  ticker  stockRates)
        ObjectProperty(rates  domain(stockRates)  range(DatedValue))
        DataProperty(isInTop500  domain(stockRates)  range(xsd:boolean))
        DataExactCardinality(1  isInTop500  stockRates)

    Class(DatedValue)
        DataProperty(value  domain(DatedValue)  range(value  xsd:double))
        DataExactCardinality(1  value  DatedValue)
        DataProperty(date  domain(DatedValue)  range(xsd:date))
        DataExactCardinality(1  date  DatedValue)

    Class(correlation)
        DataProperty(series1  domain(correlation)  range(DatedValue))
        DataProperty(series2  domain(correlation)  range(DatedValue))
        DataProperty(corr  domain(correlation)  range(xsd:double))
        DataExactCardinality(1  corr  correlation)
)
\end{verbatim}


%\hrule

%\begin{multicols}{2}

 The {\sf stockRates} class is used to denote securities, which are characterized by:
  \begin{itemize}
  \item identifier (attribute {\sf ticker});
\item time series of historical prices (attribute {\sf rates});
\item belonging to S\&P~500 list of companies (attribute {\sf isInTop500}).
\end{itemize}

  The S\&P 500 is a stock market index maintained by the Standard \& Poor's,
comprising 500 large-cap American companies.

  The \textit{correlation} class is the correlation of time series pairs. For each class
instance, the value of \textit{corr} attribute equals to the correlation of its attributes
{\sf series1} and {\sf series2} (time series).

  The conceptual specification of the problem includes two RIF documents that
correspond to the specified tasks. The first of the documents (name \textit{gex} is the
local prefix of the document) contains a program that calculates the correlation graph
of securities (predicate \textit{noncorrelated}) based on the prices in a given period of
time. For each pair of securities, a correlation of their time series (defined in
section~A1) of historical prices is calculated that is why the graph is called
``correlation graph.'' But as the candidates for the elements of the portfolio, only
  noncorrelated securities (which correlation is lower than a predefined value) are
considered. That is why the predicate is called ``\textit{noncorrelated}.'' The document
is defined in the RIF-BLD\footnote{The RIF-BLD dialect is extended with the
possibility to use the existential quantifier in the head of a rule.}
  dialect~\cite{20-kal}:

%  \end{multicols}

%  \hrule


  \begin{verbatim}
Document( Dialect(RIF-BLD)
    Import(<http://synthesis.ipi.ac.ru/optimalSecurityPortfolio>
            <http://www.w3.org/ns/entailment/OWL-Direct>)
    Prefix(srt  <http://synthesis.ipi.ac.ru/optimalSecurityPortfolio#>)
    Prefix(gex  <http://synthesis.ipi.ac.ru/graphExtraction#>)
    Group(
    Forall ?t ?symbol ?ticker ?inTop500(
        Exists ?ts( And(?ts#gex:tickers  ?ts[symbol -> ?symbol]) ):-
            And(?t#srt:stockRates  ?t[ticker->?symbol]
                ?t[isInTop500->?inTop500]
                External(pred:boolean-equal(?inTop500  true))
            )
    )

        Forall ?m ?n ?с ?ticker1 ?ticker2 ?start ?end ?rates1 ?rates2
            ?top1 ?top2 ?dv1 ?dv2 ?date1 ?date2 ?series1 ?series2 ?corr (
            Exists ?e (
                And(?e#gex:noncorrelated ?e[start->?ticker1 end->?ticker2])):-
                ?m#srt:stockRates  ?n#srt:stockRates
                ?m[ticker->?ticker1  rates->?rates1  isInTop500->?top1]
                ?n[ticker->?ticker2  rates->?rates2  isInTop500->?top2]
                ?dv1#?rates1  ?dv1[date -> ?date1]
                ?dv2#?rates2  ?dv2[date -> ?date2]
                External(pred:date-greater-than-or-equal(?date1  2012-01-01))
                External(pred:date-less-than-or-equal(?date1  2012-12-31))
                External(pred:date-greater-than-or-equal(?date2  2012-01-01))
                External(pred:date-less-than-or-equal(?date2  2012-12-31))
                ?c#srt:correlation ?c[corr->?corr
                        series1->?rates1 series2->?rates2]
                External(pred:numeric-greater-than(?corr  -0.25)
                External(pred:numeric-less-than(?corr  0.25))
                External(pred:boolean-equal(?top1  true))
                External(pred:boolean-equal(?top1  true))
                External(pred:string-less-then(?ticker1  ?ticker2))
    ))
  )
)  
\end{verbatim}
}


%\hrule


\begin{multicols}{2}
{\small
  The first rule of the document defines the predicate-collection {\sf tickers}, in
which the attribute {\sf symbol} element runs through the list of security identifiers.
Here, {\sf e\#p} is a predicate denoting membership of element~{\sf e} in collection~{\sf p};
predicate {\verb e[a->v]} means that the value of attribute~{\sf a} of object~{\sf e} is~{\sf v}.
Securities taken only from S\&P~500 list are put into {\sf tickers} collection (attribute
{\sf isInTop500} equals to {\sf true}). This means that only large companies from
S\&P 500 are considered as candidates for portfolio elements.

  The second rule defines predicate {\sf noncorrelated}, which is a noncorrelation
relation between securities. Object {\sf ?e} belongs to a relation
{\sf noncorrelated} if}
  \begin{itemize}\small
\item both securities belong to S\&P top 500 list;
\item the absolute value of correlation between the time series if historical prices of
securities for given attribute values {\sf e.start} and {\sf e.end} is less than~0.25.
Here, 0.25 is the value chosen for the use-case as the predetermined maximal
correlation value (see section~A1); and
\item identifiers of securities {\sf id1} and {\sf id2} are in lexicographic order
(${\sf id1} < {\sf id2})$. This is required to prevent equivalent pairs ({\sf id1}, {\sf id2})
and ({\sf id2}, {\sf id1})
to be both presented in a relation corresponding to the predicate
{\sf noncorrelated}~--- only one pair is sufficient.
\end{itemize}

  Dates in time series range from January~1, 2012 to December~31, 2012.

The second document ({\sf prt}) contains a program that computes the maximum clique in a
graph of correlations. The document is defined in the RIF-CASPD\footnote{The CASPD
dialect is extended with the operation {\sf :}$\sim$ for a weak constraint. Such constraints should be
satisfied if possible, but their violation on some tuples of variables values does not
invalidate the models~\cite{13-kal}.} dialect~\cite{22-kal}:

}
\end{multicols}



\hrule
{\small

  \begin{verbatim}
Document( Dialect(RIF-CASPD)
    Import(<http://synthesis.ipi.ac.ru/optimalSecurutyPortfolio>
        <http://www.w3.org/ns/entailment/OWL-Direct>)
    Module(<http://synthesis.ipi.ac.ru/graphExtraction#>)
   Prefix(prt  <http://synthesis.ipi.ac.ru/portfolio#>)
    Prefix(gex  <http://synthesis.ipi.ac.ru/graphExtraction#>)
    Group (
        Forall ?X(Or(prt:portfolio(?X) prt:nonPortfolio(?)) :- tickers@gex(?X))
        Forall ?X ?Y( :- And(prt:portfolio(?X)  prt:nonPortfolio(?X)))
        Forall ?X ?Y( :- And(prt:portfolio(?X)  prt:portfolio(?Y)
                                        (Naf noncorrelated@gex(?X ?Y))) )
        Forall ?X( :~ prt:nonPortfolio(?X))
        prt:portfolio(?X).
) )
\end{verbatim}

}

\hrule


\begin{multicols}{2}


{\small
  The program defines a predicate {\sf portfolio}, whose values are the security
identifiers in the portfolio, and predicate {\sf nonPortfolio}, whose values include all
other securities under consideration.

  The first rule states that the only securities considered are the securities which turn to
truth the predicate {\sf tickers} in document~{\sf gex}.

  The second rule states that a security cannot simultaneously belong and not
belong to the portfolio. So, the predicates {\sf portfolio} and {\sf nonPortfolio} divide
the set of all securities into two nonintersecting sets.

  The third rule claims that any two securities in {\sf portfolio} are {\sf noncorrelated}
(according to {\sf noncorrelated} predicate in the document {\sf gex}). {\sf Naf}
here means a sort of negation~--- \textit{negation as failure}~\cite{32-kal}. This
exactly means that securities belonging to {\sf portfolio} form a clique in a graph of
noncorrelated securities.
{\looseness=1

}

  The fourth rule is a weak constraint, which minimizes the number of securities not
belonging to the portfolio. So, the {\sf portfolio} itself is maximized.

  Obviously, these four rules declaratively specify the problem of finding the
maximum cliques in a graph of noncorrelated securities.

  Fifth rule is just a predicate {\sf portfolio(?X)} with a free variable {\sf ?X}. This
rule forms a result of the program, that is, a collection of sets of identifiers of
securities. Each set corresponds to a particular solution of the problem (a stable
  ASP-model for the program in the document~{\sf prt}) and forms a maximal
portfolio.

\subsection*{A3\ \ Peers of the Use-Case Infrastructure}

  \noindent
For implementation of the conceptual specification of the problem of the
investment portfolio diversification, the two peers should be produced. One of them is
the mediator used for integration of data produced by the \textit{Google Finance} and
the \textit{Yahoo! Finance} services. 

Another one is a program for the rule-based
programming system DLV~\cite{9-kal}. The mediator produces a virtually integrated
collection of facts, and DLV is a system for executing ASP-programs. Initially, the
nodes do not contain any logic programs.

\subsubsection*{A3.1\ The mediator schema}

  \noindent
  The mediator schema implements the conceptual specification of the application
domain by expressing its semantics. Schema is written in the
  SYNTHESIS~\cite{4-kal}~--- the canonical information model used for the
mediator specifications:
}
%\end{multicols}

%\hrule

{\small
  \begin{verbatim}
{ FinanceServices; in: module;
  type:
  { DatedValue; in: type;
     date: time;
     value: real;
  },
  { StockRates; in: type;
     ticker: string;
     rates: {set; type_of_element: DatedValue;};
     isInTop500:
     { in: function;
         params: {-returns/boolean};
     };
  };

  class_specification:
  { stockRates; in: class;
     instance_section: StockRates;
  };

  function:
  { correlation; in: function;
    params: {+s1/{set; 
    type_of_element: DatedValue;},
    +s2/{set; 
    type_of_element: DatedValue;}, -corr/real };
  };
}
  \end{verbatim}
  
  \vspace*{-18pt}
  


  The schema includes types {\sf DatedValue} and {\sf StockRates}, a class {\sf
stockRates}, and a function {\sf correlation}. Semantics of class {\sf stockRates},
type {\sf DatedValue}, and function \textit{correlation} correspond to the semantics of
classes {\sf stockRates}, {\sf DatedValue}, and {\sf correlation} belonging to {\sf
optimalSecurityPortfolio} ontology (see section~A2).

  The correspondence between the mediator schema and the conceptual schema is
described by the schema mapping rules (see Fig.~3, middle part) constituting a separate
RIF-document:
}

\end{multicols}

\hrule

\vspace*{6pt}

{\small
  \begin{verbatim}
Document( Dialect(RIF-BLD)
    Import(<http://synthesis.ipi.ac.ru/optimalSecurityPortfolio>
            <http://www.w3.org/ns/entailment/OWL-Direct>)
    Prefix(srt  <http://synthesis.ipi.ac.ru/optimalSecurityPortfolio#>)
    Prefix(fs  <http://synthesis.ipi.ac.ru/FinanceServices#>)
    Group(
        And(?t#srt:stockRates
            ?t[?ticker-> srt:ticker ?top-> srt:isInTop500
                    ?rates-> srt:rates]) :-
        And(External(?t#fs:stockRates)
            External(?t[?ticker->fs:ticker ?top->fs:isInTop500
                ?rates->fs:rates]) )

        And(?dv#srt:DateValue ?dv[?v->srt:value ?d->srt:date]) :-
        And(External(?dv#fs:DateValue)
            External(?dv[?v->fs:value ?d->fs:date])     )

        And(?c#correlation
            ?c[?series1->series1 ?series2->series2 ?corr->corr]) :-
        External(pred:numeric-equal(
            fs:correlation(?series1 ?series2) ?corr))
    )
)
  \end{verbatim}
}      \hrule

  \begin{multicols}{2}
{\small
 The document consists of three rules. The first one defines a correspondence
between class {\sf stockRates} of the {\sf optimalSecurityPortfolio} ontology
(denoted as {\sf srt:stockRates}) and class {\sf stockRates} of the
{\sf FinanceServices} mediator module (denoted as {\sf fs:stockRates}) as well as
correspondences between their attributes. The second and the third rules define
correspondences between ontology class {\sf srt:DatedValue} and mediator type {\sf
fs:DatedValue} and between ontology class {\sf srt:correlation} and mediator
function {\sf fs:correlation}.

  Entities of the mediator schema and conceptual schema are in one-to-one
correspondence, the names and the semantics of the relevant entities are the same.
Also, the mediator is a conformant RIF-BLD consumer. Thus, the mediator
{\sf FinanceServices} is relevant to the RIF-document {\sf gex} of the
conceptual schema.

\subsubsection*{A3.2\  Resource integration by the mediator}

  \noindent
 The mediator integrates two resources: \textit{Google Finance} and the
\textit{Yahoo! Finance services}. For both resources, specific wrappers (the
\textit{Google Finance} wrapper and the \textit{Yahoo! Finance} wrapper) were
implemented to embed the resources in the mediation system~\cite{33-kal}.

  Schema of the \textit{Google Finance} service presented for uniformity in the
canonical information model (the SYNTHESIS language) looks as follows:
}
%\end{multicols}

%\hrule

{\small
  \begin{verbatim}
{ GoogleFinance; in: module;
   type:
   { GoogleQuote; in: type;
      symbol: string;
      Date: time;
      Open: real;
      High: real;
      Low: real;
      Close: real;
      Volume: integer;
   };

   function:
   { groupByTicker; in: function;
      params: {+symbol/string, 
      +date/time, +close/real,
       -rates/{set; type_of_element: 
        DatedValue;}}
   },
   { isInTop500; in: function;
      params: {+ticker/string, 
       -returns/boolean};
   };

   class_specification:
   { historicalData; in: class;
      instance_section: GoogleQuote;
   };
}
  \end{verbatim}
 } 
 
 \vspace*{-9pt}

 % \hrule

  %\begin{multicols}{2}
{\small

 The schema includes a type {\sf GoogleQuote}, which includes attributes that
match the identifier of the company ({\sf symbol}), date ({\sf Date}), and different
indicators of the security price. Daily closing price of the shares ({\sf Close}) is used
to form the portfolio.

  The function {\sf top500} reflects the S\&P~500 index: if the value of the ticker
parameter belongs to S\&P~500 list, then the function returns {\sf true} and
{\sf false} otherwise.

  The function {\sf groupByTicker} is aimed to group flat instances of the {\sf
GoogleQuote} type by {\sf ticker} attribute to produce a time series of an indicator of
the security price for all securities identified by {\sf ticker}.

  Functions of the schema are implemented by the \textit{Google Finance} wrapper.

  Schema of the \textit{Yahoo! Finance} service presented in the canonical model of
subject mediators looks as follows:
}
%\end{multicols}

%\hrule

\noindent
{\small
  \begin{verbatim}
{ YahooFinance; in: module;
   type:
   { YahooQuote; in: type;
      symbol: string;
      Date: time;
      Open: real;
      High: real;
      Low: real;
      Close: real;
      Volume: integer;
      Adj_Close: real;
   };
   \end{verbatim}
   
   \vspace*{-18pt}
   
   \columnbreak
   

\noindent
\begin{verbatim}
   function:
   { groupByTicker; in: function;
      params: {+symbol/string, 
      +date/time, +close/real, -rates/{set; 
       type_of_element: DatedValue;}};
   },
   { isInTop500; in: function;
      params: {+ticker/string, 
       -returns/boolean};
   };

   class_specification:
   { historicalData; in: class;
      instance_section: YahooQuote;
   };
}
  \end{verbatim}
}

\vspace*{-9pt}

{\small
  The schema includes a type {\sf YahooQuote} similar to {\sf GoogleQuote} and
functions {\sf groupByTicker} and {\sf isInTop500} similar to respective functions of
\textit{GoogleFinance} schema.

  Integration of the resources in the mediator has been provided using
  GLAV-mappings. The GLAV-mapping is a combination of GAV-mappings and
  LAV-mappings. The GAV-mapping looks as follows:}
{\small
  \begin{verbatim}
financeData(x/[ticker, rates]) :-
 group_by({ticker}, [ticker, p1: partition],
  GoogleFinance.historicaldata(
   x/[ticker: symbol, date: Date, 
    value: Close]))&
 group_by({ticker}, [ticker, p2: partition],
   YahooFinance.historicaldata(
    x/[ticker: symbol, date: Date, 
     value: Close])&
  is_equal(rates, union(p1, p2))
 ).
  \end{verbatim}
}
  

%  \hrule

%  \begin{multicols}{2}

\vspace*{-9pt}

{\small

  An intermediate predicate {\sf financeData} (used to bind GAV and
  LAV-mappings) is expressed as a view over resource classes {\sf
GoogleFinance.historicaldata} and {\sf YahooFinance.historicaldata}. Flat data
structures of the resource classes are turned into hierarchical data structures of the
mediator class using {\sf group\_by} operation. Curly braces in {\sf group\_by}
operation enclose the grouping attribute ({\sf ticker}). Square brackets in {\sf
group\_by} operation enclose attributes of objects in a collection formed by {\sf
group\_by} operation~\cite{4-kal}. {\sf Partition} is a default attribute storing
grouped partitions. Here, resource attributes {\sf Date} and {\sf Close} are grouped in
the {\sf partition} attribute. Partitions from {\sf Google} and {\sf Yahoo} resource
classes are merged into {\sf rates} attribute of {\sf stockRates} class ({\sf union}
operation). Attributes of the resource classes are renamed properly in order to meet
the structure of the mediator class (for instance, {\sf Close} attribute is renamed into
{\sf value}).

 \begin{figure*}[b] %fig4
    \vspace*{1pt}
 \begin{center}
 \mbox{%
 \epsfxsize=157.485mm
 \epsfbox{kali-5.eps}
 }
 \end{center}
 \vspace*{-6pt}
\Caption{Portfolio problem infrastructure}
  \label{f5-kal}
  \end{figure*}

The  LAV mappings look as follows:
}
%\end{multicols}

%\hrule


\noindent
{\small
  \begin{verbatim}
financeData(x/[ticker, rates]) :- 
   stockRates(x/[ticker, rates]).
GoogleFinance.isInTop500(x.ticker, b) -> 
   isInTop500(x, b)
YahooFinance.isInTop500(y.ticker, b) -> 
   isInTop500(y, b)
\end{verbatim}
}


%\hrule

%\vspace*{6pt}
{\small

\noindent
  Here, {\sf financeData} predicate is expressed as a simple view over mediator
schema class {\sf stockRates}, and {\sf isInTop500} resource functions are expressed
through mediator {\sf isInTop500} function using\ \textit{functional inverse
rules}~\cite{34-kal}.
}
%\begin{multicols}{2}

{\small
\subsubsection*{A3.3\ DLV system}

\noindent
  DLV system is a conformant RIF-CASPD consumer, and initially, the system does
not contain any logic program or facts. Document {\sf prt} does not contain any
occurrence of an extensional predicate corresponding to an entity in the conceptual
schema. Thus, the system is relevant to the RIF-document {\sf prt} of the
conceptual schema, since the relevance condition (see subsection~3.2) is reduced to the
relation of dialects.
{\looseness=1

}

\subsection*{A4\ \ Portfolio Problem Infrastructure}

  \noindent
  In the provided example, the infrastructure (Fig.~\ref{f5-kal}) includes two nodes
corresponding to the mediation system (called {\sf fsv}) and to a rule-based
programming system DLV (called {\sf dlv}).



  Prefixes of the documents of the conceptual schema in the rules are replaced with
the actual node prefixes during rewriting  the conceptual programs into the
programs over the P2P network nodes. For instance, the rules of the {\sf prt} document
were rewritten as follows:
}
\end{multicols}

\hrule

{\small
  \begin{verbatim}
Document( Dialect(RIF-CASPD)
    Module(<http://synthesis.ipi.ac.ru/resources/FinanceServices#>)
    Prefix(dlv  <http://synthesis.ipi.ac.ru/resources/DLV#>)
    Prefix(fsv  <http://synthesis.ipi.ac.ru/resources/FinanceServices#>)
    Group (
        Forall ?X(Or(dlv:portfolio(?X) dlv:nonPortfolio(?)) :-
                            tickers@fsv(?X))
        Forall ?X ?Y( :- And(dlv:portfolio(?X)  dlv:nonPortfolio(?X)))
        Forall ?X ?Y( :- And(dlv:portfolio(?X)  dlv:portfolio(?Y)
                                    (Naf  noncorrelated@fsv(?X ?Y))) )
        Forall ?X( :~ dlv:nonPortfolio(?X))
        dlv:portfolio(?X).
    )
)
\end{verbatim}
}


%\hrule



{\small
  Rules from the document {\sf gex} were rewritten similarly. Rewritten
documents are sent by the supervisor to the proper nodes: {\sf gex} to the
mediation node, and {\sf prt}  to the {\sf dlv} node.

  After that, the node wrappers automatically execute distributed programs in
accordance with the algorithm described in subsection~3.3. First, the program
normalization is done. Thus, the nonlocal rule mentioned above}

\begin{multicols}{2}

\noindent
{\small
\begin{verbatim}
 Forall ?X ?Y( :- And(dlv:portfolio(?X)  
 dlv:portfolio(?Y)
    (Naf  noncorrelated@fsv(?X ?Y))) )
\end{verbatim}
}

\noindent
{\small is transformed into a delegation rule}
{\small
\begin{verbatim}
noncorrelated_fsv(?X ?Y) :- 
  noncorrelated@fsv(?X ?Y)
\end{verbatim}
and a local rule
\begin{verbatim}
Forall ?X ?Y( :- And(dlv:portfolio(?X)  
  dlv:portfolio(?Y)
     (Naf  noncorrelated_fsv(?X ?Y))) )
  \end{verbatim}
}
  

%  \hrule

\vspace*{-9pt}

 % \begin{multicols}{2}
{\small
 Remaining rules from both documents are normalized in a similar way.

  After that, the normalized program is executed. The fact delegation rule, which
sends the {\sf noncorrelated} predicate to the {\sf dlv} node, is transmitted to the
{\sf fsv} node. The bodies of the rules of the program on the {\sf fsv} node do not contain
remote terms; so, the program on the node can be executed without waiting for the
facts from the other nodes. In contrast, the {\sf dlv} node has to wait for the arrival of the
facts from the {\sf fsv} node, which turns the {\sf tickers} and the {\sf noncorrelated}
predicates to {\sf true}. After receiving all necessary facts, the program computing
portfolio is executed in the {\sf dlv} node.

  Before the execution, the normalized programs written in RIF dialects were
automatically transformed into the logic languages supported on the nodes~--- the
SYNTHESIS and the DLV, respectively. Transformations were
implemented using a model transformation language and toolkit ATL (ATLAS
Transformation Language)~\cite{35-kal}.

  Toolkit ATL is a hybrid of declarative and imperative languages. An ATL transformation
program is composed of rules that define how source model elements are matched and
navigated to create and initialize the elements of the target models.

  The result of the transformation of the {\sf prt} document into the SYNTHESIS
language looks as follows: 
}
%\end{multicols}

%\hrule


\noindent
{\small
  \begin{verbatim}
tickers(ts/[symbol]) :- 
    stockRates(t/[symbol: ticker]) &
    isInTop500(t, inTop500) & inTop500 = true.

noncorrelated(e/[start: ticker1, 
 end: ticker2]) :-
stockRates(m/[ticker1: ticker, rates1: rates, 
    top1: isInTop500]) &
stockRates(n/[ticker2: ticker, rates2: rates, 
    top2: isInTop500]) &
in_set(dv1, rates1) & in_set(dv2, rates2) &
dv1.date = date1 & dv2.date = date2 &
date1 >= '2012-01-01' & date1 <= '2012-12-31' &
date2>= '2012-01-01' & date2 <= '2012-12-31' &
correlation(c/[corr, rates1: series1, 
rates2: series2 ]) &
corr >= -0.25 & corr <= 0.25 &
top1 = true & top2 = true &
ticker1 < ticker2.
  \end{verbatim}
}
  

%  \hrule

%  \begin{multicols}{2}
{\small
  Note that existential quantifier in heads of the rules is discarded during the
mapping. According to the semantics of the SYNTHESIS logic rules, the variables in the
head of a rule which are not presented in the body of the rule are bounded by
existential quantifier by default. The RIF predicates of belonging of a variable to a class
(like \verb"?t\#stockRates") are combined with frame predicates (like
  \verb"?t[ticker->?symbol]") and mapped into a predicate-collection~\cite{4-kal} of the
SYNTHESIS language (like {\sf stockRates(t/[symbol: ticker])}).

  The result of the transformation of the {\sf gex} document into DLV language
looks as follows:}

{\small
  \begin{verbatim}
portfolio(X) v nonPortfolio(X) :- tickers(X).
:- portfolio(X), portfolio(Y),
     not noncorrelated(X, Y).
:~ nonPortfolio(X).
portfolio(?X).
  \end{verbatim}
}  

{\small
\subsection*{A5\ Result of the Use Case Program Execution}


\noindent
The maximal models of the {\sf portfolio} predicate, found during the execution of
the program on the {\sf dlv} node, are the diversified portfolios of maximal size
containing securities of companies from the S\&P 500 for 2012. As the result of the
program execution, the 11~stable models containing 10~ground atoms each were
generated (the maximum size of a diversified portfolio appeared to be equal to 10).
The models can be found below. Atoms contain symbols of different companies,
e.\,g., DUK and SBUX denote \textit{Duke Energy} and \textit{Starbucks Corp.},
respectively. The full list of abbreviations is provided in the table. The
models are outlined as the sets of company identifiers (like SBUX) related to the security ground terms
in {\sf portfolio(SBUX)}. Note that the models have nonempty intersections:



}
\end{multicols}

\hrule

\vspace*{6pt}


{\small
  Model 1: $\{\mathrm{CAH, DUK, EL, ETR, HOT, LM, PSA, TJX, TYC, UNH}\}$

  Model 2: $\{\mathrm{CAH, DUK, EL, ETR, HOT, LM, PSA, TJX, UNH, VNO}\}$

  Model 3: $\{\mathrm{CAH, DUK, EL, ETR, HON, LM, PSA, TJX, UNH, VNO}\}$

  Model 4: $\{\mathrm{CAH, DUK, EL, ETR, LM, PSA, TJX, TYC, UNH, WDC}\}$

  Model 5: $\{\mathrm{BMY, DUK, EL, FDX, IPG, KSS, PSA, STT, TJX, VIAB}\}$

  Model 6: $\{\mathrm{BA, BMY, DUK, EL, IPG, KSS, PSA, STT, TJX, VIAB}\}$

  Model 7: $\{\mathrm{BA, BMY, DUK, EL, KSS, LH, PSA, STT, TJX, VIAB}\}$

  Model 8: $\{\mathrm{AGN, BA, BMY, DUK, EL, KSS, LH, PSA, TJX, VIAB}\}$

  Model 9: $\{\mathrm{BA, BMS, BMY, DUK, EL, KSS, LH, PSA, SBUX, TJX}\}$

  Model 10: $\{\mathrm{BMY, BSX, DUK, EL, FDX, GAS, KMI, MDLZ, RRC, TJX}\}$

  Model 11:$\{\mathrm{BA, BMY, DUK, FMC, GIS, KIM, LH, LTD, MNST, SBUX}\}$
}

\pagebreak

\begin{table}\small
  \begin{center}
%  \Caption{Tickers symbols and company names}
%  \vspace*{2ex}
  \begin{tabular}{ll||ll}
  \multicolumn{4}{c}{Tickers symbols and company names}\\[6pt]
  \hline
\multicolumn{1}{c}{Ticker}&\multicolumn{1}{c||}{Company}&\multicolumn{1}{c
}{Ticker}&\multicolumn{1}{c}{Company}\\
\hline
CAH&Cardinal Health Inc.               &STT&State Street Corp.\\
DUK&Duke Energy                        &VIAB&Viacom Inc.\\
EL&Estee Lauder Cos.                   &BA&Boeing Company\\
ETR&Entergy Corp.                      &LH&Laboratory Corp. of America Holding\\
HOT&Starwood Hotels \& Resorts         &AGN&Allergan Inc\\
LM&Legg Mason                          &BMS&Bemis Company\\
PSA&Public Storage                     &SBUX&Starbucks Corp.\\
TJX&TJX Companies Inc.                 &BSX&Boston Scientific\\
TYC&Tyco International                 &GAS&AGL Resources Inc.\\
UNH&United Health Group Inc.           &KMI&Kinder Morgan\\
VNO&Vornado Realty Trust               &MDLZ&Mondelez International\\
HON&Honeywell Int'l Inc.               &RRC&Range Resources Corp.\\
WDC&Western Digital                    &FMC&FMC Corporation\\
BMY&Bristol-Myers Squibb               &GIS&General Mills\\
FDX&FedEx Corporation                  &KIM&Kimco Realty\\
IPG&Interpublic Group                  &LTD&L Brands Inc.\\
KSS&Kohl's Corp.             &MNST&Monster Beverage\\
\hline
\end{tabular}
\end{center}
\vspace*{-6pt}
\end{table}

%\hrule

\begin{figure} %fig 6
   \vspace*{1pt}
 \begin{center}
 \mbox{%
 \epsfxsize=120mm
 \epsfbox{kali-6.eps}
 }
 \end{center}
 \vspace*{-9pt}
\Caption{Models and their elements}
  \label{f6-kal}
  \vspace*{-3pt}
  \end{figure}

\begin{multicols}{2}
  

{\small
 Models and their elements are illustrated in Fig.~\ref{f6-kal}\footnote
 {Figure was
created using a graph visualization platform \textit{Gephi} ({\sf https://gephi.org/}).
The \textit{ForceAtlas} layout algorithm was applied.}.

The circles in the figure denote companies. The lines connecting two circles denote that
two companies belong to the same model. The larger is the circle, the larger is the number
of models to which a company belongs. The thicker is the line, the larger is the number
of models to which a pair of companies belongs to.}



{\small\frenchspacing
{%\baselineskip=10.8pt
\begin{thebibliography}{99}

\bibitem{2-kal} %1
Hey, T., S. Tansley, and K.~Tolle, eds. 2009. \textit{The fourth paradigm: Data-intensive
scientific discovery}. Redmond: Microsoft Research. 252~p.

\bibitem{1-kal} %2
Challenges and opportunities with big data.  A~community white paper developed by
leading re-\linebreak\vspace*{-12pt}

\pagebreak

\noindent
searchers across the United States. 2012. Available at: {\sf
http://cra.org/ccc/docs/init/bigdatawhitepaper.pdf} (accessed November 21, 2013).

\bibitem{3-kal}
\Aue{Kappel, G., M.~Wimmer, W.~Retschitzegger, and W.~Schwinger}. 2011. Leveraging
model-based tool integration by conceptual modeling techniques. \textit{The evolution of
conceptual modeling}. Eds. R.~Kaschek and L.\,M.\,L.~Delcambre. Lecture notes in computer science ser.
Berlin, Heidelberg: Springer-Verlag. 6520:254--284.
\bibitem{4-kal}
\Aue{Kalinichenko, L.\,A., S.\,A.~Stupnikov, and D.\,O.~Martynov}. 2007. SYNTHESIS:
A~language for canonical information modeling and mediator definition for problem
solving in heterogeneous information resource environments. Moscow: IPI RAN. 171~p.
\bibitem{5-kal}
\Aue{Kalinichenko, L.\,A., D.\,O.~Briukhov, D.\,O.~Martynov, N.\,A.~Skvortsov, and
S.\,A.~Stupnikov.} 2007. Mediation framework for enterprise information system
infrastructures. \textit{9th Conference (International) on Enterprise Information Systems
ICEIS 2007 Proceedings}. Funchal. Vol. Databases and information systems integration.
 246--251.
\bibitem{6-kal}
\Aue{Kalinichenko, L.\,A.} 1990. Methods and tools for equivalent data model mapping
construction. \textit{Advances in database technology~---
EDBT'90}. Eds. F.~Buncilhon, C.~Thanos, and D.~Tsichritzis. Lecture notes in
computer science ser. Berlin, Heidelberg: Springer-Verlag. 416:92--119.
\bibitem{7-kal}
\Aue{Kalinichenko, L.\,A., and S.\,A.~Stupnikov}. 2012. Synthesis of the canonical models
for database integration preserving semantics of the value inventive data models.
\textit{Advances in database and information systems}. Eds. T.~Morzy,
T.~H$\ddot{\mbox{a}}$rder, R.~Wrembel,
\textit{et al.} Lecture notes in computer science ser. Berlin, Heidelberg: Springer-Verlag.
7503:223--239.
\bibitem{8-kal}
Boley, H., and M.~Kifer, eds. 2013. RIF overview. W3C working group
note. 2nd ed. Available at: {\sf http://www.w3.org/TR/rif-overview/} (accessed November~21,  2013).


\bibitem{10-kal} %9
\Aue{Kalinichenko, L., S. Stupnikov, A.~Vovchenko, and D.~Kovalev}. 2013. Rule-based
multi-dialect infrastructure for conceptual problem solving over heterogeneous distributed
information resources. New trends in databases and information systems. \textit{Selected
Papers of the 17th European Conference on Advances in Databases and Information
Systems and Associated Satellite Events. Advances in Intelligent Systems and Computing}
241:61--68.
\bibitem{9-kal}%10
\Aue{Leone, N., G. Pfeifer, W.~Faber, T.~Eiter, G.~Gottlob, S.~Perri, and F.~Scarcello}.
2006. The DLV system for knowledge representation and reasoning. \textit{ACM
Trans. Comput. Log.} 7(3):499--562.

\bibitem{11-kal}
\Aue{Kalinichenko, L.\,A., and S.\,A.~Stupnikov}. 2008. Constructing of mappings of
heterogeneous information models into the canonical models of integrated information
systems. \textit{ADBIS 2008 Proceedings}. Pori: Tampere University of Technology.
106--122.
\bibitem{12-kal}
\Aue{Abrial, J.-R.} 1996. The B-book: Assigning programs to meanings. Cambridge:
Cambridge University Press. 816~p.
\bibitem{13-kal}
\Aue{Skvortsov, N.\,A.} 2012. Otobrazhenie modeley dannykh NoSQL v ob''ektnye
specifikatsii [Mapping of NoSQL data models to object specifications]. \textit{Trudy 14-y\linebreak
Vserossiyskoy nauchnoy konferentsii ``Elektronnye biblioteki: Perspektivnye metody i
tekhnologii, elektronnye kollektsii''' RCDL 2012} [\textit{14th Russian Conference on
Digital\linebreak \mbox{Libraries} RCDL 2012 Proceedings}]. \textit{CEUR Workshop Proceedings}
934:53--62.
\bibitem{14-kal}
\Aue{Stupnikov, S.\,A.} 2013. Otobrazhenie grafovoy modeli dannykh v kanonicheskuyu
ob''ektno-freymovuyu informa\-tsi\-on\-nuyu model' pri sozdanii sistem integratsii neodnorodnykh
informatsionnykh resursov [Mapping of a graph data model into an object-frame canonical
information model for the development of heterogeneous information resources integration
systems]. \textit{Trudy 15-y Vserossiyskoy nauchnoy konferentsii ``Elektronnye biblioteki:
Perspektivnye metody i tekhnologii, elektronnye kollektsii'' RCDL 2013} [\textit{15th
Russian Conference on Digital Libraries RCDL 2013 Proceedings}]. Yaroslavl:
P.\,G.~Demidov Yaroslavl State University. 193--202.
\bibitem{15-kal}
\Aue{Skvortsov, N.\,A.} 2013. Otobrazhenie modeli dannykh RDF v kanonicheskuyu model'
predmetnykh posrednikov [Mapping of RDF data model into the canonical model of subject
mediators]. \textit{Trudy 15-y Vserossiyskoy nauchnoy konferentsii ``Elektronnye biblioteki:
Perspektivnye metody i tekhnologii, elektronnye kollektsii'' RCDL 2013} [\textit{15th
Russian Conference on Digital Libraries RCDL 2013 Proceedings}]. Yaroslavl:
P.\,G.~Demidov Yaroslavl State University.\linebreak 202--209.
\bibitem{16-kal}
\Aue{Stupnikov, S.\,A.} 2013. Verifitsiruemoe otobrazhenie mo\-de\-li dannykh, osnovannoy na
mnogomernykh massivakh, v ob''ektnuyu model' dannykh [A~verifiable mapping of a
multidimensional array data model into an object data model].
\textit{Informatika i ee Primeneniya~--- Inform. Appl.} 7(3):22--34.
\bibitem{17-kal}
\Aue{Fagin, R., P.\,G.~Kolaitis, R.\,J.~Miller, and L.~Popa}. 2005. Data exchange:
Semantics and query answering. \textit{Theor. Comput. Sci.} 336:89--124.
\bibitem{18-kal} %18
\Aue{Lenzerini, M.} 2002. Data integration: A~theoretical perspective. \textit{ACM
Symposium on Principles of Database Systems (PODS) Proceedings}. 233--246.

\bibitem{34-kal} %19
\Aue{Kalinichenko, L.\,A., D.\,O.~Martynov, and S.\,A.~Stupnikov}. 2004. Query
rewriting using views in a typed mediator environment. \textit{Advances in databases and information
systems}. Eds. G.~Gottlob, A.~Benczur, and J.~Demetrovics. Lecture notes in
computer science ser. Berlin, Heidelberg: Springer-Verlag.
3255:37--53.


\bibitem{20-kal} %20
Boley, H., and M.~Kifer, eds. 2013. RIF basic logic dialect.  W3C recommendation.  2nd ed.
Available at: {\sf http://www.w3.org/TR/rif-bld/} (accessed November~21, 2013).

\bibitem{21-kal} %21
De Sainte Marie,~C., G.~Hallmark, and A.~Paschke, eds. 2013. RIF production rule dialect.
 W3C recommendation. 2nd ed. Available at: {\sf http://www.w3.org/TR/rif-prd/} (accessed
November~21, 2013).

\bibitem{19-kal} %22
Boley, H., and M.~Kifer, eds. 2013. RIF framework for logic dialects.  W3C
recommendation. 2nd ed. Available at: {\sf http://www.w3.org/TR/rif-fld/} (accessed
November~21, 2013).

\bibitem{23-kal} %23
Kifer,~M., ed. 2010. RIF core logic programming dialect based on the well-founded
semantics. Available at: {\sf http://ruleml.org/rif/RIF-CLPWD.html} (accessed
November~21, 2013).

\bibitem{25-kal} %24
\Aue{Van Gelder, A., K.\,A.~Ross, and J.\,S.~Schlipf}. 1991. The well-founded semantics
for general logic programs. \textit{J.~ACM} 38(3):620--650.

\bibitem{22-kal} %25
Heymans, S., and M.~Kifer, eds. 2009. RIF core answer set programming dialect. Available
at: {\sf http://ruleml.org/rif/RIF-CASPD.html} (accessed November~21, 2013).


\bibitem{26-kal} %26
\Aue{Gelfond, M., and V.~Lifschitz}. 1988. The stable model semantics for logic
programming. \textit{Logic Programming: 5th Conference and Symposium Proceedings}.
1070--1080.

\bibitem{27-kal} %27
De Bruijn, J., and C.~Welty, eds. 2013. RIF RDF and OWL Compatibility.  W3C
recommendation. 2nd ed. Available at: {\sf http://www.w3.org/TR/rif-rdf-owl/} (accessed
November~21, 2013).

\bibitem{28-kal} %28
Motik,~B., P.\,F.~Patel-Schneider, and B.~Parsia, eds. 2012. OWL~2 Web ontology language structural specification
and functional-style syntax.  W3C recommendation. 2nd ed. Available at: {\sf
http://www.w3.org/TR/owl2-syntax/} (accessed November~21, 2013).



\bibitem{29-kal}
\Aue{Shvaiko, P., and J.~Euzenat}. 2005. A survey of schema-based matching approaches.
\textit{J.~Data Semantics} IV:146--171.

\bibitem{30-kal} %30
\Aue{Abiteboul, S., M.~Bienvenu, A.~Galland, \textit{et al}.} 2011. A~rule-based
language for Web data management. \textit{30th ACM Symposium on Principles of
Database Systems Proceedings}. ACM Press. 283--292.


\bibitem{24-kal} %31
\Aue{Gelfond, M.} 2008. \textit{Answer sets. Handbook of knowledge representation}.
Elsevier. 285--316.

\bibitem{40-kal} %32
\Aue{Motik, B., and R.~Rosati}. 2007. Closing semantic Web ontologies. University of
Manchester Report. Available at: {\sf
http://www.cs.ox.ac.uk/people/boris.motik/pubs/ mr06closing-report.pdf} (accessed
November~21, 2013).

\bibitem{36-kal} %33
\Aue{Chen, P.\,P., B.~Thalheim, and L.\,Y.~Wong}. 1999. Future directions of conceptual
modeling. \textit{Conceptual modeling}. Eds. P.\,P.~Chen, J.~Akoka, 
H.~Kangassulu, and B.~Thalheim.
Lecture notes in computer science ser. Berlin, Heidelberg:
Springer-Verlag. 1565:287--301.

\bibitem{37-kal} %34
\Aue{Costal, D., C.~G$\acute{\mbox{o}}$mez, and G.~Guizzardi}. 2011. Formal
semantics and ontological analysis for understanding subsetting, specialization and
redefinition of associations in UML. \textit{Conceptual modeling~--- ER 2011}.
Eds. M.~Jeusfeld, L.~Delcambre, and T.\,W.~Ling. Lecture notes in computer
science ser. Berlin, Heidelberg: Springer-Verlag. 6998:189--203.

\bibitem{39-kal} %35
\Aue{Martinez, Y., C. Cachero, M.~Matera, S.~Abrahao, and S.~Lujan}. 2011. Impact of
MDE approaches on the maintainability of Web applications: An experimental evaluation.
\textit{Conceptual modeling~--- ER 2011}.
Eds. M.~Jeusfeld, L.~Delcambre, and T.\,W.~Ling. Lecture notes in computer
science ser. Berlin, Heidelberg: Springer-Verlag. 6998:233--246.

\bibitem{38-kal} %36
\Aue{Castro, L.\,F. Bai$\tilde{\mbox{a}}$o, and G.~Guizzardi}. 2011. A~semantic
qriented method for conceptual data modeling in \mbox{OntoUML} based on linguistic concepts.
\textit{Conceptual modeling~--- ER 2011}.
Eds. M.~Jeusfeld, L.~Delcambre, and T.\,W.~Ling. Lecture notes in computer
science ser. Berlin, Heidelberg: Springer-Verlag. 6998:486--494.

\bibitem{41-kal} %37
\Aue{Fillottrani, P.\,R., E.~Franconi, and S.~Tessaris}. 2011. The ICOM 3.0 intelligent
conceptual modelling tool and methodology. Semantic Web. \textit{IOS Press}. 1--14.
Available at: {\sf http://www.semantic-web-journal.net/sites/default/files/swj105\_1.pdf}
(accessed November 21, 2013).

\bibitem{42-kal} %38
\Aue{Maus,~C., S. Rybacki, and A.\,M.~Uhrmache}. 2011. Rule-based multi-level
modeling of cell biological systems. \textit{BMC Syst. Biol.} 5:166. Available at:
{\sf http://www.biomedcentral.com/content/pdf/1752-0509-5-166.pdf} (accessed
November~21, 2013).

\bibitem{43-kal} %39
\Aue{Calvanese,~D., G.~De Giacomo, D.~Lembo, M.~Lenzerini, A.~Poggi, and
R.~Rosati}. 2007. MASTRO-I: Efficient integration of relational data through DL
ontologies. \textit{2007 Workshop (International) on Description Logic (DL 2007)
Proceedings}. \textit{CEUR Workshop Proceedings}. 250.

\bibitem{44-kal} %40
\Aue{Cosentino,~V., M.\,D.~Del Fabro, and A.\,El~Ghali}. 2012. A~model driven
approach for bridging ILOG rule language and RIF. \textit{RuleML 2012 Proceedings.
CEUR Workshop Proceedings}. 874. Available at: {\sf http://ceur-ws.org/Vol-874/paper9.pdf} (accessed November~21, 2013).

\bibitem{45-kal} %41
\Aue{Gonzalez-Moriyon, G., L.~Polo, D.~Berrueta, and C.~Tejo-Alonso}. 2012. Final steel
industry public demonstrators. ONTORULE deliverable D5.5. Available at: {\sf
http://ontorule-project.eu/outcomes\% 3Ffunc=fileinfo\&id=94.html} (accessed
November~21, 2013).

\bibitem{48-kal} %42
\Aue{Loo, B.\,T., T.~Condie, M.~Garofalakis, D.\,E.~Gay, J.\,M.~Hellerstein, P.~Maniatis,
R.~Ramakrishnan, T.~Roscoe, and I.~Stoica}. 2006. Declarative networking: Language,
execution and optimization. \textit{ACM SIGMOD Conference Proceedings}. 97--108.


\bibitem{47-kal} %43
\Aue{Grumbach, S., and F.~Wang}. 2010. Netlog, a rule-based language for distributed
programming. \textit{Practical aspects of declarative
languages}. Eds.\ M.~Carro and R.~Pe$\tilde{\mbox{n}}$a. Lectire notes
in computer science ser. Berlin, Heidelberg: Springer-Verlag. 5937:88--103.

\bibitem{46-kal} %44
\Aue{Alvaro, P., W.\,R.~Marczak, N.~Conway, J.\,M.~Hellerstein, D.~Maier, and R.~Sears.}
2010. Dedalus: Datalog in time and space.
\textit{Datalog reloaded}. Eds. O.~de Moor, G.~Gottlob, T.~Furche, and A.~Sellers.
Lectures notes in computer science ser. Berlin, Heidelberg: Springer-Verlag. 6702: 262--281.


\bibitem{49-kal} %45
\Aue{Grossmann, G., R.~Thiagarajan, M.~Schrefl, and M.~Stumptner}. 2011. Conceptual
modeling approaches for dynamic Web service composition. \textit{The evolution of
conceptual modeling}. Eds. R.~Kaschek and L.\,M.\,L.~Delcambre. Lecture notes in computer science ser.
Berlin, Heidelberg: Springer-Verlag. 6520:180--204.

\bibitem{50-kal} %46
\Aue{Henderson-Sellers, B.} 2011. Random thoughts on multi-level conceptual modelling.
\textit{The evolution of conceptual modeling}. Eds. R.~Kaschek and L.\,M.\,L.~Delcambre. Lecture notes in computer science ser.
Berlin, Heidelberg: Springer-Verlag. 6520:93--116.
\bibitem{51-kal} %47
\Aue{Falbo, R.\,A., F.\,B. Ruy, and R.\,D.~Moro}. 2005. Using ontologies to add semantics
to a software engineering environment. \textit{SEKE 2005 Proceedings}. Curran
Associates. 151--156.
\bibitem{52-kal} %48
\Aue{Berre, A., and M.~Missikoff} (Moderators). 2011. Panel: Modeling for the future
Internet. \textit{Conceptual modeling~--- ER 2011}.
Eds. M.~Jeusfeld, L.~Delcambre, and T.\,W.~Ling. Lecture notes in
computer science ser. Berlin, Heidelberg: Springer-Verlag. 6998: 526--527.
\bibitem{53-kal} %49
\Aue{Porto, F., and S.~Spaccapietra}. 2011. Data model for scientific models and
hypotheses. \textit{The evolution of conceptual modeling}. Eds. R.~Kaschek and L.\,M.\,L.~Delcambre. Lecture notes in computer science ser.
Berlin, Heidelberg: Springer-Verlag. 6520: 285--305.
\bibitem{54-kal} %50
\Aue{Racunas, S.\,A., N.\,H. Shah, I.~Albert, and N.\,V.~Fedoroff}. 2004. Hybrow:
A~prototype system for computer-aided hypothesis evaluation. \textit{Bioinformatics}
20(1):257--264.

\bibitem{31-kal} %51
\Aue{Sharpe, W., G.\,J.~Alexander, and J.\,W.~Bailey}. 1998. \textit{Investments}.
Prentice Hall. 962~p.
\bibitem{32-kal} %52
\Aue{Gabbay, D.\,M.} 2012. What is negation as failure? \textit{Logic programs, norms and
action}. Eds. A.~Artikis, R.~Craven, N.\,K.~\!{\!\ptb\c{C}}i{\!\ptb\c{c}}ekli, \textit{et al.}
Lecture notes in computer science ser. Berlin, Heidelberg:
Springer-Verlag. 7360:52--78.
\bibitem{33-kal} %53
\Aue{Briukhov, D.\,O., A.\,E.~Vovchenko, V.\,N.~Zakharov, O.\,P.~Zhelenkova,
L.\,A.~Kalinichenko, D.\,O.~Martynov, N.\,A.~Skvortsov, and S.\,A.~Stupnikov}. 2008.
Arkhitektura promezhutochnogo sloya predmetnykh posrednikov dlya resheniya zadach nad
mnozhestvom integriruemykh neodnorodnykh raspredelennykh informatsionnykh resursov v
gibridnoy grid-infrastrukture virtual'nykh observatoriy [The middle ware architecture of the
subject mediators for problem solving over a set of integrated heterogeneous distributed
information resources in the hybrid grid-infrastucture of virtual observatories].
\textit{Informatika i ee Primeneniya~---
Inform.\ Appl.} 2(1):2--34.
\bibitem{35-kal} %54
ATL Project. 2013. Available at: {\sf http://www.eclipse.org/\linebreak atl/} (accessed November~21, 2013).

\end{thebibliography} } }

\end{multicols}

\hfill{\small\textit{Received November 25, 2013}}

\vspace*{-12pt}

\Contr

\noindent
\textbf{Kalinichenko Leonid A.} (b.\ 1937)~--- Doctor of Science in physics and mathematics; 
Head of Laboratory, Institute of Informatics Problems, Russian Academy of Sciences, Moscow 119333, Russian
Federation; 
professor, Faculty of Computational Mathematics and Cybernetics, M.\,V.~Lomonosov 
Moscow State Univer\-sity, Moscow 119991, Russian Federation; leonidandk@gmail.com 

\vspace*{3pt}

\noindent
\textbf{Stupnikov Sergey A.} (b.\ 1978)~--- Candidate of Science (PhD) in technology, 
senior scientist, Institute of Informatics Problems, Russian Academy of Sciences, 
Moscow 119333, Russian Federation; ssa@ipi.ac.ru

\vspace*{3pt}

\noindent
\textbf{Vovchenko Alexey E.} (b.\ 1984)~--- Candidate of Science (PhD) in technology, 
senior scientist, Institute of Informatics Problems, Russian Academy of Sciences,
Moscow 119333, Russian Federation; itsnein@gmail.com

\vspace*{3pt}

\noindent
\textbf{Kovalev Dmitry Yu.} (b.\ 1988)~--- PhD student, Faculty of Computational 
Mathematics and Cybernetics, M.\,V.~Lomonosov Moscow State University; 
programmer, Institute of Informatics Problems, Russian Academy of Sciences,
Moscow 119333, Russian Federation; dm.kovalev@gmail.com


\vspace*{12pt}

\hrule

\vspace*{2pt}

\hrule

%\newpage

\def\tit{КОНЦЕПТУАЛЬНЫЕ ДЕКЛАРАТИВНЫЕ СПЕЦИФИКАЦИИ И~РЕШЕНИЕ ЗАДАЧ В
ОБЛАСТЯХ С~ИНТЕНСИВНЫМ ИСПОЛЬЗОВАНИЕМ ДАННЫХ}

\def\aut{Л. Калиниченко$^1$, С.~Ступников$^2$, А.~Вовченко$^3$, Д.~Ковалев$^4$}


\def\titkol{Концептуальные декларативные спецификации и решение задач в
областях с интенсивным использованием данных}

\def\autkol{Л. Калиниченко, С.~Ступников, А.~Вовченко, Д.~Ковалев}


\titel{\tit}{\aut}{\autkol}{\titkol}

\vspace*{-12pt}

\noindent
$^1$Институт проблем информатики Российской академии наук, Москва, Россия,
leonidandk@gmail.com\\
\noindent
$^2$Институт проблем информатики Российской академии наук,
 Москва, Россия, ssa@ipi.ac.ru\\
\noindent
$^3$Институт проблем информатики Российской академии наук, Москва, Россия,
itsnein@gmail.com\\
\noindent
$^4$Институт проблем информатики Российской академии наук, Москва, Россия,
dm.kovalev@gmail.com

\vspace*{12pt}

\def\leftfootline{\small{\textbf{\thepage}
\hfill ИНФОРМАТИКА И ЕЁ ПРИМЕНЕНИЯ\ \ \ том\ 7\ \ \ выпуск\ 4\ \ \ 2013}
}%
 \def\rightfootline{\small{ИНФОРМАТИКА И ЕЁ ПРИМЕНЕНИЯ\ \ \ том\ 7\ \ \ выпуск\ 4\ \ \ 2013
\hfill \textbf{\thepage}}}


  \Abst{В сфере концептуального моделирования долгое время исследовались
разнообразные нотации, предназначенные для определения семантики
вычислений в терминах предметных областей. Подход <<сущ\-ность--связь>> и
диаграммы UML позволяют определять семантику лишь неформально.
Онтологические языки, основанные на дескриптивной логике, разрабатывались
для формализации семантики данных. Однако сейчас общепризнано, что одной
лишь семантики данных недостаточно~--- требуется
еще и представление
алгоритмов анализа данных для спецификации данных и поведения в одной
парадигме. Более того, все усиливающееся разнообразие раз\-но\-струк\-ту\-ри\-ро\-ван\-ных 
моделей данных вызывает
потребность в их
унифицированной, интегрированной абстракции для получения спецификаций,
неза-\linebreak\vspace*{-12pt}}

\Abstend{висимых от реальных данных в предметных областях с интенсивным
использованием данных.
  С целью преодоления названных недостатков предлагается новый подход к
применению семантически различных языков на правилах (диалектов) для
создания интероперабельных концептуальных спецификаций над различными
системами на правилах. Подход основан на на технике преобразования
логических программ, рекомендованной \textit{Форматом обмена правилами}
(RIF) W3C. Этот подход гармонично сочетается со спецификациями,
предназначенными для определения семантических посредников на правилах,
обеспечивающих интеграцию неоднородных баз данных. Определена
инфраструктура, реализующая мультидиалектные концептуальные
спецификации при помощи интероперабельных систем на правилах и систем
поддержки посредников. Представлен подтверждающий предложенные
концепции прототип инфраструктуры, основанный на системе поддержки
посредников СИНТЕЗ и стандарте RIF. Подход к мультидиалектной
концептуализации предметной области, делегированию правил,
интероперабельности программ на правилах и посредников подробно
рассмотрен и проиллюстрирован на реальном примере NP-пол\-ной задачи в
финансовой области. Результаты исследования свидетельствуют о
применимости подхода и инфраструктуры для концептуального,
декларативного, независимого от ресурсов и повторно используемого анализа
данных в различных предметных областях.}

  \KW{концептуальная спецификация; W3C RIF; логические языки на
правилах; СИНТЕЗ; интеграция баз данных, посредники; RIF-BLD; RIF-CASPD;
мультидиалектная инфраструктура; делегирование правил}

\DOI{10.14357/19922264130412}

%\vspace*{9pt}

%\Ack

Работа выполнена при поддержке Российского фонда фундаментальных
исследований (проект  11-07-00402-а) и Программы фундаментальных 
исследований Президиума Российской академии наук.


\renewcommand{\bibname}{\protect\rmfamily Литература}
%\renewcommand{\bibname}{\large\protect\rm References}

  \begin{multicols}{2}

{\small\frenchspacing
{%\baselineskip=10.8pt
\addcontentsline{toc}{section}{References}
\begin{thebibliography}{99}

%Список литературы к русскоязычной части

\bibitem{2-kal-1}
The fourth paradigm: Data-intensive scientific discovery~/ Eds. T.~Hey, S.~Tansley,
K.~Tolle.~--- Redmond: Microsoft Research, 2009. 252~p.
\bibitem{1-kal-1}
Challenges and opportunities with big data.  A~community white paper developed by
leading researchers across the United States, 2012. Available at: {\sf
http://cra.org/ccc/docs/init/bigdatawhitepaper.pdf} (accessed November~21, 2013).

\bibitem{3-kal-1}
\Au{Kappel G., Wimmer~M., Retschitzegger~W., Schwinger~W.} Leveraging
model-based tool integration by conceptual modeling techniques~// The evolution of
conceptual modeling. Eds. R.~Kaschek and L.\,M.\,L.~Delcambre.~---
 Lecture notes in computer science ser.~---
Berlin, Heidelberg: Springer-Verlag, 2011. Vol.~6520. P.~254--284.
\bibitem{4-kal-1}
\Au{Kalinichenko L.\,A., Stupnikov~S.\,A., Martynov~D.\,O.} SYNTHESIS: A~language
for canonical information modeling and mediator definition for problem solving in
heterogeneous information resource environments.~--- M.: IPI RAN, 2007. 171~p.
\bibitem{5-kal-1}
\Au{Kalinichenko L.\,A., Briukhov D.\,O., Martynov~D.\,O., Skvortsov~N.\,A.,
Stupnikov~S.\,A.} Mediation framework for enterprise information system infrastructures~//
9th Conference (International) on Enterprise Information Systems ICEIS 2007 Proceedings.~--- 
Funchal, 2007. Vol. Databases and information systems integration. P.~246--251.
\bibitem{6-kal-1}
\Au{Kalinichenko L.\,A.} Methods and tools for equivalent data model mapping
construction~// {Advances in database technology~---
EDBT'90}. Eds. F.~Buncilhon, C.~Thanos, and D.~Tsichritzis.~--- Lecture notes in
computer science ser.~--- Berlin, Heidelberg: Springer-Verlag, 1990. Vol.~416. P.~92--119.
\bibitem{7-kal-1}
\Au{Kalinichenko L.\,A., Stupnikov S.\,A.} Synthesis of the canonical models for database
integration preserving semantics of the value inventive data models~// 
{Advances in database and information systems}~/ Eds. T.~Morzy,
T.~H$\ddot{\mbox{a}}$rder, R.~Wrembel,
\textit{et al.} Lecture notes in computer science ser. Berlin, Heidelberg: Springer-Verlag, 
2012. Vol.~7503. P.~223--239.
\bibitem{8-kal-1}
RIF overview. W3C working group note~/ Eds. H.~Boley, M.~Kifer.~--- 2nd ed.,
2013. Available at: {\sf http://www.w3.org/TR/rif-overview/} (accessed November~21,
2013).

\bibitem{10-kal-1} %9
\Au{Kalinichenko L., Stupnikov S., Vovchenko~A., Kovalev~D.} Rule-based multi-dialect
infrastructure for conceptual problem solving over heterogeneous distributed information
resources. New trends in databases and information systems~// Selected Papers of the 17th
European Conference on Advances in Databases and Information Systems and Associated
Satellite Events. Advances in Intelligent Systems and Computing, 2013. Vol.~241.
P.~61--68.

\bibitem{9-kal-1} %10
\Au{Leone N., Pfeifer~G., Faber~W.,  Eiter~T., Gottlob~G., Perri~S., Scarcello~F.} The
DLV system for knowledge representation and reasoning~// ACM Trans.
Comput. Log., 2006. Vol.~7. No.\,3. P.~499--562.

\bibitem{11-kal-1}
\Au{Kalinichenko L.\,A., Stupnikov S.\,A.} Constructing of mappings of heterogeneous
information models into the canonical models of integrated information systems~// ADBIS
2008 Proceedings.~--- Pori: Tampere University of Technology, 2008. P.~106--122.
\bibitem{12-kal-1}
\Au{Abrial J.-R.} The B-Book: Assigning programs to meanings.~--- Cambridge:
Cambridge University Press, 1996. 816~p.
\bibitem{13-kal-1}
\Au{Скворцов Н.\,А.} Отображение моделей данных NoSQL в объектные
спецификации~// Труды 14-й Всеросс. научн. конф. <<Электронные библиотеки:
перспективные методы и технологии, электронные коллекции>> RCDL 2012.~---
Переславль-Залесский: Ун-т г.~Переславля, 2012. Т.~934. С.~53--62.
\bibitem{14-kal-1}
\Au{Ступников С.\,А.} Отображение графовой модели данных в каноническую
объектно-фреймовую информационную модель при создании систем интеграции
неоднородных информационных ресурсов~// Труды 15-й Всеросс. научн. конф.
<<Электронные библиотеки:  перспективные методы и технологии, электронные
коллекции>> RCDL 2013.~--- Ярославль: Ярославский гос. ун-т
им.\ П.\,Г.~Демидова, 2013. С.~193--202.
\bibitem{15-kal-1}
\Au{Скворцов Н.\,А.} Отображение модели данных RDF в каноническую модель
предметных посредников~// Труды 15-й Всеросс. научн. конф. <<Электронные
библиотеки: перспективные методы и технологии, электронные коллекции>> RCDL
2013.~--- Ярославль: Ярославский гос. ун-т им. П.\,Г.~Демидова,
2013. С.~202--209.
\bibitem{16-kal-1}
\Au{Ступников С.\,А.} 2013. Верифицируемое отображение модели данных,
основанной на многомерных массивах, в объектную модель данных~// Информатика и
её применения, 2013. Т.~7. Вып.~3. С.~22--34.
\bibitem{17-kal-1}
\Au{Fagin R., Kolaitis~P.\,G., Miller~R.\,J., Popa~L.} Data exchange: Semantics and query
answering~// Theor. Comput. Sci., 2005. Vol.~336. P.~89--124.
\bibitem{18-kal-1}
\Au{Lenzerini M.} Data integration: A~theoretical perspective~// ACM Symposium on
Principles of Database Systems (PODS) Proceedings, 2002. P.~233--246.

\bibitem{34-kal-1} %19
\Au{Kalinichenko L.\,A., Martynov D.\,O., Stupnikov~S.\,A.} Query rewriting using views
in a typed mediator environment~// {Advances in databases and information
systems}~/ Eds. G.~Gottlob, A.~Benczur, J.~Demetrovics.~--- Lecture notes in
computer science ser.~--- Berlin, Heidelberg: Springer-Verlag, 2004. Vol.~3255.
P.~37--53.

\bibitem{20-kal-1} %20
RIF basic logic dialect. W3C recommendation~/
Eds. H.~Boley, M.~Kifer.~--- 2nd ed., 2013.  Available at: {\sf http://www.w3.org/TR/rif-bld/}
(accessed November~21, 2013).
\bibitem{21-kal-1} %21
RIF production rule dialect. W3C recommendation~/
Eds. C.~De Sainte Marie, G.~Hallmark, A.~Paschke.~--- 2nd ed., 2013. Available at: {\sf
http://www.w3.org/TR/rif-prd/} (accessed November~21, 2013).

\bibitem{19-kal-1} %22
RIF framework for logic dialects. W3C recommendation~/
Eds. H.~Boley,  M.~Kifer.~--- 2nd ed., 2013. Available at: {\sf http://www.w3.org/TR/rif-fld/}
(accessed November~21, 2013).

\bibitem{23-kal-1} %23
RIF core logic programming dialect based on the well-founded semantics~/
Ed. M.~Kifer.~--- 2010. Available at: {\sf http://ruleml.org/rif/RIF-CLPWD.html}
(accessed November~21, 2013).

\bibitem{25-kal-1} %24
\Au{Van Gelder A., Ross K.\,A., Schlipf~J.\,S.} The well-founded semantics for general
logic programs~// J.~ACM, 1991. Vol.~8. No.\,3. P.~620--650.

\bibitem{22-kal-1} %25
RIF core answer set programming dialect~/
Eds. S.~Heymans, M.~Kifer.~--- 2009. Available at: {\sf
http://ruleml.org/rif/RIF-CASPD.html} (accessed November~21, 2013).



\bibitem{26-kal-1} %26
\Au{Gelfond M.,  Lifschitz~V.} The stable model semantics for logic programming~// Logic
programming. 5th Conference and Symposium Proceedings, 1988. P.~1070--1080.

\bibitem{27-kal-1} %27
RIF RDF and OWL compatibility. W3C Recommendation~/
Eds.\ J.~De Bruijn, C.~Welty.~--- 2nd ed., 2013. Available at: {\sf
http://www.w3.org/TR/rif-rdf-owl/} (accessed November~21, 2013).

\bibitem{28-kal-1} %28
OWL 2 Web ontology language structural specification and functional-style syntax.  W3C 
recommendation~/ B.~Motik, P.\,F.~Patel-Schneider, B.~Parsia.~--- 2nd ed., 2012. Available at: {\sf
http://www.w3.org/TR/owl2-syntax/} (accessed November~21, 2013).



\bibitem{29-kal-1}
\Au{Shvaiko P., Euzenat~J.} A survey of schema-based matching approaches~// J.~Data
Semantics, 2005. Vol.~IV. P.~146--171.

\bibitem{30-kal-1} %30
\Au{Abiteboul S., Bienvenu M., Galland~A., \textit{et al}.}  A rule-based language for
Web data management~// 30th ACM Symposium on Principles of Database Systems
Proceedings.~--- ACM Press, 2011. P.~283--292.

\bibitem{24-kal-1} %31
\Au{Gelfond M.} Answer sets. Handbook of knowledge representation.~--- Elsevier. 2008.
P.~285--316.


\bibitem{40-kal-1} %32
\Au{Motik B., Rosati~R.} Closing semantic Web ontologies. University of Manchester
Report, 2007. Available at: {\sf
http://www.cs.ox.ac.uk/people/boris.motik/pubs/\linebreak mr06closing-report.pdf} (accessed
November~21, 2013).

\bibitem{36-kal-1} %33
\Au{Chen P.\,P.,  Thalheim~B., Wong~L.\,Y.}  Future directions of conceptual modeling~//
Conceptual modeling~/ Eds. P.\,P.~Chen, J.~Akoka, B.~Thalheim.~---
Lecture notes in computer science ser.~--- Berlin, Heidelberg:
Springer-Verlag, 1999. Vol.~1565. P.~287--301.

\bibitem{37-kal-1} %34
\Au{Costal D., G$\acute{mbox{o}}$mez C., Guizzardi~G.} Formal semantics and
ontological analysis for understanding subsetting, Specialization and redefinition of
associations in UML~// Conceptual modeling~--- ER 2011~/
Eds. M.~Juesfeld, L.~Delcambre, T.\,W.~Ling.~--- Lecture notes in computer
science ser.~--- Berlin, Heidelberg: Springer-Verlag, 2011.
Vol.~6998. P.~189--203.

\bibitem{39-kal-1} %35
\Au{Mart$\iota$nez Y., Cachero~C., Matera~M., Abrahao~S.,  Lujan~S.} Impact of MDE
approaches on the maintainability of Web applications: An experimental evaluation~//
Conceptual Modeling~--- ER 2011~/ Eds. M.~Jeusfeld, L.~Delcambre, T.\,W.~Ling.~---
 Lecture notes in computer
science ser.~--- Berlin, Heidelberg: Springer-Verlag, 2011. Vol.~6998. P.~233--246.

\bibitem{38-kal-1} %36
\Au{Castro L.,  Bai$\tilde{\mbox{a}}$o~F.,  Guizzardi~G.}  A~semantic oriented method
for conceptual data modeling in OntoUML based on linguistic concepts~// Conceptual
modeling~--- ER 2011~/ Eds. M.~Jeusfeld, L.~Delcambre,  T.\,W.~Ling.~--- Lecture notes in computer
science ser.~--- Berlin, Heidelberg: Springer-Verlag, 2011. Vol.~6998. P.~486--494.

\bibitem{41-kal-1} %37
\Au{Fillottrani P.\,R., Franconi~E.,  Tessaris~S.} The ICOM 3.0 intelligent conceptual
modelling tool and methodology. Semantic Web.~--- IOS Press, 2011. P.~1--14. Available
at: {\sf http://www.semantic-web-journal.\linebreak net/sites/default/files/swj105\_1.pdf} (accessed
November~21, 2013).

\bibitem{42-kal-1} %38
\Au{Maus C.,  Rybacki~S., Uhrmache~A.\,M.} Rule-based multi-level modeling of cell
biological systems~// BMC Syst. Biol., 2011. Vol.~5. P.~166. Available at: {\sf
http://www.biomedcentral.com/content/pdf/1752-0509-5-166.pdf} (accessed
November~21, 2013).

\bibitem{43-kal-1} %39
\Au{Calvanese D., De Giacomo G.,  Lembo~D., Lenzerini~M.,  Poggi~A.,  Rosati~R.}
MASTRO-I: Efficient integration of relational data through DL ontologies~// 2007
Workshop (International) on Description Logic (DL 2007) Proceedings. CEUR Workshop
Proceedings, 2007. P.~250.

\bibitem{44-kal-1} %40
\Au{Cosentino V., Del Fabro M.\,D., El~Ghali~A.} A~model driven approach for bridging
ILOG rule language and RIF~// RuleML 2012 Proceedings. CEUR Workshop
Proceedings, 2012. P.~874. Available at: {\sf http://ceur-ws.org/Vol-874/paper9.pdf}
(accessed November~21, 2013).

\bibitem{45-kal-1} %41
\Au{Gonzalez-Moriyon G.,  Polo~L.,  Berrueta~D.,  Tejo-Alonso~C.}  Final steel industry
public demonstrators. ONTORULE deliverable D5.5.~--- 2012. Available at: {\sf
http://ontorule-project.eu/outcomes\%3Ffunc=fileinfo\&id=94.html} (accessed
November~21, 2013).




\bibitem{48-kal-1} %42
\Au{Loo B.\,T.,  Condie~T., Garofalakis~M., Gay~D.\,E., Hellerstein~J.\,M., Maniatis~P.,
Ramakrishnan~R.,  Roscoe~T., Stoica~I.} Declarative networking: Language, execution and
optimization~// ACM SIGMOD Conference Proceedings, 2006. P.~97--108.

\bibitem{47-kal-1} %43
\Au{Grumbach S., Wang~F.} Netlog, a rule-based language for distributed programming~//
Practical aspects of declarative
languages~/ Eds.\ M.~Carro, R.~Pe$\tilde{\mbox{n}}$a.~--- Lecture notes
in computer science ser.~--- Berlin, Heidelberg: Springer-Verlag, 2010. Vol.~5937. P.~88--103.

\bibitem{46-kal-1} %44
\Au{Alvaro~P., Marczak W.\,R., Conway N., Hellerstein~J.\,M., Maier~D., Sears~R.}
  Dedalus: Datalog in time and space~//
Datalog reloaded~/
Eds. O.~de Moor, G.~Gottlob, T.~Furche, and A.~Sellers.~--- Lecture notes in computer science ser.~---
Berlin, Heidelberg: Springer-Verlag,  2010. Vol.~6702. P.~262--281.

\bibitem{49-kal-1} %45
\Au{Grossmann G., Thiagarajan~R.,  Schrefl~M., Stumptner~M.} Conceptual modeling
approaches for dynamic Web service composition~// The evolution of conceptual
modeling~/ Eds. R.~Kaschek, L.\,M.\,L.~Delcambre.~--- Lecture notes in computer science ser.~---
Berlin, Heidelberg: Springer-Verlag, 2011. Vol.~6520. P.~180--204.

\bibitem{50-kal-1} %46
\Au{Henderson-Sellers B.} Random thoughts on multi-level conceptual modelling~// The
evolution of conceptual modeling~/ Eds. R.~Kaschek, L.\,M.\,L.~Delcambre.~---
 Lecture notes in computer science ser.~---
Berlin, Heidelberg: Springer-Verlag, 2011. Vol.~6520. P.~93--116.

\bibitem{51-kal-1} %47
\Au{Falbo R.\,A., Ruy F.\,B., Moro~R.\,D.} Using ontologies to add semantics to a software
engineering environment~// SEKE 2005 Proceedings.~--- Curran Associates, 2005.
P.~151--156.

\bibitem{52-kal-1} %48
Berre A.,  Missikoff~M. (Moderators).  Panel: Modeling for the future Internet~//
Conceptual modeling~--- ER 2011~/ 
Eds. M.~Jeusfeld, L.~Delcambre, T.\,W.~Ling.~--- Lecture notes in
computer science ser.~--- Berlin, Heidelberg: Springer-Verlag, 2011. Vol.~6998. P.~526--527.

\bibitem{53-kal-1} %49
\Au{Porto F., Spaccapietra~S.} Data model for scientific models and hypotheses~//
The evolution of conceptual modeling~/
Eds. R.~Kaschek, L.\,M.\,L.~Delcambre.~--- Lecture notes in computer science ser.~---
Berlin, Heidelberg: Springer-Verlag, 2011. Vol.~6520. P.~285--305.
\bibitem{54-kal-1} %50
\Au{Racunas S.\,A., Shah N.\,H., Albert~I., Fedoroff~N.\,V.} 2004. Hybrow: A~prototype
system for computer-aided hypothesis evaluation~// Bioinformatics, 2004. Vol.~20. No.\,1.
P.~257--264.

\bibitem{31-kal-1} %51
\Au{Sharpe W., Alexander G.\,J., Bailey~J.\,W.} Investments.~--- Prentice Hall, 1998.
962~p.
\bibitem{32-kal-1} %52
\Au{Gabbay D.\,M.} What is negation as failure~// Logic Programs, norms and action~/
Eds. A.~Artikis, R.~Craven, N.\,K.~\!{\!\ptb\c{C}}i{\!\!\ptb\c{c}}ekli, \textit{et al.}~---
Lecture notes in computer science ser.~--- Berlin, Heidelberg:
Springer-Verlag, 2012. Vol.~7360. P.~52--78.
\bibitem{33-kal-1} %53
\Au{Брюхов Д.\,О., Вовченко А.\,Е., Захаров~В.\,Н., Желенкова~О.\,П.,
Калиниченко~Л.\,А., Мартынов~Д.\,О., Скворцов~Н.\,А., Ступников~С.\,А.}
Архитектура  промежуточного слоя предметных посредников для решения \mbox{задач} над
множеством интегрируемых неоднородных распределенных информационных
ресурсов в гиб\-рид\-ной  грид-инф\-ра\-струк\-ту\-ре виртуальных обсерваторий~//
Информатика и её применения, 2008. Т.~2. Вып.~1. С.~2--34.

\bibitem{35-kal-1} %54
ATL Project. 2013. Available at: {\sf http://www.eclipse.org/\linebreak atl/} (accessed November~21,
2013).

 \label{end\stat}


\end{thebibliography}
} }

\end{multicols}

\hfill{\small\textit{Поступила в редакцию 25.11.13}}
%\renewcommand{\bibname}{\protect\rm Литература}
\renewcommand{\figurename}{\protect\bf Рис.}