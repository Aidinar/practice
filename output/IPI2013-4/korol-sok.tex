\def\stat{kor-gr}

\def\tit{ПРЕДЕЛЬНАЯ ТЕОРЕМА ДЛЯ ГЕОМЕТРИЧЕСКИХ СУММ НЕЗАВИСИМЫХ НЕОДИНАКОВО 
РАСПРЕДЕЛЕННЫХ СЛУЧАЙНЫХ ВЕЛИЧИН И~ЕЕ ПРИМЕНЕНИЕ К~ПРОГНОЗИРОВАНИЮ ВЕРОЯТНОСТИ 
КАТАСТРОФ В~НЕОДНОРОДНЫХ ПОТОКАХ ЭКСТРЕМАЛЬНЫХ СОБЫТИЙ$^*$}

\def\titkol{Предельная теорема для геометрических сумм независимых неодинаково 
распределенных случайных величин}
% и ее применение к прогнозированию вероятности 
%катастроф в неоднородных потоках экстремальных событий}

\def\autkol{М.\,Е.~Григорьева,  В.\,Ю.~Королев, И.\,А.~Соколов}

\def\aut{М.\,Е.~Григорьева$^1$,  В.\,Ю.~Королев$^2$, И.\,А.~Соколов$^3$}

\titel{\tit}{\aut}{\autkol}{\titkol}

{\renewcommand{\thefootnote}{\fnsymbol{footnote}}
\footnotetext[1] {Работа поддержана Российским фондом фундаментальных исследований (проекты 
11-01-00515-а, 11-07-00112-а, 12-07-00115-а).}}

\renewcommand{\thefootnote}{\arabic{footnote}}
\footnotetext[1]{Parexel International, maria-grigoryeva@yandex.ru} 
\footnotetext[2]{Факультет
вычислительной математики и кибернетики Московского государственного
университета им.\ М.\,В.~Ломоносова; Институт проблем информатики
Российской академии наук, victoryukorolev@yandex.ru} 
\footnotetext[3]{Институт проблем информатики Российской академии
наук, ipiran@ipiran.ru}


%\renewcommand{\r}{\mathbb R}
%\newcommand{\N}{\mathbb N}
%\renewcommand{\P}{{\sf P}}
%\newcommand{\E}{{\sf E}}
%\newcommand{\D}{{\sf D}}



%\newcommand{\I}{\mathbb{I}}
%\newcommand{\betm}{{\beta_{m+1+\delta}}}
%\newcommand{\bet}{\beta_{2+\delta}}
%\renewcommand{\endproof}{\hfill$\Box$}
%\renewcommand{\phi}{\varphi}
%\newcommand{\la}{\lambda}
%\newcommand{\si}{{\rm Si}\:}
%\renewcommand{\Re}{{\rm Re}\:}


 
\vspace*{-16pt}

\Abst{Рассматривается задача прогнозирования вероятностей катастроф в 
неоднородных потоках экстремальных событий. Статья развивает и обобщает 
некоторые методы, предложенные авторами в предыдущих работах. Поток экстремальных 
событий рассматривается как маркированный точечный случайный процесс с 
необязательно одинаково распределенными интервалами между точками (событиями). 
Основой предлагаемых обобщений служат предельные теоремы для геометрических 
случайных сумм независимых неодинаково распределенных случайных величин и 
теория Бал\-ке\-мы\,--\,Пи\-канд\-са\,--\,Де Ха\-ана. Рассмотрена конструкция, в рамках 
которой в качестве предельного распределения для гео\-мет\-ри\-ческих случайных сумм 
независимых неодинаково распределенных случайных величин возникает 
распределение Вей\-бул\-ла--Гне\-ден\-ко. Эффективность методов иллюстрируется 
на примере их применения к прогнозированию момента столкновения Земли с 
потенциально опасным астероидом на основе данных Центра по малым планетам 
Гарвардского университета.} %~\cite{Atkinson2001}.

\vspace*{-5pt}

\KW{катастрофа; экстремальное событие; случайная
сумма; геометрическая сумма; закон больших чисел; распределение
Вей\-бул\-ла--Гне\-ден\-ко;
теорема Бал\-ке\-мы--Пи\-канд\-са--Де Ха\-ана; обобщенное распределение Парето}

\vspace*{-3pt}

\DOI{10.14357/19922264130402}

\vskip 8pt plus 9pt minus 6pt

      \thispagestyle{headings}

      \begin{multicols}{2}

            \label{st\stat}


\section{{Введение. Постановка задачи. Определение экстремального
процесса}}

В данной статье рассматривается задача прогнозирования \textit{вероятностных} 
характеристик катастроф в неоднородных потоках
экстремальных событий. Рассмотрим некоторую систему, подвергающуюся
влиянию некоторого фактора. Предположим, что сила воздействия этого
фактора на систему в каж\-дый момент времени характеризуется некоторым
числом, причем это число изменяется во времени. Это может быть:
\begin{itemize}
\item[$\bullet$] финансовая система, которая характеризуется финансовым
индексом, таким как DAX, NIKKEI, NASDAQ и~т.\,п.; при этом резкие
колебания индекса неблагоприятны и свидетельствуют о тех или иных
кризисных явлениях;
\item[$\bullet$] экологическая система, например среда обитания человека, в
частности состояние жилых помещений в местностях, подверженных
наводнениям или землетрясениям в сейсмоопасных зонах, которое
зависит соответственно от силы подземных толчков и уровня подъема
воды;
\item[$\bullet$] социальная система, которая подвержена по\-литической
неустойчивости или воздействию терро\-ристических организаций; при
этом в качестве числовой характеристики активности неблагоприятных
воздействий может выступить, к примеру, число упоминаний некоторых
соответствующих ключевых слов или фраз в социальных информационных
сетях или сетях \mbox{связи};
\item[$\bullet$] наконец, общее состояние планеты Земля, зависящее от
расстояния, на которое подлетают к ней потенциально опасные
космические объекты~--- астероиды или кометы.
\end{itemize}

При этом естественно возникает задача прогнозирования катастроф.
Однако без применения специализированных методов, специально
ориен-\linebreak\vspace*{-12pt}

\pagebreak

\noindent
тированных на противодействие конкретным рискам, практически
никогда нельзя абсолютно точно предсказать силу воздействия фактора
на систему в каждый момент времени в будущем. Другими словами,
будущее развитие фактора непредсказуемо, вследствие чего значение
числа, характеризующего силу воздействия фактора на систему,
рас\-смат\-ри\-ва\-емое как функция времени, целесообразно рассматривать как
\textit{случайный процесс}. Поэтому задача прогнозирования самог$\acute{\mbox{о}}$
момента катастрофы сводится к прогнозированию \textit{значения}
случайного процесса (т.\,е.\ его значения на вполне определенном
элементе множества элементарных исходов) специальными методами, что
чрезвычайно трудоемко и при рассмотрении современных сложных
стохастических систем практически не реализуемо с приемлемой
точностью. 

В~то же время возникает вполне реальная и важная задача прогнозирования 
\textit{распределения} указанного случайного процесса в те или иные моменты 
времени, т.\,е.\ задача прогнозирования его статистических свойств. 
В~результате решения этой задачи появляется возможность правильно оценить 
уровни угрозы в каждой конкретной ситуации. 
{ %\looseness=1

}

Некоторым методам решения последней задачи и посвящена
данная статья.


Предположим, что очень большие изменения случайного процесса,
характеризующего воздействие фактора на систему, неблагоприятно
влияют на систему и могут вызвать ее необратимые изменения. Вместе с
тем малые флуктуации случайного процесса, характеризующего
воздействие фактора на систему, вполне допустимы (в таких случаях
говорят о <<фоновом значении>> фактора). Поэтому с целью
предсказания катастроф разумно рассматривать не все изменения
случайного процесса, а лишь такие, величина которых превышает
некоторый \textit{потенциально опасный порог}.

Будем говорить, что моменты превышений изменениями случайного процесса 
потенциально опасного порога в совокупности с самими значениями этих превышений 
образуют \textit{экстремальный случайный процесс}. Другими словами, экстремальным 
процессом будем называть маркированный точечный процесс $\{(\tau_i, 
X_i)\}_{i\geqslant1}$, где $\{\tau_i\}_{i\geqslant1}$~--- точечный случайный 
процесс, а $\{X_i\}_{i\geqslant1}$~--- случайные величины. Далее по смыслу 
задачи будет предполагаться, что $X_i\hm>0$, $i\hm=1,2,\ldots$

Среди всех превышений случайным процессом потенциально опасного
порога лишь некоторые очень большие влекут катастрофические
последствия. Поэтому наряду с \textit{потенциально опасным порогом}
рассмотрим \textit{критический порог}, превышение которого
экстремальным процессом и будем считать \textit{катастрофой}.

Для удобства точку отсчета (нуль временн$\acute{\mbox{о}}$й шкалы) поместим в то
время, которое будем считать <<настоящим>>. Тем самым <<настоящее>>
характеризуется значением $t\hm=0$.

Поскольку по условию экстремальный процесс считается случайным, то
\textit{нельзя} точно предсказать момент наступления очередной
катастрофы. Однако можно вычислить или оценить \textit{вероятности
наступления катастрофы} в течение некоторого интервала времени
$[0,\,\tau)$, где $\tau\hm>0$. Если $T$~--- момент наступления
катастрофы, то событие <<катастрофа наступила в течение интервала
времени $[0,\,\tau)$>> эквивалентно тому, что $T\hm<\tau$. В качестве
\textit{исходных данных} будем использовать информацию о развитии
экстремального процесса на некотором интервале времени
$[t_0,\,t_1]$, где $t_0\hm<t_1\hm<0$.

Простейшее (примитивное) решение задачи об отыскании вероятности
наступления катастрофы в течение интервала времени $[0,\,\tau)$ при
условии $\tau\hm<t_1\hm-t_0$ выглядит так.

Разобьем интервал времени $[t_0,\,t_1]$ на непересекающиеся
подынтервалы длиной $\tau$. Пусть внутри интервала $[t_0,\,t_1]$
поместилось $N_{\tau}$ подынтервалов длиной~$\tau$. Подсчитаем
количество подынтервалов, внутри каждого из которых наступила хотя
бы одна катастрофа. Пусть таких подынтервалов оказалось ровно~$n_{\tau}$. 
Тогда для вероятности наступления катастрофы в течение
интервала времени $[0,\,\tau)$ справедлива оценка: 

\noindent
\begin{equation}
{\sf P}(T<\tau)\approx\fr{n_{\tau}}{N_{\tau}}\,,\label{e1-kor}
\end{equation} 
основанная на
классическом определении вероятности как (предела) частоты.

Недостатки такой оценки очевидны. Например, $n_{\tau}$ просто может
оказаться равным нулю, что дает тривиально оптимистичную оценку.
Далее, и~$N_\tau$, и~$n_\tau$ могут быть (и, как правило, являются)
слишком маленькими, чтобы обеспечить приемлемую точность оценки.
Более того, од\-ной-един\-ст\-вен\-ной катастрофы может оказаться достаточно
для полного уничтожения сис\-те\-мы, так что дальнейший сбор информации
просто может оказаться невозможным.

%\vspace*{-36pt}

\section{Метод прогнозирования вероятностей катастроф 
в~неоднородных потоках экстремальных событий}

\subsection{Особенности метода}

К сожалению, именно оценками типа~(\ref{e1-kor}) за\-час\-тую пользуются на
практике для расчетов, связанных с так называемыми большими рисками
в страховании, например при страховании промышленных рисков,
связанных с крупными авариями и экологическими катастрофами. 
В~данной статье описан метод оценивания указанных вероятностей
наступления катастроф, основанный на довольно сложных математических
моделях, но свободный от указанных недостатков. Особенность этого
метода заключается в том, что для того, чтобы прогнозировать
возможности наступления катастроф, необязательно иметь статистику
самих \textit{катастроф}.

Простейший вариант этого метода описан в работах~[1--3] 
и книгах~\cite{KorolevSokolov2008, KorolevShorgin2011}, где предполагалось,
что экстремальный процесс является маркированным процессом
восстановления. В~указанных работах предполагалось, что моменты
$\tau_1,\tau_2,\ldots$ превышений исходным процессом потенциально
опасного порога образуют процесс восстановления. Это означает, что
случайные величины
\begin{equation}
\zeta_i=\tau_i-\tau_{i-1},\ \ \ i=1,2,\ldots, \ \ \
\tau_0=0\,,\label{e2-kor}
\end{equation} 
независимы и имеют одинаковое распределение, т.\,е.\ 
подчиняются одним и тем же статистическим закономерностям.
Другими словами, интенсивность потока экстремальных событий
считалась постоянной. В~то же время в реальных сложных системах,
которые в подавляющем большинстве случаев не являются информационно
и/или энергетически замкнутыми и подвержены влиянию внешней среды,
интенсивности потоков информативных со-\linebreak бытий не являются постоянными.
Например, при адекватном прогнозировании поведения фи\-нан\-совых
индексов ключевую роль играет пред\-став\-ление о том, что интенсивности
потоков ин\-формативных событий на финансовых рынках являются
случайными~\cite{Korolevetal2013}. Отказ от предположения о
постоянстве интенсивности потока экстремальных событий естественно
приводит к необходимости предположить, что случайные величины~(\ref{e2-kor})
имеют неодинаковое распределение. Именно такое обобщение методов,
предложенных в работах~[2--5], и рассматривается в данной работе.

Обозначим величину превышения исходным процессом потенциально
опасного порога в момент~$\tau_i$ символом~$X_i$, $i\hm=1,2,\ldots$
Будем считать что $X_1,X_2,\ldots$~--- независимые и одинаково
распределенные случайные величины. Это означает, что значения этих
случайных величин подчиняются одним и тем же статистическим
закономерностям, характеризуемым \textit{функцией распределения}
$$
F(x)={\sf P}(X_i<x)\,,\enskip -\infty<x<\infty\,,\enskip i=1,2,\ldots
$$
Будем считать, что последовательность $X_1,X_2,\ldots$ статистически
независима от последовательности $\tau_1,\tau_2,\ldots$

Пусть $x_0$~--- критический порог, превышение которого значением~$X_i$ и есть 
катастрофа (т.\,е.\ катастрофическое событие формально записывается в виде 
неравенства $X_i\hm\geqslant x_0$).

Очевидно, что время~$T$ наступления катастрофы (т.\,е.\ время
первого превышения уровня~$x_0$ ка\-кой-ли\-бо из величин~$X_i$) можно
представить в виде геометрической случайной суммы
\begin{equation}
T=\sum\limits_{j=1}^{N}\zeta_j\,,\label{e3-kor}
\end{equation} 
где случайные величины~$\zeta_j$ определены соотношением~(\ref{e2-kor}), а $N$~--- это случайная
величина, имеющая геометрическое распределение с параметром 
$$
{\sf P}(X_i<x_0)=F(x_0)\,.
$$ 
Это означает, что
$$
{\sf P}(N=k)=\left(F(x_0)\right)^{k-1}\left(1-F(x_0)\right)\,,\enskip
k=1,2,\ldots
$$
При этом в силу независимости последовательностей $X_1,X_2,\ldots$\ и
$\tau_1,\tau_2,\ldots$\ число~$N$ слагаемых в сумме~(\ref{e3-kor}) независимо от
самих слагаемых $\zeta_1,\zeta_2,\ldots$\ При этом принципиальным
отличием гео\-мет\-ри\-че\-ских случайных сумм, рассматриваемых здесь, от
гео\-мет\-ри\-че\-ских сумм в традиционном понимании 
(см., например,~\cite{Kalashnikov1997, KorolevBeningShorgin2011}) является то, что в
данном случае слагаемые имеют \textit{неодинаковое} распределение,
тогда как в указанных классических книгах изуча\-лись геометрические
суммы \textit{одинаково} распределенных слагаемых и, соответственно,
использовались методы, ориентированные именно на такую ситуацию.

В рамках подхода, рассматриваемого в данной статье, краеугольными
камнями являются два тео\-ре\-ти\-че\-ских результата. Первый из них~---
версия закона больших чисел для случайных сумм неодинаково
распределенных случайных величин (см.\ теорему~1 ниже), обосновывающая
использование распределения Вей\-бул\-ла--Гне\-ден\-ко в качестве мо-\linebreak дели
распределения интервалов времени между катастрофа\-ми. Второй~---
теорема Бал\-ке\-ма\,--\,Пи\-канд\-са\,--\,Де Ха\-ана (см.\ теорему~3 ниже), обосновывающая
использование обобщенного распределения Парето в качестве модели
распределения критиче\-ских значений неблагоприятного фактора. Эти два
общих результата являются основой предлагаемого метода.

\subsection{Вспомогательные результаты}

Пусть $\xi_1,\xi_2,\ldots$~--- необязательно одинаково распределенные случайные 
величины. Для каж\-до\-го натурального $n\geqslant1$ положим 
$$
S_n=\xi_1+\xi_2+\cdots+\xi_n\,.
$$ 
Рассмотрим последовательность целочисленных 
неотрицательных случайных величин $\{N_n\}_{n\geqslant1}$ и будем считать, что 
при каждом~$n$ случайные величины~$N_n$, $\xi_1,\xi_2,\ldots$ независимы в 
совокупности. Более того, предположим, что
\begin{equation}
N_n\longrightarrow \infty\ \mbox{ по вероятности при }
n\to\infty\,.\label{e4-kor}
\end{equation}
Условие~(\ref{e4-kor}) означает, что ${\sf P}(N_n\leqslant m)\hm\longrightarrow 0$ при
$n\hm\to\infty$ для любого $m\hm>0$. Везде далее символ~$\Longrightarrow$ будет 
обозначать сходимость по распределению.

\smallskip

\noindent
\textbf{Лемма 1.} \textit{Пусть для некоторой последователь\-ности положительных чисел 
$\{b_n\}_{n\geqslant1}$ выполнены условия $b_n\hm\to\infty$ при $n\hm\to\infty$ и
\begin{equation}
\fr{S_n}{b_n}\Longrightarrow 1\,,\enskip n\to\infty\,.\label{e5-kor}
\end{equation}
Предположим, что выполнено условие~$(\ref{e4-kor})$. Для того чтобы при $n\hm\to\infty$ имела 
место сходимость случайных сумм $S_{N_n}$, нормированных некоторой 
последовательностью положительных чисел $\{d_n\}_{n\geqslant1}$ такой, что 
$d_n\hm\to\infty$ при $n\hm\to\infty$, к некоторой случайной величине~$Z$:
\begin{equation}
\fr{S_{N_n}}{d_n}\Longrightarrow Z\,,\label{e6-kor}
\end{equation}
необходимо и достаточно, чтобы}
\begin{equation}
\fr{b_{N_n}}{d_n}\Longrightarrow Z\,,\enskip n\to\infty\,.\label{e7-kor}
\end{equation}


\smallskip

\noindent
\textbf{Замечание~1.} В~силу вырожденности распределения предельной
случайной величины в~(\ref{e5-kor}) сходимость по распределению~(\ref{e5-kor}) оказывается
эквивалентной сходимости по вероятности: для любого $\epsilon\hm>0$
$$
\lim\limits_{n\to\infty}{\sf P}\left(\left\vert \fr{S_n}{b_n}-1\right\vert >\epsilon\right)=0\,,
$$
которую иногда легче проверять.

\smallskip

\noindent
Д\,о\,к\,а\,з\,а\,т\,е\,л\,ь\,с\,т\,в\,о\ \ леммы~1 приведено в~\cite{Korolev1994}.

\smallskip

\noindent
\textbf{Замечание~2.} Лемма~1 является версией закона больших чисел для
случайных сумм. Согласно классическим законам больших чисел при
увеличении числа слагаемых в рассматриваемых <<средних
арифметических>> информация о конкретном виде распределений
слагаемых затухает, стягиваясь в информацию об одном лишь числе.
Точно такой же эффект наблюдается в лемме~1: при рас\-смот\-ре\-нии
<<случайных средних арифметических>> информация о распределениях
слагаемых затухает, так что предельное распределение <<случайного
среднего арифметического>> определяется видом предельного
распределения для случайного индекса (числа слагаемых в сумме) при
надлежащей нормировке.

\smallskip

Для общности пусть $x_n=x_{0,n}$~--- (возрас\-та\-ющая)
последовательность критических порогов такая, что
\begin{equation}
p_n\equiv 1-F(x_n)\longrightarrow 0 \enskip (n\to\infty)\,.\label{e8-kor}
\end{equation}
Тогда в данном случае случайная величина $N\hm=N_n$ имеет
геометрическое распределение с параметром $q_n\hm=1\hm-p_n$. При этом
условие (8) гарантирует выполнение условия~(\ref{e4-kor}). Более того, 
${\sf E}N_n\hm=p_n^{-1}$ и, как хорошо известно,
\begin{equation*}
\lim\limits_{n\to\infty}\sup\limits_{y\geqslant0}\left\vert{\sf P}\left(p_nN_n\geqslant
y\right)-e^{-y}\right\vert =0\,. %\label{e9-kor}
\end{equation*}

Предположим, что постоянные $b_n$, обеспечивающие выполнение условия~(\ref{e5-kor}), 
имеют вид $b_n\hm=bn^{\gamma}$ при некоторых $b\hm>0$ и $\gamma\hm>0$.
При этом значения $\gamma\hm>1$ соответствуют той ситуации, когда
случайные величины~$\zeta_i$ <<в среднем>> возрастают, т.\,е.\
экстремальные события происходят все реже и реже, значения
$\gamma\hm<1$ соответствуют той ситуации, когда случайные величины~$\zeta_i$ 
<<в среднем>> убывают, т.\,е.\ экстремальные события
происходят все чаще и чаще, а значение $\gamma\hm=1$ соответствует той
ситуации, когда интенсивность потока экстремальных событий <<в
среднем>> постоянна, например в поведении интенсивности наблюдаются
проявления цикличности, причем периоды изменения интенсивности
заметно меньше периода фиксации наблюдений.

Теперь выберем нормирующие постоянные~$d_n$ так, чтобы
геометрическая случайная сумма $S_{N_n}$ имела нетривиальное
предельное распределение. Из леммы~1 вытекает, что если с учетом
выбранной формы постоянных~$b_n$ и соотношения~(\ref{e7-kor}) постоянные~$d_n$
выбрать в виде $d_n\hm=bp_n^{-\gamma}$, то для любого $y\hm>0$
\begin{multline*}
\lim\limits_{n\to\infty}{\sf P}\left(\fr{b_{N_n}}{d_n}<y\right)=
\lim\limits_{n\to\infty}{\sf P}\left((p_nN_n)^{\gamma}<y\right)={}\\
{}=
\lim\limits_{n\to\infty}{\sf P}\left(p_nN_n<y^{1/\gamma}\right)=
1-\exp\left\{-y^{1/\gamma}\right\}\,.
\end{multline*}
При этом согласно лемме~1 такое же распределение Вей\-бул\-ла--Гне\-ден\-ко
с показателем $1/\gamma$ является предельным и для геометрической
случайной суммы независимых неодинаково распределенных случайных
величин $S_{N_n}$, причем в силу непрерывности предельного
распределения Вей\-бул\-ла--Гне\-ден\-ко сходимость~(\ref{e6-kor}) равномерна по
$x\hm\in\r$. Оформим сказанное в виде следующего утверждения.

\smallskip

\noindent
\textbf{Теорема~1.} \textit{Предположим, что случайная величина $N_n$
имеет геометрическое распределение с па\-ра\-мет\-ром~$p_n$, причем
$p_n\hm\to0$ при $n\hm\to\infty$. Предположим, что существуют конечные
$\gamma\hm>0$ и $b\hm>0$ такие, что
$$
\fr{S_n}{bn^{\gamma}}\Longrightarrow 1\enskip (n\to\infty)\,.
$$
Тогда}
$$
\lim\limits_{n\to\infty}\sup\limits_{x\geqslant0}\left\vert{\sf  P}
\left(p_n^{\gamma}S_{N_n}\geqslant 
bx\right)-\exp\left\{-x^{1/\gamma}\right\}\right\vert=0\,.
$$


\subsection{Описание метода}

Итак, учитывая сделанные предположения о нормирующих постоянных,
можно заключить, что при достаточно больших значениях~$x_0$
\begin{multline} 
{\sf P}(T<t)\approx{}\\
{}\approx
1-\exp\left\{-[1-F(x_0)]\left(\fr{t}{b}\right)^{1/\gamma}\right\}\,,\enskip
t>0\,.\label{e10-kor}
\end{multline}

Применение описываемого метода вычисления временн$\acute{\mbox{ы}}$х
характеристик катастроф в неоднородных потоках экстремальных событий
заключается в следующем. Пусть $\epsilon\in(0,1)$~--- произвольное
число. Решение уравнения 
$$
{\sf P}(T<t)=\epsilon
$$ относительно~$t$ обозначим $t(\epsilon)$. Если распределение случайной величины~$T$
имеет вид~(\ref{e10-kor}), то, очевидно,
$$
t(\epsilon)=b\left[\fr{\ln(1-\epsilon)}{F(x_0)-1}\right]^{\gamma}.
$$

Смысл значения $t(\epsilon)$~--- это то время, вероятность
наступления катастрофы до которого равна~$\epsilon$. Из соображений
здравого смысла особый интерес представляют значения~$\epsilon$,
близкие к нулю (соответствующее значение $t(\epsilon)$~--- это то
время, до которого катастрофа, скорее всего, не наступит), близкие к
единице (соответствующее значение $t(\epsilon)$~--- это то время, до
которого катастрофа, скорее всего, наступит), а также
$\epsilon\hm=1/2$ (соответствующее значение $t(1/2)$~--- это
<<среднее>> время до наступления катастрофы).

Особо следует сказать, что при прогнозировании <<среднего>> или
<<ожидаемого>> времени до катастрофы можно использовать как медиану
$t(1/2)$ случайной величины~$T$, которая определяется как
решение уравнения
$$
1-\exp\left\{-[1-F(x_0)]\left(\fr{t}{b}\right)^{1/\gamma}\right\}=\fr{1}{2}
$$
относительно~$t$ и, очевидно, равна
$$
t\left(\fr{1}{2}\right)=b\left[\fr{\ln 2}{1-F(x_0)}\right]^{\gamma}\,,
$$
так и математическое ожидание
$$
{\sf E}T=\fr{b\Gamma(1+\gamma)}{[1-F(x_0)]^{\gamma}}.
$$
При этом необходимо отметить, что, например, в случае $\gamma\hm=1$
медиана $t(1/2)$ случайной величины~$T$ почти в полтора раза
(точнее, в $(\ln 2)^{-1}$ раз) меньше математического ожидания ${\sf E}T$.

При этом параметры $b$ и~$\gamma$ легко оценить методом наименьших
квадратов. Предположим, что в нашем распоряжении имеется выборка
$Z_1,Z_2,\ldots,Z_n$ предыду\-щих значений случайных величин~$\zeta_j$. 
Нормирующая функция $b_k\hm=bk^{\gamma}$ параметра~$k$ имеет
смысл тренда, или основной тенденции поведения реализации
$R_k\hm=Z_1+\cdots+Z_k$ случайной функции~$S_k$. С~целью линеаризации
регрессионной задачи прологарифмируем~$b_k$ и~$R_k$, обозначим
$\beta\hm=\log b$ и получим приближенные равенства
\begin{equation}
\log R_k\approx\beta+\gamma\log k\,,\enskip k=1,\ldots,n\,,\label{e11-kor}
\end{equation}
в правой части которых стоят линейные функции параметров~$\beta$ и~$\gamma$. 
Используя стандартный метод наименьших квадратов
оценивания параметров линейной регрессии~(\ref{e11-kor}), получим оценки
\begin{align*}
\gamma&\approx\widehat\gamma=\fr{n\sum\limits_{k=1}^n(\log k\cdot\log
R_k)-\log n!\sum\limits_{k=1}^n\log R_k}{n\sum\limits_{k=1}^n(\log k)^2-(\log n!)^2}\,;
\\
b&=\exp\{\beta\}\approx\exp\left\{\fr{1}{n}\left(\sum\limits_{k=1}^n\log
R_k-\widehat\gamma\log n!\right)\right\}\,.
\end{align*}

Чтобы получить оценку величины $1\hm-F(x_0)$, необходимо построить
разумную и адекватную парамет\-рическую математическую модель
(приближение) для функции $F(x)$. С~этой целью используем метод
построения асимптотических аппрокси\-маций для $F(x)$ при больших~$x_0$, 
основанный на теореме Бал\-ке\-ма\,--\,Пи\-канд\-са\,--\,Де Ха\-ана и
называемый методом превышений порога (POT-ме\-тод, POT\;=\;Peaks Over
Threshold).

Пусть случайная величина~$\zeta$ имеет функцию распределения $F(x)$.
В~рамках рассматриваемого метода прогнозирования катастроф как
превышений экстремальным процессом критических \mbox{уровней} большой
интерес представляет описание условного распределения превышения
случайной величиной~$\zeta$ некоторого (большого) порога~$u$:
$$
F_u(y)={\sf P}(\zeta-u < y|\zeta>u), 0\leqslant y \leqslant x_F-u\,,
$$
где $y=(x-u)$~--- превышение порога и $x_F\hm =\sup\{x \hm\in \mathbb{R} | F(x)\hm<1\} 
\hm\leqslant \infty$. Функция этого условного распределения $F_u$ может быть 
выражена через~$F$:

\noindent
$$
F_u(y)=\fr{F(u+y)-F(u)}{1-F(u)}=\fr{F(x)-F(u)}{1-F(u)}\,.
$$
Если порог $u$ достаточно велик, то большинство реализаций случайной
величины~$\zeta$ лежит между~0 и~$u$, так что оценить~$F$ в этом
промежутке несложно. Но оценить $F_u$ проблематично, так как
соответствующих наблюдений мало. На помощь приходит следующая
теорема.

\smallskip

\noindent
\textbf{Теорема~2}~\cite{BalkemaDeHaan1974, Pickands1975}. \textit{Функция
распределения~$F$ принадлежит области $\max$-при\-тя\-же\-ния
распределения, предельного для экстремальных значений тогда и только
тогда, когда существует измеримая функция $\sigma(u)\hm>0$ такая, что}
$$
\lim\limits_{u \to x_F}\sup\limits_{0\leqslant y < x_F-u}|F_u(y)-G_{\delta, \sigma(u)}(y)|=0\,,
$$
{\it где $G_{\delta, \sigma}(y)$~--- функция обобщенного распределения Парето:}
$$
G_{\delta, \sigma}(y)=\begin{cases}
1-\leqslant \left(1+\fr{\delta}{\sigma}\,y\right)^{-1/\delta}\,, &\delta \neq 0\,;\\
1-e^{-y/\sigma}\,, & \delta=0\,.
\end{cases}
$$


\smallskip

Условиям теоремы удовлетворяет большинство используемых на практике
распределений. Параметр~$\delta$ показывает, насколько тяжел хвост: чем больше~$\delta$, 
тем тяжелее хвост. Например, при моделировании финансовых данных 
обычно используется $\delta \hm\geqslant 0$.

\section{Пример применения метода прогнозирования вероятностных характеристик 
глобальных катастроф}

Для иллюстрации рассмотрим интересную задачу, связанную с
определением риска глобальных катаклизмов, вызванных столкновением
Земли с довольно большими небесными телами (астероидами, кометами).

Известно, что такие объекты приближаются к Земле относительно часто.
Исследования проводились на основании данных Центра по малым
планетам Гарвардского университета, представленных в книге Остина
Аткинсона~\cite{Atkinson2001}. Это таблица, в которой содержатся
предсказания дат приближения к Земле на расстояние не более 0,2
астрономической единицы (а.е.)\ комет и малых планет на
ближайшие 33~года начиная с июня 1999~г. Всего таких предсказаний~191, 
для каждого из них известно минимальное расстояние, на которое
малая планета подойдет к Земле, год и месяц предполагаемого
сближения. 
Приводимые ниже вычисления проведены на основе метода,
предложенного в данной статье, и уточняют результаты, приведенные в~[2--5]. 
Уточнение достигнуто за счет того, что здесь используется не
модель экспоненциального распределения, а более гибкая модель
распределения Вей\-бул\-ла--Гне\-денко.

Расстояние, на которое очередной ($i$-й) космический объект
приблизится к Земле, будем считать реализацией случайной величины~$Q_i$, 
распределение которой, вообще говоря, неизвестно и подлежит
определению (оцениванию). При этом, в отличие от рассматривавшейся
ранее формальной модели экстремального процесса, интерес
представляет не {\it максимальное}, а \textit{минимальное} значение
величин~$Q_i$. Снимем формальные противоречия, полагая
$X_i\hm=Q_i^{-1}$.

Формализуем сказанное. Имеем выборку
$\mathbf{X}\hm=\{X_1,X_2,\ldots ,X_n\}\,,\enskip n=191,$
 независимых одинаково
распределенных случайных величин. Эти величины обратны расстояниям
между Землей и потенциально опасными астероидами. Предполагается,
что известны расстояния от центра Земли до центров астероидов. Все
подсчеты ведутся в а.е., 1~а.е.\;=\;149,6~млн~км.
Радиус~$R$ Земли равен $R\hm=6400$~км\;$\approx 4{,}278075\cdot
10^{-5}$~а.е.\;=\;0,00004278075~а.е. 

Как уже говорилось,
в указанной книге~\cite{Atkinson2001} приведены данные лишь о тех
потенциально опасных астероидах, которые приближаются к Земле менее
чем на 0,2~а.е. 
Таким образом, наблюдается не <<полный>>
набор величин $X_1,X_2,\ldots$, а лишь те из них, которые
превосходят $(0{,}2)^{-1}=5$~(a.e.)$^{-1}$. Будем считать, что
порог $u\hm=5$~a.e.$^{-1}$ достаточно велик, чтобы аппроксимация,
устанавливаемая теоремой Бал\-ке\-мы\,--\,Пи\-канд\-са\,--\,Де Ха\-ана для
распределения превышений такого порога, была достаточно адекватна,
так что в качестве модели распределения случайных величин~$X_i$
можно взять обобщенное распределение Парето
$$
F(x;\alpha,\sigma)=\begin{cases} 
0\,, & \mbox{если}\ x<5\,;\\ 
1-\fr{C}{(x-\alpha)^\sigma}\,, & \mbox{если}\ x\geqslant 
5\,.
\end{cases}
$$
Как показали вычисления, проведенные в работах~\cite{Korolevetal2006, Korolevetal2007}, 
эта модель действительно
демонстрирует высочайшее согласие с указанными данными. При этом
критический порог~$x_0$, превышение которого случайной величиной~$X_i$ 
означает катастрофу (столкновение астероида с Землей), равен
$$
x_0=\fr{1}{R}=\fr{1}{6400}\ \mbox{км}^{-1}=23374{,}9993~\mbox{а.е.}^{-1}\,.
$$

Для статистического оценивания параметров $C$, $\alpha$ и~$\sigma$ в
работах~\cite{Korolevetal2006, Korolevetal2007} использовалось
несколько методов, но наилучшее согласие данных наблюдалось с
моделью, построенной на основе оценок максимального правдоподобия:
 $$ \widehat\alpha=-3{,}165\,;\ \
\widehat\sigma=2{,}37\,;\ \ C=96{,}757\,. 
$$



Оценки наименьших квадратов для пара\-мет\-ров $b$ и~$\gamma$,
построенные по выборке $Z_1,\ldots,Z_{191}$ временн$\acute{\mbox{ы}}$х
промежутков между экстремальными сближениями астероидов с Землей оказались равными
$$
b\approx 0{,}1728 \mbox{ года}\approx 2{,}0733 \mbox{ мес.};\ \ \ \
\gamma\approx 1{,}0012\,.
$$



В результате применения описанного метода к вычислению оценок
временн$\acute{\mbox{ы}}$х характеристик катастрофы, связанной со столкновением
Земли с астероидом, получены следующие значения.
\begin{itemize}
\item
время $\underline t$, до которого с вероятностью 0,9999
столкновение Земли с астероидом не про\-изойдет, примерно равно 1235~годам;
\item время $\overline t$, до которого с вероятностью 0,9999
столкновение Земли с астероидом заведомо произойдет, примерно равно
111~154~073 годам;
\item <<среднее>> время $t^*$ до столкновения Земли с астероидом
примерно равно 12\,071\,039~годам (при этом в качестве ожидаемого
времени катастрофы использовалось математическое ожидание).
\end{itemize}

\vspace*{-12pt}

{\small\frenchspacing
{%\baselineskip=10.8pt
\addcontentsline{toc}{section}{Литература}
\begin{thebibliography}{99}

\bibitem{KorolevSokolov2005} 
\Au{Королев В.\,Ю., Соколов И.\,А.} Некоторые вопросы анализа
катастрофических рисков, связанных с неоднородными потоками
экстремальных событий~// Систе\-мы и средства информатики. Спец. вып.
Математические методы и модели информатики. Стохастические
технологии и сис\-те\-мы.~--- М.: ИПИ РАН, 2005. С.~109--125.

\bibitem{Korolevetal2006} %1
\Au{Королев В.\,Ю., Соколов~И.\,А., Гордеев~А.\,С.,
Григорь\-ева~М.\,Е., Попов~С.\,В., Чебоненко~Н.\,А.} Некоторые методы
анализа временн$\acute{\mbox{ы}}$х характеристик катастроф в неоднородных
потоках экстремальных событий~// Сис\-те\-мы и средства информатики.
Спец. вып. Математические методы в информационных технологиях.~---
М.: ИПИ РАН, 2006. С.~5--23.

\bibitem{Korolevetal2007}  %2
\Au{Королев В.\,Ю., Соколов~И.\,А., Гордеев~А.\,С.,
Григорьева~М.\,Е., Попов~С.\,В., Чебоненко~Н.\,А.} Некоторые методы
прогнозирования временн$\acute{\mbox{ы}}$х характеристик рисков, связанных с
катастрофическими событиями~// Актуарий, 2007. №\,1. С.~34--40.

\bibitem{KorolevSokolov2008} %3
\Au{Королев В.\,Ю., Соколов~И.\,А.} Математические
модели неоднородных потоков экстремальных событий.~--- М.: ТОРУС
ПРЕСС, 2008. 200~с.

\bibitem{KorolevShorgin2011} %4
\Au{Королев В.\,Ю., Шоргин С.\,Я.}
Математические методы анализа стохастической структуры
информационных потоков.~--- М.: ИПИ РАН, 2011. 130~с.



\bibitem{Korolevetal2013} %5
\Au{Королев В.\,Ю., Черток А.\,В., Корчагин~А.\,Ю., Горшенин~А.\,К.}
Ве\-ро\-ят\-но\-ст\-но-ста\-ти\-сти\-че\-ское моделирование информационных потоков в
сложных финансовых сис\-те\-мах на основе высокочастотных данных~//
Информатика и её применения, 2013. Т.~7. Вып.~1. С.~12--21.

\bibitem{Kalashnikov1997} %6
\Au{Kalashnikov~V.} Geometric sums: Bounds for rare events
with applications.~--- Dordrecht--Boston--London: Kluwer Academic
Publs., 1997. 288~p.

\bibitem{KorolevBeningShorgin2011}  %7
\Au{Королев В.\,Ю., Бенинг В.\,Е., Шоргин~С.\,Я.} 
Математические основы теории риска.~--- 2-е изд., перераб. и дополн.~--- М.: Физматлит, 2011.
620~с.

\bibitem{Korolev1994}  %8
\Au{Королев В.\,Ю.} Сходимость случайных последовательностей с независимыми
случайными индексами. I~// Теоpия веpоятностей и ее пpименения,
1994. Т.~39. Вып.~2. С.~313--333.

%\bibitem{GavrilenkoKorolev2010} Гавриленко С. В., Королев В. Ю. Об оценках
%вероятности разорения страховой компании, резерв которой описывается
%классическим процессом риска // Статистические методы оценивания и
%проверки гипотез. Пермь: изд-во Пермского гос. ун-та, 2010. Вып. 22.
%С. 134--143.

%\bibitem{Gavrilenko2010} Гавриленко С. В. Оценки скорости сходимости в
%предельных теоремах со случайным индексом и некоторые их применения.
%Дис. канд. физ.-матем. наук. -- Москва: МГУ им. М. В. Ломоносова,
%2010.

\bibitem{BalkemaDeHaan1974}  %9
\Au{Balkema A., de Haan~L.} Residual life time at great age~// Ann. Probab., 1974. 
Vol.~2. P.~792--804.

\bibitem{Pickands1975} %10
\Au{Pickands J.} Statistical inference using extreme order
statistics~// Ann. Stat., 1975. Vol.~3. P.~119--131.

\bibitem{Atkinson2001} %11
\Au{Аткинсон О.} Столкновение с Землей~/
Пер с англ.~--- СПб.: Ам\-фо\-ра/Эв\-ри\-ка, 2001. 400~с.
(\Au{Atkinson~O.} Impact Earth: Asteroids, comets and meteors~--- the growing threat.~---
Virgin Publ., 1999. 256~p.)
\end{thebibliography} } }



\end{multicols}

\hfill{\small\textit{Поступила в редакцию 20.10.13}}




%\vspace*{6pt}

%\hrule

%\vspace*{2pt}

%\hrule

\newpage



\def\tit{A LIMIT THEOREM FOR GEOMETRIC SUMS OF~INDEPENDENT NONIDENTICALLY DISTRIBUTED RANDOM VARIABLES AND~ITS~APPLICATION 
TO~THE~PREDICTION OF~THE~PROBABILITIES 
OF~CATASTROPHES IN~NONHOMOGENEOUS FLOWS OF~EXTREMAL EVENTS}

\def\aut{M.\,E.~Grigor'eva$^1$, V.\,Yu.~Korolev$^{2,3}$, and~I.\,A.~Sokolov$^3$}
\def\autkol{M.\,E.~Grigor'eva, V.\,Yu.~Korolev, and~I.\,A.~Sokolov}
\def\titkol{A limit theorem for geometric sums of~independent nonidentically 
distributed random variables and~its~application} %to~the~prediction of~the~probabilities  of~catastrophes in~nonhomogeneous flows of~extremal events}


\titel{\tit}{\aut}{\autkol}{\titkol}

\vspace*{-9pt}

\noindent
$^1$Parexel International, Moscow 121609, Russian Federation\\
\noindent
$^2$Faculty of Computational Mathematics and Cybernetics, M.\,V.~Lomonosov Moscow
State University, Moscow\linebreak
$\hphantom{^1}$119991, Russian Federation\\
\noindent
$^3$Institute of Informatics 
Problems, Russian Academy of Sciences, Moscow 119333, Russian Federation

\vspace*{9pt}

\def\leftfootline{\small{\textbf{\thepage}
\hfill INFORMATIKA I EE PRIMENENIYA~--- INFORMATICS AND APPLICATIONS\ \ \ 2013\ \ \ volume~7\ \ \ issue\ 4}
}%
 \def\rightfootline{\small{INFORMATIKA I EE PRIMENENIYA~--- INFORMATICS AND APPLICATIONS\ \ \ 2013\ \ \ volume~7\ \ \ issue\ 4
\hfill \textbf{\thepage}}}

\Abste{The problem of prediction of the probabilities of catastrophes 
in nonhomogeneous flows of extremal events is considered. The paper develops and 
generalizes some methods proposed by the authors in their previous
works. The flow 
of extremal events is considered as a marked point stochastic process with not 
necessarily identically distributed intervals between points (events). The proposed 
generalizations are based on limit theorems for geometric sums of independent not 
necessarily identically distributed random variables and the Balkema\,--\,Pickands\,--\,De Haan 
theory. Within the framework of the construction under consideration, the Weibull--Gnedenko 
distribution appears as a limit law for geometric sums of independent not necessarily 
identically distributed random variables. The efficiency of the proposed methods is 
illustrated by the example of their application to the problem of prediction the time 
of the impact of the Earth with a potentially dangerous asteroid based on the data of 
the IAU (International Astronomical Union)
Minor Planet Center.}

\KWE{catastrophe; extremal event; random sum; geometric sum; law of large numbers; 
Weibull--Gnedenko distribution; Balkema\,--\,Pickands\,--\,De Haan theorem; 
generalized Pareto distribution}

\DOI{10.14357/19922264130402}

%\vspace*{3pt}

\Ack
\noindent
The research was supported by the Russian Foundation for Basic Research (Projects 
Nos.\,11-01-00515-а, 11-07-00112-а, and 12-07-00115-а).


  \begin{multicols}{2}

\renewcommand{\bibname}{\protect\rmfamily References}
%\renewcommand{\bibname}{\large\protect\rm References}
%\vspace*{12pt}

{\small\frenchspacing
{%\baselineskip=10.8pt
\addcontentsline{toc}{section}{References}
\begin{thebibliography}{99}

\bibitem{1-kgr} 
\Aue{Korolev, V.\,Yu., and I.\,A.~Sokolov}. 2005. 
Nekotorye voprosy analiza katastroficheskikh riskov, svyazannykh s neodnorodnymi 
potokami ekstremal'nykh sobytiy [Some problems of the analysis of catastrophic 
risks related to \mbox{nonhomogeneous} flows of extremal events]. 
\textit{Sistemy i sredstva informatiki. Spetsial'nyy vypusk 
``Matematicheskie metody v informatsionnykh tekhnologiyakh''}
[\textit{Systems and means of informatics. 
Special issue ``Mathematical methods and models of informatics''}]. 
Moscow: IPI RAN. 109--125.

\bibitem{2-kgr} %1
\Aue{Korolev, V.\,Yu., I.\,A.~Sokolov, A.\,S.~Gordeev, M.\,E.~Grigor'eva, 
S.\,V. ~Popov, and N.\,A.~Chebonenko}. 
2006. Nekotorye metody analiza vremennykh kharakteristik katastrof v 
neodnorodnykh potokakh ekstremal'nykh sobytiy [Some methods for the 
analysis of temporal characteristics of catastrophes in nonhomogeneous 
flows of extremal
 events]. 
\textit{Sistemy i sredstva informatiki. Spetsial'nyy vypusk}\linebreak\vspace*{-12pt} 

\columnbreak

\noindent
\textit{``Matematicheskie metody v informatsionnykh tehnologiyakh''}
[\textit{Systems and means of informatics. Spetsial issue ``Mathematical 
methods in information technologies''}]. Moscow: IPI RAN. 5--23.
\bibitem{3-kgr} %2
\Aue{Korolev, V.\,Yu., I.\,A.~Sokolov, A.\,S.~Gordeev, M.\,E.~Gri\-gor'\-eva, 
S.\,V.~Popov, and N.\,A.~Chebonenko}. 2007. 
Nekotorye metody prognozirovaniya vremennykh kha\-rak\-te\-ri\-stik riskov, svyazannykh s 
katastroficheskimi sobytiyami 
[Some methods for the prediction of the temporal characteristics of risks related 
to catastrophic events]. \textit{Aktuariy} [\textit{Actuary}] 1:34--40.
\bibitem{4-kgr}%3
\Aue{Korolev, V.\,Yu., and I.\,A.~Sokolov}. 2008. 
\textit{Matema\-ti\-che\-skie modeli neodnorodnykh potokov ekstremal'nykh sobytiy} 
[\textit{Mathematical models of nonhomogeneous flows of extremal events}]. Moscow: TORUS PRESS.
200~p.
\bibitem{5-kgr} %4
\Aue{Korolev, V.\,Yu., and S.\,Ya.~Shorgin}. 2011. 
\textit{Matematicheskie metody analiza stokhasticheskoy struktury infor\-ma\-tsi\-on\-nykh potokov} 
[\textit{Mathematical methods for the analysis of the}\linebreak\vspace*{-12pt}

\pagebreak

\noindent
\textit{stochastic structure of information 
flows}]. Moscow: IPI RAN. 130~p.


\bibitem{7-kgr} %5
\Aue{Korolev, V.\,Yu., A.\,V.~Chertok, A.\,Yu.~Korchagin, and A.\,K.~Gorshenin}. 
2013. Veroyatnostno-statisticheskoe\linebreak
modelirovanie informatsionnykh potokov v 
slozhnykh finansovykh sistemakh na osnove vysokochastotnykh\linebreak
dannykh [Probability 
and statistical modeling of information flows in complex financial systems based on 
high-frequency data]. \textit{Informatika i ee Primeneniya~--- Inform. Appl.} 7(1):12--21.
\bibitem{8-kgr} %6
\Aue{Kalashnikov, V.} 1997.
\textit{Geometric sums: Bounds for rare events with applications}. 
Dordrecht--Boston--London: Kluwer Academic Publs.  288~p.
\bibitem{9-kgr} %7
\Aue{Korolev, V.\,Yu., V.\,E.~Bening, and S.\,Ya.~Shorgin}.
2011. \textit{Matematicheskie osnovy teorii riska}  
[\textit{Mathematical foundations of risk theory}]. 2nd ed. Moscow: Fizmatlit. 620~p.
\bibitem{10-kgr} %8
\Aue{Korolev, V.\,Yu.} 1994. 
Convergence of random sequences with the independent random indices. I. 
\textit{Theory Probab.  Appl.} 39(2):282--297.
\bibitem{11-kgr} %9
\Aue{Balkema, A., and L.~de Haan}. 1974. Residual life time at great age.
\textit{Ann. Probab.} 2:792--804. 

 
\bibitem{12-kgr} %10
\Aue{Pickands, J.}  1075.
Statistical inference using extreme order statistics.
\textit{Ann. Stat.} 3:119--131.



\bibitem{6-kgr}   %11
\Aue{Atkinson,~A.} 1999. \textit{Impact Earth: Asteroids, comets and meteors~--- the growing threat}.
Virgin Publ. 256~p.

\end{thebibliography}
} }

\end{multicols}

\hfill{\small\textit{Received October 20, 2013}}

\Contr

\noindent
\textbf{Grigorieva Maria E.} (b.\ 1986)~--- biostatistician II, 
Parexel International, Moscow 121609, Russian Federation;  maria-grigoryeva@yandex.ru

\vspace*{3pt}

\noindent
\textbf{Korolev Victor Yu.} (b.\ 1954)~--- Doctor of Science in 
physics and mathematics, professor, Department of Mathematical Statistics, 
Faculty of Computational Mathematics and Cybernetics, M.\,V.~Lomonosov
 Moscow State University; Moscow 119991, Russian Federation;
 leading scientist, Institute of Informatics Problems, Russian 
Academy of Sciences, Moscow 119333, Russian Federation; victoryukorolev@yandex.ru 

\vspace*{3pt}

\noindent
\textbf{Sokolov Igor A.} (b.\ 1954)~--- Academician of the Russian Academy of Sciences, 
Doctor of Science in technology, Director, Institute of Informatics Problems, 
Russian Academy of Sciences, Moscow 119333, Russian Federation;  isokolov@ipiran.ru 

 \label{end\stat}

\renewcommand{\bibname}{\protect\rm Литература}  