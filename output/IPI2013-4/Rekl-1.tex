\def\stat{rekl-1}

   
   \begin{center}

{\prgsh\LARGE СОБЫТИЯ}

\end{center}


%\hrule

   \thispagestyle{empty}
   
   \vspace*{6mm}
   
   \section*{О МЕЖДУНАРОДНЫХ КОНФЕРЕНЦИЯХ ПО~ОСНОВАМ ИНФОРМАЦИОННОЙ НАУКИ, 
ПРОВЕДЕННЫХ В 2013~ГОДУ}

\vspace*{6pt}
   
\begin{multicols}{2}



   С 21 по 23 мая 2013~г.\ в Москве при поддержке Президиума РАН, Отделения 
математических наук и Отделения нанотехнологий и информа\-ци\-онных технологий РАН, а 
также Московского гу\-манитарного университета и Института проблем информатики РАН 
состоялась \textbf{Пятая Международная конференция по фундаментальным основам 
информационной науки}. Это~--- традиционная конференция, проводящая под эгидой 
Международного общества информационных исследований (The International Society for 
Information Studies), центральный офис которого находится в Вене (Австрия). 
Конференция по данной тематике проведена в России впервые. Предыдущие конференции 
состоялись в Мадриде (1994), Вене (1997), Париже (2005) и Пекине (2010). Они оказали 
существенное влияние на развитие фундаментальных основ информационной науки в 
развитых странах и содействовали созданию в них исследовательских институтов и 
научных центров. На конференции в Москве были заслушаны 18~пленарных докладов 
ведущих ученых из Австрии, Испании, Китая, России и Франции, в которых были 
рассмотрены: 
   \begin{itemize}
\item философские и научно-ме\-то\-до\-ло\-ги\-че\-ские проблемы развития информационной 
науки как комплексного междисциплинарного научного направления, а также ее место 
в системе науки и образования;
\item основные результаты исследований в области фундаментальных основ 
информатики и опыт их использования в науке, образовании и 
со\-ци\-аль\-но-эко\-но\-ми\-че\-ском развитии общества;
\item перспективные направления дальнейшего развития фундаментальных основ 
информационной науки, а также использования ее концепций и методов в интересах 
развития науки, образования, решения глобальных проблем развития цивилизации;
\item предложения по развитию международной кооперации ученых в области 
исследования фундаментальных основ информатики и их практическому 
использованию.
\end{itemize}

   Основные материалы конференции представлены на сайте Московского 
гуманитарного уни\-вер\-ситета, на базе которого проводилась эта конференция  (см.\ {\sf 
http://mosgu.ru/nauchnaya/conference/ 2013/FIS/}). 
   
   В период с 18 по 21 октября 2013~г.\ в древней столице Китая городе Сиань состоялась 
\textbf{Первая Международная конференция по философии информации\linebreak (ISPI-2013)}. Ее 
\mbox{инициаторами} и основными организаторами стали Международный центр философии 
информации (в составе Сианьского транспортного университета) и Международное 
общество\linebreak ин\-фор\-ма\-ци\-он\-ных исследований. В конференции участвовало более 80 
ученых из различных стран мира, включая Россию. Тезисы отобранных Программным 
комитетом докладов опубликованы в\linebreak материалах конференции, которые доступны по 
адресу {\sf http://is4is.unileon.es/index.php?option=\linebreak 
com\_content\&view=article\&id=81\%3A2013-c1-start\&\linebreak catid=83\%3Aconference\&Itemid=59\&lang=en}.
   
     В мае 2015 г.\ в Вене (Австрия) планируется провести \textbf{Шестую Международную 
конференцию по фундаментальным основам информационной науки}. Параллельно с ней 
там же будет проводиться \textbf{Вторая Международная конференция по философии 
информации (ISPI-2015)}.

\end{multicols}

\hspace*{10mm}

\hfill \textit{К.\,К. Колин}

\hfill д.т.н., проф., главный научный сотрудник 

\hfill Института проблем информатики РАН