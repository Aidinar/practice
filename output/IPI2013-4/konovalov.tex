\def\stat{konovalov}

\def\tit{ОБ ОДНОЙ ЗАДАЧЕ ОПТИМАЛЬНОГО УПРАВЛЕНИЯ НАГРУЗКОЙ 
НА~СЕРВЕР$^*$}

\def\titkol{Об одной задаче оптимального управления нагрузкой 
на~сервер}

\def\autkol{М.\,Г.~Коновалов}

\def\aut{М.\,Г.~Коновалов$^1$}

\titel{\tit}{\aut}{\autkol}{\titkol}

{\renewcommand{\thefootnote}{\fnsymbol{footnote}}
\footnotetext[1] {Работа выполнена при поддержке РФФИ 
(гранты № 11-07-00112 и № 13-07-00665).}}

\renewcommand{\thefootnote}{\arabic{footnote}}
\footnotetext[1]{Институт проблем информатики Российской академии наук, mkonovalov@ipiran.ru}


\Abst{Рассматривается относительно простая постановка задачи об управлении нагрузкой на сервер с 
фиксированным количеством мест обслуживания и потенциально бесконечной очередью. Управление 
заключается в принятии решения о приеме либо об отклонении каждого вновь поступающего задания. 
Накопление очереди сопряжено с возможной потерей качества обслуживания, поскольку срок выполнения 
заданий ограничен. В то же время отклонение заявок влечет потерю дохода. Доказано, что в случае 
экспоненциально распределенного времени выполнения заданий и для входных потоков, описываемых 
процессом восстановления с произвольным распределением времени между поступлением заданий, 
оптимальной является простая пороговая стратегия. Зависимость предельного среднего дохода от значения 
порога является унимодальной. Это обстоятельство значительно облегчает поиск оптимального 
целочисленного значения порога. Экспериментальный анализ показывает, что указанная зависимость имеет 
место также для произвольного распределения времени выполнения заданий и для входных потоков, 
описываемых как марковски модулируемые процессы (ММП) общего вида.}

\KW{ограничение нагрузки; пороговое управление; потоки заданий}

\DOI{10.14357/19922264130404}

\vskip 14pt plus 9pt minus 6pt

      \thispagestyle{headings}

      \begin{multicols}{2}

            \label{st\stat}

\section{Введение}

     Эта статья продолжает тематику повышения производительности вычислительных 
комплексов за счет применения более эффективных алгоритмов распределения потоков 
заданий~[1--4]. В~отли\-чие от упомянутых работ, в которых рассматривались 
сравнительно сложные модели систем, допуска\-ющие оптимизацию по многим 
параметрам, в данном случае на простой модели анализируется только аспект, связанный 
с ограничением входной нагруз\-ки.
     
     Суть рассматриваемой задачи в следующем. Имеется система (сервер), содержащая 
конечное число обслуживающих мест (процессоров), на которых происходит выполнение 
заданий, поступа\-ющих извне случайным образом. В~каждый момент появления 
очередной заявки должно быть принято одно из двух возможных решений: задание может 
быть либо оставлено в системе для последующей обработки, либо отвергнуто. Принятое 
на обслуживание задание сразу помещается на свободный процессор, если таковой 
имеется, либо становится в очередь, которая предполагается потенциально 
неограниченной. Продвижение в очереди осуществляется по принципу <<первый 
пришел~--- первый обслуживается>>. Время выполнения задания на процессоре является 
случайным. Существенное значение имеет общее время пребывания задания в системе, 
складывающееся из времени ожидания в очереди и времени непосредственного 
выполнения, поскольку для каждого задания существует предельный срок, до истечения 
которого оно должно быть выполнено (так называемый дедлайн).
     
     Для оценки качества работы системы вводятся числовые оценки. Так, каждое 
принятое на обслуживание задание сопровождается <<доходом>>, выраженным 
некоторым числом. Точно так же любое принятое, но не выполненное в срок задание 
влечет <<штраф>>, определенный некоторой числовой величиной. В~итоге можно 
ставить задачу о выборе такой стратегии управления доступом заданий в систему, при 
которой предельный средний доход будет максимальным. 

При выборе стратегии следует 
учитывать две разнонаправленные тенденции. С~одной стороны, желательно принимать 
как можно больше заданий, поскольку это приносит доход, но одновременно нельзя 
слишком наращивать очередь, поскольку это может привести к массовому превышению 
дедлайна и получению большого штрафа.
     
     Для максимального упрощения модели условимся еще, что в момент принятия 
решения об очередной заявке неизвестно ни время, требующееся для его выполнения на 
процессоре, ни заданный срок выполнения. Обе эти характеристики становятся 
известными только в момент оконча-\linebreak\vspace*{-12pt}
\pagebreak

\noindent
ния обслуживания, и в этот же момент начисляется 
штраф, если дедлайн превышен. Сделанное предположение говорит о том, что 
наблюдаемая часть траектории процесса, на которой может основываться стратегия 
управления нагрузкой, сводится исключительно к последовательности моментов 
поступления заявок и соответствующей последовательности принимаемых решений.
     
     Описанная ситуация представляет собой едва ли не самую простую постановку 
задачи, мыслимую в связи с проблемой регулирования доступа в систему обслуживания, 
которая имеет множество оттенков и, соответственно, много названий (управ\-ле\-ние 
перегрузкой, контроль доступа, управ\-ле\-ние потоками и~т.\,д.). Источник проблемы, а 
также основные приложения разрабатываемых алгоритмов~--- разнообразные 
информационные и телекоммуникационные системы и сети. Не пытаясь дать 
характеристику этого обширного на\-прав\-ле\-ния, укажем в качестве примера на содержащие 
обзоры работы~[5, 6]. Отметим, что многочисленные пуб\-ли\-ка\-ции по этой теме насыщены 
описанием содержательной, технической стороны. Рассматриваемые в них модели 
объектов сложны для аналитического изучения, а получаемые решения имеют в основном 
эвристический характер. Решение упрощенной задачи, возможно, позволит яснее 
представить механизмы, лежащие в основе управления перегрузками.
     
     Материал оставшейся части статьи распределен следующим образом. В~разд.~2 
дается точная постановка задачи, которая была описана наглядно во введении. Раздел~3 
содержит некоторые результаты, полученные математическим путем. В~разд.~4 
приведены результаты вычислительных экспериментов, иллюстрирующие и 
дополняющие разд.~3. В~заключении статьи подводится итоговое об\-суж\-де\-ние и 
предлагается направление дальнейших исследований.

\section{Формальная постановка задачи}

  Будем считать, что процесс поступления заявок (входной поток) принадлежит классу 
так называ\-емых ММП. Произвольный ММП 
зададим с по\-мощью набора символов $M\hm=(P,\mathcal{D},\mathcal{F},G,H)$, 
означающих следующее: $P$~--- матрица $S\times S$ переходных вероятностей марковской 
цепи, име\-ющей состояния $1,\ldots , S$; 
   $\mathcal{D}\hm= \{D_s,\ s\hm=1,\ldots , S\}$~--- набор функций распределения времен 
пребывания в различных состояниях; $\mathcal{F}\hm= \{F_s,\ s\hm=1,\ldots , S\}$~---  
 набор функций распределения времен между последовательными поступлениями заявок 
в различных состояниях; $G$~--- функция распределения времени выполнения задания; 
$H$~---  функция распределения дедлайна (функции~$G$ и~$H$ одинаковы для всех 
заявок, поступающих из данного потока).
  
  Процесс протекает в непрерывном времени, $t\hm\geq0$. Состояние марковской 
цепи~$P$ в начальный момент полагаем равным $s(0)\hm=s_0$ (для целей \mbox{статьи} выбор 
начального значения не существен). \mbox{Спустя} случайное время~$\theta$, значение которого 
определяется распределением $D_{s_0}\hm\in \mathcal{D}$, цепь с ве\-ро\-ят\-ностью 
$P_{s_0s^\prime}$ переходит в состояние $s(\theta)\hm=s^\prime$ и остается в нем 
случайное время~$\theta^\prime$, имеющее распределение $D_{s^\prime}\hm\in 
\mathcal{D}$, после чего состояние цепи с ве\-ро\-ят\-ностью $P_{s^\prime s^{\prime\prime}}$ 
становится равным $s(\theta\hm+\theta^\prime)\hm=s^{\prime\prime}$ и~т.\,д.
  
  На любом промежутке пребывания марковской цепи в любом состоянии~$s$ заявки 
поступают через случайные независимые промежутки времени, имеющие одинаковое 
распределение $F_s\hm\in \mathbf{F}$. Если поступившая заявка принята, то связанное с 
ней задание либо занимает свободное место обслуживания (при его наличии), либо 
становится в конец очереди (если свободных мест нет). Предполагается, что имеется $N$ 
мест обслуживания, которые по-дру\-го\-му будем называть процессорами. 
Непосредственное выполнение задания на процессоре занимает случайное время, 
имеющее всегда одно и то же распределение~$G$. Каждая заявка, принятая на 
обслуживание, приносит доход, который полагаем равным~1. По окончании 
обслуживания общее время, проведенное заданием в системе и складывающееся из 
времени ожидания в очереди и времени непосредственного выполнения на процессоре, 
сравнивается со значением дедлайна, которое определяется распределением~$H$. Если 
дедлайн превышен, то обслуживающая система получает штраф, равный $C\hm>1$.
  
  Обозначим через $\tau_1, \tau_2,\ldots$ последовательные моменты поступления заявок. 
Каждому индексу~$n$ в последовательности~$\tau_n$ сопоставим переменную~$y_n$ 
таким образом, что
  $$
  y_n=\begin{cases}
  1,\,& \ \mbox{если заявка, поступившая}\\
  &\hspace*{23mm}\mbox{в момент $\tau_n$, принята;}\\
  0\,, &\ \mbox{если эта заявка отклонена.}
  \end{cases}
  $$
  
  Последовательность $y\hm=\{y_n\}$ называется стратегий (управления нагрузкой на 
сервер). Она является, вообще говоря, случайной, и по смыслу задачи каждое $y_n$ может 
зависеть произвольным образом от наблюдаемой части траектории процесса до момента 
$\tau_n$ включительно.
  
  Пусть имеется некоторый ММП~$M$ и задана некоторая стратегия~$y$. Пара $(M,y)$ 
порождает вероятность $\mathbf{P}\hm=\mathbf{P}_{M,y}$, определенную на 
подходящем измеримом пространстве. Интеграл по мере~$\mathbf{P}$ будем обозначать 
через~$\mathbf{E}$.
  
  Пусть $\theta_n$ означает время, проведенное в системе заданием, принятым на 
обслуживание в момент~$\tau_n$, и пусть $\delta_n \Subset H$~--- дедлайн этого задания. 
Обозначим через
  \begin{equation}
  g_n=\left( 1-CI_{\{\theta_n>\delta_n\}}\right) I_{\{y_n=1\}}
\label{e1-kon}
\end{equation}
доход, связанный с обслуживанием заявки~$n$.
  
  Основная величина, связанная с рас\-смат\-ри\-ва\-емой системой, которая представляет 
интерес,~--- это предельный (среднеарифметический) доход
  $$
  \varphi = \mathop{\underline{\lim}}\limits_{n\to\infty} n^{-1}\sum\limits_{k=1}^n g_k\,.
  $$
  Все числовые характеристики случайной величины~$\varphi$ определяются 
параметрами~$M$ и~$\sigma$ (а также изложенными выше неформальными правилами, 
касающимися постановки в очередь и~пр.). Общая задача заключается в 
максимизации~$\varphi$ в том или ином вероятностном смысле. Более конкретно, 
положим
  \begin{equation}
  w(M,y) =\mathop{\underline{\lim}}\limits_{n\to\infty} n^{-1} \sum\limits_{k=1}^n 
\mathbf{E} g_k\,.
  \label{e2-kon}
  \end{equation}
    
  Пусть заданы: множество ММП $\mathcal{M}\hm=\{M\}$; множество (допустимых) 
стратегий $\mathcal{Y} \hm= \{y\}$; число $\varepsilon\hm\geq 0$. Требуется указать 
стратегию $y(\mathcal{M},\varepsilon)\hm\in \Sigma$ такую, что для всех 
$M\hm\in\mathcal{M}$ выполняется неравенство
  $$
  w(M,y(\mathcal{M},\varepsilon))\geq \sup\limits_{y\in \mathcal{Y}} w(M,y)-\varepsilon\,.
  $$
  
  В случае, когда $\mathcal{M}$ содержит единственный процесс~$M$, имеем 
классическую задачу отыскания \mbox{$\varepsilon$-оп}\-ти\-маль\-но\-го алгоритма (в данном 
случае оптимизирующего нагрузку на сервер при конкретном входном потоке~$M$). Если 
же $\mathcal{M}$ содержит более одного элемента, то говорят о поиске стратегии, 
адап\-тив\-ной по отношению к множеству~$\mathcal{M}$. В~этом случае требование 
<<равномерной по~$M$>> оптимальности стратегии $y(\mathcal{M},\varepsilon)$ 
вызвано, как правило, тем, что априорная информация об объекте управления 
неопределенна и ограничивается описанием всего класса~$\mathcal{M}$ в целом.

\section{Теоретический анализ}

   В дальнейшем будем пользоваться следующими обозначениями:
\begin{description}
\item[\,]  $\Gamma_{a,b}$~---  гамма-распределение с параметрами $a,b$ и плот\-ностью 
$\gamma_{a,b}(t)\hm= (a^b/\Gamma(b)) t^{b-1} e^{-at}$, $t\hm\geq 0$, где $\Gamma(b)\hm= 
\int\limits_0^\infty x^{b-1} e^{-x}\,dx$, $b>0$;
   \item[\,]
   $\Gamma_a=\Gamma_{a,1}$~---   экспоненциальное распределение с параметром~$a$ и 
   плот\-ностью $\gamma_a(t)\hm= ae^{-at}$, $t\hm\geq 0$;
   \item[\,]   
   $N_{a,b}$~---  нормальное распределение со средним~$a$ и дисперсией~$b$;
      \item[\,]
   $\Pi_{a,b}$~--- распределение Парето с параметрами $a,b$ (его плотность имеет вид 
$ab^a/t^{a+1}$, $t\hm\geq b$);
      \item[\,]
   $\xi\underset{\mathrm{d}}{=}\eta$~---  случайные величины~$\xi$ и~$\eta$ имеют одинаковые 
распределения;
      \item[\,]
   $\xi\Subset F$ (или $\xi\hm\Subset f$)~---  случайная величина~$\xi$ имеет функцию 
распределения~$F$ (соответственно плотность распределения~$f$);
      \item[\,]
   $I_A$~--- индикатор события~$A$.
  \end{description}
  
  Для аналитического рассмотрения сделаем два предположения.
  \smallskip
  
  \noindent
  \textbf{П1.} Времена выполнения заданий имеют одинаковое экспоненциальное 
распределение, параметр которого, не ограничивая общности, можно положить равным~1. 
Таким образом, далее в этом разделе полагаем $G\hm=\Gamma_1$.
  
  \smallskip
  
  \noindent
  \textbf{П2.} Марковская цепь $P$ имеет всего одно состояние, так что входной поток 
представляет собой процесс восстановления $R\hm=(F,G,H)$ с независимыми 
промежутками между поступлениями заявок, имеющими одинаковое распределение~$F$, 
и одинаковыми для всех заданий распределениями времени выполнения~$G$ и 
дедлайна~$H$.
  
  (Предположение П2, в отличие от П1, не является принципиально важным для 
получения основных выводов этого раздела и сделано для сокращения выкладок.)
  
  \smallskip
  
  Пусть $v(t)$ означает общее количество заданий в системе в момент~$t$, включая 
задания, выполняющиеся на процессорах, и задания, ожидающие в очереди, и пусть 
$v_n\hm=v(\tau_n)$, $\tilde{v}_n\hm=v(\tau_n\hm+0)$. Тогда имеем соотношение:
  $$
  \tilde{v}_n=v_n+y_n\,,
  $$
где $y_n$ определено формулой~(\ref{e1-kon}).

  Введем обозначение $\kappa_n$ для количества заданий, выполненных на промежутке 
времени  $(\tau_n,\tau_{n+1}]$, $n\hm=1,2,\ldots$\ \ Ясно, что $\kappa_n\hm=\tilde{v}_n\hm-
v_{n+1}$ и что $0\hm\leq \kappa_n\hm\leq \tilde{v}_n$.
  
  Обратимся к описанию (дискретного) распределения величины~$\kappa_n$, обозначая 
через $p_m$ его компонен\-ты, которые полностью определяются значением 
величины~$\tilde{v}_n$. Рассмотрим процесс выполнения заданий на промежутке 
времени $(\tau_n,\tau_{n+1}]$ при условии $\tilde{v}\hm=i$, при котором индекс~$m$ 
пробегает значения $0,1,\ldots , i$.
  
  Пусть вначале $0\hm\leq i\hm\leq N$. В~этом случае до наступления 
момента~$\tau_{n+1}$ процессоры, на которых завершилось выполнение задания, вновь 
не загружаются. Вероятность выполнения задания на одном занятом процессоре в течение 
указанного промежутка времени составляет, в соответствии с предположением~П1, 
величину
  $$
  p=\int\limits_0^\infty (1-e^{-t})\,dF(t)\,.
  $$
  
  Следовательно, в рассматриваемом случае $\kappa_n$ имеет биномиальное 
распределение и
  \begin{multline}
  p_m =q_1(i,m) =\begin{pmatrix}
  i\\ m\end{pmatrix} p^m (1-p)^{i-m},\\
   m=0,1,\ldots, \ i\leq N\,.
  \label{e3-kon}
  \end{multline}
    
  Перейдем к случаю $i>N$. В~этой ситуации все освобождающиеся процессоры 
мгновенно загружаются вновь~--- так происходит до тех пор, пока число выполненных 
заданий не достигнет значения $r\hm=i\hm-N\hm>0$. При этом если загружены все 
$N$~процессоров, то появление свободных мест происходит согласно пуассоновскому 
закону с параметром~$N$. После того как выполнено $r$~заданий, картина аналогична 
предыдущему случаю, причем в варианте $i\hm=N$.
  
  Итак, для $0\leq m\leq r$ имеем:
  \begin{equation}
  p_m=q_2(m) =\int\limits_0^\infty \fr{(Nt)^{m}e^{-Nt}}{m!}\,dF(t)\,.
  \label{e4-kon}
\end{equation}

  Пусть $m>r$. Чтобы выполнилось более чем $r$ заданий, необходимо, чтобы момент 
$i$-го окончания обслуживания $\zeta_r$ не превосходил момента окончания 
рассматриваемого промежутка между двумя поступлениями заданий. (Согласно П2, длина 
промежутка между поступлением заданий $\tau^{(n)}\hm= \tau_{n+1}\hm-\tau_n$ имеет 
для всех $n$ одинаковое распределение, $\tau^{(n)} \underset{\mathrm{d}}{=}\tau\Subset F$.) В~свою 
очередь, время $\zeta_r$ складывается из времен между последовательными 
осво\-бож\-де\-ни\-ями $N$ одновременно занятых процессоров, 
$\zeta_r\hm=\xi_1^{(N)}+\cdots+\xi_r^{(N)}$, где $\xi_j^{(N)}\Subset \Gamma_N$, 
$j\hm=1,\ldots , r$,~--- независимые случайные величины. Таким образом, $\zeta_r\Subset 
\Gamma_{N,r}$~[7, гл.~7, \S\,6.2]. Приходим к выражению:
  \begin{multline}
  p_m=q_3(i,m) =\int\limits_0^\infty dF(t) \int\limits_0^t \gamma_{N,r}(s) \begin{pmatrix}
  N \\ m-r\end{pmatrix}\times{}\\
  {}\times
  \left[ p(t-s)\right]^{m-r}\left[ 1-p(t-s)\right]^{N-m+r}ds\,,
  \label{e5-kon}
  \end{multline}
где $\gamma_{N,r}(s)$~--- плотность гам\-ма-рас\-пре\-де\-ле\-ния $\Gamma_{N,r}$; 
$r\hm= i\hm-N$, $m\hm= r\hm+1, \ldots , r+N$.
  
  Рассмотрим далее время пребывания в системе заявки, принятой на обслуживание. Его 
можно представить как
  $$
  \theta_n=T_n+\xi_n\,,
  $$
где $T_n$~--- время ожидания до появления свободного процессора, а $\xi_n$~--- время 
непосредственного выполнения задания. Вторая из этих величин, как уже говорилось, для 
каждого $n$ определяется независимо: $\xi_n\underset{\mathrm{d}}{=} \xi\Subset \Gamma_1$. Что касается 
первого слагаемого, то его распределение полностью определяется значением~$v_n$: при 
фиксированном значении $v_n\hm=i$
$$
T_n\underset{\mathrm{d}}{=} \begin{cases}
0\,, & \ \mbox{если}\ i<N\,;\\
\xi_1^{(N)}+\cdots+ \xi_{i-N+1}^{(N)}\,, &\ \mbox{если}\ i\geq N\,,
\end{cases}
$$
где $\xi_j^{(N)}$~--- независимые случайные величины, определенные при выводе 
формулы~(\ref{e5-kon}). Они имеют экспоненциальное распределение с параметром~$N$ 
и характеризуют времена последовательного выполнения заданий, поступивших на 
обслуживание раньше, чем рассматриваемое задание. Отсюда следует, что условное 
распределение величины~$T_n$ при условии $v_n\hm=i\hm\geq N$ является 
гам\-ма-рас\-пре\-де\-ле\-ни\-ем $\Gamma_{N,i-N+1}$. Следовательно, плотность 
распределения времени, проведенного в системе заданием, принятым на обслуживание в 
тот момент, когда в системе уже находилось $i\hm\geq N$ заданий, пред\-став\-ля\-ет собой 
свертку
$$
\gamma_i(t) =\gamma_{1,1}(t) * \gamma_{N,i-N+1}(t)=e^{-t}* \gamma_{N,i-N+1} 
(t)\,.
$$
  
  Для $0\leq i\leq N$ время, проведенное в системе, совпадает со временем выполнения 
задания, поэтому $\gamma_i(t)\hm= e^{-t}$.
  
  Ключевое для этого раздела замечание заключается в том, что определенная в начале 
раздела последовательность~$v_n$ образует управляемую марковскую цепь с множеством 
состояний $\{0,1,2,\ldots\}$, множеством управлений $\{0;1\}$ и (управляемыми) 
вероятностями перехода, не зависящими от~$n$:
 \begin{multline*}
  Q_{ij}^{(k)} ={}\\
  {}=\mathbf{P}\left(v_{n+1}=v_n+\sigma_n -\kappa_n=j\vert v_n =i,\, 
\sigma_n=\kappa \right) ={}\\
{}= \mathbf{P} \left( \kappa_n=i-j+k\right)\,,\ k=0\vee 1\,,
  \end{multline*}
  
  Используя функции $q_1(i,m), q_2(m), q_3(i,m)$, полученные в 
  формулах~(\ref{e3-kon})--(\ref{e5-kon}) для условных распределений~$\kappa_n$, 
получим следующие выражения для элементов матриц~$Q^{(k)}$:

\noindent
  $$
  Q_{ij}^{(k)} = \begin{cases}
  q_1(i,i-j+k) &\hspace*{-4.5pt}\  \mbox{для} \ 0\leq i-j+k\leq i\leq N\,;\\
  q_2(i-j+k) &\hspace*{-4.5pt}\  \mbox{для}\ i>N, \\
  & \hspace*{5mm}N+k\leq j\leq i+k\,;\\
  q_3(i,i-j+k) &\hspace*{-4.5pt}\  \mbox{для}\ i>N\,,\\
  &\hspace*{11mm} 0\leq j\leq N+k\,;\\
  0 &\hspace*{-4.5pt}\  \mbox{для остальных случаев}.
  \end{cases}
  $$
  
  Сопоставим каждой паре <<со\-сто\-яние~$i$\,--\,управ\-ле\-ние~$k$>> одношаговый 
доход:
  $$
  g_{ik} =\begin{cases}
  1-C I_{\{\vartheta_i>\delta\}}\,, &\ \mbox{если}\ k=1\,;\\
  0\,,  &\ \mbox{если}\ k=0\,,
  \end{cases}
  $$
где $\vartheta_i\Subset \gamma_i(x)$, $\delta \Subset H$. (Словами: доход в состоянии~$i$ 
равен~0, если заявка не принята; равен~1, если заявка принята и обслужена до 
наступления дедлайна; равен $1\hm-C\hm<0$, если заявка принята, но обслужена позже 
требуемого срока.)

  Совокупности управляемых матриц перехода и одношаговых доходов задают 
однородную управляемую марковскую цепь с доходами и потенциально счетным 
множеством состояний. Поэтому оптимальная в смысле максимизации предельного 
среднего дохода стратегия найдется среди однородных марковских стратегий, которые в 
данном случае имеют вид:
  $$
  \sigma = \left( \sigma_0, \sigma_1, \sigma_2,\ldots \right)\,,
  $$
где $\sigma_i$~---  вероятность принять задание на обслуживание при условии, что в 
момент поступления заявки в системе уже находится $i$ заданий.
  
  Для того чтобы исключить из рассмотрения тривиальный случай, когда системе вообще 
не выгодно ничего делать, сделаем еще одно предположение.
  
  \smallskip
  
  \noindent
  \textbf{П3.} Одношаговый доход за управление~1 в состоянии~0 положителен. Иными 
словами, вероятность~$q$ того, что время выполнения одного задания на процессоре не 
превысит дедлайн, удовлетворяет неравенству
  $$
  q\leq \fr{1}{C}\,.
  $$
  
  В терминах функций распределения предположение П3 имеет вид:
  $$
  \int\limits_0^\infty \left(1-e^{-t}\right)\,dH(t) \leq \fr{1}{C}\,.
  $$
  
  \smallskip
  
  Исходя из предположения П3, заключаем, что всякая заявка, поступившая в момент, 
когда есть свободные процессоры, должна быть принята, т.\,е.\ $\sigma_0\hm=\sigma_1 
=\cdots = \sigma_{N-1}\hm=1$, однако значения остальных членов 
последовательности~$\sigma$ менее очевидны.
  
  Обозначим через $h\hm=\mathrm{inf}\,\{i:\ \sigma_i=0\}$ номер первой нулевой 
компоненты в векторе~$\sigma$.
  
  Если $h<\infty$, то значения $\sigma_{h+1},\sigma_{h+2},\ldots$\ \ не существенны, 
поскольку в этом случае число заданий, одновременно находящихся в системе, никогда не 
превышает~$h$. В~этом случае число~$h$ будем называть порогом, а стратегию 
$\sigma\hm=\sigma(h)$~--- пороговой стратегией.
  
  Легко вычислить средний одношаговый доход в состоянии~$i$ за применение 
управления~1. Он равен $r_i\hm=1\hm-C\beta_i$, где $\beta_i\hm= \int\limits_0^\infty [1\hm-
H(t)]\gamma_i(t)\,dt$~--- вероятность превышения дедлайна для задания, поступившего в 
систему в момент, когда в ней уже находилось $i$ заданий. Отсюда получаем выражение 
для среднего одношагового дохода в состоянии~$i$:
  \begin{equation}
  R_i=\sigma_i r_i\,.
  \label{e6-kon}
  \end{equation}
  
  Поскольку ожидаемое время пребывания задания в системе монотонно не убывает с 
ростом значения~$i$, то функция~$R_i$ является монотонно невозрастающей по~$i$ (а 
при всех $i\hm\geq N$, очевидно, строго монотонно убывающей), причем, очевидно, 
$\lim\limits_{i\to\infty} R_i\hm= 1-\hm C$.
  
  Обсудим целесообразность применения управ\-ле\-ния~1 (<<заявка принята>>) в 
некотором состоянии~$i$ марковской цепи~$v_n$. Пусть в некоторый момент 
поступления заявки в системе находилось~$i$ заданий. Сравним поведение системы после 
этого момента для случаев: заявка была принята~(I) и заявка была отвергнута~(II). 
Выполнение заданий, которые уже находились в системе к рассматриваемому моменту, не 
зависит от выбранного в этот момент управления и протекает одинаково (в вероятностном 
смысле) в обоих случаях. В~то же время заявки, поступающие после этого момента, 
находятся в худшем положении в случае~I по сравнению со случаем~II, поскольку перед 
ними в очереди было поставлено на одно задание больше. Постановка задания на 
обслуживание сопровождается получением дохода~$R_i$, который может 
компенсировать отрицательный эффект от дополнительного элемента в очереди. Чем 
меньше значение~$R_i$, тем, очевидно, меньше компенсация. Если доход~$R_i$ 
отрицателен, то принятие заявки заведомо невыгодно. Учитывая соображения, 
высказанные в предыдущем абзаце относительно функции~$R_i$, приходим к 
сле\-ду\-юще\-му выводу: оптимальная стратегия найдется в классе~$\Sigma$ пороговых 
стратегий, причем порог~$h^*$ оптимальной стратегии удовлетворяет неравенству
  $$
  h^*\leq \min \left\{ i:\ R_i<0\right\}\,.
  $$
  
  Заметим, что верхняя оценка оптимального значения порога не зависит от входного 
потока.


  
  Легко понять, что любая пороговая стратегия~$\sigma$ с порогом~$h$ порождает 
однородную марковскую цепь с единственным эргодическим классом состояний 
$\{0,\ldots , h\}$ и предельным распределением на этом множестве, $\pi\hm= 
\pi(\sigma)\hm=\{\pi_i\}$. Распределение~$\pi$ определяется из условия
  \begin{equation}
  \pi=Q\pi\,,
  \label{e7-kon}
  \end{equation}
где $Q=Q(\sigma)$~--- отвечающая стратегии~$\sigma$ переходная матрица, элементы 
которой имеют вид:
$$
Q_{ij} =\sigma_i Q_{ij}^{(1)}+(1-\sigma_i) Q_{ij}^{(0)}\,.
$$
  %
  Предельный средний доход~(2) за такую стратегию принимает вид:
  
  \noindent
  \begin{equation}
  w(\sigma)=\sum\limits_{i=0}^h \pi_i R_i\,.
  \label{e8-kon}
  \end{equation}
  

 
  Задача оптимизации функции предельного среднего дохода на множестве однородных 
марковских стратегий сводится, как хорошо известно, с помощью замены переменных к 
задаче линейного программирования (в данном случае эта замена имеет вид $\rho_i\hm= 
\pi_i\sigma_i$). Отсюда приходят к выводу, что оптимальная стратегия существует среди 
вы\-рож\-ден\-ных стратегий. В~рассматриваемом случае достаточно рассматривать 
пороговые стратегии, для которых $\sigma_i\hm=1$ при $i\hm<h$, а $\sigma_n\hm=0$. 
Именно такие стратегии будем называть далее простыми пороговыми стратегиями, 
взаимно однозначно сопоставляя их с соответствующими числами~$h$.
  
  Содержание этого раздела позволяет предложить способ нахождения оптимальной 
пороговой стратегии. Он заключается в нахождении с по\-мощью перебора такого значения 
$0\hm\leq h\hm\leq h^*$, при котором максимален предельный средний доход $w(h)$ за 
простую пороговую стратегию с порогом~$h$, определяемый по формуле~(\ref{e8-kon}). 
При этом вы\-чис\-ле\-ние среднего дохода осуществляется с помощью 
  выражений~(\ref{e6-kon}) и~(\ref{e7-kon}).
  
  В этой работе реализация указанного способа оптимизации нагрузки не обсуждается. 
Заметим только, что она может оказаться трудоемкой, поскольку формулы для 
вычисления элементов мат\-ри\-цы~$Q$ достаточно сложны. В~то же время есть, 
по-ви\-ди\-мо\-му, возможность сократить полный перебор, если воспользоваться вогнутостью 
функции $w(h)$, которая была обнаружена экспериментально.

\vspace*{-6pt}

\section{Результаты вычислительных экспериментов}

\vspace*{-2pt}

  Все упоминаемые ниже численные результаты получены на персональном компьютере 
с по\-мощью имитационной модели системы, описанной в предыдущих разделах.
  
  В предыдущем разделе был определен средний одношаговый доход~$r_i$ в 
состоянии~$i$ за обслуживание принятой заявки~1. Проиллюстрируем зависимость этой 
величины (а также связанной с ней вероятности превышения дедлайна~$\beta_i$) от 
размера очереди~$i$ в момент поступления заявки.



На рис.~1 изображены зависимости~$r_i$ (сплошные линии), а также~$\beta_i$ 
(штриховые линии) от~$i$ для случаев, когда число процессоров $N\hm= 1$, 2, 5, 10. 
Значение $i\hm=-1$ соответствует ситуации, когда в момент прихода заявки в системе 
есть свободные процессоры. Распределение времени выполнения задания во всех случаях 
имеет вид $F\Subset \Gamma_1$, распределение дедлайна~--- $H\Subset \Gamma_{1/30}$, а 
постоянная $C\hm=10$. Как видно из графиков, характер зависимости одинаков для 
различного числа процессоров. Масштаб горизонтальной оси изменяется по мере 
увеличения количества процессоров, что соответствует сопутствующему увеличению 
пропускной способности системы.
  
  На рис.~2 показано сравнительное поведение функции $r_i$ для четырех систем: 
\textit{1}~--- $N\hm=2$, $G\Subset \Gamma_{0{,}5}$; 
\textit{2}~--- $N\hm=4$, $G\Subset \Gamma_{0{,}25}$; 
\textit{3}~--- $N\hm=5$, $G\Subset\Gamma_{0{,}2}$; 
\textit{4}~--- $N\hm=10$, $G\Subset \Gamma_{0{,}1}$ (во 
всех случаях $H\Subset \Gamma_{1/30}$). С~помощью этого рисунка обнаруживается 
следу\-ющее обстоятельство. Параметр~$a$ распределения~$G$ можно интерпретировать 
как производительность процессора. Для всех систем выполняется условие $Na\hm=1$. 
Тем не менее производительность систем разная, причем она уменьшается с 
уменьшением~$N$. Например, система~\textit{4}, в которой в пять раз больше 
процессоров, чем в системе~\textit{а}, но они работают в~5~раз медленнее, является 
экономически нецелесообразной, поскольку при любом размере очереди средний 
одношаговый доход отрицательный.

  \begin{figure*} %fig1
   \vspace*{1pt}
 \begin{center}
 \mbox{%
 \epsfxsize=127.324mm
 \epsfbox{kon-1.eps}
 }
 \end{center}
 \vspace*{-6pt}
\Caption{Одношаговый доход в зависимости от размера очереди при числе процессоров $N\hm=1$~(\textit{а});
2~(\textit{б}); 5~(\textit{в}); 10~(\textit{г})}
\end{figure*}



  Обратимся теперь к результатам, связанным с использованием простых пороговых 
стратегий, каж\-дая из которых однозначно характеризуется неотрица\-тель\-ным целым 
чис\-лом~$h$~--- значением порога. Как было показано в предыдущем разделе, для систем 
с экспоненциальным рас\-пре\-делением времени выполнения заданий такие страте\-гии дают 
максимальный предельный средний доход. 

Про\-ведена серия экспериментов на 
имитационной модели с целью получения за\-ви\-си\-мости $w(h)$ предельного среднего 
дохода от значения по\-ро\-га. Рассматривались системы с чис\-лом процессоров 
$N\hm=1$, 2, 5, 10. В~качестве входного потока рас\-смат\-ри\-ва\-лись описанные в предыдущем 
разделе процессы восстановления $R\hm=(F,G,H)$ (см.\ предположение~П2). Функция распределения 
интервалов между поступлениями заявок~$F$ выбиралась из двух типов распределений 
(экспоненциального и Парето), причем каждый раз так, чтобы средняя нагрузка на один 
процессор составляла одну заявку в едини-\linebreak\vspace*{-12pt}
\begin{center}  %fig2
%\vspace*{-3pt}
\mbox{%
 \epsfxsize=70.591mm
 \epsfbox{kon-2.eps}
 }
  \end{center}
  
  \vspace*{-3pt}
  
\noindent
{{\figurename~2}\ \ \small{Сравнительное поведение $r_i(i)$ для систем:
\textit{1}~--- $N\hm=2$, $G\Subset \Gamma_{0{,}5}$; 
\textit{2}~--- $N\hm=4$, $G\Subset \Gamma_{0{,}25}$; 
\textit{3}~--- $N\hm=5$, $G\Subset\Gamma_{0{,}2}$; 
\textit{4}~--- $N\hm=10$, $G\Subset \Gamma_{0{,}1}$.
Вычислительную мощность выгоднее концентрировать в меньшем числе процессоров}}

\vspace*{12pt}

%\pagebreak


\addtocounter{figure}{1}


\noindent
цу времени.  Из тех же типов распределений 
выбиралась функция распределения времени выполнения задания на процессоре~$G$, 
причем так, чтобы среднее время выполнения задания составляло единицу времени. Для 
каждого значения~$N$ рас\-смат\-ри\-ва\-лись четыре возможные комбинации типов 
распределений функций~$F$ и~$G$:
\begin{enumerate}[(1)]
\item $  N=1$:
\begin{gather*}
 \hspace*{-8mm}(F\Subset \Gamma_1, G\Subset \Gamma_1)\,;\\
 \hspace*{-8mm}\left( F\Subset \Gamma_1,G\Subset 
\Pi_{1+\sqrt{2},{\sqrt{2}}/({1+\sqrt{2}})}\right)\,;
\end{gather*}

\noindent
\begin{gather*}
 \hspace*{-8mm}  \left( F\Subset \Pi_{1+\sqrt{2},{\sqrt{2}}/({1+\sqrt{2}})},G\Subset \Gamma_1\right)\,;\\
 \hspace*{-8mm} \left( F\Subset \Pi_{1+\sqrt{2},{\sqrt{2}}/({1+\sqrt{2}})}, G\Subset 
\Pi_{1+\sqrt{2},{\sqrt{2}}/({1+\sqrt{2}})}\right)\,;
  \end{gather*}
  \item $ N=2$:
    \begin{gather*}
   \hspace*{-8mm}  (F\Subset \Gamma_2, G\Subset \Gamma_1)\,;\\
   \hspace*{-8mm}  \left( F\Subset \Gamma_2,G\Subset 
\Pi_{1+\sqrt{2},{\sqrt{2}}/({1+\sqrt{2}})}\right)\,;\\
 \hspace*{-8mm}  \left( F\Subset \Pi_{1+\sqrt{2},{1}/({2+\sqrt{2}})},G\Subset \Gamma_1\right)\,;\\
 \hspace*{-8mm}  \left( 
  F\Subset \Pi_{1+\sqrt{2},{1}/({2+\sqrt{2}})}, G\Subset 
\Pi_{1+\sqrt{2},{\sqrt{2}}/({1+\sqrt{2}})}\right)\,;
  \end{gather*}
\item $N=5$:
  \begin{gather*}
   \hspace*{-8mm} (F\Subset \Gamma_5, G\Subset \Gamma_1)\,;\\
   \hspace*{-8mm} \left( F\Subset \Gamma_5,G\Subset 
\Pi_{1+\sqrt{2},{\sqrt{2}}/({1+\sqrt{2}})}\right)\,;\\
  \hspace*{-8mm} \left( F\Subset \Pi_{1+\sqrt{2},{0{,}2\sqrt{2}}/({2+\sqrt{2}})},G\Subset 
\Gamma_1\right)\,;\\
  \hspace*{-8mm} \left( 
  F\Subset \Pi_{1+\sqrt{2},{0{,}2\sqrt{2}}/({2+\sqrt{2}})}, G\Subset 
\Pi_{1+\sqrt{2},{\sqrt{2}}({1+\sqrt{2}})}\right)\,;
  \end{gather*}
\item $  N=10$:
  \begin{gather*}
   \hspace*{-8mm} (F\Subset \Gamma_{10}, G\Subset \Gamma_1)\,;\\
    \hspace*{-8mm}\left( F\Subset 
\Gamma_{10},G\Subset \Pi_{1+\sqrt{2},{\sqrt{2}}/({1+\sqrt{2}})}\right)\,;\\
 \hspace*{-8mm}  \left( F\Subset \Pi_{1+\sqrt{2},{0{,}1\sqrt{2}}/({2+\sqrt{2}})},G\Subset 
\Gamma_1\right)\,;\\
  \hspace*{-8mm} \left( 
  F\Subset \Pi_{1+\sqrt{2},{0{,}1\sqrt{2}}/({2+\sqrt{2}})}, G\Subset 
\Pi_{1+\sqrt{2},{\sqrt{2}}/({1+\sqrt{2}})}\right)\,.
  \end{gather*}
  \end{enumerate}
  
    \begin{figure*} %fig3
     \vspace*{1pt}
 \begin{center}
 \mbox{%
 \epsfxsize=124.313mm
 \epsfbox{kon-3.eps}
 }
 \end{center}
 \vspace*{-6pt}
  \Caption{Зависимость целевой функции от значения порога при числе процессоров $N\hm=1$~(\textit{а})
  2~(\textit{б});   5~(\textit{в});   10~(\textit{г});
  \textit{1}~--- Э--Э; \textit{2}~--- Э--П; \textit{3}~--- П--Э; \textit{4}~--- П--П  }
   \end{figure*}
   
\noindent
В этих формулах параметры распределения Парето подобраны так, чтобы для каждого 
значения~$N$  во всех четырех скобках дисперсии соответствующих компонент 
совпадали.


  Для всех вариантов $H\Subset N_{15{,}5}$, $C\hm=10$.
  
  На рис.~3  показаны графики зависимости $w(h)$ для всех указанных 16~вариантов 
системы. Для обозначения принадлежности функций распределения~$F$ и~$G$ 
определенному типу (экспоненциальному~--- Э или Парето~--- П) используются 
обозначения Э--Э, Э--П, П--Э и П--П.
  

   
  Во всех случаях качественный характер зависимости оказался одинаковым~--- это 
унимодальные вогнутые функции. С~ростом числа процессоров обнаруживаются две 
тенденции: сдвиг максимума в область б$\acute{\mbox{о}}$льших значений порога и 
увеличение плато на графиках в области максимума. Отсюда следует, что среди систем с 
единичной средней нагрузкой на один сервер допустимая очередь больше для тех из них, 
у которых больше мест обслуживания. При этом для систем с большим числом 
процессоров не так сказывается погрешность в определении оптимального значения 
порога.
  
  Важным обстоятельством представляется то, что различие графиков для разных типов 
функций распределения~$F$ и~$G$ незначительно. К~тому же это различие уменьшается 
с увеличением~$N$. В~связи с этим возникает вопрос: существуют ли системы с иной 
формой зависимости предельного среднего дохода от значения порога, определяющего 
прос\-тую стратегию? Многочисленные эксперименты дали отрицательный ответ. В~этих 
экспериментах варьировалось число процессоров, тип и па\-ра\-мет\-ры функций 
распределения $F,G,H$, а также па\-ра\-мет\-ры входного ММП. При этом в качестве типов 
распределений использовались экспоненциальное, равномерное, нормальное 
распределение, а также распределение Парето и Вей\-бул\-ла. Во всех случаях характер 
упомянутой зависимости оставался неизменным и представлял собой вогнутую 
унимодальную функцию. Приведем только один заключительный пример.
  
  Система содержит два процессора ($N\hm=2$). Входной поток представляет собой 
ММП со следующими параметрами: $P=\begin{pmatrix} 0&1\\ 1&0\end{pmatrix}$ 
(переходная матрица состояний входного потока); $D_1\Subset \Gamma_{0{,}001}$, 
$D_2\Subset \Pi_{4{,}75}$ (распределения времен пребывания в состояниях); $F_1\Subset 
\Gamma_1$, $F_2\Subset \Gamma_{10}$ (распределения времен между поступлениями 
заявок); $G\hm\Subset \Pi_{2{,}005, {201}/{401}}$ (распределение времени выполнения 
задания); $H\hm\Subset N_{15,5}$ (распределение дедлайна); $C\hm=10$ (штраф за 
превышение дедлайна).



  Рассматриваемая система выполняет задания за время, имеющее распределение Парето 
со средним значением~1 и большой дисперсией, равной~100. Особенность входного 
потока заключается в том, что периоды сравнительно небольшой 
нагрузки\linebreak\vspace*{-12pt}
\begin{center}  %fig4
%\vspace*{-3pt}
\mbox{%
 \epsfxsize=72.598mm
 \epsfbox{kon-4.eps}
 }
  \end{center}
  
  \vspace*{-4pt}
  
\noindent
{{\figurename~4}\ \ \small{Зависимость целевой функции от значения порога для более сложного входного потока}}

\vspace*{12pt}

%\pagebreak


\addtocounter{figure}{1}

\noindent
 (состояние~1 
цепи~$P$) сменяются периодами, когда нагрузка в 10~раз больше (состояние~2 цепи~$P$). 
При этом время пребывания в состоянии~2 хотя и меньше в среднем в 10~раз, чем время 
пребывания в состоянии~1, но имеет распределение Парето с большой дисперсией, 
равной~1250. Таким образом, поведение системы отличается большой 
<<нерегулярностью>> по сравнению с чисто <<экспоненциальной>> моделью. Тем не 
менее приведенный на рис.~4 график функции $w(h)$, полученный с помощью 
компьютерной имитации работы системы, качественно повторяет графики на рис.~3.

\section{Заключение}

  В статье изучалась работа сервера, который на конечном числе процессоров выполняет 
задания, поступающие из случайного потока. Прием задания на обслуживание приносит 
доход, но сопряжен с возможным штрафом в случае невыполнения задания в срок. 
Сформулирована задача оптимизации доступа заданий в систему с точки зрения 
увеличения предельного среднего дохода.
  
  В предположении экспоненциального распределения времени выполнения заданий для 
широкого класса входных потоков, описываемых как процесс восстановления с 
произвольным распределением времени между поступлением заданий, доказано, что 
оптимальная стратегия находится в конечном множестве простых пороговых стратегий. 
При этом вычислительные эксперименты на имитационной модели показали, что 
зависимость предельного среднего дохода от значения порога представляет собой 
унимодальную функцию, что значительно облегчает нахождение оптимального порога.
  
  На основе экспериментальных данных можно высказать гипотезу, что характер 
упомянутой зависимости сохраняется для произвольного распределения времени 
выполнения заданий и для входных потоков, описываемых как 
ММП общего вида. Это существенно облегчает нахождение оптимального значения 
порога, позволяя избежать полного перебора.
  
  В связи с изучаемой задачей остается открытым важный вопрос: можно ли увеличить 
предельный средний доход за счет расширения множества простых пороговых стратегий, 
если распределение времени выполнения заданий отличается от экспоненциального? 
В~этом случае процесс~$v_n$, обозначающий количество заданий в системе, перестает 
быть марковским. Естественно было бы ожидать, что оптимальная стратегия тогда будет 
более сложная~--- например, рандомизированная или зависящая от более глубокой 
предыстории, чем текущее значение~$v_n$. Однако эксперименты показывают, что этого 
не происходит, по крайней мере, при полной наблюдаемости процесса~$v_n$. Эти 
результаты требуют отдельного изложения, хотя бы из необходимости описать, как 
осуществляется оптимизация на множестве стратегий, более изощренных, чем простые 
пороговые стратегии.
  
  Указанное направление для продолжения исследований представляется интересным, и 
не только с теоретической точки зрения. На этом пути можно было бы подойти к 
обоснованию тех алгоритмов, которые реально применяются на практике, например 
гистерезисных стратегий ограничения нагрузки~\cite{8-kon}.

{\small\frenchspacing
{%\baselineskip=10.8pt
\addcontentsline{toc}{section}{Литература}
\begin{thebibliography}{9}
\bibitem{1-kon}
\Au{Коновалов М.\,Г.} О~планировании потоков в системах вычислительных ресурсов~// Информатика и её 
применения, 2010. Т.~4. Вып.~2. С.~3--12.
\bibitem{2-kon}
\Au{Konovalov M.} Multiagent model for jobs flows planning and pricing in distributed computing systems~// 2010 
Congress (International) on Ultra Modern Telecommunications and Control Systems and Workshops (ICUMT): 
Proceedings of ICUMT-T, ICUMT-CS and associated workshops. IEEE, Catalog Number CFP1063G-CDR, 
2010. CD-ROM. \mbox{ISBN} 978-1-4244-7286-4. Report 1569339397.
\bibitem{3-kon}
\Au{Коновалов М.\,Г., Малашенко Ю.\,Е., Назарова~И.\,А.} Управ\-ле\-ние заданиями в гетерогенных 
вычислительных сис\-те\-мах~// Известия РАН. Теория и системы управления, 2011. 
Т.~50. №\,2. С.~43--61.
\bibitem{4-kon}
\Au{Коновалов М.\,Г.} Оптимизация работы вычислительного комплекса с помощью имитационной модели 
и адаптивных алгоритмов~// Информатика и её применения, 2012. Т.~6. Вып.~1. С.~37--48.
\bibitem{5-kon}
\Au{Welzl M.} Network congestion control.~--- N.Y.: Wiley, 2005.
\bibitem{6-kon}
\Au{Hong Y., Huang~C., Yan~J.} A~comparative study of SIP overload control algorithms~// Network and traffic 
engineering in emerging distributed computing applications~/ Eds.\ J.~Abawajy, M.~Pathan, 
M.~Rahman, A.\,K.~Pathan, and M.\,M.~Deris.~--- IGI Global, 2012. P.~1--20. {\sf 
http://arxiv.org/ftp/arxiv/papers/1210/1210.1505.pdf}.
\bibitem{7-kon}
\Au{Боровков А.\,А.} Теория вероятностей.~--- 5-е изд.~--- М.: Либроком, 2009. 656~с.
\bibitem{8-kon}
\Au{Abaev~P.\,O., Gaidamaka~Y.\,V., Pechinkin~A.\,V., Razumchik~R.\,V., Shorgin~S.\,Ya.} Simulation of 
overload control in SIP server networks~// ECMS 2012: 26th European Conference on Modelling and 
Simulation Proceedings.~--- Koblenz, Germany: Digitaldruck Pirrot GmbH, 2012. P.~533--539.

\end{thebibliography} } }



\end{multicols}

\vspace*{-6pt}

\hfill{\small\textit{Поступила в редакцию 21.10.13}}


\vspace*{9pt}

\hrule

\vspace*{3pt}

\hrule

\def\tit{ABOUT ONE TASK OF OVERLOAD CONTROL}

\def\titkol{About one task of overload control}

\def\aut{M.\,G.~Konovalov}
\def\autkol{M.\,G.~Konovalov}



\titel{\tit}{\aut}{\autkol}{\titkol}

\vspace*{-15pt}


\noindent
Institute of Informatics 
Problems, Russian Academy of Sciences, Moscow 119333, Russian Federation

\def\leftfootline{\small{\textbf{\thepage}
\hfill INFORMATIKA I EE PRIMENENIYA~---  INFORMATICS AND APPLICATIONS\ \ \ 2013\ \ \ volume~7\ \ \ issue\ 4}
}%
 \def\rightfootline{\small{INFORMATIKA I EE PRIMENENIYA~--- INFORMATICS AND APPLICATIONS\ \ \ 2013\ \ \ volume~7\ \ \ issue\ 4
\hfill \textbf{\thepage}}}

\vspace*{6pt}


\Abste{The article considers the relatively simple task of congestion control. On the server with 
a finite number of places of service and potentially infinite queue, jobs are running, coming from 
the random flow. Control means the adoption of the decision on admission or rejection of each 
newly incoming job. Accumulation of the queue may result in loss of quality of service, because 
the period of execution of jobs is limited. At the same time, the rejection of application causes 
the loss of income. It is proved that in the case of exponentially distributed service 
time and for input flows, described as the renewal process with an arbitrary interarrival time 
distribution, optimum is a simple threshold strategy. The dependence of the limiting average 
income on the threshold value is unimodal. This circumstance greatly facilitates the search for 
the optimal integer value of the threshold. Experimental analysis shows that this dependence has 
a place for arbitrary distribution of service time and for general type of Markov modulated input 
flows.}

\KWE{congestion control; overload control; threshold strategy; job flow}

\DOI{10.14357/19922264130404}


\vspace*{-12pt}

\Ack
\noindent
The research was supported by the Russian Foundation for Basic Research
(grants Nos.\,11-07-00112 and 13-07-00665).



  \begin{multicols}{2}

\renewcommand{\bibname}{\protect\rmfamily References}
%\renewcommand{\bibname}{\large\protect\rm References}

{\small\frenchspacing
{%\baselineskip=10.8pt
\addcontentsline{toc}{section}{References}
\begin{thebibliography}{9}

\bibitem{1-kon-1}
\Aue{Konovalov, M.\,G.} 2010. O~planirovanii potokov v sistemakh vychislitel'nykh resursov 
[On task flow planning in computional resource systems]. \textit{Informatika i ee Primeneniya~---
Inform. Appl.}  4(2):3--12.
\bibitem{2-kon-1}
\Aue{Konovalov, M.} 2010. Multiagent model for jobs flows planning and pricing in distributed 
computing systems. \textit{2010 Congress (International) on Ultra Modern 
Telecommunications and Control Systems and Workshops (ICUMT)}. 
\textit{\mbox{ICUMT-T}, ICUMT-CS and Associated Workshops Proceedings}.  
CD-ROM. IEEE, Catalog Number CFP1063G-CDR.  Report 
1569339397.
\bibitem{3-kon-1}
\Aue{Konovalov,~M.\,G., Yu.\,E.~Malashenko, and I.\,A.~Nazarova}. 2011.
Job control in heterogeneous computing systems. \textit{J.~Comput. Syst. Sc. Int.} 
50(2):220--237.
\bibitem{4-kon-1}
\Aue{Konovalov, M.\,G.} 2012. Optimizatsiya raboty vychislitel'nogo kompleksa s pomoshch'yu 
imitatsionnoy modeli i adaptivnykh algoritmov [Computer system optimization using 
simulation model and adaptive algorithms]. \textit{Informatika i ee Primeneniya~---
Inform. Appl.} 6(1):37--48.
\bibitem{5-kon-1}
\Aue{Welzl, M.} 2005. \textit{Network congestion control}. NY: Wiley.
\bibitem{6-kon-1}
\Aue{Hong, Y., C.~Huang, and J.~ Yan}. 2012. A~comparative study of SIP overload control 
algorithms. \textit{Network and traffic engineering in emerging distributed computing 
applications}. Eds.\ J.~Abawajy, M.~Pathan, M.~Rahman, A.\,K.~Pathan, and M.\,M.~Deris. 
IGI Global. 1--20.
Available at: {\sf http://arxiv.org/ftp/arxiv/papers/1210/1210.1505.pdf}
(accessed November~5, 2013).
\bibitem{7-kon-1}
\Aue{Borovkov,~A.\,A.} 2009. \textit{ Teoriya veroyatnostey} [\textit{Probability theory}]. 
5th ed. Moscow: Librokom. 656~p.


 
\bibitem{8-kon-1}
\Aue{Abaev,~P.\,O., Y.\,V. Gaidamaka, A.\,V.~Pechinkin, R.\,V.~Razumchik, and 
S.\,Ya.~Shorgin}. 2012. Simulation of overload control in SIP server networks. 
\textit{ECMS 2012: 26th European Conference on Modelling and Simulation 
Proceedings}. Koblenz, Germany. 533--539. 
\end{thebibliography}
} }



\end{multicols}

\vspace*{-6pt}

\hfill{\small\textit{Received October 21, 2013}}

\vspace*{-18pt}

\Contr


\noindent
\textbf{Konovalov Mikhail G.} (b.\ 1950)~--- Doctor of Science in technology, Head of 
Laboratory, Institute of Informatics Problems, Russian Academy of Sciences,
Moscow 119333, Russian Federation;
mkonovalov@ipiran.ru

 \label{end\stat}
 
\renewcommand{\bibname}{\protect\rm Литература}