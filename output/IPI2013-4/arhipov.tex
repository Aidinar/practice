\def\stat{arhipov}

\def\tit{ОПТИМИЗАЦИЯ ФУНКЦИЙ LAB-КОНТРАСТНОГО 
ГРАДАЦИОННОГО ПРЕОБРАЗОВАНИЯ}

\def\titkol{Оптимизация функций Lab-контрастного 
градационного преобразования}

\def\autkol{О.\,П.~Архипов, З.\,П.~Зыкова}

\def\aut{О.\,П.~Архипов$^1$, З.\,П.~Зыкова$^2$}

\titel{\tit}{\aut}{\autkol}{\titkol}

%{\renewcommand{\thefootnote}{\fnsymbol{footnote}}\footnotetext[1] {Статья 
%рекомендована к публикации в журнале Программным комитетом конференции 
%<<Электронные библиотеки: перспективные методы и технологии, электронные 
%коллекции>> (RCDL-2012).}}

\renewcommand{\thefootnote}{\arabic{footnote}}
\footnotetext[1]{Орловский филиал Института проблем информатики Российской академии наук, arkhipov12@yandex.ru} 
\footnotetext[2]{Орловский филиал Института проблем информатики Российской академии 
наук, zykzoya@yandex.ru} 



\Abst{Рассмотрена задача персонифицированного преобразования 
распределения контрастов на ступенчатых тоновых шкалах. Решение задачи 
необходимо для управления отображениями RGB-изоб\-ра\-же\-ний на цветных 
периферийных устройствах персональных электронных вы\-чис\-ли\-тель\-ных машин (ПЭВМ) с целью улучшения их восприятия в части 
детализации. Поскольку изменение соотношения Lab-конт\-рас\-тов цифрового 
описания отображений пикселов влечет за собой подобное, хотя и, возможно, 
менее ярко выраженное изменение соотношения реальных контрастов 
отображений, то для решения задачи достаточно подобрать подходящее 
распределение Lab-конт\-рас\-тов. Для приближенного вычисления функции 
Lab-конт\-раст\-но\-го градационного преобразования ступенчатых тоновых 
шкал по образцу рассматривались два семейства параметрических алгоритмов. 
Параметры одного из них~--- подмножества пикселов шкалы, а другого~--- 
множества градаций. Задача выбора оптимальных параметров решена путем 
сравнения погрешности вычисления функции распределения 
Lab-конт\-рас\-тов на ступенчатых тоновых шкалах на типичных примерах. 
Приведен пример, демонстрирующий эффективность применения 
соответствующей функции Lab-конт\-раст\-но\-го градационного 
преобразования ступенчатых тоновых шкал. Показано, что при выборе 
подходящего образца можно не только избежать искажения детализации, но и 
добиться ее улучшения.}

\KW{цветовоспроизведение; цветовосприятие; Lab-координаты; контраст; 
градации}

\DOI{10.14357/19922264130405}

\vskip 20pt plus 9pt minus 6pt

      \thispagestyle{headings}

      \begin{multicols}{2}

            \label{st\stat}

\section{Введение}
  
  Lab-контрастное градационное преобразование ступенчатых тоновых шкал 
является необходимым компонентом информационной технологии 
равноконт\-растной градационной скелетизации цветового пространства 
восприятия произвольным пользователем вывода цветных изображений на 
периферийные устройства ПЭВМ~[1--9]. 
  
  Информационная технология применяется на практике в обычных условиях 
функционирования офисных компьютерных систем с цветной периферией для 
персонифицированного управления детализацией RGB-изоб\-ра\-же\-ний при 
их печати и выводе на монитор. 
  
  Целью такого управления является улучшение восприятия цветных 
изображений в части детализации не только для пользователей, 
цветовосприятие которых близко к стандарту, но и для тех пользователей, 
которые имеют такие аномалии цветового зрения, как частичная цветовая 
слепота. 
  
  Lab-контрастное градационное преобразование ступенчатых тоновых шкал 
по образцу является инструментом влияния на персонифицированное 
распределение контрастов в отображениях цветных изображений 
(представлениях изображений на периферийных устройствах ПЭВМ), 
позволяющим улучшить детализацию отображений в восприятии 
произвольных пользователей~[6--9]. 
  
  В основе влияния лежит тот факт, что при корректном цифровом описании 
цветовоспроизведения и цветовосприятия изменение соотношения 
  Lab-конт\-рас\-тов цифрового описания отображений пикселов влечет за 
собой подобное, хотя и, возможно, менее ярко выраженное изменение 
соотношения реальных контрастов отображений.
  
  Пусть $x_1$, $x_2$~--- RGB-пик\-се\-лы, $\Psi_n$~--- функция 
цветовосприятия $n$-го наблюдателя, а RGB-функ\-ция~$\varphi_n$~--- 
цифровое описание~$\Psi_n$. 
  
  Обозначим через $\{\Psi_n(x_i)\}$ отображения пикселов в цветовом 
пространстве $n$-го наблюдателя, $\{\varphi_n(x_i)\}$~--- цифровое описание 
этих отображений, $C_n(\Psi_n(x_1),\Psi_n(x_2))$~--- контраст отображений 
  RGB-пик\-се\-лов~$x_1$ и~$x_2$ в цветовом пространстве $n$-го 
наблюдателя. Пусть $E(\varphi_n(x_1),\varphi_n(x_2))$~--- Lab-конт\-раст этих 
пикселов:
\begin{multline*}
  E(\varphi_n(x_1),\varphi_n(x_2)) ={}\\
  {}= \sqrt{(L_{n,1}-L_{n,2})^2+(a_{n,1} - 
a_{n,2})^2 + (b_{n,1}-b_{n,2})^2}\,,\hspace*{-8.09448pt}
  \end{multline*}
где $(L_{n,1}, a_{n,1}, b_{n,1})$~--- это Lab-ко\-ор\-ди\-на\-ты, 
соответствующие RGB-ко\-ор\-ди\-на\-там $\varphi_n(x_1)$, а 
$(L_{n,2},a_{n,2},b_{n,2})$~--- $\varphi_n(x_2)$.

  Обозначим через $x_1^\prime$, $x_2^\prime$ значения пикселов~$x_1$, $x_2$ 
после некоторого преобразования. Если Lab-конт\-раст цифрового описания 
отображений новых значений пикселов значительно превосходит 
  Lab-конт\-раст цифрового описания отображений пикселов~$x_1$ и~$x_2$: 
  $$
E(\varphi_n(x^\prime_1),\varphi_n(x^\prime_2)) \gg E(\varphi_n(x_1), 
\varphi_n(x_2))\,, 
$$
то, как правило, выполняется следующее соотношение:
$$
C_n(\Psi_n(x^\prime_1),\Psi_n(x^\prime_2)) \geq C_n (\Psi_n(x_1),\Psi_n(x_2))\,. 
$$
  
  Из соотношения
  $$
E(\varphi_n(x^\prime_1),\varphi_n(x^\prime_2)) \ll E (\varphi_n(x_1),\varphi_n(x_2)) 
$$
следует
$$
C_n (\Psi_n(x^\prime_1),\Psi_n(x^\prime_2)) \leq C_n (\Psi_n(x_1),\Psi_n(x_2))\,. 
$$

  Приведенные зависимости позволяют определить подход к получению 
нужного распределения контрастов на отображениях шкал, состоящий в 
подборе подходящего распределения Lab-конт\-рас\-тов. 
  В~[6--9] предлагается подбирать распределение 
  Lab-конт\-рас\-тов по образцу. Для задания образцов используются 
неотрицательные числовые функции от пикселов шкал. 
  
  Процедура управления контрастами отображений такова:
  \begin{enumerate}[(1)]
\item задается начальный образец распределения Lab-конт\-рас\-тов;
\item определяется функция Lab-конт\-раст\-но\-го градационного 
преобразования цифрового описания отображений ступенчатых тоновых шкал 
по образцу;
\item вычисляются координаты пикселов преобразованных ступенчатых 
тоновых шкал и цифровое описание их отображений; 
\item вычисляется распределение Lab-конт\-рас\-тов на цифровом описании 
отображений ступенчатых тоновых шкал;
\item преобразованные ступенчатые шкалы выводятся на используемое 
периферийное устройство;
\item пользователем проводится визуальный анализ детализации~--- 
распределения контрастов на отображениях шкал в сопоставлении с графиками 
образца и функции Lab-конт\-рас\-тов;
\item до установления допустимости детализации отображений шкал 
проводится необходимая модификация образца и повторяются ша-\linebreak ги~2--6;
\item после нахождения подходящего образца определяется Lab-конт\-раст\-ное 
градационное преобразование всех пикселов RGB-ку\-ба;
\item выполняется предварительное Lab-конт\-раст\-ное градационное 
преобразование произвольного изображения;
\item преобразованное изображение выводится на используемое периферийное 
устройство.
\end{enumerate}

  В результате в отображениях произвольных изоб\-ражений обеспечивается 
установленное выбранным образцом распределение контрастов. В~качестве 
начального образца распределения Lab-конт\-рас\-тов можно выбрать, 
например, равноконтрастный образец, который определяется функцией 
расстояния между RGB-пик\-се\-ла\-ми, или образец подобия оригиналу, 
который определяется функцией Lab-конт\-рас\-тов исходных тоновых шкал.
  
  Модификация образца приводит к модификации распределения 
  Lab-конт\-рас\-тов. Модификация Lab-конт\-рас\-тов приводит к 
модификации распределения реальных контрастов. Таким образом, нужного 
распределения реальных контрастов на отображениях преобразованных 
ступенчатых тоновых шкал можно добиться, если модифицировать образец, 
ориентируясь на соотношение графиков функций образца и Lab-конт\-рас\-тов 
и распределение контрастов в отображениях преобразованных тоновых шкал.
  
  Из-за дискретности RGB-ко\-ор\-ди\-нат функции Lab-конт\-рас\-тов лишь 
приближенно соответствуют образцам при использовании любой функции 
  Lab-конт\-раст\-но\-го градационного преобразования. Кроме того, 
погрешность вычисления функции Lab-конт\-рас\-тов зависит от значений 
параметров применяемых функций преобразования, от значений координат 
векторов образцов, а также от характеристик и цифрового описания цветового 
пространства в восприятии пользователя. 
  
  Очевидно, что чем точнее вычисляется функция Lab-конт\-рас\-тов по 
отношению к образцу, тем быст\-рее можно получить нужный результат. 
В~связи с этим в рамках данной работы рассматривается задача оптимизации 
функции Lab-конт\-раст\-но\-го градационного преобразования~--- выбора 
такой функции преобразования, при которой распределение Lab-конт\-рас\-тов 
было бы максимально близким к образцу при произвольных образцах и 
пространствах цветовосприятия. 

\section{Обозначения и~определения}
  
  В рамках данной работы рассматривается несколько цифровых описаний 
цветовосприятия виртуальных наблюдателей. Обозначим через~$\varphi_n$,\linebreak 
$n\hm\in 0, 1, \ldots , 5$, RGB-функ\-ции, описывающие цветовосприятие 
некоторых типичных наблюдателей с аномалией цветового зрения в 
соответствии с моделями цветовосприятия из~\cite{10-ar, 11-ar}: $\varphi_0$~--- 
для дейтеранопов~\cite{10-ar}; $\varphi_1$~--- для дейтеранопов~\cite{11-ar}; 
$\varphi_2$~--- для протанопов~\cite{10-ar}; $\varphi_3$~--- для 
протанопов~\cite{11-ar}; $\varphi_4$~--- для тританопов~\cite{10-ar}; 
  $\varphi_5$~--- для тританопов~\cite{11-ar}. 
  
  Пусть $M$~--- множество RGB-пик\-се\-лов, координаты которых кратны 
семнадцати. Тогда
  $$
  \varphi_n(M)\subset M\,,\quad  n\in 0, 1, \ldots , 5\,.
  $$
  
  Обозначим через $\varphi_6$ RGB-функ\-цию, описывающую идеальное 
цветовосприятие при идеальном цветовоспроизведении, при котором
  $$
  \varphi_6(M) = M\,.
  $$
  
  Обозначим через $\varphi_n$, $n\in7, 8, \ldots , 13$, функции 
  RGB-ха\-рак\-те\-ри\-за\-ции цветовосприятия, вычисленные по результатам 
визуального тестирования различения цветных пикселов из множеств 
$\varphi_{n-6}(M)$ в соответствии с~[1--5].
  
  Рассматривается следующая совокупность тоновых шкал:
  $$
  S_i=\left\{ s_{i,j}\right\}\,,\enskip i=0,1, \ldots, 6, \ j=0,1,\ldots , J_i^S\,,
  $$
где 
\begin{align*}
S_0 &= \{s_{0,j}\},\ s_{0,j} = (j, j, j);\\[2pt]
S_1 &= \{s_{1,j}\}\cup \{s_{1,j+15}\},\ s_{1,j} = (j, 0, 0),\\
&\hspace*{40mm}s_{1,j+15} = (255, j, j);\\[2pt]
S_2 &= \{s_{2,j}\}\cup \{s_{2,j+15}\},\ s_{2,j} = (j, j, 0),\\
&\hspace*{40mm} s_{2,j+15} = (255, 255, j);\\[2pt] 
S_3 &= \{s_{3,j}\}\cup \{s_{3,j+15}\},\ s_{3,j} = (0, j, 0),\\
& \hspace*{40mm}s_{3,j+15} = (j, 255, j);\\[2pt]
S_4 &= \{s_{4,j}\}\cup \{s_{4,j+15}\},\ s_{4,j} = (0, j, j),\\
&\hspace*{40mm} s_{4,j+15} = (j, 255, 255);\\[2pt]
S_5 &= \{s_{5,j}\}\cup \{s_{5,j+15}\},\ s_{5,j} = (0, 0, j),\\
& \hspace*{40mm}s_{5,j+15} = (j, j, 255);\\[2pt]
S_6 &= \{s_{6,j}\}\cup \{s_{6,j+15}\},\ s_{6,j} = (j, 0, j),\\
& \hspace*{5mm}s_{6,j+15} = (255, j, 255), j = 0, 1, 
\ldots, 255\,.
\end{align*}
  
  Последовательность $S_i$ будем называть носителем ступенчатой тоновой 
шкалы $T_i\hm=\{t_{i,j}\}$, если $T_i$ является 
подпоследовательностью~$S_i$, состоящей из всех таких пикселов~$S_i$, 
координаты которых кратны семнадцати.
  
  Как и ранее, для описания Lab-конт\-рас\-та RGB-пик\-се\-лов~$x^\prime$ 
и~$x^{\prime\prime}$ будем использовать $E(x^\prime, x^{\prime\prime}$)~--- 
значение расстояния между образами RGB-пик\-селов~$x^\prime$ 
и~$x^{\prime\prime}$ в Lab-про\-стран\-стве.
    \mbox{Исходное} распределе\-ние Lab-конт\-рас\-тов на $\{t_{i,j}\}$ описывается 
последовательностью $\{E(t_{i,j},t_{i,j+1})\}$, на $\{\varphi_n(t_{i,j})\}$~--- 
$\{E(\varphi_n(t_{i,j}), \varphi_n(t_{i,j+1}))\}$.
  
  В качестве образцов $\tau_l(t_{i,j})$ будут использоваться произвольные 
функции, обладающие следующими свойствами:
  \begin{multline*}
\tau_l(t_{i,j})\geq 0\,,\quad \sum\limits_{j=0}^{J_i^T-1} \tau_l(t_{i,j})>0\,,\\
l=0,1,\ldots;\ i=0,1,\ldots, 6; \ j=0,1,\ldots , J_i^T\,.
\end{multline*}

\section{Lab-контрастное градационное преобразование по~образцу}

  Надо найти функцию $F(t_{i,j})$, при которой распределение контрастов на 
преобразованной шкале соответствовало бы образцу~$\tau_l$:
  \begin{multline}
  \fr{E(\varphi_n(F(t_{i,j})), \varphi_n(F(t_{i,j+1})))} 
{\sum\limits_{m=0}^{J_i^T-1}E(\varphi_n(F(t_{i,m})),\varphi_n(F(t_{i,m+1})))} = {}\\
{}=
\fr{\tau_l(t_{i,j})}{\sum\limits_{m=0}^{J_i^T-1}\tau_l(t_{i,m})}\,,\quad j<J_i^T\,.
  \label{e1-ar}
  \end{multline}
  
  Очевидно, что в поле вещественных чисел решение уравнений~(\ref{e1-ar}) 
существует. Но в данном случае из-за дискретности координат 
  RGB-пик\-се\-лов возможно только приближенное вычисление функции~$F$. 
  
  Пусть на носителях тоновых шкал выбраны градации~--- некоторые 
подпоследовательности
  \begin{multline*}
  X_k=\{x_{k,i,j}\}\,, \enskip i\in 0,1,\ldots, 6,\\ 
  j\in 0,1,\ldots ,\ \   J_{k,i}^X\,;\enskip  k\in  0,1,\ldots , K\,,
  \end{multline*}
причем
$$
x_{k,l,0} =s_{i,0}\,,\enskip  x_{k,i,J_i^X}=s_{i,J_i^S}\,.
$$
  
  По заданным градациям найдем последовательности 
  $$
  Y_{k,l,n} =\left\{ y_{k,l,n,i,j}\right\} \subset S_i\,,
  $$
удовлетворяющие соотношениям: 

\noindent
\begin{multline}
\fr{E(\varphi_n(x_{k,i,j}),\varphi_n(x_{k,i,j+1}))}
{\sum\limits_{m=0}^{J^X_{k,i}-1} 
E(\varphi_n(x_{k,i,m}),\varphi_n(x_{k,i,m+1}))}={}\\
{}=
\fr{\tau_l(y_{k,l,n,i,j})} {\sum\limits_{m=0}^{J_{k,i}^X-1} \tau_l 
(y_{k,l,n,i,m})}\,,\quad j<J^X_{k,i}\,.
\label{e2-ar}
\end{multline}
  
  Определим значения функции~$\tau_l$ на всех пикселах носителя~$s_{i,j}$ с 
помощью линейной интерполяции по двум ближайшим к ним пикселам 
ступенчатых тоновых шкал (пусть $t_{i,j^\prime}$ и $t_{i,j^\prime+1}$):
\begin{multline*}
  \tau_l(s_{i,j}) =\fr{1}{17}\left( 17(j^\prime+1)-j\right) \tau_l(t_{i,j^\prime})+{}\\
  {}+
  (j-17j^\prime)\tau_l (t_{i,j^\prime+1})\,,\enskip
  17j^\prime\leq j\leq 17(j^\prime+1)\,.
  \end{multline*}
  
  Заметим, что уравнения из~(\ref{e2-ar}) в общем случае не имеют точного 
решения. Перепишем их в эквивалентном виде
  \begin{multline*}
  \fr{\sum\limits_{m=0}^j E(\varphi_n(x_{k,i,m}),\varphi_n(x_{k,i,m+1}))}
  {\sum\limits_{m=0}^{J^X_{k,i}-1} 
E(\varphi_n(x_{k,i,m}),\varphi_n(x_{k,i,m+1}))} = {}\\
{}=
  \fr{\sum\limits_{m=0}^j \tau_l(y_{k,l,n,i,m})}
  {\sum\limits_{m=0}^{J_{k,i}^X-1} \tau_l(y_{k,l,n,i,m})}\,,\quad j< 
J_{k,i}^X\,,
  \end{multline*}
и на этой основе перейдем к условиям, из которых найдем приближенные 
значения компонентов искомых последовательностей~$Y_{k,l,n}$:
\begin{multline*}
\fr{\sum\limits_{m=0}^{j^\prime(k,l,n,i,j)-1} \tau_l(s_{i,m})}
{\sum\limits_{m=0}^{J^X_{k,i}-1}\tau_l(s_{i,m})}\leq{}\\
{}\leq
\fr{\sum\limits_{m=0}^j E(\varphi_n(x_{k,i,m}),\varphi_n(x_{k,i,m+1}))}
{\sum\limits_{m=0}^{J^X_{k,i}-1} E 
(\varphi_n(x_{k,i,m}),\varphi_n(x_{k,i,m+1}))}\leq{}\\
{}\leq
\fr{\sum\limits_{m=0}^{j^\prime(k,l,n,i,j)}\tau_l(s_{i,m})}
{\sum\limits_{m=0}^{J^X_{k,i}-1}\tau_l(s_{i,m})},\
j<  J^X_{k,i},\  j^\prime(k,l,n,i,j)<J_i^S.
\end{multline*}
  
  Полагая
  \begin{align*}
  y_{k,l,n,i,j}&\approx s_{i,j^\prime(k,l,n,i,j)}\,;\\
  x_{k,i,j} &=f_{k,l,n}\left(s_{i,j^\prime(k,l,n,i,j)}\right)\approx
  f_{k,l,n}(y_{k,l,n,i,j})\,,
  \end{align*}
получим:
{\small\begin{multline*}
\hspace*{-5.79pt}\fr{E(\varphi_n(f_{k,l,n}(y_{k,l,n,i,j})),\varphi_n(f_{k,l,n}(y_{k,l,n,i,j+1})))}
{\sum\limits_{m=0}^{J^X_{k,i}-1} 
\!E(\varphi_n(f_{k,l,n}(y_{k,l,n,i,m})),\varphi_n(f_{k,l,n}(y_{k,l,n,i,m+1})))} 
\approx{}\\
{}\approx
\fr{\tau_l(y_{k,l,n,i,j})}
{\sum\limits_{m=0}^{J^X_{k,i}-1}\tau_l(y_{k,l,n,i,m})}\,,\enskip
j<J^X_{k,i}\,.
%\label{e3-ar}
\end{multline*}}
  
  Теперь можно с помощью интерполяции определить значения функции 
$f_{k,l,n}$ во всех точках носителей, в том числе и в точках ступенчатых тоновых 
шкал
  \begin{multline*}
  \fr{E(\varphi_k(f_{k,l,n}(t_{i,j})),\varphi_k(f_{k,l,n}(t_{i,j+1})))}
  {\sum\limits_{m=0}^{J^X_{k,i}-1} E (\varphi_k(f_{k,l,n}(t_{i,m})),\varphi_k 
(f_{k,l,n}(t_{i,m+1})))}
  \approx{}\\
  {}\approx
  \fr{\tau_l(t_{i,j})}
  {\sum\limits_{m=0}^{J^X_{k,i}-1} \tau_l (y_{k,l,n,i,m})}\,,\quad 
  j<J_i^T\,.
%  \label{e4-ar}
  \end{multline*}
  
\section{Погрешность вычисления функции~$F$}

  Обозначим через $\varepsilon_{k,l,n}$ погрешность вычисления функции~$F$ в 
случае, когда функция $f_{k,l,n}$ используется в качестве ее приближенного 
значения. В~качестве значения $\varepsilon_{k,l,n}$ используем меру бли\-зости 
между векторами на основе косинуса:
  $$
  \varepsilon_{k,l,n} =\rho\left(\vec{a}_{k,l,n},\vec{b}_l\right) =1-\cos \left( 
\vec{a}_{k,l,n},\vec{b}_l\right)\,,
  $$
где
\begin{multline*}
\vec{a}_{k,l,n}= \{a_{k,l,n,i,j}\}\,;\\ 
a_{k,l,n,i,j} =  E(\varphi_n(f_{k,l,n}(t_{i,j})),\ 
\varphi_n(f_{k,l,n}(t_{i,j+1})))\,;
\end{multline*}

\vspace*{-12pt}

\noindent
\begin{multline*}
\vec{b}_l = \{b_{l,i,j}\}\,;\enskip
b_{l,i,j} = \tau_l(t_{i,j})\,,\\ i = 0, 1, \ldots , 6,\ j = 0, 1, \ldots ,  J^T_i-1\,;
\end{multline*}
$$
\cos\left( \vec{a}_{k,l,n},\vec{b}_l\right) = 
\fr{\sum\limits_{i=0}^6 \sum\limits_{j=0}^{J_i^T-1} a_{k,l,n,i,j} b_{l,i,j}}
{\sum\limits_{i=0}^6 \sum\limits_{j=0}^{J^T_i-1} a^2_{k,l,n,i,j} 
\sum\limits_{i=0}^6 \sum\limits_{j=0}^{J_i^T-1} b^2_{l,i,j}}\,.
$$


  
  Авторами была написана специальная программа, с помощью которой были 
вычислены значения $\varepsilon_{k,l,n}$ при~$X_k$, $k \hm= 1, 2, \ldots, 36$, 
$\tau_l$, $l \hm= 0, 1$, и~$\varphi_n$, $n \hm= 0, 1, \ldots , 13$.
  
  При выборе градаций носителей ступенчатых тоновых шкал~$X_k$, $k \hm= 
1, 2, \ldots , 36$, было использовано два способа. При $k \hm\leq 18$ в 
качестве~$X_k$ выбирались подпоследовательности пикселов носителей 
ступенчатых тоновых шкал с шагом по индексу, равным~$k$, а при $19 \hm\leq 
k \hm\leq 36$~--- с шагом по контрасту, равным числу $1\hm+0{,}5 (k \hm - 
18)$. 
  
  При $l = 0$ значения функции $\tau_{l}(t_{i,j})$ определялись величиной 
расстояния между компонентами ступенчатых тоновых шкал, а при $l \hm= 
1$~---  величиной Lab-конт\-рас\-та между ними.
  
\section{Выбор функции Lab-контрастного градационного 
преобразования}

  Обозначим через $\varepsilon_{37,l,n}$ погрешности вида
  $$
  \varepsilon_{37,l,n} =\min\limits_{1\leq k\leq 36} \varepsilon_{k,l,n}\,,
  $$
а через $\vec{V}_k$~--- векторы вида
$$
\vec{V}_k=\left\{ v_{k,l+2n}\right\} =\left\{\varepsilon_{k,l,n}\right\}\,,\enskip 
0\leq l+2n\leq 27\,.
$$


  
  Необходимо определить, при каком значении~$j^\prime$ выполнено 
соотношение 
  $$
  \rho\left( \vec{V}_{37},\vec{V}_{j^\prime}\right) =\min\limits_{1\leq j\leq 36} 
\rho \left( \vec{V}_{37},\vec{V}_j\right)\,.
  $$
  
  Обозначим через $\vec{P}_i\hm=\left\{ p_{i,j}\right\}$ векторы, со\-став\-лен\-ные 
из расстояний между векторами~$\vec{V}_i$ и~$\vec{V}_j$: 
  $$
  p_{i,j} =\rho \left( \vec{V}_i,\vec{V}_j\right)\,,\enskip i=1,2,\ldots , 37\,,\ 
j=1,2,\ldots , 36\,.
  $$
  
  Было установлено, что наименьшей координатой вектора~$\vec{P}_{37}$ 
является координата с индексом, равным~24. Следовательно, 
наиболее близким к вектору~$\vec{V}_{37}$  является вектор~$\vec{V}_{24}$ 
(рис.~1). Это означает оптимальность функции $f_{24,l,k}$ и соответствующего 
метода преобразования ступенчатых тоновых шкал. При определении $f_{24,l,k}$ 
градации носителей ступенчатых тоновых шкал выбирались с шагом по 
контрасту, равным четырем. 
  
  Среди функций, при определении которых градации выбирались с шагом по 
индексу, наилучшей оказалась функция $f_{15,l,k}$, поскольку наиболее близким 
к вектору~$\vec{V}_{37}$ среди соответствующего множества векторов 
является вектор~$\vec{V}_{15}$. 

\begin{center}  %fig1
\vspace*{-3pt}
\mbox{%
 \epsfxsize=72.254mm
 \epsfbox{arh-1.eps}
 }
  \end{center}
  
    \vspace*{-3pt}

\noindent
{{\figurename~1}\ \ \small{Соотношение координат векторов: \textit{1}~--- $\vec{V}_1$; \textit{2}~--- 
$\vec{V}_{15}$; \textit{3}~--- $\vec{V}_{24}$; \textit{4}~--- $\vec{V}_{37}$}}


%\pagebreak

\vspace*{12pt}

\begin{center}  %fig2
\mbox{%
 \epsfxsize=74.31mm
 \epsfbox{arh-2.eps}
 }
  \end{center}

  \vspace*{-3pt}

\noindent
{{\figurename~2}\ \ \small{Соотношение координат векторов: \textit{1}~--- $\vec{P}_1$; \textit{2}~--- 
$\vec{P}_{15}$; \textit{3}~--- $\vec{P}_{24}$; \textit{4}~--- $\vec{P}_{37}$}}

\vspace*{12pt}

\addtocounter{figure}{2}

  
  Наиболее далеким от вектора $\vec{V}_{37}$ является вектор~$\vec{V}_1$; 
следовательно, $f_{1,l,k}$ является наихудшей среди рассмотренных функций. 
  


  
  Близость векторов $\vec{V}_{15}$, $\vec{V}_{24}$, $\vec{V}_{37}$ и их 
отличие от вектора~$\vec{V}_1$ проявляется и в близости векторов 
$\vec{P}_{15}$, $\vec{P}_{24}$, $\vec{P}_{37}$ и их различии с 
вектором~$\vec{P}_1$ (рис.~2). 


  
\section{Сравнение результатов применения функций~$f_{1,l,k}$, 
$f_{15,l,k}$, $f_{24,l,k}$ на примере }
  
  Серая шкала $T_0$, исходно имеющая вид, отраженный на рис.~3,\,\textit{а}, 
в восприятии пользователя, цветовосприятие которого описывается 
функцией~$\varphi_{11}$ ($k\hm=11$), приобретает зеленоватый оттенок и 
выглядит как на рис.~3,\,\textit{б}. Можно видеть, что в области теней теряется 
контрастность пикселов.

\setcounter{figure}{5}
\begin{figure*}[b] %fig6
\vspace*{9pt}
 \begin{center}
 \mbox{%
 \epsfxsize=140.833mm
 \epsfbox{arh-6.eps}
 }
 \end{center}
 \vspace*{-6pt}
\Caption{Изображения: (\textit{а})~Img; (\textit{б})$~\varphi_{11}(\mathrm{Img})$
}
\end{figure*}
  
  После равноконтрастного градационного преобразования (рис.~3,\,\textit{в}\,--\,3,\,\textit{д}) 
  исходный контраст на\linebreak\vspace*{-12pt}



\noindent
\begin{center}  %fig3
\vspace*{-18pt}
\mbox{%
 \epsfxsize=68.608mm
 \epsfbox{arh-3.eps}
 }
  \end{center}

  \vspace*{-3pt}

\noindent
{{\figurename~3}\ \ \small{Вид пикселов шкал: (\textit{а})~$T_0$; (\textit{б})~$\varphi_{11}(T_0)$; 
(\textit{в})~$f_{1,0,11}(T_0)$; (\textit{г})~$f_{15,0,11}(T_0)$; (\textit{д})~$f_{24,0,11}(T_0)$; 
(\textit{е})~$f_{24,2,11}(T_0)$}}

\vspace*{18pt}


\noindent 
 серой шкале практически 
восстанавливается. При
 этом после преобразований с помощью функций $f_{1,0,11}$, $f_{15,0,11}$ и $f_{24,0,11}$ (рис.~4) 
наиболее точное соответствие распределения контрастов образцу 
соответствует функции $f_{24,0,11}$.
  
  Выбор образца $\tau_2$ (рис.~5) позволяет значительно увеличить 
контраст между первыми двумя пикселами ступенчатой тоновой шкалы 
(см.\ рис.~3,\,\textit{е}).

  



  Рассмотрим изображение Img (рис.~6,\,\textit{а}). В~восприятии 
пользователя, цветовосприятие которого описывается функцией~$\varphi_{11}$, 
Img принимает вид, показанный на рис.~6,\,\textit{б}. Как и у 
отображения серой шкалы, у отображения Img детализация искажается, 
поскольку уменьшается контрастность пикселов в области теней. 

  Предварительное Lab-конт\-раст\-ное градационное преобразование с 
помощью функции $f_{24,0,11}$ восстанавливает детализацию изображения в 
восприятии рассматриваемого пользователя (рис.~7,\,\textit{а}). 

%\linebreak\vspace*{-12pt} 
\begin{center}  %fig4
\mbox{%
 \epsfxsize=43.389mm
 \epsfbox{arh-4.eps}
 }
  \end{center}

  \vspace*{-9pt}

\noindent
{{\figurename~4}\ \ \small{Соотношение образцов~$\tau_0$~(\textit{1}) 
и контрастов пикселов преобразованных шкал~(\textit{2}): 
(\textit{а})~$f_{1,0,11}(T_0)$; (\textit{б})~$f_{15,0,11}(T_0)$;
(\textit{в})~$f_{24,0,11}(T_0)$}}

\vspace*{3pt}

%\addtocounter{figure}{2}




\begin{center}  %fig5
\vspace*{-2pt}
\mbox{%
 \epsfxsize=43.389mm
 \epsfbox{arh-5.eps}
 }
  \end{center}

  \vspace*{-3pt}

\noindent
{{\figurename~5}\ \ \small{Соотношение образцов~$\tau_2$~(\textit{1}) и контрастов пикселов преобразованных 
шкал~(\textit{2}) $f_{24,2,11}(T_0)$}}

\vspace*{12pt}

\addtocounter{figure}{1}

  
  

%\noindent

  
  Предварительное Lab-конт\-раст\-ное градационное преобразование с 
помощью функции $f_{24,2,11}$ улучшает детализацию изображения в 
восприятии рассматриваемого пользователя (рис.~7,\,\textit{б}), поскольку 
позволяет выявить новые детали в области теней.


\setcounter{figure}{6}
\begin{figure*} %fig7
\vspace*{1pt}
 \begin{center}
 \mbox{%
 \epsfxsize=140.864mm
 \epsfbox{arh-7.eps}
 }
 \end{center}
 \vspace*{-6pt}
\Caption{Изображения: (\textit{а})~$\varphi_{11}(f_{24,0,11}(\mathrm{Img}))$; 
(\textit{б})~$\varphi_{11}(f_{24,2,11}(\mathrm{Img}))$
}
\end{figure*}
  
\section{Заключение}
  
  Рассмотрена задача оптимизации функции Lab-конт\-раст\-но\-го 
градационного преобразования ступенчатых тоновых шкал. 
  
  Задача решена путем сравнения погрешности вычисления функции 
распределения Lab-конт\-рас\-тов на ступенчатых тоновых шкалах на 
типичных примерах. 
  
  Установлено, что наименьшую погрешность обеспечивает функция 
  Lab-конт\-раст\-но\-го градационного преобразования ступенчатых тоновых 
шкал, основанная на выборе градаций носителей ступенчатых тоновых шкал с 
шагом по контрасту, равным четырем. Эта функция рекомендуется для 
применения на практике. 
  
  Приведен пример, демонстрирующий эффективность применения функции 
Lab-конт\-раст\-но\-го градационного преобразования ступенчатых тоновых 
шкал. Показано, что при выборе подходящего образца можно не только 
избежать искажения детализации, но и добиться ее улучшения.
  
{\small\frenchspacing
{%\baselineskip=10.8pt
\addcontentsline{toc}{section}{Литература}
\begin{thebibliography}{99}
\bibitem{1-ar}
\Au{Архипов О.\,П., Зыкова З.\,П.} Допечатное тестирование индивидуального 
зрительного восприятия~// Вестник компьютерных и информационных 
технологий, 2008. №\,12. С.~2--8.
\bibitem{2-ar}
\Au{Архипов О.\,П., Зыкова З.\,П.} Интеграция гетерогенной информации о 
цветных пикселях и их цветовосприятии~// Информатика и её применения, 
2010. Т.~4. Вып.~4. С.~14--25.
\bibitem{3-ar}
\Au{Архипов О.\,П., Зыкова З.\,П.} Функциональное описание индивидуального 
цветовосприятия~// Информационные сис\-те\-мы и технологии, 2010. №\,5. 
С.~5--12.
\bibitem{4-ar}
\Au{Архипов О.\,П., Зыкова З.\,П.} RGB-ха\-рак\-те\-ри\-за\-ция пространства 
цветовосприятия~// Сис\-те\-мы и сред\-ст\-ва информатики, 
2010. Вып.~20. №\,1. С.~73--90.
\bibitem{5-ar}
\Au{Архипов О.\,П., Зыкова З.\,П.} Многокритериальный выбор тестового 
множества при исследовании цветовосприятия~// Информационные 
технологии, 2011. №\,2. С.~67--73.
\bibitem{6-ar}
\Au{Архипов О.\,П., Зыкова З.\,П.} Равноконтрастные градационные 
преобразования ступенчатых тоновых шкал~// Информационные сис\-те\-мы и 
технологии, 2011. №\,4. С.~39--46. 
\bibitem{7-ar}
\Au{Архипов О.\,П., Зыкова З.\,П.} Персонифицированное преобразование 
пред\-став\-ле\-ний цвет\-ных изоб\-ра\-же\-ний на мониторе ПЭВМ~// 
Сис\-те\-мы и средства информатики, 2012. Т.~22. №\,1. С.~22--37.
\bibitem{8-ar}
\Au{Архипов О.\,П., Зыкова З.\,П.} Метод улучшения детализации цветных 
изображений~// Вестник компьютерных и информационных технологий, 2013. 
№\,2. С.~6--11.
\bibitem{9-ar}
\Au{Архипов О.\,П., Зыкова З.\,П.} Коррекция детализации представлений 
RGB-изоб\-ра\-же\-ний на периферийных устройствах ПЭВМ~// 
Информационные технологии, 2013. №\,2. С.~56--60.
\bibitem{10-ar}
Color Oracle/Institute of Cartography.~--- Zurich: ETH, 2008. {\sf 
http://www.colororacle.org}.
\bibitem{11-ar}
Vischeck. {\sf http://www.vischeck.com}. 
\end{thebibliography} } }



\end{multicols}

\hfill{\small\textit{Поступила в редакцию 25.03.13}}


%\vspace*{12pt}

%\hrule

%\vspace*{2pt}

%\hrule

\newpage

\vspace*{-24pt}

\def\tit{FUNCTIONS OPTIMIZATION OF LAB-CONTRAST GRADED TRANSFORMATION}

\def\titkol{Functions optimization of Lab-contrast graded transformation}

\def\aut{O.\,P.~Arkhipov and Z.\,P.~Zykova}
\def\autkol{O.\,P.~Arkhipov and Z.\,P.~Zykova}


\titel{\tit}{\aut}{\autkol}{\titkol}

\vspace*{-15pt}

\noindent
Oryol Branch, Institute of Informatics 
Problems, Russian Academy of Sciences, Oryol 302025, Russian Federation


 
\def\leftfootline{\small{\textbf{\thepage}
\hfill INFORMATIKA I EE PRIMENENIYA~--- INFORMATICS AND APPLICATIONS\ \ \ 2013\ \ \ volume~7\ \ \ issue\ 4}
}%
 \def\rightfootline{\small{INFORMATIKA I EE PRIMENENIYA~--- INFORMATICS AND APPLICATIONS\ \ \ 2013\ \ \ volume~7\ \ \ issue\ 4
\hfill \textbf{\thepage}}}   

\vspace*{6pt}
  
\Abste{The purpose of this study is to create algorithm for custom conversion of 
contrast distribution on stepped-tone scales. The idea is to find suitable Lab-contrast 
distribution, keeping in mind that scaled distortion of digital representation of pixels 
would keep the real perception of color contrast all the same, if only less bright. In 
order to approximately calculate the function of Lab-contrast gradual conversion, this 
study considers two families of parametric algorithms, the first one uses subset of pixels 
on the scale as parameters, while the another one uses many gradations as they are. The problem 
of selecting the set of optimal parameters is solved by comparison of range of 
calculation errors achieved on a set of typical examples by the function of Lab-contrast 
distribution on stepped-tone scales. The study shows the function of Lab-contrast 
distribution on stepped-tone scales that gives the least error and provides example 
that proves that with optimal set of parameters chosen, it is not only possible to avoid 
visual distortion but the details can also be improved. These findings may be useful in 
management of maps of RGB-color images for PC peripherals in order to improve 
perceptive quality of details.}

\KWE{color reproduction; color perception; Lab-coordinate; contrast; gradation}
  
  
\DOI{10.14357/19922264130405}

%\Ack
%\noindent
%?????

  \begin{multicols}{2}

\renewcommand{\bibname}{\protect\rmfamily References}
%\renewcommand{\bibname}{\large\protect\rm References}

{\small\frenchspacing
{%\baselineskip=10.8pt
\addcontentsline{toc}{section}{References}
\begin{thebibliography}{99}

  \bibitem{1-ar-1}
  \Aue{Arkhipov, O.\,P., and Z.\,P.~Zykova}. 2008. Dopechatnoe testirovanie 
individual'nogo zritel'nogo vospriyatiya [Preprinting test of individual visual 
perception]. \textit{Vestnik Komp'yuternykh i
Informatsionnykh Tekhnologiy~---
Herald of Computer and Information Technologies}  12:2--8.
  \bibitem{2-ar-1}
  \Aue{Arkhipov, O.\,P., and Z.\,P.~Zykova}. 2010. Integratsiya geterogennoy
informatsii o tsvetnykh pikselyakh i ikh tsvetovospriyatii [Integration of heterogeneous 
information about color pixels and their color perception]. \textit{Informatika i ee Primeneniya~---
Inform. Appl.} 4(4):14--25.
  \bibitem{3-ar-1}
  \Aue{Arkhipov, O.\,P., and Z.\,P.~Zykova}. 2010. Funktsional'noe opisanie 
individual'nogo tsvetovospriyatiya [Characteristics of color perceptual space]. 
\textit{Informatsionnye Sistemy i Tekhnologii} [\textit{Information Systems and 
Technologies}] 5:5--12.
  \bibitem{4-ar-1}
  \Aue{Arkhipov, O.\,P., and Z.\,P.~Zykova}. 2010. RGB-kharakterizatsiya 
prostranstva tsvetovospriyatiya [RGB-characterization of color perception 
space]. \textit{Sistemy i Sredstva Informatiki~---
Systems and Means of Informatics} 20(1):73--90.
  \bibitem{5-ar-1}
  \Aue{Arkhipov, O.\,P., and Z.\,P.~Zykova}. 2011. Mnogokriterial'nyy vybor 
testovogo mnozhestva pri issledovanii tsvetovospriyatiya [Multicriterion choice of 
test set when studying the color perception]. \textit{Inform. Tekhnologii} 
[\textit{Information Technologies}] 2:67--73.
  \bibitem{6-ar-1}
  \Aue{Arkhipov, O.\,P., and Z.\,P.~Zykova}. 2011. Ravnokontrastnye gradatsionnye 
preobrazovaniya stupenchatykh tonovykh shkal [Equal contrast graded transformation 
of step tinted scales]. \textit{Informatsionnye Sistemy i Tekhnologii} 
[\textit{Information Systems and Technologies}] 4:39--46.
  \bibitem{7-ar-1}
  \Aue{Arkhipov, O.\,P., and Z.\,P.~Zykova}. 2012. Personifitsirovannoe 
preobrazovanie predstavleniy tsvetnykh izobrazheniy  na monitore PEVM [Personified 
transformation of color images presentations on a PC monitor]. \textit{Sistemy i
Sredstva Informatiki~--- Systems and 
Means of Informatics} 22(1):22--37.
  \bibitem{8-ar-1}
  \Aue{Arkhipov, O.\,P., and Z.\,P.~Zykova}. 2013. Metod uluch\-she\-niya detalizatsii 
tsvetnykh izobrazheniy [Method of improving the detail of color images]. 
\textit{Vestnik Komp'yuternykh i Informatsionnykh Tekhnologiy~--- Herald of Computer 
and Information Technologies} 2:6--11.
  \bibitem{9-ar-1}
  \Aue{Arkhipov, O.\,P., and Z.\,P.~Zykova}. 2013. Korrektsiya de\-ta\-li\-za\-tsii 
predstavleniy RGB-izobrazheniy na periferiynykh ustroystvakh PEVM [Correcting of 
detail presentations of RGB-images on peripherals of PC]. \textit{Informatsionnye 
Tekhnologii} [\textit{Information Technologies}] 2:56--60.
  \bibitem{10-ar-1}
ETH Zurich.  2008. {Color Oracle/Institute of Cartography}.   Available at: 
{\sf http://www.colororacle.org} (accessed November~5, 2013).


 
\bibitem{11-ar-1}
Vischeck. 2008. Available at: {\sf http://www.vischeck.com}
(accessed November~3, 2013).
\end{thebibliography}
} }


\end{multicols}

\vspace*{-6pt}

\hfill{\small\textit{Received March 25, 2013}}

\vspace*{-18pt}

\Contr

\noindent
\textbf{Arkhipov Oleg P.} (b.\ 1948)~--- Candidate of Science (PhD) in technology, 
Director, Oryol Branch of the Institute of Informatics Problems, Russian Academy 
of Sciences, Oryol 302025, Russian Federation; arkhipov12@yandex.ru 


\vspace*{2pt}

\noindent
\textbf{Zykova Zoya P.} (b.\ 1953)~--- Candidate of Science (PhD) in physics and 
mathematics, Head of Laboratory, Oryol Branch of the Institute of Informatics 
Problems, Russian Academy of Sciences, Oryol 302025, Russian Federation; zykzoya@yandex.ru

 \label{end\stat}
 
\renewcommand{\bibname}{\protect\rm Литература}  
  
  