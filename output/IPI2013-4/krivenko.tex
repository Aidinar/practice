\def\stat{krivenko}

\def\tit{АНАЛИЗ ОДНОРОДНОСТИ ДАННЫХ О~ХИМИЧЕСКОМ СОСТАВЕ КАМНЕЙ ПРИ~УРОЛИТИАЗЕ}

\def\titkol{Анализ однородности данных о~химическом составе камней при~уролитиазе}

\def\autkol{М.\,П.~Кривенко, С.\,А.~Голованов, А.\,В.~Сивков}

\def\aut{М.\,П.~Кривенко$^1$, С.\,А.~Голованов$^2$, А.\,В.~Сивков$^3$}

\titel{\tit}{\aut}{\autkol}{\titkol}

%{\renewcommand{\thefootnote}{\fnsymbol{footnote}}\footnotetext[1] {Статья 
%рекомендована к публикации в журнале Программным комитетом конференции 
%<<Электронные библиотеки: перспективные методы и технологии, электронные 
%коллекции>> (RCDL-2012).}}

\renewcommand{\thefootnote}{\arabic{footnote}}
\footnotetext[1]{Институт проблем информатики Российской академии наук, 
mkrivenko@ipiran.ru} 
\footnotetext[2]{Научно-исследовательский институт 
урологии, sergeygol124@mail.ru} 
\footnotetext[3]{Научно-исследовательский институт урологии, uroinfo@yandex.ru} 

 
      \Abst{Рассмотрены методы исследования однородности статистических данных о 
химическом составе камней при уролитиазе. Многомерность данных, бедный спектр 
различных значений показателей, невозможность использовать простые адекватные 
вероятностные модели для них, а также отсутствие накопленного опыта подобного 
моделирования в области урологии придают специфику рас\-смат\-ри\-ва\-емым задачам. 
Предложен и проанализирован критерий значимости, основанный на расстоянии между 
множествами точек в евклидовом пространстве. Он применен для анализа зависимости 
состава камней от пола пациента. При анализе изменения состава камней от времени 
предложенный критерий использован как элемент парных сравнений данных для различных 
временн$\acute{\mbox{ы}}$х фрагментов. В~этом случае впервые применены идеи аппроксимации 
возникающих связей между фрагментами данных с помощью отношения эквивалентности. 
Это позволяет обоснованно проводить стратификацию исходных данных. В~ходе 
экспериментов получены статистически обоснованные результаты как об особенностях 
мочекаменной болезни в зависимости от пола пациента, так и об их изменении с течением 
времени.}
      
      \KW{критерии однородности; стратификация; бут\-стреп-ме\-тод; мочекаменная 
болезнь}

\DOI{10.14357/19922264130410}

\vskip 14pt plus 9pt minus 6pt

      \thispagestyle{headings}

      \begin{multicols}{2}

            \label{st\stat}
      
\section{Введение}
      
      Заболеваемость уролитиазом существенно варьи\-руется в различных странах мира. 
В~терминах частоты встречаемости этого заболевания на протяжении жизни индивида в 
исследуемой популяции (lifetime prevalence) распространенность мочекаменной болезни 
(МКБ) составляет в странах Азии 1\%--5\%, Европы~--- 5\%--9\%, Северной Америки~--- 
13\%, достигая 20\% среди населения Саудовской Аравии~[1]. 
      
      Несмотря на то что эпидемиологические исследования, проведенные в 10~странах, 
указывают на определенное сходство распределения типов уролитиаза, определяемых по 
химическому составу конкрементов~[2], химический состав мочевых камней у больных в 
разных странах все же имеет свои особенности. Кроме того, замечено, что в одном и том же 
регионе клинические и метаболические характеристики МКБ могут существенно изменяться 
с течением времени~[1, 3--5].
      
      Существующие различия, как полагают, имеют тесную связь с инфекцией мочевых 
путей, нарушением уродинамики, изменениями обмена веществ и фи\-зи\-ко-хи\-ми\-че\-ских 
свойств мочи, а также с факторами окружающей среды, характером питания населения и 
      со\-ци\-аль\-но-эко\-но\-ми\-че\-ски\-ми условиями~[6].
      
      В связи с этим целью настоящей работы было изучение особенностей МКБ и 
распространен\-ности ее метаболических типов по результатам исследования минерального 
состава мочевых камней пациентов, проходивших лечение в клинике НИИ урологии МЗ РФ 
(г.~Моск\-ва) и городской клинической урологической больнице №\,47 г.~Моск\-вы в период 
с 2005 по 2009~гг. Данная работа является частью продолжающегося исследования 
особенностей распространенности метаболических типов МКБ в московском регионе, 
начатого в НИИ урологии с 1985~г.~[7]. 
      
      Сравнительный анализ данных по химическому составу мочевых камней за 
многолетний период представляет как эпидемиологический, так и практический интерес для 
клинической урологии, позволяя выявить определенные тенденции распространенности 
типов МКБ за исследуемый период. Например, полученные данные по частоте выявления 
основных метаболических типов мочевых камней среди мужского и женского населения\linebreak 
г.~Моск\-вы могут быть использованы при планировании оказания специализированной 
урологической помощи и разработке соответствующих ле\-чеб\-но-про\-фи\-лак\-ти\-че\-ских 
мероприятий. Результаты проведен\-ной работы могут стать основой для дальнейших 
аналитических эпидемиологических исследований, направленных на выявление факторов 
риска и их устранение, в целях снижения заболеваемости МКБ среди населения.
      
      Исследование реальных клинических баз данных играет важную роль для развития 
теории и практики методов их анализа, давая толчок к детализации постановок задач 
обработки данных, вынуждая развивать отдельные подходы по выявлению скрытых 
(латентных) связей и структур в этих данных. И~наконец, только проводя прикладные 
исследования, можно понять истинную значимость предлагаемых наработок в области 
прикладной математики и информатики. 
      
      Материалом для исследования послужили резуль\-таты анализа химического состава 
4217~мочевых конкрементов, удаленных оперативно, фрагментированных с помощью 
дистанционной ли\-то\-трип\-сии или отошедших самостоятельно. Всего обследовано 
2413~мужчин и 1804~женщин в воз\-рас\-те от 16 до 77~лет, больных МКБ.
      
      В данной работе подробно рассматриваются задачи анализа однородности данных о 
составе камней в двух конкретных постановках: по имеющимся данным о составе камней 
требуется исследовать его зависимость от пола пациента и времени. Решение этих задач 
должно включать как выявление (обнаружение) наличия зависимости, так и формирование 
предположений, какие особенности состава камней характерны для того или иного пола или 
промежутка времени. 

\section{Общая характеристика данных}

      Объектом анализа стали данные, включающие указание пола пациента (M~--- 
мужской и F~--- женский), результаты измерений состава камней, даты этих измерений. 
Мочевые камни на 98\%--99\% состоят из кристаллической фазы, представленной чаще всего 
оксалатами и фосфатами кальция, мочевой кислотой и ее солями. Кристаллическая фаза 
может содержать один, два и более минеральных компонентов~[8]. Поэтому состав камней 
представляется либо с помощью первичных признаков, отражающих тип минеральной 
кристаллической фазы, либо с помощью вторичных или интегральных признаков. Для 
первичных признаков (переменных) приняты следующие обозначения: WH~--- вевеллит 
(кальция оксалат моногидрат); WD~--- ведделлит (кальция оксалат дигидрат);\linebreak UA~--- 
мочевая кис\-ло\-та; UADH~--- мочевая кис\-ло\-та дигидрат; AMUR~--- (мо\-но)ам\-мо\-ния урат; 
NAUR~--- (мо\-но)нат\-рия урат моногидрат; WTL~--- витлокит (трикальция фосфат); CA~--- 
карбонатапатит; HAP~--- гид\-ро\-кси\-а\-па\-тит; BRU~--- брушит; STRU~--- струвит; 
      CYS~--- L-цис\-тин. Вторичные признаки суть: O~--- оксалаты; U~--- ураты; P~--- 
фосфаты. При этом 
\begin{align*}
\mathrm{O}&=\mathrm{WH+WD}\,; \\
\mathrm{U}&=\mathrm{UA}+\mathrm{UADH}+\mathrm{AMUR}+ \mathrm{NAUR}\,;\\ 
\mathrm{P}&=\mathrm{WTL}+\mathrm{CA}+\mathrm{HAP}+\mathrm{BRU}+\mathrm{STRU}\,.
\end{align*}

 Все первичные признаки характеризуют долю 
соответствующего вещества в единице веса отдельного камня, т.\,е.\ 
\begin{multline*}
\mathrm{WH}+\mathrm{WD}+\mathrm{UA}+\mathrm{UADH}+
      \mathrm{AMUR}+\mathrm{NAUR}+{}\\
      {}+\mathrm{WTL}+\mathrm{CA}+
      \mathrm{HAP}+\mathrm{BRU}+\mathrm{STRU}+\mathrm{CYS}=100\,.
\end{multline*}
      
      В имеющихся данных обращают на себя внимание следующие моменты:
      \begin{itemize}
\item высокая повторяемость значений признаков (среди 4217~наблюдений встречается 
425~различных значений 12-мер\-ных векторов первичных признаков, а различных 
значений 3-мер\-ных векторов вторичных признаков всего~114);
\item отсутствие даже намека на возможность использовать модель нормального 
распределения в качестве вероятностной модели данных.
\end{itemize}

      Особое место занимают камни, содержащие CYS. Перечень встречающихся 
комбинаций различных значений состава камней с ненулевым значением CYS приведен в 
табл.~1. Из нее видно, что: 
      \begin{itemize}
      \item
имеется 49~случаев, когда речь идет о пациентах со 100\%-ным содержанием CYS в камнях, 
они могут быть выделены в самостоятельную группу; 
\item остальные два случая из-за малочисленности не играют при анализе данных 
существенной роли.
\end{itemize}

      Наличие связей между признаками дает толчок к исследованию возможности 
снижения раз\-мер\-ности. В~этом плане эффективность перехода к первым главным 
компонентам иллюстрируется значениями доли суммарной дисперсии, приведенными в 
табл.~2 для различных величин сниженных размерностей признакового пространства. Из 
этой таб\-ли\-цы видна возможность снижения раз\-мер-\linebreak\vspace*{-12pt}

\begin{center}  %tabl1
\vspace*{6pt}
\parbox{60mm}{{{\tablename~1}\ \ \small{Состав камней с ненулевым значением CYS}}}

\vspace*{2ex}

 
{\small \begin{tabular}{|c|c|c|c|c|}
\hline
O&U&P&CYS&Число значений\\
\hline
0&0&\hphantom{9}0&100\hphantom{9}&49\\
0&0&10&90&\hphantom{9}1\\
0&0&80&20&\hphantom{9}1\\
\hline
\end{tabular}}
\end{center}

%\vspace*{6pt}

\begin{center}  %tabl2
%\vspace*{6pt}
\parbox{64mm}{{{\tablename~2}\ \ \small{Доля суммарной дисперсии (в \%) для сниженной размерности данных}}}

\vspace*{2ex}

 
{\small 
\begin{tabular}{|c|c|c|c|c|c|c|c|c|c|c|}
\hline
Сниженная &\multicolumn{2}{c|}{Признаки}\\
\cline{2-3}
размерность & Первичные &Вторичные\\
\hline
1&45&64\\
2&66&99\\
3&85&100\hphantom{9}\\
4&90&---\\
5&94&---\\
6&96&---\\
7&98&---\\
8&99&---\\
$\cdots$&$\cdots$ &---\\
12\hphantom{9}&100\hphantom{9}&---\\
\hline
\end{tabular}}
\end{center}

\vspace*{6pt}


\addtocounter{table}{2}

\noindent
ности (например, задавшись желанием 
обеспечить при снижении размерности не менее 95\% суммарной дисперсии, можно перейти 
к 6-мер\-ным данным в случае первичных признаков и к 2-мер\-ным для вторичных 
признаков). Такое снижение размерности вряд ли имеет значение для уменьшения 
алгоритмической сложности обработки данных, а вот с точки зрения повышения качества 
статистического вывода это обычно дает преимущества.
    

      
      Довольно бедный спектр различных значений данных, невозможность применить 
простые адекватные вероятностные модели для них, а также отсутствие накопленного опыта 
подобного моделирования в области урологии вынуждают обратиться в первую очередь к 
непараметрическим методам анализа данных.
      
      Задачу анализа зависимости состава камней от пола можно сформулировать как 
задачу сравнения двух выборок, соответствующих тому или иному полу. Анализ 
временн$\acute{\mbox{ы}}$х зависимостей можно также свести к парному сравнению 
подвыборок, соответствующих определенным интервалам времени. При этом важно не 
только проверить гипотезу об однородности двух выборок, но и получить представление о 
характере отличий, если они есть. 
      
\section{Критерии однородности двух выборок}

      Достаточно богатый арсенал подходов и решений двух сформулированных задач 
анализа данных (обнаружение и оценивание) имеется в одномерном случае, но это не 
означает, что их можно считать полностью решенными. Как пример здесь достаточно 
упомянуть проблему Бе\-рен\-са--Фи\-ше\-ра (Behrens--Fi\-sher) о проверке гипотезы 
равенства средних двух независимых нормальных выборок с неизвестными дисперсиями. 
Как уже отмечалось, наибольший интерес вызывают непараметрические постановки задач и 
соответствующие методы анализа, но при их обобщении на многомерный случай появляются 
дополнительные трудности: неоднозначность понятия медианы, ранга, зависимость 
статистик от исходного распределения. Как следствие, возникают и развиваются новые 
подходы к построению методов анализа однородности данных для многомерных моделей, в 
частности различные способы обобщения понятий медианы, знака и ранга на многомерный 
случай~[9, 10]. 
      
      Наиболее известными являются двухвыборочные критерии для параметра 
положения~[9, 11]. К~ним относятся критерий Хотеллинга (Hotelling), многомерный 
      Maн\-на--Уит\-ни--Вил\-кок\-со\-на (Mann--Whitney--Wil\-coxon) критерий, 
многомерный критерий Муда (Mood). В~качестве параметра положения могут выступать 
средние или медианы в за\-ви\-си\-мости от принятой модели.
      
      При использовании критерия Хотеллинга предполагается нормальное распределение 
данных и осуществляется проверка гипотезы об одно\-род\-ности относительно вектора 
средних. В~по\-стро\-ении статистики участвуют выборочные средние и ковариационные 
матрицы; ее распределение при условии нулевой гипотезы известно. Если пред\-положение о 
нормальности не действует, то мож-\linebreak но обратиться к многомерной версии критерия\linebreak 
      Maн\-на--Уит\-ни--Вил\-кок\-со\-на. Соответствующая ста\-тистика $W^2$ строится на 
основе объединения ранговых статистик для отдельных компонент, известно ее 
асимптотическое распределение. На идее объединения результатов покомпонентного 
анализа строится и знаковая статистика~$T^2$ медианного критерия Муда. Для нее известно 
предельное распределение. Для конечных объемов выборок распределения~$W^2$ и~$T^2$ 
не являются свободными от распределения данных.
      
      Необходимость предположений о непре\-рыв\-ности распределений и того, что различие 
выборок описывается параметром положения, приводит к тому, что перечисленные 
статистики малопригодны при сравнительном анализе данных о составе камней.
      
      С позиций устойчивости статистического вывода конкурентом вектора средних 
является вектор медиан~\cite{9-kri}. Здесь можно рассматривать вектор покоординатных 
медиан, пространственную медиану, многомерную медиану Ойя (Oja), многомерную 
медиану Лиу (Liu), существуют и другие менее известные обобщения. Многообразие 
вариантов объясняется желанием авторов обеспечить такие свойства характеристик 
положения, как эффективность, эквивариантность, робастность, вычислительные удобства. 
К~сожалению, для рассматриваемой ситуации с анализом состава камней все эти обобщения 
не позволяют освободиться от ярко выраженной повторяемости данных и, как следствие, 
сводят на нет декларируемые преимущества.
      
      Как новый позиционирует автор~[12] критерий, основанный на использовании 
многомерных квантильных функций и сравнении параметров биномиального распределения. 
В~предложенной редакции этот критерий не представляет интереса по следующим 
причинам: 
      \begin{itemize}
\item выделение всего одного множества, за вероятностью попадания в которое ведется 
контроль, приведет к малой мощности критерия, что наиболее ярко должно проявиться 
при возрастании размерности пространства данных в силу <<проклятия размерности>>; 
\item сводить итоговое принятие решения об однородности двух выборок на основе 
асимптотических результатов о равенстве частоты попадания данных в некоторые 
множества заданному\linebreak значению неправильно, так как существует (см.~[13,  разд.~4.5]) 
равномерно наиболее мощный несмещенный критерий сравнения двух 
биномиальных совокупностей, основанный на гипергеометрическом распределении. Для 
последнего известно, что аппроксимация с помощью нормального распределения крайне 
неудовлетворительная (вопросы аппроксимации гипергеометрического распределения 
рас\-смат\-ри\-ва\-лись в докладе~[14]). 
\end{itemize}
      
      Подводя итог, можно сделать следующий общий вывод: отсутствие 
сформировавшихся моделей и ярко выраженная дискретность данных приводят к 
необходимости привлекать при анализе критериев значимости методы управления 
обработкой выборок, а раз так, то выбор этих методов может быть достаточно 
произвольным. Важно, чтобы они по возможности были чувствительны к различным 
отклонениям от нулевой гипотезы об од\-но\-род\-ности двух выборок и не выходили бы за 
границы реальных вычислительных возможностей при их воплощении. 
      
      В данной работе получил развитие подход, основанный на использовании расстояния 
между множествами точек. Имеющиеся выборки объемом   и   представляют собой два 
множества точек в евклидовом пространстве, поэтому естественно в качестве статистики для 
проверки гипотезы об од\-но\-род\-ности взять ка\-кое-то расстояние между этими множествами. 
В~частности, это может быть сумма всех расстояний между элементами одного множества и 
элементами другого. Данное интуитивное пред\-став\-ле\-ние о мере близости двух множеств 
укрепляется на основе доказанного в~[15] сле\-ду\-юще\-го факта: для независимых случайных 
векторов $X_1$, $X_2$, $Y_1$, $Y_2$, где $X_1$, $X_2$ имеют одно и то же 
распределение~$F$ с конечным значением $E\{\parallel X_1\parallel\}$ и $Y_1$, $Y_2$ 
имеют одно и то же распределение~$G$ с конечным значением $E\{\parallel Y_1\parallel\}$, 
действует неравенство
      \begin{multline}
      E\{\parallel X_1-Y_1\parallel\}-\fr{1}{2}\,E\left\{ \parallel X_1-X_2\parallel\right\}-{}\\
      {}-
\fr{1}{2}\,E\left\{ \parallel Y_1-Y_2\parallel\right\}\geq 0\,,
      \label{e1-kri}
      \end{multline}
которое становится равенством тогда и только тогда, когда $F\hm=G$. Заменяя левую 
часть~(\ref{e1-kri}) ее выборочным аналогом, помноженным на $mn/(m+n)$, получаем 
статистику $T_{m,n}$ для проверки нулевой гипотезы~$H: F\hm=G$ против 
конку\-ри\-ру\-ющей~\mbox{$K: F\not= G$}. Она имеет следующий вид:
\begin{multline}
T_{m,n} =\fr{mn}{m+n}\left[ \fr{1}{mn} \sum\limits_{j=1}^m \sum\limits_{k=1}^n \parallel 
X_j-Y_k\parallel -{}\right.\\
{}-\fr{1}{2m^2}\sum\limits_{j=1}^m \sum\limits_{k=1}^m \parallel X_j-
X_k\parallel - {}\\
\left.{}-\fr{1}{2n^2}\sum\limits_{j=1}^n \sum\limits_{k=1}^n \parallel Y_j-Y_k\parallel 
\right]\,.
\label{e2-kri}
\end{multline}
Здесь $(X_1,\ldots ,X_m)$ суть элементы одной выборки, $(Y_1,\ldots, Y_n)$~--- другой. 
Отклонение нулевой гипотезы происходит при больших значениях статистики~$T_{m,n}$. 
Распределение статистики критерия неизвестно. Поэтому для получения критических 
уровней значимости приходится прибегать к бут\-стреп-ме\-то\-ду. В~этой связи важным 
результатом является доказательство в~\cite{15-kri} того, что распределение статистики при 
нулевой гипотезе является пределом ее бут\-стреп-рас\-пре\-де\-ле\-ния при $m,n\hm\to 
\infty$ так, что $m/n\hm\to \tau\hm\in (0,1)$.

\begin{table*}\small %tabl3
\begin{center}
\Caption{Степень сжатия представления данных}
\vspace*{2ex}

\begin{tabular}{|l|c|c|c|}
\hline
\multicolumn{1}{|c|}{Признаки}&Общее количество данных&Число различных значений&Коэффициент сжатия\\
\hline
Первичные&\multicolumn{1}{c|}{\raisebox{-6pt}[0pt][0pt]{4217}}&425&10\\
%\cline{1-1}
%\cline{3-4}
Вторичные&&114&37\\
\hline
\end{tabular}
\end{center}
\vspace*{-3pt}
\end{table*}

\begin{table*}[b]\small %tabl4
\vspace*{-9pt}
\begin{center}
\Caption{Результаты проверки нулевой гипотезы об однородности данных для различных полов}
\vspace*{2ex}

\begin{tabular}{|l|c|c|c|c|c|}
\hline
&&&&&\\[-9pt]
\multicolumn{1}{|c|}{Признаки}&Сниженная размерность&$T^*_{mn}$&$\max\limits_B T^B_{mn}$&Вывод для нулевой 
гипотезы&$\hat{a}^B$\\
\hline
Первичные &6&\hphantom{9}763,2&323,8&Отвергается&0\%\\
Вторичные &2&1339,1&356,4&Отвергается&0\%\\
\hline
\end{tabular}
\end{center}
\end{table*}


      Отсутствие априорной информации о распределении данных приводит к 
непараметрическому бут\-стреп-ме\-то\-ду, т.\,е.\ к использованию в качестве 
      бут\-стреп-рас\-пре\-де\-ле\-ния смеси 
      $$H_{m,n}=\fr{m}{m+n}\,F_m\hm+\fr{n}{m+n}\,G_n\,,
      $$
      где $F_m$ и~$G_n$~--- эмпирические функции 
распределения для $F$ и~$G$ соответственно (например, для $F_m$~--- это появление одного из 
$X_1,\ldots ,X_m$ с вероятностью~$1/m$). Для получения критических уровней значимости 
необходимо описать распределение статистики $T_{m,n}$ для данных из распределения 
$H_{m,n}$; это можно сделать либо теоретическим путем, либо с помощью метода 
статистических испытаний. Теоретический подход не годится в основном из-за 
вычислительных сложностей: достаточно большие объемы исходных данных полностью 
исключают прямое перечисление всевозможных бут\-стреп-вы\-бо\-рок, а аналитический 
вывод распределения $T_{m,n}$ при всей его неочевидности, скорее всего, приведет к 
громоздким комбинаторным формулам, расчет по которым сопряжен с большими 
вычислительными погрешностями. Поэтому использовался метод статистических 
испытаний.
      
      В этом случае многократное вычисление статистики $T_{m,n}$, включающей 
двойные суммы, создаст проблемы временного характера при реализации 
      бут\-стреп-ме\-то\-да, решить которые можно, используя особенность данных о 
составе камней~--- относительную бедность спектра значений. Таблица~3 демонстрирует 
возможности сжатия данных при переходе от исходных данных к их представлению в виде 
совокупности различных значений.
      
      Пусть $(Z_1,\ldots, Z_{m+n})\hm= (X_1,\ldots , X_m,\,Y_1,\ldots$\linebreak $\ldots , Y_n)$, 
$(\tilde{Z}_1,\ldots ,\tilde{Z}_k)$~--- различные среди $(Z_1,\ldots ,Z_{m+n})$ значения, 
$C(l)$~--- индекс $\tilde{l}$ значения $Z_l$ в последовательности $(\tilde{Z}_1,\ldots 
,\tilde{Z}_k)$, т.\,е. $\tilde{Z}_{C(l)} \hm= Z_l$, где $1\hm\leq l\hm\leq m\hm+n$ и $1\hm\leq 
\tilde{l}\hm\leq k$. Это отоб\-ра\-же\-ние~$C$ строится единожды в процессе формирования 
$(\tilde{Z}_1,\ldots ,\tilde{Z}_k)$) из $(Z_1,\ldots , Z_{m+n})$. Теперь можно заранее 
вычислить все необходимые значения $\parallel \tilde{Z}_i\hm- \tilde{Z}_j\parallel$ и 
использовать их при нахождении слагаемых в~(2), а генерацию бут\-стреп-вы\-бо\-рок 
осуществлять на множестве $\{ 1, 2, \ldots , m+n\}$
      
      Для обнаружения зависимости состава камней от пола пациента критерий $T_{m,n}$ 
используется напрямую, в случае же зависимости от времени~--- как основа для парных 
сравнений групп (фрагментов) данных, относящихся к определенному промежутку времени 
(например, в один год). 
      
\section{Эксперименты}

      Для количества бут\-стреп-экспе\-ри\-мен\-тов далее было принято значение 10$^4$, 
что обеспечивало достаточно точное представление для единиц процента при оценивании 
критического уровня $\hat{a}^B$ описанного критерия значимости. Для задачи сравнения 
составов камней у мужчин и женщин полученные результаты приведены в табл.~4. Из них 
следует, что есть веские основания для отклонения нулевой гипотезы об однородности. 
Кроме крайне малых значений оценок критического уровня для наглядности приведены 
максимальные значения статистики критерия, встретившиеся в ходе 
      бут\-стреп-экспе\-ри\-мен\-тов, которые также показывают существенное отклонение 
при нулевой гипотезе статистики критерия от обычных значений:
      $$
      T^*_{mn}>\max\limits_B T^B_{mn}\,.
      $$
      


      Таким образом, обосновано наличие зави\-си\-мости состава камней от пола, причем 
значимость этого вывода весьма высока. Но примененный метод анализа однородности 
данных ничего не говорит о том, что меняется в составе камней в зависимости от пола 
пациентов. 







Далее речь пойдет о вторичных признаках O, U и P. Напомним, что для 
большинства данных действует соотношение: 
$\mathrm{O}\hm+\mathrm{U}\hm+\mathrm{P}\hm=100$. Задача состоит в формировании 
предположений относительно того, в чем проявляется отличие состава камней для мужчин и 
женщин. Визуальный анализ пар вторичных признаков выявил следующие связи: данные 
преимущественно сосредоточиваются вдоль отрезков на осях координат и вдоль отрезков, 
принадлежащих прямым вида: 
$$
\mathrm{O}+\mathrm{U}=100;\  
\mathrm{O}+\mathrm{P}=100;\ \mathrm{U}+\mathrm{P}=100.
$$

В~плоскости 
$\mathrm{O}\hm+\mathrm{U}\hm+\mathrm{P}\hm=100$ были выделены области (для 
гистограммной оценки распределения их также называют ячейками), описанные в табл.~5.

\begin{table*}\small %tabl5
\begin{center}
\Caption{Вид ячеек для группирования данных}
\vspace*{2ex}

\begin{tabular}{|c|l|l|}
\hline
№&\multicolumn{1}{c|}{Состав ячеек}&\multicolumn{1}{c|}{Наличие в составе камней}\\
\hline
1&$\Delta_{\mathrm{O}}=\{\mathrm{abs}\,(\mathrm{O}-100)<\delta\}$&Только оксалаты\\
2&$\Delta_{\mathrm{U}}=\{\mathrm{abs}\,(\mathrm{U}-100)<\delta\}$&Только ураты\\
3&$\Delta_{\mathrm{P}}=\{\mathrm{abs}\,(\mathrm{P}-100)<\delta\}$&Только фосфаты\\
4&$\Delta_{\mathrm{OU}}= \{\mathrm{abs}\,(\mathrm{O+U-100})<\delta\}-\Delta_{\mathrm{O}}-
\Delta_{\mathrm{U}}$&Только оксалаты и ураты\\
5&$\Delta_{\mathrm{OP}}= \{\mathrm{abs}\,(\mathrm{O+P-100})<\delta\}-\Delta_{\mathrm{O}}-
\Delta_{\mathrm{P}}$&Только оксалаты и фосфаты\\
6&$\Delta_{\mathrm{UP}}= \{\mathrm{abs}\,(\mathrm{U+P-100})<\delta\}-\Delta_{\mathrm{U}}-
\Delta_{\mathrm{P}}$&Только ураты и фосфаты\\
7&$\Delta_{\mathrm{C}}=\mathcal{R}^3-\Delta_{\mathrm{O}}-\Delta_{\mathrm{U}}-
\Delta_{\mathrm{P}}-\Delta_{\mathrm{OU}}-\Delta_{\mathrm{OP}}-
\Delta_{\mathrm{UP}}$&Все остальные\\
\hline
\end{tabular}
\end{center}
\end{table*}

\begin{figure*}[b]
   \vspace*{1pt}
 \begin{center}
 \mbox{%
 \epsfxsize=161.936mm %.973mm
 \epsfbox{kri-1-2.eps}
 }
 \end{center}
 \vspace*{-6pt}
   \begin{minipage}[t]{80mm}
   \Caption{Область возможных значений состава камней на плоскости 
$\mathrm{O}\hm+\mathrm{U}\hm+\mathrm{P}\hm=100$}
%\end{figure}
\end{minipage}
\hfill
%\begin{figure}
\vspace*{-6pt}
\begin{minipage}[t]{80mm}
\Caption{Распределение данных о составе камней}
\end{minipage}
\end{figure*}

      
      Для визуализации все данные были спроецированы на плоскость 
      $\mathrm{O}\hm+\mathrm{U}\hm+\mathrm{P}\hm=100$ (рис.~1). Для этого 
достаточно воспользоваться преобразованием поворота, матрица которого имеет вид:
      $$
      T=\begin{pmatrix}
      \fr{1}{\sqrt{2}} & 0 & -\fr{1}{\sqrt{2}}\\
      -\fr{1}{\sqrt{6}} & \fr{2}{\sqrt{6}} & -\fr{1}{\sqrt{6}}
      \end{pmatrix}\,.
      $$

      Далее находились частоты попадания данных в перечисленные ячейки отдельно для 
мужчин и женщин. Общая картина распределения данных приведена на рис.~2. На нем из 
общего объема данных в количестве 4217~рассматривались лишь 114~различных вариантов 
состава камней.

      \begin{table*}\small %tabl6
%\vspace*{-12pt}
\begin{center}
\Caption{Сравнительный анализ состава камней для различных полов}
\vspace*{2ex}

\begin{tabular}{|l|c|c|c|c|c|c|}
\hline
\multicolumn{1}{|c|}{Тип}&
\multicolumn{2}{c|}{Абсолютная частота}&\multicolumn{2}{c|}{Относительная 
частота, \%}&$\hat{a}$, &Направление\\
\cline{2-5}
\multicolumn{1}{|c|}{камней}&\ \ \ \ Пол М\ \ \ \ &\ \ \ \ Пол F\ \ \ \ &\ \ \ \ Пол М\ \ \ \ &Пол F&\%& изменения\\
\hline
\hspace*{3mm}$\Delta_{\mathrm{O}}$&1144\hphantom{9}&717&47,4&39,7&0,0&Уменьшение\\
\hspace*{3mm}$\Delta_{\mathrm{U}}$&273&160&11,3&\hphantom{9}8,9&0,5&Уменьшение\\
\hspace*{3mm}$\Delta_{\mathrm{P}}$&127&213&\hphantom{9}5,3&11,8&0,0&Увеличение\\
\hspace*{3mm}$\Delta_{\mathrm{OU}}$&164&106&\hphantom{9}6,8&\hphantom{9}5,9&12,6\hphantom{9}&Нет\\
\hspace*{3mm}$\Delta_{\mathrm{OP}}$&669&579&27,7&32,1&0,1&Увеличение\\
\hspace*{3mm}$\Delta_{\mathrm{UP}}$&\hphantom{99}4&\hphantom{99}8&\hphantom{9}0,2&\hphantom{9}0,4&2,5&Нет\\
\hspace*{3mm}$\Delta_{C}$&\hphantom{9}32&\hphantom{9}21&\hphantom{9}1,3&\hphantom{9}1,2&37,4\hphantom{9}&Нет\\
\hline
Сумма&2413\hphantom{9}&1804\hphantom{9}&&&&\\
\hline
\end{tabular}
\end{center}
\end{table*}



     
      Теперь каждая полученная пара частот попадания в ячейку может быть формально 
исследована  с целью проверки гипотезы о равенстве частот и, если они не равны, для 
выяснения направления изменения. Полученные результаты сведены в табл.~6.

\begin{table*}[b]\small
\begin{center}
\parbox{355pt}{\Caption{Критические уровни значимости (в \%) критерия $T_{m,n}$ в случае мужского пола}
}


\vspace*{2ex}

%\tabcolsep=6.5pt
\begin{tabular}{|c|c|c|c|c|c|c|c|c|c|c|}
\hline
& \multicolumn{10}{c|}{$j$}\\
\cline{2-11}
$i$&\multicolumn{5}{c|}{Первичные признаки}&\multicolumn{5}{c|}{Вторичные признаки}\\
\cline{2-11}
& 1&2&3&4&5& 1&2&3&4&5\\
\multicolumn{1}{|p{20pt}|}{\hspace*{20pt}}&
\multicolumn{1}{p{20pt}|}{\hspace*{20pt}}&
\multicolumn{1}{p{20pt}|}{\hspace*{20pt}}&
\multicolumn{1}{p{20pt}|}{\hspace*{20pt}}&
\multicolumn{1}{p{20pt}|}{\hspace*{20pt}}&
\multicolumn{1}{p{20pt}|}{\hspace*{20pt}}&
\multicolumn{1}{p{20pt}|}{\hspace*{20pt}}&
\multicolumn{1}{p{20pt}|}{\hspace*{20pt}}&
\multicolumn{1}{p{20pt}|}{\hspace*{20pt}}&
\multicolumn{1}{p{20pt}|}{\hspace*{20pt}}&
\multicolumn{1}{p{20pt}|}{\hspace*{20pt}}\\[-12pt]
\hline
1&\cellcolor[gray]{.6}&1,3&\hphantom{9}2,2&\hphantom{9}0,3&0,0&\cellcolor[gray]{.6}&18,9&49,4&81,7&0,0\\
2&\cellcolor[gray]{.6}&\cellcolor[gray]{.6}&36,0&\hphantom{9}8,3&0,0&\cellcolor[gray]{.6}&\cellcolor[gray]{.6}&36,8&\hphantom{9}5,0&0,0\\
3&\cellcolor[gray]{.6}&\cellcolor[gray]{.6}&\cellcolor[gray]{.6}&46,9&0,0&
\cellcolor[gray]{.6}&\cellcolor[gray]{.6}&\cellcolor[gray]{.6}&25,4&0,0\\
4&\cellcolor[gray]{.6}&\cellcolor[gray]{.6}&\cellcolor[gray]{.6}&\cellcolor[gray]{.6}&0,0&
\cellcolor[gray]{.6}&\cellcolor[gray]{.6}&\cellcolor[gray]{.6}&\cellcolor[gray]{.6}&0,0\\
5&\cellcolor[gray]{.6}&\cellcolor[gray]{.6}&\cellcolor[gray]{.6}&\cellcolor[gray]{.6}&
\cellcolor[gray]{.6}&\cellcolor[gray]{.6}&\cellcolor[gray]{.6}&\cellcolor[gray]{.6}&\cellcolor[gray]{.6}&\cellcolor[gray]{.6}\\
      \hline
      \end{tabular}
      \end{center}
%     \end{table*}
%            \begin{table*}\small
      \begin{center}
      \Caption{Матрицы отношения совпадения в случае мужского пола}
      \vspace*{2ex}
      
%\tabcolsep=10.5pt
\begin{tabular}{|c|c|c|c|c|c|c|c|c|c|c|}
\hline
 & \multicolumn{10}{c|}{$j$}\\
\cline{2-11}
$i$&\multicolumn{5}{c|}{Первичные признаки}&\multicolumn{5}{c|}{Вторичные признаки}\\
\cline{2-11}
& 1&2&3&4&5& 1&2&3&4&5\\
\multicolumn{1}{|p{20pt}|}{\hspace*{20pt}}&
\multicolumn{1}{p{20pt}|}{\hspace*{20pt}}&
\multicolumn{1}{p{20pt}|}{\hspace*{20pt}}&
\multicolumn{1}{p{20pt}|}{\hspace*{20pt}}&
\multicolumn{1}{p{20pt}|}{\hspace*{20pt}}&
\multicolumn{1}{p{20pt}|}{\hspace*{20pt}}&
\multicolumn{1}{p{20pt}|}{\hspace*{20pt}}&
\multicolumn{1}{p{20pt}|}{\hspace*{20pt}}&
\multicolumn{1}{p{20pt}|}{\hspace*{20pt}}&
\multicolumn{1}{p{20pt}|}{\hspace*{20pt}}&
\multicolumn{1}{p{20pt}|}{\hspace*{20pt}}\\[-12pt]
\hline
1&\cellcolor[gray]{.6}&1&1&0&0&\cellcolor[gray]{.6}&1&1&1&0\\
2&\cellcolor[gray]{.6}&\cellcolor[gray]{.6}&1&1&0&\cellcolor[gray]{.6}&\cellcolor[gray]{.6}&1&1&0\\
3&\cellcolor[gray]{.6}&\cellcolor[gray]{.6}&\cellcolor[gray]{.6}&1&0&\cellcolor[gray]{.6}&\cellcolor[gray]{.6}&\cellcolor[gray]{.6}&1&0\\
4&\cellcolor[gray]{.6}&\cellcolor[gray]{.6}&\cellcolor[gray]{.6}&\cellcolor[gray]{.6}&0&
\cellcolor[gray]{.6}&\cellcolor[gray]{.6}&\cellcolor[gray]{.6}&\cellcolor[gray]{.6}&0\\
5&\cellcolor[gray]{.6}&\cellcolor[gray]{.6}&\cellcolor[gray]{.6}&\cellcolor[gray]{.6}&
\cellcolor[gray]{.6}&\cellcolor[gray]{.6}&\cellcolor[gray]{.6}&\cellcolor[gray]{.6}&\cellcolor[gray]{.6}&\cellcolor[gray]{.6}\\
\hline
\end{tabular}
\end{center}
\end{table*}


      Анализ полученных и представленных в табл.~6 результатов позволяет достаточно 
обоснованно сделать следующие выводы:
      \begin{itemize}
\item нет камней по составу явно соответствующих мужчинам или женщинам, т.\,е., 
опираясь на вторичные признаки, нельзя по составу камня определять с малыми 
ошибками пол пациента;
\item состав камней у мужчин и женщин отличаются по частоте встречаемости O и~U 
(уменьшение вероятности встречаемости у женщин по сравнению с мужчинами), а также 
по встречаемости P и O\;+\;P (увеличение вероятности встре\-ча\-емости у женщин по 
сравнению с мужчинами); 
\item ячейка $\Delta_C$ достаточно хорошо представляет тип камней, в составе которых 
присутствует только CYS; в нее попали 49~наблюдений с $\mathrm{CYS}\hm=100$, 
только 2~наблюдения с $\mathrm{CYS}\hm=20$ и $\mathrm{CYS}\hm=90$, а также 
только 2~наблюдения со следующими значениями: $\mathrm{O}\hm=70$, 
$\mathrm{U}\hm=20$, $\mathrm{P}\hm=10$; суммарное количество наблюдений~--- 53 
(см.\ в табл.~6 значения абсолютных час\-тот~32 и~21).
      \end{itemize}
      

      
      В задаче анализа зависимости состава камней от времени все данные были разбиты на 
фрагменты, соответствующие одному году обследования (всего 5~фрагментов). Далее для 
каждого пола в отдельности и каждого набора признаков (первичных или вторичных) 
строилась матрица критических уровней значимости критерия $T_{m,n}$ для сравнения 
      $i$-го и $j$-го фрагментов, $j\hm>i$. Примеры двух подобных матриц для первичных 
и вторичных признаков даны в табл.~7, затемненные клетки определяются очевидным 
образом: на диагонали стоят значения 100\% (принятие нулевой гипотезы), таблица для 
применяемого критерия является симметричной относительно диагонали.
      

      
      Все четыре матрицы (два пола и два типа признаков) свидетельствуют о следующем:
      \begin{itemize}
\item 5-й фрагмент (данные за 2009~г.)\ значимо отличается от остальных, 
\item первые четыре фрагмента в совокупности дают не совсем ясную картину, 
требующую привлечения дополнительных приемов для формирования конкретных 
предположений.
\end{itemize}

      Установим для пары фрагментов отношение совпадения распределений данных для 
этих фрагментов, оно будет выполняться при условии, если соответствующий критический 
уровень значимости будет превосходить некоторый порог. Здесь в качестве порогового было 
принято значение 1\%, оно является привычным в практике анализа данных и служит, по 
мнению авторов работы, компромиссом между ошибками первого и второго рода при 
принятии гипотезы. 

Матрицы данного отношения для двух рас\-смот\-рен\-ных примеров 
представлены в табл.~8.
      

      
      Введенное отношение необязательно является транзитивным (например, для табл.~8 
в случае первич\-ных признаков условие транзитивности нарушает\-ся, а в случае 
      вторичных~--- нет). Но в рас\-смат\-ри\-ва\-емом случае анализа однородности фрагментов 
свойство транзитивности оказывается крайне целесообразным: в дополнение к имеющимся 
рефлексивности и симметрии (см.\ ранее сделанное
 замечание о затемненных клетках 
таблиц) транзитивность приводит к отношению эквивалентности и, следовательно, к 
возможности получить разбиение для исходного множества фрагментов. Так, в случае 
вторичных признаков из табл.~8 получаем, что таковым разбиением является $\{\{1, 2, 3, 4\}, 
\{5\}\}$, что означает следующее: состав камней по вторичным признакам у мужчин в 
      2005--2008~г.\ не изменялся, а 2009~г.\ в этом смысле отличается от предыдущих.
      
      Поставим задачу нахождения транзитивного приближения~--- такого транзитивного 
отношения, которое было бы наиболее близким к задан\-ному. Если $R_1$ и~$R_2$~--- 
матрицы рефлексивных, сим\-мет\-рич\-ных отношений, то для определения их меры близости 
используем расстояние Хемминга, а именно:
\begin{multline*}
\rho (R_1,R_2)= \vert \{ i,j:\ 1< 
i<j<n\,,\ R_1(i,j)\not={}\\
{}\not= R_2(i,j)\}\vert\,.
\end{multline*}
Тогда если $R_0$~--- матрица заданного 
отношения (например, матрица из табл.~8 для первичных признаков), то искомой матрицей 
$R^*_T(R_0)$  из множества $\mathfrak{R}_T$ мат\-риц всех рефлексивных, 
симметричных, транзитивных отношений будет следующая: 
$$
R_T^*(R_0)\hm= \mathrm{arg} 
(\min\limits_{R\in\mathfrak{R}_T} \rho (R_0,R))\,.
$$

 Полученное решение $R^*_T(R_0)$ необязательно единственное. При этом $R^*_T(R_0)$ 
 необязательно является $\mathrm{TrCl}\left(R_0\right)$~--- 
транзитивным замыканием для~$R_0$. Более того, число случаев~$s$, когда $\mathrm{TrCl}\left(R_0\right)$ 
вообще не попадает в множество найден\-ных $R_T^*(R_0)$, увеличивается с ростом~$n$. 
Как иллюстрация этого, табл.~9 содержит первые значения отношения $s/r$, где $r$~--- 
общее число нетранзитивных отношений среди всех рефлексивных и сим\-мет\-рич\-ных 
отношений. Для $n\hm>7$ подобные значения получить за приемлемое время не удается. 
Следует отметить, что для проведенных в рамках работы экспериментов пришлось 
реализовывать алгоритмы перечисления всех рефлексивных и сим\-мет\-рич\-ных отношений и 
перечисления всех отношений эквивалентности (через построение всех разбиений 
множества).

      Для поиска решений $R^*_T(R_0)$ в данной работе был применен алгоритм простого 
перебора: для  всех рефлексивных, симметричных и транзитивных
 отношений 
подсчитывались расстояния $\rho(R_0,R_T)$, минимум $\rho^*$ из которых давал решение 
искомой  задачи. Подобная процедура нахождения $R^*_T(R_0)$
 реально срабатывает при 
$n\hm\leq 14$; причина проста:
быстрый рост мощности перебираемого множества\linebreak\vspace*{-12pt}
\begin{center}  %tabl9
\vspace*{3pt}
\parbox{47mm}{{{\tablename~9}\ \ \small{Доля случаев, когда транзитивное 
замыкание не попадает в множество прибли\-жа\-ющих 
транзитивных отношений}}}

\vspace*{2ex}

 
{\small 
\tabcolsep=14.3pt
\begin{tabular}{|c|c|}
\hline
$n$&$s/r$\\
\hline
3&\hphantom{9}0\%\\
4&57\%\\
5&72\%\\
6&91\%\\
7&94\%\\
$>7$\hphantom{$>\;$} &Не доступно\\
\hline
      \end{tabular}}
      \end{center}
%      \end{table*}


\begin{center}  %tabl10
\vspace*{6pt}
\parbox{60mm}{{{\tablename~10}\ \ \small{Разбиения данных по годам}}}

\vspace*{2ex}

 
{\small 
\begin{tabular}{|c|c|c|c|}
\hline
Пол&Тип признаков&$\rho^*$&$R^*_T$\\
\hline
\multicolumn{1}{|c|}{\raisebox{-6pt}[0pt][0pt]{M}}&Первичные&1&$\{\{1, 2, 3, 4\}, \{5\}\}$\\
\cline{2-4}
&Вторичные&0&$\{\{1, 2, 3, 4\}, \{5\}\}$\\
\hline
\multicolumn{1}{|c|}{\raisebox{-12pt}[0pt][0pt]{F}}&Первичные&1&$\{\{1, 2, 3, 4\}, \{5\}\}$\\
\cline{2-4}
&Вторичные&2&\tabcolsep=0pt\begin{tabular}{c}$\{\{1, 2, 3, 4\}, \{5\}\}$\\ 
$\{1, 2, 3, 4, 5\}$
\end{tabular}\\
\hline
\end{tabular}}
\vspace*{12pt}
\end{center}


      
\noindent
 отношений эквивалентности (это суть числа Белла, а именно: при $n\hm=2$~--- это~2, при 
$n\hm=5$~--- это~52, при $n\hm=10$~--- это 115\,975, при $n\hm=14$~--- это 190\,899\,322, при 
$n\hm=15$~--- это 1\,382\,958\,545).
      
      Напомним, что минимум может быть не единственный, поэтому при выборе 
соответствующего разбиения возникает некоторый произвол. Результаты анализа всех 
исследуемых случаев сведены в
 табл.~10. Для женского пола и вторичных признаков было 
отобрано разбиение $\{\{1, 2, 3, 4\}, \{5\}\}$ как наиболее отвечающее ранее сделанному 
выводу о том, что 5-й фрагмент (данные за 2009~г.)\ значимо отличается от остальных.
      


      Полученные разбиения позволяют высказать предположения о содержании 
изменений в составе камней на рубеже 4-го и 5-го фрагментов (переход от 2008 к 2009~г.). 
Для выделенных типов камней (см.\ табл.~6) исследовалось изменение частот их 
встречаемости для периодов времени 2005--2008~гг.\ и 2009~г. Формальное сравнение 
частот встре\-ча\-емости за эти периоды позволило выделить часть типов камней, для которых 
изменения были значимыми. Для мужского пола речь идет о типах камней, по составу 
входящих в ячейки $\Delta_{\mathrm{O}}$, $\Delta_{\mathrm{OP}}$, 
$\Delta_{\mathrm{UP}}$ и $\Delta_C$ (напомним, что в $\Delta_C$ оказываются камни 
фактически только с CYS), для женского пола~--- это $\Delta_{\mathrm{O}}$ и 
$\Delta_{\mathrm{OP}}$. Соответствующие данные о доле камней, входящих в 
перечисленные ячейки, приведены на рис.~3; на них штриховыми линиями обозначены 
оценки частот~$f$ для разбиения $\{1, 2, 3, 4, 5\}$, а сплошными линиями  
 выделены оценки частот~$f^*$, полученные 
для разбиения $\{\{1, 2, 3, 4\}, \{5\}\}$.

\pagebreak

\end{multicols}

\begin{figure} %fig3
   \vspace*{1pt}
 \begin{center}
 \mbox{%
 \epsfxsize=141.696mm
 \epsfbox{kri-3.eps}
 }
 \end{center}
 \vspace*{-9pt}
\Caption{Изменения состава камней в зависимости от года:
(\textit{а})~пол мужской; (\textit{б})~пол женский;
 \textit{1}~--- O; \textit{2}~--- OP;
\textit{3}~--- UP; \textit{4}~--- C; штриховые линии~--- $f$, разбиение 
$\{1, 2, 3, 4, 5\}$; сплошные линии~--- $f^\prime$, разбиение $\{\{1, 2, 3, 4\}, \{5\}\}$}
%\vspace*{9pt}
\end{figure}

\begin{multicols}{2}

      Приведенные результаты свидетельствуют о реалистичности следующих выводов:
      \begin{itemize}
\item для обоих полов изменения в первую очередь касаются камней типа O и 
$\mathrm{O}\hm+\mathrm{P}$;
\item для мужского пола также характерен рост доли камней, включающих 
$\mathrm{U\hm+P}$ и~C.
\end{itemize}
 
\vspace*{-6pt}

\section{Заключение}

\vspace*{-2pt}

      В статье рассмотрены методы исследования однородности данных, учитывающие 
специфику предметной области: многомерность данных, бедный спектр различных значений 
показателей, невозмож\-ность применить простые адекватные вероятностные модели для них. 

Предлагается и анализируется критерий значимости, основанный на расстоянии между 
множествами точек в евклидовом пространстве, строится эффективная оригинальная 
реализация бут\-стреп-ме\-то\-да. При множественном использовании данного критерия 
впервые применяются идеи аппроксимации возникающих связей между фрагментами 
данных с по\-мощью отношения эквивалентности, что позволяет обоснованно проводить 
разбиение исходных данных. 

В~ходе экспериментов получены статистически обоснованные 
результаты как об особенностях распространенности метаболических типов 
мочекаменной болезни в зависимости от пола пациента, так 
и о характерных изменениях в соотношении этих типов с течением времени.

\vspace*{-6pt}

{\small\frenchspacing
{%\baselineskip=10.8pt
\addcontentsline{toc}{section}{Литература}
\begin{thebibliography}{99}

\vspace*{-2pt}

  \bibitem{1-kri}
  \Au{Ramello A., Vitale~C., Marangella~D.} Epidemiology of nephrolithiasis~// J.~Nephrol., 
2000. Vol.~13. Suppl.~3. P.~45--50.
\bibitem{2-kri}
\Au{Pak C.\,Y., Resnick M.\,I., Preminger~G.\,M.} Ethnic and geographic diversity of stone 
disease~// Urology, 1997. Vol.~50. No.\,4. P.~504--507.
  \bibitem{3-kri}
  \Au{Takasaki E.} Chronologocal variation in the chemical composition of upper urinary tract 
calculi~// J.~Urology, 1986. Vol.~136. No.\,1. P.~5--9.
  \bibitem{4-kri}
  \Au{Trinchieri A., Coppi~F., Montanari~E., Del Nero~A., Zanetti~G., Pisani~E.} Increase in 
the prevalence of symptomatic upper urinary tract stones during the last ten years~// Eur. Urol., 
2000. Vol.~37. P.~23--25.
  \bibitem{5-kri}
  \Au{Arias Funez F., Garcia Cuerpo~E., Lovaco Castellanos~F., Escudero Barrilero~A., Avila 
Padilla~S., Villar Palasi~J.} Epidemiologia de la litiasis urinaria en nuestra Unidad. Evolucion en 
el tiempo y factores predictivos [Epidemiology of urinary lithiasis in our unit. Clinical course in 
time and predictive factors]~// Arch. Esp. Urol., 2000. Vol.~53. No.\,4. P.~343--347.
  \bibitem{6-kri}
  \Au{Тиктинский О.\,Л., Александров~В.\,П.} Мочекаменная болезнь.~--- СПб.: Питер, 
2000. 379~с.
  \bibitem{7-kri}
  \Au{Шуберт Г., Чудновская~М.\,В., Тыналиев~М.\,Т., Поповкин~Н.\,Н., Тимин~А.\,Р.} 
Особенности химического состава и структуры мочевых камней и их распространенность в 
городах Москве, Берлине и Киргизской ССР~// Урология и нефрология, 1990. №\,5. С.~49--54.
  \bibitem{8-kri}
  \Au{Фрейтаг Д., Хруска~К.} Патофизиология нефролитиаза~// Почки и гомеостаз в норме 
и при патологии~/ Под ред.\ С.~Клар; пер. с англ. Е.\,И.~Дайхина.~--- 
М.: Медицина, 1987. С.~390--420. (\Au{Klahr~S.} The kidney and body fluids in health and
desease.~--- N.Y., L.: Plenum Medical Books, 1983.)

  \bibitem{10-kri}
  \Au{Hettmansperger T.\,P.} Multivariate location tests~// Encyclopedia of statistical 
  sciences.~--- N.Y.: John Wiley\,\&\,Sons, 2006. P.~5249--5252.
  
    \bibitem{9-kri}
  \Au{Minimaa A., Oja~H.} Multivariate median~// Encyclopedia of statistical sciences.~--- 
  N.Y.: John Wiley\,\&\,Sons, 2006. P.~5258--5266.

  \bibitem{11-kri}
  \Au{Marden J.\,I.} Multivariate rank tests~// Multivariate analysis, design of experiments and 
survey sampling~/ Ed.\ S.~Ghosh.~--- N.Y.: Marcel Dekker, 1999. P.~401--432.
  \bibitem{12-kri}
  \Au{Чистяков С.\,П.} О~новом многомерном статистическом критерии однородности 
двух выборок~// Тр. Карельского научного центра РАН, 2010. №\,3. С.~93--97.
  \bibitem{13-kri}
  \Au{Леман Э.} Проверка статистических гипотез.~--- М.: Наука, 1979. 408~с.
  \bibitem{14-kri}
  \Au{Кривенко М.\,П.} Задачи выборочного контроля при досмотре лиц, багажа и 
транспорта~// Обозрение прикладной и промышленной математики, 
2011. Т.~18. Вып.~1. С.~125--126.
  \bibitem{15-kri}
  \Au{Baringhaus L., Franz~C.} On a new multivariate two-sample test~// J.~Multivariate 
Anal., 2004. Vol.~88. P.~190--206.

\end{thebibliography} } }

\end{multicols}

\hfill{\small\textit{Поступила в редакцию 05.02.13}}


\vspace*{6pt}

\hrule

\vspace*{2pt}

\hrule

\def\tit{ANALYSIS OF DATA HOMOGENEITY OF~THE~CHEMICAL COMPOSITIONS 
OF~STONES IN~CASE OF~UROLITHIASIS}

\def\aut{M.\,P.~Krivenko$^1$, S.\,A.~Golovanov$^2$,  and~A.\,V.~Sivkov$^2$}

\def\titkol{Analysis of data homogeneity of~the~chemical compositions 
of~stones in~case of~urolithiasis}

\def\autkol{M.\,P.~Krivenko, S.\,A.~Golovanov,  and~A.\,V.~Sivkov}


\titel{\tit}{\aut}{\autkol}{\titkol}

\vspace*{-11pt}

\noindent
$^1$Institute of Informatics 
Problems, Russian Academy of Sciences, Moscow 119333, Russian Federation\\
\noindent
$^2$Research Institute of Urology, Moscow 105425, Russian Federation

  
\def\leftfootline{\small{\textbf{\thepage}
\hfill INFORMATIKA I EE PRIMENENIYA~--- INFORMATICS AND APPLICATIONS\ \ \ 2013\ \ \ volume~7\ \ \ issue\ 4}
}%
 \def\rightfootline{\small{INFORMATIKA I EE PRIMENENIYA~--- INFORMATICS AND APPLICATIONS\ \ \ 2013\ \ \ volume~7\ \ \ issue\ 4
\hfill \textbf{\thepage}}}   

\vspace*{14pt}  
      
      \Abste{The article considers the methods for researching homogeneity of the data on chemical 
composition of urinary stones. Multidimensionality of the data, scarce spectrum of various values of 
indicators, impossibility to apply simple adequate likelihood models to them, and, also, absence of 
experience in such type of modeling in the field of urology are the cause of
the  considered problems 
specificity. The test of significance based on distance between sets of points in Euclidean space is 
suggested and analyzed. It is used to analyze a composition of stones dependence on the sex of a 
patient. In the case of analysis of changes in the composition of stones with time (as time goes), the 
proposed criterion is used as an element of pairwise comparisons of data for different time 
intervals. That is the first time when the idea of approximation of emerging relationships between 
groups of data with the help of equivalence relations is used. It allows reasonably carrying out the 
stratification of the original data. The experiments produced statistically valid results about the 
features of urolithiasis depending both on the sex of the patient and the changes over time.}
      
      \KWE{tests of homogeneity; stratification; bootstrap method; urolithiasis}
      
\DOI{10.14357/19922264130410}

\vspace*{6pt}

%\Ack
%\noindent

\vspace*{-3pt}

  \begin{multicols}{2}

\renewcommand{\bibname}{\protect\rmfamily References}
%\renewcommand{\bibname}{\large\protect\rm References}

{\small\frenchspacing
{%\baselineskip=10.8pt
\addcontentsline{toc}{section}{References}
\begin{thebibliography}{99}

\bibitem{1-kri-1}
\Aue{Ramello, A., C.~Vitale, and D.~Marangella}. 2000. Epidemiology of nephrolithiasis. 
\textit{J.~Nephrol.} 13(3):45--50.
\bibitem{2-kri-1}
\Aue{Pak, C.\,Y., M.\,I.~Resnick, and G.\,M.~Preminger}. 1997. Ethnic and geographic diversity of stone 
disease. \textit{Urology} 50(4):504--507.
\bibitem{3-kri-1}
\Aue{Takasaki, E.} 1986. Chronologocal variation in the chemical composition of upper urinary tract 
calculi. \textit{J.~Urology} 136(1):5--9.
\bibitem{4-kri-1}
\Aue{Trinchieri, A., F.~Coppi, E.~Montanari, A.~Del Nero, G.~Zanetti, and E.~Pisani}. 2000. Increase in 
the prevalence of symptomatic upper urinary tract stones during the last ten years. \textit{Eur. Urol.} 
37:23--25.
\bibitem{5-kri-1}
\Aue{Arias Funez, F., E.~Garcia Cuerpo, F.~Lovaco Castellanos, A.~Escudero Barrilero, S.~Avila Padilla, 
and J.~Villar Palasi}. 2000. Epidemiologia de la litiasis urinaria en nuestra Unidad. Evolucion en el tiempo y 
factores predictivos. [Epidemiology of urinary lithiasis in our unit. Clinical course in time and predictive 
factors]. \textit{Arch. Esp. Urol.} 53(4):343--347.


\bibitem{6-kri-1}
\Aue{Tiktinskij, O.\,L., and V.\,P.~Aleksandrov}. 2000. 
\textit{Mochekamennaya bolezn'} [\textit{Urolithiasis}]. St.\ Petersburg, Russia: Piter. 379~p.
\bibitem{7-kri-1}
\Aue{Shubert, G., M.\,V.~Chudnovskaja, M.\,T.~Tynaliev, N.\,N.~Popovkin, and A.\,R.~Timin}. 1990. 
Osobennosti khimicheskogo sostava i struktury mochevykh kamney i ikh rasprostranennost' v 
gorodakh Moskve, Berline i Kirgizskoy SSR [Peculiarities of chemical composition and
structure of urinary stones and their prevalence in the cities of Moscow,
Berlin, and the Kirghiz SSR]. \textit{Urologiya i Nefrologiya} [Urology and Nefrology] 5:49--54.
\bibitem{8-kri-1}
\Aue{Freitag, D., and K.~Hruska}. 1983. Pathophysiology of
nephrolithiasis. \textit{The kidney and body fluids in health
and desease}. Ed. S.~Klahr. N.Y., L.: Plenum Medical Books.

%\columnbreak 


\bibitem{10-kri-1}
\Aue{Hettmansperger, T.\,P.} 2006. Multivariate location tests. \textit{Encyclopedia of 
statistical sciences}. 
N.Y.: John Wiley\,\&\,Sons. 5249--5252.



\bibitem{9-kri-1}
\Aue{Minimaa, A., and H.~Oja}. 2006. Multivariate median. \textit{Encyclopedia of 
statistical sciences}. 
N.Y.: John Wiley\,\&\,Sons. 5258--5266.

\vspace*{5pt}

\bibitem{11-kri-1}
\Aue{Marden, J.\,I.} 1999. Multivariate rank tests. \textit{Multivariate analysis, design of experiments and 
survey sampling}. Ed. S.~Ghosh. N.Y.: Marcel Dekker. 401--432.

\vspace*{5pt}
\bibitem{12-kri-1}
\Aue{Chistjakov, S.\,P.} 2010. O~novom mnogomernom statisticheskom 
kriterii odnorodnosti dvukh vyborok [On a new multidimensional statistical test of homogeneity of
two samples]. 
\textit{Trudy Karel'skogo Nauchnogo Centra RAN} 3:93--97.

\vspace*{5pt}
\bibitem{13-kri-1}
\Aue{Leman, Je.} 1979. \textit{Proverka statisticheskikh gipotez}
[Testing of statstical hypothesis]. Moscow: Nauka.   408~p.

\vspace*{5pt}
\bibitem{14-kri-1}
\Aue{Krivenko, M.\,P.} 2011. Zadachi vyborochnogo kon\-t\-ro\-lya pri dosmotre lits, bagazha i 
transporta [The tasks of sampling during the inspection of
persons, baggage, and vehicles]. 
\textit{Obozrenie Prikladnoy i Promyshlennoy Matematiki} 18(1):125--126.

\vspace*{5pt}


 
\bibitem{15-kri-1}
\Aue{Baringhaus, L., and C.~Franz}. 2004. On a new multivariate two-sample test. \textit{J.~Multivariate 
Anal.} 88:190--206.
\end{thebibliography}
} }

\end{multicols}

\hfill{\small\textit{Received February 5, 2013}}

\Contr

\noindent
\textbf{Krivenko Michail P.} (b.\ 1946)~--- 
Doctor of Science in technology, principal scientist, Institute of Informatics 
Problems, Russian Academy of Sciences, Moscow 119333, Russian Federation;  mkrivenko@ipiran.ru

\vspace*{3pt}


\noindent\textbf{Golovanov Sergey  A.} (b.\ 1950)~--- Doctor of Science in medicine, Head of 
Laboratory, Research Institute of Urology, Moscow 105425, Russian Federation;
sergeygol124@mail.ru

\vspace*{3pt}

\noindent
\textbf{Sivkov Andrey V.} (b.\ 1957)~--- Doctor of Science in medicine, Deputy director, 
Research Institute of Urology, Moscow 105425, Russian Federation;  uroinfo@yandex.ru



 \label{end\stat}
\renewcommand{\bibname}{\protect\rm Литература}