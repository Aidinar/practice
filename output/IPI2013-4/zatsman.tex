\def\stat{zatsman}

\def\tit{МЕТОД ИЗВЛЕЧЕНИЯ БИБЛИОГРАФИЧЕСКОЙ ИНФОРМАЦИИ ИЗ~ПОЛНОТЕКСТОВЫХ 
ОПИСАНИЙ ИЗОБРЕТЕНИЙ$^*$}

\def\titkol{Метод извлечения библиографической информации из~полнотекстовых 
описаний изобретений}

\def\autkol{И.\,М.~Зацман, В.\,А.~Хавансков, С.\,К.~Шубников}

\def\aut{И.\,М.~Зацман$^1$, В.\,А.~Хавансков$^2$, С.\,К.~Шубников$^3$}

\titel{\tit}{\aut}{\autkol}{\titkol}

{\renewcommand{\thefootnote}{\fnsymbol{footnote}}\footnotetext[1] 
{Работа выполнена при частичной поддержке РГНФ (грант №\,12-02-12019в).}}

\renewcommand{\thefootnote}{\arabic{footnote}}
\footnotetext[1]{Институт проблем информатики Российской академии наук, 
iz\_ipi@a170.ipi.ac.ru} 
\footnotetext[2]{Институт проблем информатики Российской академии наук, havanskov@a170.ipi.ac.ru} 
\footnotetext[3]{Институт проблем информатики Российской академии наук, sergeysh50@yandex.ru} 

 \vspace*{6pt}
    
      \Abst{Предложен метод извлечения библиографической информации из полных 
описаний патентов, которая необходима для исследования тематических взаимосвязей науки и 
технологий. Цель исследования заключается в разработке принципов создания отечественных 
информационных систем для вычисления индикаторов тематических взаимосвязей. Этот вид 
информационных систем является новым для российской научно-технической сферы. Их 
создание необходимо для мониторинга и оценивания программ научных исследований и 
принятия решений на всех этапах программной деятельности. Предлагаемый метод 
извлечения библиографической информации из текстов на естественном языке (ЕЯ) обладает рядом 
принципиальных отличий от имеющихся зарубежных и отечественных аналогов. Во-пер\-вых, 
этот метод учитывает тот факт, что в патентных документах библиографическая информация 
может размещаться внутри текста на ЕЯ. Во-вто\-рых, библиографическая 
информация является структурированным информационным объектом, который в общем 
случае является многоязычным.}
      
      \KW{взаимосвязи науки и технологий; методология определения индикаторов; 
информационные системы; архитектурные решения; библиографическая информация; 
патентные документы}

\DOI{10.14357/19922264130406}

\vskip 20pt plus 9pt minus 6pt

      \thispagestyle{headings}

      \begin{multicols}{2}

            \label{st\stat}

\section{Введение}
      
      На протяжении более 20~лет в ИПИ РАН развивается направление исследований, 
связанное с автоматической обработкой текстов на ЕЯ с целью 
извлечения информационных объектов. Основной задачей этих исследований является создание 
нового класса интеллектуальных систем, основанных на автоматической формализации 
      ЕЯ-текс\-тов с формированием структур знаний для решения 
      ло\-ги\-ко-ана\-ли\-ти\-че\-ских задач~[1-6].
      
      Исследование тематических взаимосвязей науки и технологий также базируется на 
автоматической обработке текстов патентных документов на ЕЯ с целью извлечения 
библиографической ин\-формации, но при этом обладает рядом отличительных черт. 
      Во-пер\-вых, библиографическая информа\-ция является структурированным 
информационным объектом, который состоит из нескольких полей и который может 
размещаться внутри неструктурированного текста. Во-вто\-рых, разные поля 
библиографической информации могут быть на разных языках, отличных от языка патентного 
документа на ЕЯ. Другими словами, извлекаемый библиографический информационный объект 
в общем случае является многоязычным.
      
      До описания предлагаемого метода извлечения библиографической информации из 
полных описаний патентов, учитывающего отмеченные особенности ее извлечения, 
рассмотрим кратко ответы на два вопроса, характеризующих актуальность проб\-ле\-ма\-ти\-ки 
статьи:\\[-16pt]
      \begin{enumerate}[(1)]
      \item Каким образом процесс исследования тематических взаимосвязей науки и 
технологий связан с извлечением библиографической информации из описаний изобретений?\\[-16pt]
      \item Как сведения о тематических взаимосвязях науки и технологий могут повлиять на 
процесс принятия решений о финансировании науки?\\[-16pt]
      \end{enumerate}
      
      Ответ на первый вопрос был получен еще в 1985~г.\, когда в процессе 
сопоставительного ана\-ли\-за статей по биологии и библиографических \mbox{ссылок}\linebreak на статьи в 
описаниях изобретений по биотехнологиям были экспериментально зафиксированы 
взаимосвязи между цитируемыми научными публикациями и развитием биотехнологий. 
В~процессе\linebreak анализа различались патентные ссылки, т.\,е.\ ссылки на ранее выданные патенты, 
и непатентные ссылки, среди которых выделялись ссылки на статьи из журналов. В~результате 
сопоставительного анализа было обработано более 6500 патентных и непатент-\linebreak\vspace*{-12pt}

\pagebreak

\noindent
ных ссылок, 
извлеченных из описаний 399~патентов~[7].

    
    Второй вопрос рассматривался в рамках широкомасштабного исследования взаимосвязей 
науки и технологий с использованием более 1~млн непатентных ссылок, 
извлеченных из десятилетнего массива описаний изобретений США и Европейского патентного 
ведомства (ЕПВ). В~результате проведенного исследования было экспериментально 
определено, что 75\% научных статей, цитируемых в этих массивах описаний изобретений по 
широкому спектру технологий, были подготовлены по результатам, полученным именно в 
некоммерческом секторе научной сферы. Это дало возможность по результатам исследования 
сделать вывод о сильной зависимости технологического развития от степени государственной 
поддержки науки в США и европейских странах. Авторы этого исследования приводят пример 
существенного увеличения государственного финансирования научных проектов в Японии, 
которое было мотивировано именно выявленной зависимостью технологического развития от 
степени государственной поддержки науки~[8].
    
    При исследовании тематических взаимосвязей науки и технологий один из самых сложных 
вопросов заключается в том, как зафиксировать факт передачи и использования в 
технологической сфере результатов научных исследований~[9]. Согласно результатам 
работы~[7], один из предложенных и опробованных на практике подходов заключается в том, 
что процессы передачи знаний от науки к технологиям отслеживаются с помощью научных 
публикаций, цитируемых экспертами в отчетах о патентном поиске и/или авторами 
изобретений в их описаниях. Если публикации, цитируемые экспертами, структурно выделены 
в патентных документах, то публикации, цитируемые авторами патентов, могут быть включены 
в полнотекстовые описания изобретений. Поэтому исследования тематических взаимосвязей 
науки и технологий, как правило, основаны на извлеченной из них библиографической 
информации.
    
    Основная цель статьи заключается в описании метода извлечения библиографической 
информации из полных описаний патентов. В~следующем разделе статьи кратко 
рассматриваются основные идеи методологии определения тематических взаимосвязей науки и 
технологий. Подробное описание методологии и реализующей ее технологии,\linebreak включающей 
извлечение библиографической информации, дано в работах~[10, 11]. Раздел~3 посвящен 
методу извлечения библиографической информа\-ции из полнотекстовых описаний изобретений, 
за которым следует заключение.

\section{Методология определения тематических взаимосвязей науки 
и~технологий}
      
    Проведенный в ИПИ РАН анализ зарубежного опыта показал, что вычисление индикаторов 
тематических взаимосвязей науки и технологий требует формирования, автоматизированной 
обработки\linebreak полнотекстовых описаний изобретений с извлечением библиографической 
информации и со\-по\-став\-ле\-ния этой информации как с названиями источников научных 
публикаций (журналов или материа\-лов конференций), так и с названиями самих статей, 
хранящихся в научных электронных биб\-лио\-те\-ках. Это сопоставление дает возможность 
определить тематику тех научных направлений, к которым относятся научные публикации, 
цитируемые в описаниях изобретений~[10--14].
    
    Разработанный в ИПИ РАН вариант методологии определения значений индикаторов 
взаимосвязей позволяет вычислять их значения как для целых областей знаний, так и для 
отдельных направлений исследований~[10]. На его основе в настоящее время в ИПИ РАН 
создается ана\-ли\-ти\-ко-ин\-фор\-ма\-ци\-он\-ная сис\-те\-ма, предназначенная для 
индикаторного оценивания взаимосвязей науки и одного из направлений технологического 
развития~--- информационных технологий (ИТ).
    
    В настоящее время существует несколько вариантов методологии оценивания 
взаимосвязей науки и технологий. Перечислим основные из них, отмечая примеры 
реализованных (реализуемых) на практике вариантов, для которых решены теоретические и 
методологические вопросы.
    \begin{enumerate}[1.]
  \item Методология оценивания в статике взаимосвязей отдельных групп технологий при 
фиксированной системе классификации областей знаний и составляющих их направлений 
научных исследований на заданном интервале времени. Этот вариант является одним из 
наиболее проработанных с теоретической и методологической точек зрения. Он был 
реализован, например, в РФФИ. С~его помощью вычислены индикаторы взаимосвязей 
областей знаний и критических технологий для того массива проектов РФФИ, руководители 
которых выявили и описали эти взаимосвязи в отчетных формах по своим проектам.
  \item Методология оценивания в статике взаимосвязей по широкому спектру групп 
технологий при фиксированной системе классификации обла-\linebreak\vspace*{-12pt}

\pagebreak

\noindent
стей знаний и составляющих их 
направлений научных исследований на заданном интервале времени (с использованием 
\textit{основного индекса} Международной патентной классификации~--- МПК). Этот подход 
хорошо проработан теоретически и опробован на практике. Он был реализован в рамках 
информационной системы, разработанной по заказу Еврокомиссии с целью анализа 
многолетнего массива европейских и американских запатентованных изобретений и 
оценивания взаимосвязей науки и технологий. Этот вариант используется в настоящее время в 
директоратах Еврокомиссии.
  \item Методология оценивания в статике взаимосвязей по всем группам технологий при 
фиксированной системе классификации областей знаний и составляющих их направлений 
научных исследований на заданном интервале времени (с использованием основного и 
дополнительных индексов МПК). Этот вариант хорошо проработан теоретически. 
Продемонстрирована его реализуемость в лабораторных условиях. Он был разработан в ИПИ 
РАН в рамках проекта по гранту РГНФ №\,06-02-04043а <<Методы мониторинга и оценки 
инновационного потенциала и результативности направлений научных исследований>>.
  \end{enumerate}
  
  Наибольший интерес при оценивании взаимосвязей науки и технологий представляют не 
отдельные технологии, а группы технологий по приоритетным направлениям 
на\-уч\-но-тех\-но\-ло\-ги\-че\-ско\-го развития. Тематика каждого научного направления, результаты которого 
используются при разработке технологий, задается в виде одной или нескольких рубрик 
выбранной системы классификации областей знаний. Тематика каждой анализируемой группы 
технологий задается в виде списка рубрик МПК. Наиболее часто используются списки рубрик 
МПК из номенклатуры, разработанной Фраунгоферовским институтом системотехники и 
инновационных исследований (Fraunhofer Gesellschaft-Institute fur Systemtechnik und 
Innovationsforschung~--- FhG-ISI)~\cite{19-zac}.
  
  В номенклатуре FhG-ISI группа ИТ описывается 
следующими тремя рубриками МПК\footnote{Все рубрики МПК образуют иерархическую структуру и 
делятся на 5~категорий: разделы (верхний уровень иерархии), классы, подклассы, группы и подгруппы (нижний 
уровень иерархии).}:
  \begin{enumerate}[(1)]
\item класс G06~--- <<Вычисление; счет>> (эта рубрика МПК включает оптические 
вычислительные устройства, обработку цифровых данных с помощью компьютеров, 
аналоговые и гибридные компьютеры);
\columnbreak

\item подкласс G11C~--- <<Запоминающие устройства статического типа>>;
\item подкласс G10L~--- <<Анализирование или синтезирование речи; распознавание 
речи>>.
  \end{enumerate}
  
  Как отмечалось в начале этого раздела, для определения индикаторов взаимосвязей сначала 
должен быть сформирован массив библиографических\linebreak описаний научных публикаций, 
цитируемых в описаниях изобретений. Именно здесь возникает не\-обходимость формирования 
структурированных биб\-лио\-графических данных на основе анализа слабоструктурированных 
описаний изобретений. Затем необходимо провести рубрицирование этих данных в рамках 
заданной классификации научных на\-прав\-ле\-ний. Далее необходимо выполнить автоматическое 
сопоставление частот появления рубрик МПК, проставленных в описаниях изобретений, и 
частот появления рубрик цитируемых в этих изобретениях научных публикаций (рубрик 
заданной классификации научных направлений).
  
  Значения частот рубрик научных документов одного направления исследований, 
цитируемых в изобретениях с рубриками МПК заданной (исследуемой) группы технологий, 
представляют собой значения индикаторов взаимосвязей этого на\-прав\-ле\-ния с рубриками МПК 
заданной группы технологий. Отметим, что значения индикаторов существенно зависят от 
выбранной классификации научных направлений.
  
  Предлагаемый вариант методологии оценивания индикаторов взаимосвязей научных 
исследований и технологий предполагает в качестве входных ресурсов использование 
информационных ресурсов Роспатента. В~упрощенном виде входные и выходные 
информационные ресурсы и их взаимодействие на разных этапах методологии пред\-став\-ле\-ны на 
рис.~1. Используемые патентные ресурсы публикуются Роспатентом на CD/DVD носителях и 
доступны на его сайте. Выходные информационные ресурсы представляют собой базу данных (БД)
структурированных библиографических ссылок на публикации в описаниях изобретений, 
каждая из которых привязана, с одной стороны, к индексу МПК, с другой стороны, к одной или 
нескольким рубрикам направлений научных исследований.
  
  В разработанной методологии можно выделить два основных этапа. Первый 
(подготовительный) этап включает формирование БД структурированных ссылок на 
публикации, включенных в описания изобретений. Он предполагает обработку\linebreak\vspace*{-12pt}

\pagebreak

\end{multicols}

      \begin{figure} %fig1
         \vspace*{1pt}
 \begin{center}
 \mbox{%
 \epsfxsize=127.565mm
 \epsfbox{zats-1.eps}
 }
 \end{center}
 \vspace*{-6pt}
      \Caption{Входные и выходные информационные ресурсы (описание 
пронумерованных информационных потоков дано в работе~\cite{15-zac})}
\vspace*{3pt}
      \end{figure}

\begin{multicols}{2}

\noindent
 больших 
объемов слабоструктурированных описаний изобретений, что трудно поддается полной 
автоматизации и поэтому требует привлечения операторов. Этот этап может повторяться 
итерационно в целях пополнения БД. Второй этап включает вычисление значений 
количественных индикаторов и получение на их основе экспертных оценок взаимосвязей 
научных исследований и технологий, т.\,е.\ на этом этапе предполагается привлечение 
экспертов.
  
  Задача поиска ссылок на публикации решается с помощью метода автоматического 
выделения фрагментов неструктурированного текста по заданным признакам с использованием 
шаблонов, разработанных для разных видов ссылок на публикации. Каждый шаблон строится 
на основе признаков, характерных для одного вида ссылок на публи\-кации.
{\looseness=1

}
      

      
  Шаблоны формируются на основе данных анализа тестового массива полнотекстовых 
описаний изобретений. Все ссылки на публикации в тексте описания изобретения, найденные с 
помощью шаб\-ло\-нов, выделяются цветом. Подобное выделение дает возможность оператору 
контролировать точность поиска ссылок на публикации. В~ходе контроля оператором точности 
и полноты поиска ссылок на публикации накапливается статистика по определению успешно 
выделенных фрагментов текста, которые действительно оказались ссылками цитирования. 
Такая статистика позволяет увеличить результативность автоматического выделения ссылок 
цитирования, что, в свою очередь, служит предпосылкой для повышения результатив\-ности 
процесса автоматизированного поиска библиографических ссылок в описаниях изобретений.

\begin{figure*}[b] %fig2
         \vspace*{9pt}
 \begin{center}
 \mbox{%
 \epsfxsize=114.043mm
 \epsfbox{zats-2.eps}
 }
 \end{center}
 \vspace*{-6pt}
\Caption{Схема формирования массива патентных документов}
      \end{figure*}
  
  Однако ключевым фактором, влияющим на результативность автоматического выделения 
ссылок цитирования, является используемый метод извлечения, структуризации и 
нормализации библиографической информации, описание которого и является основной целью 
этой статьи.

\section{Метод извлечения библиографической информации}
      
      Вычисление значений количественных индикаторов тематических взаимосвязей 
научных исследований и технологий базируется на предварительной обработке патентных 
документов, содержащих слабо структурированный или полностью неструктурированный 
текст. Этими документами являются полнотекстовые описания изобретений. Массив 
отобранных документов может достигать сотен тысяч полнотекстовых описаний изобретений.
      
      Особенности технологий формирования и использования такого массива наглядно 
иллюстрирует пример из работы~\cite{20-zac}, в которой описывается процесс обработки 
массива из 656\,695~патентов на изобретения, выданных Патентным ведомством США. 
Сначала из описаний изобретений были выделены 1\,147\,160~непатентных ссылок (т.\,е.\ 
ссылки на патенты были исключены). Затем из них для дальнейшей обработки были отобраны 
только те ссылки на журнальные статьи, для которых удалось идентифицировать название 
журнала и соотнести его с нормативным списком названий журналов, в котором каждому 
названию присвоена одна или несколько рубрик научных направлений исследований. Таким 
образом было отобрано 106\,636~ссылок, т.\,е.\ менее 10\% от непатентных ссылок. 
В~результате обработки этих ссылок были вычислены значения индикаторов тематических 
взаимосвязей групп технологий и научных направлений.
      
      От используемого метода извлечения, структуризации и нормализации 
       зависит доля библиографических данных, для которых удается 
идентифицировать название журнала и соотнести его с норматив\-ным списком. Только после 
такого соотнесения появляется возможность присвоить одну или несколько рубрик научных 
направлений исследований структурированным ссылкам на публикации, извлеченным из 
описаний изобретений.

\begin{figure*}[b] %fig3
         \vspace*{1pt}
 \begin{center}
 \mbox{%
 \epsfxsize=112.636mm
 \epsfbox{zats-3.eps}
 }
 \end{center}
 \vspace*{-6pt}
\Caption{Взаимодействие подсистемы подготовки данных и ИПС MIMOSA}
      \end{figure*}
      
      Схема формирования массива патентных документов, используемых для поиска и 
выделения структурированных ссылок, а также для проведения расчетов значений индикаторов, 
показана на рис.~2. Этот массив представляет собой список патентов на изобретения и их 
библиографических описаний, отобранных на основе признаков, описывающих некоторую 
группу технологий. В~качестве базового инструмента для формирования массива используется 
ИПС (ин\-фор\-ма\-ци\-он\-но-по\-иско\-вая система) MIMOSA, разработанная в Федеральном 
институте промышленной собственности (ФИПС).


      
      Для поиска документов в БД библиографических данных и рефератов патентов на 
изобретения с помощью ИПС MIMOSA существует специально разработанный язык запросов. 
Индексы МПК в БД библиографических данных содержатся в разных полях, что необходимо 
учитывать в запросах на поиск. Ниже приводится пример запроса:
      \begin{multline*}
\hspace*{-2pt}(\mathrm{ICA=G06^*}\ \, \mathrm{OR}\ \, \mathrm{ICAA=G06}^*\  \mathrm{OR}\ \,
\mathrm{ICAI=G06^*})\\  \mbox{AND}\ \ \mathrm{DP=2012^*}\,,
\end{multline*} 
      где знак <<$*$>> служит знакозаменителем. Данный запрос интерпретируется 
следующим образом: \textit{выбрать все документы, которые содержат код МПК G06 в полях 
<<Клас\-си\-фи\-ка\-ция\,--\,рас\-ши\-рен\-ный уровень>>, 
      <<Клас\-си\-фи\-ка\-ция\,--\,до\-пол\-ни\-тель\-ная информация>>, 
      <<Клас\-си\-фи\-ка\-ция\,--\,изобре\-та\-тель\-ская информация>> и 2012~год в поле 
<<Дата публикации>>.}
      
      Информационно-поисковая сис\-те\-ма MIMOSA состоит из четырех частей. Эти части представляют собой разные 
программы:
      \begin{enumerate}[(1)]
\item MIMOSA: это основная часть, предлагающая различные функции для работы с 
БД: по\-стро\-ение запросов, выполнение этих запросов, просмотр документов 
и~т.\,д.
\item MIMOBatch: эта программа позволяет выполнить пакет запросов к одной или 
нескольким БД за одну операцию; можно просматривать, выгружать или 
печатать найденные документы.
\item MIMOViewer: эта программа предназначена в основном для просмотра электронных 
документов, ранее выгруженных из БД с помощью программ MIMOSA или 
MIMOBatch; эти документы выгружаются в форме SGML-фай\-лов и изображений.
\item JKTools: эта программа запускается только под Windows~95. Она используется для 
работы с роботизированными библиотеками в локальном режиме.
      \end{enumerate}
      
      Для формирования массива патентных документов, далее используемых для поиска и 
выделения структурированных ссылок, была спроектирована подсистема подготовки данных, 
которая использует программу MIMOBatch. С~помощью параметров настройки этой 
программы можно заранее загружать файлы пакетов запросов, выполнять пакетный поиск 
документов и формировать структурированный текстовый файл библиографических описаний 
патентов, отвечающих выбранному критерию поиска (рис.~3).


      
      Трудности, возникающие в процессе извлечения ссылок на публикации, цитируемые в 
полнотекстовых описаниях изобретений, подробно описаны в работе~\cite{16-zac}. В~числе 
основных причин этих трудностей отметим:\\[-14pt]
      \begin{enumerate}[1.]
\item Отсутствие в <<Административном регламенте\ldots>>~\cite{21-zac} требований к 
структурированию ссылок на цитируемые публикации (см.\ п.~(12) раздела регламента 
10.11~\textbf{Требования к оформлению заявки}).\\[-14pt]
\item Отсутствие в опубликованных электронных версиях полнотекстовых описаний 
изобретений групп меток, выделяющих ссылки на цитируемые публикации согласно 
рекомендациям стандарта ВОИС ST.14~\cite{22-zac}.\\[-14pt]
\item Отсутствие списка нормализованных и сокращенных названий журналов, 
используемых в ссылках на цитируемые публикации.
      \end{enumerate}
      
             \begin{table*}\small
       \begin{center}
      \begin{tabular}{|p{150mm}|}
      \multicolumn{1}{c}{ Примеры ссылок на публикации в текстах описаний 
изобретений}\\[6pt]
\hline
1.  Белоцерковский Г.\,Б. Основы радиолокации и радиолокационные устройства, М.: 
Сов. радио, 1975~г.\\
\hline
2. Г.~Корн, Т.~Корн. Электронные аналоговые и аналого-цифровые вычислительные 
машины. 1~Теория и основные функциональные блоки. изд-во Москва.: Мир, 1967.\\
\hline
3. Б.\,М. Каган и В.\,В. Сташин <<Основы проектирования микропроцессорных устройств 
автоматики, стр.11, М.: Энергоатомиздат, 1987.\\
\hline
4. Шахнович И. Век нынешний и век грядущий.~// Электроника: Наука, технология, 
бизнес, N~6,: 1999, стр.8--11.\\
\hline
5. RODECKU \textit{et al.}, Tumo grouth modulation by monoclonal antibogy to the epidermal 
grow factor receptot immunologically mediated and effector cellindependent effects., Cancer 
Research, 1987, v.47, №14, pp.3692--3696.\\
\hline
6. Высоцкого Б.\,Ф. Цифровые фильтры и устройства обработки сигналов на 
интегральных микросхемах.~--- М.: Радио и связь. 1984~г., стр.46, 54, 55.\\
\hline
7. А.\,В. Рыжкова, В.\,Н. Попова <<Синтезаторы частот в технике радиосвязи>> (М.: 
Радио и связь, 1991, с.137--142, рис.5.3--5.6)\\
\hline
8.  Holden, Science 291: 967(2001).\\
\hline
9.  <<Радио>>  №9, 2004,  с.47.\\
      \hline
      \end{tabular}
      \end{center}
      \vspace*{-6pt}
      \end{table*}
      
      Таким образом, при исследовании тематических взаимосвязей технологий и направлений 
научных исследований возникает задача анализа десятков\linebreak и сотен тысяч полнотекстовых 
описаний изобрете\-ний и поиска в тексте на ЕЯ ссылок на публика\-ции с\linebreak последующей их 
структуризацией и \mbox{привязкой}\linebreak ссылок к направлениям научных исследований. Следовательно, 
необходима автоматизация данного процесса. При этом важно учитывать, что 
биб\-лио\-гра\-фи\-че\-ская информация является структурированным информационным объектом, 
который состоит из нескольких полей и может размещаться внутри неструктурированного 
текста, а разные поля библиографической информации могут быть в общем случае на разных 
языках.
      
      Рассмотрим примеры структур объектов поиска в тексте на ЕЯ, т.\,е.\ ссылок на 
публикации, которые собраны в таблицу. Приведенные примеры ссылок на публикации были 
выделены из реального массива, который был получен в результате обработки порядка 
1300~полнотекстовых описаний изобретений.
{\looseness=-1

}
       

      
      Из представленной таблицы видно, что в обобщенном виде структура ссылок на 
публикации может быть представлена следующей схемой:

\noindent
      \begin{multline*}
       [\mbox{\textit{автор}}\ \{S_1\}]  [\mbox{\textit{название\ публикации}}] \\
       [\{S_2\}\ \mbox{\textit{название\ источника}}]\\
        \{S_3\}\ \mbox{\textit{атрибуты\ публикации}}
\end{multline*}
 %
      Наличие квадратных скобок говорит о необязательности присутствия данного элемента 
схемы в реальной ссылке. $\{S_i\}$ обозначает множество возможных 
      зна\-ков-раз\-де\-ли\-те\-лей, которые стоят между элементами этой схемы. В~отличие от 
элементов $[\mbox{\textit{название\ публикации}}]$ и $[\{S_2\}\ \mbox{\textit{название\ источника}}]$, 
которые могут рассматриваться 
как простые множества слов и знаков пунктуации, элементы 
$[\mbox{\textit{автор}}\ \{S_1\}]$ и $\mbox{\textit{атрибуты\ 
публикации}}$ имеют собственные структурные особенности.
      
      Элемент $[\mbox{\textit{автор}}\ \{S_1\}]$ структурно может иметь следующие 
      варианты представления:
      \begin{itemize}
\item[(а)] {\textit{Фамилия И.[О.]}}~--- см. ссылки на публикации~1 и~4 из таблицы;
\item[(б)] {\textit{И.[О.] Фамилия}}~--- см.\ ссылки на публикации 2, 3 и~7;
\item[(в)] {\textit{[Имя] Фамилия}} или 
{\textit{Фамилия [Имя]}}~--- см. ссылку на публикацию~5.
\end{itemize}
      
      Причем {\textit{Фамилия}} во всех вариантах представления и {\textit{Имя}} могут быть 
набраны как прописными, так и строчными буквами, за исключением первых букв, которые, 
как правило, всегда являются прописными. Кроме того, в ссылке этот элемент может 
повторяться в том варианте представления, который принят автором изобретения. В~качестве 
разделителя, как правило, используется запятая.
      
      Элемент $\{S_3\}\ \mbox{\textit{атрибуты\ публикации}}$ 
      имеет произвольную структуру, которая может 
включать данные об издательстве, томе, номере, страницах и годе пуб\-ли\-ка\-ции. Причем 
произвольность структуры выражается в том, что порядок следования, наличие и вид 
представления этих данных могут быть любыми. Кроме того, сам данный элемент может быть 
заключен в скобки (см.\ ссылки на пуб\-ли\-ка\-ции~7 и~8 из таблицы). Единственные данные, 
которые, как правило, всегда присутствуют в элементе 
$\{S3\}\mbox{\textit{\ атрибуты\ пуб\-ли\-ка\-ции}}$,~--- 
это данные о  годе публикации.
      
      Описанная структура ссылки на публикации использовалась в процессе разработки 
метода извлечения, структуризации и нормализации библиографической информации 
полнотекстовых описаний изобретений.
      
      Разработанный метод основан на процессе по\-сле\-до\-вательного перебора символов в 
линейной после\-довательности текста на ЕЯ, в которую предварительно преобразуется текст 
описания изобретения. Лингвистические методы, основанные на автоматической формализации 
текстов ЕЯ с формированием структур знаний~[1--6], не 
используются.
{\looseness=-1

}

\begin{figure*} %fig4
         \vspace*{1pt}
 \begin{center}
 \mbox{%
 \epsfxsize=112.095mm
 \epsfbox{zats-4.eps}
 }
 \end{center}
 \vspace*{-6pt}
\Caption{Фрагмент текста описания изобретения с частично выделенной ссылкой на публикацию}
      \end{figure*}
      
      В процессе перебора выбирается такая последовательность символов, которая 
удовлетворяет описанию первого элемента структуры ссылки. Далее в случае обнаружения 
этой последовательности символов ищется такая последовательность символов, которая 
соответствует описанию последнего элемента структуры. Затем выделяется фрагмент текста, 
имеющий признаки ссылки на публикацию.
      
      Анализ точности разработанного метода поиска и выделения ссылок дал в целом 
удовлетворительный результат. Из общего числа ссылок на публикацию в обработанном 
тестовом массиве описаний изобретений программа выделила около 88\% ссылок. Еще 9\% 
ссылок было выделено не полностью, как, например, во фрагменте текста описания 
изобретения на рис.~4.
      
       

      
      Как видно из приведенного фрагмента текста, алгоритм не смог правильно 
интерпретировать союз `\textit{et}' в перечислении авторов в элементе 
$[\mbox{\textit{автор}}\ \{S_1\}]$ 
описанной выше структуры ссылки на публикацию. Поэтому к знаку разделения 
пе\-ре\-чис\-ля\-емых авторов (запятой) были добавлены буквенные разделители `\textit{et}' и 
`\textit{и}' (см.\ ссылку~3 из таблицы). Кроме того, к множеству $\{S_1\}$, содержащему, как 
правило, точку и пробел, добавлено сочетание знаков `\textit{et al.}' (`и~др.'\
в русском  варианте) (см.\ ссылку на публикацию~5) для более точного выделения элемента 
$[\mbox{\textit{автор}}\  \{S_1\}]$.
      
      Из анализа приведенного примера следует, что для повышения результативности 
автоматического выделения ссылок цитирования требуется преду\-смот\-реть возможность 
внесения изменений по мере накопления и анализа новых вариантов представления ссылок на 
публикации, используемых авторами в описаниях изобретений патентов. Для реализации этого 
подхода предлагается использовать библиотеку шаблонов, описывающих различные варианты 
представления ссылок, при сохранении единого алгоритма их поиска и выделения в описаниях 
изобретений патентов.
      
      Для описания шаблонов поиска требуемой последовательности символов при разборе 
текста описа\-ния изобретений используются регулярные выражения, которые широко 
применяются в различных программных средах (см., например,~\cite{23-zac}). Приведем три 
примера представления элемента $[\mbox{\textit{автор}}\ \{S_1\}]$:
      \begin{itemize}
\item[(а)] ${\sf \backslash p\{Lu\}+(\backslash p\{Ll\}+)?[\ ](\backslash 
p\{Lu\}\backslash p\{Ll\}?[.] [\ ]?)?}$

${\sf ( \backslash p\{Lu\}[.])[,]?}$;
\item[(б)] ${\sf \backslash p\{Lu\}\backslash p\{Ll\}?[.][\ ]?( \backslash 
p\{Lu\}[.])?[\ ]? \backslash p\{Lu\}+{}}$

${\sf {}+\backslash p\{Ll\}+([.,\ ])+{}}$;
\item[(в)] ${\sf (\backslash p\{Lu\}\backslash p\{Ll\}+)[\ ] \backslash 
p\{Lu\}+\backslash p\{Ll\}+[.,][\ ]}$.
\end{itemize}
      
      Аналогичным образом можно описать варианты представления элемента $\{S_3\}\ 
\mbox{\textit{ат-}}$\linebreak $\mbox{\textit{рибуты\ публикации}}$. Все сочетания используемых описаний обоих элементов структуры 
ссылки на публикацию образуют библиотеку шаб\-ло\-нов $\{R\}$ для поиска ссылок на 
публикации в описаниях изобретений патентов.
      
      Таким образом, в процессе поиска ссылок в описании изобретения к его тексту 
применяется библиотека шаблонов $\{R\}$. Использование регулярных выражений 
обеспечивает возможность получения непересекающихся фрагментов текста, содержащих 
признаки ссылки на публикацию, для отдельно взятого шаблона. Но в то же время разные 
шаблоны коллекции могут формировать пересекающиеся фрагменты текста. Например, 
шаблоны~(а) и~(б) (см.\ выше) элемента $[\mbox{\textit{автор}}\ \{S_1\}]$ по-раз\-но\-му 
выделят в одном и том же 
фрагменте текста ссылку на публикацию:
      \begin{itemize}
      \item[(a)] \textit{действительному техническому состоянию объекта 
[\underline{Кузнецов~В.\,Е., Лихачев~А.\,М., Пара-}\linebreak \underline{щук~И.\,Б., 
Присяжнюк~С.\,П}. Телекоммуникации. 
Толковый словарь основных терминов и сокращений. Под редакцией А.\,М.~Лихачева, 
С.\,П.~Присяжнюка.~--- СПб.: Издательство МО РФ, 2001.]}.
\item [(б)] \textit{действительному техническому состоянию объекта [Кузнецов~В.\,Е., 
Лихачев~А.\,М., Паращук~И.\,Б., Присяжнюк~\underline{С.\,П. Телекоммуникации}. Толковый 
словарь основных терминов и сокращений. Под редакцией А.\,М.~Лихачева, С.\,П.~Присяжнюка.~--- 
СПб.: Издательство МО РФ, 2001.}].
\end{itemize}

      После применения к тексту библиотечных шаб\-ло\-нов выполняется процедуру интеграции 
выделенных каждым шаблоном фрагментов текста (рис.~5).
       
      \begin{figure*} %fig5
               \vspace*{1pt}
 \begin{center}
 \mbox{%
 \epsfxsize=119.949mm
 \epsfbox{zats-5.eps}
 }
 \end{center}
 \vspace*{-6pt}
\Caption{Интеграция выделенных каждым шаблоном фрагментов текста}
      \end{figure*}
      
      На этом рисунке отрезок~$TT_1$ обозначает фрагмент текста описания изобретения; 
$r_1, r_2, \ldots ,r_n$~--- шаблоны библиотеки $\{R\}$; $s_{i , j}$~--- позиция начала найденного 
фрагмента текста в соответствии с шаб\-ло\-ном~$r_i$; $e_{i , j}$~--- позиция окончания найденного 
фрагмента текста в соответствии с шаб\-ло\-ном~$r_i$; $f_m$~--- выделенный фрагмент текста, 
имеющий признаки ссылки на публикацию и являющийся элементом множества $\{F\}$ ссылок 
на публикации, которые найдены в обрабатываемом фрагменте текста описания изобретения.
      
      Как видно из рис.~5, при нахождении выделенных разными шаблонами фрагментов 
текста применяется следующее правило: из нескольких значений начала выделенных 
фрагментов $s_{i , j}$ выбирается минимальное, а для значений окончания выделенных 
фрагментов $e_{i , j}$~--- максимальное.
      
      Функциональная схема подсистемы, обеспечивающей реализацию описанного метода 
поиска и структуризации ссылок на публикации, пред\-став\-ле\-на на рис.~6. После формирования 
массива данных для проведения расчетов (см.\ рис.~2) на вход данной подсистемы поступает 
список отобранных номеров патентов. По каждому номеру производится запрос в открытый 
реестр описаний изобретений Роспатента РФ. По запросу в подсистему возвращается 
полнотекстовое описание изобретения в формате HTML. Далее текст описания программно 
обрабатывается в соответствии с методом, который описан выше (см.\ рис.~5).
      

      
      После завершения процесса поиска и выделения ссылок на публикации программа 
анализирует число ссылок в данном описании изобретения. При нулевом значении описание 
изобретения передается на рассмотрение оператору.
      
      В случае обнаружения оператором пропущенных ссылок он формирует множество 
$\{F\}$ для данного описания изобретения. Кроме того, тексты данных ссылок передаются как 
образцы на технологическую операцию ведения библиотеки шаб\-ло\-нов. Там эти образцы ссылок 
на публикации анализируются и при необходимости либо корректируются существующие 
описания шаблонов, либо формируется новый шаблон для поиска, который добавляется в 
библиотеку~$\{R\}$.
      
      В методологии поиска ссылок цитирования пуб\-ли\-ка\-ций и их привязки к направлениям 
научных исследований следующим шагом является структуризация выделенных ссылок и их 
рубрицирования по областям знаний и научным дисциплинам. Для того чтобы определить 
научную область знаний статьи, которая цитируется, используется подход, примененный в 
работе~\cite{20-zac}. В~этом подходе учитывается то, что научные журналы имеют 
тематическую направленность и соотнесены с некоторой областью знаний или научной 
дисциплиной, которая может быть выражена одной или со\-во\-куп\-ностью рубрик некоторого 
классификатора научных направлений исследований, например классификатора ГРНТИ или 
РФФИ. Таким образом, как правило, можно сделать вывод, что статья, опубликованная в 
журнале, имеет ту же рубрику классификатора, которая указана в его библиографических 
данных.
      
      Отсюда следует, что процедура рубрицирования ссылки цитируемой публикации в 
случае журнальной публикации или публикации в сборнике трудов конференции сводится к 
соотнесению выделенной ссылки цитируемой публикации с названием научного журнала или 
сборника, указанного в ссылке на публикацию, из нормативного списка журналов.
      
      Это означает, что необходимо выполнить структуризацию выделенной ссылки с целью 
поиска названия журнала с последующим извлечением из нормативного списка журналов его 
рубрик по классификатору научных направлений исследований. Иначе говоря, задачей этого 
этапа методологии является нахождение в ссылке на публикацию названия журнала и года 
публикации. Первый параметр будет использоваться для рубрицирования ссылки, а второй 
важен при построении ряда индикаторов тематических взаимосвязей направлений научных 
исследований и технологий.
      
      При технологическом решении данной задачи используется подход, основанный на 
описанной\linebreak\vspace*{-12pt}

\pagebreak

\end{multicols}

\begin{figure} %fig6
         \vspace*{1pt}
 \begin{center}
 \mbox{%
 \epsfxsize=144.222mm
 \epsfbox{zats-6.eps}
 }
 \end{center}
 \vspace*{-6pt}
\Caption{Функциональная схема поиска ссылок на публикации}
\vspace*{6pt}
      \end{figure}

\begin{multicols}{2}

\noindent
 в предыдущем разделе схеме структур ссылок на публикации и использовании 
описания элементов структур и их разделителей с помощью регулярных выражений. При этом 
для эксперта обеспечивается возможность просмотра, анализа и редактирования результатов 
рубрицирования ссылок на публикации.
      
      Таким образом, в результате накопления и обработки полнотекстовых описаний 
патентов на изобретения получается массив записей, в каждой из которых указаны: номер 
патента, индексы МПК, год публикации патента, ссылки на выявленные публикации с 
указанием рубрик(и) научных на\-прав\-ле\-ний исследований и года публикации \mbox{статьи}. Данный 
массив может быть использован для вычисления и построения различных индикаторов, 
описывающих тематические взаимосвязи научных исследований и технологий.

\section{Заключение}
      
  В работе~\cite{14-zac} была описана методология вычисления значений количественных 
индикаторов тематических взаимосвязей науки и технологий как для целых областей знаний, 
так и для отдельных направлений исследований в целях идентификации технологически 
ориентированных научных направлений. Она обладает рядом принципиальных отличий от 
зарубежных аналогов:
  \begin{itemize}
\item выбор именно тех систем классификации областей знаний, которые используются в 
процессе принятия решений;
\item использование научных публикаций, ци\-ти\-ру\-емых экспертами в отчетах о поиске и 
авторами в описаниях изобретений, для определения значений индикаторов взаимосвязей 
науки и технологий;
\item использование ключевых слов из названий пуб\-ли\-ка\-ций для уточнения рубрики 
публикации в тех случаях, когда одному источнику пуб\-ли\-ка\-ции приписано несколько 
рубрик системы классификации областей знаний.
  \end{itemize}
  
  Использование ключевых слов из названий публикаций для уточнения рубрики публикации 
предполагает, что предварительно уже выполнено базовое рубрицирование этих публикаций. 
Разработанный метод извлечения библиографической информации позволяет решить задачу 
базового руб\-ри\-ци\-ро\-ва\-ния в рамках технологии автоматизированной обработки полнотекстовых 
описаний изобретений и сопоставления извлеченной информации с названиями источников 
научных публикаций. Это сопоставление дает возможность определить тематику тех научных 
направлений, к которым относятся научные публикации, цитируемые в описаниях изобретений.
  
  Разрабатываемая в ИПИ РАН информационная система индикаторного оценивания 
тематических взаимосвязей науки и технологий не имеет аналогов в российской 
  на\-уч\-но-тех\-ни\-че\-ской сфере. Ее создание необходимо для проведения мониторинга~[20--23], 
многоаспектного оценивания программ научных исследований и прогнозирования 
  на\-уч\-но-тех\-но\-ло\-ги\-че\-ско\-го развития страны.
     
{\small\frenchspacing
{%\baselineskip=10.8pt
\addcontentsline{toc}{section}{Литература}
\begin{thebibliography}{99}

\bibitem{5-zac} %1
\Au{Kuznetsov I., Kozerenko E.} The system for extracting semantic information from natural 
language texts~// Conference (International) on Machine Learning (MLMTA-03) Proceedings.~--- 
Las Vegas, 2003. P.~75--80.
\bibitem{6-zac} %2
\Au{Кузнецов И.\,П.} Семантико-ориентированная система обработки неформализованной 
информации с выдачей результатов на естественном языке~// Сис\-те\-мы и средства 
информатики, 2006. Вып.~16. С.~235--253.
\bibitem{7-zac} %3
\Au{Кузнецов И.\,П., Мацкевич А.\,Г.} Се\-ман\-ти\-ко-ориен\-ти\-ро\-ван\-ные сис\-те\-мы на 
основе баз знаний.~--- М.: \mbox{МТУСИ}, 2007. 173~с.
\bibitem{8-zac} %4
\Au{Кузнецов И.\,П.} Объект\-но-ори\-ен\-ти\-ро\-ван\-ная сис\-те\-ма, основанная на знаниях 
в виде XML-пред\-став\-ле\-ний~// Сис\-те\-мы и средства информатики.~--- М.: Наука, 2008. 
Вып.~18. С.~96--118.
\bibitem{9-zac} %5
\Au{Kuznetsov I.\,P., Kozerenko E.\,B.} Linguistic processor Semantix for knowledge extraction 
from natural texts in Russian and English~// Conference (International) on Artificial Intelligence 
(ICAI 2008) Proceedings.~--- Las Vegas: CSREA Press, 2008. P.~835--841.
\bibitem{10-zac} %6
\Au{Кузнецов И.\,П., Сомин Н.\,В.} Выявление имплицитной информации из текстов на 
естественном языке: проблемы и методы~// Информатика и её применения, 2012. Т.~6. 
Вып.~1. С.~49--58.
\bibitem{11-zac} %7
\Au{Narin F., Noma E.} Is technology becoming science?~// Scientometrics, 1985. Vol.~7. 
No.\,3-6. P.~369--381.
\bibitem{12-zac} %8
\Au{Narin F., Olivastro D.} Linkage between patents and papers: An interim EPO/US 
comparison~// Scientometrics, 1998. Vol.~41. No.\,1-2. P.~51--59.
\bibitem{13-zac} %9
\Au{Schmoch U.} Tracing the knowledge transfer from science to technology as reflected in patent 
indicators~// Scientometrics, 1993. Vol.~26. P.~193--211.
\bibitem{14-zac} %10
\Au{Минин В.\,А., Зацман И.\,М., Кружков~М.\,Г., Норекян~Т.\,П.} Методологические основы 
создания информационных сис\-тем для вы\-чис\-ле\-ния индикаторов тематических 
взаимосвязей науки и технологий~// Информа\-тика и её применения, 2013. Т.~7. Вып.~1. 
С.~70--81.
\bibitem{15-zac} %11
\Au{Минин В.\,А., Зацман И.\,М., Хавансков~В.\,А., Шубников~С.\,К.} Архитектурные 
решения для систем вы\-чис\-ле\-ния индикаторов тематических взаимосвязей науки и 
технологий~// Сис\-те\-мы и средства информатики, 2013. Т.~23. №\,2. С.~260--283.
\bibitem{16-zac} %12
\Au{Зацман И.\,М., Шубников С.\,К.} Принципы обработки информационных ресурсов для 
оценки инновационного потенциала направлений научных исследований~// Электронные 
библиотеки: перспективные методы и технологии, электронные коллекции: Труды \mbox{9-й} 
Всеросс. научн. конф. RCDL'2007.~---  Переславль: Университет города Переславля, 2007. 
С.~35--44.
\bibitem{17-zac} %13
\Au{Зацман И.\,М., Курчавова О.\,А., Галина~И.\,В.} Информационные ресурсы и индикаторы 
для оценки инновационного потенциала направлений научных исследований~// Сис\-те\-мы 
и средства информатики, 2008. Вып.~18 (доп.). С.~159--175.
\bibitem{18-zac} %14
\Au{Кожунова О.\,С.} Цитирование документов в патентах как индикатор взаимосвязи 
областей науки и технологий~// Сис\-те\-мы и средства информатики, 2012. Т.~22. №\,2. 
С.~106--128.
\bibitem{19-zac} %15
\Au{Van Looy B., Zimmermann~E., Veugelers~R., Verbeek~A., Mello~J., Debackere~K.} Do 
science--technology interactions pay on when developing technology? An exploratory investigation 
of 10~science-intensive technology domains~// Scientometrics, 2003. Vol.~57. No.\,3. 
P.~355--367.
\bibitem{20-zac} %16
\Au{Verbeek А., Debackere~K., Luwel~M., Andries~P., Zimmermann~E., Deleus~D.} Linking 
science to technology: Using bibliographic references in patents to build linkage schemes~// 
Scientometrics, 2002. Vol.~54. No.\,3. P.~399--420.
\bibitem{21-zac} %17
Административный регламент исполнения Роспатентом приема заявок на изобретение, их 
рас\-смот\-ре\-ния и экспертизы.~--- ФИПС, 2008. {\sf 
http://www1. fips.ru/wps/wcm/connect/content\_ru/ru/documents/ 
russian\_laws/order\_minobr/administrative\_regulations/ test\_8/}.
\bibitem{22-zac} %18
Рекомендации по включению ссылок, цитируемых в патентных документах: Стандарт \mbox{ВОИС} 
ST.14. {\sf http://www.rupto.ru/rupto/nfile/52b8dfc1-1049-11e1-a520-9c8e9921fb2c/03\_14\_01.pdf}.


\bibitem{23-zac} %19
Регулярные выражения в .NET Framework. 
{\sf http:// msdn.microsoft.com/ru-ru/library/hs600312.aspx}.

\bibitem{1-zac}
\Au{Зацман И.\,М., Веревкин Г.\,Ф.} Информационный мониторинг сферы науки в задачах 
про\-грам\-мно-це\-ле\-во\-го управ\-ле\-ния~// Сис\-те\-мы и средства информатики, 2006. Вып.~16.  С.~164--189.
\columnbreak


\bibitem{2-zac}
\Au{Зацман И.\,М. Веревкин Г.\,Ф., Дрынова~И.\,В., Курчавова~О.\,А., Ларин~Н.\,В., 
Норекян~Т.\,П.} Моделирование систем информационного мониторинга как проблема 
информатики~// Сис\-те\-мы и средства информатики. Спец. вып. 
На\-уч\-но-ме\-то\-до\-ло\-ги\-че\-ские проб\-ле\-мы информатики, 2006. 
С.~112--139.
\bibitem{3-zac}
\Au{Зацман И.\,М., Кожунова О.\,С.} Семантический словарь системы информационного 
мониторинга в сфере науки: задачи и функции~// Сис\-те\-мы и средства информатики,
2007. Вып.~17. С.~124--141.
\bibitem{4-zac}
\Au{Zatsman I., Kozhunova O.} Evaluation system for the Russian Academy of Sciences: 
Objectives-resources-results approach and R\&D indicators~// 2009 Atlanta Conference on Science 
and Innovation Policy Proceedings~/ Eds. S.\,E.~Cozzens, P.~Catalаn. {\sf 
http://smartech. gatech.edu/bitstream/1853/32300/1/104-674-1-PB.\linebreak pdf}.

\end{thebibliography} } }



\end{multicols}

\hfill{\small\textit{Поступила в редакцию 07.10.13}}


\vspace*{12pt}

\hrule

\vspace*{2pt}

\hrule

\def\tit{METHOD OF BIBLIOGRAPHIC INFORMATION EXTRACTION FROM~FULL-TEXT DESCRIPTIONS 
     OF~INVENTIONS}

\def\aut{I.\,M.~Zatsman, V.\,A.~Havanskov, and~S.\,K.~Shubnikov}

\def\titkol{Method of bibliographic information extraction from~full-text descriptions 
     of~inventions}

\def\autkol{I.\,M.~Zatsman, V.\,A.~Havanskov, and~S.\,K.~Shubnikov}


\titel{\tit}{\aut}{\autkol}{\titkol}

\vspace*{-12pt}

\noindent
Institute of Informatics 
Problems, Russian Academy of Sciences, Moscow 119333, Russian Federation

%\vspace*{6pt}
  
\def\leftfootline{\small{\textbf{\thepage}
\hfill INFORMATIKA I EE PRIMENENIYA~--- INFORMATICS AND APPLICATIONS\ \ \ 2013\ \ \ volume~7\ \ \ issue\ 4}
}%
 \def\rightfootline{\small{INFORMATIKA I EE PRIMENENIYA~--- INFORMATICS AND APPLICATIONS\ \ \ 2013\ \ \ volume~7\ \ \ issue\ 4
\hfill \textbf{\thepage}}}  

\vspace*{12pt}
    
      \Abste{The method of bibliographic information extraction from full-text descriptions of 
inventions, which is necessary for analysis of thematic linkages between science and technologies, is 
considered. The research objective consists in the development of principles for creation of domestic 
information systems for calculation of indicators of thematic linkages. This type of information 
systems is new to the Russian scientific and technical sphere. Their creation is necessary for 
monitoring and evaluation of research and development programs and decision-making at all stages of program activities. 
The suggested method of bibliographic information extraction from the texts in 
natural language 
differs a lot from available foreign and domestic analogs. First, this method considers 
the fact that bibliographic information can be found inside the natural language text of descriptions of 
inventions. Second, paper bibliographic information is the structured information object, which 
is generally multilingual.}
      
      \KWE{linkages between science and technologies; methodology of indicator calculation; 
information systems; architectural decisions; bibliographic information; patent documents}
      
      
\DOI{10.14357/19922264130406}

\vspace*{-18pt}

\Ack
\noindent
The research was partially supported by the Russian Foundation
for Humanities (grant No.\,12-02-12019v).

  \begin{multicols}{2}

\renewcommand{\bibname}{\protect\rmfamily References}
%\renewcommand{\bibname}{\large\protect\rm References}

{\small\frenchspacing
{%\baselineskip=10.8pt
\addcontentsline{toc}{section}{References}
\begin{thebibliography}{99}


      \bibitem{5-zac-1}
      \Au{Kuznetsov, I., and E.~Kozerenko}. 2003. The system for extracting semantic information 
from natural language texts.  \textit{Conference (International) on Machine Learning (MLMTA-03) 
Proceedings}. Las Vegas. 75--80.
      \bibitem{6-zac-1}
      \Au{Kuznetsov, I.\,P.} 2006. Semantiko-orientirovannaya sistema obrabotki neformalizovannoy 
informatsii s vydachey rezul'tatov na estestvennom yazyke [Semantic-oriented system for processing of 
nonformalized information with outcomes in a natural language]. \textit{Sistemy i
Sredstva Informatiki~---
Systems and Means of  Informatics}  16:235--253.
\bibitem{7-zac-1}
\Aue{Kuznetsov, I.\,P., and A.\,G.~Matskevich}. 2007. \textit{Semantiko-orientirovannye sistemy na 
osnove baz znaniy} [\textit{Semantic-oriented systems based on knowledge bases}]. Moscow: MTUSI. 
173~p.

%\pagebreak

      \bibitem{8-zac-1}
      \Aue{Kuznetsov, I.\,P.} 2008. Ob''ektno-orientirovannaya sistema, osnovannaya na znaniyakh v 
vide XML-predstavleniy [The object-oriented system based on knowledge in the form of XML 
representations]. \textit{Sistemy i
Sredstva Informatiki~---
Systems and Means of Informatics} 18:96--118.
      \bibitem{9-zac-1} %5
      \Aue{Kuznetsov, I.\,P., and E.\,B.~Kozerenko}. 2008. Linguistic processor Semantix for 
knowledge extraction from natural texts in Russian and English. \textit{Conference (International) on 
Artificial Intelligence (ICAI 2008) Proceedings}. Las Vegas: CSREA Press. 835--841.
      \bibitem{10-zac-1} %6
      \Aue{Kuznetsov, I.\,P., and N.\,V.~Somin}. 2012. Vyyavlenie implitsitnoy informatsii iz tekstov 
na estestvennom yazyke: Problemy i metody [Retrieval of implicit information from natural language 
texts: Problems and methods]. \textit{Informatika i ee Primenenya~---
Inform. Appl.}  6(1):49--58.
      \bibitem{11-zac-1} %7
      \Aue{Narin, F., and E.~Noma}. 1985. Is technology becoming science? \textit{Scientometrics}  
7(3-6):369--381.
      \bibitem{12-zac-1} %8
      \Aue{Narin, F., and D.~Olivastro}. 1998. Linkage between patents and papers: An interim 
EPO/US comparison. \textit{Scientometrics}  41(1-2):51--59.
      \bibitem{13-zac-1} %9
      \Aue{Schmoch, U.} 1993. Tracing the knowledge transfer from science to technology as 
reflected in patent indicators. \textit{Scientometrics} 26:193--211.
      \bibitem{14-zac-1} %10
      \Aue{Minin, V.\,A., I.\,M.~Zatsman, M.\,G.~Kruzhkov, and T.\,P.~Norekjan}. 2013.  
Metodologicheskie osnovy so\-zda\-niya informatsionnykh sistem dlya vychisleniya indikatorov 
tematicheskikh vzaimosvyazey nauki i tekhnologiy [Methodological basis for the creation of information 
systems for the calculation of indicators of thematic linkages between science and technology]. 
\textit{Informatika i ee Primeneniya~--- Inform. Appl.} 7(1):70--81.
      \bibitem{15-zac-1} %11
      \Aue{Minin, V.\,A., I.\,M.~Zatsman, V.\,A.~Havanskov, and S.\,K.~Shubnikov}. 2013. 
Arkhitekturnye resheniya dlya sistem vychisleniya indikatorov tematicheskikh vzaimosvyazey nauki i 
tekhnologiy [Information system conceptual decisions for assessment of linkages between science and 
technologies]. \textit{Sistemy i
Sredstva Informatiki~---
Systems and Means of Informatics}   23(2):260--283.
      \bibitem{16-zac-1} %12
      \Aue{Zatsman, I., and S.~Shubnikov}. 2007. Printsipy obrabotki informatsionnykh resursov dlya 
otsenki innovatsionnogo potentsiala napravleniy nauchnykh issledovaniy [Processing principles of 
information resources for an assessment of innovation potential of the scientific domains]. 
\textit{Trudy 9-y Vserossiyskoy nauchnoy konferentsii ``Elektronnye biblioteki''} [\textit{9th 
      All-Russian Scientific Conference on Digital Libraries Proceedings}]. Pereslavl': Publishing 
House of Pereslavl' University. 35--44.
      \bibitem{17-zac-1} %13
      \Aue{Zatsman, I., O.~Kurchavova, and I.~Galina}. 2008. Informatsionnye resursy i indikatory 
dlya otsenki innova\-tsi\-on\-no\-go potentsiala napravleniy nauchnykh issledovaniy [Information resources and 
indicators for an assessment of innovation potential of the scientific domains]. 
\textit{Sistemy i
Sredstva Informatiki~--- Systems and 
Means of Informatics} 18 (add.):159--175.
      \bibitem{18-zac-1} %14
      \Aue{Kozhunova, O.} 2012. Tsitirovanie dokumentov v patentakh kak indikator vzaimosvyazi 
oblastey nauki i tehnologiy [Citing documents in patents as an indicator for 
science and technologies linkages]. \textit{Sistemy i
Sredstva Informatiki~---
Systems and Means of Informatics} 
22(2):106--128.
      \bibitem{19-zac-1} %15
      \Aue{Van Looy, B., E.~Zimmermann, R.~Veugelers, A.~Verbeek, J.~Mello, and 
K.~Debackere}. 2003. Do science-technology interactions pay on when developing technology? An 
exploratory investigation of 10~science-intensive technology domains. \textit{Scientometrics} 
57(3):355--367.
      \bibitem{20-zac-1} %16
      \Aue{Verbeek, А., K.~Debackere, M.~Luwel, P.~Andries, E.~Zimmermann, and D.~Deleus}. 
2002. Linking science to technology: Using bibliographic references in patents to build linkage 
schemes. \textit{Scientometrics} 54(3):399--420.
      \bibitem{21-zac-1} %17
FIPS. 2008.       Administrativnyy reglament ispolneniya Rospatentom priema zayavok na izobretenie, ikh 
ras\-smot\-re\-niya i ekspertizy [Rospatent administrative regulations for filing invention applications, their 
considerations and examination].  {\sf 
http://www1.fips.ru/wps/wcm/\linebreak 
connect/content\_ru/ru/documents/russian\_laws/order\_\linebreak minobr/administrative\_regulations/test\_8/}
(accessed December 10, 2013).
      \bibitem{22-zac-1} %18
      Standart VOIS ST.14 ``Rekomendatsii po vklyu\-che\-niyu ssylok, citiruemykh v patentnykh 
dokumentakh'' [WIPO Standard ST.14 ``Recommendation for the inclusion of references cited in patent 
documents''].  {\sf http://www.rupto.ru/rupto/nfile/52b8dfc1-1049-11e1-a520-9c8e9921fb2c/03\_14\_01.pdf}
(acessed December 10, 2013).


 
      \bibitem{23-zac-1} %19
      Regulyarnye vyrazheniya v .NET Framework [.NET Framework Regular Expressions].  {\sf 
http://msdn. microsoft.com/ru-ru/library/hs600312.aspx}
(accessed December 10, 2013).

      \bibitem{1-zac1}
      \Aue{Zatsman, I.\,M., and G.\,F.~Verevkin}. 2006. Informatsionnyy monitoring sfery nauki v 
zadachakh programmno-tselevogo upravleniya [Information monitoring in the\linebreak science sphere in 
problems of goal-oriented program management]. \textit{Sistemy i
Sredstva Informatiki~---
Systems and Means of Informatics} 
16:164--189. 
\bibitem{2-zac-1}
 \Aue{Zatsman, I.\,M. G.\,F.~Verevkin, I.\,V.~Drynova, O.\,A.~Kurchavova, N.\,V.~Larin, and 
T.\,P.~Norekjan}. 2006. Mo\-de\-li\-ro\-vanie sistem informatsionnogo monitoringa kak problema 
informatiki [Modeling of systems of information monitoring as informatics problem]. 
\textit{Sistemy i sredstva informatiki. Nauchno-metodologicheskie problemy informatiki} 
[\textit{Systems  and means of informatics.  Scientific and 
methodological problems of informatics}]. Moscow: IPI RAN. 112--139.
      \bibitem{3-zac1}
      \Aue{Zatsman, I., and O.~Kozhunova}. 2007. Semanticheskiy slovar' sistemy informatsionnogo 
monitoringa v sfere nauki: Zadachi i funktsii [Semantic vocabulary of the system of information 
monitoring in scientific sphere: The tasks and functions]. 
\textit{Sistemy i
Sredstva Informatiki~---
Systems and Means of Informatics}
 17:124--141.


 
\bibitem{4-zac-1}
\Aue{Zatsman, I., and O.~Kozhunova}. 2009. Evaluation system for the Russian Academy of 
Sciences: Objectives-resources-results approach and R\&D indicators. \textit{2009 Atlanta Conference 
on Science and Innovation Policy  Proceedings}. Eds.\ S.\,E.~Cozzens,  and P.~Catalаn. {\sf 
http://smartech. gatech.edu/bitstream/1853/32300/1/104-674-1-PB. pdf}.
      
      
\end{thebibliography}
} }


\end{multicols}

\hfill{\small\textit{Received October 7, 2013}}

\Contr

\noindent
\textbf{Zatsman Igor M.} (b.\ 1952)~--- Doctor of Science in technology, Head of Department, 
Institute of Informatics Problems, Russian Academy of Sciences, Moscow 119333,
Russian Federation; iz\_ipi@a170.ipi.ac.ru

\vspace*{3pt}

\noindent
\textbf{Havanskov Valerij A.} (b.\ 1950)~--- scientist, Institute of Informatics Problems, 
Russian Academy of Sciences,
Moscow 119333,
Russian Federation; havanskov@a170.ipi.ac.ru

\vspace*{3pt}

\noindent
\textbf{Shubnikov Sergej K.} (b.\ 1955)~--- senior scientist, Institute of Informatics 
Problems, Russian Academy of Sciences,Moscow 119333,
Russian Federation; sergeysh50@yandex.ru

 \label{end\stat}
 
\renewcommand{\bibname}{\protect\rm Литература}
      