
\def\stat{kor-grig}

\def\tit{О СХОДИМОСТИ РАСПРЕДЕЛЕНИЙ СЛУЧАЙНЫХ СУММ К~СКОШЕННЫМ ЭКСПОНЕНЦИАЛЬНО-СТЕПЕННЫМ 
ЗАКОНАМ$^*$}

\def\titkol{О сходимости распределений случайных сумм к скошенным экспоненциально-степенным 
законам}

\def\autkol{М.\,Е.~Григорьева,  В.\,Ю.~Королев}

\def\aut{М.\,Е.~Григорьева$^1$,  В.\,Ю.~Королев$^2$}

\titel{\tit}{\aut}{\autkol}{\titkol}

{\renewcommand{\thefootnote}{\fnsymbol{footnote}}
\footnotetext[1] {Работа поддержана Российским фондом фундаментальных исследований (проекты 
11-01-00515-а, 11-07-00112-а, 12-07-00115-а).}}

\renewcommand{\thefootnote}{\arabic{footnote}}
\footnotetext[1]{Parexel International, maria-grigoryeva@yandex.ru} 
\footnotetext[2]{Факультет
вычислительной математики и кибернетики Московского государственного
университета им.\ М.\,В.~Ломоносова; Институт проблем информатики
Российской академии наук, victoryukorolev@yandex.ru} 

\vspace*{-12pt}

\Abst{Предложено обобщение класса
экспоненциально-степенных распределений (обобщенных распределений
Лапласа) на несимметричный случай. Класс
скошенных экс\-по\-нен\-ци\-аль\-но-сте\-пен\-ных распределений (скошенных
обобщенных распределений Лапласа) вводится как семейство специальных
дис\-пер\-си\-он\-но-сдвиго\-вых смесей нормальных законов. Найдены выражения
для моментов скошенных экс\-по\-нен\-ци\-аль\-но-сте\-пен\-ных распределений.
Показано, что скошенные экс\-по\-нен\-ци\-аль\-но-сте\-пен\-ные распределения
могут использоваться в качестве асимптотических аппроксимаций. 
С~этой целью доказывается теорема о необходимых и достаточных условиях
сходимости распределений сумм случайного числа независимых одинаково
распределенных случайных величин к скошенным
экспоненциально-степенным распределениям. Для частного случая~---
специальных случайных блужданий с непрерывным временем, порожденных
обобщенными дважды стохастическими пуассоновскими процессами,~---
приводятся оценки скорости этой сходимости.}

\KW{случайная сумма; обобщенное распределение
Лапласа; скошенное обобщенное распределение Лапласа;
экспоненциально-степенное распределение; симметричное устойчивое
распределение; одностороннее устойчивое распределение;
дис\-пер\-си\-он\-но-сдвиго\-вая смесь нормальных законов; смешанное
пуассоновское распределение; смесь распределений вероятностей;
идентифицируемые смеси; аддитивно замкнутое семейство; оценка
скорости сходимости}

\DOI{10.14357/19922264130407}

\vskip 14pt plus 9pt minus 6pt

      \thispagestyle{headings}

      \begin{multicols}{2}

            \label{st\stat}


\section{Экспоненциально-степенные распределения}

Пусть $0<\alpha\leqslant 2$. \textit{Экс\-по\-нен\-ци\-аль\-но-сте\-пен\-ным распределением} 
(\textit{обобщенным рас\-пре\-де\-лением Лапласа}) называется абсолютно 
непрерывное распределение вероятностей, задаваемое лебеговой плот\-ностью
\begin{equation}
\ell_{\alpha}(x)=\fr{\alpha}{2\Gamma({1}/{\alpha})} \,
e^{-|x|^{\alpha}}\,,\ \ \ -\infty< x<\infty\,.\label{e1-gr}
\end{equation}
Чтобы упростить обозначения и вычисления, здесь и далее в
представлении~(\ref{e1-gr}) используется единственный параметр~$\alpha$,
поскольку именно этот \mbox{параметр} является в определенном смысле
характеристическим и определяет форму экс\-по\-нен\-ци\-ально-сте\-пен\-но\-го
распределения. При $\alpha\hm=1$ соотноше\-ние~(\ref{e1-gr}) определяет
классическое распределение Лапласа с нулевым средним и дисперсией~2.
При $\alpha\hm=2$ соотношение~(\ref{e1-gr}) определяет нормальное (гауссово)
распределение с нулевым средним и дисперсией~$1/2$.
\columnbreak

Класс распределений~(\ref{e1-gr}) был предложен и \mbox{изучен} М.\,Ф.~Субботиным в
1923~г.~\cite{Subbotin1923}. Наряду с термином \textit{обобщенное
распределение Лапласа}, восходящим к оригинальной работе~\cite{Subbotin1923}, 
для распределения вида~(\ref{e1-kor}) используются по
крайней мере четыре других термина. В~частности, в~\cite{BoxTiao1973} 
это распределение названо \textit{экс\-по\-нен\-ци\-аль\-но-сте\-пен\-ным}, в~\cite{Evans2000} 
и~\cite{LeemisMcQueston2008} оно названо \textit{обобщенным
распределением ошибок}, в~\cite{Morgan1996} используется 
термин~\textit{обобщенное показательное распределение}, тогда как в работах~\cite{Varanasi1989} 
и~\cite{Nadaraja2005} это распределение
соответственно названо \textit{обобщенным гауссовым} и \textit{обобщенным
нормальным}. Распределения типа~(\ref{e1-gr}) широко используются в
байесовских моделях и разнообразных прикладных задачах от астрономии
до обработки сигналов и изображений.

В работе~\cite{West1987} было замечено, что при $0\hm<\alpha\hm\leqslant2$
распределения вида~(\ref{e1-gr}) представимы в виде масштабных смесей нормальных законов 
(этот результат также цитируется в~\cite{ChoySmith1997}). 
%
Пусть 
$G_{\alpha,\theta}(x)$ и $g_{\alpha,\theta}(x)$~--- соответственно функция 
распределения и плотность строго устойчивого закона с характеристическим 
показателем~$\alpha$ и параметром~$\theta$, определяемые характеристической 
функцией
\pagebreak

\noindent
$$
\mathfrak{g}_{\alpha,\theta}(t)=\exp\left\{
-|t|^{\alpha}\exp\left\{-\fr{i\pi\theta\alpha}{2}\,\mathrm{sign}\,t\right\}\right\},\
\ \ \ t\in\r\,,
$$
где $0<\alpha\leqslant2$,
$|\theta|\hm\leqslant\theta_{\alpha}\hm=\min\{1,{2}/{\alpha}-1\}$ (см., 
например,~\cite{Zolotarev1983}). Положим
\begin{align*}
h_{\alpha/2}(z)&=\fr{\alpha}{\Gamma({1}/{\alpha})}
\sqrt{\fr{\pi}{2}}\cdot\fr{g_{\alpha/2,1}(z)}{\sqrt{z}}\,,\
\ \ z\geqslant0\,;
\\
w_{\alpha/2}(z)&=\fr{h_{\alpha/2}(z^{-1})}{z^2}={}\\
&{}=\fr{\alpha}{\Gamma({1}/{\alpha})}
\sqrt{\fr{\pi}{2}}\,\fr{g_{\alpha/2,1}(z^{-1})}{z^{3/2}}\,,\enskip
z\geqslant0\,.
\end{align*}
Несложно убедиться, что $h_{\alpha/2}(z)$ и $w_{\alpha/2}(z)$~---
плотности распределения. Будем считать, что все случайные величины,
упоминаемые в данной статье, заданы на одном и том же достаточно
богатом вероятностном пространстве $(\Omega,\mathfrak{A},{\sf P})$.
Символ~$\eqd$ обозначает совпадение распределений. Если
$V_{\alpha/2}$ и $U_{\alpha/2}$~--- неотрицательные абсолютно
непрерывные случайные величины с плотностями $h_{\alpha/2}(z)$ и
$w_{\alpha/2}(z)$ соответственно, то легко видеть, что
\begin{equation}
U_{\alpha/2}\eqd\fr{1}{V_{\alpha/2}}\,.\label{e2-gr}
\end{equation}

Хорошо известно, что если $\zeta_{\alpha,\theta}$~--- случайная
величина с устойчивым распределением, соответствующим
характеристической функции $\mathfrak{g}_{\alpha,\theta}$, то 
${\sf E}|\zeta_{\alpha,\theta}|^p\hm<\infty$ для любого $p\hm<\alpha$ (см.,
например,~\cite{Zolotarev1983}). Поэтому из определения плотности
$h_{\alpha/2}(z)$ вытекает, что ${\sf E}V_{\alpha/2}^p\hm<\infty$ для
любого $p\hm<(\alpha\hm+1)/2$, а следовательно, из~(\ref{e2-gr}) вытекает, что 
${\sf E}U_{\alpha/2}^q\hm<\infty$ для любого $q\hm>-(\alpha\hm+1)/2$.

Функции распределения, соответствующие плотностям $\ell_{\alpha}(x)$, 
$h_{\alpha/2}(z)$ и $w_{\alpha/2}(z)$, будут обозначаться соответствующими 
заглавными буквами: $L_{\alpha}(x)$, $H_{\alpha/2}(z)$ и $W_{\alpha/2}(z)$. 
Стандартная нормальная функция распределения ($\alpha\hm=2$) и ее плотность будут 
соответственно обозначаться $\Phi(x)$ и $\varphi(x)$:
$$
\varphi(x)=\fr{1}{\sqrt{2\pi}}e^{-x^2/2}\,;\quad 
\Phi(x)=\int\limits_{-\infty}^x\varphi(z)\,dz\,.
$$

\smallskip

\noindent
\textbf{Лемма 1.} \textit{При $0\hm<\alpha\hm\leqslant2$ экс\-по\-нен\-ци\-аль\-но-сте\-пен\-ное
распределение~$(\ref{e1-gr})$ является масштабной смесью нормальных законов}:
$$
L_{\alpha}(x)=\int\limits_0^{\infty}\Phi\left(x\sqrt{z}\right)\,dH_{\alpha/2}(z)\,,\quad
 x\in\mathbb{R}\,;
$$
\begin{equation}
L_{\alpha}(x)=\int\limits_0^{\infty}\Phi\left(\fr{x}{\sqrt{z}}\right)\,dW_{\alpha/2}(z)\,,\enskip
x\in\mathbb{R}\,.\label{e3-gr}
\end{equation}

\smallskip

\noindent
Д\,о\,к\,а\,з\,а\,т\,е\,л\,ь\,с\,т\,в\,о\ \ \ см.\ в~\cite{West1987} или~\cite{KBZZ2012}.

\smallskip

В работе~\cite{KBZZ2012} показано, что распределения вида~(\ref{e1-gr}) могут
выступать в качестве предельных законов в довольно простых
предельных теоремах для сумм случайного числа независимых одинаково
распределенных случайных величин и статистик, построенных по
выборкам случайного объема, и, как следствие, обосновано их
применение в качестве асимптотических аппроксимаций для
статистических закономерностей в прикладных задачах. В~данной статье
вводится несимметричное расширение класса распределений~(\ref{e1-gr}) и
доказываются теоремы, устанавливающие критерии сходимости
распределений случайных сумм к несимметричным
экс\-по\-нен\-ци\-аль\-но-сте\-пен\-ным законам и скорость этой сходимости.

\section{Скошенные экспоненциально-степенные распределения}

Все распределения вида~(\ref{e1-gr}) симметричны. Рассмотрим естественное
несимметричное расширение класса распределений вида~(\ref{e1-gr}). С~этой
\mbox{целью} используем подход, аналогичный тому, с помощью которого в 1977~г.\ 
О.~Барн\-дорфф-Ниль\-сен ввел класс обобщенных гиперболических
распределений как класс специальных дис\-пер\-си\-он\-но-сдви\-го\-вых смесей
нормальных законов~\cite{BN1977}. Основой для соответствующих
рассуждений является представление~(\ref{e3-gr}).

Пусть $\alpha\in(0,2]$, $\mu\in\mathbb{R}$. \textit{Скошенным
экс\-по\-нен\-ци\-аль\-но-сте\-пен\-ным распределением} (\textit{скошенным обобщенным
распределением Лапласа}) с параметром формы~$\alpha$ и параметром
скошенности (асимметрии)~$\mu$ назовем распределение вероятностей,
функция распределения которого имеет вид

\noindent
\begin{equation}
L_{\alpha,\mu}(x)=\int\limits_{0}^{\infty}\Phi\left(\fr{x-\mu
z}{\sqrt{z}}\right)\,dW_{\alpha/2}(z)\,,\enskip x\in\mathbb{R}\,.
\label{e4-gr}
\end{equation}
Формально в смеси~(\ref{e4-gr}) смешивание производится по обоим параметрам
нормального распределения. Однако за счет того, что эти параметры в
пред\-став\-ле\-нии~(\ref{e4-gr}) жестко связаны и математические ожидания (сдвиги)
смешиваемых нормальных законов в~(\ref{e4-gr}) оказываются пропорциональными
\textit{дисперсиям}, фактически в~(\ref{e4-gr}) смешивание производится по 
\textit{одному} параметру. В~силу этого обстоятельства Барн\-дорфф-Ниль\-сен
и его коллеги назвали смеси вида~(\ref{e4-gr}) \textit{дис\-пер\-си\-он\-но-сдви\-го\-вы\-ми}
(variance-mean mixtures)~\cite{BNKS1982}.

\pagebreak

Если $X$~--- случайная величина со стандартным нормальным
распределением, независимая от случайной величины $U_{\alpha/2}$,
введенной ранее, то, как несложно видеть, функция распределения
$L_{\alpha,\mu}(x)$ (см.~(\ref{e4-gr})) соответствует случайной величине
$$
Z_{\alpha,\mu}=X\sqrt{U_{\alpha/2}}+\mu U_{\alpha/2}\,.
$$

Найдем моменты случайной величины $Z_{\alpha,\mu}$. С~этой целью
сначала вычислим моменты случайной величины $U_{\alpha/2}$. При
$\mu=0$ имеет место соотношение 
$$
Z_{\alpha,0}\eqd
X\sqrt{U_{\alpha/2}}\,.
$$
При этом из независимости случайных величин~$X$ 
и $U_{\alpha/2}$ вытекает, что для каждого $k\hm\in\mathbb{N}$
$$
{\sf E}Z_{\alpha,0}^{2k}={\sf E}X^{2k}\cdot{\sf E}U_{\alpha/2}^k\,.
$$
Из~(\ref{e1-gr}) несложно видеть, что
\begin{multline*}
{\sf E}Z_{\alpha,0}^{2k}=\fr{\alpha}{\Gamma({1}/{\alpha})}
\int\limits_{0}^{\infty}\!x^{2k}e^{-|x|^{\alpha}}dx={}\\
{}=\fr{1}{\Gamma({1}/{\alpha})}\int\limits_{0}^{\infty}\!
x^{(2k+1)/\alpha-1}e^{-|x|}\,dx=\fr{\Gamma((2k+1)/\alpha)}{\Gamma({1}/{\alpha})}.
\end{multline*}
С другой стороны, как известно, ${\sf E}X^{2k}\hm=(2k-1)!!$. Поэтому
для любого $k\hm\in\mathbb{N}$
$$
{\sf E}U_{\alpha/2}^k=\fr{{\sf E}Z_{\alpha,0}^{2k}}{{\sf E}X^{2k}}=
\fr{\Gamma((2k+1)/\alpha)}{(2k-1)!!\,\Gamma({1}/{\alpha})}\,.
$$
Таким образом,
\begin{align*}
{\sf E}Z_{\alpha,\mu}&=\mu{\sf E}U_{\alpha/2}=
\fr{\mu\Gamma({3}/{\alpha})}{\Gamma({1}/{\alpha})}\,;\\
{\sf E}Z_{\alpha,\mu}^2&={\sf E}U_{\alpha/2}+\mu^2{\sf E}U_{\alpha/2}^2=
\fr{3\Gamma({3}/{\alpha})+\mu^2\Gamma({5}/{\alpha})}{3\Gamma({1}/{\alpha})}\,;
%\ \ {\sf D}Z_{\alpha,\mu}={\sf E}U_{\alpha/2}+\mu^2{\sf
%D}U_{\alpha/2},
\\
{\sf E}Z_{\alpha,\mu}^3&=3\mu{\sf E}U_{\alpha/2}^2+\mu^3{\sf E}
U_{\alpha/2}^3={}\\
&{}\hspace*{22mm}=\fr{15\mu\Gamma({5}/{\alpha})+\mu^3\Gamma({7}/{\alpha})}
{15\Gamma({1}/{\alpha})}\,;
\\
{\sf E}Z_{\alpha,\mu}^4&=3{\sf E}U_{\alpha/2}^2+6\mu^2{\sf E}U_{\alpha/2}^3+
\mu^4{\sf E}U_{\alpha/2}^4={}\\
&{}=\fr{105\Gamma({5}/{\alpha})+42\mu^2\Gamma({7}/{\alpha})
+\mu^4\Gamma({9}/{\alpha})}{105\Gamma({1}/{\alpha})}\,.
\end{align*}

Обозначим
$$
\Delta U_{\alpha/2}=\fr{{\sf E}U_{\alpha/2}}{{\sf D}U_{\alpha/2}}=
\fr{3\Gamma({3}/{\alpha})\Gamma({1}/{\alpha})}
{\Gamma({1}/{\alpha})\Gamma({5}/{\alpha})-3[\Gamma({3}/{\alpha})]^2}\,.
$$
Тогда коэффициенты асимметрии (скошенности) $\kappa_3(Z)$ и эксцесса
$\kappa_4(Z)$ (ос\-тро\-вер\-шин\-нос\-ти) случайной величины~$Z$ принимают вид:
\begin{multline*}
\kappa_3(Z)=\fr{{\sf E}(Z-{\sf E}Z)^3}{\left({\sf D}Z\right)^{3/2}}={}\\
{}=\mu \fr{3{\sf D}U_{\alpha/2}+\mu^2{\sf E}
(U_{\alpha/2}-{\sf E}U_{\alpha/2})^3}{\left({\sf E}
U_{\alpha/2}+\mu^2{\sf D}U_{\alpha/2}\right)^{3/2}}={}\\
{}=
\fr{\mu\kappa_3(U_{\alpha/2})}{(\mu^2+\Delta U_{\alpha/2})^{3/2}}\,;
\end{multline*}

\vspace*{-12pt}

\noindent
\begin{multline}
\kappa_4(Z)=\fr{{\sf E}(Z-{\sf E}Z)^4}{\left({\sf D}Z\right)^2}={}\\
{}=\fr{1}{(\mu^2+\Delta
U_{\alpha/2})^2}\left[
\vphantom{\fr{\kappa_3(U_{\alpha/2})}{\sqrt{{\sf D}
U_{\alpha/2}}}}
3\fr{{\sf E}U_{\alpha/2}^2}{({\sf E}U_{\alpha/2})^2}+{}\right.\\
\left.{}+
\mu^2\left(\fr{\kappa_3(U_{\alpha/2})}{\sqrt{{\sf D}
U_{\alpha/2}}}+\mu^2\kappa_4(U_{\alpha/2})\right)\right]\,,\label{e5-gr}
\end{multline}
где $\kappa_3(U_{\alpha/2})$ и $\kappa_4(U_{\alpha/2})$--- соответственно 
коэффициенты асимметрии и эксцесса случайной
величины~$U_{\alpha/2}$:
\begin{multline*}
\kappa_3\left(U_{\alpha/2}\right)=\fr{{\sf E}(U_{\alpha/2}-{\sf E}U_{\alpha/2})^3}
{({\sf D}U_{\alpha/2})^{3/2}}={}\\
{}=
\left\{\!
\left[\Gamma\left(\fr{1}{\alpha}\right)\right]^2\!\Gamma\left(\fr{7}{\alpha}\right)-
15\Gamma\left(\fr{1}{\alpha}\right)\Gamma\left(\fr{3}{\alpha}\right)
\Gamma\left(\fr{5}{\alpha}\right)+{}\right.
\\
\left.{}+
30\left[\Gamma\left(\fr{3}{\alpha}\right)\right]^3\right\}\Bigg/
\left( \vphantom{\left(\left[
\Gamma\left(\fr{3}{\alpha}
\right)\right]^2\right)^{3/2}}
5\left\{ \vphantom{\left[ \Gamma\left(\fr{3}{\alpha} \right)\right]^2}
\Gamma\left(\fr{1}{\alpha}\right)
\Gamma\left(\fr{5}{\alpha}\right)-{}\right.\right.\\
\left.\left.{}-3\left[
\Gamma\left(\fr{3}{\alpha}
\right)\right]^2\right\}^{3/2}\right)\,;
\end{multline*}

\vspace*{-12pt}

\noindent
\begin{multline*}
\kappa_4\left(U_{\alpha/2}\right)=
\fr{{\sf E}(U_{\alpha/2}-{\sf E}U_{\alpha/2})^4}
{({\sf D}U_{\alpha/2})^2}={}\\
{}=\left(\Gamma\left(\fr{9}{\alpha}\right)\left[\Gamma\left(\fr{1}{\alpha}\right)\right]^3-{}\right.\\
{}-28\left[\Gamma\left(\fr{1}{\alpha}\right)\right]^2\Gamma
\left(\fr{3}{\alpha}\right)\Gamma\left(\fr{7}{\alpha}\right)+{}\\
{}+
210\Gamma\left(\fr{1}{\alpha}\right)\left[\Gamma\left(\fr{3}{\alpha}\right)\right]^2
\Gamma\left(\fr{5}{\alpha}\right)-{}\\
\left.{}-
315\left[\Gamma\left(\fr{3}{\alpha}\right)\right]^4 \right)\Bigg/ \left(
\vphantom{\left(\left[
\Gamma\left(\fr{3}{\alpha}
\right)\right]^2\right)^{3/2}}
105 \left\{\vphantom{\left[ \Gamma\left(\fr{3}{\alpha} \right)\right]^2}
\Gamma\left(\fr{1}{\alpha}\right)\Gamma\left(\fr{5}{\alpha}\right)-{}\right.\right.\\
\left.\left.{}-
3\left[\Gamma\left(\fr{3}{\alpha}\right)\right]^2\right\}^2\right)\,.
\end{multline*}


Если $\mu=0$, то смесь~(\ref{e4-gr}) является чисто масштабной. Из~(\ref{e5-gr}) видно, что в таком 
случае $\kappa_4(Z)\hm\geqslant 3$, т.\,е.\ при $0\hm<\alpha\hm<2$ 
экс\-по\-нен\-ци\-аль\-но-сте\-пен\-ные распределения вида~(\ref{e1-gr}) 
всегда имеют более острую 
вершину и, соответственно, более тяжелые хвосты, нежели нормальное  распределение.

\section{Критерий сходимости распределений случайных сумм 
к~скошенным экспоненциально-степенным распределениям}

Пусть $\{X_{n,j}\}_{j\geqslant1}$, $n\hm=1,2,\ldots$~--- семейство
по\-следовательностей одинаково распределенных в\linebreak каж\-дой последовательности (при 
каждом фиксированном~$n$) случайных величин. Пусть $\{N_n\}_{n\geqslant1}$~--- 
последовательность целочисленных неотрицательных случайных величин таких, что 
пpи каждом $n\hm\geqslant1$ случайные величины $N_n,X_{n,1},X_{n,2},\ldots$\ 
независимы. Положим
$$
S_{n,k}=X_{n,1}+\cdots +X_{n,k}\,.
$$
Во избежание недоразумений полагаем $\sum_{j=1}^0\hm=0$. Символ~$\Longrightarrow$ 
будет обозначать слабую сходимость (сходимость по
распределению). Везде далее сходимость будет подразумеваться при $n\hm\to\infty$.

\smallskip

Как показано в~\cite{Korolev2012TVP}, дис\-пер\-си\-он\-но-сдвиго\-вые смеси нормальных 
законов вида~(\ref{e3-gr}) оказываются идентифицируемыми, поскольку при каждом 
фиксированном $\mu\hm\in\mathbb{R}$ однопараметрическое семейство распределений 
$\left\{\Phi\left((x-\mu z)/\sqrt{z}\right):\,z\hm\geqslant0\right\}$ является 
ад\-ди\-тив\-но-замкну\-тым. В~статье~\cite{Korolev2012TVP} доказано следующее общее 
утверждение (см.\ так\-же~\cite{ZaksKorolev2013}).

\smallskip

\noindent
\textbf{Теорема 1}. \textit{Предположим, что существуют последовательность
натуральных чисел $\{k_n\}_{n\hm\geqslant1}$ и число $\mu\hm\in\mathbb{R}$ такие, что}
\begin{equation}
{\sf P}\big(S_{n,k_n}<x\big)\Longrightarrow \Phi(x-\mu)\,.\label{e6-gr}
\end{equation}
\textit{Предположим, что $N_n\hm\to\infty$ по вероятности. Для того чтобы имела
место сходимость распределений случайных сумм к некоторой функции
распределения} $F(x):$
$$
{\sf P}\left(S_{n,N_n}<x\right)\Longrightarrow F(x)\,,
$$
\textit{необходимо и достаточно, чтобы существовала функция распределения
$Q(x)$ такая, что} 
\begin{gather*}
Q(0)=0\,;\quad
F(x)=\int\limits_{0}^{\infty}\Phi\left(\fr{x-\mu
z}{\sqrt{z}}\right)\,dQ(z) \,;\label{e7-gr}
\\
{\sf P}(N_n<xk_n)\Longrightarrow Q(x)\,. %\label{e8-gr}
\end{gather*}


\smallskip

\noindent
\textbf{Замечание~1.} Условие~(\ref{e6-gr}) выполняется в следующей довольно
общей ситуации. Предположим, что случайные величины $X_{n,j}$ имеют
конечные дисперсии. Также предположим, что величины $X_{n,j}$ могут
быть представлены в виде
$$
X_{n,j}=X_{n,j}^*+\mu_n\,,
$$
где $\mu_n\in\mathbb{R}$, a $X_{n,j}^*$~-- случайная величина с
${\sf E} X_{n,j}^*\hm=0$, ${\sf D} X_{n,j}^*\hm=\sigma_n^2\hm<\infty$, так
что ${\sf E} X_{n,1}\hm=\mu_n$ и ${\sf D} X_{n,1}\hm=\sigma_n^2$.
Предположим, что $\mu_nk_n\hm\to a$ и $k_n\sigma_n^2\hm\to 1$ при
$n\hm\to\infty$. Тогда вследствие хорошо известного результата о
необходимых и достаточных условиях сходимости к нормальному закону
распределений сумм независимых случайных величин с конечными
дисперсиями в схеме серий (см., например,~\cite{GnedenkoKolmogorov1949}), 
можно заметить, что соотношение~(\ref{e6-gr})
имеет место тогда и только тогда, когда выполнено условие
Линдеберга: для любого $\varepsilon\hm>0$
$$
\lim_{n\to\infty}k_n{\sf E}(X_{n,1}^*)^2\mathbb{I}(|X_{n,1}^*|\geqslant\varepsilon)=0
$$
(здесь $\mathbb{I}(A)$~--- индикаторная функция множества (события)~$A$), 
т.\,е.\ квадратичные хвосты распределений слагаемых должны убывать достаточно \mbox{быстро}.

\smallskip

\noindent
\textbf{Следствие 1}. \textit{Предположим, что существуют последовательность 
натуральных чисел $\{k_n\}_{n\geqslant1}$ и число $\mu\hm\in\mathbb{R}$ такие, что 
имеет место сходимость~$(\ref{e6-gr})$. Предположим, что $N_n\hm\to\infty$ по вероятности. 
Для того чтобы имела место сходимость}
\begin{equation}
{\sf P}\big(S_{n,N_n}<x\big)\Longrightarrow
L_{\alpha,\mu}(x)\,,\label{e10-gr}
\end{equation}
\textit{необходимо и достаточно, чтобы}
\begin{equation}
{\sf P}(N_n<xk_n)\Longrightarrow W_{\alpha/2}(x)\,.\label{e11-gr}
\end{equation}

\smallskip

\noindent
\textbf{Замечание 2.} В~соотношениях~(\ref{e6-gr}), (\ref{e10-gr}) и~(\ref{e11-gr}) пре\-дельные
функции распределения непрерывны. По\-этому в этих соотношениях
сходимость по распределению эквивалентна равномерной сходи\-мости \mbox{функций} распределения.

\section{{Оценки скорости сходимости распределений случайых 
сумм к~скошенным экспоненциально-степенным законам}}

В данном разделе будет использоваться специальная и довольно
естественная конструкция случайных блужданий, удовлетворяющая
комплексу условий, указанному в замечании~1.

Пусть $\xi_1,\xi_2,\ldots$~--- независимые одинаково распределенные
случайные величины с ${\sf E}\xi_1\hm=0$, $0\hm<{\sf D}\xi_1\hm=1$,
$\beta^3\hm={\sf E}|\xi_1|^3\hm<\infty$, $a\hm\in\r$, $n$~--- натуральное число.

Положим
\begin{equation}
X_{n,j}=\fr{\xi_j}{\sqrt{n}}+\fr {\mu}{n}\,.\label{e12-gr}
\end{equation}
В терминах случайных блужданий случайные величины $X_{n,j}$,
определенные соотношением~(\ref{e12-gr}), могут быть интерпретированы как
элементарные приращения процесса, при этом рассматриваемая их
конструкция~(\ref{e12-gr}) предполагает \textit{одинаковый порядок малости}
элементарных трендов и \textit{дисперсий}, что характерно, например,
для приращений винеровского процесса со сносом. Обозначим
$$
S_n=\sum_{j=1}^n X_{n,j}\
\left(=\fr{1}{\sqrt{n}}\sum\limits_{j=1}^n\xi_j+\mu\right)\,.
$$
В силу классической центральной предельной теоремы имеем
$$
\lim_{n\to\infty}\sup\limits_{x\in\r}\left\vert {\sf P}(S_n<x)-\Phi(x-\mu)\right\vert=0\,,
$$
т.\,е.\ так определенные случайные величины $X_{n,j}$ удовлетворяют
условию~(\ref{e6-gr}) с $k_n\hm=n$.

В книге~\cite{GnedenkoKorolev1996} и статьях~\cite{Korolev1997, Korolev2000} 
предложено моделировать эволюцию неоднородных хаотических стохастических 
процессов, в частности динамику цен финансовых активов, с помощью обобщенных 
дважды стохастических пуассоновских процессов (обобщенных процессов Кокса). 
Этот подход получил дополнительное обоснование и развитие в~[20--23]. 
В~работах~\cite{Korolev2011, KorolevSkvortsova2006} этот подход успешно применен к 
моделированию процессов плазменной турбулентности. В~соответствии с указанным 
подходом поток информативных событий, в результате каждого из которых 
появляется очередное <<наблюденное>> значение рассматриваемой характеристики, 
описывается с помощью точечного случайного процесса вида $M(\Lambda(t))$, где 
$M(t)$, $t\hm\geqslant 0$,~--- однородный пуассоновский процесс с единичной 
интенсивностью, а $\Lambda(t)$, $t\hm\geqslant 0$,~--- независимый от $M(t)$ 
случайный процесс, обладающий следующими свойствами: $\Lambda(0)\hm=0$, 
${\sf  P}(\Lambda(t)\hm<\infty)\hm=1$ для любого $t\hm>0$, траектории $\Lambda(t)$ 
не убывают и  непрерывны справа. Процесс $M(\Lambda(t))$, $t\hm\geqslant 0$, называется дважды 
стохастическим пуассоновским процессом (процессом Кокса). В~частности, если 
процесс $\Lambda(t)$ допускает представление
$$
\Lambda(t)=\int\limits_{0}^{t}\lambda(\tau)\,d\tau\,,\enskip t\geqslant0\,,
$$
в котором $\lambda(t)$~--- положительный случайный процесс с
интегрируемыми траекториями, то $\lambda(t)$ можно интерпретировать
как мгновенную стохастическую интенсивность процесса Кокса.

В соответствии с такой моделью в каждый момент времени~$t$ распределение 
случайной величины $M(\Lambda(t))$ является смешанным пуассоновским. 
С~практической точки зрения для описания статистических закономерностей поведения 
интенсивности потока информативных событий удобно использовать такую гибкую 
модель, как обобщенное обратное гауссовское распределение. Для большей 
наглядности рассмотрим случай, когда в рассматриваемой модели время~$t$ 
остается фиксированным, а $\Lambda(t)\hm=nU_{\alpha/2}$, где $n$~--- 
вспомогательный параметр; $U_{\alpha/2}$~--- введенная выше случайная величина, 
имеющая плотность распределения $w_{\alpha/2}(x)$, независимая от стандартного 
пуассоновского процесса $M(t)$, $t\hm\geqslant0$. При этом асимптотика 
$n\hm\to\infty$ может интерпретироваться как то, что (случайная) интенсивность 
потока информативных событий считается очень большой, а при использовании 
подобных\linebreak
моделей в задачах финансовой математики рас\-пределение случайной 
величины $U_{\alpha/2}$ довольно естествен\-но отождествляется со статистическими 
закономерностями поведения (случайной) волатильности. Для каждого натурального~$n$ положим
$$
N_n=M\left(nU_{\alpha/2}\right)\,.
$$
Очевидно, что так определенная случайная величина~$N_n$ имеет
смешанное пуассоновское распределение:

\noindent
\begin{multline}
{\sf P}(N_n=k)={\sf P}\left(M(nU_{\alpha/2})=k\right)={}\\
{}=
\int\limits_0^{\infty}e^{-nz}\fr{(nz)^k}{k!}w_{\alpha/2}(z)\,dz\,,\enskip
k=0,1,\ldots\label{e13-gr}
\end{multline}

Обозначим $A_n(z)\hm\equiv A_n(z;\nu,\mu,\lambda)\hm={\sf P}(N_n\hm<nz)$, $z\hm\geqslant0$ 
($A_n(z)\hm=0$ при $z\hm<0$). Несложно видеть, что $A_n(z)\hm\Longrightarrow 
W_{\alpha/2}(z)$ (см., например,~\cite{ZaksKorolev2013}), т.\,е.\ так 
определенные случайные величины $N_n$ удовлетворяют условию~(\ref{e11-gr}) с $k_n\hm=n$. 
Впредь будем считать, что при каждом $n\hm\geqslant1$ случайная величина~$N_n$ 
независима от последовательности $\{\xi_j\}_{j\geqslant1}$, что гарантирует 
независимость случайных величин $N_n,X_{n,1},X_{n,2},\ldots$

Таким образом, в силу непрерывности функции распределения
$W_{\alpha/2}(x)$ из следствия~1 вытекает, что
\begin{multline*}
D_n\equiv\sup_{x\in\r}\left\vert{\sf
P}\left(\sum_{j=1}^{N_n}X_{n,j}<x\right)-L_{\alpha,\mu}(x)\right\vert\longrightarrow{}\\
\longrightarrow 0\enskip (n\to\infty)
\end{multline*}
(см.\ замечание~2).
\pagebreak

Скорость стремления~$D_n$ к нулю описывается следующим утверждением.

\smallskip

\noindent
\textbf{Теорема 2.} \textit{Для любого $n\hm\geqslant1$ справедлива оценка}
$$
D_n\leqslant 0{,}5681\fr{\alpha}{\Gamma\left({1}/{\alpha}\right)}\,
\fr{\beta^3}{\sqrt{n}} +0{,}1210\fr{\mu^2}{n}\,.
$$

\smallskip

\noindent
Д\,о\,к\,а\,з\,а\,т\,е\,л\,ь\,с\,т\,в\,о\,.\ \ Как уже было показано, распределение случайной
величины~$N_n$ является смешанным пуассоновским (см.~(\ref{e13-gr})).
Следовательно, по теореме Фубини
\begin{multline*}
{\sf P}\left(\sum\limits_{j=1}^{N_n}X_{n,j}<x\right)=
{\sf P}\left(\sum\limits_{j=1}^{M(nU_{\alpha/2})}X_{n,j}<x\right)={}\\
{}=
\int\limits_0^{\infty}{\sf P}\left(\sum\limits_{j=1}^{M(nz)}X_{n,j}<x\right)w_{\alpha/2}(z)\,dz\,.
\end{multline*}
При этом
$$
{\sf E}X_{n,j}=\fr{\mu}{n}\,;\enskip  {\sf D}X_{n,j}=\fr{1}{n}\,;\enskip
{\sf E}|X_{n,j}-{\sf E}X_{n,j}|^3=\fr{\beta^3}{n^{3/2}}\,.
$$
Таким образом, при каждом $z\hm\in(0,\infty)$
$$
{\sf E}\sum\limits_{j=1}^{M(nz)}X_{n,j}=\mu z\,;
$$
$$
{\sf D}\sum\limits_{j=1}^{M(nz)}X_{n,j}=
nz\left(\fr{\mu^2}{n^2}+\fr{1}{n}\right)=z\left(1+\fr{\mu^2}{n}\right)\,.
$$
Из~(\ref{e13-gr}) вытекает, что
\begin{multline}
D_n=\sup\limits_x
\Bigg|\int\limits_{0}^{\infty}w_{\alpha/2}(z)\left[{\sf P}\left(\sum\limits_{j=1}^{M(nz)}
X_{n,j}<x\right)-{}\right.\\
{}-\Phi\left(\fr{x-\mu z}{\sqrt{z(1+{\mu^2}/{n})}}\right)+
\Phi\left(\fr{x-\mu z}{\sqrt{z(1+{\mu^2}/{n})}}\right)-{}\\
\left.{}-\Phi\left(\fr{x-\mu z}
{\sqrt{z}}\right)
\vphantom{\sum\limits_{j=1}^{M(nz)}}
\right]\,dz\Bigg| \leqslant I_1+I_2\,,\label{e14-gr}
\end{multline}
где
\begin{align*}
I_1&=\int\limits_{0}^{\infty}w_{\alpha/2}(z)\sup\limits_x
\left\vert 
\vphantom{\fr{x-\mu z}
{\sqrt{z(1+{\mu^2}/{n})}}}
{\sf P}
\left(\sum\limits_{j=1}^{M(nz)}X_{n,j}<x\right)-{}\right.\\
&\left.\hspace*{19mm}{}-\Phi\left(\fr{x-\mu z}
{\sqrt{z(1+{\mu^2}/{n})}}\right)\right\vert\,dz\,;
\\
I_2&=\int\limits_{0}^{\infty}w_{\alpha/2}(z)
\sup\limits_x\left\vert\Phi\left(\fr{x-\mu z}
{\sqrt{z(1+{\mu^2}/{n})}}\right)-{}\right.\\
&\hspace*{31mm}\left.{}-\Phi\left(\fr{x-az}{\sqrt{z}}\right)
\vphantom{\fr{x-\mu z}
{\sqrt{z(1+{\mu^2}/{n})}}}\right\vert\,dz\,.
\end{align*}
В дальнейшем понадобится следующее утверждение.

\medskip

\noindent
\textbf{Лемма 2.} \textit{Пусть случайные величины $X_1,X_2,\ldots$\linebreak
одинаково распределены. Пусть $N_{\lambda}$~--- пуассоновская
случайная величина с параметром $\lambda\hm>0$. Предположим, что
случайные величины $N_{\lambda},X_1,X_2,\ldots$\ \ независимы в
совокупности. Обозначим
$$
S_{\lambda}=X_1+\cdots+X_{N_{\lambda}}\,.
$$
Тогда}
\begin{multline*}
\sup\limits_x\left\vert {\sf P}(S_{\lambda}<x)-\Phi\left(\fr{x-{\sf E}S_{\lambda}}
{\sqrt{{\sf D}S_{\lambda}}}\right)\right\vert 
\leqslant{}\\
{}\leqslant \fr{0{,}4532}{\sqrt{\lambda}}\cdot\fr{{\sf E}|X_1-{\sf E}X_1|^3}{({\sf D}X_1)^{3/2}}\,.
\end{multline*}

%\medskip

\noindent
Д\,о\,к\,а\,з\,а\,т\,е\,л\,ь\,с\,т\,в\,о\ \ леммы~2 приведено в 
работе~\cite{KorolevShevtsovaShorgin2011} (см.\ так\-же~[9, с.~144]).

\smallskip

Продолжим доказательство теоремы~2. Рас\-смот\-рим~$I_1$. Применяя лемму~2, получаем
\begin{multline}
I_1\leqslant0{,}4532\fr{\beta^3}{\sqrt{n}}\int\limits_{0}^{\infty}
\fr{w_{\alpha/2}(z)}{\sqrt{z}}\,
dz={}\\
{}=0{,}4532\fr{\beta^3}{\sqrt{n}}\,{\sf E}U_{\alpha/2}^{-1/2}.\label{e15-gr}
\end{multline}

\medskip

\noindent
\textbf{Лемма 3.} \textit{Для любого $\alpha\hm\in(0,2]$}
$$
{\sf E}U_{\alpha/2}^{-1/2}=\fr{\alpha}{\Gamma\left({1}/{\alpha}\right)}\,
\sqrt{\fr{\pi}{2}}\,.
$$

\smallskip

\noindent
Д\,о\,к\,а\,з\,а\,т\,е\,л\,ь\,с\,т\,в\,о\ \ см.\ в~\cite{KBZZ2012}.

\smallskip

Продолжив~(\ref{e15-gr}) с учетом леммы~3, получим
$$
I_1\leqslant0{,}5681
\fr{\alpha}{\Gamma\left({1}/{\alpha}\right)}\,\fr{\beta^3}{\sqrt{n}}\,.
$$

Рассмотрим $I_2$. В~дальнейшем понадобится еще одно вспомогательное утверждение.

\medskip

\noindent
\textbf{Лемма 4.} \textit{Пусть $b\hm\in\r$, $0\hm<c\hm<\infty$, $0\hm<d\hm<\infty$.
Тогда}
\begin{align}
\hspace*{-2mm}\sup\limits_y|\Phi(y)-\Phi(cy)|&\leqslant
\fr{1}{\sqrt{2\pi e}}\left\vert \max\left\{c,\,\fr{1}{c}\right\}-1\right\vert;\!
\label{e16-gr}
\\
\sqrt{1+d}-1&\leqslant\fr{d}{2}\,.\label{e17-gr}
\end{align}

\smallskip

Элементарное {д\,о\,к\,а\,з\,а\,т\,е\,л\,ь\,с\,т\,в\,о}\ неравенств~(\ref{e16-gr}) 
и~(\ref{e17-gr}) можно
получить, например, с помощью формулы Лагранжа.

\pagebreak

Продолжим доказательство теоремы~2. В~лемме~4 положим
$$
y=\fr{x-\mu z}{\sqrt{z(1+{\mu^2}/{n})}}\,;\quad
c=\sqrt{1+\fr{\mu^2}{n}}\,.
$$
Тогда $c\hm\geqslant1$ и в силу утверждения~(\ref{e16-gr}) леммы~4 имеем:
$$
I_2\leqslant\fr{1}{\sqrt{2\pi e}}\left(\sqrt{1+\fr{\mu^2}{n}}-1\right)\,.
$$
При этом в силу утверждения~(\ref{e17-gr}) леммы~4
$$
\sqrt{1+\fr{\mu^2}{n}}-1\leqslant\fr{\mu^2}{2n}\,.
$$
Окончательно получаем
\begin{equation}
I_2\leqslant\fr{\mu^2}{2\sqrt{2\pi e}\cdot n}\,.\label{e18-gr}
\end{equation}
Подставляя~(\ref{e15-gr}) и~(\ref{e18-gr}) в~(\ref{e14-gr}), получаем утверждение теоремы. 
Теорема доказана.

\smallskip

Для случая симметричных экс\-по\-нен\-ци\-аль\-но-сте\-пен\-ных распределений (т.\,е.\ 
$\mu\hm=0$) оценку, представленную в теореме~2, можно уточнить не
только за счет того, что в таком случае обнуляется второе слагаемое
в правой части, но и за счет уменьшения коэффициента при первом
слагаемом. В~работе~\cite{KBZZ2012} с использованием оценки точности
нормальной аппроксимации для пуассоновских сумм, полученной 
в~\cite{KorolevShevtsova2010} в терминах начальных моментов, доказано следующее утверждение.

\smallskip

\noindent
\textbf{Теорема 3.} \textit{Пусть в дополнение к условиям теоремы~$1$
$\mu\hm=0$. Тогда}
$$
D_n\leqslant
0{,}3812\fr{\alpha}{\Gamma\left({1}/{\alpha}\right)}\,\fr{\beta^3}{\sqrt{n}}\,.
$$

{\small\frenchspacing
{%\baselineskip=10.8pt
\addcontentsline{toc}{section}{Литература}
\begin{thebibliography}{99}
\bibitem{Subbotin1923} % 1
\Au{Subbotin M.\,T.} On the law of frequency of error~//
Матем. сб., 1923. Т.~31. №\,2. С.~296--301.

\bibitem{BoxTiao1973} % 2
\Au{Box G., Tiao~G.} Bayesian inference in statistical analysis.~--- 
Reading, MA: Addison--Wesley, 1973. 608~с.

\bibitem{Evans2000} % 3
\Au{Evans M., Hastings N., Peacock~J.\,B.} Statistical distributions.~---
3rd ed.~--- N.Y.: John Wiley\,\&\,Sons, 2000. 170~p.

\bibitem{LeemisMcQueston2008} % 4
\Au{Leemis L.\,M., McQueston~J.\,T.} Univariate distribution
relationships~// Amer. Stat., 2008. Vol.~62. No.\,1. P.~45--53.

\bibitem{Morgan1996} % 5
RiskMetrics Technical Document.~--- N.Y.: RiskMetric Group, J.\,P.~Morgan, 1996.

\bibitem{Varanasi1989} % 6
\Au{Varanasi M.\,K., Aazhang~B.} Parametric generalized Gaussian
density estimation~// J.~Acoust. Soc. Am.,
1989. Vol.~86. No.\,4. P.~1404--1415.

\bibitem{Nadaraja2005} % 7
\Au{Nadaraja S.} A~generalized normal distribution~// J.~Appl. 
Stat., 2005. Vol.~32. No.\,7. P.~685--694.

\bibitem{West1987} %8
\Au{West M.} On scale mixtures of normal
distributions~// Biometrika, 1987. Vol.~74. No.\,3. P.~646--648.

\bibitem{ChoySmith1997} % 9
\Au{Choy S.\,T.\,B., Smith~A.\,F.\,F.} Hierarchical models with scale
mixtures of normal distributions~// Test, 1997. Vol.~6. P.~205--221.

\bibitem{Zolotarev1983} % 10
\Au{Золотарев В.\,М.} Одномерные устойчивые распределения.~--- М.: Наука, 1983. 304~с.

\bibitem{KBZZ2012} % 11
\Au{Korolev V.\,Yu., Bening~V.\,E., Zaks~L.\,M., Zeifman~A.\,I.}
Exponential power distributions as asymptotic approximations in
applied probability and statistics~// Applied Problems in Theory of
Probabilities and Mathematical Statistics Related to Modeling of
Information Systems (APTP\;+\;MS'2012): Book of Abstracts of the 6th
 Workshop (International) (Autumn Session).~--- М.: ИПИ РАН, 2012. P.~60--71.

\bibitem{BN1977} % 12
\Au{Barndorff-Nielsen O.\,E.} Exponentially decreasing distributions
for the logarithm of particle size~// Proc. R. Soc. Lond. Ser.
A, 1977. Vol.~353. P.~401--419.

\bibitem{BNKS1982} % 13
\Au{Barndorff-Nielsen O.\,E., Kent~J., S\!\!\!{\ptb{\o}}\,rensen M.} Normal
variance-mean mixtures and $z$-distributions~// Int.
Stat. Rev., 1982. Vol.~50. No.\,2. P.~145--159.

\bibitem{Korolev2012TVP} % 14
\Au{Королев В.\,Ю.} Обобщенные гиперболические распределения как
предельные для случайных сумм~// Теория вероятностей и ее
применения, 2013. Т.~58. Вып.~1. С.~117--132.

\bibitem{ZaksKorolev2013} %15
\Au{Закс Л.\,М., Королев~В.\,Ю.} Обобщенные дисперсионные
гамма-распределения как предельные для случайных сумм~// Информатика
и её применения, 2013. Т.~7. Вып.~1. С.~105--115.

\bibitem{GnedenkoKolmogorov1949} % 16
\Au{Гнеденко Б.\,В., Колмогоров~А.\,Н.} Предельные распределения для
сумм независимых случайных величин.~--- М.--Л.: ГИТТЛ, 1949. 264~с.

\bibitem{GnedenkoKorolev1996} % 17
\Au{Gnedenko B.\,V., Korolev~V.\,Yu.} Random summation: Limit
theorems and applications.~--- Boca Raton: CRC Press, 1996. 275~p.

\bibitem{Korolev1997} % 18
\Au{Королев В.\,Ю.} Постpоение моделей pаспpеделений биpжевых цен
пpи помощи методов асимптотической теоpии случайного суммиpования~// 
Обозpение пpомышленной и пpикладной математики. Сеp.
Финансовая и стpаховая математика, 1997. Т.~4. Вып.~1. С.~86--102.

\bibitem{Korolev2000} % 19
\Au{Королев В.\,Ю.} Асимптотические свойства экстpемумов обобщенных
пpоцессов Кокса и их пpименение к некотоpым задачам финансовой
математики~// Теоpия веpоятностей и ее пpименения, 2000. Т.~45.
Вып.~1. С.~182--194.

\bibitem{BeningKorolev2002} % 20
\Au{Bening V., Korolev~V.} Generalized Poisson models and their
applications in insurance and finance.~--- Utrecht: VSP, 2002. 434~p.

\bibitem{KorolevSokolov2008-1} % 21
\Au{Королев В.\,Ю., Соколов И.\,А.} Математические модели
неоднородных потоков экстремальных событий.~--- М.: ТОРУС ПРЕСС,
2008. 200~с.

\bibitem{KorolevBeningShorgin2011-1} % 22
\Au{Королев В.\,Ю., Бенинг~В.\,Е., Шоргин~С.\,Я.} Математические
основы теории риска.~--- 2-е изд., перераб. и доп.~--- М.: Физматлит,
2011. 620~с.

\bibitem{Korolev2011} % 23
\Au{Королев В.\,Ю.} Ве\-ро\-ят\-ност\-но-ста\-ти\-сти\-че\-ские методы декомпозиции
волатильности хаотических процессов.~--- М.: Изд-во Моск.
ун-та, 2011. 510~с.

\bibitem{KorolevSkvortsova2006} % 24
Stochastic models of
structural plasma turbulence~/
Eds. V.~Korolev, N.~Skvortsova. -- Utrecht: VSP, 2006. 400~p.

\columnbreak

\bibitem{KorolevShevtsovaShorgin2011} % 25
\Au{Королев В.\,Ю., Шевцова~И.\,Г., Шоргин~С.\,Я.} О~неравенствах типа
Бер\-ри--Эс\-се\-ена для пуассоновских случайных сумм~// Информатика и её
применения, 2011. Т.~5. Вып.~3. С.~64--66.

\bibitem{KorolevShevtsova2010} % 26
\Au{Korolev V., Shevtsova~I.} An improvement of the Berry--Esseen
inequality with applications to Poisson and mixed Poisson random
sums~// Scand. Actuar.~J., 2012. No.\,2. P.\,81--105.
Available online since June~4, 2010.
{\sf DOI:10.1080/03461238.2010.485370}.
\end{thebibliography} } }

\end{multicols}

\hfill{\small\textit{Поступила в редакцию 10.01.13}}


\vspace*{6pt}

\hrule

\vspace*{2pt}

\hrule

\def\tit{ON CONVERGENCE OF~THE~DISTRIBUTIONS OF~RANDOM SUMS TO~SKEW EXPONENTIAL POWER LAWS}

\def\aut{M.\,E.~Grigor'eva$^1$ and V.\,Yu.~Korolev$^{2,3}$}

\def\titkol{On convergence of~the~distributions of~random sums to~skew exponential power laws}

\def\autkol{M.\,E.~Grigor'eva and V.\,Yu.~Korolev}


\titel{\tit}{\aut}{\autkol}{\titkol}

\vspace*{-12pt}

\noindent
$^1$Parexel International, Moscow 121609, Russian Federation\\
\noindent
$^2$Faculty of Computational Mathematics and Cybernetics, M.\,V.~Lomonosov Moscow
State University, Moscow\linebreak
$\hphantom{^1}$119991, Russian Federation\\
$^3$Institute of Informatics  Problems, Russian Academy of Sciences,
Moscow 119333, Russian Federation

\def\leftfootline{\small{\textbf{\thepage}
\hfill INFORMATIKA I EE PRIMENENIYA~--- INFORMATICS AND APPLICATIONS\ \ \ 2013\ \ \ volume~7\ \ \ issue\ 4}
}%
 \def\rightfootline{\small{INFORMATIKA I EE PRIMENENIYA~--- INFORMATICS AND APPLICATIONS\ \ \ 2013\ \ \ volume~7\ \ \ issue\ 4
\hfill \textbf{\thepage}}}

\vspace*{8pt}


\Abste{An extension of the class of exponential power distributions 
(also known as generalized Laplace distributions) to the nonsymmetric case is proposed. 
The class of skew exponential power distributions (skew generalized Laplace distributions) 
is introduced as a family of special variance-mean normal mixtures. Expressions for 
the moments of skew exponential power distributions are given. It is demonstrated that 
skew exponential power distributions can be used as asymptotic approximations. For this 
purpose, a theorem is proved establishing necessary and sufficient conditions for the 
convergence of the distributions of sums of a random number of independent identically 
distributed random variables to skew exponential power distributions. Convergence rate 
estimates are presented for a special case of random walks generated by compound doubly 
stochastic Poisson processes.}

\KWE{random sum; generalized Laplace distribution; skew generalized Laplace distribution; 
exponential power distribution; symmetric stable distribution; one-sided stable distribution; 
variance-mean normal mixture; mixed Poisson distribution; mixture of probability distributions; 
identifiable mixtures; additively closed family; convergence rate estimate}

\DOI{10.14357/19922264130407}

\vspace*{-14pt}

\Ack
\noindent
The research was supported by the Russian Foundation for Basic Research (projects 
Nos.\,11-01-00515-а, 11-07-00112-а, and 12-07-00115-а).

  \begin{multicols}{2}

\renewcommand{\bibname}{\protect\rmfamily References}
%\renewcommand{\bibname}{\large\protect\rm References}

{\small\frenchspacing
{%\baselineskip=10.8pt
\addcontentsline{toc}{section}{References}
\begin{thebibliography}{99}


\bibitem{1-ks}
\Aue{Subbotin, M.\,T.} 1923. 
On the law of frequency of error. 
\textit{Mat. Sb.}  31(2):296--301.
\bibitem{2-ks}
\Aue{Box, G., and G.~Tiao}.  1973.
\textit{Bayesian inference in statistical analysis}.
Reading, MA: Addison--Wesley. 608~p.
\bibitem{3-ks}
\Aue{Evans, M., N.~Hastings, and J.\,B.~Peacock}.  2000.
\textit{Statistical distributions}. 3rd. ed.  N.Y.: John Wiley\,\&\,Sons. 170~p.
\bibitem{4-ks}
\Aue{Leemis, L.\,M., and J.\,T.~McQueston}. 2008.
Univariate distribution relationships. \textit{Amer. Stat.} 62(1):45--53.
\bibitem{5-ks}
RiskMetric Group, J.\,P.~Morgan.
1996. {RiskMetrics Technical Document}.  N.Y. 
\columnbreak

\bibitem{6-ks}
\Aue{Varanasi, M.\,K., and B.~Aazhang}. 
1989. Parametric generalized Gaussian density estimation.
\textit{J.~Acoust. Soc. Am.} 86(4):1404--1415.

\vspace*{1pt}

\bibitem{7-ks}
\Aue{Nadaraja, S.} 2005.
A~generalized normal distribution. \textit{J.~Appl. Stat.} 32(7):685--694.

\vspace*{1pt}

\bibitem{8-ks}
\Aue{West, M.} 1987.
On scale mixtures of normal distributions. \textit{Biometrika} 74(3):646--648.

\vspace*{1pt}

\bibitem{9-ks}
\Aue{Choy, S.\,T.\,B., and A.\,F.\,F.~Smith}. 1997. 
Hierarchical models with scale mixtures of normal distributions.
\textit{Test} 6:205--221.

\vspace*{1pt}

\bibitem{10-ks}
\Aue{Zolotarev, V.\,M.} 1983.
\textit{Odnomernye ustoychivye raspredeleniya} 
[\textit{One-dimensional stable distributions}].  Moscow: Nauka. 304~p.

\bibitem{11-ks}
\Aue{Korolev, V.\,Yu., V.\,E.~Bening, L.\,M.~Zaks, and A.\,I.~Zeifman}. 2012.
Exponential power distributions as asymptotic approximations in applied probability 
and statistics. \textit{Applied Problems in Theory of Probabilities and Mathematical 
Statistics Related to Modeling of Information Systems (APTP\;+\;MS'2012). 
Book of abstracts of the 
6th  Workshop (International) (Autumn Session)}. 
Moscow: IPI RAN. 60--71.
\bibitem{12-ks}
\Aue{Barndorff-Nielsen, O.\,E.}  1977.
Exponentially decreasing distributions for the logarithm of particle size.
\textit{Proc. R. Soc. Lond. Ser. A} 353:401--419.
\bibitem{13-ks}
\Aue{Barndorff-Nielsen, O.\,E., J.~Kent, and S\!{\ptb{\o}}rensen M.}  1982.
Normal variance-mean mixtures and $z$-distributions. 
\textit{Int. Stat. Rev.}  50(2):145--159.
\bibitem{14-ks}
\Aue{Korolev, V.\,Yu.} 2013. 
Obobshchennye giperbolicheskie raspredeleniya kak predel'nye dlya sluchaynykh summ 
[Generalized hyperbolic distributions as limit distributions for random sums] 
\textit{Teoriya Veroyatnostey i ee Primeneniya~---
Theory Probab. Appl.} 58(1):117--132.
\bibitem{15-ks}
\Aue{Zaks, L.\,M., and V.\,Yu.~Korolev}. 2013. 
Obobshchennye dispersionnye gamma-raspredeleniya kak predel'nye dlya sluchaynykh summ 
[Variance-generalized-gamma-distributions as limit laws for random sums]. 
\textit{Informatika i ee Primeneniya~--- Inform. Appl.} 7(1):105--115.
\bibitem{16-ks}
\Aue{Gnedenko, B.\,V., and A.\,N.~Kolmogorov}. 1949. 
\textit{Predel'nye raspredeleniya dlya summ nezavisimykh sluchaynykh velichin} 
[\textit{Limit distributions for sums of independent random variables}]. Moscow--Leningrad: GITTL.
264~p.
\bibitem{17-ks}
\Aue{Gnedenko, B.\,V., and V.\,Yu.~Korolev}.  1996.
Random summation: Limit theorems and applications. Boca Raton: CRC Press. 275~p.
\bibitem{18-ks}
\Aue{Korolev, V.\,Yu.} 1997. 
Postroenie modeley raspredeleniy birzhevykh tsen pri pomoshchi metodov 
asimptoti\-che\-skoy teorii sluchaynogo summirovaniya [Construction of models 
for stock prices by methods of the asymptotic theory of random summation]. 
\textit{Obozrenie Promyshlennoy i Prikladnoy Matematiki} 
[\textit{Surveys in Applied and Industrial Mathematics}] 4(1):86--102.
\columnbreak

\bibitem{19-ks}
\Aue{Korolev, V.\,Yu.} 2000. 
Asimptoticheskie svoystva ekstremumov obobshchennykh protsessov Koksa 
i ikh primenenie k nekotorym zadacham finansovoy matematiki 
[Asymptotic properties of extrema of compound Cox processes and 
their applications to some problems of financial mathematics]. 
\textit{Teoriya Veroyatnostey i ee Primeneniya~--- Theory Probab. Appl.} 45(1):182--194.
\bibitem{20-ks}
\Aue{Bening, V., and V.~Korolev}.  2002.
\textit{Generalized Poisson models and their applications in insurance and finance}. 
Utrecht: VSP. 434~p.
\bibitem{21-ks}
\Aue{Korolev, V.\,Yu., and I.\,A.~Sokolov}. 2008. 
\textit{Matema\-ti\-che\-skie modeli neodnorodnykh potokov ekstremal'nykh sobytiy} 
[\textit{Mathematical models of nonhomogeneous flows of extremal events}]. Moscow: TORUS PRESS.
200~p.
\bibitem{22-ks}
\Aue{Korolev, V.\,Yu., V.\,E.~Bening, and S.\,Ya.~Shorgin}. 
2011. \textit{Ma\-te\-ma\-ti\-che\-skie osnovy teorii riska} 
[\textit{Mathematical foundations of risk theory}]. 2nd ed. Moscow: Fizmatlit. 620~p.
\bibitem{23-ks}
\Aue{Korolev, V.\,Yu.} 2011. 
\textit{Veroyatnostno-statisticheskie metody dekompozitsii volatil'nosti khaoticheskikh processov} 
[\textit{Probabilistic and statistical methods for the decomposition of the volatility of chaotic 
processes}]. Moscow: Moscow University Press. 510~p.
\bibitem{24-ks}
Korolev, V., and N.~Skvortsova, eds.  2006.
\textit{Stochastic models of structural plasma turbulence}. Utrecht: VSP. 400~p.
\bibitem{25-ks}
\Aue{Korolev, V.\,Yu., I.\,G.~Shevtsova, and S.\,Ya.~Shorgin}. 2011. 
O~neravenstvakh tipa Berri--Esseena dlya puassonovskikh sluchaynykh summ 
[On the Berry--Esseen-type inequalities for Poisson random sums]. 
\textit{Informatika i ee Primeneniya~--- Inform. Appl.} 5(3):64--66.


 
\bibitem{26-ks}
\Aue{Korolev, V., and I.~Shevtsova}.  2012.
An improvement of the Berry--Esseen inequality with applications to Poisson and 
mixed Poisson random sums. \textit{Scand. Actuar.~J.} 2:81--105.
Available online since June~4, 2010.
{\sf DOI:10.1080/03461238.2010.485370}.
\end{thebibliography}
} }



\end{multicols}

\hfill{\small\textit{Received January 10, 2013}}

\Contr

\noindent
\textbf{Grigorieva Maria E.} (b.\ 1986)~--- biostatistician II, 
Parexel International, Moscow 121609, Russian Federation; maria-grigoryeva@yandex.ru

\vspace*{3pt}

\noindent
\textbf{Korolev Victor Yu.} (b.\ 1954)~--- Doctor of Science in 
physics and mathematics, professor, Department of Mathematical Statistics, 
Faculty of Computational Mathematics and Cybernetics, M.\,V.~Lomonosov
 Moscow State University; Moscow 119991, Russian Federation;
 leading scientist, Institute of Informatics Problems, Russian 
Academy of Sciences, Moscow 119333, Russian Federation; victoryukorolev@yandex.ru

 \label{end\stat}
\renewcommand{\bibname}{\protect\rm Литература}  