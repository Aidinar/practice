\documentclass[10pt]{book}
\usepackage[utf8]{inputenc}

\usepackage{latexsym,amssymb,amsfonts,amsmath,indentfirst,shapepar,%fleqn,%
picinpar,shadow,floatflt,enumerate,multicol,colortbl,ipi}

\usepackage{rotating}
\usepackage{mathrsfs}
\usepackage[noend]{algorithmic}
\usepackage{ulem}

\input{epsf}

\nofiles

%\includeonly{dolev,avtor,avtor-eng} %+pdf
%\includeonly{obchak,avtor}
%\includeonly{pred}      %
%\includeonly{podgot-rus,podgot-eng}  %+pdf
%\includeonly{ocherk} %+


%\includeonly{tirsin}  %1+авт+pdf+++
%\includeonly{korol-sok} %2+pdf+авт++++ 
%\includeonly{pechinkin} %3+авт+pdf+++
%\includeonly{konovalov} %+pdf %4+авт++++
%\includeonly{arhipov} %+pdf %5+авт++++
%\includeonly{zatsman} %+ pdf %6+авт+++++
%\includeonly{korolev}  %+pdf %7+авт+++++
%\includeonly{shestakov} %pdf  %8+авт++++
%\includeonly{suchkov} %+pdf %9+авт++++++
%\includeonly{krivenko} %+pdf   %10+авт+++++
%\includeonly{zaharova} %+pdf %11+авт++++нетзамечаний
%\includeonly{kalinichenko}   %12+авт+pdf
%\includeonly{dolev}  %+pdf %13+авт++++



%\includeonly{toc-rus, toc-en}
%\includeonly{obchak} %,toc-en}

%\includeonly{rekl}
%\includeonly{rekl-1}
%\includeonly{reshal}  %
%\includeonly{eng-index}
%\includeonly{cover3}

\usepackage{acad}
%\usepackage{courier}
\usepackage{decor}
\usepackage{newton}
\usepackage{pragmatica}
\usepackage{zapfchan}
\usepackage{petrotex}
\usepackage{bm}                     % полужирные греческие буквы
\usepackage{upgreek}                % прямые греческие буквы
\usepackage{eufrak}
\usepackage{verbatim}

\renewcommand{\bottomfraction}{0.99}
\renewcommand{\topfraction}{0.99}
\renewcommand{\textfraction}{0.01}

\setcounter{secnumdepth}{1} %здесь - 3 + chapter = 4

\arraycolsep=1.5pt

%\usepackage[pdftex]{graphicx}

%\usepackage{oz}

%NEW COMMANDS


\renewcommand*{\hm}[1]{#1\nobreak\discretionary{}%
            {\hbox{$\mathsurround=0pt #1$}}{}} %% Дублирует знаки операций
                               %при переносе в формуле (перед знаком, который 
                               %надо продублировать ставится команда \hm)

%\newcommand{\endproof}{\hfill$\Box$}
\renewcommand{\r}{\mathbb{R}}
\newcommand{\I}{{\rm I\hspace{-0.7mm}I}}
%\newcommand{\Ikl}{{\tt{1}}\hspace*{-1.44mm}\mathtt{1}}
\newcommand{\Ik}{\mbox{{\small \tt {1}}\hspace{-1.5mm}{\tt 1}}}
\newcommand{\argmin}{\mathop{\mathrm{arg}\,\mathrm{min}}}
\newcommand{\argmax}{\mathop{\mathrm{arg}\,\mathrm{max}}}
%\newcommand{\capr}{\mathop{\cap\,}}
%\newcommand{\cupr}{\mathop{\cup\,}}
%\def\argmin{\mathop{arg\,min}}

\def\vrp{\varphi}
\def\prt{\partial}
\def\mm{{\rm M}}

\newcommand{\il}[2]{\int\limits_{#1}^{#2}}%интеграл с пределами #1 и #2

\def\ss2{\mathop {\sum\limits^p\sum\limits^p}}
\def\sss{\sum\limits}
\def\tr{,\,\ldots\,,\,}
\def\rk{\right]}
\def\lk{\left[}
\def\rf{\right\}}
\def\lf{\left\{}
\def\lv{\,\left\vert}
\def\rv{\right\vert\,}


\def\ee{{\cal E}}
\def\ww{{\cal W}}
\def\yy{{\cal Y}}
\def\vv{{\cal V}}

\newcommand{\R}{\mathbb R}
\newcommand{\E}{\mathbb E}
\newcommand{\N}{\mathbb N}

\newcommand{\h}{{\bf H}}
\newcommand{\p}{{\sf P}}  % вероятность

\newcommand{\e}{{\sf E}}  % мат. ожидание
\newcommand{\D}{{\sf D}}  % дисперсия
\newcommand{\eps}{\varepsilon}
\newcommand{\vp}{{\mathbf p}}
\newcommand{\vz}{{\mathbf z}}
\newcommand{\vx}{{\mathbf x}}
\newcommand{\vf}{{\mathbf f}}
\newcommand{\F}{{\mathcal F}}
\def\ap{{\mathrm{ЭР}}}
\newcommand{\ud}{\Delta_n} %uniform ditance
\newcommand{\nud}{\Delta_n(x)}

\newcommand{\abs}[1]{\left\vert#1\right\vert}
\newcommand{\norm}[1]{\left\Vert#1\right\Vert}
\def\da{(\Delta_t,A)}

\def\w{\omega}
\def\W{\Omega}
\def\iii{\int\limits}
\def\inh{\int\limits_{nh}^{(n+1)h}}
\def\iin{\int\limits_{-\infty}^\infty}
\def\sumin{\sum_{i=1}^N}


\def\bxt{(Y,t)}
\def\xt{(y,t)}

\def\ovth{{\fr{\tau-nh}{h}}}


\DeclareMathOperator{\sign}{sign}

%\newcommand{\gr}{{\geqslant}}


\newcommand{\g}{\mbox{\textit{g}}}

\renewcommand{\la}{\lambda}
\newcommand{\si}{\sigma}
\newcommand{\alp}{\alpha}

%\newcommand{\pto}{\stackrel{P}{\longrightarrow}} % сходимость по веpоятности

\newcommand{\eqd}{\stackrel{\mathrm{d}}{=}} % равенство по pаспpеделению

%\newcommand{\kp}{\kappa}
%\def\Q{{\cal Q}} \def\H{{\cal H}}
%\newcommand{\bet}{\beta_{2+\delta}}


%\newtheorem{definition}{Определение}
%\renewcommand{\thedefinition}{\arabic{definition}.}
%END NEW COMMANDS

%\renewcommand{\baselinestretch}{1.2}

%\pagestyle{myheadings}

\setlength{\textwidth}{167mm}      % 122mm
\setlength{\textheight}{658pt}
%\setlength{\textheight}{635.6pt}
\setlength{\columnsep}{4.5mm}

\setcounter{secnumdepth}{4}

%\addtolength{\headheight}{2pt}
%\addtolength{\headsep}{-2mm}

%\addtolength{\topmargin}{-20mm}  % for printing


%\hoffset=-30mm  % From Yap
\hoffset=-23mm  % From Acrobat

%\voffset=0mm % From Yap
%\voffset=-15mm   % From Acrobat

\addtolength{\evensidemargin}{-9.5mm} % for printing
\addtolength{\oddsidemargin}{9.5mm}  % for printing

%\renewcommand{\thefootnote}{\fnsymbol{footnote}}
%\renewcommand{\thefootnote}{\arabic{footnote}}
\renewcommand{\figurename}{\protect\bf Рис.}
\renewcommand{\tablename}{\protect\bf Таблица}

\newcommand{\Caption}[1]{\caption{\protect\small %\baselineskip=2.5ex
#1}}

\renewcommand{\thefigure}{\arabic{figure}}
\renewcommand{\thetable}{\arabic{table}}
\renewcommand{\theequation}{\arabic{equation}}
\renewcommand{\thesection}{\arabic{section}}

\renewcommand{\contentsname}{СОДЕРЖАНИЕ}
\newcommand{\fr}[2]{\displaystyle\frac{\displaystyle #1\mathstrut}{\displaystyle #2\mathstrut}}

%\renewcommand{\thefootnote}{\fnsymbol{footnote}}
%\newcommand{\g}{\mbox{\textit{g}}}

%\newcommand{\Caption}[1]{\caption{\protect\small\baselineskip=2ex #1}}
\newcounter{razdel}
\setcounter{razdel}{0}


\newcommand{\titel}[4]{%
\

\vspace*{5pt}

\ifodd\therazdel {\raggedright\noindent\Large\textrm\textbf
 \lineskip .75em
  \baselineskip=3.2ex #1 \par}
\vskip 1em {\noindent\large\textrm\textbf #2 \par}
\addcontentsline{toc}{subsection}{{\textrm\textbf #3}\protect\newline #1}
\def\rightheadline{\underline{\noindent\hbox to \textwidth{\hfill\small\textrm{#4}
%\hfill \large\bf\thepage
}}}
\def\leftheadline{\underline{\noindent\parbox{\textwidth}{
%\raggedleft\large\bf\thepage \hfill
\small\textit{#3}\hfill}}}
\def\leftfootline{\small{\textbf{\thepage}
\hfill ИНФОРМАТИКА И ЕЁ ПРИМЕНЕНИЯ\ \ \ том~7\ \ \ выпуск 4\ \ \ 2013}
}%
 \def\rightfootline{\small{ИНФОРМАТИКА И ЕЁ ПРИМЕНЕНИЯ\ \ \ том~7\ \ \ выпуск~4\ \ \ 2013
\hfill \textbf{\thepage}}} 
\vskip 2em \setcounter{figure}{0}
\setcounter{table}{0} 
\setcounter{equation}{0} 
\setcounter{section}{0}
\setcounter{subsection}{0} 
\setcounter{subsubsection}{0}
\setcounter{footnote}{0} 
\setcounter{razdel}{0}
%\end{flushleft}
\else {
 \raggedright\noindent\Large\textrm\textbf
 \lineskip .75em
\baselineskip=3.2ex #1 \par} \vskip 1em
%\begin{flushleft}
{\noindent\large\textrm\textbf #2 \par}
\addcontentsline{toc}{subsection}{{\textrm\textbf #3}\protect\newline #1}
\def\rightheadline{\underline{\noindent\hbox to \textwidth{\hfill\small\textrm{#4}
%\hfill \large\bf\thepage
}}}
\def\leftheadline{\underline{\noindent\parbox{\textwidth}{%\raggedleft\large\bf\thepage \hfill
\small\textit{#3}\hfill}}}
\def\leftfootline{\small{\textbf{\thepage}
\hfill ИНФОРМАТИКА И ЕЁ ПРИМЕНЕНИЯ\ \ \ том~7\ \ \ выпуск~4\ \ \ 2013}
}%
 \def\rightfootline{\small{ИНФОРМАТИКА И ЕЁ ПРИМЕНЕНИЯ\ \ \ том~7\ \ \ выпуск~4\ \ \ 2013
\hfill \textbf{\thepage}}} \vskip 2em \setcounter{figure}{0}
\setcounter{table}{0} \setcounter{equation}{0} \setcounter{section}{0}
\setcounter{subsection}{0} \setcounter{subsubsection}{0}
\setcounter{footnote}{0}
%\end{flushleft}
\fi}

\newcommand{\titelr}[2]{%
\

\vspace*{5pt}

\ifodd\therazdel {\raggedright\noindent%\Large\textrm\textbf
 \lineskip .75em
  \baselineskip=3.2ex #1 \par}
\vskip 1em {\noindent\normalsize\textrm\textbf #2 \par}
\else {
 \raggedright\noindent\Large\textrm\textbf
 \lineskip .75em
\baselineskip=3.2ex #1 \par} \vskip 1em
%\begin{flushleft}
{\noindent\large\textrm\textbf #2 \par
%\noindent\normalsize\textrm\textbf #2 \par
} \fi}

\newcommand{\titele}[5]{%
\

%\vspace*{5pt}

\ifodd\therazdel {\raggedright\noindent\large
\textrm\textbf
 \lineskip .75em
%  \baselineskip=3.2ex
#1 \par}
\vskip .5em {\noindent\large\textrm\textbf #2 \par}
\vskip .5em
 {\noindent\textrm #3 \par}
\addcontentsline{toc}{subsection}{{\textrm\textbf #1}\protect\newline #2}
\def\rightheadline{\underline{\noindent\hbox to \textwidth{\hfill\small\textrm{#4}
%\hfill \large\bf\thepage
}}}
\def\leftheadline{\underline{\noindent\parbox{\textwidth}{
%\raggedleft\large\bf\thepage \hfill
\small\textrm{#5}\hfill}}}
\def\leftfootline{\small{\textbf{\thepage}
\hfill ИНФОРМАТИКА И ЕЁ ПРИМЕНЕНИЯ\ \ \ том~7\ \ \ выпуск~4\ \ \ 2013}
}%
 \def\rightfootline{\small{ИНФОРМАТИКА И ЕЁ ПРИМЕНЕНИЯ\ \ \ том~7\ \ \ выпуск~4\ \ \ 2013
\hfill \textbf{\thepage}}} \vskip 1em \setcounter{figure}{0}
\setcounter{table}{0} \setcounter{equation}{0} \setcounter{section}{0}
\setcounter{subsection}{0} \setcounter{subsubsection}{0}
\setcounter{footnote}{0} \setcounter{razdel}{0}
%\end{flushleft}
\else {
 \raggedright\noindent\large
 \textrm\textbf
 \lineskip .75em
%\baselineskip=3.2ex
#1 \par} \vskip .5em
%\begin{flushleft}
{\noindent\large\textrm\textbf #2 \par} \vskip .5em
 {\noindent\textrm #3 \par}
\addcontentsline{toc}{subsection}{{\textrm\textbf #1}\protect\newline #2}
\def\rightheadline{\underline{\noindent\hbox to \textwidth{\hfill\small\textrm{#4}
%\hfill \large\bf\thepage
}}}
\def\leftheadline{\underline{\noindent\parbox{\textwidth}{%\raggedleft\large\bf\thepage \hfill
\small\textrm{#5}\hfill}}}
\def\leftfootline{\small{\textbf{\thepage}
\hfill ИНФОРМАТИКА И ЕЁ ПРИМЕНЕНИЯ\ \ \ том~7\ \ \ выпуск~4\ \ \ 2013}
}%
 \def\rightfootline{\small{ИНФОРМАТИКА И ЕЁ ПРИМЕНЕНИЯ\ \ \ том~7\ \ \ выпуск~4\ \ \ 2013
\hfill \textbf{\thepage}}} \vskip 1em \setcounter{figure}{0}
\setcounter{table}{0} \setcounter{equation}{0} \setcounter{section}{0}
\setcounter{subsection}{0} \setcounter{subsubsection}{0}
\setcounter{footnote}{0}
%\end{flushleft}
\fi}

\def\Abst#1{
\begin{center}\small\nwt
\parbox{150mm}{%\baselineskip=2.5ex
\textbf{Аннотация:}\ \
%\hspace*{\parindent}
#1}
\end{center}}
\def\Abste#1{
\begin{center}\small\nwt
\parbox{150mm}{%\baselineskip=2.5ex
\textbf{Abstract:}\ \
%\hspace*{\parindent}
#1}
\end{center}}

\def\DOI#1{
\begin{center}\small\nwt
\parbox{150mm}{%\baselineskip=2.5ex
\textbf{DOI:}\ \
%\hspace*{\parindent}
#1}
\end{center}}

\def\Abstend#1{
\begin{center}\small\nwt
\parbox{150mm}{%\baselineskip=2.5ex
%\hspace*{\parindent}
#1}
\end{center}}


\def\KW#1{
\begin{center}\small\nwt
\parbox{150mm}{%\baselineskip=2.5ex
\textbf{Ключевые слова:}\ \ #1}
\end{center}}

\def\KWE#1{
\begin{center}\small\nwt
\parbox{150mm}{%\baselineskip=2.5ex
\textbf{Keywords:}\ \ #1}
\end{center}}


\def\KWN#1{
%\begin{center}
%\small
%\parbox{150mm}\end{center}
}

\renewcommand{\thesubsection}{\thesection.\arabic{subsection}\hspace*{-5pt}}
\renewcommand{\thesubsubsection}{\thesubsection\hspace*{5pt}.\arabic{subsubsection}\hspace*{-3pt}}

\newcommand{\Ack}{\section*{\protect\rmfamily Acknowledgments}\noindent}
\newcommand{\Contr}{\section*{\protect\rmfamily Contributors}\noindent}


\begin{document}
\Rus

\nwt
%\ptb

%\renewcommand{\contentsname}{\protect\Large\bf Содержание}

\setcounter{tocdepth}{2}

%\tableofcontents

\renewcommand{\bibname}{\protect\rmfamily Литература}
  \def\Au#1{{\it #1}}
    \def\Aue#1{{#1}}

%\newcommand{\No}{№}
  \newcommand{\tg}{\,\mathrm{tg}\,}
    \newcommand{\ctg}{\,\mathrm{ctg}\,}
  \newcommand{\arctg}{\,\mathrm{arctg}\,}
  
\def\forallb{\mathop{\forall}}
\def\cupb{\mathop{\cup}}
\def\existsb{\mathop{\exists}}

\setcounter{page}{1}

\newpage
\addtocounter{razdel}{1}
%\def\razd{РЕГУЛИРУЕМЫЙ ЭЛЕКТРОПРИВОД ДЛЯ ЭЛЕКТРОЭНЕРГЕТИКИ}


\setcounter{page}{3}


%   { %\Large  
   { %\baselineskip=16.6pt
   
   \vspace*{-48pt}
   \begin{center}\LARGE
   \textit{Предисловие}
   \end{center}
   
   %\vspace*{2.5mm}
   
   \vspace*{25mm}
   
   \thispagestyle{empty}
   
   { %\small 

    
Вниманию читателей журнала <<Информатика и её применения>> предлагается 
очередной тематический выпуск <<Вероятностно-статистические методы и 
задачи информатики и информационных технологий>>. Предыдущие тематические 
выпуски журнала по данному направлению вышли в 2008~г.\ (т.~2, вып.~2), 
в 2009~г.\ (т.~3, вып.~3) и в 2010~г.\ (т.~4, вып.~2). 

Статьи, собранные в данном журнале, посвящены разработке новых вероятностно-статистических 
методов, ориентированных на применение к решению конкретных задач информатики и информационных 
технологий, а также~--- в ряде случаев~--- и других прикладных задач. Проблематика, охватываемая 
публикуемыми работами, развивается в рамках научного сотрудничества между Институтом проблем 
информатики Российской академии наук (ИПИ РАН) и Факультетом вычислительной математики и 
кибернетики Московского государственного университета им.\ М.\,В.~Ломоносова в ходе работ 
над совместными научными проектами (в том числе в рамках функционирования 
Научно-образовательного центра <<Вероятностно-статистические методы анализа рисков>>). 
Многие из авторов статей, включенных в данный номер журнала, являются активными участниками 
традиционного международного семинара по проблемам устойчивости стохастических моделей, 
руководимого В.\,М.~Золотаревым и В.\,Ю.~Королевым; регулярные сессии этого семинара 
проводятся под эгидой МГУ и ИПИ РАН (в 2011~г.\ указанный семинар проводится в октябре 
в Калининградской области РФ). 

Наряду с представителями ИПИ РАН и МГУ в число авторов данного выпуска журнала входят 
ученые из Научно-исследовательского института системных исследований РАН, Института 
проблем технологии микроэлектроники и особочистых материалов РАН, Института 
прикладных математических исследований Карельского НЦ РАН, Московского 
авиационного института, Вологодского государственного педагогического университета, 
НИИММ им.\ Н.\,Г.~Чеботарева, Казанского государственного университета, Дебреценского 
университета (Венгрия).

Несколько статей выпуска посвящено разработке и применению стохастических методов и 
информационных технологий для решения различных прикладных задач. В~работе В.\,Г.~Ушакова 
и О.\,В.~Шестакова рассмотрена задача определения вероятностных характеристик случайных 
функций по распределениям интегральных преобразований, возникающих в задачах эмиссионной 
томографии. В~статье Д.\,О.~Яковенко и М.\,А.~Целищева рассмотрены некоторые вопросы 
математической теории риска и предложен новый подход к диверсификации инвестиционных 
портфелей. Работа И.\,А.~Кудрявцевой и А.\,В.~Пантелеева посвящена построению и 
исследованию математической модели, описывающей динамику сильноионизованной плазмы. 
В~статье П.\,П.~Кольцова изучается качество работы ряда алгоритмов сегментации изображений. 
Статья А.\,Н.~Чупрунова и И.~Фазекаша посвящена вероятностному анализу числа без\-оши\-бочных 
блоков при помехоустойчивом кодировании; получены усиленные законы больших чисел для указанных 
величин.

В данном выпуске традиционно присутствует тематика, весьма активно разрабатываемая в течение 
многих лет специалистами ИПИ РАН и МГУ,~--- методы моделирования и управления для 
информационно-телекоммуникационных и вычислительных систем, в частности методы 
теории массового обслуживания. В~статье А.\,И.~Зейфмана с соавторами рассматриваются 
модели обслуживания, описываемые марковскими цепями с непрерывным временем в случае 
наличия катастроф. В~работе М.\,М.~Лери и И.\,А.~Чеплюковой рассматриваются случайные 
графы Интернет-типа, т.\,е.\ графы, степени вершин которых имеют степенные распределения; 
такие задачи находят применение при исследовании глобальных сетей передачи данных. 
Работа Р.\,В.~Разумчика посвящена исследованию систем массового обслуживания специального 
вида~--- с отрицательными заявками и хранением вытесненных заявок.

Ряд статей посвящен развитию перспективных теоретических 
вероятностно-статистических методов, которые находят широкое применение в различных 
задачах информатики и информационных технологий. В~работе В.\,Е.~Бенинга, А.\,К.~Горшенина 
и В.\,Ю.~Королева рассмотрена задача статистической проверки гипотез о числе компонент 
смеси вероятностных распределений, приводится конструкция асимптотически наиболее мощного 
критерия. Результаты этой работы найдут применение в ряде прикладных задач, использующих 
математическую модель смеси вероятностных распределений (в информатике, моделировании 
финансовых рынков, физике турбулентной плазмы и~т.\,д.). В~статье В.\,Ю.~Королева, 
И.\,Г.~Шевцовой и С.\,Я.~Шоргина строится новая, улучшенная оценка точности нормальной 
аппроксимации для пуассоновских случайных сумм; как известно, указанные случайные суммы 
широко используются в качестве моделей многих реальных объектов, в том числе в информатике, 
физике и других прикладных областях. Работа В.\,Г.~Ушакова и Н.\,Г.~Ушакова посвящена 
исследованию ядерной оценки плотности распределения; эти результаты могут применяться, 
в част\-ности, при анализе трафика в телекоммуникационных системах. Серьезные приложения 
в статистике могут получить результаты работы О.\,В.~Шестакова, в которой доказаны оценки 
скорости сходимости распределения выборочного абсолютного медианного отклонения к нормальному 
закону. 

\smallskip

Редакционная коллегия журнала выражает надежду, что данный тематический  выпуск 
будет интересен специалистам в области теории вероятностей и математической статистики 
и их применения к решению задач информатики и информационных технологий.
     
     %\vfill 
     \vspace*{20mm}
     \noindent
     Заместитель главного редактора журнала <<Информатика и её 
применения>>,\\
     директор ИПИ РАН, академик  \hfill
     \textit{И.\,А.~Соколов}\\
     
     \noindent
     Редактор-составитель тематического выпуска,\\
     профессор кафедры математической статистики факультета\\
      вычислительной математики и кибернетики МГУ им.\ М.\,В.~Ломоносова,\\
     ведущий научный сотрудник ИПИ РАН,\\ 
доктор физико-математических наук \hfill
      \textit{В.\,Ю.~Королев}
     
     } }
     }


\def\stat{tirsin}

\def\tit{ИССЛЕДОВАНИЕ ДИНАМИКИ МНОГОМЕРНЫХ СТОХАСТИЧЕСКИХ 
СИСТЕМ НА~ОСНОВЕ ЭНТРОПИЙНОГО~МОДЕЛИРОВАНИЯ$^*$}

\def\titkol{Исследование динамики многомерных стохастических 
систем на~основе энтропийного моделирования}

\def\autkol{А.\,Н.~Тырсин, О.\,В.~Ворфоломеева}

\def\aut{А.\,Н.~Тырсин$^1$, О.\,В.~Ворфоломеева$^2$}

\titel{\tit}{\aut}{\autkol}{\titkol}

{\renewcommand{\thefootnote}{\fnsymbol{footnote}}
\footnotetext[1] {Работа выполнена при поддержке 
проекта 12-М-127-2049 фундаментальных исследований УрО РАН.}}

\renewcommand{\thefootnote}{\arabic{footnote}}
\footnotetext[1]{Научно-инженерный центр <<Надежность и ресурс больших систем и машин>> УрО РАН, 
г.~Екатеринбург, at2001@yandex.ru}
\footnotetext[2]{Челябинский государственный университет, ya.olga.work@yandex.ru}

     
  
  \Abst{Описан энтропийный подход к моделированию динамики 
стохастических систем. В его основе лежит представление системы в виде 
многомерного случайного вектора. Показано, что изменение энтропии 
многомерной стохастической сис\-те\-мы может быть выражено через дисперсии и 
условные корреляции компонент случайного вектора. Это позволяет 
обнаружить причину изменения энтропии сис\-те\-мы и оценить этот случайный вектор
количественно. Получено, что энтропия стохастической системы складывается 
из двух компонент, которые характеризуют ее свойства. Первая компонента 
определяет предельную энтропию, соответствующую полной независимости 
элементов системы, и характеризует рассмотрение целостного объекта как 
состоящего из частей (аддитивность). Вторая компонента отражает степень 
взаимосвязей между элементами сис\-те\-мы, характеризуя свойства системы как 
целого (целостность). Описанный подход делает возможным использование 
энтропийной модели в задачах диагностики и контроля состояния 
стохастических сис\-тем, а также эффективного управления ими. 
К~достоинствам предложенного подхода следует отнести простоту реализации 
и интерпретации математической модели, универсальность и применимость к 
стохастическим сис\-те\-мам различной природы, возможность ее использования 
на малых выборках данных. Приведен пример практического 
применения математической модели.}
  
  \KW{многомерная случайная величина; энтропия; динамика; стохастическая 
система; дисперсия; корреляция}

\DOI{10.14357/19922264130401}

\vskip 14pt plus 9pt minus 6pt

      \thispagestyle{headings}

      \begin{multicols}{2}

            \label{st\stat}
  
  \section{Введение}
  
  Энтропия является фундаментальным свойством любых систем с 
неоднозначным, или вероятностным, поведением~[1]. В~настоящее время 
достаточно распространено использование энтропии для описания поведения 
открытых стохастических систем в различных областях~[2--6]. Однако, 
несмотря на частое использование этого термина, использование энтропии для 
моделирования открытых систем, в отличие от термодинамики, недостаточно 
формализовано и носит в основном качественный характер. Отсутствуют 
достаточно простые и адекватные математические модели, позволяющие 
связать энтропию с фактическими характеристиками состояний стохастических 
систем.
  
  Известно~[7], что энтропия непрерывной случайной величины~$X$ 
(дифференциальная энтропия) определяется по формуле:
  \begin{equation}
  H(X) =-\int\limits_{-\infty}^{+\infty} f(x)\ln f(x)\,dx\,,
  \label{e1-t}
  \end{equation}
  где $f(x)$~--- плотность распределения случайной величины~$X$. 
Полученная по формуле~(1) энтропия называется энтропией закона 
распределения или дифференциальной энтропией.
  
  Представим стохастическую систему~$S$ в виде многомерной случайной 
величины $\mathbf{Y}\hm=(Y_1, Y_2,\ldots , Y_m)$. Будем считать, что данное 
представление является адекватной математической\linebreak моделью системы~$S$. 
Каждый элемент~$Y_i$ вектора~\textbf{Y} является одномерной случайной 
величиной, которая характеризует функционирование соответствующего 
элемента исследуемой системы. Элементы могут быть как взаимозависимыми, 
так и не зависеть друг от друга. Совместную дифференциальную энтропию 
многомерной случайной величины~$\mathbf{Y}$ будем определять по 
формуле~[7]:
  \begin{multline}
  H(\mathbf{Y})=-\int\limits_{-\infty}^{+\infty} \cdots \int\limits_{-
\infty}^{+\infty} f_{\mathbf{Y}}(x_1,x_2,\ldots ,x_m)\times{}\\
{}\times \ln  f_{\mathbf{Y}}(x_1,x_2,\ldots ,x_m)\,dx_1dx_2\ldots dx_m\,.
  \label{e2-t}
  \end{multline}
  где $ f_{\mathbf{Y}}(x_1,x_2,\ldots ,x_m)$~--- совместная плотность 
распределения случайных величин $Y_1, Y_2, \ldots , Y_m$.
  
  Аналитическое нахождение энтропии $H(\mathbf{Y})$ в настоящее время 
получено лишь для совместного нормального распределения~\cite{8-t}. 
Рассмотрение других распределений затруднено отсутствием меры нелинейной 
корреляционной взаимосвязи случайных величин с иными распределениями, 
аналогичной определителю корреляционной матрицы для совместного 
нормального распределения. В~\cite{9-t} предпринята попытка оценить 
взаимосвязь случайных величин через совместную энтропию. Но при этом 
требуется вычислить саму энтропию многомерной случайной величины по 
ограниченной выборке, что весьма затруднительно, особенно если законы 
распределения не известны. А~необходимо, наоборот, выразить совместную 
энтропию многомерной случайной величины через характеристики ее 
компонент.
  
  Таким образом, актуальна задача разработки и формального обоснования 
энтропийного подхода к моделированию открытых стохастических систем. 
Поэтому рассмотрим более общий случай, когда случайный вектор~\textbf{Y} 
не имеет многомерного нормального распределения.
  
  \section{Энтропия многомерной непрерывной случайной 
величины}
  
  \noindent
  \textbf{Теорема~1.} \textit{Пусть $X_{1}$, $X_2$~--- две непрерывные 
случайные величины, определенные на всей числовой оси и описываемые 
однотипными законами распределения с плотностями $f_1(x) \hm= 
f(x;\mu_1,\lambda_1)$, $f_2(x)\hm= f(x; \mu_2,\lambda_2)$ соответственно, где 
$\mu_1$, $\mu_2$ и $\lambda_1$, $\lambda_2$~--- параметры положения и 
масштаба случайных \mbox{величин}~$X_1$ и~$X_2$. Тогда разность 
дифференциальных энтропий случайных величин~$X_1$ и~$X_2$ равна}:
  \begin{equation}
  H(X_2) -H(X_1) =\ln \fr{\lambda_2}{\lambda_1}\,.
  \label{e3-t}
  \end{equation}
  
  \noindent
  Д\,о\,к\,а\,з\,а\,т\,е\,л\,ь\,с\,т\,в\,о\,.\ \  Выразим плотность вероятности 
случайной величины~$X_2$ через плотность вероятности случайной 
величины~$X_1$:
  $$
  f(x;\mu_2,\lambda_2) =\fr{\lambda_1}{\lambda_2}\,f\left( 
\fr{\lambda_1}{\lambda_2}\left( x+\mu_2-\mu_1\right);\mu_1,\lambda_1\right)\,.
  $$
  
  С учетом последнего соотношения разность дифференциальных энтропий 
случайных величин~$X_1$ и~$X_2$ равна:

\noindent
  \begin{multline*}
  H(X_2) -H(X_1) = {}\\
  {}= -\int\limits_{-\infty}^{+\infty} f(x;\mu_2,\lambda_2)\ln 
f(x;\mu_2,\lambda_2)\,dx
  +{}\\
  {}+ \int\limits_{-\infty}^{+\infty} f(x;\mu_1,\lambda_1)\ln 
f(x;\mu_1,\lambda_1)\,dx={}\\
  {}= -\fr{\lambda_1}{\lambda_2} \int\limits_{-\infty}^{+\infty} f\left( 
\fr{\lambda_1}{\lambda_2}\left( x+\mu_2-\mu_1\right);\mu_1,\lambda_1\right) \times{}\\
{}\times\ln 
\left[ \fr{\lambda_1}{\lambda_2}f \left( \fr{\lambda_1}{\lambda_2}\left( x+\mu_2-\mu_1\right);\mu_1,\lambda_1 
\right)\right]\,dx+{}\\
  {}+ \int\limits_{-\infty}^{+\infty} f(x;\mu_1,\lambda_1)\ln 
f(x;\mu_1,\lambda_1)\,dx= {}\\
{}=-\ln \fr{\lambda_1}{\lambda_2} \int\limits_{-
\infty}^{+\infty} f(x;\mu_2,\lambda_2)\,dx-{}\\
  {}- \int\limits_{-\infty}^{+\infty} f(t;\mu_1,\lambda_1)\ln 
f(t;,\mu_1,\lambda_1)\,dt +{}\\
{}+
  \int\limits_{-\infty}^{+\infty} f(x;\mu_1,\lambda_1)\ln 
f(x;\mu_1,\lambda_1)\,dx\,,
  \end{multline*}
  где 
  $$
  t=\fr{\lambda_1}{\lambda_2}\left( x+\mu_2-\mu_1\right)\,.
  $$ 
  Отсюда имеем:
  $$
  H(X_2)-H(X_1) =-\ln \fr{\lambda_1}{\lambda_2} + H(X_1) -H(X_1)=\ln 
\fr{\lambda_2}{\lambda_1}\,.
  $$
  
  \noindent
  \textbf{Следствие~1.} \textit{Пусть в условиях теоремы~$1$ $X_1$ 
и~$X_2$~--- две непрерывные случайные величины, имеющие конечные 
дисперсии. Поскольку среднее квадратическое отклонение непрерывной 
случайной величины, если оно существует, пропорционально параметру 
масштаба, то формулу~$(3)$ можно записать в виде}:
  \begin{equation}
  H(X_2)-H(X_1) =\ln \fr{\sigma_2}{\sigma_1}=\ln \sigma_2 -\ln \sigma_1\,,
  \label{e4-t}
  \end{equation}
  \textit{где $\sigma_1$ и $\sigma_2$~--- средние квадратические отклонения 
случайных величин~$X_1$ и~$X_2$}.
  
  \medskip
  
  \noindent
  \textbf{Следствие~2.} \textit{Дифференциальная энтропия непрерывной 
случайной величины~$X$, имеющей конечную дисперсию~$\sigma^2_X$, равна}:
   $$
   H(X)=\ln \sigma_X+C=\ln e^C\sigma_X\,,
   $$
  \textit{где $C\hm=H(\overset{\circ}{X})\hm= H(X/\sigma_X)$~--- энтропия 
случайной величины~$\overset{\circ}{X}$ с единичной дис\-пер\-си\-ей и тем же 
распределением, что и у случайной величины~$X$.}
  
  \medskip
  
  Действительно, из~(\ref{e4-t}) получим 
  $$
  H(X)- H(\overset{\circ}{X})   =\ln \sigma_X\,,
  $$ откуда
  $$
  H(X) =\ln \sigma_X +H(\overset{\circ}{X})=\ln \sigma_X +C =\ln e^C 
\sigma_X\,.
  $$
  
  Отметим, что константа~$C$ характеризует энтропию закона распределения. 
Она может быть выражена через введенный в~\cite{10-t} энтропийный 
коэффициент~$k$ закона распределения как $C\hm= \ln 2k$.
  
  
  \medskip
  
  \noindent
  \textbf{Теорема~2.} \textit{Пусть имеется две системы непрерывных 
случайных величин $\mathbf{Y}^{(1)} \hm= \left(Y_1^{(1)}, Y_2^{(1)},\ldots , 
Y_m^{(1)}\right)$ и $\mathbf{Y}^{(2)} \hm= \left(Y_1^{(2)}, Y_2^{(2)},\ldots , Y_m^{(2)}\right)$, 
каждые со\-от\-вет\-ст\-ву\-ющие компоненты которых $Y_i^{(1)}, Y_i^{(2)}$ 
$(i\hm=1,2,\ldots , m)$ определены на всей числовой оси, имеют конечные 
дисперсии и описываются однотипными законами распределения с 
некоторыми параметрами положения и масштаба. \mbox{Тогда} разность 
совместных энтропий сис\-тем случайных величин $\mathbf{Y}^{(2)} \hm= 
\left(Y_1^{(2)}, Y_2^{(2)},\ldots , Y_m^{(2)}\right)$ и 
$\mathbf{Y}^{(1)} \hm= \left(Y_1^{(1)}, 
Y_2^{(1)},\ldots , Y_m^{(1)}\right)$ равна}:
  \begin{multline}
  \Delta H\left(\mathbf{Y}\right) =H\left(\mathbf{Y}^{(2)}\right) -H\left(\mathbf{Y}^{(1)} \right)={}\\
  {}=\sum\limits_{k=1}^m \ln \fr{\sigma_{Y_k^{(2)}}}{\sigma_{Y_k^{(1)}}}+
\fr{1}{2}  \sum\limits_{k=2}^m \ln \fr{1-R^2_{Y_k^{(2)}/Y_1^{(2)}\cdots Y_{k-1}^{(2)}}} {1-
R^2_{Y_k^{(1)}/Y_1^{(1)}\cdots Y_{k-1}^{(1)}}}\,,
  \label{e5-t}
  \end{multline}
  \textit{где}
  $$
  \sigma_{Y_k^{(j)}/Y_1^{(j)}\cdots Y_{k-1}^{(j)}} =\sigma_{Y_k^{(j)}}\sqrt{1-
R^2_{Y_k^{(j)}/Y_1^{(j)}\cdots Y_{k-1}^{(j)}}}\,;
  $$
  $R^2_{Y_k^{(j)}/Y_1^{(j)}\ldots Y_{k-1}^{(j)}}$~--- \textit{коэффициенты 
детерминации соответствующих регрессионных зависимостей}, $k\hm= 2, 3, 
\ldots , m$, $j\hm=1, 2$.
  
  \medskip
  
  \noindent
  Д\,о\,к\,а\,з\,а\,т\,е\,л\,ь\,с\,т\,в\,о\,.\ \ Совместная энтропия $H(\mathbf{Y})$ 
системы случайных величин~\textbf{Y} согласно свойству иерархической 
аддитивности~\cite{11-t} равна
  \begin{multline}
  H(\mathbf{Y}) =H(Y_1)+H\left( Y_2/Y_1\right) +H\left( 
Y_3/Y_1Y_2\right)+\cdots\\
\cdots  + H\left(Y_m/Y_1\cdots Y_{m-1}\right)\,.
  \label{e6-t}
  \end{multline}
  
  Рассмотрим две системы непрерывных случайных 
величин~$\mathbf{Y}^{(1)}$ и~$\mathbf{Y}^{(2)}$, каждые соответствующие 
компоненты $Y_i^{(1)}$, $Y_i^{(2)}$ ($i\hm=1, 2,\ldots , m$) которых 
определены на всей числовой оси, имеют конечные дисперсии и описываются 
однотипными законами распределения с некоторыми параметрами положения и 
масштаба. Тогда изменение энтропии с учетом~(\ref{e6-t}) равно:
  \begin{multline*}
  \Delta H(\mathbf{Y}) =H\left(\mathbf{Y}^{(2)}\right) -H\left(\mathbf{Y}^{(1)}\right) 
={}\\
{}=
H\left(Y_1^{(2)}\right) -H\left(Y_1^{(1)}\right) +H\left(Y_2^{(2)}/Y_1^{(2)}\right)-{}\\
  {}- H\left( Y_2^{(1)}/Y_1^{(1)}\right) +\cdots + 
H\left(Y_m^{(2)}/Y_1^{(2)}\cdots Y_{m-1}^{(2)}\right) -{}\\
{}- H\left( Y_m^{(1)}/Y_1^{(1)}\cdots Y_{m-
1}^{(1)}\right)\,.
  \end{multline*}
  
  Условное математическое ожидание ${\sf E}\left[ Y_2/Y_1\hm=x\right]$ является 
регрессией~$Y_2$ на~$Y_1$ с коэффициентом 
детерминации~$R^2_{Y_2/Y_1}$. Поэтому дисперсия случайной величины 
$Y_2/Y_1$ равна дисперсии $\sigma^2_{Y_2/Y_1} \hm= \sigma^2_{Y_2}
  (1-R^2_{Y_2/Y_1})$ остаточной случайной компоненты регрессии ${\sf E}\left[ 
Y_2/Y_1=x\right]$~\cite{12-t}. Аналогично величина ${\sf E}\left[ Y_k/Y_1Y_2\cdots 
Y_{k-1}=\mathbf{x}\right]$ является регрессией~$Y_k$ на случайные величины 
$Y_1, Y_2,\ldots , Y_{k-1}$. Дисперсия случайной величины $Y_k/Y_1Y_2\cdots 
Y_{k-1}$ равна дисперсии остаточной случайной компоненты регрессии 
${\sf E}\left[ Y_k/Y_1Y_2\cdots Y_{k-1}=\mathbf{x}\right]$~\cite{12-t}. Поэтому
  \begin{equation}
  \sigma^2_{Y_k/Y_1Y_2\ldots Y_{k-1}}=\sigma^2_{Y_k}(1-
R^2_{Y_k/Y_1Y_2\ldots Y_{k-1}})\,,
  \label{e7-t}
  \end{equation}
  где $ R^2_{Y_k/Y_1Y_2\cdots Y_{k-1}}$~--- коэффициенты детерминации 
соответствующих регрессионных зависимостей, $k\hm=2, 3, \ldots , m$.
  
  Отсюда с учетом~(\ref{e4-t}) и~(\ref{e7-t}) получим
  \begin{multline*}
  \Delta H(\mathbf{Y}) =\ln \fr{\sigma_{Y_1^{(2)}}}{\sigma_{Y_1^{(1)}}}+
  \sum\limits_{k=2}^m \ln \fr{\sigma_{Y_k^{(2)}/Y_1^{(2)}\cdots Y_{k-
1}^{(2)}}}{\sigma_{Y_k^{(1)}/Y_1^{(1)}\cdots Y_{k-1}^{(1)}}} = {}\\
{}=
  \ln \fr{\sigma_{Y_1^{(2)}}}{\sigma_{Y_1^{(1)}}}+\sum\limits_{k=2}^m 
  \ln \fr{\sigma_{Y_k^{(2)}}\sqrt{1-R^2_{Y_k^{(2)}/Y_1^{(2)}\cdots 
  Y_{k-1}^{(2)}}}}{\sigma_{Y_k^{(1)}}\sqrt{1-R^2_{Y_k^{(1)}/Y_1^{(1)}\cdots 
Y_{k-1}^{(1)}}}}={}\\
  {}=\sum\limits_{k=1}^m \ln 
\fr{\sigma_{Y_k^{(2)}}}{\sigma_{Y_k^{(1)}}}+\fr{1}{2}\sum\limits_{k=2}^m 
  \ln \fr{1-R^2_{Y_k^{(2)}/Y_1^{(2)}\cdots Y_{k-1}^{(2)}}}{1-
R^2_{Y_k^{(1)}/Y_1^{(1)}\cdots Y_{k-1}^{(1)}}}\,.
  \end{multline*}
  
  
  Обозначив 
  \begin{align*}
  \Delta H(\mathbf{Y})_{\boldsymbol{\Sigma}} &= \sum\limits_{k=1}^m \ln 
\fr{\sigma_{Y_k^{(2)}}}{\sigma_{Y_k^{(1)}}}\,;\\
  \Delta H(\mathbf{Y})_{\mathbf{R}} &= \fr{1}{2}\sum\limits_{k=2}^m \ln \fr{1-
R^2_{Y_k^{(2)}/Y_1^{(2)}\cdots Y_{k-1}^{(2)}}}{1-
R^2_{Y_k^{(1)}/Y_1^{(1)}\cdots Y_{k-1}^{(1)}}}\,,
  \end{align*}  
  представим формулу~(\ref{e5-t}) как
  \begin{equation}
  \Delta H(\mathbf{Y}) =\Delta H(\mathbf{Y})_{\boldsymbol{\Sigma}}+\Delta 
H(\mathbf{Y})_{\mathbf{R}}\,,
  \label{e8-t}
  \end{equation}
  где $\Delta H(\mathbf{Y})_{\boldsymbol{\Sigma}}$ и $\Delta H(\mathbf{Y})_{\mathbf{R}}$~--- 
приращения энтропии за счет изменения дисперсий и корреляций случайных 
величин $Y_1,Y_2,\ldots , Y_m$.
  
  \medskip
  
  \noindent
  \textbf{Следствие~1.} \textit{Если случайный вектор~$\mathbf{Y}$ является 
гауссовским, то получим рассмотренный в}~\cite{13-t} \textit{частный случай}
  \begin{equation}
  \Delta H(\mathbf{Y}) =\sum\limits_{k=1}^m \ln 
\fr{\sigma_{Y_k^{(2)}}}{\sigma_{Y_k^{(1)}}} +\fr{1}{2}\ln \fr{\vert 
\mathbf{R}_{\mathbf{Y}^{(2)}}\vert}{\vert \mathbf{R}_{\mathbf{Y}^{(1)}}\vert}\,,
  \label{e9-t}
  \end{equation}
  \textit{где $\vert \mathbf{R}_{\mathbf{Y}^{(j)}}\vert$~--- определитель 
корреляционной матрицы $\mathbf{R}_{\mathbf{Y}^{(j)}}$ случайного вектора} 
$\mathbf{Y}^{(j)}$, $j\hm=1, 2$.
  
  \medskip
  
  \noindent
  \textbf{Следствие~2.} \textit{Совместная дифференциальная энтропия 
$H(\mathbf{Y})$ системы непрерывных случайных величин $\mathbf{Y}\hm= 
(Y_1,Y_2,\ldots , Y_m)$ равна}
  \begin{multline}
  H(\mathbf{Y}) ={}\\
  {}=\sum\limits_{k=1}^m H(Y_k) +\fr{1}{2} \sum\limits_{k=2}^m \ln 
\left (1-R^2_{Y_k/Y_1Y_2\cdots Y_{k-1}}\right)\,.
  \label{e10-t}
  \end{multline}
  
  \medskip
  
  Действительно, подставив в~(\ref{e5-t}) вместо $\mathbf{Y}^{(1)}$ 
и~$\mathbf{Y}^{(2)}$ соответственно две системы непрерывных случайных 
величин $\tilde{\mathbf{Y}} \hm= (\tilde{Y}_1,\tilde{Y}_2, \ldots , \tilde{Y}_m)$ 
и $\mathbf{Y}\hm= (Y_1,Y_2,\ldots , Y_m)$, каждые соответствующие 
компоненты~$\tilde{Y}_i$, $Y_i$ ($i\hm=1,2,\ldots ,m$) которых определены на 
всей числовой оси, имеют конечные дисперсии и описываются одинаковыми 
законами распределения, причем $\tilde{Y}_i$ ($i\hm=1,2,\ldots ,m$) являются 
взаимно независимыми, получим:
  \begin{multline*}
  H(\mathbf{Y}) -H(\tilde{\mathbf{Y}})= {}\\
  {}=\sum\limits_{k=1}^m \ln 
\fr{\sigma_{Y_k}}{\sigma_{\tilde{Y}_k}}+\fr{1}{2} \sum\limits_{k=2}^m \ln 
  \fr{1-R^2_{Y_k/Y_1\cdots Y_{k-1}}}{1-R^2_{\tilde{Y}_k/\tilde{Y}_1\cdots 
\tilde{Y}_{k-1}}}\,.
  \end{multline*}
  
  Поскольку $\forall k\ H(\tilde{Y}_k) \hm= H(Y_k)$, $\sigma_{Y_k}\hm= 
\sigma_{\tilde{Y}_k}$, $R^2_{\tilde{Y}_k/\tilde{Y}_1\tilde{Y}_2\cdots 
\tilde{Y}_{k-1}}\hm=0$, то
\begin{multline*}
  H(\mathbf{Y}) =H(\tilde{Y}) +\fr{1}{2}\sum\limits_{k=2}^m \ln \left( 1-
R^2_{Y_k/Y_1Y_2\cdots Y_{k-1}}\right) = {}\\
{}=\sum\limits_{k=1}^m H(Y_k) +\fr{1}{2} 
\sum\limits_{k=2}^m \ln \left (1-R^2_{Y_k/Y_1Y_2\cdots Y_{k-1}}\right)\,.
  \end{multline*}
  
  Формула~(\ref{e10-t}) является обобщением приведенного в~\cite{13-t} 
соотношения для энтропии многомерного нормального распределения
  \begin{equation}
  H(\mathbf{Y}) =\sum\limits_{k=1}^m H(Y_k) +\fr{1}{2} \ln \vert 
\mathbf{R}_{\mathbf{Y}} \vert\,,
  \label{e11-t}
  \end{equation}
  где $\vert \mathbf{R}_{\mathbf{Y}}\vert$~--- определитель корреляционной 
матрицы~$\mathbf{R}_{\mathbf{Y}}$.
  %
  Поэтому, так же как и для~(\ref{e11-t}), и в общем случае 
  согласно~(\ref{e10-t}) энтропия многомерной случайной величины 
складывается из двух со\-став\-ля\-ющих:
  \begin{equation}
      H(\mathbf{Y}) =H(\mathbf{Y})_{\boldsymbol{\Sigma}} +H(\mathbf{Y})_{\mathbf{R}}\,,
  \label{e12-t}
  \end{equation}
  где 
  \begin{align*}
  H(\mathbf{Y})_{\boldsymbol{\Sigma}} & = \sum\limits_{k=1}^m H(Y_k)\,;
\\
  H(\mathbf{Y})_{\mathbf{R}} &= (1/2)\sum\limits_{k=1}^m  \ln \left( 1-
R^2_{Y_k/Y_1Y_2\ldots Y_{k-1}}\right)\,.
\end{align*}
  
  \section{Исследование изменения состояния стохастической 
системы на~основе энтропийной модели}
  
  Аддитивные представления~(\ref{e12-t}) и~(\ref{e8-t}) в виде двух компонент 
как самой энтропии, так и ее изменения показывают ее дуализм. Компонента 
$H(\mathbf{Y})_{\boldsymbol{\Sigma}}$ определяет предельную энтропию, со\-от\-вет\-ст\-ву\-ющую 
полной независимости элементов сис\-те\-мы. Поэтому условно назовем ее 
энтропией хао\-тич\-ности. Величина $H(\mathbf{Y})_{\mathbf{R}}$ равна 
энтропии за счет совместной корреляционной взаимосвязи между элементами 
сис\-те\-мы, ее условно можно назвать энтропией самоорганизации. Следует 
заметить, что дуализм энтропии в том или ином виде отмечался в ряде 
публикаций~\cite{1-t, 2-t, 7-t, 9-t, 13-t, 14-t}.
  
  Таким образом, изменение энтропии происходит аддитивным образом: с 
одной стороны~--- за счет изменения дисперсий, а с другой стороны~--- из-за 
изменения коррелированности случайных величин $Y_1,Y_2,\ldots , Y_m$. 
Следовательно, причины рос\-та и уменьшения энтропии сис\-те\-мы могут быть 
различными. Например, энтропию системы можно увеличить (уменьшить) 
посредством увеличения (уменьшения) дисперсий~$\sigma^2_{Y_k}$ или 
уменьшения (\mbox{увеличения}) коэффициентов детерминации 
$R^2_{Y_k/Y_1Y_2\cdots Y_{k-1}}$ компонент вектора~$\mathbf{Y}$.
  
  Выражение~(\ref{e5-t}) позволяет обнаружить причину изменения энтропии 
системы и оценить его количественно. Это делает возможным использование 
энтропийной модели в задачах контроля и диагностики состояния 
стохастических сист\-ем. Пусть стохастическая система представима в виде 
случайного вектора~$\mathbf{Y}$. Тогда на основе модели~(\ref{e5-t}) можно 
осуществлять мониторинг состояния стохастической системы путем анализа 
изменения ее энтропии. Это можно сделать следующим образом. Будем 
считать, что две системы непрерывных случайных величин~$\mathbf{Y}^{(1)}$ 
и~$\mathbf{Y}^{(2)}$ соответствуют предыдущему и текущему периодам 
функционирования системы. Тогда, отслеживая изменение $\Delta 
H(\mathbf{Y})$ энтропии в целом и ее компонент $\Delta 
H(\mathbf{Y})_{\boldsymbol{\Sigma}}$, $\Delta H(\mathbf{Y})_{\mathbf{R}}$, можно сделать 
выводы о состоянии системы. Анализ изменения каждой из случайных 
величин~$Y_k$
  \begin{align*}
  \Delta H(\mathbf{Y})_{\boldsymbol{\Sigma},k} &=\ln 
\fr{\sigma_{Y_k^{(2)}}}{\sigma_{Y_k^{(1)}}}\,;\\
  \Delta H(\mathbf{Y})_{\mathbf{R},k} &=\fr{1}{2}\ln 
\fr{1-R^2_{Y_k^{(2)}/Y_1^{(2)}\cdots Y_{k-1}^{(2)}}} {1-
R^2_{Y_k^{(1)}/Y_1^{(1)}\cdots Y_{l-1}^{(1)}}}
  \end{align*}
позволит выявить те элементы системы (компоненты системы~$\mathbf{Y}$), 
которые оказали наибольшее влияние на изменение энтропии всей сис\-темы.

  Поскольку $R^2_{Y_k/Y_1Y_2\ldots Y_{k-1}}\hm\geq 
R^2_{Y_k/Y_1Y_2\ldots Y_{k-2}} \hm\geq R^2_{Y_k/Y_1Y_2\cdots Y_{k-3}}\geq 
\cdots \geq R^2_{Y_k/Y_1}$~\cite{12-t}, то оценивать вклад произвольного 
  $l$-го элемента в изменение энтропии самоорганизации целесообразно через 
их предельные значения:
  \begin{multline*}
  \Delta H(\mathbf{Y})^*_{\mathbf{R},l} = \fr{1}{2}\ln 
  \fr{1- R^2_{Y_l^{(2)}/Y_1^{(2)}\cdots Y_{l-1}^{(2)}Y_{l+1}^{(2)}\cdots Y_m^{(2)}}} 
{1-R^2_{Y_l^{(1)}/Y_1^{(1)}\cdots Y_{l-1}^{(1)}Y_{l+1}^{(1)}\cdots Y_m^{(1)}}}\,,\\ 
l=1,2,\ldots ,m\,.
  \end{multline*}
  
  Выше был рассмотрен случай, когда пары случайных величин $Y_i^{(1)}$, 
$Y_i^{(2)}$ ($i\hm=1,2,\ldots ,m$) имели однотипные распределения. Если 
данное предположение не выполняется, то тогда изменение энтропии 
хаотичности согласно~(\ref{e10-t}) равно:
  \begin{multline*}
  \Delta H(\mathbf{Y})_{\boldsymbol{\Sigma}}= H\left(\mathbf{Y}^{(2)}\right)_{\boldsymbol{\Sigma}} - 
H\left(\mathbf{Y}^{(1)}\right)_{\boldsymbol{\Sigma}} ={}\\
{}=\sum\limits_{k=1}^m \Delta 
H(\mathbf{Y})_{\boldsymbol{\Sigma},k} = \sum\limits_{k=1}^m \left[ H\left(Y_k^{(2)}\right)-H 
\left(Y_k^{(1)}\right)\right]\,,
  \end{multline*}
т.\,е.\ потребуется определять энтропии одномерных случайных величин по 
выборочным данным. Отметим, что в настоящее время предложен ряд 
алгоритмов для решения данной задачи~[15--18].
  
  \section{Пример мониторинга состояния многомерных 
стохастических систем}
  
  Рассмотрим задачу мониторинга состояния стохастической системы на 
модельном примере. Пусть некоторая стохастическая система~$S$ 
моделируется в виде случайного вектора $\mathbf{Y}\hm= (Y_1,Y_2,Y_3)$. Для 
упрощения будем считать его нормальным. Рассмотрим динамику 
функционирования системы на основе модели~(\ref{e5-t}). Пусть в 
предыдущем и текущем периодах функционирования системы имеем 
случайные векторы~$\mathbf{Y}^{(1)}$, $\mathbf{Y}^{(2)}$ с 
ковариационными матрицами, равными
  \begin{align*}
\boldsymbol{\Sigma}^{(1)} &=\begin{pmatrix}
  1{,}114 &\ 1{,}131 &\ 0{,}494\\
  1{,}131 &\ 3{,}310&\ 3{,}205\\
  0{,}494 &\ 3{,}205 &\ 5{,}348
  \end{pmatrix}\,;\\[6pt]
\boldsymbol{\Sigma}^{(2)} &= \begin{pmatrix}
  1{,}796 &\ 1{,}713 &\ 0{,}381\\
  1{,}713 &\ 3{,}589&\ 3{,}199\\
  0{,}381&\ 3{,}199&\ 4{,}841 \end{pmatrix}\,.
  \end{align*}
  
  Согласно~(\ref{e5-t}), (\ref{e8-t}) имеем:
  $$
  \fr{\sigma_{Y_1^{(2)}}}{\sigma_{Y_1^{(1)}}} =1{,}270\,;\enskip 
  \ln \fr{\sigma_{Y_1^{(2)}}}{\sigma_{Y_1^{(1)}}}=0{,}239\,;
  $$
  $$
    \fr{\sigma_{Y_2^{(2)}}}{\sigma_{Y_2^{(1)}}} =1{,}041\,; \enskip
  \ln \fr{\sigma_{Y_2^{(2)}}}{\sigma_{Y_2^{(1)}}}=0{,}040\,;
  $$
  $$
    \fr{\sigma_{Y_3^{(2)}}}{\sigma_{Y_3^{(1)}}} =0{,}951\,;\enskip
  \ln \fr{\sigma_{Y_3^{(2)}}}{\sigma_{Y_3^{(1)}}}=-0{,}050\,;
  $$
  $$
  \Delta H(\mathbf{Y})_{\boldsymbol{\Sigma}} =\sum\limits_{k=1}^3 \ln 
\fr{\sigma_{Y_k^{(2)}}}{\sigma_{Y_k^{(1)}}} =0{,}229\,;
  $$
  $$
  \fr{1-R^2_{Y_2^{(2)}/Y_1^{(2)}}}{R^2_{Y_2^{(1)}/Y_1^{(1)}}} =\fr{1-
0{,}455}{1-0{,}347}=0{,}835\,;
  $$
  $$
  \fr{1-R^2_{Y_3^{(2)}/Y_1^{(2)}Y_2^{(2)}}}{1-R^2_{Y_3^{(1)}/Y_1^{(1)}Y_2^{(1)}}} 
=\fr{1-0{,}866}{1-0{,}673}=0{,}409\,;
  $$
  $$
  \fr{1}{2}\ln \fr{1-R^2_{Y_2^{(2)}/Y_1^{(2)}}}{1-
R^2_{Y_2^{(1)}/Y_1^{(1)}}} =-0{,}090\,;
  $$
  $$
  \fr{1}{2}\ln \fr{1-R^2_{Y_3^{(2)}/Y_1^{(2)}Y_2^{(2)}}} 
  {1-R^2_{Y_3^{(1)}/Y_1^{(1)}Y_2^{(1)}}} =-0{,}447\,;
  $$
  $$
  \Delta H(\mathbf{Y})_{\mathbf{R}} =\fr{1}{2}\sum\limits_{k=2}^3 \ln 
  \fr{1-R^2_{Y_k^{(2)}/Y_1^{(2)}\cdots 
  Y_{k-2}^{(2)}}}{1-R^2_{Y_k^{(1)}/Y_1^{(1)}\cdots Y_{k-1}^{(1)}}} =
  -0{,}537\,;
  $$
 
 \vspace*{-12pt}
 
 \noindent
 \begin{multline*}
  \Delta H(\mathbf{Y}) =  \Delta H(\mathbf{Y})_{\boldsymbol{\Sigma}} +\Delta 
H(\mathbf{Y})_{\mathbf{R}} ={}\\
{}=0{,}229-0{,}537=-0{,}308\,.
  \end{multline*}
  
  Таким образом, энтропия системы в текущем периоде уменьшилась на~0,308, 
причем энтропия хаотичности выросла на~0,229 ($\Delta 
H(\mathbf{Y})_{\boldsymbol{\Sigma}}\hm=0{,}229$), а энтропия самоорганизации 
сократилась на~0,537 ($\Delta H(\mathbf{Y})_{\mathbf{R}}\hm= -0{,}537$). 
Это означает, что в текущем периоде в системе преобладала тенденция 
снижения энтропии самоорганизации.
  
  Анализ изменения каждой из компонент $\Delta H(\mathbf{Y})_{\Sigma,k}$ и 
$\Delta H(\mathbf{Y})_{\mathbf{R},k}$ показывает, что на рост энтропии 
хаотичности повлиял первый элемент сис\-те\-мы, а на снижение энтропии 
самоорганизации~--- третий элемент.
  
  \section{Заключение}
  
  \noindent
  \begin{enumerate}[1.]
  \item  Предложено энтропийное моделирование динамики многомерных 
стохастических систем. В~его основе лежит представление системы в виде 
случайного вектора, каждая из компонент которого представляет собой 
непрерывную случайную величину.
\item Получены аналитические выражения для энтропии многомерной 
случайной величины и ее динамики.
\end{enumerate}

Основные достоинства предложенного подхода:
  \begin{itemize}
  \item простота реализации и интерпретации математической модели;
  \item энтропийная модель применима при решении задач диагностики и 
контроля состояния стохастических сис\-тем, а также эффективного управ\-ле\-ния 
ими;
  \item универсальность и применимость к стохастическим системам 
различной природы;
  \item возможность использования на малых выборках данных.
  \end{itemize}
  
{\small\frenchspacing
{%\baselineskip=10.8pt
\addcontentsline{toc}{section}{Литература}
\begin{thebibliography}{99}
  \bibitem{1-t}
  \Au{Климонтович Ю.\,Л.} Введение в физику открытых сис\-тем.~--- М.: 
Янус-К, 2002. 284~с.
  \bibitem{2-t}
  \Au{Вильсон А.\,Дж.} Энтропийные методы моделирования слож\-ных 
  сис\-тем~/ Пер. с англ.~--- М.: Наука, 1978.  248~с. (\Au{Wilson~A.\,G.}
  Entropy in urban and regional modeling.~--- London: Pion, 1970. 166~p.)
  \bibitem{3-t}
  \Au{Трубецков Д.\,И., Мчедлова Е.\,С., Красичков~Л.\,В.} Введение в 
  тео\-рию самоорганизации открытых сис\-тем.~--- М.: Физматлит, 2002. 200~с.
  \bibitem{4-t}
  \Au{Романовский Ю.\,М., Степанова~Н.\,В., Чернавский~Д.\,С.} 
Математическое моделирование в биофизике.~--- Москва--Ижевск: Институт 
компьютерных исследований, 2003. 402~с.


  \bibitem{6-t} %5
  \Au{Прангишвили И.\,В.} Энтропийные и другие сис\-тем\-ные 
закономерности: Вопросы управ\-ле\-ния сложными сис\-те\-ма\-ми.~--- М.: 
Наука, 2003. 428~с.
  \bibitem{5-t} %6
  \Au{Скоробогатов С.\,М.} Катастрофы и живучесть железобетонных 
сооружений (классификация и элементы теории).~--- Екатеринбург: УрГУПС, 
2009. 512~с.

  \bibitem{7-t}
  \Au{Шеннон К.} Работы по теории информации и кибернетике~/ Пер с англ.
  С.~Карпова.~--- М.:    ИИЛ, 1963. 830~с.
  
  \bibitem{8-t}
  \Au{Cover T.\,M., Thomas~J.\,A.} Elements of information theory.~--- N.Y.: 
Wiley, 1991. 563~p.
  \bibitem{9-t}
  \Au{Pena D., Van der Linde~A.} Dimensionless measures of variability and 
dependence for multivariate continuous distributions~// Commun. 
Stat.: Theor. M., 2007. Vol.~36. Issue~10. P.~1845--1854.
  \bibitem{10-t}
  \Au{Новицкий П.\,В.} Основы информационной теории измерительных 
устройств.~--- Ленинград: Энергия, 1968. 248~с.
  \bibitem{11-t}
  \Au{Стратонович Р.\,Л.} Теория информации.~--- М.: Советское радио, 
1975. 424~с.
  \bibitem{12-t}
  \Au{Greene W.\,H.} Econometric analysis.~--- 7th ed.~--- Prentice Hall, 2011. 1230~p.
  \bibitem{13-t}
  \Au{Тырсин А.\,Н., Соколова~И.\,С.} Энтро\-пий\-но-ве\-ро\-ят\-но\-ст\-ное 
моделирование гауссовских стохастических сис\-тем~// Математическое 
моделирование, 2012. Т.~24. №\,1. С.~88--102.
  \bibitem{14-t}
  \Au{Николис Г., Пригожин~И.} Самоорганизация в неравновесных 
  сис\-те\-мах: от диссипативных структур к упорядоченности через 
флуктуации~/ Пер с англ.~--- М.: Мир, 1979. 512~с. (\Au{Nikolis~G., Prigogine~I.}
Self-organozation in nonequilibrium systems: From dissipative structures to order through fluctuations.~---
N.Y.: John Wile\,\&\,Sons, 1977. 512~p.
  \bibitem{15-t}
  \Au{Beirlant J., Dudewicz~E.\,J., Gyorfi~L., van der Meulen~E.\,C.} 
Nonparametric entropy estimation: an overview~// Int. J.~Math. Stat. Sci., 1997. Vol.~6. Issue~1. P.~17--39.
  \bibitem{16-t}
  \Au{Stowell D., Plumbley~M.\,D.} Fast multidimensional entropy estimation by 
  $k$--$d$ partitioning~// IEEE Signal Proc. Lett., 2009. Vol.~16. Issue~6. 
P.~537--540.
  \bibitem{17-t}
  \Au{Тырсин А.\,Н., Клявин~И.\,А.} Повышение точности оценки энтропии 
случайных экспериментальных данных~// Сис\-те\-мы управ\-ле\-ния и 
информационные технологии, 2010. №\,1(39). С.~87--90.
  \bibitem{18-t}
  \Au{Noughabi H.\,A., Arghami~N.\,R.} A~new estimator of entropy~// J.~Iran. 
Stat. Soc., 2010. Vol.~9. Issue~1. P.~53--64.
  
 
  \end{thebibliography} } }
  



\end{multicols}

  \hfill{\small
\textit{Поступила в редакцию 13.05.13}}




%\vspace*{12pt}

%\hrule

%\vspace*{2pt}

%\hrule

\newpage

\def\tit{STUDY OF THE DYNAMICS OF~MULTIDIMENSIONAL STOCHASTIC 
SYSTEMS BASED ON~ENTROPY MODELING}

\def\aut{A.\,N.~Tyrsin$^1$ and O.\,V.~Vorfolomeeva$^2$}

\def\autkol{A.\,N.~Tyrsin and O.\,V.~Vorfolomeeva}
\def\titkol{Study of the dynamics of~multidimensional stochastic 
systems based on~entropy modeling}

\titel{\tit}{\aut}{\autkol}{\titkol}

\vspace*{-9pt}


\noindent
$^1$Science and Engineering Center ``Reliability and Resource of Large Systems and Machines,'' 
Ural Branch,\linebreak
$\hphantom{^1}$Russian Academy of Sciences, Yekaterinburg 620049, Russian Federation\\
\noindent
$^2$Chelyabinsk State University, Chelyabinsk 454001, Russian Federation

\def\leftfootline{\small{\textbf{\thepage}
\hfill INFORMATIKA I EE PRIMENENIYA~--- INFORMATICS AND APPLICATIONS\ \ \ 2013\ \ \ volume~7\ \ \ issue\ 4}
}%
 \def\rightfootline{\small{INFORMATIKA I EE PRIMENENIYA~--- INFORMATICS AND APPLICATIONS\ \ \ 2013\ \ \ volume~7\ \ \ issue\ 4
\hfill \textbf{\thepage}}}  

\vspace*{15pt}
  
  \Abste{A new entropy approach of modeling of dynamics of stochastic systems 
is described. It is based on the representation of the system in the form of a 
multidimensional stochastic vector. It is shown that the change in entropy of  a
multivariate stochastic system can be expressed in terms of dispersions and conditional 
correlations of a component of a random vector.  This allows to reveal the cause of the 
change in the entropy of the system and to evaluate it quantitatively. It was found that 
the entropy of a stochastic system consists of two components that characterize its 
properties. The first component determines the limit entropy corresponding to the full 
independence of the elements of the system and defines the consideration of the 
integral object as consisting of components (additivity). The second component 
reflects the extent of interrelation between the elements of the system, defining the 
properties of the system as a whole (integrity).
This approach makes it possible to use 
entropy models in the diagnostics and control of stochastic systems as well as efficient
management. The advantages of the proposed approach include the simplicity 
of implementation and interpretation of the mathematical model, the universality and 
adaptability for stochastic systems of different nature, the possibility of its use on 
small samples of data. The article contains an example of the practical application of  a
mathematical model.}
  
  \KWE{multidimensional random variable; entropy; dynamics; stochastic system; 
dispersion; correlation}

\DOI{10.14357/19922264130401}

\Ack 
 The work was supported by the project 12-M-127-2049 of Basic 
Researches of the Ural Branch of the Russian Academy of Sciences. 

\vspace*{9pt}

  \begin{multicols}{2}

\renewcommand{\bibname}{\protect\rmfamily References}
%\renewcommand{\bibname}{\large\protect\rm References}

{\small\frenchspacing
{%\baselineskip=10.8pt
\addcontentsline{toc}{section}{References}
\begin{thebibliography}{99}
  
 \bibitem{1-t-1}
\Aue{Klimontovich, Ju.\,L.} 2002. \textit{Vvedenie v fiziku otkrytykh sistem}
[\textit{Introduction to the physics of open systems}].
Moscow: Janus-K Publ. 284~p.
\bibitem{2-t-1}
\Aue{Wilson A.\,G.} 1970. \textit{Entropy in urban and regional modeling}.  
L.: Pion. 166~p.
\bibitem{3-t-1}
\Aue{Trubeckov, D.\,I., E.\,S.~Mchedlova, and L.\,V.~Krasichkov}. 2002. 
\textit{Vvedenie v teoriyu samoorganizatsii otkrytykh sistem}
[\textit{Introduction to the theory of self-organization of open systems}]. Moscow: 
Publishing House of Physical-Mathematical Literature. 200~p.
\bibitem{4-t-1}
\Aue{Romanovskij, Ju.\,M., N.\,V.~Stepanova, and D.\,S.~Chernavskij}.  2003. 
\textit{Matematicheskoe modelirovanie v biofizike}
[\textit{Mathematical modeling in biophysics}]. Moscow--Izhevsk: Computer Research Institute.
402~p.
\bibitem{6-t-1}
\Aue{Prangishvili, I.\,V.} 2003. \textit{Entropiynye i drugie sistemnye 
zakonomernosti: Voprosy upravleniya slozhnymi sistemami}
[\textit{Entropic and other system regularities: Questions of management of complex
systems}]. Moscow: Nauka. 428~p.
\bibitem{5-t-1}
\Aue{Skorobogatov, S.\,M.} 2009. \textit{Katastrofy i zhivuchest' 
zhelezobetonnykh sooruzheniy (klassifikatsiya i elementy teorii)}
[\textit{Catastrophes and serviceability of reinforced concrete buildings (classification and
elements of theory)}]. Ekaterinburg: 
Ural State University of Railway Transport. 512~p.

\bibitem{7-t-1}
\Aue{Shannon C.} 1948. A mathematical theory of communication.
\textit{Bell Syst. Tech.~J.} 27(3):379--423; 4:623--656.
\bibitem{8-t-1}
\Aue{Cover, T.\,M., and J.\,A.~Thomas}. 1991. \textit{Elements of information 
theory}. N.Y.: Wiley. 563~p.
\bibitem{9-t-1}
\Aue{Pena, D., A.~Van der Linde}. 2007. Dimensionless measures of variability 
and dependence for multivariate continuous distributions. 
\textit{Commun. Stat. Theor. M.} 36(10):1845--1854.
\bibitem{10-t-1}
\Aue{Novickij, P.\,V.} 1968.  \textit{Osnovy informatsionnoy teorii 
izmeritel'nykh ustroystv}
[\textit{Fundamentals of information theory measuring devices}]. Leningrad: Energy Publs. 248~p.
\bibitem{11-t-1}
\Aue{Stratonovich, R.\,L.} 1975.  \textit{Teoriya informatsii}
[\textit{Information theory}]. Moscow: Soviet Radio Publs. 424~p.
\bibitem{12-t-1}
\Aue{Greene, W.\,H.} 2011. \textit{Econometric analysis}. 7th ed. Prentice 
Hall. 1230~p.
\bibitem{13-t-1}
\Aue{Tyrsin, A.\,N., and I.\,S.~Sokolova}. 2012. Entropiyno-veroyatnostnoe 
modelirovanie gaussovskikh sto\-khas\-ti\-che\-skikh sistem 
[Entropy-propabilistic modeling of Gaussian stochastic systems].
\textit{Matematicheskoe 
Modelirovanie} [\textit{Mathematical Modeling}] 24(1):88--102.

\vspace*{3pt}

\bibitem{14-t-1}
\Aue{Nikolis, G., and I.~Prigogine}. 1977. 
\textit{Self-oraganization in nonequilibrium systems:
From dissipative structures to order through fluctuations}.
N.Y.: John Wiley\,\&\,Sons. 512~p.

\vspace*{3pt}

\bibitem{15-t-1}
\Aue{Beirlant, J., E.\,J.~Dudewicz, L.~Gyorfi, and E.\,C.~van der Meulen}. 
1997.  Nonparametric entropy estimation: an overview. \textit{Int. 
J.~Math. Stat. Sci.} 
6(1):17--39.

\columnbreak

\bibitem{16-t-1}
\Aue{Stowell, D., and M.\,D.~Plumbley}. 2009. Fast multidimensional entropy 
estimation by $k$--$d$ partitioning. \textit{IEEE Signal Proc. Lett.} 
16(6):537--540.
\bibitem{17-t-1}
\Aue{Tyrsin, A.\,N., and I.\,A.~Kljavin}. 2010. Povyshenie toch\-nosti otsenki 
entropii sluchaynykh eksperimental'nykh dannykh
[Increase in the accuracy of the estimate of the entropy of a random experimental
data]. \textit{Sistemy Upravleniya i 
Informatsionnye Tekhnologii} [\textit{Control Systems and Information Technologies}]
1(39):87--90.


\bibitem{18-t-1}
\Aue{Noughabi, H.\,A., and N.\,R.~Arghami}. 2010. A~new estimator of entropy. 
\textit{J.~Iran. Stat. Soc.}  9(1):53--64.
  
\end{thebibliography}
} }



\end{multicols}

\hfill{\small \textit{Received May 13, 2013}}

%{%\hrule\par
%\raggedleft\Large \bf%\baselineskip=3.2ex
%C\,O\,N\,T\,R\,I\,B\,U\,T\,O\,R\,S \vskip 17pt
%    \hrule
%    \par
%\vskip 21pt plus 8pt minus 3pt }

\Contr

\noindent
\textbf{Tyrsin Alexander N.} (b.\ 1961)~--- Doctor of Science in technology, leading 
researcher, Science and Engineering Center ``Reliability and Resource of Large 
Systems and Machines,'' Ural Branch of the Russian Academy of Sciences, 
Yekaterinburg 620049, Russian Federation; at2001@yandex.ru

\vspace*{3pt}
 \label{end\stat}

\noindent
\textbf{Vorfolomeeva Olga V.} (b.\ 1987)~--- PhD student, Mathematical Faculty, Chelyabinsk 
State University, Chelyabinsk 454001, Russian Federation; ya.olga.work@yandex.ru

\renewcommand{\bibname}{\protect\rm Литература}
     %1
\def\stat{kor-gr}

\def\tit{ПРЕДЕЛЬНАЯ ТЕОРЕМА ДЛЯ ГЕОМЕТРИЧЕСКИХ СУММ НЕЗАВИСИМЫХ НЕОДИНАКОВО 
РАСПРЕДЕЛЕННЫХ СЛУЧАЙНЫХ ВЕЛИЧИН И~ЕЕ ПРИМЕНЕНИЕ К~ПРОГНОЗИРОВАНИЮ ВЕРОЯТНОСТИ 
КАТАСТРОФ В~НЕОДНОРОДНЫХ ПОТОКАХ ЭКСТРЕМАЛЬНЫХ СОБЫТИЙ$^*$}

\def\titkol{Предельная теорема для геометрических сумм независимых неодинаково 
распределенных случайных величин}
% и ее применение к прогнозированию вероятности 
%катастроф в неоднородных потоках экстремальных событий}

\def\autkol{М.\,Е.~Григорьева,  В.\,Ю.~Королев, И.\,А.~Соколов}

\def\aut{М.\,Е.~Григорьева$^1$,  В.\,Ю.~Королев$^2$, И.\,А.~Соколов$^3$}

\titel{\tit}{\aut}{\autkol}{\titkol}

{\renewcommand{\thefootnote}{\fnsymbol{footnote}}
\footnotetext[1] {Работа поддержана Российским фондом фундаментальных исследований (проекты 
11-01-00515-а, 11-07-00112-а, 12-07-00115-а).}}

\renewcommand{\thefootnote}{\arabic{footnote}}
\footnotetext[1]{Parexel International, maria-grigoryeva@yandex.ru} 
\footnotetext[2]{Факультет
вычислительной математики и кибернетики Московского государственного
университета им.\ М.\,В.~Ломоносова; Институт проблем информатики
Российской академии наук, victoryukorolev@yandex.ru} 
\footnotetext[3]{Институт проблем информатики Российской академии
наук, ipiran@ipiran.ru}


%\renewcommand{\r}{\mathbb R}
%\newcommand{\N}{\mathbb N}
%\renewcommand{\P}{{\sf P}}
%\newcommand{\E}{{\sf E}}
%\newcommand{\D}{{\sf D}}



%\newcommand{\I}{\mathbb{I}}
%\newcommand{\betm}{{\beta_{m+1+\delta}}}
%\newcommand{\bet}{\beta_{2+\delta}}
%\renewcommand{\endproof}{\hfill$\Box$}
%\renewcommand{\phi}{\varphi}
%\newcommand{\la}{\lambda}
%\newcommand{\si}{{\rm Si}\:}
%\renewcommand{\Re}{{\rm Re}\:}


 
\vspace*{-16pt}

\Abst{Рассматривается задача прогнозирования вероятностей катастроф в 
неоднородных потоках экстремальных событий. Статья развивает и обобщает 
некоторые методы, предложенные авторами в предыдущих работах. Поток экстремальных 
событий рассматривается как маркированный точечный случайный процесс с 
необязательно одинаково распределенными интервалами между точками (событиями). 
Основой предлагаемых обобщений служат предельные теоремы для геометрических 
случайных сумм независимых неодинаково распределенных случайных величин и 
теория Бал\-ке\-мы\,--\,Пи\-канд\-са\,--\,Де Ха\-ана. Рассмотрена конструкция, в рамках 
которой в качестве предельного распределения для гео\-мет\-ри\-ческих случайных сумм 
независимых неодинаково распределенных случайных величин возникает 
распределение Вей\-бул\-ла--Гне\-ден\-ко. Эффективность методов иллюстрируется 
на примере их применения к прогнозированию момента столкновения Земли с 
потенциально опасным астероидом на основе данных Центра по малым планетам 
Гарвардского университета.} %~\cite{Atkinson2001}.

\vspace*{-5pt}

\KW{катастрофа; экстремальное событие; случайная
сумма; геометрическая сумма; закон больших чисел; распределение
Вей\-бул\-ла--Гне\-ден\-ко;
теорема Бал\-ке\-мы--Пи\-канд\-са--Де Ха\-ана; обобщенное распределение Парето}

\vspace*{-3pt}

\DOI{10.14357/19922264130402}

\vskip 8pt plus 9pt minus 6pt

      \thispagestyle{headings}

      \begin{multicols}{2}

            \label{st\stat}


\section{{Введение. Постановка задачи. Определение экстремального
процесса}}

В данной статье рассматривается задача прогнозирования \textit{вероятностных} 
характеристик катастроф в неоднородных потоках
экстремальных событий. Рассмотрим некоторую систему, подвергающуюся
влиянию некоторого фактора. Предположим, что сила воздействия этого
фактора на систему в каж\-дый момент времени характеризуется некоторым
числом, причем это число изменяется во времени. Это может быть:
\begin{itemize}
\item[$\bullet$] финансовая система, которая характеризуется финансовым
индексом, таким как DAX, NIKKEI, NASDAQ и~т.\,п.; при этом резкие
колебания индекса неблагоприятны и свидетельствуют о тех или иных
кризисных явлениях;
\item[$\bullet$] экологическая система, например среда обитания человека, в
частности состояние жилых помещений в местностях, подверженных
наводнениям или землетрясениям в сейсмоопасных зонах, которое
зависит соответственно от силы подземных толчков и уровня подъема
воды;
\item[$\bullet$] социальная система, которая подвержена по\-литической
неустойчивости или воздействию терро\-ристических организаций; при
этом в качестве числовой характеристики активности неблагоприятных
воздействий может выступить, к примеру, число упоминаний некоторых
соответствующих ключевых слов или фраз в социальных информационных
сетях или сетях \mbox{связи};
\item[$\bullet$] наконец, общее состояние планеты Земля, зависящее от
расстояния, на которое подлетают к ней потенциально опасные
космические объекты~--- астероиды или кометы.
\end{itemize}

При этом естественно возникает задача прогнозирования катастроф.
Однако без применения специализированных методов, специально
ориен-\linebreak\vspace*{-12pt}

\pagebreak

\noindent
тированных на противодействие конкретным рискам, практически
никогда нельзя абсолютно точно предсказать силу воздействия фактора
на систему в каждый момент времени в будущем. Другими словами,
будущее развитие фактора непредсказуемо, вследствие чего значение
числа, характеризующего силу воздействия фактора на систему,
рас\-смат\-ри\-ва\-емое как функция времени, целесообразно рассматривать как
\textit{случайный процесс}. Поэтому задача прогнозирования самог$\acute{\mbox{о}}$
момента катастрофы сводится к прогнозированию \textit{значения}
случайного процесса (т.\,е.\ его значения на вполне определенном
элементе множества элементарных исходов) специальными методами, что
чрезвычайно трудоемко и при рассмотрении современных сложных
стохастических систем практически не реализуемо с приемлемой
точностью. 

В~то же время возникает вполне реальная и важная задача прогнозирования 
\textit{распределения} указанного случайного процесса в те или иные моменты 
времени, т.\,е.\ задача прогнозирования его статистических свойств. 
В~результате решения этой задачи появляется возможность правильно оценить 
уровни угрозы в каждой конкретной ситуации. 
{ %\looseness=1

}

Некоторым методам решения последней задачи и посвящена
данная статья.


Предположим, что очень большие изменения случайного процесса,
характеризующего воздействие фактора на систему, неблагоприятно
влияют на систему и могут вызвать ее необратимые изменения. Вместе с
тем малые флуктуации случайного процесса, характеризующего
воздействие фактора на систему, вполне допустимы (в таких случаях
говорят о <<фоновом значении>> фактора). Поэтому с целью
предсказания катастроф разумно рассматривать не все изменения
случайного процесса, а лишь такие, величина которых превышает
некоторый \textit{потенциально опасный порог}.

Будем говорить, что моменты превышений изменениями случайного процесса 
потенциально опасного порога в совокупности с самими значениями этих превышений 
образуют \textit{экстремальный случайный процесс}. Другими словами, экстремальным 
процессом будем называть маркированный точечный процесс $\{(\tau_i, 
X_i)\}_{i\geqslant1}$, где $\{\tau_i\}_{i\geqslant1}$~--- точечный случайный 
процесс, а $\{X_i\}_{i\geqslant1}$~--- случайные величины. Далее по смыслу 
задачи будет предполагаться, что $X_i\hm>0$, $i\hm=1,2,\ldots$

Среди всех превышений случайным процессом потенциально опасного
порога лишь некоторые очень большие влекут катастрофические
последствия. Поэтому наряду с \textit{потенциально опасным порогом}
рассмотрим \textit{критический порог}, превышение которого
экстремальным процессом и будем считать \textit{катастрофой}.

Для удобства точку отсчета (нуль временн$\acute{\mbox{о}}$й шкалы) поместим в то
время, которое будем считать <<настоящим>>. Тем самым <<настоящее>>
характеризуется значением $t\hm=0$.

Поскольку по условию экстремальный процесс считается случайным, то
\textit{нельзя} точно предсказать момент наступления очередной
катастрофы. Однако можно вычислить или оценить \textit{вероятности
наступления катастрофы} в течение некоторого интервала времени
$[0,\,\tau)$, где $\tau\hm>0$. Если $T$~--- момент наступления
катастрофы, то событие <<катастрофа наступила в течение интервала
времени $[0,\,\tau)$>> эквивалентно тому, что $T\hm<\tau$. В качестве
\textit{исходных данных} будем использовать информацию о развитии
экстремального процесса на некотором интервале времени
$[t_0,\,t_1]$, где $t_0\hm<t_1\hm<0$.

Простейшее (примитивное) решение задачи об отыскании вероятности
наступления катастрофы в течение интервала времени $[0,\,\tau)$ при
условии $\tau\hm<t_1\hm-t_0$ выглядит так.

Разобьем интервал времени $[t_0,\,t_1]$ на непересекающиеся
подынтервалы длиной $\tau$. Пусть внутри интервала $[t_0,\,t_1]$
поместилось $N_{\tau}$ подынтервалов длиной~$\tau$. Подсчитаем
количество подынтервалов, внутри каждого из которых наступила хотя
бы одна катастрофа. Пусть таких подынтервалов оказалось ровно~$n_{\tau}$. 
Тогда для вероятности наступления катастрофы в течение
интервала времени $[0,\,\tau)$ справедлива оценка: 

\noindent
\begin{equation}
{\sf P}(T<\tau)\approx\fr{n_{\tau}}{N_{\tau}}\,,\label{e1-kor}
\end{equation} 
основанная на
классическом определении вероятности как (предела) частоты.

Недостатки такой оценки очевидны. Например, $n_{\tau}$ просто может
оказаться равным нулю, что дает тривиально оптимистичную оценку.
Далее, и~$N_\tau$, и~$n_\tau$ могут быть (и, как правило, являются)
слишком маленькими, чтобы обеспечить приемлемую точность оценки.
Более того, од\-ной-един\-ст\-вен\-ной катастрофы может оказаться достаточно
для полного уничтожения сис\-те\-мы, так что дальнейший сбор информации
просто может оказаться невозможным.

%\vspace*{-36pt}

\section{Метод прогнозирования вероятностей катастроф 
в~неоднородных потоках экстремальных событий}

\subsection{Особенности метода}

К сожалению, именно оценками типа~(\ref{e1-kor}) за\-час\-тую пользуются на
практике для расчетов, связанных с так называемыми большими рисками
в страховании, например при страховании промышленных рисков,
связанных с крупными авариями и экологическими катастрофами. 
В~данной статье описан метод оценивания указанных вероятностей
наступления катастроф, основанный на довольно сложных математических
моделях, но свободный от указанных недостатков. Особенность этого
метода заключается в том, что для того, чтобы прогнозировать
возможности наступления катастроф, необязательно иметь статистику
самих \textit{катастроф}.

Простейший вариант этого метода описан в работах~[1--3] 
и книгах~\cite{KorolevSokolov2008, KorolevShorgin2011}, где предполагалось,
что экстремальный процесс является маркированным процессом
восстановления. В~указанных работах предполагалось, что моменты
$\tau_1,\tau_2,\ldots$ превышений исходным процессом потенциально
опасного порога образуют процесс восстановления. Это означает, что
случайные величины
\begin{equation}
\zeta_i=\tau_i-\tau_{i-1},\ \ \ i=1,2,\ldots, \ \ \
\tau_0=0\,,\label{e2-kor}
\end{equation} 
независимы и имеют одинаковое распределение, т.\,е.\ 
подчиняются одним и тем же статистическим закономерностям.
Другими словами, интенсивность потока экстремальных событий
считалась постоянной. В~то же время в реальных сложных системах,
которые в подавляющем большинстве случаев не являются информационно
и/или энергетически замкнутыми и подвержены влиянию внешней среды,
интенсивности потоков информативных со-\linebreak бытий не являются постоянными.
Например, при адекватном прогнозировании поведения фи\-нан\-совых
индексов ключевую роль играет пред\-став\-ление о том, что интенсивности
потоков ин\-формативных событий на финансовых рынках являются
случайными~\cite{Korolevetal2013}. Отказ от предположения о
постоянстве интенсивности потока экстремальных событий естественно
приводит к необходимости предположить, что случайные величины~(\ref{e2-kor})
имеют неодинаковое распределение. Именно такое обобщение методов,
предложенных в работах~[2--5], и рассматривается в данной работе.

Обозначим величину превышения исходным процессом потенциально
опасного порога в момент~$\tau_i$ символом~$X_i$, $i\hm=1,2,\ldots$
Будем считать что $X_1,X_2,\ldots$~--- независимые и одинаково
распределенные случайные величины. Это означает, что значения этих
случайных величин подчиняются одним и тем же статистическим
закономерностям, характеризуемым \textit{функцией распределения}
$$
F(x)={\sf P}(X_i<x)\,,\enskip -\infty<x<\infty\,,\enskip i=1,2,\ldots
$$
Будем считать, что последовательность $X_1,X_2,\ldots$ статистически
независима от последовательности $\tau_1,\tau_2,\ldots$

Пусть $x_0$~--- критический порог, превышение которого значением~$X_i$ и есть 
катастрофа (т.\,е.\ катастрофическое событие формально записывается в виде 
неравенства $X_i\hm\geqslant x_0$).

Очевидно, что время~$T$ наступления катастрофы (т.\,е.\ время
первого превышения уровня~$x_0$ ка\-кой-ли\-бо из величин~$X_i$) можно
представить в виде геометрической случайной суммы
\begin{equation}
T=\sum\limits_{j=1}^{N}\zeta_j\,,\label{e3-kor}
\end{equation} 
где случайные величины~$\zeta_j$ определены соотношением~(\ref{e2-kor}), а $N$~--- это случайная
величина, имеющая геометрическое распределение с параметром 
$$
{\sf P}(X_i<x_0)=F(x_0)\,.
$$ 
Это означает, что
$$
{\sf P}(N=k)=\left(F(x_0)\right)^{k-1}\left(1-F(x_0)\right)\,,\enskip
k=1,2,\ldots
$$
При этом в силу независимости последовательностей $X_1,X_2,\ldots$\ и
$\tau_1,\tau_2,\ldots$\ число~$N$ слагаемых в сумме~(\ref{e3-kor}) независимо от
самих слагаемых $\zeta_1,\zeta_2,\ldots$\ При этом принципиальным
отличием гео\-мет\-ри\-че\-ских случайных сумм, рассматриваемых здесь, от
гео\-мет\-ри\-че\-ских сумм в традиционном понимании 
(см., например,~\cite{Kalashnikov1997, KorolevBeningShorgin2011}) является то, что в
данном случае слагаемые имеют \textit{неодинаковое} распределение,
тогда как в указанных классических книгах изуча\-лись геометрические
суммы \textit{одинаково} распределенных слагаемых и, соответственно,
использовались методы, ориентированные именно на такую ситуацию.

В рамках подхода, рассматриваемого в данной статье, краеугольными
камнями являются два тео\-ре\-ти\-че\-ских результата. Первый из них~---
версия закона больших чисел для случайных сумм неодинаково
распределенных случайных величин (см.\ теорему~1 ниже), обосновывающая
использование распределения Вей\-бул\-ла--Гне\-ден\-ко в качестве мо-\linebreak дели
распределения интервалов времени между катастрофа\-ми. Второй~---
теорема Бал\-ке\-ма\,--\,Пи\-канд\-са\,--\,Де Ха\-ана (см.\ теорему~3 ниже), обосновывающая
использование обобщенного распределения Парето в качестве модели
распределения критиче\-ских значений неблагоприятного фактора. Эти два
общих результата являются основой предлагаемого метода.

\subsection{Вспомогательные результаты}

Пусть $\xi_1,\xi_2,\ldots$~--- необязательно одинаково распределенные случайные 
величины. Для каж\-до\-го натурального $n\geqslant1$ положим 
$$
S_n=\xi_1+\xi_2+\cdots+\xi_n\,.
$$ 
Рассмотрим последовательность целочисленных 
неотрицательных случайных величин $\{N_n\}_{n\geqslant1}$ и будем считать, что 
при каждом~$n$ случайные величины~$N_n$, $\xi_1,\xi_2,\ldots$ независимы в 
совокупности. Более того, предположим, что
\begin{equation}
N_n\longrightarrow \infty\ \mbox{ по вероятности при }
n\to\infty\,.\label{e4-kor}
\end{equation}
Условие~(\ref{e4-kor}) означает, что ${\sf P}(N_n\leqslant m)\hm\longrightarrow 0$ при
$n\hm\to\infty$ для любого $m\hm>0$. Везде далее символ~$\Longrightarrow$ будет 
обозначать сходимость по распределению.

\smallskip

\noindent
\textbf{Лемма 1.} \textit{Пусть для некоторой последователь\-ности положительных чисел 
$\{b_n\}_{n\geqslant1}$ выполнены условия $b_n\hm\to\infty$ при $n\hm\to\infty$ и
\begin{equation}
\fr{S_n}{b_n}\Longrightarrow 1\,,\enskip n\to\infty\,.\label{e5-kor}
\end{equation}
Предположим, что выполнено условие~$(\ref{e4-kor})$. Для того чтобы при $n\hm\to\infty$ имела 
место сходимость случайных сумм $S_{N_n}$, нормированных некоторой 
последовательностью положительных чисел $\{d_n\}_{n\geqslant1}$ такой, что 
$d_n\hm\to\infty$ при $n\hm\to\infty$, к некоторой случайной величине~$Z$:
\begin{equation}
\fr{S_{N_n}}{d_n}\Longrightarrow Z\,,\label{e6-kor}
\end{equation}
необходимо и достаточно, чтобы}
\begin{equation}
\fr{b_{N_n}}{d_n}\Longrightarrow Z\,,\enskip n\to\infty\,.\label{e7-kor}
\end{equation}


\smallskip

\noindent
\textbf{Замечание~1.} В~силу вырожденности распределения предельной
случайной величины в~(\ref{e5-kor}) сходимость по распределению~(\ref{e5-kor}) оказывается
эквивалентной сходимости по вероятности: для любого $\epsilon\hm>0$
$$
\lim\limits_{n\to\infty}{\sf P}\left(\left\vert \fr{S_n}{b_n}-1\right\vert >\epsilon\right)=0\,,
$$
которую иногда легче проверять.

\smallskip

\noindent
Д\,о\,к\,а\,з\,а\,т\,е\,л\,ь\,с\,т\,в\,о\ \ леммы~1 приведено в~\cite{Korolev1994}.

\smallskip

\noindent
\textbf{Замечание~2.} Лемма~1 является версией закона больших чисел для
случайных сумм. Согласно классическим законам больших чисел при
увеличении числа слагаемых в рассматриваемых <<средних
арифметических>> информация о конкретном виде распределений
слагаемых затухает, стягиваясь в информацию об одном лишь числе.
Точно такой же эффект наблюдается в лемме~1: при рас\-смот\-ре\-нии
<<случайных средних арифметических>> информация о распределениях
слагаемых затухает, так что предельное распределение <<случайного
среднего арифметического>> определяется видом предельного
распределения для случайного индекса (числа слагаемых в сумме) при
надлежащей нормировке.

\smallskip

Для общности пусть $x_n=x_{0,n}$~--- (возрас\-та\-ющая)
последовательность критических порогов такая, что
\begin{equation}
p_n\equiv 1-F(x_n)\longrightarrow 0 \enskip (n\to\infty)\,.\label{e8-kor}
\end{equation}
Тогда в данном случае случайная величина $N\hm=N_n$ имеет
геометрическое распределение с параметром $q_n\hm=1\hm-p_n$. При этом
условие (8) гарантирует выполнение условия~(\ref{e4-kor}). Более того, 
${\sf E}N_n\hm=p_n^{-1}$ и, как хорошо известно,
\begin{equation*}
\lim\limits_{n\to\infty}\sup\limits_{y\geqslant0}\left\vert{\sf P}\left(p_nN_n\geqslant
y\right)-e^{-y}\right\vert =0\,. %\label{e9-kor}
\end{equation*}

Предположим, что постоянные $b_n$, обеспечивающие выполнение условия~(\ref{e5-kor}), 
имеют вид $b_n\hm=bn^{\gamma}$ при некоторых $b\hm>0$ и $\gamma\hm>0$.
При этом значения $\gamma\hm>1$ соответствуют той ситуации, когда
случайные величины~$\zeta_i$ <<в среднем>> возрастают, т.\,е.\
экстремальные события происходят все реже и реже, значения
$\gamma\hm<1$ соответствуют той ситуации, когда случайные величины~$\zeta_i$ 
<<в среднем>> убывают, т.\,е.\ экстремальные события
происходят все чаще и чаще, а значение $\gamma\hm=1$ соответствует той
ситуации, когда интенсивность потока экстремальных событий <<в
среднем>> постоянна, например в поведении интенсивности наблюдаются
проявления цикличности, причем периоды изменения интенсивности
заметно меньше периода фиксации наблюдений.

Теперь выберем нормирующие постоянные~$d_n$ так, чтобы
геометрическая случайная сумма $S_{N_n}$ имела нетривиальное
предельное распределение. Из леммы~1 вытекает, что если с учетом
выбранной формы постоянных~$b_n$ и соотношения~(\ref{e7-kor}) постоянные~$d_n$
выбрать в виде $d_n\hm=bp_n^{-\gamma}$, то для любого $y\hm>0$
\begin{multline*}
\lim\limits_{n\to\infty}{\sf P}\left(\fr{b_{N_n}}{d_n}<y\right)=
\lim\limits_{n\to\infty}{\sf P}\left((p_nN_n)^{\gamma}<y\right)={}\\
{}=
\lim\limits_{n\to\infty}{\sf P}\left(p_nN_n<y^{1/\gamma}\right)=
1-\exp\left\{-y^{1/\gamma}\right\}\,.
\end{multline*}
При этом согласно лемме~1 такое же распределение Вей\-бул\-ла--Гне\-ден\-ко
с показателем $1/\gamma$ является предельным и для геометрической
случайной суммы независимых неодинаково распределенных случайных
величин $S_{N_n}$, причем в силу непрерывности предельного
распределения Вей\-бул\-ла--Гне\-ден\-ко сходимость~(\ref{e6-kor}) равномерна по
$x\hm\in\r$. Оформим сказанное в виде следующего утверждения.

\smallskip

\noindent
\textbf{Теорема~1.} \textit{Предположим, что случайная величина $N_n$
имеет геометрическое распределение с па\-ра\-мет\-ром~$p_n$, причем
$p_n\hm\to0$ при $n\hm\to\infty$. Предположим, что существуют конечные
$\gamma\hm>0$ и $b\hm>0$ такие, что
$$
\fr{S_n}{bn^{\gamma}}\Longrightarrow 1\enskip (n\to\infty)\,.
$$
Тогда}
$$
\lim\limits_{n\to\infty}\sup\limits_{x\geqslant0}\left\vert{\sf  P}
\left(p_n^{\gamma}S_{N_n}\geqslant 
bx\right)-\exp\left\{-x^{1/\gamma}\right\}\right\vert=0\,.
$$


\subsection{Описание метода}

Итак, учитывая сделанные предположения о нормирующих постоянных,
можно заключить, что при достаточно больших значениях~$x_0$
\begin{multline} 
{\sf P}(T<t)\approx{}\\
{}\approx
1-\exp\left\{-[1-F(x_0)]\left(\fr{t}{b}\right)^{1/\gamma}\right\}\,,\enskip
t>0\,.\label{e10-kor}
\end{multline}

Применение описываемого метода вычисления временн$\acute{\mbox{ы}}$х
характеристик катастроф в неоднородных потоках экстремальных событий
заключается в следующем. Пусть $\epsilon\in(0,1)$~--- произвольное
число. Решение уравнения 
$$
{\sf P}(T<t)=\epsilon
$$ относительно~$t$ обозначим $t(\epsilon)$. Если распределение случайной величины~$T$
имеет вид~(\ref{e10-kor}), то, очевидно,
$$
t(\epsilon)=b\left[\fr{\ln(1-\epsilon)}{F(x_0)-1}\right]^{\gamma}.
$$

Смысл значения $t(\epsilon)$~--- это то время, вероятность
наступления катастрофы до которого равна~$\epsilon$. Из соображений
здравого смысла особый интерес представляют значения~$\epsilon$,
близкие к нулю (соответствующее значение $t(\epsilon)$~--- это то
время, до которого катастрофа, скорее всего, не наступит), близкие к
единице (соответствующее значение $t(\epsilon)$~--- это то время, до
которого катастрофа, скорее всего, наступит), а также
$\epsilon\hm=1/2$ (соответствующее значение $t(1/2)$~--- это
<<среднее>> время до наступления катастрофы).

Особо следует сказать, что при прогнозировании <<среднего>> или
<<ожидаемого>> времени до катастрофы можно использовать как медиану
$t(1/2)$ случайной величины~$T$, которая определяется как
решение уравнения
$$
1-\exp\left\{-[1-F(x_0)]\left(\fr{t}{b}\right)^{1/\gamma}\right\}=\fr{1}{2}
$$
относительно~$t$ и, очевидно, равна
$$
t\left(\fr{1}{2}\right)=b\left[\fr{\ln 2}{1-F(x_0)}\right]^{\gamma}\,,
$$
так и математическое ожидание
$$
{\sf E}T=\fr{b\Gamma(1+\gamma)}{[1-F(x_0)]^{\gamma}}.
$$
При этом необходимо отметить, что, например, в случае $\gamma\hm=1$
медиана $t(1/2)$ случайной величины~$T$ почти в полтора раза
(точнее, в $(\ln 2)^{-1}$ раз) меньше математического ожидания ${\sf E}T$.

При этом параметры $b$ и~$\gamma$ легко оценить методом наименьших
квадратов. Предположим, что в нашем распоряжении имеется выборка
$Z_1,Z_2,\ldots,Z_n$ предыду\-щих значений случайных величин~$\zeta_j$. 
Нормирующая функция $b_k\hm=bk^{\gamma}$ параметра~$k$ имеет
смысл тренда, или основной тенденции поведения реализации
$R_k\hm=Z_1+\cdots+Z_k$ случайной функции~$S_k$. С~целью линеаризации
регрессионной задачи прологарифмируем~$b_k$ и~$R_k$, обозначим
$\beta\hm=\log b$ и получим приближенные равенства
\begin{equation}
\log R_k\approx\beta+\gamma\log k\,,\enskip k=1,\ldots,n\,,\label{e11-kor}
\end{equation}
в правой части которых стоят линейные функции параметров~$\beta$ и~$\gamma$. 
Используя стандартный метод наименьших квадратов
оценивания параметров линейной регрессии~(\ref{e11-kor}), получим оценки
\begin{align*}
\gamma&\approx\widehat\gamma=\fr{n\sum\limits_{k=1}^n(\log k\cdot\log
R_k)-\log n!\sum\limits_{k=1}^n\log R_k}{n\sum\limits_{k=1}^n(\log k)^2-(\log n!)^2}\,;
\\
b&=\exp\{\beta\}\approx\exp\left\{\fr{1}{n}\left(\sum\limits_{k=1}^n\log
R_k-\widehat\gamma\log n!\right)\right\}\,.
\end{align*}

Чтобы получить оценку величины $1\hm-F(x_0)$, необходимо построить
разумную и адекватную парамет\-рическую математическую модель
(приближение) для функции $F(x)$. С~этой целью используем метод
построения асимптотических аппрокси\-маций для $F(x)$ при больших~$x_0$, 
основанный на теореме Бал\-ке\-ма\,--\,Пи\-канд\-са\,--\,Де Ха\-ана и
называемый методом превышений порога (POT-ме\-тод, POT\;=\;Peaks Over
Threshold).

Пусть случайная величина~$\zeta$ имеет функцию распределения $F(x)$.
В~рамках рассматриваемого метода прогнозирования катастроф как
превышений экстремальным процессом критических \mbox{уровней} большой
интерес представляет описание условного распределения превышения
случайной величиной~$\zeta$ некоторого (большого) порога~$u$:
$$
F_u(y)={\sf P}(\zeta-u < y|\zeta>u), 0\leqslant y \leqslant x_F-u\,,
$$
где $y=(x-u)$~--- превышение порога и $x_F\hm =\sup\{x \hm\in \mathbb{R} | F(x)\hm<1\} 
\hm\leqslant \infty$. Функция этого условного распределения $F_u$ может быть 
выражена через~$F$:

\noindent
$$
F_u(y)=\fr{F(u+y)-F(u)}{1-F(u)}=\fr{F(x)-F(u)}{1-F(u)}\,.
$$
Если порог $u$ достаточно велик, то большинство реализаций случайной
величины~$\zeta$ лежит между~0 и~$u$, так что оценить~$F$ в этом
промежутке несложно. Но оценить $F_u$ проблематично, так как
соответствующих наблюдений мало. На помощь приходит следующая
теорема.

\smallskip

\noindent
\textbf{Теорема~2}~\cite{BalkemaDeHaan1974, Pickands1975}. \textit{Функция
распределения~$F$ принадлежит области $\max$-при\-тя\-же\-ния
распределения, предельного для экстремальных значений тогда и только
тогда, когда существует измеримая функция $\sigma(u)\hm>0$ такая, что}
$$
\lim\limits_{u \to x_F}\sup\limits_{0\leqslant y < x_F-u}|F_u(y)-G_{\delta, \sigma(u)}(y)|=0\,,
$$
{\it где $G_{\delta, \sigma}(y)$~--- функция обобщенного распределения Парето:}
$$
G_{\delta, \sigma}(y)=\begin{cases}
1-\leqslant \left(1+\fr{\delta}{\sigma}\,y\right)^{-1/\delta}\,, &\delta \neq 0\,;\\
1-e^{-y/\sigma}\,, & \delta=0\,.
\end{cases}
$$


\smallskip

Условиям теоремы удовлетворяет большинство используемых на практике
распределений. Параметр~$\delta$ показывает, насколько тяжел хвост: чем больше~$\delta$, 
тем тяжелее хвост. Например, при моделировании финансовых данных 
обычно используется $\delta \hm\geqslant 0$.

\section{Пример применения метода прогнозирования вероятностных характеристик 
глобальных катастроф}

Для иллюстрации рассмотрим интересную задачу, связанную с
определением риска глобальных катаклизмов, вызванных столкновением
Земли с довольно большими небесными телами (астероидами, кометами).

Известно, что такие объекты приближаются к Земле относительно часто.
Исследования проводились на основании данных Центра по малым
планетам Гарвардского университета, представленных в книге Остина
Аткинсона~\cite{Atkinson2001}. Это таблица, в которой содержатся
предсказания дат приближения к Земле на расстояние не более 0,2
астрономической единицы (а.е.)\ комет и малых планет на
ближайшие 33~года начиная с июня 1999~г. Всего таких предсказаний~191, 
для каждого из них известно минимальное расстояние, на которое
малая планета подойдет к Земле, год и месяц предполагаемого
сближения. 
Приводимые ниже вычисления проведены на основе метода,
предложенного в данной статье, и уточняют результаты, приведенные в~[2--5]. 
Уточнение достигнуто за счет того, что здесь используется не
модель экспоненциального распределения, а более гибкая модель
распределения Вей\-бул\-ла--Гне\-денко.

Расстояние, на которое очередной ($i$-й) космический объект
приблизится к Земле, будем считать реализацией случайной величины~$Q_i$, 
распределение которой, вообще говоря, неизвестно и подлежит
определению (оцениванию). При этом, в отличие от рассматривавшейся
ранее формальной модели экстремального процесса, интерес
представляет не {\it максимальное}, а \textit{минимальное} значение
величин~$Q_i$. Снимем формальные противоречия, полагая
$X_i\hm=Q_i^{-1}$.

Формализуем сказанное. Имеем выборку
$\mathbf{X}\hm=\{X_1,X_2,\ldots ,X_n\}\,,\enskip n=191,$
 независимых одинаково
распределенных случайных величин. Эти величины обратны расстояниям
между Землей и потенциально опасными астероидами. Предполагается,
что известны расстояния от центра Земли до центров астероидов. Все
подсчеты ведутся в а.е., 1~а.е.\;=\;149,6~млн~км.
Радиус~$R$ Земли равен $R\hm=6400$~км\;$\approx 4{,}278075\cdot
10^{-5}$~а.е.\;=\;0,00004278075~а.е. 

Как уже говорилось,
в указанной книге~\cite{Atkinson2001} приведены данные лишь о тех
потенциально опасных астероидах, которые приближаются к Земле менее
чем на 0,2~а.е. 
Таким образом, наблюдается не <<полный>>
набор величин $X_1,X_2,\ldots$, а лишь те из них, которые
превосходят $(0{,}2)^{-1}=5$~(a.e.)$^{-1}$. Будем считать, что
порог $u\hm=5$~a.e.$^{-1}$ достаточно велик, чтобы аппроксимация,
устанавливаемая теоремой Бал\-ке\-мы\,--\,Пи\-канд\-са\,--\,Де Ха\-ана для
распределения превышений такого порога, была достаточно адекватна,
так что в качестве модели распределения случайных величин~$X_i$
можно взять обобщенное распределение Парето
$$
F(x;\alpha,\sigma)=\begin{cases} 
0\,, & \mbox{если}\ x<5\,;\\ 
1-\fr{C}{(x-\alpha)^\sigma}\,, & \mbox{если}\ x\geqslant 
5\,.
\end{cases}
$$
Как показали вычисления, проведенные в работах~\cite{Korolevetal2006, Korolevetal2007}, 
эта модель действительно
демонстрирует высочайшее согласие с указанными данными. При этом
критический порог~$x_0$, превышение которого случайной величиной~$X_i$ 
означает катастрофу (столкновение астероида с Землей), равен
$$
x_0=\fr{1}{R}=\fr{1}{6400}\ \mbox{км}^{-1}=23374{,}9993~\mbox{а.е.}^{-1}\,.
$$

Для статистического оценивания параметров $C$, $\alpha$ и~$\sigma$ в
работах~\cite{Korolevetal2006, Korolevetal2007} использовалось
несколько методов, но наилучшее согласие данных наблюдалось с
моделью, построенной на основе оценок максимального правдоподобия:
 $$ \widehat\alpha=-3{,}165\,;\ \
\widehat\sigma=2{,}37\,;\ \ C=96{,}757\,. 
$$



Оценки наименьших квадратов для пара\-мет\-ров $b$ и~$\gamma$,
построенные по выборке $Z_1,\ldots,Z_{191}$ временн$\acute{\mbox{ы}}$х
промежутков между экстремальными сближениями астероидов с Землей оказались равными
$$
b\approx 0{,}1728 \mbox{ года}\approx 2{,}0733 \mbox{ мес.};\ \ \ \
\gamma\approx 1{,}0012\,.
$$



В результате применения описанного метода к вычислению оценок
временн$\acute{\mbox{ы}}$х характеристик катастрофы, связанной со столкновением
Земли с астероидом, получены следующие значения.
\begin{itemize}
\item
время $\underline t$, до которого с вероятностью 0,9999
столкновение Земли с астероидом не про\-изойдет, примерно равно 1235~годам;
\item время $\overline t$, до которого с вероятностью 0,9999
столкновение Земли с астероидом заведомо произойдет, примерно равно
111~154~073 годам;
\item <<среднее>> время $t^*$ до столкновения Земли с астероидом
примерно равно 12\,071\,039~годам (при этом в качестве ожидаемого
времени катастрофы использовалось математическое ожидание).
\end{itemize}

\vspace*{-12pt}

{\small\frenchspacing
{%\baselineskip=10.8pt
\addcontentsline{toc}{section}{Литература}
\begin{thebibliography}{99}

\bibitem{KorolevSokolov2005} 
\Au{Королев В.\,Ю., Соколов И.\,А.} Некоторые вопросы анализа
катастрофических рисков, связанных с неоднородными потоками
экстремальных событий~// Систе\-мы и средства информатики. Спец. вып.
Математические методы и модели информатики. Стохастические
технологии и сис\-те\-мы.~--- М.: ИПИ РАН, 2005. С.~109--125.

\bibitem{Korolevetal2006} %1
\Au{Королев В.\,Ю., Соколов~И.\,А., Гордеев~А.\,С.,
Григорь\-ева~М.\,Е., Попов~С.\,В., Чебоненко~Н.\,А.} Некоторые методы
анализа временн$\acute{\mbox{ы}}$х характеристик катастроф в неоднородных
потоках экстремальных событий~// Сис\-те\-мы и средства информатики.
Спец. вып. Математические методы в информационных технологиях.~---
М.: ИПИ РАН, 2006. С.~5--23.

\bibitem{Korolevetal2007}  %2
\Au{Королев В.\,Ю., Соколов~И.\,А., Гордеев~А.\,С.,
Григорьева~М.\,Е., Попов~С.\,В., Чебоненко~Н.\,А.} Некоторые методы
прогнозирования временн$\acute{\mbox{ы}}$х характеристик рисков, связанных с
катастрофическими событиями~// Актуарий, 2007. №\,1. С.~34--40.

\bibitem{KorolevSokolov2008} %3
\Au{Королев В.\,Ю., Соколов~И.\,А.} Математические
модели неоднородных потоков экстремальных событий.~--- М.: ТОРУС
ПРЕСС, 2008. 200~с.

\bibitem{KorolevShorgin2011} %4
\Au{Королев В.\,Ю., Шоргин С.\,Я.}
Математические методы анализа стохастической структуры
информационных потоков.~--- М.: ИПИ РАН, 2011. 130~с.



\bibitem{Korolevetal2013} %5
\Au{Королев В.\,Ю., Черток А.\,В., Корчагин~А.\,Ю., Горшенин~А.\,К.}
Ве\-ро\-ят\-но\-ст\-но-ста\-ти\-сти\-че\-ское моделирование информационных потоков в
сложных финансовых сис\-те\-мах на основе высокочастотных данных~//
Информатика и её применения, 2013. Т.~7. Вып.~1. С.~12--21.

\bibitem{Kalashnikov1997} %6
\Au{Kalashnikov~V.} Geometric sums: Bounds for rare events
with applications.~--- Dordrecht--Boston--London: Kluwer Academic
Publs., 1997. 288~p.

\bibitem{KorolevBeningShorgin2011}  %7
\Au{Королев В.\,Ю., Бенинг В.\,Е., Шоргин~С.\,Я.} 
Математические основы теории риска.~--- 2-е изд., перераб. и дополн.~--- М.: Физматлит, 2011.
620~с.

\bibitem{Korolev1994}  %8
\Au{Королев В.\,Ю.} Сходимость случайных последовательностей с независимыми
случайными индексами. I~// Теоpия веpоятностей и ее пpименения,
1994. Т.~39. Вып.~2. С.~313--333.

%\bibitem{GavrilenkoKorolev2010} Гавриленко С. В., Королев В. Ю. Об оценках
%вероятности разорения страховой компании, резерв которой описывается
%классическим процессом риска // Статистические методы оценивания и
%проверки гипотез. Пермь: изд-во Пермского гос. ун-та, 2010. Вып. 22.
%С. 134--143.

%\bibitem{Gavrilenko2010} Гавриленко С. В. Оценки скорости сходимости в
%предельных теоремах со случайным индексом и некоторые их применения.
%Дис. канд. физ.-матем. наук. -- Москва: МГУ им. М. В. Ломоносова,
%2010.

\bibitem{BalkemaDeHaan1974}  %9
\Au{Balkema A., de Haan~L.} Residual life time at great age~// Ann. Probab., 1974. 
Vol.~2. P.~792--804.

\bibitem{Pickands1975} %10
\Au{Pickands J.} Statistical inference using extreme order
statistics~// Ann. Stat., 1975. Vol.~3. P.~119--131.

\bibitem{Atkinson2001} %11
\Au{Аткинсон О.} Столкновение с Землей~/
Пер с англ.~--- СПб.: Ам\-фо\-ра/Эв\-ри\-ка, 2001. 400~с.
(\Au{Atkinson~O.} Impact Earth: Asteroids, comets and meteors~--- the growing threat.~---
Virgin Publ., 1999. 256~p.)
\end{thebibliography} } }



\end{multicols}

\hfill{\small\textit{Поступила в редакцию 20.10.13}}




%\vspace*{6pt}

%\hrule

%\vspace*{2pt}

%\hrule

\newpage



\def\tit{A LIMIT THEOREM FOR GEOMETRIC SUMS OF~INDEPENDENT NONIDENTICALLY DISTRIBUTED RANDOM VARIABLES AND~ITS~APPLICATION 
TO~THE~PREDICTION OF~THE~PROBABILITIES 
OF~CATASTROPHES IN~NONHOMOGENEOUS FLOWS OF~EXTREMAL EVENTS}

\def\aut{M.\,E.~Grigor'eva$^1$, V.\,Yu.~Korolev$^{2,3}$, and~I.\,A.~Sokolov$^3$}
\def\autkol{M.\,E.~Grigor'eva, V.\,Yu.~Korolev, and~I.\,A.~Sokolov}
\def\titkol{A limit theorem for geometric sums of~independent nonidentically 
distributed random variables and~its~application} %to~the~prediction of~the~probabilities  of~catastrophes in~nonhomogeneous flows of~extremal events}


\titel{\tit}{\aut}{\autkol}{\titkol}

\vspace*{-9pt}

\noindent
$^1$Parexel International, Moscow 121609, Russian Federation\\
\noindent
$^2$Faculty of Computational Mathematics and Cybernetics, M.\,V.~Lomonosov Moscow
State University, Moscow\linebreak
$\hphantom{^1}$119991, Russian Federation\\
\noindent
$^3$Institute of Informatics 
Problems, Russian Academy of Sciences, Moscow 119333, Russian Federation

\vspace*{9pt}

\def\leftfootline{\small{\textbf{\thepage}
\hfill INFORMATIKA I EE PRIMENENIYA~--- INFORMATICS AND APPLICATIONS\ \ \ 2013\ \ \ volume~7\ \ \ issue\ 4}
}%
 \def\rightfootline{\small{INFORMATIKA I EE PRIMENENIYA~--- INFORMATICS AND APPLICATIONS\ \ \ 2013\ \ \ volume~7\ \ \ issue\ 4
\hfill \textbf{\thepage}}}

\Abste{The problem of prediction of the probabilities of catastrophes 
in nonhomogeneous flows of extremal events is considered. The paper develops and 
generalizes some methods proposed by the authors in their previous
works. The flow 
of extremal events is considered as a marked point stochastic process with not 
necessarily identically distributed intervals between points (events). The proposed 
generalizations are based on limit theorems for geometric sums of independent not 
necessarily identically distributed random variables and the Balkema\,--\,Pickands\,--\,De Haan 
theory. Within the framework of the construction under consideration, the Weibull--Gnedenko 
distribution appears as a limit law for geometric sums of independent not necessarily 
identically distributed random variables. The efficiency of the proposed methods is 
illustrated by the example of their application to the problem of prediction the time 
of the impact of the Earth with a potentially dangerous asteroid based on the data of 
the IAU (International Astronomical Union)
Minor Planet Center.}

\KWE{catastrophe; extremal event; random sum; geometric sum; law of large numbers; 
Weibull--Gnedenko distribution; Balkema\,--\,Pickands\,--\,De Haan theorem; 
generalized Pareto distribution}

\DOI{10.14357/19922264130402}

%\vspace*{3pt}

\Ack
\noindent
The research was supported by the Russian Foundation for Basic Research (Projects 
Nos.\,11-01-00515-а, 11-07-00112-а, and 12-07-00115-а).


  \begin{multicols}{2}

\renewcommand{\bibname}{\protect\rmfamily References}
%\renewcommand{\bibname}{\large\protect\rm References}
%\vspace*{12pt}

{\small\frenchspacing
{%\baselineskip=10.8pt
\addcontentsline{toc}{section}{References}
\begin{thebibliography}{99}

\bibitem{1-kgr} 
\Aue{Korolev, V.\,Yu., and I.\,A.~Sokolov}. 2005. 
Nekotorye voprosy analiza katastroficheskikh riskov, svyazannykh s neodnorodnymi 
potokami ekstremal'nykh sobytiy [Some problems of the analysis of catastrophic 
risks related to \mbox{nonhomogeneous} flows of extremal events]. 
\textit{Sistemy i sredstva informatiki. Spetsial'nyy vypusk 
``Matematicheskie metody v informatsionnykh tekhnologiyakh''}
[\textit{Systems and means of informatics. 
Special issue ``Mathematical methods and models of informatics''}]. 
Moscow: IPI RAN. 109--125.

\bibitem{2-kgr} %1
\Aue{Korolev, V.\,Yu., I.\,A.~Sokolov, A.\,S.~Gordeev, M.\,E.~Grigor'eva, 
S.\,V. ~Popov, and N.\,A.~Chebonenko}. 
2006. Nekotorye metody analiza vremennykh kharakteristik katastrof v 
neodnorodnykh potokakh ekstremal'nykh sobytiy [Some methods for the 
analysis of temporal characteristics of catastrophes in nonhomogeneous 
flows of extremal
 events]. 
\textit{Sistemy i sredstva informatiki. Spetsial'nyy vypusk}\linebreak\vspace*{-12pt} 

\columnbreak

\noindent
\textit{``Matematicheskie metody v informatsionnykh tehnologiyakh''}
[\textit{Systems and means of informatics. Spetsial issue ``Mathematical 
methods in information technologies''}]. Moscow: IPI RAN. 5--23.
\bibitem{3-kgr} %2
\Aue{Korolev, V.\,Yu., I.\,A.~Sokolov, A.\,S.~Gordeev, M.\,E.~Gri\-gor'\-eva, 
S.\,V.~Popov, and N.\,A.~Chebonenko}. 2007. 
Nekotorye metody prognozirovaniya vremennykh kha\-rak\-te\-ri\-stik riskov, svyazannykh s 
katastroficheskimi sobytiyami 
[Some methods for the prediction of the temporal characteristics of risks related 
to catastrophic events]. \textit{Aktuariy} [\textit{Actuary}] 1:34--40.
\bibitem{4-kgr}%3
\Aue{Korolev, V.\,Yu., and I.\,A.~Sokolov}. 2008. 
\textit{Matema\-ti\-che\-skie modeli neodnorodnykh potokov ekstremal'nykh sobytiy} 
[\textit{Mathematical models of nonhomogeneous flows of extremal events}]. Moscow: TORUS PRESS.
200~p.
\bibitem{5-kgr} %4
\Aue{Korolev, V.\,Yu., and S.\,Ya.~Shorgin}. 2011. 
\textit{Matematicheskie metody analiza stokhasticheskoy struktury infor\-ma\-tsi\-on\-nykh potokov} 
[\textit{Mathematical methods for the analysis of the}\linebreak\vspace*{-12pt}

\pagebreak

\noindent
\textit{stochastic structure of information 
flows}]. Moscow: IPI RAN. 130~p.


\bibitem{7-kgr} %5
\Aue{Korolev, V.\,Yu., A.\,V.~Chertok, A.\,Yu.~Korchagin, and A.\,K.~Gorshenin}. 
2013. Veroyatnostno-statisticheskoe\linebreak
modelirovanie informatsionnykh potokov v 
slozhnykh finansovykh sistemakh na osnove vysokochastotnykh\linebreak
dannykh [Probability 
and statistical modeling of information flows in complex financial systems based on 
high-frequency data]. \textit{Informatika i ee Primeneniya~--- Inform. Appl.} 7(1):12--21.
\bibitem{8-kgr} %6
\Aue{Kalashnikov, V.} 1997.
\textit{Geometric sums: Bounds for rare events with applications}. 
Dordrecht--Boston--London: Kluwer Academic Publs.  288~p.
\bibitem{9-kgr} %7
\Aue{Korolev, V.\,Yu., V.\,E.~Bening, and S.\,Ya.~Shorgin}.
2011. \textit{Matematicheskie osnovy teorii riska}  
[\textit{Mathematical foundations of risk theory}]. 2nd ed. Moscow: Fizmatlit. 620~p.
\bibitem{10-kgr} %8
\Aue{Korolev, V.\,Yu.} 1994. 
Convergence of random sequences with the independent random indices. I. 
\textit{Theory Probab.  Appl.} 39(2):282--297.
\bibitem{11-kgr} %9
\Aue{Balkema, A., and L.~de Haan}. 1974. Residual life time at great age.
\textit{Ann. Probab.} 2:792--804. 

 
\bibitem{12-kgr} %10
\Aue{Pickands, J.}  1075.
Statistical inference using extreme order statistics.
\textit{Ann. Stat.} 3:119--131.



\bibitem{6-kgr}   %11
\Aue{Atkinson,~A.} 1999. \textit{Impact Earth: Asteroids, comets and meteors~--- the growing threat}.
Virgin Publ. 256~p.

\end{thebibliography}
} }

\end{multicols}

\hfill{\small\textit{Received October 20, 2013}}

\Contr

\noindent
\textbf{Grigorieva Maria E.} (b.\ 1986)~--- biostatistician II, 
Parexel International, Moscow 121609, Russian Federation;  maria-grigoryeva@yandex.ru

\vspace*{3pt}

\noindent
\textbf{Korolev Victor Yu.} (b.\ 1954)~--- Doctor of Science in 
physics and mathematics, professor, Department of Mathematical Statistics, 
Faculty of Computational Mathematics and Cybernetics, M.\,V.~Lomonosov
 Moscow State University; Moscow 119991, Russian Federation;
 leading scientist, Institute of Informatics Problems, Russian 
Academy of Sciences, Moscow 119333, Russian Federation; victoryukorolev@yandex.ru 

\vspace*{3pt}

\noindent
\textbf{Sokolov Igor A.} (b.\ 1954)~--- Academician of the Russian Academy of Sciences, 
Doctor of Science in technology, Director, Institute of Informatics Problems, 
Russian Academy of Sciences, Moscow 119333, Russian Federation;  isokolov@ipiran.ru 

 \label{end\stat}

\renewcommand{\bibname}{\protect\rm Литература}   %2
\def\stat{pechinkin}


\def\tit{СОВМЕСТНОЕ СТАЦИОНАРНОЕ РАСПРЕДЕЛЕНИЕ
ЧИСЛА ЗАЯВОК В~НАКОПИТЕЛЕ И~В~БУНКЕРЕ
ПЕРЕУПОРЯДОЧЕНИЯ В~МНОГОКАНАЛЬНОЙ СИСТЕМЕ
ОБСЛУЖИВАНИЯ С~ПЕРЕУПОРЯДОЧЕНИЕМ
ЗАЯВОК$^*$}


\def\titkol{Совместное стационарное распределение
числа заявок в~накопителе и~в~бункере
переупорядочения} % в~многоканальной системе обслуживания с~переупорядочением заявок}

\def\aut{\fbox{А.\,В.\~Печинкин}$^1$, Р.\,В.~Разумчик$^2$}

\def\autkol{А.\,В.\~Печинкин, Р.\,В.~Разумчик}

\titel{\tit}{\aut}{\autkol}{\titkol}

{\renewcommand{\thefootnote}{\fnsymbol{footnote}} \footnotetext[1]
{Работа выполнена при частичной поддержке РФФИ (проект 13-07-00223).}}


\renewcommand{\thefootnote}{\arabic{footnote}}
\footnotetext[1]{Институт проблем информатики Российской академии наук}
\footnotetext[2]{Институт проблем информатики Российской академии наук; Российский
университет дружбы народов, rrazumchik@ieee.org}

%\vspace*{3pt}

\Abst{Рассматривается функционирующая в~непрерывном времени
многоканальная система обслуживания с~накопителем
бесконечной емкости и переупорядочением заявок.
В~систему поступает пуассоновский поток заявок, время
обслуживания каждым прибором распределено по
экспоненциальному закону с~одним и~тем же параметром.
При поступлении в~систему всем заявкам  присваивается
порядковый номер. На выходе из системы сохраняется
порядок между заявками, установленный при входе в~нее.
Заявки, завершившие обслуживание и~нарушившие установленный порядок,
накапливаются на выходе системы
в~бункере переупорядочения (БП), который также имеет неограниченную емкость.
Найдено совместное стационарное распределение
числа заявок в~накопителе и~суммарного числа
заявок в~БП в~терминах
вычислительных алгоритмов и~производящих функций (ПФ).
Приведены примеры расчетов по полученным
соотношениям.}

\KW{многолинейная система массового обслуживания;
переупорядочение; стационарное распределение
числа заявок}

\DOI{10.14357/19922264140401}


%\vspace*{3pt}

\vskip 12pt plus 9pt minus 6pt

\thispagestyle{headings}

\begin{multicols}{2}

\label{st\stat}


\section{Введение}

Для функционирования ряда
ин\-фор\-ма\-ци\-он\-но-те\-ле\-ком\-му\-ни\-ка\-ци\-он\-ных сис\-тем
и для предоставления на их основе услуг
необходимо соблюдение\linebreak требования сохранения порядка в~потоке передаваемых сообщений.
Различные действия, необходимые для этого, можно объединить
в~одно понятие~--- переупорядочение.
Для изучения влияния\linebreak \mbox{переупорядочения} на качество
функционирования ин\-фор\-ма\-ци\-он\-но-те\-ле\-ком\-му\-ни\-ка\-ци\-он\-ных
сис\-тем к~настоящему времени предложено множество
моделей, которые в~своей основе используют методы
и~модели теории массового обслуживания.
Исследуемая сис\-те\-ма обычно представляется в~виде
системы или сети массового обслуживания с одним\linebreak или
несколькими входящими потоками сообщений.
Эффект переупорядочения часто моделируется с~помощью
дополнительной очереди (БП),
в~которую попадают сообщения, обработанные\linebreak в~системе,
и~ожидают там до тех пор, пока порядок следования сообщений
нельзя будет восстановить.
Некоторый обзор работ в~этом направлении можно найти
в~\cite{a1, a2},
а~некоторые последние результаты~--- в~[3--8].

Настоящая работа является развитием \cite{a8}, в~которой
рассматривается система массового обслуживания (СМО)
с~переупорядочением в~виде марковской многоканальной
системы обслуживания неограниченной емкости и~бункером
переупорядочения, также имеющим неограниченную
емкость.
В~\cite{a8} была получена система уравнений равновесия для
совместного стационарного распределения чис\-ла заявок в~системе
и~бункере переупорядочения и~приведены некоторые результаты
численных расчетов.
Однако несомненный интерес представляют
две задачи, не освещенные в~\cite{a8}, которые и~являются
предметом данной статьи, а~именно:
разработка рекуррентного алгоритма расчета вышеупомянутого
совместного стационарного распределения и~нахождение
этого распределения в~терминах ПФ.

Статья организована таким образом.
В~разд.~2 приводится подробное описание
системы.
В~разд.~3 дается рекуррентный алгоритм расчета
совместного стационарного распределения, а~в~разд.~4
показано, как совместное стационарное распределение
можно найти в~терминах ПФ.
Примеры расчетов, проведенных по формулам разд.~4,
представлены в~разд.~5.
В~заключении сформулированы основные результаты работы.

\section{Описание системы}

Рассмотрим функционирующую в~непрерывном времени
$N$-ли\-ней\-ную ($N\hm\ge 2$) СМО с накопителем
неограниченной емкости, входящим пуассоновским
потоком заявок интенсивности~$\lambda$ \mbox{и~экспоненциальным}
распределением времени
обслуживания заявки каждым прибором с~па\-ра\-мет\-ром~$\mu$.


При поступлении в~систему всем заявкам  присваивается
порядковый номер.
На выходе из СМО сохраняется порядок между заявками,
установленный при входе в~нее.
Заявки, завершившие обслуживание и~нарушившие
установленный порядок, накапливаются на выходе
системы в~БП и~покидают СМО только
после того, как закончится обслуживание всех заявок с~меньшими номерами.
Такая СМО носит название системы с переупорядочением
заявок.

Предполагается также выполнение необходимого и~достаточного условия
существования стационарного режима функционирования СМО
$$\tilde {\rho}\hm=\fr{\rho}{N}<1\,,
$$
 где $\rho\hm=\lambda/\mu$.

\vspace*{-9pt}

\section{Алгоритм нахождения совместного стационарного распределения}

Предположим, что на приборах находится $n$, $n\hm=\overline{1,N}$, заявок.
Тогда заявкой первого уровня будем называть ту из них,
которая в~систему поступила последней, второго уровня~--- предпоследней,
$\ldots,$ $n$-го уровня~--- первой. При этом если $n\hm=N$ (все приборы
заняты), то находящиеся в~БП заявки, поступившие между заявками
второго и~первого уровней, будем называть заявками первой очереди,
заявки, поступившие между заявками третьего и~второго уровней,~---
заявками второй очереди, $\ldots,$ заявки, поступившие между
заявками $N$-го и~$(N-1)$-го уровней,~--- заявками $(N-1)$-й
очереди. Если же $n<N$, то  заявками первой очереди будем называть
заявки из БП, поступившие после заявки первого уровня, заявками
второй очереди~--- заявки, поступившие между заявками второго и~первого уровней,
и~т.\,д.

При $n\ge N$ обозначим через
$p^{(m)}_{n;i}$, ${m\hm=\overline{1,N-1}}$, ${i\hm\ge 0}$,
стационарную вероятность того, что в~системе на
приборах и~в накопителе находится~$n$~заявок,
а~в~БП имеется в~сумме~$i$~заявок первой,
второй, $\ldots,$ $m$-й очереди.
Через
$p^{(m)}_{n;i}$, ${m\hm=\overline{1,n}}$, ${i\hm\ge 0}$,
обозначим аналогичную стационарную вероятность
при $n\hm=\overline{1,N-1}$.
Через~$p_n$, $n\hm\ge 0$, обозначим
стационарную вероятность того, что в~системе на
приборах и~в накопителе (без учета числа заявок в~БП) находится~$n$~заявок.
Очевидно, что стационарные вероятности~$p_n$
определяются теми же самыми формулами, что и~в~обычной
марковской СМО $M/M/N/\infty$
(см., например,~\cite{boch}):
\begin{align}
p_{0} &= \left( \sum\limits_{i=0}^{N-1} \fr{\rho^i}{i!} +
\fr{\rho^N}{(N-1)! (N-\rho)}
\right)^{-1} \,;\label{3-1}
\\
p_{i} &= \begin{cases}
\fr{\rho^i }{i!} p_{0}\,, &\ i=\overline{1,N}\,,
\\
%\label{3-3}
\fr{\rho^i}{N!\, N^{i-N}} p_{0}
= \tilde \rho^{i-N} p_{N}\,, &\ i\ge N+1\,.
\end{cases}
\label{3-2}
\end{align}

Наконец, через $p_{n;i}$, ${n\hm\ge 1}$, ${i\hm\ge 0}$, обозначим
стационарную вероятность того, что в~системе на
приборах и~в накопителе находится~$n$~заявок,
а~в~БП~--- $i$~заявок.

Используя принцип глобального баланса, можно выписать систему уравнений для
вероятностей~$p^{(m)}_{n;i}$.
Для вероятностей $p^{(1)}_{n;i}$, $n\hm\ge N$,
$i \hm\ge 0$, справедливы уравнения:
\begin{align}
\hspace*{-2.8mm}p^{(1)}_{n;0} (\lambda+N\mu) &= p^{(1)}_{n-1;0} \lambda +
p_{n+1} (N-1) \mu \,,\ n\ge N;
\!\!\label{eq-1-1}
\\
\hspace*{-2.8mm}p^{(1)}_{n;i} (\lambda+N\mu) &= p^{(1)}_{n-1;i} \lambda +
p^{(1)}_{n+1;i-1} \mu \,,\notag\\
&\hspace*{25mm} n\ge N\,,\enskip i \ge 1\,.
\label{eq-1-2}\!\!
\end{align}
%%%%%%%%%%%%%%%%%%%%%%%
%%%%%%%%%%%%%%%%%%%%%%%
Для вероятностей $p^{(1)}_{N-1;i}$,\ \ $i \ge 0$,
справедливы уравнения:
%%%%%%%%%%%%%%%%%%%
\begin{align}
\label{eq-1-3}
p^{(1)}_{N-1;0} [\lambda+(N-1)\mu] &=
p_{N-2} \lambda + p_{N} (N-1)\mu\,;
\\
\label{eq-1-4}
p^{(1)}_{N-1;i} [\lambda+(N-1)\mu] &= p^{(1)}_{N;i-1} \mu\,,\enskip i \ge 1\,.
\end{align}
Для вероятностей
$p^{(1)}_{n;i}$, $n\hm=\overline{1,N-2}$, $i \hm\ge 0$,
справедливы уравнения
\begin{align}
\label{eq-1-5}
\hspace*{-2mm}p^{(1)}_{n;0} (\lambda+n\mu) &= p_{n-1} \lambda +
p^{(1)}_{n+1;0} n\mu ,\  n=\overline{1,N-2};
\\
\label{eq-1-6}
\hspace*{-2mm}p^{(1)}_{n;i} (\lambda+n\mu) &= p^{(1)}_{n+1;i} n\mu
+ p^{(2)}_{n+1;i-1} \mu \,,\notag\\
&\hspace*{15mm}n=\overline{1,N-2},\ \ i \ge 1.
\end{align}


Для остальных вероятностей
$p^{(m)}_{n;i}$, $m\hm=\overline{2,N-1}$, справедливы формулы:
\begin{align}
p^{(m)}_{n;0} (\lambda+N\mu) &= p^{(m)}_{n-1;0} \lambda +
p^{(m-1)}_{n+1;0} (N-m) \mu\,,\notag\\
& \hspace*{30mm}n\ge N\,; \label{bat-1}
\\
p^{(m)}_{n;i} (\lambda+N\mu) &= p^{(m)}_{n-1;i} \lambda +
p^{(m-1)}_{n+1;i} (N-m) \mu +{}\notag\\
&\hspace*{-10mm}{}+p^{(m)}_{n+1;i-1} m \mu \,,\enskip
n\ge N\,,\ \ i\ge 1\,;
\label{bat-2}
\end{align}

\noindent
\begin{align}
p^{(m)}_{N-1;0} [\lambda+(N-1)\mu] &={}\notag\\
{}=p^{(m-1)}_{N-2;0} \lambda
&{}=+ p^{(m-1)}_{N;0} (N-m) \mu \,;
\label{bat-3}
\end{align}

\noindent
\begin{multline}
p^{(m)}_{N-1;i} [\lambda+(N-1)\mu] =p^{(m-1)}_{N-2;i} \lambda+{}\\
{}+
p^{(m-1)}_{N;i} (N-m) \mu +p^{(m)}_{N;i-1} m \mu\,,\enskip i\ge 1\,;
\label{bat-4}
\end{multline}

\vspace*{-12pt}



\noindent
\begin{multline}
\label{bat-5}
p^{(m)}_{n;0} (\lambda+n\mu) = p^{(m-1)}_{n-1;0} \lambda+
p^{(m)}_{n+1;0} (n-m+1) \mu \,,\\
 n=\overline{m,N-2}\,;
\end{multline}

\noindent
\begin{multline}
\label{bat-6}
p^{(m)}_{n;i} (\lambda+n\mu) = p^{(m-1)}_{n-1;i} \lambda
+
p^{(m)}_{n+1;i} (n-m+1) \mu +{}\\
{}+ p^{(m+1)}_{n+1;i-1} m \mu\,,\enskip
 n=\overline{m,N-2}\,,\ \ i\ge 1\,.
\end{multline}

Решение данной системы уравнений позволяет
найти совместное стационарное распределение
$p_{n;i}$ числа заявок на приборах и~в
накопителе и~суммарного числа заявок в~БП в~виде следующих ра\-венств:
\begin{alignat*}{2}
%\label{bat-7}
p_{n;i} &= p^{(N-1)}_{n;i}\,, &\quad  n&\ge N\,,\ \ i\ge 0\,,
\\
%\label{bat-8}
p_{n;i} &= p^{(n)}_{n;i} \,, &\quad n&=\overline{1,N-1}\,,\ \ i\ge 0\,.
\end{alignat*}

Анализ системы~\eqref{eq-1-1}--\eqref{bat-6}
показал, что вычисление стационарных
вероятностей $p^{(m)}_{n;i}$ можно проводить
рекуррентным образом по следующему алгоритму.

\bigskip

\noindent
А\,л\,г\,о\,р\,и\,т\,м~1\ (\textbf{Алгоритм решения системы уравнений равновесия}).

\noindent
\textit{Задать} $\lambda$, $\mu$ и $n$.

\noindent
\textit{Для $n\ge 0$ рассчитать $p_{n}$ по
формулам}~\eqref{3-1} и~\eqref{3-2}.

\noindent
\textit{Рассчитать $p^{(1)}_{N-1;0}$ по формуле}~\eqref{eq-1-3}.

\noindent
\textit{Для $n\ge N$ рассчитать $p^{(1)}_{n;0}$ по
формуле}~\eqref{eq-1-1}.

\noindent
\textit{Для $i\ge1$}


\textit{рассчитать $p^{(1)}_{N-1;i}$ по формуле}~\eqref{eq-1-4}.

\textit{для $n\ge N$ рассчитать $p^{(1)}_{n;i}$ по формуле}~\eqref{eq-1-2}.

\noindent
\textit{Для $n=\overline{N-2,1}$ рассчитать $p^{(1)}_{n;0}$
по формуле}~\eqref{eq-1-5}.

\noindent
\textit{Для $m=\overline{2,N-1}$}

\textit{рассчитать $p^{(m)}_{N-1;0}$ по формуле}~\eqref{bat-3}.


\textit{для $n\ge N$ рассчитать $p^{(m)}_{n;0}$
   по формуле}~\eqref{bat-1};

\textit{для} $i\hm\ge1$

    \hspace*{9pt}\textit{рассчитать $p^{(1)}_{N-m;i}$ по
    формуле}~\eqref{eq-1-6};


    \hspace*{9pt}\textit{если $m \ne 2$, для}  $j\hm=\overline{2,m-1}$ \textit{рассчитать}\linebreak\vspace*{-12pt}

 \hspace*{9pt}\textit{$p^{(j)}_{N-m+j-1;i}$ по формуле}~\eqref{bat-6};

\hspace*{9pt}\textit{рассчитать $p^{(m)}_{N-1;i}$ по формуле}~\eqref{bat-4};

\hspace*{9pt}\textit{для $n\ge N$ рассчитать $p^{(m)}_{n;i}$
    по формуле}~\eqref{bat-2};

\textit{если {$m \ne N-1$}, для $m\hm=\overline{N-2,m}$
   рассчитать}\linebreak

   \textit{$p^{(m)}_{n;0}$ по формуле}~\eqref{bat-5}.

\bigskip

В~связи с~тем, что вычисление моментов после расчета
вероятностей по представленному алгоритму
может давать погрешности, в~следующем разделе
находятся формулы для совместного стационарного
распределения в~терминах ПФ.


\section{Использование производящих функций}

Система уравнений~\eqref{eq-1-1}--\eqref{bat-6}
допускает также решение с~помощью ПФ.
Для нахождения этого решения положим
\begin{equation*}
\label{f-m}
f_m(u,z) = \lambda u^2 - (\lambda + N\mu) u + m \mu z\,,\
 m=\overline{1,N-1}\,.
\end{equation*}

Обозначим через $u_m\hm=u_m(z)$, $m\hm=\overline{1,N-1}$,
минимальное решение уравнения
$$
f_m(u,z) = 0\,,
$$
т.\,е.
\begin{equation*}
%\label{sqrt}
u_m = \fr{\lambda + N\mu - \sqrt{(\lambda + N\mu)^2 - 4 m \lambda \mu z}}
{2 \lambda }\,.
\end{equation*}


Введем ПФ
\begin{multline*}
P^{(m)}_{n}(z) = \sum\limits_{i=0}^{\infty}
z^{i} p^{(m)}_{n;i}\,, \\
0<z<1\,, \ \ n\ge1\,,\ \
m=\overline{1,\min(n,N-1)} \,;
\end{multline*}

\vspace*{-12pt}


\noindent
\begin{multline*}
P^{(m)}(u,z) = \sum\limits_{n=N}^{\infty} u^{n-N} P^{(m)}_{n}(z)\,, \\
0<u,z<1\,, \ \ m=\overline{1,N-1}\,,
\end{multline*}
и, кроме того, положим
$$
P(u) = \sum\limits_{n=N}^{\infty} u^{n-N} p_{n}
= \fr{1}{1 - \tilde{\rho} u}\, p_N \,.
$$

Тогда, умножая~\eqref{eq-1-1} и~\eqref{eq-1-2}
на~$z^i$ и~суммируя по всем~$i$ от нуля до
бесконечности, получаем:
\begin{multline*}
%\label{eq-z-1}
(\lambda+N\mu) P^{(1)}_{n}(z) =
\lambda P^{(1)}_{n-1}(z) +
(N-1) \mu p_{n+1}
+ {}\\
{}+\mu z P^{(1)}_{n+1}(z)\,,\enskip n\ge N\,.
\end{multline*}
Умножая последнее выражение на $u^{n-N}$ и~суммируя по всем значениям $n\hm\ge N$,
после приведения подобных слагаемых имеем:
\begin{multline}
\label{eq-z-2}
f_1(u,z) P^{(1)}(u,z) =
\mu z P^{(1)}_{N}(z) -{}\\
{}- \lambda u P^{(1)}_{N-1}(z) -
(N-1) \mu [P(u) - p_{N}] \,.
\end{multline}


Теперь умножим \eqref{bat-1} и~\eqref{bat-2}
на~$z^i$ и~просуммируем по всем значениям $i\hm\ge0$.
В~результате приходим к~выражению:
\begin{multline*}
%\label{bat-2*}
(\lambda+N\mu) P^{(m)}_{n}(z) = \lambda P^{(m)}_{n-1}(z)
+{}\\
{}+(N-m) \mu P^{(m-1)}_{n+1}(z) +
m \mu z P^{(m)}_{n+1}(z) \,,\enskip n\ge N\,.
\end{multline*}
Умножая последнее выражение на $u^{n-N}$, после
суммирования по всем $n\hm\ge N$ получаем:

\pagebreak

\noindent
\begin{multline}
\label{bat-2*}
f_m(u,z) P^{(m)}(u,z) = m \mu z P^{(m)}_{N}(z)
- \lambda u P^{(m)}_{N-1}(z) -{}\\
{}-
(N-m) \mu [P^{(m-1)}(u,z) - P^{(m-1)}_{N}(z)]\,,\\ m=\overline{2,N-1}\,.
\end{multline}

Из уравнений~\eqref{eq-1-3} и~\eqref{eq-1-4}
после умножения на~$z^i$ и~суммирования по
всем значениям $i \hm\ge 0$ находим:
\begin{multline}
\label{eq-z-3}
P^{(1)}_{N-1}(z)=\fr{\lambda p_{N-2} + (N-1)\mu p_{N}}
{\lambda+(N-1)\mu }+{}\\
{}+ \fr{\mu z}{\lambda+(N-1)\mu} \,P^{(1)}_N(z)\,.
\end{multline}

Действуя аналогичным образом
с~уравнениями~\eqref{bat-3} и~\eqref{bat-4}, как и~с~уравнениями~\eqref{eq-1-3}
и~\eqref{eq-1-4}, приходим к выражению:
\begin{multline}
\label{bat-4*}
P^{(m)}_{N-1}(z) = \fr{ \lambda P^{(m-1)}_{N-2}(z) + (N-m) \mu P^{(m-1)}_{N}(z)
}{\lambda+(N-1)\mu }+{}\\
{}+\fr{m \mu z}{\lambda+(N-1)\mu}\,P^{(m)}_{N}(z) \,,\enskip m=\overline{2,N-1}\,.
\end{multline}


Домножая уравнения~\eqref{eq-1-5} и~\eqref{eq-1-6}
на~$z^i$, после суммирования по всем
значениям $i \hm\ge 0$ имеем:
\begin{multline}
\label{eq-z-4}
P^{(1)}_{n}(z)= \fr{ \lambda p_{n-1} + n \mu P^{(1)}_{n+1}(z) }{
\lambda+n\mu }+ \fr{\mu z}{\lambda+n\mu}\,P^{(2)}_{n+1}(z) \,,\\
n=\overline{1,N-2}\,.
\end{multline}

Наконец, производя аналогичные преобразования
с~уравнениями~\eqref{bat-5} и~\eqref{bat-6}, получаем:
\begin{multline}
\label{bat-6*}
P^{(m)}_n(z)= \fr {\lambda P^{(m-1)}_{n-1}(z) +
(n-m+1) \mu P^{(m)}_{n+1}(z)} {\lambda+n\mu}
+{}
\\
{}+
\fr{m \mu z}{\lambda+n\mu} P^{(m+1)}_{n+1}(z)\,,\enskip
m=\overline{2,N-2}\,,\\
n=\overline{m,N-2}\,.
\end{multline}

Уравнения~\eqref{eq-z-2}--\eqref{bat-6*} позволяют
находить выражения для всех
ПФ $P^{(m)}_{n}(z)$, $m\hm=\overline{1,N-1}$,
$n\hm=\overline{1,N-1}$, а~так\-же совместное
стационарное распределение рекуррентным образом.
Подставляя выражение для $P^{(1)}_{N-1}(z)$ из
формулы~\eqref{eq-z-3} в~формулу~\eqref{eq-z-2}, получаем:
\begin{multline}
P^{(1)}(u,z) = \left(
\left[
\mu z - \fr{\lambda \mu z u}{\lambda+(N-1)\mu}
\right] P^{(1)}_N(z) -{}\right.\\
{}-
\left[
\lambda u \fr{\lambda p_{N-2} + (N-1)\mu p_{N}}{\lambda+(N-1)\mu}+{}\right.\\
\left.\left.{}+
 (N-1) \mu [P(u) - p_{N}]
\vphantom{\fr{\lambda p_{N-2} + (N-1)\mu p_{N}}{\lambda+(N-1)\mu}}\right]
\right)
\Bigg /
f_1(u,z)\,,
\label{m25}
\end{multline}
откуда из равенства нулю в~точке $u_1(z)$ числителя и~знаменателя
правой части формулы~\eqref{m25} следует:
\columnbreak


%%%%%%%%%%%%%%%%%%%%%%%%%%%
\noindent
\begin{multline*}
%\label{r1}
P^{(1)}_N(z)= \left(
\lambda u_1(z) [\lambda p_{N-2} + (N-1)\mu p_{N}]
+{}\right.\\
{}+
\left.(\lambda+(N-1)\mu)(N-1) \mu \left[P(u_1(z)) - p_{N}\right]\right)\!\!\Big/\!\!
\left(\mu z \left[\lambda+{}\right.\right.\\
\left.\left.{}+(N-1)\mu  - \lambda u_1(z)\right]\right)\,.
\end{multline*}
%%%%%%%%%%%%%%%%%%%%%%%%%%%%%%%%%%%%%%%%%%
%%%%%%%%%%%%%%%%%%%%%%%%%%%%%%%%%%%%%%%%
Теперь, возвращаясь к~формуле~\eqref{eq-z-3},
получаем выражение для $P^{(1)}_{N-1}(z)$:
\begin{multline*}
%\label{r2}
P^{(1)}_{N-1}(z)=
\left([\lambda p_{N-2} + (N-1)\mu p_{N}]+{}\right.\\
\left.{}
+ (N-1) \mu [P(u_1(z)) - p_{N}]\right)\Big /
\left(\lambda+(N-1)\mu  - {}\right.\\
\left.{}-\lambda u_1(z)\right)\,.
\end{multline*}

Далее из равенства~\eqref{eq-z-4} выражаем $P^{(1)}_{N-2}(z)$ через
$P^{(2)}_{N-1}(z)$. Из равенства~\eqref{bat-4*} выражаем
$P^{(2)}_{N-1}(z)$ через $P^{(2)}_{N}(z)$. Подставляя полученное
выражение для $P^{(2)}_{N-1}(z)$ в~формулу~\eqref{bat-2*}, из
равенства нулю в~точке~$u_2$ левой и~правой части получившегося
равенства находим $P^{(2)}(u,z)$. Затем из равенства~\eqref{eq-z-4}
выражаем $P^{(1)}_{N-3}(z)$ через $P^{(2)}_{N-2}(z)$ и~т.\,д.

Продолжая эту процедуру, можно найти
соотношения для вычисления всех
ПФ $P^{(m)}_{n}(z)$, $m\hm=\overline{1,N-1}$, $n\hm=\overline{1,N-1}$.

С каждым шагом выражение для очередной ПФ становится все сложнее,
и~в итоге при большом числе приборов выписать явный вид всех ПФ не
удается. Тем не менее нахождение значений ПФ в~каждой точке $z \hm\ne
0$ можно свести к последовательному решению систем линейных
уравнений. Для этого обозначим через $A_n(z)$, $n\hm =\overline{2,N-1}$,
мат\-ри\-цы размера $(n+1)\times (n+1)$, име\-ющие
следующую структуру:
\begin{gather*}
\setcounter{MaxMatrixCols}{3}
A_2(z)=
\begin{pmatrix}
 2 \mu z   & 0  & -2 \mu z         \\
 - \lambda u_2(z) & - \mu z   &  \lambda +(N-1) \mu       \\
0  & \lambda +(N-2)\mu &    - \lambda
\end{pmatrix}\,;
\end{gather*}

\vspace*{-12pt}

\noindent
{ %\scriptsize
\begin{multline*}
\setcounter{MaxMatrixCols}{7}
A_n(z)=\left(
\begin{matrix}
 n \mu z   & 0  & - n \mu z &      \!\cdots\!          \\
 - \lambda u_n(z) \! & 0  & \! \lambda +(N-1) \mu \! & \!\cdots\!  \\
  \vdots   & \vdots & \vdots &  \!\cdots\! \\
 0   & 0 & 0&  \!\cdots\!  \\
 0      & 0 & 0    &     \!\cdots\! \\
 0   & - \mu z  &0   &  \cdots\! \\
0 & \!\lambda +(N-n)\mu \!&0  &  \!\cdots\!
\end{matrix}\right.\\
\left.\begin{matrix}
    \cdots\!     & 0    & 0       \\
    \cdots\!  & 0 & 0 \\
    \cdots\! & \vdots     & \vdots  \\
    \cdots\!  & - 3 \mu z     & 0   \\
    \cdots\! & \! \lambda +(N-n+2) &-2\mu z\\
    \cdots\! & - \lambda  & \! \lambda+(N-n+1)\mu\\
    \cdots\!  & 0  & - \lambda
\end{matrix}\right)\,,
\\ n =\overline{3,N-1}\,.
\end{multline*}
}

\noindent
Определим вектор-стр$\acute{\mbox{о}}$\-ки $\vec{a}_n(z)$ и~$\vec{b}_n(z)$
длины $(n+1)$ следующим образом:
\begin{multline*}
\vec{a}_n(z) = \left (
P^{(n)}_{N}(z), P^{(n)}_{N-1}(z), \dots\right.\\
\left.\dots,  P^{(2)}_{N-n+1}(z), P^{(1)}_{N-n}(z)
\right )\,,\enskip
n =\overline{2,N-1}\,;
\end{multline*}

\vspace*{-12pt}
\noindent
\begin{multline*}
\vec{b}_2(z) = \left (
(N-2) \mu [P^{(1)}(u_2,z) - P^{(1)}_{N}(z)] ,
\lambda p_{N-3}+{}\right.\\
\left.{}+ (N-2) \mu P^{(1)}_{N-1}(z),
(N-2) \mu P^{(1)}_{N}(z) \right)\,;
\end{multline*}

\vspace*{-12pt}

\noindent
\begin{multline*}
\vec{b}_n(z) = \left (
(N-n) \mu
[P^{(n-1)}(u_n,z) - P^{(n-1)}_{N}(z)],\right.
\\
\lambda p_{N-1-n}+ (N-n) \mu P^{(1)}_{N-1-(n-2)}(z),\\
(N-n) \mu P^{(n-1)}_{N}(z), (N-n)\mu  P^{(n-1)}_{N-1}(z),
\dots ,
\\
\left.
(N-n)\mu  P^{(3)}_{N-n+3}(z), (N-n)\mu  P^{(2)}_{N-n+2}(z)
\right )\,,\\
n =\overline{3,N-1}\,.
\end{multline*}
Тогда алгоритм нахождения ПФ состоит в~последовательном начиная с~$n\hm=2$ решении
системы линейных уравнений
$$
\vec{a}_n(z) A_n(z) = \vec{b}_n(z) \,.
$$
Из структуры матрицы $A_n(z)$, $n \hm=\overline{3,N-1}$, видно, что
она неприводима и~обладает свойством диагонального преобладания
т.\,е.\ перестановкой строк и~столбцов можно добиться того,
что в~каждой строке модуль диагонального элемента будет либо строго
больше, либо не меньше суммы модулей всех остальных элементов в~строке.
Покажем это. Если определить матрицы перестановки~$P^L_n$ и~$P^R_n$
размера $(n+1)\times (n+1)$ при $n \hm=\overline{3,N-1}$
следующим образом:
\begin{gather*}
\setcounter{MaxMatrixCols}{5}
P^L_n=
\begin{pmatrix}
 0   & 0  & \cdots & 0& 1 \\
 1   & 0  & \cdots & 0& 0 \\
  \vdots   &  \vdots  & \cdots &  \vdots &  \vdots \\
  0   & 0  & \cdots & 0& 0 \\
   0   & 0  & \cdots & 1& 0
\end{pmatrix}\,;
\enskip
\setcounter{MaxMatrixCols}{5}
P^R_n=
\begin{pmatrix}
 0   & 1  & \cdots & 0& 0         \\
 1   & 0  & \cdots & 0& 0 \\
   \vdots   &  \vdots  & \cdots &  \vdots &  \vdots \\
  0   & 0  & \cdots & 1& 0 \\
   0   & 0  & \cdots & 0& 1
\end{pmatrix}\,,
\end{gather*}
то матрица $P^L_n A_n(z)P^R_n$, $n \hm=\overline{3,N-1}$,
примет вид:
\begin{multline*}
\setcounter{MaxMatrixCols}{7}
P^L_n A_n(z)P^R_n={}\\
{}=\left(
\begin{matrix}
 \lambda +(N-n)\mu &0 & 0  & \cdots\\
 0  &  n \mu z   & - n \mu z &  \cdots       \\
  0  & - \lambda u_n(z) & \lambda +(N-1) \mu  & \cdots  \\
  \vdots   & \vdots & \vdots & \cdots   \\
 0   & 0 & 0& \cdots \\
 0      & 0 & 0    & \cdots  \\
 - \mu z  & 0   &0   & \cdots
\end{matrix}\right.
\end{multline*}

\noindent
\begin{equation*}
\hspace*{15mm}\left.\begin{matrix}
\cdots  & 0  & - \lambda\\
\cdots        & 0    & 0       \\
\cdots    & 0    & 0      \\
\cdots   & \vdots     & \vdots       \\
\cdots  & - 3 \mu z     & 0       \\
\cdots   & \lambda +(N-n+2) \mu & - 2 \mu z       \\
\cdots      & - \lambda  & \lambda +(N-n+1)\mu
\end{matrix}\right).
\end{equation*}
Легко видеть, что в~каждой строке модуль диагонального
элемента либо строго больше, либо не меньше суммы
модулей всех остальных элементов в~строке.
Тогда, как вытекает из следствия~6.2.27 в~\cite{horn},
у~матрицы $A_n(z)$ существует обратная
и,~значит, система $\vec{a}_n(z) A_n(z) \hm= \vec{b}_n(z)$
при $z\hm\neq 0$ имеет единственное решение.

\vspace*{-4pt}

\section{Примеры расчетов}

На основе полученных в~разд.~4 результатов {были} проведены расчеты
среднего и~дисперсии чис\-ла заявок в~БП,
а~также коэффициента корреляции числа заявок в~накопителе и~числа
заявок в~БП для различного чис\-ла
приборов~$N$~и~значений загрузки системы $\rho/N$. \mbox{Напомним}, что аналогичные
показатели были рассчитаны в~\cite{a8} по определению, на основе
стационарных вероятностей, рассчитанных по приведенному выше
алгоритму. Далее можно видеть, что результаты, полученные с~по\-мощью
ПФ, как и~ожидалось, полностью совпадают с~результатами,
представленными в~\cite{a8}.

На рис.~1 отражено поведение значения среднего числа заявок
в~БП в~зависимости от загрузки системы $\rho/N$.
Отметим, что полученные в~предыдущих  разделах результаты позволяют
рассчитывать такие
 характеристики, как среднее число заявок только
в~первой очереди в~БП, в~сумме в~первой и~во второй очередях в~БП
(когда обе очереди существуют), в~сумме в~первой, второй,\ldots ,
$(N-1)$-й очере-\linebreak\vspace*{-12pt}

\vspace*{6pt}

\begin{center}  %fig1
\vspace*{2pt}
\mbox{%
 \epsfxsize=75.145mm
 \epsfbox{pec-1.eps}
 }
\end{center}

\noindent
{{\figurename~1}\ \ \small{Поведение
 среднего числа заявок в~БП в~зависимости от загрузки
системы  $\rho/N$: \textit{1}~--- $N\hm=4$; \textit{2}~--- 7;
\textit{3}~--- $N=9$}}

%\vspace*{9pt}


\addtocounter{figure}{1}


\begin{center}  %fig2
\vspace*{2pt}
 \mbox{%
 \epsfxsize=75.027mm
 \epsfbox{pec-2.eps}
 }
 \end{center}

\noindent
{{\figurename~2}\ \ \small{Поведение среднего числа заявок в~первой
очереди в~БП~(\textit{1}), в~сумме в~первой и~во второй очередях в~БП~(\textit{2}),
в~сумме в~первой, второй и~третьей очередях в~БП~(\textit{3})
в~зависимости от загрузки системы $\rho/N$. Число
приборов $N\hm=4$}}

\vspace*{18pt}


\begin{center}  %fig3
\vspace*{2pt}
 \mbox{%
 \epsfxsize=74.929mm
 \epsfbox{pec-3.eps}
 }
 \end{center}

\noindent
{{\figurename~3}\ \ \small{Поведение
 дисперсии числа заявок в~БП в~зависимости от загрузки
системы  $\rho/N$: \textit{1}~--- $N\hm=4$; \textit{2}~--- 7; \textit{3}~--- $N=9$}}

\vspace*{18pt}

\begin{center}  %fig4
\vspace*{2pt}
 \mbox{%
 \epsfxsize=75.192mm
 \epsfbox{pec-4.eps}
 }
 \end{center}

\noindent
{{\figurename~4}\ \ \small{Поведение
 коэффициента корреляции числа заявок в~накопителе и~числа
заявок в~БП в~зависимости от загрузки системы  $\rho/N$:
\textit{1}~--- $N\hm=4$; \textit{2}~--- 7; \textit{3}~--- $N=9$}}


%\vspace*{9pt}


\noindent
дях в~БП (когда каждая из очередей существует).
Поведение данных характеристик в~зависимости от загрузки системы
$\rho/N$ для случая $N\hm=4$ пред\-став\-ле\-но на рис.~2.

На рис.~3 и~4 изображено поведение дисперсии числа
заявок в~БП и~поведение
коэффициента корреляции числа заявок в~накопителе и~числа
заявок в~БП соответственно.

Во всех расчетах интенсивность обслуживания заявок~$\mu$ принималась
равной~1.

%\addtocounter{figure}{1}
%%%%%%%%%%%%%%%%%%%%%%%%%%%%%%%%%%%%%%%%%%%%%%%%%%%%%

Анализируя графики на рис.~1--4, стоит отметить два момента. Среднее
число заявок в~БП не уходит в~бесконечность с ростом загрузки
(и~даже при загрузке больше единицы), что следует из формулы Литтла.
Число заявок в~накопителе и~число заявок в~БП весьма слабо
коррелированы, и~с~рос\-том числа приборов коэффициент корреляции
уменьшается.

\section{Заключение}

В настоящей работе рассмотрена функционирующая в~непрерывном времени
многоканальная система обслуживания с~накопителем бесконечной емкости
и~переупорядочением заявок.
В~систему поступает пуассоновский поток заявок, время
обслуживания каждым прибором распределено по
экспоненциальному закону с~одним и~тем же параметром.
Для нахождения совместного стационарного распределения
числа заявок в~накопителе и~суммарного числа
заявок в~БП получен рекуррентный алгоритм.
Также показано, как можно находить совместное распределение
в~терминах ПФ, которые облегчают расчет его моментов.

{\small\frenchspacing
 {%\baselineskip=10.8pt
 \addcontentsline{toc}{section}{References}
 \begin{thebibliography}{99}
 \bibitem{a1} %1
\Au{Boxma O., Koole G., Liu~Z.}
Queueing-theoretic solution methods for
models of parallel and distributed systems~//
Performance Evaluation of Parallel and Distributed Systems Solution
Methods, 1994. CWI Tract~105 and~106. P.~1--24.

\bibitem{a2} %2
\Au{Dimitrov B.}
Queues with resequencing. A~survey and recent results~//
{2nd World Congress on Nonlinear Analysis,
Theory, Methods, Applications Proceedings}, 1997. Vol.~30. No.\,8. P.~5447--5456.

\bibitem{a3} %3
\Au{Huisman T., Boucherie R.\,J.}
The sojourn time distribution in an infinite server
resequencing queue with dependent interarrival and
service times~// J.~Appl. Probab., 2002.
Vol.~39. No.\,3. P.~590--603.

\bibitem{a5} %4
\Au{Xia Y., Tse D.\,N.\,C.}
On the large deviations of resequencing
queue size: 2-$M$/$M$/1 сase~// IEEE Trans. Inform. Theory, 2008.
Vol.~54. No.\,9. P.~4107--4118.

\bibitem{a4} %5
\Au{Leung K., Li V.\,O.\,K.}
A~resequencing model for high-speed packet-switching networks~//
J.~Comput. Commun., 2010.
Vol.~33. No.\,4. P.~443--453.

\bibitem{a7} %6
\Au{Матюшенко С.\,И.} Стационарные характеристики двухканальной
системы обслуживания с~переупорядочением заявок и~распределениями
фазового типа~// Информатика и~её применения, 2010. Т.~4. Вып.~4.
С.~67--71.

\bibitem{a6} %7
\Au{De Nicola C., Pechinkin A.\,V., Razumchik~R.\,V.}
Stationary characteristics of homogenous Geo/Geo/2
queue with resequencing in discrete time~//
27th European Conference on Modelling and
Simulation Proceedings.~---- Aalesund, 2013. P.~594--600.

\bibitem{a7+} %8
\Au{Pechinkin A.\,V., Caraccio~I., Razumchik~R.\,V.}
Joint stationary distribution of queues in
homogenous $M\vert M\vert$3 queue with resequencing~//
28th European Conference on
Modelling and Simulation Proceedings.~--- Brescia, 2014. P.~558--564.

\bibitem{a8}
\Au{Pechinkin A.\,V., Caraccio~I., Razumchik~R.\,V.}
On joint stationary distribution in exponential
multiserver reordering queue~// 12th  Conference (International) on
Numerical Analysis and Applied Mathematics Proceedings, 2014 (in press).

\bibitem{boch}
\Au{Bocharov P.\,P., D'Apice C., Pechinkin~A.\,V., Salerno~S.}
Queueing theory.~--- Urecht, Boston: VSP, 2004. 446~p.

\bibitem{horn}
\Au{Horn R.\,A., Johnson C.\,R.}
Matrix analysis.~--- 2nd ed.~--- Cambridge: Cambridge University Press, 2013.
662~p.
 \end{thebibliography}

 }
 }

\end{multicols}

\vspace*{-9pt}

\hfill{\small\textit{Поступила в редакцию 28.10.14}}

%\newpage

\vspace*{12pt}

\hrule

\vspace*{2pt}

\hrule

%\vspace*{12pt}

\def\tit{JOINT STATIONARY DISTRIBUTION OF~THE~NUMBER OF~CUSTOMERS IN~THE~SYSTEM
AND REORDERING BUFFER IN~THE~MULTISERVER REORDERING QUEUE}

\def\titkol{Joint stationary distribution of~the~number of~customers in~the~system
and reordering buffer in~the~multiserver reordering queue}



\def\aut{\fbox{A.\,V.~Pechinkin}$^1$ and R.\,V.~Razumchik$^{1,2}$}

\def\autkol{A.\,V.~Pechinkin and R.\,V.~Razumchik}

\titel{\tit}{\aut}{\autkol}{\titkol}

\vspace*{-9pt}

\noindent
$^1$Institute of Informatics Problems, Russian Academy of Sciences,
44-2 Vavilov Str., Moscow 119333, Russian\\
$\hphantom{^1}$Federation


\noindent
$^2$Peoples' Friendship University of Russia,
6~Miklukho-Maklaya Str., Moscow 117198, Russian Federation



\def\leftfootline{\small{\textbf{\thepage}
\hfill INFORMATIKA I EE PRIMENENIYA~--- INFORMATICS AND
APPLICATIONS\ \ \ 2014\ \ \ volume~8\ \ \ issue\ 4}
}%
 \def\rightfootline{\small{INFORMATIKA I EE PRIMENENIYA~---
INFORMATICS AND APPLICATIONS\ \ \ 2014\ \ \ volume~8\ \ \ issue\ 4
\hfill \textbf{\thepage}}}

\vspace*{3pt}



\Abste{The paper considers a continuous-time multiserver queueing
system with buffer on infinite capacity and reordering. The Poisson
flow of customers arrives at the system. Service times of customers at
each server are exponentially distributed with the same parameter.
Each customer obtains a~sequential number upon arrival. The order of
customers upon arrival should be preserved upon departure from the system.
Customers whose service finished but which violated the order are kept in
the reordering buffer of infinite capacity. A~joint stationary distribution
of the number of customers in the buffer, servers, and
reordering buffer is obtained in terms of a~computational algorithm and
a~generating function. A~numerical example is provided.}


\KWE{queueing system; reordering; infinite capacity; joint distribution}

\DOI{10.14357/19922264140401}

%\vspace*{3pt}

\Ack
\noindent
The research was partially financially supported by the Russian Foundation for
Basic Research (project 13-07-00223).


  \begin{multicols}{2}

\renewcommand{\bibname}{\protect\rmfamily References}
%\renewcommand{\bibname}{\large\protect\rm References}



{\small\frenchspacing
 {%\baselineskip=10.8pt
 \addcontentsline{toc}{section}{References}
 \begin{thebibliography}{99}


 \bibitem{a1-1}
\Aue{Boxma O., G. Koole, and Z.~Liu}. 1994.
Queueing-theoretic solution methods for
models of parallel and distributed systems.
\textit{Performance Evaluation of Parallel and
Distributed Systems Solution Methods}.  CWI Tract 105
and 106:1--24.

\bibitem{a2-1}
\Aue{Dimitrov, B.} 1997.
Queues with resequencing. A~survey and recent results.
\textit{2nd World Congress on Nonlinear
Analysis, Theory, Methods, Applications Proceedings}. 30(8):5447--5456.

\bibitem{a3-1}
\Aue{Huisman, T., and R.\,J.~Boucherie}. 2002.
The sojourn time distribution in an infinite server
resequencing queue with dependent interarrival and service times.
\textit{J.~Appl. Probab}. 39(3):590--603.

\bibitem{a5-1}
\Aue{Xia, Y., and D.\,N.\,C.~Tse}. 2008.
On the large deviations of resequencing
queue size: 2-$M$/$M$/1 case.
\textit{IEEE Trans. Inform. Theory} 54(9):4107--4118.

\bibitem{a4-1} %5
\Aue{Leung, K., and V.\,O.\,K.~Li}. 2010.
A~resequencing model for high-speed
packet-switching networks.
\textit{J.~ Comput. Commun.} 33(4):443--453.

\bibitem{a7-1} %6
\Aue{Matyushenko, S.\,I.} 2010.
 Statsionarnye kharakteristiki
dvukh\-ka\-nal'\-noy sistemy obsluzhivaniya s~pe\-re\-upo\-rya\-do\-chi\-va\-ni\-em zayavok
i~raspredeleniyami
fazovogo tipa [Stationary characteristics of the two-channel
queueing system with reordering customers and distributions of phase type].
\textit{Informatika i ee Primemeniya}~--- \textit{Inform. Appl.}
4(4):67--71.

\bibitem{a6-1} %7
\Aue{De Nicola, C., A.\,V.~Pechinkin, and R.\,V.~Razumchik}. 2013.
Stationary characteristics of homogenous Geo/Geo/2
queue with resequencing in discrete time.
\textit{27th European Conference
on Modelling and Simulation Proceedings}. Aalesund. 594--600.

\bibitem{a7+-1}
\Aue{Pechinkin, A.\,V., I.~Caraccio, and R.\,V.~Razumchik}. 2014.
joint stationary distribution of queues
in homogenous $M \vert M \vert3$ queue with resequencing.
\textit{28th European Conference
on Modelling and Simulation Proceedings}. Brescia. 558--564.

\bibitem{a8-1}
\Aue{Pechinkin, A.\,V., I.~Caraccio, and R.\,V.~Razumchik}. 2014 (in press).
On joint stationary distribution in exponential
multiserver reordering queue.
\textit{12th  Conference (International) on
Numerical Analysis and Applied Mathematics Proceedings}.

\bibitem{boch-1}
\Aue{Bocharov,  P.\,P., C.~D'Apice, A.\,V.~Pechinkin, and S.~Salerno}. 2004.
\textit{Queueing theory}. Urecht, Boston: VSP. 446~p.

\bibitem{horn-1}
\Aue{Horn, R.\,A., and C.\,R.~Johnson}. 2013.
\textit{Matrix analysis}. Cambridge: Cambridge University Press. 662~p.
\end{thebibliography}

 }
 }

\end{multicols}

\vspace*{-6pt}

\hfill{\small\textit{Received October 28, 2014}}

\vspace*{-18pt}

\Contr

\noindent
\textbf{Pechinkin Alexander V.} (1946--2014)~--- Doctor
of Science in physics and mathematics; principal
scientist, Institute of Informatics Problems of
the Russian Academy of Sciences, 44-2 Vavilov Str.,
Moscow 119333, Russian Federation


\vspace*{3pt}

\noindent
\textbf{Razumchik Rostislav V.} (b.\ 1984)~--- Candidate
of Science (PhD) in physics and mathematics,
senior scientist, Institute of Informatics
Problems of the Russian Academy of Sciences, 44-2 Vavilov Str.,
Moscow 119333, Russian Federation;
associate professor,
Peoples' Friendship University of Russia,
6~Miklukho-Maklaya Str., Moscow 117198, Russian Federation;
rrazumchik@ieee.org


\label{end\stat}

\renewcommand{\bibname}{\protect\rm Литература} %3

\def\stat{konovalov}

\def\tit{ОБ АДАПТИВНЫХ СТРАТЕГИЯХ И~УСЛОВИЯХ~ИХ~СУЩЕСТВОВАНИЯ$^*$}

\def\titkol{Об адаптивных стратегиях и~условиях их 
существования}

\def\autkol{М.\,Г.~Коновалов}

\def\aut{М.\,Г.~Коновалов$^1$}

\titel{\tit}{\aut}{\autkol}{\titkol}

{\renewcommand{\thefootnote}{\fnsymbol{footnote}}\footnotetext[1]
{Работа выполнена при поддержке РФФИ, грант № 11-07-00112.}}

\renewcommand{\thefootnote}{\arabic{footnote}}
\footnotetext[1]{Институт проблем информатики Российской академии наук, mkonovalov@ipiran.ru}



\Abst{Рассматривается задача оптимального управления в отсутствие априорной 
информации об управляемом объекте. Решением задачи является построение адаптивных 
стратегий на основе наблюдений, доступных в процессе управления. Изучаются 
некоторые условия адаптивной управляемости объекта. В~качестве математической 
модели используются управляемые случайные последовательности.}

\KW{управляемые случайные последовательности; адаптивные стратегии; условия 
существования}

\vskip 14pt plus 9pt minus 6pt

      \thispagestyle{headings}

      \begin{multicols}{2}

            \label{st\stat}


\section{Введение}

  Тема статьи относится к области адаптивных методов обработки информации с целью 
принятия оптимальных решений. Потребность в адаптивном\linebreak
подходе возникает в задачах 
с большой информационной неопределенностью, что наиболее характерно для 
телекоммуникационных систем, автоматизированных производственных процессов, 
робототех\-ни\-ки и других сфер, неразрывно связанных с компьютерной обработкой 
информации. Понятие неопределенности многозначно и связано с отсутствием априорных 
сведений, недетерминированностью, а также с неполнотой наблюдений. 
К~перечисленным факторам в нарастающей степени добавляется <<избыточность>> 
информации, которая порождается чрезмерно прогрессирующими объемами 
передаваемой и хранимой информации и обусловлена экспоненциальным ростом 
пропускной способности телекоммуникационных сетей, а также емкостей носителей 
информации.
  
  Идея адаптации (приспособления, самоорганизации), заимствованная из 
биологического мира, начала активно эксплуатироваться в науке примерно с середины 
прошлого века. Кратко, она заключается в том, чтобы, целенаправленно взаимодействуя с 
окружающей средой, отбирать и использовать поступающую информацию, необходимую 
для принятия оптимальных решений с точки зрения поставленной цели.
  
  Данная статья посвящена теоретическим аспектам адаптации. В~качестве исходного 
пред\-став\-ле\-ния использована схема, которая опирается на пред\-став\-ление о паре 
  <<объект--субъект>>, взаимодействующей в дискретном времени путем 
попеременного обмена сигналами. При этом субъект воздействует на объект с помощью 
управлений, получая в ответ сигналы, называемые наблюдениями. Действия субъекта 
преследуют цель, выраженную в наличии определенных свойств у траектории 
наблюдений.
  
  Основная отличительная особенность заключается в предположении, что действия 
субъекта происходят при недостаточной информации об объекте. В~качестве 
математической модели объекта взята конструкция управляемой случайной 
последовательности. В~терминах этого аппарата легко очерчиваются четыре аспекта 
информационной неопределенности:
  \begin{enumerate}[(1)]
\item недетерминированность понимается как стохастичность;
\item недостаток информации об объекте трактуется как неполное знание вероятностного 
распределения, задающего процесс;
  \item неполнота наблюдений означает, что состояния процесса наблюдаются лишь 
частично;
  \item недостаток знаний выражается в неумении \mbox{найти} или рассчитать ту или иную 
характеристику, связанную со случайной последовательностью, даже при наличии 
априорной информации о распределении процесса и полной его наблюдаемости.
  \end{enumerate}
  
  Субъект ассоциируется с алгоритмом, согласно которому выбираются управления, 
регулирующие траекторию случайной последовательности. Такой алгоритм принято 
называть стратегией управ\-ле\-ния. Задача заключается в том, чтобы выбрать стратегию, 
достигающую цели в ситуации, когда информация субъекта об объекте ограничена. 
По-дру\-го\-му можно сказать, что речь идет о построении стратегии, достигающей цели (в 
данном случае~--- максимизации предельного среднего дохода) для любого процесса из 
некоторого заданного класса объектов. Такие стратегии называют адаптивными по 
отношению к заданному классу объектов~[1].
  
  В разд.~2 даются формальные определения объекта, цели и адаптивной стратегии 
управления.
  
  В разд.~3 анализируются условия существования адаптивной стратегии. В~качестве 
необходимых условий обсуждаются два требования, которые, как представляется, должны 
выполняться из интуитивных соображений.
  
  Первое из необходимых условий связано с принципиальной особенностью адаптивных 
стратегий, которые, прежде чем выйти на <<оптимальный режим>>, должны затратить 
некоторое время на <<обуче\-ние>>. (На самом деле в рассматриваемой постановке процесс 
обучения для адаптивных стратегий длится даже неограниченно долго.) Естественно 
предположить, что подобные стратегии могут реализоваться, только если в процессе 
обучения не будут совершены <<непоправимые ошибки>>. Это соображение 
раскрывается на примерах и получает формальное описание.
  
  Второе необходимое условие является менее очевидным. Оно связано с гипотезой о 
том, что адаптивная стратегия управления классом случайных последовательностей 
существует лишь тогда, когда для данного класса возможно построение так называемой 
адаптивной стратегии перебора. Это выражается в том, что существует и заранее известно 
некоторое счетное множество вариантов поведения, среди которого для данного класса 
обязательно найдется оптимальный или близкий к нему вариант. Данное соображение 
также иллюстрировано примерами и приведена теорема о критерии существования 
адаптивной стратегии для определенного класса объектов.
  
  Подход, использованный в статье, а также полученные результаты являются 
продолжением направления, представленного в работе~[2].
  
\section{Постановка задачи адаптивного управления}
  
  Пусть  время $t$ пробегает значения 0, 1, \ldots\ и пусть заданы измеримые 
пространства $(X,\mathbf{X})$, $(Y,\mathbf{Y})$, $(Z,\mathbf{Z})$ (соответственно 
пространства \textit{состояний}, \textit{управлений} и \textit{наблюдений}).
  
  Общая траектория процесса упорядочена в виде последовательности $x_0, y_1, 
z_1,x_1,\ldots$\linebreak $\ldots , x_{t-1},y_t,z_t,x_t,\ldots$ Предыстория процесса до момента~$t$ 
включительно обозначается как

\noindent
  \begin{gather*}
 \! x^t=x_0^t=(x_0,\ldots , x_{t-1});\ \ \ y^t=y_1^t=(y_1, \ldots , y_{t-1});\\
  z^t=z_1^t=(z_1,  \ldots , z_{t-1})\,.
  \end{gather*}
  
  Траектории процесса определяются последовательностями условных вероятностных 
распределений~$\mu$, $\nu$ и~$\sigma$.
  
  Последовательность $\mu\hm=(\mu_0,\mu_1,\ldots ,\mu_t, \ldots)$ задает механизм 
смены состояний. В~этой последовательности $\mu_0$~--- вероятностное распределение 
на $(X,\mathbf{X})$; $\mu_t=\mu_t(A\vert x^{t-1},y^t)$, $t\hm>0$~---  условная 
(переходная) вероятность, которая при любых наборах $(x^{t-1},y^t)$ является 
вероятностной мерой на $(X,\mathbf{X})$ и при любом $A\hm\in X$ является измеримой 
функцией относительно $x^{t-1},y^t$.
  
  Последовательность $\nu\hm=(\nu_1, \ldots , \nu_t, \ldots)$ задает механизм появления 
наблюдений. В~этой последовательности каждый элемент $\nu_t\hm=\nu_t(C\vert x^{t-1}, 
y^t)$, $t\hm>0$, представляет собой условное распределение, которое при любом условии 
является вероятностной мерой на $(Z,\mathbf{Z})$ и для любого $C\hm\in Z$ является 
измеримой функцией относительно переменных, стоящих в условии. Пара $o\hm= 
(\mu,\nu)$ называется объектом.
  
  Последовательность $\sigma\hm= (\sigma_1, \ldots , \sigma_t. \ldots)$ называется 
(допустимой) \textit{стратегией} и определяет выбор управлений. В~этой 
последовательности:
%\smallskip
   $\sigma_1\hm=\sigma_1(\cdot)$~--- вероятностная мера на $(Y,\mathbf{Y})$; 
      $\sigma_{t+1}\hm=\sigma_{t+1}(B\vert y^t,z^t)$, $t\hm>0$,~--- условная вероятность, 
которая при любых $y^t,z^t$ является вероятностной мерой на $(Y,\mathbf{Y})$ и при 
любом $B\hm\in Y$ является измеримой функцией относительно $y^t,z^t$. Элементы 
последовательности~$\sigma$ называются (допустимыми) \textit{правилами}.

%\smallskip
  
  Введем обозначение для прямых произведений множеств:
  $$
  \Omega_0=X\,;\enskip \Omega_t=X^{t+1}\times Y^t\times Z^t\,,\enskip t>0\,,
  $$
а также для наименьших $\sigma$-ал\-гебр, порожденных соответствующими 
$\sigma$-ал\-геб\-рами:
$$
\mathbf{F}_0=\mathbf{X}\,;\enskip \mathbf{F}_t=\mathbf{X}\otimes \mathbf{Y}\otimes 
\mathbf{Z}\otimes \mathbf{X}\otimes \cdots \otimes \mathbf{Y}\otimes \mathbf{Z}\otimes 
\mathbf{X}
$$
($\mathbf{X}$ повторяется $t+1$ раз, $\mathbf{Y}$ и $\mathbf{Z}$~--- $t$ раз, $t\hm>0$).
  
  Положим
  
  \vspace*{3pt}
  
  \noindent
  $$
  \Omega =\prod\limits_{t\geq 0}\Omega_t\,;\enskip 
\mathbf{F}=\mathop{\otimes}\limits_{t\geq0}\mathbf{F}_t\,.
  $$ 
  
  Согласно общей теории~\cite{3-kon} последовательности $o\hm=(\mu,\nu)$ и~$\sigma$ 
порождают на пространстве $(\Omega, \mathbf{F})$ вероятностную меру $\mathbf{P}\hm= 
\mathbf{P}_{o,\sigma}\hm=\mathbf{P}_{\mu,\nu,\sigma}$, которая согласована с 
элементами этих последовательностей следующим образом. Случайные 
последова\-тель\-ности

\vspace*{-3pt}

\noindent
  \begin{gather*}
  x_t=x_t(\omega)\,;\enskip  
  y_{t+1}=y_{t+1}(\omega)\,;\\
  z_{t+1}= z_{t+1}(\omega)\,,\enskip  \omega\in \Omega\,,\  t\geq 0\,,
  \end{gather*}
удовлетворяют соотношениям:

\pagebreak

\noindent
$$
\mathbf{P}(x_0(\omega)\in A_0)=\int\limits_{A_0} \mu_0(dx_0)\,;
$$

\vspace*{-12pt}

\noindent
\begin{multline*}
\mathbf{P}\left(x_0(\omega)\in A_0\,,\  y_1(\omega)\in B_1\,,\ 
z_1(\omega)\in C_1, \ldots \right.\\[1pt]
\left.{}\ldots\,,
y_t(\omega)\in B_t\,,\  z_t(\omega)\in C_t\,,\  x_t(\omega)\in A_t\right)={}\\[1pt]
{}=\int\limits_{A_0}\mu_0(dx_0)\int\limits_{B_1}\sigma_1(dy_1)\int\limits_{C_1}\nu_1(dz_1
\vert x_0, y_1)\cdots{}\\[1pt]
{}\cdots
\int\limits_{B_t}\sigma_t\left(dy_t\vert y^{t-1},z^{t-1}\right) 
\int\limits_{C_t} \nu_t\left( dz_t\vert x^{t-
1},y^t\right) \times{}\\[1pt]
{}\times
\int\limits_{A_t} \mu_t\left( dx_t\vert x^{t-1},y^t\right)
\end{multline*}
для любых $A_t\in X$, $B_{t+1}\hm\in Y$, $C_{t+1}\hm\in Z$, $t\hm\geq 0$.
  
  По определению стратегии, ее правила зависят от предыдущих управлений и 
наблюдений, но не от предыдущих состояний. Это соответствует предположению о том, 
что состояния объекта не наблюдаемы в ходе процесса управления. В~частных случаях 
объект $o\hm=(\mu,\nu)$ может, конечно, описывать полностью наблюдаемый процесс. 
Например, если все множества $X_t$ содержат один и тот же единственный элемент. 
Другой простой пример~--- когда наблюдения тождественны состояниям. Однако на 
самом деле, как показывает лемма~1, с формальной точки зрения рассмотрение объекта с 
<<ненаблюдаемой>> компонентой всегда можно заменить изучением полностью 
наблюдаемого процесса.
  
  \medskip
  
  \noindent
  \textbf{Лемма 1.} \textit{Для любого объекта $o\hm=(\mu,\nu)$ условная вероятность 
$\mathbf{P}\left(dz_t\vert y^t,z^{t-1}\right)$ не зависит от стратегии~$\sigma$ при любых 
$t\hm>0$.}
  
  \medskip
  
  \noindent
  Д\,о\,к\,а\,з\,а\,т\,е\,л\,ь\,с\,т\,в\,о\,.\ Согласно отмеченной выше согласованности 
условных распределений $\mu,\nu,o$ и порождаемой ими меры~\textbf{P} имеем 
соотношения:
  \begin{multline*}
  I_1=\mathbf{P}\left(
  y_1(\omega)\in B_1,\ z_1(\omega)\in C_1\right) ={}\\[1pt]
  {}=
  \mathbf{P}\left( x_0(\omega)\in X_0\,,\ y_1(\omega)\in B_1\,,\ z_1(\omega)\in 
C_1\right)={}\\[1pt]
  {}=\int\limits_{X_0} \int\limits_{B_1} \int\limits_{C_1} \mu_0\left(dx_0\right) 
\sigma_1\left(dy_1\right) \nu_1\left(dz_1\vert x_0,y_1\right)={}\\[1pt]
  {}= \int\limits_{B_1}\int\limits_{C_1}\sigma_1\left(dy_1\right) \int\limits_{X_0}\mu_0\left( 
dx_0\right) \nu_1\left( dz_1\vert x_0,y_1\right)\,,
  \end{multline*}
справедливые при любых $B_1\hm\in Y$ и $C_1\in Z$. Кроме того, по определению 
условной вероятности
$$
I_1=\int\limits_{B_1}\int\limits_{C_1}\sigma_1\left(dy_1\right) \mathbf{P}\left(dz_1\vert 
y_1\right)\,.
$$
  
  Сравнивая оба выражения для~$I_1$, получаем, что
  $$
  \mathbf{P}\left( dz_1\vert 
y_1\right)=\int\limits_{X_0}\mu_0\left(dx_0\right)\nu_1\left(dz_1\vert x_0, y_1\right)\,,
  $$
т.\,е.\ утверждение леммы справедливо для $t\hm=1$. Пусть оно верно для $n\hm=1, 2, 
\ldots , t\hm-1$. Для любых $B_1\hm\in Y$, $C_1\hm\in Z$, \ldots , $B_{t-1}\hm\in Y$, 
$C_t\hm\in Z$ имеем:

\noindent
\begin{multline*}
I_t=\mathbf{P}\left( y_1(\omega)\in B_1\,,\ z_1(\omega)\in C_1, \ldots{}\right.\\[1pt]
\left.{}\ldots , y_t(\omega)\in B_t\,,\ 
z_t(\omega) \in C_t\right)={}\\[1pt]
{}=
\mathbf{P}\left( x_0(\omega)\in X\,,\ y_1(\omega)\in B_1\,,\ z_1(\omega)\in C_1\,, \ldots\right.\\[1pt]
\left.{}\ldots , x_{t-
1}(\omega)\in X\,,\ y_t(\omega)\in B_t\,,\ z_t(\omega)\in C_t\right)={}\\[1pt]
{}=
\int\limits_X \int\limits_{B_1} \int\limits_{C_1}\ldots \\[1pt]
\ldots\int\limits_X \int\limits_{B_t} 
\int\limits_{C_t} \mu_0\left( dx_0\right) \sigma_1\left( dy_1\right) \nu_1\left( dz_1\vert 
x_0,y_1\right)\cdots{}\\[1pt]
\cdots \mu_{t-1}\left( dx_{t-1}\vert x^{t-2} y^{t-1}\right) \sigma_t \left( dy_t\vert y^{t-
1},z^{t-1}\right)\times{}\\[1pt]
{}\times \nu_t\left( dz_t\vert x^{t-1},y^t\right)={}\\[1pt]
{}=\int\limits_{B_1} \sigma_1\left( dy_1\right) \int\limits_{C_1} \int\limits_{B_2} 
\sigma_2\left( dy_2\vert z_1\right)\cdots\\[1pt]
\cdots \int\limits_{C_{t-1}}\int\limits_{B_t} \sigma_t \left( 
dy_t\vert y^{t-1},z^{t-1}\right)\times{}\\[1pt]
{}\times \int\limits_X \mu_0\left(dx_0\right) \nu_1\left( dz_1\vert 
x_o,y_1\right)\cdots{}\\[1pt]
{}\cdots \int\limits_{X_{t-1}}\mu_{t-1}\left( dx_{t-1}\vert x^{t-2} y^{t-1}\right) \nu_t \left( 
dz_t\vert x^{t-1}, y^t\right)={}\\[1pt]
{}=\int\limits_{B_1} \sigma_1\left( dy_1\right) \int\limits_{C_1} 
\int\limits_{B_2}\sigma_2\left( dy_2\vert z_1\right)\cdots\\[1pt]
\cdots \int\limits_{C_{t-1}} 
\int\limits_{B_t} \sigma_t\left( dy_t\vert y^{t-1},z^{t-1}\right) \int\limits_{C_t} 
\mathbf{P}\left( dz_1\vert y_1\right)\ldots{}\\[1pt]
{}\cdots \mathbf{P}\left( dz_{t-1}\vert y^{t-1},z^{t-2}\right) \mathbf{P}\left( dz_t\vert y^t, 
z^{t-1}\right)\,.
\end{multline*}
Отсюда получаем, что

\noindent
  \begin{multline*}
\hspace*{-6.95218pt}\mathbf{P}\left( dz_1\vert y_1\right)\cdots \mathbf{P}\left( dz_{t-1}\vert y^{t-1},z^{t-
2}\right) \mathbf{P}\left( dz_t\vert y^t,z^{t-1}\right)={}\\[1pt]
  {}=\int\limits_X \mu_0\left( dx_0\right) \nu_1\left( dz_1\vert x_o,y_1\right)\cdots \\[1pt]
  \cdots
\int\limits_X \mu_{t-1}\left( dx_{t-1}\vert x^{t-2}y^{t-1}\right) \nu_t\left( dz_t\vert x^{t-
1},y^t\right)\,.
  \end{multline*}
  
  Следовательно, по предположению индукции $\mathbf{P}\left( dz_t\vert y^t,z^{t-
1}\right)$ не зависит от~$\sigma$.
  
  Таким образом, не уменьшая общности, можно ограничиться (что и будет сделано в 
оставшейся части текста) рассмотрением полностью наблюда-\linebreak\vspace*{-12pt}

\pagebreak

\noindent
емых объектов $o\hm=\mu$, 
управляемых (допустимыми) стратегиями~$\sigma$ c правилами вида
  $$
  \sigma_1=\sigma_1\left(\cdot\right)\,;\enskip \sigma_{t+1}=\sigma_{t+1}\left( \cdot \vert 
y^t,x^t\right)\,,\enskip t>0\,.
  $$
(Множество всех таких стратегий при заданных пространствах состояний и управлений 
далее обозначается через~$\Sigma$.) В~этом случае вероятностная мера 
$\mathbf{P}\hm=\mathbf{P}_{\mu,\sigma}$ определена на пространстве $(\Omega, 
\mathbf{F})$, в котором $\Omega\hm=\prod\limits_{t\geq0} X^{t+1}\times Y^t$, 
$\mathbf{F}\mathop{\otimes}\limits_{t\geq0} \mathbf{F}_t$, где $\mathbf{F}_0\hm=\mathbf{X}$; 
$\mathbf{F}_t=\mathbf{X}\otimes \mathbf{Y}\otimes \mathbf{X}\otimes \cdots \otimes 
\mathbf{Y}\otimes \mathbf{X}$ и согласована с последовательностями~$\mu$ и~$\sigma$. 
Через $\mathbf{F}_t$ обозначена $\sigma$-ал\-геб\-ра, порожденная предысторией 
$(x^t,y^t)$ до момента~$t$ включительно.
  
  В то же время необходимо заметить, что предположение о наличии 
<<двухступенчатой>> структуры у объектов (со\-сто\-яние--наблю\-де\-ние) может 
принести пользу при их изучении. Так происходит, например, в теории частично 
наблюдаемых управляемых марковских процессов.
  
  Предположим далее, что на наблюдаемой части траектории процесса задан 
одношаговый доход (в момент~$t$), и будем считать, что этот доход имеет вид 
$g_t\hm=g(x_t)$, где $g:\ X\rightarrow (0,\,1)\subset \mathbb{R}$~--- измеримая числовая 
функция со значениями из интервала (0,\,1).
  
  Обозначим через $v_{t,s}\hm=s^{-1}\sum\limits_{n=1}^s g_{t+n}$ среднее 
арифметическое доходов на промежутке от $t+1$ до $t\hm+s$ ($t\hm\geq0$, $s\hm\geq 1$).
  
  Если объект~$\mu$ управляется согласно стратегии~$\sigma$, то число
  $$
  w_t(\mu,\sigma) =\sup \left\{ c:\ \mathbf{P}_{\mu,\sigma} \left( 
\lim\limits_{\overline{s\rightarrow\infty}} v_{t,s}>c\right) =1\right\}
  $$
характеризует получаемый при этом гарантированный предельный средний доход 
начиная с момента $t=1$. Поскольку $\lim\limits_{\overline{s\rightarrow\infty}} v_{t,s}$ не 
зависит от~$t$, то $w_0(\mu,\sigma)\hm=w_1(\mu,\sigma)\hm=w_2(\mu,\sigma)\hm=\cdots$. 
Величина $w(\mu,\sigma)\hm=w_0(\mu,\sigma)$ играет в дальнейшем роль целевой 
функции и называется просто \textit{доходом} (при управлении объектом~$\mu$ с 
помощью стратегии~$\sigma$).
  
  Из определения дохода следует, что для любого $t>0$ выполняется условие
  $$
  \mathbf{P}_{\mu,\sigma}\left( \lim\limits_{\overline{s\rightarrow\infty}} v_{t,s}\geq 
w(\mu,\sigma)\vert \mathbf{F}_{t-1}\right)=1
  $$
почти наверное.
  Столь общее определение дохода, без предположений об эргодичности, оказывается 
полезным в теоретических рассмотрениях, однако на практике все же среднее 
арифметическое ведет себя более или менее регулярным образом. Поэтому введем 
следующее определение.
{ %\looseness=1

}
  
  Стратегия~$\sigma$ называется \textit{эргодической} по отношению к классу~$M$, 
если для любого объекта $\mu\hm\in M$ и любого $\varepsilon\hm>0$ выполняется 
условие $\sum\limits_{s=1}^\infty a_s\hm<\infty$, где $a_s\hm= 
a_s(\mu,\sigma,\varepsilon)\hm=\sup\limits_{t\geq0} \mathbf{P}_{\mu,\sigma}\left( \left\vert 
v_{t,s}-w(\mu,\sigma)\right\vert >\varepsilon\vert \mathbf{F}_t\right)$. Обозначим еще
  $$
  W=W(\mu) =\sup\limits_\sigma w(\mu,\sigma)\,,
  $$
где точная верхняя грань берется по всем допустимым стратегиям. Стратегия~$\sigma$ 
называется $\varepsilon$-\textit{оп\-ти\-маль\-ной}, если выполняется неравенство
$$
w(\mu,\sigma)\geq W-\varepsilon\,,\enskip \varepsilon\geq 0\,.
$$
  
  Далее объекты будут объединяться в множества объектов (классы объектов). При этом 
без дополнительных оговорок всюду предполагается, что
  \begin{itemize}
  \item все объекты из класса имеют одинаковые пространства состояний, управлений (и 
наблюдений);
  \item в качестве множества допустимых стратегий берется определенное выше 
множество~$\Sigma$;
  \item функция одношаговых доходов~$g$ одна и та же для всех объектов.
  \end{itemize}
  
  Пусть $M$~--- класс объектов. Стратегия~$\sigma$ является равномерно 
  $\varepsilon$-оп\-ти\-маль\-ной относительно этого класса, если последнее неравенство 
выполняется для всех $\mu\hm\in M$. Такую стратегию будем называть также 
  $\varepsilon$-\textit{адап\-тив\-ной} по отношению к классу~$M$. Класс объектов, для 
которого существует $\varepsilon$-адап\-тив\-ная стратегия, называется 
  $\varepsilon$-\textit{адап\-тив\-но управ\-ля\-емым}. (Если $\varepsilon\hm=0$, то 
приставка <<$\varepsilon$->> в этих определениях опускается.)
  
  Основная задача адаптивного управления заключается в построении адаптивных 
стратегий для различных классов объектов. 

К~настоящему вре\-ме\-ни получено много 
решений для многочисленных вариантов этой задачи. Подобные результаты являются 
фактически достаточными условиями адаптивной управ\-ля\-емости. Ниже, однако, будет 
уделено внимание также необходимым условиям существования адаптивных стратегий. 
Подчеркнем, что рассматриваемая постановка задачи предполагает, по сути, наличие 
лишь минимальной априорной информации об объекте управления~--- необходимо знать 
множество управлений~$Y$.

\section{Некоторые условия адаптивной управляемости}

  Пусть $\mu\in M$~--- фиксированный объект, а $\sigma\hm\in \Sigma$~--- 
фиксированная стратегия из некоторой среды. Набор, состоящий из первых $t$ правил 
стратегии~$\sigma$, будем обозначать через $\sigma^t\hm=(\sigma_1, \ldots , \sigma_t)$. 
Таким образом, $\sigma\hm=(\sigma^t, \sigma_{t+1},\sigma_{t+2}, \ldots)$. Положим
  $$
  w_t^*(\mu,\sigma) =w_t^*(\mu,\sigma^t)=\sup\limits_{\sigma_{t+1},\sigma_{t+2}, \ldots} 
w_t(\mu,\sigma)\,,
  $$
где верхняя грань берется по всем допустимым правилам начиная с момента $t\hm+1$. В 
этих обозначениях $w_0^*(\mu,\sigma) \hm=W(\mu)$. Ясно, что $W(\mu)\hm\geq 
w_1^*(\mu,\sigma)\hm\geq w_2^*(\mu,\sigma)\geq \cdots$
  
  Стратегию~$\sigma$ назовем $\varepsilon$-\textit{по\-вреж\-да\-ющей} для 
объекта~$\mu$, если
  $$
  \inf\left\{ t:\ w_t^*(\mu,\sigma)<W(\mu)-\varepsilon\right\} <\infty\,,\enskip \varepsilon>0\,.
  $$
  
  Пример~1 показывает, что существуют объекты, для которых каждая стратегия~--- 
$\varepsilon$-по\-вреж\-да\-ющая (с разными значениями~$\varepsilon$).
  
  \medskip
  
  \noindent
  \textbf{Пример~1.} Множество~$X$ состояний объекта~$\mu$ образовано точками с 
неотрицательными целочисленными координатами на плоскости, $X\hm=\{ (i,j), 
i\hm\geq0,\ j\hm\geq0\}$. Множество управлений $Y\hm=\{1;2\}$. Начальное состояние 
$x_0=(0,\,0)$. Детерминированные переходы между состояниями заданы следующим 
образом ($t\hm>0$, $i\hm\geq0$):
  \begin{align*}
  \mu_t\left( x_t=(i+1{,}0)\vert x_{t-1}=(i,0),y_t=1\right)&=1\,;\\
  \mu_t\left( x_t=(i,j+1)\vert x_{t-1}=(i,j),y_t=1\right)&=1\,,\ j>0\,;\\
  \mu_t\left( x_t=(i,j+1)\vert x_{t-1}=(i,j),y_t=2\right)&=1\,, j\geq 0\,.
  \end{align*}
  
  Одношаговые доходы определены как $g(i,0)\hm=0$, $g(i,j)\hm=1-2^{-i}$ для $i\geq 0$, 
$j\hm>0$.
  
  Стратегия, состоящая из бесконечного повторения управления~1, приносит доход~0. 
Стратегия, в которой управление~2 первый раз применяется (детерминировано) в 
момент~$t$, приносит доход $1\hm-2^{t-1}$, что меньше максимально возможного 
на~$2^{t-1}$. Рандомизация правил и их зависимость от предыстории не вносит 
принципиальных изменений~--- каждая стратегия остается 
  $\varepsilon$-по\-вреж\-да\-ющей относительно предельно наибольшего, но 
недостижимого значения~1.
  
  В примере~2 оптимальная стратегия для любого объекта из класса является 
повреждающей для остальных объектов.
  
  \medskip
  
  \noindent
  \textbf{Пример~2.} Пусть $X\hm= \{0, 1, 2, \ldots\}\cup \{a,b\}$; $Y\hm=\{0;\,1\}$; 
$g(a)\hm=1$; $g(b)\hm=g(i)\hm=0$, $i\hm\geq0$. Зададим счетное множество объектов 
$M\hm=\{\mu^{(k)},\ k\hm=0, 1, \ldots\}$. Пусть для всех~$k$:
  \begin{align*}
  \mu^{(k)}(x_0=0)&=1\,;\\
  \mu^{(k)}(x_{t+1}=i+1\vert x_t=i, y_t=0)&=1\,,\enskip i\geq0\,;\\
     \mu^{(k)}(x_{t+1}=a\vert x_t=k,y_t=1)&=1\,;\\
     \mu^{(k)}(x_{t+1}=b\vert x_t=i,y_t=1) &=1\,,\enskip i\not=k\,;\\
     \mu^{(k)}(x_{t+1}=a\vert x_t=a,y_t=j)&={}\\
&\hspace*{-45mm}{}=\mu^{(k)}(x_{t+1}=b\vert 
x_t=b,y_t=j)=1\,,\enskip j=0\vee 1\,.
     \end{align*}
  
  Таким образом, состояния $a$ и $b$~--- погло\-ща\-ющие, причем в состояние~$a$, 
приносящее максимальный доход, объект~$\mu^{(k)}$ может попасть, только если 
применить управление~1, находясь в со\-сто\-янии~$k$. Первые (существенные) правила 
оптимальной стратегии для объекта~$\mu^{(k)}$ требуют применения управления~0 до 
достижения состояния~$k$, а затем применения в этом состоянии управления~1. Однако 
такая стратегия является повреждающей для всех остальных объектов. Следовательно, для 
класса~$M$ не существует равномерно оптимальной стра\-тегии.
{\looseness=1

}
  
  Пусть $M$~--- класс объектов. Обозначим через $\Sigma_\varepsilon(\mu)$ множество 
$\varepsilon$-по\-вреж\-да\-ющих стратегий для объекта~$\mu$, $\mu\hm\in M$. Положим 
$\Sigma_\varepsilon(M)\bigcap\limits_{\mu\in M}\left( \Sigma\backslash 
\Sigma_\varepsilon(\mu)\right)$.
  
  \medskip
  
  \noindent
  \textbf{Лемма~2.} \textit{Для того чтобы существовала $\varepsilon$-адап\-тив\-ная 
стратегия, необходимо, чтобы $\Sigma_\varepsilon(M)\not=\emptyset$.}
  \medskip
  
  \noindent
  Д\,о\,к\,а\,з\,а\,т\,е\,л\,ь\,с\,т\,в\,о\,.\ Если $\Sigma_\varepsilon\not= \emptyset$, то любая 
допустимая стратегия хотя бы для одного из объектов является 
  $\varepsilon$-по\-вреж\-да\-ющей и, следовательно, не является 
  $\varepsilon$-оп\-ти\-маль\-ной, а потому не может быть равномерно 
  $\varepsilon$-оп\-ти\-маль\-ной по отношению к классу~$M$.
  
  В примере~3, несмотря на наличие по\-вреж\-да\-ющих стратегий, адаптивная стратегия 
существует.
  
  \medskip
  
  \noindent
  \textbf{Пример~3.} Пусть $X\hm=Y\hm=\{1, \ldots , K\}$ и пусть задана 
детерминированная функция~$f:\ X\hm\rightarrow X$, которая представляет собой 
циклическую подстановку на множестве~$X$,  т.\,е.\ $f(i)\not= f(j)$, если $i\not= j$; 
$i,j\hm=1, \ldots , K$. Рассмотрим следующий неоднородный во времени 
детерминированный объект. Положим
  \begin{align*}
  \mu_0(x_0=1)&=1\,;\\
  \mu_t(x_t=f(k)\vert x^{t-1},y^t) &= I_{\{y_t=k\}}\,,\ 0<k\,,\ t\leq K\,;\\
  \mu_t(x_t=f(k)\vert x^{t-1},y^t) &=I_{\{y_{K+1}=k}\,,\\
  & \hspace*{10mm}0<k\leq K\,,\enskip t>K
  \end{align*}
($I_A$~--- индикатор события~$A$).
  
  Одношаговые доходы определим как $g(i)\hm=i$, $i\hm\in X$.
  
  Так определенный объект обозначим через~$\mu^f$. Ясно, что для этого объекта 
траектория управ\-ля\-емо\-го процесса, начиная с момента $K+1$, и, следовательно, доход 
зависят исключительно от управ\-ле\-ния, примененного в момент $K+1$. Доход будет 
максимален (и равен~$K$) тогда и только тогда, когда $y_{K+1}\hm= k^\prime \hm= 
k^*(f)\hm=\argmax\limits_{1\leq k\leq K} f(k)$.
  
  Пусть $M=\{\mu^f\}$~--- совокупность всех объектов данного вида (которая содержит 
$K!$ элементов). Очевидно, для класса~$M$ существует равномерно оптимальная 
стратегия, доставляющая доход, равный~$K$. Например, достаточно вначале в моменты 
$t\hm=1, \ldots , K$ по одному разу применить каждое из управлений, а затем в момент 
$K+1$ применить управление~$k^*$, которое будет выявлено путем наблюдения за 
полученными одношаговыми доходами. Таким образом, на первых тактах необходимо совершить 
<<обучение>>~--- выявить управление, приносящее наибольший одношаговый доход. 
В~то же время существуют и повреждающие стратегии. Например, стратегия, в которой 
первые $K$ правил заключаются в применении управления~1. Правило~$\sigma_{K+1}$ 
такой стратегии может быть построено только в виде зависимости от управления~1 и от 
значения $f(1)$, поэтому при любом его определении найдется объект~$\mu^f$, для 
которого в момент $K+1$ будет с положительной вероятностью предписано применение 
неоптимального управления, и, следовательно, доход будет меньше~$K$.
  
  В примере 3 <<обучение>> оказалось возможным только благодаря знанию структуры 
процессов. Если бы заранее не было известно, что необходимо на первых тактах по разу 
<<испробовать>> все управ\-ле\-ния, то легко можно было пропустить период, когда 
возможно обучение, и совершить тем самым <<непоправимую ошибку>>. Следовательно, 
для того чтобы конструктивно построить равномерно оптимальную стратегию, 
необходима дополнительная информация. Это противоречит избранному принципу 
постановки задачи~--- минимальности априорной информации об объекте. 

Введем более 
жесткое определение адаптивной стратегии, которое, в част\-ности, устраняет указанное 
несоответствие.
  
  Пусть $M$~--- некоторый класс объектов. Эргодическая стратегия~$\sigma$ (ее 
определение дано в конце разд.~2) называется \textit{устойчивой} по отношению к 
классу~$M$, если для любого объекта $\mu\hm\in M$ стратегия~$\tilde{\sigma}$, 
полученная из стратегии~$\sigma$ путем произвольной (допустимой) замены конечного 
числа правил, (1)~имеет одинаковый со стратегией доход 
$w(\mu,\sigma)\hm=w(\mu,\tilde{\sigma})$ и (2)~является эргодической по отношению к 
классу~$M$.

%\columnbreak
  
  Адаптивная стратегия для класса~$M$ называется \textit{строго адаптивной}, если она 
устойчивая по отношению к этому классу.
  
  \medskip
  
  \noindent
  \textbf{Пример~4.} Легко показать, что строго адаптивными являются 
многочисленные адаптивные стратегии для класса управляемых конечных связных 
марковских цепей~[1, 2].
  
  Рассмотрим еще один мотив, выдвигаемый в качестве необходимого условия 
адаптивной управ\-ля\-емости.
  
  \medskip
  
  \noindent
  \textbf{Пример~5.} Пусть класс объектов состоит из функций вещественного 
аргумента~$u$ вида $\mu^y\hm=\mu^y(u)\hm=I_{\{u=y\}}$, $y\hm\in [0,\,1]$. (В~терминах 
управляемых случайных последовательностей: $X\hm= \{0;1]\}$, $Y\hm=[0,1]$; 
$\mu_t(x_t\vert x^{t-1},y^t)\hm=x_t I_{\{y_t=y\}}+ (1-x_t)I_{\{y_t=y\}}$; $g(x)\hm=x$, 
$x\hm\in X$.) Интуитивно представляется очевидным, что невозможно найти максимум 
такой функции за счетное число шагов, если не знать значение, в котором она обращается 
в единицу. В~то же время формально для каждого объекта~$\mu^y$ существует 
оптимальная стратегия. Например, можно постоянно повторять управление~$y$. Однако 
не существует стратегии, равномерно оптимальной по отношению к классу 
$M\hm=\{\mu^y\}$. В~такой стратегии для каждого $y\hm\in [0,\,1]$ необходимо должно 
было бы выполняться следующее условие: $\sigma_t(y_t=y\vert \cdot)>0$ хотя бы для 
одного значения~$t$. Но это невозможно, поскольку для фиксированного значения~$t$ 
данное неравенство может быть выполнено лишь для счетного множества значений~$y$, а 
$t$ также пробегает счетное множество значений. Счетное объединение счетных 
множеств само счетно, поэтому необходимое неравенство не может быть выполнено для 
всех точек на отрезке [0,\,1].
  
  Аналогичные рассуждения показывают, что в данном примере не существует счетного 
множества стратегий, обладающего тем свойством, что для любого объекта найдется 
$\varepsilon$-оп\-ти\-маль\-ная стратегия из этого множества.
  
  Конечное или счетное множество стратегий $\Sigma\hm=\{\sigma(1),\sigma(2), \ldots \}$ 
назовем \textit{базовым} по отношению к классу объектов $M\hm\in \mathcal{M}$, если:
  \begin{enumerate}[(1)]
  \item для любого объекта из $M$ и любого $\varepsilon\hm>0$ существует оптимальная 
стратегия из множества~$\Sigma$;
  \item любая стратегия $\sigma(i)$ является устойчивой по отношению к классу~$M$.
  \end{enumerate}
  
  \smallskip
  
  \noindent
  \textbf{Теорема.} \textit{Строго адаптивная стратегия для класса объектов~$M$ 
существует тогда и только тогда, когда для этого класса существует базовое 
множество стратегий~$\Sigma$.}


%\hfill {\large Приложение~1}

\bigskip

%\pagebreak

\noindent
Д\,о\,к\,а\,з\,а\,т\,е\,л\,ь\,с\,т\,в\,о\ \ теоремы.

Необходимость условий в данном случае является тривиальной, поскольку строго 
адаптивная стратегия, если она существует, образует базовое множество 
стратегий~$\Sigma$, состоящее из одного элемента.
  
  Докажем достаточность. Определим с по\-мощью стратегий из~$\Sigma$ новую 
стратегию $a$ следующим образом. Обозначим
  $$
  \theta_{t,n}=\mathrm{Int}\left(\left( 1-v_{t,n}\right)^{-n}\right)\,,
  $$
где $\mathrm{Int}\left(a\right)$ означает целую часть числа~$a$, и зададим 
последовательность марковских моментов $\tau\hm=\{\tau_n\}$ с помощью рекуррентных 
соотношений

\pagebreak

\noindent
$$
\tau_0=0\,,\enskip \tau_n=\tau_{n-1}+n+\theta_n\,,
$$
где $\theta_n\hm=\theta_{\tau_{n-1},n}$. Соответствующие $\sigma$-ал\-геб\-ры обозначим 
$\mathbf{F}_{(n)}\hm=\mathbf{F}_{\tau_{n-1}}$.
  
  Будем считать, что на пространстве $(\Omega,\mathbf{F})$ задана последовательность 
случайных величин $\beta\hm=\{\beta_n\}$, независимых 
относительно~$\mathbf{F}_{(n)}$. Каждая случайная величина имеет одно и то же 
невырожденное распределение $\{b_i\}$ на множестве номеров стратегий из~$\Sigma$.
  
  Определим правила стратегии $a\hm=a(\Sigma,\beta)$ формулой
  $$
  a_t=\sum\limits_{n=1}^\infty \sigma_t(\beta_n) I_{\{\tau_{n-1}<t\leq \tau_n\}}\,,
  $$
где $\sigma_t(\beta_n)$~--- правило стратегии $\sigma(i)\hm\in\Sigma$ в момент~$t$, если 
$\beta_n\hm=i$.
  
  Наглядно работа стратегии~$a$ выглядит следующим образом. Процесс управления 
разбивается на этапы. Этап с номером $n$ начинается в момент $\tau_{n-1}+1$ и 
оканчивается в момент~$\tau_n;\tau_0\hm=0$. В~момент, предшествующий началу 
очередного этапа, определяется номер стратегии в множестве~$\Sigma$, из которой будут 
взяты правила для применения на данном этапе. Этот номер равен значению случайной 
величины~$\beta_n$. Продолжительность $n$-го этапа равна $n\hm+\theta_n$ и зависит, 
следовательно, от номера этапа и от оценки качества применяемой стратегии, полученной 
в течение первых $n$ тактов этапа. Стратегия~$a$ называется стратегией перебора~[2]. 
Таким образом, последовательность~$\beta$ определяет на каждом этапе выбор стратегии 
из множества~$\Sigma$, правила из которой применяются на этом этапе.
  
  Пусть задан объект $\mu\hm\in M$ и пусть $W\hm=W(\mu)$~--- точная верхняя грань 
доходов для этого объекта, взятая по всем допустимым стратегиям, и пусть %также
  \begin{alignat*}{2}
  W_i&=w(\mu,\sigma(i))\,; &\enskip v_n^{(1)}&=v_{\tau_{n-1},n}\,;\\
  v_n^{(2)}&=v_{\tau_{n-1},n+\theta_n}\,; &\enskip \Delta_n&=\tau_n-\tau_{n-1}=n+\theta_n\,.
  \end{alignat*}
  
  Для произвольного $\varepsilon>0$ определим множества
  $$
  A_n^{(k)}(\varepsilon)=\left\{ v_n^{(k)}\geq W-\varepsilon\right\}\,,
  $$
обозначая их дополнения $\overline{A_n^{(k)}(\varepsilon)}$, $k=1, 2$.
  
  Обозначим
  \begin{align*}
  s_n^{(1)} &= \sum\limits_{l=1}^n I_{A_l^{(1)}(\varepsilon)\cap 
{A_l^{(2)}(2\varepsilon)}} \Delta_l\,;\\
  s_n^{(2)} &= \sum\limits_{l=1}^n I_{A_l^{(1)}\cap 
\overline{A_l^{(2)}(2\varepsilon)}}\Delta_l\,;\\
  s_n^{(3)} &= \sum\limits_{l=1}^n I_{\overline{A_l^{(1)}(\varepsilon)}}\Delta_l\,,
  \end{align*}
так что $\tau_n\hm=\sum\limits_{l=1}^n \Delta_l\hm= s_n^{(1)}\hm+ s_n^{(2)}\hm+ 
s_n^{(3)}$.

\columnbreak

  
  С~помощью введенных обозначений запишем оценку для усредненного дохода к 
моменту~$\tau_n$:
  \begin{multline}
  w_n=\fr{1}{\tau_n}\sum\limits_{t=1}^{\tau_n} g_t=\fr{\sum\limits_{l=1}^n 
v_l^{(2)}\Delta_l} {\sum\limits_{l=1}^n \Delta_l}\geq{}\\
{}\geq (W-2\varepsilon) \fr{s_n^{(1)}} 
{s_n^{(1)}+s_n^{(2)}+s_n^{(3)}}\,.
  \label{e1-kon}
  \end{multline}
  
  Для оценки суммы $s_n^{(1)}$ запишем неравенство
  $$
  s_n^{(1)}\geq \Delta_{v_n}\,,
  $$
в котором обозначено
$$
v_n=\max\left\{ l:\ l\leq n,\ A_l^{(1)}(\varepsilon)\cap A_l^{(2)}(2\varepsilon)\right\}\,.
$$
  
  Оценим вероятность события $B_n\hm=\{v_n\hm\leq n-\ln n\}$, для которого выполняется 
включение
  $$
  B_n\subset \bigcap\limits_{n-\ln n<l\leq n} 
  \overline{A_l^{(1)}(\varepsilon)}\cap \overline{A_l^{(2)}(2\varepsilon)}\,.
  $$
  
  Согласно определениям эргодической стратегии, базового множества стратегий и 
семейства случайных величин~$\beta$ имеем:
  \begin{multline*}
  \mathbf{P}_{a} \left( \overline{A_l^{(1)}(\varepsilon)}\cup\overline{A_l^{(2)} 
(2\varepsilon)}\,\Big\vert \mathbf{F}_{(l)}\right)\leq{}\\
  {}\leq
  \sum\limits_{\substack{{i\in \mathcal{I};}\\ {W_i\leq W-\varepsilon/2}}}\!\!\!\!
   \mathbf{P}_{a}\left(\beta_l=i\vert 
\mathbf{F}_{(l)}\right)+{}\\
{}+  %\substack{{i=\overline{1,n}}\\ {j=\overline{1,l}}}
\sum\limits_{\substack{{i\in \mathcal{I};}\\ {W_i\leq W-\varepsilon/2}}}\!\!\!\!
\mathbf{P}_{a}\left( \overline{A_l^{(1)}(\varepsilon)}, \ \beta_l=i
\vert \mathbf{F}_{(l)}\right)\leq{}\\
  {}\leq \sum\limits_{\substack{{i\in \mathcal{I};}\\ {W_i\leq W-\varepsilon/2}}}\!\!\!\!
  \mathrm{P}_{a}(\beta_l=i)+{}\\
{}+\sum\limits_{\substack{{i\in \mathcal{I};}\\ {W_i> W-
\varepsilon/2}}}
\!\!\!\!\mathbf{P}_{a}\left( v_l^{(1)}\leq W_i-\fr{\varepsilon}{2}, \beta_l=i\vert 
\mathbf{F}_{(l)}\right) \leq{}\\
  {}\leq \sum\limits_{\substack{{i\in \mathcal{I};}\\ {W_i\leq W-\varepsilon/2}}}\!\!\!\!
   b_i+a_l\left( 
\fr{\varepsilon}{2}\right) \leq q<1
  \end{multline*}
при всех достаточно больших~$l$. Отсюда следует, что для всех достаточно больших 
значений~$n$ выполняется неравенство
$$
\mathbf{P}_a(B_n)\leq q^{n-\ln n}\,.
$$
  
  Следовательно, согласно лемме Бо\-ре\-ля--Кан\-тел\-ли
  \begin{equation}
  \mathbf{P}_{a}\left( \overline{\lim\limits_{n\rightarrow\infty}} B_n\right)=0\,.
  \label{e2-kon}
  \end{equation}
  
  Это означает, что
  $$
  s_n^{(1)}\geq \Delta_{v_n}\geq (1-W-\varepsilon)^{-n+\ln n}\,.
  $$
  
  Оценим сумму $s_n^{(2)}$. Обозначив 
$C_n\hm=A_n^{(1)}(\varepsilon)\cap$\linebreak 
$\cap\overline{A_n^{(2)}(2\varepsilon)}$ и $W_{(n)}\hm=\sum\limits_{i\in 
I} W_i I_{\{\beta_n=i\}}$, получим:
  \begin{multline*}
  \mathrm{P}_{a}\left(C_n\vert \mathrm{ F}_{(n)}\right)=
  \mathrm{P}_{a|} \left( C_n, W_{(n)}<W-\fr{3\varepsilon}{2}\vert \mathrm{
  F}_{(n)}\right) +{}\\
  {}+ \mathrm{P}_{a}\left( 
  C_n, W_{(n)}\geq W-\fr{3\varepsilon}{2}\vert \mathrm{
  F}_{(n)}\right)\leq{}\\
  {}\leq \mathrm{P}_{a}\left( v_n^{(1)}>W-\varepsilon,\, W_{(n)}<W-\fr{3\varepsilon}{2}\vert \mathrm{
  F}_{(n)}\right)+{}\\
  {}+
  \mathrm{P}_{a} \left( v_n^{(2)}\leq W-2\varepsilon,\, W_{(n)}\geq W-
\fr{3\varepsilon}{2}\vert \mathrm{
  F}_{(n)}\right)\leq{}\\
  {}\leq \sum\limits_{i\in \mathcal{I}; W_i\leq W- \varepsilon/2} \mathrm{P}_{a}\left(
  v_{\tau_n,n}>W_i+\fr{\varepsilon}{2},\, \beta_l=i\vert\mathrm{F}_{(n)}\right)+{}\\
  {}+\sum\limits_{\substack{{i\in \mathcal{I};}\\ {W_i> W- 3\varepsilon/2}}}\!\!\!\!
   \mathbf{P}_{a} \left( 
v_{\tau_n,n+\theta_n}\leq W_i-\fr{\varepsilon}{2},\,\beta_l=i\vert\mathbf{F}_{(n)}\right)\leq {}\\
{}\leq
a_n\left( \fr{\varepsilon}{2}\right)\,.
  \end{multline*}
  
  Из определения базового множества стратегий следует, что
  $$
  \sum\limits_{n=1}^\infty \mathbf{P}_{a} (C_n)<\infty\,,
  $$
поэтому согласно лемме Бо\-ре\-ля--Кан\-тел\-ли полу\-чаем:
\begin{equation}
\mathbf{P}_{a}\left( \overline{\lim\limits_{n\rightarrow\infty}} C_n\right) =0\,.
\label{e3-kon}
\end{equation}
  
  Отсюда следует, что
  $$
  \sup\limits_n s_n^{(2)}\leq c<\infty\,.
  $$
  
  Для суммы $s_n^{(3)}$ имеем следующую оценку:
  $$
  s_n^{(3)}\geq \sum\limits_{l=1}^n \left(n+(1-W+\varepsilon)^{-l}\right)< n^2+n(1-
W+\varepsilon)^{-n}.
  $$
  
  Подставляя оценки, полученные для сумм $s_n^{(k)}$, в неравенство~(\ref{e1-kon}), 
получаем:
  \begin{multline*}
  w_n\geq (W-\varepsilon) \left( 1+\fr{s_n^{(2)}+s_n^{(3)}}{s_n^{(1)}}\right)^{-1}\geq 
{}\\
  {}\geq (W-\varepsilon)\left( 1+\fr{c+n^2+n(1-W+\varepsilon)^{-n}}{(1-W-\varepsilon/2)^{-
n+\ln n}}\right)^{-1}\geq{}\\
{}\geq W-3\varepsilon
  \end{multline*}
для всех достаточно больших значений~$n$. Отсюда
\begin{equation}
\lim\limits_{\overline{n\rightarrow\infty}} w_n\geq W\,.
\label{e4-kon}
\end{equation}
  
  Рассмотрим далее множество
  $$
  \Omega^\prime =\left\{ \lim\limits_{n\rightarrow\infty} w_n =W\right\}\cap 
\overline{B}\cap\overline{C}\,,
  $$
где $\overline{B}$ и $\overline{C}$ означают соответственно дополнения к множествам 
$B\hm= \overline{\lim\limits_{n\rightarrow\infty}} B_n$ и $C\hm= 
\overline{\lim\limits_{n\rightarrow\infty}} C_n$.
  
  Согласно формулам~(\ref{e2-kon})--(\ref{e4-kon})
  $$
  \mathbf{P}_{a}\left(\Omega^\prime\right) =1\,.
  $$
  
  Определим следующие события:
  
  \noindent
  \begin{align*}
  D_{n,t}^{(1)} &= \left\{ \tau_{n-1}<t\leq \tau_{n-1}+n\right\} \cap \Omega^\prime\,;\\
  D_{n,t}^{(2)} &= \left\{\tau_{n-1}+n<t\leq \tau_n\right\}\cap \Omega^\prime\,;\\
  D_{n,t}^{(3)} &= \left\{ \tau_{n-1}<t\leq \tau_n\right\} \cap \Omega^\prime\,.
  \end{align*}
  
  На множестве $D_{n,t}^{(1)}$ усредненный доход $v_t\hm=v_{0,t}\hm=
  t^{-1}\sum\limits_{s=1}^t g_s$ оценивается с помощью формулы~(\ref{e1-kon}) как
  
    \noindent
  $$
  v_t\geq \fr{\tau_{n-1} w_n}{\tau_{n-1}+n+\theta_n}\geq W-\varepsilon_n^{(1)}\,,
  $$
где $\varepsilon_n^{(1)}\hm\rightarrow0$ при $n\hm\rightarrow\infty$.
  
  Пусть событие $D_{n,t}^{(2)}$ имеет место. Тогда $\theta_n\geq (1\hm- 
W\hm+\varepsilon)^{-n}$. Кроме того, из определения событий $B_n$, $B$, 
$D_{n,t}^{(2)}$ следует, что для всех достаточно больших значений~$n$ выполняется 
неравенство $v_n\hm> n-\ln n$. Следовательно, на множестве~$D_n^{(2)}$ справедлива 
оценка

  \noindent
  $$
  v_t\geq \fr{\tau_{n-1} w_n}{\tau_{n-1}+n+\theta_n}\geq W-\varepsilon_n^{(2)}\,,
  $$
где $\varepsilon_n^{(2)}\hm\rightarrow0$ при $n\hm\rightarrow\infty$.
  
  Из определения событий $C_n$, $C$, $D_{n,t}^{(3)}$ вытекает, что
  
    \noindent
  $$
  D_{n,t}^{(3)} \subset \left\{ \min\limits_{n<m\leq n+\theta_n} v_{n,m}\geq W-
2\varepsilon\right\}\,,
  $$
поэтому на множестве $D_n^{(3)}$ справедливы неравенства:

  \noindent
\begin{multline*}
\!\!v_t\geq \fr{\tau_{n-1} w_n}{t}+\left(1- \fr{\tau_{n-1}}{t}\right) \left( 1-\tau_n\right)^{-1} 
\!\!\sum\limits_{s=\tau_{n-1}+1}^t \!\!\!\!g_s\geq{}\\
{}\geq \fr{\tau_{n-1} w_n}{t}+\left( 1-\fr{\tau_{n-1}}{t}\right)\left( W-2\varepsilon\right) \geq 
W-2\varepsilon -\varepsilon_n^{(3)},
\end{multline*}
где $\varepsilon_n^{(3)}\rightarrow0$ при $n\hm\rightarrow\infty$.

\pagebreak
  
  Таким образом, на множестве
  $$
  D_{n,t}=\bigcup\limits_{k=1}^3 D_{n,t}^{(k)} = \left\{ \tau_{n-1}<t\leq \tau_n\right\} \cap 
\Omega^\prime
  $$
имеет место оценка $v_n\hm\geq W-\varepsilon-\varepsilon_n$, где 
$\varepsilon_n\hm\rightarrow 0$ при $n\hm\rightarrow\infty$. Достаточность утверждения 
теоремы следует из соотношений $\Omega\hm= \bigcup\limits_{n=1}^\infty \left\{ \tau_{n-
1}\hm<t\hm\leq \tau_n\right\}$ и $\lim\limits_{t\rightarrow\infty} I_{D_{n,t}}\hm=0$.

\section{Заключение}

  Адаптивные стратегии, позволяющие достигать цели в условиях информационной 
неопреде\-лен\-ности, основываясь на <<обучении>> в процессе взаимодействия с объектом, 
находят все более широкое практическое применение. 

В~этой работе было уделено 
внимание теоретическим аспектам адаптивного подхода. Сформулированы определения 
адаптивных стратегий и приведена формальная постановка задачи адаптивного 
управления. Сформулированы и доказаны некоторые утверждения о необходимых 
условиях и достаточных условиях адап\-тив\-ной управляемости. 

Продолжение исследований 
в данном на\-прав\-ле\-нии позволит найти ответы на принципиальные вопросы, в каких 
ситуациях можно рассчитывать на <<приспособление к неизвестной среде>> и сколь 
универсальными могут быть <<обучающиеся>> алгоритмы.



{\small\frenchspacing
{%\baselineskip=10.8pt
\addcontentsline{toc}{section}{Литература}
\begin{thebibliography}{9}


  \bibitem{1-kon}
  \Au{Sragovich~V.\,G.}
  Mathematical theory of adaptive control.~--- Singapore: World Scientific, 2006.
  \bibitem{2-kon}
  \Au{Коновалов~М.\,Г.}
  Методы адаптивной обработки информации и их приложения.~--- М.: ИПИ РАН, 2007.
  
  \label{end\stat}
  
  \bibitem{3-kon}
  \Au{Неве~Ж.}
  Математические основы теории вероятностей.~--- М.: Мир, 1969.
\end{thebibliography}
}
}


\end{multicols}  %4
\def\stat{arhipov}

\def\tit{ОПТИМИЗАЦИЯ ФУНКЦИЙ LAB-КОНТРАСТНОГО 
ГРАДАЦИОННОГО ПРЕОБРАЗОВАНИЯ}

\def\titkol{Оптимизация функций Lab-контрастного 
градационного преобразования}

\def\autkol{О.\,П.~Архипов, З.\,П.~Зыкова}

\def\aut{О.\,П.~Архипов$^1$, З.\,П.~Зыкова$^2$}

\titel{\tit}{\aut}{\autkol}{\titkol}

%{\renewcommand{\thefootnote}{\fnsymbol{footnote}}\footnotetext[1] {Статья 
%рекомендована к публикации в журнале Программным комитетом конференции 
%<<Электронные библиотеки: перспективные методы и технологии, электронные 
%коллекции>> (RCDL-2012).}}

\renewcommand{\thefootnote}{\arabic{footnote}}
\footnotetext[1]{Орловский филиал Института проблем информатики Российской академии наук, arkhipov12@yandex.ru} 
\footnotetext[2]{Орловский филиал Института проблем информатики Российской академии 
наук, zykzoya@yandex.ru} 



\Abst{Рассмотрена задача персонифицированного преобразования 
распределения контрастов на ступенчатых тоновых шкалах. Решение задачи 
необходимо для управления отображениями RGB-изоб\-ра\-же\-ний на цветных 
периферийных устройствах персональных электронных вы\-чис\-ли\-тель\-ных машин (ПЭВМ) с целью улучшения их восприятия в части 
детализации. Поскольку изменение соотношения Lab-конт\-рас\-тов цифрового 
описания отображений пикселов влечет за собой подобное, хотя и, возможно, 
менее ярко выраженное изменение соотношения реальных контрастов 
отображений, то для решения задачи достаточно подобрать подходящее 
распределение Lab-конт\-рас\-тов. Для приближенного вычисления функции 
Lab-конт\-раст\-но\-го градационного преобразования ступенчатых тоновых 
шкал по образцу рассматривались два семейства параметрических алгоритмов. 
Параметры одного из них~--- подмножества пикселов шкалы, а другого~--- 
множества градаций. Задача выбора оптимальных параметров решена путем 
сравнения погрешности вычисления функции распределения 
Lab-конт\-рас\-тов на ступенчатых тоновых шкалах на типичных примерах. 
Приведен пример, демонстрирующий эффективность применения 
соответствующей функции Lab-конт\-раст\-но\-го градационного 
преобразования ступенчатых тоновых шкал. Показано, что при выборе 
подходящего образца можно не только избежать искажения детализации, но и 
добиться ее улучшения.}

\KW{цветовоспроизведение; цветовосприятие; Lab-координаты; контраст; 
градации}

\DOI{10.14357/19922264130405}

\vskip 20pt plus 9pt minus 6pt

      \thispagestyle{headings}

      \begin{multicols}{2}

            \label{st\stat}

\section{Введение}
  
  Lab-контрастное градационное преобразование ступенчатых тоновых шкал 
является необходимым компонентом информационной технологии 
равноконт\-растной градационной скелетизации цветового пространства 
восприятия произвольным пользователем вывода цветных изображений на 
периферийные устройства ПЭВМ~[1--9]. 
  
  Информационная технология применяется на практике в обычных условиях 
функционирования офисных компьютерных систем с цветной периферией для 
персонифицированного управления детализацией RGB-изоб\-ра\-же\-ний при 
их печати и выводе на монитор. 
  
  Целью такого управления является улучшение восприятия цветных 
изображений в части детализации не только для пользователей, 
цветовосприятие которых близко к стандарту, но и для тех пользователей, 
которые имеют такие аномалии цветового зрения, как частичная цветовая 
слепота. 
  
  Lab-контрастное градационное преобразование ступенчатых тоновых шкал 
по образцу является инструментом влияния на персонифицированное 
распределение контрастов в отображениях цветных изображений 
(представлениях изображений на периферийных устройствах ПЭВМ), 
позволяющим улучшить детализацию отображений в восприятии 
произвольных пользователей~[6--9]. 
  
  В основе влияния лежит тот факт, что при корректном цифровом описании 
цветовоспроизведения и цветовосприятия изменение соотношения 
  Lab-конт\-рас\-тов цифрового описания отображений пикселов влечет за 
собой подобное, хотя и, возможно, менее ярко выраженное изменение 
соотношения реальных контрастов отображений.
  
  Пусть $x_1$, $x_2$~--- RGB-пик\-се\-лы, $\Psi_n$~--- функция 
цветовосприятия $n$-го наблюдателя, а RGB-функ\-ция~$\varphi_n$~--- 
цифровое описание~$\Psi_n$. 
  
  Обозначим через $\{\Psi_n(x_i)\}$ отображения пикселов в цветовом 
пространстве $n$-го наблюдателя, $\{\varphi_n(x_i)\}$~--- цифровое описание 
этих отображений, $C_n(\Psi_n(x_1),\Psi_n(x_2))$~--- контраст отображений 
  RGB-пик\-се\-лов~$x_1$ и~$x_2$ в цветовом пространстве $n$-го 
наблюдателя. Пусть $E(\varphi_n(x_1),\varphi_n(x_2))$~--- Lab-конт\-раст этих 
пикселов:
\begin{multline*}
  E(\varphi_n(x_1),\varphi_n(x_2)) ={}\\
  {}= \sqrt{(L_{n,1}-L_{n,2})^2+(a_{n,1} - 
a_{n,2})^2 + (b_{n,1}-b_{n,2})^2}\,,\hspace*{-8.09448pt}
  \end{multline*}
где $(L_{n,1}, a_{n,1}, b_{n,1})$~--- это Lab-ко\-ор\-ди\-на\-ты, 
соответствующие RGB-ко\-ор\-ди\-на\-там $\varphi_n(x_1)$, а 
$(L_{n,2},a_{n,2},b_{n,2})$~--- $\varphi_n(x_2)$.

  Обозначим через $x_1^\prime$, $x_2^\prime$ значения пикселов~$x_1$, $x_2$ 
после некоторого преобразования. Если Lab-конт\-раст цифрового описания 
отображений новых значений пикселов значительно превосходит 
  Lab-конт\-раст цифрового описания отображений пикселов~$x_1$ и~$x_2$: 
  $$
E(\varphi_n(x^\prime_1),\varphi_n(x^\prime_2)) \gg E(\varphi_n(x_1), 
\varphi_n(x_2))\,, 
$$
то, как правило, выполняется следующее соотношение:
$$
C_n(\Psi_n(x^\prime_1),\Psi_n(x^\prime_2)) \geq C_n (\Psi_n(x_1),\Psi_n(x_2))\,. 
$$
  
  Из соотношения
  $$
E(\varphi_n(x^\prime_1),\varphi_n(x^\prime_2)) \ll E (\varphi_n(x_1),\varphi_n(x_2)) 
$$
следует
$$
C_n (\Psi_n(x^\prime_1),\Psi_n(x^\prime_2)) \leq C_n (\Psi_n(x_1),\Psi_n(x_2))\,. 
$$

  Приведенные зависимости позволяют определить подход к получению 
нужного распределения контрастов на отображениях шкал, состоящий в 
подборе подходящего распределения Lab-конт\-рас\-тов. 
  В~[6--9] предлагается подбирать распределение 
  Lab-конт\-рас\-тов по образцу. Для задания образцов используются 
неотрицательные числовые функции от пикселов шкал. 
  
  Процедура управления контрастами отображений такова:
  \begin{enumerate}[(1)]
\item задается начальный образец распределения Lab-конт\-рас\-тов;
\item определяется функция Lab-конт\-раст\-но\-го градационного 
преобразования цифрового описания отображений ступенчатых тоновых шкал 
по образцу;
\item вычисляются координаты пикселов преобразованных ступенчатых 
тоновых шкал и цифровое описание их отображений; 
\item вычисляется распределение Lab-конт\-рас\-тов на цифровом описании 
отображений ступенчатых тоновых шкал;
\item преобразованные ступенчатые шкалы выводятся на используемое 
периферийное устройство;
\item пользователем проводится визуальный анализ детализации~--- 
распределения контрастов на отображениях шкал в сопоставлении с графиками 
образца и функции Lab-конт\-рас\-тов;
\item до установления допустимости детализации отображений шкал 
проводится необходимая модификация образца и повторяются ша-\linebreak ги~2--6;
\item после нахождения подходящего образца определяется Lab-конт\-раст\-ное 
градационное преобразование всех пикселов RGB-ку\-ба;
\item выполняется предварительное Lab-конт\-раст\-ное градационное 
преобразование произвольного изображения;
\item преобразованное изображение выводится на используемое периферийное 
устройство.
\end{enumerate}

  В результате в отображениях произвольных изоб\-ражений обеспечивается 
установленное выбранным образцом распределение контрастов. В~качестве 
начального образца распределения Lab-конт\-рас\-тов можно выбрать, 
например, равноконтрастный образец, который определяется функцией 
расстояния между RGB-пик\-се\-ла\-ми, или образец подобия оригиналу, 
который определяется функцией Lab-конт\-рас\-тов исходных тоновых шкал.
  
  Модификация образца приводит к модификации распределения 
  Lab-конт\-рас\-тов. Модификация Lab-конт\-рас\-тов приводит к 
модификации распределения реальных контрастов. Таким образом, нужного 
распределения реальных контрастов на отображениях преобразованных 
ступенчатых тоновых шкал можно добиться, если модифицировать образец, 
ориентируясь на соотношение графиков функций образца и Lab-конт\-рас\-тов 
и распределение контрастов в отображениях преобразованных тоновых шкал.
  
  Из-за дискретности RGB-ко\-ор\-ди\-нат функции Lab-конт\-рас\-тов лишь 
приближенно соответствуют образцам при использовании любой функции 
  Lab-конт\-раст\-но\-го градационного преобразования. Кроме того, 
погрешность вычисления функции Lab-конт\-рас\-тов зависит от значений 
параметров применяемых функций преобразования, от значений координат 
векторов образцов, а также от характеристик и цифрового описания цветового 
пространства в восприятии пользователя. 
  
  Очевидно, что чем точнее вычисляется функция Lab-конт\-рас\-тов по 
отношению к образцу, тем быст\-рее можно получить нужный результат. 
В~связи с этим в рамках данной работы рассматривается задача оптимизации 
функции Lab-конт\-раст\-но\-го градационного преобразования~--- выбора 
такой функции преобразования, при которой распределение Lab-конт\-рас\-тов 
было бы максимально близким к образцу при произвольных образцах и 
пространствах цветовосприятия. 

\section{Обозначения и~определения}
  
  В рамках данной работы рассматривается несколько цифровых описаний 
цветовосприятия виртуальных наблюдателей. Обозначим через~$\varphi_n$,\linebreak 
$n\hm\in 0, 1, \ldots , 5$, RGB-функ\-ции, описывающие цветовосприятие 
некоторых типичных наблюдателей с аномалией цветового зрения в 
соответствии с моделями цветовосприятия из~\cite{10-ar, 11-ar}: $\varphi_0$~--- 
для дейтеранопов~\cite{10-ar}; $\varphi_1$~--- для дейтеранопов~\cite{11-ar}; 
$\varphi_2$~--- для протанопов~\cite{10-ar}; $\varphi_3$~--- для 
протанопов~\cite{11-ar}; $\varphi_4$~--- для тританопов~\cite{10-ar}; 
  $\varphi_5$~--- для тританопов~\cite{11-ar}. 
  
  Пусть $M$~--- множество RGB-пик\-се\-лов, координаты которых кратны 
семнадцати. Тогда
  $$
  \varphi_n(M)\subset M\,,\quad  n\in 0, 1, \ldots , 5\,.
  $$
  
  Обозначим через $\varphi_6$ RGB-функ\-цию, описывающую идеальное 
цветовосприятие при идеальном цветовоспроизведении, при котором
  $$
  \varphi_6(M) = M\,.
  $$
  
  Обозначим через $\varphi_n$, $n\in7, 8, \ldots , 13$, функции 
  RGB-ха\-рак\-те\-ри\-за\-ции цветовосприятия, вычисленные по результатам 
визуального тестирования различения цветных пикселов из множеств 
$\varphi_{n-6}(M)$ в соответствии с~[1--5].
  
  Рассматривается следующая совокупность тоновых шкал:
  $$
  S_i=\left\{ s_{i,j}\right\}\,,\enskip i=0,1, \ldots, 6, \ j=0,1,\ldots , J_i^S\,,
  $$
где 
\begin{align*}
S_0 &= \{s_{0,j}\},\ s_{0,j} = (j, j, j);\\[2pt]
S_1 &= \{s_{1,j}\}\cup \{s_{1,j+15}\},\ s_{1,j} = (j, 0, 0),\\
&\hspace*{40mm}s_{1,j+15} = (255, j, j);\\[2pt]
S_2 &= \{s_{2,j}\}\cup \{s_{2,j+15}\},\ s_{2,j} = (j, j, 0),\\
&\hspace*{40mm} s_{2,j+15} = (255, 255, j);\\[2pt] 
S_3 &= \{s_{3,j}\}\cup \{s_{3,j+15}\},\ s_{3,j} = (0, j, 0),\\
& \hspace*{40mm}s_{3,j+15} = (j, 255, j);\\[2pt]
S_4 &= \{s_{4,j}\}\cup \{s_{4,j+15}\},\ s_{4,j} = (0, j, j),\\
&\hspace*{40mm} s_{4,j+15} = (j, 255, 255);\\[2pt]
S_5 &= \{s_{5,j}\}\cup \{s_{5,j+15}\},\ s_{5,j} = (0, 0, j),\\
& \hspace*{40mm}s_{5,j+15} = (j, j, 255);\\[2pt]
S_6 &= \{s_{6,j}\}\cup \{s_{6,j+15}\},\ s_{6,j} = (j, 0, j),\\
& \hspace*{5mm}s_{6,j+15} = (255, j, 255), j = 0, 1, 
\ldots, 255\,.
\end{align*}
  
  Последовательность $S_i$ будем называть носителем ступенчатой тоновой 
шкалы $T_i\hm=\{t_{i,j}\}$, если $T_i$ является 
подпоследовательностью~$S_i$, состоящей из всех таких пикселов~$S_i$, 
координаты которых кратны семнадцати.
  
  Как и ранее, для описания Lab-конт\-рас\-та RGB-пик\-се\-лов~$x^\prime$ 
и~$x^{\prime\prime}$ будем использовать $E(x^\prime, x^{\prime\prime}$)~--- 
значение расстояния между образами RGB-пик\-селов~$x^\prime$ 
и~$x^{\prime\prime}$ в Lab-про\-стран\-стве.
    \mbox{Исходное} распределе\-ние Lab-конт\-рас\-тов на $\{t_{i,j}\}$ описывается 
последовательностью $\{E(t_{i,j},t_{i,j+1})\}$, на $\{\varphi_n(t_{i,j})\}$~--- 
$\{E(\varphi_n(t_{i,j}), \varphi_n(t_{i,j+1}))\}$.
  
  В качестве образцов $\tau_l(t_{i,j})$ будут использоваться произвольные 
функции, обладающие следующими свойствами:
  \begin{multline*}
\tau_l(t_{i,j})\geq 0\,,\quad \sum\limits_{j=0}^{J_i^T-1} \tau_l(t_{i,j})>0\,,\\
l=0,1,\ldots;\ i=0,1,\ldots, 6; \ j=0,1,\ldots , J_i^T\,.
\end{multline*}

\section{Lab-контрастное градационное преобразование по~образцу}

  Надо найти функцию $F(t_{i,j})$, при которой распределение контрастов на 
преобразованной шкале соответствовало бы образцу~$\tau_l$:
  \begin{multline}
  \fr{E(\varphi_n(F(t_{i,j})), \varphi_n(F(t_{i,j+1})))} 
{\sum\limits_{m=0}^{J_i^T-1}E(\varphi_n(F(t_{i,m})),\varphi_n(F(t_{i,m+1})))} = {}\\
{}=
\fr{\tau_l(t_{i,j})}{\sum\limits_{m=0}^{J_i^T-1}\tau_l(t_{i,m})}\,,\quad j<J_i^T\,.
  \label{e1-ar}
  \end{multline}
  
  Очевидно, что в поле вещественных чисел решение уравнений~(\ref{e1-ar}) 
существует. Но в данном случае из-за дискретности координат 
  RGB-пик\-се\-лов возможно только приближенное вычисление функции~$F$. 
  
  Пусть на носителях тоновых шкал выбраны градации~--- некоторые 
подпоследовательности
  \begin{multline*}
  X_k=\{x_{k,i,j}\}\,, \enskip i\in 0,1,\ldots, 6,\\ 
  j\in 0,1,\ldots ,\ \   J_{k,i}^X\,;\enskip  k\in  0,1,\ldots , K\,,
  \end{multline*}
причем
$$
x_{k,l,0} =s_{i,0}\,,\enskip  x_{k,i,J_i^X}=s_{i,J_i^S}\,.
$$
  
  По заданным градациям найдем последовательности 
  $$
  Y_{k,l,n} =\left\{ y_{k,l,n,i,j}\right\} \subset S_i\,,
  $$
удовлетворяющие соотношениям: 

\noindent
\begin{multline}
\fr{E(\varphi_n(x_{k,i,j}),\varphi_n(x_{k,i,j+1}))}
{\sum\limits_{m=0}^{J^X_{k,i}-1} 
E(\varphi_n(x_{k,i,m}),\varphi_n(x_{k,i,m+1}))}={}\\
{}=
\fr{\tau_l(y_{k,l,n,i,j})} {\sum\limits_{m=0}^{J_{k,i}^X-1} \tau_l 
(y_{k,l,n,i,m})}\,,\quad j<J^X_{k,i}\,.
\label{e2-ar}
\end{multline}
  
  Определим значения функции~$\tau_l$ на всех пикселах носителя~$s_{i,j}$ с 
помощью линейной интерполяции по двум ближайшим к ним пикселам 
ступенчатых тоновых шкал (пусть $t_{i,j^\prime}$ и $t_{i,j^\prime+1}$):
\begin{multline*}
  \tau_l(s_{i,j}) =\fr{1}{17}\left( 17(j^\prime+1)-j\right) \tau_l(t_{i,j^\prime})+{}\\
  {}+
  (j-17j^\prime)\tau_l (t_{i,j^\prime+1})\,,\enskip
  17j^\prime\leq j\leq 17(j^\prime+1)\,.
  \end{multline*}
  
  Заметим, что уравнения из~(\ref{e2-ar}) в общем случае не имеют точного 
решения. Перепишем их в эквивалентном виде
  \begin{multline*}
  \fr{\sum\limits_{m=0}^j E(\varphi_n(x_{k,i,m}),\varphi_n(x_{k,i,m+1}))}
  {\sum\limits_{m=0}^{J^X_{k,i}-1} 
E(\varphi_n(x_{k,i,m}),\varphi_n(x_{k,i,m+1}))} = {}\\
{}=
  \fr{\sum\limits_{m=0}^j \tau_l(y_{k,l,n,i,m})}
  {\sum\limits_{m=0}^{J_{k,i}^X-1} \tau_l(y_{k,l,n,i,m})}\,,\quad j< 
J_{k,i}^X\,,
  \end{multline*}
и на этой основе перейдем к условиям, из которых найдем приближенные 
значения компонентов искомых последовательностей~$Y_{k,l,n}$:
\begin{multline*}
\fr{\sum\limits_{m=0}^{j^\prime(k,l,n,i,j)-1} \tau_l(s_{i,m})}
{\sum\limits_{m=0}^{J^X_{k,i}-1}\tau_l(s_{i,m})}\leq{}\\
{}\leq
\fr{\sum\limits_{m=0}^j E(\varphi_n(x_{k,i,m}),\varphi_n(x_{k,i,m+1}))}
{\sum\limits_{m=0}^{J^X_{k,i}-1} E 
(\varphi_n(x_{k,i,m}),\varphi_n(x_{k,i,m+1}))}\leq{}\\
{}\leq
\fr{\sum\limits_{m=0}^{j^\prime(k,l,n,i,j)}\tau_l(s_{i,m})}
{\sum\limits_{m=0}^{J^X_{k,i}-1}\tau_l(s_{i,m})},\
j<  J^X_{k,i},\  j^\prime(k,l,n,i,j)<J_i^S.
\end{multline*}
  
  Полагая
  \begin{align*}
  y_{k,l,n,i,j}&\approx s_{i,j^\prime(k,l,n,i,j)}\,;\\
  x_{k,i,j} &=f_{k,l,n}\left(s_{i,j^\prime(k,l,n,i,j)}\right)\approx
  f_{k,l,n}(y_{k,l,n,i,j})\,,
  \end{align*}
получим:
{\small\begin{multline*}
\hspace*{-5.79pt}\fr{E(\varphi_n(f_{k,l,n}(y_{k,l,n,i,j})),\varphi_n(f_{k,l,n}(y_{k,l,n,i,j+1})))}
{\sum\limits_{m=0}^{J^X_{k,i}-1} 
\!E(\varphi_n(f_{k,l,n}(y_{k,l,n,i,m})),\varphi_n(f_{k,l,n}(y_{k,l,n,i,m+1})))} 
\approx{}\\
{}\approx
\fr{\tau_l(y_{k,l,n,i,j})}
{\sum\limits_{m=0}^{J^X_{k,i}-1}\tau_l(y_{k,l,n,i,m})}\,,\enskip
j<J^X_{k,i}\,.
%\label{e3-ar}
\end{multline*}}
  
  Теперь можно с помощью интерполяции определить значения функции 
$f_{k,l,n}$ во всех точках носителей, в том числе и в точках ступенчатых тоновых 
шкал
  \begin{multline*}
  \fr{E(\varphi_k(f_{k,l,n}(t_{i,j})),\varphi_k(f_{k,l,n}(t_{i,j+1})))}
  {\sum\limits_{m=0}^{J^X_{k,i}-1} E (\varphi_k(f_{k,l,n}(t_{i,m})),\varphi_k 
(f_{k,l,n}(t_{i,m+1})))}
  \approx{}\\
  {}\approx
  \fr{\tau_l(t_{i,j})}
  {\sum\limits_{m=0}^{J^X_{k,i}-1} \tau_l (y_{k,l,n,i,m})}\,,\quad 
  j<J_i^T\,.
%  \label{e4-ar}
  \end{multline*}
  
\section{Погрешность вычисления функции~$F$}

  Обозначим через $\varepsilon_{k,l,n}$ погрешность вычисления функции~$F$ в 
случае, когда функция $f_{k,l,n}$ используется в качестве ее приближенного 
значения. В~качестве значения $\varepsilon_{k,l,n}$ используем меру бли\-зости 
между векторами на основе косинуса:
  $$
  \varepsilon_{k,l,n} =\rho\left(\vec{a}_{k,l,n},\vec{b}_l\right) =1-\cos \left( 
\vec{a}_{k,l,n},\vec{b}_l\right)\,,
  $$
где
\begin{multline*}
\vec{a}_{k,l,n}= \{a_{k,l,n,i,j}\}\,;\\ 
a_{k,l,n,i,j} =  E(\varphi_n(f_{k,l,n}(t_{i,j})),\ 
\varphi_n(f_{k,l,n}(t_{i,j+1})))\,;
\end{multline*}

\vspace*{-12pt}

\noindent
\begin{multline*}
\vec{b}_l = \{b_{l,i,j}\}\,;\enskip
b_{l,i,j} = \tau_l(t_{i,j})\,,\\ i = 0, 1, \ldots , 6,\ j = 0, 1, \ldots ,  J^T_i-1\,;
\end{multline*}
$$
\cos\left( \vec{a}_{k,l,n},\vec{b}_l\right) = 
\fr{\sum\limits_{i=0}^6 \sum\limits_{j=0}^{J_i^T-1} a_{k,l,n,i,j} b_{l,i,j}}
{\sum\limits_{i=0}^6 \sum\limits_{j=0}^{J^T_i-1} a^2_{k,l,n,i,j} 
\sum\limits_{i=0}^6 \sum\limits_{j=0}^{J_i^T-1} b^2_{l,i,j}}\,.
$$


  
  Авторами была написана специальная программа, с помощью которой были 
вычислены значения $\varepsilon_{k,l,n}$ при~$X_k$, $k \hm= 1, 2, \ldots, 36$, 
$\tau_l$, $l \hm= 0, 1$, и~$\varphi_n$, $n \hm= 0, 1, \ldots , 13$.
  
  При выборе градаций носителей ступенчатых тоновых шкал~$X_k$, $k \hm= 
1, 2, \ldots , 36$, было использовано два способа. При $k \hm\leq 18$ в 
качестве~$X_k$ выбирались подпоследовательности пикселов носителей 
ступенчатых тоновых шкал с шагом по индексу, равным~$k$, а при $19 \hm\leq 
k \hm\leq 36$~--- с шагом по контрасту, равным числу $1\hm+0{,}5 (k \hm - 
18)$. 
  
  При $l = 0$ значения функции $\tau_{l}(t_{i,j})$ определялись величиной 
расстояния между компонентами ступенчатых тоновых шкал, а при $l \hm= 
1$~---  величиной Lab-конт\-рас\-та между ними.
  
\section{Выбор функции Lab-контрастного градационного 
преобразования}

  Обозначим через $\varepsilon_{37,l,n}$ погрешности вида
  $$
  \varepsilon_{37,l,n} =\min\limits_{1\leq k\leq 36} \varepsilon_{k,l,n}\,,
  $$
а через $\vec{V}_k$~--- векторы вида
$$
\vec{V}_k=\left\{ v_{k,l+2n}\right\} =\left\{\varepsilon_{k,l,n}\right\}\,,\enskip 
0\leq l+2n\leq 27\,.
$$


  
  Необходимо определить, при каком значении~$j^\prime$ выполнено 
соотношение 
  $$
  \rho\left( \vec{V}_{37},\vec{V}_{j^\prime}\right) =\min\limits_{1\leq j\leq 36} 
\rho \left( \vec{V}_{37},\vec{V}_j\right)\,.
  $$
  
  Обозначим через $\vec{P}_i\hm=\left\{ p_{i,j}\right\}$ векторы, со\-став\-лен\-ные 
из расстояний между векторами~$\vec{V}_i$ и~$\vec{V}_j$: 
  $$
  p_{i,j} =\rho \left( \vec{V}_i,\vec{V}_j\right)\,,\enskip i=1,2,\ldots , 37\,,\ 
j=1,2,\ldots , 36\,.
  $$
  
  Было установлено, что наименьшей координатой вектора~$\vec{P}_{37}$ 
является координата с индексом, равным~24. Следовательно, 
наиболее близким к вектору~$\vec{V}_{37}$  является вектор~$\vec{V}_{24}$ 
(рис.~1). Это означает оптимальность функции $f_{24,l,k}$ и соответствующего 
метода преобразования ступенчатых тоновых шкал. При определении $f_{24,l,k}$ 
градации носителей ступенчатых тоновых шкал выбирались с шагом по 
контрасту, равным четырем. 
  
  Среди функций, при определении которых градации выбирались с шагом по 
индексу, наилучшей оказалась функция $f_{15,l,k}$, поскольку наиболее близким 
к вектору~$\vec{V}_{37}$ среди соответствующего множества векторов 
является вектор~$\vec{V}_{15}$. 

\begin{center}  %fig1
\vspace*{-3pt}
\mbox{%
 \epsfxsize=72.254mm
 \epsfbox{arh-1.eps}
 }
  \end{center}
  
    \vspace*{-3pt}

\noindent
{{\figurename~1}\ \ \small{Соотношение координат векторов: \textit{1}~--- $\vec{V}_1$; \textit{2}~--- 
$\vec{V}_{15}$; \textit{3}~--- $\vec{V}_{24}$; \textit{4}~--- $\vec{V}_{37}$}}


%\pagebreak

\vspace*{12pt}

\begin{center}  %fig2
\mbox{%
 \epsfxsize=74.31mm
 \epsfbox{arh-2.eps}
 }
  \end{center}

  \vspace*{-3pt}

\noindent
{{\figurename~2}\ \ \small{Соотношение координат векторов: \textit{1}~--- $\vec{P}_1$; \textit{2}~--- 
$\vec{P}_{15}$; \textit{3}~--- $\vec{P}_{24}$; \textit{4}~--- $\vec{P}_{37}$}}

\vspace*{12pt}

\addtocounter{figure}{2}

  
  Наиболее далеким от вектора $\vec{V}_{37}$ является вектор~$\vec{V}_1$; 
следовательно, $f_{1,l,k}$ является наихудшей среди рассмотренных функций. 
  


  
  Близость векторов $\vec{V}_{15}$, $\vec{V}_{24}$, $\vec{V}_{37}$ и их 
отличие от вектора~$\vec{V}_1$ проявляется и в близости векторов 
$\vec{P}_{15}$, $\vec{P}_{24}$, $\vec{P}_{37}$ и их различии с 
вектором~$\vec{P}_1$ (рис.~2). 


  
\section{Сравнение результатов применения функций~$f_{1,l,k}$, 
$f_{15,l,k}$, $f_{24,l,k}$ на примере }
  
  Серая шкала $T_0$, исходно имеющая вид, отраженный на рис.~3,\,\textit{а}, 
в восприятии пользователя, цветовосприятие которого описывается 
функцией~$\varphi_{11}$ ($k\hm=11$), приобретает зеленоватый оттенок и 
выглядит как на рис.~3,\,\textit{б}. Можно видеть, что в области теней теряется 
контрастность пикселов.

\setcounter{figure}{5}
\begin{figure*}[b] %fig6
\vspace*{9pt}
 \begin{center}
 \mbox{%
 \epsfxsize=140.833mm
 \epsfbox{arh-6.eps}
 }
 \end{center}
 \vspace*{-6pt}
\Caption{Изображения: (\textit{а})~Img; (\textit{б})$~\varphi_{11}(\mathrm{Img})$
}
\end{figure*}
  
  После равноконтрастного градационного преобразования (рис.~3,\,\textit{в}\,--\,3,\,\textit{д}) 
  исходный контраст на\linebreak\vspace*{-12pt}



\noindent
\begin{center}  %fig3
\vspace*{-18pt}
\mbox{%
 \epsfxsize=68.608mm
 \epsfbox{arh-3.eps}
 }
  \end{center}

  \vspace*{-3pt}

\noindent
{{\figurename~3}\ \ \small{Вид пикселов шкал: (\textit{а})~$T_0$; (\textit{б})~$\varphi_{11}(T_0)$; 
(\textit{в})~$f_{1,0,11}(T_0)$; (\textit{г})~$f_{15,0,11}(T_0)$; (\textit{д})~$f_{24,0,11}(T_0)$; 
(\textit{е})~$f_{24,2,11}(T_0)$}}

\vspace*{18pt}


\noindent 
 серой шкале практически 
восстанавливается. При
 этом после преобразований с помощью функций $f_{1,0,11}$, $f_{15,0,11}$ и $f_{24,0,11}$ (рис.~4) 
наиболее точное соответствие распределения контрастов образцу 
соответствует функции $f_{24,0,11}$.
  
  Выбор образца $\tau_2$ (рис.~5) позволяет значительно увеличить 
контраст между первыми двумя пикселами ступенчатой тоновой шкалы 
(см.\ рис.~3,\,\textit{е}).

  



  Рассмотрим изображение Img (рис.~6,\,\textit{а}). В~восприятии 
пользователя, цветовосприятие которого описывается функцией~$\varphi_{11}$, 
Img принимает вид, показанный на рис.~6,\,\textit{б}. Как и у 
отображения серой шкалы, у отображения Img детализация искажается, 
поскольку уменьшается контрастность пикселов в области теней. 

  Предварительное Lab-конт\-раст\-ное градационное преобразование с 
помощью функции $f_{24,0,11}$ восстанавливает детализацию изображения в 
восприятии рассматриваемого пользователя (рис.~7,\,\textit{а}). 

%\linebreak\vspace*{-12pt} 
\begin{center}  %fig4
\mbox{%
 \epsfxsize=43.389mm
 \epsfbox{arh-4.eps}
 }
  \end{center}

  \vspace*{-9pt}

\noindent
{{\figurename~4}\ \ \small{Соотношение образцов~$\tau_0$~(\textit{1}) 
и контрастов пикселов преобразованных шкал~(\textit{2}): 
(\textit{а})~$f_{1,0,11}(T_0)$; (\textit{б})~$f_{15,0,11}(T_0)$;
(\textit{в})~$f_{24,0,11}(T_0)$}}

\vspace*{3pt}

%\addtocounter{figure}{2}




\begin{center}  %fig5
\vspace*{-2pt}
\mbox{%
 \epsfxsize=43.389mm
 \epsfbox{arh-5.eps}
 }
  \end{center}

  \vspace*{-3pt}

\noindent
{{\figurename~5}\ \ \small{Соотношение образцов~$\tau_2$~(\textit{1}) и контрастов пикселов преобразованных 
шкал~(\textit{2}) $f_{24,2,11}(T_0)$}}

\vspace*{12pt}

\addtocounter{figure}{1}

  
  

%\noindent

  
  Предварительное Lab-конт\-раст\-ное градационное преобразование с 
помощью функции $f_{24,2,11}$ улучшает детализацию изображения в 
восприятии рассматриваемого пользователя (рис.~7,\,\textit{б}), поскольку 
позволяет выявить новые детали в области теней.


\setcounter{figure}{6}
\begin{figure*} %fig7
\vspace*{1pt}
 \begin{center}
 \mbox{%
 \epsfxsize=140.864mm
 \epsfbox{arh-7.eps}
 }
 \end{center}
 \vspace*{-6pt}
\Caption{Изображения: (\textit{а})~$\varphi_{11}(f_{24,0,11}(\mathrm{Img}))$; 
(\textit{б})~$\varphi_{11}(f_{24,2,11}(\mathrm{Img}))$
}
\end{figure*}
  
\section{Заключение}
  
  Рассмотрена задача оптимизации функции Lab-конт\-раст\-но\-го 
градационного преобразования ступенчатых тоновых шкал. 
  
  Задача решена путем сравнения погрешности вычисления функции 
распределения Lab-конт\-рас\-тов на ступенчатых тоновых шкалах на 
типичных примерах. 
  
  Установлено, что наименьшую погрешность обеспечивает функция 
  Lab-конт\-раст\-но\-го градационного преобразования ступенчатых тоновых 
шкал, основанная на выборе градаций носителей ступенчатых тоновых шкал с 
шагом по контрасту, равным четырем. Эта функция рекомендуется для 
применения на практике. 
  
  Приведен пример, демонстрирующий эффективность применения функции 
Lab-конт\-раст\-но\-го градационного преобразования ступенчатых тоновых 
шкал. Показано, что при выборе подходящего образца можно не только 
избежать искажения детализации, но и добиться ее улучшения.
  
{\small\frenchspacing
{%\baselineskip=10.8pt
\addcontentsline{toc}{section}{Литература}
\begin{thebibliography}{99}
\bibitem{1-ar}
\Au{Архипов О.\,П., Зыкова З.\,П.} Допечатное тестирование индивидуального 
зрительного восприятия~// Вестник компьютерных и информационных 
технологий, 2008. №\,12. С.~2--8.
\bibitem{2-ar}
\Au{Архипов О.\,П., Зыкова З.\,П.} Интеграция гетерогенной информации о 
цветных пикселях и их цветовосприятии~// Информатика и её применения, 
2010. Т.~4. Вып.~4. С.~14--25.
\bibitem{3-ar}
\Au{Архипов О.\,П., Зыкова З.\,П.} Функциональное описание индивидуального 
цветовосприятия~// Информационные сис\-те\-мы и технологии, 2010. №\,5. 
С.~5--12.
\bibitem{4-ar}
\Au{Архипов О.\,П., Зыкова З.\,П.} RGB-ха\-рак\-те\-ри\-за\-ция пространства 
цветовосприятия~// Сис\-те\-мы и сред\-ст\-ва информатики, 
2010. Вып.~20. №\,1. С.~73--90.
\bibitem{5-ar}
\Au{Архипов О.\,П., Зыкова З.\,П.} Многокритериальный выбор тестового 
множества при исследовании цветовосприятия~// Информационные 
технологии, 2011. №\,2. С.~67--73.
\bibitem{6-ar}
\Au{Архипов О.\,П., Зыкова З.\,П.} Равноконтрастные градационные 
преобразования ступенчатых тоновых шкал~// Информационные сис\-те\-мы и 
технологии, 2011. №\,4. С.~39--46. 
\bibitem{7-ar}
\Au{Архипов О.\,П., Зыкова З.\,П.} Персонифицированное преобразование 
пред\-став\-ле\-ний цвет\-ных изоб\-ра\-же\-ний на мониторе ПЭВМ~// 
Сис\-те\-мы и средства информатики, 2012. Т.~22. №\,1. С.~22--37.
\bibitem{8-ar}
\Au{Архипов О.\,П., Зыкова З.\,П.} Метод улучшения детализации цветных 
изображений~// Вестник компьютерных и информационных технологий, 2013. 
№\,2. С.~6--11.
\bibitem{9-ar}
\Au{Архипов О.\,П., Зыкова З.\,П.} Коррекция детализации представлений 
RGB-изоб\-ра\-же\-ний на периферийных устройствах ПЭВМ~// 
Информационные технологии, 2013. №\,2. С.~56--60.
\bibitem{10-ar}
Color Oracle/Institute of Cartography.~--- Zurich: ETH, 2008. {\sf 
http://www.colororacle.org}.
\bibitem{11-ar}
Vischeck. {\sf http://www.vischeck.com}. 
\end{thebibliography} } }



\end{multicols}

\hfill{\small\textit{Поступила в редакцию 25.03.13}}


%\vspace*{12pt}

%\hrule

%\vspace*{2pt}

%\hrule

\newpage

\vspace*{-24pt}

\def\tit{FUNCTIONS OPTIMIZATION OF LAB-CONTRAST GRADED TRANSFORMATION}

\def\titkol{Functions optimization of Lab-contrast graded transformation}

\def\aut{O.\,P.~Arkhipov and Z.\,P.~Zykova}
\def\autkol{O.\,P.~Arkhipov and Z.\,P.~Zykova}


\titel{\tit}{\aut}{\autkol}{\titkol}

\vspace*{-15pt}

\noindent
Oryol Branch, Institute of Informatics 
Problems, Russian Academy of Sciences, Oryol 302025, Russian Federation


 
\def\leftfootline{\small{\textbf{\thepage}
\hfill INFORMATIKA I EE PRIMENENIYA~--- INFORMATICS AND APPLICATIONS\ \ \ 2013\ \ \ volume~7\ \ \ issue\ 4}
}%
 \def\rightfootline{\small{INFORMATIKA I EE PRIMENENIYA~--- INFORMATICS AND APPLICATIONS\ \ \ 2013\ \ \ volume~7\ \ \ issue\ 4
\hfill \textbf{\thepage}}}   

\vspace*{6pt}
  
\Abste{The purpose of this study is to create algorithm for custom conversion of 
contrast distribution on stepped-tone scales. The idea is to find suitable Lab-contrast 
distribution, keeping in mind that scaled distortion of digital representation of pixels 
would keep the real perception of color contrast all the same, if only less bright. In 
order to approximately calculate the function of Lab-contrast gradual conversion, this 
study considers two families of parametric algorithms, the first one uses subset of pixels 
on the scale as parameters, while the another one uses many gradations as they are. The problem 
of selecting the set of optimal parameters is solved by comparison of range of 
calculation errors achieved on a set of typical examples by the function of Lab-contrast 
distribution on stepped-tone scales. The study shows the function of Lab-contrast 
distribution on stepped-tone scales that gives the least error and provides example 
that proves that with optimal set of parameters chosen, it is not only possible to avoid 
visual distortion but the details can also be improved. These findings may be useful in 
management of maps of RGB-color images for PC peripherals in order to improve 
perceptive quality of details.}

\KWE{color reproduction; color perception; Lab-coordinate; contrast; gradation}
  
  
\DOI{10.14357/19922264130405}

%\Ack
%\noindent
%?????

  \begin{multicols}{2}

\renewcommand{\bibname}{\protect\rmfamily References}
%\renewcommand{\bibname}{\large\protect\rm References}

{\small\frenchspacing
{%\baselineskip=10.8pt
\addcontentsline{toc}{section}{References}
\begin{thebibliography}{99}

  \bibitem{1-ar-1}
  \Aue{Arkhipov, O.\,P., and Z.\,P.~Zykova}. 2008. Dopechatnoe testirovanie 
individual'nogo zritel'nogo vospriyatiya [Preprinting test of individual visual 
perception]. \textit{Vestnik Komp'yuternykh i
Informatsionnykh Tekhnologiy~---
Herald of Computer and Information Technologies}  12:2--8.
  \bibitem{2-ar-1}
  \Aue{Arkhipov, O.\,P., and Z.\,P.~Zykova}. 2010. Integratsiya geterogennoy
informatsii o tsvetnykh pikselyakh i ikh tsvetovospriyatii [Integration of heterogeneous 
information about color pixels and their color perception]. \textit{Informatika i ee Primeneniya~---
Inform. Appl.} 4(4):14--25.
  \bibitem{3-ar-1}
  \Aue{Arkhipov, O.\,P., and Z.\,P.~Zykova}. 2010. Funktsional'noe opisanie 
individual'nogo tsvetovospriyatiya [Characteristics of color perceptual space]. 
\textit{Informatsionnye Sistemy i Tekhnologii} [\textit{Information Systems and 
Technologies}] 5:5--12.
  \bibitem{4-ar-1}
  \Aue{Arkhipov, O.\,P., and Z.\,P.~Zykova}. 2010. RGB-kharakterizatsiya 
prostranstva tsvetovospriyatiya [RGB-characterization of color perception 
space]. \textit{Sistemy i Sredstva Informatiki~---
Systems and Means of Informatics} 20(1):73--90.
  \bibitem{5-ar-1}
  \Aue{Arkhipov, O.\,P., and Z.\,P.~Zykova}. 2011. Mnogokriterial'nyy vybor 
testovogo mnozhestva pri issledovanii tsvetovospriyatiya [Multicriterion choice of 
test set when studying the color perception]. \textit{Inform. Tekhnologii} 
[\textit{Information Technologies}] 2:67--73.
  \bibitem{6-ar-1}
  \Aue{Arkhipov, O.\,P., and Z.\,P.~Zykova}. 2011. Ravnokontrastnye gradatsionnye 
preobrazovaniya stupenchatykh tonovykh shkal [Equal contrast graded transformation 
of step tinted scales]. \textit{Informatsionnye Sistemy i Tekhnologii} 
[\textit{Information Systems and Technologies}] 4:39--46.
  \bibitem{7-ar-1}
  \Aue{Arkhipov, O.\,P., and Z.\,P.~Zykova}. 2012. Personifitsirovannoe 
preobrazovanie predstavleniy tsvetnykh izobrazheniy  na monitore PEVM [Personified 
transformation of color images presentations on a PC monitor]. \textit{Sistemy i
Sredstva Informatiki~--- Systems and 
Means of Informatics} 22(1):22--37.
  \bibitem{8-ar-1}
  \Aue{Arkhipov, O.\,P., and Z.\,P.~Zykova}. 2013. Metod uluch\-she\-niya detalizatsii 
tsvetnykh izobrazheniy [Method of improving the detail of color images]. 
\textit{Vestnik Komp'yuternykh i Informatsionnykh Tekhnologiy~--- Herald of Computer 
and Information Technologies} 2:6--11.
  \bibitem{9-ar-1}
  \Aue{Arkhipov, O.\,P., and Z.\,P.~Zykova}. 2013. Korrektsiya de\-ta\-li\-za\-tsii 
predstavleniy RGB-izobrazheniy na periferiynykh ustroystvakh PEVM [Correcting of 
detail presentations of RGB-images on peripherals of PC]. \textit{Informatsionnye 
Tekhnologii} [\textit{Information Technologies}] 2:56--60.
  \bibitem{10-ar-1}
ETH Zurich.  2008. {Color Oracle/Institute of Cartography}.   Available at: 
{\sf http://www.colororacle.org} (accessed November~5, 2013).


 
\bibitem{11-ar-1}
Vischeck. 2008. Available at: {\sf http://www.vischeck.com}
(accessed November~3, 2013).
\end{thebibliography}
} }


\end{multicols}

\vspace*{-6pt}

\hfill{\small\textit{Received March 25, 2013}}

\vspace*{-18pt}

\Contr

\noindent
\textbf{Arkhipov Oleg P.} (b.\ 1948)~--- Candidate of Science (PhD) in technology, 
Director, Oryol Branch of the Institute of Informatics Problems, Russian Academy 
of Sciences, Oryol 302025, Russian Federation; arkhipov12@yandex.ru 


\vspace*{2pt}

\noindent
\textbf{Zykova Zoya P.} (b.\ 1953)~--- Candidate of Science (PhD) in physics and 
mathematics, Head of Laboratory, Oryol Branch of the Institute of Informatics 
Problems, Russian Academy of Sciences, Oryol 302025, Russian Federation; zykzoya@yandex.ru

 \label{end\stat}
 
\renewcommand{\bibname}{\protect\rm Литература}  
  
   %5
\def\stat{zatsman}

\def\tit{ТРАНСФОРМАЦИИ ОБЪЕКТОВ ПЕРВОГО И~ВТОРОГО ПОРЯДКА 
В~ЛЕКСИКОГРАФИЧЕСКОЙ ИНФОРМАЦИОННОЙ СИСТЕМЕ$^*$}

\def\titkol{Трансформации объектов первого и~второго порядка 
в~лексикографической информационной системе}

\def\aut{И.\,М.~Зацман$^1$}

\def\autkol{И.\,М.~Зацман}

\titel{\tit}{\aut}{\autkol}{\titkol}

\index{Зацман И.\,М.}
\index{Zatsman I.\,M.}


{\renewcommand{\thefootnote}{\fnsymbol{footnote}} \footnotetext[1]
{Исследование выполнено в~ФИЦ ИУ РАН за счет гранта Российского научного фонда №\,24-18-00155, {\sf 
https://rscf.ru/project/24-18-00155}. Работа выполнялась с~использованием инфраструктуры Центра 
коллективного пользования <<Высокопроизводительные вычисления и~большие данные>> (ЦКП 
<<Информатика>>) ФИЦ ИУ РАН (г.\ Москва).}}


\renewcommand{\thefootnote}{\arabic{footnote}}
\footnotetext[1]{ Федеральный исследовательский центр <<Информатика и~управление>> Российской академии наук, 
\mbox{izatsman@yandex.ru}}

\vspace*{-12pt}


  
  \Abst{Рассматриваются теоретические основания проектирования информационных 
технологий (ИТ) интеграции двуязычных словарей и~параллельных корпусов. Дано описание 
первых результатов создания третьего уровня классификации трансформаций объектов 
предметной области информатики, которую предполагается использовать при создании 
концепции лексикографической информационной системы, обеспечивающей интеграцию. 
Все сущности информатики в~статье разделены на два глобальных класса: объекты и~их 
трансформации. Для каждого такого класса конструируется своя классификация. Ранее были 
описаны два верхних уровня классификации трансформаций объектов предметной области. 
В~данной статье рассматривается третий уровень этой классификации. Основанием для 
построения самого верхнего ее уровня служило деление предметной области информатики 
на среды (ментальная, сенсорно воспринимаемая, цифровая и~ряд других сред), каждая из 
которых по определению включает объекты одной природы. Основанием для построения 
второго уровня классификации трансформаций объектов служила типология знаковых  
сис\-тем А.~Соломоника. Цель статьи состоит в~систематизации трансформаций первого 
и~второго порядка объектов предметной области на третьем уровне этой классификации. 
Основанием для систематизации служит средовая версия иерархии Акоффа.}
  
  \KW{объекты предметной области; трансформации объектов; классификация; данные; 
информация; знание; лексикографическая информационная сис\-тема}

\DOI{10.14357/19922264240211}{VZTGVV}
  
\vspace*{3pt}


\vskip 10pt plus 9pt minus 6pt

\thispagestyle{headings}

\begin{multicols}{2}

\label{st\stat}
  
\section{Введение}

\vspace*{-9pt}

  Возникновение параллельных корпусов, в~которых предложениям 
оригинального текста со\-по\-став\-ле\-ны предложения его перевода, обеспечило 
возможность контрастивного лингвистического\linebreak \mbox{анализа} на принципиально 
новом уровне полноты и~точности, недостижимом в~докорпусную эпоху. 
Пионерскими в~этой области стали работы \mbox{1990-х~гг}. Стига Йоханссона  
с~анг\-ло-нор\-веж\-ским корпусом~[1]. В России параллельные корпусы стали 
формироваться в~начале XXI~века в~рамках Национального корпуса русского 
языка~[2].
  
  Создатели двуязычных словарей используют параллельные корпусы для 
сбора материала и~эмпирической проверки своих гипотез, касающихся 
межъязы\-ко\-вой эквивалентности. Ценность параллельных корпусов 
определяется тем, что в~лингвистике этап сбора исходного материала считается 
наиболее трудоемким и~наименее творческим, а~параллельные корпусы 
позволяют значительно сэкономить время и~силы для творческого этапа 
создания словарей~[3].
 % 
  При этом двуязычные словари, создаваемые на основе исходного материала, 
извлеченного из параллельных корпусов, сейчас формируются без связей с~их 
текстами. Другими словами, онлайновые связи созданных словарей 
с~параллельными корпусами, которые служили источниками исходного 
материала, отсутствуют. 

Параллельные корпусы постоянно пополняются 
новыми текстами, в~предложениях которых можно обнаружить новые значения 
слов и~устойчивых словосочетаний. Однако при этом отсутствуют методы 
и~средства оперативного обновления словарей по корпусным данным. 
В~настоящее время проблема установления связей между двуязычными 
словарями и~параллельными корпусами (далее~--- проблема интеграции) 
находится на стадии поиска концептуальных подходов к~их интеграции на 
уровне значений.
  
  Подход к~решению проблемы интеграции, предлагаемый в~статье, учитывает 
  и~появление новых значений слов и~устойчивых словосочетаний, и~динамику 
смысловых значений, которая обусловлена развитием и~пополнением знания 
лингвистов, фиксирующих эти значения в~результате семантического анализа 
пополняемых корпусных данных. Проведенные эксперименты показали, что 
обнаружение нового лингвистического знания обусловливает и~формирование 
дефиниций новых значений, и~пересмотр уже существующих дефиниций~[4, 5].
  
  Например, в~проведенных экспериментах с~использованием ЦКП 
<<Информатика>> ФИЦ ИУ РАН фиксировалась эволюция значений немецких 
модальных глаголов, исходное состояние значений которых было описано 
в~не\-мец\-ко-рус\-ском словаре. В~экспериментальном массиве текстов как 
потенциальных источниках нового знания 16\,268 предложений содержали 
немецкие модальные глаголы и~в~2041 из них встречался глагол sollen. 
В~начале эксперимента в~словаре были описаны~12~значений этого модального 
глагола. По окончании эксперимента лингвисты обнаружили два новых его 
значения, согласовали их дефиниции и~описали эволюцию дефиниций~[6, 7].
  
  Таким образом, для решения проблемы интеграции требуется фиксировать 
новое знание, обнаруженное лингвистами в~текстовых данных параллельных 
корпусов, отслеживать эволюцию знания, представленного в~виде дефиниций 
значений слов и~устойчивых словосочетаний, и,~соответственно, 
актуализировать электронные двуязычные словари. Предлагаемый 
концептуальный подход к~интеграции, который планируется реализовать 
в~процессе проектирования лексикографической информационной сис\-те\-мы, 
фиксирующей эволюцию лингвистического знания, основан на решении 
следующих задач:\\[-14pt]
  \begin{itemize}
  \item категоризация трех базовых понятий информатики, включенных 
  в~иерархию Акоффа~[8] (данные, информация, знание), на объекты 
проектируемой сис\-те\-мы, которая необходима, чтобы фиксировать 
<<кванты>> нового знания и~отслеживать его эволюцию в~этой сис\-теме;\\[-15pt]
  \item  систематизация трансформаций объектов этой сис\-темы.\\[-14pt]
  \end{itemize}
  
  Цель статьи и~состоит в~решении двух задач: категоризации трех базовых 
понятий информатики на объекты лексикографической информационной  
сис\-те\-мы и~сис\-те\-ма\-ти\-за\-ции трансформаций первого и~второго порядка 
ее объектов.
  
  Трансформациями первого порядка, о которых сказано в~формулировке цели 
статьи, называются взаимные преобразования между двумя объектами  
сис\-те\-мы одной природы. Например, перевод в~сис\-те\-ме текста с~русского 
языка на английский относится к~ним. Трансформациями второго порядка 
и~выше называются взаимные преобразования между двумя и~более объектами 
разной природы. Например, кодирование символов текс\-та компьютерными 
кодами и~их декодирование относятся по определению к~трансформациям 
второго порядка.

%\vspace*{-9pt}
  
\section{Процессы трансформаций в~информатике}

%\vspace*{-3pt}

Процессы трансформаций, рассматриваемые в~статье, относятся к~теоретическому ядру информатики, а~не 
только к~проектированию лексикографической информационной сис\-те\-мы. Например, из трех основных 
подходов к~описанию предметной об\-ласти информатики\footnote{В статье предметная область информатики 
трактуется согласно концепции полиадического компьютинга Пола Розенблума~\cite{9-zac}.} (объектный, 
трансформационный и~синтетический) сис\-те\-ма\-ти\-за\-ция трансформаций ближе всего ко второму 
подходу. Примерами первого подхода, в~рамках которого основное внимание уделяется объектам предметной 
области информатики и~в~меньшей степени отношениям\linebreak между ними, могут служить  
работы~\cite{8-zac, 10-zac, 11-zac}; \mbox{примерами} второго подхода, в~рамках которого основное внимание 
уделяется трансформациям и~в~меньшей степени трансформируемым объектам,~---  
работы~\cite{12-zac, 13-zac}; примерами третьего, синтетического подхода, в~котором уделяется внимание 
и~объектам предметной об\-ласти информатики, и~отношениям между ними, могут служить работы~\cite{14-zac, 
15-zac, 16-zac, 17-zac, 18-zac}.

  Таким образом, для описания трансформаций объектов лексикографической 
информационной\linebreak системы предпочтительнее всего трансформационный 
подход, который упоминается и~в определениях информатики. Например, 
в~2009~г.\ П.~Деннинг и~П.~Розенблум сформулировали суть \mbox{информатики} как 
компьютинга следующим образом: <<$\ldots$информатика~--- это не просто 
алгоритмы и~структуры данных; это преобразования [трансформации] 
представлений>>~\cite{12-zac}. Чуть позже, в~контексте краткого описания 
парадигмы информатики как компьютинга, П.~Деннинг и~П.~Фриман изменили 
эту формулировку на такую: <<Центральный объект внимания в~информатике 
можно определить как информационные процессы~--- \textit{естественные или 
искусственные процессы, преобразующие информацию} (курсив мой~--- 
И.\,З.)>>~\cite{13-zac}. Согласно парадигме, предлагаемой авторами этой 
статьи, на начальном этапе проектирования автоматизированных систем 
базовыми элементами моделей их функционирования служат 
\textit{информационные про\-цессы}.
  
  Однако если 15~лет назад в~формулировке из работы~\cite{13-zac} шла речь 
о~процессах, преобразующих информацию, то в~последние~10~лет в~спектр 
процессов трансформаций все чаще стали включать процессы, преобразующие 
не только информацию, но также и~другие объекты автоматизированных 
систем, в~первую очередь данные и~знания~[19--21]. Например, Виктория 
Стодден, позиционируя науку о~данных как одну из дисциплин информатики, 
говорит, что центральный объект исследований в~науке о~данных~--- это 
<<изучение обобщаемого извлечения знания из данных>>~\cite{21-zac}. 
Увеличение и~чис\-ла объектов, и~спект\-ра процессов их трансформаций 
в~автоматизированных сис\-те\-мах обуслов\-ли\-ва\-ет не\-об\-хо\-ди\-мость 
систематизации и~объектов, и~процессов их трансформаций на начальном этапе 
проектирования сис\-тем.
  
  Для создания концепции лексикографической информационной сис\-те\-мы 
и~проектирования ИТ, обеспечивающих интеграцию 
двуязычных словарей и~параллельных корпусов, сначала выполним 
категоризацию на объекты этой сис\-те\-мы трех базовых понятий информатики 
(данные, информация, знание) в~контексте построения классификаций 
сущностей ее предметной об\-ласти.
  
  Необходимость использования классификаций информатики в~процессе 
создания концепции проиллюстрируем, используя иерархию  
Акоффа~\cite{8-zac}. Он использовал принцип их вертикального размещения 
в~иерархии снизу вверх: данные, информация и~знание. Еще в~ней есть термин 
<<мудрость>>, который в~статье не рассматривается. Такое размещение Акофф 
прокомментировал так: <<Каждое из пе\-ре\-чис\-лен\-ных понятий [кроме данных] 
содержит в~себе нижестоящие$\ldots$>>~\cite{8-zac}.
  
  Этому принципу размещения и~комментарию Акоффа свойственны 
недостатки, проанализированные, в~частности, в~работе~\cite{10-zac}. Главный 
вывод, к~которому пришла Роули после изучения иерархии Акоффа, 
заключается в~следующем: <<$\ldots$информация определяется в~терминах 
данных, знание~--- в~терминах информации$\ldots$ но существует меньше 
консенсуса в~описании трансформаций, которые преобразуют сущности, 
расположенные ниже в~иерархии, в~те, которые находятся над ними, что 
приводит к~их терминологической неопределенности>>~\cite{10-zac}. Причина 
этой неопределенности, скорее всего, в~том, что базовые понятия информатики 
включены в~иерархию Акоффа изолированно от общего контекста 
классификаций сущностей ее предметной об\-ласти.

%\vspace*{-9pt}
  
\section{Классификации сущностей информатики}


%\vspace*{-2pt}

  Все сущности предметной области информатики в~работах~[22, 23] 
разделены на два глобальных класса: ее объекты и~их трансформации. Для 
каждого такого класса была предложена своя классификация. 
В~работе~\cite{22-zac} дано описание классификации объектов предметной 
области информатики, первый уровень которой содержит базовые понятия ее 
предметной области (данные, информация, знания и~др.).  
В~работе~\cite{23-zac} дано описание двух верхних уровней классификации 
трансформаций объектов предметной об\-ласти (см.\ рисунок 
в~работе~\cite{23-zac}). Основанием для построения самого верхнего ее уровня послужило деление 
предметной области информатики на среды\footnote{В~работе~\cite{24-zac} дано описание пяти сред 
предметной области информатики (ментальная; сенсорно воспринимаемая, или информационная; 
цифровая; нейро- и~ДНК-среда), каждая из которых по определению включает объекты одной и~той же 
природы.} и~степень разнообразия природы объектов, вовлеченных в~трансформации:
\begin{itemize}
\item  первый класс верхнего уровня классификации включает 
трансформации объектов в~пределах среды только одной природы 
(трансформации первого порядка);
\item  второй класс включает трансформации объектов, относящихся 
к~двум средам разной природы (трансформации второго порядка);
\item третий и~последующие классы включают трансформации объектов, 
относящихся к~трем и~более средам разной природы (трансформации 
третьего и~более высоких порядков).
\end{itemize}

  В работе~\cite{23-zac} были приведены примеры для трех первых классов 
трансформаций, включая пример трансформаций объектов, относящихся 
к~двум средам разной природы (компьютерное кодирование символов текстов 
с~по\-мощью таб\-лиц Unicode).
  
Основанием для построения второго уровня классификации трансформаций объектов послужила типология 
знаковых сис\-тем А.~Соломоника~\cite[c.~131]{25-zac}: естественные знаковые сис\-те\-мы, образные,  
ес\-тест\-вен\-но-язы\-ко\-в$\acute{\mbox{ы}}$е,  
вер\-баль\-но-не\-сло\-вес\-ные сис\-те\-мы записи\footnote{Под системой записи понимается знаковая 
система, сочетающая вербальные знаки с~несловесными (языки нотной записи, карт, таблиц и~др.).} 
и~формализованные знаковые сис\-те\-мы, включая математические. Введем понятие обобщенного текста~--- 
это текст, который может быть создан в~любой из перечисленных знаковых систем. Тогда обобщенные тексты 
могут быть естественными, образными, ес\-тест\-вен\-но-язы\-ко\-в$\acute{\mbox{ы}}$\-ми,  
вер\-баль\-но-не\-сло\-вес\-ны\-ми и~формализованными. Второй уровень классификации трансформаций 
охватывает не все виды объектов предметной  
об\-ласти информатики, а~только перечисленные~5~видов текс\-тов и~их представления, вовлеченные 
в~процессы трансформаций в~одной или более средах вместе с~данными, знанием и~его концептами.

\begin{figure*}[b] %fig1
\vspace*{6pt}
      \begin{center}
     \mbox{%
\epsfxsize=121.191mm 
\epsfbox{zac-1.eps}
}
\end{center}
\vspace*{-6pt}
\Caption{Средовая версия иерархии Акоффа}
\end{figure*}

\section{Классификация трансформаций: построение~третьего 
уровня}

  Основанием для систематизации трансформаций первого и~второго порядка 
на третьем уровне этой классификации служит иерархия Акоффа~\cite{8-zac}, 
на основе которой и~была создана ее средов$\acute{\mbox{а}}$я версия~[26, 
27]. Для создания средов$\acute{\mbox{о}}$й версии была выполнена 
категоризация трех базовых понятий информатики (данные, информация, 
знания) на объекты лексикографической информационной сис\-те\-мы 
в~процессе создания ее концепции\linebreak (рис.~1).
  


  В отличие от классической иерархии Акоффа, в~ее 
средов$\acute{\mbox{о}}$й версии различаются три вида данных: сенсорно 
воспринимаемые, цифровые и~те данные, которые генерируются 
искусственными нейронными сетями (ИНС) в~системах искусственного интеллекта 
(далее~--- ИИ-дан\-ные). Последний вид данных необходим, например, для 
различения входа и~выхода процесса применения обученной 
ИНС в~цифровой модели генерации знания, описанию которой 
посвящена работа~\cite{27-zac}.
  
  Также предлагается различать два вида информации: сенсорно 
воспринимаемая и~цифровая. Кроме знания в~средов$\acute{\mbox{у}}$ю 
версию добавлены концепты и~ментальные образы сенсорно воспринимаемых 
данных. Последние служат промежуточной сущностью между сенсорно 
воспринимаемыми данными и~генерируемым знанием при описании процессов 
извлечения знания из текстовых данных лексикографической информационной 
системы. Описание объектов средов$\acute{\mbox{о}}$й версии иерархии 
Акоффа (см.\ рис.~1) и~отношений между ними дано в~работах~\cite{26-zac, 28-zac}.
  
  В средов$\acute{\mbox{о}}$й версии число объектов равно восьми. Если 
учитывать направления трансформаций, то между восемью объектами на 
рис.~1 она включает~16 их видов (трансформации на границе между сенсорно 
воспринимаемыми данными и~информацией, обозначенные символом~<<?>>, 
в~статье не рас\-смат\-ри\-ва\-ют\-ся). В~будущем число объектов 
в~средов$\acute{\mbox{о}}$й версии, которая выбрана как основание для 
сис\-те\-ма\-ти\-за\-ции трансформаций первого и~второго порядка, может быть 
увеличено. Для построения классификации трансформаций 
важ\-но не возможное увеличение числа объектов 
и~трансформаций между ними, а то, что их виды в~средов$\acute{\mbox{о}}$й 
версии распределены между трансформациями первого и~второго порядка. Из 
16~видов на рис.~1 шесть относятся к~трансформациям первого порядка, это\linebreak 
виды с~номерами~7, 8, 13--16 (далее~--- типология трансформаций первого 
порядка), а~десять~--- к~трансформациям второго порядка, это виды 
с~\mbox{номерами}~1--6 и~9--12 (далее~--- типология трансформаций второго 
порядка). Разместим обе типологии на третьем уровне классификации (см.\ ее 
схему на рис.~2). Перечислим виды трансформаций первой типологии, вводя 
в~скобках их краткие названия, используемые ниже на рис.~3:
  \begin{description}
  \item[\,] 7~--- членение знания на концепты с~помощью одной или нескольких 
знаковых систем (далее~--- членение знания);
  \item[\,] 8~--- формирование знания на основе концептов (формирование 
знания);
  \item[\,] 13~--- обучение ИНС;
  \end{description}
  
  \vspace*{-6pt}
  
  \pagebreak
  
  \end{multicols}
  
  \begin{figure*} %fig2
\vspace*{1pt}
      \begin{center}
     \mbox{%
\epsfxsize=127.513mm 
\epsfbox{zac-2.eps}
}
\end{center}
\vspace*{-9pt}
\Caption{Схема трех верхних уровней классификации трансформаций объектов (объединены 
по три слоя и~для второго, и~для третьего уровней этой классификации)}
\end{figure*}
  
  \begin{multicols}{2}
  
  \noindent
  \begin{description}
  \item[\,] 14~--- восстановление обучающей информации на основе 
содержания обученной ИНС (обращение ИНС);
  \item[\,] 15~--- использование обученной ИНС (использование ИНС);



  \item[\,] 16~--- восстановление исходных данных, соответствующих 
полученным результатам работы обучен\-ной ИНС (восстановление исходных данных 
по результатам ИНС).
  \end{description}
  
  
  Не все виды трансформаций 13--16 поддерживаются в~конкретных системах 
искусственного интеллекта, но с~теоретической точки зрения все их 
предлагается включить в~первую типологию для полноты спектра видов 
трансформаций.
  
  Перечислим виды трансформаций второй типологии:
  \begin{description}
  \item[\,] 1~--- декодирование цифровых данных в~компьютерных системах 
(декодирование данных);
  \item[\,]  2~--- кодирование сенсорно воспринимаемых данных (кодирование 
данных);
  \item[\,] 3~--- ментальное копирование сенсорно воспринимаемых данных 
(ментальное копирование);
  \item[\,] 4~--- восстановление сенсорно воспринимаемых данных по 
ментальным образам (восстановление по образам);
  \item[\,] 5~--- смысловая интерпретация без деления на концепты ментальных 
образов сенсорно воспринимаемых данных (смысловая интерпретация);
  \item[\,] 6~--- восстановление ментальных образов (восстановление образов);
  \item[\,] 9~--- представление концептов в~виде сенсорно воспринимаемой 
информации, например текс\-та\-ми, формулами, таблицами, рисунками и~т.\,д.\ 
(представление концептов);
  \item[\,] 10~--- понимание смысла сенсорно воспринимаемой информации 
(понимание смысла);
  \item[\,] 11~--- кодирование сенсорно воспринимаемой информации 
(кодирование информации);
\end{description}

\vspace*{-6pt}

\pagebreak

\end{multicols}

\begin{figure*} %fig3
\vspace*{1pt}
      \begin{center}
     \mbox{%
\epsfxsize=163mm 
\epsfbox{zac-3.eps}
}
\end{center}
\vspace*{-9pt}
\Caption{Схема частного случая классификации трансформаций объектов (трансформации 
пронумерованы согласно рис.~1)}
\end{figure*}

\begin{multicols}{2}

\noindent
\begin{description}

  \item[\,] 12~--- декодирование цифровой информации (декодирование 
информации).
  \end{description}
  
  Отметим, что в~существующих ИТ
  и~компьютерных системах наиболее часто используются виды 
трансформаций~13 и~15 типологии первого порядка и~1, 2, 11 и~12 типологии 
второго порядка. На рис.~2 в~первом слое третьего уровня классификации 
показаны типологии первого порядка без указания числа трансформаций в~них 
и~без детализации трансформируемых объектов.
  
  Во втором слое третьего уровня классификации условно (без названий) 
показаны типологии второго порядка. Также на рис.~2 в~третьем слое третьего 
уровня классификации условно (также без названий) показаны типологии 
третьего порядка, которые планируется рассмотреть в~отдельной статье. По 
определению они должны включать трансформации между тремя объектами 
разной природы, но средов$\acute{\mbox{а}}$я версия иерархии Акоффа 
включает трансформации только между двумя объектами разной природы. 
Поэтому потребуется другое основание для их систематизации (ранее были 
рассмотрены отдельные примеры трансформаций третьего 
порядка\footnote{Далеко не всегда трансформации третьего и~более высоких порядков можно 
рассматривать как последовательность трансформаций второго порядка. Примером этого могут 
служить трансформации в~процессе обучения пациента пользованию роботизированной рукой, 
охватывающие личностные концепты пациента, релевантные его намерениям, сигналы активности 
мозга как объекты нейросреды и~компьютерные коды~\cite{29-zac}.}~\cite{29-zac}).

\section{Классификация трансформаций: частный~случай}

  Выше было отмечено, что в~будущем число объектов 
в~средов$\acute{\mbox{о}}$й версии иерархии Акоффа может быть увеличено. 
Это означает, что увеличатся и~чис\-ло объектов, и~чис\-ло трансформаций между 
ними в~классификации трансформаций, так как эта средов$\acute{\mbox{а}}$я 
версия служит по определению основанием для систематизации 
трансформаций первого и~второго порядка. Поэтому на третьем уровне рис.~2 
указаны типологии без детализации объектов и~без указания числа 
трансформаций в~каждой из них. С~одной стороны, при таком подходе 
получаем достаточно общий вид этой классификации, так как она не зависит от 
числа объектов в~том или ином варианте средов$\acute{\mbox{о}}$й версии 
(и~это существенно упрощает рис.~2). С~другой стороны, на третьем уровне 
такой общей классификации подразумевается, но не эксплицируется природа 
трансформируемых объектов и~их возможные сочетания в~трансформациях. 

При проектировании лексикографической информационной системы важно 
эксплицировать природу трансформируемых объектов и~их возможные 
сочетания.
  %
  Поэтому в~парадигму информатики~\cite{30-zac} кроме общей 
классификации трансформаций предлагается включать и~ее частные случаи, 
эксплицирующие природу трансформируемых объектов. 

В~этом разделе 
рассмотрим один частный случай, когда используются только естественные 
знаковые сис\-те\-мы из типологии А.~Соломоника~\cite{25-zac} вместе 
с~данными, знанием и~его концептами. Чис\-ло естественных языков при этом не 
ограничено. И~этот частный случай классификации включает только три 
класса природных трансформаций (первого, второго и~третьего порядка, см.\ 
схему классификации на рис.~3).
  
  Первый и~второй уровни схемы общей классификации (см.\ рис.~2) можно 
объединить в~один уровень в~этом частном случае. Ниже этого уровня 
приведено содержание типологий первого и~второго порядка без содержания 
типологий третьего по\-рядка.




  Наполнение типологий первого и~второго порядка соответствует 
средов$\acute{\mbox{о}}$й версии иерархии Акоффа на рис.~1, содержащей 
6~видов трансформаций типологии первого порядка и~10~видов 
трансформаций типологии второго порядка (на рис.~3 стрелки указывают 
направления трансформаций согласно средов$\acute{\mbox{о}}$й версии на рис.~1).
  
  Таким образом, частный случай классификации содержит для этих двух 
типологий 16~теоретически возможных трансформаций, 6 из которых 
в~настоящее время в~существующих ИТ применяются наиболее часто: виды 
трансформаций~1, 2, 11 и~12 типологии второго порядка реализуются 
с~помощью тех или иных методов ко\-ди\-ро\-ва\-ния/де\-ко\-ди\-ро\-ва\-ния 
(например, с~использованием таблиц Unicode), а~виды трансформаций~13 и~15
 в~типологии первого порядка реализуются полностью с~по\-мощью процессов 
цифровой обработки компьютерами.
  
  Остальные виды трансформаций или применяются намного реже (это 
виды~3, 5, 7, 9 и~10), или находятся в~стадии поиска и~разработки (14 и~16) или 
в~настоящее время носят только теоретический характер, обеспечивая полноту 
первой и~второй типологий (4, 6 и~8). Знаком~<<?>> обозначены те виды 
трансформаций, которые по определению не существуют в~используемой 
парадигме информатики~\cite{30-zac}. Однако возможно, что в~других 
будущих подходах к~построению ее парадигмы эти виды трансформаций будут 
существовать.
  
\section{Заключение}

  На сегодняшний день процесс построения классификаций объектов 
предметной области информатики~\cite{22-zac} и~их  
трансформаций~\cite{23-zac} еще не завершен. Однако первые результаты их 
построения уже используются для создания концепции лексикографической 
информационной сис\-те\-мы, обеспечивающей интеграцию двуязычных 
словарей и~параллельных корпусов.
  
  \bigskip
  
  
  Автор признателен рецензентам за помощь в~улучшении статьи.
  
{\small\frenchspacing
 { %\baselineskip=10.6pt
 %\addcontentsline{toc}{section}{References}
 \begin{thebibliography}{99}
\bibitem{1-zac}
\Au{Aijmer K., Altenberg~B.} Advances in corpus-based contrastive linguistics. Studies in honour 
of Stig Johansson.~--- Amsterdam: John Benjamins, 2013. 295~p.  doi: 10.1075/scl.54.
\bibitem{2-zac}
\Au{Добровольский Д.\,О., Кретов~А.\, А., Шаров~С.\,А.} Корпус параллельных текстов~// 
Научная и~техническая информация. Сер.~2: Информационные процессы и~сис\-те\-мы, 2005. 
№\,6. С.~16--27.
\bibitem{3-zac}
\Au{Добровольский Д.\,О.} Корпус параллельных текстов и~сопоставительная 
лексикология~// Труды Института русского языка им.\ В.\,В.~Виноградова, 2015. №\,6. 
С.~413--449. EDN: VJQBHP.
\bibitem{4-zac}
\Au{Гончаров А.\,А., Зацман~И.\,М., Кружков~М.\,Г.} Эволюция классификаций 
в~надкорпусных базах данных~// Информатика и~её применения, 2020. Т.~14. Вып.~4. 
С.~108--116. doi: 10.14357/19922264200415.  
EDN: \mbox{GKWBZT}.
\bibitem{5-zac}
\Au{Гончаров А.\, А., Зацман И. \,М., Кружков~М.\, Г}. Представление новых 
лексикографических знаний в~динамических классификационных сис\-те\-мах~// 
Информатика и~её применения, 2021. Т.~15. Вып.~1. С.~86--93.  doi: 10.14357/19922264210112. EDN: OPEFXW.
\bibitem{6-zac}
\Au{Zatsman I.} Finding and filling lacunas in linguistic typologies~// 15th Forum (International) 
on Knowledge Asset Dynamics Proceedings.~--- Matera, Italy: Institute of Knowledge Asset 
Management, 2020. P.~780--793.
\bibitem{7-zac}
\Au{Zatsman I.} Three-dimensional encoding of emerging meanings in AI-systems~// 21st 
European Conference on Knowledge Management Proceedings.~--- Reading, U.K.: Academic 
Publishing International Ltd., 2020. P.~878--887.
\bibitem{8-zac}
\Au{Ackoff R.} From data to wisdom~// J.~Applied Systems Analysis, 1989. Vol.~16. No.\,1. P.~3--9.
\bibitem{9-zac}
\Au{Rosenbloom P.\,S.} On computing: The fourth great scientific domain.~--- Cambridge, MA, 
USA: MIT Press, 2013. 307~p.
\bibitem{10-zac}
\Au{Rowley J.} The wisdom hierarchy: Representations of the DIKW hierarchy~// J.~Inf. 
Sci., 2007. Vol.~33. Iss.~2. P.~163--180. doi: 10.1177/0165551506070706.
\bibitem{11-zac} 
\Au{Frick$\acute{\mbox{e}}$~M.\,H.} Data--Information--Knowledge--Wisdom (DIKW) pyramid, 
framework, continuum~// Encyclopedia of big data~/ Eds. L.~Schintler, C.~McNeely.~--- Cham: 
Springer, 2018. 4~p. doi: 10.1007/978-3-319-32001-4\_331-1.
\bibitem{12-zac}
\Au{Denning P., Rosenbloom~P.} Computing: The fourth great domain of science~// Commun. 
ACM, 2009. Vol.~52. Iss.~9. P.~27--29.
\bibitem{13-zac}
\Au{Denning P., Freeman~P.} Computing's paradigm~// Commun.  ACM, 2009. Vol.~52. 
Iss.~12. P.~28--30. doi: 10.1145/ 1610252.1610265.
\bibitem{17-zac} %14
\Au{Farradane J.} Knowledge, information, and information science~// J.~Inf. Sci., 
1980. Vol.~2. Iss.~2. P.~75--80. doi: 10.1177/01655515800020020.

\bibitem{15-zac}
\Au{Шрейдер Ю.\,А.} Информация и~знание~// Сис\-тем\-ная концепция информационных 
процессов.~--- М.: ВНИИСИ, 1988. С.~47--52.
\bibitem{16-zac}
\Au{Ingwersen P.} Information and information science~// Enclyclopaedie of library and 
information science~/ Eds. J.\,D.~McDonald, 
M.~Levine-Clark.~--- New York, NY, USA: Marcel Dekker Inc., 1992. Vol.~56. Sup.~19. 
P.~137--174.

\bibitem{14-zac} %17
Информатика как наука об информации: Информационный, документальный, 
технологический, экономический, социальный и~организационный аспекты~/ Под ред. 
Р.\,С.~Гиляревского.~--- М.: Фаир-Пресс, 2006. 592~с.

\bibitem{18-zac}
\Au{Hjorland B.} Library and information science: practice, theory, and philosophical basis~// 
Inform. Process. Manag., 2000. Vol.~36. Iss.~3. P.~501--531. doi:  
10.1016/S0306-\mbox{4573(99)00038-2}.
\bibitem{19-zac}
Deep shift~--- technology tipping points and societal impact.~--- Geneva: WE Forum, 2015. 44~p. 
{\sf http://www3.weforum.org/docs/WEF\_GAC15\_ Technological\_Tipping\_Points\_report\_2015.pdf}.
\bibitem{20-zac}
\Au{Berman F., Rutenbar~R., Hailpern~B., Christensen~H., Davidson~S., Estrin~D., 
Franklin~M., Martonosi~M., Raghavan~P., Stodden~V., Szalay~A.\,S.} Realizing the potential of 
data science~// Commun.  ACM, 2018. Vol.~61. Iss.~4. P.~67--72. doi: 10.1145/3188721.

\bibitem{21-zac}
\Au{Stodden V.} The data science life cycle: A~disciplined approach to advancing data science as 
a~science~// Commun.  ACM, 2020. Vol.~63. Iss.~7. P.~58--66. doi: 10.1145/ 3360646.


\bibitem{23-zac} %22
\Au{Зацман И.\,М.} Научная парадигма информатики: классификация трансформаций 
объектов предметной об\-ласти~// Системы и~средства информатики, 2023. Т.~33. №\,4. 
С.~126--138. doi: 10.14357/08696527230412. EDN: ZIKUWO.

\bibitem{22-zac} %23
\Au{Зацман И.\,М.} Научная парадигма информатики: классификация объектов предметной  
об\-ласти~// Информатика и~её применения, 2023. Т.~17. Вып.~4. С.~96--103. doi: 
10.14357/19922264230413. EDN: FIUQAT.

\bibitem{24-zac}
\Au{Зацман И.\,М.} О~научной парадигме информатики: верхний уровень классификации 
объектов ее предметной об\-ласти~// Информатика и~её применения, 2022. Т.~16. Вып.~4. 
С.~73--79. doi: 10.14357/ 19922264220411. EDN: XZNKVI.

\bibitem{25-zac}
\Au{Соломоник А.\,Б.} Философия знаковых систем и~язык.~--- М.: ЛКИ, 2011. 408~с.
\bibitem{26-zac}
\Au{Зацман И.\,М.} Трансформация иерархии Акоффа в~научной парадигме информатики~// 
Информатика и~её применения, 2023. Т.~17. Вып.~3. С.~107--113. doi: 
10.14357/19922264230315. EDN: UMVRRV.

\bibitem{27-zac}
\Au{Zatsman I.} Building digital spiral models of knowledge generation~// 19th Forum 
(International) on Knowledge Asset Dynamics Proceedings.~--- Matera, Italy: Arts for Business 
Institute, 2024. P.~2185--2196.
\bibitem{28-zac}
\Au{Zatsman I.} Digital spiral model of knowledge creation and encoding its dynamics~// 18th 
Forum (International) on Knowledge Asset Dynamics Proceedings.~--- Matera, Italy: Arts for 
Business Institute, 2023. P.~581--596.
\bibitem{29-zac}
\Au{Зацман И.\,М.} Интерфейсы третьего порядка в~информатике~// Информатика и~её 
применения, 2019. Т.~13. Вып.~3. С.~82--89. doi: 10.14357/19922264190312. EDN: 
EHRQLF.

\bibitem{30-zac}
\Au{Зацман И.\,М.} Научная парадигма информатики как третьей культуры~//  
На\-уч\-но-тех\-ни\-че\-ская информация. Сер.~1: Организация и~методика информационной 
работы, 2023. №\,11. С.~1--14.

\end{thebibliography}

 }
 }

\end{multicols}

\vspace*{-9pt}

\hfill{\small\textit{Поступила в~редакцию 14.04.24}}

\vspace*{4pt}

%\pagebreak

%\newpage

%\vspace*{-28pt}

\hrule

\vspace*{2pt}

\hrule



\def\tit{OBJECT TRANSFORMATIONS OF~THE~FIRST AND~SECOND ORDER
IN~A~LEXICOGRAPHIC INFORMATION SYSTEM\\[-5pt]}


\def\titkol{Object transformations of~the~first and~second order
in~a~lexicographic information system}


\def\aut{I.\,M.~Zatsman}

\def\autkol{I.\,M.~Zatsman}

\titel{\tit}{\aut}{\autkol}{\titkol}

\vspace*{-13pt}


\noindent
Federal Research Center ``Computer Science and Control'' of the Russian Academy of Sciences, 
44-2~Vavilov Str., Moscow 119133, Russian Federation


\def\leftfootline{\small{\textbf{\thepage}
\hfill INFORMATIKA I EE PRIMENENIYA~--- INFORMATICS AND
APPLICATIONS\ \ \ 2024\ \ \ volume~18\ \ \ issue\ 2}
}%
 \def\rightfootline{\small{INFORMATIKA I EE PRIMENENIYA~---
INFORMATICS AND APPLICATIONS\ \ \ 2024\ \ \ volume~18\ \ \ issue\ 2
\hfill \textbf{\thepage}}}

\vspace*{2pt}



\Abste{The theoretical foundations of the design of information technologies used for 
the integration of bilingual dictionaries and parallel corpora are considered. The 
description of the first outcomes of the creation of the third\linebreak\vspace*{-12pt}}

\Abstend{ level of object 
transformations classification in the subject domain of informatics, which is supposed 
to be used
in creating the lexicographic information system providing integration, is 
given. All the entities of informatics are divided into two global classes: objects and 
their transformations. For each such class, its own classification is constructed. 
Previously, the two upper levels of the object transformation classification in the subject 
domain have been described. The present paper discusses the third level of this classification. The 
basis for the construction of its highest level was the division of the subject domain of 
informatics into media (mental, sensory, digital, and a~number of other media), each 
of which by definition includes objects of the same nature. The Solomonick's 
typology of sign systems served as the basis for constructing the second level of the 
object transformation classification. The aim of the paper is to systematize object 
transformations of the first and second orders at the third level of this classification. 
The basis for systematization is the medium version of the Ackoff's hierarchy.}

\KWE{subject domain objects; object transformations; classification; data; 
information; knowledge; lexicographic information system}


\DOI{10.14357/19922264240211}{VZTGVV}

\vspace*{-12pt}

\Ack

\vspace*{-3pt}


\noindent
The reported study was funded by the Russian Science Foundation, project  
No.\,24-18-00155, {\sf 
https://rscf.ru/project/24-18-00155}. The research was carried out using the infrastructure of the Shared 
Research Facilities ``High Performance Computing and Big Data'' (CKP 
``Informatics'') of FRC CSC RAS (Moscow) .
   


  \begin{multicols}{2}

\renewcommand{\bibname}{\protect\rmfamily References}
%\renewcommand{\bibname}{\large\protect\rm References}

{\small\frenchspacing
 {%\baselineskip=10.8pt
 \addcontentsline{toc}{section}{References}
 \begin{thebibliography}{99} 
\bibitem{1-zac-1}
\Aue{Aijmer, K., and B.~Altenberg.} 2013. \textit{Advances in corpus-based 
contrastive linguistics. Studies in honour of Stig Johansson}. Amsterdam: John 
Benjamins. 295~p. doi: 10.1075/scl.54.
\bibitem{2-zac-1}
\Aue{Dobrovolskiy, D.\,O., A.\,A.~Kretov, and S.\,A.~Sharov.} 2005. Korpus 
parallel'nykh tekstov [Corpus of parallel texts]. \textit{Nauchnaya i~tekhnicheskaya 
informatsiya. Ser. 2. Informatsionnye protsessy i~sistemy} [Scientific and Technical 
Information. Ser.~2: Information Processes and Systems] 6:16--27.
\bibitem{3-zac-1}
\Aue{Dobrovolskiy, D.\,O.} 2015. Korpus parallel'nykh tekstov i~sopostavitel'naya 
leksikologiya [The corpus of parallel texts and contrastive lexicology]. \textit{Trudy 
Instituta russkogo yazyka im. V.\,V.~Vinogradova} [Proceedings of the 
V.\,V.~Vinogradov Russian Language Institute] 6:413--449. EDN: VJQBHP.
\bibitem{4-zac-1}
\Aue{Goncharov, A.\,A., I.\,M.~Zatsman, and M.\,G.~Kruzhkov.} 2020. Evolyutsiya 
klassifikatsiy v~nadkorpusnykh ba\-zakh dannykh [Evolution of classifications in 
supracorpora databases]. \textit{Informatika i~ee Primeneniya~--- Inform. \mbox{Appl.}}  
14(4):108--116. doi: 10.14357/19922264200415.  
EDN: GKWBZT.
\bibitem{5-zac-1}
\Aue{Goncharov, A.\,A., I.\,M.~Zatsman, and M.\,G.~Kruzhkov.} 2021. 
Predstavlenie novykh leksikograficheskikh znaniy v~dinamicheskikh 
klassifikatsionnykh sistemakh [Representation of new lexicographical knowledge in 
dynamic classification systems]. \textit{Informatika i~ee Primeneniya~--- Inform. 
Appl.} 15(1):86--93. doi: 10.14357/19922264210112. EDN: OPEFXW.
\bibitem{6-zac-1}
\Aue{Zatsman, I.} 2020. Finding and filling lacunas in linguistic typologies. 
\textit{15th Forum (International) on Knowledge Asset Dynamics Proceedings}. 
Matera, Italy: Institute of Knowledge Asset Management. 780--793.
\bibitem{7-zac-1}
\Aue{Zatsman, I.} 2020. Three-dimensional encoding of emerging meanings in  
AI-systems. \textit{21st European Conference on Knowledge Management 
Proceedings}. Reading, U.K.: Academic Publishing International Ltd. 878--887.
\bibitem{8-zac-1}
\Aue{Ackoff, R.} 1989. From data to wisdom. \textit{J.~Applied Systems Analysis} 
16(1):3--9.
\bibitem{9-zac-1}
\Aue{Rosenbloom, P.\,S.} 2013. \textit{On computing: The fourth great scientific 
domain}. Cambridge, MA: MIT Press. 307~p.
\bibitem{10-zac-1}
\Aue{Rowley, J.} 2007. The wisdom hierarchy: Representations of the DIKW 
hierarchy. \textit{J.~Inf. Sci.} 33(2):163--180. doi: 10.1177/0165551506070706.
\bibitem{11-zac-1}
\Aue{Frick$\acute{\mbox{e}}$, M.\,H.} 2018.  
Data-Information-Knowledge-Wisdom (DIKW) pyramid, framework, continuum. 
\textit{Encyclopedia of big data}. Eds. L.~Schintler and C.~McNeely. Cham: 
Springer. 4~p. doi: 10.1007/978-3-319-32001- 4\_331-1.
\bibitem{12-zac-1}
\Aue{Denning, P., and P.~Rosenbloom.} 2009. Computing: The fourth great domain 
of science. \textit{Commun. ACM} 52(9):27--29.
\bibitem{13-zac-1}
\Aue{Denning, P., and P.~Freeman.} 2009. Computing's paradigm. \textit{Commun. 
ACM} 52(12):28--30. doi: 10.1145/ 1610252.1610265.

\bibitem{17-zac-1} %14
\Aue{Farradane, J.} 1980. Knowledge, information, and information science. 
\textit{J.~Inf. Sci.} 2(2):75--80. doi: 10.1177/ 01655515800020020.

\bibitem{15-zac-1}
\Aue{Shreyder, Yu.\,A.} 1988. Informatsiya i~znanie [Information and knowledge]. 
\textit{Sistemnaya kontseptsiya in\-for\-ma\-tsi\-on\-nykh protsessov} [System concept of 
information processes]. Moscow: VNIISI. 47--52.
\bibitem{16-zac-1}
\Aue{Ingwersen, P.} 1995. Information and information science. 
\textit{Encyclopedia of library and information science}. Eds. J.\,D.~McDonald and 
M.~Levine-Clark. New York, NY: Marcel Dekker Inc. 56(19):137--174.

\bibitem{14-zac-1} %17
Gilyarevskiy, R.\,S., ed. 2006. \textit{Informatika kak nauka ob informatsii: 
informatsionnyy, dokumental'nyy, tekh\-no\-lo\-gi\-che\-skiy, ekonomicheskiy, sotsial'nyy 
i~organizatsionnyy aspekty} [Informatics as information science: Informational, 
documentary, technological, economic, social, and organizational dimensions]. 
Moscow: FAIR-PRESS. 592~p.

\bibitem{18-zac-1}
\Aue{Hjorland, B.} 2000. Library and information science: Practice, theory, and 
philosophical basis. \textit{Inform. Process. Manag.} 36(3):501--531. doi:  
10.1016/S0306-\mbox{4573(99)00038-2}.
\bibitem{19-zac-1}
Deep shift~--- technology tipping points and societal impact. 2015. \textit{World Economic 
Forum}. Geneva. 44~p. Available at: {\sf 
http://www3.weforum.org/docs/WEF\_ GAC15\_Technological\_Tipping\_Points\_report\_2015.pdf} (accessed May~20, 
2024).
\bibitem{20-zac-1}
\Aue{Berman, F., R.~Rutenbar, B.~Hailpern, H.~Christensen, S.~Davidson, 
D.~Estrin, M.~Franklin, M.~Martonosi, P.~Raghavan, V.~Stodden, and 
A.\,S.~Szalay.} 2018. Realizing the potential of data science. \textit{Commun. ACM} 
61(4):67--72. doi: 10.1145/3188721.
\bibitem{21-zac-1}
\Aue{Stodden, V.} 2020. The data science life cycle: A~disciplined approach to 
advancing data science as a~science. \textit{Commun. ACM} 
 63(7):58--66. doi: 10.1145/3360646.

\bibitem{23-zac-1} %22
\Aue{Zatsman, I.\,M.} 2023. Nauchnaya paradigma informatiki: klassifikatsiya 
transformatsiy ob''ektov predmetnoy oblasti [Scientific paradigm of informatics: 
Transformation classification of domain objects]. \textit{Sistemy i~Sredstva 
Informatiki~--- Systems and Means of Informatics} 33(4):126--138. doi: 
10.14357/08696527230412. EDN: ZIKUWO.

\bibitem{22-zac-1} %23
\Aue{Zatsman, I.\,M.} 2023. Nauchnaya paradigma informatiki: klassifikatsiya 
ob''ektov predmetnoy oblasti [Scientific paradigm of informatics: Classification of 
domain objects]. \textit{Informatika i~ee Primeneniya~--- Inform. Appl.} 
 17(4):96--103. doi: 10.14357/19922264230413. EDN: FIUQAT.
 
\bibitem{24-zac-1}
\Aue{   Zatsman, I.\,M.} 2022. O nauchnoy paradigme informatiki: verkhniy uroven' 
klassifikatsii ob''ektov ee predmetnoy oblasti [On the scientific paradigm of 
informatics: The classification high level of its objects]. \textit{Informatika i~ee 
Primeneniya~--- Inform. Appl.} 16(4):73--79. doi: 10.14357/19922264220411. EDN: 
XZNKVI.
\bibitem{25-zac-1}
\Aue{Solomonick, A.\,B.} 2011. \textit{Filosofiya znakovykh system i~yazyk} 
[Philosophy of sign systems and language]. Moscow: LKI. 408~p.
\bibitem{26-zac-1}
\Aue{Zatsman, I.\,M.} 2023. Transformatsiya ierarkhii Akoffa v~nauchnoy 
paradigme informatiki [Transformation of the Ackoff's hierarchy in the scientific 
paradigm of informatics]. \textit{Informatika i~ee Primeneniya~--- Inform. \mbox{Appl.}} 
17(3):107--113. doi: 10.14357/19922264230315. EDN: UMVRRV.
\bibitem{27-zac-1}
\Aue{Zatsman, I.} 2024. Building digital spiral models of knowledge 
generation. \textit{19th Forum (International) on Knowledge Asset Dynamics 
Proceedings}. Matera, Italy: Arts for Business Institute. 2185--2196.
\bibitem{28-zac-1}
\Aue{Zatsman, I.} 2023. Digital spiral model of knowledge creation and encoding its 
dynamics. \textit{18th Forum (International) on Knowledge Asset Dynamics 
Proceedings}. Matera, Italy: Arts for Business Institute. 581--596.
\bibitem{29-zac-1}
\Aue{Zatsman, I.\,M.} 2019. Interfeysy tret'ego poryadka v~informatike 
 [Third-order interfaces in informatics]. \textit{Informatika i~ee Primeneniya~--- 
Inform. Appl.} 13(3):82--89. doi: 10.14357/19922264190312. EDN: EHRQLF.
\bibitem{30-zac-1}
\Aue{Zatsman, I.} 2023. Scientific paradigm of informatics as a~third culture. 
\textit{Scientific Technical Information Processing} 50(4):246--258. doi: 
10.3103/S0147688223040111. EDN: CKHMYS.

\end{thebibliography}

 }
 }

\end{multicols}

\vspace*{-6pt}

\hfill{\small\textit{Received April 14, 2024}} 


\vspace*{-12pt}


\Contrl

\vspace*{-3pt}

\noindent
\textbf{Zatsman Igor M.} (b.\ 1952)~--- Doctor of Science in technology, head of 
department, Federal Research Center ``Computer Science and Control'' of the 
Russian Academy of Sciences, 44-2~Vavilov Str., Moscow 119333, Russian 
Federation; \mbox{izatsman@yandex.ru}





\label{end\stat}

\renewcommand{\bibname}{\protect\rm Литература}    %6
\def\stat{kor-kor}



\def\tit{МОДИФИЦИРОВАННЫЙ СЕТОЧНЫЙ МЕТОД РАЗДЕЛЕНИЯ ДИСПЕРСИОННО-СДВИГОВЫХ
СМЕСЕЙ НОРМАЛЬНЫХ ЗАКОНОВ$^*$}



\def\titkol{Модифицированный сеточный метод разделения дисперсионно-сдвиговых
смесей нормальных законов}

\def\aut{В.\,Ю.~Королев$^1$,  А.\,Ю.~Корчагин$^2$}

\def\autkol{В.\,Ю.~Королев,  А.\,Ю.~Корчагин}

\titel{\tit}{\aut}{\autkol}{\titkol}

{\renewcommand{\thefootnote}{\fnsymbol{footnote}} \footnotetext[1]
{Работа поддержана Российским научным фондом (проект 14-11-00364).}}


\renewcommand{\thefootnote}{\arabic{footnote}}
\footnotetext[1]{Факультет
вычислительной математики и кибернетики Московского государственного
университета им.\ М.\,В.~Ломоносова; Институт проблем информатики
Российской академии наук; victoryukorolev@yandex.ru}
\footnotetext[2]{Факультет вычислительной математики и кибернетики
Московского государственного университета им.\ М.\,В.~Ломоносова;
sasha.korchagin@gmail.com}

%\vspace*{2pt}



\Abst{Описывается модифицированный двухэтапный
сеточный метод разделения дис\-пер\-си\-он\-но-сдви\-го\-вых смесей нормальных
законов, представляющий собой альтернативу чистому ЕМ (expectation-maximization)
ал\-го\-рит\-му. На
первом этапе этого алгоритма строится дискретная аппроксимация для
смешивающего распределения, на втором этапе подбирается абсолютно
непрерывное распределение из заранее заданного семейства, например,
обобщенных обратных гауссовских законов, ближайшее к~дискретному
распределению, полученному на первом этапе. Обсуждаются вопросы
сходимости этого двухэтапного алгоритма. Доказана монотонность
сеточного итерационного метода, используемого на первом этапе.
Подробно обсуждается вопрос оптимального выбора параметров метода,
прежде всего сетки, накидываемой на носитель смешивающего
распределения. С~этой целью предложены статистические оценки
квантилей смешивающего распределения. Эффективность метода
иллюстрируется примерами конкретных вычислений оценок параметров
обобщенных гиперболических распределений.}

\KW{смесь распределений вероятностей;
дис\-пер\-си\-он\-но-сдви\-го\-вая смесь нормальных законов; обобщенное
гиперболическое распределение; ЕМ-ал\-го\-ритм; сеточный метод
разделения смесей}

\vspace*{1pt}

%\vspace*{2pt}

\DOI{10.14357/19922264140402}


\vskip 12pt plus 9pt minus 6pt

\thispagestyle{headings}

\begin{multicols}{2}

\label{st\stat}

\section{Введение}

При {\it практическом} решении задачи моделирования и исследования
волатильности (изменчивости) хаотических стохастических процессов
ключевым этапом является статистическое разделение смесей
вероятностных распределений. Задача разделения смесей~---
статистического оценивания параметров смесей вероятностных
распределений~--- в~деталях разобрана, например, в~книге~\cite{k2011}.

Для решения задачи разделения смесей вероятностных распределений
традиционно используются итерационные процедуры типа ЕМ-ал\-го\-рит\-ма.
К~сожалению, классический ЕМ-ал\-го\-ритм обладает рядом серьезных
недостатков при его применении к~смесям нормальных законов, а~именно:
он демонстрирует крайнюю неустойчивость по отношению к~исходным
данным и~начальным приближениям.

Для преодоления этих недостатков
предложено много модификаций ЕМ-ал\-го\-рит\-ма (см., например,~\cite{k2011}).
Вместе с тем в~указанной книге предложен и~исследован
принципиально новый~--- сеточный~--- метод приближенного решения
задачи разделения смесей. В~работе~\cite{n2013} подробно исследованы
вопросы сходимости сеточных методов разделения смесей.

В соответствии с подходом к~статистическому анализу хаотических
стохастических процессов, в~частности к~решению задачи декомпозиции
волатильности таких процессов, развитом в~книге~\cite{k2011},
в~общем случае на практике приходится решать задачу разделения
конечных смесей нормальных законов с~произвольно большим числом
неизвестных параметров (параметров компонент и~их весов).
И~хотя в~большинстве приложений возникают смеси не более чем с~пятью--семью
компонентами, даже при использовании таких смесей, скажем, в~задачах
анализа и~прогнозирования финансовых рисков приходится моделировать
траекторию движения точки в~пространствах, размерность которых
соответственно лежит в~пределах от~14 (для пятикомпонентных смесей)
до~20 (для семикомпонентных смесей), что существенно увеличивает
вычислительные и~временн$\acute{\mbox{ы}}$е ресурсы, необходимые для практического
решения указанных задач.

Поскольку во многих ситуациях (например,
при прогнозировании на основе высокочастотных данных) эти задачи
необходимо решать в~режиме, близком к~реальному времени, для
создания эффективных методов статистического анализа на основе
смешанных моделей на первый план выходит проб\-ле\-ма снижения
размерности решаемой задачи, т.\,е.\ параметрического пространства.

Одним из возможных подходов к~снижению размерности является
априорное сужение классов допусти\-мых смесей. К~примеру, при решении
многих задач, связанных с~анализом процессов атмосферной или
плазменной турбулентности, а~так\-же процессов, описывающих эволюцию
различных финансовых индексов, высочайшую адекватность
продемонстрировали модели, основанные на дис\-пер\-си\-он\-но-сдви\-го\-вых
смесях нормальных законов. Класс таких смесей очень обширен
и,~в~част\-ности, включает в~себя обобщенные гиперболические распределения,
которые были введены О.-Е.~Барн\-дорфф-Ниль\-се\-ном в~1977--1978~гг.\ как
класс специальных сдвиг-мас\-штаб\-ных смесей нормальных законов~\cite{BN1977, BN1978}.
Пусть $\alpha\hm\in\r$, $\beta\hm\in\r$. Если
функцию распределения обобщенного гиперболического закона
с~параметрами~$\alpha$, $\beta$, $\nu$, $\mu$, $\lambda$ обозначить
$P_{GH}(x;\alpha,\beta,\nu,\mu,\lambda)$, то по определению
\begin{multline}
P_{GH}(x;\alpha,\beta,\nu,\mu,\lambda)={}\\
{}=
\int\limits_{0}^{\infty}\Phi\left(\fr{x-\beta-\alpha
z}{\sqrt{z}}\right)\,p_{GIG}(z;\nu,\mu,\lambda)\,dz\,,\\
x\in\r\,,
\label{e1-kor}
\end{multline}
где $\Phi(x)$~--- стандартная нормальная функция распределения:
$$
\Phi(x)=\int\limits_{-\infty}^{x}\varphi(z)\,dz\,,\enskip
\varphi(x)=\fr{1}{\sqrt{2\pi}}e^{-x^2/2}\,,\enskip  x\in\mathbb{R}\,;
$$
$p_{GIG}(x;\nu,\mu,\lambda)$~--- плот\-ность обобщенного обратного
гауссовского распределения:
\begin{multline*}
p_{GIG}(x;\nu,\mu,\lambda)={}\\
{}=\fr{\lambda^{\nu/2}}{2\mu^{\nu/2}
K_{\nu}\left(\sqrt{\mu\lambda}\right)}\,
x^{\nu-1}\exp\left\{-\fr{1}{2}\left(\fr{\mu}{x}+\lambda
x\right)\right\}\,,\\ x>0\,.
\end{multline*}
Здесь $\nu\in\r$;
$$
\begin{array}{lll}
\mu>0\,, & \lambda\geqslant0\,, & \mbox{если }\nu<0\,;\\[6pt]
\mu>0\,, & \lambda>0\,, & \mbox{если }\nu=0\,;\\[6pt]
\mu\geqslant0\,, & \lambda>0\,, & \mbox{если }\nu>0\,;
\end{array}
$$
$K_{\nu}(z)$~--- модифицированная бесселева функция третьего рода
порядка~$\nu$:

\noindent
\begin{multline*}
K_{\nu}(z)=\fr{1}{2}\int\limits_{0}^{\infty}y^{\nu-1}\exp
\left\{-\fr{z}{2}\left(y+\fr{1}{y}\right)\right\}\,dy\,,\\
z\in\mathbb{C}\,,\enskip \mathrm{Re}\,z>0\,.
\end{multline*}
Обратим внимание, что в~(1) смешивание происходит одновременно и~по
параметру сдвига, и~по параметру масштаба, но так как эти параметры
в~(1)  связаны жесткой зависимостью, так что параметр сдвига
смешиваемого распределения пропорционален его дисперсии, то
фактически смесь~(1) является {\it однопараметрической} и~поэтому
называется {\it дис\-пер\-си\-он\-но-сдви\-го\-вой} (см., например,~\cite{BN1982}).

Другим примером дис\-пер\-си\-он\-но-сдви\-го\-вых смесей нормальных законов
являются обобщенные дисперсионные гам\-ма-рас\-пре\-де\-ле\-ния, в~которых
смешивающими являются обобщенные гам\-ма-рас\-пре\-де\-ле\-ния~\cite{ks2012, zk2013}.

В указанных семействах смесей число неизвестных параметров равно
пяти или шести (если\linebreak учитывать неслучайный сдвиг). Вместе
с~тем у~подоб\-ных моделей имеются довольно серьезные тео\-ре\-ти\-че\-ские
обоснования: в~работах~\cite{zk2013, k2013} показано, что указанные
модели являются асимптотическими аппроксимациями в~простой
предельной схеме случайного суммирования и~потому могут успешно
применяться для анализа процессов типа остановленных случайных
блужданий. Эти выводы подтверждены статистическим анализом
вы\-со\-ко\-час\-тот\-ных финансовых данных, в~результате которого выявлен
синхронизированный характер изменения интенсивностей потоков заявок
в~сис\-те\-мах электронных торгов, что естественно приводит к~синхронизированному
поведению па\-ра\-мет\-ров сдвига и~диффузии в~соответствующих моделях вида смесей
нормальных законов~\cite{kckg2013}.

\section{Описание моди\-фи\-ци\-ро\-ван\-но\-го
сеточного ме\-то\-да разделения дисперсионно-сдвиговых смесей
нормальных законов и~его свойства}

Оказывается, что сеточные методы разделения смесей довольно
эффективны не только при разделении конечных смесей нормальных
законов, но и~при разделении произвольных дис\-пер\-си\-он\-но-сдви\-го\-вых
смесей нормальных законов. Поясним сказанное на примере задачи
оценивания па\-ра\-мет\-ров обобщенных гиперболических распределений.

Для решения задачи оценивания параметров обобщенных гиперболических
распределений традиционно используется метод, предложенный в~статье~\cite{p2004}
и~по сути являющийся классическим ЕМ-ал\-го\-рит\-мом,
приспособленным к~конкретной задаче, и,~соответственно, наследующий
присущие ЕМ-ал\-го\-рит\-мам недостатки.

Рассмотрим следующий альтернативный двухэтапный метод. На первом
этапе на поло\-жи\-тельной полупрямой выделим основную часть носителя
смешивающего распределения, т.\,е.\ \mbox{ограниченный} интервал,
вероятность которого, вычисленная в~соответствии со смешивающим
распределением, практически равна единице. На этот интервал накинем
конечную сетку, содержащую, возможно, очень много {\it известных}
узлов $u_1,\ldots,u_K$. Считая параметр сдвига~$\beta$ равным нулю,
приблизим искомое обобщенное гиперболическое распределение конечной
смесью нормальных законов:

\noindent
\begin{multline}
P_{GH}(x;\,\alpha,0,\nu,\mu,\lambda)\approx{}\\
{}\approx \sum\limits_{i=1}^K
p_i\Phi\left(\fr{x-\alpha u_i}{\sqrt{u_i}}\right)\,,\enskip
x\in\mathbb{R}\,.\label{e2-kor}
\end{multline}
В смеси, стоящей в~правой части соотношения~(2), неизвестными
являются только параметры $p_1,\ldots,p_{K-1}$ и~$\alpha$. Пусть
$x_1,\ldots,x_n$~--- анализируемая выборка значений случайной
величины с~оцениваемым обобщенным гиперболическим распределением.
Итерационный процесс, определяющий сеточный ЕМ-ал\-го\-ритм для данной
задачи, задается следующим образом. Пусть
$p_1^{(m)},\ldots,p_{K-1}^{(m)}$ и~$\alpha^{(m)}$~--- оценки параметров
$p_1,\ldots,p_{K-1}$ и~$\alpha$ на $m$-й итерации,
$p_K^{(m)}\hm=1\hm-p_1^{(m)}-\cdots-p_{K-1}^{(m)}$. Обозначим

\noindent
\begin{align*}
\varphi_{ij}^{(m)}&=\fr{1}{\sqrt{u_i}}\varphi\left(\fr{x_j-\alpha^{(m)}u_i}{\sqrt{u_i}}\right)\,;
\\
g_{ij}^{(m)}&=\fr{p_i^{(m)}\varphi_{ij}^{(m)}}{\sum\limits_{r=1}^K
p_r^{(m)}\varphi_{rj}^{(m)}}\,,\\
&\hspace*{14mm}i=1,\ldots,K\,;\enskip j=1,\ldots,n\,.
\end{align*}
Тогда, используя стандартные рассуждения, определяющие
вычислительные формулы EM-ал\-го\-рит\-ма для параметров конечной смеси
нормальных законов (см, например,~[1, разд.~5.3.7--5.3.8]),
следует положить

\noindent
\begin{equation}
p_i^{(m+1)}=\fr{1}{n}\sum\limits_{j=1}^n g_{ij}^{(m)}\,, \enskip
i=1,\ldots,K\,.\label{e3-kor}
\end{equation}
Обозначим $\overline{x}=(1/n)\sum\limits_{j=1}^nx_j$. Используя
соотношение~(5.3.24) в~\cite{k2011}, с~учетом очевидного равенства
$\sum\limits_{i=1}^K g_{ij}^{(m)}\hm=1$ можно заметить, что уточненная
оценка параметра~$\alpha$ имеет вид:

\columnbreak

\noindent
\begin{equation}
\alpha^{(m+1)}=\fr{\overline{x}}{\sum\limits_{i=1}^K u_ip_i^{(m+1)}}\,,
\label{e4-kor}
\end{equation}
т.\,е.\ равна отношению генерального выборочного среднего и~текущего
эмпирического среднего смешивающего распределения, что вполне
согласуется с~тем, что в~соответствии с~приводимым ниже соотношением~(\ref{e5-kor})
в~данном случае ${\sf E}X\hm=\alpha{\sf E}U$.

В силу монотонности классического ЕМ-ал\-го\-рит\-ма справедливо следующее
утверждение.

\smallskip

\noindent
\textbf{Теорема~1.} {\it Пусть узлы $u_1,\ldots,u_K$ сетки различны,
неотрицательны и~известны. Тогда итерационный процесс $(3)$--$(4)$
является монотонным, т.\,е.\ каждая его итерация не уменьшает
целевую сеточную функцию правдоподобия}
\begin{multline*}
L(p_1,\ldots,p_K,\alpha;x_1,\ldots,x_n)={}\\
{}=
\prod\nolimits_{j=1}^n\left[\sum\nolimits_{i=1}^K
\fr{p_i}{\sqrt{u_i}}\,\varphi\left(\fr{x_j-\alpha^{(m)}u_i}{\sqrt{u_i}}\right)\right].
\end{multline*}

\smallskip

\noindent
\textbf{Замечание~1.} В~разд.~5.7.4 книги~\cite{k2011} показано, что
при каждом фиксированном значении параметра~$\alpha$ сеточная
функция правдоподобия\linebreak
$L(p_1,\ldots,p_{K-1},\alpha;\,x_1,\ldots,x_n)$ вогнута по
аргументам $p_1,\ldots,p_{K-1}$. Поэтому на каждом шаге
итерационного процесса вместо соотношения~(3) можно\linebreak использо\-вать
любой более быстрый алгоритм максимизации функции
$L(p_1,\ldots,p_{K-1},\alpha^{(m)};\,x_1,\ldots$\linebreak $\ldots,x_n)$ по переменным
$p_1,\ldots,p_{K-1}$. Например, оценки весов $p_1,\ldots,p_K$ можно
искать методом условного градиента~\cite{k2011, kn2010}.

\smallskip

Таким образом, на первом этапе получаются оценки параметра~$\alpha$
и~весов всех узлов~$u_i$ конечной сетки, накинутой на носитель
смешивающего обобщенного обратного гауссовского распределения
$P_{\mathrm{GIG}}(z;\,\nu,\mu,\lambda)$.

На втором этапе остается применить ка\-кой-ли\-бо стандартный метод
подгонки обобщенного обратного гауссовского распределения
$P_{\mathrm{GIG}}(z;\,\nu,\mu,\lambda)$ к~эмпирическим данным типа
гистограммы $(u_1, p_1),\ldots, (u_K, p_K)$. Например, параметры~$\nu$,
$\mu$ и~$\lambda$ можно оценить, минимизируя соответствующую
статистику хи-квад\-рат. Или же, например, можно решить задачу
наименьших квад\-ратов:
\begin{multline*}
(\nu^*,\mu^*,\lambda^*)={}\\
{}=\arg\min\limits_{\nu,\mu,\lambda}\sum\limits_{i=1}^K
\left[p_i- \!\!\!\!\!
\int\limits_{(1/2)\left(u_{i-1}+u_i\right)}^{(1/2)(u_i+u_{i+1})}\!\!\!\!\!\!\!\!\!\!\!\!\!\!\!
p_{GIG}(u;\,\nu,\mu,\lambda)\,du\right]^2,
\end{multline*}
где $u_0=0$; $u_{K+1}\hm=\infty$.

На практике хорошие результаты показал подход с решением задачи
наименьших квадратов. Для поиска параметров использовался алгоритм
ns2sol, описанный в~книге~\cite{DSch1983}. Указанный алгоритм
доступен во многих статистических пакетах, отличается высоким
быстродействием и~возможностью при желании задавать разумные
интервалы для поиска параметров.

%\vspace*{-9pt}

\section{О практическом выборе сетки
на~первом этапе моди\-фи\-ци\-ро\-ван\-но\-го
сеточного метода разделения дисперсионно-сдвиговых смесей нормальных
законов}

Естественно, что при использовании указанного двухэтапного метода
в~динамическом режиме крайне важным становится вопрос о~выборе
наиболее эффективных и~быстродействующих численных процедур и~их
параметров. В~частности, исключительную важность приобретает
правильный выбор сетки на первом этапе. Рассмотрим этот вопрос
подробнее.

Формально рассматриваемая задача выглядит так: по наблюдаемым
значениям $x_1,\ldots,x_n$ требуется построить статистическую оценку
верхней границы квантилей заданного порядка сме\-ши\-ва\-юще\-го закона так,
чтобы как можно точнее оценить носитель смешивающего распределения.

В дальнейшем будем считать, что $x_1,\ldots,x_n$~--- независимые
реализации случайной величины $X\hm=Y\sqrt{U}+\alpha U$, где $Y$~---
случайная величина со стандартным нормальным распределением, а~$U$~---
независимая от нее случайная величина с~обобщенным обратным
гауссовским распределением. Тогда, очевидно, распределение случайной
величины~$X$ имеет вид~(1). Предположим, что у~случайной величины~$U$
существуют моменты первых двух порядков. Тогда, как несложно видеть,
\begin{equation}
{\sf E}X={\sf E}Y\cdot{\sf E}\sqrt{U}+\alpha{\sf E}U=\alpha{\sf
E}U\,.\label{e5-kor}
\end{equation}
При этом по усиленному закону больших чисел с~вероятностью единица
$\overline x\hm\longrightarrow {\sf E}X$ $(n\hm\to\infty)$, так что при
больших~$n$ справедливо приближенное равенство ${\sf E}X\hm\approx\overline x$
и~с учетом~(\ref{e5-kor})
\begin{equation}
{\sf E}U\approx\fr{\overline x}{\alpha}\,.\label{e6-kor}
\end{equation}
Далее, очевидно,

\columnbreak

\noindent
\begin{multline}
{\sf E}X^2={\sf E}Y^2\cdot{\sf E}U+2\alpha{\sf E}X\cdot{\sf E}U^{3/2}+{}\\
{}+
\alpha^2{\sf E}U^2={\sf E}U+\alpha^2{\sf E}U^2\,.
\label{e7-kor}
\end{multline}

\noindent
Поэтому, обозначив
$$
m^2=\fr{1}{n}\sum\limits_{i=1}^nx_i^2\,,
$$
получаем приближенное равенство ${\sf E}X^2\hm\approx m^2$, так что
с~учетом~(\ref{e6-kor}) и~(\ref{e7-kor}) имеем:
\begin{equation}
{\sf E}U^2\approx\fr{1}{\alpha^2}\left(m^2-\fr{\overline
x}{\alpha}\right)\,.\label{e8-kor}
\end{equation}
Если параметр~$\alpha$ известен, то для определения верхней границы~$u^*$
сетки, накидываемой на носитель распределения случайной
величины~$U$, можно задать малое положительное число~$\varepsilon$
и~воспользоваться требованием
\begin{equation}
{\sf P}(U\geqslant u^*)\leqslant\varepsilon\,.\label{e9-kor}
\end{equation}
А~для гарантированного выполнения требования~(\ref{e9-kor}) можно использовать
неравенство Маркова:
$$
{\sf P}(U\geqslant u^*)\leqslant\fr{{\sf E}U^2}{(u^*)^2}\leqslant \varepsilon\,,
$$
откуда с учетом~(\ref{e8-kor})
$$
(u^*)^2\geqslant\fr{{\sf E}U^2}{\varepsilon}\approx
\fr{1}{\alpha^2\varepsilon}\left( m^2-\fr{\overline x}{\alpha}\right)
$$
или
\begin{equation}
u^*\approx\fr{1}{\alpha\sqrt{\varepsilon}}\sqrt{m^2-
\fr{\overline x}{\alpha}}\,.\label{e10-kor}
\end{equation}

\begin{figure*}[b] %fig1
\vspace*{1pt}
 \begin{center}
 \mbox{%
 \epsfxsize=161.718mm
 \epsfbox{kor-1.eps}
 }
 \end{center}
 \vspace*{-9pt}
\Caption{Примеры применения модифицированного двухэтапного сеточного
ЕМ-ал\-го\-рит\-ма для подгонки обобщенного гиперболического распределения
к искусственным данным, $\beta\hm=0$: (\textit{a})~$n\hm=1000$, $\alpha\hm=0{,}3$,
$\nu\hm=1{,}3$, $\mu\hm=1{,}6$, $\lambda\hm=0{,}2$;
(\textit{б})~$n\hm=1000$, $\alpha\hm=0{,}5$, $\nu\hm=1$, $\mu\hm=1$,
$\lambda\hm=3$;
(\textit{в})~$n\hm=1000$, $\alpha\hm=3$,
 $\nu\hm=1{,}3$, $\mu\hm=1{,}6$, $\lambda\hm=2$;
(\textit{г})~$n\hm=10\,000$,
$\alpha\hm=0{,}3$, $\nu\hm=1{,}3$, $\mu\hm=1{,}6$, $\lambda\hm=0{,}2$}
\end{figure*}


Если же параметр~$\alpha$, определяющий асим\-мет\-рию распределения
случайной величины~$X$, неизвестен, то можно воспользоваться
следующими рассуждениями. Обозначим
$$
q_n=\fr{1}{n}\sum\limits_{i=1}^n{\bf 1}(x_i<0)\,,
$$
где ${\bf 1}(A)$~--- индикаторная функция множества (события)~$A$.
При этом по усиленному закону больших чисел с~вероятностью единица
$q_n\hm\longrightarrow {\sf P}(X\hm<0)$ $(n\hm\to\infty)$, так что при
больших~$n$ справедливо приближенное равенство
\begin{equation}
q_n\approx{\sf P}(X<0)\,.\label{e11-kor}
\end{equation}
Но
\begin{multline}
{\sf P}(X<0)=\int\limits_{0}^{\infty}\Phi
\left(-\alpha\sqrt{u}\right) p_{\mathrm{GIG}}(u;\nu,\mu,\lambda)\,du={}\\
{}=
{\sf E}\Phi\left(-\alpha\sqrt{U}\right)\,.\label{e12-kor}
\end{multline}

\pagebreak

\noindent
Предположим сначала, что $q_n\hm<1/2$. Если~$n$ достаточно велико,
то можно с~большой степенью
 уверенности утверж\-дать, что тогда
$\overline x\hm>0$ и~$-\alpha\hm<0$, т.\,е.
 $\alpha\hm>0$ и,~стало быть, на
положительной полуоси значений аргумента~$u$ функция $\Phi(\alpha u)$
вогнута, т.\,е.\ выпукла вверх. Тогда из~(\ref{e11-kor}) и~(\ref{e12-kor}), дважды
применяя неравенство Иенсена, в~силу монотонности функции~$\Phi$
получаем:
\begin{multline}
1-q_n\approx 1-{\sf E}\Phi\left(-\alpha\sqrt{U}\right)=
          {\sf E}\Phi\left(\alpha\sqrt{U}\right)\leqslant{}\\
          {}\leqslant\Phi
          \left(\alpha{\sf E}\sqrt{U}\right)\leqslant
          \Phi\left(\alpha\sqrt{{\sf E}U}\right)\,.\label{e13-kor}
\end{multline}
Если теперь для $t\hm\in(0,1)$ символом~$v_t$ обозначить $t$-кван\-тиль
стандартного нормального закона, то из~(\ref{e13-kor}) и~(\ref{e6-kor}) вытекает
<<приближенное неравенство>>
$$
v_{1-q_n}\hm\leqslant \alpha\sqrt{{\sf E}U}\,,
$$
т.\,е.
$$
\alpha\geqslant\fr{v_{1-q_n}}{\sqrt{{\sf E}U}}\approx
\fr{v_{1-q_n}\sqrt{\alpha}}{\sqrt{\overline x}}\,,
$$
откуда получаем, что при достаточно больших~$n$
\begin{equation}
\alpha\geqslant\fr{v_{1-q_n}^2}{\overline x}\,.\label{e14-kor}
\end{equation}
Если теперь задать малое положительное число~$\varepsilon$, то
для определения верхней границы~$u^*$ сетки, накидываемой на
носитель распределения случайной величины~$U$, можно воспользоваться
требованием~(\ref{e9-kor}), для гарантированного выполнения которого
с~учетом~(\ref{e6-kor}) и~(\ref{e14-kor}) можно использовать неравенство Маркова:
$$
{\sf P}(U\geqslant u^*)\leqslant \fr{{\sf E}U}{u^*}\approx\fr{\overline
x}{\alpha u^*}\leqslant \fr{(\overline x)^2}{v_{1-q_n}^2 u^*}\leqslant
\varepsilon\,,
$$
откуда окончательно вытекает оценка
\begin{equation}
u^*\approx\fr{(\overline x)^2}{v_{1-q_n}^2 \varepsilon}\,.\label{e15-kor}
\end{equation}

\begin{figure*}[b] %fig2
\vspace*{18pt}
 \begin{center}
 \mbox{%
 \epsfxsize=162.433mm
 \epsfbox{kor-3.eps}
 }
 \end{center}
 \vspace*{-9pt}
\Caption{Примеры применения модифицированного двухэтапного
сеточного ЕМ-ал\-го\-рит\-ма для подгонки обобщенного гиперболического
распределения к~искусственным данным, $n=10\,000$, $\beta\hm=0$:
(\textit{а})~$\alpha\hm=0{,}3$,
$\nu\hm=2$, $\mu\hm=2$, $\lambda\hm=2{,}5$;
(\textit{б})~$\alpha\hm=0{,}5$,  $\nu\hm=1$, $\mu\hm=1$, $\lambda\hm=3$;
(\textit{в})~$\alpha\hm=0{,}8$,
$\nu\hm=1{,}3$, $\mu\hm=1{,}6$, $\lambda\hm=2$;
(\textit{г})~$\alpha\hm=1{,}3$, $\nu\hm=2$, $\mu\hm=2$, $\lambda\hm=2{,}5$}
\end{figure*}



В случае $q_n\hm\geqslant1/2$, если $n$ достаточно велико, то можно
с~большой степенью уверенности утверж\-дать, что $\overline x\hm\leqslant 0$
и~$-\alpha\hm\geqslant 0$, т.\,е.\ на положительной\linebreak\vspace*{-12pt}

\pagebreak

%\end{multicols}


%\begin{multicols}{2}

\noindent
 полуоси значений аргумента~$u$
функция $\Phi(-\alpha u)$ вогнута, т.\,е.\ выпукла вверх. Тогда
из~(\ref{e11-kor}) и~(\ref{e12-kor}), дважды применяя неравенство Иенсена, в~силу
монотонности функции~$\Phi$ получаем
$$
q_n\approx {\sf E}\Phi\left(-\alpha\sqrt{U}\right)\leqslant
\Phi\left(-\alpha\sqrt{{\sf E}U}\right)\,,
$$
откуда вытекает <<приближенное неравенство>> $v_{q_n}\hm \leqslant
-\alpha\sqrt{{\sf E}U}$,
т.\,е.
$$
-\alpha\geqslant\fr{v_{q_n}}{\sqrt{{\sf E}U}}\approx
\fr{v_{q_n}\sqrt{|\alpha|}}{\sqrt{|\overline x|}}
$$
и при достаточно больших~$n$
\begin{equation}
|\alpha|\geqslant\fr{v_{q_n}^2}{|\overline x|}\,.\label{e16-kor}
\end{equation}
Для определения верхней границы~$u^*$ сетки, накидываемой на
носитель распределения случайной величины~$U$, снова зададим малое
положительное число~$\varepsilon$ и~потребуем, чтобы было
справедливо условие~(\ref{e9-kor}), для гарантированного выполнения которого
с~учетом~(\ref{e6-kor}) и~(\ref{e16-kor}) используем неравенство Маркова и~тот факт, что
$\mathrm{sign}\, \overline x\hm=\mathrm{sign}\,\alpha$ при достаточно
больших~$n$:
\begin{multline}
{\sf P}(U\geqslant u^*)\leqslant \fr{{\sf E}U}{u^*}\approx
\fr{\overline x}{\alpha u^*}=
\fr{|\overline x|}{|\alpha| u^*} \leqslant{}\\
{}\leqslant
\fr{(\overline x)^2}{v_{q_n}^2 u^*}\leqslant
\varepsilon\,.\label{e17-kor}
\end{multline}
В силу симметричности нормального распределения $v_{t}\hm=-v_{1-t}$ для
любого $t\hm\in(0,1)$, поэтому $v_{q_n}^2\hm=v_{1-q_n}^2$ и~в~случае
$q_n\hm\geqslant1/2$ соотношение~(\ref{e17-kor}) снова приводит к~оценке~(\ref{e15-kor}).

Справедливости ради необходимо отметить, что оценки~(\ref{e10-kor}) и~(\ref{e15-kor})
являются завышенными, но они гарантируют, что
$(1-\varepsilon)$-почти-весь носитель распределения случайной
величины~$U$ будет лежать внутри интервала $[0, u^*]$.

\section{Результаты численных экспериментов}

Приводимые в~данном разделе графики иллюстрируют качество работы
модифицированного сеточного метода разделения дис\-пер\-си\-он\-но-сдви\-го\-вых
смесей нормальных законов на примере его\linebreak применения к~оцениванию
параметров обоб\-щенных гиперболических распределений с~ис\-поль\-зованием
указанного алгоритма выбора сетки\linebreak с~умеренным чис\-лом узлов $K\hm=40$.
Для вы\-чис\-ле\-ний использовались искусственно сгенерированные выборки
объемов $n\hm=1000$ и~$n\hm=10\,000$ с~разными наборами параметров, значения
которых указаны на рисунках. На рис.~1 и~2 изображены гистограммы
(серые столбики) и~графики
истинной плот\-ности (штриховые линии), промежуточной
оценки, полученной сеточным ЕМ-ал\-го\-рит\-мом (пунктирные линии)
и~итоговой оценки (непрерывные линии). На рис.~1 и~2 так\-же указаны
значения полученных оценок параметров. Как видно из приводимых
рисунков, параметры~$\alpha$ оцениваются очень точно. Точность
оценок остальных параметров удовлетворительная и~может быть повышена
за счет использования более частых сеток и~более чувствительных
критериев остановки ЕМ-ал\-го\-рит\-ма на первом этапе. Следует отметить,
что даже в~тех случаях, в~которых наблюдаются заметные расхождения
оценок параметров и~их точных значений, оценки самих плотностей
довольно \mbox{точны}.




{\small\frenchspacing
 {%\baselineskip=10.8pt
 \addcontentsline{toc}{section}{References}
 \begin{thebibliography}{99}
\bibitem{k2011}
\Au{Королев В.\,Ю.} Ве\-ро\-ят\-но\-ст\-но-ста\-ти\-сти\-че\-ские методы
декомпозиции волатильности хаотических процессов.~--- М.: Изд-во
Московского ун-та, 2011.

\bibitem{n2013}
\Au{Назаров А.\,Л.} Приближенные методы разделения смесей
вероятностных распределений: Дисс.\ \ldots\  канд. физ.-мат. наук.~--- М.:
МГУ им.\ М.\,В.~Ломоносова, 2013.

\bibitem{BN1977}
\Au{Barndorff-Nielsen~O.-E.} Exponentially decreasing distributions
for the logarithm of particle size~// Proc. Roy. Soc. Lond.~A,
1977. Vol.~353. P.~401--419.

\bibitem{BN1978}
\Au{Barndorff-Nielsen~O.-E.} Hyperbolic distributions and
distributions of hyperbolae~// Scand. J. Statist., 1978. Vol.~5.
P.~151--157.

\bibitem{BN1982}
\Au{Barndorff-Nielsen~O.-E., Kent~J., S\!{\!\ptb{\!\o}}\,rensen~M.} Normal
variance-mean mixtures and $z$-distributions~// Int. Statist. Rev.,
1982. Vol.~50. No.\,2. P.~145--159.

\bibitem{ks2012}
\Aue{Королев В.\,Ю., Соколов И.\,А.} Скошенные распределения
Стьюдента, дисперсионные гам\-ма-рас\-пре\-де\-ле\-ния и~их обобщения как
асимптотические аппроксимации~// Информатика и~её применения, 2012.
Т.~6. Вып.~1. С.~2--10.

\bibitem{zk2013}
\Au{Закс Л.\,М., Королев В.\,Ю.} Обобщенные дисперсионные
гам\-ма-рас\-пре\-де\-ле\-ния как предельные для случайных сумм~// Информатика
и её применения, 2013. Т.~7. Вып.~1. С.~105--115.

\bibitem{k2013}
\Au{Королев В.\,Ю.} Обобщенные гиперболические
распределения как предельные для случайных сумм~// Тео\-рия
вероятностей и~ее применения, 2013. Т.~58. Вып.~1. С.~117--132.

\bibitem{kckg2013}
\Au{Королев В.\,Ю., Черток А.\,В., Корчагин~А.\,Ю.,
Горшенин~А.\,К.} Ве\-ро\-ят\-но\-ст\-но-ста\-ти\-сти\-че\-ское моделирование
информационных потоков в~сложных финансовых системах на основе
высокочастотных данных~// Информатика и~её применения, 2013. Т.~7.
Вып.~1. С.~12--21.

\bibitem{p2004}
\Au{Protassov R.\,S.} EM-based maximum likelihood parameter
estimation for a~multivariate generalized hyperbolic distribution
with fixed~$\lambda$~// Statistics Computing, 2004. Vol.~14.
P.~67--77.

\bibitem{kn2010}
\Au{Королев В.\,Ю., Назаров А.\,Л.} Разделение смесей
вероятностных распределений при помощи сеточных методов моментов и~максимального правдоподобия~//
Автоматика и~телемеханика, 2010. Вып.~3. С.~98--116.

\bibitem{DSch1983}
\Au{Dennis J.\,E., Schnabel R.\,B.} Numerical methods for
unconstrained optimization and nonlinear equations.~--- Englewood
Cliffs: Prentice-Hall, 1983. 378~p.
 \end{thebibliography}

 }
 }

\end{multicols}

\vspace*{-6pt}

\hfill{\small\textit{Поступила в редакцию 01.10.14}}

\newpage

%\vspace*{12pt}

%\hrule

%\vspace*{2pt}

%\hrule

%\vspace*{12pt}

\def\tit{A MODIFIED GRID METHOD FOR~STATISTICAL SEPARATION
OF~NORMAL VARIANCE-MEAN MIXTURES}

\def\titkol{A modified grid method for statistical separation
of~normal variance-mean mixtures}

\def\aut{V.\,Yu.~Korolev$^{1,2}$ and~A.\,Yu.~Korchagin$^1$}

\def\autkol{V.\,Yu.~Korolev and~A.\,Yu.~Korchagin}

\titel{\tit}{\aut}{\autkol}{\titkol}

\vspace*{-9pt}


\noindent
$^1$Faculty of Computational Mathematics and Cybernetics,
M.\,V.~Lomonosov Moscow State University,\linebreak
$\hphantom{^1}$1-52 Leninskiye Gory, GSP-1, Moscow 119991, Russian Federation


\noindent
$^2$Institute of Informatics Problems, Russian Academy of Sciences,
44-2~Vavilov Str., Moscow 119333, Russian\linebreak
$\hphantom{^1}$Federation

\def\leftfootline{\small{\textbf{\thepage}
\hfill INFORMATIKA I EE PRIMENENIYA~--- INFORMATICS AND
APPLICATIONS\ \ \ 2014\ \ \ volume~8\ \ \ issue\ 4}
}%
 \def\rightfootline{\small{INFORMATIKA I EE PRIMENENIYA~---
INFORMATICS AND APPLICATIONS\ \ \ 2014\ \ \ volume~8\ \ \ issue\ 4
\hfill \textbf{\thepage}}}

\vspace*{3pt}

\Abste{A~modified two-stage grid method for
statistical separation of normal variance-mean mixtures is described
as an alternative to a pure EM (expectation-maximization) algorithm.
At the first stage of this
algorithm, a~discrete approximation is constructed to the mixing
distribution. At the second stage, the obtained discrete
distribution is approximated by an absolutely continuous
distribution from a~predetermined family, say, by a generalized
inverse Gaussian distribution. The convergence of this two-stage
procedure is discussed. The monotonicity of the grid procedure used
at the first stage is proved. The problem of the optimal choice of
the parameters of the method is discussed in detail. First of all,
the problem of the optimal choice of the grid thrown on the support
of the mixing distribution is considered. Statistical estimators are
proposed for the quantiles of the mixing law. The efficiency of the
method is illustrated by examples of its application to the
estimation of the parameters of generalized hyperbolic
distributions.}

\smallskip

\KWE{mixture of probability distributions; normal
variance-mean mixture; generalized hyperbolic distribution;
EM-algorithm; grid method of separation of mixtures}

\DOI{10.14357/19922264140402}

\Ack
\noindent
The research was supported by the Russian Science Foundation (project 14-11-00364).

%\vspace*{3pt}

  \begin{multicols}{2}

\renewcommand{\bibname}{\protect\rmfamily References}
%\renewcommand{\bibname}{\large\protect\rm References}



{\small\frenchspacing
 {%\baselineskip=10.8pt
 \addcontentsline{toc}{section}{References}
 \begin{thebibliography}{99}
 \bibitem{k2011eng}
 \Aue{Korolev, V.\,Yu.} 2011.
\textit{Veroyatnostno-statisticheskie metody dekompozitsii
volatil'nosti khaoticheskikh protsessov}
[Probabilistic and statistical methods for the decomposition of volatility
of chaotic processes].
Moscow: Moscow University Press. 510~p.

\bibitem{n2013eng}
\Aue{Nazarov, A.\,L.} 2013.
{Priblizhennye metody razdeleniya smesey veroyatnostnykh raspredeleniy}
[Approximate methods for the decomposition of volatility of chaotic processes].
Ph.D. Thesis. Moscow: Moscow State University.

\bibitem{BN1977eng}
\Aue{Barndorff-Nielsen, O.\,E.} 1977.
Exponentially decreasing distributions for the logarithm of particle size.
\textit{Proc. Roy. Soc. Lond. A} 353:401--419.

\bibitem{BN1978eng}
\Aue{Barndorff-Nielsen, O.\,E.} 1978.
Hyperbolic distributions and distributions of hyperbolae.
\textit{Scand. J. Statist.} 5:151--157.

\bibitem{BN1982eng}
\Aue{Barndorff-Nielsen, O.\,E., J.~Kent, and M.~S\!{\ptb{\o}}rensen}. 1982.
Normal variance-mean mixtures and $z$-distributions.
\textit{Int. Statist. Rev.} 50(2):145--159.

\bibitem{ks2012eng}
\Aue{Korolev, V.\,Yu., and I.\,A. Sokolov}. 2012.
{Skoshennye raspredeleniya St'yudenta, dispersionnye
gam\-ma-ras\-pre\-de\-le\-niya i~ikh obobshcheniya kak asimptoticheskie
approksimatsii}
[Skewed Student's distributions, variance gamma distributions, and their
generalizations as asymptotic approximations].
\textit{Informatika i ee Primeneniya}~--- \textit{Inform. Appl.} 6(1):2--10.

\bibitem{zk2013eng}
\Aue{Korolev, V.\,Yu., and L.\,M.~Zaks}. 2013.
{Obobshchennye dispersionnye gam\-ma-ras\-pre\-de\-le\-niya kak
predel'nye dlya sluchaynykh summ}
[Generalized variance gamma distributions as limiting for random sums].
\textit{Informatika i ee Primeneniya}~--- \textit{Inform. Appl.} 7(1):105--115.

\bibitem{k2013eng} \Aue{Korolev, V.\,Yu.} 2013.
{Obobshchennye giperbolicheskie raspredeleniya kak predel'nye dlya sluchaynykh summ}
[Generalized hyperbolic distributions as limiting for random sums]
\textit{Theory Probab. Appl.} 58(1):117--132.

\bibitem{kckg2013eng}
\Aue{Korolev, V.\,Yu., A.\,V. Chertok, A.\,Yu.~Korchagin, and A.\,K.~Gorshenin}.
2013. {Ve\-ro\-yat\-no\-st\-no-sta\-ti\-sti\-che\-skoe
mo\-de\-li\-ro\-va\-nie informatsionnykh potokov v~slozhnykh finansovykh sistemakh
na osnove vysokochastotnykh dannykh}
[Probability and statistical modeling of information flows in complex
financial systems from high-frequency data].
\textit{Informatika i~ee Primeneniya}~--- \textit{Inform.  Appl.} 7(1):12--21.

\bibitem{p2004eng-1}
\Aue{Protassov, R.\,S.} 2004.
EM-based maximum likelihood parameter estimation for a multivariate
generalized hyperbolic distribution with fixed~$\lambda$.
\textit{Statistics Computing} 14:67--77.

\bibitem{kn2010eng-1}
\Aue{Korolev, V.\,Yu., and A.\,L.~Nazarov}. 2010.
{Razdelenie smesey veroyatnostnykh raspredeleniy pri pomoshchi
setochnykh metodov momentov i~maksimal'nogo pravdopodobiya}
[Separation of mixtures using grid moment-based methods and maximum likelihood].
\textit{Avtomatika i~Telemekhanika} [Automatics and Telemechanics] 3:98--116.

\bibitem{DSch1983eng}
\Aue{Dennis, J.\,E., and R.\,B.~Schnabel}. 1983.
\textit{Numerical methods for unconstrained optimization and nonlinear equations}.
Englewood Cliffs: Prentice-Hall. 378~p.


\end{thebibliography}

 }
 }

\end{multicols}

\vspace*{-6pt}

\hfill{\small\textit{Received October 01, 2014}}

\vspace*{-18pt}

\Contr

\noindent
\textbf{Korolev Victor Yu.} (b.\ 1954)~---
Doctor of Science in physics and mathematics, professor,
Department of Mathematical Statistics, Faculty of Computational Mathematics
and Cybernetics, M.\,V.~Lomonosov Moscow State University,
1-52 Leninskiye Gory, GSP-1, Moscow 119991, Russian Federation;
leading scientist, Institute of Informatics Problems,
Russian Academy of Sciences, 44-2~Vavilov Str., Moscow 119333, Russian
Federation; victoryukorolev@yandex.ru

\vspace*{3pt}

\noindent
\textbf{Korchagin Alexander Yu.} (b.\ 1989)~---
PhD student, Faculty of Computational Mathematics and Cybernetics,
M.\,V.~Lomonosov Moscow State University,
1-52 Leninskiye Gory, GSP-1, Moscow 119991, Russian Federation;
sasha.korchagin@gmail.com


\label{end\stat}

\renewcommand{\bibname}{\protect\rm Литература} %7
\def\stat{shestakov+vor}

\def\tit{АСИМПТОТИЧЕСКАЯ НОРМАЛЬНОСТЬ И~СИЛЬНАЯ СОСТОЯТЕЛЬНОСТЬ ОЦЕНКИ РИСКА ПРИ~ИСПОЛЬЗОВАНИИ FDR-ПОРОГА В УСЛОВИЯХ СЛАБОЙ ЗАВИСИМОСТИ}

\def\titkol{Асимптотическая нормальность и~сильная состоятельность оценки риска при~использовании FDR-порога} % в~условиях слабой зависимости}

\def\aut{М.\,О.~Воронцов$^1$, О.\,В.~Шестаков$^2$}

\def\autkol{М.\,О.~Воронцов, О.\,В.~Шестаков}

\titel{\tit}{\aut}{\autkol}{\titkol}

\index{Воронцов М.\,О.}
\index{Шестаков О.\,В.}
\index{Vorontsov M.\,O.}
\index{Shestakov O.\,V.}


%{\renewcommand{\thefootnote}{\fnsymbol{footnote}} \footnotetext[1]
%{Работа 
%выполнена при поддержке Программы развития МГУ, проект №\,23-Ш03-03. При анализе 
%данных использовалась инфраструктура Центра коллективного пользования 
%<<Высокопроизводительные вычисления и~большие данные>> 
%(ЦКП <<Информатика>>) ФИЦ ИУ РАН (г.~Москва)}}


\renewcommand{\thefootnote}{\arabic{footnote}}
\footnotetext[1]{Московский государственный университет 
имени~М.\,В.~Ломоносова, факультет вычислительной математики и~кибернетики;  
Московский центр фундаментальной и~прикладной математики, \mbox{m.vtsov@mail.ru}}
\footnotetext[2]{Московский государственный университет 
имени М.\,В.~Ломоносова, факультет вычислительной математики и~кибернетики; 
Федеральный исследовательский центр <<Информатика и~управление>> Российской 
академии наук; Московский центр фундаментальной и~прикладной математики, 
\mbox{oshestakov@cs.msu.ru}}


\vspace*{-12pt}





\Abst{Рассматривается подход к~решению задачи удаления шума в~большом массиве 
разреженных данных, основанный на методе контроля средней доли ложных отклонений 
гипотез (False Discovery Rate, FDR). Данный подход эквивалентен процедурам 
пороговой обработки, обнуляющим компоненты массива, значения которых не 
превосходят некоторого заданного порога.  Наблюдения в~модели считаются слабо 
зависимыми. Для контроля степени зависимости используются ограничения на 
коэффициент сильного перемешивания и~максимальный коэффициент корреляции. 
В~качестве меры эффективности рассматриваемого подхода используется 
среднеквадратичный риск. Вычислить значение риска можно только на тестовых 
данных, поэтому в~работе рассматривается его статистическая оценка и~исследуются 
ее свойства. Показана асимптотическая нормальность и~сильная состоятельность 
оценки риска при использовании FDR-по\-ро\-га в~условиях слабой зависимости в~данных.}

\KW{пороговая обработка; множественная проверка гипотез; 
оценка риска}

\DOI{10.14357/19922264240309}{ZOQVTO}
  
%\vspace*{-6pt}


\vskip 10pt plus 9pt minus 6pt

\thispagestyle{headings}

\begin{multicols}{2}

\label{st\stat}



\section{Введение}

Во многих прикладных областях возникает задача обработки больших массивов 
зашумленных данных. Примерами служат задачи обработки изоб\-ра\-же\-ний с~высоким 
разрешением~\cite{FDRImage}, задачи множественной проверки гипотез, возникающие 
в~\mbox{исследованиях} в~об\-ласти генетики~\cite{MultipleTesting}, и~другие проб\-ле\-мы. 
В~связи с~этим рас\-смот\-рим модель
$$
x_i = \mu_i + z_i, \enskip i=\overline{1,n}\,,
$$
где $\mu_i\in\mathbb{R}$~--- <<полезные>> данные; $z_i \sim N(0,\sigma^2)$~--- 
шум. Задача заключается в~нахождении оценки неизвестного вектора $\mu \hm= 
(\mu_1,\ldots,\mu_n)$ как функции вектора $x \hm= (x_1,\ldots,x_n)$ и~может 
рассматриваться как задача множественной проверки гипотез о~равенстве нулю 
компонент вектора~$\mu$~\cite{AdaptingFDR}. При этом обычно предполагается, что 
вектор~$\mu$ имеет в~определенном смысле <<разреженную>> структуру, т.\,е.\ для 
<<полезных>> данных используется <<экономное>> представление.



В работе~\cite{AdaptingFDR} для решения рассматриваемой задачи в~условиях 
независимости компонент вектора~$x$ и~разреженности вектора~$\mu$ была 
предложена процедура построения оценки~$\hat{\mu}_F$ вектора~$\mu$, основанная 
на методе контроля средней доли ложных отклонений (FDR) 
гипотез при помощи алгоритма Бен\-жа\-ми\-ни--Хох\-бер\-га,
и~было проведено исследование асимптотики ее среднеквадратичного риска. 
В~работах~\cite{ZasShe17,Mathematics2020} была показана состоятельность 
и~асимптотическая нормальность оценки риска данной процедуры. Аналогичные 
результаты для других методов построения~$\hat{\mu}_F$ получены в~работах~\cite{Shestakov2021-1,Shestakov2021-2,Shestakov2022}.

В то же время в~определенных приложениях, например  при анализе полученных 
в~результате использования ДНК-мик\-ро\-чи\-пов данных~\cite{ResultsOnFDRUnderDependence}, исследовании геофизических процессов 
и~анализе помех\linebreak в~телекоммуникационных каналах, условие незави\-си\-мости компонент 
вектора $x$ может не выполняться. Ранее в~работах~\cite{VorontsovShestakov2023,Vorontsov2024} была \mbox{исследована} асимп\-то\-ти\-ка 
среднеквадратичного риска оценки~$\hat{\mu}_F$ \mbox{в~случае}, когда~$\mu$ принадлежит 
одному из классов разреженности
$$
l_0[\eta] = \left\{\mu\,:\, ||\mu||_0 \leq \eta n\right\}, \enskip \eta \in 
(0,1),
$$

\vspace*{-12pt}

\noindent
\begin{multline*}
m_p[\eta] \equiv{}\\
{}\equiv \left\{\mu \in \mathbb{R}^n : |\mu|_{(k)} \leq \eta n^{1/p} 
k^{-1/p},\ k=\overline{1,n}\right\}, \\
 p\in(0, 2),
\end{multline*}
а компоненты вектора~$x$ слабо зависимы~--- имеют достаточно быстро убывающий 
коэффициент сильного перемешивания~\cite{Bosq}

\noindent
\begin{multline*}
\alpha(k) = \sup\limits_{1\leq m\leq n}\alpha\left(\sigma(x_i, i\leq m), 
\sigma(x_i, i\geq m+k)\right), \\ 
k=\overline{1,n-1}\,,
\end{multline*}
где символом $\sigma(x_i, i\in I)$ обозначена сиг\-ма-ал\-геб\-ра, порожденная 
множеством случайных величин $\{x_i, i \hm\in I\}$, а~мера  $\alpha(\cdot, \cdot)$ 
близости двух сиг\-ма-ал\-гебр определяется как
$$
\alpha(\mathcal{B},\mathcal{C}) = \sup\limits_{B\in\mathcal{B}, 
C\in\mathcal{C}} \left|\p(BC)-\p(B)\p(C)\right|.
$$

В настоящей работе показана асимптотическая нормальность и~сильная 
состоятельность оценки риска при применении FDR-про\-це\-ду\-ры в~случае, когда 
компоненты вектора~$x$ слабо зависимы, а~$\mu$ принадлежит одному из классов 
раз\-ре\-жен\-ности: 
$l_0[\eta]$ или $m_p[\eta]$.


\section{Обработка вектора данных с~помощью FDR-процедуры}

Широким классом методов построения оценки~$\hat{\mu}$ стала пороговая обработка 
вектора~$x$ с~некоторым порогом~$T$. Различают жесткую пороговую обработку, при 
которой полагается
\begin{equation*}
\left(\hat{\mu}\right)_i  = p_H(x_i,T) \equiv
 \begin{cases}
   x_i, & |x_i| > T\,;\\
   0, & |x_i| \leq T\,,
 \end{cases}
\end{equation*}
и мягкую пороговую обработку, для которой
\begin{equation*}
(\hat{\mu})_i  = p_S(x_i,T) \equiv
 \begin{cases}
   x_i-T, & \hphantom{\vert\vert}x_i > T;\\
   x_i+T, & \hphantom{\vert\vert}x_i <- T;\\
   0, & |x_i| \leq T.
 \end{cases}
\end{equation*}
Среднеквадратичный риск подобных процедур определяется как
\begin{equation}
\label{riskDef}
R(T) = {\mathsf E} ||\hat{\mu}-\mu||^2 = \sum\limits_{i=1}^n {\mathsf E} \left((\hat{\mu})_i-
\mu_i\right)^2.
\end{equation}
Обозначим через~$T_m$ наилучшее значение порога:
$$
T_m : \, R(T_m) = \min\limits_{T} R(T).
$$

Предложенная в~\cite{AdaptingFDR} процедура заключается в~жесткой пороговой 
обработке компонент вектора~$x$ с~порогом $\hat{t}_F \hm= \hat{t}_F(x)$, и~ее 
результат~--- оценка $\hat{\mu}_F$ вектора~$\mu$ с~компонентами $(\hat{\mu}_F)_i  
\hm= p_H(x_i,\hat{t}_F)$, где
\begin{multline*}
\hat{t}_F = \sigma z\left(\fr{q \hat{k}_F}{2n}\right), \enskip
\hat{k}_F = \max 
\left\{k \, :\, |x|_{(k)} \geq t_k \right\}, \\
 t_k = \sigma z\left(\fr{q  k}{2n}\right);
\end{multline*}
$z(\alpha)$ --- квантиль уровня $(1\hm-\alpha)$ стандартного нормального 
распределения; $|x|_{(k)}$~--- $k$-й элемент вектора, получаемого в~результате 
упорядочения вектора~$|x|$ по невозрастанию:
$$
|x|_{(1)} \geq |x|_{(2)} \geq \cdots \geq |x|_{(n)};
$$
$q\in(0;1)$~--- управ\-ля\-ющий параметр FDR-ме\-то\-да.
Далее полагается, что $q\hm\equiv q_n$ зависит от~$n$. В~\cite{AdaptingFDR} 
показано, что эта процедура эквивалентна множественной проверке гипотез 
о~равенстве нулю компонент наблюдаемого вектора. Также показано, что с~помощью 
метода штрафных функций данную процедуру можно свести к~другим видам пороговой 
обработки, в~част\-ности к~мягкой пороговой обработке.

В работах~\cite{VorontsovShestakov2023, Vorontsov2024} была исследована 
асимптотика среднеквадратичного риска~$R(\hat{t}_F)$ описанной процедуры 
в~случае, когда компоненты вектора $x$ слабо зависимы, а $\mu$ принадлежит классу 
разреженности~$\Theta_n$, где~$\Theta_n$ есть~$l_0[\eta_n]$ или~$m_p[\eta_n]$. 
Было показано, что~$R(\hat{t}_F)$ асимптотически отличается от минимаксного 
риска
$\inf\nolimits_{\hat{\mu}\hm=\hat{\mu}(x)} \sup\nolimits_{\mu\in \Theta_n} {\mathsf E} 
||\hat{\mu}-\mu||^2$
на множитель не более чем логарифмического по\-рядка.

Отметим, что в~выражении для среднеквадратичного риска~(\ref{riskDef}) 
присутствуют неизвестные величины~$\mu_i$, а~потому вычислить~$R(T_m)$ и~$T_m$ 
не представляется возможным. На практике можно пользоваться, например, следующей 
оценкой среднеквадратичного риска~\cite{Mallat}:
$$
\hat{R}(T) = \sum\limits_{i=1}^n F[x_i, T],
$$
где  
\begin{multline*}
F[x_i, T] = {}\\[3pt]
{}=\!\begin{cases}
\left(x_i^2-\sigma^2\right) \Ik(|x_i|\leq T) + \sigma^2 \Ik\left(|x_i|>T\right) &\\[3pt]
&\hspace*{-53mm}\mbox{для\ жесткой\ пороговой\ обработки};\\[3pt]
\left(x_i^2-\sigma^2\right) \Ik\left(|x_i|\leq T\right) + (\sigma^2+T^2) 
\Ik \left(|x_i|>T\right) \hspace*{-11.21576pt}&\\[3pt]
&\hspace*{-51mm}\mbox{для\ мягкой\ пороговой\ обработки}.
\end{cases}\hspace*{-7.17859pt}
\end{multline*}


\noindent
\textbf{Замечание}.\ При пороговой обработке иногда также используется так 
называемый универсальный порог $T_U\hm = \sigma \sqrt{2\ln n}$, предложенный 
в~работе~\cite{spatialAdaptation}. Исследования в~\cite{AdaptingSURE, ExactRisk} 
показали, что порог~$T_U$ в~определенном смысле максимальный, и~рас\-смат\-ри\-вать 
пороги выше него не имеет смысла. Более того, нетрудно показать, что $t_k \hm< T_U$ 
для всех~$k$ и~всех достаточно больших~$n$, в~связи с~чем всюду далее полагаем, 
что порог~$\hat{t}_F$ выбирается на отрезке $[0; T_U]$.

\section{Вспомогательные утверждения}

Кроме коэффициента сильного перемешивания~$\alpha(\cdot)$ также понадобится 
следующее понятие~\cite{Bosq}.

\smallskip

\noindent
\textbf{Определение.} %\label{defRho}
Максимальным коэффициентом корреляции~$\rho(\cdot)$ компонент вектора~$x$ 
называется
\begin{multline*}
\rho (k) \equiv \rho_n (k) = {}\\
{}=\sup\limits_{1\leq m\leq n}\rho\left(\sigma(x_i, 
i\leq m), \sigma(x_i, i\geq m+k)\right), \\
 k=\overline{1,n-1}\,,
\end{multline*}
где мера $\rho(\cdot, \cdot)$ близости двух сиг\-ма-ал\-гебр определяется как
$$
\rho(\mathcal{B},\mathcal{C}) = \sup\limits_{\substack{\xi 
\in\mathcal{L}^2(\mathcal{B}) \\
 \eta \in\mathcal{L}^2(\mathcal{C})}} 
\left|\mathrm{corr}\,(\xi, \eta)\right|.
$$


Введем обозначения:
$$
T_1 = \sqrt{2\ln \eta_n^{-p}};  \,\gamma_n = \fr{1}{\ln\ln n}; \, \kappa_n 
= \fr{n \eta_n^p T_1^{-p}}{1 - q_n - \gamma_n}; 
$$
$$ 
\kappa_n^0 = \fr{[n \eta_n]}{1 - q_n - \gamma_n} ;\, \rho^\star (k) = 
\sup\limits_{n\geq k+1} \rho(k), k \in \mathbb{N} ;
$$
$$
t_{\kappa_n} = \sigma z\left(\fr{q_n \kappa_n }{2n}\right) , \,\, t_{\kappa_n^0} 
= \sigma z\left(\fr{q_n \kappa_n^0 }{2n}\right).
$$


Следующие два утверждения показывают, что случайный порог~$\hat{t}_F$ в~случае 
$\mu\hm\in m_p[\eta_n]$ (соответственно $\mu\hm\in l_0[\eta_n]$) с~большой 
вероятностью будет не меньше~$t_{\kappa_n}$ (соответственно~$ t_{\kappa_n^0}$). 
Их  доказательства приведены в~работах~\cite{VorontsovShestakov2023, Vorontsov2024}.

\smallskip

\noindent
%\begin{lem}\label{lem5}
\textbf{Лемма~1.}\ \textit{Пусть $n^{-\delta_1} \hm\leq \eta_n^p \hm\leq n^{-\delta_2}$, 
$0\hm<\delta_2\hm<\delta_1<1$, $\mathrm{lim\,inf} q_n \ln n \hm\geq C \hm> 0$, 
$m\hm\in[1;n/2]\cap\mathbb{N}$, а $\alpha(\cdot)$~--- коэффициент сильного 
перемешивания компонент вектора~$x$. Для некоторого $N\hm\in\mathbb{N}$ при $n \hm\geq 
N$ справедливо}
\begin{multline*}
\hspace*{-3pt}\sup\limits_{\mu\in m_p[\eta_n]} \p \left(\hat{k}_F \geq \kappa_n \right) \leq 
4 n \exp\left\{-\fr{m}{256n}  \kappa_n q_n \gamma_n^2    \right\}+{}\\
{}+ 22\left(1+\fr{8n}{\kappa_n q_n \gamma_n}\right)^{1/2} n m 
\alpha\left(\left[\fr{n}{2m}\right]\right).
\end{multline*}



\smallskip

\noindent
\textbf{Лемма 2.}\ 
%\label{lem1}
\textit{Пусть $\eta_n \hm\leq b\hm<1$, $m\in[1;n/2]\cap\mathbb{N}$, а~$\alpha(\cdot)$~--- 
коэффициент сильного перемешивания компонент вектора~$x$. Для некоторого 
$N\hm\in\mathbb{N}$ при $n \hm\geq N$ справедливо}
\begin{multline*}
\sup\limits_{\mu\in l_0[\eta_n]} \p \left(\hat{k}_F \geq \kappa_n^0 \right) 
\leq{}\\
{}\leq 4 n \exp\left\{-\fr{(1-b)m}{64n}\,  \kappa_n^0 q_n \gamma_n^2    
\right\}+{}\\
{}+ 22\left(1+\fr{4n}{(1-b)\kappa_n^0 q_n \gamma_n}\right)^{1/2} n m 
\alpha\left(\left[\fr{n}{2m}\right]\right).
\end{multline*}

Следующие два утверждения доказаны в~\cite{Bosq} и~представляют собой аналоги 
неравенств Хеффдинга и~Бернштейна для слабо зависимых случайных величин.


\smallskip

\noindent
\textbf{Лемма 3.}\
\textit{Пусть для набора действительных случайных величин $X_1, \ldots, X_n$ 
с~коэффициентом сильного перемешивания $\alpha(\cdot)$ выполняется ${\mathsf E} X_i \hm=0$, 
$|X_i|\hm\leq b$, $i\hm=\overline{1,n}$. Тогда для любого целого числа $m\hm\in[1; n/2]$ 
и~любого $\eps\hm>0$ справедливо}
\begin{multline*}
\p\left(\left|\sum\limits_{i=1}^n X_i\right| > n\eps \right) \leq 4 
\exp\left\{-\fr{\eps^2 m}{8 b^2}\right\}+ {}\\
{}+
22\left(1+\fr{4b}{\eps}\right)^{1/2} m\, 
\alpha\left(\left[\fr{n}{2m}\right]\right).
\end{multline*}


\smallskip

\noindent
\textbf{Лемма 4.}\
\textit{Пусть для набора действительных случайных величин $X_1, \ldots, X_k$ 
с~коэффициентом сильного перемешивания $\alpha(\cdot)$ выполняется ${\mathsf E} X_i \hm=0$, 
$|X_i|\hm\leq b$, $i\hm=\overline{1,k}$. Тогда для любого целого числа $m\hm\in[1; k/2]$ 
и~любого $\eps\hm>0$ справедливо}
\begin{multline*}
\p\left(\left|\sum\limits_{i=1}^k X_i\right| > \eps \right) \leq 4 
\exp\left\{-\fr{\eps^2 m}{8 v^2 k^2}\right\}+{}\\
{}+ 22\left(1+\fr{4bk}{\eps}\right)^{1/2} m\, 
\alpha\left(\left[\fr{k}{2m}\right]\right),
\end{multline*}
\textit{где $p = k/(2m)$}:
\begin{multline*}
v^2 =
 \fr{b \eps}{2k} + {}\\
 {}+\fr{2}{p^2} \,  \max\limits_{ j\in[0,\,2m-1]} 
{\mathsf E} \big( ([jp]+1-jp)X_{[jp]+1} + X_{[jp]+2}+{}\\
{}+ \cdots +  X_{[(j+1)p]} + ((j+1)p-[(j+1)p])X_{[(j+1)p+1]}\big)^2.
\end{multline*}

\noindent
\textbf{Замечание.}
Если существует такое число $S \hm> 0$, что сразу для всех $i\hm\in[1;k]$  выполняется 
${\mathsf E} X_i^2 \hm\leq S^2$, то в~качестве~$v^2$ можно взять
$$
v^2 = \fr{b \eps}{2k} + 8 S^2.
$$


Д\,о\,к\,а\,з\,а\,т\,е\,л\,ь\,с\,т\,в\,о\ \ сле\-ду\-юще\-го утверж\-де\-ния приведено в~работе~\cite{AdaptingFDR}.

\smallskip

\noindent
\textbf{Лемма 5.}\ 
\textit{Для $y\leq 0{,}01$ справедливы представления}
\begin{multline}
\label{lem1eq1}
z^2(y) = 2 \ln y^{-1} - \ln \ln y^{-1} - r_2(y), \\
 r_2(y) \in [1{,}8; 3];
\end{multline}

\noindent
\begin{equation}
\label{lem1eq2}
z(y) = \sqrt{2 \ln y^{-1}} - r_1(y), \, \, r_1(y) \in [0; 1{,}5].
\end{equation}


\section{Асимптотическая нормальность оценки риска при~применении FDR-процедуры в~условиях слабой зависимости}

Перейдем к~описанию достаточных условий для асимптотической нормальности оценки 
риска $\hat{R}(\hat{t}_F)$ в~случае $\mu \hm\in m_p[\eta_n]$.

\smallskip

\noindent
\textbf{Теорема~1.}\
\textit{Пусть $\mu \hm\in m_p[\eta_n],$ $\eta_n^p \hm\in[n^{-\delta_1}; n^{-\delta_2}],$ $1/2 \hm< 
\delta_2 \hm< \delta_1<1;$ имеются такие константы $c_1, c_2>0$, что для 
коэффициента сильного перемешивания $\alpha(\cdot)$ компонент вектора $x$ 
справедливо  $\alpha(k) \hm\leq c_1 k^{-1-(5/2)\delta_1/(1-\delta_1)-c_2},$ 
$k\hm=\overline{1,n-1};$ $q_n \hm< c_3 \hm< 1;$ $\mathrm{lim\,inf} q_n \ln n \hm= c_4 \hm> 0;$ и,~кроме того, 
для максимального коэффициента корреляции $\rho(\cdot)$ компонент вектора~$x$ 
справедливо}
$$
\sum\limits_{k = 1}^{\infty} \sup\limits_{n\geq k+1} \rho(k) \equiv 
\sum\limits_{k = 1}^{\infty}  \rho^\star (k) = c_5 < \infty. 
$$
\textit{Тогда при $n \to \infty$}
$$
\fr{\hat{R}(\hat{t}_F) - R(T_m)}{C_\rho \sqrt{2n}} \Rightarrow N(0, 1),
$$
\textit{где}
$$
C_\rho = \sigma^2\sqrt{1 +  \lim\limits_{n\to\infty} \fr{1}{n} \sum\limits_{j\neq i} \mathrm{corr}^2 (x_i, x_j)}.
$$

\noindent
Д\,о\,к\,а\,з\,а\,т\,е\,л\,ь\,с\,т\,в\,о\  \
 приводится для метода мягкой пороговой обработки; в~случае жесткой пороговой 
обработки доказательство аналогично. Обозначим
$$
U(T) = \hat{R}(T) -  \hat{R}(T_m) = \sum \limits_{i=1}^n H_i(T, T_m),
$$
где
$$
H_i(T, T_m) = F[x_i, T] - F[x_i, T_m].
$$
Имеем

\vspace*{-3pt}

\noindent
\begin{multline}
\label{D00}
\hat{R}(\hat{t}_F) - R(T_m) + \hat{R}(T_m) - \hat{R}(T_m) ={}\\
{}= \hat{R}(T_m) - 
R(T_m) + U(\hat{t}_F).
\end{multline}
Покажем, что
\begin{equation}
\label{D0}
\fr{\hat{R}(T_m) - R(T_m)}{C_\rho\sqrt{2n}} \Rightarrow N(0, 1).
\end{equation}


Повторяя рассуждения из~\cite{KuShe2016_1,KuShe2016_2,Jansen}, можно показать, 
что $T_m \hm\geq t_{\kappa_n}$. Учитывая также $T_m\hm \leq T_U$, имеем 
$$
C \sqrt{\ln n} \leq T_m \leq C^\prime \sqrt{\ln n}
$$ 
для некоторых положительных констант $C$ и~$C^\prime$.

\columnbreak

В случае мягкой пороговой обработки $\hat{R}(T_m)$ представляет собой 
несмещенную оценку~$R(T_m)$, а~при жесткой пороговой обработке и~выполнении 
условий теоремы смещение стремится к~нулю при делении на $\sqrt{n}$~\cite{Mallat}.

Для дисперсии числителя~(\ref{D0}) имеем:
\begin{multline*}
{\mathsf D} \left(\hat{R}(T_m) - R(T_m)\right) = \sum\limits_{i=1}^n {\mathsf D} F[x_i, T_m] + {}\\
{}+
\sum\limits_{i=1}^n\sum\limits_{\substack{j=1 \\  j\neq i}}^n \mathrm{cov}\left( F[x_i, T_m], F[x_j, 
T_m] \right).
\end{multline*}

Поскольку $\mu \in m_p[\eta_n]$,
\begin{equation}
\left.
\begin{array}{l}
 \displaystyle\sum\limits_{i: |\mu_i| > 1/T_1} {\mathsf D} F[x_i, T_m]  \leq{}\\
 \hspace*{15mm}{}\leq  4\left(\sigma^2 + T_m^2\right)^2 n \eta_n^p 
T_1^p = o(n);
\\[6pt]
\displaystyle \sum\limits_{\substack{{i,j: \max\{|\mu_i|, |\mu_j|\} > 1/T_1,}\\{j\neq i}}}  \hspace*{-12mm}\mathrm{cov}\,(F[x_i, 
T_m],F[x_j, T_m])  \leq{}\\
\hspace*{10mm}{}\leq 16\left(\sigma^2 + T_m^2\right)^2 n \eta_n^p T_1^p c_5 = o(n). 
\end{array}
\right\}    
\label{D2}
\end{equation}
Далее, учитывая что ${\mathsf D} x_i^2 \hm= 2\sigma^4 \hm+ 4\sigma^2 \mu_i^2$, нетрудно 
убедиться, что
\begin{multline}
\label{D3}
\sum\limits_{i: |\mu_i| \leq 1/T_1}\hspace*{-4mm} {\mathsf D} F[x_i, T_m] ={}\\
{}= \sum\limits_{i: |\mu_i| \leq 1/T_1} \hspace*{-4mm} {\mathsf D} 
x_i^2 + o(n) = 2\sigma^4 n + o(n).
\end{multline}


Введем обозначение 
$$
D_n = \left\{(i,j) : \max\left\{|\mu_i|, |\mu_j|\right\}  \leq \fr{1}{T_1}\,, \enskip j\hm\neq i\right\}.
$$
 Для суммы ковариаций аналогично~(\ref{D3}) получим
\begin{multline*}
\sum\limits_{(i,j)\in D_n} \hspace*{-2mm}\mathrm{cov}\left( F[x_i, T_m], F[x_j, T_m] \right) = {}\\
{}=
\sum\limits_{(i,j)\in D_n} \hspace*{-2mm}\mathrm{cov}\left( x_i^2, x_j^2 \right) + o(n).
\end{multline*}
Воспользуемся тождеством~\cite{Eroshenko}
$$
\mathrm{cov}\left (x_i^2, x_j^2\right) = 4 {\mathsf E} x_i {\mathsf E} x_j \mathrm{cov}\left(x_i, x_j\right) + 2 \mathrm{cov}^2 \left(x_i, x_j\right)
$$
для вектора $(x_i, x_j)$, имеющего двумерное нормальное распределение. Заметим, 
что
\begin{gather*}
 \sum\limits_{(i,j)\in D_n} 4 | {\mathsf E} x_i {\mathsf E} x_j \mathrm{cov}\left(x_i, x_j\right)| \leq 8 T_1^{-2} 
\sigma^2 n c_5 = o(n);
\\
\sum\limits_{(i,j)\in D_n} 2 \mathrm{cov}^2 (x_i, x_j)  = 2\sigma^4 \sum\limits_{(i,j)\in D_n} 
\mathrm{corr}^2 (x_i, x_j). 
\end{gather*}
Более того, поскольку  %< 4 \sigma^2 n c_5.$$
\begin{equation*}
\sum\limits_{\substack{{i,j: \max\{|\mu_i|, |\mu_j|\} > 1/T_1} \\ {j\neq i}}}
\hspace*{-10mm}\mathrm{corr}^2 (x_i, x_j)  
\leq  4 n \eta_n^p T_1^p c_5 =  o(n),
\end{equation*}
имеем
\begin{multline*}
\sum\limits_{(i,j)\in D_n} \mathrm{corr}^2 (x_i, x_j) ={}\\
{}= \sum\limits_{j\neq i} \mathrm{corr}^2 (x_i, x_j) 
+o(n)= c_6 n + o(n),
\end{multline*}
где
$$
c_6 = \lim\limits_{n\to\infty} \fr{1}{n} \sum\limits_{j\neq i} \mathrm{corr}^2 (x_i, x_j) 
\leq 2 c_5.
$$
Полагая $C_\rho \hm= \sigma^2\sqrt{1 + c_6}$, получим, наконец,
\begin{equation}
\label{D1}
{\mathsf D} \left(\hat{R}(T_m) - R(T_m)\right)  =  2 n C_\rho^2 + o(n).
\end{equation}
Заметим, что из~(\ref{D2}), (\ref{D3}) и~(\ref{D1}) следует, что
\begin{equation}
\label{D5}
\sup\limits_{n} \fr{\sum\nolimits_{i=1}^n {\mathsf D} F[x_i, T_m]}{V_n^2} < \infty\,,
\end{equation}
где 
$$
V_n^2 = {\mathsf D} \sum\limits_{i=1}^n \left(F[x_i, T_m] \hm- {\mathsf E} F[x_i, T_m]\right).
$$
Кроме того, поскольку $F[x_i, T_m]$ по модулю ограничены величиной $\sigma^2 \hm+ 
T_m^2$, выполнено условие Линдеберга: для любого $\eps\hm>0$ при $n \hm\to \infty$
\begin{multline}
\label{D6}
\!\!\!\fr{1}{V_n^2}\sum\limits_{i=1}^n {\mathsf E} \left( \!\left( F\left[x_i, T_m\right]\! -\! {\mathsf E} F\left[x_i, T_m\right]\right)^2 
\Ik \left(\vert F\left[x_i, T_m\right] -{}\right.\right.\hspace*{-2.69505pt}\\
\left.\left.{}- {\mathsf E} F\left[x_i, T_m\right]\vert >\eps V_n\right)\!
\vphantom{\left( F\left[x_i, T_m\right]\! -\! {\mathsf E} F\left[x_i, T_m\right]\right)^2}
\right) 
\to  0\,.
\end{multline}
Из~(\ref{D1})--(\ref{D6}), очевидного неравенства
$$ 
\lim\limits_{k\to\infty} \sup\limits_{n\geq k+1}\rho(k) \equiv 
\lim\limits_{k\to\infty} \rho^\star (k)  < 1
$$
 и~центральной предельной теоремы для сильно перемешанных случайных величин~\cite{Peligrad} следует~(\ref{D0}).

Перейдем к~доказательству того, что $U(\hat{t}_F) \, n^{-1/2} \overset{\, \p \, }{\to} 0$.
Всюду далее, не ограничивая общности, полагаем $\sigma=1$. 
Введем обозначения:

\noindent
\begin{align*}
S_1(T) &= \sum\limits_{i: |\mu_i| > 1/T_1} H_i(T, T_m); \\
S_2(T) &= \sum\limits_{i: |\mu_i| \leq 1/T_1} H_i(T, T_m); 
\\
N_1(a, b) &= \sum\limits_{i: |\mu_i| > 1/T_1} \Ik (a<|x_i|\leq b); \\ 
N_2(a, b) &= \sum\limits_{i: |\mu_i| \leq 1/T_1} \Ik (a<|x_i|\leq b);
\end{align*}

\noindent
\begin{align*}
Z_l(T) &= S_l(T) - {\mathsf E} S_l(T),\enskip l = 1,2\,; \\  
d_n &= \fr{T_U -  t_{\kappa_n}}{n};\\
T_j^{\prime} &= t_{\kappa_n}+j d_n,\enskip j = \overline{0,n-1}\,.
\end{align*} 

\vspace*{-3pt}

\noindent
Для произвольного $\eps>0$

\vspace*{-3pt}

\noindent
\begin{multline}
\p \left( \fr{|U(\hat{t}_F)|}{\sqrt{n}}> 4\eps \right) \leq 
\p\left(\hat{t}_F \leq t_{\kappa_n}\right) + {}\\
{}+\p \left(\fr{\sup\nolimits_{T\in 
[t_{\kappa_n}, T_U]} |U(T)|}{\sqrt{n}}>4\eps \right)\leq  {}\\
{}\leq \p\left(\hat{t}_F \leq t_{\kappa_n}\right) + \p\left(\fr{\sup\nolimits_{T\in 
[t_{\kappa_n}, T_U]} |{\mathsf E} U(T)|}{\sqrt{n}}>\eps\right)+{}\\
{}+ \p \left(\sup\limits_{T\in [t_{\kappa_n}, T_U]} |Z_1(T)| > 
\eps\sqrt{n}\right) +{}\\
{}+ \p \left(\sup\limits_{j \in [0, n-1]} |Z_2(T_j^{\prime})| > 
\eps\sqrt{n}\right) +{}\\
{}+ \p \left(\sup\limits_{\substack{j \in [0, n-1] \\
 T\in [T_j^{\prime},T_j^{\prime}+d_n]}} |Z_2(T)-Z_2(T_j^{\prime})| > \eps\sqrt{n}\right).
\label{M1}
\end{multline}
Заметим, что $\gamma_n\hm > \ln^{-1} n$, $\kappa_n\hm > n \eta_n^p \ln ^{-1} n \hm\geq 
n^{1-\delta_1} \ln ^{-1} n$ и~$q_n\hm > c_4 \ln ^{-1} n /2$ для всех достаточно 
больших~$n$.
Для первого слагаемого в~(\ref{M1}) по лемме~1 с~$m \hm= n^{\delta_1} \ln 
^7 n$ для  больших~$n$ имеем

\vspace*{-3pt}

\noindent
\begin{multline}
\label{M1next}
\p\left(\hat{t}_F \leq t_{\kappa_n}\right)  = \p \left(\hat{k}_F \geq \kappa_n 
\right) \leq 4 n e^{-\ln^2 n} + {}\\
{}+n^{1+(3/2)\,\delta_1} \ln^9 n \, 
\alpha\left(\left[\fr{n^{1-\delta_1}}{\ln^{7} n}\right]\right) = o(1)
\end{multline}
при $n\to\infty$. 
Для оценки второго слагаемого в~(\ref{M1}) заметим, что при $T \hm\in 
[t_{\kappa_n}, T_U]$ справедливо
\begin{equation}
\label{M2}
{\mathsf E} H_i(T, T_m) \leq T_U^2 + 1.
\end{equation}
Если же кроме $T \hm\in [t_{\kappa_n}, T_U]$ также выполнено $|\mu_i| \hm\leq T_1^{-1}$, то

\vspace*{-6pt}

\noindent
\begin{multline*}
|{\mathsf E} H_i (T, T_m)| \leq 2 T_U^2 \, \p \left(|x_i| > t_{\kappa_n}\right) \leq {}\\
{}\leq2 
T_U^2 \, \p \left(|x_i-\mu_i| > t_{\kappa_n}-T_1^{-1}\right) \leq{}\\
{}\leq 2 T_U^2  \exp\left\{ -\fr{1}{2} \left(t_{\kappa_n} - T_1^{-
1}\right)^2 \right\}  \leq{}\\
{}\leq
 4 (\ln n)  \exp\left\{ -\fr{1}{2} 
\left(z\left(\fr{q_n\kappa_n}{2n}\right)\right)^2 + t_{\kappa_n} T_1^{-
1}\right\},
\end{multline*}

\vspace*{-2pt}

\noindent
где использовано неравенство 

\noindent
$$
2(1-\Phi(x))\hm \leq \fr{e^{-x^2/2}}{x}
$$

\pagebreak


\noindent
 для $x\hm\geq 0$ 
($\Phi(x)$~--- функция распределения $N(0,1)$). Рас\-смот\-рим выражение 
в~экспоненте. Второе слагаемое не превышает $1\hm+o(1)$ при $n\hm\to\infty$, поскольку 
$t_{\kappa_n} \hm\leq T_1 (1+o(1))$ при $\sigma\hm=1$, что нетрудно получить из 
определения~$t_{\kappa_n}$, пред\-став\-ле\-ния~(\ref{lem1eq2}) и~ограничения на~$q_n$ 
из формулировки тео\-ре\-мы. Для первого слагаемого, используя пред\-став\-ле\-ние~(\ref{lem1eq1}) 
и~ограничения, наложенные на~$q_n$, при больших~$n$ получим
\begin{multline*}
-\fr{1}{2}\left(z\left(\fr{q_n \kappa_n}{2n}\right)\right)^2 \leq - \ln 
\fr{2n (1-q_n-\gamma_n)}{q_n n \eta_n^p T_1^{-p}} + {}\\
{}+\fr{1}{2} \ln 
\left((1+o(1)) \ln \eta_n^{-p}\right) + \fr{3}{2} \leq{}\\
{}\leq \ln \fr{c_3}{1-c_3} + \ln \eta_n^p + \ln T_1^{-p} + \ln T_1 + 
\fr{3}{2}+ o(1).
\end{multline*}
Из приведенных соотношений следует, что с~некоторой константой $c_7 = c_7(c_3, 
p, \delta_1, \delta_2, c_4)$
\begin{equation}\label{M3}
\sup\limits_{\substack{i: |\mu_i| \leq 1/T_1 \\ T\in [t_{\kappa_n}, T_U]}} |{\mathsf E} 
H_i (T, T_m)|  \leq c_7 (\ln n)^{(3-p)/2}\eta_n^p.
\end{equation}
Из (\ref{M2}) и~(\ref{M3}) с~учетом $\delta_2 \hm> 1/2$ следует
\begin{multline*}
\sup\limits_{T\in [t_{\kappa_n}, T_U]} |{\mathsf E} U(T)| \leq{}\\
{}\leq 
 n\eta_n^p T_1^p 
(T_U^2+1) + c_7 (\ln n)^{(3-p)/2} n \eta_n^p = o(\sqrt{n})
\end{multline*}
при $n\to\infty$, а следовательно, для любого $\eps\hm>0$ второе слагаемое в~(\ref{M1}) обращается в~ноль для всех достаточно больших~$n$.

Далее, поскольку при $T \hm\leq T_U$ и~$\sigma\hm=1$
$$
|H_i(T, T_m) - {\mathsf E} H_i(T, T_m)| \leq 2 (T_U^2 +2), \enskip i=\overline{1, n}\,,
$$
а число слагаемых в~$Z_1(T)$ не превосходит $n\eta_n^p T_1^p$, имеем
$$
\sup\limits_{T\in [t_{\kappa_n}, T_U]} |Z_1(T)|  \leq 2 n\eta_n^p T_1^p (T_U^2 
+2) = o(\sqrt{n})
$$
при $n\to\infty$, а следовательно, для любого $\eps\hm>0$ и~третье слагаемое в~(\ref{M1}) обращается в~ноль для всех достаточно больших~$n$.

Перейдем к~оценке четвертого слагаемого в~(\ref{M1}). Аналогично~(\ref{M3}) 
можно получить:
\begin{multline}
\label{M10}
\!\!\sup\limits_{\substack{i: |\mu_i| \leq 1/T_1 \\ T\in [t_{\kappa_n}, T_U]}} \!{\mathsf D} 
H_i (T, T_m)  \leq \!\sup\limits_{\substack{i: |\mu_i| \leq 1/T_1 \\ T\in 
[t_{\kappa_n}, T_U]}} \!{\mathsf E} \left(H_i (T, T_m)\right)^2  \leq{}\\
{}\leq 2 c_7 (\ln n)^{(5-p)/2} \eta_n^p.
\end{multline}
По лемме~4 с~$m \hm= \sqrt{n} (\ln n)^3$ и~$k \hm= n-[n\eta_n^p T_1^p]$ 
для четвертого слагаемого в~(\ref{M1}) имеем:

\noindent
\begin{multline}
\p \left(\sup\limits_{j \in [0, n-1]} |Z_2(T_j^\prime)| > \eps\sqrt{n}\right) 
\leq {}\\
{}\leq \sum\limits_{j \in [0, n-1]} \hspace*{-3mm}\p \left( |Z_2(T_j^\prime)| > \varepsilon\sqrt{n}\right)\leq{}\\
{}\leq 4 n \exp \left\{ - \fr{\eps^2 n^{3/2} (\ln n)^3}{n-[n\eta_n^p T_1^p]}\!\Bigg/\! \big( 8 (T_U^2+2)\eps\sqrt{n} +{}\right.\\
\left.{}+ 128 c_7 (\ln n)^{(5-p)/2} \eta_n^p  (n-
[n\eta_n^p T_1^p])\big) 
\vphantom{ \fr{\eps^2 n^{3/2} (\ln n)^3}{n-[n\eta_n^p T_1^p]}}
\right\} +{}\\
{}
+ 22 \left(1+\fr{8(T_U^2+2) (n-[n\eta_n^p T_1^p])}{\eps 
\sqrt{n}}\right)^{1/2}\times{}\\
{}\times n^{3/2} (\ln n)^3 \alpha\left(\left[\fr{n-[n\eta_n^p 
T_1^p]}{2 (\ln n)^3 \sqrt{n}}\right]\right).
\label{M5}
\end{multline}
Используя ограничения $n^{-\delta_1}\hm\leq \eta_n^p \leq n^{-\delta_2}$ 
и~$1/2\hm<\delta_2\hm<\delta_1\hm<1$, из~(\ref{M5}) получим для любого $\eps\hm>0$
$$
\p \left(\sup\limits_{j \in [0, n-1]} |Z_2(T_j^\prime)| > \eps\sqrt{n}\right) 
\to 0
$$
при $n \to \infty$.

Рассмотрим, наконец, пятое слагаемое в~(\ref{M1})). Заметим, что при $0\hm< a \hm< b$ 
справедливо
$$
|Z_2(b)-Z_2(a)| \leq 2 |N_2(a,b)-{\mathsf E} N_2(a,b)| + n (b^2-a^2).
$$
Полагая $a = T_j^\prime$, $b \hm= T \hm\in [T_j^\prime, T_j^\prime+d_n]$ для 
произвольного $j \hm\in [0, n-1]$ и~учитывая, что
$$
(T^2 - (T_j^\prime )^2) = (T - T_j^\prime)(T+ T_j^\prime ) \leq  2 d_n T_U < 2 
T_U^2 n^{-1}; 
$$

\vspace*{-12pt}

\noindent
\begin{multline*}
\p\left(T_j^\prime < |x_i| \leq T \right) \leq \p\left(T_j^\prime < |x_i| \leq 
T_j^\prime+d_n\right) <{}\\
{}< d_n < T_U n^{-1}, 
\end{multline*}
получим  оценку
$$
|Z_2(T)-Z_2(T_j^\prime)| \leq 2 N_2(T_j^\prime, T) +  3 T_U^2 .
$$
Далее, поскольку $N_2 (T_j^\prime, T) \hm\leq N_2 (T_j^\prime, T_j^\prime+d_n)$ и~${\mathsf E} N_2 (T_j^\prime, T_j^\prime+d_n) \hm< T_U^2$,
имеем
\begin{multline*}
\sup\limits_{T \in [T_j^\prime, T_j^\prime+d_n]} |Z_2(T)-Z_2(T_j^\prime)| \leq {}\\
{}\leq
2 \left|N_2 (T_j^\prime, T_j^\prime+d_n) - {\mathsf E} N_2 (T_j^\prime, 
T_j^\prime+d_n)\right| +  5 T_U^2 .
\end{multline*}
Аналогично~(\ref{M3}) показывается, что
\begin{multline}
\label{M11}
\sup\limits_{\substack{i : |\mu_i| \leq 1/T_1 \\ j \in [0, n-1]}} {\mathsf D} \Ik 
(T_j^\prime < |x_i| \leq T_j^\prime + d_n) <{}\\
{}< c_7 (\ln n)^{(1-p)/2} \eta_n^p.
\end{multline}
Пусть $n > N(\eps)$ настолько, что 
$$
\fr{\eps\sqrt{n} - 5 T_U^2}{2} > \fr{\eps \sqrt{n} }{4}\,.
$$
%
 Тогда для пятого слагаемого в~(\ref{M1}) по лемме~4 с~$m \hm= 
\sqrt{n} (\ln n)^2$ и~$k \hm= n\hm-[n\eta_n^p T_1^p]$ имеем
\begin{multline}
\p \left(\sup\limits_{\substack{j \in [0, n-1] \\ T\in 
[T_j^{\prime},T_j^{\prime}+d_n]}} |Z_2(T)-Z_2(T_j^{\prime})| > 
\eps\sqrt{n}\right) \leq{}\\
{}\leq  \sum\limits_{j \in [0, n-1]} \p \left(  \left|N_2 (T_j^\prime, 
T_j^\prime+d_n) -{}\right.\right.\\
\left.\left.{}- {\mathsf E} N_2 (T_j^\prime, T_j^\prime+d_n)\right| > \fr{\eps\sqrt{n}}{4} 
\right) \leq{}\\
{}\leq  4n \exp \left\{ -  \fr{\eps^2 n^{3/2} (\ln n)^2}{(n-[n\eta_n^p T_1^p])^{-1}}\Bigg/ 
\big( 16 \eps \sqrt{n} +{}\right.\\
\left.{}+ 64 c_7 (\ln n)^{(1-p)/2} \eta_n^p (n-[n\eta_n^p 
T_1^p]) \big) 
\vphantom{\fr{\eps^2 n^{3/2} (\ln n)^2}{(n-[n\eta_n^p T_1^p])^{-1}}}
\right\} +{}\\
{}+ 22 \left(1+\fr{16 (n-[n\eta_n^p T_1^p])}{\eps \sqrt{n}}\right)^{1/2}\times{}\\
{}\times 
n^{3/2} (\ln n)^2 \alpha\left(\left[\fr{n-[n\eta_n^p T_1^p]}{2 (\ln n)^2 
\sqrt{n}}\right]\right).
\label{M6}
\end{multline}
Используя ограничения $n^{-\delta_1}\hm\leq \eta_n^p\hm \leq n^{-\delta_2}$ 
и~$1/2\hm<\delta_2\hm<\delta_1<1$, из~(\ref{M6}) получим для любого $\eps\hm>0$
$$
\p \left(\sup\limits_{\substack{j \in [0, n-1] \\ T\in 
[T_j^{\prime},T_j^{\prime}+d_n]}} |Z_2(T)-Z_2(T_j^{\prime})| > 
\eps\sqrt{n}\right) \to 0
$$
при $n \to \infty$.

Таким образом, показано, что для любого $\eps>0$ все слагаемые в~(\ref{M1}) 
стремятся к~нулю при $n\to\infty$. Следовательно,
$$
\fr{|U(\hat{t}_F)|}{\sqrt{n}}  \overset{\, \p \, }{\to} 0 \,,
$$
что вместе с~(\ref{D0}) завершает доказательство тео\-ремы.~\hfill$\square$

\smallskip

Следующая теорема дает достаточные условия для асимптотической нормальности 
оценки риска $\hat{R}(\hat{t}_F)$ в~случае $\mu \hm\in l_0[\eta_n]$.

\smallskip

\noindent
\textbf{Теорема 2.}\ 
\textit{Пусть $\mu \hm\in l_0[\eta_n]$, $\eta_n\hm\in[n^{-\delta_1}, n^{-\delta_2}]$, $1/2\hm < 
\delta_2\hm < \delta_1\hm<1;$ имеются такие константы $c_1, c_2\hm>0$, что для 
коэффициента сильного перемешивания $\alpha(\cdot)$ компонент вектора~$x$ 
справедливо} 
\begin{gather*}
\alpha(k) \leq c_1 k^{-1-(5/2)\delta_1/(1\hm-\delta_1)\hm-c_2},\enskip 
k=\overline{1,n-1};\\
 q_n < c_3 < 1;\enskip \mathrm{lim\,inf} q_n \ln n = c_4 > 0;
\end{gather*}
\textit{для максимального коэффициента корреляции~$\rho(\cdot)$ компонент вектора~$x$ 
справедливо}
$$
\sum\limits_{k = 1}^{\infty} \sup\limits_{n\geq k+1} \rho(k) \equiv 
\sum\limits_{k = 1}^{\infty}  \rho^\star (k) = c_5 < \infty. 
$$
\textit{Тогда при $n \to \infty$}
$$
\fr{\hat{R}(\hat{t}_F) - R(T_m)}{C_\rho \sqrt{2n}} \Rightarrow N(0, 1),
$$
\textit{где}
$$
C_\rho = \sigma^2\sqrt{1 +   \lim\limits_{n\to\infty} \fr{1}{n} 
\sum\limits_{j\neq i} \mathrm{corr}^2 (x_i, x_j)}\,.
$$

\noindent
Д\,о\,к\,а\,з\,а\,т\,е\,л\,ь\,с\,т\,в\,о\  проводится аналогично доказательству теоремы~1. 
Переменная~$D_n$ теперь определяется как $D_n \hm= \{(i,j) : 
|\mu_i|\hm=|\mu_j|=0$, $j\hm\neq i\}$. Условия вида $|\mu_i|\hm<T_1^{-1}$ (вида 
$|\mu_i|\hm\geq T_1^{-1}$) заменяются условиями  $\mu_i\hm=0$ (соответственно 
$|\mu_i|\hm>0$).
Поскольку $\mu \hm\in l_0[\eta_n]$, количество~$i$ таких, что $|\mu_i|\hm>0$ 
(а~значит, и~число слагаемых в~$Z_1(T)$), не превышает~$[n \eta_n]$.

Для оценки первого слагаемого в~(\ref{M1}) используется лемма~2, 
в~которой можно взять, например, $b\hm=1/2$, а~для~$\kappa_n^0$ использовать оценку 
$\kappa_n^0 \hm> n \eta_n$. Формулы (\ref{M3}),  (\ref{M10}) и~(\ref{M11}) 
принимают вид соответственно
\begin{align*}
\sup\limits_{\substack{i: \mu_i =0 \\ T\in [t_{\kappa_n^0}, T_U]}} |{\mathsf E} H_i (T, 
T_m)| & \leq c_8 (\ln n)^{3/2} \eta_n ;
\\
\sup\limits_{\substack{i: \mu_i =0 \\ T\in [t_{\kappa_n^0}, T_U]}} {\mathsf D} H_i (T, 
T_m)  & \leq 2 c_8 (\ln n)^{5/2} \eta_n;
\\
\sup\limits_{\substack{i : \mu_i =0 \\ j \in [0, n-1]}} {\mathsf D} \Ik (T_j^\prime < 
|x_i| \leq T_j^\prime + d_n) &< c_8 (\ln n)^{1/2} \eta_n,
\end{align*}
где $c_8 = c_8(c_3,\delta_1, \delta_2, c_4)$. В~остальном доказательство 
аналогично.~\hfill$\square$

\section{Сильная состоятельность оценки риска при~применении FDR-процедуры 
в~условиях слабой зависимости}

Следующая теорема дает достаточные условия для сильной состоятельности оценки 
риска $\hat{R}(\hat{t}_F)$ в~случаях $\mu \hm\in m_p[\eta_n]$ и~$\mu \hm\in 
l_0[\eta_n]$.

\smallskip

\noindent
\textbf{Теорема 3.}
\textit{Пусть $\mu\hm \in m_p[\eta_n]$, $\eta_n^p\hm\in[n^{-\delta_1}, n^{-\delta_2}]$ либо 
$\mu \hm\in l_0[\eta_n]$, $\eta_n\hm\in[n^{-\delta_1}, n^{-\delta_2}]$; $0 \hm< \delta_2 
\hm< \delta_1<1$; имеются такие константы $c_1, c_2\hm>0$, что для коэффициента 
сильного перемешивания $\alpha(\cdot)$ компонент вектора~$x$ справедливо}  
$\alpha(k) \hm\leq c_1 k^{-2-(7/2)\delta_1/(1\hm-\delta_1)\hm-c_2}$, $k\hm=\overline{1,n-1}$; 
$q_n \hm< c_3 \hm< 1$; $\mathrm{lim\,inf} q_n \ln n \hm= c_4 \hm> 0$. \textit{Тогда при} $n \hm\to \infty$
$$
\fr{\hat{R}(\hat{t}_F) - R(T_m)}{n} \rightarrow 0 \, \, \,\textit{п.~в.}
$$


\noindent
Д\,о\,к\,а\,з\,а\,т\,е\,л\,ь\,с\,т\,в\,о\,.  Воспользуемся представлением~(\ref{D00}).

Покажем, что $(\hat{R}(T_m)-R(T_m))n^{-1}\hm \to 0$ п.~в.\ при $n\hm\to\infty$. 
При мягкой пороговой обработке ${\mathsf E} \hat{R}(T_m) \hm= R(T_m)$, а~при жесткой 
пороговой обработке
\begin{multline*}
\fr{\hat{R}(T_m)-R(T_m)}{n} = {}\\
{}=\fr{\hat{R}(T_m)-{\mathsf E} \hat{R}(T_m)}{n} 
+\fr{{\mathsf E}\hat{R}(T_m)-R(T_m)}{n}\,,
\end{multline*}
где второе слагаемое стремится к~нулю при $n\to\infty$ \cite{Mallat}. 
Следовательно, достаточно показать, что $(\hat{R}(T_m)\hm-{\mathsf E}\hat{R}(T_m))n^{-1} \hm\to 0$ п.~в.

Полагая в~лемме~3 $X_i \hm= F[x_i, T_m] \hm- {\mathsf E} F[x_i, T_m]$, $b \hm= 
2(\sigma^2\hm+T_m^2)$ и~$m \hm= n^{1/4}$ и~учитывая ограничения на $\alpha(\cdot)$ из 
условия, нетрудно убедиться, что для всех~$n$
$$
\p \left(\left| \fr{\hat{R}(T_m)-{\mathsf E} \hat{R}(T_m)}{n}\right| >\eps \right) 
\leq \fr{c_5}{n^{1+c_6}}\,, 
$$
где константы $c_5$, $c_6$ положительны. Отсюда
$$
\sum\limits_{n=1}^{\infty}\p \left(\left|\fr{\hat{R}(T_m)-{\mathsf E} 
\hat{R}(T_m)}{n}\right| >\eps \right) < \infty,
$$
и по теореме~1.3.4 из~\cite{Serfling2002} 
$$
\left(\hat{R}(T_m)-{\mathsf E}\hat{R}(T_m)\right)n^{-1} \to 0~\mbox{п.~в.}
$$



Покажем теперь, что  $U(\hat{t}_F) \, n^{-1}\hm \to 0$ п.~в. Доказательство 
проведено для $\mu \hm\in m_p[\eta_n]$, в~случае $\mu\hm \in l_0[\eta_n]$ 
доказательство аналогично.
Аналогично формуле~(\ref{M1}), для произвольного $\eps\hm>0$ в~терминах тео\-ре\-мы~1 имеем
\begin{multline*}
\p \left( \fr{|U(\hat{t}_F)|}{n}> 4\eps \right) \leq \p\left(\hat{t}_F 
\leq t_{\kappa_n}\right) +{}\\
{}+ \p\left(\fr{\sup\nolimits_{T\in [t_{\kappa_n}, T_U]} |{\mathsf E} 
U(T)|}{n}>\eps\right)+{}\\
{}+ \p \left(\sup\limits_{T\in [t_{\kappa_n}, T_U]} |Z_1(T)| > \eps n\right) +{}
\end{multline*}

\noindent
\begin{multline}
{}+ \p  \left(\sup\limits_{j \in [0, n-1]} |Z_2(T_j^{\prime})| > \eps n\right) +{}\\
{}+ \p \left(\sup\limits_{\substack{j \in [0, n-1] \\ T\in 
[T_j^{\prime},T_j^{\prime}+d_n]}} |Z_2(T)-Z_2(T_j^{\prime})| > \eps n\right).
\label{M1SC}
\end{multline}
Применяя рассуждения, аналогичные приведенным в~доказательстве теоремы~1, можно показать, что
$$
\sup\limits_{T\in [t_{\kappa_n}, T_U]} |{\mathsf E} U(T)| = o(n); \enskip
\sup\limits_{T\in [t_{\kappa_n}, T_U]} |Z_1(T)|  = o(n),
$$
откуда следует, что второе и~третье слагаемые в~(\ref{M1SC}) обращаются в~ноль 
для всех достаточно больших~$n$.

Для некоторых положительных констант  $c_7$ и~$c_8$ первое, четвертое и~пятое 
слагаемые  в~(\ref{M1SC}) не превышают $c_7 n^{-1-c_8}$ для всех достаточно 
боль\-ших~$n$, что можно показать с~помощью ограничения на $\alpha(\cdot)$ из 
условия и~рассуждений, аналогичных приведенным при выводе соответственно формул~(\ref{M1next}), (\ref{M5}) и~(\ref{M6}), с~тем отличием, что при применении 
леммы~4 полагается $m \hm= (\ln n)^3$.

Из доказанного следует, что
$$
\sum\limits_{n=1}^{\infty}\p \left( \fr{|U(\hat{t}_F)|}{n}> 4\eps \right) 
< \infty,
$$
и по теореме~1.3.4 из~\cite{Serfling2002} $U(\hat{t}_F) \, n^{-1} \to 0$ п.~в., 
что завершает доказательство теоремы.~\hfill$\square$



{\small\frenchspacing
 {\baselineskip=11.5pt
 %\addcontentsline{toc}{section}{References}
 \begin{thebibliography}{99}
\bibitem{FDRImage}
\Au{Krylov V.\,A., Moser~G., Serpico~S.\,B., Zerubia~J.}
False discovery rate approach to unsupervised image change detection~// IEEE 
T. Image Process., 2016. Vol.~25. No.\,10. P.~4704--4718. doi: 10.1109/TIP.2016.2593340.

\bibitem{MultipleTesting} %2
\Au{Menyhart~O., Weltz~B., Gyorffy~B.}
MultipleTesting.com: A~tool for life science researchers for multiple hypothesis 
testing correction~// PLoS One, 2021. Vol.~16. No.\,6. Art.~0245824. doi: 10.1371/journal.pone.0245824.

\bibitem{AdaptingFDR} %3
\Au{Abramovich~F., Benjamini~Y., Donoho~D., Johnstone~I.}
Adapting to unknown sparsity by controlling the false discovery rate~// Ann. Stat., 2006. Vol.~34. No.\,2. P.~584--653.
doi: 10.1214/009053606000000074.

\bibitem{ZasShe17} %4
\Au{Заспа~А.\,Ю., Шестаков~О.\,В.}
Состоятельность оценки риска при множественной проверке гипотез с~FDR-по\-ро\-гом~// 
Вестник ТвГУ. Сер. Прикладная математика, 2017. Вып.~1. С.~5--16.
doi: 10.26456/vtpmk119. EDN: YFYJXT.

\bibitem{Mathematics2020} %5
\Au{Palionnaya~S.\,I., Shestakov~O.\,V.}
Asymptotic properties of MSE estimate for the false discovery rate controlling 
procedures in multiple hypothesis testing // Mathematics, 2020. Vol.~8. No.~11. 
Art.~1913. 11~p. doi: 10.3390/ math8111913.

\bibitem{Shestakov2021-1} %6
\Au{Шестаков~О.\,В.}
Анализ несмещенной оценки среднеквадратичного риска метода блочной пороговой 
обработки~// Информатика и~её применения, 2021. Т.~15. Вып.~2. С.~30--35.
doi: 10.14357/19922264210205. EDN: DSQQAU.

\bibitem{Shestakov2021-2} %7
\Au{Шестаков~О.\,В.}
Пороговые функции в~методах подавления шума, основанных на вейв\-лет-раз\-ло\-же\-нии 
сигнала~// Информатика и~её применения, 2021. Т.~15. Вып.~3. С.~51--56.
doi: 10.14357/19922264210307. EDN: WSEAYG.

\bibitem{Shestakov2022} %8
\Au{Шестаков~О.\,В.}
Несмещенная оценка риска пороговой обработки с~двумя пороговыми значениями~// 
Информатика и~её применения, 2022. Т.~16. Вып.~4. С.~14--19.
doi: 10.14357/19922264220403. EDN: \mbox{DZBVLC}.

\bibitem{ResultsOnFDRUnderDependence} %9
\Au{Farcomeni~A.}
Some results on the control of the false discovery rate under dependence~// 
Scand. J. Stat., 2007. Vol.~34. No.\,2. P.~275--297.
doi: 10.1111/j.1467-9469.2006.00530.x.

\bibitem{VorontsovShestakov2023} %10
\Au{Воронцов~М.\,О., Шестаков~О.\,В.}
Среднеквадратичный риск FDR-про\-це\-ду\-ры в~условиях слабой за\-ви\-си\-мости~// 
Информатика и~её применения, 2023. Т.~17. Вып.~2. С.~34--40.
doi: 10.14357/19922264230205. EDN: AVJZDX.

\bibitem{Vorontsov2024} %11
\Au{Воронцов~М.\,О.}
Анализ среднеквадратичного риска при использовании методов множественной 
проверки гипотез для выбора параметров пороговой обработки в~условиях слабой 
зависимости~// Вестник Московского университета. Сер. 15: Вычислительная 
математика и~кибернетика, 2024. №\,2. С.~18--24.

\bibitem{Bosq} %12
\Au{Bosq~D.}
Nonparametric statistics for stochastic processes: Estimation and prediction.~--- 
Lecture notes in statistics ser.~--- New York, NY, USA: Springer, 1996. Vol.~110. 
188~p.

\bibitem{Mallat} %13
\Au{Mallat~S.}
A wavelet tour of signal processing.~--- New York, NY, USA: Academic Press, 1999. 
857~p.

\bibitem{spatialAdaptation} %14
\Au{Donoho~D., Johnstone~I.}
Ideal spatial adaptation via wavelet shrinkage~// Biometrika, 1994. Vol.~81. 
No.\,3. P.~425--455. doi: 10.1093/biomet/81.3.425.

\bibitem{AdaptingSURE} %15
\Au{Donoho D., Johnstone I.\,M.}
Adapting to unknown smoothness via wavelet shrinkage~// J.~Amer. Stat. Assoc., 
1995. Vol.~90. P.~1200--1224.

\bibitem{ExactRisk} %16
\Au{Marron J.\,S., Adak~S., Johnstone~I.\,M., Neumann~M.\,H., Patil~P.}
Exact risk analysis of wavelet regression~// J.~Comput. Graph. Stat., 1998. 
Vol.~7. P.~278--309. doi: 10.1080/ 10618600.1998.10474777.

\bibitem{Jansen} %17
\Au{Jansen~M.}
Noise reduction by wavelet thresholding.~-- Lecture notes in statistics ser.~--- 
New York, NY, USA: Springer, 2001. Vol.~161. 217~p.

\bibitem{KuShe2016_1} %18
\Au{Кудрявцев~А.\,А., Шестаков~О.\,В.}
Асимптотическое поведение порога, минимизирующего усредненную\linebreak вероятность ошибки 
вычисления вейв\-лет-ко\-эф\-фи\-ци\-ен\-тов~// Докл. Акад. наук, 2016. Т.~468. №\,5. 
С.~487--491.

\bibitem{KuShe2016_2} %19
\Au{Кудрявцев~А.\,А., Шестаков~О.\,В.}
Асимптотически оптимальная пороговая обработка вейв\-лет-ко\-эф\-фи\-ци\-ен\-тов в~моделях с~негауссовым распределением шума~// Докл. Акад. наук, 2016. Т.~471. №\,1. 
С.~11--15.



\bibitem{Eroshenko} %20
\Au{Ерошенко~А.\,А.}
Статистические свойства оценок сигналов и~изображений при пороговой обработке 
коэффициентов в~вейв\-лет-раз\-ло\-же\-ни\-ях: Дис.\ \ldots\ канд. физ.-мат. наук.~--- 
М.: МГУ, 2015. 82~с.

\bibitem{Peligrad} %21
\Au{Peligrad~M.}
On the asymptotic normality of sequences of weak dependent random variables~// 
J. Theor. Probab., 1996. Vol.~9. No.\,3. P.~703--715. doi: 10.1007/BF02214083.

\bibitem{Serfling2002} %22
\Au{Serfling~R.\,J.}
Approximation theorems of mathematical statistics.~--- New York, NY, USA: John Wiley \&~Sons, Inc., 2002. 371~p.

\end{thebibliography}

 }
 }

\end{multicols}

\vspace*{-6pt}

\hfill{\small\textit{Поступила в~редакцию 21.05.24}}

\vspace*{8pt}

%\pagebreak

%\newpage

%\vspace*{-28pt}

\hrule

\vspace*{2pt}

\hrule



\def\tit{ASYMPTOTIC NORMALITY AND STRONG CONSISTENCY\\ OF~RISK ESTIMATE WHEN USING THE~FDR THRESHOLD\\ UNDER WEAK DEPENDENCE CONDITION}


\def\titkol{Asymptotic normality and strong consistency of~risk estimate when using the~FDR threshold under weak dependence condition}


\def\aut{M.\,O.~Vorontsov$^{1,2}$ and~O.\,V.~Shestakov$^{1,2,3}$}

\def\autkol{M.\,O.~Vorontsov and~O.\,V.~Shestakov}

\titel{\tit}{\aut}{\autkol}{\titkol}

\vspace*{-13pt}


\noindent
$^{1}$Department of Mathematical Statistics, Faculty of Computational Mathematics and Cybernetics,
 M.\,V.~Lo\-mo-\linebreak
 $\hphantom{^1}$nosov Moscow State University, 1-52~Leninskie Gory, GSP-1, Moscow 119991, Russian Federation

\noindent
$^{2}$Moscow Center for Fundamental and Applied Mathematics, M.\,V.~Lomonosov Moscow State University,\linebreak
$\hphantom{^1}$1~Leninskie Gory, GSP-1, Moscow 119991, Russian Federation

\noindent
$^{3}$Federal Research Center ``Computer Science and Control'' of the Russian Academy of Sciences, 44-2~Vavilov\linebreak
$\hphantom{^1}$Str., Moscow 119333, Russian Federation


\def\leftfootline{\small{\textbf{\thepage}
\hfill INFORMATIKA I EE PRIMENENIYA~--- INFORMATICS AND
APPLICATIONS\ \ \ 2024\ \ \ volume~18\ \ \ issue\ 3}
}%
 \def\rightfootline{\small{INFORMATIKA I EE PRIMENENIYA~---
INFORMATICS AND APPLICATIONS\ \ \ 2024\ \ \ volume~18\ \ \ issue\ 3
\hfill \textbf{\thepage}}}

\vspace*{2pt}






\Abste{An approach to solving the problem of noise removal in a large array of sparse data is considered
 based on the method of controlling the average proportion of false hypothesis rejections (False Discovery Rate, FDR). 
 This approach is equivalent to threshold processing procedures that remove array components whose values do not exceed 
 some specified threshold. The observations in the model are considered weakly dependent. To control the\linebreak\vspace*{-12pt}}
 
 \Abstend{degree of dependence, 
 restrictions on the strong mixing coefficient and the maximum correlation coefficient are used. The mean-square risk is 
 used as a measure of the effectiveness of the considered approach. It is possible to calculate the risk value only on the test data;
  therefore, its statistical estimate is considered in the work and its properties are investigated. The asymptotic normality and
   strong consistency of the risk estimate are proved when using the FDR threshold under conditions of weak dependence in the data.}

\KWE{thresholding; multiple hypothesis testing; risk estimate}

\DOI{10.14357/19922264240309}{ZOQVTO}

%\vspace*{-12pt}


    
   %   \Ack

%\vspace*{-3pt}
%\noindent



  \begin{multicols}{2}

\renewcommand{\bibname}{\protect\rmfamily References}
%\renewcommand{\bibname}{\large\protect\rm References}

{\small\frenchspacing
 {\baselineskip=10.8pt
 \addcontentsline{toc}{section}{References}
 \begin{thebibliography}{99} 

%1
\bibitem{FDRImage-1}
\Aue{Krylov, V.\,A., G.~Moser, S.\,B.~Serpico, and J.~Zerubia.} 2016. 
False discovery rate approach to unsupervised image change detection. 
\textit{IEEE T. Image Process.} 25(10):4704--4718. doi: 10.1109/TIP.2016.2593340.

%2
\bibitem{MultipleTesting-1}
\Aue{Menyhart, O., B.~Weltz, and B.~Gyorffy.} 2021. 
MultipleTesting.com: A~tool for life science researchers for multiple hypothesis testing correction. 
\textit{PLoS One} 16(6):0245824. 
doi: 10.1371/journal.pone.0245824.

%3
\bibitem{AdaptingFDR-1}
\Aue{Abramovich, F., Y.~Benjamini, D.~Donoho, and I.\,M.~Johnstone.} 2006. 
Adapting to unknown sparsity by controlling the false discovery rate. 
\textit{Ann. Stat.} 34(2):584--653. 
doi: 10.1214/009053606000000074.


%4
\bibitem{ZasShe17-1}
\Aue{Zaspa, A.\,Yu., and O.\,V.~Shestakov.} 2017.
Sostoyatel'nost' otsenki riska pri mnozhestvennoy proverke gipotez s~FDR-porogom
 [Consistency of the risk estimate of the multiple hypothesis testing with the FDR threshold]. 
\textit{Vestnik TvGU. Ser.: Prikladnaya matematika} [Herald of Tver State University. Ser. Applied Mathematics] 1:5--16.
doi: 10.26456/vtpmk119. EDN: YFYJXT.

%5
\bibitem{Mathematics2020-1}
\Aue{Palionnaya, S.\,I., and O.\,V.~Shestakov.} 2020. 
Asymptotic properties of MSE estimate for the false discovery rate controlling procedures in multiple hypothesis testing. 
\textit{Mathematics} 8(11):1913. 11~p.
doi: 10.3390/math8111913.

%6
\bibitem{Shestakov2021-1-1}
\Aue{Shestakov, O.\,V.} 2021.
Analiz nesmeshchennoy otsenki srednekvadratichnogo riska metoda blochnoy po\-ro\-go\-voy obrabotki 
[Analysis of the unbiased mean-square risk estimate of the block thresholding method]. 
\textit{Informatika i~ee Primeneniya~--- Inform. Appl.} 15(2):30--35.
doi: 10.14357/19922264210205. EDN: DSQQAU.

%7
\bibitem{Shestakov2021-2-1}
\Aue{Shestakov, O.\,V.} 2021.
Porogovye funktsii v~metodakh podavleniya shuma, osnovannykh na veyvlet-razlozhenii signala 
[Thresholding functions in the noise suppression methods based on the wavelet expansion of the signal]. 
\textit{Informatika i~ee Primeneniya~--- Inform. Appl.} 15(3):51--56.
doi: 10.14357/19922264210307. EDN: WSEAYG.

%8
\bibitem{Shestakov2022-1}
\Aue{Shestakov, O.\,V.} 2022.
Nesmeshchennaya otsenka riska porogovoy obrabotki s dvumya porogovymi znacheniyami [Unbiased thresholding risk estimate with two threshold values]. 
\textit{Informatika i~ee Primeneniya~--- Inform. Appl.} 16(4):14--19.
doi: 10.14357/19922264220403. EDN: DZBVLC.

%9
\bibitem{ResultsOnFDRUnderDependence-1}
\Aue{Farcomeni, A.} 2007. Some results on the control of the false discovery rate under dependence. 
\textit{Scand. J. Stat.} 34(2):275--297. 
doi: 10.1111/j.1467-9469.2006.00530.x.

%10
\bibitem{VorontsovShestakov2023-1}
\Aue{Vorontsov, M.\,O., and O.\,V.~Shestakov.} 2023.
Sred\-ne\-kvad\-ra\-tich\-nyy risk FDR-protsedury v~usloviyakh slaboy za\-vi\-si\-mosti [Mean-square risk of the FDR procedure under weak dependence]. 
\textit{Informatika i~ee Primeneniya~--- Inform. Appl.} 17(2):34--40.
doi: 10.14357/19922264230205. EDN: AVJZDX.

%11
\bibitem{Vorontsov2024-1}
\Aue{Vorontsov, M.\,O.} 2024. 
RMS risk analysis when using multiple hypothesis testing select parameters of thresholding under conditions of weak dependence. 
\textit{Moscow University Computational Mathematics Cybernetics} 48:91--97. 
doi: 10.3103/S027864192470002X.

%12
\bibitem{Bosq-1}
\Aue{Bosq, D.} 1996. 
\textit{Nonparametric statistics for stochastic processes: Estimation and prediction}. 
Lecture notes in statistics ser. New York, NY: Springer Verlag. Vol.~110. 188~p.

%13
\bibitem{Mallat-1}
\Aue{Mallat, S.} 1999. 
\textit{A wavelet tour of signal processing}. New York, NY: Academic Press. 857~p.

%14
\bibitem{spatialAdaptation-1}
\Aue{Donoho, D., and I.\,M.~Johnstone.} 1994. 
Ideal spatial adaptation via wavelet shrinkage. 
\textit{Biometrika} 81(3):425--455. doi: 10.1093/biomet/81.3.425.

%15
\bibitem{AdaptingSURE-1}
\Aue{Donoho, D., and I.\,M.~Johnstone.} 1995. 
Adapting to unknown smoothness via wavelet shrinkage. 
\textit{J. Am. Stat. Assoc.} 90(432):1200--1224. doi: 10.1080/01621459. 1995.10476626.

%16
\bibitem{ExactRisk-1}
\Aue{Marron, J.\,S., S.~Adak, I.\,M.~Johnstone, M.\,H.~Neumann, and P.~Patil.} 1998. 
Exact risk analysis of wavelet regression. 
\textit{J.~Comput. Graph. Stat.} 7(3):278-309. doi: 10.1080/ 10618600.1998.10474777.

%17
\bibitem{Jansen-1}
\Aue{Jansen, M.} 2001. 
\textit{Noise reduction by wavelet thresholding}. Lecture notes in statistics ser. New York, NY: Springer Verlag. Vol.~161. 217~p.

%18
\bibitem{KuShe2016_1-1}
\Aue{Kudryavtsev, A.\,A., and O.\,V.~Shestakov.} 2016. 
Asymptotic behavior of the threshold minimizing the average probability of error in calculation of wavelet coefficients. 
\textit{Dokl. Math.} 93(3):295--299.
doi: 10.1134/S1064562416030212. EDN: WUMUEV. 

%19
\bibitem{KuShe2016_2-1}
\Aue{Kudryavtsev, A.\,A., and O.\,V.~Shestakov.} 2016. 
Asymptotically optimal wavelet thresholding in the models with non-Gaussian noise distributions. 
\textit{Dokl. Math.} 94(3):615--619.
doi: 10.1134/S1064562416060028. EDN: YUYVUP.




%20
\bibitem{Eroshenko-1}
\Aue{Eroshenko, A.\,A.} 2015. Statisticheskie svoystva otsenok signalov i~izobrazheniy pri porogovoy obrabotke ko\-ef\-fi\-tsi\-en\-tov 
v~veyvlet-razlozheniyakh 
[Statistical properties of signal and image estimates under thresholding of coefficients in wavelet decompositions]. Moscow: MSU. PhD Diss. 82~p.

%21
\bibitem{Peligrad-1}
\Aue{Peligrad, M.} 1996. 
On the asymptotic normality of sequences of weak dependent random variables. 
\textit{J. Theor. Probab.} 9(3):703--715. doi: 10.1007/BF02214083.

%22
\bibitem{Serfling2002-1}
\Aue{Serfling, R.\,J.} 2002. 
\textit{Approximation theorems of mathematical statistics}. New York, NY: John Wiley \&~Sons. 371~p.
\end{thebibliography}

 }
 }

\end{multicols}

\vspace*{-6pt}

\hfill{\small\textit{Received May 21, 2024}} 

%\vspace*{-18pt}

\Contr

\vspace*{-3pt}


\noindent
\textbf{Vorontsov Mikhail O.} (b.\ 1996)~--- PhD student, Department of Mathematical Statistics, 
Faculty of Computational Mathematics and Cybernetics, M.\,V.~Lomonosov Moscow State University, 1-52~Leninskie Gory, GSP-1, Moscow 119991, Russian Federation;  
mathematician, Moscow Center for Fundamental and Applied Mathematics, M.\,V.~Lomonosov Moscow State University, 1~Leninskie Gory, GSP-1, Moscow 119991, Russian Federation;
\mbox{m.vtsov@mail.ru}

\vspace*{6pt}

\noindent
\textbf{Shestakov Oleg V.} (b.\ 1976)~--- Doctor of Science in physics and mathematics, professor, Department of Mathematical Statistics,
 Faculty of Computational Mathematics and Cybernetics, M.\,V.~Lomonosov Moscow State University, 1-52~Leninskie Gory, GSP-1, Moscow 119991, Russian Federation; 
 senior scientist, Federal Research Center ``Computer Science and Control'' of the Russian Academy of Sciences, 44-2~Vavilov Str., Moscow 119333, 
 Russian Federation; leading scientist, Moscow Center for Fundamental and Applied Mathematics, M.\,V.~Lomonosov Moscow State University, 
 1~Leninskie Gory, GSP-1, Moscow 119991, Russian Federation; \mbox{oshestakov@cs.msu.su}


\label{end\stat}

\renewcommand{\bibname}{\protect\rm Литература}  %8
\def\stat{suchkov}

\def\tit{ИНФОРМАЦИОННО-АНАЛИТИЧЕСКАЯ АВТОМАТИЗИРОВАННАЯ 
СИСТЕМА <<МЕГАЛИТ>> В~ОПТИМИЗАЦИИ ДИАГНОСТИКИ И ЛЕЧЕНИЯ МОЧЕКАМЕННОЙ БОЛЕЗНИ}

\def\titkol{Информационно-аналитическая автоматизированная 
система <<Мегалит>> в~оптимизации диагностики} % и лечения мочекаменной болезни}

\def\autkol{М.\,П.~Кривенко, С.\,А.~Голованов,  П.\,А.~Савченко
 и др.}

\def\aut{М.\,П.~Кривенко$^1$, С.\,А.~Голованов$^2$, П.\,А.~Савченко$^3$, 
А.\,В.~Сивков$^4$,  А.\,П.~Сучков$^5$}

\titel{\tit}{\aut}{\autkol}{\titkol}

%{\renewcommand{\thefootnote}{\fnsymbol{footnote}}\footnotetext[1] {Статья 
%рекомендована к публикации в журнале Программным комитетом конференции 
%<<Электронные библиотеки: перспективные методы и технологии, электронные 
%коллекции>> (RCDL-2012).}}

\renewcommand{\thefootnote}{\arabic{footnote}}
\footnotetext[1]{Институт проблем информатики Российской академии наук, mkrivenko@ipiran.ru} 
\footnotetext[2]{Научно-исследовательский институт урологии, sergeygol124@mail.ru} 
\footnotetext[3]{Институт проблем информатики Российской академии наук, psavchenko@ipiran.ru} 
\footnotetext[4]{Научно-исследовательский институт урологии, uroinfo@yandex.ru} 
\footnotetext[5]{Институт проблем информатики Российской академии наук, asuchkov@ipiran.ru}


\Abst{В статье, первой из предполагаемой серии научных публикаций, рассматриваются 
результаты исследований по автоматизации информационных и аналитических процессов 
обследования, диагностирования и лечения мочекаменной болезни (МКБ). Существенную 
роль в создании систем диагностики МКБ играет разработка информационных технологий 
сбора клинических данных и формирования специализированных баз данных (БД). Изучена 
возможность создания и способы реализации ин\-фор\-ма\-ци\-он\-но-ана\-ли\-ти\-че\-ской 
автоматизированной системы (ИААС) по сбору, хранению и обработке клинических данных 
обследования больных, а также алгоритмизации процессов принятия решений при 
диагностике МКБ и выборе схем лечения и профилактики этого заболевания. Предложенные 
математические методы и алгоритмы могут найти применение при дальнейшем развитии 
фундаментальных научных исследований в области разработки математических методов 
моделирования ме\-ди\-ко-био\-ло\-ги\-че\-ских сис\-тем, а также при создании необходимого 
математического инструментария.}
      
\KW{информационно-аналитическая система; урология; компьютерная диагностика; схема 
лечения; схема профилактики}

\DOI{10.14357/19922264130409}

\vskip 14pt plus 9pt minus 6pt

      \thispagestyle{headings}

      \begin{multicols}{2}

            \label{st\stat}

\section{Введение}

      В настоящее время доля людей, у которых на протяжении их жизни диагностируется 
МКБ, довольно значительна и составляет в странах Западной Европы 5\%--9\%, в Канаде и 
США~--- 7\%--12\%, в странах Азии~--- 1\%--5\%~[1--4]. 
  %    
      Эпидемиологические исследования, проводимые в ряде индустриально развитых стран, 
указывают на сохранение тенденции к росту частоты возникновения МКБ 
среди населения. Так, число  впервые выявленных случаев
МКБ на 100\,000 населения за последние 
десятилетия возросло в США с 58,7 (1950--1954~гг.)\ до 85,1 (2000~г.)~\cite{4-su, 3-su}, 
в Японии~--- с~43,7 
(1965~г.)\ до~134 (2005~г.)~\cite{6-su, 5-su}, в России~--- со 123,3 (2002~г.)\ до~138,6 
(2010~г.)~[7, 8].
      
      По данным исследований~[9] с использованием БД Pediatric Health 
Information System (национальная БД, в которую включены данные об амбулаторных 
визитах, срочных госпитализациях и стационарном лечении детей из 42~детских больниц 
США) по сравнению с общим количеством госпитализированных пациентов число пациентов 
с МКБ увеличилось с 18,4 на 100\,000 населения в 1999~г.\ до 57,0 в 2008~г., годовой прирост 
составил 10,6\% ($p \hm<0{,}0001$). 
      
      В основе развития МКБ лежат характерные нарушения обмена веществ, приводящие к 
образованию камней в мочевых путях. Эти литогенные (камнеобразующие) нарушения обмена 
веществ характеризуются большим многообразием и проявляются различными 
патологическими изменениями биохимического состава крови и мочи пациента.
      
      Необходимым условием для выбора правильной тактики консервативного лечения с 
целью предупреждения повторного камнеобразования является исследование всего комплекса 
метаболических факторов риска (МФР), ответственных за развитие МКБ.
{\looseness=-1

}
      
      В этой связи большое внимание придается изуче\-нию особенностей фи\-зи\-ко-хи\-ми\-че\-ских 
па\-ра\-мет\-ров мочи, во многом определяющих вероятность образования мочевых камней~[10]. Кроме 
того, литогенные нарушения метаболизма зачастую имеют сложный многофакторный 
характер воздействия на процесс формирования камня. Это создает особые трудности для 
врача в полной и объективной оценке всех влияющих литогенных факторов обмена веществ, а 
также в принятии решения по диагностике и выбору лечебной тактики для конкретного 
больного. Отсюда возникает необходимость формирования БД анкетных и 
лабораторных исследований, систем, связанных с диагностикой МКБ и формирования базы 
знаний по профилактике и лечению этого заболевания.
      
\section{Системы компьютерной диагностики в~области урологии}

      Существенную роль в создании систем диагностики МКБ является разработка 
информационных технологий сбора клинических данных и формирования 
специализированных БД. К~ним относится упомянутая Pediatric Health Information 
System. В~ряде медицинских работ упоминается реестр по уролитиазу (БД по 
больным и результатам лечения) Юго-за\-пад\-но\-го медицинского центра Техасского 
университета: <<Retrospective data from the University of Texas Southwestern Medical Center 
\textit{Nephrolithiasis Registry} from 17~studies that dealt with physiologic and physicochemical 
effects of various magnesium and potassium salts were categorized into three groups and 
analyzed\ldots>>~[11]. Однако подробного описания данного реестра не приведено. 
      
      Задачи диагностики, дифференциальной диагностики, прогнозирования, выбора 
стратегии и тактики лечения позволяют решать экспертные медицинские системы~[12].
      
      Ряд работ посвящен использованию в урологии компьютерных диагностических систем 
на основе методов искусственных нейронных сетей (ИНС)~[13].
 Так, в онкоурологии смогли 
прогнозировать 5-лет\-нюю выживаемость пациентов, перенесших радикальную цистэктомию 
по поводу\linebreak
рака мочевого пузы\-ря~[14]. Искусственные ней\-ронные сети применили также для 
автоматизи\-рованного анализа показаний к биопсии предстательной железы~[15]. 
Методика основывалась на\linebreak 
выявлении общего прос\-тат-спе\-ци\-фи\-че\-ско\-го антигена (ПСА) и определении доли 
свободного ПСА. Чувствительность составила 95\%, специфичность~--- 34\%. При 
дополне\-нии нейросети моделью логистической регрессии специфичность возросла до 95\%. 
Искусственная нейронная сеть использовалась для выявле\-ния группы риска рака предстательной железы в сравнении с 
моделью логистической регрессии~[15]. Искусственная нейронная сеть так\-же 
продемонстрировала более точные 
прогностические возможности. Компьютерных систем диагностики именно МКБ по 
литературным данным не выявлено.
{ %\looseness=-1

}
      
      Отсюда ясно, что имеется настоятельная необходимость разработки аналитической 
системы диагностики и лечения больных МКБ в процессе их динами\-ческого наблюдения 
(мониторинге) для пред\-упреж\-де\-ния повторного камнеобразования. Отсутствие подобных 
аналитических систем для мониторинга больных МКБ послужило основанием для разработки 
опытного образца ИААС 
<<Мегалит>>. Создание системы осуществляется ИПИ РАН совместно с НИИ урологии 
Минздравсоцразвития России в рамках серии совместных на\-уч\-но-ис\-сле\-до\-ва\-тель\-ских работ.
      
\subsection*{Основные цели и~задачи создания информационно-аналитической автоматизированной системы
 <<Мегалит>>}

      \noindent
      \begin{enumerate}[1.]
      \item  Создание БД по результатам обследования пациента, включающей:\\[-15pt]
      \begin{itemize}
\item формализованные данные опроса пациента при первом и последующих визитах, 
содержащие информацию о факторах, способных оказывать влияние на возникновение и 
особенности клинического течения МКБ (lifestyle-фак\-то\-ры индивида, факторы среды, 
питания, профессии и проч.);\\[-15pt]
\item данные лабораторного обследования (результатов простого или расширенного 
лабораторного обследования).\\[-15pt] 
\end{itemize}
      \item  Создание аналитической подсистемы, обеспечивающей решение следующих 
задач:\\[-15pt]
      \begin{itemize}
\item на основании данных первичного опроса выявление наличия или отсутствия, а также 
степень риска развития МКБ и определение объема предполагаемого лабораторного 
обследования пациента (простое или расширенное обследование);\\[-15pt]
\item на основе анализа входных данных лабораторного обследования осуществление выбора 
дальнейшей тактики ведения больного~--- дополнительные виды исследования, выбор 
лечебных мероприятий (тип хирургического лечения, схема медикаментозной терапии, 
коррекция диеты и проч.);\\[-15pt]
\item реализация методов оптимального выбора (с учетом показаний и противопоказаний) вида 
хирургического лечения или схемы проведения профилактического лечения (включая прием 
специальных фармпрепаратов, рекомендации по модификации диеты и образа жизни).
\end{itemize}
\end{enumerate}

        При создании опытного образца ИААС <<Мегалит>> учитывались следующие 
требования.
      \begin{enumerate}[1.]
\item Опытный образец аналитической системы <<Мегалит>> должен иметь возможность 
ведения распределенной БД пациентов, содержащей результаты обследований, профили 
МФР и относительный индекс перенасыщенности мочи (ОИП) как 
исходные, так и измененные в результате назначенного лечения, и включать набор подсистем, 
вклю\-ча\-ющих программную реализацию разработанных методов диагностирования и лечения.
\item Данные простого лабораторного обследования пациента должны включать: 
\begin{itemize}
\item исследование химического состава мочевого камня; 
\item биохимическое исследование крови и мочи по различным параметрам; 
\item клинический анализ мочи с посевом на мик\-ро\-флору; 
\item обзорный рентгеновский снимок, сонограмму и другие виды инструментального 
обследования пациента.
\end{itemize}
%\end{enumerate}
       Данные биохимического исследования представлены величинами содержания в крови 
и моче ионов и соединений, способных приводить к образованию мочевых камней. При 
наличии патологических отклонений в биохимических исследования проводится расширенное 
лабораторное обследование.
 \item Данные расширенного лабораторного обследования включают протокол диагностики 
типа гиперкальциурии (ПД-ГКУ) (при выявлении повышен\-ной суточной экскреции кальция у 
пациента). Выполняется поэтапно, с помощью модифицированной по кальцию диеты. 
В~расширенное лабораторное обследование входит также полный диагностический протокол (ПДП)
больного МКБ. 
\item Полный диагностический протокол пред\-став\-ля\-ет собой выраженное в 
графическом виде исходное состояние обмена веществ у пациента с МКБ с выявленными 
МФР и динамику изменения показателей обмена веществ 
в результате проводимого лечения. Графическое отображение МФР и ОИП больного МКБ 
позволяет оценить степень выявленных нарушений и их динамику в процессе 
профилактического лечения и вносить в лечебную схему необходимые коррекции, также 
выбираемые по особому алгорит\-му. Выявленные при первичном обследовании МФР и ОИП 
служат основой для програм\-мно\-го выбора схем коррекции метаболических нарушений и 
предупреждения рецидивов МКБ. Коррекция включает в себя лечебные мероприятия, прием 
специальных фармпрепаратов, рекомендации по модификации диеты и образа жизни.
\item В~аналитической системе <<Мегалит>> пред\-усмат\-ри\-ва\-ет\-ся возможность ее обучения и 
настройки на основе получаемых новых данных о результатах лечения пациента (пациентов) 
на \mbox{каждом} этапе наблюдения.
\item Предусмотреть в программной реализации алгоритмов экспертного модуля: 
\begin{itemize}
\item
алгоритм оценки эффективности выбранной схемы лечения;
\item
алгоритм принятия решения по дальнейшему лечению;
\item
алгоритм поиска и выбора рациональной схемы профилактического лечения.
\end{itemize}
\item Оценить возможности разработки метода корректировки параметров подсистемы 
диагностирования и лечения на основе анализа вновь поступающих данных (обратная связь).
\item Разработанные аналитические методы и алгоритмы, реализованные в составе опытного 
образца аналитической системы <<Мегалит>>, должны пройти апробацию и тестирование в 
реальных клинических условиях. По результатам применения опытного образца должны быть 
сформулированы рекомендации по его совершенствованию и развитию.
\end{enumerate}

\begin{figure*} %fig1
   \vspace*{1pt}
 \begin{center}
 \mbox{%
 \epsfxsize=143.69mm
 \epsfbox{such-1.eps}
 }
 \end{center}
 \vspace*{-6pt}
\Caption{Структура ИААС}
\end{figure*}

\section{Основные подходы к~созданию информационно-аналитической
автоматизированной системы <<Мегалит>> и~их~реализация}
      Основные функции ИААС:
      \begin{itemize}
\item сбор и формализация данных, включая ведение реестра пациентов, системы словарей и 
справочников;
\item поддержка принятия решения по назначению и сбор данных диагностических 
исследований;
\item первичный и ретроспективный анализ тестов;
\item поддержка принятия решения по выбору схемы лечения, оценка эффективности схемы 
лечения;
\item поддержка принятия решения по дальнейшему лечению;
\item поиск и поддержка принятия решения по выбору рациональной схемы 
профилактического лечения.
\end{itemize}

      Информационно-аналитическая автоматизированная система
       <<Мегалит>> включает в себя подсистемы:
      \begin{itemize}
\item администрирования;
\item регистрации пациентов и сбора данных анкет, анамнеза;
\item ведения лингвистического обеспечения;
\item первичного обследования;
\item диагностических исследований;
\item экспертный модуль (поддержки процессов лечения).
\end{itemize}

Структурная схема ИААС <<Мегалит>> пред\-став\-ле\-на на рис.~1.


\begin{figure*}[b] %fig2
   \vspace*{1pt}
 \begin{center}
 \mbox{%
 \epsfxsize=161.589mm
 \epsfbox{such-2.eps}
 }
 \end{center}
 \vspace*{-6pt}
\Caption{Пирамида анализа данных}
\end{figure*}


      Для обеспечения возможности коллективной работы по формированию БД
системы и многопользовательского режима работы с ее аналитическим модулем она 
проектируется в виде веб-сай\-та, доступного авторизованным пользователям в сети Интернет. 
Основные базовые функции информационного сайта должны быть реализованы 
общесистемным функционалом его платформы. 
%
Таким образом, 
в про\-грам\-мно-тех\-но\-ло\-ги\-че\-ской платформе должны быть заложены следующие функции:
      \begin{enumerate}[(1)]
\item выполнение приложений~--- позволяет легко разрабатывать, развертывать различные 
приложения и управлять ими;
\item возможность совместной работы~--- позволяет отдельным пользователям и крупным 
организациям объединить свои ресурсы и работать вместе через Интернет;
\item управление содержимым~--- придает гибкость производству и управлению 
отдельными веб-уз\-ла\-ми, позволяя поставлять конечному пользователю 
приспособленное под него (персонифицированное) содержимое сайта;
\item управление пользователями~--- позволяет организации управлять пользователями, 
ресурсами и безопасностью внутри и вне системы сетевой защиты, а также предоставлять 
канал для внешних связей и проведения электронных транзакций;
\item контроль и управление про\-из\-во\-ди\-тель\-ностью~--- позволяет улучшать качество 
пользовательского интерфейса, обеспечивая:
\begin{itemize}
\item управление знаниями~--- помогает объединять внутреннюю и внешнюю 
информацию и предоставлять информацию, основанную на контекстной 
концепции;
\item поддержку поиска~--- обеспечивает клиента доступом к широкому спектру 
источников информации как внутри, так и вне сайта;
\item безопасность~--- защиту данных, приложений и транзакций;
\item стандартный www-до\-ступ к сайту~--- для технического обеспечения 
функционирования его содержимого.
\end{itemize}
\end{enumerate}
      
      Определяющими характеристиками веб-ре\-шений являются масштабируемость, 
доступность, надежность, защита данных от несанкционированного доступа, транзакционная 
целостность и распространение.
      
      Важной особенностью платформы является то, что она объединяет все необходимые 
модули, которые позволяют выполнять практически любую работу, связанную с созданием и 
обновлением сайта специалистом предметной области. 
      
      В качестве языка программирования выбран один из самых современных языков~--- 
C\#. ASP.NET~--- технология, которая является частью .NET и используется для разработки 
ин\-тер\-нет-ори\-ен\-ти\-ро\-ван\-но\-го программного обеспечения и ин\-тер\-нет-сай\-тов. 

Для работы ин\-тер\-нет-сай\-тов используется связка: операционная система Windows Server 
2008\;+\;ин\-тер\-нет-сер\-вер IIS~7.0\;+\;СУБД Ms SQL Server~2008.

\section{Концепция экспертного модуля системы}

\subsection{Основные подходы к~использованию статистических методов анализа данных 
в~урологии}

      Клинические БД содержат большое количество информации о пациентах и их 
заболеваниях. Скрытые (латентные) связи и структуры в этих данных могут быть источником 
новых медицинских знаний. К~сожалению, немногие из существующих технологий анализа 
данных оказываются непосредственно применимыми и действенными при обнаружении и 
описании этих латентных знаний, но, безусловно, универсальной из них является технология 
на принципах \textit{Data Mining}~--- извлечение скрытой информации из уже накопленных и 
пополняемых сведений об объекте исследования. Эта и ряд других сформировавшихся 
технологий, ориентированных на анализ массивов данных, терминологически пересекаются 
или оказываются взглядом на одном и том же, но с разных точек зрения (краткое освещение 
данного вопроса дано в~[17, разд.~1]). В~первую очередь речь идет о следующих 
подходах:
      \begin{itemize}
\item разведочный анализ данных~--- Exploratory Data Analysis (EDA);
\item извлечение скрытой информации из данных~--- Data Mining 
(DM);
\item обнаружение знаний в данных~--- Knowledge Discovery in Databases (KDD);
\item машинное обучение~--- Machine Learning (ML).
\end{itemize}

      Таким образом, обнаружение в данных ранее не известных, нетривиальных, 
практически полезных и доступных интерпретации знаний, необходимых для поддержки 
принятия решений в различных сферах человеческой деятельности, составляет суть DM. 
Говоря далее об анализе данных, будем понимать при этом цели, задачи, технологии, методы и 
алгоритмы, присущие DM. 
      
      На рис.~2 схематично изображена иерархия содержательной стороны анализа данных. 
Вертикальная стрелка слева показывает направление роста отдельных характеристик задач 
анализа данных в зависимости от их уровня. Примеры постановок практических задач 
приведены справа. Надо понимать, что <<восхождение>> по пирамиде анализа данных 
должно обеспечиваться обязательным существенным ростом объема используемой 
информации (данных и предположений об объектах исследования), а также глубиной 
проработки вопроса о качестве предлагаемых решений.


      
      \textbf{Основные принципы анализа данных.} Среди методов, которые 
использовались при решении проб\-ле\-мы обучения в ML, те, которые представляют\linebreak 
наибольший интерес при анализе данных (снижение размерности, оценивание распределения 
данных, регрессионный анализ, классификация, клас\-те\-ри\-за\-ция), теперь все вместе 
упоминаются как статисти\-ческое обучение. Проблема обучения делится на различные 
категории: две из них, наиболее близкие к статистике, суть контролируемое обучение или 
обучение с <<учителем>> и не\-конт\-ро\-ли\-ру\-емое, без <<учителя>>.
      
      Одна из самых важных задач в анализе данных состоит в том, чтобы оценить качество 
полученных решений, в частности точность предложенного прогноза (например, качество 
построенного классификатора). В~качестве меры точности прогноза обычно используется 
ошибка прогноза. Простейшая оценка ошибки прогноза строится с помощью тех же данных, 
которые используются для построения модели (такой вариант оценки называют самооценкой, 
оценкой переподстановки). Понятно, что в результате сформируется чрезмерно 
оптимистический взгляд на точность прогноза. 
      
      Очевидный способ улучшения состоит в обобщении: оценивать точность прогноза с 
помощью данных, независимых от тех, которые использовались для подгонки модели. 
Получить подобные независимые данные можно путем сбора новых данных. Если это 
невозможно, то имеет смысл разделить исходные данные на части и воспользоваться ими для 
решения самостоятельных задач. Обычная практика заключается в следующем: если набор 
данных достаточно велик, то необходимо использовать случайный механизм для разделения 
данных на два непересекающихся и независимых набора: 
      \begin{enumerate}[(1)]
\item данные для обучения, которые можно использовать для предварительного контроля 
данных, для формирования моделей;
\item тестирующие данные, которые будут использоваться для оценки 
качества построенной модели.
      \end{enumerate}
      
      Альтернативные методы расщепления данных для того, чтобы оценить тестовую 
ошибку, основаны на перепроверке~[16] и бут\-стреп-ме\-то\-де~[17].
     
     Суть вероятностной модели бутстреп-метода в данном случае состоит в следующем. 
Предположим, что по выборке $x\hm=(x_1,\ldots ,x_N)$ данных лабораторных исследований 
из распределения $F(u)$ оценивается значение $\vartheta\hm=\vartheta(F)$ некоторого 
функционала (например, классификатора заболеваний), заданного на семействе~$\mathbf{F}$. 
Качество оценки $\vartheta^*(X)$ характеризуется величиной
     $$
     R(\vartheta^*(X),\vartheta(F))=E_F\{L(\vartheta^*(X),\vartheta(F))\}\,,
     $$
где $L(\vartheta^*(X),\vartheta(F))$~--- потери от принятия оценки $\vartheta^*(X)$ вместо 
неизвестного значения $\vartheta(F)$. Бут\-стреп-ме\-тод позволяет оценить $ 
R(\vartheta^*(X),\vartheta(F))$ с помощью замены распределения~$F$ его некоторой оценкой 
$F^B$ и вычисления статистики $\vartheta^*$ по выборке $x^B$ объемом~$N$ из~$F^B$. 
Совокупность $x^B$ называется бут\-стреп-вы\-бор\-кой, статистика $\vartheta^*(x^B)$~--- 
бут\-стреп-реа\-ли\-за\-ци\-ей~$\vartheta^*$. 
     
     Условное распределение
     \begin{multline*}
     \mathrm{Pr}\left\{ \vartheta^*\left( X^B\right) <u\vert x_1,\ldots , x_N\right\} = {}\\
     {}=
     \int\limits_{\{y:\ \vartheta^*(y)<u\}} dF^B (y_1)\cdots dF^B(y_N)
     \end{multline*}
является бут\-стреп-оцен\-кой функции распределения $\mathrm{Pr}\left\{ 
\vartheta^*(X)<u\right\}$ статистики~$\vartheta^*$. 
     
     Процедура выбора оценки $F^B$ для~$F$ мотивируется наличием априорной 
информации. В~параметрической ситуации, когда $\mathbf{F}\hm= \left\{ F_\lambda, \, 
\lambda\in \Lambda\right\}$, оценка $F^B$ часто оказывается результатом подстановки 
вместо~$\lambda$ некоторой оценки~$\lambda^*$, т.\,е.\ $F^B\hm=F_{\lambda^*}$. 
{\looseness=1

}

Другая  ситуация относится к области непараметрической статистики. Здесь $F^B$ обычно 
оказывается эмпирической функцией распределения, т.\,е.\ каждому наблюденному значению 
(элементу исходной выборки) приписывается вероятность $1/N$. Бут\-стреп-вы\-бор\-ки тогда 
подчиняются условному полиномиальному распределению, сосредоточенному на  $x_1,\ldots , 
x_N$. 
     
     Наиболее трудную часть бутстреп-метода со\-став\-ля\-ет нахождение распределения 
$\vartheta^*(X^B)$, для чего применяются три приема:
    \begin{enumerate}[(1)]
\item прямое теоретическое вычисление;
\item аппроксимация с помощью метода статистических испытаний;
\item аппроксимация с помощью аналитических методов (например, используя разложение в 
ряд Тейлора).
\end{enumerate}

     Прямое теоретическое вычисление распределения $\vartheta^*(X^B)$ может 
осуществляться либо аналитическим путем, либо путем непосредственного перечис\-ле\-ния 
     бут\-стреп-вы\-бо\-рок и подсчета соответствующих вероятностей. Если оба приема 
недоступны (первый из-за аналитических сложностей, второй из-за вычислительных), то 
приходится прибегать к методу статистических испытаний, т.\,е.\ к повторению экспериментов 
по случайному формированию бут\-стреп-вы\-бор\-ки $x^B$ и подсчету значения 
$\vartheta^*(x^B)$.
     
     Следует обратить внимание на реальные возможности бут\-стреп-ме\-то\-да: он не 
позволяет получить новую информацию о наблюдаемых объектах, его назначение~--- 
сформировать объективное представление о свойствах использованных процедур анализа 
данных.
      
      В аналитической системе <<Мегалит>> накапливаются данные следующих типов:
      \begin{enumerate}[1.]
\item Неформализованные (неструктурированные), представленные в виде текста (например, 
текст назначения врача).
\item Формализованные:
\begin{enumerate}[{2.}1.]
\item Качественные:
\begin{enumerate}[{2.1.}1.]
\item Измеренные по шкале наименований (например, пол пациента).
\item Измеренные по порядковой шкале (например, порядковый номер сезона, когда 
обследовался пациент).
\end{enumerate}
\item Количественные:
\begin{enumerate}[{2.2.}1.]
\item Измеренные по одной из соответствующих шкал и при\-ни\-ма\-ющие значения из 
небольшого \mbox{набора} числовых значений (например, дата взятие анализов или 
количество обнаруженных у пациента камней).
\item Измеренные по одной из соответствующих шкал и принимающие значения в виде 
действительных чисел (например, уровень кальция в анализе крови пациента).
\end{enumerate}
\end{enumerate}
\end{enumerate}
      
      Приведенный систематизированный перечень встречающихся типов данных требует 
привлечения разнообразного арсенала средств, таких как лингвистический анализ (п.~1), 
статистический анализ категориальных данных (п.~2.1.1), ранговые процедуры 
(п.~2.1.2), статистический анализ на основе моделей дискретных и непрерывных 
распределений (пп.~2.2.1 и~2.2.2).
      
      Ошибки есть во всех видах БД; к сожалению, встречаются они и в данном 
случае. В~различных прикладных областях накоплен опыт (см., в частности,~[18]), 
позволяющий привести типичный перечень источников ошибок: фальсификация, неполнота, 
несогласованность, дублирование. 
      
      Те ошибки, которые легко обнаружить, вероятнее всего можно найти на стадии 
<<очистки>> данных, более же скрытые, неочевидные могут быть обнаружены только при 
анализе данных. <<Очистка>> данных обычно происходит, когда данные получены и прежде, 
чем они сохраняются в формате только для чтения в хранилище данных. В~частности, должны 
быть исключены ошибки, при которых переменные принимают значения, противоречащие 
естественным ограничениям (например, при описании химического состава камней значения 
отдельных переменных не могут превосходить 100\%). Доля подобных грубых ошибок в 
медицинских исследованиях может превышать~10\%. 
      
      Ошибки недопустимости значений должны быть описаны с помощью логических 
выражений, истинность которых проверяется на этапе <<очистки>>. В~случаях, когда их не 
удается исправить автоматически или автоматизированно, результат должен помечаться 
специальным образом. 
      
      Для данных, уже хранящихся в БД и явля\-ющих\-ся объектом анализа, могут 
быть характерны сле\-ду\-ющие проблемы: 
\begin{itemize}
\item наличие аномальных наблюдений (значения, которые 
существенно отличаются от основной массы наблюдений);
\item пропуски в данных;
\item малочисленность данных (ситуация, когда количество переменных превышает число 
наблюдений).
\end{itemize}
      
      Таким образом, статистический анализ конкретных данных является многоэтапным 
процессом, включающим планирование статистического исследования, организацию сбора 
необходимых статистических данных, первичное описание данных, оценивание характеристик 
данных, проверку статистических гипотез, анализ полученных решений, формулировку 
выводов, составление итоговых документов. 
      
      В основе принципов построения статистического вывода относительно данных лежат 
следующие положения:
      \begin{itemize}
\item при выборе семейства вероятностных распределений, описывающих данные, 
существенную роль играет предварительный анализ данных; последующий итеративный 
процесс уточнения априорных предположений направлен на построение модели, являющейся 
достаточно реалистичной и позволяющей строить содержательные выводы;
\item при построении методов анализа наряду с постановкой задач разработки оптимальных 
процедур и попыткой их решения следует не пренебрегать разумными подходами к созданию 
ка\-ких-ли\-бо процедур с последующим обязательным анализом предлагаемых решений; 
\item завершающим этапом построения методов анализа должен быть количественный или 
качественный анализ влияния на предлагаемые реше\-ния отклонений от априорных 
предположений, при этом исследование качества полученных решений реальнее всего 
проводить с по\-мощью бут\-стреп-ме\-тода.
\end{itemize}

\subsection{Принципиальные возможности создания экспертного модуля системы 
<<Мегалит>>}

      Повседневная деятельность врача требует решения задач интерпретации, диагностики, 
контроля и прогнозирования, т.\,е.\ таких задач, которые могут быть решены с помощью 
систем поддержки принятия решений. Медицина представляет одну из областей человеческой 
деятельности, где знания специалистов трудно формализуемы, однако разработка 
диагностических медицинских систем в настоящее время является актуальной задачей.
      
      При создании экспертного модуля системы <<Мегалит>>, предназначенного для 
поддержки принятия решения по диагностике заболевания, предполагается:
      \begin{itemize}
\item разработать систему представления медицинских знаний (с использованием данных 
анкет, анамнеза, данных инструментальных и лабораторных методов исследования);
\item разработать алгоритм механизма логического вывода (выполнение диагностики типа 
литогенного нарушения обмена веществ; выбора адекватной схемы лечения, оценки 
эффективности заданной схемы лечения с возможностью ее коррекции при дальнейшем 
мониторинге пациента).
\end{itemize}
      
      Система представления медицинских знаний позволяет выделять значимые для 
принятия врачебного решения или постановки медицинского диагноза данные (качественные 
или количественные). Так, анализ качественных данных анамнеза, анкетных данных позволяет 
сделать заключение о силе влияния наследственных, средовых и социальных факторов риска 
развития МКБ; потенциальной активности процесса камнеобразования. Этой же цели служат 
качественные и количественные данные, полученные при инструментальном/лабораторном 
(рентгенологическом, микробиологическом, ультразвуковом или антропометрическом) 
обследовании пациента.
      
      Большой массив количественных данных в виде числовых значений показателей 
получают при биохимическом исследовании. Именно он является основным объектом 
алгоритмизации при разработке экспертного модуля системы <<Мегалит>>. Применение 
этого модуля предназначено для объективной и более точной диагностики метаболического 
литогенного синдрома, выбора адекватной схемы лечения, качественной оценки результатов 
лечения, коррекции лечебной схемы на основе полученных данных в целях выбора 
оптимального лечебного воздействия на нарушенный обмен веществ у пациента.
      
      Учитывая, что указанные процессы являются алгоритмизуемыми, можно полагать, что 
принципиальные возможности создания экспертного модуля для системы <<Мегалит>> 
имеются.

\subsection{Качественное описание алгоритмической базы экспертного~модуля }

      Качественное описание алгоритмической базы включает: постановку задачи, описание 
входных и выходных данных, описание вход\-ных/вы\-ход\-ных форм пользовательского 
интерфейса, описание событий и реакций системы в рамках поддержки процесса оценки 
эффективности.
      
      \subsubsection*{Алгоритм оценки эффективности заданной схемы лечения}
      
      \paragraph*{Постановка задачи.} Качественно и количественно оценить эффективность 
применения выбранной схемы лечения (с выводом о продолжении ее использования в 
лечении; ее модификации в той или иной степени; замены на другую схему лечения).
      
      \paragraph*{Входные данные.} Входными данными служат те количественно измененные 
биохимические показатели, которые на предыдущем этапе про\-грам\-мно\-го анализа были 
определены (диагностированы) как характерные для данного метаболического синдрома. 
      
      \paragraph*{Выходные данные.} Выходными данными служат количественные значения 
биохимических признаков пролеченного метаболического синдрома, полученные при 
лабораторном исследовании после курса лечения. Эти данные должны быть про\-грам\-мно 
проанализированы в сравнении с их исходными (до начала лечения) значениями. 
Используется \textit{алгоритм <<Оценка качества лечебного эффекта>>}, который 
предполагает следующие варианты вывода о качестве лечения:
      \begin{itemize}
\item <<отсутствие эффекта>>;
\item <<слабо выраженный положительный эффект>>;
\item <<выраженный положительный эффект>>;
\item <<слабо выраженный отрицательный эффект>>; 
\item <<выраженный отрицательный эффект>>.
\end{itemize}
      
      \paragraph*{Выходные формы пользовательского интерфейса.} Выходные формы 
пользовательского интерфейса при этом могут быть представлены в виде таб\-ли\-цы со 
значениями биохимических признаков метаболического синдрома до и после лечения. 
Возможна опция отображения исходных данных в виде диаграммы или графика.

      \subsubsection*{Алгоритм принятия решения по дальнейшему лечению}
      
      \paragraph*{Постановка задачи.} Построить алгоритмические правила, позволяющие 
пользователю сделать вывод и принять решение об использовании данной схемы в 
дальнейшем лечении; модификации схемы в той или иной степени; замены данной схемы на 
другую схему лечения.
      
      \paragraph*{Входные данные:}
      \begin{itemize}
\item данные сравнительного анализа численных биохимических величин до и после лечения;
\item данные качественной оценки лечебного эффекта, получаемые в результате обработки 
данных сравнительного анализа численных биохимических величин до и после лечения.
     \end{itemize}
     
     При формировании и сборе данных первого типа выполняется процедура сравнения 
достигнутых в результате лечения величин значимых для данного метаболического синдрома 
показателей с их исходными значениями, диагностированными до начала лечения в ходе 
биохимического лабораторного исследования.
     
     Данные второго типа являются качественными, производными от данных первого типа. 
Эти данные представляют собой возможные варианты вывода о качестве лечения.
     
     \paragraph*{Выходные данные.} Выходными данными алгоритма принятия решения по 
дальнейшему лечению служат установленные типы рекомендаций по применявшейся схеме 
лечения: 
     \begin{itemize}
\item сохранение схемы без изменений и продолжение лечения;
\item модификация схемы трех степеней выраженности 
(незначительная, умеренная, существенная);
\item отказ от применения данной схемы и выбор новой схемы лечения.
\end{itemize}

\section{Перспективы развития информационно-аналитической автоматизированной
системы~<<Мегалит>>}

      Таким образом, разработана методологическая и техническая база для экспертной 
системы комплексной диагностики и профилактического \mbox{лечения} пациентов с МКБ. Учитывая 
не только медицинскую, но и социальную актуальность проб\-ле\-мы МКБ, а также трудности 
принятия врачебного решения в выборе адекватной тактики противорецидивного лечения 
этого заболевания, следует считать целесообразным создание технологий расширения и 
совершенствования функционала экспертного модуля системы <<Мегалит>> на основе 
анализа вновь поступающих данных с использованием принципа обратной связи.
      
      На следующих этапах планируется работа по оптимизации практического применения 
ИААС <<Мегалит>> в 
клинической урологии.
      
      Предложенные математические методы и алгоритмы найдут применение при 
дальнейшем развитии фундаментальных научных исследований в области разработки 
математических методов моделирования медико-биологических систем, а также при создании 
необходимого математического инструментария.
    %  
      В первую очередь это касается постановки развернутого диагноза, наиболее полно 
отражающего особенности метаболического типа конкретного больного (обследуемого), 
специфику функционального состояния почек и мочевых путей пациента. При этом также 
учитывается влияние различных модифицирующих факторов, таких как наличие или 
отсутствие инфекции и степени ее вы\-ра\-жен\-ности, воздействие социальных факторов, 
факторов питания, наследственности и~др. 
      
      Внедрение и практическое использование сис\-те\-мы <<Мегалит>> позволит 
сформировать представительный набор данных, на основе которого разработать методы 
выбора оптимальной лечебной тактики. Таким образом, конечные пользователи получат 
возможность дистанционной диагностики метаболических литогенных синдромов у пациента, 
оценки степени риска развития МКБ, выбора адекватных терапевтических схем лечения МКБ 
и/или профилактики рецидивов камнеобразования, ввода данных о пациенте в единый банк 
данных для последующего мониторинга и~др. 

{\small\frenchspacing
{%\baselineskip=10.8pt
\addcontentsline{toc}{section}{Литература}
\begin{thebibliography}{99}

\bibitem{1-su} %1
\Au{Ramello, A., Vitale C., Marangella D.} Epidemiology of nephrolithiasis~// J.~Nephrol., 2000. 
Vol.~13. Suppl.~3. P.~45--50.

\bibitem{4-su} %2
\Au{Trinchieri A., Coppi F., Montanari~E., Del Nero~A., Zanetti~G., Pisani~E.} Increase in the 
prevalence of symptomatic upper urinary tract stones during the last ten years~// Eur. Urol., 2000. 
Vol.~37. P.~23--25.

\bibitem{2-su} %3
\Au{Pearle M.\,S., Calhoun E.\,A., Curhan~G.\,C.} Urologic diseases in America project: 
Urolithiasis~// J.~Urology, 2005. Vol.~173. P.~848--857. 

\bibitem{3-su} %4
\Au{Lieske J.\,C., Pena de la Vega~L.\,S., Slezak~J.\,M., Bergstralh~E.\,J., Leibson~C.\,L., 
Ho~K.\,L., Gettman~M.\,T.} Renal stone epidemiology in Rochester, Minnesota: An update~// Kidney 
Int., 2006. Vol.~69. No.\,4. P.~760--764.


\bibitem{6-su} %5
\Au{Johnson C.\,M., Wilson D.\,M., O'Fallon~W.\,M., Malek~R.\,S., Kurland~L.\,T.} 
Renal stone epidemiology: A~25-year study in Rochester, Minnesota~// Kidney Int., 1979. Vol.~16. 
No.\,5. P.~624--631.

\bibitem{5-su} %6
\Au{Yasui T., Iguchi M., Suzuki~S., Kohri~K.} Prevalence and epidemiological characteristics of 
urolithiasis in Japan: National trends between 1965 and 2005~// Urology, 2008. Vol.~71. No.\,2. 
P.~209--213.

\bibitem{7-su}
Заболеваемость населения России в 2003~году: Статистические материалы.~--- М., 2004 
(электронная версия МЗ и СР РФ и ЦНИИ организации и информатизации здравоохранения 
МЗ и СР РФ). 
{\sf http:// www.minzdravsoc.ru/docs/mzsr/stat/17}.
\bibitem{8-su}
\Au{Аполихин О.\,И., Сивков А.\,В., Солнцева Т.\,В., Комарова~В.\,А.}
Анализ урологической заболеваемости в Российской Федерации в 2005--2010~годах~//
Экспериментальная и клиническая урология, 2012. №\,2. C.~4--12.
{\sf http://ecuro.ru/article/analiz-urologicheskoi-zabolevaemosti-v-rossiiskoi-federatsii-v-2005-2010-godakh}.
%\bibitem{9-su}
%Заболеваемость населения России в 2007~году: Статистические материалы.~--- М., 2008 
%(электронная версия МЗ и СР РФ и ЦНИИ организации и информатизации здравоохранения 
%МЗ и СР РФ). {\sf http:// www.minzdravsoc.ru/docs/mzsr/stat/27}.
\bibitem{11-su} %9
\Au{Routh J.\,C., Graham D.\,A., Nelson~C.\,P.} Epidemiological trends in pediatric urolithiasis at 
United States freestanding pediatric hospitals~// J.~Urology, 2010. Vol.~184. No.\,3. P.~1100--1104.

\bibitem{10-su} %10
\Au{Голованов С.\,А., Дрожжева В.\,В.}
Кристаллообразующая активность мочи при оксалатном уролитиазе~//
Экспериментальная и клиническая урология, 2010. №\,2. C.~24--29.
{\sf http://ecuro.ru/ article/kristalloobrazuyushchaya-aktivnost-mochi-pri-oksalatnom-urolitiaze}.
\bibitem{12-su} %11
\Au{Bonny O., Rubin A., Huang~Ch.-L., Frawley~W.\,H., Pak~C.\,Y.\,C., Moe~O.\,W.} Mechanism 
of urinary calcium regulation by urinary magnesium and pH~// J.~Am. Soc. Nephrol., 2008. Vol.~19. 
No.\,8. P.~1530--1537.
\bibitem{13-su} %12
\Au{Дюк В.\,А., Эмануэль В.\,Л.} Информационные технологии в 
ме\-ди\-ко-био\-ло\-ги\-че\-ских исследованиях.~--- СПб.: Питер, 2003. 525~с.
\bibitem{14-su} %13
\Au{Liew P.\,L., Lee Y.\,C., Lin~Y.\,C., \textit{et al}.} Comparison of artificial neural networks with 
logistic regression in prediction of gallbladder disease among obese patients~// Digest. Liver Dis., 2007. 
Vol.~39. No.\,4. P.~356--362.
\bibitem{15-su} %14
\Au{Bassi P., Sacco E., De Marco~V., \textit{et al}.} Prognostic accuracy of an artificial neural 
network in patients undergoing radical cystectomy for bladder cancer: A~comparison with logistic 
regression analysis~// BJU Int., 2007. Vol.~99. No.\,5. P.~1007--1012.
\bibitem{16-su} %15
\Au{Stephan C., Xu C., Finne~P., \textit{et al}.} Comparison of two different artificial neural 
networks for prostate biopsy indication in two different patient populations~// J.~Urology, 2007. 
Vol.~70. No.\,3. P.~596--601.
%\bibitem{17-su} 
%\Au{Chun F.\,K., Karakiewicz P.\,I., Briganti~A., \textit{et al}.} A~critical appraisal of logistic 
%regression-based nomograms, artificial neural networks, classification and regression-tree models, 
%look-up tables and risk-group stratification models for prostate cancer ~/ BJU Intern., 2007. Vol.~99. 
%No.\,4. P.~794--800.


\bibitem{19-su} %16
\Au{Stone M.} Cross-validatory choice and assessment of statistical predictions (with discussion)~// 
J.~Roy. Stat. Soc. B, 1974. Vol.~36. P.~111--147.

\bibitem{18-su} %17
\Au{Efron B.} Bootstrap methods: Another look at the jackknife~// Ann. Stat., 1979. Vol.~7. 
P.~1--26.

\bibitem{21-su} %18
\Au{Izenman A.\,J.} Modern multivariate statistical techniques.~--- Springer, 2008. 731~p. 
%\bibitem{20-su} %20
%\Au{Breiman L.} The 1991 census adjustment: Undercount or bad data~// Stat. Sci., 1994. 
%Vol.~9. P.~458--475.


\end{thebibliography} } }

\end{multicols}

\hfill{\small\textit{Поступила в редакцию 17.04.13}}
%\vspace*{12pt}

%\hrule

%\vspace*{2pt}

%\hrule

\newpage

\def\tit{THE INFORMATION-ANALYTICAL COMPUTER SYSTEM ``MEGALITH'' 
IN~OPTIMIZATION OF~THE~DIAGNOSIS AND~TREATMENT OF~UROLITHIASIS}

\def\titkol{The information-analytical computer system ``Megalith'' 
in~the~field of~urology}

\def\aut{M.\,P.~Krivenko$^1$, S.\,A.~Golovanov$^2$, P.\,A.~Savchenko$^1$, A.\,V.~Sivkov$^2$, 
 and~A.\,P.~Suchkov$^1$}
 
 \def\autkol{S.\,A.~Golovanov, M.\,P.~Krivenko, P.\,A.~Savchenko, et al.}


\titel{\tit}{\aut}{\autkol}{\titkol}

\vspace*{-12pt}


\noindent
$^1$Institute of Informatics 
Problems, Russian Academy of Sciences, Moscow 119333, Russian Federation\\
\noindent $^2$Research Institute of Urology, Moscow 105425, Russian Federation

\vspace*{12pt}

\def\leftfootline{\small{\textbf{\thepage}
\hfill INFORMATIKA I EE PRIMENENIYA~--- INFORMATICS AND APPLICATIONS\ \ \ 2013\ \ \ volume~7\ \ \ issue\ 4}
}%
 \def\rightfootline{\small{INFORMATIKA I EE PRIMENENIYA~--- INFORMATICS AND APPLICATIONS\ \ \ 2013\ \ \ volume~7\ \ \ issue\ 4
\hfill \textbf{\thepage}}}

\Abste{In this article, that is the first of an expected series of scientific publications, the results of 
research on automation of the information and analytical processes of the urolithic disease (ULD) 
survey, diagnosis, and treatment are discussed. A significant role in creating the systems of ULD 
diagnostics has the development of information technologies for clinical data collection and 
formation of specialized databases. The possibility of creation and the ways of realization of 
information-analytical computer system of collection, storage, and processing of the clinical data of 
patients examination, as well as programming decision-making processes in the diagnosis ULD and 
the choice of schemes of treatment and prevention of this disease has been studied. The developed 
mathematical methods and algorithms may be applied to the further fundamental scientific researches 
in the field of development of mathematical methods of medical and biological systems modeling; 
besides, they may be applied for necessary mathematical tools creation.}

\KWE{informational-analytical system; urology; computer diagnostics; treatment 
scheme; scheme of prevention} 

\DOI{10.14357/19922264130409}

%\Ack
%\noindent
%?????

\vspace*{3pt}

  \begin{multicols}{2}

\renewcommand{\bibname}{\protect\rmfamily References}
%\renewcommand{\bibname}{\large\protect\rm References}

{\small\frenchspacing
{%\baselineskip=10.8pt
\addcontentsline{toc}{section}{References}
\begin{thebibliography}{99}


\bibitem{1-su-1}
\Aue{Ramello, A., C.~Vitale, and D.~Marangella}. 2000. Epidemiology of nephrolithiasis. 
\textit{J.~Nephrol.} 13(Suppl.~3):45--50.

\bibitem{4-su-1} %2
\Aue{Trinchieri, A., F.~Coppi, E.~Montanari, A.~Del Nero, G.~Zanetti, and E.~Pisani}. 2000. 
Increase in the prevalence of symptomatic upper urinary tract stones during the last ten years. 
\textit{Eur. Urol.} 37:23--25.

\bibitem{2-su-1} %3
\Aue{Pearle, M.\,S., E.\,A.~Calhoun, and G.\,C.~Curhan}. 2005. Urologic diseases in America 
project: Urolithiasis. \textit{J.~Urology} 173:848--857. 
\bibitem{3-su-1} %4
\Aue{Lieske, J.\,C., L.\,S.~Pena de la Vega, J.\,M.~Slezak, E.\,J.~Bergstralh; C.\,L.~Leibson, 
K.\,L.~Ho, and M.\,T.~Gettman}. 2006. Renal stone epidemiology in Rochester, Minnesota: An 
update. \textit{Kidney Int.} 69(4):760--768.


\bibitem{6-su-1} %5
\Aue{Johnson, C.\,M., D.\,M.~Wilson, W.\,M.~O'Fallon, R.\,S.~Malek, and L.\,T.~Kurland}. 
1979. Renal stone epidemiology: A~25-year study in Rochester, Minnesota. \textit{Kidney 
Int.} 16(5):624--631.

\bibitem{5-su-1} %6
\Aue{Yasui, T, M.~Iguchi, S.~Suzuki, and K.~Kohri}. 2008. Prevalence and epidemiological 
characteristics of urolithiasis in Japan: National trends between 1965 and 2005. \textit{Urology}  
 71(2):209--213.

\bibitem{7-su-1} %7
Russian Ministry of Health:
Central Research Institute of Organization and Informatization of Population.
2004. Zabolevaemost' naseleniya Rossii v 2003 godu: Sta\-ti\-sti\-che\-skie materialy [Morbidity of 
population of Russia in 2003: Statistical materials]. Мoscow.  Electronic version.
{\sf http://www.minzdravsoc.ru/docs/mzsr/stat/17}.

\bibitem{8-su-1}
\Aue{Apolikhin,~O.\,I., A.\,V.~Sivkov, T.\,V.~Solntseva, and V.\,A.~Komarova.}
2012. Analysis of urological morbidity in the Russian Federation within the period of 2005--2010.
\textit{Experimental and Clinical Urology} 2:4--12.
{\sf http://ecuro.ru/en/article/analysis-urological-morbidity-russian-federation-within-period-2005-2010}.


\bibitem{11-su-1} %9
\Aue{Routh, J.\,C., D.\,A.~Graham, and C.\,P.~Nelson}. 2100. Epidemiological trends in 
pediatric urolithiasis at United States freestanding pediatric hospitals. \textit{J.~Urology} 
184(3):1100--1104.

%\bibitem{9-su-1}
%Russian Ministry of Health:
%Central Research Institute of Organization and Informatization of Population.
%2008. Zabolevaemost' naseleniya Rossii v 2007 godu: Sta\-ti\-sti\-che\-skie materialy [Morbidity of 
%population of Russia in 2007: Statistical materials]. Мoscow. Electronic version.
%{\sf http://www.minzdravsoc.ru/docs/mzsr/stat/27}.
\bibitem{10-su-1} %10
\Aue{Golovanov, S.\,A., and V.\,V.~Drozhzheva}.
2010. Crystal formation activity of urine in oxalate urolithiasis.
\textit{Experimental and Clinical Urology} 2:24--29.
{\sf http://ecuro.ru/en/article/crystal-formation-activity-urine-oxalate-urolithiasis}.



\bibitem{12-su-1} %12
\Aue{Bonny, O., A.~Rubin, Ch.-L.~Huang, W.\,H.~Frawley, C.\,Y.\,C.~Pak, and 
O.\,W.~Moe}. 2008. Mechanism of urinary calcium regulation by urinary magnesium and $pH$. 
\textit{J.~Am. Soc. Nephrol.} 19(8):1530--1537.
\bibitem{13-su-1}
\Aue{Djuk, V.\,A., and V.\,L.~Jemanujel'}. 2003. \textit{Informatsionnye tekhnologii v 
mediko-biologicheskikh issledovaniyakh} [\textit{Information technologies in medical and 
biological researches}]. St.\ Petersburg, Russia: Piter, 2003. 525~p.
\bibitem{14-su-1}
\Aue{Liew, P.\,L., Y.\,C.~Lee, Y.\,C.~Lin, \textit{et al}.} 2007. Comparison of artificial neural 
networks with logistic regression\linebreak\vspace*{-12pt}

\pagebreak

\noindent
 in prediction of gallbladder disease among obese patients. 
\textit{Digest. Liver Dis.} 39(4):356--362.

%\pagebreak


\bibitem{15-su-1}
\Aue{Bassi, P., E. Sacco, V.~De Marco, \textit{et al}.} 2007. Prognostic accuracy of an 
artificial neural network in patients undergoing radical cystectomy for bladder cancer: 
A~comparison with logistic regression analysis. \textit{BJU Int.}  99(5):1007--1012.
\bibitem{16-su-1}
\Aue{Stephan, C., C.~Xu, P.~Finne, \textit{et al}.} 2007. Comparison of two different artificial 
neural networks for prostate biopsy indication in two different patient populations. 
\textit{J.~Urology}  70(3):596--601.

%\columnbreak

\bibitem{19-su-1} %18
\Aue{Stone, M.} 1974. Cross-validatory choice and assessment of statistical predictions (with 
discussion). \textit{J.~Roy. Stat. Soc. B} 36:111--147.
%\bibitem{17-su-1}
%\Aue{Chun, F.\,K., P.\,I.~Karakiewicz, A.~Briganti, \textit{et al}.} 2007. A~critical appraisal 
%of logistic regression-based nomograms, artificial neural networks, classification and 
%regression-tree models, look-up tables and risk-group stratification models for prostate cancer. 
%\textit{BJU Intern}.  99(4):794--800.



\bibitem{18-su-1} %19
\Aue{Efron, B.} 1979. Bootstrap methods: Another look at the jackknife. \textit{Ann.  
Stat.}  7:1--26.

\bibitem{21-su-1} %17
\Aue{Izenman, A.\,J.} 2008. \textit{Modern multivariate statistical techniques}. Springer. 
731~p.


 
 
 
% \bibitem{20-su-1} %20
%\Aue{Breiman, L.} 1994. The 1991 census adjustment: Undercount or bad data.  
%\textit{Stat. Sci.}  9:458--75.
 

\end{thebibliography}
} }

\end{multicols}

\hfill{\small\textit{Received April 17, 2013}}

\Contr

\noindent
\textbf{Krivenko Michail P.} (b.\ 1946)~--- 
Doctor of Science in technology, principal scientist, Institute of Informatics 
Problems, Russian Academy of Sciences, Moscow 119333, Russian Federation;  mkrivenko@ipiran.ru

\vspace*{3pt}


\noindent\textbf{Golovanov Sergey  A.} (b.\ 1950)~--- Doctor of Science in medicine, Head of 
Laboratory, Research Institute of Urology, Moscow 105425, Russian Federation;
sergeygol124@mail.ru
 

\vspace*{3pt}

\noindent
\textbf{Savchenko Pavel A.} (b.\ 1967)~--- software engineer, Institute of Informatics 
Problems, Russian Academy of Sciences, Moscow 119333, Russian Federation;  
psavchenko@ipiran.ru

\vspace*{3pt}

\noindent
\textbf{Sivkov Andrey V.} (b.\ 1957)~--- Doctor of Science in medicine, Deputy director, 
Research Institute of Urology, Moscow 105425, Russian Federation;  uroinfo@yandex.ru

\vspace*{3pt}

\noindent
\textbf{Suchkov Alexander P.} (b. 1954)~--- Doctor of Science in technology, principal 
scientist, Institute of Informatics Problems, Russian Academy of Sciences, Moscow 119333, Russian Federation;  
asuchkov@ipiran.ru 

\label{end\stat}

\renewcommand{\bibname}{\protect\rm Литература}   %9
\def\stat{krivenko}

\def\tit{МНОГОМЕРНЫЙ РЕФЕРЕНСНЫЙ РЕГИОН\\ ВЫСОКОЙ ПЛОТНОСТИ}

\def\titkol{Многомерный референсный регион высокой плотности}

\def\aut{М.\,П.~Кривенко$^1$}

\def\autkol{М.\,П.~Кривенко}

\titel{\tit}{\aut}{\autkol}{\titkol}

\index{Кривенко М.\,П.}
\index{Krivenko M.\,P.}


%{\renewcommand{\thefootnote}{\fnsymbol{footnote}} \footnotetext[1]
%{Работа выполнена при финансовой поддержке РФФИ (проекты 16-07-00677 
%и~15-37-20611-мол\_а\_вед).}}


\renewcommand{\thefootnote}{\arabic{footnote}}
\footnotetext[1]{Институт проблем информатики Федерального исследовательского центра <<Информатика и~управление>> Российской академии наук,
\mbox{mkrivenko@ipiran.ru}}

\vspace*{4pt}



\Abst{Рассматриваются принципы построения многомерных референсных регионов
(MRR~--- multivariate reference region). 
Предложен оригинальный метод построения региона на основе областей с~высокой 
плотностью точек и~аппроксимации распределения данных с~помощью смеси нормальных 
распределений. Для оценки порога для плотности распределения используется  
бут\-стреп-ме\-тод. В~качестве эксперимента рассмотрена задача построения 
и~использования эталонной области для прогнозирования типа мочевого камня. Обработка 
реальных данных продемонстрировала преимущества предлагаемых решений.}

\KW{многомерный референсный регион; область высокой плотности; бут\-стреп-ме\-тод; 
смесь многомерных нормальных распределений}

\vspace*{6pt}

\DOI{10.14357/19922264170207} 


\vskip 10pt plus 9pt minus 6pt

\thispagestyle{headings}

\begin{multicols}{2}

\label{st\stat}

\section{Введение}

     Многомерный референсный регион 
был предложен в~литературе по клинической химии в~начале 1970-х~гг.\ как 
альтернатива одномерным референсным интервалам~[1]. Там излагались 
преимущества предлагаемых множественных тестов, хоть и~имеющих 
упрощенный вид, но снижающих (по отношению к~одномерным вариантам) 
число ложных положительных результатов. Появление MRR оказалось 
особенно привлекательным для интерпретации результатов наборов 
медицинских тестов. Тем не менее возникали трудности в~построении 
и~использовании процедур многомерного анализа (см., например,~[2]), 
связанные, в~частности, с~быстрым увеличением числа параметров, которые 
должны быть оценены. Немногие лаборатории использовали MRR в~своей 
практике, причем в~экспериментальном режиме, и,~как следствие, на 
сегодняшний день имеется относительно малое количество соответствующих 
публикаций. 

\vspace*{-6pt}

\section{Многомерный референсный регион на основе расстояния Махалонобиса}

\vspace*{-2pt}

     Одномерный референсный интервал, полученный статистическим путем, 
использует центральную часть значений анализируемого показателя, обычно 
соответствующую~95\% некоторой популяции~--- совокупности особей 
определенного вида (например, здоровой части населения определенного пола 
из некоторого диапазона возрастов). Одномерные референсные интервалы 
применялись в~течение многих лет в~качестве стандартного приема 
интерпретации лабораторных данных. Они легко формируются, хранятся, 
извлекаются и~передаются в~лабораторных информационных системах, просты 
в~понимании, хорошо воспринимаются медицинским сообществом в~ходе 
длительного использования. Тем не менее одномерные референсные интервалы 
при классификации данных могут дать большое число ложно аномальных 
результатов. Этот далеко не единственный недостаток однофакторного 
референсного интервала может быть полностью или частично устранен 
с~помощью MRR.
     
     Простейшим и~весьма распространенным способом построения MRR 
является использование прямого произведения отдельных референсных 
интервалов в~предположении, что они статистически независимы. Пусть 
$(1\hm-\alpha)$~--- вероятность попадания в~MRR, а~$p_0$~--- вероятность 
попадания в~референсный интервал для любого из~$d$~признаков, тогда 
$p_0\hm= \sqrt[d]{1-\alpha}$. С~ростом размерности~$d$ значения~$p_0$ 
быстро приближаются к~1, что фактически лишает смысла применение MRR.
     
     Как и~в одномерном случае, отправной точкой для построения MRR 
может стать нормальное распределение. Идеи центрального расположения 
референсного региона и~заданной вероятности попадания в~него приводят для 
$d$-мер\-но\-го нормального распределения, имеющего плотность 
распределения
     \begin{multline*}
     \varphi(y,\mu,\Sigma) ={}\\
     {}=(2\pi)^{-d/2}\vert\Sigma\vert^{-1/2}\exp \left( -\fr{\left(y-
\mu\right)^{\mathrm{T}} \Sigma^{-1}(y-\mu)}{2}\right),
   \end{multline*}
где величина $(y-\mu)^{\mathrm{T}} \Sigma^{-1} (y-\mu)$ есть квадрат так 
называемого расстояния Махаланобиса между~$y$ и~$\mu$, к~использованию 
многомерного эллипсоида
\begin{multline*}
(2\pi)^{-d/2}\vert\Sigma\vert^{-1/2}\exp \left( -\fr{\left(y-\mu\right)^{\mathrm{T}}
\Sigma^{-1} 
(y-\mu)}{2}\right) ={}\\
{}=const
\end{multline*}
или, что то же самое, 
$$ 
(y-\mu)^{\mathrm{T}} \Sigma^{-1}(y-\mu)=const\,.
$$
Его называют эллипсоидом равной плотности распределения (или просто 
эллипсоидом равной вероятности). 
     
     Если задаться вероятностью $(1\hm-\alpha)$ попадания в~эллипсоид 
равной вероятности вида $(y\hm-\mu)^{\mathrm{T}}\Sigma^{-1} (y\hm-\mu)\hm= 
\rho$, то параметр~$\rho$ удовлетворяет уравнению $\mathrm{Pr}\left\{ 
\chi_d^2\leq \rho\right\} \hm=1\hm-\alpha$.
     
     Использование эллипсоида в~качестве MRR будет оправдано только 
тогда, когда исходное распределение данных есть многомерное нормаль-\linebreak ное. 
Поэтому становятся актуальными критерии\linebreak подгонки, а~также использование 
процедур норма\-ли\-зации распределения данных в~многомерном\linebreak случае.
 Если 
с~помощью тестов выявляется, что распределение не является нормальным, то 
Международная федерация клинической химии и~лабораторной медицины 
рекомендует, согласно~[3], использовать двухступенчатую процедуру 
нормализации. Следует обратить внимание, что многошаговость здесь 
относится не к~многомерности, а касается лишь покоординатного 
преобразования распределения данных к~нормальному.
     
     Первые же попытки применения MRR на основе расстояния 
Махалонобиса (фактически это означает принятие модели нормального 
распределения референсных значений) выявили ряд недостатков (более 
подробно смотри в~\cite[разд.~6.2]{4-kri}):
     \begin{itemize}
\item проявление <<проклятий>> размерности при механическом 
увеличении~$d$, в~особенности если игнорируется этап анализа состава 
признаков~[1, 5, 6];
\item из-за небольших объемов обучающей выборки невысокая устойчивость 
при применении, в~частности чувствительность к~увеличению неточностей 
измерений после того, как регион был установлен~\cite{5-kri, 7-kri}. 
\item предположение о нормальном распределении и~попытки <<подправить>> 
действительность с~помощью преобразований реальных данных для их 
нормализации при увеличении размерности данных становятся все более 
шаткими~\cite{5-kri};
\item представление и~интерпретация выводов на основе MRR трудно 
понимаемы не только для специалистов в~предметной области~[8].
\end{itemize}

\vspace*{-9pt}

\section{Многомерный референсный регион высокой плотности}

\vspace*{-2pt}

     Заметим, что в~случае нормального распределения референсных значений 
для точек внут\-ри построенного эллипсоида значения плотности\linebreak распределения 
больше, чем на границе, а~вне~--- меньше. Это замечание позволяет 
предложить другой подход к~построению MRR.
     
     \smallskip
     
     \noindent
     \textbf{Определение.}\ Eсли плотность распределения референсных 
значений есть $f(y)$, то MRR есть область $A_t\hm= \left\{ y\in 
\mathcal{R}^d\vert f(y)\hm\geq t\right\}$ для некоторого порогового 
значения~$t$. 
     
     \smallskip
     
     Для нормального распределения это уже упомянутый эллипсоид равной 
вероятности. Если задается вероятность $(1\hm-\alpha)$ попадания в~$A_t$, то 
пороговое значение~$t$ есть решение уравнения $\int\nolimits_{A_t} 
f(u)\,du\hm=1\hm-\alpha$, получить которое аналитически в~случае 
произвольной плотности распределения вряд ли возможно. Здесь присутствуют 
две проблемы: вычисление многомерного интеграла и~зависимость области 
интегрирования от неизвестного значения. Для решения их предлагается 
привлечь метод моделирования.
     
     Сгенерируем выборку из $f(y)$, которую обозначим как $Y^f\hm= \left\{ 
y_1^f, \ldots, y_m^f\right\}$. Для оценки $\int\nolimits_{A_t} f(u)\,du$ 
используем отношение:

\noindent
\begin{multline*}
     \fr{\left\vert \left\{ y_i^f\vert y_i^f\in A_t\right\}\right\vert }{m} =
      \fr{\left\vert\left\{ y_i^f\vert 
f\left(y_i^f\right) \geq t\right\}\right\vert }{m} ={}\\
{}= 1-\fr{\left\vert \left\{ y_i^f\vert f(y_i^f)<t\right\}\right\vert }{m}=1-
F_m(t)\,,
     \end{multline*}
где $F_m(t)$~--- эмпирическая функция распределения случайной 
величины~$f(y)$, т.\,е.\ случайной величины, являющейся результатом 
преобразования с~помощью функции~$f(\cdot)$ случайной величины, име\-ющей 
плотность распределения~$f(u)$.

     Таким образом, искомая оценка~$t^*$ должна удовле\-тво\-рять уравнению 
$F_m(t^*)\hm=\alpha$ и~может быть получена как непараметрическая оценка 
квантиля\linebreak\vspace*{-12pt}

\pagebreak

\noindent
 порядка~$\alpha$ из распределения $F_m(\cdot)$. Если обозначить 
$f_i\hm= f(y_i^f)$, то~$t^*$ есть~$f_{(r)}$, где
     $$
     r= \begin{cases}
     m\alpha, &\ m\alpha~\mbox{---~целое}\,;\\
     \lfloor m\alpha+1\rfloor\,, & m\alpha~\mbox{--- не целое}\,.
     \end{cases}
     $$
     Заметим, что для такой оценки можно указать доверительный интервал.
     
     Для построения MRR необходимо знать распределение данных. При 
реализации принципа точек высокой плотности в~первую очередь следует 
обратиться к~параметрическим моделям, в~част\-ности к~смеси нормальных 
распределений, име\-ющей плотность распределения
     $$
     f(u) =\sum\limits_{j=1}^k p_j \varphi\left (u,\mu_j, \Sigma_j\right)\,.
     $$
Если $\hat{f}(u)$~--- оценка смеси, то~$t^*$ строится сле\-ду\-ющим образом:
\begin{itemize}
\item генерируется выборка $\left\{ y_1^f,\ldots , y_m^f\right\}$ из $\hat{f}(u)$ и~
для каждого ее $i$-го элемента подсчитывается значение $\hat{f}\left( 
y_i^f\right)$;
\item в~качестве~$t^*$ берется непараметрическая оценка квантиля 
порядка~$\alpha$ (в случае необходимости дополнительно находится 
непараметрическая оценка доверительного интервала для~$t^*$, что 
может характеризовать правильность выбранного объема для 
генерируемой выборки).
\end{itemize}

     Пусть для $f(u)$ имеется~$A_t$, а также получена $\hat{f}(u)$ 
и~соответствующий MRR вида~$\hat{A}_t$. Качество аппроксимации~$A_t$ 
с~по\-мощью~$\hat{A}_t$ можно оценить с~по\-мощью вероятности совпадения 
этих областей, т.\,е. 
     $$
     P_c= \int\limits_{\{ u\in A_t\}\cup \{u\in \hat{A}_t\}} \hspace*{-6mm}
f(u)\,du+\int\limits_{\{u\not\in A_t\} \cup\{ u\not\in \hat{A}_t\}}\hspace*{-6mm} f(u)\,du\,.
     $$
     
     Для оценки  $P_c$ можно использовать величину
     \begin{multline*}
     \hat{P}_c= \fr{\left\vert \left\{ 
     y_i^f\vert y_i^f \in \left\{\left\{ y_i^f\in A_t\right\}\cup \left\{y_i^f\in 
\hat{A}_t\right\}\right\}\right\}\right\vert}{m}+{}\\
{}+\fr{\left\vert \left\{ y_i^f\vert y_i^f \in \left\{\left\{ y_i^f\not\in A_t\right\}\cup 
\left\{ y_i^f\not\in \hat{A}_t\right\}\right\}\right\}\right\vert}{m}\,.
     \end{multline*}
     
     Использование MRR высокой плотности для диагностирования сводится 
к~реализации так называемого слабого критерия значимости для наблюденного 
значения~$x$: нулевая гипотеза заключается в~том, что $x\hm\in A_t$, 
статистика критерия есть $\hat{f}(x)$ и~решение о~принадлежности 
критической об\-ласти~$A_t$ принимается при больших значениях~$\hat{f}(x)$.
     
     Для медицинской практики важна возможность использования 
референсного региона при интерпретации результатов обследования 
некоторого пациента с~вектором признаков~$x$. В~подобных случаях 
сложившейся практикой для слабых критериев значимости является 
использование критического уровня~$\alpha_{\mathrm{cr}}$ (более распространенным 
в~медицине является употребление термина $p$-зна\-че\-ние)  $\alpha_{\mathrm{cr}}\hm= 
\mathrm{Pr}\left\{ \hat{f}(y)\hm\leq \hat{f}(x)\right\}$, где $y$~--- случайная 
величина, имеющая плотность распределения~$\hat{f}(u)$, а $\hat{f}(x)$~--- 
значение плотности распределения~$\hat{f}(u)$ в~точке~$x$. Эта 
характеристика дает представление о~том, насколько сильно данное 
наблюденное значение~$x$ противоречит гипотезе (или подкрепляет ее) 
о~принадлежности данных MRR. При выбранном же заранее уровне 
значимости с~помощью~$\alpha_{\mathrm{cr}}$ сразу же можно принять конкретное 
решение. 

\vspace*{-9pt}

\section{Эксперименты}

\vspace*{-2pt}

     Для демонстрации возможностей MRR использовались данные по 
прогнозу химического состава мочевых камней по метаболическим 
показателям мочи и~сыворотки крови, а также антропологическим 
характеристикам пациентов~[9]. В качестве исходной классификации камней 
рассматривалась следующая: чисто оксалатные (далее обозначены как O), чисто 
уратные (U), чисто фосфатные (P), смесь только оксалатных и~уратных (OU), 
смесь только оксалатных и~фосфатных (OP), смесь только уратных 
и~фосфатных (UP), все остальные. Данная классификация была построена 
в~[10] на основе доминирующих частот встречаемости основных компонентов. 
В~качестве референсных значений рассматривались наборы метаболических 
и~антропологических показателей (их всего было~14), соответствующих 
определенному классу камней.

\begin{table*}\small
\begin{center}


\begin{tabular}{|c|c|c|c|c|c|c|}
\multicolumn{7}{c}{Качество классификации с~помощью MRR}\\
\multicolumn{7}{c}{\ }\\[-6pt]
\hline
\multicolumn{1}{|c|}{\raisebox{-6pt}[0pt][0pt]{\tabcolsep=0pt\begin{tabular}{c}Тип\\ камня\end{tabular}}}&
\multicolumn{1}{c|}{\raisebox{-6pt}[0pt][0pt]{$N$}}&$(1-\alpha)$, 
&\multicolumn{2}{c|}{MRR(5)}&\multicolumn{2}{c|}{MRR(1)}\\
\cline{4-7}
&&&&&&\\[-9pt]
&&\%&$(1-\hat{\alpha})$, \%&$\hat{\beta}$, \%&$(1-\hat{\alpha})$, \%&$\hat{\beta}$, \%\\
\hline
\multicolumn{1}{|c|}{\raisebox{-18pt}[0pt][0pt]{O}}&
\multicolumn{1}{c|}{\raisebox{-18pt}[0pt][0pt]{82}}
&95&100\hphantom{9}&71&90&24\\
&&85&96&78&89&36\\
&&75&91&85&77&44\\
&&65&76&88&74&50\\
\hline
\multicolumn{1}{|c|}{\raisebox{-18pt}[0pt][0pt]{U}}&
\multicolumn{1}{c|}{\raisebox{-18pt}[0pt][0pt]{76}}&95&100\hphantom{9}&75&91&24\\
&&85&99&85&80&35\\
&&75&82&89&74&48\\
&&65&71&91&68&56\\
\hline
\multicolumn{1}{|c|}{\raisebox{-18pt}[0pt][0pt]{P}}&
\multicolumn{1}{c|}{\raisebox{-18pt}[0pt][0pt]{83}}&95&100\hphantom{9}&66&87&25\\
&&85&94&78&86&33\\
&&75&86&82&82&41\\
&&65&77&87&75&47\\
\hline
\end{tabular}
\end{center}
\end{table*}
     
     
     Для каждого из основных классов O, U, P, OU, OP и~UP перед построением 
MRR проводилась селекция признаков и~принималось то значение размерности 
признакового пространства~$d$ и~соответствующий набор показателей, 
которые позволяли прогнозировать состав камней без потери качества 
(методика описана в~\cite{9-kri} и~привела к~значению $d\hm=9$). В~качестве 
модели данных в~первую очередь рассматривалась смесь многомерных 
нормальных распределений из пяти элементов (подбор числа элементов смеси 
проводился с~по\-мощью AIC~--- Akaike information criterion), для соответствующего региона было принято 
обозначение MRR(5). Для сравнения также использовалась модель 
нормального распределения, которой соответствовал MRR(1). Полученные 
результаты приводятся час\-тич\-но в~таблице, где $N$~--- объем 
классифицируемых данных; $\hat{\alpha}$~--- оценка для~$\alpha$; 
$\hat{\beta}$~--- оценка мощности критерия при определении типа камня на 
основании MRR.


     Одной из базовых характеристик является вероятность попадания в~MRR 
$(1\hm-\alpha)$ и~ее оценка $(1\hm-\hat{\alpha})$. Сравнение соответствующих 
столбцов с~учетом значений~$N$ и~ориентировочных значений разброса 
(стандартные отклонения на основе биномиального распределения) не 
позволило выявить явных отклонений. Необходимо, правда, отметить, что во 
всех проанализированных случаях для MRR(5) оказалось, что $1\hm-
\hat{\alpha}\hm\geq 1\hm-\alpha$.
     
     Назначение MRR, заключающееся в~сжатом представлении референсных 
значений, в~многомерном случае практически не проявляется. Для задания 
MRR(5) необходимо указать следующие величины: $1\hm-\alpha$, $t$, 
$p_1,\ldots, p_{k-1}$, $\mu_1, \Sigma_1,\ldots , \mu_k,\Sigma_k$, общее 
количество которых равно  $[2\hm+ (k\hm-1)\hm+ k(d\hm+ d(d\hm+1)/2)]$ 
и,~в~частности, в~рассматриваемых экспериментах~--- 276. Для MRR(1) это 
значение меньше и~равно~56. При этом для обрабатываемой обучающей 
выборки в~зависимости от класса камней речь идет о~порядка~10$^2$ векторах 
данных (см.\ столбец со значениями~$N$), что приблизительно 
дает~10$^3$~скалярных величин.
     
     Другое назначение MRR состоит в~его использовании для 
диагностирования (классификации). В~этой связи в~первую очередь 
проводился сравнительный анализ MRR(1) (фактически это означает, что 
построение региона осуществляется на основе расстояния Махаланобиса) 
и~MRR(5) (модель смеси нормальных распределений и~предложенный 
в~данной работе метод оценивания па\-ра\-мет\-ров региона). Показателем 
информативности метода построения многомерного региона выступала 
мощность соответствующего слабого критерия значимости, а~именно: 
вероятность не попасть в~MRR при условии, что данные берутся из дополнения 
к~классу, для которого построена MRR. Сравнение соответствующих столбцов 
говорит о~явном преимуществе двух предложенных моментов: усложнение 
модели данных путем перехода от нормального распределения к~смеси 
нормальных распределений и~построение региона высокой плотности.
     
     Использование критического уровня можно продемонстрировать  
с~по\-мощью зависимости результатов сравнения двух классов от того, какой 
класс взять за основу. Введем для возможных значений $p$-ве\-ли\-чи\-ны три 
интервала: $(-\infty, 1\%)$, $[1\%, 5\%)$, $[5\%, 100\%)$ с~соответствующей 
интерпретацией положения наблюденного набора показателей для пациента 
относительно построенного MRR: уверенное непопадание, неуверенное 
попадание, уверенное попадание. Если MRR построить для оксалатных камней, 
то результаты для анализа пациентов с~фосфатными камнями дадут следующий 
вектор относительных частот попадания $p$-ве\-ли\-чин в~указанные 
интервалы: $(60\%, 18\%, 22\%)$. Если же MRR строить для фосфатных 
камней, то получим $(71\%, 5\%, 24\%)$. Таким образом, для классификации 
указанных камней при приблизительно одинаковых частотах попадания в~MRR 
(22\% или~24\%) уверенный отказ от референсного региона происходит чаще, 
если принять за базовый MRR регион для фосфатных камней. Построение 
шкалы, подобной рассмотренной, является прерогативой специалистов 
в~предметной области, в~данной работе она использовалась только для 
иллюстрации. 

\vspace*{-6pt}

\section{Заключение}

\vspace*{-2pt}

     На настоящий момент имеется относительно мало примеров применения 
MRR в~клинической практике. Тому есть несколько причин. Математическое 
обеспечение, необходимое для получения и~применения MRR, не отвечает 
возможностям большинства клинических лабораторий. Лаборатории слабо 
оснащены программными средствами\linebreak для реализации достаточно сложного 
математического аппарата многомерного анализа, а~еще важнее, что 
отсутствуют методики, инструкции по\linebreak использованию соответствующих 
средств. Лишь немногие клинические применения демонстрируют 
преимущества MRR, хотя свидетельств неудачных попыток больше.
     
     Несмотря на сложности внедрения мно\-го\-мерно\-го анализа референсных 
значений, можно сформулировать некоторые рекомендации по иссле\-до\-ва\-нию 
и~разработке MRR. Во-пер\-вых, эффективная размерность в~MRR должна 
быть как можно меньше, чтобы избежать затенения диагностически полезной 
информации тестами, со\-зда\-ющи\-ми шум. Низкая размерность также должна 
уменьшить неблагоприятные последствия увеличения неточности результатов 
в~связи с~ростом числа анализируемых показателей. Во-вто\-рых, показатели 
(тес\-ты), включенные в~MRR, должны быть физиологически релевантными 
исследуемому кругу расстройств, чтобы максимизировать информацию, 
полученную от MRR. В-треть\-их, чтобы учесть эффекты долгосрочной 
лабораторной из\-мен\-чи\-вости, данные, используемые для получения MRR, 
долж\-ны быть собраны и~проанализированы в~течение достаточно большого 
периода времени (от нескольких недель до нескольких месяцев).  
В-чет\-вер\-тых, представление результатов лабораторных исследований 
следует осуществлять в~графическом виде, чтобы помочь врачам лучше понять 
MRR. Различные подходы к~уменьшению размерности помогут выполнить это 
требование.
     
     Необходима дальнейшая разработка пояснительных инструментов, 
способных воспринять результаты анализа MRR. При этом дополнительно 
необходима информация о~том, какие именно тес\-ты являются важнейшими 
факторами нарушения нормы. Надо признать, что соответствующий 
математический аппарат еще предстоит разработать. Решение перечисленных 
вопросов играет важную роль для обеспечения постоянного клинического 
применения MRR. 

\vspace*{-6pt}
     
{\small\frenchspacing
 {%\baselineskip=10.8pt
 \addcontentsline{toc}{section}{References}
 \begin{thebibliography}{99}
 
 \vspace*{-2pt}
 
\bibitem{1-kri}
\Au{Boyd J.\,C.} Reference regions of two or more dimensions~// Clin. Chem. Lab. 
Med., 2004. Vol.~42. No.\,7. P.~739--746.
\bibitem{2-kri}
\Au{Winkel P.} Patterns and clusters~--- multivariate approach for interpreting 
clinical chemistry results~// Clin. Chem., 1973. Vol.~19. No.\,12. P.~1329--1333.
\bibitem{3-kri}
IFCC. Expert panel on theory of reference values. Approved recommendation on the 
theory of reference values. Part~5. Statistical treatment of collected reference values. 
Determination of reference limits~// J.~Clin. Chem. Clin. Biochem., 1987. Vol.~25. 
No.\,9. P.~645--656.
\bibitem{4-kri}
\Au{Кривенко М.\,П.} Статистические методы представления и~предварительной 
обработки референсных значений.~--- М.: ФИЦ ИУ РАН, 2016. 160~с.
\bibitem{5-kri}
\Au{Boyd J.\,C., Lacher~D.\,A.} The multivariate reference range: An alternative 
interpretation of multi-test profiles~// Clin. Chem., 1982. Vol.~28. No.\,2.  
P.~259--265.
\bibitem{6-kri}
\Au{Albert A., Harris~E.\,K.} Multivariate interpretation of clinical laboratory  
data.~--- New York, NY, USA: CRC Press, 1987. 328~p.
\bibitem{7-kri}
\Au{Linnet K.} Influence of sampling variation and analytical errors on the 
performance of the multivariate reference region~// Meth. Inf. Med., 1988. Vol.~27. 
No.\,1. P.~37--42.
\bibitem{8-kri}
\Au{Durbridge T.\,C.} Clinical acceptance of a multi-test reference region for 
biochemical-panel results~// Clin. Chem., 1983. Vol.~29. No.\,10. P.~1724--1726.
\bibitem{9-kri}
\Au{Кривенко М.\,П.} Критерии значимости отбора признаков классификации~// 
Информатика и~её применения, 2016. Т.~10. Вып.~3. С.~32--40.
\bibitem{10-kri}
\Au{Кривенко М.\,П., Голованов~С.\,А., Сивков~А.\,В.} Анализ однородности 
данных о химическом составе камней при уролитиазе~// Информатика и~её 
применения, 2013. Т.~7. Вып.~4. С.~94--104.
 \end{thebibliography}

 }
 }

\end{multicols}

\vspace*{-10pt}

\hfill{\small\textit{Поступила в~редакцию 5.12.16}}

\vspace*{4pt}

%\newpage

%\vspace*{-24pt}

\hrule

\vspace*{2pt}

\hrule

\vspace*{-3pt}


\def\tit{HIGH-DENSITY MULTIVARIATE REFERENCE REGION\\[-5pt]}

\def\titkol{High-density multivariate reference region}

\def\aut{M.\,P.~Krivenko\\[-7pt]}

\def\autkol{M.\,P.~Krivenko}

\titel{\tit}{\aut}{\autkol}{\titkol}

\vspace*{-16pt}


\noindent
Institute of Informatics Problems, Federal Research Center 
``Computer Science and Control'' of the Russian
Academy of Sciences,  44-2~Vavilov Str., Moscow 119333, Russian Federation



\def\leftfootline{\small{\textbf{\thepage}
\hfill INFORMATIKA I EE PRIMENENIYA~--- INFORMATICS AND
APPLICATIONS\ \ \ 2017\ \ \ volume~11\ \ \ issue\ 2}
}%
 \def\rightfootline{\small{INFORMATIKA I EE PRIMENENIYA~---
INFORMATICS AND APPLICATIONS\ \ \ 2017\ \ \ volume~11\ \ \ issue\ 2
\hfill \textbf{\thepage}}}

\vspace*{2pt}




\Abste{The paper considers the principles of construction of multivariate 
reference regions. An original method of construction of 
a~region on the basis of areas of high density of points and approximation 
of data distribution with a~mixture of normal distributions is suggested. 
To estimate the threshold for the probability density, the bootstrap method is used. 
As an experiment, the paper considers the problem of description and use of 
the reference region for predicting the type of urinary stones. 
Real data treatment demonstrated the benefits of the proposed solutions.}

\KWE{multivariate reference region; high-density region; bootstrap method; 
multivariate normal mixture}

\DOI{10.14357/19922264170207} 

%\vspace*{-18pt}

%\Ack
%\noindent



%\vspace*{3pt}

  \begin{multicols}{2}

\renewcommand{\bibname}{\protect\rmfamily References}
%\renewcommand{\bibname}{\large\protect\rm References}

{\small\frenchspacing
 {%\baselineskip=10.8pt
 \addcontentsline{toc}{section}{References}
 \begin{thebibliography}{99}
\bibitem{1-kri-1}
\Aue{Boyd, J.\,C.} 2004. Reference regions of two or more dimensions. \textit{Clin. 
Chem. Lab. Med.} 42(7):739--746.

\bibitem{2-kri-1}
\Aue{Winkel, P.} 1973. Patterns and clusters~--- multivariate approach for interpreting 
clinical chemistry results. \textit{Clin. Chem.} 19(12):1329--1333.
\bibitem{3-kri-1}
IFCC. 1987. Expert panel on theory of reference values. Approved recommendation on the 
theory of reference values. Part~5. Statistical treatment of collected reference values. 
Determination of reference limits. \textit{J.~Clin. Chem. Clin. Biochem.} 
25(9):645--656.
\bibitem{4-kri-1}
\Aue{Krivenko, M.\,P.} 2016. \textit{Statisticheskie metody predstavleniya 
i~predvaritel'noy obrabotki referensnykh znacheniy}
[Statistical methods for representation and preliminary processing of
reference values]. Moscow: FRC CSC RAS. 160~p.

\bibitem{5-kri-1}
\Aue{Boyd, J.\,C., and D.\,A.~Lacher.} 1982. The multivariate reference range: An 
alternative interpretation of multi-test profiles. \textit{Clin. Chem.}  
28(2):259--265.
\bibitem{6-kri-1}
\Aue{Albert, A., and E.\,K.~Harris.} 1987. \textit{Multivariate interpretation of 
clinical laboratory data}. New York, NY: CRC Press. 328~p.
\bibitem{7-kri-1}
\Aue{Linnet, K.} 1988. Influence of sampling variation and analytical errors on the 
performance of the multivariate reference region. \textit{Meth. Inf. Med.}  
27(1):37--42.
\bibitem{8-kri-1}
\Aue{Durbridge, T.\,C.} 1983. Clinical acceptance of a multi-test reference region 
for biochemical-panel results. \textit{Clin. Chem.} 29(10):1724--1726.
\bibitem{9-kri-1}
\Aue{Krivenko, M.\,P.} 2016. Kriterii znachimosti otbora priznakov klassifikatsii
[Significance tests of feature selection for~classification]. \textit{Informatika i~ee 
Primeneniya~--- Inform. Appl.} 10(3):32--40.
\bibitem{10-kri-1}
\Aue{Krivenko, M.\,P., S.\,A.~Golovanov, and A.\,V.~Sivkov}. 2013. Analiz 
odnorodnosti dannykh o~khimicheskom sostave kamney pri urolitiaze
[Analysis of data homogeneity of~the~chemical compositions 
of~stones in~case of~urolithiasis]. \textit{Informatika i~ee Primeneniya~---
Inform Appl.} 7(4):94--104.
\end{thebibliography}

 }
 }

\end{multicols}

\vspace*{-3pt}

\hfill{\small\textit{Received December 5, 2016}}


\Contrl

\noindent
\textbf{Krivenko Michail P.} (b.\ 1946)~--- Doctor of Science in technology, 
professor, leading scientist, Institute of Informatics Problems, Federal Research 
Center ``Computer Science and Control'' of the Russian Academy of Sciences, 
\mbox{44-2}~Vavilov Str., Moscow 119333, Russian Federation; \mbox{mkrivenko@ipiran.ru}

\label{end\stat}


\renewcommand{\bibname}{\protect\rm Литература}   %10
\def\stat{zaharova}

\def\tit{ОЦЕНКА УРОВНЯ ЗНАЧИМОСТИ КРИТЕРИЯ ШУИРМАННА ДЛЯ~ПРОВЕРКИ ГИПОТЕЗЫ 
БИОЭКВИВАЛЕНТНОСТИ ПРИ~НАЛИЧИИ ПРОПУЩЕННЫХ ДАННЫХ$^*$}

\def\titkol{Оценка уровня значимости критерия Шуирманна для~проверки гипотезы 
биоэквивалентности} % при~наличии пропущенных данных}

\def\aut{Т.\,В.~Захарова$^1$,  А.\,А.~Тархов$^2$}

\def\autkol{Т.\,В.~Захарова,  А.\,А.~Тархов}

\titel{\tit}{\aut}{\autkol}{\titkol}

\index{Захарова Т.\,В.}
\index{Тархов А.\,А.}
\index{Zakharova T.\,V.}
\index{Tarkhov A.\,A.}


{\renewcommand{\thefootnote}{\fnsymbol{footnote}} \footnotetext[1]
{Работа выполнена при поддержке РФФИ (проект 18-07-00252).}}


\renewcommand{\thefootnote}{\arabic{footnote}}
\footnotetext[1]{Московский государственный университет им.\ М.\,В.~Ломоносова,
факультет вычислительной математики и~кибернетики; Институт 
проблем информатики Федерального исследовательского центра <<Информатика и~управление>> 
Российской академии наук, \mbox{tvzaharova@mail.ru}}
\footnotetext[2]{Московский государственный университет им.\ М.\,В.~Ломоносова, 
факультет вычислительной математики и~кибернетики, \mbox{alexeytarkhov@gmail.com}}

%\vspace*{-2pt}



\Abst{Задача проверки гипотезы биоэквивалентности имеет важное 
значение в~фармакокинетике. С~ее помощью принимают решение об 
эквивалентности воспроизведенного лекарственного препарата референтному 
лекарственному препарату. Одна из проблем исследований биоэквивалентности~--- 
наличие пропущенных данных. Так как объем исследуемых данных достаточно мал,
 то удаление данных субъекта, у~которого есть пропущенные данные, нежелательно. 
 Поэтому стоит задача оценить влияние пропущенных данных при принятии решения
  о~биоэквивалентности, а~именно: дать оценку уровня значимости.
Основным методом проверки гипотезы биоэквивалентности является 
процедура двух односторонних тес\-тов Шуирманна. В~статье дана оценка уровня 
значимости данной процедуры при наличии пропущенных данных. 
В~явном виде получена компонента оценки уровня значимости, зависящая от уровня 
полноты данных.}

\KW{биоэквивалентность; уровень значимости; ошибка первого рода;
 пропущенные данные;  процедура двух односторонних тестов Шуирманна}


\DOI{10.14357/19922264190309} 
  
\vspace*{3pt}


\vskip 10pt plus 9pt minus 6pt

\thispagestyle{headings}

\begin{multicols}{2}

\label{st\stat}

\section{Введение}

\vspace*{-3pt}

Предположим, что имеется лекарственный препа\-рат, для которого 
был проведен набор  широкомасштабных клинических исследований, 
доказавших его безопасность и~медицинскую эф\-фективность. Данный 
препарат будем называть\linebreak референтным лекарственным препаратом. 
На основе действующих веществ референтного лекарственного препарата 
могут быть созданы новые лекар\-ст\-вен\-ные препараты с~таким же 
количественным и~качественным составом. Далее будем называть 
их воспроизведенными лекарственными препаратами. Чтобы 
перенести имеющиеся сведения о безопасности и~эффективности 
референтного\linebreak лекарственного препарата на воспроизведенный 
препарат без проведения широкомасштабных исследований, исследуют 
биоэквивалентность лекарственных препаратов.


Понятие биоэквивалентности тесно связано с~понятием биодоступности.

Биодоступность~-- скорость и~степень, с~которыми действующее вещество 
или его активная часть молекулы из дозированной лекарственной формы 
всасываются и~становятся доступными в~месте действия. Два лекарственных 
препарата, содержащих одинаковое количество действующего вещества, 
считаются биоэквивалентными, если они являются фармацевтически 
эквивалентными или фармацевтически альтернативными и~их биодоступность 
(по скорости и~степени) после применения в~одинаковой молярной дозе 
укладывается в~заранее установленные допустимые пределы~\cite{defin}.


В процессе сбора данных некоторые полученные значения могут быть утеряны. 
Так как число испытуемых ограничено, то исключать данные испытуемого,
 для которого было утеряно одно значение концентрации действующего вещества,
  нерационально. Вместо этого пропущенные данные заполняют нулем или другим значением,
   полученным на основе информации о~других значениях. 
   
   При использовании недостаточно 
   точных методов заполнения данных, таких как заполнение нулем, уменьшается часть 
   значений исследуемых данных.
Хотелось бы оценить, как наличие пропущенных данных влияет на проверку 
гипотезы биоэквивалентности.


В данной работе будет рассмотрена процедура двух односторонних тестов Шуирманна, 
которая в~настоящее время используется при проверке гипотезы биоэквивалентности, 
при наличии пропусков в~исследуемых данных.



\section{Задача проверки гипотезы биоэквивалентности}

Пусть $T$~--- воспроизведенный лекарственный препарат, а~$R$ --- 
референтный лекарственный препарат и~соответственно~$\mu_T$ и~$\mu_R$~--- 
математические ожидания сравнительных характеристик для лекарственных препаратов~$T$ и~$R$.

Основными сравнительными характеристиками биоэквивалентности служат максимальная 
концентрация  в~крови~$C_{\max}$ и~площадь под кривой <<кон\-цент\-ра\-ция\,--\,вре\-мя>> 
$\mathrm{AUC}$ (от \textit{англ.}\ Area Under the Curve).

Будем следовать предположению, что рас\-смат\-ри\-ва\-емые сравнительные 
характеристики имеют логнормальное распределение~[2--4].
%\cite{schuirmann, book, article}.
Например, для $\mathrm{AUC}$:
\begin{equation*}
%\label{log-norm}
    \mathrm{AUC}_i \sim \mathrm{Log}\,N\left(a_i, \sigma_i\right),\enskip i \in \{T,R\}.
\end{equation*}

Пусть $\theta_1$ и~$\theta_2$~--- соответственно нижний и~верхний принятый 
допустимый предел признания биоэквивалентности. Следовательно, гипотеза 
о~биоэквивалентности может быть записана следующим образом:
\begin{align*}
%\left.
%\begin{array}{rl}
&    H_0: \fr{\mu_T}{\mu_R} \leqslant q \theta_1\enskip \mbox{или}\enskip  
\fr{\mu_T}{\mu_R} \ge \theta_2; \\
 &   H_A: \theta_1 <\fr{\mu_T}{\mu_R} < \theta_2.  
% \end{array}
% \right\}
 %\label{h-bio}
\end{align*}

Сделав логарифмическое преобразование, можем перейти к~следующей 
постановке рассматриваемой гипотезы:
\begin{align*}
%\left.
%\begin{array}{rl}
    &H_0': \mu'_T - \mu'_R \leqslant q \delta_1\enskip \mbox{или}\enskip  \mu'_T - \mu'_R \ge \delta_2; \\
    &H_A': \delta_1 <\mu'_T - \mu'_R < \delta_2,  
 %   \end{array}
  %  \right\}
   % \label{h-bio-log}
\end{align*}
где $\delta_1 = \ln\theta_1$ и~$\delta_2 \hm= \ln\theta_2$, а~$\mu'_T$ и~$\mu'_R$~--- 
математические ожидания логарифмов сравнительных характеристик 
для лекарственных препаратов~$T$ и~$R$.
Например, для~$\ln{\mathrm{AUC}_T}$ из свойств логнормального распределения следует, 
что $ \mu'_T \hm= a_T$.

Гипотеза $H_0'$ соответствует небиоэквивалент\-ности исследуемых 
лекарственных препаратов, в~то время как~$H_A'$ утверждает, что они 
биоэквивалентны. Выбор такого порядка основной и~альтернативной гипотез 
обусловлен тем, что в~таком случае ошибка первого рода соответствует 
признанию лекарственных средств биоэквивалентными, хотя на самом деле 
они такими и~не является. В~этом случае пациент несет риск получить препарат, 
который может не обладать такими же эффективностью и~безопасностью, как 
референтный лекарственный препарат~\cite{article}.


\section{Процедура двух односторонних тестов Шуирманна}

Разобьем гипотезы $H'_0$ и~$H'_1$ на два множества односторонних гипотез:
\begin{equation*}
\left\{
\begin{array}{rl}
    H_{01}:& \mu'_T - \mu'_R \leqslant q \delta_1;  \\[6pt]
    H_{A1}:& \mu'_T - \mu'_R > \delta_1;  
    \end{array}
    \right.
   % \label{h-bio-log1}
\end{equation*}
\begin{equation*}
\left\{
\begin{array}{rl}
    H_{02}:& \mu'_T - \mu'_R \geqslant \delta_2; \\[6pt]
    H_{A2}:& \mu'_T - \mu'_R < \delta_2.  
    \end{array}
    \right.
   % \label{h-bio-log2}
\end{equation*}
Процедура двух односторонних тестов заключается в~том, что~$H'_0$ 
отвергаем при уровне зна\-чи\-мости~$\alpha$, тем самым устанавливая 
эквивалентность~$\mu_T$ и~$\mu_R$,  только в~том случае, если отвергаются 
обе гипотезы~$H_{01}$ и~$H_{02}$ при заданном уровне зна\-чи\-мости~$\alpha$~\cite{schuirmann, book}.

Таким образом, два односторонних теста проверяются с~использованием односторонних 
t-кри\-те\-ри\-ев, т.\,е.\ 
характеристики биодоступности признаются эквивалентными, если
\begin{equation}
\left.
\begin{array}{rl}
    \hspace*{-2mm}t_1& =  \fr{\bar{Y_T} -\bar{Y_R}-\delta_1}
    {\hat\sigma_d\sqrt{{1}/{n_1} + {1}/{n_2}}} > t\left(\alpha, n_1 + n_2-2\right); \\[6pt]
        \hspace*{-2mm}t_2& =  \fr{\bar{Y_T} - \bar{Y_R}-\delta_2}
    {\hat\sigma_d\sqrt{{1}/{n_1} + {1}/{n_2}}} < -t\left(\alpha, n_1 + n_2-2\right), 
    \end{array}\!
    \right\}\!\!
    \label{t}
\end{equation}
где $n_1$ и~$n_2$~--- число субъектов в~последовательностях клинического 
исследования с~перекрестным\linebreak двухпоследовательным дизайном, 
$t(\alpha, n_1\hm + n_2-2)$~--- $(1\hm - \alpha)$-кван\-тиль центрального 
t-рас\-пре\-де\-ле\-ния с~$n_1\hm+n_2-2$ степенями свободы; $\hat\sigma_d$~--- 
обобщенная выборочная дисперсия разностей между периодами (для обоих 
последовательностей в~исследовании), которая является несмещенной оценкой~$\sigma_d$,\linebreak 
причем
$$
\sigma_d^2 = \fr{\sigma_w^2}{2}\,,
$$
где $\sigma_w$~--- внутрисубъектная вариабельность изуча\-емых параметров~\cite{article}.

Процедура двух односторонних тестов эквивалентна подходу с~построением 
доверительного интервала для разности выборочных средних, т.\,е.\ 
получению следующей интервальной оценки:

\noindent
\begin{multline*}
%\label{interv}
    \left(\bar{Y_T} - \bar{Y_R} + t\left(\alpha, n_1 + n_2-2\right)\hat\sigma_d
    \sqrt{\fr{1}{n_1} + \fr{1}{n_2}};\right.\\
    \left.   \bar{Y_T} - \bar{Y_R} - t\left(\alpha, n_1 + n_2-2\right)
    \hat\sigma_d\sqrt{\fr{1}{n_1} + \fr{1}{n_2}}\right).
\end{multline*}

Признание эквивалентности параметров биодоступности на
 уровне значимости~$\alpha$ может быть сделано, только если 
 полученный доверительный $(1\hm-2\alpha)100\%$-ный интервал для 
 $\mu'_T \hm- \mu'_R$ полностью содержится в~интервале 
 $\left(\delta_1, \delta_2\right)$~\cite{schuirmann}.


Так как рассматриваем сбалансированный дизайн, то $n_1\hm=n_2\hm=n$, 
и,~учитывая~(\ref{t}), получаем, что t-кри\-те\-рии принимают вид:
\begin{align*}
%\left.
%\begin{array}{rl}
    t'_1 &=  \fr{\bar{Y_T} - \bar{Y_R}-\delta_1}{\hat\sigma_d\sqrt{2/n}} > t(2n-2, \alpha); \\
    t'_2 &=  \fr{\bar{Y_T} - \bar{Y_R}-\delta_2}{\hat\sigma_d\sqrt{2/n}} < -t(2n-2, \alpha)  
%    \end{array}
 %   \right\}
  %  \label{t_2}
\end{align*}
и соответствующий доверительный интервал принимает вид:
\begin{multline*}
%\label{interv_2}
    \left(\bar{Y_T} - \bar{Y_R} + t(\alpha, 2n-2)\hat\sigma_d\sqrt{\fr{2}{n}};\  \right.\\
\left.     \bar{Y_T} - \bar{Y_R} - t(\alpha, 2n-2)\hat\sigma_d\sqrt{\fr{2}{n}}\right).
\end{multline*}

\vspace*{-9pt}

\section{Оценка уровня значимости при~наличии пропущенных данных}

\vspace*{-3pt}

Рассмотрим выборочное пространство $\chi$~--- пространство элементарных событий. 
Статистический критерий разбивает пространство элементарных событий~$\chi$ 
на два подмножества:
\begin{enumerate}[(1)]
\item область принятия гипотезы $\chi_0$~--- множество, состоящее 
из точек, для которых гипотеза~$H_0$ принимается;\\[-14pt]
\item  область отклонения гипотезы $\chi_A$~--- множество, 
состоящее из точек, для которых гипотеза~$H_0$ отвергается.
\end{enumerate}


Говорят, что критерий имеет уровень зна\-чи\-мости~$\alpha$, если вероятность 
наступления ошибки первого рода не превышает~$\alpha$, $0 \hm< \alpha\hm < 1$, 
для $\delta\hm \in \chi_A$:

\noindent
\begin{multline*}
{\sf P}\left\{\mbox{отклонить } H_0 \mbox{ при\ истинной}\right.\\[-1pt]
\left.\mbox{небиоэквивалентности}\right\} ={}\\[-1pt]
{}
= {\sf P}\left\{\mbox{отклонить\ } H_0, \delta \in \chi_0\right\} \leqslant \alpha\,.
\end{multline*}



Рассмотрим задачу при наличии пропусков в~данных:
пусть $q$~--- уровень полноты данных,
 т.\,е.\ 
доля данных, оставшихся от изначальных дан-\linebreak\vspace*{-12pt}

\columnbreak

\noindent
ных, $0 \hm<q\hm \leqslant 1$
($1 - q$~--- доля пропущенных данных в~выборке).



Тогда $\tilde{Y_T} = \bar{Y_T} + \ln(q)$~--- 
выборочное среднее логарифмов сравнительных характеристик для лекарственного 
препарата~$T$ при наличии пропущенных данных.


Критическая область для данной задачи имеет следующий вид:

\vspace*{-4pt}

\noindent
\begin{multline*}
\chi_A' = \left\{(\tilde{Y_T} - \bar{Y_R}, \hat{\sigma_d}): 
\delta_1 + t\left(\alpha, 2n-2\right)\hat\sigma_d\sqrt{\fr{2}{n}} <{}\right.\\
\left.{}< \tilde{Y_T} - \bar{Y_R} <  \delta_2 - t\left(\alpha, 2n-2\right)\hat\sigma_d
\sqrt{\fr{2}{n}}\right\}.
\end{multline*}

\vspace*{-2pt}

Рассмотрим функцию мощности:

\vspace*{-3pt}

\noindent
\begin{align*}
    \phi_{\hat{\sigma_d}}'(\delta) &= {\sf P}\{\mbox{отклонить } H_0 \mbox{ при истинной}\\
&    \mbox{биоэквивалентности}\}={} \\
    &{}= {\sf P}\{(\tilde{Y_T} - \bar{Y_R}, \hat{\sigma_d}): \chi_A', если \delta \in \chi_A'\}.
\end{align*}

\vspace*{-2pt}

\noindent
Фиксируем $\delta = \delta_0$, получаем:

\vspace*{-2pt}

\noindent
\begin{multline*}
    \phi_{\hat{\sigma_d}}'(\delta_0) = {} \\
{}= {\sf P}\left(\left(\tilde{Y_T} - \bar{Y_R}, \hat{\sigma_d}\right): 
\delta_1 + t(\alpha, 2n-2)\hat\sigma_d\sqrt{\fr{2}{n}} < {}\right.\\
\left.{}<\tilde{Y_T} - \bar{Y_R} <  \delta_2 - t(\alpha, 2n-2)
\hat\sigma_d\sqrt{\fr{2}{n}}\vert  \delta =\delta_0\right) ={} \\
{}={\sf P}\left( \delta_1 + t(\alpha, 2n-2)\hat\sigma_d\sqrt{\fr{2}{n}}
 < \bar{Y_T} + \ln(q) - {}\right.\\
\left. {}-\bar{Y_R} <  \delta_2 - t(\alpha, 2n-2)\hat\sigma_d
 \sqrt{\fr{2}{n}}\right) ={} \\
 {}={\sf P}\Biggl(\fr{\delta_1 + t(\alpha, 2n-2)\hat\sigma_d\sqrt{{2}/{n}}- 
\ln(q)-\delta_0}{\sigma_d\sqrt{{2}/{n}}}<{}\\
{}< \fr{\bar{Y_T} - 
\bar{Y_R}-\delta_0}{\sigma_d\sqrt{{2}/{n}}} <{}\\
{} < \fr{ \delta_2 - t(\alpha, 2n-2)\hat\sigma_d\sqrt{{2}/{n}}- 
\ln(q)-\delta_0}{\sigma_d\sqrt{{2}/{n}}}\Biggr) = {}\\
{}= \{\mbox{фиксируем } \hat\sigma_d\} = {}\\
{}=E\Biggl[{\sf P}\Biggl(\fr{\delta_1 + t(\alpha, 2n-2)\hat\sigma_d\sqrt{{2}/{n}}
- \ln(q)-\delta_0}{\sigma_d\sqrt{{2}/{n}}}< {}\\
{}<\fr{\bar{Y_T} - 
\bar{Y_R}-\delta_0}{\sigma_d\sqrt{{2}/{n}}} <{}\\
 {} < \fr{ \delta_2 - t(\alpha, 2n-2)\hat\sigma_d\sqrt{{2}/{n}}- 
\ln(q)-\delta_0}{\sigma_d\sqrt{{2}/{n}}}|\hat\sigma_d\Biggr)\Biggr] = {}
\end{multline*}

 \noindent
 \begin{multline*}
\hspace*{-4pt}{}=E\Biggl[\Phi\left(\fr{\delta_1 + t(\alpha, 2n-2)\hat\sigma_d\sqrt{{2}/{n}}-
 \ln(q)-\delta_0}{\sigma_d\sqrt{{2}/{n}}}\right) - {}\\[3pt]
 {}-
 \Phi\left( \fr{ \delta_2 - t(\alpha, 2n-2)\hat\sigma_d\sqrt{{2}/{n}}- 
 \ln(q)-\delta_0}{\sigma_d\sqrt{{2}/{n}}}\right)\Biggr], 
\end{multline*}
где $\Phi(x)$~--- функция стандартного нормального распределения.

Процедура двух односторонних тестов Шуирманна используется в~условиях 
решающего правила 80/125~\cite{schuirmann, article, hsu}. Это значит, что
 $\delta_1\hm = \ln(0,80)\hm \approx -0{,}2231$ и~$\delta_2\hm = \ln(1,25)
 \hm \approx 0{,}2231$; следовательно, $\delta_1 \hm\approx -\delta_2$. 
 Тогда видим, что функция $\phi_{\hat{\sigma_d}}'(\delta)$ сим\-мет\-рич\-на 
 относительно точки $\delta \hm= -\ln(q)$ и~достигает максимума в~этой точке.

Тогда 

\vspace*{-6pt}

\noindent
\begin{multline*}
\max_{\delta \in \chi_0} {\sf P}\{\mbox{отклонить } H_0\} \hm= 
\phi_{\hat{\sigma_d}}'(\delta_2) ={}\\
{}\{\mbox{подставим }  \delta = \delta_2,\ 
\delta_1 = -\delta_2, \sigma_d=\hat\sigma_d \} ={}\\
    {}={\sf P}\left( \fr{-2\delta_2 - \ln(q)}{\hat\sigma_d\sqrt{{2}/{n}}} 
    +  t(\alpha, 2n-2)<{}\right.\\
   \left. {}<\fr{\bar{Y_T} - \bar{Y_R}-\delta_2}{\hat\sigma_d
    \sqrt{{2}/{n}}}< \fr{- \ln(q)}{\hat\sigma_d\sqrt{{2}/{n}}}- 
    t(\alpha, 2n-2)\right) = {}\\
    {}={\sf P}\left( \fr{-2\delta_2 - \ln(q)}{\hat\sigma_d\sqrt{{2}/{n}}} + 
     t(\alpha, 2n-2)<{}\right.\\
\left.     {}<\fr{\bar{Y_T} - \bar{Y_R}-\delta_2}{\hat\sigma_d\sqrt{{2}/{n}}}<
      - t(\alpha, 2n-2)\right) + {}\\
    {}+{\sf P}\left( - t(\alpha, 2n-2)<\fr{\bar{Y_T} - \bar{Y_R}-\delta_2}
    {\hat\sigma_d\sqrt{{2}/{n}}}<{}\right.\\
\left.    {}< \fr{- \ln(q)}{\hat\sigma_d\sqrt{{2}/{n}}}- 
    t(\alpha, 2n-2)\right) \leqslant {}\\
{}\leqslant {\sf P}\left(\fr{\bar{Y_T} - \bar{Y_R}-\delta_2}{\hat\sigma_d\sqrt{{2}/{n}}}< 
- t(\alpha, 2n-2)\right) + {}\\
    {}+ {\sf P}\left( - t(\alpha, 2n-2)<\fr{\bar{Y_T} - \bar{Y_R}-\delta_2}
    {\hat\sigma_d\sqrt{{2}/{n}}}< {}\right.\\
\left.    {}<\fr{- \ln(q)}{\hat\sigma_d\sqrt{{2}/{n}}}- 
    t(\alpha, 2n-2)\right) = {}
   \\
    {}=\alpha + {\sf P}\left( - t(\alpha, 2n-2)<\fr{\bar{Y_T} - 
    \bar{Y_R}-\delta_2}{\hat\sigma_d\sqrt{{2}/{n}}}< {}\right.\\
\left.    {}<\fr{- \ln(q)}
    {\hat\sigma_d\sqrt{{2}/{n}}}- t(\alpha, 2n-2)\right) =\alpha + \alpha'. \\
\end{multline*}

\vspace*{-18pt}

\noindent
Полученная оценка показывает, что величина ошибки первого рода не превосходит
 $\alpha \hm+ \alpha'$. При чем в~работах~\cite{book, article} показано, что 
 при исполь-\linebreak\vspace*{-12pt}
 
 \columnbreak
 
 \noindent
зо\-вании двух односторонних тестов Шуирманна\linebreak
  величина 
 вероятности ошибки первого рода не превосходит~$\alpha$. 
 В~рассматриваемой постановке уровень значимости критерия повышается на 
 величину~$\alpha'$, что обусловлено наличием пропущенных данных для 
 воспроизводимого лекарственного препарата. Таким образом, риск потенциального 
 выхода на рынок небиоэквивалентного лекарственного препарата повышается.
 
 \vspace*{-15pt}

\section{Заключение}

 \vspace*{-5pt}

Процедура двух односторонних тестов Шуирманна~--- одно из основных средств 
при проверке гипотезы биоэквивалентности. При исследовании критериев принятия 
гипотезы биоэквивалентности важную роль играет оценка вероятности наступления 
ошибки первого рода. Важность ее контроля\linebreak обусловлена риском пациента получить 
препарат с~несоответствующими эффективностью и~без\-опас\-ностью.
%
В~данной статье впервые дана оценка уровня значимости процедуры двух односторонних
 тестов Шуирманна при наличии пропущенных данных. В~част\-ности, в~явном виде
  показана та ее часть, которая зависит от уровня полноты данных.
%
В практическом плане данная оценка может быть использована для корректировки 
задаваемого уровня зна\-чи\-мости при известном уровне полноты данных, чтобы 
обеспечить гарантированную эффективность и~безопасность воспроизведенных лекарств.

 \vspace*{-15pt}


{\small\frenchspacing
 { %\baselineskip=10.5pt
 \addcontentsline{toc}{section}{References}
 \begin{thebibliography}{9}
 
  \vspace*{-5pt}
  
    \bibitem{defin} 
    Правила проведения исследований биоэквивалентности лекарственных 
    средств Евразийского экономического союза. 
    {\sf 
    http://www.eurasiancommission.org/ ru/act/texnreg/deptexreg/konsultComitet/Documents/\linebreak Правила\%20БЭИ\%20итог\%2020.02.2015\%20на\%20\linebreak сайт.pdf}.
    \bibitem{schuirmann}
    \Au{Schuirmann D.\,J.}
    A~comparison of the two one-sided tests procedure and the power approach 
    for assessing the equivalence of average bioavailability~// J.~Pharmacokinet. 
    Biop., 1987. Vol.~15. P.~657--680.
    \bibitem{book}
    \Au{Chow Shein-Chung, Liu Jen-pei.}
    Design and analysis of bioavailability and bioequivalence studies.~--- 
    Chapman \& Hall/CRC, 2009. 735~p.
    \bibitem{article}
\Au{Драницына М.\,А., Захарова~Т.\,В., Ниязов~Р.\,Р.}
    Свойства процедуры двух односторонних тестов 
    для признания биоэквивалентности лекарственных препаратов~// Ремедиум. 
    Журнал о рынке лекарств и~медицинской техники, 2019. №\,3. С.~40--47.
    \bibitem{hsu} \Au{Berger R.\,L., Hsu~J.\,C.} 
    Bioequivalence trials, intersection--union tests and equivalence confidence sets~// 
    Stat. Sci., 1996. Vol.~11. No.~4. P.~283--319.
    
     \end{thebibliography}

 }
 }

\end{multicols}

\vspace*{-12pt}

\hfill{\small\textit{Поступила в~редакцию 09.05.19}}

%\vspace*{8pt}

\pagebreak

%\newpage

\vspace*{-28pt}

%\hrule

%\vspace*{2pt}

%\hrule

%\vspace*{-2pt}

\def\tit{EVALUATION OF THE SIGNIFICANCE LEVEL IN~SCHUIRMANN'S TEST FOR~CHECKING 
THE~BIOEQUIVALENCE HYPOTHESIS IN~MISSING DATA CONDITIONS}


\def\titkol{Evaluation of the significance level in~Schuirmann's test for~checking 
the~bioequivalence hypothesis in~missing data conditions}

\def\aut{T.\,V.~Zakharova$^{1,2}$ and A.\,A.~Tarkhov$^1$}

\def\autkol{T.\,V.~Zakharova and A.\,A.~Tarkhov}

\titel{\tit}{\aut}{\autkol}{\titkol}

\vspace*{-11pt}


\noindent
$^1$Department of Mathematical Statistics, Faculty of Computational Mathematics 
 and Cybernetics, M.\,V.~Lo\-mo-\linebreak
 $\hphantom{^1}$nosov Moscow State University, 1-52~Leninskiye Gory, 
 GSP-1, Moscow 119991, Russian Federation
 
 \noindent
 $^2$Institute of 
 Informatics Problems, Federal Research Center ``Computer Science and Control'' 
 of the Russian\linebreak
  $\hphantom{^1}$Academy of Sciences, 44-2~Vavilov Str., Moscow 119333, 
 Russian Federation

\def\leftfootline{\small{\textbf{\thepage}
\hfill INFORMATIKA I EE PRIMENENIYA~--- INFORMATICS AND
APPLICATIONS\ \ \ 2019\ \ \ volume~13\ \ \ issue\ 3}
}%
 \def\rightfootline{\small{INFORMATIKA I EE PRIMENENIYA~---
INFORMATICS AND APPLICATIONS\ \ \ 2019\ \ \ volume~13\ \ \ issue\ 3
\hfill \textbf{\thepage}}}

\vspace*{3pt}    



\Abste{The bioequivalence hypothesis testing is the important task 
in pharmacokinetics. It helps to make a~decision about the equivalence 
of the reproduced drug to the reference drug. One of the problems of bioequivalence 
studies is the availability of missing data. 
A~small amount of data entails the inability to delete a~data sample with 
missing data. Therefore, there is a~task to estimate the impact of missing data 
on bioequivalence testing task, in particular, to estimate the significance level. 
The main method of the bioequivalence hypothesis testing is Schuirmann's 
two one-sided tests procedure. The article shows the significance level evaluation 
of this procedure in the case of missing data. The evaluation component, depending 
on the level of data completeness, is shown in the explicit form.}


\KWE{bioequivalence; significance level; type I error; missing data; 
Schuirmann's two one-sided tests procedure}


\DOI{10.14357/19922264190309} 

%\vspace*{-14pt}

\Ack
   \noindent
   The paper was supported by the Russian Foundation for Basic Research (project  
18-07-00252).


%\vspace*{-6pt}

  \begin{multicols}{2}

\renewcommand{\bibname}{\protect\rmfamily References}
%\renewcommand{\bibname}{\large\protect\rm References}

{\small\frenchspacing
 {%\baselineskip=10.8pt
 \addcontentsline{toc}{section}{References}
 \begin{thebibliography}{9}

\bibitem{1-zah}
Pravila provedeniya issledovaniy bioekvivalentnosti 
lekarstvennykh sredstv Evraziyskogo ekonomicheskogo soyuza. Available at:
{\sf http://www.eurasiancommission.\linebreak org/ru/act/texnreg/deptexreg/konsultComitet/\linebreak Documents/Pravila\%20BEI\%20itog\%2020.02.2015\%20\linebreak na\%20sajt.pdf}
  (accessed May~8, 2019).
\bibitem{2-zah}
\Aue{Schuirmann, D.\,J.} 1987. 
A~comparison of the two one-sided tests procedure and the power approach for 
assessing the equivalence of average bioavailability.
\textit{J.~Pharmacokinet. Biop.} 15:657--680.

\columnbreak 

\bibitem{3-zah}
\Aue{Chow, Shein-Chung, and Jen-pei Liu.} 2009. \textit{Design 
and analysis of bioavailability and bioequivalence studies.} 
Chapman \& Hall/CRC. 735~p.

\vspace*{-2pt}

\bibitem{4-zah}
\Aue{Dranitsyna, M.\,A., T.\,V.~Zakharova, and R.\,R.~Niyazov.}
 2019. Svoystva protsedury dvukh odnostoronnikh testov dlya priznaniya
  bioekvivalentnosti lekarstvennykh preparatov [Properties of the 
   two-sided tests procedure for the  bioequivalence assessment of medical products]. 
  \textit{Remedium. Zh.~o~rynke lekarstv i~meditsinskoy tekhniki}
   [Remedium: J.~of the Market of Medicines and Medical Equipment] 2019(3):40--47.
   
   \vspace*{-2pt}
   
\bibitem{5-zah}
\Aue{Berger, R.\,L.,  and J.\,C.~Hsu.} 1996. 
Bioequivalence trials, intersection--union tests and equivalence confidence sets. 
\textit{Stat. Sci.} 11(4):283--319.
\end{thebibliography}

 }
 }

\end{multicols}

%\vspace*{-7pt}

\hfill{\small\textit{Received May 9, 2019}}

%\pagebreak

\vspace*{-12pt}

\Contr

\noindent
\textbf{Zakharova Tatiana V.} (b.\ 1962)~--- 
Candidate of Science (PhD) in physics and mathematics, associate professor,
 Department of Mathematical Statistics, Faculty of Computational Mathematics 
 and Cybernetics, M.\,V.~Lomonosov Moscow State University, 1-52~Leninskiye Gory, 
 GSP-1, Moscow 119991, Russian Federation; senior scientist, Institute of 
 Informatics Problems, Federal Research Center ``Computer Science and Control'' 
 of the Russian Academy of Sciences, 44-2~Vavilov Str., Moscow 119333, 
 Russian Federation; \mbox{tvzaharova@mail.ru}
 
 \vspace*{3pt}

\noindent
\textbf{Tarkhov Alexey A.} (b.\ 1995)~--- 
master student, Department of Mathematical Statistics, Faculty of Computational 
Mathematics and Cybernetics, M.\,V.~Lomonosov Moscow State University, 
1-52~Leninskiye Gory, GSP-1, Moscow 119991, Russian Federation; 
\mbox{alexeytarkhov@gmail.com}

\label{end\stat}

\renewcommand{\bibname}{\protect\rm Литература}   %11
\renewcommand{\figurename}{\protect\bf Figure}

\def\stat{kalinichenko}


\def\tit{METHODS AND TOOLS FOR HYPOTHESIS-DRIVEN
RESEARCH SUPPORT: A~SURVEY$^*$}

\def\titkol{Methods and tools for hypothesis-driven
research support: A~survey}

\def\autkol{L.~Kalinichenko, D.~Kovalev, D.~Kovaleva,
and~O.~Malkov}

\def\aut{L.~Kalinichenko$^1$, D.~Kovalev$^1$, D.~Kovaleva$^2$,
and~O.~Malkov$^2$}

\titel{\tit}{\aut}{\autkol}{\titkol}

{\renewcommand{\thefootnote}{\fnsymbol{footnote}}
\footnotetext[1] {This work has been partially supported by the RFBR grants 13-07-00579
and 14-07-00548.}}

\renewcommand{\thefootnote}{\arabic{footnote}}
\footnotetext[1]{Institute of Informatics Problems, Russian Academy of Sciences, 44-2 Vavilov Str., Moscow 119333, Russian
Federation}
\footnotetext[2]{Institute of Astronomy, Russian Academy of Sciences, 48 Pyatnitskaya Str., Moscow 119017, Russian Federation}


%\vspace*{-12pt}

\def\leftfootline{\small{\textbf{\thepage}
\hfill INFORMATIKA I EE PRIMENENIYA~--- INFORMATICS AND APPLICATIONS\ \ \ 2015\ \ \ volume~9\ \ \ issue\ 1}
}%
 \def\rightfootline{\small{INFORMATIKA I EE PRIMENENIYA~--- INFORMATICS AND APPLICATIONS\ \ \ 2015\ \ \ volume~9\ \ \ issue\ 1
\hfill \textbf{\thepage}}}


\Abste{Data intensive research (DIR) is being developed in frame of the new
paradigm of research study known as the Fourth paradigm, emphasizing an
increasing role of observational, experimental, and computer simulated data
practically in all research domains. The principal goal of DIR is an extraction
(inference) of knowledge from data.
  The intention of this work is to make an overview of the existing approaches,
methods, and infrastructures of the data analysis in DIR accentuating the role of
hypotheses in such process and efficient support of hypothesis formation,
evaluation, and selection in course of the natural phenomena modeling and
experiments carrying out.  An introduction into various concepts, methods, and
tools intended for effective organization of hypothesis-driven experiments in DIR
is presented.}

\KWE{data intensive research; Fourth paradigm; hypotheses; models; theories;
hypothetico-deductive method; hypothesis testing; hypothesis lattice; Galaxy
model; connectome analysis; automated hypothesis generation}

\DOI{10.14357/19922264150104}

%\vspace*{6pt}


\vskip 12pt plus 9pt minus 6pt

      \thispagestyle{myheadings}

      \begin{multicols}{2}

                  \label{st\stat}


\section{Hypotheses, Theories, Models and~Laws in~Data Intensive Science}

  \noindent
  Data intensive research is being developed in accordance with the Fourth
Paradigm~[1] of research study (following three previous historical paradigms of the
science development (empirical science, theoretical science,
and computational science))
emphasizing that science as a~whole is becoming increasingly dependent on data as
the core source for discovery. Emerging of the Fourth Paradigm is motivated by the huge
amount of data coming from scientific instruments, sensors, simulations, as well as
from people accumulating data in Web or social nets. The basic objective of DIR is to
infer knowledge from the integrated data organized in networked infrastructures
(such as warehouses, grids, clouds). At the same time, ``Big Data'' movement has
emerged as a~recognition of the increased significance of massive data in various
domains. Open access to large volumes of data, therefore, becomes a~key prerequisite
for discoveries in the XXI~century. Data intensive research denotes a~crosscut of
DIR/IT areas aimed at the creation of effective data analysis technologies for DIR
covering scientific and other data intensive domains (including finance, economy,
social environment, business, etc.).
{\looseness=1

}

  Science endeavors to give a~meaningful description of the world of natural
phenomena using that are known as laws, hypotheses, and theories. Hypotheses,
theories,\linebreak
\begin{center}  %fig1
\vspace*{-6pt}
\mbox{%
 \epsfxsize=72.481mm
 \epsfbox{kal-1.eps}
 }

\vspace*{9pt}

\noindent
{{\figurename~1}\ \ \small{Multiple incarnations of hypotheses}}

\end{center}


\vspace*{6pt}


\noindent
 and laws in their essence have the same fundamental character (Fig.~1)~[2].


  \textit{A scientific hypothesis} is a~proposed explanation of a~phenomenon which
still has to be rigorously tested. In contrast, \textit{a~scientific theory} has undergone
extensive testing and is generally accepted to be the accurate explanation behind an
observation. A~\textit{scientific law} is a~proposition, which points out any such
orderliness or regularity in nature, \textit{the prevalence of an invariable association
between a~particular set of conditions and particular phenomena}. In the exact
sciences, laws can often be expressed in the form of mathematical relationships.
Hypotheses explain laws, and well-tested, corroborated hypotheses become theories
(see Fig.~1). At the same time, the laws do not cease to be laws, just because they did not
appear first as hypotheses and pass through the stage of theories.

  Though theories and laws are different kinds of knowledge, actually, they represent
different forms of the same knowledge construct. Laws are generalizations, principles,
or patterns in nature, and theories are the explanations of those generalizations.
However, classification expressed in Fig.~1 is subjective. Article~[3]
provides examples
showing that the differences between laws, hypotheses, and theories consist only in
that they stand at different levels in their claim for acceptance  depending on how
much empirical evidence is amassed. Therefore, there is no essential difference
between constructs used for expressing hypotheses, theories, and laws. Important role
of hypotheses in scientific research can scarcely be
 overestimated. In the edition of
M.~Poincar$\acute{\mbox{e}}$'s book~[4],
 it is stressed that \textit{without
hypotheses, there is no science}. Thus, it is not surprising that so much attention in the
scientific research and the respective publications is devoted to the methods for
hypothesis manipulation in experimenting and modeling of various phenomena
applying the means of informatics. The idea that the new approaches are needed that
can address both \textit{data-} and \textit{hypothesis-driven sciences} runs all through
this paper.  Such symbiosis alongside with the hypothesis-driven tradition of science
(``first hypothesize-then-experiment'') might cause wide application of another one
that is typified by ``first experiment-then-hypothesize'' mode of research. Often, the
``first experiment'' ordering in DIR is motivated by the necessity of analysis of the
existing massive data to generate a~hypothesis.



  In the course of the present study, paying attention to the issue of
  inductive and deductive
reasoning in hypothesis-driven sciences will be emphasized.  In Fig.~2,
such ways of
knowledge production are shown~[2]. Here, ``generalization'' means any subset of
hypotheses, theories, and laws and ``Evidence'' is any subset of all facts accumulated
in a~specific DIR.

  All researchers collect and interpret empirical evidence through the process called
\textit{induction}. This is a~technique by which individual pieces of evidence are
collected and examined until a~law is discovered or a~theory is invented. Frances
Bacon first formalized induction~\cite{5-kl}. The method of
(naive) induction (see Fig.~2), he suggested, is, in part, the
principal way by which humans traditionally have produced generalizations that
permit predictions. The problem with induction is that\linebreak

\begin{center}  %fig2
\vspace*{-3pt}
\mbox{%
 \epsfxsize=65.106mm
 \epsfbox{kal-2.eps}
 }

\vspace*{6pt}

\noindent
{{\figurename~2}\ \ \small{Enhanced knowledge production diagram}}

\end{center}


%\vspace*{9pt}



\addtocounter{figure}{2}


\noindent
 it is impossible to collect
all observations pertaining to a~given situation in all time~--- past, present, and future.

  The formulation of a~new law begins through induction as facts are heaped upon
other relevant facts. Deduction is useful in checking the validity of a~law. Figure~2
shows that a~valid law would permit the accurate prediction  of  facts  not  yet
known.  Also an \textit{abduction}~\cite{6-kl} is the process of validating a~given
hypothesis through reasoning by successive approximation. Under this principle, an
explanation is valid if it is the best possible explanation of a~set of known data.
Abductive validation is common practice in hypothesis formation in science.
Hypothesis related logic reasoning issues are considered in more details in section~3.

  In~\cite{4-kl}, the useful hypotheses of science are considered to be of two kinds:
  \begin{enumerate}[(1)]
\item the hypotheses which are valuable \textit{precisely} because they are either
verifiable or, else, refutable through a~definite appeal to the tests furnished by
experience; and
\item the hypotheses which, despite the fact that experience suggests them, are
valuable \textit{despite}, or even \textit{because}, of the fact that experience can
neither confirm nor refute them.
\end{enumerate}

  Aspects of science which are determined by the use of the hypotheses of the
second kind are considered in~\cite{4-kl}
as ``constituting an essential human way of viewing nature, an interpretation rather
than a~portrayal or a~prediction of the objective facts of nature, an adjustment of our
conceptions of things to the internal needs of our intelligence.'' According to
Poincar$\acute{\mbox{e}}$'s discussion, the central problem of the logic of
science becomes the problem of the relation between the two fundamentally distinct
kinds of hypotheses, i.\,e., between those which cannot be verified or refuted through
experience and those which can be empirically tested.

  The analysis in this paper will be focused mostly on the modeling of hypotheses of
the first kind, leaving issues of analysis of the relations between such two kinds of
hypotheses to further study.

  The rest of the paper is organized as follows.  Section~2 discusses the basic
concepts defining the role of hypotheses in the formation of scientific knowledge and
the respective organization of the scientific experiments. Approaches for hypothesis
formulation, logical reasoning, hypothesis modeling, and testing are briefly
introduced in section~3. In section~4,
 a~general overview of the basic facilities provided by
informatics for the hypothesis-driven experimentation scenarios, including conceptual
modeling, simulations, statistics and machine learning methods is given. In
section~5, several examples of organization of hypothesis-driven scientific
experiments are included. Concluding remarks summarize the discussion.

\section{Role of Hypotheses in~Scientific Experiments: Basic Principles}

  \noindent
  Normally, scientific hypotheses have the form of a~mathematical model.
Sometimes, one can also formulate them as existential statements, stating that some
particular instance of the phenomenon under examination has some characteristic and
causal explanations, which have the general form of universal statements, stating that
every instance of the phenomenon has a~particular characteristic (e.\,g., \textit{for all
x, if x is a~swan, then x is white}). Scientific hypothesis considered as a~declarative
statement identifies the predicted relationship (associative or causal) between two or
more variables (independent and dependent).  In causal relationship, a~change caused
by the independent variable is predicted in the dependent variable. Variables are more
commonly related in noncausal (associative) way~\cite{7-kl}.

  In experimental studies, the researcher manipulates the independent variable. The
dependent variable is often referred to as consequence or the presumed effect that
varies with a~change of the independent variable. The dependent variable is not
manipulated. It is observed and assumed to vary with changes in the independent
variable. Predictions are made from the independent variable to the dependent
variable. It is the dependent variable that the researcher is interested in understanding,
explaining, or predicting~\cite{7-kl}.

  In case when a~possible correlation or similar relation between variables is
investigated (such as, for example, whether a~proposed medication is effective in treating a~disease, that is, at least to some extent and for some patients), a~few cases in which
the tested remedy shows no effect do not falsify the hypothesis. Instead, statistical
tests are used to determine how likely it is that the overall effect would be observed if
no real relation as hypothesized exists. If that likelihood is sufficiently small, the
existence of a~relation may be assumed. In statistical hypothesis testing, two
hypotheses are compared, which are called the \textit{null hypothesis} and the
\textit{alternative hypothesis}. The null hypothesis states that there is no relationship
between the phenomena (variables) whose relation is under investigation or, at least,
not of the form given by the alternative hypothesis. The alternative hypothesis, as the
name suggests, is the alternative to the null hypothesis: it states that there \textit{is}
some kind of relation.

  Alternative hypotheses are generally used more often than null hypotheses because
they are more desirable to state the researcher's expectations. But in any study that
involves statistical analysis, the underlying null hypothesis is usually
assumed~\cite{7-kl}. It is important that the conclusion ``do not reject the null
hypothesis'' does not necessarily mean that the null hypothesis is true. It suggests that
there is not sufficient evidence against the null hypothesis in favor of the alternative
hypothesis.  Rejecting the null hypothesis suggests that the alternative hypothesis
may be true.

  Any useful hypothesis will enable \textit{predictions by reasoning} (including
\textit{deductive reasoning}). It might predict the outcome of an experiment in a~laboratory setting or the observation of a~phenomenon in nature. The prediction may
also invoke statistics assuming that a~hypothesis must be
  \textit{falsifiable}~\cite{8-kl} and that one cannot regard a~proposition or theory
as scientific if it does not admit the possibility of being shown false. The way to
demarcate between hypotheses is to call \textit{scientific} those for which we can
specify (beforehand) one or more potential falsifiers as the respective experiments.
Falsification was supposed to proceed deductively instead of inductively.
{\looseness=1

}

  Other philosophers of science have rejected the criterion of falsifiability or
supplemented it with other criteria, such as verifiability (only statements about the
world that are empirically confirmable or logically necessary are cognitively
meaningful). They claim that science proceeds by ``induction''~--- that is, by finding
confirming instances of a~conjecture. Popper treated confirmation as never
certain~\cite{8-kl}. However, a~falsification can be sudden and definitive. Einstein
said: ``No amount of experimentation can ever prove me right; a~single experiment
can prove me wrong.'' To scientists and philosophers outside the Popperian
belief~\cite{8-kl}, science operates mainly by induction (confirmation), and also and
less often by disconfirmation (falsification). Its language is almost always one of
induction. For this survey both philosophical treatment of hypotheses are acceptable.
Sometimes such way of reasoning is called the \textit{hypothetico-deductive
method}. According to it, scientific inquiry proceeds by formulating a~hypothesis in a~form that could conceivably be falsified by a~test on observable data. A~test that
could and does run contrary to predictions of the hypothesis is taken as a~falsification
of the hypothesis. A~test that could but does not run contrary to the hypothesis
corroborates the theory.
{\looseness=1

}

  A scientific method involves experiment to test the ability of some hypothesis to
adequately answer the question under investigation. A~prediction enabled by
hypothesis suggests a~test (observation or experiment) for the hypothesis thus
becoming testable. If a~hypothesis does not generate any observational tests, there is
nothing that a~scientist can do with it.

  For example, not testable hypothesis: ``Our universe is surrounded by another,
larger universe, with which we can have absolutely no contact;'' not verifiable
(though testable) hypothesis: ``There are other inhabited planets in the universe;''
scientific hypothesis (both testable and verifiable):  ``Any two objects dropped from
the same height above the surface of the earth will hit the ground at the same time as
long as air resistance is not a~factor'' ({\sf
http://www.batesville.k12.in.us/physics/phynet/\linebreak aboutscience/hypotheses.html}).

  A \textit{problem} (\textit{research question}) should be formulated as an issue of
what relation exists between two or more variables. The problem statement should be
such as to imply possibilities of empirical testing; otherwise, this will not be a~scientific problem.
Problems and hypotheses being generalized relational statements
enable to deduce specific empirical manifestations implied by the problem and
hypotheses. In this process, hypotheses can be deduced from theory and from other
hypotheses. A~problem cannot be scientifically solved unless it is reduced to
hypothesis form, because a~problem is not directly testable~\cite{9-kl}.

  Most formal hypotheses connect concepts by specifying the expected relationships
between \textit{propositions}. When a~set of hypotheses are grouped together, they
become a~type of \textit{conceptual framework}. When a~conceptual framework is
complex and incorporates causality or explanation, it is generally referred to as a~\textit{theory}~\cite{10-kl}.  In general, hypotheses have to reflect the multivariate
complexity of the reality. A~scientific theory summarizes a~hypothesis or a~group of
hypotheses that have been supported with repeated testing. A~theory is valid as long
as there is no evidence to dispute it. \textit{Scientific paradigm} explains the working
set of theories under which science operates.

  Elements of hypothesis-driven research and their relationships are shown in
Fig.~3~\cite{12-kl, 11-kl}. The hypothesis triangle relations, \textit{explains},
\textit{formulates}, and \textit{represents}, are functional in the scientist's final decision in
adopting a~particular model $m_1$ to formulate a~hypothesis~$h_1$, which
is meant to explain phenomenon~$p_1$.

  In~\cite{11-kl}, the lattice structure for hypothesis interconnection is proposed as
shown in Fig.~4. A~hypothesis lattice is formed by considering a~set of hypotheses
equipped with \textit{wasDerivedFrom} as a~strict order (from the bottom to
the top). Hypotheses directly derived from exactly one hypothesis are \textit{atomic},
while those directly derived from at least two hypotheses are \textit{complex}.



  The hypothesis lattice is unfolded into model and phenomena isomorphic lattices
according to the hypothesis triangle (see Fig.~3)~\cite{11-kl}. The lattices are
isomorphic if one takes subsets of~$M$ (Model), $H$ (Hypotheses), and~$P$
(Phenomenon) such that \textit{formulates, explains, and represents} are both
  one-to-one and onto mappings (i.\,e., bijections), seen as structure-preserving
mappings (morphisms). Example of the isomorphic lattice is shown in
  Fig.~\ref{f5-kl}~\cite{11-kl}. This particular lattice corresponds to the case in
Computational Hemodynamics considered in~\cite{11-kl}. Here, model~$m_1$
formulates hypothesis~$h_1$, which explains phenomenon~$p_1$.
Similarly,  $m_2$ formulates~$h_2$, which explains~$p_2$, and so on. Prop-\linebreak\vspace*{-12pt}
\begin{center}  %fig3
\vspace*{-3pt}
 \mbox{%
 \epsfxsize=77.487mm
 \epsfbox{kal-3.eps}
 }

\vspace*{6pt}

\noindent
{{\figurename~3}\ \ \small{Elements of hypothesis-driven research}}

\end{center}


\vspace*{12pt}

\begin{center}  %fig4
\vspace*{-3pt}
\mbox{%
 \epsfxsize=74.477mm
 \epsfbox{kal-4.eps}
 }
 \end{center}

%\vspace*{6pt}

\noindent
{{\figurename~4}\ \ \small{A lattice theoretic representation for hypothesis relationship}}


\vspace*{16pt}



\addtocounter{figure}{2}


\noindent
erties of the
hypothesis lattices and operations over them are considered in~\cite{13-kl}.



  \textit{Models} are one of the principal instruments of modern science. Models can
perform two fundamentally different representational functions: a~model can be a~representation of a~selected part of the world, or a~model can represent a~theory in the
sense that it interprets the laws and hypotheses of that theory.

  Here, let consider scientific models to be representations in both senses at the same
time. One of the most perplexing questions in connection with models is how they
relate to theories. In this respect, models can be considered as a~complement to
theories, as preliminary theories, can be used as substitutions of theories when the
latter are too complicated to handle. Learning about the model is done through
experiments, thought experiments, and simulation. Given a~set of parameters, a~model
can generate expectations about how the system will behave in a~particular situation.
A~model and the hypotheses it is based upon are supported when the model
generates expectations that match the behavior of its real-world counterpart.

\pagebreak

\end{multicols}

\begin{figure} %fig5
\vspace*{1pt}
 \begin{center}
 \mbox{%
 \epsfxsize=155.928mm
 \epsfbox{kal-5.eps}
 }
 \end{center}
 \vspace*{-9pt}
\Caption{Hypothesis lattice unfolded into model and phenomenon isomorphic lattice}
\label{f5-kl}
\end{figure}

\begin{multicols}{2}

  A law generalizes a~body of observations. Generally, a~law represents a~group of
related undisputable hypotheses using a~handful of fundamental concepts and
equations to define the rules governing a~set of phenomena. A~law does not attempt
to explain why something happens~--- it simply states that it does.

  Facilities for support of the hypothesis-driven experimentation will be discussed in
the remaining sections.

\section{Hypothesis Manipulation in~Scientific Experiments}

\subsection{Hypothesis generation}

  \noindent
  Researchers that support rationality of scientific discovery presented several
methods for hypothesis generation, including discovery as abduction, induction,
anomaly detection, heuristics programming, and use of analogies~\cite{14-kl}.

  \textit{Discovery as abduction} characterizes reasoning processes that take place
before a~new hypothesis is justified. The abductive model of reasoning that leads to
plausible hypotheses formulation is conceptualized as an inference beginning with
data. According to~\cite{15-kl}, an abduction happens as follows:
\begin{enumerate}[(1)]
\item some
phenomena $p_1, p_2, p_3,\ldots$ are encountered for which there is no or little
explanation;
\item however,  $p_1, p_2, p_3,\ldots$ would not be surprising if a~hypothesis~$H$ were added. They would certainly follow from something like~$H$
and would be explained by it; and
\item therefore, there is a~good reason for elaborating
a~hypothesis~$H$~--- for proposing it as a~possible hypothesis from which the
assumption $p_1, p_2, p_3,\ldots$ might follow.
\end{enumerate}
 The abductive model of reasoning is
primarily a~process of explaining anomalies or surprising phenomena~\cite{16-kl}.
The scientists' reasoning proceeds abductively from an anomaly to an explanatory
hypothesis in light of which the phenomena would no longer be surprising. There can
be several different hypotheses that can serve as the explanations for phenomena; so,
additionally some criteria for choosing among different hypotheses are required.

  One way to implement abductive model of reasoning is the abductive logic
programming~\cite{17-kl}. Hypothesis generation in abduction logical framework is
organized as follows. During the experiment, some new observations are
encountered. Let~$B$ represents the background knowledge and $O$~is the set of facts
that represents observations. Both~$B$ and~$O$ are the logic programs (set of rules in
some rule language). In addition, $\Gamma$ stands for a~set of literals representing
the set of abducibles, which are candidate assumptions to be added to~$B$ for
explaining~$O$. Given~$B$, $O$, and~$\Gamma$, the hypothesis-generation
problem is to find a~set~$H$ of literals (called a~hypothesis) such that:
\begin{enumerate}[(1)]
\item $B$ and~$H$ entail~$O$;
\item $B$ and~$H$ are consistent; and
\item $H$ is some subset
of~$\Gamma$.
\end{enumerate}
  If all conditions are met, then~$H$ is an explanation of~$O$ (with
respect to~$B$ and~$\Gamma$).  Examples of abductive logic programming systems
include ACLP~\cite{18-kl}, A-system~\cite{19-kl}, ABDUAL~\cite{20-kl}, and
ProLogICA~\cite{21-kl}. Abductive logic programming can also be implemented by
means of Answer Set Programming systems, e.\,g., by the DLV system~\cite{22-kl}.

  The example abductive logic program in ProLogICA describes a~simple model of
the lactose metabolism of the bacterium E.Coli~\cite{21-kl}. The background
knowledge~$B$ describes that E.coli can feed on the sugar lactose if it makes two
enzymes permease and galactosidase. Like all enzymes (E), these are made if they are
coded by a~gene (G) that is expressed. These enzymes are coded by two genes (lac(y)
and lac(z)) in cluster of genes (lac(X)) called an operon that is expressed when the
amounts (amt) of glucose are low and lactose are high or when they are both at
medium level. The abducibles, $\Gamma$, declare all ground instances of the
predicates ``amount'' as assumable. This reflects the fact that in the model, it is not
known what are the amounts at any time of the various substances. This is incomplete
information that should be found out in each problem case that is examined. The
integrity constraints state that the amount of a~substance (S) can only take one value.
 {\small \begin{verbatim}
##  Background Knowledge (B)
feed(lactose):- make(permease),
   make(galactosidase).
make(Enzyme):- code(Gene,Enzyme),express(Gene).
express(lac(X)):-amount(glucose,low),
   amount(lactose,hi).
express(lac(X)):-amount(glucose,medium),
   amount(lactose,medium).
code(lac(y),permease).
code(lac(z),galactosidase).
temperature(low):-amount(glucose,low).
false :- amount(S,V1), amount(S,V2), V1 != V2.

##  Abducibles (Г)
abducible_predicate(amount).

## Observation (O)
feed(lactose).

This goal generates two possible hypotheses:
{amount(lactose,hi), amount(glucose,low)}
{amount(lactose,medium),amount(glucose,medium)}
\end{verbatim}
}

Below, just a~couple of another examples of real rule-based systems, where abductive
logic programming is used, are presented.
Robot Scientist (see subsection~4.4) abductively hypothesizes new
facts about the yeast functional biology by inferring what is missing from a~model~\cite{23-kl}. In~\cite{24-kl}, both abduction and induction are used to
formulate hypotheses about inhibition in metabolic pathways. Augmenting
background knowledge is done with abduction; after that, induction is used for
learning general rules.  Authors of~\cite{25-kl} use SOLAR reasoning system to
abductively generate hypotheses about the inhibitory effects of toxins on the rat
metabolisms.

  The process of discovery is deeply connected also with the search of
\textit{anomalies}. There are a~lot of methods and algorithms to discover anomalies.
Anomaly detection is an important research problem in data mining
aimed at search of the
objects that are considerably dissimilar, exceptional, and inconsistent with respect to
the majority data in an input database~\cite{26-kl}.

  \textit{Analogies} play several roles in science. Not only do they contribute to
discovery but they also play a~role in the development and evaluation of scientific
theories (new hypotheses) by analogical reasoning.

\subsection{Hypothesis evaluation}

  \noindent
  Being testable and falsifiable, a~scientific hypothesis provides a~solid basis to its
further modeling and testing. There are several ways to do it, including the use of
statistics, machine learning, and logic reasoning techniques.

\subsubsection{Statistical testing of hypotheses }

  \noindent
  The classical (frequentist) and Bayesian statistic approaches are applicable for
hypothesis testing and selection. Brief summary of the basic differences between
these approaches are as follows~\cite{27-kl}.

  Classical (frequentist) statistics is based on the following beliefs:
  \begin{itemize}
\item probabilities refer to relative frequencies of events. They are objective
properties of the real world;
\item parameters of hypotheses (models) are fixed, unknown constants. Because
they are not fluctuating, probability statements about parameters are meaningless; and
\item statistical procedures should have well-defined long-run frequency
properties.
\end{itemize}

  In contrast, Bayesian approach takes the following assumptions:
  \begin{itemize}
\item probability describes the degree of subjective belief, not the limiting
frequency. Probability statements can be made about things other than data,
including hypotheses (models) themselves as well as their parameters;
and
\item inferences about a~parameter are made by producing its probability
distribution~--- this distribution quantifies the uncertainty of our knowledge about
that parameter. Various point estimates, such as expectation value, may then be
readily extracted from this distribution.
\end{itemize}

  The Bayesian interpretation of probability can be seen as an extension
of propositional logic that enables reasoning with hypotheses, i.e.,
the propositions whose truth or falsity is uncertain.

  Bayesian probability belongs to the category of evidential probabilities; to evaluate
the probability of a~hypothesis, the Bayesian probabilist specifies some prior
probability, which is then updated in the light of new,
relevant data (evidence)~\cite{28-kl}. The Bayesian interpretation provides a~standard set of procedures and formulae to perform this calculation.

\vspace*{-6pt}

  \paragraph*{Hypothesis testing in classical statistic style.} After null and alternative
hypotheses are stated, some statistical assumptions about data samples should be
done, e.\,g., assumptions about statistical independence or distributions of observations.
Failure in providing correct assumptions leads to the invalid test results.

  A common problem in classical statistics is to ask whether a~given sample is
consistent with some hypothesis. For example, one might be interested in whether a~measured value~$x_i$, or the whole set $\{x_i\}$, is consistent with being drawn
from a~Gaussian distribution $N(\mu ,\sigma)$. Here, $N(\mu,\sigma$) is the
\textit{null hypothesis}.

  It is always assumed that we know how to compute the probability of a~given
outcome from the null hypothesis: for example, given the cumulative distribution
function, $0 \leq H_0(x) \leq 1$, the probability that we would get a~value at least as
large as $x_i$ is $p(x > x_i ) = 1 - H_0(x_i)$ and is called the $p$-\textit{value}.
Typically, a~threshold~$p$ value is adopted, called \textit{the significance
level}~$\alpha$, and the null hypothesis is rejected when $p\leq \alpha$ (e.\,g., if
$\alpha = 0.05$ and $p < 0.05$, the null hypothesis is rejected at a~0.05~significance
level). If one fails to reject a~hypothesis, it does not mean that
its correctness is proved
because it may be that the sample is simply not large enough to detect an effect.

  When performing these tests, one can meet with two types of errors, which
statisticians call \textit{Type~I and Type~II errors}. Type~I errors are
the cases when the
null hypothesis is true but incorrectly rejected. In the context of source detection,
these errors represent spurious sources or, more generally, false positives (with
respect to the alternative hypothesis). The false-positive probability when testing a~single datum is limited by the adopted significance level~$\alpha$. Cases when the
null hypothesis is false but it is not rejected are called Type~II errors (missed
sources, or false negatives (again, with respect to the alternative hypothesis)). The
false-negative probability when testing a~single datum is usually called~$\beta$ and
is related to \textit{the power of}~$\alpha$~\textit{test as} $(1 -\beta)$. Hypothesis
testing is intimately related to comparisons of distributions.

  As the significance level~$\alpha$ is decreased (the criterion for rejecting the null
hypothesis becomes more conservative), the number of false positives decreases and
the number of false negatives increases. Therefore, there is a~trade-off to be made to
find an optimal value of~$\alpha$, which depends on the relative importance of false
negatives and positives in a~particular problem. Both the acceptance of false
hypotheses and the rejection of true ones are errors that scientists should try to avoid.
There is discussion as to what states of affairs is less desirable; many people think
that the acceptance of a~false hypothesis is always worse than failure to accept a~true
one and that science should in the first place try to avoid the former kind of error.

  When many instances of hypothesis testing are performed, a~process called
\textit{multiple hypothesis testing}, the fraction of false positives can significantly
exceed the value of~$\alpha$. The fraction of false positives depends not only
on~$\alpha$ and the number of data points, but also on the number of true positives
(the latter is proportional to the number of instances when an alternative hypothesis is
true).

  Depending on data type (discrete vs.\ continuous random variables) and what one
can assume (or not) about the underlying distributions, and the specific question
one asks, different statistical tests can be used. The underlying idea of statistical tests is to
use data to compute an appropriate statistic and then compare the resulting
  data-based value to its expected distribution. The expected distribution is evaluated
by \textit{assuming that the null hypothesis is true}. When this expected distribution
implies that the data-based value is unlikely to have arisen from it by chance (i.\,e.,
the corresponding $p$ value is small), the null hypothesis is rejected with some
threshold probability~$\alpha$, typically 0.05 or 0.01 ($p<\alpha$). Note again
that~$p>\alpha$ does \textit{not} mean that the hypothesis is \textit{proven} to be
correct.

  The number of various statistical tests in the literature is overwhelming and their
applicability is often hard to decide (see~\cite{29-kl, 30-kl} for variety of statistical
methods in SPSS (Statical Package for the Social Sciences)).
When the distributions are not known, tests are called
nonparametric, or distribution-free tests. The most popular nonparametric test is the
Kolmogorov--Smirnov (K-S) test, which compares the cumulative distribution
function, $F (x)$, for two samples, $\{x_{1i}\}$, $i = 1,\ldots  , N_1$, and $\{x_{2i}\}$,
$i = 1,\ldots  , N_2$. The K-S test is not the only option for nonparametric comparison
of distributions. The Cram$\acute{\mbox{e}}$r\,--\,von Mises criterion, the Watson
test, and the Anderson--Darling test are similar in spirit to the K-S test, but consider
somewhat different statistics. The Mann--Whitney--Wilcoxon test (or the Wilcoxon
rank-sum test) is a~nonparametric test for testing whether two data sets are drawn
from distributions with different location parameters (if these distributions are known
to be Gaussian, the standard classical test is called the~$t$~test). A~few standard
statistical tests can be used when it is known, or can be assumed, that both~$h(x)$ and~$f(x)$
are the Gaussian distributions (e.\,g., the Anderson--Darling test, the Shapiro--Wilk
test)~\cite{27-kl}. More on statistical tests can be found
in~\cite{27-kl, 29-kl, 30-kl, 31-kl}.

\vspace*{-6pt}

  \paragraph*{Hypothesis (model) selection and testing in Bayesian style.} The
Bayesian approach can be thought of as formalizing the process of continually
refining our state of knowledge about the world, beginning with no data (as encoded
by the \textit{prior}), then updating that by multiplying in the likelihood once the
data  are observed to obtain the \textit{posterior}. When more data are taken, then the
posterior based on the first data set can be used as the prior for the second analysis.
Indeed, the data sets can be different.

  The question often arises as to which is the `best' model (hypothesis) to use;
`\textit{model selection}' is a~technique that can be used when we wish to
discriminate between competing models (hypotheses) and identify the best model
(hypothesis) in a~set, $\{M_1,\ldots , M_n\}$, given the data.

  Let remind the basic notation. The Bayes theorem can be applied to
calculate the posterior probability $p(M_j\vert d)$ for each model (or hypothesis) $M_j$
representing our state of knowledge about  the truth of the model (hypothesis) in the
light of the data~$d$ as follows:
  $$
  p(M_j\vert d) = p(d\vert M_j) \fr{p(M_j)}{p(d)}
  $$
  where $p(M_j)$ is the prior belief in the model (hypothesis) that represents our
state of knowledge (or ignorance) about the truth of the model (hypothesis) before
the  current data have been analyzed; $p(d\vert M_j)$ is the model
 (hypothesis)
\textit{likelihood} (represents the probability that some data are produced under the
assumption of this model);  and $p(d)$ is the normalization constant given by
  $$
  p(d) = \sum\limits_i p(d\vert M_i) p(M_i)\,.
  $$

  The relative `goodness' of models is given by a~comparison of their posterior
probabilities; so, to compare two models $M_a$ and~$M_b$, let look at the ratio of the
model posterior probabilities:
  $$
  \fr{p(M_a\vert d)}{p(M_b\vert d)} = \fr{ p(d\vert M_a)p(M_a)}{p(d\vert M_b)p(M_b)}\,.
  $$
  The Bayes factor, $B_{ab}$, can be computed as the ratio of the model likelihoods:
  $$
  B_{ab} = \fr{p(d\vert M_a)}{p(d\vert M_b)}\,.
  $$
  Empirical scale for evaluating the strength of evidence from the Bayes
factor~$B_{ij}$ between two models is shown in the table~\cite{32-kl}.

\vspace*{3pt}

%\begin{table*}
{\small
  \begin{center}


\begin{tabular}{ccc}
\multicolumn{3}{c}{Strength of evidence for Bayes factor $B_{ij}$ for two models}\\
&&\\[-6pt]
\hline
  $\vert \ln B_{ij}\vert$&Odds&Strength of evidence\\
  \hline
  $<$1.0\hphantom{$<$}&$< 3 : 1$&Inconclusive\\
  1.0&$\sim 3 : 1$&Weak evidence\\
  2.5&$\sim 12 : 1$\hphantom{9}&Moderate evidence\\
  5.0&$\sim150 : 1$\hphantom{99}&Strong evidence\\
  \hline
  \end{tabular}
  \end{center}}
  %  \end{table*}

  \vspace*{12pt}


\noindent


  The Bayes factor gives a~measure of the `goodness' of a~model regardless of the
prior belief about the model; the higher the Bayes factor, the better the model is. In
many cases, the prior belief in each model in the set of proposed models will be
equal; so, the Bayes factor will be equivalent to the ratio of the posterior probabilities
of the models. The `best' model in the Bayesian sense is the one which gives the best
fit to the data with the smallest parameter space.

  A special case of model (hypothesis) selection is \textit{Bayesian hypothesis
testing}~\cite{27-kl, 33-kl}. Taking $M_1$ to be the ``null'' hypothesis, one can ask
whether the data supports the alternative hypothesis~$M_2$, i.\,e., whether one
can reject the null hypothesis. Taking equal priors $p(M_1) = p(M_2)$, the odds
ratio is
  $$
  B_{21} =\fr{p(d\vert M_1)}{p(d\vert M_2)}\,.
  $$

  The inability to reject $M_1$ in the absence of an alternative hypothesis is very different from the hypothesis testing procedure in classical statistics. The latter procedure rejects the null hypothesis if it does not provide a~good description of the data, that is, when it is very unlikely that the given data could have been generated as
prescribed by the null hypothesis. In contrast, the Bayesian approach is
based on the posterior rather than on the data likelihood
and cannot reject a~hypothesis if there are no alternative
explanations for observed data~\cite{27-kl}.

  Comparing classical and Bayesian approaches~\cite{27-kl}, it is rare for a~mission-critical analysis be done in the ``fully Bayesian'' manner, i.\,e., without the
use of the frequentist tools at the various stages. Philosophy and
beauty aside, the reliability and efficiency of the underlying computations
required by the Bayesian
framework are the main practical issues. A~central technical issue at the
heart of this is that it is much easier to do optimization (reliably
and efficiently) in high dimensions than it is to do integration in high
dimensions.
Thus, the usable machine learning methods, while there are ongoing
efforts to adapt them to Bayesian framework, are almost all rooted in
frequentist methods.

  Most users of Bayesian estimation methods, in practice, are likely to use a~mix of Bayesian and frequentist tools. The reverse is also true~--- frequentist data analysts, even if they stay formally within the frequentist framework, are often influenced by
``Bayesian thinking,'' referring to ``priors'' and ``posteriors.'' The most advisable
position is probably to know both paradigms well, in order to make informed
judgments about which tools to apply in which situations~\cite{27-kl}.  More details on Bayesian style of hypothesis testing can be found
in~\cite{27-kl,  28-kl, 33-kl}.

\vspace*{-8pt}

\subsubsection{Logic-based hypothesis testing}

%\vspace*{-1pt}

\noindent
  According to the hypothetico-deductive approach, the hypotheses are tested by
deducing predictions or other empirical consequences from general theories. If these
predictions are verified by experiments, this supports the hypothesis. It should be noted that not
everything that is logically entailed by a~hypothesis can be confirmed by
a proper test for it. The relation between hypothesis and evidence is often
empirical rather than logical. A~clean deduction of empirical consequences from
a~hypothesis,
as it may sometimes exist in physics, is practically inapplicable in biology.
Thus, entailment of the evidence by hypotheses under test is neither sufficient
nor necessary
for a~good test. Inference to the best explanation is usually construed
as a~form of inductive inference (see abduction in subsection~3.1) where hypothesis' explanatory
credentials are taken to indicate its truth~[34].

  An inductive logic is a~system of evidential support that extends deductive
  logic to less-than-certain inferences.  For valid deductive arguments, the
  premises logically
entail the conclusion where the entailment means that the truth of the premises
provides a~guarantee of the truth of the conclusion. Similarly, in a~good
inductive argument, the premises should provide some degree of support for the conclusion,
where such support means that the truth of the premises indicates with some degree of strength that the conclusion is true. If the logic of good inductive arguments is to be of any real value, the measure of support it articulates should meet the Criterion of
Adequacy (CoA): as evidence accumulates, the degree to which the collection of true evidence statements comes to support a~hypothesis, as measured by the logic, should
tend to indicate that the hypotheses are probably false or probably true.
  In~\cite{35-kl}, the extent to which a~kind of logic based on the Bayes theorem
can estimate how the implications of hypotheses about evidence claims influences the degree to which hypotheses are supported is discussed in detail. In particular, it is
shown how such a~logic may be applied to satisfy the CoA: as evidence accumulates, false hypotheses will very probably come to have evidential support values (as
measured by their posterior probabilities) that approach~0; and as this happens, a~true hypothesis will very probably acquire evidential support values (measured by
their posterior probabilities) that approach~1.

\vspace*{-8pt}

\subsubsection{Parameter estimation }

%\vspace*{-1pt}

  \noindent
  Models (hypotheses) are typically described by parameters~$\theta$  whose
  values are to be estimated from data. The authors describe this process according
  to~\cite{27-kl}.
For a~particular model~$M$ and prior information~$I$, one gets:
  $$
  p(M, \theta\vert d, I) =
  \fr{p(d\vert M, \theta, I) p(M, \theta\vert I)}{p(d\vert I)}\,.
  $$
  The result $p(M, \theta\vert d, I)$ is called the \textit{posterior} probability density
function (pdf) for model~$M$ and parameters~$\theta$, given data~$d$ and other
prior information~$I$. This term is a~$(k + 1)$-dimensional pdf in the space spanned
by $k$~model parameters and the model~$M$. The term $p(d\vert M, \theta, I)$ is
the \textit{likelihood} of data \textit{given} some model~$M$ and some fixed values of parameters~$\theta$
describing it and all other prior information~$I$. The term
$p(M, \theta\vert I)$ is the \textit{a~priori} joint probability for model~$M$ and its
parameters~$\theta$ in the absence of any of the data used to compute likelihood
and is often simply called the \textit{prior}.

  In the Bayesian formalism, $p(M, \theta\vert d, I)$ corresponds to the state of our \textit{knowledge} (i.\,e., belief) about a~model and its parameters, given data~$d$. To simplify the notation, $M(\theta)$ will be substituted by~$M$ whenever the
absence of explicit dependence on~$\theta$ is not confusing.
A~completely Bayesian data analysis has the following conceptual steps.
  \begin{enumerate}[1.]
\item Formulation of the data likelihood $p(d\vert M, I)$.\\[-14pt]
\item Choice of the prior $p(\theta\vert M,I)$, which incorporates all other
knowledge that might exist, but is \textit{not} used when computing the likelihood
(e.\,g., prior measurements of the same type, different measurements, or simply an
uninformative prior). Several methods for constructing ``objective'' priors have
been proposed. One of them is the \textit{principle of maximum entropy} for
assigning uninformative priors by maximizing the entropy over a~suitable set of
pdfs, finding the distribution that is least informative (given the constraints).
Entropy maximization with no testable information takes place under a~single
constraint: the sum of the probabilities must be one. Under this constraint, the
maximum entropy for a~discrete probability distribution is given by the uniform
distribution.\\[-14pt]
\item Determination of the posterior $p(M\vert d, I)$, using Bayes theorem above.
In practice, this step can be computationally intensive for complex
multidimensional problems.\\[-14pt]
\item The search for the best model~$M$ parameters, which maximizes $p(M\vert
d, I)$, yielding the \textit{maximum a~posteriori} (MAP) estimate. This point
estimate is the natural analog to the \textit{maximum likelihood estimate} (MLE)
from classical statistics.\\[-14pt]
\item Quantification of uncertainty in parameter estimates, via \textit{credible
regions}. As in MLE, such an estimate can be obtained analytically by doing
mathematical derivations specific to the chosen model. The same as in MLE, various
numerical techniques can be used to simulate samples from the posterior. This can be viewed as an analogy to the frequentist approach, which can simulate draws of samples from the true underlying distribution of the data. In both cases, various descriptive statistics can then be computed on such samples to examine the
uncertainties surrounding the data and estimators of
model parameters based on that data.\\[-14pt]
\item Hypothesis testing as needed to make other conclusions about the model
(hypothesis) or parameter estimates.
\end{enumerate}

\vspace*{-6pt}

\subsection{Algorithmic generation and evaluation of~hypotheses}

\vspace*{-2pt}

\noindent
  Two cultures of data analysis (\textit{formulaic modeling}\footnote{In~\cite{36-kl},
instead of ``formulaic modeling,'' the term ``data modeling'' is used that looks misleading in the computer science
context.} and \textit{algorithmic modeling}) distinguished here in accordance
with~\cite{36-kl} can be applied to the hypothesis extraction and generation based on data.

  \textit{Formulaic modeling} is a~process for estimating the relationships among variables. It includes many techniques for modeling and analyzing several variables,
when the focus is on the formulae $y = f(x)$ that give a~relation specifying a~vector of dependent variables~$y$ in terms of a~vector of independent variables~$x$. In a~statistics experiment
(based on various regression techniques), the dependent variable
defines the event studied and is expected to change whenever the independent
variable (\textit{predictor} variables, extraneous variables) is altered.
Such methods as linear regression, logistic regression, and multiple regression are
the well-known examples of the representatives of this modeling approach.

%\pagebreak

  In the \textit{algorithmic modeling} culture, the approach is to find an algorithm
that operates on~$x$ to predict the responses~$y$. What is observed is a~set of $x$'s that go in and a~subsequent set of $y$'s that come out. Predictive accuracy and properties of the algorithms (such as,
for example, their convergence if they are iterative) are the issues to be investigated. \textit{Machine learning algorithms} focus on prediction, based on known properties learned from the training data. Such machine learning algorithms as decision tree, association rule, neural networks, support vector
machines as well as other techniques of learning in Bayesian and probabilistic
models~\cite{38-kl, 37-kl} are examples of the methods that belong to this second
culture.

  The models that best emulate the nature in terms of predictive accuracy are also the
most complex and inscrutable. Nature forms the outputs~$y$
from the inputs~$x$ by means of a~black box with complex and unknown interior.
Current accurate
prediction methods are also \textit{complex black boxes} (such as neural nets,
forests, support vectors). So, we are facing two black boxes, where ours seem
only slightly less inscrutable than nature's~\cite{36-kl}. In a~choice
between \textit{accuracy} and
\textit{interpretability}, in applications, people sometimes prefer interpretability.

  However, the goal of a~model is not interpretability (a~way of getting information), but getting useful, accurate information about the relation between the response and
predictor variables. It is stated in~\cite{36-kl} that algorithmic models can
give better predictive accuracy than formulaic models, providing also better
information about the underlying mechanism. And actually, this is what the goal of statistical analysis is.
The researchers should be focused on solving the problems instead of asking
what regression model they can create.

  An objection to this idea (expressed by Cox) is that prediction without some understanding of underlying process and linking with other sources of information
becomes more and more tentative. Due to that, it is suggested to construct the
stochastic calculation models that summarize the understanding of the phenomena under study. One of the objectives of such approach might be an understanding and
test of hypotheses about underlying process. Given the relatively small sample size,
following such direction could be productive. But data characteristics are rapidly changing. In many of the most interesting current problems, the idea of starting with
a~formal model is not tenable. The methods used in statistics for small
sample sizes and a~small number of variables are not applicable.
Data analytics need to be more
pragmatic. Given a~statistical problem, find a~good solution, whether
it is a~formulaic model, an algorithmic model, or a~Bayesian model
 or a~completely different approach.
 {\looseness=-1

 }

  In the context of the hypothesis-driven analysis, one should pay attention to the question how far can we go applying the algorithmic modeling for hypothesis
generation and testing. Various approaches to machine learning use related to
hypothesis formation and selection can be found
in~\cite{27-kl, 36-kl, 37-kl}.

  Besides machine learning, an interesting example of algorithmic generation of hypotheses can be found in the IBM Watson project~\cite{39-kl} where the
symbiosis of the  general-purpose reusable natural language processing (NLP) and knowledge representation and reasoning (KRR) technologies (under the name
DeepQA) is exploited for answering arbitrary questions over the existing natural language documents as well as structured data resources. Hypothesis generation takes the results of question analysis and produces candidate answers by searching the available data sources and extracting answer-sized snippets from the search results.
Each candidate answer plugged back into the question is considered a~hypothesis, which the system has to prove correct with some degree of confidence. After
merging, the system must rank the hypotheses and estimate confidence based on their merged scores.
A~machine-learning approach adopted is based on running the system over a~set of training questions with known answers and training a~model based on the scores. An important consideration in dealing with NLP-based scorers is that the
features they produce may be quite sparse, and so, accurate confidence estimation requires the application of confidence-weighted learning techniques~\cite{39-kl}~---
a~new class of online learning methods that maintain a~probabilistic measure of confidence in each parameter. It is important to note that instead of statistics based hypothesis testing, contextual evaluation of a~wide range of loosely coupled
probabilistic question and semantic based content analytics is applied for scoring different questions (hypotheses) and content interpretations. Training different
models on different portions of the data in parallel and combining the learned classifiers into a~single classifier
allow to make the process applicable to the large collections of data. More details on that can be found
in~\cite{39-kl, 40-kl} as well as in other Watson project related publications.
{\looseness=1

}

\subsection{Bayesian motivation for discovery }

\noindent
  One way for discriminating between competing models of some phenomenon is to use Bayesian model selection approach
  (see paragraph~3.2.1), the Bayesian evidences for each of
the proposed models (hypotheses) can be computed and the models can then be
ranked by their Bayesian evidence. This is a~good method for identifying which is the best model in a~given set of models, but it gives no indication of the \textit{absolute
goodness} of the model. Bayesian model selection says nothing about the
\textit{overall quality} of the set of models (hypotheses) as a~whole~---
the best model in the set may merely be the best of in a~set of poor models.
Knowing that the best model in the current set of models is not particularly
good model would provide \textit{motivation to search for a~better model} and,
hence, may lead to model discovery.

  One way of assigning some measure of the absolute goodness of a~model is
  to use the concept of Bayesian doubt first introduced in~\cite{41-kl}. Bayesian doubt
works by comparing all the known models in a~set with an idealized model, which acts as a~benchmark model.

  An application of the Bayesian doubt method for the cosmological model building
  is given in~\cite{32-kl, 42-kl}. One of the most important questions in cosmology is to identify the fundamental model underpinning the vast amount of observations nowadays available. The so-called `cosmological concordance model' is based on the cosmological principle (i.\,e., the Universe is isotropic and homogeneous, at least on large enough scales) and on the hot big bang scenario, complemented by an inflationary epoch. This remarkably simple model is able to explain with only half a~dozen free parameter observations spanning
  a~huge range of time and length-scales.
Since both a~cold dark matter (CDM) and a~cosmological constant ($\Lambda$)
component are required to fit the data, the concordance model is often referred
to as `the $\Lambda$CDM model.'

   Several different types of explanation are possible for the apparent late time
acceleration of the Universe, including different classes of dark energy model such as $\Lambda$CDM, $w$CDM; theories of modified gravity; void models or the back reaction~\cite{32-kl}. The methodology of Bayesian doubt which gives an absolute measure of the degree of goodness of a~model has been applied to the issue of
whether the $\Lambda$CDM model should be doubted.

  The methodology of Bayesian doubt dictates that an unknown idealized
model~$X$ should be introduced against which the other models may be compared.
Following~\cite{41-kl}, `doubt' may be defined as the posterior probability of the unknown model:
  $$
  D \equiv  p(X\vert d) = \fr{p(d\vert X)p(X)}{p(d)}\,.
  $$
  Here, $p(X)$ is the prior doubt, i.\,e., the prior on the unknown model, which represents the degree of belief that the list of known models does not contain the true model. The sum of  all the model
  priors must be unity.

  The methodology of Bayesian doubt requires a~baseline model
  (the best model in the set of known models), for which, in this application,
  the $\Lambda$CDM has been chosen. The average Bayes factor
  between $\Lambda$CDM and each of the known models is given by:
  $$
  \langle B_{i\Lambda}\rangle \equiv \fr{1}{N} \sum\limits_{i=1}^N
B_{i\Lambda}\,.
  $$

  The ratio $R$ between the posterior doubt and prior doubt,
  which is called the relative change in doubt, is:
  $$
  R\equiv \fr{D}{p(X)}\,.
  $$

  For doubt to grow, i.\,e., the posterior doubt to be
  greater than the prior doubt ($R\ll 1$), the Bayes factor between
  the unknown model~$X$ and the baseline model must be much greater
  than the average Bayes factor:
  $$
  \fr{\langle B_{i\Lambda}\rangle}{BX\Lambda} \ll 1\,.
  $$

  To genuinely doubt the baseline model, $\Lambda$CDM, it is not sufficient that
$R > 1$, but additionally, the probability of $\Lambda$CDM must also decrease
such that its posterior probability is greater than its prior probability, i.\,e.,
$p(\Lambda \vert d) < p(\Lambda)$. One can define:
  $$
  R_\Lambda \equiv \fr{p(\Lambda\vert d)}{p(\Lambda)}\,.
  $$

  For $\Lambda$CDM to be doubted, the following two conditions must be fulfilled:
  $$
  R>1\,;\quad R_\lambda <1\,.
  $$
  If these two conditions are fulfilled, then it suggests that the set of
  known models is incomplete, and gives motivation to search for a~better
  model not yet included, which may lead to model discovery.

  In~\cite{41-kl}, a~way of computing an absolute upper bound for $p(d\vert X)$
achievable among the class of known models  has been proposed. Finally, it was
found that current cosmic microwave background (CMB), matter power spectrum
(mpk), and Type~Ia supernovae (SNIa) observations do not require the introduction of
an alternative model to the baseline $\Lambda$CDM model. The upper bound of the Bayesian evidence for a~presently unknown dark energy model against
$\Lambda$CDM gives only weak evidence in favor of the unknown model. Since
this is an absolute upper bound, it was concluded that $\Lambda$CDM remains a~sufficient phenomenological description of currently available observations.

\section{Facilities for~the~Scientific Hypothesis-Driven Experiment Support}
\subsection{Conceptualization of~scientific experiments}

  \noindent
  Data intensive research increasingly becomes dependent on computational resources to aid complex
researches. It becomes paramount to offer scientists mechanisms to manage the
variety of knowledge produced during such investigations. Specific conceptual
modeling facilities~\cite{43-kl} are investigated to allow scientists to represent
scientific hypotheses, models, and associated computational or simulation
interpretations which can be compared against phenomena observations (see Fig.~3). The model allows scientists to record the existing knowledge about an observable
investigated phenomenon, including a~formal mathematical interpretation of it, if any.
Model evolution and model sharing need also to be supported taking either a~mathematical or computational view (e.\,g., expressed by scientific workflows).
Declarative representation of scientific model allows scientists to concentrate on the
scientific issues to be investigated. Hypotheses can be used also to bridge the gap between an ontological description of studied phenomena and the simulations.
Conceptual views on scientific domain entities allow for searching for definitions supporting scientific models sharing among different scientific groups.

  In~\cite{11-kl}, the engineering of hypothesis as linked data is addressed.
A~semantic view on scientific hypotheses shows their existence apart from a~particular statement formulation in some mathematical framework. The mathematical
equation is considered as not enough to identify the hypothesis: first,
because it must be physically interpreted, and second, because there can be many
ways to formulate the  same hypothesis. The link to a~mathematical expression,
however, brings to the  hypothesis concept higher semantic precision.
Another link, in addition, to an explicit  description of the explained
phenomenon (emphasizing its ``physical interpretation'')  can bring forth the
intended meaning. By dealing with that hypothesis as a~conceptual  entity,
the scientists make it possible to change its statement formulation or even
to  assert a~semantic mapping to another incarnation of the hypothesis in
case someone  else reformulates it.

  In~\cite{43-kl}, the following elements related to hypothesis-driven science are
conceptualized: a~phenomenon observed, a~model interpreting this phenomenon, the
metadata defining the related computation together with the simulation definition
(for
simulation, a~declarative logic-based language is proposed).
In this work, specific attention is devoted to hypothesis definition.
The explanation, a~scientific hypothesis  conveys, is a~relationship between the causal phenomena and the simulated one,
namely, that the simulated phenomenon is caused by or produced under the
conditions set by the causal phenomena. By running the simulations defined by the antecedents in the causal relationship, the scientist aims at providing hypothetical  analysis of the studied phenomenon.

  Thus, the scientific hypothesis becomes an element of the scientific model that may replace a~phenomenon. When computing a~simulation based on a~scientific
hypothesis, i.\,e., according to the causal relationship it establishes, the output results
may be compared against phenomenon observations to assess the quality of the
hypothesis. Such interpretation provides for bridging the gap between qualitative
description of the phenomenon domain (scientific hypotheses may be used in
qualitative (i.\,e., ontological) assertions) and the corresponding quantitative
valuation obtained through simulations. According to the approach~\cite{43-kl},
complex scientific models can be expressed as the composition of computation
models similarly to database views.
{\looseness=1

}

\subsection{Hypothesis space browsers}

  \noindent
  In the HyBrow (Hypothesis Space Browser) project~\cite{44-kl}, the hypotheses
for the biology domain are represented as a~set of first-order predicate calculus
sentences. In conjunction with an axiom set specified as rules that model known
biological facts over the same universe and experimental data, the knowledge base
may contradict or validate some of the sentences in hypotheses, leaving the remaining
ones as candidates for new discovery. As more experimental data are obtained and
rules are identified, discoveries become positive facts or are contradicted. In the case of
contradictions, the rules that caused the problems must be identified and eliminated
from the theory formed by the hypotheses. In such model-theoretical approach, the
validation of hypotheses considers the satisfiability of the logical implications
defined in the model with respect to an interpretation. This might be applicable also
for simulation-based research, in which validation is solved based on the
quantitative analysis between the simulation results and the
  observations~\cite{43-kl}. HyBrow is based on an OWL ontology and application-
level rules to contradict or validate hypothetical statements. HyBrow provides for
designing hypotheses and evaluating them for consistency with existing knowledge
and uses an ontology of hypotheses to represent hypotheses in machine understandable
form as relations between objects (agents) and processes~\cite{45-kl}.

  As an upgrade of HyBrow, the HyQue~\cite{46-kl} framework adopts linked data
technologies and employs Bio2RDF linked data to add to HyBrow semantic
interoperability capabilities. HyBrow/HyQue's hypotheses are domain-specific
statements that correlate biological processes (seen as events) in the First-Order
Logic (FOL). Hypotheses are formulated as instances of the HyQue Hypothesis
Ontology and are evaluated through a~set of SPARQL queries against
  biologically-typed OWL and HyBrow data. The query results are scored in terms
of how the set of events correspond to background expectations. A score indicates the
level of support the data lend the hypothesis.  Each event is evaluated independently
in order to quantify the degree of support it provides for the hypothesis posed.
Hypothesis scores are linked as properties to the respective hypothesis.

  OBI (the Ontology for Biomedical Investigations) project
  ({\sf http://obi-ontology.org}) aims to model the design of an investigation: the
protocols, the instrumentation, and the materials used in experiments and the data
generated~\cite{47-kl}. Ontologies such as EXPO and OBI enable the recording of
the whole structure of scientific investigations: how and why an investigation was
executed, what conclusions were made, the basis for these conclusions, etc. As a~result of these generic ontology development efforts, the Minimum Information about
a Genotyping Experiment (MIGen) recommends the use of terms defined in OBI. The
use of a~generic or a~compliant ontology to supply terms will stimulate
  cross-disciplinary data-sharing and reuse. As much detail about an investigation as
possible in order to make the investigation more reproducible and reusable can be
collected~\cite{48-kl}.

  Hypothesis modeling is embedded into the knowledge infrastructures being
developed in various branches of science. One example of such infrastructure is
considered under the name SWAN~--- a~Semantic Web Application in
Neuromedicine~\cite{47-kl}. SWAN is a~project for developing an integrated
knowledge infrastructure for the Alzheimer disease (AD) research community.
SWAN incorporates the full biomedical research knowledge lifecycle in its
ontological model, including support for personal data organization, hypothesis
generation, experimentation, laboratory data organization, and digital prepublication
collaboration. The common ontology is specified in an RDF Schema. SWAN's
content is intended to cover all stages of the ``truth discovery'' process in biomedical
research, from formulation of questions and hypotheses to capture of experimental
data, sharing data with colleagues, and ultimately, the full discovery and publication
process.

  \begin{figure*}[b] %fig6
  \vspace*{1pt}
 \begin{center}
 \mbox{%
 \epsfxsize=160.208mm
 \epsfbox{kal-6.eps}
 }
 \end{center}
 \vspace*{-9pt}
  \Caption{Elements of the scientific hypothesis model}
  \label{f6-kl}
  \end{figure*}

  Several information categories created and managed in SWAN are defined as
subclasses of Assertion. They include Publication, Hypothesis, Claim, Concept,
Manuscript, DataSet, and Annotation. An Assertion may be made upon any other
Assertion, or upon any object specifiable by URL. For example, a~scientist can make
a Comment upon, or classify, the Hypothesis of another scientist. Linking to objects
``outside'' SWAN  by URL allows one to use SWAN as metadata to organize, for
example, all one's PDFs of publications, or the Excel files in which one's
laboratory data are stored, or all the websites of tools  relevant to Neuroscience.
Annotation may be structured or unstructured. Structured annotation means attaching
a Concept (tag or term) to an Assertion. Unstructured annotation means attaching free
text. Concepts are nodes in controlled vocabularies, which may also be hierarchical
(taxonomies).

\subsection{Scientific hypothesis formalization}

  \noindent
  An example showing in Fig.~\ref{f6-kl} the diversity of the components of a~scientific hypothesis model has been borrowed from the applications in Neuroscience
[43, 49] and in a~human cardiovascular system in Computational
Hemodynamics~\cite{11-kl, 50-kl}. The formalization of a~scientific hypothesis was
provided by a~mathematical model, by a~set of differential equations for continuous
processes, quantifying the variations of physical quantities in continuous space--time,
and by the mathematical solver (HEMOLAB) for discrete processes. The
mathematical equations were represented in MathML, enabling models interchange
and reuse.

  In~\cite{51-kl}, the formalism of quantitative process models is presented that
provides for encoding of scientific models formally as a~set of equations and
informally in terms of processes expressing those equations. The model revision
works as follows. For input, it is required an initial model; a~set of constraints
representing acceptable changes to the initial model in terms of processes; a~set of
generic processes that may be added to the initial model;
and observations to which the
revised model should fit. These data provide the approach with a~heuristics that guides
search toward parts of the model space that are consistent with the observations. The
algorithm generates a~set of revised models that are sorted by their distance from the
initial model and presented with their mean squared error on the training data. The
distance between a~revised model and the initial model is defined as the number of
processes that are present in one but not in the other. The abilities of the approach
have been successfully checked in several environmental domains.

  \begin{figure*}[b] %fig7
  \vspace*{1pt}
 \begin{center}
 \mbox{%
 \epsfxsize=150.042mm
 \epsfbox{kal-7.eps}
 }
 \end{center}
 \vspace*{-9pt}
  \Caption{Hypothesis-driven closed-loop learning}
  \label{f7-kl}
  \end{figure*}

  Formalisms for hypothesis formation are mostly monotonic and are considered to
be not quite suitable for knowledge representation, especially in dealing with
incomplete knowledge, which is often the case with respect to biochemical networks.
In~\cite{52-kl}, knowledge-based framework for the general problem of hypothesis
formation is presented. The framework has been implemented by extending
BioSigNet-RR~--- a~ knowledge-based system that supports elaboration tolerant
representation and nonmonotonic reasoning. The main features of the extended
system provide:
\begin{enumerate}[(1)]
\item seamless integration of hypothesis formation with knowledge
representation and reasoning;
\item use of various resources of biological data as well
as human expertise to intelligently generate hypotheses; and
\item support for ranking
hypotheses and for designing experiments to verify hypotheses.
\end{enumerate}
 The extended system
is positioned as a~prototype of an intelligent research assistant of molecular
biologists.

\subsection{Hypothesis-driven robots}

\noindent
  The Robot Scientist~\cite{53-kl} oriented on genomic applications is a~physically
implemented system which is capable of running cycles of scientific experimentation
and discovery in a~fully automatic manner: hypothesis formation, experiment
selection to test these hypotheses, experiment execution using robotic system, results
analysis and interpretation, repeating the cycle  (closed-loop in which the results
obtained are used for learning from them and feeding the resulting knowledge back
into the experimental models). Deduction, induction, and abduction are the types of
logical reasoning used in scientific discovery (see section~3). The full automation of
science requires `closed-loop learning,' where the computer not only analyses the
results, but learns from them and feeds the resulting knowledge back into the next
cycle of the process (Fig.~\ref{f7-kl}).

  In the Robot Scientist, the automated formation of hypotheses is based on the
following key components:
  \begin{enumerate}[(1)]
\item machine-computable representation of the domain knowledge;
\item abductive or inductive inference of novel hypotheses;
\item an algorithm for the selection of hypotheses; and
\item deduction of the experimental consequences of hypotheses.
\end{enumerate}

  Adam, the first Robot Scientist prototype, was designed to carry out microbial
growth experiments to study functional genomics in the yeast \textit{Saccharomyces
cerevisiae}, specifically to identify the genes encoding `locally orphan enzymes.'
Adam uses a~comprehensive logical model of yeast metabolism, coupled with a~bioinformatic database (Kyoto Encyclopaedia of Genes and Genomes~--- KEGG)
and standard bioinformatics homology search techniques (PSI-BLAST and FASTA)
to hypothesize likely candidate genes that may encode the locally orphan enzymes.
This hypothesis generation process is abductive.

  To formalize Adam's functional genomics experiments, the LABORS ontology
(LABoratory Ontology for Robot Scientists) has been developed. LABORS is a~version of the ontology EXPO (as an upper layer ontology) customized for Robot
scientists to describe biological knowledge. LABORS is expressed in OWL-DL.
LABORS defines various structural research units, e.\,g., trial, study, cycle of study
and replicate as well as design strategy, plate layout, expected actual results. The
respective concepts and relations in the functional genomics data and metadata are
also defined. Both LABORS and the corresponding database (used for storing the
instances of the classes) are translated into Datalog in order to use the SWI-Prolog
reasoner for required applications~\cite{48-kl}.

  There were two types of hypotheses generated. The first level links an orphan
enzyme, represented by its enzyme class (E.C.)\ number, to a~gene (ORF) that
potentially encodes it. This relation is expressed as a~two-place predicate where the
first argument is the ORF and the second is the E.C.\ number. An example of
hypothesis at this level is: \textit{encodesORFtoEC(`YBR166C', `1.1.1.25')}.

  The second level of hypothesis involves the association between a~specific strain,
referenced via the name of its missing ORF, and a~chemical compound which should
affect the growth of the strain, if added as a~nutrient to its environment. This level of
hypothesis is derived from the first by logical inference using a~specific model of
yeast metabolism. An example of such a~hypothesis is: \textit{affects
growth(`C00108',`YBR166C')}, where the first argument is the compound (names
according to KEGG) and the second argument is the  strain considered.

  Adam then designs the experimental assays required to test these hypotheses for
execution on the laboratory robotic system. These experiments are based on a~two-
factor design that compares multiple replicates of the strains with and without
metabolites compared against wild type strain controls with and without metabolites.



  Adam follows a~hypothetico-deductive methodology (see section~2). Adam
abductively hypothesizes new facts about yeast functional biology, then it deduces
the experimental consequences of these facts using its model of metabolism, which it
then experimentally tests. To select experiments, Adam takes into account the variable
cost of experiments, and the different probabilities of hypotheses. Adam chooses its
experiments to minimize the expected cost of eliminating all but one hypothesis.
This
is, in general, an~NP complete problem and Adam uses heuristics to find a~solution~\cite{45-kl}.

  It is now likely that the majority of hypotheses in biology are computer-generated.
Computers are increasingly automating the process of hypothesis formation, for
example: machine learning programs (based on induction) are used in chemistry to
help design drugs; and in biology, genome annotation is essentially a~vast process of
(abductive) hypothesis formation. Such computer-generated hypotheses have been
necessarily expressed in a~computationally amenable way, but it is still not common
practice to deposit them into a~public database and make them available for
processing by other applications~\cite{45-kl}.

  The details describing the software and informatics decisions in the Robot Scientist
project can be found in~\cite{45-kl, 53-kl} and online at the website {\sf
http://www.aber.ac. uk/compsci/Research/bio/robotsci/data/informatics/}. The details
for developing the formalization used for Adam's functional genomics investigations
can be found in~\cite{48-kl, 54-kl}. An ontology-based formalization based on graph
theory and logical modeling makes it possible to keep an accurate track of all the
result units used for different goals, while preserving the semantics of all the
experimental entities involved in all the investigations. It is shown how
experimentation and machine learning are used to identify additional knowledge to
improve the metabolic model~\cite{54-kl}.

\subsection{Hypotheses as~data in~probabilistic databases}

\noindent
  Another view of hypotheses encoding and management is presented
  in~\cite{55-kl}. Authors use probabilistic database techniques for hypotheses
systematic construction and management. MayBMS~\cite{56-kl}, a~probabilistic
database management system, is used as a~core for hypothesis management. This
methodology (called $\gamma$-DB) enables researchers to maintain several
hypotheses explaining some phenomena and provides evaluation mechanism based
on Bayesian approach to rank them.

  The construction of $\gamma$-DB database comprises several steps. In the first
step, phenomenon and hypothesis entities are provided as input to the system.
Hypothesis is a~set of mathematical equations expressed as functions in W3C
MathML-based format and is associated with one or more simulation trial dataset,
consisting of tuples with input variables of equation and its corresponding output as
functionally dependent variables (the predictions). Phenomenon is represented
by at least one empirical dataset similar to simulation trials. In the next step, the
system deals with hypotheses  and phenomena in the following way:
\begin{enumerate}[(1)]
\item researcher
has to provide some metadata about hypotheses and phenomena; e.\,g., hypotheses
need to be associated with the respective phenomena and assigned a~prior confidence
distribution (uniform by default according to the principle of maximum
entropy~(see paragraph~3.2.3));
\item functional dependencies (FD) are extracted from equations in
order to obtain database schema to store simulations and experimental data; it should
be mentioned that to precisely identify hypothesis formulation, the special attributes
for phenomena and hypothesis references are introduced into FD;
\item tuples are
synthesized from simulation trials and observational data by uncertain
  pseudotransitive closure and  reasoning; and finally,
  \item  the probabilistic
  $\gamma$-DB database is formed.
  \end{enumerate}

   Once phenomenon and hypothesis (with
empirical datasets and simulation trials) are produced, it becomes possible to
manipulate them with database tools.

  MayBMS provides tools to evaluate competing hypotheses for the explanation of a~single phenomenon. With prior probabilities already provided,
  the system allows to
make one or more (if new observational data appears) Bayesian inference steps. In
each step, the prior probability is updated to posterior according to Bayes' theorem.
As a~result, hypotheses which better explain phenomenon get higher probabilities
enabling researchers to make more confident decisions (see also
paragraph~3.2.1). The
  $\gamma$-DB approach provides a~promising way to analyze hypotheses in
  large-scale DIR as uncertain predictive database in face of empirical data.

  \vspace*{-6pt}

\section{Examples of~Hypothesis-Driven Scientific Research}

\vspace*{-3pt}

\subsection{Hypotheses in~Besan{\!\!\fontsize{14pt}{10pt}\selectfont\ptb{\c{c}}}on Galaxy model}

\vspace*{-1pt}

  \noindent
  Various models in astronomy heavily rely on hypotheses. One of the most
impressive is the Besan{\!\fontsize{12pt}{10pt}\selectfont\ptb{\c{c}}}on galaxy model (BGM)~[57--59]
evolving for many years and representing the population and structure synthesis
model for the Milky Way. It allows astronomers to test hypotheses on the star
formation history, star evolution, and chemical and dynamical evolution of the
Galaxy.  As the result of simulation process, one can get the following:
multidimensional histograms of intrinsic star properties or observable properties, a~catalog
of pseudoobservations, or the integrated luminosity in a~specified photometric
band~\cite{60-kl}. From the beginning, the aim of the BGM was not only to be able to
simulate reasonable star counts but further to test scenarios of Galactic evolution
from assumptions on the rate of star formation (SFR), initial mass function (IMF),
and stellar evolution.

  The model has explicit and implicit hypotheses associated with it. Explicit
hypotheses are usually some sets of equations, taken from the literature studies and
put as the ingredient of the model. Some of explicit hypotheses are passed as the
input of the model, e.\,g., star formation rate, initial mass function, evolutionary
tracks, chemical evolution, atmosphere models, density laws, interstellar extinction
model.

  The model has some implicit hypotheses as well. For example, it is assumed that
no star population comes from the outside of the Galaxy. There are several more
implicit hypotheses about disk formation and dark matter assumptions encoded inside
the model. It is usually much harder to get all the implicit hypotheses, since many of
them are not described in the articles and are difficult to pin from the code.

  BGM has not only the large number of explicit and implicit hypotheses, but also a~complex interrelations between them. So, some of the hypotheses are being
independent, e.\,g., IFM and SFR; so, it is possible to change them independently. On
the other side, some of the hypotheses are connected, e.\,g., the age distribution,
the density laws, and the potential are linked with the age--velocity dispersion via the
Boltzmann equation and need to be consistent. Such kind of dependencies make the
model hard to be tested and to keep it consistent while varying different parameters
during model fitting. Another example of interrelations of hypotheses is competing
hypotheses.

  BGM has changed drastically over the last 30~years. This has happened because of
the appearance of new data surveys, technologies, and methods of observation
development. As an example of such evolution, the model developed in~2014
compared to previous versions handles variations of the SFR, IMF, evolutionary
tracks, and atmosphere models. These hypotheses are passed as input parameters to
the model; so, the user can vary them.

The second improvement of the model is the
implementation of the stellar binarity, being an important change since binaries can
account for about 50\% of the total stellar content of the Milky Way. The authors of
the new version underline the importance of understanding interrelationships between
different hypotheses and need for model evolution tools~\cite{60-kl}: ``In practice, to build a~Galaxy from the fundamental building-blocks, we had to reconstruct the previous
model and apply important changes in the code arrangement. That required to
understand well the underlying relations between all mentioned
components.''

  It is planned further to focus on the renewed BGM~\cite{57-kl}, in which authors
draw their attention to the Galaxy thin disk treatment and use of Tycho-2 as a~testing
dataset. The parameters of BGM (such as IMF, SFR and evolutionary track sets)
explicitly and model ingredients implicitly can be treated as hypotheses. Model
ingredients include the treatment of binarity, the local stellar mass densities of thin
disk, extinction model,  age-metallicity and age-velocity relations, radial scale length,
the age of the Galaxy thin disc, different sets of the star atmosphere models, etc.

  Tycho-2 dataset  and $\chi^2$-type statistics test is used to test various versions of
these hypotheses in order to choose the most appropriate ones and update model to
better fit the provided data. The tests were made by comparing star counts and
  $(B-V)_T$ color distributions between data and simulations. Two different tests
were used to evaluate the adequacy of the stellar densities globally and to test the
shape of the color distribution. Other parameters to be tested are: star counts, radio
velocity, magnitudes, colors, proper motions, parallax, effective temperatures,
gravity, and metallicity. Authors use histograms, 2~goodness of fit (maximum
likelihood and $\chi^2$-test) and for velocity parameter, Kolmogorov--Smirnov
and Henderson--Darling tests.

  Due to the fact that some ingredients of the model are highly correlated (such as
the IMF, SFR, and the local mass density), the authors defined default models as a~combination of a~new set of ingredients that significantly improve the fit to Tycho
data.  So, 11~IMF functions, 2~SFR functions, 2~evolutionary track sets, 3~sets of
atmosphere models, 3~values for the age of the formation of the thin disk,
and 3~sets of
values of the thin disk local stellar volume mass density were tested.
As a~result of
testing, the two most appropriate IMS and SFR hypotheses were chosen.

%\pagebreak

  BGM authors have plans to incorporate other star surveys and test the model
against them. To do simulations directly comparable with data, the selected
magnitudes from the surveys need to be complete in terms of magnitude. Among
these surveys, there are the Geneva-Copenhagen survey, SDSS-II/III,
SEGUE/SEGUE2, APOGEE, RAVE, LAMOST, Gaia, Gaia-ESO, GALAH LSST,
WEAVE, 4MOST, and \mbox{MOONS} surveys~\cite{61-kl}.

\subsection{Hypothesis testing applying connectome data }

\noindent
  In the neuroscience community, the development of common paradigms for
interrogating the myriad functional systems in the brain remains to be the core
challenge. Building on the term ``\textit{connectome},'' coined to describe the
comprehensive map of neural connections in the human brain, the ``functional
connectome'' denotes the collective set of functional connections in the human brain
(its ``wiring diagram'')~\cite{62-kl}. More broadly, a~connectome would include the
mapping of all neural connections within an organism's nervous system. The
production and study of connectomes, known as \textit{connectomics}, may range in
scale from a~detailed map of the full set of neurons and synapses within part or all of
the nervous system of an organism to a~macroscale description~\cite{63-kl} of the
functional and structural connectivity between all cortical areas and subcortical
structures. The ultimate goal of connectomics is to map the human brain. In
functional magnetic resonance imaging (fMRI), associations are thought to represent
functional connectivity in the sense that the two regions of the brain participate
together in the achievement of some higher-order function, often in the context of
performing some task. fMRI has emerged as a~powerful tool used to interrogate a~multitude of functional circuits simultaneously. This has elicited the interest of
statisticians working in that area. At the level of basic measurements, neuroimaging
data can be considered to consist typically of a~set of signals (usually, time series) at
each of a~collection of pixels (in two dimensions) or voxels (in three dimensions).
Building from such data, various forms of higher-level data representations are
employed in neuroimaging. In recent years, a~substantial interest in network-based
representations has emerged in neuroimaging to use \textit{networks} to summarize
relational information in a~set of measurements, typically assumed to be reflective of
either functional or structural relationships between regions of interest in the brain.
With neuroimaging, now, a~standard tool in clinical neuroscience, quickly moving
towards a~time in which we will have available databases composed of large
collections of secondary data in the form of \textit{network-based data objects}, is
predictable.

  One of the most basic tasks of interest in the analysis of such data is the testing of
hypotheses in answer to questions such as ``Is there a~difference between the
networks of these two groups of subjects?'' Networks are not Euclidean objects and,
hence, classical methods of statistics do not directly apply. Network-based analogues
of classical tools for statistical estimation and hypothesis testing are
investigated in~\cite{64-kl, 65-kl}. Such research is motivated by the 1000 Functional
Connectomes Project (FCP) launched in 2010~\cite{62-kl}.
  The 1000 FCP~\cite{66-kl} constitutes the largest data set of its kind similarly to
large data sets in genetics. Other projects (such as the Human Connectome Project
(HCP)) are aimed to build a~network map of the human brain in healthy, living adults.
The total volume of data produced by the HCP will likely be multiple
petabytes~\cite{67-kl}. HCP informatics platform includes data management system
ConnectomeDB that is based on the XNAT (eXtensive Neuroimaging Archive Toolkit)
imaging informatics
  platform~\cite{68-kl}, a~widely used open source system for managing and sharing
imaging and related data.

  Now, HCP has information about more than 500~subjects including structural scans
(T1w and T2w), resting-state fMRI (rfMRI), task fMRI (tfMRI), and high angular
resolution diffusion imaging (dMRI). In addition, some resting-state MEG (rMEG)
and/or task MEG (tMEG) data are available.

  Data come in several formats: ``unprocessed'' raw data, ``minimally
preprocessed,'' and ``analysis'' datasets. Preprocessed datasets have spatial distortions
minimized and data have been aligned across modalities and across subjects using
appropriate volume-based and surface-based registration methods. HCP consortium
recommends to use the preprocessing dataset.

  Visualization, processing, and analysis of high-dimensional data such as images
often require some kind of preprocessing to reduce the dimensionality of the data
and find a~mapping from the original representation to a~low-dimensional vector
space. The assumption is that the original data resides in a~low-dimensional subspace
or manifold~\cite{69-kl}, embedded in the original space. This topic of research is
called dimensionality reduction, nonlinear dimensionality reduction, including
methods for parameterization of data using low-dimensional manifolds as models.
Within the neural information processing community, this has become known as
manifold learning. Methods for manifold learning are able to find nonlinear
manifold parameterizations of datapoints residing in high-dimensional spaces, very
much like Principal Component Analysis (PCA) is able to learn or identify the most
important linear subspace of a~set of data points (projecting data on a~$n$-dimensional
linear subspace which maximizes the variance of the data in the new space).
{\looseness=1

}

  In~\cite{64-kl}, necessary mathematical properties associated with a~certain notion
of a~`space' of networks used to interpret functional neuroimaging
  connectome-oriented data are established. Extension of the classical statistics tools
to network-based datasets, however, appeared to be highly nontrivial. The main
challenge in such an extension is due to the fact that networks are not Euclidean
objects (for which classical methods were developed)~--- rather, they are
combinatorial objects, defined through their sets of vertices and edges.
  In~\cite{64-kl}, it was shown that networks can be associated with certain natural
subsets of Euclidean space and demonstrated that through a~combination of tools
from geometry, probability on manifolds, and high-dimensional statistical analysis, it
is possible to develop a~principled and practical framework in analogy to classical
tools. In particular, an asymptotic framework for one- and two-sample hypothesis
testing has been developed. Key to this approach is the correspondence between an
undirected graph and its Laplacian, where the latter is defined as a~matrix (associating
with a~network). Graph Laplacian appeared to be particularly appropriate to be used
for such matrices. The space of graph Laplacians is used working in certain subsets of
Euclidian space which are some submanifolds of the standard Euclidian space.

  The 1000 FCP describes functional neuroimaging data from 1093~subjects,
located in 24~community-based centers. The mean age of the participants was
29~years, and all subjects were 18~years old or older. It is of interest to compare the
subject-specific networks of males and females in the 1000 FCP data set.
  In~\cite{64-kl},  for the 1000 FCP, database comparing networks with respect to
the sex of the subjects, over different age group, and over various collection sites is
considered. It is shown that it is necessary to compute the means in each subgroup of
networks. This was done by constructing the Euclidean mean of the Laplacians for
each group of subjects in different age groups. Such group-specific mean Laplacians
can then be interpreted as the mean functional connectivity in each group. Such
approach provides for building the hypothesis tests about the average of networks or
groups of networks to investigate the effect of sex differences on entire networks.

  For the 1000 FCP data set, it was tested using the two-sample test for Laplacians
whether sex differences were significant to influence patterns of brain connectivity.
The null hypothesis of no group differences was rejected with high probability.
Similarly for the three different age cohorts, the null hypothesis of no cohort
differences also was rejected with high probability.

  On such examples, it was shown~\cite{64-kl} that the proposed global test has
sufficient power to reject the null hypothesis in cases when mass-univariate approach
(considered to be the gold standard in fMRI research~\cite{70-kl}) fails to detect the
differences at the local level. According to the mass-univariate approach, statistical
analysis is performed iteratively on all voxels to identify brain regions whose fMRI
detected responses display significant statistical effects. Thus,
it was shown that a~framework for network-based statistical testing is more
statistically powerful than a~mass-univariate approach.

  It is expected that in the near future, there will be a~plethora of databases of
network-based objects in neuroscience motivating the development and extension of
various tools from classical statistics to global network data.


  In paper~\cite{71-kl} discussing the relationship between neuroimaging and
Big Data areas, it is analyzed how modern neuroimaging research represents a~multifactorial and broad ranging data challenge, involving the growing size of the
data being acquired; sociological and logistical sharing issues; infrastructural
challenges for multisite, multidatatype archiving; and the means by which to
explore and mine these data. As neuroimaging advances further, e.\,g., aging,
genetics, and age-related disease, new vision is needed to manage and process this
information while marshalling of these resources into novel results.
It is predicted
that on this way, ``big data'' can become ``big'' brain science.

  In~\cite{72-kl}, authors formulate a~hypothesis about the brain connectivity and
evaluate it against HCP data. They use the task fMRI data, there is specific data about
a well-validated task used to probe animate motion detection. The audience was
shown short videoclips (20~s) of objects (squares, circles, and triangles) either
interacting in some way (animate motion) or moving mechanically (inanimate
motion). Participants rated the video by selecting if there was any social interaction,
no interaction, or not sure for interaction. There were 2~sessions comprised
of~5~videoblocks.

  Hypothesis states that some regions of the brain (V5 and pSTS) are effectively
connected and impacted by animate motion.

  To test it, general linear models were used. The time series were modeled with
regressors All motion\,--\,No motion, Animate--Inanimate motion. Together with
regressors about head, tongue, and finger movement, these regressors were used to
build general linear model. A~group level ANOVA was performed to identify
significant regional effects for the All Motion contrast and a~contrast for
Animate--Inanimate motion. For effective connectivity discovery,
 Dynamic Causal Modeling
(DCM) technique was used. DCM tells about self-, forwards, and
backward connections between active brain regions during an experiment, enabling
to infer the way of brain regions impact each other mostly. As the result of DCM
modeling, 16~models were created and passed as the input to Bayesian Model
Selection procedure, which chose the winning model among them.
{\looseness=1

}

  The results show that there is a~connectivity between V5 and the pSTS brain
regions in both hemispheres, which was independent of the type of motion. Animate
motion stimulates the forward and backward connection between V5 and the pSTS in
both hemispheres.

\subsection{Climate in~Australia}

\noindent
  Another view on hypothesis representation and evaluation is presented
  in~\cite{73-kl}. Authors argue that as long as in DIR data relevant to some
hypotheses get continuously aggregated as time passes, hypotheses should be
represented as programs that are executed repeatedly, as new relevant amounts of
data get aggregated. Their method and techniques are illustrated by examining
hypotheses about temperature trends in Australia during the 20th century. The
hypothesis being tested comes from~\cite{74-kl}, stated that the temperature series is
not stationary and is integrated of order~1 (I(1)). Nonstationarity means that the
level of the time series is not stable in time and can show increasing and decreasing
trends; I(1) means that by differentiating the stochastic process,
a~stationary process
(main statistical properties of the series remain unchanged) is obtained. Phillips--Perron test and the Kwiatkowski--Phillips--Schmidt--Shin (KPSS) test are used and
both of them are executed in~R. Several data sources are crawled: ($i$)~The National
Oceanographic and Atmospheric Administration marine and weather information; and
($ii$)~Australian Bureau of Meteorology dataset. The framework consists
of~R~interpreter and R~\textit{SPARQL}, \textit{tseries} packages. Authors also
used \mbox{agINFRA} for computation and rich semantics to support traditional scientific
workflows for natural sciences. Authors received further evidence on different
independent dataset that time series is integrated of order~1.

\subsection{Financial market}

\noindent
  Efficient-market hypothesis (EMH) is one of the most prominent in finance and
``\textit{asserts that financial markets are ``informationally efficient}.''
  In~\cite{75-kl}, authors test the weak form of EMH, stating that prices on traded
assets (e.\,g., stocks, bonds, or property) already reflect all past publicly available
information. The null hypothesis states that successive prices changes are
independent (random walk). The alternative hypothesis states that they are dependent.
To check if the successive closing prices are dependent of each other, the following
statistical tests were used: a~serial correlation test, a~runs test, an augmented
Dickey--Fuller test, and the multiple variance ratio test. Tests were performed on daily closing
prices from the six European stock markets (France, Germany, U.K., Greece,
Portugal, and Spain) during the period between~1993 and~2007. The result of each
test states whether successive closing prices are dependent of each other.

     Test provides evidence that for monthly prices and returns, the null hypothesis
should not be rejected for all six markets. If daily prices are concerned, the null
hypothesis is not rejected for France, Germany, U.K., and Spain, but this hypothesis is
rejected for Greece and Portugal. However, on the 2003--2007 dataset, the null
hypothesis for these two countries is not rejected as well.

  In \cite{76-kl}, Bollen \textit{et al.}\ use different approach to test EMH. Authors
investigate whether public sentiment, as expressed in large-scale collections of daily
Twitter posts, can be used to predict the stock market. They build public mood time
series by sentiment analysis of tweets from February~28 to December~19,
2008 and try to show that it can predict Dow Jones Index corresponding values. The
null hypothesis states that the mood time series do not predict DJIA
(Dow Jones Industrial Average) values. Granger
causality analysis in which Dow Jones values and mood time series are correlated is
used to test the null hypothesis. Granger causality analysis is used to determine if one
time series can predict another time series. Its results reject the null hypothesis and
claim that public opinion is predictive of changes in DJIA closing values.

\subsection{Publication-based automated hypothesis generation in~life sciences}

  \noindent
  Researchers and scientists from leading academic, pharmaceutical, and other
research centers have begun deploying  IBM's  Watson Discovery Advisor to rapidly
analyze and test hypotheses using data in millions of scientific papers available in
public databases. A~new scientific research paper is published nearly every
30~s, which equals more than a~million annually. According to the National
Institutes of Health, a~typical researcher reads about 23~scientific papers per month,
which translates to nearly~300~per~year, making it humanly impossible to keep up
with the evergrowing body of scientific material available. Building on Watson's
ability to understand nuances in natural language, Watson Discovery Advisor can
understand the language of science, such as how chemical compounds interact,
making it a~uniquely powerful tool for researchers in life sciences and other research
and industrial domains.  Specifically, the Watson Discovery Advisor for Life Sciences
is armed with expertise and understands field-specific lexicon in areas such as
clinical trial data, genomics, drugs, and human anatomy.

  Recently, scientists of Baylor College of Medicine and IBM using the Baylor
Knowledge Integration Toolkit (KnIT), based on Watson technology, identified new
enzymes (called kinases) that can modify p53, an important protein related to many
cancers~\cite{77-kl}. There are over 240,000~papers that mention one or more of
500+ known human kinases in their Medline abstract. There are over 70,000~papers
published on p53 to make their analysis manually is completely unrealistic task.
Watson analyzed the scientific articles related to p53 to predict proteins that turn on
or off p53's activity. This automated analysis led the Baylor cancer researchers to
identify six potential p53 kinases to target for new research. These results are notable,
considering that over the last 30~years, scientists averaged one p53 kinase discovery
per year. Knowing which proteins are modified by each kinase and, therefore, which
kinases would make good drug targets is a~difficult and unsolved problem. There are
over~500 known human kinases and tens of thousands of possible proteins they can
target.

  KnIT collects the abstracts to be mined applying queries. A~specific kinase name
and its synonyms are used in this process. Entity resolution process looks as follows.
The words and phrases that make up the document feature space are determined by
counting the number of documents in which each word appears and identifying the
words with the highest counts.  A~phrase is considered to be a~sequence of two
words. Only the~$N$~most frequent words and phrases are selected. This becomes the
feature space.

  Once a~feature space is received, a~representation of each kinase by averaging the
feature vectors of all documents that contain the kinase is created. This is the kinase
centroid. Next, a~distance matrix is calculated that measures the distance between each
kinase and every other kinase in the space.

  Finally, a~meaningful picture of kinase--kinase relationships is obtained.  Thus, it is
possible to identify a~set of kinases that may modify p53. However, some sort of
principled ranking scheme is needed in order to prioritize the kinases for further
experimentation. To provide such a~scheme, the graph diffusion method~\cite{78-kl}
was used. Graph diffusion is a~semisupervised learning approach for classification
based on labeled and unlabeled data. It takes known information (initial labels) and
then constrains the new labels to be smooth in respect to a~defined structure (e.\,g.,
a~network). In the case considered, it is known which kinases can modify p53
(initial labels); one
would like to know which other proteins can modify p53 (final labels). The distance
matrix based on the literature gives the structure of the kinase network. The initial
labels are extracted from current knowledge found in review articles.

  To test the algorithm, it was first applied in a~retrospective analysis to show
whether recent annotations of new p53 kinases occurring after a~certain date (2003)
could be predicted from a~model that only took into account papers written before
that date, at a~time when these discoveries of p53 kinases were still unknown. Next,
it was asked whether some variations in the algorithm could improve p53 kinase
prediction as  its performance  was compared to the common approach used most
typically to identify functionally similar proteins in biology. Finally, the analysis was
expanded to a~larger set of proteins to test scalability.

  This research represents the first stage in the IBM--Baylor collaborative effort
and as such, it proves the principle that mining past literature is a~viable strategy for
predicting previously unknown biological events. It was shown that p53 kinases
predicted with the text mining methods are supported by laboratory findings. In the
future, it should be possible to make many other kinds of predictions on a~much
larger scale as the infrastructure and capabilities will be increased. In the future, it is
planned to focus on a~wider area of proteins and functions, building up
comprehensive networks of interactions and predicting where new connections ought
to exist based on everything else that is known. It is expected that  this will ultimately
accelerate the pace of cancer discoveries by an order of magnitude and allow
scientists to come to a~much more complete understanding of the mechanisms behind
this disease.

  Expanding KnIT to other areas of biology or the physical sciences is not
straightforward. For example, to generalize to more proteins and genes is a~big
problem. In subjects like physics, results tend to be presented using equations and
graphs rather than words. However, data-mining groups are working to retrieve
information from these, too.

\vspace*{-6pt}

\section{Concluding Remarks}



\noindent
  The objective of this study is to analyze, collect, and systematize information on the
role of hypotheses in the DIR process as well as on support of
hypothesis formation, evaluation, selection, and refinement in course of the natural
phenomena modeling and scientific experiments. The discussion is started with the
basic concepts defining the role of hypotheses in the formation of scientific
knowledge and organization of the scientific experiments. Based on such concepts,
the basic approaches for hypothesis formulation applying logical reasoning, various
methods for hypothesis modeling and testing (including classical statistics, Bayesian
hypothesis, and parameter estimation methods, hypothetico-deductive approaches)
are briefly introduced. Special attention is given to discussion of the data mining and
machine learning methods role in process of generation, selection, and evaluation of
hypotheses as well as the methods for motivation of new hypothesis formulation.
Facilities of informatics for support of hypothesis-driven experiments, considered in
the paper, are aimed at the conceptualization of scientific experiments, hypothesis
formulation, and browsing in various domains (including biology, biomedical
investigations, neuromedicine, and astronomy), automatic organization of
hypothesis-driven   experiments. Examples of scientific researches applying hypotheses
considered in the paper include modeling of population and structure synthesis of the
Galaxy, connectome-related hypothesis testing,  studying of temperature trends in
Australia, analysis of stock markets applying the EMH,
as well as algorithmic generation of hypotheses in the collaborative project based on
IBM Watson--Baylor Knowledge Integration Toolkit applying the NLP and
knowledge representation and reasoning technologies. An introduction into the state
of the art of the hypothesis-driven research presented in the paper opens a~way for
investigation of the generalized approaches for efficient organization of
hypothesis-driven experiments applicable for various branches of DIR.

\renewcommand{\bibname}{\protect\rmfamily References}

\vspace*{-6pt}

{\small\frenchspacing
{%\baselineskip=10.8pt
\begin{thebibliography}{99}



\bibitem{1-kl}
Hey, T., S. Tansley, and K. Tolle, eds. 2009. \textit{The Fourth paradigm:
Data-intensive scientific discovery}. Redmond, Microsoft Research. 252~p.
\bibitem{2-kl}
\Aue{McComas, W.\,F.} 1998. The principal elements of the nature of science:
Dispelling the myths of science. \textit{Nature of science in science education:
Rationales and strategies}. Ed.\ W.\,F.~McComas.
Kluwer Academic Publs. 53--70.
\bibitem{3-kl}
\Aue{Lakshmana Rao, J.\,R.} 1998. Scientific `Laws,' `Hypotheses' and `Theories'.
\textit{Meanings Distinctions Reson.} 3:69--74.
\bibitem{4-kl}
\Aue{Poincar$\acute{\mbox{e}}$, H.} 2012. The foundations of science: Science and
hypothesis, the value of science, science and method. \textit{The Project
Gutenberg EBook}.  No.\,39713. 554~p. Available at:
{\sf http://www.gutenberg.org/files/39713/39713-8.txt}
(accessed February~10, 2015).
\bibitem{5-kl}
\Aue{Bacon, F.}
1952. {The new organon}. \textit{Great
books of the Western World. Vol.~30. The works of Francis Bacon}.
Ed. R.\,M.~Hutchins. Chicago: Encyclopedia
Britannica, Inc. 107--195.
\bibitem{6-kl}
\Aue{Menzies, T.} 1996. Applications of abduction: Knowledge-level modeling.
\textit{Int. J.~Hum.-Comput. St.} 45(3):305--335.
\bibitem{7-kl}
\Aue{Haber, J.} 2010. Research questions, hypotheses, and clinical questions.
\textit{Evolve resources for nursing research}. 7th ed. Elsevier. 27--55.
\bibitem{8-kl}
\Aue{Popper, K.} 2005. \textit{The logic of scientific discovery}.
  London\,--\,New York: Routledge,  Taylor \& Francis. 545~p.
  Available at: {\sf http://strangebeautiful.com/other-texts/popper-logic-scientific-discovery.pdf}
  (accessed February~10, 2015).
\bibitem{9-kl}
\Aue{Kerlinger, F.\,N., and H.\,B.~Lee}. 1964. \textit{Foundations of behavioral
research: Educational and psychological inquiry}. New York: Holt, Rinehart and
Winston. 739~p.
\bibitem{10-kl}
\Aue{Hempel, C.\,G.} 1952. Fundamentals of concept formation in empirical science.
\textit{Int. Encyclopedia Unified Sci.} 2(7). Available at: {\sf
http://www.iep.utm.edu/hempel/} (accessed February~10, 2015).

\bibitem{12-kl} %11
\Aue{Porto, F., and S.~Spaccapietra}. 2011. Data model for scientific models and
hypotheses. \textit{Evolution Conceptual Modeling} 6520:285--305.

\bibitem{11-kl} %12
\Aue{Gon{\!\,\fontsize{10pt}{10pt}\selectfont\ptb{\!\c{c}}}alves, B., and F.~Porto}. 2013. A~lattice-theoretic approach
for representing and managing hypothesis-driven research. \textit{25th Conference
(International) on Scientific and Statistical Database Management (ACM)
Proceedings}. Baltimore. 41.


\bibitem{13-kl}
\Aue{Gon{\!\fontsize{10pt}{10pt}\selectfont\ptb{\!\c{c}}}alves, B., F.~Porto, and A.\,M.\,C.~Moura}. 2012. On the
semantic engineering of scientific hypotheses as linked data. \textit{2nd Workshop
(International) on Linked Science Proceedings}. Boston.
\bibitem{14-kl}
\Aue{Woodward, J.} 2011. Scientific explanation. \textit{The Stanford Encyclopedia
of Philosophy}. Available at:  {\sf
http://plato. stanford.edu/archives/win2011/entries/scientific-explanation/} (accessed
February~10, 2015).
\bibitem{15-kl}
Nickles, T., ed. 1980. \textit{Scientific discovery: Case studies}. Taylor \& Francis.
501~p.
\bibitem{16-kl}
\Aue{Schickore, J.} 2014. Scientific discovery. \textit{The Stanford
Encyclopedia of Philosophy}. Available at: {\sf
http://plato. stanford.edu/archives/spr2014/entries/scientific-discovery/}
(accessed February~10, 2015).
\bibitem{17-kl}
\Aue{Kakas, A.\,C., R.\,A. Kowalski, and F.~Toni}. 1993. Abductive logic
programming. \textit{J.~Logic Comput.} 2(6):719--770.
\bibitem{18-kl}
\Aue{Kakas, A.\,C., A. Michael, and C.~Mourlas}. 2000. ACLP: Abductive constraint
logic programming. \textit{J.~Logic Program.} 44(1):129--177.
\bibitem{19-kl}
\Aue{Van Nuffelen, B., and A.~Kakas}. 2001. A-system: Declarative programming
with abduction. \textit{Logic programming and nonmotonic reasoning}.
Eds.\ T.~Eiter, W.~Faber, and M.~Truszczy$\acute{\mbox{n}}$ski.
Lecture notes in computer science ser. Berlin--Heidelberg:
Springer. 2173:393--397.
\bibitem{20-kl}
\Aue{Alferes, J.\,J., L.\,M. Pereira, and T.~Swift}. 2004. Abduction in well-founded
semantics and generalized stable models via tabled dual programs. \textit{Theor.
Pract. Log. Progr.} 4(4):383--428.
\bibitem{21-kl}
\Aue{Ray, O., and A.~Kakas.} 2006. ProLogICA: A~practical system for Abductive
Logic Programming. \textit{11th Workshop (International) on Non-Monotonic
Reasoning Proceedings}. 304--312.
\bibitem{22-kl}
\Aue{Citrigno, S., T. Eiter, W.~Faber, G.~Gottlob, C.~Koch, N.~Leone, and
F.~Scarcello}. 1997. The dlv system: Model generator and application frontends.
\textit{12th Workshop on Logic Programming Proceedings}.
128--137.
\bibitem{23-kl}
\Aue{King, R.\,D., M. Liakata, C.~Lu, S.\,G.~Oliver, and L.\,N.~Soldatova}. 2011.
On the formalization and reuse of scientific research. \textit{J.~Roy. Soc.
Interface} 8(63):1440--1448.
\bibitem{24-kl}
\Aue{Tamaddoni-Nezhad, A., R. Chaleil, A.~Kakas, and S.\,H.~Muggleton}. 2006.
Application of abductive ILP to learning metabolic network inhibition from temporal
data. \textit{Mach. Learn.} 64:209--230.
\bibitem{25-kl}
\Aue{Inoue K., T. Sato, M.~Ishihata, Y.~Kameya, and H.~Nabeshima}. 2009.
Evaluating abductive hypotheses using and EM algorithm on BDDs. \textit{21st Joint
Conference (International) on Artificial Intelligence (IJCAI09) Proceedings}.
Pasadena. 810--815.
\bibitem{26-kl}
\Aue{Bartha, P.} 2013. Analogy and analogical reasoning. \textit{The Stanford
Encyclopedia of Philosophy}. Available at:    {\sf
http://plato.stanford.edu/archives/fall2013/entries/ reasoning-analogy/} (accessed
February~10, 2015).
\bibitem{27-kl}
\Aue{Ivezi$\acute{\mbox{c}}$, {\ptb{\v{Z}}}., A.\,J.~Connolly, J.\,T.~VanderPlas,
and A.~Gray}. 2014. \textit{Statistics, data mining, and machine learning in
astronomy: A~practical Python guide for the analysis of survey data}. Princeton
University Press. 552~p.
\bibitem{28-kl}
\Aue{Sivia, D.\,S., and J.~Skilling}. 2006. \textit{Data analysis. A~Bayesian tutorial}.
New York: Oxford University Press Inc. 264~p.
\bibitem{29-kl}
\Aue{Field, A.} 2013. \textit{Discovering statistics using IBM SPSS statistics}.
4th ed. Sage.  915~p.
\bibitem{30-kl}
IBM SPSS Statistics for Windows, Version 22.0. 2013. Armonk, N.Y.: IBM Corp. IBM
SPSS Statistics base.  Available at: {\sf
https://www.uio.no/tjenester/it/forskning/\linebreak
statistikk/hjelp/programveilednigner/ibm\_spss\_\linebreak statistics\_brief\_guide-2.pdf}
(accessed February~10, 2015).
\bibitem{31-kl}
\Aue{Ihaka, R., and R.~Gentleman}. 1996. R: A~language for data analysis and
graphics. \textit{J.~Comput. Graph. Stat.} 5(3):299--314.
\bibitem{32-kl}
\Aue{March, M.\,C., G.\,D.~Starkman, R.~Trotta, and P.\,M.~Vaudrevange}. 2011.
Should we doubt the cosmological constant? \textit{Mon. Not. Roy.
Astron. Soc.} 410(4):2488--2496.
\bibitem{33-kl}
\Aue{Rouder, J.\,N., P.\,L.~Speckman, D.~Sun, R.\,D.~Morey, and G.~Iverson}.
2009. Bayesian t tests for accepting and rejecting the null hypothesis.
\textit{Psychon. Bull. Rev.} 16(2):225--237.
\bibitem{34-kl}
\Aue{Weber, M.} 2014. Experiment in biology. \textit{The Stanford Encyclopedia of
Philosophy}. Available at: {\sf
http://plato. stanford.edu/archives/fall2014/entries/biology-experiment/} (accessed
February~10, 2015).
\bibitem{35-kl}
\Aue{Hawthorne, J.} 2014. Inductive logic. \textit{The Stanford Encyclopedia of
Philosophy}. Available at: {\sf
http://plato. stanford.edu/archives/sum2014/entries/logic-inductive/} (accessed
February~10, 2015).
\bibitem{36-kl}
\Aue{Breiman, L.} 2001. Statistical modeling: The two cultures. \textit{Stat.
Sci.} 16(3):199--231.

\bibitem{38-kl} %37
\Aue{Hastie, T., R.~Tibshirani, J.~Friedman, and J.~Franklin}. 2005. The elements
of statistical learning: Data mining, inference and prediction. \textit{Math.
Intell.} 27(2):83--85.

\bibitem{37-kl} %38
\Aue{Barber, D.} 2010. \textit{Bayesian reasoning and machine learning}.
Cambridge University Press. 720~p.


\bibitem{39-kl}
\Aue{Ferrucci, D., E. Brown, J.~Chu-Carroll, J.~Fan, D.~Gondek,
A.\,A.~Kalyanpur, and C.~Welty}. 2010. Building Watson: An overview of the
DeepQA project. \textit{AI Mag.} 31(3):59--79.
\bibitem{40-kl}
\Aue{Dredze, M., K. Crammer, and F.~Pereira}. 2008. Confidence-weighted linear
classification. \textit{25th Conference (International) on Machine Learning
Proceedings}. Helsinki. 264--271.
\bibitem{41-kl}
\Aue{Starkman, G.\,D., R. Trotta, and P.\,M.~Vaudrevange}. 2008. Introducing doubt
in Bayesian model comparison. arXiv preprint arXiv:0811.2415.
\bibitem{42-kl}
\Aue{March, M.\,C.} 2013. Advanced statistical methods for astrophysical probes of
cosmology. {Springer Theses}. Vol.~20. 177~p.
\bibitem{43-kl}
\Aue{Porto, F.} 2013. Big data in astronomy. The LIneA-DEXL case.
\textit{EMC Summer School on BIG DATA~--- NCE/UFRJ}.
Available at: {\sf
http://www.slideshare.net/ fabiomporto/emc-2013-big-data-in-astronomy} (accessed
February~10, 2015).
\bibitem{44-kl}
\Aue{Racunas, S.\,A., N.\,H.~Shah, I.~Albert, and N.\,V.~Fedoroff}. 2004. Hybrow:
A~prototype system for computer-aided hypothesis evaluation.
\textit{Bioinformatics} 20(1):257--264.
\bibitem{45-kl}
\Aue{Soldatova, L.\,N., A.~Rzhetsky, and R.\,D.~King}. 2011. Representation of
research hypotheses. \textit{J.~Biomed. Semantics} 2(S-2):S9.
\bibitem{46-kl}
\Aue{Callahan, A., M.~Duumontier, and N.~Shah}. 2011. HyQue: Evaluating
hypotheses using Semantic Web technologies. \textit{J.~Biomed. Semantics}
 2(S-2):S3.
\bibitem{47-kl}
\Aue{Gao, Y., J.~Kinoshita, E.~Wu, E.~Miller, R.~Lee, A.~Seaborne, and T.~Clark}.
2006. SWAN: A~distributed knowledge infrastructure for Alzheimer disease
research. \textit{J.~Web Semant.} 4(3):222--228.
\bibitem{48-kl}
\Aue{King, R.\,D., K.\,E. Whelan, F.\,M.~Jones, P.\,G.~Reiser, C.\,H.~Bryant,
S.\,H.~Muggleton, and S.\,G.~Oliver}. 2004. Functional genomic hypothesis
generation and experimentation by a~robot scientist. \textit{Nature}
427(6971):247--252.
\bibitem{49-kl}
\Aue{Porto, F., A.\,M.\,C.~Moura, B.~Gon{\!\fontsize{10pt}{10pt}\selectfont\ptb{\!\c{c}}}alves, R.~Costa, and
S.\,A.~Spaccapietra}. 2012. A~scientific hypothesis conceptual model.
\textit{Advances in conceptual modeling.}
Eds. S.~Castano, P.~Vassiliadis, L.\,V.~Lakshmanan, and M.~Li~Lee.
Lecture notes in computer science ser. Berlin--Heidelberg: Springer.
 7518:101--110.
\bibitem{50-kl}
\Aue{Porto, F., and A.\,M.\,C.~Moura}. 2011. Scientific hypothesis database. Report.
Available at: {\sf
http://livroaberto. ibict.br/bitstream/1/869/1/Scientific\%20Hypothesis\%\linebreak 20Database.pdf} (accessed February~10, 2015).
\bibitem{51-kl}
\Aue{Asgharbeygi, N., P.~Langley, S.~Bay, and K.~Arrigo}. 2006. Inductive revision
of quantitative process models. \textit{Ecol. Model.} 194(1):70--79.
\bibitem{52-kl}
\Aue{Tran, N., C. Baral, V.\,J.~Nagaraj, and L.~Joshi}. 2005. Knowledge-based
integrative framework for hypothesis formation in biochemical networks.
\textit{Data integration in the life sciences}.
Eds. B.~Lud$\ddot{\mbox{a}}$scher and L.~Raschid.
Lecture notes in computer science ser. Berlin--Heidelberg: Springer.
3615:121--136.
\bibitem{53-kl}
\Aue{Sparkes, A., W. Aubrey, E.~Byrne, A.~Clare, M.\,N.~Khan, M.~Liakata, and
R.\,D.~King}. 2010. Towards Robot Scientists for autonomous scientific discovery.
\textit{Autom. Exp.} 2(1). Available at: {\sf
http://www.aejournal.net/content/2/1/1} (accessed February~10, 2015).
\bibitem{54-kl}
Castrillo, J.\,I., and S.\,G.~Oliver, eds. 2011. \textit{Yeast systems biology: Methods and
protocols}. {Methods in molecular biology ser}. Berlin--Heidelberg:
Springer. Vol.~759. 549~p.
\bibitem{55-kl}
\Aue{Plotkin, G.\,D.} 1970. A~note on inductive generalization. \textit{Mach.
Intell.} 5:153--163.
\bibitem{56-kl}
\Aue{Huang, J., L. Antova, C.~Koch, and D.~Olteanu}. 2009. MayBMS: A~probabilistic database management system. \textit{2009 ACM SIGMOD Conference
(International) on Management of Data Proceedings}. Rhode Island. 1071--1074.

\bibitem{59-kl} %57
\Aue{Robin, A., and M.~Cr$\acute{\mbox{e}}$z$\acute{\mbox{e}}$}.
1986. Stellar
populations in the Milky Way~--- a~synthetic model. \textit{Astron.
Astrophys.} 157:71--90.

\bibitem{58-kl}
\Au{Robin, A.\,C., C.~Reyl$\acute{\mbox{e}}$~C., S.~Derri{\!\!\ptb{\`{e}}}re,
and S.~Picaud.} 2006. A~synthetic view on structure and evolution of the Milky Way.
{arXiv preprint astro-ph}/0401052.

 \bibitem{57-kl} %59
\Aue{Czekaj, M.\,A., A.\,C.~Robin, F.~Figueras, X.~Luri, and M.~Haywood}. 2014.
The Besan{$\negthickspace$\!\,\fontsize{10pt}{10pt}\selectfont\ptb{\c{c}}}on Galaxy model \mbox{renewed-I}. Constraints on the local star formation
history from Tycho data. \textit{Astron. Astrophys.} 564:A102.


\bibitem{60-kl}
\Aue{Czekaj, M.\,A.} 2012. Galaxy evolution: A~new version of the
Besan{\fontsize{10pt}{10pt}\selectfont\ptb{\!\c{c}}}on
Galaxy Model constrained with Tycho data. PhD Thesis. Barcelona: Universitet de
Barcelona. 167~p.
\bibitem{61-kl}
\Aue{Martins, A.\,M.\,M.} 2014. Statistical analysis  of large scale surveys for
constraining the Galaxy evolution. PhD Thesis. Barcelona: Universitet de Barcelona.
221~p.
\bibitem{62-kl}
\Aue{Biswal, B.\,B., M.~Mennes, X.\,N.~Zuo, S.~Gohel, C.~Kelly,
S.\,M.~Smith, and C.~Windischberger}. 2010. Toward discovery science of
human brain function. \textit{Proc. Nat. Acad. Sci. USA}
107(10):4734--4739.
\bibitem{63-kl}
\Aue{Craddock, R.\,C., S.~Jbabdi, C.\,G.~Yan, J.\,T.~Vogelstein, F.\,X.~Castellanos,
A.~Di~Martino, and M.\,P.~Milham}. 2013. Imaging human connectomes at the
macroscale. \textit{Nat. Methods} 10(6):524--539.
\bibitem{64-kl}
\Aue{Ginestet, C.\,E., P.~Balanchandran, S.~Rosenberg, and E.\,D.~Kolaczyk}.
2014. Hypothesis testing for network data in functional neuroimaging. arXiv
preprint \mbox{arXiv}:1407.5525.
\bibitem{65-kl}
\Aue{Ginestet, C.\,E., A.\,P.~Fournel, and A.~Simmons}. 2014. Statistical network
analysis for functional MRI: Summary networks and group comparisons.
\textit{Front. Comput. Neurosci.} 8:51. Available at: {\sf
http://www.ncbi.nlm. nih.gov/pmc/articles/PMC4018548/} (accessed February~10,
2015).
\bibitem{66-kl}
\Aue{Yan, C.\,G., R.\,C.~Craddock, X.\,N.~Zuo, Y.\,F.~Zang, and M.\,P.~Milham}.
2013. Standardizing the intrinsic bra towards robust measurement of inter-individual
variation in 1000 functional connectomes. \textit{Neuroimage} 80:246--262.
\bibitem{67-kl}
\Aue{Marcus, D.\,S., J. Harwel, T.~Olsen, M.~Hodge, M.\,F.~Glasser, F.~Prior, and
D.\,C.~Van~Essen}. 2011. Informatics and data mining tools and strategies for the
human connectome project. \textit{Front. Neuroinform.} 5. Available at:
{\sf http://www.ncbi.nlm.nih.gov/pmc/\linebreak articles/PMC3127103/} (accessed
February~10, 2015).

\columnbreak

\bibitem{68-kl}
\Aue{Marcus, D.\,S., T.\,R. Olsen, M.~Ramaratnam, and R.\,L.~Buckner}. 2007. The
extensible neuroimaging archive toolkit. \textit{Neuroinformatics} 5(1):11--33.
\bibitem{69-kl}
\Aue{Brun, A.} 2006. {Manifold learning and representations for image analysis
and visualization}. Department of Biomedical Engineering,
Link$\ddot{\mbox{o}}$pings Universitet. 104~p.
\bibitem{70-kl}
\Aue{Mahmoudi, A., S.~Takerkart, F.~Regragui, D.~Boussaoud, and A.~Brovelli}.
2012. Multivoxel pattern analysis for fMRI data: A~review. \textit{Comput.
Math. Methods Med}. Available at: {\sf
http://www.hindawi.com/journals/ cmmm/2012/961257/} (accessed February~10,
2015).
\bibitem{71-kl}
\Aue{Van Horn, J.\,D., and A.\,W.~Toga}. 2014. Human neuroimaging as a~``Big
Data'' science. \textit{Brain Imaging Behavior} 8(2):323--331.
\bibitem{72-kl}
\Aue{Hillebrandt, H., K.\,J. Friston, and S.\,J.~Blakemore}. 2014. Effective
connectivity during animacy perception-dynamic causal modelling of Human
Connectome Project data. \textit{Sci. Rep.} 4. Available at: {\sf
http://www.ncbi.nlm. nih.gov/pmc/articles/PMC4150124/} (accessed February~10,
2016).
\bibitem{73-kl}
\Aue{Lappalainen, J., M.\,A. Sicilia, and B.~Hern$\acute{\mbox{a}}$ndez}. 2013.
Automatic hypothesis checking using eScience Research Infrastructures, ontologies,
and linked data: A~case study in climate change research. \textit{Procedia Comput.
Sci.} 18:1172--1178.
\bibitem{74-kl}
\Aue{Lenten, L.\,J., and I.\,A.~Moosa}. 2003. An empirical investigation into
long-term climate change in Australia. \textit{Environ. Modell. Softw.}
18(1):59--70.
\bibitem{75-kl}
\Aue{Borges, M.\,R.} 2010. Efficient market hypothesis in European stock markets.
\textit{Eur. J.~Financ.} 16(7):711--726.
\bibitem{76-kl}
\Aue{Bollen, J., H.~Mao, and X.~Zeng}. 2011. Twitter mood predicts the stock
market. \textit{J.~Comput. Sci.} 2(1):1--8.
\bibitem{77-kl}
\Aue{Spangler, S., A.\,D.~Wilkins, B.\,J.~Bachman, \textit{et al}.} 2014. Automated
hypothesis generation based on mining scientific literature. \textit{KDD'14
Proceedings}. New York. 1877--1886.
\bibitem{78-kl}
\Aue{Zhou, D., O. Bousquet, T.\,N.~Lal, J.~Weston, and
B.~Sch$\ddot{\mbox{o}}$lkopf}. 2004. Learning with local and global consistency.
\textit{Adv. Neur. Inform. Proc. Syst.} 16(16):321--328.

\end{thebibliography} } }

\end{multicols}

\vspace*{-3pt}

\hfill{\small\textit{Received February 10, 2015}}

\vspace*{-36pt}

\Contr

\noindent
\textbf{Kalinichenko Leonid A.} (b.\ 1937)~--- Doctor of Science in physics and mathematics, professor;
Head of Laboratory, Institute of Informatics Problems, Russian Academy of Sciences; 44-2 Vavilov Str.,
Moscow 119333, Russian Federation; professor, Faculty of Computational Mathematics and
Cybernetics, M.\,V.~Lomonosov Moscow State University, 1-52 Leninskiye Gory, GSP-1, Moscow
119991, Russian Federation; leonidandk@gmail.com

\vspace*{3pt}

\noindent
\textbf{Kovalev Dmitry Yu.} (b.\ 1988)~--- junior scientist, Institute of Informatics Problems, Russian
Academy of Sciences, 44-2 Vavilov Str., Moscow 119333, Russian Federation;
dkovalev@ipiran.ru


\vspace*{3pt}

\noindent
\textbf{Kovaleva Dana A.} (b.\ 1973)~--- Candidate of Science (PhD)
in physics and mathematics, scientist, Institute of
Astronomy, Russian Academy of Sciences, 48 Pyatnitskaya Str., Moscow 119017, Russian Federation;
dana@inasan.ru

\vspace*{3pt}

\noindent
\textbf{Malkov Oleg Yu.} (b.\ 1961)~---
Doctor of Science in physics and mathematics, associate professor; Head of Department, Institute of
Astronomy, Russian Academy of Sciences; 48 Pyatnitskaya Str., Moscow 119017, Russian Federation;
professor, Faculty of Physics, M.\,V.~Lomonosov Moscow State University, 1-52 Leninskiye Gory, GSP-1,
Moscow 119991, Russian Federation; malkov@inasan.ru

%\vspace*{8pt}

%\hrule

%\vspace*{2pt}

%\hrule

%\vspace*{-6pt}

\newpage

\vspace*{-18pt}


\def\tit{МЕТОДЫ И~СРЕДСТВА ПОДДЕРЖКИ ИССЛЕДОВАНИЙ, ДВИЖИМЫХ ГИПОТЕЗАМИ: ОБЗОР}

\def\aut{Л.\,А.~Калиниченко$^1$, Д.\,Ю.~Ковалев$^2$, Д.\,А.~Ковалева$^3$, О.\,Ю.~Малков$^4$}


\def\titkol{Методы и средства поддержки исследований, движимых гипотезами: Обзор}

\def\autkol{Л.\,А.~Калиниченко, Д.\,Ю.~Ковалев, Д.\,А.~Ковалева, О.\,Ю.~Малков}

%{\renewcommand{\thefootnote}{\fnsymbol{footnote}}
%\footnotetext[1]{Работа проводится при финансовой поддержке Программы
%стратегического развития Петрозаводского государственного университета в рамках
%на\-уч\-но-ис\-сле\-до\-ва\-тель\-ской деятельности.}}


\titel{\tit}{\aut}{\autkol}{\titkol}

\vspace*{-12pt}

\noindent
$^1$Институт проблем информатики Российской академии наук; leonidandk@gmail.com

\noindent
$^2$Институт проблем информатики Российской академии наук; dkovalev@ipiran.ru

\noindent
$^3$Институт астрономии Российской академии наук; dana@inasan.ru

\noindent
$^4$Институт астрономии Российской академии наук;  malkov@inasan.ru


\vspace*{6pt}

\def\leftfootline{\small{\textbf{\thepage}
\hfill ИНФОРМАТИКА И ЕЁ ПРИМЕНЕНИЯ\ \ \ том\ 9\ \ \ выпуск\ 1\ \ \ 2015}
}%
 \def\rightfootline{\small{ИНФОРМАТИКА И ЕЁ ПРИМЕНЕНИЯ\ \ \ том\ 9\ \ \ выпуск\ 1\ \ \ 2015
\hfill \textbf{\thepage}}}

\Abst{Исследования с интенсивным использованием данных (ИИИД), развиваемые в рамках новой
парадигмы изучения естественных явлений, именуемой Четвертой парадигмой, придают особое
значение все возрастающей роли, которую играют данные, полученные в результате наблюдений,
экспериментов или компьютерного моделирования,  практически во всех областях анализа и
накопления информации. Главной целью ИИИД является извлечение (вывод) знаний из данных.
Целью настоящей работы является обзор существующих подходов, методов и инфраструктур анализа
данных в ИИИД с акцентом на роли гипотез в процессе анализа информации и эффективной
поддержки формирования, оценки и выбора гипотез при моделировании естественных явлений и
проведении экспериментов. Статья включает введение в разнообразные понятия, методы и средства
эффективной организации  движимых гипотезами экспериментов в ИИИД.}

\KW{исследования с интенсивным использованием данных; Четвертая парадигма; гипотезы; модели;
теории; ги\-по\-те\-ти\-ко-де\-дук\-тив\-ный метод; проверка гипотез; решетка гипотез; модель
Галактики, анализ коннектома; автоматизированная генерация гипотез}

\DOI{10.14357/19922264150104}

%\vspace*{6pt}


 \begin{multicols}{2}

\renewcommand{\bibname}{\protect\rmfamily Литература}
%\renewcommand{\bibname}{\large\protect\rm References}

{\small\frenchspacing
{%\baselineskip=10.8pt
\begin{thebibliography}{99}
\bibitem{1-kl-1}
The Fourth paradigm: Data-intensive scientific discovery~/
Eds. T.~Hey, S.~Tansley, K.~Tolle.~--- Redmond, Microsoft Research, 2009.  252~p.
\bibitem{2-kl-1}
\Au{McComas W.\,F.} The principal elements of the nature of science: Dispelling the
myths of science~// Nature of science in science education: Rationales and
strategies~/ Ed. W.\,F.~McComas.~--- Kluwer Academic Publs., 1998. P.~53--70.
\bibitem{3-kl-1}
\Au{Lakshmana Rao J.\,R.} Scientific `Laws', `Hypotheses' and `Theories'~//
Meanings Distinctions Reson., 1998. Vol.~3. P.~69--74.
\bibitem{4-kl-1}
\Au{Poincar$\acute{\mbox{e}}$ H.} The foundations of science: Science and
hypothesis, the value of science,  science and method. The Project Gutenberg
EBook, 2012. No.~39713. P.~554.
{\sf http://www.gutenberg.org/files/39713/39713-8.txt}.
\bibitem{5-kl-1}
\Au{Bacon F.} The new organon~// Great books of the Western World. Vol.~30. The works of
Francis Bacon~/ Ed.\ R.\,M.~Hutchins.~--- Chicago, Encyclopedia Britannica, Inc.,
1952.  P.~107--195.
\bibitem{6-kl-1}
\Au{Menzies T.} Applications of abduction: Knowledge-level modeling~//
Int. J.~Hum.-Comput. St., 1996. Vol.~45. No.\,3. P.~305--335.
\bibitem{7-kl-1}
\Au{Haber J.} Research questions, hypotheses, and clinical questions~// Evolve
resources for nursing research.~--- 7th ed.~--- Elsevier, 2010. P.~27--55.
\bibitem{8-kl-1}
\Au{Popper K.} The logic of scientific discovery.~---
London\,--\,New York: Routledge, Taylor \& Francis,
2005. 545~p. {\sf http://strangebeautiful.com/other-texts/popper-logic-scientific-discovery.pdf}.
\bibitem{9-kl-1}
\Au{Kerlinger F.\,N.,  Lee H.\,B.} Foundations of behavioral research: Educational
and psychological inquiry.~--- New York: Holt, Rinehart and Winston, 1964. 739~p.
\bibitem{10-kl-1}
\Au{Hempel C.\,G.} Fundamentals of concept formation in empirical science~// Int.
Encyclopedia Unified Sci., 1952. Vol.~2. No.\,7.
{\sf
http://www.iep.utm.edu/hempel/}.


\bibitem{12-kl-1} %11
\Au{Porto F., Spaccapietra~S.} Data model for scientific models and hypotheses~//
Evolution Conceptual Modeling, 2011, Vol.~6520. P.~285--305.
\bibitem{11-kl-1} %12
\Au{Gon\/{\!\fontsize{10pt}{10pt}\selectfont\ptb{\!\!\c{c}}}alves~B., Porto~F.} A~lattice-theoretic approach for
representing and managing hypothesis-driven research~// AMW, 2013.

\bibitem{13-kl-1}
\Au{Gon\/{\!\fontsize{10pt}{10pt}\selectfont\ptb{\!\!\c{c}}}alves~B., Porto~F., Moura~A.\,M.\,C.} On the semantic
engineering of scientific hypotheses as linked data~// 2nd Workshop (International)
on Linked Science Proceedings, 2012.
\bibitem{14-kl-1}
\Au{Woodward J.} Scientific explanation~// The Stanford Encyclopedia of
Philosophy, 2011. {\sf http://plato.stanford.edu/ archives/win2011/entries/scientific-explanation/}.
\bibitem{15-kl-1}
Scientific discovery: Case studies~/
Ed. T.~Nickles.~--- Taylor \& Francis, 1980. Vol.~2. 501~p.
\bibitem{16-kl-1}
\Au{Schickore J.} Scientific discovery~// The Stanford Encyclopedia of Philosophy,
2014. {\sf http://plato.stanford.edu/ archives/spr2014/entries/scientific-discovery/}.
\bibitem{17-kl-1}
\Au{Kakas A.\,C., Kowalski R.\,A.,  Toni~F.} Abductive logic programming~//
J.~Logic Comput., 1993.  Vol.~2. No.\,6.  P.~719--770.
\bibitem{18-kl-1}
\Au{Kakas A.\,C., Michael~A.,  Mourlas~C.} ACLP: Abductive constraint logic
programming~// J.~Logic Program., 2000. Vol.~44. No.\,1. P.~129--177.
\bibitem{19-kl-1}
\Au{Van Nuffelen B., Kakas A.} A-system: Declarative programming with
abduction~// Logic programming and nonmotonic reasoning~/
Eds. T.~Eiter, W.~Faber, M.~Truszczy$\acute{\mbox{n}}$ski.~---
Lecture notes in computer science ser.~--- Berlin--Heidelberg:
Springer, 2001. Vol.~2173. P.~393--397.
\bibitem{20-kl-1}
\Au{Alferes J.\,J., Pereira L.\,M., Swift~T.} Abduction in well-founded semantics and
generalized stable models via tabled dual programs~// Theor. Pract. Log. Prog.,
2004. Vol.~4. No.\,4.
P.~383--428.
\bibitem{21-kl-1}
\Au{Ray O., Kakas A.} ProLogICA: A~practical system for Abductive Logic
Programming~// 11th Workshop (International) on Non-Monotonic Reasoning
Proceedings, 2006. P.~304--312.
\bibitem{22-kl-1}
\Au{Citrigno S., Eiter T., Faber~W., Gottlob~G., Koch~C., Leone~N., Scarcello~F.}
The dlv system: Model generator and application frontends~// 12th Workshop on
Logic Programming Proceedings, 1997.  P.~128--137.
\bibitem{23-kl-1}
\Au{King R.\,D., Liakata M., Lu~C., Oliver~S.\,G., Soldatova~L.\,N.} On the
formalization and reuse of scientific research~// J.~Roy. Soc. Interface,
2011. Vol.~8. No.\,63.  P.~1440--1448.
\bibitem{24-kl-1}
\Au{Tamaddoni-Nezhad A., Chaleil~R., Kakas~A., Muggleton~S.\,H.} Application of
abductive ILP to learning metabolic network inhibition from temporal data~//
Mach. Learn., 2006. Vol.~64.  P.~209--230.
\bibitem{25-kl-1}
\Au{Inoue K., Sato T., Ishihata M., Kameya~Y., Nabeshima~H.} Evaluating
abductive hypotheses using and EM algorithm on BDDs~// IJCAI-09 Proceedings,
2009. P.~810--815.
\bibitem{26-kl-1}
\Au{Bartha P.} Analogy and analogical reasoning~// The Stanford Encyclopedia of
Philosophy, 2013. {\sf
http://plato. stanford.edu/archives/fall2013/entries/reasoning-analogy/}.
\bibitem{27-kl-1}
\Au{Ivezi$\acute{\mbox{c}}$~{\ptb{\v{Z}}}., Connolly~A.\,J., VanderPlas~J.\,T.,
Gray~A.} Statistics, data mining, and machine learning in astronomy: A~practical
Python guide for the analysis of survey data.~--- Princeton University Press, 2014.
552~p.
\bibitem{28-kl-1}
\Au{Sivia D.\,S., Skilling~J.} Data analysis. A~Bayesian tutorial.~--- New
York: Oxford University Press Inc., 2006.  264~p.
\bibitem{29-kl-1}
\Au{Field A.} Discovering statistics using IBM SPSS statistics.~---
4th ed.~--- Sage, 2013.  915~p.
\bibitem{30-kl-1}
IBM SPSS Statistics for Windows, Version 22.0. Armonk, N.Y.: IBM Corp. IBM
SPSS Statistics base, 2013.
{\sf
https://www.uio.no/tjenester/it/forskning/statistikk/
hjelp/programveilednigner/ibm\_spss\_statistics\_brief\_ guide-2.pdf}.

\bibitem{31-kl-1}
\Au{Ihaka R., Gentleman R.} R:~A~language for data analysis and graphics~//
J.~Comput. Graph. Stat., 1996. Vol.~5. No.\,3. P.~299--314.
\bibitem{32-kl-1}
\Au{March M.\,C., Starkman G.,D., Trotta~R., Vaudrevange~P.\,M.} Should we
doubt the cosmological constant?~// Mon. Not. Roy. Astron. Soc.,
2011. Vol.~410. No.\,4. P.~2488--2496.
\bibitem{33-kl-1}
\Au{Rouder J.\,N., Speckman~P.\,L., Sun~D., Morey~R.\,D., Iverson~G.} Bayesian t
tests for accepting and rejecting the null hypothesis~// Psychon. Bull.
Rev., 2009.  Vol.~16. No.\,2. P.~225--237.
\bibitem{34-kl-1}
\Au{Weber M.} Experiment in biology~// The Stanford Encyclopedia of Philosophy,
2014. {\sf  http://plato.stanford.edu/ archives/fall2014/entries/biology-experiment/}.
\bibitem{35-kl-1}
\Au{Hawthorne J.} Inductive logic~// The Stanford Encyclopedia of Philosophy,
2014. {\sf http://plato.stanford.edu/ archives/sum2014/entries/logic-inductive/}.
\bibitem{36-kl-1}
\Au{Breiman L.} Statistical modeling: The two cultures~// Stat. Sci., 2001.
Vol.~16. No.\,3. P.~199--231.

\bibitem{38-kl-1} %37
\Au{Hastie T., Tibshirani R., Friedman~J.,  Franklin~J.} The elements of statistical
learning: Data mining, inference and prediction~// Math. Intell.,
2005. Vol.~27. No.\,2. P.~83--85.

\bibitem{37-kl-1} %38
\Au{Barber D.} Bayesian reasoning and machine learning.~--- Cambridge University
Press, 2010.  720~p.

\bibitem{39-kl-1}
\Au{Ferrucci D., Brown E., Chu-Carroll~J., Fan~J., Gondek~D., Kalyanpur~A.\,A.,
Welty~C.} Building Watson: An overview of the DeepQA project~// AI Mag.,
2010. Vol.~31. No.\,3. P.~59--79.
\bibitem{40-kl-1}
\Au{Dredze M, Crammer K., Pereira~F.} Confidence-weighted linear
classification~// 25th Conference (International) on Machine Learning Proceedings.
Helsinki, Finland, 2008.  P.~264--271.
\bibitem{41-kl-1}
\Au{Starkman G.\,D., Trotta~R., Vaudrevange~P.\,M.} Introducing doubt in Bayesian
model comparison. arXiv preprint arXiv:0811.2415, 2008.
\bibitem{42-kl-1}
\Au{March M.\,C.} Advanced statistical methods for astrophysical probes of
cosmology. Springer Theses, 2013.  Vol.~20. 177~p.
\bibitem{43-kl-1}
\Au{Porto F.} Big data in astronomy. The LIneA-DEXL case~// Presentation at the
EMC Summer School on BIG DATA~--- NCE/UFRJ, 2013.
{\sf
http://www.slideshare.net/ fabiomporto/emc-2013-big-data-in-astronomy}.
\bibitem{44-kl-1}
\Au{Racunas S.\,A., Shah~N.\,H., Albert~I., Fedoroff~N.\,V.} Hybrow: A~prototype
system for computer-aided hypothesis evaluation~// Bioinformatics, 2004. Vol.~20.
No.\,1. P.~257--264.
\bibitem{45-kl-1}
\Au{Soldatova L.\,N., Rzhetsky~A., King~R.\,D.} Representation of research
hypotheses~// J.~Biomed. Semantics, 2011. Vol.~2. No.\,S-2. P.~S9.
\bibitem{46-kl-1}
\Au{Callahan A., Duumontier~M., Shah~N.} HyQue: Evaluating hypotheses using
Semantic Web technologies~// J.~Biomed. Semantics, 2011. Vol.~2. No.\,S-2.
P.~S3.
\bibitem{47-kl-1}
\Au{Gao Y., Kinoshita J., Wu~E., Miller~E., Lee~R., Seaborne~A., Clark~T.}
SWAN: A~distributed knowledge infrastructure for Alzheimer disease
research~// J.~Web Semant., 2006. Vol.~4. No.\,3. P.~222--228.
\bibitem{48-kl-1}
\Au{King R.\,D., Whelan K.\,E., Jones~F.\,M., Reiser~P.\,G., Bryant~C.\,H.,
Muggleton~S.\,H., Oliver~S.\,G.} Functional genomic hypothesis generation and
experimentation by a~robot scientist~// Nature, 2004.  Vol.~427. No.\,6971.
P.~247--252.
\bibitem{49-kl-1}
\Au{Porto F., Moura~A.\,M.\,C., Gon\/{\!\fontsize{10pt}{10pt}\selectfont\ptb{\!\!\c{c}}}alves~B., Costa~R., Spaccapietra~S.\,A.}
A~scientific hypothesis conceptual model~//
{Advances in conceptual modeling}~/
Eds. S.~Castano, P.~Vassiliadis, L.\,V.~Lakshmanan, M.~Li~Lee.~---
Lecture notes in computer science ser.~--- Berlin--Heidelberg: Springer, 2012.
Vol.~7518. P.~101--110.
\bibitem{50-kl-1}
\Au{Porto F., Moura A.\,M.\,C.} Scientific hypothesis database. Report, 2011.
{\sf
http://livroaberto.ibict.br/bitstream/1/ 869/1/Scientific\%20Hypothesis\%20Database.pdf}.
\bibitem{51-kl-1}
\Au{Asgharbeygi N., Langley P., Bay~S., Arrigo~K.} Inductive revision of
quantitative process models~// Ecol. Model., 2006. Vol.~194. No.\,1.
P.~70--79.
\bibitem{52-kl-1}
\Au{Tran N., Baral C., Nagaraj~V.\,J., Joshi~L.} Knowledge-based integrative
framework for hypothesis formation in biochemical networks~// Data
integration in the life sciences~/
Eds. B.~Lud$\ddot{\mbox{a}}$scher, L.~Raschid.~---
Lecture notes in computer science ser. Berlin--Heidelberg: Springer,
2005. Vol.~3615. P.~121--136.

\bibitem{53-kl-1}
\Au{Sparkes A., Aubrey W., Byrne~E., Clare~A., Khan~M.\,N., Liakata~M.,
King~R.\,D.} Towards Robot Scientists for autonomous scientific discovery~//
Autom. Exp., 2010.  Vol.~2. No.\,1. {\sf
http://www.aejournal.net/content/2/1/1}.
\bibitem{54-kl-1}
Yeast systems biology: Methods and protocols~/
Eds. J.\,I.~Castrillo,  S.\,G.~Oliver.~---  Methods in molecular biology ser.~---
Berlin--Heidelberg: Springer, 2011. Vol.~759. 549~p.
\bibitem{55-kl-1}
\Au{Plotkin G.\,D.} A~note on inductive generalization~// Mach. Intell.,
1970.  Vol.~5. P.~153--163.
\bibitem{56-kl-1}
\Au{Huang J., Antova L., Koch~C., Olteanu~D.} MayBMS: A~probabilistic database
management system~// 2009 ACM SIGMOD Conference (International) on
Management of Data Proceedings, 2009. P.~1071--1074.

\bibitem{59-kl-1} %57
\Au{Robin A., Cr$\acute{\mbox{e}}$z$\acute{\mbox{e}}$~M.} Stellar populations in
the Milky Way~--- a~synthetic model~// Astron. Astrophys., 1986. Vol.~157.
P.~71--90.

\bibitem{58-kl-1}
\Au{Robin A.\,C., Reyl$\acute{\mbox{e}}$~C., Derri{\!\!\!\ptb{\`{e}}}re~S.,
Picaud~S.} A~synthetic view on structure and evolution of the Milky Way.
arXiv preprint astro-ph/0401052, 2004.

\bibitem{57-kl-1} %59
\Au{Czekaj M.\,A., Robin A.\,C., Figueras~F., Luri~X., Haywood~M.}
The Besan{\!\fontsize{10pt}{10pt}\selectfont\ptb{\!\c{c}}}on
Galaxy model renewed-I. Constraints on the local star formation history from Tycho
data~//Astron. Astrophys., 1986. Vol.~564. P.~A102.

\bibitem{60-kl-1}
\Au{Czekaj~M.\,A.} Galaxy evolution: A~new version of the
Besan{\!\fontsize{10pt}{10pt}\selectfont\ptb{\!\c{c}}}on Galaxy
Model constrained with Tycho data. PhD Thesis, 2012. Universitet de
Barcelona, Spain. 167~p.
\bibitem{61-kl-1}
\Au{Martins A.\,M.\,M.} Statistical analysis  of large scale surveys for constraining
the Galaxy evolution. PhD Thesis, 2014. Universitet de Barcelona, Spain.
221~p.
\bibitem{62-kl-1}
\Au{Biswal B.\,B., Mennes~M., Zuo~X.\,N., Gohel~S., Kelly~C., Smith~S.\,M.,
Windischberger~C.} Toward discovery science of human brain function~// Proc.
Nat. Acad. Sci. USA, 2010.  Vol.~107. No.\,10. P.~4734--4739.
\bibitem{63-kl-1}
\Au{Craddock R.\,C., Jbabdi~S., Yan~C.\,G., Vogelstein~J.\,T., Castellanos~F.\,X.,
Di~Martino~A., Milham~M.\,P.} Imaging human connectomes at the macroscale~//
Nat. Methods, 2013. Vol.~10. No.\,6. P.~524--539.
\bibitem{64-kl-1}
\Au{Ginestet C.\,E., Balanchandran~P., Rosenberg~S., Kolaczyk~E.\,D.} Hypothesis
testing for network data in functional neuroimaging. arXiv preprint arXiv:1407.5525,
2014.
\bibitem{65-kl-1}
\Au{Ginestet C.\,E., Fournel~A.\,P., Simmons~A.} Statistical network analysis for
functional MRI: Summary networks and group comparisons~// Front.
Comput. Neurosci., 2014. Vol.~8. P.~51.
{\sf
http://www.ncbi.nlm.nih.gov/ pmc/articles/PMC4018548/}.
\bibitem{66-kl-1}
\Au{Yan C.\,G., Craddock~R.\,C., Zuo~X.\,N., Zang~Y.\,F., Milham~M.\,P.}
Standardizing the intrinsic bra towards robust measurement of inter-individual
variation in 1000 functional connectomes~// Neuroimage, 2013. Vol.~80.
P.~246--262.
\bibitem{67-kl-1}
\Au{Marcus D.\,S., Harwell J., Olsen~T., Hodge~M., Glasser~M.\,F.,
Prior~F., Van Essen~D.\,C.} Informatics and data mining tools and strategies
for the human connectome project~// Front. Neuroinform., 2011.
Vol.~5. {\sf http://www.ncbi.nlm.nih.gov/pmc/\linebreak articles/PMC3127103/}.
\bibitem{68-kl-1}
\Au{Marcus D.\,S., Olsen T.\,R., Ramaratnam~M., Buckner~R.\,L.} The extensible
neuroimaging archive toolkit~// Neuroinformatics, 2007. Vol.~5. No.\,1. P.~11--33.
\bibitem{69-kl-1}
\Au{Brun A.} Manifold learning and representations for image analysis and
visualization. Department of Biomedical Engineering, Link$\ddot{\mbox{o}}$pings
Universitet, 2006. 104~p.
\bibitem{70-kl-1}
\Au{Mahmoudi A., Takerkart~S., Regragui~F., Boussaoud~D., Brovelli~A.}
Multivoxel pattern analysis for fMRI data: A~review~// Comput.
Math. Methods Med., 2012.
{\sf
http://www.hindawi.com/journals/cmmm/2012/\linebreak 961257/}.
\bibitem{71-kl-1}
\Au{Van Horn J.\,D., Toga~A.\,W.} Human neuroimaging as a~``Big Data''
science~// Brain Imaging Behavior, 2014.  Vol.~8. No.\,2.  P.~323--331.
\bibitem{72-kl-1}
\Au{Hillebrandt H., Friston~K.\,J., Blakemore~S.\,J.} Effective connectivity during
animacy perception-dynamic causal modelling of Human Connectome Project data~//
Sci. Rep., 2014.  Vol.~4. {\sf
http://www.ncbi.nlm.nih.gov/ pmc/articles/PMC4150124/}.
\bibitem{73-kl-1}
\Au{Lappalainen J., Sicilia~M.\,$\acute{\mbox{A}}$.,
Hern$\acute{\mbox{a}}$ndez~B.} Automatic hypothesis checking using eScience
Research Infrastructures, ontologies, and linked data: A~case study in climate change
research~// Procedia Comput. Sci., 2013. Vol.~18. P.~1172--1178.
\bibitem{74-kl-1}
\Au{Lenten L.\,J., Moosa I.\,A.} An empirical investigation into long-term climate
change in Australia~// Environ. Modell. Softw., 2003. Vol.~18. No.\,1.
P.~59--70.
\bibitem{75-kl-1}
\Au{Borges M.\,R.} Efficient market hypothesis in European stock markets~//
Eur. J.~Financ., 2010. Vol.~16. No.\,7. P.~711--726.
\bibitem{76-kl-1}
\Au{Bollen J., Mao H., Zeng~X.} Twitter mood predicts the stock market~//
J.~Comput. Sci., 2011. Vol.~2. No.\,1. P.~1--8.
\bibitem{77-kl-1}
\Au{Spangler S., Wilkins A.\,D., Bachman~B.\,J., \textit{et al.}} Automated
hypothesis generation based on mining scientific literature~// KDD'14 Proceedings,
2014. P.~1877--1886.
\bibitem{78-kl-1}
\Au{Zhou D., Bousquet~O., Lal~T.\,N., Weston~J.,  Sch$\ddot{\mbox{o}}$lkopf~B.}
Learning with local and global consistency~// Adv. Neur. Inform.
Proc. Syst., 2004.  Vol.~16. No.\,16. P.~321--328.

\end{thebibliography}
} }

\end{multicols}

 \label{end\stat}

 \vspace*{-3pt}

\hfill{\small\textit{Поступила в редакцию 10.02.2015}}
\renewcommand{\bibname}{\protect\rm Литература}
\renewcommand{\figurename}{\protect\bf Рис.}
   %12
\renewcommand{\bibname}{\protect\rmfamily References}
\renewcommand{\figurename}{\protect\bf Figure}

\def\stat{dolev}

\def\tit{PROBABILISTIC METHODS FOR SELF-CORRECTING HARDWARE DESIGN$^*$}

\def\titkol{Probabilistic methods for self-correcting hardware design}

\def\autkol{S.~Dolev,  S.~Frenkel, and D.\,E.~Tamir}

\def\aut{S.~Dolev$^1$,  S.~Frenkel$^{2,3}$, and D.\,E.~Tamir$^4$}

\titel{\tit}{\aut}{\autkol}{\titkol}

{\renewcommand{\thefootnote}{\fnsymbol{footnote}}
\footnotetext[1] {Extended abstract of this work was presented at the 13th International Conference
on Applied Stochastic Models and Data Analysis (ASMDA), 2009. This research is 
partially supported by the Russian 
Foundation for Basic Research (Grant RFBR 12-07-00109) and by the Rita Altura Trust Chair in
Computer Sciences, Lynne and William Frankel Center for Computer Science, Israel Science Foundation
(Grant No.\,428/11), Cabarnit Cyber Security MAGNET Consortium, Grant from the
Institute for Future Defense Technologies Research named for the Medvedi of the Technion,
and the Israeli Internet Association.}}

\renewcommand{\thefootnote}{\arabic{footnote}}
\footnotetext[1]{Department of Computer Science, Ben-Gurion University,  
Beer-Sheva 84105, Israel} 
\footnotetext[2]{Institute of Informatics 
Problems, Russian Academy of Sciences, Moscow 119333, Russian Federation}
\footnotetext[3]{Moscow Institute of Radio, Electronics, and Automation
``MIREA,'' Moscow 119454, Russian Federation} 
\footnotetext[4]{Department of Computer Science, Texas State University, San Marcos, TX 78666, 
USA} 

%\vspace*{6pt}

\def\leftfootline{\small{\textbf{\thepage}
\hfill INFORMATIKA I EE PRIMENENIYA~--- INFORMATICS AND APPLICATIONS\ \ \ 2013\ \ \ volume~7\ \ \ issue\ 4}
}%
 \def\rightfootline{\small{INFORMATIKA I EE PRIMENENIYA~--- INFORMATICS AND APPLICATIONS\ \ \ 2013\ \ \ volume~7\ \ \ issue\ 4
\hfill \textbf{\thepage}}}
 

\Abste{This paper presents several ways for extending the scope of program self-correction 
methods, based on the ``random self-reducibility'' property, to hardware design. The 
concept can be utilized for both 
analog and digital hardware-design. The extension is based on sampling, 
polynomial-interpolation, and error-correcting codes. In particular, the authors suggest 
using the well-known reconstruction of real-numerical functions for correcting faults 
remaining in analog and digital hardware, e.\,g., arithmetic logic units (ALU), after 
manufacturing testing. 
The present approach can complement the state-of-the-art technique of program self-correction 
by uniformly testing samples of operations and verifying the results of these samples.}

%\vspace*{2pt}


\KWE{self-correcting; real function computation; data analysis; interpolation}

%\vspace*{2pt}

\DOI{10.14357/19922264130413}


\vskip 20pt plus 9pt minus 6pt

      \thispagestyle{myheadings}

      \begin{multicols}{2}

            \label{st\stat}


\section{Introduction}

\noindent
The reliability of computation in the presence of errors is an important research topic. 
In particular, it is crucial in the 
scope of Digital Signal Processors (DSP) based classification tasks
and for embedded devices of safety-critical systems. 
Robust methods for identifying incoming waveforms, referred to as Modulation 
Classification~[1] are  some 
examples of this type of classification problems. The objective in these computation tasks is to minimize the error 
probability.

Since it is very difficult to test and detect all of the possible manufacturing faults in the stage of fabrication of modern 
hardware, there is an increasing interest in methods for self-correction~[2]. The 
methods presented in the literature (cf.~[2]), 
however, require knowledge related to the logical structure of the target design 
as well as sophisticated models to 
analyze their reliability.

In addition, the task of verifying the correctness of information processing devices 
such as microprocessors is a very 
challenging task since, typically, an exhaustive verification is an exponential function 
of the device complexity. 

   Generally, self-correction is based on simple estimation of the error probability using sampling rather than proving 
correctness or exhaustive evaluation.  Sampling can be used to efficiently identify the probability of a given ``black-box'' 
device to correctly compute the results for uniformly selected inputs. 
   
   Originally, the scope of self-correction has involved a program that computes functions-over-finite-fields, overcoming 
computation errors on a small fraction ($\varepsilon$) of their input~ [3, 4]. 
In this paper, however, it is shown that the 
random reducibility-based self-correction approach, originally suggested to \textit{amplify} the reliability of 
programs~[3, 4], can be used in the scope of nonfinite fields for self-correcting hardware. 
   
   One of the main contributions of the current paper is the introduction of novel ways 
   to extend the software-based self-correction 
paradigm to digital and analog hardware. 
   
The proposed approach can complement state-of-the-art techniques 
by uniformly testing samples of operations and verifying 
the results of these samples. Hence, it enables tolerating a small percentage of incorrect results due to manufacturing defects,
thereby facilitating the use of nonperfect hardware. In 
operation mode when an operation op$_i$ has to be executed, a uniformly chosen set of operations op$_j$, op$_k$  that 
can imply the result of the operation in hand are executed in order to maintain a verified result.

It is essential to note that the domain/range of computed functions cannot be restricted to 
finite fields and, by using some 
techniques~\cite{4-dol, 12-dol}, can include the real numbers. In addition,  sampling, error correction codes, 
polynomial interpolation, and segmentation are used
to increase the efficiency of self-correction for any given function over 
the real numbers. 

   The paper, which is an extended version of paper~\cite{14-dol}, is organized as follows. Section~2 provides the 
problem definition and surveys related research. Section~3 presents a methodology for increasing the computation 
accuracy by polynomial interpolation with error correction. Section~4 considers possible ways to reconstruct real 
functions using interpolation and section~5 presents a synopsis of approaches to possible implementation of 
self-correcting based hardware. Conclusions and proposals for future work 
are included in section~6.

\section{Problem Analysis and~Related Work}
   
   \noindent
   The following aspects of self-correcting computations
   are considered in this section:  ($i$)~incorrect function on a small 
fraction of the inputs; and ($ii$)~sampling-based self-correcting. 

\subsection{Incorrect function on~a~small fraction of~the~inputs}
    
\noindent
    Consider a hardware computation device, such as an ALU, designed to compute a function 
$f(x)$ of input values from the domain~$X$; and assume that the device produces incorrect results $f^*(x)\not= f(x)$ 
for a small fraction of~$X$. That is, $f^*(x)\not= f(x)$ for $x\in X_C\subset X$, such that $\vert X_C\vert \ll \vert 
X\vert$ where $\vert A\vert$ denotes the rank of a set~$A$. This is depicted in Fig.~1.

    Generally, the correctness of general-purpose and application-specific 
    microprocessors is verified by manufacturer-testing 
    at production time and/or self-checking procedures which are 
    based on online detection.
    
     The detection of all possible permanent faults through testing at the manufacturing cycle, however, is not feasible; 
and self-checking covers only a small fraction of erroneous bits. Hence, it requires specific knowledge about the logical 
structure of the target design. 

\begin{center}  %fig1
\vspace*{6pt}
\mbox{%
 \epsfxsize=40mm
 \epsfbox{dol-1.eps}
 }
  \end{center}
 \vspace*{6pt}
{{\figurename~1}\ \ \small{Incorrect results of the computation of the target function $f(x)$}}



\addtocounter{figure}{1}


The Floating Point divide instruction on the Pentium$^\registered$ processor 
is one of the well-known 
examples for this phenomenon. Despite more than
 10~years of debugging and enhancements, the Pentium$^\registered$ 
processor Floating Point divide instructions have produced inaccurate results for a fraction of 
inputs~\cite{5-dol}.
    
    Nevertheless, sampling can be used to efficiently identify the probability of a given device to correctly compute the 
results for inputs selected consistently according to a probability distribution such as uniform distribution. Indeed, 
sampling-based self-correction along with testing and self-checking is suggested in 
literature~\cite{3-dol, 4-dol}.

\subsection{Sampling-based self-correction}

\noindent
     A function is \textit{random self-reducible} of order~$k$ over a set~$D$ if its value at a given point can be 
efficiently reconstructed from its evaluation at   random points~[3, 4]. The reconstruction is possible if, and only if, 
there exists a function~$\varphi$ and a set of random functions $\sigma_1, \sigma_2,\ldots , \sigma_n$ such that 
$f(x)=\varphi(x,r,f(\sigma_1(x,r),\ldots , f(\sigma_n(x,r)))$ for $x,r\in D$. This property allows reconstruction of the 
value of a function~$f$ using a finite number of elements taken from its domain without requiring any knowledge about 
the implementation of the device that implements the function, e.\,g., a hardware operational block or a program, which 
performs the computation. Note that polynomials of degree~$d$ over a finite field are random self-reducible using 
$d+1$ random points~\cite{4-dol}.
     
     Consider a function with no input/output domain restrictions. For example, these domains might include integer 
values, real numbers, vectors, etc. In order to use the \textit{reliability amplification} technique 
utilizing the random 
self-reducible property~\cite{3-dol, 6-dol}, it is necessary to provide a specific number of 
\textit{batches} that yield sufficient 
probability for the majority of the batches to be correct; thereby, enabling using majority vote procedures for self-
correction. In this context, the term \textit{batch} denotes a
\textit{set of program input/output instances}.
In other words, in the context of this paper, reliability amplification denotes the increase of correct 
computation probability due to computation reorganization, for example, using the fact that the functions are random 
self-reducible to reorganize the functions.
       
       Let $n$ be the number of batches and let $p$ denote a fraction of the inputs for which the computations can be 
incorrect. The probability of correct computation can be calculated as the probability that the outputs obtained for more 
than $n/2$  of the batches are correct. This is given by:
       \begin{equation}
       \mathrm{Pr}\left( k\geq \left\lfloor \fr{n}{2}\right\rfloor+1\right)=1-\sum\limits_{k=1}^L C_n^k p^k q^{n-k}
       \label{e1-dol}
       \end{equation}
where $k$ is the number of correct outputs; $p=1-q$; $L=\lfloor 
n/2\rfloor +1$ is the probability of correct computation for each batch; and
$C_n^k$ is the Binomial coefficient. 

   The reliability of the computations depends on the number of batches and on the 
   choice of the reliability-parameter (or confidence level)~$r$ which is the probability 
   of obtaining a majority of wrong results.  
     
   According to the Chernoff inequality, the required number of batches can be expressed as
   \begin{equation}
   n\geq \fr{1}{(p-1/2)^2}\,\ln\left( \fr{1}{\sqrt{1-r}}\right)\,.
   \label{e2-dol}
   \end{equation}
   
   For example, if the function computed is a quadratic polynomial then 
   $p=(1-r)^3$ as each batch 
must include at least three input points (vectors). Equations~(1) and~(2) show that the use of majority-vote based choice 
among the results obtained from uniformly chosen batches can ``amplify'' the original reliability of devices if  
enough batches are used. Nevertheless, the minimum number of batches required for obtaining a correct computation results with 
a confidence level~$r$ might be very large, even when the device has a small probability of errors. For example, more than 
10,000 batches are required in the case of quadratic polynomials for $\varepsilon=0.2$ and $r=0.05$, where 
$\varepsilon$ is the small fraction the inputs and~$r$  is the reliability parameter. On the other hand, according to Eq.~(1), the 
probability of correct computation with a reasonable number of batches is less than the original $\varepsilon =0.2$. 
Hence, there is no amplification of the computation reliability, that is, 
using the Chernoff majority rule~(1) will not lead 
to an increase in correct computations. Consequently, in this example,
given~$r$, it is impossible to provide correct 
computations on the basis of the majority vote rule, if the number of inputs is less than 10,000 batches, as the 
computations are erroneous on a fraction of inputs that is greater than~$\varepsilon$.                               
    
    A function over a group $G$ is \textit{linear} if it maps the group~$G$ to a 
    group~$H$ so that $(x_1\oplus 
x_2)=f(x_1)\otimes f(x_2)$ where $\oplus$ and~$\otimes$ are the
group operations. Integer multiplication and modular 
multiplication are some examples for such functions. 
From the point of view of the computation overhead, one benefit of 
the linearity is that given the values $f(x_1)$ and $f(x_2)$ and given that $x=x_1\oplus x_2$, the function $f(x)$ can be 
computed as $f(x_1)\otimes f(x_2)$ which might be an easier computation task. An important aspect of the 
self-correction methods proposed in this paper is that the linearity properties of functions 
defined over finite fields can be 
utilized to increase the probability of success of a batch and, therefore, 
reduce the required number of batches. Error 
correcting codes can be used to obtain a better success rate for a batch result~\cite{6-dol}. This approach, referred to as 
\textit{batch self-corrector}, has been applied to the function $f(x)=x\,\mathrm{mod}\, R$ over the positive integers 
domain~\cite{6-dol}. In addition, it has been used for self-testing, which is a part of the self-correction techniques. For 
example, Spielman suggested using the result of encoding functions defined over a finite field in order to 
increase the probability of correct computations of batches~\cite{7-dol}. 
    
    Assume that one is able to digitize (discretize) the input domain~$X$  
    for a set of integer or rational 
    values~\cite{8-dol}, thereby transforming the given function to the domain/range of finite fields. Then, if the 
function is a polynomial, it becomes an integer function. Furthermore, if one is able to use the linearity properties to 
reduce the number of batches, then only the ``improved'' integer function has to be applied to the final real computation. 
This results in the following stages for computing $f(x)$:
    \begin{enumerate}[(1)]
\item digitize the original function to a finite field (cf.~\cite{8-dol, 9-dol});
\item select a polynomial interpolation function over the finite field; 
\item use error correction (e.\,g., the Berlekamp--Welch algorithm~\cite{9-dol}) to correct 
a batch, leading to a reduction 
of the necessary number of batches according to the Chernoff-bound; and
\item reconstruct the digitized real number function from the discrete domain to the real domain using polynomial 
interpolation such as Tailor, Chebyshev, etc. Alternatively, 
this stage can be implemented by checking whether the 
original result of the device is close to the discrete value obtained.
\end{enumerate}

\begin{figure*} %fig2
   \vspace*{1pt}
 \begin{center}
 \mbox{%
 \epsfxsize=150.174mm
 \epsfbox{dol-2.eps}
 }
 \end{center}
 \vspace*{-6pt}
\Caption{Correctness probability vs.\ correction rate~(\textit{a})
and vs.\ the logarithm of the number of batches~(\textit{b}) for different ALU error rates: 
\textit{1}~--- 0.2; \textit{2}~--- 0.3; and 
\textit{3}~--- 0.4}
\end{figure*}

\section{Amplification by~Polynomial Interpolation with~Error Correction}

\noindent
   Error correcting codes can be used to exploit linearity~\cite{8-dol}. For example, the 
   Reed--Solomon (RS) code of a 
polynomial over a finite field has linear properties~\cite{7-dol}. Let $(E,D)$ be an encoding-decoding pair for an error-
correcting code of a code-word of length~$n$ with rate~$T$ (that is, a code that can correct~$T$ bits or symbols) for a 
polynomial function. For example, consider the Berlekamp--Welch algorithm, of RS codes polynomials 
computation in the presence of errors of interpolation over a finite field~\cite{9-dol}.  The polynomial~$P$  is unknown; 
and the only information about~$P$ is that it is of degree of~$l$ (say, $l=2$). The polynomial~$E$ is unknown as well. 
Using the relationship defined above, one can produce a linear system whose solutions are the coefficients of~$P$ 
and~$E$.  This is shown in the following equations.
   
   Let $Q(X)=aX^3+bX^2+cX+d=P(X) E(X)$ where $a$, $b$, $c$, and~$d$ are unknown coefficients. 
Substituting $P(X)$ by $R(X)=P(X)/Q(X)$,  one obtains:

\noindent
   $$
   aX^3+bX^2+cX+d=R(X) E(X)=R(X) (X-e)
   $$
   which can be rewritten as: 
   $$
   aX^3+bX^2+cX+d+R(X) e =R(X) X\,.
   $$ 
   Now, one can substitute $X$ by $\{0,1,2,3,4\}$  to obtain five linear equations in five unknowns. Solving this linear 
system for $a$, $b$, $c$, $d$, and~$e$ provides the polynomials $Q(X)$ and $E(X)$ which enable finding $P(X)$ by 
computing the quotient $Q(X)/E(X)$, and from~$P$ it is possible to recover the original (uncorrupted) values. In this 
case, the computation correctness probability can be defined as the probability that the number of incorrect symbols 
(from the specific finite field) is at most~$T$. This probability is given by:  
   \begin{equation}
   P_{\mathrm{corr}}=\sum\limits_{i=0}^T C_N^i \varepsilon^i (1-\varepsilon)^{N-i}
   \label{e3-dol}
   \end{equation}
where $N$ is the degree of the polynomials, which is the 
number of points used for interpolation, referred to as the 
block size ($N=2$ in the case of a quadratic interpolating 
polynomials), and $\varepsilon$ corresponds to the probability of a symbol error. Obviously, the code distance is $T=(n-
k)/2$, where $k$ is the number of data symbols that has to be maintained. That is, for $T=1$,  a (5,3) code is obtained, 
$T=2$ results in a (7,3) code, and $T=3$ provides a (9,3) code, where the left value in a pair of numbers that describe a 
code is the code-word length, and the right value is the number of data symbols. 

   Figure~2\textit{a} shows the batches correctness probability ($P_{\mathrm{corr}}$) 
   computed using Eq.~(2) vs.\ the correction 
rate~$T$ (the $x$ axis) for different device errors. 

   As seen in the figure, the probability of a correct result can be increased using the data correction encoding (e.\,g., the 
RS codes).   Correspondingly, in some cases, 
due to this encoding, one can afford to work with greater 
fraction of erroneously computed inputs. This means that amplification of the original probability (by using 3-symbols 
error correction) is possible even when the error probability of the device is approximately 0.3; moreover, such a 
correction is essential in the case that the error probability is around 0.4. Thus, if one deals with a discrete function 
(whose domain is a finite field) where RS-encoding can be used, it is possible to improve the probability of correct 
computation by repeating the computation using uniform random inputs (from this finite field), interpolating them, and 
choosing the result according to a majority vote rule.
   
   Figure~2\textit{b} presents an example of the computation of correctness probability vs.\ the logarithm (log base~10 is denoted 
as ``lg'') of the number of batches for different ALU error rates with RS (7,3). This corresponds to RS encoding that 
corrects two symbols ($T=2$), when the probability of a symbol error, which is equal in this instance to the erroneous 
fraction of inputs~$\varepsilon$, is $P_s=0.3$; and $P_{\mathrm{corr}}=0.647$. In this case, the amplification of the correct 
probability starts when 21~batches are used, a considerably lower number of batches than for the case of noncoded batches. In 
comparison, in the case of $\varepsilon=0.3$, error correction is impossible for noncoded batches as $(1-r)^3<1/2$ and 
is irrelevant for the majority-vote-based choice algorithm. One should take into account, however, the need for 4~extra 
points for each batch (7~instead of~3) to achieve this improvement. 
{ %\looseness=1

}
   
   Nevertheless, adding points to a batch can significantly reduce the required number 
   of batches even when~$p$  is 
only slightly larger than~$1/2$.  
   
   
\section{Real Function Reconstruction}

\noindent
   The feasibility of implementation of the self-correcting algorithm proposed in this paper
    for an error correction in analog hardware
    depends on existence of appropriate transformation of real signals input (numbers) to finite fields.  
As shown below, the state-of-the-art of analog-digital design allows finding proper solutions of this problem.   
   
       Note that the explicit reconstructing polynomials and rational functions over finite fields are presented by Sigal 
\textit{et al}.\ in~\cite{3-dol}. 
Sigal \textit{et al}.\ use the fact that multiplication of any fixed element of 
finite fields by a random uniformly distributed element of the field gives a result that is uniformly distributed over the 
field. Therefore, in order to use the random self-reducibility-based approach to self-correction of real functions, one 
should consider discrete transformation of these functions to finite fields and commence with reconstruction of the 
functions. 
   
   In general, the reconstruction of real continuous function from digital data is governed by the \textit{Nyquist 
sampling theorem}~\cite{6-dol}, which requires that a band-limited continuous function is sampled with a frequency 
equal to, or greater than, twice the maximum frequency of the signal. This digitization-reconstruction model, however, is 
not suitable in the context of self-correction, since the function is interpolated by algebraic polynomials.  Moreover, the 
batches include randomly generated points. Hence, nonuniform discretization is required. This raises additional 
difficulties in the reconstruction~\cite{8-dol}. In addition, the \textit{quantization} of function values implies 
representation by a finite number of bits (say, $n$~bits). Due to the finite precision representation of real numbers in 
computational devices; roundoff errors might occur during the calculations. The problem is finding the minimum 
accuracy necessary to ensure that the inverse quantization transformation that is a part of the 
digital-to-analog 
transformation can perform rounding and roundoff. This would make the function result equivalent to the rounding of 
the exact result (which could be obtained by the device) for all possible inputs. Since real-valued polynomial 
interpolations, say, Tailor polynomials, are defined over input variables given as real numbers, they cannot be used to 
express the finite bit-width limitations. Thus, one should coordinate the number of Tailor series 
terms and the number of bits in 
the aforementioned finite numbers representation. In order to resolve this problem, it is possible to use the technique 
presented in~\cite{12-dol}, where the coefficients of the series are expressed in terms of a finite number of bits referred 
to as fractional bits (FB). Several techniques for finding the necessary numbers of the Tailor series coefficients given a 
specific number of FBs are suggested in~\cite{12-dol}.
     
     Another mean for increasing the accuracy is segmentation, which refers to dividing the input into subintervals, 
slices, or segments~\cite{10-dol}. Generally, a set of coefficients of a low-degree polynomial can be used to evaluate 
each segment, and the error probability is computed independently for each segment~\cite{10-dol}. Note that the degree 
is an important parameter since a small degree enables correcting more errors using the 
Berlekamp--Welch algorithm. An 
evaluation of the obtained accuracy can be controlled by varying the number of segments and/or the polynomial degree. 
Using online segmentation requires predicting the interpolation error for each segment. For several differential 
functions, this error depends on the first $d+1$ derivatives $f^{(d+1)}(x)$, where $d$ is the
 degree of the interpolation 
polynomial, can be calculated during a preprocessing stage~[13]. 
     
\section{Synopsys}

\noindent
The computation of the value of $f(x)$ at a given point   by evaluation at several random points can be implemented via 
interpolation of $f(x)$ using samples of~$x$. For example, the input data of an ALU, which can be faulty with some known 
probability, can be considered as a set or series of batches, each of which is a series of~$k$  randomly generated 
arguments $r_{i,j}$ where $i=1,\ldots ,m$  is the number of the ALU random inputs 
needed for interpolation; for example, $m=2$ 
for the linear interpolation and $k=l+1$ in the case of the interpolating polynomial 
of degree~$l$. The variable $j=1,\ldots ,n$ is the 
number of interpolations (number of batches of computations). The series quantity must provide a reliable choice of result of 
interpolation by majority in the sequence of results of interpolation obtained from the batches. 
    
    The model suggested in~[3, 4] which uses uniform batches of random inputs is practical only in the case of 
relatively small error probability~$\varepsilon$. Nevertheless, even in the relatively simple case of the quadratic 
polynomial, the batch correctness probability defined by the Chernoff-bound success probability of~0.512 for 
$\varepsilon=0.2$ might imply a much smaller success probability for all polynomials of degrees $d>2$.
    
    Computation in a finite field is one possible way to increase the batches' correctness probability. This means 
operating with encoded data using coding methods such as RS codes and the Berlekamp--Welch 
algorithm~\cite{7-dol}. This coding is an interpolation as it provides computation of polynomials in all the required 
points, using several points where the polynomial is known. In fact, the building of polynomials in the Berlekamp-Welch 
decoding algorithm is similar to Lagrange interpolation. Note that, in effect, different types of errors of 
computation (referred to as ``ALU\_errors'') are considered in the compared approaches. While the symbol (bit) error relates to a specific 
error rate~$T$ (say, $T=3$), the error considered in self-correcting program theory (e.\,g., in~\cite{3-dol}) is only a 
small fraction~$\varepsilon$ of erroneous computed  inputs. This means that  for the $(1-\varepsilon)$ fraction of 
inputs $\mu(f(x),f_C(x))\leq \sigma$ where $f_C(x)$ is the function $f(x)$ computed by the ALU;
$\mu (f(x), f_C(x))$ 
is the measure of distance between the exact value $f(x)$ and the computed function $f_C(x)$;
and $\sigma$ is the threshold 
error value. In general, $\sigma$ corresponds to other erroneous quantities of bits. 
    
     As for DSP-based classification tasks, mentioned above as a prospective field of the self-correcting approach 
application, many current approaches use various polynomials to compute the classification characteristics, e.\,g., spline 
approximation in image processing. Therefore, in the event that the DSP computes the characteristics correctly on all but 
a small fraction~$\varepsilon$ of inputs, the algorithms mentioned above are suitable, and implementation of the 
computation schema, presented in sections~2--4, can essentially improve the classification reliability in comparison with the 
DSP characteristic~$\varepsilon$.
     
Since the error correction can be interpreted as a ``decoding of code-words,'' one can 
borrow  several ideas from Locally 
Decodable Codes (LDC)~\cite{13-dol}. Locally decodable codes are the correcting codes where in order to retrieve the correct value of 
just one position of the input with high probability, it is sufficient to read a small number of positions of the 
corresponding possibly corrupted code-word. The locally decodable code can recover from a much higher 
     error-rate~\cite{4-dol}. One of the reasons for using LDC is that the previously used RS code consists 
of complete evaluations of polynomials of total degree up to~$d$. In particular, there are LDCs which provide reduction 
of the error rate of the code with the number of queries which can be essentially 
higher than the polynomial degree~$d$. 
Hence, the polynomial 
degree is not a limiting factor for the fraction of erroneous results reduction. 
In this context, the term \textit{query} is a measure of complexity computed by the
number of bits that need to be read from a corrupted code-word in order to recover
a single bit of the encoded word~[14]. It should be noted that LDCs are based on the classical Reed--Muller
(RM) codes, which have rather simple and fast hardware implementation~[15].
     
     
\section{Concluding Remarks and~Proposals for~Future Research}
     
\noindent
In this paper, recent results in self-correcting computations have been presented.
     As the results show, in spite of essential reduction in the number of batches needed for suitable computation 
accuracy, this number might be rather significant. The complexity of the proposed
approach depends on the number of\, batches as 
well as on the complexity of decoding the codes used for increasing correct computation probability for each batch. 
Feasible ways  for improving the  amplification have been proposed
and it has been demonstrated that these methods can minimize the number of 
batches of the computation function used to correct the computed value and provide a significant decrease in the error 
probability with the number of the batches used. 
Furthermore, a hardware implementation of this approach to self-correction can be
derived from hardware implementation of the coding methods such as RS and RM~[15, 16].
     
     In the future, both the theory of random self-reducibility and new results in LDC 
will be explored for the problem of reconstruction of real numerical functions for correcting faults remaining in hardware after 
manufacturing testing. In addition, the authors plan to explore nonuniform sampling methods such as compressive sensing. 
Furthermore, a study of technical details of hardware implementation as well as DSP-based solutions will be performed.

{\small\frenchspacing
{%\baselineskip=10.8pt
\begin{thebibliography}{99}
\bibitem{1-dol}
\Aue{Bil$\acute{\mbox{e}}$n,~S., and A. Price}.  2007. 
Modulation classification for radio interoperability via SDR. \textit{SDR 07 Technical Conference and Product Exposition 
Proceedings}. {\sf http://www.slideshare.net/kirill443/12-4-5647963} (accessed November~7, 2013).
\bibitem{2-dol}
\Aue{Lala,~P.} 2000. 
\textit{Self-checking and fault-tolerant digital design}. Morgan Kaufmann Publs. 400~p.
\bibitem{3-dol}
\Aue{Sigal,~A., R.~Lipton, R.~Rubinfeld, and M.~Sudan}. 1990. Reconstructing algebraic functions from mixed data. \textit{33rd 
Annual Symposium on Foundations of Computer Science.} 503--512.
\bibitem{4-dol} %4
\Aue{Gemmell,~P., R. Lipton, R.~Rubinfeld, M.~Sudan, and A.~Wigderson}. 1991.
Self-testing/correcting for polynomials and for approximate functions. \textit{23rd Annual ACM Symposium on Theory of 
Computing Proceedings}. 32--34.
\bibitem{12-dol} %5
\Aue{Lee, D.\,U., R.~Cheung, W. Luk, and J.~Villasenor}. 2008.  Hardware implementation trade-offs of polynomial 
approximations and interpolations. \textit{IEEE Trans. Comput.}  57(5):686--701.

\bibitem{14-dol} %14
\Aue{Dolev,~Sh., and  S.~Frenkel}. 2009. Extending the scope of self-correcting. 
\textit{13th Conference (International) on Applied 
Stochastic Models and Data Analysis (ASMDA2009)  Proceedings}. 458--462. 

\bibitem{5-dol} %6
\Aue{Nicely, T.\,R.} {Some results of computational research in prime numbers (Computational number theory)}.
{\sf  http://www.trnicely.net/pentbug/pentbug.html} (accessed December 2010).

\bibitem{6-dol} %7
\Aue{Rubinfeld,~R.} 1992. Batch checking with applications to linear functions. 
\textit{Inform. Process. Lett.} 42:77--80.
\bibitem{7-dol} %8
\Aue{Spielman, D. } 1996. Highly fault-tolerant parallel computation. \textit{37th IEEE  Annual Symposium on Foundations of 
Computer Science Proceeding}.  154--163.
\bibitem{8-dol} %9
\Aue{Oppenheim, A.\,V.,  R.\,W.~Schafer, and J.\,R.~Buck}. 1999.  
\textit{Discrete-time signal processing}. Upper Saddle River, 
NJ: Prentice-Hall. 871~p.
\bibitem{9-dol} %10
\Aue{Berlekamp,~E., and L.~Welch}. 1986.  Error correction  of algebraic block codes. U.S. Patent No.\,4,633,470. 
\bibitem{10-dol} %11
\Aue{Tertinek,~S., and C.~Vogel}. 2008. Reconstruction of nonuniformly sampled bandlimited  signals using a 
differentiator-multiplier cascade. \textit{IEEE Trans. Circuits  Syst.} 55(8):2273---2286.
\bibitem{11-dol}
\Aue{Pang,~Y., and K. Radecka}. 2008. Optimizing imprecise fixed-point arithmetic circuits specified by Taylor series through 
arithmetic transform. \textit{Design Automation Conference  DAC'08 Proceedings}. 397--402.

\bibitem{13-dol} %12
\Aue{Yekhanin, S.} 2011. Locally decodable codes. \textit{Foundations Trends Theoretical Computer Sci.} 7(1):1--117.

\bibitem{14-1-dol}
\Aue{Rahardja,~S., and B.\,J.~Falkowski}. 2001. Efficient algorithm to calculate Reed--Muller
expansions over GF(4). \textit{IEE Proceedings~--- Circuits, Devices and  
Systems}. 148(6):289, 297.

\bibitem{14-2-dol}
\Aue{Leroux,~C., G.~Le Mestre, C.~Jego, P.~Adde, and M.~Jezequel.}
2008. A~5-Gbps FPGA prototype of a (31,29)$^2$ Reed--Solomon turbo decoder.
\textit{5th Symposium (International) on Turbo Codes and Related Topics Proceedings}.
67--72.



\end{thebibliography} } }

\end{multicols}

\vspace*{-6pt}

\hfill{\small\textit{Received October 23, 2013}}

\vspace*{-12pt}

\Contr

\noindent
\textbf{Dolev Shlomi} (b.\ 1958)~--- professor, Doctor of Science in computer science, Dean 
of the Faculty of Natural Sciences, Ben-Gurion University of the Negev, 
Beer-Sheva 84105, Israel; dolev@cs.bgu.ac.il 

\vspace*{3pt}

\noindent
\textbf{Frenkel Sergey L.} (b.\ 1951)~--- Candidate of Science (PhD) in 
technology, senior scientist, Institute of Informatics Problems, Russian 
Academy of Sciences, Moscow 119333, Russian Federation; associate professor, Moscow Institute of Radio, 
Electronics, and Automation (MIREA), Moscow 119454, Russian Federation;
fsergei@mail.ru

\vspace*{3pt}

\noindent
\textbf{Tamir Dan E.} (b.\ 1955)~---PhD-CS, 
associate professor in the Department of Computer Science, Texas State 
University, San Marcos, TX 78666, USA; dt19@txstate.edu




\vspace*{12pt}

\hrule

\vspace*{2pt}

\hrule

%\newpage

\def\tit{ВЕРОЯТНОСТНЫЙ ПОДХОД К~САМОКОРРЕКТИРУЮЩИМСЯ ВЫЧИСЛЕНИЯМ 
В~ПРОЕКТИРОВАНИИ АППАРАТУРЫ}

\def\aut{Ш.~Долев$^1$,  С.~Френкель$^2$, Д.\,Е.~Тамир$^3$}


\def\titkol{Вероятностный подход к~самокорректирующимся вычислениям 
в~проектировании аппаратуры}

\def\autkol{Ш.~Долев,  С.~Френкель, Д.\,Е.~Тамир}


\titel{\tit}{\aut}{\autkol}{\titkol}

\vspace*{-12pt}

\noindent $^1$Университет им.\ Бен-Гуриона в Негаве, Беэр-Шева, Израиль, dolev@cs.bgu.ac.il\\ 
\noindent $^2$Институт проблем информатики Российской академии наук;
Московский государственный технический\linebreak
$\hphantom{^1}$университет радиотехники,
электроники и автоматики (МГТУ МИРЭА), Москва, Россия,\linebreak
$\hphantom{^1}$fsergei@mail.ru\\
\noindent
$^3$Университет Техаса, г.\ Сан-Маркос, США,  dt19@txstate.edu

\vspace*{6pt}

\def\leftfootline{\small{\textbf{\thepage}
\hfill ИНФОРМАТИКА И ЕЁ ПРИМЕНЕНИЯ\ \ \ том\ 7\ \ \ выпуск\ 4\ \ \ 2013}
}%
 \def\rightfootline{\small{ИНФОРМАТИКА И ЕЁ ПРИМЕНЕНИЯ\ \ \ том\ 7\ \ \ выпуск\ 4\ \ \ 2013
\hfill \textbf{\thepage}}}
 

\Abst{Описаны некоторые подходы к распространению метода самокоррекции 
программ, основанного на свойстве <<случайной самосокращаемости>> (random 
self-reducibility), на задачи проектирования  аппаратной части вычислительных систем.
Данная концепция может быть использована для проектирования как цифровой, так и 
аналоговой аппаратуры. Расширение метода основано на использовании случайных 
выборок, полиномиальной интерполяции и теории самокорректирующихся кодов.
В~частности, предлагается использовать известные методы реконструкции числовых 
функций для коррекции ошибок, вызываемых неисправностями, остающимися в 
аппаратуре после производственного контроля. 
Предлагаемый подход может дополнять известные методы тестирования цифровых и 
аналоговых  приборов посредством использования равновероятной выборки операций 
и верификации результатов их выполнения, обеспечивая приемлемый уровень (небольшую 
долю) неправильных результатов.} 

\KW{самокоррекция; вычисление действительных функций; анализ данных; интерполяция}


\DOI{10.14357/19922264130413}

%\vspace*{9pt}

%\Ack
%\noindent
Расширенные тезисы данной статьи были представлены на 13-й Международной конференции
по прикладным стохастическим моделям и анализу данных (ASMDA-2009).
Работа выполнена при час\-тич\-ной поддержке Российского фонда фундаментальных исследований 
(грант №\,12-07-00109),  Фонда главы отделения информатики Риты Алтура, Центра
вычислительной техники им.\ Линне и Уильяма Франкелей, Израильского научного фонда
(грант №\,428/11), Кабарнит кибербезопасности консорциума <<Магнит>>, гранта
Института перспективных оборонных технологий им.\ Медведи (Технион) и
Израильской Интернет ассоциации.

  \begin{multicols}{2}

\renewcommand{\bibname}{\protect\rmfamily Литература}
%\renewcommand{\bibname}{\large\protect\rm References}

{\small\frenchspacing
{%\baselineskip=10.8pt
\addcontentsline{toc}{section}{References}
\begin{thebibliography}{99}

\bibitem{1-dol-1}
\Au{Bil$\acute{\mbox{e}}$n~S., Price~A.}
Modulation classification for radio interoperability via SDR~// {SDR 07 Technical Conference and Product Exposition 
Proceedings},  2007. {\sf http://www.slideshare.net/kirill443/12-4-5647963} (accessed November~7, 2013).
\bibitem{2-dol-1}
\Au{Lala~P.} Self-checking and fault-tolerant digital design.~--- Morgan Kaufmann Publs., 2000.
400~с.
\bibitem{3-dol-1}
\Au{Sigal~A., Lipton~R., Rubinfeld~R.,  Sudan~M}.  Reconstructing algebraic functions from mixed data~// {33rd Annual 
Symposium on Foundations of Computer Science}, 1990. P.~503--512.
\bibitem{4-dol-1}
\Au{Gemmell~P., Lipton~R., Rubinfeld~R., Sudan~M., Wigderson~A}.
Self-testing/correcting for polynomials and for approximate functions~// {23rd Annual ACM Symposium on Theory of Computing 
Proceedings},  1991. P.~32--34.

\bibitem{12-dol-1} %5
\Au{Lee D.\,U., Cheung~R., Luk~W., Villasenor J.}  Hardware implementation trade-offs of polynomial approximations and 
interpolations~// {IEEE Trans. Comput.}, 2008.  Vol.~57. No.\,5. P.~686--701.

\bibitem{14-dol-1}
\Au{Dolev~Sh., Frenkel~S}.  Extending the scope of self-correcting~// {13th 
Conference (International) on Applied Stochastic Models 
and Data Analysis (ASMDA2009)  Proceedings}, 2009.  С.~458--462.

\bibitem{5-dol-} %6
\Au{Nicely T.\,R.} {Some results of computational research in prime numbers (Computational number theory)}.
{\sf  http://www.trnicely.net/pentbug/pentbug.html} (last retrieved December 2010).

\bibitem{6-dol-1} %7
\Au{Rubinfeld~R.} Batch checking with applications to linear functions~// 
{Inform. Process. Lett.}, 1992. Vol.~42. P.~77--80.
\bibitem{7-dol-1} %8
\Au{Spielman D.} Highly fault-tolerant parallel computation~// {37th IEEE  Annual Symposium on Foundations of Computer 
Science Proceeding}, 1996. P.~ 154--163.
\bibitem{8-dol-1} %9
\Au{Oppenheim A.\,V.,  Schafer R.\,W., Buck~J.\,R.}  {Discrete-time signal processing}.~--- Upper Saddle River, NJ: Prentice-
Hall, 1999. 871~p.
\bibitem{9-dol-1} %10
\Au{Berlekamp~E., Welch L.}  Error correction  of algebraic block codes. U.S. Patent No.\,4,633,470, 1986. 
\bibitem{10-dol-1} %11
\Au{Tertinek~S., Vogel~C}.  Reconstruction of nonuniformly sampled bandlimited  signals using a 
differentiator-multiplier cascade~// {IEEE Trans. Circuits  Syst.}, 2008. Vol.~55. No.\,8. P.~2273--2286.
\bibitem{11-dol-1}
\Au{Pang~Y., Radecka~K}.  Optimizing imprecise fixed-point arithmetic circuits specified by Taylor series through arithmetic 
transform~// {Design Automation Conference  DAC'08 Proceedings}, 2008.  P.~397--402.

\bibitem{13-dol-1} %12
\Au{Yekhanin S.}  Locally decodable codes~// {Foundations  Trends Theoretical Computer Sci.}, 
2011. Vol.~7. Iss.~1.  P.~1--117.

\bibitem{14-1-dol-1}
\Au{Rahardja~S., Falkowski B.\,J.} Efficient algorithm to calculate Reed--Muller
expansions over GF(4)~// IEE Proceedings~--- Circuits, Devices and  
Systems, 2001. Vol.~148. No.\,6. P.~289, 297.

\bibitem{14-2-dol-1}
\Au{Leroux~C., Le Mestre~G., Jego~C., Adde~P., Jezequel~M.}
 A~5-Gbps FPGA prototype of a (31,29)$^2$ Reed--Solomon turbo decoder~//
5th Symposium (International) on Turbo Codes and Related Topics Proceedings, 2008.
P.~67--72.


 

\end{thebibliography}
} }

\end{multicols}

 \label{end\stat}

\hfill{\small\textit{Поступила в редакцию 23.10.13}}
%\renewcommand{\bibname}{\protect\rm Литература}  
\renewcommand{\figurename}{\protect\bf Рис.}  %13


%\end{document}


%%%%%%%%%%%%%%%%%%%%%%%%%%%%%%%%%%%%%%%%%%%%%%%


%\def\stat{rez}
{%\hrule\par
%\vskip 7pt % 7pt
\raggedleft\Large \bf%\baselineskip=3.2ex
Р\,Е\,Ц\,Е\,Н\,З\,И\,И \vskip 17pt
    \hrule
    \par
\vskip 6pt plus 6pt minus 3pt }

%\thispagestyle{headings} %с верхним колонтитулом
%\thispagestyle{myheadings} %с нижним колонтитулом, но в верхнем РЕЦЕНЗИИ

\def\tit{НОВАЯ КНИГА И.\,Н.~СИНИЦЫНА, А.\,С.~ШАЛАМОВА <<ЛЕКЦИИ ПО ТЕОРИИ 
ИНТЕГРИРОВАННОЙ ЛОГИСТИЧЕСКОЙ ПОДДЕРЖКИ>> (М.: ТОРУС ПРЕСС, 2012. 624~с.)}

%1
\def\aut{Д.ф.-м.н., профессор С.\,Я.~Шоргин}

\def\auf{\ }

\def\leftkol{\ % РЕЦЕНЗИИ
}

\def\rightkol{ \ } 

%\def\leftkol{\ } % ENGLISH ABSTRACTS}

%\def\rightkol{\ } %ENGLISH ABSTRACTS}

%\def\leftkol{РЕЦЕНЗИИ}

%\def\rightkol{РЕЦЕНЗИИ}

\titele{\tit}{\aut}{\auf}{\leftkol}{\rightkol}
\vspace*{-18pt}


     \label{st\stat}

     \begin{multicols}{2}
     {\small
     {\baselineskip=10.1pt
     

      В книге представлено системное изложение теоретических основ одного из новейших 
направлений в \mbox{об\-ласти} экономики послепродажного обслуживания изделий наукоемкой 
продукции (ИНП) длительного пользования~--- интегрированной логистической поддержки
(ИЛП). 
{\looseness=1

}

Приведены также результаты новых работ, выполненных в Институте проблем информатики 
Российской академии наук в рамках научного направления <<Информационные технологии и 
анализ сложных сис\-тем>>.
 {%\looseness=1

}
     
      Излагаемые в книге научные подходы позво\-ляют карди\-наль\-но реформировать 
существующие системы производства и эксплуатации ИНП путем создания и внед\-ре\-ния 
методов рационального и оптимального управ\-ле\-ния процессами расходования 
вре\-мен\-н$\acute{\mbox{ы}}$х, 
мате\-ри\-аль\-ных, трудовых и других ресурсов на всех стадиях жизненного цикла изделий (ЖЦИ) по 
критериям экономической целесообразности и эф\-фек\-тив\-ности.
  {\looseness=1

}
    
      В книге приведен краткий обзор причин возник\-новения и
      развития CALS-методологии как основы 
современных международных стандартов по созданию и функционированию глобальных 
ин\-фор\-ма\-ци\-он\-но-ком\-му\-ни\-ка\-ци\-он\-ных систем, ее ключевых возможностей и эффективности 
результатов ее использования. 
Авторы %\linebreak 
предлагают ряд научных обоснований для разработки 
единой теории проектирования и управления систем ИЛП для полноценного использования 
преимуществ %\linebreak
 суще\-ст\-ву\-ющей методологии, определяют \mbox{общую} структурную схему 
комплексной системы <<ИНП-СППО>> и необходимость разработки для ее описания 
гибридных стохастических моделей.
{%\looseness=1

}

%\columnbreak
      
      Книга состоит из пяти частей, где последовательно излагается материал по каждой из 
следующих тем: <<Интегрированная логистическая поддержка>>, <<Теория гибридных 
стохастических систем и компьютерная поддержка исследований и разработок>>, <<Основы 
математического моделирования, анализа и синтеза систем послепродажного обслуживания>>, 
<<Определение и анализ показателей экспортного потенциала ИНП при проектировании>>, 
<<Задачи управления поддержкой послепродажного обслуживания>>, а также 
<<Моделирование инвестиционных процессов ИЛП в условиях неравновесных финансовых 
рынков>>. 
   
      В конце каждой главы приведены выводы и даны вопросы и задания для 
самоконтроля. В~приложениях содержатся основные определения по программам работ по 
анализу ИЛП, логистическим базам данных и компьютерным решениям, эквивалентной статистической 
линеаризации нелинейных преобразований ИЛП, справочный материал, а также развернутые 
уравнения для вероятностных характеристик.


      \def\leftkol{РЕЦЕНЗИИ}

\def\rightkol{РЕЦЕНЗИИ} 

      
      Книга заинтересует широкий круг специалистов и может быть использована научными 
проектными организациями в сфере промышленного производства ИНП. Большое количество 
иллюстраций, примеров и вопросов, обращенных к читателю, позволяет использовать книгу 
также в качестве учебного пособия для студентов и аспирантов машиностроительных, 
транспортных и~других специальностей, а также для самостоятельного изучения. 
{%\looseness=-1

}

Книга 
представляет несомненный интерес для специалистов и студентов в области прикладной 
математики и информатики.
    

}

}
\end{multicols}

%\newpage

%\end{document}

%\include{obchak}

%\end{document}


\def\stat{authorsrus}
{%\hrule\par
%\vskip 7pt % 7pt
\raggedleft\Large \bf%\baselineskip=3.2ex
О\,Б\ \ А\,В\,Т\,О\,Р\,А\,Х \vskip 17pt
    \hrule
    \par
\vskip 21pt plus 8pt minus 4pt }


\def\tit{\ }

\def\aut{\ }

\def\auf{\ }

\def\leftkol{\ } % ENGLISH ABSTRACTS}

\def\rightkol{ОБ АВТОРАХ} %ENGLISH ABSTRACTS}

\titele{\tit}{\aut}{\auf}{\leftkol}{\rightkol}
      
            \label{st\stat}



\vspace*{24pt}

\begin{multicols}{2}




\noindent
\textbf{Архипов Олег Петрович} (р.\ 1948)~---
кандидат технических наук, директор Орловского филиала Института проб\-лем информатики
Российской академии наук
%302025, г.Орел, Московское шоссе, д.137

\vspace*{3pt}

\noindent
\textbf{Бирюкова Татьяна Константиновна} (р.\ 1968)~---
кандидат фи\-зи\-ко-ма\-те\-ма\-ти\-че\-ских наук, старший научный сотрудник Института проб\-лем информатики
Российской академии наук

\vspace*{3pt}

\noindent 
\textbf{Бобков  Сергей Геннадьевич} (р.\ 1955)~---
доктор технических наук,  заведующий отделением На\-уч\-но-ис\-сле\-до\-ва\-тель\-ско\-го 
института системных исследований Российской академии наук
%117218, Москва, Нахимовский просп., 36, к.1 

\vspace*{3pt}

\noindent \textbf{Васильев Николай Семенович} (р.\ 1952)~--- доктор 
фи\-зи\-ко-ма\-те\-ма\-ти\-че\-ских наук, профессор, 
МГТУ им.\ Н.\,Э.~Баумана 
%, Москва 105005, 2-я Бауманская ул., д.~5,

\vspace*{3pt}

\noindent
\textbf{Гершкович Максим Михайлович} (р.\ 1968)~---
старший научный сотрудник Института проб\-лем информатики
Российской академии наук

\vspace*{3pt}

\noindent 
\textbf{Дьяченко Юрий Георгиевич} (р.\ 1958)~--- кандидат технических наук, 
старший научный сотрудник Института проб\-лем информатики
Российской академии наук

\vspace*{3pt}

\noindent 
\textbf{Ерошенко Александр Андреевич} (р.\ 1989)~--- аспирант кафедры 
математической статистики факультета вычисли\-тельной математики и кибернетики 
Московского государственного университета им.\ М.\,В.~Ломоносова
%119991, Москва ГСП-1, Ленинские горы, д.\ 1, стр. 52

\vspace*{3pt}
 
\noindent 
\textbf{Захаров Виктор Николаевич} (р.\ 1948)~--- 
доктор технических наук, доцент, ученый секретарь Института проб\-лем информатики
Российской академии наук

\vspace*{3pt}

\noindent
\textbf{Зейфман Александр Израилевич} (р.\ 1954)~---
доктор фи\-зи\-ко-ма\-те\-ма\-ти\-че\-ских наук, профессор, 
заведующий кафедрой Вологодского государственного университета; 
старший научный сотрудник Института проб\-лем информатики
Российской академии наук; главный научный сотрудник ИСЭРТ Российской академии наук

\vspace*{3pt}

\noindent
\textbf{Зыкин Сергей Владимирович} (р.\ 1959)~--- 
доктор технических наук, профессор, заведующий лабораторией Института математики 
им.\ С.\,Л.~Соболева Сибирского отделения Российской академии наук, Новосибирск 
%630090, пр.\ ак.\ Коптюга, 4 

\vspace*{4pt}

\noindent
\textbf{Киреев Владимир Иванович} (р.\ 1938)~---
доктор фи\-зи\-ко-ма\-те\-ма\-ти\-че\-ских наук, профессор Московского 
государственного горного университета
%Адрес: Россия, 119991, г. Москва, Ленинский проспект, д. 6

%\columnbreak

\vspace*{4pt}

\noindent
\textbf{Козеренко Елена Борисовна} (р.\ 1959)~---
кандидат филологических наук, заведующая лабораторией Института проб\-лем информатики
Российской академии наук

\vspace*{4pt}

\noindent
\textbf{Королев Виктор Юрьевич} (р.\ 1954)~--- доктор
фи\-зи\-ко-ма\-те\-ма\-ти\-че\-ских наук, профессор кафедры математической 
статистики факультета вычисли\-тельной математики и кибернетики 
Московского государственного университета; 
ведущий научный сотрудник Института проб\-лем информатики
Российской академии наук

\vspace*{4pt}

\noindent
\textbf{Коротышева Анна Владимировна} (р.\ 1988)~---
старший преподаватель Вологодского государственного университета

\vspace*{4pt}

\noindent 
\textbf{Кун Де Турк} (р.\ 1981)~--- научный сотрудник 
исследовательской группы SMACS факультета телекоммуникаций и обработки информации
Университета Гента, Бельгия
%В-9000 Гент, Бельгия

\vspace*{4pt}

\noindent
\textbf{Лупенцов Олег Сергеевич} (р.\ 1986)~---
аспирант Омского государственного института сервиса
%Омск 644043, ул.\ Певцова 13

\vspace*{4pt}

\noindent
\textbf{Лучко Олег Николаевич} (р.\ 1961)~---
кандидат педагогических наук, профессор, заведующий кафедрой 
Омского государственного института сервиса
%Омск 644043, ул.\ Певцова 13

\vspace*{4pt}

\noindent
\textbf{Малашенко Юрий Евгеньевич} (р.\ 1946)~---
доктор фи\-зи\-ко-ма\-те\-ма\-ти\-че\-ских наук, заведующий сектором 
Вычислительного центра им.\ А.\,А.~Дородницына Российской академии наук
%Адрес: 119333, Москва, ул. Вавилова, 40,

\vspace*{4pt}

\noindent
\textbf{Маньяков Юрий Анатольевич} (р.\ 1984)~---
кандидат технических наук, научный сотрудник Орловского филиала Института проб\-лем информатики
Российской академии наук
%302025, г.Орел, Московское шоссе, д.137

\vspace*{4pt}

\noindent
\textbf{Маренко Валентина Афанасьевна} (р.\ 1951)~---
кандидат технических наук, доцент, старший научный сотрудник 
Института математики им.\ С.\,Л.~Соболева Сибирского отделения Российской академии наук
%Новосибирск 630090, пр. ак. Коптюга, 4 

\vspace*{3pt}

\noindent 
\textbf{Морозов Евсей Викторович} (р.\ 1947)~--- доктор 
фи\-зи\-ко-ма\-те\-ма\-ти\-че\-ских, профессор, ведущий научный сотрудник 
Института прикладных математических исследований Карельского научного центра Российской
академии наук; 
%%185910 Россия, Республика Карелия, г.\ Петрозаводск, ул.\ Пушкинская, 11
профессор Петрозаводского государственного университета, Петрозаводск
%185910 Россия, Республика Карелия, г.\ Петрозаводск, пр.\ Ленина, 33

%\pagebreak

\vspace*{3pt}

\noindent
\textbf{Назарова Ирина Александровна} (р.\ 1966)~---
кандидат фи\-зи\-ко-ма\-те\-ма\-ти\-че\-ских наук, 
научный сотрудник Вычислительного центра им.\ А.\,А.~Дородницына Российской академии наук 
%Адрес: 119333, Москва, ул. Вавилова, 40

\vspace*{3pt}

\noindent
\textbf{Павлов Игорь Валерианович} (р.\ 1945)~--- 
доктор фи\-зи\-ко-ма\-те\-ма\-ти\-че\-ских наук, профессор МГТУ им.\ Н.\,Э.~Баумана 
%Москва 105005, 2-я Бауманская ул., д.~5 

%\pagebreak

\vspace*{3pt}

\noindent 
\textbf{Потахина Любовь Викторовна} (р.\ 1989)~--- аспирантка
Института прикладных математических исследований Карельского научного центра
Российской академии наук; 
%%185910 Россия, Республика Карелия, г.\ Петрозаводск, ул.\ Пушкинская, 11
инженер Петрозаводского государственного университета, Петрозаводск
%185910 Россия, Республика Карелия, г.\ Петрозаводск, пр.\ Ленина, 33

\vspace*{3pt}

\noindent 
\textbf{Рождественский Юрий Владимирович} (р.\ 1952)~--- 
кандидат технических наук, заведующий сектором Института проб\-лем информатики
Российской академии наук

\vspace*{3pt}

\noindent 
\textbf{Синицын Игорь Николаевич} (р.\ 1940)~--- доктор технических наук,
профессор, заслуженный деятель\linebreak\vspace*{-12pt}

\columnbreak

\noindent
 науки РФ, заведующий отделом Института проб\-лем информатики
Российской академии наук

\vspace*{7pt}


\noindent
\textbf{Сиротинин Денис Олегович} (р.\ 1984)~---
кандидат технических наук, научный сотрудник Орловского филиала Института проб\-лем информатики
Российской академии наук
%302025, г.Орел, Московское шоссе, д.137

\vspace*{7pt}

%\columnbreak

\noindent 
\textbf{Соколов  Игорь Анатольевич} (р.\ 1954)~--- академик (действительный член) Российской 
академии наук, доктор технических наук, директор Института проб\-лем информатики
Российской академии наук

\vspace*{7pt}

\noindent
\textbf{Степченков Юрий Афанасьевич} (р.\ 1951)~---
кандидат технических наук, заведующий отделом Института проб\-лем информатики
Российской академии наук

\vspace*{7pt}

\noindent
\textbf{Сурков Алексей Викторович} (р.\ 1978)~--- 
старший научный сотрудник На\-уч\-но-ис\-сле\-до\-ва\-тель\-ско\-го 
института системных исследований Российской академии наук
%117218, Москва, Нахимовский просп., 36, к.1 

\vspace*{7pt}

\noindent 
\textbf{Шестаков Олег Владимирович} (р.\ 1976)~--- доктор 
фи\-зи\-ко-ма\-те\-ма\-ти\-че\-ских, доцент кафедры математической статистики 
факультета вычисли\-тельной математики и кибернетики Московского 
государственного университета им.\ М.\,В.~Ломоносова; 
%119991, Москва ГСП-1, Ленинские горы, д.\ 1, стр. 52
старший научный сотрудник Института проб\-лем информатики
Российской академии наук
%, Москва 119333, ул. Вавилова, д.~44, корп.~2

\vspace*{7pt}

\noindent 
\textbf{Шоргин Сергей Яковлевич} (р.\ 1952.)~--- доктор
фи\-зи\-ко-ма\-те\-ма\-ти\-че\-ских наук, профессор, заместитель директора Института 
проб\-лем информатики Российской академии наук





%%%%%%%%%%%%%%%%%%%%%%%%%%%%%%%%%%%%%%%%%%%%%%%%%%%%%%%%%%%%%%%%%%%%%%%%%%%%%%%




%\def\rightkol{ОБ АВТОРАХ}
%\def\leftkol{ОБ АВТОРАХ}

 \label{end\stat}





%\def\leftfootline{\small{\textbf{\thepage}
%\hfill ИНФОРМАТИКА И ЕЁ ПРИМЕНЕНИЯ\ \ \ том~7\ \ \ выпуск~1\ \ \ 2013}
%}%
% \def\rightfootline{\small{ИНФОРМАТИКА И ЕЁ ПРИМЕНЕНИЯ\ \ \ том~7\ \ \ выпуск~1\ \ \ 2013
%\hfill \textbf{\thepage}}}


%\thispagestyle{myheadings}



\end{multicols}

\newpage



\def\stat{cont}
{%\hrule\par
%\vskip 7pt % 7pt
\raggedleft\Large \bf%\baselineskip=3.2ex
А\,В\,Т\,О\,Р\,С\,К\,И\,Й\ \ У\,К\,А\,З\,А\,Т\,Е\,Л\,Ь\ \ З\,А\ \ 2\,0\,1\,0 г. \vskip 17pt
    \hrule
    \par
\vskip 21pt plus 6pt minus 3pt }

\label{st\stat}

\def\tit{\ }

\def\aut{\ }
\def\auf{\ }

\def\leftkol{\ } % ENGLISH ABSTRACTS}

\def\rightkol{\ } %АВТОРСКИЙ УКАЗАТЕЛЬ ЗА 2010 г.} %ENGLISH ABSTRACTS}

\titele{\tit}{\aut}{\auf}{\leftkol}{\rightkol}

\vspace*{-12pt}

{\tabcolsep=3pt
\begin{tabular}{p{388pt}rr}
&\textbf{Выпуск} & \textbf{Стр.}\\[6pt]
\hangindent=23pt\noindent\textbf{Арутюнян~А.\,Р.} Моделирование влияния деформаций отпечатков пальцев на 
точность\linebreak
\vspace*{-12pt}\\
\hspace*{23pt}дактилоскопической идентификации$\dotfill$&1&51\\
\hangindent=23pt\noindent\textbf{Архипов~О.\,П., Зыкова~З.\,П.} Интеграция гетерогенной информации о цветных 
пикселях\linebreak
\vspace*{-12pt}\\
\hspace*{23pt}и их цветовосприятии$\dotfill$&4&15\\
\hangindent=23pt\noindent\textbf{Баранов~С.\,И., Френкель~С.\,Л., Захаров~В.\,Н.} Полуформальная верификация 
цифрового устройства с конвейером, основанная на использовании алгоритмических машин\linebreak
\vspace*{-12pt}\\
\hspace*{23pt}состояния$\dotfill$&4&49\\
\textbf{Бекетова~И.\,В.} см.~Каратеев~С.\,Л.&&\\
\textbf{Белоусов~В.\,В.} см.~Синицын~И.\,Н.&&\\
\hangindent=23pt\noindent\textbf{Бенинг~В.\,Е., Королев~Р.\,А.} О предельном поведении мощностей критериев в 
случае\linebreak
\vspace*{-12pt}\\
\hspace*{23pt}распределения Лапласа$\dotfill$&2&63\\
\hangindent=23pt\noindent\textbf{Бенинг~В.\,Е., Сипина~А.\,В.} Асимптотическое разложение для мощности 
критерия,\linebreak
\vspace*{-12pt}\\
\hspace*{23pt}основанного на выборочной медиане, в случае распределения Лапласа$\dotfill$&1&18\\
\textbf{Бондаренко~А.\,В.} см.~Каратеев~С.\,Л.&&\\
\hangindent=23pt\noindent\textbf{Бородина~А.\,В., Морозов~Е.\,В.} Об оценивании асимптотики вероятности 
большого\linebreak
\vspace*{-12pt}\\
\hspace*{23pt}уклонения стационарной регенеративной очереди с одним прибором$\dotfill$&3&29\\
\hangindent=23pt\noindent\textbf{Бунтман~Н.\,В., Минель~Ж.-Л., Ле~Пезан~Д., Зацман~И.\,М.} Типология и 
компьютерное\linebreak
\vspace*{-12pt}\\
\hspace*{23pt}моделирование трудностей перевода$\dotfill$&3&77\\
\textbf{Визильтер~Ю.\,В.} см.~Каратеев~С.\,Л.&&\\
\hangindent=23pt\noindent\textbf{Гавриленко~С.\,В.} Оценки скорости сходимости распределений случайных сумм с 
безгранично делимыми индексами к нормальному закону$\dotfill$&4&81\\
\hangindent=23pt\noindent\textbf{Григорьева~М.\,Е., Шевцова~И.\,Г.} Уточнение неравенства 
Каца--Берри--Эссеена$\dotfill$&2&75\\
\hangindent=23pt\noindent\textbf{Грушо~А.\,А., Грушо~Н.\,А., Тимонина~Е.\,Е.} Поиск конфликтов в политиках 
безопасности: модель случайных графов$\dotfill$&3&38\\
\textbf{Грушо~Н.\,А.} см.~Грушо~А.\,А.&&\\
\hangindent=23pt\noindent\textbf{Гудков~В.\,Ю.} Математические модели изображения отпечатка пальца на основе 
описания линий$\dotfill$&1&58\\
\textbf{Гуртов~А.\,В.} см.~Лукьяненко~А.\,С.&&\\
\textbf{Желтов~С.\,Ю.} см.~Каратеев~С.\,Л.&&\\
\hangindent=23pt\noindent\textbf{Захаров~А.\,А., Серебряков~В.\,А.} Система управления электронной библиотекой 
LibMeta$\dotfill$&4&2\\
\textbf{Захаров~В.\,Н.} см.~Баранов~С.\,И.&&\\
\textbf{Захарова~Т.\,В.} см.~Матвеева~С.\,С.&&\\
\hangindent=23pt\noindent\textbf{Зацаринный~А.\,А., Чупраков~К.\,Г.} Некоторые аспекты выбора технологии для 
постро-\linebreak
\vspace*{-12pt}\\
\hspace*{23pt}ения систем отображения информации ситуационного центра$\dotfill$&3&59\\
\textbf{Зацман~И.\,М.} см.~Бунтман~Н.\,В.&&\\
\hangindent=23pt\noindent\textbf{Зейфман~А.\,И., Коротышева~А.\,В., Сатин~Я.\,А., Шоргин~С.\,Я.} Об 
устойчивости нестаци-\linebreak
\vspace*{-12pt}\\
\hspace*{23pt}онарных систем обслуживания с катастрофами$\dotfill$&3&9\\
\textbf{Зыкова~З.\,П.} см.~Архипов~О.\,П.&&\\
\hangindent=23pt\noindent\textbf{Илюшин~Г.\,Я., Соколов~И.\,А.} Организация управляемого доступа пользователей 
к\linebreak
\vspace*{-12pt}\\
\hspace*{23pt}разнородным ведомственным информационным ресурсам$\dotfill$&1&24\\
\hangindent=23pt\noindent\textbf{Кавагучи~Ю., Ульянов~В.\,В., Фуджикоши~Я.} Приближения для статистик, 
описывающих\linebreak
\vspace*{-12pt}\\
\hspace*{23pt}геометрические свойства данных большой размерности, с оценками 
ошибок$\dotfill$&1&12\\
\hangindent=23pt\noindent\textbf{Каратеев~С.\,Л., Бекетова~И.\,В., Ососков~М.\,В., Князь~В.\,А., 
Визильтер~Ю.\,В., Бондаренко~А.\,В., Желтов~С.\,Ю.} Автоматизированный контроль 
качества цифровых\linebreak
\vspace*{-12pt}\\
\hspace*{23pt}изображений для персональных документов$\dotfill$&1&65\\
\end{tabular}
}

\pagebreak

\def\leftkol{АВТОРСКИЙ УКАЗАТЕЛЬ ЗА 2010 г.} % ENGLISH ABSTRACTS}

\def\rightkol{АВТОРСКИЙ УКАЗАТЕЛЬ ЗА 2010 г.} %ENGLISH ABSTRACTS}

{\tabcolsep=3pt
\begin{tabular}{p{388pt}rr}
&\textbf{Выпуск} & \textbf{Стр.}\\[3pt]
\hangindent=23pt\noindent\textbf{Козеренко~Е.\,Б.} Лингвистические фильтры в статистических моделях машинного\linebreak
\vspace*{-12pt}\\
\hspace*{23pt}перевода$\dotfill$&2&83\\
\hangindent=23pt\noindent\textbf{Козеренко~Е.\,Б., Кузнецов~И.\,П.} Когнитивно-лингвистические представления в 
систе-\linebreak
\vspace*{-12pt}\\
\hspace*{23pt}мах обработки текстов$\dotfill$&3&69\\
\textbf{Князь~В.\,А.} см.~Каратеев~С.\,Л.&&\\
\hangindent=23pt\noindent\textbf{Колесников~А.\,В., Солдатов~С.\,А.} Алгоритм координации для гибридной 
интеллектуальной системы решения сложной задачи оперативно-производственного\linebreak
\vspace*{-12pt}\\
\hspace*{23pt}планирования$\dotfill$&4&61\\
\hangindent=23pt\noindent\textbf{Коновалов~М.\,Г.} О планировании потоков в системах вычислительных 
ресурсов$\dotfill$&2&3\\
\textbf{Конушин~А.\,С.} см.~Конушин~В.\,С.&&\\
\hangindent=23pt\noindent\textbf{Конушин~В.\,С., Кривовязь~Г.\,Р., Конушин~А.\,С.} Алгоритм распознавания людей 
в видео-\linebreak
\vspace*{-12pt}\\
\hspace*{23pt}последовательности по одежде$\dotfill$&1&74\\
\textbf{Корепанов~Э.\, Р.} см.~Синицын~И.\,Н.&&\\
\textbf{Королев~В.\,Ю.} см.~Соколов~И.\,А.&&\\
\textbf{Королев~Р.\,А.} см.~Бенинг~В.\,Е.&&\\
\textbf{Коротышева~А.\,В.} см.~Зейфман~А.\,И.&&\\
\hangindent=23pt\noindent\textbf{Кривенко~М.\,П.} Непараметрическое оценивание элементов байесовского 
клас\-си-\linebreak
\vspace*{-12pt}\\
\hspace*{23pt}фикатора$\dotfill$&2&13\\
\textbf{Кривовязь~Г.\,Р.} см.~Конушин~В.\,С.&&\\
\textbf{Крылов~А.\,С.} см.~Павельева~Е.\,А.&&\\
\hangindent=23pt\noindent\textbf{Крылов~В.\,А.} Моделирование и классификация многоканальных дистанционных\linebreak
\vspace*{-12pt}\\
\hspace*{23pt}изображений с использованием копул$\dotfill$&4&34\\
\hangindent=23pt\noindent\textbf{Крючин~О.\,В.} Разработка параллельных эвристических алгоритмов подбора 
весовых\linebreak
\vspace*{-12pt}\\
\hspace*{23pt}коэффициентов искусственной нейтронной сети$\dotfill$&2&53\\
\hangindent=23pt\noindent\textbf{Кудрявцев~А.\,А., Шоргин~С.\,Я.} Байесовские модели массового обслуживания и 
надеж-\linebreak
\vspace*{-12pt}\\
\hspace*{23pt}ности: характеристики среднего числа заявок в системе $M\vert M \vert 1\vert 
\infty$$\dotfill$&3&16\\
\hangindent=23pt\noindent\textbf{Кузнецов~А.\,А.} Связь между временными и структурно-топологическими 
характери-\linebreak
\vspace*{-12pt}\\
\hspace*{23pt}стиками диаграмм ритма сердца здоровых людей$\dotfill$&4&39\\
\textbf{Кузнецов~И.\,П.} см.~Козеренко~Е.\,Б.&&\\
\textbf{Ле~Пезан~Д.} см.~Бунтман~Н.\,В.&&\\
\hangindent=23pt\noindent\textbf{Лукьяненко~А.\,С., Морозов~Е.\,В., Гуртов~А.\,В.} Анализ сетевого протокола с общей 
функ-\linebreak
\vspace*{-12pt}\\
\hspace*{23pt}цией расширения окна передачи сообщения при конфликтах$\dotfill$&2&46\\
\hangindent=23pt\noindent\textbf{Лямин~О.\,О.} О предельном поведении мощностей критериев в случае обобщенного\linebreak
\vspace*{-12pt}\\
\hspace*{23pt}распределения Лапласа$\dotfill$&3&47\\
\hangindent=23pt\noindent\textbf{Маркин~А.\,В., Шестаков~О.\,В.} Асимптотики оценки риска при пороговой 
обработке\linebreak
\vspace*{-12pt}\\
\hspace*{23pt}вейвлет-вейглет коэффициентов в задаче томографии$\dotfill$&2&36\\
\hangindent=23pt\noindent\textbf{Матвеева~С.\,С., Захарова~Т.\,В.} Сети массового обслуживания с наименьшей 
длиной\linebreak
\vspace*{-12pt}\\
\hspace*{23pt}очереди$\dotfill$&3&22\\
\hangindent=23pt\noindent\textbf{Матюшенко~С.\,И.} Стационарные характеристики двухканальной системы 
обслужива-\linebreak
\vspace*{-12pt}\\
\hspace*{23pt}ния с переупорядочиванием заявок и распределениями фазового типа$\dotfill$&4&68\\
\textbf{Минель~Ж.-Л.} см.~Бунтман~Н.\,В.&&\\
\textbf{Морозов~Е.\,В.} см.~Бородина~А.\,В.&&\\
\textbf{Морозов~Е.\,В.} см.~Лукьяненко~А.\,С.&&\\
\textbf{Ососков~М.\,В.} см.~Каратеев~С.\,Л.&&\\
\hangindent=23pt\noindent\textbf{Павельева~Е.\,А., Крылов~А.\,С.} Поиск и анализ ключевых точек радужной 
оболочки\linebreak
\vspace*{-12pt}\\
\hspace*{23pt}глаза методом преобразования Эрмита$\dotfill$&1&79\\
\textbf{Печинкин~А.\,В.} см.~Френкель~С.\,Л.,&&\\
\hangindent=23pt\noindent\textbf{Протасов~В.\,И.} Составление субъективного портрета с использованием 
эволюционно-\linebreak
\vspace*{-12pt}\\
\hspace*{23pt}го морфинга и квалиметрия метода$\dotfill$&1&83\\
\hangindent=23pt\noindent\textbf{Рудаков~К.\,В., Торшин~И.\,Ю.} Вопросы разрешимости задачи распознавания 
вторичной\linebreak
\vspace*{-12pt}\\
\hspace*{23pt}структуры белка$\dotfill$&2&25\\
\textbf{Сатин~Я.\,А.} см.~Зейфман~А.\,И.&&\\
\hangindent=23pt\noindent\textbf{Сейфуль-Мулюков~Р.\,Б.} Нефть как носитель информации о своем 
происхождении,\linebreak
\vspace*{-12pt}\\
\hspace*{23pt}структуре и эволюции$\dotfill$&1&41\\
\end{tabular}
}

{\tabcolsep=3pt
\begin{tabular}{p{388pt}rr}
&\textbf{Выпуск} & \textbf{Стр.}\\[6pt]
\textbf{Семендяев~Н.\,Н.} см.~Синицын~И.\,Н.&&\\
\textbf{Серебряков~В.\,А.} см.~Захаров~А.\,А.&&\\
\textbf{Синицын~В.\,И.} см.~Синицын~И.\,Н.&&\\
\hangindent=23pt\noindent\textbf{Синицын~И.\,Н., Синицын~В.\,И., Корепанов~Э.\, Р., Белоусов~В.\,В., 
Семендяев~Н.\,Н.} Оперативное построение информационных моделей движения полюса 
Земли\linebreak
\vspace*{-12pt}\\
\hspace*{23pt}методами линейных и линеаризованных фильтров$\dotfill$&1&2\\
\textbf{Сипина~А.\,В.} см.~Бенинг~В.\,Е.&&\\
\hangindent=23pt\noindent\textbf{Соколов~И.\,А.} О работах заслуженного деятеля науки Российской Федерации 
И.\,Н.~Синицына в области информационных технологий и автоматизации (к 70-летию\linebreak
\vspace*{-12pt}\\
\hspace*{23pt}со дня рождения)$\dotfill$&3&84\\
\textbf{Соколов~И.\,А.} см.~Илюшин~Г.\,Я.&&\\
\hangindent=23pt\noindent\textbf{Соколов~И.\,А., Королев~В.\,Ю.} Предисловие$\dotfill$&2&2\\
\textbf{Солдатов~С.\,А.} см.~Колесников~А.\,В.&&\\
\hangindent=23pt\noindent\textbf{Степанов~С.\,Ю.} Использование координатного метода фрагментации 
коммутаторной\linebreak
\vspace*{-12pt}\\
\hspace*{23pt}нейронной сети для сокращения трафика$\dotfill$&2&57\\
\textbf{Тимонина~Е.\,Е.} см.~Грушо~А.\,А.&&\\
\textbf{Торшин~И.\,Ю.} см.~Рудаков~К.\,В.&&\\
\textbf{Ульянов~В.\,В.} см.~Кавагучи~Ю.&&\\
\textbf{Фазекаш~И.} см.~Чупрунов~А.\,Н.&&\\
\textbf{Френкель~С.\,Л.} см.~Баранов~С.\,И.&&\\
\hangindent=23pt\noindent\textbf{Френкель~С.\,Л., Печинкин~А.\,В.} Оценка времени самовосстановления в 
цифровых\linebreak
\vspace*{-12pt}\\
\hspace*{23pt}системах после сбоев, вызываемых переходными помехами$\dotfill$&3&2\\
\textbf{Фуджикоши~Я.} см.~Кавагучи~Ю.&&\\
\hangindent=23pt\noindent\textbf{Цискаридзе~А.\,К.} Математическая модель и метод восстановления позы человека 
по\linebreak
\vspace*{-12pt}\\
\hspace*{23pt}стереопаре силуэтных изображений$\dotfill$&4&27\\
\hangindent=23pt\noindent\textbf{Чупраков~К.\,Г.} К вопросу о размещении коллективных средств отображения в 
ситуа-\linebreak
\vspace*{-12pt}\\
\hspace*{23pt}ционном зале с заданными параметрами$\dotfill$&4&89\\
\textbf{Чупраков~К.\,Г.} см.~Зацаринный~А.\,А.&&\\
\hangindent=23pt\noindent\textbf{Чупрунов~А.\,Н., Фазекаш~И.} Законы повторного логарифма для числа 
безошибочных\linebreak
\vspace*{-12pt}\\
\hspace*{23pt}блоков при помехоустойчивом кодировании$\dotfill$&3&42\\
\textbf{Шевцова~И.\,Г.} см.~Григорьева~М.\,Е.&&\\
\hangindent=23pt\noindent\textbf{Шестаков~О.\,В.} Аппроксимация распределения оценки риска пороговой 
обработки вейвлет-коэффициентов нормальным распределением при использовании 
выбо-\linebreak
\vspace*{-12pt}\\
\hspace*{23pt}рочной дисперсии$\dotfill$&4&73\\
\textbf{Шестаков~О.\,В.} см.~Маркин~А.\,В.&&\\
\textbf{Шоргин~С.\,Я.} см.~Зейфман~А.\,И.&&\\
\textbf{Шоргин~С.\,Я.} см.~Кудрявцев~А.\,А.&&\\
\end{tabular}
}

%\thispagestyle{myheadings}
\def\leftfootline{\small{\textbf{\thepage}
\hfill ИНФОРМАТИКА И ЕЁ ПРИМЕНЕНИЯ\ \ \ том~4\ \ \ выпуск~4\ \ \ 2010}
}%
 \def\rightfootline{\small{ИНФОРМАТИКА И ЕЁ ПРИМЕНЕНИЯ\ \ \ том~4\ \ \ выпуск~4\ \ \ 2010
 \hfill \textbf{\thepage}}}
 \label{end\stat}


%Том 10 Выпуск 1-4 Год 2016

\def\stat{cont-e}
{%\hrule\par
%\vskip 7pt % 7pt
\raggedleft\Large \bf%\baselineskip=3.2ex
2\,0\,1\,6\ \ A\,U\,T\,H\,O\,R\ \ I\,N\,D\,E\,X \vskip 17pt
 \hrule
 \par
\vskip 21pt plus 6pt minus 3pt }

\label{st\stat}

\def\tit{\ }

\def\aut{\ }
\def\auf{\ }

\def\leftkol{\ } %2016 AUTHOR INDEX} % ENGLISH ABSTRACTS}

\def\rightkol{\ } %2016 AUTHOR INDEX} %ENGLISH ABSTRACTS}

\titele{\tit}{\aut}{\auf}{\leftkol}{\rightkol}

\def\leftfootline{\small{\textbf{\thepage}
\hfill INFORMATIKA I EE PRIMENENIYA~--- INFORMATICS AND APPLICATIONS\ \ \ 2016\
\ \ volume~10\ \ \ issue\ 4}
}%
 \def\rightfootline{\small{INFORMATIKA I EE PRIMENENIYA~--- INFORMATICS AND APPLICATIONS\ \ \ 2016\ \ \ volume~10\ \ \ issue\ 4
\hfill \textbf{\thepage}}}

\vspace*{-12pt}
\vspace*{-18pt}

{\tabcolsep=2.8pt
\begin{tabular}{p{382pt}cc}
&\textbf{Issue} & \textbf{Page}\\[6pt]
\Avtors{Agalarov~M.\,Ya.} see~Agalarov~Ya.\,M.&&\\
\Avtors{Agalarov~Ya.\,M., Agalarov~M.\,Ya., and
Shorgin~V.\,S.} About the optimal threshold of queue\linebreak
\\[-12pt]
\hspace*{23pt}length in a~particular problem of profit maximization
in the $M/G/1$ queuing system&2&70--79\\
\Avtors{Alexeyevsky~D.\,A.} BioNLP ontology extraction from 
a~restricted language corpus with\linebreak
\\[-12pt]
\hspace*{23pt}context-free grammars&1&119--128\\
\Avtors{Andreev~S.\,D.} see~Gaidamaka~Yu.\,V.&&\\
\Avtors{Andreev~S.\,D.} see~Ometov~A.\,Ya.&&\\
\Avtors{Arkhipov~O.\,P., Arkhipov~P.\,O., and Sidorkin~I.\,I.} The
option to create a~local coordinate\linebreak
\\[-12pt]
\hspace*{23pt}system for synchronization of selected images&3&91--97\\
\Avtors{Arkhipov~P.\,O.} see~Arkhipov~O.\,P.&&\\
\Avtors{Belousov~V.\,V.} see~Shnurkov~P.\,V.&&\\
\Avtors{Belousov~V.\,V.} see~Shnurkov~P.\,V.&&\\
\Avtors{Bening~V.\,E.} Calculation of~the~asymptotic deficiency
of~some statistical procedures based\linebreak
\\[-12pt]
\hspace*{23pt}on~samples with~random sizes&4&34--45\\
\Avtors{Borisov~A.\,V., Bosov~A.\,V., and Miller~G.\,B.} Modeling and
monitoring of VoIP connection&2&\hphantom{1}2--13\\
\Avtors{Bosov~A.\,V.} see~Borisov~A.\,V.&&\\
\Avtors{Briukhov~D.\,O.} see~Stupnikov~S.\,A.&&\\
\Avtors{Callaos~N.\,K.\ and Seyful-Mulyukov~R.\,B.} Complexity and
its information content&1&129--139\\
\Avtors{Chertok~A.\,V., Kadaner~A.\,I., Khazeeva~G.\,T., and
Sokolov~I.\,A.} Regime switching detection\linebreak
\\[-12pt]
\hspace*{23pt}for~the~Levy driven
Ornstein--Uhlenbeck process using CUSUM methods&4&46--56\\
\Avtors{Chichagov~V.\,V.} Asymptotic expansions of mean absolute
error of uniformly minimum variance unbiased and maximum likelihood
estimators on the one-parameter exponential\linebreak
\\[-12pt]
\hspace*{23pt}family model of lattice distributions&3&66--76\\
\Avtors{Danishevsky~V.\,I.} see~Kolesnikov A.\,V.&&\\
\Avtors{Fazliev~A.\,Z.} see~Kalinichenko~L.\,A.&&\\
\Avtors{Fedoseev~A.\,A.} What is behind the concept of ``knowledge in
small packages''&3&105--110\\
\Avtors{Gaidamaka~Yu.\,V., Andreev~S.\,D., Sopin~E.\,S.,
Samouylov~K.\,E., and Shorgin~S.\,Ya.} Interference analysis
of~the~device-to-device communications model with~regard to~a~signal\linebreak
\\[-12pt]
\hspace*{23pt}propagation environment&4&\hphantom{1}2--10\\
\Avtors{Gasilov~A.\,V.} see~Yakovlev~O.\,A.&&\\
\Avtors{Goncharov~A.\,V.\ and Strijov~V.\,V.} Metric time series
classification using weighted dynamic\linebreak
\\[-12pt]
\hspace*{23pt}warping relative to centroids of classes&2&36--47\\
\Avtors{Gordov~E.\,P.} see~Kalinichenko~L.\,A.&&\\
\Avtors{Gorshenin~A.\,K.} Concept of online service for stochastic
modeling of real processes&1&72--81\\
\Avtors{Gorshenin~A.\,K.} see~Shnurkov~P.\,V.&&\\
\Avtors{Gorshenin~A.\,K.} see~Shnurkov~P.\,V.&&\\
\Avtors{Grusho~A.\,A., Grusho~N.\,A., Zabezhailo~M.\,I., and
Timonina~E.\,E.} Integration of statistical and\linebreak
\\[-12pt]
\hspace*{23pt}deterministic methods for
analysis of information security&3&2--8\\
\Avtors{Grusho~A.\,A., Zabezhailo~M.\,I., and Zatsarinny~A.\,A.} On
the advanced procedure to reduce\linebreak
\\[-12pt]
\hspace*{23pt}calculation of Galois closures&4&\hphantom{1}96--104\\
\Avtors{Grusho~N.\,A.} see~Grusho~A.\,A.&&\\
\Avtors{Havanskov~V.\,A.} see~Minin~V.\,A.&&\\
\Avtors{Inkova~O.\,Yu.} see~Zatsman~I.\,M.&&\\
\Avtors{Isachenko~R.\,V.\ and Strijov~V.\,V.} Metric learning in
multiclass time series classification\linebreak
\\[-12pt]
\hspace*{23pt}problem&2&48--57\\
\end{tabular}
}
\pagebreak

\def\leftfootline{\small{\textbf{\thepage}
\hfill INFORMATIKA I EE PRIMENENIYA~--- INFORMATICS AND APPLICATIONS\ \ \ 2016\
\ \ volume~10\ \ \ issue\ 4}
}%
 \def\rightfootline{\small{INFORMATIKA I EE PRIMENENIYA~---
INFORMATICS AND APPLICATIONS\ \ \ 2016\ \ \ volume~10\ \ \ issue\ 4
\hfill \textbf{\thepage}}}

\def\leftkol{2016 AUTHOR INDEX} % ENGLISH ABSTRACTS}

\def\rightkol{2016 AUTHOR INDEX} %ENGLISH ABSTRACTS}


{\tabcolsep=2.83pt
\begin{tabular}{p{382pt}cc}
&\textbf{Issue} & \textbf{Page}\\[6pt]
\Avtors{Kadaner~A.\,I.} see~Chertok~A.\,V.&&\\[.255pt]
\Avtors{Kalinichenko~L.\,A., Volnova~A.\,A., Gordov~E.\,P.,
Kiselyova~N.\,N., Kovaleva~D.\,A., Malkov~O.\,Yu., Okladnikov~I.\,G.,
Podkolodnyy~N.\,L., Pozanenko~A.\,S., Ponomareva~N.\,V.,
Stupnikov~S.\,A.,} \textbf{and Fazliev~A.\,Z.} Data access challenges for data
intensive\linebreak
\\[-12pt]
\hspace*{23pt}research in Russia&1& 2--22\\[.255pt]
\Avtors{Karasikov~M.\,E.\ and Strijov~V.\,V.} Feature-based
time-series classification&4&121--131\\[.255pt]
\Avtors{Khazeeva~G.\,T.} see~Chertok~A.\,V.&&\\[.255pt]
\Avtors{Khokhlov~Yu.\,S.} Multivariate fractional Levy motion and its
applications&2&\hphantom{1}98--106\\[.255pt]
\Avtors{Kirikov~I.\,A., Kolesnikov~A.\,V., Listopad~S.\,V., and
Rumovskaya~S.\,B.} Fine-grained hybrid\linebreak
\\[-12pt]
\hspace*{23pt}intelligent systems. Part 2:
Bidirectional hybridization&1&\hphantom{1}96--105\\[.255pt]
\Avtors{Kirikov~I.\,A., Kolesnikov~A.\,V., Listopad~S.\,V., and
Rumovskaya~S.\,B.} ``Virtual council''~---\linebreak
\\[-12pt]
\hspace*{23pt}source environment
supporting complex diagnostic decision making&3&81--90\\[.255pt]
\Avtors{Kiselyova~N.\,N.} see~Kalinichenko~L.\,A.&&\\[.255pt]
\Avtors{Kolesnikov A.\,V., Listopad~S.\,V., Rumovskaya~S.\,B., and
Danishevsky~V.\,I.} Informal axiomatic\linebreak
\\[-12pt]
\hspace*{23pt}theory of~the~role visual models&4&114--120\\[.255pt]
\Avtors{Kolesnikov~A.\,V.} see~Kirikov~I.\,A.&&\\[.255pt]
\Avtors{Kolesnikov~A.\,V.} see~Kirikov~I.\,A.&&\\[.255pt]
\Avtors{Kolin~K.\,K.} Humanitarian aspects of information
security&3&111--121\\[.255pt]
\Avtors{Konovalov~M.\,G.\ and Razumchik~R.\,V.} Dispatching
to~two parallel nonobservable queues using\linebreak
\\[-12pt]
\hspace*{23pt}only static
information&4&57--67\\[.255pt]
\Avtors{Korchagin~A.\,Yu.} see~Korolev~V.\,Yu.&&\\[.255pt]
\Avtors{Korchagin~A.\,Yu.} see~Korolev~V.\,Yu.&&\\[.255pt]
\Avtors{Korepanov~E.\,R.} see~Sinitsyn~I.\,N.&&\\[.255pt]
\Avtors{Korepanov~E.\,R.} see~Sinitsyn~I.\,N.&&\\[.255pt]
\Avtors{Korolev~V.\,Yu., Korchagin~A.\,Yu., and Zeifman~A.\,I.} The
Poisson theorem for Bernoulli trials\linebreak
\\[-12pt]
\hspace*{23pt}with~a~random probability
of~success and~a~discrete analog of~the~Weibull distribution&4&11--20\\[.255pt]
\Avtors{Korolev~V.\,Yu., Zeifman~A.\,I., and Korchagin~A.\,Yu.}
Asymmetric Linnik distributions as~limit\linebreak
\\[-12pt]
\hspace*{23pt}laws for~random sums
of~independent random variables with~finite variances&4&21--33\\[.255pt]
\Avtors{Koucheryavy~E.\,A.} see~Ometov~A.\,Ya.&&\\[.255pt]
\Avtors{Kovaleva~D.\,A.} see~Kalinichenko~L.\,A.&&\\[.255pt]
\Avtors{Kovalyov~S.\,P.} Metaprogramming to increase
manufacturability of large-scale software-\linebreak
\\[-12pt]
\hspace*{23pt}intensive systems&1&56--66\\[.255pt]
\Avtors{Krivenko~M.\,P.} Significance tests of feature selection for
classification&3&32--40\\[.255pt]
\Avtors{Kruzhkov~M.\,G.} see~Zalizniak~Anna~A.&&\\[.255pt]
\Avtors{Kruzhkov~M.\,G.} see~Zatsman~I.\,M.&&\\[.255pt]
\Avtors{Kudryavtsev~A.\,A.} Bayesian queueing and reliability models:
\textit{A~priori} distributions with\linebreak
\\[-12pt]
\hspace*{23pt}compact support&1&67--71\\[.255pt]
\Avtors{Kudryavtsev~A.\,A.} Characteristics dependent on the balance
coefficient in Bayesian models\linebreak
\\[-12pt]
\hspace*{23pt}with compact support of \textit{a priori}
distributions&3&77--80\\[.255pt]
\Avtors{Kudryavtsev~A.\,A.\ and Palionnaia~S.\,I.} Bayesian recurrent
model of reliability growth:\linebreak
\\[-12pt]
\hspace*{23pt}Parabolic distribution of parameters&2&80--83\\[.255pt]
\Avtors{Kudryavtsev~A.\,A.\ and Titova~A.\,I.} Bayesian queuing
and~reliability models: Degenerate-\linebreak
\\[-12pt]
\hspace*{23pt}Weibull case&4&68--71\\[.255pt]
\Avtors{Leontyev~N.\,D.\ and Ushakov~V.\,G.} Analysis of a queueing
system with autoregressive arrivals\linebreak
\\[-12pt]
\hspace*{23pt}and nonpreemptive priority&3&15--22\\[.255pt]
\Avtors{Listopad~S.\,V.} see~Kirikov~I.\,A.&&\\[.255pt]
\Avtors{Listopad~S.\,V.} see~Kirikov~I.\,A.&&\\[.255pt]
\Avtors{Listopad~S.\,V.} see~Kolesnikov A.\,V.&&\\[.255pt]
\Avtors{Malkov~O.\,Yu.} see~Kalinichenko~L.\,A.&&\\[.255pt]
\Avtors{Markov~A.\,S., Monakhov~M.\,M., and
Ulyanov~V.\,V.} Generalized Cornish--Fisher expansions\linebreak
\\[-12pt]
\hspace*{23pt}for distributions of statistics based on samples
of random size&2&84--91\\[.255pt]
\Avtors{Melnikov~A.\,K.\ and Ronzhin~A.\,F.} Generalized statistical
method of~text analysis based\linebreak
\\[-12pt]
\hspace*{23pt}on~calculation of~probability distributions
of~statistical values&4&89--95\\
\end{tabular}
}
\pagebreak

\def\leftfootline{\small{\textbf{\thepage}
\hfill INFORMATIKA I EE PRIMENENIYA~--- INFORMATICS AND APPLICATIONS\ \ \ 2016\
\ \ volume~10\ \ \ issue\ 4}
}%
 \def\rightfootline{\small{INFORMATIKA I EE PRIMENENIYA~---
INFORMATICS AND APPLICATIONS\ \ \ 2016\ \ \ volume~10\ \ \ issue\ 4
\hfill \textbf{\thepage}}}

\def\leftkol{2016 AUTHOR INDEX} % ENGLISH ABSTRACTS}

\def\rightkol{2016 AUTHOR INDEX} %ENGLISH ABSTRACTS}


{\tabcolsep=3pt
\begin{tabular}{p{381pt}cc}
&\textbf{Issue} & \textbf{Page}\\[6pt]
\Avtors{Meykhanadzhyan~L.\,A.} Stationary characteristics of the finite
capacity queueing system with\linebreak
\\[-12pt]
\hspace*{23pt}inverse service order and generalized
probabilistic priority&2&123--131\\[.23pt]
\Avtors{Miller~G.\,B.} see~Borisov~A.\,V.&&\\[.23pt]
\Avtors{Minin~V.\,A., Zatsman~I.\,M., Havanskov~V.\,A., and
Shubnikov~S.\,K.} Intensity of citation of scientific publications in
inventions on information and computer technologies patented\linebreak
\\[-12pt]
\hspace*{23pt}in Russia by domestic and foreign applicants&2&107--122\\[.23pt]
\Avtors{Monakhov~M.\,M.} see~Markov~A.\,S.&&\\[.23pt]
\Avtors{Naumov~V.\,A.\ and Samouylov~K.\,E.} On relationship
between queuing systems with resources\linebreak
\\[-12pt]
\hspace*{23pt}and Erlang networks&3&\hphantom{1}9--14\\[.23pt]
\Avtors{Okladnikov~I.\,G.} see~Kalinichenko~L.\,A.&&\\[.23pt]
\Avtors{Ometov~A.\,Ya., Andreev~S.\,D., Turlikov~A.\,M., and
Koucheryavy~E.\,A.} Performance analysis of\linebreak
\\[-12pt]
\hspace*{23pt}a wireless data
aggregation system with contention for contemporary sensor
networks&3&23--31\\[.23pt]
\Avtors{Palionnaia~S.\,I.} see~Kudryavtsev~A.\,A.&&\\[.23pt]
\Avtors{Podkolodnyy~N.\,L.} see~Kalinichenko~L.\,A.&&\\[.23pt]
\Avtors{Ponomareva~N.\,V.} see~Kalinichenko~L.\,A.&&\\[.23pt]
\Avtors{Popkova~N.\,A.} see~Zatsman~I.\,M.&&\\[.23pt]
\Avtors{Pozanenko~A.\,S.} see~Kalinichenko~L.\,A.&&\\[.23pt]
\Avtors{Razumchik~R.\,V.} see~Konovalov~M.\,G.&&\\[.23pt]
\Avtors{Ronzhin~A.\,F.} see~Melnikov~A.\,K.&&\\[.23pt]
\Avtors{Rumovskaya~S.\,B.} see~Kirikov~I.\,A.&&\\[.23pt]
\Avtors{Rumovskaya~S.\,B.} see~Kirikov~I.\,A.&&\\[.23pt]
\Avtors{Rumovskaya~S.\,B.} see~Kolesnikov A.\,V.&&\\[.23pt]
\Avtors{Samouylov~K.\,E.} see~Gaidamaka~Yu.\,V.&&\\[.23pt]
\Avtors{Samouylov~K.\,E.} see~Naumov~V.\,A.&&\\[.23pt]
\Avtors{Serebryanskii~S.\,M.} see~Tyrsin~A.\,N.&&\\[.23pt]
\Avtors{Seyful-Mulyukov~R.\,B.} see~Callaos~N.\,K.&&\\[.23pt]
\Avtors{Shestakov~O.\,V.} Statistical properties of the denoising method
based on the stabilized hard\linebreak
\\[-12pt]
\hspace*{23pt}thresholding&2&65--69\\[.23pt]
\Avtors{Shestakov~O.\,V.} The strong law of large numbers for the risk
estimate in the problem of\linebreak
\\[-12pt]
\hspace*{23pt}tomographic image reconstruction from
projections with a correlated noise&3&41--45\\[.23pt]
\Avtors{Shestakov~O.\,V.} see~Zakharova~T.\,V.&&\\[.23pt]
\Avtors{Shnurkov~P.\,V., Gorshenin~A.\,K., and Belousov~V.\,V.}
Analytical solution of~the~optimal control\linebreak
\\[-12pt]
\hspace*{23pt}task of~a~semi-Markov
process with~finite set of~states&4&72--88\\[.23pt]
\Avtors{Shnurkov~P.\,V., Zasypko~V.\,V., Belousov~V.\,V., and
Gorshenin~A.\,K.} Development of the algorithm of numerical solution
of the optimal investment control problem\linebreak
\\[-12pt]
\hspace*{23pt}in the closed dynamical model of three-sector economy&1&82--95\\[.23pt]
\Avtors{Shorgin~S.\,Ya.} see~Gaidamaka~Yu.\,V.&&\\[.23pt]
\Avtors{Shorgin~V.\,S.} see~Agalarov~Ya.\,M.&&\\[.23pt]
\Avtors{Shubnikov~S.\,K.} see~Minin~V.\,A.&&\\[.23pt]
\Avtors{Sidorkin~I.\,I.} see~Arkhipov~O.\,P.&&\\[.23pt]
\Avtors{Sinitsyn~I.\,N.} Analytical modeling of processes in stochastic
systems with complex fractional\linebreak
\\[-12pt]
\hspace*{23pt}order Bessel nonlinearities&3&55--65\\[.23pt]
\Avtors{Sinitsyn~I.\,N.} Orthogonal supoptimal filters for nonlinear
stochastic systems on manifolds&1&34--44\\[.23pt]
\Avtors{Sinitsyn~I.\,N.\ and Korepanov~E.\,R.} Normal Pugachev
conditionally-optimal filters and extra-\linebreak
\\[-12pt]
\hspace*{23pt}polators for state linear stochastic systems&2&14--23\\[.23pt]
\Avtors{Sinitsyn~I.\,N.\ and Sinitsyn~V.\,I.} Analytical modeling of
distributions in stochastic systems on\linebreak
\\[-12pt]
\hspace*{23pt}manifolds based on ellipsoidal approximation&1&45--55\\[.23pt]
\Avtors{Sinitsyn~I.\,N., Sinitsyn~V.\,I., and
Korepanov~E.\,R.} Ellipsoidal suboptimal filters for nonlinear\linebreak
\\[-12pt]
\hspace*{23pt}stochastic systems on manifolds&2&24--35\\[.23pt]
\Avtors{Sinitsyn~V.\,I.} see~Sinitsyn~I.\,N.&&\\[.23pt]
\Avtors{Sinitsyn~V.\,I.} see~Sinitsyn~I.\,N.&&\\[.23pt]
\Avtors{Skvortsov~N.\,A.} see~Stupnikov~S.\,A.&&\\[.23pt]
\Avtors{Sokolov~I.\,A.} see~Chertok~A.\,V.&&\\
\end{tabular}
}
\pagebreak

\def\leftfootline{\small{\textbf{\thepage}
\hfill INFORMATIKA I EE PRIMENENIYA~--- INFORMATICS AND APPLICATIONS\ \ \ 2016\
\ \ volume~10\ \ \ issue\ 4}
}%
 \def\rightfootline{\small{INFORMATIKA I EE PRIMENENIYA~---
INFORMATICS AND APPLICATIONS\ \ \ 2016\ \ \ volume~10\ \ \ issue\ 4
\hfill \textbf{\thepage}}}

\def\leftkol{2016 AUTHOR INDEX} % ENGLISH ABSTRACTS}

\def\rightkol{2016 AUTHOR INDEX} %ENGLISH ABSTRACTS}


{\tabcolsep=3pt
\begin{tabular}{p{382pt}cc}
&\textbf{Issue} & \textbf{Page}\\[6pt]
\Avtors{Sopin~E.\,S.} see~Gaidamaka~Yu.\,V.&&\\
\Avtors{Strijov~V.\,V.} see~Goncharov~A.\,V.&&\\
\Avtors{Strijov~V.\,V.} see~Isachenko~R.\,V.&&\\
\Avtors{Strijov~V.\,V.} see~Karasikov~M.\,E.&&\\
\Avtors{Stupnikov~S.\,A., Briukhov~D.\,O., and Skvortsov~N.\,A.}
Co-lending systemic risk analysis over\linebreak
\\[-12pt]
\hspace*{23pt}heterogeneous data collections&1&23--33\\
\Avtors{Stupnikov~S.\,A.} see~Kalinichenko~L.\,A.&&\\
\Avtors{Suchkov~A.\,P.} see~Zatsarinny~A.\,A.&&\\
\Avtors{Timonina~E.\,E.} see~Grusho~A.\,A.&&\\
\Avtors{Titova~A.\,I.} see~Kudryavtsev~A.\,A.&&\\
\Avtors{Turlikov~A.\,M.} see~Ometov~A.\,Ya.&&\\
\Avtors{Tyrsin~A.\,N.\ and Serebryanskii~S.\,M.} Recognition of
dependences on the basis of inverse\linebreak
\\[-12pt]
\hspace*{23pt}mapping&2&58--64\\
\Avtors{Ulyanov~V.\,V.} see~Markov~A.\,S.&&\\
\Avtors{Ushakov~V.\,G.} Queueing system with working vacations and
hyperexponential input stream&2&92--97\\
\Avtors{Ushakov~V.\,G.} see~Leontyev~N.\,D.&&\\
\Avtors{Volnova~A.\,A.} see~Kalinichenko~L.\,A.&&\\
\Avtors{Yakovlev~O.\,A.\ and Gasilov~A.\,V.} Speeded-up stereo
matching using geodesic support weights&3&\hphantom{1}98--104\\
\Avtors{Zabezhailo~M.\,I.} see~Grusho~A.\,A.&&\\
\Avtors{Zabezhailo~M.\,I.} see~Grusho~A.\,A.&&\\
\Avtors{Zakharova~T.\,V.\ and Shestakov~O.\,V.} Precision analysis of
wavelet processing of aerodynamic\linebreak
\\[-12pt]
\hspace*{23pt}flow patterns&3&46--54\\
\Avtors{Zalizniak~Anna~A.\ and Kruzhkov~M.\,G.} Database
of~Russian impersonal verbal constructions&4&132--141\\
\Avtors{Zasypko~V.\,V.} see~Shnurkov~P.\,V.&&\\
\Avtors{Zatsarinny~A.\,A.\ and Suchkov~A.\,P.} Systems engineering
approaches to~the~establishment of\linebreak
\\[-12pt]
\hspace*{23pt}a~system for~decision support based
on~situational analysis&4&105--113\\
\Avtors{Zatsarinny~A.\,A.} see~Grusho~A.\,A.&&\\
\Avtors{Zatsman~I.\,M., Inkova~O.\,Yu., Kruzhkov~M.\,G., and
Popkova~N.\,A.} Representation of cross-\linebreak
\\[-12pt]
\hspace*{23pt}lingual knowledge about
connectors in supracorpora databases&1&106--118\\
\Avtors{Zatsman~I.\,M.} see~Minin~V.\,A.&&\\
\Avtors{Zeifman~A.\,I.} see~Korolev~V.\,Yu.&&\\
\Avtors{Zeifman~A.\,I.} see~Korolev~V.\,Yu.&&\\
\end{tabular}
}

%\thispagestyle{myheadings}
\def\leftfootline{\small{\textbf{\thepage}
\hfill INFORMATIKA I EE PRIMENENIYA~--- INFORMATICS AND APPLICATIONS\ \ \ 2016\
\ \ volume~10\ \ \ issue\ 4}
}%
 \def\rightfootline{\small{INFORMATIKA I EE PRIMENENIYA~---
INFORMATICS AND APPLICATIONS\ \ \ 2016\ \ \ volume~10\ \ \ issue\ 4
\hfill \textbf{\thepage}}}

 \label{end\stat}

\newpage

\def\stat{rekl}
%\label{preobr}

%\def\tit{АКАДЕМИК ПУГАЧЁВ  ВЛАДИМИР СЕМЁНОВИЧ\\
%25.03.1911--25.03.1998}


%   \vspace*{-48pt}
%   \begin{center}\LARGE
%Академик Пугачёв  Владимир Семёнович\\ (25.03.1911--25.03.1998)
%   \end{center}
   
   %\vspace*{2.5mm}
   
   \begin{center}

{\prgsh\LARGE
ОБЪЯВЛЕНИЯ О КОНФЕРЕНЦИЯХ}

\end{center}
%\hrule

\vspace*{6pt}

   
   \vspace*{10mm}
   
   \thispagestyle{empty}

\noindent
\begin{tabular}{cc}
%\begin{center}
\multicolumn{1}{c}{\raisebox{-40pt}[0pt][0pt]{\mbox{%
\epsfxsize=33mm
\epsfbox{vspu.eps}
}}}
%\end{center}
&
\tabcolsep=0pt\begin{tabular}{c}
{\prg{\Large\textbf{XII Всероссийское совещание}}}\\[6pt]
{\prg{\Large\textbf{по проблемам управления}}}\\[12pt]
{\prg{\large 16--19 июня 2014~г.}}\\[6pt] 
{\prg{\large Институт проблем управления имени В.\,А.~Трапезникова РАН}}\\[6pt]
{\prg{\large Москва, Россия}}
\end{tabular}
\end{tabular}

\vspace*{60pt}

     
 { %\large    
 XII Всероссийское совещание по проблемам управления (ВСПУ XII), посвященное 75-летию 
Института проблем управления (ИПУ) имени В.\,А.~Трапезникова РАН, проводится 16--19~июня 
2014~г.\ 
в ИПУ РАН (г.~Москва, Россия). ВСПУ XII организуется ИПУ РАН при поддержке РФФИ, Отделения 
энергетики, машиностроения, механики и процессов управления Российской академии наук, 
Российского 
национального комитета по автоматическому управлению, Академии навигации и управ\-ле\-ния 
движением, 
Научного совета РАН по комплексным проблемам управления и автоматизации, Совета по 
мехатронике и робототехнике РАН. Официальный язык Совещания~--- русский.

\vspace*{24pt}
     
     \textbf{Направления работы}
     \begin{enumerate}[1.]
\item Теория систем управления
\item Управление подвижными объектами и навигация
\item Интеллектуальные системы управления
\item Управление в промышленности, транспортом и логистикой
\item Управление системами междисциплинарной природы
\item Средства измерения, вычислений и контроля в управлении
\item Системный анализ и принятие решений в задачах управления
\item Информационные технологии в управлении
\item Проблемы образования в области управления: современное содержание и технологии обучения
\end{enumerate}

\vspace*{24pt}

     Подробная информация о Совещании находится на сайте {\sf http://vspu2014.ipu.ru}. Срок 
окончательной подачи докладов через систему подачи докладов на сайте~--- \textbf{30~ноября} 
2013~г.
}

\include{rekl-1}

%\end{document}

%   \vspace*{-48pt}

\begin{center}
\vspace*{6pt}
\mbox{%
\epsfxsize=53.502mm
\epsfbox{foto-1.eps}
}
\end{center}

\vspace*{6pt} %Академик


   \begin{center}
\fbox{\Large\textbf{Профессор Игорь Алексеевич Ушаков}}\\[12pt]
\textbf{\large 22.01.1935--27.02.2015}
   \end{center}


   %\vspace*{2.5mm}

   \vspace*{5mm}

   \thispagestyle{empty}

%\

%\vspace*{-12pt}


Редакционный совет и редакционная коллегия журнала <<Информатика и~её применения>> с~глубоким прискорбием извещают, что 27~февраля 2015~г.\ после тяжелой
и~продолжительной болезни скончался Игорь Алексеевич Ушаков~--- доктор технических наук, профессор, член редколлегии журнала <<Информатика и ее применения>>.

Игорь Алексеевич Ушаков окончил Московский авиационный институт, в~1963~г.\ защитил кандидатскую, а~в~1968~г.~--- докторскую диссертацию. С~1958 по 1989~гг.\ работал в~ряде научно-исследовательских организаций СССР, в~том числе руководил отделами в~НИИ АА и~ВЦ АН СССР; с 1969 по 1989 гг. преподавал в~МФТИ (был профессором, а~затем заведующим кафедрой) и~в~МЭИ. С~1989~г.~---- в~США: являлся профессором университета Дж.\ Вашингтона, университета Дж.\ Мэйсона и~Калифорнийского университета, сотрудником компаний MCI, Qualcomm и Hughes.

И.\,А.~Ушаков с момента основания журнала <<Надежность и~контроль качества>> был заместителем ответственного редактора, а~затем на протяжении многих лет членом редколлегии. В~2006~г.\ основал электронный международный журнал ``Reliability: Theory \& Application'', главным редактором которого оставался до конца жизни.

Учебниками и справочниками по теории надежности, написанными И.\,А.~Ушаковым, пользовались и~пользуются несколько поколений ученых и~специалистов в~разных странах мира.

Игорь Алексеевич всегда уделял огромное внимание работе с~молодежью; более~50 его учеников защитили докторские и~кандидатские диссертации.

И.\,А.~Ушаков вел активную научно-про\-све\-ти\-тель\-скую деятельность. В~частности, он был одним из организаторов и~руководителей Московского кабинета качества и~надежности при Политехническом музее (целью этого Кабинета было оказание консультаций работникам промышленных предприятий и~чтение курсов лекций для инженеров, занимающихся проблемой надежности). Находясь в~США, И.\,А.~Ушаков создал международный ин\-тер\-нет-фо\-рум им.\ Б.\,В.~Гнеденко, объединивший около~400~видных специалистов по приложениям теории вероятностей и~математической статистики, преимущественно в~об\-ласти теории надежности и~анализа риска, из десятков стран мира; коллективным членов этого Форума является и~наш журнал. Цели Форума~--- содействие контактам между специалистами из разных стран, организация обмена профессиональными 
новостями и~информацией (новые публикации, предстоящие события и~др.). Также необходимо отметить большое число на\-уч\-но-по\-пу\-ляр\-ных работ, опубликованных И.\,А.~Ушаковым.

И.\,А.~Ушаков обладал большим личным обаянием, имел широкий круг интересов. Все знавшие И.\,А.~Ушакова всегда будут помнить его как замечательного ученого и~прекрасного человека.

\bigskip

Редакционный совет и редакционная коллегия журнала <<Информатика и~её применения>> 
выражают глубокие соболезнования родным и близким покойного, всем, кто его знал и~работал с~ним.






%\end{document}

%\include{IPPM-25}



\def\stat{cont-rus}
{%\hrule\par
%\vskip 7pt % 7pt
\vspace*{-24pt}
\raggedleft\Large \bf%\baselineskip=3.2ex
Правила подготовки рукописей  для публикации в журнале
<<Информатика~и~её~применения>> \vskip 8pt
    \hrule
    \par
\vskip 14pt plus 6pt minus 3pt }

\label{st\stat}

\def\tit{\ }

\def\aut{\ }
\def\auf{\ }

\def\leftkol{\ }
% Правила подготовки рукописей  для публикации в журнале
%<<Информатика и её применения>>

\def\rightkol{\ }
%Правила подготовки рукописей  для публикации в журнале
%<<Информатика и её применения>>}


\titele{\tit}{\aut}{\auf}{\leftkol}{\rightkol}


\vspace*{-60pt}
{ %\small

Журнал <<Информатика и её применения>>
публикует теоретические, обзорные и дискуссионные статьи,
посвященные научным исследованиям и разработкам в области
информатики и ее приложений.

Журнал издается на русском языке. По специальному решению
редколлегии отдельные статьи могут печататься на английском языке.

Тематика журнала охватывает следующие направления:
\begin{itemize}
\item теоретические основы информатики;\\[-15pt]
      \item
математические методы исследования сложных систем и процессов;\\[-15pt]
           \item
информационные системы и сети;\\[-15pt]
                \item
информационные технологии;\\[-15pt]
                     \item
архитектура и программное обеспечение вычислительных комплексов и сетей.\\[-15pt]
\end{itemize}


\noindent
\begin{enumerate}[1.]
\item В журнале печатаются статьи, содержащие результаты, ранее не опубликованные и
не предназначенные к одновременной публикации в других изданиях.

%Публикация не должна нарушать закон об авторских правах.
Публикация предоставленной автором(ами) рукописи не должна нарушать 
положений глав~69, 70 раздела~VII части~IV Гражданского кодекса, 
которые определяют права на результаты интеллектуальной деятельности 
и~средства индивидуализации, в~том числе авторские права, в~РФ.

Ответственность за нарушение авторских прав, в~случае предъявления претензий к~редакции журнала,  
несут авторы статей.



Направляя рукопись в редакцию, авторы сохраняют свои права на данную
рукопись и при этом передают учредителям и редколлегии журнала неисключительные права на
издание статьи на русском языке 
(или на языке статьи, если он отличен от рус\-ско\-го) и~на перевод ее на английский
язык, а~также на
ее распространение в России и за рубежом. 
Каждый автор должен представить в~редакцию подписанный 
с~его стороны <<Лицензионный договор о~передаче неисключительных прав 
на использование произведения>>, текст которого размещен по адресу 
{\sf http://www.ipiran.ru/publications/licence.doc}. 
Этот договор может быть пред\-став\-лен в~бумажном (в~2-х экз.)\ 
или в~электронном виде (отсканированная копия заполненного и~подписанного документа).




Редколлегия вправе запросить у авторов экспертное заключение о возможности
пуб\-ли\-ка\-ции пред\-став\-лен\-ной статьи в открытой печати.\\[-13.5pt]

\item К статье прилагаются данные автора (авторов) (см.\ п.~8). При наличии нескольких
авторов указывается фамилия автора, ответственного за переписку с редакцией.\\[-13.5pt]

\item Редакция журнала осуществляет экспертизу присланных статей в соответствии с
принятой в журнале процедурой рецензирования.

Возвращение рукописи на доработку не означает ее принятия к печати.

Доработанный вариант с ответом на замечания рецензента необходимо прислать в
редакцию.\\[-13.5pt]

\item Решение редколлегии о публикации статьи или ее отклонении сообщается авторам.

Редколлегия может также направить авторам текст рецензии на их статью. Дискуссия по
поводу отклоненных статей не ведется.\\[-13.5pt]

%\pagebreak

\item Редактура статей высылается авторам для просмотра. Замечания к редактуре должны
быть присланы авторами в кратчайшие сроки.\\[-13.5pt]

\item Рукопись предоставляется в электронном виде в форматах MS WORD (.doc или
.docx) или \LaTeX\  (.tex), дополнительно~--- в формате .pdf, на дискете, лазерном диске
или электронной почтой. Предоставление бумажной рукописи необязательно.\\[-13.5pt]

\item При подготовке рукописи в MS Word рекомендуется использовать следующие
настройки.

Параметры страницы:
формат~--- А4; ориентация~--- книжная; поля (см): внутри~--- 2,5, снаружи~--- 1,5,
сверху~--- 2, снизу~--- 2, от края до нижнего колонтитула~--- 1,3.

Основной текст: стиль~--- <<Обычный>>, шрифт~--- Times New Roman, размер~---
14~пунк\-тов, абзацный отступ~--- 0,5~см, 1,5~интервала, выравнивание~--- по ширине.

\pagebreak

\def\leftkol{Правила подготовки рукописей  для публикации в журнале
<<Информатика и её применения>>}

\def\rightkol{Правила подготовки рукописей  для публикации в журнале
<<Информатика и её применения>>}



Рекомендуемый объем рукописи~--- не свыше 10~страниц указанного формата.
При превышении указанного объема редколлегия вправе потребовать от 
автора сокращения объема рукописи.


Сокращения слов, помимо стандартных, не допускаются. Допускается минимальное
количество аббревиатур.


Все страницы рукописи нумеруются.

Шаблоны оформления представлены в интернете:

\noindent
 {\sf
http://www.ipiran.ru/journal/template\_iiep\_ssi\_2024.zip}\\[-14pt]

\item Статья должна содержать следующую информацию на {\bfseries\textit{русском и
английском языках}}:\\[-16pt]

\begin{itemize}
\item название статьи;\\[-15pt]
\item Ф.И.О.\ авторов, на английском можно только имя и фамилию;\\[-15pt]
\item место работы, с указанием почтового адреса организации и электронного адреса каждого
автора;\\[-15pt]
\item сведения об авторах, в соответствии с форматом, образцы которого
представлены на страницах:



\def\leftfootline{\small{\textbf{\thepage}
\hfill ИНФОРМАТИКА И ЕЁ ПРИМЕНЕНИЯ\ \ \ том\ 18\ \ \ выпуск\ 3\ \ \ 2024}
}%
 \def\rightfootline{\small{ИНФОРМАТИКА И ЕЁ ПРИМЕНЕНИЯ\ \ \ том\ 18\ \ \ выпуск\ 3\ \ \ 2024
\hfill \textbf{\thepage}}}



{\sf http://www.ipiran.ru/journal/issues/2013\_07\_01/authors.asp} и

{\sf http://www.ipiran.ru/journal/issues/2013\_07\_01\_eng/authors.asp};
\item аннотация (не менее 100~слов на каждом из языков). Аннотация~--- это краткое
резюме работы, которое может публиковаться отдельно. Она является основным
источником информации в~ин\-фор\-ма\-ци\-он\-ных системах и базах данных. Английская
аннотация должна быть оригинальной, может не быть дословным переводом русского
текста и должна быть написана хорошим английским языком. В~аннотации не должно
быть ссылок на литературу и, по возможности, формул;\\[-15pt]
\item ключевые слова~--- желательно из принятых в мировой
на\-уч\-но-тех\-ни\-че\-ской литературе тематических тезаурусов. Предложения не
могут быть ключевыми словами;\\[-15pt]
\item источники финансирования работы (ссылки на гранты, проекты,
поддерживающие организации и~т.\,п.).
\end{itemize}



%\pagebreak

\item  Требования к спискам литературы.\\[-14pt]

Ссылки на литературу в тексте статьи нумеруются (в квадратных скобках) и
располагаются в каждом из списков литературы в порядке  первых упоминаний. Если источник имеет DOI и/или EDN,
то их необходимо указывать.

Списки литературы представляются в двух вариантах:\\[-14pt]


\noindent
\begin{enumerate}[(1)]
\item \textbf{Список литературы к русскоязычной части}. Русские и английские
работы~---  на языке и в алфавите оригинала;\\[-14.5pt]
\item  \textbf{References}. Русские работы и работы на других языках~--- в латинской
транслитерации с переводом на английский язык; английские работы и работы на других
языках~--- на языке оригинала.
\end{enumerate}

Необходимо для составления списка ``References'' пользоваться размещенной на сайте
{\sf http://www. translit.net/ru/bgn/} бесплатной программой транслитерации русского
 текста в~латиницу. %, при этом в~за\-клад\-ке <<варианты\ldots>> следует выбратьопцию BGN.

Список литературы ``References'' приводится полностью отдельным блоком, повторяя все
позиции из списка литературы к русскоязычной части, независимо от того, имеются или
нет в нем иностранные источники. Если в списке литературы к русскоязычной части есть
ссылки на иностранные публикации, набранные латиницей, они полностью повторяются в
списке ``References''.

Ниже приведены примеры ссылок на различные виды публикаций в списке ``References''.

\def\leftfootline{\small{\textbf{\thepage}
\hfill ИНФОРМАТИКА И ЕЁ ПРИМЕНЕНИЯ\ \ \ том\ 18\ \ \ выпуск\ 3\ \ \ 2024}
}%
 \def\rightfootline{\small{ИНФОРМАТИКА И ЕЁ ПРИМЕНЕНИЯ\ \ \ том\ 18\ \ \ выпуск\ 3\ \ \ 2024
\hfill \textbf{\thepage}}}

{\small

\noindent
\textbf{Описание статьи из журнала:}

\Aue{Zagurenko, A.\,G., V.\,A.~Korotovskikh, A.\,A.~Kolesnikov, A.\,V.~Timonov, and D.\,V.~Kardymon}. 2008.
Tekhniko-ekonomicheskaya optimizatsiya dizayna gidrorazryva plasta [Technical and
economic optimization of the design
of hydraulic fracturing]. \textit{Neftyanoe hozyaystvo} [\textit{Oil Industry}] 11:54--57.

\Aue{Zhang, Z., and D.~Zhu}. 2008. Experimental research on the localized
electrochemical micromachining. \textit{Russ. J.~Electrochem.}  44(8):926--930.
{\sf doi:10.1134/S1023193508080077}.

\noindent
\textbf{Описание статьи из электронного журнала:}

\Aue{Swaminathan, V., E.~Lepkoswka-White, and B.\,P.~Rao}. 1999. Browsers or buyers in cyberspace? An
investigation of electronic factors influencing electronic exchange. \textit{JCMC}
5(2). Available at: {\sf http://www.ascusc.org/jcmc/vol5/issue2/} (accessed April~28, 2011).

\def\leftkol{Правила подготовки рукописей  для публикации в журнале
<<Информатика и её применения>>}

\def\rightkol{Правила подготовки рукописей  для публикации в журнале
<<Информатика и её применения>>}


\noindent
\textbf{Описание статьи из продолжающегося издания (сборника трудов):}

\Aue{Astakhov, M.\,V., and T.\,V.~Tagantsev}. 2006. Eksperimental'noe
issledovanie prochnosti soedineniy ``stal'--kompozit'' [Experimental study of
the strength of joints ``steel--composite'']. \textit{Trudy MGTU
``Matematicheskoe modelirovanie slozhnykh tekh\-ni\-che\-skikh sistem''}
[\textit{Bauman MSTU ``Mathematical Modeling of Complex Technical
Systems'' Proceedings}]. 593:125--130.


\pagebreak



\noindent
\textbf{Описание материалов конференций:}

\Aue{Usmanov, T.\,S., A.\,A.~Gusmanov, I.\,Z.~Mullagalin, R.\,Ju.~Muhametshina, A.\,N.~Chervyakova, and
A.\,V.~Sveshnikov}. 2007. Osobennosti proektirovaniya razrabotki mestorozhdeniy
s primeneniem gidrorazryva
plasta [Features of the design of field development with the use of hydraulic fracturing].
\textit{Trudy 6-go
Mezhdu\-na\-rod\-no\-go Simpoziuma ``Novye resursosberegayushchie tekhnologii nedropol'zovaniya i povysheniya
neftegazootdachi''} [\textit{6th  Symposium (International) ``New Energy Saving Subsoil Technologies and
the Increasing of the Oil and Gas Impact'' Proceedings}]. Moscow. 267--272.



\def\leftfootline{\small{\textbf{\thepage}
\hfill ИНФОРМАТИКА И ЕЁ ПРИМЕНЕНИЯ\ \ \ том\ 18\ \ \ выпуск\ 3\ \ \ 2024}
}%
 \def\rightfootline{\small{ИНФОРМАТИКА И ЕЁ ПРИМЕНЕНИЯ\ \ \ том\ 18\ \ \ выпуск\ 3\ \ \ 2024
\hfill \textbf{\thepage}}}



\noindent
\textbf{Описание книги (монографии, сборники):}



Lindorf, L.\,S., and L.\,G.~Mamikoniants, eds. 1972.
\textit{Ekspluatatsiya turbogeneratorov s neposredstvennym
okhlazhdeniem} [\textit{Operation of turbine generators with direct cooling}].
Moscow: Energy Publs. 352~p.


\Aue{Latyshev, V.\,N.} 2009. \textit{Tribologiya rezaniya. Kn.~1: Friktsionnye protsessy
pri rezanii metallov}
[\textit{Tribology of cutting. Vol.~1: Frictional processes in metal cutting}]. Ivanovo: Ivanovskii
State Univ. 108~p.

\def\leftkol{Правила подготовки рукописей  для публикации в журнале
<<Информатика и её применения>>}

\def\rightkol{Правила подготовки рукописей  для публикации в журнале
<<Информатика и её применения>>}

\noindent
\textbf{Описание переводной книги}
(в списке литературы к русскоязычной части необходимо указать:~/ Пер.\ с англ.~---
после названия книги, а в конце ссылки указать оригинал книги в круглых скобках):
\begin{enumerate}[1.]
\item  В русскоязычной части:

\def\leftfootline{\small{\textbf{\thepage}
\hfill ИНФОРМАТИКА И ЕЁ ПРИМЕНЕНИЯ\ \ \ том\ 18\ \ \ выпуск\ 3\ \ \ 2024}
}%
 \def\rightfootline{\small{ИНФОРМАТИКА И ЕЁ ПРИМЕНЕНИЯ\ \ \ том\ 18\ \ \ выпуск\ 3\ \ \ 2024
\hfill \textbf{\thepage}}}

\Au{Тимошенко С.\,П., Янг Д.\,Х., Уивер~У.}
Колебания в инженерном деле~/ Пер.\ с англ.~--- М.: Машиностроение, 1985. 472~с.
(\Au{Timoshenko~S.\,P., Young~D.\,H., Weaver~W.}
Vibration problems in engineering.~--- 4th ed.~--- New York, NY, USA: Wiley, 1974. 521~p.)\\[-13.5pt]
\item  В англоязычной части:

\Aue{Timoshenko, S.\,P., D.\,H.~Young, and W.~Weaver}.
1974. \textit{Vibration problems in engineering}. 4th ed. New York: 
Wiley. 521~p.
\end{enumerate}

\vspace*{-3pt}


\noindent
\textbf{Описание неопубликованного документа:}


\Aue{Latypov, A.\,R., M.\,M.~Khasanov, and V.\,A.~Baikov}.
2004 (unpubl.). Geologiya i~dobycha (NGT GiD) [Geology and production (NGT GiD)]. Certificate on official registration of the computer program
No.\,2004611198. 

\noindent
\textbf{Описание интернет-ресурса:}


Pravila tsitirovaniya istochnikov [Rules for the citing of sources]. Available at: {\sf
http://www.scribd.com/doc/1034528/} (accessed February~7, 2011).

%\pagebreak

\noindent
\textbf{Описание диссертации или автореферата диссертации:}

\Aue{Semenov, V.\,I.}
2003. Matematicheskoe modelirovanie plazmy v sisteme kompaktnyy tor [Mathematical
modeling of the plasma in the compact torus].  Moscow.  D.Sc.\ Diss. 272~p.

\Aue{Kozhunova, O.\,S.} 2009. Tekhnologiya razrabotki semanticheskogo
slovarya informatsionnogo monitoringa [Technology of development of
semantic dictionary of information monitoring system].  Moscow: IPI RAN. PhD Thesis. 23~p.


\noindent
\textbf{Описание ГОСТа:}

GOST 8.586.5-2005. 2007. Metodika vypolneniya izmereniy. Izmerenie raskhoda i~kolichestva zhidkostey i~gazov
s~pomoshch'yu standartnykh suzhayushchikh ustroystv [Method of measurement.
Measurement of flow rate and volume of liquids and gases by means of orifice devices]. Moscow:
Standardinform  Publs. 10~p.

\noindent
\textbf{Описание патента:}

\Aue{Bolshakov, M.\,V., A.\,V.~Kulakov, A.\,N.~Lavrenov, and M.\,V.~Palkin}.
2006. Sposob orientirovaniya po krenu letatel'nogo
apparata s opti\-che\-skoy golovkoy
samonavedeniya [The way to orient on the roll of aircraft with optical homing head].
Patent RF No.\,2280590.
}

\item Присланные в редакцию материалы авторам не возвращаются.\\[-13.5pt]

\item При отправке файлов по электронной почте просим придерживаться следующих
правил:
\begin{itemize}
\item указывать в поле subject (тема) название журнала и фамилию автора;\\[-13.5pt]
\item указывать в тексте письма название статьи, авторов и~журнал, в~который направляется статья;\\[-13.5pt]
\item использовать attach (присоединение);\\[-13.5pt]
\item в состав электронной версии статьи должны входить: файл, содержащий текст
статьи, и файл(ы), содержащий(е) иллюстрации.\\[-13.5pt]
\end{itemize}

\item Журнал <<Информатика и её применения>> является некоммерческим изданием.
Плата за публикацию не взимается, гонорар авторам не выплачивается.
\end{enumerate}



\def\leftfootline{\small{\textbf{\thepage}
\hfill ИНФОРМАТИКА И ЕЁ ПРИМЕНЕНИЯ\ \ \ том\ 18\ \ \ выпуск\ 3\ \ \ 2024}
}%
 \def\rightfootline{\small{ИНФОРМАТИКА И ЕЁ ПРИМЕНЕНИЯ\ \ \ том\ 18\ \ \ выпуск\ 3\ \ \ 2024
\hfill \textbf{\thepage}}}


\vspace*{-1mm}

\begin{center}

\textbf{Адрес редакции журнала <<Информатика и её применения>>:} \\




Москва 119333, ул.~Вавилова, д.~44, корп.~2, ФИЦ ИУ РАН\\[-10pt]

\

Тел.: +7\,(499)\,135-86-92\ \ Факс:  +7\,(495)\,930-45-05\\[-10pt]

 \

e-mail:   {\sf iiep@frccsc.ru} (Стригина Светлана Николаевна)\\[-10pt]

\

{\sf http://www.ipiran.ru/journal/issues/}
\end{center}
}


\def\leftkol{Правила подготовки рукописей  для публикации в журнале
<<Информатика и её применения>>}

\def\rightkol{Правила подготовки рукописей  для публикации в журнале
<<Информатика и её применения>>}


\def\leftfootline{\small{\textbf{\thepage}
\hfill ИНФОРМАТИКА И ЕЁ ПРИМЕНЕНИЯ\ \ \ том\ 18\ \ \ выпуск\ 3\ \ \ 2024}
}%
 \def\rightfootline{\small{ИНФОРМАТИКА И ЕЁ ПРИМЕНЕНИЯ\ \ \ том\ 18\ \ \ выпуск\ 3\ \ \ 2024
\hfill \textbf{\thepage}}} 
\def\stat{podg-e}
{%\hrule\par
%\vskip 7pt % 7pt
\vspace*{-24pt}
\raggedleft\Large \bf%\baselineskip=3.2ex
Requirements for manuscripts submitted to Journal
``Informatics~and~Applications'' \vskip 8pt
    \hrule
    \par
\vskip 21pt plus 6pt minus 3pt }

\label{st\stat}

\def\tit{\ }

\def\aut{\ }
\def\auf{\ }

\def\leftkol{\ }

\def\rightkol{\ }
%Requirements for manuscripts submitted to Journal
%``Informatics~and~Applications''}

\titele{\tit}{\aut}{\auf}{\leftkol}{\rightkol}

\def\leftfootline{\small{\textbf{\thepage}
\hfill INFORMATIKA I EE PRIMENENIYA~--- INFORMATICS AND APPLICATIONS\ \ \ 2019\
\ \ volume~13\ \ \ issue\ 4}
}%
 \def\rightfootline{\small{INFORMATIKA I EE PRIMENENIYA~--- INFORMATICS AND APPLICATIONS\ \ \ 2019\ \ \ volume~13\ \ \ issue\ 4
\hfill \textbf{\thepage}}}

\vspace*{-60pt}

{\small

\noindent
Journal ``Informatics and Applications'' (Inform.\ Appl.)
publishes theoretical, review, and discussion
articles on the research and development in the
field of informatics and its applications.

The journal is published in Russian.
By a special decision of the editorial
board, some articles can be published in English.


The topics covered include the following areas:
\begin{itemize}
               \item
     theoretical fundamentals of informatics; \\[-14pt]
\item
mathematical methods for studying complex systems and processes; \\[-14pt]
\item
information systems and networks;\\[-14pt]
\item
information technologies; and \\[-14pt]
\item
architecture and software of computational complexes and networks. \\[-14pt]
\end{itemize}

\noindent
\begin{enumerate}[1.]
\item The Journal publishes original articles which have not been published before and are not
intended for simultaneous publication in other editions. An article submitted to the Journal must not violate the
Copyright law. Sending the manuscript to the Editorial Board, the authors retain all rights of the
owners of the manuscript and transfer the nonexclusive rights to publish the article in Russian
(or the language of the article, if not Russian) and its distribution in Russia and abroad to the
Founders and the Editorial Board. Authors should submit a letter to the Editorial Board in the
following form:

{\bfseries\textit{Agreement on the transfer of rights to publish:}}

``\textit{We, the undersigned authors of the manuscript ``\ldots'', pass to the
Founder and the Editorial Board of the Journal ``Informatics and Applications''
the nonexclusive right to publish the manuscript of the article in Russian (or
in English) in both print and electronic versions of the Journal. We affirm
that this publication does not violate the Copyright of other persons or
organizations.}

\textit{Author(s) signature(s): (name(s), address(es), date).}

This agreement should be submitted in paper form or in the form of a scanned copy (signed by
the authors).


%The Editorial Board has the right to request from the authors an official expert conclusion that
%the submitted article has no secret data prohibited for publication. \\[-13.5pt]
\item
A submitted article should be attached with \textbf{the data on the author(s)} (see item~8). If
there are several authors, the contact person should be indicated who is responsible for
correspondence with the Editorial Board and other authors about revisions and final approval
of the proofs.\\[-13.5pt]

\item The Editorial Board of the Journal examines the article according to the established
reviewing procedure. If the authors receive their article for correction after reviewing, it does not
mean that the article is approved for publication. The corrected article should be sent to the
Editorial Board for the subsequent review and approval.\\[-13.5pt]

\item The decision on the article publication or its rejection is communicated to the authors. The
Editorial Board may also send the reviews on the submitted articles to the authors. Any
discussion upon the rejected articles is not possible.\\[-13.5pt]

\item The edited articles will be sent to the authors for proofread. The comments of the authors
to the edited text of the article should be sent to the Editorial Board as soon as possible.\\[-13.5pt]

\item The manuscript of the article should be presented electronically in the MS WORD (.doc or
.docx) or \LaTeX\ (.tex) formats, and additionally in the .pdf format. All documents
 may be sent
by e-mail or provided on a CD or diskette. A~hard copy submission is not necessary.\\[-13.5pt]

\item The recommended typesetting instructions for manuscript.

Pages parameters: format A4, portrait orientation, document margins (cm): left~--- 2.5, right~---
1.5, above~--- 2.0, below~--- 2.0, footer 1.3.

Text: font~---Times New Roman, font size~--- 14, paragraph indent~--- 0.5, line spacing~--- 1.5,
justified alignment.

The recommended manuscript size: not more than 15~pages of the specified format.
If the specified size exceeded, the editorial board is entitled to require the author
to reduce the manuscript.

Use only standard abbreviations. Avoid  abbreviations in the title and
abstract. The full term for which an abbreviation stands should precede
its first use in the text unless it is a standard unit of measurement.

All pages of the manuscript should be numbered.

The templates for the manuscript typesetting are presented on site: {\sf
http://www.ipiran.ru/journal/template.doc}.\\[-13.5pt]


%\def\leftkol{Requirements for manuscripts submitted to Journal
%``Informatics~and~Applications''}

\item The articles should enclose data both in \textbf{Russian and English}:
\begin{itemize}
\item title;\\[-13.5pt]
\item author's name and surname;\\[-13.5pt]
\item affiliation~--- organization, its address with ZIP code, city, country, and
official e-mail address;\\[-13.5pt]
\item data on authors according to the format: (see site)

{\sf http://www.ipiran.ru/journal/issues/2013\_07\_01/authors.asp}  and

{\sf  http://www.ipiran.ru/journal/issues/2013\_07\_01\_eng/authors.asp};\\[-13.5pt]

\pagebreak

\def\leftfootline{\small{\textbf{\thepage}
\hfill INFORMATIKA I EE PRIMENENIYA~--- INFORMATICS AND APPLICATIONS\ \ \ 2019\
\ \ volume~13\ \ \ issue\ 4}
}%
 \def\rightfootline{\small{INFORMATIKA I EE PRIMENENIYA~--- INFORMATICS AND APPLICATIONS\ \ \ 2019\ \ \ volume~13\ \ \ issue\ 4
\hfill \textbf{\thepage}}}


%\def\leftkol{Requirements for manuscripts submitted to Journal
%``Informatics~and~Applications''}

%\def\rightkol{Requirements for manuscripts submitted to Journal
%``Informatics~and~Applications''}



\item abstract (not less than 100 words) both in Russian and in English. Abstract is a short
summary of the article that can be published separately. The abstract is the
main source of information on the article and it could be included in leading information
systems and data bases. The abstract in English has to be an original text and should
not be an exact translation of the Russian one. Good English is required.
In abstracts, avoid references and formulae;\\[-13.5pt]
\item indexing is performed on the basis of keywords. The use of keywords from the
internationally accepted thematic Thesauri is recommended.

%\def\leftkol{Requirements for manuscripts submitted to Journal
%``Informatics~and~Applications''}

%\def\rightkol{Requirements for manuscripts submitted to Journal
%``Informatics~and~Applications''}

Important! Keywords must not be sentences;
\item Acknowledgments.
\end{itemize}

\item References. Russian references have to be presented both in English translation and Latin
transliteration (refer {\sf http://www.translit.net/ru/bgn/}).

Please take into account the following examples of Russian references appearance:

\noindent
\textbf{Article in journal:}

\Aue{Zhang, Z., and D.~Zhu}. 2008. Experimental research on the localized electrochemical
micromachining.
\textit{Rus. J.~Electrochem.}  44(8):926--930. {\sf doi:10.1134/S1023193508080077}.


\noindent
\textbf{Journal article in electronic format:}

\Aue{Swaminathan, V., E.~Lepkoswka-White, and B.\,P.~Rao}. 1999. Browsers or buyers in
cyberspace? An
investigation of electronic factors influencing electronic exchange. \textit{JCMC}
5(2). Available at: {\sf http://www.ascusc.org/jcmc/vol5/issue2/} (accessed April~28, 2011).




\noindent
\textbf{Article from the continuing publication (collection of works, proceedings):}

\Aue{Astakhov, M.\,V., and T.\,V.~Tagantsev}. 2006. Eksperimental'noe
issledovanie prochnosti soedineniy ``stal'--kompozit'' [Experimental study of
the strength of joints ``steel--composite'']. \textit{Trudy MGTU
``Matematicheskoe modelirovanie slozhnykh tekh\-ni\-che\-skikh sistem''}
[\textit{Bauman MSTU ``Mathematical Modeling of Complex Technical
Systems'' Proceedings}]. 593:125--130.

\def\leftfootline{\small{\textbf{\thepage}
\hfill INFORMATIKA I EE PRIMENENIYA~--- INFORMATICS AND APPLICATIONS\ \ \ 2019\
\ \ volume~13\ \ \ issue\ 4}
}%
 \def\rightfootline{\small{INFORMATIKA I EE PRIMENENIYA~--- INFORMATICS AND APPLICATIONS\ \ \ 2019\ \ \ volume~13\ \ \ issue\ 4
\hfill \textbf{\thepage}}}

\def\leftkol{Requirements for manuscripts submitted to Journal
``Informatics~and~Applications''}

\def\rightkol{Requirements for manuscripts submitted to Journal
``Informatics~and~Applications''}

\noindent
\textbf{Conference proceedings:}

\Aue{Usmanov, T.\,S., A.\,A.~Gusmanov, I.\,Z.~Mullagalin, R.\,Ju.~Muhametshina,
A.\,N.~Chervyakova, and
A.\,V.~Sveshnikov}. 2007. Osobennosti proektirovaniya razrabotki mestorozhdeniy
s primeneniem gidrorazryva
plasta [Features of the design of field development with the use of hydraulic fracturing].
\textit{Trudy 6-go
Mezhdu\-na\-rod\-no\-go Simpoziuma ``Novye resursosberegayushchie tekhnologii
nedropol'zovaniya i povysheniya
neftegazootdachi''} [\textit{6th  Symposium (International) ``New Energy Saving Subsoil
Technologies and
the Increasing of the Oil and Gas Impact'' Proceedings}]. Moscow. 267--272.


\noindent
\textbf{Books and other monographs:}




Lindorf, L.\,S., and L.\,G.~Mamikoniants, eds. 1972.
\textit{Ekspluatatsiya turbogeneratorov s neposredstvennym
okhlazhdeniem} [\textit{Operation of turbine generators with direct cooling}].
Moscow: Energy Publs. 352~p.


%\Aue{Latyshev, V.\,N.} 2009. \textit{Tribologiya rezaniya. Kn.~1: Frikcionnye prosessy
%pri rezanii metallov}
%[\textit{Tribology of cutting. Vol.~1: Frictional processes in metal cutting}]. Ivanovo: Ivanovskii
%State Univ. 108~p.


%\noindent
%\textbf{Unpublished material:}

%\Aue{Latypov, A.\,R., M.\,M.~Khasanov, and V.\,A.~Baikov}.
%2004. Geology and production (NGT GiD). Certificate on official registration of the computer
%program
%No.\,2004611198. (In Russian, unpubl.)

%\noindent
%\textbf{Internet-source:}

%APA Style. 2011. Available at: {\sf http://www.apastyle.org/apa-style-help.aspx} (accessed
%February~5, 2011).

%Pravila citirovaniya istochnikov [Rules for the citing of sources]. Available at: {\sf
%http://www.scribd.com/doc/1034528/} (accessed February~7, 2011).


\noindent
\textbf{Dissertation and Thesis:}

%\Aue{Semenov, V.\,I.}
%2003. Matematicheskoe modelirovanie plazmy v sisteme kompaktnyy tor. [Mathematical
%modeling of the plasma in the compact torus]. D.Sc.\ Diss. Moscow. 272~p.

\Aue{Kozhunova, O.\,S.} 2009. Tekhnologiya razrabotki semanticheskogo
slovarya informatsionnogo monitoringa [Technology of development of
semantic dictionary of information monitoring system]. PhD Thesis. Moscow: IPI RAN. 23~p.


\noindent
\textbf{State standards and patents:}

GOST 8.586.5-2005. 2007. Metodika vypolneniya izmereniy. Izmerenie raskhoda i~kolichestva
zhidkostey i gazov 
s~pomoshch'yu standartnykh suzhayushchikh ustroystv [Method of measurement.
Measurement of flow rate and volume of liquids and gases by means of orifice devices]. M.:
Standardinform
Publs. 10~p.

%\noindent
%\textbf{Patent:}

\Aue{Bolshakov, M.\,V., A.\,V.~Kulakov, A.\,N.~Lavrenov, and M.\,V.~Palkin}.
2006. Sposob orientirovaniya po krenu letatel'nogo
apparata s opti\-che\-skoy golovkoy
samonavedeniya [The way to orient on the roll of aircraft with optical homing head].
Patent RF No.\,2280590.

References in Latin transcription are presented in the original language.

References in the text are numbered according to the order of their
first appearance; the number is
placed in square brackets. All items from the reference list should be
cited.\\[-13.5pt]

\item Manuscripts and additional materials are not returned to Authors by the Editorial Board.\\[-13.5pt]

\item Submissions of files by e-mail must include:\\[-13.5pt]
\begin{itemize}
\item   the journal title and author's name in the ``Subject'' field; \\[-13.5pt]
\item   an article and additional materials have to be attached using the ``attach'' function;\\[-13.5pt]
\item   an electronic version of the article should contain the file with the text and a separate file
with figures.\\[-13.5pt]
\end{itemize}

\item ``Informatics and Applications'' journal is not a profit publication. There are no
charges for the authors as well as there are no royalties.\\[-13.5pt]
\end{enumerate}

\def\leftfootline{\small{\textbf{\thepage}
\hfill INFORMATIKA I EE PRIMENENIYA~--- INFORMATICS AND APPLICATIONS\ \ \ 2019\
\ \ volume~13\ \ \ issue\ 4}
}%
 \def\rightfootline{\small{INFORMATIKA I EE PRIMENENIYA~--- INFORMATICS AND APPLICATIONS\ \ \ 2019\ \ \ volume~13\ \ \ issue\ 4
\hfill \textbf{\thepage}}}

\def\leftkol{Requirements for manuscripts submitted to Journal
``Informatics~and~Applications''}

\def\rightkol{Requirements for manuscripts submitted to Journal
``Informatics~and~Applications''}


%\vspace*{5mm}


\begin{center}
\textbf{Editorial Board address:} \\

%ABOUT AUTHORS



FRC CSC RAS, 44, block~2, Vavilov Str., Moscow 119333, Russia\\[-10pt]

\

Ph.: +7\,(499)\,135\,86\,92,\ \ Fax: +7\,(495)\,930\,45\,05\\[-10pt]

\

 e-mail: {\sf rust@ipiran.ru} (to Prof.\ Rustem Seyful-Mulyukov)\\[-10pt]

\

 {\sf http://www.ipiran.ru/english/journal.asp}
\end{center}
 }
%\thispagestyle{myheadings}

\def\leftkol{Requirements for manuscripts submitted to Journal
``Informatics~and~Applications''}

\def\rightkol{Requirements for manuscripts submitted to Journal
``Informatics~and~Applications''}

\def\leftfootline{\small{\textbf{\thepage}
\hfill INFORMATIKA I EE PRIMENENIYA~--- INFORMATICS AND APPLICATIONS\ \ \ 2019\
\ \ volume~13\ \ \ issue\ 4}
}%
 \def\rightfootline{\small{INFORMATIKA I EE PRIMENENIYA~--- INFORMATICS AND APPLICATIONS\ \ \ 2019\ \ \ volume~13\ \ \ issue\ 4
\hfill \textbf{\thepage}}}

 \label{end\stat}

\newpage


%\vspace*{-60pt} {\small
{\baselineskip=9.1pt
\section*{Правила подготовки рукописей статей для публикации в журнале
<<Информатика и её применения>>}

\thispagestyle{empty}

 Журнал <<Информатика и её применения>> публикует
теоретические, обзорные и дискуссионные статьи, посвященные научным
исследованиям и разработкам в области информатики и ее приложений. Журнал
издается на русском языке. По специальному решению редколлегии отдельные статьи,
в виде исключения, могут печататься на английском языке.
Тематика журнала охватывает следующие направления:
\begin{itemize}
\item теоретические основы информатики; %\\[-13.5pt]
\item математические методы исследования сложных систем и процессов; %\\[-13.5pt]
\item информационные системы и сети; %\\[-13.5pt]
\item информационные технологии; %\\[-13.5pt]
\item архитектура и программное
обеспечение вычислительных комплексов и сетей.
\end{itemize}
\begin{enumerate}
\item В журнале печатаются результаты, ранее не
опубликованные и не предназначенные к одновременной публикации в других
изданиях. Публикация не должна нарушать закон об авторских правах. Направляя
свою рукопись в редакцию, авторы автоматически передают учредителям и
редколлегии неисключительные права на издание данной статьи на русском языке и
на ее распространение в России и за рубежом. При этом за авторами сохраняются
все права как собственников данной рукописи. В связи с этим авторами должно
быть представлено в редакцию письмо в следующей форме:
Соглашение о передаче права на публикацию:

\textit{<<Мы, нижеподписавшиеся, авторы рукописи <<$\qquad\qquad$>>, передаем
учредителям и редколлегии журнала <<Информатика и её применения>>
неисключительное право опубликовать данную рукопись статьи на русском языке как
в печатной, так и в электронной версиях журнала. Мы подтверждаем, что данная
публикация не нарушает авторского права других лиц или организаций. Подписи
авторов: (ф.\,и.\,о., дата, адрес)>>.}

Указанное соглашение может быть представлено 
как в бумажном виде, так и в виде отсканированной копии (с подписями авторов).


Редколлегия вправе запросить у авторов экспертное заключение о возможности
опубликования представленной статьи в открытой печати. %\\[-13.5pt]
\item Статья
подписывается всеми авторами. На отдельном листе представляются данные автора
(или всех авторов): фамилия, полные имя и отчество, телефон, факс, e-mail,
почтовый адрес. Если работа выполнена несколькими авторами, указывается фамилия
одного из них, ответственного за переписку с редакцией. %\\[-13.5pt]
\item Редакция журнала
осуществляет самостоятельную экспертизу присланных статей. Возвращение рукописи
на доработку не означает, что статья уже принята к печати. Доработанный вариант
с ответом на замечания рецензента необходимо прислать в редакцию. %\\[-13.5pt]
\item Решение
редакционной коллегии о принятии статьи к печати или ее отклонении сообщается
авторам. Редколлегия не обязуется направлять рецензию авторам отклоненной
статьи. %\\[-13.5pt]
\item Корректура статей высылается авторам для просмотра. Редакция
просит авторов присылать свои замечания в кратчайшие сроки. %\\[-13.5pt]
\item При
подготовке рукописи в MS Word рекомендуется использовать следующие настройки.
Параметры страницы: формат~--- А4; ориентация~--- книжная; поля (см): внутри~---
2,5, снаружи~--- 1,5, сверху~--- 2, снизу~--- 2, от края до нижнего
колонтитула~--- 1,3. Основной текст: стиль~--- <<Обычный>>: шрифт Times New
Roman, размер 14~пунктов, абзацный отступ~--- 0,5~см, 1,5 интервала,
выравнивание~--- по ширине. Рекомендуемый объем рукописи~--- не свыше
25~страниц указанного формата. Ознакомиться с шаблонами, содержащими примеры
оформления, можно по адресу в Интернете:
\textsf{http://www.ipiran.ru/journal/template.doc}.
\item К рукописи, предоставляемой в 2-х
экземплярах, обязательно прилагается электронная версия статьи (как правило, в
форматах MS WORD (.doc) или \LaTeX\ (.tex), а также~--- дополнительно~--- в
формате .pdf) на дискете, лазерном диске или по электронной почте. Сокращения
слов, кроме стандартных, не применяются. Все страницы рукописи должны быть
пронумерованы. %\\[-13.5pt]
\item Статья должна содержать следующую информацию на русском и
английском языках: название, Ф.И.О. авторов, места работы авторов и их
электронные адреса, подробные сведения об авторах, оформленные в соответствии с форматом, 
определяемым файлами {\sf http://www.ipiran.ru/journal/issues/2011\_05\_01/authors.asp} и 
{\sf http://www.ipiran.ru/journal/issues/2011\_01\_eng/authors.asp},
аннотация (не более 100~слов), ключевые слова. Ссылки на
литературу в тексте статьи нумеруются (в квадратных скобках) и располагаются в
порядке их первого упоминания. В~списке литературы не должно быть позиций, на которые нет ссылки в тексте статьи.
Все фамилии авторов, заглавия статей, названия
книг, конференций и~т.\,п.\ даются на языке оригинала, если этот язык
использует кириллический или латинский алфавит. %\\[-13.5pt]
\item Присланные в редакцию материалы авторам не возвращаются.
\item При отправке файлов по электронной
почте просим придерживаться следующих правил:
\begin{itemize}
\item указывать в поле subject (тема) название журнала и фамилию автора; %\\[-13.5pt]
\item использовать attach (присоединение); %\\[-13.5pt]
\item в случае больших объемов информации возможно
использование общеизвестных архиваторов (ZIP, RAR); %\\[-13.5pt]
\item в состав электронной версии статьи должны входить: файл, содержащий текст статьи, и файл(ы),
содержащий(е) иллюстрации. %\\[-13.5pt]
\end{itemize}
\item Журнал <<Информатика и её применения>> является некоммерческим изданием. 
Плата за публикацию с авторов не взимается, гонорар авторам не выплачивается.
\end{enumerate}
\thispagestyle{empty}
\textbf{Адрес редакции:} Москва 119333,
ул.~Вавилова, д.~44, корп.~2, ИПИ РАН\\
\hphantom{\textbf{Адрес редакции:} }Тел.: +7 (499) 135-86-92\ \
Факс:  +7 (495) 930-45-05\ \  E-mail:   rust@ipiran.ru }
}

\end{document}


%\tableofcontents

%\end{document}





%\def\stat{cont}
{%\hrule\par
%\vskip 7pt % 7pt
\raggedleft\Large \bf%\baselineskip=3.2ex
А\,В\,Т\,О\,Р\,С\,К\,И\,Й\ \ У\,К\,А\,З\,А\,Т\,Е\,Л\,Ь\ \ З\,А\ \ 2\,0\,0\,7 г. \vskip 17pt
    \hrule
    \par
\vskip 21pt plus 6pt minus 3pt }

\label{st\stat}

\def\tit{\ }

\def\aut{\ }
\def\auf{\ }

\def\leftkol{\ } % ENGLISH ABSTRACTS}

\def\rightkol{\ } %ENGLISH ABSTRACTS}

\titele{\tit}{\aut}{\auf}{\leftkol}{\rightkol}


\contentsline {chapter}{\ }{Выпуск \quad Стр.} 
\contentsline {section}{\textbf{Батракова Д.\,А., Королев В.\,Ю., Шоргин С.\,Я.}\ \ Новый метод вероятностно-ста\-ти\-сти\-че\-ско\-го анализа информационных потоков в\nobreakspace {}телекоммуникационных сетях}{\qquad 1 \qquad 40} 
\contentsline {section}{\textbf{Борисов А.\,В.}\ \ Байесовское оценивание в системах наблюдения с\nobreakspace {}марковскими скачкообразными процессами: игровой подход}{\qquad 2 \qquad 65}
\contentsline {section}{\textbf{Босов А.\,В., Иванов А.\,В.}\ \ Программная инфраструктура информационного Web-пор\-тала}{\qquad 2 \qquad 50}
\contentsline {section}{\textbf{Захаров В.\,Н., Калиниченко Л.\,А., Соколов И.\,А., Ступников С.\,А.}\ \ Конструирование канонических информационных моделей для интегрированных информационных систем}{\qquad 2 \qquad 15}
\contentsline {section}{\textbf{Захаров В.\,Н., Козмидиади В.\,А.}\ \ Средства обеспечения отказоустойчивости при\-ло\-жений}{\qquad 1 \qquad 14} 
\contentsline {section}{\textbf{Иванов А.\,В.}\ \ см. Босов А.\,В.\hfill\hfill\hfill\hfill\hfill\hfill\hfill\hfill\hfill\hfill\hfill\hfill\hfill\hfill\hfill\hfill\hfill\hfill\hfill\hfill\hfill\hfill\hfill\hfill\hfill\hfill\hfill\hfill\hfill\hfill\hfill\hfill\hfill\hfill\hfill}{\ }
\contentsline {section}{\textbf{Ильин В.\,Д., Соколов И.\,А.}\ \ Символьная модель системы знаний информатики в\nobreakspace {}че\-ло\-ве\-ко-автоматной среде}{\qquad 1 \qquad 66} 
\contentsline {section}{\textbf{Калиниченко Л.\,А.}\ \ см. Захаров В.\,Н.\hfill\hfill\hfill\hfill\hfill\hfill\hfill\hfill\hfill\hfill\hfill\hfill\hfill\hfill\hfill\hfill\hfill\hfill\hfill\hfill\hfill\hfill\hfill\hfill\hfill\hfill\hfill\hfill\hfill\hfill\hfill\hfill\hfill\hfill\hfill}{\ }
\contentsline {section}{\textbf{Козеренко Е.\,Б.}\ \ Лингвистическое моделирование для систем машинного перевода и обработки знаний}{\qquad 1 \qquad 54} 
\contentsline {section}{\textbf{Козмидиади В.\,А.}\ \ см. Захаров В.\,Н.\hfill\hfill\hfill\hfill\hfill\hfill\hfill\hfill\hfill\hfill\hfill\hfill\hfill\hfill\hfill\hfill\hfill\hfill\hfill\hfill\hfill\hfill\hfill\hfill\hfill\hfill\hfill\hfill\hfill\hfill\hfill\hfill\hfill\hfill\hfill }{\ } 
\contentsline {section}{\textbf{Королев В.\,Ю.}\ \ см. Батракова Д.\,А.\hfill\hfill\hfill\hfill\hfill\hfill\hfill\hfill\hfill\hfill\hfill\hfill\hfill\hfill\hfill\hfill\hfill\hfill\hfill\hfill\hfill\hfill\hfill\hfill\hfill\hfill\hfill\hfill\hfill\hfill\hfill\hfill\hfill\hfill\hfill}{\ } 
\contentsline {section}{\textbf{Кудрявцев А.\,А., Шоргин С.\,Я.}\ \ Байесовский подход к\nobreakspace {}анализу систем массового обслуживания и\nobreakspace {}показателей надежности}{\qquad 2 \qquad 76}
\contentsline {section}{\textbf{Печинкин А.\,В., Соколов И.\,А., Чаплыгин В.\,В.}\ \ Многолинейная система массового обслуживания с конечным накопителем и ненадежными приборами}{\qquad 1 \qquad 27} 
\contentsline {section}{\textbf{Печинкин А.\,В., Соколов И.\,А., Чаплыгин В.\,В.}\ \ Стационарные характеристики многолинейной\nobreakspace {}системы массового обслуживания с\nobreakspace {}одновременными отказами приборов}{\qquad 2 \qquad 39}
\contentsline {section}{\textbf{Синицын И.\,Н.}\ \ Корреляционные методы построения аналитических информационных моделей флуктуаций полюса Земли по априорным данным}{\qquad 2 \qquad \hphantom{9}2}
\contentsline {section}{\textbf{Синицын И.\,Н.}\ \ Развитие теории фильтров Пугачева для оперативной обработки информации в стохастических системах}{{\qquad 1 \qquad \hphantom{9}3}} 
\contentsline {section}{\textbf{Соколов И.\,А.}\ \ см. Захаров В.\,Н.\hfill\hfill\hfill\hfill\hfill\hfill\hfill\hfill\hfill\hfill\hfill\hfill\hfill\hfill\hfill\hfill\hfill\hfill\hfill\hfill\hfill\hfill\hfill\hfill\hfill\hfill\hfill\hfill\hfill\hfill\hfill\hfill\hfill\hfill\hfill}{\ }
\contentsline {section}{\textbf{Соколов И.\,А.}\ \ см. Ильин В.\,Д.\hfill\hfill\hfill\hfill\hfill\hfill\hfill\hfill\hfill\hfill\hfill\hfill\hfill\hfill\hfill\hfill\hfill\hfill\hfill\hfill\hfill\hfill\hfill\hfill\hfill\hfill\hfill\hfill\hfill\hfill\hfill\hfill\hfill\hfill\hfill}{\ } 
\contentsline {section}{\textbf{Соколов И.\,А.}\ \ см. Печинкин А.\,В.\hfill\hfill\hfill\hfill\hfill\hfill\hfill\hfill\hfill\hfill\hfill\hfill\hfill\hfill\hfill\hfill\hfill\hfill\hfill\hfill\hfill\hfill\hfill\hfill\hfill\hfill\hfill\hfill\hfill\hfill\hfill\hfill\hfill\hfill\hfill}{\ } 
\contentsline {section}{\textbf{Соколов И.\,А.}\ \ см. Печинкин А.\,В.\hfill\hfill\hfill\hfill\hfill\hfill\hfill\hfill\hfill\hfill\hfill\hfill\hfill\hfill\hfill\hfill\hfill\hfill\hfill\hfill\hfill\hfill\hfill\hfill\hfill\hfill\hfill\hfill\hfill\hfill\hfill\hfill\hfill\hfill\hfill}{\ }
\contentsline {section}{\textbf{Ступников С.\,А.}\ \ см. Захаров В.\,Н.\hfill\hfill\hfill\hfill\hfill\hfill\hfill\hfill\hfill\hfill\hfill\hfill\hfill\hfill\hfill\hfill\hfill\hfill\hfill\hfill\hfill\hfill\hfill\hfill\hfill\hfill\hfill\hfill\hfill\hfill\hfill\hfill\hfill\hfill\hfill}{\ }
\contentsline {section}{\textbf{Чаплыгин В.\,В.}\ \ см. Печинкин А.\,В.\hfill\hfill\hfill\hfill\hfill\hfill\hfill\hfill\hfill\hfill\hfill\hfill\hfill\hfill\hfill\hfill\hfill\hfill\hfill\hfill\hfill\hfill\hfill\hfill\hfill\hfill\hfill\hfill\hfill\hfill\hfill\hfill\hfill\hfill\hfill}{\ } 
\contentsline {section}{\textbf{Чаплыгин В.\,В.}\ \ см. Печинкин А.\,В.\hfill\hfill\hfill\hfill\hfill\hfill\hfill\hfill\hfill\hfill\hfill\hfill\hfill\hfill\hfill\hfill\hfill\hfill\hfill\hfill\hfill\hfill\hfill\hfill\hfill\hfill\hfill\hfill\hfill\hfill\hfill\hfill\hfill\hfill\hfill}{\ }
\contentsline {section}{\textbf{Шоргин С.\,Я.}\ \ см. Батракова Д.\,А.\hfill\hfill\hfill\hfill\hfill\hfill\hfill\hfill\hfill\hfill\hfill\hfill\hfill\hfill\hfill\hfill\hfill\hfill\hfill\hfill\hfill\hfill\hfill\hfill\hfill\hfill\hfill\hfill\hfill\hfill\hfill\hfill\hfill\hfill\hfill}{\ } 
\contentsline {section}{\textbf{Шоргин С.\,Я.}\ \ см. Кудрявцев А.\,А.\hfill\hfill\hfill\hfill\hfill\hfill\hfill\hfill\hfill\hfill\hfill\hfill\hfill\hfill\hfill\hfill\hfill\hfill\hfill\hfill\hfill\hfill\hfill\hfill\hfill\hfill\hfill\hfill\hfill\hfill\hfill\hfill\hfill\hfill\hfill}{\ }
%\thispagestyle{myheadings}
\def\leftfootline{\small{\textbf{\thepage}
\hfill ИНФОРМАТИКА И ЕЁ ПРИМЕНЕНИЯ\ \ \ том~1\ \ \ выпуск~2\ \ \ 2007}
}%
 \def\rightfootline{\small{ИНФОРМАТИКА И ЕЁ ПРИМЕНЕНИЯ\ \ \ том~1\ \ \ выпуск~2\ \ \ 2007
 \hfill \textbf{\thepage}}}
 \label{end\stat}

%\def\stat{cont-e}
{%\hrule\par
%\vskip 7pt % 7pt
\raggedleft\Large \bf%\baselineskip=3.2ex
2\,0\,0\,7\ \ A\,U\,T\,H\,O\,R\ \ I\,N\,D\,E\,X \vskip 17pt
    \hrule
    \par
\vskip 21pt plus 6pt minus 3pt }

\label{st\stat}

\def\tit{\ }

\def\aut{\ }
\def\auf{\ }

\def\leftkol{\ } % ENGLISH ABSTRACTS}

\def\rightkol{\ } %ENGLISH ABSTRACTS}

\titele{\tit}{\aut}{\auf}{\leftkol}{\rightkol}


\contentsline {chapter}{\ }{Issue \quad Page} 
\contentsline {subsection}{\textbf{Batrakova D.\,A., Korolev V.\,Yu., Shorgin S.\,Ya.}\ \ A New Method for the Probabilistic and Statistical Analysis of Information Flows in Telecommunication Networks}{\qquad 1 \qquad 40} 
\contentsline {subsection}{\textbf{Borisov A.\,V.}\ \ Bayesian Estimation in\nobreakspace {}Observation Systems with\nobreakspace {}Markov Jump Processes: Game-Theoretic Approach}{\qquad 2 \qquad 65} 
\contentsline {subsection}{\textbf{Bosov A.\,V., Ivanov A.\,V.}\ \ Linguistic Simulation for Machine Translation and Knowledge Management Systems}{\qquad 2 \qquad 50} 
\contentsline {subsection}{\textbf{Chaplygin V.\,V.} see Pechinkin A.\,V.\hfill\hfill\hfill\hfill\hfill\hfill\hfill\hfill\hfill\hfill\hfill\hfill\hfill\hfill\hfill\hfill\hfill\hfill\hfill\hfill\hfill\hfill\hfill\hfill\hfill\hfill\hfill\hfill\hfill\hfill\hfill\hfill\hfill\hfill\hfill}{\ }
\contentsline {subsection}{\textbf{Chaplygin V.\,V.} see Pechinkin A.\,V.\hfill\hfill\hfill\hfill\hfill\hfill\hfill\hfill\hfill\hfill\hfill\hfill\hfill\hfill\hfill\hfill\hfill\hfill\hfill\hfill\hfill\hfill\hfill\hfill\hfill\hfill\hfill\hfill\hfill\hfill\hfill\hfill\hfill\hfill\hfill}{\ }
\contentsline {subsection}{\textbf{Ilyin V.\,D., Sokolov I.\,A.}\ \ The Symbol Model of Informatics Knowledge System in Human-Automaton Environment}{\qquad 1 \qquad 66} 
\contentsline {subsection}{\textbf{Ivanov A.\,V.} see Bosov A.\,V.\hfill\hfill\hfill\hfill\hfill\hfill\hfill\hfill\hfill\hfill\hfill\hfill\hfill\hfill\hfill\hfill\hfill\hfill\hfill\hfill\hfill\hfill\hfill\hfill\hfill\hfill\hfill\hfill\hfill\hfill\hfill\hfill\hfill\hfill\hfill}{\ }
\contentsline {subsection}{\textbf{Kalinichenko L.\,A.} see Zakharov V.\,N.\hfill\hfill\hfill\hfill\hfill\hfill\hfill\hfill\hfill\hfill\hfill\hfill\hfill\hfill\hfill\hfill\hfill\hfill\hfill\hfill\hfill\hfill\hfill\hfill\hfill\hfill\hfill\hfill\hfill\hfill\hfill\hfill\hfill\hfill\hfill}{\ }
\contentsline {subsection}{\textbf{Korolev V.\,Yu.} see Batrakova D.\,A.\hfill\hfill\hfill\hfill\hfill\hfill\hfill\hfill\hfill\hfill\hfill\hfill\hfill\hfill\hfill\hfill\hfill\hfill\hfill\hfill\hfill\hfill\hfill\hfill\hfill\hfill\hfill\hfill\hfill\hfill\hfill\hfill\hfill\hfill\hfill}{\ }
\contentsline {subsection}{\textbf{Kozerenko E.\,B.}\ \ Linguistic Simulation for Machine Translation and Knowledge Management Systems}{\qquad 1 \qquad 54} 
\contentsline {subsection}{\textbf{Kozmidiady V.\,A.} see Zakharov V.\,N.\hfill\hfill\hfill\hfill\hfill\hfill\hfill\hfill\hfill\hfill\hfill\hfill\hfill\hfill\hfill\hfill\hfill\hfill\hfill\hfill\hfill\hfill\hfill\hfill\hfill\hfill\hfill\hfill\hfill\hfill\hfill\hfill\hfill\hfill\hfill}{\ }
\contentsline {subsection}{\textbf{Kudryavtsev A.\,A., Shorgin S.\,Ya.}\ \ Bayesian Approach to Queueing Systems and Reliability Characteristics}{\qquad 2 \qquad 76} 
\contentsline {subsection}{\textbf{Pechinkin A.\,V., Sokolov I.\,A., Chaplygin V.\,V.}\ \ Multichannel Queuing System with Finite Buffer and Unreliable Servers}{\qquad 1 \qquad 27} 
\contentsline {subsection}{\textbf{Pechinkin A.\,V., Sokolov I.\,A., Chaplygin V.\,V.}\ \ Stationary Characteristics of a Multichannel Queueing System with\nobreakspace {}Simultaneous Refusals of Servers}{\qquad 2 \qquad 39} 
\contentsline {subsection}{\textbf{Shorgin S.\,Ya.} see Batrakova D.\,A.\hfill\hfill\hfill\hfill\hfill\hfill\hfill\hfill\hfill\hfill\hfill\hfill\hfill\hfill\hfill\hfill\hfill\hfill\hfill\hfill\hfill\hfill\hfill\hfill\hfill\hfill\hfill\hfill\hfill\hfill\hfill\hfill\hfill\hfill\hfill}{\ }
\contentsline {subsection}{\textbf{Shorgin S.\,Ya.} see Kudryavtsev A.\,A.\hfill\hfill\hfill\hfill\hfill\hfill\hfill\hfill\hfill\hfill\hfill\hfill\hfill\hfill\hfill\hfill\hfill\hfill\hfill\hfill\hfill\hfill\hfill\hfill\hfill\hfill\hfill\hfill\hfill\hfill\hfill\hfill\hfill\hfill\hfill}{\ }
\contentsline {subsection}{\textbf{Sinitsyn I.\,N.}\ \ Correlational Methods for Analytical Informational Models of the Earth Pole Fluctuations Design Based on a priori Data}{\qquad 2 \qquad \hphantom{9}2}
\contentsline {subsection}{\textbf{Sinitsyn I.\,N.}\ \ Development of Pugachev Filtering for Stochastic Systems}{\qquad 1 \qquad \hphantom{9}3}
\contentsline {subsection}{\textbf{Sokolov I.\,A.} see Ilyin V.\,D.\hfill\hfill\hfill\hfill\hfill\hfill\hfill\hfill\hfill\hfill\hfill\hfill\hfill\hfill\hfill\hfill\hfill\hfill\hfill\hfill\hfill\hfill\hfill\hfill\hfill\hfill\hfill\hfill\hfill\hfill\hfill\hfill\hfill\hfill\hfill}{\ }
\contentsline {subsection}{\textbf{Sokolov I.\,A.} see Pechinkin A.\,V.\hfill\hfill\hfill\hfill\hfill\hfill\hfill\hfill\hfill\hfill\hfill\hfill\hfill\hfill\hfill\hfill\hfill\hfill\hfill\hfill\hfill\hfill\hfill\hfill\hfill\hfill\hfill\hfill\hfill\hfill\hfill\hfill\hfill\hfill\hfill}{\ }
\contentsline {subsection}{\textbf{Sokolov I.\,A.} see Pechinkin A.\,V.\hfill\hfill\hfill\hfill\hfill\hfill\hfill\hfill\hfill\hfill\hfill\hfill\hfill\hfill\hfill\hfill\hfill\hfill\hfill\hfill\hfill\hfill\hfill\hfill\hfill\hfill\hfill\hfill\hfill\hfill\hfill\hfill\hfill\hfill\hfill}{\ }
\contentsline {subsection}{\textbf{Sokolov I.\,A.} see Zakharov V.\,N.\hfill\hfill\hfill\hfill\hfill\hfill\hfill\hfill\hfill\hfill\hfill\hfill\hfill\hfill\hfill\hfill\hfill\hfill\hfill\hfill\hfill\hfill\hfill\hfill\hfill\hfill\hfill\hfill\hfill\hfill\hfill\hfill\hfill\hfill\hfill}{\ }
\contentsline {subsection}{\textbf{Stupnikov S.\,A.} see Zakharov V.\,N.\hfill\hfill\hfill\hfill\hfill\hfill\hfill\hfill\hfill\hfill\hfill\hfill\hfill\hfill\hfill\hfill\hfill\hfill\hfill\hfill\hfill\hfill\hfill\hfill\hfill\hfill\hfill\hfill\hfill\hfill\hfill\hfill\hfill\hfill\hfill}{\ }
\contentsline {subsection}{\textbf{Zakharov V.\,N., Kalinichenko L.\,A., Sokolov I.\,A., Stupnikov S.\,A.}\ \ Development of Canonical Information Models for Integrated Information Systems}{\qquad 2 \qquad 15} 
\contentsline {subsection}{\textbf{Zakharov V.\,N., Kozmidiady V.\,A.}\ \ Means Providing Applications Fault Tolerance}{\qquad 1 \qquad 14} 
\def\leftfootline{\small{\textbf{\thepage}
\hfill ИНФОРМАТИКА И ЕЁ ПРИМЕНЕНИЯ\ \ \ том~1\ \ \ выпуск~2\ \ \ 2007}
}%
 \def\rightfootline{\small{ИНФОРМАТИКА И ЕЁ ПРИМЕНЕНИЯ\ \ \ том~1\ \ \ выпуск~2\ \ \ 2007
 \hfill \textbf{\thepage}}}
 \label{end\stat}


%\tableofcontents


\end{document}

\newcommand{\Ack}{\subsection*{\protect\large\bf Acknowledgments}}