\def\stat{suchkov}

\def\tit{ИНФОРМАЦИОННО-АНАЛИТИЧЕСКАЯ АВТОМАТИЗИРОВАННАЯ 
СИСТЕМА <<МЕГАЛИТ>> В~ОПТИМИЗАЦИИ ДИАГНОСТИКИ И ЛЕЧЕНИЯ МОЧЕКАМЕННОЙ БОЛЕЗНИ}

\def\titkol{Информационно-аналитическая автоматизированная 
система <<Мегалит>> в~оптимизации диагностики} % и лечения мочекаменной болезни}

\def\autkol{М.\,П.~Кривенко, С.\,А.~Голованов,  П.\,А.~Савченко
 и др.}

\def\aut{М.\,П.~Кривенко$^1$, С.\,А.~Голованов$^2$, П.\,А.~Савченко$^3$, 
А.\,В.~Сивков$^4$,  А.\,П.~Сучков$^5$}

\titel{\tit}{\aut}{\autkol}{\titkol}

%{\renewcommand{\thefootnote}{\fnsymbol{footnote}}\footnotetext[1] {Статья 
%рекомендована к публикации в журнале Программным комитетом конференции 
%<<Электронные библиотеки: перспективные методы и технологии, электронные 
%коллекции>> (RCDL-2012).}}

\renewcommand{\thefootnote}{\arabic{footnote}}
\footnotetext[1]{Институт проблем информатики Российской академии наук, mkrivenko@ipiran.ru} 
\footnotetext[2]{Научно-исследовательский институт урологии, sergeygol124@mail.ru} 
\footnotetext[3]{Институт проблем информатики Российской академии наук, psavchenko@ipiran.ru} 
\footnotetext[4]{Научно-исследовательский институт урологии, uroinfo@yandex.ru} 
\footnotetext[5]{Институт проблем информатики Российской академии наук, asuchkov@ipiran.ru}


\Abst{В статье, первой из предполагаемой серии научных публикаций, рассматриваются 
результаты исследований по автоматизации информационных и аналитических процессов 
обследования, диагностирования и лечения мочекаменной болезни (МКБ). Существенную 
роль в создании систем диагностики МКБ играет разработка информационных технологий 
сбора клинических данных и формирования специализированных баз данных (БД). Изучена 
возможность создания и способы реализации ин\-фор\-ма\-ци\-он\-но-ана\-ли\-ти\-че\-ской 
автоматизированной системы (ИААС) по сбору, хранению и обработке клинических данных 
обследования больных, а также алгоритмизации процессов принятия решений при 
диагностике МКБ и выборе схем лечения и профилактики этого заболевания. Предложенные 
математические методы и алгоритмы могут найти применение при дальнейшем развитии 
фундаментальных научных исследований в области разработки математических методов 
моделирования ме\-ди\-ко-био\-ло\-ги\-че\-ских сис\-тем, а также при создании необходимого 
математического инструментария.}
      
\KW{информационно-аналитическая система; урология; компьютерная диагностика; схема 
лечения; схема профилактики}

\DOI{10.14357/19922264130409}

\vskip 14pt plus 9pt minus 6pt

      \thispagestyle{headings}

      \begin{multicols}{2}

            \label{st\stat}

\section{Введение}

      В настоящее время доля людей, у которых на протяжении их жизни диагностируется 
МКБ, довольно значительна и составляет в странах Западной Европы 5\%--9\%, в Канаде и 
США~--- 7\%--12\%, в странах Азии~--- 1\%--5\%~[1--4]. 
  %    
      Эпидемиологические исследования, проводимые в ряде индустриально развитых стран, 
указывают на сохранение тенденции к росту частоты возникновения МКБ 
среди населения. Так, число  впервые выявленных случаев
МКБ на 100\,000 населения за последние 
десятилетия возросло в США с 58,7 (1950--1954~гг.)\ до 85,1 (2000~г.)~\cite{4-su, 3-su}, 
в Японии~--- с~43,7 
(1965~г.)\ до~134 (2005~г.)~\cite{6-su, 5-su}, в России~--- со 123,3 (2002~г.)\ до~138,6 
(2010~г.)~[7, 8].
      
      По данным исследований~[9] с использованием БД Pediatric Health 
Information System (национальная БД, в которую включены данные об амбулаторных 
визитах, срочных госпитализациях и стационарном лечении детей из 42~детских больниц 
США) по сравнению с общим количеством госпитализированных пациентов число пациентов 
с МКБ увеличилось с 18,4 на 100\,000 населения в 1999~г.\ до 57,0 в 2008~г., годовой прирост 
составил 10,6\% ($p \hm<0{,}0001$). 
      
      В основе развития МКБ лежат характерные нарушения обмена веществ, приводящие к 
образованию камней в мочевых путях. Эти литогенные (камнеобразующие) нарушения обмена 
веществ характеризуются большим многообразием и проявляются различными 
патологическими изменениями биохимического состава крови и мочи пациента.
      
      Необходимым условием для выбора правильной тактики консервативного лечения с 
целью предупреждения повторного камнеобразования является исследование всего комплекса 
метаболических факторов риска (МФР), ответственных за развитие МКБ.
{\looseness=-1

}
      
      В этой связи большое внимание придается изуче\-нию особенностей фи\-зи\-ко-хи\-ми\-че\-ских 
па\-ра\-мет\-ров мочи, во многом определяющих вероятность образования мочевых камней~[10]. Кроме 
того, литогенные нарушения метаболизма зачастую имеют сложный многофакторный 
характер воздействия на процесс формирования камня. Это создает особые трудности для 
врача в полной и объективной оценке всех влияющих литогенных факторов обмена веществ, а 
также в принятии решения по диагностике и выбору лечебной тактики для конкретного 
больного. Отсюда возникает необходимость формирования БД анкетных и 
лабораторных исследований, систем, связанных с диагностикой МКБ и формирования базы 
знаний по профилактике и лечению этого заболевания.
      
\section{Системы компьютерной диагностики в~области урологии}

      Существенную роль в создании систем диагностики МКБ является разработка 
информационных технологий сбора клинических данных и формирования 
специализированных БД. К~ним относится упомянутая Pediatric Health Information 
System. В~ряде медицинских работ упоминается реестр по уролитиазу (БД по 
больным и результатам лечения) Юго-за\-пад\-но\-го медицинского центра Техасского 
университета: <<Retrospective data from the University of Texas Southwestern Medical Center 
\textit{Nephrolithiasis Registry} from 17~studies that dealt with physiologic and physicochemical 
effects of various magnesium and potassium salts were categorized into three groups and 
analyzed\ldots>>~[11]. Однако подробного описания данного реестра не приведено. 
      
      Задачи диагностики, дифференциальной диагностики, прогнозирования, выбора 
стратегии и тактики лечения позволяют решать экспертные медицинские системы~[12].
      
      Ряд работ посвящен использованию в урологии компьютерных диагностических систем 
на основе методов искусственных нейронных сетей (ИНС)~[13].
 Так, в онкоурологии смогли 
прогнозировать 5-лет\-нюю выживаемость пациентов, перенесших радикальную цистэктомию 
по поводу\linebreak
рака мочевого пузы\-ря~[14]. Искусственные ней\-ронные сети применили также для 
автоматизи\-рованного анализа показаний к биопсии предстательной железы~[15]. 
Методика основывалась на\linebreak 
выявлении общего прос\-тат-спе\-ци\-фи\-че\-ско\-го антигена (ПСА) и определении доли 
свободного ПСА. Чувствительность составила 95\%, специфичность~--- 34\%. При 
дополне\-нии нейросети моделью логистической регрессии специфичность возросла до 95\%. 
Искусственная нейронная сеть использовалась для выявле\-ния группы риска рака предстательной железы в сравнении с 
моделью логистической регрессии~[15]. Искусственная нейронная сеть так\-же 
продемонстрировала более точные 
прогностические возможности. Компьютерных систем диагностики именно МКБ по 
литературным данным не выявлено.
{ %\looseness=-1

}
      
      Отсюда ясно, что имеется настоятельная необходимость разработки аналитической 
системы диагностики и лечения больных МКБ в процессе их динами\-ческого наблюдения 
(мониторинге) для пред\-упреж\-де\-ния повторного камнеобразования. Отсутствие подобных 
аналитических систем для мониторинга больных МКБ послужило основанием для разработки 
опытного образца ИААС 
<<Мегалит>>. Создание системы осуществляется ИПИ РАН совместно с НИИ урологии 
Минздравсоцразвития России в рамках серии совместных на\-уч\-но-ис\-сле\-до\-ва\-тель\-ских работ.
      
\subsection*{Основные цели и~задачи создания информационно-аналитической автоматизированной системы
 <<Мегалит>>}

      \noindent
      \begin{enumerate}[1.]
      \item  Создание БД по результатам обследования пациента, включающей:\\[-15pt]
      \begin{itemize}
\item формализованные данные опроса пациента при первом и последующих визитах, 
содержащие информацию о факторах, способных оказывать влияние на возникновение и 
особенности клинического течения МКБ (lifestyle-фак\-то\-ры индивида, факторы среды, 
питания, профессии и проч.);\\[-15pt]
\item данные лабораторного обследования (результатов простого или расширенного 
лабораторного обследования).\\[-15pt] 
\end{itemize}
      \item  Создание аналитической подсистемы, обеспечивающей решение следующих 
задач:\\[-15pt]
      \begin{itemize}
\item на основании данных первичного опроса выявление наличия или отсутствия, а также 
степень риска развития МКБ и определение объема предполагаемого лабораторного 
обследования пациента (простое или расширенное обследование);\\[-15pt]
\item на основе анализа входных данных лабораторного обследования осуществление выбора 
дальнейшей тактики ведения больного~--- дополнительные виды исследования, выбор 
лечебных мероприятий (тип хирургического лечения, схема медикаментозной терапии, 
коррекция диеты и проч.);\\[-15pt]
\item реализация методов оптимального выбора (с учетом показаний и противопоказаний) вида 
хирургического лечения или схемы проведения профилактического лечения (включая прием 
специальных фармпрепаратов, рекомендации по модификации диеты и образа жизни).
\end{itemize}
\end{enumerate}

        При создании опытного образца ИААС <<Мегалит>> учитывались следующие 
требования.
      \begin{enumerate}[1.]
\item Опытный образец аналитической системы <<Мегалит>> должен иметь возможность 
ведения распределенной БД пациентов, содержащей результаты обследований, профили 
МФР и относительный индекс перенасыщенности мочи (ОИП) как 
исходные, так и измененные в результате назначенного лечения, и включать набор подсистем, 
вклю\-ча\-ющих программную реализацию разработанных методов диагностирования и лечения.
\item Данные простого лабораторного обследования пациента должны включать: 
\begin{itemize}
\item исследование химического состава мочевого камня; 
\item биохимическое исследование крови и мочи по различным параметрам; 
\item клинический анализ мочи с посевом на мик\-ро\-флору; 
\item обзорный рентгеновский снимок, сонограмму и другие виды инструментального 
обследования пациента.
\end{itemize}
%\end{enumerate}
       Данные биохимического исследования представлены величинами содержания в крови 
и моче ионов и соединений, способных приводить к образованию мочевых камней. При 
наличии патологических отклонений в биохимических исследования проводится расширенное 
лабораторное обследование.
 \item Данные расширенного лабораторного обследования включают протокол диагностики 
типа гиперкальциурии (ПД-ГКУ) (при выявлении повышен\-ной суточной экскреции кальция у 
пациента). Выполняется поэтапно, с помощью модифицированной по кальцию диеты. 
В~расширенное лабораторное обследование входит также полный диагностический протокол (ПДП)
больного МКБ. 
\item Полный диагностический протокол пред\-став\-ля\-ет собой выраженное в 
графическом виде исходное состояние обмена веществ у пациента с МКБ с выявленными 
МФР и динамику изменения показателей обмена веществ 
в результате проводимого лечения. Графическое отображение МФР и ОИП больного МКБ 
позволяет оценить степень выявленных нарушений и их динамику в процессе 
профилактического лечения и вносить в лечебную схему необходимые коррекции, также 
выбираемые по особому алгорит\-му. Выявленные при первичном обследовании МФР и ОИП 
служат основой для програм\-мно\-го выбора схем коррекции метаболических нарушений и 
предупреждения рецидивов МКБ. Коррекция включает в себя лечебные мероприятия, прием 
специальных фармпрепаратов, рекомендации по модификации диеты и образа жизни.
\item В~аналитической системе <<Мегалит>> пред\-усмат\-ри\-ва\-ет\-ся возможность ее обучения и 
настройки на основе получаемых новых данных о результатах лечения пациента (пациентов) 
на \mbox{каждом} этапе наблюдения.
\item Предусмотреть в программной реализации алгоритмов экспертного модуля: 
\begin{itemize}
\item
алгоритм оценки эффективности выбранной схемы лечения;
\item
алгоритм принятия решения по дальнейшему лечению;
\item
алгоритм поиска и выбора рациональной схемы профилактического лечения.
\end{itemize}
\item Оценить возможности разработки метода корректировки параметров подсистемы 
диагностирования и лечения на основе анализа вновь поступающих данных (обратная связь).
\item Разработанные аналитические методы и алгоритмы, реализованные в составе опытного 
образца аналитической системы <<Мегалит>>, должны пройти апробацию и тестирование в 
реальных клинических условиях. По результатам применения опытного образца должны быть 
сформулированы рекомендации по его совершенствованию и развитию.
\end{enumerate}

\begin{figure*} %fig1
   \vspace*{1pt}
 \begin{center}
 \mbox{%
 \epsfxsize=143.69mm
 \epsfbox{such-1.eps}
 }
 \end{center}
 \vspace*{-6pt}
\Caption{Структура ИААС}
\end{figure*}

\section{Основные подходы к~созданию информационно-аналитической
автоматизированной системы <<Мегалит>> и~их~реализация}
      Основные функции ИААС:
      \begin{itemize}
\item сбор и формализация данных, включая ведение реестра пациентов, системы словарей и 
справочников;
\item поддержка принятия решения по назначению и сбор данных диагностических 
исследований;
\item первичный и ретроспективный анализ тестов;
\item поддержка принятия решения по выбору схемы лечения, оценка эффективности схемы 
лечения;
\item поддержка принятия решения по дальнейшему лечению;
\item поиск и поддержка принятия решения по выбору рациональной схемы 
профилактического лечения.
\end{itemize}

      Информационно-аналитическая автоматизированная система
       <<Мегалит>> включает в себя подсистемы:
      \begin{itemize}
\item администрирования;
\item регистрации пациентов и сбора данных анкет, анамнеза;
\item ведения лингвистического обеспечения;
\item первичного обследования;
\item диагностических исследований;
\item экспертный модуль (поддержки процессов лечения).
\end{itemize}

Структурная схема ИААС <<Мегалит>> пред\-став\-ле\-на на рис.~1.


\begin{figure*}[b] %fig2
   \vspace*{1pt}
 \begin{center}
 \mbox{%
 \epsfxsize=161.589mm
 \epsfbox{such-2.eps}
 }
 \end{center}
 \vspace*{-6pt}
\Caption{Пирамида анализа данных}
\end{figure*}


      Для обеспечения возможности коллективной работы по формированию БД
системы и многопользовательского режима работы с ее аналитическим модулем она 
проектируется в виде веб-сай\-та, доступного авторизованным пользователям в сети Интернет. 
Основные базовые функции информационного сайта должны быть реализованы 
общесистемным функционалом его платформы. 
%
Таким образом, 
в про\-грам\-мно-тех\-но\-ло\-ги\-че\-ской платформе должны быть заложены следующие функции:
      \begin{enumerate}[(1)]
\item выполнение приложений~--- позволяет легко разрабатывать, развертывать различные 
приложения и управлять ими;
\item возможность совместной работы~--- позволяет отдельным пользователям и крупным 
организациям объединить свои ресурсы и работать вместе через Интернет;
\item управление содержимым~--- придает гибкость производству и управлению 
отдельными веб-уз\-ла\-ми, позволяя поставлять конечному пользователю 
приспособленное под него (персонифицированное) содержимое сайта;
\item управление пользователями~--- позволяет организации управлять пользователями, 
ресурсами и безопасностью внутри и вне системы сетевой защиты, а также предоставлять 
канал для внешних связей и проведения электронных транзакций;
\item контроль и управление про\-из\-во\-ди\-тель\-ностью~--- позволяет улучшать качество 
пользовательского интерфейса, обеспечивая:
\begin{itemize}
\item управление знаниями~--- помогает объединять внутреннюю и внешнюю 
информацию и предоставлять информацию, основанную на контекстной 
концепции;
\item поддержку поиска~--- обеспечивает клиента доступом к широкому спектру 
источников информации как внутри, так и вне сайта;
\item безопасность~--- защиту данных, приложений и транзакций;
\item стандартный www-до\-ступ к сайту~--- для технического обеспечения 
функционирования его содержимого.
\end{itemize}
\end{enumerate}
      
      Определяющими характеристиками веб-ре\-шений являются масштабируемость, 
доступность, надежность, защита данных от несанкционированного доступа, транзакционная 
целостность и распространение.
      
      Важной особенностью платформы является то, что она объединяет все необходимые 
модули, которые позволяют выполнять практически любую работу, связанную с созданием и 
обновлением сайта специалистом предметной области. 
      
      В качестве языка программирования выбран один из самых современных языков~--- 
C\#. ASP.NET~--- технология, которая является частью .NET и используется для разработки 
ин\-тер\-нет-ори\-ен\-ти\-ро\-ван\-но\-го программного обеспечения и ин\-тер\-нет-сай\-тов. 

Для работы ин\-тер\-нет-сай\-тов используется связка: операционная система Windows Server 
2008\;+\;ин\-тер\-нет-сер\-вер IIS~7.0\;+\;СУБД Ms SQL Server~2008.

\section{Концепция экспертного модуля системы}

\subsection{Основные подходы к~использованию статистических методов анализа данных 
в~урологии}

      Клинические БД содержат большое количество информации о пациентах и их 
заболеваниях. Скрытые (латентные) связи и структуры в этих данных могут быть источником 
новых медицинских знаний. К~сожалению, немногие из существующих технологий анализа 
данных оказываются непосредственно применимыми и действенными при обнаружении и 
описании этих латентных знаний, но, безусловно, универсальной из них является технология 
на принципах \textit{Data Mining}~--- извлечение скрытой информации из уже накопленных и 
пополняемых сведений об объекте исследования. Эта и ряд других сформировавшихся 
технологий, ориентированных на анализ массивов данных, терминологически пересекаются 
или оказываются взглядом на одном и том же, но с разных точек зрения (краткое освещение 
данного вопроса дано в~[17, разд.~1]). В~первую очередь речь идет о следующих 
подходах:
      \begin{itemize}
\item разведочный анализ данных~--- Exploratory Data Analysis (EDA);
\item извлечение скрытой информации из данных~--- Data Mining 
(DM);
\item обнаружение знаний в данных~--- Knowledge Discovery in Databases (KDD);
\item машинное обучение~--- Machine Learning (ML).
\end{itemize}

      Таким образом, обнаружение в данных ранее не известных, нетривиальных, 
практически полезных и доступных интерпретации знаний, необходимых для поддержки 
принятия решений в различных сферах человеческой деятельности, составляет суть DM. 
Говоря далее об анализе данных, будем понимать при этом цели, задачи, технологии, методы и 
алгоритмы, присущие DM. 
      
      На рис.~2 схематично изображена иерархия содержательной стороны анализа данных. 
Вертикальная стрелка слева показывает направление роста отдельных характеристик задач 
анализа данных в зависимости от их уровня. Примеры постановок практических задач 
приведены справа. Надо понимать, что <<восхождение>> по пирамиде анализа данных 
должно обеспечиваться обязательным существенным ростом объема используемой 
информации (данных и предположений об объектах исследования), а также глубиной 
проработки вопроса о качестве предлагаемых решений.


      
      \textbf{Основные принципы анализа данных.} Среди методов, которые 
использовались при решении проб\-ле\-мы обучения в ML, те, которые представляют\linebreak 
наибольший интерес при анализе данных (снижение размерности, оценивание распределения 
данных, регрессионный анализ, классификация, клас\-те\-ри\-за\-ция), теперь все вместе 
упоминаются как статисти\-ческое обучение. Проблема обучения делится на различные 
категории: две из них, наиболее близкие к статистике, суть контролируемое обучение или 
обучение с <<учителем>> и не\-конт\-ро\-ли\-ру\-емое, без <<учителя>>.
      
      Одна из самых важных задач в анализе данных состоит в том, чтобы оценить качество 
полученных решений, в частности точность предложенного прогноза (например, качество 
построенного классификатора). В~качестве меры точности прогноза обычно используется 
ошибка прогноза. Простейшая оценка ошибки прогноза строится с помощью тех же данных, 
которые используются для построения модели (такой вариант оценки называют самооценкой, 
оценкой переподстановки). Понятно, что в результате сформируется чрезмерно 
оптимистический взгляд на точность прогноза. 
      
      Очевидный способ улучшения состоит в обобщении: оценивать точность прогноза с 
помощью данных, независимых от тех, которые использовались для подгонки модели. 
Получить подобные независимые данные можно путем сбора новых данных. Если это 
невозможно, то имеет смысл разделить исходные данные на части и воспользоваться ими для 
решения самостоятельных задач. Обычная практика заключается в следующем: если набор 
данных достаточно велик, то необходимо использовать случайный механизм для разделения 
данных на два непересекающихся и независимых набора: 
      \begin{enumerate}[(1)]
\item данные для обучения, которые можно использовать для предварительного контроля 
данных, для формирования моделей;
\item тестирующие данные, которые будут использоваться для оценки 
качества построенной модели.
      \end{enumerate}
      
      Альтернативные методы расщепления данных для того, чтобы оценить тестовую 
ошибку, основаны на перепроверке~[16] и бут\-стреп-ме\-то\-де~[17].
     
     Суть вероятностной модели бутстреп-метода в данном случае состоит в следующем. 
Предположим, что по выборке $x\hm=(x_1,\ldots ,x_N)$ данных лабораторных исследований 
из распределения $F(u)$ оценивается значение $\vartheta\hm=\vartheta(F)$ некоторого 
функционала (например, классификатора заболеваний), заданного на семействе~$\mathbf{F}$. 
Качество оценки $\vartheta^*(X)$ характеризуется величиной
     $$
     R(\vartheta^*(X),\vartheta(F))=E_F\{L(\vartheta^*(X),\vartheta(F))\}\,,
     $$
где $L(\vartheta^*(X),\vartheta(F))$~--- потери от принятия оценки $\vartheta^*(X)$ вместо 
неизвестного значения $\vartheta(F)$. Бут\-стреп-ме\-тод позволяет оценить $ 
R(\vartheta^*(X),\vartheta(F))$ с помощью замены распределения~$F$ его некоторой оценкой 
$F^B$ и вычисления статистики $\vartheta^*$ по выборке $x^B$ объемом~$N$ из~$F^B$. 
Совокупность $x^B$ называется бут\-стреп-вы\-бор\-кой, статистика $\vartheta^*(x^B)$~--- 
бут\-стреп-реа\-ли\-за\-ци\-ей~$\vartheta^*$. 
     
     Условное распределение
     \begin{multline*}
     \mathrm{Pr}\left\{ \vartheta^*\left( X^B\right) <u\vert x_1,\ldots , x_N\right\} = {}\\
     {}=
     \int\limits_{\{y:\ \vartheta^*(y)<u\}} dF^B (y_1)\cdots dF^B(y_N)
     \end{multline*}
является бут\-стреп-оцен\-кой функции распределения $\mathrm{Pr}\left\{ 
\vartheta^*(X)<u\right\}$ статистики~$\vartheta^*$. 
     
     Процедура выбора оценки $F^B$ для~$F$ мотивируется наличием априорной 
информации. В~параметрической ситуации, когда $\mathbf{F}\hm= \left\{ F_\lambda, \, 
\lambda\in \Lambda\right\}$, оценка $F^B$ часто оказывается результатом подстановки 
вместо~$\lambda$ некоторой оценки~$\lambda^*$, т.\,е.\ $F^B\hm=F_{\lambda^*}$. 
{\looseness=1

}

Другая  ситуация относится к области непараметрической статистики. Здесь $F^B$ обычно 
оказывается эмпирической функцией распределения, т.\,е.\ каждому наблюденному значению 
(элементу исходной выборки) приписывается вероятность $1/N$. Бут\-стреп-вы\-бор\-ки тогда 
подчиняются условному полиномиальному распределению, сосредоточенному на  $x_1,\ldots , 
x_N$. 
     
     Наиболее трудную часть бутстреп-метода со\-став\-ля\-ет нахождение распределения 
$\vartheta^*(X^B)$, для чего применяются три приема:
    \begin{enumerate}[(1)]
\item прямое теоретическое вычисление;
\item аппроксимация с помощью метода статистических испытаний;
\item аппроксимация с помощью аналитических методов (например, используя разложение в 
ряд Тейлора).
\end{enumerate}

     Прямое теоретическое вычисление распределения $\vartheta^*(X^B)$ может 
осуществляться либо аналитическим путем, либо путем непосредственного перечис\-ле\-ния 
     бут\-стреп-вы\-бо\-рок и подсчета соответствующих вероятностей. Если оба приема 
недоступны (первый из-за аналитических сложностей, второй из-за вычислительных), то 
приходится прибегать к методу статистических испытаний, т.\,е.\ к повторению экспериментов 
по случайному формированию бут\-стреп-вы\-бор\-ки $x^B$ и подсчету значения 
$\vartheta^*(x^B)$.
     
     Следует обратить внимание на реальные возможности бут\-стреп-ме\-то\-да: он не 
позволяет получить новую информацию о наблюдаемых объектах, его назначение~--- 
сформировать объективное представление о свойствах использованных процедур анализа 
данных.
      
      В аналитической системе <<Мегалит>> накапливаются данные следующих типов:
      \begin{enumerate}[1.]
\item Неформализованные (неструктурированные), представленные в виде текста (например, 
текст назначения врача).
\item Формализованные:
\begin{enumerate}[{2.}1.]
\item Качественные:
\begin{enumerate}[{2.1.}1.]
\item Измеренные по шкале наименований (например, пол пациента).
\item Измеренные по порядковой шкале (например, порядковый номер сезона, когда 
обследовался пациент).
\end{enumerate}
\item Количественные:
\begin{enumerate}[{2.2.}1.]
\item Измеренные по одной из соответствующих шкал и при\-ни\-ма\-ющие значения из 
небольшого \mbox{набора} числовых значений (например, дата взятие анализов или 
количество обнаруженных у пациента камней).
\item Измеренные по одной из соответствующих шкал и принимающие значения в виде 
действительных чисел (например, уровень кальция в анализе крови пациента).
\end{enumerate}
\end{enumerate}
\end{enumerate}
      
      Приведенный систематизированный перечень встречающихся типов данных требует 
привлечения разнообразного арсенала средств, таких как лингвистический анализ (п.~1), 
статистический анализ категориальных данных (п.~2.1.1), ранговые процедуры 
(п.~2.1.2), статистический анализ на основе моделей дискретных и непрерывных 
распределений (пп.~2.2.1 и~2.2.2).
      
      Ошибки есть во всех видах БД; к сожалению, встречаются они и в данном 
случае. В~различных прикладных областях накоплен опыт (см., в частности,~[18]), 
позволяющий привести типичный перечень источников ошибок: фальсификация, неполнота, 
несогласованность, дублирование. 
      
      Те ошибки, которые легко обнаружить, вероятнее всего можно найти на стадии 
<<очистки>> данных, более же скрытые, неочевидные могут быть обнаружены только при 
анализе данных. <<Очистка>> данных обычно происходит, когда данные получены и прежде, 
чем они сохраняются в формате только для чтения в хранилище данных. В~частности, должны 
быть исключены ошибки, при которых переменные принимают значения, противоречащие 
естественным ограничениям (например, при описании химического состава камней значения 
отдельных переменных не могут превосходить 100\%). Доля подобных грубых ошибок в 
медицинских исследованиях может превышать~10\%. 
      
      Ошибки недопустимости значений должны быть описаны с помощью логических 
выражений, истинность которых проверяется на этапе <<очистки>>. В~случаях, когда их не 
удается исправить автоматически или автоматизированно, результат должен помечаться 
специальным образом. 
      
      Для данных, уже хранящихся в БД и явля\-ющих\-ся объектом анализа, могут 
быть характерны сле\-ду\-ющие проблемы: 
\begin{itemize}
\item наличие аномальных наблюдений (значения, которые 
существенно отличаются от основной массы наблюдений);
\item пропуски в данных;
\item малочисленность данных (ситуация, когда количество переменных превышает число 
наблюдений).
\end{itemize}
      
      Таким образом, статистический анализ конкретных данных является многоэтапным 
процессом, включающим планирование статистического исследования, организацию сбора 
необходимых статистических данных, первичное описание данных, оценивание характеристик 
данных, проверку статистических гипотез, анализ полученных решений, формулировку 
выводов, составление итоговых документов. 
      
      В основе принципов построения статистического вывода относительно данных лежат 
следующие положения:
      \begin{itemize}
\item при выборе семейства вероятностных распределений, описывающих данные, 
существенную роль играет предварительный анализ данных; последующий итеративный 
процесс уточнения априорных предположений направлен на построение модели, являющейся 
достаточно реалистичной и позволяющей строить содержательные выводы;
\item при построении методов анализа наряду с постановкой задач разработки оптимальных 
процедур и попыткой их решения следует не пренебрегать разумными подходами к созданию 
ка\-ких-ли\-бо процедур с последующим обязательным анализом предлагаемых решений; 
\item завершающим этапом построения методов анализа должен быть количественный или 
качественный анализ влияния на предлагаемые реше\-ния отклонений от априорных 
предположений, при этом исследование качества полученных решений реальнее всего 
проводить с по\-мощью бут\-стреп-ме\-тода.
\end{itemize}

\subsection{Принципиальные возможности создания экспертного модуля системы 
<<Мегалит>>}

      Повседневная деятельность врача требует решения задач интерпретации, диагностики, 
контроля и прогнозирования, т.\,е.\ таких задач, которые могут быть решены с помощью 
систем поддержки принятия решений. Медицина представляет одну из областей человеческой 
деятельности, где знания специалистов трудно формализуемы, однако разработка 
диагностических медицинских систем в настоящее время является актуальной задачей.
      
      При создании экспертного модуля системы <<Мегалит>>, предназначенного для 
поддержки принятия решения по диагностике заболевания, предполагается:
      \begin{itemize}
\item разработать систему представления медицинских знаний (с использованием данных 
анкет, анамнеза, данных инструментальных и лабораторных методов исследования);
\item разработать алгоритм механизма логического вывода (выполнение диагностики типа 
литогенного нарушения обмена веществ; выбора адекватной схемы лечения, оценки 
эффективности заданной схемы лечения с возможностью ее коррекции при дальнейшем 
мониторинге пациента).
\end{itemize}
      
      Система представления медицинских знаний позволяет выделять значимые для 
принятия врачебного решения или постановки медицинского диагноза данные (качественные 
или количественные). Так, анализ качественных данных анамнеза, анкетных данных позволяет 
сделать заключение о силе влияния наследственных, средовых и социальных факторов риска 
развития МКБ; потенциальной активности процесса камнеобразования. Этой же цели служат 
качественные и количественные данные, полученные при инструментальном/лабораторном 
(рентгенологическом, микробиологическом, ультразвуковом или антропометрическом) 
обследовании пациента.
      
      Большой массив количественных данных в виде числовых значений показателей 
получают при биохимическом исследовании. Именно он является основным объектом 
алгоритмизации при разработке экспертного модуля системы <<Мегалит>>. Применение 
этого модуля предназначено для объективной и более точной диагностики метаболического 
литогенного синдрома, выбора адекватной схемы лечения, качественной оценки результатов 
лечения, коррекции лечебной схемы на основе полученных данных в целях выбора 
оптимального лечебного воздействия на нарушенный обмен веществ у пациента.
      
      Учитывая, что указанные процессы являются алгоритмизуемыми, можно полагать, что 
принципиальные возможности создания экспертного модуля для системы <<Мегалит>> 
имеются.

\subsection{Качественное описание алгоритмической базы экспертного~модуля }

      Качественное описание алгоритмической базы включает: постановку задачи, описание 
входных и выходных данных, описание вход\-ных/вы\-ход\-ных форм пользовательского 
интерфейса, описание событий и реакций системы в рамках поддержки процесса оценки 
эффективности.
      
      \subsubsection*{Алгоритм оценки эффективности заданной схемы лечения}
      
      \paragraph*{Постановка задачи.} Качественно и количественно оценить эффективность 
применения выбранной схемы лечения (с выводом о продолжении ее использования в 
лечении; ее модификации в той или иной степени; замены на другую схему лечения).
      
      \paragraph*{Входные данные.} Входными данными служат те количественно измененные 
биохимические показатели, которые на предыдущем этапе про\-грам\-мно\-го анализа были 
определены (диагностированы) как характерные для данного метаболического синдрома. 
      
      \paragraph*{Выходные данные.} Выходными данными служат количественные значения 
биохимических признаков пролеченного метаболического синдрома, полученные при 
лабораторном исследовании после курса лечения. Эти данные должны быть про\-грам\-мно 
проанализированы в сравнении с их исходными (до начала лечения) значениями. 
Используется \textit{алгоритм <<Оценка качества лечебного эффекта>>}, который 
предполагает следующие варианты вывода о качестве лечения:
      \begin{itemize}
\item <<отсутствие эффекта>>;
\item <<слабо выраженный положительный эффект>>;
\item <<выраженный положительный эффект>>;
\item <<слабо выраженный отрицательный эффект>>; 
\item <<выраженный отрицательный эффект>>.
\end{itemize}
      
      \paragraph*{Выходные формы пользовательского интерфейса.} Выходные формы 
пользовательского интерфейса при этом могут быть представлены в виде таб\-ли\-цы со 
значениями биохимических признаков метаболического синдрома до и после лечения. 
Возможна опция отображения исходных данных в виде диаграммы или графика.

      \subsubsection*{Алгоритм принятия решения по дальнейшему лечению}
      
      \paragraph*{Постановка задачи.} Построить алгоритмические правила, позволяющие 
пользователю сделать вывод и принять решение об использовании данной схемы в 
дальнейшем лечении; модификации схемы в той или иной степени; замены данной схемы на 
другую схему лечения.
      
      \paragraph*{Входные данные:}
      \begin{itemize}
\item данные сравнительного анализа численных биохимических величин до и после лечения;
\item данные качественной оценки лечебного эффекта, получаемые в результате обработки 
данных сравнительного анализа численных биохимических величин до и после лечения.
     \end{itemize}
     
     При формировании и сборе данных первого типа выполняется процедура сравнения 
достигнутых в результате лечения величин значимых для данного метаболического синдрома 
показателей с их исходными значениями, диагностированными до начала лечения в ходе 
биохимического лабораторного исследования.
     
     Данные второго типа являются качественными, производными от данных первого типа. 
Эти данные представляют собой возможные варианты вывода о качестве лечения.
     
     \paragraph*{Выходные данные.} Выходными данными алгоритма принятия решения по 
дальнейшему лечению служат установленные типы рекомендаций по применявшейся схеме 
лечения: 
     \begin{itemize}
\item сохранение схемы без изменений и продолжение лечения;
\item модификация схемы трех степеней выраженности 
(незначительная, умеренная, существенная);
\item отказ от применения данной схемы и выбор новой схемы лечения.
\end{itemize}

\section{Перспективы развития информационно-аналитической автоматизированной
системы~<<Мегалит>>}

      Таким образом, разработана методологическая и техническая база для экспертной 
системы комплексной диагностики и профилактического \mbox{лечения} пациентов с МКБ. Учитывая 
не только медицинскую, но и социальную актуальность проб\-ле\-мы МКБ, а также трудности 
принятия врачебного решения в выборе адекватной тактики противорецидивного лечения 
этого заболевания, следует считать целесообразным создание технологий расширения и 
совершенствования функционала экспертного модуля системы <<Мегалит>> на основе 
анализа вновь поступающих данных с использованием принципа обратной связи.
      
      На следующих этапах планируется работа по оптимизации практического применения 
ИААС <<Мегалит>> в 
клинической урологии.
      
      Предложенные математические методы и алгоритмы найдут применение при 
дальнейшем развитии фундаментальных научных исследований в области разработки 
математических методов моделирования медико-биологических систем, а также при создании 
необходимого математического инструментария.
    %  
      В первую очередь это касается постановки развернутого диагноза, наиболее полно 
отражающего особенности метаболического типа конкретного больного (обследуемого), 
специфику функционального состояния почек и мочевых путей пациента. При этом также 
учитывается влияние различных модифицирующих факторов, таких как наличие или 
отсутствие инфекции и степени ее вы\-ра\-жен\-ности, воздействие социальных факторов, 
факторов питания, наследственности и~др. 
      
      Внедрение и практическое использование сис\-те\-мы <<Мегалит>> позволит 
сформировать представительный набор данных, на основе которого разработать методы 
выбора оптимальной лечебной тактики. Таким образом, конечные пользователи получат 
возможность дистанционной диагностики метаболических литогенных синдромов у пациента, 
оценки степени риска развития МКБ, выбора адекватных терапевтических схем лечения МКБ 
и/или профилактики рецидивов камнеобразования, ввода данных о пациенте в единый банк 
данных для последующего мониторинга и~др. 

{\small\frenchspacing
{%\baselineskip=10.8pt
\addcontentsline{toc}{section}{Литература}
\begin{thebibliography}{99}

\bibitem{1-su} %1
\Au{Ramello, A., Vitale C., Marangella D.} Epidemiology of nephrolithiasis~// J.~Nephrol., 2000. 
Vol.~13. Suppl.~3. P.~45--50.

\bibitem{4-su} %2
\Au{Trinchieri A., Coppi F., Montanari~E., Del Nero~A., Zanetti~G., Pisani~E.} Increase in the 
prevalence of symptomatic upper urinary tract stones during the last ten years~// Eur. Urol., 2000. 
Vol.~37. P.~23--25.

\bibitem{2-su} %3
\Au{Pearle M.\,S., Calhoun E.\,A., Curhan~G.\,C.} Urologic diseases in America project: 
Urolithiasis~// J.~Urology, 2005. Vol.~173. P.~848--857. 

\bibitem{3-su} %4
\Au{Lieske J.\,C., Pena de la Vega~L.\,S., Slezak~J.\,M., Bergstralh~E.\,J., Leibson~C.\,L., 
Ho~K.\,L., Gettman~M.\,T.} Renal stone epidemiology in Rochester, Minnesota: An update~// Kidney 
Int., 2006. Vol.~69. No.\,4. P.~760--764.


\bibitem{6-su} %5
\Au{Johnson C.\,M., Wilson D.\,M., O'Fallon~W.\,M., Malek~R.\,S., Kurland~L.\,T.} 
Renal stone epidemiology: A~25-year study in Rochester, Minnesota~// Kidney Int., 1979. Vol.~16. 
No.\,5. P.~624--631.

\bibitem{5-su} %6
\Au{Yasui T., Iguchi M., Suzuki~S., Kohri~K.} Prevalence and epidemiological characteristics of 
urolithiasis in Japan: National trends between 1965 and 2005~// Urology, 2008. Vol.~71. No.\,2. 
P.~209--213.

\bibitem{7-su}
Заболеваемость населения России в 2003~году: Статистические материалы.~--- М., 2004 
(электронная версия МЗ и СР РФ и ЦНИИ организации и информатизации здравоохранения 
МЗ и СР РФ). 
{\sf http:// www.minzdravsoc.ru/docs/mzsr/stat/17}.
\bibitem{8-su}
\Au{Аполихин О.\,И., Сивков А.\,В., Солнцева Т.\,В., Комарова~В.\,А.}
Анализ урологической заболеваемости в Российской Федерации в 2005--2010~годах~//
Экспериментальная и клиническая урология, 2012. №\,2. C.~4--12.
{\sf http://ecuro.ru/article/analiz-urologicheskoi-zabolevaemosti-v-rossiiskoi-federatsii-v-2005-2010-godakh}.
%\bibitem{9-su}
%Заболеваемость населения России в 2007~году: Статистические материалы.~--- М., 2008 
%(электронная версия МЗ и СР РФ и ЦНИИ организации и информатизации здравоохранения 
%МЗ и СР РФ). {\sf http:// www.minzdravsoc.ru/docs/mzsr/stat/27}.
\bibitem{11-su} %9
\Au{Routh J.\,C., Graham D.\,A., Nelson~C.\,P.} Epidemiological trends in pediatric urolithiasis at 
United States freestanding pediatric hospitals~// J.~Urology, 2010. Vol.~184. No.\,3. P.~1100--1104.

\bibitem{10-su} %10
\Au{Голованов С.\,А., Дрожжева В.\,В.}
Кристаллообразующая активность мочи при оксалатном уролитиазе~//
Экспериментальная и клиническая урология, 2010. №\,2. C.~24--29.
{\sf http://ecuro.ru/ article/kristalloobrazuyushchaya-aktivnost-mochi-pri-oksalatnom-urolitiaze}.
\bibitem{12-su} %11
\Au{Bonny O., Rubin A., Huang~Ch.-L., Frawley~W.\,H., Pak~C.\,Y.\,C., Moe~O.\,W.} Mechanism 
of urinary calcium regulation by urinary magnesium and pH~// J.~Am. Soc. Nephrol., 2008. Vol.~19. 
No.\,8. P.~1530--1537.
\bibitem{13-su} %12
\Au{Дюк В.\,А., Эмануэль В.\,Л.} Информационные технологии в 
ме\-ди\-ко-био\-ло\-ги\-че\-ских исследованиях.~--- СПб.: Питер, 2003. 525~с.
\bibitem{14-su} %13
\Au{Liew P.\,L., Lee Y.\,C., Lin~Y.\,C., \textit{et al}.} Comparison of artificial neural networks with 
logistic regression in prediction of gallbladder disease among obese patients~// Digest. Liver Dis., 2007. 
Vol.~39. No.\,4. P.~356--362.
\bibitem{15-su} %14
\Au{Bassi P., Sacco E., De Marco~V., \textit{et al}.} Prognostic accuracy of an artificial neural 
network in patients undergoing radical cystectomy for bladder cancer: A~comparison with logistic 
regression analysis~// BJU Int., 2007. Vol.~99. No.\,5. P.~1007--1012.
\bibitem{16-su} %15
\Au{Stephan C., Xu C., Finne~P., \textit{et al}.} Comparison of two different artificial neural 
networks for prostate biopsy indication in two different patient populations~// J.~Urology, 2007. 
Vol.~70. No.\,3. P.~596--601.
%\bibitem{17-su} 
%\Au{Chun F.\,K., Karakiewicz P.\,I., Briganti~A., \textit{et al}.} A~critical appraisal of logistic 
%regression-based nomograms, artificial neural networks, classification and regression-tree models, 
%look-up tables and risk-group stratification models for prostate cancer ~/ BJU Intern., 2007. Vol.~99. 
%No.\,4. P.~794--800.


\bibitem{19-su} %16
\Au{Stone M.} Cross-validatory choice and assessment of statistical predictions (with discussion)~// 
J.~Roy. Stat. Soc. B, 1974. Vol.~36. P.~111--147.

\bibitem{18-su} %17
\Au{Efron B.} Bootstrap methods: Another look at the jackknife~// Ann. Stat., 1979. Vol.~7. 
P.~1--26.

\bibitem{21-su} %18
\Au{Izenman A.\,J.} Modern multivariate statistical techniques.~--- Springer, 2008. 731~p. 
%\bibitem{20-su} %20
%\Au{Breiman L.} The 1991 census adjustment: Undercount or bad data~// Stat. Sci., 1994. 
%Vol.~9. P.~458--475.


\end{thebibliography} } }

\end{multicols}

\hfill{\small\textit{Поступила в редакцию 17.04.13}}
%\vspace*{12pt}

%\hrule

%\vspace*{2pt}

%\hrule

\newpage

\def\tit{THE INFORMATION-ANALYTICAL COMPUTER SYSTEM ``MEGALITH'' 
IN~OPTIMIZATION OF~THE~DIAGNOSIS AND~TREATMENT OF~UROLITHIASIS}

\def\titkol{The information-analytical computer system ``Megalith'' 
in~the~field of~urology}

\def\aut{M.\,P.~Krivenko$^1$, S.\,A.~Golovanov$^2$, P.\,A.~Savchenko$^1$, A.\,V.~Sivkov$^2$, 
 and~A.\,P.~Suchkov$^1$}
 
 \def\autkol{S.\,A.~Golovanov, M.\,P.~Krivenko, P.\,A.~Savchenko, et al.}


\titel{\tit}{\aut}{\autkol}{\titkol}

\vspace*{-12pt}


\noindent
$^1$Institute of Informatics 
Problems, Russian Academy of Sciences, Moscow 119333, Russian Federation\\
\noindent $^2$Research Institute of Urology, Moscow 105425, Russian Federation

\vspace*{12pt}

\def\leftfootline{\small{\textbf{\thepage}
\hfill INFORMATIKA I EE PRIMENENIYA~--- INFORMATICS AND APPLICATIONS\ \ \ 2013\ \ \ volume~7\ \ \ issue\ 4}
}%
 \def\rightfootline{\small{INFORMATIKA I EE PRIMENENIYA~--- INFORMATICS AND APPLICATIONS\ \ \ 2013\ \ \ volume~7\ \ \ issue\ 4
\hfill \textbf{\thepage}}}

\Abste{In this article, that is the first of an expected series of scientific publications, the results of 
research on automation of the information and analytical processes of the urolithic disease (ULD) 
survey, diagnosis, and treatment are discussed. A significant role in creating the systems of ULD 
diagnostics has the development of information technologies for clinical data collection and 
formation of specialized databases. The possibility of creation and the ways of realization of 
information-analytical computer system of collection, storage, and processing of the clinical data of 
patients examination, as well as programming decision-making processes in the diagnosis ULD and 
the choice of schemes of treatment and prevention of this disease has been studied. The developed 
mathematical methods and algorithms may be applied to the further fundamental scientific researches 
in the field of development of mathematical methods of medical and biological systems modeling; 
besides, they may be applied for necessary mathematical tools creation.}

\KWE{informational-analytical system; urology; computer diagnostics; treatment 
scheme; scheme of prevention} 

\DOI{10.14357/19922264130409}

%\Ack
%\noindent
%?????

\vspace*{3pt}

  \begin{multicols}{2}

\renewcommand{\bibname}{\protect\rmfamily References}
%\renewcommand{\bibname}{\large\protect\rm References}

{\small\frenchspacing
{%\baselineskip=10.8pt
\addcontentsline{toc}{section}{References}
\begin{thebibliography}{99}


\bibitem{1-su-1}
\Aue{Ramello, A., C.~Vitale, and D.~Marangella}. 2000. Epidemiology of nephrolithiasis. 
\textit{J.~Nephrol.} 13(Suppl.~3):45--50.

\bibitem{4-su-1} %2
\Aue{Trinchieri, A., F.~Coppi, E.~Montanari, A.~Del Nero, G.~Zanetti, and E.~Pisani}. 2000. 
Increase in the prevalence of symptomatic upper urinary tract stones during the last ten years. 
\textit{Eur. Urol.} 37:23--25.

\bibitem{2-su-1} %3
\Aue{Pearle, M.\,S., E.\,A.~Calhoun, and G.\,C.~Curhan}. 2005. Urologic diseases in America 
project: Urolithiasis. \textit{J.~Urology} 173:848--857. 
\bibitem{3-su-1} %4
\Aue{Lieske, J.\,C., L.\,S.~Pena de la Vega, J.\,M.~Slezak, E.\,J.~Bergstralh; C.\,L.~Leibson, 
K.\,L.~Ho, and M.\,T.~Gettman}. 2006. Renal stone epidemiology in Rochester, Minnesota: An 
update. \textit{Kidney Int.} 69(4):760--768.


\bibitem{6-su-1} %5
\Aue{Johnson, C.\,M., D.\,M.~Wilson, W.\,M.~O'Fallon, R.\,S.~Malek, and L.\,T.~Kurland}. 
1979. Renal stone epidemiology: A~25-year study in Rochester, Minnesota. \textit{Kidney 
Int.} 16(5):624--631.

\bibitem{5-su-1} %6
\Aue{Yasui, T, M.~Iguchi, S.~Suzuki, and K.~Kohri}. 2008. Prevalence and epidemiological 
characteristics of urolithiasis in Japan: National trends between 1965 and 2005. \textit{Urology}  
 71(2):209--213.

\bibitem{7-su-1} %7
Russian Ministry of Health:
Central Research Institute of Organization and Informatization of Population.
2004. Zabolevaemost' naseleniya Rossii v 2003 godu: Sta\-ti\-sti\-che\-skie materialy [Morbidity of 
population of Russia in 2003: Statistical materials]. Мoscow.  Electronic version.
{\sf http://www.minzdravsoc.ru/docs/mzsr/stat/17}.

\bibitem{8-su-1}
\Aue{Apolikhin,~O.\,I., A.\,V.~Sivkov, T.\,V.~Solntseva, and V.\,A.~Komarova.}
2012. Analysis of urological morbidity in the Russian Federation within the period of 2005--2010.
\textit{Experimental and Clinical Urology} 2:4--12.
{\sf http://ecuro.ru/en/article/analysis-urological-morbidity-russian-federation-within-period-2005-2010}.


\bibitem{11-su-1} %9
\Aue{Routh, J.\,C., D.\,A.~Graham, and C.\,P.~Nelson}. 2100. Epidemiological trends in 
pediatric urolithiasis at United States freestanding pediatric hospitals. \textit{J.~Urology} 
184(3):1100--1104.

%\bibitem{9-su-1}
%Russian Ministry of Health:
%Central Research Institute of Organization and Informatization of Population.
%2008. Zabolevaemost' naseleniya Rossii v 2007 godu: Sta\-ti\-sti\-che\-skie materialy [Morbidity of 
%population of Russia in 2007: Statistical materials]. Мoscow. Electronic version.
%{\sf http://www.minzdravsoc.ru/docs/mzsr/stat/27}.
\bibitem{10-su-1} %10
\Aue{Golovanov, S.\,A., and V.\,V.~Drozhzheva}.
2010. Crystal formation activity of urine in oxalate urolithiasis.
\textit{Experimental and Clinical Urology} 2:24--29.
{\sf http://ecuro.ru/en/article/crystal-formation-activity-urine-oxalate-urolithiasis}.



\bibitem{12-su-1} %12
\Aue{Bonny, O., A.~Rubin, Ch.-L.~Huang, W.\,H.~Frawley, C.\,Y.\,C.~Pak, and 
O.\,W.~Moe}. 2008. Mechanism of urinary calcium regulation by urinary magnesium and $pH$. 
\textit{J.~Am. Soc. Nephrol.} 19(8):1530--1537.
\bibitem{13-su-1}
\Aue{Djuk, V.\,A., and V.\,L.~Jemanujel'}. 2003. \textit{Informatsionnye tekhnologii v 
mediko-biologicheskikh issledovaniyakh} [\textit{Information technologies in medical and 
biological researches}]. St.\ Petersburg, Russia: Piter, 2003. 525~p.
\bibitem{14-su-1}
\Aue{Liew, P.\,L., Y.\,C.~Lee, Y.\,C.~Lin, \textit{et al}.} 2007. Comparison of artificial neural 
networks with logistic regression\linebreak\vspace*{-12pt}

\pagebreak

\noindent
 in prediction of gallbladder disease among obese patients. 
\textit{Digest. Liver Dis.} 39(4):356--362.

%\pagebreak


\bibitem{15-su-1}
\Aue{Bassi, P., E. Sacco, V.~De Marco, \textit{et al}.} 2007. Prognostic accuracy of an 
artificial neural network in patients undergoing radical cystectomy for bladder cancer: 
A~comparison with logistic regression analysis. \textit{BJU Int.}  99(5):1007--1012.
\bibitem{16-su-1}
\Aue{Stephan, C., C.~Xu, P.~Finne, \textit{et al}.} 2007. Comparison of two different artificial 
neural networks for prostate biopsy indication in two different patient populations. 
\textit{J.~Urology}  70(3):596--601.

%\columnbreak

\bibitem{19-su-1} %18
\Aue{Stone, M.} 1974. Cross-validatory choice and assessment of statistical predictions (with 
discussion). \textit{J.~Roy. Stat. Soc. B} 36:111--147.
%\bibitem{17-su-1}
%\Aue{Chun, F.\,K., P.\,I.~Karakiewicz, A.~Briganti, \textit{et al}.} 2007. A~critical appraisal 
%of logistic regression-based nomograms, artificial neural networks, classification and 
%regression-tree models, look-up tables and risk-group stratification models for prostate cancer. 
%\textit{BJU Intern}.  99(4):794--800.



\bibitem{18-su-1} %19
\Aue{Efron, B.} 1979. Bootstrap methods: Another look at the jackknife. \textit{Ann.  
Stat.}  7:1--26.

\bibitem{21-su-1} %17
\Aue{Izenman, A.\,J.} 2008. \textit{Modern multivariate statistical techniques}. Springer. 
731~p.


 
 
 
% \bibitem{20-su-1} %20
%\Aue{Breiman, L.} 1994. The 1991 census adjustment: Undercount or bad data.  
%\textit{Stat. Sci.}  9:458--75.
 

\end{thebibliography}
} }

\end{multicols}

\hfill{\small\textit{Received April 17, 2013}}

\Contr

\noindent
\textbf{Krivenko Michail P.} (b.\ 1946)~--- 
Doctor of Science in technology, principal scientist, Institute of Informatics 
Problems, Russian Academy of Sciences, Moscow 119333, Russian Federation;  mkrivenko@ipiran.ru

\vspace*{3pt}


\noindent\textbf{Golovanov Sergey  A.} (b.\ 1950)~--- Doctor of Science in medicine, Head of 
Laboratory, Research Institute of Urology, Moscow 105425, Russian Federation;
sergeygol124@mail.ru
 

\vspace*{3pt}

\noindent
\textbf{Savchenko Pavel A.} (b.\ 1967)~--- software engineer, Institute of Informatics 
Problems, Russian Academy of Sciences, Moscow 119333, Russian Federation;  
psavchenko@ipiran.ru

\vspace*{3pt}

\noindent
\textbf{Sivkov Andrey V.} (b.\ 1957)~--- Doctor of Science in medicine, Deputy director, 
Research Institute of Urology, Moscow 105425, Russian Federation;  uroinfo@yandex.ru

\vspace*{3pt}

\noindent
\textbf{Suchkov Alexander P.} (b. 1954)~--- Doctor of Science in technology, principal 
scientist, Institute of Informatics Problems, Russian Academy of Sciences, Moscow 119333, Russian Federation;  
asuchkov@ipiran.ru 

\label{end\stat}

\renewcommand{\bibname}{\protect\rm Литература}