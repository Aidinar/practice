
\documentclass[10pt]{book}
\usepackage[utf8]{inputenc}

\usepackage{latexsym,amssymb,amsfonts,amsmath,amsxtra,dsfont,
indentfirst,shapepar,%fleqn,%
picinpar,shadow,floatflt,enumerate,multicol,colortbl,moreverb,cite,ipi}

\usepackage{rotating}
\usepackage{mathrsfs}
\usepackage[noend]{algorithmic}
\usepackage{ulem}
\usepackage{graphicx}
%\usepackage{algorithm2e}
\usepackage[linesnumbered,boxed,ruled]{algorithm2e}
%\usepackage{xypic}
\usepackage{oldgerm}
\usepackage{epic}
\usepackage{eepic}


\SetAlgorithmName{Algorithm}{алгоритм}{Список алгоритмов}

%из Дюковой

\newcommand{\algKeyword}[1]{{\bf #1}}
\newcommand{\Proc}[1]{\text{\tt #1}}
\def\CALL{\algKeyword{call}~}

\newenvironment{AlgProcedure}[1]
{
    \small
    \medskip
    %    \hrule
    \medskip
    \algKeyword{PROCEDURE} #1
    \begin{algorithmic}[1]}
    {\end{algorithmic}
    %    \hrule
    \bigskip
}

\def\CALL{\algKeyword{call}~}

%конец для Дюковой

%\RequirePackage[ruled]{algorithm}


\input{epsf}

%\nofiles

%\includeonly{avtor}             %+pdf+
%\includeonly{obchak,avtor}
%\includeonly{pred}                 %+
%\includeonly{podgot-rus-site,podgot-eng-site}  
%\includeonly{podgot-rus,podgot-eng}  
%\includeonly{ocherk} 
%\includeonly{nekrol} 
%\includeonly{ipi-ind} 
%\includeonly{index13}
%\includeonly{toc-rus, toc-en}
%\includeonly{toc-rus}
%\includeonly{toc-en} 



%\includeonly{shestakov}  %1+pdfавт+
%\includeonly{gorshenin}  %2pdfавт послали повт
%\includeonly{borisov}    %3+pdfавт послали повт-ответил+
%\includeonly{bosov}      %4+pdfавт+ 
%\includeonly{goncharov}  %5+pdfавт+
%\includeonly{br-stup}    %6+pdfавт+
%\includeonly{danilishin} %7+pdfавт послали повт-ответил
%\includeonly{serebr}     %8+pdfавт+
%\includeonly{sevast}     %9+pdfавт+
%\includeonly{pop+sim}    %10+pdfавт+
%\includeonly{grusho}     %11+pdfавт+
%\includeonly{husainov}   %12+pdfавт+
%\includeonly{farh}       %13+pdfавт+
%\includeonly{vohmin}     %14+pdf
%\includeonly{kozerenko}  %15+pdfавт+
%\includeonly{shihiev}    %16+pdfавт+
%\includeonly{dulin}      %17+pdf??









%\includeonly{nekrol}             %+


%\includeonly{obchak}
%\includeonly{rekl}
%\includeonly{rekl-1}
%\includeonly{reshal}  %
%\includeonly{cover3}

\usepackage{acad}
%\usepackage{courier}
\usepackage{decor}
\usepackage{newton}
\usepackage{pragmatica}
\usepackage{zapfchan}
\usepackage{petrotex}
\usepackage{bm}                     % полужирные греческие буквы
\usepackage{upgreek}                % прямые греческие буквы
\usepackage{eufrak}
\usepackage{verbatim}

\renewcommand{\bottomfraction}{0.99}
\renewcommand{\topfraction}{0.99}
\renewcommand{\textfraction}{0.01}

\setcounter{secnumdepth}{1} %здесь - 3 + chapter = 4

\arraycolsep=1.5pt

%\usepackage[pdftex]{graphicx}

%\usepackage{oz}

%NEW COMMANDS


\renewcommand*{\hm}[1]{#1\nobreak\discretionary{}%
            {\hbox{$\mathsurround=0pt #1$}}{}} %% Дублирует знаки операций
                               %при переносе в формуле (перед знаком, который
                               %надо продублировать ставится команда \hm)

%\newcommand{\endproof}{\hfill$\Box$}
\renewcommand{\r}{\mathbb{R}}
%\newcommand{\I}{{\rm I\hspace{-0.7mm}I}}
%\newcommand{\Ikl}{{\tt{1}}\hspace*{-1.44mm}\mathtt{1}}
\newcommand{\Ik}{\mbox{{\small \tt {1}}\hspace{-1.3mm}{\tt 1}}}
\newcommand{\argmin}{\mathop{\mathrm{arg}\,\mathrm{min}}}
\newcommand{\argmax}{\mathop{\mathrm{arg}\,\mathrm{max}}}
%\newcommand{\capr}{\mathop{\cap\,}}
%\newcommand{\cupr}{\mathop{\cup\,}}
%\def\argmin{\mathop{arg\,min}}

\def\vrp{\varphi}
\def\prt{\partial}
\def\mm{{\sf M}}
\def\modnop#1{\mathop{#1}\limits_{n}}
\def\eam{\mathbin{{\mathop{=}\limits^{\mathrm{def}}}}}
\def\dey#1#2{#1 (#2)}
\def\deyc#1#2{#1 \cdot  #2}
\def\ra#1{\;\mathop{\to}\limits^{#1}\;}
\def\raz#1{\;\mathop{\longrightarrow}\limits^{\!\!\!#1}\;}
\def\ral#1{\;\mathop{\longrightarrow}\limits^{#1}\;}

\newcommand{\Nor}{\mathcal{N}}
\newcommand{\T}{\mathbb{T}}
\newcommand{\Z}{\mathbb{Z}}



\newcommand{\il}[2]{\int\limits_{#1}^{#2}}%интеграл с пределами #1 и #2

\def\sm2{\mathop {\sum\limits^{n^\Theta}\sum\limits^{n^\Theta}}}
\def\sss{\sum\limits}
\def\tr{,\,\ldots\,,\,}
\def\rk{\right]}
\def\lk{\left[}
\def\rf{\right\}}
\def\lf{\left\{}
\def\lv{\,\left\vert}
\def\rv{\right\vert\,}
\def\iii{\int\limits}
\def\iin{\int\limits_{-\infty}^\infty}
\def\rrv{\right\vert}


\def\ee{{\cal E}}
\def\ww{{\cal W}}
\def\yy{{\cal Y}}
\def\vv{{\cal V}}

\newcommand{\R}{\mathbb R}
\newcommand{\E}{\mathbb E}
\newcommand{\N}{\mathbb N}

\renewcommand{\P}{\mathbb{P}}

\newcommand{\h}{{\bf H}}
\newcommand{\p}{{\sf P}}  % вероятность

\newcommand{\e}{{\sf E}}  % мат. ожидание
\newcommand{\D}{{\sf D}}  % дисперсия
\newcommand{\eps}{\varepsilon}
\newcommand{\vp}{{\mathbf p}}
\newcommand{\vz}{{\mathbf z}}
\newcommand{\vx}{{\mathbf x}}
\newcommand{\vf}{{\mathbf f}}
\newcommand{\F}{{\mathcal F}}
\def\ap{{\mathrm{ЭР}}}
\newcommand{\ud}{\Delta_n} %uniform ditance
\newcommand{\nud}{\Delta_n(x)}
%\renewcommand{\Re}{\mathrm{Re}\,}

\newcommand{\abs}[1]{\left\vert#1\right\vert}

\newcommand{\norm}[1]{\left\Vert#1\right\Vert}
\def\da{(\Delta_t,A)}

\newcommand{\corr}{\mathrm{corr}}

\newcommand{\cov}{\mathrm{cov}}
\newcommand{\Expect}{\mathbb{E}}

\def\w{\omega}
\def\W{\Omega}

\def\inh{\int\limits_{nh}^{(n+1)h}}

\def\sumin{\sum_{i=1}^N}


\def\bxt{(Y,t)}
\def\xt{(y,t)}

\def\ovth{{\fr{\tau-nh}{h}}}
\def\ov{\overline}
\def\tm{\tilde m}
\def\tl{\tilde\lambda}
\def\tB{\widetilde B}
\def\tb{\tilde b}
\def\ld{\ldots}
\def\cd{\cdots}


\DeclareMathOperator{\sign}{sign}

%\newcommand{\gr}{{\geqslant}}


\newcommand{\g}{\mbox{\textit{g}}}

\renewcommand{\la}{\lambda}
\newcommand{\si}{\sigma}
\newcommand{\alp}{\alpha}

\newcommand{\pto}{\stackrel{P}{\longrightarrow}} % сходимость по веpоятности

\newcommand{\eqd}{\stackrel{\mathrm{d}}{=}} % равенство по pаспpеделению
\newcommand{\eqdelta}{\stackrel{\triangle}{=}} % равенство по pаспpеделению

\def\be#1{\begin{equation}\label{#1}}
\def\ee{\end{equation}}
\def\re#1{(\ref{#1})}

\def\bn{\begin{enumerate}}
\def\en{\end{enumerate}}
\def\bi{\begin{itemize}}
\def\ei{\end{itemize}}
%\def\i{\item}

%\newcommand{\kp}{\kappa}
%\def\Q{{\cal Q}} \def\H{{\cal H}}
%\newcommand{\bet}{\beta_{2+\delta}}


%\newtheorem{definition}{Определение}
%\renewcommand{\thedefinition}{\arabic{definition}.}
%END NEW COMMANDS

%\renewcommand{\baselinestretch}{1.2}

%\pagestyle{myheadings}

\setlength{\textwidth}{167mm}      % 122mm
\setlength{\textheight}{658pt}
%\setlength{\textheight}{635.6pt}
\setlength{\columnsep}{4.5mm}

\setcounter{secnumdepth}{4}

%\addtolength{\headheight}{2pt}
%\addtolength{\headsep}{-2mm}

\addtolength{\topmargin}{-7mm}  % for printing


%\hoffset=-30mm  % From Yap
\hoffset=-23mm  % From Acrobat

%\voffset=0mm % From Yap
\voffset=-5mm   % From Acrobat

%\addtolength{\evensidemargin}{-2.5mm} % for printing
%\addtolength{\oddsidemargin}{2.5mm}  % for printing

\addtolength{\evensidemargin}{-12mm} % for printing
\addtolength{\oddsidemargin}{8mm}  % for printing

%\renewcommand{\thefootnote}{\fnsymbol{footnote}}
%\renewcommand{\thefootnote}{\arabic{footnote}}
\renewcommand{\figurename}{\protect\bf Рис.}
\renewcommand{\tablename}{\protect\bf Таблица}

\newcommand{\Caption}[1]{\caption{\protect\small %\baselineskip=2.5ex
#1}}

\renewcommand{\thefigure}{\arabic{figure}}
\renewcommand{\thetable}{\arabic{table}}
\renewcommand{\theequation}{\arabic{equation}}
\renewcommand{\thesection}{\arabic{section}}

\renewcommand{\contentsname}{СОДЕРЖАНИЕ}
\newcommand{\fr}[2]{\displaystyle\frac{\displaystyle #1\mathstrut}{\displaystyle #2\mathstrut}}

%\renewcommand{\thefootnote}{\fnsymbol{footnote}}
%\newcommand{\g}{\mbox{\textit{g}}}

%\newcommand{\Caption}[1]{\caption{\protect\small\baselineskip=2ex #1}}
\newcounter{razdel}
\setcounter{razdel}{0}


\newcommand{\titel}[4]{%
\

\vspace*{5pt}

\ifodd\therazdel {\raggedright\noindent\Large\textrm\textbf
 \lineskip .75em
  \baselineskip=3.2ex #1 \par}
\vskip 1em {\noindent\large\textrm\textbf #2 \par}
\addcontentsline{toc}{subsection}{{\textrm\textbf #1}\protect\newline #2}
\def\rightheadline{\underline{\noindent\hbox to \textwidth{\hfill\small\textrm{#4}
%\hfill \large\bf\thepage
}}}
\def\leftheadline{\underline{\noindent\parbox{\textwidth}{
%\raggedleft\large\bf\thepage \hfill
\small\textit{#3}\hfill}}}
\def\leftfootline{\small{\textbf{\thepage}
\hfill ИНФОРМАТИКА И ЕЁ ПРИМЕНЕНИЯ\ \ \ том~14\ \ \ выпуск 1\ \ \ 2020}
}%
 \def\rightfootline{\small{ИНФОРМАТИКА И ЕЁ ПРИМЕНЕНИЯ\ \ \ том~14\ \ \ выпуск~1\ \ \ 2020
\hfill \textbf{\thepage}}}
\vskip 2em \setcounter{figure}{0}
\setcounter{table}{0}
\setcounter{equation}{0}
\setcounter{section}{0}
\setcounter{subsection}{0}
\setcounter{subsubsection}{0}
\setcounter{footnote}{0}
\setcounter{razdel}{0}
%\end{flushleft}
\else {
 \raggedright\noindent\Large\textrm\textbf
 \lineskip .75em
\baselineskip=3.2ex #1 \par} \vskip 1em
%\begin{flushleft}
{\noindent\large\textrm\textbf #2 \par}
\addcontentsline{toc}{subsection}{{\textrm\textbf #1}\protect\newline #2}
\def\rightheadline{\underline{\noindent\hbox to \textwidth{\hfill\small\textrm{#4}
%\hfill \large\bf\thepage
}}}
\def\leftheadline{\underline{\noindent\parbox{\textwidth}{%\raggedleft\large\bf\thepage \hfill
\small\textit{#3}\hfill}}}
\def\leftfootline{\small{\textbf{\thepage}
\hfill ИНФОРМАТИКА И ЕЁ ПРИМЕНЕНИЯ\ \ \ том~14\ \ \ выпуск~1\ \ \ 2020}
}%
 \def\rightfootline{\small{ИНФОРМАТИКА И ЕЁ ПРИМЕНЕНИЯ\ \ \ том~14\ \ \ выпуск~1\ \ \ 2020
\hfill \textbf{\thepage}}} \vskip 2em \setcounter{figure}{0}
\setcounter{table}{0} \setcounter{equation}{0} \setcounter{section}{0}
\setcounter{subsection}{0} \setcounter{subsubsection}{0}
\setcounter{footnote}{0}
%\end{flushleft}
\fi}

\newcommand{\titelr}[2]{%
\

\vspace*{5pt}

\ifodd\therazdel {\raggedright\noindent%\Large\textrm\textbf
 \lineskip .75em
  \baselineskip=3.2ex #1 \par}
\vskip 1em {\noindent\normalsize\textrm\textbf #2 \par}
\else {
 \raggedright\noindent\Large\textrm\textbf
 \lineskip .75em
\baselineskip=3.2ex #1 \par} \vskip 1em
%\begin{flushleft}
{\noindent\large\textrm\textbf #2 \par
%\noindent\normalsize\textrm\textbf #2 \par
} \fi}

\newcommand{\titele}[5]{%
\

%\vspace*{5pt}

\ifodd\therazdel {\raggedright\noindent\large
\textrm\textbf
 \lineskip .75em
%  \baselineskip=3.2ex
#1 \par}
\vskip .5em {\noindent\large\textrm\textbf #2 \par}
\vskip .5em
 {\noindent\textrm #3 \par}
\addcontentsline{toc}{subsection}{{\textrm\textbf #1}\protect\newline #2}
\def\rightheadline{\underline{\noindent\hbox to \textwidth{\hfill\small\textrm{#4}
%\hfill \large\bf\thepage
}}}
\def\leftheadline{\underline{\noindent\parbox{\textwidth}{
%\raggedleft\large\bf\thepage \hfill
\small\textrm{#5}\hfill}}}
\def\leftfootline{\small{\textbf{\thepage}
\hfill ИНФОРМАТИКА И ЕЁ ПРИМЕНЕНИЯ\ \ \ том~14\ \ \ выпуск~1\ \ \ 2020}
}%
 \def\rightfootline{\small{ИНФОРМАТИКА И ЕЁ ПРИМЕНЕНИЯ\ \ \ том~14\ \ \ выпуск~1\ \ \ 2020
\hfill \textbf{\thepage}}} \vskip 1em \setcounter{figure}{0}
\setcounter{table}{0} \setcounter{equation}{0} \setcounter{section}{0}
\setcounter{subsection}{0} \setcounter{subsubsection}{0}
\setcounter{footnote}{0} \setcounter{razdel}{0}
%\end{flushleft}
\else {
 \raggedright\noindent\large
 \textrm\textbf
 \lineskip .75em
%\baselineskip=3.2ex
#1 \par} \vskip .5em
%\begin{flushleft}
{\noindent\large\textrm\textbf #2 \par} \vskip .5em
 {\noindent\textrm #3 \par}
\addcontentsline{toc}{subsection}{{\textrm\textbf #1}\protect\newline #2}
\def\rightheadline{\underline{\noindent\hbox to \textwidth{\hfill\small\textrm{#4}
%\hfill \large\bf\thepage
}}}
\def\leftheadline{\underline{\noindent\parbox{\textwidth}{%\raggedleft\large\bf\thepage \hfill
\small\textrm{#5}\hfill}}}
\def\leftfootline{\small{\textbf{\thepage}
\hfill ИНФОРМАТИКА И ЕЁ ПРИМЕНЕНИЯ\ \ \ том~14\ \ \ выпуск~1\ \ \ 2020}
}%
 \def\rightfootline{\small{ИНФОРМАТИКА И ЕЁ ПРИМЕНЕНИЯ\ \ \ том~14\ \ \ выпуск~1\ \ \ 2020
\hfill \textbf{\thepage}}} \vskip 1em \setcounter{figure}{0}
\setcounter{table}{0} \setcounter{equation}{0} \setcounter{section}{0}
\setcounter{subsection}{0} \setcounter{subsubsection}{0}
\setcounter{footnote}{0}
%\end{flushleft}
\fi}

\def\Abst#1{
\begin{center}\small\nwt
\parbox{150mm}{%\baselineskip=2.5ex
\textbf{Аннотация:}\ \
%\hspace*{\parindent}
#1}
\end{center}}
\def\Abste#1{
\begin{center}\small\nwt
\parbox{150mm}{%\baselineskip=2.5ex
\textbf{Abstract:}\ \
%\hspace*{\parindent}
#1}
\end{center}}

\def\DOI#1{
\begin{center}\small\nwt
\parbox{150mm}{%\baselineskip=2.5ex
\textbf{DOI:}\ \
%\hspace*{\parindent}
#1}
\end{center}}

\def\Abstend#1{
\begin{center}\small\nwt
\parbox{150mm}{%\baselineskip=2.5ex
%\hspace*{\parindent}
#1}
\end{center}}


\def\KW#1{
\begin{center}\small\nwt
\parbox{150mm}{%\baselineskip=2.5ex
\textbf{Ключевые слова:}\ \ #1}
\end{center}}

\def\KWE#1{
\begin{center}\small\nwt
\parbox{150mm}{%\baselineskip=2.5ex
\textbf{Keywords:}\ \ #1}
\end{center}}


\def\KWN#1{
%\begin{center}
%\small
%\parbox{150mm}\end{center}
}

\newcommand{\Avtors}[1]{%\smallskip
%\vspace*{.5pt}
\hangindent=23pt\noindent
%\nwt
{\bfseries#1}\
}


\renewcommand{\thesubsection}{\thesection.\arabic{subsection}\hspace*{-5pt}}
\renewcommand{\thesubsubsection}{\thesubsection\hspace*{5pt}.\arabic{subsubsection}\hspace*{-3pt}}

\newcommand{\Ack}{\section*{\protect\rmfamily Acknowledgments}\noindent}
\newcommand{\Contr}{\section*{\protect\rmfamily Contributors}\noindent}
\newcommand{\Contrl}{\section*{\protect\rmfamily Contributor}\noindent}

\makeindex


\begin{document}
\Rus

\nwt
%\ptb


%\renewcommand{\contentsname}{\protect\Large\bf Содержание}

\setcounter{tocdepth}{2}

%\tableofcontents

\renewcommand{\bibname}{\protect\rmfamily Литература}
  \def\Au#1{{\it #1}}
    \def\Aue#1{{#1}}

%\newcommand{\No}{№}
  \newcommand{\tg}{\,\mathrm{tg}\,}
    \newcommand{\ctg}{\,\mathrm{ctg}\,}
  \newcommand{\arctg}{\,\mathrm{arctg}\,}

\def\forallb{\mathop{\forall}}
\def\cupb{\mathop{\cup}}
\def\existsb{\mathop{\exists}}


\newpage
\addtocounter{razdel}{1}
%\def\razd{РЕГУЛИРУЕМЫЙ ЭЛЕКТРОПРИВОД ДЛЯ ЭЛЕКТРОЭНЕРГЕТИКИ}


\setcounter{page}{3}

%   { %\Large  
   { %\baselineskip=16.6pt
   
   \vspace*{-48pt}
   \begin{center}\LARGE
   \textit{Предисловие}
   \end{center}
   
   %\vspace*{2.5mm}
   
   \vspace*{25mm}
   
   \thispagestyle{empty}
   
   { %\small 

    
Вниманию читателей журнала <<Информатика и её применения>> предлагается 
очередной тематический выпуск <<Вероятностно-статистические методы и 
задачи информатики и информационных технологий>>. Предыдущие тематические 
выпуски журнала по данному направлению вышли в 2008~г.\ (т.~2, вып.~2), 
в 2009~г.\ (т.~3, вып.~3) и в 2010~г.\ (т.~4, вып.~2). 

Статьи, собранные в данном журнале, посвящены разработке новых вероятностно-статистических 
методов, ориентированных на применение к решению конкретных задач информатики и информационных 
технологий, а также~--- в ряде случаев~--- и других прикладных задач. Проблематика, охватываемая 
публикуемыми работами, развивается в рамках научного сотрудничества между Институтом проблем 
информатики Российской академии наук (ИПИ РАН) и Факультетом вычислительной математики и 
кибернетики Московского государственного университета им.\ М.\,В.~Ломоносова в ходе работ 
над совместными научными проектами (в том числе в рамках функционирования 
Научно-образовательного центра <<Вероятностно-статистические методы анализа рисков>>). 
Многие из авторов статей, включенных в данный номер журнала, являются активными участниками 
традиционного международного семинара по проблемам устойчивости стохастических моделей, 
руководимого В.\,М.~Золотаревым и В.\,Ю.~Королевым; регулярные сессии этого семинара 
проводятся под эгидой МГУ и ИПИ РАН (в 2011~г.\ указанный семинар проводится в октябре 
в Калининградской области РФ). 

Наряду с представителями ИПИ РАН и МГУ в число авторов данного выпуска журнала входят 
ученые из Научно-исследовательского института системных исследований РАН, Института 
проблем технологии микроэлектроники и особочистых материалов РАН, Института 
прикладных математических исследований Карельского НЦ РАН, Московского 
авиационного института, Вологодского государственного педагогического университета, 
НИИММ им.\ Н.\,Г.~Чеботарева, Казанского государственного университета, Дебреценского 
университета (Венгрия).

Несколько статей выпуска посвящено разработке и применению стохастических методов и 
информационных технологий для решения различных прикладных задач. В~работе В.\,Г.~Ушакова 
и О.\,В.~Шестакова рассмотрена задача определения вероятностных характеристик случайных 
функций по распределениям интегральных преобразований, возникающих в задачах эмиссионной 
томографии. В~статье Д.\,О.~Яковенко и М.\,А.~Целищева рассмотрены некоторые вопросы 
математической теории риска и предложен новый подход к диверсификации инвестиционных 
портфелей. Работа И.\,А.~Кудрявцевой и А.\,В.~Пантелеева посвящена построению и 
исследованию математической модели, описывающей динамику сильноионизованной плазмы. 
В~статье П.\,П.~Кольцова изучается качество работы ряда алгоритмов сегментации изображений. 
Статья А.\,Н.~Чупрунова и И.~Фазекаша посвящена вероятностному анализу числа без\-оши\-бочных 
блоков при помехоустойчивом кодировании; получены усиленные законы больших чисел для указанных 
величин.

В данном выпуске традиционно присутствует тематика, весьма активно разрабатываемая в течение 
многих лет специалистами ИПИ РАН и МГУ,~--- методы моделирования и управления для 
информационно-телекоммуникационных и вычислительных систем, в частности методы 
теории массового обслуживания. В~статье А.\,И.~Зейфмана с соавторами рассматриваются 
модели обслуживания, описываемые марковскими цепями с непрерывным временем в случае 
наличия катастроф. В~работе М.\,М.~Лери и И.\,А.~Чеплюковой рассматриваются случайные 
графы Интернет-типа, т.\,е.\ графы, степени вершин которых имеют степенные распределения; 
такие задачи находят применение при исследовании глобальных сетей передачи данных. 
Работа Р.\,В.~Разумчика посвящена исследованию систем массового обслуживания специального 
вида~--- с отрицательными заявками и хранением вытесненных заявок.

Ряд статей посвящен развитию перспективных теоретических 
вероятностно-статистических методов, которые находят широкое применение в различных 
задачах информатики и информационных технологий. В~работе В.\,Е.~Бенинга, А.\,К.~Горшенина 
и В.\,Ю.~Королева рассмотрена задача статистической проверки гипотез о числе компонент 
смеси вероятностных распределений, приводится конструкция асимптотически наиболее мощного 
критерия. Результаты этой работы найдут применение в ряде прикладных задач, использующих 
математическую модель смеси вероятностных распределений (в информатике, моделировании 
финансовых рынков, физике турбулентной плазмы и~т.\,д.). В~статье В.\,Ю.~Королева, 
И.\,Г.~Шевцовой и С.\,Я.~Шоргина строится новая, улучшенная оценка точности нормальной 
аппроксимации для пуассоновских случайных сумм; как известно, указанные случайные суммы 
широко используются в качестве моделей многих реальных объектов, в том числе в информатике, 
физике и других прикладных областях. Работа В.\,Г.~Ушакова и Н.\,Г.~Ушакова посвящена 
исследованию ядерной оценки плотности распределения; эти результаты могут применяться, 
в част\-ности, при анализе трафика в телекоммуникационных системах. Серьезные приложения 
в статистике могут получить результаты работы О.\,В.~Шестакова, в которой доказаны оценки 
скорости сходимости распределения выборочного абсолютного медианного отклонения к нормальному 
закону. 

\smallskip

Редакционная коллегия журнала выражает надежду, что данный тематический  выпуск 
будет интересен специалистам в области теории вероятностей и математической статистики 
и их применения к решению задач информатики и информационных технологий.
     
     %\vfill 
     \vspace*{20mm}
     \noindent
     Заместитель главного редактора журнала <<Информатика и её 
применения>>,\\
     директор ИПИ РАН, академик  \hfill
     \textit{И.\,А.~Соколов}\\
     
     \noindent
     Редактор-составитель тематического выпуска,\\
     профессор кафедры математической статистики факультета\\
      вычислительной математики и кибернетики МГУ им.\ М.\,В.~Ломоносова,\\
     ведущий научный сотрудник ИПИ РАН,\\ 
доктор физико-математических наук \hfill
      \textit{В.\,Ю.~Королев}
     
     } }
     }

\def\stat{shestakov+vor}

\def\tit{АСИМПТОТИЧЕСКАЯ НОРМАЛЬНОСТЬ И~СИЛЬНАЯ СОСТОЯТЕЛЬНОСТЬ ОЦЕНКИ РИСКА ПРИ~ИСПОЛЬЗОВАНИИ FDR-ПОРОГА В УСЛОВИЯХ СЛАБОЙ ЗАВИСИМОСТИ}

\def\titkol{Асимптотическая нормальность и~сильная состоятельность оценки риска при~использовании FDR-порога} % в~условиях слабой зависимости}

\def\aut{М.\,О.~Воронцов$^1$, О.\,В.~Шестаков$^2$}

\def\autkol{М.\,О.~Воронцов, О.\,В.~Шестаков}

\titel{\tit}{\aut}{\autkol}{\titkol}

\index{Воронцов М.\,О.}
\index{Шестаков О.\,В.}
\index{Vorontsov M.\,O.}
\index{Shestakov O.\,V.}


%{\renewcommand{\thefootnote}{\fnsymbol{footnote}} \footnotetext[1]
%{Работа 
%выполнена при поддержке Программы развития МГУ, проект №\,23-Ш03-03. При анализе 
%данных использовалась инфраструктура Центра коллективного пользования 
%<<Высокопроизводительные вычисления и~большие данные>> 
%(ЦКП <<Информатика>>) ФИЦ ИУ РАН (г.~Москва)}}


\renewcommand{\thefootnote}{\arabic{footnote}}
\footnotetext[1]{Московский государственный университет 
имени~М.\,В.~Ломоносова, факультет вычислительной математики и~кибернетики;  
Московский центр фундаментальной и~прикладной математики, \mbox{m.vtsov@mail.ru}}
\footnotetext[2]{Московский государственный университет 
имени М.\,В.~Ломоносова, факультет вычислительной математики и~кибернетики; 
Федеральный исследовательский центр <<Информатика и~управление>> Российской 
академии наук; Московский центр фундаментальной и~прикладной математики, 
\mbox{oshestakov@cs.msu.ru}}


\vspace*{-12pt}





\Abst{Рассматривается подход к~решению задачи удаления шума в~большом массиве 
разреженных данных, основанный на методе контроля средней доли ложных отклонений 
гипотез (False Discovery Rate, FDR). Данный подход эквивалентен процедурам 
пороговой обработки, обнуляющим компоненты массива, значения которых не 
превосходят некоторого заданного порога.  Наблюдения в~модели считаются слабо 
зависимыми. Для контроля степени зависимости используются ограничения на 
коэффициент сильного перемешивания и~максимальный коэффициент корреляции. 
В~качестве меры эффективности рассматриваемого подхода используется 
среднеквадратичный риск. Вычислить значение риска можно только на тестовых 
данных, поэтому в~работе рассматривается его статистическая оценка и~исследуются 
ее свойства. Показана асимптотическая нормальность и~сильная состоятельность 
оценки риска при использовании FDR-по\-ро\-га в~условиях слабой зависимости в~данных.}

\KW{пороговая обработка; множественная проверка гипотез; 
оценка риска}

\DOI{10.14357/19922264240309}{ZOQVTO}
  
%\vspace*{-6pt}


\vskip 10pt plus 9pt minus 6pt

\thispagestyle{headings}

\begin{multicols}{2}

\label{st\stat}



\section{Введение}

Во многих прикладных областях возникает задача обработки больших массивов 
зашумленных данных. Примерами служат задачи обработки изоб\-ра\-же\-ний с~высоким 
разрешением~\cite{FDRImage}, задачи множественной проверки гипотез, возникающие 
в~\mbox{исследованиях} в~об\-ласти генетики~\cite{MultipleTesting}, и~другие проб\-ле\-мы. 
В~связи с~этим рас\-смот\-рим модель
$$
x_i = \mu_i + z_i, \enskip i=\overline{1,n}\,,
$$
где $\mu_i\in\mathbb{R}$~--- <<полезные>> данные; $z_i \sim N(0,\sigma^2)$~--- 
шум. Задача заключается в~нахождении оценки неизвестного вектора $\mu \hm= 
(\mu_1,\ldots,\mu_n)$ как функции вектора $x \hm= (x_1,\ldots,x_n)$ и~может 
рассматриваться как задача множественной проверки гипотез о~равенстве нулю 
компонент вектора~$\mu$~\cite{AdaptingFDR}. При этом обычно предполагается, что 
вектор~$\mu$ имеет в~определенном смысле <<разреженную>> структуру, т.\,е.\ для 
<<полезных>> данных используется <<экономное>> представление.



В работе~\cite{AdaptingFDR} для решения рассматриваемой задачи в~условиях 
независимости компонент вектора~$x$ и~разреженности вектора~$\mu$ была 
предложена процедура построения оценки~$\hat{\mu}_F$ вектора~$\mu$, основанная 
на методе контроля средней доли ложных отклонений (FDR) 
гипотез при помощи алгоритма Бен\-жа\-ми\-ни--Хох\-бер\-га,
и~было проведено исследование асимптотики ее среднеквадратичного риска. 
В~работах~\cite{ZasShe17,Mathematics2020} была показана состоятельность 
и~асимптотическая нормальность оценки риска данной процедуры. Аналогичные 
результаты для других методов построения~$\hat{\mu}_F$ получены в~работах~\cite{Shestakov2021-1,Shestakov2021-2,Shestakov2022}.

В то же время в~определенных приложениях, например  при анализе полученных 
в~результате использования ДНК-мик\-ро\-чи\-пов данных~\cite{ResultsOnFDRUnderDependence}, исследовании геофизических процессов 
и~анализе помех\linebreak в~телекоммуникационных каналах, условие незави\-си\-мости компонент 
вектора $x$ может не выполняться. Ранее в~работах~\cite{VorontsovShestakov2023,Vorontsov2024} была \mbox{исследована} асимп\-то\-ти\-ка 
среднеквадратичного риска оценки~$\hat{\mu}_F$ \mbox{в~случае}, когда~$\mu$ принадлежит 
одному из классов разреженности
$$
l_0[\eta] = \left\{\mu\,:\, ||\mu||_0 \leq \eta n\right\}, \enskip \eta \in 
(0,1),
$$

\vspace*{-12pt}

\noindent
\begin{multline*}
m_p[\eta] \equiv{}\\
{}\equiv \left\{\mu \in \mathbb{R}^n : |\mu|_{(k)} \leq \eta n^{1/p} 
k^{-1/p},\ k=\overline{1,n}\right\}, \\
 p\in(0, 2),
\end{multline*}
а компоненты вектора~$x$ слабо зависимы~--- имеют достаточно быстро убывающий 
коэффициент сильного перемешивания~\cite{Bosq}

\noindent
\begin{multline*}
\alpha(k) = \sup\limits_{1\leq m\leq n}\alpha\left(\sigma(x_i, i\leq m), 
\sigma(x_i, i\geq m+k)\right), \\ 
k=\overline{1,n-1}\,,
\end{multline*}
где символом $\sigma(x_i, i\in I)$ обозначена сиг\-ма-ал\-геб\-ра, порожденная 
множеством случайных величин $\{x_i, i \hm\in I\}$, а~мера  $\alpha(\cdot, \cdot)$ 
близости двух сиг\-ма-ал\-гебр определяется как
$$
\alpha(\mathcal{B},\mathcal{C}) = \sup\limits_{B\in\mathcal{B}, 
C\in\mathcal{C}} \left|\p(BC)-\p(B)\p(C)\right|.
$$

В настоящей работе показана асимптотическая нормальность и~сильная 
состоятельность оценки риска при применении FDR-про\-це\-ду\-ры в~случае, когда 
компоненты вектора~$x$ слабо зависимы, а~$\mu$ принадлежит одному из классов 
раз\-ре\-жен\-ности: 
$l_0[\eta]$ или $m_p[\eta]$.


\section{Обработка вектора данных с~помощью FDR-процедуры}

Широким классом методов построения оценки~$\hat{\mu}$ стала пороговая обработка 
вектора~$x$ с~некоторым порогом~$T$. Различают жесткую пороговую обработку, при 
которой полагается
\begin{equation*}
\left(\hat{\mu}\right)_i  = p_H(x_i,T) \equiv
 \begin{cases}
   x_i, & |x_i| > T\,;\\
   0, & |x_i| \leq T\,,
 \end{cases}
\end{equation*}
и мягкую пороговую обработку, для которой
\begin{equation*}
(\hat{\mu})_i  = p_S(x_i,T) \equiv
 \begin{cases}
   x_i-T, & \hphantom{\vert\vert}x_i > T;\\
   x_i+T, & \hphantom{\vert\vert}x_i <- T;\\
   0, & |x_i| \leq T.
 \end{cases}
\end{equation*}
Среднеквадратичный риск подобных процедур определяется как
\begin{equation}
\label{riskDef}
R(T) = {\mathsf E} ||\hat{\mu}-\mu||^2 = \sum\limits_{i=1}^n {\mathsf E} \left((\hat{\mu})_i-
\mu_i\right)^2.
\end{equation}
Обозначим через~$T_m$ наилучшее значение порога:
$$
T_m : \, R(T_m) = \min\limits_{T} R(T).
$$

Предложенная в~\cite{AdaptingFDR} процедура заключается в~жесткой пороговой 
обработке компонент вектора~$x$ с~порогом $\hat{t}_F \hm= \hat{t}_F(x)$, и~ее 
результат~--- оценка $\hat{\mu}_F$ вектора~$\mu$ с~компонентами $(\hat{\mu}_F)_i  
\hm= p_H(x_i,\hat{t}_F)$, где
\begin{multline*}
\hat{t}_F = \sigma z\left(\fr{q \hat{k}_F}{2n}\right), \enskip
\hat{k}_F = \max 
\left\{k \, :\, |x|_{(k)} \geq t_k \right\}, \\
 t_k = \sigma z\left(\fr{q  k}{2n}\right);
\end{multline*}
$z(\alpha)$ --- квантиль уровня $(1\hm-\alpha)$ стандартного нормального 
распределения; $|x|_{(k)}$~--- $k$-й элемент вектора, получаемого в~результате 
упорядочения вектора~$|x|$ по невозрастанию:
$$
|x|_{(1)} \geq |x|_{(2)} \geq \cdots \geq |x|_{(n)};
$$
$q\in(0;1)$~--- управ\-ля\-ющий параметр FDR-ме\-то\-да.
Далее полагается, что $q\hm\equiv q_n$ зависит от~$n$. В~\cite{AdaptingFDR} 
показано, что эта процедура эквивалентна множественной проверке гипотез 
о~равенстве нулю компонент наблюдаемого вектора. Также показано, что с~помощью 
метода штрафных функций данную процедуру можно свести к~другим видам пороговой 
обработки, в~част\-ности к~мягкой пороговой обработке.

В работах~\cite{VorontsovShestakov2023, Vorontsov2024} была исследована 
асимптотика среднеквадратичного риска~$R(\hat{t}_F)$ описанной процедуры 
в~случае, когда компоненты вектора $x$ слабо зависимы, а $\mu$ принадлежит классу 
разреженности~$\Theta_n$, где~$\Theta_n$ есть~$l_0[\eta_n]$ или~$m_p[\eta_n]$. 
Было показано, что~$R(\hat{t}_F)$ асимптотически отличается от минимаксного 
риска
$\inf\nolimits_{\hat{\mu}\hm=\hat{\mu}(x)} \sup\nolimits_{\mu\in \Theta_n} {\mathsf E} 
||\hat{\mu}-\mu||^2$
на множитель не более чем логарифмического по\-рядка.

Отметим, что в~выражении для среднеквадратичного риска~(\ref{riskDef}) 
присутствуют неизвестные величины~$\mu_i$, а~потому вычислить~$R(T_m)$ и~$T_m$ 
не представляется возможным. На практике можно пользоваться, например, следующей 
оценкой среднеквадратичного риска~\cite{Mallat}:
$$
\hat{R}(T) = \sum\limits_{i=1}^n F[x_i, T],
$$
где  
\begin{multline*}
F[x_i, T] = {}\\[3pt]
{}=\!\begin{cases}
\left(x_i^2-\sigma^2\right) \Ik(|x_i|\leq T) + \sigma^2 \Ik\left(|x_i|>T\right) &\\[3pt]
&\hspace*{-53mm}\mbox{для\ жесткой\ пороговой\ обработки};\\[3pt]
\left(x_i^2-\sigma^2\right) \Ik\left(|x_i|\leq T\right) + (\sigma^2+T^2) 
\Ik \left(|x_i|>T\right) \hspace*{-11.21576pt}&\\[3pt]
&\hspace*{-51mm}\mbox{для\ мягкой\ пороговой\ обработки}.
\end{cases}\hspace*{-7.17859pt}
\end{multline*}


\noindent
\textbf{Замечание}.\ При пороговой обработке иногда также используется так 
называемый универсальный порог $T_U\hm = \sigma \sqrt{2\ln n}$, предложенный 
в~работе~\cite{spatialAdaptation}. Исследования в~\cite{AdaptingSURE, ExactRisk} 
показали, что порог~$T_U$ в~определенном смысле максимальный, и~рас\-смат\-ри\-вать 
пороги выше него не имеет смысла. Более того, нетрудно показать, что $t_k \hm< T_U$ 
для всех~$k$ и~всех достаточно больших~$n$, в~связи с~чем всюду далее полагаем, 
что порог~$\hat{t}_F$ выбирается на отрезке $[0; T_U]$.

\section{Вспомогательные утверждения}

Кроме коэффициента сильного перемешивания~$\alpha(\cdot)$ также понадобится 
следующее понятие~\cite{Bosq}.

\smallskip

\noindent
\textbf{Определение.} %\label{defRho}
Максимальным коэффициентом корреляции~$\rho(\cdot)$ компонент вектора~$x$ 
называется
\begin{multline*}
\rho (k) \equiv \rho_n (k) = {}\\
{}=\sup\limits_{1\leq m\leq n}\rho\left(\sigma(x_i, 
i\leq m), \sigma(x_i, i\geq m+k)\right), \\
 k=\overline{1,n-1}\,,
\end{multline*}
где мера $\rho(\cdot, \cdot)$ близости двух сиг\-ма-ал\-гебр определяется как
$$
\rho(\mathcal{B},\mathcal{C}) = \sup\limits_{\substack{\xi 
\in\mathcal{L}^2(\mathcal{B}) \\
 \eta \in\mathcal{L}^2(\mathcal{C})}} 
\left|\mathrm{corr}\,(\xi, \eta)\right|.
$$


Введем обозначения:
$$
T_1 = \sqrt{2\ln \eta_n^{-p}};  \,\gamma_n = \fr{1}{\ln\ln n}; \, \kappa_n 
= \fr{n \eta_n^p T_1^{-p}}{1 - q_n - \gamma_n}; 
$$
$$ 
\kappa_n^0 = \fr{[n \eta_n]}{1 - q_n - \gamma_n} ;\, \rho^\star (k) = 
\sup\limits_{n\geq k+1} \rho(k), k \in \mathbb{N} ;
$$
$$
t_{\kappa_n} = \sigma z\left(\fr{q_n \kappa_n }{2n}\right) , \,\, t_{\kappa_n^0} 
= \sigma z\left(\fr{q_n \kappa_n^0 }{2n}\right).
$$


Следующие два утверждения показывают, что случайный порог~$\hat{t}_F$ в~случае 
$\mu\hm\in m_p[\eta_n]$ (соответственно $\mu\hm\in l_0[\eta_n]$) с~большой 
вероятностью будет не меньше~$t_{\kappa_n}$ (соответственно~$ t_{\kappa_n^0}$). 
Их  доказательства приведены в~работах~\cite{VorontsovShestakov2023, Vorontsov2024}.

\smallskip

\noindent
%\begin{lem}\label{lem5}
\textbf{Лемма~1.}\ \textit{Пусть $n^{-\delta_1} \hm\leq \eta_n^p \hm\leq n^{-\delta_2}$, 
$0\hm<\delta_2\hm<\delta_1<1$, $\mathrm{lim\,inf} q_n \ln n \hm\geq C \hm> 0$, 
$m\hm\in[1;n/2]\cap\mathbb{N}$, а $\alpha(\cdot)$~--- коэффициент сильного 
перемешивания компонент вектора~$x$. Для некоторого $N\hm\in\mathbb{N}$ при $n \hm\geq 
N$ справедливо}
\begin{multline*}
\hspace*{-3pt}\sup\limits_{\mu\in m_p[\eta_n]} \p \left(\hat{k}_F \geq \kappa_n \right) \leq 
4 n \exp\left\{-\fr{m}{256n}  \kappa_n q_n \gamma_n^2    \right\}+{}\\
{}+ 22\left(1+\fr{8n}{\kappa_n q_n \gamma_n}\right)^{1/2} n m 
\alpha\left(\left[\fr{n}{2m}\right]\right).
\end{multline*}



\smallskip

\noindent
\textbf{Лемма 2.}\ 
%\label{lem1}
\textit{Пусть $\eta_n \hm\leq b\hm<1$, $m\in[1;n/2]\cap\mathbb{N}$, а~$\alpha(\cdot)$~--- 
коэффициент сильного перемешивания компонент вектора~$x$. Для некоторого 
$N\hm\in\mathbb{N}$ при $n \hm\geq N$ справедливо}
\begin{multline*}
\sup\limits_{\mu\in l_0[\eta_n]} \p \left(\hat{k}_F \geq \kappa_n^0 \right) 
\leq{}\\
{}\leq 4 n \exp\left\{-\fr{(1-b)m}{64n}\,  \kappa_n^0 q_n \gamma_n^2    
\right\}+{}\\
{}+ 22\left(1+\fr{4n}{(1-b)\kappa_n^0 q_n \gamma_n}\right)^{1/2} n m 
\alpha\left(\left[\fr{n}{2m}\right]\right).
\end{multline*}

Следующие два утверждения доказаны в~\cite{Bosq} и~представляют собой аналоги 
неравенств Хеффдинга и~Бернштейна для слабо зависимых случайных величин.


\smallskip

\noindent
\textbf{Лемма 3.}\
\textit{Пусть для набора действительных случайных величин $X_1, \ldots, X_n$ 
с~коэффициентом сильного перемешивания $\alpha(\cdot)$ выполняется ${\mathsf E} X_i \hm=0$, 
$|X_i|\hm\leq b$, $i\hm=\overline{1,n}$. Тогда для любого целого числа $m\hm\in[1; n/2]$ 
и~любого $\eps\hm>0$ справедливо}
\begin{multline*}
\p\left(\left|\sum\limits_{i=1}^n X_i\right| > n\eps \right) \leq 4 
\exp\left\{-\fr{\eps^2 m}{8 b^2}\right\}+ {}\\
{}+
22\left(1+\fr{4b}{\eps}\right)^{1/2} m\, 
\alpha\left(\left[\fr{n}{2m}\right]\right).
\end{multline*}


\smallskip

\noindent
\textbf{Лемма 4.}\
\textit{Пусть для набора действительных случайных величин $X_1, \ldots, X_k$ 
с~коэффициентом сильного перемешивания $\alpha(\cdot)$ выполняется ${\mathsf E} X_i \hm=0$, 
$|X_i|\hm\leq b$, $i\hm=\overline{1,k}$. Тогда для любого целого числа $m\hm\in[1; k/2]$ 
и~любого $\eps\hm>0$ справедливо}
\begin{multline*}
\p\left(\left|\sum\limits_{i=1}^k X_i\right| > \eps \right) \leq 4 
\exp\left\{-\fr{\eps^2 m}{8 v^2 k^2}\right\}+{}\\
{}+ 22\left(1+\fr{4bk}{\eps}\right)^{1/2} m\, 
\alpha\left(\left[\fr{k}{2m}\right]\right),
\end{multline*}
\textit{где $p = k/(2m)$}:
\begin{multline*}
v^2 =
 \fr{b \eps}{2k} + {}\\
 {}+\fr{2}{p^2} \,  \max\limits_{ j\in[0,\,2m-1]} 
{\mathsf E} \big( ([jp]+1-jp)X_{[jp]+1} + X_{[jp]+2}+{}\\
{}+ \cdots +  X_{[(j+1)p]} + ((j+1)p-[(j+1)p])X_{[(j+1)p+1]}\big)^2.
\end{multline*}

\noindent
\textbf{Замечание.}
Если существует такое число $S \hm> 0$, что сразу для всех $i\hm\in[1;k]$  выполняется 
${\mathsf E} X_i^2 \hm\leq S^2$, то в~качестве~$v^2$ можно взять
$$
v^2 = \fr{b \eps}{2k} + 8 S^2.
$$


Д\,о\,к\,а\,з\,а\,т\,е\,л\,ь\,с\,т\,в\,о\ \ сле\-ду\-юще\-го утверж\-де\-ния приведено в~работе~\cite{AdaptingFDR}.

\smallskip

\noindent
\textbf{Лемма 5.}\ 
\textit{Для $y\leq 0{,}01$ справедливы представления}
\begin{multline}
\label{lem1eq1}
z^2(y) = 2 \ln y^{-1} - \ln \ln y^{-1} - r_2(y), \\
 r_2(y) \in [1{,}8; 3];
\end{multline}

\noindent
\begin{equation}
\label{lem1eq2}
z(y) = \sqrt{2 \ln y^{-1}} - r_1(y), \, \, r_1(y) \in [0; 1{,}5].
\end{equation}


\section{Асимптотическая нормальность оценки риска при~применении FDR-процедуры в~условиях слабой зависимости}

Перейдем к~описанию достаточных условий для асимптотической нормальности оценки 
риска $\hat{R}(\hat{t}_F)$ в~случае $\mu \hm\in m_p[\eta_n]$.

\smallskip

\noindent
\textbf{Теорема~1.}\
\textit{Пусть $\mu \hm\in m_p[\eta_n],$ $\eta_n^p \hm\in[n^{-\delta_1}; n^{-\delta_2}],$ $1/2 \hm< 
\delta_2 \hm< \delta_1<1;$ имеются такие константы $c_1, c_2>0$, что для 
коэффициента сильного перемешивания $\alpha(\cdot)$ компонент вектора $x$ 
справедливо  $\alpha(k) \hm\leq c_1 k^{-1-(5/2)\delta_1/(1-\delta_1)-c_2},$ 
$k\hm=\overline{1,n-1};$ $q_n \hm< c_3 \hm< 1;$ $\mathrm{lim\,inf} q_n \ln n \hm= c_4 \hm> 0;$ и,~кроме того, 
для максимального коэффициента корреляции $\rho(\cdot)$ компонент вектора~$x$ 
справедливо}
$$
\sum\limits_{k = 1}^{\infty} \sup\limits_{n\geq k+1} \rho(k) \equiv 
\sum\limits_{k = 1}^{\infty}  \rho^\star (k) = c_5 < \infty. 
$$
\textit{Тогда при $n \to \infty$}
$$
\fr{\hat{R}(\hat{t}_F) - R(T_m)}{C_\rho \sqrt{2n}} \Rightarrow N(0, 1),
$$
\textit{где}
$$
C_\rho = \sigma^2\sqrt{1 +  \lim\limits_{n\to\infty} \fr{1}{n} \sum\limits_{j\neq i} \mathrm{corr}^2 (x_i, x_j)}.
$$

\noindent
Д\,о\,к\,а\,з\,а\,т\,е\,л\,ь\,с\,т\,в\,о\  \
 приводится для метода мягкой пороговой обработки; в~случае жесткой пороговой 
обработки доказательство аналогично. Обозначим
$$
U(T) = \hat{R}(T) -  \hat{R}(T_m) = \sum \limits_{i=1}^n H_i(T, T_m),
$$
где
$$
H_i(T, T_m) = F[x_i, T] - F[x_i, T_m].
$$
Имеем

\vspace*{-3pt}

\noindent
\begin{multline}
\label{D00}
\hat{R}(\hat{t}_F) - R(T_m) + \hat{R}(T_m) - \hat{R}(T_m) ={}\\
{}= \hat{R}(T_m) - 
R(T_m) + U(\hat{t}_F).
\end{multline}
Покажем, что
\begin{equation}
\label{D0}
\fr{\hat{R}(T_m) - R(T_m)}{C_\rho\sqrt{2n}} \Rightarrow N(0, 1).
\end{equation}


Повторяя рассуждения из~\cite{KuShe2016_1,KuShe2016_2,Jansen}, можно показать, 
что $T_m \hm\geq t_{\kappa_n}$. Учитывая также $T_m\hm \leq T_U$, имеем 
$$
C \sqrt{\ln n} \leq T_m \leq C^\prime \sqrt{\ln n}
$$ 
для некоторых положительных констант $C$ и~$C^\prime$.

\columnbreak

В случае мягкой пороговой обработки $\hat{R}(T_m)$ представляет собой 
несмещенную оценку~$R(T_m)$, а~при жесткой пороговой обработке и~выполнении 
условий теоремы смещение стремится к~нулю при делении на $\sqrt{n}$~\cite{Mallat}.

Для дисперсии числителя~(\ref{D0}) имеем:
\begin{multline*}
{\mathsf D} \left(\hat{R}(T_m) - R(T_m)\right) = \sum\limits_{i=1}^n {\mathsf D} F[x_i, T_m] + {}\\
{}+
\sum\limits_{i=1}^n\sum\limits_{\substack{j=1 \\  j\neq i}}^n \mathrm{cov}\left( F[x_i, T_m], F[x_j, 
T_m] \right).
\end{multline*}

Поскольку $\mu \in m_p[\eta_n]$,
\begin{equation}
\left.
\begin{array}{l}
 \displaystyle\sum\limits_{i: |\mu_i| > 1/T_1} {\mathsf D} F[x_i, T_m]  \leq{}\\
 \hspace*{15mm}{}\leq  4\left(\sigma^2 + T_m^2\right)^2 n \eta_n^p 
T_1^p = o(n);
\\[6pt]
\displaystyle \sum\limits_{\substack{{i,j: \max\{|\mu_i|, |\mu_j|\} > 1/T_1,}\\{j\neq i}}}  \hspace*{-12mm}\mathrm{cov}\,(F[x_i, 
T_m],F[x_j, T_m])  \leq{}\\
\hspace*{10mm}{}\leq 16\left(\sigma^2 + T_m^2\right)^2 n \eta_n^p T_1^p c_5 = o(n). 
\end{array}
\right\}    
\label{D2}
\end{equation}
Далее, учитывая что ${\mathsf D} x_i^2 \hm= 2\sigma^4 \hm+ 4\sigma^2 \mu_i^2$, нетрудно 
убедиться, что
\begin{multline}
\label{D3}
\sum\limits_{i: |\mu_i| \leq 1/T_1}\hspace*{-4mm} {\mathsf D} F[x_i, T_m] ={}\\
{}= \sum\limits_{i: |\mu_i| \leq 1/T_1} \hspace*{-4mm} {\mathsf D} 
x_i^2 + o(n) = 2\sigma^4 n + o(n).
\end{multline}


Введем обозначение 
$$
D_n = \left\{(i,j) : \max\left\{|\mu_i|, |\mu_j|\right\}  \leq \fr{1}{T_1}\,, \enskip j\hm\neq i\right\}.
$$
 Для суммы ковариаций аналогично~(\ref{D3}) получим
\begin{multline*}
\sum\limits_{(i,j)\in D_n} \hspace*{-2mm}\mathrm{cov}\left( F[x_i, T_m], F[x_j, T_m] \right) = {}\\
{}=
\sum\limits_{(i,j)\in D_n} \hspace*{-2mm}\mathrm{cov}\left( x_i^2, x_j^2 \right) + o(n).
\end{multline*}
Воспользуемся тождеством~\cite{Eroshenko}
$$
\mathrm{cov}\left (x_i^2, x_j^2\right) = 4 {\mathsf E} x_i {\mathsf E} x_j \mathrm{cov}\left(x_i, x_j\right) + 2 \mathrm{cov}^2 \left(x_i, x_j\right)
$$
для вектора $(x_i, x_j)$, имеющего двумерное нормальное распределение. Заметим, 
что
\begin{gather*}
 \sum\limits_{(i,j)\in D_n} 4 | {\mathsf E} x_i {\mathsf E} x_j \mathrm{cov}\left(x_i, x_j\right)| \leq 8 T_1^{-2} 
\sigma^2 n c_5 = o(n);
\\
\sum\limits_{(i,j)\in D_n} 2 \mathrm{cov}^2 (x_i, x_j)  = 2\sigma^4 \sum\limits_{(i,j)\in D_n} 
\mathrm{corr}^2 (x_i, x_j). 
\end{gather*}
Более того, поскольку  %< 4 \sigma^2 n c_5.$$
\begin{equation*}
\sum\limits_{\substack{{i,j: \max\{|\mu_i|, |\mu_j|\} > 1/T_1} \\ {j\neq i}}}
\hspace*{-10mm}\mathrm{corr}^2 (x_i, x_j)  
\leq  4 n \eta_n^p T_1^p c_5 =  o(n),
\end{equation*}
имеем
\begin{multline*}
\sum\limits_{(i,j)\in D_n} \mathrm{corr}^2 (x_i, x_j) ={}\\
{}= \sum\limits_{j\neq i} \mathrm{corr}^2 (x_i, x_j) 
+o(n)= c_6 n + o(n),
\end{multline*}
где
$$
c_6 = \lim\limits_{n\to\infty} \fr{1}{n} \sum\limits_{j\neq i} \mathrm{corr}^2 (x_i, x_j) 
\leq 2 c_5.
$$
Полагая $C_\rho \hm= \sigma^2\sqrt{1 + c_6}$, получим, наконец,
\begin{equation}
\label{D1}
{\mathsf D} \left(\hat{R}(T_m) - R(T_m)\right)  =  2 n C_\rho^2 + o(n).
\end{equation}
Заметим, что из~(\ref{D2}), (\ref{D3}) и~(\ref{D1}) следует, что
\begin{equation}
\label{D5}
\sup\limits_{n} \fr{\sum\nolimits_{i=1}^n {\mathsf D} F[x_i, T_m]}{V_n^2} < \infty\,,
\end{equation}
где 
$$
V_n^2 = {\mathsf D} \sum\limits_{i=1}^n \left(F[x_i, T_m] \hm- {\mathsf E} F[x_i, T_m]\right).
$$
Кроме того, поскольку $F[x_i, T_m]$ по модулю ограничены величиной $\sigma^2 \hm+ 
T_m^2$, выполнено условие Линдеберга: для любого $\eps\hm>0$ при $n \hm\to \infty$
\begin{multline}
\label{D6}
\!\!\!\fr{1}{V_n^2}\sum\limits_{i=1}^n {\mathsf E} \left( \!\left( F\left[x_i, T_m\right]\! -\! {\mathsf E} F\left[x_i, T_m\right]\right)^2 
\Ik \left(\vert F\left[x_i, T_m\right] -{}\right.\right.\hspace*{-2.69505pt}\\
\left.\left.{}- {\mathsf E} F\left[x_i, T_m\right]\vert >\eps V_n\right)\!
\vphantom{\left( F\left[x_i, T_m\right]\! -\! {\mathsf E} F\left[x_i, T_m\right]\right)^2}
\right) 
\to  0\,.
\end{multline}
Из~(\ref{D1})--(\ref{D6}), очевидного неравенства
$$ 
\lim\limits_{k\to\infty} \sup\limits_{n\geq k+1}\rho(k) \equiv 
\lim\limits_{k\to\infty} \rho^\star (k)  < 1
$$
 и~центральной предельной теоремы для сильно перемешанных случайных величин~\cite{Peligrad} следует~(\ref{D0}).

Перейдем к~доказательству того, что $U(\hat{t}_F) \, n^{-1/2} \overset{\, \p \, }{\to} 0$.
Всюду далее, не ограничивая общности, полагаем $\sigma=1$. 
Введем обозначения:

\noindent
\begin{align*}
S_1(T) &= \sum\limits_{i: |\mu_i| > 1/T_1} H_i(T, T_m); \\
S_2(T) &= \sum\limits_{i: |\mu_i| \leq 1/T_1} H_i(T, T_m); 
\\
N_1(a, b) &= \sum\limits_{i: |\mu_i| > 1/T_1} \Ik (a<|x_i|\leq b); \\ 
N_2(a, b) &= \sum\limits_{i: |\mu_i| \leq 1/T_1} \Ik (a<|x_i|\leq b);
\end{align*}

\noindent
\begin{align*}
Z_l(T) &= S_l(T) - {\mathsf E} S_l(T),\enskip l = 1,2\,; \\  
d_n &= \fr{T_U -  t_{\kappa_n}}{n};\\
T_j^{\prime} &= t_{\kappa_n}+j d_n,\enskip j = \overline{0,n-1}\,.
\end{align*} 

\vspace*{-3pt}

\noindent
Для произвольного $\eps>0$

\vspace*{-3pt}

\noindent
\begin{multline}
\p \left( \fr{|U(\hat{t}_F)|}{\sqrt{n}}> 4\eps \right) \leq 
\p\left(\hat{t}_F \leq t_{\kappa_n}\right) + {}\\
{}+\p \left(\fr{\sup\nolimits_{T\in 
[t_{\kappa_n}, T_U]} |U(T)|}{\sqrt{n}}>4\eps \right)\leq  {}\\
{}\leq \p\left(\hat{t}_F \leq t_{\kappa_n}\right) + \p\left(\fr{\sup\nolimits_{T\in 
[t_{\kappa_n}, T_U]} |{\mathsf E} U(T)|}{\sqrt{n}}>\eps\right)+{}\\
{}+ \p \left(\sup\limits_{T\in [t_{\kappa_n}, T_U]} |Z_1(T)| > 
\eps\sqrt{n}\right) +{}\\
{}+ \p \left(\sup\limits_{j \in [0, n-1]} |Z_2(T_j^{\prime})| > 
\eps\sqrt{n}\right) +{}\\
{}+ \p \left(\sup\limits_{\substack{j \in [0, n-1] \\
 T\in [T_j^{\prime},T_j^{\prime}+d_n]}} |Z_2(T)-Z_2(T_j^{\prime})| > \eps\sqrt{n}\right).
\label{M1}
\end{multline}
Заметим, что $\gamma_n\hm > \ln^{-1} n$, $\kappa_n\hm > n \eta_n^p \ln ^{-1} n \hm\geq 
n^{1-\delta_1} \ln ^{-1} n$ и~$q_n\hm > c_4 \ln ^{-1} n /2$ для всех достаточно 
больших~$n$.
Для первого слагаемого в~(\ref{M1}) по лемме~1 с~$m \hm= n^{\delta_1} \ln 
^7 n$ для  больших~$n$ имеем

\vspace*{-3pt}

\noindent
\begin{multline}
\label{M1next}
\p\left(\hat{t}_F \leq t_{\kappa_n}\right)  = \p \left(\hat{k}_F \geq \kappa_n 
\right) \leq 4 n e^{-\ln^2 n} + {}\\
{}+n^{1+(3/2)\,\delta_1} \ln^9 n \, 
\alpha\left(\left[\fr{n^{1-\delta_1}}{\ln^{7} n}\right]\right) = o(1)
\end{multline}
при $n\to\infty$. 
Для оценки второго слагаемого в~(\ref{M1}) заметим, что при $T \hm\in 
[t_{\kappa_n}, T_U]$ справедливо
\begin{equation}
\label{M2}
{\mathsf E} H_i(T, T_m) \leq T_U^2 + 1.
\end{equation}
Если же кроме $T \hm\in [t_{\kappa_n}, T_U]$ также выполнено $|\mu_i| \hm\leq T_1^{-1}$, то

\vspace*{-6pt}

\noindent
\begin{multline*}
|{\mathsf E} H_i (T, T_m)| \leq 2 T_U^2 \, \p \left(|x_i| > t_{\kappa_n}\right) \leq {}\\
{}\leq2 
T_U^2 \, \p \left(|x_i-\mu_i| > t_{\kappa_n}-T_1^{-1}\right) \leq{}\\
{}\leq 2 T_U^2  \exp\left\{ -\fr{1}{2} \left(t_{\kappa_n} - T_1^{-
1}\right)^2 \right\}  \leq{}\\
{}\leq
 4 (\ln n)  \exp\left\{ -\fr{1}{2} 
\left(z\left(\fr{q_n\kappa_n}{2n}\right)\right)^2 + t_{\kappa_n} T_1^{-
1}\right\},
\end{multline*}

\vspace*{-2pt}

\noindent
где использовано неравенство 

\noindent
$$
2(1-\Phi(x))\hm \leq \fr{e^{-x^2/2}}{x}
$$

\pagebreak


\noindent
 для $x\hm\geq 0$ 
($\Phi(x)$~--- функция распределения $N(0,1)$). Рас\-смот\-рим выражение 
в~экспоненте. Второе слагаемое не превышает $1\hm+o(1)$ при $n\hm\to\infty$, поскольку 
$t_{\kappa_n} \hm\leq T_1 (1+o(1))$ при $\sigma\hm=1$, что нетрудно получить из 
определения~$t_{\kappa_n}$, пред\-став\-ле\-ния~(\ref{lem1eq2}) и~ограничения на~$q_n$ 
из формулировки тео\-ре\-мы. Для первого слагаемого, используя пред\-став\-ле\-ние~(\ref{lem1eq1}) 
и~ограничения, наложенные на~$q_n$, при больших~$n$ получим
\begin{multline*}
-\fr{1}{2}\left(z\left(\fr{q_n \kappa_n}{2n}\right)\right)^2 \leq - \ln 
\fr{2n (1-q_n-\gamma_n)}{q_n n \eta_n^p T_1^{-p}} + {}\\
{}+\fr{1}{2} \ln 
\left((1+o(1)) \ln \eta_n^{-p}\right) + \fr{3}{2} \leq{}\\
{}\leq \ln \fr{c_3}{1-c_3} + \ln \eta_n^p + \ln T_1^{-p} + \ln T_1 + 
\fr{3}{2}+ o(1).
\end{multline*}
Из приведенных соотношений следует, что с~некоторой константой $c_7 = c_7(c_3, 
p, \delta_1, \delta_2, c_4)$
\begin{equation}\label{M3}
\sup\limits_{\substack{i: |\mu_i| \leq 1/T_1 \\ T\in [t_{\kappa_n}, T_U]}} |{\mathsf E} 
H_i (T, T_m)|  \leq c_7 (\ln n)^{(3-p)/2}\eta_n^p.
\end{equation}
Из (\ref{M2}) и~(\ref{M3}) с~учетом $\delta_2 \hm> 1/2$ следует
\begin{multline*}
\sup\limits_{T\in [t_{\kappa_n}, T_U]} |{\mathsf E} U(T)| \leq{}\\
{}\leq 
 n\eta_n^p T_1^p 
(T_U^2+1) + c_7 (\ln n)^{(3-p)/2} n \eta_n^p = o(\sqrt{n})
\end{multline*}
при $n\to\infty$, а следовательно, для любого $\eps\hm>0$ второе слагаемое в~(\ref{M1}) обращается в~ноль для всех достаточно больших~$n$.

Далее, поскольку при $T \hm\leq T_U$ и~$\sigma\hm=1$
$$
|H_i(T, T_m) - {\mathsf E} H_i(T, T_m)| \leq 2 (T_U^2 +2), \enskip i=\overline{1, n}\,,
$$
а число слагаемых в~$Z_1(T)$ не превосходит $n\eta_n^p T_1^p$, имеем
$$
\sup\limits_{T\in [t_{\kappa_n}, T_U]} |Z_1(T)|  \leq 2 n\eta_n^p T_1^p (T_U^2 
+2) = o(\sqrt{n})
$$
при $n\to\infty$, а следовательно, для любого $\eps\hm>0$ и~третье слагаемое в~(\ref{M1}) обращается в~ноль для всех достаточно больших~$n$.

Перейдем к~оценке четвертого слагаемого в~(\ref{M1}). Аналогично~(\ref{M3}) 
можно получить:
\begin{multline}
\label{M10}
\!\!\sup\limits_{\substack{i: |\mu_i| \leq 1/T_1 \\ T\in [t_{\kappa_n}, T_U]}} \!{\mathsf D} 
H_i (T, T_m)  \leq \!\sup\limits_{\substack{i: |\mu_i| \leq 1/T_1 \\ T\in 
[t_{\kappa_n}, T_U]}} \!{\mathsf E} \left(H_i (T, T_m)\right)^2  \leq{}\\
{}\leq 2 c_7 (\ln n)^{(5-p)/2} \eta_n^p.
\end{multline}
По лемме~4 с~$m \hm= \sqrt{n} (\ln n)^3$ и~$k \hm= n-[n\eta_n^p T_1^p]$ 
для четвертого слагаемого в~(\ref{M1}) имеем:

\noindent
\begin{multline}
\p \left(\sup\limits_{j \in [0, n-1]} |Z_2(T_j^\prime)| > \eps\sqrt{n}\right) 
\leq {}\\
{}\leq \sum\limits_{j \in [0, n-1]} \hspace*{-3mm}\p \left( |Z_2(T_j^\prime)| > \varepsilon\sqrt{n}\right)\leq{}\\
{}\leq 4 n \exp \left\{ - \fr{\eps^2 n^{3/2} (\ln n)^3}{n-[n\eta_n^p T_1^p]}\!\Bigg/\! \big( 8 (T_U^2+2)\eps\sqrt{n} +{}\right.\\
\left.{}+ 128 c_7 (\ln n)^{(5-p)/2} \eta_n^p  (n-
[n\eta_n^p T_1^p])\big) 
\vphantom{ \fr{\eps^2 n^{3/2} (\ln n)^3}{n-[n\eta_n^p T_1^p]}}
\right\} +{}\\
{}
+ 22 \left(1+\fr{8(T_U^2+2) (n-[n\eta_n^p T_1^p])}{\eps 
\sqrt{n}}\right)^{1/2}\times{}\\
{}\times n^{3/2} (\ln n)^3 \alpha\left(\left[\fr{n-[n\eta_n^p 
T_1^p]}{2 (\ln n)^3 \sqrt{n}}\right]\right).
\label{M5}
\end{multline}
Используя ограничения $n^{-\delta_1}\hm\leq \eta_n^p \leq n^{-\delta_2}$ 
и~$1/2\hm<\delta_2\hm<\delta_1\hm<1$, из~(\ref{M5}) получим для любого $\eps\hm>0$
$$
\p \left(\sup\limits_{j \in [0, n-1]} |Z_2(T_j^\prime)| > \eps\sqrt{n}\right) 
\to 0
$$
при $n \to \infty$.

Рассмотрим, наконец, пятое слагаемое в~(\ref{M1})). Заметим, что при $0\hm< a \hm< b$ 
справедливо
$$
|Z_2(b)-Z_2(a)| \leq 2 |N_2(a,b)-{\mathsf E} N_2(a,b)| + n (b^2-a^2).
$$
Полагая $a = T_j^\prime$, $b \hm= T \hm\in [T_j^\prime, T_j^\prime+d_n]$ для 
произвольного $j \hm\in [0, n-1]$ и~учитывая, что
$$
(T^2 - (T_j^\prime )^2) = (T - T_j^\prime)(T+ T_j^\prime ) \leq  2 d_n T_U < 2 
T_U^2 n^{-1}; 
$$

\vspace*{-12pt}

\noindent
\begin{multline*}
\p\left(T_j^\prime < |x_i| \leq T \right) \leq \p\left(T_j^\prime < |x_i| \leq 
T_j^\prime+d_n\right) <{}\\
{}< d_n < T_U n^{-1}, 
\end{multline*}
получим  оценку
$$
|Z_2(T)-Z_2(T_j^\prime)| \leq 2 N_2(T_j^\prime, T) +  3 T_U^2 .
$$
Далее, поскольку $N_2 (T_j^\prime, T) \hm\leq N_2 (T_j^\prime, T_j^\prime+d_n)$ и~${\mathsf E} N_2 (T_j^\prime, T_j^\prime+d_n) \hm< T_U^2$,
имеем
\begin{multline*}
\sup\limits_{T \in [T_j^\prime, T_j^\prime+d_n]} |Z_2(T)-Z_2(T_j^\prime)| \leq {}\\
{}\leq
2 \left|N_2 (T_j^\prime, T_j^\prime+d_n) - {\mathsf E} N_2 (T_j^\prime, 
T_j^\prime+d_n)\right| +  5 T_U^2 .
\end{multline*}
Аналогично~(\ref{M3}) показывается, что
\begin{multline}
\label{M11}
\sup\limits_{\substack{i : |\mu_i| \leq 1/T_1 \\ j \in [0, n-1]}} {\mathsf D} \Ik 
(T_j^\prime < |x_i| \leq T_j^\prime + d_n) <{}\\
{}< c_7 (\ln n)^{(1-p)/2} \eta_n^p.
\end{multline}
Пусть $n > N(\eps)$ настолько, что 
$$
\fr{\eps\sqrt{n} - 5 T_U^2}{2} > \fr{\eps \sqrt{n} }{4}\,.
$$
%
 Тогда для пятого слагаемого в~(\ref{M1}) по лемме~4 с~$m \hm= 
\sqrt{n} (\ln n)^2$ и~$k \hm= n\hm-[n\eta_n^p T_1^p]$ имеем
\begin{multline}
\p \left(\sup\limits_{\substack{j \in [0, n-1] \\ T\in 
[T_j^{\prime},T_j^{\prime}+d_n]}} |Z_2(T)-Z_2(T_j^{\prime})| > 
\eps\sqrt{n}\right) \leq{}\\
{}\leq  \sum\limits_{j \in [0, n-1]} \p \left(  \left|N_2 (T_j^\prime, 
T_j^\prime+d_n) -{}\right.\right.\\
\left.\left.{}- {\mathsf E} N_2 (T_j^\prime, T_j^\prime+d_n)\right| > \fr{\eps\sqrt{n}}{4} 
\right) \leq{}\\
{}\leq  4n \exp \left\{ -  \fr{\eps^2 n^{3/2} (\ln n)^2}{(n-[n\eta_n^p T_1^p])^{-1}}\Bigg/ 
\big( 16 \eps \sqrt{n} +{}\right.\\
\left.{}+ 64 c_7 (\ln n)^{(1-p)/2} \eta_n^p (n-[n\eta_n^p 
T_1^p]) \big) 
\vphantom{\fr{\eps^2 n^{3/2} (\ln n)^2}{(n-[n\eta_n^p T_1^p])^{-1}}}
\right\} +{}\\
{}+ 22 \left(1+\fr{16 (n-[n\eta_n^p T_1^p])}{\eps \sqrt{n}}\right)^{1/2}\times{}\\
{}\times 
n^{3/2} (\ln n)^2 \alpha\left(\left[\fr{n-[n\eta_n^p T_1^p]}{2 (\ln n)^2 
\sqrt{n}}\right]\right).
\label{M6}
\end{multline}
Используя ограничения $n^{-\delta_1}\hm\leq \eta_n^p\hm \leq n^{-\delta_2}$ 
и~$1/2\hm<\delta_2\hm<\delta_1<1$, из~(\ref{M6}) получим для любого $\eps\hm>0$
$$
\p \left(\sup\limits_{\substack{j \in [0, n-1] \\ T\in 
[T_j^{\prime},T_j^{\prime}+d_n]}} |Z_2(T)-Z_2(T_j^{\prime})| > 
\eps\sqrt{n}\right) \to 0
$$
при $n \to \infty$.

Таким образом, показано, что для любого $\eps>0$ все слагаемые в~(\ref{M1}) 
стремятся к~нулю при $n\to\infty$. Следовательно,
$$
\fr{|U(\hat{t}_F)|}{\sqrt{n}}  \overset{\, \p \, }{\to} 0 \,,
$$
что вместе с~(\ref{D0}) завершает доказательство тео\-ремы.~\hfill$\square$

\smallskip

Следующая теорема дает достаточные условия для асимптотической нормальности 
оценки риска $\hat{R}(\hat{t}_F)$ в~случае $\mu \hm\in l_0[\eta_n]$.

\smallskip

\noindent
\textbf{Теорема 2.}\ 
\textit{Пусть $\mu \hm\in l_0[\eta_n]$, $\eta_n\hm\in[n^{-\delta_1}, n^{-\delta_2}]$, $1/2\hm < 
\delta_2\hm < \delta_1\hm<1;$ имеются такие константы $c_1, c_2\hm>0$, что для 
коэффициента сильного перемешивания $\alpha(\cdot)$ компонент вектора~$x$ 
справедливо} 
\begin{gather*}
\alpha(k) \leq c_1 k^{-1-(5/2)\delta_1/(1\hm-\delta_1)\hm-c_2},\enskip 
k=\overline{1,n-1};\\
 q_n < c_3 < 1;\enskip \mathrm{lim\,inf} q_n \ln n = c_4 > 0;
\end{gather*}
\textit{для максимального коэффициента корреляции~$\rho(\cdot)$ компонент вектора~$x$ 
справедливо}
$$
\sum\limits_{k = 1}^{\infty} \sup\limits_{n\geq k+1} \rho(k) \equiv 
\sum\limits_{k = 1}^{\infty}  \rho^\star (k) = c_5 < \infty. 
$$
\textit{Тогда при $n \to \infty$}
$$
\fr{\hat{R}(\hat{t}_F) - R(T_m)}{C_\rho \sqrt{2n}} \Rightarrow N(0, 1),
$$
\textit{где}
$$
C_\rho = \sigma^2\sqrt{1 +   \lim\limits_{n\to\infty} \fr{1}{n} 
\sum\limits_{j\neq i} \mathrm{corr}^2 (x_i, x_j)}\,.
$$

\noindent
Д\,о\,к\,а\,з\,а\,т\,е\,л\,ь\,с\,т\,в\,о\  проводится аналогично доказательству теоремы~1. 
Переменная~$D_n$ теперь определяется как $D_n \hm= \{(i,j) : 
|\mu_i|\hm=|\mu_j|=0$, $j\hm\neq i\}$. Условия вида $|\mu_i|\hm<T_1^{-1}$ (вида 
$|\mu_i|\hm\geq T_1^{-1}$) заменяются условиями  $\mu_i\hm=0$ (соответственно 
$|\mu_i|\hm>0$).
Поскольку $\mu \hm\in l_0[\eta_n]$, количество~$i$ таких, что $|\mu_i|\hm>0$ 
(а~значит, и~число слагаемых в~$Z_1(T)$), не превышает~$[n \eta_n]$.

Для оценки первого слагаемого в~(\ref{M1}) используется лемма~2, 
в~которой можно взять, например, $b\hm=1/2$, а~для~$\kappa_n^0$ использовать оценку 
$\kappa_n^0 \hm> n \eta_n$. Формулы (\ref{M3}),  (\ref{M10}) и~(\ref{M11}) 
принимают вид соответственно
\begin{align*}
\sup\limits_{\substack{i: \mu_i =0 \\ T\in [t_{\kappa_n^0}, T_U]}} |{\mathsf E} H_i (T, 
T_m)| & \leq c_8 (\ln n)^{3/2} \eta_n ;
\\
\sup\limits_{\substack{i: \mu_i =0 \\ T\in [t_{\kappa_n^0}, T_U]}} {\mathsf D} H_i (T, 
T_m)  & \leq 2 c_8 (\ln n)^{5/2} \eta_n;
\\
\sup\limits_{\substack{i : \mu_i =0 \\ j \in [0, n-1]}} {\mathsf D} \Ik (T_j^\prime < 
|x_i| \leq T_j^\prime + d_n) &< c_8 (\ln n)^{1/2} \eta_n,
\end{align*}
где $c_8 = c_8(c_3,\delta_1, \delta_2, c_4)$. В~остальном доказательство 
аналогично.~\hfill$\square$

\section{Сильная состоятельность оценки риска при~применении FDR-процедуры 
в~условиях слабой зависимости}

Следующая теорема дает достаточные условия для сильной состоятельности оценки 
риска $\hat{R}(\hat{t}_F)$ в~случаях $\mu \hm\in m_p[\eta_n]$ и~$\mu \hm\in 
l_0[\eta_n]$.

\smallskip

\noindent
\textbf{Теорема 3.}
\textit{Пусть $\mu\hm \in m_p[\eta_n]$, $\eta_n^p\hm\in[n^{-\delta_1}, n^{-\delta_2}]$ либо 
$\mu \hm\in l_0[\eta_n]$, $\eta_n\hm\in[n^{-\delta_1}, n^{-\delta_2}]$; $0 \hm< \delta_2 
\hm< \delta_1<1$; имеются такие константы $c_1, c_2\hm>0$, что для коэффициента 
сильного перемешивания $\alpha(\cdot)$ компонент вектора~$x$ справедливо}  
$\alpha(k) \hm\leq c_1 k^{-2-(7/2)\delta_1/(1\hm-\delta_1)\hm-c_2}$, $k\hm=\overline{1,n-1}$; 
$q_n \hm< c_3 \hm< 1$; $\mathrm{lim\,inf} q_n \ln n \hm= c_4 \hm> 0$. \textit{Тогда при} $n \hm\to \infty$
$$
\fr{\hat{R}(\hat{t}_F) - R(T_m)}{n} \rightarrow 0 \, \, \,\textit{п.~в.}
$$


\noindent
Д\,о\,к\,а\,з\,а\,т\,е\,л\,ь\,с\,т\,в\,о\,.  Воспользуемся представлением~(\ref{D00}).

Покажем, что $(\hat{R}(T_m)-R(T_m))n^{-1}\hm \to 0$ п.~в.\ при $n\hm\to\infty$. 
При мягкой пороговой обработке ${\mathsf E} \hat{R}(T_m) \hm= R(T_m)$, а~при жесткой 
пороговой обработке
\begin{multline*}
\fr{\hat{R}(T_m)-R(T_m)}{n} = {}\\
{}=\fr{\hat{R}(T_m)-{\mathsf E} \hat{R}(T_m)}{n} 
+\fr{{\mathsf E}\hat{R}(T_m)-R(T_m)}{n}\,,
\end{multline*}
где второе слагаемое стремится к~нулю при $n\to\infty$ \cite{Mallat}. 
Следовательно, достаточно показать, что $(\hat{R}(T_m)\hm-{\mathsf E}\hat{R}(T_m))n^{-1} \hm\to 0$ п.~в.

Полагая в~лемме~3 $X_i \hm= F[x_i, T_m] \hm- {\mathsf E} F[x_i, T_m]$, $b \hm= 
2(\sigma^2\hm+T_m^2)$ и~$m \hm= n^{1/4}$ и~учитывая ограничения на $\alpha(\cdot)$ из 
условия, нетрудно убедиться, что для всех~$n$
$$
\p \left(\left| \fr{\hat{R}(T_m)-{\mathsf E} \hat{R}(T_m)}{n}\right| >\eps \right) 
\leq \fr{c_5}{n^{1+c_6}}\,, 
$$
где константы $c_5$, $c_6$ положительны. Отсюда
$$
\sum\limits_{n=1}^{\infty}\p \left(\left|\fr{\hat{R}(T_m)-{\mathsf E} 
\hat{R}(T_m)}{n}\right| >\eps \right) < \infty,
$$
и по теореме~1.3.4 из~\cite{Serfling2002} 
$$
\left(\hat{R}(T_m)-{\mathsf E}\hat{R}(T_m)\right)n^{-1} \to 0~\mbox{п.~в.}
$$



Покажем теперь, что  $U(\hat{t}_F) \, n^{-1}\hm \to 0$ п.~в. Доказательство 
проведено для $\mu \hm\in m_p[\eta_n]$, в~случае $\mu\hm \in l_0[\eta_n]$ 
доказательство аналогично.
Аналогично формуле~(\ref{M1}), для произвольного $\eps\hm>0$ в~терминах тео\-ре\-мы~1 имеем
\begin{multline*}
\p \left( \fr{|U(\hat{t}_F)|}{n}> 4\eps \right) \leq \p\left(\hat{t}_F 
\leq t_{\kappa_n}\right) +{}\\
{}+ \p\left(\fr{\sup\nolimits_{T\in [t_{\kappa_n}, T_U]} |{\mathsf E} 
U(T)|}{n}>\eps\right)+{}\\
{}+ \p \left(\sup\limits_{T\in [t_{\kappa_n}, T_U]} |Z_1(T)| > \eps n\right) +{}
\end{multline*}

\noindent
\begin{multline}
{}+ \p  \left(\sup\limits_{j \in [0, n-1]} |Z_2(T_j^{\prime})| > \eps n\right) +{}\\
{}+ \p \left(\sup\limits_{\substack{j \in [0, n-1] \\ T\in 
[T_j^{\prime},T_j^{\prime}+d_n]}} |Z_2(T)-Z_2(T_j^{\prime})| > \eps n\right).
\label{M1SC}
\end{multline}
Применяя рассуждения, аналогичные приведенным в~доказательстве теоремы~1, можно показать, что
$$
\sup\limits_{T\in [t_{\kappa_n}, T_U]} |{\mathsf E} U(T)| = o(n); \enskip
\sup\limits_{T\in [t_{\kappa_n}, T_U]} |Z_1(T)|  = o(n),
$$
откуда следует, что второе и~третье слагаемые в~(\ref{M1SC}) обращаются в~ноль 
для всех достаточно больших~$n$.

Для некоторых положительных констант  $c_7$ и~$c_8$ первое, четвертое и~пятое 
слагаемые  в~(\ref{M1SC}) не превышают $c_7 n^{-1-c_8}$ для всех достаточно 
боль\-ших~$n$, что можно показать с~помощью ограничения на $\alpha(\cdot)$ из 
условия и~рассуждений, аналогичных приведенным при выводе соответственно формул~(\ref{M1next}), (\ref{M5}) и~(\ref{M6}), с~тем отличием, что при применении 
леммы~4 полагается $m \hm= (\ln n)^3$.

Из доказанного следует, что
$$
\sum\limits_{n=1}^{\infty}\p \left( \fr{|U(\hat{t}_F)|}{n}> 4\eps \right) 
< \infty,
$$
и по теореме~1.3.4 из~\cite{Serfling2002} $U(\hat{t}_F) \, n^{-1} \to 0$ п.~в., 
что завершает доказательство теоремы.~\hfill$\square$



{\small\frenchspacing
 {\baselineskip=11.5pt
 %\addcontentsline{toc}{section}{References}
 \begin{thebibliography}{99}
\bibitem{FDRImage}
\Au{Krylov V.\,A., Moser~G., Serpico~S.\,B., Zerubia~J.}
False discovery rate approach to unsupervised image change detection~// IEEE 
T. Image Process., 2016. Vol.~25. No.\,10. P.~4704--4718. doi: 10.1109/TIP.2016.2593340.

\bibitem{MultipleTesting} %2
\Au{Menyhart~O., Weltz~B., Gyorffy~B.}
MultipleTesting.com: A~tool for life science researchers for multiple hypothesis 
testing correction~// PLoS One, 2021. Vol.~16. No.\,6. Art.~0245824. doi: 10.1371/journal.pone.0245824.

\bibitem{AdaptingFDR} %3
\Au{Abramovich~F., Benjamini~Y., Donoho~D., Johnstone~I.}
Adapting to unknown sparsity by controlling the false discovery rate~// Ann. Stat., 2006. Vol.~34. No.\,2. P.~584--653.
doi: 10.1214/009053606000000074.

\bibitem{ZasShe17} %4
\Au{Заспа~А.\,Ю., Шестаков~О.\,В.}
Состоятельность оценки риска при множественной проверке гипотез с~FDR-по\-ро\-гом~// 
Вестник ТвГУ. Сер. Прикладная математика, 2017. Вып.~1. С.~5--16.
doi: 10.26456/vtpmk119. EDN: YFYJXT.

\bibitem{Mathematics2020} %5
\Au{Palionnaya~S.\,I., Shestakov~O.\,V.}
Asymptotic properties of MSE estimate for the false discovery rate controlling 
procedures in multiple hypothesis testing // Mathematics, 2020. Vol.~8. No.~11. 
Art.~1913. 11~p. doi: 10.3390/ math8111913.

\bibitem{Shestakov2021-1} %6
\Au{Шестаков~О.\,В.}
Анализ несмещенной оценки среднеквадратичного риска метода блочной пороговой 
обработки~// Информатика и~её применения, 2021. Т.~15. Вып.~2. С.~30--35.
doi: 10.14357/19922264210205. EDN: DSQQAU.

\bibitem{Shestakov2021-2} %7
\Au{Шестаков~О.\,В.}
Пороговые функции в~методах подавления шума, основанных на вейв\-лет-раз\-ло\-же\-нии 
сигнала~// Информатика и~её применения, 2021. Т.~15. Вып.~3. С.~51--56.
doi: 10.14357/19922264210307. EDN: WSEAYG.

\bibitem{Shestakov2022} %8
\Au{Шестаков~О.\,В.}
Несмещенная оценка риска пороговой обработки с~двумя пороговыми значениями~// 
Информатика и~её применения, 2022. Т.~16. Вып.~4. С.~14--19.
doi: 10.14357/19922264220403. EDN: \mbox{DZBVLC}.

\bibitem{ResultsOnFDRUnderDependence} %9
\Au{Farcomeni~A.}
Some results on the control of the false discovery rate under dependence~// 
Scand. J. Stat., 2007. Vol.~34. No.\,2. P.~275--297.
doi: 10.1111/j.1467-9469.2006.00530.x.

\bibitem{VorontsovShestakov2023} %10
\Au{Воронцов~М.\,О., Шестаков~О.\,В.}
Среднеквадратичный риск FDR-про\-це\-ду\-ры в~условиях слабой за\-ви\-си\-мости~// 
Информатика и~её применения, 2023. Т.~17. Вып.~2. С.~34--40.
doi: 10.14357/19922264230205. EDN: AVJZDX.

\bibitem{Vorontsov2024} %11
\Au{Воронцов~М.\,О.}
Анализ среднеквадратичного риска при использовании методов множественной 
проверки гипотез для выбора параметров пороговой обработки в~условиях слабой 
зависимости~// Вестник Московского университета. Сер. 15: Вычислительная 
математика и~кибернетика, 2024. №\,2. С.~18--24.

\bibitem{Bosq} %12
\Au{Bosq~D.}
Nonparametric statistics for stochastic processes: Estimation and prediction.~--- 
Lecture notes in statistics ser.~--- New York, NY, USA: Springer, 1996. Vol.~110. 
188~p.

\bibitem{Mallat} %13
\Au{Mallat~S.}
A wavelet tour of signal processing.~--- New York, NY, USA: Academic Press, 1999. 
857~p.

\bibitem{spatialAdaptation} %14
\Au{Donoho~D., Johnstone~I.}
Ideal spatial adaptation via wavelet shrinkage~// Biometrika, 1994. Vol.~81. 
No.\,3. P.~425--455. doi: 10.1093/biomet/81.3.425.

\bibitem{AdaptingSURE} %15
\Au{Donoho D., Johnstone I.\,M.}
Adapting to unknown smoothness via wavelet shrinkage~// J.~Amer. Stat. Assoc., 
1995. Vol.~90. P.~1200--1224.

\bibitem{ExactRisk} %16
\Au{Marron J.\,S., Adak~S., Johnstone~I.\,M., Neumann~M.\,H., Patil~P.}
Exact risk analysis of wavelet regression~// J.~Comput. Graph. Stat., 1998. 
Vol.~7. P.~278--309. doi: 10.1080/ 10618600.1998.10474777.

\bibitem{Jansen} %17
\Au{Jansen~M.}
Noise reduction by wavelet thresholding.~-- Lecture notes in statistics ser.~--- 
New York, NY, USA: Springer, 2001. Vol.~161. 217~p.

\bibitem{KuShe2016_1} %18
\Au{Кудрявцев~А.\,А., Шестаков~О.\,В.}
Асимптотическое поведение порога, минимизирующего усредненную\linebreak вероятность ошибки 
вычисления вейв\-лет-ко\-эф\-фи\-ци\-ен\-тов~// Докл. Акад. наук, 2016. Т.~468. №\,5. 
С.~487--491.

\bibitem{KuShe2016_2} %19
\Au{Кудрявцев~А.\,А., Шестаков~О.\,В.}
Асимптотически оптимальная пороговая обработка вейв\-лет-ко\-эф\-фи\-ци\-ен\-тов в~моделях с~негауссовым распределением шума~// Докл. Акад. наук, 2016. Т.~471. №\,1. 
С.~11--15.



\bibitem{Eroshenko} %20
\Au{Ерошенко~А.\,А.}
Статистические свойства оценок сигналов и~изображений при пороговой обработке 
коэффициентов в~вейв\-лет-раз\-ло\-же\-ни\-ях: Дис.\ \ldots\ канд. физ.-мат. наук.~--- 
М.: МГУ, 2015. 82~с.

\bibitem{Peligrad} %21
\Au{Peligrad~M.}
On the asymptotic normality of sequences of weak dependent random variables~// 
J. Theor. Probab., 1996. Vol.~9. No.\,3. P.~703--715. doi: 10.1007/BF02214083.

\bibitem{Serfling2002} %22
\Au{Serfling~R.\,J.}
Approximation theorems of mathematical statistics.~--- New York, NY, USA: John Wiley \&~Sons, Inc., 2002. 371~p.

\end{thebibliography}

 }
 }

\end{multicols}

\vspace*{-6pt}

\hfill{\small\textit{Поступила в~редакцию 21.05.24}}

\vspace*{8pt}

%\pagebreak

%\newpage

%\vspace*{-28pt}

\hrule

\vspace*{2pt}

\hrule



\def\tit{ASYMPTOTIC NORMALITY AND STRONG CONSISTENCY\\ OF~RISK ESTIMATE WHEN USING THE~FDR THRESHOLD\\ UNDER WEAK DEPENDENCE CONDITION}


\def\titkol{Asymptotic normality and strong consistency of~risk estimate when using the~FDR threshold under weak dependence condition}


\def\aut{M.\,O.~Vorontsov$^{1,2}$ and~O.\,V.~Shestakov$^{1,2,3}$}

\def\autkol{M.\,O.~Vorontsov and~O.\,V.~Shestakov}

\titel{\tit}{\aut}{\autkol}{\titkol}

\vspace*{-13pt}


\noindent
$^{1}$Department of Mathematical Statistics, Faculty of Computational Mathematics and Cybernetics,
 M.\,V.~Lo\-mo-\linebreak
 $\hphantom{^1}$nosov Moscow State University, 1-52~Leninskie Gory, GSP-1, Moscow 119991, Russian Federation

\noindent
$^{2}$Moscow Center for Fundamental and Applied Mathematics, M.\,V.~Lomonosov Moscow State University,\linebreak
$\hphantom{^1}$1~Leninskie Gory, GSP-1, Moscow 119991, Russian Federation

\noindent
$^{3}$Federal Research Center ``Computer Science and Control'' of the Russian Academy of Sciences, 44-2~Vavilov\linebreak
$\hphantom{^1}$Str., Moscow 119333, Russian Federation


\def\leftfootline{\small{\textbf{\thepage}
\hfill INFORMATIKA I EE PRIMENENIYA~--- INFORMATICS AND
APPLICATIONS\ \ \ 2024\ \ \ volume~18\ \ \ issue\ 3}
}%
 \def\rightfootline{\small{INFORMATIKA I EE PRIMENENIYA~---
INFORMATICS AND APPLICATIONS\ \ \ 2024\ \ \ volume~18\ \ \ issue\ 3
\hfill \textbf{\thepage}}}

\vspace*{2pt}






\Abste{An approach to solving the problem of noise removal in a large array of sparse data is considered
 based on the method of controlling the average proportion of false hypothesis rejections (False Discovery Rate, FDR). 
 This approach is equivalent to threshold processing procedures that remove array components whose values do not exceed 
 some specified threshold. The observations in the model are considered weakly dependent. To control the\linebreak\vspace*{-12pt}}
 
 \Abstend{degree of dependence, 
 restrictions on the strong mixing coefficient and the maximum correlation coefficient are used. The mean-square risk is 
 used as a measure of the effectiveness of the considered approach. It is possible to calculate the risk value only on the test data;
  therefore, its statistical estimate is considered in the work and its properties are investigated. The asymptotic normality and
   strong consistency of the risk estimate are proved when using the FDR threshold under conditions of weak dependence in the data.}

\KWE{thresholding; multiple hypothesis testing; risk estimate}

\DOI{10.14357/19922264240309}{ZOQVTO}

%\vspace*{-12pt}


    
   %   \Ack

%\vspace*{-3pt}
%\noindent



  \begin{multicols}{2}

\renewcommand{\bibname}{\protect\rmfamily References}
%\renewcommand{\bibname}{\large\protect\rm References}

{\small\frenchspacing
 {\baselineskip=10.8pt
 \addcontentsline{toc}{section}{References}
 \begin{thebibliography}{99} 

%1
\bibitem{FDRImage-1}
\Aue{Krylov, V.\,A., G.~Moser, S.\,B.~Serpico, and J.~Zerubia.} 2016. 
False discovery rate approach to unsupervised image change detection. 
\textit{IEEE T. Image Process.} 25(10):4704--4718. doi: 10.1109/TIP.2016.2593340.

%2
\bibitem{MultipleTesting-1}
\Aue{Menyhart, O., B.~Weltz, and B.~Gyorffy.} 2021. 
MultipleTesting.com: A~tool for life science researchers for multiple hypothesis testing correction. 
\textit{PLoS One} 16(6):0245824. 
doi: 10.1371/journal.pone.0245824.

%3
\bibitem{AdaptingFDR-1}
\Aue{Abramovich, F., Y.~Benjamini, D.~Donoho, and I.\,M.~Johnstone.} 2006. 
Adapting to unknown sparsity by controlling the false discovery rate. 
\textit{Ann. Stat.} 34(2):584--653. 
doi: 10.1214/009053606000000074.


%4
\bibitem{ZasShe17-1}
\Aue{Zaspa, A.\,Yu., and O.\,V.~Shestakov.} 2017.
Sostoyatel'nost' otsenki riska pri mnozhestvennoy proverke gipotez s~FDR-porogom
 [Consistency of the risk estimate of the multiple hypothesis testing with the FDR threshold]. 
\textit{Vestnik TvGU. Ser.: Prikladnaya matematika} [Herald of Tver State University. Ser. Applied Mathematics] 1:5--16.
doi: 10.26456/vtpmk119. EDN: YFYJXT.

%5
\bibitem{Mathematics2020-1}
\Aue{Palionnaya, S.\,I., and O.\,V.~Shestakov.} 2020. 
Asymptotic properties of MSE estimate for the false discovery rate controlling procedures in multiple hypothesis testing. 
\textit{Mathematics} 8(11):1913. 11~p.
doi: 10.3390/math8111913.

%6
\bibitem{Shestakov2021-1-1}
\Aue{Shestakov, O.\,V.} 2021.
Analiz nesmeshchennoy otsenki srednekvadratichnogo riska metoda blochnoy po\-ro\-go\-voy obrabotki 
[Analysis of the unbiased mean-square risk estimate of the block thresholding method]. 
\textit{Informatika i~ee Primeneniya~--- Inform. Appl.} 15(2):30--35.
doi: 10.14357/19922264210205. EDN: DSQQAU.

%7
\bibitem{Shestakov2021-2-1}
\Aue{Shestakov, O.\,V.} 2021.
Porogovye funktsii v~metodakh podavleniya shuma, osnovannykh na veyvlet-razlozhenii signala 
[Thresholding functions in the noise suppression methods based on the wavelet expansion of the signal]. 
\textit{Informatika i~ee Primeneniya~--- Inform. Appl.} 15(3):51--56.
doi: 10.14357/19922264210307. EDN: WSEAYG.

%8
\bibitem{Shestakov2022-1}
\Aue{Shestakov, O.\,V.} 2022.
Nesmeshchennaya otsenka riska porogovoy obrabotki s dvumya porogovymi znacheniyami [Unbiased thresholding risk estimate with two threshold values]. 
\textit{Informatika i~ee Primeneniya~--- Inform. Appl.} 16(4):14--19.
doi: 10.14357/19922264220403. EDN: DZBVLC.

%9
\bibitem{ResultsOnFDRUnderDependence-1}
\Aue{Farcomeni, A.} 2007. Some results on the control of the false discovery rate under dependence. 
\textit{Scand. J. Stat.} 34(2):275--297. 
doi: 10.1111/j.1467-9469.2006.00530.x.

%10
\bibitem{VorontsovShestakov2023-1}
\Aue{Vorontsov, M.\,O., and O.\,V.~Shestakov.} 2023.
Sred\-ne\-kvad\-ra\-tich\-nyy risk FDR-protsedury v~usloviyakh slaboy za\-vi\-si\-mosti [Mean-square risk of the FDR procedure under weak dependence]. 
\textit{Informatika i~ee Primeneniya~--- Inform. Appl.} 17(2):34--40.
doi: 10.14357/19922264230205. EDN: AVJZDX.

%11
\bibitem{Vorontsov2024-1}
\Aue{Vorontsov, M.\,O.} 2024. 
RMS risk analysis when using multiple hypothesis testing select parameters of thresholding under conditions of weak dependence. 
\textit{Moscow University Computational Mathematics Cybernetics} 48:91--97. 
doi: 10.3103/S027864192470002X.

%12
\bibitem{Bosq-1}
\Aue{Bosq, D.} 1996. 
\textit{Nonparametric statistics for stochastic processes: Estimation and prediction}. 
Lecture notes in statistics ser. New York, NY: Springer Verlag. Vol.~110. 188~p.

%13
\bibitem{Mallat-1}
\Aue{Mallat, S.} 1999. 
\textit{A wavelet tour of signal processing}. New York, NY: Academic Press. 857~p.

%14
\bibitem{spatialAdaptation-1}
\Aue{Donoho, D., and I.\,M.~Johnstone.} 1994. 
Ideal spatial adaptation via wavelet shrinkage. 
\textit{Biometrika} 81(3):425--455. doi: 10.1093/biomet/81.3.425.

%15
\bibitem{AdaptingSURE-1}
\Aue{Donoho, D., and I.\,M.~Johnstone.} 1995. 
Adapting to unknown smoothness via wavelet shrinkage. 
\textit{J. Am. Stat. Assoc.} 90(432):1200--1224. doi: 10.1080/01621459. 1995.10476626.

%16
\bibitem{ExactRisk-1}
\Aue{Marron, J.\,S., S.~Adak, I.\,M.~Johnstone, M.\,H.~Neumann, and P.~Patil.} 1998. 
Exact risk analysis of wavelet regression. 
\textit{J.~Comput. Graph. Stat.} 7(3):278-309. doi: 10.1080/ 10618600.1998.10474777.

%17
\bibitem{Jansen-1}
\Aue{Jansen, M.} 2001. 
\textit{Noise reduction by wavelet thresholding}. Lecture notes in statistics ser. New York, NY: Springer Verlag. Vol.~161. 217~p.

%18
\bibitem{KuShe2016_1-1}
\Aue{Kudryavtsev, A.\,A., and O.\,V.~Shestakov.} 2016. 
Asymptotic behavior of the threshold minimizing the average probability of error in calculation of wavelet coefficients. 
\textit{Dokl. Math.} 93(3):295--299.
doi: 10.1134/S1064562416030212. EDN: WUMUEV. 

%19
\bibitem{KuShe2016_2-1}
\Aue{Kudryavtsev, A.\,A., and O.\,V.~Shestakov.} 2016. 
Asymptotically optimal wavelet thresholding in the models with non-Gaussian noise distributions. 
\textit{Dokl. Math.} 94(3):615--619.
doi: 10.1134/S1064562416060028. EDN: YUYVUP.




%20
\bibitem{Eroshenko-1}
\Aue{Eroshenko, A.\,A.} 2015. Statisticheskie svoystva otsenok signalov i~izobrazheniy pri porogovoy obrabotke ko\-ef\-fi\-tsi\-en\-tov 
v~veyvlet-razlozheniyakh 
[Statistical properties of signal and image estimates under thresholding of coefficients in wavelet decompositions]. Moscow: MSU. PhD Diss. 82~p.

%21
\bibitem{Peligrad-1}
\Aue{Peligrad, M.} 1996. 
On the asymptotic normality of sequences of weak dependent random variables. 
\textit{J. Theor. Probab.} 9(3):703--715. doi: 10.1007/BF02214083.

%22
\bibitem{Serfling2002-1}
\Aue{Serfling, R.\,J.} 2002. 
\textit{Approximation theorems of mathematical statistics}. New York, NY: John Wiley \&~Sons. 371~p.
\end{thebibliography}

 }
 }

\end{multicols}

\vspace*{-6pt}

\hfill{\small\textit{Received May 21, 2024}} 

%\vspace*{-18pt}

\Contr

\vspace*{-3pt}


\noindent
\textbf{Vorontsov Mikhail O.} (b.\ 1996)~--- PhD student, Department of Mathematical Statistics, 
Faculty of Computational Mathematics and Cybernetics, M.\,V.~Lomonosov Moscow State University, 1-52~Leninskie Gory, GSP-1, Moscow 119991, Russian Federation;  
mathematician, Moscow Center for Fundamental and Applied Mathematics, M.\,V.~Lomonosov Moscow State University, 1~Leninskie Gory, GSP-1, Moscow 119991, Russian Federation;
\mbox{m.vtsov@mail.ru}

\vspace*{6pt}

\noindent
\textbf{Shestakov Oleg V.} (b.\ 1976)~--- Doctor of Science in physics and mathematics, professor, Department of Mathematical Statistics,
 Faculty of Computational Mathematics and Cybernetics, M.\,V.~Lomonosov Moscow State University, 1-52~Leninskie Gory, GSP-1, Moscow 119991, Russian Federation; 
 senior scientist, Federal Research Center ``Computer Science and Control'' of the Russian Academy of Sciences, 44-2~Vavilov Str., Moscow 119333, 
 Russian Federation; leading scientist, Moscow Center for Fundamental and Applied Mathematics, M.\,V.~Lomonosov Moscow State University, 
 1~Leninskie Gory, GSP-1, Moscow 119991, Russian Federation; \mbox{oshestakov@cs.msu.su}


\label{end\stat}

\renewcommand{\bibname}{\protect\rm Литература}   %1
\def\stat{gorshenin}

\def\tit{ЗАШУМЛЕНИЕ ДАННЫХ КОНЕЧНЫМИ СМЕСЯМИ НОРМАЛЬНЫХ 
И~ГАММА-РАСПРЕДЕЛЕНИЙ\\ С~ПРИМЕНЕНИЕМ К~ЗАДАЧЕ ОКРУГЛЕНИЯ НАБЛЮДЕНИЙ$^*$}

\def\titkol{Зашумление данных конечными смесями нормальных 
и~гамма-распределений с~применением к~задаче округления} % наблюдений}

\def\aut{А.\,К.~Горшенин$^1$}

\def\autkol{А.\,К.~Горшенин}

\titel{\tit}{\aut}{\autkol}{\titkol}

\index{Горшенин А.\,К.}
\index{Gorshenin A.\,K.}


{\renewcommand{\thefootnote}{\fnsymbol{footnote}} \footnotetext[1]
{Работа выполнена при поддержке РНФ (проект 18-71-00156).}}


\renewcommand{\thefootnote}{\arabic{footnote}}
\footnotetext[1]{Институт проблем информатики Федерального исследовательского центра 
<<Информатика и~управление>> Российской академии наук, \mbox{agorshenin@frccsc.ru}}

\vspace*{-12pt}




\Abst{Во многих реальных задачах проводится статистический анализ данных, 
содержащих дополнительные ошибки измерения, в~том числе в~виде округления, 
что в~ряде ситуаций может приводить к~достаточно существенным искажениям. 
В~настоящей статье для одной из возможных моделей округления получены оценки 
для неизвестного математического ожидания наблюдений в~предположении, что 
исходные данные дополнительно зашумлены с~по\-мощью случайных величин, 
име\-ющих распределения типа конечных смесей нормальных и~гам\-ма-за\-ко\-нов. 
Построены доверительные интервалы для неизвестного математического ожидания 
с~использованием уточненной оценки для дисперсии целой части случайной величины. 
Обсуждается алгоритм определения значения параметра для искусственного шума, 
добавление которого к~исходным данным способствует повышению качества работы 
метода скользящего разделения смесей.}

\KW{зашумленные данные; округленные наблюдения; конечные смеси нормальных 
распределений; конечные смеси гам\-ма-рас\-пре\-де\-ле\-ний; доверительные интервалы;  
метод скользящего разделения смесей}

\DOI{10.14357/19922264180304}
  
\vspace*{-4pt}


\vskip 10pt plus 9pt minus 6pt

\thispagestyle{headings}

\begin{multicols}{2}

\label{st\stat}


\section{Введение}

Во многих реальных задачах данные, являющиеся непрерывными по своей сути, 
регистрируются с~помощью инструментов, вносящих дополнительные ошибки 
измерения, в~том чис\-ле в~виде округления. Таким образом, статистический 
анализ проводится не для исходных, а для преобразованных некоторым 
случайным образом наблюдений, что в~ряде ситуаций может приводить к~достаточно
 существенным искажениям.

Для преодоления данной проблемы развивались различные подходы, в~том числе 
на основе смешанных моделей (см., например, статью~\cite{Wright2003}, в~которой 
различные компоненты  используются для пред\-став\-ле\-ния уровней округления). 
В~работе~\cite{Bai2009} приводятся результаты для моделей авторегрессии и~скользящего 
среднего для округленных данных, а~в~статье~\cite{Zhang2010} эти результаты 
развиваются и~исследуются их асимптотические свойства. 
В~статье~\cite{Zhao2012} исследован метод оценивания па\-ра\-мет\-ров конечных смесей 
вероятностных распределений (в~том чис\-ле, и~многомерных) 
на основе использования EM (expectation-maximization) 
алгоритма~\cite{Korolev2011-i} с~\mbox{целью} получения состоятельных 
и~асимптотически нормальных оценок.

В настоящей статье развиваются результаты для моделей округления, 
описанных в~работах~\cite{Ushakov2015,Ushakov2017a,Ushakov2017b}. 
В~их рамках будут получены оценки для неизве\-ст\-ного математического ожидания 
наблюдений в~предположении, что исходные данные зашумлены с~по\-мощью случайных 
величин, имеющих распределения типа конечных смесей нормальных и~гам\-ма-за\-ко\-нов. 
Это позволяет учесть большее количество случайных факторов, влия\-ющих на величину 
<<дополнительной>> ошибки. Также будут построены доверительные интервалы для 
неизвестного математического ожидания. Выражения для гам\-ма-рас\-пре\-де\-ле\-ний 
получены впервые. Также обсуждается алгоритм определения значения па\-ра\-мет\-ра для 
искусственного шума, добавление которого к~исходным данным способствует 
повышению качества работы метода скользящего разделения смесей~\cite{Gorshenin2016}.

\vspace*{-12pt}

\section{Предположения и~базовые отношения}

Для сокращения формулировок теорем в~сле\-ду\-ющих разделах сделаем ряд 
предположений, на которые будем ссылаться в~дальнейшем. Итак, пусть:
\begin{itemize}
\item[(A)] $X_1,X_2,\ldots$~--- независимые одинаково распределенные 
случайные величины с~неизвестным математическим ожиданием ${\sf E}_X\hm<+\infty$;
\item[(B)] $\varepsilon_1,\varepsilon_2,\ldots$~--- независимые одинаково 
распределенные случайные величины с~математическим ожиданием 
${\sf E}_\varepsilon\hm<+\infty$; %\label{B}
\item[(C)] $X_1,X_2,\ldots$ и~$\varepsilon_1,\varepsilon_2,\ldots$ 
являются независимыми;
\item[(D)] $Y_j=\left[X_j+\varepsilon_j+1/2\right]$ для всех $j\hm=1,2,\ldots$ 
представляют собой округление значения суммы случайных величин $X_j\hm+\varepsilon_j$ 
до ближайшего целого сверху (при этом запись~$[\cdot]$ соответствует целой 
части выражения).
\end{itemize}

В рамках данных предположений в~статье будут рассмотрены вопросы качества 
приближения неизвестного математического ожидания~${\sf E}_X$ для исходных данных 
в~ситуации, когда наблюдения для анализа получены с~аддитивной ошибкой c известными 
распределениями (см.\ предположение~(B)) и~дополнительно округляются до 
ближайшего целого (см.\ предположение~(D)).

Заметим, что в~силу усиленного закона больших чисел справедливы следующие выражения:
\begin{multline}
\fr{1}{n}\sum\limits_{j=1}^n Y_j\xrightarrow[n\to\infty]{\text{п.н.}}
{\sf E}_Y\equiv\mathbb{E}\left[X_1+\varepsilon_1+\fr{1}{2}\right]={}\\
{}=\mathbb{E}\left(X_j+\varepsilon_j+\fr{1}{2}\right)-\mathbb{E}
\left\{X_j+\varepsilon_j+\fr{1}{2}\right\}={}\\
{}={\sf E}_X+{\sf E}_\varepsilon+\fr{1}{2}-\mathbb{E}\left\{X_j+\varepsilon_j+\fr{1}{2}\right\}. 
\label{Law}
\end{multline}

Запись $\{\cdot\}$ в~формуле~\eqref{Law} соответствует дробной 
части выражения, а~п.н.\ обозначает сходимость в~смысле почти наверное.

Для доказательства результатов в~дальнейшем потребуется следующее 
представления для дробной части  абсолютно непрерывной случайной величины~$Z$ 
с~абсолютно  интегрируемой характеристической функцией~$\varphi_Z(t)$
 (см., например, Лемму~4 в~работе~\cite{Ushakov2017b}):
\begin{equation}
\label{Fract}
\mathbb{E}\{Z\}=\fr{1}{2}-\sum\limits_{n=1}^\infty 
\fr{\mathrm{Im}\left (\varphi_Z(2\pi n)\right)}{\pi n}\,.
\end{equation}

Через $\mathrm{Im}\,(\cdot)$ в~формуле~\eqref{Fract} обозначена мнимая часть 
соответствующей функции.

При построении доверительных интервалов в~дальнейшем будет 
использована следующая оценка, справедливая для любой случайной величины~$Z$:
\begin{equation}
\mathbb{D}[Z]\leqslant \left(\sqrt{\mathbb{D} Z}+\fr{1}{2}\right)^2.
\label{Var}
\end{equation}
Она может быть проверена непосредственно с~учетом представления 
$\mathbb{D} [Z]\hm=\mathbb{D}\left(Z\hm-\{Z\}\right)$, неравенства 
Ко\-ши--Бу\-ня\-ков\-ско\-го для ковариации и~соотношения 
 $\mathbb{D}\{Z\}\hm\leqslant 1/4$, справедливого для любой случайной величины~$Z$ 
 (см., например, статью~\cite{Ushakov2017b}). Отметим, что данная оценка 
 является более точной по сравнению с~использованным для аналогичных 
 целей в~работе~\cite{Ushakov2017b} соотношением 
 $\mathbb{D} [Z]\hm\leqslant 2\mathbb{D} Z\hm+1/2$. Действительно,
\begin{equation*}
2\mathbb{D} Z+\fr{1}{2}-\left(\sqrt{\mathbb{D} Z}+\fr{1}{2}\right)^2=
\left(\sqrt{\mathbb{D} Z}-\fr{1}{2}\right)^2\geqslant0\,,
\end{equation*}
причем для всех $\sqrt{\mathbb{D} Z}\hm\neq {1}/{2}$ 
данное неравенство является строгим.

\section{Конечные смеси нормальных законов}

Для случайной величины~$X$, имеющей распределение типа 
конечной смеси нормальных законов~\cite{Korolev2011-i} с~параметрами 
${\bf a}\hm=(a_1,\ldots, a_k)$, $a_j\hm\in \mathbb{R}$, 
$\boldsymbol{\sigma}\hm=(\sigma_1,\ldots, \sigma_k)$, 
$\sigma_j\hm>0$, ${\bf p}\hm=(p_1,\ldots, p_k)$, $p_j\hm\geqslant 0$, 
$\sum\nolimits_{j=1}^{k}p_j\hm=1$, плот\-ность которого задается выражением
\begin{equation}
f_X(x)=\sum\limits_{j=1}^{k}\fr{p_j}{\sigma_j\sqrt{2\pi}}\,e^{-(x-a_j)^2/(2\sigma_j^2)}\,,
\label{FinNormMixt}
\end{equation}
характеристическая функция имеет вид:
\begin{equation}
\varphi_X(t)=\int\limits_{-\infty}^{+\infty}\!\!e^{itx} f_X(x)\, dx = 
\sum\limits_{j=1}^{k}p_j e^{ita_j-\sigma_j^2 t^2/2}.
\label{ChiFinNormMixt}
\end{equation}

Абсолютная интегрируемость  $\varphi_X(t)$ вытекает из свойств 
характеристической функции нормального распределения. 
Заметим, что в~точке $t\hm=2\pi n$ выражение~\eqref{ChiFinNormMixt} принимает 
сле\-ду\-ющий вид:
\begin{equation}
\label{ChiFinNormMixt2npi}
\varphi_X(2\pi n)= \sum\limits_{j=1}^{k}p_j e^{-2\pi^2 \sigma_j^2 n^2}\,.
\end{equation}

Рассмотрим вопрос точности оценивания неизвестного математического ожидания~${\sf E}_X$ 
при до\-бав\-ле\-нии зашумления.

\smallskip

\noindent
\textbf{Теорема~1.}\ 
\textit{Пусть выполнены предположения}~(A)--(D), 
\textit{причем случайные величины~$\varepsilon_j$, $j\hm=1,2,\ldots$, 
имеют распределение типа конечной $k$-ком\-по\-нент\-ной смеси нормальных законов 
вида}~\eqref{FinNormMixt} \textit{с~па\-ра\-мет\-ра\-ми~${\bf a}$, $\boldsymbol{\sigma}$ 
и~${\bf p}$. Тогда}
\begin{equation}
\label{Th1Eq}
\left\lvert {\sf E}_Y-{\sf E}_X\right\rvert \leqslant 
A+\fr{1}{\pi}\left(1+\fr{1}{4\pi^2\sigma^2}\right)e^{-2\pi^2\sigma^2}\,, 
\end{equation}
\textit{где} $A=\max(|a_1|,\ldots,|a_k|)$, $\sigma\hm=\min(\sigma_1,\ldots,\sigma_k)$.

\smallskip


\noindent
Д\,о\,к\,а\,з\,а\,т\,е\,л\,ь\,с\,т\,в\,о\,.\ \
С~учетом пред\-став\-ле\-ний~\eqref{Law},~\eqref{Fract} и~\eqref{ChiFinNormMixt2npi}, 
ограниченности модуля характеристической функции, а~также не\-за\-ви\-си\-мости 
случайных величин~$X_j$ и~$\varepsilon_j$ имеем:
\begin{multline*}
\left\lvert {\sf E}_Y-{\sf E}_X\right\rvert =
\left\lvert {\sf E}_\varepsilon+\fr{1}{2}-\mathbb{E}\left\{X_j+
\varepsilon_j+\fr{1}{2}\right\}\right\rvert={}\\
{}=\left\lvert {\sf E}_\varepsilon+\sum\limits_{n=1}^\infty
\fr{\mathrm{Im} \left(\varphi_{X_j}(2\pi n)\varphi_{\varepsilon_j}(2\pi n)
\varphi_{1/2}(2\pi n)\right)}{\pi n}\right\rvert={}\\
=\left\lvert 
\vphantom{\fr{(-1)^n\sum\nolimits_{j=1}^{k}p_j e^{-2\pi^2 \sigma_j^2 n^2} 
\mathrm{Im} \left(\varphi_{X_j}(2\pi n)\right)}{\pi n}}
{\sf E}_\varepsilon+{}\right.\\
\left.{}+\sum\limits_{n=1}^\infty
\fr{\mathrm{Im} \left(\varphi_{X_j}(2\pi n) 
\sum\nolimits_{j=1}^{k}p_j e^{-2\pi^2 \sigma_j^2 n^2} 
e^{\pi n}\right)}{\pi n}\right\rvert={}\\
{}=\left\lvert 
\vphantom{\fr{(-1)^n\sum\nolimits_{j=1}^{k}p_j e^{-2\pi^2 \sigma_j^2 n^2} 
\mathrm{Im} \left(\varphi_{X_j}(2\pi n)\right)}{\pi n}}
{\sf E}_\varepsilon+{}\right.\\
\left.{}+\sum\limits_{n=1}^\infty
\fr{(-1)^n\sum\nolimits_{j=1}^{k}p_j e^{-2\pi^2 \sigma_j^2 n^2} 
\mathrm{Im} \left(\varphi_{X_j}(2\pi n)\right)}{\pi n}\right\rvert\leqslant{}\\
{}\leqslant \left\lvert {\sf E}_\varepsilon\right\rvert+\left\lvert\
\sum\limits_{j=1}^{k}p_j\sum\limits_{n=1}^\infty 
\fr{1}{\pi n} e^{-2\pi^2 \sigma_j^2 n^2}\right\rvert\leqslant {}\\
\\
{}\leqslant
\max\left(|a_1|,\ldots,|a_k|\right)+{}\\
{}+\sum\limits_{j=1}^{k} 
\fr{p_j}{\pi} \left(\!1+\fr{1}{4\pi^2\sigma_j^2}\!\right)\!e^{-2\pi^2\sigma_j^2}\leqslant{}\\
{}\leqslant
A+\fr{1}\pi\left(1+\fr{1}{4\pi^2\sigma^2}\right)e^{-2\pi^2\sigma^2}\,.
\end{multline*}

Справедливость использованной оценки 
\begin{equation*}
\sum\limits_{n=1}^\infty
\fr{e^{-2\pi^2 \sigma_j^2 n^2}}{n}\leqslant 
\left(1+\fr{1}{4\pi^2\sigma_j^2}\right)e^{-2\pi^2\sigma_j^2}
\end{equation*}
может быть проверена непосредственно (например, см.\ доказательство Теоремы~6 
в~статье~\cite{Ushakov2017b}).~\hfill$\square$

\smallskip

\noindent
\textbf{Замечание~1.}
В~случае, если зашумление производится нормально распределенными случайными 
величинами c нулевыми средними (т.\,е.\ в~формуле~\eqref{Th1Eq} необходимо считать 
$A\hm=0$, $k\hm=1$), то Тео\-ре\-ма~1 совпадает с~результатом, 
полученным в~работе~\cite{Ushakov2017b}.


\smallskip

Рассмотрим вопросы построения доверительного интервала для неизвестного 
математического ожидания~${\sf E}_X$ в~предположении, что случайные величины~$X_j$ не 
содержат ошибок измерения, а~все погрешности учтены исключительно в~за\-шум\-ля\-ющих 
элементах~$\varepsilon_j$.

\smallskip

\noindent
\textbf{Теорема~2.}\ 
\textit{Пусть выполнены предположения}~(A)--(D), 
\textit{причем случайные величины~$\varepsilon_j$, $j\hm=1,2,\ldots$, имеют 
распределение типа конечной $k$-ком\-по\-нент\-ной смеси нормальных законов 
вида}~\eqref{FinNormMixt} \textit{с~параметрами~${\bf a}$, $\boldsymbol{\sigma}$ 
и~${\bf p}$, а~случайные величины} $X_j\stackrel{\text{п.н.}}{=}{\sf E}_X$, $j\hm=1,2,\ldots$ 
\textit{Тогда доверительный интервал для~${\sf E}_X$ при условии $0\hm<\alpha\hm<1$ имеет вид}:
\begin{equation} 
\label{Th2Eq}
\hat{{\sf E}}_X - f({\bf a},\boldsymbol{\sigma},\alpha,n) 
\leqslant {\sf E}_X \leqslant  \hat{{\sf E}}_X + f({\bf a},\boldsymbol{\sigma},\alpha,n),
\end{equation}
\textit{где}

\vspace*{-2pt}

\noindent
\begin{align}
\hat{{\sf E}}_X&=\fr{1}{n} \sum\limits_{j=1}^{n} Y_j\,; \label{Th2hatE}\\
f({\bf a},\boldsymbol{\sigma},\alpha,n)&=
\fr{z_{1-{\alpha}/2}}{\sqrt{n}} \left(\sqrt{A^2+\Sigma^2}+\fr{1}{2}\right) +{}\notag\\
&{}+A+\fr{1}\pi\left(1+\fr{1}{4\pi^2\sigma^2}\right)e^{-2\pi^2\sigma^2}\,;
  \label{Th2f}
\end{align}
\textit{$z_{1-{\alpha}/2}$~--- $\left(1-{\alpha}/2\right)$-кван\-тиль 
стандартного нормального распределения; $A\hm=\max(|a_1|,\ldots,|a_k|)$; 
$\Sigma\hm=\max(\sigma_1,\ldots,\sigma_k)$; $\sigma\hm=\min(\sigma_1,\ldots,\sigma_k)$}. 


\smallskip

\noindent
\noindent
Д\,о\,к\,а\,з\,а\,т\,е\,л\,ь\,с\,т\,в\,о\,.\ \
Из центральной предельной тео\-ре\-мы с~учетом условия~(A) следует, 
что величина~$\hat{{\sf E}}_X$~\eqref{Th2hatE} асимптотически нормальна с~математическим 
ожиданием 
\begin{equation}
{\sf E}_Y\equiv \mathbb{E}\left[{\sf E}_X+\varepsilon_1+\fr{1}{2}\right] \label{EY}
\end{equation}
и дисперсией
\begin{equation}
\fr{1}{n} {\sf D}_Y\equiv \fr{1}{n}\mathbb{D}\left[{\sf E}_X+\varepsilon_1+
\fr{1}{2}\right]. \label{DY}
\end{equation}

Воспользовавшись оценкой~\eqref{Var}, получим:

\vspace*{-2pt}

\noindent
\begin{multline*}
{\sf D}_Y \leqslant  \left(\sqrt{\mathbb{D} \left({\sf E}_X+\varepsilon_1+\fr{1}{2}\right)}+
\fr{1}{2}\right)^2={}\\
{}=
\left(\sqrt{\mathbb{D}\varepsilon_1}+\fr{1}{2}\right)^2= {}\\
{}= \left(\sqrt{\sum\limits_{j=1}^{k}p_j\left(\left(a_j-\sum\limits_{t=1}^{k}
p_t a_t\right)^2+\sigma_j^2\right)}+\fr{1}{2}\right)^2\leqslant {}\\ 
{}\leqslant \left(\sqrt{A^2+\Sigma^2}+\fr{1}{2}\right)^2\,.
\end{multline*}
Тогда доверительный интервал уровня $1\hm-\alpha$ для математического ожидания~${\sf E}_Y$ 
имеет вид:
\begin{equation*}
\mathbb{P}\left(\left\lvert \hat{{\sf E}}_X-{\sf E}_Y\right\rvert \leqslant 
\fr{z_{1-{\alpha}/2}}{\sqrt{n}} 
\left(\sqrt{A^2+\Sigma^2}+\fr{1}{2}\right)\right)\geqslant 1-\alpha\,.
\end{equation*}

\begin{table*}[b]\small
\begin{center}

\begin{tabular}{|c|c|c|c|c|c|c|c|}
\multicolumn{7}{p{100mm}}{Численные решения уравнений~\eqref{f1} и~\eqref{f2} относительно 
параметра~$\sigma$ для некоторых значений~$n$ и~$\alpha$}\\
\multicolumn{7}{c}{\ }\\[-6pt]
\hline
\multicolumn{1}{|c|}{Размер}  & \multicolumn{2}{c|}{Уровень $\alpha=0{,}1$}& 
\multicolumn{2}{c|}{Уровень $\alpha=0{,}05$}& 
\multicolumn{2}{c|}{Уровень $\alpha=0{,}01$}\\
\cline{2-7}
\multicolumn{1}{|c|}{выборки $n$}&$\sigma_1$&$\sigma_2$&$\sigma_1$&$\sigma_2$&$\sigma_1$&$\sigma_2$\\
\hline
$\hphantom{000}100$&$0{,}4302$&$0{,}435$&$0{,}419$&$0{,}425$&$0{,}4002$&$0{,}408$\\
%\hline
$\hphantom{000}200$&$0{,}452$&$0{,}455$ &$0{,}441$&$0{,}445$&$0{,}424$&$0{,}429$\\
%\hline
$\hphantom{00}1000$&$0{,}499$&$0{,}499$ &$0{,}489$&$0{,}489$&$0{,}473$&$0{,}475$\\
%\hline
$\hphantom{0}10000$&$0{,}558$&$0{,}556$ &$0{,}549$&$0{,}547$&$0{,}536$&$0{,}534$\\
%\hline
$100000$&$0{,}611$&$0{,}607$ &$0{,}603$&$0{,}599$&$0{,}591$&$0{,}588$\\
\hline
\end{tabular}
\end{center}
\end{table*}


\noindent
Откуда следует справедливость соотношения~\eqref{Th2Eq} c~уче\-том 
очевидного неравенства

\pagebreak

\noindent
\begin{equation*}
\left\lvert \hat{{\sf E}}_X-{\sf E}_X\right\rvert \leqslant 
\left\lvert \hat{{\sf E}}_X-{\sf E}_Y\right\rvert +\left\lvert {\sf E}_Y-{\sf E}_X\right\rvert 
\end{equation*}
и оценки~\eqref{Th1Eq} из Теоремы~1.~\hfill$\square$

\smallskip

\noindent
\textbf{Замечание~2.}
В~работе~\cite{Gorshenin2016} было продемонстрировано повышение точ\-ности 
работы метода скользящего разделения конечных нормальных смесей за счет 
введения дополнительной компоненты, имеющей нормальное 
распределение $\mathcal{N}(0,\sigma^2)$ с~математическим ожиданием, равным~$0$, 
и~стандартным отклонением~$\sigma$. При этом была отмечена сложность выбора 
параметра~$\sigma$ для сохранения структуры выборки, близкой к~исходной. 
Результат Теоремы~2 может быть использован с~данной целью, если положить $k\hm=1$, 
$a_j\hm=0$ для всех $j\hm=1,2,\ldots$ и~выбирать величину~$\sigma$ как 
минимизирующую длину доверительного интервала~\eqref{Th2Eq}. Для 
этого необходимо найти производную функции $f(0,\sigma,\alpha,n)$~\eqref{Th2f} 
и~численно решить уравнение
\begin{multline}
f_\sigma'(0,\sigma,\alpha,n)\equiv \fr{z_{1-{\alpha}/2}}{\sqrt{n}} - {}\\
{}-
e^{-2\pi^2\sigma^2}\left(4\pi\sigma+\fr{1}{2\pi^3\sigma^3}+
\fr{1}{\pi\sigma}\right)=0
\label{f1}
\end{multline}
относительно неизвестного параметра~$\sigma$ при выбранных значениях величин~$n$ 
и~$\alpha$. В~качестве альтернативы можно использовать вид доверительного интервала 
из статьи~\cite{Ushakov2017b}, полученный с~помощью неравенства $\mathbb{D} [Z]
\hm\leqslant 2\mathbb{D} Z\hm+{1}/{2}$, и~искать решение уравнения вида:
\begin{multline}
\hspace*{-2.90578pt}\fr{2\sigma z_{1-{\alpha}/2}}{\sqrt{n (2\sigma^2+{1}/{2})}} -
 e^{-2\pi^2\sigma^2}\left(4\pi\sigma+\fr{1}{2\pi^3\sigma^3}+
 \fr{1}{\pi\sigma}\right)={}\\
 {}=0\,.\label{f2}
\end{multline}

Примеры найденных значений~$\sigma$ для типичных размеров выборок в~методе 
скользящего разделения смесей (учитываются как возможная ширина окна, 
так и~общее количество наблюдений в~анализируемом ряде) приведены в~таблице 
(использован метод оптимизации \verb"Trust-Region Dogleg" пакета \verb"MATLAB" 
c~настройками по умолчанию), в~которой через~$\sigma_1$ обозначено приближенное  
решение уравнения~\eqref{f1}, a~через $\sigma_2$~--- уравнения~\eqref{f2}.


Проверка практической эффективности данного подхода в~качестве 
критерия выбора параметров зашумляющего распределения для повышения 
точности работы метода скользящего разделения смесей может быть отмечена 
как задача для дальнейших исследований.


\section{Конечные смеси гамма-распределений}

Для случайной величины~$X$, имеющей распределение типа конечной смеси 
гам\-ма-рас\-пре\-де\-ле\-ний с~параметрами ${\bf r}\hm=(r_1,\ldots, r_k)$,
 $r_j\hm>0$, $\boldsymbol{\lambda}\hm=(\lambda_1,\ldots, \lambda_k)$, $\lambda_j\hm>0$, 
 ${\bf p}\hm=(p_1,\ldots, p_k)$, $p_j\hm\geqslant 0$, $\sum\nolimits_{j=1}^{k}p_j\hm=1$, 
 плот\-ность которого задается выражением
\begin{equation}
f_X(x)=\sum\limits_{j=1}^{k}p_j\fr{\lambda_j^{r_j} e^{-\lambda_j x}}
{\Gamma(r_j)}\,x^{r_j-1}\,,
\label{FinGammaMixt}
\end{equation}
характеристическая функция имеет следующий вид:
%характеристическая функция задается следующим выражением:
\begin{equation}
\varphi_X(t)=\!\int\limits_{-\infty}^{+\infty}\!\!\!e^{itx} f_X(x)\, dx = \!
\sum\limits_{j=1}^{k}p_j \left(\!1-\fr{it}{\lambda_j}\right)^{-r_j}\!.\!
\label{ChiFinGammaMixt}
\end{equation}

Отметим, что подобные модели зашумления разумно использовать в~случае, 
если известно, что данные сосредоточены на положительной полуоси, например 
при анализе различных информационных потоков (см., в~част\-ности, 
 работу~\cite{Gorshenin2013}). 

Проверим абсолютную интегрируемость функции $\varphi_X(t)$~\eqref{ChiFinGammaMixt}. 
Имеем:
\begin{multline*}
\int\limits_{-\infty}^{+\infty}\left\lvert\varphi_X(t)\right\rvert\, dt\leqslant 
\sum\limits_{j=1}^{k}p_j \int\limits_{-\infty}^{+\infty}\left\lvert \left(
1-\fr{it}{\lambda_j}\right)^{-r_j}\right\rvert \, dt={}\\
{}=\sum\limits_{j=1}^{k}p_j \int\limits_{-\infty}^{+\infty} \left\lvert\left(
\fr{\lambda_j(\lambda_j+it)}{\lambda_j^2+t^2}\right)^{r_j}\right\rvert\, dt \leqslant{}\\
{}\leqslant\sum\limits_{j=1}^{k}p_j \lambda_j \int\limits_{-\infty}^{+\infty}\left(
1+y^2\right)^{-{r_j}/{2}}\, dy\,.
\end{multline*}

Подынтегральное выражение при $r_j\hm\geqslant 2$ может быть оценено сверху 
функцией $1/({1+y^2})$, при этом соответствующий интеграл равен~$\pi$, что влечет 
абсолютную интегрируемость характеристической функции для конечной смеси 
гам\-ма-рас\-пре\-де\-ле\-ний. Поэтому в~дальнейшем будем предполагать,
 что $r_j\hm\geqslant 2$ для всех возможных значений $j\hm=1,2,\ldots$

Рассмотрим вопрос точ\-ности оценивания неизвестного математического ожидания ${\sf E}_X\hm>0$ 
при добавлении зашумления.

\smallskip

\noindent
\textbf{Теорема~3.}
\textit{Пусть выполнены предположения}~(A)--(D), 
\textit{причем случайные величины~$\varepsilon_j$, $j\hm=1,2,\ldots$, имеют 
распределение типа конечной $k$-ком\-по\-нент\-ной смеси 
гам\-ма-рас\-пре\-де\-ле\-ний вида}~\eqref{FinGammaMixt} 
\textit{с~па\-ра\-мет\-ра\-ми~${\bf r}$, $\boldsymbol{\lambda}$ и~${\bf p}$. Тогда}
\begin{equation}
\label{Th3Eq}
\left\lvert {\sf E}_Y-{\sf E}_X\right\rvert \leqslant \fr{R}{\lambda}+
\fr{\Lambda^{R}}{2^{r}\pi^{r+1}}\left(1+\frac1{r}\right)\,,
\end{equation}
\textit{где} $r=\min(r_1, \ldots,r_k)$; $R\hm=\max(r_1, \ldots,r_k)$; 
$\lambda\hm=\max(\lambda_1, \ldots,\lambda_k)$; 
$\Lambda\hm=\max(\lambda_1, \ldots,\lambda_k)$.

\smallskip

\noindent
Д\,о\,к\,а\,з\,а\,т\,е\,л\,ь\,с\,т\,в\,о\,.\ \
С~учетом пред\-став\-ле\-ний~\eqref{Law} и~\eqref{Fract}, ограниченности 
модуля характеристической функции, перехода от тригонометрической к~показательной 
записи комплексных чисел, а~также независимости случайных величин~$X_j$ 
и~$\varepsilon_j$ \mbox{имеем}:
\begin{multline*}
\left\lvert {\sf E}_Y-{\sf E}_X\right\rvert
\leqslant \left\lvert {\sf E}_\varepsilon\right\rvert+ {}\\
{}+\left\lvert\sum\limits_{n=1}^\infty
\left(
(-1)^n\mathrm{Im} \left(\sum\limits_{j=1}^{k}p_j \varphi_{X_j}(2\pi n)\left(
\vphantom{\fr{2\pi n}{\lambda_j}}
1-{}\right.\right.\right.\right.\\
\left.\left.\left.\left.{}-i\left(\fr{2\pi n}{\lambda_j}\right)\right)^{-r_j}\right)
\Bigg/ ({\pi n})
\vphantom{\sum\limits_{j=1}^{k}}
\right)\right\rvert={}\\
{}=\left\lvert {\sf E}_\varepsilon\right\rvert+ 
\left\lvert\sum\limits_{n=1}^\infty
\left(\!(-1)^n\mathrm{Im} \!\left(\sum\limits_{j=1}^{k}p_j \left(\!
1+\fr{4\pi^2 n^2}{\lambda_j^2}\right)^{- {r_j}/2}\!\times{}\right.\right.\right.\hspace*{-2.8663pt}\\
\left.\left.\left.{}\times \varphi_{X_j}(2\pi n)\,
e^{-ir_j\mathrm{arctan}\,({{t}/{\lambda_j}})}\right)
\Bigg/
({\pi n})
\vphantom{\left(
1+\fr{4\pi^2 n^2}{\lambda_j^2}\right)^{- {r_j}/2}}
\right)\right\rvert\leqslant{}\\
{}\leqslant \left\lvert {\sf E}_\varepsilon\right\rvert+\sum\limits_{j=1}^{k}
p_j\sum\limits_{n=1}^\infty\fr{1}{\pi n}\left(
1+\fr{4\pi^2 n^2}{\lambda_j^2}\right)^{-{r_j}/2}\leqslant{}\\
{}\leqslant  \fr{R}\lambda + \sum\limits_{j=1}^{k}p_j
\sum\limits_{n=1}^\infty\left(\fr{1}{\pi n}\,
\fr{\lambda_j^{r_j}}{(2\pi)^{r_j} n^{r_j}}\right)\leqslant {}
\\
{}\leqslant  \fr{R}{\lambda} + \sum\limits_{j=1}^{k}p_j 
\fr{\lambda_j^{r_j}}{2^{r_j}\pi^{r_j+1}}\left(1+\int\limits_{1}^{\infty}
\fr{1}{ x^{r_j+1}}\,dx\right)
\leqslant{}\\
{}\leqslant \fr{R}{\lambda}+\fr{\Lambda^{R}}{2^{r}\pi^{r+1}}\left(1+\fr{1}{r}\right).
\end{multline*}

При переходе от суммы к~интегралу используется факт убывания функции как переменной~$n$ 
(или~$x$).~\hfill$\square$


\smallskip

\noindent
\textbf{Замечание~3.}\
Теорема~3 описывает соответ\-ст\-ву\-ющий результат для гам\-ма-рас\-пре\-де\-лен\-ных 
за\-шум\-ля\-ющих случайных величин, если положить $k\hm=1$ в~выражении~\eqref{Th3Eq}. 
При этом, очевидно, $r\hm\equiv R$ и~$\lambda\hm\equiv \Lambda$.


\smallskip

Рассмотрим вопросы построения доверительного интервала для неизвестного 
математического ожидания ${\sf E}_X\hm>0$ в~предположении, что случайные величины~$X_j$ 
не содержат ошибок измерения, а все погрешности учтены исключительно в~за\-шум\-ля\-ющих 
элементах~$\varepsilon_j$.

\smallskip

\noindent
\textbf{Теорема~4.}
\textit{Пусть выполнены предположения}~(A)--(D), 
\textit{причем случайные величины~$\varepsilon_j$, $j\hm=1,2,\ldots$, имеют 
распределение типа конечной $k$-ком\-по\-нент\-ной смеси 
гам\-ма-рас\-пре\-де\-ле\-ний вида}~\eqref{FinGammaMixt} 
\textit{с~па\-ра\-мет\-ра\-ми~${\bf r}$, $\boldsymbol{\lambda}$ и~${\bf p}$, 
а~случайные величины} $X_j\stackrel{\text{п.н.}}{=}{\sf E}_X$, $j=1,2,\ldots$ 
\textit{Тогда доверительный интервал для~${\sf E}_X$ при условии $0\hm<\alpha\hm<1$ имеет вид}:
\begin{equation} 
\label{Th4Eq}
\left\lvert {\sf E}_X - \hat{{\sf E}}_X\right\rvert \leqslant  
f({\bf r},\boldsymbol{\lambda},\alpha,n),
\end{equation}
\textit{где}

\vspace*{-9pt}

\noindent
\begin{align}
\hat{{\sf E}}_X&=\fr{1}{n} \sum\limits_{j=1}^{n} Y_j\,; \label{Th4hatE}\\[-4pt]
f({\bf r}, \boldsymbol{\lambda},\alpha,n)&=\fr{z_{1-{\alpha}/2}}{\sqrt{n}} \left(
\sqrt{\fr{R(R+1)}{\lambda^2}-\fr{r^2}{\Lambda^2}}+\fr{1}{2}\right) +{}\notag\\[-1pt]
&\hspace*{20mm}{}+
\fr{R}{\lambda}+\fr{\Lambda^{R}}{2^{r}\pi^{r+1}}\left(1+\fr{1}{r}\right); \notag
\end{align}
\textit{$z_{1-{\alpha}/2}$~--- $\left(1-{\alpha}/2\right)$-кван\-тиль 
стандартного нормального распределения; $r\hm=\min(r_1, \ldots,r_k)$; 
$R\hm=\max(r_1, \ldots,r_k)$; $\lambda\hm=\max(\lambda_1, \ldots,\lambda_k)$; 
$\Lambda\hm=\max(\lambda_1, \ldots,\lambda_k)$}. 

\smallskip

\noindent
Д\,о\,к\,а\,з\,а\,т\,е\,л\,ь\,с\,т\,в\,о\,.\ \
Из центральной предельной теоремы с~учетом условия~(A) 
следует, что величина~$\hat{{\sf E}}_X$~\eqref{Th4hatE} асимптотически нормальна 
с~математическим ожиданием~${\sf E}_Y$~\eqref{EY} и~дисперсией $(1/n){\sf D}_Y$~\eqref{DY}. 
Пользуясь определением и~свойствами гам\-ма-функ\-ции, а~также оценкой~\eqref{Var} 
получим:

\noindent
\begin{multline*}
{\sf D}_Y \leqslant \left(\sqrt{\sum\limits_{j=1}^k p_j
\fr{\lambda_j^{r_j}}{\Gamma(r_j)} \int\limits_{0}^{+\infty} 
e^{\lambda_j x}x^{r_j+1}\, dx}+\fr{1}{2}\right)^2= {}\\[-0.5pt]
= \left(\sqrt{\sum\limits_{j=1}^{k}p_j
\fr{r_j(r_j+1)}{\lambda_j^2}-\left(\sum\limits_{j=1}^{k}p_j
\fr{r_j}{\lambda_j}\right) ^2}+\fr{1}{2}\right)^2\leqslant {}\\[-1.5pt]
{}\leqslant \left(\sqrt{\fr{R(R+1)}{\lambda^2}-\fr{r^2}{\Lambda^2}}+\fr{1}{2}\right)^2\,.
\end{multline*}

Аналогично доказательству Тео\-ре\-мы~2 с~учетом оценки~\eqref{Th3Eq} 
отсюда следует справедливость соотношения~\eqref{Th4Eq}.~\hfill$\square$

\vspace*{-12pt}

\section{Заключение}

Итак, в~работе получены оценки для математического ожидания наблюдений в~предположении 
зашумления конечными смесями нормальных\linebreak (Тео\-ре\-ма~1) 
и~гам\-ма-рас\-пре\-де\-ле\-ний (Тео\-ре\-ма~3). 
%
Построены доверительные интервалы 
для неизвестного математического ожидания в~этих случаях с~использованием 
уточненной оценки~\eqref{Var} 
(Тео\-ре\-мы~2 и~4 соответственно). Отметим, что соответствующие соотношения 
зависят только от <<экстремальных>> значений параметров смесей, но не от числа 
компонент и~весов в~распределении зашумляющих наблюдений. 
%
Замечание~2 
предлагает подход, который  может быть использован для определения неизвестного 
параметра искусственно добавляемого к~исходным данным шума для улучшения качества 
работы метода скользящего разделения смесей.

\smallskip
Автор выражает признательность доктору фи\-зи\-ко-ма\-те\-ма\-ти\-че\-ских наук, 
профессору Виктору Юрьевичу Королеву за идею использования оценки 
дисперсии вида~\eqref{Var} и~другие полезные обсуждения в~рамках 
работы над данной статьей.

\vspace*{-12pt}

{\small\frenchspacing
 {%\baselineskip=10.8pt
 \addcontentsline{toc}{section}{References}
 \begin{thebibliography}{99}
\bibitem{Wright2003} \Au{Wright~D.\,E., Bray~I.} 
A~mixture model for rounded data~// J.~Roy. Stat. Soc.~D 
Sta., 2003. Vol.~52. P.~3--13.

\columnbreak

\bibitem{Bai2009} \Au{Bai~Z., Zheng~S., Zhang~B., Hu~G.} 
Statistical analysis for rounded data~// J.~Stat. Plan.  Infer., 2009. 
Vol.~139. Iss.~8. P.~2526--2542.

\bibitem{Zhang2010} \Au{Zhang~B., Liu~T., Bai~Z.\,D.} 
Analysis of rounded data from dependent sequences~// 
Ann. I.~Stat. Math., 2010. Vol.~62. Iss.~6. P.~1143--1173.

\bibitem{Zhao2012} \Au{Zhao~N., Bai~Z.} 
Analysis of rounded data in mixture normal model~// Stat. Pap., 2012. 
Vol.~53. P.~895--914.

\bibitem{Korolev2011-i} \Au{Королев~В.\,Ю.} 
Ве\-ро\-ят\-но\-ст\-но-ста\-ти\-сти\-че\-ские методы декомпозиции волатильности 
хаотических процессов.~--- М.: Изд-во Моск. ун-та, 2011. 512~с.

\bibitem{Ushakov2015} \Au{Ушаков В.\,Г., Ушаков Н.\,Г.} 
Об усреднении округленных данных~// Информатика и~её применения, 2015. Т.~9. 
Вып.~4. С.~106--109.

\bibitem{Ushakov2017a} \Au{Ушаков~В.\,Г., Ушаков~Н.\,Г.} 
Границы точ\-ности восстановления информации, 
теряемой при округлении результатов наблюдений~// 
Вестник Московского университета. Серия~15: Вычислительная математика и~кибернетика, 
2017. №\,2. С.~26--30.

\bibitem{Ushakov2017b} \Au{Ushakov~N.\,G., Ushakov~V.\,G.} 
Statistical analysis of rounded data: Recovering of information lost due to rounding~// 
J.~Korean Stat. Soc., 2017.  Vol.~46. No.\,3. P.~426--437.

\bibitem{Gorshenin2016} \Au{Gorshenin~A.\,K., Korolev~V.\,Yu.} 
A~noising method for the identification of the stochastic structure of 
information flows~// Comm. Com. Inf. Sc., 2017. 
Vol.~678. P.~279--289.

\bibitem{Gorshenin2013} 
\Au{Gorshenin~A., Korolev~V.} Modelling of statistical
fluctuations of information flows by mixtures of gamma distributions~// 
27th European Conference on Modelling and Simulation Proceedings.~--- 
Dudweiler, Germany: Digitaldruck Pirrot GmbHP, 2013. P.~569--572.
 \end{thebibliography}

 }
 }

\end{multicols}

\vspace*{-6pt}

\hfill{\small\textit{Поступила в~редакцию 03.08.18}}

\vspace*{6pt}

%\newpage

%\vspace*{-24pt}

\hrule

\vspace*{2pt}

\hrule

\vspace*{-2pt}


\def\tit{DATA NOISING BY FINITE NORMAL AND~GAMMA MIXTURES WITH~APPLICATION 
TO~THE~PROBLEM OF~ROUNDED OBSERVATIONS}


\def\titkol{Data noising by finite normal and~gamma mixtures with~application 
to~the~problem of~rounded observations}



\def\aut{A.\,K.~Gorshenin}

\def\autkol{A.\,K.~Gorshenin}

\titel{\tit}{\aut}{\autkol}{\titkol}

\vspace*{-11pt}


\noindent
Institute of Informatics Problems, Federal Research Center ``Computer Science and
Control'' of the Russian Academy of Sciences, 44-2~Vavilov Str., Moscow 119333,
Russian Federation


\def\leftfootline{\small{\textbf{\thepage}
\hfill INFORMATIKA I EE PRIMENENIYA~--- INFORMATICS AND
APPLICATIONS\ \ \ 2018\ \ \ volume~12\ \ \ issue\ 3}
}%
 \def\rightfootline{\small{INFORMATIKA I EE PRIMENENIYA~---
INFORMATICS AND APPLICATIONS\ \ \ 2018\ \ \ volume~12\ \ \ issue\ 3
\hfill \textbf{\thepage}}}

\vspace*{3pt}



\Abste{In many real problems, statistical analysis of data containing additional 
measurement errors, including rounding, is performed, which in some situations can 
lead to sufficiently significant distortions. In this paper, estimates for an 
unknown expectation of observations are obtained for one of the possible 
rounding models under the assumption that the original data are additionally 
noised with random variables having distributions of the type of finite 
mixtures of normal and gamma laws. Confidence intervals for an 
unknown expectation are constructed using the refined estimate for 
the variance of the integer part of the random variable. An algorithm 
for determining the value of the parameter of artificial noise, which 
can be added to the initial data to improve the quality of the 
method of moving separation of mixtures, is discussed.}


\KWE{noisy data; rounded data; finite normal mixtures; finite gamma mixtures; 
confidence intervals; moving separation of mixtures}



\DOI{10.14357/19922264180304}

%\vspace*{-14pt}

\Ack
\noindent
The research was supported by the Russian Science Foundation (project 18-71-00156).



%\vspace*{6pt}

  \begin{multicols}{2}

\renewcommand{\bibname}{\protect\rmfamily References}
%\renewcommand{\bibname}{\large\protect\rm References}

{\small\frenchspacing
 {%\baselineskip=10.8pt
 \addcontentsline{toc}{section}{References}
 \begin{thebibliography}{99}
\bibitem{1-gor-1}
\Aue{Wright,~D.\,E., and I.~Bray.} 2003.
A~mixture model for rounded data.  \textit{J.~Roy. Stat. Soc.~D Sta.} 52:3--13.

\bibitem{2-gor-1}
\Aue{Bai,~Z., S.~Zheng, B.~Zhang, and G.~Hu.} 2009. 
Statistical analysis for rounded data. \textit{J.~Stat. Plan. 
Infer.} 139(8):2526--2542.

\bibitem{3-gor-1}
\Aue{Zhang,~B., T.~Liu, and Z.\,D.~Bai.} 2010. 
Analysis of rounded data from dependent sequences. 
\textit{Ann. I.~Stat. Math.} 62(6):1143--1173.

\bibitem{4-gor-1}
\Aue{Zhao,~N., and Z.~Bai.} 2012. Analysis of rounded data in mixture normal model. 
\textit{Stat. Pap.} 53:895--914.

\bibitem{5-gor-1}
\Aue{Korolev, V.\,Yu.} 2011. 
\textit{Veroyatnostno-statisticheskie metody dekompozitsii volatil'nosti 
khaoticheskikh protsessov} [Probabilistic and statistical methods of 
decomposition of volatility of chaotic processes]. 
Moscow: Moscow University Publishing House. 512~p.

\bibitem{6-gor-1}
\Aue{Ushakov, V.\,G., and N.\,G.~Ushakov.} 
2015. Ob usrednenii okruglennykh dannykh [On averaging of rounded data].
\textit{Informatika i~ee Primeneniya~--- Inform. Appl.} 9(4):106--109.

\bibitem{7-gor-1}
\Aue{Ushakov,~V.\,G., and N.\,G.~Ushakov.} 2017. 
Boundaries of the precision of restoring information lost after rounding
 the results from observations. 
 \textit{Moscow University Computational Math. Cybernetics} 41(2):76--80.

\bibitem{8-gor-1}
\Aue{Ushakov,~N.\,G., and  V.\,G.~Ushakov.} 2017. 
Statistical analysis of rounded data: Recovering of information lost due to rounding. 
\textit{J.~Korean Stat. Soc.} 46(3):426--437.

\bibitem{9-gor-1}
\Aue{Gorshenin,~A.\,K., and V.\,Yu.~Korolev.} 2016. 
A~noising method for the identification of the stochastic structure of information 
flows. \textit{Comm. Com. Inf. Sc.} 678:279--289.

\bibitem{10-gor-1}
\Aue{Gorshenin,~A., and V.~Korolev.} 2013.  Modelling of statistical fluctuations of
information flows by mixtures of gamma distributions. 
\textit{27th European Conference on Modelling and Simulation Proceedings}. 
Dudweiler, Germany: Digitaldruck Pirrot GmbHP. 569--572.

\end{thebibliography}

 }
 }

\end{multicols}

\vspace*{-6pt}

\hfill{\small\textit{Received August 3, 2018}}

%\pagebreak

%\vspace*{-18pt}

\Contrl

\noindent
\textbf{Gorshenin Andrey K.} (b.\ 1986)~--- Candidate of Science (PhD) in physics and
mathematics, associate professor, leading scientist, Institute of Informatics Problems,
Federal Research Center ``Computer Science and Control'' of the Russian Academy of
Sciences, 44-2 Vavilov Str., Moscow 119333, Russian Federation; 
\mbox{agorshenin@frccsc.ru}
\label{end\stat}

\renewcommand{\bibname}{\protect\rm Литература}         %2
%\newcommand {\ff}{{\mathcal F}}
\newcommand {\ebd}{\triangleq}
\newcommand{\me}[2]{\mathbf{E}_{ #1 }\left\{ \mathop{#2} \right\} }



\def\stat{borisov}

\def\tit{ФИЛЬТРАЦИЯ СОСТОЯНИЙ МАРКОВСКИХ СКАЧКООБРАЗНЫХ ПРОЦЕССОВ 
ПО~ДИСКРЕТИЗОВАННЫМ НАБЛЮДЕНИЯМ$^*$}

\def\titkol{Фильтрация состояний марковских скачкообразных процессов 
по~дискретизованным наблюдениям}

\def\aut{А.\,В.~Борисов$^1$}

\def\autkol{А.\,В.~Борисов}

\titel{\tit}{\aut}{\autkol}{\titkol}

\index{Борисов А.\,В.}
\index{Borisov A.\,A.}




{\renewcommand{\thefootnote}{\fnsymbol{footnote}} \footnotetext[1]
{Работа выполнена при частичной поддержке РФФИ (проект 16-07-00677).}}


\renewcommand{\thefootnote}{\arabic{footnote}}
\footnotetext[1]{Институт проблем информатики Федерального исследовательского центра <<Информатика 
и~управление>> Российской академии наук,
\mbox{aborisov@frccsc.ru}}

%\vspace*{8pt}



\Abst{Статья посвящена решению задачи оптимальной 
фильтрации состояний однородного марковского скачкообразного процесса (МСП). 
Наблюдения представляют собой приращения случайных процессов~--- интегральных 
преобразований состояний, зашумленные винеровскими процессами, интенсивность 
которых также зависит от оцениваемого состояния. Оптимальная оценка в~моменты 
получения нового наблюдения вычисляется как функция предыдущей оценки и~новых 
наблюдений, а~между моментами наблюдений~--- простейшим прогнозом в~силу системы 
уравнений Колмогорова. Рекуррентная формула пересчета ресурсозатратна, так как 
содержит  интегралы~--- мас\-штаб\-но-сдви\-го\-вые смеси многомерных гауссиан, 
где в~качестве смешивающих выступают распределения времени пребывания 
состояния в~каждом из возможных значений. Предложены более простые аппроксимации, 
основанные на предположении об ограниченности числа скачков состояния за время между 
наблюдениями. Получены универсальные локальная и~глобальная характеристики точности 
аппроксимаций, зависящие от па\-ра\-мет\-ров оцениваемого процесса, величины 
временн$\acute{\mbox{о}}$го шага  между наблюдениями и~максимального числа учитываемых скачков.}

\KW{марковский скачкообразный процесс; оптимальная фильтрация; мультипликативные 
шумы в~наблюдениях; стохастическое дифференциальное уравнение; численная аппроксимация}

\DOI{10.14357/19922264180316}
  
%\vspace*{4pt}


\vskip 10pt plus 9pt minus 6pt

\thispagestyle{headings}

\begin{multicols}{2}

\label{st\stat}



 \section{Введение}
 
 Фильтр Вонэма~\cite{Won_65}~--- один из редких удачных случаев, когда 
 оценка оптимальной фильтрации состо\-яния стохастической системы наблюдения 
 выражается в~виде решения некоторой замк\-ну\-той\linebreak конечномерной сис\-те\-мы 
 стохастических дифференциальных уравнений. 
 
 Алгоритм данного фильт\-ра 
 позволяет вычислить оценку фильт\-ра\-ции со\-сто\-яния \textit{марковского скачкообразного 
 процесса} с~\mbox{конечным} множеством состояний по наблюдениям в~присутствии 
 аддитивных винеровских шумов. Теоретически оптимальная оценка со\-сто\-яния~--- 
 его условное распределение в~текущий момент времени~--- 
 обладает очевидными свойствами неотрицательности и~нормировки. 
 При чис\-лен\-ной реализации данного фильтра классическим методом 
 Эй\-ле\-ра--Ма\-ру\-ямы~\cite{KP_92} данные свойства могут не сохраняться и~процедура 
 вы\-чис\-ле\-ния становится неустойчивой.  В~связи с~этим обстоятельством разрабатывались 
 другие алгоритмы чис\-лен\-но\-го решения уравнения фильтра Вонэма, обладающие 
 требуемыми свойствами устойчивости (см.~\cite{YZL_04, PR_10} и~библиографию в~них). 
 В~час\-ти этих работ доказана лишь слабая сходимость пред\-ла\-га\-емых аппроксимационных 
 схем к~оценке фильт\-ра Вонэма, в~то время как ка\-кая-ли\-бо 
 характеризация точ\-ности этих приближений отсутствует.
 
 В~\cite{B_18} было представлено распространение фильт\-ра Вонэма на случай 
 наблюдений с~мультипликативными шумами. При этом уравнение обобщенного 
 фильт\-ра содержит в~правой части квадратическую характеристику шумов в~наблюдениях. 
 Данный процесс на практике никогда не наблюдается непосредственно, а~является лишь 
 некоторым нелинейным интегральным преобразованием наблюдений. Очевидно, что 
 имеющиеся в~настоящий момент времени алгоритмы приближенного вычисления оценки 
 фильтрации Вонэма для данной системы не подходят. 
 
 Целью предлагаемой работы является ис\-поль\-зование результатов оптимальной 
 фильтрации со\-стояний сис\-тем с~дискретным временем для аппроксимации решения 
 аналогичной задачи для\linebreak стохастических дифференциальных сис\-тем. 
 
 Статья организована следующим образом. Раздел~2 содержит формальную постановку 
 задачи фильт\-ра\-ции со\-сто\-яний однородного МСП с~конечным множеством со\-сто\-яний 
 по наблюдениям, полученным путем временн$\acute{\mbox{о}}$й дискретизации процессов с~непрерывным 
 временем~--- интегральных преобразований со\-сто\-яния сис\-те\-мы в~присутствии 
 мультипликативных винеровских шумов.\linebreak
  В~разд.~3 пред\-став\-ле\-но решение поставленной 
 задачи фильт\-ра\-ции: пересчет оценок со\-сто\-яний в~момент получения новых 
 дискретизованных наблюдений выполняется в~соответствии с~некоторыми\linebreak 
 рекуррентными интегральными соотношениями, в~то время как между 
 моментами наблюдений оценка корректируется в~соответствии с~прогнозом в~силу 
 сис\-те\-мы уравнений Колмогорова. Вы\-чис\-ли\-тель\-ная слож\-ность 
 упомянутых выше интегральных\linebreak 
 соотношений связана с~тем, что в~расчет принимается воз\-мож\-ность того, что между 
 моментами наблюдений оцениваемое со\-сто\-яние может совершить произвольное чис\-ло 
 скачков. В~разд.~4 пред\-став\-лен более простой алгоритм приближенного вы\-чис\-ле\-ния 
 оценки фильт\-ра\-ции, основанный на ограничении возможного числа учитываемых скачков 
 МСП. Доказана тео\-ре\-ма, опре\-де\-ля\-ющая как\linebreak
  локальную (одношаговую), так и~глобальную 
 (многошаговую) характеристики точ\-ности предложенного при\-бли\-же\-ния~--- 
 $\ell_1$-нор\-мы ошибки аппроксимации. Полученные характеристики являются\linebreak 
 универсальными, т.\,е.\ не асимптотическими по шагу дискретизации, и~зависят от характеристик 
 самого МСП, %\linebreak
  шага временн$\acute{\mbox{о}}$й дискретизации и~чис\-ла
  скачков со\-сто\-яния, учи\-ты\-ва\-емых 
 на шаге. Об\-суж\-де\-ние результатов и~заключительные комментарии пред\-став\-ле\-ны 
 в~разд.~5.
 
 \section{Постановка задачи фильтрации}
 
 На полном вероятностном пространстве с~фильт\-ра\-цией 
 $(\Omega,\mathcal{F},\mathcal{P},\{\mathcal{F}_{t}\}_{t \geqslant 0})$ рассматривается система наблюдений
\begin{equation}
 \left.
 \begin{array}{rl}
 \displaystyle X_t &=X_0 +  \displaystyle
 \int\limits_0^t \Lambda^{\top}X_{s}\,ds + \mu_s\,;  \\[6pt]
 \displaystyle Y_k &=  \displaystyle\int\limits_{t_{k-1}}^{t_k}fX_s\,ds+
 \int\limits_{t_{k-1}}^{t_k} 
 \sum\limits_{n=1}^NX_s^ng_n \,dW_s, \\[6pt]
 &\hspace*{10mm}\{t_k\}_{k \geqslant 0}: \; 0 = t_0 < t_1 < t_2\cdots,
 \end{array}
 \right\}
 \label{eq:obsys_1}
 \end{equation}
 где
  \begin{itemize}
  \item
  $X_t \ebd \mathrm{col}\left(X_t^1,\ldots,X_t^N\right) \hm\in \mathbb{S}^N$~--- 
  ненаблюда\-емое состояние системы, являющееся однородным МСП с~конечным 
  множеством состояний $ \mathbb{S}^N \ebd$\linebreak $\ebd \{e_1,\ldots,e_N\}$ ($\mathbb{S}^N$~--- 
  множество единичных векторов евклидова пространства~$\mathbb{R}^N$), 
  матрицей интенсивностей переходов~$\Lambda$ и~начальным распределением~$\pi$;
  \item
  $\mu_t \ebd \mathrm{col}\left(
  \mu_t^1,\ldots,\mu_t^N\right)\hm\in \mathbb{R}^N$~--- 
  ${\mathcal{F}}_t$-со\-гла\-со\-ван\-ный мартингал;
  \item
  $\{Y_k\}_{k \in \mathbb{N}}:\;  Y_k \ebd \mathrm{col}\left(Y_k^1,\ldots,Y_k^M\right) 
  \hm\in \mathbb{R}^M$~--- последовательность дискретизованных наблюдений, 
  доступных в~известные неслучайные  моменты времени~$\{t_k\}_{k \in \mathbb{N}}$,
в~которых $W_t \ebd$\linebreak $\ebd \mathrm{col}\left(W_t^1,\ldots,W_t^M\right) \hm\in \mathbb{R}^M$
 является ${\mathcal{F}}_t$-со\-гла\-со\-ван\-ным стандартным винеровским процессом, 
 определяющим шумы в~наблюдениях,\linebreak  $f$~--- $(M \times N)$-мер\-ная 
 мат\-ри\-ца плана наблюдений, а~набор мат\-риц~$\{g_n\}_{n=\overline{1,N}}$ 
 характеризует интенсивности шумов в~зависимости от текущего состояния~$X_t$.
  \end{itemize}
  
  Введем также в~рассмотрение неубывающие семейства $\sigma$-ал\-гебр 
  $\mathcal{O}_k \ebd \sigma\{ Y_{\ell}: \; 1 \hm\leqslant \ell \hm\leqslant k\}$ 
  и~$\mathcal{O}_t \ebd  \mathcal{O}_{k(t)}$, где 
  $k(t) \ebd \sum\nolimits_{j \in \mathbb{N}}\mathbf{I}(t-t_{j})$; 
  $\mathcal{O}_0 \ebd \{\varnothing,\; \Omega\}$.
  
   \textit{Задача оптимальной фильтрации состояния~$X$ по наблюдениям~$Y$} 
   заключается в~нахождении \textit{условного математического ожидания} (УМО)
  \begin{equation*}
  \widehat{X}_t \ebd {\sf E}\left\{X_t|\mathcal{O}_{t} \right\}\,.
 % \label{eq:fest_1}
  \end{equation*}
  
  Относительно системы~(\ref{eq:obsys_1})  сделаны следующие предположения:
   \begin{itemize}
 \item[(а)]
 ${\mathcal{F}}_t \equiv {\mathcal{F}}_{t}^X \bigvee 
 {\mathcal{F}}_{t}^W $ для любого $t \hm\geqslant 0$;
 \item[(б)]
 шумы в~наблюдениях равномерно невырожденные, т.\,е.\
  $g_ng_n^{\top} \hm\geqslant \alpha I \hm> 0$ для всех $n\hm=\overline{1,N}$ 
  и~некоторого $\alpha\hm>0$.
% \item
 % Верно неравенство
  %\begin{equation}
  %\min_{1\leqslant k \leqslant N}|\lambda_{kk}| > 0.
  %\label{eq:ineq_0}
  % \end{equation}
 %\item
 %Для любого $t \geqslant 0$ все компоненты вектора $p_t \ebd \me{}{X_t}$ строго %положительны. 
 \end{itemize} 

 \section{Уравнения оптимального фильтра} 
 
 Для получения уравнений оптимального фильт\-ра воспользуемся подходом, 
 применяемым для решения аналогичной задачи в~стохастических сис\-те\-мах 
 наблюдения с~дискретным временем~\cite{BSh_85}. 
 Воспользу\-ем\-ся методом математической индукции. 
 
 При $r=0$ 
 \begin{equation}
 \widehat{X}_{t_0}={\sf E}\{X_0|\mathcal{O}_0\}={\sf E}\{X_0\}=\pi\,.
 \label{eq:in_cond}
 \end{equation} 
 
 Пусть для некоторого $ r \hm\geqslant 0$ известна оценка оптимальной 
 фильтрации~$\widehat{X}_{t_r} \hm= {\sf E}{X_{t_r} |\mathcal{O}_r}$. 
 Определим оценку оптимальной фильтрации~$\widehat{X}_{t} $ для $t\hm \in (t_r,t_{r+1}]$. 
 
 Для произвольного момента $t \hm\in (t_r,t_{r+1})$ в~силу мартингального 
 разложения МСП~$X_t$ и~свойств УМО верна следующая цепочка равенств:
 \begin{multline*}
 \widehat{X}_{t} = {\sf E}\left\{X_t | \mathcal{O}_r\right\}={}\\
 {}=
 {\sf E}\left\{{\cal P}^{\top}(t_r,t)X_{t_r}+
 \int\limits_{t_r}^t{\cal P}^{\top}(t_r,s)\,dM_s\big\vert \mathcal{O}_r\right\} = {}
\end{multline*}

\noindent
   \begin{multline}
 \hspace*{-11.66pt}{}=\mathcal{P}^{\top}(t_r,t)\widehat{X}_{t_r} + {\sf E}\hspace*{-2pt}
 \left\{{\sf E}\hspace*{-2pt}\left\{\int\limits_{t_r}^t\hspace*{-2pt}\mathcal{P}^{\top}(t_r,s)\,dM_s |
 {\mathcal{F}}_{t_r}\right\}\!\big\vert 
 \mathcal{O}_r\!\right\} ={}\hspace*{-4.24124pt}\\
 {}=
  \mathcal{P}^{\top}(t_r,t)\widehat{X}_{t_r}\,,
 \label{eq:bw_obs}
 \end{multline}
 где $\mathcal{P}(s,t)$ $(s \hm\leqslant t)$~--- матрица переходной ве\-ро\-ят\-ности МСП 
 на промежутке $[s,t]$, являющаяся решением сис\-те\-мы дифференциальных 
 уравнений Колмогорова
 \begin{equation*}
 \mathcal{P}'_t(s,t) = \mathcal{P}(s,t) \Lambda, \enskip t > s, \enskip \mathcal{P}(s,s) = I.
 \end{equation*}
 В случае однородного МСП $\mathcal{P}(s,t) \hm= e^{(t-s)\Lambda}$.
 
 Далее необходимо определить совместное распределение $(X_{t_{r+1}},Y_{r+1})$ 
 относительно~$ \mathcal{O}_r$. Из модели наблюдений следует, что 
 распределение~$Y_{r+1}$ относительно 
 $\sigma$-ал\-геб\-ры~$\mathcal{F}^X_{t_{r+1}} \vee \mathcal{O}_r$~---
 гауссовское с~параметрами 
 \begin{align*}
{\sf E}\left\{Y_{r+1}|{\mathcal{F}}^X_{t_{r+1}}\right\}& = f \tau_{r+1}\,; \\[6pt]
 \mathrm{cov} \left(Y_{r+1},Y_{r+1}|{\mathcal{F}}^X_{t_{r+1}}\right) &= 
 \displaystyle\sum\limits_{n=1}^N \tau_{r+1}^n g_ng_n^{\top}\,,
% \label{eq:occup_1}
 \end{align*}
 где $\tau_{r+1} \hm= \tau_{r+1}(X(\omega))=
 \mathrm{col}\left(\tau_{r+1}^1,\ldots,\tau_{r+1}^N\right) \ebd$\linebreak
 $\ebd 
 \int\nolimits_{t_r}^{t_{r+1}}X_s\,ds$~--- случайный вектор, $n$-я 
 компонента которого равна времени пребывания процесса~$X$ в~со\-сто\-янии~$e_n$ 
 на  интервале времени $[t_r, t_{r+1}]$. 
 Обозначим через $\mathcal{D}_{r+1} \ebd \{u=\mathrm{col}\,(u^1,\ldots,u^N):\; 
 u_m \hm\geqslant 0,\; \sum\nolimits_{m=1}^Mu_m\hm= t_{r+1}-t_r\}$ $(M-1)$-мер\-ный 
 симплекс в~пространстве~$\mathbb{R}^M$, являющийся носителем распределения 
 вектора~$\tau_{r+1}$. Пусть $\rho^{k,\ell}_{r+1}(du)$~--- 
 распределение вектора $\tau_{r+1} X_{t_{r+1}}^{\ell}$ при условии $X_{t_r}\hm=e_k$, 
 т.\,е.\ 
 для любого $\mathcal{A} \hm\in \mathcal{B}(\mathbb{R}^M)$ верно тождество:
\begin{multline*}
 \mathbf{P}\left\{\omega: \; X_{t_{r+1}}(\omega)=e_{\ell},\right.\\
 \left. 
 \tau_{r+1}(X(\omega)) \in \mathcal{A}\;|\;X_{t_r}=e_k\right\} \equiv
   \rho^{k,\ell}_{r+1}(\mathcal{A})\,.
\end{multline*}
 
Обозначим через
\begin{multline*}
 \mathcal{N}(y,m,K) \ebd (2\pi)^{-M/2} \mathrm{ det}^{-1/2} K\times{}\\
 {}\times\exp
 \left\{ -\fr{1}{2}\left(y-m\right)^{\top}K^{-1}(y-m)\right\}
\end{multline*}
 $M$-мер\-ную плот\-ность гауссовского распределения с~математическим 
 ожиданием~$m$ и~ковариационной матрицей~$K$.
 
 Из марковского свойства  $\{X_{t_{r}},Y_{r})\}_{r \geqslant 0}$ 
 относительно~${\mathcal{F}}_{t_{r}}$~\cite{ZhSh_95} и~теоремы Фубини следует, что 
 для любого  множества $\mathcal{A} \hm\in \mathcal{B}(\mathbb{R}^M)$ 
 верна следующая цепочка равенств:
 \begin{multline*}
 {\sf E}\left\{X_{t_{r+1}}\mathbf{I}_{\mathcal{A}}
 \left(Y_{r+1}\right)\big|\mathcal{O}_r\right\}={}\\
 {}=
{\sf E}\left\{{\sf E}\left\{X_{t_{r+1}}\mathbf{I}_{\mathcal{A}}
\left(Y_{r+1}\right)\big|
\mathcal{F}^X_{t_{r+1}} \vee \mathcal{O}_r\right\}
 \big|\mathcal{O}_r\right\} = {}
\end{multline*}

\noindent
\begin{multline*}
 %{}=
% {\sf E}\left\{{\sf E}\left\{X_{t_{r+1}}\mathbf{I}_{\mathcal{A}}
% \left(Y_{r+1}\right)\vert X_{t_r}\right\}
% \vert\mathcal{O}_r\right\} = {}\\
% {}=
%{\sf E}\left\{\sum\limits_{k=1}^N {\sf E}\left\{X_{t_{r+1}}\mathbf{I}_{\mathcal{A}}
%\left(Y_{r+1}\right)  \big| X_{t_r}=e_k\right\}X_{t_r}^k
% \big|\mathcal{O}_r\right\} = {}\\ 
% {}=
% \sum\limits_{k=1}^N{\sf E}
% \left\{X_{t_{r+1}}\mathbf{I}_{\mathcal{A}}\left(Y_{r+1}\right)\bigl| X_{t_r}=e_k\right\} 
% \widehat{X}_{t_r}^k ={}\\
% {}=\!
% \sum\limits_{k=1}^N{\sf E}
% \left\{{\sf E}\left\{X_{t_{r+1}}\mathbf{I}_{\mathcal{A}}
% \left(Y_{r+1}\right)\!\bigl| {\mathcal{F}}_{t_{r+1}}\right\}\!\bigl| 
% X_{t_r}\!=e_k\right\} \widehat{X}_{t_r}^k ={}\\
% {}=
% \sum\limits_{k=1}^N {\sf E}\left\{
% \vphantom{\int\limits_A\left(\sum\limits_{p=1}^N\right)}
% X_{t_{r+1}} \times{}\right.\\
% {}\times\int\limits_{\mathcal{A}}  
% \mathcal{N}\left(y,f \tau_{r+1}(X),\sum\limits_{p=1}^N \tau_{r+1}^p(X) g_pg_p^{\top}\right)dy
% \Biggl| X_{t_r}={}\\
%\left. {}=e_k
% \vphantom{\int\limits_A\left(\sum\limits_{p=1}^N\right)}
%\right\} \widehat{X}_{t_r}^k = 
% \sum\limits_{k=1}^N \int\limits_{\mathcal{A}}{\sf E}\left\{ 
% \vphantom{\sum\limits_{p=1}^N}
% X_{t_{r+1}} \times{}\right.\\
% {}\times\mathcal{N}\left(y,f \tau_{r+1}(X),\sum\limits_{p=1}^N \tau_{r+1}^p(X) 
% g_p g_p^{\top}\right)
% \Biggl| X_{t_r}={}\\
%\left. {}=e_k
%\vphantom{\sum\limits^N_{p=1}}
%\right\} \widehat{X}_{t_r}^k\, dy
 %={}\\
 {}=
 \sum\limits_{\ell=1}^N e_{\ell} \int\limits_{\mathcal{A}} 
 \left[ \sum\limits_{k=1}^N 
 \int\limits_{\mathcal{D}_{r+1}} 
 \mathcal{N}\left(y,f u,\sum_{p=1}^N u^p g_pg_p^{\top}\right)\times{}\right.\\
\left. {}\times
 \rho^{k,\ell}_{r+1}(du)\widehat{X}_{t_r}^k
 \vphantom{\int\limits_A\sum\limits_{p=1}^N}
 \right] 
 dy,
 \end{multline*}
 из чего следует, что интегранд в~квадратных скобках в~последнем выражении 
 определяет искомое совместное распределение $(X_{t_{r+1}},Y_{r+1})$ 
 относительно~$ \mathcal{O}_r$. Оценка~$\widehat{X}_{t_{r+1}}$ покомпонентно 
 определяется~\cite{BSh_85} с~помощью обобщенного варианта формулы Байеса:
 \begin{multline}
 \widehat{X}_{t_{r+1}}^j = {}\\
 \hspace*{-1mm}{}=
 \fr{\int\nolimits_{\mathcal{D}_{r+1}}\hspace*{-6mm} 
 \mathcal{N}\left(Y_{r+1},f u,\sum\nolimits_{p=1}^N \hspace*{-2mm}
 u^p g_pg_p^{\top}\!\right)\hspace*{-1mm}
 \sum\nolimits_{k=1}^N \hspace*{-2mm}
 \widehat{X}_{t_r}^k
 \rho^{k,j}_{r+1}(du)
 }
 { \int\nolimits_{\mathcal{D}_{r+1}} \hspace*{-6mm}
 \mathcal{N}\left(Y_{r+1},f v,\sum\nolimits_{q=1}^N \hspace*{-2mm}
 v^q g_qg_q^{\top}\!\right)\hspace*{-1mm}
 \sum\nolimits_{i,\ell=1}^N \hspace*{-2mm}
 \widehat{X}_{t_r}^i
 \rho^{i,\ell}_{r+1}(dv)
  },  \\ 
  j = \overline{1,N}\,.
 \label{eq:filt_1}
 \end{multline}
 Таким образом, доказана следующая
 
 %\smallskip
 
 \noindent
 \textbf{Лемма~1.}
\textit{Если для системы наблюдения}~(\ref{eq:obsys_1}) 
\textit{верны условия~(а) и~(б), то оценка~$\widehat{X}_t$ оптимальной фильтрации 
определяется формулой}~(\ref{eq:in_cond}) 
\textit{при $t\hm=0$, рекуррентным соотношением}~(\ref{eq:filt_1})~---
\textit{в~моменты~$t_{r+1}$ получения наблюдений~$Y_{r+1}$ 
и~формулой}~(\ref{eq:bw_obs})~--- 
\textit{в~промежутках времени между моментами получения наблюдений}.


\smallskip
 

 
 Несмотря на компактную запись~(\ref{eq:filt_1}), их прямая численная реализация 
 ресурсозатратна. Во-пер\-вых, в~(\ref{eq:filt_1}) требуется вычислять 
 распределения мас\-штаб\-но-сдви\-го\-вых смесей многомерных нормальных 
 распределений, что является трудоемкой\linebreak процедурой. Во-вто\-рых, 
 распределения~$\rho^{k,j}_{r+1}$ вре-\linebreak мени пребывания представляют собой 
 сумму\linebreak бесконечного ряда, слагаемые которого вычис\-ляются с~помощью 
 некоторой рекуррентной про\-це\-дуры~\cite{S_00}. В-третьих, 
 распределения~$\rho^{k,j}_{r+1}$ не являются абсолютно непрерывными 
 относительно меры Ле\-бега.
 { %\looseness=1
 
 }
 
 Следующий раздел посвящен численной аппроксимации~(\ref{eq:filt_1}) и~исследованию 
 ее точностных характеристик.
 
 \section{Приближенное вычисление оценки фильтрации}
 
 Без ограничения общности будем считать, что сетка~$\{t_r\}_{r \geqslant 0}$ 
 является равномерной с~шагом~$\Delta$, т.\,е.\ $t_r \hm= r \Delta$ 
 и~$\mathcal{D}_r \hm\equiv \mathcal{D}$.
 Обозначим через~$N_{r+1}$ об-\linebreak\vspace*{-12pt}
 
 \pagebreak
 
 \noindent
 щее число скачков процесса~$X_t$, имевших место 
 на промежутке $(t_r,t_{r+1}]$. Тогда из формулы полной вероятности следует, 
 что~(\ref{eq:filt_1}) представима в~виде:
 \begin{multline}
 \widehat{X}_{t_{r+1}}^j =  \left(
 \int\limits_{\mathcal{D}} 
 \mathcal{N}\left(Y_{r+1},f u,\sum\limits_{p=1}^N u^p g_pg_p^{\top}\right)\times{}\right.\\
\left. {}\times
 \sum\limits_{h=0}^{\infty}\sum\limits_{k=1}^N \widehat{X}_{t_r}^k
 \rho^{k,j,h}_{r+1}(du)
 \right)\Bigg/ \\
 \left(
 \vphantom{\sum\limits_{m=0}^{\infty}
 \sum\limits_{i,\ell=1}^N \widehat{X}_{t_r}^i
 \rho^{i,\ell,m}_{r+1}(dv)}
 \int\limits_{\mathcal{D}} 
 \mathcal{N}\left(Y_{r+1},f v,\sum\limits_{q=1}^N v^q g_qg_q^{\top}\right)\times{}\right.\\
\left.{}\times \sum\limits_{m=0}^{\infty}
 \sum\limits_{i,\ell=1}^N \widehat{X}_{t_r}^i
 \rho^{i,\ell,m}_{r+1}(dv)
 \right)
  \,, \enskip j = \overline{1,N}\,,
  \label{eq:filt_1_1}
 \end{multline}
 где 
 $ \rho^{k,j,h}_{r+1}(du)$~--- распределение вектора 
 $\tau_{r+1}X_{t_{r+1}}^{j}\mathbf{I}_{\{h\}}(N_{r+1})$ при 
 условии $X_{t_r}\hm=e_k$, т.\,е.\ 
 для любого $\mathcal{A} \hm\in \mathcal{B}(\mathbb{R}^M)$ верно тождество
\begin{multline*}
 \mathbf{P}\left\{\omega: \; X_{t_{r+1}}(\omega)=e_{j}, \; N_{r+1} = h,\right.\\ 
\left. \tau_{r+1}(X(\omega)) \in \mathcal{A}\;|\;X_{t_r}=e_k\right\} \equiv
  \rho^{k,j,h}_{r+1}(\mathcal{A}).
\end{multline*}
В качестве аппроксимации оценок можно использовать  
 $\overline{X}_{t_{r+1}}^n \ebd 
 \mathrm{col}\,(\overline{X}_{t_{r+1}}^{n,1},\ldots,\overline{X}_{t_{r+1}}^{n,N})$, 
 полученные из~(\ref{eq:filt_1_1}) путем урезания сумм ряда в~числителе и~знаменателе:
 
 \noindent
 \begin{multline}
 \overline{X}_{t_{r+1}}^{n,j} = 
 \left(
 \int\limits_{\mathcal{D}} 
 \mathcal{N}\left(Y_{r+1},f u,\sum\limits_{p=1}^N u^p g_pg_p^{\top}\right)\times{}\right.\\[-1pt]
\left.{}\times \sum\limits_{h=0}^{n}\sum\limits_{k=1}^N \overline{X}_{t_r}^k
 \rho^{k,j,h}_{r+1}(du)
 \right)\Bigg/ \\[-1pt]
 \left(
 \int\limits_{\mathcal{D}} 
 \mathcal{N}\left(Y_{r+1},f v,\sum\limits_{q=1}^N v^q g_qg_q^{\top}\right)\times{}\right.\\[-1pt]
\left. {}\times
 \sum\limits_{m=0}^{n}
 \sum\limits_{i,\ell=1}^N \overline{X}_{t_r}^i
 \rho^{i,\ell,m}_{r+1}(dv)
  \right)\,, \enskip
   j = \overline{1,N}.
  \label{eq:filt_2}
 \end{multline}
 Ниже по формуле полной вероятности получены интегралы из~(\ref{eq:filt_2}) для 
 $h\hm=0,1,2$:
 
\vspace*{-3pt}

 \noindent
  \begin{multline*}
 \int\limits_{\mathcal{D}}  \mathcal{N}
 \left(Y_{r+1},f u,\sum\limits_{p=1}^N u^p g_pg_p^{\top}\right) 
 \rho^{k,j,0}_{r+1}(du) = {}\\[-1pt]
 {}=
 \delta_{kj}\mathcal{N}\left(Y_{r+1},\Delta f^j,\Delta g_jg_j^{\top}\right)
 e^{\lambda_{jj}\Delta};
 %\label{eq:h0}
\\[-1pt]
 \int\limits_{\mathcal{D}}  \mathcal{N}\left(
 Y_{r+1},f u,\sum\limits_{p=1}^N u^p g_pg_p^{\top}\right) 
 \rho^{k,j,1}_{r+1}(du) ={} 
 \end{multline*}
 
 \noindent
 \begin{multline}
 \hspace*{-6.7pt}{}=\left(1-\delta_{kj}\right)\lambda_{kj}e^{\lambda_{jj}\Delta}
\! \int\limits_0^{\Delta}\!
 e^{(\lambda_{kk}-\lambda_{jj})u^k}
 \mathcal{N}\left(Y_{r+1},u^kf^k +{}\right.\hspace*{-0.28818pt}\\[-1pt]
\hspace*{-3mm}\left. {}+ \left(\Delta - u^k\right)f^j, u^k g_kg_k^{\top}+
 \left(\Delta-u^k\right)g_jg_j^{\top}\right)\,du^k;
 \label{eq:h1}
 \end{multline}
 
 \vspace*{-12pt}
 
 \noindent
 \begin{multline}
 \int\limits_D \mathcal{N}\left( 
Y_{r+1},f u,\sum\limits_{p=1}^N u^p g_pg_p^{\top}\right)du ={}\\[-1pt]
{}=
\sum\limits_{\substack{{\ell:\ell \neq k,}\\ {\ell \neq j}}}
 \lambda_{k\ell}\lambda_{\ell j} e^{\lambda_{jj}\Delta}\times {}\\[-1pt] 
 {}\times
 \int\limits_0^{\Delta} \int\limits_0^{\Delta-u^k} \!
e^{(\lambda_{kk}-\lambda_{\ell\ell})u^k+(\lambda_{\ell\ell}-
 \lambda_{jj})u^{\ell}}\times{} \\[-1pt] 
{}  \times
 \mathcal{N}\left(Y_{r+1},u^k f^k+u^{\ell}f^{\ell}+\left(
 \Delta-u^k-u^{\ell} \right)f^j,\right.\\[-1pt]
 \hspace*{-1mm}\left.
 u^k g_kg_k^{\top}+u^{\ell}g_{\ell}g_{\ell}^{\top}+\left(
 \Delta-u^k-u^{\ell} \right)
 g_jg_j^{\top}
 \right) du^{\ell}du^{k}, \!\!
  \label{eq:h2}
 \end{multline} 
 
\vspace*{-2pt}
 
 \noindent
  где  $\delta_{ij}$~--- символ Кронекера. Интегралы для $h\hm>2$ также могут 
  быть получены в~явном виде, однако их сложность резко возрастает.
 

   Так как система~(\ref{eq:obsys_1}) является автономной, то в~качестве локальной 
   характеристики бли\-зости~$\{\overline{X}_{t_r}\}$ 
   к~$\{\widehat{X}_{t_r}\}$ может быть выбрана величина
   
\noindent
 \begin{multline*}
 \overline{\sigma}(\pi) \ebd {\sf E}\left\{
 \|\widehat{X}_{t_{1}}(\pi, Y_{1}) - \overline{X}_{t_{1}}
 \left(\pi,Y_{1}\right)\|_{1}\right\} = {}\\
 {}=
 \sum\limits_{j=1}^N{\sf E}
 \left\{\left\vert \widehat{X}^j_{t_{1}}\left(\pi, Y_{1}\right) - \overline{X}^{n,j}_{t_{1}}
 \left(\pi,Y_{1}\right)\right\vert\right\}.
 %\label{eq:prec_1}
 \end{multline*}
 При этом начальное распределение $\pi \hm\in \mathcal{D}_1 \ebd $\linebreak $\ebd
 \{\mathrm{col}\,(\pi^1,\ldots,\pi^N):\;\pi^j > 0$, 
 $\sum\nolimits_{j=1}^N\pi^j\hm=1\}$ является начальным условием применения 
 одного шага рекурсии~(\ref{eq:filt_1}) или~(\ref{eq:filt_2}) для вычисления 
 оценки~$\widehat{X}_{t_{1}}$
   или~$\overline{X}_{t_{1}}$ соответственно. Фактически, 
 характеристика~$\overline{\sigma}(\pi)$ определяет, насколько сильно 
 рекурсивные схемы~(\ref{eq:filt_1}) и~(\ref{eq:filt_2}) разойдутся за 
 один шаг, стартуя из общей точки~$\pi$.
 
 Рекуррентные схемы~(\ref{eq:filt_1}) и~(\ref{eq:filt_2}), примененные~$r$~раз, 
 позволяют вычислить оценки~$\widehat{X}_{t_r}$ и~$\overline{X}_{t_r}$ 
 в~точке~$t_r$. В~качестве характеристики точности глобальной аппроксимации в~этом 
 случае естественно рассмотреть величину
 
 \vspace*{-2pt}
 
 \noindent
 \begin{equation*}
 \overline{\Sigma}_{t_r}(\pi) \ebd {\sf E}
 \left\{\|\widehat{X}_{t_{r}} - \overline{X}_{t_{r}}\|_{1}\right\} = 
 \!\sum\limits_{j=1}^N\!{\sf E}
 \left\{\left\vert \widehat{X}^j_{t_{r}} - 
 \overline{X}^{n,j}_{t_{r}}\right\vert \right\}.
% \label{eq:prec_2}
 \end{equation*}
 
 Следующее утверждение определяет оценки локальной и~глобальной 
 точности схемы аппроксимации~(\ref{eq:filt_2}).
 
 %\smallskip
 
 \noindent
 \textbf{Теорема~1.}\
\textit{Выполняются неравенства} 

%\vspace*{-2pt}

\noindent
 \begin{equation}
 \sup_{\pi \in \mathcal{D}_1} \overline{\sigma}(\pi) 
 \leqslant 2 \fr{(\overline{\lambda}\Delta)^{n+1}}{(n+1)!}\,;
 \label{eq:prec_loc}
\end{equation}

\noindent
\begin{align}
  \sup\limits_{\pi \in \mathcal{D}_1} \overline{\Sigma}_{t_r}(\pi)
   &\leqslant 2r \fr{(\overline{\lambda}\Delta)^{n+1}}{(n+1)!} +{}\notag\\[-0.5pt]
   &\hspace*{-20mm}{}+
  r(r-1)\left(
  \fr{(\overline{\lambda}\Delta)^{n+1}}{(n+1)!}
  \right)^2
  \left(
  1-\fr{(\overline{\lambda}\Delta)^{n+1}}{(n+1)!}
  \right)^{r-2},
 \label{eq:prec_glob}
 \end{align}
 
 \vspace*{-2pt}
 
 \noindent
 \textit{где} $\overline{\lambda} \ebd \max_{1 \leqslant j \leqslant N}|\lambda_{jj}|$.


%\smallskip

 Доказательство теоремы~1 приведено в~приложении.
 
 Данное утверждение представляет полезные оценки точности. Во-пер\-вых, 
 они являются равномерными по начальному распределению $\pi \hm\in \mathcal{D}_1$. 
 Во-вто\-рых, оценки носят универсальный, а~не асимптотический характер. Это 
 существенно в~практических задачах оценивания по дискретизованным 
 наблюдениям с~физическими или алгоритмическими ограничениями на шаг 
 по времени. Например, в~случае наблюдаемого процесса восстановления в~силу 
 центральной предельной теоремы для процессов восстановления~\cite{B_80} его
  приращения можно рассматривать как гауссовские случайные величины. 
  Однако данная аппроксимация обладает удовлетворительной точностью 
  только в~случае, когда шаг дискретизации по времени достаточно большой. 
 %
 В-третьих, неравенство~(\ref{eq:prec_glob}) позволяет получить порядок 
 аппроксимации при $\Delta \hm\to 0$. Зафиксируем момент времени $t\hm=T$ и~рассмотрим 
 характеристику $\sup\nolimits_{\pi \in \mathcal{D}_1} 
 \overline{\Sigma}_{T}(\pi)$ при $r\hm={T}/{\Delta}$ и~$\Delta \hm\to 0$. 
 Как только~$\Delta$ становится настолько мало, что 
 $\max\left({(\overline{\lambda}\Delta)^{n+1}}/{(n+1)!}, 
 \Delta ({T\lambda^{n+1}}/{(n+1)!})\right)\hm< 1$, из~(\ref{eq:prec_glob}) 
 следует неравенство
  %\begin{equation}
  $\sup\nolimits_{\pi \in \mathcal{D}_1} \overline{\Sigma}_{T}(\pi) 
  \hm\leqslant  ({3\overline{\lambda}^{n+1}}/{(n+1)!}) T\Delta^n.$
 %\label{eq:prec_asympt}
 %\end{equation}
 Это значит, что с~ростом времени~$T$ 
 ошибка аппроксимации копится пропорционально~$T$ и~при этом порядок точности 
 по~$\Delta$ равен~$n$.
 
 %\vspace*{-7pt}
 
  \section{Заключение}
  
  \vspace*{-4pt}
 
  В работе решена задача оценивания состояния однородного МСП по 
  дискретизованным наблюдениям. Получено аналитическое решение и~его 
  чис\-лен\-ные аппроксимации. Локальные и~глобальные показатели точ\-ности этих 
  приближений в~статье так\-же пред\-став\-ле\-ны. Примечательно, что  част\-ный случай 
  аппроксимаций~(\ref{eq:filt_2}) при $n\hm=0$ и~$\Lambda\hm=0$ был ранее 
  пред\-став\-лен в~\cite{B_17_1,B_17_2} для решения задачи байесовской классификации 
  случайного вектора по непрерывным наблюдениям с~мультипликативными шумами. 
 % 
Алгоритм оптимальной фильт\-ра\-ции и~его субоптимальные версии могут 
рас\-смат\-ри\-вать\-ся в~качестве основы чис\-лен\-ной реализации обобщения фильт\-ра 
Вонэма для сис\-тем с~мультипликативными шумами в~наблюдениях. 
Однако для их непосредственного использования необходимо решить 
следующие проб\-ле\-мы. Во-пер\-вых, в~(\ref{eq:h1}) и~(\ref{eq:h2}) присутствуют
 многомерные интегралы. Следует выяснить, какую результирующую погрешность 
 будут вносить ошибки их вы\-чис\-ле\-ния. Во-вто\-рых, представляется интересным 
 определить характеристики точ\-ности оптимальной фильт\-ра\-ции по дискретизованным 
 наблюдениям по отношению к~оптимальной фильт\-ра\-ции по непрерывным наблюдениям: 
 каков порядок точ\-ности по шагу временной дискретизации~$\Delta$? Для случая 
 вы\-чис\-ле\-ния классического фильт\-ра Вонэма с~по\-мощью алгоритма Эй\-ле\-ра--Ма\-ру\-ямы 
 подобный результат известен: порядок глобальной ошибки равен~${1}/{2}$. 
 Перечисленные задачи являются предметом дальнейших исследований.
 
 
  \vspace*{-10pt}
 
{\small
\subsection*{\raggedleft Приложение} 

\vspace*{-2pt}


\noindent
Д\,о\,к\,а\,з\,а\,т\,е\,л\,ь\,с\,т\,в\,о\ \ теоремы~1.\ \ Введем следующие 
обозначения для случайных величин и~мат\-риц, составленных из них:
\begin{align*}
\xi^{ji}(\ell)&\ebd 
\sum\limits_{h=0}^n \int\limits_{\mathcal{D}} 
 \mathcal{N}\left(Y_{\ell},f u,\sum\limits_{p=1}^N u^p g_pg_p^{\top}\right)
 \rho^{j,i,h}_{1}(du)\,; \\
  \theta^{ji}(\ell)&\ebd 
\sum\limits_{h=n+1}^{\infty} \int\limits_{\mathcal{D}} 
 \mathcal{N}\left(Y_{\ell},f u,\sum\limits_{p=1}^N u^p g_pg_p^{\top}\right)
 \rho^{j,i,h}_{1}(du)\,;
\\
 \xi(\ell)&\ebd \|\xi^{ji}(\ell)\|_{j,i=\overline{1,N}}\,,\quad 
 \Xi(r) \ebd \xi(r) \xi(r-1)\cdots \xi(1)\,;
 \\
 \theta(\ell)&\ebd \|\theta^{ji}(\ell)\|_{j,i=\overline{1,N}}\,, \quad 
 \Theta(r) \ebd \theta(r) \theta(r-1)\cdots \theta(1)\,.
%\label{eq:not_1}
\end{align*}
 
 Рекуррентные формулы~(\ref{eq:filt_1}) и~(\ref{eq:filt_2}) можно записать в~явной 
 форме
 
 
\noindent
\begin{align*}
 \widehat{X}_{t_r}& = \left( \mathbf{1}\left(\Xi(r) + 
 \Theta(r)\right)\pi\right)^{-1} \left(\Xi(r) + \Theta(r)\right)\pi\,;
\\
 \overline{X}_{t_r} &= \left( \mathbf{1}\Xi(r)\pi\right)^{-1} \Xi(r) \pi,
\end{align*}

\vspace*{-2pt}

\noindent
где $\mathbf{1} \ebd (1,\ldots,1)$~--- век\-тор-стро\-ка 
подходящей раз\-мер\-ности, составленная из единиц.

%Далее для краткости записи зависимость от~$r$ в~обозначениях~$\Xi(r)$ 
%и~$\Theta(r)$ будет опущена. 
Верна следующая цепочка неравенств:

 \vspace*{-3pt}

\noindent
\begin{multline}
\overline{\Sigma}_{t_r}(\pi)=%
%\me{}{\left\| 
%\widehat{X}_{t_r}(\pi, Y_1,\ldots,Y_r) - \overline{X}_{t_r}(\pi, Y_1,\ldots,Y_r)
%\right\|_1} =\\=
{\sf E}\left\{\left\| 
\fr{1}{\mathbf{1}\left(\Xi(r) + \Theta(r)\right)\pi} \left(\Xi(r) +{}\right.\right.\right.\\[-1pt]
\left.\left.\left.{}+ \Theta(r)\right)\pi
- \fr{1}{\mathbf{1}\Xi(r)\pi}\,\Xi(r) \pi
\right\|_1\right\} ={} \\[-1pt]
{}=
{\sf E}\left\{\fr{1}{\mathbf{1}\left(\Xi(r) + \Theta(r)\right)\pi \mathbf{1}\Xi(r)\pi}
\left\|
 \mathbf{1}\Xi(r) \pi \Theta(r)\pi -{}\right.\right.\\[-1pt]
\left.\left. {}- \mathbf{1}\Theta(r)\pi \Xi(r) \pi
 \right\|_1
 \vphantom{\fr{1}{\mathbf{1}\left(\Xi(r) + \Theta(r)\right)\pi \mathbf{1}\Xi(r)\pi}}
\right\} \leqslant {}\\[-1pt]
{}\leqslant 
{\sf E}\left\{\fr{1}{\mathbf{1}\left(\Xi(r) + \Theta(r)\right)\pi \mathbf{1}\Xi(r)\pi}
\left(
\mathbf{1}\Xi(r)\pi \| \Theta(r)\pi \|_1 +{}\right.\right.\\[-1pt]
\left.\left.{}+ \mathbf{1}\Theta(r)\pi 
\|
\Xi(r) \pi
\|_1
\right)
 \vphantom{\fr{1}{\mathbf{1}\left(\Xi(r) + \Theta(r)\right)\pi \mathbf{1}\Xi(r)\pi}}
\right\} ={}\\[-1pt]
{}=
2\,{\sf E}\left\{\fr{1}{\mathbf{1}\left(\Xi(r) + \Theta(r)\right)\pi}\mathbf{1}\Theta(r)\pi 
\right\}.
\label{eq:ineq_1}
\end{multline}

 
 \noindent
 Рассмотрим случайные события $a_{\ell} \ebd \{\omega \in \Omega: 
 N_{\ell}(\omega) \hm\leqslant n\}$, $\ell \hm= \overline{1,r}$, и~$A_r \ebd \{
 \omega\hm \in \Omega: \max_{1 \leqslant {\ell} \leqslant r}N_{\ell}(\omega) 
 \hm\leqslant n
 \}\hm=\prod\nolimits_{\ell=1}^r a_{\ell}$ и~оценку 
 $
 \widetilde{X}_{t_r}(\pi, Y_1,\ldots,Y_r)\ebd$\linebreak $\ebd
 {\sf E}\left\{X_{t_r}(\omega)\mathbf{I}_{A_r}(\omega)|\mathcal{O}_r\right\}.
 $
 Используя введенные выше обозначе\-ния и~абстрактный вариант формулы Байеса, 
 получаем, что
 
 \noindent
\begin{align}
\widetilde{X}_{t_r}& = \fr{1}{{\mathbf{1}\left(\Xi(r) + 
 \Theta(r)\right)\pi}}\,\Xi(r)\pi\,;\notag
 \\
\widehat{X}_{t_r} - \widetilde{X}_{t_r} &=
{\sf E}\left\{X_{t_r}(\omega)\mathbf{I}_{\overline{A}_r}(\omega)|\mathcal{O}_r\right\} ={}\notag\\[-1pt]
&\hspace*{17mm}{}= 
\fr{1}{\mathbf{1}\left(\Xi(r) + \Theta(r)\right)\pi}\Theta(r)\pi\,. 
\label{eq:eq_2}
 \end{align}
 Из (\ref{eq:ineq_1}) и~(\ref{eq:eq_2}) для $r\hm=1$ следует, что
 
 \vspace*{-4pt}
 
 \noindent
 \begin{multline}
 \overline{\sigma}(\pi) \leqslant 2\,{\sf E}
 \left\{\|{\sf E}\left\{X_{t_1}(\omega)\mathbf{I}_{\overline{a}_1}(\omega)|\mathcal{O}_1
 \right\}\|_1
 \right\} ={}\\[-1.5pt]
 {}=
 2\,{\sf E}\left\{\sum\limits_{n=1}^N {\sf E}
 \left\{X^n_{t_1}(\omega)\mathbf{I}_{\overline{a}_1}
 (\omega)|\mathcal{O}_1\right\}\right\} ={} \\[-2pt] 
 {}=
  2\,{\sf E}\left\{{\sf E}\left\{\mathbf{I}_{\overline{a}_1}(\omega)|\mathcal{O}_1
  \right\}\right\} =
   2 \mathbf{P}\left\{\overline{a}_1(\omega)\right\}.
\label{eq:ineq_3}
\end{multline}

 \vspace*{-2pt}
 
 \noindent
 Процесс $N^X_t$ общего числа скачков состояния~$X_t$ является считающим, и~его
  квадратическая характеристика равна 
  
\vspace*{-2pt}
  
  \noindent
 $$
 \langle N^X, N^X\rangle_t = - \int\limits_0^t \sum\limits_{n=1}^N \lambda_{nn} X_s^n\,ds\,,
 $$
 поэтому искомая вероятность ограничена сверху:
 $$ 
 \mathbf{P}\left\{\overline{a}_1(\omega)\right\} \leqslant 
 e^{-\overline{\lambda}\Delta}\sum\limits_{k=n+1}^{\infty} 
 \fr{(\overline{\lambda}\Delta)^{k}}{k!} <
 \fr{(\overline{\lambda}\Delta)^{n+1}}{(n+1)!}.
 $$
 
  \vspace*{-2pt}
  
  \noindent
 Из последнего неравенства и~(\ref{eq:ineq_3}) следует, что  для любого 
 начального распределения~$\pi$ выполняется неравенство $\overline{\sigma}(\pi)  
 \hm< 2({(\overline{\lambda}\Delta)^{n+1}}/{(n+1)!})$, т.\,е.\ 
 локальная оценка~(\ref{eq:prec_loc}) верна.
 
 С помощью марковского свойства пары $(X_t, N^X_t)$ и~последнего 
 неравенства можно оценить сверху вероятность 
 $\mathbf{P}\left\{\overline{A}_r(\omega)\right\}$:
 
  \vspace*{-2pt}
 
 \noindent
 \begin{multline*}
 \mathbf{P}\left\{\overline{A}_r(\omega)\right\} \leqslant 1 - \left(
 1- \fr{(\overline{\lambda}\Delta)^{n+1}}{(n+1)!}
 \right)^r \leqslant r \fr{(\overline{\lambda}\Delta)^{n+1}}{(n+1)!} + {}\\[-1pt]
 {}+\left|
 \sum\limits_{k=2}^r C_r^k \left(-\fr{(\overline{\lambda}\Delta)^{n+1}}{(n+1)!}
 \right)^k
 \right| \leqslant
 r \fr{(\overline{\lambda}\Delta)^{n+1}}{(n+1)!} +{}\\[-1pt]
 {}+\fr{r(r-1)}{2}
 \left(
 \fr{(\overline{\lambda}\Delta)^{n+1}}{(n+1)!}
 \right)^2
 \left(
 1-\fr{(\overline{\lambda}\Delta)^{n+1}}{(n+1)!}
 \right)^{r-2},
 \end{multline*} 
 из чего следует истинность глобальной оценки~(\ref{eq:prec_glob}).
Теорема~1 доказана.

}

%\vspace*{-12pt}

{\small\frenchspacing
 {%\baselineskip=10.8pt
 \addcontentsline{toc}{section}{References}
 \begin{thebibliography}{99}

\bibitem{Won_65}
\Au{Wonham W.} 
Some applications of stochastic differential equations to optimal
  nonlinear filtering~//
SIAM~J.~Control, 1965. Vol.~2. P.~347--369. 

\bibitem{KP_92}
\Au{Kloeden P., Platen E.} Numerical solution of stochastic
differential equations.~--- Berlin: Springer, 1992.~636~p.

\bibitem{YZL_04}
\Au{Yin G., Zhang Q., Liu Y.} 
Discrete-time approximation of Wonham filters~//
J.~Control Theory Applications, 2004. Iss.~2. P.~1--10.

\bibitem{PR_10}
\Au{Platen E., Rendek R.}
Quasi-exact approximation of hidden Markov chain filters~//
Communicat.~Stoch.~Analys., 2010. Vol.~4. Iss.~1. P.~129--142.

\bibitem{B_18}
\Au{Борисов А.} Фильтрация Вонэма по наблюдениям с~мультипликативными шумами~// 
Автоматика и~телемеханика, 2018.
№~1. C.~52--65. 
 
  \bibitem{BSh_85} %6
\Au{Бертсекас Д., Шрив С.} Стохастическое оптимальное управление. 
Случай дискретного времени~/ Пер. с~англ.~--- М.: Наука, 1985.~280~c.
(\Au{Betsekas~D.\,P., Shreve~S.\,E.} Stochastic optimal control:
The discrete-time case.~--- Orlando, FL, USA:
Academic Press Inc., 1978. 323~p.)

  \bibitem{ZhSh_95} %7
\Au{Жакод Ж., Ширяев А.} Предельные теоремы для случайных процессов,~I.~/
Пер. с~англ.~--- 
М.: Физматлит, 1995.~544~c.
(\Au{Jacod~J., Shiryaev~A.} Limit theorems for stochastic processes.~---
Berlin: Springer, 2003. 664~p.)

\bibitem{S_00}
\Au{Sericola B.} Occupation times in Markov processes~//
Commun. Stat. Stochastic Models, 2000. Vol.~16. Iss.~5. P.~479--510. 

  \bibitem{B_80}
\Au{Боровков А.} Асимптотические методы в~тео\-рии массового обслуживания.~--- 
М.: Физматлит, 1995.~384~c.

  \bibitem{B_17_1}
\Au{Борисов А.} Классификация по непрерывным наблюдениям с~мультипликативными шумами.~I. 
Формулы байесовской оценки~// Информатика и~её применения, 2017. Т.~11. Вып.~1. C.~11--19.
doi: 10.14357/19922264170102.

  \bibitem{B_17_2}
\Au{Борисов А.} Классификация по непрерывным наблюдениям с~мультипликативными 
шумами.~II. Алгоритм численной реализации оценки~// Информатика и~её 
применения, 2017. Т.~11. Вып.~2. C.~33--41.
doi: 10.14357/19922264170204.

 \end{thebibliography}

 }
 }

\end{multicols}

\vspace*{-4pt}

\hfill{\small\textit{Поступила в~редакцию 10.07.18}}

\vspace*{6pt}

%\pagebreak

%\newpage

%\vspace*{-28pt}

\hrule

\vspace*{2pt}

\hrule

%\vspace*{-2pt}

\def\tit{FILTERING OF~MARKOV JUMP PROCESSES\\ BY~DISCRETIZED OBSERVATIONS}

\def\titkol{Filtering of Markov jump processes by discretized observations}

\def\aut{A.\,V.~Borisov}

\def\autkol{A.\,V.~Borisov}

\titel{\tit}{\aut}{\autkol}{\titkol}

\vspace*{-11pt}


\noindent
Institute of Informatics Problems, Federal Research Center ``Computer Science 
and Control'' of the Russian Academy of Sciences, 44-2~Vavilov Str., Moscow 
119333, Russian Federation


\def\leftfootline{\small{\textbf{\thepage}
\hfill INFORMATIKA I EE PRIMENENIYA~--- INFORMATICS AND
APPLICATIONS\ \ \ 2018\ \ \ volume~12\ \ \ issue\ 3}
}%
 \def\rightfootline{\small{INFORMATIKA I EE PRIMENENIYA~---
INFORMATICS AND APPLICATIONS\ \ \ 2018\ \ \ volume~12\ \ \ issue\ 3
\hfill \textbf{\thepage}}}

\vspace*{6pt}



\Abste{The article is devoted to a~solution of the optimal filtering problem 
of a~homogenous Markov
jump process state. The available observations represent 
time increments of the integral transformations of the Markov\linebreak\vspace*{-12pt}}

\Abstend{state corrupted by 
Wiener processes. The noise intensity is also state-dependent. At the instant of 
the consecutive
observation obtaining, the optimal estimate is calculated recursively 
as a~function of previous estimate and the new observation, meanwhile between 
observations the filtering estimate is a simple forecast by virtue of the Kolmogorov 
differential system. The recursion is rather expensive because of  need to calculate 
the integrals, which are the location-scale mixtures of Gaussians. The mixing 
distributions represent the occupation of the state in each of possible values 
during the mid-observation intervals. The paper contains numerically cheaper 
approximations, based on the restriction of the state transitions number between 
the observations. Both the local and global characteristics of approximation 
accuracy are obtained as functions of the dynamics parameters, mid-observation 
interval length, and upper bound of transitions number.}

\KWE{Markov jump process; optimal filtering; multiplicative observation noises; 
stochastic differential equation; numerical approximation}




\DOI{10.14357/19922264180316}

%\vspace*{-14pt}

\Ack
\noindent
The work was supported in part by the Russian Foundation
for Basic Research (Project No.\,16-07-00677).



%\vspace*{6pt}

  \begin{multicols}{2}

\renewcommand{\bibname}{\protect\rmfamily References}
%\renewcommand{\bibname}{\large\protect\rm References}

{\small\frenchspacing
 {%\baselineskip=10.8pt
 \addcontentsline{toc}{section}{References}
 \begin{thebibliography}{99}
\bibitem{Won_65-1}
\Aue{Wonham, W.} 1965.
Some applications of stochastic differential equations to optimal
  nonlinear filtering.
\textit{SIAM~J.~Control} 2:347--369. 

\bibitem{KP_92-1}
\Aue{Kloeden,~P., and E.~Platen.} 1992. \textit{Numerical solution of stochastic
differential equations.} Berlin: Springer. 636~p.

\bibitem{YZL_04-1}
\Aue{Yin,~G., Q.~Zhang, and Y.~Liu.} 2004.
Discrete-time approximation of Wonham filters.
\textit{J.~Control Theory Applications} 2:1--10.

\bibitem{PR_10-1}
\Aue{Platen, E., and R.~Rendek.} 2010.
Quasi-exact approximation of hidden Markov chain filters.
\textit{Communicat. Stoch. Analys.} 4(1):129--142.

\bibitem{B_18-1}
\Aue{Borisov, A.} 2018. Wonham filtering by observations
with multiplicative noises. \textit{Automat.~Rem.~Contr.} 79(1):39--50.  
doi: 10.1134/ S0005117918010046.
 
  \bibitem{BSh_85-1}
\Aue{Bertsekas, D., and S.~Shreve.} 1996.
\textit{Stochastic optimal control: The discrete-time case}.
Nashua, NH: Athena Scientific. 330~p.
  
  \bibitem{ZhSh_95-1}
  \Aue{Jacod,~J., and A.~Shiryaev.} 2003.
\textit{Limit theorems for stochastic processes.}
Berlin: Springer. 664~p.

\bibitem{S_00-1}
\Aue{Sericola, B.}
2000. Occupation times in Markov processes.
\textit{Commun. Stat.} 16(5):479--510. 

  \bibitem{B_80-1}
\Aue{Borovkov, A.} 1984.
 \textit{Asymptotic methods in queueing theory}. 
 Hoboken, NJ: Wiley-Blackwell.~304~p.

  \bibitem{B_17_1-1}
  \Aue{Borisov, A.} 2017. 
  Klassifikatsiya po ne\-pre\-ryv\-nym nablyu\-de\-miyam s~mul'tiplikativnymi shumami. I. 
  Formuly bayesov\-skoy otsenki [Classification by continuous-time observations
in multiplicative noise. I.~Formulae for Bayesian 
estimate]. \textit{Informatika i~ee Primeneniya~--- Inform.~Appl.}
11(1):11--19. doi: 10.14357/19922264170102.

  \bibitem{B_17_2-1}
\Aue{Borisov, A.} 2017. Klassifikatsiya po nepreryvnym nablyudemiyam 
s~mul'tiplikativnymi summami. II.~Formuly bayesovskoy otsenki 
[Classification by continuous-time observations
in multiplicative noise. II.~Numerical algorithm].
\textit{Informatika i~ee Primeneniya~--- Inform.~Appl.}
11(2):33--41. doi: 10.14357/19922264170204.

\end{thebibliography}

 }
 }

\end{multicols}

\vspace*{-6pt}

\hfill{\small\textit{Received July 10, 2018}}

%\pagebreak

%\vspace*{-18pt}

\Contrl

\noindent
\textbf{Borisov Andrey V.} (b.\ 1965)~--- 
Doctor of Science in physics and mathematics, principal scientist, Institute of
Informatics Problems, Federal Research Center ``Computer Science and Control''
 of the Russian Academy of
Sciences, 44-2 Vavilov Str., Moscow 119333, Russian Federation; 
\mbox{aborisov@frccsc.ru}
\label{end\stat}

\renewcommand{\bibname}{\protect\rm Литература}         %3
\def\stat{bosov+stef}

\def\tit{УПРАВЛЕНИЕ ВЫХОДОМ СТОХАСТИЧЕСКОЙ ДИФФЕРЕНЦИАЛЬНОЙ СИСТЕМЫ 
ПО~КВАДРАТИЧНОМУ КРИТЕРИЮ. I.~ОПТИМАЛЬНОЕ РЕШЕНИЕ МЕТОДОМ 
ДИНАМИЧЕСКОГО ПРОГРАММИРОВАНИЯ$^*$}

\def\titkol{Управление выходом стохастической дифференциальной системы 
по~квадратичному критерию. I}
%.~Оптимальное решение методом 
%динамического программирования}

\def\aut{А.\,В.~Босов$^1$, А.\,И.~Стефанович$^2$}

\def\autkol{А.\,В.~Босов, А.\,И.~Стефанович}

\titel{\tit}{\aut}{\autkol}{\titkol}

\index{Босов А.\,В.}
\index{Стефанович А.\,И.}
\index{Bosov A.\,V.}
\index{Stefanovich A.\,I.}




{\renewcommand{\thefootnote}{\fnsymbol{footnote}} \footnotetext[1]
{Работа выполнена при частичной поддержке РФФИ (проект 16-07-00677).}}


\renewcommand{\thefootnote}{\arabic{footnote}}
\footnotetext[1]{Институт проблем информатики Федерального исследовательского центра <<Информатика 
и~управление>> Российской академии наук, \mbox{AVBosov@ipiran.ru}}
\footnotetext[2]{Институт проблем информатики Федерального исследовательского центра <<Информатика 
и~управление>> Российской академии наук, \mbox{AStefanovich@frccsc.ru}}

%\vspace*{8pt}



  
  \Abst{Решается задача оптимального управления для диффузионного процесса 
Ито и~линейного управ\-ля\-емо\-го выхода. Рассматриваемая постановка близка 
к~классической ли\-ней\-но-квад\-ра\-тич\-ной гауссовской задаче управления 
(linear-quadratic Gaussian (LQG) control). Отличия состоят в~том, что состояние описывается нелинейным 
дифференциальным уравнение Ито $dy_t\hm= A_t(y_t) \,dt\hm+ \Sigma_t(y_t)\,dv_t$ 
и~не зависит от управ\-ле\-ния~$u_t$, оптимизации подлежит управ\-ля\-емый 
линейный выход $dz_t\hm= a_t y_t\,dt\hm+ b_t z_t \,dt\hm+ c_t u_t \,dt\hm+ \sigma_t\, 
dw_t$. Дополнительные обобщения внесены в~квад\-ра\-тич\-ный критерий качества 
с~целью воз\-мож\-ности постановки таких задач, как отслеживание выходом 
состояния или управ\-ле\-ни\-ем~--- линейной комбинации состояния и~выхода. Для 
решения используется метод динамического программирования. Функцию 
Беллмана позволяет найти предположение о~ее структуре вида $V_t(y,z)\hm= 
\alpha_t z^2\hm+ \beta_t(y)z \hm+\gamma_t(y)$. Решение дают три 
дифференциальных уравнения для коэффициентов~$\alpha_t$, $\beta_t(y)$ 
и~$\gamma_t(y)$. Эти уравнения со\-став\-ля\-ют оптимальное решение 
рас\-смат\-ри\-ва\-емой задачи.}
  
  \KW{стохастическое дифференциальное уравнение; оптимальное управ\-ле\-ние; 
динамическое программирование; функция Беллмана; уравнение Риккати; 
линейные уравнения параболического типа}

\DOI{10.14357/19922264180314}
  
%\vspace*{4pt}


\vskip 10pt plus 9pt minus 6pt

\thispagestyle{headings}

\begin{multicols}{2}

\label{st\stat}

\section{Введение}

     Ключевые результаты в~области оптимизации стохастических 
динамических систем, со\-став\-ля\-ющие классическую теорию управления, 
получены более~40~лет назад (такова работа~[1] в~отношении задачи 
управ\-ле\-ния ли\-ней\-но-гаус\-сов\-ски\-ми стохастическими сис\-те\-ма\-ми по 
квад\-ра\-тич\-но\-му критерию). К~классической тео\-рии следует относить 
линейные модели стохастических сис\-тем и~квадратичный критерий качества. 
Это исходный базис, на котором основано множество успешно 
исследованных и~решенных задач стохастического управ\-ле\-ния 
и~оптимизации. 

Дальнейшее развитие~--- это новые модели и~критерии, но 
прежде всего это новые методы: от тео\-рии линейных регуляторов, метода 
динамического программирования и~принципа максимума к~адаптивному 
и~минимаксному подходу, импульсному управ\-ле\-нию и~т.\,д. Множество 
инноваций как в~час\-ти моделей, так и~в~час\-ти математического аппарата, 
имевших мес\-то в~по\-сле\-ду\-ющие годы, существенно обогатили тео\-рию 
управ\-ле\-ния. Но и~до настоящего времени линейные модели и~квадратичный 
критерий, несмотря на всю справедливую критику в~отношении их 
аде\-кват\-ности и~гиб\-кости, сохраняют исследовательский интерес и~находят 
современные области приложения.
     
     Не претендуя на сколь\-ко-ни\-будь полное обосно\-ва\-ние последнего 
тезиса, приведем несколько примеров, показавшихся наиболее ин\-те\-рес\-ными. 

Так, в~[2] решается ли\-ней\-но-квад\-ра\-тич\-ная за\-да\-ча в~игровой 
постановке с~запаздыванием. В~близ\-кой по модели работе~[3] задача 
управ\-ле\-ния ставится в~терминах $H_\infty$-ро\-баст\-ности. Точнее \mbox{называть} 
эту тематику $H_2/H_\infty$-управ\-ле\-ни\-ем, и~работ по этой теме очень 
много. Аккуратности ради следует уточнить, что под линейными 
понимаются модели с~мультипликативными по состоянию воз\-му\-ще\-ниями. 

Совсем другой класс моделей, особо популярных в~по\-след\-ние годы, 
составляют скачкообразные процессы. Например, линейные уравнения 
в~сочетании с~пуассоновскими скачками в~[4] используются в~моделях, 
описывающих различные показатели функционирования сетевых протоколов 
передачи данных транспортного уровня. Телекоммуникации представляют 
в~последние годы самый популярный прикладной материал для 
исследований, работ по этой проб\-ле\-ма\-ти\-ке множество, математические 
техники привлекаются самые разные и~самые современные, но и~линейным 
моделям место находится. Еще один любопытный пример исследования 
скачкообразного процесса и~оптимизации на основе квад\-ра\-тич\-но\-го критерия 
можно найти в~[5] применительно к~задаче инвестирования на финансовом 
рынке. Наконец, упомянем еще работу~[6], подводящую итог исследований 
в~отношении классической детерминированной  
ли\-ней\-но-квад\-ра\-тич\-ной задачи с~использованием техники матричных 
неравенств.
     
     В данной работе также эксплуатируются привлекательные свойства 
линейных моделей и~квад\-ра\-тич\-но\-го критерия, причем в~стохастической 
постановке. На\-прав\-ле\-ни\-ем для обобщения \mbox{выбрана} модель динамики 
сис\-те\-мы: основные усилия на\-прав\-ле\-ны на то, чтобы сделать ее нелинейной. 
Кроме того, пред\-став\-лен\-ная постановка может рас\-смат\-ри\-вать\-ся и~как 
обобщение ранее решенной задачи в~дискретном времени~[7, 8] на время 
непрерывное. В~упомянутых работах помимо собственно модельной 
постановки важна еще и~привлекаемая прикладная об\-ласть~--- 
функционирование сложных программных сис\-тем. Результатов, 
ориентированных непосредственно на такие приложения, к~настоящему 
времени пренебрежимо мало, поэтому~[7, 8]~--- это еще и~прикладное 
обоснование рас\-смат\-ри\-ва\-емой далее задачи.
     
     Оптимизируемая динамическая сис\-те\-ма описывается двумя 
уравнениями. Состояние задается нелинейным стохастическим 
дифференциальным уравнением Ито, не содержащим управ\-ля\-емой 
переменной. Возмущение здесь описывается стандартным винеровским 
процессом, накладываются простые условия существования 
и~един\-ст\-вен\-ности решения. Поскольку состояние не управ\-ля\-ет\-ся, то уместно 
его интерпретировать как слож\-ное внешнее возмущение. Вторая 
переменная~--- управ\-ля\-емый выход~--- задается линейным стохастическим 
дифференциальным уравнением. Цель оптимизации выхода формируется 
квадратичным критерием общего вида. Формальная постановка задачи 
приведена в~сле\-ду\-ющем разделе.
     
     Для решения задачи используется метод динамического 
программирования, решается уравнение Беллмана~[9]. Соответственно, 
в~результате получаются аналитические выражения и~для оптимального 
управ\-ле\-ния, и~для значения функционала качества. Технически 
традиционный, стандартный подход к~задаче обременен, пожалуй, 
единственной проблемой~--- поиском верного пред\-став\-ле\-ния структуры 
функции Беллмана. Справиться с~этой проблемой в~большей степени удается 
за счет результата, полученного при решении дискретного по времени 
аналога рассматриваемой постановки~\cite{8-bos}. Конечные соотношения 
для оптимального решения, как и~во всех подобных задачах, включая 
классическую ли\-ней\-но-квад\-ра\-тич\-ную, содержат решения 
определенных дифференциальных уравнений (обыкновенных и~в~частных 
производных). Вывод этих уравнений и~со\-став\-ля\-ет содержание первой час\-ти 
данной работы. Во второй части будет обсуждаться их приближенное 
чис\-лен\-ное решение и~компьютерные эксперименты.
     
     Кратко обозначим основные положения, при\-вле\-ка\-емые далее 
к~решению задачи, следуя в~основном обозначениям 
и~терминологии~\cite{9-bos}, а~именно: будем рассматривать задачу 
оптимального управления в~стохастической динамической сис\-те\-ме по полной 
информации, применяя метод динамического программирования. В~качестве 
целевого функционала, опре\-де\-ля\-юще\-го качество управ\-ле\-ния $U_0^T\hm= \{ 
u_t,\ 0\leq t\leq T\}$, выступает
     \begin{equation}
     J\left(U_0^T\right)={\sf E}\left\{ \int\limits_0^T L_t \left(x_t, u_t\right)\,dt+ 
l\left(x_T\right)\right\}\,.
     \label{e1-bos}
     \end{equation}
Здесь ${\sf E}\{\cdot\}$~--- оператор математического ожидания; $x_t$~--- 
случайный процесс, описываемый стохастическим дифференциальным 
уравнением Ито
     \begin{equation}
     dx_t=m_t\left( x_t, u_t\right) dt+ \sigma_t\left( x_t\right)dW_t\,,\enskip 
x_0=X\,,
     \label{e2-bos}
     \end{equation}
где $W_t$~--- стандартный винеровский процесс подходящей раз\-мер\-ности; 
$X$~--- случайный вектор.

     $U_0^T$ будем выбирать из класса допустимых неупреждающих (по 
отношению к~$W_t$) управлений~\cite{9-bos}. Соответственно, 
относительно функций сноса и~диффузии~$m_t$ и~$\sigma_t$  
в~(\ref{e2-bos}) будем предполагать выполненными ка\-кие-ли\-бо условия 
существования сильного решения для заданного до\-пус\-ти\-мо\-го управ\-ле\-ния. 
Например, для управ\-ле\-ния с~обратной связью $u_t\hm= u_t(x_t)$ будем 
считать, что $m_t(x,u_t(x))$ и~$\sigma_t(x)$ удовлетворяют условию 
линейного рос\-та и~локальному условию Липшица по~$x$ равномерно 
по~$t$ (т.\,е.\ условиям Ито).
     
     Для поиска оптимального управления, минимизирующего $J(U_0^T)$, 
рас\-смат\-ри\-ва\-ет\-ся функция Беллмана
     \begin{equation}
     V_t(x)=\left.\mathop{\mathrm{inf}}\limits_{U_t^T} {\sf E} \left\{ \int\limits_t^T 
L_t \left( x_t, u_t\right)\,dt+l\left( x_T\right) \right\vert \mathcal{F}_t^x\right\}\,,
     \label{e3-bos}
     \end{equation}
где $\mathcal{F}_t^x$~--- $\sigma$-ал\-геб\-ра, по\-рож\-ден\-ная~$x_\tau$, 
$0\hm\leq \tau\hm\leq t$, ${\sf E}\{\cdot\vert \mathcal{F}\}$~--- оператор условного 
математического ожидания относительно~$\mathcal{F}$. Соответственно, 
в~качестве достаточного условия оп\-ти\-маль\-ности воспользуемся уравнением 
динамического программирования
\begin{multline}
\fr{\partial V_t(x)}{\partial t} +\fr{1}{2}\sum\limits^n_{i,j=1} \sigma^2_{t_{ij}}
\fr{\partial^2 V_t(x)}{\partial x_i \partial x_j}+{}\\
{}+\min\limits_u\left[  
\sum\limits^n_{i=1} m_{t_i} \fr{\partial V_t(x)}{\partial x_i} + L_t(x,u)\right] 
=0\,,\\
V_T(x)=l(x)\,,
\label{e4-bos}
\end{multline}
где $m_{t_i}$~--- $i$-й элемент век\-тор-функ\-ции~$m_t(x,u)$; 
$\sigma^2_{t_{ij}} \hm= \sum\nolimits^m_{k=1} 
\sigma_{t_{ik}}\sigma_{t_{ki}}$, $\sigma_{t_{ij}}$~--- $i$-й по строке, $j$-й 
по столб\-цу элемент мат\-рич\-ной функции~$\sigma_t(x)$; $n$ и~$m$~--- 
размерности~$x_t$ и~$W_t$ соответственно.

     Традиционно в~рамках применения метода динамического 
программирования будем предполагать, что функции~$L_t$, $l$, $m_t$ 
и~$\sigma_t$ обеспечивают существование хотя бы одного решения 
уравнения~(\ref{e4-bos}), а~следовательно, и~оптимального 
управления~$u_t^*$, $0\hm\leq t\hm\leq T$, до\-став\-ля\-юще\-го минимум 
целевому функционалу~(\ref{e1-bos}). Задача оптимизации далее получается 
путем указания конкретных выражений для~$L_t$, $l$, $m_t$ и~$\sigma_t$.

\section{Постановка задачи управления выходом}

     Рассматриваемые далее случайные функции будут предполагаться 
скалярными. Такое упрощение позволит разгрузить выкладки и~итоговые 
выражения от не самых существенных деталей.
     
     Рассмотрим стохастическую дифференциальную сис\-те\-му, со\-сто\-яние 
которой представляет диффузи\-он\-ный процесс~$y_t$, описываемый 
нелинейным стохастическим дифференциальным уравнением Ито
     \begin{equation}
     dy_t=A_t\left( y_t\right) dt +\Sigma_t \left( y_t\right) dv_t\,,\enskip 
y_0=Y\,,
     \label{e5-bos}
     \end{equation}
где $v_t$~--- стандартный (одномерный) винеровский процесс; $Y$~--- 
случайная величина с~конечным вторым моментом; функции~$A_t$ 
и~$\Sigma_t$ удовлетворяют условиям Ито:
\begin{equation*}
\left\vert A_t(y)\right\vert +\left\vert \Sigma_t(y)\right\vert \leq C(1+\vert y\vert )\ 
\mbox{для\ всех } 0\leq t\leq T\,;
\end{equation*}

\vspace*{-12pt}

\noindent
\begin{multline*}
\hspace*{-2.10051pt}\left\vert A_t\left(y_1\right) -A_t \left( y_2\right) \right\vert +\left\vert 
\Sigma_t\left( y_1\right) -\Sigma_t \left(y_2\right)\right\vert \leq
C\left\vert y_1-y_2\right\vert\\
 \mbox{для\ всех\ } 0\leq t\leq T\ \mbox{и } 
y_1,y_2\in \mathbb{R}^1\,,
\end{multline*}
обеспечивающим существование единственного сильного (потраекторного) 
решения уравнения.
     
     Будем считать, что~$y_t$ описывает состояние некоторой 
динамической системы. Соответственно, поведение этой сис\-те\-мы опишем 
выходом, линейно связанным с~со\-сто\-янием:
     \begin{equation}
     dz_t=a_t y_t \,dt+ b_t z_t \,dt+ c_t u_t \,dt+\sigma_t \,dw_t\,,\enskip
     z_0=Z\,.
     \label{e6-bos}
     \end{equation}
Здесь $w_t$~--- не зависящий от~$v_t$, $Y$ и~$Z$ стандартный (одномерный) 
винеровский процесс; $Z$~--- случайная величина с~конечным вторым 
моментом; $u_t$~--- допустимое неупреждающее управ\-ле\-ние, качество 
которого определяется целевым функционалом следующего вида:
\begin{multline}
\!\hspace*{-3.98538pt}J\left( U_0^T\right) ={\sf E}\left\{ \int\limits_0^T \!\left( S_t\left( s_ty_t-g_t z_t -h_t 
u_t\right)^2 +G_t z_t^2+{}\right.\right.\\
\left.\left.{}+ H_t u_t^2
\vphantom{S_t\left( s_ty_t-g_t z_t -h_t 
u_t\right)^2}
\right) dt+S_T\left( s_T y_T -g_T 
z_T\right)^2+G_T z_T^2
\vphantom{\int\limits_0^T}\right\}\,,
\label{e7-bos}
\end{multline}
где $S_t$, $G_t$ и~$H_t$~--- неотрицательные функции\linebreak
$0\hm\leq t\hm\leq T$. 
Такой критерий отражает физический смысл задачи распределения ресурсов 
со\-глас\-но аналогичной~(\ref{e5-bos})--(\ref{e7-bos}) задаче для дис\-крет\-но\-го 
времени, рас\-смот\-рен\-ной в~\cite{7-bos}. В~част\-ности,  
функци\-онал~(\ref{e7-bos}) поз\-во\-ля\-ет ставить задачи отслеживания
 выходом 
со\-сто\-яния сис\-те\-мы, используя сла\-га\-емое $(y_t\hm- z_t)^2$, или 
управлением~--- линейной комбинации со\-сто\-яния и~выхода, сла\-га\-емое типа\linebreak 
$(y_t\hm+ z_t\hm- u_t)^2$. Поскольку задача формулируется 
в~предположении наличия пол\-ной информации о~со\-сто\-янии~$y_t$ 
и~выходе~$z_t$ (соответствующую $\sigma$-ал\-геб\-ру 
обозначим~$\mathcal{F}_t^{y,z}$), то допустимое управ\-ле\-ние ищется 
в~классе~$\mathcal{F}_t^{y,z}$-из\-ме\-ри\-мых неупреждающих функций 
(и,~как будет показано далее, оказывается управ\-ле\-ни\-ем с~обратной связью).

     Функции~$a_t$, $b_t$, $c_t$ и~$\sigma_t$ будем предполагать 
ограниченными: $\vert a_t\vert \hm+ \vert b_t\vert \hm+\vert c_t\vert \hm+ \vert 
\sigma_t \vert \hm\leq C$ для всех $0\hm\leq t\hm\leq T$, процесс  
управления~--- допустимым не\-упреж\-да\-ющим~\cite{9-bos}, обеспечивая, 
таким образом, существование сильного решения урав\-не\-ния~(\ref{e6-bos}) 
для любого допустимого управ\-ления.
     
     Задачу составляет поиск~$u_t^*$~--- допустимого управ\-ле\-ния, 
доставляющего минимум квад\-ра\-тич\-но\-му функционалу~$J(U_0^T)$.
      
     Поставленная задача очевидным образом формулируется в~терминах 
введенных выше в~(\ref{e1-bos})--(\ref{e3-bos}) обозначений, а~именно: 
     требуется обозначить
     \begin{gather*}
      x_t=\begin{pmatrix}
     y_t\\ z_t\end{pmatrix};\quad  m_t(x_t, u_t)=\begin{pmatrix}
     A_t(y_t)\\ a_t y_t +b_t z_t +c_t u_t\end{pmatrix};\\
     \sigma_t(x_t)= \begin{pmatrix}
     \Sigma_t(y_t)& 0\\
     0& \sigma_t\end{pmatrix};\quad W_t=\begin{pmatrix}
     v_t \\ w_t\end{pmatrix}
     %     \label{e8-bos}
     \end{gather*}
для записи уравнения со\-сто\-яния типа~(\ref{e2-bos}) и
\begin{align*}
L_t(x,u)&= L_t(y,z,u) ={}\\
&\hspace*{3mm}{}=S_t\left( s_t y-g_t z -h_t u\right)^2 +G_t z^2 +H_t  u^2\,;\\
l(x)&= l(y,z) =S_T \left( S_T y-g_T z\right)^2 +G_T z^2
%\label{e9-bos}
\end{align*}
для записи целевого функционала в~виде~(\ref{e1-bos}).

     Функция Беллмана~(\ref{e3-bos}) принимает вид 
     $V_t(x)\hm= V_t(y,z)$. Для записи со\-от\-вет\-ст\-ву\-юще\-го~(\ref{e4-bos}) 
уравнения Беллмана для~$V_t(y,z)$ заметим, что
     $$
     \left( \sigma^2_{t_{ij}}\right)_{i,j=1,2}= \begin{pmatrix}
     \Sigma_t^2(y) & 0\\
     0 & \sigma_t^2\end{pmatrix}\,.
     $$
     
     С~учетом перечисленных обозначений урав\-не\-ние динамического 
программирования~(\ref{e4-bos}) принимает вид:
     \begin{multline}
     \fr{\partial V_t(y,z)}{\partial t} +\fr{1}{2}\left( \Sigma_t^2(y) \fr{\partial^2 
V_t(y,z)} {\partial y^2}+\sigma_t^2\fr{\partial^2 V_t(y,z)} {\partial 
z^2}\right)+{}\\
    {}+\min\limits_u\! \left[ A_t(y) \fr{\partial V_t(y,z)}{\partial y}+\left( a_t 
y+b_t z+c_t u\right) \fr{\partial V_t(y,z)}{\partial z} +{}\right.\hspace*{-3pt}\\
\left.{}+ S_t\left( s_t y-g_t z-h_t 
u\right)^2+G_t z^2+H_t u^2
     \vphantom{\fr{\partial V_t(y,z)}{\partial y}}\right] =0\,,\\
     V_T(y,z)=S_T\left( s_T y-g_T z\right)^2+G_T z^2\,.
     \label{e10-bos}
     \end{multline}
     Это и~есть то самое уравнение, которое требуется решить: 
существование решения данного урав\-не\-ния суть достаточное условие 
оптимальности; оптимальное управ\-ле\-ние при этом~--- точ\-ка минимума 
со\-от\-вет\-ст\-ву\-юще\-го сла\-га\-емого.
     
\section{Динамическое программирование и~оптимальное 
управление}

     В рассматриваемой постановке линейность\linebreak выхода и~квадратичность 
критерия дают те же преимущества, что и~в~классической  
ли\-ней\-но-квад\-ра\-тич\-ной задаче управ\-ле\-ния~\cite{1-bos}, а~именно: 
позволяют сразу определить вид оптимального управ\-ле\-ния и~фактические 
условия его существования. Действительно, со\-хра\-няя в~(\ref{e10-bos}) под 
знаком $\min\nolimits_u$ только члены, зависящие от~$u$, получаем
     \begin{multline*}
     \fr{\partial V_t(y,z)}{\partial t} +\fr{1}{2}\left( \Sigma_t^2(y) \fr{\partial^2 
V_t(y,z)} {\partial y^2}+\sigma_t^2\fr{\partial^2 V_t(y,z)} {\partial 
z^2}\right)+{}\\
     {}+A_t(y)\fr{\partial V_t(y,z)}{\partial y}+\left( a_t y+b_t z\right) 
\fr{\partial V_t(y,z)}{\partial z}+{}\\
{}+S_t\left( s_t y-g_t z\right)^2 +G_t z^2+{}
\end{multline*}

\noindent
\begin{multline*}
     {}+\min\limits_u \left[ \left( c_t \fr{\partial V_t(y,z)}{\partial z}-2S_t \left( 
s_t y-g_t z\right) h_t\right)u +{}\right.\\
\left.{}+\left( S_t h_t^2+H_t\right) u^2
\vphantom{\fr{\partial V_t(y,z)}{\partial z}}
\right]=0\,,
     %\label{e11-bos}
     \end{multline*}
откуда в~предположении $S_t h_t^2\hm+ H_t\hm>0$ следует, что существует 
оптимальное управ\-ле\-ние, которое определяется равенством
\begin{multline}
u_t^* = u_t^*(y,z)=-\fr{1}{2}\left( S_t h_t^2 +H_t\right)^{-1} \left( c_t 
\fr{\partial V_t(y,z)}{\partial z}-{}\right.\\
\left.{}-2S_t\left( s_t y-g_t z\right) h_t
\vphantom{\fr{\partial V_t(y,z)}{\partial z}}
\right)
\label{e12-bos}
\end{multline}
и доставляет минимум соответствующему сла\-га\-емо\-му в~урав\-не\-нии Беллмана, 
равный
$-\left( S_t h_t^2\hm+\right.$\linebreak
$\left.{}+H_t\right)^{-1} \left( c_t 
{\partial V_t(y,z)}/{\partial 
z}\hm-2S_t\left( s_t y \hm-g_t z\right) h_t \right)^2/4.
$ 
     
     Отметим, что, как и~в~классической ли\-ней\-но-квад\-ра\-тич\-ной 
задаче, управ\-ле\-ние из класса до\-пус\-ти\-мых не\-упреж\-да\-ющих получилось 
управ\-ле\-ни\-ем с~обратной связью.
     
     Таким образом, функция Беллмана описывается сле\-ду\-ющим 
дифференциальным уравнением:
     \begin{multline}
     \fr{\partial V_t(y,z)}{\partial t} +\fr{1}{2}\left( \Sigma_t^2(y) \fr{\partial^2 
V_t(y,z)} {\partial y^2}+\sigma_t^2\fr{\partial^2 V_t(y,z)} {\partial 
z^2}\right)+{}\\
     {}+ A_t(y) \fr{\partial V_t(y,z)}{\partial y}+\left( a_t y+b_t z\right) 
\fr{\partial V_t(y,z)}{\partial z}+{}\\
{}+ S_t \left( s_t y- g_t z\right)^2 +G_t z^2-
 \fr{1}{4}\left( S_t h_t^2+H_t\right)^{-1}\times{}\\
 {}\times \left( c_t \fr{\partial V_t(y,z)} 
{\partial z}-2S_t\left( s_t y -g_t z\right) h_t \right)^2=0\,.
     \label{e13-bos}
     \end{multline}
     
     Возводя в~квадрат по\-след\-нее сла\-га\-емое в~(\ref{e13-bos}), перепишем 
его в~виде:
     \begin{multline}
     \fr{\partial V_t(y,z)}{\partial t} +\fr{1}{2}\left( \Sigma_t^2(y) \fr{\partial^2 
V_t(y,z)} {\partial y^2}+\sigma_t^2\fr{\partial^2 V_t(y,z)} {\partial 
z^2}\!\right)+{}\\
{}+A_t(y) \fr{\partial V_t(y,z)}{\partial y}
+ \left( 
\vphantom{\left( S_t h_t^2 +H_t\right)^{-1}}
a_t y+b_t z+{}\right.\\
\left.{}+\left( S_t h_t^2 +H_t\right)^{-1}
 c_t S_t \left( s_t y-g_t z\right) h_t
\right) 
     \fr{\partial V_t(y,z)}{\partial z}+{}\\
     {}+\left( S_t-\left( S_t h_t^2 +H_t\right)^{-1} S_t^2 h_t^2\right)\left( s_t y -
g_t z\right)^2+{}\\
     \!\!{}+
     G_t z^2 -\fr{1}{4}\left( S_t h_t^2+H_t\right)^{-1}\! c_t^2
     \left(\fr{\partial V_t(y,z)}{\partial z}\right)^{\!2}=0\,.\!\!
     \label{e14-bos}
     \end{multline}
     
     Рассматривая полученное уравнение, заметим, что его решение может 
быть пред\-став\-ле\-но в~виде:
   \begin{equation}
     V_t(y,z)= \alpha_t z^2+\beta_t(y) z +\gamma_t(y)\,,
     \label{e15-bos}
     \end{equation}
т.\,е.\ будем искать решение при дополнительном предположении 
о~квад\-ра\-тич\-ности функции Белл\-ма\-на по переменной~$z$, и~сведем, таким 
образом, поиск оптимального решения к~уравнениям относительно функций 
$\alpha_t$, $\beta_t(y)$ и~$\gamma_t(y)$. Отметим сразу, что явный вид 
функции~$\gamma_t(y)$ для реализации оптимального управ\-ле\-ния не 
требуется, однако далее будет предложен вариант вы\-чис\-ле\-ния и~этой 
функции, что пред\-став\-ля\-ет\-ся небесполезным, поскольку позволит выполнять 
расчет минимума целевого функционала. Источником для 
предложения~(\ref{e15-bos}) является уже упоминавшаяся аналогичная 
задача для случая дис\-крет\-но\-го времени~\cite{7-bos, 8-bos}. В~той задаче 
выражение для функции Беллмана получается формально без 
дополнительных усилий. При этом форма~(\ref{e15-bos}) обнаруживается 
как свойство оптимального решения. В~рассматриваемом случае 
непрерывного времени~(\ref{e15-bos}) постулируется, а~пра\-виль\-ность 
постулата под\-тверж\-да\-ет\-ся далее ре\-зуль\-ти\-ру\-ющи\-ми уравнениями 
для~$\alpha_t$, $\beta_t(y)$ и~$\gamma_t(y)$ Кроме того, данное 
предположение пред\-став\-ля\-ет\-ся вы\-те\-ка\-ющим из линейной структуры задачи 
в~отношении переменной~$z$, в~част\-ности, тем фактом, что такой вид 
функции Беллмана обеспечивает линейность оптимального 
управ\-ле\-ния~(\ref{e12-bos}) по~$z$.

     Граничное условие при выбранном предположении~(\ref{e15-bos}) 
принимает вид:

\noindent
     \begin{multline*}
     V_T(y,z)= S_T\left( s_T y- g_T z\right)^2+G_T z^2 ={}\\[-0.5pt]
     {}=\alpha_T z^2 
+\beta_T(y) z +\gamma_T(y)\,,
    \end{multline*}
т.\,е.

\noindent
\begin{align*}
\alpha_T&= S_T g_T^2 +G_T\,;\\[-0.5pt]
\beta_T(y)&=-2S_T s_T g_T y\,;\\[-0.5pt]
\gamma_T(y)&=S_T s_T^2 y^2\,.
%\label{e16-bos}
\end{align*}
          При этом само оптимальное управ\-ле\-ние, определенное 
выражением~(\ref{e12-bos}), оказывается управ\-ле\-ни\-ем с~обратной связью 
по~$y_t$ и~$z_t$:

\noindent
     \begin{multline}
     u_t^*=u_t^*(y,z) ={}\\[-0.5pt]
     {}=
     -\fr{1}{2}\left( S_t h_t^2 +H_t\right)^{-1}
     \left( c_t \left( 2\alpha_t z +\beta_t(y)\right) +{}\right.\\[-0.5pt]
    \left. {}+2S_t\left( s_t y-g_t z\right) 
h_t\right)\,.
     \label{e17-bos}
     \end{multline}
          Подставляем $V_t(y,z)\hm= \alpha_t z^2 \hm+ \beta_t(y) 
z\hm+\gamma_t(y)$ в~(\ref{e14-bos}):

\noindent
     \begin{multline*}
     \fr{\partial \alpha_t}{\partial t}\, z^2 +
     \fr{\partial \beta_t(y)}{\partial t}\,z +
     \fr{\partial \gamma_t(y)}{\partial t}+{}\\[-0.5pt]
     {}+\fr{1}{2}\left( \Sigma_t^2(y) \left( 
\fr{\partial^2\beta_t(y)}{\partial y^2}\,z +\fr{\partial^2 \gamma_t(y)}{\partial 
y^2}\right) +2\sigma_t^2\alpha_t\right)+{}\\[-0.5pt]
 {}+A_t(y)\left(\fr{\partial \beta_t(y)}{\partial y}\,z + \fr{\partial 
\gamma_t(y)}{\partial y}\right) +{}\\[-0.5pt]
\hspace*{-0.22987pt}{}+\left( a_t y+b_t z+\left( S_t h_t^2 +H_t\right)^{-1} c_t S_t \left( s_t y-
g_t z\right) h_t\right)\times{}
\end{multline*}

\noindent
\begin{multline*}
         {}\times \left( 2\alpha_t z+\beta_t(y)\right)+{}\\
     {}+\left( S_t-\left( S_t h_t^2 +H_t\right)^{-1} S_t^2 h_t^2\right)\left( s_t y-
g_t z\right)^2+{}\\
     {}+ G_t z^2 -\fr{1}{4}\left( S_t h_t^2 +H_t\right)^{-1} c_t^2 \left( 
2\alpha_t z+\beta_t(y)\right)^2=0\,.
     \end{multline*}
          Далее выделяем слагаемые при~$z^2$, $z$ и~$z^0$
          
          \noindent
     \begin{multline*}
     \fr{\partial \alpha_t}{\partial t}\, z^2 +\fr{\partial \beta_t(y)}{\partial t}\,z +
     \fr{\partial \gamma_t(y)}{\partial 
t}+\fr{1}{2}\,\Sigma_t^2(y)\fr{\partial^2\beta_t(y)}{\partial y^2}\,z+ {}\\
{}+
\fr{1}{2}\,\Sigma_t^2(y)\fr{\partial^2\gamma_t(y)}{\partial 
y^2}+\sigma_t^2\alpha_t+A_t(y)\fr{\partial \beta_t(y)}{\partial y}\,z +{}\\
{}+A_t(y) \fr{\partial 
\gamma_t(y)}{\partial y}+{}\\
{}+ 2\alpha_t \left( b_t -\left( S_t h_t^2+H_t\right)^{-1} c_t 
S_t h_t g_t \right)z^2+{}\\
     {}+
     \left( 2\alpha_t\left( \alpha_t+\left( S_t h_t^2+H_t\right)^{-1} c_t S_t h_t 
s_t\right)y +{}\right.\\
\left.{}+\beta_t(y) \left( b_t-\left( S_t h_t^2+H_t\right)^{-1} c_t S_t h_t 
g_t\right) \right) z+{}\\
     {}+\beta_t(y)\left( a_t +\left( S_t h_t^2+H_t\right)^{-1} c_t S_t h_t s_t\right) 
y+{}\\
{}+ \left( S_t -\left( S_t h_t^2+H_t\right)^{-1} S_t^2 h_t^2\right) g_t^2 z^2-{}\\
     {}- 2\left( S_t -\left( S_t h_t^2+H_t\right)^{-1} S_t^2 h_t^2\right) s_t g_t yz 
+{}\\
{}+
     \left( S_t-\left( S_t h_t^2+H_t\right)^{-1} S_t^2 h_t^2\right) s_t^2 y^2+{}\\
     {}+G_t z^2 -\left( S_t h_t^2 +H_t\right)^{-1} c_t^2 \alpha_t^2 z^2 -{}\\
     {}-\left( 
S_t h_t^2+H_t\right)^{-1} c_t^2 \alpha_t \beta_t(y) z-{}\\
{}-
\fr{1}{4}\left( S_t h_t^2+H_t\right)^{-1}  c_t^2 \beta_t^2(y)=0\,,
     \end{multline*}
группируем их и~получаем сле\-ду\-ющие уравнения:
\begin{itemize}
\item  для~$\alpha_t$:

\noindent
\begin{multline}
\fr{\partial\alpha_t}{\partial t}+2\alpha_t\left( b_t-\left( S_t h_t^2+H_t\right)^{-1} c_t 
S_t h_t g_t\right)+{}\\
{}+ \left( S_t- \left( S_t h_t^2+H_t\right)^{-1} S_t^2 h_t^2\right) 
g_t^2+G_t-{}\\
\hspace*{-8mm}{}-\left( S_t h_t^2+H_t\right)^{-1} c_t^2 \alpha_t^2 =0\,,\enskip \alpha_T=S_T 
g_t^2+G_T\,;\!\!
\label{e18-bos}
\end{multline}
\item для $\beta_t$:

\noindent
\begin{multline}
\fr{\partial\beta_t(y)}{\partial 
t}+\fr{1}{2}\,\Sigma_t^2(y)\fr{\partial^2\beta_t(y)}{\partial y^2} 
+A_t(y)\fr{\partial \beta_t(y)}{\partial y}+{}\\
{}+ 2\alpha_t\left( a_t +\left( S_t h_t^2+H_t\right)^{-1} c_t S_t h_t s_t\right) y+{}\\
{}+
\beta_t(y)\left( b_t -\left( S_t h_t^2 +H_t\right)^{-1} c_t S_t h_t g_t\right)-{}\\
{}-2\left( S_t-\left( S_t h_t^2+H_t\right)^{-1} S_t^2 h_t^2\right) s_t g_t y-{}
\\
{}-
\left( S_t h_t^2+H_t\right)^{-1} c_t^2 \alpha_t \beta_t(y)=0\,,\\
\beta_T(y)=-2S_T s_T g_T y\,;
\label{e19-bos}
\end{multline}
\item  для $\gamma_t$:
\begin{multline}
\hspace*{-0.8pt}\fr{\partial \gamma_t(y)}{\partial t}+\fr{1}{2}\,\Sigma_t^2(y)
\fr{\partial^2 \gamma_t(y)}{\partial y^2} +\sigma_t^2 \alpha_t +A_t(y)
\fr{\partial \gamma_t(y)}{\partial y}+{}\\
{}+ \beta_t(y)\left( a_t +\left( S_t h_t^2+H_t\right)^{-1} c_t S_t h_t s_t\right) y+{}\\
{}+
\left( S_t-\left( S_t h_t^2+H_t\right)^{-1} S_t^2 h_t^2\right)  s_t^2 y^2-{}\\
{}-\fr{1}{4}\left( S_t h_t^2+H_t\right)^{-1} c_t^2 \beta_t^2(y) =0\,,\\
\gamma_T(y)=S_T s_T^2 y^2\,.
\label{e20-bos}
\end{multline}
\end{itemize}
     
     Уравнение~(\ref{e18-bos}), легко заметить, является уравнением 
Риккати, которое в~силу сформулированного выше условия   
имеет единственное неотрицательное решение для всех $0\hm\leq t\hm\leq T$. 
Этот факт требует дополнительного комментария. Для получения 
уравнения~(\ref{e18-bos}) рас\-смот\-рим исходную задачу при дополнительных 
условиях $a_t\hm=0$ и~$s_t\hm=0$ для всех $0\hm\leq t\hm\leq T$. Нетрудно 
видеть, что эти условия рассматриваемую по\-ста\-нов\-ку сводят фактически 
к~классической ли\-ней\-но-квад\-ра\-тич\-ной задаче. Имеющуюся 
в~рассматриваемой формулировке чуть более общую форму целевой 
функции (принципиального значения это обобщение, конечно, не имеет) 
сведем к~классической еще одним предположением: $S_t\hm=0$ для всех 
$0\hm\leq t\hm\leq T$. Теперь уравнение для~$\alpha_t$ принимает хорошо 
известный вид:
     \begin{equation}
     \fr{\partial \alpha_t}{\partial t}+2\alpha_t b_t +G_t- H_t^{-1} c_t^2 
\alpha_t^2=0\,,\enskip \alpha_T=G_T\,.
     \label{e21-bos}
     \end{equation}

     В таком случае, как известно~\cite{10-bos}, существует единственное 
оптимальное управление~--- линейное с~обратной связью по выходу~$z_t$, 
с~коэффициентом усиления, опи\-сы\-ва\-емым уравнением  
Риккати~(\ref{e21-bos}). Именно этот результат дают  
уравнения~(\ref{e18-bos})--(\ref{e20-bos}) и~описываемая ими функция 
Беллмана~(\ref{e15-bos}), так как из $a_t\hm=0$ и~$s_t\hm=0$ немедленно 
следует, что $\beta_t(y)\hm=0$, откуда, в~свою очередь, с~учетом 
не\-за\-ви\-си\-мости решения от~$y_t$ следует, что $\gamma_t(y)\hm=\gamma_t$, 
т.\,е.\ не зависит от~$y$ и~задается уравнением: 
     $$
     \fr{\partial \gamma_t(y)}{\partial t} +\sigma^2_t \alpha_t=0\,,\enskip 
\gamma_T=0\,.
     $$ 
     Оптимальное управ\-ле\-ние при этом 
     $$
     u_t^*= -H_t^{-1} c_t \alpha_t z_t\,,
     $$
      т.\,е.\ все полностью совпадает с~известным классическим решением.
     
     С уравнениями~(\ref{e19-bos}) и~(\ref{e20-bos}) ситуация, естественно, 
обстоит сложнее. Это линейные уравнения второго порядка параболического 
типа, поскольку\linebreak
 $\Sigma_t^2(y)\hm>0$. Фактически отсутствуют 
конструктивные условия, гарантирующие существование их\linebreak
 решений 
(требовать, чтобы все фигурирующие в~уравнениях коэффициенты были 
представлены аналитическими функциями на всем пространстве значений, 
вряд ли целесообразно), поэтому далее будем предполагать, что данные 
уравнения имеют на рас\-смат\-ри\-ва\-емом интервале $0\hm\leq t\hm\leq T$ хотя 
бы одно ограниченное решение и~именно эти условия будем рас\-смат\-ри\-вать 
как достаточные условия существования оптимального решения 
рассматриваемой задачи.
     
     Таким образом, доказана следующая тео\-рема.
     
     \smallskip
     
     \noindent
     \textbf{Теорема.}\ \textit{Пусть для диффузионного 
процесса}~(\ref{e5-bos}) \textit{выполнены условия Ито, для 
     процесса}~(\ref{e6-bos})~--- \textit{ограничены коэффициенты, 
уравнения}~(\ref{e18-bos})--(\ref{e20-bos}) \textit{имеют ограниченные 
решения для $0\hm\leq t\hm\leq T$. Тогда минимум  
функционалу}~(\ref{e7-bos}) \textit{доставляет оптимальное 
управ\-ле\-ние}~(\ref{e17-bos}), \textit{где} $y\hm= y_t$; $z\hm=z_t$.
     
\section{Заключение}

     Рассмотренная задача оптимизации в~целом близка и~по модели, и~по 
критерию к~классической ли\-ней\-но-квад\-ра\-тич\-ной постановке. 
Принципиальным отличием является нелинейная модель для описания 
со\-сто\-яния динамической сис\-те\-мы, в~которой отсутствует управ\-ля\-ющее 
воздействие.\linebreak
 Такую модель наряду с~традиционной интер\-пре\-тацией  
<<со\-сто\-яние--вы\-ход>> мож\-но понимать как\linebreak модель неконтролируемого 
слож\-но\-го внешнего воздействия. Небольшое дополнительное отличие дает 
предложенная форма квад\-ра\-тич\-но\-го критерия, поз\-во\-ля\-ющая, в~част\-ности, 
ставить такие задачи, как отслеживание выходом или управ\-ле\-ни\-ем со\-сто\-яния 
сис\-те\-мы или ее выхода.
     
     Поскольку обсуждать возможности точного решения уравнений, 
определяющих оптимальное управ\-ле\-ние, не имеет смыс\-ла, наиболее 
актуальной далее является задача их приближенного чис\-лен\-но\-го решения 
и~анализа воз\-мож\-ности практической реализации. Этому посвящена вторая 
часть данной работы, пла\-ни\-ру\-емая к~выходу в~ближайшее время.

{\small\frenchspacing
 {%\baselineskip=10.8pt
 \addcontentsline{toc}{section}{References}
 \begin{thebibliography}{99}
\bibitem{1-bos}
\Au{Athans M.} Editorial on the LQG problem~// IEEE~T. Automat. Contr., 1971. Vol.~16. 
No.\,6. P.~528--552. doi: 10.1109/TAC.1971.1099845.
\bibitem{2-bos}
\Au{Wu Z.} Forward-backward stochastic differential equations, linear quadratic stochastic 
optimal control and nonzero sum differential games~// J.~Syst. Sci. Complex., 2005. Vol.~18. 
No.\,2. P.~179--192.
\bibitem{3-bos}
\Au{Chen B.\,S., Zhang~W.} Stochastic H2/H1 control with state-dependent noise~// IEEE 
T.~Automat. Contr., 2004. Vol.~49. No.\,1. P.~45--56. doi: 10.1109/TAC.2003.821400.
\bibitem{4-bos}
\Au{Bohacek S.} A~stochastic model of TCP and fair video transmission~// IEEE 
INFOCOM, 2003. Vol.~2. P.~1134--1144. doi: 10.1109/INFCOM.2003.1208950.
\bibitem{5-bos}
\Au{Домбровский В.\,В., Объедко~Т.\,Ю.} Управление с~прогнозированием системами 
с~марковскими скачками при ограничениях и~применение к~оптимизации 
инвестиционного портфеля~// Автомат. телемех., 2011. №\,5. С.~96--112. doi: 
10.1134/S0005117911050079.
\bibitem{6-bos}
\Au{Баландин Д.\,В., Коган~М.\,М.} Оптимальное линейно-квад\-ра\-тич\-ное управление: от 
матричных уравнений к~линейным матричным неравенствам~// Автомат. телемех., 2011. 
№\,11. С.~60--69. doi: 10.1134/ S0005117911110038.
\bibitem{7-bos}
\Au{Босов А.\,В.} Обобщенная задача распределения ресурсов программной системы~// 
Информатика и~её применения, 2014. Т.~8. Вып.~2. С.~39--47. doi: 
10.14357/19922264140204.
\bibitem{8-bos}
\Au{Босов А.\,В.} Управление линейным выходом дискретной стохастической системы по 
квадратичному критерию~// Изв. РАН. Теория и~системы управления, 2016. №\,3.  
С.~19--35. doi: 10.1134/S1064230716030060.
\bibitem{9-bos}
\Au{Флеминг У., Ришел~Р.} Оптимальное управление детерминированными 
и~стохастическими системами~/ Пер. с~англ.~--- М.: Мир, 1978. 316~с. 
(\Au{Fleming~W.\,H., Rishel~R.\,W.} Deterministic and stochastic optimal control.~--- New 
York, NY, USA: Springer-Verlag, 1975. 222~p.)
\bibitem{10-bos}
\Au{Девис М.\,Х.\,А.} Линейное оценивание и~стохастическое управление~/ Пер. с~англ.~--- 
М.: Наука, 1984. 206~с. (\Au{Davis~M.\,H.\,A.} Linear estimation and stochastic control.~--- 
London: Chapman and Hall, 1977. 224~p.)

 \end{thebibliography}

 }
 }

\end{multicols}

\vspace*{-6pt}

\hfill{\small\textit{Поступила в~редакцию 30.03.18}}

\vspace*{4pt}

%\newpage

%\vspace*{-24pt}

\hrule

\vspace*{2pt}

\hrule

\vspace*{-2pt}


\def\tit{STOCHASTIC DIFFERENTIAL SYSTEM OUTPUT CONTROL 
BY~THE~QUADRATIC CRITERION.~I.~DYNAMIC\\ PROGRAMMING 
OPTIMAL SOLUTION}


\def\titkol{Stochastic differential system output control 
by~the~quadratic criterion. I.~Dynamic programming 
optimal solution}

\def\aut{A.\,V.~Bosov and~A.\,I.~Stefanovich}

\def\autkol{A.\,V.~Bosov and~A.\,I.~Stefanovich}

\titel{\tit}{\aut}{\autkol}{\titkol}

\vspace*{-11pt}


\noindent
Institute of Informatics Problems, Federal Research Center ``Computer Science 
and Control'' of the Russian Academy of Sciences, 44-2~Vavilov Str., Moscow 
119333, Russian Federation


\def\leftfootline{\small{\textbf{\thepage}
\hfill INFORMATIKA I EE PRIMENENIYA~--- INFORMATICS AND
APPLICATIONS\ \ \ 2018\ \ \ volume~12\ \ \ issue\ 3}
}%
 \def\rightfootline{\small{INFORMATIKA I EE PRIMENENIYA~---
INFORMATICS AND APPLICATIONS\ \ \ 2018\ \ \ volume~12\ \ \ issue\ 3
\hfill \textbf{\thepage}}}

\vspace*{3pt}



\Abste{The problem of optimal control for the Ito diffusion 
process and a~controlled linear output is solved. The considered 
statement is close to the classical linear-quadratic Gaussian 
control  (LQG control) problem. Differences consist in the fact 
that the state is described by the nonlinear differential Ito equation  $dy_y = A_t(y_t) 
\,dt+\Sigma_t(y_t)\,dv_t$ and does not depend on the control~$u_t$, 
optimization subject is controlled linear output 
 $dz_t=a_ty_t\,dt +b_tz_t\,dt +c_t u_t\,dt +\sigma_t \,dw_t$. 
Additional generalizations are included in the quadratic 
quality criterion for the purpose of statement such problems 
as state tracking by output or a linear combination of state 
and output tracking by control. The method of dynamic programming 
is used for the solution. 
The assumption about Bellman function in the form  $V_t(y,z)= \alpha_t 
z^2+\beta_t(y) z+\gamma_t(y)$ allows one to find it. 
Three differential equations for the coefficients $\alpha_t$,  $\beta_t(y)$,
and $\gamma_t(y)$ give the solution. 
These equations constitute the optimal solution of the problem under consideration.}

\KWE{stochastic differential equation; optimal control; dynamic programming; 
Bellman function; Riccati equation; linear differential equations of parabolic type}


\DOI{10.14357/19922264180314}

\vspace*{-12pt}

\Ack
\noindent
This work was partially supported by the Russian Science Foundation (grant  
16-07-00677).



%\vspace*{6pt}

  \begin{multicols}{2}

\renewcommand{\bibname}{\protect\rmfamily References}
%\renewcommand{\bibname}{\large\protect\rm References}

{\small\frenchspacing
 {%\baselineskip=10.8pt
 \addcontentsline{toc}{section}{References}
 \begin{thebibliography}{99}
\bibitem{1-bos-1}
\Aue{Athans, M.} 1971. Editorial on the LQG problem. \textit{IEEE~T. 
Automat. Contr.} 16(6):528--552. doi: 10.1109/ TAC.1971.1099845.
\bibitem{2-bos-1}
\Aue{Wu, Z.} 2005. Forward-backward stochastic differential equations, linear 
quadratic stochastic optimal control and\linebreak\vspace*{-12pt}

\columnbreak

\noindent
 nonzero sum differential games. 
\textit{J.~Syst. Sci. Complex.} 18(2):179--192.
\bibitem{3-bos-1}
\Aue{Chen, B.\,S. and W.~Zhang.} 2004. Stochastic H2/H1 control with  
state-dependent noise. \textit{IEEE~T. Automat. Contr.} 49(1):45--56.
doi: 10.1109/TAC.2003.821400.
\bibitem{4-bos-1}
\Aue{Bohacek, S.} 2003. A~stochastic model of TCP and fair video 
transmission. \textit{IEEE INFOCOM}. 2:1134--1144.
doi: 10.1109/INFCOM.2003.1208950.
\bibitem{5-bos-1}
\Aue{Dombrovskii, V.\,V., and T.\,Yu.~Ob''edko.} 2011. Predictive control of 
systems with Markovian jumps under constraints and its application to the 
investment portfolio optimization. \textit{Automat. Rem. Contr.}  
72(5):989--1003.
\bibitem{6-bos-1}
\Aue{Balandin, D.\,V., and M.\,M.~Kogan.} 2011. Optimal linear-quadratic 
control: From matrix equations to linear matrix inequalities. \textit{Automat. 
Rem. Contr.} 72(11):2276--2284.
\bibitem{7-bos-1}
\Aue{Bosov, A.\,V.} 2014. Obobshchennaya zadacha raspredeleniya resursov 
programmnoy sistemy [The generalized problem of software system resources 
distribution]. \textit{Informatika i~ee Primeneniya~--- Inform. Appl.}  
8(2):39--47. doi: 
10.14357/19922264140204.
\bibitem{8-bos-1}
\Aue{Bosov, A.\,V.} 2016. Discrete stochastic system linear output control 
with respect to a quadratic criterion. \textit{J.~Comput. Syst. Sc. 
Int.} 55(3):349--364.
\bibitem{9-bos-1}
\Aue{Fleming, W.\,H., and R.\,W.~Rishel.} 1975. \textit{Deterministic and 
stochastic optimal control.} New York, NY: Springer-Verlag. 222~p.
\bibitem{10-bos-1}
\Aue{Davis, M.\,H.\,A.} 1977. \textit{Linear estimation and stochastic 
control.} London: Chapman and Hall. 224~p.
\end{thebibliography}

 }
 }

\end{multicols}

\vspace*{-6pt}

\hfill{\small\textit{Received March 30, 2018}}

%\pagebreak

%\vspace*{-18pt}
     
     \Contr
     
       \noindent
       \textbf{Bosov Alexey V.} (b.\ 1969)~--- Doctor of Science in technology, 
principal scientist, Institute of Informatics Problems, Federal Research 
Center ``Computer Science and Control'' of the Russian Academy of Sciences, 
44-2~Vavilov Str., Moscow 119333, Russian Federation; 
\mbox{AVBosov@ipiran.ru}
       
       \vspace*{3pt}
       
       \noindent
       \textbf{Stefanovich Alexey I.} (b.\ 1983)~--- principal specialist, 
Institute of Informatics Problems, Federal Research Center ``Computer Science 
and Control'' of the Russian Academy of Sciences, 44-2~Vavilov Str., Moscow 
119333, Russian Federation; \mbox{AStefanovich@frccsc.ru}
\label{end\stat}

\renewcommand{\bibname}{\protect\rm Литература}       

       %4
\def\stat{goncharov}

\def\tit{ВЫРАВНИВАНИЕ ДЕКАРТОВЫХ ПРОИЗВЕДЕНИЙ УПОРЯДОЧЕННЫХ МНОЖЕСТВ$^*$}

\def\titkol{Выравнивание декартовых произведений упорядоченных множеств}

\def\aut{А.\,В.~Гончаров$^1$, В.\,В.~Стрижов$^2$}

\def\autkol{А.\,В.~Гончаров, В.\,В.~Стрижов}

\titel{\tit}{\aut}{\autkol}{\titkol}

\index{Гончаров А.\,В.}
\index{Стрижов В.\,В.}
\index{Goncharov A.\,V.}
\index{Strijov V.\,V.}


{\renewcommand{\thefootnote}{\fnsymbol{footnote}} \footnotetext[1]
{Работа выполнена при частичной финансовой поддержке РФФИ 
(проекты 19-07-1155 и~19-07-00885). Настоящая статья содержит 
результаты проекта <<Статистические методы машинного обучения>>, 
выполняемого в~рамках реализации Программы Центра компетенций 
Национальной технологической инициативы <<Центр хранения 
и~анализа больших данных>>, поддерживаемого Министерством науки 
и~высшего образования Российской Федерации по договору МГУ им.\ 
М.\,В.~Ломоносова  с~Фондом поддержки проектов Национальной 
технологической инициативы от 11.12.2018 №\,13/1251/2018.}}


\renewcommand{\thefootnote}{\arabic{footnote}}
\footnotetext[1]{Московский физико-технический институт, alex.goncharov@phystech.edu}
\footnotetext[2]{Вычислительный центр им.\ А.\,А.~Дородницына Федерального исследовательского 
центра <<Информатика и~управ\-ле\-ние>> Российской академии наук; 
Московский фи\-зи\-ко-тех\-ни\-че\-ский институт, \mbox{strijov@ccas.ru}}

%\vspace*{-12pt}



\Abst{Работа посвящена исследованию метрических методов анализа 
объектов сложной структуры. Предлагается обобщить метод динамического 
выравнивания двух временных рядов на случай объектов, определенных на 
двух и~более осях времени. В~дискретном представлении такие объекты 
являются матрицами. Метод динамического выравнивания временных рядов 
обобщается как метод динамического выравнивания матриц. Предложена 
функция расстояния, устойчивая к~монотонным нелинейным деформациям 
декартова произведения двух и~более временных шкал. Определен выравнивающий 
путь между объектами. В~дальнейшем объектом называется матрица, 
в~которой строки и~столбцы соответствуют осям времени. Исследованы 
свойства предложенной функции расстояния. Для иллюстрации метода 
решаются задачи метрической классификации объектов на модельных 
данных и~данных из набора MNIST.}

\KW{функция расстояния; динамическое выравнивание; расстояние между матрицами; 
нелинейные деформации времени; про\-стран\-ст\-вен\-но-вре\-мен\-ные ряды}

\DOI{10.14357/19922264200105} 
  
\vspace*{-3pt}


\vskip 10pt plus 9pt minus 6pt

\thispagestyle{headings}

\begin{multicols}{2}

\label{st\stat}


\section{Введение}

Временн$\acute{\mbox{ы}}$е ряды представляют собой набор измерений, упорядоченных 
по оси времени. Анализ временн$\acute{\mbox{ы}}$х рядов производится при решении задач, 
связанных с~классификацией активности человека по измерениям акселерометра 
телефона, поиском паттернов в~EEG-сиг\-на\-лах (электроэнцефалограмма), 
кластеризации набора ECoG (электрокортикограмма) данных и~во многих других 
задачах~\cite{0}. Рассматриваются объекты, для которых время между измерениями 
фиксированно. В~данной работе для построения адекватной функции 
расстояния между объектами требуется учесть нелинейные деформации 
относительно оси времени: глобальные и~локальные сдвиги, растяжения 
и~сжатия~\cite{1}.

В~\cite{2} приводятся различные методы решения задач анализа 
временн$\acute{\mbox{ы}}$х рядов: классификации, детектирования паттернов, 
кластеризации и~др. В~\cite{3} описание временных рядов 
строится с~по\-мощью анализа параметров моделей, в~\cite{4} 
используется их признаковое описание, в~\cite{5} анализируется их форма. 
Комбинации этих подходов описаны в~\cite{2}.

Метрические методы находят схожие объекты в~наборе. Используются 
функции расстояния над временн$\acute{\mbox{ы}}$ми рядами: расстояние Хаусдорфа~\cite{10}, 
MODH~\cite{11}, расстояние, основанное на HMM
(hiden Markov model)~\cite{6}, евклидово расстояние 
в~исходном пространстве или в~пространстве сниженной размерности~\cite{5}, 
\mbox{LCSS} (longest common\linebreak subsequence)~\cite{7}. Показано~\cite{8}, что в~случае локальных или глобальных 
деформаций времени при решении задач, требующих анализа исходной формы 
временн$\acute{\mbox{о}}$го ряда, метод динамического выравнивания оси времени 
DTW (Dynamic Time Warping) 
превосходит другие функции расстояния~\cite{9} по качеству итогового 
решения задачи, так как при наличии смещений двух объектов относительно 
друг друга требуется выравнивать их оптимальным образом для вычисления 
расстояния между ними.

В данной работе предлагается перейти от рас\-смот\-ре\-ния объекта~$\textbf{s}(t)$, 
временн$\acute{\mbox{о}}$го ряда, к~более общему случаю $\textbf{s}(\textbf{t})$, 
в~котором компоненты вектора~$\textbf{t}$~--- оси времени. Из-за 
существенного рос\-та вы\-чис\-ли\-тель\-ной слож\-ности при увеличении чис\-ла 
осей времени предлагается рас\-смот\-реть объекты $\textbf{s}(t_1, t_2)$, 
определенные на двух осях времени. Оси времени считаются независимыми. 
В~случае единственной дискретной и~ограниченной сверху шкалы времени 
объект представим вектором фиксированной размерности. 
Аналогично объект настоящего исследования представим мат\-ри\-цей.

Вводятся ограничения на зависимости осей времени в~декартовом 
произведении для таких объектов. Определена гипотеза порождения данных: 
объекты одного класса эквивалентности получены при помощи допустимых 
преобразований, а~именно: локальных деформаций (растяжений и~сжатий) 
каждой из осей времени по отдельности. В~дискретном случае преобразование 
представимо дуп\-ли\-ци\-ро\-ва\-ни\-ем строк и~столбцов матриц. 
В~число допустимых преобразований попадают и~глобальные деформации: 
сдвиги по осям времени, представимые добавлением и~удалением крайних 
строк и~столбцов исходных матриц. Для каждой из осей времени выполняются 
свойства времени: монотонность и~непрерывность. Похожими на описанные 
свойствами обладает, например, частотный спектр сигнала, где одна ось 
определяет время, а другая~--- частоту, величину, обратную времени.


Между двумя объектами, матрицами, в~случае допустимых преобразований 
требуется определить инвариантную к~преобразованиям осей времени функцию 
расстояния, которая сможет выделить классы эквивалентности множества 
преобразованных объектов. Работа посвящена определению такой функции 
расстояния, как обобщения метода динамического выравнивания временных рядов 
DTW для матриц.

Цель данной работы~--- построение метода, основанного на динамическом 
выравнивании осей времени для матриц. Метод динамического выравнивания 
временн$\acute{\mbox{ы}}$х рядов~\cite{33} определен только для объектов с~одной осью времени, 
что делает его неприменимым для описанного случая. Однако концепции, 
используемые на каждой стадии вы\-чис\-ле\-ния оптимального выравнивания, обобщены 
на рассматриваемый случай. Работа исследует свойства предложенного 
метода и~сравнивает результаты применения метода к~задачам классификации 
изображений~\cite{12} с~результатами функции расстояния~$L_2$.

Для иллюстрации и~анализа результатов решается задача метрической 
классификации объектов (матриц низкой размерности). Используются наборы данных: 
модельные данные, которые согласуются с~выдвинутой гипотезой порождения 
данных для временн$\acute{\mbox{ы}}$х рядов, подмножество набора MNIST сниженной 
размерности и~частотный спектр сигнала.

\vspace*{-10pt}

\section{Постановка задачи построения функции расстояния}

\vspace*{-2pt}

Рассмотрим задачу построения функции расстояния между объектами. 
Функция расстояния инвариантна к~допустимым преобразованиям осей времени: 
глобальным и~локальным линейным и~нелинейным деформациям временн$\acute{\mbox{о}}$й шкалы. 
Ниже приведены две постановки задачи, с~помощью которых определены свойства 
предложенной функции расстояния, оценено ее качество и~проведено сравнение 
нескольких функций расстояния: предложенной и~$L_2$.

Первая постановка задачи использует общее свойство функций расстояния: 
объединение схожих объектов и~разделение непохожих объектов. 
Вводится определение свойства инвариантности функции расстояния к~допустимым 
преобразованиям осей времени.
Вторая постановка задачи уточняет первую и~заключается в~проведении метрической 
классификации методом ближайшего соседа.

\textbf{Постановка задачи выбора функции расстояния между двумя объектами.}
На двух временн$\acute{\mbox{ы}}$х осях заданы объекты вида 
$\textbf{A}(t_1,t_2)\hm \in \mathbb{R}^{n \times n}$. 
Функция $G_w(\textbf{A}):\mathbb{R}^{n \times n} \hm\rightarrow 
\mathbb{R}^{\hat{n} \times \hat{n}}$ задает допустимые преобразования 
исходного объекта~$\textbf{A}$: глобальные сдвиги, локальные линейные 
и~нелинейные деформации, а~именно: растяжения и~сжатия оси времени, 
сдвиги значений по оси времени. Скалярный параметр $w \hm\in \mathbb{R}^+$
 функции~$G$ фиксирует набор этих преобразований.

Допустимым элементарным преобразованием матрицы~$\textbf{A}$ назовем 
дуплицирование случайных строк и~столбцов исходной матрицы, добавление 
или удаление крайних строк и~столбцов. Допустимым преобразованием 
примем некоторую последовательность допустимых элементарных 
преобразований матрицы~$\textbf{A}$ и~обозначим как~$G_w(\textbf{A})$.

Будем называть объект~$\textbf{B} \hm\in \mathbb{R}^{\hat{n} \times \hat{n}}$ 
полученным из объекта~$\textbf{A}$ при помощи допустимых 
преобразований~$G_{\hat{w}}$, если существует $\hat{w}\hm\in \mathbb{R}^+ : 
\textbf{B} \hm= G_{\hat{w}}(\textbf{A})$.

Функцию расстояния между двумя объектами $\rho: 
\mathbb{R}^{{n} \times {n}} \times \mathbb{R}^{\hat{n} \times \hat{n}} 
\hm\rightarrow  \mathbb{R}^+$ оценим на выборке $\mathfrak{D } \hm= 
\{ \textbf{A}_i \}_{i=1}^m$ объектов вида $\textbf{A}_i \hm\in 
\mathbb{R}^{n \times n}$.

Для каждого объекта выборки~$\textbf{A}_i$ и~объекта~$\textbf{B}_j$ его 
класса эквивалентности $\{\textbf{B}_j\}_i \hm= \{  \textbf{B} 
\hm\in \mathfrak{D} | \exists w_i,w_j: G_{w_i}(\textbf{A}_i) \hm= G_{w_j}
(\textbf{B}_j)   \}$ заданы допустимые трансформации с~параметрами~$w_i$ 
и~$w_j$, такие что $G_{w_i}(\textbf{A}_i)\hm = G_{w_j}(\textbf{B}_j)$. 
Для каждого объекта выборки~$\textbf{A}_i$ и~объекта~$\textbf{C}_j$ 
из других классов эквивалентности $\{ \textbf{C}_k\}_i \hm= 
\{  \textbf{C} \hm\in \mathfrak{D} | \nexists w_i,w_k: G_{w_i}(\textbf{A}_i)
\hm = G_{w_k}(\textbf{C})   \}$ не существует таких $ w_i, w_k : G_{w_i}
(\textbf{A}_i) \hm= G_{w_k}(\textbf{C}_k)$.

Решается задача поиска функции расстояния~$\rho$, значение
 которой на паре объектов одного класса эквивалентности меньше, 
 чем на любой паре объектов из разных: для любых $i,j,k \hm\in 
 \{1,\dots,m\}$ $\quad \rho(\textbf{A}_i,\textbf{B}_j) \hm< 
 \rho(\textbf{A},\textbf{C}_k)$. Функцию расстояния, обладающую 
 таким свойством, назовем инвариантной на классах эквивалентности.

Критерием качества для функции расстояния~$\rho$ на выборке~$\mathfrak{D}$ 
примем долю объектов, для которых указанное неравенство выполняется:
$$
S_{\rho}(\mathfrak{D}) = \fr{1}{m} \sum\limits_{i=1}^m 
\prod\limits_{\{ \textbf{B}_j\}_i} 
\prod\limits_{\{ \textbf{C}_k\}_i}  
\left[  \rho(\textbf{A}_i,\textbf{B}_j) < \rho(\textbf{A}_i,\textbf{C}_k)  
 \right].
 $$
Постановка задачи выбора функции расстояния~$\rho$ 
сводится к~задаче максимизации критерия качества.

\textbf{Прикладное использование функции расстояния.}
Задана выборка $\mathfrak{D}\hm = \{(\textbf{A}_i,y_i)\}^m_{i=1}$, 
состоящая из пар объ\-ект--от\-вет. Объектами служат объекты сложной 
структуры: $\textbf{A}_i\hm \in \mathbb{R}^{n\times n}$, 
а~ответами выступают метки класса~---~$y_i\hm \in Y \hm= \{1,\ldots,E\}$, 
где $E \hm\ll m$. Выборка разделена на обучение $\mathfrak{D}_l \hm= 
\{(\textbf{A}_i,y_i)\}^{m_1}_{i=1}$ и~контроль $\mathfrak{D}_t \hm= 
\{(\textbf{A}_i,y_i)\}_{m_1}^{m_1+m_2}$.

Модель классификации~$f$ принадлежит множеству моделей метрической 
классификации 1NN, которые классифицируемому объекту ставят 
в~соответствие метку класса ближайшего объекта из обучающей 
выборки по заданной функции расстояния~$\rho$:
$$ 
\hat{y} = f(\textbf{B} | \rho) = y \argmin\limits_{i = 1,\dots, m_1} 
\rho\left(B,A_i\right)\,.
$$
Критерий качества $S$ модели~$f$ для задачи классификации~--- 
доля правильно проставленного класса на контрольной выборке:
 $$ 
 S(f | \rho) = \fr{1}{m_2}\sum\limits_{i=m_1}^{m_1+m_2} 
 \left[f(\textbf{A}_i | \rho) = y_i\right].
 $$

Требуется выбрать функцию расстояния~$\rho$ для модели 
классификации~$f:~\mathbb{R}^{n\times n} \hm\rightarrow~Y$, 
максимизируюшую критерий качества~$S$ на контрольной выборке:
\begin{equation*}
f =  \argmax\limits_{\rho \in \{\mathrm{mDTW}, L_2\}}\left(S(f | \rho)\right).
\end{equation*}

\section{Вычисление матричного расстояния mDTW}

Предлагается использовать функцию расстояния DTW, 
модифицированную для случая выравнивания двойной шкалы времени.

\smallskip

\noindent
\textbf{Определение~1.} {Даны два объекта~$\textbf{A},\textbf{B}\hm \in 
\mathbb{R}^{n\times n}$. Тензор 
невязок~$\boldsymbol{\Omega}^{n \times n \times n \times n}$~--- 
такой тензор, что его элемент~$\boldsymbol{\Omega}(i,j,k,l)$ 
равен квадрату разности между элементами~$\textbf{A}(i,j)$ и~$\textbf{B}(k,l)$:}
\begin{equation*}
\boldsymbol{\Omega}(i,j,k,l)=(\textbf{A}(i,j) - \textbf{B}(k,l))^2.
\end{equation*}

\noindent
\textbf{Определение 2.} {Путем~$\boldsymbol{\pi}$ между двумя 
объектами $\textbf{A},\textbf{B} \hm\in \mathbb{R}^{n\times n}$ 
назовем множество индексов тензора~$\boldsymbol{\Omega}$: }
$$
\boldsymbol{\pi} = \{(i,j,k,l)\},\quad i,j,k,l \in \{1,\ldots,n\} ,
$$
\textit{удовлетворяющее следующим условиям:}

{\bfseries\textit{Частичный порядок.}}
Для элементов пути~$\boldsymbol{\pi}$ с~фиксированными значениями~$i,k$ 
задан порядок: выравнивающий путь для фиксированных строк двух 
матриц упорядочен~--- $\{(i,j_r,k,l_r))\}_{r=1}^{R} \hm\subset 
\boldsymbol{\pi}$ мощностью~$R$. Аналогично для фиксированных столбцов 
с~индексами~$j,l$.

{\bfseries\textit{Граничные условия.}}
 Пусть $(i,j,k,l) \in \boldsymbol{\pi}$, тогда $(1,j,1,l) \hm\in 
 \boldsymbol{\pi}$ и~$(i,1,k,1) \hm\in \boldsymbol{\pi}$.
Путь $\boldsymbol{\pi}$ содержит элементы тензора~$\boldsymbol{\Omega}$: 
$(1,1,1,1) \hm\in \boldsymbol{\pi}$ и~$(n,n,n,n) \hm\in \boldsymbol{\pi}$.

{\bfseries\textit{Непрерывность по направлению.}}
Для упорядоченного подмножества пути $\{(i,j_r,k,l_r)\}_{r=1}^{R}
\hm\subset\boldsymbol{\pi}$ выполняется условие непрерывности:
$$
j_{r}-j_{r-1}\leq1\,,\quad l_r-l_{r-1}\leq1\,, \quad r = 2,\ldots,R\,.
$$
На~шаге пути~$\boldsymbol{\pi}$ по фиксированному направлению времени~$i,k$ 
встречаются только соседние элементы матрицы (включая соседние по диагонали). 
Аналогично для фиксированных~$j,l$.

{\bfseries\textit{Монотонность по направлению.}}
Для упорядоченного подмножества пути  $\{(i,j_r,k,l_r)\}_{r=1}^{R}
\hm\subset\boldsymbol{\pi}$ выполняется хотя бы одно из условий 
монотонности функции выравнивания времени: 
$$
j_{r}-j_{r-1}\geq1\,,\quad l_r-l_{r-1}\geq1\,, \quad r = 2,\ldots,R\,.
$$

Свойства пути между матрицами обобщают свойства пути между двумя 
временными рядами.

\smallskip

\noindent
\textbf{Определение~3.}\ {Стоимость 
$\mathrm{Cost}\,(\textbf{A},\textbf{B},{\boldsymbol{\pi}})$ пути $\boldsymbol{\pi}$ 
между объектами $\textbf{A}, \textbf{B}$:
\begin{equation*}
\mathrm{Cost}\,(\textbf{A},\textbf{B},{\boldsymbol{\pi}}) = 
\sum\limits_{(i,j,k,l) \in \boldsymbol{\pi}}{\boldsymbol{\Omega}}(i,j,k,l).
\end{equation*}}

\noindent
\textbf{Определение~4.}\ 
{Выравнивающий путь~$\hat{\boldsymbol{\pi}}$ между 
объектами $\textbf{A},\textbf{B}$~--- путь наименьшей стоимости 
среди всех возможных путей между объектами:
\begin{equation*}
\hat{\boldsymbol{\pi}} = 
\argmin\limits_{{\boldsymbol{\pi}}} \mathrm{Cost}
\left(\textbf{A},\textbf{B},{\boldsymbol{\pi}}\right).
\end{equation*}}
Функция расстояния~$\rho (\textbf{A},\textbf{B})\hm = \mathrm{mDTW}\,
(\textbf{A},\textbf{B})$ между объектами~$\textbf{A}$ и~$\textbf{B}$ 
рассчитывается как стоимость выравнивающего пути~$\hat{\boldsymbol{\pi}}$:
\begin{equation}
\mathrm{mDTW}(\textbf{A},\textbf{B}) = \mathrm{Cost}\left(\textbf{A},
\textbf{B},\hat{\boldsymbol{\pi}}\right).
\end{equation}

\setcounter{figure}{1}
\begin{figure*}[b] %fig2
{\small 
\begin{center}
\begin{tabular}{l}
\hline
DTW(\textbf{s},\textbf{c}):\\
\hspace*{3mm}$\boldsymbol{D}$(1:n+1,1:m+1) = inf;\\
\hspace*{3mm}$\boldsymbol{D}$(1,1) = 0;\\
\hspace*{3mm}for $i = 2$: $n+1$\\
\hspace*{6mm}for $j = 2$ : $m+1$\\
\hspace*{9mm}$d = (\textbf{s}(i-1)-\textbf{c}(j-1))^2$;\\
\hspace*{9mm}$\boldsymbol{D}(i,j) = d + \min( 
[ \boldsymbol{D}(i-1,j), \boldsymbol{D}(i,j-1), \boldsymbol{D}(i-1,j-1) ])$;\\
return\ sqrt$(\boldsymbol{D}(n+1,m+1))$\\
\hline
\end{tabular}
\end{center}}
\vspace*{-9pt}

\Caption{Алгоритм вычисления DTW для временных рядов
\label{ris:dtwts}}
%\end{figure*}
%\begin{figure*} %fig3
\vspace*{6pt}
{\small 
\begin{center}
\begin{tabular}{l}
\hline
\\[-9pt]
Correction $(\overline{i,j,k,l}, \boldsymbol{\pi}(\overline{i,j,k,l})):$\\
\hspace*{3mm}if $\overline{i,j,k,l} \in \{ (i-1, j, k,l)  ;  
(i, j, k-1, l)  ;  (i-1, j, k-1, l) \}$:\\
\hspace*{6mm}$ \widehat{\pi} = \{ (\overline{i}, r, \overline{k}, f) \in 
\boldsymbol{\pi}(\overline{i, j, k, l}) \vert r, f \in \mathbb{N} \}$\\
\hspace*{3mm}elif $\overline{i,j,k,l}\in \{  
(i, j-1, k, l); (i, j, k, l-1); (i, j-1, k, l-1) \}$:\\
\hspace*{6mm}$\widehat{\pi} = \{ (r, \overline{j}, f, \overline{l}) 
\in \boldsymbol{\pi}(\overline{i, j, k, l}) \vert r, f \in \mathbb{N} \}$\\
\hspace*{3mm}elif $\overline{i,j,k,l} =  i-1,j-1,k-1,l-1:$\\
\hspace*{6mm}$\widehat{\pi} = \{ (\overline{i}, r, \overline{k}, f) 
\in \boldsymbol{\pi}(\overline{i, j, k, l}) \vert r,f \in \mathbb{N} \} \cup$\\
\hspace*{6mm}$\cup \{ (r, \overline{j}, f, \overline{l}) \in \boldsymbol{\pi}
(\overline{i, j, k, l}) \vert r,f \in \mathbb{N} \}$\\
\hspace*{3mm}$\boldsymbol{d\pi} = \{ \mathrm{element} \in \widehat{\pi}: 
\mbox{произведены\ замены\ индексов } 
\overline{i} = i,\ \overline{j} = j,\ \overline{k} = k,\ \overline{l} = l \}$\\
return $\boldsymbol{d\pi}$\\
\hline
\end{tabular}
\end{center}
}
\vspace*{-9pt}

\Caption{Алгоритм вычисления поправки $\boldsymbol{d\pi}$ 
пути $\boldsymbol{\pi}$
\label{ris:codedpi}}
\end{figure*}


\textbf{Алгоритм вычисления значения расстояния~(4).}
Построение алгоритма вычисления значения функции расстояния 
между матрицами основан на алгоритме расчета функции расстояния 
между временн$\acute{\mbox{ы}}$ми рядами. В~случае выравнивания одной\linebreak\vspace*{-12pt}

{ \begin{center}  %fig1
 \vspace*{-3pt}
    \mbox{%
 \epsfxsize=79mm 
 \epsfbox{gon-1.eps}
 }


\end{center}


\noindent
{{\figurename~1}\ \ \small{Матрица стоимости оптимального выравнивания, по обеим 
осям отложены временные отсчеты}}
}

\vspace*{12pt}


\noindent 
временн$\acute{\mbox{о}}$й шкалы
 итоговая матрица расстояний~$\boldsymbol{D}$ (рис.~1) в~каждом 
 элементе~$\boldsymbol{D}(i,j)$ содержит рас\-сто\-яние между подрядом 
 первого временн$\acute{\mbox{о}}$го ряда и~подрядом второго временн$\acute{\mbox{о}}$го ряда. 
 Рас\-смот\-рим алгоритм динамического выравнивания двух временн$\acute{\mbox{ы}}$х 
 рядов $\textbf{s} \hm\in R^n$ и~$\textbf{c} \hm\in R^m$ на рис.~2.
 
 

Элемент $\boldsymbol{D}(i,j)$ матрицы~$\boldsymbol{D}$ соответствует 
стоимости выравнивающего пути между подпоследовательностями 
исходных временн$\acute{\mbox{ы}}$х рядов: $\textbf{s}(1:i) \hm= \textbf{s}(t)$, 
$t \hm= 1,\ldots,i,$ и~$\textbf{c}(1:j) \hm= \textbf{c}(t)$, $t \hm= 1,\ldots,j$. 
Алгоритм построения наилучшего выравнивания времени 
подразумевает, что выравнивающий путь между этими 
подпоследовательностями получен одним из трех способов~--- 
если стоимость выравнивающего пути между 
подпоследовательностями~$\textbf{s}(1:\overline{i}) $ 
и~$\textbf{c}(1:\overline{j})$ минимальна для~$\overline{i,j}$ из множества
$$
\overline{i,j} \in \left\{ \{i-1,j\},\{i,j-1\},\{i-1,j-1\} \right\},$$
тогда выравнивающий путь между $\textbf{s}(1:i)$ и~$\textbf{c}(1:j)$ получен добавлением пары~$(i,j)$ к~выбранному 
выравнивающему пути с~минимальной стоимостью из трех.



Предложенный алгоритм переносит эти рас\-суж\-де\-ния на случай 
выравнивания двух матриц~$\textbf{A}$ и~$\textbf{B}$. 
Элемент~$\boldsymbol{D}(i,j,k,l)$ четырехиндексного
 тензора расстояний~$\boldsymbol{D}$ соответствует стоимости выравнивающего 
 пути между $\textbf{A}(1:i,1:j) \hm= \textbf{A}(t_1,t_2)$, 
 $t_1 \hm= 1,\ldots, i$, $t_2 \hm= 1,\ldots, j,$ 
 и~$\textbf{B}(1:k,1:l) \hm= \textbf{B}(t_1,t_2)$, $t_1 \hm= 1,\ldots, k$,
 $t_2 \hm= 1,\ldots, l$. Выравнивающий путь между этими 
 подматрицами получен одним из семи способов~--- 
 если стоимость выравнивающего пути между 
 подматрицами $\textbf{A}(1:\overline{i},1:\overline{j})$ 
 и~$\textbf{B}(1:\overline{k},1:\overline{l})$ 
 минимальна для~$\overline{i,j,k,l}$ из множества
\begin{multline*} 
\overline{i,j,k,l} \in 
\left\{ \{i-1,j,k,l\},\{i,j-1,k,l\},\right.\\
\{i,j,k-1,l\},
\{i,j,k,l-1\}, \{i-1,j,k-1,l\},\\
\left.
\{i,j-1,k,l-1\},\{i-1,j-1,k-1,l-1\}\right\},
\end{multline*}

\setcounter{figure}{3}
\begin{figure*} %fig4
{\small 
\begin{center}
\begin{tabular}{l}
\hline
$\mathrm{mDTW}\left(\textbf{A},\textbf{B}\right):$\\
\hspace*{3mm}$\textbf{D}(1:n+1,1:n+1, 1:n+1, 1:n+1) = inf$;\\
\hspace*{3mm}$\textbf{D}(1,1,1,1) = 0;$\\
\hspace*{3mm}$\boldsymbol{\pi}(1,1,1,1) = ((1,1),(1,1))$\\
\hspace*{3mm}$for\ i,j,k,l  \in \mathbb{N}^{2 : n+1} \times 
\mathbb{N}^{2 : n+1} \times \mathbb{N}^{2 : n+1} \times \mathbb{N}^{2 : n+1}:$\\
\hspace*{6mm}$\overline{i,j,k,l} = \argmin($ [ \textbf{D}(i-1, j, k, l), 
\textbf{D}(i, j-1, k, l), \textbf{D}(i, j, k-1, l), 
\textbf{D}(i, j, k, l-1),    \\
\hspace*{9mm}$\textbf{D}(i-1, j, k-1, l), \textbf{D}(i, j-1, k, l-1), 
\textbf{D}(i-1, j-1, k-1, l-1) ])$;\\
\hspace*{3mm}$\boldsymbol{d \pi} = \mathrm{Correction}\,(\overline{i,j,k,l}, 
\boldsymbol{\pi}(\overline{i,j,k,l}))$\\
\hspace*{3mm}$\boldsymbol{\pi}(i, j, k, l) = \boldsymbol{d \pi} \cup 
\{(\overline{i,j,k,l})\}$\\
\hspace*{3mm}$\mathrm{cost} = (\textbf{A}(i, j)-\textbf{B}(k, l))^2 + 
\sum\nolimits_{(r,f,t,g) \in \boldsymbol{d \pi}}
(\textbf{A}(r, f)-\textbf{B}(t, g))^2$;\\
\hspace*{3mm}$\textbf{D}(i,j,k,l) = \mathrm{cost} + \textbf{D}
(\overline{i,j,k,l})$\\
return  sqrt$(\textbf{D}(n+1,n+1,n+1,n+1))$\\
\hline
\end{tabular}
\end{center}
}
\vspace*{-9pt}

\Caption{Алгоритм вычисления расстояния между матрицами
\label{ris:matrixdtw}}
\end{figure*}

\begin{table*}[b]\small
\begin{center}
\begin{tabular}{|l|c|c|c|c|}
\multicolumn{5}{c}{Снижение расстояний при выполнении преобразований 
для различных наборов данных}\\
\multicolumn{5}{c}{\ }\\[-6pt]
\hline
 &\multicolumn{4}{c|}{Метод}\\
 \cline{2-5}
\multicolumn{1}{|c|}{Данные}  & \multicolumn{2}{c|}{$L_2$} & \multicolumn{2}{c|}{MatrixDTW} \\
\cline{2-5}
& $S(f|p)$  &  $S_{\rho}(\mathfrak{D})$ &  $S(f|p)$ & $S_{\rho}(\mathfrak{D})$ \\
\hline
Модельные данные без преобразований& 92\% & 78\% & 100\%\hphantom{9} & 85\% \\
Модельные данные с~преобразованиями & 86\% & 65\% &  100\%\hphantom{9} & 82\% \\
Модельные данные с~преобразованиями и~шумом& 69\% & 61\% &  92\% & 78\% \\
MNIST без преобразований& 95\% & --- & 95\% & --- \\
MNIST с~преобразованиями & 53\% & --- & 92\% & --- \\
Спектр сигнала& 83\% & --- & 96\% & --- \\
\hline
\end{tabular}
\end{center}
\end{table*}

\noindent
то к~выравнивающему пути между этими под\-мат\-ри\-ца\-ми 
добавляется элемент пути $(i,j,k,l)$ и~поправка~$\boldsymbol{d\pi} $ 
пути~$\boldsymbol{\pi}$, алгоритм вычисления которой приведен ниже.

Обозначим выравнивающий путь между $\textbf{A}(1:i,\linebreak 1:j)$
 и~$\textbf{B}(1:k,1:l)$ как~$\boldsymbol{\pi}(i,j,k,l)$, тогда 
 поправка~$\boldsymbol{d\pi} $ пути~$\boldsymbol{\pi}(i,j,k,l)$ 
 при фиксированных~$\overline{i,j,k,l}$ вычисляется приведенным на рис.~3 
 образом.





Алгоритм динамического выравнивания двух матриц и~вычисления 
расстояния $\mathrm{mDTW}$ между ними с~учетом приведенного выше 
алгоритма примет вид, представленный на рис.~4.





\begin{figure*} %fig5
\vspace*{1pt}
    \begin{center}  
  \mbox{%
 \epsfxsize=161.412mm 
 \epsfbox{gon-5.eps}
 }
\end{center}
\vspace*{-12.5pt}
\Caption{Выравнивание модельных данных: (\textit{а})~один класс без шума; 
(\textit{б})~разные классы без шума; 
(\textit{в})~один класс с~шумом; (\textit{г})~разные классы с~шумом
\label{ris:random}}
%\end{figure*}
%\begin{figure*} %fig6
\vspace*{1pt}
    \begin{center}  
  \mbox{%
 \epsfxsize=163mm 
 \epsfbox{gon-6.eps}
 }
\end{center}
\vspace*{-12.5pt}
\Caption{Выравнивание данных MNIST: левый столбец~--- один класс; 
правый столбец~--- разные 
классы;
(\textit{а})~$\mathrm{mDTW}\hm=720{,}1$; 
(\textit{б})~948,6;
(\textit{в})~2017,0;
(\textit{г})~$\mathrm{mDTW}\hm=2071{,}4$
\label{ris:mnist}}
\end{figure*}


Следует отметить, что алгоритм~\cite{15} имеет\linebreak высокую сложность 
вычисления~--- $O(n^4)$. Предполагается ускорение метода 
с~использованием ограниче\-ния Sakoe-Chiba band, что сократит 
вычислительную сложность алгоритма до $O(n^2k^2)$, где~$k$~--- 
параметр ограничения.


\section{Вычислительный эксперимент}

Вычислительный эксперимент проведен на модельных данных с~допустимыми 
преобразованиями и~на реальных данных: объектах коллекции MNIST с~допустимыми 
преобразованиями и~на спектрограммах зашумленных сигналов.





Решается задача метрической классификации методом ближайшего соседа. В~таблице 
приведены значения критерия качества функции расстояния 
$S_{\rho}(\mathfrak{D})$ и~критерия качества метрической классификации $S(f|p)$ 
при использовании двух функций расстояния: предложенной в~работе $\mathrm{mDTW}$ 
и~$L_2$.

Модельные данные~--- это нулевые матрицы со случайными ненулевыми 
строками, столбцами, подпрямоугольниками с~наложенным шумом. 
К~ним применены допустимые преобразования, согласованные с~гипотезой 
наличия локальных и~глобальных искажений. На рис.~\ref{ris:random} 
показан пример оптимального выравнивания двух объектов. 
Линиями показаны элементы пути~$\boldsymbol{\pi}$.

Подготовлена подвыборка набора данных MNIST. Она 
состоит из~100 объектов классов 0 и~1 сниженной размерности
 с~допустимыми преобразованиями. На рис.~\ref{ris:mnist} 
 показан пример оптимального выравнивания объектов.


Аналогичный эксперимент проведен для решения задачи метрической 
классификации спектров различных сигналов, пример которых приведен на 
рис.~\ref{ris:spectr}. На рисунке показаны примеры Фурье-спект\-ров 
этих сигналов. Спектр получен путем применения быстрого преобразования 
Фурье к~исходному сигналу для различных окон с~фиксированным размером и~сдвигом. 
Исходные временн$\acute{\mbox{ы}}$е ряды обладали свойством периодичности, период выбирался 
случайным образом.



Тестирование проведено на разного рода данных: исходных 
модельных данных без наложения\linebreak\vspace*{-12pt}

\pagebreak

\end{multicols}

\begin{figure*} %fig7
\vspace*{1pt}
    \begin{center}  
  \mbox{%
 \epsfxsize=149.062mm 
 \epsfbox{gon-7.eps}
 }
\end{center}
\vspace*{-8pt}
\Caption{Данные спектров сигнала: (\textit{а})~класс~1; (\textit{б})~спектр 
класса~1; (\textit{в})~класс~2; (\textit{г})~спектр класса~2; 
(\textit{д})~класс~3; (\textit{е})~спектр класса~3
\label{ris:spectr}}
\vspace*{9pt}
\end{figure*}

\begin{multicols}{2}

\noindent допустимых преобразований, с~ними, а~также 
на модельных данных с~наложенным поверх объектов случайным шумом.



В каждом из проведенных экспериментов была продемонстрирована 
устойчивость предложенного подхода к~допустимым преобразованиям. 
Наилучшее значение критерия качества задачи классификации было 
достигнуто при использовании предложенной функции расстояния.

\vspace*{-5pt}

\section{Заключение}

В работе предложено обобщение метода динамического выравнивания
 временн$\acute{\mbox{ы}}$х рядов для случая объектов, определенных на двух осях времени. 
 Существует теоретическое обобщение предлагаемых методов на случай 
 конечного множества осей времени. Вычислительный эксперимент позволил 
 проанализировать свойства подхода: устойчивость к~допустимым 
 преобразованиям и~разделяющая способность функции расстояния как 
 на реальных, так и~на модельных данных. Качество решения задачи 
 метрической классификации выше решения, основанного на евклидовом 
 расстоянии. Вычислительная сложность метода высокая, что ограничивает 
 его применимость на объектах высокой размерности.

\vspace*{-2pt}

{\small\frenchspacing
 {%\baselineskip=10.8pt
 \addcontentsline{toc}{section}{References}
 \begin{thebibliography}{99}
%\bibitem{Karasikov2016}
%\Au{Карасиков~М.\,Е., Стрижов~В.\,В.} Классификация временных рядов 
%в~пространстве параметров по\-рож\-да\-ющих моделей~// Информатика и~её 
%применения,~2016. T.~10. Вып.~4. С.~121--131.

\bibitem{0}
\Au{Hill~N.\,J., Lal~T.\,N., Schroder~M., Hinterberger~T., 
Wilhelm~B., Nijboer~F., Mochty~U., Widman~G., Elger~C., 
Scholkopf~B., Kubler~A., Birbaumer~N.} Classifying EEG and 
ECoG signals without subject training for fast BCI implementation: 
Comparison of nonparalyzed and completely paralyzed subjects~//  
IEEE~T. Neur. Sys. Reh., 2006. Vol.~14. 
Iss.~2. P.~183--186.

\bibitem{1}
\Au{Sakoe~H., Chiba~S.} 
A~dynamic programming approach to continuous speech recognition~// 
7th  Congress (International) on Acoustics Proceedings, 1971. Vol.~3. P.~65--69.

\bibitem{2} %3
\Au{Aghabozorgi~S., Ali~S.\,S., Wah~T.\,Y.} 
Time-series clustering~--- a~decade review~// Inform. Syst., 
2015. Vol.~53. P.~16--38.

\bibitem{3} %4
\Au{Warrenliao~T.} Clustering of time series data~--- a~survey~// 
Pattern Recogn., 2005. Vol.~38. Iss.~11. P.~1857--1874.



\bibitem{4} %5
\Au{Hautamaki~V., Nykanen~P., Franti~P.} 
Time-series clustering by approximate prototypes~// 
19th  Conference (International) on Pattern Recognition Proceedings, 2008. No.\,D. 
P.~1--4.

\bibitem{5} %6
\Au{Faloutsos~C., Ranganathan~M., Manolopoulos~Y.} 
Fast subsequence matching in time-series databases~// \mbox{SIGMOD} Rec., 1994. 
Vol.~23. Iss.~2. P.~419--429.

\bibitem{10} %7
\Au{Basalto~N., Bellotti~R., Carlo~F.\,D., Facchi~P., 
Pascazio~S.} Hausdorff clustering of financial time series~// 
Physica~A, 2007. Vol.~379. Iss.~2. P.~635--644.

\bibitem{11} %8
\Au{Gorelick~L., Blank~M., Shechtman~E., Irani~M., Basri~R.} 
Actions as space-time shapes~// IEEE~T. Pattern Anal., 
2007. Vol.~29. Iss.~12. P.~2247--2253.

\bibitem{6} %9
\Au{Smyth~P.} Clustering sequences with hidden Markov models~// 
Adv. Neural In., 1997. Vol.~9. P.~648--654.

\bibitem{7} %10
\Au{Banerjee~A., Ghosh~J.} Clickstream clustering using weighted 
longest common subsequences~// 
Workshop on Web Mining, SIAM Conference on Data Mining
Proceedings, 2001. P.~33--40.

\bibitem{8} %11
\Au{Aach~J., Church~G.M.} Aligning gene expression time series
 with time warping algorithms~// Bioinformatics, 2001. Vol.~17. Iss.~6. P.~495--508.

\bibitem{9} %12
\Au{Yi~B.\,K., Faloutsos~C.} Fast time sequence indexing 
for arbitrary $\mathcal{L}_p$ norms~// 
26th  Conference (International) on Very Large Data Bases Proceedings, 2000. P.~385--394.

\bibitem{33} %13
\Au{Goncharov~A.\,V., Strijov~V.\,V.} 
Analysis of dissimilarity set between time series~// Computational 
Mathematics Modeling, 2018. Vol.~29. Iss.~3. P.~359--366.

\bibitem{12} %14
\Au{Alon~J., Athitsos~V., Sclaroff~S.}
 Online and offline character recognition using alignment to prototypes~// 
 8th  Conference (International) on Document Analysis and Recognition, 2005. 
 Vol.~2. P.~839--843.

\bibitem{15} %15
\Au{Гончаров~А.\,В.} 
Выравнивания декартовых произведений упорядоченных множеств mDTW. 
Про\-грам\-мная реализация алгоритма, 2019. 
{\sf https://github.
com/Intelligent-Systems-Phystech/PhDThesis/tree/\linebreak  master/Goncharov2019/MatrixDTW/code}.
 \end{thebibliography}

 }
 }

\end{multicols}

\vspace*{-9pt}

\hfill{\small\textit{Поступила в~редакцию 24.04.19}}

\vspace*{6pt}

%\pagebreak

%\newpage

%\vspace*{-28pt}

\hrule

\vspace*{2pt}

\hrule

\vspace*{-4pt}

\def\tit{ALIGNMENT OF~ORDERED SET CARTESIAN PRODUCT\\[-5pt]}


\def\titkol{Alignment of~ordered set cartesian product}

\def\aut{A.\,V.~Goncharov$^1$ and~V.\,V.~Strijov$^{1,2}$}

\def\autkol{A.\,V.~Goncharov and~V.\,V.~Strijov}

\titel{\tit}{\aut}{\autkol}{\titkol}

\vspace*{-13pt}


\noindent
$^1$ Moscow Institute of Physics and Technology, 
9~Institutskiy Per., Dolgoprudny, Moscow Region 141700, Russian\linebreak
$\hphantom{^1}$Federation


\noindent
$^2$A.\,A.~Dorodnicyn Computing Center, Federal Research Center 
``Computer Science and Control'' of the Russian\linebreak
$\hphantom{^1}$Academy of Sciences, 
40~Vavilov Str., Moscow 119333, Russian Federation

\def\leftfootline{\small{\textbf{\thepage}
\hfill INFORMATIKA I EE PRIMENENIYA~--- INFORMATICS AND
APPLICATIONS\ \ \ 2020\ \ \ volume~14\ \ \ issue\ 1}
}%
 \def\rightfootline{\small{INFORMATIKA I EE PRIMENENIYA~---
INFORMATICS AND APPLICATIONS\ \ \ 2020\ \ \ volume~14\ \ \ issue\ 1
\hfill \textbf{\thepage}}}

\vspace*{2pt} 



\Abste{The work is devoted to the study of metric methods for analyzing 
objects with complex structure. It proposes to generalize the dynamic 
time warping method of two time series for the case of objects defined 
on two or more time axes. Such objects are matrices in the discrete 
representation. The DTW (Dynamic Time Warping) method of time series is generalized as 
a~method of matrices dynamic alignment. The paper proposes 
a~distance function resistant to monotonic nonlinear deformations of the 
Cartesian product of two time scales. The alignment path between objects is 
defined. An object is called a~matrix in which the rows and columns correspond 
to the axes of time. The properties of the proposed distance function 
are investigated. To illustrate the method, the problems of metric 
classification of objects are solved on model data and data from the 
MNIST dataset.}

\KWE{distance function; dynamic alignment; distance between matrices; 
nonlinear time warping; space--time series}



\DOI{10.14357/19922264200105} 

%\vspace*{-14pt}

\Ack
\noindent
This work was supported by the Russian Foundation for Basic
Research (projects 19-07-1155 and 19-07-00885). 
The paper contains results of the project Statistical 
methods of machine learning, which is carried out within the 
framework of the Program ``Center of Big Data Storage and Analysis'' 
of the National Technology Initiative Competence Center. 
It is supported by the Ministry of Science and Higher Education 
of the Russian Federation according to the agreement between the
 M.\,V.~Lomonosov Moscow State University and the Foundation 
 of project support of the National Technology Initiative from 11.12.2018, 
 No.\,13/1251/2018.
 


%\vspace*{6pt}

  \begin{multicols}{2}

\renewcommand{\bibname}{\protect\rmfamily References}
%\renewcommand{\bibname}{\large\protect\rm References}

{\small\frenchspacing
 {%\baselineskip=10.8pt
 \addcontentsline{toc}{section}{References}
 \begin{thebibliography}{99}

 \bibitem{0-1}   
\Aue{Hill, N.\,J., T.\,N.~Lal, M.~Schroder, T.~Hinterberger, B.~Wilhelm, 
F.~Nijboer, U.~Mochty, G.~Widman, C.~Elger, B.~Scholkopf, A.~Kubler, and 
N.~Birbaumer.} 2006. Classifying EEG and ECoG signals without subject 
training for fast BCI implementation: Comparison of nonparalyzed and completely 
paralyzed subjects. \textit{IEEE~T. Neur. Sys. 
Reh.} 14(2):183--186.

\bibitem{1-1}   
\Aue{Sakoe, H., and S.~Chiba.} 1971. A~dynamic programming approach 
to continuous speech recognition. \textit{7th 
 Congress (International) on Acoustics Proceedings}. 3:65--69.

\bibitem{2-1}    %2
\Aue{Aghabozorgi,~S., S.\,S.~Ali, and T.\,Y.~Wah.} 2015. 
Time-series clustering~--- a~decade review.  \textit{Inform. Syst.} 
53:16--38.

\bibitem{3-1}   %4 
\Aue{Warrenliao,~T.} 2005. Clustering of time series data~--- a~survey. 
\textit{Pattern Recogn.}
38(11):1857--1874.



\bibitem{4-1}    %5
\Aue{Hautamaki,~V., P.~Nykanen, and P.~Franti.} 2008. 
Time-series clustering by approximate prototypes. 
 \textit{19th  Conference (International) on Pattern Recognition Proceedings}. 
 D:1--4.

\bibitem{5-1}    %6
\Aue{Faloutsos,~C., M.~Ranganathan, and Y.~Manolopoulos.} 1994. 
Fast subsequence matching in time-series databases.  \textit{SIGMOD Rec}. 
23(2):419--429.

\bibitem{10-1}    %7
\Aue{Basalto, N., R.~Bellotti, F.\,D.~Carlo, P.~Facchi, and S.~Pascazio.} 
2007. Hausdorff clustering of financial time series. 
\textit{Physica~A} 379(2):635--644.

\bibitem{11-1}   %8
\Aue{Gorelick, L., M.~Blank, E.~Shechtman, M.~Irani, and R.~Basri.} 
2007. Actions as space-time shapes.
\textit{IEEE~T. Pattern Anal.} 29(12):2247--2253.

\bibitem{6-1}    %9
\Aue{Smyth, P.} 1997. 
Clustering sequences with hidden Markov models. \textit{Adv. Neural In.} 9:648--654.

\bibitem{7-1}    %10
\Aue{Banerjee,~A., and J.~Ghosh.} 2001. 
Clickstream clustering using weighted longest common subsequences.  
\textit{Workshop on Web Mining, SIAM Conference 
on Data Mining Proceedings.} 33--40.

\bibitem{8-1}    %11
\Aue{Aach, J., and G.\,M.~Church.} 2001. 
Aligning gene expression time series with time warping algorithms. 
\textit{Bioinformatics} 17(6):495--508.

\bibitem{9-1}   %12
\Aue{Yi, B.\,K., and C.~Faloutsos.} 2000. 
Fast time sequence indexing for arbitrary $\mathcal{L}_p$ norms. 
\textit{26th  Conference (International) 
on Very Large Data Bases Proceedings}. 385--394.

\bibitem{33-1}   %13 
\Aue{Goncharov,~A.\,V., and V.\,V.~Strijov.} 2018. 
Analysis of dissimilarity set between time series. 
\textit{Computational Mathematics Modeling } 29(3):359--366.



\bibitem{12-1}    %14
\Aue{Alon, J., V.~Athitsos, and S.~Sclaroff.} 2005.
 Online and offline character recognition using alignment to prototypes. 
 \textit{8th  Conference (International) on Document Analysis and Recognition}. 
 2:839--843.

\bibitem{15-1}    %15
\Aue{Goncharov, A.\,V.} Alignment of 
Ordered Set Cartesian Product mDTW. Software implementation of the algorithm. 
Available at: {\sf https://github.com/Intelligent-\linebreak 
Systems-Phystech/PhDThesis/tree/master/Goncharov\linebreak 2019/MatrixDTW/code} 
(accessed December~27, 2019).
\end{thebibliography}

 }
 }

\end{multicols}

%\vspace*{-7pt}

\hfill{\small\textit{Received April 24, 2019}}

%\pagebreak

%\vspace*{-22pt}



\Contr

\noindent
\textbf{Goncharov Alexey V.} (b.\ 1995)~--- 
PhD student, Moscow Institute of Physics and Technology, 
9~Institutskiy Per., Dolgoprudny, Moscow Region 141701, 
Russian Federation; \mbox{alex.goncharov@phystech.edu}

\vspace*{3pt}

\noindent
\textbf{Strijov Vadim V.} (b.\ 1967)~--- 
Doctor of Science in physics and mathematics, leading scientist, 
A.\,A.~Dorodnicyn Computing Centre, Federal Research Center 
``Computer Science and Control'' of the Russian Academy of Sciences, 
40~Vavilov Str., Moscow 119333, Russian Federation;
 professor, Moscow Institute of Physics and Technology, 
 9~Institutskiy Per., Dolgoprudny, Moscow Region 141701, Russian Federation; 
 \mbox{strijov@ccas.ru}
\label{end\stat}

\renewcommand{\bibname}{\protect\rm Литература} %5
\def\stat{br-stup}

\def\tit{НЕЙРОФИЗИОЛОГИЯ КАК~ПРЕДМЕТНАЯ ОБЛАСТЬ ДЛЯ~РЕШЕНИЯ ЗАДАЧ 
С~ИНТЕНСИВНЫМ\\ ИСПОЛЬЗОВАНИЕМ ДАННЫХ$^*$}

\def\titkol{Нейрофизиология как предметная область для~решения задач 
с~интенсивным использованием данных}

\def\aut{Д.\,О.~Брюхов$^1$, С.\,А.~Ступников$^2$, Д.\,Ю.~Ковалёв$^3$, 
И.\,А.~Шанин$^4$}

\def\autkol{Д.\,О.~Брюхов, С.\,А.~Ступников, Д.\,Ю.~Ковалёв, 
И.\,А.~Шанин}

\titel{\tit}{\aut}{\autkol}{\titkol}

\index{Брюхов Д.\,О.}
\index{Ступников С.\,А.}
\index{Ковалёв Д.\,Ю.} 
\index{Шанин И.\,А.}
\index{Briukhov D.\,O.}
\index{Stupnikov S.\,A.}
\index{Kovalev D.\,Yu.}
\index{Shanin I.\,A.}


{\renewcommand{\thefootnote}{\fnsymbol{footnote}} \footnotetext[1]
{Работа выполнена при частичной финансовой поддержке РФФИ (проект 18-29-22096).}}


\renewcommand{\thefootnote}{\arabic{footnote}}
\footnotetext[1]{Институт проблем информатики Федерального исследовательского центра <<Информатика и~управлени>> 
Российской академии наук, \mbox{dbriukhov@ipiran.ru}}
\footnotetext[2]{Институт проблем информатики Федерального исследовательского центра <<Информатика и~управление>> 
Российской академии наук, \mbox{sstupnikov@ipiran.ru}}
\footnotetext[3]{Институт проблем информатики Федерального исследовательского центра <<Информатика и~управление>> 
Российской академии наук, \mbox{dm.kovalev@gmail.com}}
\footnotetext[4]{Институт проблем информатики Федерального исследовательского центра 
<<Информатика и~управление>> Российской академии наук, \mbox{ivan.shanin@gmail.com}}

%\vspace*{-12pt}


  

\Abst{Цель данного обзора~--- анализ нейрофизиологии как области с~интенсивным 
использованием данных. В~настоящее время происходит заметный рост числа 
исследований в~области изучения человеческого мозга. Появляются крупные 
международные проекты, поддерживающие исследования, направленные на улучшение 
понимания работы человеческого мозга, а также на обнаружение и~поиск способов 
лечения основных заболеваний, связанных с~человеческим мозгом. Объем данных, 
генерируемых в~типичной лаборатории, проводящей исследования в~области 
нейрофизиологии, растет в~геометрической прогрессии. При этом данные представляются с~использованием большого числа разнообразных форматов. Это приводит 
к~необходимости создания инфраструктур и~баз данных, а также веб-сай\-тов, 
предоставляющих единый доступ к~данным и~обеспечивающим обмен этими данными 
между исследователями по всему миру. Для анализа собранных данных применяются 
методы и~средства из области нейроинформатики~--- науки на стыке нейрофизиологии 
и~информатики. Для решения нейрофизиологических задач применяются различные 
методы информатики, такие как статистический анализ и~машинное обучение, в~частности 
нейронные сети.}

\KW{нейрофизиология; нейроинформатика; интенсивное использование данных; анализ 
данных}

\DOI{10.14357/19922264200106} 
  
\vspace*{6pt}


\vskip 10pt plus 9pt minus 6pt

\thispagestyle{headings}

\begin{multicols}{2}

\label{st\stat}

\section{Введение}

%\vspace*{-3pt}

    Нейрофизиология~--- один из ярких примеров научной области 
с~интенсивным использованием данных. Она представляет собой 
комбинацию различных областей знаний: анатомии, физиологии, генетики, 
биохимии, психологии~--- и~стала передовой областью в~исследовании 
и~моделировании работы человеческого мозга.
    
    В настоящее время во всем мире растет интерес к~научному пониманию 
работы человеческого мозга, выражающийся в~количестве исследований 
в~области нейрофизиологии. Появляется новое, более качественное 
оборудование, поз\-во\-ля\-ющее получать более точные данные различного вида, 
в~част\-ности данные маг\-нит\-но-ре\-зо\-нанс\-ной томографии (МРТ), 
электроэнцефалографии (ЭЭГ), магнитоэнцефалографии (МЭГ) и~др. Новое 
оборудование позволяет за несколько дней собирать больше данных, чем 
всего десять лет назад собиралось за целый год.
    
    С увеличением объема данных встает проблема совместного 
использования этих данных для решения разнообразных задач в~области 
нейрофизиологии.
Стали появляться как региональные консорциумы 
и~проекты, поддерживающие\linebreak исследователей для решения различных задач 
в~об\-ласти нейрофизиологии (например, американская инициатива BRAIN
(The Brain Research through Advancing Innovative 
Neurotechnologies$^\registered$ Initiative), 
европейские проекты HBP (Human Brain Project)
и~BNCI (Brain-Neural-Computer-Interaction)
из программы Horizon 2020), так 
и~консорциумы, объединяющие исследователей во всем мире для решения 
конкретных задач, такие как проект исследования коннектома человека HCP
(Human Connectome Project), 
инициатива по нейровизуализации болезни Альцгеймера ADNI
(Alzheimer's Disease Neuroimaging 
Initiative), инициатива 
по развитию маркеров для болезни Паркинсона PPMI
(Parkinson's Progression Markers Initiative).
{\looseness=1

} 
    
    С ростом объема данных растет и~разнообразие этих данных. 
К~сожалению, в~об\-ласти нейрофизиологии в~настоящее время нет единых 
стандартов для пред\-став\-ле\-ния данных. Это относится практически ко всем 
видам данных, в~част\-ности к~нейроизображениям и~биомедицинским 
сигналам. Разнообразие форматов данных вызвано большим чис\-лом видов 
медицинского оборудования, а~также средств визуализации и~анализа 
получаемых данных.
{\looseness=1

}
    
    С ростом объема и~разнообразия данных продолжает усиливаться 
стремление разместить их в~доступных репозиториях (базах данных, веб-сай\-тах). 
Такие репозитории могут содержать петабайты 
нейрофизиологических данных и~позволяют обмениваться ими 
исследователям по всему миру. Некоторые базы содержат данные для 
решения конкретного класса задач, другие~--- широкий набор различных 
данных. При добавлении новых данных в~эти базы данные обычно проходят 
процесс рецензирования. 

Базы данных могут содержать данные как  
в~ка\-ком-то определенном формате, так и~поддерживать несколько разных 
форматов, принятых в~сообществе. Базы данных и~веб-сай\-ты предоставляют 
единый интерфейс доступа к~зарегистрированным в~них данным. Некоторые 
сайты предоставляют также программные средства для визуализации 
содержащихся в~них данных. 
{ %\looseness=1

}
    
    За последние годы созданы десятки про\-грам\-мных средств для сбора, 
обработки, анализа и~визуализации данных в~области нейрофизиологии, 
основанные на методах и~средствах из области информатики и~применяющих 
методы моделирования из области нейрофизиологии. Таким образом,\linebreak 
формируется нейроинформатика как меж\-дис\-цип\-ли\-нар\-ная об\-ласть 
сотрудничества ис\-сле\-до\-ва\-те\-лей-ней\-ро\-фи\-зио\-ло\-гов  
с~ис\-сле\-до\-ва\-те\-ля\-ми-ин\-фор\-ма\-ти\-ками.
    
    В рамках статьи предоставлена информация о~текущем состоянии дел 
в~об\-ласти нейрофизиологии (при этом основное внимание уделяется 
моделированию когнитивных функций на основе нейрофизиологических 
данных): основные мировые стратегические инициативы и~проекты (разд.~2), 
крупные базы данных, содержащие данные исследований (разд.~3), основные 
форматы представления нейроизображений и~биомедицинских сигналов 
(разд.~4), программные средства для обработки и~анализа нейроизображений 
(разд.~5). 

\section{Крупные международные консорциумы и~проекты 
в~области нейрофизиологии}

    С увеличением объема данных и~числа исследований в~области 
нейрофизиологии встает задача компьютерной поддержки этих исследований и~совместного использования полученных данных. Появляются как 
региональные консорциумы и~проекты, поддерживающие исследователей 
для решения различных задач в~области нейрофизиологии, так 
и~консорциумы, объединяющие исследователей во всем мире для решения 
конкретных задач или для лечения различных заболеваний, связанных 
с~мозгом.
    
    \textit{Инициатива исследования мозга с~помощью продвинутых 
инновационных технологий} (BRAIN)~[1] была объявлена 
в~США в~2013~г.\ и~представляет собой 10-лет\-нюю программу, 
направленную на революцию в~понимании работы человеческого мозга. 
    
    Начатый в~2013~г.\ проект \textit{Human Brain Project}~[2]~--- 
это десятилетний проект поддержки исследований человеческого мозга, 
курируемый Европейским Союзом (ЕС). Цель проекта~--- создание 
современной исследовательской инфраструктуры, которая позволит 
исследователям расширять знания в~понимании работы человеческого мозга. 
    
    Стартовавший в~2010~г.\ проект \textit{Human Connectome Project}~[3] 
    является попыткой картирования нервных путей, лежащих 
в~основе функционирования человеческого мозга. Цель проекта~--- сбор 
и~обмен данными о структурной и~функциональной связанности 
человеческого мозга (коннектома) в~макромасштабе (в~сантиметровом 
и~миллиметровом масштабе). 
    
    Проект \textit{BNCI Horizon 2020}~[4] в~рамках \mbox{7-й} рамочной программы 
ЕС направлен на поддержку и~координацию усилий в~области интерфейсов 
мозг--компьютер (BCI, Brain--Computer Interface) и~нейроинтерфейсов 
мозг--компьютер (\mbox{BNCI}). 
Основная цель этого проекта~--- разработка дорожной карты для области 
BCI с~особым упором на промышленные приложения BCI и~конечных 
пользователей. Этот проект объединяет~12~европейских университетов. 
    
    Потребность в~использовании данных различных дисциплин для 
исследования процессов и~способов лечения основных заболеваний была 
признана несколько лет назад~[5]. Также была осознана\linebreak необходимость 
сотрудничества между центрами и~дисциплинами для интеграции 
и~совместного использования разнообразных данных~[6] путем организации 
междисциплинарных консорциумов.
    
    Примером такого консорциума может служить \textit{инициатива по 
нейровизуализации болезни Альцгеймера} (ADNI)~[7], объединяющая исследователей с~данными 
исследований для улучшения профилактики и~лечения болезни Альцгеймера. 
Основные цели инициативы: выявление болезни на ранней стадии 
и~определение способа отслеживания болезни с~помощью биомаркеров, 
применение методов ранней диагностики (когда вмешательство может быть 
наиболее эффективным), предостав\-ле\-ние данных исследований для ученых 
всего \mbox{мира}. 
{\looseness=1

}
    
    Другими примерами междисциплинарных консорциумов, 
использующих обработку ней\-ро\-изоб\-ра\-же\-ний, являются инициативы, 
направленные на лечения таких заболеваний, как болезнь Паркинсона (PPMI), 
психиатрические 
расстройства. Поддерживаются базы данных для сбора нейровизуальных, 
генетических и~феноменальных данных об аутизме (National Database of 
Autism Research) и~повреждениях головного мозга (Federal Interacgency 
Traumatic Brain Injury Research).
    
\section{Инфраструктуры и~базы данных в~области 
нейрофизиологии}

    С целью дальнейшего использования данных, полученных 
исследователями со всего мира, создаются и~поддерживаются 
инфраструктуры доступа к~данным и~отдельные базы данных, объеди\-ня\-ющие 
данные от различных исследовательских групп и~пред\-остав\-ля\-ющие единый 
интерфейс доступа к~этим данным. Инфраструктуры пред\-остав\-ля\-ют единую 
среду для доступа к~различным данным и~использования различных 
программных средств для обработки этих данных. Ниже рассмотрены 
основные современные инфраструктуры и~базы данных.

\medskip
    
    Проект \textit{1000 функциональных коннектомов} (1000 Functional 
Connectomes project)~\cite{8-bs} предоставляет доступ  
к~фМРТ-изоб\-ра\-же\-ни\-ям со всего мира.\linebreak Проект содержит данные 
о~более~1200~наборах фМРТ-изображений состояния покоя, собранных 
с~33~разных сайтов. Проект содержит как сырые, так и~пред\-об\-ра\-бо\-тан\-ные 
данные, представленные в~формате BIDS  (brain imaging data structure).

\medskip
    
    \textit{OpenNEURO}~\cite{9-bs}~--- бесплатная и~открытая платформа 
для обмена данными МРТ, МЭГ и~ЭЭГ. Она является развитием проекта по 
созданию базы данных \textit{OpenfMRI}, законченного в~2010~г. 
Первоначально база данных включала только наборы данных, содержащих 
фМРТ-дей\-ст\-вия (task based \mbox{fMRI}). В~настоящее время она открыта для 
любых видов  
МРТ-ней\-ро\-изоб\-ра\-же\-ний. Все изображения, хранящиеся в~базе 
данных, представлены в~формате BIDS.
    
    \textit{База данных ConnectomeDB}~\cite{10-bs} была разработана 
в~рамках проекта HCP~[3] и~содержит данные 
о~структурной и~функциональной связанности человеческого мозга 
(коннектома). База данных в~настоящее время включает в~себя несколько 
видов данных МРТ, ЭЭГ и~МЭГ. Изображения, хранящиеся в~ConnectomeDB, 
представлены в~формате NIFTI. Для обработки данных проекта был создан 
\textit{Connectome Workbench}~--- свободно предоставляемый инструмент 
для визуализации и~анализа данных, полученных в~рамках проекта HCP.
    
    \textit{XNAT}~\cite{11-bs}~--- это открытая информационная платформа 
для работы с~нейроизображениями, разработанная исследовательской 
группой по нейроинформатике в~Вашингтонском университете. Она 
облегчает общие задачи управления, обеспечения производительности 
и~качества обработки нейроизображений и~связанных данных. \textit{XNAT 
Central} является общедоступным хранилищем медицинских изображений, 
основанным на открытой информационной платформе обработки 
изображений XNAT. В~отличие от большинства других хранилищ, таких как 
ConnectomeDB и~Open fMRI, XNAT Central не модерируется для контроля 
содержимого и~не предназначен для поддержки решения ка\-ких-ли\-бо 
конкретных научных задач и~подходов. Все изображения, хранящиеся 
в~XNAT Central, представлены в~формате DICOM.
    
    \textit{NITRC}~\cite{12-bs}~--- это бесплатный веб-ре\-сурс, который 
предлагает информацию о постоянно расширяющемся наборе программного 
обеспечения и~данных для нейроинформатики. Он состоит из трех 
компонентов: реестра ресурсов (NITRC-R), репозитория изображений 
(NITRC-IR) и~вычислительной среды (NITRC-CE). \textit{Вычислительная 
среда NITRC-CE} представляет собой виртуальную облачную платформу, 
содержащую предустановленный набор программных средств для работы 
с~нейроизображениями. \textit{Репозиторий изображений NITRC} включает 
в~себя изображения в~форматах DICOM и~NIFTI.
    
    \textit{База данных, разработанная в~рамках проекта BNCI Horizon 
2020}~\cite{4-bs}, является общедоступной коллекцией наборов данных 
в~области~BCI. Цель создания базы данных~--- 
повышение научной про\-зрач\-ности и~эф\-фек\-тив\-ности. База данных 
способствует также валидации опубликованных методов и~способствует 
разработке новых алгоритмов. Данные могут храниться в~различных 
форматах ЭЭГ-дан\-ных.
    
\section{Форматы данных в~нейрофизиологии}

    В области нейрофизиологии в~настоящее время нет единых стандартов 
для хранения данных~\cite{13-bs}. Это относится как к~нейроизображениям, 
так и~к~биомедицинским сигналам. Многообразие форматов представления 
данных вызвано разнообразием как медицинского оборудования, так 
и~средств визуализации и~анализа получаемых данных.

\vspace*{-4pt}
    
\subsection{Форматы магнитно-резонансной томографии}

%\vspace*{-4pt}

    Данные нейрофизиологических изображений должны содержать не 
только сами изображения, но и~дополнительную информацию (метаданные), 
обеспечивающую интероперабельность и~повторное использование этих 
данных. Изображения без связанных с~ними метаданных практически 
бесполезны. К~метаданным относятся информация об изображении (размер 
пикселя, ширина и~высота изображения, число изображений), информация об 
оборудовании, информация об объекте наблюдения, информация 
о~положении объекта наблюдения относительно оборудования.
    
    Наиболее распространенные форматы пред\-став\-ле\-ния 
нейроизображений~--- DICOM
(digital imaging and communications in medicine)~[14], 
используемый в~большинстве 
медицинских сканеров, и~ANALYZE~7.5~[15], разработанный в~клинике 
Mayo в~рамках создания пакета программ Analyze для хранения, 
визуализации и~обработки многомерных биомедицинских изображений. 
Среди других форматов данных в~нейрофизиологии можно отметить 
NIFTI~[16] и~BIDS~[17].
    
    Форматы определяют, как изображения и~метаданные хранятся в~файле. 
Названия конкретных метаданных в~каждом формате свои, и~их число 
варь\-и\-ру\-ет\-ся от сотни (\mbox{ANALYZE}, NIFTI) до нескольких тысяч (DICOM). 
В~ряде форматов (\mbox{ANALYZE}, \mbox{NIFTI}, \mbox{BIDS}) определяется неизменяемый 
список используемых метаданных, и~любой файл должен содержать значения 
всех этих метаданных, даже если они неизвестны. Другие форматы (DICOM) 
используют гибкий набор метаданных, когда конкретные метаданные 
присутствуют в~файле только в~том случае, если они определены.
    
    Нейрофизиологические изображения пред\-ставля\-ются в~виде 
трехмерного массива вокселей, описывающего положение вокселей 
в~трехмерном пространстве. Также может добавляться четвертое 
измерение~--- время. Каждый формат определяет свой способ представления 
этого массива в~файле в~виде одномерной последовательности вокселей. 
Форматы отличаются способом задания ориентации изображения 
относительно сканера: неявная фиксированная ориентация (\mbox{ANALYZE}), 
кватернионы (NIFTI) и~направляющие косинусы (\mbox{DICOM}, NIFTI). Для 
определения ориентации объекта наблюдения относительно сканера 
используются в~основном два подхода: нейрологический (NIFTI) 
и~радиологический (DICOM).

\vspace*{-4pt}
    
\subsection{Форматы электроэнцефалографии}

%\vspace*{-4pt}

    В области хранения биомедицинских сигналов существует множество 
разнообразных форматов~[18]. Форматы определяют, как метаданные 
(заголовки) и~данные хранятся в~файле. Заголовки файлов (метаданные) 
обычно хранятся в~бинарном виде (EDF (European data format)~[19], 
GDF (general data format)~[20]), но в~некоторых 
форматах они хранятся в~текстовом виде или в~виде XML (OpenXDF~[21]). 
    
    Некоторые биомедицинские данные могут содержать различные виды 
биомаркеров, для этого форматы (EDF, GDF, OpenXDF) должны 
поддерживать частоту дискретизации и~коэффициенты масштабирования. 
Первоначально форматы поддерживали хранение 8-бит\-ных данных, затем  
16-бит\-ных (EDF), а~все последние форматы поддерживают и~типы данных 
более 16~бит (GDF, OpenXDF). При хранении данных важно знать 
физическую единицу записанного сигнала, т.\,е.\ представляют ли значения 
выборки милливольт (мВ) или микровольт (мкВ). Большинство форматов 
поддерживают все физические единицы, представленные в~стандарте 
ISO11073:10101. Но некоторые старые форматы (EDF, GDF~1.0) отводят на 
это только~8~байт, чего недостаточно для хранения всех единиц.
    
    Биомедицинские сигналы зачастую содержат артефакты. Часть 
форматов (GDF, OpenXDF) позволяют задавать диапазон изменения значения 
единиц, что позволяет автоматически находить некорректные данные. Для 
анализа больших баз данных и~архивов важно иметь доступную информацию 
(поддерживаются в~форматах OpenXDF, GDF~2.1) о~демографии пациентов, 
записывающем оборудовании, исследователе и~т.\,д.

\vspace*{-4pt}
    
\section{Программные средства работы с~нейрофизиологическими 
данными}

\vspace*{-4pt}

    Программные средства работы с~нейроизображениями помогают 
исследователям в~изучении мозга человека. Они позволяют визуализировать 
данные в~виде трехмерных изображений, применять различные методы 
анализа данных. 
    
    \textit{Средства визуализации} позволяют визуализировать как 2D-, так 
и~3D- и~4D-ней\-ро\-изоб\-ра\-же\-ния (3D Slicer~[22], Mango~[23]). Кроме 
визуализации они также позволяют выполнять операции над изображениями, 
такие как ручная сегментация и~создание трехмерной модели поверхности 
(3D Slicer), создание и~редактирование областей интереса в~изображениях, 
рендеринг поверхности, наложение изоб\-ра\-же\-ний (Mango).
    
    \textit{Средства анализа нейроизображений} позволяют применять 
различные методы информатики для анализа нейроизображений. Написанное 
на \mbox{MATLAB} программное обеспечение \textit{CONN}~[24] предназначено 
для вычисления, визуализации и~ана\-ли\-за функциональных связностей 
в~\mbox{фМРТ}. Пред\-остав\-ля\-ют\-ся также функции обнаружения и~очистки 
артефактов, динамического анализа связности и~анализа на основе 
по\-верх\-ности и~\mbox{объема}. Сис\-те\-ма \textit{SPM}~[25] предназначена для 
статистического параметрического картирования, используемого для 
определения различий в~зарегистрированной активности мозга 
с~использованием пространственно расширенных статистических процессов.
    
    Набор инструментов для работы с~ЭЭГ \textit{NBT}~[26] обеспечивает 
расчет и~интеграцию ней\-ро\-фи\-зио\-логических биомаркеров. NBT предлагает 
конвей\-ер, включающий различные этапы обработки данных: от хранения 
данных до применения статистических методов, вычисление отклонения 
\mbox{артефактов}, визуализацию сигналов, вычисление биомаркеров 
и~статистическое тестирование.\linebreak Программные средства 
\textit{EEGLAB}~[27], \textit{FieldTrip}~[28] и~\textit{BioSig}~[29], 
реализованные в~MATLAB, предназначены для обработки биомедицинских 
сигналов, таких как ЭЭГ, МЭГ и~других электрофизиологических сигналов. 
EEGLAB реализует\linebreak метод независимых компонент,  
час\-тот\-но-вре\-мен\-ной анализ, вычисление отклонения артефактов 
и~несколько режимов визуализации данных. \mbox{FieldTrip} предлагает методы 
предварительной об\-работки и~расширенного анализа, такие как 
час\-тот\-но-вре\-мен\-ной анализ, восстановление источников с~использованием диполей, 
распределенных источников и~непараметрическое статистическое 
тес\-тирование. BioSig предоставляет средства визуализации данных 
и~средства для сбора данных, обработки артефактов, контроля качества, 
извлечения характеристик, классификации, моделирования данных.
{ %\looseness=1

}
    
    \textit{Библиотеки обработки нейроизображений на языке Python} 
помогают разрабатывать собственные программы для работы 
с~нейроизображениями.
    
    Библиотека \textit{NiPy}~[30]~--- библиотека, состоящая из нескольких 
частей, которые позволяют пользователю выполнять как простые операции 
с~изоб\-ра\-же\-ни\-ями fMRI (например, чтение и~запись), так и~сложные алгоритмы 
анализа нейроизображений. \textit{Nibabel} предоставляет прикладной 
программный интерфейс для чтения и~записи различных форматов файлов 
нейроизображений, таких как \mbox{ANALYZE}, NIFTI, MINC, MGH. 
\textit{Niwidgets} пред\-остав\-ля\-ет средства визуализации нейроизображений. 
Nitime пред\-остав\-ля\-ет средства для анализа временн$\acute{\mbox{ы}}$х рядов в~области 
нейровизуализации. \textit{Nilearn} пред\-остав\-ля\-ет средства для 
статистического исследования данных нейровизуализации на основе метода 
независимых компонент CanICA.
    
    \textit{MNE-Python}~[31]~--- это программный пакет с~открытым 
исходным кодом, предназначенный для анализа данных 
МЭГ и~ЭЭГ. Он пред\-остав\-ля\-ет 
современные алгоритмы, которые охватывают несколько методов 
предварительной обработки данных, локализации источников, 
статистического анализа, методы машинного обучения.

\vspace*{-6pt}

\section{Заключение}

 \vspace*{-3pt}

    Нейрофизиологию можно рассматривать как область с~интенсивным 
использованием данных, где данные играют ключевую роль в~исследованиях 
в~понимании работы головного мозга и~в~обнаружении и~лечении 
заболеваний, связанных с~головным мозгом. По всему миру создаются 
проекты, поддерживающие исследования в~этой области. Рост числа 
исследований и~появление нового оборудования ведут к~лавинообразному 
увеличению объема данных. Эти данные могут представляться в~различных 
форматах. Требуются новые средства для хранения данных больших объемов 
(которые могут достигать нескольких петабайт), средства для интеграции 
данных, представленных в~разных форматах, средства анализа такого объема 
данных. Отдельные компьютеры больше не подходят для анализа данных 
в~области нейрофизиологии, а потому необходимо развивать новые 
инфраструктуры, позволяющие хранить и~обрабатывать такие объемы 
данных. Актуальные задачи в~области нейрофизиологии требуют применения 
современных методов анализа данных, включая статистический анализ 
и~машинное обучение, реализованных в~распределенных вычислительных 
инфраструктурах.

\vspace*{-6pt}


{\small\frenchspacing
 {%\baselineskip=10.8pt
 \addcontentsline{toc}{section}{References}
 \begin{thebibliography}{99}
 
 \vspace*{-3pt}
 
  \bibitem{1-bs}
  BRAIN Initiative. {\sf https://braininitiative.nih.gov}.
  \bibitem{2-bs}
  Human Brain Project home page. {\sf https://www.\linebreak humanbrainproject. eu}.
  \bibitem{3-bs}
  \Au{Elam J.\,S., Van Essen~D.} Human Connectome project~// Encyclopedia of 
computational neuroscience~/
Eds. D.~Jaeger, R.~Jung.~--- New York, NY, USA: Springer, 2013. 4~p.
  \bibitem{4-bs}
  \Au{Brunner C., Blankertz~B., Cincotti~F., \textit{et al.}} BNCI Horizon  
2020~--- towards a~roadmap for brain/neural computer interaction~// 8th 
Conference (International) on Universal Access in Human--Computer Interaction 
Proceedings.~--- Lecture notes in computer science ser.~--- Springer, 2014. 
Vol.~8513. P.~475--486. 
  \bibitem{5-bs}
  \Au{Jiang T., Liu~Y., Shi~F., Shu~N., Liu~B., Jiang~J., Zhou~Y.} Multimodal 
magnetic resonance imaging for brain disorders: Advances and perspectives~// 
Brain Imaging Behav., 2008. Vol.~2. Iss.~4. P.~249--257.
  \bibitem{6-bs}
  \Au{Van Horn J.\,D., Toga~A.\,W.} Multisite neuroimaging trials~// Curr. 
Opin. Neurol., 2009. Vol.~22. Iss.~4. P.~370--378. 
  \bibitem{7-bs}
  \Au{Jack C.\,R., Bernstein~M.\,A., Fox~N.\,C., \textit{et al.}} The Alzheimer's 
disease neuroimaging initiative (ADNI): MRI methods~// J.~Magn. Reson. 
Imaging, 2008. Vol.~27. Iss.~4. P.~685--691.
  \bibitem{8-bs}
  \Au{Biswal B.\,B., Mennes~M., Zuo~X.\,N., \textit{et al.}} Toward discovery 
science of human brain function~// P.~Natl. Acad. Sci. USA, 
2010. Vol.~107. Iss.~10. P.~4734--4739.
  \bibitem{9-bs}
  \Au{Poldrack R.\,A., Barch~D.,M., Mitchell~J., \textit{et al.}} Toward open 
sharing of task-based fMRI data: The OpenfMRI project~// Front. Neuroinform., 
2013. Vol.~7. Art. No.\,12. P.~1--12.
  \bibitem{10-bs}
  \Au{Hodge M.\,R., Horton~W., Brown~T., \textit{et al.}}  
ConnectomeDB-sharing human brain connectivity data~// NeuroImage, 2016. 
Vol.~124. P.~1102--1107.
  \bibitem{11-bs}
  \Au{Marcus D., Olsen~T.\,R., Ramaratnam~M., Buckner~R.\,L.} The extensible 
neuroimaging archive toolkit (XNAT): An informatics platform for managing, 
exploring, and sharing neuroimaging data~// Neuroinformatics, 2007. Vol.~5. 
P.~11--34.
  \bibitem{12-bs}
  NITRC home page. {\sf https://www.nitrc.org}.
  \bibitem{13-bs}
  \Au{Neu S.\,C., Crawford~K.\,L., Toga~A.\,W.} Practical management of 
heterogeneous neuroimaging metadata by global neuroimaging data repositories~// 
Front. Neuroinform., 2012. Vol.~6. Art. No.\,8. P.~1--9.
  \bibitem{14-bs}
  Digital Imaging Communication in Medicine (DICOM): NEMA Standards 
Publication PS~3.~--- Washington, DC, USA: National Electrical Manufacturers 
Association, 1999.
  \bibitem{15-bs}
  ANALYZE 7.5 file format. {\sf http://eeg.sourceforge.net/ ANALYZE75.pdf}.
  \bibitem{16-bs}
  NIFTI home page. {\sf http://nifti.nimh.nih.gov}.
  \bibitem{17-bs}
  The Brain Imaging Data Structure (BIDS) specification. {\sf 
https://bids.neuroimaging.io/bids\_spec.pdf}.
  \bibitem{18-bs}
  \Au{Schl$\ddot{\mbox{o}}$gl A.} An overview on data formats for biomedical 
signals~// World Congress on Medical Physics and Biomedical Engineering.~--- 
Berlin--Heidelberg: Springer, 2009. P.~1557--1560.
  \bibitem{19-bs}
  \Au{Kemp B., V$\ddot{\mbox{a}}$rri~A., Rosa~A.\,C., Nielsen~K.\,D., 
Gade~J.} A~simple format for exchange of digitized polygraphic recordings~// 
Electroen. Clin. Neuro., 1992. Vol.~82. Iss.~5. 
P.~391--393.
  \bibitem{20-bs}
  \Au{Schl$\ddot{\mbox{o}}$gl~A.} GDF~--- a~general data format for 
biomedical signals~// arXiv.org, 11~Aug 2006 (v.~1), 26~Mar 2013 (v.~10). 
arxiv:cs/0608052.
  \bibitem{21-bs}
  \Au{Smith J., Johnson~J., Schubert~J., Widell~R.} A~new format for 
polysomnography data~// Sleep, 2005. Vol.~28. Iss.~11. P.~1473--1473.
  \bibitem{22-bs}
  \Au{Fedorov A., Beichel~R., Kalpathy-Cramer~J., \textit{et al.}} 3D slicer as 
an image computing platform for the quantitative imaging network~// Magn. 
Reson. Imaging, 2012. Vol.~30. Iss.~9. P.~1323--1241.
  \bibitem{23-bs}
  \Au{Sadigh-Eteghad S., Majdi~A., Farhoudi~M., Talebi~M., Mahmoudi~J.} 
Different patterns of brain activation in normal aging and Alzheimer's disease from 
cognitional sight: Meta analysis using activation likelihood estimation~// 
J.~Neurol. Sci., 2014. Vol.~343. Iss.~1-2. P.~159--166. 
  \bibitem{24-bs}
  \Au{Whitfield-Gabrieli S., Nieto-Castanon~A.} Conn: A~functional 
connectivity toolbox for correlated and anticorrelated brain networks~// Brain 
Connectivity, 2012. Vol.~2. Iss.~3. P.~125--141. 
  \bibitem{25-bs}
  \Au{Friston~K.\,J., Ashburner~J.\,T., Kiebel~S.\,J., 
Nichols~T.\,E., Penny W.\,D.} Statistical parametric mapping: The analysis of functional brain 
images: The analysis of functional brain images.~--- Academic Press, 2011. 688~p.
  \bibitem{26-bs}
  \Au{Poil S.} Neurophysiological Biomarkers of cognitive decline: From 
criticality to toolbox.~--- Amsterdam: VU University, 2013. 218~p.
  \bibitem{27-bs}
  \Au{Delorme A., Makeig~S.} EEGLAB: An open source toolbox for analysis of 
single-trial EEG dynamics~// J.~Neurosci. Meth., 2004. Vol.~134. P.~9--21.
  \bibitem{28-bs}
  \Au{Oostenveld R., Fries~P., Maris~E., Schoffelen~J.\,M.} FieldTrip: Open 
source software for advanced analysis of MEG, EEG, and invasive 
electrophysiological data~// Comput. Intell. Neurosc., 2011. 
Vol.~2011. Art. ID: 156869. P.~1--9.
  \bibitem{29-bs}
  \Au{Vidaurre C., Sander~T.\,H., Schl$\ddot{\mbox{o}}$gl~A.} BioSig: The free 
and open source software library for biomedical signal processing~// 
Comput. Intell. Neurosc., 2011. Vol.~2011. Art. ID: 935364. 
P.~1--12.
  \bibitem{30-bs}
  \Au{Brett M., Taylor~J., Burns~C., \textit{et al.}} NIPY: An open library and 
development framework for FMRI data analysis~// NeuroImage, 2009. Vol.~47. 
Suppl.~1. P.~S196.
  \bibitem{31-bs}
  \Au{Gramfort A., Luessi~M., Larson~E., \textit{et al.}}  MEG and EEG data 
analysis with MNE-Python~// Front. Neurosci., 2013. Vol.~7. Art. No.\,267.  
P.~1--13.
 \end{thebibliography}

 }
 }

\end{multicols}

\vspace*{-3pt}

\hfill{\small\textit{Поступила в~редакцию 14.11.19}}

%\vspace*{8pt}

%\pagebreak

\newpage

\vspace*{-28pt}

%\hrule

%\vspace*{2pt}

%\hrule

%\vspace*{-2pt}

\def\tit{NEUROPHYSIOLOGY AS~A~SUBJECT DOMAIN\\ FOR~DATA 
INTENSIVE PROBLEM SOLVING}


\def\titkol{Neurophysiology as~a~subject domain for~data 
intensive problem solving}

\def\aut{D.\,O.~Briukhov, S.\,A.~Stupnikov, D.\,Yu.~Kovalev, and~I.\,A.~Shanin}

\def\autkol{D.\,O.~Briukhov, S.\,A.~Stupnikov, D.\,Yu.~Kovalev, and~I.\,A.~Shanin}

\titel{\tit}{\aut}{\autkol}{\titkol}

\vspace*{-11pt}


\noindent
Institute of Informatics Problems, Federal Research Center ``Computer Science 
and Control'' of the Russian Academy of Sciences, 44-2~Vavilov Str., Moscow 
119333, Russian Federation

\def\leftfootline{\small{\textbf{\thepage}
\hfill INFORMATIKA I EE PRIMENENIYA~--- INFORMATICS AND
APPLICATIONS\ \ \ 2020\ \ \ volume~14\ \ \ issue\ 1}
}%
 \def\rightfootline{\small{INFORMATIKA I EE PRIMENENIYA~---
INFORMATICS AND APPLICATIONS\ \ \ 2020\ \ \ volume~14\ \ \ issue\ 1
\hfill \textbf{\thepage}}}

\vspace*{6pt} 
  


\Abste{The goal of this survey is to analyze neurophysiology as a data intensive domain. 
Nowadays, the number of researches on the human brain is increasing. International projects and 
researches are aimed at improvement of the understanding of the human brain function. The 
amount of data obtained in typical laboratories in the field of neurophysiology is growing 
exponentially. The data are represented using a~large number of various formats. This requires 
creation of infrastructures, databases, and websites that provide unified access to data and support 
the exchange of data between researchers all over the world. Specific methods and tools forming 
the field of neuroinformatics (that is, an intersection of neurophysiology and computer science) 
are used to analyze collected data and to solve neurophysiological problems. These methods 
include, in particular, statistical analysis, machine learning, and neural networks.}

\KWE{neurophysiology; neurophysiological resources; neuroinformatics; data intensive 
research; analysis of neurophysiological data}



\DOI{10.14357/19922264200106} 

%\vspace*{-14pt}

\Ack
\noindent
This research was partially supported by the Russian Foundation 
for Basic Research (project   18-29-22096).


 


\vspace*{6pt}

  \begin{multicols}{2}

\renewcommand{\bibname}{\protect\rmfamily References}
%\renewcommand{\bibname}{\large\protect\rm References}

{\small\frenchspacing
 {%\baselineskip=10.8pt
 \addcontentsline{toc}{section}{References}
 \begin{thebibliography}{99}
\bibitem{1-bs-1}
BRAIN Initiative Home Page. Available at: {\sf https://\linebreak braininitiative.nih.gov/} 
(accessed November~12, 2019)
\bibitem{2-bs-1}
Human Brain Project Home Page. Available at: {\sf 
https://\linebreak www.humanbrainproject.eu} (accessed November~12, 2019).
\bibitem{3-bs-1}
\Aue{Elam, J.\,S., and D.~Van Essen.} 2013. Human Connectome project. 
\textit{Encyclopedia of computational neuroscience}. Eds. D.~Jaeger and R.~Jung.
New York, NY: Springer.  4~p.
\bibitem{4-bs-1}
\Aue{Brunner, C., B.~Blankertz, F.~Cincotti, \textit{et al.}} 2014. 
\mbox{BNCI} Horizon 
2020~--- towards a~roadmap for brain/neural computer interaction. \textit{8th 
 Conference (International) on Universal Access in Human--Computer 
Interaction Proceedings.} Lecture notes in computer science ser. Springer. 
8513: 475--486.
\bibitem{5-bs-1}
\Aue{Jiang, T., Y.~Liu, F.~Shi, N.~Shu, B.~Liu, J.~Jiang, and Y.~Zhou.} 2008. 
Multimodal magnetic resonance imaging for brain disorders: advances and 
perspectives. \textit{Brain Imaging Behav.} 2(4):249--257.
\bibitem{6-bs-1}
\Aue{Van Horn, J.\,D., and A.\,W.~Toga.} 2009. Multisite neuroimaging trials. 
\textit{Curr. Opin. Neurol.} 22(4):370--378. 
\bibitem{7-bs-1}
\Aue{Jack, C.\,R., M.\,A.~Bernstein, N.\,C.~Fox, \textit{et al.}} 2008. The 
Alzheimer's disease neuroimaging initiative (ADNI): MRI methods. 
\textit{J.~Magn. Reson. Imaging} 27(4):685--691.
\bibitem{8-bs-1}
\Aue{Biswal, B.\,B., M.~Mennes, X.\,N.~Zuo, \textit{et al.}} 2010. Toward 
discovery science of human brain function. \textit{P.~Natl. 
Acad. Sci. USA} 107(10):4734--4739.
\bibitem{9-bs-1}
\Aue{Poldrack, R.\,A., D.\,M.~Barch, J.~Mitchell, \textit{et al.}} 2013. Toward 
open sharing of task-based fMRI data: The \mbox{OpenfMRI} project. \textit{Front. 
Neuroinform.} 7:12.
\bibitem{10-bs-1}
\Aue{Hodge, M.\,R., W.~Horton, T.~Brown, \textit{et al.}} 2016. 
ConnectomeDB-sharing human brain connectivity data. \textit{NeuroImage} 
124:1102--1107.
\bibitem{11-bs-1}
\Aue{Marcus, D., T.\,R.~Olsen, M.~Ramaratnam, and R.\,L.~Buckner.} 2007. The 
extensible neuroimaging archive toolkit (XNAT): An informatics platform for 
managing, exploring, and sharing neuroimaging data. 
\textit{Neuroinformatics} 5:11--34.
\bibitem{12-bs-1}
NITRC Home Page. Available at: {\sf https://www.nitrc.org/} (accessed 
November~12, 2019).
\bibitem{13-bs-1}
\Aue{Neu, S.\,C., K.\,L.~Crawford, and A.\,W.~Toga.} 2012. Practical 
management of heterogeneous neuroimaging metadata by global neuroimaging 
data repositories. \textit{Front. Neuroinform.} 6:8.
\bibitem{14-bs-1}
Digital Imaging Communication in Medicine (DICOM). 1999. NEMA Standards 
Publication PS~3. Washington, DC: National Electrical Manufacturers 
Association.
\bibitem{15-bs-1}
ANALYZE 7.5 file format. Available at: {\sf 
http://eeg.\linebreak sourceforge.net/ANALYZE75.pdf} (accessed November~12, 2019).
\bibitem{16-bs-1}
NIFTI home page. Available at: {\sf http://nifti.nimh.nih.gov} (accessed 
November~12, 2019).
\bibitem{17-bs-1}
The Brain Imaging Data Structure (BIDS) specification. Available at: {\sf 
https://bids.neuroimaging.io/bids\_spec.pdf} (accessed November~12, 2019).
\bibitem{18-bs-1}
\Aue{Schl$\ddot{\mbox{o}}$gl,~A.} 2009. An overview on data formats for 
biomedical signals. \textit{World Congress on Medical Physics and 
Biomedical Engineering}. Berlin--Heidelberg: Springer. 1557--1560.
\bibitem{19-bs-1}
\Aue{Kemp, B., A.~V$\ddot{\mbox{a}}$rri, A.\,C.~Rosa, K.\,D.~Nielsen, and 
J.~Gade.} 1992. A~simple format for exchange of digitized polygraphic 
recordings. \textit{Electroen. Clin. Neuro.} 
82(5):391--393.
\bibitem{20-bs-1}
\Aue{Schl$\ddot{\mbox{o}}$gl, A.} GDF~--- a~general data format for 
biomedical signals Version~2.51. Available at: {\sf 
https://arxiv.org/\linebreak abs/cs/0608052} (accessed November~12, 2019).
\bibitem{21-bs-1}
\Aue{Smith, J., J.~Johnson, J.~Schubert, and R.~Widell.} 2005. A~new file 
format for polysomnography data. \textit{Sleep} 28(11):1473--1473.
\bibitem{22-bs-1}
\Aue{Fedorov, A., R.~Beichel, J.~Kalpathy-Cramer, \textit{et al.}} 2012. 3D 
slicer as an image computing platform for the quantitative imaging network. 
\textit{Magn. Reson. Imaging} 30(9):1323--1241. 
\bibitem{23-bs-1}
\Aue{Sadigh-Eteghad, S., A.~Majdi, M.~Farhoudi, M.~Talebi, and 
J.~Mahmoudi.} 2014. Different patterns of brain activation in normal aging 
and Alzheimer's disease from cognitional sight: Meta analysis using activation 
likelihood estimation. \textit{J.~Neurol. Sci.} 343(1-2):159--166.



\bibitem{24-bs-1}
\Aue{Whitfield-Gabrieli, S., and A.~Nieto-Castanon.} 2012. Conn: A~functional 
connectivity toolbox for correlat-\linebreak\vspace*{-12pt}

\columnbreak 

\noindent
ed and anticorrelated brain networks. 
\textit{Brain Connectivity} 2(3):125--141.
\bibitem{25-bs-1}
\Aue{Friston, K.\,J., J.\,T.~Ashburner, S.\,J.~Kiebel, 
T.\,E.~Nichols, and W.\,D.~Penny.} 2011. \textit{Statistical parametric mapping: The analysis of 
functional brain images}. Academic Press. 688~p.
\bibitem{26-bs-1}
\Aue{Poil, S.} 2013. \textit{Neurophysiological Biomarkers of cognitive decline: 
From criticality to toolbox}. Amsterdam: VU University. 218~p.
\bibitem{27-bs-1}
\Aue{Delorme, A., and S.~Makeig.} 2004. EEGLAB: An open source toolbox for 
analysis of single-trial EEG dynamics. \textit{J.~Neurosci. Meth.} 
134:9--21.
\bibitem{28-bs-1}
\Aue{Oostenveld, R., P.~Fries, E.~Maris, and J.\,M.~Schoffelen.} 2011. FieldTrip: 
Open source software for advanced analysis of MEG, EEG, and invasive 
electrophysiological data. \textit{Comput. Intell. 
Neurosc.} 2011:156869.
\bibitem{29-bs-1}
\Aue{Vidaurre, C., T.\,H.~Sander, and A.~Schl$\ddot{\mbox{o}}$gl.} 2011. 
BioSig: The free and open source software library for biomedical signal 
processing. \textit{Comput. Intell. Neurosc.} 
2011:935364. 
  \bibitem{30-bs-1}
\Aue{Brett, M., J.~Taylor, C.~Burns, \textit{et al.}} 2009. NIPY: An open library 
and development framework for FMRI data analysis. \textit{NeuroImage} 
47:S196.
  \bibitem{31-bs-1}
\Aue{Gramfort, A., M.~Luessi, and E.~Larson.} 2013. EEG data analysis with 
MNE-Python. \textit{Front. Neurosci.} 7:267.
%\vspace*{-18pt}

\end{thebibliography}

 }
 }

\end{multicols}

%\vspace*{-7pt}

\hfill{\small\textit{Received November 14, 2019}}

%\pagebreak

%\vspace*{-22pt}

\Contr

    \noindent
\textbf{Briukhov Dmitry O.} (b.\ 1971)~--- Candidate of Science (PhD) in technology, senior 
scientist, Institute of Informatics Problems, Federal Research Center ``Computer Science and 
Control'' of the Russian Academy of Sciences, 44-2~Vavilov Str., Moscow 119333, Russian 
Federation; \mbox{dbriukhov@ipiran.ru}
    
    \vspace*{3pt}
    
    \noindent
    \textbf{Stupnikov Sergey A.} (b.\ 1978)~--- Candidate of Science (PhD) in 
technology, lead scientist, Institute of Informatics Problems, Federal Research 
Center ``Computer Science and Control'' of the Russian Academy of Sciences,  
44-2~Vavilov Str., Moscow 119333, Russian Federation; 
\mbox{sstupnikov@ipiran.ru}
    
    \vspace*{3pt}
    
    \noindent
    \textbf{Kovalev Dmitry Yu.} (b.\ 1988)~--- junior scientist, Institute of 
Informatics Problems, Federal Research Center ``Computer Science and Control'' 
of the Russian Academy of Sciences, 44-2~Vavilov Str.,  Moscow 119333, Russian 
Federation; \mbox{dkovalev@ipiran.ru}
    
    \vspace*{3pt}
    
    \noindent
    \textbf{Shanin Ivan A.} (b.\ 1991)~--- junior scientist, Institute of Informatics 
Problems, Federal Research Center ``Computer Science and Control'' of the 
Russian Academy of Sciences, 44-2~Vavilov Str., Moscow 119333, Russian 
Federation; \mbox{v08shanin@gmail.com}
\label{end\stat}

\renewcommand{\bibname}{\protect\rm Литература}  %6
\def\stat{danilishin}

\def\tit{ОЦЕНКА СТОИМОСТИ ОПЦИОНОВ НА~ОСНОВЕ 
МОДЕЛЕЙ ARIMA--GARCH С~ОШИБКАМИ, РАСПРЕДЕЛЕННЫМИ 
ПО~ЗАКОНУ~$S_U$ ДЖОНСОНА}

\def\titkol{Оценка стоимости опционов на основе моделей  
ARIMA--GARCH с~ошибками, распределенными по закону~JSU} %$S_U$  Джонсона}

\def\aut{А.\,Р.~Данилишин$^1$, Д.\,Ю.~Голембиовский$^2$}

\def\autkol{А.\,Р.~Данилишин, Д.\,Ю.~Голембиовский}

\titel{\tit}{\aut}{\autkol}{\titkol}

\index{Данилишин А.\,Р.}
\index{Голембиовский Д.\,Ю.}
\index{Danilishin A.\,R.}
\index{Golembiovsky D.\,Yu.}


%{\renewcommand{\thefootnote}{\fnsymbol{footnote}} \footnotetext[1]
%{Работа выполнена при частичной поддержке РФФИ (проект 19-07-00187-A).}}


\renewcommand{\thefootnote}{\arabic{footnote}}
\footnotetext[1]{Московский государственный университет имени М.\,В.~Ломоносова, факультет 
вычислительной математики и~кибернетики, \mbox{danilishin-artem@mail.ru}}
\footnotetext[2]{Московский государственный университет имени М.\,В.~Ломоносова, факультет 
вычислительной математики и~кибернетики; Московский фи\-нан\-со\-во-про\-мыш\-лен\-ный университет 
<<Синергия>>, \mbox{golemb@cs.msu.su}}

%\vspace*{-12pt}


\Abst{В продолжение статьи <<Риск-нейтральная динамика для модели ARIMA--GARCH 
с~ошибками, распределенными по закону $S_U$ Джонсона>> в~данной работе приводятся 
результаты экспериментов для моделей ARIMA--GARCH 
(autoregressive integrated moving average\,--\,generalized autoregressive conditional heteroskedasticity)
с~нормальными (N), 
экспоненциальными бета второго типа (EGB2) и~$S_U$ Джонсона (JSU) распределениями 
ошибок. Стоимость европейских опционов оценивается методом Мон\-те-Кар\-ло на основе 
результатов, полученных в~указанной статье при помощи расширенного принципа 
Гирсанова. Параметры моделей ARIMA--GARCH-N, ARIMA--GARCH-EGB2 и~ARIMA--GARCH-JSU 
были найдены методом квазимаксимального правдоподобия. Эффективность 
полученных риск-нейтральных моделей исследовалась на примере биржевых европейских 
опционов PUT и~CALL на базовые активы DAX  (Deutscher Aktienindex)
и~Light Sweet Crude Oil. }

\KW{ARIMA; GARCH; риск-нейтральная мера; расширенный принцип Гирсанова; 
распределение $S_U$ Джонсона; ценообразование опционов}

\DOI{10.14357/19922264200412} 
  
%\vspace*{9pt}


\vskip 10pt plus 9pt minus 6pt

\thispagestyle{headings}

\begin{multicols}{2}

\label{st\stat}


\section{Введение}

Данная работа является продолжением \mbox{статьи}~[1]. В~указанной статье была 
введена модель  
ARIMA$\,(p,d,q)$--GARCH$\,(P,Q)$-JSU$\,(\xi,\lambda, \gamma,\delta)$ для 
доходности базового актива в~следующем виде (для распределения JSU 
$\tilde{Y}_t\hm= S_t/S_{t-1}\hm -1$, для N- и~EGB2-рас\-пре\-де\-ле\-ний 
$Y_t\hm= \ln(S_t/S_{t-1})$)~[2--5]:
\begin{equation}
\left.
\begin{array}{l}
\!\!\!\Delta^d Y_t=m_t +\sqrt{h_t}\,\varepsilon_t\,,\ \varepsilon_t\vert \sim 
\mathrm{JSU}(\xi,\lambda,\gamma,\delta)\,;\\[12pt]
\!\!\!m_t=\mathbb{E}\left[ \Delta^dY_t\vert \mathcal{F}_{t-1}\right]=
\phi_0+\cdots+\phi_p\Delta^d Y_{t-p} +{}\\[6pt]
\!\!\!\hspace*{10mm}{}+\theta_1\sqrt{h_{t-1}}\,\varepsilon_{t-1}+
\cdots+\theta_q \sqrt{h_{t-q}}\,\varepsilon_{t-q}\,;\\[12pt]
\!\!\!h_t=\mathrm{Var}\left[ \Delta^d Y_t\vert \mathcal{F}_{t-1}\right]=
\alpha_0+\cdots+\alpha_P h_{t-P} +{}\\[6pt]
\!\!\!\hspace*{15mm}{}+\beta_1h_{t-1}\varepsilon^2_{t-1}+\cdots+\beta_Q h_{t-Q} 
\varepsilon^2_{t-Q}
\end{array} \!\!
\right\} \!\!
\label{e1-dan}
\end{equation}
с ограничениями на математическое ожидание и~дисперсию ошибки
\begin{align*}
\mathbb{E}[\varepsilon_t] &=\xi-\lambda e^{1/(2\delta^2)}\mathrm{sinh}\left(
\fr{\gamma}{\delta}\right)=0\,;\\
\mathrm{Var}\left[\varepsilon_t\right]&=\fr{\lambda^2}{2}\left( e^{1/\delta^2}-1\right)\times{}\\
&\hspace*{10mm}{}\times \left( 
e^{1/\delta^2} \mathrm{cosh}\left( \fr{2\gamma}{\delta}\right) +1\right)=1\,.
\end{align*}

%\label{e2-dan}

  Далее к~разностному (стационарному) ряду~(\ref{e1-dan}) применялся 
расширенный принцип Гирсанова~[6, 7], а~в~случае распределения~JSU~---
 его найденная модификация. Для распределения JSU %$S_U$ Джонсона 
полученные коэффициенты модели, обес\-пе\-чи\-ва\-ющие риск-ней\-т\-раль\-ную 
динамику процесса даются сле\-ду\-ющи\-ми соотношениями:
  \begin{equation}
  \left.
  \begin{array}{rl}
  Y_t&=r+\delta_t\fr{1+r}{1+m_t}\,\varepsilon_t\,;\\[9pt]
   \varepsilon_{t\vert
  \mathcal{F}_{t-1}} &\sim \mathrm{JSU}\left( 
\tilde{\xi},\tilde{\lambda},\gamma,\delta\right);\\[9pt]
  \tilde{\xi}&=\tilde{\lambda}e^{1/(2\delta^2)}\mathrm{sinh}\left( \fr{\gamma} 
{\delta} \right);\\[9pt] 
\tilde{\lambda}&=\sqrt{2}\left( \left( e^{1/\delta^2}-1\right)\times{}\right.\\[9pt]
&\hspace*{3mm}\left.{}\times\left(
  e^{1/\delta^2}\mathrm{cosh}\left( \fr{2\gamma}{\delta}\right)+1\right)\right)^{-1/2}.
  \end{array}
  \right\}
  \label{e3-dan}
  \end{equation}
  
  Риск-нейтральные коэффициенты для моделей ARIMA--GARCH-EGB2  
и~ARIMA--GARCH-N представлены соотношениями~\cite{7-dan}:
  \begin{multline}
  Y_t=r-\ln\fr{B(\alpha+\delta_t\overline{\delta},\beta-
\delta_t\overline{\delta})}{B(\alpha,\beta)}+{}\\
{}+\delta_t\overline{\delta}\overline{\omega}(\alpha,\beta)+
  \delta_t\varepsilon_t\,;
  \label{e4-dan}
  \end{multline}
  
  \noindent
  \begin{equation}
  \left.
  \begin{array}{c}
  \varepsilon_{t\vert\mathcal{F}_{t-1}}\sim 
\mathrm{EGB2}\left(\alpha,\beta,\overline{\delta},\overline{\mu}\right);\\[6pt]
  \overline{\delta}=\fr{1}{\sqrt{l(\alpha,\beta)}};\quad
   \overline{\mu}=-
\fr{\overline{\omega}(\alpha,\beta)}{\sqrt{l(\alpha,\beta)}};\\[12pt]
  Y_t=r-\fr{1}{2}\,\delta^2_t+\delta_t \varepsilon_t\,; \quad 
\varepsilon_{t\vert\mathcal{F}_{t-1}} \sim N(0,1)\,.
  \end{array}
  \right\}
  \label{e5-dan}
  \end{equation}
  
  В данной статье приводятся результаты чис\-лен\-ных экспериментов, 
подтверждающие корректность теоретических результатов первой работы 
и~эффективность полученных риск-ней\-траль\-ных моделей  
ARIMA--GARCH-N, ARIMA--GARCH-EGB2 и~ARIMA--GARCH-JSU. 
  
  Работа построена следующим образом. В~разд.~2 приводятся формулы для 
оценки справедливой стоимости опционов CALL и~PUT методом Мон\-те-Кар\-ло. 
В~разд.~3 описываются два набора данных для проведения 
численных экспериментов: цены закрытия торгов по опционам на индекс 
немецких акций DAX и~на нефть Light Sweet Crude Oil, а~также 
соответствующие ряды цен базовых активов. В~разд.~4 даются формулы 
оценки параметров моделей ARIMA--GARCH методом квазимаксимального 
правдоподобия~\cite{8-dan}. В~разд.~5 приводятся тесты, определяющие 
спецификации моделей и~результаты оценок параметров моделей. Раздел~6 
содержит результаты оценки справедливой стоимости опционов, полученные 
при использовании различных моделей динамики базовых активов. 
В~заключении формулируются выводы исследования.

\vspace*{-6pt}

\section{Оценка справедливой стоимости опциона методом  
Монте-Карло}

\vspace*{-3pt}

  Оценка справедливой стоимости опционов проводится по методу Мон\-те-Кар\-ло~\cite{9-dan}. 
  Европейские опционы CALL и~PUT с~ценой 
исполнения~$X$ и~стоимостью базового актива~$S_T$ в~день 
экспирации~$T$ характеризуются функциями выплат $b_c (S_T,X)\hm= \max 
(S_T\hm - X,0)$ и~$b_p(S_T, X)\hm= \max (X\hm- S_T,0)$. Стоимость опционов 
определяется как среднее значение соответствующей функции выплаты 
относительно риск-ней\-траль\-ной меры~$\mathbb{Q}$, приведенное 
к~текущему моменту времени~\cite{10-dan}:
  \begin{equation}
\! \fr{p_{\mathrm{call}}}{p_{\mathrm{put}}} \!=\!\fr{\mathbb{E}^{\mathbb{Q}} [b_{c/p}(S_T,X)]} {(1+r)^T} 
\!=\! \fr{\mathbb{E}^{\mathbb{P}} [b_{c/p}(S_T,X)d\mathbb{Q}/d\mathbb{P}]} 
{(1+r)^T},\!\!
  \label{e6-dan}
  \end{equation}
где $d\mathbb{Q}/d\mathbb{P}$~--- производная Ра\-до\-на--Ни\-ко\-ди\-ма  
риск-ней\-т\-раль\-ной меры (в~рамках данной работы это мера, полученная на 
основе расширенного принципа Гирсанова либо его модификации) 
относительно физической меры~$\mathbb{P}$~\cite{11-dan, 12-dan}. Метод 
Мон\-те-Кар\-ло позволяет по реализациям построенного процесса ARIMA--GARCH 
оценить среднее значение относительно  
риск-ней\-т\-раль\-ной меры~$\mathbb{Q}$:
\vspace*{-6pt}

\noindent
\begin{multline}
\fr{1}{M}\sum\limits_{m=1}^M b_{c/p}\left( S_T, X\right) 
\fr{d\mathbb{Q}}{d\mathbb{P}}\,(m) 
\xrightarrow[M\to\infty]{P}{} \\
{}\xrightarrow[M\to\infty]{P}
\mathbb{E}^{\mathbb{P}} \left[ b_{c/p}\left( 
S_T,X\right)\fr{d\mathbb{Q}}{d\mathbb{P}}\right]\,.
\label{e7-dan}
\end{multline}

\vspace*{-9pt}

\section{Описание данных}

\vspace*{-3pt}

  Данные для проведения численных экспериментов состоят из двух 
однотипных наборов цен закрытия торгов по опционам на 3~июня 2019~г. По 
дате экспирации опционы были поделены на самые ближние и~дальние 
с~учетом ликвидности.
  
  Первый набор данных представлен европейскими опционами на фондовый 
индекс DAX. Индекс отражает суммарный доход по 
капиталу, поэтому при его расчете учитываются полученные дивидендные 
доходы по акциям, которые, как предполагается, реинвестируются в~акцию, 
по которой был получен дивиденд. Рас\-смат\-ри\-ва\-ют\-ся 19~опционов CALL 
и~19~опционов PUT, величина страйка варьируется от~9\,400 до~13\,000 
с~шагом~200. Базовым активом выступает фьючерс с~датой экспирации, 
соответствующей дате экспирации самого опциона. Дата экспирации ближних 
опционов~--- 22~июня 2019~г., а~дальних~--- 22~декабря 2023~г. Валюта 
опционов~--- евро. Данные взяты с~сайта {\sf www.eurexchange.com}.
  
  Во второй набор данных вошли 10 европейских опционов CALL и~PUT на 
фьючерс, базовым активом которого выступает нефть (Light Sweet Crude Oil). 
Величина страйка варьируется от~51,0 до~55,5 с~шагом~0,5. Дата экспирации 
ближних опционов~--- 20~июня 2019~г., дальних~--- 22~июня 2020~г. 
Валюта~--- доллар США. Источник данных: {\sf www.cmegroup.com}.
  
  В представленных данных фигурируют два вида валют; соответственно, при 
расчете справедливой стоимости опционов использовались две\linebreak безрисковые 
процентные ставки. В~качестве таковых были взяты соответствующие ставки 
LIBOR. На дату расчета (03~июня 2019~г.)\ ставка USD \mbox{LIBOR} равнялась 
2,36075\%, а~ставка EUR \mbox{LIBOR} со\-став\-ля\-ла~0,46614\%. Источник данных: 
{\sf www.global-rates.com}.

\vspace*{-12pt}

\section{Калибровка моделей ARIMA--GARCH}

\vspace*{-3pt}

\begin{table*}[b] %\small %tabl1
\vspace*{-3pt}
\begin{minipage}[t]{80mm}
\begin{center}
{\small 
\Caption{Результаты оценивания моделей ARIMA(0,0,1)--GARCH(1,1) для 
$Y_t^{\mathrm{N/EGB2}}\hm=\ln (S_t/S_{t-2})$  
и~$\tilde{Y}_t^{\mathrm{JSU}} \hm= S_t/S_{t-2}\hm-1$ фондового индекса DAX}
\vspace*{2ex}

\tabcolsep=8.15pt
\begin{tabular}{|c|c|c|c|}
\hline
\tabcolsep=0pt\begin{tabular}{c}Распре-\\ деление\\ ошибок\end{tabular}&N&EGB2&JSU \\
\hline
$L_n(\hat{v})$&1652,609&1658,26&1658,657\\
\hline
\tabcolsep=0pt \begin{tabular}{c} $\phi_0$\\ Std.\ error\\ $t$-value\end{tabular}&
\tabcolsep=0pt \begin{tabular}{c} $-$0,000100\\ (0,000798)\\ $-$0,12480 \end{tabular}&
\tabcolsep=0pt \begin{tabular}{c} $-$0,00006\\ (0,000786)\\ $-$0,076381 \end{tabular}&
\tabcolsep=0pt \begin{tabular}{c}  $-$0,000074\\ (0,000792)\\ $-$0,093104 \end{tabular}\\
\hline
\tabcolsep=0pt \begin{tabular}{c} $\theta_1$\\ Std.\ error\\ $t$-value\end{tabular}&
\tabcolsep=0pt \begin{tabular}{c} 0,950806\\ (0,013106)\\ 72,54849 \end{tabular}&
\tabcolsep=0pt \begin{tabular}{c} 0,948445\\ (0,015226)\\ 62,292279 \end{tabular}&
\tabcolsep=0pt \begin{tabular}{c} 0,949900\\ (0,013229)\\ 71,806222 \end{tabular}\\
\hline
\tabcolsep=0pt \begin{tabular}{c} $\alpha_0$\\ Std.\ error\\ $t$-value\end{tabular}&
\tabcolsep=0pt \begin{tabular}{c} 0,000003\\ (0,000005)\\ 0,55453 \end{tabular}&
\tabcolsep=0pt \begin{tabular}{c} 0,000003\\ (0,000004)\\ 0,687438 \end{tabular}&
\tabcolsep=0pt \begin{tabular}{c} 0,000003\\ (0,000004)\\ 0,718274 \end{tabular}\\
\hline
\tabcolsep=0pt \begin{tabular}{c} $\alpha_1$\\ Std.\ error\\ $t$-value \end{tabular}&
\tabcolsep=0pt \begin{tabular}{c} 0,071761\\ (0,031491)\\ 2,27881 \end{tabular}&
\tabcolsep=0pt \begin{tabular}{c} 0,067895\\ (0,027377)\\ 2,480017 \end{tabular}&
\tabcolsep=0pt \begin{tabular}{c} 0,067271\\ (0,030275)\\ 2,222013 \end{tabular}\\
\hline
\tabcolsep=0pt \begin{tabular}{c} $\beta_1$\\ Std.\ error\\ $t$-value \end{tabular}&
\tabcolsep=0pt \begin{tabular}{c} 0,894232\\ (0,050380)\\ 17,74964 \end{tabular}&
\tabcolsep=0pt \begin{tabular}{c} 0,902903\\ (0,037822)\\ 23,872549 \end{tabular}&
\tabcolsep=0pt \begin{tabular}{c} 0,900226\\ (0,041630)\\ 21,624622 \end{tabular}\\
\hline
AIC&$-$6,5773&$-$6,5919&$-$6,5934\\
\hline
BIC&$-$6,5352&$-$6,5369&$-$6,5385\\
\hline
$\xi$&---&$-$0,224916&$-$0,543876\\
\hline
$\kappa$&---&2,897165&2,298773\\
\hline
\end{tabular}
}
\end{center}
\end{minipage}
%\end{table*}
\hfill
%\begin{table*}%{\small %tabl2
\begin{minipage}[t]{80mm}
\begin{center}
{\small 
\Caption{Результаты оценивания моделей ARIMA(2,0,0)--GARCH(1,1) 
для $Y_t^{\mathrm{N/EGB2}}\hm= \ln (S_t/S_{t-2})$ 
и $\tilde{Y}_t^{\mathrm{JSU}}\hm= S_t/S_{t-2}\hm-1$ (Light Sweet Crude Oil)}
\vspace*{2ex}

\tabcolsep=8.14pt
\begin{tabular}{|c|c|c|c|}
\hline
\tabcolsep=0pt\begin{tabular}{c}Распре-\\ деление\\ ошибок\end{tabular}&N&EGB2&JSU\\
\hline
$L_n(\hat{v})$&1261,086&1267,017&1268,022\\
\hline
\tabcolsep=0pt \begin{tabular}{c} $\phi_1$\\ Std.\ error\\ $t$-value \end{tabular}&
\tabcolsep=0pt \begin{tabular}{c} 0,667116\\ (0,043985)\\ 15,1668 \end{tabular}&
\tabcolsep=0pt \begin{tabular}{c} 0,657088\\ (0,043122)\\ 15,2379 \end{tabular}&
\tabcolsep=0pt \begin{tabular}{c} 0,659745\\ (0,043152)\\ 15,2888\end{tabular}\\
\hline
\tabcolsep=0pt \begin{tabular}{c} $\phi_2$\\ Std.\ error\\ $t$-value \end{tabular}&
\tabcolsep=0pt \begin{tabular}{c} $-$0,313186\\ (0,043636)\\ $-$7,1773 \end{tabular}&
\tabcolsep=0pt \begin{tabular}{c} $-$0,323465\\ (0,042236)\\ $-$7,6585 \end{tabular}&
\tabcolsep=0pt \begin{tabular}{c} $-$0,322464\\ (0,042439)\\ $-$7,5983 \end{tabular}\\
\hline
\tabcolsep=0pt \begin{tabular}{c} $\alpha_0$\\ Std.\ error\\ $t$-value \end{tabular}&
\tabcolsep=0pt \begin{tabular}{c} 0,000015\\ (0,000002)\\ 8,341500 \end{tabular}&
\tabcolsep=0pt \begin{tabular}{c} 0,000013\\ (0,000001)\\ 16,4056 \end{tabular}&
\tabcolsep=0pt \begin{tabular}{c} 0,000013\\ (0,000001)\\ 16,1163\end{tabular}\\
\hline
\tabcolsep=0pt \begin{tabular}{c} $\alpha_1$\\ Std.\ error\\ $t$-value \end{tabular}&
\tabcolsep=0pt \begin{tabular}{c} 0,049707\\ (0,007002)\\ 7,0988\end{tabular}&
\tabcolsep=0pt \begin{tabular}{c} 0,042324\\ (0,005701)\\ 7,424 \end{tabular}&
\tabcolsep=0pt \begin{tabular}{c} 0,041834\\ (0,005506)\\ 7,5984\end{tabular}\\
\hline
\tabcolsep=0pt \begin{tabular}{c} $\beta_1$\\ Std.\ error\\ $t$-value \end{tabular}&
\tabcolsep=0pt \begin{tabular}{c} 0,913128\\ (0,013747)\\ 66,424300 \end{tabular}&
\tabcolsep=0pt \begin{tabular}{c} 0,92728\\ (0,011196)\\ 82,8231 \end{tabular}&
\tabcolsep=0pt \begin{tabular}{c} 0,928020\\ (0,011169)\\ 83,0925\end{tabular}\\
\hline
AIC&$-$4,9944&1267,752&$-$5,0140\\ 
\hline
BIC&$-$4,9524&$-$4,9542&$-$4,9553\\
\hline
$\xi$&&$-$0,70877&$-$0,27232\\
\hline
$\kappa$&&4,405384&2,663401\\
\hline
\end{tabular}
}
\end{center}
\end{minipage}
\vspace*{-3pt}
\end{table*}

  Пусть $\ln(L_n(v))$~--- логарифм функции правдоподобия модели  
ARIMA--GARCH для доходности базового актива с~вектором параметров 
$v\hm\in \Theta$. В~случае модели  
ARIMA$\,(p,d,q)$--GARCH$\,(P,Q)$-JSU$\,(\xi,\lambda,\gamma,\delta)$~(1) 
имеется вектор параметров 
\begin{multline*}
v= \left(\gamma, \delta, \phi_0, \phi_1, \ldots , \phi_p, 
\theta_1, \ldots , \theta_q, \alpha_0, \alpha_1, \ldots , \alpha_P,\right.\\
\left. \beta_1, \ldots , 
\beta_Q\right).
\end{multline*}
 Число параметров равно $n\hm= p\hm+ 1\hm+q\hm+ P\hm+Q\hm+2$. 
Оптимальные параметры определяются исходя из максимума функции  
правдоподобия~\cite{8-dan}:
  \begin{multline}
  \hat{v}_n=\argmax\limits_{v\in\Theta} \ln (L_n(v))={}\\[-3pt]
  {}=\argmax\limits_{v\in \Theta} \sum\limits^T_{t=0} \left( \ln(\delta)-
\ln \left(\sqrt{\fr{2h_t}{A}}\right)-{}\right.\\[-1pt]
\left.{}- \fr{1}{2}\ln\left(1+\left( 
\fr{\varepsilon_t}{\sqrt{2h_t/A}}-B\right)^{\!2}\right) \right.-{}\\[-3pt]
  \left.{}-\fr{1}{2}\left(\gamma+\delta\mathrm{sinh}^{-1}\left( 
\fr{\varepsilon_t}{\sqrt{2h_t/A}}-B\right)\right)^{\!2}\right)\,,\\[-1pt]
\delta, \alpha_0>0,\alpha_1,\ldots , \alpha_P, \beta_1,\ldots , \beta_Q\geq 0\,,
\label{e8-dan}
\end{multline}
где 
%\vspace*{-6pt}


\noindent
$$
A=\left(e^{1/\delta^2}-1\right)\left(e^{1/\delta^2}\mathrm{cosh}\left(\fr{2\gamma}{\delta}\right)+1\right);
$$ 
$$
B=e^{1/(2\delta^2)}\mathrm{sinh}\left(\fr{\gamma}{\delta}\right); \enskip
\sum\limits_{i=1}^P 
\alpha_i+ \sum\limits^Q_{j=1} \beta_j <1\,.
$$ 
  
  Для случая EGB2-распределения число па\-ра\-мет\-ров $v\hm= (\alpha, \beta, 
\phi_0, \ldots , \phi_p, \theta_1, \ldots , \theta_q, \alpha_0, \alpha_1, \ldots$\linebreak $\ldots , \alpha_P, 
\beta_1, \ldots , \beta_Q)$ такое же, как в~предыдущем случае. 
Оптимизационная задача имеет следующий вид:

\vspace*{-6pt}

\noindent
  \begin{multline}
  \hat{v}_n=\argmax\limits_{v\in \Theta} \ln (L_n(v))={}\\
  {}=\argmax\limits_{v\in \Theta} \sum\limits^T_{t=0} \left( 
  \vphantom{\fr{\varepsilon_t\sqrt{l(\alpha,\beta)}}{\sqrt{h_t}}}
  \ln \left( \sqrt{l(\alpha,\beta)}\right) -
\ln(B(\alpha,\beta)) +{}\right.\\
{}+ \alpha\overline{\omega}(\alpha,\beta)+ 
\fr{\alpha\varepsilon_t\sqrt{l(\alpha,\beta)}}{\sqrt{h_t}}
- \ln\left(\sqrt{h_t}\right)-{}\\
{} -(\alpha+\beta) \ln \left( 1+\exp \left( 
\fr{\varepsilon_t\sqrt{l(\alpha,\beta)}}{\sqrt{h_t}}
%+{}\right.\right.\\\left.\left.
\left.{}+\overline{\omega}(\alpha,\beta)
 \vphantom{\fr{\varepsilon_t\sqrt{l(\alpha,\beta)}}{\sqrt{h_t}}}
 \right)\right)\right)\!,\\
  \alpha,\beta,\alpha_0>0\,, \enskip \alpha_1,\ldots, \alpha_P,\beta_1,\ldots , \beta_Q\geq 0\,,
  \label{e9-dan}
  \end{multline}
\vspace*{-6pt}
  
\noindent
где $\Gamma(c)$~--- гамма-функция;
\begin{align*}
\overline{\omega}(\alpha,\beta)&=\left.\fr{d\ln\Gamma(c)}{dc}\right\vert_{c=\alpha}-
\left.\fr{d\ln\Gamma(c)}{dc}\right\vert_{c=\beta}\,;\\
l(\alpha,\beta)&=\left.\fr{d^2\ln\Gamma(c)}{dc^2}\right\vert_{c=\alpha} +
\left.\fr{d^2\ln\Gamma(c)}{dc^2}\right\vert_{c=\beta}\,;
\end{align*}
$$
\sum\limits^P_{i=1}\alpha_i+\sum\limits^Q_{j=1} \beta_j<1\,.
$$
  
\begin{figure*}[b] %fig1
\vspace*{1pt}
    \begin{center}  
  \mbox{%
 \epsfxsize=162.997mm 
 \epsfbox{dan-1.eps}
 }
\end{center}
\vspace*{-12pt}
\Caption{Графики ACF~(\textit{а}) и~PACF~(\textit{б}) для $Y_t\hm= \ln(S_t/S_{t-2})$ (DAX)}
%\end{figure*}
%\begin{figure*} %fig2
\vspace*{5pt}
    \begin{center}  
  \mbox{%
 \epsfxsize=162.997mm 
 \epsfbox{dan-2.eps}
 }
\end{center}
   \vspace*{-12pt}
  \Caption{Графики ACF~(\textit{а}) и~PACF~(\textit{б}) для $Y_t\hm= \ln (S_t/S_{t-2})$ (Light Sweet Crude Oil)}
  \end{figure*}

  Для случая нормального распределения использовались функции 
библиотеки rugarch среды~R~\cite{8-dan}.

\vspace*{-9pt}
   
\section{Калибровка параметров моделей ARIMA--GARCH}

\vspace*{-3pt}

  В табл.~1 и~2 представлены результаты оценки параметров моделей  
ARIMA--GARCH. Для обоих временн$\acute{\mbox{ы}}$х рядов был проведен Q-тест  
Льюн\-га--Бок\-са для разного числа лагов (нулевая гипотеза заключается 
в~отсутствии автокорреляций для первых~$k$~лагов), который показал, что 
автокорреляционные связи для рядов $\ln (S_t/S_{t-1})$ и~$S_t/S_{t-1}\hm-1$ 
отсутствуют с~вероятностью 99\% (значение статистики составило~30,123 для 
индекса DAX и~24,576 для Light Sweet Crude Oil, в~то время как критическое 
значение для $k\hm=30$~лагов и~уровня значимости~99\% равно~50,892). 
В~результате были рассмотрены ряды $\ln(S_t/S_{t-2})$  
и~$S_t/S_{t-2}\hm- 1$ и~на основе их коррелограмм ACF и~PACF (рис.~1 и~2) 
был сделан вывод о спецификации ARIMA части моделей ARIMA--GARCH.
{\looseness=-1

}

  Для коррелограмм индекса DAX характерна модель ARIMA$(\,(0,0,1)$ 
(ненулевое значение автокорреляции первого лага и~затухающая динамика 
значений частных автокорреляций). Коррелограммы цен на нефть определяют 
модель ARIMA$(\,2,0,0)$  (ненулевые значения частных автокорреляций двух 
первых лагов и~затухающее поведение автокорреляций). 
{\looseness=-1

}
  
  Коэффициенты всех трех моделей имеют одинаковый знак и~порядок. Все 
полученные коэффициенты моделей статистически значимы на уровне 
значимости~99\% (следует из $t$-кри\-те\-рия). Можно также заметить, что 
коэффициенты, соответствующие моделям с~распределениями ошибок EGB2 
и~JSU, почти идентичны.

  
\begin{figure*}[b] %fig3
\vspace*{1pt}
 \begin{center}  
 \mbox{%
 \epsfxsize=161.099mm 
 \epsfbox{dan-3.eps}
 }
\end{center}
\vspace*{-11pt}
\Caption{Абсолютные ошибки цен опционов CALL~(\textit{а}) и~PUT~(\textit{б})
 на индекс DAX со сроком 22~июня 2019~г.: \textit{1}~---  
ARIMA--GARCH-N; \textit{2}~--- 
ARIMA--GARCH-EGB2; \textit{3}--- ARIMA--GARCH-JSU }
%\end{figure*}
%\begin{figure*} %fig4
\vspace*{5pt}
    \begin{center}  
  \mbox{%
 \epsfxsize=162.999mm 
 \epsfbox{dan-5.eps}
 }
\end{center}
\vspace*{-11pt}
\Caption{Абсолютные ошибки цен опционов CALL~(\textit{а}) и~PUT~(\textit{б})
 на индекс DAX со сроком 22~декабря 2023~г.: \textit{1}~--- 
ARIMA--GARCH-N; \textit{2}~--- 
ARIMA--GARCH-EGB2; \textit{3}--- ARIMA--GARCH-JSU}
\end{figure*}

  Предсказательная сила моделей оценивалась с~помощью информационных 
критериев Акаике (AIC) и~Байеса (BIC), статистики которых рассчитываются 
по следующим формулам:

\noindent
  \begin{align*}
    \mathrm{AIC} &= 2k-2\ln\left( L_n\!\left( \hat{v}\right)\right)\,;\\[6pt]
  \mathrm{BIC} &= k\ln(N) -2\ln\left( L_n\!\left( \hat{v}\right)\right)\,,
  %  \label{e10-dan}
  \end{align*}
где $N$~--- объем выборки; $k$~--- число параметров; $L_n(\hat{v})$~--- 
значение функции правдоподобия для найден\-ных оптимальных 
параметров~$\hat{v}$. Таблицы~1 и~2 показывают, что асимметричные 
распределения EGB2 и~JSU лучше моделируют ряд, чем нормальное 
распределение.


\section{Ценообразование опционов}

  В рамках данной работы справедливая стоимость опционов оценивалась 
  с~помощью метода Мон\-те-Кар\-ло в~соответствии с~формулами~(\ref{e6-dan}) 
и~(\ref{e7-dan}) по риск-ней\-т\-раль\-ным траекториям моделей  
ARIMA--GARCH~(\ref{e3-dan})--(\ref{e5-dan}), где число 
реализаций доходности базового актива $M\hm= 10\,000$. Эффективность 
каждой модели ARIMA--GARCH оценивалась по абсолютной ошибке (AE):
  \begin{multline*}
  \mathrm{АО}(\mathrm{Moneyness})={}\\
\!=\!\,\left\vert p^m_{\mathrm{call}}/p^m_{\mathrm{put}}
  \left( \mathrm{Moneyness}\right) -
p_{\mathrm{call}}/p_{\mathrm{put}}\left( \mathrm{Moneyness}\right)
  \!\,\right\vert\!,\hspace*{-5.9pt}
  %\label{e11-dan}
  \end{multline*}
где $p^m_{\mathrm{call}}/p^m_{\mathrm{put}}$~---  рыночные котировки опционов; 
$\mathrm{Moneyness}\hm=X/s_0$. 



  Абсолютные ошибки оценивания опционов иллюстрируются на рис.~3--6. 
Для опционов на индекс DAX наиболее близкие значения к~рыночным 
котировкам дает модель с~распределением ошибок JSU, модель 
с~нормальным распределением хуже всего находит стоимость опционов DAX. 
При этом стоит также отметить, что для опционов с~датой экспирации 
22~декабря 2023~г.\ расхождения между моделями становятся существенней, 
показывая неэффективность использования моделей с~нормальным 
распределением ошибок на длинных временн$\acute{\mbox{ы}}$х
горизонтах. 
\pagebreak

\end{multicols}


\begin{figure*} %fig5
\vspace*{.1pt}
    \begin{center}  
  \mbox{%
 \epsfxsize=162.238mm 
 \epsfbox{dan-7.eps}
 }
\end{center}
\vspace*{-12.5pt}
\Caption{Абсолютные ошибки цен опционов CALL~(\textit{а}) и~PUT~(\textit{б})
 Light Sweet Crude Oil (20~июня 2019~г.): \textit{1}~---  
ARIMA--GARCH-N; \textit{2}~--- 
ARIMA--GARCH-EGB2; \textit{3}--- ARIMA--GARCH-JSU}
%\end{figure*}
%\begin{figure*} %fig6
\vspace*{5pt}
    \begin{center}  
  \mbox{%
 \epsfxsize=161.542mm 
 \epsfbox{dan-9.eps}
 }
\end{center}
\vspace*{-12.5pt}
\Caption{Абсолютные ошибки цен опционов CALL~(\textit{а}) и~PUT~(\textit{б})
 Light Sweet Crude Oil (22~июня 2020~г.): \textit{1}~---  
ARIMA--GARCH-N; \textit{2}~--- 
ARIMA--GARCH-EGB2; \textit{3}--- ARIMA--GARCH-JSU}
\vspace*{-1pt}
\end{figure*}


\begin{multicols}{2}

Результаты 
оценки спра\-вед\-ли\-вой стои\-мости опционов на индекс DAX аналогичны 
результатам работы~\cite{7-dan}, в~которой среди рас\-смат\-ри\-ва\-емых опционов 
присутствовали опционы на индекс S\&P~500 и~делался вывод о~том, что 
модели ARIMA--GARCH с~ошибками, распределенными по закону EGB2, дают 
лучшие оценки стоимости опционов в~деньгах (для опционов CALL 
$\mathrm{Moneyness}\hm< 1$, для PUT $\mathrm{Moneyness}\hm>1$) по сравнению с~моделями, 
где ошибки распределены нормально. 
  
  Опционы на базовый актив Light Sweet Crude Oil также характеризуются 
существенным преимуществом модели JSU. Однако теперь уже модель 
с~распределением EGB2 показывает результаты хуже, чем модель с~нормальным 
распределением.
  
\vspace*{-7pt}

\section{Заключение}
\vspace*{-2pt}

  В данной статье были рассмотрены три альтернативные модели временн$\acute{\mbox{ы}}$х 
рядов с~условным средним ARIMA и~условной дисперсией GARCH 
структуры. Оценка справедливой стоимости опционов проводилась по методу  
Мон\-те-Кар\-ло~(\ref{e6-dan}), (\ref{e7-dan}) на основе риск-ней\-т\-раль\-ной 
динамики моделей в~соответствии с~формулами~(\ref{e3-dan})--(\ref{e5-dan}). 
Калибровка моделей осуществлялась методом 
квазимаксимального правдоподобия в~соответствии 
с~соотношениями~(\ref{e8-dan}) и~(\ref{e9-dan}). Определение порядка моделей 
было выполнено на основе Q-тес\-та  
Льюн\-га--Бок\-са и~графиков коррелограмм ACF и~PACF. 
  
  Результаты эмпирических исследований показывают, что на небольших 
временн$\acute{\mbox{ы}}$х промежутках модели обеспечивают близкие значения 
справедливой стоимости опционов. Для опционов с~дальней датой 
экспирации (больше года) модели показывают существенно разные 
результаты. Модель, построенная для распределения JSU, дает 
оценки стоимости, максимально близкие к~ценам закрытия биржевых торгов 
для всех рас\-смат\-ри\-ва\-емых опционных контрактов. 
  
{\small\frenchspacing
 {%\baselineskip=10.8pt
 %\addcontentsline{toc}{section}{References}
 \begin{thebibliography}{99}
  
\bibitem{1-dan}
\Au{Данилишин А.\,Р., Голембиовский~Д.\,Ю.} Риск-нейт\-раль\-ная динамика для модели 
ARIMA--GARCH с~ошиб\-ка\-ми, распределенными по закону $S_U$ Джонсона~// 
Информатика и~её применения, 2020. Т.~14. Вып.~1. С.~48--55.

\bibitem{5-dan} %2
\Au{Bollerslev T.} A~conditionally heteroskedastic time series model for speculative prices and 
rates of return~//  Rev. Econ. Stat., 1987. Vol.~69. Iss.~3. P.~542--547. doi: 10.2307/1925546.
\bibitem{3-dan} %3
\Au{Akgiray V.} Conditional heteroscedasticity in time series of stock returns: Evidence and 
forecasts~// J.~Bus., 1989. Vol.~62. Iss.1. P.~55--80. doi: 10.1086/296451.
\bibitem{4-dan} %4
\Au{Follmer H., Schied A.} Stochastic finance: An introduction in discrete time.~--- Berlin: 
Walter de Gruyter, 2002.\linebreak 422~p.
\bibitem{2-dan} %5
\Au{Terasvirta T.} An introduction to univariate GARCH models~//  Handbook of 
financial time series~/
Eds. T.\,G.~Andersen, R.\,A.~Davis, J.-P.~Kreiss, Th.\,V.~Mikosch.~--- Berlin--Heidelberg: 
Springer, 2009. Vol.~10.  
P.~17--42. doi: 10.1007/978-3-540-71297-8\_1.
\bibitem{6-dan} %6
\Au{Elliott R.\,J., Madan D.\,B.} A~discrete time equivalent martingale measure~// Math. 
Financ., 1998. Vol.~8. Iss.~2. P.~127--152. doi: 10.1111/1467-9965.00048.
\bibitem{7-dan}
\Au{Yi Xi.} Comparison of option pricing between ARMA-GARCH and GARCH-M models.~--- 
London, Ontario, Canada: University of Western Ontario, 2013. MoS Thesis. 73~p.
\bibitem{8-dan}
\Au{Christian F., Francq M.} GARCH models: Structure, statistical inference and financial 
applications.~--- New York, NY, USA: Wiley, 2019. 504~p.
\bibitem{9-dan}
\Au{Boyle T.} A~Monte Carlo approach~// J.~Financ. Econ., 2012. Vol.~4. P.~323--338. 
doi: 10.1016/0304-405x(77)90005-8.
\bibitem{10-dan}
\Au{Hull J.} Options, futures, and other derivatives.~--- 10th ed.~--- Pearson, 2018. 896~p.
\bibitem{11-dan}
\Au{Williams D.} Probability with martingales.~--- Cambridge: Cambridge University Press, 
1991. 251~p.
\bibitem{12-dan}
\Au{Bell D.} Transformations of measure on an infinite dimensional vector space~// Seminar on 
stochastic processes, 1990~/ Eds. \mbox{E.~{\ptb{\c{C}}}inlar}, P.\,J.~Fitzsimmons, 
R.\,J.~Williams.~--- Progress in probability book ser.~--- Boston, MA, USA:
Birkh$\ddot{\mbox{a}}$user, 1991. 
Vol.~24. P.~15--25. doi: 10.1007/978-1-4684-0562-0\_3.
\end{thebibliography}

 }
 }

\end{multicols}

\vspace*{-12pt}

\hfill{\small\textit{Поступила в~редакцию 01.10.19}}

\vspace*{6pt}

%\pagebreak

%\newpage

%\vspace*{-28pt}

\hrule

\vspace*{2pt}

\hrule

\vspace*{-2pt}

\def\tit{ESTIMATING THE FAIR VALUE OF~OPTIONS\\
 BASED~ON~ARIMA--GARCH MODELS WITH~ERRORS\\ DISTRIBUTED ACCORDING 
TO~THE~JOHNSON'S $S_U$ LAW}


\def\titkol{Estimating the fair value of options based on~ARIMA--GARCH models 
with~errors distributed according to~the~Johnson's $S_U$ law}


\def\aut{A.\,R.~Danilishin$^1$ and D.\,Yu.~Golembiovsky$^{1,2}$}

\def\autkol{A.\,R.~Danilishin and D.\,Yu.~Golembiovsky}

\titel{\tit}{\aut}{\autkol}{\titkol}

\vspace*{-11pt}


\noindent
$^1$Department of Operations Research, Faculty of Computational Mathematics and Cybernetics, 
M.\,V.~Lomonosov\linebreak
$\hphantom{^1}$Moscow State University,  
1-52~Leninskie Gory, Moscow 119991, GSP-1, Russian Federation

\noindent
$^2$Department of Banking, Sinergy University, 80-G~Leningradskiy Prosp., Moscow 125190, Russian 
Federation


\def\leftfootline{\small{\textbf{\thepage}
\hfill INFORMATIKA I EE PRIMENENIYA~--- INFORMATICS AND
APPLICATIONS\ \ \ 2020\ \ \ volume~14\ \ \ issue\ 4}
}%
 \def\rightfootline{\small{INFORMATIKA I EE PRIMENENIYA~---
INFORMATICS AND APPLICATIONS\ \ \ 2020\ \ \ volume~14\ \ \ issue\ 4
\hfill \textbf{\thepage}}}

\vspace*{6pt} 


\Abste{In continuation of the article ``Risk-neutral dynamics for the ARIMA--GARCH 
(autoregressive integrated moving average\,--\,generalized autoregressive conditional heteroskedasticity)
random process with errors distributed according to the Johnson's $S_U$ law,'' this 
paper presents the experimental results for the ARIMA--GARCH 
(autoregressive integrated moving average\,--\,generalized autoregressive conditional heteroskedasticity)
models with normal (N), exponential beta of the second type 
(EGB2), and $S_U$ Johnson (JSU) error distributions. The fair value of European 
options is estimated by the Monte-Carlo method based on the results obtained in the 
specified article by using the extended Girsanov principle. The parameters of the 
ARIMA--GARCH-N,  
ARIMA--GARCH-EGB2, and ARIMA--GARCH-JSU models were found by the 
quasi-maximum likelihood method. The efficiency of the resulting risk-neutral models 
was studied using the example of European exchange-traded options PUT and CALL 
on basic assets DAX and Light Sweet Crude Oil.}

\KWE{ARIMA; GARCH; risk-neutral measure; Girsanov extended principle; 
Johnson's $S_U$ distribution; option pricing}

\DOI{10.14357/19922264200412} 

%\vspace*{-20pt}

%\Ack
%\noindent


%\vspace*{-6pt}

  \begin{multicols}{2}

\renewcommand{\bibname}{\protect\rmfamily References}
%\renewcommand{\bibname}{\large\protect\rm References}

{\small\frenchspacing
 {%\baselineskip=10.8pt
 \addcontentsline{toc}{section}{References}
 \begin{thebibliography}{99}
\vspace*{-4pt}

\bibitem{1-dan-1}
\Aue{Danilishin, A.\,R., and D.\,Y. Golembiovsky.} 2020. Risk-neytral'naya dinamika 
dlya ARIMA--GARCH modeli
%\linebreak
%\vspace*{-12pt}
%\columnbreak
%\noindent
s~oshib\-ka\-mi, raspredelennymi po zakonu $S_U$ 
Dzhonsona. [Risk-neutral dynamics for ARIMA--GARCH random process with errors 
distributed according to the Johnson's $S_U$ law]. \textit{Informatika i~ee 
Primeneniya~--- Inform. Appl.} 14(1):56--62.

\bibitem{5-dan-1} %2
\Aue{Bollerslev, T.} 1987. A~conditionally heteroskedastic time series model for 
speculative prices and rates of return. \textit{Rev. Econ. Stat.} 69(3):542--547. doi: 
10.2307/ 1925546.
{\looseness=1

}
\bibitem{3-dan-1} %3
\Aue{Akgiray, V.} 1989. Conditional heteroscedasticity in time series of stock returns: 
Evidence and forecasts. \textit{J.~Bus.} 62(1):55--80. doi: 10.1086/296451.
\bibitem{4-dan-1} %4
\Aue{Follmer, H., and A. Schied.} 2002. \textit{Stochastic finance: An introduction in 
discrete time}. Berlin: Walter de Gruyter. 422~p.

\bibitem{2-dan-1} %5
\Aue{Terasvirta, T.} 2009. An introduction to univariate GARCH models. 
\textit{Handbook of financial time series}. Eds. T.\,G.~Andersen, R.\,A.~Davis, 
J.-P.~Kreiss, and Th.\,V.~Mikosch. Berlin--Heidelberg: Springer. 10:17--42. doi:  
10.1007/978-3-540-71297-8\_1.

\bibitem{6-dan-1}
\Aue{Elliott, R.\,J., and D.\,B. Madan.} 1998. A~discrete time equivalent martingale 
measure. \textit{Math. Financ.} 8(2):127--152. doi: 10.1111/1467-9965.00048.
\bibitem{7-dan-1}
\Aue{Yi, Xi.} 2013. Comparison of option pricing between ARMA--GARCH and 
GARCH-M models. London, Ontario, Canada: University of Western Ontario. MoS 
Thesis. 73~p.
\bibitem{8-dan-1}
\Aue{Christian, F., and M. Francq.} 2019. \textit{GARCH models: Structure, 
statistical inference and financial applications.} New York, NY: Wiley. 504~p.
\bibitem{9-dan-1}
\Aue{Boyle, P.} 2012. Options: A~Monte Carlo approach. \textit{J.~Financ. 
Econ.} 4(3):323--338.  doi: 10.1016/0304-405x(77)90005-8.
\bibitem{10-dan-1}
\Aue{Hull, J.} 2018. \textit{Options, futures, and other derivatives}. 10th ed. Pearson. 
896~p.
\bibitem{11-dan-1}
\Aue{Williams, D.} 1991. \textit{Probability with martingales}. Cambridge: 
Cambridge University Press. 251~p.
\bibitem{12-dan-1}
\Aue{Bell, D.} 1991. Transformations of measure on an infinite dimensional vector 
space. \textit{Seminar on stochastic processes, 1990}. Eds. \mbox{E.~{\ptb{\c{C}}}inlar}, 
P.\,J.~Fitzsimmons, and R.\,J.~Williams. Progress in probability book ser. Boston, MA: 
Birkh$\ddot{\mbox{a}}$user. 
24:15--25. doi: 10.1007/978-1-4684-0562-0\_3.
\end{thebibliography}

 }
 }

\end{multicols}

\vspace*{-3pt}

\hfill{\small\textit{Received October 1, 2019}}

%\pagebreak

%\vspace*{-24pt}


\Contr

\noindent
\textbf{Danilishin Artem R.} (b.\ 1992)~--- PhD student, Department of Operations Research, Faculty of 
Computational Mathematics and Cybernetics, M.\,V.~Lomonosov Moscow State University, 1-52~Leninskie 
Gory, GSP-1, Moscow 119991, Russian Federation; \mbox{danilishin-artem@mail.ru}

\vspace*{3pt}

\noindent
\textbf{Golembiovsky Dmitry Y.} (b.\ 1960)~--- Doctor of Science in technology, professor, Department of 
Operation Research, Faculty of Computational Mathematics and Cybernetics, M.\,V.~Lomonosov Moscow 
State University, 1-52~Leninskie Gory, GSP-1, Moscow 119991, Russian Federation; professor, Department 
of Banking, Sinergy University, 80-G~Leningradskiy Prosp., Moscow 125190, Russian Federation; 
\mbox{golemb@cs.msu.su}
\label{end\stat}

\renewcommand{\bibname}{\protect\rm Литература} 
      
  %7
\def\stat{serebr+ataeva}

\def\tit{ОНТОЛОГИЯ ЦИФРОВОЙ СЕМАНТИЧЕСКОЙ БИБЛИОТЕКИ LibMeta}

\def\titkol{Онтология цифровой семантической библиотеки LibMeta}

\def\aut{О.\,М.~Атаева$^1$, В.\,А.~Серебряков$^2$}

\def\autkol{О.\,М.~Атаева, В.\,А.~Серебряков}

\titel{\tit}{\aut}{\autkol}{\titkol}

\index{Атаева О.\,М.}
\index{Серебряков В.\,А.}
\index{Serebryakov V.\,A.}
\index{Ataeva O.\,M.}




%{\renewcommand{\thefootnote}{\fnsymbol{footnote}} \footnotetext[1]
%{Работа выполнена при финансовой поддержке РФФИ (проект 17-01-00816).}}


\renewcommand{\thefootnote}{\arabic{footnote}}
\footnotetext[1]{Вычислительный центр им.\ А.\,А.~Дородницына Федерального исследовательского 
центра <<Информатика и~управ\-ле\-ние>> Российской академии наук, 
\mbox{oli@ultimeta.ru}}
\footnotetext[2]{Вычислительный центр им.\ А.\,А.~Дородницына Федерального исследовательского 
центра <<Информатика и~управ\-ле\-ние>> Российской академии наук, \mbox{serebr@ultimeta.ru}}
%\vspace*{-6pt}



\Abst{При разработке цифровых библиотек особое внимание уделяют модели данных 
содержимого библиотеки. При этом контент цифровых библиотек может быть описан 
различными форматами и~представлен различными способами. Библиотека, определяемая 
с~по\-мощью системы LibMeta, рассматривается как хранилище структурированных 
разнообразных данных с~возможностью их интеграции с~другими источниками данных 
и~предполагает возможность специфицирования своего контента за счет описания 
предметной области. В~качестве средства формализации выступает онтология контента 
семантической библиотеки. Также вводятся основные понятия для описания задачи 
интеграции данных из источников Linked Open Data (LOD), понятия для определения 
произвольного тезауруса. Онтология построена таким образом, чтобы иметь возможность 
определения семантической библиотеки в~произвольной предметной области.}

\KW{семантические библиотеки; модель данных; онтологии; источники данных; поиск 
в~LOD}

  \DOI{10.14357/19922264180101} 
  
%\vspace*{9pt}


\vskip 10pt plus 9pt minus 6pt

\thispagestyle{headings}

\begin{multicols}{2}

\label{st\stat}

\section{Введение}

     В различных предметных областях модель данных содержимого 
цифровых семантических биб\-лио\-тек может существенно отличаться как по 
типам ресурсов, так и~по их структуре. При разработке таких биб\-лио\-тек особое 
внимание уделяют модели данных содержимого биб\-лио\-теки.
     
     Говоря о библиотеках, авторы прежде всего \mbox{имеют} в~виду 
разработанную информационную сис\-те\-му для создания семантических 
библиотек LibMeta~[1--3], с~по\-мощью которой создается и~описывается 
семантическая биб\-лио\-те\-ка некоторой предметной об\-ласти. 
     
     LibMeta представляет собой информационную сис\-те\-му, которая 
реализует функциональность, необходимую для работы с~контентом 
семантической биб\-лио\-те\-ки. LibMeta не является традиционной сис\-те\-мой 
управления электронными библиотеками (СУЭБ). 

Развитие современных 
технологий подталкивает к~переопределению как понятия биб\-лио\-те\-ки, так 
и~контента биб\-лио\-те\-ки, в~качестве которых не обязательно могут выступать 
традиционные описания печатных изданий, но и~любые другие типы 
циф\-ро\-вых объектов. При этом контент циф\-ро\-вых биб\-лио\-тек может быть 
описан различными форматами и~пред\-став\-лен различными способами. 
Биб\-лио\-те\-ка, реализуемая с~помощью LibMeta, рассматривается как 
хранилище структурированных разнообразных\ данных с~воз\-мож\-ностью их 
интеграции с~другими источниками данных и~предполагает возможность 
специфицирования своего контента путем описания предметной об\-ласти. 
     
     Определение предметной области задается тезаурусом~[4], который 
содержит основные термины этой предметной об\-ласти, связанные 
иерархическими и~горизонтальными связями между собой. Содержимое 
биб\-лио\-те\-ки задается типами ресурсов, описание которых задает,
в~свою очередь, множество 
допустимых объектов, возможно объединенных в~разнообразные коллекции, 
со\-став\-ля\-ющие вместе с~тезаурусом ее контент.
     
     Статья посвящена исследованию средств пред\-став\-ле\-ния знаний 
о~контенте семантической биб\-лио\-те\-ки. Эти средства необходимы для 
автоматизации описания ресурсов биб\-лио\-те\-ки конкретной предметной 
области и~воз\-мож\-ности их автоматизированной интеграции с~данными 
внеш\-них открытых источников. Необходимым условием для этого является 
структуризация и~формализация знаний в~об\-ласти описания контента 
семантической биб\-лио\-теки. 
     
     При реализации LibMeta авторы руководствовались набором основных 
задач, которые должна решать разрабатываемая система:
     \begin{enumerate}[(1)]
\item библиотека должна поддерживать возможность использования 
медийных объектов или ссылки на них при описании своих объектов, 
включая текст, аудио- и~видеофайлы или любую их комбинацию. Это 
требование отражается в~названии словом <<цифровая>>;\\[-10pt]
\item типы используемых ресурсов и~связи между ними должны быть 
описаны средствами сис\-те\-мы в~рамках определенных в~предыду\-щей работе 
понятий, составляющих семантическое описание ресурсов контента 
биб\-лио\-те\-ки. При этом, согласно принципам LOD, при описании ресурсов 
поддерживается использование классов и~свойств ранее используемых 
онтологий в~сообществе, поддерживающем LOD. Эта поддержка 
выражается либо в~непосредственном использовании готовых онтологий 
при описании ресурсов и~связей между ними, либо в~возможности ссылок 
на их элементы, используя связи на уровне описания ресурсов. Это 
требование отражается в~названии словом <<семантическая>>;\\[-10pt]
\item библиотека должна служить интеграционным узлом, предоставляя 
возможность связывания своих данных с~данными из разных источников, 
которые включены в~облако LOD. Должна также обеспечиваться 
возможность извлекать данные этой биб\-лио\-те\-ки в~машиночитаемом 
формате. Это требование отражается в~на\-зва\-нии словом <<открытая>>;\\[-10pt]
\item пользователи биб\-лио\-те\-ки должны иметь возможность организовывать 
свои коллекции по интересующему их научному на\-прав\-ле\-нию, добавляя 
новые термины в~предметный тезаурус, уточняя таким образом об\-ласть 
своих интересов. Пользователи должны также иметь возможность 
осуществлять поиск не только среди объектов в~рамках сис\-те\-мы, но и~по 
источникам данных, без необходимости использования 
специализированного языка для поисковых запросов. Это требование 
отражается в~названии словом <<персональная>>.
\end{enumerate}

     Основные требования, предъявляемые при этом к~контенту  
системы,~--- \textit{универсальность}, \textit{структурированность}, 
\textit{адаптируемость}~--- не противоречат этим свойствам и~обеспечивают 
поддержку настраиваемого хранилища метаданных для объектов 
и~расширяемый набор информационных ресурсов. \textit{Универсальность} 
обеспечивает описания типов ее ресурсов и~объектов независимо от 
предметной об\-ласти и~об\-ласти интересов пользователей. 
\textit{Структурированность} описания обеспечивает\linebreak\vspace*{-12pt}

\columnbreak

\noindent
 поддержку связей 
между различными типами ресурсов как внутри сис\-те\-мы, так и~вне нее, 
исходя из определений LOD. \textit{Адаптируемость} описания ресурсов 
обеспечивает возможность добавления новых свойств и~связей в~процессе 
развития сис\-те\-мы и~обеспечивает настройку пользовательских интерфейсов 
под эти изменения. 
{ %\looseness=1

}
     
     В качестве средства формализации выступает онтология~\cite{5-ser} 
контента семантической биб\-лио\-те\-ки. На основе этого описания можно 
выделить основные понятия описания задачи интеграции данных из 
открытых источников.
     
     В качестве открытых источников рас\-смат\-ри\-ва\-ют\-ся источники данных, 
включенные в~LOD~\cite{6-ser} и~соответствующие основным 
требованиям, предъявляемым к~таким источникам данных.
     
     В~качестве основных разделов освещаемой задачи рассмотрим 
определение тезауруса и~основные стандарты, выделим понятия, 
необходимые для описания контента семантической биб\-лио\-те\-ки 
в~произвольной предметной об\-ласти, определим основные понятия, 
необходимые для описания задачи интеграции данных из открытых 
источников и~выделим основные типы связей между этими понятиями.

\section{Тезаурус и~стандарты}

     Для описания какой-либо предметной области всегда используется 
определенный набор терминов, каждый из которых обозначает или 
описывает ка\-кую-ли\-бо концепцию из этой предметной об\-ласти. 
Совокупность терминов, описывающих предметную область с~указанием 
семантических отношений (связей) между ними, является тезаурусом. Такие 
отношения в~тезаурусе всегда указывают на наличие смысловой 
(семантической) связи между терминами.
     
     При этом модель тезауруса не должна быть ориентирована ни на одну 
из конкретных предметных областей и~быть достаточно гибкой для того, 
чтобы позволить всегда сохранять актуальность словаря и~удобство его 
использования для определения любой предметной об\-ласти.
     
     Тезаурус с~наличием связей различных типов позволяет реализовать 
гибкий настраиваемый поиск, результатом которого будет список объектов 
предметной об\-ласти, со\-от\-вет\-ст\-ву\-ющий выбранным терминам. 
     
     Рассматриваемая в~статье модель тезауруса соответствует стандарту 
ISO~2788-1986. Этот стандарт определяет тезаурус как набор терминов, 
связанных между собой со\-от\-вет\-ст\-ву\-ющи\-ми связями (отношениями). 

Термины могут иметь следующие атрибуты:
     \begin{itemize}
\item $\mathrm{SN}$~--- Scope Note. Комментарий к~термину. Например, представляет 
вербальное пояснение термина или правила его использования;
\item $\mathrm{TT}$~--- Top Term. Признак, выделяющий термины на самом верхнем 
уровне иерархии (термины наиболее общих понятий в~иерархии понятий).
\end{itemize}

     Связи между терминами могут быть сле\-ду\-ющими:
     \begin{itemize}
\item $\mathrm{USE}$~--- связывает термин с~наиболее предпочтительным термином 
для понятия. $A$~$\mathrm{USE}$~$B$ означает, что термин~$B$ является наиболее 
предпочтительным для понятия, обозначаемого термином~$A$;
\item $\mathrm{UF}$~--- Used For. Обращение связи $\mathrm{USE}$. Связывает наиболее 
подходящий термин с~синонимами и~квазисинонимами (менее 
подходящими терминами);
\item $\mathrm{BT}$~--- Broader Term. Связь термина с~термином более общего 
понятия. $A$~$\mathrm{BT}$~$B$ означает, что термин~$B$ обозначает более общее 
понятие по сравнению с~понятием, обозначаемым термином~$A$;
\item $\mathrm{BTG}$~--- Broader Term Generic. Вариант связи~$\mathrm{BT}$ в~случае, 
когда 
термин характеризует разно\-вид\-ность понятия, определяемого более 
общим термином. Например, <<попугаи>> и~<<птицы>>. Наличие связи 
$\mathrm{BTG}$ подразумевает наличие связи~$\mathrm{BT}$; 
\item $\mathrm{BTP}$~--- Broader Term Partitive. Вариант связи $\mathrm{BT}$ в~случае, когда 
термин характеризует часть понятия, определяемого более общим 
термином. Например, <<математика>> и~<<тео\-рия чисел>>. Наличие 
связи $\mathrm{BTP}$ подразумевает наличие связи~BT; 
\item $\mathrm{NT}$, $\mathrm{NTG}$ и~$\mathrm{NTP}$~--- Narrower Term, 
Narrower Term Generic и~Narrower 
Term Partitive~--- обращение связей $\mathrm{BT}$, $\mathrm{BTG}$ и~$\mathrm{BTP}$ 
со\-от\-вет\-ст\-венно; 
{\looseness=1

}
\item $\mathrm{RT}$~--- Related Term. Ассоциативная связь. Используется для 
семантически связанных между собою терминов, не находящихся при 
этом в~одной иерархии и~не являющихся синонимами или 
квазисинонимами. Эта связь проставляется в~тех случаях, когда 
пользователю тезауруса может быть полезно осуществлять поиск или 
индексацию не только по данному термину, но и~по связанному с~ним.
\end{itemize}

\section{Онтология}

     Исходя из вышесказанного, тезаурус~--- это \mbox{полный} 
сис\-те\-ма\-ти\-зи\-ро\-ван\-ный набор терминов о~ка\-кой-ли\-бо об\-ласти знаний 
и~больше относится к~лексике, используемой в~конкретной об\-ласти, тогда 
как онтология описывает ресурсы предметной об\-ласти и~их взаимосвязи. Для 
каждой предметной об\-ласти набор ресурсов может отличаться как по 
формату, так и~по набору самих ресурсов. Поэтому, задавая определение 
самой библиотеки, предлагается использовать для описания ресурсов, 
со\-став\-ля\-ющих контент конкретной предметной об\-ласти, понятия, общие для 
любой из них, т.\,е.\ набор понятий, формирующих описание контента 
биб\-лио\-те\-ки, должен быть настолько универсальным, чтобы мог 
адаптироваться под нужды конкретной об\-ласти. 

Так как одной из основных 
задач, решаемых в~рамках биб\-лио\-те\-ки, как было сказано выше, является 
интеграция данных из различных источников, такой подход поз\-во\-ля\-ет 
реализовать средства интеграции данных в~рамках биб\-лио\-те\-ки, 
адап\-ти\-ру\-емые под условия любой предметной об\-ласти без оглядки на ее 
специфику.
     
     Понятия, составляющие онтологию библиотеки LibMeta, условно 
делятся на предназначенные для:
     \begin{itemize}
\item описания контента предметной об\-ласти;
\item формирования тезауруса любой предметной об\-ласти;
\item описания тематических коллекций; 
\item описания задачи интеграции контента биб\-лио\-те\-ки с~данными 
источников из LOD.
\end{itemize}

     Между этими группами понятий определены семантически значимые 
связи.
     
     Рассмотрим далее основные формальные определения, необходимые 
для описания онтологии.
     
     \smallskip
     
     \noindent
\textbf{Определение~1.} \textit{Контент библиотеки} $C\hm=\langle \mathrm{IR}, A, 
\mathrm{IO}\rangle$ определяется типами ее информационных ресурсов, описанных 
связанными с~ними наборами атрибутов~$A$ и~набором входных данных, 
опре\-де\-ля\-ющих информационные объекты~$\mathrm{IO}$, которые являются 
непосредственно объектами, хранящимися в~биб\-лио\-теке.
\smallskip

\noindent
\textbf{Определение~2.} \textit{Тезаурус библиотеки} $\mathrm{TH}\hm=\langle T, 
R\rangle$ определяется терминами~$T$ и~связями~$R$ между ними. Набор 
терминов~$T$, со\-став\-ля\-ющих описание предметной об\-ласти, строго задан.

\smallskip

\noindent
\textbf{Определение~3.} \textit{Семантические метки} $M\hm=\left\{ 
m_i\right\}$ информационного объекта~--- это термины, которые не попали 
в~тезаурус, но являются необходимыми для специфицирования тематики 
информационного объекта. Семантические метки не связаны, в~отличие от 
терминов тезауруса, связями между собой или с~терминами тезауруса, но 
дают возможность дополнительного тематического разделения 
информационных объектов в~рамках предметной об\-ласти.

\smallskip

\noindent
\textbf{Определение~4.} \textit{Задача интеграции данных биб\-лио\-те\-ки} 
$\mathrm{IT}\hm = \langle \mathrm{DS}, R, A, M, D, D_S\rangle$ \textit{с~внешними 
источниками}~$\mathrm{DS}$ определяется типами ресурсов биб\-лио\-те\-ки и~набором 
их атрибутов~$A$, отображением~$M$ ресурсов~$R$ на схему источника 
данных~$S$ и~набором связей~$D_S$ с~данными из источника.

\smallskip

\noindent
\textbf{Определение~5.} \textit{Коллекция информационных объектов} 
$C\hm= \langle \mathrm{IO}, T, M, \mathrm{DS}\rangle$ представляет собой набор объектов, 
объединенных на основе совокупности признаков:
\begin{enumerate}[(1)]
\item по их термину тезауруса предметной об\-ласти; 
\item по семантическим меткам; 
\item по источнику данных, из которого поступили объекты.
\end{enumerate}

В коллекцию могут входить объекты различных типов ресурсов, заданных 
при описании контента биб\-лио\-те\-ки. При этом коллекции по каждому 
признаку могут формироваться автоматически и~будем называть их 
автоматическими коллекциями. В~случае, когда признаки определяет 
пользователь, будем называть такие коллекции просто \textit{коллекциями.}

\smallskip

\noindent
\textbf{Определение~6.} \textit{Семантически значимыми связями 
библиотеки} $P\hm=\left\{P_i\right\}$ назовем связи, определенные между 
контентом библиотеки, ее предметной об\-ластью (тезаурусом), 
семантическими метками и~объектами источника данных. Выделим 
сле\-ду\-ющие основные связи:
\begin{itemize}
\item $P_1(t, \mathrm{io})$~--- термин те\-за\-у\-ру\-са\,--\,ин\-фор\-ма\-ци\-он\-ный 
объект;
\item $P_2(\mathrm{io}, t)$~--- информационный объ\-ект\,--\,тер\-мин тезауруса;
\item $P_3(r, s)$~--- информационный ре\-сурс\,--\,класс объектов 
источника, где информационный ресурс~--- это общее определение для 
информационных объектов, хранящихся в~сис\-те\-ме; таким образом, 
фактически информационные объекты являются экземплярами 
информационных ресурсов;
\item $P_4(a, s_a)$~--- атрибут информационного ре\-сур\-са\,--\,свой\-ст\-во 
класса источника;
\item $P_5(\mathrm{io}, o_s)$~--- информационный объ\-ект\,--\,эк\-земп\-ляр класса из 
источника данных;
\item $P_6(m, \mathrm{io})$~--- семантическая мет\-ка\,--\,ин\-фор\-ма\-ци\-он\-ный 
объект;
\item $P_7(\mathrm{io}, m)$~--- информационный объ\-ект\,--\,се\-ман\-ти\-че\-ская 
метка.
\end{itemize}

На основе введенных явных связей можно определить связи, которые 
назовем \textit{неявными значимыми связями} (т.\,е.\ заданными по 
некоторым определенным заранее правилам) между семантическими 
метками и~терминами тезауруса и~объектами как самой биб\-лио\-те\-ки, так 
и~экземплярами связанных данных из источников: 
\begin{itemize}
\item $P_8(m, t) \leftarrow P_6(m, \mathrm{io}) \wedge P_2(\mathrm{io}, t)$ семантическая 
мет\-ка\,--\,ин\-фор\-ма\-ци\-он\-ный объект\,--\,тер\-мин тезауруса;
\item $P_9(t, m) \leftarrow P_1(t, \mathrm{io}) \wedge P_7(\mathrm{io}, m)$ термин  
те\-за\-у\-ру\-са\,--\,ин\-фор\-ма\-ци\-он\-ный  
объ\-ект\,--\,се\-ман\-ти\-че\-ская метка;
\item $P_{10}(m, o_s) \leftarrow P_6(m, \mathrm{io}) \wedge P_5(\mathrm{io}, o_s)$ семантическая  
мет\-ка\,--\,ин\-фор\-ма\-ци\-он\-ный объект\,--\,эк\-земп\-ляр класса из 
источника данных;
\item $P_{11}(t, o_s) \leftarrow P_1(t, \mathrm{io}) \wedge P_5(\mathrm{io}, o_s)$ термин  
те\-за\-у\-ру\-са\,--\,ин\-фор\-ма\-ци\-он\-ный объект~--- экземпляр класса из 
источника данных.
\end{itemize}

     Для представления онтологии LibMeta был выбран язык описания 
онтологий OWL (Web Ontology Language)\footnote{{\sf https://www.w3.org/TR/owl-ref}.}. Такая онтология 
со\-сто\-ит из классов, свойств классов и~индивидов. В~терминах 
OWL~$P_1$~\textit{инверсивно}~$P_2$, $P_6$~\textit{инверсивно}~$P_7$, 
$P_8$~\textit{инверсивно}~$P_9$ и~$P_{10}$ \textit{инверсивно}~$P_{11}$. При этом 
правила для неявных связей задаются с~по\-мощью правил SWRL
(Semantic Web Rule Language)\footnote{ {\sf 
https://www.w3.org/Submission/SWRL}.}. Правила SWRL как расширение 
OWL помогают описать 
абстрактный механизм оперирования объектами предметной области и~ее 
закономерности. Правила SWRL дают возможность выводить новые факты из 
существующих утверж\-де\-ний, что повышает эффективность описания 
предметной об\-ласти.
     
     В соответствии с~определениями были введены основные классы 
онтологии. Исходя из определения~1, вводятся классы:
     \begin{enumerate}[1.]
\item $\mathrm{IResource}$ (информационный ресурс биб\-лио\-те\-ки), который 
содержит общую информацию о~типе ресурса, название, 
$URI$ (Universal Resource Identifier)\footnote{{\sf https://tools.ietf.org/html/rfc3986}.} и~информацию об 
ис\-поль\-зу\-емом наборе атрибутов для описания структуры ресурса.
\item $\mathrm{IObject}$ (информационный объект библиотеки), который 
фактически пред\-став\-ля\-ет собой экземпляр некоторого ресурса и~по 
составу ат-\linebreak рибутов соответствует набору атрибутов связанного с~ним 
ресурса. Для описания со\-от\-вет\-ст\-ву\-ющих значений для информационного 
объекта имеется многозначное свойство $\mathrm{value}$, значениями 
которого являются экземпляры вспомогательного класса $\mathrm{AttributeValue}$, 
содержащие информацию о конкретном значении объекта и~соответствующем атрибуте.
\item $\mathrm{Attribute}$ (атрибут, элемент описания информационного ресурса), 
который имеет следующие свойства:
\begin{itemize}
\item[(а)] $\mathrm{name}$~--- название;
\item[(б)] $\mathrm{type}$~--- содержит информацию о типе значений 
этого атрибута и~может включать такие значения, как 
\textit{строка}, \textit{число}, \textit{дата}, \textit{тип ресурса} 
(т.\,е.\ значениями являются объекты некоторого выбранного типа 
ресурса);
\item[(в)] $\mathrm{view}$~--- указывает на об\-ласть 
применения ат-\linebreak рибута в~рамках системы. Может иметь зна-\linebreak чения 
\textit{поисковый} 
(участвует в~формировании поисковых форм), 
\textit{иден\-ти\-фи\-ци\-ру\-ющий} (является обязательным) 
и~\textit{описательный} (содержит дополнительную информацию 
об описываемом объекте).
\end{itemize}
\item $\mathrm{AttributeSet}$ (набор атрибутов, группирующий атрибуты, 
со\-от\-вет\-ст\-ву\-ющие одному пред\-став\-ле\-нию ресурса).
\end{enumerate}

\begin{figure*}[b] %fig1
\vspace*{1pt}
 \begin{center}
 \mbox{%
 \epsfxsize=105.266mm 
 \epsfbox{ata-1.eps}
 }
 \end{center}
\vspace*{-9pt}
\caption{Пример описания информационного ресурса в~терминах онтологии LibMeta}
\end{figure*}

Исходя из определения~2, согласно описанному ранее стандарту  
ISO~2788-1986 для тезаурусов, вводятся классы:
\begin{enumerate}[1.]
\setcounter{enumi}{4}
\item $\mathrm{Thesaurus}$ (тезаурус предметной области)~--- содержит в~себе общую 
информацию о~тезаурусе: название и~авторов (организации и~персоны). 
Наличие этой сущности позволяет загружать готовые тезаурусы, не 
смешивая их с~теми, что уже, быть может, есть в~сис\-теме.
\item $\mathrm{Concept}$~--- сущность, содержащая информацию о~понятиях 
тезауруса. Содержит следующие атрибуты:
\begin{itemize}
\item[(a)] $\mathrm{Name}$~--- название понятия. В~случае, если понятие 
не может иметь названия, пред\-став\-лен\-но\-го в~виде текста, 
используется ка\-кой-ли\-бо идентификатор;
\item[(б)] $\mathrm{RepresentationType}$~--- тип представления понятия. 
Понятие не всегда можно описать словами, иногда для этого 
гораздо больше подходит формула или изображение, поэтому 
необходимо иметь возможность добавления понятия в~любом виде;
\item[(в)] $\mathrm{Image}$~--- изображение;
\item[(г)] $\mathrm{Note}$~--- примечание.
\end{itemize}
\item $\mathrm{ConceptGroup}$~--- тематическое разделение понятий тезауруса. 
\item $\mathrm{HierarchicalRel}$~--- иерархические связи, определяющие 
древовидную структуру словаря. Содержит атрибуты, определяющие 
связи в~соответствии со стандартом ($\mathrm{BT}$, $\mathrm{BTG}$, $\mathrm{BTP}$).
\item $\mathrm{FamilyRel}$~--- горизонтальные связи. Они задают родственные 
отношения между понятиями и~позволяют находить публикации по 
похожим тематикам. Содержит также атрибуты, определяющие связи 
в~соответствии со стандартом ($\mathrm{NT}$, $\mathrm{NTG}$, $\mathrm{NTP}$).
\item $\mathrm{PrefferedTerm}$~--- дескрипторы понятия. Каж\-до\-му понятию 
соответствует единственный дескриптор на каждом языке. 
\item $\mathrm{NonPrefferedTerm}$~--- сюда включаются синонимы. Один 
дескриптор может иметь множество синонимов. В~этот класс объектов 
добавлен атрибут $\mathrm{Visibility}$~--- свойство, от\-ве\-ча\-ющее за видимость 
термина. Имеет два значения~--- $\mathrm{global}$ и~$\mathrm{private}$, 
глобальная и~приватная об\-ласти видимости соответственно. Этот атрибут 
введен для решения проблемы множественных терминологий~--- разные 
люди могут называть одни и~те же объекты по-раз\-но\-му (пусть даже эти 
названия будут похожи). Для того чтобы каж\-до\-му пользователю было 
комфортно работать в~сис\-те\-ме, ему дается возможность создавать свои 
термины, если таковых нет в~глобальной части тезауруса. Эти термины он 
может связывать с~другими терминами из глобальной части и~размечать 
ими свои публикации. Таким образом, если два пользователя создали 
в~своих локальных репозиториях удобные для них ключевые слова, 
разметили ими свои публикации и~связали эти ключевые слова с~одним 
и~тем же термином из глобального тезауруса, то они смогут находить 
и~получать публикации друг друга, пользуясь при этом своими 
терминологиями.
\item $\mathrm{Term}$~--- общий класс, объединяющий дескрипторы и~синонимы. 
Содержит набор свойств, который при необходимости позволяет 
произвольно расширять текстовые описания терминов и~определять связи 
с~информационными объектами системы.
\end{enumerate}

Исходя из определений~3 и~5, вводятся классы:
\begin{enumerate}[1.]
\setcounter{enumi}{12}
\item  $\mathrm{SemanticTag}$~--- класс семантических меток, который обладает 
следующими свойствами:
\begin{itemize}
\item[(а)] $\mathrm{title}$~--- краткое название семантической метки;
\item[(б)] $\mathrm{description}$~--- расширенное описание 
семантической метки.
\end{itemize}
\item $\mathrm{ICollection}$~--- класс коллекций, определенных человеком, 
который обладает следующими свойствами:
\begin{itemize}
\item[(а)] $\mathrm{name}$~--- название коллекции;
\item[(б)] $\mathrm{definition}$~--- описание коллекции;
\item[(в)] $\mathrm{resources}$~--- типы ресурсов, включаемых в~эту 
коллекцию.
\end{itemize}
\end{enumerate}

Исходя из определения 4 вводятся классы:
\begin{enumerate}[1.]
\setcounter{enumi}{14}
\item $\mathrm{DataSource}$ (источники данных LOD)~--- класс, который имеет 
следующие свойства:
\begin{itemize}
\item[(а)]  $\mathrm{name}$~--- название источника;
\item[(б)] $\mathrm{description}$~--- описание источника;
\item[(в)] $\mathrm{url}$~--- точка входа для извлечения данных;
\item[(г)] $\mathrm{resourceMapping}$~--- содержит информацию о~типах 
ресурсов, отображаемых на этот источник, и~со\-от\-вет\-ст\-ву\-ющие классы 
источника. Значениями являются экземпляры класса 
$\mathrm{ResourceMapping}$.
\end{itemize}
\item $\mathrm{ResourceMapping}$~--- класс, содержащий информацию об 
отображаемых на источник данных информационных ресурсах 
библиотеки:
\begin{itemize}
\item[(а)] $\mathrm{resource}$~--- тип ресурсов, отображаемых на этот 
источник;
\item[(б)] $\mathrm{class}$~--- ссылка на соответствующий класс 
источника данных;
\item[(в)] $\mathrm{attributeMappings}$~--- содержит экземпляры класса 
$\mathrm{AttributeMapping}$, содержащих информацию об отображении 
со\-от\-вет\-ст\-ву\-ющих ресурсу атрибутов.
\end{itemize}
\item $\mathrm{AttributeMapping}$~--- класс, содержащий информацию об 
отображаемых на источник данных атрибутах из набора атрибутов, 
со\-от\-вет\-ст\-ву\-юще\-го информационному ресурсу библиотеки:
\begin{itemize}
\item[(а)] $\mathrm{attribute}$~--- атрибут, отображаемый на этот 
источник;
\item[(б)] $\mathrm{property}$~--- ссылка на соответствующее свойство 
класса источника данных.
\end{itemize}
\end{enumerate}

     На рис.~1 и~2 приведены примеры описания конкретного 
информационного ресурса и~информационного объекта в~терминах этой 
онтологии согласно определению~1. 



\begin{figure*} %fig2
\vspace*{1pt}
 \begin{center}
 \mbox{%
 \epsfxsize=155.106mm 
 \epsfbox{ata-2.eps}
 }
 \end{center}
\vspace*{-9pt}
\Caption{Пример описания информационного объекта в~терминах онтологии LibMeta}
\vspace*{6pt}
\end{figure*}

\section{Использование онтологии контента библиотеки и~тезауруса
предметной области при~конструировании семантической библиотеки 
в~LibMeta}

     Для применения тезауруса конкретной предметной 
об\-ласти и~онтологии контента биб\-лио\-те\-ки необходимо придерживаться 
сле\-ду\-ющей последовательности их использования при конструировании 
семантической биб\-лио\-те\-ки в~рамках LibMeta:
     \begin{enumerate}[(1)]
\item на основе введенной модели задается набор информационных 
ресурсов, ис\-поль\-зу\-емых в~биб\-лио\-те\-ке. Для этого необходимо пред\-ста\-вить 
описания содержимого будущей биб\-лио\-те\-ки в~терминах предложенной 
модели. На базе классов, заданных для описания контента биб\-лио\-те\-ки, 
реализован модуль, в~котором задаются базовые свойства, атрибуты для 
ресурсов и~связи между ними;
\item осуществляется окончательная настройка структуры тезауруса. На 
базе определенных классов согласно определению тезауруса реализован 
модуль для его по\-стро\-ения, в~котором задаются используемые связи 
между терминами, расширяется при необходимости описание термина, 
определяются связи с~ресурсами сис\-темы; 
\item для выбора семантических меток можно использовать 
дополнительные словари по предметной области или оставить 
возможность их определения (доопределения) позднее; 
\item на основе заданных классов согласно определению задачи 
интеграции реализован модуль, в~рамках которого осуществляется 
подключение внешних источников данных. Это действие можно 
выполнить на любом этапе жизнедеятельности системы;
\item на основе заданных классов согласно определению коллекций 
реализован модуль, в~рамках которого осуществляются создание 
коллекций и~их наполнение; это можно выполнить также на 
любом этапе.
\end{enumerate}

     На основе выполненных действий происходит автоматическая 
адаптация пользовательских интер\-фей\-сов системы под заданные описания 
ресурсов, со\-став\-ля\-ющих содержимое биб\-лио\-те\-ки. Пользовательский 
интерфейс делится услов\-но на сле\-ду\-ющие категории:
     \begin{itemize}
\item интерфейсы поиска;\\[-13.5pt]
\item интерфейсы просмотра;\\[-13.5pt]
\item интерфейсы редактирования;\\[-13.5pt]
\item интерфейсы загрузки данных.
\end{itemize}

\vspace*{-8pt}

     \subsection*{Пример}
     
     \vspace*{-1pt}
     
     На основе предложенной модели была сконструирована 
библиотека для предметной об\-ласти обыкновенных дифференциальных 
урав\-нений (ОДУ). В~качестве тезауруса использован тезаурус ОДУ, 
разработанный коллективом специалистов в~этой области~\cite{7-ser}. 
     
     Объектами библиотеки рассматривались журнальные математические 
статьи. В~качестве примеров типов ресурсов, соответственно, 
рассматривались \textit{Авторы} и~\textit{Публикации}. Был определен набор 
атрибутов для каждого типа ресурсов в~рамках минимального набора свойств 
на основе Dublin Core\footnote{{\sf  http://dublincore.org.}} для публикаций 
и~FOAF (Friend of a~Friend)\footnote{{\sf http://xmlns.com/foaf/spec.}} для описания авторов. В~качестве 
примера источника данных рассматривались данные о~персонах из сис\-те\-мы 
MathNet\footnote{{\sf http://www.mathnet.ru}.}, которые были смоделированы в~виде 
источника, интегрированного в~LOD. Были определены отобра\-же\-ния 
атрибутов \textit{Авторов} на свойства персон из этого источника 
и~выявлены связи у~почти~50\% авторов рас\-смат\-ри\-ва\-емых пуб\-ли\-ка\-ций, при 
этом авторов было около~700.
     Средствами системы для каждой публикации на основе ее названия, 
аннотации и~ключевых слов были выявлены связи с~тезаурусом ОДУ. 
В~качестве семантических меток были использованы термины 
математической энциклопедии\footnote{{\sf 
https://www.encyclopediaofmath.org.}}~\cite{8-ser}, что позволило дополнительно 
выявить смежные предметные области и~произвести дополнительное 
тематическое разбиение пуб\-ли\-ка\-ций в~рамках предметной области. Такое 
связывание позволило выявить с~некоторой долей вероятности статьи, 
относящиеся к~предметной области ОДУ, и~организовать их в~коллекции на 
основе тезауруса и~выявленных семантических меток. Было использовано 
описание около~2000~пуб\-ли\-ка\-ций, из них около~30\% были отнесены 
к~об\-ласти ОДУ и~имели связи со смежными предметными областями, 
выявленными согласно семантическим меткам.

\vspace*{-10pt}
     
\section{Дальнейшее направление работ}

\vspace*{-3pt}

Работа с~полными текстами предоставленных статей пока находится 
в~активной стадии. Предполагается создание информационного образа 
статей для выделения мик\-ро\-те\-зау\-ру\-са на основе семантических меток 
и~терминов предметной области по каж\-дой статье с~дальнейшим 
определением возможностей расширения используемых тезаурусов или для 
создания облака ключевых понятий отдельных областей знания. 

Отдельной 
задачей является семантическая обработка формул из полных текстов 
и~определение их ключевых слов с~воз\-мож\-ностью дальнейшего поиска по 
формулам, а~также выделение отдельных направлений и~математических школ. При 
этом формулы рас\-смат\-ри\-ва\-ют\-ся как отдельный тип ресурсов сис\-темы.

\vspace*{-6pt}
     
{\small\frenchspacing
 {%\baselineskip=10.8pt
 \addcontentsline{toc}{section}{References}
 \begin{thebibliography}{9}
 
 \vspace*{-2pt}

\bibitem{2-ser} %1
\Au{Серебряков В.\,А., Атаева~О.\,М.} Персональная цифровая библиотека LibMeta как 
среда интеграции связанных открытых данных~// Электронные библиотеки: 
перспективные методы и~технологии, электронные коллекции: Тр. XVI Всеросс. науч. 
конф. RCDL'2014.~--- Дубна: ОИЯИ, 2014. С.~66--71.
\bibitem{1-ser} %2
\Au{Серебряков~В.\,А., Атаева О.\,М.} Основные понятия формальной 
модели семантических биб\-лио\-тек и~формализация процессов интеграции в~ней~// 
Программные продукты и~сис\-те\-мы, 2015. №\,4. С.~180--187.

\bibitem{3-ser}
\Au{Серебряков В.\,А., Атаева~О.\,М.} Информационная модель открытой персональной 
семантической библиотеки LibMeta~// Научный сервис в~сети Интернет: Тр. XVIII 
Всеросс. науч. конф.~--- М.: ИПМ им.\ М.\,В.~Келдыша, 2016. С.~304--313.
\bibitem{4-ser}
\Au{Нгуен М.\,Х., Аджиев~А.\,С.} Описание и~использование тезаурусов 
в~информационных системах, подходы и~реализация~// Электронные библиотеки, 2004. 
Т.~7. №\,1. С.~16--45.
\bibitem{5-ser}
\Au{Gruber T.\,R.} A~translation approach to portable ontologies~// Knowl. Acquis., 
1993. Vol.~5. No.\,2. P.~199--220.
\bibitem{6-ser}
\Au{Bizer C., Heath~T., Berners-Lee~T.} Linked data~--- the story so far~// Int. J.~Semantic 
Web Inf., 2009. Vol.~5. No.\,3. P.~1--22.
\bibitem{7-ser}
\Au{Моисеев~Е.\,И., Муромский~А.\,А., Тучкова~Н.\,П.} Тезаурус  
ин\-фор\-ма\-ци\-он\-но-по\-иско\-вый по предметной области <<обыкновенные 
дифференциальные уравнения>>.~--- М.: МАКС Пресс, 2005. 116~с.
\bibitem{8-ser}
Математическая энциклопедия: В~5~т.~/ Гл.\ ред. И.\,М.~Виноградов.~--- М.: Советская 
энциклопедия, 1977.
 \end{thebibliography}

 }
 }

\end{multicols}

\vspace*{-6pt}

\hfill{\small\textit{Поступила в~редакцию 03.05.17}}

%\vspace*{8pt}

\newpage

\vspace*{-28pt}

%\hrule

%\vspace*{2pt}

%\hrule

%\vspace*{8pt}


\def\tit{ONTOLOGY OF~THE~DIGITAL SEMANTIC LIBRARY LibMeta}

\def\titkol{Ontology of~the~digital semantic library LibMeta}

\def\aut{V.\,A.~Serebryakov and O.\,M.~Ataeva}

\def\autkol{V.\,A.~Serebryakov and O.\,M.~Ataeva}

\titel{\tit}{\aut}{\autkol}{\titkol}

\vspace*{-9pt}


\noindent
A.\,A.~Dorodnicyn Computing Center, Federal Research Center ``Computer Science and 
Control'' of the Russian Academy of Sciences,  40~Vavilov Str., Moscow 119333, Russian 
Federation 



\def\leftfootline{\small{\textbf{\thepage}
\hfill INFORMATIKA I EE PRIMENENIYA~--- INFORMATICS AND
APPLICATIONS\ \ \ 2018\ \ \ volume~12\ \ \ issue\ 1}
}%
 \def\rightfootline{\small{INFORMATIKA I EE PRIMENENIYA~---
INFORMATICS AND APPLICATIONS\ \ \ 2018\ \ \ volume~12\ \ \ issue\ 1
\hfill \textbf{\thepage}}}

\vspace*{3pt}



\Abste{During development of digital libraries, рarticular attention is paid to the 
library content data model. In this case, the content of digital libraries can be 
described in various formats and presented in various ways. The library defined by 
the LibMeta system is considered as a storehouse of structured diverse data with 
the possibility of their integration with other data sources and assumes the 
possibility of specifying its content by describing the subject area. The ontology of 
the semantic library content serves as a means of formalization. It also introduces 
the basic concepts for describing the task of data integration from sources of 
Linked Open Data (LOD), concepts for defining an arbitrary thesaurus. The 
ontology is constructed in such a~way that it is possible to determine the semantic 
library in an arbitrary domain.}

\KWE{semantic library; data model; ontology; data source; search in LOD}

  \DOI{10.14357/19922264180101} 

%\vspace*{-12pt}

%\Ack
%\noindent



%\vspace*{3pt}

  \begin{multicols}{2}

\renewcommand{\bibname}{\protect\rmfamily References}
%\renewcommand{\bibname}{\large\protect\rm References}

{\small\frenchspacing
 {%\baselineskip=10.8pt
 \addcontentsline{toc}{section}{References}
 \begin{thebibliography}{9} 

\bibitem{2-ser-1}
\Aue{Serebryakov, V.\,A., and O.\,M.~Ataeva.} 2014. Personal'naya tsifrovaya 
biblioteka LibMeta kak sreda integratsii svyazannykh otkrytykh dannykh [Personal 
Digital Library Libmeta as an integration environment of linked data]. \textit{Tr. XVI 
Vseross. nauch. konf. RCDL'2014}
[16th All-Russia Scientific Conference RCDL'2014 Proceedings]. Dubna: OIYI. 66--71.
\bibitem{1-ser-1}
\Aue{Serebryakov, V.\,A., and O.\,M.~Ataeva.} 2015. Osnovnye ponyatiya  
formal'noy modeli se\-man\-ti\-che\-skikh bib\-lio\-tek i~formalizatsiya protsessov 
integratsii v~ney [The basic concepts of a~formal model of semantic libraries 
and formalization of the integration processes in it]. \textit{Programmnye produkty 
i~sistemy} [Software Systems] 4:180--187.

\bibitem{3-ser-1}
\Aue{Serebryakov, V.\,A., and O.\,M.~Ataeva.} 2016. Informatsionnaya model' 
otkrytoy personal'noy semanticheskoy biblioteki LibMeta
[Information model of the open
personal semantic library LibMeta]. 
\textit{Nauchnyy servis v~seti Internet: Tr.\ XVIII Vseross. nauch. konf.}   
[Scietifical service in the Internet: 18th All-Russia Scientific Conference Proceedings]. 
Moscow: IPM.  304--313.
\bibitem{4-ser-1}
\Aue{Nguen, M.\,H., and A.\,S.~Adzhiev.} 2004. Opisanie i~ispol'zovanie tezaurusov 
v~informatsionnykh sistemakh, podkhody i~realizatsiya [Description and use of thesauri 
in information systems, approaches and implementation].\linebreak
 \textit{Elektronnye biblioteki} 
[Digital Library] 7(1):16--45.
\bibitem{5-ser-1}
\Aue{Gruber, T.\,R.} 1993. A~translation approach to portable ontologies. 
\textit{Knowl. Acquis.} 5(2):199--220.
\bibitem{6-ser-1}
\Aue{Bizer, C., T.~Heath, and T.~Berners-Lee.} 2009. Linked data~--- the story so far. 
\textit{Int. J.~Semantic Web Inf.} 5(3):1--22.
\bibitem{7-ser-1}
\Aue{Moiseev, E.\,I., A.\,A.~Muromskiy, and N.\,P.~Tuchkova.} 2005. \textit{Tezaurus 
informatsionno-poiskovyy po predmetnoy oblasti ``obyknovennye differentsial'nye 
uravneniya''} [Information search with thesaurus in application area of ordinary 
differential equations].  Moscow: MAKS Press. 116~p.
\bibitem{8-ser-1}
Vinogradov, I.\,M., ed. 1977.
\textit{Matematicheskaya enciklopediya: V~5~t.} [Mathematical encyclopedia: In 5~vols.]. 
Moscow: Sovetskaya Entsiklopediya.

\end{thebibliography}

 }
 }

\end{multicols}

\vspace*{-6pt}

\hfill{\small\textit{Received May 3, 2017}}

%\vspace*{-10pt}

\Contr

\noindent
\textbf{Ataeva Olga M.} (b.\ 1978)~--- junior scientist, A.\,A.~Dorodnicyn 
Computing Centre, Federal Research Center 
``Computer Science and Control'' of the Russian Academy of Sciences, 
40~Vavilov Str., Moscow 119333, Russian Federation; \mbox{oli@ultimeta.ru}

\vspace*{3pt}


\noindent
\textbf{Serebryakov Vladimir A.} (b.\ 1946)~--- Doctor of Science in physics 
and mathematics, professor, Head of Department, A.\,A.~Dorodnicyn Computing 
Centre, Federal Research Center ``Computer Science and Control'' of the Russian 
Academy of Sciences, 40~Vavilov Str., Moscow 119333, Russian Federation; 
\mbox{serebr@ultimeta.ru}


\label{end\stat}


\renewcommand{\bibname}{\protect\rm Литература}  %8
\def\stat{sevast}

\def\tit{О МЕТОДАХ ПОВЫШЕНИЯ ТОЧНОСТИ МНОГОКЛАССОВОЙ КЛАССИФИКАЦИИ 
НА~НЕСБАЛАНСИРОВАННЫХ ДАННЫХ$^*$}

\def\titkol{О методах повышения точности многоклассовой классификации 
на~несбалансированных данных}

\def\aut{Л.\,А.~Севастьянов$^1$, Е.\,Ю.~Щетинин$^2$}

\def\autkol{Л.\,А.~Севастьянов, Е.\,Ю.~Щетинин}

\titel{\tit}{\aut}{\autkol}{\titkol}

\index{Севастьянов Л.\,А.}
\index{Щетинин Е.\,Ю.}
\index{Sevastianov L.\,A.}
\index{Shchetinin E.\,Yu.}


{\renewcommand{\thefootnote}{\fnsymbol{footnote}} \footnotetext[1]
{Работа выполнена при поддержке РФФИ (проект 18-07-00567).}}


\renewcommand{\thefootnote}{\arabic{footnote}}
\footnotetext[1]{Российский университет дружбы народов, leonid.sevast@gmail.com}
\footnotetext[2]{Финансовый университет при Правительстве РФ, riviera-molto@mail.ru}

\vspace*{12pt}

  
  
      
  
  \Abst{Проведены исследования методов преодоления разбалансированности классов 
в~данных с~целью повышения качества классификации с~точностью, более высокой, чем при 
непосредственном использовании алгоритмов классификации к~несбалансированным 
данным. Для повышения точности классификации в~работе предложена схема, состоящая 
в~использовании комбинации алгоритмов классификации и~методов отбора признаков RFE
(Recursive Feature Elimination), 
Random Forest и~Boruta с~предварительным использованием балансирования классов 
методами случайного семплирования, SMOTE (Synthetic Minority
Oversamplimg TEchnique) и~ADASYN (ADAptive SYNthetic sampling). 
На примере данных 
о~заболеваниях кожи проведены компьютерные эксперименты, показавшие, что применение 
алгоритмов семплирования для устранения дисбаланса классов, а также отбора наиболее 
информативных признаков значительно повышает точность результатов классификации. 
Наиболее эффективным по точности классификации оказался алгоритм случайного леса при 
семплировании данных с~использованием алгоритма ADASYN.}
  
  \KW{классификация; несбалансированные данные; семплирование; случайный лес; 
ADASYN; SMOTE}

\DOI{10.14357/19922264200109} 
  
\vspace*{6pt}


\vskip 10pt plus 9pt minus 6pt

\thispagestyle{headings}

\begin{multicols}{2}

\label{st\stat}
  
\section{Введение}

  Задачи классификации относятся к~наиболее популярным в~анализе 
данных~[1]. В~качестве методов, используемых для установления 
принадлежности объекта к~тому или иному классу, чаще всего используют 
машинное обучение с~учителем. Основная идея этого подхода~--- индуктивный 
вывод функции на основе размеченных данных для обучения. Это означает, что 
успешность применения алгоритма машинного обучения с~учителем во многом 
зависит от той выборки объектов, на основе которых он <<обучается>>. 
Большинство подобных алгоритмов требуют от исследователя включения 
сопоставимого числа примеров для каждого из классов, однако зачастую 
сделать сбалансированные наборы данных не представляется возможным 
в~связи с~рядом факторов. Нередко возникают ситуации, когда в~наборе 
данных доля примеров некоторого класса незначительна (этот класс будем 
называть миноритарным, а~другой, пре\-об\-ла\-да\-ющий над первым,~--- 
мажоритарным). Ключевые из них~--- специфика целевой области 
(балансировка данных может понизить показатель их репрезентативности) 
и~разная цена ошибок первого и~второго рода при классификации. Такие 
тенденции хорошо заметны, например, в~кредитном скоринге, медицине, 
маркетинге~[2, 3].  
  
  Вследствие этого возникает проблема обучения модели на 
несбалансированных данных (таковыми являются данные, в~распределении 
которых наблюдается асимметрия, а показатели моды и~среднего значения не 
равны): в~соответствии с~базовыми предположениями, заключенными 
в~большинстве алгоритмов, целью обучения ставится максимизация доли 
правильных решений по отношению ко всем принятым решениям, а~данные 
для обучения и~генеральная совокупность подчиняются одному и~тому же 
распределению. Однако учет данных предположений и~несбалансированности 
выборки приводит к~тому, что модель оказывается неспособна 
классифицировать данные лучше, чем тривиальная модель, полностью 
игнорирующая менее представленный класс и~маркирующая все объекты для 
классификации как принадлежащие к~мажоритарному классу.
  
  С другой стороны, возможно построение слишком сложной модели, 
включающей большое мно\-же\-ство правил, которое при этом будет охватывать\linebreak 
малое число объектов. Такой классификатор может оказаться неэффективным, 
что приведет модель к~переобучению и~некорректным оценкам прогноза. 
Следует отметить, что могут отличаться и~последствия ошибочной 
классификации, причем неверная классификация примеров миноритарного 
класса, как правило, обходится в~разы дороже, чем ошибочная классификация 
объекта из мажоритарного класса. Правильный выбор признаков может 
оказаться более значимой задачей, чем уменьшение времени обработки данных 
или повышения точности классификации. К~примеру, в~медицине нахождение 
минимального набора признаков, оптимального для задачи классификации, 
может стать необходимым условием для постановки диагноза. Таким образом, 
чтобы избежать подобного явления и~достичь хорошего результата, 
необходимо исследовать методы работы с~несбалансированными данными.
  
  В настоящей работе проведены исследования методов преодоления 
разбалансированности классов с~целью повышения качества классификации 
с~точ\-ностью, более высокой, чем при непосредственном применении 
алгоритмов классификации к~несбалансированным классам. Для повышения 
точности классификации в~работе предложена схема, состоящая 
в~использовании комбинации алгоритмов классификации и~методов отбора 
признаков RFE, Random Forest и~Boruta с~предварительным использованием 
балансирования классов методами случайного семплирования, SMOTE 
и~ADASYN.

%\vspace*{-6pt}
  
\section{Алгоритмы балансирования классов}

  Один из подходов к~решению указанной проблемы~--- применение различных 
стратегий семплинга, которые можно разделить на две группы: случайные 
и~специальные~\cite{3-sev}. В~первом случае удаляют некоторое число 
примеров мажоритарного класса (undersampling), во втором~--- увеличивают 
число примеров миноритарного класса (oversampling). 

%\vspace*{-9pt}
  
\subsection{Удаление примеров мажоритарного класса. Алгоритм 
случайного семплирования мажоритарного класса (random 
undersampling)}

  Сначала рассчитывается~$K$~--- число мажоритарных примеров, которые 
необходимо удалить для достижения требуемого уровня соотношения 
различных классов. Затем случайным образом 
выбираются~$K$~мажоритарных примеров и~удаляются. В~случае 
исследуемых данных естественными представляются методы по увеличению 
миноритарного класса. Перейдем к~рассмотрению таких стратегий.

%\vspace*{-9pt}
  
  \subsection{Увеличение миноритарного класса. Дублирование примеров 
миноритарного класса (oversampling). Случайная наивная выборка}
  
  Самый простой способ увеличить число примеров миноритарного класса~--- 
случайным образом выбрать наблюдения из него и~добавить их в~общий набор 
данных, пока не будет достигнут баланс между классами большинства 
и~меньшинства. В~зависимости от того, какое соотношение классов 
необходимо, выбирается число случайных записей для дублирования. Одна из 
проблем со случайной наивной выборкой заключается в~том, что она просто 
дублирует уже существующие данные. К~достоинствам такого подхода 
относятся его простота, легкость реализации и~предоставляемая им 
возможность изменить баланс в~любую нужную сторону. Про недостатки 
нужно говорить отдельно в~соответствии с~тем, какая стратегия семплинга 
используется: несмотря на то что обе из них изменяют общий размер данных 
с~целью поиска баланса, их применение влечет разные последствия. 

В~случае 
undersampling удаление данных может привести к~потере классом важной 
информации и,~как следствие, к~понижению показателя его презентативности. 
В~свою очередь, применение oversampling может привести 
к~переобучению~\cite{3-sev}. 

Такой подход к~восстановлению баланса не 
всегда эффективен, поэтому был предложен специальный метод увеличения 
числа примеров миноритарного класса~--- алгоритм SMOTE~\cite{5-sev}.
  
  Алгоритм SMOTE основан на идее генерации некоторого числа 
искусственных примеров, которые были бы <<похожи>> на имеющиеся 
в~миноритарном классе, но при этом не дублировали их. Для создания новой 
записи находят разность  $d\hm= X_b\hm-X_a$, где~$X_a$ и~$X_b$~--- векторы 
признаков <<соседних>> примеров~$a$ и~$b$ из миноритарного класса. Их 
находят, используя алгоритм ближайших соседей (KNN, k-nearest neighbors). В~данном случае 
необходимо и~достаточно для примера~$b$ получить набор из~$k$~соседей, из 
которого в~дальнейшем будет выбрана запись~$b$. Остальные шаги алгоритма 
KNN не требуются. Далее из~$d$ путем умножения каждого его элемента на 
случайное число в~интервале $(0, 1)$ получают~$\tilde{d}$. Вектор признаков 
нового примера вычисляется путем сложения~$X_a$ и~$\tilde{d}$. Алгоритм 
SMOTE позволяет задавать число записей, которое необходимо искусственно 
сгенерировать. Степень сходства примеров~$a$ и~$b$ можно регулировать 
путем изменения значения~$k$ (числа ближайших соседей). 
  
 Алгоритм SMOTE решает многие проблемы, которые присущи методу случайной 
выборки, и~действительно увеличивает изначальный набор данных таким 
образом, что модель обучается гораздо эффективнее~\cite{9-sev}. Тем не менее 
данный алгоритм имеет и~свои недостатки, главный из которых~--- 
игнорирование мажоритарного класса. Это может проявиться в~том, что при 
сильно разреженном распределении объектов миноритарного класса 
относительно мажоритарного наборы данных <<смешаются>>, т.\,е.\ 
расположатся в~таком виде, что отделить объекты одного класса от другого 
будет очень трудно. Примером данного явления может служить случай, когда
между объектом и~его соседом, на основе которых генерируется новый 
экземпляр, находится объект другого класса. В~результате синтетически 
созданный объект будет находиться ближе к~противоположному классу, чем 
к~классу своих родителей. Кроме того, число сгенерированных с~помощью 
SMOTE экземпляров задается заранее; следовательно, уменьшается 
возможность изменения баланса и~гибкость метода. 

Важно отметить 
существенное ограничение SMOTE. Поскольку он работает путем 
интерполяции между редкими примерами, то может генерировать примеры 
только внутри тела доступных примеров~--- никогда снаружи. Формально 
SMOTE может только заполнить выпуклую оболочку существующих примеров 
меньшинства, но не создавать для них новые внешние области. 

Основное 
преимущество SMOTE по сравнению с~традиционной случайной наивной 
чрезмерной выборкой заключается в~том, что при создании синтетических 
наблюдений вместо повторного использования существующих наблюдений 
данный классификатор с~меньшей вероятностью будет переобучен. В~то же 
время всегда необходимо убедиться, что наблюдения, созданные SMOTE, 
реалистичны.
  
  
 % \vspace*{-6pt}
  
  \subsection{Адаптивный синтетический семплинг и~его обобщения}
  
  В основе данного метода лежат алгоритмы синтетического семплинга, 
основные из которых~--- Borderline-SMOTE и~ADASYN~\cite{6-sev, 7-sev}.
 Borderline-SMOTE накладывает ограничения на 
выбор \mbox{объектов} миноритарного класса, на основе которых генерируются новые 
экземпляры. Происходит это следующим образом: для каждого объекта 
миноритарного класса определяется набор~$k$~ближайших соседей, затем 
производится подсчет, сколько экземпляров из этого набора принадлежат 
к~мажоритарному классу (это число принимается за~$m$). После этого 
отбираются те объекты миноритарного класса, для которых верно неравенство  
  $k/2\hm\leq m\hm<k$. Полученный набор представляет собой экземпляры 
миноритарного класса, находящиеся на границе распределения, и~именно у них 
вероятность оказаться некорректно классифицированными выше, чем у прочих. 
Следует отметить, почему неравенство, определяющее отбор объектов, 
исключает случаи, при которых все~$k$~соседей принадлежат 
к~мажоритарному классу: это связано с~тем, что подобные экземпляры 
расположены в~зоне <<смешивания>> двух классов и~на их основе могут быть 
сгенерированы лишь искажающие процесс обучения модели объекты. В~связи 
с~этим они объявляются шумом (\textit{англ}.\ noise) и~игнорируются алгоритмом. 
  
  Алгоритм ADASYN же, в~свою очередь, основывается на систематическом 
методе, позволяющем адаптивно генерировать разные объемы данных 
в~соответствии с~их распределениями~\cite{6-sev}. Входные данные для 
алгоритма~--- обучающий набор данных: $D_r$ с~$m$~выборками с~$\{x_i, 
y_i\}$, $i\hm=\overline{1,m}$, где~$x_i$~---\linebreak $n$-мер\-ный вектор 
в~пространстве признаков, а~$y_i$~--- соответствующий класс. 

Пусть~$m_r$ 
и~$m_x$~--- число образцов классов меньшинства и~большинства 
соответственно, такие что $m_r\ll m_x$ и~$m_r\hm+m_x\hm=m$. Псевдокод 
алгоритма имеет следующий вид.
  \begin{enumerate}[1.]
\item Вычислить пропорцию классов $d \hm= m_r/m_x$.
\item Если $d < d_x$ (где $d_x$~--- заданный порог для максимально 
допустимого дисбаланса классов), то:
\begin{itemize}
\item[(a)] найти число синтетически создаваемых образцов минорного класса 
$G\hm= (m_x\hm- m_r)\beta$,  где $\beta$~---
параметр, используемый для определения 
желаемого уровня баланса ($\beta\hm=1$  означает полный баланс классов); 
\item[(б)] для каждого $x_i\hm\in \mbox{minority\ class}$ найти~$K$~ближайших 
соседей, используя евклидово расстояние, и~вычислить $r_i\hm= \Delta_i/K$;
\item[(в)] нормализовать $r_x\hm= r_i/\sum\nolimits_i r_i$ так, чтобы~$r_x$ стал 
плотностью распределения;
\item[(г)] вычислить $g_i\hm= r_x  G$ синтетической выборки, сформированной 
для каждого образа из класса меньшинства, где~$G$~--- общее число примеров 
синтетических данных; 
\item[(д)] для каждого примера данных из класса меньшинства~$x_i$ создать 
примеры синтетических данных~$g_i$ в~соответствии со сле\-ду\-ющи\-ми шагами: 
\begin{itemize}

\pagebreak

\item[--]  \textit{в цикле от~$1$ до~$i$}:
  \begin{itemize}
  \item[(i)] случайным образом выбрать один пример данных меньшинства, 
$x_u$ из~$K$~ближайших соседей для данных~$x_i$;
  \item[(ii)] создать пример синтетических данных: 
  $$
  g_i= x_i+ (x_u-  x_i)\lambda,
  $$
   где $(x_u\hm- x_i)$~--- $n$-мер\-ный вектор евклидова 
пространства; $\lambda$~---  случайное число: $\lambda\hm\in [0,1]$.
  \end{itemize}
  \end{itemize}
  \end{itemize}
\end{enumerate}
  
  Основное различие между SMOTE и~ADASYN заключается в~способах 
создания синтетических выборочных образцов для класса меньшинства. 
В~ADASYN используется функция плотности~$r_x$, определяющая число 
синтетических образцов, которые будут созданы для конкретной точки, тогда 
как в~SMOTE существует единый вес для всех точек меньшинства.
  
\section{Исследуемые данные: описание и~характеристики}

  В настоящей работе для тестирования и~сравнительного анализа описанных 
выше методов устранения дисбаланса классов был использован набор данных 
о~заболеваниях кожи. Диагностика эри\-те\-ма\-тоз\-но-плос\-ко\-кле\-точ\-ных 
заболеваний~--- серь\-ез\-ная проб\-ле\-ма в~дерматологии, а современные принципы 
диагностики и~лечения опираются на наиболее раннее обнаружение заболевания. 
Все они имеют общие клинические особенности с~очень небольшими 
различиями. Еще одна трудность для диагностики заключается в~том, что 
заболевание может проявлять признаки другого заболевания на начальной 
стадии и~иметь характерные признаки на последующих стадиях. 
  
  Исследуемые данные были созданы компанией Nielsen в~1998~г.\ и~содержат 
366~наблюдений, формирующих 6~классов, которые могут быть 
охарактеризованы 34~признаками~\cite{8-sev}. Классами являются: 
\begin{itemize}
\item псориаз 
(класс~1)~--- 112~случаев; 
\item себорейный дерматит (класс~2)~--- 72~случая; 
\item плоский лишай (класс~3)~--- 61~случай; 
\item розовый лишай (класс~4)~--- 49~случаев; 
\item хронический дерматит (класс~5)~--- 52~случая; 
\item красный волосяной 
лишай (класс~6)~--- 20~случаев.
\end{itemize}
 Полное описание данных приведено 
в~\cite{11-sev}.
  
\section{Компьютерные эксперименты}

  Исследования данных проводились по сле\-ду\-юще\-му алгоритму.
  \begin{enumerate}[1.]
\item Предварительная обработка данных: заполнение пропусков в~данных 
и~использование кодирования признаков. 
\item Балансирование классов с~помощью описанных выше алгоритмов 
семплинга.
\item Отбор признаков по их важности. 
\item Классификация с~использованием логистической регрессии и~метода 
опорных векторов (SVM~--- Support Vector Machine).
\item Оценка качества классификации.
\end{enumerate}

  В настоящей работе отбор признаков по их важности и~информативности 
был осуществлен следующими методами:
\begin{itemize}
\item[(а)]~рекурсивное исключение 
признаков RFE~\cite{9-sev}; 
\item[(б)] деревья решений RF~\cite{10-sev}; 
\item[(в)] Boruta~\cite{4-sev}.
\end{itemize}
  
  Алгоритм Random Forest представляет собой ансамбль многочисленных 
алгоритмов классификации (деревьев решений). Каждый из этих 
классификаторов строится на случайном подмножестве объектов 
и~случайном подмножестве признаков. Пусть обучающая выборка состоит 
из~$N$~примеров, размерность пространства признаков равна~$M$ 
и~задан дополнительный параметр~$m$. Все деревья строятся независимо 
друг от друга по следующей процедуре.
  \begin{enumerate}[1.]
  \item Сгенерируем случайную подвыборку с~повторением размером~$n$ из 
обучающей выборки.
  \item Построим решающее дерево, классифицирующее примеры данной 
подвыборки, причем в~ходе создания очередного узла дерева будем выбирать 
признак, на основе которого производится разбиение, не из 
всех~$M$~признаков, а~лишь из~$m$~случайно выбранных.
  \item Дерево строится до полного исчерпания подвыборки и~не 
подвергается процедуре отсечения ветвей.
  \end{enumerate}
  
     \begin{table*}[b]\small %tabl1
  \begin{center}
  \Caption{Результаты классификации методом опорных векторов}
  \vspace*{2ex}
  
  \begin{tabular}{|l|l|c|c|c|c|}
  \hline
\multicolumn{1}{|c|}{\tabcolsep=0pt\begin{tabular}{c}Семплинг\\ для 
несбалансированных\\ классов\end{tabular}}&
\multicolumn{1}{c|}{\tabcolsep=0pt\begin{tabular}{c}Методы\\ выбора\\ 
признаков\end{tabular}}&
\tabcolsep=0pt\begin{tabular}{c}Число\\ выбранных\\ 
признаков\end{tabular}&Точность&F1-мера&Полнота\\
\hline
\raisebox{-18pt}[0pt][0pt]{Несбалансированные данные}&Все признаки&630&0,9324&0,9337&0,9324\\
&RFE&\hphantom{9}65&0,9595&0,9598&0,9595\\
&Случайный лес&\hphantom{9}32&0,9595&0,9590&0,9595\\
&Boruta&207&0,9324&0,9330&0,9324\\
\hline
\raisebox{-18pt}[0pt][0pt]{Случайная выборка}&Все признаки&630&0,9324&0,9337&0,9324\\
&RFE&\hphantom{9}44&0,9359&0,9468&0,9459\\
&Случайный лес&\hphantom{9}44&0,9465&0,9466&0,9465\\
&Boruta&284&0,9595&0,9598&0,9595\\
\hline
\raisebox{-18pt}[0pt][0pt]{SMOTE}&Все признаки&630&0,9324&0,9337&0,9324\\
&RFE&\hphantom{9}68&0,9595&0,9730&0,9730\\
&Случайный лес&\hphantom{9}42&0,9595&0,9072&0,9054\\
&Boruta&257&0,9459&0,9337&0,9324\\
\hline
\raisebox{-18pt}[0pt][0pt]{ADASYN}&Все признаки&630&0,9324&0,9337&0,9324\\
&RFE&\hphantom{9}44&0,9459&0,9459&0,9459\\
&Случайный лес&\hphantom{9}40&\textbf{0,9845}&0,9330&0,9324\\
&Boruta&276&0,9459&0,9602&0,9595\\
\hline
\end{tabular}
\end{center}
\end{table*}
  
  Классификация объектов проводится путем голо\-со\-ва\-ния: каждое дерево 
ансамбля относит классифицируемый объект к~одному из классов, 
и~побеждает класс, за который проголосовало наибольшее число деревьев. 
Для применения RF в~задаче оценки важности признаков необходимо обучить 
алгоритм на выборке и~для каждого примера обучающей выборки посчитать 
out-of-bag-ошиб\-ку~\cite{10-sev}. 

Пусть~$X_n$~--- бутстрэпированная 
выборка дере\-ва~$b_n$. Бутстрэппинг представляет собой 
выбор~$l$~объектов из выборки с~возвращением, в~результате чего 
некоторые объекты выбираются\linebreak несколь\-ко раз, а~некоторые~--- ни разу. 
Помещение нескольких копий одного объекта в~бутстрэпированную выборку 
соответствует выставлению веса при данном объекте, соответствующее ему 
слагаемое несколько раз \mbox{войдет} в~функционал, и~поэтому штраф за 
ошибку на нем будет больше. Пусть~$L(y, z)$~--- функция потерь,  
$y_i$~--- ответ на $i$-м объекте обучающей выборки, тогда  
out-of-bag-ошиб\-ка вычисляется по следующей формуле:
  $$
  \mathrm{OOB}=\sum\limits^l_{i=1} L\left( y_i, \fr{\sum\nolimits^N_{n=1} [x_i\not\in 
X^l_n] b_n(x_i)} {\sum\nolimits^N_{n=1} [x_i\not\in X^l_n]}\right)\,.
  $$
  
  Затем для каждого объекта такая ошибка усредняется по всему 
случайному лесу. Чтобы оценить важность признака, его значения 
перемешиваются для всех объектов обучающей выборки  
и~out-of-bag-ошиб\-ка считается снова. Важность признака оценивается 
путем усреднения по всем деревьям разности показателей  
out-of-bag-оши\-бок до и~после перемешивания значений. При этом значения 
таких ошибок нормализуются на стандартное отклонение. 
  
  Boruta~--- эвристический алгоритм отбора значимых признаков, 
основанный на использовании Random Forest~\cite{4-sev}. На каждой 
итерации удаляются признаки, у~которых Z-ме\-ра меньше максимальной  
Z-ме\-ры среди добавленных признаков. Чтобы получить Z-ме\-ру 
признака, необходимо посчитать важность признака, полученную  
с~по\-мощью встроенного алгоритма в~Random Forest, и~поделить ее на 
стандартное отклонение важности признака. Добавленные признаки 
получаются следующим образом: копируются признаки, имеющиеся 
в~выборке, а затем каждый новый признак заполняется путем перетасовки 
его значений. В~целях получения статистически значимых результатов эта 
процедура повторяется несколько раз, переменные генерируются независимо 
на каждой итерации. Запишем пошагово алгоритм Boruta.
  \begin{enumerate}[1.]
  \item Добавить в~данные копии всех признаков. В~дальнейшем копии 
будем называть скрытыми признаками.
  \item Случайным образом перемешать каждый скрытый признак. 
   \item Запустить Random Forest и~получить Z-ме\-ру всех признаков.
  \item Найти максимальную Z-ме\-ру из всех Z-мер для скрытых 
признаков.
  \item Удалить признаки, у которых Z-ме\-ра меньше, чем найденная на 
предыдущем шаге. 
  \item Повторять все шаги до тех пор, пока Z-ме\-ра всех признаков не 
станет больше, чем максимальная Z-ме\-ра скрытых признаков.
  \end{enumerate}
  
   \begin{table*}\small %tabl2
  \begin{center}
  \Caption{Результаты классификации с~использованием логистической регрессии}
  \vspace*{2ex}
  
  \begin{tabular}{|l|l|c|c|c|c|}
  \hline
\multicolumn{1}{|c|}{\tabcolsep=0pt\begin{tabular}{c}Семплинг\\ для 
несбалансированных\\ классов\end{tabular}}&
\multicolumn{1}{c|}{\tabcolsep=0pt\begin{tabular}{c}Методы\\ выбора\\ 
признаков\end{tabular}}&
\tabcolsep=0pt\begin{tabular}{c}Число\\ выбранных\\ 
признаков\end{tabular}&Точность&F1-мера&Полнота\\
\hline
\raisebox{-18pt}[0pt][0pt]{Несбалансированные данные}&Все признаки&630&0,9330&0,9234&0,9231\\
&RFE&\hphantom{9}19&0,9459&0,9459&0,9459\\
&Случайный лес&\hphantom{9}32&0,9595&0,9590&0,9595\\
&Boruta&200&0,9595&0,9590&0,9595\\
\hline
\raisebox{-18pt}[0pt][0pt]{Случайная выборка}&Все признаки&630&0,9630&0,9634&0,9630\\
&RFE&\hphantom{9}48&0,9665&0,9866&0,9865\\
&Случайный лес&\hphantom{9}44&0,9730&0,9730&0,9730\\
&Boruta&290&0,9730&0,9765&0,9730\\
\hline
\raisebox{-18pt}[0pt][0pt]{SMOTE}&Все признаки&630&0,9730&0,9734&0,9730\\
&RFE&\hphantom{9}20&0,9459&0,9459&0,9459\\
&Случайный лес&\hphantom{9}41&0,9324&0,9330&0,9324\\
&Boruta&264&0,9595&0,9590&0,9595\\
\hline
\raisebox{-18pt}[0pt][0pt]{ADASYN}&Все признаки&630&0,9530&0,9534&0,9530\\
&RFE&\hphantom{9}67&0,9595&0,9602&0,9595\\
&Случайный лес&\hphantom{9}42&\textbf{0,9895}&\textbf{0,9859}&\textbf{0,9893}\\
&Boruta&245&0,9595&0,9590&0,9595\\
\hline
\end{tabular}
\end{center}
\end{table*}
  
\section{Результаты исследования и~их~обсуждение}
 
  Для решения задачи многоклассовой классификации на несбалансированных 
данных были вы\-бра\-ны алгоритмы машинного обучения: ло\-ги\-стическая 
регрессия и~метод опорных векторов\linebreak с~линейным ядром (Linear SVM). Все 
вычисления были программно реализованы на языке PYTHON, их результаты, 
данные, а также коды программ размещены в~репозитории авторов данной 
статьи~\cite{11-sev}. Для сравнения результатов классификации были 
использованы три метрики: точность (accuracy), полнота (precision) и~F1-ме\-ра. 
Результаты проведенных исследований представлены в~табл.~1.
  

  
  В ее первом столбце перечислены применявшиеся методы семплирования. 
Во втором столбце приведены применявшиеся методы отбора признаков, 
в~третьей~--- число отобранных при этом признаков. В~остальных столбцах 
приведены значения метрик качества, полученные в~результате применения 
к~преобразованным данным алгоритма опорных векторов (SVM). 

Аналогично 
построена табл.~2, содержащая результаты классификации с~применением 
логистической регрессии.
  
 
  
  Из анализа полученных результатов, приведенных в~табл.~1 и~2, можно 
видеть, что во всех случа\-ях применение методов семплирования позволило\linebreak 
получить более высокую точность классификации, чем на несбалансированных 
данных. В~рамках применения описанной в~работе схемы наилучшая точность 
классификации была достигнута в~результате применения алгоритма 
балансирования классов ADASYN и~затем отбора признаков алгоритмом 
случайного леса. Для сравнения, в~работах других исследователей, 
проводивших подобные исследования, например~\cite{9-sev, 10-sev}, точность 
классификации достигала лишь~93\%.
  
\section{Заключение}

  В настоящей работе предложена схема повышения точности классификации 
на несбалансированных данных с~использованием алгоритмов балансирования 
классов и~отбора признаков, таких как RFE, Boruta, Random Forest и~др. 
Результаты проведенных в~работе вычислительных экспериментов показали 
эффективность ее применения для решения поставленной задачи. В частности, 
алгоритм ADASYN, по сравнению с~другими алгоритмами, повысил точность 
классификации до~98\%. В~заключение стоит отметить, что рассмотренная 
в~работе проблема по-прежнему актуальна, а существующие методы могут 
быть улучшены.
  
{\small\frenchspacing
 {%\baselineskip=10.8pt
 \addcontentsline{toc}{section}{References}
 \begin{thebibliography}{99}
\bibitem{1-sev}
\Au{Паттерсон Дж., Гибсон~А.} Глубокое обучение с~точки зрения практика~/ Пер. с~англ. 
А.\,А.~Слинкина.~--- М.: ДМК Пресс, 2018. 417~с. (\Au{Patterson~J., Gibson~A.} Deep 
learning: A~practitioner's approach.~--- O'Reilly Media, 2017. 532~p.)

\bibitem{3-sev} %2
\Au{Japkowicz N., Stephen~S.} 
The class imbalance problem: A~systematic study~// Intell. 
Data Anal., 2002. Vol.~6. Iss.~5. P.~429--449. doi: 10.3233/IDA-2002-6504.

\bibitem{2-sev} %3
\Au{He H., Garcia~A.} Learning from imbalanced data~// IEEE~T. Knowl. Data 
En., 2009. Vol.~21. Iss.~9. P.~1263--1284. doi: 10.1109/TKDE.2008.239.


\bibitem{5-sev} %4
\Au{Chawla N.\,V., Bowyer~K.\,W., Hall~L.\,O., Kegelmeyer~W.\,P.} SMOTE: Synthetic minority 
over-sampling technique~// J.~Artif. Intell. Res., 2002. Vol.~16. P.~321--357. doi: 
10.1613/jair.953.

\bibitem{9-sev} %5
\Au{Lin X., Yang~F., Zhou~L.} A~support vector machine recursive feature elimination feature 
selection method based on artificial contrast variables and mutual information~// 
J.~Chromatogr.~B, 2012. 
Vol.~10. P.~149--155. doi: 10.1016/j.jchromb.2012.05.020.

\bibitem{7-sev} %6
\Au{Han H., Wen-Yuan~W., Bing-Huan~M.} Borderline-SMOTE: 
A~new over-sampling method in 
imbalanced data sets learning~// Advances in
intelligent computing~/
Eds. De-Shuang Huang, Xiao-Ping Zhang, Guang-Bin Huang.~--- 
Lecture notes in computer science book ser.~--- Springer, 2005. Vol.~3644. P.~878--887. 
doi: 10.1007/11538059\_91.

\bibitem{6-sev} %7
\Au{He H., Bai~Ya., Garcia~A., Li~Sh.} ADASYN: Adaptive synthetic sampling approach for 
imbalanced learning~// IEEE  Joint Conference (International) on Neural Networks (IEEE World 
Congress on Computational Intelligence).~--- IEEE, 2008. P.~1322--1328.

\bibitem{8-sev} %8
\Au{Murphy P.\,M., Aha~D.\,W.} UCI repository of machine learning databases.~--- 
Irvine, CA, USA: 
University of California, Department of Information and Computer Science, 
1998. {\sf 
https://www.ics.uci.edu/mlearn/\linebreak MLRepository.html}.


\bibitem{11-sev} %9
Dermatology-article. {\sf https://github.com/riviera2015/\linebreak Dermatology-article}.
\bibitem{10-sev} %10
\Au{Tuv E., Borisov~A., Runger~G., Torkkola~K.} Feature selection with ensembles, artificial 
variables, and redundancy elimination~// J.~Mach. Learn. Res., 2009. Vol.~10.  
P.~1341--1366.

\bibitem{4-sev} %11
\Au{Kursa M., Rudnicki~W.} Feature selection with the Boruta package~// J.~Stat. 
Softw.,  2010. Vol.~36. Iss.~11. P.~1--13. doi: 10.18637/jss.v036.i11.

 \end{thebibliography}

 }
 }

\end{multicols}

\vspace*{-9pt}

\hfill{\small\textit{Поступила в~редакцию 29.11.19}}

\vspace*{6pt}

%\pagebreak

%\newpage

%\vspace*{-28pt}

\hrule

\vspace*{2pt}

\hrule

\vspace*{-4pt}

\def\tit{ON METHODS FOR~IMPROVING THE~ACCURACY OF~MULTICLASS 
CLASSIFICATION ON~IMBALANCED DATA\\[-5pt]}


\def\titkol{On methods for~improving the~accuracy of~multiclass 
classification on~imbalanced data}

\def\aut{L.\,A.~Sevastianov$^1$ and~E.\,Yu.~Shchetinin$^2$}

\def\autkol{L.\,A.~Sevastianov and~E.\,Yu.~Shchetinin}

\titel{\tit}{\aut}{\autkol}{\titkol}

\vspace*{-16pt}


\noindent
$^1$Peoples' Friendship University of Russia (RUDN University), 6~Miklukho-Maklaya Str., 
Moscow 117198, Russian\linebreak
$\hphantom{^1}$Federation

\noindent
$^2$Financial University under the Government of the Russian Federation, 49~Leningradsky 
Prospekt, Moscow\linebreak
$\hphantom{^1}$125993, Russian Federation

\def\leftfootline{\small{\textbf{\thepage}
\hfill INFORMATIKA I EE PRIMENENIYA~--- INFORMATICS AND
APPLICATIONS\ \ \ 2020\ \ \ volume~14\ \ \ issue\ 1}
}%
 \def\rightfootline{\small{INFORMATIKA I EE PRIMENENIYA~---
INFORMATICS AND APPLICATIONS\ \ \ 2020\ \ \ volume~14\ \ \ issue\ 1
\hfill \textbf{\thepage}}}

\vspace*{1pt} 
  
  
  
\Abste{This paper studies methods to overcome the imbalance of classes in order to improve the 
quality of classification with accuracy higher than the direct use of classification algorithms to 
unbalanced data. The scheme to improve the accuracy of classification is proposed, consisting in the 
use of a~combination of classification algorithms and methods of selection of features such as RFE
(Recursive Feature Elimination), 
Random Forest, and Boruta with the preliminary use of balancing classes by random sampling 
methods, SMOTE (Synthetic Minority
Oversamplimg TEchnique)
and ADASYN (ADAptive SYNthetic sampling). 
By the example of data on skin diseases, computer experiments 
were conducted which showed that the use of sampling algorithms to eliminate the imbalance of 
classes as well as the selection of the most informative features significantly increases the accuracy 
of the classification results. The most effective classification accuracy was the Random Forest 
algorithm for sampling data using the ADASYN algorithm.}
  
\KWE{imbalanced data; classification; sampling; random forest; ADASYN; SMOTE}
  

  
\DOI{10.14357/19922264200109} 

\vspace*{-24pt}

\Ack

\vspace*{-3pt}

\noindent
The paper was prepared with the support of the Russian Foundation for Basic 
Research (project 18-07-00567).

%\vspace*{6pt}

  \begin{multicols}{2}

\renewcommand{\bibname}{\protect\rmfamily References}
%\renewcommand{\bibname}{\large\protect\rm References}

{\small\frenchspacing
 {%\baselineskip=10.8pt
 \addcontentsline{toc}{section}{References}
 \begin{thebibliography}{99}
\bibitem{1-sev-1} %1
\Aue{Patterson, J., and А.~Gibson.} 2017. \textit{Deep learning: A~practitioner's approach}. 
O'Reilly Media. 532~p.

\bibitem{3-sev-1} %2
\Aue{Japkowicz, N., and S.~Stephen.} 2002. The class imbalance problem: A~systematic study. 
\textit{Intell. Data Anal.} 6(5):429--449. doi: 10.3233/IDA-2002-6504.

\bibitem{2-sev-1} %3
\Aue{He, H., and A.~Garcia.} 2009. Learning from imbalanced data. \textit{IEEE~T. 
Knowl. Data En.} 21(9):1263--1284. doi: 10.1109/TKDE.2008.239.


\bibitem{5-sev-1} %4
\Aue{Chawla, N.\,V., K.\,W.~Bowyer, L.~O.Hall, and W.\,P.~Kegelmeyer.} 2002. SMOTE: 
Synthetic minority over-sampling technique. \textit{J.~Artif. Intell. Res.} 
16:321--357.

\bibitem{9-sev-1} %5
\Aue{Lin, X., F.~Yang, and L.~Zhou.} 2012. A support vector machine recursive feature 
elimination feature selection method based on artificial contrast variables and mutual information. 
\textit{J.~Chromatogr.~B} 
10:149--155. doi: 10.1016/ j.jchromb.2012.05.020.

\bibitem{7-sev-1} %6
\Aue{Han, H., W.~Wen-Yuan, and M.~Bing-Huan.} 2005. Borderline-SMOTE: A~new  
over-sampling method in imbalanced data sets learning. \textit{Advances  
in intelligent computing}. 
Eds. De-Shuang Huang, Xiao-Ping Zhang, and Guang-Bin Huang. 
Lecture notes in computer science book ser. Springer.
3644:878--887. http://dx.doi.org/ 10.1007/11538059\_91.

\bibitem{6-sev-1} %7
\Aue{He, H., Ya.~Bai, A.~Garcia, and Sh.~Li.} 2008. ADASYN: Adaptive synthetic sampling 
approach for imbalanced learning. \textit{IEEE Joint Conference (International) on Neural 
Networks (IEEE World Congress on Computational Intelligence)}. China. 1322--1328.

\bibitem{8-sev-1}
\Aue{Murphy, P.\,M., and D.\,W.~Aha.} 1998. UCI repository
 of machine learning databases. 
Irvine, CA: University of California-Irvine, Department of Information 
and Computer Science. Available at: 
{\sf https://www.ics. uci.edu/mlearn/MLRepository.html} (accessed December~27, 2019).


\bibitem{11-sev-1} %9
Dermatology-article. Available at: {\sf  
https://github.com/ riviera2015/Dermatology-article} (accessed December~27, 2019).
\bibitem{10-sev-1}
\Aue{Tuv, E., A.~Borisov, G.~Runger, and K.~Torkkola.} 2009. Feature selection with ensembles, 
artificial variables, and redundancy elimination. \textit{J.~Mach. Learn. Res.}
10:1341--1366.


\bibitem{4-sev-1} %11
\Aue{Kursa, M., and W.~Rudnicki.} 2010. Feature selection with the Boruta package. 
\textit{J.~Stat. Softw.} 36(11):1--13. doi: 10.18637/jss.v036.i11.

\end{thebibliography}

 }
 }

\end{multicols}

%\vspace*{-7pt}

\hfill{\small\textit{Received November 29, 2019}}

%\pagebreak

%\vspace*{-22pt}
  
\Contr

\noindent
\textbf{Sevastianov Leonid A.} (b.\ 1949)~--- Doctor of Science in physics and mathematics, 
professor, Peoples' Friendship University of Russia (RUDN University), 6~Miklukho-Maklaya Str., 
Moscow 117198, Russian Federation; \mbox{leonid.sevast@gmail.com}

\vspace*{3pt}

\noindent
\textbf{Shchetinin Eugene Yu.} (b.\ 1962)~--- Doctor of Science in physics and mathematics, 
professor, Department of Data Analysis, Decision-Making and Financial Technology, Financial 
University under the Government of the Russian Federation, 49~Leningradsky Prosp., Moscow 
125993, Russian Federation; \mbox{riviera-molto@mail.ru} 
  

  
\label{end\stat}

\renewcommand{\bibname}{\protect\rm Литература}  %9
\def\stat{pop+sim}

\def\tit{МОДЕЛИРОВАНИЕ ПРОЦЕССА МОНИТОРИНГА СИСТЕМ ИНФОРМАЦИОННОЙ 
БЕЗОПАСНОСТИ\\ НА~ОСНОВЕ СИСТЕМ МАССОВОГО ОБСЛУЖИВАНИЯ$^*$}

\def\titkol{Моделирование процесса мониторинга систем информационной 
безопасности на~основе СМО} %систем массового обслуживания}

\def\aut{Г.\,А.~Попов$^1$, С.\,Ж.~Симаворян$^2$, А.\,Р.~Симонян$^3$, 
Е.\,И.~Улитина$^4$}

\def\autkol{Г.\,А.~Попов, С.\,Ж.~Симаворян, А.\,Р.~Симонян, 
Е.\,И.~Улитина}

\titel{\tit}{\aut}{\autkol}{\titkol}

\index{Попов Г.\,А.}
\index{Симаворян С.\,Ж.}
\index{Симонян А.\,Р.} 
\index{Улитина Е.\,И.}
\index{Popov G.\,A.}
\index{Simavoryan S.\,Zh.}
\index{Simonyan A.\,R.} 
\index{Ulitina E.\,I.}


{\renewcommand{\thefootnote}{\fnsymbol{footnote}} \footnotetext[1]
{Исследование выполнено при финансовой поддержке РФФИ (проект 19-01-00383).}}


\renewcommand{\thefootnote}{\arabic{footnote}}
\footnotetext[1]{Астраханский государственный технический университет, popov@astu.org}
\footnotetext[2]{Сочинский государственный университет, simsim58@mail.ru}
\footnotetext[3]{Сочинский государственный университет, oppm@mail.ru}
\footnotetext[4]{Сочинский государственный университет, ulitinaelena@mail.ru}

%\vspace*{-12pt}
  

  \Abst{Рассматривается задача моделирования процесса мониторинга в~системах 
информационной безопасности по выявлению необнаруженных злоумышленных атак на 
основе использования методов теории массового обслуживания. Процесс мониторинга 
сводится к~анализу потока заявок на обслуживание системой обработки данных как потока 
потенциально возможных злоумышленных действий. При выявлении вызова мониторинг 
немедленно прекращается и~начинается обслуживание выявленного вызова. В~рамках 
указанной модели получены функциональные соотношения для следующих двух наиболее 
важных характеристик: вероятности состояний системы и~вероятности числа невыявленных 
вызовов в~моменты окончания обслуживания. Нахождение указанных характеристик 
позволит более эффективно организовать процесс выявления злоумышленных атак на 
систему обработки данных при данной схеме обработки выявленных вызовов.}
   
  \KW{защита информации; информационная безопасность; система массового 
обслуживания; вероятность}

\DOI{10.14357/19922264200110} 
  
%\vspace*{-3pt}


\vskip 10pt plus 9pt minus 6pt

\thispagestyle{headings}

\begin{multicols}{2}

\label{st\stat}

  \section{Введение}
  
  Одной из актуальных проблем процесса обеспечения информационной 
безопасности в~системах обработки данных является проблема выявления 
успешных злоумышленных атак, оказавшихся незамеченными для систем 
обнаружения вторжений, т.\,е.\ атак, которые система обнаружения 
вторжений не обнаружила на этапе поступления запроса на обработку 
и~пропустила как незлоумышленное действие. В~этом случае система 
обработки данных либо даже не имеет понятия о том, что была подвергнута 
успешной злоумышленной атаке, либо узнает об этом слишком поздно, когда 
уже необходимо предпринять меры по локализации и~ликвидации последствий 
несанкционированного вторжения. Примером может служить появление новых 
вирусов, шпионских программ, когда су\-щест\-ву\-ющие средства защиты еще не 
выработали соответствующих механизмов противодействия и~в~течение 
некоторого периода новый вирус или программа абсолютно безнаказанно 
действует в~компьютерных системах. 
  
  В~настоящее время большое внимание уделяется разработкам 
интеллектуальных систем защиты информации на основе нейронных сетей для 
решения задач, связанных с~обнаружением атак, и~механизмов искусственных 
иммунных структур, которые успешно решают задачу противодействия 
выявленным угрозам на этапе проникновения их в~системы обработки 
данных~[1, 2]. Статей по разработке эффективных методов и~механизмов 
противодействия невыявленным угрозам почти нет~\cite{2-sim}.
  
  Можно перечислить ряд методов, позволяющих в~определенных 
ограниченных рамках выявлять и/или нейтрализовать невыявленные 
злоумышленные атаки~\cite{2-sim}. В~част\-ности, периодическое полное 
обновление программного обеспечения, периодический мониторинг систем 
обработки данных с~целью выявления ка\-ких-ли\-бо нетиповых отклонений, 
использование интеллектуальных методов выявления атак. В~данной работе 
рассматривается процедура мониторинга систем обработки данных с~\mbox{целью} 
выявления возможных следов злоумышленных атак.
  
  
  Для формализованного изучения процесса мониторинга как составной части 
системы ин\-фор\-мационной безопас\-ности в~работе предлагается использо\-вать 
аппарат теории систем массового обслуживания (СМО)~[3, 4]. Использованы 
методы анализа соответствующих СМО, аналогичные приведенным в~[5].

\vspace*{-6pt}
  
  \section{Описание модели системы массового обслуживания 
с~мониторингом} 
  
  Рассматривается СМО, в~которую поступает простейший поток вызовов 
с~интенсивностью~$a$. Поступающий вызов направляется в~очередь. Процесс 
обслуживания состоит из двух этапов. На первом этапе проводится мониторинг 
системы с~целью\linebreak выявления хотя бы одного вызова, который не\-об\-ходимо 
обслужить. Будем различать два вида мониторинга. Первый выполняется перед 
каж\-дым обслуживанием: он заключается в~проверке всех\linebreak стандартных 
атрибутов с~целью выявления потребности в~обслуживании (например, 
противодействии атаке). Решение о~том, какой из вызовов обслуживается 
следующим, принимается, исходя из результатов мониторинга. Назовем этот 
мониторинг мониторингом очереди. По\-треб\-ность в~обслуживании вызова, 
находящегося в~сис\-те\-ме, при мониторинге очереди возникает с~ве\-ро\-ят\-ностью 
$\alpha\hm >0$. Обслуженный вызов покидает сис\-те\-му. Функции 
распределения (ФР) времени мониторинга и~времени обслуживания 
рав\-ны~$B_1(t)$ и~$B_2(t)$ соответственно. В~случае если вызов не был 
выбран для обслуживания в~процессе мониторинга, он остается в~очереди. 
Предположим, что после цикла мониторинга не был выбран ни один вызов для 
обслуживания. Тогда обслуживающий прибор начинает выполнять общий 
мониторинг, предполагающий более системный, общий и~глубокий анализ 
и~контроль состояния системы. Этот мониторинг назовем общим 
мониторингом. Он выполняется периодически и~непрерывно, после окончания 
одного цикла мониторинга сразу же начинается другой. Если в~процессе 
данного мониторинга возникает потребность в~обслуживании, то мониторинг 
сразу же прерывается и~начинается обслуживание вызова. Вероятность 
обнаружения потребности в~обслуживании (например, скрытой или явной атаки 
в~системах безопасности) равна~$\gamma$, а ФР длительности одного цикла 
мониторинга равна~$B_3(t)$. Пусть $B_1(+0)\hm=0$, $B_3(+0)\hm=0$, т.\,е.\ 
мгновенный мониторинг исключается.

\vspace*{-6pt}
  
  \section{Основной результат}
  
  Ниже используется следующая лемма. % (СВ~--- случайная величина).
  
  \smallskip
  
  \noindent
  \textbf{Лемма.} \textit{Пусть случайные величины (СВ) $\{\xi_i, i\hm= 
\overline{-n,M}\}$ ($n\hm\geq 0$) независимы и~равномерно распределены на 
промежутке $[0,A]$, где $M$~--- СВ, имеющая пуассоновское распределение 
с~параметром~$\lambda$. Каждая СВ реализуется с~вероятностью~$\gamma$ 
и~<<окрашивается>> в~красный цвет с~вероятностью~$z_1$ для 
$i\hm\leq 0$ и~вероятностью~$z_2$ 
для $i\hm\geq 1$; $\zeta\hm=\min \xi_i$, где минимум 
берется по всем реализованным СВ. Тогда вероятность $\Phi(z,x,A)\hm= 
\Phi(z,x)$ события <<все имеющиеся СВ окрашены в~красный 
цвет и~$\zeta\hm< x$>> равна}:
  \begin{multline*}
  \Phi(z,x)={}\\
  {}=
  \begin{cases}
  0 & \hspace*{-20mm}\mbox{при } x<0\,;\\
  z_1^n e^{-\lambda A(1-z_2)} \left( 1-\left( 1-\gamma\fr{x}{A}\right)^n e^{-\lambda 
z_2 \gamma x}\right) &\\
&\hspace*{-20mm} \mbox{при } x\in [0,A)\,;\\
  z_1^n &  \hspace*{-20mm}\mbox{при } x\geq A\,.
  \end{cases}
  \end{multline*}
  
  \noindent
  Д\,о\,к\,а\,з\,а\,т\,е\,л\,ь\,с\,т\,в\,о\,.\ \ Введем 
СВ~$\eta_i$ ($i\hm\geq -n$):
  $$
  \eta_i= \begin{cases}
  \xi_i & \mbox{с~вероятностью } \gamma\,;\\
  +\infty & \mbox{с~вероятностью } 1-\gamma\,.
  \end{cases}
  $$
  
  Тогда $\zeta=\min \{ \eta_i \vert -n\leq i\leq M\}$. Пусть~$B(N)$ есть событие 
<<СВ $\{\xi_i,\ i\hm= \overline{-n,N}\}$ окрашены в~красный 
цвет>>. Для любого фиксированного $N\hm\geq 0$ и~любого $x\hm\in [0,A]$ 
имеем
  \begin{multline*}
  {\sf P}\left\{ B(N),\min\left( \eta_i\vert i\in [-n;N]\right)<x\right\} ={}\\
  {}={\sf P}\{B(N)\}- {\sf P}\left\{ B(N),\min\left( \eta_i\vert i\in [-n;N]\right) \geq 
x\right\}={}\\
  {}=z_1^n z_2^N \left(1-\prod\limits^N_{i=-n} P\left( \eta_i\geq 
x\right)\right)={}\\
  {}=z_1^n z_2^N\left( 1-\prod^N_{i=-n} \left( 1-\gamma+\gamma\left( 1-
\fr{x}{A}\right)^{N+n}\right)\right)={}\\
  {}= z_1^n z_2^N -z_1^n z_2^N \left(1-\gamma\fr{x}{A}\right)^{N+n}\,,
  \end{multline*}
откуда
\begin{multline*}
\Phi(z,x) =\sum\limits_{N\geq 0} \! {\sf P}(M=N) \times{}\\
{}\times {\sf P}\left\{ B(N),\min \left( \eta_i\vert -
n\leq i\leq N\right) <x\right\}={}\\
{}=
\sum\limits^\infty_{N=0} \fr{(\lambda A)^N}{N!}\,e^{-\lambda A} z_1^n 
z_2^N\left[ 1-\left( 1-\gamma\fr{x}{A}\right)^{N+n}\right]={}\\
{}=
z_1^n e^{-\lambda A(1-z_2)}\left( 1-\left( 1-\fr{\gamma x}{A}\right)^n e^{1-
\gamma x/A}\right)\,,
\end{multline*}
что влечет утверждение леммы. 

\smallskip

  На основе леммы доказывается следующая теорема.
  
  \smallskip
  
  \noindent
  \textbf{Теорема.}\ \textit{Справедливо соотношение}
  \begin{multline*}
  \Phi^\prime_{x} (z,x)={}\\
  {}= \begin{cases}
  0\,, &\!\mbox{если } x\notin (0,A)\,;\\
  z_1^n e^{-\lambda A(1-z_2)-\lambda z_2\gamma x}\overline{\Phi}(z,x)\,, & 
\!\mbox{если } x\in (0,A)\,,
  \end{cases}
  \end{multline*}
\textit{где} 
$$
\overline{\Phi}(z,x)=\left( \fr{n}{A}\left( 1- \gamma\fr{x}{A}\right)^{n-1} +\left( 
1-\gamma\fr{x}{A}\right)^n \lambda z_2 \gamma\right)\,.
$$

\smallskip
  
  Введем следующие обозначения. Пусть $q(m,n,t)$ ($n\hm\geq 0$; $0\hm\leq 
m\hm\leq n$; $t\hm\geq0$) есть вероятность того, что в~очереди в~момент~$t$ 
находится~$n$~вызовов, из которых~$m$ поступили в~сис\-те\-му во время 
обслуживания других вызовов; $q(z,w,t)\hm= \sum\nolimits_{n\geq0} 
\sum\nolimits^n_{m=0} q(m,n,t) z^m w^n$ ($0\hm\leq z\hm\leq 1$); 
$\beta_i(s)\hm= \int\nolimits_0^\infty e^{-s t}\,dB_i(t)$~--- преобразование 
Лап\-ла\-са--Стилть\-еса (ПЛС) ФР $B_i(t)$ ($i\hm= \overline{1,3}$); 
$\beta(s)\hm= \beta_1(s)\beta_2(s)$; $p(m,n,t)$~--- вероятность следующего 
события: в~момент~$t$ заканчивается обслуживание вызова, в~системе 
имеется~$n$~вызовов, из которых~$m$ пришли при обслуживании других 
вызовов, и~нет выявленных для обслуживания вызовов ($m\hm\leq n$). 
  
  Выведем соотношение для $q(z,w,t)$. Заметим, что функции $q(z,w,t)$ можно 
дать следующую вероятностную интерпретацию. Предположим, что каждый 
вызов, поступивший во время обслуживания другого вызова, 
с~вероятностью~$w$ окрашивается в~розовый цвет, а~с~вероятностью $(1\hm-
w)$ не окрашивается; кроме того, каждый вызов, поступивший во время 
обслуживания другого вызова, окрашивается с~вероятностью~$z$ в~красный 
цвет, а~с~вероятностью $(1\hm- z)$~--- в~синий. Тогда $q(z,w,t)$ есть 
вероятность того, что в~очереди в~момент~$t$ все вызовы окрашены в~красный 
цвет (если очередь не пуста) и~нет синих вызовов, пришедших в~систему во 
время обслуживания других вызовов. Потоки как красных, так и~синих вызовов 
получаются из поступающего простейшего потока на основе процедуры его 
просеивания с~вероятностями~$w$ и~$(1\hm- w)$ соответственно 
и,~следовательно, также являются пуассоновскими с~параметрами~$a w$ 
и~$a(1\hm-w)$. Аналогичные утверждения справедливы для потоков красных 
и~неокрашенных вызовов, а~также для комбинаций цветов.
  
  Назовем вызов плохим, если за время его обслуживания пришли синие или 
неокрашенные вызовы. Вероятность того, что данный вызов не является ни 
синим, ни окрашенным, равна $1\hm-zw$. Поэтому вероятность того, что 
данный вызов является хорошим (т.\,е.\ за время его обслуживания не пришло 
ни одного синего или неокрашенного вызова), равна $\beta(a\hm- azw) 
\stackrel{\mathrm{def}}{=} \tau$.
  
  Составим соотношения для потока хороших вызовов, поступивших в~систему 
за время от~0 до~$t$. Поток плохих вызовов является просеянным 
пуассоновским потоком с~вероятностью просеивания $1\hm-\tau$. 
Следовательно, вероятность того, что за время~$t$ в~систему не поступило ни 
одного плохого вызова, равна $\exp (-a(1\hm- \tau)t)$. Отметим, что при этом 
синие вызовы могли поступить в~систему в~промежутках, когда система была 
свободна от обслуживания и~занята общим мониторингом; это вызовы, 
с~которых начинается период занятости, и~вызовы, которые оказались 
необнаруженными обслуживающим прибором (с~вероятностью $1\hm-\alpha$ 
или $1\hm-\gamma$ в~зависимости от этапа работы обслуживающего 
устройства). Описанное событие имеет место в~следующих случаях.\\[-14pt]
  \begin{enumerate}
  \item За время~$t$ вообще не было синих и~неокрашенных вызовов 
(вероятность равна $e^{-a(1-zw)t}$) и~в~очереди в~момент~$t$ нет синих 
и~неокрашенных вызовов (вероятность равна $q(z,w,t)$). Таким образом, 
вероятность указанного случая равна $e^{-a(1-zw)t} q(z,w,t)$.\\[-14pt]
  \item  Первый синий вызов поступил в~систему, когда она не обслуживала 
вызовов, а занималась общим мониторингом, а~именно: в~некоторый момент 
времени~$u$ система оказалась в~состоянии, описываемом вероятностью 
$p(n,m,u)$; при этом вызовы, пришедшие в~систему во время обслуживания 
других вызовов, были красными (вероятность~$z^m$). Просмотрев все вызовы, 
обслуживающий прибор не обнаружил ни одного вызова, нуждающегося 
в~обслуживании (вероятность $(1\hm-\alpha)^n$), и~длительность промежутка 
поиска вызовов для обслуживания (т.\,е.\ длительность мониторинга очереди) 
равна~$v_0$ (вероятность $dB_1(v_0)$); за этот промежуток не пришли 
неокрашенные и~плохие вызовы, а~также вызовы, которые будут выявлены до 
конца $N$-го этапа общего мониторинга (т.\,е.\ при\linebreak $(N\hm+1)$-й попытке 
выявления, включая\linebreak
 данный этап). Процедуру выявления можно рас\-смат\-ри\-вать 
как процедуру просеивания с~ве\-ро\-ят\-ностью просеивания~$\alpha$ на данном 
этапе и~вероятностью~$\gamma$ на последующих этапах. \mbox{Вероятность} того, 
что вызов не будет выявлен на этапах с~данного по $N$-й, 
равна~$\overline{\alpha}\,\overline{\gamma}^N$, где $\overline{\alpha}\hm=1\hm-
\alpha$, $\overline{\gamma}\hm= 1\hm-\gamma$. Поэтому вероятность того, что 
хотя бы один из уже просеянных по критериям окраски (с вероятностью $1\hm- 
\tau w$) вызовов будет выявлен, равна $(1\hm-\tau w) (1\hm- \overline{\alpha}\, 
\overline{\gamma}^N)$. А~вероятность того, что за время~$v_0$ подобных 
вызовов не придет, равна $e^{-a(1-\tau w)(1-
\overline{\alpha}\,\overline{\gamma}^N)v_0}$. Далее начался цикл из~$N$~этапов 
общей профилактики ($N\hm\geq0$), длительность $i$-го этапа равна~$v_i$ 
(вероятность $dB_3(v_i)$) ($1\hm\leq i\hm\leq N$). Ни один из первоначально 
имевшихся~$n$~вызовов за все~$N$~этапов не был выявлен с~целью 
обслуживания (вероятность $(1\hm- \gamma)^{nN}$); не был выявлен также 
и~ни один из вызовов, пришедших в~систему во время общего мониторинга. 
Так как при общем мониторинге процесс выявления вызовов с~\mbox{целью} 
обслуживания можно рассматривать как процедуру просеивания 
с~вероятностью~$\gamma$, то поток выявленных вызовов, пришедших на \mbox{$i$-м} 
этапе, во время обслуживания которых не было синих и~неокрашенных 
вызовов, является просеянным пуассоновским с~вероятностью просеивания 
$\gamma^{N-i+1} \tau w$, а~вероятность того, что ни один из пришедших на 
$i$-м этапе вызовов не будет выявлен до конца $N$-го этапа и~является 
хорошим, равна 
 $e^{-a(1-\gamma^{N-i+1}\tau w)v_i}$. В~силу свойств пуассоновского потока 
промежутки вре\-ме\-ни между последовательными моментами выявления, 
а~также остаточные времена до выявления вызовов имеют показательное 
распределение с~параметром $a\gamma^{N-i+1}\tau w$. Наконец, на 
($N\hm+1$)-м этапе, длительность которого равна~$v_{N+1}$ (вероятность 
$dB_3(v_{N+1})$), один из пришедших вызовов был выявлен через время~$v$ 
($v\hm\leq v_{N+1}$) после начала этого этапа, и~ни один из пришедших за это 
время вызовов не был неокрашенным или плохим; вероятность этого события, 
в~силу леммы~1, равна $\Phi^\prime_v(\tau w, v, v_{N+1})\,dv$. А~затем за 
оставшийся промежуток длиной ($t\hm-u\hm-\sum\nolimits^N_{i=0} v_i\hm- v$) 
не пришли вызовы, за время обслуживания которых поступили в~систему синие и~неокрашенные вызовы (вероятность $e^{-a(1-\tau w) (t -u-
\sum\nolimits^N_{i=0} v_i-v)}$).\\[-13pt]
\end{enumerate}
  
  Просуммировав по всем значениям $n$, $m$, $N$, $u$, ${v}$, $v_i$ 
($0\hm\leq i \hm \leq N\hm+1$), получим следующее выражение для вероятности 
события, описываемого в~п.~2:

\vspace*{-4pt}

\noindent
  \begin{multline*}
  \sum\limits_{n>0} \sum\limits^n_{m=0} 
  \sum\limits_{N\geq0} \idotsint\limits_D 
\int\limits_{v=0}^{v_{N+1}} p(m,n,u) z^m w^n \times{}\\
{}\times (1-\alpha)^n (1-\gamma)^{nN} (1-\gamma)^{nN} e^{-a(1-\alpha\gamma^N \tau w)v_0} \times{}\\
  {}\times
\prod\limits^N_{i=1} e^{-a(1-\gamma^{N-i+1}\tau w)\sum\nolimits^N_{j=i} v_j} 
\tau^n e^{-av_{N+1} (1-\tau w)}\times{}\\[-2pt]
  {}\times  
d_v\left(1-\left( 1-\gamma\fr{v}{v_{N+1}}\right)^n e^{-a\tau w\gamma v}\right)\times{}\\[-2pt]
{}\times
e^{-a(1-\tau w) (t-u-\sum\nolimits^N_{i=0} v_i-v)}
du dB_1(v_0) \prod\limits_{i=1}^{N+1} dB_3(v_i)\,,
\end{multline*}

\vspace*{-6pt}

\noindent
где область интегрирования 

\vspace*{-3pt}

\noindent
\begin{multline*}
D= \{(u;v_0;v_1;\ldots ; v_{N+1}): u+ 
v_0+v_1+\cdots\\
\cdots +v_{N+1}< t,\ u\geq 0\,,\ v_i\geq 0\ (0\leq 
i\leq N+1)\}.
\end{multline*}

 Таким образом, получаем уравнение
\begin{multline*}
e^{-a(1-\tau)t} = e^{-a(1-zw)t} q(z,t)\times{}\\
{}\times\sum\limits_{n>0} \sum\limits_{m=0}^n 
\sum\limits_{N\geq0} \idotsint\limits_D \int\limits_{v=0}^{v_{N+1}} p(m,n,u)
 z^m w^n\times{}\\
 {}\times (1-\alpha)^n 
(1-\gamma)^{n N} e^{-a(1-\tau w)v_0} \times{}\\
{}\times
e^{-a\sum\nolimits^N_{i=1} (1-\gamma^{N-i+1}\tau)\sum\nolimits^N_{j=1} v_j e^{-
av_{N+1}(1-\tau w)}}\times\\
\times d_v\left( 1-\left( 1-\gamma\fr{v}{v_{N+1}}\right)^n 
e^{-a\tau w\gamma v}\right)\times{}\\
{}\times
e^{-a(1-\tau w)(t-u-\sum\nolimits^N_{i=0} v_i-v)} du 
dB_1(v_0)\prod\limits_{i=1}^{N+1} dB_3(v_i)\,.
\end{multline*}
 % 
  Отсюда, полагая 
  $$
  p(w,z,t)= \sum\limits_{n\geq0} \sum\limits^n_{m=0} 
p(m,n,t) z^m w^n,
$$
 выводим ($\theta\hm= v/v_{N+1}$):
  \begin{multline*}
  q(z,t)= e^{a[(1-zw)-(1-\tau)]t}\Bigg( 
  1+ {}\\
    {}+\sum\limits_{N\geq0} \idotsint\limits_D \!\!\int\limits_{v=0}^{v_{N+1}} 
    \hspace*{-2mm}\hspace*{-1.51018pt}
d_\theta\left(p \left( (1\!-\!\alpha)(1\!-\!\gamma)^N (1\!-\!\gamma\theta), z ,
u\right) \times\right.\\
\left.{}\times e^{-a\tau w 
\gamma v_{N+1}\theta}\right) 
  e^{-a(1-\tau w)v_0} e^{-av_{N+1}(1-\tau w)}\times{}\\
  {}\times e^{-a(1-\tau w)(t-u-
\sum\nolimits^N_{i=0} v_i-v)}\times{}\\
{}\times e^{-a\sum\nolimits^N_{j=1} v_j \left(j-\tau w ({\gamma^{N-j+1}-
\gamma^n})/({1-\gamma})\right)} du dB_1(v_0)\times{}\\
{}\times \prod\limits_{i=1}^{N+1} 
dB_3(v_i)\Bigg)\,.
  \end{multline*}
  
  При достаточно малых $z\hm>0$ выражение $\beta(a\hm- az)\hm-zw\hm>0$. 
Поэтому при $t\hm\to \infty$ величина $e^{a[\beta(a-az))-zw]t} \hm\to\infty$ при 
указанных значениях~$z$. Но так как величина~$q(z,t)$ при $t\hm\to\infty$ 
ограничена, то выражение в~правой части в~скобках должно стремиться к~нулю, и~после алгебраических преобразований получаем:
  \begin{multline}
 \hspace*{-2.56012pt} \sum\limits_{N\geq0} 
 \int\limits^\infty_{\theta=0}\hspace*{-1mm}
   d_\theta \!\left(\! \right.
  p\left( (1\!-\!\alpha) (1\!-\!\gamma)^N (1\!-\!\gamma\theta), z, 
  a\tau(1\!-\!w)\right)\times{}\\
{}\times
\beta_1\left(a(1-\tau w)\right)\times{}\\
{}\times \beta_3\left( a\left(1+\theta\right)+\left(\tau 
w(\gamma\theta-1)-\tau\theta\right) v_{N+1}\right) \times{}\\
{}\times
  \prod^N_{j=1}\beta_3 \left(a\left( 
  j-\fr{\gamma^{N-j+1}-\gamma^n}{1-\gamma}\tau w\right)-a(1-\tau w)\right)={}\\
  {}=-1\,.
  \label{e1-sim}
  \end{multline}
  
  Равенство~(1) требует дальнейшего анализа. Оно может быть использовано 
в~процессе компьютерного моделирования процесса мониторинга.
  
  Исследуем теперь характеристику, которая описывает число невыявленных 
атак. Пусть $q_n(z,\overline{Z})$ есть вероятность того, что после окончания 
\mbox{$n$-го} по порядку обслуживания вызовов в~системе нет\linebreak выявленных  
$z$-си\-них вызовов, ожидающих обслуживания, все вызовы, не выявленные 
за~$i$~попыток, являются $i$-крас\-ны\-ми ($i\hm\geq 1$); здесь 
$\overline{Z}\hm= (z_1, z_2, \ldots , z_i, \ldots)$. Запишем рекуррентное 
\mbox{соотношение}, связывающее $q_{n+1}(z,\overline{Z})$ с~$q_n(z,\overline{Z})$.
  
  В момент окончания $(n\hm+1)$-го по порядку обслуживания возможны 
следующие ситуации.
  \begin{enumerate}
  \item В момент окончания $n$-го по порядку обслуживания в~очереди 
имеются выявленные вызовы (ве\-роят\-ность $q_n(z,\overline{z})\hm- 
q_n(0,\overline{z})$), с~ве\-ро\-ят\-ностью~$\alpha_i$ каждый вызов, не выявленный 
в~предыду\-щих ($i\hm-1$)-й попытках, будет выявлен и~перейдет в~основную 
очередь, а~с~дополнительной вероятностью $\overline{\alpha}_i\hm= 1\hm- 
\alpha_i$ не будет выявлен. В~последнем случае он с~вероятностью~$z_i$ 
получит дополнительную окраску $i$-крас\-но\-го цвета и~перейдет в~очередь 
из вызовов, не выявленных при~$i$~попытках. Распишем указанное событие 
в~терминах исходных вероятностей: $\sum\nolimits_{N\geq1} 
\sum\nolimits_{N_i,\ i\geq1} Q_n(N; N_i,\ i\hm\geq1) z^N \prod\nolimits_{i\geq 1} 
Z_i^{N_i}$, где $Z_i\hm= \prod\nolimits^i_{j=1} z_j$. 
С~ве\-ро\-ят\-ностью~$\alpha_{i+1}$ каждый вызов переходит в~основную очередь 
и~окрашивается в~красный цвет с~ве\-ро\-ят\-ностью~$z$ (при этом старая окраска 
убирается (делится на~$Z_i$)~--- в~момент  
($n\hm+1$)-го окончания его прошлая окрас\-ка не представляет интереса), 
а~с~вероятностью $\overline{\alpha_{i+1}}$ вызов переходит  
в~($i\hm+1$)-очередь и~получает дополнительную  
($i\hm+1$)-окраску с~ве\-ро\-ят\-ностью~$z_{i+1}$. Таким образом, необходимо 
заменить~$Z_i$ на~$z$ с~ве\-ро\-ят\-ностью~$\alpha_{i+1}$ и~заменить~$Z_i$ 
на~$Z_{i+1}$ с~вероятностью~$\overline{\alpha_{i+1}}$, т.\,е.~$Z_i$ 
заменяется на $\alpha_{i+1} z\hm+ (1\hm- \alpha_{i+1})Z_{i+1}$. Далее 
начинается обслуживание одного из основных вызовов (по предположению, 
в~случае~1 эта очередь не пуста); при этом необходимо убрать окраску этого 
вызова (т.\,е.\ разделить на~$z$). За время обслуживания этого вызова не 
должно поступить ни одного синего вызова; при этом каж\-дый поступающий 
вызов с~ве\-ро\-ят\-ностью~$\alpha_1$ становится выявленным и~ставится 
в~основную очередь, а~с~ве\-ро\-ят\-ностью~$\overline{\alpha_1}\hm=1-\alpha_1$ не 
выявляется и~ставится в~1-оче\-редь. Это равносильно тому, что по\-сту\-па\-ющий 
поток просеивается на три потока: выявленных 0-крас\-ных вызовов (параметр 
потока~$\alpha_1z$), не выявленных 1-крас\-ных вызовов параметр потока 
$(1\hm-\alpha_1)z_1$ и~на поток всех остальных вызовов, и~требуется, чтобы за 
время обслуживания не поступали вызовы третьего потока. Ве\-ро\-ят\-ность этого 
события равна $\beta(a\hm- a(\alpha_1 z\hm+ (1\hm- \alpha_1)z_1))$. Таким 
образом, получаем, что\linebreak вероятность, описываемая в~случае~1, имеет вид: 
$z^{-1} \left( q_n\left(z, Az+(1-A) *\overline{Z}_{+1}\right) -\right.$\linebreak
$\left.-q_n\left( 0, Az+(1\!\hm-\!A)* \overline{Z}_{+1}\right)\right)
 \beta\left( a\hm\!-\!a\left( \alpha_1z+(1-\right.\right.$\linebreak
 $\left.\left.-\alpha_1z_1\right)
 \right)$.
  Здесь <<*>> означает покомпонентное умножение векторов, 
а~индекс~<<$+1$>> указывает на увеличение на~1 индексов всех компонентов 
вектора.
  \item  В~момент окончания $n$-го по порядку обслуживания в~очереди нет 
выявленных вызовов (вероятность $q_n(0,\overline{Z})$). Затем началось 
проведение общего мониторинга системы, и~в течение~$N$~циклов ($N\hm\geq 
1$) не было выявлено ни одного вызова из имевшихся (вероятность 
$q_n(0,\{\prod\nolimits^N_{j=1} (1\hm-A_j)\}_{j\geq1} * \overline{Z}_{+N})$), 
ни одного из вызовов, пришедших на $i$-м цикле мониторинга (вероятность 
$\beta_3(a\hm- a\prod\nolimits^N_{j=i} (1\hm- \alpha_j)z_i)$ для всех $i\hm= 
\overline{1,N}$. Наконец, на ($N\hm+1$)-м цикле один из вызовов был выявлен. 
Так как поток красных вызовов, не выявленных на этапах с~$i$-го по $N$-й, 
может быть получен на основе процедуры просеивания с~вероятностью 
$\prod\nolimits^N_{j=i} (1\hm- \alpha_j)z_i$, то вероятность того, что за 
время~$t$ не будет выявлен ни один из вызовов, пришедших в~сис\-те\-му на $i$-м 
цикле, равна

\vspace*{-10pt}

\noindent
\begin{multline*}
  \hspace*{-1pt}q_n\left(0,\overline{Z}\right) =q_n\left(0, \left\{ \prod\limits^N_{j=1} \left(1-
A_j\right)\right\}_{j\geq1}\hspace*{-9pt} * \overline{Z}_{+N}\right) \times{}\\
{}\times\beta_3 \left( 
a-a\prod\limits^N_{j=i} \left(1-\alpha_j\right) z_i\right)\,.
\end{multline*}
  \end{enumerate}
  
  \vspace*{-6pt}
  
  Пусть $\overline{M_0}$~--- вектор числа невыявленных вызовов, прошедших 
различные стадии выявления, в~момент $n$-го окончания обслуживания (когда 
нет выявленных вызовов); $M_j$~--- число вызовов, поступивших во время  
$j$-го цикла общего мониторинга (и~они не были выявлены) ($1\hm\leq j\hm\leq 
N$); $M_{N+1}$~--- число вызовов, поступивших во время $(N\hm+1)$-го 
цикла общего мониторинга до момента~$\tau$, когда был выявлен первый из 
всех перечисленных вызовов (т.\,е.\ до момента~$\tau$ они не были выявлены). 
Тогда вероятность того, что хотя бы один из этих вызовов будет выявлен на 
$(N\hm+1)$-м цикле и~не будет выявлен на предыдущих, а~все вызовы~--- 
соответствующего типа красности, равна 

\vspace*{-2pt}

\noindent
  \begin{multline*}
  \sum\limits_{l\geq1} \alpha_{l+N+1}\times{}\\
  {}\times \prod\limits^N_{k=0} \left(1-
\alpha_{l+k}\right)^{M_{0,l}}
M_{0,l}\left( 1-\alpha_{l+N+1}\right)^{M_{0,l}-1}\times{}
\end{multline*}

\noindent
 \begin{multline*}
{}\times  
z_{l+N+1}^{M_{0,l}-1} \prod\limits_{j=1}^{N+1}
  \left[(1-\alpha_{N-j+1})^{M_j} z^{M_j}_{N-j+1}\right] \times{}\\
  {}\times \prod\limits_{j\geq1, j\not=l} \left\{
  \prod\limits_{k=0}^{N+1}\left( 1-\alpha_{j+k}\right) 
z_{j+N+1}\right\}^{M_{0,j}}+{}\\
{}+
 \sum\limits_{l=1}^{N+2} \alpha_l M_l\left( 1-\alpha_{l+N+1}\right)^{M_{l}-1}
  z^{M_l-1}_{l+N+1}\times{}\\
 {}\times
  \prod\limits_{j\geq1} \left(1-\alpha_{N-j+1}\right)^{M_{0,j}} 
  z^{M_j}_{n-j+1}\times{}\\
  \times 
 \prod\limits_{j=1,i\not=l}^{N+2} 
  \left( 1-\alpha_{N-j+1}\right)^{M_j} 
z^{M_j}_{N-j+1}\,.
  \end{multline*}

  Считаем, что момент~$\tau$ выявления вызова является 
СВ, равномерно распределенной на интервале~$\tau_{N+1}$ 
длительности $(N\hm+1)$-го цикла общего мониторинга.
  
  Заметим, что распределение вектора~$M_0$ задается производящей 
функцией $q_n(0,\overline{Z})$; $M_j$ распределено по закону Пуассона 
с~параметром~$a$ на промежутке~$\tau_j$ длительности $j$-го цикла. 
Просуммировав последнее выражение по всем возможным состояниям 
векторов~$M_0$ и~$M_j$ ($1\hm\leq j\hm\leq N$), получаем:
  \begin{multline*}
  \sum\limits_{l\geq1} \alpha_{l+N+1} \prod\limits^N_{k=0} \left( 1-
\alpha_{l+k}\right)\times{}\\
{}\times q^\prime_{n,l}\left( 0, 
\left\{ \prod\limits_{k=0}^{N+1} \left( 1-
\alpha_{l+k}\right) z_{l+N+1}\right\}_{l\geq1}\right)\times{}\\
{}\times b_3\left( a-a(\left( 1-
\alpha_1\right)z_1\right)\times{}\\
  {}\times
  \prod\limits^N_{j=1} \beta_3\left( a- a\prod\limits_{k=1}^{N-j+2}\left( 1-
\alpha_k\right) z_{N-j+1}\right) +{}\\
{}+\sum\limits_{l=1}^{N+1} \alpha_{N-l+2} 
\beta_{3}^\prime \left( \prod\limits_{k=1}^{N-l+1} \left( 1-\alpha_k\right) z_{N-
j+1}\right)\times{}\\
  {}\times
\prod\limits_{k=1}^{N-l+1}\hspace*{-2mm}
\left( 1-\alpha_k\right) q_n\!\left(\! 0,\left\{ 
\prod\limits_{k=0}^{N+1} \left( 1-\alpha_{l+k}\right) z_{l+N+1}\right\}\!\right) \times{}\\
{}\times
b_3\left( a-a\left( 1-\alpha_1\right)z_1\right) \times{}\\
{}\times
\prod\limits^N_{j=1,i\not=l} \beta_3 \left(a-a\prod\limits_{k=1}^{N-j+2} \left(1-
\alpha_k\right)z_{N-j+1}\right)\times{}\\
{}\times \prod\limits^N_{j=1,j\not=l} \beta_3 \left(
a-a\prod\limits_{k=1}^{N-j+2} \left(1-\alpha_k\right) z_{N-j+1}\right)+{}\\
{}+\alpha_1\prod\limits^N_{j=1} \beta_3\left( \prod\limits_{k=1}^{N-j+2} \left(
1-\alpha_k\right) z_{N-j+1}\right) \times{}\\
{}\times b_3^\prime \left(a-a\left(1-\alpha_1\right)z_1\right)\,,
\end{multline*}
где 

\noindent
\begin{align*}
q_{n,l^\prime} \left(z,\overline{Z}\right)&= \fr{\partial}{\partial z_l}\,
q_n\left(z,\overline{Z}\right)\,;
\\
b_3(s)&= \int\limits_0^\infty \fr{1-e^{-s\tau_{N+1}}}{s\tau_{N+1}}\,dB_3 
\left(\tau_{N+1}\right)\,.
\end{align*}
  
  Дополнительно надо умножить приведенную вероятность на 
вероятность~$z$ того, что выявленный вызов~--- красный, и~на вероятность 
$\beta(a\hm- a(\alpha_1z\hm+ (1\hm- \alpha_1)z_1))$ того, что за время его 
обслуживания не поступят в~систему синие вызовы и~невыявленные 1-си\-ние 
вызовы. Предполагается для простоты, что по окончании обслуживания 
очередной мониторинг не производится.
  
  При этом возможны следующие три схемы обработки выявленного вызова.
  \begin{enumerate}[1.]
  \item При выявлении вызова мониторинг немедленно прекращается 
и~начинается обслуживание выявленного вызова.
  \item При выявлении вызова мониторинг доводится до конца и~только затем 
начинается обслуживание выявленных вызовов.
  \item При выявлении вызова на первом цикле мониторинга прерывание 
обслуживания не происходит; на втором и~последующих циклах мониторинга 
происходит его прерывание и~начинается обслуживание вызова.
  \end{enumerate}
  
  В статье рассматривается только первая схема. Остальные схемы 
предполагается рассмотреть в~после\-ду\-ющих работах авторов. Тогда надо 
потребовать, чтобы за время начавшегося обслуживания красного вызова не 
пришли выявленные синие вызовы и~1-крас\-ные вызовы, не выявленные по 
окончании обслуживания этого вызова (вероятность $\beta(a\hm- 
a(\alpha_1z\hm+(1\hm- \alpha_1)z_1))$). Получаем:
  \begin{multline*}
  q_{n+1} \left(z,\overline{Z}\right) =
  \beta\left(a-a\left( \alpha_1 z+\left( 1-
\alpha_1\right) z_1\right)\right)\times{}\\
{}\times
  \Bigg\{ z^{-1} \left( q_n \left(z, Az+(1-A)*\overline{Z}_{+1}\right)\right.-{}\\
  \left.{}-
  q_n\left(0,Az+(1-A)*\overline{Z}_{+1}\right)\right)+{}\\
  {}+\Bigg[
  \sum\limits_{l\geq1} \alpha_{l+N+1}\prod\limits^N_{k=0} \left(1-
\alpha_{l+k}\right)\times{} \\
  {}\times
  q^\prime_{n,l} \left(0, \left\{ \prod\limits_{k=0}^{N+1} \left(1-
\alpha_{l+k}\right) z_{l+N+1}\right\}_{l\geq1}\right) \times{}\\
{}\times \prod\limits^N_{j=1} \beta_3 
\left( a-a\prod\limits_{k=1}^{N-j+2}\left( 1-\alpha_k\right) z_{N-j+1}\right)\times{}
  \end{multline*}
  
  \noindent
    \begin{multline*}
    {}\times
  b_3\left(a-a\left(1-\alpha_1\right)z_1\right) +\sum\limits^{N+1}_{l=1} 
  \alpha_{N-l+2} \times{}\\
  {}\times
  \prod\limits_{k=1}^{N-l+1}\left(1-\alpha_k\right) \beta_3^\prime \left( 
\prod\limits_{k=1}^{N-l+1}\left(1-\alpha_k\right) z_{N-j+1}\right)\times{}\\
  {}\times
q_n\left(0,\left\{ \prod\limits_{k=0}^{N+1} \left(1-\alpha_{l+k}\right) 
z_{l+N+1}\right\}\right)\times{}\\
{}\times
\prod\limits^N_{j=1,j\not=l} \beta_3 \left( a-
a\prod\limits_{k=1}^{N-j+2}\left( 1-\alpha_k\right) z_{N-j+1}\right)\times{}\\
{}\times
b_3\left( a-a\left(1-\alpha_1\right)z_1\right) +{}\\
{}+\alpha_1\prod\limits^N_{j=1} \beta_3 \left( 
\prod\limits_{k=1}^{N-j+2}\left( 1-\alpha_k\right) z_{N-j+1}\right)\times{}\\
{}\times b_3^\prime
\left( a-a\left( 1-\alpha_1\right)z_1\right)\Bigg]\Bigg\}\,.
\end{multline*}
  %
  Помножим данное соотношение на~$w^n$ и~просуммируем по $n\hm\geq1$. 
Получим:
  \begin{multline*}
  \fr{q(w,z,\overline{Z})-q_0(z,\overline{Z})}{w}={}\\
{}=\beta\left(
  a-a\left( \alpha_1 z+\left( 1-\alpha_1\right) z_1\right)\right)\times{}\\
  {}\times
  \Bigg\{
  \fr{q(w,z,Az+(1-A)*\overline{Z}_{+1})}{z}-{}\\
  {}-\fr{q(w,z,Az+(1-A)*\overline{Z}_{+1})}{z}+{}\\
  {}+\Bigg[
  \sum\limits_{l\geq0} q_l^\prime \Bigg( w,0,\left\{
  \prod\limits_{j=0}^{N+1} \left(1-\alpha_{l+j}\right)
   z_{l+N+1}\right\}_{l\geq1} \times{}\\
   {}\times \alpha_{l+N+1}\Bigg)
  \prod\limits^N_{k=0} \left(1-\alpha_{l+k}\right) 
  \times{}\\
  {}\times
  \prod\limits^N_{j=1} \beta_3 
\left( a-a\prod\limits_{k=1}^{N-j+2}\left(1-\alpha_k\right)
  z_{N-j+1}\right)\times{}\\
  {}\times b_3\left( a-a\left( 1-\alpha_1\right) z_1\right)+{}\\
  {}+q\left( w,0,\left\{ \prod\limits_{j=0}^{N+1} \left(1-\alpha_{l+j}\right) 
z_{l+N+1}\right\} \right)\times{}\\
{}\times
  \left( \sum\limits_{l=1}^{N+1} \alpha_{N-l+2} \beta_3^\prime \left(
  \prod\limits_{k=1}^{N-l+1}\left(1-\alpha_k\right) z_{N-
j+1}\right)\times{}\right.\\
\left.{}\times\prod\limits_{k=1}^{N-l+1}\left(1-\alpha_k\right)\right)
 b_3\left(a-a\left( 1-\alpha_1\right)z_1\right) \times{}\\
 {}\times \prod\limits^N_{j=1,j\not=l} 
\beta_3 \left(a-a\prod\limits_{k=1}^{N-j+2}\left(1-\alpha_k\right) 
z_{N-j+1}\right)+{}
\end{multline*}

\noindent
\begin{multline}
{}+
  \alpha_1 \prod\limits^N_{j=1} \beta_3 \left( \prod\limits_{k=1}^{N-j+2}\left( 1-
\alpha_k\right) z_{N-j+1}\right) \times{}\\
{}\times b_3^\prime \left(a-a\left(1-
\alpha_1\right)z_1\right)
  \vphantom{\prod^N_{l=1}}
  \Bigg]\Bigg\}\,,
  \label{e2-sim}
  \end{multline}
где $q_0(z,\overline{Z})$ есть производящая функция числа вызовов разных 
типов в~системе в~начальный момент времени. В~частности, если вызовов 
в~начальный момент нет, то $q_0(z,\overline{Z})\hm=1$. Заметим, что 
$q_0(z,\overline{Z})\hm=q(0,z,\overline{Z})$.
  
  Соотношение~(2) является достаточно сложным  
ин\-тег\-ро\-диф\-фе\-рен\-ци\-аль\-ным уравнением для на\-хож\-де\-ния функции 
$q(w,z,\overline{Z})$, описывающей распределение во времени числа 
невыявленных атак. В~настоящее время нет эффективных методов решения 
указанного уравнения. Однако соотношение~(2) позволяет разработать 
рекуррентные вы\-чис\-ли\-тель\-ные процедуры для нахождения 
$q(w,z,\overline{Z})$, которые предполагается привести в~последующих 
работах.
  
  \section{Заключение}
  
  В работе рассмотрена задача построения модели мониторинга систем 
обработки данных по показателям информационной безопас\-ности на основе 
использования методов теории массового обслуживания. Получены 
соотношения для следующих двух наиболее важных характеристик указанных 
систем: вероятности состояний сис\-те\-мы и~вероятности чис\-ла 
невыявленных вызовов в~моменты окончания обслуживания. Здесь под 
состоянием понимается чис\-ло вызовов, ожидающих обслуживания (т.\,е.\ атак, 
ожидающих нейтрализации), включая невыявленные атаки. Нахождение 
указанных характеристик позволит более эффективно организовать процесс 
выявления злоумышленных атак на сис\-те\-му обработки данных. 
  
{\small\frenchspacing
 {%\baselineskip=10.8pt
 \addcontentsline{toc}{section}{References}
 \begin{thebibliography}{9}
 
  \bibitem{2-sim}
  \Au{Грушо А.\,А., Грушо~Н.\,А., Тимонина~Е.\,Е.} Методы защиты информации от атак 
с~помощью скрытых каналов и~враждебных про\-граммно-ап\-па\-рат\-ных агентов 
в~распределенных системах~// Вестник РГГУ. Сер.: Документоведение и~архивоведение. 
Информатика. Защита информации и~информационная безопасность, 2009. №\,10. С.~33--45.

 \bibitem{1-sim}
  \Au{Kopyrin A.\,S., Simavoryan~S.\,Zh., Simonyan~A.\,R., Ulitina~E.\,I.} The methodology of 
risk analysis in assessing information security threats~// Modeling Artificial Intelligence, 2017. 
No.\,4-2. P.~78--85. doi: 10.13187/mai.2017.2.78.

  
  \bibitem{4-sim} %3
  \Au{Бажаев Н.\,А., Давыдов~А.\,Е., Кривцова~И.\,Е., Лебедев~И.\,С., 
Салахутдинова~К.\,И.} Подход к~анализу состояния информационной безопасности 
беспроводной сети~// Прикладная информатика, 2016. Т.~11. №\,6(66). С.~121--128.

\bibitem{3-sim} %4
  \Au{Коляденко Ю.\,Ю., Лукинов~И.\,Г.} Модель распределенных атак 
  в~программно-конфигурируемых сетях связи~// Вестник ЮУрГУ. Сер.: Компьютерные технологии, 
управление, радиоэлектроника, 2017. Т.~17. №\,3. С.~34--43. doi: 10.14529/ctcr170304.

  \bibitem{5-sim}
  \Au{Гнеденко Б.\,В., Даниелян~Э.\,А., Димитров~Б.\,Н., Климов~Г.\,П., Матвеев~В.\,Ф.} 
Приоритетные системы массового обслуживания.~--- М.: МГУ, 1973. 448~с.
 \end{thebibliography}

 }
 }

\end{multicols}

\vspace*{-3pt}

\hfill{\small\textit{Поступила в~редакцию 03.08.19}}

\vspace*{8pt}

%\pagebreak

%\newpage

%\vspace*{-28pt}

\hrule

\vspace*{2pt}

\hrule

%\vspace*{-2pt}

\def\tit{MODELING OF~MONITORING OF~INFORMATION SECURITY 
PROCESS ON~THE~BASIS OF~QUEUING SYSTEMS}


\def\titkol{Modeling of~monitoring of~information security 
process on~the~basis of~queuing systems}

\def\aut{G.\,A.~Popov$^1$, S.\,Zh.~Simavoryan$^2$, A.\,R.~Simonyan$^2$, 
and~E.\,I.~Ulitina$^2$}

\def\autkol{G.\,A.~Popov, S.\,Zh.~Simavoryan, A.\,R.~Simonyan, 
and~E.\,I.~Ulitina}

\titel{\tit}{\aut}{\autkol}{\titkol}

\vspace*{-11pt}


\noindent
   $^1$Astrakhan State Technical University, 16~Tatischeva Str., Astrakhan 414056, 
Russian Federation
  
  \noindent
  $^2$Sochi State University, 94~Plastunskaya Str., Sochi 354003, Russian 
Federation

\def\leftfootline{\small{\textbf{\thepage}
\hfill INFORMATIKA I EE PRIMENENIYA~--- INFORMATICS AND
APPLICATIONS\ \ \ 2020\ \ \ volume~14\ \ \ issue\ 1}
}%
 \def\rightfootline{\small{INFORMATIKA I EE PRIMENENIYA~---
INFORMATICS AND APPLICATIONS\ \ \ 2020\ \ \ volume~14\ \ \ issue\ 1
\hfill \textbf{\thepage}}}

\vspace*{3pt} 
  
  
    
\Abste{The paper is devoted to the mathematical modeling of monitoring process 
by the information security systems, aimed at detection of hidden malicious 
attacks. The modeling is based on the queueing theory formalism. The monitoring 
process is reduced to the analysis of the customer flow arriving at the queueing 
system, in which each customer is regarded as carrying potential malicious 
attacks. Functional relations between the system state probability distribution and 
the distribution of the number of undetected malicious attacks on service 
completion epochs are obtained. These characteristics may allow one to improve 
the efficiency of malicious attacks detection process in the data processing 
systems.}

\KWE{protection of information; information security; queuing system; 
probability}
  
  


\DOI{10.14357/19922264200110} 

%\vspace*{-14pt}

\Ack
  \noindent
  The reported study was funded by the Russian Foundation for Basic Researh, 
project No.\,19-01-00383.

 


%\vspace*{6pt}

  \begin{multicols}{2}

\renewcommand{\bibname}{\protect\rmfamily References}
%\renewcommand{\bibname}{\large\protect\rm References}

{\small\frenchspacing
 {%\baselineskip=10.8pt
 \addcontentsline{toc}{section}{References}
 \begin{thebibliography}{9}
  
  \bibitem{2-sim-1}
  \Aue{Grusho, A.\,A., N.\,A.~Grusho, and E.\,E.~Timonina}. 2009. Metody 
zashchity informatsii ot atak s pomoshch'yu skrytykh kanalov i~vrazhdebnykh 
programmno-apparatnykh agentov v~raspredelennykh sistemakh [Methods of 
information protection aganst covert channels attacks and 
malicious software/hardware agents in distributed systems]. 
\textit{Vestnik RGGU. Ser. Dokumentovedenie i~arkhivovedenie. Informatika. 
Zashchita informatsii i~informatsionnaya bezopasnost'} [RGGU Bulletin. Informatics. Information 
security. Mathematician ser.] 10:33--45. 

\bibitem{1-sim-1}
  \Aue{Kopyrin, A.\,S., S.\,Zh.~Simavoryan, A.\,R.~Simonyan, and E.\,I.~Ulitina.} 
2017. The methodology of risk analysis in assessing information security threats. 
\textit{Modeling Artificial Intelligence} 4-2:78--85.

  
  \bibitem{4-sim-1} %3
  \Aue{Bazhayev, N.\,A., A.\,E.~Davydov, I.\,E.~Krivtsova, I.\,S.~Lebedev, and 
K.\,I.~Salakhutdinova.} 2016. Podkhod k~analizu so\-sto\-yaniya informatsionnoy 
bezopasnosti besprovodnoy seti [Wireless security information analysis approach]. 
\textit{Prikladnaya informatika} [J.~Applied Informatics] 11(6(66)):121--128.

\bibitem{3-sim-1} %4
  \Aue{Kolyadenko Yu.\,Yu., and I.\,G.~Lukinov}. 2017. Model' raspredelennykhatak 
v~programmno-konfiguriruemykh setyakh svyazi [Model of distributed attacks in 
program-configurable communication networks]. \textit{Vestnik YUUrGU. Ser. 
Komp'yuternye tekhnologii, upravleniye, radioelektronika} [Bulletin of SUSU. 
Computer technologies, automatic control, radioelectronics ser.] 17(3):34--43.

  \bibitem{5-sim-1}
  \Aue{Gnedenko, B.\,V., E.\,A.~Danielyan, B.\,N.~Dimitrov, G.\,P.~Klimov, and 
V.\,F.~Matveev.} 1973. \textit{Prioritetnyye sistemy massovogo obsluzhivaniya} 
[Priority queues]. Moscow: MSU. 448p.
\end{thebibliography}

 }
 }

\end{multicols}

%\vspace*{-7pt}

\hfill{\small\textit{Received August 3, 2019}}

%\pagebreak

%\vspace*{-22pt}
  
  \Contr
  
  \noindent
  \textbf{Popov Georgy A.} (b.\ 1950)~--- Doctor of Science in technology, 
professor, Head of Department, Astrakhan State Technical University, 16~Tatischeva 
Str., Astrakhan 414056, Russian Federation; \mbox{popov@astu.org}
  
  \vspace*{6pt}
  
  \noindent
  \textbf{Simavoryan Simon Zh.} (b.\ 1958)~--- Candidate of Science (PhD) in 
technology, associate professor, Sochi State University, 94~Plastunskaya Str., Sochi 
354003, Russian Federation; \mbox{simsim58@mail.ru}
  
  \vspace*{6pt}
  
  \noindent
  \textbf{Simonyan Arsen R.} (b.\ 1960)~--- Candidate of Science (PhD) in physics 
and mathematics, associate professor, Sochi State University, 94~Plastunskaya Str., 
Sochi 354003, Russian Federation; \mbox{oppm@mail.ru}
  
  \vspace*{6pt}
  
  \noindent
  \textbf{Ulitina Elena I.} (b.\ 1978)~--- Candidate of Science (PhD) in physics and 
mathematics, associate professor, Sochi State University, 94~Plastunskaya Str., 
Sochi 354003, Russian Federation; \mbox{ulitinaelena@mail.ru}
\label{end\stat}

\renewcommand{\bibname}{\protect\rm Литература}  %10
\def\stat{grusho}

\def\tit{АРХИТЕКТУРНЫЕ РЕШЕНИЯ В~ЗАДАЧЕ ВЫЯВЛЕНИЯ МОШЕННИЧЕСТВА ПРИ~АНАЛИЗЕ 
ИНФОРМАЦИОННЫХ ПОТОКОВ В~ЦИФРОВОЙ ЭКОНОМИКЕ$^*$}

\def\titkol{Архитектурные решения в~задаче выявления мошенничества при~анализе 
информационных потоков в
%~цифровой 
экономике}

\def\aut{А.\,А.~Грушо$^1$, М.\,И.~Забежайло$^2$, Н.\,А.~Грушо$^3$, 
Е.\,Е.~Тимонина$^4$}

\def\autkol{А.\,А.~Грушо, М.\,И.~Забежайло, Н.\,А.~Грушо, 
Е.\,Е.~Тимонина}

\titel{\tit}{\aut}{\autkol}{\titkol}

\index{Грушо А.\,А.}
\index{Забежайло М.\,И.}
\index{Грушо Н.\,А.}
\index{Тимонина Е.\,Е.}
\index{Grusho A.\,A.}
\index{Zabezhailo M.\,I.}
\index{Grusho N.\,A.}
\index{Timonina E.\,E.}


{\renewcommand{\thefootnote}{\fnsymbol{footnote}} \footnotetext[1]
{Работа частично поддержана РФФИ (проекты 18-29-03081 и~18-07-00274).}}


\renewcommand{\thefootnote}{\arabic{footnote}}
\footnotetext[1]{Институт проблем информатики Федерального исследовательского центра <<Информатика и~управление>> 
Российской академии наук, grusho@yandex.ru}
\footnotetext[2]{Институт проблем информатики Федерального исследовательского центра <<Информатика и~управление>> 
Российской академии наук, m.zabezhailo@yandex.ru}
\footnotetext[3]{Институт проблем информатики Федерального исследовательского центра <<Информатика и~управление>> 
Российской академии наук, info@itake.ru}
\footnotetext[4]{Институт проблем информатики Федерального исследовательского центра <<Информатика и~управление>> 
Российской академии наук, eltimon@yandex.ru}

\vspace*{-12pt}
   

 
  
  \Abst{Cформулирован подход к~исследованию некоторых видов мошенничества в~цифровой 
экономике с~использованием причинно-следственных связей. Во всех видах рассматриваемых 
мошенничеств должно наблюдаться несоответствие между целями финансовых транзакций 
и~реальной стоимостью достижения этих целей. Данные о транзакциях можно собирать, 
наблюдая информационные потоки, в~которых отражаются эти транзакции. Архитектура сбора 
данных и~их анализа может быть организована с~помощью распределенных реестров 
с~централизованным консенсусом, что позволяет создать аналог электронной бухгалтерской 
книги, фиксирующей финансово-экономическую деятельность субъектов цифровой экономики в~регионе. 
  Рассматриваемые методы выявления мошенничества основаны на противоречиях 
между действиями, описанными в~транзакциях, и~информацией, содержащейся в~планах, 
стандартах, прецедентах и~др. Рассмотрен метод, основанный на некоторой упрощенной схеме 
реализации абстрактного проекта. Для выявления противоречий необходимо проводить анализ 
от следствия к~причине, т.\,е.\ искать аномалии в~информации, описывающей порождение 
наблюдаемых следствий. 
  Показано, как в~реализации проекта можно выделять простые <<необходимые условия>> 
нарушения при\-чин\-но-след\-ст\-вен\-ных связей, т.\,е.\ множество <<необходимых условий>>, 
нарушение которых свидетельствует о наличии мошенничества. Это множество <<необходимых 
условий>> можно назвать метаданными для контроля проекта на выявление мошенничества.} 
 
 
  \KW{цифровая экономика; информационные потоки; при\-чин\-но-след\-ст\-вен\-ные связи; 
выявление мошеннических схем} 

\DOI{10.14357/19922264190204}
  
\vspace*{-4pt}


\vskip 10pt plus 9pt minus 6pt

\thispagestyle{headings}

\begin{multicols}{2}

\label{st\stat}

\section{Введение}

\vspace*{3pt}

  В работе сформулирован подход к~исследованию некоторых видов 
мошенничества в~цифровой экономике с~использованием  
при\-чин\-но-след\-ст\-вен\-ных связей. Рассматриваются три вида мошенничества, 
а именно:
  \begin{enumerate}[(1)]
\item отмыв денег; 
\item обман при выполнении договорных обязательств при реализации 
технических проектов (строительные проекты и~др.); 
\item незаконный вывод денег. 
\end{enumerate}

  Названные виды мошенничества могут быть сведены к~решению одного типа 
задач. Для отмывания денег источник должен заключать фиктивные контракты, 
в~соответствии с~которыми будут переводиться средства за заведомо ненужную 
работу и~материалы. 
  
  Мошенничество, связанное с~невыполнением договорных обязательств, связано 
со снижением качества услуг, качества и~количества закупаемых 
материалов, выполнением работ с~ненадлежащим качеством. 
  
  Вывод денег связан с~переводом средств фир\-мам-од\-но\-днев\-кам, которые 
заведомо не могут выполнить обязательства по контрактам, за которые им 
переводятся средства. 
  
  Таким образом, во всех трех видах рассматриваемых мошенничеств должно 
наблюдаться несоответствие между целями финансовых транзакций и~реальной 
стоимостью достижения этих целей. Данные о транзакциях можно собирать, 
наблюдая информационные потоки, в~которых отражаются эти транзакции. 
  
  Однако для наблюдения таких информационных потоков необходимо создавать 
архитектуру\linebreak телекоммуникационной системы, позволяющей перехватывать 
и~собирать данные о всех транзакциях. Например, такая архитектура может быть 
организована с~помощью распределенных реестров с~централизованным 
консенсусом, т.\,е.\ все информационные потоки, сформированные в~цифровой 
экономике и~несущие информацию о транзакциях, проходят через некоторый 
центральный узел, запоминающий их в~форме распределенного реестра. Такие 
реестры могут дублироваться в~аналогичных центрах различных регионов, что 
позволяет создать аналог электронной бухгалтерской книги, фиксирующей 
фи\-нан\-со\-во-эко\-но\-ми\-че\-скую деятельность субъектов цифровой экономики. Такой 
подход предложено реализовать на базе системы ситуационных центров, что 
отражено в~работах~[1, 2].
  
  Собранная из информационных потоков информация о~транзакциях, т.\,е.\ 
о~контрактах, договорах, платежах, отчетах, закупленных материалах, 
характеристиках исполнителей работ и~др., собирается в~базе данных в~указанном 
центре. Согласно теории интеллектуальных сис\-тем~[3], эту базу данных можно 
называть базой фактов (БФ). Базу фактов можно представить как бинарную мат\-ри\-цу, 
строки которой описывают характеристики, входящие в~транзакции, а столбцы 
нумеруются характеристиками. Строки матрицы будем называть 
\textit{объектами}~[4, 5]. 
  
  Рассматриваемые в~работе методы выявления мошенничества будут основаны 
на противоречиях между действиями, описанными в~транзакциях, и~информацией, 
содержащейся в~планах, стандартах, прецедентах и~др. Для нахождения 
противоречий в~архитектуре центра предусмотрена другая база данных~--- база 
знаний (БЗ)~\cite{3-gr, 6-gr}, которая устроена так же, как БФ. 
  
  Информация в~БЗ собирается на основе положительного опыта или расчетов. 
Используя БЗ, можно выводить факты нарушения при\-чин\-но-след\-ст\-вен\-ных 
связей. Нарушения при\-чин\-но-след\-ст\-вен\-ных связей будем называть 
\textit{аномалиями}. 
  
  Для упрощения дальнейшее изложение будет вестись в~рамках поиска 
противоречий при выполнении некоторого абстрактного проекта. Выявление 
аномалий будет происходить на основе фактов из БФ с~помощью знаний из БЗ 
методами искусственного интеллекта и~интеллектуального анализа 
данных~\cite{6-gr}. 

\vspace*{-10pt}
  
  \section{Модели}
  
  \vspace*{-3pt}
  
  Наиболее сложная из рассмотренных выше задач~--- выявление противоречий, 
т.\,е.\ использование БЗ для получения новых знаний и~выявление аномалий из 
полученных фактов. 
  
  Все способы выявления противоречий основаны на определении 
  причинно-следственных связей. При этом противоречия в~параметрах транзакций по 
отношению к~требуемым в~БЗ составляют сущность аномалий. 
  
   Далее будет рассмотрен метод, основанный на некоторой упрощенной схеме 
реализации абстрактного проекта. 
  
  Каждый проект имеет цель: например, цель представляет собой построение 
некоторой системы. Воспользуемся структурным подходом, который позволяет 
строить проект на основе разбиения системы на подсистемы и~определения 
взаимодействий подсистем~\cite{7-gr}. При этом каждая подсистема также 
представима структурной моделью. 
  
  Как сама система, так и~каждая ее подсистема имеют свой функционал 
и~спецификацию, па\-ра\-мет\-ры настройки и~домены параметров настройки. Кроме 
этих характеристик существует множество характеристик, связанных 
с~<<жизненным циклом>> создания системы. Сюда входят работы, ресурсы, 
сроки выполнения работ по созданию подсистем и~самой системы, стоимости 
компонентов и~материалов, стоимости работ, схемы поставок, договорные 
обязательства и~др. Все характеристики связаны между собой, поэтому можно 
говорить о стоимости и~времени изготовления структурных компонентов системы. 
  
  Одной из важнейших характеристик является смета (система смет для 
подсистем). Смета сопоставляет каждому компоненту системы стоимость его 
изготовления и~настройки. 
  
  Схема построения системы может быть пред\-став\-ле\-на диаграммой, 
изображенной на рис.~1. 

{ \begin{center}  %fig1
 \vspace*{9pt}
   \mbox{%
 \epsfxsize=79mm 
 \epsfbox{gru-1.eps}
 }


\vspace*{9pt}


\noindent
{{\figurename~1}\ \ \small{Диаграмма достижения цели}}
\end{center}
}

\vspace*{9pt}

\addtocounter{figure}{1}
  
  


  Представленная на рис.~1 диаграмма позволяет описать основные классы 
возможных противоречий при достижении цели. Противоречия возникают, когда 
данные БФ не соответствуют требуемым характеристикам. 
  
  
  \section{Потенциальные классы аномалий при~достижении цели}
  
  Выделим четыре потенциальных класса противоречий, которые показывают, 
каким образом нужно искать эти противоречия.
  
 
  Противоречие цели и~проекта (рис.~2) возникает при отсутствии обоснования 
или в~случае логического противоречия между возможностями проектируемого 
функционала и~целью системы. Отметим, что в~проект входят сроки, перечень 
работ, материалы, настройки, которые описываются соответствующими 
параметрами и~допустимыми значениями этих параметров. Проект формируется 
на основе БЗ и~расчетов, исходя из информации, полученной по аналогии 
с~другими проектами и~решениями, которые считаются апробированными. 
  
  Отметим, что цель порождает проект и~в этом смысле является причиной 
проекта. Однако для анализа противоречий необходимо двигаться по штриховой 
стрелке диаграммы (см.\ рис.~2) от проекта к~цели. В~самом деле, любой компонент 
проекта направлен на теоретическое достижение цели. Цель~--- сложный объект, 
поэтому в~проекте могут возникнуть характеристики, противоречащие хотя бы 
некоторым характеристикам цели. Это делает проект противоречивым, но вывод 
об этом может быть сделан только на уровне описания цели. 
  

  Противоречия между проектом и~его реализацией, исключая настройки 
(рис.~3), могут возникать, например, при закупке исполнителем материалов более 
низкого качества по более низким ценам, при попытках достижения требуемых 
сроков работы за счет снижения качества выполнения работ, за счет нахождения 
<<объективных>> причин для увеличения сроков работы и,~следовательно, 
увеличения цены реализации проекта. 


  Для выявления указанных противоречий необходимо двигаться по диаграмме 
(см.\ рис.~3) в~обратную сторону в~соответствии со~штриховыми стрелками. 
Действительно, выявить противоречия между характеристиками закупленных 
материалов и~требуемыми по проекту можно только при обращении к~проекту 
и~его спецификациям. Манипуляции со сроками работы также можно выявить 
только при обращении к~соответствующим расчетам в~проекте. Задержки в~сроках 
работы, связанные с~поставками материалов, можно определить только на 
предыдущем этапе диаграммы (см.\ рис.~3) в~описании проекта. 


  


  Противоречия между реализацией проекта и~его настройкой (рис.~4) возникает, 
когда не удается добиться требуемых значений параметров функционала, не 
удается обеспечить необходимый уровень\linebreak\vspace*{-12pt}

{ \begin{center}  %fig2
 \vspace*{-6pt}
   \mbox{%
 \epsfxsize=16mm 
 \epsfbox{gru-2.eps}
 }


\vspace*{6pt}


\noindent
{{\figurename~2}\ \ \small{Противоречия цели и~проекта}}
\end{center}
}

%\vspace*{9pt}

\addtocounter{figure}{1}

{ \begin{center}  %fig3
 \vspace*{6pt}
    \mbox{%
 \epsfxsize=79mm 
 \epsfbox{gru-3.eps}
 }


\end{center}

\vspace*{-2pt}


\noindent
{{\figurename~3}\ \ \small{Противоречия проекта и~его реализации (без настройки)}}
}

\vspace*{6pt}

\addtocounter{figure}{1}

{ \begin{center}  %fig4
 \vspace*{1pt}
   \mbox{%
 \epsfxsize=54.5mm 
 \epsfbox{gru-4.eps}
 }


\end{center}


\noindent
{{\figurename~4}\ \ \small{Противоречия реализации проекта и~его на\-стройки}}
}

%\vspace*{9pt}

\addtocounter{figure}{1}

{ \begin{center}  %fig5
 \vspace*{5pt}
    \mbox{%
 \epsfxsize=79mm 
 \epsfbox{gru-5.eps}
 }


\end{center}



\noindent
{{\figurename~5}\ \ \small{Противоречия цели и~достигнутой реализации проекта}}
}

\vspace*{6pt}

\addtocounter{figure}{1}

\noindent
 качества реализации проекта. Для 
определения противоречия в~настройках надо опять же двигаться по диаграмме 
(см.\ рис.~4) в~обратную сторону по штриховым стрелкам, так как для выявления 
характеристик результатов работы, которые не дают возможности реализации 
определенного функционала, необходимо иметь информацию о результатах этой 
работы. 


  



  Противоречие между целью и~достигнутой реализацией проекта (рис.~5) 
возникает, когда реализованная система не позволяет достичь цели. В~этом случае 
опять противоречие нужно искать, двигаясь от цели к~реальному достигнутому 
функционалу по штриховой стрелке (см.\ рис.~5).
  
  Суммируя положения, изложенные в~данном разделе, приходим к~выводу, что 
для выявления противоречий необходимо проводить анализ от следствия 
к~причине, т.\,е.\ искать аномалии в~информации, описывающей порождение 
наблюдаемых следствий. 
  
  
  \section{Связь противоречий и~причин}
  
  Прежде чем построить связь между причинами и~противоречиями, кратко 
опишем простейшую модель связи этих понятий. Причины и~противоречия будут 
сформулированы для представления компонентов системы как объектов, 
обладающих наборами известных характеристик~\cite{4-gr, 5-gr}. 
  
  Пусть $U\hm=\{\alpha, \beta, \ldots\}$~--- совокупность характеристик 
(пространство характеристик). Согласно~\cite{4-gr} \textit{объектом}~$O$ 
называется любое подмножество характеристик $O\hm\subseteq U$. Рассмотрим 
последовательность объектов, возможно в~различных пространствах 
характеристик. 
  
  \smallskip
  
  \noindent
  \textbf{Определение~1.}\ Объект~$P$ с~числом характеристик, большим или 
равным~2, является \textit{причиной} объекта (\textit{свойства})~$B$ в~цепочке 
наблюдаемых объектов тогда и~только тогда, когда выполнены следующие 
условия:
  \begin{enumerate}[(1)]
\item для каждого объекта~$C$, если $P\hm\subseteq C$, то $C\mapsto B$, где 
$C\mapsto B$ означает, что объект~$B$ присутствует в~объекте, следующем за 
объектом~$C$;
\item объект~$P$ является минимальным объектом, удовлетворяющим 
условию~1, а~именно: $\forall \alpha\hm\in P$ объект~$P\backslash \{\alpha\}$ 
не является причиной, т.\,е.\ $\exists C:\ \alpha\not\in C$, $P\backslash 
\{\alpha\}\hm\subseteq C$ и~$C\not\mapsto B$, где $C\not\mapsto B$ означает, 
что~$B$ не может содержаться в~объекте, следующем за объектом~$C$. 
\end{enumerate}

  Приведенное определение причины является упрощением причин, 
возникающих в~реальном мире. Например, реальные причины могут возникать\linebreak 
как совокупность характеристик из разных пространств. Одно следствие может 
порождаться разными причинами или возникать из внешних\linebreak и~ненаблюдаемых 
характеристик. Однако пред\-став\-лен\-ная далее формализация позволяет доступно 
изложить при\-чин\-но-след\-ст\-вен\-ные истоки противоречий, которые 
инициируют в~дальнейшем глубокое исследование рассматриваемых процессов.
  
  Будем считать, что для любого интересующего нас свойства~$B$ существует 
причина. Тогда справедлива следующая теорема.
  
  \smallskip
  
  \noindent
  \textbf{Теорема~1.}\ \textit{Для любого свойства~$B$ существует 
единственная причина}. 
  
  \smallskip
  
  \noindent
  Д\,о\,к\,а\,з\,а\,т\,е\,л\,ь\,с\,т\,в\,о\,.\ \ Доказательство будем вести от противного, 
т.\,е.\ предположим, что существуют две причины свойства~$B$: $P$ 
и~$P^\prime$, $P\hm\not= P^\prime$. Тогда существует $\alpha\hm\in U$, которое 
удовлетворяет одному из двух условий:
  \begin{itemize}
\item[(а)] $\alpha\in P$, $\alpha\notin P^\prime$;
\item[(б)] $\alpha\notin P$, $\alpha \in P^\prime$.
\end{itemize}

  Пусть выполняется условие~(б). Тогда $P^\prime\backslash \{\alpha\}$ не 
является причиной по условию~2 определения~1, т.\,е.\ $\exists C$ такое, что 
$\alpha\notin C$, $P^\prime\backslash \{\alpha\}\hm\subseteq C$ и~$C\not\mapsto B$. 
Но если~$B$ произошло и~$P$ его причина, то $C\mapsto B$, что противоречит 
предположению. Теорема~1 доказана.
  
  \smallskip
  
  \noindent
  \textbf{Лемма.} \textit{Если $P$~--- причина появления свойства~$B$, то 
объект~$B$ определяет существование свойства~$P$ в~объекте, 
предшествующем~$B$. }
  
  \smallskip
  
  \noindent
  Д\,о\,к\,а\,з\,а\,т\,е\,л\,ь\,с\,т\,в\,о\,.\ \ Из предположения, что у~каж\-до\-го 
свойства~$B$ есть причина, и~условия, что~$P$ является причиной~$B$, следует, 
что при появлении в~данных свойства~$B$ объект~$C$, предшествующий 
появлению~$B$, содержит как часть объект~$P$. Это следует из теоремы~1 
и~определения причины. 
  
  Докажем принцип <<необходимого условия>>, который, несмотря на простоту 
доказательства, будет играть в~дальнейшем существенную роль.
  
  \smallskip
  
  \noindent
  \textbf{Теорема~2.} \textit{Если~$P$~--- причина появления свойства~$B$ 
и~$A\hm\subseteq P$, то объект~$B$ определяет наличие свойства~$A$ 
в~объекте, предшествующем~$B$}. 
  
  \smallskip
  
  \noindent
  Д\,о\,к\,а\,з\,а\,т\,е\,л\,ь\,с\,т\,в\,о\,.\ \ Пусть в~данных имеется объект~$B$ 
и~$P\mapsto B$, тогда в~силу существования и~единственности причины~$B$ 
в~данных должен существовать объект~$C$, предшествующий~$B$ 
и~содержащий причину~$P$. Поскольку $A\hm\subseteq P$ и~$B$ содержит 
причину~$P$, то $B\mapsto A$. С~учетом леммы теорема~2 доказана.
  
  \smallskip
  
  Пусть даны пространства $U_1, U_2,\ldots$ и~имеется последовательность 
данных (процесс выполнения этапов проекта в~соответствии с~рис.~1) $A, B, 
\ldots$, где каждый объект является подмножеством некоторого 
пространства~$U_i$, $i\hm=1,\ldots$ Тогда в~объекте~$A$ присутствует 
причина~$P$ появления интересующего нас свойства~$C$ в~объекте~$B$. Пусть 
$P\hm\subseteq A$, тогда по теореме~2 $\forall \alpha\hm\in P$:  
$C\mapsto \{\alpha\}$, т.\,е.\ из появления~$C$ следует появление 
характеристики~$\alpha$ в~предшествующем объекте. Это необходимое условие 
того, что~$C$ удовлетворяет причинно-следственным связям развития процесса 
выполнения проекта. Если для~$C$ нет характеристики~$\alpha$, которую можно 
отнести к~причине~$C$, то можно считать, что нарушена  
при\-чин\-но-след\-ст\-вен\-ная связь и~$C$~--- аномальный объект. 
  
  \smallskip
  
  \noindent
  \textbf{Пример.} Если объект~$C$ состоит в~получении суммы~$a$ 
фирмой~$K$, то согласно теореме~2 в~пред\-шест\-ву\-ющем объекте должна 
существовать причина перевода суммы~$a$ на фирму~$K$. Если эта причина 
в~проекте отсутствует, то это можно считать признаком мошеннической схемы. 
Все проекты по предположению собираются из <<кубиков>>, содержащихся в~БЗ. 
Тогда можно сравнить цену объекта~$C$, породившего получение суммы~$a$, 
и~сумму, присутствующую в~смете проекта. Если разница велика, то это либо 
ошибка проекта, либо признак мошеннической схемы.
  
  \section{Поиск противоречий на~основе~принципа <<необходимых~условий>>}
   
  Как было показано в~разд.~3, нахождение противоречий соответствуют 
движению от следствия к~причине. Для каждого объекта в~наблюдаемых данных 
выявление причин его появления является трудоемкой задачей. Кроме того, при 
реализации контроля соблюдения при\-чин\-но-след\-ст\-вен\-ных связей на 
большом множестве участников экономической деятельности задача анализа 
причин становится трудоемкой. Поэтому процедуру контроля необходимо разбить 
на два этапа, где первый этап состоит в~анализе простых <<необходимых 
условий>> проявления мошенничества, когда используется хотя бы одна 
известная характеристика причины. Второй этап (в~режиме офлайн) состоит 
в~выявлении причин, позволяющих провести анализ источников мошеннических 
схем. 
  
  Один из подходов к~выбору <<необходимых условий>> состоит в~построении 
множества подцелей исходной цели проекта (структурный метод построения 
проекта~\cite{7-gr}). Каждая подцель описывается диаграммой на рис.~1, 
и~реализации подцелей должны образовывать полный функционал цели. Это 
является необходимым, но не достаточным условием достижения цели, так как 
при таком подходе отсутствует компонент согласования всех подцелей в~единую 
систему. Однако такой подход значительно упрощает анализ выполнения проекта 
на предмет поиска мошенничества. Если признаки мошенничества будут 
обнаружены в~реализации хотя бы одной из подцелей, то это значит, что 
мошенничество присутствует в~реализации всего проекта. 
  
  Аналогично в~реализации каждого этапа в~любой из подцелей можно выделять 
простые <<необходимые условия>> нарушения при\-чин\-но-след\-ст\-венн\-ых 
связей. 
  
  Таким образом, получается множество <<необходимых условий>>, нарушение 
которых свидетельствует о наличии мошенничества. Это множество 
<<необходимых условий>> можно назвать метаданными~[8, 9] для контроля 
проекта на выявление мошенничества. 
  
  
  \section{Заключение }
  
  В поиске противоречий необходимо от транзакций, соответствующих 
следствиям при\-чин\-но-след\-ст\-вен\-ных связей, переходить к~анализу причин 
наблюдаемых следствий. Это сложная задача, которая связана с~описанием причин 
определенных свойств. 
  
  В работе представлена модель, позволяющая строить множество необходимых 
условий соответствия наблюдаемого следствия вызвавшей его причине. Этот 
подход делает поиск противоречий вполне вычислимой задачей, но не гарантирует 
успех. 
  
  {\small\frenchspacing
 {%\baselineskip=10.8pt
 \addcontentsline{toc}{section}{References}
 \begin{thebibliography}{9}
\bibitem{1-gr}
\Au{Грушо А.\,А., Зацаринный~А.\,А., Тимонина~Е.\,Е.} Блокчейны цифровой экономики на базе 
системы ситуационных центров и~централизованного консенсуса~// Радиолокация, навигация, 
связь: Мат-лы XXV Междунар. научн.-технич. конф.~---
Воронеж: Издательский дом ВГУ, 2019. Т.~6. С.~183--191. 
\bibitem{2-gr}
\Au{Grusho A., Zatsarinny~A., Timonina~E.} A~system approach to information security in 
distributed ledgers on the situational centers platform.~---
Lecture notes in computer science ser.~--- Springer, 2019 
(in press).
\bibitem{3-gr}
\Au{Финн В.\,К.} Искусственный интеллект: Методология, применения, философия.~--- М.: 
Красанд, 2011. 448~с.

\bibitem{5-gr} %4
\Au{Аншаков~О.\,М., Фабрикантова~Е.\,Ф.} ДСМ-ме\-тод автоматического порождения 
гипотез: Логические и~эпистемологические основания.~--- М.: Либроком, 2009. 432~с.

\bibitem{4-gr} %5
\Au{Poelmans J., Elzinga~P., Viaene~S., Dedene~G.} Formal concept analysis in knowledge 
discovery: A~survey~// Conceptual structures: From information to intelligence~/ Eds.\ M.~Croitoru, 
S.~Ferr$\acute{\mbox{e}}$, and D.~Lukose.~--- Lecture notes in computer science 
ser.~--- Berlin--Heidelberg: Springer, 2010. Vol.~6208.  P.~139--153.

\bibitem{6-gr}
\Au{Панкратова~Е.\,С., Финн~В.\,К.} Автоматическое по\-рож\-де\-ние гипотез в~интеллектуальных 
системах.~--- М.: Либроком, 2009. 528~с. 
\bibitem{7-gr}
\Au{Денисов А.\,А., Колесников~Д.\,Н.} Теория больших систем управления.~--- Л.: Энергоиздат, 1982. 488~с.

\bibitem{9-gr}
\Au{Грушо А.\,А., Грушо Н.\,А., Забежайло~М.\,И., Смирнов~Д.\,В., Тимонина~Е.\,Е.} 
Параметризация в~прикладных задачах поиска эмпирических причин~// Информатика и~её 
применения, 2018. Т.~12. Вып.~3. С.~62--66.

\bibitem{8-gr}
\Au{Грушо А.\,А., Грушо Н.\,А., Левыкин~М.\,В., Тимонина~Е.\,Е.} Методы идентификации 
захвата хоста в~распределенной ин\-фор\-ма\-ци\-он\-но-вы\-чис\-ли\-тель\-ной сис\-те\-ме, 
защищенной с~помощью метаданных~// Информатика и~её применения, 2018. Т.~12. Вып.~4. 
С.~41--45.

 \end{thebibliography}

 }
 }

\end{multicols}

\vspace*{-3pt}

\hfill{\small\textit{Поступила в~редакцию 03.04.19}}

%\vspace*{8pt}

%\pagebreak

\newpage

\vspace*{-28pt}

%\hrule

%\vspace*{2pt}

%\hrule

%\vspace*{-2pt}

\def\tit{ARCHITECTURAL DECISIONS IN~THE~PROBLEM 
OF~IDENTIFICATION OF~FRAUD IN~THE~ANALYSIS 
OF~INFORMATION FLOWS IN~DIGITAL ECONOMY\\[-5pt]}


\def\titkol{Architectural decisions in~the~problem 
of~identification of~fraud in~the~analysis 
of~information flows in~digital economy}

\def\aut{A.\,A.~Grusho, M.\,I.~Zabezhailo, N.\,A.~Grusho, and~E.\,E.~Timonina}

\def\autkol{A.\,A.~Grusho, M.\,I.~Zabezhailo, N.\,A.~Grusho, and~E.\,E.~Timonina}

\titel{\tit}{\aut}{\autkol}{\titkol}

\vspace*{-13pt}


 \noindent
   Institute of Informatics Problems, Federal Research Center ``Computer Sciences and 
Control'' of the Russian Academy of Sciences; 44-2~Vavilov Str., Moscow 119133, 
Russian Federation

\def\leftfootline{\small{\textbf{\thepage}
\hfill INFORMATIKA I EE PRIMENENIYA~--- INFORMATICS AND
APPLICATIONS\ \ \ 2019\ \ \ volume~13\ \ \ issue\ 2}
}%
 \def\rightfootline{\small{INFORMATIKA I EE PRIMENENIYA~---
INFORMATICS AND APPLICATIONS\ \ \ 2019\ \ \ volume~13\ \ \ issue\ 2
\hfill \textbf{\thepage}}}

\vspace*{3pt}


   
     
   \Abste{An approach to a~research of some types of fraud in digital economy with the usage of relationships of 
cause and effect is formulated. In all types of the considered frauds, the discrepancy between the 
purposes of financial transactions and actual cost of achievement of these purposes
has to be observed. Data on 
transactions can be collected by observing information flows in which these transactions are reflected. 
The architecture of data collection and their analysis can be organized by means of the distributed 
ledgers with the centralized consensus that allows creating an analog of the electronic account book 
fixing financial and economic activity of subjects of digital economy in the region. 
   The methods of fraud identification considered are based on the contradictions 
between actions described in transactions and information, which is contained in plans, standards, 
precedents, etc. 
   The method based on a~simplified scheme of implementation of the abstract project is considered. 
For identification of contradictions, it is necessary to carry out the analysis from the effect to the cause, 
i.\,e., to look for anomalies in information describing the generation of the observed effects. 
   It is shown how in implementation of the project it is possible to allocate simple ``necessary 
conditions'' of violation of cause and effect relationships, i.\,e., a~set of ``necessary conditions'' 
violation of which demonstrates fraud existence. It is possible to call this set of "necessary conditions" 
by metadata for control of the project for fraud identification.} 
   
   \KWE{digital economy; information flows; relationships of reason and effect; detection of 
fraudulent schemes}
   
  

 \DOI{10.14357/19922264190204}

\vspace*{-20pt}

 \Ack
   \noindent
   The work was partially supported by the Russian Foundation for Basic Research (projects  
18-29-03081 and 18-07-00274).



%\vspace*{6pt}

  \begin{multicols}{2}

\renewcommand{\bibname}{\protect\rmfamily References}
%\renewcommand{\bibname}{\large\protect\rm References}

{\small\frenchspacing
 {\baselineskip=10.5pt
 \addcontentsline{toc}{section}{References}
 \begin{thebibliography}{9}
\bibitem{1-gr-1}
\Aue{Grusho, A.\,A., A.\,A.~Zatsarinny, and E.\,E.~Timonina.} 2019. Blokcheyny tsifrovoy ekonomiki 
na baze sistemy situatsionnykh tsentrov i~tsentralizovannogo konsensusa [Blockchains of digital 
economy on the basis of the system of the situational centres and the centralized consensus]. 
\textit{25th Scientific and Technical Conference (International) ``Radar-Location, Navigation, 
Communication'' Proceedings}. Voronezh: VSU Publs. 6:183--191.
\bibitem{2-gr-1}
\Aue{Grusho, A., A.~Zatsarinny, and E.~Timonina.} 2019 (in press). 
A~system approach to information security 
in distributed ledgers on the situational centers platform. 
Lecture notes in computer science ser. Springer.
\bibitem{3-gr-1}
\Aue{Finn, V.\,K.} 2011. \textit{Iskusstvennyy intellekt: Metodologiya, primeneniya, filosofiya} 
[Artificial intelligence: Methodology, applications, philosophy]. Moscow: KRASAND. 448~p.

\bibitem{5-gr-1}
\Aue{Anshakov, O.\,M., and E.\,F.~Fabrikantova}. 2009. \textit{DSM-metod avtomaticheskogo porozhdeniya gipotez: Logicheskie 
i~epistemologicheskie osnovaniya} [JSM-method of automatic hypothesis generation: Logical and 
epistemological]. Moscow: KD LIBROKOM. 432~p.
\bibitem{4-gr-1} %5
\Aue{Poelmans, J., P.~Elzinga, S.~Viaene, and G.~Dedene.} 2010. Formal concept analysis in 
knowledge discovery: A~survey. \textit{Conceptual structures: From information to intelligence}. 
Eds.\ M.~Croitoru, S.~Ferr$\acute{\mbox{e}}$, and D.~Lukose. Lecture notes in 
computer science ser. Berlin--Heidelberg: Springer. 6208:139--153.

\bibitem{6-gr-1}
\Aue{Pankratov, E.\,S., and V.\,K.~Finn}. 
2009. \textit{Avtomaticheskoe porozhdenie gipotez v~intellektual'nykh 
sistemakh} [Automatic hypotheses generation in intelligent systems]. Moscow: KD 
\mbox{LIBROKOM}.  528~p. 
\bibitem{7-gr-1}
\Aue{Denisov, A.\,A., and D.\,N.~Kolesnikov.} 1982. \textit{Teoriya bol'shikh 
sistem upravleniya} [Theory of big control systems]. Leningrad: Energoizdat. 488~p.

\bibitem{9-gr-1}
\Aue{Grusho, A.\,A., N.\,A.~Grusho, M.\,I.~Zabezhailo, D.\,V.~Smirnov, and 
E.\,E.~Timonina.} 2018. 
Parametrizatsiya v~prikladnykh zadachakh poiska empiricheskikh prichin 
[Parametrization in applied 
problems of search of the empirical reasons]. 
\textit{Informatika i~ee Primeneniya~--- 
Inform. Appl.} 12(3):62--66.

\bibitem{8-gr-1}
\Aue{Grusho, A.\,A., N.\,A.~Grusho, M.\,V.~Levykin, and E.\,E.~Timonina.} 2018. Metody 
identifikatsii zakhvata khosta v~raspredelennoy informatsionno-vychislitel'noy sisteme, 
zashchishchennoy s~pomoshch'yu metadannykh [Methods of identification of host capture 
in the  distributed information system which is protected on the base of meta data].
\textit{Informatika i~ee 
Primeneniya~--- Inform. Appl.} 12(4):41--45.
{ %\looseness=1

}

\end{thebibliography}

 }
 }

\end{multicols}

\vspace*{-12pt}

\hfill{\small\textit{Received April 3, 2019}}

%\pagebreak

%\vspace*{-18pt}

\Contr

\noindent
\textbf{Grusho Alexander A.} (b.\ 1946)~--- Doctor of Science in physics and 
mathematics, professor, principal scientist, Institute of Informatics Problems, 
Federal Research Center ``Computer Sciences and Control'' of the Russian 
Academy of Sciences; 44-2~Vavilov Str., Moscow 119133, Russian Federation; 
\mbox{grusho@yandex.ru} 

\vspace*{3pt}

\noindent
\textbf{Zabezhailo Michael I.} (b.\ 1956)~--- Doctor of Science in physics and 
mathematics, principal scientist, Institute of Informatics Problems, Federal Research 
Center ``Computer Sciences and Control'' of the Russian Academy of Sciences;  
44-2~Vavilov Str., Moscow 119133, Russian Federation; 
\mbox{m.zabezhailo@yandex.ru} 

\vspace*{3pt}


\noindent
\textbf{Grusho Nikolai A.} (b.\ 1982)~--- Candidate of Science (PhD) in physics 
and mathematics, senior scientist, Institute of Informatics Problems, Federal 
Research Center ``Computer Sciences and Control'' of the Russian Academy of 
Sciences; 44-2~Vavilov Str., Moscow 119133, Russian Federation; 
\mbox{info@itake.ru} 

\vspace*{3pt}


\noindent
\textbf{Timonina Elena E.} (b.\ 1952)~--- Doctor of Science in technology, 
professor, leading scientist, Institute of Informatics Problems, Federal Research 
Center ``Computer Sciences and Control'' of the Russian Academy of Sciences;  
44-2~Vavilov Str., Moscow 119133, Russian Federation; 
\mbox{eltimon@yandex.ru} 

\label{end\stat}

\renewcommand{\bibname}{\protect\rm Литература}    %11
\def\stat{husainov}

\def\tit{ПРОИЗВОДИТЕЛЬНОСТЬ ОГРАНИЧЕННОГО КОНВЕЙЕРА}

\def\titkol{Производительность ограниченного конвейера}

\def\aut{А.\,А.~Хусаинов$^1$}

\def\autkol{А.\,А.~Хусаинов}

\titel{\tit}{\aut}{\autkol}{\titkol}

\index{Khusainov A.\,A.}
\index{Хусаинов А.\,А.}


%{\renewcommand{\thefootnote}{\fnsymbol{footnote}} \footnotetext[1]
%{Работа выполнена при финансовой поддержке Российского научного фонда (проект 18-11-00155).}}


\renewcommand{\thefootnote}{\arabic{footnote}}
\footnotetext[1]{Комсомольский-на-Амуре государственный университет, 
\mbox{husainov51@yandex.ru}}

%\vspace*{-12pt}



  \Abst{Работа посвящена изучению производительности ограниченного конвейера~--- 
вычислительного конвейера, число активных ступеней которого в~каждый момент 
времени ограничено сверху некоторым значением. Рассмотрены ограниченные 
конвейеры с~заданными суммой и~максимумом задержек ступеней. Ступени могут иметь 
разные задержки. Основная задача~--- построение аналитической модели для расчета 
времени обработки заданного объема данных с~помощью этого ограниченного 
конвейера. Решение упрощается, если ограничение рассматривать как структурный 
конфликт конвейера. Эта аналитическая модель построена для случая, когда работа 
ограниченного конвейера обладает свойством непрерывности обработки каждого 
входного элемента. Для таких конвейеров в~работе доказана гипотеза о~том, что 
минимальное число процессоров, при котором достигается наибольшая 
производительность, равно наименьшему целому числу, не меньшему отношения суммы 
задержек ступеней к~их максимальной задержке. Установлено, что если не требовать 
свойства непрерывности, то эта гипотеза неверна. Построенная модель может быть 
применена для синхронизации работы ступеней ограниченного конвейера со свойством 
непрерывности. Если не требовать свойства непрерывности, то получаем асинхронный 
ограниченный конвейер, синхронизация работы ступеней которого осуществляется на 
основе го\-тов\-ности данных. Разработано программное обеспечение, позволяющее 
вычислять время обработки данных с~по\-мощью асинхронного ограниченного 
конвейера.}
   
  \KW{вычислительный конвейер; моноид трасс; нормальная форма Фоаты; 
производительность конвейера; структурный конфликт}

\DOI{10.14357/19922264200112} 
  
\vspace*{-3pt}


\vskip 10pt plus 9pt minus 6pt

\thispagestyle{headings}

\begin{multicols}{2}

\label{st\stat}

\section{Введение}

  Вычислительный конвейер, состоящий из~$p$ ступеней, называется 
\textit{ограниченным} некоторым чис\-лом~$q$, если в~каждый момент 
времени могут одновременно выполняться не более чем~$q$~ступеней. 
В~данной работе найдена формула для расчета времени обработки~$n$ 
входных элементов с~помощью ограниченного конвейера, обладающего 
свойством непрерывности обработки для каждого входного элемента 
конвейера. С~по\-мощью этой формулы для ограниченного конвейера со 
свойством непрерывности подтверждена выдвинутая в~[1] на основании 
экспериментов гипотеза о~том, что минимальное число процессоров, при 
котором достигается наибольшая производительность ограниченного 
конвейера, будет равно наименьшему целому~$q$, удовлетворяющему 
неравенству $q\hm\geq \sigma/\mu$, где $\sigma$~--- сумма задержек 
ступеней конвейера, а~$\mu$~--- задержка самой медленной ступени. 
Приведен пример, показывающий, что в~общем случае эта гипотеза неверна.
  
  Проведенные исследования тесно связаны с~конфликтами, 
возникающими при работе конвейера. Под конфликтами подразумеваются 
со\-стояния, приводящие к~замедлению работы\linebreak конвейера. Тео\-рия 
конфликтов применяется при разработке конвейерных процессоров~[2], 
сигнальных процессоров~[3], сопроцессоров~[4]. Существуют три типа 
конфликтов~[5]: структурные конфликты, конфликты по данным 
и~конфликты управления. Структурный конфликт~--- оборудование не 
может поддержать комбинацию инструкций, которые необходимо 
выполнить одновременно в~некоторый момент времени. Ограниченный 
конвейер можно рассматривать как конвейер со структурным конфликтом. 
Аналитические модели для расчета производительности конвейеров 
с~конфликтами построены в~[2, 6, 7]. Эти модели предназначены для 
однородных конвейеров~--- конвейеров, ступени которых имеют 
одинаковые задержки. Заметим, что в~работах~[8, 9] изучались 
неоднородные конвейеры и~был предложен метод динамического 
отоб\-ра\-же\-ния конвейера (dynamic pipeline mapping) для улучшения 
про\-из\-во\-ди\-тель\-ности, решались задачи, где чис\-ло процессоров превышает 
число ступеней, но проб\-ле\-мы, связанные с~расчетом про\-из\-во\-ди\-тель\-ности 
ограниченных конвейеров, не были решены. 

Автором в~работе~[10] была 
построена аналитическая модель для неоднородного конвейера 
с~единственным конфликтом, вызывающим ре\-старт. В~предлагаемой 
работе строится аналогичная модель для конвейера с~одним конфликтом, 
соответствующего ограниченному конвейеру со свойством непрерывности. 
  
  Для оценки времени ускорения работы программы с~помощью~$q$ 
процессоров можно использовать закон Амдала~[11]. Для конвейеров 
существуют некоторые варианты этого закона, описанные 
в~\cite[п.~1.4.1.3]{12-h}. Естественно предположить, что для ограниченного 
конвейера со свойством непрерывности имеет место 
$$
T_q(n)\approx 
\sigma+ \fr{(n-1)\sigma}{q}\,.
$$

 Будет доказано, что это равенство верно 
с~точностью до суммы задержек ступеней~$\sigma$. 
  
  В разд.~2 рассмотрен однородный ограниченный конвейер. Для 
построения аналитической модели для него достаточно рассмотреть 
таблицу занятости процесса обработки~$n$~входных элементов. В~разд.~3 
построена и~доказана формула для расчета времени обработки данных 
с~помощью неоднородного ограниченного конвейера со свойством 
непрерывности. В~разд.~4 описано программное обеспечение для расчета 
производительности асинхронных ограниченных конвейеров, 
синхронизация работы ступеней которых осуществляется на основе 
готовности данных, передаваемых между ступенями. В~конце разд.~4 
приведен пример, показывающий, что в~общем случае гипотеза 
о~минимальном числе процессоров конвейера неверна. 
     
\section{Однородный ограниченный конвейер}

  Обозначим ступени вычислительного конвейера через $a_1,\ldots , a_p$. 
Ступень, выполняющаяся в~некоторый момент времени на некотором 
процессоре (функциональном устройстве) конвейера, называется 
\textit{активной} в~этот момент. Задержкой ступени называется время 
обработки ступенью одного входного элемента конвейера. Это время 
включает в~себя логические операции и~операции обмена данными 
с~другими ступенями через входные и~выходные каналы. Будем 
предполагать, что процессоры конвейера имеют одинаковую тактовую 
частоту, и~измерять время в~тактах.
  
  Под \textit{таблицей занятости}~[13] конвейера будем подразумевать 
матрицу, строки которой соответствуют ступеням конвейера и~имеют 
номера $1\hm\leq i\hm\leq p$, а~столбцы~--- тактам времени $1, 2, 3,\ldots$ 
Коэффициенты этой матрицы~$a_{ij}$ равны $k\hm\geq1$\linebreak тогда и~только 
тогда, когда $i$-я ступень обрабатывает $k$-й входной элемент в~течение 
такта~$j$. В~этом случае в~клетку $(i,j)$ ставится чис\-ло~$k$. Если $i$-я 
ступень в~момент~$j$ не активна, то $a_{ij}\hm=0$ и~со\-от\-вет\-ст\-ву\-ющая 
клетка в~таблице остается пустой.
  
  Конвейер, состоящий из~$p$~ступеней, называется \textit{ограниченным 
числом}~$q$, если в~каждый момент времени число активных ступеней не 
больше~$q$. 
  
  Ограниченный конвейер имеет следующие свойства:
  \begin{itemize}
\item в~каждый момент времени активна по крайней мере одна ступень;
\item в~каждый момент времени активно не более~$q$~ступеней;
\item перед обработкой входного элемента конвейера для каждого 
$i\hm>1$ ступень~$a_i$ ожидает окончания обработки этого входного 
элемента с~помощью ступени~$a_{i-1}$; 
\item для каждого $i\hm\geq 1$ ступень~$a_i$ ожидает окончания своего 
предыдущего выполнения.
\end{itemize}

  Конвейер называется имеющим свойство \textit{непрерывности} 
(работы), если для всякого входного элемента разность между временем 
конца обработки и~временем начала обработки этого элемента равна сумме 
задержек ступеней конвейера. 
  %
  В част\-ности, свойством непрерывности обладает однородный  
конвейер~--- конвейер, все ступени которого имеют одинаковые задержки, 
равные некоторому числу~$h$.

 Обозначим через~$p$ число его ступеней. 
Если нет конфликтов, то время обработки~$n$~элементов равно 
$(p\hm+n\hm-1)h$. Пусть однородный конвейер ограничен числом~$q$, 
$1\hm\leq q\hm\leq p$. На вход конвейера поступает~$n$ элементов входных 
данных. Таблица~1 показывает занятость конвейера при $p\hm=4$, 
$q\hm=3$ и~$n\hm=5$. При попытке запустить больше чем~$q$ параллельно 
работающих ступеней возникает структурный конфликт, в~результате 
которого каж\-дая ступень будет ожидать освобождения одного из 
процессоров и~время работы этой ступени увеличится на $(p\hm-q)h$. 

\begin{center}
\vspace*{3pt}
%\noindent
{{\tablename~1}\ \ \small{Однородный ограниченный конвейер }}
%\vspace*{2ex}

\vspace*{9pt}


{\small
\tabcolsep=5.8pt
\begin{tabular}{|c|c|c|c|c|c|c|c|c|c|c|}
\hline
&01&02&03&04&05&06&07&08&09&10\\
\hline
1&1&2&3&&4&5&&&&\\
2&&1&2&3&&4&5&&&\\
3&&&1&2&3&&4&5&&\\
4&&&&1&2&3&&4&5&\\
\hline
\end{tabular}
}
\end{center}

%\end{table*}

\vspace*{9pt}

  
  Учитывая случай $p\hm<q$, приходим к~следующему утверждению.
  
  \smallskip
  
  \noindent
  \textbf{Предложение~1}~\cite[Prop.~1]{14-h}. \textit{Время 
обработки~$n$~элементов с~помощью~$q$~процессоров для однородного 
конвейера из~$p$~ступеней с~задержкой~$h$ равно}
  \begin{equation}
  T_q(n)=\left( p+n-1+(p-q)^+\left[ \fr{n-1}{q}\right]\right)h\,.
  \label{e1-h}
  \end{equation}
  Здесь $[x]$ обозначает целую часть вещественного чис\-ла~$x$, а~$(x)^+$ 
означает чис\-ло, равное~$x$, если $x\hm\geq0$, и~равное~0 в~случае 
$x\hm\leq 0$. 
  
  Эта формула была применена в~[14] для расчета оптимальной глубины 
однородного ограниченного конвейера. 

\section{Неоднородный ограниченный конвейер}

  В данном разделе всюду, где не оговорено противное, неоднородный 
ограниченный конвейер будет обладать свойством непрерывности. 
  
  Ступени неоднородного конвейера могут иметь разные задержки 
$\tau_1,\ldots , \tau_p$. Время обработки~$n$~элементов равно
  $$
  T_p(n)=\sigma+(n-1)\mu\,,
  $$
где $\sigma =\sum\nolimits^p_{i=1} \tau_i$~--- сумма, а~$\mu\hm=\max \{ 
\tau_i\vert 1\hm\leq i\hm\leq p\}$~--- максимум задержек ступеней конвейера. 
Правую часть этой формулы можно получить из времени обработки для 
однородного конвейера $(p\hm+n\hm-1)h$, подставляя вместо~$p$ 
отношение $\sigma/\mu$, а~вместо~$h$~--- максимальную задержку 
ступеней~$\mu$. Это приводит к~предположению о том, что аналогичным 
образом из формулы~(1) может быть получена формула для времени 
обработки с~помощью ограниченного неоднородного конвейера. Из этой 
формулы придется удалить слагаемые, для которых $\sigma/\mu\hm- 
q\hm<0$.    Сформулируем и~докажем полученное утверждение. 

\smallskip

\noindent
  \textbf{Теорема~1.} \textit{Время обработки~$n$~элементов  
с~по\-мощью неоднородного конвейера со свойством не\-пре\-рыв\-ности, 
ограниченного числом~$q$ и~состоящего из~$p$~ступеней, равно}
  \begin{equation}
  T_q(n)=\sigma +(n-1)\mu +(\sigma-q\mu)^+\left[ \fr{n-1}{q}\right]\,.
  \label{e2-h}
  \end{equation}
  
  \noindent
  Д\,о\,к\,а\,з\,а\,т\,е\,л\,ь\,с\,т\,в\,о\,.\ \ Обычный конвейер 
обрабатывает~$n$ элементов за время $T_p(n)\hm= \sigma \hm+ (n\hm-
1)\mu$. В~случае, когда активны $q\hm<p$ ступеней, к~этому времени 
добавляется время ожидания свободных процессоров. Это время ожидания 
называется штрафным. Рассмотрим таблицу занятости при 
обработке~$n$~элементов. Первые~$q$~элементов будут обрабатываться 
без лишних торможений. Они будут обработаны за время $T(q)\hm= 
\sigma\hm+ (q\hm-1)\mu$. При попытке обработать $(q+1)$-й элемент 
возникает (структурный) конфликт, связанный с~тем, что чис\-ло 
одновременно работающих ступеней не должно быть больше чем~$q$. Этот 
конфликт будет разрешен после окончания обработки первого элемента, 
ибо в~этом случае появится свободный процессор. В~силу свойства 
непрерывности отсюда вытекает, что обработку $(q+1)$-го элемента можно 
начать в~момент времени~$\sigma$. Время обработки $q+1$~элементов 
будет равно~$2\sigma$. Поскольку для обычного конвейера время 
обработки $q\hm+1$~элементов равно $\sigma\hm+q\mu$, то штрафное 
время будет равно $2\sigma\hm- (\sigma \hm+ q\mu)\hm=\sigma -\hm q\mu$. 
Эти конфликты возникают при обработке элементов 
с~номерами $q\hm+1, 2q+1, \ldots , mq\hm+1$, где~$m$~--- наибольшее 
целое, для которого $mq\hm+1\hm\leq n$. Ясно, что $m\hm= \left[ ({n\hm-
1})/{q}\right]$. При обработке остальных элементов конфликты не 
возникают. Отсюда вытекает, что если $\sigma\hm- q\mu\hm\geq 0$, то 
$T_q(n)\hm=\sigma\hm+ (n\hm-1)\mu\hm+m(\sigma\hm- q\mu)$. Если же 
$\sigma\hm- q\mu\hm\leq 0$, то в~момент времени~$q\mu$ начала обработки 
$(q+1)$-го элемента первый элемент будет обработан и~закончит занимать 
один из процессоров. В~этом случае штрафное время будет равно нулю 
и~$T_q(n)\hm= \sigma\hm+ (n\hm-1)\mu$. Теорема~1 доказана.
  
  \smallskip
  
  
  Из доказанной формулы~(2) вытекает, что при $n\hm-1\hm\geq q$ время 
обработки~$n$~элементов будет минимальным тогда и~только тогда, когда 
имеет место неравенство $\sigma\hm- q\mu\hm\leq 0$. Это приводит 
к~следующему утверждению. 
  
  \smallskip
  
  \noindent
  \textbf{Следствие~1.} Минимальное число процессоров, при котором 
достигается наибольшая производительность ограниченного конвейера со 
свойством непрерывности, равно наименьшему целому~$q$, 
удовле\-тво\-ря\-юще\-му неравенству $q\hm\geq \sigma/\mu$. В~этом случае 
время обработки равно $\sigma\hm+ (n\hm-1)\mu$.
  
  \smallskip
  
  Отсюда вытекает, что предположение, выдвинутое в~работе~[1], верно 
для ограниченных конвейеров со свойством непрерывности. 
  
  Обозначим через $T_q^A(n)$ следующую оценку для 
производительности конвейера:
  $$
  T_q^A(n)=\begin{cases}
  \left(1+\fr{n-1}{q}\right)\sigma\,, &\mbox{если } q\leq \fr{\sigma}{\mu}\,;\\
  \sigma+(n-1)\mu\,, &\mbox{если } q\geq \fr{\sigma}{\mu}\,.
  \end{cases}
  $$
  
  \noindent
  \textbf{Следствие~2.}\ Имеют место неравенства: 
  $0\hm\leq 
T_q^A(n)\hm-T_q(n)\hm<\sigma$.
  
  \smallskip
  
  \noindent
  Д\,о\,к\,а\,з\,а\,т\,е\,л\,ь\,с\,т\,в\,о\,.\ \  Пусть $\sigma\hm\geq q\mu$. 
Воспользуемся тем, что $n\hm-1\hm- q\left[ ({n-1})/{q}\right]$ равно 
остатку $(n\hm-1)\mathrm{mod}\,q$ от деления числа $n\hm-1$ на~$q$. Из 
теоремы~1 вытекает, что 
  \begin{multline*}
  T_q(n)=\sigma +(n-1)\mu +(\sigma-q\mu)\left[ \fr{n-1}{q}\right] ={}\\
  {}=\sigma 
\left( 1+\left[ \fr{n-1}{q}\right]\right)+\mu \left( \left( n-
1\right)\mathrm{mod}\,q\right)\,.
\end{multline*}
   Следовательно, 
  $$
  T_q^A(n)-T_q(n)=(\sigma -q\mu)\fr{(n-1)\mathrm{mod}\,q}{q}<\sigma\,.
  $$
  
  Пусть $S_q(n)=T_1(n)/T_q(n)$~--- ускорение вы\-чис\-ле\-ния с~по\-мощью 
ограниченного конвейера. Обозначим $S_q\hm = \lim\nolimits_{n\to\infty} 
S_q(n)$. Рассматривая случаи $q\hm<\sigma/\mu$ и~$q\hm\geq \sigma/\mu$, 
получаем 
  
  \smallskip
  
  \noindent
  \textbf{Следствие~3.}\ Для конвейера со свойством непрерывности, 
ограниченного числом $q\hm>1$, имеет место равенство $S_q\hm=\min 
(q,\sigma/\mu)$.
    
\section{Асинхронный ограниченный конвейер}

  Попытаемся сравнить производительность огра\-ни\-чен\-но\-го конвейера со 
свойством непрерывности с~производительностью асинхронного 
ограниченного конвейера, синхронизация работы\linebreak ступеней которого 
осуществляется на основе го\-тов\-ности данных, передаваемых между 
ступенями. Но не ясно, как вычислять производительность асинхронного 
конвейера. Возможный ответ\linebreak дает компьютерная программа, которой 
посвящен данный раздел. Эта программа для введенных пользователем 
числовых значений задержек ступеней асинхронного ограниченного 
конвейера и~объема данных вычисляет значения времени обработки\linebreak данных 
в~зависимости от чис\-ла процессоров и~выводит эти значения в~виде 
графиков. Она создает также и~сохраняет в~файл таб\-ли\-цы за\-ня\-тости 
конвейера при различных чис\-лах процессоров.
  
  Программа основана на методе, предложенном Дикертом~[15] 
и~использующем теорию трасс~--- слов, состоящих из букв алфавита, на 
котором задано антирефлексивное симметричное бинарное отношение 
независимости. 
  
  Опишем этот метод. Рассмотрим множество операций (машинных 
команд) $A\hm= \{a_0, \ldots , a_{m-1}\}$ и~отношение 
\textit{независимости} $I\hm\subseteq A^2$, состоящее из пар операций 
$(a_i,a_j)$, которые могут выполняться одновременно. Каждое слово можно 
интерпретировать как процесс, состоящий из команд, принадлежащих этому 
слову. Если команды могут выполняться одновременно, то их можно 
переставлять между собой. Два слова, составленные из букв алфавита~$A$, 
определим как \textit{эквивалентные}, если одно из них можно получить из 
другого с~по\-мощью последовательности перестановок рядом стоящих 
независимых букв. \textit{Трассой} называется класс эквивалентности 
множества~$A^*$ всех слов по этому отношению эквивалентности. Для 
произвольного слова $w\hm\in A^*$ обозначим через~$[w]$ его класс 
эквивалентности. Определим \textit{композицию} по формуле 
$[w_1][w_2]\hm=[w_1 w_2]$. Операция композиции превращает множество 
классов эк\-ви\-ва\-лент\-ности в~моноид, который называется \textit{моноидом 
трасс}. Идея алгоритма вычисления времени работы параллельного 
процесса описана в~[15]. Пусть время выполнения каждой команды, 
принадлежащей алфавиту~$A$, равно одному такту. Каждая трасса может 
быть представлена в~виде последовательности максимальных блоков 
(ярусов) параллельно выполняющихся команд. Это представление 
называется \textit{нормальной формой Фоаты}. Число блоков называется 
\textit{высотой} нормальной формы, и~оно равно времени выполнения 
трассы.
  
   Аналогично нормальной форме Фоаты введем \textit{нормальную форму 
относительно числа} $q\hm\geq 1$ как состоящую из последовательности 
блоков, длины которых не превышают число~$q$. Для этой цели 
модифицируем алгоритм приведения к~нормальной форме Фоаты и~за 
определение возьмем результат этого алгоритма.
   
   Опишем алгоритм. Пусть на входе задано некоторое непустое слово 
$w\hm\in A^*$. Считываем из него первый символ и~нормальную форму 
полагаем равной одному блоку, состоящему из этого символа. Далее 
в~цикле считываем очередной символ~$x$  и~для него 
выполняем~3~действия:
   \begin{enumerate}[(1)]
\item ищем блок с~наименьшим номером $k\hm\geq1$, такой что все 
элементы блоков, имеющих номера $\geq k$, независимы от~$x$. Если 
таких~$k$ нет, то добавляем новый блок, содержащий единственный 
элемент~$x$, и~переходим к~следующему символу;
\item цикл: пока $k$-й блок имеет~$q$~элементов, увеличиваем~$k$ 
на~1;
\item если~$k$ остался не больше номера последнего блока, то 
добавляем~$x$ в~$k$-й блок. Иначе до\-бав\-ля\-ем новый блок, состоящий из 
элемента~$x$. 
  \end{enumerate}
  
    \setcounter{table}{1}
    \begin{table*}[b]\small %tabl2
  \begin{center}
  \Caption{Работа асинхронного ограниченного конвейера}
  \vspace*{2ex}
  
  \begin{tabular}{|c|c|c|c|c|c|c|c|c|c|c|c|c|c|c|}
  \hline
&01 &02&03&04&05&06&07&08&09&10&11&12&13&14\\
\hline
1&1&2&3&&&&&&&&&&&\\
2&&1&1&1&2&2&2&3&3&3&&&&\\
3&&&&&1&&&2&&&3&&&\\
4&&&&&&1&1&&2&2&&3&3&\\
\hline
\end{tabular}
\end{center}
\end{table*}
     
  
  Число слов, из которых состоит нормальная форма относительно~$q$, 
называется ее \textit{высотой относительно}~$q$. Если каждая буква 
обозначает операцию, время выполнения которой равно единице измерения, 
и~независимые операции могут выполняться параллельно, то время 
выполнения операций слова~$w$ будет равно высоте нормальной формы 
этого слова относительно~$q$. 
  
  Для того чтобы находить время выполнения процесса обработки 
асинхронным ограниченным конвейером, зададим алфавит $A\hm= \{a_0, 
\ldots, a_{m-1}\}$, состоящий из $m\hm=3p$ букв. Слово~$w$, 
соответствующее обработке~$n$~элементов конвейером, задается 
следующим образом. Всякая ступень с~номером~$i$ разбивается на 
полутакты и~записывается как слово $a_{3i}a_{3i+1}^{2\tau_i-2} a_{3i+2}$. 
Здесь~$a_{3i}$~--- начальный полутакт,\linebreak\vspace*{-12pt}

{ \begin{center}  %fig1
 \vspace*{-1pt}
    \mbox{%
 \epsfxsize=79.105mm 
 \epsfbox{hus-1.eps}
 }


\vspace*{6pt}


\noindent
{\small Сравнение асинхронной и~непрерывной обработки}
\end{center}
}


\vspace*{12pt}

%\addtocounter{figure}{1}



\noindent
 а~$a_{3i+2}$~--- последний 
полутакт ступени. Между ними находятся полутакты, выполняющиеся 
последовательно, и~их можно обозначить одинаковой буквой~$a_{3i+1}$. 
Слово~$w$ будет равно 
  $$
  \left( a_0 a_1^{2\tau_0-2} a_2 a_3 a_4^{2\tau_1-2} a_5\cdots a_{3(p-1)} 
  a_{3p-2}^{2\tau_{p-1}-2} a_{3p-1}\right)^n\,.
  $$
Отношение независимости состоит из пар полутактов, которые могут 
выполняться параллельно: 
\begin{multline*}
I=A^2\backslash \left( \left\{ a_0, a_1, a_2\right\}^2 \cup
\left\{ a_3, a_4, a_5\right\}^2 \cup\cdots \right.\\
\cdots \cup \left\{ a_{3p-3},
a_{3p-2}, s_{3p-1}\right\}^2\cup{}\\
\left.{}\cup \left\{ a_2, a_3\right\}^2 \cup
\left\{ a_5, a_6\right\}^2\cup \cdots \cup \left\{ a_{3p-4}, a_{3p-
3}\right\}^2\right)\,.
\end{multline*}
  
  Нормальная форма относительно~$q$ дает последовательность 
параллельно работающих блоков (ярусов) операций. Поскольку время 
выполнения операции равно полутакту, то время обработки~$n$~элементов 
с~по\-мощью ограниченного конвейера будет равно половине высоты этой 
нормальной формы.
  
  Рассмотрим пример работы программы. На рисунке крупными точками 
показан график зависимости времени обработки трех входных элементов 
с~по\-мощью асинхронного конвейера, ограниченного числом 
процессоров~$X$. Задержки ступеней равны $\tau_0\hm=1$, $\tau_1\hm=3$, 
$\tau_2\hm=1$, $\tau_3\hm=2$. График, полученный по формуле~(\ref{e2-h}), 
изображен в~виде ломаной линии. По формуле~(\ref{e2-h}) время 
$T_2(3)\hm=14$. Крупная точка $(2,13)$ показывает, что для асинхронного 
конвейера это время равно~13. Приведенный пример показывает также, что 
для асинхронного конвейера следствие~1 неверно, поскольку наименьшее 
целое~$q$, удовлетворяющее $q\hm\geq \sigma/\mu$, равно~3, 
а~минимальное число процессоров, при котором достигается наибольшая 
производительность, равно~2.
  


  Таблица~2 представляет собой таблицу занятости для этого примера.
  
\vspace*{-6pt}

\section{Заключение}

  Получена формула для расчета производительности ограниченного 
конвейера (теорема~1). Вытекающее из нее следствие~1 говорит о том, что\linebreak 
для некоторых конвейеров число функциональных устройств можно 
уменьшить и~это не приведет к~снижению производительности. Это \mbox{можно} 
применять для совершенствования архитектуры процес\-со\-ра. Другим 
возможным продолжением данной работы может стать развитие метода, 
подсказавшего формулировку теоремы~1. Он позволяет обобщать 
некоторые утверждения об однородных конвейерах на неоднородные. 
  
  В последнее время конвейеры широко применяются при разработке 
многопоточных приложений для облачных вычислений. Многопоточный 
конвейер, работающий на компьютере с~многоядерным процессором, 
ограничен числом процессорных ядер. В~этом случае работа каждого 
процессорного ядра сопровождается переключением контекста нити, что 
приводит к~замедлению работы конвейера. Аналогичная проблема 
возникает при реализации ограниченных конвейеров для многоядерных 
сигнальных процессоров и~вы\-чис\-ли\-тель\-ных сис\-тем с~массовым 
параллелизмом~--- появляются дополнительные накладные расходы, 
связанные с~общим управ\-ле\-ни\-ем ядрами в~рамках одного процессора. 
Полученная в~работе формула для расчета производительности не 
учитывает такого замедления. Построение аналитической модели, 
учитывающей эти накладные расходы, было бы одним из перспективных 
продолжений данной работы.

\vspace*{-6pt} 
    
{\small\frenchspacing
 {%\baselineskip=10.8pt
 \addcontentsline{toc}{section}{References}
 \begin{thebibliography}{99}
\bibitem{1-h}
\Au{Хусаинов А.\,А., Чернов~А.\,М., Маевская~Е.\,Д., Романченко~А.\,А.} Модели для 
расчета времени работы вычислительных конвейеров~// Актуальные проблемы 
науки: Мат-лы XXIII Междунар. научно-практич. конф.~--- М.: Спутник+, 2016. 
С.~83--91. 
\bibitem{2-h}
\Au{Emma P.\,G., Davidson~E.\,S.} Characterization of branch and data dependencies in 
programs for evaluating pipeline performance~// IEEE~T. Comput., 1987. Vol.~7. 
P.~859--875.
\bibitem{3-h}
\Au{Cheah H.\,Y., Fahmy~S.\,A., Kapre~N.} On data forwarding in deeply pipelined soft 
processors~// ACM/SIGDA Symposium (International) on Field-Programmable Gate 
Arrays Proceedings.~--- New York, NY, USA: ACM, 2015. P.~181--189.
\bibitem{4-h}
\Au{Merchant F., Chattopadhyay~A., Raha~S., Nandy~S.\,K., Narayan~R.} 
Accelerating 
BLAS and LAPACK via efficient floating point architecture design~// Parallel Process. 
Lett., 2017. Vol.~27. No.\,03n04. P.~1--7.
\bibitem{5-h}
\Au{Паттерсон Д., Хеннесси~Дж.} Архитектура компьютера и~проектирование 
компьютерных систем~/ Пер. с~англ.~--- СПб.: Питер, 2012. 784~с. 
(\Au{Patterson~D.\,A., Hennessy~J.\,L.} Computer organization and design.~---
4th ed.~--- 
Amsterdam: Elsevier, 2012. 703~p.)
\bibitem{6-h}
\Au{Hartstein A., Puzak~T.\,R.} The optimum pipeline depth for a~microprocessor~// ACM 
 Comp. Ar., 2002. Vol.~30. Iss.~2. P.~7--13. 
\bibitem{7-h}
\Au{Yao J., Miwa~S., Shimada~H.} Optimal pipeline depth with pipeline stage unification 
adoption~// ACM Comp. Ar., 2007. Vol.~35. Iss.~5. P.~3--9.
\bibitem{8-h}
\Au{Moreno A., C$\acute{\mbox{e}}$sar~E., Guevara~A., Sorribes~J., Margalef~T.} Load 
balancing in homogeneous pipeline based applications~// Parallel Comput., 2012. Vol.~38. 
Iss.~3. P.~125--139.
\bibitem{9-h}
\Au{Moreno A., Sikora~A., C$\acute{\mbox{e}}$sar~E., Sorribes~J., Margalef~T.} 
HeDPM: Load balancing of linear pipeline applications on heterogeneous systems~// 
J.~Supercomput., 2017. Vol.~73. Iss.~9. P.~3738--3760.
\bibitem{10-h}
\Au{Хусаинов А.\,А., Титова~Е.\,А.} Оптимальная глубина вычислительного 
конвейера при заданном объеме данных~// Вычислительные технологии, 2018. Т.~23. 
№\,1. С.~96--104.
\bibitem{11-h}
\Au{Amdahl G.\,M.} Validity of the single processor approach to achieving large scale 
computing capabilities~// AFIPS Spring Joint Computer Conference Proceedings.~--- New 
York, NY, USA: ACM, 1967. P.~483--485.
\bibitem{12-h}
\Au{Shen J.\,P., Lipasti~M.\,H.} Model processor design: Fundamental of superscalar 
processors.~--- New York, NY, USA: McGraw-Hill, 2005. 643~p.
\bibitem{13-h}
\Au{Коуги П.\,М.} Архитектура конвейерных ЭВМ~/ Пер. с~англ.~--- М.: Радио 
и~связь, 1985. 360~с. (\Au{Kogge~P.\,M.} The architecture of pipelined computers.~--- 
Washington, D.C., USA: McGraw-Hill, 1981. 335~p.)
\bibitem{14-h}
\Au{Husainov A.\,A.} Optimum depth of the bounded pipeline.~--- New York, 
NY, USA: Cornell University, 2018.  Preprint.
11~p. {\sf http://arxiv.org/abs/cs.DC/1807.11022v1}.
\bibitem{15-h}
\Au{Diekert V.} Combinatorics on traces.~--- Lecture notes in computer science ser.~--- 
Berlin: Springer-Verlag, 1990. Vol.~454. 169~p.
 \end{thebibliography}

 }
 }

\end{multicols}

\vspace*{-3pt}

\hfill{\small\textit{Поступила в~редакцию 30.08.19}}

\vspace*{8pt}

%\pagebreak

%\newpage

%\vspace*{-28pt}

\hrule

\vspace*{2pt}

\hrule

%\vspace*{-2pt}

\def\tit{PERFORMANCE OF~THE~BOUNDED PIPELINE}


\def\titkol{Performance of~the~bounded pipeline}

\def\aut{A.\,A.~Khusainov}

\def\autkol{A.\,A.~Khusainov}

\titel{\tit}{\aut}{\autkol}{\titkol}

\vspace*{-11pt}


\noindent
Komsomolsk-na-Amure State University, 27~Lenina Prosp.,  
Komsomolsk-on-Amur, Khabarovsk Region 681013, Russian Federation

\def\leftfootline{\small{\textbf{\thepage}
\hfill INFORMATIKA I EE PRIMENENIYA~--- INFORMATICS AND
APPLICATIONS\ \ \ 2020\ \ \ volume~14\ \ \ issue\ 1}
}%
 \def\rightfootline{\small{INFORMATIKA I EE PRIMENENIYA~---
INFORMATICS AND APPLICATIONS\ \ \ 2020\ \ \ volume~14\ \ \ issue\ 1
\hfill \textbf{\thepage}}}

\vspace*{3pt} 



  \Abste{The paper is devoted to studying the performance of a bounded 
pipeline that is a computational pipeline, the number of active stages of which is 
bounded at any time by a fixed number. The bounded pipelines with the given 
sum and the maximum of delays of stages are considered. The stages can have 
different delays. The main problem is to build an analytical model for calculating 
the processing time of a given amount of data using this bounded pipeline. The 
solution is simplified if the constraint is treated as a structural pipeline hazard. 
This analytical model
is constructed for the case when the operation of a bounded 
pipeline has the property of continuity of processing for each input element. 
For 
such pipelines, the conjecture is proved in the paper that the minimum number 
of 
processors at which the greatest productivity is achieved 
is equal to the smallest 
integer not less than the ratio of the sum of stage delays to the 
maximum delay. It 
is established that if the property of continuity is not required,
then this conjecture 
is not true. The constructed model can be used to synchronize the operation of the 
stages of a bounded pipeline with the continuity property. If we do not require the 
property of continuity, then we get an asynchronous bounded pipeline, the 
synchronization of the work for the stages is carried out on the basis of the data 
readiness. The software is developed, which is based on the theory of trace 
monoids and allows one to calculate the processing time with an asynchronous 
bounded pipeline.}
  
  \KWE{computational pipeline; trace monoid; Foata normal form; pipeline 
performance; structural hazard}


\DOI{10.14357/19922264200112} 


\pagebreak

%\vspace*{-14pt}

%\Ack
%\noindent

 


%\vspace*{6pt}

  \begin{multicols}{2}

\renewcommand{\bibname}{\protect\rmfamily References}
%\renewcommand{\bibname}{\large\protect\rm References}

{\small\frenchspacing
 {%\baselineskip=10.8pt
 \addcontentsline{toc}{section}{References}
 \begin{thebibliography}{99}
\bibitem{1-h-1}
\Aue{Khusainov, A.\,A., A.\,M.~Chernov, E.\,D.~Mayevskaya, and 
A.\,A.~Romanchenko.} 2016. Modeli dlya rascheta vremeni raboty 
vychislitel'nykh konveyerov [Models for calculating the operating time of 
computational pipelines]. \textit{Aktual'nyye problemy nauki: Mat-ly XXIII 
Mezhdunar. nauchno-praktich. konf.}
[23rd Conference (International) on Actual Problems of Science Proceedings].
83--91.
\bibitem{2-h-1}
\Aue{Emma, P.\,G., and E.\,S.~Davidson.} 1987. Characterization of branch and 
data dependencies in programs for evaluating pipeline performance. \textit{IEEE~T.
 Comput.} 7:859--875.
\bibitem{3-h-1}
\Aue{Cheah, H.\,Y., S.\,A.~Fahmy, and N.~Kapre.} 2015. On data forwarding in 
deeply pipelined soft processors. \textit{ACM/SIGDA International Symposium 
on Field-Programmable Gate Arrays Proceedings}. New York, NY: ACM.  
181--189.
\bibitem{4-h-1}
\Aue{Merchant, F., A.~Chattopadhyay, S.~Raha, S.\,K.~Nandy, and 
R.~Narayan.} 2017. Accelerating BLAS and LAPACK via efficient floating 
point architecture design. \textit{Parallel Process. Lett.} 27(03n04):1--7.
\bibitem{5-h-1}
\Aue{Patterson, D.\,A., and J.\,L.~Hennessy.} 2012. Computer organization and 
design: The hardware/software interface. 4th ed. Amsterdam: Elsevier. 703~p.
\bibitem{6-h-1}
\Aue{Hartstein, A., and T.\,R.~Puzak.} 2002. The optimum pipeline depth for 
a~microprocessor. \textit{ACM Comp. Ar.}. 30(2):7--13. 
\bibitem{7-h-1}
\Aue{Yao, J., S.~Miwa, and H.~Shimada.} 2007. Optimal pipeline depth with 
pipeline stage unification adoption. \textit{ACM Comput. 
Ar.} 35(5):3--9.
\bibitem{8-h-1}
\Aue{Moreno, A., E.~C$\acute{\mbox{e}}$sar, A.~Guevara, J.~Sorribes, and 
T.~Margalef.} 2012. Load balancing in homogeneous pipelinebased applications. 
\textit{Parallel Comput.} 38(3):125--139.
\bibitem{9-h-1}
\Aue{Moreno, A., A.~Sikora, E.~C$\acute{\mbox{e}}$sar, J.~Sorribes, and 
T.~Margalef.} 2017. HeDPM: Load balancing of linear pipeline applications on 
heterogeneous systems. \textit{J.~Supercomput.} 73(9):3738--3760.
\bibitem{10-h-1}
\Aue{Husainov, A.\,A., and E.\,A.~Titova.} 2018. Optimal'naya glubina 
vychislitel'nogo konveyera pri zadannom ob''eme dannykh [Optimal depth of 
the computational pipeline for a~given amount of input data]. 
\textit{Vychislitel'nyye tekhnologii} [Computational Technologies]  
23(1):96--104.
\bibitem{11-h-1}
\Aue{Amdahl, G.\,M.} 1967. Validity of the single processor approach to 
achieving large scale computing capabilities. \textit{AFIPS Spring Joint 
Computer Conference Proceedings}. New York, NY: ACM. 483--485.
\bibitem{12-h-1}
\Aue{Shen, J.\,P., and M.\,H.~Lipasti.} 2005. \textit{Model processor design: 
Fundamental of superscalar processors.} New York, NY: McGraw-Hill. 643~p.
\bibitem{13-h-1}
\Aue{Kogge, P.\,M.} 1981. \textit{The architecture of pipelined computers.} 
Washington, D.C.: McGraw-Hill. 335~p.
\bibitem{14-h-1}
\Aue{Husainov, A.\,A.} 2018. Optimum depth of the bounded pipeline. 
{arXiv} 1807.11022 v1[cs.DC]. Available at: {\sf 
http://arxiv.org/abs/cs.DC/1807.11022v1} (accessed December~27, 2019).
\bibitem{15-h-1}
\Aue{Diekert, V.} 1990. \textit{Combinatorics on traces}. Lecture notes in 
computer science ser. Berlin: Springer-Verlag. Vol.~454. 169~p.
\end{thebibliography}

 }
 }

\end{multicols}

%\vspace*{-7pt}

\hfill{\small\textit{Received August 30, 2019}}

%\pagebreak

%\vspace*{-22pt}

\Contrl

\noindent
\textbf{Khusainov Akhmet A.} (b.\ 1951)~--- Doctor of Science in physics and 
mathematics, professor, Komsomolsk-na-Amure State University, 27~Lenina 
Prosp., Komsomolsk-on-Amur, Khabarovsk Region 681013, Russian 
Federation; \mbox{husainov51@yandex.ru}




\label{end\stat}

\renewcommand{\bibname}{\protect\rm Литература}     %12  
\def\stat{farh}

\def\tit{МЕТОД ЗАДАНИЯ КОНЕЧНЫХ НЕКОММУТАТИВНЫХ АССОЦИАТИВНЫХ АЛГЕБР 
ПРОИЗВОЛЬНОЙ ЧЕТНОЙ РАЗМЕРНОСТИ ДЛЯ~ПОСТРОЕНИЯ ПОСТКВАНТОВЫХ 
КРИПТОСХЕМ}

\def\titkol{Метод задания КНАА %конечных некоммутативных ассоциативных алгебр 
произвольной четной размерности для~построения постквантовых 
криптосхем}

\def\aut{А.\,А.~Костина$^1$, А.\,Ю.~Мирин$^2$, Д.\,Н.~Молдовян$^3$, 
Р.\,Ш.~Фахрутдинов$^4$}

\def\autkol{А.\,А.~Костина, А.\,Ю.~Мирин, Д.\,Н.~Молдовян, 
Р.\,Ш.~Фахрутдинов}

\titel{\tit}{\aut}{\autkol}{\titkol}

\index{Костина А.\,А.}
\index{Мирин А.\,Ю.}
\index{Молдовян Д.\,Н.} 
\index{Фахрутдинов Р.\,Ш.}
\index{Kostina A.\,A.}
\index{Mirin A.\,Yu.}
\index{Moldovyan D.\,N.}
\index{Fahrutdinov R.\,Sh.}


%{\renewcommand{\thefootnote}{\fnsymbol{footnote}} \footnotetext[1]
%{Работа выполнена при финансовой поддержке Российского научного фонда (проект 18-11-00155).}}


\renewcommand{\thefootnote}{\arabic{footnote}}
\footnotetext[1]{Санкт-Петербургский институт информатики 
и~автоматизации Российской академии наук, \mbox{anna-kostina1805@mail.ru}}
\footnotetext[2]{Санкт-Петербургский институт информатики и~автоматизации Российской 
      академии наук, \mbox{mirin@cobra.ru}}
\footnotetext[3]{Санкт-Петербургский институт информатики и~автоматизации Российской академии наук, 
\mbox{mdn.spectr@mail.ru}}
\footnotetext[4]{Санкт-Петербургский институт информатики и~автоматизации Российской 
академии наук, \mbox{fahr@cobra.ru}}

%\vspace*{-12pt}
      
     
  
  \Abst{Представлен новый унифицированный метод задания конечных некоммутативных 
ассоциативных алгебр (КНАА) произвольной четной размерности~$m$ и~описаны исследуемые 
свойства алгебр для случаев $m \hm= 4$ и~$6$ при задании алгебр над конечным 
простым полем $GF(p)$ с~большим размером простого числа~$p$. Получены формулы, 
описывающие множество~$p^2$ ($p^4$) глобальных левосторонних единиц, содержащихся 
в~4-мер\-ной (6-мер\-ной) алгебре. В~исследованных алгебрах имеет место только локальная 
обратимость. Для каждой из алгебр выведены формулы для вычисления единственного 
локального двустороннего элемента, связанного с~фиксированным локально обратимым 
вектором. Новая форма скрытой задачи дискретного логарифмирования  (СЗДЛ)
предложена 
в~качестве постквантового криптографического примитива и~использована для разработки 
постквантовой схемы электронной цифровой подписи (ЭЦП).}
  
  \KW{конечная некоммутативная алгебра; ассоциативная алгебра; вычислительно трудная 
задача; дискретный логарифм; цифровая подпись; постквантовая криптография}

\DOI{10.14357/19922264200113} 
  
%\vspace*{-3pt}


\vskip 10pt plus 9pt minus 6pt

\thispagestyle{headings}

\begin{multicols}{2}

\label{st\stat}

\section{Введение}

  Актуальным текущим вызовом в~области крип\-то\-гра\-фии стала разработка 
криптосхем с~открытым ключом, пригодных для принятия на их основе 
постквантовых криптографических стандартов~[1, 2] взамен текущих 
стандартов, основанных на вычислительной трудности задачи дискретного 
логарифмирования (ЗДЛ), которая решается на пока еще гипотетическом 
квантовом компьютере за полиномиальное время~[3]. Перспективным 
на\-прав\-ле\-ни\-ем решения указанной проб\-ле\-мы пред\-став\-ля\-ет\-ся использование 
вычислительной трудности СЗДЛ, за\-да\-ва\-емой в~КНАА~[4]. Для реализации 
потенциала СЗДЛ как постквантового криптографического примитива важное 
значение имеют задачи поиска и~исследования новых КНАА как носителей 
СЗДЛ и~новых форм последней, для которых СЗДЛ не будет сводиться к~ЗДЛ 
в~конечном поле~[5]. 
  
  В настоящей работе предлагается общий способ задания КНАА 
  произвольной четной 
размерности $m\hm> 2$ и~исследуются свойства 4- и~6-мер\-ных 
КНАА, построенных в~соответствии с~предложенным общим способом. 
Характерным свойством рас\-смот\-рен\-ных алгебр является наличие большого 
множества глобальных левосторонних единиц, и~для реализации на их основе 
постквантовых криптосхем предлагается новая форма СЗДЛ, использующая 
указанную особенность примененных в~качестве ее алгебраического носителя 
КНАА.
  
\section{Задача дискретного логарифмирования в~скрытой~группе}

  Известно автоморфное отображение некоммутативной группы~$\Gamma$, 
задаваемое следующей формулой:
  $$
  \varphi_V(G) =V^{-1}\circ G\circ V\,,
  $$
  где $V$~--- фиксированный элемент группы~$\Gamma$; $G$~--- элемент, 
пробегающий все значения в~группе~$\Gamma$. Элементы~$G$ и~$Y\hm=V^{-
1}\circ G\circ V$ называются элементами, сопряженными через элемент~$V$. 
Для фиксированного значения~$G$ операция автоморфного отоб\-ра\-же\-ния 
и~операция возведения в~степень являются перестановочными (взаимно 
коммутативными), т.\,е.\ имеет место соотношение: 
  $$
  \left( V^{-1}\circ G\circ V\right)^x=V^{-1}\circ (G^x)\circ V\,.
  $$
  %
  Это свойство может быть использовано для формулирования следующей 
вычислительно трудной задачи, пригодной для использования в~качестве 
примитива криптосистем с~открытым ключом. 
  
  Пусть задан некоторый элемент~$G$. Из некоторой коммутативной 
подгруппы~$\Gamma_{\mathrm{комм}}\subset \Gamma$ вы\-бираются элемент~$X$ 
и~произвольное число~$x$ и~вы\-чис\-ля\-ет\-ся элемент $Y\hm=X^{-1}\circ (G^x)\circ 
X$. По \mbox{заданным}~$Y$ и~$G$ требуется вычислить~$X$ и~$x$. Поскольку 
вычислить эти два неизвестных элемента по отдельности нельзя, то эта задача 
в~общем случае не сводится к~задаче дискретного логарифмирования 
в~циклической подгруппе. Нахождение неизвестных~$X$ и~$x$ по 
значениям~$Y$ и~$G$ представляет собой самостоятельную трудную 
вычислительную задачу, отличную от задачи дискретного логарифмирования. 
При известном значении~$X$ можно вычислить $Y^\prime \hm=X\circ Y\circ 
X^{-1}$ или $G^\prime\hm=X^{-1} \circ G\circ X$, после чего число~$x$ можно 
найти из уравнения $Y^\prime\hm=G^x$ или из уравнения  $Y\hm=G^{\prime x}$ 
соответственно, т.\,е.\ решая задачу дискретного логарифмирования. Однако 
значение~$X$ остается неизвестным, поэтому ЗДЛ в~явном виде не стоит. 
Криптосхемы на основе СЗДЛ, заданной в~этой форме, описаны 
в~работе~\cite{4-f}.
  
\section{Конечные некоммутативные ассоциативные алгебры}

  \subsection{Алгебры как~векторные пространства с~дополнительной 
операцией умножения векторов}
  
  Рассмотрим $m$-мер\-ное векторное пространство, элементами которого 
выступают всевозможные векторы вида
  \begin{multline*}
  A= \left(a_0, a_1, \ldots , a_1\right)= {}\\
  {}=\left(a_0\mathbf{e}_0\hm+ 
a_1\mathbf{e}_1+\cdots + a_{m-1}\mathbf{e}_{m-1}\right),
\end{multline*}
 где $a_i\hm\in GF(p)$, 
$p$~--- простое число; $\mathbf{e}_i$~--- формальные базисные векторы. 
Дополнительно к~стандартным операциям в~векторном пространстве~--- 
операции сложения векторов и~операции умножения вектора на скаляр~--- 
определим операцию умножения ($\circ$) векторов $A\hm= 
\sum\nolimits_{i=0}^{m-1} a_i\mathbf{e}_i$ и~$B\hm= 
  \sum\nolimits_{j=0}^{m-1} b_j\mathbf{e}_j$ в~соответствии со следующей 
формулой:  
  \begin{equation}
  A\circ B= \sum\limits_{i=0}^{m-1} \sum\limits_{j=0}^{m-1} a_i b_j 
\mathbf{e}_i\circ \mathbf{e}_j\,,
  \label{e1-f}
  \end{equation}
где каждое из всевозможных произведений пар базисных векторов заменяется 
на однокомпонентный вектор в~соответствии с~некоторым правилом, 
задаваемым в~виде таблицы умножения базисных векторов (ТУБВ). Векторное 
пространство с~определенной таким образом операцией умножения векторов 
называется $m$-мер\-ной алгеброй. Для построения криптосхем на основе 
СЗДЛ интерес представляют КНАА.

\subsection{Общий способ задания конечных некоммутативных ассоциативных
алгебр произвольной четной 
размерности $m\hm\geq4$}

  В качестве общего способа задания конечной ассоциативной алгебры четной 
размерности $m\hm>1$ предлагается задать значение произведения 
$\mathbf{e}_i\circ \mathbf{e}_j$ в~формуле~(1) в~соответствии со следующим 
выражением:
  \begin{equation}
  \mathbf{e}_i\circ\mathbf{e}_j=\begin{cases}
  \mathbf{e}_j\,, &\ \mbox{если } i\,\mathrm{mod}\,2=0\,;\\
  \mathbf{e}_{m-1-j}\,, &\ \mbox{если } i\,\mathrm{mod}\,2=1\,.
  \end{cases}
  \label{e2-f}
  \end{equation}
  
  Докажем, что правило~(2) задает ассоциативное умножение векторов. 
Рассмотрим произведение векторов~$A$, $B$ и~$C\hm= 
\sum\nolimits_{k=0}^{m-1} c_k\mathbf{e}_k$, осуществляемое в~соответствии 
со следующими двумя вариантами: 
  \begin{align}
(A\circ B)\circ C &=\notag{}\\
  &\hspace*{-15mm}{}= \left( \sum\limits_{i=0}^{m-1} \sum\limits_{j=0}^{m-1} a_i 
b_j \mathbf{e}_i\circ \mathbf{e}_j\right) \circ \sum\limits_{k=0}^{m-1} 
c_k\mathbf{e}_k ={}\notag\\
&{}= \sum\limits_{i=0}^{m-1} \sum\limits_{j=0}^{m-1} 
\sum\limits_{k=0}^{m-1} a_i b_j c_k \left( \mathbf{e}_i\circ \mathbf{e}_j 
\right)\circ \mathbf{e}_k\,;
  \label{e3.1-f}
  \\
 A\circ( B\circ C) &={}\notag\\
  &\hspace*{-15mm}{}= \left( \sum\limits^{m-1}_{i=0} a_i \mathbf{e}_i\right) \circ 
\left( \sum\limits^{m-1}_{j=0} \sum\limits^{m-1}_{k=0} b_j c_k \mathbf{e}_i\circ 
\mathbf{e}_j\right) ={}\notag\\
&{}=\sum\limits_{i=0}^{m-1} \sum\limits_{j=0}^{m-1} 
\sum\limits^{m-1}_{k=0} a_i b_j c_k \mathbf{e}_i\circ \left( \mathbf{e}_j\circ 
\mathbf{e}_k \right)\,.
  \label{e3.2-f}
  \end{align}
  
  Равенство правых частей выражений~(\ref{e3.1-f}) и~(\ref{e3.2-f}) имеет 
место, если равенство
  \begin{equation}
  \left(\mathbf{e}_i\circ\mathbf{e}_j\right)\circ\mathbf{e}_k=\mathbf{e}_i\circ\left
( \mathbf{e}_j\circ\mathbf{e}_k\right)
  \label{e4-f}
  \end{equation}
справедливо для всех возможных троек значений $(i, j, k)$. Справедливость 
равенства~(\ref{e4-f}) можно легко показать, рассматривая следующие четыре 
случая:
\begin{description}
\item[\,]  \textbf{Случай~1}: $i$ и~$j$~--- четные значения: 
  \begin{multline*}
  \left\{ \begin{matrix}
  \left( \mathbf{e}_i\circ \mathbf{e}_j\right) \circ\mathbf{e}_k = \mathbf{e}_j 
\circ \mathbf{e}_k =\mathbf{e}_k\\
  \mathbf{e}_i \circ \left(\mathbf{e}_j\circ \mathbf{e}_k\right) 
=\mathbf{e}_i\circ \mathbf{e}_k =\mathbf{e}_k
  \end{matrix}
  \right\} \Rightarrow {}\\
  {}\Rightarrow \left(\mathbf{e}_i\circ \mathbf{e}_j\right)\circ 
\mathbf{e}_k=\mathbf{e}_i\circ\left(\mathbf{e}_j\circ \mathbf{e}_k\right)\,.
\end{multline*}
\item[\,]  
  \textbf{Случай~2}: $i$~--- четное значение; $j$~--- нечетное значение: 
\begin{multline*}
  \left\{
  \begin{matrix}
  \left(\mathbf{e}_i\circ \mathbf{e}_j\right)  \circ\mathbf{e}_k 
=\mathbf{e}_j\circ\mathbf{e}_k =\mathbf{e}_{m-1-k}\\
  \mathbf{e}_i \circ\left(\mathbf{e}_j\circ \mathbf{e}_k\right) = 
\mathbf{e}_i\circ \mathbf{e}_{m-1-k} =\mathbf{e}_{m-1-k}
  \end{matrix}
  \right\} \Rightarrow{}\\
  {}\Rightarrow \left(\mathbf{e}_i\circ \mathbf{e}_j\right)\circ 
\mathbf{e}_k=\mathbf{e}_i\circ\left(\mathbf{e}_j\circ \mathbf{e}_k\right)\,.
\end{multline*}
  \item[\,]
  \textbf{Случай~3}: $i$~--- нечетное значение; $j$~--- четное значение: 
  \begin{multline*}
  \left\{ \begin{matrix}
  \left(\mathbf{e}_i\circ \mathbf{e}_j\right)\circ \mathbf{e}_k=\mathbf{e}_{m-
1-j}\circ \mathbf{e}_k =\mathbf{e}_{m-1-k}\\
  \mathbf{e}_i\circ \left(\mathbf{e}_j\circ \mathbf{e}_k\right) 
=\mathbf{e}_i\circ \mathbf{e}_k =\mathbf{e}_{m-1-k}
  \end{matrix}
  \right\} \Rightarrow {}\\
  {}\Rightarrow\left(\mathbf{e}_i\circ \mathbf{e}_j\right) \circ 
\mathbf{e}_k = \mathbf{e}_i\circ \left(\mathbf{e}_j\circ \mathbf{e}_k\right)\,.
  \end{multline*}
  \item[\,]
  \textbf{Случай~4}: $i$ и~$j$~--- нечетные значения: 
  \begin{multline*}
  \left\{ 
  \begin{matrix}
  \left(\mathbf{e}_i\circ \mathbf{e}_j\right) \circ \mathbf{e}_k=\mathbf{e}_{m-
1-j}\circ \mathbf{e}_k =\mathbf{e}_k\\
  \mathbf{e}_i\circ\left(\mathbf{e}_j\circ \mathbf{e}_k\right) =
  \mathbf{e}_i\circ 
\mathbf{e}_{m-1-k} ={}\\
\hspace*{22mm}{}= \mathbf{e}_{m-1-(m-1-k)}= \mathbf{e}_k
  \end{matrix}
  \right\} \Rightarrow{}\\
  {}\Rightarrow \left(\mathbf{e}_i\circ \mathbf{e}_j\right)\circ 
\mathbf{e}_k = \mathbf{e}_i\circ\left(\mathbf{e}_j\circ \mathbf{e}_k\right)\,.
 \end{multline*}
 \end{description}
  
  Таким образом, операция умножения векторов, задаваемая правилом~(2), 
ассоциативна, а при $m\hm\geq4$ также и~некоммутативна. 
  
  \subsection{Четырехмерная алгебра}
  
  Рассмотрим случай 4-мер\-ной КНАА, для которой найдено распределение 
структурного коэффициента $\mu\hm\in GF(p)$, представленное в~табл.~1.
  
  
  
  Нахождение левосторонних единиц 4-мер\-ной алгебры, задаваемой табл.~1, 
связано с~решением следующего векторного уравнения:
  \begin{equation}
  X\circ A=A\,,
  \label{e5-f}
  \end{equation}
где $A=(a_0, a_1, a_2, a_3)$~--- некоторый заданный вектор, для которого 
требуется найти левостороннюю
единицу; $X\hm= (x_0, x_1, x_2, x_3)$~--- 
неизвестный век-\linebreak\vspace*{-12pt}

\vspace*{9pt} %tabl1

\begin{center}
\noindent
\parbox{50mm}{{{\tablename~1}\ \ \small{Предлагаемая ТУБВ для случая $m = 4$}}}
%\vspace*{2ex}

\vspace*{8pt}

\tabcolsep=8pt
{\small
\begin{tabular}{|c|c|c|c|c|}
  \hline
\multicolumn{1}{|c|}{$\circ$}&\multicolumn{1}{c|}{$\mathbf{e}_0$}&$\mathbf{e}_1$&
$\mathbf{e}_2$&$\mathbf{e}_3$\\
\hline
$\mathbf{e}_0$&$\mu\mathbf{e}_0$&$\mu\mathbf{e}_1$&$\mu\mathbf{e}_2$&$\mu\mathbf{e}_3$\\
\hline
$\mathbf{e}_1$&$\mathbf{e}_3$&$\mathbf{e}_2$&$\mathbf{e}_1$&$\mathbf{e}_0$\\
\hline
$\mathbf{e}_2$&$\mathbf{e}_0$&$\mathbf{e}_1$&$\mathbf{e}_2$&$\mathbf{e}_3$\\
\hline
$\mathbf{e}_3$&$\mu\mathbf{e}_3$&$\mu\mathbf{e}_2$&$\mu\mathbf{e}_1$&$\mu\mathbf{e}_0$\\
\hline
\end{tabular}
}
\vspace*{2pt}
\end{center}

%\end{table*}





\noindent
тор. С~учетом табл.~1 данное векторное уравнение   сводится 
к~решению следующей системы линейных уравнений с~четырьмя 
неизвестными:

\vspace*{1pt}

\noindent
\begin{equation}
\left.
\begin{array}{l}
\mu x_0 a_0 +x_1a_3 +x_2a_0+\mu x_3a_3=a_0\,;\\[3pt]
\mu x_0 a_1+x_1a_2+x_2a_1+\mu x_3a_2=a_1\,;\\[3pt]
\mu x_0a_2+x_1a_1+x_2a_2+\mu x_3a_1=a_2\,;\\[3pt]
\mu x_0 a_3+x_1a_0 +x_2 a_3 +\mu x_3a_0=a_3\,.
\end{array}
\right\}
\label{e6-f}
\end{equation}

\vspace*{-1pt}

  Эта система распадается на следующие две независимые системы из двух 
уравнений:

\vspace*{1pt}

\noindent
  \begin{equation}
  \left.
  \begin{array}{l}
  \left\{ 
  \begin{array}{l}
  (\mu x_0+x_2)a_0+(x_1+\mu x_3)a_3=a_0\,;\\[3pt]
  (x_1+\mu x_3)a_0+(\mu x_0+x_2)a_3=a_3\,;
  \end{array}
  \right.\\[12pt]
  \left\{ \begin{array}{l}
  (\mu x_0+x_2)a_1+(x_1+\mu x_3)a_2=a_1\,;\\[3pt]
  (x_1+\mu x_3)a_1+(\mu x_0+x_2)a_2=a_2\,.
  \end{array}
  \right.
  \end{array}
  \right\}
  \label{e7-f}
  \end{equation}
  
  \vspace*{-1pt}
  
  Выполнив в~(\ref{e7-f}) замену переменных по формулам $z_1\hm= \mu 
x_0\hm+x_2$ и~$z_1\hm= z_2\hm+\mu x_3$, легко показать, что решения 
системы~(\ref{e6-f}) совпадают с~решениями следующей системы из двух 
уравнений с~четырьмя неизвестными~$x_0$, $x_1$, $x_2$ и~$x_3$:

\vspace*{1pt}

\noindent
  \begin{equation*}
  \left. 
  \begin{array}{l}
  \mu x_0+x_2=1\,;\\[3pt]
  x_1+\mu x_3=0\,.
  \end{array}
  \right\}
  %\label{e8-f}
  \end{equation*}
  
  \vspace*{-1pt}
  
  Поскольку решения системы~(\ref{e6-f}) не зависят от\linebreak координат 
вектора~$A$, это означает, что найденные решения описывают глобальные 
левосторонние единицы, т.\,е.\ левосторонние единицы, действующие на все 
элементы рассматриваемой \mbox{4-мер\-ной} КНАА. Множество всех~$p^2$ 
глобальных левосторонних единиц описывается следующей формулой:
  \begin{equation*}
  L=\left( l_0, l_1, l_2, l_3\right) =\left( x_0, x_1, 1-\mu x_0, -\mu^{-1}x_1\right)\,,
  %\label{e9-f}
  \end{equation*}
где $x_0, x_1=0, 1, \ldots , p-1$.
  
  Для нахождения правосторонних единиц вектора $A\hm= (a_0, a_1, a_2, a_3)$ 
рассмотрим следующее векторное уравнение:

\vspace*{1pt}

\noindent
  \begin{equation}
  A\circ X=A\,,
  \label{e10-f}
  \end{equation}
  
  \vspace*{-1pt}
  
  \noindent
которое сводится к~системе линейных уравнений вида
\begin{equation}
\left.
\begin{matrix}
\mu a_0 x_0+a_1x_3+a_2x_0+\mu a_3x_3=a_0\,;\\[2pt]
\mu a_0x_1+a_1x_2+a_2x_1+\mu a_3x_2 =a_1\,;\\[2pt]
\mu a_0 x_2+a_1x_1+a_2x_2+\mu a_3 x_1=a_2\,;\\[2pt]
\mu a_0x_3+a_1x_0+a_2x_3+\mu a_3x_0=a_3\,.
\end{matrix}
\right\}
\label{e11-f}
\end{equation}
  
  Если координаты вектора~$A$ удовлетворяют неравенству 
    \begin{equation*}
  \Delta = \left( \mu a_0+a_2\right)^2 -\left( a_1+\mu a_3\right)^2\not= 0\,,
 % \label{e12-f}
  \end{equation*}
то система~(\ref{e11-f}) имеет единственное решение:

\noindent
\begin{align*}
x_0&=r_0=\fr{a_0(\mu a_0+a_2)-a_3(a_1+\mu a_3)}{\Delta}\,;\\ 
x_1&=r_1=\fr{\mu(a_0a_1-a_2a_3)}{\Delta}\,;\\
x_2&=r_2=\fr{a_2(\mu a_0+a_2)-a_1(a_1+\mu a_3)}{\Delta}\,;\\ 
x_3&=r_3=\fr{a_2 a_3-a_0 a_1}{\Delta}\,,
\end{align*}
которое определяет существование единственной локальной правосторонней 
единицы $R_A\hm= (r_0, r_1, r_2, r_3)$, соответствующей вектору~$A$. 
Вектор~$R_A$ действует как правая единица на \mbox{некотором} подмножестве 
элементов рассматриваемой ал\-геб\-ры, поэтому она называется локальной. 

  \subsection{Шестимерная алгебра}
  
  Для случая 6-мер\-ной КНАА найдены распределения независимых 
структурных коэффициентов~$\mu, \lambda \hm\in GF(p)$, представленные 
в~табл.~2.

\vspace*{6pt} %tabl2

\begin{center}
\noindent
{{\tablename~2}\ \ \small{Предлагаемая ТУБВ для случая $m = 6$}}
%\vspace*{2ex}

\vspace*{6pt}

\tabcolsep=8pt
{\small
 \begin{tabular}{|c|c|c|c|c|c|c|}
  \hline
$\circ$&$\mathbf{e}_0$&$\mathbf{e}_1$&$\mathbf{e}_2$&$\mathbf{e}_3$&
$\mathbf{e}_4$&$\mathbf{e}_5$\\
\hline
$\mathbf{e}_0$&$\mu \mathbf{e}_0$&$\mu \mathbf{e}_1$&
$\mu \mathbf{e}_2$&$\mu \mathbf{e}_3$&$\mu \mathbf{e}_4$&$\mu \mathbf{e}_5$\\
\hline
$\mathbf{e}_1$&$\mathbf{e}_5$&$\mathbf{e}_4$&$\mathbf{e}_3$&$\mathbf{e}_2$&
$\mathbf{e}_1$&$\mathbf{e}_0$\\
\hline
$\mathbf{e}_2$&$\lambda \mathbf{e}_0$&$\lambda \mathbf{e}_1$&$\lambda 
\mathbf{e}_2$&$\lambda \mathbf{e}_3$&$\lambda\mathbf{e}_4$&$\lambda 
\mathbf{e}_5$\\
\hline
$\mathbf{e}_3$&$\lambda\mathbf{e}_5$&$\lambda\mathbf{e}_4$&$\lambda\mathbf{e}_3
$&$\lambda\mathbf{e}_2$&$\lambda\mathbf{e}_1$&$\lambda \mathbf{e}_0$\\
\hline
$\mathbf{e}_4$&$\mathbf{e}_0$&$\mathbf{e}_1$&$\mathbf{e}_2$&$\mathbf{e}_3$&
$\mathbf{e}_4$&$\mathbf{e}_5$\\
\hline
$\mathbf{e}_5$&$\mu \mathbf{e}_5$&$\mu\mathbf{e}_4$&$\mu 
\mathbf{e}_3$&$\mu\mathbf{e}_2$&$\mu \mathbf{e}_1$&$\mu \mathbf{e}_0$\\
\hline
\end{tabular}
}
%\vspace*{2pt}
\end{center}

\vspace*{3pt}

%\end{table*}
  

  
  
  Нахождение левосторонних единиц 6-мер\-ной КНАА, задаваемой табл.~2, 
по векторному уравнению~(\ref{e5-f}), в~котором $A\hm= (a_0, a_1, a_2, a_3, 
a_4, a_5)$ и~$X\hm= (x_0, x_1, x_2, x_3, x_4, x_5)$, приводит к~решению 
следующей системы из шести линейных уравнений с~неизвестными 
координатами вектора~$X$:
  \begin{equation}
  \left.
  \begin{array}{rl}
  \mu x_0a_0+x_1a_5+\lambda x_2a_0+\lambda x_3 a_5+{}&\\[1pt]
  &\hspace*{-20mm}{}+x_4a_0+\mu x_5a_5=a_0\,;\\[3pt]
  \mu x_0a_1+x_1 a_4+\lambda x_2a_1+\lambda x_3 a_4+{}&\\[1pt]
&  \hspace*{-20mm}{}+x_4a_1+\mu x_5a_4=a_1\,;\\[3pt]
  \mu x_0a_2+x_1 a_3+\lambda x_2a_2+\lambda x_3 a_3+{}&\\[1pt]
&\hspace*{-20mm}{}+  x_4a_2+\mu x_5a_3=a_2\,;\\[3pt]
  \mu x_0a_3+x_1 a_2+\lambda x_2a_3+\lambda x_3 a_2+{}&\\[1pt]
  &\hspace*{-20mm}{}+x_4a_3+\mu  x_5a_2=a_3\,;\\[3pt]
  \mu x_0a_4+x_1 a_1+\lambda x_2a_4+\lambda x_3 a_1+{}&\\[1pt]
  &\hspace*{-20mm}{}+x_4a_4+\mu x_5a_1=a_4\,;\\[3pt]
  \mu x_0a_5+x_1 a_0+\lambda x_2a_5+\lambda x_3 a_0+{}&\\[1pt]
  &\hspace*{-20mm}{}+x_4a_5+\mu  x_5a_0=a_5\,.
  \end{array}
  \right\}
  \label{e13-f}
  \end{equation}
  
  

  

  Выделим в~этой системе следующие три системы из двух уравнений:
  
  \noindent
  \begin{equation}
  \left.
  \begin{array}{l}
    \left\{
\begin{array}{l}
    (\mu x_0+\lambda x_2+x_4)a_0+{}\\[1pt]
    \hspace*{15mm}{}+(x_1+\lambda x_3+\mu x_5)a_5=a_0\,;\\[1pt]
  (x_1+\lambda x_3+\mu x_5)a_0 +{}\\[1pt]
   \hspace*{15mm}{}+(\mu x_0+\lambda x_2+x_4)a_5=a_5\,;
  \end{array}
  \right.\\[9pt]
  \left\{
  \begin{array}{l}
  (\mu x_0+\lambda x_2+x_4)a_1+{}\\[1pt]
  \hspace*{15mm}{}+(x_1+\lambda x_3+\mu x_5)a_4=a_1\,;\\[1pt]
 (x_1+\lambda x_3 +\mu x_5)a_1+{}\\[1pt]
  \hspace*{15mm}{}+(\mu x_0+\lambda x_2 +x_4)a_4=a_4\,;
  \end{array}
  \right.\\[9pt]
  \left\{ 
  \begin{array}{l}
(\mu x_0+\lambda x_2 +x_4)a_2+{}\\[1pt]
 \hspace*{15mm} {}+(x_1+\lambda x_3+\mu x_5)a_3=a_2\,;\\[1pt]
  (x_1+\lambda x_3+\mu x_5)a_2+{}\\[1pt]
   \hspace*{15mm}{}+(\mu x_0+\lambda x_2+x_4)a_3=a_3\,.
    \end{array}
  \right.
\end{array}
  \right\}
  \label{e14-f}
  \end{equation}
  
  Легко видеть, что решение системы~(\ref{e14-f}) можно найти, выполнив  
замену переменных по формулам $z_1\hm= \mu x_0\hm+ \lambda x_2\hm+ x_4$ 
и~$z_2\hm= x_1\hm+\lambda x_3\hm+\mu x_5$. После такой замены 
переменных каждая из трех подсистем системы~(\ref{e14-f}) включает два 
уравнения с~одинаковыми двумя неизвестными~$z_1$ и~$z_2$ и~приобретает 
вид:
  \begin{equation}
  \left.
  \begin{array}{l}
  \left\{
  \begin{array}{c}
  z_1a_0+z_2a_5=a_0\,;\\[3pt]
  z_1a_5+z_2a_0=a_5\,;
  \end{array}
  \right.\\[12pt]
  \left\{
  \begin{array}{c}
  z_1a_1+z_2a_4=a_1\,;\\[3pt]
  z_1a_4+z_2a_2=a_4\,;
  \end{array}
  \right.\\[12pt]
  \left\{
  \begin{array}{c}
  z_1a_2+z_2a_3=a_2\,;\\[3pt]
  z_1a_3+z_2a_2=a_3\,.
  \end{array}
  \right.
  \end{array}
  \right\}
  \label{e15-f}
  \end{equation}
                  
  
  Система~(\ref{e15-f}) имеет единственное решение в~виде пары значений 
$z_1\hm=1$ и~$z_2\hm=0$ для всех возможных значений вектора~$A$, кроме 
случая одновременного выполнения условий $a_0\hm=a_5$, $a_1\hm= a_4$ 
и~$a_2\hm= a_3$. В~последнем случае имеется множество дополнительных 
решений в~виде пар значений $z_1\hm\in GF(p)$ и~$z_2\hm= 1\hm -z_1$. Этот 
особый случай выпадает из множества значений векторов, используемых при 
построении криптосхем на основе рассматриваемой 6-мер\-ной конечной 
алгебры.
  
  Выполнение обратной замены переменных показывает, что исходная 
система~(\ref{e13-f}) имеет решения, совпадающие с~решениями следующей 
системы из двух линейных уравнений с~шестью неизвестными~$x_0$, $x_1$, 
$x_2$, $x_3$, $x_4$ и~$x_5$:
  \begin{equation}
  \left.
  \begin{matrix}
  \mu x_0+\lambda x_2+x_4=1\,;\\
  x_1+\lambda x_3+\mu x_5=0\,.
  \end{matrix}
  \right\}
  \label{e16-f}
  \end{equation}
  
  Решения системы~(\ref{e16-f}) не зависят от координат вектора~$A$, т.\,е.\ 
они описывают следующее множество~$p^4$ глобальных левосторонних 
единиц:
  \begin{multline}
  L=\left( l_0, l_1, l_2, l_3, l_4, l_5\right)={}\\
  \!\!\!=\left( x_0, x_1, x_2, x_3, 1\!-\!\mu x_0\!-\!
\lambda x_2, -\mu^{-1}(x_1\!+\!\lambda x_3)\right),\!\!\!\!
  \label{e17-f}
  \end{multline}
где $x_0, x_1, x_2, x_3 \hm=0, 1, \ldots ,  p-1$.
  
  Правосторонние единицы, соответствующие вектору $A\hm= (a_0, a_1, a_2, 
a_3, a_4, a_5)$, удовлетворяют векторному уравнению~(\ref{e10-f}), 
рассмотрение которого приводит к~системе уравнений, представимой
в~виде трех независимых систем из двух линейных уравнений с~двумя 
неизвестными:
\begin{equation}
\left.
\begin{array}{l}
\left\{
\begin{array}{l}
k_1x_0+k_2x_5=a_0\,;\\[3pt]
k_2x_0+k_1x_5=a_5\,;
\end{array}
\right.\\[12pt]
\left\{
\begin{array}{l}
k_1x_1+k_2x_4=a_1\,;\\[3pt]
k_2x_1+k_1x_4=a_4\,;
\end{array}
\right.\\[12pt]
\left\{
\begin{array}{l}
k_1x_2+k_2x_3=a_2\,;\\[6pt]
k_2x_2+k_1x_3=a_3\,,
\end{array}
\right.
\end{array}
\right\}
\label{e18-f}
\end{equation}
%     
где введены обозначения $k_1\hm= \mu a_0\hm+\lambda a_2\hm+a_4$ 
и~$k_2\hm= a_1\hm+\lambda a_3 \hm+\mu a_5$. В~каждой из трех независимых 
систем из двух уравнений главный определитель равен одному и~тому же 
значению: 
$$
\Delta =\left( \mu a_0+\lambda a_2+a_4\right)^2-\left( a_1+\lambda a_3+\mu 
a_5\right)^2\,.
$$
  
  При выполнении условия $\Delta\hm= k_1^2\hm- k_2^2\not= 0$ 
сис\-те\-ма~(\ref{e18-f}) имеет единственное решение:
\begin{equation}
\left.
\begin{array}{rl}
x_0&=r_0=\fr{a_0k_1-a_5k_2}{\Delta}\,;\\[6pt]
x_1&=r_1= \fr{a_1k_1-a_4k_2}{\Delta}\,;\\[6pt]
x_2&=r_2=\fr{a_2k_1-a_3k_2}{\Delta}\,;\\[6pt]
x_3&=r_3=\fr{a_3k_1-a_2k_2}{\Delta}\,;\\[6pt]
x_4&=r_4=\fr{a_4k_1-a_1k_2}{\Delta}\,;\\[6pt]
x_5&=r_5=\fr{a_5k_1-a_0k_2}{\Delta}\,,
\end{array}
\right\}
\label{e19-f}
\end{equation}
которое определяет существование единственной локальной правосторонней 
единицы $R_A\hm= (r_0, r_1, r_2, r_3, r_4, r_5)$, соответствующей вектору~$A$ 
и~всевозможным степеням последнего. 

  Подставляя значения $x_0\hm=r_0$, $x_1\hm=r_1$, $x_2\hm=r_2$, 
$x_3\hm=r_3$ из формул~(\ref{e19-f}) в~формулу~(\ref{e17-f}), описывающую 
множество глобальных левосторонних единиц, можно получить $x_4\hm= 
1\hm- \mu r_0 \hm- \lambda r_2\hm=r_4$ и~$x_5\hm= -\mu^{-1} (r_1\hm+ \lambda 
r_3)\hm=r_5$ Последнее означает, что локальная правосторонняя 
единица~$R_A$ содержится в~множестве глобальных левосторонних 
единиц~(\ref{e17-f}), т.\,е.\ она является локальной двухсторонней 
единицей~$E_A$ вектора~$A$. 
  
  Легко показать, что в~бесконечной последовательности $A, A^2, A^3, \ldots , 
A^i,\ldots$ отсутствует нулевой вектор и~при некоторой минимальной 
натуральной степени~$d$ имеет место $A^d\hm= A$. Следовательно, $A^d\hm= 
A\hm\Rightarrow A^{d-1}\circ A\hm= A\circ A^{d-1}$, т.\,е.\ вектор 
$E_A\hm=A^\omega$, где $\omega \hm= d\hm-1$, есть локальная двухсторонняя 
единица вектора~$A$ (значение~$\omega$ будем называть локальным 
порядком локально обратимого вектора~$A$). Множество $\{ A, A^2, A^3, 
\ldots , A^i, \ldots , A^\omega\}$ представляет собой циклическую 
мультипликативную группу с~единицей~$E_A$.
  
\section{Задание новой формы скрытой задачи дискретного логарифмирования
и~схема цифровой подписи  на~ее основе}

  Для построения алгоритмов ЭЦП 
предлагается новая форма СЗДЛ, которая отличается использованием 
открытого ключа в~виде трех элементов КНАА, принадлежащих разным 
циклическим группам, и~описывается следующим образом. 
  \begin{enumerate}[1.]
  \item В качестве характеристики поля берем простое число~$p$ достаточно 
большой разрядности (например, 384~бит). 
  \item Выбираем три случайных локально обратимых вектора~$A$, $B$ 
и~$N$, локальный порядок которых содержит достаточно большой простой 
делитель.
  \item  Выбираем две случайные глобальные левосторонние единицы~$L_1$ 
и~$L_2$.
  \item  Вычисляем вектор~$A^\prime$ из уравнения 
  \begin{equation}
  A\circ A^\prime =L_1\,.
  \label{e20-f}
  \end{equation}
  \item Вычисляем вектор~$B^\prime$ из уравнения 
  \begin{equation}
  B\circ B^\prime =L_2\,.
  \label{e21-f}
  \end{equation}
  \item Вычисляем векторы~$T$ и~$L_3$ из уравнения 
  \begin{equation}
  A\circ T\circ B^\prime =L_3\,.
  \label{e22-f}
  \end{equation}
  \item Выбираем случайное натуральное число $x\hm<\omega$, 
где~$\omega$~--- значение порядка вектора~$N$.
  \item Вычисляем векторы~$Y$ и~$U$ по формулам:
  $$
  Y=A^\prime \circ N^x\circ A\,;\enskip U=B^\prime\circ N\circ B\,.
  $$
  \end{enumerate}
  
  В силу локальной обратимости векторов~$A$ и~$B$ уравнения~(\ref{e20-f}) 
и~(\ref{e21-f}) имеют единственное решение (см.\ решение  
системы~(\ref{e18-f})). Уравнение~(\ref{e22-f}) решается в~два этапа. Сначала 
вычисляются векторы~$T^\prime$ и~$L_3$ как неизвестные в~уравнении 
$T^\prime\circ B^\prime \hm=L_3$ (решение является единственным), а~затем 
находится вектор~$T$ из уравнения $A\circ T\hm= T^\prime$, которое имеет 
единственное решение.
  
  Открытым ключом служит тройка векторов~$Y$, $U$ и~$T$. Число~$x$ 
и~все другие векторы, использованные для вычисления открытого ключа, 
остаются секретными. Владелец открытого ключа должен хранить в~качестве 
своего личного секретного ключа число~$x$ и~два вектора~$A^\prime$ и~$B$. 
Остальные секретные элементы могут быть уничтожены после завершения 
процедуры вычисления открытого ключа. Предлагаемая форма СЗДЛ состоит 
в~вычислении значения~$x$ по открытому ключу. Схема ЭЦП на ее основе 
описывается следующим образом. 

  \vspace*{-6pt}
  
  \subsection*{Алгоритм генерации электронной цифровой подписи}
  
  \noindent
  \begin{enumerate}[1.]
  \item Выбрать случайное число $k\hm<\omega$ и~вычислить вектор $V\hm= 
A^\prime \circ N^k\circ B$.
  \item Вычислить значение $e\hm= F_h(M,V)$, где $F_h$~--- некоторая 
специфицированная хеш-функ\-ция; $M$~--- электронный документ, который 
должен быть подписан.
  \item Вычислить число $s\hm= k\hm+ex\,\mathrm{mod}\,q$.
  \end{enumerate}
  
  Пара чисел $(e,s)$ представляет собой ЭЦП к~документу~$M$.
  
  \vspace*{-6pt}
  
  \subsection*{Алгоритм проверки подлинности электронной цифровой подписи}
  
  \noindent
  \begin{enumerate}[1.]
  
  \item По значениям~$e$ и~$s$ вычислить вектор $V^\prime\hm= Y^e\circ 
T\circ U^s$. 
  \item Используя хеш-функ\-цию~$F_h$, вычислить значение $e^\prime\hm= 
F_h(M,V^\prime)$.
  \item Если $e^\prime\hm=e$, то подпись признается подлинной, иначе 
подпись отвергается как ложная. 
  \end{enumerate}
  
  Доказательство корректности работы схемы ЭЦП: 
  \begin{multline*}
 V^\prime =Y^e\circ T\circ U^s ={}\\
 {}=\left( A^\prime\circ N^x\circ A\right)^e\circ 
T\circ \left( B^\prime \circ N\circ B\right)^s={}
\end{multline*}

\noindent
  \begin{multline*}
    {}= A^\prime \circ N^{xe} \circ(A\circ T\circ B^\prime)\circ N^s\circ B= {}\\
{}=A^\prime \circ N^{xe}\circ L_3\circ N^{k-xe}\circ B={}\\
  {}= A^\prime \circ N^{xe+k-xe} \circ B =A^\prime \circ N^k \circ B 
=V\Rightarrow{}\\
{}\Rightarrow 
  e^\prime =F_h(M,V^\prime) =F_h(M,V)=e\,.
  \end{multline*}
  
  Предложенная в~данном разделе схема ЭЦП расширяет ранее известные 
типы криптосхем с~открытым ключом~\cite{4-f}, основанные на 
вычислительной трудности СЗДЛ. Вопрос о сверхполиномиальной сложности 
решения предложенной формы СЗДЛ на квантовом компьютере требует 
выполнения специальных математических исследований с~привлечением 
теории конечных алгебр и~связан с~изучением возможности и~трудоемкости 
сведения СЗДЛ к~обычной ЗДЛ. Это представляет собой самостоятельную 
исследовательскую задачу. 

  \vspace*{-9pt}
  
\section{Заключение}

\vspace*{-2pt}

  Разработан общий способ задания $m$-мер\-ных конечных некоммутативных 
ассоциативных алгебр для произвольного четного значения размерности 
$m\hm\geq4$, свойства которых позволили предложить новую форму СЗДЛ
 и~постквантовую схему цифровой 
подписи. 

  \vspace*{-9pt}
  
{\small\frenchspacing
 {%\baselineskip=10.8pt
 \addcontentsline{toc}{section}{References}
 \begin{thebibliography}{9}
 
 \vspace*{-2pt}
 
\bibitem{1-f}
Announcing request for nominations for public-key post-quantum cryptographic 
algoritms. Federal Register. Department of Commerce.  Vol.~81. No.\,244. P.~92787--92788.
{\sf  
https://www.gpo.gov/fdsys/pkg/FR-2016-12-20/pdf/2016-30615.pdf}.
\bibitem{2-f}
Post-quantum cryptography~/ Eds. T.~Lange, R.~Steinwandt.~--- 
Security and cryptology ser.~--- Springer, 2018. Vol.~10786. 542~p.
\bibitem{3-f}
\Au{Shor P.\,W.} Polynomial-time algorithms for prime factorization and discrete logarithms 
on quantum computer~// SIAM J.~Comput., 1997. Vol.~26. P.~1484--1509.
\bibitem{4-f}
\Au{Moldovyan~D.\,N.} Non-commutative finite groups as primitive of public-key 
cryptoschemes~// Quasigroups Related Systems, 2010. Vol.~18. P.~165--176.
\bibitem{5-f}
\Au{Kuzmin A.\,S., Markov~V.\,T., Mikhalev~A.\,A., Mikhalev~A.\,V., Nechaev~A.\,A.} 
Cryptographic algorithms on groups and algebras~// J.~Math. Sci., 2017. Vol.~223. Iss.~5. 
P.~629--641.
\end{thebibliography}

 }
 }

\end{multicols}

\vspace*{-6pt}

\hfill{\small\textit{Поступила в~редакцию 27.06.19}}

\vspace*{8pt}

%\pagebreak

\newpage

\vspace*{-28pt}

%\hrule

%\vspace*{2pt}

%\hrule

%\vspace*{-2pt}

\def\tit{METHOD FOR~DEFINING FINITE NONCOMMUTATIVE 
ASSOCIATIVE ALGEBRAS OF~ARBITRARY EVEN~DIMENSION 
FOR~DEVELOPMENT OF~THE~POSTQUANTUM CRYPTOSCHEMES}


\def\titkol{Method for~defining finite noncommutative 
associative algebras of~arbitrary even~dimension 
for~development of %~the~postquantum 
cryptoschemes}

\def\aut{A.\,A.~Kostina, A.\,Yu.~Mirin, D.\,N.~Moldovyan, and~R.\,Sh.~Fahrutdinov}

\def\autkol{A.\,A.~Kostina, A.\,Yu.~Mirin, D.\,N.~Moldovyan, and~R.\,Sh.~Fahrutdinov}

\titel{\tit}{\aut}{\autkol}{\titkol}

\vspace*{-11pt}


 \noindent
  St.\ Petersburg Institute for Informatics and Automation of the Russian Academy 
of Sciences, 39,~14th Line V.O., St.\ Petersburg 199178, Russian 
Federation

\def\leftfootline{\small{\textbf{\thepage}
\hfill INFORMATIKA I EE PRIMENENIYA~--- INFORMATICS AND
APPLICATIONS\ \ \ 2020\ \ \ volume~14\ \ \ issue\ 1}
}%
 \def\rightfootline{\small{INFORMATIKA I EE PRIMENENIYA~---
INFORMATICS AND APPLICATIONS\ \ \ 2020\ \ \ volume~14\ \ \ issue\ 1
\hfill \textbf{\thepage}}}

\vspace*{3pt} 

 

\Abste{The paper introduces a new unified method for defining finite noncommutative associative 
algebras of arbitrary even dimension~$m$ and describes the investigated properties of the algebras 
for the cases $m = 4$ and~$6$, when the algebras are defined over the ground field $GF(p)$ 
with a~large size of the prime number~$p$. Formulas describing the set of~$p^2$ ($p^4$) global 
left-sided units contained in the 4-dimensional (6-dimensional) algebra are derived. Only local 
invertibility takes place in the algebras investigated. Formulas for computing the unique local 
two-sided unit related to the fixed locally invertible vector are derived for each of the algebras. A~new 
form of the hidden discrete logarithm problem is proposed as postquantum cryptographic 
primitive. The latter was used to develop the postquantum digital signature scheme.} 

  \KWE{finite noncommutative algebra; associative algebra; computationally difficult problem; 
discrete logarithm; digital signature; postquantum cryptography}
  
\DOI{10.14357/19922264200113} 

%\vspace*{-14pt}

%\Ack
%\noindent

 


%\vspace*{6pt}

  \begin{multicols}{2}

\renewcommand{\bibname}{\protect\rmfamily References}
%\renewcommand{\bibname}{\large\protect\rm References}

{\small\frenchspacing
 {%\baselineskip=10.8pt
 \addcontentsline{toc}{section}{References}
 \begin{thebibliography}{9}
\bibitem{1-f-1}
 Depatment of Commerce. 2016. Announcing request for nominations for 
 public-key post-quantum cryptographic algorithms. Federal Register 
81(244):92787--92788. Available at: 
{\sf https://www.gpo.gov/fdsys/pkg/FR-2016-12-20/pdf/2016-30615.pdf} (accessed 
March~2, 2020).
\bibitem{2-f-1}
Lange,~T., and R.~Steinwandt, eds.
2018. \textit{Post-quantum cryptography}. Security and cryptology ser.
Springer. Vol.~10786. 542~p.
\bibitem{3-f-1}
\Aue{Shor, P.\,W.} 1997. Polynomial-time algorithms for prime factorization and 
discrete logarithms on quantum computer. \textit{SIAM J.~Comput.} 
26:1484--1509.
\bibitem{4-f-1}
\Aue{Moldovyan, D.\,N.} 2010. Non-commutative finite groups as primitive of 
public-key cryptoschemes. \textit{Quasigroups Related Systems} 18:165--176.
\bibitem{5-f-1}
\Aue{Kuzmin, A.\,S., V.\,T.~Markov, A.\,A.~Mikhalev, A.\,V.~Mikhalev, and 
A.\,A.~Nechaev.} 2017. Cryptographic algorithms on groups and algebras. 
\textit{J.~Math. Sci.} 223(5):629--641.
 \end{thebibliography}

 }
 }

\end{multicols}

%\vspace*{-7pt}

\hfill{\small\textit{Received June 27, 2019}}

%\pagebreak

%\vspace*{-22pt} 
  
  \Contr
  
  \noindent
  \textbf{Kostina Anna A.} (b.\ 1983)~--- scientist, Laboratory of Cybersecurity 
and Postquantum Cryptosystems, St.\ Petersburg Institute for Informatics and 
Automation of the Russian Academy of Sciences, 39,~14th Line V.O., St.\ 
Petersburg 199178, Russian Federation; \mbox{anna-kostina1805@mail.ru}
  
  \vspace*{3pt}
  
  \noindent
  \textbf{Mirin Anatoly Yu.} (b.\ 1979)~--- Candidate of Science (PhD), senior 
scientist, Laboratory of  Cybersecurity and Postquantum Cryptosystems, 
  St.\ Petersburg Institute for Informatics and Automation of the Russian Academy 
of Sciences, 39,~14th Line V.O., St.\ Petersburg 199178, Russian Federation; 
\mbox{mirin@cobra.ru}
  
  \vspace*{3pt}
  
  \noindent
  \textbf{Moldovyan Dmitriy N.} (b.\ 1986)~--- Candidate of Science (PhD), 
scientist, Laboratory of  Cybersecurity and Postquantum Cryptosystems, St. 
Petersburg Institute for Informatics and Automation of the Russian Academy of 
Sciences, 
  39,~14th Line V.O., St.\ Petersburg 199178, Russian Federation; 
\mbox{mdn.spectr@mail.ru}
  \vspace*{3pt}
  
  \noindent
  \textbf{Fahrutdinov Roman Sh.} (b.\ 1972)~--- Candidate of Science (PhD), 
Head of Laboratory of Cybersecurity and Postquantum Cryptography, St.\ Petersburg 
Institute for Informatics and Automation of the Russian Academy of Sciences, 
  39, 14th Line V.O., St.\ Petersburg 199178, Russian Federation; 
\mbox{fahr@cobra.ru}
  



\label{end\stat}

\renewcommand{\bibname}{\protect\rm Литература} 
       %13
\def\stat{vohmin}

\def\tit{МЕТОД НАВИГАЦИИ И~СОСТАВЛЕНИЯ КАРТЫ В~ТРЕХМЕРНОМ ПРОСТРАНСТВЕ 
НА~ОСНОВЕ КОМБИНИРОВАННОГО РЕШЕНИЯ ВАРИАЦИОННОЙ ПОДЗАДАЧИ 
ТОЧКА--ТОЧКА~ICP ДЛЯ~АФФИННЫХ ПРЕОБРАЗОВАНИЙ$^*$}

\def\titkol{Метод навигации и~составления карты в~трехмерном пространстве 
на~основе комбинированного решения} % вариационной подзадачи 
%точка--точка ICP для аффинных преобразований}

\def\aut{А.\,В.~Вохминцев$^1$, А.\,В.~Мельников$^2$, C.\,А.~Пачганов$^3$}

\def\autkol{А.\,В.~Вохминцев, А.\,В.~Мельников, C.\,А.~Пачганов}

\titel{\tit}{\aut}{\autkol}{\titkol}

\index{Вохминцев А.\,В.}
\index{Мельников А.\,В.}
\index{Пачганов C.\,А.}
\index{Vokhmintcev A.\,V.}
\index{Melnikov A.\,V.}
\index{Pachganov S.\,A.}


{\renewcommand{\thefootnote}{\fnsymbol{footnote}} \footnotetext[1]
{Работа выполнена при поддержке РФФИ (проект 18-37-20032) 
и~Российского научного фонда (проект  
15-19-10010).}}


\renewcommand{\thefootnote}{\arabic{footnote}}
\footnotetext[1]{Челябинский государственный университет; 
Югорский государственный университет, 
\mbox{vav@csu.ru}}
\footnotetext[2]{Югорский государственный университет, \mbox{melnikovav@uriit.ru}}
\footnotetext[3]{Югорский государственный университет, \mbox{pachganovsa@uriit.ru}}

%\vspace*{-12pt}
  
  
      
  
  \Abst{Одновременная навигация и~картографирование относятся к~проблеме, в~которой 
данные кадра используются в~качестве единственного источника внешней информации для 
того, чтобы установить положение движущейся камеры в~пространстве и~в~то же время 
построить карту зоны исследования. На сегодняшний день эта проблема считается решенной 
для построения двумерных карт небольших статических сцен с~использованием датчиков 
дальности. Однако для динамичных, сложных и~крупномасштабных сцен построение точной 
трехмерной карты окружающего пространства стало активной об\-ластью научных 
исследований. Для решения поставленной проблемы в~работе предложено решение задачи 
точка--точка для аффинных преобразований и~разработан быстрый итерационный алгоритм 
регистрации кадров в~трехмерном пространстве. Производительность и~вычислительная 
сложность предлагаемого метода реконструкции трехмерных сцен представлены 
и~обсуждены на примере эталонных данных. Результаты могут быть применены в~задачах 
навигации мобильного робота в~реальном масштабе времени.}
  
  \KW{задача регистрации данных; локализация; методы одновременной навигации 
и~со\-став\-ле\-ния карты; аффинное преобразование; двумерные дескрипторы; итеративный 
алгоритм ближайших точек}

\DOI{10.14357/19922264200114} 
  
\vspace*{-3pt}


\vskip 10pt plus 9pt minus 6pt

\thispagestyle{headings}

\begin{multicols}{2}

\label{st\stat}
  
\section{Введение}

  Разработка динамической системы для надежного решения проблемы 
одновременной навигации мобильного робота и~составления карты 
окру\-жа\-ющей его среды (Simultaneous Localization And Mapping, SLAM) 
в~реальном масштабе времени стала одной из ключевых задач в~современной 
робототехнике и~машинном зрении, так как на ее решении основано создание 
автономных интеллектуальных робототехнических комплексов и~сис\-тем~[1--3]. 
Для построения качественных трехмерных моделей необходимо совместное 
использование полученных с~разных датчиков данных, таких как изображения, 
положение используемого датчика и~карты глубины~[4, 5]. В~большинстве 
случаев для построения трехмерных моделей по картам глубины используется 
итеративный алгоритм ближайших точек (Iterative Closest Point, ICP)~[6]. 
Главный этап алгоритма регистрации ICP связан с~поиском соответствующего 
геометрического преобразования (ортогонального или аффинного), которое 
наилучшим образом совмещает два облака точек в~разных RGB-D-кад\-рах для 
выбранной метрики (вариационная подзадача алгоритма). Точ\-ность 
реконструкции трехмерной сцены существенно зависит от выбора метрики для 
оценки гео\-мет\-ри\-че\-ско\-го преобразования и~метода решения вариационной 
задачи~[7]. 

Результат применения итерационных методов для решения задачи 
минимизации выбранного функционала зависит от правильности выбора 
начального приближения па\-ра\-мет\-ров гео\-мет\-ри\-че\-ско\-го преобразования: 
итерационный процесс может сходиться медленно, сходиться к~локальному 
оптимуму или вообще не сходиться. Использование решений вариационной 
задачи в~замк\-ну\-той форме позволяет избежать этих проблем~[8, 9]. Выбор 
класса гео\-мет\-ри\-че\-ских преобразований так\-же оказывает значительное влияние 
на результат реконструкции трехмерной сцены~[10].

 Для класса ортогональных 
преобразований решение задачи точ\-ка--точ\-ка в~замк\-ну\-той форме получено 
с~по\-мощью кватернионов~\cite{8-voh} или с~помощью ортогональных 
мат\-риц~\cite{9-voh}. На основе метода Хорна сформулирован алгоритм ICP 
в~варианте точ\-ка--точ\-ка~[11]. Известно, что метрика точ\-ка--плос\-кость 
превосходит метрику точ\-ка--точ\-ка по точ\-ности и~ско\-рости схо\-ди\-мости. 

Использование аффинных преобразований позво\-ля\-ет решать задачу 
регистрации для нежестких объектов~[12]. Отметим, что если истинное 
гео\-мет\-ри\-че\-ское преобразование, связывающее два облака точек, ортогонально, 
то применение вариационной задачи точ\-ка--плос\-кость для аффинных 
преобразований даст правильный результат только в~том случае, если 
соответствие между точками двух облаков близко к~идеальному. На практике 
соответствие между точками двух облаков в~большинстве случаев неидеально: 
например, все точки первого облака могут соответствовать одной точке второго 
облака или небольшому локальному подмножеству точек второго облака~[13]. 
Решение описанной выше условной вариационной задачи для метрики  
точ\-ка--точ\-ка известно как метод Хорна~\cite{8-voh}. Для мет\-ри\-ки  
точ\-ка--плос\-кость~\cite{14-voh} существуют решения в~замк\-ну\-той форме для 
аффинных преобразований. 
  
\section{Метод SLAM и~постановка задачи}

  Решение задачи одновременной навигации и~картографирования со\-сто\-ит из 
сле\-ду\-ющих этапов: 
\begin{itemize}
\item сопоставление и~регистрация последовательностей 
изображений в~RGB-D-кад\-рах; 
\item пространственное совмещение трехмерных 
облаков точек в~RGB-D-кад\-рах; 
\item обнаружение замыканий цикла; 
\item построение 
трехмерной карты доступной окружающей среды;
\item определение позиции 
робота в~относительной системе координат в~каждый момент времени. 
\end{itemize}

В~данной работе предложено новое решение вариационной задачи  
точ\-ка--точ\-ка в~замк\-ну\-той форме на основе комбинации данных о~глубине 
и~цвете в~кадре, которое на\-прав\-ле\-но на решение второго этапа задачи SLAM. 
Основные недостатки итерационных методов регистрации данных связаны 
с~ограничением области схо\-ди\-мости и~большой вы\-чис\-ли\-тель\-ной слож\-ностью. 
Кроме того, результат решения вариационной задачи зависит от пра\-виль\-ности 
выбора начального приближения. Для преодоления данного недостатка 
в~работе предлагается использовать визуально связанные характеристики 
RGB-D-кад\-ра (особые точки), которые позволяют совмещать кадры без 
требования начальной инициализации. Хорн предложил решение условной 
вариационной задачи для метрики точ\-ка--точ\-ка в~замкнутой форме для 
ортогональных преобразований. В~данной работе получено решение 
в~замк\-ну\-той форме для аффинных преобразований, что, во-пер\-вых, создает 
математическую основу для решения задачи регистрации неригидных объектов 
на сцене; во-вто\-рых, позволяет находить точное решение вариационной 
задачи для вырожденных случаев, например, когда все точки трехмерного 
облака точек находятся в~одной плоскости. При решении задачи SLAM 
в~динамическом пространстве такая необходимость возникает при 
идентификации и~отслеживании основных структурных элементов сцены: 
например, для замк\-ну\-тых пространств (помещений) такими элементами могут 
выступать потолок, стены, пол.
  
\section{Сопоставление визуальных признаков на~RGB-D-кадрах}

  Для обработки визуальных характеристик \mbox{сцены} используется алгоритм 
сопоставления изоб\-ра\-же\-ний на основе рекурсивного вы\-чис\-ле\-ния гистограмм 
на\-прав\-лен\-ных градиентов (ГНГ) по нескольким круглым скользящим окнам 
и~\mbox{пирамидальному} разложению изображения~[15, 16]. Для работы с~особыми 
признаками используется сле\-ду\-ющая схема.
  \begin{enumerate}[1.]
  \item  Вычисление ГНГ на изображениях.
  \item  Сопоставление между особыми точками для выбранных подмножеств.
  \item  Отбрасывание некоторых пар особых точек для поиска~[17].
  \item Решение вариационной задачи регистрации данных для визуально 
связанных характеристик изображения.
  \end{enumerate}
  
  Рассмотрим более подробно пп.~2 и~4. В~работе используется 
корреляционный оператор, при помощи которого осуществляется процедура 
со\-по\-став\-ле\-ния данных из различных RGB-D-кад\-ров. Введем формулу для 
определения нормализованной центрированной ГНГ эталонного изображения:
  \begin{equation*}
  \overline{\mathrm{HOG}_i^R}(\alpha) =\fr{\mathrm{HOG}_i^R(\alpha) -
  \mathrm{Mean}^R}{\sqrt{\mathrm{Var}^R}}\,,
  %\label{e1-voh}
  \end{equation*}
где $\mathrm{Mean}^R$~--- среднее значение ГНГ; $\mathrm{Var}^R$~--- дисперсия ГНГ. Тогда 
для каж\-до\-го $i$-го медианного фильт\-ра (МФ) в~позиции~$k$ можно 
определить корреляционную функцию:
\begin{multline*}
C_i^k(\alpha) ={}\\
{}=\mathrm{IFT} \left[ \fr{\mathrm{HS}_i^k(\omega) \mathrm{HR}_i^*(\omega)} {\sqrt{Q 
\sum\nolimits_{q=0}^{Q-1}\left( \mathrm{HOG}_i^k(q)\right)^2- \left( 
\mathrm{HS}_i^k(0)\right)^2}}\right],\hspace*{-0.56015pt}
%\label{e2-voh}
\end{multline*}
где $\mathrm{HS}_i^k(\omega)$~--- преобразование Фурье ГНГ внутри $i$-го МФ 
входной сцены; $\mathrm{HR}_i(\omega)$~--- преобразование Фурье 
$\overline{\mathrm{HOG}_i^R}(\alpha)$; $*$ для $i$-го преобразования Фурье обозначает 
комплексное сопряжение. Для определения подобия двух ГНГ применяется 
корреляционный пик $P_i^k\hm= \max\nolimits_\alpha \left\{ 
C_i^k(\alpha)\right\}$.
  
  Решение вариационной задачи для визуально связанных характеристик 
изоб\-ра\-же\-ния может быть пред\-став\-ле\-но в~виде:
  \begin{equation*}
  J(\mathrm{RV}\,)=\fr{1}{\left\vert {A}_f \right\vert} 
  \sum\limits^n_{i\in A_f} {w}_i\left\| 
M\left(\mathrm{RV}\,F_x^i\right) - M\left(F_y^i\right)\right\|^2\,.
  %\label{e3-voh}
  \end{equation*}
Здесь $\mathrm{RV}$~--- матрица аффинного преобразования для визуально связанных 
характеристик сцены; ${w}_i$~--- весовые характеристики данных; 
$F_x^i$ и~$F_y^i$~--- визуально связанные характеристики сцены в~исходном и~целевом кадре соответственно:
\begin{equation*}
F_x^i=\left( x^i_{1f}, x^i_{2f}, x^i_{3f}\right)^{\mathrm{T}}\,;\enskip
F_y^i= \left( y^i_{1f}, y^i_{2f}, y^i_{3f}\right)^{\mathrm{T}}\,.
%\label{e4-voh}
\end{equation*}
  
  Функция~$M$ осуществляет преобразование координат точек~$F_x^i$ 
и~$F_y^i\hm\in \mathbb{R}^3$ из трехмерной сис\-те\-мы координат относительно 
камеры в~систему координат камеры $C^i\hm= \left( C_x^i, C_y^i, 
D^i\right)\hm\in \mathbb{R}^3$, где $C_x^i$ и~$C_y^i$~--- соответствующие 
координаты точек в~пиксельном пространстве; $D^i$~--- значение глубины 
в~пиксельном про\-стран\-стве:
  \begin{align*}
  C_x^i&=\fr{f}{x^i_{3f}}\,x^i_{1f}+O_x\,;\\
  C_y^i&=\fr{f}{x^i_{3f}}\,x^i_{2f}+O_y\,;\\
  D^i&=\sqrt{{x^i_{1f}}^2 +{x^i_{2f}}^2+{x^i_{3f}}^2}\,.
  \end{align*}
%  \label{e5-voh}
 Здесь $O_x$ и~$O_y$~--- координаты центра изображения в~пиксельном 
пространстве; $f$~--- фокус камеры. Аналогичным образом могут быть 
определены координаты точек в~сис\-те\-ме координат камеры для точек~$F_y^i$.

\section{Решение задачи точка--точка метода ICP для~аффинных 
преобразований}

  Задача регистрации трехмерных данных (данных кадра о глубине) на основе 
алгоритма ICP со\-сто\-ит из следующих шагов.
  \begin{enumerate}[1.]
\item Формирование разреженных подмножеств точек из двух плот\-ных 
трехмерных облаков точек.
\item Определение соответствующих точек в~каждом из разреженных 
подмножеств.
\item Определение весовых коэффициентов для каж\-дой полученной пары.
\item Отбрасывание некоторых пар в~облаках точек (RANSAC-ме\-тод или 
аналог).
\item Выбор метрики ошибки для пар точек.
\item Решение вариационной задачи на основе минимизации функции ошибки.
\end{enumerate}
  
  Обозначим через $X\hm=\{x_1, \ldots , x_n\}$ множество точек в~исходном 
RGB-кадре и~через $Y\hm= \{ y_1, \ldots , y_m\}$ множество точек в~целевом 
RGB-D-кад\-ре в~$\mathbb{R}^3$. Предположим, что отношения между 
точками в~кадрах~$X$ и~$Y$ такие, что для каждой точки в~$x_i$ можно 
вычислить со\-от\-вет\-ст\-ву\-ющую точ\-ку в~$y_i$. Тогда алгоритм ICP можно 
рассмотреть как гео\-мет\-ри\-че\-ское преобразование из~$X$ в~$Y$ сле\-ду\-юще\-го 
вида:
  \begin{equation*}
  Rx_i+T\,,
  %\label{e6-voh}
  \end{equation*}
где $R$~--- матрица поворота; $T$~--- вектор параллельного переноса; 
$i\hm=1,\ldots , n$. Тогда аффинное преобра\-зо\-ва\-ние в~$\mathbb{R}^3$ можно 
представить в~виде функции от двенадцати переменных и~решить 
вариационную задачу алгоритма регистрации кад\-ров для произвольного 
преобразования. 

Пусть $J(R,T)$~--- функция вида:
\begin{equation*}
J(R,T)=\sum\limits_{i=1}^n \| R x_i+T-y_i\|^2\,,
%\label{e7-voh}
\end{equation*}
 где
 $$
x_i=\begin{pmatrix}
x_{1i}\\ x_{2i}\\ x_{3i}
\end{pmatrix}\,;\enskip
y_i=\begin{pmatrix}
y_{1i}\\ y_{2i}\\ y_{3i}
\end{pmatrix}\,.
$$
  %
  Тогда вариационная задача ICP может быть определена как
%  \begin{equation*}
 $\mathop{\mathrm{arg\,min}}\limits_{ R,T} J(R,T)$,
 %  \end{equation*}
где 
$$
R=\begin{pmatrix}
r_{11} & r_{12} & r_{13}\\
r_{21} & r_{22} & r_{23}\\
r_{31} & r_{32} & r_{33}\end{pmatrix}\,;\enskip
T=\begin{pmatrix}
t_1\\ t_2\\ t_3
\end{pmatrix}\,.
\vspace*{6pt}
$$

  
  Можно заметить, что
  \begin{multline}
  J(R,T)={}\\
  {}=\sum\limits_{i=1}^n  \left( 
r_{11}x_{1i}+r_{12}x_{2i}+r_{13}x_{3i}+t_1-y_{1i}\right)^2+{}\\
  {}+ \left( r_{21}x_{1i}+r_{22}x_{2i}+r_{23}x_{3i}+t_2 -y_{2i}\right)^2+{}\\
  {}+ \left( r_{31}x_{1i}+r_{32}x_{2i}+r_{33}x_{3i}+t_3-y_{3i}\right)^2\,.
  \label{e9-voh}
  \end{multline}
  
  Применим к~множеству точек~$X$ преобразование переноса сле\-ду\-юще\-го 
вида:

\noindent
  \begin{multline*}
  \left( x^t_{1i}=x_{1i}-\fr{1}{n}\sum\limits^n_{j=1}x_{1j}\,,
  x^t_{2i}=x_{2i}-\fr{1}{n}\sum\limits^n_{j=1} x_{2j}\,,\right.\\
  \left. x^t_{3i}=x_{3i}-\fr{1}{n}\sum\limits^n_{j=1}x_{3j}\right)\,.
 % \label{e10-voh}
  \end{multline*}
  
  Далее выполним аналогичное преобразование\linebreak для облака точек~$Y$ 
и~подставим новые координаты в~выражение~(\ref{e9-voh}). Очевидно, что 
функционал~$J(R,T)$ не зависит от элементов вектора параллельного переноса. 
Решение вариационной \mbox{задачи} ICP может быть найдено методом Хор\-на 
в~общем случае. Распространим метод Хорна на случай аффинных 
преобразований. Решение задачи для вы\-рож\-ден\-ных случаев представлено 
в~работе~\cite{17-voh}. Положим, что 
$$
\sum\limits^n_{i=1} x^2_{1i}\not=0\,;\enskip 
  \sum\limits^n_{i=1} x^2_{2i}\not=0\,;\enskip
  \sum\nolimits^n_{i=1}  x^2_{3i}\not=0\,.
  $$
  
   Тогда можно решить вариационную задачу относительно 
мат\-ри\-цы~${R}$:

\noindent
  \begin{multline*}
  \fr{\partial J(R)}{\partial r_{1k}} ={}\\
  {}=\sum\limits^n_{i=1} 2\left( 
r_{11}x_{1i}+r_{12} x_{2i}+r_{13}x_{3i}-y_{1i}\right) x_{ki} =0\,,\\
  k=1,2,3\,;
 % \label{e11-voh}
  \end{multline*}
  
\vspace*{-14pt}

  \noindent
  \begin{multline}
  \sum\limits^n_{i=1} \left( 
r_{1{m}}x_{{m}i}+r_{1{n}}x_{{n}i} -
y_{1i}\right) x_{ki} +r_{1k} \sum\limits^n_{i=1} x^2_{ki}=0\,,\\
  k,m,n=1,2,3\,;\ m,n\not= k\,.
  \label{e12-voh}
  \end{multline}
  
  Из выражения~(\ref{e12-voh}) можем получить значения параметров 
мат\-ри\-цы поворота:
  \begin{equation}
  r_{1k}=-\fr{\sum\nolimits^n_{i=1} (r_{1{m}} 
x_{{m}i}+r_{1{n}}x_{{n}i}-
y_{1i})x_{ki}}{\sum\nolimits^n_{i=1} x^2_{ki}}\,.
  \label{e13-voh}
  \end{equation}
  
  Учитывая выражение~(\ref{e13-voh}), можем пред\-ста\-вить $J(R)$ 
в~сле\-ду\-ющем виде:

\noindent
  \begin{multline}
  J(R)= \sum\limits^n_{i=1} \left(
  \vphantom{\fr{\sum\nolimits^n_{j=1} (r_{1{m}}x_{{m}j}+r_{1{n}} 
x_{{n}j}-y_{1j})x_{kj}}{\sum\nolimits^n_{j=1} x^2_{kj}}}
 r_{1{m}}x_{{m}i}-{}\right.\\
  {}-
\fr{\sum\nolimits^n_{j=1} (r_{1{m}}x_{{m}j}+r_{1{n}} 
x_{{n}j}-y_{1j})x_{kj}}{\sum\nolimits^n_{j=1} x^2_{kj}}\,x_{ki} 
+r_{1{n}} x_{{n}i} -{}\\
\left.{}-y_{1i}
  \vphantom{\fr{\sum\nolimits^n_{j=1} (r_{1\mathrm{m}}x_{\mathrm{m}j}+r_{1\mathrm{n}} 
x_{\mathrm{n}j}-y_{1j})x_{kj}}{\sum\nolimits^n_{j=1} x^2_{kj}}}
\right)^2+  \left( r_{21}x_{1i}+r_{22}x_{2i}+r_{23}x_{3i}-y_{2i}\right)^2+{}\\
  {}+ \left(  r_{31}x_{1i} +r_{32}x_{2i}+r_{33} x_{3i} -y_{3i}\right)^2\,.
  \label{e14-voh}
  \end{multline}
  
  Рассмотрим более подробно первое сла\-га\-емое в~выражении~(\ref{e14-voh}) 
и~раскроем скоб\-ки под знаком суммы. Если подставить полученное выражение 
в~функционал~$J(R)$, то имеем
  \begin{multline}
  J(R)=\sum\limits^n_{i=1}\left( r_{1m}\left(
  x_{mi}-x_{ki}\fr{\sum\nolimits^n_{j=1} x_{mj} x_{kj}}
  {\sum\nolimits^n_{j=1} x^2_{kj}}\right)
  +{}\right.\\
  {}+r_{1n}\left( x_{ni}-x_{ki}\fr{\sum\nolimits^n_{j=1} x_{nj} x_{kj}}
  {\sum\nolimits^n_{j=1} x^2_{kj}}\right)-{}\\
 \left. {}-\left( y_{1i}-x_{ki}
  \fr{\sum\nolimits^n_{j=1} y_{1j} x_{kj}}
  {\sum\nolimits^n_{j=1} x^2_{kj}}\right)\right)^2+{}\\
  {}+
 \left( r_{21}x_{1i}+r_{22} x_{2i}+r_{23}x_{3i}-
y_{2i}\right)^2 +{}\\
  {}+\left( r_{31}x_{1i}+r_{32} x_{2i}+r_{33}x_{3i}-y_{3i}\right)^2\,.
  \label{e15-voh}
  \end{multline}
  
  Введем следующие обозначения для упрощения вида  
выражения~(\ref{e15-voh}):
  \begin{align*}
  G_{mi}&=x_{{m}i}-x_{ki} \fr{\sum\nolimits^n_{j=1} x_{mj} x_{kj}} 
{\sum\nolimits^n_{j=1} x^2_{kj}}\,;\\
  G_{pi}&=x_{ni}- x_{ki} \fr{\sum\nolimits^n_{j=1} x_{nj} x_{kj}} 
{\sum\nolimits^n_{j=1} x^2_{kj}}\,;\\
  G_{ki}&=y_{1i}- x_{ki} \fr{\sum\nolimits^n_{j=1} y_{1j} x_{kj}} 
{\sum\nolimits^n_{j=1} x^2_{kj}}\,.
  \end{align*}
Тогда с~учетом обозначений можем переписать выражение~(\ref{e15-voh}) как
\begin{multline}
J(R)=\sum\limits^n_{i=1}\left( r_{1{m}} G_{mi}+r_{1{n}}G_{pi} 
-G_{ki}\right)^2+{}\\
{}+ \left( r_{21}x_{1i} +r_{22}x_{2i} +r_{23}x_{3i}-y_{2i}\right)^2+{}\\
{}+ \left( r_{31}x_{1i} +r_{32}x_{2i} +r_{33}x_{3i}-y_{3i}\right)^2\,.
\label{e16-voh}
\end{multline}
  
  Определим частную производную~$J(R)$ относительно~$r_{1m}$:
  \begin{equation*}
  \hspace*{-1.63478pt}\fr{\partial J(R)}{\partial r_{1m}} =2\sum\limits^n_{i=1} \left( 
r_{1{m}}G_{mi} +r_{1{n}}G_{pi} -G_{ki}\right) G_{mi}=0.
%  \label{e17-voh}
  \end{equation*}
  Тогда найдем~$r_{1{m}}$:
  $$
  r_{1{m}}= -\fr{r_{1n}\sum\nolimits^n_{i=1} G_{mi} G_{pi}-
\sum\nolimits^n_{i=1} G_{mi} G_{ki}}{\sum\nolimits^n_{i=1} G_{mi}^2}\,.
  $$
  Подставим полученное решение в~функционал~$J(R)$:
  \begin{multline}
  J(R)={}\\
  {}=\sum\limits^n_{i=1} \!\left(\! -\fr{r_{1{n}}\sum\nolimits^n_{j=1} 
G_{mj} G_{pj} \!-\!\sum\nolimits^n_{j=1} G_{mj} G_{kj}} {\sum\nolimits^n_{j=1} 
G_{mj}^2}\,G_{mi} +\right.\\
\left.{}+r_{1n}G_{pi} -G_{ki}
\vphantom{\fr{r_{1\mathrm{n}}\sum\nolimits^n_{j=1} 
G_{mj} G_{pj} -\sum\nolimits^n_{j=1} G_{mj} G_{kj}} {\sum\nolimits^n_{j=1} 
G_{mj}^2}}
\right)^2+
 \left( r_{21}x_{1i} +r_{22}x_{2i} + r_{23}x_{3i}-{}\right.\\
\left. {}-y_{2i}\right)^2+ \left( r_{31}x_{1i} +r_{32}x_{2i} +r_{33}x_{3i}-y_{3i}\right)^2\,.
  \label{e18-voh}
  \end{multline}
  
  Произведем некоторые преобразования в~первом сла\-га\-емом 
выражения~(\ref{e18-voh}):
  \begin{multline*}
  J(R)=\sum\limits^n_{i=1}
  \left(  \left(
  - r_{1{n}} G_{mi} 
\sum\limits^n_{j=1} G_{mj} G_{pj} -{}\right.\right.\\
\left.\left.{}-G_{mi} \sum\limits^n_{j=1} G_{mj} 
G_{kj}\!\right) \!\Bigg/\!  \sum\limits^n_{j=1} G_{mj}^2+
r_{1n} G_{pi} -G_{ki}\!\right)^2\!\!+\\
  {}+
  \left( r_{21}x_{1i} +r_{22}x_{2i} +r_{23}x_{3i}-y_{2i}\right)^2+{}\\
  {}+ \left( r_{31}x_{1i} +r_{32}x_{2i} +r_{33}x_{3i}-y_{3i}\right)^2\,.
  %\label{e19-voh}
  \end{multline*}
  %
  Затем сделаем группировку слагаемых сле\-ду\-юще\-го вида:
  \begin{multline}
  J(R)= \sum\limits^n_{i=1} \left(\! r_{1n}\left(\! G_{pi} -\fr{G_{mi} 
\sum\nolimits^n_{j=1} G_{mj} G_{pj}} {\sum\nolimits^n_{j=1} G_{mj}^2} 
\!\right)-{}\right.\\[2pt]
  \left.{}-\left( G_{ki} -\fr{ G_{mi} \sum\nolimits^n_{j=1} G_{mj} G_{kj}}{ 
\sum\nolimits^n_{j=1} G_{mj}^2}\right) \right)^2+{}\\[2pt]
  {}+ \left( r_{21}x_{1i} +r_{22}x_{2i} +r_{23}x_{3i}-y_{2i}\right)^2+{}\\[2pt]
  {}+ \left( r_{31}x_{1i} +r_{32}x_{2i} +r_{33}x_{3i}-y_{3i}\right)^2\,.
  \label{e20-voh}
  \end{multline}
  
  Введем следующие обозначения:
  \begin{equation}
  \left.
  \begin{array}{rl}
  \Omega_1 &= G_{pi} -\fr{G_{mi} \sum\nolimits^n_{j=1} G_{mj}G_{pj}} 
{\sum\nolimits^n_{j=1} G_{mj}^2}\,;\\[16pt]
  \Omega_2 &= G_{ki} -\fr{ G_{mi} \sum\nolimits^n_{j=1} G_{mj}G_{kj}} 
{\sum\nolimits^n_{j=1} G_{mj}^2}\,.
  \end{array}
  \right\}
  \label{e21-voh}
  \end{equation}
  
  С учетом~(\ref{e20-voh}) и~(\ref{e21-voh}) выражение~(\ref{e16-voh}) может 
быть пред\-став\-ле\-но в~виде:
  \begin{multline*}
  J(R)=\sum\limits^n_{i=1} \left( r_{1n} \Omega_1-\Omega_2\right)^2+{}\\
  {}+
  \left( r_{21}x_{1i} +r_{22}x_{2i} +r_{23}x_{3i}-y_{2i}\right)^2+{}\\
  {}+ \left( r_{31}x_{1i} +r_{32}x_{2i} +r_{33}x_{3i}-y_{3i}\right)^2\,.
 % \label{e222-voh}
  \end{multline*}
  
  Теперь определим част\-ную производную~$J(R)$ относительно~$r_{1n}$:
  \begin{equation*}
  \fr{\partial J(R)}{\partial r_{1{n}}} =2\sum\limits^n_{j=1} \left( r_{1n} 
\Omega_1 -\Omega_2\right) \Omega_1=0\,.
%  \label{e23-voh}
  \end{equation*}
  %
  Тогда 
  $$
  r_{1n}=\fr{\sum\nolimits^n_{k=1} \Omega_1\Omega_2}{\sum\nolimits^n_{k=1} 
\Omega_1^2}\,.
  $$
  
  При $\sum\nolimits^n_{k=1} \Omega_1^2\not=0$ можем определить 
па\-ра\-мет\-ры мат\-ри\-цы поворота. Тогда первая строка мат\-ри\-цы будет выглядеть 
сле\-ду\-ющим образом:
  \begin{align*}
  r_{1{m}}& =-\fr{r_{1{n}}\sum\nolimits^n_{i=1} G_{mi} G_{pi} 
-\sum\nolimits^n_{i=1} G_{mi} G_{ki}} {\sum\nolimits^n_{i=1} G_{mi}^2}\,;\\ 
r_{1n} &=\fr{\sum\nolimits^n_{k=1} \Omega_1\Omega_2}{\sum\nolimits^n_{k=1} 
\Omega_1^2}\,;\\
  r_{1k} &=-\fr{\sum\nolimits^n_{i=1} \left( r_{1{m}} x_{{m}i} 
+r_{1{n}} x_{{n}i} -y_{1i}\right) x_{ki}} {\sum\nolimits^n_{i=1} 
x^2_{ki}}\,.
  \end{align*}
  %
  Вторая строка параметров мат\-ри\-цы поворота:
  \begin{align*}
  r_{2{m}} &=-\fr{r_{2{n}}\sum\nolimits^n_{i=1} G_{mi} G_{pi} 
-\sum\nolimits^n_{i=1} G_{mi} G_{ki}} {\sum\nolimits^n_{i=1} G_{mi}^2}\,;\\ 
r_{2n} &=\fr{\sum\nolimits^n_{k=1} \Omega_1\Omega_2}{\sum\nolimits^n_{k=1} 
\Omega_1^2}\,;\\
  r_{2k} &=-\fr{\sum\nolimits^n_{i=1} \left( r_{2{m}} x_{{m}i} 
+r_{2{n}} x_{{n}i} -y_{2i}\right) x_{ki}} {\sum\nolimits^n_{i=1} 
x^2_{ki}}\,.
  \end{align*}
  %
  Третья строка параметров мат\-ри\-цы поворота:
  \begin{align*}
  r_{3{m}}& =-\fr{r_{3{n}}\sum\nolimits^n_{i=1} G_{mi} G_{pi} 
-\sum\nolimits^n_{i=1} G_{mi} G_{ki}} {\sum\nolimits^n_{i=1} G_{mi}^2}\,;\\ 
r_{3n} &=\fr{\sum\nolimits^n_{k=1} \Omega_1\Omega_2}{\sum\nolimits^n_{k=1} 
\Omega_1^2}\,;\\
  r_{3k} &=-\fr{\sum\nolimits^n_{i=1} \left( r_{3{m}} x_{{m}i} 
+r_{3{n}} x_{{n}i} -y_{3i}\right) x_{ki}} {\sum\nolimits^n_{i=1} 
x^2_{ki}}\,.
  \end{align*}
  
  Определим элементы вектора параллельного переноса~$T$ через элементы 
мат\-ри\-цы поворота:
  \begin{multline*}
  t_k=\fr{1}{n}\sum\limits^n_{i=1} \left( y_{ki}-\left( r_{k1}x_{1i} 
+r_{k2}x_{2i} +r_{k3} x_{3i}\right)\right)=0\,,\\
  k=1,2,3\,.
 % \label{e24-voh}
  \end{multline*}
  
  \vspace*{-6pt}

\section{Комбинированное решение задачи точка--точка 
для~аффинных преобразований в~трехмерном пространстве}

  Вариационную задачу точ\-ка--точ\-ка для визуально связанных характеристик 
сцены и~данных глубины мож\-но представить в~виде:
  $\mathop{\mathrm{arg\,min}}\limits_{  \mathrm{RV}, \mathrm{RD}}
  J\left(\mathrm{RV}, \mathrm{RD}\right)$, где 
  
  \noindent
  \begin{multline*}
  J\left(\mathrm{RV}, \mathrm{RD}\right)={}\\
  {}= \alpha \fr{1}{{W}} \,\fr{1}{\vert 
{A}_{{f}} \vert} \sum\limits^m_{i\in A_f} {w}_i\left\|
  M\left( \mathrm{RV}\,F_x^i\right) -M\left( F_y^i\right)\right\|^2+{}\\
  {}+(1-\alpha) \fr{1}{{W}}\,
  \fr{1}{\vert {A}_{{d}}\vert } \sum\limits^n_{j\in A_d} {w}_j\left\| 
\mathrm{RD}\,x_j +T-y_j\right\|^2\,,
%  \label{e25-voh}
  \end{multline*}
где $\mathrm{RV}$ и~$\mathrm{RD}$~--- мат\-ри\-цы аффинного преобразования для визуально 
связанных характеристик сцены и~для данных о~глубине сцены соответственно; 
$\alpha$ и~${W}$~--- набор параметров для нормировки данных, 
подбираемый эмпирическим путем; ${w}_i$ и~${w}_j$~--- 
весовые характеристики данных, связанные с~семантической маркировкой 
про\-стран\-ст\-ва; ${A}_{{f}}$~--- подмножество, которое 
содержит связи между особыми точками в~двух кадрах; 
${A}_{{d}}$~--- содержит связи между соответствующими 
точками~$x_j$ и~$y_j$ в~трехмерных облаках точек в~двух кадрах данных; 
$F_x^i$ и~$F_y^i$~--- визуально связанные характеристики сцены. 

В~общем 
случае $\mathrm{RV}\not= \mathrm{RD}$, в~данной работе находится совместное решение 
вариационной задачи для част\-но\-го случая, когда $\mathrm{RV}\hm= \mathrm{RD}
\hm= \mathrm{RT}$. Пусть~${T}_{km}$~--- начальная оценка для ICP с~использованием 
кинематической модели движения камеры; ${k}_{\max}$  
и~$\varepsilon$~--- пороги алгоритма ICP по числу шагов и~по величине 
ошибки~${E}$ соответственно; RT$^*$~--- лучшее преобразование на 
$i$-м шаге метода; $\delta$~--- порог для точек ин\-лай\-не\-ров. Общая схема 
комбинированного метода регистрации может быть представлена в~сле\-ду\-ющем 
виде.
\begin{description}
\item[Шаг~1.] Вычисление ГНГ на 
изображениях.
\item[Шаг~2.] Сопоставление между особыми точками~$F_x^i$ и~$F_y^i$ для 
выбранных подмножеств. Решение вариационной задачи регистрации данных 
для визуально связанных характеристик изоб\-ра\-же\-ния. Получим RT$^*$ 
и~${A}_{{f}}$.
\item[Шаг~3.] Проверка: если $\vert {A}_{{f}} \vert \hm <\delta$, 
то $\mathrm{RT}^* \hm={T}_{{km}}$ 
и~${A}_{{f}}\hm=\emptyset$. Положим $i\hm = 1$.
\item[Шаг~4.] Определение соответствующих 
точек~${A}_{{d}}$ в~исходном и~целевом облаке точек 
с~использованием метода ближайших соседей. Определение весовых 
коэффициентов для каж\-дой полученной пары~${A}_{{d}}$.
\item[Шаг~5.] Решение комбинированной вариационной задачи относительно 
RT$^*$, ${A}_{{f}}$ и~${A}_{d}$.
\item[Шаг~6.] Проверка: $({E}(\mathrm{RT}^*_i)-
{E}(\mathrm{RT}^*_{i+1})\leq \varepsilon)$ или (Номер  
ите\-ра\-ции\;$>$\;$\mathrm{k}_{\max}$). Если условие неверно, то возврат на 
шаг~4 и~$i \hm= i \hm+ 1$, иначе получено искомое преобразование RT$^*$.
\end{description}
  
  В качестве продолжения направления исследований представляет интерес 
разработка метода решения комбинированной вариационной задачи  
точ\-ка--плос\-кость в~замкнутой форме для аффинных и~ортогональных 
преобразований. Обозначим: 
  \begin{equation*}
  \eta_1=\alpha\fr{1}{{W}}\,\fr{1}{\vert 
{A}_{{f}}\vert}\,,\
  \eta_2=(1-\alpha) \fr{1}{{W}}\,\fr{1}{\vert 
{A}_{d}\vert}.
\end{equation*}
 Пусть~$n^i$ есть унитарная нормаль 
к~касательной плос\-кости~$T({y}^i)$ к~поверхности~$S(Y)$ в~точке~$y^i$; $\mathrm{RD}^M$~--- комбинированная матрица аффинных 
преобразований, содержащая компоненты параллельного переноса и~поворота. 
Тогда
  \begin{multline*}
  \hspace*{-2.90392pt}J\left( \mathrm{RD}^M\right) =\sum\limits^m_{i=1} \eta_1{w}_i \left( M\left( 
\mathrm{RD}^M F_x^i\right) -M\left( F_y^i\right)\right)^2+{}\\
  {}+\sum\limits_{j=1}^n \eta_2 {w}_j \left( \left\langle \mathrm{RD}^M 
x^{\prime\,j} -y^{\prime\,j}, n^j\right\rangle \right)^2\,,
  %\label{e26-voh}
  \end{multline*}
где 
$$
x^{\prime\,j}=\left( x_1^{\prime\,j}, x_2^{\prime\,j},
x_3^{\prime\,j},1\right)^{\mathrm{T}}\,;\ 
y^{\prime j}= \left(y_1^{\prime\,j}, y_2^{\prime\,j}, 
y_3^{\prime\,j},1\right)^{\mathrm{T}}.
$$
Предполагается, что 
метод будет основан на проецировании элемента многообразия мат\-риц 
линейных преобразований на подмногообразие ортогональных мат\-риц.

  \begin{figure*}[b] %fig1
  \vspace*{1pt}
    \begin{center}  
  \mbox{%
 \epsfxsize=163mm 
 \epsfbox{voh-1.eps}
 }
\end{center}
\vspace*{-10pt}
  \Caption{Тестовый набор данных Classrooms (классная комната) из NYU Depth Dataset: 
(\textit{а})~визуальные данные RGB-D-кад\-ра; 
(\textit{б})~данные глубины RGB-D-кад\-ра}
  \end{figure*}

\section{Компьютерное моделирование}

  В данном разделе представлены результаты компьютерного моделирования. 
Для проведения компьютерного моделирования были выбраны четыре набора 
данных из базы данных NYU Depth Dataset~V2~\cite{18-voh, 19-voh}, 
содержащих фрагменты крупномасштабных сцен помещений: Classrooms, 
Living Rooms~(1/4), Offices~(1/2) и~Offices~(2/2). Компьютерное моделирование 
проводилось на персональном компьютере Intel Core~i7 с~многоядерным 
графическим процессором. Для экспериментальных исследований 
использовалась камера Kinect~2.0 в~качестве RGB-D-сен\-со\-ра. Каждый из 
четырех наборов данных из тестовой базы содержит файлы дампа, изображение 
в~формате ppm (рис.~1,\,\textit{а}), данные о~глубине в~формате pgm 
(рис.~1,\,\textit{б}). Данные в~RGB-D-кад\-рах получены из разных положений 
камеры на сцене. Например, набор данных Classrooms содержит~688~ключевых 
кад\-ров размером $640\times 480$.
  

  
  Для повышения точности решения вариационной задачи точка--точка 
в~предложенном методе используются данные о~визуально связанных 
характеристиках изображений, а~значит решение вариационной задачи в~целом 
зависит от точности сопоставления данных на изображениях в~RGB-D-кад\-ре. 
Проведем сравнительный анализ известных дескрипторных методов (SIFT
(scale-invariant feature transform), 
SURF (speeded-up robust features) и~ORB (oriented fast and rotated brief)) 
и~предложенного метода со\-по\-став\-ле\-ния особых точек 
(HOGs, histogram oriented gradients)~\cite{15-voh, 16-voh}. 

Будем оценивать точность методов по чис\-лу 
правильных сопоставлений при разных значениях угла поворота и~изменения 
мас\-шта\-ба. Были получены следующие результаты зависимости точности 
реконструкции трехмерной комбинированной карты и~производительности от 
выбранного типа двумерного дескриптора (см.\ таб\-ли\-цу)~\cite{20-voh}. 
В~таб\-ли\-це приведены средние значения точности для разного типа аффинных 
преобразований (угол поворота и~масштаб) для различных наборов данных из 
тес\-то\-вой базы данных и~различных двумерных дескрипторов.
  
  \begin{table*}\small
  \begin{center}
  \begin{tabular}{|l|c|c|c|c|}
  \multicolumn{5}{p{118mm}}{Зависимость точности/производительности метода 
реконструкции трехмерной комбинированной плотной карты от типа двумерного 
дескриптора}\\
  \multicolumn{5}{c}{\ }\\[-6pt]
  \hline
\multicolumn{1}{|c|}{Метод}&\tabcolsep=0pt\begin{tabular}{c}Набор данных\\ Classrooms\end{tabular}&
\tabcolsep=0pt\begin{tabular}{c}Набор данных\\ Living Rooms (1/4)\end{tabular}&
\tabcolsep=0pt\begin{tabular}{c}Набор данных\\ Offices (1/2)\end{tabular}&
\tabcolsep=0pt\begin{tabular}{c}Набор данных\\ Offices (2/2)\end{tabular}\\
\hline
&\multicolumn{4}{c|}{Точность, \%}\\
\hline
SIFT&93&93&94&95\\
SURF&57&61&63&59\\
ORB&78&82&75&75\\
HOGs&93&96&97&96\\
\hline
&\multicolumn{4}{c|}{Производительность, с}\\
\hline
SIFT&17,2&26,21\hphantom{9}&12,82\hphantom{9}&18,9\hphantom{99}\\
SURF&0,23&0,7\hphantom{9}&0,4\hphantom{9}&0,44\\
ORB&3,34&3,42&5,15&3,33\\
HOGs&0,79&1,11&0,68&0,81\\
\hline
\end{tabular}
\end{center}
\vspace*{-2pt}
\end{table*}

\begin{figure*}[b] %fig2
   \vspace*{1pt}
    \begin{center}  
  \mbox{%
 \epsfxsize=163mm 
 \epsfbox{voh-2.eps}
 }
\end{center}
\vspace*{-10pt}
  \Caption{Сравнение скорости сходимости в~зависимости 
  от ошибки метрики и~условий 
наблюдения: (\textit{а})~изменение ско\-рости сходимости в~контролируемых 
условиях; 
(\textit{б})~изменение скорости сходимости в~неконтролируемых условиях;
  \textit{1}~--- точ\-ка--точ\-ка; \textit{2}~--- точ\-ка--точ\-ка с~экстраполяцией; \textit{3}~--- 
точ\-ка--плос\-кость; \textit{4}~--- комбинированный подход}
  \end{figure*}
  
  В работе проведен сравнительный анализ скорости сходимости 
предложенного метода регистрации данных и~известных методов регистрации 
на основе итеративного алгоритма ближайших точек\linebreak
 в~зависимости от выбора 
метрики ошибки пар точек: точ\-ка--точ\-ка~\cite{6-voh}, точ\-ка--точ\-ка 
с~экстраполяцией~\cite{8-voh}, точ\-ка--плос\-кость~\cite{14-voh} в~терминах 
среднеквадратичной ошибки. Для проведения \mbox{компьютерного} моделирования 
было выбрано два набора данных из базы данных NYU Depth Dataset: 
в~контролируемых условиях (Offices~(1/2), сцена~1) и~в~неконтролируемых 
условиях (Offices~(1/2), сцена~2). 

В~результате серии тестов установлены 
зависимости скорости сходимости методов от выбора\linebreak ошибки метрики 
и~условий проведения экс\-пе\-ри\-мен\-тов (рис.~2). Было установлено, что 
в~конт\-ро\-ли\-ру\-емых условиях предложенный метод регистрации данных имеет 
сходимость, близкую \mbox{к~обеспечиваемой} методом, который использует метрику 
точ\-ка--плос\-кость (рис.~2,\,\textit{а}). В~неконтролируемых условиях (при 
неравномерном освещении) предложенный метод показывает лучшую 
сходимость, чем указанные выше методы регистрации данных 
(рис.~2,\,\textit{б}). Дополнительно было установлено, что точность метода 
реконструкции зависит от числа особых точек в~RGB-D-кад\-ре нелинейным 
образом~--- в~виде функции с~одним ярко выраженным пиком для всех типов 
дескрипторов~\cite{21-voh, 22-voh}.
  
  
  
  Как известно, дескриптор HOG инвариантен к~неравномерному освещению 
и~фотометрическим преобразованиям, которые проявляются на больших 
сценах~\cite{15-voh}. Был проведен сравнительный анализ за\-ви\-си\-мости 
ско\-рости схо\-ди\-мости методов регистрации от выбора типа дескриптора 
и~условий проведения экспериментов (рис.~3). Было установлено, что 
в~конт\-ро\-ли\-ру\-емых условиях тип используемого дескриптора имеет 
ограниченное влияние на схо\-ди\-мость метода регистрации данных\linebreak 
(рис.~3,\,\textit{а}). В~неконтролируемых условиях ис\-поль\-зование дескриптора 
HOGs позволяет получить\linebreak лучшую схо\-ди\-мость в~сравнении с~другими 
дескрипторными методами: предложенный метод регистрации сходится уже 
после 11-й итерации, тогда как при использовании дескриптора ORB метод 
сходится только после 16-й итерации (рис.~3,\,\textit{б}).

 На рис.~4 показаны 
результаты применения предложенного метода регистрации для 
по\-сле\-до\-ва\-тель\-ности ключевых RGB-D-кадров из базы 
дан-\linebreak\vspace*{-10pt}

\pagebreak

\end{multicols}

\begin{figure*} %fig3
   \vspace*{1pt}
    \begin{center}  
  \mbox{%
 \epsfxsize=163.096mm 
 \epsfbox{voh-3.eps}
 }
\end{center}
\vspace*{-10pt}
  \Caption{Сравнение скорости сходимости в~зависимости от типа дескриптора и~условий 
наблюдения: (\textit{а})~изменение скорости сходимости в~контролируемых условиях; 
(\textit{б})~изменение скорости сходимости в~неконтролируемых условиях; \textit{1}~--- 
SIFT; \textit{2}~--- SURF; \textit{3}~--- ORB; \textit{4}~--- HOGs}
  %\end{figure*}
 % 
 % \begin{figure*} %fig4
   \vspace*{15pt}
    \begin{center}  
  \mbox{%
 \epsfxsize=163mm 
 \epsfbox{voh-4.eps}
 }
\end{center}
\vspace*{-10pt}
  \Caption{Реконструкция трехмерного объекта на основе предложенного метода 
регистрации данных с~разных точек обзора}
  \end{figure*}
  

\begin{multicols}{2}

\noindent
ных NYU Depth 
Dataset: исходное облако точек визуально совмещено с~целевым облаком точек 
на примере статуи головы человека. Недостаток классического метода 
ICP~\cite{6-voh}~--- большая вычислительная слож\-ность. Проекционные 
методы могут сократить вы\-чис\-ли\-тель\-ную слож\-ность метода регистрации ICP 
с~$O(N_S\log(N_T))$ для метода ICP с~k-D-де\-ре\-вом до~$O(N_S)$ для метода 
ICP c~ограничением в~виде сферы или треугольника.
  
  
  
  Для ускорения процедуры сопоставления данных в~работе используется 
пирамидальный подход, основанный на низкочастотной фильт\-ра\-ции 
и~по\-сле\-ду\-ющей дискретизации результирующего изоб\-ра\-же\-ния/об\-ла\-ка 
точек. Вы\-чис\-ли\-тель\-ная слож\-ность пред\-ла\-га\-емо\-го метода регистрации может 
быть оценена сле\-ду\-ющим образом: $n_1+ n_2O_1/F^1+\cdots\linebreak \cdots +  
n_kO_1/F^{k-1}$,  где $k$~--- чис\-ло шагов дискретизации; $n_i$~--- чис\-ло 
итераций в~методе регистрации, выполненных на шаге~$k$; $O_1$~--- 
вы\-чис\-ли\-тель\-ная слож\-ность первого шага алгоритма; $F$~--- параметр, 
опре\-де\-ля\-ющий разбиение RGB-D-кадра.

\vspace*{-6pt}

\section{Выводы}

  В работе предложен комбинированной метод решения вариационной задачи 
точка--точка в~замк\-ну\-той форме для аффинных преобразований, который 
создает основу для распространения метода Хорна на случай с~неригидными 
объектами на сцене. Было проведено сравнение пред\-ла\-га\-емо\-го метода 
регистрации данных с~методом Хорна для метрики точ\-ка--точ\-ка 
с~экстраполяцией и~без экстраполяции, а~также с~методом регистрации для\linebreak 
метрики точ\-ка--плос\-кость. В~результате компьютерного моделирования было 
установлено, что применение визуально связанных характеристик для решения 
вариационной задачи алгоритма ICP позво\-ля\-ет улучшить сходимость метода 
в~неконтролируемых условиях. Использование визуально\linebreak связанных 
характеристик изображений поз\-во\-ля\-ет решить проб\-ле\-му за\-ви\-си\-мости 
результата решения вариационной задачи от пра\-виль\-ности выбора начальных 
значений. Двумерный дескриптор HOGs обладает лучшими характеристиками 
по сравнению с~известными дескрипторами при малых поворотах в~об\-ласти 
сцены. Предложенный метод используется для регистрации облаков точек 
с~произвольным пространственным разрешением и~масштабом относительно 
друг друга, дает точные оценки для слож\-ных крупномасштабных сцен.
  
{\small\frenchspacing
 {\baselineskip=11.5pt
 \addcontentsline{toc}{section}{References}
 \begin{thebibliography}{99}
\bibitem{1-voh}
\Au{Vidal-Calleja T.\,A., Berger~C., Sola~J., Lacroix~S.} Large scale multiple 
robot visual mapping with heterogeneous landmarks in semi-structured terrain~// 
J.~Robotics Autonomous Systems, 2011. Vol.~59. Iss.~9. P.~654--674. doi: 
10.1016/j.robot.2011.05.008.
\bibitem{2-voh}
\Au{Vokmintsev A., Timchenko~M., Yakovlev~K.} Simultaneous localization and 
mapping in unknown environment using dynamic matching of images and 
registration of point clouds~// 2nd Conference (\mbox{International}) on Industrial 
Engineering, Applications and Manufacturing.~--- IEEE, 2017. Art. ID 7910967. 
6~p. doi: 10.1109/ICIEAM.2016.7910967.
\bibitem{3-voh}
\Au{Bokovoy A., Yakovlev~K.} Sparse 3D point-cloud map upsampling and noise 
removal as a~vSLAM post-processing step: Experimental evaluation~// 
Interactive collaborative robotics~/ 
Eds. A.~Ronzhin, G.~Rigoll, R.~Meshcheryakov.~--- Lecture 
notes in computer science ser.~--- Springer, 2018. Vol.~11097. P.~23--33.

\bibitem{5-voh}
\Au{Tam G., Cheng~Z.-Q., Lai~Y.-K., Langbein~F., Liu~Y., Marshall~D., 
Martin~R., Rosin~P.} Registration of 3D point clouds and meshes: A~survey 
from rigid to nonrigid~// IEEE~T. Vis. Comput. Gr., 2013. Vol.~19. Iss.~7. 
P.~1199--1217. doi: 10.1109/tvcg.2012.310.

\bibitem{4-voh}
\Au{Picos K., Diaz-Ramirez~V.\,H., Kober~V., Montemayor~A.\,S., 
Pantrigo~J.\,J.} Accurate three-dimensional pose recognition from monocular 
images using template matched filtering~// Opt. Eng., 2016. Vol.~55. 
Iss.~6. Art. ID 063102. doi: 10.1117/1.oe.55.6.063102.

\bibitem{6-voh}
\Au{Besl P., McKay~N.} A~method for registration of \mbox{3-D} shapes~// 
IEEE~T. Pattern Anal., 1992. Vol.~14. Iss.~2. P.~239--256. doi: 
10.1109/34.121791.
\bibitem{7-voh}
\Au{Cheng~S., Marras~I., Zafeiriou~S., Pantic~M.} Statistical non-rigid ICP 
algorithm and its application to 3D face alignment~// Image Vision Comput., 2017. 
Vol.~58. P.~3--12. doi: 10.1016/j.imavis.2016.10.007.
\bibitem{8-voh}
\Au{Horn B.} Closed-form solution of absolute orientation using unit 
quaternions~// J.~Opt. Soc. Am.~A, 1987. Vol.~4. Iss.~4.  
P.~629--642. doi: 10.1364/josaa.4.000629.
\bibitem{9-voh}
\Au{Horn B., Hilden H., Negahdaripour~S.} Closed-form solution of absolute 
orientation using orthonormal matrices~// J.~Opt. Soc. Am.~A, 1988. 
Vol.~5. Iss.~7. P.~1127--1135. doi: 10.1364/josaa.5.001127.
\bibitem{10-voh}
\Au{Khoshelham K.} Closed-form solutions for estimating a~rigid motion from 
plane correspondences extracted from point clouds~// ISPRS J.~Photogramm., 
2016. Vol.~114. Р.~78--91. doi: 10.1016/j.isprsjprs.2016.01.010.
\bibitem{11-voh}
\Au{Du S., Liu J., Zhang~C., Zhu~J., Li~K.} Probability iterative closest point 
algorithm for m-D point set registration with noise~// Neurocomputing, 2015. 
Vol.~157. Iss.~1. Р.~187--198. doi: 10.1016/j.neucom.2015.01.019.
\bibitem{12-voh}
\Au{Cheng S., Marras I., Zafeiriou~S.} Active nonrigid ICP algorithm~// 11th 
IEEE Conference (International) and Workshops on Automatic Face and Gesture 
Recognition Proceedings.~--- IEEE, 2015. Art. ID 7163161. 8~p. doi: 
10.1109/FG.2015.7163161.
\bibitem{13-voh}
\Au{Echeagaray-Patron B.\,A., Kober~V., Karnaukhov~V., Kuznetsov~V.} 
A~method of face recognition using 3D facial surfaces~// J.~Commun. 
Technol. El., 2017. Vol.~62. Iss.~6. Р.~648--652. doi: 
10.1134/s1064226917060067.
\bibitem{14-voh}
\Au{Low K.\,L.} Linear least-squares optimization for point-to-plane ICP surface 
registration.~--- Chapel Hill, NC, USA: University of 
North Carolina at Chapel Hill, Department of Computer Science, 2004. 
 Technical Report TTR04-004.
 {\sf 
https://www.comp.nus.edu.sg/$\sim$lowkl/\linebreak publications/lowk\_point-to-plane\_icp\_techrep.pdf}.
\bibitem{15-voh}
\Au{Вохминцев А.\,В., Соченков~И.\,В., Кузнецов~В.\,В., Тихоньких~Д.\,В.} 
Распознавание лиц на основе алгоритма сопоставления изображений 
с~рекурсивным вычислением гистограмм направленных градиентов~// 
Докл.  Акад. наук, 2016. Т.~466. №\,3. 
С.~261. doi: 10.7868/ S0869565216030087.


\bibitem{16-voh}
\Au{Diaz-Escobar J., Kober~V.} A~robust HOG-based descriptor for pattern 
recognition~// Proc. SPIE, 2016. Vol.~9971.
 Art. ID 99712A. doi: 10.1117/12.2237963.
\bibitem{17-voh}
\Au{Vokhmintcev A., Yakovlev~K.} A~real-time algorithm for mobile robot 
mapping based on rotation-invariant descriptors and ICP~// Comm. 
Comp. Inf. Sc., 2016. Vol.~661. P.~357--369.
\bibitem{18-voh}
\Au{Silberman N., Kohli~P., Hoiem~D., Fergus~R.} NYU Depth Dataset~V2. {\sf 
https://cs.nyu.edu/$\sim$silberman/datasets/\linebreak  nyu\_depth\_v2.html}.
\bibitem{19-voh}
\Au{Silberman N., Hoiem~D., Kohli~P., Fergus~R.} Indoor segmentation and 
support inference from RGBD images~// Computer 
vision~/ Eds. A.\,W.~Fitzgibbon, S.~Lazebnik, P.~Perona,
\textit{et al.}~--- Lecture notes in computer science ser.~--- 2012. Vol.~7576.  
P.~746--760.
\bibitem{20-voh}
\Au{Vokhmintcev A., Botova~T., Sochenkov~I., Sochenkova~A., Makovetskii~A.} 
Robot mapping algorithm based on Kalman filtering and symbolic tags~// Proc. 
SPIE, 2017. Vol.~10396. Art. 
ID~103962I. doi: 10.1117/12.2273562.
\bibitem{21-voh}
\Au{Vokhmintcev A., Timchenko~M., Melnikov~A., Kozko~A., Makovetskii~A.} 
Robot path planning algorithm based on symbolic tags in dynamic environment~// 
Proc. SPIE, 2017. Vol.~10396. Art. 
ID~103962E. doi: 10.1117/ 12.2273279.
\bibitem{22-voh}
\Au{Sochenkov I., Vokhmintsev~A.} Visual duplicates image search for  
a~non-cooperative person recognition at a~distance~// Procedia Engineer., 
2015. Vol.~129. Р.~440--445. doi: 10.1016/j.proeng.2015.12.147.
 \end{thebibliography}

 }
 }

\end{multicols}

\vspace*{-3pt}

\hfill{\small\textit{Поступила в~редакцию 25.02.19}}

\vspace*{8pt}

%\pagebreak

%\newpage

%\vspace*{-28pt}

\hrule

\vspace*{2pt}

\hrule

%\vspace*{-2pt}

\def\tit{SIMULTANEOUS LOCALIZATION AND~MAPPING 
METHOD IN~THREE-DIMENSIONAL SPACE BASED 
ON~THE~COMBINED SOLUTION OF~THE~POINT--POINT VARIATION 
PROBLEM ICP FOR~AN~AFFINE TRANSFORMATION}


\def\titkol{Simultaneous localization and~mapping 
method in~three-dimensional space based 
on~the~combined solution}
% of~the~point--point variation 
%problem ICP for~an~affine transformation}

\def\aut{A.\,V.~Vokhmintcev$^{1,2}$, A.\,V.~Melnikov$^2$, and~S.\,A.~Pachganov$^2$}

\def\autkol{A.\,V.~Vokhmintcev, A.\,V.~Melnikov, and~S.\,A.~Pachganov}

\titel{\tit}{\aut}{\autkol}{\titkol}

\vspace*{-11pt}


\noindent
$^1$Chelyabinsk State University, 129 Br.~Kashirinyh Str., Chelyabinsk 454001, 
Russian Federation


\noindent
$^2$Ugra State University, 16~Chekhov Str., Khanty-Mansiysk 628012, Russian 
Federation

\def\leftfootline{\small{\textbf{\thepage}
\hfill INFORMATIKA I EE PRIMENENIYA~--- INFORMATICS AND
APPLICATIONS\ \ \ 2020\ \ \ volume~14\ \ \ issue\ 1}
}%
 \def\rightfootline{\small{INFORMATIKA I EE PRIMENENIYA~---
INFORMATICS AND APPLICATIONS\ \ \ 2020\ \ \ volume~14\ \ \ issue\ 1
\hfill \textbf{\thepage}}}

\vspace*{3pt} 
     


\Abste{Simultaneous localization and mapping is a~problem in which frame data are 
used as the only source of external information to define the position of a~moving 
camera in space and at the same time, to reconstruct a~map of the study area. 
Nowadays, this problem is considered solved for the construction of two-dimensional 
maps for small static scenes using range sensors such as lasers or sonar. However, 
for dynamic, complex, and large-scale scenes, the construction of an accurate  
three-dimensional map of the surrounding space is an active area of research. To 
solve this problem, the authors propose a~solution of the point--point problem 
for  an affine transformation and develop a~fast iterative algorithm 
for point  clouds registering in three-dimensional space. The performance and computational complexity 
of the proposed method are presented and discussed by an example of reference data. 
The results can be applied  for navigation tasks of a~mobile robot
in real-time.}

\KWE{registration problem; localization; simultaneous localization and mapping; 
affine transformation; two-dimensional descriptors; iterative closest point}



\DOI{10.14357/19922264200114} 

\vspace*{-6pt}

\Ack
\noindent
This work was partially supported by the Russian Foundation for Basic Research 
(grant 18-37-20032) and by the Russian Science Foundation (project  
No.\,15-19-10010).

\vspace*{9pt}

  \begin{multicols}{2}

\renewcommand{\bibname}{\protect\rmfamily References}
%\renewcommand{\bibname}{\large\protect\rm References}

{\small\frenchspacing
 {%\baselineskip=10.8pt
 \addcontentsline{toc}{section}{References}
 \begin{thebibliography}{99}
 
 %\vspace*{-3pt}
\bibitem{1-voh-1}
\Aue{Vidal-Calleja, T.\,A., C.~Berger, J.~Sola, and S.~Lacroix.} 2011. Large 
scale multiple robot visual mapping with heterogeneous landmarks in semi-
structured terrain. \textit{J.~Robotics Autonomous Systems} 59(9):654--674. doi: 
10.1016/j.robot.2011.05.008.
\bibitem{2-voh-1}
\Aue{Vokmintsev, A., M.~Timchenko, and K.~Yakovlev.} 2017. Simultaneous 
localization and mapping in unknown\linebreak environment using dynamic matching of 
images and registration of point clouds. \textit{2nd  Conference 
(International) on Industrial Engineering, Applications and Manufacturing}. 
IEEE. Art. 
ID 7910967. 6~p. doi: 10.1109/ ICIEAM.2016.7910967.
\bibitem{3-voh-1}
\Aue{Bokovoy, A., and K.~Yakovlev.} 2018. Sparse 3D point-cloud map 
upsampling and noise removal as a~vSLAM post-processing step: Experimental 
evaluation. \textit{Interactive collaborative 
robotics}. Eds. A.~Ronzhin, G.~Rigoll, and R.~Meshcheryakov.
Lecture notes in computer science ser. Springer. 11097:23--33.

\bibitem{5-voh-1}
\Aue{Tam, G., Z.-Q.~Cheng, Y.-K.~Lai, F.~Langbein, Y.~Liu, D.~Marshall, 
R.~Martin, and P.~Rosin.} 2013. Registration of 3D point clouds and meshes: 
A~survey from rigid to nonrigid. \textit{IEEE~T. Vis. Comput. Gr.}  
19(7):1199--1217. doi: 10.1109/tvcg.2012.310.

\bibitem{4-voh-1}
\Aue{Picos, K., V.\,H.~Diaz-Ramirez, V.~Kober, A.\,S.~Montemayor, and 
J.\,J.~Pantrigo.} 2016. Accurate three-dimensional pose recognition from 
monocular images using template matched filtering. \textit{Opt. 
Eng.} 55(6):063102. doi: 10.1117/1.oe.55.6.063102.


\bibitem{6-voh-1}
\Aue{Besl, P., and N.~McKay.} 1992. A~method for registration of 3-D shapes. 
\textit{IEEE~T. Pattern Anal.} 14(2):239--256. 
doi: 10.1109/34.121791.
\bibitem{7-voh-1}
\Aue{Cheng, S., I.~Marras, S.~Zafeiriou, and M.~Pantic.} 2017. Statistical 
non-rigid ICP algorithm and its application to 3D face alignment. \textit{IEEE 
Image Vision Comput.} 58:3--12. doi: 10.1016/j.imavis.2016.10.007.
\bibitem{8-voh-1}
\Aue{Horn, B.} 1987. Closed-form solution of absolute orientation using unit 
quaternions. \textit{J.~Opt. Soc. Am.~A} 4(4):629--642. doi: 
10.1364/josaa.4.000629.
\bibitem{9-voh-1}
\Aue{Horn, B., H.~Hilden, and S.~Negahdaripour.} 1988. Closed-form solution 
of absolute orientation using orthonormal matrices. \textit{J.~Opt. Soc. 
Am.~A} 5(7):1127--1135. doi: 10.1364/JOSAA.5.001127.
\bibitem{10-voh-1}
\Aue{Khoshelham, K.} 2016. Closed-form solutions for estimating a rigid motion 
from plane correspondences extracted from point clouds. \textit{J.~ISPRS 
Photogramm.} 114:78--91. doi: 
10.1016/j.isprsjprs.2016.01.010.
\bibitem{11-voh-1}
\Aue{Du, S., J.~Liu, C.~Zhang, J.~Zhu, and K.~Li.} 2015. Probability iterative 
closest point algorithm for m-D point set registration with noise. 
\textit{Neurocomputing} 157(1):187--198. doi: 10.1016/j.neucom.2015.01.019.
\bibitem{12-voh-1}
\Aue{Cheng, S., I.~Marras, and S.~Zafeiriou.} 2015. Active nonrigid ICP 
algorithm. \textit{IEEE 11th  Conference (International) and Workshops on 
Automatic Face and Gesture Recognition Proceedings}. Art. ID 7163161. 8~p.
doi: 10.1109/FG.2015.7163161.
\bibitem{13-voh-1}
\Aue{Echeagaray-Patron, B.\,A., V.~Kober, V.~Karnaukhov, and V.~Kuznetsov.} 
2017. A~method of face recognition using 3D facial surfaces. 
\textit{J.~Commun. Technol. El.} 62(6):648--652. doi: 
10.1134/s1064226917060067.
\bibitem{14-voh-1}
\Aue{Low, K.\,L.} 2004. Linear least-squares optimization for point-to-plane ICP 
surface registration.  Chapel Hill, NC: University of 
North Carolina at Chapel Hill, Department of Computer Science. 
Technical Report TTR04-004. Available at: 
{\sf  
https://www.comp.nus.edu.sg/$\sim$lowkl/\linebreak publications/lowk\_point-to-plane\_icp\_techrep.pdf} (accessed December~17, 2019).
\bibitem{15-voh-1}
\Aue{Vokhmintcev, A.\,V., I.\,V.~Sochenkov, V.\,V.~Kuznetsov, and 
D.\,V.~Tikhonkikh.} 2016. Face recognition based on a~matching algorithm with 
recursive calculation of oriented gradient histograms. \textit{Doklady 
Mathematics}  
93(1):37--41. doi: 10.1134/s1064562416010178.
\bibitem{16-voh-1}
\Aue{Diaz-Escobar, J., and V.~Kober.} 2016. A~robust HOG-based descriptor 
for pattern recognition. \textit{Proc. SPIE} 9971:99712A. doi: 10.1117/12.2237963.
\bibitem{17-voh-1}
\Aue{Vokhmintcev, A., and K.~Yakovlev.} 2016. A~real-time algorithm for 
mobile robot mapping based on rotation-invariant descriptors and ICP. 
\textit{Comm. Comp. Inf. Sc.} 661:357--369. 
\bibitem{18-voh-1}
\Aue{Silberman, N., P.~Kohli, D.~Hoiem, and R.~Fergus}. NYU depth dataset 
V2. Available at: {\sf 
https://cs.nyu.edu/\linebreak $\sim$silberman/datasets/nyu\_depth\_v2.html} (accessed 
December~17, 2019).
\bibitem{19-voh-1}
\Aue{Silberman, N., D.~Hoiem, P.~Kohli, and R.~Fergus.} 2012. Indoor 
segmentation and support inference from RGBD Images. \textit{Computer vision}. 
Eds. A.\,W.~Fitzgibbon, S.~Lazebnik, P.~Perona,
\textit{et al.}
Lecture notes in computer science ser.  
7576:746--760. 
\bibitem{20-voh-1}
\Aue{Vokhmintcev, A., T.~Botova, I.~Sochenkov, A.~Sochenkova, and 
A.~Makovetskii.} 2017. Robot mapping algorithm based on Kalman filtering and 
symbolic tags. \textit{Proc. SPIE} 10396:103962I. doi: 10.1117/12.2273562.
\bibitem{21-voh-1}
\Aue{Vokhmintcev, A., M.~Timchenko, A.~Melnikov, A.~Kozko, and 
A.~Makovetskii.} 2017. Robot path planning algorithm based on symbolic tags in 
dynamic environment. \textit{Proc. SPIE} 10396:103962E. doi: 10.1117/12.2273279.
\bibitem{22-voh-1}
\Aue{Sochenkov, I., and A.~Vokhmintsev.} 2015. Visual duplicates image search 
for a~non-cooperative person recognition at a~distance. \textit{Procedia 
Engineer.} 129:440--445. doi: 10.1016/j.proeng.2015.12.147.
\end{thebibliography}

 }
 }

\end{multicols}

%\vspace*{-7pt}

\hfill{\small\textit{Received February 25, 2019}}

%\pagebreak

%\vspace*{-22pt}

\Contr

\noindent
\textbf{Vokhmintcev Alexander V.} (b.\ 1978)~--- 
Candidate of Science (PhD) in technology, associate professor; 
head of laboratory, Chelyabinsk State University, 
129~Br.~Kashirinyh Str., Chelyabinsk 454001, Russian Federation; 
associate professor, Ugra State University, 16~Chekhov Str.,
 Khanty-Mansiysk, 628012, Russian Federation; 
\mbox{vav@csu.ru}

\vspace*{3pt}

\noindent
\textbf{Melnikov Andrey V.} (b.\ 1956)~--- Doctor of Science in technology, 
professor, Ugra State University, 16~Chekhov Str., Khanty-Mansiysk 628012, 
Russian Federation; \mbox{melnikovav@uriit.ru}

\vspace*{3pt}

\noindent
\textbf{Pachganov Stepan A.} (b.\ 1994)~---
 PhD student, Ugra State University, 16~Chekhov Str., Khanty-Mansiysk 628012, 
Russian Federation; \mbox{pachganovsa@uriit.ru}
     
\label{end\stat}

\renewcommand{\bibname}{\protect\rm Литература}    %14
\def\stat{kozerenko}

\def\tit{КОГНИТИВНО-ЛИНГВИСТИЧЕСКИЕ ПРЕДСТАВЛЕНИЯ 
В~СИСТЕМАХ ОБРАБОТКИ ТЕКСТОВ}

\def\titkol{Когнитивно-лингвистические представления 
в~системах обработки текстов}

\def\autkol{Е.\,Б.~Козеренко, И.\,П.~Кузнецов}
\def\aut{Е.\,Б.~Козеренко$^1$, И.\,П.~Кузнецов$^2$}

\titel{\tit}{\aut}{\autkol}{\titkol}

%{\renewcommand{\thefootnote}{\fnsymbol{footnote}}\footnotetext[1]
%{Работа выполнена при поддержке Российского фонда фундаментальных
%исследований, проект~10-01-00480. Статья написана на основе материалов доклада, 
%представленного на IV Международном семинаре <<Прикладные задачи теории вероятностей 
%и математической статистики, связанные с моделированием информационных систем>> 
%(зимняя сессия, Аоста, Италия, январь--февраль 2010 г.).}}

\renewcommand{\thefootnote}{\arabic{footnote}}
\footnotetext[1]{Институт проблем информатики Российской академии наук, kozerenko@mail.ru}
\footnotetext[2]{Институт проблем информатики Российской академии наук, igor-kuz@mtu-net.ru}


\Abst{Рассмотрены вопросы проектирования и развития 
семантико-синтаксических и лексико-семантических представлений в 
лингвистических процессорах ряда систем, основанных на аппарате расширенных 
семантических сетей (РСС). Системы этого класса создаются для извлечения знаний из 
текстов на естественных языках, отображения извлеченных сущностей и связей в 
структуры базы знаний (БЗ) и использования знаний для поддержки экспертных 
аналитических решений в различных сферах приложения. В~фокусе внимания 
находятся ин\-же\-нер\-но-линг\-ви\-сти\-че\-ские представления, позволяющие 
построить целостную работающую лингвистическую модель, которая 
модифицируется в зависимости от конкретной задачи: от <<тяжелой>> формы на 
основе детальных глубинных представлений до фокусных редуцированных 
оболочек, настроенных на узкую предметную область (ПО) и ограниченный язык 
общения. Особое внимание уделяется способам описания 
дис\-три\-бу\-тив\-но-транс\-фор\-ма\-ци\-он\-ных признаков языковых объектов.}

\KW{интеллектуальные системы; семантические представления; лингвистические 
процессоры; обработка естественного языка; извлечение знаний}

       \vskip 14pt plus 9pt minus 6pt

      \thispagestyle{headings}

      \begin{multicols}{2}

      \label{st\stat}

\section{Введение}

     Данная работа посвящена проблемам создания\linebreak 
     когни\-тив\-но-линг\-ви\-сти\-че\-ских моделей естественного языка для 
различных классов информационных систем и описанию опыта создания 
линг\-ви\-сти\-че\-ских представлений для интеллектуальных\linebreak технологий 
обработки текстов. Вопросы извлечения знаний из текстов и создания модели 
естественного языка рассматриваются в единстве. В центре внимания будут 
находиться лингвистические процессоры интеллектуальных систем, 
разработанных на основе аппарата \textit{расширенных семантических 
сетей}~[1--5]. %\cite{1koz}--\cite{3koz}, \cite{18koz}--\cite{19koz}. 
Будем 
называть их \textit{РСС-сис\-те\-мы}. Эти системы создавались коллективом 
разработчиков, включая авторов данной статьи в Институте проб\-лем 
информатики РАН на протяжении целого ряда лет в рамках 
исследовательских проектов и прикладных систем, ориентированных на 
конкретные ПО заказчиков. Можно выделить четыре 
поколения РСС-систем. Ко\-гни\-тив\-но-линг\-ви\-сти\-че\-ские 
представления, заложенные в основу систем этого класса, прошли 
определенный эволюционный путь. 
     
     Интеллектуальные РСС-сис\-те\-мы содержат развитые \textit{базы 
знаний}, при этом знания представлены в виде записей на языке 
РСС, называемых 
     \textit{РСС-струк\-ту\-ра\-ми}. Лингвистические знания, таким 
образом, являются частным случаем <<знаний>> и также представлены в 
виде записей на языке РСС. Основным 
конструктивным элементом РСС\linebreak является именованный $N$-мест\-ный 
предикат, на\-зы\-ва\-емый <<\textit{фрагментом}>>. Все множество языковых 
объектов задается в виде системы пре\-ди\-кат\-но-ак\-тант\-ных структур, при этом 
поддерживаются механизмы представления вложенных структур, что дает 
очень мощные изобразительные возможности для описания объектов 
различных языковых уровней. Очень важными факторами являются 
однородность и единообразие лингвистических представлений. 
     
     В процессе анализа и синтеза предложений естественного языка 
используется фор\-маль\-но-грам\-ма\-ти\-че\-ский аппарат, сходный с 
грамматиками зависимостей. При этом подходе опорными элементами 
служат слова и конструкции, выполняющие роль предикатов в предложении, 
и результатом анализа предложения должен стать один предикат, 
соответствующий сказуемому рассматриваемого предложения (т.\,е.\ 
основному глаголу в личной форме или другому основному предикатному 
выражению). Таким образом, в процессе анализа происходит выявление 
\textit{когнитивных опор} предложения: <<слов-дейст\-вий>> и 
     <<слов-от\-но\-ше\-ний>>, т.\,е.\ глаголов и других слов, имеющих 
синтактико-семантические валентности. Примером <<слов-от\-но\-ше\-ний>> 
могут служить, например, слова <<отец>>, <<друг>> и~т.\,п., т.\,е.\ в данном 
случае <<отношения>> (или \textit{функции}~--- в терминах языка логики 
предикатов 1-го порядка)~--- это слова, которые задают сильные, четко 
выраженные син\-так\-ти\-ко-се\-ман\-ти\-че\-ские ожидания. 
     
     Семантический анализ в ин\-же\-нер\-но-линг\-ви\-сти\-че\-ском 
понимании~--- это процесс перевода ес\-тест\-вен\-но-язы\-ко\-вых 
выражений во <<внутренние>> структуры БЗ, в 
рассматриваемой ситуации этими <<внутренними>> структурами являются 
записи на языке РСС. Таким образом, структуры БЗ~--- это код смысла в 
интеллектуальных информационных системах подобного рода. 
     
     В работе рассматриваются ин\-же\-нер\-но-линг\-ви\-сти\-че\-ские 
решения в системах с <<пол\-ным>> линг\-ви\-сти\-че\-ским анализом~--- это 
     сис\-те\-мы 1-го и 2-го поколения: ДИЕС1, ДИЕС2, 
     Логос-Д~\cite{2koz, 3koz}~--- и сис\-те\-мах с <<фактографическим>> 
подходом: интеллектуальных системах поддержки аналитических решений 
(ИСПАР)~\cite{18koz, 19koz}, где целью анализа является выделение 
сущностей и связей из текстов,~--- это системы 3-го и 4-го поколения. 

\section{Процесс концептуально-лингвистического моделирования 
в системах, основанных на аппарате расширенных семантических сетей}
     
\subsection{Центральные вопросы семантического моделирования} %2.1
     
     Концептуально-лингвистическое моделирование (КЛМ)~--- это 
процесс построения ес\-тест\-вен\-но-язы\-ко\-вой модели ПО (рис.~1), синтезирующий в себе подходы 
концептуального и лингвистического моделирования~[1--3]. 
По\-стро\-ение концептуально-лингвистической модели некоторой 
ПО подразделяется на следующие этапы:
     \begin{itemize}
     \item построение собственно концептуальной модели, т.\,е.\ вычленение 
базовых понятий, организация их в ро\-до-ви\-до\-вые деревья и определение 
связей между ними;
     \item разработка идеографического словаря ПО, т.\,е.\ 
лексическое наполнение концептуальной модели;
     \item ввод базовых правил, описывающих на естественном языке 
<<модель мира>>, релевантную данной ПО.
     \end{itemize}
     
     
     Методика КЛМ на 
основе аппарата РСС базируется на следующих принципах:
     \begin{itemize}
\item модель должна быть <<открытой>>, т.\,е.\ поддерживать эффективный 
механизм расширения и обновления информации;
\begin{center} %fig1
%\vspace*{3pt}
\hspace*{-10.7158pt}\mbox{%
\epsfxsize=77.871mm
\epsfbox{koz-1.eps}
}\hspace{10.7158pt}
%\end{center}
\vspace*{4pt}
%\begin{center}
{{\figurename~1}\ \ \small{Процесс КЛМ}}
\end{center}
\vspace*{3pt}

%\bigskip
\addtocounter{figure}{1}
\item модель представления <<смысла>> должна учитывать факты 
экстралингвистической реаль\-ности, которые в виде правил и отношений 
составляют некоторую базовую <<модель мира>>, достраиваемую 
конкретными моделями ПО;
\item модель должна быть практичной, т.\,е.\ не перегруженной детальными 
описаниями связей и отношений между понятиями, чтобы обеспечить 
возможность ее реализации, но в то же время отражать всю релевантную 
конкретной задаче информацию.
\end{itemize}

     \begin{figure*} %fig2
%     \begin{center}
\hspace*{23mm}\{(ВЫРАБАТЫВА895\_\_)(DICSEM)\\
\hspace*{23mm}COORD(PROGNOZ1,RUS,ВЫРАБАТЫВА895\_\_,S50\_31\_51\_20,\%)\\
\hspace*{23mm}SUB(UNIV,0+)~SUB(UNIV,1+)~SUB(UNIV,2+)\\
\hspace*{23mm}ВЫРАБАТЫВ(0-,1-,2-/3+)~INFI(3-)~ПРИДЕТСЯ(3-)~ПРИДЕТСЯ(3$-$/4+) \\
\hspace*{23mm}FUT1(4$-$)~SUB(СРЕД,5+)
%\end{center}
%\vspace*{2pt}
\Caption{Пример записи представления глагола <<вырабатывать>> в семантическом 
словаре
\label{f2koz}}
%\vspace*{6pt}
\end{figure*}

     Реалистичный подход к постановке задачи диктует необходимость 
ограничения моделируемого подмножества естественного языка. Суть 
ограничений сводится к следующему:
     \begin{enumerate}[(1)]
     \item анализируемые текстовые материалы содержат 
экспертные знания из конкретных ПО (в разработанных 
авторами системах это были такие ПО, как диагностика 
брака при изготовлении микросхем, социальное прогнозирование, 
криминалистика и другие);
     \item в целях максимально возможного устранения 
неоднозначности словарь строится по модульному принципу: есть некоторая 
наиболее общая часть (1--2~уровня), которая достраивается специальными 
словарями для каж\-дой отдельной~ПО.
     \end{enumerate}
     
     Предлагаемая модель лексической семантики основана на принципе 
<<ядерного>> значения, реализуемого в контексте данной 
ПО, с последующим индуктивным наращиванием других значений (если 
они актуализируются в рас\-смат\-ри\-ва\-емых контекстах). Также используется 
таксономия, которая реализуется в виде иерархических деревьев классов 
слов. 
     
     Общая <<модель мира>> системы является основой для моделей ПО. 
Элементами этой модели служат классы слов, которые подразделяются на 
понятия/имена, отношения, действия, свойства, характеристики действий, 
временные и пространственные характеристики.
     
     Самым общим понятием является \textit{концепт}, или 
\textit{универсальный класс}, который подразделяется на объект, ситуацию, 
процесс и~др. 
     
     Слова, относящиеся к классам действий и отношений, представлены 
как се\-ман\-ти\-ко-син\-так\-си\-че\-ские фреймы, задающие 
     пре\-ди\-кат\-но-ак\-тант\-ные структуры (модель управления). Однако 
в описываемом подходе (назовем его РСС-под\-хо\-дом) существенно 
расширена область значений актантов. Суть расширения состоит, во-первых, 
в том, что в роли актантов могут выступать не только простые объекты, 
соответствующие отдельным словам, но и структурные объекты, 
представляющие словосочетания и фразы, а во-вторых, в том, что понятие 
падежа включает в себя не только семантические, но и синтаксические 
признаки.
     
     Подход, основанный на РСС, позволяет отражать произвольный 
уровень вложенности структур за счет пропозициональных вершин 
семантической сети. Это обеспечивает представление\linebreak сложных 
синтаксических конструкций фраз\linebreak естественного языка, а также позволяет 
отразить\linebreak структурный характер лексической семантики,\linebreak которая в 
предлагаемой модели имеет иерар\-хи\-че\-ски-се\-те\-вую структуру. 
Линг\-ви\-сти\-че\-ские зна-\linebreak ния пред\-став\-ле\-ны в системном словаре и 
декла\-ра\-тивных модулях линг\-ви\-сти\-че\-ско\-го процессора.\linebreak В РСС-сис\-те\-мах 
так\-же реализована функция динамически форми\-ру\-емо\-го семантического 
словаря, который на основе исходной лингвистической информации 
достраивается системой автоматически в процессе об\-ра\-бот\-ки конкретных 
текстов. На рис.~\ref{f2koz} пред\-став\-ле\-но \mbox{такое} <<внутреннее>> описание 
глагола в семантическом словаре. Этот словарь автоматически генерируется 
РСС-системами ДИЕС2, ЛОГОС-Д, ИКС в процессе обработки 
     естест\-вен\-но-язы\-ко\-вых \mbox{текстов}. 
     {\looseness=1
     
     }
     
     
\subsection{Особенности применения аппарата расширенных семантических сетей 
в~когнитивно-лингвистическом моделировании} %2.2
     
     Дадим краткое описание аппарата РСС и  
обос\-ну\-ем выбор именно этого метода представления для моделирования 
естественного языка. Классическое понятие семантической сети сводится к 
следующему: задаются некоторые вершины, соответствующие объектам,  
вершины связываются дугами, которые помечаются именами отношений. 
Однако с помощью подобных сетей оказывается трудно представлять 
сложные виды информации, например, когда объекты, связанные 
отношениями, образуют агрегаты и когда отношения связываются между 
собой отношениями и~др. Поэтому в сети вводятся вершины, 
соответствующие именам отношений, а также специальный композиционный 
элемент, называемый вершиной связи. Вершина связи как бы <<разрывает>> 
дугу и подсоединяется одним ребром к вершине-отношению, а другими 
ребрами~--- к вершинам-объектам. Расширенная семантическая сеть является развитием такого сорта 
сетей в направлении повышения изобразительных возможностей при 
сохранении свойства однородности.
     
     Основой РСС является множество вершин ($V$), из которых 
составляются элементарные фрагменты (ЭФ) вида
     $
     V_0(V_1,V_2,\ldots ,V_k/V_{k+1})
     $, 
     где
$V_0, V_1, V_2,\ldots , V_k, V_{k+1}>0$.
     
     
     Такой фрагмент представляет $k$-местное отношение. Позиции 
вершин в ЭФ определяют их роли. 
Вершина~$V_0$ ставится в соответствие имени отношения, 
вершины~$V_1$, $V_2$, \ldots , $V_k$~--- объектам, участ\-ву\-ющим в 
отношении, а вершина~$V_{k+1}$, отделенная косой линией,~--- всей 
совокупности упомянутых объектов с учетом их отношения. В~дальнейшем 
будем $V_{k+1}$ называть $C$-вершиной ЭФ.\linebreak 
Множество ЭФ образует РСС. 
С~помощью РСС представляются наборы отношений, различные ситуации, 
сце\-нарии. Сильной стороной РСС-под\-хо\-да является возможность 
однородного пред\-став\-ле\-ния как предметной (концептуальной), так и 
лингвистической информации, что обеспечивает эффективную обработку 
знаний и поддержание непротиворечи\-вости~БЗ.
          \begin{figure*} %fig3
     \vspace*{1pt}
\begin{center}
\mbox{%
\epsfxsize=125.039mm
\epsfbox{koz-3.eps}
}
\end{center}
\vspace*{-9pt}
     \Caption{Семантико-синтаксический анализ без выявления глагольных 
словоформ
      \label{f3koz}}
\vspace*{12pt}
 %     \end{figure*}
%            \begin{figure*} %fig4
           \vspace*{1pt}
\begin{center}
\mbox{%
\epsfxsize=103.129mm
\epsfbox{koz-4.eps}
}
\end{center}
\vspace*{-9pt}
      \Caption{Целостная семантическая структура предложения
      \label{f4koz}}
      \end{figure*}

     
     Посредством РСС в БЗ представлены лингвистические  и 
предметные знания. Обработка этих знаний осуществляется 
продукциями языка ДЕКЛ, на котором реализованы сле\-ду\-ющие шесть 
блоков: морфологического анализа, семанти\-ческого анализа слов, 
син\-так\-ти\-ко-се\-ман\-ти\-че\-ско\-го анализа форм, 
прагматических функций, организации системной активности и 
обратный лингвистический процессор. С~помощью продукций 
осущест\-вля\-ет\-ся последовательное преобразование сети~--- РСС. При этом 
проходятся фазы, соответствующие уровню понимания входного текста. 
Рас\-смот\-рим~их.
     \begin{enumerate}[1.]
     \item На первом шаге анализа строится 
пространственная структура предложения с морфологической информацией 
для каждого слова.\linebreak Каж\-дый член предложения представляется вершиной 
семантической сети. Вместо слова генерируется код (если слово 
многозначно, т.\,е.\ принадлежит к нескольким классам,~--- то более одного 
кода). Основой кода служит корень слова. На этом этапе предложение 
представляется в виде набора фрагментов типа LRR (специальных меток 
результатов 1-го этапа анализа), объединяемых в целостную структуру 
посредством вершины связи. Результат 1-го этапа постоянно обращается к 
словарю: <<Что значит данное слово?>>
     \item На втором этапе каждой вершине сопоставляется семантический 
класс и присваивается новый код. За словами (т.\,е.\ конкретными вершинами 
РСС) система видит объекты, действия, свойства, т.\,е.\ строит 
классификации. Производится се\-ман\-ти\-ко-син\-так\-си\-че\-ский анализ 
без выявления глагольных словоформ, при этом предложение представляется 
в виде совокупности фрагментов типа SEM и SEMD~--- специальных меток 
результатов 2-го этапа анализа (рис.~\ref{f3koz}).
     \item На третьем этапе происходит частичное <<сворачивание>> 
синтаксических структур в более компактные (например, свойство объекта и 
сам объект) с присваиванием нового кода и строится фрагмент для объекта, 
обладающего этим свойством.
     \begin{figure*}[b] %fig5
          \vspace*{12pt}
\begin{center}
\mbox{%
\epsfxsize=147.485mm
\epsfbox{koz-5.eps}
}
\end{center}
\vspace*{-9pt}
     \Caption{Глубинная структура предложений
      \label{f5koz}}
      \end{figure*}      
     \item На четвертом этапе выявляются отношения и действия и 
производится анализ непосредственного контекста на соответствие заданным 
семантическим падежам. Система проверяет, подходят ли объекты 
(концепты, понятия) на аргументные места данного действия или отношения. 
При этом отглагольные существительные (<<делатель>>, т.\,е.\ агент 
действия, или <<делание>>~--- процесс~--- анализируются как слова с 
двойной природой: вначале как действия, а затем как объекты). Результатом 
этого этапа является целостная семантическая структура предложения, 
которая представляется фрагментом типа SEMSTR~--- метки результата 4-го 
этапа анализа (рис.~\ref{f4koz}).
     \item На пятом этапе происходит анализ прагматики: установление 
кореференциальных отношений, частичное восстановление эллиптических 
конструкций, система производит дальнейшие действия с построенными 
фрагментами.
     \end{enumerate}

     
Система ДИЕС допускает ввод полисемичных форм глаголов. Для этого следует 
воспользоваться формальной записью лингвистических знаний. 
     В~сис\-те\-мах, основанных на РСС, все функции реализованы на 
единой основе~--- в рамках языков РСС и ДЕКЛ, которые были разработаны 
с ориентацией на задачи обработки естественного языка.

%\vspace*{-6pt}

\section{Представление семантики глаголов, глубинные 
и~поверхностные структуры}
     
     В процессе анализа выявляются семантические вершины предложения: 
происходит выявление <<слов-дей\-ст\-вий>>, т.\,е.\ глаголов, и 
     <<слов-от\-но\-ше\-ний>>. Что же является конструктивной основой\linebreak 
задания семантических представлений предикатных слов и выражений? Как 
убедительно показано в работе~\cite{4koz}, семантика глагола 
определяется его дис\-три\-бу\-тив\-но-транс\-фор\-ма\-ци\-он\-ны\-ми\linebreak 
свойствами. Поэтому смысл предикатных выражений должен кодироваться с 
учетом их дистрибутивных и трансформационных признаков. 
     
     Выдвинутая рядом лингвистов (Хомский, Филлмор) гипотеза о том, что 
все предложения имеют глубинные и поверхностные 
     структуры~[7--10], явилась очень продуктивным 
источником проектных решений при создании первых РСС-сис\-тем и 
развивалась в дальнейшем. 

В~тео\-ре\-ти\-ко-линг\-ви\-сти\-че\-ском 
понимании глубинная структура~--- это абстракция, содержащая все 
элементы, необходимые для образования поверхностных структур 
предложений со сходной семантикой. 

     В~ин\-же\-нер\-но-линг\-ви\-сти\-че\-ском понимании\linebreak глубинная 
структура~--- это запись на языке БЗ, например на языке РСС, 
которая может быть представлена в <<поверхностном>> виде на одном из 
естественных языков в результате конечного числа определенных 
преобразований. Например, предложения

\noindent
\begin{align*}    
(1)\ &\mbox{\textit{The programmer writes the code}}\\
(2)\ &\mbox{\textit{The code is written by the programmer}}
\end{align*}
имеют истоком одну глубинную структуру:

\medskip

\noindent
     \begin{verbatim}
  Programmer <---- write ----> Code
      agent                   object,
\end{verbatim}

\medskip

\noindent
хотя и отличаются своими поверхностными структурами. В~каждом из них 
имеется агент (the programmer), объект (the code) и действие (write).\linebreak Согласно 
концепции \textit{падежной грамматики} Филлмора~\cite{5koz} глубинная 
структура для обоих предложений инвариантна. Эту структуру можно 
представить в виде скобочной записи $V(\mathrm{AGENT}, \mathrm{OBJECT})$. В~графическом 
виде глубинная структура предложения также может быть представлена 
диаграммой в виде дерева, где отражены инвариантные отношения 
зависимости между предикатной вершиной и актантами (рис.~\ref{f5koz}), 
причем в таком представлении явным образом разграничиваются 
\textit{модальность} (MOD) и \textit{пропозиция} (PROP).
     

     В исходном варианте~\cite{5koz} теория признавала шесть падежей: 
агентив, инструменталис, датив, объектив, локатив и фактитив. По мере 
развития теории~\cite{8koz} происходило увеличение числа падежей, однако 
<<умножение>> количества падежей утяжеляет первоначальную 
конфигурацию, поэтому при построении инженерных семантических 
представлений требуется некоторый <<компромиссный>> вариант, 
сочетающий в себе необходимую полноту, с одной стороны, и простоту и 
гибкость, с другой.

\begin{figure*}[b] %fig6
\vspace*{24pt}
\begin{center}
\mbox{%
\epsfxsize=156.873mm
\epsfbox{koz-6.eps}
}
\end{center}
%\vspace*{-9pt}
\Caption{Обобщенное функциональное представление систем ИСПАР
\label{f6koz}}
\end{figure*}
     
%\vspace*{-6pt}

\section{Некоторые базовые аспекты построения многоязычных 
систем}
     
     Одним из приоритетных направлений развития РСС-сис\-тем является 
обеспечение обработки текстов на нескольких языках, прежде всего для 
рус\-ско-анг\-лий\-ской языковой пары. В системах 2-го поколения~--- ДИЕС2, 
ИКС, ЛОГОС-Д были реализованы лингвистические процессоры и словари 
для русского и английского языка, позволявшие обрабатывать тексты для 
ряда ПО. При этом поддерживался как режим ввода 
лингвистических знаний линг\-вис\-том-ана\-ли\-ти\-ком, так и 
автоматический режим самообучения системы по вводимым \mbox{текстам}. 
{\looseness=1

}

Проводились также эксперименты с итальянским и французским языком. 
При создании многоязычных систем авторы обращались к европейским 
языкам. Очевидно, что европейские языки обладают большим числом общих 
правил, чем любой из них с языками других групп. Но при этом все 
естественные языки обладают общей структурой на самом глубинном 
уровне. На этом уровне располагаются главные элементы естественного 
языка: \textit{предложение}, \textit{модальность}, \textit{пропозиция}.
     
     Моделирование смысловых представлений~--- это процесс, 
развивающийся в направлении от поверхностных семантических структур к 
глубинным. Поиск такого внутреннего представления смысла в условиях 
многоязычной ситуации является на\-прав\-ле\-ни\-ем развития методов 
     КЛМ на базе  РСС. 
     
%     \vspace*{-48pt}
     
\section{Интеллектуальные системы поддержки аналитических 
решений}
     
Системы РСС 3-го и 4-го поколения на\-прав\-ле\-ны на извлечение знаний 
в виде \textit{объектов}, или \textit{сущностей}, и связей между ними из 
пред\-мет\-но-ориен\-ти\-ро\-ван\-ных текстов на русском и английском 
языке~\cite{18koz, 19koz}.

    
В настоящее время во всем мире активно ведутся работы по созданию 
систем извлечения фактов из текстов на естественных языках~[11--14], создаются развитые тезаурусы и 
онтологии~\cite{17koz}. Сис\-те\-мы РСС функционально шире, поскольку 
имеют возможность не только извлекать факты, но и поддерживать 
механизмы логического анализа и экспертного вывода на основе 
извлеченных знаний. Сис\-те\-ма\-ми такого рода являются ИСПАР. В~целом это 
направление исследований требует дальнейшей проработки 
     лек\-си\-ко-се\-ман\-ти\-че\-ских представлений, создания 
     пред\-мет\-но-ориен\-ти\-ро\-ван\-ных семантических словарей. 

Обобщенное функциональное представление систем ИСПАР дано на 
рис.~\ref{f6koz}. 
     
     В рамках ИСПАР на основе РСС 
(\mbox{ИСПАР}--РСС) были реализованы полномасштабные и\linebreak пилотные 
проекты для ряда ПО: криминалистики, управления 
кадрами, мониторинга финансово-экономического кризиса и 
др.~\cite{18koz, 19koz}.

\section{Применение аппарата расширенных семантических сетей в~лингвистических 
исследованиях}
     
     В настоящее время в рамках проектов, на\-прав\-лен\-ных на создание 
открытых лингвистических ресурсов~\cite{20koz} для 
     на\-уч\-но-прак\-ти\-че\-ских целей, ведутся работы по выравниванию 
параллельных текстов научных статей, патентов и 
     фи\-нан\-со\-во-эко\-но\-ми\-че\-ских текстов. В~качестве одного из 
методов выравнивания используется РСС-под\-ход, поскольку он позволяет 
отразить глу\-бин\-но-се\-ман\-ти\-че\-ский уровень языковых структур. 

На  рис.~7 представлен фрагмент первого этапа лингвистического 
анализа в многоязычных системах. Для <<идеальной>> ситуации, когда 
структуры исходного текста и текста перевода практически совпадают, такая 
ситуация имеет место в меньшинстве случаев. Основные трудности 
возникают при наличии переводческих трансформаций в параллельных 
текстах. Особое внимание следует уделять гла\-голь\-но-имен\-ным 
трансформациям, например явлению \textit{номинализации}, поскольку она 
очень продуктивна для всех исследовавшихся языков.

     
     Ключевой задачей при разработке методов сопоставления 
параллельных текстов является выявление и детальное описание тех 
языковых трансформаций, которые имеют место при переводе 
     естест\-вен\-но-язы\-ко\-вых конструкций с одного языка на 
другой~\cite{9koz}, потому что далеко не всегда некое содержание 
передается струк\-тур\-но-по\-доб\-ны\-ми средствами в текстах на разных 
языках. Сравнительное исследование употребления различных частей речи в 
параллельных текстах на разных языках создает основу для выявления и 
описания языковых транс-\linebreak

\begin{center} %fig7
\vspace*{3pt}
\mbox{%
\epsfxsize=79.726mm
\epsfbox{koz-7.eps}
}
\end{center}
\vspace*{4pt}
%\begin{center}
{{\figurename~7}\ \ \small{Первый этап анализа параллельных текстов ($W_n$
обозначает словоформу с номером~$n$, $1\leq n\geq 5$)}}
%\end{center}
%\vspace*{9pt}

%\bigskip
\addtocounter{figure}{1}
      

\noindent 
формаций, при этом центральной трансформацией
является \textit{номинализация}. Явление номинализации
было исследовано в 
ряде работ отечественных и зарубежных лингвистов~[17--20]. 
Ближе всего к правильному, по мнению авторов данной статьи, 
пониманию этого явления следующие определения номинализации: 
<<конструкции\ldots называются номинализованными~--- в том смысле, что 
их естественно рассматривать как результат номинализации конструкций с 
предикативным употреблением глаголов и прилагательных>>; 
<<номинализация~--- это синтаксический процесс, который соотносит 
предложения с именными группами>>~\cite{9koz, 10koz}. Выявление 
номинализованных конструкций в параллельных научных и патентных 
текстах на русском, английском, французском и немецком языках в научных 
и патентных текстах и сопоставительное описание гла\-голь\-но-имен\-ных 
межъязыковых трансформаций~--- одна из центральных задач 
     ин\-же\-нер\-но-линг\-ви\-сти\-че\-ских исследований. 
     
     Следующей базовой трансформацией в исследуемых текстах на 
нескольких европейских языках является адъек\-тив\-но-ад\-вер\-би\-аль\-ное 
преобразование. Это означает, что при переводе с одного языка на другой 
происходит синтаксическое преобразование имен прилагательных в наречия 
и обратное преобразование~--- наречий в прилагательные. Установление 
семантических соответствий между этими языковыми объектами также 
возможно осуществить посредством аппарата~РСС. 
     
     При семантическом выравнивании непараллельных текстов, имеющих 
одну и ту же денотативную составляющую, аппарат РСС позволяет выявить в 
текстах когнитивные опоры (слова с сильной валентностью~--- 
     <<сло\-ва-дейст\-вия>> и <<сло\-ва-от\-но\-ше\-ния>>) и установить 
между ними семантические соответствия.

\section{Заключение}

     В данной работе представлен опыт создания и развития 
     когни\-тив\-но-линг\-ви\-сти\-че\-ских пред\-став\-ле\-ний в 
интеллектуальных информационных сис\-те\-мах, разработанных на основе 
аппарата РСС. Аппарат РСС 
обеспечивает мощные изобразительные возможности для описания всех 
уровней естественного языка, включая уровень 
     глу\-бин\-но-се\-ман\-ти\-че\-ских представлений и межъязыковых 
соответствий. Конкретные лингвистические процессоры, которые были 
созданы на основе этого подхода, прошли определенный путь развития и 
позволили выработать проектные решения для основных задач текущего 
этапа~--- извлечения и обработки содержательных знаний из текстов на 
естественных языках и сопоставления языковых структур в текстах на 
различных языках с учетом базовых трансформаций.
     
     Проблема извлечения и обработки знаний открывает перспективы 
развития интеллектуальных направлений компьютерной лингвистики, 
поскольку ее основной акцент смещен в сторону\linebreak глубинных представлений 
языка, в которых используются как грамматические (морфологические и 
синтаксические), так и семантические атрибуты для описания языковых 
объектов. Проводи-\linebreak мые авторами исследования параллельных текстов 
направлены также на рассмотрение этой проблемы~\cite{20koz}. 
Центральное место в проводящихся линг\-ви\-сти\-че\-ских исследованиях 
занимает изучение и формализация процессов трансформации языковых 
структур, особенно все варианты глагольно-но\-ми\-на\-тив\-ных трансформаций, 
создание развитых дис\-три\-бу\-тив\-но-транс\-фор\-ма\-ци\-он\-ных 
описаний предикатых структур для рассматриваемых языков. 
     
     Для задач извлечения знаний и создания \mbox{ИСПАР} 
     дис\-три\-бу\-тив\-но-транс\-фор\-ма\-ци\-он\-ные описания имеют 
особое значение, поскольку таким образом задаются все возможные способы 
перевода языковых структур в пре\-ди\-кат\-но-ар\-гу\-мент\-ные 
представления, которые затем используются в процедурах обработки знаний.

{\small\frenchspacing
{%\baselineskip=10.8pt
%\addcontentsline{toc}{section}{Литература}
\begin{thebibliography}{99}

     \bibitem{1koz}
     \Au{Кузнецов~И.\,П.}
     Семантические представления.~--- М.: Наука, 1986. 290~с.
     
     \bibitem{2koz}
     \Au{Козеренко~Е.\,Б.}
     Кон\-цеп\-ту\-аль\-но-линг\-вис\-ти\-че\-ское моделирование в среде 
интеллектуального редактора знаний ИКС~// Проблемы проектирования и 
использования баз знаний.~--- Киев: Ин-т кибернетики им.\ В.\,М.~Глушкова, 
1992. C.~73--79.
     
     \bibitem{3koz}
     \Au{Kozerenko~E.\,B.}
     Multilingual processors: A unified approach to semantic and syntactic 
knowledge presentation~// Conference (International ) on Artificial Intelligence 
IC-AI'2001 Proceedings. Las Vegas, Nevada, USA. June 25--28, 2001.~--- Las 
Vegas: CSREA Press, 2001. P.~1277--1282.

     \bibitem{18koz} %4
     \Au{Kuznetsov~I.\,P., Efimov~D.\,A., Kozerenko~E.\,B.}
     Tools for tuning the semantic processor to application areas~// ICAI'09 
Proceedings, WORLDCOMP'09. July 13--16, 2009. Las Vegas, Nevada, USA. 
Vol.~I.~--- Las Vegas: CRSEA Press, 2009. P.~467--472.
     
     \bibitem{19koz} %5
     \Au{Kuznetsov~I.\,P., Kozerenko~E.\,B., Kuznetsov~K.\,I., 
Timonina~N.\,O.}
     Intelligent system for entities extraction (ISEE) from natural language 
texts~// Workshop (International) on Conceptual Structures for Extracting Natural 
Language Semantics (Sense'09) at the 17th Conference 
(International ) on Conceptual Structures (ICCS'09) Proceedings. University Higher School of 
Economics. Moscow, Russia, 2009. P.~17--25.
     
     \bibitem{4koz} %6
     \Au{Апресян~Ю.\,Д.}
     Экспериментальное исследование семантики русского глагола.~--- М.: 
Наука, 1967.  252~с.
     
     \bibitem{5koz} %7
     \Au{Филлмор~Ч.}
     Дело о падеже~// Новое в зарубежной линг\-вистике, 1968. Вып.~X. С.~369--495.
     
     \bibitem{6koz} %8
     \Au{Хомский~Н.}
     Аспекты теории синтаксиса.~--- М.: МГУ, 1972.
     
     \bibitem{7koz} %9
     \Au{Хомский Н.}
     Язык и мышление.~--- М.: МГУ, 1972.
     
     
     \bibitem{8koz} %10
     \Au{Fillmore~C.}
     The case for case reopened~// Syntax and Semantics. Vol.~8.~--- N.Y.: 
Academic Press, 1977. 
     

          \bibitem{15koz} %11
     FASTUS: A cascaded finite-state trasducer for extracting information from 
natural-language text~// AIC, SRI International, Menlo Park, California, 1996. 
     
     \bibitem{16koz} %12
     \Au{Han~J., Pei~Y., Mao~R.}
     Mining frequent patterns without candidate generation: A frequent-pattern 
tree approach~// Data Mining and Knowledge Discovery, 2004. Vol.~8. No.\,1. 
P.~53--87.
     
     
     \bibitem{13koz} %13
     \Au{Cunningham~H.}
     Automatic information extraction~// Encyclopedia of Language and 
Linguistics. 2nd ed.~--- Elsevier, 2005.
     
     \bibitem{14koz} %14
     \Au{Han~J., Kamber~M.}
     Data mining: Concepts and techniques.~--- Morgan Kaufmann, 2006.
     
     
     \bibitem{17koz} %15
     \Au{Добров~Б.\,В., Лукашевич~Н.\,В.}
     Онтологии для автоматической обработки текстов: Описание понятий 
и лексических значений~// Компьютерная лингвистика и интеллектуальные 
технологии: Тр. межд. конф. <<Диалог'06>>. Бекасово, 31~мая\,--\,4~июня 
2006. С.~138--142.

     \bibitem{20koz} %16
     \Au{Kozerenko~E.\,B.}
     INTERTEXT: A multilingual knowledge base for machine translation~// 
Conference (International) on Machine Learning, Models, Technologies and 
Applications Proceedings. June 25--28, 2007. Las Vegas, USA.~--- Las Vegas: 
CSREA Press, 2007. P.~238--243.

     \bibitem{9koz} %17
     \Au{Жолковский~А.\,К., Мельчук~И.\,А.}
     О семантическом синтезе~// Проблемы кибернетики, 1967. Вып.~19.
     
         
     \bibitem{11koz} %18
     \Au{Jacobs~R.\,A., Rosenbaum P.\,S.}
     English transformational grammar.~--- Blaisdell, 1968.
     

\label{end\stat}
     
          \bibitem{12koz} %19
     \Au{Балли~Ш.}
     Общая лингвистика и вопросы французского языка. 2-е изд.~--- М.: 
УРСС, 2001.

\bibitem{10koz} %20
     \Au{Падучева~Е.\,В.}
     О~семантике синтаксиса: Мат-лы к трансформационной 
грамматике русского языка. 2-е изд.~--- М: КомКнига, 2007.  296~с. 
     
 \end{thebibliography}
}
}


\end{multicols} %15
\def\stat{shihiev}

\def\tit{ИНКАПСУЛЯЦИЯ СЕМАНТИЧЕСКИХ ПРЕДСТАВЛЕНИЙ В~ЭЛЕМЕНТЫ 
ГРАММАТИКИ}

\def\titkol{Инкапсуляция семантических представлений в~элементы 
грамматики}

\def\aut{Ш.\,Б.~Шихиев$^1$, Ф.\,Ш.~Шихиев$^2$}

\def\autkol{Ш.\,Б.~Шихиев, Ф.\,Ш.~Шихиев}

\titel{\tit}{\aut}{\autkol}{\titkol}

\index{Шихиев Ш.\,Б.}
\index{Шихиев Ф.\,Ш.}
\index{Shihiev Sh.\,B.}
\index{Shihiev F.\,Sh.}


%{\renewcommand{\thefootnote}{\fnsymbol{footnote}} \footnotetext[1]
%{Работа выполнена при частичной поддержке РФФИ (проекты 19-07-00352 и~18-29-03100) и~Стипендии Президента Российской Федерации молодым ученым и~аспирантам (СП-538.2018.5). Для ускорения обучения был использован гибридный высокопроизводительный вычислительный комплекс ЦКП <<Информатика>> ФИЦ ИУ РАН: 
%{\sf http://ckp.frccsc.ru}.}}


\renewcommand{\thefootnote}{\arabic{footnote}}
\footnotetext[1]{Дагестанский государственный университет, sh\_sh\_b51@mail.ru}
\footnotetext[2]{Дагестанский государственный университет, fuad@mail.ru}

%\vspace*{-12pt}


  

     \Abst{Предлагается новый математический аппарат представления естественного 
языка (ЕЯ) для компьютерной лингвистики~--- морфология, синтаксис и~семантика описаны 
как предметы дискретной математики, образующие иерархию и~целостную 
информационную систему. 
     Предлагаемая конструктивная теория языка представляет собой новый подход 
     к~изучению языка путем разделения полномочий синтаксиса и~семантики; построения 
автономных моделей синтаксиса и~семантики; формирования языка как отображения 
элементов двух множеств: синтаксиса и~семантики.}
     
      \KW{естественный язык; граф; синтаксис; семантика; лексика; словоформа; 
морфологический признак; лексическая группа; словарь; предложение; алгоритм}

\DOI{10.14357/19922264200116} 
  
%\vspace*{-3pt}


\vskip 10pt plus 9pt minus 6pt

\thispagestyle{headings}

\begin{multicols}{2}

\label{st\stat}
    
\section{Введение}

    \textbf{А.} В~ЕЯ переплетены два автономных 
явления: дискретное (грамматика) и~аналоговое (семантика). Грамматика 
(морфология и~синтаксис) может быть описана на языке математики. 
Морфологические формы слов позволяют их различать и~приписывать им 
и~их сочетаниям различные значения, которые фиксируются в~семантическом 
словаре. В~данной статье показано, как это можно сделать.
    
    \textit{Модель морфологии} заключает в~себе правила построения 
словоформ; словоформа представлена \textit{словом} в~исходной форме и~ее 
\textit{морфологическими параметрами} из десятичных цифр. Словоформы 
с~одинаковыми \textit{морфологическими параметрами} (\textit{формами}) 
образуют \textit{лексическую группу}. 
    
     \textit{Модель синтаксиса} ЕЯ строится исходя из следующих 
предположений: в~ЕЯ имеются такие \textit{элементарные предложения}, что 
из $n$ членов предложения всегда можно составить ($n\hm-1$) 
\textit{словосочетаний} (\textit{синтаксически связанных пар словоформ}), 
таких что члены предложения образуют \textit{связный граф}; следовательно, 
объединение \textit{элементарных предложений} образует 
\textit{синтаксический граф} $\mathrm{Sint}\hm = (X, Y)$ из 
множеств~$X$ (\textit{словоформ}) и~$Y$ (\textit{словосочетаний}). 
Критерий связности двух словоформ определяется через их 
\textit{морфологические формы} (\textit{параметры}). Нетрудно 
проверить~[1], что миллионы \textit{словосочетаний}, имеющих место 
в~грамматике русского языка, распределены по трем сотням 
\textit{синтаксических отношений}~--- прямых произведений лексических 
групп, что упрощает представление \textit{графа} $\mathrm{Sint}$ в~памяти 
компьютера.
     
    \textbf{Б.}\ Предлагаемая сетевая модель грамматики $\mathrm{Sint}$ открывает 
следующие возможности в~изучении ЕЯ и~его реализации на компьютере:
    \begin{enumerate}[1.]
    \item Корневые деревья из модели $\mathrm{Sint}$ порождают 
\textit{элементарные пред\-ло\-же\-ния-де\-ревья}. Обход 
\textit{пред\-ло\-же\-ния-де\-ре\-ва} сопоставляет ему 
\textit{пред\-ло\-же\-ние-по\-сле\-до\-ва\-тель\-ность}; 
корректность обратной задачи, известной как 
\textit{синтаксический анализ}\linebreak
\textit{предложения}, зависит от того, в~какой степени\linebreak 
соблюдены принципы \textit{фрагментарности сегментов} между этими 
предложениями, т.\,е.\ являются ли \textit{сегменты} (вершины ветвей)  
в~\textit{пред\-ло\-же\-нии-де\-ре\-ве фрагментами} в~соответствующем 
\textit{пред\-ло\-же\-нии-по\-сле\-до\-ва\-тель\-ности}~\cite{2-shi}. 
    
    \item Из элементарных предложений строятся более сложные 
предложения по правилам синтаксиса. Из правил построения модели $\mathrm{Sint}$ 
следует возможность построения такой модель грамматики, что любое 
предложение русского языка (например, предложения, встречающиеся 
в~литературе на русском языке) порождается элементами синтаксиса $\mathrm{Sint}$; 
следовательно, имеется формальное определение \textit{синтаксически 
правильно построенного предложения} и~соответствующий алгоритм 
распознавания таких предложений~\cite{2-shi}.
    \item \textit{Модель морфологии} (правила преобразования слов) 
представлена в~\textit{морфологическом словаре}; программа, реализующая 
эти правила, образует \textit{компьютерную модель морфологии}. Приемы 
реализации \textit{модели синтаксиса} (алгоритмов анализа и~синтеза 
предложений) демонстрируются в~\cite{3-shi}. Значения предложений 
в~синтаксисе $\mathrm{Sint}$ представлены в~\textit{семантическом словаре} в~виде 
классов для реализации \textit{семантической модели семантики} 
посредством объектных технологий программирования.
    \end{enumerate}
    
    \textbf{В.} Обращаясь к~истории вопроса, можно на\-пом\-нить следующее. 
В~данной работе реализована идея Ф.~де Сос\-сю\-ра и~Л.~Ельм\-сле\-ва, 
согласно которой строится автономное и~конструктивное\linebreak описание 
синтаксиса, порождающего <<планы выражений>>, далее в~каждый <<план 
выражения>> инкапсулируется <<план содержания>>~--- элемент 
семантики, а~\textit{язык} становится отображением (биекцией) элементов 
двух множеств: синтаксиса и~семантики.
    
    Представляется, что только разделение полномочий синтаксиса 
и~семантики открывает путь к~формализации ЕЯ, иначе придется 
согласиться с~утверждениями типа: <<Никто не может сформулировать все 
правила английской грамматики$\ldots$>>~[4].
    
    В учебниках по <<Общему синтаксису>>~[5] и~в~работах 
И.\,А.~Мельчука~[6] осторожно указывалось на древовидность структуры 
предложения и~графы использовались только для демонстрации сетевой 
структуры предложения. Нужно было решиться и~выделить класс 
\textit{элементарных предложений}, имеющих структуру \textit{корневого 
дерева}, а~далее заметить, что все другие формы предложения 
(с~однородными членами, сложные предложения и~т.\,д.)\ собраны из 
элементарных предложений.
    
    Многие правила построения синтаксически правильных сочетаний 
можно представить в~виде правил кон\-текст\-но-сво\-бод\-ной грамматики 
(например, выражения, образованные из согласованных и~несогласованных 
определений). Этот факт был использован для представления \textit{правил 
синтаксиса} ЕЯ посредством \textit{подстановок} и~\textit{деревьев 
разбора}. Предложение <<Сколько чувствительности контекста требуется, 
чтобы предоставлять разумные структурные описания?>> (How much 
context-sensitivity is required to provide reasonable structural descriptions?) 
из~[7] указывает на то, что автор <<Грамматики сложения деревьев>> (Tree 
adjoining grammars) далек от мысли раздельного исследования синтаксиса 
и~семантики.
    
\section{Морфология}

    \textit{Морфология} есть структура, заданная тройкой ($A, L_0, F_0$), где 
$A$~--- \textit{алфавит}; $L_0$~--- \textit{исходная лексика}, 
представляющая собой конечное множество \textit{исходных слов} над 
алфавитом~$A$; $F_0$~--- конечный набор \textit{исходных 
морфологических признаков}, пред\-став\-ля\-ющих собой двухразрядные 
десятичные числа.
    
    \textit{Исходные признаки} разбиты на непересекающиеся 
подмножества; элементы каждого подмножества образуют линейный массив, 
который называется \textit{категорией} (\textit{признаков}). Категорий 
в~морфологии русского языка меньше десяти, и~они именованы кодами: 
10~(род), 20~(число), 30~(падеж), 40~(степень) и~т.\,д., а~исходные признаки в~категориях кодированы следующим образом: $10\hm = (11, 12, 13)$, $20 
\hm= (21, 22)$, $30 \hm= (31, 32, 33, 34, 35, 36)$, $40 \hm= (41, 42, 43)$ и~т.\,д., 
или в~более привычной записи: $10 \hm= (\mathrm{м.\ род, ср.\ род, ж.\ 
род})$, $20 \hm= (\mathrm{ед.\ число},\linebreak \mathrm{мн.\ число})$, $30\hm = (\mathrm{И., 
Р., Д., В., Т., П.})$, $40 \hm= (\mathrm{полная}\linebreak
\mathrm{форма\ ИП,\ краткая\ форма\ 
ИП, сравнительная\ сте\mbox{-}}\linebreak \mathrm{пень\ ИП})$ и~т.\,д.
    
    \textit{Исходная лексика} также разбита на непересекающиеся 
подмножества~--- \textit{исходные части речи}; к~каждой \textit{исходной 
части речи}~$D_0$ прикреплен свой набор категорий~$\Psi$ и~множество 
\textit{признаков}~$\Omega$; \textit{признак} представляет собой строку 
$f\hm = \mbox{<<}\alpha_1\alpha_2\cdots \alpha_k\mbox{>>}$ из 
\textit{исходных признаков}, принадлежащих различным категориям 
из~$\Psi$. Один из признаков~$f_0$ называется \textit{начальным 
признаком}. Слова из~$D_0$ обладают \textit{начальным признаком}; 
\textit{исходное слово}~${s}_0$ представлено строкой вида 
<<${s}_0:f_0$>>; форма~${s}$ слова~${s}_0$ 
с~признаком~$f$ будет представлена в~виде <<${s}_0:f$>>: 
например, \textbf{дом}\;:\;2133\;=\;\textbf{дому}.
    
    Части речи именованы кодами: 01~--- имя существительное (ИС), 02~--- 
имя прилагательное (ИП), 07~--- глагол и~т.\,д. Категориями для ИС 
являются~20 и~30, а признаками будут $\Omega\hm = \{2131, 2132, 2133, 
2134, 2135, 2136, 2231, 2232,\linebreak
 2233, 2234, 2235, 2236\}$. Для ИП из 
категорий~10, 20, 30 и~40 составлено 29~признаков: $41112131,\linebreak 41112132, 
41112133, 41112134, 41112135, \ldots, 43$. 
    
    Множество ${s}_0:\Omega\hm = \{{s}_0:f \vert  f\hm\in 
\Omega\}$ называется \textit{морфологической группой} 
слова~${s}_0$, а~ее элементы называются формами 
слова~${s}_0$, или просто \textit{словоформами}. Группа 
${s}_0:\Omega$ для ИС~$s_0$ состоит из~12~слов, а~для ИП~--- из 
29~словоформ; нетрудно заметить, что элементы множества 
${s}_0:\Omega$ различны. 
    
    Объединение множеств ${s}_0:\Omega$ по всем~$s_0$ 
из~$D_0$ обозначается через~$D$ и~называется \textit{частью речи}. 
Морфология теперь может быть представлена чет\-вер\-кой ($A, L, \Psi1, 
\Omega1$), где лексика~$L$~--- объединение всех \textit{частей речи}; 
$\Psi1$~--- множество категорий; $\Omega1$~--- множество признаков.
    
    Числовые признаки при исходной форме слова обозначают формы слова, 
на которых будет построен синтаксис (и~предложения) языка. Для 
преобразования словоформ ЕЯ в~слова с~числовыми признаками и~обратно 
нужны соответствующие \textit{морфологические правила}; их, как известно, 
можно найти в~словообразовательных словарях.
    
     Рассмотрим слово $\mathrm{s}_0:\alpha_1\alpha_2\cdots \alpha_k$. По 
определению исходный признак $\alpha_k$ принадлежит некоторой 
категории~$F_k$. Морфология ЕЯ обладает таким свойством, что 
признаками морфологии выступают все строки $\alpha_1\alpha_2\cdots \alpha_k$, 
где~$\alpha_k$ пробегает элементы массива~$F_k$; они образуют множество 
$\alpha_1\alpha_2\cdots \alpha_{k-1}F_k$~--- \textit{парадигму} слова~${s}_0$ по 
категории~$F_k$. Формы слова сгруппированы по \textit{парадигмам}. 
В~электронном словаре вместо элементов \textit{парадигм} будут записаны 
алгоритмы (морфологические правила), порождающие их элементы.
     
     Например, в~строке 
     
     <<2130дом,дома,дому,дом,домом,доме>> 
     
     \noindent
     за 
\textit{парадигмой} 2130~слова \textbf{дом} перечислены его элементы. Есть 
возможность вместо словоформ записать их постфиксы следующим образом:
     \begin{multline}
      \mbox{дом0115:213000,а,у,,ом,е;}\\
      \mbox{223000а,ов,ам,а,ами,ах}.
      \label{e1-shi}
      \end{multline}
      
     В первой части <<дом0115>> статьи~(1) за словом <<дом>> указаны 
его \textit{грамматические атрибуты} (01~--- код ИС, 15~--- мужской род 
и~неодушевленное); во второй части перечислены два 
\textit{морфологических правила} (разделенных точкой с~запятой), по 
которым строятся элементы \textit{парадигмы}~2130 и~2230. К~коду 
\textit{парадигмы} приписано двухразрядное число~--- длина изменяемой 
части словоформ из этой парадигмы. Для экономии памяти имеет смысл 
хранить отдельным списком постфиксы элементов \textit{парадигмы}, 
а~в~статьях словаря указать их порядковые номера. 
     
     Построение словарных статей представляет собой рутинную работу, 
которую также можно запрограммировать.
     
     Пусть $s$~--- обычная форма слова~${s}_0$ с~признаком 
$\alpha_1\alpha_2\cdots \alpha_k$, т.\,е.\ $s \hm= s_0:\alpha_1\alpha_2\cdots\alpha_k$. В~языковом 
явлении морфология решает две задачи: \textit{синтеза}~--- перехода от 
$s_0:\alpha_1\alpha_2\cdots\alpha_k$ к~$s$~--- и~\textit{анализа}~--- перехода 
от~$s$ 
к~$s_0:\alpha_1\alpha_2\cdots\alpha_k$. 
     
     Чтобы осуществить \textit{анализ} словоформы, потребуется 
осуществить синтез всех элементов множества $s_0:\Omega$. Анализ 
словоформы усложняется тем, что по форме слова~$s$ практически 
невозможно точно угадать его исходную форму~$s_0$; а~словарная \mbox{статья} 
начинается с~исходной формы слова~$s$. Анализ словоформы~--- 
трудоемкая процедура, для повышения ее эффективности требуется 
использовать различные приемы поиска из дискретного анализа.

\section{Синтаксис}

    Пусть $D^1, D^2, \ldots, D^q$~--- коды частей речи (как изменяемых, так 
и~наречия, которые образуют одну \textit{лексическую группу}) в~морфологии 
$\mu\hm= (A, L, \Psi1, \Omega1)$, для них определены множества признаков 
$\Omega^1, \Omega^2,\ldots , \Omega^q$ соответственно.
    
    Если $a\in \Omega^i$, то через $a:D^i$, или $D^ia$, обозначается 
множество слов из~$D^i$, обладающих признаком~$a$, и~называется 
\textit{лексической группой} по признаку~$a$. Например, $2135(01) \hm= 
012135\hm = \{\mbox{\textbf{домом}, \textbf{точкой}, 
\textbf{едой},}\ldots\}$.
    
    Через $\Omega^i(D^i)$, или $[D^i]$, обозначается множество, состоящее 
из \textit{лексических групп} $D^ia$ по всем признакам~$a$ из~$\Omega^i$. 
Например, множество~[01] состоит из~12 \textit{лексических групп}: $012131, 
012132, 012133, 012134, \ldots , 012236$.
    
    Через $\Lambda_\mu$ обозначено объединение всех $[D^1], [D^2], \ldots, 
[D^q]$. Построение синтаксиса начинается с~выбора нескольких пар 
\textit{лексических групп} из множества~$\Lambda_\mu$:
    \begin{equation}
    {X}_1\ \mbox{и } {Y}_1\,;\ 
    {X}_2\ \mbox{и } {Y}_2\,;\ 
    {X}_3\ \mbox{и } {Y}_3\,;\ldots ;
    {X}_k\ \mbox{и } {Y}_k\,.
    \label{e2-shi}
    \end{equation} 
     Через $R$ обозначено объединение произведений:
    \begin{equation}
    R={X}_1*{Y}_1\cup
    {X}_2*{Y}_2\cup\cdots
    \cup {X}_k*{Y}_k
    \,.
    \label{e3-shi}
    \end{equation}
   
   Произведение \textit{лексических групп} ${X}_i*{Y}_i$ 
($i\hm = 1, \ldots , k$) называется \textit{синтаксическим отношением} (СО), 
их элементы~--- \textit{синтаксически связанными словоформами} или 
\textit{словосочетаниями}; в~\textit{словосочетании} ($x, y$) слово~$x$~--- 
главный член, $y$~--- зависимый член сочетания. Например, $012131*012132 
\hm= \{\mbox{(\textbf{дом}, \textbf{моды}), (\textbf{запах}, \textbf{дыма}), 
(\textbf{небо}, \textbf{сна}),}\ldots\}$.
   
    Орграф ($L, R$) задает \textit{синтаксис} $\mathrm{Sint}$. Орграф ($L1, R1$), где 
$L1$ состоит из \textit{лексических групп}~(2), а~$R1$~--- из пар 
(${X}_i, {Y}_i$), где $i \hm= 1,\ldots , k$, также задает 
\textit{синтаксис} $\mathrm{Sint}$ в~упакованном виде; 
пусть $\mathrm{Sint}1 \hm= (L1, R1)$. 
Через $\mathrm{Sint}\,(\Lambda_\mu)$ обозначается некоторый синтаксис (грамматика), 
заданный на~$\Lambda_\mu$.
    
    Синтаксические отношения~(\ref{e3-shi}) задаются парами морфологических признаков
    \begin{equation}
    f_1\ \mbox{и } g_1\,;\ 
    f_2\ \mbox{и } g_2\,; \ldots ; 
    f_k\ \mbox{и } g_k\,,
    \label{e4-shi}
    \end{equation}
где признаки $f_i$ и~$g_i$ определяют \textit{лексические группы} 
${X}_i$ и~${Y}_i$ ($i \hm= 1, \ldots , k$). Следовательно, 
необъятного размера синтаксис $\mathrm{Sint}$ задается небольшим набором (около 
двухсот пар) признаков~(\ref{e4-shi}) из~$\Omega1$.
    
    Например, если $D^1$~--- ИС, $D^2$~--- ИП, а~$f_1 \hm= 012131$ и~$g_1 
\hm= 012132$, $f_2 \hm= 012131$ и~$g_2 \hm= 022131$, $f_3 \hm= 012132$ 
и~$g_3 \hm= 022132$, то в~графе ($A, L, \Psi1, \Omega1$) будут связаны дугой 
$(v, w)$ только те вершины~$v$ и~$w$, которые принадлежат СО: 
$012131*012132$, $012131*022131$, $012132*022132$.
    
    В каждом из трех множеств содержатся сотни тысяч элементов. 
Элементами СО $012131*012132$ ($012131*022131$, $012132*022132$) 
являются \textit{несогласованные} (\textit{согласованные}) определения.
    
    Корневое дерево в~графе $\mathrm{Sint}$ называется \textit{вы\-ра\-же\-ни\-ем-де\-ре\-вом}. 
Среди лексических групп имеются\linebreak\vspace*{-12pt}

{ \begin{center}  %fig1
 \vspace*{-1pt}
   
 \mbox{%
 \epsfxsize=79.057mm 
 \epsfbox{shi-1.eps}
 }


\vspace*{6pt}


\noindent
{{\figurename~1}\ \ \small{Корневое дерево в~$\mathrm{Sint1}$}}
\end{center}
}

\vspace*{12pt}

\addtocounter{figure}{1}

\noindent
 две группы: \textit{группа сказуемых} 
$\mathrm{GV}$ и~\textit{группа подлежащих} $\mathrm{GS}$. Если корень \textit{вы\-ра\-же\-ния-де\-ре\-ва} 
принадлежит $\mathrm{GV}$ и~связан дугой с~вершиной из $\mathrm{GS}$, то такое 
\textit{вы\-ра\-же\-ние-де\-ре\-во} называется  
\textit{пред\-ложением-де\-ре\-вом}. Обход  
\textit{вы\-ра\-же\-ния-де\-ре\-ва} называется  
\textit{вы\-ра\-же\-ни\-ем-по\-сле\-до\-ва\-тель\-ностью} (или прос\-то 
выражением), обход \textit{пред\-ло\-же\-ния-де\-ре\-ва} называется\linebreak  
\textit{пред\-ло\-же\-ни\-ем-по\-сле\-до\-ва\-тель\-ностью} (просто 
\textit{предложением}).
    
    Если в~грамматике $\mathrm{Sint}\,(\Lambda_\mu)$:
    \begin{enumerate}[(1)]
\item множество $L$~--- лексика русского языка, (3)~--- функции из 
морфологии русского языка; 
\item элементы ${X}_i*{Y}_i$ ($i \hm= 1,\ldots , k$)~--- 
словосочетания (пары связанных словоформ), допустимые в~синтаксисе 
русского языка; 
\item связанные пары словоформ в~выражениях русского языка образуют 
корневое дерево, 
\end{enumerate}
то $\mathrm{Sint}\,(\Lambda_\mu)$ должен иметь много общего с~синтаксисом русского 
языка; поэтому $\mathrm{Sint}\,(\Lambda_\mu)$ будем называть \textit{моделью} 
грамматики русского языка. Нетрудно показать существование такой 
грамматики $\mathrm{Sint}\,(\Lambda_\mu)$, что предложения, встречающиеся 
в~литературе на русском языке, будут предложениями в~грамматике 
$\mathrm{Sint}\,(\Lambda_\mu)$. (Но в~грамматике $\mathrm{Sint}\,(\Lambda_\mu)$ будут 
предложения <<сомнительного>> значения.)

    Грамматика $\mathrm{Sint}$ будет использована для программного построения 
и~распознавания \textit{выражений} в~синтаксисе $\mathrm{Sint}$. $\mathrm{Sint}$~--- открытая 
система, в~ней могут появляться новые слова и~синтаксически связанные 
пары слов.
    
    В частности, известная задача \textit{синтаксического анализа 
предложения} одинаково формулируется и~решается как в~ЕЯ, так 
и~в~грамматике $\mathrm{Sint}$: \textit{синтаксически правильно построенное} 
в~грамматике $\mathrm{Sint}$ \textit{предложение} будет \textit{синтаксически 
правильно построенным предложением} и~в~грамматике русского языка. 
Последовательность словоформ образует \textit{синтаксически правильно 
построенное} в~грамматике $\mathrm{Sint}$ \textit{предложение}, если в~графе 
$\mathrm{Sint}$ 
найдется \textit{пред\-ло\-же\-ние-де\-ре\-во}, по\-рож\-ден\-ное этим набором 
словоформ. А~алгоритмы поиска корневого дерева, по\-рож\-ден\-но\-го заданным 
множеством вершин, хорошо известны. Строгая формулировка задачи 
\textit{синтаксического анализа предложения} и~наличие алгоритма ее 
решения в~грамматике ЕЯ уже дорогого стоит.
    
     Важным понятием в~грамматике $\mathrm{Sint}$ является СФ~--- 
\textit{синтаксическая форма}. Показанное на рис.~1 дерево имеет 
скобочные формы представления: через вершины~--- $A(B, C(D(H), E))$~--- 
и~дуги~--- $0(1, 2(3(5), 4))$.
                    
 
     
     Вершинами дерева $A(B, C(D(H), E))$ являются \textit{лексические 
группы}, поэтому оно порождает пред\-ло\-же\-ния-де\-ревья $a(b, c(d(h), e))$, 
где строчной буквой обозначена словоформа из группы, обозначенной этой 
же буквой в~верхнем регистре. Предложения из $\mathrm{Sint}$, порожденные 
\textit{корневым деревом} из $\mathrm{Sint1}$, имеют одну и~ту же синтаксическую 
форму, поэтому такие деревья называются СФ. Более того, СФ представляет 
собой правило, порождающее класс предложений определенной формы.
     
     Предположим, что дуга~$i$ ($i \hm= 1, \ldots, 5$) дерева с~рис.~1 (СФ1) 
представлена произведением пары признаков $f_i*g_i$. Синтаксические связи 
между вершинами дерева требуют, чтобы $f_3\hm=f_4\hm =g_2$, $f_5\hm=g_3$; 
любые шесть слов с~указанными признаками могут оказаться вершинами 
предложения, порожденного~СФ1.
     
     Присваивая различным вершинам СФ1 различные словоформы, можно 
строить различные предложения. Обозначив через $g(t)$ словоформу $t:g$ 
(форму~$g$ слова~$t$), можно выписать форму по\-рож\-да\-емых СФ1 
предложений:

\vspace*{3pt} 

\noindent
     \begin{equation}
               f_1(s_1)\left(g_1(t_1), g_2(t_2)\left(g_3(t_3)(g_5(t_5)), 
g_4(t_4)\right)\right).
               \label{e5-shi}
               \end{equation}
Выражение~(\ref{e5-shi}) удобно представить в~виде двух изоморфных 
деревьев:

\vspace*{3pt}

\noindent
\begin{equation}
f_1
\left(g_1, g_2\left(g_3(g_5), g_4\right)\right):
s_1\left(t_1, t_2\left(t_3(t_5), 
t_4\right)\right).
               \label{e6-shi}
               \end{equation}
Нетрудно заметить, что СФ1 представлена выражением $f_1(g_1, g_2(g_3(g_5), 
g_4))$. Примером СФ служит выражение 

\vspace*{3pt}

\noindent
\begin{equation*}
               01112131(02112131, 01112132(02112132))\,,
              %\label{e7-shi}
              \end{equation*}
которое порождает выражения типа <<\textbf{белый дом старого 
охотника}>>.
     
     В форме~(\ref{e6-shi}) будут храниться выражения 
в~\textit{семантическом словаре}. Исследования текстов показывают, что 
число различных СФ, порождающих простые предложения на русском языке, 
не превышает сотни; три десятка СФ позволяют носителю русского языка 
вполне красноречиво выражать свои мысли; а~у~каждого автора текстов 
имеются характерные для него СФ, которыми он пользуется для выражения 
своих мыслей.

     \begin{figure*}[b] %fig2
     \begin{center}
     \begin{tabular}{lr}
  {аист040613/01птица}(      &       (1)\\
  {\textbf{форма}}: (01аист; 02{аистовый}, 02{аистиный}; 07); 
&   (2)\\
  {\textbf{свойство}}: &                        (3)\\
  \hspace*{5mm}({СФ}1: {цвет}({белый}); &          
    (4)\\
  \hspace*{5mm}{СФ}2: {вес}({до}~7~кг); &          
    (5)\\
  \hspace*{5mm}{СФ}4: {местонахождение}({деревня}, 
{поле}); &   (6)\\
  \hspace*{5mm}{СФ}3: 
{местожительство}({гнездо}({СФ}6: {кровля}, {СФ}: 
{дерево})); &(7)\\
  {\textbf{элемент}}: ({СФ}5: {клюв}({СФ}1: {длинный})); &
    (8)\\
  {\textbf{событие}}: ({СФ}6: {сидеть}, {летать}); &  (9)\\
  {\textbf{метод}}: ({СФ}7: {клевать}({трава})) & (10)\\
  )  &          (11)
  \end{tabular}
\end{center}
 %  \vspace{6pt}
\Caption{Статья семантического словаря, описывающая понятие \textit{аист }}
\end{figure*}



%\vspace*{-8pt}

\section{Семантика}

%\vspace*{-2pt}

     При всей своей содержательности формальная грамматика без 
семантики не образует языка. \textit{Семантика} строится на элементах 
синтаксиса; элементы синтаксиса (слова и~их сочетания) должны 
\textit{выражать знания}. Знание, выраженное элементом синтаксиса, 
называется его \textit{значением} или \textit{семантикой}. 
     
     Знание есть специфическая форма \textit{ощущения} сознанием 
активного состояния определенной об\-ласти памяти человека; свидетелем 
существования \textit{знания} является человек, \textit{ощущающий} его; 
\textit{знание} о~слове (как последовательности букв)~--- назовем его 
\textit{именем} слова~--- также хранится в~памяти; к~\textit{знанию} о~слове 
прикреплено иное \textit{знание}, называемое его \textit{значением}; человек 
способен воспроизводить слово; воспринятое человеком \textit{слово} 
активизирует его \textit{значение}; слово и~его значение способны 
активизировать друг друга~--- в~этом сущность ЕЯ. \textit{Значение} 
и~\textit{имя} слова состоят в~таких же отношениях, как информация 
в~ячейке оперативной памяти и~адрес этой ячейки, который хранится 
в~другой ячейке. 
     
     Элементы синтаксиса и~семантики~--- проявления в~физиологии 
человека. Отношения между двумя явлениями: 
\textit{ощущением}-\textit{словом} и~\textit{ощу\-ще\-ни\-ем}-\textit{зна\-ни\-ем}, видимо, имел в~виду 
Ф.~де Соссюр, говоря о~языке как об отображении друг в~друга <<двух 
сущностей>>: \textit{элементов грамматики} (<<план выражения>>) 
и~\textit{элементов семантики} (<<план значения>>).
     
     Чтобы цифровая техника, способная оперировать элементами 
грамматики, стала имитатором языка, требуется в~ней (в~технике) найти 
нечто,\linebreak представляющее собой знание. Например, функции <<плана 
значений>> могли бы сыграть <<нейронные сети>> (электронные схемы), 
если бы ОП состояла из них; но в~современных компьютерах \mbox{организация} 
памяти такова, что слово на экране монитора и~в~памяти компьютера~--- 
сущности одной и~той же природы.
     
     Собеседники, пользуясь только элементами синтаксиса, обмениваются 
знаниями. Такое общение доступно и~двум компьютерам, если они будут 
наделены одной и~той же \textit{моделью мира} (сетью знаний) 
и~идентичными правилами трансформации знаний в~предложения 
и~наоборот.
     
     Знание в~\textit{модели мира} может быть сохранено в~памяти машины 
и~в~форме элементов синтаксиса. Остается придумать технологию хранения, 
подобную той, которая наблюдается в~языковой способности человека, а~не 
в~статьях \textit{толкового словаря}.
     
     Далее излагается один из возможных вариантов построения 
\textit{семантического словаря}, позволяющего хранить большие объемы 
\textit{знаний} в~форме, удобной для программной обработки. 
     
     В статье $W$ словаря хранятся \textit{элементарные знания} о понятии 
$W$ в~виде корневого дерева $T(W)$, в~котором отображены 
\textit{семантические отношения} между~$W$ и~другими понятиями. При 
дереве~$T(W)$ имеется СФ для преобразования $T(W)$ 
в~\textit{элементарное} синтаксически правильное выражение.
     
     Синтаксическую форму, привязанную к~дереву $T(W)$, обозначим через 
     $\mathrm{SF}\,(T(W))$. 
Семантический словарь состоит из пар $\langle T, \mathrm{SF}\,(T)\rangle$, которые 
были описаны в~(\ref{e6-shi}) и~являются предложениями синтаксиса $\mathrm{Sint}$. 
Таким образом, элементы синтаксиса используются для представления 
знания.
     
     На примере \textit{статьи}, посвященной понятию \textit{аист} 
(рис.~2), рассмотрим структуру самой \mbox{статьи} и~оценим, какие возможности 
кроются в~таком словаре для формирования языка.
     

     
     По структуре статья состоит из двух частей: заголовка~(1)  
и~тела~(2)--(11). В~заголовке статьи за именем \textit{понятия} следует его 
\textit{полный код}~--- 040613, в~котором~04 и~0406~--- вложенные друг 
в~друга коды двух предков (\textit{животное} и~\textit{птица}) понятия 
\textit{аист}. За косой чертой указаны имя родительского понятия 
(\textit{птица}) и~код части речи~(01) понятия \textit{аист}.
     
     Тело статьи состоит из пяти разделов: \textit{форма}, \textit{свойство}, 
\textit{элемент}, \textit{событие}, \textit{метод}. В~разделе 
<<\textit{форма}>> перечислены \textit{семантические формы} слова 
\textit{аист}; в~разделе <<\textit{свойство}>>~--- качества данного понятия 
(\textit{цвет}, \textit{вес}, \textit{форма} и~т.\,д.); в~разделе 
<<\textit{элемент}>>~--- составные части, образующие описываемое 
понятие. В~разделе <<событие>> перечислены действия и~состояния, 
которым может подвергаться опи\-сы\-ва\-емое понятие. В~разделе 
<<\textit{метод}>> указаны действия, которые может совершать 
<<\textit{аист}>> над другими понятиями. 
     
     В каждом разделе фиксированы знания, представленные в~виде 
\textit{семантического дерева} в~скобочной форме; корнем для всех деревьев 
служит имя описываемого понятия (в~теле статьи оно не дублируется). 
Деревьям предписаны СФ для преобразования их в~выражения синтаксиса 
$\mathrm{Sint}$.
     
     Например, запись <<{СФ}1: {цвет}({белый})>> может 
обозначать 
\begin{multline*}
\mbox{<<01112131(01112132(02112132)):}\\
\mbox{{аист}({цвета}({белого
}))>>},
\end{multline*}
 т.\,е.\ выражение <<{аист цвета белого}>>.
     
     Далее для построения сложного предложения из \textit{элементарных} 
можно использовать богатый опыт \textit{синтаксической} 
и~\textit{логической семантик}.
    
{\small\frenchspacing
 {%\baselineskip=10.8pt
 \addcontentsline{toc}{section}{References}
 \begin{thebibliography}{9}
\bibitem{1-shi}
Грамматика русского языка~/ Под ред.\ В.\,В.~Виноградова, Е.\,С.~Истриной, 
С.\,Г.~Бархударова.~--- В~2~т.~--- М.: Изд-во Академии наук СССР, 1960. 720~c.
\bibitem{2-shi}
\Au{Шихиев Ф.\,Ш.} Формализация и~сетевая формулировка задачи синтаксического 
анализа: Дис.\ \ldots\ канд. физ.-мат. наук.~--- СПб.: СпбГУ, 2006. 171~с.
\bibitem{3-shi}
\Au{Мирзабеков Я.\,М., Шихиев~Ш.\,Б.} Формальная грамматика русского языка 
в~примерах~// Прикладная дискретная математика, 2018. №\,40. С.~114--126.
\bibitem{4-shi}
\Au{Слобин Д., Грин Дж.} Психолингвистика.~--- М.: Прогресс, 1976. 336~с.
\bibitem{5-shi}
\Au{Тестелец Я.\,Г.} Введение в~общий синтаксис.~--- М.: РГГУ, 2001. 798~с.
\bibitem{6-shi}
\Au{Мельчук И.\,А.} Опыт теории лингвистических моделей Смысл--Текст.~--- М.: 
Языки русской культуры, 1999. 346~с.
\bibitem{7-shi}
Natural language parsing~/ Eds. D.~Dowty, L.~Karttunen, A.~Zwicky.~--- Cambridge: 
Cambridge University Press, 1985. 413~p.
 \end{thebibliography}

 }
 }

\end{multicols}

\vspace*{-12pt}

\hfill{\small\textit{Поступила в~редакцию 25.12.18}}

\vspace*{6pt}

%\pagebreak

%\newpage

%\vspace*{-28pt}

\hrule

\vspace*{2pt}

\hrule

\vspace*{-4pt}

\def\tit{INCAPSULATION OF~SEMANTIC REPRESENTATIONS INTO~ELEMENTS 
OF~A~GRAMMAR}


\def\titkol{Incapsulation of~semantic representations into~elements 
of~a~grammar}

\def\aut{Sh.\,B.~Shihiev and F.\,Sh.~Shihiev}

\def\autkol{Sh.\,B.~Shihiev and F.\,Sh.~Shihiev}

\titel{\tit}{\aut}{\autkol}{\titkol}

\vspace*{-11pt}


\noindent
\noindent
Department of Discrete Mathematics and Computer Science, Dagestan State University,  
43-a~Gadzhiyev Str., Makhachkala 367000, Republic of Dagestan, Russian Federation

\def\leftfootline{\small{\textbf{\thepage}
\hfill INFORMATIKA I EE PRIMENENIYA~--- INFORMATICS AND
APPLICATIONS\ \ \ 2020\ \ \ volume~14\ \ \ issue\ 1}
}%
 \def\rightfootline{\small{INFORMATIKA I EE PRIMENENIYA~---
INFORMATICS AND APPLICATIONS\ \ \ 2020\ \ \ volume~14\ \ \ issue\ 1
\hfill \textbf{\thepage}}}

\vspace*{3pt} 



\Abste{The article proposes a new mathematical apparatus of natural language representation for 
computer linguistics: morphology, syntax, and semantics are described as the objects of discrete 
mathematics forming a~hierarchy and an integral information system. The proposed constructive 
language theory is a new approach to language learning by separating the domains of syntax and 
semantics, constructing the autonomous models of syntax and semantics, language formation as 
the mapping of elements of two sets: syntax and semantics.}

\KWE{natural language; graph; syntax; semantics; lexicon; word form; morphological feature; 
lexical group; dictionary; sentence; algorithm}

\DOI{10.14357/19922264200116} 

%\vspace*{-24pt}

%\Ack
%\noindent



%\vspace*{6pt}

  \begin{multicols}{2}

\renewcommand{\bibname}{\protect\rmfamily References}
%\renewcommand{\bibname}{\large\protect\rm References}

{\small\frenchspacing
 {%\baselineskip=10.8pt
 \addcontentsline{toc}{section}{References}
 \begin{thebibliography}{9}
\bibitem{1-shi-1}
Vinogradov, V.\,V., E.\,S.~Istrina, and S.\,G.~Barkhudarova, eds. 1960. \textit{Grammatika 
russkogo yazyka} [Russian language grammar]. Moscow: USSR Acad. Sci. Publs. 720~p.
\bibitem{2-shi-1}
\Aue{Shihiev, F.\,Sh.} 2006. Formalizatsiya i~setevaya formulirovka zadachi 
sintaksicheskogo analiza [Formalization and network interpretation of 
a~parsing task].  
St.\ Petersburg: St.\ Petersburg State University. PhD Diss. 171~p.
\bibitem{3-shi-1}
\Aue{Mirzabekov, Ya.\,M., and Sh.\,B.~Shihiev.} 2018. Formal'naya grammatika russkogo 
yazyka v~primerakh [Formal grammar of Russian language in examples]. 
\textit{Prikladnaya diskretnaya matematika} [Applied Discrete Mathematics] 40:114--126.
{\looseness=1

}
\bibitem{4-shi-1}
\Aue{Slobin, D., and G.~Green.} 1976. \textit{Psikholingvistika} [Psycholinguistics]. 
Moscow: Progress. 336~p.
\bibitem{5-shi-1}
\Aue{Testelets, Ya.\,G.} 2001. \textit{Vvedenie v~obshchiy sintaksis} [Introduction to 
general syntax]. Moscow: RGGU. 798~p.
\bibitem{6-shi-1}
\Aue{Mel'chuk, I.\,A.} 1999. \textit{Opyt teorii lingvisticheskikh modeley Smysl--Tekst} 
[Experience in the theory of linguistic models Sense--Text]. Moscow: Yazyki russkoy 
kul'tury. 346~p.
\bibitem{7-shi-1}
Dowty, D., L.~Karttunen, and A.~Zwicky, eds. 1985. \textit{Natural language parsing}. 
Cambridge: Cambridge University Press. 413~p.

\end{thebibliography}

 }
 }

\end{multicols}

\vspace*{-6pt}

\hfill{\small\textit{Received December 25, 2018}}

%\pagebreak

%\vspace*{-24pt}

\Contr


\noindent
\textbf{Shihiev Shukur B.} (b.\ 1951)~--- Candidate of Science (PhD) in physics and 
mathematics, associate professor, Department of Discrete Mathematics and Computer Science, 
Dagestan State University, 43-a~Gadzhiyev Str., Makhachkala 367000, Republic of Dagestan, 
Russian Federation; \mbox{sh\_sh\_b51@mail.ru}

\vspace*{6pt}

\noindent
\textbf{Shihiev Fuad B.} (b.\ 1980)~--- Candidate of Science (PhD) in physics and 
mathematics, associate professor, Department of Discrete Mathematics and Computer Science, 
Dagestan State University, 43-a~Gadzhiyev Str., Makhachkala 367000, Republic of Dagestan, 
Russian Federation; \mbox{fuad@mail.ru}

\label{end\stat}

\renewcommand{\bibname}{\protect\rm Литература}    %16
\renewcommand{\figurename}{\protect\bf Figure}
\renewcommand{\tablename}{\protect\bf Table}

\def\stat{dulin}


\def\tit{INFORMATION FUSION OF~DOCUMENTS}

\def\titkol{Information fusion of~documents}

\def\autkol{S.\,K.~Dulin, N.\,G.~Dulina, and~P.\,V.~Ermakov}

\def\aut{ S.\,K.~Dulin$^1$, N.\,G.~Dulina$^2$, and~P.\,V.~Ermakov$^3$}

\titel{\tit}{\aut}{\autkol}{\titkol}



\renewcommand{\thefootnote}{\arabic{footnote}}
\footnotetext[1]{Institute of Informatics Problems, Federal Research Center ``Computer Science and Control'' 
of the Russian Academy of Sciences, 44-2~Vavilov Str., Moscow 119333, Russian Federation, 
skdulin@mail.ru}
\footnotetext[2]{A.\,A.~Dorodnicyn Computing Center, Federal Research Center ``Computer Science and 
Control'' of the Russian Academy of Sciences, 40~Vavilov Str., Moscow 119333, Russian Federation, 
ngdulina@mail.ru}
\footnotetext[3]{ TeleRetail GmbH, 30~\mbox{Markenstra{\!\ptb{\ss}}e}, 
D$\ddot{\mbox{u}}$sseldorf 40227,  Germany; petcazay@gmail.com}


\index{Dulin S.\,K.}
\index{Dulina N.\,G.}
\index{Ermakov P.\,V.}
\index{Дулин С.\,К.}
\index{Дулина Н.\,Г.}
\index{Ермаков П.\,В.}


\def\leftfootline{\small{\textbf{\thepage}
\hfill INFORMATIKA I EE PRIMENENIYA~--- INFORMATICS AND
APPLICATIONS\ \ \ 2020\ \ \ volume~14\ \ \ issue\ 1}
}%
 \def\rightfootline{\small{INFORMATIKA I EE PRIMENENIYA~---
INFORMATICS AND APPLICATIONS\ \ \ 2020\ \ \ volume~14\ \ \ issue\ 1
\hfill \textbf{\thepage}}}

%\vspace*{-2pt}



       \Abste{The paper considers the problems associated with the 
creation of an expert base of documents that require prompt 
processing of incoming information and, as a consequence, 
restructuring of the knowledge base. The authors propose procedures 
that reduce the search of the optimal consistent state of 
interrelated documents. An approach to assessing the relationship of 
text documents and informational messages as poorly structured 
objects was developed. The practical implementation of this approach 
is described.}
      
      \KWE{information fusion; controlled data and knowledge consistency; 
knowledge base restructuring}
      
\DOI{10.14357/19922264200117} 
      
      %\vspace*{8pt}
      
      
      \vskip 12pt plus 9pt minus 6pt
      
       \thispagestyle{myheadings}
      
       \begin{multicols}{2}
      
       \label{st\stat}
     
     \section{Introduction}
      
     \noindent
     Combining information of various origins for integrative analysis and 
processing has been called ``Information Fusion''[1], implying that the synthesized 
data carrying information combine type properties of source data and possess 
more information than merely conjunction of information sources considered 
separately. The main difficulty of the synthesis problem is that information sources 
contain heterogeneous data represented by various formats and structures and 
employed in different types of platforms.
     
     The main factors of data heterogeneity and their sources are: various types 
of data, diversity in data origin, various models of database representation, various 
data presentation formats, differentiating in the organization of data storage 
systems, differences in the degree of reliability and accuracy of data, and
variety of  a~degree and form of data structure.
     
     The process of information fusion is a~multilevel process that includes five 
basic stages~\cite{2-d, 3-d, 4-d}:
     \begin{itemize}
\item zero stage~--- the stage of combining sensor signals, designed to obtain 
data indicating semantically clear and interpretable attributes of objects and 
participating in the applications of the research being performed;
\item the first stage is aimed at processing data of the zero stage in order to 
make a decision on the classes of the objects in question and the states of these 
objects;
\item the second stage of Information Fusion, designed to assess the situation, 
including the zero and the first stages. It is used to assess the situational 
interaction of objects considered as a whole;
\item the third stage~--- the stage of evaluation of the interaction ``Impact 
Assessment,'' designed to perform an antagonistic assessment, based on the 
prediction of the situation;
\item the fourth stage~--- the stage of feedbacks, evaluating the possibility of 
using feedbacks in the system in question; and
\item the fifth stage~--- the final stage, the level of man--machine interaction, 
performing correctional actions of the operator for the sake of the system 
control.
\end{itemize}

     Research in the field of Information Fusion mainly focuses on the synthesis 
of data represented by digital images and arrays of data and  
documents~\cite{1-d, 4-d, 5-d}.
     
     Current trends in the development of corporative informational systems 
show that, along with traditional informational resources, the results of intelligent 
activity of experts and analysts become very important for the successful operation 
of large and middle-sized companies. A~unified informational environment of the 
company incorporates these formalized results in an accumulated form such that 
all executives can jointly use this resource in the context of their assignments. The 
role played by the knowledge accumulated in such a~way in the enterprise-wide 
systems allows us to consider this knowledge as very valuable and a~notably 
important resource for a~company, which, together with the traditional resources, 
such as financial, material, human, etc., characterizes the reliability of the 
company. The totality of this knowledge, presented mainly in text form, is the 
intelligent assets of the company, and the competitiveness of the company and its 
adaptability to changing the business environment depends on how efficiently this 
resource is used.
{\looseness=-1

}
     
     An intelligent asset is a~specific resource that requires specialized 
knowledge management systems. These systems enable the search, accumulation, 
and processing of knowledge by experts in solving various analytical problems. 
This tendency in knowledge engineering appeared relatively recently, but interest 
in the development and usage of such systems is permanently growing. This is 
largely due to the significant results achieved by some companies that have 
successfully implemented knowledge management systems into their 
manufacturing activity.
     
     Complex technological solutions designed to support various stages of 
composition and usage of corporative data and knowledge have been embodied in 
the knowledge management systems. At each of these stages, individual problems 
are solved, with the most important of them being associated with tasks related to 
searching, processing documents, and extracting knowledge from them.
     
     Text processing tasks are solved in practically all fields of human activity, 
and the analysis of the current environment is an integral part of practically 
each 
corporative management system securing a timely and adequate reaction to 
changes in the business environment. Actually, operativeness is the basic 
characteristics of monitoring problems, which distinguishes them from the problems 
related to prediction, planning, etc., because the main goal of the monitoring is the 
timely reaction of corresponding management subsystems of the general 
technological scheme of company functioning to changes of internal or external 
factors.
     
     In the general case, the purpose of text processing tasks is to accumulate 
necessary information from different sources, process it analytically, and, on this 
basis, generate corresponding decisions. The character of text processing tasks is 
permanent in the sense that the environment and the parameters of the company 
operation are subject to permanent changes, which requires regular (or periodic) 
sampling of ever changing information.
     
     Text processing tasks can conventionally be divided into two classes: internal 
monitoring and external monitoring.
     
     Internal monitoring is associated mainly with the monitoring of internal 
operation parameters, e.\,g., regular monitoring of the operation of complex installa-
tions, cargo moving, etc. Possible examples are control systems for energy plants, 
freight management, etc. The typical feature of these problems is a relatively 
constant set of parameters used to estimate the state of the process (production, 
physical parameters of an installation, etc.).
     
     In contrast to the internal monitoring, the external monitoring is mainly 
related to the estimation of the state of the environment and external conditions of 
the company operation. As an example, an analysis of consumer demand carried 
out by a commodity-producing company falls into this category. The typical 
feature of these problems is that, first, the parameters to be estimated are poorly 
formalized and, second, the set of these parameters is variable. The latter factor 
requires the restructuring of the analyst knowledge according to the changed 
conditions. All this makes us consider the ``restructurability'' of the expert 
knowledge base as one of the characteristic features of the problems of external 
monitoring.
     
     In the problems of external monitoring, special requirements must be 
imposed on the sources of information used by experts for the localization of 
required knowledge and data. The development of informational technologies 
during recent years has strongly suggested that the Internet is gradually becoming 
the most important source of information in solving analytical problems in 
practically all areas of human activity. Coming up to printed and electronic mass 
media, Internet is often ranked first in operativeness, which makes the Internet the 
most valuable information source in monitoring problems. It is for this reason that, 
in this work, special attention is paid to the solution of monitoring problems 
associated with search and processing of text information in Internet.
     
     \section{Approach to~Provision of~Knowledge Consistency}
     
     \noindent
     In previous works (see~\cite{4-d, 7-d, 6-d}), the authors put forward a procedure 
providing the consistency of the knowledge base dynamically formed by an 
expert, which is based on the analysis of structural interrelations between separate 
components of the knowledge base with subsequent restructuring of it aimed at 
reducing existing inconsistency. In so doing, the basic criterion of structural 
consistency was a concept of polyconsonance of power~$n$~\cite{2-d}.
     
     Consider a knowledge base formed on the basis of search and analysis of 
Internet information. In solving the monitoring problems associated with the 
formation of such a knowledge base, the application of this procedure faces certain 
difficulties resulting from poor formalization and an obscure or ambiguous 
structure of the data (text or multimedia documents). Besides, for the monitoring 
problems considered here, a large number of informational messages directed to the 
expert for analytical processing and replenishment of the knowledge base are 
characteristic. As a result, the amount of resources (especially, time) required for 
the restructuring of a dynamically changing knowledge base is increased 
significantly, which is, perhaps, the main obstacle to the successful practical 
implementation of any procedure of the above type.
     
     One of the major disadvantages of the algorithm proposed in~\cite{4-d} is 
that it is oriented to problems of the search type; that is why, the authors made 
special efforts to reduce the search and thus increase the algorithm efficiency in its 
practical implementation. The results presented below are aimed at the solution of 
the latter problem.
     
     Consider a set of mutually related objects $O = \{o_i\}$ with a similarity 
function~$f$~\cite{3-d} satisfying the condition
     $$
     0\leq f\left( o_i, o_j\right)\leq 1\,.
     $$
     
     Numbers $\alpha$ and~$\beta$ will denote the lower and upper similarity 
thresholds, respectively, satisfying the condition
     $$
     0\leq \alpha\leq \beta\leq 1\,.
     $$
     
    Now, let us introduce the concepts of a negative, positive, and indifferent link 
between two arbitrary elements~$o_i$ and~$o_j$ of the set~$O$. The link is called 
``negative'' if its value does not exceed the lower similarity threshold: $0\leq 
f(o_i,o_j)\leq \alpha$; it is called ``positive'' if the value of the similarity function is 
not less than the upper similarity threshold: $\beta\leq f(o_i,o_j)\leq 1$; and, if 
$\alpha<f<\beta$, it is called ``indifferent'' (zero).
     
     Consider a partition of the given set into a number of nonempty subsets 
$K_1,\ldots , K_n$.
     
     A link between two arbitrary elements~$o_i$ and~$o_j$ of the entire 
set~$O$ is called ``bad'' if one of the following conditions is satisfied:
     \begin{enumerate}[(1)] 
     \item the elements~$o_i$ and~$o_j$ belong to the same subset~$K_x$, and 
the link between them is negative; or
\item the elements~$o_i$ and~$o_j$ belong to different subsets~$K_1$ 
and~$K_2$, and the link between them is positive.
\end{enumerate}

     Using this definition, let us to each object~$o_k$ from the set 
considered   assign the number~$v_k$ of its bad links for a~given partition into subsets. 
Now, let us construct a~vector~$V$ consisting of these values (this vector has 
a~dimension equal to the number of objects in the set) and call it the nodewise 
difference vector (NDV)~\cite{4-d}. The sum of the elements of this vector is 
denoted by $S_{\mathrm{NDV}}$.
     
     Clearly, different partitions of the original set correspond to different NDVs 
and different values of $S_{\mathrm{NDV}}$. According to the algorithm considered, 
the main problem is to find a partition of the given set~$O$ such that the sum 
$S_{\mathrm{NDV}}$ 
takes its minimal value; i.\,e., the total number of bad links tends to zero.
     
     The algorithm~\cite{4-d} developed by the authors consists in 
successive transformations of the set of informational objects on the basis of the 
condition
     $$
     S_{\mathrm{NDV}} > \fr{n(N-n)}{2}
     $$
     where $S_{\mathrm{NDV}}$ is the sum of nodewise differences for the given 
set of~$n$ elements belonging to a pair of consonant subsets of the total 
cardinality~$N$.  If this condition is fulfilled, then the restructuring of the 
considered set results in a decrease of the total sum~$S_{\mathrm{NDV}}$.
     \smallskip
     
     \noindent
     \textbf{Theorem~1.} \textit{Let~$K_1$ and~$K_2$ be two subsets of 
a~given set of mutually related objects~$O$}:
     \begin{align*}
     K_1 &= \left\{ o_i\right\}\,,\ i=1,\ldots, n_1\,;\\
     K_2&= \left\{ o_j\right\}\,, \ j=1,\ldots , n_2\,.
     \end{align*}
     
     \textit{A set containing~$m$~elements from these two subsets satisfies the 
condition of the algorithm if, and only if, the set consisting of all remaining 
elements of these two subsets satisfies the same condition.}
     
     \smallskip
     
     \noindent
     P\,r\,o\,o\,f\,.\ \  First, let us prove the necessity. Let the set of 
objects~$\{o_k\}$, $k = 1,\ldots , m$, satisfy the condition of the algorithm:
     $$
     \sum v_k> \fr{m(n_1+n_2-m)}{2}
     $$
     where $v_k$ are the NDV values for the element with the number~$k$. 
This formula can be transformed to the form:
     $$
     \sum v_k > \fr{\left(n_1+n_2-m\right)
     \left(\left(n_1+n_2\right)-\left(n_1+n_2-m\right)\right)}{2}
     $$
     which means that the set of $n_1+n_2-m$ vectors not belonging to the 
original set also satisfies the condition of the algorithm.
     
     The sufficiency of the condition is proved similarly. The theorem is proved.
     
     \smallskip
     
     \noindent
     \textbf{Corollary.} In order to find a set of objects from two given subsets 
that satisfies the condition of the algorithm, it is sufficient to check the fulfillment 
of this condition only for the subsets consisting of $(n_1+n_2)/2$ objects. In other 
words, only subsets with cardinalities not exceeding half of the sum of the 
cardinalities of the original subsets~$K_1$ and~$K_2$ should be checked.
     \smallskip
     
     \noindent
     P\,r\,o\,o\,f\,.\ \ Indeed, if some set consisting of more than $(n_1+n_2)/2$ 
elements satisfies the condition, then the complement to it also satisfies this 
condition, with the cardinality of the complement being not greater than 
$(n_1+n_2)/2$.

\begin{figure*}[b] %fig1
\vspace*{1pt}
    \begin{center}  
  \mbox{%
 \epsfxsize=160.967mm 
 \epsfbox{dul-1.eps}
 }
\end{center}
\vspace*{-10pt}
\Caption{Determination of vocabulary groups}
\end{figure*}

     
     \smallskip
     
     \noindent
     \textbf{Theorem~2.}\  \textit{Let~$K_1$ and~$K_2$ be two subsets of 
a~given set of mutually related objects~$O$}:
     \begin{align*}
     K_1 &= \left\{o_i\right\}\,,\ i=1,\ldots , n_1\,;\\
     K_2&= \left\{o_j\right\}\,,\ i=1,\ldots , n_2\,.
     \end{align*}
     \textit{Let a set $\{o_k\}$ of $m < (n_1 + n_2)/2$ elements belonging to 
these two subsets satisfy the condition of the algorithm. If a zero NDV element 
corresponds to some element~$o_x$ from this set, then the set of the vectors 
corresponding $O^*=\{o_1, \ldots, o_{x-1}, o_{x+1}, \ldots, o_m\}$ also satisfies the 
condition of the algorithm.}
     
     \smallskip
     
     \noindent
     P\,r\,o\,o\,f\,.\ \ According to the assumption of the theorem, the sum 
$S^*_{\mathrm{NDV}}$ for the set $O^*=\{o_1, \ldots\linebreak
\ldots, o_{x-1}, o_{x+1}, \ldots, o_m\}$ is 
equal to the sum $S_{\mathrm{NDV}}$ of the original set of the elements from the two 
subsets~$K_1$ and~$K_2$:
     $$
S^*_{\mathrm{NDV}} = S_{\mathrm{NDV}}\,.
$$
     
     Denote by~$N$ the total cardinality of the considered subsets: $N = 
n_1+n_2$. Then,
     $$
     (m-1)(N-(m-1)) = m(N-m)+(2m-N-1)\,.
     $$
     
     According to the assumption of the theorem, $m \leq N/2$; hence, $2m-N-1 
< 0$. To complete the proof, let us write the following inequality:
     \begin{multline*}
     S^*_{\mathrm{NDV}} = S_{\mathrm{NDV}} = \sum v_k >\fr{m(N - 
m)}{2} >{}\\
{}> \fr{(m-1)(N - (m-1))}{ 2}
   \end{multline*}
     which means that the set $\{o_1, \ldots, o_{x-1}, o_{x+1}, \ldots, o_m\}$ satisfies 
the condition of the algorithm.
     
     Obviously enough, it follows from this theorem that, in the practical 
implementation of the proposed algorithm, it is sufficient to search for a set of 
elements for the next iteration among those with nonzero NDV values.
{\looseness=1

}
     
\section{Thematic Role of~Similarity}

     \noindent
     The most significant factor affecting the operation of the algorithm 
considered is the similarity function on the basis of which interrelations between 
different elements of a given set are determined. As far as the support of 
monitoring problems is considered, with the texts (in particular, news) and the 
Internet being the elements and the main information source, respectively, the 
construction of the similarity function becomes a fairly difficult problem. Perhaps, 
one of the solutions to this problem could be the use of various methods of 
linguistic analysis to determine the degree of ``likeness'' of two different 
documents, although these methods are not free from some shortcomings 
associated with the hardship of their implementation, adjustment, etc. To 
determine the similarity function in practical applications, the authors have put 
forward another approach. One of the advantages of this new approach is the 
simplicity of implementation and the ``notional transparency.''
     
     The basis of this approach schematically shown in Fig.~1 is the 
determination of vocabulary groups~\cite{7-d}, which denote the sets of keywords 
defined by the expert. The expert assorts the keywords according to 
some criterion, e.\,g., ``thematic meaning:''
     $$
     G_k= \left\{w_i\right\},\enskip i = 1,\ldots ,n_k.
     $$

\begin{figure*}[b] %fig2
\vspace*{1pt}
    \begin{center}  
  \mbox{%
 \epsfxsize=94.043mm 
 \epsfbox{dul-2.eps}
 }
\end{center}
\vspace*{-10pt}
\Caption{A general scheme of operation of iiProcessor system}
\end{figure*}

     Consider an arbitrary element~$o_j$ from a given set~$O$. This object is a 
text document; so, it can be represented as an aggregate of lexical units, i.\,e., 
words. For~$o_j$, let us define its coefficient of correspondence with the dictionary 
group~$G_i$ as the ratio $S(G_i)_j$ of the number of keywords specified in this 
dictionary group and available in the text of the information object itself, to the 
total number of keywords from all dictionary groups, $S(G)_j$ found in this text. 
Then, one can define the factor of correspondence of the object~$o_j$ to the 
vocabulary group~$G_i$ as
     $$
     L^i_j = \fr{S(G_i)_j}{S(G)_j}.
        $$
     
     On the basis of these coefficients, let us define the degree of thematic coupling 
between two arbitrary informational objects as follows:
     \begin{itemize}
     \item[(A)] $f(o_k, o_l) = 1$ if $ S(G)_k = 0$ and  $S(G)_l = 0$;
     \item[(B)] $f(o_k, o_l) = 0$ if  $S(G)_k\not= S(G)_l$ 
     and $S(G)_k S(G)_l\linebreak = 0$; and
     \item[(C)] $f(o_k, o_l) = \max\left( \min\left(L^i_k, L^i_l\right) \right)$, $i = 1, 
\ldots, n$,  for $S(G)_k  S(G)_l\not= 0$
     where $n$ is the number of the vocabulary groups.
     \end{itemize}

     
     Note that the similarity function defined above takes the values on the 
interval from~0 to~1 but lacks associativity, because $0 \leq f(o_i, o_j) \leq 1$. In 
the works devoted to the theoretical grounds of the considered algorithm of 
structural transformations of a set of objects, the associativity of the similarity 
function has not been used; therefore, the fact that the function introduced above is 
not associative does not require any changes in the proposed algorithm. Moreover, 
the lack of associativity here has an additional meaning, which makes it possible to 
treat the function introduced above as a~\textit{thematic} similarity function.
     
     Indeed, if, in the considered text, there are keywords from different 
vocabulary groups, then all the coefficients~$L^i_j$ for this element will be less 
than one. Hence, the value of the similarity function~$f$ will also be less than one, 
and the more the number of the vocabulary groups, the less this value. In practice, 
this could mean that the considered document is of a review nature and, most 
probably, has no distinct ``thematic meaning.''

\begin{figure*} %fig3
\vspace*{1pt}
    \begin{center}  
  \mbox{%
 \epsfxsize=156.872mm 
 \epsfbox{dul-3.eps}
 }
\end{center}
\vspace*{-10pt}
\Caption{Example of use of vocabulary group technique to establish
links between different documents}
\end{figure*}
     
\section{Consistency Controlling Module iiProcessor}

     \noindent
     The authors' technique for providing structural consistency of the knowledge 
base in solving monitoring problems has been implemented in a specialized system 
called an iiProcessor. This system is designed to compose expert knowledge bases 
for social, political, and international sciences. The knowledge bases are 
constructed from the information supplied by various mass media through their 
Internet servers. The main purpose of the system is to accumulate informational 
messages (news) related to the themes of user's interest from various Internet 
sources, to integrate the information into a unified knowledge base, to create links 
between different elements of the knowledge base, and to make subsequent 
restructuring of the knowledge base on the basis of these links, with the result of 
this restructuring being the representation of the body of the information 
accumulated as a logical system of classes. The latter system can be treated as an 
informational model of the problem examined by the expert (for example, the 
social and political situation in a particular region of the world). A~general scheme 
of operation of the system is shown in Fig.~2.



     As a source of information, this system uses the CNN Internet site ({\sf 
http://cnn.com}). Several times a day, this site publishes information covering many 
aspects of social and political life in many countries. In most cases, the 
informational messages are weakly-structured text documents. In order to establish 
links between different documents, the vocabulary group technique described 
above is used (Fig.~3). If various informational messages contain common 
keywords belonging to different vocabulary groups, this technique estimates the 
``likeness'' of the messages. The similarity function classifies these links as 
positive or negative, which makes it possible to construct a~connectivity matrix on 
the set of the informational messages received by the user (see Fig.~3).
     
    


     The mode of ``Keywords'' allows one to get~10 of the most significant key 
words for a~given document with an indication of their weighting factors (Fig.~4).
     
     
     The mode of interrelations (``Correlations'') will allow to get several 
documents that have the greatest interrelations with selected document. This mode 
works only if the loaded document belongs to the current project of the iiProcessor 
system, in which the relationship was evaluated (Fig.~5).
    
     The choice of the CNN server as a source of information is explained by the 
fact that this server is one of the most informationally abundant servers providing 
real-time information. Of course, the choice of the sources of information is 
strongly determined by the character of the problem considered. In this sense, the 
CNN server is not universal. In view of the above considerations, the Restructor 
system is implemented as a~complex of two program modules. The rsn.exe module 
is the basic one. An auxiliary iip.class module executes a real-time search for new 
information in a specified information source in the Internet. With such an 
architecture, this\linebreak\vspace*{-12pt}

{ \begin{center}  %fig4
 \vspace*{-7pt}
     \mbox{%
 \epsfxsize=79mm 
 \epsfbox{dul-4.eps}
 }


\vspace*{4pt}


\noindent
{{\figurename~4}\ \ \small{``Keywords'' mode}}
\end{center}
}

\vspace*{2pt}


{ \begin{center}  %fig5
 \vspace*{-1pt}
    \mbox{%
 \epsfxsize=79mm 
 \epsfbox{dul-5.eps}
 }


\vspace*{4pt}


\noindent
{{\figurename~5}\ \ \small{``Correlation'' mode}}
\end{center}
}

%\vspace*{3pt}



\noindent
 system can be adopted to operation with any informational 
servers in the Internet (and beyond) by replacing only the auxiliary module, 
without changing its kernel where the major mathematical results of the authors' 
approach are implemented.
     
\section{Concluding Remarks}

\noindent
The implementation of the results of Theorems~1 and~2 in the inference engine 
made it possible to considerably reduce the time expenses of the built-in algorithm 
for restructuring the database. The use of the connectivity matrix as the major 
visualization means for the informational objects improved the clearness of the 
representation of the information model of the problem considered by an expert. 
The system has been tested in analyzing the events related to NASA's 
aerospace research.
     
    % \Ack
    % \noindent
    % This work was supported by the Russian Foundation for Basic Research, 
%project No.\,20-07-00329~А.
     
     \renewcommand{\bibname}{\protect\rmfamily References}
     
     
     \vspace*{-9pt}
     
     {\small\frenchspacing
     {\baselineskip=10.45pt
     \begin{thebibliography}{99}
     
     \bibitem{1-d} %1
\Aue{Dasarathy, B.} 2001. Information fusion~--- what, where, why, when, and how? 
\textit{Inform. Fusion} 2(2):75--76.
     
     \bibitem{4-d} %2
\Aue{Dulin, S.\,K.} 1995. The approach to structural consistency of situations' models in 
an active knowledge base. \textit{Workshop of 10th IEEE Symposium 
(International) on Intelligent Control Proceedings}. Monterey, CA: AdRem, Inc. 
253--258.

\bibitem{3-d} %3
\Aue{Duckham, M., and M.~Worboys.} 2007. Automated geographic information 
fusion and ontology alignment. \textit{Spatial data on the Web}. Eds. A.~Belussi, 
B.~Catania, E.~Clementini, and E.~Ferrari.
Berlin: Springer. Ch.~6:109--132. 

\bibitem{2-d} %4
\Aue{Pravia, M.} 2008. Generation of a~fundamental data set for hard/soft information 
fusion. \textit{11th Conference (International) on Information Fusion Proceedings}. 
Cologne: International Society of Information Fusion. 134--145.





\bibitem{5-d} %5
\Aue{Landauer, T.\,K., K.~Kireyev, and C.~Panaccione.} 2011. Word maturity: A~new 
metric for word knowledge. \textit{Sci. Stud. Read.} 15(1):92--108. 

\bibitem{7-d} %6
\Aue{Dulina, N., and O.~Kozhunova.} 2010. Information monitoring system: 
A~problem 
of linguistic resources consistency and verification. \textit{Problems of Cybernetics and 
Informatics: 3rd Conference (International) Proceedings}. Baku.  
56--58.
\bibitem{6-d} %7
\Aue{Dulin, S.\,K., and  N.\,G.~Dulina.} 2018. Ispol'zovanie disseminatsionnykh 
algoritmov dlya formirovaniya nestrukturirovannoy tekstovoy informatsii v~baze 
geodannykh [Using dissemination algorithms for the formation of unstructured textual 
information in the geodatabase]. \textit{Sistemy i~Sredstva Informatiki~--- Systems and 
Means of Informatics} 28(2):42--59.

\end{thebibliography}}}

\end{multicols}

\vspace*{-6pt}

\hfill{\small\textit{Received February 26, 2019}}

\vspace*{-16pt}

\Contr

%\vspace*{-3pt}

\noindent
\textbf{Dulin Sergey K.} (b.\ 1950)~--- Doctor of Science in technology, 
professor, leading scientist, Institute of Informatics Problems, Federal Research 
Center ``Computer Science and Control'' of the Russian Academy of Sciences,  
44-2~Vavilov Str., Moscow 119333, Russian Federation; principal scientist, 
Research \& Design Institute for Information Technology, Signalling and 
Telecommunications on Railway Transport (JSC NIIAS), 27-1~Nizhegorodskaya 
Str., Moscow 109029, Russian Federation; \mbox{skdulin@mail.ru} 

\vspace*{3pt}

\noindent
\textbf{Dulina Natalia G.} (b.\ 1947)~--- Candidate of Science (PhD) in 
technology, leading programmer, A.\,A.~Dorodnicyn Computing Center, Federal 
Research Center ``Computer Science and Control'' of the Russian Academy of 
Sciences, 40~Vavilov Str., Moscow 119333, Russian Federation; 
\mbox{ngdulina@mail.ru}
\vspace*{3pt}

\noindent
\textbf{Ermakov Petr V.} (b.\ 1985)~--- Senior Software Developer, TeleRetail 
GmbH, 30~\mbox{Markenstra{\!\ptb{\ss}}e}, D$\ddot{\mbox{u}}$sseldorf 
40227,  Germany; \mbox{petcazay@gmail.com}

 

%\newpage

\vspace*{8pt}

\hrule

\vspace*{2pt}

\hrule

%\vspace*{-7pt}

%\newpage

%\vspace*{-28pt}

\def\tit{ИНФОРМАЦИОННЫЙ СИНТЕЗ ДОКУМЕНТОВ}

\def\titkol{Информационный синтез документов}

\def\aut{С.\,К.~Дулин$^1$, Н.\,Г.~Дулина$^2$, П.\,В.~Ермаков$^3$}

\def\autkol{С.\,К.~Дулин, Н.\,Г.~Дулина, П.\,В.~Ермаков}

%{\renewcommand{\thefootnote}{\fnsymbol{footnote}} \footnotetext[1]
%{Работа was supported by the Russian Foundation for Basic Research, project No.\,20-07-00329~А.}}



\titel{\tit}{\aut}{\autkol}{\titkol}

\vspace*{-11pt}

\noindent
$^1$Институт проблем информатики Федерального исследовательского центра <<Информатика 
и~управление>>\linebreak
$\hphantom{^1}$Российской академии наук, \mbox{skdulin@mail.ru}

\noindent
$^2$Вычислительный центр им.\ А.\,А.~Дородницына Федерального исследовательского центра 
<<Информатика\linebreak
$\hphantom{^1}$и~управление>> Российской академии наук, \mbox{ngdulina@mail.ru}

\noindent
$^3$TeleRetail GmbH, D$\ddot{\mbox{u}}$sseldorf, Germany

\vspace*{1pt}

\def\leftfootline{\small{\textbf{\thepage}
\hfill ИНФОРМАТИКА И ЕЁ ПРИМЕНЕНИЯ\ \ \ том\ 14\ \ \ выпуск\ 1\ \ \ 2020}
}%
 \def\rightfootline{\small{ИНФОРМАТИКА И ЕЁ ПРИМЕНЕНИЯ\ \ \ том\ 14\ \ \ 
выпуск\ 1\ \ \ 2020
\hfill \textbf{\thepage}}}

\vspace*{-1pt}



\Abst{Рассматриваются проблемы, связанные с созданием экспертной 
базы документов, требующей оперативной обработки поступающей 
информации и, как следствие, реструктуризации базы знаний. 
Предложены процедуры, уменьшающие время поиска оптимального 
согласованного состояния взаимосвязанных документов. Был 
разработан подход к~оценке взаимосвязи текстовых документов 
и~информационных сообщений как плохо структурированных 
объектов. Описана практическая реализация этого подхода.}

\KW{информационный синтез; контролируемая согласованность 
данных и~знаний; реструктуризация базы знаний}


\DOI{10.14357/19922264200117} 

%\vspace*{-3pt}


 \begin{multicols}{2}

\renewcommand{\bibname}{\protect\rmfamily Литература}
%\renewcommand{\bibname}{\large\protect\rm References}

{\small\frenchspacing
{\baselineskip=10.5pt
\begin{thebibliography}{99}
%\vspace*{-3pt} 

\bibitem{1-d-1} %1
\Au{Dasarathy B.} Information fusion~--- what, where, why, when, and how?~// 
Inform. Fusion, 2001. Vol.~2. Iss.~2. P.~75--76.

\bibitem{4-d-1} %2
\Au{Dulin S.\,K.} The approach to structural consistency of situations' models in an 
active knowledge base~// Workshop of 10th IEEE Symposium 
(International) on Intelligent Control Proceedings.~--- Monterey, CA, USA: AdRem, 
Inc., 1995. P.~253--258.

\bibitem{3-d-1} %3
\Au{Duckham M., Worboys~M.} Automated geographic information fusion and 
ontology alignment~// Spatial data on the Web~/ Eds. A.~Belussi, B.~Catania, 
E.~Clementini, E.~Ferrari.~--- Berlin: Springer, 2007. Ch.~6. P.~109--132. 

\bibitem{2-d-1} %4
\Au{Pravia M.} Generation of a fundamental data set for hard/soft information 
fusion~// 11th Conference (International) on Information Fusion.~--- Cologne: 
International Society of Information Fusion, 2008. P.~134--145.




\bibitem{5-d-1} %5
\Au{Landauer T.\,K., Kireyev~K., Panaccione~C.} Word maturity: A~new metric 
for word knowledge~// Sci. Stud. Read., 2011. Vol.~15. Iss.~1. 
P.~92--108. 

\bibitem{7-d-1} %6
\Au{Dulina N., Kozhunova~O.} Information monitoring system: A~problem of 
linguistic resources consistency and verification~// Problems of Cybernetics and 
Informatics: 3rd Conference (International) Proceedings.~--- Baku, 2010.  
P.~56--58.
\bibitem{6-d-1} %7
\Au{Дулин С.\,К., Дулина~Н.\,Г.} Использование диссеминационных 
алгоритмов для формирования неструктурированной текстовой информации 
в базе геоданных~// Системы и средства информатики, 2018. Т.~28. №\,2. 
С.~42--59. 

\end{thebibliography}
} }

\end{multicols}

 \label{end\stat}

 \vspace*{-9pt}

\hfill{\small\textit{Поступила в~редакцию 26.02.2019}}


%\renewcommand{\bibname}{\protect\rm Литература}
\renewcommand{\figurename}{\protect\bf Рис.}
\renewcommand{\tablename}{\protect\bf Таблица}  %17





%%%%%%%%%%%%%%%%%%%%%%%%%%%%%%%%%%%%%%%%%%%%%%%

%\def\stat{rez}
{%\hrule\par
%\vskip 7pt % 7pt
\raggedleft\Large \bf%\baselineskip=3.2ex
Р\,Е\,Ц\,Е\,Н\,З\,И\,И \vskip 17pt
    \hrule
    \par
\vskip 6pt plus 6pt minus 3pt }

%\thispagestyle{headings} %с верхним колонтитулом
%\thispagestyle{myheadings} %с нижним колонтитулом, но в верхнем РЕЦЕНЗИИ

\def\tit{НОВАЯ КНИГА И.\,Н.~СИНИЦЫНА, А.\,С.~ШАЛАМОВА <<ЛЕКЦИИ ПО ТЕОРИИ 
ИНТЕГРИРОВАННОЙ ЛОГИСТИЧЕСКОЙ ПОДДЕРЖКИ>> (М.: ТОРУС ПРЕСС, 2012. 624~с.)}

%1
\def\aut{Д.ф.-м.н., профессор С.\,Я.~Шоргин}

\def\auf{\ }

\def\leftkol{\ % РЕЦЕНЗИИ
}

\def\rightkol{ \ } 

%\def\leftkol{\ } % ENGLISH ABSTRACTS}

%\def\rightkol{\ } %ENGLISH ABSTRACTS}

%\def\leftkol{РЕЦЕНЗИИ}

%\def\rightkol{РЕЦЕНЗИИ}

\titele{\tit}{\aut}{\auf}{\leftkol}{\rightkol}
\vspace*{-18pt}


     \label{st\stat}

     \begin{multicols}{2}
     {\small
     {\baselineskip=10.1pt
     

      В книге представлено системное изложение теоретических основ одного из новейших 
направлений в \mbox{об\-ласти} экономики послепродажного обслуживания изделий наукоемкой 
продукции (ИНП) длительного пользования~--- интегрированной логистической поддержки
(ИЛП). 
{\looseness=1

}

Приведены также результаты новых работ, выполненных в Институте проблем информатики 
Российской академии наук в рамках научного направления <<Информационные технологии и 
анализ сложных сис\-тем>>.
 {%\looseness=1

}
     
      Излагаемые в книге научные подходы позво\-ляют карди\-наль\-но реформировать 
существующие системы производства и эксплуатации ИНП путем создания и внед\-ре\-ния 
методов рационального и оптимального управ\-ле\-ния процессами расходования 
вре\-мен\-н$\acute{\mbox{ы}}$х, 
мате\-ри\-аль\-ных, трудовых и других ресурсов на всех стадиях жизненного цикла изделий (ЖЦИ) по 
критериям экономической целесообразности и эф\-фек\-тив\-ности.
  {\looseness=1

}
    
      В книге приведен краткий обзор причин возник\-новения и
      развития CALS-методологии как основы 
современных международных стандартов по созданию и функционированию глобальных 
ин\-фор\-ма\-ци\-он\-но-ком\-му\-ни\-ка\-ци\-он\-ных систем, ее ключевых возможностей и эффективности 
результатов ее использования. 
Авторы %\linebreak 
предлагают ряд научных обоснований для разработки 
единой теории проектирования и управления систем ИЛП для полноценного использования 
преимуществ %\linebreak
 суще\-ст\-ву\-ющей методологии, определяют \mbox{общую} структурную схему 
комплексной системы <<ИНП-СППО>> и необходимость разработки для ее описания 
гибридных стохастических моделей.
{%\looseness=1

}

%\columnbreak
      
      Книга состоит из пяти частей, где последовательно излагается материал по каждой из 
следующих тем: <<Интегрированная логистическая поддержка>>, <<Теория гибридных 
стохастических систем и компьютерная поддержка исследований и разработок>>, <<Основы 
математического моделирования, анализа и синтеза систем послепродажного обслуживания>>, 
<<Определение и анализ показателей экспортного потенциала ИНП при проектировании>>, 
<<Задачи управления поддержкой послепродажного обслуживания>>, а также 
<<Моделирование инвестиционных процессов ИЛП в условиях неравновесных финансовых 
рынков>>. 
   
      В конце каждой главы приведены выводы и даны вопросы и задания для 
самоконтроля. В~приложениях содержатся основные определения по программам работ по 
анализу ИЛП, логистическим базам данных и компьютерным решениям, эквивалентной статистической 
линеаризации нелинейных преобразований ИЛП, справочный материал, а также развернутые 
уравнения для вероятностных характеристик.


      \def\leftkol{РЕЦЕНЗИИ}

\def\rightkol{РЕЦЕНЗИИ} 

      
      Книга заинтересует широкий круг специалистов и может быть использована научными 
проектными организациями в сфере промышленного производства ИНП. Большое количество 
иллюстраций, примеров и вопросов, обращенных к читателю, позволяет использовать книгу 
также в качестве учебного пособия для студентов и аспирантов машиностроительных, 
транспортных и~других специальностей, а также для самостоятельного изучения. 
{%\looseness=-1

}

Книга 
представляет несомненный интерес для специалистов и студентов в области прикладной 
математики и информатики.
    

}

}
\end{multicols}

%\newpage

\def\stat{authorsrus}
{%\hrule\par
%\vskip 7pt % 7pt
\raggedleft\Large \bf%\baselineskip=3.2ex
О\,Б\ \ А\,В\,Т\,О\,Р\,А\,Х \vskip 17pt
    \hrule
    \par
\vskip 21pt plus 8pt minus 4pt }


\def\tit{\ }

\def\aut{\ }

\def\auf{\ }

\def\leftkol{\ } % ENGLISH ABSTRACTS}

\def\rightkol{ОБ АВТОРАХ} %ENGLISH ABSTRACTS}

\titele{\tit}{\aut}{\auf}{\leftkol}{\rightkol}
      
            \label{st\stat}



\vspace*{24pt}

\begin{multicols}{2}




\noindent
\textbf{Архипов Олег Петрович} (р.\ 1948)~---
кандидат технических наук, директор Орловского филиала Института проб\-лем информатики
Российской академии наук
%302025, г.Орел, Московское шоссе, д.137

\vspace*{3pt}

\noindent
\textbf{Бирюкова Татьяна Константиновна} (р.\ 1968)~---
кандидат фи\-зи\-ко-ма\-те\-ма\-ти\-че\-ских наук, старший научный сотрудник Института проб\-лем информатики
Российской академии наук

\vspace*{3pt}

\noindent 
\textbf{Бобков  Сергей Геннадьевич} (р.\ 1955)~---
доктор технических наук,  заведующий отделением На\-уч\-но-ис\-сле\-до\-ва\-тель\-ско\-го 
института системных исследований Российской академии наук
%117218, Москва, Нахимовский просп., 36, к.1 

\vspace*{3pt}

\noindent \textbf{Васильев Николай Семенович} (р.\ 1952)~--- доктор 
фи\-зи\-ко-ма\-те\-ма\-ти\-че\-ских наук, профессор, 
МГТУ им.\ Н.\,Э.~Баумана 
%, Москва 105005, 2-я Бауманская ул., д.~5,

\vspace*{3pt}

\noindent
\textbf{Гершкович Максим Михайлович} (р.\ 1968)~---
старший научный сотрудник Института проб\-лем информатики
Российской академии наук

\vspace*{3pt}

\noindent 
\textbf{Дьяченко Юрий Георгиевич} (р.\ 1958)~--- кандидат технических наук, 
старший научный сотрудник Института проб\-лем информатики
Российской академии наук

\vspace*{3pt}

\noindent 
\textbf{Ерошенко Александр Андреевич} (р.\ 1989)~--- аспирант кафедры 
математической статистики факультета вычисли\-тельной математики и кибернетики 
Московского государственного университета им.\ М.\,В.~Ломоносова
%119991, Москва ГСП-1, Ленинские горы, д.\ 1, стр. 52

\vspace*{3pt}
 
\noindent 
\textbf{Захаров Виктор Николаевич} (р.\ 1948)~--- 
доктор технических наук, доцент, ученый секретарь Института проб\-лем информатики
Российской академии наук

\vspace*{3pt}

\noindent
\textbf{Зейфман Александр Израилевич} (р.\ 1954)~---
доктор фи\-зи\-ко-ма\-те\-ма\-ти\-че\-ских наук, профессор, 
заведующий кафедрой Вологодского государственного университета; 
старший научный сотрудник Института проб\-лем информатики
Российской академии наук; главный научный сотрудник ИСЭРТ Российской академии наук

\vspace*{3pt}

\noindent
\textbf{Зыкин Сергей Владимирович} (р.\ 1959)~--- 
доктор технических наук, профессор, заведующий лабораторией Института математики 
им.\ С.\,Л.~Соболева Сибирского отделения Российской академии наук, Новосибирск 
%630090, пр.\ ак.\ Коптюга, 4 

\vspace*{4pt}

\noindent
\textbf{Киреев Владимир Иванович} (р.\ 1938)~---
доктор фи\-зи\-ко-ма\-те\-ма\-ти\-че\-ских наук, профессор Московского 
государственного горного университета
%Адрес: Россия, 119991, г. Москва, Ленинский проспект, д. 6

%\columnbreak

\vspace*{4pt}

\noindent
\textbf{Козеренко Елена Борисовна} (р.\ 1959)~---
кандидат филологических наук, заведующая лабораторией Института проб\-лем информатики
Российской академии наук

\vspace*{4pt}

\noindent
\textbf{Королев Виктор Юрьевич} (р.\ 1954)~--- доктор
фи\-зи\-ко-ма\-те\-ма\-ти\-че\-ских наук, профессор кафедры математической 
статистики факультета вычисли\-тельной математики и кибернетики 
Московского государственного университета; 
ведущий научный сотрудник Института проб\-лем информатики
Российской академии наук

\vspace*{4pt}

\noindent
\textbf{Коротышева Анна Владимировна} (р.\ 1988)~---
старший преподаватель Вологодского государственного университета

\vspace*{4pt}

\noindent 
\textbf{Кун Де Турк} (р.\ 1981)~--- научный сотрудник 
исследовательской группы SMACS факультета телекоммуникаций и обработки информации
Университета Гента, Бельгия
%В-9000 Гент, Бельгия

\vspace*{4pt}

\noindent
\textbf{Лупенцов Олег Сергеевич} (р.\ 1986)~---
аспирант Омского государственного института сервиса
%Омск 644043, ул.\ Певцова 13

\vspace*{4pt}

\noindent
\textbf{Лучко Олег Николаевич} (р.\ 1961)~---
кандидат педагогических наук, профессор, заведующий кафедрой 
Омского государственного института сервиса
%Омск 644043, ул.\ Певцова 13

\vspace*{4pt}

\noindent
\textbf{Малашенко Юрий Евгеньевич} (р.\ 1946)~---
доктор фи\-зи\-ко-ма\-те\-ма\-ти\-че\-ских наук, заведующий сектором 
Вычислительного центра им.\ А.\,А.~Дородницына Российской академии наук
%Адрес: 119333, Москва, ул. Вавилова, 40,

\vspace*{4pt}

\noindent
\textbf{Маньяков Юрий Анатольевич} (р.\ 1984)~---
кандидат технических наук, научный сотрудник Орловского филиала Института проб\-лем информатики
Российской академии наук
%302025, г.Орел, Московское шоссе, д.137

\vspace*{4pt}

\noindent
\textbf{Маренко Валентина Афанасьевна} (р.\ 1951)~---
кандидат технических наук, доцент, старший научный сотрудник 
Института математики им.\ С.\,Л.~Соболева Сибирского отделения Российской академии наук
%Новосибирск 630090, пр. ак. Коптюга, 4 

\vspace*{3pt}

\noindent 
\textbf{Морозов Евсей Викторович} (р.\ 1947)~--- доктор 
фи\-зи\-ко-ма\-те\-ма\-ти\-че\-ских, профессор, ведущий научный сотрудник 
Института прикладных математических исследований Карельского научного центра Российской
академии наук; 
%%185910 Россия, Республика Карелия, г.\ Петрозаводск, ул.\ Пушкинская, 11
профессор Петрозаводского государственного университета, Петрозаводск
%185910 Россия, Республика Карелия, г.\ Петрозаводск, пр.\ Ленина, 33

%\pagebreak

\vspace*{3pt}

\noindent
\textbf{Назарова Ирина Александровна} (р.\ 1966)~---
кандидат фи\-зи\-ко-ма\-те\-ма\-ти\-че\-ских наук, 
научный сотрудник Вычислительного центра им.\ А.\,А.~Дородницына Российской академии наук 
%Адрес: 119333, Москва, ул. Вавилова, 40

\vspace*{3pt}

\noindent
\textbf{Павлов Игорь Валерианович} (р.\ 1945)~--- 
доктор фи\-зи\-ко-ма\-те\-ма\-ти\-че\-ских наук, профессор МГТУ им.\ Н.\,Э.~Баумана 
%Москва 105005, 2-я Бауманская ул., д.~5 

%\pagebreak

\vspace*{3pt}

\noindent 
\textbf{Потахина Любовь Викторовна} (р.\ 1989)~--- аспирантка
Института прикладных математических исследований Карельского научного центра
Российской академии наук; 
%%185910 Россия, Республика Карелия, г.\ Петрозаводск, ул.\ Пушкинская, 11
инженер Петрозаводского государственного университета, Петрозаводск
%185910 Россия, Республика Карелия, г.\ Петрозаводск, пр.\ Ленина, 33

\vspace*{3pt}

\noindent 
\textbf{Рождественский Юрий Владимирович} (р.\ 1952)~--- 
кандидат технических наук, заведующий сектором Института проб\-лем информатики
Российской академии наук

\vspace*{3pt}

\noindent 
\textbf{Синицын Игорь Николаевич} (р.\ 1940)~--- доктор технических наук,
профессор, заслуженный деятель\linebreak\vspace*{-12pt}

\columnbreak

\noindent
 науки РФ, заведующий отделом Института проб\-лем информатики
Российской академии наук

\vspace*{7pt}


\noindent
\textbf{Сиротинин Денис Олегович} (р.\ 1984)~---
кандидат технических наук, научный сотрудник Орловского филиала Института проб\-лем информатики
Российской академии наук
%302025, г.Орел, Московское шоссе, д.137

\vspace*{7pt}

%\columnbreak

\noindent 
\textbf{Соколов  Игорь Анатольевич} (р.\ 1954)~--- академик (действительный член) Российской 
академии наук, доктор технических наук, директор Института проб\-лем информатики
Российской академии наук

\vspace*{7pt}

\noindent
\textbf{Степченков Юрий Афанасьевич} (р.\ 1951)~---
кандидат технических наук, заведующий отделом Института проб\-лем информатики
Российской академии наук

\vspace*{7pt}

\noindent
\textbf{Сурков Алексей Викторович} (р.\ 1978)~--- 
старший научный сотрудник На\-уч\-но-ис\-сле\-до\-ва\-тель\-ско\-го 
института системных исследований Российской академии наук
%117218, Москва, Нахимовский просп., 36, к.1 

\vspace*{7pt}

\noindent 
\textbf{Шестаков Олег Владимирович} (р.\ 1976)~--- доктор 
фи\-зи\-ко-ма\-те\-ма\-ти\-че\-ских, доцент кафедры математической статистики 
факультета вычисли\-тельной математики и кибернетики Московского 
государственного университета им.\ М.\,В.~Ломоносова; 
%119991, Москва ГСП-1, Ленинские горы, д.\ 1, стр. 52
старший научный сотрудник Института проб\-лем информатики
Российской академии наук
%, Москва 119333, ул. Вавилова, д.~44, корп.~2

\vspace*{7pt}

\noindent 
\textbf{Шоргин Сергей Яковлевич} (р.\ 1952.)~--- доктор
фи\-зи\-ко-ма\-те\-ма\-ти\-че\-ских наук, профессор, заместитель директора Института 
проб\-лем информатики Российской академии наук





%%%%%%%%%%%%%%%%%%%%%%%%%%%%%%%%%%%%%%%%%%%%%%%%%%%%%%%%%%%%%%%%%%%%%%%%%%%%%%%




%\def\rightkol{ОБ АВТОРАХ}
%\def\leftkol{ОБ АВТОРАХ}

 \label{end\stat}





%\def\leftfootline{\small{\textbf{\thepage}
%\hfill ИНФОРМАТИКА И ЕЁ ПРИМЕНЕНИЯ\ \ \ том~7\ \ \ выпуск~1\ \ \ 2013}
%}%
% \def\rightfootline{\small{ИНФОРМАТИКА И ЕЁ ПРИМЕНЕНИЯ\ \ \ том~7\ \ \ выпуск~1\ \ \ 2013
%\hfill \textbf{\thepage}}}


%\thispagestyle{myheadings}



\end{multicols}

\newpage  

%\def\stat{cont}
{%\hrule\par
%\vskip 7pt % 7pt
\raggedleft\Large \bf%\baselineskip=3.2ex
А\,В\,Т\,О\,Р\,С\,К\,И\,Й\ \ У\,К\,А\,З\,А\,Т\,Е\,Л\,Ь\ \ З\,А\ \ 2\,0\,0\,7 г. \vskip 17pt
    \hrule
    \par
\vskip 21pt plus 6pt minus 3pt }

\label{st\stat}

\def\tit{\ }

\def\aut{\ }
\def\auf{\ }

\def\leftkol{\ } % ENGLISH ABSTRACTS}

\def\rightkol{\ } %ENGLISH ABSTRACTS}

\titele{\tit}{\aut}{\auf}{\leftkol}{\rightkol}


\contentsline {chapter}{\ }{Выпуск \quad Стр.} 
\contentsline {section}{\textbf{Батракова Д.\,А., Королев В.\,Ю., Шоргин С.\,Я.}\ \ Новый метод вероятностно-ста\-ти\-сти\-че\-ско\-го анализа информационных потоков в\nobreakspace {}телекоммуникационных сетях}{\qquad 1 \qquad 40} 
\contentsline {section}{\textbf{Борисов А.\,В.}\ \ Байесовское оценивание в системах наблюдения с\nobreakspace {}марковскими скачкообразными процессами: игровой подход}{\qquad 2 \qquad 65}
\contentsline {section}{\textbf{Босов А.\,В., Иванов А.\,В.}\ \ Программная инфраструктура информационного Web-пор\-тала}{\qquad 2 \qquad 50}
\contentsline {section}{\textbf{Захаров В.\,Н., Калиниченко Л.\,А., Соколов И.\,А., Ступников С.\,А.}\ \ Конструирование канонических информационных моделей для интегрированных информационных систем}{\qquad 2 \qquad 15}
\contentsline {section}{\textbf{Захаров В.\,Н., Козмидиади В.\,А.}\ \ Средства обеспечения отказоустойчивости при\-ло\-жений}{\qquad 1 \qquad 14} 
\contentsline {section}{\textbf{Иванов А.\,В.}\ \ см. Босов А.\,В.\hfill\hfill\hfill\hfill\hfill\hfill\hfill\hfill\hfill\hfill\hfill\hfill\hfill\hfill\hfill\hfill\hfill\hfill\hfill\hfill\hfill\hfill\hfill\hfill\hfill\hfill\hfill\hfill\hfill\hfill\hfill\hfill\hfill\hfill\hfill}{\ }
\contentsline {section}{\textbf{Ильин В.\,Д., Соколов И.\,А.}\ \ Символьная модель системы знаний информатики в\nobreakspace {}че\-ло\-ве\-ко-автоматной среде}{\qquad 1 \qquad 66} 
\contentsline {section}{\textbf{Калиниченко Л.\,А.}\ \ см. Захаров В.\,Н.\hfill\hfill\hfill\hfill\hfill\hfill\hfill\hfill\hfill\hfill\hfill\hfill\hfill\hfill\hfill\hfill\hfill\hfill\hfill\hfill\hfill\hfill\hfill\hfill\hfill\hfill\hfill\hfill\hfill\hfill\hfill\hfill\hfill\hfill\hfill}{\ }
\contentsline {section}{\textbf{Козеренко Е.\,Б.}\ \ Лингвистическое моделирование для систем машинного перевода и обработки знаний}{\qquad 1 \qquad 54} 
\contentsline {section}{\textbf{Козмидиади В.\,А.}\ \ см. Захаров В.\,Н.\hfill\hfill\hfill\hfill\hfill\hfill\hfill\hfill\hfill\hfill\hfill\hfill\hfill\hfill\hfill\hfill\hfill\hfill\hfill\hfill\hfill\hfill\hfill\hfill\hfill\hfill\hfill\hfill\hfill\hfill\hfill\hfill\hfill\hfill\hfill }{\ } 
\contentsline {section}{\textbf{Королев В.\,Ю.}\ \ см. Батракова Д.\,А.\hfill\hfill\hfill\hfill\hfill\hfill\hfill\hfill\hfill\hfill\hfill\hfill\hfill\hfill\hfill\hfill\hfill\hfill\hfill\hfill\hfill\hfill\hfill\hfill\hfill\hfill\hfill\hfill\hfill\hfill\hfill\hfill\hfill\hfill\hfill}{\ } 
\contentsline {section}{\textbf{Кудрявцев А.\,А., Шоргин С.\,Я.}\ \ Байесовский подход к\nobreakspace {}анализу систем массового обслуживания и\nobreakspace {}показателей надежности}{\qquad 2 \qquad 76}
\contentsline {section}{\textbf{Печинкин А.\,В., Соколов И.\,А., Чаплыгин В.\,В.}\ \ Многолинейная система массового обслуживания с конечным накопителем и ненадежными приборами}{\qquad 1 \qquad 27} 
\contentsline {section}{\textbf{Печинкин А.\,В., Соколов И.\,А., Чаплыгин В.\,В.}\ \ Стационарные характеристики многолинейной\nobreakspace {}системы массового обслуживания с\nobreakspace {}одновременными отказами приборов}{\qquad 2 \qquad 39}
\contentsline {section}{\textbf{Синицын И.\,Н.}\ \ Корреляционные методы построения аналитических информационных моделей флуктуаций полюса Земли по априорным данным}{\qquad 2 \qquad \hphantom{9}2}
\contentsline {section}{\textbf{Синицын И.\,Н.}\ \ Развитие теории фильтров Пугачева для оперативной обработки информации в стохастических системах}{{\qquad 1 \qquad \hphantom{9}3}} 
\contentsline {section}{\textbf{Соколов И.\,А.}\ \ см. Захаров В.\,Н.\hfill\hfill\hfill\hfill\hfill\hfill\hfill\hfill\hfill\hfill\hfill\hfill\hfill\hfill\hfill\hfill\hfill\hfill\hfill\hfill\hfill\hfill\hfill\hfill\hfill\hfill\hfill\hfill\hfill\hfill\hfill\hfill\hfill\hfill\hfill}{\ }
\contentsline {section}{\textbf{Соколов И.\,А.}\ \ см. Ильин В.\,Д.\hfill\hfill\hfill\hfill\hfill\hfill\hfill\hfill\hfill\hfill\hfill\hfill\hfill\hfill\hfill\hfill\hfill\hfill\hfill\hfill\hfill\hfill\hfill\hfill\hfill\hfill\hfill\hfill\hfill\hfill\hfill\hfill\hfill\hfill\hfill}{\ } 
\contentsline {section}{\textbf{Соколов И.\,А.}\ \ см. Печинкин А.\,В.\hfill\hfill\hfill\hfill\hfill\hfill\hfill\hfill\hfill\hfill\hfill\hfill\hfill\hfill\hfill\hfill\hfill\hfill\hfill\hfill\hfill\hfill\hfill\hfill\hfill\hfill\hfill\hfill\hfill\hfill\hfill\hfill\hfill\hfill\hfill}{\ } 
\contentsline {section}{\textbf{Соколов И.\,А.}\ \ см. Печинкин А.\,В.\hfill\hfill\hfill\hfill\hfill\hfill\hfill\hfill\hfill\hfill\hfill\hfill\hfill\hfill\hfill\hfill\hfill\hfill\hfill\hfill\hfill\hfill\hfill\hfill\hfill\hfill\hfill\hfill\hfill\hfill\hfill\hfill\hfill\hfill\hfill}{\ }
\contentsline {section}{\textbf{Ступников С.\,А.}\ \ см. Захаров В.\,Н.\hfill\hfill\hfill\hfill\hfill\hfill\hfill\hfill\hfill\hfill\hfill\hfill\hfill\hfill\hfill\hfill\hfill\hfill\hfill\hfill\hfill\hfill\hfill\hfill\hfill\hfill\hfill\hfill\hfill\hfill\hfill\hfill\hfill\hfill\hfill}{\ }
\contentsline {section}{\textbf{Чаплыгин В.\,В.}\ \ см. Печинкин А.\,В.\hfill\hfill\hfill\hfill\hfill\hfill\hfill\hfill\hfill\hfill\hfill\hfill\hfill\hfill\hfill\hfill\hfill\hfill\hfill\hfill\hfill\hfill\hfill\hfill\hfill\hfill\hfill\hfill\hfill\hfill\hfill\hfill\hfill\hfill\hfill}{\ } 
\contentsline {section}{\textbf{Чаплыгин В.\,В.}\ \ см. Печинкин А.\,В.\hfill\hfill\hfill\hfill\hfill\hfill\hfill\hfill\hfill\hfill\hfill\hfill\hfill\hfill\hfill\hfill\hfill\hfill\hfill\hfill\hfill\hfill\hfill\hfill\hfill\hfill\hfill\hfill\hfill\hfill\hfill\hfill\hfill\hfill\hfill}{\ }
\contentsline {section}{\textbf{Шоргин С.\,Я.}\ \ см. Батракова Д.\,А.\hfill\hfill\hfill\hfill\hfill\hfill\hfill\hfill\hfill\hfill\hfill\hfill\hfill\hfill\hfill\hfill\hfill\hfill\hfill\hfill\hfill\hfill\hfill\hfill\hfill\hfill\hfill\hfill\hfill\hfill\hfill\hfill\hfill\hfill\hfill}{\ } 
\contentsline {section}{\textbf{Шоргин С.\,Я.}\ \ см. Кудрявцев А.\,А.\hfill\hfill\hfill\hfill\hfill\hfill\hfill\hfill\hfill\hfill\hfill\hfill\hfill\hfill\hfill\hfill\hfill\hfill\hfill\hfill\hfill\hfill\hfill\hfill\hfill\hfill\hfill\hfill\hfill\hfill\hfill\hfill\hfill\hfill\hfill}{\ }
%\thispagestyle{myheadings}
\def\leftfootline{\small{\textbf{\thepage}
\hfill ИНФОРМАТИКА И ЕЁ ПРИМЕНЕНИЯ\ \ \ том~1\ \ \ выпуск~2\ \ \ 2007}
}%
 \def\rightfootline{\small{ИНФОРМАТИКА И ЕЁ ПРИМЕНЕНИЯ\ \ \ том~1\ \ \ выпуск~2\ \ \ 2007
 \hfill \textbf{\thepage}}}
 \label{end\stat} 
                     
%\def\stat{cont-e}
{%\hrule\par
%\vskip 7pt % 7pt
\raggedleft\Large \bf%\baselineskip=3.2ex
2\,0\,0\,7\ \ A\,U\,T\,H\,O\,R\ \ I\,N\,D\,E\,X \vskip 17pt
    \hrule
    \par
\vskip 21pt plus 6pt minus 3pt }

\label{st\stat}

\def\tit{\ }

\def\aut{\ }
\def\auf{\ }

\def\leftkol{\ } % ENGLISH ABSTRACTS}

\def\rightkol{\ } %ENGLISH ABSTRACTS}

\titele{\tit}{\aut}{\auf}{\leftkol}{\rightkol}


\contentsline {chapter}{\ }{Issue \quad Page} 
\contentsline {subsection}{\textbf{Batrakova D.\,A., Korolev V.\,Yu., Shorgin S.\,Ya.}\ \ A New Method for the Probabilistic and Statistical Analysis of Information Flows in Telecommunication Networks}{\qquad 1 \qquad 40} 
\contentsline {subsection}{\textbf{Borisov A.\,V.}\ \ Bayesian Estimation in\nobreakspace {}Observation Systems with\nobreakspace {}Markov Jump Processes: Game-Theoretic Approach}{\qquad 2 \qquad 65} 
\contentsline {subsection}{\textbf{Bosov A.\,V., Ivanov A.\,V.}\ \ Linguistic Simulation for Machine Translation and Knowledge Management Systems}{\qquad 2 \qquad 50} 
\contentsline {subsection}{\textbf{Chaplygin V.\,V.} see Pechinkin A.\,V.\hfill\hfill\hfill\hfill\hfill\hfill\hfill\hfill\hfill\hfill\hfill\hfill\hfill\hfill\hfill\hfill\hfill\hfill\hfill\hfill\hfill\hfill\hfill\hfill\hfill\hfill\hfill\hfill\hfill\hfill\hfill\hfill\hfill\hfill\hfill}{\ }
\contentsline {subsection}{\textbf{Chaplygin V.\,V.} see Pechinkin A.\,V.\hfill\hfill\hfill\hfill\hfill\hfill\hfill\hfill\hfill\hfill\hfill\hfill\hfill\hfill\hfill\hfill\hfill\hfill\hfill\hfill\hfill\hfill\hfill\hfill\hfill\hfill\hfill\hfill\hfill\hfill\hfill\hfill\hfill\hfill\hfill}{\ }
\contentsline {subsection}{\textbf{Ilyin V.\,D., Sokolov I.\,A.}\ \ The Symbol Model of Informatics Knowledge System in Human-Automaton Environment}{\qquad 1 \qquad 66} 
\contentsline {subsection}{\textbf{Ivanov A.\,V.} see Bosov A.\,V.\hfill\hfill\hfill\hfill\hfill\hfill\hfill\hfill\hfill\hfill\hfill\hfill\hfill\hfill\hfill\hfill\hfill\hfill\hfill\hfill\hfill\hfill\hfill\hfill\hfill\hfill\hfill\hfill\hfill\hfill\hfill\hfill\hfill\hfill\hfill}{\ }
\contentsline {subsection}{\textbf{Kalinichenko L.\,A.} see Zakharov V.\,N.\hfill\hfill\hfill\hfill\hfill\hfill\hfill\hfill\hfill\hfill\hfill\hfill\hfill\hfill\hfill\hfill\hfill\hfill\hfill\hfill\hfill\hfill\hfill\hfill\hfill\hfill\hfill\hfill\hfill\hfill\hfill\hfill\hfill\hfill\hfill}{\ }
\contentsline {subsection}{\textbf{Korolev V.\,Yu.} see Batrakova D.\,A.\hfill\hfill\hfill\hfill\hfill\hfill\hfill\hfill\hfill\hfill\hfill\hfill\hfill\hfill\hfill\hfill\hfill\hfill\hfill\hfill\hfill\hfill\hfill\hfill\hfill\hfill\hfill\hfill\hfill\hfill\hfill\hfill\hfill\hfill\hfill}{\ }
\contentsline {subsection}{\textbf{Kozerenko E.\,B.}\ \ Linguistic Simulation for Machine Translation and Knowledge Management Systems}{\qquad 1 \qquad 54} 
\contentsline {subsection}{\textbf{Kozmidiady V.\,A.} see Zakharov V.\,N.\hfill\hfill\hfill\hfill\hfill\hfill\hfill\hfill\hfill\hfill\hfill\hfill\hfill\hfill\hfill\hfill\hfill\hfill\hfill\hfill\hfill\hfill\hfill\hfill\hfill\hfill\hfill\hfill\hfill\hfill\hfill\hfill\hfill\hfill\hfill}{\ }
\contentsline {subsection}{\textbf{Kudryavtsev A.\,A., Shorgin S.\,Ya.}\ \ Bayesian Approach to Queueing Systems and Reliability Characteristics}{\qquad 2 \qquad 76} 
\contentsline {subsection}{\textbf{Pechinkin A.\,V., Sokolov I.\,A., Chaplygin V.\,V.}\ \ Multichannel Queuing System with Finite Buffer and Unreliable Servers}{\qquad 1 \qquad 27} 
\contentsline {subsection}{\textbf{Pechinkin A.\,V., Sokolov I.\,A., Chaplygin V.\,V.}\ \ Stationary Characteristics of a Multichannel Queueing System with\nobreakspace {}Simultaneous Refusals of Servers}{\qquad 2 \qquad 39} 
\contentsline {subsection}{\textbf{Shorgin S.\,Ya.} see Batrakova D.\,A.\hfill\hfill\hfill\hfill\hfill\hfill\hfill\hfill\hfill\hfill\hfill\hfill\hfill\hfill\hfill\hfill\hfill\hfill\hfill\hfill\hfill\hfill\hfill\hfill\hfill\hfill\hfill\hfill\hfill\hfill\hfill\hfill\hfill\hfill\hfill}{\ }
\contentsline {subsection}{\textbf{Shorgin S.\,Ya.} see Kudryavtsev A.\,A.\hfill\hfill\hfill\hfill\hfill\hfill\hfill\hfill\hfill\hfill\hfill\hfill\hfill\hfill\hfill\hfill\hfill\hfill\hfill\hfill\hfill\hfill\hfill\hfill\hfill\hfill\hfill\hfill\hfill\hfill\hfill\hfill\hfill\hfill\hfill}{\ }
\contentsline {subsection}{\textbf{Sinitsyn I.\,N.}\ \ Correlational Methods for Analytical Informational Models of the Earth Pole Fluctuations Design Based on a priori Data}{\qquad 2 \qquad \hphantom{9}2}
\contentsline {subsection}{\textbf{Sinitsyn I.\,N.}\ \ Development of Pugachev Filtering for Stochastic Systems}{\qquad 1 \qquad \hphantom{9}3}
\contentsline {subsection}{\textbf{Sokolov I.\,A.} see Ilyin V.\,D.\hfill\hfill\hfill\hfill\hfill\hfill\hfill\hfill\hfill\hfill\hfill\hfill\hfill\hfill\hfill\hfill\hfill\hfill\hfill\hfill\hfill\hfill\hfill\hfill\hfill\hfill\hfill\hfill\hfill\hfill\hfill\hfill\hfill\hfill\hfill}{\ }
\contentsline {subsection}{\textbf{Sokolov I.\,A.} see Pechinkin A.\,V.\hfill\hfill\hfill\hfill\hfill\hfill\hfill\hfill\hfill\hfill\hfill\hfill\hfill\hfill\hfill\hfill\hfill\hfill\hfill\hfill\hfill\hfill\hfill\hfill\hfill\hfill\hfill\hfill\hfill\hfill\hfill\hfill\hfill\hfill\hfill}{\ }
\contentsline {subsection}{\textbf{Sokolov I.\,A.} see Pechinkin A.\,V.\hfill\hfill\hfill\hfill\hfill\hfill\hfill\hfill\hfill\hfill\hfill\hfill\hfill\hfill\hfill\hfill\hfill\hfill\hfill\hfill\hfill\hfill\hfill\hfill\hfill\hfill\hfill\hfill\hfill\hfill\hfill\hfill\hfill\hfill\hfill}{\ }
\contentsline {subsection}{\textbf{Sokolov I.\,A.} see Zakharov V.\,N.\hfill\hfill\hfill\hfill\hfill\hfill\hfill\hfill\hfill\hfill\hfill\hfill\hfill\hfill\hfill\hfill\hfill\hfill\hfill\hfill\hfill\hfill\hfill\hfill\hfill\hfill\hfill\hfill\hfill\hfill\hfill\hfill\hfill\hfill\hfill}{\ }
\contentsline {subsection}{\textbf{Stupnikov S.\,A.} see Zakharov V.\,N.\hfill\hfill\hfill\hfill\hfill\hfill\hfill\hfill\hfill\hfill\hfill\hfill\hfill\hfill\hfill\hfill\hfill\hfill\hfill\hfill\hfill\hfill\hfill\hfill\hfill\hfill\hfill\hfill\hfill\hfill\hfill\hfill\hfill\hfill\hfill}{\ }
\contentsline {subsection}{\textbf{Zakharov V.\,N., Kalinichenko L.\,A., Sokolov I.\,A., Stupnikov S.\,A.}\ \ Development of Canonical Information Models for Integrated Information Systems}{\qquad 2 \qquad 15} 
\contentsline {subsection}{\textbf{Zakharov V.\,N., Kozmidiady V.\,A.}\ \ Means Providing Applications Fault Tolerance}{\qquad 1 \qquad 14} 
\def\leftfootline{\small{\textbf{\thepage}
\hfill ИНФОРМАТИКА И ЕЁ ПРИМЕНЕНИЯ\ \ \ том~1\ \ \ выпуск~2\ \ \ 2007}
}%
 \def\rightfootline{\small{ИНФОРМАТИКА И ЕЁ ПРИМЕНЕНИЯ\ \ \ том~1\ \ \ выпуск~2\ \ \ 2007
 \hfill \textbf{\thepage}}}
 \label{end\stat} 


%\end{document}

%
\def\stat{rekl}
%\label{preobr}

%\def\tit{АКАДЕМИК ПУГАЧЁВ  ВЛАДИМИР СЕМЁНОВИЧ\\
%25.03.1911--25.03.1998}


%   \vspace*{-48pt}
%   \begin{center}\LARGE
%Академик Пугачёв  Владимир Семёнович\\ (25.03.1911--25.03.1998)
%   \end{center}

   %\vspace*{2.5mm}

   \begin{center}

{\prgsh\LARGE
ЮБИЛЕИ}

\end{center}
%\hrule

\vspace*{6pt}


   \vspace*{8mm}

   \thispagestyle{empty}


%\def\stat{emel}


\section*{К 70-летию заместителя директора ИПИ РАН,\\ члена редколлегии журнала
<<Информатика и её применения>>\\ доктора технических наук В.\,И.~Будзко}

\vspace*{18pt}




          \begin{multicols}{2}

%            \label{st\stat}

\begin{center}
\vspace*{1pt}
\mbox{%
\epsfxsize=78mm
\epsfbox{bud-1.eps}
}
\end{center}

\vspace*{12pt}

      14 августа 2014~г.\ исполнилось 70~лет за\-мес\-ти\-те\-лю директора ИПИ РАН по
научной работе доктору технических наук Владимиру Игоревичу Будзко.

      Владимир Игоревич Будзко родился в г.~Москве. Высшее образование получил на факультете
элект\-рон\-но-вы\-чис\-ли\-тель\-ных устройств в Московском
ин\-же\-нер\-но-фи\-зи\-че\-ском институте
(МИФИ), который он окончил в 1968~г., после чего был на\-прав\-лен для прохождения
службы в одну из войс\-ко\-вых частей, где прошел путь от инженера до первого заместителя
командира войсковой части.

      С приходом В.\,И.~Будзко в ИПИ РАН (2001~г.)\ в институте
сформировалось новое научное на\-прав\-ле\-ние теоретических исследований~--- <<Постро\-ение
ин\-фор\-ма\-ци\-он\-но-те\-ле\-ком\-му\-ни\-ка\-ци\-он\-ных\linebreak сис\-тем
высокой до\-ступ\-ности>>. В~рамках этого
направления выполнен широкий круг фундаментальных исследований по поиску подходов и
определению принципов построения средств обеспечения доступности, конфиденциальности
и целостности современных крупномасштабных
ин\-фор\-ма\-ци\-он\-но-те\-ле\-ком\-му\-ни\-ка\-ци\-он\-ных
сис\-тем (ИТС). Разработаны основные сис\-тем\-но-тех\-ни\-че\-ские принципы и базовые
архитектурные решения построения перспективных для условий России ИТС с
централизованной обработкой и хранением информации, сочетающих в себе свойства
высокой доступности, отказо- и катастрофоустойчивости, информационной защищенности.
Определены принципы, методы и математические основы рационального построения и
оптимизации средств восстановления функционирования центров обработки данных (ЦОД)
после возникновения отказов и катастроф, передачи и хранения данных, обеспечения
информационной безопасности при достижении минимальной совокупной стоимости
владения такими системами. Результаты нашли практическое воплощение при реализации
проектов в интересах ряда отечественных государственных и негосударственных
организаций, таких как Банк России (БР), Внешторгбанк, ОАО <<ГМК <<Норильский Никель>>,
<<Газпром>>, Минэкономразвития России, Правительство Москвы, а также ряд силовых
ведомств.

      Под руководством В.\,И.~Будзко начиная с 2001~г.\ выполнен комплекс
      на\-уч\-но-ис\-сле\-до\-ва\-тель\-ских и
      опыт\-но-кон\-ст\-рук\-тор\-ских работ (свыше 100~проектов),
направленных на развитие электронной информационной технологии БР.
Разработаны концепции развития ИТС БР сначала до 2008~г., а затем до 2013~г., которые
были приняты в качестве основы проведения технической политики. За реализацию проекта
<<Катастрофоустойчивая тер\-ри\-то\-ри\-аль\-но-рас\-пре\-де\-лен\-ная
      ин\-фор\-ма\-ци\-он\-но-те\-ле\-ком\-му\-ни\-ка\-ци\-он\-ная сис\-те\-ма централизованной
обработки банковской информации>> В.\,И.~Будзко удостоен Премии Правительства РФ в
области науки и техники за 2010~г.

      В.\,И.~Будзко возглавлял и возглавляет работы по ряду других прикладных проектов,
связанных с созданием, совершенствованием и развитием крупномасштабных ИТС.

      В.\,И.~Будзко~--- генерал-майор, доктор технических наук, член-кор\-рес\-пон\-дент
Академии криптографии РФ, известный ученый в области информатики и применения
информационных технологий при построении территориально распределенных ИТС
различного назначения. Является автором свыше 250~научных работ, опубликованных в
на\-уч\-но-тех\-ни\-че\-ских и специальных изданиях.

    \thispagestyle{empty}

      В.\,И.~Будзко уделяет большое внимание подготовке научных кадров. Под его
руководством защищено 6~диссертаций на соискание ученой степени кандидата
технических наук. Свыше 30~лет он читает лекции в ИКСИ Академии ФСБ, профессор
кафедры НИЯУ МИФИ. Является членом двух диссертационных советов, главным
редактором журнала <<Системы высокой доступности>> и членом редколлегии журнала
<<Информатика и её применения>>.

      \bigskip

      Редакционный совет и Редакционная коллегия журнала <<Информатика и её
применения>> сердечно поздравляют Владимира Игоревича Будзко с 70-ле\-ти\-ем и желают
крепкого здоровья и новых научных достижений.

\end{multicols}

%%Информатика и её применения
%Том 13 Выпуск 1-4 Год 2019

\def\stat{cont}
{%\hrule\par
%\vskip 7pt % 7pt
\raggedleft\Large \bf%\baselineskip=3.2ex
А\,В\,Т\,О\,Р\,С\,К\,И\,Й\ \ У\,К\,А\,З\,А\,Т\,Е\,Л\,Ь\ \ З\,А\ \ 2\,0\,1\,9 г. \vskip 17pt
 \hrule
 \par
\vskip 21pt plus 6pt minus 3pt }

\label{st\stat}

\def\tit{\ }

\def\aut{\ }
\def\auf{\ }

\def\leftkol{\ } % ENGLISH ABSTRACTS}

\def\rightkol{\ } %АВТОРСКИЙ УКАЗАТЕЛЬ ЗА 2019 г.} %ENGLISH ABSTRACTS}

\titele{\tit}{\aut}{\auf}{\leftkol}{\rightkol}
\addcontentsline{toc}{subsection}{\textrm\textbf Авторский указатель за 2019 г.}

%\vspace*{-12pt}

\noindent
{\tabcolsep=3pt
\begin{tabular}{p{397pt}cc}
&\textbf{Вып.} & \textbf{Стр.}\\[6pt]
\Avtors{Абгарян~К.\,К., Осипова~В.\,А.} Применение методов поддержки принятия решений для\linebreak
\\[-12pt]
\hspace*{23pt}многокритериальной задачи отбора многомасштабных композиций&2&47--53\\
\Avtors{Агаларов~Я.\,М., Коновалов~М.\,Г.} Доказательство унимодальности целевой функции\linebreak
\\[-12pt]
\hspace*{23pt}в~задаче порогового управления нагрузкой на~сервер&2&2--6\\
\Avtors{Агаларов~Я.\,М., Ушаков~В.\,Г.} Об унимодальности функции дохода системы массового\linebreak
\\[-12pt]
\hspace*{23pt}обслуживания типа $G|M|s$ с~управляемой очередью&1&55--61\\
\Avtors{Агасандян~Г.\,А.} Вычисление показателей оптимальных по CC-VaR портфелей на~рынках\linebreak
\\[-12pt]
\hspace*{23pt}опционов&3&72--81\\
\Avtors{Агасандян~Г.\,А.} Теоретические основы оптимизации по континуальному критерию VaR на совокупности рынков&4&36--41\\
\Avtors{Анашин~В.\,С.} О теоретико-автоматных моделях блокчейн-среды&2&29--36\\
\Avtors{Аникеев~Д.\,А., Пенкин~Г.\,О., Стрижов~В.\,В.} Классификация физической активности\linebreak
\\[-12pt]
\hspace*{23pt}человека с~помощью локальных аппроксимирующих моделей&1&40--48\\
\Avtors{Арутюнов~Е.\,Н., Кудрявцев~А.\,А., Титова~А.\,И.} Байесовские модели баланса факторов, \linebreak
\\[-12pt]
\hspace*{23pt}имеющих априорные распределения Вейбулла и~Накагами&2&71--75\\
\Avtors{Бахтеев~О.\,Ю.} см.\ Грабовой~А.\,В.&&\\
\Avtors{Бондаренко~Н.\,Н.} см.\ Журавлев~Ю.\,И.&&\\
\Avtors{Борисов~А.\,В.} Численные схемы фильтрации марковских скачкообразных процессов по\linebreak
\\[-12pt]
\hspace*{23pt}дискретизованным наблюдениям~I: характеристики точности&4&68--75\\
\Avtors{Босов~А.\,В., Миллер~Г.\,Б.} О развитии концепции условно-минимаксной нелинейной\linebreak
\\[-12pt]
\hspace*{23pt}фильтрации: модифицированный фильтр и~его анализ&2&\hphantom{1}7--15\\
\Avtors{Босов~А.\,В., Мхитарян~Г.\,А., Наумов~А.\,В., Сапунова~А.\,П.} Использование модели гамма-распределения в~задаче формирования ограниченного по времени теста в~системе\linebreak
\\[-12pt]
\hspace*{23pt}дистанционного обучения&4&11--17\\
\Avtors{Босов~А.\,В., Стефанович~А.\,И.} Управление выходом стохастической дифференциальной системы по~квадратичному критерию. II.~Численное решение уравнений динами-\linebreak
\\[-12pt]
\hspace*{23pt}ческого программирования&1&\hphantom{1}9--15\\
\Avtors{Босов~А.\,В., Стефанович~А.\,И.} Управление выходом стохастической дифференциальной системы по квадратичному критерию. III.~Анализ свойств оптимального управ-\linebreak
\\[-12pt]
\hspace*{23pt}ления&3&41--49\\
\Avtors{Бурлуцкий~В.\,В., Якимчук~А.\,В., Мельников~А.\,В., Царегородцев~А.\,Л., Волошин~С.\,В.} Разработка метода формирования признакового пространства и~модели для оценки и~прогнозирования антропогенного влияния на окружающую среду (на примере\linebreak
\\[-12pt]
\hspace*{23pt}лесного фонда нефтедобывающего региона)&3&131--136\\
\Avtors{Вахтанов~Н.\,А.} см.\ Шнурков~П.\,В.&&\\
\Avtors{Вахтанов~Н.\,А.} см.\ Шнурков~П.\,В.&&\\
\Avtors{Виноградов~А.\,П.} см.\ Журавлев~Ю.\,И.&&\\
\Avtors{Волошин~С.\,В.} см.\ Бурлуцкий~В.\,В.&&\\
\Avtors{Вышинский~Л.\,Л., Курьянский~М.\,К., Флеров~Ю.\,А.} Цифровая модель весового паспорта\linebreak
\\[-12pt]
\hspace*{23pt}летательного аппарата&4&\hphantom{1}3--10\\
\Avtors{Гайдамака~А.\,А., Чухно~Н.\,В., Чухно~О.\,В., Самуйлов~К.\,Е., Шоргин~С.\,Я.} Формализация метода ранжирования альтернатив для процесса группового принятия решений при\linebreak
\\[-12pt]
\hspace*{23pt}анализе социальных сетей&3&63--71\\
\Avtors{Гайдамака~Ю.\,В.} см.\ Горбунова~А.\,В.&&\\
\end{tabular}
}

\pagebreak

\def\leftkol{АВТОРСКИЙ УКАЗАТЕЛЬ ЗА 2019 г.} % ENGLISH ABSTRACTS}

\def\rightkol{АВТОРСКИЙ УКАЗАТЕЛЬ ЗА 2019 г.} %ENGLISH ABSTRACTS}

%\thispagestyle{myheadings}
\def\leftfootline{\small{\textbf{\thepage}
\hfill ИНФОРМАТИКА И ЕЁ ПРИМЕНЕНИЯ\ \ \ том~13\ \ \ выпуск~4\ \ \ 2019}
}%
 \def\rightfootline{\small{ИНФОРМАТИКА И ЕЁ ПРИМЕНЕНИЯ\ \ \ том~13\ \ \ выпуск~4\ \ \ 2019
 \hfill \textbf{\thepage}}}


\noindent
{\tabcolsep=3pt
\begin{tabular}{p{394pt}cc}
&\textbf{Вып.} & \textbf{Стр.}\\[3pt]
\Avtors{Гольская~А.\,А.} см.\ Маркова~Е.\,В.&&\\
\Avtors{Гончаров~А.\,А., Зацман~И.\,М., Кружков~М.\,Г.} Темпоральные данные в~лексикографиче-\linebreak
\\[-12pt]
\hspace*{23pt}ских базах знаний&4&90--96\\
\Avtors{Гончаров~А.\,А., Инькова~О.\,Ю.} Методика поиска имплицитных логико-семантических\linebreak
\\[-12pt]
\hspace*{23pt}отношений в~тексте&3&\hphantom{1}97--104\\
\Avtors{Горбунова~А.\,В., Наумов~В.\,А., Гайдамака~Ю.\,В., Самуйлов~К.\,Е.} Ресурсные системы\linebreak
\\[-12pt]
\hspace*{23pt}массового обслуживания с~произвольным обслуживанием&1&\hphantom{1}99--107\\
\Avtors{Горшенин~А.\,К., Кузьмин~В.\,Ю.} Оптимизация гиперпараметров нейронных сетей с~ис-\linebreak
\\[-12pt]
\hspace*{23pt}пользованием высокопроизводительных вычислений для~предсказания осадков&1&75--81\\
\Avtors{Горшенин~А.\,К., Кузьмин~В.\,Ю.} Применение рекуррентных нейронных сетей для\linebreak
\\[-12pt]
\hspace*{23pt}прогнозирования моментов конечных нормальных смесей&3&114--121\\
\Avtors{Горшенин~А.\,К., Мартынов~О.\,П.} Гибридные модели экстремального градиентного\linebreak
\\[-12pt]
\hspace*{23pt}бустинга для восстановления пропущенных значений в~данных об~осадках&3&34--40\\
\Avtors{Грабовой~А.\,В., Бахтеев~О.\,Ю., Стрижов~В.\,В.} Определение релевантности параметров\linebreak
\\[-12pt]
\hspace*{23pt}нейросети&2&62--70\\
\Avtors{Гринченко~С.\,Н.} О генезисе информационного общества: информатико-кибернетиче-\linebreak
\\[-12pt]
\hspace*{23pt}ское модельное представление&2&100--108\\
\Avtors{Грушо~А.\,А., Грушо~Н.\,А., Тимонина~Е.\,Е.} Использование метаданных для реализации\linebreak
\\[-12pt]
\hspace*{23pt}требований политики безопасности MLS&4&85--89\\
\Avtors{Грушо~А.\,А., Грушо~Н.\,А., Тимонина~Е.\,Е.} Методы выявления <<слабых>> признаков\linebreak
\\[-12pt]
\hspace*{23pt}нарушений информационной безопасности&3&3--8\\
\Avtors{Грушо~А.\,А., Забежайло~М.\,И., Грушо~Н.\,А., Тимонина~Е.\,Е.} Архитектурные решения в~задаче выявления мошенничества при анализе информационных потоков\linebreak
\\[-12pt]
\hspace*{23pt}в~цифровой экономике&2&22--28\\
\Avtors{Грушо~А.\,А., Забежайло~М.\,И., Грушо~Н.\,А., Тимонина~Е.\,Е.} Формирование концептов\linebreak
\\[-12pt]
\hspace*{23pt}на основе малых выборок&4&81--84\\
\Avtors{Грушо~Н.\,А.} см.\ Грушо~А.\,А.&&\\
\Avtors{Грушо~Н.\,А.} см.\ Грушо~А.\,А.&&\\
\Avtors{Грушо~Н.\,А.} см.\ Грушо~А.\,А.&&\\
\Avtors{Грушо~Н.\,А.} см.\ Грушо~А.\,А.&&\\
\Avtors{Гудкова~И.\,А.} см.\ Маркова~Е.\,В.&&\\
\Avtors{Дзантиев~И.\,Л.} см.\ Маркова~Е.\,В.&&\\
\Avtors{Докукин~А.\,А.} см.\ Журавлев~Ю.\,И.&&\\
\Avtors{Дулин~С.\,К., Дулина~Н.\,Г., Кожунова~О.\,С.} Синтез геоданных в пространственных\linebreak
\\[-12pt]
\hspace*{23pt}инфраструктурах на~основе связанных данных&1&82--90\\
\Avtors{Дулина~Н.\,Г.} см.\ Дулин~С.\,К.&&\\
\Avtors{Дюкова~Е.\,В., Масляков~Г.\,О., Прокофьев~П.\,А.} О числе максимальных независимых\linebreak
\\[-12pt]
\hspace*{23pt}элементов частичных порядков (случай цепей)&1&25--32\\
\Avtors{Журавлев~Ю.\,И., Сенько~О.\,В., Бондаренко~Н.\,Н., Рязанов~В.\,В., Докукин~А.\,А., Виноградов~А.\,П.} Исследование возможности прогнозирования изменения финансового\linebreak
\\[-12pt]
\hspace*{23pt}состояния кредитной организации на основе публикуемой отчетности&4&30--35\\
\Avtors{Забежайло~М.\,И.} см.\ Грушо~А.\,А.&&\\
\Avtors{Забежайло~М.\,И.} см.\ Грушо~А.\,А.&&\\
\Avtors{Захарова~Т.\,В., Тархов~А.\,А.} Оценка уровня значимости критерия Шуирманна для\linebreak
\\[-12pt]
\hspace*{23pt}проверки гипотезы биоэквивалентности при наличии пропущенных данных&3&58--62\\
\Avtors{Зацаринный~А.\,А., Коротков~В.\,В., Матвеев~М.\,Г.} Моделирование процессов сетевого планирования портфеля проектов с~неоднородными ресурсами в~условиях нечет-\linebreak
\\[-12pt]
\hspace*{23pt}кой информации&2&92--99\\
\Avtors{Зацман~И.\,М.} Интерфейсы третьего порядка в~информатике&3&82--89\\
\Avtors{Зацман~И.\,М.} Кодирование концептов в~цифровой среде&4&\hphantom{1}97--106\\
\Avtors{Зацман~И.\,М.} Целенаправленное развитие систем лингвистических знаний: выявление\linebreak
\\[-12pt]
\hspace*{23pt}и~заполнение лакун&1&91--98\\
\Avtors{Зацман~И.\,М.} см.\ Гончаров~А.\,А.&&\\
\end{tabular}
}

\pagebreak

\def\leftkol{АВТОРСКИЙ УКАЗАТЕЛЬ ЗА 2019 г.} % ENGLISH ABSTRACTS}

\def\rightkol{АВТОРСКИЙ УКАЗАТЕЛЬ ЗА 2019 г.} %ENGLISH ABSTRACTS}

%\thispagestyle{myheadings}
\def\leftfootline{\small{\textbf{\thepage}
\hfill ИНФОРМАТИКА И ЕЁ ПРИМЕНЕНИЯ\ \ \ том~13\ \ \ выпуск~4\ \ \ 2019}
}%
 \def\rightfootline{\small{ИНФОРМАТИКА И ЕЁ ПРИМЕНЕНИЯ\ \ \ том~13\ \ \ выпуск~4\ \ \ 2019
 \hfill \textbf{\thepage}}}


\noindent
{\tabcolsep=3pt
\begin{tabular}{p{394pt}cc}
&\textbf{Вып.} & \textbf{Стр.}\\[3pt]
\Avtors{Зейфман~А.\,И., Сатин~Я.\,А., Киселева~К.\,М.} Об оценках скорости сходимости для некоторых моделей массового обслуживания с~неполно заданными интенсивно-\linebreak
\\[-12pt]
\hspace*{23pt}стями&3&14--19\\
\Avtors{Инькова~О.\,Ю., Кружков~М.\,Г.} Сочетаемость логико-семантических отношений: коли-\linebreak
\\[-12pt]
\hspace*{23pt}чественные методы анализа&2&83--91\\
\Avtors{Инькова~О.\,Ю.} см.\ Гончаров~А.\,А.&&\\
\Avtors{Кириков~И.\,А.} см.\ Румовская~С.\,Б.&&\\
\Avtors{Киселева~К.\,М.} см.\ Зейфман~А.\,И.&&\\
\Avtors{Ковалёв~Д.\,Ю., Тарасов~Е.\,А.} Виртуальные эксперименты в~исследованиях с~интенсив-\linebreak
\\[-12pt]
\hspace*{23pt}ным использованием данных&2&117--125\\
\Avtors{Кожунова~О.\,С.} см.\ Дулин~С.\,К.&&\\
\Avtors{Колесников~А.\,В., Листопад~С.\,В.} Протокол гетерогенного мышления гибридной интеллектуальной многоагентной системы для решения проблемы восстановления\linebreak
\\[-12pt]
\hspace*{23pt}распределительной электросети&2&76--82\\
\Avtors{Коновалов~М.\,Г., Разумчик~Р.\,В.} Комплексное управление в~одном классе систем\linebreak
\\[-12pt]
\hspace*{23pt}с~параллельным обслуживанием&4&54--59\\
\Avtors{Коновалов~М.\,Г.} см.\ Агаларов~Я.\,М.&&\\
\Avtors{Коротков~В.\,В.} см.\ Зацаринный~А.\,А.&&\\
\Avtors{Кривенко~М.\,П.} Выбор модели данных в~задачах медицинской диагностики&4&27--29\\
\Avtors{Кружков~М.\,Г.} см.\ Гончаров~А.\,А.&&\\
\Avtors{Кружков~М.\,Г.} см.\ Инькова~О.\,Ю.&&\\
\Avtors{Кудрявцев~А.\,А.} Априорное обобщенное гамма-распределение в~байесовских моделях\linebreak
\\[-12pt]
\hspace*{23pt}баланса&3&27--33\\
\Avtors{Кудрявцев~А.\,А.} О представлении 
гамма-экспоненциального и~обобщенного отрица-\linebreak
\\[-12pt]
\hspace*{23pt}тельного биномиального распределений&4&76--80\\
\Avtors{Кудрявцев~А.\,А., Палионная~С.\,И., Шоргин~В.\,С.} Априорные Фреше и масштабированное\linebreak
\\[-12pt]
\hspace*{23pt}обратное хи-распределение в~байесовских моделях баланса&1&62--66\\
\Avtors{Кудрявцев~А.\,А.} см.\ Арутюнов~Е.\,Н.&&\\
\Avtors{Кузьмин~В.\,Ю.} см.\ Горшенин~А.\,К.&&\\
\Avtors{Кузьмин~В.\,Ю.} см.\ Горшенин~А.\,К.&&\\
\Avtors{Курьянский~М.\,К.} см.\ Вышинский~Л.\,Л.&&\\
\Avtors{Ланге~M.\,M.} О~сравнительной эффективности схем классификации данных на~ансамбле\linebreak
\\[-12pt]
\hspace*{23pt}источников с~использованием средней взаимной информации&4&18--26\\
\Avtors{Лебедев~А.\,В.} Нетранзитивные триплеты непрерывных случайных величин и~их прило-\linebreak
\\[-12pt]
\hspace*{23pt}жения&3&20--26\\
\Avtors{Листопад~С.\,В.} см.\ Колесников~А.\,В.&&\\
\Avtors{Логачев~О.\,А., Сукаев~А.\,А., Федоров~С.\,Н.} Об одном методе решения систем 
квад\-ра\-тич\-ных булевых уравнений, использующем локальные аффинности булевых\linebreak
\\[-12pt]
\hspace*{23pt}функций&2&37--46\\
\Avtors{Логачев~О.\,А., Сукаев~А.\,А., Федоров~С.\,Н.} Полиномиальные алгоритмы вычисления\linebreak
\\[-12pt]
\hspace*{23pt}локальных аффинностей квадратичных булевых функций&1&67--74\\
\Avtors{Лукашенко~О.\,В., Морозов~Е.\,В., Пагано~М.} Гауссовская аппроксимация процесса\linebreak
\\[-12pt]
\hspace*{23pt}распределенных вычислений&2&109--116\\
\Avtors{Малашенко~Ю.\,Е., Назарова~И.\,А., Новикова~Н.\,М.} Анализ уязвимости многополюсных\linebreak
\\[-12pt]
\hspace*{23pt}сетей при~структурных повреждениях&1&33--39\\
\Avtors{Маркова~Е.\,В., Гольская~А.\,А., Дзантиев~И.\,Л., Гудкова~И.\,А., Шоргин~С.\,Я.} Сравнительный анализ показателей эффективности модели беспроводной сети меж\-ма\-шин\-ного взаимодействия, работающей в~рамках двух политик разделения радиоре-\linebreak
\\[-12pt]
\hspace*{23pt}сурсов&1&108--116\\
\Avtors{Мартынов~О.\,П.} см.\ Горшенин~А.\,К.&&\\
\Avtors{Масляков~Г.\,О.} см.\ Дюкова~Е.\,В.&&\\
\Avtors{Матвеев~М.\,Г.} см.\ Зацаринный~А.\,А.&&\\
\Avtors{Мейханаджян~Л.\,А., Разумчик~Р.\,В.} Система массового обслуживания Geo$/G/1/\infty$\linebreak
\\[-12pt]
\hspace*{23pt}синверсионным порядком обслуживания и~ресамплингом в~дискретном времени&4&60--67\\
\end{tabular}
}

\pagebreak

\def\leftkol{АВТОРСКИЙ УКАЗАТЕЛЬ ЗА 2019 г.} % ENGLISH ABSTRACTS}

\def\rightkol{АВТОРСКИЙ УКАЗАТЕЛЬ ЗА 2019 г.} %ENGLISH ABSTRACTS}

%\thispagestyle{myheadings}
\def\leftfootline{\small{\textbf{\thepage}
\hfill ИНФОРМАТИКА И ЕЁ ПРИМЕНЕНИЯ\ \ \ том~13\ \ \ выпуск~4\ \ \ 2019}
}%
 \def\rightfootline{\small{ИНФОРМАТИКА И ЕЁ ПРИМЕНЕНИЯ\ \ \ том~13\ \ \ выпуск~4\ \ \ 2019
 \hfill \textbf{\thepage}}}


\noindent
{\tabcolsep=3pt
\begin{tabular}{p{394pt}cc}
&\textbf{Вып.} & \textbf{Стр.}\\[3pt]
\Avtors{Мельников~А.\,В.} см.\ Бурлуцкий~В.\,В.&&\\
\Avtors{Миллер~Г.\,Б.} см.\ Босов~А.\,В.&&\\
\Avtors{Морозов~Е.\,В.} см.\ Лукашенко~О.\,В.&&\\
\Avtors{Мхитарян~Г.\,А.} см.\ Босов~А.\,В.&&\\
\Avtors{Назарова~И.\,А.} см.\ Малашенко~Ю.\,Е.&&\\
\Avtors{Наумов~А.\,В.} см.\ Босов~А.\,В.&&\\
\Avtors{Наумов~В.\,А.} см.\ Горбунова~А.\,В.&&\\
\Avtors{Новикова~Н.\,М.} см.\ Малашенко~Ю.\,Е.&&\\
\Avtors{Нуриев~В.\,А.} Архитектура системы нейронного машинного перевода&3&90--96\\
\Avtors{Осипова~В.\,А.} см.\ Абгарян~К.\,К.&&\\
\Avtors{Павлов~Ю.\,Л.} Об асимптотике кластерного коэффициента конфигурационного графа\linebreak
\\[-12pt]
\hspace*{23pt}с~неизвестным распределением степеней вершин&3&\hphantom{1}9--13\\
\Avtors{Пагано~М.} см.\ Лукашенко~О.\,В.&&\\
\Avtors{Палионная~С.\,И.} см.\ Кудрявцев~А.\,А.&&\\
\Avtors{Панов~А.\,И.} см.\ Смирнов~И.\,В.&&\\
\Avtors{Пенкин~Г.\,О.} см.\ Аникеев~Д.\,А.&&\\
\Avtors{Прокофьев~П.\,А.} см.\ Дюкова~Е.\,В.&&\\
\Avtors{Разумчик~Р.\,В.} см.\ Коновалов~М.\,Г.&&\\
\Avtors{Разумчик~Р.\,В.} см.\ Мейханаджян~Л.\,А.&&\\
\Avtors{Румовская~С.\,Б., Кириков~И.\,А.} Методы моделирования и~визуального представления\linebreak
\\[-12pt]
\hspace*{23pt}конфликта в~малом коллективе экспертов, решающих проблемы (обзор)&3&122--130\\
\Avtors{Рыбаков~К.\,А.} Об одном классе задач фильтрации на многообразиях&1&16--24\\
\Avtors{Рязанов~В.\,В.} см.\ Журавлев~Ю.\,И.&&\\
\Avtors{Самуйлов~К.\,Е.} см.\ Гайдамака~А.\,А.&&\\
\Avtors{Самуйлов~К.\,Е.} см.\ Горбунова~А.\,В.&&\\
\Avtors{Сапунова~А.\,П.} см.\ Босов~А.\,В.&&\\
\Avtors{Сатин~Я.\,А.} см.\ Зейфман~А.\,И.&&\\
\Avtors{Сейфуль-Мулюков~Р.\,Б.} Законы информатики и~синергетики в~познании сложных\linebreak
\\[-12pt]
\hspace*{23pt}систем&4&107--113\\
\Avtors{Сенько~О.\,В.} см.\ Журавлев~Ю.\,И.&&\\
\Avtors{Синицын~И.\,Н.} Интерполяционное аналитическое моделирование распределений\linebreak
\\[-12pt]
\hspace*{23pt}в~сложных стохастических системах&1&2--8\\
\Avtors{Скрынник~А.\,А.} см.\ Смирнов~И.\,В.&&\\
\Avtors{Смирнов~И.\,В., Панов~А.\,И., Скрынник~А.\,А., Чистова~Е.\,В.} Персональный когнитивный\linebreak
\\[-12pt]
\hspace*{23pt}ассистент: концепция и~принципы работы&3&105--113\\
\Avtors{Стефанович~А.\,И.} см.\ Босов~А.\,В.&&\\
\Avtors{Стефанович~А.\,И.} см.\ Босов~А.\,В.&&\\
\Avtors{Стрижов~В.\,В.} см.\ Аникеев~Д.\,А.&&\\
\Avtors{Стрижов~В.\,В.} см.\ Грабовой~А.\,В.&&\\
\Avtors{Сукаев~А.\,А.} см.\ Логачев~О.\,А.&&\\
\Avtors{Сукаев~А.\,А.} см.\ Логачев~О.\,А.&&\\
\Avtors{Сучков~А.\,П.} Научный результат как информационный объект в~контексте системы\linebreak
\\[-12pt]
\hspace*{23pt}управления научными сервисами&3&137--144\\
\Avtors{Тарасов~Е.\,А.} см.\ Ковалёв~Д.\,Ю.&&\\
\Avtors{Тархов~А.\,А.} см.\ Захарова~Т.\,В.&&\\
\Avtors{Тимонина~Е.\,Е.} см.\ Грушо~А.\,А.&&\\
\Avtors{Тимонина~Е.\,Е.} см.\ Грушо~А.\,А.&&\\
\Avtors{Тимонина~Е.\,Е.} см.\ Грушо~А.\,А.&&\\
\Avtors{Тимонина~Е.\,Е.} см.\ Грушо~А.\,А.&&\\
\Avtors{Титова~А.\,И.} см.\ Арутюнов~Е.\,Н.&&\\
\Avtors{Ушаков~В.\,Г., Ушаков~Н.\,Г.} Выходящие потоки в~однолинейной системе с~относитель-\linebreak
\\[-12pt]
\hspace*{23pt}ным приоритетом&4&42--47\\
\Avtors{Ушаков~В.\,Г.} см.\ Агаларов~Я.\,М.&&\\
\Avtors{Ушаков~Н.\,Г.} см.\ Ушаков~В.\,Г.&&\\
\end{tabular}
}

\pagebreak

\def\leftkol{АВТОРСКИЙ УКАЗАТЕЛЬ ЗА 2019 г.} % ENGLISH ABSTRACTS}

\def\rightkol{АВТОРСКИЙ УКАЗАТЕЛЬ ЗА 2019 г.} %ENGLISH ABSTRACTS}

%\thispagestyle{myheadings}
\def\leftfootline{\small{\textbf{\thepage}
\hfill ИНФОРМАТИКА И ЕЁ ПРИМЕНЕНИЯ\ \ \ том~13\ \ \ выпуск~4\ \ \ 2019}
}%
 \def\rightfootline{\small{ИНФОРМАТИКА И ЕЁ ПРИМЕНЕНИЯ\ \ \ том~13\ \ \ выпуск~4\ \ \ 2019
 \hfill \textbf{\thepage}}}


\noindent
{\tabcolsep=3pt
\begin{tabular}{p{394pt}cc}
&\textbf{Вып.} & \textbf{Стр.}\\[3pt]
\Avtors{Федоров~С.\,Н.} см.\ Логачев~О.\,А.&&\\
\Avtors{Федоров~С.\,Н.} см.\ Логачев~О.\,А.&&\\
\Avtors{Флеров~Ю.\,А.} см.\ Вышинский~Л.\,Л.&&\\
\Avtors{Царегородцев~А.\,Л.} см.\ Бурлуцкий~В.\,В.&&\\
\Avtors{Чистова~Е.\,В.} см.\ Смирнов~И.\,В.&&\\
\Avtors{Чухно~Н.\,В.} см.\ Гайдамака~А.\,А.&&\\
\Avtors{Чухно~О.\,В.} см.\ Гайдамака~А.\,А.&&\\
\Avtors{Шестаков~О.\,В.} Обращение однородных операторов с помощью стабилизированной\linebreak
\\[-12pt]
\hspace*{23pt}жесткой пороговой обработки при неизвестной дисперсии шума&1&49--54\\
\Avtors{Шестаков~О.\,В.} Свойства вейвлет-оценок сигналов, регистрируемых в~случайные\linebreak
\\[-12pt]
\hspace*{23pt}моменты времени&2&16--21\\
\Avtors{Шестаков~О.\,В.} Среднеквадратичный риск нелинейной регуляризации задачи обраще-\linebreak
\\[-12pt]
\hspace*{23pt}ния линейных однородных операторов при случайном объеме выборки&4&48--53\\
\Avtors{Шнурков~П.\,В., Вахтанов~Н.\,А.} Исследование проблемы оптимального управления запасом дискретного продукта в~стохастической модели регенерации с~непрерывно\linebreak
\\[-12pt]
\hspace*{23pt}происходящим потреблением и~случайной задержкой поставки&2&54--61\\
\Avtors{Шнурков~П.\,В., Вахтанов~Н.\,А.} О~решении проблемы оптимального управления запасом дискретного продукта в~стохастической модели регенерации с непрерывно\linebreak
\\[-12pt]
\hspace*{23pt}происходящим потреблением&3&50--57\\
\Avtors{Шоргин~В.\,С.} см.\ Кудрявцев~А.\,А.&&\\
\Avtors{Шоргин~С.\,Я.} см.\ Гайдамака~А.\,А.&&\\
\Avtors{Шоргин~С.\,Я.} см.\ Маркова~Е.\,В.&&\\
\Avtors{Якимчук~А.\,В.} см.\ Бурлуцкий~В.\,В.&&\\
\end{tabular}
}

%\thispagestyle{myheadings}
\def\leftfootline{\small{\textbf{\thepage}
\hfill ИНФОРМАТИКА И ЕЁ ПРИМЕНЕНИЯ\ \ \ том~13\ \ \ выпуск~4\ \ \ 2019}
}%
 \def\rightfootline{\small{ИНФОРМАТИКА И ЕЁ ПРИМЕНЕНИЯ\ \ \ том~13\ \ \ выпуск~4\ \ \ 2019
 \hfill \textbf{\thepage}}}

 \label{end\stat}

\newpage

\def\stat{cont-e}
{%\hrule\par
%\vskip 7pt % 7pt
\raggedleft\Large \bf%\baselineskip=3.2ex
2\,0\,1\,9\ \ A\,U\,T\,H\,O\,R\ \ I\,N\,D\,E\,X \vskip 17pt
 \hrule
 \par
\vskip 21pt plus 6pt minus 3pt }

\label{st\stat}

\def\tit{\ }

\def\aut{\ }
\def\auf{\ }

\def\leftkol{\ } %2019 AUTHOR INDEX} % ENGLISH ABSTRACTS}

\def\rightkol{\ } %2019 AUTHOR INDEX} %ENGLISH ABSTRACTS}

\titele{\tit}{\aut}{\auf}{\leftkol}{\rightkol}
\addcontentsline{toc}{subsection}{\textrm\textbf 2019 Author Index}

\def\leftfootline{\small{\textbf{\thepage}
\hfill INFORMATIKA I EE PRIMENENIYA~--- INFORMATICS AND APPLICATIONS\ \ \ 2019\
\ \ volume~13\ \ \ issue\ 4}
}%
 \def\rightfootline{\small{INFORMATIKA I EE PRIMENENIYA~--- INFORMATICS AND APPLICATIONS\ \ \ 2019\ \ \ volume~13\ \ \ issue\ 4
\hfill \textbf{\thepage}}}

%\vspace*{-12pt}

\noindent
{\tabcolsep=3pt
\begin{tabular}{p{396pt}cc}
&\textbf{Issue} & \textbf{Page}\\[6pt]
\Avtors{Abgaryan~K.\,K.\ and Osipova~V.\,A.} Application of decision support methods for the multicriterial\linebreak
\\[-12pt]
\hspace*{23pt}selection of multiscale compositions&2&47--53\\
\Avtors{Agalarov~Ya.\,M.\ and Konovalov~M.\,G.} Proof of the unimodality of the objective function in\linebreak
\\[-12pt]
\hspace*{23pt}$M/M/N$ queue with threshold-based congestion control&2&2--6\\
\Avtors{Agalarov~Ya.\,M.\ and Ushakov~V.\,G.} On the unimodality of the~income function of a~type $G|M|s$\linebreak
\\[-12pt]
\hspace*{23pt}queueing system with controlled queue&1&55--61\\
\Avtors{Agasandyan~G.\,A.} Performance estimations for optimal-on-CC-VaR portfolios in option markets&3&72--81\\
\Avtors{Agasandyan~G.\,A.} Theoretical foundations of~continuous VaR criterion optimization in~the~col-\linebreak
\\[-12pt]
\hspace*{23pt}lection of~markets&4&36--41\\
\Avtors{Anashin~V.\,S.} On automata models of blockchain&2&29--36\\
\Avtors{Anikeyev~D.\,A., Penkin~G.\,O., and Strijov~V.\,V.} Local approximation models for~human physical\linebreak
\\[-12pt]
\hspace*{23pt}activity classification&1&40--48\\
\Avtors{Arutyunov~E.\,N., Kudryavtsev~A.\,A., and Titova~A.\,I.} Bayesian models of factors balance with\linebreak
\\[-12pt]
\hspace*{23pt}\textit{a~priori} Weibull and Nakagami distributions&2&71--75\\
\Avtors{Bakhteev~O.\,Yu.} see Grabovoy~A.\,V.&&\\
\Avtors{Bondarenko~N.\,N.} see Zhuravlev~Yu.\,I.&&\\
\Avtors{Borisov~A.\,V.} Numerical schemes of markov jump process filtering given discretized observa-\linebreak
\\[-12pt]
\hspace*{23pt}tions~I:~Accuracy characteristics&4&68--75\\
\Avtors{Bosov~A.\,V.\ and Miller~G.\,B.} On the conditionally minimax nonlinear filtering concept\linebreak
\\[-12pt]
\hspace*{23pt}development: Filter modification and analysis&2&\hphantom{1}7--15\\
\Avtors{Bosov~A.\,V., Naumov~A.\,V., Mkhitaryan~G.\,A., and Sapunova~A.\,P.} Using the model of~gamma\linebreak
\\[-12pt]
\hspace*{23pt}distribution in~the~problem of~forming a~time-limited test in~a~distance learning system&4&11--17\\
\Avtors{Bosov~A.\,V.\ and Stefanovich~A.\,I.} Stochastic differential system output control by~the~quadratic\linebreak
\\[-12pt]
\hspace*{23pt}criterion. II.~Dynamic programming equations numerical solution&1&\hphantom{1}9--15\\
\Avtors{Bosov~A.\,V.\ and Stefanovich~A.\,I.} Stochastic differential system output control by~the~quadratic\linebreak
\\[-12pt]
\hspace*{23pt}criterion. III.~Optimal control properties analysis&3&41--49\\

\Avtors{Burlutskiy~V.\,V., Yakimchuk~A.\,V., Melnikov~A.\,V., Tsaregorodtsev~A.\,L., and Voloshin~S.\,V.} Development of a method for the formation of~attribute space and a~model for~the~assessment and prediction of anthropogenic influence on~the~environment (on~the~example of~the~forest fund of the~oil-producing region)&3& 131--136\\
\Avtors{Chistova~E.\,V.} see Smirnov~I.\,V.&&\\
\Avtors{Chukhno~N.\,V.} see Gaidamaka~A.\,A.&&\\
\Avtors{Chukhno~O.\,V.} see Gaidamaka~A.\,A.&&\\
\Avtors{Djukova~E.\,V., Maslyakov~G.\,O., and Prokofyev~P.\,A.} On the number of maximal independent\linebreak
\\[-12pt]
\hspace*{23pt}elements of~partially ordered sets (the case of~chains)&1&25--32\\
\Avtors{Dokukin~A.\,A.} see Zhuravlev~Yu.\,I.&&\\
\Avtors{Dulin~S.\,K., Dulina~N.\,G., and Kozhunova~O.\,S.} Synthesis of geodata in spatial infrastructures\linebreak
\\[-12pt]
\hspace*{23pt}based on related data&1&82--90\\
\Avtors{Dulina~N.\,G.} see Dulin~S.\,K.&&\\
\Avtors{Dzantiev~I.\,L.} see Markova~E.\,V.&&\\
\Avtors{Fedorov~S.\,N.} see Logachev~O.\,A.&&\\
\Avtors{Fedorov~S.\,N.} see Logachev~O.\,A.&&\\
\end{tabular}
}
\pagebreak

\def\leftfootline{\small{\textbf{\thepage}
\hfill INFORMATIKA I EE PRIMENENIYA~--- INFORMATICS AND APPLICATIONS\ \ \ 2019\
\ \ volume~13\ \ \ issue\ 4}
}%
 \def\rightfootline{\small{INFORMATIKA I EE PRIMENENIYA~---
INFORMATICS AND APPLICATIONS\ \ \ 2019\ \ \ volume~13\ \ \ issue\ 4
\hfill \textbf{\thepage}}}

\def\leftkol{2019 AUTHOR INDEX} % ENGLISH ABSTRACTS}

\def\rightkol{2019 AUTHOR INDEX} %ENGLISH ABSTRACTS}


\noindent
{\tabcolsep=3pt
\begin{tabular}{p{395.48108pt}cc}
&\textbf{Issue} & \textbf{Page}\\[6pt]
\Avtors{Flerov~Yu.\,A.} see Vyshinsky~L.\,L.&&\\
\Avtors{Gaidamaka~A.\,A., Chukhno~N.\,V., Chukhno~O.\,V., Samouylov~K.\,E., and Shorgin~S.\,Ya.} Formalization of the alternatives ranking method for group decision making in social net-\linebreak
\\[-12pt]
\hspace*{23pt}works&3&63--71\\
\Avtors{Gaidamaka~Yu.\,V.} see Gorbunova~A.\,V.&&\\
\Avtors{Golskaia~A.\,A.} see Markova~E.\,V.&&\\
\Avtors{Goncharov~A.\,A.\ and Inkova~O.\,Yu.} Methods for identification of implicit logical-semantic\linebreak
\\[-12pt]
\hspace*{23pt}relations in~texts&3&\hphantom{1}97--104\\
\Avtors{Goncharov~A.\,A., Zatsman~I.\,M., and Kruzhkov~M.\,G.} Temporal data in~lexicographic databases&4&90--96\\
\Avtors{Gorbunova~A.\,V., Naumov~V.\,A., Gaidamaka~Yu.\,V., and Samouylov~K.\,E.} Resource queuing\linebreak
\\[-12pt]
\hspace*{23pt}systems with general service discipline&1&\hphantom{1}99--107\\
\Avtors{Gorshenin~A.\,K.\ and Kuzmin~V.\,Yu.} Application of recurrent neural networks to~forecasting\linebreak
\\[-12pt]
\hspace*{23pt}the~moments of~finite normal mixtures&3&114--121\\
\Avtors{Gorshenin~A.\,K.\ and Kuzmin~V.\,Yu.} Optimization of hyperparameters of neural networks using\linebreak
\\[-12pt]
\hspace*{23pt}high-performance computing for prediction of precipitation&1&75--81\\
\Avtors{Gorshenin~A.\,K.\ and Martynov~O.\,P.} Hybrid extreme gradient boosting models to~impute\linebreak
\\[-12pt]
\hspace*{23pt}the~missing data in~precipitation records&3&34--40\\
\Avtors{Grabovoy~A.\,V., Bakhteev~O.\,Yu., and Strijov~V.\,V.} Estimation of the relevance of the neural\linebreak
\\[-12pt]
\hspace*{23pt}network parameters&2&62--70\\
\Avtors{Grinchenko~S.\,N.} On the genesis of the information society: Informatics-cybernetic model\linebreak
\\[-12pt]
\hspace*{23pt}representation&2&100--108\\
\Avtors{Grusho~A.\,A., Grusho~N.\,A., and Timonina~E.\,E.} Methods of identification of ``weak'' signs of\linebreak
\\[-12pt]
\hspace*{23pt}violations of information security&3&3--8\\
\Avtors{Grusho~A.\,A., Grusho~N.\,A., and Timonina~E.\,E.} Using metadata to~implement multilevel security\linebreak
\\[-12pt]
\hspace*{23pt}policy requirements&4&85--89\\
\Avtors{Grusho~A.\,A., Zabezhailo~M.\,I., Grusho~N.\,A., and Timonina~E.\,E.} Architectural decisions in the problem of identification of~fraud in~the~analysis of~information flows in~digital eco-\linebreak
\\[-12pt]
\hspace*{23pt}nomy&2&22--28\\
\Avtors{Grusho~A.\,A., Zabezhailo~M.\,I., Grusho~N.\,A., and Timonina~E.\,E.} Concepts forming on~the~basis\linebreak
\\[-12pt]
\hspace*{23pt}of~small samples&4&81--84\\
\Avtors{Grusho~N.\,A.} see Grusho~A.\,A.&&\\
\Avtors{Grusho~N.\,A.} see Grusho~A.\,A.&&\\
\Avtors{Grusho~N.\,A.} see Grusho~A.\,A.&&\\
\Avtors{Grusho~N.\,A.} see Grusho~A.\,A.&&\\
\Avtors{Gudkova~I.\,A.} see Markova~E.\,V.&&\\
\Avtors{Inkova~O.\,Yu.\ and Kruzhkov~M.\,G.} Compatibility of logical semantic relations: Methods\linebreak
\\[-12pt]
\hspace*{23pt}of~quantitative analysis&2&83--91\\
\Avtors{Inkova~O.\,Yu.} see Goncharov~A.\,A.&&\\
\Avtors{Kirikov~I.\,A.} see Rumovskaya~S.\,B.&&\\
\Avtors{Kiseleva~K.\,M.} see Zeifman~A.\,I.&&\\
\Avtors{Kolesnikov~A.\,V.\ and Listopad~S.\,V.} Heterogeneous thinking protocol of hybrid intelligent\linebreak
\\[-12pt]
\hspace*{23pt}multiagent system for~solving distributional power grid recovery problem&2&76--82\\
\Avtors{Konovalov~M.\,G.\ and Razumchik~R.\,V.} Mixed policies for~online job allocation in~one class\linebreak
\\[-12pt]
\hspace*{23pt}of~systems with~parallel service&4&54--59\\
\Avtors{Konovalov~M.\,G.} see Agalarov~Ya.\,M.&&\\
\Avtors{Korotkov~V.\,V.} see Zatsarinny~A.\,A.&&\\
\Avtors{Kovalev~D.\,Y.\ and Tarasov~E.\,A.} Virtual experiments in data intensive research&2&117--125\\
\Avtors{Kozhunova~O.\,S.} see Dulin~S.\,K.&&\\
\Avtors{Krivenko~M.\,P.} Data model selection in~medical diagnostic tasks&4&27--29\\
\Avtors{Kruzhkov~M.\,G.} see Goncharov~A.\,A.&&\\
\Avtors{Kruzhkov~M.\,G.} see Inkova~O.\,Yu.&&\\
\Avtors{Kudryavtsev~A.\,A.} \textit{A priori} generalized gamma distribution in Bayesian balance models&3&27--33\\
\Avtors{Kudryavtsev~A.\,A.} On the representation of gamma-exponential and~generalized negative\linebreak
\\[-12pt]
\hspace*{23pt}binomial distributions&4&76--80\\
\end{tabular}
}
\pagebreak

\def\leftfootline{\small{\textbf{\thepage}
\hfill INFORMATIKA I EE PRIMENENIYA~--- INFORMATICS AND APPLICATIONS\ \ \ 2019\
\ \ volume~13\ \ \ issue\ 4}
}%
 \def\rightfootline{\small{INFORMATIKA I EE PRIMENENIYA~---
INFORMATICS AND APPLICATIONS\ \ \ 2019\ \ \ volume~13\ \ \ issue\ 4
\hfill \textbf{\thepage}}}

\def\leftkol{2019 AUTHOR INDEX} % ENGLISH ABSTRACTS}

\def\rightkol{2019 AUTHOR INDEX} %ENGLISH ABSTRACTS}


\noindent
{\tabcolsep=3pt
\begin{tabular}{p{395.48108pt}cc}
&\textbf{Issue} & \textbf{Page}\\[6pt]
\Avtors{Kudryavtsev~A.\,A., Palionnaia~S.\,I., and Shorgin~V.\,S.} \textit{A priori} Frechet and~scaled inverse chi\linebreak
\\[-12pt]
\hspace*{23pt}distribution in~Bayesian balance models&1&62--66\\
\Avtors{Kudryavtsev~A.\,A.} see Arutyunov~E.\,N.&&\\
\Avtors{Kuryansky~M.\,K.} see Vyshinsky~L.\,L.&&\\
\Avtors{Kuzmin~V.\,Yu.} see Gorshenin~A.\,K.&&\\
\Avtors{Kuzmin~V.\,Yu.} see Gorshenin~A.\,K.&&\\
\Avtors{Lange~M.\,M.} On comparative efficiency of classification schemes in an ensemble of data\linebreak
\\[-12pt]
\hspace*{23pt}sources using average mutual information&4&18--26\\
\Avtors{Lebedev~A.\,V.} Nontransitive triplets of continuous random variables and their applications&3&20--26\\
\Avtors{Listopad~S.\,V.} see Kolesnikov~A.\,V.&&\\
\Avtors{Logachev~O.\,A., Sukayev~A.\,A., and Fedorov~S.\,N.} On local affinity based method of solving\linebreak
\\[-12pt]
\hspace*{23pt}systems of quadratic Boolean equations&2&37--46\\
\Avtors{Logachev~O.\,A., Sukayev~A.\,A., and Fedorov~S.\,N.} Polynomial algorithms for~constructing local\linebreak
\\[-12pt]
\hspace*{23pt}affinities of~quadratic Boolean functions&1&67--74\\
\Avtors{Lukashenko~O.\,V., Morozov~E.\,V., and Pagano~M.} A~Gaussian approximation of~the~distributed\linebreak
\\[-12pt]
\hspace*{23pt}computing process&2&109--116\\
\Avtors{Malashenko~Yu.\,E., Nazarova~I.\,A., and Novikova~N.\,M.} Vulnerability analysis of multipolar\linebreak
\\[-12pt]
\hspace*{23pt}networks after structural damages&1&33--39\\
\Avtors{Markova~E.\,V., Golskaia~A.\,A., Dzantiev~I.\,L., Gudkova~I.\,A., and Shorgin~S.\,Ya.} Comparative analysis of performance measures for a wireless machine-to-machine network model\linebreak
\\[-12pt]
\hspace*{23pt}operating within two radio resource management policies&1&108--116\\
\Avtors{Martynov~O.\,P.} see Gorshenin~A.\,K.&&\\
\Avtors{Maslyakov~G.\,O.} see Djukova~E.\,V.&&\\
\Avtors{Matveev~M.\,G.} see Zatsarinny~A.\,A.&&\\
\Avtors{Melnikov~A.\,V.} see Burlutskiy~V.\,V.&&\\
\Avtors{Meykhanadzhyan~L.\,A.\ and Razumchik~R.\,V.} Discrete-time $\mathrm{GEO}/G/1/\infty$ LIFO queue with\linebreak
\\[-12pt]
\hspace*{23pt}resampling policy&4&60--67\\
\Avtors{Miller~G.\,B.} see Bosov~A.\,V.&&\\
\Avtors{Mkhitaryan~G.\,A.} see Bosov~A.\,V.&&\\
\Avtors{Morozov~E.\,V.} see Lukashenko~O.\,V.&&\\
\Avtors{Naumov~A.\,V.} see Bosov~A.\,V.&&\\
\Avtors{Naumov~V.\,A.} see Gorbunova~A.\,V.&&\\
\Avtors{Nazarova~I.\,A.} see Malashenko~Yu.\,E.&&\\
\Avtors{Novikova~N.\,M.} see Malashenko~Yu.\,E.&&\\
\Avtors{Nuriev~V.\,A.} Architecture of a~machine translation system&3&90--96\\
\Avtors{Osipova~V.\,A.} see Abgaryan~K.\,K.&&\\
\Avtors{Pagano~M.} see Lukashenko~O.\,V.&&\\
\Avtors{Palionnaia~S.\,I.} see Kudryavtsev~A.\,A.&&\\
\Avtors{Panov~A.\,I.} see Smirnov~I.\,V.&&\\
\Avtors{Pavlov~Yu.\,L.} On the asymptotics of clustering coefficient in~a~configuration graph with unknown\linebreak
\\[-12pt]
\hspace*{23pt}distribution of~vertex degrees&3&\hphantom{1}9--13\\
\Avtors{Penkin~G.\,O.} see Anikeyev~D.\,A.&&\\
\Avtors{Prokofyev~P.\,A.} see Djukova~E.\,V.&&\\
\Avtors{Razumchik~R.\,V.} see Konovalov~M.\,G.&&\\
\Avtors{Razumchik~R.\,V.} see Meykhanadzhyan~L.\,A.&&\\
\Avtors{Rumovskaya~S.\,B.\ and Kirikov~I.\,A.} Methods of modeling and visual representation of~a~conflict\linebreak
\\[-12pt]
\hspace*{23pt}in~a~small collective of experts solving problems (review)&3&122--130\\
\Avtors{Ryazanov~V.\,V.} see Zhuravlev~Yu.\,I.&&\\
\Avtors{Rybakov~K.\,A.} On a class of filtering problems on~manifolds&1&16--24\\
\Avtors{Samouylov~K.\,E.} see Gaidamaka~A.\,A.&&\\
\Avtors{Samouylov~K.\,E.} see Gorbunova~A.\,V.&&\\
\Avtors{Sapunova~A.\,P.} see Bosov~A.\,V.&&\\
\Avtors{Satin~Y.\,A.} see Zeifman~A.\,I.&&\\
\end{tabular}
}
\pagebreak

\def\leftfootline{\small{\textbf{\thepage}
\hfill INFORMATIKA I EE PRIMENENIYA~--- INFORMATICS AND APPLICATIONS\ \ \ 2019\
\ \ volume~13\ \ \ issue\ 4}
}%
 \def\rightfootline{\small{INFORMATIKA I EE PRIMENENIYA~---
INFORMATICS AND APPLICATIONS\ \ \ 2019\ \ \ volume~13\ \ \ issue\ 4
\hfill \textbf{\thepage}}}

\def\leftkol{2019 AUTHOR INDEX} % ENGLISH ABSTRACTS}

\def\rightkol{2019 AUTHOR INDEX} %ENGLISH ABSTRACTS}


\noindent
{\tabcolsep=3pt
\begin{tabular}{p{395.48108pt}cc}
&\textbf{Issue} & \textbf{Page}\\[6pt]
\Avtors{Sen'ko~O.\,V.} see Zhuravlev~Yu.\,I.&&\\
\Avtors{Seyful-Mulyukov~R.\,B.} Understanding of~complex systems using~the~laws of~synergetics\linebreak
\\[-12pt]
\hspace*{23pt}and~informatics&4&107--113\\
\Avtors{Shestakov~O.\,V.} Inversion of homogeneous operators using stabilized hard thresholding with\linebreak
\\[-12pt]
\hspace*{23pt}unknown noise variance&1&49--54\\
\Avtors{Shestakov~O.\,V.} Properties of wavelet estimates of signals recorded at random time points&2&16--21\\
\Avtors{Shestakov~O.\,V.} The mean square risk of~nonlinear regularization in~the~problem of~inversion\linebreak
\\[-12pt]
\hspace*{23pt}of~linear homogeneous operators with~a~random sample size&4&48--53\\
\Avtors{Shnurkov~P.\,V.\ and Vakhtanov~N.\,A.} On the solution of the optimal control problem of inventory of~a~discrete product in~the~stochastic model of~regeneration with continuously\linebreak
\\[-12pt]
\hspace*{23pt}occuring consumption&3&50--57\\
\Avtors{Shnurkov~P.\,V.\ and Vakhtanov~N.\,A.} Research of the optimal control problem of~inventory of~a~discrete product in~the~stochastic regeneration model with continuously\linebreak
\\[-12pt]
\hspace*{23pt}occuring consumption and random delivery delay&2&54--61\\
\Avtors{Shorgin~S.\,Ya.} see Gaidamaka~A.\,A.&&\\
\Avtors{Shorgin~S.\,Ya.} see Markova~E.\,V.&&\\
\Avtors{Shorgin~V.\,S.} see Kudryavtsev~A.\,A.&&\\
\Avtors{Sinitsyn~I.\,N.} Interpolatonal analytical modeling in~complex stochastic systems&1&2--8\\
\Avtors{Skrynnik~A.\,A.} see Smirnov~I.\,V.&&\\
\Avtors{Smirnov~I.\,V., Panov~A.\,I., Skrynnik~A.\,A., and Chistova~E.\,V.} Personal cognitive assistant: \linebreak
\\[-12pt]
\hspace*{23pt}Concept and key principals&3&105--113\\
\Avtors{Stefanovich~A.\,I.} see Bosov~A.\,V.&&\\
\Avtors{Stefanovich~A.\,I.} see Bosov~A.\,V.&&\\
\Avtors{Strijov~V.\,V.} see Anikeyev~D.\,A.&&\\
\Avtors{Strijov~V.\,V.} see Grabovoy~A.\,V.&&\\
\Avtors{Suchkov~A.\,P.} The scientific result as~the~information object in~the~context of~the~scientific\linebreak
\\[-12pt]
\hspace*{23pt}services system management&3&137--144\\
\Avtors{Sukayev~A.\,A.} see Logachev~O.\,A.&&\\
\Avtors{Sukayev~A.\,A.} see Logachev~O.\,A.&&\\
\Avtors{Tarasov~E.\,A.} see Kovalev~D.\,Y.&&\\
\Avtors{Tarkhov~A.\,A.} see Zakharova~T.\,V.&&\\
\Avtors{Timonina~E.\,E.} see Grusho~A.\,A.&&\\
\Avtors{Timonina~E.\,E.} see Grusho~A.\,A.&&\\
\Avtors{Timonina~E.\,E.} see Grusho~A.\,A.&&\\
\Avtors{Timonina~E.\,E.} see Grusho~A.\,A.&&\\
\Avtors{Titova~A.\,I.} see Arutyunov~E.\,N.&&\\
\Avtors{Tsaregorodtsev~A.\,L.} see Burlutskiy~V.\,V.&&\\
\Avtors{Ushakov~N.\,G.} see Ushakov~V.\,G.&&\\
\Avtors{Ushakov~V.\,G.\ and Ushakov~N.\,G.} The output streams in~the~single server queueing system\linebreak
\\[-12pt]
\hspace*{23pt}with~a~head of~the~line priority&4&42--47\\
\Avtors{Ushakov~V.\,G.} see Agalarov~Ya.\,M.&&\\
\Avtors{Vakhtanov~N.\,A.} see Shnurkov~P.\,V.&&\\
\Avtors{Vakhtanov~N.\,A.} see Shnurkov~P.\,V.&&\\
\Avtors{Vinogradov~A.\,P.} see Zhuravlev~Yu.\,I.&&\\
\Avtors{Voloshin~S.\,V.} see Burlutskiy~V.\,V.&&\\
\Avtors{Vyshinsky~L.\,L., Kuryansky~M.\,K., and Flerov~Yu.\,A.} Digital model of the aircraft's weight\linebreak
\\[-12pt]
\hspace*{23pt}passport&4&\hphantom{1}3--10\\
\Avtors{Yakimchuk~A.\,V.} see Burlutskiy~V.\,V.&&\\
\Avtors{Zabezhailo~M.\,I.} see Grusho~A.\,A.&&\\
\Avtors{Zabezhailo~M.\,I.} see Grusho~A.\,A.&&\\
\Avtors{Zakharova~T.\,V.\ and Tarkhov~A.\,A.} Evaluation of the significance level in schuirmann's test for\linebreak
\\[-12pt]
\hspace*{23pt}checking the~bioequivalence hypothesis in~missing data conditions&3&58--62\\
\end{tabular}
}
\pagebreak

\def\leftfootline{\small{\textbf{\thepage}
\hfill INFORMATIKA I EE PRIMENENIYA~--- INFORMATICS AND APPLICATIONS\ \ \ 2019\
\ \ volume~13\ \ \ issue\ 4}
}%
 \def\rightfootline{\small{INFORMATIKA I EE PRIMENENIYA~---
INFORMATICS AND APPLICATIONS\ \ \ 2019\ \ \ volume~13\ \ \ issue\ 4
\hfill \textbf{\thepage}}}

\def\leftkol{2019 AUTHOR INDEX} % ENGLISH ABSTRACTS}

\def\rightkol{2019 AUTHOR INDEX} %ENGLISH ABSTRACTS}


\noindent
{\tabcolsep=3pt
\begin{tabular}{p{395.48108pt}cc}
&\textbf{Issue} & \textbf{Page}\\[6pt]
\Avtors{Zatsarinny~A.\,A., Korotkov~V.\,V., and Matveev~M.\,G.} Modeling the process of network planning\linebreak
\\[-12pt]
\hspace*{23pt}of~a~portfolio of~projects with heterogeneous resources under fuzziness&2&92--99\\
\Avtors{Zatsman~I.\,M.} Digital encoding of~concepts&4&\hphantom{1}97--106\\
\Avtors{Zatsman~I.\,M.} Goal-oriented development of~linguistic knowledge systems: Identifying and\linebreak
\\[-12pt]
\hspace*{23pt}filling of~lacunae&1&91--98\\
\Avtors{Zatsman~I.\,M.} Third-order interfaces in informatics&3&82--89\\
\Avtors{Zatsman~I.\,M.} see Goncharov~A.\,A.&&\\
\Avtors{Zeifman~A.\,I., Satin~Y.\,A., and Kiseleva~K.\,M.} On the bounds of the rate of convergence for\linebreak
\\[-12pt]
\hspace*{23pt}some queueing models with incompletely defined intensities&3&14--19\\
\Avtors{Zhuravlev~Yu.\,I., Sen'ko~O.\,V., Bondarenko~N.\,N., Ryazanov~V.\,V., Dokukin~A.\,A., and Vinogradov~A.\,P.} Research of~the~possibility to~forecast changes in~financial state of~a~credit\linebreak
\\[-12pt]
\hspace*{23pt}organization on~the~basis of~public financial statements&4&30--35\\
\end{tabular}
}

%\thispagestyle{myheadings}
\def\leftfootline{\small{\textbf{\thepage}
\hfill INFORMATIKA I EE PRIMENENIYA~--- INFORMATICS AND APPLICATIONS\ \ \ 2019\
\ \ volume~13\ \ \ issue\ 4}
}%
 \def\rightfootline{\small{INFORMATIKA I EE PRIMENENIYA~---
INFORMATICS AND APPLICATIONS\ \ \ 2019\ \ \ volume~13\ \ \ issue\ 4
\hfill \textbf{\thepage}}}

 \label{end\stat}

\newpage

%   \vspace*{-48pt}

\begin{center}
\vspace*{6pt}
\mbox{%
\epsfxsize=53.502mm
\epsfbox{foto-1.eps}
}
\end{center}

\vspace*{6pt} %Академик


   \begin{center}
\fbox{\Large\textbf{Профессор Игорь Алексеевич Ушаков}}\\[12pt]
\textbf{\large 22.01.1935--27.02.2015}
   \end{center}


   %\vspace*{2.5mm}

   \vspace*{5mm}

   \thispagestyle{empty}

%\

%\vspace*{-12pt}


Редакционный совет и редакционная коллегия журнала <<Информатика и~её применения>> с~глубоким прискорбием извещают, что 27~февраля 2015~г.\ после тяжелой
и~продолжительной болезни скончался Игорь Алексеевич Ушаков~--- доктор технических наук, профессор, член редколлегии журнала <<Информатика и ее применения>>.

Игорь Алексеевич Ушаков окончил Московский авиационный институт, в~1963~г.\ защитил кандидатскую, а~в~1968~г.~--- докторскую диссертацию. С~1958 по 1989~гг.\ работал в~ряде научно-исследовательских организаций СССР, в~том числе руководил отделами в~НИИ АА и~ВЦ АН СССР; с 1969 по 1989 гг. преподавал в~МФТИ (был профессором, а~затем заведующим кафедрой) и~в~МЭИ. С~1989~г.~---- в~США: являлся профессором университета Дж.\ Вашингтона, университета Дж.\ Мэйсона и~Калифорнийского университета, сотрудником компаний MCI, Qualcomm и Hughes.

И.\,А.~Ушаков с момента основания журнала <<Надежность и~контроль качества>> был заместителем ответственного редактора, а~затем на протяжении многих лет членом редколлегии. В~2006~г.\ основал электронный международный журнал ``Reliability: Theory \& Application'', главным редактором которого оставался до конца жизни.

Учебниками и справочниками по теории надежности, написанными И.\,А.~Ушаковым, пользовались и~пользуются несколько поколений ученых и~специалистов в~разных странах мира.

Игорь Алексеевич всегда уделял огромное внимание работе с~молодежью; более~50 его учеников защитили докторские и~кандидатские диссертации.

И.\,А.~Ушаков вел активную научно-про\-све\-ти\-тель\-скую деятельность. В~частности, он был одним из организаторов и~руководителей Московского кабинета качества и~надежности при Политехническом музее (целью этого Кабинета было оказание консультаций работникам промышленных предприятий и~чтение курсов лекций для инженеров, занимающихся проблемой надежности). Находясь в~США, И.\,А.~Ушаков создал международный ин\-тер\-нет-фо\-рум им.\ Б.\,В.~Гнеденко, объединивший около~400~видных специалистов по приложениям теории вероятностей и~математической статистики, преимущественно в~об\-ласти теории надежности и~анализа риска, из десятков стран мира; коллективным членов этого Форума является и~наш журнал. Цели Форума~--- содействие контактам между специалистами из разных стран, организация обмена профессиональными 
новостями и~информацией (новые публикации, предстоящие события и~др.). Также необходимо отметить большое число на\-уч\-но-по\-пу\-ляр\-ных работ, опубликованных И.\,А.~Ушаковым.

И.\,А.~Ушаков обладал большим личным обаянием, имел широкий круг интересов. Все знавшие И.\,А.~Ушакова всегда будут помнить его как замечательного ученого и~прекрасного человека.

\bigskip

Редакционный совет и редакционная коллегия журнала <<Информатика и~её применения>> 
выражают глубокие соболезнования родным и близким покойного, всем, кто его знал и~работал с~ним.


%\def\stat{cont}
{%\hrule\par
%\vskip 7pt % 7pt
\raggedleft\Large \bf%\baselineskip=3.2ex
А\,В\,Т\,О\,Р\,С\,К\,И\,Й\ \ У\,К\,А\,З\,А\,Т\,Е\,Л\,Ь\ \ З\,А\ \ 2\,0\,1\,0 г. \vskip 17pt
    \hrule
    \par
\vskip 21pt plus 6pt minus 3pt }

\label{st\stat}

\def\tit{\ }

\def\aut{\ }
\def\auf{\ }

\def\leftkol{\ } % ENGLISH ABSTRACTS}

\def\rightkol{\ } %АВТОРСКИЙ УКАЗАТЕЛЬ ЗА 2010 г.} %ENGLISH ABSTRACTS}

\titele{\tit}{\aut}{\auf}{\leftkol}{\rightkol}

\vspace*{-12pt}

{\tabcolsep=3pt
\begin{tabular}{p{388pt}rr}
&\textbf{Выпуск} & \textbf{Стр.}\\[6pt]
\hangindent=23pt\noindent\textbf{Арутюнян~А.\,Р.} Моделирование влияния деформаций отпечатков пальцев на 
точность\linebreak
\vspace*{-12pt}\\
\hspace*{23pt}дактилоскопической идентификации$\dotfill$&1&51\\
\hangindent=23pt\noindent\textbf{Архипов~О.\,П., Зыкова~З.\,П.} Интеграция гетерогенной информации о цветных 
пикселях\linebreak
\vspace*{-12pt}\\
\hspace*{23pt}и их цветовосприятии$\dotfill$&4&15\\
\hangindent=23pt\noindent\textbf{Баранов~С.\,И., Френкель~С.\,Л., Захаров~В.\,Н.} Полуформальная верификация 
цифрового устройства с конвейером, основанная на использовании алгоритмических машин\linebreak
\vspace*{-12pt}\\
\hspace*{23pt}состояния$\dotfill$&4&49\\
\textbf{Бекетова~И.\,В.} см.~Каратеев~С.\,Л.&&\\
\textbf{Белоусов~В.\,В.} см.~Синицын~И.\,Н.&&\\
\hangindent=23pt\noindent\textbf{Бенинг~В.\,Е., Королев~Р.\,А.} О предельном поведении мощностей критериев в 
случае\linebreak
\vspace*{-12pt}\\
\hspace*{23pt}распределения Лапласа$\dotfill$&2&63\\
\hangindent=23pt\noindent\textbf{Бенинг~В.\,Е., Сипина~А.\,В.} Асимптотическое разложение для мощности 
критерия,\linebreak
\vspace*{-12pt}\\
\hspace*{23pt}основанного на выборочной медиане, в случае распределения Лапласа$\dotfill$&1&18\\
\textbf{Бондаренко~А.\,В.} см.~Каратеев~С.\,Л.&&\\
\hangindent=23pt\noindent\textbf{Бородина~А.\,В., Морозов~Е.\,В.} Об оценивании асимптотики вероятности 
большого\linebreak
\vspace*{-12pt}\\
\hspace*{23pt}уклонения стационарной регенеративной очереди с одним прибором$\dotfill$&3&29\\
\hangindent=23pt\noindent\textbf{Бунтман~Н.\,В., Минель~Ж.-Л., Ле~Пезан~Д., Зацман~И.\,М.} Типология и 
компьютерное\linebreak
\vspace*{-12pt}\\
\hspace*{23pt}моделирование трудностей перевода$\dotfill$&3&77\\
\textbf{Визильтер~Ю.\,В.} см.~Каратеев~С.\,Л.&&\\
\hangindent=23pt\noindent\textbf{Гавриленко~С.\,В.} Оценки скорости сходимости распределений случайных сумм с 
безгранично делимыми индексами к нормальному закону$\dotfill$&4&81\\
\hangindent=23pt\noindent\textbf{Григорьева~М.\,Е., Шевцова~И.\,Г.} Уточнение неравенства 
Каца--Берри--Эссеена$\dotfill$&2&75\\
\hangindent=23pt\noindent\textbf{Грушо~А.\,А., Грушо~Н.\,А., Тимонина~Е.\,Е.} Поиск конфликтов в политиках 
безопасности: модель случайных графов$\dotfill$&3&38\\
\textbf{Грушо~Н.\,А.} см.~Грушо~А.\,А.&&\\
\hangindent=23pt\noindent\textbf{Гудков~В.\,Ю.} Математические модели изображения отпечатка пальца на основе 
описания линий$\dotfill$&1&58\\
\textbf{Гуртов~А.\,В.} см.~Лукьяненко~А.\,С.&&\\
\textbf{Желтов~С.\,Ю.} см.~Каратеев~С.\,Л.&&\\
\hangindent=23pt\noindent\textbf{Захаров~А.\,А., Серебряков~В.\,А.} Система управления электронной библиотекой 
LibMeta$\dotfill$&4&2\\
\textbf{Захаров~В.\,Н.} см.~Баранов~С.\,И.&&\\
\textbf{Захарова~Т.\,В.} см.~Матвеева~С.\,С.&&\\
\hangindent=23pt\noindent\textbf{Зацаринный~А.\,А., Чупраков~К.\,Г.} Некоторые аспекты выбора технологии для 
постро-\linebreak
\vspace*{-12pt}\\
\hspace*{23pt}ения систем отображения информации ситуационного центра$\dotfill$&3&59\\
\textbf{Зацман~И.\,М.} см.~Бунтман~Н.\,В.&&\\
\hangindent=23pt\noindent\textbf{Зейфман~А.\,И., Коротышева~А.\,В., Сатин~Я.\,А., Шоргин~С.\,Я.} Об 
устойчивости нестаци-\linebreak
\vspace*{-12pt}\\
\hspace*{23pt}онарных систем обслуживания с катастрофами$\dotfill$&3&9\\
\textbf{Зыкова~З.\,П.} см.~Архипов~О.\,П.&&\\
\hangindent=23pt\noindent\textbf{Илюшин~Г.\,Я., Соколов~И.\,А.} Организация управляемого доступа пользователей 
к\linebreak
\vspace*{-12pt}\\
\hspace*{23pt}разнородным ведомственным информационным ресурсам$\dotfill$&1&24\\
\hangindent=23pt\noindent\textbf{Кавагучи~Ю., Ульянов~В.\,В., Фуджикоши~Я.} Приближения для статистик, 
описывающих\linebreak
\vspace*{-12pt}\\
\hspace*{23pt}геометрические свойства данных большой размерности, с оценками 
ошибок$\dotfill$&1&12\\
\hangindent=23pt\noindent\textbf{Каратеев~С.\,Л., Бекетова~И.\,В., Ососков~М.\,В., Князь~В.\,А., 
Визильтер~Ю.\,В., Бондаренко~А.\,В., Желтов~С.\,Ю.} Автоматизированный контроль 
качества цифровых\linebreak
\vspace*{-12pt}\\
\hspace*{23pt}изображений для персональных документов$\dotfill$&1&65\\
\end{tabular}
}

\pagebreak

\def\leftkol{АВТОРСКИЙ УКАЗАТЕЛЬ ЗА 2010 г.} % ENGLISH ABSTRACTS}

\def\rightkol{АВТОРСКИЙ УКАЗАТЕЛЬ ЗА 2010 г.} %ENGLISH ABSTRACTS}

{\tabcolsep=3pt
\begin{tabular}{p{388pt}rr}
&\textbf{Выпуск} & \textbf{Стр.}\\[3pt]
\hangindent=23pt\noindent\textbf{Козеренко~Е.\,Б.} Лингвистические фильтры в статистических моделях машинного\linebreak
\vspace*{-12pt}\\
\hspace*{23pt}перевода$\dotfill$&2&83\\
\hangindent=23pt\noindent\textbf{Козеренко~Е.\,Б., Кузнецов~И.\,П.} Когнитивно-лингвистические представления в 
систе-\linebreak
\vspace*{-12pt}\\
\hspace*{23pt}мах обработки текстов$\dotfill$&3&69\\
\textbf{Князь~В.\,А.} см.~Каратеев~С.\,Л.&&\\
\hangindent=23pt\noindent\textbf{Колесников~А.\,В., Солдатов~С.\,А.} Алгоритм координации для гибридной 
интеллектуальной системы решения сложной задачи оперативно-производственного\linebreak
\vspace*{-12pt}\\
\hspace*{23pt}планирования$\dotfill$&4&61\\
\hangindent=23pt\noindent\textbf{Коновалов~М.\,Г.} О планировании потоков в системах вычислительных 
ресурсов$\dotfill$&2&3\\
\textbf{Конушин~А.\,С.} см.~Конушин~В.\,С.&&\\
\hangindent=23pt\noindent\textbf{Конушин~В.\,С., Кривовязь~Г.\,Р., Конушин~А.\,С.} Алгоритм распознавания людей 
в видео-\linebreak
\vspace*{-12pt}\\
\hspace*{23pt}последовательности по одежде$\dotfill$&1&74\\
\textbf{Корепанов~Э.\, Р.} см.~Синицын~И.\,Н.&&\\
\textbf{Королев~В.\,Ю.} см.~Соколов~И.\,А.&&\\
\textbf{Королев~Р.\,А.} см.~Бенинг~В.\,Е.&&\\
\textbf{Коротышева~А.\,В.} см.~Зейфман~А.\,И.&&\\
\hangindent=23pt\noindent\textbf{Кривенко~М.\,П.} Непараметрическое оценивание элементов байесовского 
клас\-си-\linebreak
\vspace*{-12pt}\\
\hspace*{23pt}фикатора$\dotfill$&2&13\\
\textbf{Кривовязь~Г.\,Р.} см.~Конушин~В.\,С.&&\\
\textbf{Крылов~А.\,С.} см.~Павельева~Е.\,А.&&\\
\hangindent=23pt\noindent\textbf{Крылов~В.\,А.} Моделирование и классификация многоканальных дистанционных\linebreak
\vspace*{-12pt}\\
\hspace*{23pt}изображений с использованием копул$\dotfill$&4&34\\
\hangindent=23pt\noindent\textbf{Крючин~О.\,В.} Разработка параллельных эвристических алгоритмов подбора 
весовых\linebreak
\vspace*{-12pt}\\
\hspace*{23pt}коэффициентов искусственной нейтронной сети$\dotfill$&2&53\\
\hangindent=23pt\noindent\textbf{Кудрявцев~А.\,А., Шоргин~С.\,Я.} Байесовские модели массового обслуживания и 
надеж-\linebreak
\vspace*{-12pt}\\
\hspace*{23pt}ности: характеристики среднего числа заявок в системе $M\vert M \vert 1\vert 
\infty$$\dotfill$&3&16\\
\hangindent=23pt\noindent\textbf{Кузнецов~А.\,А.} Связь между временными и структурно-топологическими 
характери-\linebreak
\vspace*{-12pt}\\
\hspace*{23pt}стиками диаграмм ритма сердца здоровых людей$\dotfill$&4&39\\
\textbf{Кузнецов~И.\,П.} см.~Козеренко~Е.\,Б.&&\\
\textbf{Ле~Пезан~Д.} см.~Бунтман~Н.\,В.&&\\
\hangindent=23pt\noindent\textbf{Лукьяненко~А.\,С., Морозов~Е.\,В., Гуртов~А.\,В.} Анализ сетевого протокола с общей 
функ-\linebreak
\vspace*{-12pt}\\
\hspace*{23pt}цией расширения окна передачи сообщения при конфликтах$\dotfill$&2&46\\
\hangindent=23pt\noindent\textbf{Лямин~О.\,О.} О предельном поведении мощностей критериев в случае обобщенного\linebreak
\vspace*{-12pt}\\
\hspace*{23pt}распределения Лапласа$\dotfill$&3&47\\
\hangindent=23pt\noindent\textbf{Маркин~А.\,В., Шестаков~О.\,В.} Асимптотики оценки риска при пороговой 
обработке\linebreak
\vspace*{-12pt}\\
\hspace*{23pt}вейвлет-вейглет коэффициентов в задаче томографии$\dotfill$&2&36\\
\hangindent=23pt\noindent\textbf{Матвеева~С.\,С., Захарова~Т.\,В.} Сети массового обслуживания с наименьшей 
длиной\linebreak
\vspace*{-12pt}\\
\hspace*{23pt}очереди$\dotfill$&3&22\\
\hangindent=23pt\noindent\textbf{Матюшенко~С.\,И.} Стационарные характеристики двухканальной системы 
обслужива-\linebreak
\vspace*{-12pt}\\
\hspace*{23pt}ния с переупорядочиванием заявок и распределениями фазового типа$\dotfill$&4&68\\
\textbf{Минель~Ж.-Л.} см.~Бунтман~Н.\,В.&&\\
\textbf{Морозов~Е.\,В.} см.~Бородина~А.\,В.&&\\
\textbf{Морозов~Е.\,В.} см.~Лукьяненко~А.\,С.&&\\
\textbf{Ососков~М.\,В.} см.~Каратеев~С.\,Л.&&\\
\hangindent=23pt\noindent\textbf{Павельева~Е.\,А., Крылов~А.\,С.} Поиск и анализ ключевых точек радужной 
оболочки\linebreak
\vspace*{-12pt}\\
\hspace*{23pt}глаза методом преобразования Эрмита$\dotfill$&1&79\\
\textbf{Печинкин~А.\,В.} см.~Френкель~С.\,Л.,&&\\
\hangindent=23pt\noindent\textbf{Протасов~В.\,И.} Составление субъективного портрета с использованием 
эволюционно-\linebreak
\vspace*{-12pt}\\
\hspace*{23pt}го морфинга и квалиметрия метода$\dotfill$&1&83\\
\hangindent=23pt\noindent\textbf{Рудаков~К.\,В., Торшин~И.\,Ю.} Вопросы разрешимости задачи распознавания 
вторичной\linebreak
\vspace*{-12pt}\\
\hspace*{23pt}структуры белка$\dotfill$&2&25\\
\textbf{Сатин~Я.\,А.} см.~Зейфман~А.\,И.&&\\
\hangindent=23pt\noindent\textbf{Сейфуль-Мулюков~Р.\,Б.} Нефть как носитель информации о своем 
происхождении,\linebreak
\vspace*{-12pt}\\
\hspace*{23pt}структуре и эволюции$\dotfill$&1&41\\
\end{tabular}
}

{\tabcolsep=3pt
\begin{tabular}{p{388pt}rr}
&\textbf{Выпуск} & \textbf{Стр.}\\[6pt]
\textbf{Семендяев~Н.\,Н.} см.~Синицын~И.\,Н.&&\\
\textbf{Серебряков~В.\,А.} см.~Захаров~А.\,А.&&\\
\textbf{Синицын~В.\,И.} см.~Синицын~И.\,Н.&&\\
\hangindent=23pt\noindent\textbf{Синицын~И.\,Н., Синицын~В.\,И., Корепанов~Э.\, Р., Белоусов~В.\,В., 
Семендяев~Н.\,Н.} Оперативное построение информационных моделей движения полюса 
Земли\linebreak
\vspace*{-12pt}\\
\hspace*{23pt}методами линейных и линеаризованных фильтров$\dotfill$&1&2\\
\textbf{Сипина~А.\,В.} см.~Бенинг~В.\,Е.&&\\
\hangindent=23pt\noindent\textbf{Соколов~И.\,А.} О работах заслуженного деятеля науки Российской Федерации 
И.\,Н.~Синицына в области информационных технологий и автоматизации (к 70-летию\linebreak
\vspace*{-12pt}\\
\hspace*{23pt}со дня рождения)$\dotfill$&3&84\\
\textbf{Соколов~И.\,А.} см.~Илюшин~Г.\,Я.&&\\
\hangindent=23pt\noindent\textbf{Соколов~И.\,А., Королев~В.\,Ю.} Предисловие$\dotfill$&2&2\\
\textbf{Солдатов~С.\,А.} см.~Колесников~А.\,В.&&\\
\hangindent=23pt\noindent\textbf{Степанов~С.\,Ю.} Использование координатного метода фрагментации 
коммутаторной\linebreak
\vspace*{-12pt}\\
\hspace*{23pt}нейронной сети для сокращения трафика$\dotfill$&2&57\\
\textbf{Тимонина~Е.\,Е.} см.~Грушо~А.\,А.&&\\
\textbf{Торшин~И.\,Ю.} см.~Рудаков~К.\,В.&&\\
\textbf{Ульянов~В.\,В.} см.~Кавагучи~Ю.&&\\
\textbf{Фазекаш~И.} см.~Чупрунов~А.\,Н.&&\\
\textbf{Френкель~С.\,Л.} см.~Баранов~С.\,И.&&\\
\hangindent=23pt\noindent\textbf{Френкель~С.\,Л., Печинкин~А.\,В.} Оценка времени самовосстановления в 
цифровых\linebreak
\vspace*{-12pt}\\
\hspace*{23pt}системах после сбоев, вызываемых переходными помехами$\dotfill$&3&2\\
\textbf{Фуджикоши~Я.} см.~Кавагучи~Ю.&&\\
\hangindent=23pt\noindent\textbf{Цискаридзе~А.\,К.} Математическая модель и метод восстановления позы человека 
по\linebreak
\vspace*{-12pt}\\
\hspace*{23pt}стереопаре силуэтных изображений$\dotfill$&4&27\\
\hangindent=23pt\noindent\textbf{Чупраков~К.\,Г.} К вопросу о размещении коллективных средств отображения в 
ситуа-\linebreak
\vspace*{-12pt}\\
\hspace*{23pt}ционном зале с заданными параметрами$\dotfill$&4&89\\
\textbf{Чупраков~К.\,Г.} см.~Зацаринный~А.\,А.&&\\
\hangindent=23pt\noindent\textbf{Чупрунов~А.\,Н., Фазекаш~И.} Законы повторного логарифма для числа 
безошибочных\linebreak
\vspace*{-12pt}\\
\hspace*{23pt}блоков при помехоустойчивом кодировании$\dotfill$&3&42\\
\textbf{Шевцова~И.\,Г.} см.~Григорьева~М.\,Е.&&\\
\hangindent=23pt\noindent\textbf{Шестаков~О.\,В.} Аппроксимация распределения оценки риска пороговой 
обработки вейвлет-коэффициентов нормальным распределением при использовании 
выбо-\linebreak
\vspace*{-12pt}\\
\hspace*{23pt}рочной дисперсии$\dotfill$&4&73\\
\textbf{Шестаков~О.\,В.} см.~Маркин~А.\,В.&&\\
\textbf{Шоргин~С.\,Я.} см.~Зейфман~А.\,И.&&\\
\textbf{Шоргин~С.\,Я.} см.~Кудрявцев~А.\,А.&&\\
\end{tabular}
}

%\thispagestyle{myheadings}
\def\leftfootline{\small{\textbf{\thepage}
\hfill ИНФОРМАТИКА И ЕЁ ПРИМЕНЕНИЯ\ \ \ том~4\ \ \ выпуск~4\ \ \ 2010}
}%
 \def\rightfootline{\small{ИНФОРМАТИКА И ЕЁ ПРИМЕНЕНИЯ\ \ \ том~4\ \ \ выпуск~4\ \ \ 2010
 \hfill \textbf{\thepage}}}
 \label{end\stat}
%
%Том 10 Выпуск 1-4 Год 2016

\def\stat{cont-e}
{%\hrule\par
%\vskip 7pt % 7pt
\raggedleft\Large \bf%\baselineskip=3.2ex
2\,0\,1\,6\ \ A\,U\,T\,H\,O\,R\ \ I\,N\,D\,E\,X \vskip 17pt
 \hrule
 \par
\vskip 21pt plus 6pt minus 3pt }

\label{st\stat}

\def\tit{\ }

\def\aut{\ }
\def\auf{\ }

\def\leftkol{\ } %2016 AUTHOR INDEX} % ENGLISH ABSTRACTS}

\def\rightkol{\ } %2016 AUTHOR INDEX} %ENGLISH ABSTRACTS}

\titele{\tit}{\aut}{\auf}{\leftkol}{\rightkol}

\def\leftfootline{\small{\textbf{\thepage}
\hfill INFORMATIKA I EE PRIMENENIYA~--- INFORMATICS AND APPLICATIONS\ \ \ 2016\
\ \ volume~10\ \ \ issue\ 4}
}%
 \def\rightfootline{\small{INFORMATIKA I EE PRIMENENIYA~--- INFORMATICS AND APPLICATIONS\ \ \ 2016\ \ \ volume~10\ \ \ issue\ 4
\hfill \textbf{\thepage}}}

\vspace*{-12pt}
\vspace*{-18pt}

{\tabcolsep=2.8pt
\begin{tabular}{p{382pt}cc}
&\textbf{Issue} & \textbf{Page}\\[6pt]
\Avtors{Agalarov~M.\,Ya.} see~Agalarov~Ya.\,M.&&\\
\Avtors{Agalarov~Ya.\,M., Agalarov~M.\,Ya., and
Shorgin~V.\,S.} About the optimal threshold of queue\linebreak
\\[-12pt]
\hspace*{23pt}length in a~particular problem of profit maximization
in the $M/G/1$ queuing system&2&70--79\\
\Avtors{Alexeyevsky~D.\,A.} BioNLP ontology extraction from 
a~restricted language corpus with\linebreak
\\[-12pt]
\hspace*{23pt}context-free grammars&1&119--128\\
\Avtors{Andreev~S.\,D.} see~Gaidamaka~Yu.\,V.&&\\
\Avtors{Andreev~S.\,D.} see~Ometov~A.\,Ya.&&\\
\Avtors{Arkhipov~O.\,P., Arkhipov~P.\,O., and Sidorkin~I.\,I.} The
option to create a~local coordinate\linebreak
\\[-12pt]
\hspace*{23pt}system for synchronization of selected images&3&91--97\\
\Avtors{Arkhipov~P.\,O.} see~Arkhipov~O.\,P.&&\\
\Avtors{Belousov~V.\,V.} see~Shnurkov~P.\,V.&&\\
\Avtors{Belousov~V.\,V.} see~Shnurkov~P.\,V.&&\\
\Avtors{Bening~V.\,E.} Calculation of~the~asymptotic deficiency
of~some statistical procedures based\linebreak
\\[-12pt]
\hspace*{23pt}on~samples with~random sizes&4&34--45\\
\Avtors{Borisov~A.\,V., Bosov~A.\,V., and Miller~G.\,B.} Modeling and
monitoring of VoIP connection&2&\hphantom{1}2--13\\
\Avtors{Bosov~A.\,V.} see~Borisov~A.\,V.&&\\
\Avtors{Briukhov~D.\,O.} see~Stupnikov~S.\,A.&&\\
\Avtors{Callaos~N.\,K.\ and Seyful-Mulyukov~R.\,B.} Complexity and
its information content&1&129--139\\
\Avtors{Chertok~A.\,V., Kadaner~A.\,I., Khazeeva~G.\,T., and
Sokolov~I.\,A.} Regime switching detection\linebreak
\\[-12pt]
\hspace*{23pt}for~the~Levy driven
Ornstein--Uhlenbeck process using CUSUM methods&4&46--56\\
\Avtors{Chichagov~V.\,V.} Asymptotic expansions of mean absolute
error of uniformly minimum variance unbiased and maximum likelihood
estimators on the one-parameter exponential\linebreak
\\[-12pt]
\hspace*{23pt}family model of lattice distributions&3&66--76\\
\Avtors{Danishevsky~V.\,I.} see~Kolesnikov A.\,V.&&\\
\Avtors{Fazliev~A.\,Z.} see~Kalinichenko~L.\,A.&&\\
\Avtors{Fedoseev~A.\,A.} What is behind the concept of ``knowledge in
small packages''&3&105--110\\
\Avtors{Gaidamaka~Yu.\,V., Andreev~S.\,D., Sopin~E.\,S.,
Samouylov~K.\,E., and Shorgin~S.\,Ya.} Interference analysis
of~the~device-to-device communications model with~regard to~a~signal\linebreak
\\[-12pt]
\hspace*{23pt}propagation environment&4&\hphantom{1}2--10\\
\Avtors{Gasilov~A.\,V.} see~Yakovlev~O.\,A.&&\\
\Avtors{Goncharov~A.\,V.\ and Strijov~V.\,V.} Metric time series
classification using weighted dynamic\linebreak
\\[-12pt]
\hspace*{23pt}warping relative to centroids of classes&2&36--47\\
\Avtors{Gordov~E.\,P.} see~Kalinichenko~L.\,A.&&\\
\Avtors{Gorshenin~A.\,K.} Concept of online service for stochastic
modeling of real processes&1&72--81\\
\Avtors{Gorshenin~A.\,K.} see~Shnurkov~P.\,V.&&\\
\Avtors{Gorshenin~A.\,K.} see~Shnurkov~P.\,V.&&\\
\Avtors{Grusho~A.\,A., Grusho~N.\,A., Zabezhailo~M.\,I., and
Timonina~E.\,E.} Integration of statistical and\linebreak
\\[-12pt]
\hspace*{23pt}deterministic methods for
analysis of information security&3&2--8\\
\Avtors{Grusho~A.\,A., Zabezhailo~M.\,I., and Zatsarinny~A.\,A.} On
the advanced procedure to reduce\linebreak
\\[-12pt]
\hspace*{23pt}calculation of Galois closures&4&\hphantom{1}96--104\\
\Avtors{Grusho~N.\,A.} see~Grusho~A.\,A.&&\\
\Avtors{Havanskov~V.\,A.} see~Minin~V.\,A.&&\\
\Avtors{Inkova~O.\,Yu.} see~Zatsman~I.\,M.&&\\
\Avtors{Isachenko~R.\,V.\ and Strijov~V.\,V.} Metric learning in
multiclass time series classification\linebreak
\\[-12pt]
\hspace*{23pt}problem&2&48--57\\
\end{tabular}
}
\pagebreak

\def\leftfootline{\small{\textbf{\thepage}
\hfill INFORMATIKA I EE PRIMENENIYA~--- INFORMATICS AND APPLICATIONS\ \ \ 2016\
\ \ volume~10\ \ \ issue\ 4}
}%
 \def\rightfootline{\small{INFORMATIKA I EE PRIMENENIYA~---
INFORMATICS AND APPLICATIONS\ \ \ 2016\ \ \ volume~10\ \ \ issue\ 4
\hfill \textbf{\thepage}}}

\def\leftkol{2016 AUTHOR INDEX} % ENGLISH ABSTRACTS}

\def\rightkol{2016 AUTHOR INDEX} %ENGLISH ABSTRACTS}


{\tabcolsep=2.83pt
\begin{tabular}{p{382pt}cc}
&\textbf{Issue} & \textbf{Page}\\[6pt]
\Avtors{Kadaner~A.\,I.} see~Chertok~A.\,V.&&\\[.255pt]
\Avtors{Kalinichenko~L.\,A., Volnova~A.\,A., Gordov~E.\,P.,
Kiselyova~N.\,N., Kovaleva~D.\,A., Malkov~O.\,Yu., Okladnikov~I.\,G.,
Podkolodnyy~N.\,L., Pozanenko~A.\,S., Ponomareva~N.\,V.,
Stupnikov~S.\,A.,} \textbf{and Fazliev~A.\,Z.} Data access challenges for data
intensive\linebreak
\\[-12pt]
\hspace*{23pt}research in Russia&1& 2--22\\[.255pt]
\Avtors{Karasikov~M.\,E.\ and Strijov~V.\,V.} Feature-based
time-series classification&4&121--131\\[.255pt]
\Avtors{Khazeeva~G.\,T.} see~Chertok~A.\,V.&&\\[.255pt]
\Avtors{Khokhlov~Yu.\,S.} Multivariate fractional Levy motion and its
applications&2&\hphantom{1}98--106\\[.255pt]
\Avtors{Kirikov~I.\,A., Kolesnikov~A.\,V., Listopad~S.\,V., and
Rumovskaya~S.\,B.} Fine-grained hybrid\linebreak
\\[-12pt]
\hspace*{23pt}intelligent systems. Part 2:
Bidirectional hybridization&1&\hphantom{1}96--105\\[.255pt]
\Avtors{Kirikov~I.\,A., Kolesnikov~A.\,V., Listopad~S.\,V., and
Rumovskaya~S.\,B.} ``Virtual council''~---\linebreak
\\[-12pt]
\hspace*{23pt}source environment
supporting complex diagnostic decision making&3&81--90\\[.255pt]
\Avtors{Kiselyova~N.\,N.} see~Kalinichenko~L.\,A.&&\\[.255pt]
\Avtors{Kolesnikov A.\,V., Listopad~S.\,V., Rumovskaya~S.\,B., and
Danishevsky~V.\,I.} Informal axiomatic\linebreak
\\[-12pt]
\hspace*{23pt}theory of~the~role visual models&4&114--120\\[.255pt]
\Avtors{Kolesnikov~A.\,V.} see~Kirikov~I.\,A.&&\\[.255pt]
\Avtors{Kolesnikov~A.\,V.} see~Kirikov~I.\,A.&&\\[.255pt]
\Avtors{Kolin~K.\,K.} Humanitarian aspects of information
security&3&111--121\\[.255pt]
\Avtors{Konovalov~M.\,G.\ and Razumchik~R.\,V.} Dispatching
to~two parallel nonobservable queues using\linebreak
\\[-12pt]
\hspace*{23pt}only static
information&4&57--67\\[.255pt]
\Avtors{Korchagin~A.\,Yu.} see~Korolev~V.\,Yu.&&\\[.255pt]
\Avtors{Korchagin~A.\,Yu.} see~Korolev~V.\,Yu.&&\\[.255pt]
\Avtors{Korepanov~E.\,R.} see~Sinitsyn~I.\,N.&&\\[.255pt]
\Avtors{Korepanov~E.\,R.} see~Sinitsyn~I.\,N.&&\\[.255pt]
\Avtors{Korolev~V.\,Yu., Korchagin~A.\,Yu., and Zeifman~A.\,I.} The
Poisson theorem for Bernoulli trials\linebreak
\\[-12pt]
\hspace*{23pt}with~a~random probability
of~success and~a~discrete analog of~the~Weibull distribution&4&11--20\\[.255pt]
\Avtors{Korolev~V.\,Yu., Zeifman~A.\,I., and Korchagin~A.\,Yu.}
Asymmetric Linnik distributions as~limit\linebreak
\\[-12pt]
\hspace*{23pt}laws for~random sums
of~independent random variables with~finite variances&4&21--33\\[.255pt]
\Avtors{Koucheryavy~E.\,A.} see~Ometov~A.\,Ya.&&\\[.255pt]
\Avtors{Kovaleva~D.\,A.} see~Kalinichenko~L.\,A.&&\\[.255pt]
\Avtors{Kovalyov~S.\,P.} Metaprogramming to increase
manufacturability of large-scale software-\linebreak
\\[-12pt]
\hspace*{23pt}intensive systems&1&56--66\\[.255pt]
\Avtors{Krivenko~M.\,P.} Significance tests of feature selection for
classification&3&32--40\\[.255pt]
\Avtors{Kruzhkov~M.\,G.} see~Zalizniak~Anna~A.&&\\[.255pt]
\Avtors{Kruzhkov~M.\,G.} see~Zatsman~I.\,M.&&\\[.255pt]
\Avtors{Kudryavtsev~A.\,A.} Bayesian queueing and reliability models:
\textit{A~priori} distributions with\linebreak
\\[-12pt]
\hspace*{23pt}compact support&1&67--71\\[.255pt]
\Avtors{Kudryavtsev~A.\,A.} Characteristics dependent on the balance
coefficient in Bayesian models\linebreak
\\[-12pt]
\hspace*{23pt}with compact support of \textit{a priori}
distributions&3&77--80\\[.255pt]
\Avtors{Kudryavtsev~A.\,A.\ and Palionnaia~S.\,I.} Bayesian recurrent
model of reliability growth:\linebreak
\\[-12pt]
\hspace*{23pt}Parabolic distribution of parameters&2&80--83\\[.255pt]
\Avtors{Kudryavtsev~A.\,A.\ and Titova~A.\,I.} Bayesian queuing
and~reliability models: Degenerate-\linebreak
\\[-12pt]
\hspace*{23pt}Weibull case&4&68--71\\[.255pt]
\Avtors{Leontyev~N.\,D.\ and Ushakov~V.\,G.} Analysis of a queueing
system with autoregressive arrivals\linebreak
\\[-12pt]
\hspace*{23pt}and nonpreemptive priority&3&15--22\\[.255pt]
\Avtors{Listopad~S.\,V.} see~Kirikov~I.\,A.&&\\[.255pt]
\Avtors{Listopad~S.\,V.} see~Kirikov~I.\,A.&&\\[.255pt]
\Avtors{Listopad~S.\,V.} see~Kolesnikov A.\,V.&&\\[.255pt]
\Avtors{Malkov~O.\,Yu.} see~Kalinichenko~L.\,A.&&\\[.255pt]
\Avtors{Markov~A.\,S., Monakhov~M.\,M., and
Ulyanov~V.\,V.} Generalized Cornish--Fisher expansions\linebreak
\\[-12pt]
\hspace*{23pt}for distributions of statistics based on samples
of random size&2&84--91\\[.255pt]
\Avtors{Melnikov~A.\,K.\ and Ronzhin~A.\,F.} Generalized statistical
method of~text analysis based\linebreak
\\[-12pt]
\hspace*{23pt}on~calculation of~probability distributions
of~statistical values&4&89--95\\
\end{tabular}
}
\pagebreak

\def\leftfootline{\small{\textbf{\thepage}
\hfill INFORMATIKA I EE PRIMENENIYA~--- INFORMATICS AND APPLICATIONS\ \ \ 2016\
\ \ volume~10\ \ \ issue\ 4}
}%
 \def\rightfootline{\small{INFORMATIKA I EE PRIMENENIYA~---
INFORMATICS AND APPLICATIONS\ \ \ 2016\ \ \ volume~10\ \ \ issue\ 4
\hfill \textbf{\thepage}}}

\def\leftkol{2016 AUTHOR INDEX} % ENGLISH ABSTRACTS}

\def\rightkol{2016 AUTHOR INDEX} %ENGLISH ABSTRACTS}


{\tabcolsep=3pt
\begin{tabular}{p{381pt}cc}
&\textbf{Issue} & \textbf{Page}\\[6pt]
\Avtors{Meykhanadzhyan~L.\,A.} Stationary characteristics of the finite
capacity queueing system with\linebreak
\\[-12pt]
\hspace*{23pt}inverse service order and generalized
probabilistic priority&2&123--131\\[.23pt]
\Avtors{Miller~G.\,B.} see~Borisov~A.\,V.&&\\[.23pt]
\Avtors{Minin~V.\,A., Zatsman~I.\,M., Havanskov~V.\,A., and
Shubnikov~S.\,K.} Intensity of citation of scientific publications in
inventions on information and computer technologies patented\linebreak
\\[-12pt]
\hspace*{23pt}in Russia by domestic and foreign applicants&2&107--122\\[.23pt]
\Avtors{Monakhov~M.\,M.} see~Markov~A.\,S.&&\\[.23pt]
\Avtors{Naumov~V.\,A.\ and Samouylov~K.\,E.} On relationship
between queuing systems with resources\linebreak
\\[-12pt]
\hspace*{23pt}and Erlang networks&3&\hphantom{1}9--14\\[.23pt]
\Avtors{Okladnikov~I.\,G.} see~Kalinichenko~L.\,A.&&\\[.23pt]
\Avtors{Ometov~A.\,Ya., Andreev~S.\,D., Turlikov~A.\,M., and
Koucheryavy~E.\,A.} Performance analysis of\linebreak
\\[-12pt]
\hspace*{23pt}a wireless data
aggregation system with contention for contemporary sensor
networks&3&23--31\\[.23pt]
\Avtors{Palionnaia~S.\,I.} see~Kudryavtsev~A.\,A.&&\\[.23pt]
\Avtors{Podkolodnyy~N.\,L.} see~Kalinichenko~L.\,A.&&\\[.23pt]
\Avtors{Ponomareva~N.\,V.} see~Kalinichenko~L.\,A.&&\\[.23pt]
\Avtors{Popkova~N.\,A.} see~Zatsman~I.\,M.&&\\[.23pt]
\Avtors{Pozanenko~A.\,S.} see~Kalinichenko~L.\,A.&&\\[.23pt]
\Avtors{Razumchik~R.\,V.} see~Konovalov~M.\,G.&&\\[.23pt]
\Avtors{Ronzhin~A.\,F.} see~Melnikov~A.\,K.&&\\[.23pt]
\Avtors{Rumovskaya~S.\,B.} see~Kirikov~I.\,A.&&\\[.23pt]
\Avtors{Rumovskaya~S.\,B.} see~Kirikov~I.\,A.&&\\[.23pt]
\Avtors{Rumovskaya~S.\,B.} see~Kolesnikov A.\,V.&&\\[.23pt]
\Avtors{Samouylov~K.\,E.} see~Gaidamaka~Yu.\,V.&&\\[.23pt]
\Avtors{Samouylov~K.\,E.} see~Naumov~V.\,A.&&\\[.23pt]
\Avtors{Serebryanskii~S.\,M.} see~Tyrsin~A.\,N.&&\\[.23pt]
\Avtors{Seyful-Mulyukov~R.\,B.} see~Callaos~N.\,K.&&\\[.23pt]
\Avtors{Shestakov~O.\,V.} Statistical properties of the denoising method
based on the stabilized hard\linebreak
\\[-12pt]
\hspace*{23pt}thresholding&2&65--69\\[.23pt]
\Avtors{Shestakov~O.\,V.} The strong law of large numbers for the risk
estimate in the problem of\linebreak
\\[-12pt]
\hspace*{23pt}tomographic image reconstruction from
projections with a correlated noise&3&41--45\\[.23pt]
\Avtors{Shestakov~O.\,V.} see~Zakharova~T.\,V.&&\\[.23pt]
\Avtors{Shnurkov~P.\,V., Gorshenin~A.\,K., and Belousov~V.\,V.}
Analytical solution of~the~optimal control\linebreak
\\[-12pt]
\hspace*{23pt}task of~a~semi-Markov
process with~finite set of~states&4&72--88\\[.23pt]
\Avtors{Shnurkov~P.\,V., Zasypko~V.\,V., Belousov~V.\,V., and
Gorshenin~A.\,K.} Development of the algorithm of numerical solution
of the optimal investment control problem\linebreak
\\[-12pt]
\hspace*{23pt}in the closed dynamical model of three-sector economy&1&82--95\\[.23pt]
\Avtors{Shorgin~S.\,Ya.} see~Gaidamaka~Yu.\,V.&&\\[.23pt]
\Avtors{Shorgin~V.\,S.} see~Agalarov~Ya.\,M.&&\\[.23pt]
\Avtors{Shubnikov~S.\,K.} see~Minin~V.\,A.&&\\[.23pt]
\Avtors{Sidorkin~I.\,I.} see~Arkhipov~O.\,P.&&\\[.23pt]
\Avtors{Sinitsyn~I.\,N.} Analytical modeling of processes in stochastic
systems with complex fractional\linebreak
\\[-12pt]
\hspace*{23pt}order Bessel nonlinearities&3&55--65\\[.23pt]
\Avtors{Sinitsyn~I.\,N.} Orthogonal supoptimal filters for nonlinear
stochastic systems on manifolds&1&34--44\\[.23pt]
\Avtors{Sinitsyn~I.\,N.\ and Korepanov~E.\,R.} Normal Pugachev
conditionally-optimal filters and extra-\linebreak
\\[-12pt]
\hspace*{23pt}polators for state linear stochastic systems&2&14--23\\[.23pt]
\Avtors{Sinitsyn~I.\,N.\ and Sinitsyn~V.\,I.} Analytical modeling of
distributions in stochastic systems on\linebreak
\\[-12pt]
\hspace*{23pt}manifolds based on ellipsoidal approximation&1&45--55\\[.23pt]
\Avtors{Sinitsyn~I.\,N., Sinitsyn~V.\,I., and
Korepanov~E.\,R.} Ellipsoidal suboptimal filters for nonlinear\linebreak
\\[-12pt]
\hspace*{23pt}stochastic systems on manifolds&2&24--35\\[.23pt]
\Avtors{Sinitsyn~V.\,I.} see~Sinitsyn~I.\,N.&&\\[.23pt]
\Avtors{Sinitsyn~V.\,I.} see~Sinitsyn~I.\,N.&&\\[.23pt]
\Avtors{Skvortsov~N.\,A.} see~Stupnikov~S.\,A.&&\\[.23pt]
\Avtors{Sokolov~I.\,A.} see~Chertok~A.\,V.&&\\
\end{tabular}
}
\pagebreak

\def\leftfootline{\small{\textbf{\thepage}
\hfill INFORMATIKA I EE PRIMENENIYA~--- INFORMATICS AND APPLICATIONS\ \ \ 2016\
\ \ volume~10\ \ \ issue\ 4}
}%
 \def\rightfootline{\small{INFORMATIKA I EE PRIMENENIYA~---
INFORMATICS AND APPLICATIONS\ \ \ 2016\ \ \ volume~10\ \ \ issue\ 4
\hfill \textbf{\thepage}}}

\def\leftkol{2016 AUTHOR INDEX} % ENGLISH ABSTRACTS}

\def\rightkol{2016 AUTHOR INDEX} %ENGLISH ABSTRACTS}


{\tabcolsep=3pt
\begin{tabular}{p{382pt}cc}
&\textbf{Issue} & \textbf{Page}\\[6pt]
\Avtors{Sopin~E.\,S.} see~Gaidamaka~Yu.\,V.&&\\
\Avtors{Strijov~V.\,V.} see~Goncharov~A.\,V.&&\\
\Avtors{Strijov~V.\,V.} see~Isachenko~R.\,V.&&\\
\Avtors{Strijov~V.\,V.} see~Karasikov~M.\,E.&&\\
\Avtors{Stupnikov~S.\,A., Briukhov~D.\,O., and Skvortsov~N.\,A.}
Co-lending systemic risk analysis over\linebreak
\\[-12pt]
\hspace*{23pt}heterogeneous data collections&1&23--33\\
\Avtors{Stupnikov~S.\,A.} see~Kalinichenko~L.\,A.&&\\
\Avtors{Suchkov~A.\,P.} see~Zatsarinny~A.\,A.&&\\
\Avtors{Timonina~E.\,E.} see~Grusho~A.\,A.&&\\
\Avtors{Titova~A.\,I.} see~Kudryavtsev~A.\,A.&&\\
\Avtors{Turlikov~A.\,M.} see~Ometov~A.\,Ya.&&\\
\Avtors{Tyrsin~A.\,N.\ and Serebryanskii~S.\,M.} Recognition of
dependences on the basis of inverse\linebreak
\\[-12pt]
\hspace*{23pt}mapping&2&58--64\\
\Avtors{Ulyanov~V.\,V.} see~Markov~A.\,S.&&\\
\Avtors{Ushakov~V.\,G.} Queueing system with working vacations and
hyperexponential input stream&2&92--97\\
\Avtors{Ushakov~V.\,G.} see~Leontyev~N.\,D.&&\\
\Avtors{Volnova~A.\,A.} see~Kalinichenko~L.\,A.&&\\
\Avtors{Yakovlev~O.\,A.\ and Gasilov~A.\,V.} Speeded-up stereo
matching using geodesic support weights&3&\hphantom{1}98--104\\
\Avtors{Zabezhailo~M.\,I.} see~Grusho~A.\,A.&&\\
\Avtors{Zabezhailo~M.\,I.} see~Grusho~A.\,A.&&\\
\Avtors{Zakharova~T.\,V.\ and Shestakov~O.\,V.} Precision analysis of
wavelet processing of aerodynamic\linebreak
\\[-12pt]
\hspace*{23pt}flow patterns&3&46--54\\
\Avtors{Zalizniak~Anna~A.\ and Kruzhkov~M.\,G.} Database
of~Russian impersonal verbal constructions&4&132--141\\
\Avtors{Zasypko~V.\,V.} see~Shnurkov~P.\,V.&&\\
\Avtors{Zatsarinny~A.\,A.\ and Suchkov~A.\,P.} Systems engineering
approaches to~the~establishment of\linebreak
\\[-12pt]
\hspace*{23pt}a~system for~decision support based
on~situational analysis&4&105--113\\
\Avtors{Zatsarinny~A.\,A.} see~Grusho~A.\,A.&&\\
\Avtors{Zatsman~I.\,M., Inkova~O.\,Yu., Kruzhkov~M.\,G., and
Popkova~N.\,A.} Representation of cross-\linebreak
\\[-12pt]
\hspace*{23pt}lingual knowledge about
connectors in supracorpora databases&1&106--118\\
\Avtors{Zatsman~I.\,M.} see~Minin~V.\,A.&&\\
\Avtors{Zeifman~A.\,I.} see~Korolev~V.\,Yu.&&\\
\Avtors{Zeifman~A.\,I.} see~Korolev~V.\,Yu.&&\\
\end{tabular}
}

%\thispagestyle{myheadings}
\def\leftfootline{\small{\textbf{\thepage}
\hfill INFORMATIKA I EE PRIMENENIYA~--- INFORMATICS AND APPLICATIONS\ \ \ 2016\
\ \ volume~10\ \ \ issue\ 4}
}%
 \def\rightfootline{\small{INFORMATIKA I EE PRIMENENIYA~---
INFORMATICS AND APPLICATIONS\ \ \ 2016\ \ \ volume~10\ \ \ issue\ 4
\hfill \textbf{\thepage}}}

 \label{end\stat}

\newpage

%\def\stat{rekl}
%\label{preobr}

%\def\tit{АКАДЕМИК ПУГАЧЁВ  ВЛАДИМИР СЕМЁНОВИЧ\\
%25.03.1911--25.03.1998}


%   \vspace*{-48pt}
%   \begin{center}\LARGE
%Академик Пугачёв  Владимир Семёнович\\ (25.03.1911--25.03.1998)
%   \end{center}
   
   %\vspace*{2.5mm}
   
   \begin{center}

{\prgsh\LARGE
ОБЪЯВЛЕНИЯ О КОНФЕРЕНЦИЯХ}

\end{center}
%\hrule

\vspace*{6pt}

   
   \vspace*{10mm}
   
   \thispagestyle{empty}

\noindent
\begin{tabular}{cc}
%\begin{center}
\multicolumn{1}{c}{\raisebox{-40pt}[0pt][0pt]{\mbox{%
\epsfxsize=33mm
\epsfbox{vspu.eps}
}}}
%\end{center}
&
\tabcolsep=0pt\begin{tabular}{c}
{\prg{\Large\textbf{XII Всероссийское совещание}}}\\[6pt]
{\prg{\Large\textbf{по проблемам управления}}}\\[12pt]
{\prg{\large 16--19 июня 2014~г.}}\\[6pt] 
{\prg{\large Институт проблем управления имени В.\,А.~Трапезникова РАН}}\\[6pt]
{\prg{\large Москва, Россия}}
\end{tabular}
\end{tabular}

\vspace*{60pt}

     
 { %\large    
 XII Всероссийское совещание по проблемам управления (ВСПУ XII), посвященное 75-летию 
Института проблем управления (ИПУ) имени В.\,А.~Трапезникова РАН, проводится 16--19~июня 
2014~г.\ 
в ИПУ РАН (г.~Москва, Россия). ВСПУ XII организуется ИПУ РАН при поддержке РФФИ, Отделения 
энергетики, машиностроения, механики и процессов управления Российской академии наук, 
Российского 
национального комитета по автоматическому управлению, Академии навигации и управ\-ле\-ния 
движением, 
Научного совета РАН по комплексным проблемам управления и автоматизации, Совета по 
мехатронике и робототехнике РАН. Официальный язык Совещания~--- русский.

\vspace*{24pt}
     
     \textbf{Направления работы}
     \begin{enumerate}[1.]
\item Теория систем управления
\item Управление подвижными объектами и навигация
\item Интеллектуальные системы управления
\item Управление в промышленности, транспортом и логистикой
\item Управление системами междисциплинарной природы
\item Средства измерения, вычислений и контроля в управлении
\item Системный анализ и принятие решений в задачах управления
\item Информационные технологии в управлении
\item Проблемы образования в области управления: современное содержание и технологии обучения
\end{enumerate}

\vspace*{24pt}

     Подробная информация о Совещании находится на сайте {\sf http://vspu2014.ipu.ru}. Срок 
окончательной подачи докладов через систему подачи докладов на сайте~--- \textbf{30~ноября} 
2013~г.
}

%\include{rekl-1}

%\end{document}

%   \vspace*{-48pt}

\begin{center}
\vspace*{6pt}
\mbox{%
\epsfxsize=53.502mm
\epsfbox{foto-1.eps}
}
\end{center}

\vspace*{6pt} %Академик


   \begin{center}
\fbox{\Large\textbf{Профессор Игорь Алексеевич Ушаков}}\\[12pt]
\textbf{\large 22.01.1935--27.02.2015}
   \end{center}


   %\vspace*{2.5mm}

   \vspace*{5mm}

   \thispagestyle{empty}

%\

%\vspace*{-12pt}


Редакционный совет и редакционная коллегия журнала <<Информатика и~её применения>> с~глубоким прискорбием извещают, что 27~февраля 2015~г.\ после тяжелой
и~продолжительной болезни скончался Игорь Алексеевич Ушаков~--- доктор технических наук, профессор, член редколлегии журнала <<Информатика и ее применения>>.

Игорь Алексеевич Ушаков окончил Московский авиационный институт, в~1963~г.\ защитил кандидатскую, а~в~1968~г.~--- докторскую диссертацию. С~1958 по 1989~гг.\ работал в~ряде научно-исследовательских организаций СССР, в~том числе руководил отделами в~НИИ АА и~ВЦ АН СССР; с 1969 по 1989 гг. преподавал в~МФТИ (был профессором, а~затем заведующим кафедрой) и~в~МЭИ. С~1989~г.~---- в~США: являлся профессором университета Дж.\ Вашингтона, университета Дж.\ Мэйсона и~Калифорнийского университета, сотрудником компаний MCI, Qualcomm и Hughes.

И.\,А.~Ушаков с момента основания журнала <<Надежность и~контроль качества>> был заместителем ответственного редактора, а~затем на протяжении многих лет членом редколлегии. В~2006~г.\ основал электронный международный журнал ``Reliability: Theory \& Application'', главным редактором которого оставался до конца жизни.

Учебниками и справочниками по теории надежности, написанными И.\,А.~Ушаковым, пользовались и~пользуются несколько поколений ученых и~специалистов в~разных странах мира.

Игорь Алексеевич всегда уделял огромное внимание работе с~молодежью; более~50 его учеников защитили докторские и~кандидатские диссертации.

И.\,А.~Ушаков вел активную научно-про\-све\-ти\-тель\-скую деятельность. В~частности, он был одним из организаторов и~руководителей Московского кабинета качества и~надежности при Политехническом музее (целью этого Кабинета было оказание консультаций работникам промышленных предприятий и~чтение курсов лекций для инженеров, занимающихся проблемой надежности). Находясь в~США, И.\,А.~Ушаков создал международный ин\-тер\-нет-фо\-рум им.\ Б.\,В.~Гнеденко, объединивший около~400~видных специалистов по приложениям теории вероятностей и~математической статистики, преимущественно в~об\-ласти теории надежности и~анализа риска, из десятков стран мира; коллективным членов этого Форума является и~наш журнал. Цели Форума~--- содействие контактам между специалистами из разных стран, организация обмена профессиональными 
новостями и~информацией (новые публикации, предстоящие события и~др.). Также необходимо отметить большое число на\-уч\-но-по\-пу\-ляр\-ных работ, опубликованных И.\,А.~Ушаковым.

И.\,А.~Ушаков обладал большим личным обаянием, имел широкий круг интересов. Все знавшие И.\,А.~Ушакова всегда будут помнить его как замечательного ученого и~прекрасного человека.

\bigskip

Редакционный совет и редакционная коллегия журнала <<Информатика и~её применения>> 
выражают глубокие соболезнования родным и близким покойного, всем, кто его знал и~работал с~ним.



%\end{document}

%\include{IPPM-25}

\def\stat{cont-rus}
{%\hrule\par
%\vskip 7pt % 7pt
\vspace*{-24pt}
\raggedleft\Large \bf%\baselineskip=3.2ex
Правила подготовки рукописей  для публикации в журнале
<<Информатика~и~её~применения>> \vskip 8pt
    \hrule
    \par
\vskip 14pt plus 6pt minus 3pt }

\label{st\stat}

\def\tit{\ }

\def\aut{\ }
\def\auf{\ }

\def\leftkol{\ }
% Правила подготовки рукописей  для публикации в журнале
%<<Информатика и её применения>>

\def\rightkol{\ }
%Правила подготовки рукописей  для публикации в журнале
%<<Информатика и её применения>>}


\titele{\tit}{\aut}{\auf}{\leftkol}{\rightkol}


\vspace*{-60pt}
{ %\small

Журнал <<Информатика и её применения>>
публикует теоретические, обзорные и дискуссионные статьи,
посвященные научным исследованиям и разработкам в области
информатики и ее приложений.

Журнал издается на русском языке. По специальному решению
редколлегии отдельные статьи могут печататься на английском языке.

Тематика журнала охватывает следующие направления:
\begin{itemize}
\item теоретические основы информатики;\\[-15pt]
      \item
математические методы исследования сложных систем и процессов;\\[-15pt]
           \item
информационные системы и сети;\\[-15pt]
                \item
информационные технологии;\\[-15pt]
                     \item
архитектура и программное обеспечение вычислительных комплексов и сетей.\\[-15pt]
\end{itemize}


\noindent
\begin{enumerate}[1.]
\item В журнале печатаются статьи, содержащие результаты, ранее не опубликованные и
не предназначенные к одновременной публикации в других изданиях.

%Публикация не должна нарушать закон об авторских правах.
Публикация предоставленной автором(ами) рукописи не должна нарушать 
положений глав~69, 70 раздела~VII части~IV Гражданского кодекса, 
которые определяют права на результаты интеллектуальной деятельности 
и~средства индивидуализации, в~том числе авторские права, в~РФ.

Ответственность за нарушение авторских прав, в~случае предъявления претензий к~редакции журнала,  
несут авторы статей.



Направляя рукопись в редакцию, авторы сохраняют свои права на данную
рукопись и при этом передают учредителям и редколлегии журнала неисключительные права на
издание статьи на русском языке 
(или на языке статьи, если он отличен от рус\-ско\-го) и~на перевод ее на английский
язык, а~также на
ее распространение в России и за рубежом. 
Каждый автор должен представить в~редакцию подписанный 
с~его стороны <<Лицензионный договор о~передаче неисключительных прав 
на использование произведения>>, текст которого размещен по адресу 
{\sf http://www.ipiran.ru/publications/licence.doc}. 
Этот договор может быть пред\-став\-лен в~бумажном (в~2-х экз.)\ 
или в~электронном виде (отсканированная копия заполненного и~подписанного документа).




Редколлегия вправе запросить у авторов экспертное заключение о возможности
пуб\-ли\-ка\-ции пред\-став\-лен\-ной статьи в открытой печати.\\[-13.5pt]

\item К статье прилагаются данные автора (авторов) (см.\ п.~8). При наличии нескольких
авторов указывается фамилия автора, ответственного за переписку с редакцией.\\[-13.5pt]

\item Редакция журнала осуществляет экспертизу присланных статей в соответствии с
принятой в журнале процедурой рецензирования.

Возвращение рукописи на доработку не означает ее принятия к печати.

Доработанный вариант с ответом на замечания рецензента необходимо прислать в
редакцию.\\[-13.5pt]

\item Решение редколлегии о публикации статьи или ее отклонении сообщается авторам.

Редколлегия может также направить авторам текст рецензии на их статью. Дискуссия по
поводу отклоненных статей не ведется.\\[-13.5pt]

%\pagebreak

\item Редактура статей высылается авторам для просмотра. Замечания к редактуре должны
быть присланы авторами в кратчайшие сроки.\\[-13.5pt]

\item Рукопись предоставляется в электронном виде в форматах MS WORD (.doc или
.docx) или \LaTeX\  (.tex), дополнительно~--- в формате .pdf, на дискете, лазерном диске
или электронной почтой. Предоставление бумажной рукописи необязательно.\\[-13.5pt]

\item При подготовке рукописи в MS Word рекомендуется использовать следующие
настройки.

Параметры страницы:
формат~--- А4; ориентация~--- книжная; поля (см): внутри~--- 2,5, снаружи~--- 1,5,
сверху~--- 2, снизу~--- 2, от края до нижнего колонтитула~--- 1,3.

Основной текст: стиль~--- <<Обычный>>, шрифт~--- Times New Roman, размер~---
14~пунк\-тов, абзацный отступ~--- 0,5~см, 1,5~интервала, выравнивание~--- по ширине.

\pagebreak

\def\leftkol{Правила подготовки рукописей  для публикации в журнале
<<Информатика и её применения>>}

\def\rightkol{Правила подготовки рукописей  для публикации в журнале
<<Информатика и её применения>>}



Рекомендуемый объем рукописи~--- не свыше 10~страниц указанного формата.
При превышении указанного объема редколлегия вправе потребовать от 
автора сокращения объема рукописи.


Сокращения слов, помимо стандартных, не допускаются. Допускается минимальное
количество аббревиатур.


Все страницы рукописи нумеруются.

Шаблоны оформления представлены в интернете:

\noindent
 {\sf
http://www.ipiran.ru/journal/template\_iiep\_ssi\_2024.zip}\\[-14pt]

\item Статья должна содержать следующую информацию на {\bfseries\textit{русском и
английском языках}}:\\[-16pt]

\begin{itemize}
\item название статьи;\\[-15pt]
\item Ф.И.О.\ авторов, на английском можно только имя и фамилию;\\[-15pt]
\item место работы, с указанием почтового адреса организации и электронного адреса каждого
автора;\\[-15pt]
\item сведения об авторах, в соответствии с форматом, образцы которого
представлены на страницах:



\def\leftfootline{\small{\textbf{\thepage}
\hfill ИНФОРМАТИКА И ЕЁ ПРИМЕНЕНИЯ\ \ \ том\ 18\ \ \ выпуск\ 3\ \ \ 2024}
}%
 \def\rightfootline{\small{ИНФОРМАТИКА И ЕЁ ПРИМЕНЕНИЯ\ \ \ том\ 18\ \ \ выпуск\ 3\ \ \ 2024
\hfill \textbf{\thepage}}}



{\sf http://www.ipiran.ru/journal/issues/2013\_07\_01/authors.asp} и

{\sf http://www.ipiran.ru/journal/issues/2013\_07\_01\_eng/authors.asp};
\item аннотация (не менее 100~слов на каждом из языков). Аннотация~--- это краткое
резюме работы, которое может публиковаться отдельно. Она является основным
источником информации в~ин\-фор\-ма\-ци\-он\-ных системах и базах данных. Английская
аннотация должна быть оригинальной, может не быть дословным переводом русского
текста и должна быть написана хорошим английским языком. В~аннотации не должно
быть ссылок на литературу и, по возможности, формул;\\[-15pt]
\item ключевые слова~--- желательно из принятых в мировой
на\-уч\-но-тех\-ни\-че\-ской литературе тематических тезаурусов. Предложения не
могут быть ключевыми словами;\\[-15pt]
\item источники финансирования работы (ссылки на гранты, проекты,
поддерживающие организации и~т.\,п.).
\end{itemize}



%\pagebreak

\item  Требования к спискам литературы.\\[-14pt]

Ссылки на литературу в тексте статьи нумеруются (в квадратных скобках) и
располагаются в каждом из списков литературы в порядке  первых упоминаний. Если источник имеет DOI и/или EDN,
то их необходимо указывать.

Списки литературы представляются в двух вариантах:\\[-14pt]


\noindent
\begin{enumerate}[(1)]
\item \textbf{Список литературы к русскоязычной части}. Русские и английские
работы~---  на языке и в алфавите оригинала;\\[-14.5pt]
\item  \textbf{References}. Русские работы и работы на других языках~--- в латинской
транслитерации с переводом на английский язык; английские работы и работы на других
языках~--- на языке оригинала.
\end{enumerate}

Необходимо для составления списка ``References'' пользоваться размещенной на сайте
{\sf http://www. translit.net/ru/bgn/} бесплатной программой транслитерации русского
 текста в~латиницу. %, при этом в~за\-клад\-ке <<варианты\ldots>> следует выбратьопцию BGN.

Список литературы ``References'' приводится полностью отдельным блоком, повторяя все
позиции из списка литературы к русскоязычной части, независимо от того, имеются или
нет в нем иностранные источники. Если в списке литературы к русскоязычной части есть
ссылки на иностранные публикации, набранные латиницей, они полностью повторяются в
списке ``References''.

Ниже приведены примеры ссылок на различные виды публикаций в списке ``References''.

\def\leftfootline{\small{\textbf{\thepage}
\hfill ИНФОРМАТИКА И ЕЁ ПРИМЕНЕНИЯ\ \ \ том\ 18\ \ \ выпуск\ 3\ \ \ 2024}
}%
 \def\rightfootline{\small{ИНФОРМАТИКА И ЕЁ ПРИМЕНЕНИЯ\ \ \ том\ 18\ \ \ выпуск\ 3\ \ \ 2024
\hfill \textbf{\thepage}}}

{\small

\noindent
\textbf{Описание статьи из журнала:}

\Aue{Zagurenko, A.\,G., V.\,A.~Korotovskikh, A.\,A.~Kolesnikov, A.\,V.~Timonov, and D.\,V.~Kardymon}. 2008.
Tekhniko-ekonomicheskaya optimizatsiya dizayna gidrorazryva plasta [Technical and
economic optimization of the design
of hydraulic fracturing]. \textit{Neftyanoe hozyaystvo} [\textit{Oil Industry}] 11:54--57.

\Aue{Zhang, Z., and D.~Zhu}. 2008. Experimental research on the localized
electrochemical micromachining. \textit{Russ. J.~Electrochem.}  44(8):926--930.
{\sf doi:10.1134/S1023193508080077}.

\noindent
\textbf{Описание статьи из электронного журнала:}

\Aue{Swaminathan, V., E.~Lepkoswka-White, and B.\,P.~Rao}. 1999. Browsers or buyers in cyberspace? An
investigation of electronic factors influencing electronic exchange. \textit{JCMC}
5(2). Available at: {\sf http://www.ascusc.org/jcmc/vol5/issue2/} (accessed April~28, 2011).

\def\leftkol{Правила подготовки рукописей  для публикации в журнале
<<Информатика и её применения>>}

\def\rightkol{Правила подготовки рукописей  для публикации в журнале
<<Информатика и её применения>>}


\noindent
\textbf{Описание статьи из продолжающегося издания (сборника трудов):}

\Aue{Astakhov, M.\,V., and T.\,V.~Tagantsev}. 2006. Eksperimental'noe
issledovanie prochnosti soedineniy ``stal'--kompozit'' [Experimental study of
the strength of joints ``steel--composite'']. \textit{Trudy MGTU
``Matematicheskoe modelirovanie slozhnykh tekh\-ni\-che\-skikh sistem''}
[\textit{Bauman MSTU ``Mathematical Modeling of Complex Technical
Systems'' Proceedings}]. 593:125--130.


\pagebreak



\noindent
\textbf{Описание материалов конференций:}

\Aue{Usmanov, T.\,S., A.\,A.~Gusmanov, I.\,Z.~Mullagalin, R.\,Ju.~Muhametshina, A.\,N.~Chervyakova, and
A.\,V.~Sveshnikov}. 2007. Osobennosti proektirovaniya razrabotki mestorozhdeniy
s primeneniem gidrorazryva
plasta [Features of the design of field development with the use of hydraulic fracturing].
\textit{Trudy 6-go
Mezhdu\-na\-rod\-no\-go Simpoziuma ``Novye resursosberegayushchie tekhnologii nedropol'zovaniya i povysheniya
neftegazootdachi''} [\textit{6th  Symposium (International) ``New Energy Saving Subsoil Technologies and
the Increasing of the Oil and Gas Impact'' Proceedings}]. Moscow. 267--272.



\def\leftfootline{\small{\textbf{\thepage}
\hfill ИНФОРМАТИКА И ЕЁ ПРИМЕНЕНИЯ\ \ \ том\ 18\ \ \ выпуск\ 3\ \ \ 2024}
}%
 \def\rightfootline{\small{ИНФОРМАТИКА И ЕЁ ПРИМЕНЕНИЯ\ \ \ том\ 18\ \ \ выпуск\ 3\ \ \ 2024
\hfill \textbf{\thepage}}}



\noindent
\textbf{Описание книги (монографии, сборники):}



Lindorf, L.\,S., and L.\,G.~Mamikoniants, eds. 1972.
\textit{Ekspluatatsiya turbogeneratorov s neposredstvennym
okhlazhdeniem} [\textit{Operation of turbine generators with direct cooling}].
Moscow: Energy Publs. 352~p.


\Aue{Latyshev, V.\,N.} 2009. \textit{Tribologiya rezaniya. Kn.~1: Friktsionnye protsessy
pri rezanii metallov}
[\textit{Tribology of cutting. Vol.~1: Frictional processes in metal cutting}]. Ivanovo: Ivanovskii
State Univ. 108~p.

\def\leftkol{Правила подготовки рукописей  для публикации в журнале
<<Информатика и её применения>>}

\def\rightkol{Правила подготовки рукописей  для публикации в журнале
<<Информатика и её применения>>}

\noindent
\textbf{Описание переводной книги}
(в списке литературы к русскоязычной части необходимо указать:~/ Пер.\ с англ.~---
после названия книги, а в конце ссылки указать оригинал книги в круглых скобках):
\begin{enumerate}[1.]
\item  В русскоязычной части:

\def\leftfootline{\small{\textbf{\thepage}
\hfill ИНФОРМАТИКА И ЕЁ ПРИМЕНЕНИЯ\ \ \ том\ 18\ \ \ выпуск\ 3\ \ \ 2024}
}%
 \def\rightfootline{\small{ИНФОРМАТИКА И ЕЁ ПРИМЕНЕНИЯ\ \ \ том\ 18\ \ \ выпуск\ 3\ \ \ 2024
\hfill \textbf{\thepage}}}

\Au{Тимошенко С.\,П., Янг Д.\,Х., Уивер~У.}
Колебания в инженерном деле~/ Пер.\ с англ.~--- М.: Машиностроение, 1985. 472~с.
(\Au{Timoshenko~S.\,P., Young~D.\,H., Weaver~W.}
Vibration problems in engineering.~--- 4th ed.~--- New York, NY, USA: Wiley, 1974. 521~p.)\\[-13.5pt]
\item  В англоязычной части:

\Aue{Timoshenko, S.\,P., D.\,H.~Young, and W.~Weaver}.
1974. \textit{Vibration problems in engineering}. 4th ed. New York: 
Wiley. 521~p.
\end{enumerate}

\vspace*{-3pt}


\noindent
\textbf{Описание неопубликованного документа:}


\Aue{Latypov, A.\,R., M.\,M.~Khasanov, and V.\,A.~Baikov}.
2004 (unpubl.). Geologiya i~dobycha (NGT GiD) [Geology and production (NGT GiD)]. Certificate on official registration of the computer program
No.\,2004611198. 

\noindent
\textbf{Описание интернет-ресурса:}


Pravila tsitirovaniya istochnikov [Rules for the citing of sources]. Available at: {\sf
http://www.scribd.com/doc/1034528/} (accessed February~7, 2011).

%\pagebreak

\noindent
\textbf{Описание диссертации или автореферата диссертации:}

\Aue{Semenov, V.\,I.}
2003. Matematicheskoe modelirovanie plazmy v sisteme kompaktnyy tor [Mathematical
modeling of the plasma in the compact torus].  Moscow.  D.Sc.\ Diss. 272~p.

\Aue{Kozhunova, O.\,S.} 2009. Tekhnologiya razrabotki semanticheskogo
slovarya informatsionnogo monitoringa [Technology of development of
semantic dictionary of information monitoring system].  Moscow: IPI RAN. PhD Thesis. 23~p.


\noindent
\textbf{Описание ГОСТа:}

GOST 8.586.5-2005. 2007. Metodika vypolneniya izmereniy. Izmerenie raskhoda i~kolichestva zhidkostey i~gazov
s~pomoshch'yu standartnykh suzhayushchikh ustroystv [Method of measurement.
Measurement of flow rate and volume of liquids and gases by means of orifice devices]. Moscow:
Standardinform  Publs. 10~p.

\noindent
\textbf{Описание патента:}

\Aue{Bolshakov, M.\,V., A.\,V.~Kulakov, A.\,N.~Lavrenov, and M.\,V.~Palkin}.
2006. Sposob orientirovaniya po krenu letatel'nogo
apparata s opti\-che\-skoy golovkoy
samonavedeniya [The way to orient on the roll of aircraft with optical homing head].
Patent RF No.\,2280590.
}

\item Присланные в редакцию материалы авторам не возвращаются.\\[-13.5pt]

\item При отправке файлов по электронной почте просим придерживаться следующих
правил:
\begin{itemize}
\item указывать в поле subject (тема) название журнала и фамилию автора;\\[-13.5pt]
\item указывать в тексте письма название статьи, авторов и~журнал, в~который направляется статья;\\[-13.5pt]
\item использовать attach (присоединение);\\[-13.5pt]
\item в состав электронной версии статьи должны входить: файл, содержащий текст
статьи, и файл(ы), содержащий(е) иллюстрации.\\[-13.5pt]
\end{itemize}

\item Журнал <<Информатика и её применения>> является некоммерческим изданием.
Плата за публикацию не взимается, гонорар авторам не выплачивается.
\end{enumerate}



\def\leftfootline{\small{\textbf{\thepage}
\hfill ИНФОРМАТИКА И ЕЁ ПРИМЕНЕНИЯ\ \ \ том\ 18\ \ \ выпуск\ 3\ \ \ 2024}
}%
 \def\rightfootline{\small{ИНФОРМАТИКА И ЕЁ ПРИМЕНЕНИЯ\ \ \ том\ 18\ \ \ выпуск\ 3\ \ \ 2024
\hfill \textbf{\thepage}}}


\vspace*{-1mm}

\begin{center}

\textbf{Адрес редакции журнала <<Информатика и её применения>>:} \\




Москва 119333, ул.~Вавилова, д.~44, корп.~2, ФИЦ ИУ РАН\\[-10pt]

\

Тел.: +7\,(499)\,135-86-92\ \ Факс:  +7\,(495)\,930-45-05\\[-10pt]

 \

e-mail:   {\sf iiep@frccsc.ru} (Стригина Светлана Николаевна)\\[-10pt]

\

{\sf http://www.ipiran.ru/journal/issues/}
\end{center}
}


\def\leftkol{Правила подготовки рукописей  для публикации в журнале
<<Информатика и её применения>>}

\def\rightkol{Правила подготовки рукописей  для публикации в журнале
<<Информатика и её применения>>}


\def\leftfootline{\small{\textbf{\thepage}
\hfill ИНФОРМАТИКА И ЕЁ ПРИМЕНЕНИЯ\ \ \ том\ 18\ \ \ выпуск\ 3\ \ \ 2024}
}%
 \def\rightfootline{\small{ИНФОРМАТИКА И ЕЁ ПРИМЕНЕНИЯ\ \ \ том\ 18\ \ \ выпуск\ 3\ \ \ 2024
\hfill \textbf{\thepage}}} 
\def\stat{podg-e}
{%\hrule\par
%\vskip 7pt % 7pt
\vspace*{-24pt}
\raggedleft\Large \bf%\baselineskip=3.2ex
Requirements for manuscripts submitted to Journal
``Informatics~and~Applications'' \vskip 8pt
    \hrule
    \par
\vskip 21pt plus 6pt minus 3pt }

\label{st\stat}

\def\tit{\ }

\def\aut{\ }
\def\auf{\ }

\def\leftkol{\ }

\def\rightkol{\ }
%Requirements for manuscripts submitted to Journal
%``Informatics~and~Applications''}

\titele{\tit}{\aut}{\auf}{\leftkol}{\rightkol}

\def\leftfootline{\small{\textbf{\thepage}
\hfill INFORMATIKA I EE PRIMENENIYA~--- INFORMATICS AND APPLICATIONS\ \ \ 2019\
\ \ volume~13\ \ \ issue\ 4}
}%
 \def\rightfootline{\small{INFORMATIKA I EE PRIMENENIYA~--- INFORMATICS AND APPLICATIONS\ \ \ 2019\ \ \ volume~13\ \ \ issue\ 4
\hfill \textbf{\thepage}}}

\vspace*{-60pt}

{\small

\noindent
Journal ``Informatics and Applications'' (Inform.\ Appl.)
publishes theoretical, review, and discussion
articles on the research and development in the
field of informatics and its applications.

The journal is published in Russian.
By a special decision of the editorial
board, some articles can be published in English.


The topics covered include the following areas:
\begin{itemize}
               \item
     theoretical fundamentals of informatics; \\[-14pt]
\item
mathematical methods for studying complex systems and processes; \\[-14pt]
\item
information systems and networks;\\[-14pt]
\item
information technologies; and \\[-14pt]
\item
architecture and software of computational complexes and networks. \\[-14pt]
\end{itemize}

\noindent
\begin{enumerate}[1.]
\item The Journal publishes original articles which have not been published before and are not
intended for simultaneous publication in other editions. An article submitted to the Journal must not violate the
Copyright law. Sending the manuscript to the Editorial Board, the authors retain all rights of the
owners of the manuscript and transfer the nonexclusive rights to publish the article in Russian
(or the language of the article, if not Russian) and its distribution in Russia and abroad to the
Founders and the Editorial Board. Authors should submit a letter to the Editorial Board in the
following form:

{\bfseries\textit{Agreement on the transfer of rights to publish:}}

``\textit{We, the undersigned authors of the manuscript ``\ldots'', pass to the
Founder and the Editorial Board of the Journal ``Informatics and Applications''
the nonexclusive right to publish the manuscript of the article in Russian (or
in English) in both print and electronic versions of the Journal. We affirm
that this publication does not violate the Copyright of other persons or
organizations.}

\textit{Author(s) signature(s): (name(s), address(es), date).}

This agreement should be submitted in paper form or in the form of a scanned copy (signed by
the authors).


%The Editorial Board has the right to request from the authors an official expert conclusion that
%the submitted article has no secret data prohibited for publication. \\[-13.5pt]
\item
A submitted article should be attached with \textbf{the data on the author(s)} (see item~8). If
there are several authors, the contact person should be indicated who is responsible for
correspondence with the Editorial Board and other authors about revisions and final approval
of the proofs.\\[-13.5pt]

\item The Editorial Board of the Journal examines the article according to the established
reviewing procedure. If the authors receive their article for correction after reviewing, it does not
mean that the article is approved for publication. The corrected article should be sent to the
Editorial Board for the subsequent review and approval.\\[-13.5pt]

\item The decision on the article publication or its rejection is communicated to the authors. The
Editorial Board may also send the reviews on the submitted articles to the authors. Any
discussion upon the rejected articles is not possible.\\[-13.5pt]

\item The edited articles will be sent to the authors for proofread. The comments of the authors
to the edited text of the article should be sent to the Editorial Board as soon as possible.\\[-13.5pt]

\item The manuscript of the article should be presented electronically in the MS WORD (.doc or
.docx) or \LaTeX\ (.tex) formats, and additionally in the .pdf format. All documents
 may be sent
by e-mail or provided on a CD or diskette. A~hard copy submission is not necessary.\\[-13.5pt]

\item The recommended typesetting instructions for manuscript.

Pages parameters: format A4, portrait orientation, document margins (cm): left~--- 2.5, right~---
1.5, above~--- 2.0, below~--- 2.0, footer 1.3.

Text: font~---Times New Roman, font size~--- 14, paragraph indent~--- 0.5, line spacing~--- 1.5,
justified alignment.

The recommended manuscript size: not more than 15~pages of the specified format.
If the specified size exceeded, the editorial board is entitled to require the author
to reduce the manuscript.

Use only standard abbreviations. Avoid  abbreviations in the title and
abstract. The full term for which an abbreviation stands should precede
its first use in the text unless it is a standard unit of measurement.

All pages of the manuscript should be numbered.

The templates for the manuscript typesetting are presented on site: {\sf
http://www.ipiran.ru/journal/template.doc}.\\[-13.5pt]


%\def\leftkol{Requirements for manuscripts submitted to Journal
%``Informatics~and~Applications''}

\item The articles should enclose data both in \textbf{Russian and English}:
\begin{itemize}
\item title;\\[-13.5pt]
\item author's name and surname;\\[-13.5pt]
\item affiliation~--- organization, its address with ZIP code, city, country, and
official e-mail address;\\[-13.5pt]
\item data on authors according to the format: (see site)

{\sf http://www.ipiran.ru/journal/issues/2013\_07\_01/authors.asp}  and

{\sf  http://www.ipiran.ru/journal/issues/2013\_07\_01\_eng/authors.asp};\\[-13.5pt]

\pagebreak

\def\leftfootline{\small{\textbf{\thepage}
\hfill INFORMATIKA I EE PRIMENENIYA~--- INFORMATICS AND APPLICATIONS\ \ \ 2019\
\ \ volume~13\ \ \ issue\ 4}
}%
 \def\rightfootline{\small{INFORMATIKA I EE PRIMENENIYA~--- INFORMATICS AND APPLICATIONS\ \ \ 2019\ \ \ volume~13\ \ \ issue\ 4
\hfill \textbf{\thepage}}}


%\def\leftkol{Requirements for manuscripts submitted to Journal
%``Informatics~and~Applications''}

%\def\rightkol{Requirements for manuscripts submitted to Journal
%``Informatics~and~Applications''}



\item abstract (not less than 100 words) both in Russian and in English. Abstract is a short
summary of the article that can be published separately. The abstract is the
main source of information on the article and it could be included in leading information
systems and data bases. The abstract in English has to be an original text and should
not be an exact translation of the Russian one. Good English is required.
In abstracts, avoid references and formulae;\\[-13.5pt]
\item indexing is performed on the basis of keywords. The use of keywords from the
internationally accepted thematic Thesauri is recommended.

%\def\leftkol{Requirements for manuscripts submitted to Journal
%``Informatics~and~Applications''}

%\def\rightkol{Requirements for manuscripts submitted to Journal
%``Informatics~and~Applications''}

Important! Keywords must not be sentences;
\item Acknowledgments.
\end{itemize}

\item References. Russian references have to be presented both in English translation and Latin
transliteration (refer {\sf http://www.translit.net/ru/bgn/}).

Please take into account the following examples of Russian references appearance:

\noindent
\textbf{Article in journal:}

\Aue{Zhang, Z., and D.~Zhu}. 2008. Experimental research on the localized electrochemical
micromachining.
\textit{Rus. J.~Electrochem.}  44(8):926--930. {\sf doi:10.1134/S1023193508080077}.


\noindent
\textbf{Journal article in electronic format:}

\Aue{Swaminathan, V., E.~Lepkoswka-White, and B.\,P.~Rao}. 1999. Browsers or buyers in
cyberspace? An
investigation of electronic factors influencing electronic exchange. \textit{JCMC}
5(2). Available at: {\sf http://www.ascusc.org/jcmc/vol5/issue2/} (accessed April~28, 2011).




\noindent
\textbf{Article from the continuing publication (collection of works, proceedings):}

\Aue{Astakhov, M.\,V., and T.\,V.~Tagantsev}. 2006. Eksperimental'noe
issledovanie prochnosti soedineniy ``stal'--kompozit'' [Experimental study of
the strength of joints ``steel--composite'']. \textit{Trudy MGTU
``Matematicheskoe modelirovanie slozhnykh tekh\-ni\-che\-skikh sistem''}
[\textit{Bauman MSTU ``Mathematical Modeling of Complex Technical
Systems'' Proceedings}]. 593:125--130.

\def\leftfootline{\small{\textbf{\thepage}
\hfill INFORMATIKA I EE PRIMENENIYA~--- INFORMATICS AND APPLICATIONS\ \ \ 2019\
\ \ volume~13\ \ \ issue\ 4}
}%
 \def\rightfootline{\small{INFORMATIKA I EE PRIMENENIYA~--- INFORMATICS AND APPLICATIONS\ \ \ 2019\ \ \ volume~13\ \ \ issue\ 4
\hfill \textbf{\thepage}}}

\def\leftkol{Requirements for manuscripts submitted to Journal
``Informatics~and~Applications''}

\def\rightkol{Requirements for manuscripts submitted to Journal
``Informatics~and~Applications''}

\noindent
\textbf{Conference proceedings:}

\Aue{Usmanov, T.\,S., A.\,A.~Gusmanov, I.\,Z.~Mullagalin, R.\,Ju.~Muhametshina,
A.\,N.~Chervyakova, and
A.\,V.~Sveshnikov}. 2007. Osobennosti proektirovaniya razrabotki mestorozhdeniy
s primeneniem gidrorazryva
plasta [Features of the design of field development with the use of hydraulic fracturing].
\textit{Trudy 6-go
Mezhdu\-na\-rod\-no\-go Simpoziuma ``Novye resursosberegayushchie tekhnologii
nedropol'zovaniya i povysheniya
neftegazootdachi''} [\textit{6th  Symposium (International) ``New Energy Saving Subsoil
Technologies and
the Increasing of the Oil and Gas Impact'' Proceedings}]. Moscow. 267--272.


\noindent
\textbf{Books and other monographs:}




Lindorf, L.\,S., and L.\,G.~Mamikoniants, eds. 1972.
\textit{Ekspluatatsiya turbogeneratorov s neposredstvennym
okhlazhdeniem} [\textit{Operation of turbine generators with direct cooling}].
Moscow: Energy Publs. 352~p.


%\Aue{Latyshev, V.\,N.} 2009. \textit{Tribologiya rezaniya. Kn.~1: Frikcionnye prosessy
%pri rezanii metallov}
%[\textit{Tribology of cutting. Vol.~1: Frictional processes in metal cutting}]. Ivanovo: Ivanovskii
%State Univ. 108~p.


%\noindent
%\textbf{Unpublished material:}

%\Aue{Latypov, A.\,R., M.\,M.~Khasanov, and V.\,A.~Baikov}.
%2004. Geology and production (NGT GiD). Certificate on official registration of the computer
%program
%No.\,2004611198. (In Russian, unpubl.)

%\noindent
%\textbf{Internet-source:}

%APA Style. 2011. Available at: {\sf http://www.apastyle.org/apa-style-help.aspx} (accessed
%February~5, 2011).

%Pravila citirovaniya istochnikov [Rules for the citing of sources]. Available at: {\sf
%http://www.scribd.com/doc/1034528/} (accessed February~7, 2011).


\noindent
\textbf{Dissertation and Thesis:}

%\Aue{Semenov, V.\,I.}
%2003. Matematicheskoe modelirovanie plazmy v sisteme kompaktnyy tor. [Mathematical
%modeling of the plasma in the compact torus]. D.Sc.\ Diss. Moscow. 272~p.

\Aue{Kozhunova, O.\,S.} 2009. Tekhnologiya razrabotki semanticheskogo
slovarya informatsionnogo monitoringa [Technology of development of
semantic dictionary of information monitoring system]. PhD Thesis. Moscow: IPI RAN. 23~p.


\noindent
\textbf{State standards and patents:}

GOST 8.586.5-2005. 2007. Metodika vypolneniya izmereniy. Izmerenie raskhoda i~kolichestva
zhidkostey i gazov 
s~pomoshch'yu standartnykh suzhayushchikh ustroystv [Method of measurement.
Measurement of flow rate and volume of liquids and gases by means of orifice devices]. M.:
Standardinform
Publs. 10~p.

%\noindent
%\textbf{Patent:}

\Aue{Bolshakov, M.\,V., A.\,V.~Kulakov, A.\,N.~Lavrenov, and M.\,V.~Palkin}.
2006. Sposob orientirovaniya po krenu letatel'nogo
apparata s opti\-che\-skoy golovkoy
samonavedeniya [The way to orient on the roll of aircraft with optical homing head].
Patent RF No.\,2280590.

References in Latin transcription are presented in the original language.

References in the text are numbered according to the order of their
first appearance; the number is
placed in square brackets. All items from the reference list should be
cited.\\[-13.5pt]

\item Manuscripts and additional materials are not returned to Authors by the Editorial Board.\\[-13.5pt]

\item Submissions of files by e-mail must include:\\[-13.5pt]
\begin{itemize}
\item   the journal title and author's name in the ``Subject'' field; \\[-13.5pt]
\item   an article and additional materials have to be attached using the ``attach'' function;\\[-13.5pt]
\item   an electronic version of the article should contain the file with the text and a separate file
with figures.\\[-13.5pt]
\end{itemize}

\item ``Informatics and Applications'' journal is not a profit publication. There are no
charges for the authors as well as there are no royalties.\\[-13.5pt]
\end{enumerate}

\def\leftfootline{\small{\textbf{\thepage}
\hfill INFORMATIKA I EE PRIMENENIYA~--- INFORMATICS AND APPLICATIONS\ \ \ 2019\
\ \ volume~13\ \ \ issue\ 4}
}%
 \def\rightfootline{\small{INFORMATIKA I EE PRIMENENIYA~--- INFORMATICS AND APPLICATIONS\ \ \ 2019\ \ \ volume~13\ \ \ issue\ 4
\hfill \textbf{\thepage}}}

\def\leftkol{Requirements for manuscripts submitted to Journal
``Informatics~and~Applications''}

\def\rightkol{Requirements for manuscripts submitted to Journal
``Informatics~and~Applications''}


%\vspace*{5mm}


\begin{center}
\textbf{Editorial Board address:} \\

%ABOUT AUTHORS



FRC CSC RAS, 44, block~2, Vavilov Str., Moscow 119333, Russia\\[-10pt]

\

Ph.: +7\,(499)\,135\,86\,92,\ \ Fax: +7\,(495)\,930\,45\,05\\[-10pt]

\

 e-mail: {\sf rust@ipiran.ru} (to Prof.\ Rustem Seyful-Mulyukov)\\[-10pt]

\

 {\sf http://www.ipiran.ru/english/journal.asp}
\end{center}
 }
%\thispagestyle{myheadings}

\def\leftkol{Requirements for manuscripts submitted to Journal
``Informatics~and~Applications''}

\def\rightkol{Requirements for manuscripts submitted to Journal
``Informatics~and~Applications''}

\def\leftfootline{\small{\textbf{\thepage}
\hfill INFORMATIKA I EE PRIMENENIYA~--- INFORMATICS AND APPLICATIONS\ \ \ 2019\
\ \ volume~13\ \ \ issue\ 4}
}%
 \def\rightfootline{\small{INFORMATIKA I EE PRIMENENIYA~--- INFORMATICS AND APPLICATIONS\ \ \ 2019\ \ \ volume~13\ \ \ issue\ 4
\hfill \textbf{\thepage}}}

 \label{end\stat}

\newpage

%\vspace*{-60pt} {\small
{\baselineskip=9.1pt
\section*{Правила подготовки рукописей статей для публикации в журнале
<<Информатика и её применения>>}

\thispagestyle{empty}

 Журнал <<Информатика и её применения>> публикует
теоретические, обзорные и дискуссионные статьи, посвященные научным
исследованиям и разработкам в области информатики и ее приложений. Журнал
издается на русском языке. По специальному решению редколлегии отдельные статьи,
в виде исключения, могут печататься на английском языке.
Тематика журнала охватывает следующие направления:
\begin{itemize}
\item теоретические основы информатики; %\\[-13.5pt]
\item математические методы исследования сложных систем и процессов; %\\[-13.5pt]
\item информационные системы и сети; %\\[-13.5pt]
\item информационные технологии; %\\[-13.5pt]
\item архитектура и программное
обеспечение вычислительных комплексов и сетей.
\end{itemize}
\begin{enumerate}
\item В журнале печатаются результаты, ранее не
опубликованные и не предназначенные к одновременной публикации в других
изданиях. Публикация не должна нарушать закон об авторских правах. Направляя
свою рукопись в редакцию, авторы автоматически передают учредителям и
редколлегии неисключительные права на издание данной статьи на русском языке и
на ее распространение в России и за рубежом. При этом за авторами сохраняются
все права как собственников данной рукописи. В связи с этим авторами должно
быть представлено в редакцию письмо в следующей форме:
Соглашение о передаче права на публикацию:

\textit{<<Мы, нижеподписавшиеся, авторы рукописи <<$\qquad\qquad$>>, передаем
учредителям и редколлегии журнала <<Информатика и её применения>>
неисключительное право опубликовать данную рукопись статьи на русском языке как
в печатной, так и в электронной версиях журнала. Мы подтверждаем, что данная
публикация не нарушает авторского права других лиц или организаций. Подписи
авторов: (ф.\,и.\,о., дата, адрес)>>.}

Указанное соглашение может быть представлено 
как в бумажном виде, так и в виде отсканированной копии (с подписями авторов).


Редколлегия вправе запросить у авторов экспертное заключение о возможности
опубликования представленной статьи в открытой печати. %\\[-13.5pt]
\item Статья
подписывается всеми авторами. На отдельном листе представляются данные автора
(или всех авторов): фамилия, полные имя и отчество, телефон, факс, e-mail,
почтовый адрес. Если работа выполнена несколькими авторами, указывается фамилия
одного из них, ответственного за переписку с редакцией. %\\[-13.5pt]
\item Редакция журнала
осуществляет самостоятельную экспертизу присланных статей. Возвращение рукописи
на доработку не означает, что статья уже принята к печати. Доработанный вариант
с ответом на замечания рецензента необходимо прислать в редакцию. %\\[-13.5pt]
\item Решение
редакционной коллегии о принятии статьи к печати или ее отклонении сообщается
авторам. Редколлегия не обязуется направлять рецензию авторам отклоненной
статьи. %\\[-13.5pt]
\item Корректура статей высылается авторам для просмотра. Редакция
просит авторов присылать свои замечания в кратчайшие сроки. %\\[-13.5pt]
\item При
подготовке рукописи в MS Word рекомендуется использовать следующие настройки.
Параметры страницы: формат~--- А4; ориентация~--- книжная; поля (см): внутри~---
2,5, снаружи~--- 1,5, сверху~--- 2, снизу~--- 2, от края до нижнего
колонтитула~--- 1,3. Основной текст: стиль~--- <<Обычный>>: шрифт Times New
Roman, размер 14~пунктов, абзацный отступ~--- 0,5~см, 1,5 интервала,
выравнивание~--- по ширине. Рекомендуемый объем рукописи~--- не свыше
25~страниц указанного формата. Ознакомиться с шаблонами, содержащими примеры
оформления, можно по адресу в Интернете:
\textsf{http://www.ipiran.ru/journal/template.doc}.
\item К рукописи, предоставляемой в 2-х
экземплярах, обязательно прилагается электронная версия статьи (как правило, в
форматах MS WORD (.doc) или \LaTeX\ (.tex), а также~--- дополнительно~--- в
формате .pdf) на дискете, лазерном диске или по электронной почте. Сокращения
слов, кроме стандартных, не применяются. Все страницы рукописи должны быть
пронумерованы. %\\[-13.5pt]
\item Статья должна содержать следующую информацию на русском и
английском языках: название, Ф.И.О. авторов, места работы авторов и их
электронные адреса, подробные сведения об авторах, оформленные в соответствии с форматом, 
определяемым файлами {\sf http://www.ipiran.ru/journal/issues/2011\_05\_01/authors.asp} и 
{\sf http://www.ipiran.ru/journal/issues/2011\_01\_eng/authors.asp},
аннотация (не более 100~слов), ключевые слова. Ссылки на
литературу в тексте статьи нумеруются (в квадратных скобках) и располагаются в
порядке их первого упоминания. В~списке литературы не должно быть позиций, на которые нет ссылки в тексте статьи.
Все фамилии авторов, заглавия статей, названия
книг, конференций и~т.\,п.\ даются на языке оригинала, если этот язык
использует кириллический или латинский алфавит. %\\[-13.5pt]
\item Присланные в редакцию материалы авторам не возвращаются.
\item При отправке файлов по электронной
почте просим придерживаться следующих правил:
\begin{itemize}
\item указывать в поле subject (тема) название журнала и фамилию автора; %\\[-13.5pt]
\item использовать attach (присоединение); %\\[-13.5pt]
\item в случае больших объемов информации возможно
использование общеизвестных архиваторов (ZIP, RAR); %\\[-13.5pt]
\item в состав электронной версии статьи должны входить: файл, содержащий текст статьи, и файл(ы),
содержащий(е) иллюстрации. %\\[-13.5pt]
\end{itemize}
\item Журнал <<Информатика и её применения>> является некоммерческим изданием. 
Плата за публикацию с авторов не взимается, гонорар авторам не выплачивается.
\end{enumerate}
\thispagestyle{empty}
\textbf{Адрес редакции:} Москва 119333,
ул.~Вавилова, д.~44, корп.~2, ИПИ РАН\\
\hphantom{\textbf{Адрес редакции:} }Тел.: +7 (499) 135-86-92\ \
Факс:  +7 (495) 930-45-05\ \  E-mail:   rust@ipiran.ru }
}

%\include{ipi-ind}

%\tableofcontents

\end{document}

%\tableofcontents

%\end{document}

%\tableofcontents


\end{document}

\newcommand{\Ack}{\subsection*{\protect\large\bf Acknowledgments}}