\def\stat{goncharov}

\def\tit{ВЫРАВНИВАНИЕ ДЕКАРТОВЫХ ПРОИЗВЕДЕНИЙ УПОРЯДОЧЕННЫХ МНОЖЕСТВ$^*$}

\def\titkol{Выравнивание декартовых произведений упорядоченных множеств}

\def\aut{А.\,В.~Гончаров$^1$, В.\,В.~Стрижов$^2$}

\def\autkol{А.\,В.~Гончаров, В.\,В.~Стрижов}

\titel{\tit}{\aut}{\autkol}{\titkol}

\index{Гончаров А.\,В.}
\index{Стрижов В.\,В.}
\index{Goncharov A.\,V.}
\index{Strijov V.\,V.}


{\renewcommand{\thefootnote}{\fnsymbol{footnote}} \footnotetext[1]
{Работа выполнена при частичной финансовой поддержке РФФИ 
(проекты 19-07-1155 и~19-07-00885). Настоящая статья содержит 
результаты проекта <<Статистические методы машинного обучения>>, 
выполняемого в~рамках реализации Программы Центра компетенций 
Национальной технологической инициативы <<Центр хранения 
и~анализа больших данных>>, поддерживаемого Министерством науки 
и~высшего образования Российской Федерации по договору МГУ им.\ 
М.\,В.~Ломоносова  с~Фондом поддержки проектов Национальной 
технологической инициативы от 11.12.2018 №\,13/1251/2018.}}


\renewcommand{\thefootnote}{\arabic{footnote}}
\footnotetext[1]{Московский физико-технический институт, alex.goncharov@phystech.edu}
\footnotetext[2]{Вычислительный центр им.\ А.\,А.~Дородницына Федерального исследовательского 
центра <<Информатика и~управ\-ле\-ние>> Российской академии наук; 
Московский фи\-зи\-ко-тех\-ни\-че\-ский институт, \mbox{strijov@ccas.ru}}

%\vspace*{-12pt}



\Abst{Работа посвящена исследованию метрических методов анализа 
объектов сложной структуры. Предлагается обобщить метод динамического 
выравнивания двух временных рядов на случай объектов, определенных на 
двух и~более осях времени. В~дискретном представлении такие объекты 
являются матрицами. Метод динамического выравнивания временных рядов 
обобщается как метод динамического выравнивания матриц. Предложена 
функция расстояния, устойчивая к~монотонным нелинейным деформациям 
декартова произведения двух и~более временных шкал. Определен выравнивающий 
путь между объектами. В~дальнейшем объектом называется матрица, 
в~которой строки и~столбцы соответствуют осям времени. Исследованы 
свойства предложенной функции расстояния. Для иллюстрации метода 
решаются задачи метрической классификации объектов на модельных 
данных и~данных из набора MNIST.}

\KW{функция расстояния; динамическое выравнивание; расстояние между матрицами; 
нелинейные деформации времени; про\-стран\-ст\-вен\-но-вре\-мен\-ные ряды}

\DOI{10.14357/19922264200105} 
  
\vspace*{-3pt}


\vskip 10pt plus 9pt minus 6pt

\thispagestyle{headings}

\begin{multicols}{2}

\label{st\stat}


\section{Введение}

Временн$\acute{\mbox{ы}}$е ряды представляют собой набор измерений, упорядоченных 
по оси времени. Анализ временн$\acute{\mbox{ы}}$х рядов производится при решении задач, 
связанных с~классификацией активности человека по измерениям акселерометра 
телефона, поиском паттернов в~EEG-сиг\-на\-лах (электроэнцефалограмма), 
кластеризации набора ECoG (электрокортикограмма) данных и~во многих других 
задачах~\cite{0}. Рассматриваются объекты, для которых время между измерениями 
фиксированно. В~данной работе для построения адекватной функции 
расстояния между объектами требуется учесть нелинейные деформации 
относительно оси времени: глобальные и~локальные сдвиги, растяжения 
и~сжатия~\cite{1}.

В~\cite{2} приводятся различные методы решения задач анализа 
временн$\acute{\mbox{ы}}$х рядов: классификации, детектирования паттернов, 
кластеризации и~др. В~\cite{3} описание временных рядов 
строится с~по\-мощью анализа параметров моделей, в~\cite{4} 
используется их признаковое описание, в~\cite{5} анализируется их форма. 
Комбинации этих подходов описаны в~\cite{2}.

Метрические методы находят схожие объекты в~наборе. Используются 
функции расстояния над временн$\acute{\mbox{ы}}$ми рядами: расстояние Хаусдорфа~\cite{10}, 
MODH~\cite{11}, расстояние, основанное на HMM
(hiden Markov model)~\cite{6}, евклидово расстояние 
в~исходном пространстве или в~пространстве сниженной размерности~\cite{5}, 
\mbox{LCSS} (longest common\linebreak subsequence)~\cite{7}. Показано~\cite{8}, что в~случае локальных или глобальных 
деформаций времени при решении задач, требующих анализа исходной формы 
временн$\acute{\mbox{о}}$го ряда, метод динамического выравнивания оси времени 
DTW (Dynamic Time Warping) 
превосходит другие функции расстояния~\cite{9} по качеству итогового 
решения задачи, так как при наличии смещений двух объектов относительно 
друг друга требуется выравнивать их оптимальным образом для вычисления 
расстояния между ними.

В данной работе предлагается перейти от рас\-смот\-ре\-ния объекта~$\textbf{s}(t)$, 
временн$\acute{\mbox{о}}$го ряда, к~более общему случаю $\textbf{s}(\textbf{t})$, 
в~котором компоненты вектора~$\textbf{t}$~--- оси времени. Из-за 
существенного рос\-та вы\-чис\-ли\-тель\-ной слож\-ности при увеличении чис\-ла 
осей времени предлагается рас\-смот\-реть объекты $\textbf{s}(t_1, t_2)$, 
определенные на двух осях времени. Оси времени считаются независимыми. 
В~случае единственной дискретной и~ограниченной сверху шкалы времени 
объект представим вектором фиксированной размерности. 
Аналогично объект настоящего исследования представим мат\-ри\-цей.

Вводятся ограничения на зависимости осей времени в~декартовом 
произведении для таких объектов. Определена гипотеза порождения данных: 
объекты одного класса эквивалентности получены при помощи допустимых 
преобразований, а~именно: локальных деформаций (растяжений и~сжатий) 
каждой из осей времени по отдельности. В~дискретном случае преобразование 
представимо дуп\-ли\-ци\-ро\-ва\-ни\-ем строк и~столбцов матриц. 
В~число допустимых преобразований попадают и~глобальные деформации: 
сдвиги по осям времени, представимые добавлением и~удалением крайних 
строк и~столбцов исходных матриц. Для каждой из осей времени выполняются 
свойства времени: монотонность и~непрерывность. Похожими на описанные 
свойствами обладает, например, частотный спектр сигнала, где одна ось 
определяет время, а другая~--- частоту, величину, обратную времени.


Между двумя объектами, матрицами, в~случае допустимых преобразований 
требуется определить инвариантную к~преобразованиям осей времени функцию 
расстояния, которая сможет выделить классы эквивалентности множества 
преобразованных объектов. Работа посвящена определению такой функции 
расстояния, как обобщения метода динамического выравнивания временных рядов 
DTW для матриц.

Цель данной работы~--- построение метода, основанного на динамическом 
выравнивании осей времени для матриц. Метод динамического выравнивания 
временн$\acute{\mbox{ы}}$х рядов~\cite{33} определен только для объектов с~одной осью времени, 
что делает его неприменимым для описанного случая. Однако концепции, 
используемые на каждой стадии вы\-чис\-ле\-ния оптимального выравнивания, обобщены 
на рассматриваемый случай. Работа исследует свойства предложенного 
метода и~сравнивает результаты применения метода к~задачам классификации 
изображений~\cite{12} с~результатами функции расстояния~$L_2$.

Для иллюстрации и~анализа результатов решается задача метрической 
классификации объектов (матриц низкой размерности). Используются наборы данных: 
модельные данные, которые согласуются с~выдвинутой гипотезой порождения 
данных для временн$\acute{\mbox{ы}}$х рядов, подмножество набора MNIST сниженной 
размерности и~частотный спектр сигнала.

\vspace*{-10pt}

\section{Постановка задачи построения функции расстояния}

\vspace*{-2pt}

Рассмотрим задачу построения функции расстояния между объектами. 
Функция расстояния инвариантна к~допустимым преобразованиям осей времени: 
глобальным и~локальным линейным и~нелинейным деформациям временн$\acute{\mbox{о}}$й шкалы. 
Ниже приведены две постановки задачи, с~помощью которых определены свойства 
предложенной функции расстояния, оценено ее качество и~проведено сравнение 
нескольких функций расстояния: предложенной и~$L_2$.

Первая постановка задачи использует общее свойство функций расстояния: 
объединение схожих объектов и~разделение непохожих объектов. 
Вводится определение свойства инвариантности функции расстояния к~допустимым 
преобразованиям осей времени.
Вторая постановка задачи уточняет первую и~заключается в~проведении метрической 
классификации методом ближайшего соседа.

\textbf{Постановка задачи выбора функции расстояния между двумя объектами.}
На двух временн$\acute{\mbox{ы}}$х осях заданы объекты вида 
$\textbf{A}(t_1,t_2)\hm \in \mathbb{R}^{n \times n}$. 
Функция $G_w(\textbf{A}):\mathbb{R}^{n \times n} \hm\rightarrow 
\mathbb{R}^{\hat{n} \times \hat{n}}$ задает допустимые преобразования 
исходного объекта~$\textbf{A}$: глобальные сдвиги, локальные линейные 
и~нелинейные деформации, а~именно: растяжения и~сжатия оси времени, 
сдвиги значений по оси времени. Скалярный параметр $w \hm\in \mathbb{R}^+$
 функции~$G$ фиксирует набор этих преобразований.

Допустимым элементарным преобразованием матрицы~$\textbf{A}$ назовем 
дуплицирование случайных строк и~столбцов исходной матрицы, добавление 
или удаление крайних строк и~столбцов. Допустимым преобразованием 
примем некоторую последовательность допустимых элементарных 
преобразований матрицы~$\textbf{A}$ и~обозначим как~$G_w(\textbf{A})$.

Будем называть объект~$\textbf{B} \hm\in \mathbb{R}^{\hat{n} \times \hat{n}}$ 
полученным из объекта~$\textbf{A}$ при помощи допустимых 
преобразований~$G_{\hat{w}}$, если существует $\hat{w}\hm\in \mathbb{R}^+ : 
\textbf{B} \hm= G_{\hat{w}}(\textbf{A})$.

Функцию расстояния между двумя объектами $\rho: 
\mathbb{R}^{{n} \times {n}} \times \mathbb{R}^{\hat{n} \times \hat{n}} 
\hm\rightarrow  \mathbb{R}^+$ оценим на выборке $\mathfrak{D } \hm= 
\{ \textbf{A}_i \}_{i=1}^m$ объектов вида $\textbf{A}_i \hm\in 
\mathbb{R}^{n \times n}$.

Для каждого объекта выборки~$\textbf{A}_i$ и~объекта~$\textbf{B}_j$ его 
класса эквивалентности $\{\textbf{B}_j\}_i \hm= \{  \textbf{B} 
\hm\in \mathfrak{D} | \exists w_i,w_j: G_{w_i}(\textbf{A}_i) \hm= G_{w_j}
(\textbf{B}_j)   \}$ заданы допустимые трансформации с~параметрами~$w_i$ 
и~$w_j$, такие что $G_{w_i}(\textbf{A}_i)\hm = G_{w_j}(\textbf{B}_j)$. 
Для каждого объекта выборки~$\textbf{A}_i$ и~объекта~$\textbf{C}_j$ 
из других классов эквивалентности $\{ \textbf{C}_k\}_i \hm= 
\{  \textbf{C} \hm\in \mathfrak{D} | \nexists w_i,w_k: G_{w_i}(\textbf{A}_i)
\hm = G_{w_k}(\textbf{C})   \}$ не существует таких $ w_i, w_k : G_{w_i}
(\textbf{A}_i) \hm= G_{w_k}(\textbf{C}_k)$.

Решается задача поиска функции расстояния~$\rho$, значение
 которой на паре объектов одного класса эквивалентности меньше, 
 чем на любой паре объектов из разных: для любых $i,j,k \hm\in 
 \{1,\dots,m\}$ $\quad \rho(\textbf{A}_i,\textbf{B}_j) \hm< 
 \rho(\textbf{A},\textbf{C}_k)$. Функцию расстояния, обладающую 
 таким свойством, назовем инвариантной на классах эквивалентности.

Критерием качества для функции расстояния~$\rho$ на выборке~$\mathfrak{D}$ 
примем долю объектов, для которых указанное неравенство выполняется:
$$
S_{\rho}(\mathfrak{D}) = \fr{1}{m} \sum\limits_{i=1}^m 
\prod\limits_{\{ \textbf{B}_j\}_i} 
\prod\limits_{\{ \textbf{C}_k\}_i}  
\left[  \rho(\textbf{A}_i,\textbf{B}_j) < \rho(\textbf{A}_i,\textbf{C}_k)  
 \right].
 $$
Постановка задачи выбора функции расстояния~$\rho$ 
сводится к~задаче максимизации критерия качества.

\textbf{Прикладное использование функции расстояния.}
Задана выборка $\mathfrak{D}\hm = \{(\textbf{A}_i,y_i)\}^m_{i=1}$, 
состоящая из пар объ\-ект--от\-вет. Объектами служат объекты сложной 
структуры: $\textbf{A}_i\hm \in \mathbb{R}^{n\times n}$, 
а~ответами выступают метки класса~---~$y_i\hm \in Y \hm= \{1,\ldots,E\}$, 
где $E \hm\ll m$. Выборка разделена на обучение $\mathfrak{D}_l \hm= 
\{(\textbf{A}_i,y_i)\}^{m_1}_{i=1}$ и~контроль $\mathfrak{D}_t \hm= 
\{(\textbf{A}_i,y_i)\}_{m_1}^{m_1+m_2}$.

Модель классификации~$f$ принадлежит множеству моделей метрической 
классификации 1NN, которые классифицируемому объекту ставят 
в~соответствие метку класса ближайшего объекта из обучающей 
выборки по заданной функции расстояния~$\rho$:
$$ 
\hat{y} = f(\textbf{B} | \rho) = y \argmin\limits_{i = 1,\dots, m_1} 
\rho\left(B,A_i\right)\,.
$$
Критерий качества $S$ модели~$f$ для задачи классификации~--- 
доля правильно проставленного класса на контрольной выборке:
 $$ 
 S(f | \rho) = \fr{1}{m_2}\sum\limits_{i=m_1}^{m_1+m_2} 
 \left[f(\textbf{A}_i | \rho) = y_i\right].
 $$

Требуется выбрать функцию расстояния~$\rho$ для модели 
классификации~$f:~\mathbb{R}^{n\times n} \hm\rightarrow~Y$, 
максимизируюшую критерий качества~$S$ на контрольной выборке:
\begin{equation*}
f =  \argmax\limits_{\rho \in \{\mathrm{mDTW}, L_2\}}\left(S(f | \rho)\right).
\end{equation*}

\section{Вычисление матричного расстояния mDTW}

Предлагается использовать функцию расстояния DTW, 
модифицированную для случая выравнивания двойной шкалы времени.

\smallskip

\noindent
\textbf{Определение~1.} {Даны два объекта~$\textbf{A},\textbf{B}\hm \in 
\mathbb{R}^{n\times n}$. Тензор 
невязок~$\boldsymbol{\Omega}^{n \times n \times n \times n}$~--- 
такой тензор, что его элемент~$\boldsymbol{\Omega}(i,j,k,l)$ 
равен квадрату разности между элементами~$\textbf{A}(i,j)$ и~$\textbf{B}(k,l)$:}
\begin{equation*}
\boldsymbol{\Omega}(i,j,k,l)=(\textbf{A}(i,j) - \textbf{B}(k,l))^2.
\end{equation*}

\noindent
\textbf{Определение 2.} {Путем~$\boldsymbol{\pi}$ между двумя 
объектами $\textbf{A},\textbf{B} \hm\in \mathbb{R}^{n\times n}$ 
назовем множество индексов тензора~$\boldsymbol{\Omega}$: }
$$
\boldsymbol{\pi} = \{(i,j,k,l)\},\quad i,j,k,l \in \{1,\ldots,n\} ,
$$
\textit{удовлетворяющее следующим условиям:}

{\bfseries\textit{Частичный порядок.}}
Для элементов пути~$\boldsymbol{\pi}$ с~фиксированными значениями~$i,k$ 
задан порядок: выравнивающий путь для фиксированных строк двух 
матриц упорядочен~--- $\{(i,j_r,k,l_r))\}_{r=1}^{R} \hm\subset 
\boldsymbol{\pi}$ мощностью~$R$. Аналогично для фиксированных столбцов 
с~индексами~$j,l$.

{\bfseries\textit{Граничные условия.}}
 Пусть $(i,j,k,l) \in \boldsymbol{\pi}$, тогда $(1,j,1,l) \hm\in 
 \boldsymbol{\pi}$ и~$(i,1,k,1) \hm\in \boldsymbol{\pi}$.
Путь $\boldsymbol{\pi}$ содержит элементы тензора~$\boldsymbol{\Omega}$: 
$(1,1,1,1) \hm\in \boldsymbol{\pi}$ и~$(n,n,n,n) \hm\in \boldsymbol{\pi}$.

{\bfseries\textit{Непрерывность по направлению.}}
Для упорядоченного подмножества пути $\{(i,j_r,k,l_r)\}_{r=1}^{R}
\hm\subset\boldsymbol{\pi}$ выполняется условие непрерывности:
$$
j_{r}-j_{r-1}\leq1\,,\quad l_r-l_{r-1}\leq1\,, \quad r = 2,\ldots,R\,.
$$
На~шаге пути~$\boldsymbol{\pi}$ по фиксированному направлению времени~$i,k$ 
встречаются только соседние элементы матрицы (включая соседние по диагонали). 
Аналогично для фиксированных~$j,l$.

{\bfseries\textit{Монотонность по направлению.}}
Для упорядоченного подмножества пути  $\{(i,j_r,k,l_r)\}_{r=1}^{R}
\hm\subset\boldsymbol{\pi}$ выполняется хотя бы одно из условий 
монотонности функции выравнивания времени: 
$$
j_{r}-j_{r-1}\geq1\,,\quad l_r-l_{r-1}\geq1\,, \quad r = 2,\ldots,R\,.
$$

Свойства пути между матрицами обобщают свойства пути между двумя 
временными рядами.

\smallskip

\noindent
\textbf{Определение~3.}\ {Стоимость 
$\mathrm{Cost}\,(\textbf{A},\textbf{B},{\boldsymbol{\pi}})$ пути $\boldsymbol{\pi}$ 
между объектами $\textbf{A}, \textbf{B}$:
\begin{equation*}
\mathrm{Cost}\,(\textbf{A},\textbf{B},{\boldsymbol{\pi}}) = 
\sum\limits_{(i,j,k,l) \in \boldsymbol{\pi}}{\boldsymbol{\Omega}}(i,j,k,l).
\end{equation*}}

\noindent
\textbf{Определение~4.}\ 
{Выравнивающий путь~$\hat{\boldsymbol{\pi}}$ между 
объектами $\textbf{A},\textbf{B}$~--- путь наименьшей стоимости 
среди всех возможных путей между объектами:
\begin{equation*}
\hat{\boldsymbol{\pi}} = 
\argmin\limits_{{\boldsymbol{\pi}}} \mathrm{Cost}
\left(\textbf{A},\textbf{B},{\boldsymbol{\pi}}\right).
\end{equation*}}
Функция расстояния~$\rho (\textbf{A},\textbf{B})\hm = \mathrm{mDTW}\,
(\textbf{A},\textbf{B})$ между объектами~$\textbf{A}$ и~$\textbf{B}$ 
рассчитывается как стоимость выравнивающего пути~$\hat{\boldsymbol{\pi}}$:
\begin{equation}
\mathrm{mDTW}(\textbf{A},\textbf{B}) = \mathrm{Cost}\left(\textbf{A},
\textbf{B},\hat{\boldsymbol{\pi}}\right).
\end{equation}

\setcounter{figure}{1}
\begin{figure*}[b] %fig2
{\small 
\begin{center}
\begin{tabular}{l}
\hline
DTW(\textbf{s},\textbf{c}):\\
\hspace*{3mm}$\boldsymbol{D}$(1:n+1,1:m+1) = inf;\\
\hspace*{3mm}$\boldsymbol{D}$(1,1) = 0;\\
\hspace*{3mm}for $i = 2$: $n+1$\\
\hspace*{6mm}for $j = 2$ : $m+1$\\
\hspace*{9mm}$d = (\textbf{s}(i-1)-\textbf{c}(j-1))^2$;\\
\hspace*{9mm}$\boldsymbol{D}(i,j) = d + \min( 
[ \boldsymbol{D}(i-1,j), \boldsymbol{D}(i,j-1), \boldsymbol{D}(i-1,j-1) ])$;\\
return\ sqrt$(\boldsymbol{D}(n+1,m+1))$\\
\hline
\end{tabular}
\end{center}}
\vspace*{-9pt}

\Caption{Алгоритм вычисления DTW для временных рядов
\label{ris:dtwts}}
%\end{figure*}
%\begin{figure*} %fig3
\vspace*{6pt}
{\small 
\begin{center}
\begin{tabular}{l}
\hline
\\[-9pt]
Correction $(\overline{i,j,k,l}, \boldsymbol{\pi}(\overline{i,j,k,l})):$\\
\hspace*{3mm}if $\overline{i,j,k,l} \in \{ (i-1, j, k,l)  ;  
(i, j, k-1, l)  ;  (i-1, j, k-1, l) \}$:\\
\hspace*{6mm}$ \widehat{\pi} = \{ (\overline{i}, r, \overline{k}, f) \in 
\boldsymbol{\pi}(\overline{i, j, k, l}) \vert r, f \in \mathbb{N} \}$\\
\hspace*{3mm}elif $\overline{i,j,k,l}\in \{  
(i, j-1, k, l); (i, j, k, l-1); (i, j-1, k, l-1) \}$:\\
\hspace*{6mm}$\widehat{\pi} = \{ (r, \overline{j}, f, \overline{l}) 
\in \boldsymbol{\pi}(\overline{i, j, k, l}) \vert r, f \in \mathbb{N} \}$\\
\hspace*{3mm}elif $\overline{i,j,k,l} =  i-1,j-1,k-1,l-1:$\\
\hspace*{6mm}$\widehat{\pi} = \{ (\overline{i}, r, \overline{k}, f) 
\in \boldsymbol{\pi}(\overline{i, j, k, l}) \vert r,f \in \mathbb{N} \} \cup$\\
\hspace*{6mm}$\cup \{ (r, \overline{j}, f, \overline{l}) \in \boldsymbol{\pi}
(\overline{i, j, k, l}) \vert r,f \in \mathbb{N} \}$\\
\hspace*{3mm}$\boldsymbol{d\pi} = \{ \mathrm{element} \in \widehat{\pi}: 
\mbox{произведены\ замены\ индексов } 
\overline{i} = i,\ \overline{j} = j,\ \overline{k} = k,\ \overline{l} = l \}$\\
return $\boldsymbol{d\pi}$\\
\hline
\end{tabular}
\end{center}
}
\vspace*{-9pt}

\Caption{Алгоритм вычисления поправки $\boldsymbol{d\pi}$ 
пути $\boldsymbol{\pi}$
\label{ris:codedpi}}
\end{figure*}


\textbf{Алгоритм вычисления значения расстояния~(4).}
Построение алгоритма вычисления значения функции расстояния 
между матрицами основан на алгоритме расчета функции расстояния 
между временн$\acute{\mbox{ы}}$ми рядами. В~случае выравнивания одной\linebreak\vspace*{-12pt}

{ \begin{center}  %fig1
 \vspace*{-3pt}
    \mbox{%
 \epsfxsize=79mm 
 \epsfbox{gon-1.eps}
 }


\end{center}


\noindent
{{\figurename~1}\ \ \small{Матрица стоимости оптимального выравнивания, по обеим 
осям отложены временные отсчеты}}
}

\vspace*{12pt}


\noindent 
временн$\acute{\mbox{о}}$й шкалы
 итоговая матрица расстояний~$\boldsymbol{D}$ (рис.~1) в~каждом 
 элементе~$\boldsymbol{D}(i,j)$ содержит рас\-сто\-яние между подрядом 
 первого временн$\acute{\mbox{о}}$го ряда и~подрядом второго временн$\acute{\mbox{о}}$го ряда. 
 Рас\-смот\-рим алгоритм динамического выравнивания двух временн$\acute{\mbox{ы}}$х 
 рядов $\textbf{s} \hm\in R^n$ и~$\textbf{c} \hm\in R^m$ на рис.~2.
 
 

Элемент $\boldsymbol{D}(i,j)$ матрицы~$\boldsymbol{D}$ соответствует 
стоимости выравнивающего пути между подпоследовательностями 
исходных временн$\acute{\mbox{ы}}$х рядов: $\textbf{s}(1:i) \hm= \textbf{s}(t)$, 
$t \hm= 1,\ldots,i,$ и~$\textbf{c}(1:j) \hm= \textbf{c}(t)$, $t \hm= 1,\ldots,j$. 
Алгоритм построения наилучшего выравнивания времени 
подразумевает, что выравнивающий путь между этими 
подпоследовательностями получен одним из трех способов~--- 
если стоимость выравнивающего пути между 
подпоследовательностями~$\textbf{s}(1:\overline{i}) $ 
и~$\textbf{c}(1:\overline{j})$ минимальна для~$\overline{i,j}$ из множества
$$
\overline{i,j} \in \left\{ \{i-1,j\},\{i,j-1\},\{i-1,j-1\} \right\},$$
тогда выравнивающий путь между $\textbf{s}(1:i)$ и~$\textbf{c}(1:j)$ получен добавлением пары~$(i,j)$ к~выбранному 
выравнивающему пути с~минимальной стоимостью из трех.



Предложенный алгоритм переносит эти рас\-суж\-де\-ния на случай 
выравнивания двух матриц~$\textbf{A}$ и~$\textbf{B}$. 
Элемент~$\boldsymbol{D}(i,j,k,l)$ четырехиндексного
 тензора расстояний~$\boldsymbol{D}$ соответствует стоимости выравнивающего 
 пути между $\textbf{A}(1:i,1:j) \hm= \textbf{A}(t_1,t_2)$, 
 $t_1 \hm= 1,\ldots, i$, $t_2 \hm= 1,\ldots, j,$ 
 и~$\textbf{B}(1:k,1:l) \hm= \textbf{B}(t_1,t_2)$, $t_1 \hm= 1,\ldots, k$,
 $t_2 \hm= 1,\ldots, l$. Выравнивающий путь между этими 
 подматрицами получен одним из семи способов~--- 
 если стоимость выравнивающего пути между 
 подматрицами $\textbf{A}(1:\overline{i},1:\overline{j})$ 
 и~$\textbf{B}(1:\overline{k},1:\overline{l})$ 
 минимальна для~$\overline{i,j,k,l}$ из множества
\begin{multline*} 
\overline{i,j,k,l} \in 
\left\{ \{i-1,j,k,l\},\{i,j-1,k,l\},\right.\\
\{i,j,k-1,l\},
\{i,j,k,l-1\}, \{i-1,j,k-1,l\},\\
\left.
\{i,j-1,k,l-1\},\{i-1,j-1,k-1,l-1\}\right\},
\end{multline*}

\setcounter{figure}{3}
\begin{figure*} %fig4
{\small 
\begin{center}
\begin{tabular}{l}
\hline
$\mathrm{mDTW}\left(\textbf{A},\textbf{B}\right):$\\
\hspace*{3mm}$\textbf{D}(1:n+1,1:n+1, 1:n+1, 1:n+1) = inf$;\\
\hspace*{3mm}$\textbf{D}(1,1,1,1) = 0;$\\
\hspace*{3mm}$\boldsymbol{\pi}(1,1,1,1) = ((1,1),(1,1))$\\
\hspace*{3mm}$for\ i,j,k,l  \in \mathbb{N}^{2 : n+1} \times 
\mathbb{N}^{2 : n+1} \times \mathbb{N}^{2 : n+1} \times \mathbb{N}^{2 : n+1}:$\\
\hspace*{6mm}$\overline{i,j,k,l} = \argmin($ [ \textbf{D}(i-1, j, k, l), 
\textbf{D}(i, j-1, k, l), \textbf{D}(i, j, k-1, l), 
\textbf{D}(i, j, k, l-1),    \\
\hspace*{9mm}$\textbf{D}(i-1, j, k-1, l), \textbf{D}(i, j-1, k, l-1), 
\textbf{D}(i-1, j-1, k-1, l-1) ])$;\\
\hspace*{3mm}$\boldsymbol{d \pi} = \mathrm{Correction}\,(\overline{i,j,k,l}, 
\boldsymbol{\pi}(\overline{i,j,k,l}))$\\
\hspace*{3mm}$\boldsymbol{\pi}(i, j, k, l) = \boldsymbol{d \pi} \cup 
\{(\overline{i,j,k,l})\}$\\
\hspace*{3mm}$\mathrm{cost} = (\textbf{A}(i, j)-\textbf{B}(k, l))^2 + 
\sum\nolimits_{(r,f,t,g) \in \boldsymbol{d \pi}}
(\textbf{A}(r, f)-\textbf{B}(t, g))^2$;\\
\hspace*{3mm}$\textbf{D}(i,j,k,l) = \mathrm{cost} + \textbf{D}
(\overline{i,j,k,l})$\\
return  sqrt$(\textbf{D}(n+1,n+1,n+1,n+1))$\\
\hline
\end{tabular}
\end{center}
}
\vspace*{-9pt}

\Caption{Алгоритм вычисления расстояния между матрицами
\label{ris:matrixdtw}}
\end{figure*}

\begin{table*}[b]\small
\begin{center}
\begin{tabular}{|l|c|c|c|c|}
\multicolumn{5}{c}{Снижение расстояний при выполнении преобразований 
для различных наборов данных}\\
\multicolumn{5}{c}{\ }\\[-6pt]
\hline
 &\multicolumn{4}{c|}{Метод}\\
 \cline{2-5}
\multicolumn{1}{|c|}{Данные}  & \multicolumn{2}{c|}{$L_2$} & \multicolumn{2}{c|}{MatrixDTW} \\
\cline{2-5}
& $S(f|p)$  &  $S_{\rho}(\mathfrak{D})$ &  $S(f|p)$ & $S_{\rho}(\mathfrak{D})$ \\
\hline
Модельные данные без преобразований& 92\% & 78\% & 100\%\hphantom{9} & 85\% \\
Модельные данные с~преобразованиями & 86\% & 65\% &  100\%\hphantom{9} & 82\% \\
Модельные данные с~преобразованиями и~шумом& 69\% & 61\% &  92\% & 78\% \\
MNIST без преобразований& 95\% & --- & 95\% & --- \\
MNIST с~преобразованиями & 53\% & --- & 92\% & --- \\
Спектр сигнала& 83\% & --- & 96\% & --- \\
\hline
\end{tabular}
\end{center}
\end{table*}

\noindent
то к~выравнивающему пути между этими под\-мат\-ри\-ца\-ми 
добавляется элемент пути $(i,j,k,l)$ и~поправка~$\boldsymbol{d\pi} $ 
пути~$\boldsymbol{\pi}$, алгоритм вычисления которой приведен ниже.

Обозначим выравнивающий путь между $\textbf{A}(1:i,\linebreak 1:j)$
 и~$\textbf{B}(1:k,1:l)$ как~$\boldsymbol{\pi}(i,j,k,l)$, тогда 
 поправка~$\boldsymbol{d\pi} $ пути~$\boldsymbol{\pi}(i,j,k,l)$ 
 при фиксированных~$\overline{i,j,k,l}$ вычисляется приведенным на рис.~3 
 образом.





Алгоритм динамического выравнивания двух матриц и~вычисления 
расстояния $\mathrm{mDTW}$ между ними с~учетом приведенного выше 
алгоритма примет вид, представленный на рис.~4.





\begin{figure*} %fig5
\vspace*{1pt}
    \begin{center}  
  \mbox{%
 \epsfxsize=161.412mm 
 \epsfbox{gon-5.eps}
 }
\end{center}
\vspace*{-12.5pt}
\Caption{Выравнивание модельных данных: (\textit{а})~один класс без шума; 
(\textit{б})~разные классы без шума; 
(\textit{в})~один класс с~шумом; (\textit{г})~разные классы с~шумом
\label{ris:random}}
%\end{figure*}
%\begin{figure*} %fig6
\vspace*{1pt}
    \begin{center}  
  \mbox{%
 \epsfxsize=163mm 
 \epsfbox{gon-6.eps}
 }
\end{center}
\vspace*{-12.5pt}
\Caption{Выравнивание данных MNIST: левый столбец~--- один класс; 
правый столбец~--- разные 
классы;
(\textit{а})~$\mathrm{mDTW}\hm=720{,}1$; 
(\textit{б})~948,6;
(\textit{в})~2017,0;
(\textit{г})~$\mathrm{mDTW}\hm=2071{,}4$
\label{ris:mnist}}
\end{figure*}


Следует отметить, что алгоритм~\cite{15} имеет\linebreak высокую сложность 
вычисления~--- $O(n^4)$. Предполагается ускорение метода 
с~использованием ограниче\-ния Sakoe-Chiba band, что сократит 
вычислительную сложность алгоритма до $O(n^2k^2)$, где~$k$~--- 
параметр ограничения.


\section{Вычислительный эксперимент}

Вычислительный эксперимент проведен на модельных данных с~допустимыми 
преобразованиями и~на реальных данных: объектах коллекции MNIST с~допустимыми 
преобразованиями и~на спектрограммах зашумленных сигналов.





Решается задача метрической классификации методом ближайшего соседа. В~таблице 
приведены значения критерия качества функции расстояния 
$S_{\rho}(\mathfrak{D})$ и~критерия качества метрической классификации $S(f|p)$ 
при использовании двух функций расстояния: предложенной в~работе $\mathrm{mDTW}$ 
и~$L_2$.

Модельные данные~--- это нулевые матрицы со случайными ненулевыми 
строками, столбцами, подпрямоугольниками с~наложенным шумом. 
К~ним применены допустимые преобразования, согласованные с~гипотезой 
наличия локальных и~глобальных искажений. На рис.~\ref{ris:random} 
показан пример оптимального выравнивания двух объектов. 
Линиями показаны элементы пути~$\boldsymbol{\pi}$.

Подготовлена подвыборка набора данных MNIST. Она 
состоит из~100 объектов классов 0 и~1 сниженной размерности
 с~допустимыми преобразованиями. На рис.~\ref{ris:mnist} 
 показан пример оптимального выравнивания объектов.


Аналогичный эксперимент проведен для решения задачи метрической 
классификации спектров различных сигналов, пример которых приведен на 
рис.~\ref{ris:spectr}. На рисунке показаны примеры Фурье-спект\-ров 
этих сигналов. Спектр получен путем применения быстрого преобразования 
Фурье к~исходному сигналу для различных окон с~фиксированным размером и~сдвигом. 
Исходные временн$\acute{\mbox{ы}}$е ряды обладали свойством периодичности, период выбирался 
случайным образом.



Тестирование проведено на разного рода данных: исходных 
модельных данных без наложения\linebreak\vspace*{-12pt}

\pagebreak

\end{multicols}

\begin{figure*} %fig7
\vspace*{1pt}
    \begin{center}  
  \mbox{%
 \epsfxsize=149.062mm 
 \epsfbox{gon-7.eps}
 }
\end{center}
\vspace*{-8pt}
\Caption{Данные спектров сигнала: (\textit{а})~класс~1; (\textit{б})~спектр 
класса~1; (\textit{в})~класс~2; (\textit{г})~спектр класса~2; 
(\textit{д})~класс~3; (\textit{е})~спектр класса~3
\label{ris:spectr}}
\vspace*{9pt}
\end{figure*}

\begin{multicols}{2}

\noindent допустимых преобразований, с~ними, а~также 
на модельных данных с~наложенным поверх объектов случайным шумом.



В каждом из проведенных экспериментов была продемонстрирована 
устойчивость предложенного подхода к~допустимым преобразованиям. 
Наилучшее значение критерия качества задачи классификации было 
достигнуто при использовании предложенной функции расстояния.

\vspace*{-5pt}

\section{Заключение}

В работе предложено обобщение метода динамического выравнивания
 временн$\acute{\mbox{ы}}$х рядов для случая объектов, определенных на двух осях времени. 
 Существует теоретическое обобщение предлагаемых методов на случай 
 конечного множества осей времени. Вычислительный эксперимент позволил 
 проанализировать свойства подхода: устойчивость к~допустимым 
 преобразованиям и~разделяющая способность функции расстояния как 
 на реальных, так и~на модельных данных. Качество решения задачи 
 метрической классификации выше решения, основанного на евклидовом 
 расстоянии. Вычислительная сложность метода высокая, что ограничивает 
 его применимость на объектах высокой размерности.

\vspace*{-2pt}

{\small\frenchspacing
 {%\baselineskip=10.8pt
 \addcontentsline{toc}{section}{References}
 \begin{thebibliography}{99}
%\bibitem{Karasikov2016}
%\Au{Карасиков~М.\,Е., Стрижов~В.\,В.} Классификация временных рядов 
%в~пространстве параметров по\-рож\-да\-ющих моделей~// Информатика и~её 
%применения,~2016. T.~10. Вып.~4. С.~121--131.

\bibitem{0}
\Au{Hill~N.\,J., Lal~T.\,N., Schroder~M., Hinterberger~T., 
Wilhelm~B., Nijboer~F., Mochty~U., Widman~G., Elger~C., 
Scholkopf~B., Kubler~A., Birbaumer~N.} Classifying EEG and 
ECoG signals without subject training for fast BCI implementation: 
Comparison of nonparalyzed and completely paralyzed subjects~//  
IEEE~T. Neur. Sys. Reh., 2006. Vol.~14. 
Iss.~2. P.~183--186.

\bibitem{1}
\Au{Sakoe~H., Chiba~S.} 
A~dynamic programming approach to continuous speech recognition~// 
7th  Congress (International) on Acoustics Proceedings, 1971. Vol.~3. P.~65--69.

\bibitem{2} %3
\Au{Aghabozorgi~S., Ali~S.\,S., Wah~T.\,Y.} 
Time-series clustering~--- a~decade review~// Inform. Syst., 
2015. Vol.~53. P.~16--38.

\bibitem{3} %4
\Au{Warrenliao~T.} Clustering of time series data~--- a~survey~// 
Pattern Recogn., 2005. Vol.~38. Iss.~11. P.~1857--1874.



\bibitem{4} %5
\Au{Hautamaki~V., Nykanen~P., Franti~P.} 
Time-series clustering by approximate prototypes~// 
19th  Conference (International) on Pattern Recognition Proceedings, 2008. No.\,D. 
P.~1--4.

\bibitem{5} %6
\Au{Faloutsos~C., Ranganathan~M., Manolopoulos~Y.} 
Fast subsequence matching in time-series databases~// \mbox{SIGMOD} Rec., 1994. 
Vol.~23. Iss.~2. P.~419--429.

\bibitem{10} %7
\Au{Basalto~N., Bellotti~R., Carlo~F.\,D., Facchi~P., 
Pascazio~S.} Hausdorff clustering of financial time series~// 
Physica~A, 2007. Vol.~379. Iss.~2. P.~635--644.

\bibitem{11} %8
\Au{Gorelick~L., Blank~M., Shechtman~E., Irani~M., Basri~R.} 
Actions as space-time shapes~// IEEE~T. Pattern Anal., 
2007. Vol.~29. Iss.~12. P.~2247--2253.

\bibitem{6} %9
\Au{Smyth~P.} Clustering sequences with hidden Markov models~// 
Adv. Neural In., 1997. Vol.~9. P.~648--654.

\bibitem{7} %10
\Au{Banerjee~A., Ghosh~J.} Clickstream clustering using weighted 
longest common subsequences~// 
Workshop on Web Mining, SIAM Conference on Data Mining
Proceedings, 2001. P.~33--40.

\bibitem{8} %11
\Au{Aach~J., Church~G.M.} Aligning gene expression time series
 with time warping algorithms~// Bioinformatics, 2001. Vol.~17. Iss.~6. P.~495--508.

\bibitem{9} %12
\Au{Yi~B.\,K., Faloutsos~C.} Fast time sequence indexing 
for arbitrary $\mathcal{L}_p$ norms~// 
26th  Conference (International) on Very Large Data Bases Proceedings, 2000. P.~385--394.

\bibitem{33} %13
\Au{Goncharov~A.\,V., Strijov~V.\,V.} 
Analysis of dissimilarity set between time series~// Computational 
Mathematics Modeling, 2018. Vol.~29. Iss.~3. P.~359--366.

\bibitem{12} %14
\Au{Alon~J., Athitsos~V., Sclaroff~S.}
 Online and offline character recognition using alignment to prototypes~// 
 8th  Conference (International) on Document Analysis and Recognition, 2005. 
 Vol.~2. P.~839--843.

\bibitem{15} %15
\Au{Гончаров~А.\,В.} 
Выравнивания декартовых произведений упорядоченных множеств mDTW. 
Про\-грам\-мная реализация алгоритма, 2019. 
{\sf https://github.
com/Intelligent-Systems-Phystech/PhDThesis/tree/\linebreak  master/Goncharov2019/MatrixDTW/code}.
 \end{thebibliography}

 }
 }

\end{multicols}

\vspace*{-9pt}

\hfill{\small\textit{Поступила в~редакцию 24.04.19}}

\vspace*{6pt}

%\pagebreak

%\newpage

%\vspace*{-28pt}

\hrule

\vspace*{2pt}

\hrule

\vspace*{-4pt}

\def\tit{ALIGNMENT OF~ORDERED SET CARTESIAN PRODUCT\\[-5pt]}


\def\titkol{Alignment of~ordered set cartesian product}

\def\aut{A.\,V.~Goncharov$^1$ and~V.\,V.~Strijov$^{1,2}$}

\def\autkol{A.\,V.~Goncharov and~V.\,V.~Strijov}

\titel{\tit}{\aut}{\autkol}{\titkol}

\vspace*{-13pt}


\noindent
$^1$ Moscow Institute of Physics and Technology, 
9~Institutskiy Per., Dolgoprudny, Moscow Region 141700, Russian\linebreak
$\hphantom{^1}$Federation


\noindent
$^2$A.\,A.~Dorodnicyn Computing Center, Federal Research Center 
``Computer Science and Control'' of the Russian\linebreak
$\hphantom{^1}$Academy of Sciences, 
40~Vavilov Str., Moscow 119333, Russian Federation

\def\leftfootline{\small{\textbf{\thepage}
\hfill INFORMATIKA I EE PRIMENENIYA~--- INFORMATICS AND
APPLICATIONS\ \ \ 2020\ \ \ volume~14\ \ \ issue\ 1}
}%
 \def\rightfootline{\small{INFORMATIKA I EE PRIMENENIYA~---
INFORMATICS AND APPLICATIONS\ \ \ 2020\ \ \ volume~14\ \ \ issue\ 1
\hfill \textbf{\thepage}}}

\vspace*{2pt} 



\Abste{The work is devoted to the study of metric methods for analyzing 
objects with complex structure. It proposes to generalize the dynamic 
time warping method of two time series for the case of objects defined 
on two or more time axes. Such objects are matrices in the discrete 
representation. The DTW (Dynamic Time Warping) method of time series is generalized as 
a~method of matrices dynamic alignment. The paper proposes 
a~distance function resistant to monotonic nonlinear deformations of the 
Cartesian product of two time scales. The alignment path between objects is 
defined. An object is called a~matrix in which the rows and columns correspond 
to the axes of time. The properties of the proposed distance function 
are investigated. To illustrate the method, the problems of metric 
classification of objects are solved on model data and data from the 
MNIST dataset.}

\KWE{distance function; dynamic alignment; distance between matrices; 
nonlinear time warping; space--time series}



\DOI{10.14357/19922264200105} 

%\vspace*{-14pt}

\Ack
\noindent
This work was supported by the Russian Foundation for Basic
Research (projects 19-07-1155 and 19-07-00885). 
The paper contains results of the project Statistical 
methods of machine learning, which is carried out within the 
framework of the Program ``Center of Big Data Storage and Analysis'' 
of the National Technology Initiative Competence Center. 
It is supported by the Ministry of Science and Higher Education 
of the Russian Federation according to the agreement between the
 M.\,V.~Lomonosov Moscow State University and the Foundation 
 of project support of the National Technology Initiative from 11.12.2018, 
 No.\,13/1251/2018.
 


%\vspace*{6pt}

  \begin{multicols}{2}

\renewcommand{\bibname}{\protect\rmfamily References}
%\renewcommand{\bibname}{\large\protect\rm References}

{\small\frenchspacing
 {%\baselineskip=10.8pt
 \addcontentsline{toc}{section}{References}
 \begin{thebibliography}{99}

 \bibitem{0-1}   
\Aue{Hill, N.\,J., T.\,N.~Lal, M.~Schroder, T.~Hinterberger, B.~Wilhelm, 
F.~Nijboer, U.~Mochty, G.~Widman, C.~Elger, B.~Scholkopf, A.~Kubler, and 
N.~Birbaumer.} 2006. Classifying EEG and ECoG signals without subject 
training for fast BCI implementation: Comparison of nonparalyzed and completely 
paralyzed subjects. \textit{IEEE~T. Neur. Sys. 
Reh.} 14(2):183--186.

\bibitem{1-1}   
\Aue{Sakoe, H., and S.~Chiba.} 1971. A~dynamic programming approach 
to continuous speech recognition. \textit{7th 
 Congress (International) on Acoustics Proceedings}. 3:65--69.

\bibitem{2-1}    %2
\Aue{Aghabozorgi,~S., S.\,S.~Ali, and T.\,Y.~Wah.} 2015. 
Time-series clustering~--- a~decade review.  \textit{Inform. Syst.} 
53:16--38.

\bibitem{3-1}   %4 
\Aue{Warrenliao,~T.} 2005. Clustering of time series data~--- a~survey. 
\textit{Pattern Recogn.}
38(11):1857--1874.



\bibitem{4-1}    %5
\Aue{Hautamaki,~V., P.~Nykanen, and P.~Franti.} 2008. 
Time-series clustering by approximate prototypes. 
 \textit{19th  Conference (International) on Pattern Recognition Proceedings}. 
 D:1--4.

\bibitem{5-1}    %6
\Aue{Faloutsos,~C., M.~Ranganathan, and Y.~Manolopoulos.} 1994. 
Fast subsequence matching in time-series databases.  \textit{SIGMOD Rec}. 
23(2):419--429.

\bibitem{10-1}    %7
\Aue{Basalto, N., R.~Bellotti, F.\,D.~Carlo, P.~Facchi, and S.~Pascazio.} 
2007. Hausdorff clustering of financial time series. 
\textit{Physica~A} 379(2):635--644.

\bibitem{11-1}   %8
\Aue{Gorelick, L., M.~Blank, E.~Shechtman, M.~Irani, and R.~Basri.} 
2007. Actions as space-time shapes.
\textit{IEEE~T. Pattern Anal.} 29(12):2247--2253.

\bibitem{6-1}    %9
\Aue{Smyth, P.} 1997. 
Clustering sequences with hidden Markov models. \textit{Adv. Neural In.} 9:648--654.

\bibitem{7-1}    %10
\Aue{Banerjee,~A., and J.~Ghosh.} 2001. 
Clickstream clustering using weighted longest common subsequences.  
\textit{Workshop on Web Mining, SIAM Conference 
on Data Mining Proceedings.} 33--40.

\bibitem{8-1}    %11
\Aue{Aach, J., and G.\,M.~Church.} 2001. 
Aligning gene expression time series with time warping algorithms. 
\textit{Bioinformatics} 17(6):495--508.

\bibitem{9-1}   %12
\Aue{Yi, B.\,K., and C.~Faloutsos.} 2000. 
Fast time sequence indexing for arbitrary $\mathcal{L}_p$ norms. 
\textit{26th  Conference (International) 
on Very Large Data Bases Proceedings}. 385--394.

\bibitem{33-1}   %13 
\Aue{Goncharov,~A.\,V., and V.\,V.~Strijov.} 2018. 
Analysis of dissimilarity set between time series. 
\textit{Computational Mathematics Modeling } 29(3):359--366.



\bibitem{12-1}    %14
\Aue{Alon, J., V.~Athitsos, and S.~Sclaroff.} 2005.
 Online and offline character recognition using alignment to prototypes. 
 \textit{8th  Conference (International) on Document Analysis and Recognition}. 
 2:839--843.

\bibitem{15-1}    %15
\Aue{Goncharov, A.\,V.} Alignment of 
Ordered Set Cartesian Product mDTW. Software implementation of the algorithm. 
Available at: {\sf https://github.com/Intelligent-\linebreak 
Systems-Phystech/PhDThesis/tree/master/Goncharov\linebreak 2019/MatrixDTW/code} 
(accessed December~27, 2019).
\end{thebibliography}

 }
 }

\end{multicols}

%\vspace*{-7pt}

\hfill{\small\textit{Received April 24, 2019}}

%\pagebreak

%\vspace*{-22pt}



\Contr

\noindent
\textbf{Goncharov Alexey V.} (b.\ 1995)~--- 
PhD student, Moscow Institute of Physics and Technology, 
9~Institutskiy Per., Dolgoprudny, Moscow Region 141701, 
Russian Federation; \mbox{alex.goncharov@phystech.edu}

\vspace*{3pt}

\noindent
\textbf{Strijov Vadim V.} (b.\ 1967)~--- 
Doctor of Science in physics and mathematics, leading scientist, 
A.\,A.~Dorodnicyn Computing Centre, Federal Research Center 
``Computer Science and Control'' of the Russian Academy of Sciences, 
40~Vavilov Str., Moscow 119333, Russian Federation;
 professor, Moscow Institute of Physics and Technology, 
 9~Institutskiy Per., Dolgoprudny, Moscow Region 141701, Russian Federation; 
 \mbox{strijov@ccas.ru}
\label{end\stat}

\renewcommand{\bibname}{\protect\rm Литература}