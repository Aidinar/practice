\def\stat{br-stup}

\def\tit{НЕЙРОФИЗИОЛОГИЯ КАК~ПРЕДМЕТНАЯ ОБЛАСТЬ ДЛЯ~РЕШЕНИЯ ЗАДАЧ 
С~ИНТЕНСИВНЫМ\\ ИСПОЛЬЗОВАНИЕМ ДАННЫХ$^*$}

\def\titkol{Нейрофизиология как предметная область для~решения задач 
с~интенсивным использованием данных}

\def\aut{Д.\,О.~Брюхов$^1$, С.\,А.~Ступников$^2$, Д.\,Ю.~Ковалёв$^3$, 
И.\,А.~Шанин$^4$}

\def\autkol{Д.\,О.~Брюхов, С.\,А.~Ступников, Д.\,Ю.~Ковалёв, 
И.\,А.~Шанин}

\titel{\tit}{\aut}{\autkol}{\titkol}

\index{Брюхов Д.\,О.}
\index{Ступников С.\,А.}
\index{Ковалёв Д.\,Ю.} 
\index{Шанин И.\,А.}
\index{Briukhov D.\,O.}
\index{Stupnikov S.\,A.}
\index{Kovalev D.\,Yu.}
\index{Shanin I.\,A.}


{\renewcommand{\thefootnote}{\fnsymbol{footnote}} \footnotetext[1]
{Работа выполнена при частичной финансовой поддержке РФФИ (проект 18-29-22096).}}


\renewcommand{\thefootnote}{\arabic{footnote}}
\footnotetext[1]{Институт проблем информатики Федерального исследовательского центра <<Информатика и~управлени>> 
Российской академии наук, \mbox{dbriukhov@ipiran.ru}}
\footnotetext[2]{Институт проблем информатики Федерального исследовательского центра <<Информатика и~управление>> 
Российской академии наук, \mbox{sstupnikov@ipiran.ru}}
\footnotetext[3]{Институт проблем информатики Федерального исследовательского центра <<Информатика и~управление>> 
Российской академии наук, \mbox{dm.kovalev@gmail.com}}
\footnotetext[4]{Институт проблем информатики Федерального исследовательского центра 
<<Информатика и~управление>> Российской академии наук, \mbox{ivan.shanin@gmail.com}}

%\vspace*{-12pt}


  

\Abst{Цель данного обзора~--- анализ нейрофизиологии как области с~интенсивным 
использованием данных. В~настоящее время происходит заметный рост числа 
исследований в~области изучения человеческого мозга. Появляются крупные 
международные проекты, поддерживающие исследования, направленные на улучшение 
понимания работы человеческого мозга, а также на обнаружение и~поиск способов 
лечения основных заболеваний, связанных с~человеческим мозгом. Объем данных, 
генерируемых в~типичной лаборатории, проводящей исследования в~области 
нейрофизиологии, растет в~геометрической прогрессии. При этом данные представляются с~использованием большого числа разнообразных форматов. Это приводит 
к~необходимости создания инфраструктур и~баз данных, а также веб-сай\-тов, 
предоставляющих единый доступ к~данным и~обеспечивающим обмен этими данными 
между исследователями по всему миру. Для анализа собранных данных применяются 
методы и~средства из области нейроинформатики~--- науки на стыке нейрофизиологии 
и~информатики. Для решения нейрофизиологических задач применяются различные 
методы информатики, такие как статистический анализ и~машинное обучение, в~частности 
нейронные сети.}

\KW{нейрофизиология; нейроинформатика; интенсивное использование данных; анализ 
данных}

\DOI{10.14357/19922264200106} 
  
\vspace*{6pt}


\vskip 10pt plus 9pt minus 6pt

\thispagestyle{headings}

\begin{multicols}{2}

\label{st\stat}

\section{Введение}

%\vspace*{-3pt}

    Нейрофизиология~--- один из ярких примеров научной области 
с~интенсивным использованием данных. Она представляет собой 
комбинацию различных областей знаний: анатомии, физиологии, генетики, 
биохимии, психологии~--- и~стала передовой областью в~исследовании 
и~моделировании работы человеческого мозга.
    
    В настоящее время во всем мире растет интерес к~научному пониманию 
работы человеческого мозга, выражающийся в~количестве исследований 
в~области нейрофизиологии. Появляется новое, более качественное 
оборудование, поз\-во\-ля\-ющее получать более точные данные различного вида, 
в~част\-ности данные маг\-нит\-но-ре\-зо\-нанс\-ной томографии (МРТ), 
электроэнцефалографии (ЭЭГ), магнитоэнцефалографии (МЭГ) и~др. Новое 
оборудование позволяет за несколько дней собирать больше данных, чем 
всего десять лет назад собиралось за целый год.
    
    С увеличением объема данных встает проблема совместного 
использования этих данных для решения разнообразных задач в~области 
нейрофизиологии.
Стали появляться как региональные консорциумы 
и~проекты, поддерживающие\linebreak исследователей для решения различных задач 
в~об\-ласти нейрофизиологии (например, американская инициатива BRAIN
(The Brain Research through Advancing Innovative 
Neurotechnologies$^\registered$ Initiative), 
европейские проекты HBP (Human Brain Project)
и~BNCI (Brain-Neural-Computer-Interaction)
из программы Horizon 2020), так 
и~консорциумы, объединяющие исследователей во всем мире для решения 
конкретных задач, такие как проект исследования коннектома человека HCP
(Human Connectome Project), 
инициатива по нейровизуализации болезни Альцгеймера ADNI
(Alzheimer's Disease Neuroimaging 
Initiative), инициатива 
по развитию маркеров для болезни Паркинсона PPMI
(Parkinson's Progression Markers Initiative).
{\looseness=1

} 
    
    С ростом объема данных растет и~разнообразие этих данных. 
К~сожалению, в~об\-ласти нейрофизиологии в~настоящее время нет единых 
стандартов для пред\-став\-ле\-ния данных. Это относится практически ко всем 
видам данных, в~част\-ности к~нейроизображениям и~биомедицинским 
сигналам. Разнообразие форматов данных вызвано большим чис\-лом видов 
медицинского оборудования, а~также средств визуализации и~анализа 
получаемых данных.
{\looseness=1

}
    
    С ростом объема и~разнообразия данных продолжает усиливаться 
стремление разместить их в~доступных репозиториях (базах данных, веб-сай\-тах). 
Такие репозитории могут содержать петабайты 
нейрофизиологических данных и~позволяют обмениваться ими 
исследователям по всему миру. Некоторые базы содержат данные для 
решения конкретного класса задач, другие~--- широкий набор различных 
данных. При добавлении новых данных в~эти базы данные обычно проходят 
процесс рецензирования. 

Базы данных могут содержать данные как  
в~ка\-ком-то определенном формате, так и~поддерживать несколько разных 
форматов, принятых в~сообществе. Базы данных и~веб-сай\-ты предоставляют 
единый интерфейс доступа к~зарегистрированным в~них данным. Некоторые 
сайты предоставляют также программные средства для визуализации 
содержащихся в~них данных. 
{ %\looseness=1

}
    
    За последние годы созданы десятки про\-грам\-мных средств для сбора, 
обработки, анализа и~визуализации данных в~области нейрофизиологии, 
основанные на методах и~средствах из области информатики и~применяющих 
методы моделирования из области нейрофизиологии. Таким образом,\linebreak 
формируется нейроинформатика как меж\-дис\-цип\-ли\-нар\-ная об\-ласть 
сотрудничества ис\-сле\-до\-ва\-те\-лей-ней\-ро\-фи\-зио\-ло\-гов  
с~ис\-сле\-до\-ва\-те\-ля\-ми-ин\-фор\-ма\-ти\-ками.
    
    В рамках статьи предоставлена информация о~текущем состоянии дел 
в~об\-ласти нейрофизиологии (при этом основное внимание уделяется 
моделированию когнитивных функций на основе нейрофизиологических 
данных): основные мировые стратегические инициативы и~проекты (разд.~2), 
крупные базы данных, содержащие данные исследований (разд.~3), основные 
форматы представления нейроизображений и~биомедицинских сигналов 
(разд.~4), программные средства для обработки и~анализа нейроизображений 
(разд.~5). 

\section{Крупные международные консорциумы и~проекты 
в~области нейрофизиологии}

    С увеличением объема данных и~числа исследований в~области 
нейрофизиологии встает задача компьютерной поддержки этих исследований и~совместного использования полученных данных. Появляются как 
региональные консорциумы и~проекты, поддерживающие исследователей 
для решения различных задач в~области нейрофизиологии, так 
и~консорциумы, объединяющие исследователей во всем мире для решения 
конкретных задач или для лечения различных заболеваний, связанных 
с~мозгом.
    
    \textit{Инициатива исследования мозга с~помощью продвинутых 
инновационных технологий} (BRAIN)~[1] была объявлена 
в~США в~2013~г.\ и~представляет собой 10-лет\-нюю программу, 
направленную на революцию в~понимании работы человеческого мозга. 
    
    Начатый в~2013~г.\ проект \textit{Human Brain Project}~[2]~--- 
это десятилетний проект поддержки исследований человеческого мозга, 
курируемый Европейским Союзом (ЕС). Цель проекта~--- создание 
современной исследовательской инфраструктуры, которая позволит 
исследователям расширять знания в~понимании работы человеческого мозга. 
    
    Стартовавший в~2010~г.\ проект \textit{Human Connectome Project}~[3] 
    является попыткой картирования нервных путей, лежащих 
в~основе функционирования человеческого мозга. Цель проекта~--- сбор 
и~обмен данными о структурной и~функциональной связанности 
человеческого мозга (коннектома) в~макромасштабе (в~сантиметровом 
и~миллиметровом масштабе). 
    
    Проект \textit{BNCI Horizon 2020}~[4] в~рамках \mbox{7-й} рамочной программы 
ЕС направлен на поддержку и~координацию усилий в~области интерфейсов 
мозг--компьютер (BCI, Brain--Computer Interface) и~нейроинтерфейсов 
мозг--компьютер (\mbox{BNCI}). 
Основная цель этого проекта~--- разработка дорожной карты для области 
BCI с~особым упором на промышленные приложения BCI и~конечных 
пользователей. Этот проект объединяет~12~европейских университетов. 
    
    Потребность в~использовании данных различных дисциплин для 
исследования процессов и~способов лечения основных заболеваний была 
признана несколько лет назад~[5]. Также была осознана\linebreak необходимость 
сотрудничества между центрами и~дисциплинами для интеграции 
и~совместного использования разнообразных данных~[6] путем организации 
междисциплинарных консорциумов.
    
    Примером такого консорциума может служить \textit{инициатива по 
нейровизуализации болезни Альцгеймера} (ADNI)~[7], объединяющая исследователей с~данными 
исследований для улучшения профилактики и~лечения болезни Альцгеймера. 
Основные цели инициативы: выявление болезни на ранней стадии 
и~определение способа отслеживания болезни с~помощью биомаркеров, 
применение методов ранней диагностики (когда вмешательство может быть 
наиболее эффективным), предостав\-ле\-ние данных исследований для ученых 
всего \mbox{мира}. 
{\looseness=1

}
    
    Другими примерами междисциплинарных консорциумов, 
использующих обработку ней\-ро\-изоб\-ра\-же\-ний, являются инициативы, 
направленные на лечения таких заболеваний, как болезнь Паркинсона (PPMI), 
психиатрические 
расстройства. Поддерживаются базы данных для сбора нейровизуальных, 
генетических и~феноменальных данных об аутизме (National Database of 
Autism Research) и~повреждениях головного мозга (Federal Interacgency 
Traumatic Brain Injury Research).
    
\section{Инфраструктуры и~базы данных в~области 
нейрофизиологии}

    С целью дальнейшего использования данных, полученных 
исследователями со всего мира, создаются и~поддерживаются 
инфраструктуры доступа к~данным и~отдельные базы данных, объеди\-ня\-ющие 
данные от различных исследовательских групп и~пред\-остав\-ля\-ющие единый 
интерфейс доступа к~этим данным. Инфраструктуры пред\-остав\-ля\-ют единую 
среду для доступа к~различным данным и~использования различных 
программных средств для обработки этих данных. Ниже рассмотрены 
основные современные инфраструктуры и~базы данных.

\medskip
    
    Проект \textit{1000 функциональных коннектомов} (1000 Functional 
Connectomes project)~\cite{8-bs} предоставляет доступ  
к~фМРТ-изоб\-ра\-же\-ни\-ям со всего мира.\linebreak Проект содержит данные 
о~более~1200~наборах фМРТ-изображений состояния покоя, собранных 
с~33~разных сайтов. Проект содержит как сырые, так и~пред\-об\-ра\-бо\-тан\-ные 
данные, представленные в~формате BIDS  (brain imaging data structure).

\medskip
    
    \textit{OpenNEURO}~\cite{9-bs}~--- бесплатная и~открытая платформа 
для обмена данными МРТ, МЭГ и~ЭЭГ. Она является развитием проекта по 
созданию базы данных \textit{OpenfMRI}, законченного в~2010~г. 
Первоначально база данных включала только наборы данных, содержащих 
фМРТ-дей\-ст\-вия (task based \mbox{fMRI}). В~настоящее время она открыта для 
любых видов  
МРТ-ней\-ро\-изоб\-ра\-же\-ний. Все изображения, хранящиеся в~базе 
данных, представлены в~формате BIDS.
    
    \textit{База данных ConnectomeDB}~\cite{10-bs} была разработана 
в~рамках проекта HCP~[3] и~содержит данные 
о~структурной и~функциональной связанности человеческого мозга 
(коннектома). База данных в~настоящее время включает в~себя несколько 
видов данных МРТ, ЭЭГ и~МЭГ. Изображения, хранящиеся в~ConnectomeDB, 
представлены в~формате NIFTI. Для обработки данных проекта был создан 
\textit{Connectome Workbench}~--- свободно предоставляемый инструмент 
для визуализации и~анализа данных, полученных в~рамках проекта HCP.
    
    \textit{XNAT}~\cite{11-bs}~--- это открытая информационная платформа 
для работы с~нейроизображениями, разработанная исследовательской 
группой по нейроинформатике в~Вашингтонском университете. Она 
облегчает общие задачи управления, обеспечения производительности 
и~качества обработки нейроизображений и~связанных данных. \textit{XNAT 
Central} является общедоступным хранилищем медицинских изображений, 
основанным на открытой информационной платформе обработки 
изображений XNAT. В~отличие от большинства других хранилищ, таких как 
ConnectomeDB и~Open fMRI, XNAT Central не модерируется для контроля 
содержимого и~не предназначен для поддержки решения ка\-ких-ли\-бо 
конкретных научных задач и~подходов. Все изображения, хранящиеся 
в~XNAT Central, представлены в~формате DICOM.
    
    \textit{NITRC}~\cite{12-bs}~--- это бесплатный веб-ре\-сурс, который 
предлагает информацию о постоянно расширяющемся наборе программного 
обеспечения и~данных для нейроинформатики. Он состоит из трех 
компонентов: реестра ресурсов (NITRC-R), репозитория изображений 
(NITRC-IR) и~вычислительной среды (NITRC-CE). \textit{Вычислительная 
среда NITRC-CE} представляет собой виртуальную облачную платформу, 
содержащую предустановленный набор программных средств для работы 
с~нейроизображениями. \textit{Репозиторий изображений NITRC} включает 
в~себя изображения в~форматах DICOM и~NIFTI.
    
    \textit{База данных, разработанная в~рамках проекта BNCI Horizon 
2020}~\cite{4-bs}, является общедоступной коллекцией наборов данных 
в~области~BCI. Цель создания базы данных~--- 
повышение научной про\-зрач\-ности и~эф\-фек\-тив\-ности. База данных 
способствует также валидации опубликованных методов и~способствует 
разработке новых алгоритмов. Данные могут храниться в~различных 
форматах ЭЭГ-дан\-ных.
    
\section{Форматы данных в~нейрофизиологии}

    В области нейрофизиологии в~настоящее время нет единых стандартов 
для хранения данных~\cite{13-bs}. Это относится как к~нейроизображениям, 
так и~к~биомедицинским сигналам. Многообразие форматов представления 
данных вызвано разнообразием как медицинского оборудования, так 
и~средств визуализации и~анализа получаемых данных.

\vspace*{-4pt}
    
\subsection{Форматы магнитно-резонансной томографии}

%\vspace*{-4pt}

    Данные нейрофизиологических изображений должны содержать не 
только сами изображения, но и~дополнительную информацию (метаданные), 
обеспечивающую интероперабельность и~повторное использование этих 
данных. Изображения без связанных с~ними метаданных практически 
бесполезны. К~метаданным относятся информация об изображении (размер 
пикселя, ширина и~высота изображения, число изображений), информация об 
оборудовании, информация об объекте наблюдения, информация 
о~положении объекта наблюдения относительно оборудования.
    
    Наиболее распространенные форматы пред\-став\-ле\-ния 
нейроизображений~--- DICOM
(digital imaging and communications in medicine)~[14], 
используемый в~большинстве 
медицинских сканеров, и~ANALYZE~7.5~[15], разработанный в~клинике 
Mayo в~рамках создания пакета программ Analyze для хранения, 
визуализации и~обработки многомерных биомедицинских изображений. 
Среди других форматов данных в~нейрофизиологии можно отметить 
NIFTI~[16] и~BIDS~[17].
    
    Форматы определяют, как изображения и~метаданные хранятся в~файле. 
Названия конкретных метаданных в~каждом формате свои, и~их число 
варь\-и\-ру\-ет\-ся от сотни (\mbox{ANALYZE}, NIFTI) до нескольких тысяч (DICOM). 
В~ряде форматов (\mbox{ANALYZE}, \mbox{NIFTI}, \mbox{BIDS}) определяется неизменяемый 
список используемых метаданных, и~любой файл должен содержать значения 
всех этих метаданных, даже если они неизвестны. Другие форматы (DICOM) 
используют гибкий набор метаданных, когда конкретные метаданные 
присутствуют в~файле только в~том случае, если они определены.
    
    Нейрофизиологические изображения пред\-ставля\-ются в~виде 
трехмерного массива вокселей, описывающего положение вокселей 
в~трехмерном пространстве. Также может добавляться четвертое 
измерение~--- время. Каждый формат определяет свой способ представления 
этого массива в~файле в~виде одномерной последовательности вокселей. 
Форматы отличаются способом задания ориентации изображения 
относительно сканера: неявная фиксированная ориентация (\mbox{ANALYZE}), 
кватернионы (NIFTI) и~направляющие косинусы (\mbox{DICOM}, NIFTI). Для 
определения ориентации объекта наблюдения относительно сканера 
используются в~основном два подхода: нейрологический (NIFTI) 
и~радиологический (DICOM).

\vspace*{-4pt}
    
\subsection{Форматы электроэнцефалографии}

%\vspace*{-4pt}

    В области хранения биомедицинских сигналов существует множество 
разнообразных форматов~[18]. Форматы определяют, как метаданные 
(заголовки) и~данные хранятся в~файле. Заголовки файлов (метаданные) 
обычно хранятся в~бинарном виде (EDF (European data format)~[19], 
GDF (general data format)~[20]), но в~некоторых 
форматах они хранятся в~текстовом виде или в~виде XML (OpenXDF~[21]). 
    
    Некоторые биомедицинские данные могут содержать различные виды 
биомаркеров, для этого форматы (EDF, GDF, OpenXDF) должны 
поддерживать частоту дискретизации и~коэффициенты масштабирования. 
Первоначально форматы поддерживали хранение 8-бит\-ных данных, затем  
16-бит\-ных (EDF), а~все последние форматы поддерживают и~типы данных 
более 16~бит (GDF, OpenXDF). При хранении данных важно знать 
физическую единицу записанного сигнала, т.\,е.\ представляют ли значения 
выборки милливольт (мВ) или микровольт (мкВ). Большинство форматов 
поддерживают все физические единицы, представленные в~стандарте 
ISO11073:10101. Но некоторые старые форматы (EDF, GDF~1.0) отводят на 
это только~8~байт, чего недостаточно для хранения всех единиц.
    
    Биомедицинские сигналы зачастую содержат артефакты. Часть 
форматов (GDF, OpenXDF) позволяют задавать диапазон изменения значения 
единиц, что позволяет автоматически находить некорректные данные. Для 
анализа больших баз данных и~архивов важно иметь доступную информацию 
(поддерживаются в~форматах OpenXDF, GDF~2.1) о~демографии пациентов, 
записывающем оборудовании, исследователе и~т.\,д.

\vspace*{-4pt}
    
\section{Программные средства работы с~нейрофизиологическими 
данными}

\vspace*{-4pt}

    Программные средства работы с~нейроизображениями помогают 
исследователям в~изучении мозга человека. Они позволяют визуализировать 
данные в~виде трехмерных изображений, применять различные методы 
анализа данных. 
    
    \textit{Средства визуализации} позволяют визуализировать как 2D-, так 
и~3D- и~4D-ней\-ро\-изоб\-ра\-же\-ния (3D Slicer~[22], Mango~[23]). Кроме 
визуализации они также позволяют выполнять операции над изображениями, 
такие как ручная сегментация и~создание трехмерной модели поверхности 
(3D Slicer), создание и~редактирование областей интереса в~изображениях, 
рендеринг поверхности, наложение изоб\-ра\-же\-ний (Mango).
    
    \textit{Средства анализа нейроизображений} позволяют применять 
различные методы информатики для анализа нейроизображений. Написанное 
на \mbox{MATLAB} программное обеспечение \textit{CONN}~[24] предназначено 
для вычисления, визуализации и~ана\-ли\-за функциональных связностей 
в~\mbox{фМРТ}. Пред\-остав\-ля\-ют\-ся также функции обнаружения и~очистки 
артефактов, динамического анализа связности и~анализа на основе 
по\-верх\-ности и~\mbox{объема}. Сис\-те\-ма \textit{SPM}~[25] предназначена для 
статистического параметрического картирования, используемого для 
определения различий в~зарегистрированной активности мозга 
с~использованием пространственно расширенных статистических процессов.
    
    Набор инструментов для работы с~ЭЭГ \textit{NBT}~[26] обеспечивает 
расчет и~интеграцию ней\-ро\-фи\-зио\-логических биомаркеров. NBT предлагает 
конвей\-ер, включающий различные этапы обработки данных: от хранения 
данных до применения статистических методов, вычисление отклонения 
\mbox{артефактов}, визуализацию сигналов, вычисление биомаркеров 
и~статистическое тестирование.\linebreak Программные средства 
\textit{EEGLAB}~[27], \textit{FieldTrip}~[28] и~\textit{BioSig}~[29], 
реализованные в~MATLAB, предназначены для обработки биомедицинских 
сигналов, таких как ЭЭГ, МЭГ и~других электрофизиологических сигналов. 
EEGLAB реализует\linebreak метод независимых компонент,  
час\-тот\-но-вре\-мен\-ной анализ, вычисление отклонения артефактов 
и~несколько режимов визуализации данных. \mbox{FieldTrip} предлагает методы 
предварительной об\-работки и~расширенного анализа, такие как 
час\-тот\-но-вре\-мен\-ной анализ, восстановление источников с~использованием диполей, 
распределенных источников и~непараметрическое статистическое 
тес\-тирование. BioSig предоставляет средства визуализации данных 
и~средства для сбора данных, обработки артефактов, контроля качества, 
извлечения характеристик, классификации, моделирования данных.
{ %\looseness=1

}
    
    \textit{Библиотеки обработки нейроизображений на языке Python} 
помогают разрабатывать собственные программы для работы 
с~нейроизображениями.
    
    Библиотека \textit{NiPy}~[30]~--- библиотека, состоящая из нескольких 
частей, которые позволяют пользователю выполнять как простые операции 
с~изоб\-ра\-же\-ни\-ями fMRI (например, чтение и~запись), так и~сложные алгоритмы 
анализа нейроизображений. \textit{Nibabel} предоставляет прикладной 
программный интерфейс для чтения и~записи различных форматов файлов 
нейроизображений, таких как \mbox{ANALYZE}, NIFTI, MINC, MGH. 
\textit{Niwidgets} пред\-остав\-ля\-ет средства визуализации нейроизображений. 
Nitime пред\-остав\-ля\-ет средства для анализа временн$\acute{\mbox{ы}}$х рядов в~области 
нейровизуализации. \textit{Nilearn} пред\-остав\-ля\-ет средства для 
статистического исследования данных нейровизуализации на основе метода 
независимых компонент CanICA.
    
    \textit{MNE-Python}~[31]~--- это программный пакет с~открытым 
исходным кодом, предназначенный для анализа данных 
МЭГ и~ЭЭГ. Он пред\-остав\-ля\-ет 
современные алгоритмы, которые охватывают несколько методов 
предварительной обработки данных, локализации источников, 
статистического анализа, методы машинного обучения.

\vspace*{-6pt}

\section{Заключение}

 \vspace*{-3pt}

    Нейрофизиологию можно рассматривать как область с~интенсивным 
использованием данных, где данные играют ключевую роль в~исследованиях 
в~понимании работы головного мозга и~в~обнаружении и~лечении 
заболеваний, связанных с~головным мозгом. По всему миру создаются 
проекты, поддерживающие исследования в~этой области. Рост числа 
исследований и~появление нового оборудования ведут к~лавинообразному 
увеличению объема данных. Эти данные могут представляться в~различных 
форматах. Требуются новые средства для хранения данных больших объемов 
(которые могут достигать нескольких петабайт), средства для интеграции 
данных, представленных в~разных форматах, средства анализа такого объема 
данных. Отдельные компьютеры больше не подходят для анализа данных 
в~области нейрофизиологии, а потому необходимо развивать новые 
инфраструктуры, позволяющие хранить и~обрабатывать такие объемы 
данных. Актуальные задачи в~области нейрофизиологии требуют применения 
современных методов анализа данных, включая статистический анализ 
и~машинное обучение, реализованных в~распределенных вычислительных 
инфраструктурах.

\vspace*{-6pt}


{\small\frenchspacing
 {%\baselineskip=10.8pt
 \addcontentsline{toc}{section}{References}
 \begin{thebibliography}{99}
 
 \vspace*{-3pt}
 
  \bibitem{1-bs}
  BRAIN Initiative. {\sf https://braininitiative.nih.gov}.
  \bibitem{2-bs}
  Human Brain Project home page. {\sf https://www.\linebreak humanbrainproject. eu}.
  \bibitem{3-bs}
  \Au{Elam J.\,S., Van Essen~D.} Human Connectome project~// Encyclopedia of 
computational neuroscience~/
Eds. D.~Jaeger, R.~Jung.~--- New York, NY, USA: Springer, 2013. 4~p.
  \bibitem{4-bs}
  \Au{Brunner C., Blankertz~B., Cincotti~F., \textit{et al.}} BNCI Horizon  
2020~--- towards a~roadmap for brain/neural computer interaction~// 8th 
Conference (International) on Universal Access in Human--Computer Interaction 
Proceedings.~--- Lecture notes in computer science ser.~--- Springer, 2014. 
Vol.~8513. P.~475--486. 
  \bibitem{5-bs}
  \Au{Jiang T., Liu~Y., Shi~F., Shu~N., Liu~B., Jiang~J., Zhou~Y.} Multimodal 
magnetic resonance imaging for brain disorders: Advances and perspectives~// 
Brain Imaging Behav., 2008. Vol.~2. Iss.~4. P.~249--257.
  \bibitem{6-bs}
  \Au{Van Horn J.\,D., Toga~A.\,W.} Multisite neuroimaging trials~// Curr. 
Opin. Neurol., 2009. Vol.~22. Iss.~4. P.~370--378. 
  \bibitem{7-bs}
  \Au{Jack C.\,R., Bernstein~M.\,A., Fox~N.\,C., \textit{et al.}} The Alzheimer's 
disease neuroimaging initiative (ADNI): MRI methods~// J.~Magn. Reson. 
Imaging, 2008. Vol.~27. Iss.~4. P.~685--691.
  \bibitem{8-bs}
  \Au{Biswal B.\,B., Mennes~M., Zuo~X.\,N., \textit{et al.}} Toward discovery 
science of human brain function~// P.~Natl. Acad. Sci. USA, 
2010. Vol.~107. Iss.~10. P.~4734--4739.
  \bibitem{9-bs}
  \Au{Poldrack R.\,A., Barch~D.,M., Mitchell~J., \textit{et al.}} Toward open 
sharing of task-based fMRI data: The OpenfMRI project~// Front. Neuroinform., 
2013. Vol.~7. Art. No.\,12. P.~1--12.
  \bibitem{10-bs}
  \Au{Hodge M.\,R., Horton~W., Brown~T., \textit{et al.}}  
ConnectomeDB-sharing human brain connectivity data~// NeuroImage, 2016. 
Vol.~124. P.~1102--1107.
  \bibitem{11-bs}
  \Au{Marcus D., Olsen~T.\,R., Ramaratnam~M., Buckner~R.\,L.} The extensible 
neuroimaging archive toolkit (XNAT): An informatics platform for managing, 
exploring, and sharing neuroimaging data~// Neuroinformatics, 2007. Vol.~5. 
P.~11--34.
  \bibitem{12-bs}
  NITRC home page. {\sf https://www.nitrc.org}.
  \bibitem{13-bs}
  \Au{Neu S.\,C., Crawford~K.\,L., Toga~A.\,W.} Practical management of 
heterogeneous neuroimaging metadata by global neuroimaging data repositories~// 
Front. Neuroinform., 2012. Vol.~6. Art. No.\,8. P.~1--9.
  \bibitem{14-bs}
  Digital Imaging Communication in Medicine (DICOM): NEMA Standards 
Publication PS~3.~--- Washington, DC, USA: National Electrical Manufacturers 
Association, 1999.
  \bibitem{15-bs}
  ANALYZE 7.5 file format. {\sf http://eeg.sourceforge.net/ ANALYZE75.pdf}.
  \bibitem{16-bs}
  NIFTI home page. {\sf http://nifti.nimh.nih.gov}.
  \bibitem{17-bs}
  The Brain Imaging Data Structure (BIDS) specification. {\sf 
https://bids.neuroimaging.io/bids\_spec.pdf}.
  \bibitem{18-bs}
  \Au{Schl$\ddot{\mbox{o}}$gl A.} An overview on data formats for biomedical 
signals~// World Congress on Medical Physics and Biomedical Engineering.~--- 
Berlin--Heidelberg: Springer, 2009. P.~1557--1560.
  \bibitem{19-bs}
  \Au{Kemp B., V$\ddot{\mbox{a}}$rri~A., Rosa~A.\,C., Nielsen~K.\,D., 
Gade~J.} A~simple format for exchange of digitized polygraphic recordings~// 
Electroen. Clin. Neuro., 1992. Vol.~82. Iss.~5. 
P.~391--393.
  \bibitem{20-bs}
  \Au{Schl$\ddot{\mbox{o}}$gl~A.} GDF~--- a~general data format for 
biomedical signals~// arXiv.org, 11~Aug 2006 (v.~1), 26~Mar 2013 (v.~10). 
arxiv:cs/0608052.
  \bibitem{21-bs}
  \Au{Smith J., Johnson~J., Schubert~J., Widell~R.} A~new format for 
polysomnography data~// Sleep, 2005. Vol.~28. Iss.~11. P.~1473--1473.
  \bibitem{22-bs}
  \Au{Fedorov A., Beichel~R., Kalpathy-Cramer~J., \textit{et al.}} 3D slicer as 
an image computing platform for the quantitative imaging network~// Magn. 
Reson. Imaging, 2012. Vol.~30. Iss.~9. P.~1323--1241.
  \bibitem{23-bs}
  \Au{Sadigh-Eteghad S., Majdi~A., Farhoudi~M., Talebi~M., Mahmoudi~J.} 
Different patterns of brain activation in normal aging and Alzheimer's disease from 
cognitional sight: Meta analysis using activation likelihood estimation~// 
J.~Neurol. Sci., 2014. Vol.~343. Iss.~1-2. P.~159--166. 
  \bibitem{24-bs}
  \Au{Whitfield-Gabrieli S., Nieto-Castanon~A.} Conn: A~functional 
connectivity toolbox for correlated and anticorrelated brain networks~// Brain 
Connectivity, 2012. Vol.~2. Iss.~3. P.~125--141. 
  \bibitem{25-bs}
  \Au{Friston~K.\,J., Ashburner~J.\,T., Kiebel~S.\,J., 
Nichols~T.\,E., Penny W.\,D.} Statistical parametric mapping: The analysis of functional brain 
images: The analysis of functional brain images.~--- Academic Press, 2011. 688~p.
  \bibitem{26-bs}
  \Au{Poil S.} Neurophysiological Biomarkers of cognitive decline: From 
criticality to toolbox.~--- Amsterdam: VU University, 2013. 218~p.
  \bibitem{27-bs}
  \Au{Delorme A., Makeig~S.} EEGLAB: An open source toolbox for analysis of 
single-trial EEG dynamics~// J.~Neurosci. Meth., 2004. Vol.~134. P.~9--21.
  \bibitem{28-bs}
  \Au{Oostenveld R., Fries~P., Maris~E., Schoffelen~J.\,M.} FieldTrip: Open 
source software for advanced analysis of MEG, EEG, and invasive 
electrophysiological data~// Comput. Intell. Neurosc., 2011. 
Vol.~2011. Art. ID: 156869. P.~1--9.
  \bibitem{29-bs}
  \Au{Vidaurre C., Sander~T.\,H., Schl$\ddot{\mbox{o}}$gl~A.} BioSig: The free 
and open source software library for biomedical signal processing~// 
Comput. Intell. Neurosc., 2011. Vol.~2011. Art. ID: 935364. 
P.~1--12.
  \bibitem{30-bs}
  \Au{Brett M., Taylor~J., Burns~C., \textit{et al.}} NIPY: An open library and 
development framework for FMRI data analysis~// NeuroImage, 2009. Vol.~47. 
Suppl.~1. P.~S196.
  \bibitem{31-bs}
  \Au{Gramfort A., Luessi~M., Larson~E., \textit{et al.}}  MEG and EEG data 
analysis with MNE-Python~// Front. Neurosci., 2013. Vol.~7. Art. No.\,267.  
P.~1--13.
 \end{thebibliography}

 }
 }

\end{multicols}

\vspace*{-3pt}

\hfill{\small\textit{Поступила в~редакцию 14.11.19}}

%\vspace*{8pt}

%\pagebreak

\newpage

\vspace*{-28pt}

%\hrule

%\vspace*{2pt}

%\hrule

%\vspace*{-2pt}

\def\tit{NEUROPHYSIOLOGY AS~A~SUBJECT DOMAIN\\ FOR~DATA 
INTENSIVE PROBLEM SOLVING}


\def\titkol{Neurophysiology as~a~subject domain for~data 
intensive problem solving}

\def\aut{D.\,O.~Briukhov, S.\,A.~Stupnikov, D.\,Yu.~Kovalev, and~I.\,A.~Shanin}

\def\autkol{D.\,O.~Briukhov, S.\,A.~Stupnikov, D.\,Yu.~Kovalev, and~I.\,A.~Shanin}

\titel{\tit}{\aut}{\autkol}{\titkol}

\vspace*{-11pt}


\noindent
Institute of Informatics Problems, Federal Research Center ``Computer Science 
and Control'' of the Russian Academy of Sciences, 44-2~Vavilov Str., Moscow 
119333, Russian Federation

\def\leftfootline{\small{\textbf{\thepage}
\hfill INFORMATIKA I EE PRIMENENIYA~--- INFORMATICS AND
APPLICATIONS\ \ \ 2020\ \ \ volume~14\ \ \ issue\ 1}
}%
 \def\rightfootline{\small{INFORMATIKA I EE PRIMENENIYA~---
INFORMATICS AND APPLICATIONS\ \ \ 2020\ \ \ volume~14\ \ \ issue\ 1
\hfill \textbf{\thepage}}}

\vspace*{6pt} 
  


\Abste{The goal of this survey is to analyze neurophysiology as a data intensive domain. 
Nowadays, the number of researches on the human brain is increasing. International projects and 
researches are aimed at improvement of the understanding of the human brain function. The 
amount of data obtained in typical laboratories in the field of neurophysiology is growing 
exponentially. The data are represented using a~large number of various formats. This requires 
creation of infrastructures, databases, and websites that provide unified access to data and support 
the exchange of data between researchers all over the world. Specific methods and tools forming 
the field of neuroinformatics (that is, an intersection of neurophysiology and computer science) 
are used to analyze collected data and to solve neurophysiological problems. These methods 
include, in particular, statistical analysis, machine learning, and neural networks.}

\KWE{neurophysiology; neurophysiological resources; neuroinformatics; data intensive 
research; analysis of neurophysiological data}



\DOI{10.14357/19922264200106} 

%\vspace*{-14pt}

\Ack
\noindent
This research was partially supported by the Russian Foundation 
for Basic Research (project   18-29-22096).


 


\vspace*{6pt}

  \begin{multicols}{2}

\renewcommand{\bibname}{\protect\rmfamily References}
%\renewcommand{\bibname}{\large\protect\rm References}

{\small\frenchspacing
 {%\baselineskip=10.8pt
 \addcontentsline{toc}{section}{References}
 \begin{thebibliography}{99}
\bibitem{1-bs-1}
BRAIN Initiative Home Page. Available at: {\sf https://\linebreak braininitiative.nih.gov/} 
(accessed November~12, 2019)
\bibitem{2-bs-1}
Human Brain Project Home Page. Available at: {\sf 
https://\linebreak www.humanbrainproject.eu} (accessed November~12, 2019).
\bibitem{3-bs-1}
\Aue{Elam, J.\,S., and D.~Van Essen.} 2013. Human Connectome project. 
\textit{Encyclopedia of computational neuroscience}. Eds. D.~Jaeger and R.~Jung.
New York, NY: Springer.  4~p.
\bibitem{4-bs-1}
\Aue{Brunner, C., B.~Blankertz, F.~Cincotti, \textit{et al.}} 2014. 
\mbox{BNCI} Horizon 
2020~--- towards a~roadmap for brain/neural computer interaction. \textit{8th 
 Conference (International) on Universal Access in Human--Computer 
Interaction Proceedings.} Lecture notes in computer science ser. Springer. 
8513: 475--486.
\bibitem{5-bs-1}
\Aue{Jiang, T., Y.~Liu, F.~Shi, N.~Shu, B.~Liu, J.~Jiang, and Y.~Zhou.} 2008. 
Multimodal magnetic resonance imaging for brain disorders: advances and 
perspectives. \textit{Brain Imaging Behav.} 2(4):249--257.
\bibitem{6-bs-1}
\Aue{Van Horn, J.\,D., and A.\,W.~Toga.} 2009. Multisite neuroimaging trials. 
\textit{Curr. Opin. Neurol.} 22(4):370--378. 
\bibitem{7-bs-1}
\Aue{Jack, C.\,R., M.\,A.~Bernstein, N.\,C.~Fox, \textit{et al.}} 2008. The 
Alzheimer's disease neuroimaging initiative (ADNI): MRI methods. 
\textit{J.~Magn. Reson. Imaging} 27(4):685--691.
\bibitem{8-bs-1}
\Aue{Biswal, B.\,B., M.~Mennes, X.\,N.~Zuo, \textit{et al.}} 2010. Toward 
discovery science of human brain function. \textit{P.~Natl. 
Acad. Sci. USA} 107(10):4734--4739.
\bibitem{9-bs-1}
\Aue{Poldrack, R.\,A., D.\,M.~Barch, J.~Mitchell, \textit{et al.}} 2013. Toward 
open sharing of task-based fMRI data: The \mbox{OpenfMRI} project. \textit{Front. 
Neuroinform.} 7:12.
\bibitem{10-bs-1}
\Aue{Hodge, M.\,R., W.~Horton, T.~Brown, \textit{et al.}} 2016. 
ConnectomeDB-sharing human brain connectivity data. \textit{NeuroImage} 
124:1102--1107.
\bibitem{11-bs-1}
\Aue{Marcus, D., T.\,R.~Olsen, M.~Ramaratnam, and R.\,L.~Buckner.} 2007. The 
extensible neuroimaging archive toolkit (XNAT): An informatics platform for 
managing, exploring, and sharing neuroimaging data. 
\textit{Neuroinformatics} 5:11--34.
\bibitem{12-bs-1}
NITRC Home Page. Available at: {\sf https://www.nitrc.org/} (accessed 
November~12, 2019).
\bibitem{13-bs-1}
\Aue{Neu, S.\,C., K.\,L.~Crawford, and A.\,W.~Toga.} 2012. Practical 
management of heterogeneous neuroimaging metadata by global neuroimaging 
data repositories. \textit{Front. Neuroinform.} 6:8.
\bibitem{14-bs-1}
Digital Imaging Communication in Medicine (DICOM). 1999. NEMA Standards 
Publication PS~3. Washington, DC: National Electrical Manufacturers 
Association.
\bibitem{15-bs-1}
ANALYZE 7.5 file format. Available at: {\sf 
http://eeg.\linebreak sourceforge.net/ANALYZE75.pdf} (accessed November~12, 2019).
\bibitem{16-bs-1}
NIFTI home page. Available at: {\sf http://nifti.nimh.nih.gov} (accessed 
November~12, 2019).
\bibitem{17-bs-1}
The Brain Imaging Data Structure (BIDS) specification. Available at: {\sf 
https://bids.neuroimaging.io/bids\_spec.pdf} (accessed November~12, 2019).
\bibitem{18-bs-1}
\Aue{Schl$\ddot{\mbox{o}}$gl,~A.} 2009. An overview on data formats for 
biomedical signals. \textit{World Congress on Medical Physics and 
Biomedical Engineering}. Berlin--Heidelberg: Springer. 1557--1560.
\bibitem{19-bs-1}
\Aue{Kemp, B., A.~V$\ddot{\mbox{a}}$rri, A.\,C.~Rosa, K.\,D.~Nielsen, and 
J.~Gade.} 1992. A~simple format for exchange of digitized polygraphic 
recordings. \textit{Electroen. Clin. Neuro.} 
82(5):391--393.
\bibitem{20-bs-1}
\Aue{Schl$\ddot{\mbox{o}}$gl, A.} GDF~--- a~general data format for 
biomedical signals Version~2.51. Available at: {\sf 
https://arxiv.org/\linebreak abs/cs/0608052} (accessed November~12, 2019).
\bibitem{21-bs-1}
\Aue{Smith, J., J.~Johnson, J.~Schubert, and R.~Widell.} 2005. A~new file 
format for polysomnography data. \textit{Sleep} 28(11):1473--1473.
\bibitem{22-bs-1}
\Aue{Fedorov, A., R.~Beichel, J.~Kalpathy-Cramer, \textit{et al.}} 2012. 3D 
slicer as an image computing platform for the quantitative imaging network. 
\textit{Magn. Reson. Imaging} 30(9):1323--1241. 
\bibitem{23-bs-1}
\Aue{Sadigh-Eteghad, S., A.~Majdi, M.~Farhoudi, M.~Talebi, and 
J.~Mahmoudi.} 2014. Different patterns of brain activation in normal aging 
and Alzheimer's disease from cognitional sight: Meta analysis using activation 
likelihood estimation. \textit{J.~Neurol. Sci.} 343(1-2):159--166.



\bibitem{24-bs-1}
\Aue{Whitfield-Gabrieli, S., and A.~Nieto-Castanon.} 2012. Conn: A~functional 
connectivity toolbox for correlat-\linebreak\vspace*{-12pt}

\columnbreak 

\noindent
ed and anticorrelated brain networks. 
\textit{Brain Connectivity} 2(3):125--141.
\bibitem{25-bs-1}
\Aue{Friston, K.\,J., J.\,T.~Ashburner, S.\,J.~Kiebel, 
T.\,E.~Nichols, and W.\,D.~Penny.} 2011. \textit{Statistical parametric mapping: The analysis of 
functional brain images}. Academic Press. 688~p.
\bibitem{26-bs-1}
\Aue{Poil, S.} 2013. \textit{Neurophysiological Biomarkers of cognitive decline: 
From criticality to toolbox}. Amsterdam: VU University. 218~p.
\bibitem{27-bs-1}
\Aue{Delorme, A., and S.~Makeig.} 2004. EEGLAB: An open source toolbox for 
analysis of single-trial EEG dynamics. \textit{J.~Neurosci. Meth.} 
134:9--21.
\bibitem{28-bs-1}
\Aue{Oostenveld, R., P.~Fries, E.~Maris, and J.\,M.~Schoffelen.} 2011. FieldTrip: 
Open source software for advanced analysis of MEG, EEG, and invasive 
electrophysiological data. \textit{Comput. Intell. 
Neurosc.} 2011:156869.
\bibitem{29-bs-1}
\Aue{Vidaurre, C., T.\,H.~Sander, and A.~Schl$\ddot{\mbox{o}}$gl.} 2011. 
BioSig: The free and open source software library for biomedical signal 
processing. \textit{Comput. Intell. Neurosc.} 
2011:935364. 
  \bibitem{30-bs-1}
\Aue{Brett, M., J.~Taylor, C.~Burns, \textit{et al.}} 2009. NIPY: An open library 
and development framework for FMRI data analysis. \textit{NeuroImage} 
47:S196.
  \bibitem{31-bs-1}
\Aue{Gramfort, A., M.~Luessi, and E.~Larson.} 2013. EEG data analysis with 
MNE-Python. \textit{Front. Neurosci.} 7:267.
%\vspace*{-18pt}

\end{thebibliography}

 }
 }

\end{multicols}

%\vspace*{-7pt}

\hfill{\small\textit{Received November 14, 2019}}

%\pagebreak

%\vspace*{-22pt}

\Contr

    \noindent
\textbf{Briukhov Dmitry O.} (b.\ 1971)~--- Candidate of Science (PhD) in technology, senior 
scientist, Institute of Informatics Problems, Federal Research Center ``Computer Science and 
Control'' of the Russian Academy of Sciences, 44-2~Vavilov Str., Moscow 119333, Russian 
Federation; \mbox{dbriukhov@ipiran.ru}
    
    \vspace*{3pt}
    
    \noindent
    \textbf{Stupnikov Sergey A.} (b.\ 1978)~--- Candidate of Science (PhD) in 
technology, lead scientist, Institute of Informatics Problems, Federal Research 
Center ``Computer Science and Control'' of the Russian Academy of Sciences,  
44-2~Vavilov Str., Moscow 119333, Russian Federation; 
\mbox{sstupnikov@ipiran.ru}
    
    \vspace*{3pt}
    
    \noindent
    \textbf{Kovalev Dmitry Yu.} (b.\ 1988)~--- junior scientist, Institute of 
Informatics Problems, Federal Research Center ``Computer Science and Control'' 
of the Russian Academy of Sciences, 44-2~Vavilov Str.,  Moscow 119333, Russian 
Federation; \mbox{dkovalev@ipiran.ru}
    
    \vspace*{3pt}
    
    \noindent
    \textbf{Shanin Ivan A.} (b.\ 1991)~--- junior scientist, Institute of Informatics 
Problems, Federal Research Center ``Computer Science and Control'' of the 
Russian Academy of Sciences, 44-2~Vavilov Str., Moscow 119333, Russian 
Federation; \mbox{v08shanin@gmail.com}
\label{end\stat}

\renewcommand{\bibname}{\protect\rm Литература} 