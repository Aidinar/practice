\newcommand {\ebd}{\triangleq}
\newcommand {\col}{\mathop{\mathrm{col}\,}}
\newcommand{\me}[2]{\mathbf{E}_{ #1 }\left\{ \mathop{#2} \right\} }






%\newcommand {\s}{^{(s)}}
%\newcommand {\st}{^{(\prime)}}



\def\stat{borisov}

\def\tit{ЧИСЛЕННЫЕ СХЕМЫ ФИЛЬТРАЦИИ МАРКОВСКИХ СКАЧКООБРАЗНЫХ ПРОЦЕССОВ 
ПО~ДИСКРЕТИЗОВАННЫМ НАБЛЮДЕНИЯМ II:~СЛУЧАЙ АДДИТИВНЫХ ШУМОВ$^*$}

\def\titkol{Численные схемы фильтрации марковских скачкообразных процессов 
по~дискретизованным наблюдениям II}
%:~случай аддитивных шумов}

\def\aut{А.\,В.~Борисов$^1$}

\def\autkol{А.\,В.~Борисов}

\titel{\tit}{\aut}{\autkol}{\titkol}

\index{Борисов А.\,В.}
\index{Borisov A.\,V.}


{\renewcommand{\thefootnote}{\fnsymbol{footnote}} \footnotetext[1]
{Работа выполнена при частичной поддержке РФФИ (проект 19-07-00187~А).}}


\renewcommand{\thefootnote}{\arabic{footnote}}
\footnotetext[1]{Институт проблем информатики Федерального исследовательского центра 
<<Информатика и~управление>> Российской академии наук,
\mbox{aborisov@frccsc.ru}}

\vspace*{-8pt}



\Abst{Статья продолжает цикл исследований, начатых в~работе  
<<Численные схемы фильтрации марковских скачкообразных процессов 
по дискретизованным наблюдениям I:~характеристики точности>>. 
На основании полученных оценок точности приближенного решения задачи 
фильтрации состояний однородных марковских скачкообразных процессов (МСП) по 
косвенным непрерывным зашумленным наблюдениям предложены различные 
алгоритмы ее численной реализации и~проведен их сравнительный анализ. 
При этом класс систем наблюдения ограничен системами с~\textit{аддитивными} 
винеровскими шумами: интенсивность шумов в~наблюдениях является 
неслучайной постоянной. Для построения аппроксимаций использовались 
численные схемы  <<левых>> и~<<средних>> прямоугольников порядка точности~2 и~3 
соответственно, а также квадратуры Гаусса порядка~5. 
В~итоге были получены численные схемы порядка точности $1/2$, 1 и~2.}


\KW{марковский скачкообразный процесс; оптимальная фильтрация; 
аддитивные шумы в~наблюдениях; стохастическое дифференциальное 
уравнение; аналитическая и~численная аппроксимация}

\DOI{10.14357/19922264200103} 
  
\vspace*{-3pt}


\vskip 10pt plus 9pt minus 6pt

\thispagestyle{headings}

\begin{multicols}{2}

\label{st\stat}



 \section{Введение}
 
 Данная статья продолжает исследования, начатые в~\cite{B_18_IA, B_20_1_IA}. 
 Они посвящены численной реализации решения задачи фильтрации состояний 
 однородных \textit{марковских скачкообразных процессов} по непрерывным 
  косвенным наблюдениям в~присутствии винеровских шумов~\cite{Wonham_64,  B_18}. 
   Оценка оптимальной фильтрации является решением сис\-те\-мы нелинейных 
   \textit{стохастических дифференциальных уравнений} (СДУ) типа
 Куш\-не\-ра--Стра\-то\-но\-ви\-ча, не имеющим аналитического представления. 
 Искомая оценка оптимальной фильтрации является условным распределением 
 состояния МСП, имеющим неотрицательные компоненты и~удовлетворяющим 
 условию нормировки. Традиционные методы решения систем СДУ, например метод
 Эй\-ле\-ра--Ма\-ру\-ямы~\cite{KP_92}, не гарантируют сохранение этих 
 свойств для своих реализаций и~поэтому бесполезны для решения систем 
 Куш\-не\-ра--Стра\-то\-но\-ви\-ча.  Те численные методы, которые 
 обеспечивают своим оценкам выполнение свойств не\-от\-ри\-ца\-тель\-ности 
 компонентов и~нормировки, названы в~\cite{B_18_IA, B_20_1_IA}  
 \textit{устойчивыми}. Для их построения предлагается аппроксимировать 
 исходную задачу фильтрации по непрерывным наблюдениям задачей оптимальной 
 фильтрации по дискретизованным наблюдениям. В~\cite{B_18_IA} эта задача решена:
  соответствующая рекуррентная формула представляет собой вариант 
  абстрактной формулы Байеса. Она содержит бесконечные суммы, в~которых 
  слагаемые~--- интегралы. Предложена замена бесконечного ряда его 
  конечным отрезком длины~$s$, соответствующая учету в~оценке возможности 
  того, что состояние на временн$\acute{\mbox{о}}$м 
  интервале дискретизации наблюдений совершит не более~$s$ переходов; 
  полученное таким образом приближение названо в~работе 
  \textit{аналитической аппроксимацией порядка~$s$}. 
  В~\cite{B_18_IA} доказано утверждение, определяющее потерю точности 
  при переходе от оптимальной оценки к~ее аналитической 
  аппроксимации в~зависимости от параметров системы наблюдения и~выбранного 
  порядка~$s$.

 Аналитические аппроксимации также не могут быть вычислены точно, 
 так как содержат интегралы, не имеющие явного аналитического пред\-став\-ле\-ния. 
 В~\cite{B_20_1_IA} предложено заменить их конечными интегральными суммами. 
 Полученные аппроксимации названы \textit{численными}. Они уже могут 
 быть реализованы средствами вычислительной техники. Доказано утверждение, 
 определяющее потерю точности при переходе от аналитической 
 аппроксимации к~ее численной реализации в~зависимости от точности
  используемой схемы интегрирования.  Результаты~\cite{B_18_IA,   B_20_1_IA} также
 позволяют оценить общую потерю точности при переходе от оценки 
 оптимальной фильтрации по дискретизованным наблюдениям к~ее численной 
 аппроксимации, заданной схемой численного интегрирования.

Цель данной статьи~--- разработка и~сравнительный анализ различных 
численных аппроксимаций оценок оптимальной фильтрации для класса 
сис\-тем с~аддитивными шумами в~наблюдениях. Это значит, что интенсивность 
шумов является неслучайной.  Несмотря на то что теоретическое решение 
данной задачи в~непрерывном времени известно давно и~определяется 
классическим фильтром Во\-нэ\-ма~\cite{Wonham_64}, создание алгоритмического 
обеспечения для его эффективной устойчивой численной реализации до сих
 пор является актуальной задачей~\cite{YZL_04, PR_10, BGH_16}.

Статья организована следующим образом. Раздел~2 содержит постановку 
задачи оптимальной фильтрации по дискретизованным наблюдениям и~сведения 
из~\cite{B_18_IA, B_20_1_IA}, необходимые для ее численного решения. 
В~разд.~3 рассматривается подкласс сис\-тем наблюдения, включающий в~себя 
лишь сис\-те\-мы с~аддитивными шумами в~наблюдениях. Для них выбраны 
аналитические аппроксимации порядка $s\hm=1$ и~$2$, для реализации 
которых использованы схемы численного интегрирования <<левых>> и~<<средних>> 
прямоугольников (для $s\hm=1$), а~также комбинация квадратур Гаусса 
(для $s\hm=2$). Полученные в~результате численные аппроксимации имеют 
порядок ${1}/{2}$, 1 и~2. Заключительные замечания представлены в~разд.~4.


 \section{Постановка задачи фильтрации и~необходимые сведения 
 об~аналитических и~численных аппроксимациях}
 
 На триплете с~фильтрацией 
 $(\Omega^X \times \Omega^W,\mathcal{F}^X \hm\times \mathcal{F}^W,
 \mathcal{P}^X \times \mathcal{P}^W, 
 \{\mathcal{F}^X_t \times \mathcal{F}^W_t\}_{t \geqslant 0})$
  рассматривается сис\-те\-ма наблюдения
\begin{align*}
X_t &= \displaystyle X_0 + \int\limits_0^t \Lambda^{\top}X_{s}\,ds + \mu_t\,; \\
 Y_r &= \displaystyle\!\!\!\!\!\int\limits_{(r-1)h}^{rh}\!\!\!\!\!
 fX_s \,ds+\!\!\!\!\!\int\limits_{(r-1)h}^{rh}
  \sum\limits_{n=1}^NX_s^ng_n^{1/2} \,dW_s,\enskip 
 r \in \mathbb{N},
 % \label{eq:obsys_1}
 \end{align*}
 где
  \begin{itemize}
  \item
  $X_t  \ebd \col(X_t^1,\ldots,X_t^N) \in \mathbb{S}^N$~--- 
  не\-наблю\-да\-емое состояние сис\-те\-мы~--- 
  однородный МСП с~конечным множеством со\-сто\-яний 
  $\mathbb{S}^N \ebd\linebreak \ebd \{e_1,\ldots,e_N\}$ ($\mathbb{S}^N$~--- 
  множество единичных век\-то\-ров евклидова пространства~$\mathbb{R}^N$), 
  матрицей интенсивностей переходов~$\Lambda$ и~начальным распределением~$\pi$;
  \item
  $\mu_t \ebd \col(
  \mu_t^1,\ldots,\mu_t^N)\in \mathbb{R}^N$~--- 
  $\mathcal{F}_t^X$-со\-гла\-со\-ван\-ный мартингал;
  \item
  $\{Y_r\}_{r \in \mathbb{N}}:\;  Y_r \ebd \col(Y_r^1,\ldots,Y_r^M) 
  \hm\in \mathbb{R}^M$~--- последовательность дискретизованных 
  наблюдений, доступных в~известные равноотстоящие моменты 
  времени $\{rh\}_{r \in \mathbb{N}}$;
  \item
$W_t \ebd \col(W_t^1,\ldots,W_t^M) \in \mathbb{R}^M$~--- 
$\mathcal{F}_t^W$-со\-гла\-со\-ван\-ный стандартный винеровский процесс;  
\item $f$~--- $(M \times N)$-мер\-ная матрица;
\item $\{g_n\}_{n=\overline{1,N}}$~--- 
симметрические положительно определенные матрицы.
  \end{itemize}

   \textit{Задача оптимальной фильтрации состояния~$X$ 
   по дискретизованным наблюдениям~$Y$} заключается в~нахождении 
   \textit{условного математического ожидания} (УМО)
  \begin{equation*}
  \widehat{X}_r \ebd \me{}{X_{t_r}|\mathcal{O}_{r} },
  %\label{eq:fest_1}
  \end{equation*}
  где $\mathcal{O}_r \ebd \sigma\{ Y_{\ell}: \; 1 \leqslant
   \ell \leqslant r\}$~--- $\sigma$-ал\-геб\-ра, по\-рож\-ден\-ная 
   наблюдениями, полученными до момента времени~$rh$ включительно; 
   $\mathcal{O}_0 \ebd \{\varnothing,\; \Omega\}$.

    Пусть $N_r^X(\omega)$~--- число скачков процесса~$X$, произошедших 
    на отрезке $[(r-1)h,rh]$; $\tau_r \ebd \int\nolimits_0^t X_s\,ds$~--- 
    случайный вектор времени пребывания процесса~$X$ в~различных состояниях 
    на отрезке $[(r-1)h,rh]$, а~$\rho^{n,j,m}(\cdot)$~--- 
    распределение вектора
 $\tau_{r}X_{t_{r}}^{j}\mathbf{I}_{\{m\}}(N_{r}^X)$ 
 при условии $X_{t_{r-1}}\hm=e_k$. Это означает, что
 для любого $\mathcal{G} \hm\in \mathcal{B}(\mathbb{R}^M)$ верно равенство
$$
\me{}{\mathbf{I}_{\mathcal{G}}(\tau_r)X_{t_r}^j\mathbf{I}_{\{m\}}
\left(N_r^X\right)\vert X_{t_{r-1}}=e_k}\!
=
 \!\! \int\limits_{\mathcal{G}}\! \rho^{k,j,m}(du).
$$

\textit{Аналитическая аппроксимация}~$\overline{X}_r$ порядка~$s$ 
определяется рекурсивной схемой
 \begin{equation}
\overline{X}_r = (\mathbf{1}\xi_{r}^{\top}\overline{X}_{r-1})^{-1} 
\xi_{r}^{\top}\overline{X}_{r-1}, \enskip
r \geqslant 1, \enskip \overline{X}_0=\pi,
 \label{eq:filt_3}
 \end{equation}
 где $\mathbf{1} = \mathrm{row} (1,\ldots,1)$~--- век\-тор-стро\-ка 
 подходящей размерности,
 а $\xi_q \ebd \|\xi^{ij}(Y_q)\|_{i,j=\overline{1,N}}$~---
  $(N \times N)$-мер\-ные случайные матрицы~--- функции наблюдений~$Y_q$:
 \begin{equation}
 \xi^{ij}(y)\ebd
\sum\limits_{m=0}^s \int\limits_{\mathcal{D}}
 \mathcal{N}\left(y,f u,\sum\limits_{p=1}^N u^p g_p\right)
 \rho^{i,j,m}(du).
 \label{eq:xi_def}
 \end{equation}
 В последней формуле 
 $\mathcal{N}(y,m,K) \ebd\linebreak \ebd (2\pi)^{-M/2} \mathrm{det}^{-1/2} 
 K \exp\left\{ -({1}/{2})\|y-m)\|^2_{K^{-1}}\right\}$~---\linebreak
  $M$-мер\-ная 
 плотность гауссовского распределения с~математическим ожиданием~$m$ 
 и~невырожденной ковариационной матрицей~$K$.

 Из построения оценки~$\overline{X}_r$~(\ref{eq:filt_3}) следует, что 
 она обладает свойством устойчивости.

 Обычно значения интегралов~$\xi^{ij}(y)$ приближенно вычисляются в~виде 
 интегральных сумм:
  \begin{align*}
  \xi^{ij}(y) &\approx \psi^{ij}(y) \ebd
 \sum\limits_{\ell=1}^{L} \mathcal{N}\left(y,f w_{\ell},\sum\limits_{p=1}^N 
 w^p_{\ell} g_p\right)\varrho_{\ell}^{ij}; \\
 \psi(y) &\ebd \|\psi^{ij}(y)\|_{i,j=\overline{1,N}},
%  \label{eq:int_sum}
  \end{align*}
  определяемых набором пар $\{(w_{\ell},
  \varrho_{\ell}^{ij})\}_{\ell=\overline{1,L}}$. 
  Здесь $\varrho_{\ell}^{ij} \hm\geqslant 0$, $\ell\hm=\overline{1,L}$,~--- 
  веса, $\sum\nolimits_{\ell=1}^L\varrho_{\ell}^{ij}\hm \leqslant 1$, 
  а~$w_{\ell}\ebd \col(w^1_{\ell},\ldots,w^N_{\ell}) \hm\in \mathcal{D}$~--- 
  точки
  и~$\psi_q \ebd\linebreak \ebd \|\psi^{ij}(Y_q)\|_{ij=\overline{1,N}}$.

  По построению~$\psi^{ij}_q$ являются положительными случайными 
  величинами, поэтому \textit{численная аппроксимация}~$\widetilde{X}_r$ 
  оценки~$\overline{X}_r$, вычисляемая ре\-кур\-сивно
  \begin{equation}
  \widetilde{X}_r \ebd (\mathbf{1}\psi_r^{\top} 
  \widetilde{X}_{r-1})^{-1}\psi_r^{\top} \widetilde{X}_{r-1}, \enskip
   r\geqslant 1\,, \enskip \widetilde{X}_{0} = \pi\,,
  \label{eq:pp_est}
  \end{equation}
  также обладает свойством устойчивости.

Если для схемы численного интегрирования выполнено условие
$$
\int\limits_{\mathbb{R}^M}|\psi^{kj}(y) - \xi^{kj}(y)|dy < \delta\,,
$$
то глобальный показатель точности ограничен сверху
 \begin{multline}
\mathop{\mathrm{sup}}\limits_{\pi \in \Pi}
\me{}{\|\widehat{X}_r - \widetilde{X}_r\|_1} \leqslant  {}\\
{}\leqslant 
 4
\left[ 1-\left(1-\fr{(\overline{\lambda}h)^{s+1}}{(s+1)!}\right)^r
\right] + 2r\delta\,,
 \label{eq:tot_glob}
 \end{multline}
 где $\overline{\lambda} \ebd \max_k|\lambda_{kk}|$.
 Для фиксированного момента времени~$T$ с~уменьшением шага дискретизации 
 $h \hm\to 0$ неравенство~(\ref{eq:tot_glob}) принимает асимптотический вид:
 \begin{multline}
\mathop{\mathrm{sup}}\limits_{\pi \in \Pi}\me{}{\|\widetilde{X}_{T/h} - 
\widehat{X}_{T/h}\|_1}  \leqslant{}\\
{}\leqslant
2T\left(2\overline{\lambda} \fr{(\overline{\lambda}h)^s}{(s+1)!}
+  \fr{\delta}{h}\right).
 \label{eq:asympt_glob}
\end{multline}
Для эффективного выбора схемы численного интегрирования необходимо 
подбирать~$\delta$ так, чтобы
 ${\overline{\lambda}\delta}/{(\overline{\lambda}h)^{s+1}} \hm\to C 
 \hm\geqslant 0$ при $h \hm\to 0$.

 \section{Приближенное решение задачи фильтрации по~наблюдениям 
 с~аддитивными шумами}
 
 Ниже исследуются аппроксимации порядка $s\hm=1$ и~$2$. 
 Для них с~по\-мощью обобщенной формулы полной вероятности легко
  получить вид интегралов~(\ref{eq:xi_def}),  используемых в~дальнейшем 
  изложении:
   \begin{multline}
 \int\limits_{\mathcal{D}}  \mathcal{N}\left(Y_{r},f u,\sum\limits_{p=1}^N u^p g_p\right)
  \rho^{k,j,0}(du) ={}\\
  {}=
 \delta_{kj}e^{\lambda_{kk}h} \mathcal{N}\left(Y_{r},h f^k,h g_k\right);
 \label{eq:h0}
 \end{multline}
 
 \vspace*{-12pt}
 
 \noindent
 \begin{multline}
 \int\limits_{\mathcal{D}}  \mathcal{N}\left(Y_{r},f u,\sum\limits_{p=1}^N 
 u^p g_p\right) \rho^{k,j,1}(du) ={} \\
 {}=\left(1-\delta_{kj}\right)\lambda_{kj}e^{\lambda_{jj}h}
 \int\limits_0^{h}
 e^{(\lambda_{kk}-\lambda_{jj})u}\times{}\\
 \hspace*{-3mm}{}\times
 \mathcal{N}\left(Y_{r},uf^k + (h - u)f^j, u g_k+(h-u)g_j\right)\,du;\!\!
 \label{eq:h1}
 \end{multline}
 
 \vspace*{-12pt}
 
 \noindent
\begin{multline}
\int\limits_{\mathcal{D}}  \!\!\mathcal{N}\left(Y_{r},f u,\sum\limits_{p=1}^N u^p g_p\right)
 \rho^{k,j,2}(du)  ={}\\
 {}=
  \sum\limits_{\substack{{i:i \neq k,}\\ {i \neq j}}} 
  \lambda_{k i}\lambda_{i j} 
  e^{\lambda_{jj}h}
\int\limits_0^{h} \!\int\limits_0^{h-u^k}
\!\!
e^{(\lambda_{kk}-\lambda_{i i})u^k+
\left(\lambda_{i i}-\lambda_{jj}\right)u^{i}}\times{} \\
{}\times
\mathcal{N}\left(Y_{r},u^kf^k+u^{i}f^{i}+(h -u^k-u^{i} )f^j,\right.\\
\left. u^kg_k+u^{i}g_{i}+(h -  u^k  - u^{i} )g_j
 \right) du^{i}du^{k},
 \label{eq:h2}
 \end{multline}
  где $f^j$~--- $j$-й столбец матрицы~$f$. За исключением~(\ref{eq:h0}) 
  остальные интегралы не имеют явного аналитического представления, и~для 
  них будут рассмотрены различные варианты численной реализации.

 \subsection{Порядок $s=1$, схема <<левых>> прямоугольников}
 
  Для системы наблюдения с~чисто аддитивными шумами ($g_n \hm\equiv g$) 
  исследуем точность аппроксимации первого порядка ($s\hm=1$) при 
  использовании численного интегрирования методом <<левых>> прямоугольников. 
  В~этом случае
 \begin{multline}
 \xi^{kj}(y)=\delta_{kj}e^{\lambda_{jj}h}\mathcal{N}(y,hf^j,hg)+ {}\\
 {}+
 (1-\delta_{kj})\lambda_{kj}e^{\lambda_{jj}h}
 \int\limits_0^h \!e^{(\lambda_{kk}-\lambda_{jj})u}\times{}\\
 {}\times
 \mathcal{N}\left(y,uf^k+(h-u)f^j,hg\right)du\,;
 \label{eq:xi_1}
 \end{multline}
 
 \vspace*{-12pt}

\noindent
 \begin{multline}
\psi^{kj}(y)={}\\
{}=e^{\lambda_{jj}h}\mathcal{N}(y,hf^j,hg)
 \left[\delta_{kj}+(1-\delta_{kj})\lambda_{kj}h\right].
 \label{eq:psi_left_tr}
 \end{multline}
 Ошибка численного интегрирования~\cite{IK_94} определяется следующим образом:
 
 \noindent
 \begin{multline*} 
 \gamma^{kj}(y)=\left(1-\delta_{kj}\right)\fr{\lambda_{kj}h^2}{2}\,
 e^{\lambda_{jj}h}\fr{d}{du}\big[
 e^{(\lambda_{kk}-\lambda_{jj})u}\times{}\\
 {}\times 
 \mathcal{N}\left(y,uf^k+(h-u)f^j,hg\right)
 \big]\Bigl|_{u=z} ={} \\ 
 {}=
 \left(1-\delta_{kj}\right)
 \fr{\lambda_{kj}h^2}{2}e^{\lambda_{jj}h}
 e^{(\lambda_{kk}-\lambda_{jj})z}\times{}\\
 {}\times
 \mathcal{N}
 \left(y,zf^k+(h-z)f^j,hg\right) \zeta_0(y,z),
 \end{multline*}
 где $z = z (y) \hm\in [0,h]$~--- некоторый параметр, зависящий от~$y$, и
 \begin{multline} 
 \zeta_0(y,z) \ebd \lambda_{kk}-\lambda_{jj}
 +\langle f^j,f^k-f^j\rangle_{g^{-1}}- {}\\
 {}-\fr{z}{h}\,
 \|f^k-f^j\|^2_{g^{-1}} + \fr{1}{h}\langle y,f^k-f^j\rangle_{g^{-1}}.
 \label{eq:zeta_0_def}
 \end{multline}

 Непосредственно интегрировать абсолютную величину~$\gamma^{kj}$ 
 проблематично, так как
 $$
 \int\limits_{\mathbb{R}^M}|\gamma^{kj}(y)|\,dy= 
 \int\limits_{\mathbb{R}^M}|\gamma^{kj}(y, z^{kj}(y))|\,dy\,,
 $$ 
 а~зависимость~$z^{kj} (y)$ в~общем случае неизвестна. 
 Поэтому предварительно оценим~$|\gamma^{kj}|$ сверху. Прежде всего, 
 можно непосредственно проверить истинность следующего неравенства:
 \begin{multline}
 \| y - z^{kj}f^k-(h-z^{kj})f^j
 \|^2_{(hg)^{-1}} \geqslant{}\\
 {}\geqslant \|y\|^2_{(2hg)^{-1}} - \|z^{kj}f^k+(h-z^{kj})f^j\|^2_{(hg)^{-1}}.
 \label{eq:ineq_1}
 \end{multline}
 Отсюда следует, что
 \begin{multline}
 \hspace*{-3mm}h^2 \fr{\lambda_{kj}}{2}\,e^{\lambda_{jj}h+(\lambda_{kk}-\lambda_{jj})z}
 \mathcal{N}(y,z^{kj}f^k+(h-z^{kj})f^j,hg) ={} \hspace*{-1.51866pt}\\ 
 {}=
 h^2 \fr{\lambda_{kj}}{2}\,e^{\lambda_{jj}h+(\lambda_{kk}-\lambda_{jj})z}
 (2\pi)^{-M/2}|hg|^{-1/2} \times{}\\
 {}\times
 \exp  \left(
 -\fr{1}{2} \| y - z^{kj}f^k-(h-z^{kj})f^j
 \|^2_{(hg)^{-1}}
 \right) \leqslant {}\\ 
 {}\leqslant
  h^2 \fr{\lambda_{kj}}{2}\,e^{\lambda_{jj}h+(\lambda_{kk}-\lambda_{jj})z}
 (2\pi)^{-M/2}|hg|^{-1/2}\times{}\\
 {}\times
 \exp\left( 
 -\fr{1}{2}\|y\|^2_{(2hg)^{-1}}
 \right)
 \times{} \\ 
 {}\times
 \exp
 \left( 
 \fr{1}{2}\| z^{kj}f^k+(h-z^{kj})f^j \|^2_{(hg)^{-1}}
 \right) \leqslant{}\\
 {}\leqslant
  h^2C \mathcal{N}(y,0,2hg),
 \label{eq:gauss}
 \end{multline}
 где
 $$
 C \ebd \max\limits_{\substack{{k,j:}\\
 {k \neq j}}}\lambda_{kj}e^{({h}/{2})\max_k\|f^k\|_{g^{-1}}^2}.
 $$
 Тогда для интеграла от абсолютной величины ошибки верна 
 следующая оценка сверху:
 
 \noindent
  \begin{multline}
   \!\!\int\limits_{\mathbb{R}^M}\!\! |\gamma^{kj}(y)|\,dy 
   \leqslant  Ch^2\int_{\mathbb{R}^M}\mathcal{N}(y,0,2hg)|\zeta_0(y,z)|\,dy 
   \leqslant {}\hspace*{-1.41057pt}\\ 
   {}\leqslant
Ch^2\int\limits_{\mathbb{R}^M}
\left|
\vphantom{\fr{z}{h}}
\lambda_{kk}-\lambda_{jj}
 +\langle f^j,f^k-f^j\rangle_{g^{-1}}- {}\right.\\
\left. {}-\fr{z}{h}\|f^k-f^j\|^2_{g^{-1}}\right|
\mathcal{N}(y,0,2hg)\,dy +{} \\ 
{}+ 
Ch^2\int\limits_{\mathbb{R}^M}
\left|\fr{1}{h}\langle y,f^k-f^j\rangle_{g^{-1}}\right|
\mathcal{N}(y,0,2hg)\,dy ={} \\ 
{}= 
Ch^2\int\limits_{\mathbb{R}^M}
\left|
\vphantom{\fr{z}{h}}
\lambda_{kk}-\lambda_{jj}
 +\langle f^j,f^k-f^j\rangle_{g^{-1}}- {}\right.\\
\left. {}-\fr{z}{h}\|f^k-f^j\|^2_{g^{-1}}\right|
\mathcal{N}(y,0,2hg)\,dy +{} \\ 
{}+ 
\sqrt{2}Ch^{{3}/{2}}\int\limits_{\mathbb{R}^M}\!
\left|\fr{1}{h}\left\langle y,g^{-{1}/{2}}
\left(f^k-f^j\right)\right\rangle_{I}\right|\times{}\\
{}\times
\mathcal{N}(y,0,I)dy = K_1h^2 + K_2 h^{{3}/{2}}
\label{eq:ineq_2}
 \end{multline}
 для некоторых неотрицательных констант~$K_1$ и~$K_2$.
 Окончательно получаем, что 
 $$
 \int\limits_{\mathbb{R}^M}|\gamma^{kj}(y)|\,dy =O\left(h^{{3}/{2}}\right),
 $$
  и~в~этом 
 случае согласно~(\ref{eq:asympt_glob}) 
 $$\mathop{\mathrm{sup}}\limits_{\pi \in \Pi}
 \me{}{\|\widetilde{X}_{T/h} - \widehat{X}_{T/h}\|_1} 
 \leqslant CT h^{{1}/{2}}
 $$ 
 для некоторой константы $C \hm> 0$ и~достаточно малого шага~$h$. 
 Данный результат вполне согласуется с~известным фактом, согласно 
 которому порядок глобальной точности аппроксимации методом Эй\-ле\-ра--Ма\-ру\-ямы 
  решения СДУ, описывающего 
  фильтр Во\-нэ\-ма, равен~${1}/{2}$. Однако, в~отличие от последнего, 
  алгоритм~(\ref{eq:pp_est}), (\ref{eq:psi_left_tr}) обеспечивает 
  своим реализациям устойчивость.
  
  \subsection{Порядок $s=1$, схема <<средних>> прямоугольников}
  
   Несмотря на широкое применение метода <<левых>> прямоугольников, 
   его выбор при фиксированном порядке аналитической аппроксимации $s\hm=1$ 
   не является оптимальным: выбранная схема чис\-лен\-но\-го интегрирования снижает 
   общий порядок точности до~${1}/{2}$.

 Вновь рассмотрим аналитическую аппроксимацию~(\ref{eq:xi_1}) порядка $s\hm=1$ 
 и~для ее приближения вмес\-то~(\ref{eq:psi_left_tr}) используем схему 
 <<средних>> пря\-мо\-уголь\-ников:
 
 \noindent
 \begin{multline*}
 \psi^{kj}(y)=\delta_{kj}e^{\lambda_{jj}h}\mathcal{N}(y,hf^j,hg)
+{}\\
{}+
 (1-\delta_{kj})\lambda_{kj}he^{{(\lambda_{jj}+\lambda_{kk})h}/{2}}
 \mathcal{N}\!\left( \! y,\fr{h}{2}\left(f^j+f^k\right),hg\!\right)\!.
% \label{eq:psi_mid_tr}
 \end{multline*}
 При этом ошибка численного интегрирования определяется формулой:
 \begin{multline*} 
 \gamma^{kj}(y)=\left(1-\delta_{kj}\right)
 \fr{\lambda_{kj}h^3}{24}\,e^{\lambda_{jj}h}\times{}\\
 {}\times
 \fr{d^2}{du^2}\left[
 e^{(\lambda_{kk}-\lambda_{jj})u}\mathcal{N}(y,uf^k+(h-u)f^j,hg)
 \right]\Bigl|_{u=z} = {}\\ 
 {}= 
 \left(1-\delta_{kj}\right)\fr{\lambda_{kj}h^2}{2}\,e^{\lambda_{jj}h}
 e^{(\lambda_{kk}-\lambda_{jj})z}\times{}\\
 {}\times\mathcal{N}
 \left(y,zf^k+(h-z)f^j,hg\right) \left[\zeta_0^2(y,z)-\zeta_1\right],
 \end{multline*}
 где вновь $z = z (y) \hm\in [0,h]$~--- некоторый параметр, зависящий от~$y$, а
 \begin{equation} 
 \zeta_1 \ebd \fr{\partial }{\partial z}\zeta_0(y,z)=
 \fr{1}{h}\|f^j-f^k\|^2_{g^{-1}}.
 \label{eq:zeta_1_def}
 \end{equation}
 Используя неравенства~(\ref{eq:ineq_1}) и~(\ref{eq:gauss})  
 и~действуя аналогично выводу~(\ref{eq:ineq_2}), можно показать, 
 что для схемы <<средних>> прямоугольников верна оценка
$$
 \int\limits_{\mathbb{R}^M}|\gamma^{kj}(y)|\,dy = O(h^2),
$$
 и~в этом случае согласно~(\ref{eq:asympt_glob}) 
 $$
 \mathop{\mathrm{sup}}\limits_{\pi \in \Pi}
 \me{}{\|\widetilde{X}_{T/h} - \widehat{X}_{T/h}\|_1} \leqslant CT h
 $$ 
 для некоторой константы $C\hm > 0$ и~достаточно малого шага~$h$. 
 Таким образом, заменой одной схемы численного интегрирования другой,
  без увеличения вычислительных затрат, возможно повысить общий порядок 
  точ\-ности до первого.

Необходимо отметить, что дальнейшая фиксация порядка $s\hm=1$ и~использование 
более точных методов численного интегрирования не приведет к~значительному 
уточнению оценок, так как в~суммарной погрешности основную роль будет играть 
ошибка аналитической аппроксимации, а~не чис\-лен\-но\-го интегрирования. Для 
увеличения общей точности следует увеличить порядок аналитической 
аппроксимации до второго.

\vspace*{-6pt}

   \subsection{Порядок $s=2$, квадратуры Гаусса}
   
   Формулы (\ref{eq:h0})--(\ref{eq:h2}) для $s\hm=2$ позволяют получить 
   вид функций $\xi^{kj}$:
   
   \noindent
   \begin{multline}
   \xi^{kj}(y)= \delta_{kj}
   e^{\lambda_{jj}h}\mathcal{N}(y,hf^j,hg)+ {}\\
   {}+
 \left(1-\delta_{kj}\right)\lambda_{kj}e^{\lambda_{jj}h}
 \int\limits_0^h\! e^{(\lambda_{kk}-\lambda_{jj})u}\times{}\\
 {}\times
 \mathcal{N}\left(y,uf^k+(h-u)f^j,hg\right)du + {}\\ 
 {}+
 \sum\limits_{\substack{{i:i \neq k,}\\ {i \neq j}}} 
 \lambda_{k i}\lambda_{i j} e^{\lambda_{jj}h}
\int\limits_0^{h} \!\int\limits_0^{h-u}\!
e^{(\lambda_{kk}-\lambda_{i i})u+(\lambda_{i i}-\lambda_{jj})v}\times{} \\
{}\times
\mathcal{N}\left(y,uf^k+vf^{i}+(h -u-v )f^j, hg \right) dvdu\,.
 \label{eq:xi_2}
   \end{multline}
   В случае наблюдений с~аддитивными шумами для эффективного 
   использования повышения порядка
    аналитической аппроксимации~$s$ удается 
   подобрать
    схемы вычисления интегралов без дополнительного дробления шага~$h$. 
   Для вычисления \mbox{одномерного} интеграла в~(\ref{eq:xi_2}) будем использовать 
   двухточечную квадратуру Гаусса, для повторного интеграла~--- трехточечную:
   \begin{multline*}
   \int\limits_0^h e^{(\lambda_{kk}-\lambda_{jj})u}
   \mathcal{N}(y,uf^k+(h-u)f^j,hg)\,du
   = {}\\
   {} = 
   \fr{h}{2}
   \left[
   \vphantom{\fr{h}{2\sqrt{3}}}
   e^{(\lambda_{kk}-\lambda_{jj})(\sqrt{3}-1){h}/({2\sqrt{3}})}\times{}\right.\\
   {}\times
   \mathcal{N}\left(y,2hf^j + \fr{h}{2\sqrt{3}}
   \left(f^k-f^j\right),hg\right) +{}  \\
{} +
   e^{(\lambda_{kk}-\lambda_{jj})
   ({\sqrt{3}+1})h/({2\sqrt{3}})}\times{}\\
  \left. {}\times
   \mathcal{N}\left(y,2hf^k + 
   \fr{h}{2\sqrt{3}}\left(f^j-f^k\right),hg\right)
   \right]+e_1(y);
   \end{multline*}
   
\vspace*{-12pt}

   \noindent
   \begin{multline*}
   \int\limits_0^{h} \int\limits_0^{h-u}\!
e^{(\lambda_{kk}-\lambda_{i i})u+(\lambda_{i i}-\lambda_{jj})v}\times{}\\
{}\times
\mathcal{N}\left(y,uf^k+vf^{i}+(h - u- v )f^j,
 hg \right) dvdu ={} \\ 
 {}=
   \fr{h^2}{6}
   \left[
   \vphantom{\fr{2}{6}}
e^{(\lambda_{kk}-\lambda_{i i}){h}/{6}+(\lambda_{i i}-\lambda_{jj})
{h}/{6}}\times{}\right.\\
{}\times
\mathcal{N}\left(y,\fr{h}{6}\,f^k+\fr{h}{6}\,f^{i}+\fr{2h}{3}\,f^j,
 hg \right) + {}\\ 
 {}+ 
 e^{(\lambda_{kk}-\lambda_{i i}){2h}/{3}+
 (\lambda_{i i}-\lambda_{jj}){h}/{6}}\times{}\\
 {}\times
\mathcal{N}\left(y,\fr{h}{6}\,f^k+\fr{2h}{3}\,f^{i}+\fr{h}{6}\,f^j,
 hg \right) +{} \\ 
 {}+ 
 e^{(\lambda_{kk}-\lambda_{i i}){h}/{6}+
 (\lambda_{i i}-\lambda_{jj}){2h}/{3}}\times{}\\
 \left.{}\times
\mathcal{N}\left(y,\fr{2h}{3}\,f^k+\fr{h}{6}\,f^{i}+\fr{h}{6}\,f^j,
 hg \right)
    \right] + e_2(y),
   \end{multline*}
   где $e_1(y)$ и~$e_2(y)$~--- ошибки интегрирования. Согласно~\cite{IK_94}, 
   абсолютные значения ошибок ограничены следующим образом:
   
   \noindent
   \begin{multline}
   |e_1(y)| \leqslant h^5 K_3\max\limits_{u \in [0,h]}
   \left|\fr{\partial^4}{\partial u^4}
   \Big[e^{(\lambda_{kk}-\lambda_{jj})u}\times{}
   \right.\\
   \left.{}\times
   \mathcal{N}(y,uf^k+(h-u)f^j,hg)
   \Big]
   \vphantom{\fr{\partial^4}{\partial u^4}}
   \right|;
   \label{eq:e_1_def}
   \end{multline}
   
\vspace*{-12pt}

   \noindent
   \begin{multline*}
   |e_2(y)| \leqslant {}\\
   {} \leqslant h^5 K_4
   \max\limits_{\substack{{(u,v) \in \mathcal{D},}\\
{k=1,2,3}}}
   \left\vert
   \fr{\partial^3}{\partial u^k\partial v^{3-k}}\Big[
   e^{(\lambda_{kk}-\lambda_{i i})u+(\lambda_{i i}-\lambda_{jj})v}\times{}\right.\\
   \left.{}\times
\mathcal{N}\left(y,uf^k+vf^{i}+(h - u- v )f^j, hg \right)
 \Big]
 \vphantom{   \fr{\partial^3}{\partial u^k\partial v^{3-k}}}
 \right\vert,
 %\label{eq:e_2_def}
   \end{multline*}
   где $K_3$ и~$K_4$~--- некоторые положительные кон\-станты.

   Производная в~(\ref{eq:e_1_def}) имеет вид:
   \begin{multline*}
     \fr{\partial^4}{\partial u^4}\left[e^{(\lambda_{kk}-\lambda_{jj})u}
   \mathcal{N}(y,uf^k+(h-u)f^j,hg)\right]={} \\ 
   {}=
   e^{(\lambda_{kk}-\lambda_{jj})u}
   \mathcal{N}\left(y,uf^k+(h-u)f^j,hg\right)\times{}\\
   {}\times\left(\zeta_0^4(y,z)+
   6\zeta_0^2(y,z)\zeta_1 +3\zeta_1^2\right),
      \end{multline*}
      где $\zeta_0$ и~$\zeta_1$ определены формулами~(\ref{eq:zeta_0_def}) 
      и~(\ref{eq:zeta_1_def}).
      Строя оценки сверху интеграла от абсолютного значения~$e_1(y)$ 
      подобно~(\ref{eq:ineq_2}), можно получить неравенство
\begin{equation*}
\int\limits_{\mathbb{R}^M}|e_1(y)|\,dy \leqslant K_5 h^3,
\end{equation*}
и аналогичная оценка для $|e_2(y)|$ имеет вид:
\begin{equation*}
\int\limits_{\mathbb{R}^M}|e_2(y)|\,dy \leqslant K_6 h^3
\end{equation*}
для некоторых неотрицательных констант~$K_5$ и~$K_6$. 
В~этом случае согласно~(\ref{eq:asympt_glob}) 
$$
\mathop{\mathrm{sup}}\limits_{\pi \in \Pi}\me{}{\|\widetilde{X}_{T/h}
- \widehat{X}_{T/h}\|_1} \leqslant CT h^2
$$ 
для некоторой константы $C 
\hm> 0$ и~достаточно малого шага~$h$. Таким образом, путем незначительного 
увеличения вычислительных затрат возможно повысить общий порядок 
точности аппроксимации до второго.

\vspace*{-9pt}

 \section{Заключение}

 На основании результатов~\cite{B_18_IA, B_20_1_IA}
 в~статье
 раз\-работан ряд численных алгоритмов решения зада-\linebreak чи фильтрации 
 состояний однородных МСП по дискретизованным косвенным наблюдениям 
 в~присутствии аддитивных винеровских шумов. Эти алгоритмы имеют 
 одинаковую структуру: оценки считаются рекуррентно в~виде дроби, 
 в~которой чис\-ли\-тель и~знаменатель являются суммами некоторых интегралов. 
 Отличие алгоритмов заключается лишь в~числе сла\-га\-емых в~этих суммах 
 ($s\hm=1, 2$) и~схеме, реализующей численное интегрирование. 
 Примечательно, что в~случае аддитивных шумов в~наблюдениях 
 интегралы таковы, что для их чис\-лен\-но\-го интегрирования с~выбранной 
 точностью не требуется дополнительного дроб\-ле\-ния интервала отрезка 
 интегрирования. В~итоге получены аппроксимации с~точ\-ностью порядка 
 ${1}/{2}$, 1 и~2.
 
 \vspace*{-9pt}
 
{\small\frenchspacing
 {%\baselineskip=10.8pt
 \addcontentsline{toc}{section}{References}
 \begin{thebibliography}{9}
  \bibitem{B_18_IA}
  \Au{Борисов~А.} Фильтрация состояний марковских скачкообразных процессов 
  по дискретизованным наблюдениям~// Информатика и~её применения,~2018. 
  Т.~12.~Вып.~3.~C.~115--121. doi: 10.14357/19922264180316.

  \bibitem{B_20_1_IA}
  \Au{Борисов А.} Численные схемы фильтрации марковских скачкообразных 
  процессов по дискретизованным наблюдениям~I:  характеристики точности~// 
  Информатика и~её применения,~2019. Т.~13.~Вып.~4. C.~68--75. 
  doi: 10.14357/1992226419041.

\bibitem{Wonham_64}
\Au{Wonham W.} Some applications of stochastic differential 
equations to optimal nonlinear
filtering~// SIAM J.~Control Optim., 1964. Vol.~2. No.\,3. P.~347--369. doi: 
10.1137/0302028.

\bibitem{B_18}
\Au{Борисов А.} Фильтрация Вонэма по наблюдениям с~мультипликативными шумами~// 
Автоматика и~телемеханика, 2018.
№\,1. C.~52--65. %doi: 10.1134/S0005117918010046.

\bibitem{KP_92}
\Au{Kloeden P., Platen~E.} Numerical solution of stochastic differential 
equations.~--- Berlin: Springer, 1992. 636~p. %doi: 10.1007/978-3-662-12616-5.

\bibitem{YZL_04}
\Au{Yin G., Zhang~Q., Liu~Y.}
Discrete-time approximation of Wonham filters~//
J.~Control Theory Applications, 2004. No.\,2. P.~1--10.

\bibitem{PR_10}
\Au{Platen E., Rendek~R.}
Quasi-exact approximation of hidden Markov chain filters~// 
Commun. Stoch. Anal., 2010. Vol.~4. Iss.~1. P.~129--142.

\bibitem{BGH_16}
\Au{B$\ddot{\mbox{a}}$uerle~N., Gilitschenski~I., Hanebeck~U.} 
Exact and approximate hidden Markov chain filters based on discrete 
observations~// Statistics Risk Modeling, 2016. Vol.~32. Iss.~3-4.
P.~159--176.

\bibitem{IK_94}
\Au{Isaacson E., Keller~H.} Analysis of numerical methods.~--- 
New York, NY, USA: Dover Publications, 1994. 541~p.
 \end{thebibliography}

 }
 }

\end{multicols}

\vspace*{-3pt}

\hfill{\small\textit{Поступила в~редакцию 11.10.19}}

%\vspace*{8pt}

%\pagebreak

\newpage

\vspace*{-28pt}

%\hrule

%\vspace*{2pt}

%\hrule

%\vspace*{-2pt}

\def\tit{NUMERICAL SCHEMES OF~MARKOV JUMP PROCESS FILTERING GIVEN 
DISCRETIZED OBSERVATIONS II:~ADDITIVE NOISE CASE}


\def\titkol{Numerical schemes of~Markov jump process filtering given 
discretized observations II:~Additive noise case}

\def\aut{A.\,V.~Borisov}

\def\autkol{A.\,V.~Borisov}

\titel{\tit}{\aut}{\autkol}{\titkol}

\vspace*{-11pt}


\noindent
Institute of Informatics Problems, Federal Research Center 
``Computer Science and Control'' of the Russian
Academy of Sciences, 44-2~Vavilov Str., Moscow 119333, Russian Federation

\def\leftfootline{\small{\textbf{\thepage}
\hfill INFORMATIKA I EE PRIMENENIYA~--- INFORMATICS AND
APPLICATIONS\ \ \ 2020\ \ \ volume~14\ \ \ issue\ 1}
}%
 \def\rightfootline{\small{INFORMATIKA I EE PRIMENENIYA~---
INFORMATICS AND APPLICATIONS\ \ \ 2020\ \ \ volume~14\ \ \ issue\ 1
\hfill \textbf{\thepage}}}

\vspace*{3pt} 



\Abste{The note is a sequel of investigations initialized in 
the article Borisov,~A. 2019. Numerical schemes of Markov jump 
process filtering given discretized observations~I: Accuracy characteristics. 
\textit{Inform.~Appl.} 13(4):68--75. 
The basis is the accuracy characteristics of the approximated solution 
of the filtering problem for the state of homogeneous Markov jump 
processes given the continuous indirect noisy observations. The 
paper presents a~number of the algorithms of their numerical 
realization together with the comparative analysis. The class 
of observation systems under investigation is bounded by ones 
with additive observation noises. This presumes that the observation 
noise intensity is a~nonrandom constant. To construct the approximation, 
the authors use the left and midpoint rectangle rule of the accuracy 
order~2 and~3, respectively, and the Gaussian quadrature of the order~5. 
Finally, the presented numerical schemes have the accuracy of the 
order~${1}/{2}$, 1, and~2.}

\KWE{Markov jump process; optimal filtering; additive and multiplicative 
observation noises; stochastic differential equation; analytical 
and numerical approximation}



\DOI{10.14357/19922264200103} 

%\vspace*{-14pt}

\Ack
\noindent
The work was supported in part by the Russian Foundation
for Basic Research (Project No.\,19-07-00187~A).


%\vspace*{6pt}

  \begin{multicols}{2}

\renewcommand{\bibname}{\protect\rmfamily References}
%\renewcommand{\bibname}{\large\protect\rm References}

{\small\frenchspacing
 {%\baselineskip=10.8pt
 \addcontentsline{toc}{section}{References}
 \begin{thebibliography}{9}


  \bibitem{B_18_IA-1}
\Aue{Borisov, A.} 2018. Fil'tratsiya sostoyaniy markovskikh skachkoobraznykh 
protsessov po diskretizovannym nablyudeniyam [Filtering of Markov 
jump processes by discretized observations].
\textit{Informatika i~ee Primeneniya~--- Inform.~Appl.}
12(3):115--121.~doi:~10.14357/19922264180316.

  \bibitem{B_20_1_IA-1}
\Aue{Borisov, A.} 2019. Chislennye skhemy fil'tratsii 
mar\-kov\-skikh skachkoobraznykh protsessov po dis\-kre\-ti\-zo\-van\-nym nablyudeniyam~I:  
kharakteristiki tochnosti [Numerical schemes of Markov jump process 
filtering given discretized observations I: Accuracy characteristics].
\textit{Informatika i~ee Primeneniya~--- Inform.~Appl.}
13(4):68--75.~doi:~10.14357/19922264190411.

\bibitem{Won_65-1}
\Aue{Wonham, W.} 1964.
Some applications of stochastic differential equations to optimal
  nonlinear filtering. \textit{SIAM~J.~Control Optim.} 2:347--369. 

\bibitem{B_18-1}
\Aue{Borisov, A.} 2018. 
Wonham filtering by observations with multiplicative noises.
\textit{Automat.~Rem.~Contr.} 79(1):39--50.  
doi: 10.1134/S0005117918010046.

\bibitem{KP_92-1}
\Aue{Kloeden,~P., and E.~Platen.} 1992.
\textit{Numerical solution of stochastic
differential equations.} Berlin: Springer. 636~p.

\bibitem{YZL_04-1}
\Aue{Yin, G., Q.~Zhang, and Y.~Liu.} 2004. 
Discrete-time approximation of Wonham filters.
\textit{J.~Control Theory Applications} 2:1--10.

\bibitem{PR_10-1}
\Aue{Platen, E., and R.~Rendek.} 2010.
Quasi-exact approximation of hidden Markov chain filters.
\textit{Commun. Stoch. Anal.} 4(1):129--142.

\bibitem{BGH_16-1}
\Aue{B$\ddot{\mbox{a}}$auerle, N., I.~Gilitschenski, and U.~Hanebeck.} 2016. 
Exact and approximate hidden Markov chain filters based on discrete observations.
\textit{Statistics Risk Modeling} 32(3-4):159--176.

\bibitem{IK_94-1}
\Aue{Isaacson, E., and H.~Keller.} 1994.
\textit{Analysis of numerical methods.} New York, NY: Dover Publications.
541~p.
\end{thebibliography}

 }
 }

\end{multicols}

%\vspace*{-7pt}

\hfill{\small\textit{Received October 11, 2019}}

%\pagebreak

%\vspace*{-22pt}


\Contrl


\noindent
\textbf{Borisov Andrey V.} (b.\ 1965)~--- Doctor of Science in physics and 
 mathematics, principal scientist, Institute of Informatics Problems, 
 Federal Research Center ``Computer Science and Control'' of the Russian 
 Academy of Sciences, 44-2~Vavilov Str., Moscow 119333, Russian 
 Federation; \mbox{aborisov@frccsc.ru}
\label{end\stat}

\renewcommand{\bibname}{\protect\rm Литература} 