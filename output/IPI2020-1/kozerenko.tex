\def\stat{kozerenko}

\def\tit{АНАЛИТИЧЕСКАЯ ТЕКСТОЛОГИЯ В~СИСТЕМАХ ИНТЕЛЛЕКТУАЛЬНОЙ 
ОБРАБОТКИ\\ НЕСТРУКТУРИРОВАННЫХ ДАННЫХ$^*$}

\def\titkol{Аналитическая текстология в~системах интеллектуальной 
обработки неструктурированных данных}

\def\aut{Е.\,Б.~Козеренко$^1$, М.\,Ю.~Михеев$^2$, Н.\,В.~Сомин$^3$, 
Л.\,И.~Эрлих$^4$, К.\,И.~Кузнецов$^5$}

\def\autkol{Е.\,Б.~Козеренко, М.\,Ю.~Михеев, Н.\,В.~Сомин и~др.}
%Л.\,И.~Эрлих$^4$, К.\,И.~Кузнецов$^5$}

\titel{\tit}{\aut}{\autkol}{\titkol}

\index{Козеренко Е.\,Б.}
\index{Михеев М.\,Ю.}
\index{Сомин Н.\,В.} 
\index{Эрлих Л.\,И.}
\index{Кузнецов К.\,И.}
\index{Kozerenko E.\,B.}
\index{Mikheev M.\,Y.}
\index{Somin N.\,V.} 
\index{Ehrlich L.\,I.}
\index{Kuznetsov K.\,I.}


{\renewcommand{\thefootnote}{\fnsymbol{footnote}} \footnotetext[1]
{Работа выполнена при частичной поддержке РФФИ (проект 18-012-00220-а).}}


\renewcommand{\thefootnote}{\arabic{footnote}}
\footnotetext[1]{Институт проблем информатики Федерального исследовательского центра 
<<Информатика и~управление>> Российской академии наук, \mbox{kozerenko@mail.ru}}
\footnotetext[2]{Научно-исследовательский вычислительный центр Московского государственного 
университета им.\ М.\,В.~Ломоносова, \mbox{mihej57@yandex.ru}}
\footnotetext[3]{Институт проблем информатики Федерального исследовательского центра 
<<Информатика и~управление>> Российской академии наук, \mbox{chri-soc@yandex.ru}}
\footnotetext[4]{Научно-исследовательский вычислительный центр Московского государственного 
университета им.\ М.\,В.~Ломоносова, \mbox{levehr@yandex.ru}}
\footnotetext[5]{Институт проблем информатики Федерального исследовательского центра 
<<Информатика и~управление>> Российской академии наук, \mbox{k.smith@mail.ru}}

%\vspace*{-12pt}

  

      \Abst{Представлено новое направление исследований на пересечении лингвистики, 
информатики и~филологии с~привлечением ло\-ги\-ко-ста\-ти\-сти\-че\-ских методов 
анализа неструктурированных данных в~виде ес\-тест\-вен\-но-язы\-ко\-вых текстов 
с~целью решения целого ряда задач извлечения эксплицитных и~имплицитных знаний из 
текстов с~использованием се\-ман\-ти\-че\-ски-ори\-ен\-ти\-ро\-ван\-но\-го 
лингвистического процессора (СОЛП), формирования лек\-си\-ко-ста\-ти\-сти\-че\-ских 
представлений текстов, построения аналитических заключений, определения идиостиля 
автора и~текстуального сходства произведений на основе анализа служебных слов 
и~других микротекстовых элементов; выявления эмоциональной окрашенности текстов, 
построения полного профиля авторского текста на основе суперпозиции методов. 
Рассматривается пример текстологического анализа <<Синей книги>> из <<Петербургского 
дневника>> З.\,Н.~Гиппиус.}
      
      \KW{обработка естественного языка; статистические методы; когнитивные 
технологии; лек\-си\-ко-се\-ман\-ти\-че\-ский анализ; извлечение знаний из текстов; 
аналитические системы} 

\DOI{10.14357/19922264200115} 
  
\vspace*{2pt}


\vskip 10pt plus 9pt minus 6pt

\thispagestyle{headings}

\begin{multicols}{2}

\label{st\stat}
      
    \section{Введение}
    
    \vspace*{-5pt}
    
     В статье рассматриваются актуальные задачи и~методы обработки 
естественно-языковых текс\-тов с~целью решения целого ряда задач 
извлечения\linebreak эксплицитных и~имплицитных знаний из текс\-тов~[1] 
с~использованием се\-ман\-ти\-чески-ори\-ен\-ти\-ро\-ван\-но\-го 
лингвистического процессора~[2], формирования  
лек\-си\-ко-ста\-ти\-сти\-че\-ских представлений текстов~[3--8], построения 
аналитических заключений~[9, 10], определения идиостиля автора 
и~текстуального сходства произведений на основе анализа служебных слов 
и~других микротекстовых элементов~[11]; выявления эмоциональной 
окрашенности текстов, построения полного профиля авторского текста на 
основе суперпозиции методов. Настоящая работа ставит своей целью описать 
методы аналитической текстологии и~их возможные применения для 
обработки неструктурированных данных~[12--15]. 
     
     Способы представления информации, знаний многообразны. Огромный 
объем данных пред\-став\-лен в~виде текстов естественного языка, что делает 
задачу извлечения и~структурирования информации из текстов весьма 
важной. Это относится к~различным предметным областям. Для 
оперирования данными на компьютере необходимо выделить из текста 
объекты, их атрибуты, связи между объектами, процессы, в~которых эти 
объекты задействованы, другую важную информацию, которая бы позволяла 
не только описать ситуацию, но и~строить выводы, характерные для 
конкретной предметной области, прогнозировать развитие  
ситуации~[14--18].
     
     Решение проблемы связано с~анализом больших массивов текстов 
русского языка на основе технологической цепочки, организованной как 
последовательное применение инструментов построения  
лек\-си\-ко-ста\-ти\-сти\-че\-ско\-го представления ис\-сле\-ду\-емо\-го текс\-та, 
извлечения именованных сущностей и~связей, определения идиостиля автора 
на основе служебных слов и~микротекстовых элементов и~определения 
близости текстов различных авторов.
{\looseness=-1

}
    
    \section{Формирование лексико-статистических 
представлений текстов}
     
     Для получения лексико-частотных характеристик текстов была 
разработана программная система ISV (Intelligent Statistical Verbalizer), 
реализующая три основные функции анализа текстов (разработчик 
программного обеспечения~--- Н.\,В.~Сомин):
     \begin{enumerate}[(1)]
     \item лексический анализ, определяющий лексические особенности 
лексем;
     \item морфологический анализ русских и~английских слов с~выдачей 
канонической формы слова;
     \item каунтинг, т.\,е.\ подсчет числа вхождений слов 
с~предварительным вычислением их канонической формы. 
     \end{enumerate}
     
     Необходимость разработки вербалайзера обусловлена прежде всего 
расширением и~усложнением работ, связанных с~аналитической обработкой 
текстов на естественном языке. В~очень многих задачах предварительный 
морфологический анализ текста позволяет получить очень важную 
информацию о лексемах, которая в~дальнейшем может использоваться для 
синтаксического анализа, семантического анализа и~других фаз 
аналитической обработки текстов. 
     
     Предложим пример. Так, в~комплексе задач, связанных 
с~обнаружением плагиата и~выяснением авторства, очень важен 
статистический анализ текстов. Однако прямой подсчет вхождений лексем 
оказывается малоэффективным, поскольку статистика числа вхождений 
<<расплывается>> по различным словоформам одного и~того же слова и~не 
дает объективной картины. Гораздо более точные результаты можно 
получить, применяя подсчет числа вхождений не для словоформ, 
а~канонических форм слова.
     
     Помимо этого, поскольку очень часто верхние\linebreak строчки файла 
вхождений занимают предлоги, союзы, местоимения и~другие 
вспомогательные слова, то в~процессе анализа их следует отделять в~особую 
группу (для последующей обработки \mbox{микротекстовых} элементов), для чего 
необходим достаточно глубокий морфологический анализ. 
     
     Кроме того, с~помощью морфологического анализа возможен подсчет 
статистики по частям речи: только для существительных, только для 
глаголов и~т.\,д., что позволит сделать анализ более дифференцированным, 
а~значит выявить более тонкие закономерности.
     
     Таким образом, одним из требований, предъявляемых к~вербалайзеру, 
должна быть возможность комбинирования всех его возможностей 
с~выбором необходимых для данной задачи.
     
     Другое существенное требование~--- обеспечение высокой 
эффективности вычислений. Анализ больших текстов позволяет строить, 
например, терминологические портреты многих предметных областей, 
а~также выявлять закономерности связей между текстами.
     
     Чтобы успешно обрабатывать тексты больших объемов, необходима 
очень высокая скорость поиска и~осуществления морфологического анализа. 
Это требует включения в~вербалайзер техники обработки больших данных 
(big data), а также разработки особой структуры информационных массивов 
вербалайзера. 
     
     Данным блоком прежде всего решается задача структуризации текста. 
От правильного распознавания структуры текста в~значительной степени 
зависит корректность всего анализа. При этом задача структуризации 
распадается на цепочку локальных задач: выделение из входного потока 
лексем; выделение предложений; выделение абзацев; унификация текста; 
исправление опечаток и~грамматических ошибок; определение лексических 
признаков слов.
     
     Отметим, что блок морфологического анализа выдает информацию 
о~97~различных морфологических признаках русского и~английского 
языков, которых достаточно для анализа самых сложных языковых нюансов. 
В~общем случае морфологический блок выдает несколько вариантов 
морфологического анализа (так называемая морфологическая омонимия). 
Это свойство морфологического анализа обеспечивается особой структурой 
словарей, а~именно: в~Словаре основ (СО) может быть несколько записей 
с~одинаковой основой (но с~разными классами окончаний), а на один и~тот 
же класс окончаний может ссылаться несколько слов с~разными основами. 
Возможны случаи пустой основы (пример: <<хорошо>>--<<лучше>>) 
и~пустого класса окончаний (для неизменяемых слов). Кроме основы 
и~вариантов окончаний в~Словаре классов окончаний (СКО) хранятся 
морфологические признаки, соответствующие определенному классу 
окончаний в~целом (постоянная морфологическая информация) и~каждому 
окончанию парадигмы в~отдельности (переменная морфологическая 
информация). Алгоритм предполагает, что в~общем случае 
могут быть найдены несколько вариантов морфологического разбора. Этот 
факт хорошо известен лингвистам как морфологическая омонимия. 
Например, слово <<стекло>> имеет по крайней мере два варианта разбора: 
как существительное (вставить \textit{стек\-ло}) и~как глагол (что-то 
\textit{стек\-ло} с~крыши). 
    
    \section{Концептуальный анализ текстов на~основе 
семантически-ориентированного лингвистического процессора}
    
     Задача автоматического анализа текстовой информации, 
представленной в~интернете, актуальна во всем мире. Для решения полного 
спект\-ра задач обработки естественного языка создан  
СОЛП~\cite{2-koz}. Центральным компонентом СОЛП является 
инструментальный пакет (SDK-модуль) PullEnti (Puller of Entities). 
Этот процессор в~рамках 
соревнований, проводившихся на конференции <<Диалог-2016>>, занял два 
первых места при анализе текстов в~рамках решения задач извлечения 
именованных сущностей. Разработчик PullEnti~--- Константин Игоревич 
Кузнецов. В~системе PullEnti используются динамически подключаемые 
компоненты (плагины), что позволяет без перекомпилирования запускать 
различные функциональные возможности. Именно таким образом 
активируется блок семантического анализа.
     
     В процессе анализа выделяются семантические единицы (токены), 
которые представляют собой типизированные фразы, такие как текстовые, 
чис\-ло\-вые и~т.\,д. Например, в~результате анализа фразы <<в~1917~году>> 
будут выделены три токена: <<в>>~--- текс\-то\-вый, <<году>>~--- текс\-то\-вый, 
<<1917>>~--- чис\-ло\-вой. Такие токены можно назвать простыми. \mbox{Кроме} того, 
выделяются метатокены~--- сложные токены, которые объединяют 
несколько прос\-тых токенов, например существительные с~определителями, 
скоб\-ки, кавычки и~т.\,п.
     
     Первый этап концептуального анализа текстовых сообщений~--- 
выделение параметров команд. Этот этап проводится с~помощью 
инструментального пакета PullEnti, предназначенного для 
решения задачи выделения именованных сущностей, их свойств и~связей из 
неструктурированных русскоязычных текстов в~рамках информационных 
систем, разрабатываемых на .NET Framework~2.0 и~выше. PullEnti состоит из 
общей и~специализированной частей. Общая часть обеспечивает реализацию 
общих алгоритмов морфологического и~синтаксического анализа, а~также 
поддержку модели данных. Специализированная часть состоит из отдельных 
сборок (анализаторов), реализующих выделение именованных сущностей 
определенных типов (персоны, даты, локации, организации и~др.). 
Предполагается также выявление ассоциативных связей между выделенными 
сущностями в~определенной предметной области. При этом для расчета силы 
ассоциативной связи между именованными сущностями используется 
косинусная мера между контекстными векторами (компонентами вектора 
именованных сущностей служат частоты их совместной встречаемости 
в~одном и~том же контексте). Такие векторы образуют семантическое 
контекстное пространство. Для вычисления косинусной меры между 
контекстными векторами используется следующая формула:
     $$
     \fr{x\cdot y}{\vert x\vert \cdot \vert y\vert} =\fr{\sum\nolimits^n_{i=1} x_i 
y_i} {\sqrt{\sum\nolimits^n_{i=1} x_i^2}\sqrt{\sum\nolimits^n_{i=1} y_i^2}}\,.
     $$
    
     В зависимости от того, какие контексты считаются идентичными, 
различают типы контекстов. Классификация и~даже перечисление типов 
контекстов~--- проблема, которая в~силу своей новизны требует особого 
рассмотрения~\cite{7-koz, 16-koz}. 
     
   \section{Стейплинг: построение идиостиля автора 
и~определение близости текстов}
    
     Метод стейплинга (от английского staple~--- скрепа) основан на 
результатах исследований М.\,Ю.~Михеева и~Л.\,И.~Эрлиха, в~которых 
продемонстрирована высокая степень информативности слов закрытых 
классов (служебных слов~--- союзов, предлогов, частиц,~--- дискурсивных 
слов, вводных оборотов, устойчивых наречных конструкций и~т.\,п.)\ для 
решения задач определения авторства текста по частотам служебных слов, 
построения идиостилистического профиля автора, установления бли\-зости 
текстов, определения текстовых заимствований~\cite{11-koz}. Такие 
элементы языка получили название языковой, или текстовой, скрепы (staple) 
и~служат потенциальными статистическими маркерами (СМ)
автора. 

Стейплинг-под\-ход к~текстологическим 
исследованиям является новым и~менее из\-вес\-тен, 
чем составление авторских словарей ключевых и~опорных слов. Автора и~его 
текст могут исчерпывающим образом характеризовать не только ключевые 
слова, но и~слова наименее значимые, при помощи списка из сотни наиболее 
частотных подобных скреп можно будет находить реального автора текста, 
определять близость исследуемого текста к~стилистике текстов базы 
сравнения. Для составления 100-слов\-но\-го, а~затем 300-слов\-но\-го 
списка русских служебных слов и~выражений, т.\,е.\ наиболее частых в~языке 
союзов, предлогов, час\-тиц, дискурсивных слов, вводных оборотов, 
устойчивых наречных конструкций, фразеологизмов, стандартных средств 
пара- и~гипотаксиса, а~также наиболее употребительных\linebreak сочетаний из них, 
проведены исследования в~рамках двух этапов проекта РФФИ <<Создание 
алгоритма идентификации авторского идиостиля на основании час\-тот\-ности 
употребления служебных слов>> (руководитель М.\,Ю.~Михеев). 
В~настоящее время ведутся работы по расширению, уточнению списка 
скреп, рас\-смот\-ре\-нию их контекстных характеристик, выявлению 
дистрибутивных признаков этих элементов текста.
     
     Для проведения достоверных исследований необходима база авторских 
текстов. За исходную базу взят Национальный корпус русского языка
(НКРЯ), а~из 
него~--- тексты семи наиболее известных русских  
пи\-са\-те\-лей-про\-заи\-ков XIX--XX~вв.: Гоголя, Тургенева, 
Достоевского, Толстого, Бунина, Горького и~Набокова. В этих подкорпусах 
подсчитано чис\-ло употреблений всех единиц 100-слов\-но\-го списка 
и~вычислены их частоты (в~миллионных долях~--- ipm, или миллипромилле, 
т.\,е.\ 1/10000-й части процента). 
     
     Как показали системные исследования авторских текстов, высокой 
информативностью обладают относительные частоты маркеров: более 
наглядным представляется сравнивать не сами значения частот данного 
СМ в~разных текстах разных авторов, а их отношения 
к~среднему уровню, т.\,е.\ выраженное в~процентах отношение частоты  
ка\-кой-то конкретной скрепы у конкретного автора к~средней частоте этой 
же скрепы по НКРЯ. Метод стейплинга эффективно применяется для 
создания идиостилистического профиля автора. Истоки метода 
обнаруживаются в~первых исследованиях идиостиля и~связаны с~именами 
Ю.\,Н.~Тынянова, Ю.\,Н.~Караулова и~В.\,В.~Виноградова. В~част\-ности, 
В.\,В.~Виноградов ввел термин <<языковая личность>>. Идиостиль~--- это 
система содержательных и~формальных лингвистических характеристик, 
присущих произведениям определенного автора, которая делает уникальным 
воплощенный в~этих произведениях авторский способ языкового выражения. 
Идиостиль близок к~понятию идио\-лект. 
{\looseness=1

}
     
     В задаче распознавания авторства более важным оказывается не то, 
{\bfseries\textit{о~чем}} говорит автор, а~то, {\bfseries\textit{как}} он это 
говорит. В~лингвостатистике существует 100-слов\-ный список Сводеша, 
задающий, как известно, лексику, наименее подверженную изменениям 
в~данном языке, по которой можно рассчитать скорость синонимических 
замен базового лексического фонда. Сам список предполагается примерно 
одинаковым для любых языков и~служит как бы <<лингвистическими 
часами>>~--- по изменениям в~нем можно определить время распада\linebreak языка. 
     
     Методика выявления идиостилистического профиля автора: выявляется 
набор наиболее час\-тых в~его текстах, характерных именно для него 
служебных единиц языка, элементов 100-слов\-но\-го списка. Для краткости 
далее все элементы такого списка будут именоваться просто текстовыми, или 
языковыми скрепами. Они же выступают потенциальными 
СМ стиля писателя, обособляя один идиостиль от другого. 
    
   \section{Полный профиль авторского текста (на примере 
<<Синей книги>> из <<Петербургского дневника>> 
З.\,Н.~Гиппиус)}
   
    
     Авторский текст~--- непосредственный вход в~интеллектуальный 
и~эмо\-цио\-наль\-но-аф\-фек\-тив\-ный мир человека, данный нам 
в~<<ощущении через восприятие текста>>, актуализированные мысль 
и~чувство пишущего. В~дополнение к~текстам, представленным 
в~НКРЯ был проведен ряд исследований 
текстов <<Петербургского дневника>> З.\,Н.~Гиппиус~\cite{9-koz, 10-koz}. 
Несмотря на большое влияние поэта, критика и~философа З.\,Н.~Гиппиус на 
современников, ее произведения~\cite{19-koz, 20-koz} относительно мало 
изучены отечественной лингвистикой в~связи с~их не\-до\-ступ\-ностью 
в~советский период. На примере анализа текста <<Синей книги>> 
рассмотрим последовательное применение методов аналитической 
текс\-то\-логии.
     
     <<Синяя книга>>~--- первая часть <<Петербургского дневника>> 
З.\,Н.~Гиппиус~\cite[с.~51--241; 19]{18-koz}~--- содержит описание 
хронологии событий и~настроений с~августа 1914~г.\ по ноябрь 1917~г., 
впечатления и~мысли автора. Ярко и~образно представлены события 
предреволюционного периода и~непосредственно двух всплесков 
революции~--- в~феврале и~октябре 1917~г. Построение 
лек\-си\-ко-ста\-ти\-сти\-че\-ско\-го образа текста с~по\-мощью системы ISV позволило 
получить наиболее частотные ключевые слова и~скрепы. Далее были 
выделены семантически значимые объекты, именованные сущности и~связи 
с~использованием СОЛП.
     
     Рассмотрим полученные результаты.
     
     Наиболее частотные значимые (опорные) слова~--- <<герои 
повествования>>: война, революция.
     
     Базовые семантические классы (типы именованных сущностей): даты, 
события, люди, стихии, атмосфера, общий строй жизни, настроение автора, 
впечатления и~оценки автора.
     
     Основная тема предреволюционного периода звучит рефреном: тяжесть 
войны и~ожидание надвигающейся катастрофы: <<Война~--- в~статике>>, 
<<Греция замерла>>, <<никому нет никуда выхода. И~не предвидится>>; 
<<Ка\-кая-то ЧРЕВАТОСТЬ в~воздухе; ведь нельзя же только~--- 
ЖДАТЬ!>>; <<В~атмосфере глубокий и~зловещий ШТИЛЬ>>; 
<<Оцепенели>>; <<Спокойствие$\ldots$ отчаянья>>.
     
     Ключевые узлы развития событий по датам:
     
     Зима 1916~года; Лето 1916~года; 1917 2~февраля. Четверг; 11~февраля. 
Суббота; 22~февраля. Среда; 23~февраля. Четверг: начало движения; 
<<26~февраля. Воскресенье>>; 27~февраля. Понедельник 12~ч.\ дня; 2~ч.\ 
дня; 3~ч.\ дня; 4~1/2~часа; 5~часов; 5~1/2.
     
     Персоны: Чхенкели, Вильсон, Керенский, Милюков, Гришка, 
Пуришкевич, Протопопов, Род\-зян\-ко, Брусилов, Рузский, ген.\ Алексеев, 
Коновалов, Дмитрюков, Чхеидзе, Шульгин, Шидловский, Милюков, 
Караулов, Львов и~Ржевский.
     
     Локации: Санкт-Петербург, Выборгская сторона, Таврический сад.
     
     Организации: Дума, Городская Дума, Комитет <<для водворения 
порядка и~для сношения с~учреж\-де\-ни\-ями и~лицами>>.
     
     Оценочные суждения автора: <<При этом плохо везде. Истощение 
и~неустройство>>; <<У нас особенно худо. Нынешняя зима впятеро тяжелее 
и~дороже прошлогодней. Рядом~--- постыдная роскошь наживателей>>; 
<<Грозная, страш\-ная сказка>>; <<$\ldots$столько знакомых, милых лиц, 
молодых и~старых>>; <<Но все лица, и~незнакомые,~--- милые, радостные, 
верящие ка\-кие-то$\ldots$ Незабвенное утро, алые \mbox{крылья} и~марсельеза 
в~снежной, золотом отливающей, белости$\ldots$>>
     
     Скрепы в~<<Синей книге>> образуют отчетливое тематическое гнездо 
<<неопределенности>>: какая-то, кто-то где-то, будто бы, кем-то и~другие:
никому нет никуда $\{$выхода$\}$; какая-то $\{$ЧРЕВАТОСТЬ$\}$; никто 
не $\{$сомневается$\}$; никто не $\{$знает$\}$; никто не $\{$думает$\}$; 
кто-то где-то $\{$обмолвился$\}$; $\{$опираются$\}$ на ка\-кие-то 
$\{$слова$\}$; $\{$случилось большое$\}$ <<Ничего>>; $\{$было~---$\}$ 
Ничего; как-то $\{$внезапно$\}$; никто, $\{$конечно, в~точ\-ности$\}$ 
ничего не $\{$знает$\}$; кое-где $\{$остановили трамваи (и~разбили)$\}$; 
будто бы $\{$убили$\}$; будто бы $\{$пошли$\}$; $\{$все$\}$ <<будто бы>>; 
где оно и~кто; и~что бы ни было $\{$дальше$\}$; $\{$верящие$\}$ ка\-кие-то; 
$\{$не прос\-тит\-ся~---$\}$ кем-то, чем-то; кто-ни\-будь. Ка\-кие-ни\-будь 
$\{$третьи$\}$; ка\-кой-то $\{$подлый слой$\}$; $\{$министерством$\}$ 
якобы $\{$<<доверия>>$\}$; $\{$Дума$\}$ будто бы $\{$решила$\}$; 
$\{$она$\}$, кажется, $\{$там сидит$\}$; $\{$Солдаты$\}$, кажется, $\{$были 
выпивши$\}$; в~какой $\{$зале$\}$~--- не $\{$знаю$\}$; В~какой $\{$они 
связи с~Комитетом$\}$~--- не $\{$выясняется$\}$.
    
   \section{Заключение}
     
     Лексико-статистический анализ текстов естественных языков 
предназначен для установления статистических закономерностей 
встре\-ча\-емости наименований понятий, служебных слов, оборотов. 
Полученные в~результате такого анализа закономерности позволяют не 
только автоматически распознавать именованные сущности, но 
и~использовать их для установления системы взаимосвязей понятий при 
формировании предварительных словарей парадигматических 
и~ассоциативных связей.
{ %\looseness=1

}
     
     Привлечение ряда методов текстологического анализа и~выстраивание 
их в~последовательную технологическую цепочку:  
лек\-си\-ко-ста\-ти\-сти\-че\-ский анализ, извлечение именованных сущностей 
и~формирование классификации ключевых и~опорных слов (основы для 
построения онтологии), выявление и~частотный анализ служебных слов 
и~других скреп позволяет более эффективно решать целый ряд задач 
аналитической обработки неструктурированных данных, каковыми являются 
ес\-тест\-вен\-но-язы\-ко\-вые тексты.

\vspace*{-12pt} 
    
{\small\frenchspacing
 {\baselineskip=10.9pt
 \addcontentsline{toc}{section}{References}
 \begin{thebibliography}{99}
     
\bibitem{1-koz}
\Au{Kuznetsov I.\,P., Kozerenko~E.\,B., Matskevich~A.\,G.} Intelligent extraction of knowledge 
structures from natural language texts~// IEEE/WIC/ACM Joint Conferences 
(International) on Web Intelligence and Intelligent Agent Technology~--- 
Workshops WI-IAT Proceedings.~--- Lyon, France: IEEE Computer Society, 2011. 
P.~269--272. 
\bibitem{2-koz}
\Au{Kozerenko E.\,B., Kuznetsov~K.\,I., Morozova~Yu.\,I., Romanov~D.\,A.} Semantic 
proximity establishment in the tasks of knowledge extraction and named entities recognition~// 
19th Conference (International) on Artificial Intelligence, WORLDCOMP'17 
Proceedings.~--- Las Vegas, NV, USA: CSREA Press, 2017. P.~339--344.

\bibitem{7-koz} %3
\Au{Dempster A.\,P., Laird~N.\,M., Rubin~D.\,B.} Maximum likelihood from incomplete data 
via the EM algorithm~// J.~Roy. Stat. Soc.~B, 1977. Vol.~39. No.\,1. P.~1--22.

\bibitem{5-koz} %4
\Au{Rapp R.} Word sense discovery based on sense descriptor dissimilarity~//
9th Machine Translation Summit Proceedings.~--- New Orleans, LA, USA, 2003. P.~315--322.

\bibitem{3-koz} %5
\Au{Lenci A.} Distributional semantics in linguistic and cognitive research~// 
Riv. Linguist., 2008. Vol.~1. Р.~1--30.


\bibitem{6-koz} %6
\Au{Turney P.} A~uniform approach to analogies, synonyms, 
antonyms and associations~// 22nd 
Conference (International) on Computational Linguistic Proceedings.~--- 
Manchester, 2008. P.~905--912.

\bibitem{4-koz} %7
\Au{Baroni M., Lenci~A.} Distributional memory: A~general framework for corpus-based 
semantics~// Comput. Linguist., 2010. Vol.~36. Iss.~4. P.~673--721.

\bibitem{8-koz}
\Au{Schumann A.} Towards the automated enrichment of multilingual terminology databases 
with knowledge-rich contexts~// Computational Linguistics and Intellectual Technologies: 
Conference (International) ``Dialogue 2012'' Proceedings.~---  Moscow: RGGU, 2012. Vol.~1. 
Iss.~11. P.~559--567. 
\bibitem{9-koz}
\Au{Козеренко Е.\,Б.} <<Наших дедов мечта невозможная>>~--- Учредительное собрание 
в~Черных тетрадях Зинаиды Гиппиус~// Мар\-ги\-на\-лии-2015: границы культуры 
и~текста: Тезисы докл. Междунар. конф.~/
Под ред. Е.\,Б.~Козеренко, А.\,Г.~Кравецкого, М.\,Ю.~Ми\-хе\-ева.~--- 
Полоцк, 2015. {\sf http://uni-persona.srcc.msu.ru/ site/conf/marginalii-2015/thesis.htm}.
\bibitem{10-koz}
\Au{Козеренко Е.\,Б.} Февраль 17-го в~<<Синей книге>> З.\,Н.~Гиппиус: опыт 
текстологического анализа~// Мар\-ги\-на\-лии-2017: границы культуры и~текста: Тезисы 
докл. Междунар. конф.~/
Под ред.\ А.\,Г.~Кра\-вец\-ко\-го, М.\,Ю.~Ми\-хе\-ева.--- Торжок, 
2017. {\sf http://uni-\linebreak persona.srcc.msu.ru/site/conf/marginalii-2017/thesis. htm}.
\bibitem{11-koz}
\Au{Михеев М.\,Ю., Эрлих~Л.\,И.} Идиостилевой профиль и~определение авторства текста 
по частотам служебных слов~// НТИ. Сер.~2: Информ. процессы и~системы, 2018. №\,2. 
С.~25--34.
\bibitem{12-koz}
\Au{Charnine M.\,M., Kuznetsov~I.\,P., Kozerenko~E.\,B.} Semantic navigator for Internet 
search~// Conference (International) on Machine Learning Proceeding.~--- Las 
Vegas, NV, USA: CSREA Press, 2005. P.~60--68.
\bibitem{13-koz}
\Au{Кузнецов И.\,П., Сомин~Н.\,В.} Выявление имплицитной информации из текстов на 
естественном языке: проб\-ле\-мы и~методы~// Информатика и~её применения, 2012. Т.~6. 
Вып.~1. С.~48--57.
\bibitem{14-koz}
\Au{Kuznetsov I.\,P., Kozerenko~E.\,B., Charnine~M.\,M.} Technological peculiarity of 
knowledge extraction for logical-analytical systems~// WORLDCOMP'12: ICAI'12 
Proceedings.~--- Las Vegas, NV, USA: CSREA Press, 2012. Vol.~II. P.~762--768.
\bibitem{15-koz}
\Au{Шарнин М.\,М., Кузнецов~И.\,П.} Особенности семантического поиска 
информационных объектов на основе технологии баз знаний~// Информатика и~её 
применения, 2012. Т.~6. Вып.~2. С.~47--56.
\bibitem{16-koz}
\Au{Lund K., Burgess C.} Producing high-dimensional semantic spaces from lexical  
co-occurrence~// Behav. Res. Meth. Ins.~C., 1996. Vol.~28. 
No.\,2. P.~203--208.
\bibitem{17-koz}
\Au{McCarthy D., Koeling~R., Weeds~J., Carroll~J.} Finding predominant senses in untagged 
text~// 42nd Annual Meeting of the Association for Computational Linguistics Proceedings.~--- 
Barcelona, Spain: ACL, 2004. P.~280--287.
\bibitem{18-koz}
\Au{Baroni M., Zamparelli~R.} Nouns are vectors, adjectives are matrices: Representing 
adjective-noun constructions in semantic space~// Conference on Empirical Methods in 
Natural Language Processing Proceedings.~--- Stroudsburg, PA, USA: ACL, 2010.  
P.~1183--1193.
\bibitem{19-koz}
\Au{Гиппиус З.\,Н.} Дневники.~--- М.: Захаров, 2017. 528~с.
\bibitem{20-koz}
Синяя книга~// Ин\-тер\-нет-ре\-сурс произведений З.\,Н.~Гип\-пи\-ус. {\sf 
https://gippius.com/doc/memory/\linebreak sinyaya-kniga.html}.
 \end{thebibliography}

 }
 }

\end{multicols}

\vspace*{-9pt}

\hfill{\small\textit{Поступила в~редакцию 15.01.20}}

\vspace*{6pt}

%\pagebreak

%\newpage

%\vspace*{-28pt}

\hrule

\vspace*{2pt}

\hrule

%\vspace*{-2pt}

\def\tit{ANALYTICAL TEXTOLOGY IN~INTELLIGENT PROCESSING SYSTEMS 
FOR~UNSTRUCTURED DATA}


\def\titkol{Analytical textology in~intelligent processing systems 
for~unstructured data}

\def\aut{E.\,B.~Kozerenko$^1$, M.\,Y.~Mikheev$^2$, N.\,V.~Somin$^1$, 
L.\,I.~Ehrlich$^2$, and~K.\,I.~Kuznetsov$^1$}

\def\autkol{E.\,B.~Kozerenko, M.\,Y.~Mikheev, N.\,V.~Somin, et al.} 
%L.\,I.~Ehrlich$^2$, and~K.\,I.~Kuznetsov$^1$}

\titel{\tit}{\aut}{\autkol}{\titkol}

\vspace*{-11pt}


\noindent
$^1$Institute of Informatics Problems, Federal Research Center ``Computer 
Science and Control'' of the Russian\linebreak
$\hphantom{^1}$Academy of Sciences, 44-2~Vavilov Str., 
Moscow 119133, Russian Federation


\noindent
$^2$Research Computing Center Lomonosov Moscow State University, 1, bld.~4 
Leninskie Gory, Moscow, GSP-1,\linebreak
$\hphantom{^1}$119991, Russian Federation

\def\leftfootline{\small{\textbf{\thepage}
\hfill INFORMATIKA I EE PRIMENENIYA~--- INFORMATICS AND
APPLICATIONS\ \ \ 2020\ \ \ volume~14\ \ \ issue\ 1}
}%
 \def\rightfootline{\small{INFORMATIKA I EE PRIMENENIYA~---
INFORMATICS AND APPLICATIONS\ \ \ 2020\ \ \ volume~14\ \ \ issue\ 1
\hfill \textbf{\thepage}}}

\vspace*{3pt} 



\Abste{The paper presents a new field of research at the intersection of linguistics, 
computer science, and philology involving logical and statistical methods of analyzing 
unstructured data in the form of natural language texts in order to solve a number 
of the tasks of extracting explicit and implicit knowledge from texts using a 
semantics-oriented linguistic processor, forming lexical statistical representations 
of texts, building analytical conclusions,  discovery of the author's idiostyle and 
textual similarity of literary works based on the analysis of service words and other 
microtext elements; identifying the sentiment of texts, building a full profile of the 
author's text based on the superposition of methods. The example of the 
textological analysis of the ``Blue Book'' of the ``Petersburg Diary'' by Zinaida 
Hippius is considered.} 

\KWE{natural language processing; statistical methods; cognitive technology; 
lexical semantic analysis; knowledge extraction from texts; analytical systems} 



\DOI{10.14357/19922264200115} 

\vspace*{-14pt}

\Ack
\noindent
The paper was partially supported by the Russian Foundation for Basic Research 
(project 18-012-00220-a).
%\vspace*{6pt}

  \begin{multicols}{2}

\renewcommand{\bibname}{\protect\rmfamily References}
%\renewcommand{\bibname}{\large\protect\rm References}

{\small\frenchspacing
 {%\baselineskip=10.8pt
 \addcontentsline{toc}{section}{References}
 \begin{thebibliography}{99}
\bibitem{1-koz-1}
\Aue{Kuznetsov, I.\,P., E.\,B.~Kozerenko, and A.\,G.~Matskevich.} 2011. Intelligent extraction 
of knowledge structures from natural language texts. \textit{IEEE/WIC/ACM Joint 
Conferences (International) on Web Intelligence and Intelligent Agent Technology 
Proceedings~--- Workshops WI-IAT Proceedings}. Lyon, 
France: IEEE Computer Society. 269--272. 
\bibitem{2-koz-1}
\Aue{Kozerenko, E.\,B., K.\,I.~Kuznetsov, Yu.\,I.~Morozova, and D.\,A.~Romanov.} 2017. 
Semantic proximity establishment in the tasks of knowledge extraction and named entities 
recognition. \textit{Conference (International) on Artificial Intelligence, 
WORLDCOMP'17 Proceedings}.  Las Vegas, NV: CSREA. 339--344.

\bibitem{7-koz-1} %3
\Aue{Dempster, A.\,P., N.\,M.~Laird, and D.\,B.~Rubin.} 1977. Maximum likelihood from 
incomplete data via the EM algorithm. \textit{J.~Roy. Stat. Soc.~B} 39(1):1--22.
\bibitem{5-koz-1} %4
\Aue{Rapp, R.} 2003. Word sense discovery based on sense descriptor dissimilarity. \textit{9th 
Machine Translation Summit Proceedings}. New Orleans, LA. 315--322.

\bibitem{3-koz-1} %5
\Aue{Lenci, A.} 2008. Distributional semantics in linguistic and cognitive research. 
\textit{Riv. Linguist.} 1:1--30.


\bibitem{6-koz-1} %6
\Aue{Turney, P.} 2008. A~uniform approach to analogies, synonyms, antonyms and 
associations. \textit{22nd 
Conference (International) on Computational Linguistic Proceedings}. Manchester. 905--912.

\bibitem{4-koz-1} %7
\Aue{Baroni, M., and A.~Lenci.} 2010. Distributional memory: A~general framework for 
corpus-based semantics. \textit{Comput. Linguist.} 36(4):673--721.

\bibitem{8-koz-1} %8
\Aue{Schumann, A.} 2012. Towards the automated enrichment of multilingual terminology 
databases with knowledge-rich contexts. \textit{Computational Linguistics and Intellectual Technologies: 
Conference (International) ``Dialogue 2012'' Proceedings}. Moscow. 1(11):559--567. 
\bibitem{9-koz-1}
\Aue{Kozerenko, E.\,B.} 2015 <<Nashikh dedov mechta nevozmozhnaya>>~--- 
Uchreditel'noe sobranie v~Chernykh Tetradyakh Zinaidy Gippius [``The impossible dream of 
our grandfathers''~--- Constituent assembly in the Black Notebooks of 
Z.\,N.~Hippius]. Eds. 
E.\,B.~Ko\-ze\-ren\-ko, A.\,G.~Kra\-vet\-sky, and M.\,Y.~Mikheev. \textit{Conference (International) 
``Marginalia-2015: Borders of Culture and Text'' Proceedings}. Polotsk. Available at: 
{\sf http://uni-persona.srcc.msu.ru/site/conf/marginalii-2015/thesis.htm}
(accessed March~10, 2020).
\bibitem{10-koz-1}
\Aue{Kozerenko, E.\,B.} 2017. Fevral' 17-go v~<<Siney kni\-ge>> Z.\,N.~Gippius: opyt 
tekstologicheskogo ana\-li\-za [February of 17th in the ``Blue book''
 of Z.\,N.~Hippius: The\linebreak case 
of the textological analysis]. Eds. A.\,G.~Kra\-vet\-sky and M.\,Y.~Mi\-khe\-ev. \textit{Conference 
(International) ``Marginalia-2017: Borders of Culture and Text'' Proceedings}.
 Torzhok. Available at: 
 {\sf  http://uni-persona. srcc.msu.ru/site/conf/marginalii-2017/thesis.htm}
 (accessed March~10, 2020).
\bibitem{11-koz-1}
\Aue{Mikheev, M.\,Yu., and L.\,I.~Ehrlich.} 2018. Idiostilevoy profil'
 i~opredelenie avtorstva 
teksta po chastotam sluzhebnykh slov 
[Individual style profile and text authorship\linebreak detection 
based on the service words frequencies]. \textit{Nauchno-technicheskaya informatsia. Ser.~2. 
Informatsionnye pro\-tses\-sy i~sistemy} [Scientific Technical Information. Ser.~2. Information 
Processes and Systems] 2:25--34.
\bibitem{12-koz-1}
\Aue{Charnine, M.\,M., I.\,P.~Kuznetsov, and E.\,B.~Kozerenko.} 2005. Semantic navigator for 
Internet search. \textit{Conference (International) on Machine Learning 
Proceeding}. Las Vegas, NV: CSREA Press. 60--68.
\bibitem{13-koz-1}
\Aue{Kuznetsov, I.\,P., and N.\,V.~Somin}. 2012. Vyyavlenie implitsitnoy informatsii iz 
tekstov na estestvennom yazyke: problemy i~metody [Revealing implicit information from texts 
in natural language: Problems and methods]. \textit{Informatika i~ee Primeneniya~--- Inform. 
Appl.} 6(1):48--57. 
\bibitem{14-koz-1}
\Aue{Kuznetsov, I.\,P., E.\,B.~Kozerenko, and M.\,M.~Charnine.} 2012. Technological 
peculiarity of knowledge extraction for logical-analytical systems. \textit{WORLDCOMP'12: 
ICAI'12 Proceedings}. Las Vegas, NV: CSREA Press. II:762--768.
\bibitem{15-koz-1}
\Aue{Charnine, M.\,M., and I.\,P.~Kuznetsov.} 2012. Oso\-ben\-no\-sti semanticheskogo poiska 
informatsionnykh ob''ektov na osnove tekhnologii baz znaniy [The peculiarities of the semantic 
search of information objects founded on the knowledge bases technology]. \textit{Informatika 
i~ee Primeneniya~--- Inform. Appl.} 6(2):47--56.
\bibitem{16-koz-1}
\Aue{Lund, K., and C.~Burgess.} 1996. Producing high-dimensional semantic spaces from 
lexical co-occurrence. \textit{Behav. Res. Meth. Ins.~C.} 
28(2):203--208.
\bibitem{17-koz-1}
\Aue{McCarthy, D., R.~Koeling, J.~Weeds, and J.~Carroll.} 2004. Finding predominant senses 
in untagged text. \textit{42nd Annual Meeting of the Association for Computational Linguistics.} 
Barcelona, Spain: ACL. 280--287.
\bibitem{18-koz-1}
\Aue{Baroni, M., and R.~Zamparelli.} 2010. Nouns are vectors, adjectives are matrices: 
Representing adjective-noun constructions in semantic space. \textit{Conference on 
Empirical Methods in Natural Language Processing}. Stroudsburg, PA: ACL. 1183--1193.
\bibitem{19-koz-1}
\Aue{Hippius, Z.\,N.} 2017. \textit{Dnevniki} [Diaries]. Moscow: Zakharov. 528~p. 
\bibitem{20-koz-1}
The Internet resource of Z.\,N.~Hippius works, ``Sinyaya kniga.''
 Available at: {\sf 
https://gippius.com/doc/memory/\linebreak sinyaya-kniga.html} (accessed January~27, 2020).

\end{thebibliography}

 }
 }

\end{multicols}

%\vspace*{-7pt}

\hfill{\small\textit{Received January 15, 2020}}

%\pagebreak

\vspace*{-9pt}

\Contr


\noindent
\textbf{Kozerenko Elena B.} (b.\ 1959)~---  Candidate of Science (PhD) in linguistics, leading 
scientist, Institute of Informatics Problems, Federal Research Center 
``Computer Science and 
Control'' of the Russian Academy of Sciences, 44-2 Vavilov Str., Moscow 119133, Russian 
Federation; \mbox{kozerenko@mail.ru}

%\vspace*{3pt}

\pagebreak

\noindent
\textbf{Mikheev Michael Yu.} (b.\ 1957)~--- Doctor of Science in linguistics, leading scientist, 
Research Computing Center Lomonosov Moscow State University, 1, bld.~4 Leninskie Gory, 
Moscow, GSP-1, 119991, Russian Federation;  \mbox{mihej57@yandex.ru}

\vspace*{6pt}

\noindent
\textbf{Somin Nikolai V.} (b.\ 1947)~--- Candidate of Science (PhD) in physics and 
mathematics, leading scientist, Institute of Informatics Problems, Federal Research Center 
``Computer Science and Control'' of the Russian Academy of Sciences, 44-2~Vavilov Str., 
Moscow 119133, Russian Federation; \mbox{chri-soc@yandex.ru}

\vspace*{6pt}

\noindent
\textbf{Ehrlich Lev I.} (b.\ 1948)~--- leading engineer, Research Computing Center Lomonosov 
Moscow State University, 1, bld.~4 Leninskie Gory, Moscow, GSP-1, 119991, Russian 
Federation; \mbox{levehr@yandex.ru}

\vspace*{6pt}

\noindent
\textbf{Kuznetsov Konstantin I.} (b.\ 1968)~--- leading engineer, Institute of Informatics 
Problems, Federal Research Center ``Computer Science and Control'' of the Russian Academy 
of Sciences, 44-2 Vavilov Str., Moscow 119133, Russian Federation; \mbox{k.smith@mail.ru}
\label{end\stat}

\renewcommand{\bibname}{\protect\rm Литература} 
 

 