\def\stat{husainov}

\def\tit{ПРОИЗВОДИТЕЛЬНОСТЬ ОГРАНИЧЕННОГО КОНВЕЙЕРА}

\def\titkol{Производительность ограниченного конвейера}

\def\aut{А.\,А.~Хусаинов$^1$}

\def\autkol{А.\,А.~Хусаинов}

\titel{\tit}{\aut}{\autkol}{\titkol}

\index{Khusainov A.\,A.}
\index{Хусаинов А.\,А.}


%{\renewcommand{\thefootnote}{\fnsymbol{footnote}} \footnotetext[1]
%{Работа выполнена при финансовой поддержке Российского научного фонда (проект 18-11-00155).}}


\renewcommand{\thefootnote}{\arabic{footnote}}
\footnotetext[1]{Комсомольский-на-Амуре государственный университет, 
\mbox{husainov51@yandex.ru}}

%\vspace*{-12pt}



  \Abst{Работа посвящена изучению производительности ограниченного конвейера~--- 
вычислительного конвейера, число активных ступеней которого в~каждый момент 
времени ограничено сверху некоторым значением. Рассмотрены ограниченные 
конвейеры с~заданными суммой и~максимумом задержек ступеней. Ступени могут иметь 
разные задержки. Основная задача~--- построение аналитической модели для расчета 
времени обработки заданного объема данных с~помощью этого ограниченного 
конвейера. Решение упрощается, если ограничение рассматривать как структурный 
конфликт конвейера. Эта аналитическая модель построена для случая, когда работа 
ограниченного конвейера обладает свойством непрерывности обработки каждого 
входного элемента. Для таких конвейеров в~работе доказана гипотеза о~том, что 
минимальное число процессоров, при котором достигается наибольшая 
производительность, равно наименьшему целому числу, не меньшему отношения суммы 
задержек ступеней к~их максимальной задержке. Установлено, что если не требовать 
свойства непрерывности, то эта гипотеза неверна. Построенная модель может быть 
применена для синхронизации работы ступеней ограниченного конвейера со свойством 
непрерывности. Если не требовать свойства непрерывности, то получаем асинхронный 
ограниченный конвейер, синхронизация работы ступеней которого осуществляется на 
основе го\-тов\-ности данных. Разработано программное обеспечение, позволяющее 
вычислять время обработки данных с~по\-мощью асинхронного ограниченного 
конвейера.}
   
  \KW{вычислительный конвейер; моноид трасс; нормальная форма Фоаты; 
производительность конвейера; структурный конфликт}

\DOI{10.14357/19922264200112} 
  
\vspace*{-3pt}


\vskip 10pt plus 9pt minus 6pt

\thispagestyle{headings}

\begin{multicols}{2}

\label{st\stat}

\section{Введение}

  Вычислительный конвейер, состоящий из~$p$ ступеней, называется 
\textit{ограниченным} некоторым чис\-лом~$q$, если в~каждый момент 
времени могут одновременно выполняться не более чем~$q$~ступеней. 
В~данной работе найдена формула для расчета времени обработки~$n$ 
входных элементов с~помощью ограниченного конвейера, обладающего 
свойством непрерывности обработки для каждого входного элемента 
конвейера. С~по\-мощью этой формулы для ограниченного конвейера со 
свойством непрерывности подтверждена выдвинутая в~[1] на основании 
экспериментов гипотеза о~том, что минимальное число процессоров, при 
котором достигается наибольшая производительность ограниченного 
конвейера, будет равно наименьшему целому~$q$, удовлетворяющему 
неравенству $q\hm\geq \sigma/\mu$, где $\sigma$~--- сумма задержек 
ступеней конвейера, а~$\mu$~--- задержка самой медленной ступени. 
Приведен пример, показывающий, что в~общем случае эта гипотеза неверна.
  
  Проведенные исследования тесно связаны с~конфликтами, 
возникающими при работе конвейера. Под конфликтами подразумеваются 
со\-стояния, приводящие к~замедлению работы\linebreak конвейера. Тео\-рия 
конфликтов применяется при разработке конвейерных процессоров~[2], 
сигнальных процессоров~[3], сопроцессоров~[4]. Существуют три типа 
конфликтов~[5]: структурные конфликты, конфликты по данным 
и~конфликты управления. Структурный конфликт~--- оборудование не 
может поддержать комбинацию инструкций, которые необходимо 
выполнить одновременно в~некоторый момент времени. Ограниченный 
конвейер можно рассматривать как конвейер со структурным конфликтом. 
Аналитические модели для расчета производительности конвейеров 
с~конфликтами построены в~[2, 6, 7]. Эти модели предназначены для 
однородных конвейеров~--- конвейеров, ступени которых имеют 
одинаковые задержки. Заметим, что в~работах~[8, 9] изучались 
неоднородные конвейеры и~был предложен метод динамического 
отоб\-ра\-же\-ния конвейера (dynamic pipeline mapping) для улучшения 
про\-из\-во\-ди\-тель\-ности, решались задачи, где чис\-ло процессоров превышает 
число ступеней, но проб\-ле\-мы, связанные с~расчетом про\-из\-во\-ди\-тель\-ности 
ограниченных конвейеров, не были решены. 

Автором в~работе~[10] была 
построена аналитическая модель для неоднородного конвейера 
с~единственным конфликтом, вызывающим ре\-старт. В~предлагаемой 
работе строится аналогичная модель для конвейера с~одним конфликтом, 
соответствующего ограниченному конвейеру со свойством непрерывности. 
  
  Для оценки времени ускорения работы программы с~помощью~$q$ 
процессоров можно использовать закон Амдала~[11]. Для конвейеров 
существуют некоторые варианты этого закона, описанные 
в~\cite[п.~1.4.1.3]{12-h}. Естественно предположить, что для ограниченного 
конвейера со свойством непрерывности имеет место 
$$
T_q(n)\approx 
\sigma+ \fr{(n-1)\sigma}{q}\,.
$$

 Будет доказано, что это равенство верно 
с~точностью до суммы задержек ступеней~$\sigma$. 
  
  В разд.~2 рассмотрен однородный ограниченный конвейер. Для 
построения аналитической модели для него достаточно рассмотреть 
таблицу занятости процесса обработки~$n$~входных элементов. В~разд.~3 
построена и~доказана формула для расчета времени обработки данных 
с~помощью неоднородного ограниченного конвейера со свойством 
непрерывности. В~разд.~4 описано программное обеспечение для расчета 
производительности асинхронных ограниченных конвейеров, 
синхронизация работы ступеней которых осуществляется на основе 
готовности данных, передаваемых между ступенями. В~конце разд.~4 
приведен пример, показывающий, что в~общем случае гипотеза 
о~минимальном числе процессоров конвейера неверна. 
     
\section{Однородный ограниченный конвейер}

  Обозначим ступени вычислительного конвейера через $a_1,\ldots , a_p$. 
Ступень, выполняющаяся в~некоторый момент времени на некотором 
процессоре (функциональном устройстве) конвейера, называется 
\textit{активной} в~этот момент. Задержкой ступени называется время 
обработки ступенью одного входного элемента конвейера. Это время 
включает в~себя логические операции и~операции обмена данными 
с~другими ступенями через входные и~выходные каналы. Будем 
предполагать, что процессоры конвейера имеют одинаковую тактовую 
частоту, и~измерять время в~тактах.
  
  Под \textit{таблицей занятости}~[13] конвейера будем подразумевать 
матрицу, строки которой соответствуют ступеням конвейера и~имеют 
номера $1\hm\leq i\hm\leq p$, а~столбцы~--- тактам времени $1, 2, 3,\ldots$ 
Коэффициенты этой матрицы~$a_{ij}$ равны $k\hm\geq1$\linebreak тогда и~только 
тогда, когда $i$-я ступень обрабатывает $k$-й входной элемент в~течение 
такта~$j$. В~этом случае в~клетку $(i,j)$ ставится чис\-ло~$k$. Если $i$-я 
ступень в~момент~$j$ не активна, то $a_{ij}\hm=0$ и~со\-от\-вет\-ст\-ву\-ющая 
клетка в~таблице остается пустой.
  
  Конвейер, состоящий из~$p$~ступеней, называется \textit{ограниченным 
числом}~$q$, если в~каждый момент времени число активных ступеней не 
больше~$q$. 
  
  Ограниченный конвейер имеет следующие свойства:
  \begin{itemize}
\item в~каждый момент времени активна по крайней мере одна ступень;
\item в~каждый момент времени активно не более~$q$~ступеней;
\item перед обработкой входного элемента конвейера для каждого 
$i\hm>1$ ступень~$a_i$ ожидает окончания обработки этого входного 
элемента с~помощью ступени~$a_{i-1}$; 
\item для каждого $i\hm\geq 1$ ступень~$a_i$ ожидает окончания своего 
предыдущего выполнения.
\end{itemize}

  Конвейер называется имеющим свойство \textit{непрерывности} 
(работы), если для всякого входного элемента разность между временем 
конца обработки и~временем начала обработки этого элемента равна сумме 
задержек ступеней конвейера. 
  %
  В част\-ности, свойством непрерывности обладает однородный  
конвейер~--- конвейер, все ступени которого имеют одинаковые задержки, 
равные некоторому числу~$h$.

 Обозначим через~$p$ число его ступеней. 
Если нет конфликтов, то время обработки~$n$~элементов равно 
$(p\hm+n\hm-1)h$. Пусть однородный конвейер ограничен числом~$q$, 
$1\hm\leq q\hm\leq p$. На вход конвейера поступает~$n$ элементов входных 
данных. Таблица~1 показывает занятость конвейера при $p\hm=4$, 
$q\hm=3$ и~$n\hm=5$. При попытке запустить больше чем~$q$ параллельно 
работающих ступеней возникает структурный конфликт, в~результате 
которого каж\-дая ступень будет ожидать освобождения одного из 
процессоров и~время работы этой ступени увеличится на $(p\hm-q)h$. 

\begin{center}
\vspace*{3pt}
%\noindent
{{\tablename~1}\ \ \small{Однородный ограниченный конвейер }}
%\vspace*{2ex}

\vspace*{9pt}


{\small
\tabcolsep=5.8pt
\begin{tabular}{|c|c|c|c|c|c|c|c|c|c|c|}
\hline
&01&02&03&04&05&06&07&08&09&10\\
\hline
1&1&2&3&&4&5&&&&\\
2&&1&2&3&&4&5&&&\\
3&&&1&2&3&&4&5&&\\
4&&&&1&2&3&&4&5&\\
\hline
\end{tabular}
}
\end{center}

%\end{table*}

\vspace*{9pt}

  
  Учитывая случай $p\hm<q$, приходим к~следующему утверждению.
  
  \smallskip
  
  \noindent
  \textbf{Предложение~1}~\cite[Prop.~1]{14-h}. \textit{Время 
обработки~$n$~элементов с~помощью~$q$~процессоров для однородного 
конвейера из~$p$~ступеней с~задержкой~$h$ равно}
  \begin{equation}
  T_q(n)=\left( p+n-1+(p-q)^+\left[ \fr{n-1}{q}\right]\right)h\,.
  \label{e1-h}
  \end{equation}
  Здесь $[x]$ обозначает целую часть вещественного чис\-ла~$x$, а~$(x)^+$ 
означает чис\-ло, равное~$x$, если $x\hm\geq0$, и~равное~0 в~случае 
$x\hm\leq 0$. 
  
  Эта формула была применена в~[14] для расчета оптимальной глубины 
однородного ограниченного конвейера. 

\section{Неоднородный ограниченный конвейер}

  В данном разделе всюду, где не оговорено противное, неоднородный 
ограниченный конвейер будет обладать свойством непрерывности. 
  
  Ступени неоднородного конвейера могут иметь разные задержки 
$\tau_1,\ldots , \tau_p$. Время обработки~$n$~элементов равно
  $$
  T_p(n)=\sigma+(n-1)\mu\,,
  $$
где $\sigma =\sum\nolimits^p_{i=1} \tau_i$~--- сумма, а~$\mu\hm=\max \{ 
\tau_i\vert 1\hm\leq i\hm\leq p\}$~--- максимум задержек ступеней конвейера. 
Правую часть этой формулы можно получить из времени обработки для 
однородного конвейера $(p\hm+n\hm-1)h$, подставляя вместо~$p$ 
отношение $\sigma/\mu$, а~вместо~$h$~--- максимальную задержку 
ступеней~$\mu$. Это приводит к~предположению о том, что аналогичным 
образом из формулы~(1) может быть получена формула для времени 
обработки с~помощью ограниченного неоднородного конвейера. Из этой 
формулы придется удалить слагаемые, для которых $\sigma/\mu\hm- 
q\hm<0$.    Сформулируем и~докажем полученное утверждение. 

\smallskip

\noindent
  \textbf{Теорема~1.} \textit{Время обработки~$n$~элементов  
с~по\-мощью неоднородного конвейера со свойством не\-пре\-рыв\-ности, 
ограниченного числом~$q$ и~состоящего из~$p$~ступеней, равно}
  \begin{equation}
  T_q(n)=\sigma +(n-1)\mu +(\sigma-q\mu)^+\left[ \fr{n-1}{q}\right]\,.
  \label{e2-h}
  \end{equation}
  
  \noindent
  Д\,о\,к\,а\,з\,а\,т\,е\,л\,ь\,с\,т\,в\,о\,.\ \ Обычный конвейер 
обрабатывает~$n$ элементов за время $T_p(n)\hm= \sigma \hm+ (n\hm-
1)\mu$. В~случае, когда активны $q\hm<p$ ступеней, к~этому времени 
добавляется время ожидания свободных процессоров. Это время ожидания 
называется штрафным. Рассмотрим таблицу занятости при 
обработке~$n$~элементов. Первые~$q$~элементов будут обрабатываться 
без лишних торможений. Они будут обработаны за время $T(q)\hm= 
\sigma\hm+ (q\hm-1)\mu$. При попытке обработать $(q+1)$-й элемент 
возникает (структурный) конфликт, связанный с~тем, что чис\-ло 
одновременно работающих ступеней не должно быть больше чем~$q$. Этот 
конфликт будет разрешен после окончания обработки первого элемента, 
ибо в~этом случае появится свободный процессор. В~силу свойства 
непрерывности отсюда вытекает, что обработку $(q+1)$-го элемента можно 
начать в~момент времени~$\sigma$. Время обработки $q+1$~элементов 
будет равно~$2\sigma$. Поскольку для обычного конвейера время 
обработки $q\hm+1$~элементов равно $\sigma\hm+q\mu$, то штрафное 
время будет равно $2\sigma\hm- (\sigma \hm+ q\mu)\hm=\sigma -\hm q\mu$. 
Эти конфликты возникают при обработке элементов 
с~номерами $q\hm+1, 2q+1, \ldots , mq\hm+1$, где~$m$~--- наибольшее 
целое, для которого $mq\hm+1\hm\leq n$. Ясно, что $m\hm= \left[ ({n\hm-
1})/{q}\right]$. При обработке остальных элементов конфликты не 
возникают. Отсюда вытекает, что если $\sigma\hm- q\mu\hm\geq 0$, то 
$T_q(n)\hm=\sigma\hm+ (n\hm-1)\mu\hm+m(\sigma\hm- q\mu)$. Если же 
$\sigma\hm- q\mu\hm\leq 0$, то в~момент времени~$q\mu$ начала обработки 
$(q+1)$-го элемента первый элемент будет обработан и~закончит занимать 
один из процессоров. В~этом случае штрафное время будет равно нулю 
и~$T_q(n)\hm= \sigma\hm+ (n\hm-1)\mu$. Теорема~1 доказана.
  
  \smallskip
  
  
  Из доказанной формулы~(2) вытекает, что при $n\hm-1\hm\geq q$ время 
обработки~$n$~элементов будет минимальным тогда и~только тогда, когда 
имеет место неравенство $\sigma\hm- q\mu\hm\leq 0$. Это приводит 
к~следующему утверждению. 
  
  \smallskip
  
  \noindent
  \textbf{Следствие~1.} Минимальное число процессоров, при котором 
достигается наибольшая производительность ограниченного конвейера со 
свойством непрерывности, равно наименьшему целому~$q$, 
удовле\-тво\-ря\-юще\-му неравенству $q\hm\geq \sigma/\mu$. В~этом случае 
время обработки равно $\sigma\hm+ (n\hm-1)\mu$.
  
  \smallskip
  
  Отсюда вытекает, что предположение, выдвинутое в~работе~[1], верно 
для ограниченных конвейеров со свойством непрерывности. 
  
  Обозначим через $T_q^A(n)$ следующую оценку для 
производительности конвейера:
  $$
  T_q^A(n)=\begin{cases}
  \left(1+\fr{n-1}{q}\right)\sigma\,, &\mbox{если } q\leq \fr{\sigma}{\mu}\,;\\
  \sigma+(n-1)\mu\,, &\mbox{если } q\geq \fr{\sigma}{\mu}\,.
  \end{cases}
  $$
  
  \noindent
  \textbf{Следствие~2.}\ Имеют место неравенства: 
  $0\hm\leq 
T_q^A(n)\hm-T_q(n)\hm<\sigma$.
  
  \smallskip
  
  \noindent
  Д\,о\,к\,а\,з\,а\,т\,е\,л\,ь\,с\,т\,в\,о\,.\ \  Пусть $\sigma\hm\geq q\mu$. 
Воспользуемся тем, что $n\hm-1\hm- q\left[ ({n-1})/{q}\right]$ равно 
остатку $(n\hm-1)\mathrm{mod}\,q$ от деления числа $n\hm-1$ на~$q$. Из 
теоремы~1 вытекает, что 
  \begin{multline*}
  T_q(n)=\sigma +(n-1)\mu +(\sigma-q\mu)\left[ \fr{n-1}{q}\right] ={}\\
  {}=\sigma 
\left( 1+\left[ \fr{n-1}{q}\right]\right)+\mu \left( \left( n-
1\right)\mathrm{mod}\,q\right)\,.
\end{multline*}
   Следовательно, 
  $$
  T_q^A(n)-T_q(n)=(\sigma -q\mu)\fr{(n-1)\mathrm{mod}\,q}{q}<\sigma\,.
  $$
  
  Пусть $S_q(n)=T_1(n)/T_q(n)$~--- ускорение вы\-чис\-ле\-ния с~по\-мощью 
ограниченного конвейера. Обозначим $S_q\hm = \lim\nolimits_{n\to\infty} 
S_q(n)$. Рассматривая случаи $q\hm<\sigma/\mu$ и~$q\hm\geq \sigma/\mu$, 
получаем 
  
  \smallskip
  
  \noindent
  \textbf{Следствие~3.}\ Для конвейера со свойством непрерывности, 
ограниченного числом $q\hm>1$, имеет место равенство $S_q\hm=\min 
(q,\sigma/\mu)$.
    
\section{Асинхронный ограниченный конвейер}

  Попытаемся сравнить производительность огра\-ни\-чен\-но\-го конвейера со 
свойством непрерывности с~производительностью асинхронного 
ограниченного конвейера, синхронизация работы\linebreak ступеней которого 
осуществляется на основе го\-тов\-ности данных, передаваемых между 
ступенями. Но не ясно, как вычислять производительность асинхронного 
конвейера. Возможный ответ\linebreak дает компьютерная программа, которой 
посвящен данный раздел. Эта программа для введенных пользователем 
числовых значений задержек ступеней асинхронного ограниченного 
конвейера и~объема данных вычисляет значения времени обработки\linebreak данных 
в~зависимости от чис\-ла процессоров и~выводит эти значения в~виде 
графиков. Она создает также и~сохраняет в~файл таб\-ли\-цы за\-ня\-тости 
конвейера при различных чис\-лах процессоров.
  
  Программа основана на методе, предложенном Дикертом~[15] 
и~использующем теорию трасс~--- слов, состоящих из букв алфавита, на 
котором задано антирефлексивное симметричное бинарное отношение 
независимости. 
  
  Опишем этот метод. Рассмотрим множество операций (машинных 
команд) $A\hm= \{a_0, \ldots , a_{m-1}\}$ и~отношение 
\textit{независимости} $I\hm\subseteq A^2$, состоящее из пар операций 
$(a_i,a_j)$, которые могут выполняться одновременно. Каждое слово можно 
интерпретировать как процесс, состоящий из команд, принадлежащих этому 
слову. Если команды могут выполняться одновременно, то их можно 
переставлять между собой. Два слова, составленные из букв алфавита~$A$, 
определим как \textit{эквивалентные}, если одно из них можно получить из 
другого с~по\-мощью последовательности перестановок рядом стоящих 
независимых букв. \textit{Трассой} называется класс эквивалентности 
множества~$A^*$ всех слов по этому отношению эквивалентности. Для 
произвольного слова $w\hm\in A^*$ обозначим через~$[w]$ его класс 
эквивалентности. Определим \textit{композицию} по формуле 
$[w_1][w_2]\hm=[w_1 w_2]$. Операция композиции превращает множество 
классов эк\-ви\-ва\-лент\-ности в~моноид, который называется \textit{моноидом 
трасс}. Идея алгоритма вычисления времени работы параллельного 
процесса описана в~[15]. Пусть время выполнения каждой команды, 
принадлежащей алфавиту~$A$, равно одному такту. Каждая трасса может 
быть представлена в~виде последовательности максимальных блоков 
(ярусов) параллельно выполняющихся команд. Это представление 
называется \textit{нормальной формой Фоаты}. Число блоков называется 
\textit{высотой} нормальной формы, и~оно равно времени выполнения 
трассы.
  
   Аналогично нормальной форме Фоаты введем \textit{нормальную форму 
относительно числа} $q\hm\geq 1$ как состоящую из последовательности 
блоков, длины которых не превышают число~$q$. Для этой цели 
модифицируем алгоритм приведения к~нормальной форме Фоаты и~за 
определение возьмем результат этого алгоритма.
   
   Опишем алгоритм. Пусть на входе задано некоторое непустое слово 
$w\hm\in A^*$. Считываем из него первый символ и~нормальную форму 
полагаем равной одному блоку, состоящему из этого символа. Далее 
в~цикле считываем очередной символ~$x$  и~для него 
выполняем~3~действия:
   \begin{enumerate}[(1)]
\item ищем блок с~наименьшим номером $k\hm\geq1$, такой что все 
элементы блоков, имеющих номера $\geq k$, независимы от~$x$. Если 
таких~$k$ нет, то добавляем новый блок, содержащий единственный 
элемент~$x$, и~переходим к~следующему символу;
\item цикл: пока $k$-й блок имеет~$q$~элементов, увеличиваем~$k$ 
на~1;
\item если~$k$ остался не больше номера последнего блока, то 
добавляем~$x$ в~$k$-й блок. Иначе до\-бав\-ля\-ем новый блок, состоящий из 
элемента~$x$. 
  \end{enumerate}
  
    \setcounter{table}{1}
    \begin{table*}[b]\small %tabl2
  \begin{center}
  \Caption{Работа асинхронного ограниченного конвейера}
  \vspace*{2ex}
  
  \begin{tabular}{|c|c|c|c|c|c|c|c|c|c|c|c|c|c|c|}
  \hline
&01 &02&03&04&05&06&07&08&09&10&11&12&13&14\\
\hline
1&1&2&3&&&&&&&&&&&\\
2&&1&1&1&2&2&2&3&3&3&&&&\\
3&&&&&1&&&2&&&3&&&\\
4&&&&&&1&1&&2&2&&3&3&\\
\hline
\end{tabular}
\end{center}
\end{table*}
     
  
  Число слов, из которых состоит нормальная форма относительно~$q$, 
называется ее \textit{высотой относительно}~$q$. Если каждая буква 
обозначает операцию, время выполнения которой равно единице измерения, 
и~независимые операции могут выполняться параллельно, то время 
выполнения операций слова~$w$ будет равно высоте нормальной формы 
этого слова относительно~$q$. 
  
  Для того чтобы находить время выполнения процесса обработки 
асинхронным ограниченным конвейером, зададим алфавит $A\hm= \{a_0, 
\ldots, a_{m-1}\}$, состоящий из $m\hm=3p$ букв. Слово~$w$, 
соответствующее обработке~$n$~элементов конвейером, задается 
следующим образом. Всякая ступень с~номером~$i$ разбивается на 
полутакты и~записывается как слово $a_{3i}a_{3i+1}^{2\tau_i-2} a_{3i+2}$. 
Здесь~$a_{3i}$~--- начальный полутакт,\linebreak\vspace*{-12pt}

{ \begin{center}  %fig1
 \vspace*{-1pt}
    \mbox{%
 \epsfxsize=79.105mm 
 \epsfbox{hus-1.eps}
 }


\vspace*{6pt}


\noindent
{\small Сравнение асинхронной и~непрерывной обработки}
\end{center}
}


\vspace*{12pt}

%\addtocounter{figure}{1}



\noindent
 а~$a_{3i+2}$~--- последний 
полутакт ступени. Между ними находятся полутакты, выполняющиеся 
последовательно, и~их можно обозначить одинаковой буквой~$a_{3i+1}$. 
Слово~$w$ будет равно 
  $$
  \left( a_0 a_1^{2\tau_0-2} a_2 a_3 a_4^{2\tau_1-2} a_5\cdots a_{3(p-1)} 
  a_{3p-2}^{2\tau_{p-1}-2} a_{3p-1}\right)^n\,.
  $$
Отношение независимости состоит из пар полутактов, которые могут 
выполняться параллельно: 
\begin{multline*}
I=A^2\backslash \left( \left\{ a_0, a_1, a_2\right\}^2 \cup
\left\{ a_3, a_4, a_5\right\}^2 \cup\cdots \right.\\
\cdots \cup \left\{ a_{3p-3},
a_{3p-2}, s_{3p-1}\right\}^2\cup{}\\
\left.{}\cup \left\{ a_2, a_3\right\}^2 \cup
\left\{ a_5, a_6\right\}^2\cup \cdots \cup \left\{ a_{3p-4}, a_{3p-
3}\right\}^2\right)\,.
\end{multline*}
  
  Нормальная форма относительно~$q$ дает последовательность 
параллельно работающих блоков (ярусов) операций. Поскольку время 
выполнения операции равно полутакту, то время обработки~$n$~элементов 
с~по\-мощью ограниченного конвейера будет равно половине высоты этой 
нормальной формы.
  
  Рассмотрим пример работы программы. На рисунке крупными точками 
показан график зависимости времени обработки трех входных элементов 
с~по\-мощью асинхронного конвейера, ограниченного числом 
процессоров~$X$. Задержки ступеней равны $\tau_0\hm=1$, $\tau_1\hm=3$, 
$\tau_2\hm=1$, $\tau_3\hm=2$. График, полученный по формуле~(\ref{e2-h}), 
изображен в~виде ломаной линии. По формуле~(\ref{e2-h}) время 
$T_2(3)\hm=14$. Крупная точка $(2,13)$ показывает, что для асинхронного 
конвейера это время равно~13. Приведенный пример показывает также, что 
для асинхронного конвейера следствие~1 неверно, поскольку наименьшее 
целое~$q$, удовлетворяющее $q\hm\geq \sigma/\mu$, равно~3, 
а~минимальное число процессоров, при котором достигается наибольшая 
производительность, равно~2.
  


  Таблица~2 представляет собой таблицу занятости для этого примера.
  
\vspace*{-6pt}

\section{Заключение}

  Получена формула для расчета производительности ограниченного 
конвейера (теорема~1). Вытекающее из нее следствие~1 говорит о том, что\linebreak 
для некоторых конвейеров число функциональных устройств можно 
уменьшить и~это не приведет к~снижению производительности. Это \mbox{можно} 
применять для совершенствования архитектуры процес\-со\-ра. Другим 
возможным продолжением данной работы может стать развитие метода, 
подсказавшего формулировку теоремы~1. Он позволяет обобщать 
некоторые утверждения об однородных конвейерах на неоднородные. 
  
  В последнее время конвейеры широко применяются при разработке 
многопоточных приложений для облачных вычислений. Многопоточный 
конвейер, работающий на компьютере с~многоядерным процессором, 
ограничен числом процессорных ядер. В~этом случае работа каждого 
процессорного ядра сопровождается переключением контекста нити, что 
приводит к~замедлению работы конвейера. Аналогичная проблема 
возникает при реализации ограниченных конвейеров для многоядерных 
сигнальных процессоров и~вы\-чис\-ли\-тель\-ных сис\-тем с~массовым 
параллелизмом~--- появляются дополнительные накладные расходы, 
связанные с~общим управ\-ле\-ни\-ем ядрами в~рамках одного процессора. 
Полученная в~работе формула для расчета производительности не 
учитывает такого замедления. Построение аналитической модели, 
учитывающей эти накладные расходы, было бы одним из перспективных 
продолжений данной работы.

\vspace*{-6pt} 
    
{\small\frenchspacing
 {%\baselineskip=10.8pt
 \addcontentsline{toc}{section}{References}
 \begin{thebibliography}{99}
\bibitem{1-h}
\Au{Хусаинов А.\,А., Чернов~А.\,М., Маевская~Е.\,Д., Романченко~А.\,А.} Модели для 
расчета времени работы вычислительных конвейеров~// Актуальные проблемы 
науки: Мат-лы XXIII Междунар. научно-практич. конф.~--- М.: Спутник+, 2016. 
С.~83--91. 
\bibitem{2-h}
\Au{Emma P.\,G., Davidson~E.\,S.} Characterization of branch and data dependencies in 
programs for evaluating pipeline performance~// IEEE~T. Comput., 1987. Vol.~7. 
P.~859--875.
\bibitem{3-h}
\Au{Cheah H.\,Y., Fahmy~S.\,A., Kapre~N.} On data forwarding in deeply pipelined soft 
processors~// ACM/SIGDA Symposium (International) on Field-Programmable Gate 
Arrays Proceedings.~--- New York, NY, USA: ACM, 2015. P.~181--189.
\bibitem{4-h}
\Au{Merchant F., Chattopadhyay~A., Raha~S., Nandy~S.\,K., Narayan~R.} 
Accelerating 
BLAS and LAPACK via efficient floating point architecture design~// Parallel Process. 
Lett., 2017. Vol.~27. No.\,03n04. P.~1--7.
\bibitem{5-h}
\Au{Паттерсон Д., Хеннесси~Дж.} Архитектура компьютера и~проектирование 
компьютерных систем~/ Пер. с~англ.~--- СПб.: Питер, 2012. 784~с. 
(\Au{Patterson~D.\,A., Hennessy~J.\,L.} Computer organization and design.~---
4th ed.~--- 
Amsterdam: Elsevier, 2012. 703~p.)
\bibitem{6-h}
\Au{Hartstein A., Puzak~T.\,R.} The optimum pipeline depth for a~microprocessor~// ACM 
 Comp. Ar., 2002. Vol.~30. Iss.~2. P.~7--13. 
\bibitem{7-h}
\Au{Yao J., Miwa~S., Shimada~H.} Optimal pipeline depth with pipeline stage unification 
adoption~// ACM Comp. Ar., 2007. Vol.~35. Iss.~5. P.~3--9.
\bibitem{8-h}
\Au{Moreno A., C$\acute{\mbox{e}}$sar~E., Guevara~A., Sorribes~J., Margalef~T.} Load 
balancing in homogeneous pipeline based applications~// Parallel Comput., 2012. Vol.~38. 
Iss.~3. P.~125--139.
\bibitem{9-h}
\Au{Moreno A., Sikora~A., C$\acute{\mbox{e}}$sar~E., Sorribes~J., Margalef~T.} 
HeDPM: Load balancing of linear pipeline applications on heterogeneous systems~// 
J.~Supercomput., 2017. Vol.~73. Iss.~9. P.~3738--3760.
\bibitem{10-h}
\Au{Хусаинов А.\,А., Титова~Е.\,А.} Оптимальная глубина вычислительного 
конвейера при заданном объеме данных~// Вычислительные технологии, 2018. Т.~23. 
№\,1. С.~96--104.
\bibitem{11-h}
\Au{Amdahl G.\,M.} Validity of the single processor approach to achieving large scale 
computing capabilities~// AFIPS Spring Joint Computer Conference Proceedings.~--- New 
York, NY, USA: ACM, 1967. P.~483--485.
\bibitem{12-h}
\Au{Shen J.\,P., Lipasti~M.\,H.} Model processor design: Fundamental of superscalar 
processors.~--- New York, NY, USA: McGraw-Hill, 2005. 643~p.
\bibitem{13-h}
\Au{Коуги П.\,М.} Архитектура конвейерных ЭВМ~/ Пер. с~англ.~--- М.: Радио 
и~связь, 1985. 360~с. (\Au{Kogge~P.\,M.} The architecture of pipelined computers.~--- 
Washington, D.C., USA: McGraw-Hill, 1981. 335~p.)
\bibitem{14-h}
\Au{Husainov A.\,A.} Optimum depth of the bounded pipeline.~--- New York, 
NY, USA: Cornell University, 2018.  Preprint.
11~p. {\sf http://arxiv.org/abs/cs.DC/1807.11022v1}.
\bibitem{15-h}
\Au{Diekert V.} Combinatorics on traces.~--- Lecture notes in computer science ser.~--- 
Berlin: Springer-Verlag, 1990. Vol.~454. 169~p.
 \end{thebibliography}

 }
 }

\end{multicols}

\vspace*{-3pt}

\hfill{\small\textit{Поступила в~редакцию 30.08.19}}

\vspace*{8pt}

%\pagebreak

%\newpage

%\vspace*{-28pt}

\hrule

\vspace*{2pt}

\hrule

%\vspace*{-2pt}

\def\tit{PERFORMANCE OF~THE~BOUNDED PIPELINE}


\def\titkol{Performance of~the~bounded pipeline}

\def\aut{A.\,A.~Khusainov}

\def\autkol{A.\,A.~Khusainov}

\titel{\tit}{\aut}{\autkol}{\titkol}

\vspace*{-11pt}


\noindent
Komsomolsk-na-Amure State University, 27~Lenina Prosp.,  
Komsomolsk-on-Amur, Khabarovsk Region 681013, Russian Federation

\def\leftfootline{\small{\textbf{\thepage}
\hfill INFORMATIKA I EE PRIMENENIYA~--- INFORMATICS AND
APPLICATIONS\ \ \ 2020\ \ \ volume~14\ \ \ issue\ 1}
}%
 \def\rightfootline{\small{INFORMATIKA I EE PRIMENENIYA~---
INFORMATICS AND APPLICATIONS\ \ \ 2020\ \ \ volume~14\ \ \ issue\ 1
\hfill \textbf{\thepage}}}

\vspace*{3pt} 



  \Abste{The paper is devoted to studying the performance of a bounded 
pipeline that is a computational pipeline, the number of active stages of which is 
bounded at any time by a fixed number. The bounded pipelines with the given 
sum and the maximum of delays of stages are considered. The stages can have 
different delays. The main problem is to build an analytical model for calculating 
the processing time of a given amount of data using this bounded pipeline. The 
solution is simplified if the constraint is treated as a structural pipeline hazard. 
This analytical model
is constructed for the case when the operation of a bounded 
pipeline has the property of continuity of processing for each input element. 
For 
such pipelines, the conjecture is proved in the paper that the minimum number 
of 
processors at which the greatest productivity is achieved 
is equal to the smallest 
integer not less than the ratio of the sum of stage delays to the 
maximum delay. It 
is established that if the property of continuity is not required,
then this conjecture 
is not true. The constructed model can be used to synchronize the operation of the 
stages of a bounded pipeline with the continuity property. If we do not require the 
property of continuity, then we get an asynchronous bounded pipeline, the 
synchronization of the work for the stages is carried out on the basis of the data 
readiness. The software is developed, which is based on the theory of trace 
monoids and allows one to calculate the processing time with an asynchronous 
bounded pipeline.}
  
  \KWE{computational pipeline; trace monoid; Foata normal form; pipeline 
performance; structural hazard}


\DOI{10.14357/19922264200112} 


\pagebreak

%\vspace*{-14pt}

%\Ack
%\noindent

 


%\vspace*{6pt}

  \begin{multicols}{2}

\renewcommand{\bibname}{\protect\rmfamily References}
%\renewcommand{\bibname}{\large\protect\rm References}

{\small\frenchspacing
 {%\baselineskip=10.8pt
 \addcontentsline{toc}{section}{References}
 \begin{thebibliography}{99}
\bibitem{1-h-1}
\Aue{Khusainov, A.\,A., A.\,M.~Chernov, E.\,D.~Mayevskaya, and 
A.\,A.~Romanchenko.} 2016. Modeli dlya rascheta vremeni raboty 
vychislitel'nykh konveyerov [Models for calculating the operating time of 
computational pipelines]. \textit{Aktual'nyye problemy nauki: Mat-ly XXIII 
Mezhdunar. nauchno-praktich. konf.}
[23rd Conference (International) on Actual Problems of Science Proceedings].
83--91.
\bibitem{2-h-1}
\Aue{Emma, P.\,G., and E.\,S.~Davidson.} 1987. Characterization of branch and 
data dependencies in programs for evaluating pipeline performance. \textit{IEEE~T.
 Comput.} 7:859--875.
\bibitem{3-h-1}
\Aue{Cheah, H.\,Y., S.\,A.~Fahmy, and N.~Kapre.} 2015. On data forwarding in 
deeply pipelined soft processors. \textit{ACM/SIGDA International Symposium 
on Field-Programmable Gate Arrays Proceedings}. New York, NY: ACM.  
181--189.
\bibitem{4-h-1}
\Aue{Merchant, F., A.~Chattopadhyay, S.~Raha, S.\,K.~Nandy, and 
R.~Narayan.} 2017. Accelerating BLAS and LAPACK via efficient floating 
point architecture design. \textit{Parallel Process. Lett.} 27(03n04):1--7.
\bibitem{5-h-1}
\Aue{Patterson, D.\,A., and J.\,L.~Hennessy.} 2012. Computer organization and 
design: The hardware/software interface. 4th ed. Amsterdam: Elsevier. 703~p.
\bibitem{6-h-1}
\Aue{Hartstein, A., and T.\,R.~Puzak.} 2002. The optimum pipeline depth for 
a~microprocessor. \textit{ACM Comp. Ar.}. 30(2):7--13. 
\bibitem{7-h-1}
\Aue{Yao, J., S.~Miwa, and H.~Shimada.} 2007. Optimal pipeline depth with 
pipeline stage unification adoption. \textit{ACM Comput. 
Ar.} 35(5):3--9.
\bibitem{8-h-1}
\Aue{Moreno, A., E.~C$\acute{\mbox{e}}$sar, A.~Guevara, J.~Sorribes, and 
T.~Margalef.} 2012. Load balancing in homogeneous pipelinebased applications. 
\textit{Parallel Comput.} 38(3):125--139.
\bibitem{9-h-1}
\Aue{Moreno, A., A.~Sikora, E.~C$\acute{\mbox{e}}$sar, J.~Sorribes, and 
T.~Margalef.} 2017. HeDPM: Load balancing of linear pipeline applications on 
heterogeneous systems. \textit{J.~Supercomput.} 73(9):3738--3760.
\bibitem{10-h-1}
\Aue{Husainov, A.\,A., and E.\,A.~Titova.} 2018. Optimal'naya glubina 
vychislitel'nogo konveyera pri zadannom ob''eme dannykh [Optimal depth of 
the computational pipeline for a~given amount of input data]. 
\textit{Vychislitel'nyye tekhnologii} [Computational Technologies]  
23(1):96--104.
\bibitem{11-h-1}
\Aue{Amdahl, G.\,M.} 1967. Validity of the single processor approach to 
achieving large scale computing capabilities. \textit{AFIPS Spring Joint 
Computer Conference Proceedings}. New York, NY: ACM. 483--485.
\bibitem{12-h-1}
\Aue{Shen, J.\,P., and M.\,H.~Lipasti.} 2005. \textit{Model processor design: 
Fundamental of superscalar processors.} New York, NY: McGraw-Hill. 643~p.
\bibitem{13-h-1}
\Aue{Kogge, P.\,M.} 1981. \textit{The architecture of pipelined computers.} 
Washington, D.C.: McGraw-Hill. 335~p.
\bibitem{14-h-1}
\Aue{Husainov, A.\,A.} 2018. Optimum depth of the bounded pipeline. 
{arXiv} 1807.11022 v1[cs.DC]. Available at: {\sf 
http://arxiv.org/abs/cs.DC/1807.11022v1} (accessed December~27, 2019).
\bibitem{15-h-1}
\Aue{Diekert, V.} 1990. \textit{Combinatorics on traces}. Lecture notes in 
computer science ser. Berlin: Springer-Verlag. Vol.~454. 169~p.
\end{thebibliography}

 }
 }

\end{multicols}

%\vspace*{-7pt}

\hfill{\small\textit{Received August 30, 2019}}

%\pagebreak

%\vspace*{-22pt}

\Contrl

\noindent
\textbf{Khusainov Akhmet A.} (b.\ 1951)~--- Doctor of Science in physics and 
mathematics, professor, Komsomolsk-na-Amure State University, 27~Lenina 
Prosp., Komsomolsk-on-Amur, Khabarovsk Region 681013, Russian 
Federation; \mbox{husainov51@yandex.ru}




\label{end\stat}

\renewcommand{\bibname}{\protect\rm Литература} 