\def\stat{bosov+st}

\def\tit{УПРАВЛЕНИЕ ВЫХОДОМ СТОХАСТИЧЕСКОЙ ДИФФЕРЕНЦИАЛЬНОЙ СИСТЕМЫ 
ПО~КВАДРАТИЧНОМУ КРИТЕРИЮ. IV.~АЛЬТЕРНАТИВНОЕ ЧИСЛЕННОЕ 
РЕШЕНИЕ$^*$}

\def\titkol{Управление выходом стохастической дифференциальной системы 
по~квадратичному критерию. IV%.  Альтернативное численное решение
}

\def\aut{А.\,В.~Босов$^1$, А.\,И.~Стефанович$^2$}

\def\autkol{А.\,В.~Босов, А.\,И.~Стефанович}

\titel{\tit}{\aut}{\autkol}{\titkol}

\index{Босов А.\,В.}
\index{Стефанович А.\,И.}
\index{Bosov A.\,V.}
\index{Stefanovich A.\,I.}


{\renewcommand{\thefootnote}{\fnsymbol{footnote}} \footnotetext[1]
{Работа выполнена при частичной поддержке РФФИ (проект 19-07-00187-A).}}


\renewcommand{\thefootnote}{\arabic{footnote}}
\footnotetext[1]{Институт проблем информатики Федерального исследовательского центра <<Информатика 
и~управление>> Российской академии наук, \mbox{ABosov@frccsc.ru}}
\footnotetext[2]{Институт проблем информатики Федерального исследовательского центра <<Информатика 
и~управление>> Российской академии наук, \mbox{AStefanovich@frccsc.ru}}

%\vspace*{-12pt}


 

\Abst{В исследовании задачи оптимального управления для диффузионного процесса Ито 
и~линейного управляемого выхода с~квадратичным критерием качества подводится 
промежуточный итог: для приближенного вычисления оптимального решения предлагается 
альтернативный классическому чис\-лен\-но\-му интегрированию метод на основе компьютерного 
моделирования. Метод позволяет применять статистическое оценивание для определения 
коэффициентов $\beta_t(y)$ и~$\gamma_t(y)$ полученной ранее функции Беллмана $V_t(y,z)\hm= 
\alpha_t z^2 \hm+ \beta_t(y)z\hm+ \gamma_t(y)$, определяющей оптимальное решение в~исходной 
задаче оптимального стохастического управления. Реализуется метод на основании свойств 
линейных уравнений в~частных производных параболического типа, описывающих~$\beta_t(y)$ 
и~$\gamma_t(y)$,~--- их эквивалентного описания в~форме стохастических дифференциальных 
уравнений и~тео\-ре\-ти\-ко-ве\-ро\-ят\-ност\-но\-го пред\-став\-ле\-ния решения, известного как 
уравнение А.\,Н.~Колмогорова, или эквивалентной интегральной форме, известной как формула 
Фейн\-ма\-на--Ка\-ца. Стохастические уравнения, соотношения для оптимального управления 
и~ряд вспомогательных параметров объединяются в~одну дифференциальную систему, для 
которой формулируется алгоритм имитационного моделирования решения, обеспечивающий 
необходимые выборки для статистического оценивания коэффициентов $\beta_t(y)$
и~$\gamma_t(y)$. 
Поставленный ранее численный эксперимент дополняется расчетами, выполненными 
пред\-став\-лен\-ным альтернативным методом, и~сравнительным анализом результатов.}

\KW{стохастическое дифференциальное уравнение; оптимальное управление; функция Беллмана; 
линейные уравнения параболического типа; уравнение А.\,Н.~Колмогорова; формула  
Фейн\-ма\-на--Ка\-ца; имитационное компьютерное моделирование; метод Мон\-те-Карло}

\DOI{10.14357/19922264200104} 
  
\vspace*{9pt}


\vskip 10pt plus 9pt minus 6pt

\thispagestyle{headings}

\begin{multicols}{2}

\label{st\stat}

\section{Введение}

\vspace*{-4pt}

     В работах~[1--3] представлено детальное исследование задачи 
управления линейным выходом стохастической дифференциальной системы 
по квадратичному критерию качества: решены уравнения динамического 
программирования и~получено оптимальное управление, применены 
традиционные сеточные методы для приближенного вычисления 
коэффициентов найденной функции Беллмана, предложено альтернативное 
описание этих коэффициентов, базирующееся на уравнении 
А.\,Н.~Колмогорова~\cite{4-bos} или эквивалентной ему интегральной 
формуле Фейн\-ма\-на--Ка\-ца~\cite{5-bos}. Основание результатам 
обеспечило то обстоятельство, что функцию Беллмана в~рассматриваемой 
задаче управления удалось привести к~виду

\noindent
\begin{equation}
V_t(y,z)=\alpha_t z^2+ \beta_t(y)z+ \gamma_t(y)\,,
\label{e1a-bos}
\end{equation}
 в~котором коэффициенты $\beta_t(y)$ 
и~$\gamma_t(y)$ описываются линейными уравнениями в~частных производных 
второго порядка параболического типа, коэффициент~$\alpha_t$ описывается 
уравнением Риккати и~легко вычисляется любым методом решения 
обыкновенных дифференциальных уравнений.
     
     Формула Фейн\-ма\-на--Ка\-ца или тео\-ре\-ти\-ко-ве\-ро\-ят\-ност\-ное 
представление терминального условия решения задачи Коши для уравнения 
А.\,Н.~Колмогорова в~данной работе используются как основание для 
представления численного метода,\linebreak альтернативного классическим сеточным 
методам решения параболических уравнений, а~именно: дополнив 
имеющееся описание исходной стохастической дифференциальной системы 
уравнениями для оптимального управления, для коэффициента~$\alpha_t$ 
и~для нескольких вспомогательных переменных, приближенное вычисление 
коэффициентов $\beta_t(y)$ и~$\gamma_t(y)$ удается реализовать в~форме 
статистического оценивания параметров случайных функций~--- решений 
общей дифференциальной сис\-те\-мы. Таким образом, приближенное решение 
обеспечивается методом Мон\-те-Кар\-ло, примененным к~смоделированным 
решениям стохастических дифференциальных уравнений. 
     
     Структура статьи такова. Привлекаемые исходные положения из~[1--3] 
кратко приведены в~разд.~2, описание общей дифференциальной системы 
дано в~разд.~3, процедура имитационного моделирования детально 
расписана в~разд.~4. Анализу качества предложенной численной процедуры 
посвящен разд.~5, где продолжено исследование того же численного 
эксперимента, что был пред\-став\-лен в~\cite{2-bos}, а~именно: проделанные 
в~\cite{2-bos} расчеты дополнены аналогичными расчетами, в~которых 
вмес\-то сеточных методов для расчета~$\beta_t(y)$ используется 
имитационное моделирование. Сравнительный анализ результатов 
представлен как в~форме сравнения поверхностей~$\beta_t(y)$, вычисленных 
разными способами, так и~динамикой целевого функционала в~исходной 
задаче управления.

\section{Краткие сведения о~задаче управления выходом}
     
     Рассматривается задача оптимизации квадратичного целевого 
функционала 
     \begin{multline}
     J\left( U_0^T\right)= {}\\
     {}= E\left\{ \int\limits^T_0 \big( S_t\left( s_ty_t-g_tz_t-
h_tu_t\right)^2 +G_t z_t^2+{}\right.\\
\left.{}+ H_tu_t^2\big)dt+
     S_T\left( s_T y_T-g_T z_T\right)^2+G_T z_T^2
     \vphantom{\int\limits^T_0}
     \right\}\,,\\ 
U_0^T=\left\{ u_t, 0\leq t\leq T
\right\}\,,
     \label{e1-bos}
     \end{multline}
     определяемого состоянием~$y_t$ 
стохастической дифференциальной системы
     \begin{equation}
     dy_t=A_t(y_t)\,dt+ \Sigma_t (y_t)\,dv_t\,,\enskip y_0=Y\,,
     \label{e2-bos}
     \end{equation}
     и~выходом~$z_t$, связанным с~$y_t$ линейно:
     \begin{equation*}
dz_t=a_ty_t\,dt +b_tz_t\,dt+c_tu_t\,dt+\sigma_t\,dw_t\,,\enskip z_0=Z\,,
%\label{e3-bos}
\end{equation*}
где $v_t$ и~$w_t$~--- независимые стандартные винеровские процессы; $Y$ 
и~$Z$~--- случайные величины с~конечным вторым моментом; $u_t$~--- 
управление. Ограничения на параметры модели, обеспечивающие 
корректность постановки задачи и~существование решения, а также решение 
уравнений динамического программирования приведены в~\cite{1-bos}. 
В~частности, показано, что оптимальное управление
\begin{multline}
u_t^*=-\fr{1}{2}\left( S_th_t^2+H_t\right)^{-1} \left(c_t\left(2\alpha_t 
z_t+\beta_t(y_t)\right)+{}\right.\\
\left.{}+2S_t\left( s_t y_t -g_t z_t\right) h_t\right)\,,
\label{e4-bos}
\end{multline}
доставляющее минимум функционалу $J(U_0^T)$, обеспечивается 
представлением функции Беллмана в~виде~(\ref{e1a-bos}), 
в~част\-ности имеет место равенство:
\begin{multline}
\min\limits_{U_0^T} J\left( U_0^T\right) =J\left(\left( U^*\right)^T_0\right)=
E\left\{ V_0(Y,Z)\right\} ={}\\
{}=E\left\{ \alpha_0 
Z^2+\beta_0(Y)Z+\gamma_0(Y)\right\}\,.
\label{e5-bos}
\end{multline}
     
     Коэффициент~$\alpha_t$ описывается уравнением Риккати (приведено 
ниже), коэффициенты $\beta_t(y)$ и~$\gamma_t(y)$ задаются следующими 
дифференциальными уравнениями в~частных производных:
     \begin{multline*}
     \fr{\partial\beta_t(y)}{\partial t} +A_t(y) \fr{\partial\beta_t(y)}{\partial y} 
+\fr{1}{2} \Sigma_t^2(y) \fr{\partial^2\beta_t(y)}{\partial^2 y} +M_t y+{}\\
{}+N_t\beta_t (y) =0\,,\enskip 
     \beta_T(y)=-2S_T s_T g_T y\,;
     %\label{e6-bos}
     \end{multline*}
     
     \vspace*{-12pt}
     
     \noindent
     \begin{multline}
\fr{\partial\gamma_t(y)}{\partial t}+A_t(y) \fr{\partial\gamma_t(y)}{\partial 
y}+\fr{1}{2}\Sigma_t^2(y)
\fr{\partial^2\gamma_t(y)}{\partial y^2} +L_t(y)=0\,,\\
\gamma_T(y)= S_T s_T^2 y^2\,,
\label{e7-bos}
\end{multline}
где
\begin{align*}
M_t&= 2\left( \alpha_t\left(a_t+\left( S_t h_t^2+H_t\right)^{-1} c_t S_t h_t 
s_t\right) - {}\right.\\
&\left.\hspace*{10mm}{}-\left( S_t -\left( S_t h_t^2+H_t\right)^{-1} S_t^2 h_t^2\right) s_t 
g_t\right)\,;\\
N_t&= b_t-\left( S_t h_t^2 +H_t\right)^{-1} c_t S_t h_t g_t -{}\\
&\hspace*{32mm}{}-\left( S_t h_t^2 +H_t\right)^{-1} c_t^2 \alpha_t\,;
\\
L_t(y) &=
\beta_t(y)\left( a_t+\left( S_th_t^2 +H_t\right)^{-1} c_t S_t h_t s_t\right) y+ {}\\
&\hspace*{10mm}{}+\left( S_t-\left( S_t h_t^2+H_t\right)^{-1} S_t^2 h_t^2\right) s_t^2 y^2 -{}\\
&\hspace*{25mm}{}-
\fr{1}{4}\left( S_t h_t^2 +H_t\right)^{-1} c_t^2 \beta_t^2(y)\,.
\end{align*}
     
     Исследование уравнений~(\ref{e4-bos}) и~(\ref{e5-bos}), выполненное 
в~\cite{3-bos}, позволило записать систему стохастических 
дифференциальных уравнений А.\,Н.~Колмогорова, решение которой 
связано тео\-ре\-ти\-ко-ве\-ро\-ят\-ност\-ным соотношением с~искомыми 
коэффициентами $\beta_t(y)$ и~$\gamma_t(y)$. Соответствующее свойство 
в~окончательном виде представлено следующими соотношениями:

\noindent
     \begin{multline}
\beta_t(y) =\displaystyle
     E\left\{ -2S_T s_T g_T \exp\left\{ \int\limits_t^T N_\tau\,d\tau\right\} 
y_T+{}\right.\\
\displaystyle \left.{}+\int\limits_t^T \exp
     \left\{ \int\limits_t^\tau N_s\,ds\right\} M_\tau y_\tau\,d\tau
     \left\vert \mathcal{F}_t^y\right.\right\}\,;
          \label{e8-bos}
     \end{multline}
     
     \noindent
     \begin{equation}
\gamma_t(y)=\displaystyle
     E\left\{ S_T s_T^2 y_T^2 +\int\limits_t^T L_\tau(y_\tau)\,d\tau\left\vert 
\mathcal{F}_t^y\right.\right\}\,,
     \label{e8a-bos}
     \end{equation}
где $\mathcal{F}_t^y$~--- $\sigma$-ал\-геб\-ра, порожденная 
значениями~$y_\tau$ до момента~$t$ включительно. Равенство~(\ref{e8-bos}) 
пред\-став\-ля\-ет собой частный случай формулы  
Фейн\-ма\-на--Ка\-ца~\cite{5-bos}.

\section{Общая дифференциальная система}
     
     Соотношения~(\ref{e8-bos})--(\ref{e8a-bos})
      в~полной мере выражают тео\-ре\-ти\-ко-ве\-ро\-ят\-ност\-ную 
      связь между решениями линейных уравнений в~частных 
производных второго порядка параболического типа, т.\,е.\ уравнений для 
искомых коэффициентов $\beta_t(y), \gamma_t(y)$ и~стохастическим 
дифференциальным уравнением~(\ref{e2-bos}) для состояния 
оптимизируемой динамической системы. Но форма их записи не слишком 
удобна для численной реализации, поэтому искомое решение ниже 
представлено в~эквивалентной дифференциальной форме:
     \begin{equation}
     dy_t=A_t(y_t)\,dt+\Sigma_t(y_t)\,dv_t\,,\enskip y_0=Y\,;
     \label{e9-bos}
     \end{equation}
     
\vspace*{-12pt}

     \noindent
     \begin{multline}
          \fr{\partial \alpha_t}{\partial t} +2\alpha_t \left( b_t- \left( S_t 
h_t^2+H_t\right)^{-1} c_t S_t h_t g_t\right)+{}\\
{}+
     \left( S_t-\left( S_t h_t^2 +H_t\right)^{-1} S_t^2 h_t^2\right) g_t^2+
     G_t-{}\\
     \hspace*{-10mm}{}- \left( S_t h_t^2+H_t\right)^{-1} c_t^2 \alpha_t^2=0\,,\enskip
     \alpha_T=S_T g_t^2+G_T\,;
     \label{e10-bos}
     \end{multline}
     
     \noindent
\begin{equation}
\left.
\begin{array}{rl}
dy_t^{(1)}&= M_t I_t^{-1} y_t \,dt\,,\enskip
y_0^{(1)}=0\,;\\[6pt]
 dy_t^{(2)}&=L_t(y_t)\,dt\,,\enskip y_0^{(2)}=0\,;\\[6pt]
 di_t&=N_t\,dt\,,\enskip
i_0=0\,,\ I_t=\exp \{-i_t\}\,.
\end{array}
\right\}
\label{e11-bos}
\end{equation}
    % 
     Здесь уравнение~(\ref{e9-bos}) такое же, как
      и~уравнение~(\ref{e2-bos}), но формально оно должно пониматься 
в~смысле другого вероятностного пространства, так как описывает не 
состояние исходной стохастической сис\-те\-мы, а~некоторый 
инструментальный процесс, построенный для уравнения 
А.\,Н.~Колмогорова. То обстоятельство, что использованы одинаковые 
обозначения~$y_t$, объясняется желанием подчеркнуть стохастическую 
эквивалентность~(\ref{e9-bos}) и~(\ref{e2-bos}) и~не вводить без 
необходимости дополнительные обозначения. Уравнение~(\ref{e10-bos})~--- уравнение 
Риккати для коэффициента~$\alpha_t$. Соотношениями~(\ref{e11-bos}) 
задаются вспомогательные переменные, используемые для более компактной 
записи результата.
     
     Предложенная дифференциальная  
система~(\ref{e9-bos})--(\ref{e11-bos}) позволяет записать следующие 
выражения для коэффициентов~$\beta_t(y)$ и~$\gamma_t(y)$:
     \begin{equation}
     \left.
     \begin{array}{rl}
     \beta_t(y)&=I_t E\left\{
     \vphantom{y_t^{(1)}}
      -2S_T s_T g_T I_T^{-1} y_T(t,y)+{}\right.\\[6pt]
     &\left.\hspace*{15mm}{}+y_T^{(1)}(t,y)-
y_t^{(1)}(t,y)\right\}\,;\\[6pt]
     \gamma_t(y)&=E\left\{ S_T s_T^2 y_T^2(t,y)+y_T^{(2)}(t,y)-{}\right.\\[6pt]
&\left.\hspace*{32mm}{}-y_t^{(2)}(t,y)
     \vphantom{y_t^{(1)}}
     \right\}\,,
     \end{array}
     \right\}
     \label{e12-bos}
     \end{equation}
где через~$y_T(t,y)$ обозначена терминальная точка решения 
уравнения~(\ref{e9-bos}) при условии $y_t\hm=y$, через $y_t^{(1)}(t,y)$ 
и~$y_t^{(2)}(t,y)$ обозначены решения~(\ref{e11-bos}), удовле\-тво\-ря\-ющие тому 
же условию $y_t\hm=y$, в~том чис\-ле и~терминальные значения 
$y_T^{(1)}(t,y)$ и~$y_T^{(2)}(t,y)$. 
     
     Заметим, что совокупность выписанных уравнений очевидным образом 
декомпозируется на три самостоятельные части. Во-пер\-вых, это 
обыкновенные дифференциальные уравнения для~$\alpha_t$ и~$i_t$, 
которые решаются первыми любым численным методом. Вторая часть~--- 
это стохастическое дифференциальное уравнение~(\ref{e7-bos}), решение 
которого требует соответствующего подхода. Третья часть~--- это уравнения 
для вспомогательных переменных, которые представляют собой 
среднеквадратические интегралы от решения~(\ref{e9-bos}).
     
\section{Приближенное решение методом имитационного 
моделирования}

     Представление уравнений для искомых коэффициентов с~помощью 
системы~(\ref{e9-bos})--(\ref{e11-bos}) преследует очевидную цель~--- 
использовать для ее решения метод Мон\-те-Кар\-ло. Для этого под 
решением уравнения~(\ref{e9-bos}) будем понимать смоделированный пучок 
$\{ (y_t)_l\}^L_{l=1}$, содержащий~$L$~траекторий процесса~$y_t$. Тогда 
вместо точных расчетов по формулам~(\ref{e10-bos}) можно получить 
приближенные решения, используя статистические оценки 
$\overline{E}\{X\}\hm= (1/L)\sum\nolimits^L_{l=1} X_l$, построенные по 
выборке $\{ X_l\}^L_{l=1}$, порождаемой пучком $\{(y_t)_l\}^L_{l=1}$, 
вместо математических ожиданий~$E\{X\}$.
     
     Для формального описания алгоритма введем обозначения:
     \begin{itemize}
\item разбиение (для простоты равномерное) во временн$\acute{\mbox{о}}$й области $0\hm\leq 
t\hm\leq T$: $\{t_n\}^N_{n=0}$, $t_0\hm=0\hm< t_1<\cdots < t_{N-1}\hm< t_N 
\hm= T$, $\delta\hm= t_n\hm-t_{n-1}\hm= T/N\hm\ll 1$;
\item разбиение (равномерное и~одинаковое для всех~$t$) в~об\-ласти 
значений~$y_t$:  $\{y_m\}^M_{m=0}$, $-\infty\hm< y_0\hm< y_1<\cdots <
y_{M-1} \hm< y_M\hm< +\infty$, $\varepsilon\hm= y_m\hm-y_{m-1}\hm= 
y_M/M\hm\ll 1$;
\item оператор <<ближайшей точки>> $y_{m^*}\hm=\mathrm{near}\,(y)$, 
где~$y_{m^*}$~--- та точка разбиения $\{y_m\}^M_{m=0}$, для которой 
$m^*\hm= \argmin\limits_{0\leq m\leq M} \vert y\hm- y_m\vert$.
\end{itemize}
     
     При росте~$N$ будет уменьшаться~$\delta$. При росте~$M$ 
дополнительно потребуем, чтобы расширялся интервал $[y_0, y_M]$ 
и~уменьшалось~$\varepsilon$ так, чтобы при $M\hm\to +\infty$ выполнялось 
$y_0\hm\to -\infty$, $y_M\hm\to +\infty$, $\varepsilon\hm\to 0$. Ближайшая 
точка в~рассматриваемом скалярном случае, естественно, определяется 
элементарно~--- сравнением двух расстояний от заданной точки до границ 
интервала разбиения, в~который она попала. Однако сохраним этот оператор 
как заготовку для обобщения на многомерный случай, что добавит 
содержательности этому определению. Вопросы сходимости выходят за 
рамки данного исследования, точнее в~рассмотрении этих вопросов не 
видится какого-либо принципиального содержания, поскольку относятся они 
не столько к~предлагаемому алгоритму, сколько к~используемым в~нем 
классическим методам. Итак, алгоритм.
     \begin{description}
\item[Шаг~1.] Используя разбиение $\{t_n\}^N_{n=0}$ решить численно 
обыкновенные дифференциальные уравнения~(\ref{e10-bos}) и~(\ref{e11-bos})
для~$\alpha_t$ и~$i_t$. Упрощая обозначения, для полученных 
приближенных решений будем использовать те же, что и~для точных: 
$\alpha_{t_n}$ и~$i_{t_n}$, $n\hm=0,\ldots , N$. Аналогично поступим далее 
с~обозначениями для приближенных решений стохастических уравнений.
     
   \item[Шаг~2.] Используя разбиения $\{t_n\}^N_{n=0}$ 
и~$\{y_m\}^M_{m=0}$, смоделировать приближенно пучок~$L$~траекторий 
$\{ (y_{t_n})_l, (y_{t_n}^{(1)})_l, (y_{t_n}^{(2)})_l\}^L_{l=1}$ процессов 
$y_t, y_t^{(1)}, y_t^{(2)}$ для всех $t\hm\in \{t_n\}^N_{n=0}$, так чтобы 
$(y_{t_n})_{l}\hm\in \{y_m\}^M_{m=0}$, т.\,е.\ значения всех смоделированных 
приближенно траекторий~$y_t$ проходили через узлы выбранного 
разбиения. Для этого выполнить следующие шаги.
     \begin{description}
\item[Шаг~2.1.] Записать приближенные уравнения для $y_t, y_t^{(1)}, 
y_t^{(2)}$, заменив в~точных уравнениях~(\ref{e9-bos}), (\ref{e11-bos}) 
коэффициенты~$\alpha_t$ и~$i_t$ их приближенными значениями, 
полученными на шаге~1.
     \item[Шаг~2.2.] Записать разностный аналог стохастической 
дифференциальной системы~(\ref{e9-bos}), используя любую 
(предпочтительно явную) схему численного интегрирования~\cite{6-bos}, 
например Эйлера, Мильштейна, Тейлора, Рун\-ге--Кут\-ты.
     \item[Шаг~2.3.] Используя на шаге~$n$, т.\,е.\ для $t\hm= t_n$, решение $\{ 
(y_{t_{n-1}})_l, (y^{(1)}_{t_{n-1}})_l, (y^{(2)}_{t_{n-1}})_l\}^L_{l=1}$, 
полученное на шаге $n\hm-1$ и~удовлетворяющее привязке к~разбиению 
$\{y_m\}^M_{m=0}$, смоделировать (согласно выбранной схеме численного 
решения стохастического уравнения) промежуточное решение для $\{ 
(y_{t_{n}})_l, (y^{(1)}_{t_{n}})_l, (y^{(2)}_{t_{n}})_l\}^L_{l=1}$ 
и~заменить точку $(y_{t_{n}})_l$ промежуточного решения на 
<<ближайшую точку>>: $\mathrm{near}((y_{t_{n}})_l)$. Отличие для 
$t\hm=0$ состоит в~том, что вместо разностной схемы моделируется выборка 
для случайной величины~$Y$, задающей начальное условие в~(\ref{e2-bos}). 
Шаги~2.1--2.3 повторяются вплоть до $t\hm=T$ или $n\hm=N$.
 \end{description}    
     
    \item[Шаг~3.] Вычислить приближенные значения 
$\{\beta_{t_n}(y_m),\gamma_{t_n}(y_m)\}^{n=0,\ldots, N}_{m=0,\ldots , M}$, 
для чего выполнить следующие шаги.
   \begin{description}  
\item[Шаг~3.1.] Для каждого $t\hm=t_n$ и~каждого $y\hm=y_m$ из 
имеющегося пучка траекторий выбрать те, что отвечают условию 
$y_t\hm=y$, и~сформировать выборки $\{ ( y_T(t,y))_l, (y_T^{(1)}(t,y))_l, 
(y_T^{(2)}(t,y))_l\}^{L_{n,m}}_{l=1}$ некоторой длины~$L_{n,m}$.
     
     \item[Шаг~3.2.] Вычислить приближенные решения,   
используя в~соответствующих соотношениях в~(\ref{e12-bos}) вместо 
математического ожидания~$E\{\cdot\}$ аппроксимацию 
$\overline{E}\{\cdot\}$, суммируя в~ней именно тот пучок, что выбран на 
предыду\-щем шаге для $t\hm=t_n$ и~$y\hm=y_m$ Повторять шаг~3.1 для 
сле\-ду\-ющих значений~$t,y$.
\end{description}
     
     \item[Шаг~4.] Дополнив имеющиеся смоделированные данные выборкой для 
начального значения $z_0\hm=Z$ (с использованием уже имеющейся 
выборки для~$Y$ и~вычисленных на предыдущем шаге $\beta_0(Y_l)$ 
и~$\gamma_0(Y_l)$), вычислить, используя~(\ref{e5-bos}), оценку  
Мон\-те-Кар\-ло для $J((U^*)_0^T)$.
     
    \item[Шаг~5.] Смоделировать приближенно пучок~$L$~траекторий $\{ 
(z_{t_n})_l, (u^*_{t_n})_l\}^L_{l=1}$ процессов~$z_t$, $u_t^*$ для всех 
$t\hm\in \{t_n\}^N_{n=0}$.
\end{description}

\begin{figure*}[b] %fig1
\vspace*{-3pt}
    \begin{center}  
  \mbox{%
 \epsfxsize=163mm 
 \epsfbox{bos-1.eps}
 }
\end{center}
\vspace*{-17pt}
\begin{minipage}[t]{80mm}
\Caption{Поверхность $\beta_t^0(y)$}
\end{minipage}
%\end{figure*}
\hfill
%\begin{figure*} %fig2
\vspace*{-17pt}
\begin{minipage}[t]{80mm}
\Caption{Поверхность $\beta_t^*(y)$}
\end{minipage}
\vspace*{12pt}
\end{figure*}
     
     Выбор на шаге~1 метода приближенного интегрирования не имеет 
принципиального значения для алгоритма в~целом. Это может быть и~явный 
метод Эйлера, и~метод Рун\-ге--Кут\-ты 4-го порядка, как и~любой неявный 
метод численного интегрирования обыкновенного дифференциального 
уравнения. От этого метода требуется быть устойчивым в~имеющейся 
постановке и~обеспечивать сходимость приближенного решения к~точному 
при расширении разбиения $\{t_n\}^N_{n=0}$. Выбор на шаге~2.2 
конкретного метода для решения уравнения Ито~\cite{6-bos} также не 
принципиален, естественно, при условии, что будет получено <<хорошее>> 
приближение искомого сильного решения. Соответственно, условия 
и~свойства сходимости этого метода принимаются такими, как есть. Следует 
обратить внимание, что на шаге~2 привязка к~разбиению выполняется только 
для переменной, соответствующей~$y_t$, что, как уже несколько раз 
подчеркивалось, вытекает из вспомогательного характера $y_t^{(1)}, 
y_t^{(2)}$. Точность, обеспечиваемая шагом~3, конечно, во многом будет 
определяться как качеством выбора разбиения фазовой переменной, так 
и~гладкостью всех элементов системы. Если система окажется <<удачной>>, 
то исходно смоделированного пучка траекторий окажется достаточно для 
выполнения всех осреднений на всем интервале управления. Если нет, то 
нехватку, возможно, компенсирует то обстоятельство, что вероятность 
траекторий, проходящих через <<неудачные>> точки разбиения 
$\{y_m\}^M_{m=0}$, мала, а~значит, мал их вклад в~целевую 
функцию~(\ref{e1-bos}). В~любом случае следует исходить из того, что, 
наращивая объем моделируемой информации, всегда можно добиться любой 
точности оценивания, оправдывая это действием закона больших 
чисел~\cite{7-bos}. Наконец, отметим, что последние два шага выполняются в~том случае, когда требуется провести ка\-кой-ли\-бо анализ оптимального 
управления (определить достижимое качество управления, сравнить 
с~другими управлениями, визуально охарактеризовать траектории выхода 
и/или управ\-ле\-ния и~т.\,п.).

\vspace*{-9pt}

\section{Численный пример}

\vspace*{-2pt}
     
     Для применения предложенного алгоритма были продолжены 
эксперименты с~модельным примером, детально проанализированным 
в~\cite{2-bos}, а~именно: использована простая модель для показателя RTT 
(Round-Trip Time) сетевого протокола TCP (Transmission Control Protocol), 
предложенная в~\cite{8-bos} и~конкретизированная следующим уравнением:

\noindent
     \begin{multline}
     dy_t=\left( 1-0{,}1y_t\right) dt+0{,}5\sqrt{y_t}\,dv_t\,,\\ 
     y_0=Y\sim N(15{,}9)\,.
     \label{e13-bos}
     \end{multline}
Здесь $N(M,D)$~--- нормальное распределение со средним~$M$ 
и~дисперсией~$D$.
     
     Равномерное разбиение временн$\acute{\mbox{о}}$й области $0\hm\leq t\hm\leq T$ для 
$T\hm=5$ выполнено для разбиения в~об\-ласти $[-10, 40]$ значений~$y_t$~--- 
для $\varepsilon\hm=0{,}01$, $M\hm=5000$. Итоговый расчет в~отношении 
$\beta_t(y)$ в~\cite{2-bos} проведен для неявной сеточной схемы 
и~граничного условия в~задаче Дирихле: $\beta_t(y)\hm=0$, $y\hm=-10;40$. 
Соответствующее приближенное решение обозначается~$\beta_t^0(y)$, 
порождаемое им управление~--- $u_t^0$.
     
     Альтернативное численное решение~$\beta_t^*(y)$ (соответствующее 
управление~$u_t^*$), реализованное представленным выше алгоритмом, 
использует ту же сеточную структуру и~выборку из $L\hm= 
N \,\mathrm{M}\cdot 1000\hm= 25\cdot 10^9$ траекторий~$y_t$, 
аппроксимирующих~(\ref{e13-bos}) согласно явному методу Эйлера.
     
     Результаты иллюстрируются, во-пер\-вых, рис.~1, на котором 
представлена поверхность $\beta_t^0(y)$, и~рис.~2, на котором изображена 
поверхность $\beta_t^*(y)$, а также табл.~1, в~которой демонстрируются 
отклонения $\vert \beta_t^0(y)\hm- \beta_t^*(y)\vert$, и~табл.~2, в~которой 
показана динамика целевых функционалов $J((U^0)_0^T)$ и~$J((U^*)_0^T)$. 

     
     В отношении показанных на рис.~1 и~2 поверхностей можно сделать 
вывод, что они аппроксимируют одну и~ту же функцию, причем первая 
аппроксимация обеспечивает некоторую визуальную гладкость, вторая 
отличается некоторым <<дрожанием>>, характерным для оценок  
Мон\-те-Карло.
     
\begin{table*}\small %tabl1
\begin{center}
\Caption{Отклонения $\vert \beta_t^0(y)-\beta_t^*(y)\vert$}
\vspace*{2ex}

\begin{tabular}{|c|l|l|c|c|c|c|}
\hline
\multicolumn{1}{|c|}{\raisebox{-6pt}[0pt][0pt]{$t$}}&\multicolumn{6}{c|}{$y$}\\
\cline{2-7}
&\multicolumn{1}{c|}{$-10{,}0$}&\multicolumn{1}{c|}{$0{,}0$}&$10{,}0$&$20{,}0$&$30{,}0$&$40{,}0$\\
\hline
0&0,57 (6,2\%)&0,00 (0,7\%)&0,16 (1,5\%)&0,17 (0,8\%)&0,25 (0,8\%)&1,41 (3,5\%)\\
1&1,07 (11,7\%)&0,00 (0,3\%)&0,06 (0,5\%)&0,13 (0,6\%)&0,57 (1,8\%)&1,33 (3,3\%)\\
2&0,47 (5,1\%)&0,01 (1,3\%)&0,30 (2,8\%)&0,39 (1,9\%)&0,14 (0,4\%)&0,56 (1,4\%)\\
3&0,64 (7,0\%)&0,04 (7,5\%)&0,25 (2,4\%)&0,75 (3,6\%)&0,16 (0,5\%)&0,98 (2,4\%)\\
4&0,43 (4,6\%)&0,07 (12,4\%)&0,18 (1,6\%)&0,27 (1,2\%)&0,31 (1,0\%)&2,51 (6,1\%)\\
5&0,00 (0,0\%)&0,00 (0,0\%)&0,00 (0,0\%)&0,00 (0,0\%)&0,00 (0,0\%)&0,00 (0,0\%)\\
\hline
\end{tabular}
\end{center}
\vspace*{-6pt}
\end{table*}
     
     В табл.~1 каждое отклонение дополнено в~скобках относительным 
отклонением по отношению\linebreak к~поверхности $\beta_t^*(y)$. Как видно, 
совпадение поверхностей с~точностью 1\%--4\% имеет место\linebreak в~большинстве 
точек внут\-ри интервала $[-10,40]$.
Исключение составляют точки левой 
границы $y\;=$\linebreak\vspace*{-12pt}

\pagebreak

%\begin{table*}
%tabl2

\noindent
{{\tablename~2}\ \ \small{Динамика целевых функционалов $J((U^0)_0^T)$ и~$J((U^*)_0^T)$ }}
%\vspace*{2ex}

\vspace*{3pt}

\begin{center}
{\small
\tabcolsep=8pt
\begin{tabular}{|c|c|c|r|}
\hline
&&&\\[-9pt]
$t$&$J((U^0)_0^t)$&$J((U^*)_0^t)$&
\multicolumn{1}{c|}{$\vert J((U^0)_0^t) - J((U^*)_0^t)\vert$}\\
\hline
1&134,17&132,10&2,08 (1,5\%)\hspace*{8mm}\\
2&266,11&258,87&7,23 (2,7\%)\hspace*{8mm}\\
3&394,32&381,55&12,77 (3,2\%)\hspace*{8mm}\\
4&517,93&497,49&20,44 (3,9\%)\hspace*{8mm}\\
5&742,21&704,48&37,73 (5,1\%)\hspace*{8mm}\\
\hline
\end{tabular}
}
\end{center}

%\end{table*}

\vspace*{9pt}

     
     
     \noindent
      $=-10{,}0$, относительно которых можно предполагать неудачу 
в~выборе граничного условия. Также несколько неудачных значений 
присутствуют в~точке $y\hm=0{,}0$, где сечение поверхности $\beta_t^0(y)$ 
принимает значения, близкие к~нулю. Заметим, что, хотя поверхность 
$\beta_t^0(y)$ интерпретируется как некоторый эталон, на самом деле 
неизвестно, какое именно решение ближе к~$\beta_t(y)$. Косвенно оценить 
это помогают результаты применения соответствующих управлений 
$(U^0)_0^T\hm= \{ u_t^0, 0\hm\leq t\hm\leq 5\}$ и~$(U^*)_0^T\hm= \{ u_t^*, 
0\hm\leq t\hm\leq 5\}$, иллюстрируемые в~динамике табл.~2.
     

     Приведенные результаты не только подтвердили\linebreak высокую точ\-ность 
расчета, обеспечиваемую предложенным алгоритмом, но и~показали 
превосходство, обеспечиваемое управ\-ле\-ни\-ем~$u_t^*$, что косвенно 
свидетельствует о~более высокой точ\-ности аппрок\-си\-ма\-ции~$\beta_t(y)$, 
обеспечиваемой пред\-став\-лен\-ным алгоритмом.

\vspace*{-11pt}

\section{Заключение}

\vspace*{-3pt}
     
     В статье подведен промежуточный итог исследованию задачи 
оптимизации линейного выхода нелинейной дифференциальной системы по 
квад\-ратичному критерию, выполненному в~[1--3].\linebreak В~рассматриваемой 
задаче оптимизации имеется оптимальное решение, полученное методом 
динамического программирования, приближенное решение на основе 
сеточных методов решения дифференциальных уравнений и~альтернативное\linebreak 
приближенное решение, базирующееся на тео\-ре\-ти\-ко-ве\-ро\-ят\-ност\-ной связи 
решений параболических уравнений в~частных производных 
и~стохастических дифференциальных систем. Завершающим в~полном 
объеме исследование вопросом могло бы стать изучение рассматриваемой 
постановки для случая неполной информации о состоянии системы~$y_t$, 
т.\,е.\ предположения, что состояние системы доступно посредством 
косвенных наблюдений, обеспечиваемых выходом~$z_t$. Данный вопрос 
предполагается рассмотреть в~последующих пуб\-ли\-ка\-циях.

\vspace*{-8pt}

{\small\frenchspacing
 {%\baselineskip=10.8pt
 \addcontentsline{toc}{section}{References}
 \begin{thebibliography}{9}
\bibitem{1-bos}
\Au{Босов А.\,В., Стефанович~А.\,И.} Управление выходом стохастической 
дифференциальной системы по квадратичному критерию. I.~Оптимальное решение 
методом динамического программирования~// Информатика и~её применения, 2018. 
Т.~12. Вып.~3. С.~99--106.
\bibitem{2-bos}
\Au{Босов А.\,В., Стефанович~А.\,И.} Управление выходом стохастической 
дифференциальной системы по квадратичному критерию. II.~Численное решение 
уравнений динамического программирования~// Информатика и~её применения, 2019. 
Т.~13. Вып.~1. С.~9--15.
\bibitem{3-bos}
\Au{Босов А.\,В., Стефанович~А.\,И.} Управление выходом стохастической 
дифференциальной системы по квадратичному критерию. III.~Анализ свойств 
оптимального управления~// Информатика и~её применения, 2019. Т.~13. Вып.~3.  
С.~41--49.
\bibitem{4-bos}
\Au{Гихман И.\,И., Скороход~А.\,В.} Теория случайных процессов.~--- М.: Наука, 
1975.  Т.~III. 496~с.
\bibitem{5-bos}
\Au{{\ptb{\O}}\,\,ksendal B.} Stochastic differential equations. An introduction with  
applications.~--- New York, NY, USA: Springer-Verlag, 2003. 379~p.
\bibitem{6-bos}
\Au{Kloden P.\,E., Platen~E.} Numerical solution of stochastic differential equations.~--- 
Berlin--Heidelberg: Springer-Verlag, 1992. 636~p.
\bibitem{7-bos}
\Au{Ширяев А.\,Н.} Вероятность.~--- 2-е изд.~--- М.: Наука, 1989. 640~с.
\bibitem{8-bos}
\Au{Bohacek S., Rozovskii B.} A~diffusion model of roundtrip time~// Comput. Stat.  
Data An., 2004. Vol.~45. Iss.~1. P.~25--50.
 \end{thebibliography}

 }
 }

\end{multicols}

\vspace*{-12pt}

\hfill{\small\textit{Поступила в~редакцию 28.08.19}}

%\vspace*{8pt}

%\pagebreak

\newpage

\vspace*{-28pt}

%\hrule

%\vspace*{2pt}

%\hrule

%\vspace*{-2pt}

\def\tit{STOCHASTIC DIFFERENTIAL SYSTEM OUTPUT CONTROL 
BY~THE~QUADRATIC CRITERION.\\ IV.~ALTERNATIVE NUMERICAL 
DECISION}


\def\titkol{Stochastic differential system output control 
by~the~quadratic criterion. IV.~Alternative numerical 
decision}

\def\aut{A.\,V.~Bosov and~A.\,I.~Stefanovich}

\def\autkol{A.\,V.~Bosov and~A.\,I.~Stefanovich}

\titel{\tit}{\aut}{\autkol}{\titkol}

\vspace*{-16pt}


\noindent
Institute of Informatics Problems, Federal Research Center ``Computer Science 
and Control'' of the Russian Academy of Sciences, 44-2~Vavilov Str., Moscow 
119333, Russian Federation

\def\leftfootline{\small{\textbf{\thepage}
\hfill INFORMATIKA I EE PRIMENENIYA~--- INFORMATICS AND
APPLICATIONS\ \ \ 2020\ \ \ volume~14\ \ \ issue\ 1}
}%
 \def\rightfootline{\small{INFORMATIKA I EE PRIMENENIYA~---
INFORMATICS AND APPLICATIONS\ \ \ 2020\ \ \ volume~14\ \ \ issue\ 1
\hfill \textbf{\thepage}}}

\vspace*{2pt} 



\Abste{In the study of the optimal control problem for the Ito diffusion process and the controlled linear 
output with a~quadratic quality criterion, an intermediate result is resumed: for approximate calculation of 
the optimal solution, an alternative to classical numerical integration method based on computer 
simulation is proposed. The method allows applying statistical estimation to determine the coefficients 
$\beta_t(y)$ and~$\gamma_t(y)$ of the previously obtained Bellman function $V_t(y,z)=\alpha_t 
z^2+\beta_t(y)z+\gamma_t(y)$, determining the optimal solution in the original problem of optimal 
stochastic control. The method is implemented on the basis of the properties of linear parabolic partial 
differential equations describing $\beta_t(y)$ and~$\gamma_t(y)$~--- their equivalent description in the form of 
stochastic differential equations and a theoretical-probability representation of the solution, known as 
A.\,N.~Kolmogorov equation, or an equivalent integral form known as the Feynman--Katz formula. 
Stochastic equations, relations for optimal control and for auxiliary parameters are combined into one 
differential system, for which an algorithm for simulating a~solution is stated. The algorithm provides the 
necessary samples for statistical estimation of the coefficients~$\beta_t(y)$
and~$\gamma_t(y)$. The previously 
performed numerical experiment is supplemented by calculations presented by an alternative method and 
a~comparative analysis of the results.}

\KWE{stochastic differential equation; optimal control; Bellman function; linear 
differential equations of parabolic type; Kolmogorov equation; Feynman--Katz formula; 
computer simulations; Monte-Carlo method}




\DOI{10.14357/19922264200104} 

\vspace*{-20pt}

\Ack
\noindent
This work was partially supported by the Russian Foundation for Basic Research 
(grant 19-07-00187-A).


%\vspace*{6pt}

  \begin{multicols}{2}

\renewcommand{\bibname}{\protect\rmfamily References}
%\renewcommand{\bibname}{\large\protect\rm References}

{\small\frenchspacing
 {%\baselineskip=10.8pt
 \addcontentsline{toc}{section}{References}
 \begin{thebibliography}{9}
\bibitem{1-bos-1}
\Aue{Bosov, A.\,V., and A.\,I.~Stefanovich.} 2018. Upravlenie vykhodom 
stokhasticheskoy differentsial'noy sistemy po kvadratichnomu kriteriyu. 
I.~Optimal'noe reshenie metodom dinamicheskogo programmirovaniya 
[Stochastic differential system output control by the quadratic criterion. 
I.~Dynamic programming optimal solution]. \textit{Informatika i~ee 
Primeneniya~--- Inform. Appl.} 12(3):99--106.
\bibitem{2-bos-1}
\Aue{Bosov, A.\,V., and A.\,I.~Stefanovich.} 2019. Upravlenie vykhodom 
stokhasticheskoy differentsial'noy sistemy po kvadratichnomu kriteriyu. 
II.~Chislennoe reshenie uravneniy dinamicheskogo programmirovaniya 
[Stochastic differential system output control by the quadratic criterion. 
II.~Dynamic programming equations numerical solution]. \textit{Informatika 
i~ee Primeneniya~--- Inform. Appl.} 13(1):9--15.
\bibitem{3-bos-1}
\Aue{Bosov, A.\,V., and A.\,I.~Stefanovich.} 2019. Upravlenie vykhodom 
stokhasticheskoy differentsial'noy sistemy po kvadratichnomu kriteriyu. 
III.~Analiz svoystv optimal'nogo upravleniya [Stochastic differential system 
output control by the quadratic criterion. III.~Optimal control properties 
analysis]. \textit{Informatika i~ee Primeneniya~--- Inform. Appl.}
13(3):41--49.
\bibitem{4-bos-1}
\Aue{Gihman, I.I., and A.\,V.~Skorohod.} 2012. \textit{The theory of stochastic 
processes}. New York, NY: Springer-Verlag. Vol.~III. 388~p.
\bibitem{5-bos-1}
\Aue{{\ptb{\O}}ksendal, B.} 2003. \textit{Stochastic differential equations. An 
introduction with applications}. New York, NY: Springer-Verlag. 379~p.
\bibitem{6-bos-1}
\Aue{Kloden, P.\,E., and E.~Platen.} 1992. \textit{Numerical solution of 
stochastic differential equations}. Berlin--Heidelberg: Springer-Verlag. 636~p.
\bibitem{7-bos-1}
\Aue{Shiryaev, A.\,N.} 1996. \textit{Probability}. New York, NY:  
Springer-Verlag. 624~p.
\bibitem{8-bos-1}
\Aue{Bohacek, S., and B.~Rozovskii.} 2004. A~diffusion model of roundtrip 
time. \textit{Comput. Stat. Data An.} 45(1):25--50.
\end{thebibliography}

 }
 }

\end{multicols}

\vspace*{-9pt}

\hfill{\small\textit{Received August 28, 2019}}

%\pagebreak

\vspace*{-24pt}

\Contr

\vspace*{-4pt}

\noindent
\textbf{Bosov Alexey V.} (b.\ 1969)~--- Doctor of Science in technology, 
principal scientist, Institute of Informatics Problems, Federal Research Center 
``Computer Science and Control'' of the Russian Academy of Sciences,  
44-2~Vavilov Str., Moscow 119333, Russian Federation; 
\mbox{AVBosov@ipiran.ru}

%\vspace*{3pt}

\noindent
\textbf{Stefanovich Alexey I.} (b.\ 1983)~--- principal specialist, Institute of 
Informatics Problems, Federal Research Center ``Computer Science and Control'' 
of the Russian Academy of Sciences, 44-2~Vavilov Str., Moscow 119333, Russian 
Federation; \mbox{AStefanovich@frccsc.ru}
\label{end\stat}

\renewcommand{\bibname}{\protect\rm Литература} 
      