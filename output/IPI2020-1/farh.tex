\def\stat{farh}

\def\tit{МЕТОД ЗАДАНИЯ КОНЕЧНЫХ НЕКОММУТАТИВНЫХ АССОЦИАТИВНЫХ АЛГЕБР 
ПРОИЗВОЛЬНОЙ ЧЕТНОЙ РАЗМЕРНОСТИ ДЛЯ~ПОСТРОЕНИЯ ПОСТКВАНТОВЫХ 
КРИПТОСХЕМ}

\def\titkol{Метод задания КНАА %конечных некоммутативных ассоциативных алгебр 
произвольной четной размерности для~построения постквантовых 
криптосхем}

\def\aut{А.\,А.~Костина$^1$, А.\,Ю.~Мирин$^2$, Д.\,Н.~Молдовян$^3$, 
Р.\,Ш.~Фахрутдинов$^4$}

\def\autkol{А.\,А.~Костина, А.\,Ю.~Мирин, Д.\,Н.~Молдовян, 
Р.\,Ш.~Фахрутдинов}

\titel{\tit}{\aut}{\autkol}{\titkol}

\index{Костина А.\,А.}
\index{Мирин А.\,Ю.}
\index{Молдовян Д.\,Н.} 
\index{Фахрутдинов Р.\,Ш.}
\index{Kostina A.\,A.}
\index{Mirin A.\,Yu.}
\index{Moldovyan D.\,N.}
\index{Fahrutdinov R.\,Sh.}


%{\renewcommand{\thefootnote}{\fnsymbol{footnote}} \footnotetext[1]
%{Работа выполнена при финансовой поддержке Российского научного фонда (проект 18-11-00155).}}


\renewcommand{\thefootnote}{\arabic{footnote}}
\footnotetext[1]{Санкт-Петербургский институт информатики 
и~автоматизации Российской академии наук, \mbox{anna-kostina1805@mail.ru}}
\footnotetext[2]{Санкт-Петербургский институт информатики и~автоматизации Российской 
      академии наук, \mbox{mirin@cobra.ru}}
\footnotetext[3]{Санкт-Петербургский институт информатики и~автоматизации Российской академии наук, 
\mbox{mdn.spectr@mail.ru}}
\footnotetext[4]{Санкт-Петербургский институт информатики и~автоматизации Российской 
академии наук, \mbox{fahr@cobra.ru}}

%\vspace*{-12pt}
      
     
  
  \Abst{Представлен новый унифицированный метод задания конечных некоммутативных 
ассоциативных алгебр (КНАА) произвольной четной размерности~$m$ и~описаны исследуемые 
свойства алгебр для случаев $m \hm= 4$ и~$6$ при задании алгебр над конечным 
простым полем $GF(p)$ с~большим размером простого числа~$p$. Получены формулы, 
описывающие множество~$p^2$ ($p^4$) глобальных левосторонних единиц, содержащихся 
в~4-мер\-ной (6-мер\-ной) алгебре. В~исследованных алгебрах имеет место только локальная 
обратимость. Для каждой из алгебр выведены формулы для вычисления единственного 
локального двустороннего элемента, связанного с~фиксированным локально обратимым 
вектором. Новая форма скрытой задачи дискретного логарифмирования  (СЗДЛ)
предложена 
в~качестве постквантового криптографического примитива и~использована для разработки 
постквантовой схемы электронной цифровой подписи (ЭЦП).}
  
  \KW{конечная некоммутативная алгебра; ассоциативная алгебра; вычислительно трудная 
задача; дискретный логарифм; цифровая подпись; постквантовая криптография}

\DOI{10.14357/19922264200113} 
  
%\vspace*{-3pt}


\vskip 10pt plus 9pt minus 6pt

\thispagestyle{headings}

\begin{multicols}{2}

\label{st\stat}

\section{Введение}

  Актуальным текущим вызовом в~области крип\-то\-гра\-фии стала разработка 
криптосхем с~открытым ключом, пригодных для принятия на их основе 
постквантовых криптографических стандартов~[1, 2] взамен текущих 
стандартов, основанных на вычислительной трудности задачи дискретного 
логарифмирования (ЗДЛ), которая решается на пока еще гипотетическом 
квантовом компьютере за полиномиальное время~[3]. Перспективным 
на\-прав\-ле\-ни\-ем решения указанной проб\-ле\-мы пред\-став\-ля\-ет\-ся использование 
вычислительной трудности СЗДЛ, за\-да\-ва\-емой в~КНАА~[4]. Для реализации 
потенциала СЗДЛ как постквантового криптографического примитива важное 
значение имеют задачи поиска и~исследования новых КНАА как носителей 
СЗДЛ и~новых форм последней, для которых СЗДЛ не будет сводиться к~ЗДЛ 
в~конечном поле~[5]. 
  
  В настоящей работе предлагается общий способ задания КНАА 
  произвольной четной 
размерности $m\hm> 2$ и~исследуются свойства 4- и~6-мер\-ных 
КНАА, построенных в~соответствии с~предложенным общим способом. 
Характерным свойством рас\-смот\-рен\-ных алгебр является наличие большого 
множества глобальных левосторонних единиц, и~для реализации на их основе 
постквантовых криптосхем предлагается новая форма СЗДЛ, использующая 
указанную особенность примененных в~качестве ее алгебраического носителя 
КНАА.
  
\section{Задача дискретного логарифмирования в~скрытой~группе}

  Известно автоморфное отображение некоммутативной группы~$\Gamma$, 
задаваемое следующей формулой:
  $$
  \varphi_V(G) =V^{-1}\circ G\circ V\,,
  $$
  где $V$~--- фиксированный элемент группы~$\Gamma$; $G$~--- элемент, 
пробегающий все значения в~группе~$\Gamma$. Элементы~$G$ и~$Y\hm=V^{-
1}\circ G\circ V$ называются элементами, сопряженными через элемент~$V$. 
Для фиксированного значения~$G$ операция автоморфного отоб\-ра\-же\-ния 
и~операция возведения в~степень являются перестановочными (взаимно 
коммутативными), т.\,е.\ имеет место соотношение: 
  $$
  \left( V^{-1}\circ G\circ V\right)^x=V^{-1}\circ (G^x)\circ V\,.
  $$
  %
  Это свойство может быть использовано для формулирования следующей 
вычислительно трудной задачи, пригодной для использования в~качестве 
примитива криптосистем с~открытым ключом. 
  
  Пусть задан некоторый элемент~$G$. Из некоторой коммутативной 
подгруппы~$\Gamma_{\mathrm{комм}}\subset \Gamma$ вы\-бираются элемент~$X$ 
и~произвольное число~$x$ и~вы\-чис\-ля\-ет\-ся элемент $Y\hm=X^{-1}\circ (G^x)\circ 
X$. По \mbox{заданным}~$Y$ и~$G$ требуется вычислить~$X$ и~$x$. Поскольку 
вычислить эти два неизвестных элемента по отдельности нельзя, то эта задача 
в~общем случае не сводится к~задаче дискретного логарифмирования 
в~циклической подгруппе. Нахождение неизвестных~$X$ и~$x$ по 
значениям~$Y$ и~$G$ представляет собой самостоятельную трудную 
вычислительную задачу, отличную от задачи дискретного логарифмирования. 
При известном значении~$X$ можно вычислить $Y^\prime \hm=X\circ Y\circ 
X^{-1}$ или $G^\prime\hm=X^{-1} \circ G\circ X$, после чего число~$x$ можно 
найти из уравнения $Y^\prime\hm=G^x$ или из уравнения  $Y\hm=G^{\prime x}$ 
соответственно, т.\,е.\ решая задачу дискретного логарифмирования. Однако 
значение~$X$ остается неизвестным, поэтому ЗДЛ в~явном виде не стоит. 
Криптосхемы на основе СЗДЛ, заданной в~этой форме, описаны 
в~работе~\cite{4-f}.
  
\section{Конечные некоммутативные ассоциативные алгебры}

  \subsection{Алгебры как~векторные пространства с~дополнительной 
операцией умножения векторов}
  
  Рассмотрим $m$-мер\-ное векторное пространство, элементами которого 
выступают всевозможные векторы вида
  \begin{multline*}
  A= \left(a_0, a_1, \ldots , a_1\right)= {}\\
  {}=\left(a_0\mathbf{e}_0\hm+ 
a_1\mathbf{e}_1+\cdots + a_{m-1}\mathbf{e}_{m-1}\right),
\end{multline*}
 где $a_i\hm\in GF(p)$, 
$p$~--- простое число; $\mathbf{e}_i$~--- формальные базисные векторы. 
Дополнительно к~стандартным операциям в~векторном пространстве~--- 
операции сложения векторов и~операции умножения вектора на скаляр~--- 
определим операцию умножения ($\circ$) векторов $A\hm= 
\sum\nolimits_{i=0}^{m-1} a_i\mathbf{e}_i$ и~$B\hm= 
  \sum\nolimits_{j=0}^{m-1} b_j\mathbf{e}_j$ в~соответствии со следующей 
формулой:  
  \begin{equation}
  A\circ B= \sum\limits_{i=0}^{m-1} \sum\limits_{j=0}^{m-1} a_i b_j 
\mathbf{e}_i\circ \mathbf{e}_j\,,
  \label{e1-f}
  \end{equation}
где каждое из всевозможных произведений пар базисных векторов заменяется 
на однокомпонентный вектор в~соответствии с~некоторым правилом, 
задаваемым в~виде таблицы умножения базисных векторов (ТУБВ). Векторное 
пространство с~определенной таким образом операцией умножения векторов 
называется $m$-мер\-ной алгеброй. Для построения криптосхем на основе 
СЗДЛ интерес представляют КНАА.

\subsection{Общий способ задания конечных некоммутативных ассоциативных
алгебр произвольной четной 
размерности $m\hm\geq4$}

  В качестве общего способа задания конечной ассоциативной алгебры четной 
размерности $m\hm>1$ предлагается задать значение произведения 
$\mathbf{e}_i\circ \mathbf{e}_j$ в~формуле~(1) в~соответствии со следующим 
выражением:
  \begin{equation}
  \mathbf{e}_i\circ\mathbf{e}_j=\begin{cases}
  \mathbf{e}_j\,, &\ \mbox{если } i\,\mathrm{mod}\,2=0\,;\\
  \mathbf{e}_{m-1-j}\,, &\ \mbox{если } i\,\mathrm{mod}\,2=1\,.
  \end{cases}
  \label{e2-f}
  \end{equation}
  
  Докажем, что правило~(2) задает ассоциативное умножение векторов. 
Рассмотрим произведение векторов~$A$, $B$ и~$C\hm= 
\sum\nolimits_{k=0}^{m-1} c_k\mathbf{e}_k$, осуществляемое в~соответствии 
со следующими двумя вариантами: 
  \begin{align}
(A\circ B)\circ C &=\notag{}\\
  &\hspace*{-15mm}{}= \left( \sum\limits_{i=0}^{m-1} \sum\limits_{j=0}^{m-1} a_i 
b_j \mathbf{e}_i\circ \mathbf{e}_j\right) \circ \sum\limits_{k=0}^{m-1} 
c_k\mathbf{e}_k ={}\notag\\
&{}= \sum\limits_{i=0}^{m-1} \sum\limits_{j=0}^{m-1} 
\sum\limits_{k=0}^{m-1} a_i b_j c_k \left( \mathbf{e}_i\circ \mathbf{e}_j 
\right)\circ \mathbf{e}_k\,;
  \label{e3.1-f}
  \\
 A\circ( B\circ C) &={}\notag\\
  &\hspace*{-15mm}{}= \left( \sum\limits^{m-1}_{i=0} a_i \mathbf{e}_i\right) \circ 
\left( \sum\limits^{m-1}_{j=0} \sum\limits^{m-1}_{k=0} b_j c_k \mathbf{e}_i\circ 
\mathbf{e}_j\right) ={}\notag\\
&{}=\sum\limits_{i=0}^{m-1} \sum\limits_{j=0}^{m-1} 
\sum\limits^{m-1}_{k=0} a_i b_j c_k \mathbf{e}_i\circ \left( \mathbf{e}_j\circ 
\mathbf{e}_k \right)\,.
  \label{e3.2-f}
  \end{align}
  
  Равенство правых частей выражений~(\ref{e3.1-f}) и~(\ref{e3.2-f}) имеет 
место, если равенство
  \begin{equation}
  \left(\mathbf{e}_i\circ\mathbf{e}_j\right)\circ\mathbf{e}_k=\mathbf{e}_i\circ\left
( \mathbf{e}_j\circ\mathbf{e}_k\right)
  \label{e4-f}
  \end{equation}
справедливо для всех возможных троек значений $(i, j, k)$. Справедливость 
равенства~(\ref{e4-f}) можно легко показать, рассматривая следующие четыре 
случая:
\begin{description}
\item[\,]  \textbf{Случай~1}: $i$ и~$j$~--- четные значения: 
  \begin{multline*}
  \left\{ \begin{matrix}
  \left( \mathbf{e}_i\circ \mathbf{e}_j\right) \circ\mathbf{e}_k = \mathbf{e}_j 
\circ \mathbf{e}_k =\mathbf{e}_k\\
  \mathbf{e}_i \circ \left(\mathbf{e}_j\circ \mathbf{e}_k\right) 
=\mathbf{e}_i\circ \mathbf{e}_k =\mathbf{e}_k
  \end{matrix}
  \right\} \Rightarrow {}\\
  {}\Rightarrow \left(\mathbf{e}_i\circ \mathbf{e}_j\right)\circ 
\mathbf{e}_k=\mathbf{e}_i\circ\left(\mathbf{e}_j\circ \mathbf{e}_k\right)\,.
\end{multline*}
\item[\,]  
  \textbf{Случай~2}: $i$~--- четное значение; $j$~--- нечетное значение: 
\begin{multline*}
  \left\{
  \begin{matrix}
  \left(\mathbf{e}_i\circ \mathbf{e}_j\right)  \circ\mathbf{e}_k 
=\mathbf{e}_j\circ\mathbf{e}_k =\mathbf{e}_{m-1-k}\\
  \mathbf{e}_i \circ\left(\mathbf{e}_j\circ \mathbf{e}_k\right) = 
\mathbf{e}_i\circ \mathbf{e}_{m-1-k} =\mathbf{e}_{m-1-k}
  \end{matrix}
  \right\} \Rightarrow{}\\
  {}\Rightarrow \left(\mathbf{e}_i\circ \mathbf{e}_j\right)\circ 
\mathbf{e}_k=\mathbf{e}_i\circ\left(\mathbf{e}_j\circ \mathbf{e}_k\right)\,.
\end{multline*}
  \item[\,]
  \textbf{Случай~3}: $i$~--- нечетное значение; $j$~--- четное значение: 
  \begin{multline*}
  \left\{ \begin{matrix}
  \left(\mathbf{e}_i\circ \mathbf{e}_j\right)\circ \mathbf{e}_k=\mathbf{e}_{m-
1-j}\circ \mathbf{e}_k =\mathbf{e}_{m-1-k}\\
  \mathbf{e}_i\circ \left(\mathbf{e}_j\circ \mathbf{e}_k\right) 
=\mathbf{e}_i\circ \mathbf{e}_k =\mathbf{e}_{m-1-k}
  \end{matrix}
  \right\} \Rightarrow {}\\
  {}\Rightarrow\left(\mathbf{e}_i\circ \mathbf{e}_j\right) \circ 
\mathbf{e}_k = \mathbf{e}_i\circ \left(\mathbf{e}_j\circ \mathbf{e}_k\right)\,.
  \end{multline*}
  \item[\,]
  \textbf{Случай~4}: $i$ и~$j$~--- нечетные значения: 
  \begin{multline*}
  \left\{ 
  \begin{matrix}
  \left(\mathbf{e}_i\circ \mathbf{e}_j\right) \circ \mathbf{e}_k=\mathbf{e}_{m-
1-j}\circ \mathbf{e}_k =\mathbf{e}_k\\
  \mathbf{e}_i\circ\left(\mathbf{e}_j\circ \mathbf{e}_k\right) =
  \mathbf{e}_i\circ 
\mathbf{e}_{m-1-k} ={}\\
\hspace*{22mm}{}= \mathbf{e}_{m-1-(m-1-k)}= \mathbf{e}_k
  \end{matrix}
  \right\} \Rightarrow{}\\
  {}\Rightarrow \left(\mathbf{e}_i\circ \mathbf{e}_j\right)\circ 
\mathbf{e}_k = \mathbf{e}_i\circ\left(\mathbf{e}_j\circ \mathbf{e}_k\right)\,.
 \end{multline*}
 \end{description}
  
  Таким образом, операция умножения векторов, задаваемая правилом~(2), 
ассоциативна, а при $m\hm\geq4$ также и~некоммутативна. 
  
  \subsection{Четырехмерная алгебра}
  
  Рассмотрим случай 4-мер\-ной КНАА, для которой найдено распределение 
структурного коэффициента $\mu\hm\in GF(p)$, представленное в~табл.~1.
  
  
  
  Нахождение левосторонних единиц 4-мер\-ной алгебры, задаваемой табл.~1, 
связано с~решением следующего векторного уравнения:
  \begin{equation}
  X\circ A=A\,,
  \label{e5-f}
  \end{equation}
где $A=(a_0, a_1, a_2, a_3)$~--- некоторый заданный вектор, для которого 
требуется найти левостороннюю
единицу; $X\hm= (x_0, x_1, x_2, x_3)$~--- 
неизвестный век-\linebreak\vspace*{-12pt}

\vspace*{9pt} %tabl1

\begin{center}
\noindent
\parbox{50mm}{{{\tablename~1}\ \ \small{Предлагаемая ТУБВ для случая $m = 4$}}}
%\vspace*{2ex}

\vspace*{8pt}

\tabcolsep=8pt
{\small
\begin{tabular}{|c|c|c|c|c|}
  \hline
\multicolumn{1}{|c|}{$\circ$}&\multicolumn{1}{c|}{$\mathbf{e}_0$}&$\mathbf{e}_1$&
$\mathbf{e}_2$&$\mathbf{e}_3$\\
\hline
$\mathbf{e}_0$&$\mu\mathbf{e}_0$&$\mu\mathbf{e}_1$&$\mu\mathbf{e}_2$&$\mu\mathbf{e}_3$\\
\hline
$\mathbf{e}_1$&$\mathbf{e}_3$&$\mathbf{e}_2$&$\mathbf{e}_1$&$\mathbf{e}_0$\\
\hline
$\mathbf{e}_2$&$\mathbf{e}_0$&$\mathbf{e}_1$&$\mathbf{e}_2$&$\mathbf{e}_3$\\
\hline
$\mathbf{e}_3$&$\mu\mathbf{e}_3$&$\mu\mathbf{e}_2$&$\mu\mathbf{e}_1$&$\mu\mathbf{e}_0$\\
\hline
\end{tabular}
}
\vspace*{2pt}
\end{center}

%\end{table*}





\noindent
тор. С~учетом табл.~1 данное векторное уравнение   сводится 
к~решению следующей системы линейных уравнений с~четырьмя 
неизвестными:

\vspace*{1pt}

\noindent
\begin{equation}
\left.
\begin{array}{l}
\mu x_0 a_0 +x_1a_3 +x_2a_0+\mu x_3a_3=a_0\,;\\[3pt]
\mu x_0 a_1+x_1a_2+x_2a_1+\mu x_3a_2=a_1\,;\\[3pt]
\mu x_0a_2+x_1a_1+x_2a_2+\mu x_3a_1=a_2\,;\\[3pt]
\mu x_0 a_3+x_1a_0 +x_2 a_3 +\mu x_3a_0=a_3\,.
\end{array}
\right\}
\label{e6-f}
\end{equation}

\vspace*{-1pt}

  Эта система распадается на следующие две независимые системы из двух 
уравнений:

\vspace*{1pt}

\noindent
  \begin{equation}
  \left.
  \begin{array}{l}
  \left\{ 
  \begin{array}{l}
  (\mu x_0+x_2)a_0+(x_1+\mu x_3)a_3=a_0\,;\\[3pt]
  (x_1+\mu x_3)a_0+(\mu x_0+x_2)a_3=a_3\,;
  \end{array}
  \right.\\[12pt]
  \left\{ \begin{array}{l}
  (\mu x_0+x_2)a_1+(x_1+\mu x_3)a_2=a_1\,;\\[3pt]
  (x_1+\mu x_3)a_1+(\mu x_0+x_2)a_2=a_2\,.
  \end{array}
  \right.
  \end{array}
  \right\}
  \label{e7-f}
  \end{equation}
  
  \vspace*{-1pt}
  
  Выполнив в~(\ref{e7-f}) замену переменных по формулам $z_1\hm= \mu 
x_0\hm+x_2$ и~$z_1\hm= z_2\hm+\mu x_3$, легко показать, что решения 
системы~(\ref{e6-f}) совпадают с~решениями следующей системы из двух 
уравнений с~четырьмя неизвестными~$x_0$, $x_1$, $x_2$ и~$x_3$:

\vspace*{1pt}

\noindent
  \begin{equation*}
  \left. 
  \begin{array}{l}
  \mu x_0+x_2=1\,;\\[3pt]
  x_1+\mu x_3=0\,.
  \end{array}
  \right\}
  %\label{e8-f}
  \end{equation*}
  
  \vspace*{-1pt}
  
  Поскольку решения системы~(\ref{e6-f}) не зависят от\linebreak координат 
вектора~$A$, это означает, что найденные решения описывают глобальные 
левосторонние единицы, т.\,е.\ левосторонние единицы, действующие на все 
элементы рассматриваемой \mbox{4-мер\-ной} КНАА. Множество всех~$p^2$ 
глобальных левосторонних единиц описывается следующей формулой:
  \begin{equation*}
  L=\left( l_0, l_1, l_2, l_3\right) =\left( x_0, x_1, 1-\mu x_0, -\mu^{-1}x_1\right)\,,
  %\label{e9-f}
  \end{equation*}
где $x_0, x_1=0, 1, \ldots , p-1$.
  
  Для нахождения правосторонних единиц вектора $A\hm= (a_0, a_1, a_2, a_3)$ 
рассмотрим следующее векторное уравнение:

\vspace*{1pt}

\noindent
  \begin{equation}
  A\circ X=A\,,
  \label{e10-f}
  \end{equation}
  
  \vspace*{-1pt}
  
  \noindent
которое сводится к~системе линейных уравнений вида
\begin{equation}
\left.
\begin{matrix}
\mu a_0 x_0+a_1x_3+a_2x_0+\mu a_3x_3=a_0\,;\\[2pt]
\mu a_0x_1+a_1x_2+a_2x_1+\mu a_3x_2 =a_1\,;\\[2pt]
\mu a_0 x_2+a_1x_1+a_2x_2+\mu a_3 x_1=a_2\,;\\[2pt]
\mu a_0x_3+a_1x_0+a_2x_3+\mu a_3x_0=a_3\,.
\end{matrix}
\right\}
\label{e11-f}
\end{equation}
  
  Если координаты вектора~$A$ удовлетворяют неравенству 
    \begin{equation*}
  \Delta = \left( \mu a_0+a_2\right)^2 -\left( a_1+\mu a_3\right)^2\not= 0\,,
 % \label{e12-f}
  \end{equation*}
то система~(\ref{e11-f}) имеет единственное решение:

\noindent
\begin{align*}
x_0&=r_0=\fr{a_0(\mu a_0+a_2)-a_3(a_1+\mu a_3)}{\Delta}\,;\\ 
x_1&=r_1=\fr{\mu(a_0a_1-a_2a_3)}{\Delta}\,;\\
x_2&=r_2=\fr{a_2(\mu a_0+a_2)-a_1(a_1+\mu a_3)}{\Delta}\,;\\ 
x_3&=r_3=\fr{a_2 a_3-a_0 a_1}{\Delta}\,,
\end{align*}
которое определяет существование единственной локальной правосторонней 
единицы $R_A\hm= (r_0, r_1, r_2, r_3)$, соответствующей вектору~$A$. 
Вектор~$R_A$ действует как правая единица на \mbox{некотором} подмножестве 
элементов рассматриваемой ал\-геб\-ры, поэтому она называется локальной. 

  \subsection{Шестимерная алгебра}
  
  Для случая 6-мер\-ной КНАА найдены распределения независимых 
структурных коэффициентов~$\mu, \lambda \hm\in GF(p)$, представленные 
в~табл.~2.

\vspace*{6pt} %tabl2

\begin{center}
\noindent
{{\tablename~2}\ \ \small{Предлагаемая ТУБВ для случая $m = 6$}}
%\vspace*{2ex}

\vspace*{6pt}

\tabcolsep=8pt
{\small
 \begin{tabular}{|c|c|c|c|c|c|c|}
  \hline
$\circ$&$\mathbf{e}_0$&$\mathbf{e}_1$&$\mathbf{e}_2$&$\mathbf{e}_3$&
$\mathbf{e}_4$&$\mathbf{e}_5$\\
\hline
$\mathbf{e}_0$&$\mu \mathbf{e}_0$&$\mu \mathbf{e}_1$&
$\mu \mathbf{e}_2$&$\mu \mathbf{e}_3$&$\mu \mathbf{e}_4$&$\mu \mathbf{e}_5$\\
\hline
$\mathbf{e}_1$&$\mathbf{e}_5$&$\mathbf{e}_4$&$\mathbf{e}_3$&$\mathbf{e}_2$&
$\mathbf{e}_1$&$\mathbf{e}_0$\\
\hline
$\mathbf{e}_2$&$\lambda \mathbf{e}_0$&$\lambda \mathbf{e}_1$&$\lambda 
\mathbf{e}_2$&$\lambda \mathbf{e}_3$&$\lambda\mathbf{e}_4$&$\lambda 
\mathbf{e}_5$\\
\hline
$\mathbf{e}_3$&$\lambda\mathbf{e}_5$&$\lambda\mathbf{e}_4$&$\lambda\mathbf{e}_3
$&$\lambda\mathbf{e}_2$&$\lambda\mathbf{e}_1$&$\lambda \mathbf{e}_0$\\
\hline
$\mathbf{e}_4$&$\mathbf{e}_0$&$\mathbf{e}_1$&$\mathbf{e}_2$&$\mathbf{e}_3$&
$\mathbf{e}_4$&$\mathbf{e}_5$\\
\hline
$\mathbf{e}_5$&$\mu \mathbf{e}_5$&$\mu\mathbf{e}_4$&$\mu 
\mathbf{e}_3$&$\mu\mathbf{e}_2$&$\mu \mathbf{e}_1$&$\mu \mathbf{e}_0$\\
\hline
\end{tabular}
}
%\vspace*{2pt}
\end{center}

\vspace*{3pt}

%\end{table*}
  

  
  
  Нахождение левосторонних единиц 6-мер\-ной КНАА, задаваемой табл.~2, 
по векторному уравнению~(\ref{e5-f}), в~котором $A\hm= (a_0, a_1, a_2, a_3, 
a_4, a_5)$ и~$X\hm= (x_0, x_1, x_2, x_3, x_4, x_5)$, приводит к~решению 
следующей системы из шести линейных уравнений с~неизвестными 
координатами вектора~$X$:
  \begin{equation}
  \left.
  \begin{array}{rl}
  \mu x_0a_0+x_1a_5+\lambda x_2a_0+\lambda x_3 a_5+{}&\\[1pt]
  &\hspace*{-20mm}{}+x_4a_0+\mu x_5a_5=a_0\,;\\[3pt]
  \mu x_0a_1+x_1 a_4+\lambda x_2a_1+\lambda x_3 a_4+{}&\\[1pt]
&  \hspace*{-20mm}{}+x_4a_1+\mu x_5a_4=a_1\,;\\[3pt]
  \mu x_0a_2+x_1 a_3+\lambda x_2a_2+\lambda x_3 a_3+{}&\\[1pt]
&\hspace*{-20mm}{}+  x_4a_2+\mu x_5a_3=a_2\,;\\[3pt]
  \mu x_0a_3+x_1 a_2+\lambda x_2a_3+\lambda x_3 a_2+{}&\\[1pt]
  &\hspace*{-20mm}{}+x_4a_3+\mu  x_5a_2=a_3\,;\\[3pt]
  \mu x_0a_4+x_1 a_1+\lambda x_2a_4+\lambda x_3 a_1+{}&\\[1pt]
  &\hspace*{-20mm}{}+x_4a_4+\mu x_5a_1=a_4\,;\\[3pt]
  \mu x_0a_5+x_1 a_0+\lambda x_2a_5+\lambda x_3 a_0+{}&\\[1pt]
  &\hspace*{-20mm}{}+x_4a_5+\mu  x_5a_0=a_5\,.
  \end{array}
  \right\}
  \label{e13-f}
  \end{equation}
  
  

  

  Выделим в~этой системе следующие три системы из двух уравнений:
  
  \noindent
  \begin{equation}
  \left.
  \begin{array}{l}
    \left\{
\begin{array}{l}
    (\mu x_0+\lambda x_2+x_4)a_0+{}\\[1pt]
    \hspace*{15mm}{}+(x_1+\lambda x_3+\mu x_5)a_5=a_0\,;\\[1pt]
  (x_1+\lambda x_3+\mu x_5)a_0 +{}\\[1pt]
   \hspace*{15mm}{}+(\mu x_0+\lambda x_2+x_4)a_5=a_5\,;
  \end{array}
  \right.\\[9pt]
  \left\{
  \begin{array}{l}
  (\mu x_0+\lambda x_2+x_4)a_1+{}\\[1pt]
  \hspace*{15mm}{}+(x_1+\lambda x_3+\mu x_5)a_4=a_1\,;\\[1pt]
 (x_1+\lambda x_3 +\mu x_5)a_1+{}\\[1pt]
  \hspace*{15mm}{}+(\mu x_0+\lambda x_2 +x_4)a_4=a_4\,;
  \end{array}
  \right.\\[9pt]
  \left\{ 
  \begin{array}{l}
(\mu x_0+\lambda x_2 +x_4)a_2+{}\\[1pt]
 \hspace*{15mm} {}+(x_1+\lambda x_3+\mu x_5)a_3=a_2\,;\\[1pt]
  (x_1+\lambda x_3+\mu x_5)a_2+{}\\[1pt]
   \hspace*{15mm}{}+(\mu x_0+\lambda x_2+x_4)a_3=a_3\,.
    \end{array}
  \right.
\end{array}
  \right\}
  \label{e14-f}
  \end{equation}
  
  Легко видеть, что решение системы~(\ref{e14-f}) можно найти, выполнив  
замену переменных по формулам $z_1\hm= \mu x_0\hm+ \lambda x_2\hm+ x_4$ 
и~$z_2\hm= x_1\hm+\lambda x_3\hm+\mu x_5$. После такой замены 
переменных каждая из трех подсистем системы~(\ref{e14-f}) включает два 
уравнения с~одинаковыми двумя неизвестными~$z_1$ и~$z_2$ и~приобретает 
вид:
  \begin{equation}
  \left.
  \begin{array}{l}
  \left\{
  \begin{array}{c}
  z_1a_0+z_2a_5=a_0\,;\\[3pt]
  z_1a_5+z_2a_0=a_5\,;
  \end{array}
  \right.\\[12pt]
  \left\{
  \begin{array}{c}
  z_1a_1+z_2a_4=a_1\,;\\[3pt]
  z_1a_4+z_2a_2=a_4\,;
  \end{array}
  \right.\\[12pt]
  \left\{
  \begin{array}{c}
  z_1a_2+z_2a_3=a_2\,;\\[3pt]
  z_1a_3+z_2a_2=a_3\,.
  \end{array}
  \right.
  \end{array}
  \right\}
  \label{e15-f}
  \end{equation}
                  
  
  Система~(\ref{e15-f}) имеет единственное решение в~виде пары значений 
$z_1\hm=1$ и~$z_2\hm=0$ для всех возможных значений вектора~$A$, кроме 
случая одновременного выполнения условий $a_0\hm=a_5$, $a_1\hm= a_4$ 
и~$a_2\hm= a_3$. В~последнем случае имеется множество дополнительных 
решений в~виде пар значений $z_1\hm\in GF(p)$ и~$z_2\hm= 1\hm -z_1$. Этот 
особый случай выпадает из множества значений векторов, используемых при 
построении криптосхем на основе рассматриваемой 6-мер\-ной конечной 
алгебры.
  
  Выполнение обратной замены переменных показывает, что исходная 
система~(\ref{e13-f}) имеет решения, совпадающие с~решениями следующей 
системы из двух линейных уравнений с~шестью неизвестными~$x_0$, $x_1$, 
$x_2$, $x_3$, $x_4$ и~$x_5$:
  \begin{equation}
  \left.
  \begin{matrix}
  \mu x_0+\lambda x_2+x_4=1\,;\\
  x_1+\lambda x_3+\mu x_5=0\,.
  \end{matrix}
  \right\}
  \label{e16-f}
  \end{equation}
  
  Решения системы~(\ref{e16-f}) не зависят от координат вектора~$A$, т.\,е.\ 
они описывают следующее множество~$p^4$ глобальных левосторонних 
единиц:
  \begin{multline}
  L=\left( l_0, l_1, l_2, l_3, l_4, l_5\right)={}\\
  \!\!\!=\left( x_0, x_1, x_2, x_3, 1\!-\!\mu x_0\!-\!
\lambda x_2, -\mu^{-1}(x_1\!+\!\lambda x_3)\right),\!\!\!\!
  \label{e17-f}
  \end{multline}
где $x_0, x_1, x_2, x_3 \hm=0, 1, \ldots ,  p-1$.
  
  Правосторонние единицы, соответствующие вектору $A\hm= (a_0, a_1, a_2, 
a_3, a_4, a_5)$, удовлетворяют векторному уравнению~(\ref{e10-f}), 
рассмотрение которого приводит к~системе уравнений, представимой
в~виде трех независимых систем из двух линейных уравнений с~двумя 
неизвестными:
\begin{equation}
\left.
\begin{array}{l}
\left\{
\begin{array}{l}
k_1x_0+k_2x_5=a_0\,;\\[3pt]
k_2x_0+k_1x_5=a_5\,;
\end{array}
\right.\\[12pt]
\left\{
\begin{array}{l}
k_1x_1+k_2x_4=a_1\,;\\[3pt]
k_2x_1+k_1x_4=a_4\,;
\end{array}
\right.\\[12pt]
\left\{
\begin{array}{l}
k_1x_2+k_2x_3=a_2\,;\\[6pt]
k_2x_2+k_1x_3=a_3\,,
\end{array}
\right.
\end{array}
\right\}
\label{e18-f}
\end{equation}
%     
где введены обозначения $k_1\hm= \mu a_0\hm+\lambda a_2\hm+a_4$ 
и~$k_2\hm= a_1\hm+\lambda a_3 \hm+\mu a_5$. В~каждой из трех независимых 
систем из двух уравнений главный определитель равен одному и~тому же 
значению: 
$$
\Delta =\left( \mu a_0+\lambda a_2+a_4\right)^2-\left( a_1+\lambda a_3+\mu 
a_5\right)^2\,.
$$
  
  При выполнении условия $\Delta\hm= k_1^2\hm- k_2^2\not= 0$ 
сис\-те\-ма~(\ref{e18-f}) имеет единственное решение:
\begin{equation}
\left.
\begin{array}{rl}
x_0&=r_0=\fr{a_0k_1-a_5k_2}{\Delta}\,;\\[6pt]
x_1&=r_1= \fr{a_1k_1-a_4k_2}{\Delta}\,;\\[6pt]
x_2&=r_2=\fr{a_2k_1-a_3k_2}{\Delta}\,;\\[6pt]
x_3&=r_3=\fr{a_3k_1-a_2k_2}{\Delta}\,;\\[6pt]
x_4&=r_4=\fr{a_4k_1-a_1k_2}{\Delta}\,;\\[6pt]
x_5&=r_5=\fr{a_5k_1-a_0k_2}{\Delta}\,,
\end{array}
\right\}
\label{e19-f}
\end{equation}
которое определяет существование единственной локальной правосторонней 
единицы $R_A\hm= (r_0, r_1, r_2, r_3, r_4, r_5)$, соответствующей вектору~$A$ 
и~всевозможным степеням последнего. 

  Подставляя значения $x_0\hm=r_0$, $x_1\hm=r_1$, $x_2\hm=r_2$, 
$x_3\hm=r_3$ из формул~(\ref{e19-f}) в~формулу~(\ref{e17-f}), описывающую 
множество глобальных левосторонних единиц, можно получить $x_4\hm= 
1\hm- \mu r_0 \hm- \lambda r_2\hm=r_4$ и~$x_5\hm= -\mu^{-1} (r_1\hm+ \lambda 
r_3)\hm=r_5$ Последнее означает, что локальная правосторонняя 
единица~$R_A$ содержится в~множестве глобальных левосторонних 
единиц~(\ref{e17-f}), т.\,е.\ она является локальной двухсторонней 
единицей~$E_A$ вектора~$A$. 
  
  Легко показать, что в~бесконечной последовательности $A, A^2, A^3, \ldots , 
A^i,\ldots$ отсутствует нулевой вектор и~при некоторой минимальной 
натуральной степени~$d$ имеет место $A^d\hm= A$. Следовательно, $A^d\hm= 
A\hm\Rightarrow A^{d-1}\circ A\hm= A\circ A^{d-1}$, т.\,е.\ вектор 
$E_A\hm=A^\omega$, где $\omega \hm= d\hm-1$, есть локальная двухсторонняя 
единица вектора~$A$ (значение~$\omega$ будем называть локальным 
порядком локально обратимого вектора~$A$). Множество $\{ A, A^2, A^3, 
\ldots , A^i, \ldots , A^\omega\}$ представляет собой циклическую 
мультипликативную группу с~единицей~$E_A$.
  
\section{Задание новой формы скрытой задачи дискретного логарифмирования
и~схема цифровой подписи  на~ее основе}

  Для построения алгоритмов ЭЦП 
предлагается новая форма СЗДЛ, которая отличается использованием 
открытого ключа в~виде трех элементов КНАА, принадлежащих разным 
циклическим группам, и~описывается следующим образом. 
  \begin{enumerate}[1.]
  \item В качестве характеристики поля берем простое число~$p$ достаточно 
большой разрядности (например, 384~бит). 
  \item Выбираем три случайных локально обратимых вектора~$A$, $B$ 
и~$N$, локальный порядок которых содержит достаточно большой простой 
делитель.
  \item  Выбираем две случайные глобальные левосторонние единицы~$L_1$ 
и~$L_2$.
  \item  Вычисляем вектор~$A^\prime$ из уравнения 
  \begin{equation}
  A\circ A^\prime =L_1\,.
  \label{e20-f}
  \end{equation}
  \item Вычисляем вектор~$B^\prime$ из уравнения 
  \begin{equation}
  B\circ B^\prime =L_2\,.
  \label{e21-f}
  \end{equation}
  \item Вычисляем векторы~$T$ и~$L_3$ из уравнения 
  \begin{equation}
  A\circ T\circ B^\prime =L_3\,.
  \label{e22-f}
  \end{equation}
  \item Выбираем случайное натуральное число $x\hm<\omega$, 
где~$\omega$~--- значение порядка вектора~$N$.
  \item Вычисляем векторы~$Y$ и~$U$ по формулам:
  $$
  Y=A^\prime \circ N^x\circ A\,;\enskip U=B^\prime\circ N\circ B\,.
  $$
  \end{enumerate}
  
  В силу локальной обратимости векторов~$A$ и~$B$ уравнения~(\ref{e20-f}) 
и~(\ref{e21-f}) имеют единственное решение (см.\ решение  
системы~(\ref{e18-f})). Уравнение~(\ref{e22-f}) решается в~два этапа. Сначала 
вычисляются векторы~$T^\prime$ и~$L_3$ как неизвестные в~уравнении 
$T^\prime\circ B^\prime \hm=L_3$ (решение является единственным), а~затем 
находится вектор~$T$ из уравнения $A\circ T\hm= T^\prime$, которое имеет 
единственное решение.
  
  Открытым ключом служит тройка векторов~$Y$, $U$ и~$T$. Число~$x$ 
и~все другие векторы, использованные для вычисления открытого ключа, 
остаются секретными. Владелец открытого ключа должен хранить в~качестве 
своего личного секретного ключа число~$x$ и~два вектора~$A^\prime$ и~$B$. 
Остальные секретные элементы могут быть уничтожены после завершения 
процедуры вычисления открытого ключа. Предлагаемая форма СЗДЛ состоит 
в~вычислении значения~$x$ по открытому ключу. Схема ЭЦП на ее основе 
описывается следующим образом. 

  \vspace*{-6pt}
  
  \subsection*{Алгоритм генерации электронной цифровой подписи}
  
  \noindent
  \begin{enumerate}[1.]
  \item Выбрать случайное число $k\hm<\omega$ и~вычислить вектор $V\hm= 
A^\prime \circ N^k\circ B$.
  \item Вычислить значение $e\hm= F_h(M,V)$, где $F_h$~--- некоторая 
специфицированная хеш-функ\-ция; $M$~--- электронный документ, который 
должен быть подписан.
  \item Вычислить число $s\hm= k\hm+ex\,\mathrm{mod}\,q$.
  \end{enumerate}
  
  Пара чисел $(e,s)$ представляет собой ЭЦП к~документу~$M$.
  
  \vspace*{-6pt}
  
  \subsection*{Алгоритм проверки подлинности электронной цифровой подписи}
  
  \noindent
  \begin{enumerate}[1.]
  
  \item По значениям~$e$ и~$s$ вычислить вектор $V^\prime\hm= Y^e\circ 
T\circ U^s$. 
  \item Используя хеш-функ\-цию~$F_h$, вычислить значение $e^\prime\hm= 
F_h(M,V^\prime)$.
  \item Если $e^\prime\hm=e$, то подпись признается подлинной, иначе 
подпись отвергается как ложная. 
  \end{enumerate}
  
  Доказательство корректности работы схемы ЭЦП: 
  \begin{multline*}
 V^\prime =Y^e\circ T\circ U^s ={}\\
 {}=\left( A^\prime\circ N^x\circ A\right)^e\circ 
T\circ \left( B^\prime \circ N\circ B\right)^s={}
\end{multline*}

\noindent
  \begin{multline*}
    {}= A^\prime \circ N^{xe} \circ(A\circ T\circ B^\prime)\circ N^s\circ B= {}\\
{}=A^\prime \circ N^{xe}\circ L_3\circ N^{k-xe}\circ B={}\\
  {}= A^\prime \circ N^{xe+k-xe} \circ B =A^\prime \circ N^k \circ B 
=V\Rightarrow{}\\
{}\Rightarrow 
  e^\prime =F_h(M,V^\prime) =F_h(M,V)=e\,.
  \end{multline*}
  
  Предложенная в~данном разделе схема ЭЦП расширяет ранее известные 
типы криптосхем с~открытым ключом~\cite{4-f}, основанные на 
вычислительной трудности СЗДЛ. Вопрос о сверхполиномиальной сложности 
решения предложенной формы СЗДЛ на квантовом компьютере требует 
выполнения специальных математических исследований с~привлечением 
теории конечных алгебр и~связан с~изучением возможности и~трудоемкости 
сведения СЗДЛ к~обычной ЗДЛ. Это представляет собой самостоятельную 
исследовательскую задачу. 

  \vspace*{-9pt}
  
\section{Заключение}

\vspace*{-2pt}

  Разработан общий способ задания $m$-мер\-ных конечных некоммутативных 
ассоциативных алгебр для произвольного четного значения размерности 
$m\hm\geq4$, свойства которых позволили предложить новую форму СЗДЛ
 и~постквантовую схему цифровой 
подписи. 

  \vspace*{-9pt}
  
{\small\frenchspacing
 {%\baselineskip=10.8pt
 \addcontentsline{toc}{section}{References}
 \begin{thebibliography}{9}
 
 \vspace*{-2pt}
 
\bibitem{1-f}
Announcing request for nominations for public-key post-quantum cryptographic 
algoritms. Federal Register. Department of Commerce.  Vol.~81. No.\,244. P.~92787--92788.
{\sf  
https://www.gpo.gov/fdsys/pkg/FR-2016-12-20/pdf/2016-30615.pdf}.
\bibitem{2-f}
Post-quantum cryptography~/ Eds. T.~Lange, R.~Steinwandt.~--- 
Security and cryptology ser.~--- Springer, 2018. Vol.~10786. 542~p.
\bibitem{3-f}
\Au{Shor P.\,W.} Polynomial-time algorithms for prime factorization and discrete logarithms 
on quantum computer~// SIAM J.~Comput., 1997. Vol.~26. P.~1484--1509.
\bibitem{4-f}
\Au{Moldovyan~D.\,N.} Non-commutative finite groups as primitive of public-key 
cryptoschemes~// Quasigroups Related Systems, 2010. Vol.~18. P.~165--176.
\bibitem{5-f}
\Au{Kuzmin A.\,S., Markov~V.\,T., Mikhalev~A.\,A., Mikhalev~A.\,V., Nechaev~A.\,A.} 
Cryptographic algorithms on groups and algebras~// J.~Math. Sci., 2017. Vol.~223. Iss.~5. 
P.~629--641.
\end{thebibliography}

 }
 }

\end{multicols}

\vspace*{-6pt}

\hfill{\small\textit{Поступила в~редакцию 27.06.19}}

\vspace*{8pt}

%\pagebreak

\newpage

\vspace*{-28pt}

%\hrule

%\vspace*{2pt}

%\hrule

%\vspace*{-2pt}

\def\tit{METHOD FOR~DEFINING FINITE NONCOMMUTATIVE 
ASSOCIATIVE ALGEBRAS OF~ARBITRARY EVEN~DIMENSION 
FOR~DEVELOPMENT OF~THE~POSTQUANTUM CRYPTOSCHEMES}


\def\titkol{Method for~defining finite noncommutative 
associative algebras of~arbitrary even~dimension 
for~development of %~the~postquantum 
cryptoschemes}

\def\aut{A.\,A.~Kostina, A.\,Yu.~Mirin, D.\,N.~Moldovyan, and~R.\,Sh.~Fahrutdinov}

\def\autkol{A.\,A.~Kostina, A.\,Yu.~Mirin, D.\,N.~Moldovyan, and~R.\,Sh.~Fahrutdinov}

\titel{\tit}{\aut}{\autkol}{\titkol}

\vspace*{-11pt}


 \noindent
  St.\ Petersburg Institute for Informatics and Automation of the Russian Academy 
of Sciences, 39,~14th Line V.O., St.\ Petersburg 199178, Russian 
Federation

\def\leftfootline{\small{\textbf{\thepage}
\hfill INFORMATIKA I EE PRIMENENIYA~--- INFORMATICS AND
APPLICATIONS\ \ \ 2020\ \ \ volume~14\ \ \ issue\ 1}
}%
 \def\rightfootline{\small{INFORMATIKA I EE PRIMENENIYA~---
INFORMATICS AND APPLICATIONS\ \ \ 2020\ \ \ volume~14\ \ \ issue\ 1
\hfill \textbf{\thepage}}}

\vspace*{3pt} 

 

\Abste{The paper introduces a new unified method for defining finite noncommutative associative 
algebras of arbitrary even dimension~$m$ and describes the investigated properties of the algebras 
for the cases $m = 4$ and~$6$, when the algebras are defined over the ground field $GF(p)$ 
with a~large size of the prime number~$p$. Formulas describing the set of~$p^2$ ($p^4$) global 
left-sided units contained in the 4-dimensional (6-dimensional) algebra are derived. Only local 
invertibility takes place in the algebras investigated. Formulas for computing the unique local 
two-sided unit related to the fixed locally invertible vector are derived for each of the algebras. A~new 
form of the hidden discrete logarithm problem is proposed as postquantum cryptographic 
primitive. The latter was used to develop the postquantum digital signature scheme.} 

  \KWE{finite noncommutative algebra; associative algebra; computationally difficult problem; 
discrete logarithm; digital signature; postquantum cryptography}
  
\DOI{10.14357/19922264200113} 

%\vspace*{-14pt}

%\Ack
%\noindent

 


%\vspace*{6pt}

  \begin{multicols}{2}

\renewcommand{\bibname}{\protect\rmfamily References}
%\renewcommand{\bibname}{\large\protect\rm References}

{\small\frenchspacing
 {%\baselineskip=10.8pt
 \addcontentsline{toc}{section}{References}
 \begin{thebibliography}{9}
\bibitem{1-f-1}
 Depatment of Commerce. 2016. Announcing request for nominations for 
 public-key post-quantum cryptographic algorithms. Federal Register 
81(244):92787--92788. Available at: 
{\sf https://www.gpo.gov/fdsys/pkg/FR-2016-12-20/pdf/2016-30615.pdf} (accessed 
March~2, 2020).
\bibitem{2-f-1}
Lange,~T., and R.~Steinwandt, eds.
2018. \textit{Post-quantum cryptography}. Security and cryptology ser.
Springer. Vol.~10786. 542~p.
\bibitem{3-f-1}
\Aue{Shor, P.\,W.} 1997. Polynomial-time algorithms for prime factorization and 
discrete logarithms on quantum computer. \textit{SIAM J.~Comput.} 
26:1484--1509.
\bibitem{4-f-1}
\Aue{Moldovyan, D.\,N.} 2010. Non-commutative finite groups as primitive of 
public-key cryptoschemes. \textit{Quasigroups Related Systems} 18:165--176.
\bibitem{5-f-1}
\Aue{Kuzmin, A.\,S., V.\,T.~Markov, A.\,A.~Mikhalev, A.\,V.~Mikhalev, and 
A.\,A.~Nechaev.} 2017. Cryptographic algorithms on groups and algebras. 
\textit{J.~Math. Sci.} 223(5):629--641.
 \end{thebibliography}

 }
 }

\end{multicols}

%\vspace*{-7pt}

\hfill{\small\textit{Received June 27, 2019}}

%\pagebreak

%\vspace*{-22pt} 
  
  \Contr
  
  \noindent
  \textbf{Kostina Anna A.} (b.\ 1983)~--- scientist, Laboratory of Cybersecurity 
and Postquantum Cryptosystems, St.\ Petersburg Institute for Informatics and 
Automation of the Russian Academy of Sciences, 39,~14th Line V.O., St.\ 
Petersburg 199178, Russian Federation; \mbox{anna-kostina1805@mail.ru}
  
  \vspace*{3pt}
  
  \noindent
  \textbf{Mirin Anatoly Yu.} (b.\ 1979)~--- Candidate of Science (PhD), senior 
scientist, Laboratory of  Cybersecurity and Postquantum Cryptosystems, 
  St.\ Petersburg Institute for Informatics and Automation of the Russian Academy 
of Sciences, 39,~14th Line V.O., St.\ Petersburg 199178, Russian Federation; 
\mbox{mirin@cobra.ru}
  
  \vspace*{3pt}
  
  \noindent
  \textbf{Moldovyan Dmitriy N.} (b.\ 1986)~--- Candidate of Science (PhD), 
scientist, Laboratory of  Cybersecurity and Postquantum Cryptosystems, St. 
Petersburg Institute for Informatics and Automation of the Russian Academy of 
Sciences, 
  39,~14th Line V.O., St.\ Petersburg 199178, Russian Federation; 
\mbox{mdn.spectr@mail.ru}
  \vspace*{3pt}
  
  \noindent
  \textbf{Fahrutdinov Roman Sh.} (b.\ 1972)~--- Candidate of Science (PhD), 
Head of Laboratory of Cybersecurity and Postquantum Cryptography, St.\ Petersburg 
Institute for Informatics and Automation of the Russian Academy of Sciences, 
  39, 14th Line V.O., St.\ Petersburg 199178, Russian Federation; 
\mbox{fahr@cobra.ru}
  



\label{end\stat}

\renewcommand{\bibname}{\protect\rm Литература} 
  