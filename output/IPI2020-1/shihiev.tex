\def\stat{shihiev}

\def\tit{ИНКАПСУЛЯЦИЯ СЕМАНТИЧЕСКИХ ПРЕДСТАВЛЕНИЙ В~ЭЛЕМЕНТЫ 
ГРАММАТИКИ}

\def\titkol{Инкапсуляция семантических представлений в~элементы 
грамматики}

\def\aut{Ш.\,Б.~Шихиев$^1$, Ф.\,Ш.~Шихиев$^2$}

\def\autkol{Ш.\,Б.~Шихиев, Ф.\,Ш.~Шихиев}

\titel{\tit}{\aut}{\autkol}{\titkol}

\index{Шихиев Ш.\,Б.}
\index{Шихиев Ф.\,Ш.}
\index{Shihiev Sh.\,B.}
\index{Shihiev F.\,Sh.}


%{\renewcommand{\thefootnote}{\fnsymbol{footnote}} \footnotetext[1]
%{Работа выполнена при частичной поддержке РФФИ (проекты 19-07-00352 и~18-29-03100) и~Стипендии Президента Российской Федерации молодым ученым и~аспирантам (СП-538.2018.5). Для ускорения обучения был использован гибридный высокопроизводительный вычислительный комплекс ЦКП <<Информатика>> ФИЦ ИУ РАН: 
%{\sf http://ckp.frccsc.ru}.}}


\renewcommand{\thefootnote}{\arabic{footnote}}
\footnotetext[1]{Дагестанский государственный университет, sh\_sh\_b51@mail.ru}
\footnotetext[2]{Дагестанский государственный университет, fuad@mail.ru}

%\vspace*{-12pt}


  

     \Abst{Предлагается новый математический аппарат представления естественного 
языка (ЕЯ) для компьютерной лингвистики~--- морфология, синтаксис и~семантика описаны 
как предметы дискретной математики, образующие иерархию и~целостную 
информационную систему. 
     Предлагаемая конструктивная теория языка представляет собой новый подход 
     к~изучению языка путем разделения полномочий синтаксиса и~семантики; построения 
автономных моделей синтаксиса и~семантики; формирования языка как отображения 
элементов двух множеств: синтаксиса и~семантики.}
     
      \KW{естественный язык; граф; синтаксис; семантика; лексика; словоформа; 
морфологический признак; лексическая группа; словарь; предложение; алгоритм}

\DOI{10.14357/19922264200116} 
  
%\vspace*{-3pt}


\vskip 10pt plus 9pt minus 6pt

\thispagestyle{headings}

\begin{multicols}{2}

\label{st\stat}
    
\section{Введение}

    \textbf{А.} В~ЕЯ переплетены два автономных 
явления: дискретное (грамматика) и~аналоговое (семантика). Грамматика 
(морфология и~синтаксис) может быть описана на языке математики. 
Морфологические формы слов позволяют их различать и~приписывать им 
и~их сочетаниям различные значения, которые фиксируются в~семантическом 
словаре. В~данной статье показано, как это можно сделать.
    
    \textit{Модель морфологии} заключает в~себе правила построения 
словоформ; словоформа представлена \textit{словом} в~исходной форме и~ее 
\textit{морфологическими параметрами} из десятичных цифр. Словоформы 
с~одинаковыми \textit{морфологическими параметрами} (\textit{формами}) 
образуют \textit{лексическую группу}. 
    
     \textit{Модель синтаксиса} ЕЯ строится исходя из следующих 
предположений: в~ЕЯ имеются такие \textit{элементарные предложения}, что 
из $n$ членов предложения всегда можно составить ($n\hm-1$) 
\textit{словосочетаний} (\textit{синтаксически связанных пар словоформ}), 
таких что члены предложения образуют \textit{связный граф}; следовательно, 
объединение \textit{элементарных предложений} образует 
\textit{синтаксический граф} $\mathrm{Sint}\hm = (X, Y)$ из 
множеств~$X$ (\textit{словоформ}) и~$Y$ (\textit{словосочетаний}). 
Критерий связности двух словоформ определяется через их 
\textit{морфологические формы} (\textit{параметры}). Нетрудно 
проверить~[1], что миллионы \textit{словосочетаний}, имеющих место 
в~грамматике русского языка, распределены по трем сотням 
\textit{синтаксических отношений}~--- прямых произведений лексических 
групп, что упрощает представление \textit{графа} $\mathrm{Sint}$ в~памяти 
компьютера.
     
    \textbf{Б.}\ Предлагаемая сетевая модель грамматики $\mathrm{Sint}$ открывает 
следующие возможности в~изучении ЕЯ и~его реализации на компьютере:
    \begin{enumerate}[1.]
    \item Корневые деревья из модели $\mathrm{Sint}$ порождают 
\textit{элементарные пред\-ло\-же\-ния-де\-ревья}. Обход 
\textit{пред\-ло\-же\-ния-де\-ре\-ва} сопоставляет ему 
\textit{пред\-ло\-же\-ние-по\-сле\-до\-ва\-тель\-ность}; 
корректность обратной задачи, известной как 
\textit{синтаксический анализ}\linebreak
\textit{предложения}, зависит от того, в~какой степени\linebreak 
соблюдены принципы \textit{фрагментарности сегментов} между этими 
предложениями, т.\,е.\ являются ли \textit{сегменты} (вершины ветвей)  
в~\textit{пред\-ло\-же\-нии-де\-ре\-ве фрагментами} в~соответствующем 
\textit{пред\-ло\-же\-нии-по\-сле\-до\-ва\-тель\-ности}~\cite{2-shi}. 
    
    \item Из элементарных предложений строятся более сложные 
предложения по правилам синтаксиса. Из правил построения модели $\mathrm{Sint}$ 
следует возможность построения такой модель грамматики, что любое 
предложение русского языка (например, предложения, встречающиеся 
в~литературе на русском языке) порождается элементами синтаксиса $\mathrm{Sint}$; 
следовательно, имеется формальное определение \textit{синтаксически 
правильно построенного предложения} и~соответствующий алгоритм 
распознавания таких предложений~\cite{2-shi}.
    \item \textit{Модель морфологии} (правила преобразования слов) 
представлена в~\textit{морфологическом словаре}; программа, реализующая 
эти правила, образует \textit{компьютерную модель морфологии}. Приемы 
реализации \textit{модели синтаксиса} (алгоритмов анализа и~синтеза 
предложений) демонстрируются в~\cite{3-shi}. Значения предложений 
в~синтаксисе $\mathrm{Sint}$ представлены в~\textit{семантическом словаре} в~виде 
классов для реализации \textit{семантической модели семантики} 
посредством объектных технологий программирования.
    \end{enumerate}
    
    \textbf{В.} Обращаясь к~истории вопроса, можно на\-пом\-нить следующее. 
В~данной работе реализована идея Ф.~де Сос\-сю\-ра и~Л.~Ельм\-сле\-ва, 
согласно которой строится автономное и~конструктивное\linebreak описание 
синтаксиса, порождающего <<планы выражений>>, далее в~каждый <<план 
выражения>> инкапсулируется <<план содержания>>~--- элемент 
семантики, а~\textit{язык} становится отображением (биекцией) элементов 
двух множеств: синтаксиса и~семантики.
    
    Представляется, что только разделение полномочий синтаксиса 
и~семантики открывает путь к~формализации ЕЯ, иначе придется 
согласиться с~утверждениями типа: <<Никто не может сформулировать все 
правила английской грамматики$\ldots$>>~[4].
    
    В учебниках по <<Общему синтаксису>>~[5] и~в~работах 
И.\,А.~Мельчука~[6] осторожно указывалось на древовидность структуры 
предложения и~графы использовались только для демонстрации сетевой 
структуры предложения. Нужно было решиться и~выделить класс 
\textit{элементарных предложений}, имеющих структуру \textit{корневого 
дерева}, а~далее заметить, что все другие формы предложения 
(с~однородными членами, сложные предложения и~т.\,д.)\ собраны из 
элементарных предложений.
    
    Многие правила построения синтаксически правильных сочетаний 
можно представить в~виде правил кон\-текст\-но-сво\-бод\-ной грамматики 
(например, выражения, образованные из согласованных и~несогласованных 
определений). Этот факт был использован для представления \textit{правил 
синтаксиса} ЕЯ посредством \textit{подстановок} и~\textit{деревьев 
разбора}. Предложение <<Сколько чувствительности контекста требуется, 
чтобы предоставлять разумные структурные описания?>> (How much 
context-sensitivity is required to provide reasonable structural descriptions?) 
из~[7] указывает на то, что автор <<Грамматики сложения деревьев>> (Tree 
adjoining grammars) далек от мысли раздельного исследования синтаксиса 
и~семантики.
    
\section{Морфология}

    \textit{Морфология} есть структура, заданная тройкой ($A, L_0, F_0$), где 
$A$~--- \textit{алфавит}; $L_0$~--- \textit{исходная лексика}, 
представляющая собой конечное множество \textit{исходных слов} над 
алфавитом~$A$; $F_0$~--- конечный набор \textit{исходных 
морфологических признаков}, пред\-став\-ля\-ющих собой двухразрядные 
десятичные числа.
    
    \textit{Исходные признаки} разбиты на непересекающиеся 
подмножества; элементы каждого подмножества образуют линейный массив, 
который называется \textit{категорией} (\textit{признаков}). Категорий 
в~морфологии русского языка меньше десяти, и~они именованы кодами: 
10~(род), 20~(число), 30~(падеж), 40~(степень) и~т.\,д., а~исходные признаки в~категориях кодированы следующим образом: $10\hm = (11, 12, 13)$, $20 
\hm= (21, 22)$, $30 \hm= (31, 32, 33, 34, 35, 36)$, $40 \hm= (41, 42, 43)$ и~т.\,д., 
или в~более привычной записи: $10 \hm= (\mathrm{м.\ род, ср.\ род, ж.\ 
род})$, $20 \hm= (\mathrm{ед.\ число},\linebreak \mathrm{мн.\ число})$, $30\hm = (\mathrm{И., 
Р., Д., В., Т., П.})$, $40 \hm= (\mathrm{полная}\linebreak
\mathrm{форма\ ИП,\ краткая\ форма\ 
ИП, сравнительная\ сте\mbox{-}}\linebreak \mathrm{пень\ ИП})$ и~т.\,д.
    
    \textit{Исходная лексика} также разбита на непересекающиеся 
подмножества~--- \textit{исходные части речи}; к~каждой \textit{исходной 
части речи}~$D_0$ прикреплен свой набор категорий~$\Psi$ и~множество 
\textit{признаков}~$\Omega$; \textit{признак} представляет собой строку 
$f\hm = \mbox{<<}\alpha_1\alpha_2\cdots \alpha_k\mbox{>>}$ из 
\textit{исходных признаков}, принадлежащих различным категориям 
из~$\Psi$. Один из признаков~$f_0$ называется \textit{начальным 
признаком}. Слова из~$D_0$ обладают \textit{начальным признаком}; 
\textit{исходное слово}~${s}_0$ представлено строкой вида 
<<${s}_0:f_0$>>; форма~${s}$ слова~${s}_0$ 
с~признаком~$f$ будет представлена в~виде <<${s}_0:f$>>: 
например, \textbf{дом}\;:\;2133\;=\;\textbf{дому}.
    
    Части речи именованы кодами: 01~--- имя существительное (ИС), 02~--- 
имя прилагательное (ИП), 07~--- глагол и~т.\,д. Категориями для ИС 
являются~20 и~30, а признаками будут $\Omega\hm = \{2131, 2132, 2133, 
2134, 2135, 2136, 2231, 2232,\linebreak
 2233, 2234, 2235, 2236\}$. Для ИП из 
категорий~10, 20, 30 и~40 составлено 29~признаков: $41112131,\linebreak 41112132, 
41112133, 41112134, 41112135, \ldots, 43$. 
    
    Множество ${s}_0:\Omega\hm = \{{s}_0:f \vert  f\hm\in 
\Omega\}$ называется \textit{морфологической группой} 
слова~${s}_0$, а~ее элементы называются формами 
слова~${s}_0$, или просто \textit{словоформами}. Группа 
${s}_0:\Omega$ для ИС~$s_0$ состоит из~12~слов, а~для ИП~--- из 
29~словоформ; нетрудно заметить, что элементы множества 
${s}_0:\Omega$ различны. 
    
    Объединение множеств ${s}_0:\Omega$ по всем~$s_0$ 
из~$D_0$ обозначается через~$D$ и~называется \textit{частью речи}. 
Морфология теперь может быть представлена чет\-вер\-кой ($A, L, \Psi1, 
\Omega1$), где лексика~$L$~--- объединение всех \textit{частей речи}; 
$\Psi1$~--- множество категорий; $\Omega1$~--- множество признаков.
    
    Числовые признаки при исходной форме слова обозначают формы слова, 
на которых будет построен синтаксис (и~предложения) языка. Для 
преобразования словоформ ЕЯ в~слова с~числовыми признаками и~обратно 
нужны соответствующие \textit{морфологические правила}; их, как известно, 
можно найти в~словообразовательных словарях.
    
     Рассмотрим слово $\mathrm{s}_0:\alpha_1\alpha_2\cdots \alpha_k$. По 
определению исходный признак $\alpha_k$ принадлежит некоторой 
категории~$F_k$. Морфология ЕЯ обладает таким свойством, что 
признаками морфологии выступают все строки $\alpha_1\alpha_2\cdots \alpha_k$, 
где~$\alpha_k$ пробегает элементы массива~$F_k$; они образуют множество 
$\alpha_1\alpha_2\cdots \alpha_{k-1}F_k$~--- \textit{парадигму} слова~${s}_0$ по 
категории~$F_k$. Формы слова сгруппированы по \textit{парадигмам}. 
В~электронном словаре вместо элементов \textit{парадигм} будут записаны 
алгоритмы (морфологические правила), порождающие их элементы.
     
     Например, в~строке 
     
     <<2130дом,дома,дому,дом,домом,доме>> 
     
     \noindent
     за 
\textit{парадигмой} 2130~слова \textbf{дом} перечислены его элементы. Есть 
возможность вместо словоформ записать их постфиксы следующим образом:
     \begin{multline}
      \mbox{дом0115:213000,а,у,,ом,е;}\\
      \mbox{223000а,ов,ам,а,ами,ах}.
      \label{e1-shi}
      \end{multline}
      
     В первой части <<дом0115>> статьи~(1) за словом <<дом>> указаны 
его \textit{грамматические атрибуты} (01~--- код ИС, 15~--- мужской род 
и~неодушевленное); во второй части перечислены два 
\textit{морфологических правила} (разделенных точкой с~запятой), по 
которым строятся элементы \textit{парадигмы}~2130 и~2230. К~коду 
\textit{парадигмы} приписано двухразрядное число~--- длина изменяемой 
части словоформ из этой парадигмы. Для экономии памяти имеет смысл 
хранить отдельным списком постфиксы элементов \textit{парадигмы}, 
а~в~статьях словаря указать их порядковые номера. 
     
     Построение словарных статей представляет собой рутинную работу, 
которую также можно запрограммировать.
     
     Пусть $s$~--- обычная форма слова~${s}_0$ с~признаком 
$\alpha_1\alpha_2\cdots \alpha_k$, т.\,е.\ $s \hm= s_0:\alpha_1\alpha_2\cdots\alpha_k$. В~языковом 
явлении морфология решает две задачи: \textit{синтеза}~--- перехода от 
$s_0:\alpha_1\alpha_2\cdots\alpha_k$ к~$s$~--- и~\textit{анализа}~--- перехода 
от~$s$ 
к~$s_0:\alpha_1\alpha_2\cdots\alpha_k$. 
     
     Чтобы осуществить \textit{анализ} словоформы, потребуется 
осуществить синтез всех элементов множества $s_0:\Omega$. Анализ 
словоформы усложняется тем, что по форме слова~$s$ практически 
невозможно точно угадать его исходную форму~$s_0$; а~словарная \mbox{статья} 
начинается с~исходной формы слова~$s$. Анализ словоформы~--- 
трудоемкая процедура, для повышения ее эффективности требуется 
использовать различные приемы поиска из дискретного анализа.

\section{Синтаксис}

    Пусть $D^1, D^2, \ldots, D^q$~--- коды частей речи (как изменяемых, так 
и~наречия, которые образуют одну \textit{лексическую группу}) в~морфологии 
$\mu\hm= (A, L, \Psi1, \Omega1)$, для них определены множества признаков 
$\Omega^1, \Omega^2,\ldots , \Omega^q$ соответственно.
    
    Если $a\in \Omega^i$, то через $a:D^i$, или $D^ia$, обозначается 
множество слов из~$D^i$, обладающих признаком~$a$, и~называется 
\textit{лексической группой} по признаку~$a$. Например, $2135(01) \hm= 
012135\hm = \{\mbox{\textbf{домом}, \textbf{точкой}, 
\textbf{едой},}\ldots\}$.
    
    Через $\Omega^i(D^i)$, или $[D^i]$, обозначается множество, состоящее 
из \textit{лексических групп} $D^ia$ по всем признакам~$a$ из~$\Omega^i$. 
Например, множество~[01] состоит из~12 \textit{лексических групп}: $012131, 
012132, 012133, 012134, \ldots , 012236$.
    
    Через $\Lambda_\mu$ обозначено объединение всех $[D^1], [D^2], \ldots, 
[D^q]$. Построение синтаксиса начинается с~выбора нескольких пар 
\textit{лексических групп} из множества~$\Lambda_\mu$:
    \begin{equation}
    {X}_1\ \mbox{и } {Y}_1\,;\ 
    {X}_2\ \mbox{и } {Y}_2\,;\ 
    {X}_3\ \mbox{и } {Y}_3\,;\ldots ;
    {X}_k\ \mbox{и } {Y}_k\,.
    \label{e2-shi}
    \end{equation} 
     Через $R$ обозначено объединение произведений:
    \begin{equation}
    R={X}_1*{Y}_1\cup
    {X}_2*{Y}_2\cup\cdots
    \cup {X}_k*{Y}_k
    \,.
    \label{e3-shi}
    \end{equation}
   
   Произведение \textit{лексических групп} ${X}_i*{Y}_i$ 
($i\hm = 1, \ldots , k$) называется \textit{синтаксическим отношением} (СО), 
их элементы~--- \textit{синтаксически связанными словоформами} или 
\textit{словосочетаниями}; в~\textit{словосочетании} ($x, y$) слово~$x$~--- 
главный член, $y$~--- зависимый член сочетания. Например, $012131*012132 
\hm= \{\mbox{(\textbf{дом}, \textbf{моды}), (\textbf{запах}, \textbf{дыма}), 
(\textbf{небо}, \textbf{сна}),}\ldots\}$.
   
    Орграф ($L, R$) задает \textit{синтаксис} $\mathrm{Sint}$. Орграф ($L1, R1$), где 
$L1$ состоит из \textit{лексических групп}~(2), а~$R1$~--- из пар 
(${X}_i, {Y}_i$), где $i \hm= 1,\ldots , k$, также задает 
\textit{синтаксис} $\mathrm{Sint}$ в~упакованном виде; 
пусть $\mathrm{Sint}1 \hm= (L1, R1)$. 
Через $\mathrm{Sint}\,(\Lambda_\mu)$ обозначается некоторый синтаксис (грамматика), 
заданный на~$\Lambda_\mu$.
    
    Синтаксические отношения~(\ref{e3-shi}) задаются парами морфологических признаков
    \begin{equation}
    f_1\ \mbox{и } g_1\,;\ 
    f_2\ \mbox{и } g_2\,; \ldots ; 
    f_k\ \mbox{и } g_k\,,
    \label{e4-shi}
    \end{equation}
где признаки $f_i$ и~$g_i$ определяют \textit{лексические группы} 
${X}_i$ и~${Y}_i$ ($i \hm= 1, \ldots , k$). Следовательно, 
необъятного размера синтаксис $\mathrm{Sint}$ задается небольшим набором (около 
двухсот пар) признаков~(\ref{e4-shi}) из~$\Omega1$.
    
    Например, если $D^1$~--- ИС, $D^2$~--- ИП, а~$f_1 \hm= 012131$ и~$g_1 
\hm= 012132$, $f_2 \hm= 012131$ и~$g_2 \hm= 022131$, $f_3 \hm= 012132$ 
и~$g_3 \hm= 022132$, то в~графе ($A, L, \Psi1, \Omega1$) будут связаны дугой 
$(v, w)$ только те вершины~$v$ и~$w$, которые принадлежат СО: 
$012131*012132$, $012131*022131$, $012132*022132$.
    
    В каждом из трех множеств содержатся сотни тысяч элементов. 
Элементами СО $012131*012132$ ($012131*022131$, $012132*022132$) 
являются \textit{несогласованные} (\textit{согласованные}) определения.
    
    Корневое дерево в~графе $\mathrm{Sint}$ называется \textit{вы\-ра\-же\-ни\-ем-де\-ре\-вом}. 
Среди лексических групп имеются\linebreak\vspace*{-12pt}

{ \begin{center}  %fig1
 \vspace*{-1pt}
   
 \mbox{%
 \epsfxsize=79.057mm 
 \epsfbox{shi-1.eps}
 }


\vspace*{6pt}


\noindent
{{\figurename~1}\ \ \small{Корневое дерево в~$\mathrm{Sint1}$}}
\end{center}
}

\vspace*{12pt}

\addtocounter{figure}{1}

\noindent
 две группы: \textit{группа сказуемых} 
$\mathrm{GV}$ и~\textit{группа подлежащих} $\mathrm{GS}$. Если корень \textit{вы\-ра\-же\-ния-де\-ре\-ва} 
принадлежит $\mathrm{GV}$ и~связан дугой с~вершиной из $\mathrm{GS}$, то такое 
\textit{вы\-ра\-же\-ние-де\-ре\-во} называется  
\textit{пред\-ложением-де\-ре\-вом}. Обход  
\textit{вы\-ра\-же\-ния-де\-ре\-ва} называется  
\textit{вы\-ра\-же\-ни\-ем-по\-сле\-до\-ва\-тель\-ностью} (или прос\-то 
выражением), обход \textit{пред\-ло\-же\-ния-де\-ре\-ва} называется\linebreak  
\textit{пред\-ло\-же\-ни\-ем-по\-сле\-до\-ва\-тель\-ностью} (просто 
\textit{предложением}).
    
    Если в~грамматике $\mathrm{Sint}\,(\Lambda_\mu)$:
    \begin{enumerate}[(1)]
\item множество $L$~--- лексика русского языка, (3)~--- функции из 
морфологии русского языка; 
\item элементы ${X}_i*{Y}_i$ ($i \hm= 1,\ldots , k$)~--- 
словосочетания (пары связанных словоформ), допустимые в~синтаксисе 
русского языка; 
\item связанные пары словоформ в~выражениях русского языка образуют 
корневое дерево, 
\end{enumerate}
то $\mathrm{Sint}\,(\Lambda_\mu)$ должен иметь много общего с~синтаксисом русского 
языка; поэтому $\mathrm{Sint}\,(\Lambda_\mu)$ будем называть \textit{моделью} 
грамматики русского языка. Нетрудно показать существование такой 
грамматики $\mathrm{Sint}\,(\Lambda_\mu)$, что предложения, встречающиеся 
в~литературе на русском языке, будут предложениями в~грамматике 
$\mathrm{Sint}\,(\Lambda_\mu)$. (Но в~грамматике $\mathrm{Sint}\,(\Lambda_\mu)$ будут 
предложения <<сомнительного>> значения.)

    Грамматика $\mathrm{Sint}$ будет использована для программного построения 
и~распознавания \textit{выражений} в~синтаксисе $\mathrm{Sint}$. $\mathrm{Sint}$~--- открытая 
система, в~ней могут появляться новые слова и~синтаксически связанные 
пары слов.
    
    В частности, известная задача \textit{синтаксического анализа 
предложения} одинаково формулируется и~решается как в~ЕЯ, так 
и~в~грамматике $\mathrm{Sint}$: \textit{синтаксически правильно построенное} 
в~грамматике $\mathrm{Sint}$ \textit{предложение} будет \textit{синтаксически 
правильно построенным предложением} и~в~грамматике русского языка. 
Последовательность словоформ образует \textit{синтаксически правильно 
построенное} в~грамматике $\mathrm{Sint}$ \textit{предложение}, если в~графе 
$\mathrm{Sint}$ 
найдется \textit{пред\-ло\-же\-ние-де\-ре\-во}, по\-рож\-ден\-ное этим набором 
словоформ. А~алгоритмы поиска корневого дерева, по\-рож\-ден\-но\-го заданным 
множеством вершин, хорошо известны. Строгая формулировка задачи 
\textit{синтаксического анализа предложения} и~наличие алгоритма ее 
решения в~грамматике ЕЯ уже дорогого стоит.
    
     Важным понятием в~грамматике $\mathrm{Sint}$ является СФ~--- 
\textit{синтаксическая форма}. Показанное на рис.~1 дерево имеет 
скобочные формы представления: через вершины~--- $A(B, C(D(H), E))$~--- 
и~дуги~--- $0(1, 2(3(5), 4))$.
                    
 
     
     Вершинами дерева $A(B, C(D(H), E))$ являются \textit{лексические 
группы}, поэтому оно порождает пред\-ло\-же\-ния-де\-ревья $a(b, c(d(h), e))$, 
где строчной буквой обозначена словоформа из группы, обозначенной этой 
же буквой в~верхнем регистре. Предложения из $\mathrm{Sint}$, порожденные 
\textit{корневым деревом} из $\mathrm{Sint1}$, имеют одну и~ту же синтаксическую 
форму, поэтому такие деревья называются СФ. Более того, СФ представляет 
собой правило, порождающее класс предложений определенной формы.
     
     Предположим, что дуга~$i$ ($i \hm= 1, \ldots, 5$) дерева с~рис.~1 (СФ1) 
представлена произведением пары признаков $f_i*g_i$. Синтаксические связи 
между вершинами дерева требуют, чтобы $f_3\hm=f_4\hm =g_2$, $f_5\hm=g_3$; 
любые шесть слов с~указанными признаками могут оказаться вершинами 
предложения, порожденного~СФ1.
     
     Присваивая различным вершинам СФ1 различные словоформы, можно 
строить различные предложения. Обозначив через $g(t)$ словоформу $t:g$ 
(форму~$g$ слова~$t$), можно выписать форму по\-рож\-да\-емых СФ1 
предложений:

\vspace*{3pt} 

\noindent
     \begin{equation}
               f_1(s_1)\left(g_1(t_1), g_2(t_2)\left(g_3(t_3)(g_5(t_5)), 
g_4(t_4)\right)\right).
               \label{e5-shi}
               \end{equation}
Выражение~(\ref{e5-shi}) удобно представить в~виде двух изоморфных 
деревьев:

\vspace*{3pt}

\noindent
\begin{equation}
f_1
\left(g_1, g_2\left(g_3(g_5), g_4\right)\right):
s_1\left(t_1, t_2\left(t_3(t_5), 
t_4\right)\right).
               \label{e6-shi}
               \end{equation}
Нетрудно заметить, что СФ1 представлена выражением $f_1(g_1, g_2(g_3(g_5), 
g_4))$. Примером СФ служит выражение 

\vspace*{3pt}

\noindent
\begin{equation*}
               01112131(02112131, 01112132(02112132))\,,
              %\label{e7-shi}
              \end{equation*}
которое порождает выражения типа <<\textbf{белый дом старого 
охотника}>>.
     
     В форме~(\ref{e6-shi}) будут храниться выражения 
в~\textit{семантическом словаре}. Исследования текстов показывают, что 
число различных СФ, порождающих простые предложения на русском языке, 
не превышает сотни; три десятка СФ позволяют носителю русского языка 
вполне красноречиво выражать свои мысли; а~у~каждого автора текстов 
имеются характерные для него СФ, которыми он пользуется для выражения 
своих мыслей.

     \begin{figure*}[b] %fig2
     \begin{center}
     \begin{tabular}{lr}
  {аист040613/01птица}(      &       (1)\\
  {\textbf{форма}}: (01аист; 02{аистовый}, 02{аистиный}; 07); 
&   (2)\\
  {\textbf{свойство}}: &                        (3)\\
  \hspace*{5mm}({СФ}1: {цвет}({белый}); &          
    (4)\\
  \hspace*{5mm}{СФ}2: {вес}({до}~7~кг); &          
    (5)\\
  \hspace*{5mm}{СФ}4: {местонахождение}({деревня}, 
{поле}); &   (6)\\
  \hspace*{5mm}{СФ}3: 
{местожительство}({гнездо}({СФ}6: {кровля}, {СФ}: 
{дерево})); &(7)\\
  {\textbf{элемент}}: ({СФ}5: {клюв}({СФ}1: {длинный})); &
    (8)\\
  {\textbf{событие}}: ({СФ}6: {сидеть}, {летать}); &  (9)\\
  {\textbf{метод}}: ({СФ}7: {клевать}({трава})) & (10)\\
  )  &          (11)
  \end{tabular}
\end{center}
 %  \vspace{6pt}
\Caption{Статья семантического словаря, описывающая понятие \textit{аист }}
\end{figure*}



%\vspace*{-8pt}

\section{Семантика}

%\vspace*{-2pt}

     При всей своей содержательности формальная грамматика без 
семантики не образует языка. \textit{Семантика} строится на элементах 
синтаксиса; элементы синтаксиса (слова и~их сочетания) должны 
\textit{выражать знания}. Знание, выраженное элементом синтаксиса, 
называется его \textit{значением} или \textit{семантикой}. 
     
     Знание есть специфическая форма \textit{ощущения} сознанием 
активного состояния определенной об\-ласти памяти человека; свидетелем 
существования \textit{знания} является человек, \textit{ощущающий} его; 
\textit{знание} о~слове (как последовательности букв)~--- назовем его 
\textit{именем} слова~--- также хранится в~памяти; к~\textit{знанию} о~слове 
прикреплено иное \textit{знание}, называемое его \textit{значением}; человек 
способен воспроизводить слово; воспринятое человеком \textit{слово} 
активизирует его \textit{значение}; слово и~его значение способны 
активизировать друг друга~--- в~этом сущность ЕЯ. \textit{Значение} 
и~\textit{имя} слова состоят в~таких же отношениях, как информация 
в~ячейке оперативной памяти и~адрес этой ячейки, который хранится 
в~другой ячейке. 
     
     Элементы синтаксиса и~семантики~--- проявления в~физиологии 
человека. Отношения между двумя явлениями: 
\textit{ощущением}-\textit{словом} и~\textit{ощу\-ще\-ни\-ем}-\textit{зна\-ни\-ем}, видимо, имел в~виду 
Ф.~де Соссюр, говоря о~языке как об отображении друг в~друга <<двух 
сущностей>>: \textit{элементов грамматики} (<<план выражения>>) 
и~\textit{элементов семантики} (<<план значения>>).
     
     Чтобы цифровая техника, способная оперировать элементами 
грамматики, стала имитатором языка, требуется в~ней (в~технике) найти 
нечто,\linebreak представляющее собой знание. Например, функции <<плана 
значений>> могли бы сыграть <<нейронные сети>> (электронные схемы), 
если бы ОП состояла из них; но в~современных компьютерах \mbox{организация} 
памяти такова, что слово на экране монитора и~в~памяти компьютера~--- 
сущности одной и~той же природы.
     
     Собеседники, пользуясь только элементами синтаксиса, обмениваются 
знаниями. Такое общение доступно и~двум компьютерам, если они будут 
наделены одной и~той же \textit{моделью мира} (сетью знаний) 
и~идентичными правилами трансформации знаний в~предложения 
и~наоборот.
     
     Знание в~\textit{модели мира} может быть сохранено в~памяти машины 
и~в~форме элементов синтаксиса. Остается придумать технологию хранения, 
подобную той, которая наблюдается в~языковой способности человека, а~не 
в~статьях \textit{толкового словаря}.
     
     Далее излагается один из возможных вариантов построения 
\textit{семантического словаря}, позволяющего хранить большие объемы 
\textit{знаний} в~форме, удобной для программной обработки. 
     
     В статье $W$ словаря хранятся \textit{элементарные знания} о понятии 
$W$ в~виде корневого дерева $T(W)$, в~котором отображены 
\textit{семантические отношения} между~$W$ и~другими понятиями. При 
дереве~$T(W)$ имеется СФ для преобразования $T(W)$ 
в~\textit{элементарное} синтаксически правильное выражение.
     
     Синтаксическую форму, привязанную к~дереву $T(W)$, обозначим через 
     $\mathrm{SF}\,(T(W))$. 
Семантический словарь состоит из пар $\langle T, \mathrm{SF}\,(T)\rangle$, которые 
были описаны в~(\ref{e6-shi}) и~являются предложениями синтаксиса $\mathrm{Sint}$. 
Таким образом, элементы синтаксиса используются для представления 
знания.
     
     На примере \textit{статьи}, посвященной понятию \textit{аист} 
(рис.~2), рассмотрим структуру самой \mbox{статьи} и~оценим, какие возможности 
кроются в~таком словаре для формирования языка.
     

     
     По структуре статья состоит из двух частей: заголовка~(1)  
и~тела~(2)--(11). В~заголовке статьи за именем \textit{понятия} следует его 
\textit{полный код}~--- 040613, в~котором~04 и~0406~--- вложенные друг 
в~друга коды двух предков (\textit{животное} и~\textit{птица}) понятия 
\textit{аист}. За косой чертой указаны имя родительского понятия 
(\textit{птица}) и~код части речи~(01) понятия \textit{аист}.
     
     Тело статьи состоит из пяти разделов: \textit{форма}, \textit{свойство}, 
\textit{элемент}, \textit{событие}, \textit{метод}. В~разделе 
<<\textit{форма}>> перечислены \textit{семантические формы} слова 
\textit{аист}; в~разделе <<\textit{свойство}>>~--- качества данного понятия 
(\textit{цвет}, \textit{вес}, \textit{форма} и~т.\,д.); в~разделе 
<<\textit{элемент}>>~--- составные части, образующие описываемое 
понятие. В~разделе <<событие>> перечислены действия и~состояния, 
которым может подвергаться опи\-сы\-ва\-емое понятие. В~разделе 
<<\textit{метод}>> указаны действия, которые может совершать 
<<\textit{аист}>> над другими понятиями. 
     
     В каждом разделе фиксированы знания, представленные в~виде 
\textit{семантического дерева} в~скобочной форме; корнем для всех деревьев 
служит имя описываемого понятия (в~теле статьи оно не дублируется). 
Деревьям предписаны СФ для преобразования их в~выражения синтаксиса 
$\mathrm{Sint}$.
     
     Например, запись <<{СФ}1: {цвет}({белый})>> может 
обозначать 
\begin{multline*}
\mbox{<<01112131(01112132(02112132)):}\\
\mbox{{аист}({цвета}({белого
}))>>},
\end{multline*}
 т.\,е.\ выражение <<{аист цвета белого}>>.
     
     Далее для построения сложного предложения из \textit{элементарных} 
можно использовать богатый опыт \textit{синтаксической} 
и~\textit{логической семантик}.
    
{\small\frenchspacing
 {%\baselineskip=10.8pt
 \addcontentsline{toc}{section}{References}
 \begin{thebibliography}{9}
\bibitem{1-shi}
Грамматика русского языка~/ Под ред.\ В.\,В.~Виноградова, Е.\,С.~Истриной, 
С.\,Г.~Бархударова.~--- В~2~т.~--- М.: Изд-во Академии наук СССР, 1960. 720~c.
\bibitem{2-shi}
\Au{Шихиев Ф.\,Ш.} Формализация и~сетевая формулировка задачи синтаксического 
анализа: Дис.\ \ldots\ канд. физ.-мат. наук.~--- СПб.: СпбГУ, 2006. 171~с.
\bibitem{3-shi}
\Au{Мирзабеков Я.\,М., Шихиев~Ш.\,Б.} Формальная грамматика русского языка 
в~примерах~// Прикладная дискретная математика, 2018. №\,40. С.~114--126.
\bibitem{4-shi}
\Au{Слобин Д., Грин Дж.} Психолингвистика.~--- М.: Прогресс, 1976. 336~с.
\bibitem{5-shi}
\Au{Тестелец Я.\,Г.} Введение в~общий синтаксис.~--- М.: РГГУ, 2001. 798~с.
\bibitem{6-shi}
\Au{Мельчук И.\,А.} Опыт теории лингвистических моделей Смысл--Текст.~--- М.: 
Языки русской культуры, 1999. 346~с.
\bibitem{7-shi}
Natural language parsing~/ Eds. D.~Dowty, L.~Karttunen, A.~Zwicky.~--- Cambridge: 
Cambridge University Press, 1985. 413~p.
 \end{thebibliography}

 }
 }

\end{multicols}

\vspace*{-12pt}

\hfill{\small\textit{Поступила в~редакцию 25.12.18}}

\vspace*{6pt}

%\pagebreak

%\newpage

%\vspace*{-28pt}

\hrule

\vspace*{2pt}

\hrule

\vspace*{-4pt}

\def\tit{INCAPSULATION OF~SEMANTIC REPRESENTATIONS INTO~ELEMENTS 
OF~A~GRAMMAR}


\def\titkol{Incapsulation of~semantic representations into~elements 
of~a~grammar}

\def\aut{Sh.\,B.~Shihiev and F.\,Sh.~Shihiev}

\def\autkol{Sh.\,B.~Shihiev and F.\,Sh.~Shihiev}

\titel{\tit}{\aut}{\autkol}{\titkol}

\vspace*{-11pt}


\noindent
\noindent
Department of Discrete Mathematics and Computer Science, Dagestan State University,  
43-a~Gadzhiyev Str., Makhachkala 367000, Republic of Dagestan, Russian Federation

\def\leftfootline{\small{\textbf{\thepage}
\hfill INFORMATIKA I EE PRIMENENIYA~--- INFORMATICS AND
APPLICATIONS\ \ \ 2020\ \ \ volume~14\ \ \ issue\ 1}
}%
 \def\rightfootline{\small{INFORMATIKA I EE PRIMENENIYA~---
INFORMATICS AND APPLICATIONS\ \ \ 2020\ \ \ volume~14\ \ \ issue\ 1
\hfill \textbf{\thepage}}}

\vspace*{3pt} 



\Abste{The article proposes a new mathematical apparatus of natural language representation for 
computer linguistics: morphology, syntax, and semantics are described as the objects of discrete 
mathematics forming a~hierarchy and an integral information system. The proposed constructive 
language theory is a new approach to language learning by separating the domains of syntax and 
semantics, constructing the autonomous models of syntax and semantics, language formation as 
the mapping of elements of two sets: syntax and semantics.}

\KWE{natural language; graph; syntax; semantics; lexicon; word form; morphological feature; 
lexical group; dictionary; sentence; algorithm}

\DOI{10.14357/19922264200116} 

%\vspace*{-24pt}

%\Ack
%\noindent



%\vspace*{6pt}

  \begin{multicols}{2}

\renewcommand{\bibname}{\protect\rmfamily References}
%\renewcommand{\bibname}{\large\protect\rm References}

{\small\frenchspacing
 {%\baselineskip=10.8pt
 \addcontentsline{toc}{section}{References}
 \begin{thebibliography}{9}
\bibitem{1-shi-1}
Vinogradov, V.\,V., E.\,S.~Istrina, and S.\,G.~Barkhudarova, eds. 1960. \textit{Grammatika 
russkogo yazyka} [Russian language grammar]. Moscow: USSR Acad. Sci. Publs. 720~p.
\bibitem{2-shi-1}
\Aue{Shihiev, F.\,Sh.} 2006. Formalizatsiya i~setevaya formulirovka zadachi 
sintaksicheskogo analiza [Formalization and network interpretation of 
a~parsing task].  
St.\ Petersburg: St.\ Petersburg State University. PhD Diss. 171~p.
\bibitem{3-shi-1}
\Aue{Mirzabekov, Ya.\,M., and Sh.\,B.~Shihiev.} 2018. Formal'naya grammatika russkogo 
yazyka v~primerakh [Formal grammar of Russian language in examples]. 
\textit{Prikladnaya diskretnaya matematika} [Applied Discrete Mathematics] 40:114--126.
{\looseness=1

}
\bibitem{4-shi-1}
\Aue{Slobin, D., and G.~Green.} 1976. \textit{Psikholingvistika} [Psycholinguistics]. 
Moscow: Progress. 336~p.
\bibitem{5-shi-1}
\Aue{Testelets, Ya.\,G.} 2001. \textit{Vvedenie v~obshchiy sintaksis} [Introduction to 
general syntax]. Moscow: RGGU. 798~p.
\bibitem{6-shi-1}
\Aue{Mel'chuk, I.\,A.} 1999. \textit{Opyt teorii lingvisticheskikh modeley Smysl--Tekst} 
[Experience in the theory of linguistic models Sense--Text]. Moscow: Yazyki russkoy 
kul'tury. 346~p.
\bibitem{7-shi-1}
Dowty, D., L.~Karttunen, and A.~Zwicky, eds. 1985. \textit{Natural language parsing}. 
Cambridge: Cambridge University Press. 413~p.

\end{thebibliography}

 }
 }

\end{multicols}

\vspace*{-6pt}

\hfill{\small\textit{Received December 25, 2018}}

%\pagebreak

%\vspace*{-24pt}

\Contr


\noindent
\textbf{Shihiev Shukur B.} (b.\ 1951)~--- Candidate of Science (PhD) in physics and 
mathematics, associate professor, Department of Discrete Mathematics and Computer Science, 
Dagestan State University, 43-a~Gadzhiyev Str., Makhachkala 367000, Republic of Dagestan, 
Russian Federation; \mbox{sh\_sh\_b51@mail.ru}

\vspace*{6pt}

\noindent
\textbf{Shihiev Fuad B.} (b.\ 1980)~--- Candidate of Science (PhD) in physics and 
mathematics, associate professor, Department of Discrete Mathematics and Computer Science, 
Dagestan State University, 43-a~Gadzhiyev Str., Makhachkala 367000, Republic of Dagestan, 
Russian Federation; \mbox{fuad@mail.ru}

\label{end\stat}

\renewcommand{\bibname}{\protect\rm Литература} 