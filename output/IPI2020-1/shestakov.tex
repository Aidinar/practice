

\newcommand{\mujk}{\mu_{jk}}
\newcommand{\sumk}{\sum\limits_{k=0}^{2^j-1}}
\newcommand{\Prob}{\sf P}


\def\stat{shestakov}

\def\tit{АСИМПТОТИЧЕСКАЯ РЕГУЛЯРНОСТЬ ВЕЙВЛЕТ-МЕТОДОВ
ОБРАЩЕНИЯ ЛИНЕЙНЫХ ОДНОРОДНЫХ ОПЕРАТОРОВ
ПО~НАБЛЮДЕНИЯМ, РЕГИСТРИРУЕМЫМ В СЛУЧАЙНЫЕ
МОМЕНТЫ ВРЕМЕНИ$^*$}

\def\titkol{Асимптотическая регулярность вейвлет-методов
обращения линейных однородных операторов
по~наблюдениям} %, регистрируемым в~случайные моменты времени}

\def\aut{О.\,В.~Шестаков$^1$}

\def\autkol{О.\,В.~Шестаков}

\titel{\tit}{\aut}{\autkol}{\titkol}

\index{Шестаков О.~В.}
\index{Shestakov O.\,V.}


{\renewcommand{\thefootnote}{\fnsymbol{footnote}} \footnotetext[1]
{Работа выполнена при финансовой поддержке Российского научного фонда (проект 18-11-00155).}}


\renewcommand{\thefootnote}{\arabic{footnote}}
\footnotetext[1]{Московский государственный университет им.\ М.\,В.~Ломоносова, 
кафедра математической статистики факультета вычислительной математики и~кибернетики; 
Институт проблем информатики Федерального исследовательского центра <<Информатика и~управление>> 
Российской академии наук, \mbox{oshestakov@cs.msu.ru}}

%\vspace*{-12pt}



\Abst{При решении обратных статистических задач часто приходится обращать некоторый линейный однородный 
оператор, и~обычно бывает необходимо использовать методы регуляризации, поскольку 
наблюдаемые данные, как правило, зашумлены. Популярными методами 
подавления шума являются процедуры пороговой обработки коэффициентов 
разложения наблюдаемой функции по специальному базису. Преимущества 
данных методов заключаются в~их вычислительной эффективности и~возможности 
адаптации как к~виду оператора, так и~к~локальным особенностям оцениваемой функции. 
Анализ погрешностей этих методов представляет собой важную практическую задачу, 
поскольку позволяет оценить качество как самих методов, так и~используемого оборудования. 
Иногда природа данных такова, что регистрация наблюдений проводится 
в~случайные моменты времени. 
Если точки отсчетов образуют вариационный ряд, построенный по выборке из равномерного 
распределения на отрезке регистрации данных, то использование обычных процедур пороговой
 обработки оказывается адекватным. В~данной работе проводится анализ оценки среднеквадратичного 
 риска при обращении линейных однородных операторов и~показывается, что при определенных условиях 
 данная оценка является сильно состоятельной и~асимптотически нормальной.}


\KW{пороговая обработка; линейный однородный оператор; случайные отсчеты; оценка среднеквадратичного риска}

\DOI{10.14357/19922264200101} 
  
\vspace*{-3pt}


\vskip 10pt plus 9pt minus 6pt

\thispagestyle{headings}

\begin{multicols}{2}

\label{st\stat}


\section{Введение}

Во многих прикладных задачах используются математические модели, в~которых предполагается, 
что данные наблюдаются не напрямую, а после некоторого линейного преобразования, 
и~если в~наблюдаемых данных содержится шум, то необходимо применять методы регуляризации. 
Нелинейные методы подавления шума с~помощью вейв\-лет-раз\-ло\-же\-ния и~процедур пороговой обработки приобрели 
значительную популярность. Эти методы хорошо изучены, и~предложены способы нахождения 
оптимальных параметров для различных классов функций, описывающих наблюдаемые данные~\cite{D94, Lee97, AS98, J99}. 
Также изучены статистические свойства оценки среднеквадратичного риска. 
Показано, что при определенных условиях она оказывается сильно состоятельной 
и~асимптотически нормальной~\cite{KS11-1, ESH14-1, EKS15}.

В некоторых ситуациях нет возможности (или она сильно затруднена) регистрировать 
наблюдения через равные промежутки времени~\cite{CB98}. Иногда природа поступающих данных 
такова, что ре\-гист\-ра\-ция отсчетов производится в~случайные моменты времени. 
В~работе~\cite{CB99} показано, что если\linebreak точки отсчетов образуют вариационный ряд, 
построенный по выборке из равномерного распределения на отрезке 
регистрации данных, то при использовании обычной пороговой 
обработки вейвлет-коэффициентов порядок среднеквадратичного 
риска остается с~точностью до логарифмического множителя 
равным оптимальному порядку в~классе функций, регулярных по Липшицу. 

В~данной работе рассматривается статистическая оценка 
среднеквадратичного риска пороговой обработки коэффициентов 
при обращении линейных однородных операторов и~показывается, 
что статистические свойства этой оценки также не меняются 
при переходе от фиксированной равномерной сетки отсчетов 
к~случайной. Оценка остается сильно состоятельной и~асимптотически 
нормальной, т.\,е.\ сохраняет асимптотическую регулярность.

\section{Обращение линейных однородных операторов}

Линейным однородным оператором называется такое линейное преобразование~$K$ искомой функции~$f$, что
$$
K\left[f(a(x-x_0))\right]=a^{-\beta}(Kf)\left[a(x-x_0)\right]
$$
для любого $x_0$ и~любого $a\hm>0$. Параметр~$\beta$ называется показателем однородности. 
Примерами линейных однородных операторов служат оператор интегрирования, 
преобразование Гильберта и~преобразование Абеля. Математические 
модели с~такими операторами используются при решении задач 
вычислительной томографии, физики плазмы, оптики и~др.

Рассмотрим методы обращения оператора~$K$, 
основанные на свойствах вейв\-лет-раз\-ло\-же\-ний~\cite{D94, Lee97, AS98}. 
Преимущество этих методов заключается в~адап\-та\-ции не 
только к~свойствам оператора~$K$, но и~к~свойствам самой искомой функции~$f$.

Вейвлет-разложение функции $f\hm\in L^2(\mathbb{R})$ имеет вид:
%
\begin{equation}                                                                   
\label{waveletdecomp}
f = \sum\limits_{j,k\in Z} \langle f,\psi_{j,k}\rangle  \psi_{j,k},
\end{equation}
%
где $\psi_{j.k}(x)\hm=2^{j/2}\psi(2^jx-k)$, 
а~$\psi(x)$~--- некоторая материнская вейв\-лет-функ\-ция (семейство $\{\psi_{j,k}\}_{j,k\in Z}$ 
образует ортонормированный базис в~$L^2(\mathbb{R})$). Индекс~$j$ 
в~\eqref{waveletdecomp} называется масштабом, а~индекс~$k$~--- сдвигом. 
В~дальнейшем будут рассматриваться функции~$f$ на отрезке $[0,1]$, 
равномерно регулярные по Липшицу с~некоторым показателем $\gamma\hm>\beta$ 
и~константой Липшица $L\hm>0$. Для таких функций известно~\cite{Mall99}, 
что если вейв\-лет-функ\-ция~$M$ раз непрерывно дифференцируема ($M\hm\geqslant\gamma$),
 имеет~$M$ нулевых моментов и~достаточно быстро убывает на бесконечности, 
 т.\,е.\ существует такая константа $C_A\hm>0$, что
$$
\int\limits_{-\infty}^{\infty}\left(1+\abs{t}^{\gamma}\right)
\abs{\psi(t)}dt\leqslant C_A,
$$
то найдется такая константа $A>0$, что
%
\begin{equation}                                                                      
\label{Coeff_Decay}
\abs{\langle f, \psi_{j,k} \rangle} \leqslant \fr{A}{2^{j \left( \gamma + 1/2 \right)}}\,.
\end{equation}

Поскольку оператор~$K$ линеен и~однороден, существуют такие функции~$\xi_{j,k}$, 
что $\langle K f, \xi_{j,k}\rangle \hm= \langle f, \psi_{j,k}\rangle$~\cite{D94}. 
Функции~$\xi_{j,k}$ называются вейглетами. По своим свойствам они похожи на 
вейвлеты и~также представляют собой сдвиги и~растяжения некоторой материнской функции~$\xi$.

Далее пусть $\xi_{j,k}\hm = \lambda_{j,k} u_{j,k}$, где $\lambda_{j,k} \hm= \
\|{(K^*)^{-1} \psi_{j,k}}\|$. Можно показать, что $\lambda_{j,k}\hm=2^{\beta j}\lambda_{0,0}$.
При этом функция~$f$ представляется в~виде ряда
\begin{equation} 
\label{WVD}
f=\sum\limits_{j,k\in Z}\lambda_{j,k}\langle Kf,u_{j,k}\rangle\psi_{j,k}.
\end{equation}
Как видно, в~\eqref{WVD} коэффициенты разложения выражаются через~$Kf$, 
а~не через~$f$. Эта формула лежит в~основе метода обращения~$K$, который называется вейв\-лет-вейг\-лет-раз\-ло\-же\-ни\-ем.


Аналогично по базису вейвлет-фук\-ций можно разложить~$Kf$:
\begin{equation*}
Kf = \sum\limits_{j,k\in Z} \langle Kf,\psi_{j,k}\rangle \psi_{j,k}.
\end{equation*}
Функции $\psi_{j,k}$ не обязаны совпадать с~функциями в~разложении~\eqref{waveletdecomp}, 
но для удобства будем обозначать их так же. Если функции~$Kf$ и~$\psi$ 
удовлетворяют перечисленным выше условиям, то найдется такая константа $C_K\hm>0$, что
\begin{equation}                                                                           
\label{VWD_Coeff_Decay}
\abs{\langle Kf,\psi_{j,k}\rangle} \leqslant \fr{C_K}{2^{j \left(\gamma + 1/2 \right)}}\,.
\end{equation}

Далее через $\mathrm{Lip}(\gamma)$ будем обозначать класс регулярных по 
Липшицу функций, коэффициенты разложения которых удовлетворяют~\eqref{Coeff_Decay} 
или~\eqref{VWD_Coeff_Decay} в~зависимости от используемого метода обращения.

%% у f и~Kf разная гладкость (показатель Липшица отличается на \beta), поэтому скорость убывания коэффициентов в~риске и~его оценке одинаковая

Пусть теперь $\lambda_{j,k} \hm= \norm{K^{-1}\psi_{j,k}}$, тогда $\lambda_{j,k} 
\hm= 2^{\beta j}\lambda_{0,0}$, а функция~$f$ представляется в~виде 
ряда~\cite{AS98}:
\begin{equation} 
\label{VWD}
f = \sum\limits_{j,k\in Z}\lambda_{j,k}\langle Kf,\psi_{j,k}\rangle u_{j,k},
\end{equation}
где $u_{j,k} = K^{-1}\psi_{j,k}/\lambda_{j,k}$. Функции~$u_{j,k}$ не совпадают 
с~функциями в~разложении~\eqref{WVD}, однако по аналогии также называются вейглетами. 
Формула~\eqref{VWD} лежит в~основе еще одного метода обращения, который 
называется вейг\-лет-вейв\-лет-раз\-ло\-же\-нием.

Последовательности $\{u_{j,k}\}$ в~обоих разложениях не образуют ортонормированную систему,
однако если выполнены некоторые условия гладкости, то они образуют
устойчивые базисы~\cite{Lee97, KS11-1}.

\section{Пороговая обработка коэффициентов}

Пусть функция $Kf(x)$ задана на отрезке $[0,1]$. Предположим, что отсчеты~$Kf(x)$ 
регистрируются в~случайные моменты времени и~содержат аддитивный шум, т.\,е.\
 рассмотрим следующую модель данных:
\begin{equation*}
Y_i = K f(x_i) + z_i, \qquad i = 1, \dots, N\enskip \left(N=2^J\right)\,,
\end{equation*}
где $x_i$ независимы и~равномерно распределены на $[0,1]$, а~$z_{i}$~--- 
не зависящие от~$x_i$ и~между собой <<шумовые>> коэффициенты, относительно 
которых предполагается, что они имеют нормальное распределение с~нулевым 
средним и~дисперсией~$\sigma^2$.

Пусть $0\leqslant x_{(1)}<\cdots< x_{(N)}\hm\leqslant 1$~--- 
вариационный ряд, построенный по выборке~$x_i$, $i\hm=1,\ldots,N$. Тогда, перенумеровав~$Y_i$ 
и~$z_i$, получаем модель
\begin{equation}
\label{rand_sample}
Y_i=Kf\left(x_{(i)}\right)+\varepsilon_i,\enskip i=1,\ldots,N\,,
\end{equation}
где $\varepsilon_i$ имеют такую же структуру, как~$z_{i}$.
Наблюдения состоят из пар $(x_{(1)},Y_1),\ldots, (x_{(N)},Y_N)$, 
в~которых расстояния между отсчетами в~общем случае не равны. При этом 
$\e x_{(i)}\hm=i/(N+1)$. Наряду с~\eqref{rand_sample} рассмотрим выборку с~равными 
расстояниями между отсчетами
\begin{equation}
\label{eqspace_sample}
\left(\fr{1}{N+1},Z_1\right),\ldots, \left(\fr{N}{N+1},Z_N\right).
\end{equation}
где
\begin{equation*}
Z_i=Kf\left(\fr{i}{N+1}\right)+\varepsilon_i,\enskip i=1,\ldots,N\,.
\end{equation*}

Применяя к~выборке~\eqref{eqspace_sample} дискретное вейглет- или 
вейв\-лет-пре\-об\-ра\-зо\-ва\-ние~\cite{ESH14-1, EKS15}, можно перейти к~моделям дискретных коэффициентов.

Для метода вейв\-лет-вейг\-лет-раз\-ло\-же\-ния имеем
\begin{equation}                                                                         
\label{WVD_model}
Z^{W}_{j,k} = \mu^{W}_{j,k} +  w_{j,k},
\end{equation}
где $\mu^{W}_{j,k}\approx 2^{J/2}\langle K f, u_{j,k}\rangle$, а шумовые коэффициенты~$w_{j,k}$ 
имеют нормальное распределение с~нулевым средним и~не являются независимыми.\linebreak 
Дисперсии~$\sigma_1^2$ коэффициентов~$w_{j,k}$ зависят от вида оператора и~выбранного вейв\-лет-ба\-зи\-са, 
но не зависят от~$j$ и~$k$~\cite{J99}.

Модель вейглет-вейвлет-ко\-эф\-фи\-ци\-ен\-тов имеет вид:
%
\begin{equation}                                                            
   \label{VWD_model}
Z^{V}_{j,k} = \mu^{V}_{j,k} +   v_{j,k},
\end{equation}
%
где $\mu^{V}_{j,k}\approx 2^{J/2}\langle Kf,\psi_{j,k}\rangle$, 
а~шумовые коэффициенты~$v_{j,k}$ независимы и~имеют нормальное распределение с~нулевым 
средним и~дисперсией $\sigma_2^2\hm=\sigma^2$.

Популярным методом подавления шума является пороговая обработка 
эмпирических коэффициентов. К~коэффициентам в~моделях~\eqref{WVD_model} 
или~\eqref{VWD_model} применяется функция жесткой пороговой 
обработки $\rho_{H}(x,T)\hm =y\mathbf{1}(\abs{x}>T)$ или мягкой пороговой обработки $\rho_{S}(x,T)
\hm=\textbf{sgn}(x)\left(\abs{x}-T\right)_{+}$ с~порогом~$T$. Смысл пороговой 
обработки заключается в~удалении достаточно маленьких коэффициентов, которые считаются шумом.

Далее для сокращения записи будем обозначать через~$W_{j,k}$ 
<<зашумленные>> коэффициенты моделей~\eqref{WVD_model} и~\eqref{VWD_model}, а через~$\mu_{j,k}$~--- 
<<чистые>> коэффициенты этих моделей. Через~$\widehat{W}_{j,k}$ будем обозначать оценки~$\mu_{j,k}$, 
полученные с~помощью пороговой обработки. Также дисперсии шумовых коэффициентов~$\sigma_1^2$ 
и~$\sigma_2^2$ будем обозначать одним символом~$\sigma^2$ (хотя эти дисперсии, вообще говоря, различны).

Если применить дискретное вейглет- или вейв\-лет-пре\-об\-ра\-зо\-ва\-ние к~выборке~\eqref{rand_sample}, 
то получится набор эмпирических коэффициентов
$$
V_{j,k}=\nu_{j,k}+\xi_{j,k},\ j=0,\ldots,J-1,\ k=0,\ldots,2^{j}-1,
$$
где $\xi_{j,k}$ равны $w_{j,k}$ или~$v_{j,k}$ в~зависимости от того, используется 
модель~\eqref{WVD_model} или~\eqref{VWD_model}. Здесь~$\nu_{j,k}$~--- 
коэффициенты дискретного преобразования <<чис\-той>> выборки 
$Kf\left(x_{(1)}\right),\ldots, Kf\left(x_{(N)}\right).$
В~общем случае~$V_{j,k}$ не равны~$W_{j,k}$ и~$\nu_{j,k}$ не равны~$\mu_{j,k}$. Однако к~$V_{j,k}$ 
можно применить ту же процедуру, что и~к~коэффициентам~$W_{j,k}$, 
и~получить оценки~$\widehat{V}_{j,k}$. В~следующих разделах обсуждаются свойства таких оценок.

\vspace*{-4pt}

\section{Среднеквадратичный риск пороговой обработки}

\vspace*{-2pt}

Среднеквадратичный риск пороговой обработки для выборки со случайными точками отсчетов определим как
\begin{equation*}% \label{MSE_Rand}
R_{\nu}(f,T)=\sum\limits_{j=0}^{J-1}\sumk\lambda^2_{j,k}\e(\widehat{V}_{j,k}-\mu_{j,k})^2.
\end{equation*}
Также определим среднеквадратичный риск для выборки с~равными расстояниями между отсчетами:
\begin{equation*}% \label{MSE_Eq}
R_{\mu}(f,T)=\sum\limits_{j=0}^{J-1}\sumk\lambda^2_{j,k}\e(\widehat{W}_{j,k}-\mu_{j,k})^2.
\end{equation*}


Выбор величины порога~--- одна из основных задач при пороговой обработке. 
Для класса~$\mathrm{Lip}(\gamma)$ близким к~оптимальному является порог 
$$
T_\gamma
=\sigma\sqrt{\fr{4\gamma}{{2\gamma+1}}(1+2\beta)\ln 2^J}\,.
$$ 

Используя результаты 
работ~\cite{D94, J99}, можно убедиться, что справедливо следующее утверждение о~порядке~$R_{\mu}(f,T_\gamma)$.

\smallskip

\noindent
\textbf{Теорема~1.}\ \textit{Пусть $Kf\in\mathrm{Lip}(\gamma)$ на 
отрезке $[0,1]$ с~$\gamma\hm>\beta$ 
и~вейв\-лет-функ\-ция удовлетворяет перечисленным выше условиям. Тогда при выборе порога~$T_\gamma$ справедливо}
\begin{equation*}% \label{MSE_Rand_Rate}
R_{\mu}(f,T_\gamma)\leqslant  C\cdot 
2^{J({2\beta+1})/({2\gamma+1})}
J^{({2\gamma+2\beta+2})/({2\gamma+1})},  
\end{equation*}
\textit{где $C$~--- некоторая положительная константа}.

\pagebreak

\smallskip

Также, повторяя рассуждения работы~\cite{CB99}, можно показать, что при $\gamma\hm>\max(\beta,1/2)$ 
аналогичное утверждение справедливо для~$R_{\nu}(f,T_\gamma)$. Таким образом, замена 
равноотстоящих точек отсчетов на случайные не ухудшает оценку порядка 
сред\-не\-квад\-ра\-тич\-но\-го риска. 
% ограничение $\gamma>1/2$ из статьи \cite{CB99}, $\gamma>\beta$, появляется, чтобы сама f была регулярна по Липшицу с~положительным показателем, а не только Kf. При этом нормированный среднеквадратичный риск будет стремиться к~нулю.


\section{Свойства статистической оценки среднеквадратичного риска}

В практических ситуациях вычислить значение среднеквадратичного риска нельзя, 
поскольку оно зависит от ненаблюдаемых <<чистых>> коэффициентов. Однако можно 
построить его оценку, используя только наблюдаемые данные. Эта оценка 
определяется выражением~\cite{Mall99}:
\begin{equation}
\label{MSE_Estimate}
\widehat{R}_{\nu}(f,T)=\sum\limits_{j=0}^{J-1}\sumk \lambda^2_{j,k}F\left[V_{j,k},T\right],
\end{equation}
где 
\begin{multline*}
F\left[V_{j,k},T\right]={}\\
{}=
\begin{cases}
\left(V_{j,k}^2\hm-\sigma^2\right)\mathbf{1}(|V_{j,k}|\leqslant T)
+\sigma^2\mathbf{1}(|V_{j,k}|\hm>T) &\\
& 
\hspace*{-65mm}\mbox{в~случае жесткой пороговой обработки};\\ 
\left(V_{j,k}^2-\sigma^2\right)\mathbf{1}(|V_{j,k}|\leqslant T)+{}\\
\hspace*{30mm}{}+
\left(\sigma^2+T^2\right)\mathbf{1}(|V_{j,k}|>T) &\\ 
&\hspace*{-65mm}\mbox{в~случае мягкой пороговой обработки.}
\end{cases}\hspace*{-12pt}
\end{multline*}

Оценка \eqref{MSE_Estimate} дает возможность 
получить представление о погрешности, 
с~которой оценивается функция~$f$.  
Докажем утверждение о ее асимптотической нормальности.

\smallskip

\noindent
\textbf{Теорема~2.}\ \textit{Пусть $Kf\in\mathrm{Lip}(\gamma)$ на отрезке $[0,1]$ 
с~$\gamma\hm>1/2\hm+\beta$ и~вейв\-лет-функ\-ция удовлетворяет пе\-ре\-чис\-лен\-ным выше условиям. 
Тогда при жесткой и~мягкой пороговых обработках}
 % условие $\gamma>\max(\beta,1/2)$ может быть другим. В лемме об асимптотике дисперсии в~работах с~Ерошенко и~Ерошенко, Кудрявцевым там другие условия. Посмотреть внимательнее, может быть исправить требования на гладкость. Update: условие $Kf\in\mathrm{Lip}(\gamma)$ с~$\gamma>1/2+\beta$ заведомо жестче условий из работ с~Ерошенко и~Кудрявцевым (или такое же в~случае VWD). Напишем его.
\begin{multline*}
%\label{MSE_CLT}
\p\left(\fr{\widehat{R}_{\nu}\left(f,T_\gamma\right)-
R_{\nu}\left(f,T_\gamma\right)}{D_J}<x\right)\to \Phi(x)\\
 \mbox{при}\ J\to\infty, %\notag
\end{multline*}
\textit{где $\Phi(x)$~--- функция распределения стандартного нормального закона. 
При использовании метода вейг\-лет-вейв\-лет-раз\-ло\-же\-ния} 
$$
D_J=\sigma^2\lambda_{0,0}^2
\sqrt{2(2^{4\beta+1}-1)^{-1}}2^{J(1/2+2\beta)}\,,
$$
 \textit{а~при использовании метода вейв\-лет-вейг\-лет-раз\-ло\-же\-ния}
$$
D_J=C_\beta2^{J(1/2+2\beta)}\,,
$$
\textit{где константа~$C_\beta$ имеет более сложную структуру}~\cite{ESH14-1} 
\textit{и~зависит от используемого базиса и~вида оператора~$K$}.

\smallskip

\noindent
Д\,о\,к\,а\,з\,а\,т\,е\,л\,ь\,с\,т\,в\,о\,.\ \
 Докажем теорему для случая жесткой пороговой обработки. 
 В~случае мягкой пороговой обработки доказательство аналогично.

Наряду с~$\widehat{R}_{\nu}(f,T_\gamma)$ рассмотрим
\begin{equation*}
\widehat{R}_{\mu}(f,T_\gamma)=\sum\limits_{j=0}^{J-1}\sumk \lambda^2_{j,k}F[W_{j,k},T_\gamma]
\end{equation*}
и запишем разность $\widehat{R}_{\nu}(f,T_\gamma)\hm-R_{\nu}(f,T_\gamma)$ в~виде
$$
\widehat{R}_{\nu}\left(f,T_\gamma\right)-R_{\nu}\left(f,T_\gamma\right)=
\widehat{R}_{\mu}\left(f,T_\gamma\right)-R_{\mu}\left(f,T_\gamma\right)+\widetilde{R},
$$
где
$$
\widetilde{R}=\widehat{R}_{\nu}\left(f,T_\gamma\right)-\widehat{R}_{\mu}\left(f,T_\gamma\right)-
\left(R_{\nu}(f,T_\gamma)-R_{\mu}\left(f,T_\gamma\right)\!\right).
$$
В~\cite{ESH14-1, SH16-1} показано, что при $\gamma\hm>1/2\hm+\beta$ 
% условие $Kf\in\mathrm{Lip}(\gamma)$ с~$\gamma>1/2+\beta$ заведомо жестче условий из работ с~Ерошенко и~Кудрявцевым (или такое же в~случае VWD).
\begin{multline*}
\p\left(\fr{\widehat{R}_{\mu}(f,T_\gamma)-R_{\mu}\left(f,T_\gamma\right)}{D_J}
<x\right)\to \Phi(x)\\ 
\mbox{при}\ J\to\infty\,.
\end{multline*}
Следовательно, для доказательства теоремы достаточно показать, что
$$
\fr{\widetilde{R}}{2^{J(1/2+2\beta)}}\xrightarrow{\Prob} 0\ \mbox{при}\ J\to\infty\,.
$$
Если $\gamma>\max(\beta,1/2)$, то в~силу теоремы~1 и~аналогичного утверждения для~$R_{\nu}(f,T)$
$$
\fr{R_{\nu}(f,T_\gamma)-R_{\mu}(f,T_\gamma)}{2^{J(1/2+2\beta)}}\to 0\ \mbox{при}\ 
J\to\infty\,.
$$ 
% здесь еще добавляется ограничение $\gamma>(8\beta+2)^{-1}$, которое поглощается ограничением $\gamma>1/2$ при $\beta>0$.
Далее пусть
$$
j_0\approx\fr{J}{2\gamma+1}+\fr{1}{2\gamma+1}\log_2 J\,.
$$
Представим $\widehat{R}_{\nu}(f,T_\gamma)-\widehat{R}_{\mu}(f,T_\gamma)$ в~виде
$$
\widehat{R}_{\nu}(f,T_\gamma)-\widehat{R}_{\mu}(f,T_\gamma)=S_1+S_2,
$$
где
$$
S_1=\sum\limits_{j=0}^{j_0-1}\sumk \lambda^2_{j,k}\left(F\left[V_{j,k},T_\gamma\right]-F\left[W_{j,k},T_\gamma\right]
\right);
$$
$$
S_2=\sum\limits_{j=j_0}^{J-1}\sumk \lambda^2_{j,k}\left(F\left[V_{j,k},T_\gamma\right]-
F\left[W_{j,k},T_\gamma\right]\right).
$$
Поскольку как в~случае жесткой, так и~в~случае мягкой пороговой обработки
\begin{equation}
\left.
\begin{array}{rl}
\abs{F[V_{j,k},T_\gamma]}&\leq T_\gamma^2+\sigma^2,\\[6pt]
\abs{F[W_{j,k},T_\gamma]}&\leq T_\gamma^2+\sigma^2\;\mbox{ п.в.,}
\end{array}
\right\}
\label{Term_Bound}
\end{equation}
то для $\gamma>\max(\beta,1/2)$
$$
\fr{S_1}{2^{J(1/2+2\beta)}}\xrightarrow{\Prob} 0\ \mbox{при}\ J\to\infty.$$
Далее
\begin{multline}
\label{Sums}
S_2=\sum\limits_{j=j_0}^{J-1}\sumk \lambda^2_{j,k}\left(F\left[V_{j,k},
T_\gamma]-F[W_{j,k},T_\gamma\right]\right)={}\\
{}=
\sum\limits_{j=j_0}^{J-1}\sumk\lambda^2_{j,k}(V_{j,k}^2-W_{j,k}^2)+{}\\
{}+\sum\limits_{j=j_0}^{J-1}\sumk\lambda^2_{j,k}(W_{j,k}^2-2\sigma^2)\times{}\\
{}\times 
\mathbf{1}(|V_{j,k}|\leqslant T_\gamma, 
|W_{j,k}|> T_\gamma)+{}\\
{}+\sum\limits_{j=j_0}^{J-1}\sumk\!\lambda^2_{j,k}(2\sigma^2-V_{j,k}^2)\mathbf{1}(|V_{j,k}|> 
T_\gamma, |W_{j,k}|\leqslant T_\gamma)+{}\\
{}
+\sum\limits_{j=j_0}^{J-1}\sumk\lambda^2_{j,k}(W_{j,k}^2-V_{j,k}^2)\times{}\\
{}\times \mathbf{1}(|V_{j,k}|> T_\gamma, |W_{j,k}|> 
T_\gamma).
\end{multline}
Рассмотрим сумму %$\sum\limits_{j=j_0}^{J-1}\sumk\lambda^2_{j,k}(V_{j,k}^2-W_{j,k}^2)$.
\begin{multline*}
\sum\limits_{j=j_0}^{J-1}\sumk\lambda^2_{j,k}(V_{j,k}^2-W_{j,k}^2)={}\\
{}=
\sum\limits_{j=j_0}^{J-1}\sumk\lambda^2_{j,k}(\nu_{j,k}^2-\mu_{j,k}^2)+{}\\
{}+
2\sum\limits_{j=j_0}^{J-1}\sumk\xi_{j,k}\lambda^2_{j,k}(\nu_{j,k}-\mu_{j,k}).
\end{multline*}
Учитывая устойчивость вейг\-лет-ба\-зи\-са и~результаты работ~\cite{CB99, SH19}, можно показать, 
что условное распределение этой суммы при фиксированных~$x_i$ нормально с~математическим ожиданием
$\sum\nolimits_{j=j_0}^{J-1}\sum\nolimits_{k=0}^{2^J-1}
\lambda^2_{j,k}(\nu_{j,k}^2-\mu_{j,k}^2)
$
и~дисперсией, не превосходящей
$C_{\lambda}\sigma^2\sum\nolimits_{j=j_0}^{J-1}
\sum\nolimits_{k=0}^{2J-1}\lambda^4_{j,k}(\nu_{j,k}-\mu_{j,k})^2,
$
где константа~$C_{\lambda}$ зависит от выбранного базиса и~вида оператора~$K$.

Так как $Kf\in\mathrm{Lip}(\gamma)$, то, повторяя рассуждения работы~\cite{SH19}, можно показать, что

\noindent
\begin{multline*}
\e_x\abs{\sum\limits_{j=j_0}^{J-1}\sumk\lambda^2_{j,k}
\left(\nu_{j,k}^2-\mu_{j,k}^2\right)}\leqslant {}\\
{}\leqslant
C_1 \cdot 2^{J(1+2\beta-\min(2,\gamma))}+C_2\cdot 
2^{J\left(1+({2\beta-2\gamma})/({2\gamma+1})\right)},
\end{multline*}
 %% здесь вынести максимальное $2^{2\beta J}$, а дальше как в~статье article50 оценка модуля разности сумм. Второе слагаемое - оценка суммы $\mu_{j,k}^2$.
где $C_1$ и~$C_2$~--- некоторые положительные кон\-станты.

Также, используя оценку из работы~\cite{CB99}, получаем, что
\begin{multline}
\label{VAR_Decay}
\e_x\sum\limits_{j=j_0}^{J-1}\sumk\lambda^4_{j,k}
\left(\nu_{j,k}-\mu_{j,k}\right)^2\leqslant{}\\
{}\leqslant C_3 \cdot 2^{J(1+4\beta-\min(1,\gamma))}, 
%% здесь вынести максимальное $2^{4\beta J}$, а дальше как в~статье article50.
\end{multline}
где $C_3$~--- некоторая положительная константа. Следовательно, если 
$\gamma\hm>\max(\beta,1/2)$, то, применяя неравенство Маркова, получаем, что
$$
\fr{1}{2^{J(1/2+2\beta)}}\sum\limits_{j=j_0}^{J-1}\sumk\lambda^2_{j,k}
\left(\nu_{j,k}^2-\mu_{j,k}^2\right)\xrightarrow{\Prob} 0\,,
$$
$$
\fr{1}{2^{J(1+4\beta)}}\sum\limits_{j=j_0}^{J-1}\sumk\lambda^4_{j,k}(\nu_{j,k}
-\mu_{j,k})^2\xrightarrow{\Prob} 0
$$
при $J\to\infty$. Таким образом,
$$
\fr{\sum\nolimits_{j=j_0}^{J-1}\sum\nolimits_{k=0}^{2^J-1}\lambda^2_{j,k}\left(V_{j,k}^2-W_{j,k}^2\right)}
{2^{J(1/2+2\beta)}}\xrightarrow{\Prob} 0\ \mbox{при}\ J\to\infty\,.
$$


В оставшихся суммах в~\eqref{Sums} содержатся индикаторы, в~которых либо $|V_{j,k}|\hm> T_\gamma$, 
либо $|W_{j,k}|\hm> T_\gamma$, причем в~силу~\eqref{Coeff_Decay} и~\eqref{VWD_Coeff_Decay} для всех 
слагаемых $\abs{\mujk}\hm\leq C_4 J^{-1/2}$ с~некоторой константой $C_4\hm>0$. 
Повторяя рассуждения из работы~\cite{SH10} с~использованием~\eqref{VAR_Decay}, можно показать, 
что эти суммы при делении на $2^{J(1/2+2\beta)}$ также сходятся к~нулю по вероятности. Теорема доказана.

Помимо асимптотической нормальности оценка~\eqref{MSE_Estimate} также обладает 
свойством сильной состоятельности.

\smallskip

\noindent
\textbf{Теорема~3.}  \textit{Пусть выполнены условия теоремы~$2$. 
Тогда при жесткой и~мягкой пороговых обработках для любого} $\alpha\hm>1/2$
\begin{equation*}
\fr{\widehat{R}_{\nu}(f,T_\gamma)-R_{\nu}(f,T_\gamma)}{2^{(\alpha+2\beta) J}}\rightarrow 0 \ \mbox{п.в.\ при } 
J\rightarrow\infty.
\end{equation*}

Поскольку выполнено~\eqref{Term_Bound} и~при фиксированных~$x_i$ слагаемые 
в~\eqref{MSE_Estimate} условно независимы
(или слабо зависимы), доказательство этой теоремы 
аналогично доказательству соответствующего утверж\-де\-ния из работы~\cite{SH16-2}.
% здесь, в~отличие от неслучайных отсчетов, требуется регулярность по Липшицу из-за порядка теоретического риска.


{\small\frenchspacing
 {%\baselineskip=10.8pt
 \addcontentsline{toc}{section}{References}
 \begin{thebibliography}{99}
\bibitem{D94}
\Au{Donoho~D.}  
Nonlinear solution of linear inverse problems by wavelet-vaguelette decomposition~// 
Appl. Comp. Harm. Anal., 1995. Vol.~2. P.~101--126.

\bibitem{Lee97}
\Au{Lee N.} Wavelet-vaguelette decompositions and homogenous equations.~--- 
West Lafayette, IN, USA: Purdue University, 1997. PhD Thesis. 103~p.

\bibitem{AS98}
\Au{Abramovich F., Silverman B. W.} Wavelet decomposition approaches to 
statistical inverse problems~// Biometrika, 1998. Vol.~85. No.~1. P. 115--129.

\bibitem{J99}
\Au{Johnstone I.\,M.} 
Wavelet shrinkage for correlated data and inverse problems adaptivity results~// 
Statist. Sinica, 1999. Vol.~9. P.~51--83.

\bibitem{KS11-1}
\Au{Кудрявцев А.\,А., Шестаков~О.\,В.} 
Асимптотика оценки риска при вейг\-лет-вейв\-лет разложении наблюдаемого сигнала~// T-Comm~--- 
Телекоммуникации и~транспорт, 2011. №\,2. С.~54--57.

\bibitem{ESH14-1}
\Au{Ерошенко А.\,А., Шестаков~О.\,В.} 
Асимптотическая нормальность оценки риска при вейв\-лет-вейг\-лет-раз\-ло\-же\-нии 
функции сигнала в~модели с~коррелированным шумом~// Вестн. Моск. ун-та. Сер.~15: 
Вычисл. матем. и~киберн., 2014. №\,3. C.~23--30.

\bibitem{EKS15}
\Au{Ерошенко А.\,А., Кудрявцев~А.\,А., Шестаков~О.\,В.} 
Предельное распределение оценки риска метода вейг\-лет-вейв\-лет-раз\-ло\-же\-ния 
сигнала в~модели с~коррелированным шумом // Вестн. Моск. ун-та. Сер.~15: Вычисл. матем. и~киберн., 2015. №\,1. C.~12--18.

\bibitem{CB98}
\Au{Cai T., Brown~L.} Wavelet shrinkage for nonequispaced samples~// 
Ann. Stat., 1998. Vol.~26. No.\,5. P.~1783--1799.

\bibitem{CB99}
\Au{Cai T., Brown~L.} Wavelet estimation for samples with random uniform 
design~// Stat. Probabil. Lett., 1999. 
Vol.~42. P.~313--321.

\bibitem{Mall99}
\Au{Mallat S.} A~wavelet tour of signal processing.~--- New York, NY, USA: Academic Press, 1999. 857~p.

\bibitem{SH16-1}
\Au{Шестаков О.\,В.} Ве\-ро\-ят\-ност\-но-ста\-ти\-сти\-че\-ские 
методы анализа и~обработки сигналов на основе вейв\-лет-ал\-го\-рит\-мов.~--- М.:~АРГАМАК-МЕДИА, 2016. 200~с.

\bibitem{SH19}
\Au{Шестаков О.\,В.} Свойства вейв\-лет-оце\-нок сигналов, регистрируемых в~случайные моменты времени~// 
Информатика и~её применения, 2019. Т.~13. Вып.~2. С. 30--35.

\bibitem{SH10}
\Au{Шестаков О.\,В.} Аппроксимация распределения оценки риска пороговой обработки 
вейв\-лет-ко\-эф\-фи\-ци\-ен\-тов нормальным распределением при использовании выборочной дисперсии~// Информатика 
и~её применения, 2010. Т.~4. Вып.~4. С.~73--81.

\bibitem{SH16-2}
\Au{Shestakov O.\,V.} On the strong consistency of the adaptive risk estimator for wavelet thresholding~// 
J.~Math. Sci., 2016. Vol.~214. No.\,1. P.~115--118.
 \end{thebibliography}

 }
 }

\end{multicols}

\vspace*{-3pt}

\hfill{\small\textit{Поступила в~редакцию 16.12.19}}

\vspace*{8pt}

%\pagebreak

%\newpage

%\vspace*{-28pt}

\hrule

\vspace*{2pt}

\hrule

%\vspace*{-2pt}

\def\tit{ASYMPTOTIC REGULARITY OF~THE~WAVELET METHODS OF~INVERTING LINEAR HOMOGENEOUS
 OPERATORS FROM~OBSERVATIONS RECORDED AT~RANDOM TIMES}


\def\titkol{Asymptotic regularity of~the~wavelet methods of~inverting linear homogeneous
 operators from~observations %recorded 
 at~random times}

\def\aut{O.\,V.~Shestakov$^{1,2}$}

\def\autkol{O.\,V.~Shestakov}

\titel{\tit}{\aut}{\autkol}{\titkol}

\vspace*{-11pt}


\noindent
$^1$Department of Mathematical Statistics, Faculty of Computational Mathematics and Cybernetics, 
M.\,V.~Lomo-\linebreak
$\hphantom{^1}$nosov Moscow State University, 1-52~Leninskiye Gory, GSP-1, Moscow 119991, Russian Federation

\noindent
$^2$Institute of Informatics Problems, Federal Research Center ``Computer Science and Control''
 of the Russian\linebreak
 $\hphantom{^1}$Academy of Sciences, 44-2~Vavilov Str., Moscow 119333, Russian Federation

\def\leftfootline{\small{\textbf{\thepage}
\hfill INFORMATIKA I EE PRIMENENIYA~--- INFORMATICS AND
APPLICATIONS\ \ \ 2020\ \ \ volume~14\ \ \ issue\ 1}
}%
 \def\rightfootline{\small{INFORMATIKA I EE PRIMENENIYA~---
INFORMATICS AND APPLICATIONS\ \ \ 2020\ \ \ volume~14\ \ \ issue\ 1
\hfill \textbf{\thepage}}}

\vspace*{3pt} 







\Abste{When solving inverse statistical problems, it is often necessary to invert some linear 
homogeneous operator and it is usually necessary to use regularization methods, since 
the observed data are noisy. Popular methods for noise suppression are 
the procedures 
of thresholding the expansion coefficients of the observed function. The advantages 
of these methods are their computational efficiency and the ability to adapt to both the 
type of operator and the local features of the estimated function. An analysis of 
the errors of these methods is an important practical task, since it allows one to 
evaluate the quality of both the methods themselves and the equipment used. 
Sometimes, the nature of the data is such that observations are recorded at random times. 
If the observation points form a~variational series constructed from a~sample of a~uniform distribution 
on the data recording interval, then the use of conventional threshold processing procedures is 
adequate. The present author analyzes the estimate of the mean square risk in the problem 
of inversion of linear homogeneous operators and demonstrates that under 
certain conditions, 
this estimate is strongly consistent and asymptotically normal.}

\KWE{threshold processing; linear homogeneous operator; random observation points; mean square risk estimate}




\DOI{10.14357/19922264200101} 

%\vspace*{-14pt}

\Ack
\noindent
This research is supported by the Russian Science Foundation (project No.\,18-11-00155).
 


%\vspace*{6pt}

  \begin{multicols}{2}

\renewcommand{\bibname}{\protect\rmfamily References}
%\renewcommand{\bibname}{\large\protect\rm References}

{\small\frenchspacing
 {%\baselineskip=10.8pt
 \addcontentsline{toc}{section}{References}
 \begin{thebibliography}{99}
 

  \bibitem{1-sh-1}
\Aue{Donoho, D.} 1995. Nonlinear solution of linear inverse problems by wavelet-vaguelette decomposition.
\textit{Appl. Comp. Harm. Anal.} 2:101--126.

\bibitem{2-sh-1}
\Aue{Lee, N.} 1997. Wavelet-vaguelette decompositions and 
homogenous equations. West Lafayette, IN: Purdue University.  PhD Thesis. 103~p.

\bibitem{3-sh-1}
\Aue{Abramovich, F., and B.\,W.~Silverman.}
 1998. Wavelet decomposition approaches to statistical inverse problems. \textit{Biometrika} 85(1):115--129.

\bibitem{4-sh-1}
\Aue{Johnstone, I.\,M.}
 1999. Wavelet shrinkage for correlated data and inverse problems adaptivity results. 
  \textit{Statist. Sinica} 9:51--83.

\bibitem{5-sh-1}
\Aue{Kudryavtsev, A.\,A., and O.\,V.~Shestakov.}
 2011. Asimp\-to\-ti\-ka otsen\-ki riska pri veyglet-veyvlet razlozhenii 
 nablyu\-da\-emo\-go signala 
 [The average risk assessment of the wavelet decomposition of the signal]. 
  \textit{T-Comm~--- Telekommunikatsii i~transport} [T-Comm~--- Telecommunications and Transport] 2:54--57.

\bibitem{6-sh-1}
\Aue{Eroshenko, A.\,A., and O.\,V.~Shestakov.}
 2014. Asymptotic normality of estimating risk upon the wavelet-vaguelette decomposition of 
 a~signal function in a~model with correlated noise. 
  \textit{Mosc. Univ. Comput. Math. Cybern.} 38(3):110--117.

\bibitem{7-sh-1}
\Aue{Eroshenko, A.\,A., A.\,A.~Kudryavtsev, and O.\,V.~Shestakov.}
 2015. Limit distribution of a~risk estimate using the vaguelette-wavelet decomposition of signals in 
 a~model with correlated noise.  \textit{Mosc. Univ. Comput. Math. Cybern.} 39(1):6--13.

\bibitem{8-sh-1}
\Aue{Cai, T., and L.~Brown.} 1998. Wavelet shrinkage for nonequispaced samples. 
 \textit{Ann. Stat.} 26(5):1783--1799.

\bibitem{9-sh-1}
\Aue{Cai, T., and L.~Brown.} 1999. Wavelet estimation for samples with random uniform design. 
 \textit{Stat. Probabil. Lett.} 42:313--321.

\bibitem{10-sh-1}
\Aue{Mallat, S.} 1999.  \textit{A~wavelet tour of signal processing.} New York, NY: Academic Press. 857~p.

\bibitem{11-sh-1}
\Aue{Shestakov, O.\,V.} 2016.  \textit{Veroyatnostno-statisticheskie 
metody analiza 
i~obrabotki signalov na osnove veyvlet-algoritmov} [Probabilistic-statistical methods 
of signal analysis and processing based on wavelet algorithms]. Moscow: 
Argamak-Media Publs. 200~p.

\bibitem{12-sh-1}
\Aue{Shestakov, O.\,V.} 2019. Svoystva veyvlet-otsenok signalov,
 registriruemykh 
v~sluchaynye momenty vremeni [Properties of wavelet estimates of signals recorded at random time points]. 
 \textit{Informatika i~ee Primeneniya~--- Inform. Appl.} 13(2):30--35.

\bibitem{13-sh-1}
\Aue{Shestakov, O.\,V.} 2010. Approksimatsiya raspredeleniya otsen\-ki riska porogovoy obrabotki 
veyvlet-koeffitsientov normal'nym raspredeleniem pri ispol'zovanii vyborochnoy dispersii 
[Normal approximation for distribution of risk estimate for wavelet coefficients thresholding 
when using sample variance].  \textit{Informatika i~ee Primeneniya~--- Inform. Appl.} 4(4):73--81.

\bibitem{14-sh-1}
\Aue{Shestakov, O.\,V.}
 2016. On the strong consistency of the adaptive risk estimator for wavelet thresholding.
 \textit{J.~Math. Sci.} 214(1):115--118.
 \end{thebibliography}

 }
 }

\end{multicols}

%\vspace*{-7pt}

\hfill{\small\textit{Received December 16, 2019}}

%\pagebreak

%\vspace*{-22pt}


\Contrl

\noindent
\textbf{Shestakov Oleg V.} (b.\ 1976)~--- 
Doctor of Science in physics and mathematics, professor, Department of 
Mathematical Statistics, Faculty of Computational Mathematics and Cybernetics, 
M.\,V.~Lomonosov Moscow State University, 1-52~Leninskiye Gory, GSP-1, Moscow 119991, 
Russian Federation; senior scientist, 
Institute of Informatics Problems, Federal Research Center ``Computer Science and Control'' 
of the Russian Academy of Sciences, 44-2~Vavilov Str., Moscow 119333, Russian Federation; \mbox{oshestakov@cs.msu.su}


\label{end\stat}

\renewcommand{\bibname}{\protect\rm Литература}  
   