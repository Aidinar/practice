\def\stat{grusho}

\def\tit{О~КАУЗАЛЬНОЙ РЕПРЕЗЕНТАТИВНОСТИ\\ ОБУЧАЮЩИХ ВЫБОРОК ПРЕЦЕДЕНТОВ\\ В~ЗАДАЧАХ 
ДИАГНОСТИЧЕСКОГО ТИПА$^*$}

\def\titkol{О~каузальной репрезентативности обучающих выборок прецедентов в~задачах 
диагностического типа}

\def\aut{А.\,А.~Грушо$^1$, М.\,И.~Забежайло$^2$, Е.\,Е.~Тимонина$^3$}

\def\autkol{А.\,А.~Грушо, М.\,И.~Забежайло, Е.\,Е.~Тимонина}

\titel{\tit}{\aut}{\autkol}{\titkol}

\index{Грушо А.\,А.}
\index{Забежайло М.\,И.}
\index{Тимонина Е.\,Е.}
\index{Grusho A.\,A.}
\index{Zabezhailo M.\,I.}
\index{Timonina E.\,E.}


{\renewcommand{\thefootnote}{\fnsymbol{footnote}} \footnotetext[1]
{Работа частично поддержана РФФИ (проект 18-29-03081).}}


\renewcommand{\thefootnote}{\arabic{footnote}}
\footnotetext[1]{Институт проблем информатики Федерального исследовательского центра <<Информатика и~управление>> 
Российской академии наук, \mbox{grusho@yandex.ru}}
\footnotetext[2]{Вычислительный центр им.\ А.\,А.~Дородницына Федерального исследовательского центра <<Информатика 
и~управ\-ле\-ние>> Российской академии наук, \mbox{m.zabezhailo@yandex.ru}}
\footnotetext[3]{Институт проблем информатики Федерального исследовательского центра <<Информатика и~управ\-ле\-ние>> 
Российской академии наук, \mbox{eltimon@yandex.ru}}

\vspace*{-12pt}



  
  
  \Abst{Работа посвящена некоторым особенностям анализа причинности в~задачах 
интеллектуального анализа данных. Обсуждаются возможности использования так называемых 
открытых логических теорий в~задачах диагностики (классификации) для описания 
пополняемых наборов эмпирических данных. В~задачах этого типа необходимо установить 
(спрогнозировать, диагностировать и~др.)\ наличие или отсутствие целевого свойства у нового 
прецедента, заданного описанием на том же языке представления гетерогенных данных, 
которым описаны примеры, обладающие целевым свойством, и~контрпримеры, не обладающие 
целевым свойством. Представлен вариант построения открытых теорий, опи\-сы\-ва\-ющих 
коллекции прецедентов средствами специальных логических выражений~--- 
характеристических функций (ХФ). Характеристические функции позволяют избавиться от 
гетерогенности в~описаниях прецедентов. Предложена процедурная конструкция формирования 
ХФ обучающей выборки прецедентов. Исследованы свойства 
ХФ и~некоторые условия их существования.}
   
  \KW{диагностика; каузальный анализ; интеллектуальный анализ данных; открытые 
логические теории} 

\DOI{10.14357/19922264200111} 
  
\vspace*{-3pt}


\vskip 10pt plus 9pt minus 6pt

\thispagestyle{headings}

\begin{multicols}{2}

\label{st\stat}

\section{Введение }

  Работа посвящена некоторым особенностям анализа причинности в~задачах 
диагностического типа. К~этому классу традиционно относят такие задачи, 
в~которых исходные данные пред\-став\-ле\-ны описаниями двух типов  
прецедентов~--- тех, которые обладают заданным целевым свойством (будем 
называть их \textit{примерами} наличия анализируемого эффекта), и~тех, которые, 
будучи <<похожими>> на примеры, тем не менее таковым свойством не обладают 
(будем называть их \textit{контрпримерами}). В~задачах этого типа необходимо 
установить (спрогнозировать, диагностировать и~др.)\ наличие или отсутствие 
целевого свойства у нового (предлагаемого для формирования <<диагноза>>) 
прецедента, заданного описанием на том же языке представления  
дан\-ных/зна\-ний, которым описаны примеры и~контрпримеры.
  
  По-видимому, исторически первые примеры подобных проблем сформировала 
медицина, однако сегодня предлагаемый перечень содержит и~такие не менее 
востребованные предметные области, как:
  \begin{itemize}
\item техническая диагностика~[1] (рассматриваемая в~широком диапазоне от 
диагностики отказов и~поддержания работоспособности сложного 
оборудования~--- транспортного, энергетического, нефтегазового и~т.\,п.~--- до 
обеспечения устойчивой работы крупных центров обработки данных, 
компьютерных сетей, телекоммуникационной инфраструктуры и~др.);
\item финансовый мониторинг и~противодействие мошенничеству 
в~финансовой сфере~[2];
\item информационная безопасность (идентификация компьютерных атак 
и~организация противодействия целенаправленным вредоносным 
воздействиям в~сфере ин\-фор\-ма\-ци\-он\-но-ком\-му\-ни\-ка\-ци\-он\-ных 
технологий)~[3, 4];
\item экологический мониторинг (в~частности, выявление потенциально 
опасных химических соединений в~окружающей человека среде)~[5, 6];
\item социологический анализ (типология социума и~анализ рациональных 
оснований принятия решений различными социальными группами)~[7] и~др. 
\end{itemize}

  Соответствующие подходы, математические модели, методы и~алгоритмы 
формируют в~этих областях своего рода сквозную технологию компьютерного 
анализа данных и~поддержки принятия управленческих решений.
  
  Наиболее распространенными в~рас\-смат\-ри\-ва\-емой области анализа данных 
и~поддержки принятия решений оказались так называемые  
\textit{ин\-тер\-по\-ля\-ци\-он\-но-экстра\-по\-ля\-ци\-он\-ные} математические 
\mbox{техники} решения задач диагностического типа: име\-юща\-яся <<обучающая>> 
выборка описаний прецедентов (примеров и~контрпримеров) интерполируется той 
или иной сис\-те\-мой эмпирических зависимостей (ЭЗ) (регрессионных, логических 
и~т.\,п.)\ так, чтобы <<диагностика>> нового прецедента могла быть сведена 
к~проверке корректной экстраполяции ка\-ких-ли\-бо из уже найденных ЭЗ на 
описание этого нового прецедента. 
  
  Критически значимым аспектом, характерным для задач этого типа, 
оказывается проблема анализа причинности (идентификации каузальных 
оснований) формируемых диагностических заключений. С~управленческой точки 
зрения существенно иметь возможности выделять именно причинные факторы 
влияния, <<вынуждающие>> появление исследуемого целевого эффекта. Это 
позволяет, оказывая влияние именно на причины, целенаправленно 
воздействовать на ситуацию и~объект управления, чтобы не допустить 
возникновения негативных последствий.
  
  При изучении эффектов причинности (каузальных <<влияний>>) путем анализа 
эмпирического материала важно принять во внимание ряд фундаментальных 
обстоятельств, к~наиболее критически значимым из которых следует отнести:
  \begin{itemize}
\item отсутствие ясных, корректных и~точных представлений о~<<структуре>> 
искомых причинных <<влияний>>, в~частности об исчерпывающем перечне 
факторов <<влияния>> и~полном комплексе взаимосвязей между ними; 
\item  отсутствие достаточно надежных оснований считать используемый язык 
представления анализируемых эмпирических данных \textit{каузально 
полным}, т.\,е.\ адекватно представляющим как все факторы причинного 
влияния, так и~все взаимосвязи между ними~[8].
\end{itemize}

  В подобной ситуации говорить об анализе собственно причин в~задачах 
рассматриваемого типа не совсем корректно. Более естественным представляется 
ограничиться лишь эмпирическими <<причинами>>~--- описаниями каузальных 
влияний, которые могут быть выделены (найдены, восстановлены из данных 
и~т.\,п.)\ в~име\-ющей\-ся <<обучающей>> выборке прецедентов 
и~задействованном для описания накапливаемого эмпирического материала языке 
представления данных.
  
  Далее в~этой работе, упоминая о причинности и~каузальных влияниях, будем 
иметь в~виду именно такое, учитывающее специфику анализа эмпирических 
данных, уточнение представлений о причинности.
  
  Принимая во внимание особую значимость каузальных факторов <<влияния>>, 
характеризующих в~каждом конкретном случае наличие изучаемого целевого 
эффекта, и~дополнительно отмечая критически важную роль идентификации таких 
факторов для организации целенаправленного противодействия негативным 
последствиям их влияния, более подробно рассмотрим возможности их выявления с~помощью подходящих средств компьютерного анализа данных. Результативной 
здесь оказывается следующая очевидная эвристика: так как необходимые факторы 
влияния должны отражать <<каузальность>> во всех описаниях прецедентов 
наличия анализируемого целевого эффекта (примеров), то, как следствие, 
значимые для возникновения исследуемого целевого эффекта комбинации таких 
факторов должны отображаться сходствами описаний примеров~[9, 10]. При этом 
естественно потребовать, чтобы выделяемые таким образом значимые 
комбинации факторов влияния не <<встречались>> ни в~одном из описаний 
контрпримеров. Случаи, когда <<причина>> имеется, а эффекта~--- нет, должны 
быть использованы для удаления артефактов из формируемого на примерах 
множества значимых комбинаций факторов целенаправленного каузального 
влияния.
  
  Располагая <<непротиворечивыми>>, харак\-тер\-ными только для примеров 
значимыми комбинациями факторов влияния, можно строить соответствующие 
логические условия. Будем называть их\linebreak характеристическими функциями  
текущей базы фактов (БФ). Характеристические функции охватывают имеющееся на текущий момент 
множество прецедентов (примеров и~контрпримеров), каждое из которых истинно 
на всех примерах и~ложно на всех контрпримерах из текущей БФ. Непустота 
множества ХФ, порождаемых на текущей выборке прецедентов БФ, может 
рассматриваться как характеристика ее каузальной репрезентативности, т.\,е.\ 
корректной <<разделимости>> примеров и~контрпримеров логическими 
условиями каузального характера.
  
  Учитывая, что <<природа>> диагностических задач\linebreak требует возможностей 
оперировать расширя\-ющимися за счет новых описаний прецедентов коллекциями 
эмпирических данных, естественно\linebreak рас\-смат\-ри\-вать соответствующую 
<<динамику>> накапливания и~изменений анализируемого эмпирического 
материала, т.\,е.\ динамику изменений каж\-до\-го конкретного множества 
соответствующих ХФ. Особый интерес вызывают такие подсемейства ХФ, 
которые <<наследуются>>, т.\,е.\ сохраняют корректность описания конкретной 
БФ при ее конкретном расширении. С~математической точки зрения интерес 
представляет проб\-ле\-ма эффективного выделения именно таких подсемейств ХФ. 
В~данном случае речь идет о~проб\-ле\-ме эффективного поиска ЭЗ, 
описывающих природу диагностируемого целевого эффекта.
  
  \section{Математическая модель формирования факторов каузального 
влияния}
   
  Пусть заданы два множества:
  \begin{align*}
  \boldsymbol{U}&= \left\{a_1, a_2,\ldots ,a_n\right\};\\ 
\boldsymbol{\Omega}&= \left\{O_1, O_2,\ldots , O_m\right\}\subseteq 
2^U\backslash\phi\,,
\end{align*}
 первое из которых будем называть исходным алфавитом, или 
множеством образующих элементов для описания анализируемых прецедентов 
из~$\Omega$, а~второе~--- множеством собственно описаний прецедентов, 
построенных над универсумом~$U$. 
  
  \textit{Примерами} будем называть такие прецеденты $O_1^+, O_2^+,\ldots , 
O^+_{m^+}$ из множества~$\boldsymbol{\Omega}$, которые обладают 
исследуемым целевым свойством~$P$. Обозначим это подмножество 
множества~$\boldsymbol{\Omega}$ через~$\boldsymbol{\Omega}^+$. 
\textit{Контрпримерами} будем называть такие прецеденты $O_1^-, O_2^-, \ldots , 
O^-_{m^-}$ из множества~$\boldsymbol{\Omega}$, которые не обладают этим 
целевым свойством~$P$. Обозначим это подмножество 
множества~$\boldsymbol{\Omega}$ через~$\boldsymbol{\Omega}^-$. Таким 
образом, множество~$\boldsymbol{\Omega}$ будет представлять описание 
текущего состояния~БФ. 
  
  Для операции $\otimes$ сходства описаний прецедентов из БФ используем 
операцию~$\cap$ пересечения множеств образующих, формирующих описание 
каждого прецедента из~$\boldsymbol{\Omega}$, т.\,е.\ результатом ее применения 
будет множество попарно совпадающих значений признаков из множества 
образующих~$\boldsymbol{U}$. Несложно убедиться, что такое определение 
операции сходства корректно, т.\,е.\ по непустой операции это отношение 
рефлексивно, симметрично и~ассоциативно. 
  
  Пусть $\boldsymbol{\Omega}\hm = \{O_1, O_2, \ldots , O_m\}$. Построим 
по~$\boldsymbol{\Omega}$ и~$\otimes$ множество $\mathrm{Dom}\,(\boldsymbol{\Omega})$ 
следующим образом: 
  \begin{itemize}
\item[(а)] $\boldsymbol{\Omega} \hm= \left\{ O_1, O_2, \ldots , 
O_m\right\}\hm\subset \mathrm{Dom}\,(\boldsymbol{\Omega})$;
\item[(б)] $\{ [A\in \mathrm{Dom}\,(\boldsymbol{\Omega})] \& [B\hm\in 
\mathrm{Dom}\,(\boldsymbol{\Omega})] \& [(A \otimes B) \hm\not=$\linebreak $\not= \phi]\}\to [ (A\otimes  
B)\hm\in \mathrm{Dom}\,(\boldsymbol{\Omega})]$;
\item[(в)] других элементов в~множестве $\mathrm{Dom}\,(\boldsymbol{\Omega})$нет.
\end{itemize}
  
  Аналогичным образом определяются множества 
$\mathrm{Dom}\,(\boldsymbol{\Omega}^+)$ и~$\mathrm{Dom}\,(\boldsymbol{\Omega}^-)$. 
  
  Фиксируя каждый конкретный, непустой результат $V\hm=V_0$ вычисления 
операции~$\otimes$ сходства на элементах множества 
$\mathrm{Dom}\,(\boldsymbol{\Omega})$, где $A\otimes B\hm= V_0$, можно выделять 
соответствующие классы сходства~$\mathbf{E}^{\otimes}_{V_0}$: 
  $$
  \mathbf{E}^{\otimes}_{V_0}=\left\{ \langle O_{i_1}, O_{i_2} \rangle \vert 
O_{i_1} \otimes O_{i_2}\otimes V_0=V_0\right\}\,.
  $$
  
   
  
  Таким образом, фиксируя конкретное значение параметра сходства~$V_0$, 
можно определить множество всех прецедентов~$\mathbf{E}_{V_0}$, которые 
содержат\linebreak множество признаков~$V_0$. Тогда можно разбить исходное множество 
прецедентов~$\boldsymbol{\Omega}$ на два непересекающихся подмножества 
$\mathbf{E}_{V_0}$ и~$\boldsymbol{\Omega}\backslash \mathbf{E}_{V_0}$: 
$\mathbf{E}_{V_0}\cap \boldsymbol{\Omega}\backslash 
\mathbf{E}_{V_0}\hm=\phi$.
  %
  При этом надо рассматривать каж\-дый раз при таком разделении 
множества~$\boldsymbol{\Omega}$ на два не\-пе\-ре\-се\-ка\-ющих\-ся подмножества 
с~по\-мощью соответствующего~$V_0$ все содержащие общую часть~$V_0$ 
прецеденты. 
  
  Для каждого сформированного таким образом класса сходства примеров 
должно дополнительно\linebreak выполняться логическое условие, называемое 
\textit{запретом на контрпримеры} (ЗКП)~\cite{9-gr}. В~случае использования 
ЗКП запрещается вложимость\linebreak {примеров} одного <<знака>> в~ка\-кой-ли\-бо из 
сформированных классов сходства противоположного <<знака>>.
  
  Условие ЗКП для примеров можно представить в~следующем виде:
    \begin{multline*}
  \forall\ V \left\{ \left[ \left(V\in \mathrm{Dom}\,
  \left(\boldsymbol{\Omega}^+\right)\right)\& (V\not= \phi) \right]
\&\right.\\
\&
\left[ \exists\,O_r^+, \exists\,O_s^+  \left( O_r^+\otimes O_s^+\otimes 
V=V\right)\right]\to\\
  \left. {}\to \neg \left( \exists\,O_p^- \left[
  \left(O_p^-\in (\boldsymbol{\Omega}^-)\right)\& 
  \left(O_p^-
\otimes V=V\right)\right]\right)\right\}\,.
  \end{multline*}
% 



\noindent 
  При этом условие ЗКП для контрпримеров может быть формализовано 
симметричным образом заменой~$\boldsymbol{\Omega}^+$ 
на~$\boldsymbol{\Omega}^-$, примеров~$O_r^+$ и~$O_s^+$~--- на контрпримеры 
$O_r^-$ и~$O_s^-$, а~вместе с~ними контрпримера~$O_p^-$ на пример~$O_p^+$. 
  
  Исходя из того принципа, что для появления свойства~$P$ должна быть 
причина, при выполнении условия ЗКП можно ожидать, что причина содержится 
в~примерах~$\boldsymbol{\Omega}^+$. Однако неизвестны характеристики 
причины, а~именно: является ли причина единственной и~какова структура 
причины. Из упомянутого выше принципа сходства можно искать причины 
свойства~$P$ в~по\-рож\-да\-ющих элементах~$V$ классов сходства. Но в~силу 
указанной неопределенности о~всех порождающих классы сходства элементах 
необходимо говорить как о~возможных факторах влияния на прогноз появления 
свойства~$P$, или факторах эмпирической причинности свойства~$P$. При этом 
сами порождающие элементы классов сходства можно называть ЭЗ. Аналогично 
можно определять ЭЗ на множестве~$\boldsymbol{\Omega}^-$. 
  
  Итак, чтобы ожидать, что новый предложенный для прогноза свойств 
прецедент~$O_0$ также имеет свойство~$P$, в~обсуждаемой схеме рассуждений 
следует убедиться, что данный~$O_0$ <<попадает>> хотя бы в~один из классов 
сходства, сформированных на исходном множестве 
примеров~$\boldsymbol{\Omega}^+$.
  
  Легко видеть, что покрытие всего исходного\linebreak множества 
прецедентов~$\boldsymbol{\Omega}^+$ классами сходства  
пре\-це\-ден\-тов-при\-ме\-ров, порождаемыми на его примерах с~непременным 
выполнением условия ЗКП,\linebreak позволяет разделить с~по\-мощью выделяемых 
комбинаций значений факторов <<причинности>> примеры и~контрпримеры 
в~текущем множестве~$\boldsymbol{\Omega}$ (текущей БФ). 
  
  Ясно, что один класс сходства $\mathbf{E}_V$ может не покрывать 
множество~$\boldsymbol{\Omega}^+$. Однако несколько классов сходства 
$\mathbf{E}_{V_1},\ldots , \mathbf{E}_{V_k}$ могут покрыть все 
множество~$\boldsymbol{\Omega}^+$. Такое покрытие, если оно существует, 
может не быть единственным. Из условия ЗКП для любого покрытия 
множества~$\boldsymbol{\Omega}^+$ классами сходства 
$\mathbf{E}_{V_1},\ldots , \mathbf{E}_{V_k}$ порождающие элементы 
$V_1,\ldots , V_k$ не могут встречаться в~прецедентах из 
множества~$\boldsymbol{\Omega}^-$. 
  
  Для покрытия $\mathbf{E}_{V_1},\ldots , \mathbf{E}_{V_k}$ определим 
бинарные переменные следующим образом:
  $$
  x_{ij}=\begin{cases}
  1\,, & \mbox{если\ характеристика } a_i\in V_j\,;\\
  0 & \mbox{в~противном случае}.
  \end{cases}
  $$
  Тогда с~множеством~$V_j$ взаимно однозначно связана конъюнкция 
$\mathop{\wedge}\limits_{a_i\in V_j} x_{ij}$.
  
  \smallskip
  
  \noindent
  \textbf{Определение.} Характеристической функцией 
покрытия~$\boldsymbol{\Omega}^+$ классами сходства $\mathbf{E}_{V_1},\ldots 
, \mathbf{E}_{V_k}$ называется двоичная функция  
$\mathop{\vee}\limits^k_{j=1}\mathop{\wedge}\limits_{a_i\in V_j} x_{ij}$.
  
  Если возможно построить~$s$~покрытий классами сходства 
множества~$\boldsymbol{\Omega}^+$, то существует~$s$~ХФ. 
  
  \smallskip
  
  \noindent
  \textbf{Утверждение.} Каждая построенная в~соответствии с~определением 
функция~$\mathrm{ХФ}$ принимает на факте~$\varphi$ из текущей БФ значение~1 (истина) 
тогда и~только тогда, когда данный факт характеризуется наличием 
анализируемого целевого свойства~$P$, и~0 (ложь) тогда и~только тогда, когда 
данный факт характеризуется отсутствием анализируемого целевого 
свойства~$P$.
  
  \smallskip
  
  \noindent
  Д\,о\,к\,а\,з\,а\,т\,е\,л\,ь\,с\,т\,в\,о\,.\ \ \textit{Необходимость}. Если~$\varphi$ не 
обладает свойством~$P$, то $\varphi\not\in \boldsymbol{\Omega}^+$. Тогда при 
любом покрытии~$\boldsymbol{\Omega}^+$ в~силу ЗКП ни один из порождающих 
элементов из покрытия не может входить в~$\varphi$. Следовательно, ни одна 
конъюнкция ХФ не может равняться~1, а~тогда значение 
ХФ на~$\varphi$ равно~0. 
  
  Если $\varphi$ обладает свойством~$P$, то 
$\varphi\hm\in\boldsymbol{\Omega}^+$ и,~следовательно, входит в~один из 
классов сходства покрытия~$\boldsymbol{\Omega}^+$. Тогда порождающий 
элемент этого класса сходства принадлежит~$\varphi$ и~соответствующая 
конъюнкция равняется~1. 
  
  \noindent
  \textit{Достаточность.} Если ХФ\,$(\varphi)\hm=1$, то в~определяющем ХФ 
покрытии существует конъюнкция, принимающая значение~1. Тогда~$\varphi$ 
принадлежит соответствующему классу сходства в~$\boldsymbol{\Omega}^+$. 
Если ХФ\,$(\varphi)\hm=0$, то в~определяющем ХФ покрытии не существует ни 
одной конъюнкции, принимающей значение~1. Тогда~$\varphi$ не принадлежит 
ни одному классу сходства в~$\boldsymbol{\Omega}^+$. Тогда из покрытия 
следует, что $\varphi \hm \in \boldsymbol{\Omega}^-$.
  
  \section{Примеры}
  
  Несложно убедиться, что в~общем случае ХФ, формируемая на текущей БФ, не 
является единственной. 
  
  \smallskip
  
  \noindent
  \textbf{Пример~1.}\ Пусть $\boldsymbol{U}\hm = \{a_1, a_2, a_3, a_4\}$ 
и~$\boldsymbol{\Omega}\hm= \{O_1, O_2, \ldots , O_7\}$, где 
  \begin{align*}
O_1&=\left\{ a_1, a_4\right\}\,;\\
  O_2&=\left\{ a_1, a_2\right\}\,;\\
  O_3&=\left\{ a_2, a_4\right\}\,;,\\
  O_4&=\left\{ a_1, a_3\right\}\,;\\
  O_5 &= \left\{ a_1, a_2, a_3, a_4\right\}\,;\\
  O_6 &= \left\{ a_2, a_3\right\}\,;\\
  O_7 &= \left\{ a_3, a_4\right\}\,,
  \end{align*}
при этом множество примеров $\boldsymbol{\Omega}^+\hm =\{O_1^+, O_2^+, 
\ldots , O_7^+\}$, а~множество контрпримеров~--- $\boldsymbol{\Omega}^-\hm= \phi$. 

  Несложно убедиться, что дизъюнкции 
  \begin{gather*}
   \hspace*{-7mm}\mathrm{ХФ}_1(\boldsymbol{\Omega})= a_1\vee a_2 \vee a_3
   \left( \left\{ O_1, O_2, O_4, O_5\right\}\cup \right.\\
   \hspace*{7mm} \left.\cup\left\{ O_2, O_3, O_5, O_6\right\} \cup 
\left\{ O_4, O_5, O_6, O_7\right\}\right)\,;\\
   \hspace*{-7mm}\mathrm{ХФ}_2(\boldsymbol{\Omega})=a_1\vee a_3\vee a_4
  \left( \left\{ O_1, O_2, O_4, O_5\right\}\cup\right.\\
 \hspace*{7mm}\left.  \cup \left\{ O_4, O_5, O_6, O_7\right\} \cup 
\left\{ O_1, O_3, O_5, O_7\right\}\right)\,;\\
   \hspace*{-7mm}\mathrm{ХФ}_3(\boldsymbol{\Omega})=a_2\vee a_3\vee a_4
  \left( \left\{ O_2, O_3, O_5, O_6\right\}\cup\right.\\
 \hspace*{7mm}\left.  \cup \left\{ O_4, O_5, O_6, O_7\right\} \cup 
\left\{ O_1, O_3, O_5, O_7\right\}\right)
  \end{gather*}
  соответствуют трем различным покрытиям исходного множества 
примеров~$\boldsymbol{\Omega}^+$ классами сходства примеров 
из~$\boldsymbol{\Omega}$.

  Множество всех ХФ, построенных на заданной БФ, может оказаться пустым.
  \smallskip
  
  \noindent
  \textbf{Пример~2.} Пусть $\boldsymbol{U}\hm= \{a_1, a_2,\ldots , a_n, x,y,z\}$ 
и~$\boldsymbol{\Omega}\hm =\{ O_1, O_2, O_3\}$, где
  \begin{align*}
   O_1&= \left\{a_1, a_2, \ldots , a_n, x\right\}\,;\\
  O_2&= \left\{a_1, a_2, \ldots , a_n, y\right\}\,;\\
  O_3&= \left\{a_1, a_2, \ldots , a_n, z\right\}\,,
  \end{align*}
  при этом множество примеров $\boldsymbol{\Omega}^+\hm= \{O_1^+, O_2^+\}$, 
а~множество контрпримеров $\boldsymbol{\Omega}^-\hm= \{O_3^-\}$. 

  Тогда единственная ЭЗ, формируемая на примерах $\{O_1^+, O_2^+\}$, 
порождается множеством признаков $\{a_1, a_2,\ldots , a_n\}$, и~при этом условие 
ЗКП на контрпримере~$O_3^-$ не выполняется. Таким образом, множество ХФ, 
формируемых на БФ, оказывается пустым.
  
  \section{Заключение }
  
  \noindent
  \begin{enumerate}[1.]
\item В работе исследован подход к~построению методов описания причинности 
появления свойства~$P$ в~БФ. Открытость множества БФ не позволяет 
однозначно определять причину\linebreak свойства~$P$, если не определена структура 
этой причины и~единственность причины. Поэтому в~исследовании речь идет 
о~выявлении фактов влияния на появление ЭП. Пополнение БФ позволяет лишь 
делать предположения о~появлении свойства~$P$ в~новых фактах. 
\item Для удобства анализа факторов влияния на появление ЭП определены ХФ, 
которые позволяют формализовать рассматриваемый подход в~условиях 
выполнения ЗКП. 
\item В терминах ХФ можно автоматизировать поиск свойства~$P$ в~новых 
прецедентах. 
\item Развитие предложенного метода предполагает его практическое 
использование в~областях, перечисленных во введении, а также дальнейшее 
исследование при\-чин\-но-след\-ст\-вен\-ных связей в~условиях сложных 
причин и~сложной структуры свойства~$P$. 
\end{enumerate}
  
{\small\frenchspacing
 {%\baselineskip=10.8pt
 \addcontentsline{toc}{section}{References}
 \begin{thebibliography}{99}
\bibitem{1-gr}
\Au{Грушо А.\,А., Забежайло~М.\,И., Грушо~Н.\,А., Тимонина~Е.\,Е.} Поиск эмпирических 
причин сбоев и~ошибок в~компьютерных системах и~сетях с~использованием метаданных~// 
Системы и~средства информатики, 2019. Т.~29. №\,4. С.~28--38. doi: 10.14357/08696527190403.
\bibitem{2-gr}
\Au{Грушо А.\,А., Забежайло~М.\,И., Грушо~Н.\,А., Тимонина~Е.\,Е.} Архитектурные решения 
в~задаче выявления мошенничества при анализе информационных потоков в~цифровой 
экономике~// Информатика и~её применения, 2019. Т.~13. Вып.~2. С.~21--27. doi: 
10.14357/19922264190204.

\bibitem{4-gr}
\Au{Грушо А.\,А., Забежайло~М.\,И., Грушо~Н.\,А., Тимонина~Е.\,Е.} Интеллектуальный анализ 
данных в~обеспечении информационной безопасности~// Проб\-ле\-мы информационной 
безопасности. Компьютерные сис\-те\-мы, 2016. №\,3. С.~55--60.

\bibitem{3-gr}
\Au{Grusho A.\,A.} Data mining and information security~// Computing network security~/
Eds. J.~Rak, J.~Bay, I.~Kotenko, \textit{et al.}~--- Lecture notes in computer science ser.~--- Springer, 
2017. Vol.~10446. P.~28--33.

\bibitem{5-gr}
\Au{Забежайло М.\,И.} Комбинаторные средства формализации эмпирической индукции: Дис.\ 
\ldots\  докт. физ.-мат. наук.~--- М., 2015. 440~с.
\bibitem{6-gr}
\Au{Забежайло М.\,И., Трунин~Ю.\,Ю.} К~проб\-ле\-ме доказательности медицинского 
диагноза: интеллектуальный анализ данных о пациентах в~выборках ограниченного размера~// 
На\-уч\-но-тех\-ни\-че\-ская \mbox{информация}. Сер.~2, 2019. №\,12. C.~12--18.
\bibitem{7-gr}
\Au{Михеенкова М.\,А. и~др.} ДСМ-метод в~социологии: анализ данных и~прогнозирование~// 
Автоматическое порождение гипотез в~интеллектуальных системах~/ Под общ. ред. 
В.\,К.~Финна.~--- М.: ЛИБРОКОМ, 2009. С.~409--492.
\bibitem{8-gr}
\Au{Грушо А.\,А., Грушо~Н.\,А., Забежайло~М.\,И., Смирнов~Д.\,В., Тимонина~Е.\,Е.} 
Параметризация в~прикладных задачах поиска эмпирических причин~// Информатика и~её 
применения, 2018. Т.~12. Вып.~3. С.~62--66.
\bibitem{9-gr}
\Au{Финн В.\,К.} Индуктивные методы Д.\,С.~Милля в~сис\-те\-мах искусственного интеллекта. 
Часть~I~// Искусственный интеллект и~принятие решений, 2010. №\,3. С.~3--21.
\bibitem{10-gr}
\Au{Финн В.\,К.} Индуктивные методы Д.\,С.~Милля в~сис\-те\-мах искусственного интеллекта. 
Часть~II~// Искусственный интеллект и~принятие решений, 2010. №\,4. С.~14--40.
 \end{thebibliography}

 }
 }

\end{multicols}

\vspace*{-3pt}

\hfill{\small\textit{Поступила в~редакцию 12.01.20}}

%\vspace*{8pt}

%\pagebreak

\newpage

\vspace*{-28pt}

%\hrule

%\vspace*{2pt}

%\hrule

%\vspace*{-2pt}

\def\tit{ON CAUSAL REPRESENTATIVENESS OF~TRAINING SAMPLES OF~PRECEDENTS 
IN~DIAGNOSTIC TYPE TASKS}


\def\titkol{On causal representativeness of~training samples of~precedents 
in~diagnostic type tasks}

\def\aut{A.\,A.~Grusho$^1$, M.\,I.~Zabezhailo$^2$, and~E.\,E.~Timonina$^1$}

\def\autkol{A.\,A.~Grusho, M.\,I.~Zabezhailo, and~E.\,E.~Timonina}

\titel{\tit}{\aut}{\autkol}{\titkol}

\vspace*{-11pt}


\noindent
   $^1$Institute of Informatics Problems, Federal Research Center ``Computer 
Science and Control'' of the Russian\linebreak
$\hphantom{^1}$Academy of Sciences; 44-2~Vavilov Str., Moscow 
119133, Russian Federation
   
   \noindent
   $^2$Dorodnicyn Computing Center, Federal Research Center ``Computer Science 
and Control'' of the Russian\linebreak
$\hphantom{^1}$Academy of Sciences, 40~Vavilov Str., Moscow 119333, 
Russian Federation

\def\leftfootline{\small{\textbf{\thepage}
\hfill INFORMATIKA I EE PRIMENENIYA~--- INFORMATICS AND
APPLICATIONS\ \ \ 2020\ \ \ volume~14\ \ \ issue\ 1}
}%
 \def\rightfootline{\small{INFORMATIKA I EE PRIMENENIYA~---
INFORMATICS AND APPLICATIONS\ \ \ 2020\ \ \ volume~14\ \ \ issue\ 1
\hfill \textbf{\thepage}}}

\vspace*{3pt} 


   
   
   \Abste{The work focuses on some features of causality analysis in data mining tasks. The 
possibilities of using so-called open logic theories in diagnostic (classification) tasks to describe 
replenished sets of empirical data are discussed. In tasks of this type, it is necessary to establish 
(predict, diagnose, etc.)\ the presence or absence of a~target property in a~new precedent given by 
a~description in the same presentation language of heterogeneous data, which describes examples 
having a target property and counter-examples not having a target property. The variant of construction 
of open theories describing collections of precedents by means of special logical expressions~--- 
characteristic functions~--- is presented. Characteristic functions allow to get rid of heterogeneity in 
descriptions of precedents. The procedural design of formation of characteristic functions of a~training 
sample of precedents is proposed. The properties of characteristic functions and some conditions of 
their existence are studied.}
   
   \KWE{diagnostics; causal analysis; intelligent data analysis; open logic theories}
   
   

\DOI{10.14357/19922264200111} 

%\vspace*{-14pt}

\Ack
   \noindent
   The paper was partially supported by the Russian Foundation for Basic Research (project  
18-29-03081).


%\vspace*{6pt}

  \begin{multicols}{2}

\renewcommand{\bibname}{\protect\rmfamily References}
%\renewcommand{\bibname}{\large\protect\rm References}

{\small\frenchspacing
 {%\baselineskip=10.8pt
 \addcontentsline{toc}{section}{References}
 \begin{thebibliography}{99} 
\bibitem{1-gr-1}
\Aue{Grusho, A.\,A., N.\,A.~Grusho, M.\,I.~Zabezhailo, and E.\,E.~Timonina.} 2019. Poisk 
empiricheskikh prichin sboev i~oshibok v~komp'yuternykh sistemakh i~setyakh s~ispol'zovaniem 
metadannykh [Search of empirical causes of failures and errors in computer systems and networks 
using metadata]. \textit{Sistemy i~Sredstva Informatiki~--- Systems and Means of Informatics}  
29(4):28--38. 
\bibitem{2-gr-1}
\Aue{Grusho, A.\,A., N.\,A.~Grusho, M.\,I.~Zabezhailo, and E.\,E.~Timonina.} 2019. Arkhitekturnye 
resheniya v~zadache vyyavleniya moshennichestva pri analize informatsionnykh potokov v~tsifrovoy 
ekonomike [Architectural decisions in the problem of identification of fraud in the analysis of 
information flows in digital economy]. \textit{Informatika i~ee Primeneniya~--- Inform. Appl.} 
13(2):21--27. 

\bibitem{4-gr-1}
\Aue{Grusho, A.\,A., N.\,A.~Grusho, M.\,I.~Zabezhailo, and E.\,E.~Timonina.} 2016. Intelligent data 
analysis in information security. \textit{Autom. Control Comp.~S.} 50(8):722--725.

\bibitem{3-gr-1}
\Aue{Grusho, A.} 2017. Data mining and information security. 
\textit{Computer network security}. Eds. 
J.~Rak, J.~Bay, I.~Kotenko, \textit{et al.} Lecture notes in computer science ser. Springer.  
10446:28--33.

\bibitem{5-gr-1}
\Aue{Zabezhailo, M.\,I.} 2015. Kombinatornye sredstva for\-ma\-li\-za\-tsii empiricheskoy induktsii 
[Combinatorial means of formalizing empirical induction]. 
 Moscow. D.Sc.\ Diss. 440~p.
\bibitem{6-gr-1}
\Aue{Zabezhailo, M.\,I., and Y.\,Y.~Trunin.} 2019. 
%K~probleme dokazatel'nosti meditsinskogo 
%diagnoza: intellektual'nyy analiz dannykh o~patsientakh v~vyborkakh ogranichennogo razmera [To 
%the problem of evidence of medical diagnosis: Intelligent analysis of patient data in samples of limited 
%size]. \textit{Nauchno-tekhnicheskaya informatsiya} [Scientific and Technical Information Processing] 
%Ser.~2. 12:12--18.
%
On the problem of medical diagnostic evidence: 
Intelligent analysis of empirical data on patients in samples of limited size. 
\textit{Autom. Doc. Math. Linguist.} 53:322--328. 
doi: 10.3103/ S0005105519060086.
\bibitem{7-gr-1}
\Aue{Mikheenkova, M.\,A., \textit{et al.}} 2009. DSM-metod v~sotsiologii: analiz dannykh 
i~prognozirovanie [DSM method in sociology: Data analysis and forecasting]. 
\textit{Av\-to\-ma\-ti\-che\-skoe porozhdenie gipotez v~intellektual'nykh sistemakh} [Automatic hypotheses 
generation in intelligent systems]. Ed. V.\,K.~Finn. Moscow: KD LIBROKOM. 409--492.
\bibitem{8-gr-1}
\Aue{Grusho, A.\,A., N.\,A.~Grusho, M.\,I.~Zabezhailo, D.\,V.~Smirnov, and E.\,E.~Timonina.} 2018. 
Parametrizatsiya v~prikladnykh zadachakh poiska empiricheskikh prichin [Parametrization in applied 
problems of search of the empirical reasons]. \textit{Informatika i~ee Primeneniya~--- Inform. Appl.} 
12(3):62--66.
\bibitem{9-gr-1}
\Aue{Finn, V.\,K.} 2011. J.\,S.~Mill's 
inductive methods in artificial intelligence systems. Part~I. 
\textit{Scientific Technical Information Processing} 38(6):385--302. 
\bibitem{10-gr-1}
\Aue{Finn, V.\,K.} 2012. J.\,S.~Mill's 
inductive methods in artificial intelligence systems. Part~II. 
\textit{Scientific Technical Information Processing} 39(5):241--260.
\end{thebibliography}

 }
 }

\end{multicols}

%\vspace*{-7pt}

\hfill{\small\textit{Received January 12, 2020}}

%\pagebreak

%\vspace*{-22pt}


\Contr

\noindent
\textbf{Grusho Alexander A.} (b.\ 1946)~--- 
Doctor of Science in physics and mathematics, professor, principal scientist, Institute of Informatics 
Problems, Federal Research Center ``Computer Science and Control'' of the Russian Academy of 
Sciences; 44-2~Vavilov Str., Moscow 119133, Russian Federation;  \mbox{grusho@yandex.ru}

\vspace*{6pt}

\noindent
\textbf{Zabezhailo Michael I.} (b.\ 1956)~--- Doctor of Science in physics and mathematics, principal 
scientist, Dorodnicyn Computing Center, Federal Research Center 
``Computer Science and Control'' of the Russian Academy of Sciences, 40~Vavilov Str., Moscow 119333, 
Russian Federation; \mbox{m.zabezhailo@yandex.ru}

\vspace*{6pt}

\noindent
\textbf{Timonina Elena E.} (b.\ 1952)~--- 
Doctor of Science in technology, professor, leading scientist, Institute of Informatics 
Problems, Federal Research Center ``Computer Science and Control'' 
of the Russian Academy of Sciences; 44-2~Vavilov Str., Moscow 119133, Russian Federation; 
\mbox{eltimon@yandex.ru}



\label{end\stat}

\renewcommand{\bibname}{\protect\rm Литература} 