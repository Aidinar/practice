\def\stat{pop+sim}

\def\tit{МОДЕЛИРОВАНИЕ ПРОЦЕССА МОНИТОРИНГА СИСТЕМ ИНФОРМАЦИОННОЙ 
БЕЗОПАСНОСТИ\\ НА~ОСНОВЕ СИСТЕМ МАССОВОГО ОБСЛУЖИВАНИЯ$^*$}

\def\titkol{Моделирование процесса мониторинга систем информационной 
безопасности на~основе СМО} %систем массового обслуживания}

\def\aut{Г.\,А.~Попов$^1$, С.\,Ж.~Симаворян$^2$, А.\,Р.~Симонян$^3$, 
Е.\,И.~Улитина$^4$}

\def\autkol{Г.\,А.~Попов, С.\,Ж.~Симаворян, А.\,Р.~Симонян, 
Е.\,И.~Улитина}

\titel{\tit}{\aut}{\autkol}{\titkol}

\index{Попов Г.\,А.}
\index{Симаворян С.\,Ж.}
\index{Симонян А.\,Р.} 
\index{Улитина Е.\,И.}
\index{Popov G.\,A.}
\index{Simavoryan S.\,Zh.}
\index{Simonyan A.\,R.} 
\index{Ulitina E.\,I.}


{\renewcommand{\thefootnote}{\fnsymbol{footnote}} \footnotetext[1]
{Исследование выполнено при финансовой поддержке РФФИ (проект 19-01-00383).}}


\renewcommand{\thefootnote}{\arabic{footnote}}
\footnotetext[1]{Астраханский государственный технический университет, popov@astu.org}
\footnotetext[2]{Сочинский государственный университет, simsim58@mail.ru}
\footnotetext[3]{Сочинский государственный университет, oppm@mail.ru}
\footnotetext[4]{Сочинский государственный университет, ulitinaelena@mail.ru}

%\vspace*{-12pt}
  

  \Abst{Рассматривается задача моделирования процесса мониторинга в~системах 
информационной безопасности по выявлению необнаруженных злоумышленных атак на 
основе использования методов теории массового обслуживания. Процесс мониторинга 
сводится к~анализу потока заявок на обслуживание системой обработки данных как потока 
потенциально возможных злоумышленных действий. При выявлении вызова мониторинг 
немедленно прекращается и~начинается обслуживание выявленного вызова. В~рамках 
указанной модели получены функциональные соотношения для следующих двух наиболее 
важных характеристик: вероятности состояний системы и~вероятности числа невыявленных 
вызовов в~моменты окончания обслуживания. Нахождение указанных характеристик 
позволит более эффективно организовать процесс выявления злоумышленных атак на 
систему обработки данных при данной схеме обработки выявленных вызовов.}
   
  \KW{защита информации; информационная безопасность; система массового 
обслуживания; вероятность}

\DOI{10.14357/19922264200110} 
  
%\vspace*{-3pt}


\vskip 10pt plus 9pt minus 6pt

\thispagestyle{headings}

\begin{multicols}{2}

\label{st\stat}

  \section{Введение}
  
  Одной из актуальных проблем процесса обеспечения информационной 
безопасности в~системах обработки данных является проблема выявления 
успешных злоумышленных атак, оказавшихся незамеченными для систем 
обнаружения вторжений, т.\,е.\ атак, которые система обнаружения 
вторжений не обнаружила на этапе поступления запроса на обработку 
и~пропустила как незлоумышленное действие. В~этом случае система 
обработки данных либо даже не имеет понятия о том, что была подвергнута 
успешной злоумышленной атаке, либо узнает об этом слишком поздно, когда 
уже необходимо предпринять меры по локализации и~ликвидации последствий 
несанкционированного вторжения. Примером может служить появление новых 
вирусов, шпионских программ, когда су\-щест\-ву\-ющие средства защиты еще не 
выработали соответствующих механизмов противодействия и~в~течение 
некоторого периода новый вирус или программа абсолютно безнаказанно 
действует в~компьютерных системах. 
  
  В~настоящее время большое внимание уделяется разработкам 
интеллектуальных систем защиты информации на основе нейронных сетей для 
решения задач, связанных с~обнаружением атак, и~механизмов искусственных 
иммунных структур, которые успешно решают задачу противодействия 
выявленным угрозам на этапе проникновения их в~системы обработки 
данных~[1, 2]. Статей по разработке эффективных методов и~механизмов 
противодействия невыявленным угрозам почти нет~\cite{2-sim}.
  
  Можно перечислить ряд методов, позволяющих в~определенных 
ограниченных рамках выявлять и/или нейтрализовать невыявленные 
злоумышленные атаки~\cite{2-sim}. В~част\-ности, периодическое полное 
обновление программного обеспечения, периодический мониторинг систем 
обработки данных с~целью выявления ка\-ких-ли\-бо нетиповых отклонений, 
использование интеллектуальных методов выявления атак. В~данной работе 
рассматривается процедура мониторинга систем обработки данных с~\mbox{целью} 
выявления возможных следов злоумышленных атак.
  
  
  Для формализованного изучения процесса мониторинга как составной части 
системы ин\-фор\-мационной безопас\-ности в~работе предлагается использо\-вать 
аппарат теории систем массового обслуживания (СМО)~[3, 4]. Использованы 
методы анализа соответствующих СМО, аналогичные приведенным в~[5].

\vspace*{-6pt}
  
  \section{Описание модели системы массового обслуживания 
с~мониторингом} 
  
  Рассматривается СМО, в~которую поступает простейший поток вызовов 
с~интенсивностью~$a$. Поступающий вызов направляется в~очередь. Процесс 
обслуживания состоит из двух этапов. На первом этапе проводится мониторинг 
системы с~целью\linebreak выявления хотя бы одного вызова, который не\-об\-ходимо 
обслужить. Будем различать два вида мониторинга. Первый выполняется перед 
каж\-дым обслуживанием: он заключается в~проверке всех\linebreak стандартных 
атрибутов с~целью выявления потребности в~обслуживании (например, 
противодействии атаке). Решение о~том, какой из вызовов обслуживается 
следующим, принимается, исходя из результатов мониторинга. Назовем этот 
мониторинг мониторингом очереди. По\-треб\-ность в~обслуживании вызова, 
находящегося в~сис\-те\-ме, при мониторинге очереди возникает с~ве\-ро\-ят\-ностью 
$\alpha\hm >0$. Обслуженный вызов покидает сис\-те\-му. Функции 
распределения (ФР) времени мониторинга и~времени обслуживания 
рав\-ны~$B_1(t)$ и~$B_2(t)$ соответственно. В~случае если вызов не был 
выбран для обслуживания в~процессе мониторинга, он остается в~очереди. 
Предположим, что после цикла мониторинга не был выбран ни один вызов для 
обслуживания. Тогда обслуживающий прибор начинает выполнять общий 
мониторинг, предполагающий более системный, общий и~глубокий анализ 
и~контроль состояния системы. Этот мониторинг назовем общим 
мониторингом. Он выполняется периодически и~непрерывно, после окончания 
одного цикла мониторинга сразу же начинается другой. Если в~процессе 
данного мониторинга возникает потребность в~обслуживании, то мониторинг 
сразу же прерывается и~начинается обслуживание вызова. Вероятность 
обнаружения потребности в~обслуживании (например, скрытой или явной атаки 
в~системах безопасности) равна~$\gamma$, а ФР длительности одного цикла 
мониторинга равна~$B_3(t)$. Пусть $B_1(+0)\hm=0$, $B_3(+0)\hm=0$, т.\,е.\ 
мгновенный мониторинг исключается.

\vspace*{-6pt}
  
  \section{Основной результат}
  
  Ниже используется следующая лемма. % (СВ~--- случайная величина).
  
  \smallskip
  
  \noindent
  \textbf{Лемма.} \textit{Пусть случайные величины (СВ) $\{\xi_i, i\hm= 
\overline{-n,M}\}$ ($n\hm\geq 0$) независимы и~равномерно распределены на 
промежутке $[0,A]$, где $M$~--- СВ, имеющая пуассоновское распределение 
с~параметром~$\lambda$. Каждая СВ реализуется с~вероятностью~$\gamma$ 
и~<<окрашивается>> в~красный цвет с~вероятностью~$z_1$ для 
$i\hm\leq 0$ и~вероятностью~$z_2$ 
для $i\hm\geq 1$; $\zeta\hm=\min \xi_i$, где минимум 
берется по всем реализованным СВ. Тогда вероятность $\Phi(z,x,A)\hm= 
\Phi(z,x)$ события <<все имеющиеся СВ окрашены в~красный 
цвет и~$\zeta\hm< x$>> равна}:
  \begin{multline*}
  \Phi(z,x)={}\\
  {}=
  \begin{cases}
  0 & \hspace*{-20mm}\mbox{при } x<0\,;\\
  z_1^n e^{-\lambda A(1-z_2)} \left( 1-\left( 1-\gamma\fr{x}{A}\right)^n e^{-\lambda 
z_2 \gamma x}\right) &\\
&\hspace*{-20mm} \mbox{при } x\in [0,A)\,;\\
  z_1^n &  \hspace*{-20mm}\mbox{при } x\geq A\,.
  \end{cases}
  \end{multline*}
  
  \noindent
  Д\,о\,к\,а\,з\,а\,т\,е\,л\,ь\,с\,т\,в\,о\,.\ \ Введем 
СВ~$\eta_i$ ($i\hm\geq -n$):
  $$
  \eta_i= \begin{cases}
  \xi_i & \mbox{с~вероятностью } \gamma\,;\\
  +\infty & \mbox{с~вероятностью } 1-\gamma\,.
  \end{cases}
  $$
  
  Тогда $\zeta=\min \{ \eta_i \vert -n\leq i\leq M\}$. Пусть~$B(N)$ есть событие 
<<СВ $\{\xi_i,\ i\hm= \overline{-n,N}\}$ окрашены в~красный 
цвет>>. Для любого фиксированного $N\hm\geq 0$ и~любого $x\hm\in [0,A]$ 
имеем
  \begin{multline*}
  {\sf P}\left\{ B(N),\min\left( \eta_i\vert i\in [-n;N]\right)<x\right\} ={}\\
  {}={\sf P}\{B(N)\}- {\sf P}\left\{ B(N),\min\left( \eta_i\vert i\in [-n;N]\right) \geq 
x\right\}={}\\
  {}=z_1^n z_2^N \left(1-\prod\limits^N_{i=-n} P\left( \eta_i\geq 
x\right)\right)={}\\
  {}=z_1^n z_2^N\left( 1-\prod^N_{i=-n} \left( 1-\gamma+\gamma\left( 1-
\fr{x}{A}\right)^{N+n}\right)\right)={}\\
  {}= z_1^n z_2^N -z_1^n z_2^N \left(1-\gamma\fr{x}{A}\right)^{N+n}\,,
  \end{multline*}
откуда
\begin{multline*}
\Phi(z,x) =\sum\limits_{N\geq 0} \! {\sf P}(M=N) \times{}\\
{}\times {\sf P}\left\{ B(N),\min \left( \eta_i\vert -
n\leq i\leq N\right) <x\right\}={}\\
{}=
\sum\limits^\infty_{N=0} \fr{(\lambda A)^N}{N!}\,e^{-\lambda A} z_1^n 
z_2^N\left[ 1-\left( 1-\gamma\fr{x}{A}\right)^{N+n}\right]={}\\
{}=
z_1^n e^{-\lambda A(1-z_2)}\left( 1-\left( 1-\fr{\gamma x}{A}\right)^n e^{1-
\gamma x/A}\right)\,,
\end{multline*}
что влечет утверждение леммы. 

\smallskip

  На основе леммы доказывается следующая теорема.
  
  \smallskip
  
  \noindent
  \textbf{Теорема.}\ \textit{Справедливо соотношение}
  \begin{multline*}
  \Phi^\prime_{x} (z,x)={}\\
  {}= \begin{cases}
  0\,, &\!\mbox{если } x\notin (0,A)\,;\\
  z_1^n e^{-\lambda A(1-z_2)-\lambda z_2\gamma x}\overline{\Phi}(z,x)\,, & 
\!\mbox{если } x\in (0,A)\,,
  \end{cases}
  \end{multline*}
\textit{где} 
$$
\overline{\Phi}(z,x)=\left( \fr{n}{A}\left( 1- \gamma\fr{x}{A}\right)^{n-1} +\left( 
1-\gamma\fr{x}{A}\right)^n \lambda z_2 \gamma\right)\,.
$$

\smallskip
  
  Введем следующие обозначения. Пусть $q(m,n,t)$ ($n\hm\geq 0$; $0\hm\leq 
m\hm\leq n$; $t\hm\geq0$) есть вероятность того, что в~очереди в~момент~$t$ 
находится~$n$~вызовов, из которых~$m$ поступили в~сис\-те\-му во время 
обслуживания других вызовов; $q(z,w,t)\hm= \sum\nolimits_{n\geq0} 
\sum\nolimits^n_{m=0} q(m,n,t) z^m w^n$ ($0\hm\leq z\hm\leq 1$); 
$\beta_i(s)\hm= \int\nolimits_0^\infty e^{-s t}\,dB_i(t)$~--- преобразование 
Лап\-ла\-са--Стилть\-еса (ПЛС) ФР $B_i(t)$ ($i\hm= \overline{1,3}$); 
$\beta(s)\hm= \beta_1(s)\beta_2(s)$; $p(m,n,t)$~--- вероятность следующего 
события: в~момент~$t$ заканчивается обслуживание вызова, в~системе 
имеется~$n$~вызовов, из которых~$m$ пришли при обслуживании других 
вызовов, и~нет выявленных для обслуживания вызовов ($m\hm\leq n$). 
  
  Выведем соотношение для $q(z,w,t)$. Заметим, что функции $q(z,w,t)$ можно 
дать следующую вероятностную интерпретацию. Предположим, что каждый 
вызов, поступивший во время обслуживания другого вызова, 
с~вероятностью~$w$ окрашивается в~розовый цвет, а~с~вероятностью $(1\hm-
w)$ не окрашивается; кроме того, каждый вызов, поступивший во время 
обслуживания другого вызова, окрашивается с~вероятностью~$z$ в~красный 
цвет, а~с~вероятностью $(1\hm- z)$~--- в~синий. Тогда $q(z,w,t)$ есть 
вероятность того, что в~очереди в~момент~$t$ все вызовы окрашены в~красный 
цвет (если очередь не пуста) и~нет синих вызовов, пришедших в~систему во 
время обслуживания других вызовов. Потоки как красных, так и~синих вызовов 
получаются из поступающего простейшего потока на основе процедуры его 
просеивания с~вероятностями~$w$ и~$(1\hm- w)$ соответственно 
и,~следовательно, также являются пуассоновскими с~параметрами~$a w$ 
и~$a(1\hm-w)$. Аналогичные утверждения справедливы для потоков красных 
и~неокрашенных вызовов, а~также для комбинаций цветов.
  
  Назовем вызов плохим, если за время его обслуживания пришли синие или 
неокрашенные вызовы. Вероятность того, что данный вызов не является ни 
синим, ни окрашенным, равна $1\hm-zw$. Поэтому вероятность того, что 
данный вызов является хорошим (т.\,е.\ за время его обслуживания не пришло 
ни одного синего или неокрашенного вызова), равна $\beta(a\hm- azw) 
\stackrel{\mathrm{def}}{=} \tau$.
  
  Составим соотношения для потока хороших вызовов, поступивших в~систему 
за время от~0 до~$t$. Поток плохих вызовов является просеянным 
пуассоновским потоком с~вероятностью просеивания $1\hm-\tau$. 
Следовательно, вероятность того, что за время~$t$ в~систему не поступило ни 
одного плохого вызова, равна $\exp (-a(1\hm- \tau)t)$. Отметим, что при этом 
синие вызовы могли поступить в~систему в~промежутках, когда система была 
свободна от обслуживания и~занята общим мониторингом; это вызовы, 
с~которых начинается период занятости, и~вызовы, которые оказались 
необнаруженными обслуживающим прибором (с~вероятностью $1\hm-\alpha$ 
или $1\hm-\gamma$ в~зависимости от этапа работы обслуживающего 
устройства). Описанное событие имеет место в~следующих случаях.\\[-14pt]
  \begin{enumerate}
  \item За время~$t$ вообще не было синих и~неокрашенных вызовов 
(вероятность равна $e^{-a(1-zw)t}$) и~в~очереди в~момент~$t$ нет синих 
и~неокрашенных вызовов (вероятность равна $q(z,w,t)$). Таким образом, 
вероятность указанного случая равна $e^{-a(1-zw)t} q(z,w,t)$.\\[-14pt]
  \item  Первый синий вызов поступил в~систему, когда она не обслуживала 
вызовов, а занималась общим мониторингом, а~именно: в~некоторый момент 
времени~$u$ система оказалась в~состоянии, описываемом вероятностью 
$p(n,m,u)$; при этом вызовы, пришедшие в~систему во время обслуживания 
других вызовов, были красными (вероятность~$z^m$). Просмотрев все вызовы, 
обслуживающий прибор не обнаружил ни одного вызова, нуждающегося 
в~обслуживании (вероятность $(1\hm-\alpha)^n$), и~длительность промежутка 
поиска вызовов для обслуживания (т.\,е.\ длительность мониторинга очереди) 
равна~$v_0$ (вероятность $dB_1(v_0)$); за этот промежуток не пришли 
неокрашенные и~плохие вызовы, а~также вызовы, которые будут выявлены до 
конца $N$-го этапа общего мониторинга (т.\,е.\ при\linebreak $(N\hm+1)$-й попытке 
выявления, включая\linebreak
 данный этап). Процедуру выявления можно рас\-смат\-ри\-вать 
как процедуру просеивания с~ве\-ро\-ят\-ностью просеивания~$\alpha$ на данном 
этапе и~вероятностью~$\gamma$ на последующих этапах. \mbox{Вероятность} того, 
что вызов не будет выявлен на этапах с~данного по $N$-й, 
равна~$\overline{\alpha}\,\overline{\gamma}^N$, где $\overline{\alpha}\hm=1\hm-
\alpha$, $\overline{\gamma}\hm= 1\hm-\gamma$. Поэтому вероятность того, что 
хотя бы один из уже просеянных по критериям окраски (с вероятностью $1\hm- 
\tau w$) вызовов будет выявлен, равна $(1\hm-\tau w) (1\hm- \overline{\alpha}\, 
\overline{\gamma}^N)$. А~вероятность того, что за время~$v_0$ подобных 
вызовов не придет, равна $e^{-a(1-\tau w)(1-
\overline{\alpha}\,\overline{\gamma}^N)v_0}$. Далее начался цикл из~$N$~этапов 
общей профилактики ($N\hm\geq0$), длительность $i$-го этапа равна~$v_i$ 
(вероятность $dB_3(v_i)$) ($1\hm\leq i\hm\leq N$). Ни один из первоначально 
имевшихся~$n$~вызовов за все~$N$~этапов не был выявлен с~целью 
обслуживания (вероятность $(1\hm- \gamma)^{nN}$); не был выявлен также 
и~ни один из вызовов, пришедших в~систему во время общего мониторинга. 
Так как при общем мониторинге процесс выявления вызовов с~\mbox{целью} 
обслуживания можно рассматривать как процедуру просеивания 
с~вероятностью~$\gamma$, то поток выявленных вызовов, пришедших на \mbox{$i$-м} 
этапе, во время обслуживания которых не было синих и~неокрашенных 
вызовов, является просеянным пуассоновским с~вероятностью просеивания 
$\gamma^{N-i+1} \tau w$, а~вероятность того, что ни один из пришедших на 
$i$-м этапе вызовов не будет выявлен до конца $N$-го этапа и~является 
хорошим, равна 
 $e^{-a(1-\gamma^{N-i+1}\tau w)v_i}$. В~силу свойств пуассоновского потока 
промежутки вре\-ме\-ни между последовательными моментами выявления, 
а~также остаточные времена до выявления вызовов имеют показательное 
распределение с~параметром $a\gamma^{N-i+1}\tau w$. Наконец, на 
($N\hm+1$)-м этапе, длительность которого равна~$v_{N+1}$ (вероятность 
$dB_3(v_{N+1})$), один из пришедших вызовов был выявлен через время~$v$ 
($v\hm\leq v_{N+1}$) после начала этого этапа, и~ни один из пришедших за это 
время вызовов не был неокрашенным или плохим; вероятность этого события, 
в~силу леммы~1, равна $\Phi^\prime_v(\tau w, v, v_{N+1})\,dv$. А~затем за 
оставшийся промежуток длиной ($t\hm-u\hm-\sum\nolimits^N_{i=0} v_i\hm- v$) 
не пришли вызовы, за время обслуживания которых поступили в~систему синие и~неокрашенные вызовы (вероятность $e^{-a(1-\tau w) (t -u-
\sum\nolimits^N_{i=0} v_i-v)}$).\\[-13pt]
\end{enumerate}
  
  Просуммировав по всем значениям $n$, $m$, $N$, $u$, ${v}$, $v_i$ 
($0\hm\leq i \hm \leq N\hm+1$), получим следующее выражение для вероятности 
события, описываемого в~п.~2:

\vspace*{-4pt}

\noindent
  \begin{multline*}
  \sum\limits_{n>0} \sum\limits^n_{m=0} 
  \sum\limits_{N\geq0} \idotsint\limits_D 
\int\limits_{v=0}^{v_{N+1}} p(m,n,u) z^m w^n \times{}\\
{}\times (1-\alpha)^n (1-\gamma)^{nN} (1-\gamma)^{nN} e^{-a(1-\alpha\gamma^N \tau w)v_0} \times{}\\
  {}\times
\prod\limits^N_{i=1} e^{-a(1-\gamma^{N-i+1}\tau w)\sum\nolimits^N_{j=i} v_j} 
\tau^n e^{-av_{N+1} (1-\tau w)}\times{}\\[-2pt]
  {}\times  
d_v\left(1-\left( 1-\gamma\fr{v}{v_{N+1}}\right)^n e^{-a\tau w\gamma v}\right)\times{}\\[-2pt]
{}\times
e^{-a(1-\tau w) (t-u-\sum\nolimits^N_{i=0} v_i-v)}
du dB_1(v_0) \prod\limits_{i=1}^{N+1} dB_3(v_i)\,,
\end{multline*}

\vspace*{-6pt}

\noindent
где область интегрирования 

\vspace*{-3pt}

\noindent
\begin{multline*}
D= \{(u;v_0;v_1;\ldots ; v_{N+1}): u+ 
v_0+v_1+\cdots\\
\cdots +v_{N+1}< t,\ u\geq 0\,,\ v_i\geq 0\ (0\leq 
i\leq N+1)\}.
\end{multline*}

 Таким образом, получаем уравнение
\begin{multline*}
e^{-a(1-\tau)t} = e^{-a(1-zw)t} q(z,t)\times{}\\
{}\times\sum\limits_{n>0} \sum\limits_{m=0}^n 
\sum\limits_{N\geq0} \idotsint\limits_D \int\limits_{v=0}^{v_{N+1}} p(m,n,u)
 z^m w^n\times{}\\
 {}\times (1-\alpha)^n 
(1-\gamma)^{n N} e^{-a(1-\tau w)v_0} \times{}\\
{}\times
e^{-a\sum\nolimits^N_{i=1} (1-\gamma^{N-i+1}\tau)\sum\nolimits^N_{j=1} v_j e^{-
av_{N+1}(1-\tau w)}}\times\\
\times d_v\left( 1-\left( 1-\gamma\fr{v}{v_{N+1}}\right)^n 
e^{-a\tau w\gamma v}\right)\times{}\\
{}\times
e^{-a(1-\tau w)(t-u-\sum\nolimits^N_{i=0} v_i-v)} du 
dB_1(v_0)\prod\limits_{i=1}^{N+1} dB_3(v_i)\,.
\end{multline*}
 % 
  Отсюда, полагая 
  $$
  p(w,z,t)= \sum\limits_{n\geq0} \sum\limits^n_{m=0} 
p(m,n,t) z^m w^n,
$$
 выводим ($\theta\hm= v/v_{N+1}$):
  \begin{multline*}
  q(z,t)= e^{a[(1-zw)-(1-\tau)]t}\Bigg( 
  1+ {}\\
    {}+\sum\limits_{N\geq0} \idotsint\limits_D \!\!\int\limits_{v=0}^{v_{N+1}} 
    \hspace*{-2mm}\hspace*{-1.51018pt}
d_\theta\left(p \left( (1\!-\!\alpha)(1\!-\!\gamma)^N (1\!-\!\gamma\theta), z ,
u\right) \times\right.\\
\left.{}\times e^{-a\tau w 
\gamma v_{N+1}\theta}\right) 
  e^{-a(1-\tau w)v_0} e^{-av_{N+1}(1-\tau w)}\times{}\\
  {}\times e^{-a(1-\tau w)(t-u-
\sum\nolimits^N_{i=0} v_i-v)}\times{}\\
{}\times e^{-a\sum\nolimits^N_{j=1} v_j \left(j-\tau w ({\gamma^{N-j+1}-
\gamma^n})/({1-\gamma})\right)} du dB_1(v_0)\times{}\\
{}\times \prod\limits_{i=1}^{N+1} 
dB_3(v_i)\Bigg)\,.
  \end{multline*}
  
  При достаточно малых $z\hm>0$ выражение $\beta(a\hm- az)\hm-zw\hm>0$. 
Поэтому при $t\hm\to \infty$ величина $e^{a[\beta(a-az))-zw]t} \hm\to\infty$ при 
указанных значениях~$z$. Но так как величина~$q(z,t)$ при $t\hm\to\infty$ 
ограничена, то выражение в~правой части в~скобках должно стремиться к~нулю, и~после алгебраических преобразований получаем:
  \begin{multline}
 \hspace*{-2.56012pt} \sum\limits_{N\geq0} 
 \int\limits^\infty_{\theta=0}\hspace*{-1mm}
   d_\theta \!\left(\! \right.
  p\left( (1\!-\!\alpha) (1\!-\!\gamma)^N (1\!-\!\gamma\theta), z, 
  a\tau(1\!-\!w)\right)\times{}\\
{}\times
\beta_1\left(a(1-\tau w)\right)\times{}\\
{}\times \beta_3\left( a\left(1+\theta\right)+\left(\tau 
w(\gamma\theta-1)-\tau\theta\right) v_{N+1}\right) \times{}\\
{}\times
  \prod^N_{j=1}\beta_3 \left(a\left( 
  j-\fr{\gamma^{N-j+1}-\gamma^n}{1-\gamma}\tau w\right)-a(1-\tau w)\right)={}\\
  {}=-1\,.
  \label{e1-sim}
  \end{multline}
  
  Равенство~(1) требует дальнейшего анализа. Оно может быть использовано 
в~процессе компьютерного моделирования процесса мониторинга.
  
  Исследуем теперь характеристику, которая описывает число невыявленных 
атак. Пусть $q_n(z,\overline{Z})$ есть вероятность того, что после окончания 
\mbox{$n$-го} по порядку обслуживания вызовов в~системе нет\linebreak выявленных  
$z$-си\-них вызовов, ожидающих обслуживания, все вызовы, не выявленные 
за~$i$~попыток, являются $i$-крас\-ны\-ми ($i\hm\geq 1$); здесь 
$\overline{Z}\hm= (z_1, z_2, \ldots , z_i, \ldots)$. Запишем рекуррентное 
\mbox{соотношение}, связывающее $q_{n+1}(z,\overline{Z})$ с~$q_n(z,\overline{Z})$.
  
  В момент окончания $(n\hm+1)$-го по порядку обслуживания возможны 
следующие ситуации.
  \begin{enumerate}
  \item В момент окончания $n$-го по порядку обслуживания в~очереди 
имеются выявленные вызовы (ве\-роят\-ность $q_n(z,\overline{z})\hm- 
q_n(0,\overline{z})$), с~ве\-ро\-ят\-ностью~$\alpha_i$ каждый вызов, не выявленный 
в~предыду\-щих ($i\hm-1$)-й попытках, будет выявлен и~перейдет в~основную 
очередь, а~с~дополнительной вероятностью $\overline{\alpha}_i\hm= 1\hm- 
\alpha_i$ не будет выявлен. В~последнем случае он с~вероятностью~$z_i$ 
получит дополнительную окраску $i$-крас\-но\-го цвета и~перейдет в~очередь 
из вызовов, не выявленных при~$i$~попытках. Распишем указанное событие 
в~терминах исходных вероятностей: $\sum\nolimits_{N\geq1} 
\sum\nolimits_{N_i,\ i\geq1} Q_n(N; N_i,\ i\hm\geq1) z^N \prod\nolimits_{i\geq 1} 
Z_i^{N_i}$, где $Z_i\hm= \prod\nolimits^i_{j=1} z_j$. 
С~ве\-ро\-ят\-ностью~$\alpha_{i+1}$ каждый вызов переходит в~основную очередь 
и~окрашивается в~красный цвет с~ве\-ро\-ят\-ностью~$z$ (при этом старая окраска 
убирается (делится на~$Z_i$)~--- в~момент  
($n\hm+1$)-го окончания его прошлая окрас\-ка не представляет интереса), 
а~с~вероятностью $\overline{\alpha_{i+1}}$ вызов переходит  
в~($i\hm+1$)-очередь и~получает дополнительную  
($i\hm+1$)-окраску с~ве\-ро\-ят\-ностью~$z_{i+1}$. Таким образом, необходимо 
заменить~$Z_i$ на~$z$ с~ве\-ро\-ят\-ностью~$\alpha_{i+1}$ и~заменить~$Z_i$ 
на~$Z_{i+1}$ с~вероятностью~$\overline{\alpha_{i+1}}$, т.\,е.~$Z_i$ 
заменяется на $\alpha_{i+1} z\hm+ (1\hm- \alpha_{i+1})Z_{i+1}$. Далее 
начинается обслуживание одного из основных вызовов (по предположению, 
в~случае~1 эта очередь не пуста); при этом необходимо убрать окраску этого 
вызова (т.\,е.\ разделить на~$z$). За время обслуживания этого вызова не 
должно поступить ни одного синего вызова; при этом каж\-дый поступающий 
вызов с~ве\-ро\-ят\-ностью~$\alpha_1$ становится выявленным и~ставится 
в~основную очередь, а~с~ве\-ро\-ят\-ностью~$\overline{\alpha_1}\hm=1-\alpha_1$ не 
выявляется и~ставится в~1-оче\-редь. Это равносильно тому, что по\-сту\-па\-ющий 
поток просеивается на три потока: выявленных 0-крас\-ных вызовов (параметр 
потока~$\alpha_1z$), не выявленных 1-крас\-ных вызовов параметр потока 
$(1\hm-\alpha_1)z_1$ и~на поток всех остальных вызовов, и~требуется, чтобы за 
время обслуживания не поступали вызовы третьего потока. Ве\-ро\-ят\-ность этого 
события равна $\beta(a\hm- a(\alpha_1 z\hm+ (1\hm- \alpha_1)z_1))$. Таким 
образом, получаем, что\linebreak вероятность, описываемая в~случае~1, имеет вид: 
$z^{-1} \left( q_n\left(z, Az+(1-A) *\overline{Z}_{+1}\right) -\right.$\linebreak
$\left.-q_n\left( 0, Az+(1\!\hm-\!A)* \overline{Z}_{+1}\right)\right)
 \beta\left( a\hm\!-\!a\left( \alpha_1z+(1-\right.\right.$\linebreak
 $\left.\left.-\alpha_1z_1\right)
 \right)$.
  Здесь <<*>> означает покомпонентное умножение векторов, 
а~индекс~<<$+1$>> указывает на увеличение на~1 индексов всех компонентов 
вектора.
  \item  В~момент окончания $n$-го по порядку обслуживания в~очереди нет 
выявленных вызовов (вероятность $q_n(0,\overline{Z})$). Затем началось 
проведение общего мониторинга системы, и~в течение~$N$~циклов ($N\hm\geq 
1$) не было выявлено ни одного вызова из имевшихся (вероятность 
$q_n(0,\{\prod\nolimits^N_{j=1} (1\hm-A_j)\}_{j\geq1} * \overline{Z}_{+N})$), 
ни одного из вызовов, пришедших на $i$-м цикле мониторинга (вероятность 
$\beta_3(a\hm- a\prod\nolimits^N_{j=i} (1\hm- \alpha_j)z_i)$ для всех $i\hm= 
\overline{1,N}$. Наконец, на ($N\hm+1$)-м цикле один из вызовов был выявлен. 
Так как поток красных вызовов, не выявленных на этапах с~$i$-го по $N$-й, 
может быть получен на основе процедуры просеивания с~вероятностью 
$\prod\nolimits^N_{j=i} (1\hm- \alpha_j)z_i$, то вероятность того, что за 
время~$t$ не будет выявлен ни один из вызовов, пришедших в~сис\-те\-му на $i$-м 
цикле, равна

\vspace*{-10pt}

\noindent
\begin{multline*}
  \hspace*{-1pt}q_n\left(0,\overline{Z}\right) =q_n\left(0, \left\{ \prod\limits^N_{j=1} \left(1-
A_j\right)\right\}_{j\geq1}\hspace*{-9pt} * \overline{Z}_{+N}\right) \times{}\\
{}\times\beta_3 \left( 
a-a\prod\limits^N_{j=i} \left(1-\alpha_j\right) z_i\right)\,.
\end{multline*}
  \end{enumerate}
  
  \vspace*{-6pt}
  
  Пусть $\overline{M_0}$~--- вектор числа невыявленных вызовов, прошедших 
различные стадии выявления, в~момент $n$-го окончания обслуживания (когда 
нет выявленных вызовов); $M_j$~--- число вызовов, поступивших во время  
$j$-го цикла общего мониторинга (и~они не были выявлены) ($1\hm\leq j\hm\leq 
N$); $M_{N+1}$~--- число вызовов, поступивших во время $(N\hm+1)$-го 
цикла общего мониторинга до момента~$\tau$, когда был выявлен первый из 
всех перечисленных вызовов (т.\,е.\ до момента~$\tau$ они не были выявлены). 
Тогда вероятность того, что хотя бы один из этих вызовов будет выявлен на 
$(N\hm+1)$-м цикле и~не будет выявлен на предыдущих, а~все вызовы~--- 
соответствующего типа красности, равна 

\vspace*{-2pt}

\noindent
  \begin{multline*}
  \sum\limits_{l\geq1} \alpha_{l+N+1}\times{}\\
  {}\times \prod\limits^N_{k=0} \left(1-
\alpha_{l+k}\right)^{M_{0,l}}
M_{0,l}\left( 1-\alpha_{l+N+1}\right)^{M_{0,l}-1}\times{}
\end{multline*}

\noindent
 \begin{multline*}
{}\times  
z_{l+N+1}^{M_{0,l}-1} \prod\limits_{j=1}^{N+1}
  \left[(1-\alpha_{N-j+1})^{M_j} z^{M_j}_{N-j+1}\right] \times{}\\
  {}\times \prod\limits_{j\geq1, j\not=l} \left\{
  \prod\limits_{k=0}^{N+1}\left( 1-\alpha_{j+k}\right) 
z_{j+N+1}\right\}^{M_{0,j}}+{}\\
{}+
 \sum\limits_{l=1}^{N+2} \alpha_l M_l\left( 1-\alpha_{l+N+1}\right)^{M_{l}-1}
  z^{M_l-1}_{l+N+1}\times{}\\
 {}\times
  \prod\limits_{j\geq1} \left(1-\alpha_{N-j+1}\right)^{M_{0,j}} 
  z^{M_j}_{n-j+1}\times{}\\
  \times 
 \prod\limits_{j=1,i\not=l}^{N+2} 
  \left( 1-\alpha_{N-j+1}\right)^{M_j} 
z^{M_j}_{N-j+1}\,.
  \end{multline*}

  Считаем, что момент~$\tau$ выявления вызова является 
СВ, равномерно распределенной на интервале~$\tau_{N+1}$ 
длительности $(N\hm+1)$-го цикла общего мониторинга.
  
  Заметим, что распределение вектора~$M_0$ задается производящей 
функцией $q_n(0,\overline{Z})$; $M_j$ распределено по закону Пуассона 
с~параметром~$a$ на промежутке~$\tau_j$ длительности $j$-го цикла. 
Просуммировав последнее выражение по всем возможным состояниям 
векторов~$M_0$ и~$M_j$ ($1\hm\leq j\hm\leq N$), получаем:
  \begin{multline*}
  \sum\limits_{l\geq1} \alpha_{l+N+1} \prod\limits^N_{k=0} \left( 1-
\alpha_{l+k}\right)\times{}\\
{}\times q^\prime_{n,l}\left( 0, 
\left\{ \prod\limits_{k=0}^{N+1} \left( 1-
\alpha_{l+k}\right) z_{l+N+1}\right\}_{l\geq1}\right)\times{}\\
{}\times b_3\left( a-a(\left( 1-
\alpha_1\right)z_1\right)\times{}\\
  {}\times
  \prod\limits^N_{j=1} \beta_3\left( a- a\prod\limits_{k=1}^{N-j+2}\left( 1-
\alpha_k\right) z_{N-j+1}\right) +{}\\
{}+\sum\limits_{l=1}^{N+1} \alpha_{N-l+2} 
\beta_{3}^\prime \left( \prod\limits_{k=1}^{N-l+1} \left( 1-\alpha_k\right) z_{N-
j+1}\right)\times{}\\
  {}\times
\prod\limits_{k=1}^{N-l+1}\hspace*{-2mm}
\left( 1-\alpha_k\right) q_n\!\left(\! 0,\left\{ 
\prod\limits_{k=0}^{N+1} \left( 1-\alpha_{l+k}\right) z_{l+N+1}\right\}\!\right) \times{}\\
{}\times
b_3\left( a-a\left( 1-\alpha_1\right)z_1\right) \times{}\\
{}\times
\prod\limits^N_{j=1,i\not=l} \beta_3 \left(a-a\prod\limits_{k=1}^{N-j+2} \left(1-
\alpha_k\right)z_{N-j+1}\right)\times{}\\
{}\times \prod\limits^N_{j=1,j\not=l} \beta_3 \left(
a-a\prod\limits_{k=1}^{N-j+2} \left(1-\alpha_k\right) z_{N-j+1}\right)+{}\\
{}+\alpha_1\prod\limits^N_{j=1} \beta_3\left( \prod\limits_{k=1}^{N-j+2} \left(
1-\alpha_k\right) z_{N-j+1}\right) \times{}\\
{}\times b_3^\prime \left(a-a\left(1-\alpha_1\right)z_1\right)\,,
\end{multline*}
где 

\noindent
\begin{align*}
q_{n,l^\prime} \left(z,\overline{Z}\right)&= \fr{\partial}{\partial z_l}\,
q_n\left(z,\overline{Z}\right)\,;
\\
b_3(s)&= \int\limits_0^\infty \fr{1-e^{-s\tau_{N+1}}}{s\tau_{N+1}}\,dB_3 
\left(\tau_{N+1}\right)\,.
\end{align*}
  
  Дополнительно надо умножить приведенную вероятность на 
вероятность~$z$ того, что выявленный вызов~--- красный, и~на вероятность 
$\beta(a\hm- a(\alpha_1z\hm+ (1\hm- \alpha_1)z_1))$ того, что за время его 
обслуживания не поступят в~систему синие вызовы и~невыявленные 1-си\-ние 
вызовы. Предполагается для простоты, что по окончании обслуживания 
очередной мониторинг не производится.
  
  При этом возможны следующие три схемы обработки выявленного вызова.
  \begin{enumerate}[1.]
  \item При выявлении вызова мониторинг немедленно прекращается 
и~начинается обслуживание выявленного вызова.
  \item При выявлении вызова мониторинг доводится до конца и~только затем 
начинается обслуживание выявленных вызовов.
  \item При выявлении вызова на первом цикле мониторинга прерывание 
обслуживания не происходит; на втором и~последующих циклах мониторинга 
происходит его прерывание и~начинается обслуживание вызова.
  \end{enumerate}
  
  В статье рассматривается только первая схема. Остальные схемы 
предполагается рассмотреть в~после\-ду\-ющих работах авторов. Тогда надо 
потребовать, чтобы за время начавшегося обслуживания красного вызова не 
пришли выявленные синие вызовы и~1-крас\-ные вызовы, не выявленные по 
окончании обслуживания этого вызова (вероятность $\beta(a\hm- 
a(\alpha_1z\hm+(1\hm- \alpha_1)z_1))$). Получаем:
  \begin{multline*}
  q_{n+1} \left(z,\overline{Z}\right) =
  \beta\left(a-a\left( \alpha_1 z+\left( 1-
\alpha_1\right) z_1\right)\right)\times{}\\
{}\times
  \Bigg\{ z^{-1} \left( q_n \left(z, Az+(1-A)*\overline{Z}_{+1}\right)\right.-{}\\
  \left.{}-
  q_n\left(0,Az+(1-A)*\overline{Z}_{+1}\right)\right)+{}\\
  {}+\Bigg[
  \sum\limits_{l\geq1} \alpha_{l+N+1}\prod\limits^N_{k=0} \left(1-
\alpha_{l+k}\right)\times{} \\
  {}\times
  q^\prime_{n,l} \left(0, \left\{ \prod\limits_{k=0}^{N+1} \left(1-
\alpha_{l+k}\right) z_{l+N+1}\right\}_{l\geq1}\right) \times{}\\
{}\times \prod\limits^N_{j=1} \beta_3 
\left( a-a\prod\limits_{k=1}^{N-j+2}\left( 1-\alpha_k\right) z_{N-j+1}\right)\times{}
  \end{multline*}
  
  \noindent
    \begin{multline*}
    {}\times
  b_3\left(a-a\left(1-\alpha_1\right)z_1\right) +\sum\limits^{N+1}_{l=1} 
  \alpha_{N-l+2} \times{}\\
  {}\times
  \prod\limits_{k=1}^{N-l+1}\left(1-\alpha_k\right) \beta_3^\prime \left( 
\prod\limits_{k=1}^{N-l+1}\left(1-\alpha_k\right) z_{N-j+1}\right)\times{}\\
  {}\times
q_n\left(0,\left\{ \prod\limits_{k=0}^{N+1} \left(1-\alpha_{l+k}\right) 
z_{l+N+1}\right\}\right)\times{}\\
{}\times
\prod\limits^N_{j=1,j\not=l} \beta_3 \left( a-
a\prod\limits_{k=1}^{N-j+2}\left( 1-\alpha_k\right) z_{N-j+1}\right)\times{}\\
{}\times
b_3\left( a-a\left(1-\alpha_1\right)z_1\right) +{}\\
{}+\alpha_1\prod\limits^N_{j=1} \beta_3 \left( 
\prod\limits_{k=1}^{N-j+2}\left( 1-\alpha_k\right) z_{N-j+1}\right)\times{}\\
{}\times b_3^\prime
\left( a-a\left( 1-\alpha_1\right)z_1\right)\Bigg]\Bigg\}\,.
\end{multline*}
  %
  Помножим данное соотношение на~$w^n$ и~просуммируем по $n\hm\geq1$. 
Получим:
  \begin{multline*}
  \fr{q(w,z,\overline{Z})-q_0(z,\overline{Z})}{w}={}\\
{}=\beta\left(
  a-a\left( \alpha_1 z+\left( 1-\alpha_1\right) z_1\right)\right)\times{}\\
  {}\times
  \Bigg\{
  \fr{q(w,z,Az+(1-A)*\overline{Z}_{+1})}{z}-{}\\
  {}-\fr{q(w,z,Az+(1-A)*\overline{Z}_{+1})}{z}+{}\\
  {}+\Bigg[
  \sum\limits_{l\geq0} q_l^\prime \Bigg( w,0,\left\{
  \prod\limits_{j=0}^{N+1} \left(1-\alpha_{l+j}\right)
   z_{l+N+1}\right\}_{l\geq1} \times{}\\
   {}\times \alpha_{l+N+1}\Bigg)
  \prod\limits^N_{k=0} \left(1-\alpha_{l+k}\right) 
  \times{}\\
  {}\times
  \prod\limits^N_{j=1} \beta_3 
\left( a-a\prod\limits_{k=1}^{N-j+2}\left(1-\alpha_k\right)
  z_{N-j+1}\right)\times{}\\
  {}\times b_3\left( a-a\left( 1-\alpha_1\right) z_1\right)+{}\\
  {}+q\left( w,0,\left\{ \prod\limits_{j=0}^{N+1} \left(1-\alpha_{l+j}\right) 
z_{l+N+1}\right\} \right)\times{}\\
{}\times
  \left( \sum\limits_{l=1}^{N+1} \alpha_{N-l+2} \beta_3^\prime \left(
  \prod\limits_{k=1}^{N-l+1}\left(1-\alpha_k\right) z_{N-
j+1}\right)\times{}\right.\\
\left.{}\times\prod\limits_{k=1}^{N-l+1}\left(1-\alpha_k\right)\right)
 b_3\left(a-a\left( 1-\alpha_1\right)z_1\right) \times{}\\
 {}\times \prod\limits^N_{j=1,j\not=l} 
\beta_3 \left(a-a\prod\limits_{k=1}^{N-j+2}\left(1-\alpha_k\right) 
z_{N-j+1}\right)+{}
\end{multline*}

\noindent
\begin{multline}
{}+
  \alpha_1 \prod\limits^N_{j=1} \beta_3 \left( \prod\limits_{k=1}^{N-j+2}\left( 1-
\alpha_k\right) z_{N-j+1}\right) \times{}\\
{}\times b_3^\prime \left(a-a\left(1-
\alpha_1\right)z_1\right)
  \vphantom{\prod^N_{l=1}}
  \Bigg]\Bigg\}\,,
  \label{e2-sim}
  \end{multline}
где $q_0(z,\overline{Z})$ есть производящая функция числа вызовов разных 
типов в~системе в~начальный момент времени. В~частности, если вызовов 
в~начальный момент нет, то $q_0(z,\overline{Z})\hm=1$. Заметим, что 
$q_0(z,\overline{Z})\hm=q(0,z,\overline{Z})$.
  
  Соотношение~(2) является достаточно сложным  
ин\-тег\-ро\-диф\-фе\-рен\-ци\-аль\-ным уравнением для на\-хож\-де\-ния функции 
$q(w,z,\overline{Z})$, описывающей распределение во времени числа 
невыявленных атак. В~настоящее время нет эффективных методов решения 
указанного уравнения. Однако соотношение~(2) позволяет разработать 
рекуррентные вы\-чис\-ли\-тель\-ные процедуры для нахождения 
$q(w,z,\overline{Z})$, которые предполагается привести в~последующих 
работах.
  
  \section{Заключение}
  
  В работе рассмотрена задача построения модели мониторинга систем 
обработки данных по показателям информационной безопас\-ности на основе 
использования методов теории массового обслуживания. Получены 
соотношения для следующих двух наиболее важных характеристик указанных 
систем: вероятности состояний сис\-те\-мы и~вероятности чис\-ла 
невыявленных вызовов в~моменты окончания обслуживания. Здесь под 
состоянием понимается чис\-ло вызовов, ожидающих обслуживания (т.\,е.\ атак, 
ожидающих нейтрализации), включая невыявленные атаки. Нахождение 
указанных характеристик позволит более эффективно организовать процесс 
выявления злоумышленных атак на сис\-те\-му обработки данных. 
  
{\small\frenchspacing
 {%\baselineskip=10.8pt
 \addcontentsline{toc}{section}{References}
 \begin{thebibliography}{9}
 
  \bibitem{2-sim}
  \Au{Грушо А.\,А., Грушо~Н.\,А., Тимонина~Е.\,Е.} Методы защиты информации от атак 
с~помощью скрытых каналов и~враждебных про\-граммно-ап\-па\-рат\-ных агентов 
в~распределенных системах~// Вестник РГГУ. Сер.: Документоведение и~архивоведение. 
Информатика. Защита информации и~информационная безопасность, 2009. №\,10. С.~33--45.

 \bibitem{1-sim}
  \Au{Kopyrin A.\,S., Simavoryan~S.\,Zh., Simonyan~A.\,R., Ulitina~E.\,I.} The methodology of 
risk analysis in assessing information security threats~// Modeling Artificial Intelligence, 2017. 
No.\,4-2. P.~78--85. doi: 10.13187/mai.2017.2.78.

  
  \bibitem{4-sim} %3
  \Au{Бажаев Н.\,А., Давыдов~А.\,Е., Кривцова~И.\,Е., Лебедев~И.\,С., 
Салахутдинова~К.\,И.} Подход к~анализу состояния информационной безопасности 
беспроводной сети~// Прикладная информатика, 2016. Т.~11. №\,6(66). С.~121--128.

\bibitem{3-sim} %4
  \Au{Коляденко Ю.\,Ю., Лукинов~И.\,Г.} Модель распределенных атак 
  в~программно-конфигурируемых сетях связи~// Вестник ЮУрГУ. Сер.: Компьютерные технологии, 
управление, радиоэлектроника, 2017. Т.~17. №\,3. С.~34--43. doi: 10.14529/ctcr170304.

  \bibitem{5-sim}
  \Au{Гнеденко Б.\,В., Даниелян~Э.\,А., Димитров~Б.\,Н., Климов~Г.\,П., Матвеев~В.\,Ф.} 
Приоритетные системы массового обслуживания.~--- М.: МГУ, 1973. 448~с.
 \end{thebibliography}

 }
 }

\end{multicols}

\vspace*{-3pt}

\hfill{\small\textit{Поступила в~редакцию 03.08.19}}

\vspace*{8pt}

%\pagebreak

%\newpage

%\vspace*{-28pt}

\hrule

\vspace*{2pt}

\hrule

%\vspace*{-2pt}

\def\tit{MODELING OF~MONITORING OF~INFORMATION SECURITY 
PROCESS ON~THE~BASIS OF~QUEUING SYSTEMS}


\def\titkol{Modeling of~monitoring of~information security 
process on~the~basis of~queuing systems}

\def\aut{G.\,A.~Popov$^1$, S.\,Zh.~Simavoryan$^2$, A.\,R.~Simonyan$^2$, 
and~E.\,I.~Ulitina$^2$}

\def\autkol{G.\,A.~Popov, S.\,Zh.~Simavoryan, A.\,R.~Simonyan, 
and~E.\,I.~Ulitina}

\titel{\tit}{\aut}{\autkol}{\titkol}

\vspace*{-11pt}


\noindent
   $^1$Astrakhan State Technical University, 16~Tatischeva Str., Astrakhan 414056, 
Russian Federation
  
  \noindent
  $^2$Sochi State University, 94~Plastunskaya Str., Sochi 354003, Russian 
Federation

\def\leftfootline{\small{\textbf{\thepage}
\hfill INFORMATIKA I EE PRIMENENIYA~--- INFORMATICS AND
APPLICATIONS\ \ \ 2020\ \ \ volume~14\ \ \ issue\ 1}
}%
 \def\rightfootline{\small{INFORMATIKA I EE PRIMENENIYA~---
INFORMATICS AND APPLICATIONS\ \ \ 2020\ \ \ volume~14\ \ \ issue\ 1
\hfill \textbf{\thepage}}}

\vspace*{3pt} 
  
  
    
\Abste{The paper is devoted to the mathematical modeling of monitoring process 
by the information security systems, aimed at detection of hidden malicious 
attacks. The modeling is based on the queueing theory formalism. The monitoring 
process is reduced to the analysis of the customer flow arriving at the queueing 
system, in which each customer is regarded as carrying potential malicious 
attacks. Functional relations between the system state probability distribution and 
the distribution of the number of undetected malicious attacks on service 
completion epochs are obtained. These characteristics may allow one to improve 
the efficiency of malicious attacks detection process in the data processing 
systems.}

\KWE{protection of information; information security; queuing system; 
probability}
  
  


\DOI{10.14357/19922264200110} 

%\vspace*{-14pt}

\Ack
  \noindent
  The reported study was funded by the Russian Foundation for Basic Researh, 
project No.\,19-01-00383.

 


%\vspace*{6pt}

  \begin{multicols}{2}

\renewcommand{\bibname}{\protect\rmfamily References}
%\renewcommand{\bibname}{\large\protect\rm References}

{\small\frenchspacing
 {%\baselineskip=10.8pt
 \addcontentsline{toc}{section}{References}
 \begin{thebibliography}{9}
  
  \bibitem{2-sim-1}
  \Aue{Grusho, A.\,A., N.\,A.~Grusho, and E.\,E.~Timonina}. 2009. Metody 
zashchity informatsii ot atak s pomoshch'yu skrytykh kanalov i~vrazhdebnykh 
programmno-apparatnykh agentov v~raspredelennykh sistemakh [Methods of 
information protection aganst covert channels attacks and 
malicious software/hardware agents in distributed systems]. 
\textit{Vestnik RGGU. Ser. Dokumentovedenie i~arkhivovedenie. Informatika. 
Zashchita informatsii i~informatsionnaya bezopasnost'} [RGGU Bulletin. Informatics. Information 
security. Mathematician ser.] 10:33--45. 

\bibitem{1-sim-1}
  \Aue{Kopyrin, A.\,S., S.\,Zh.~Simavoryan, A.\,R.~Simonyan, and E.\,I.~Ulitina.} 
2017. The methodology of risk analysis in assessing information security threats. 
\textit{Modeling Artificial Intelligence} 4-2:78--85.

  
  \bibitem{4-sim-1} %3
  \Aue{Bazhayev, N.\,A., A.\,E.~Davydov, I.\,E.~Krivtsova, I.\,S.~Lebedev, and 
K.\,I.~Salakhutdinova.} 2016. Podkhod k~analizu so\-sto\-yaniya informatsionnoy 
bezopasnosti besprovodnoy seti [Wireless security information analysis approach]. 
\textit{Prikladnaya informatika} [J.~Applied Informatics] 11(6(66)):121--128.

\bibitem{3-sim-1} %4
  \Aue{Kolyadenko Yu.\,Yu., and I.\,G.~Lukinov}. 2017. Model' raspredelennykhatak 
v~programmno-konfiguriruemykh setyakh svyazi [Model of distributed attacks in 
program-configurable communication networks]. \textit{Vestnik YUUrGU. Ser. 
Komp'yuternye tekhnologii, upravleniye, radioelektronika} [Bulletin of SUSU. 
Computer technologies, automatic control, radioelectronics ser.] 17(3):34--43.

  \bibitem{5-sim-1}
  \Aue{Gnedenko, B.\,V., E.\,A.~Danielyan, B.\,N.~Dimitrov, G.\,P.~Klimov, and 
V.\,F.~Matveev.} 1973. \textit{Prioritetnyye sistemy massovogo obsluzhivaniya} 
[Priority queues]. Moscow: MSU. 448p.
\end{thebibliography}

 }
 }

\end{multicols}

%\vspace*{-7pt}

\hfill{\small\textit{Received August 3, 2019}}

%\pagebreak

%\vspace*{-22pt}
  
  \Contr
  
  \noindent
  \textbf{Popov Georgy A.} (b.\ 1950)~--- Doctor of Science in technology, 
professor, Head of Department, Astrakhan State Technical University, 16~Tatischeva 
Str., Astrakhan 414056, Russian Federation; \mbox{popov@astu.org}
  
  \vspace*{6pt}
  
  \noindent
  \textbf{Simavoryan Simon Zh.} (b.\ 1958)~--- Candidate of Science (PhD) in 
technology, associate professor, Sochi State University, 94~Plastunskaya Str., Sochi 
354003, Russian Federation; \mbox{simsim58@mail.ru}
  
  \vspace*{6pt}
  
  \noindent
  \textbf{Simonyan Arsen R.} (b.\ 1960)~--- Candidate of Science (PhD) in physics 
and mathematics, associate professor, Sochi State University, 94~Plastunskaya Str., 
Sochi 354003, Russian Federation; \mbox{oppm@mail.ru}
  
  \vspace*{6pt}
  
  \noindent
  \textbf{Ulitina Elena I.} (b.\ 1978)~--- Candidate of Science (PhD) in physics and 
mathematics, associate professor, Sochi State University, 94~Plastunskaya Str., 
Sochi 354003, Russian Federation; \mbox{ulitinaelena@mail.ru}
\label{end\stat}

\renewcommand{\bibname}{\protect\rm Литература} 