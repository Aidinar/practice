\def\stat{serebr}

\def\tit{ПОВЫШЕНИЕ ТОЧНОСТИ РЕШЕНИЯ ОБРАТНЫХ ЗАДАЧ ЗА~СЧЕТ~УТОЧНЕНИЯ 
ГРАНИЧНЫХ УСЛОВИЙ$^*$}

\def\titkol{Повышение точности решения обратных задач за~счет уточнения 
граничных условий}

\def\aut{С.\,М.~Серебрянский$^1$, А.\,Н.~Тырсин$^2$}

\def\autkol{С.\,М.~Серебрянский, А.\,Н.~Тырсин}

\titel{\tit}{\aut}{\autkol}{\titkol}

\index{Серебрянский С.\,М.}
\index{Тырсин А.\,Н.}
\index{Serebryanskii S.\,M.}
\index{Tyrsin A.\,N.}


{\renewcommand{\thefootnote}{\fnsymbol{footnote}} \footnotetext[1]
{Работа выполнена при финансовой поддержке РФФИ (проект 20-41-660008~р\_а).}}


\renewcommand{\thefootnote}{\arabic{footnote}}
\footnotetext[1]{Троицкий филиал Челябинского государственного университета, 
\mbox{tf\_chelgu@mail.ru}}
\footnotetext[2]{Научно-инженерный центр <<Надежность и~ресурс больших сис\-тем 
и~машин>> УрО РАН; Уральский федеральный университет имени первого 
Президента России Б.\,Н.~Ельцина, \mbox{at2001@yandex.ru}}

%\vspace*{-12pt}



  \Abst{Рассматриваются вопросы устойчивости решения обратных задач относительно 
точного задания граничных условий. В~практических приложениях, как правило, 
теоретический вид функциональной зависимости граничных условий не определен или 
неизвестен, а также присутствуют случайные по\-греш\-ности измерений. Исследования 
показали, что это приводит к~существенному снижению точ\-ности решения обратной задачи. 
С~целью повышения точ\-ности решения обратных задач предложено уточнять 
функциональный вид граничных условий с~помощью распознавания вида математической 
модели зависимости с~последующей аппроксимацией этой функцией поведения физической 
величины на границе. Восстановление вида зависимости выполнено методами распознавания 
зависимостей на основе структурных разностных схем и~распознавания на основе обратного 
отображения. Приведены модельные примеры реализации в~условиях присутствия 
аддитивных случайных погрешностей измерений и~неизвестного вида зависимости 
граничных условий.}
  
  \KW{обратная задача; распознавание; функциональная зависимость; модель; разностная 
схема; обратная функция; выборка; дисперсия; аппроксимация}

\DOI{10.14357/19922264200108} 
  
\vspace*{-3pt}


\vskip 10pt plus 9pt minus 6pt

\thispagestyle{headings}

\begin{multicols}{2}

\label{st\stat}
  
\section{Введение}

  Исследование так называемых обратных задач, когда исходя из некоторых 
характеристик физического поля необходимо восстановить характеристики 
самой среды, порождающей это поле, можно описать операторным уравнением 
1-го рода~[1]:
  $$
  Au=f\,.
  $$
  
  Трудности, возникающие при исследовании \mbox{таких} уравнений, связаны, 
главным образом, с~неограниченностью обратного оператора~$A$ 
и~отсутствием непрерывной зависимости решения от \mbox{правой} части 
(неустойчивость или некорректность задачи). 
  
  Обычные методы, используемые для приближенного решения корректных 
задач, оказываются, как правило, непригодными. Для эффективного решения 
неустойчивых задач к~настоящему времени созданы специальные регулярные 
методы, основанные на замене исходной некорректной (неустойчивой) задачи 
задачей или последовательностью задач, корректных в~обычном смысле~[2].
  
  При решении таких задач могут возникать возможные ошибки из-за 
неточного задания вида функциональных зависимостей, описывающих 
граничные условия, а также наличия случайных погрешностей измерений.
  
  Рассматривается обратная задача для уравнения теплопроводности 
с~ненулевыми граничными условиями~[3]. Граничные условия заданы 
приближенно с~некоторой погрешностью, поэтому определение вида 
функциональной зависимости в~них имеет большой практический смысл. Как 
показывает практика, достаточно некоторой ошибки в~начальных данных для 
нарушения устойчивости решения задачи.
  
  Цель исследования~--- попытка преодоления этого недостатка за счет 
восстановления граничных условий задачи с~помощью методов 
идентификации~[4--10].

\vspace*{-6pt}


\section{Постановка задачи}

  Рассмотрим задачу
  \begin{equation}
  \fr{\partial u_1(x,t)}{\partial t} =\fr{\partial^2 u_1(x,t)}{\partial x^2}\,,\enskip 
0\leq x\leq x_2\,,\ t\geq0\,;\label{e1-sr}
  \end{equation}
  \begin{equation}
  \fr{\partial u_2(x,t)}{\partial t} =\kappa\fr{\partial^2 u_2(x,t)}{\partial 
x^2}\,,\enskip x_2<x\leq 1\,,\ t\geq 0\,;\label{e2-sr}
  \end{equation}
  \begin{equation}
  u_1(x,0)=0\,,\ 0\leq x\leq x_2\,;\label{e3-sr}
  \end{equation}
  \begin{equation}
  u_2(x,0)=0\,,\ x_2\leq x\leq1\,;\label{e4-sr}
  \end{equation}
      \begin{equation}
  u_1(0,t)=f_\delta(t)\,,\ t\geq 0\,;
  \label{e5-sr}
  \end{equation}
  \begin{equation}
  u_1(x_1,t)=\psi_\delta(t)\,,\ 0<x_1<x_2\,,\ t\geq0\,;
  \label{e6-sr}
  \end{equation}
  \begin{equation}
  u_1(x_2,t)=u_2(x_2,t)\,,\ t\geq 0\,;
  \label{e7-sr}
  \end{equation}
  \begin{equation}
  \lambda_1\fr{\partial u_1(x_2,t)}{\partial x} =\lambda_2 \fr{\partial 
u_2(x_2,t)}{\partial x}\,,\ t\geq0\,.
  \label{e8-sr}
  \end{equation}
Здесь $x$~--- пространственная координата; $t$~--- время; $u_i(x,t)$~--- значение 
температуры в~точке~$x$ в~момент времени~$t$ в~$i$-м слое; 
$\kappa\hm=\kappa_1/\kappa_2$, где $\kappa_1\hm= \lambda_1/(\rho_1c_1)$, 
$\kappa_2\hm= \lambda_2/(\rho_2 c_2)$, $c_i$~--- теплоемкость в~$i$-м слое, 
$\rho_i$~--- плотность материала в~$i$-м слое, $\lambda_i$~--- коэффициент 
теплопроводности $i$-го слоя; $f_\delta$~--- значение температуры в~точке~0, 
рассчитанное в~результате эксперимента, при этом если бы было известно 
точное значение~$f_0$ температурного распределения в~этой точке, то 
выполнялось бы условие $\| f_\delta (t) \hm- f_0(t)\| \hm\leq \delta$, 
где~$\delta$~--- 
погрешность измерений; $\psi_\delta$~--- значение температуры 
в~промежуточной точке~$x_1$, рассчитанное в~результате эксперимента, при 
этом если бы было известно точное значение~$\psi_0$ температурного 
распределения в~этой точке, то выполнялось бы условие $\| \psi_\delta(t)\hm- 
\psi_0(t)\| \hm\leq \delta$, где $\delta$~--- погрешность измерений.
  
  Физическая картина смоделированных процессов восстановления функции 
температуры пред\-став\-ле\-на на рис.~1.
  

  
  Требуется, используя исходную информацию $f_\delta$, $\psi_\delta$ и~$\delta$ 
задачи~(1)--(8), рассчитать значения~$u_{2\delta}(1,t)$, наиболее близкие по 
норме значениям~$u_{20}(1,t)$.
  
  Задача (1)--(8) считается некорректно по\-став\-лен\-ной и~требует специального 
решения~\cite{2-sr, 11-sr, 12-sr, 13-sr, 14-sr}. Ее решение подробно 
рассматривалось в~[3]. В~результате была получена точная по порядку оценка
 \begin{multline*}
  \!\sqrt{\!\| u_{2\varepsilon}(1,t)\!-\!u_{20}(1,t)\|^2\!+ \!\left \| 
  v_{2\varepsilon}(1,t)\!-\!
\fr{\partial u_{20}(1,t)}{\partial x}\right\|^2}\! \leq\\
{}\leq \sqrt{2}\,l_2\ln^{-
2}\fr{1}{\varepsilon}\,,
\end{multline*}
  где 
  $$
  \varepsilon=\fr{\lambda_1}{\lambda_2} \,\delta \left[  \sqrt{2}\,r^{2/3}\left[ 
\fr{1+e^{-(x_1/\sqrt{2}}}{1-e^{-\sqrt{2}\,x_1}}\right] 
+\fr{4}{\sqrt{2}\,x_1}\right]\,,
  $$
  а $r\hm=const$~--- радиус на множестве~$M_r$, являющемся классом 
корректности.

{ \begin{center}  %fig1
 \vspace*{9pt}
    \mbox{%
 \epsfxsize=76.504mm 
 \epsfbox{ser-1.eps}
 }


\end{center}


\noindent
{{\figurename~1}\ \ \small{Восстановление функции температуры в~композиционных материалах}}
}

%\vspace*{3pt}

%\vspace*{-6pt}

\section{Численная реализация}

  В~\cite{15-sr} приведено численное решение задачи~(1)--(8), которое 
выглядит следующим образом:

\noindent
  \begin{multline}
  u\left( 1,\overline{t}\right) \approx\psi_0(\overline{t})+\left( 
  \fr{\psi_0(\overline{t})-f_0(\overline{t})}{x_1}\right) \left(1-x_1\right) 
+{}\\
{}+\fr{1}{2}\,\psi_0^\prime (\overline{t}) \left(1-x_1\right)^2\kappa\,.
  \label{e9-sr}
  \end{multline}
  
  Если вместо точных значений~$f_0(\overline{t})$ и~$\psi_0(\overline{t})$ 
даны приближения, то использование~(\ref{e9-sr}), очевидно, приведет 
к~неверному результату.
  
  Для решения этой проблемы в~\cite{15-sr} подробно рассмотрена методика, 
построенная на выполнении условий невязки и~нахождении параметра 
регуляризации~$\alpha$, согласованного с~погрешностью задания входных 
данных.
  
  В качестве входных данных для численной реализации задачи~(1)--(8) были 
использованы: шаг по координате $\Delta x\hm= 0{,}1$; длина отрезка по 
координате $lx\hm=1$; длина отрезка по времени $lt\hm=10$; параметр 
регуляризации $\alpha\hm=0{,}05$; погрешности исходных данных $\delta\hm= 
0{,}01$, 0,03, 0,05 и~0,1; исходная функция $u(1,t)\hm=te^{-t}$.
  
  Экспериментально полученные результаты представлены на рис.~2.
  
  Аппроксимация проводилась с~помощью схожих по свойствам моделей: 
$\overline{u}(1,t)\hm= 1{,}1 te^{-t}$; $\overline{u}(1,t)\hm= t^\alpha e^{-t}$ 
и~$\overline{u}(1,t)\hm= t\cdot 3^{-\alpha t}$.
  



  Результаты вычислений для модели $u(1,t)\hm= te^{-t}$ с~различными 
погрешностями представлены в~табл.~1. Погрешность~$\delta$ распределена 
равномерно с~нулевым математическим ожиданием. Выделим в~качестве 
характеристики точности среднеквадратичную ошибку между точным 
и~приближенным решением, вычисляемую по формуле: 
  $$
  S_n^1 = \sqrt{\fr{1}{n}\sum\limits^n_{i=1} \left( u_i -\overline{u}_i\right)^2}\,.
  $$
  

  
  Наличие ошибок неизбежно приводит к~понижению устойчивости решения 
задачи, что наглядно представлено в~табл.~1. Необходима более точная 
идентификация граничных условий задачи.
  
  Для идентификации вида зависимостей в~граничных  
условиях~(\ref{e5-sr})--(\ref{e6-sr}) были использованы следующие методы:
  \begin{enumerate}[(1)]
\item метод распознавания зависимостей на основе структурных разностных 
схем (РС)~\cite{4-sr};
\item метод распознавания зависимостей на основе обратного 
отображения~\cite{5-sr} (расширение области применения метода 
распознавания на основе структурных РС).
\end{enumerate}

{ \begin{center}  %fig2
 \vspace*{-9pt}
    \mbox{%
 \epsfxsize=79mm 
 \epsfbox{ser-2.eps}
 }


\end{center}

\vspace*{-4pt}


\noindent
{{\figurename~2}\ \ \small{Аппроксимация моделями $\overline{u}(1,t)\hm=1{,}1 te^{-t}$ 
(погрешность 
$\delta\hm=0{,}01$)~(\textit{а}),
$\overline{u}(1,t)\hm=t^\alpha e^{-t}$~(\textit{б}) 
и~$\overline{u}(1,t)\hm=t\cdot 3^{-\alpha t}$~(\textit{в}): 
\textit{1}~--- приближенное решение; \textit{2}~--- точное решение}}
}

\vspace*{12pt}

\addtocounter{figure}{1}

  Результаты апробации на модельных данных, выполненной методом 
статистических испытаний~\cite{16-sr, 17-sr}, показали достаточно высокую 
достоверность распознавания. В~результате распознавания функциональных 
зависимостей $f_\delta(t)$ и~$\psi_\delta(t)$ в~граничных  
условиях~(\ref{e5-sr})--(\ref{e6-sr}) можно получить более полную картину 
распространения тепла в~ма-\linebreak\vspace*{-12pt}

%{\begin{table*}\small %tabl1

\begin{center}

\noindent
\parbox{174pt}{{{\tablename~1}\ \ \small{Среднеквадратичная ошибка при различных погрешностях}}
}

\vspace*{6pt}

{\small
\begin{tabular}{|c|c|}
\hline
&\\[-9pt]
Погрешность $\delta$&\tabcolsep=0pt\begin{tabular}{c}Среднеквадратичная\\
 ошибка $S^1_n$\end{tabular}\\
\hline
0,01&0,04\\
0,03&0,07\\
0,05&0,12\\
0,10&0,37\\
\hline
\end{tabular}
}
\end{center}
%\end{table*}}

\vspace*{12pt}

\addtocounter{table}{1}

\noindent
териалах и~тем самым повысить точность решения 
задачи~(\ref{e1-sr})--(\ref{e8-sr}).


  
  
\section{Примеры повышения точности численного решения обратной 
задачи за~счет распознавания зависимостей в~граничных условиях}
 
  Для распознавания зависимостей на основе структурных 
  РС~\cite{5-sr, 18-sr, 19-sr} рассмотрим в~табл.~2 некоторый класс моделей 
с~соответствующими им разностными схемами, которые могут быть взяты 
в~качестве граничных условий задачи~(1)--(8).

{\begin{table*}\small %tabl2
\begin{center}
\Caption{Класс моделей с~соответствующими им РС}
\vspace*{2ex}

\begin{tabular}{|c|l|c|}
\hline
№ модели &
\multicolumn{1}{c|}{Зависимость} &
Структурная РС\\
\hline
&&\\[-9pt]
1 &$y_k=A+B\sin (\omega t_k+\varphi)$ &
$\displaystyle y_k=a_1\fr{y_{k-1}(y_{k-1}+y_{k-3})}
{y_{k-2}}+a_2y_{k-2}$, $a_1=1$, $a_2=-1$\\
 &&\\[-9pt]
\hline
&&\\[-9pt]
2 &
$y_k=A+B\ln (Ct_k)$ &
\tabcolsep=0pt\begin{tabular}{c}
$\displaystyle
y_k=a_1b_k^{(2)} y_{k-1} +a_2 c_k^{(2)} y_{k-2}\,,\ a_1=1,\ a_2=-1,\ C>0\,,$\\
$b_k^{(2)}=\fr{\ln (k/(k-2))}{\ln ((k-1)/(k-2))}\,,\ c_k^{(2)}=
\fr{\ln (k/(k-1))}{\ln ((k-1)/(k-2))}$ \end{tabular}  \\
 &&\\[-9pt]
 \hline
 &&\\[-9pt]
3 & $\displaystyle  y_k=\fr{At_k^2+Bt_k}{t_k^2+C}$ &
\tabcolsep=0pt\begin{tabular}{c}
$\displaystyle v_k=a_1v_{k-1}+a_2v_{k-2}\,,\ a_1=2\,,\ a_2=-1\,,$\\
$\displaystyle v_k=2\Delta 
\fr{y_k[(k-1)(2k-2)y_{k-2}-(k-2)(2k-1)y_{k-1}]}
{(k-1) (k-2)y_k-2k(k-2)y_{k-1}+k(k-1)y_{k-2}}$
\end{tabular}
 \\
 &&\\[-9pt]
 \hline
 &&\\[-9pt]
4 & $\displaystyle y_k=A\exp \left\{-Be^{-\alpha t_k}\right\}$&
$\displaystyle v_k=a_1v_{k-1}+a_2\fr{ (v_{k-1} - 
v_{k-2})^2}{v_{k-2}-v_{k-3}}$, $a_1=1$, 
$a_2=1$, $v_k=\ln y_k$\\
\hline
%&&\\[-9pt]
5 &
$y_k=A+Be^{\alpha t_k}$ &
\tabcolsep=0pt
\begin{tabular}{c}$v_k=a_1v_{k-1}+a_2v_{k-2}$, 
$a_1=2$, $a_2=-1$, \\
$v_k=\ln [ r(y_k-y_{k-1})]$, 
$r=\mathrm{sign}\left(B\alpha\right)$\end{tabular}\\
\hline
%&&\\[-9pt]
6 & $y_k=A+B t_k$ &
$y_k=a_1y_{k-1}+a_2y_{k-2}$, $a_1=2$, 
$a_2=-1$\\
\hline
\end{tabular}
\end{center}
\end{table*}}
  
  \textbf{Пример~1.}\ Пусть задана последовательность
  $$
  y_k=f(k)=A+Bt_k=A+B\Delta k\,,\ n_0\leq k\leq n_1\,,
  $$
где $A=5$, $B = 3$, $\sigma_\varepsilon\hm=0{,}02$, $\Delta\hm=1$, 
$n_0\hm=6$ и~$n_1\hm=100$.
  
  Идентифицируем каждую из зависимостей 
  $$
  \overset{\smallfrown}{f}_j(k)=  f_j(k)\hm+\varepsilon_k
$$ 
по экспериментальным данным, т.\,е.\ вычислим 
оценки ее параметров~$\overset{\smallfrown}{\mathbf{b}}^j$, найденные методом 
наименьших квадратов, которые являются состоятельными и~несмещенными 
($\varepsilon_k$~--- случайные ошибки, имеющие нормальное распределение 
$N(0,\sigma_\varepsilon^2)$).
  
  Используя определение обратного отображения, для каждой 
$\overset{\smallfrown}{f}_j(k)$ построим последовательность 
$\{\overset{\smallfrown}{u}_k^j\}$, элементы которой $\overset{\smallfrown}{u}_k^j \hm= 
\overset{\smallfrown}{f}_j^{-1}(y_k)$. Отсюда, построив для последовательности 
$\mathbf{u}\hm= \{u_k\}$, элементы которой $u_k\hm= f^{-1}[f(k)]\hm= k$, РС 
2-го порядка, получим:
  $$
  u_k=a_1 u_{k-1} +a_2 u_{k-2}\,,\enskip
  \mathbf{a}^\circ = \left(a_1,a_2\right)=(2,-1)\,,
  $$
т.\,е.\ отображение функции~$f$ в~точку~$\mathbf{a}^\circ\hm=(2,-1)$. Таким 
образом, $\mathbf{u}\overset{F_p}{\to} \mathbf{a}^\circ$.

%\linebreak\vspace*{-12pt}

%{\begin{table*}\small %tabl3

\begin{center}
\vspace*{-1pt}

\noindent
\parbox{150pt}{{{\tablename~3}\ \ \small{Результаты идентификации модели $y_k\hm= A\hm+Bt_k$}}
}

\vspace*{6pt}

{\small
\begin{tabular}{|c|c|}
\hline
№ модели&Расстояние до ОДЗ\\
\hline
1&0,0355\\
2&0,0238\\
3&0,0238\\
4&0,0373\\
5&0,0238\\
6&0,0076\\
\hline
\end{tabular}
\vspace*{6pt}
}
\end{center}
%\end{table*}}

%\vspace*{12pt}

\addtocounter{table}{1}

%\begin{table*}\small %tabl4
\begin{center}
\noindent
\parbox{231pt}{{{\tablename~4}\ \ 
\small{Набор моделей и~их обратные преобразования}}
}

\vspace*{6pt}

{\small 
\tabcolsep=1pt
\begin{tabular}{|c|c|}
\hline
&\\[-9pt]
Модель $u=F_j(t)$ &Обратное преобразование $t=F_j^{-1}(u)$\\
&\\[-9pt]
\hline
&\\[-9pt]
$u=te^{-t}$&$-W(-t)$, $W(\cdot)$~--- функция Ламберта\\
\hline
&\\[-9pt]
$u=t^\beta e^{-\alpha t}$&$\displaystyle -\fr{1{,}2W(-0{,}83333\alpha^{5/6} t)}{t}$\\
\hline
&\\[-9pt]
$u=ta^{-\alpha t}$&$\displaystyle -\fr{W(\alpha a \ln(t))}{\alpha\ln(t)}$ \\
&\\[-9pt]
\hline
&\\[-9pt]
$u=1-e^{-(t/\lambda)^k}$&$\lambda \sqrt[t]{-\ln (1-k)}$ \\
\hline
\end{tabular}
}
\end{center}
%\end{table*}

\vspace*{3pt}



  
  Если $\overset{\smallfrown}{f}_j\not= f$, то 
  
  \noindent
  $$
  u_k^j= \overset{\smallfrown}{f}_j^{-
1}(y_k)\overset{F_p}{\to} \left( a_1^j, a_2^j\right) = \mathbf{a}^j\not= 
\mathbf{a}^\circ\,.
$$

\vspace*{-9pt}
  
  Поставим в~соответствие каждой функции~$\overset{\smallfrown}{f}_j$ 
функционал вида
  $$
  d(\mathbf{y},f_j) =
  \rho_{{R}} \left( F\left[ f_l^{-1}(\mathbf{y})\right]\!,
 F\left[ f^{-1}(\mathbf{y})\right]\right) =
\sqrt{\left( \mathbf{a}^j, \mathbf{a}^\circ\right)}\,.
  $$ 
  
  Выражение представляет собой евклидово расстояние. В~итоге получим 
последовательность евклидовых расстояний для выбора наилучшей модели по 
критерию минимума расстояния до области допустимых значений (ОДЗ) 
вектора оценок коэффициентов АР-мо\-де\-лей (авторегрессионных моделей). 
Результаты идентификации моделей приведены в~табл.~3.


  



  Из табл.~3 видно, что расстояние до ОДЗ в~случае модели $y_k\hm= 
A\hm+Bt_k$ существенно меньше, чем для остальных моделей~\cite{20-sr}. 
Поэтому в~данном примере, безусловно, лучшей оказалась фактическая модель 
$y_k\hm= A\hm+Bt_k$.
  
  Для распознавания зависимостей на основе обратного  
отображения~\cite{15-sr} рассмотрим в~табл.~4 некоторый набор моделей 
заданного конечного
 множества моделей и~их обратные преобразования, 
которые могут быть взяты в~качестве граничных условий задачи~(1)--(8).

%\begin{table*}\small %tabl5
\begin{center}
\vspace*{-6pt}

\noindent
\parbox{184pt}{{{\tablename~5}\ \ 
\small{Среднеквадратичная ошибка с~аддитивным шумом}}
}

\vspace*{6pt}

{\small
\begin{tabular}{|c|c|}
\hline
Аддитивный шум $\delta$&\tabcolsep=0pt\begin{tabular}{c}Среднеквадратичная\\
 ошибка $S_n^2$\end{tabular}\\
\hline
0,01&0,0015\\
0,03&0,0027\\
0,05&0,0035\\
0,10&0,0080\\
\hline
\end{tabular}
}
\end{center}
%\end{table*}

%\vspace*{12pt}

{ \begin{center}  %fig5
% \vspace*{-3pt}
    \mbox{%
 \epsfxsize=79mm 
 \epsfbox{ser-5.eps}
 }


\end{center}

\vspace*{-9pt}


\noindent
{{\figurename~3}\ \ \small{Аппроксимация моделью $\overline{u}(1,t)=t^{1{,}05} e^{-0{,}95t}$: 
\textit{1}~--- приближенное решение; \textit{2}~--- точное решение}}
}

%\vspace*{9pt}

\setcounter{table}{5}
  \begin{table*}\small %tabl6
  %\vspace*{-12pt}
  \begin{center}
  \Caption{Сравнение среднеквадратичных ошибок при наличии аддитивного шума}
  \vspace*{2ex}
  
  \begin{tabular}{|c|c|c|c|}
  \hline
\tabcolsep=0pt\begin{tabular}{c}Аддитивный\\ шум, $\delta$\end{tabular}&
\tabcolsep=0pt\begin{tabular}{c}Среднеквадратичная\\ ошибка, $S_n^1$\end{tabular}&
\tabcolsep=0pt\begin{tabular}{c}Среднеквадратичная\\ 
ошибка, $S_n^2$\end{tabular}&
\tabcolsep=0pt\begin{tabular}{c}Отношение\\ среднеквадратичных\\ 
ошибок, $S_n^1/S_n^2$\end{tabular}\\
\hline
0,01&0,04&0,0015&26,66\\
0,03&0,07&0,0027&25,92\\
0,05&0,12&0,0035&34,28\\
0,10&0,37&0,0080&46,25\\
\hline
\end{tabular}
\end{center}
\end{table*}

  

  В процессе применения предлагаемого метода распознается функциональная 
зависимость с~различными параметрами и~производится ее сглаживание.
  
  \textbf{Пример~2.}\ Пусть задана модель $u\hm=t^\beta e^{-\alpha t}$, где 
$\alpha\hm=\beta\hm=1$.
  
  Результаты вычислений~\cite{15-sr} для рассматриваемой модели 
с~аддитивным шумом представлены в~табл.~5. 
  

  
  Подбор оптимальных параметров аппроксимируемой модели выполнялся по 
сетке в~пределах: $\alpha\hm= [0{,}7;1{,}3]$, $\beta\hm= [0{,}7;1{,}3]$ 
с~шагом~0,05.



  
  На рис.~3 представлена графическая интерпретация проведенных 
вычислений.


  
  В табл.~6 показано сравнение среднеквадратичных ошибок~$S_n^1$ 
и~$S_n^2$, полученных регулярным методом с~параметром 
регуляризации~$\alpha$, результаты которого представлены в~табл.~1, 
и~методом распознавания на основе обратного отображения, которое 
показывает, насколько точнее происходит распознавание модели.
  
    \vspace*{-9pt}

\section{Заключение}

  В статье были рассмотрены вопросы, связанные с~устойчивостью решения 
обратных задач относительно правильного задания граничных условий. 

Показано, что наличие случайных погрешностей измерений и~неправильное 
задание функциональной зависимости граничных условий приводит 
к~значительному снижению точности решения обратной задачи. 

Для решения 
этой проблемы предложено предварительно идентифицировать вид 
функциональной зависимости граничных условий и~затем\linebreak аппроксимировать 
результаты измерений на границе данной моделью. Для идентификации вида 
функциональной зависимости можно использовать предложенные методы 
распознавания. Результаты апробации на модельных данных, выполненные 
методом статистических испытаний, показали значительное повышение 
точности решения обратной задачи за счет идентификации вида 
функциональной зависимости граничных условий.
  
  
  \vspace*{-6pt}
  
{\small\frenchspacing
 {%\baselineskip=10.8pt
 \addcontentsline{toc}{section}{References}
 \begin{thebibliography}{99}
    \bibitem{1-sr}
    \Au{Иванов В.\,К., Мельникова~И.\,В., Филинков~А.\,И.}  
Диф\-фе\-рен\-ци\-аль\-но-опе\-ра\-тор\-ные уравнения и~некорректные 
задачи.~--- М.: Физматлит, 1995. 176~с.
    \bibitem{2-sr}
    \Au{Иванов В.\,К., Васин~В.\,В., Танана~В.\,П.} Теория линейных 
некорректных задач и~её приложения.~--- М.: Наука, 1978. 206~с.
    \bibitem{3-sr}
    \Au{Серебрянский С.\,М.} Об оценках погрешности методов 
приближенного решения одной обратной задачи~// Сибирский 
ж.~индустриальной математики, 2010. №\,2(42). С.~135--148.
   
   
    \bibitem{6-sr} %4
    \Au{Налимов В.\,В.} Теория эксперимента.~--- М.: Наука, Физматлит, 1971. 
208~с.
    \bibitem{7-sr} %5
    \Au{Айвазян С.\,А., Енюков~И.\,С., Мешалкин~Л.\,Д.} Прикладная 
статистика: исследование зависимостей.~--- М.: Финансы и~статистика, 1985. 
487~с.
    \bibitem{8-sr} %6
    \Au{Клейнер Г.\,Б., Смоляк~С.\,А.} Эконометрические зависимости: 
принципы и~методы построения.~--- М.: Наука, 2000. 104~с.
 \bibitem{10-sr} %7
    \Au{Тырсин А.\,Н.} Идентификация зависимостей на основе моделей 
авторегрессии~// Автометрия, 2005. Т.~41. №\,1. С.~43--49.
    \bibitem{9-sr} %8
    \Au{Орлов А.\,И.} Прикладная статистика.~--- 2-е изд., перераб. и~доп.~--- 
М.: Экзамен, 2007. 671~с.

 \bibitem{4-sr} %9
    \Au{Тырсин А.\,Н., Серебрянский~С.\,М.} Распознавание зависимостей во 
временных рядах на основе структурных разностных схем~// Автометрия, 2015. 
Т.~51. №\,2. C.~54--60.

 \bibitem{5-sr} %10
    \Au{Тырсин~А.\,Н., Серебрянский~С.\,М.} Распознавание типа зависимости 
на основе обратного отображения~// Информатика и~её применения, 2016. 
Т.~10. Вып.~2. С.~58--64.
   
    
    \bibitem{12-sr} %11
    \Au{Тихонов А.\,Н.} Об устойчивости обратных задач~// Докл. Акад. наук СССР, 1943. 
Т.~39. №\,5. С.~195--198.
    \bibitem{13-sr} %12
    \Au{Танана В.\,П.} Методы решения операторных уравнений.~--- М.: 
Наука, 1981. 160~с.
\bibitem{11-sr} %13
    \Au{Алифанов О.\,М.} Обратные задачи теплообмена.~--- М.: 
Машиностроение, 1988. 279~с.
    \bibitem{14-sr}
    \Au{Танана В.\,П., Япарова~Н.\,М.} Об оптимальном по порядку методе 
решения условно-корректных задач~// Сибирский ж.~вычислительной 
математики, 2006. Т.~9. №\,4. С.~353--368.
    \bibitem{15-sr}
    \Au{Серебрянский С.\,М., Тырсин~А.\,Н.} Повышение точ\-ности решения 
обратных задач при ошибках в~начальных данных~// Вестник Бурятского 
государственного университета. Математика, информатика, 2018. Вып.~4. 
С.~58--71. 
   
    \bibitem{17-sr} %16
    \Au{Ермаков С.\,Е.} Метод Мон\-те-Кар\-ло и~смежные вопросы.~--- 2-е 
изд.~--- М.: Наука, Физматлит, 1975. 472~с. 

 \bibitem{16-sr} %17
    \Au{Михайлов Г.\,А., Войтишек~А.\,В.} Численное статистическое 
моделирование. Методы Мон\-те-Кар\-ло.~--- М.: Академия, 2006. 368~с.

    \bibitem{18-sr}
    \Au{Семенычев В.\,К.} Идентификация экономической динамики на основе 
моделей авторегрессии.~--- Самара: СНЦ РАН, 2004. 243~с.
    \bibitem{19-sr}
    \Au{Зотеев В.\,Е.} Параметрическая идентификация диссипативных 
механических систем на основе разностных уравнений.~--- М.: 
Машиностроение, 2009. 344~с.
    \bibitem{20-sr}
    \Au{Серебрянский С.\,М.} Распознание зависимостей на основе 
структурных разностных схем: Свидетельство о~регистрации электронного 
ресурса №\,19869.~--- М.: ИНИПИ РАО, ОФЭРНиО; зарегистр. 10.01.14.
 \end{thebibliography}

 }
 }

\end{multicols}

\vspace*{-3pt}

\hfill{\small\textit{Поступила в~редакцию 21.04.19}}

\vspace*{8pt}

%\pagebreak

%\newpage

%\vspace*{-28pt}

\hrule

\vspace*{2pt}

\hrule

%\vspace*{-2pt}

\def\tit{IMPROVEMENT OF~THE~ACCURACY OF~SOLUTION OF~TASKS FOR~THE~ACCOUNT 
OF~THE~CONSTRUCTION OF~BOUNDARY CONDITIONS}


\def\titkol{Improvement of~the~accuracy of~solution of~tasks for~the~account 
of~the~construction of~boundary conditions}

\def\aut{S.\,M.~Serebryanskii$^1$ and~A.\,N.~Tyrsin$^2$}

\def\autkol{S.\,M.~Serebryanskii and~A.\,N.~Tyrsin}

\titel{\tit}{\aut}{\autkol}{\titkol}

\vspace*{-11pt}


\noindent
$^1$Troitsk Branch of Chelyabinsk State University, 9~S.~Rasin Str., Troitsk 457100, Russian Federation

\noindent
$^2$Science and Engineering Center ``Reliability and Resource of Large Systems and Machines,'' Ural Branch\linebreak
$\hphantom{^1}$of the Russian Academy of Sciences; 54a~Studencheskaya Str., Yekaterinburg 620049, Russian Federation

\def\leftfootline{\small{\textbf{\thepage}
\hfill INFORMATIKA I EE PRIMENENIYA~--- INFORMATICS AND
APPLICATIONS\ \ \ 2020\ \ \ volume~14\ \ \ issue\ 1}
}%
 \def\rightfootline{\small{INFORMATIKA I EE PRIMENENIYA~---
INFORMATICS AND APPLICATIONS\ \ \ 2020\ \ \ volume~14\ \ \ issue\ 1
\hfill \textbf{\thepage}}}

\vspace*{3pt} 



   \Abste{The problems of stability of the solution of inverse problems with respect to the exact setting of 
boundary conditions are considered. In practical applications, as a~rule, the theoretical form of the 
functional dependence of the boundary conditions is a~form that is not defined or not known, and there 
are also random measurement errors. Studies have shown that this leads to a~significant reduction in the 
accuracy of solving the inverse problem. In order to increase the accuracy of solving inverse problems, it 
was proposed to refine the functional form of the boundary conditions by recognizing the form of the 
mathematical model of dependence with the subsequent approximation by this function of the behavior of 
a~physical quantity at the boundary. Dependency recovery was performed using dependency recognition 
methods based on structural difference schemes and inverse mapping recognition. Model examples of 
implementation in the presence of additive random measurement errors and an unknown type of 
dependence of the boundary conditions are given.}
   \KWE{inverse problem; recognition; functional dependence; model; difference schemes; inverse 
function; sampling; variance; approximation}


   
\DOI{10.14357/19922264200108} 

\vspace*{-18pt}

\Ack
\noindent
This research was partially supported by the Russian Foundation for Basic Research 
(project 20-41-660008~r\_а).

\vspace*{6pt}

  \begin{multicols}{2}

\renewcommand{\bibname}{\protect\rmfamily References}
%\renewcommand{\bibname}{\large\protect\rm References}

{\small\frenchspacing
 {%\baselineskip=10.8pt
 \addcontentsline{toc}{section}{References}
 \begin{thebibliography}{99}
 
 \vspace*{-4pt}
 
  \bibitem{1-sr-1}
  \Aue{Ivanov, V.\,K., I.\,V.~Melnikova, and A.\,I.~Filinkov.} 1995.  
\textit{Differentsial'no-operatornye uravneniya i~nekorrektnye zadachi} 
[Differential-operator equations and ill-posed problems]. Moscow: Fizmatlit. 
176~p.
  \bibitem{2-sr-1}
  \Aue{Ivanov, V.\,K., V.\,V.~Vasin, and V.\,P.~Tanana.} 1978. \textit{Teoriya 
lineynykh nekorrektnykh zadach i~ee prilozheniya} [Theory of linear ill-posed 
problems and its applications]. Moscow: Nauka. 206~p.
  \bibitem{3-sr-1}
  \Aue{Serebryanskii, S.\,M.} 2010. Ob otsenkakh po\-gresh\-nosti metodov 
priblizhennogo resheniya odnoy obratnoy zadachi 
[On estimating the errors of approximate solution 
methods for an inverse problem]. \textit{Sibirskiy zh.~industrial'noy 
matematiki} [Siberian J.~Iindustrial Mathematics] 2(42):135--148.

  \bibitem{6-sr-1} %4
  \Aue{Nalimov, V.\,V.} 1971. \textit{Teoriya eksperimenta} [Theory of experiment]. 
Moscow: Nauka, Fizmatlit. 208~p.
  \bibitem{7-sr-1} %5
  \Aue{Aivazyan, S.\,A., I.\,S.~Enyukov, and L.\,D.~Meshalkin.} 1985. 
\textit{Prikladnaya statistika: issledovanie zavisimostey} [Applied statistics: The 
study of dependencies]. Moscow: Finansy i~statistika. 487~p.
  \bibitem{8-sr-1} %6
  \Aue{Kleiner, G.\,B., and S.\,A.~Smolyak.} 2000. \textit{Eko\-no\-met\-ri\-che\-skie 
zavisimosti: printsipy i~metody postroeniya} [Econometric dependences: Principles 
and methods of construction]. Moscow: Nauka. 104~p.
\bibitem{10-sr-1} %7
  \Aue{Tyrsin, A.\,N.} 2005. Identifikatsiya zavisimostey na osno\-ve modeley 
avtoregressii [Identification of dependencies based on autoregression models]. 
\textit{Optoelectronics Instrumentation Data Processing} 
41(1):43--49.
  \bibitem{9-sr-1} %8
  \Aue{Orlov, A.\,I.} 2007. \textit{Prikladnaya statistika} [Applied statistics]. 2nd ed. 
Moscow: Ekzamen. 671~p.
    \bibitem{4-sr-1} %9
  \Aue{Tyrsin, A.\,N., and S.\,M.~Serebryanskii.} 2015.  
Dependence identification in a time series on the basis of structural difference 
schemes. \textit{Optoelectronics Instrumentation Data 
Processing} 51(2):149--154.
  \bibitem{5-sr-1} %10
  \Aue{Tyrsin, A.\,N., and S.\,M.~Serebryanskii.} 2016. Raspoznavanie tipa 
zavisimosti na osnove obratnogo otobrazheniya [Recognition of dependences on the 
basis of inverse mapping]. \textit{Informatika i~ee Primeneniya~--- Inform. 
\mbox{Appl}.}  10(2):58--64.
  
  \bibitem{12-sr-1} %11
  \Aue{Tikhonov, A.\,N.} 1943. Ob ustoychivosti obratnykh zadach [On the stability 
of inverse problems]. \textit{Dokl. Akad. nauk SSSR} 39(5):195--198.
  \bibitem{13-sr-1} %12
  \Aue{Tanana, V.\,P.} 1981. \textit{Metody resheniya operatornykh uravneniy} 
[Methods for solving operator equations]. Moscow: Nauka. 160~p.
\bibitem{11-sr-1} %13
  \Aue{Alifanov, O.\,M.} 1988. \textit{Obratnye zadachi teploobmena} [Inverse 
heat transfer problems]. Moscow: Mechanical Engineering. 279~p.
  \bibitem{14-sr-1} %14
  \Aue{Tanana, V.\,P., and N.\,M.~Yaparova.} 2006. Ob optimal'nom po poryadku 
metode resheniya uslovno-korrektnykh zadach [On the optimal order method for 
solving ill-posed problems]. \textit{Sibirskiy zh.~vychislitel'noy matematiki} [Siberian 
J.~Numerical Mathematics] 9(4):353--368.
  \bibitem{15-sr-1}
  \Aue{Serebryansky, S.\,M., and A.\,N.~Tyrsin.} 2018. Povyshenie tochnosti 
resheniya obratnykh zadach pri oshibkakh v~nachal'nykh dannykh [Improving the 
accuracy of solving inverse problems with inherent
errors]. \textit{Vestnik 
Buryatskogo gosudarstvennogo universiteta. Matematika, informatika} 
[BSU Bull. Mathematics Informatics] 4:58--71.  
 \bibitem{17-sr-1} %16
  \Aue{Ermakov, S.\,E.} 1975. \textit{Metod Monte-Karlo i~smezhnye voprosy} 
[Monte-Carlo method and related questions]. 2nd ed. Moscow: Nauka, Fizmatlit. 
472~p. 
  \bibitem{16-sr-1} %17
  \Aue{Mikhailov, G.\,A., and A.\,V.~Voytishek.} 2006. \textit{Chislennoe 
statisticheskoe modelirovanie. Metody Monte-Karlo} [Numerical statistical 
modeling.  Monte-Carlo methods]. Moscow: Akademiya. 368~p.
 
  \bibitem{18-sr-1}
  \Aue{Semenychev, V.\,K.} 2004. \textit{Identifikatsiya ekonomicheskoy dinamiki 
na osnove modeley avtoregressii} [Identification of economic dynamics on the basis 
of autoregressive models]. Samara: SNTs RAN. 243~p.
  \bibitem{19-sr-1}
  \Aue{Zoteev, V.\,E.} 2009. \textit{Parametricheskaya identifikatsiya dissipativnykh 
mekhanicheskikh sistem na osnove raznostnykh uravneniy} [Parametric 
identification of dissipative mechanical systems on the basis of difference 
equations]. Moscow: Mashinostroenie. 344~p.
  \bibitem{20-sr-1}
  \Aue{Serebryanskii, S.\,M.} 2014. Raspoznanie zavisimostey na osnove 
strukturnykh raznostnykh skhem [Recognition of dependences on the basis of 
structural difference schemes]. Certificate of registration of electronic resource 
No.\,19869. 
\end{thebibliography}

 }
 }

\end{multicols}

%\vspace*{-7pt}

\hfill{\small\textit{Received April 21, 2019}}

%\pagebreak

%\vspace*{-22pt}

\Contr

\noindent
\textbf{Serebryanskii Sergey M.} (b.\ 1983)~--- senior lecturer, Troitsk branch of 
Chelyabinsk State University, 9~S.~Razina Str., Troitsk 457100, Chelyabinsk Region, 
Russian Federation; \mbox{tf\_chelgu@mail.ru}

\vspace*{3pt}

\noindent
\textbf{Tyrsin Alexander N.} (b.\ 1961)~--- Doctor of Science in technology, leading 
scientist, Science and Engineering Center ``Reliability and Resource of Large Systems 
and Machines,'' Ural Branch of the Russian Academy of Sciences, 
54a~Studencheskaya Str., Ekaterinburg 620049, Russian Federation; 
\mbox{at2001@yandex.ru}

\label{end\stat}

\renewcommand{\bibname}{\protect\rm Литература} 