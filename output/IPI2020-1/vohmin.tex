\def\stat{vohmin}

\def\tit{МЕТОД НАВИГАЦИИ И~СОСТАВЛЕНИЯ КАРТЫ В~ТРЕХМЕРНОМ ПРОСТРАНСТВЕ 
НА~ОСНОВЕ КОМБИНИРОВАННОГО РЕШЕНИЯ ВАРИАЦИОННОЙ ПОДЗАДАЧИ 
ТОЧКА--ТОЧКА~ICP ДЛЯ~АФФИННЫХ ПРЕОБРАЗОВАНИЙ$^*$}

\def\titkol{Метод навигации и~составления карты в~трехмерном пространстве 
на~основе комбинированного решения} % вариационной подзадачи 
%точка--точка ICP для аффинных преобразований}

\def\aut{А.\,В.~Вохминцев$^1$, А.\,В.~Мельников$^2$, C.\,А.~Пачганов$^3$}

\def\autkol{А.\,В.~Вохминцев, А.\,В.~Мельников, C.\,А.~Пачганов}

\titel{\tit}{\aut}{\autkol}{\titkol}

\index{Вохминцев А.\,В.}
\index{Мельников А.\,В.}
\index{Пачганов C.\,А.}
\index{Vokhmintcev A.\,V.}
\index{Melnikov A.\,V.}
\index{Pachganov S.\,A.}


{\renewcommand{\thefootnote}{\fnsymbol{footnote}} \footnotetext[1]
{Работа выполнена при поддержке РФФИ (проект 18-37-20032) 
и~Российского научного фонда (проект  
15-19-10010).}}


\renewcommand{\thefootnote}{\arabic{footnote}}
\footnotetext[1]{Челябинский государственный университет; 
Югорский государственный университет, 
\mbox{vav@csu.ru}}
\footnotetext[2]{Югорский государственный университет, \mbox{melnikovav@uriit.ru}}
\footnotetext[3]{Югорский государственный университет, \mbox{pachganovsa@uriit.ru}}

%\vspace*{-12pt}
  
  
      
  
  \Abst{Одновременная навигация и~картографирование относятся к~проблеме, в~которой 
данные кадра используются в~качестве единственного источника внешней информации для 
того, чтобы установить положение движущейся камеры в~пространстве и~в~то же время 
построить карту зоны исследования. На сегодняшний день эта проблема считается решенной 
для построения двумерных карт небольших статических сцен с~использованием датчиков 
дальности. Однако для динамичных, сложных и~крупномасштабных сцен построение точной 
трехмерной карты окружающего пространства стало активной об\-ластью научных 
исследований. Для решения поставленной проблемы в~работе предложено решение задачи 
точка--точка для аффинных преобразований и~разработан быстрый итерационный алгоритм 
регистрации кадров в~трехмерном пространстве. Производительность и~вычислительная 
сложность предлагаемого метода реконструкции трехмерных сцен представлены 
и~обсуждены на примере эталонных данных. Результаты могут быть применены в~задачах 
навигации мобильного робота в~реальном масштабе времени.}
  
  \KW{задача регистрации данных; локализация; методы одновременной навигации 
и~со\-став\-ле\-ния карты; аффинное преобразование; двумерные дескрипторы; итеративный 
алгоритм ближайших точек}

\DOI{10.14357/19922264200114} 
  
\vspace*{-3pt}


\vskip 10pt plus 9pt minus 6pt

\thispagestyle{headings}

\begin{multicols}{2}

\label{st\stat}
  
\section{Введение}

  Разработка динамической системы для надежного решения проблемы 
одновременной навигации мобильного робота и~составления карты 
окру\-жа\-ющей его среды (Simultaneous Localization And Mapping, SLAM) 
в~реальном масштабе времени стала одной из ключевых задач в~современной 
робототехнике и~машинном зрении, так как на ее решении основано создание 
автономных интеллектуальных робототехнических комплексов и~сис\-тем~[1--3]. 
Для построения качественных трехмерных моделей необходимо совместное 
использование полученных с~разных датчиков данных, таких как изображения, 
положение используемого датчика и~карты глубины~[4, 5]. В~большинстве 
случаев для построения трехмерных моделей по картам глубины используется 
итеративный алгоритм ближайших точек (Iterative Closest Point, ICP)~[6]. 
Главный этап алгоритма регистрации ICP связан с~поиском соответствующего 
геометрического преобразования (ортогонального или аффинного), которое 
наилучшим образом совмещает два облака точек в~разных RGB-D-кад\-рах для 
выбранной метрики (вариационная подзадача алгоритма). Точ\-ность 
реконструкции трехмерной сцены существенно зависит от выбора метрики для 
оценки гео\-мет\-ри\-че\-ско\-го преобразования и~метода решения вариационной 
задачи~[7]. 

Результат применения итерационных методов для решения задачи 
минимизации выбранного функционала зависит от правильности выбора 
начального приближения па\-ра\-мет\-ров гео\-мет\-ри\-че\-ско\-го преобразования: 
итерационный процесс может сходиться медленно, сходиться к~локальному 
оптимуму или вообще не сходиться. Использование решений вариационной 
задачи в~замк\-ну\-той форме позволяет избежать этих проблем~[8, 9]. Выбор 
класса гео\-мет\-ри\-че\-ских преобразований так\-же оказывает значительное влияние 
на результат реконструкции трехмерной сцены~[10].

 Для класса ортогональных 
преобразований решение задачи точ\-ка--точ\-ка в~замк\-ну\-той форме получено 
с~по\-мощью кватернионов~\cite{8-voh} или с~помощью ортогональных 
мат\-риц~\cite{9-voh}. На основе метода Хорна сформулирован алгоритм ICP 
в~варианте точ\-ка--точ\-ка~[11]. Известно, что метрика точ\-ка--плос\-кость 
превосходит метрику точ\-ка--точ\-ка по точ\-ности и~ско\-рости схо\-ди\-мости. 

Использование аффинных преобразований позво\-ля\-ет решать задачу 
регистрации для нежестких объектов~[12]. Отметим, что если истинное 
гео\-мет\-ри\-че\-ское преобразование, связывающее два облака точек, ортогонально, 
то применение вариационной задачи точ\-ка--плос\-кость для аффинных 
преобразований даст правильный результат только в~том случае, если 
соответствие между точками двух облаков близко к~идеальному. На практике 
соответствие между точками двух облаков в~большинстве случаев неидеально: 
например, все точки первого облака могут соответствовать одной точке второго 
облака или небольшому локальному подмножеству точек второго облака~[13]. 
Решение описанной выше условной вариационной задачи для метрики  
точ\-ка--точ\-ка известно как метод Хорна~\cite{8-voh}. Для мет\-ри\-ки  
точ\-ка--плос\-кость~\cite{14-voh} существуют решения в~замк\-ну\-той форме для 
аффинных преобразований. 
  
\section{Метод SLAM и~постановка задачи}

  Решение задачи одновременной навигации и~картографирования со\-сто\-ит из 
сле\-ду\-ющих этапов: 
\begin{itemize}
\item сопоставление и~регистрация последовательностей 
изображений в~RGB-D-кад\-рах; 
\item пространственное совмещение трехмерных 
облаков точек в~RGB-D-кад\-рах; 
\item обнаружение замыканий цикла; 
\item построение 
трехмерной карты доступной окружающей среды;
\item определение позиции 
робота в~относительной системе координат в~каждый момент времени. 
\end{itemize}

В~данной работе предложено новое решение вариационной задачи  
точ\-ка--точ\-ка в~замк\-ну\-той форме на основе комбинации данных о~глубине 
и~цвете в~кадре, которое на\-прав\-ле\-но на решение второго этапа задачи SLAM. 
Основные недостатки итерационных методов регистрации данных связаны 
с~ограничением области схо\-ди\-мости и~большой вы\-чис\-ли\-тель\-ной слож\-ностью. 
Кроме того, результат решения вариационной задачи зависит от пра\-виль\-ности 
выбора начального приближения. Для преодоления данного недостатка 
в~работе предлагается использовать визуально связанные характеристики 
RGB-D-кад\-ра (особые точки), которые позволяют совмещать кадры без 
требования начальной инициализации. Хорн предложил решение условной 
вариационной задачи для метрики точ\-ка--точ\-ка в~замкнутой форме для 
ортогональных преобразований. В~данной работе получено решение 
в~замк\-ну\-той форме для аффинных преобразований, что, во-пер\-вых, создает 
математическую основу для решения задачи регистрации неригидных объектов 
на сцене; во-вто\-рых, позволяет находить точное решение вариационной 
задачи для вырожденных случаев, например, когда все точки трехмерного 
облака точек находятся в~одной плоскости. При решении задачи SLAM 
в~динамическом пространстве такая необходимость возникает при 
идентификации и~отслеживании основных структурных элементов сцены: 
например, для замк\-ну\-тых пространств (помещений) такими элементами могут 
выступать потолок, стены, пол.
  
\section{Сопоставление визуальных признаков на~RGB-D-кадрах}

  Для обработки визуальных характеристик \mbox{сцены} используется алгоритм 
сопоставления изоб\-ра\-же\-ний на основе рекурсивного вы\-чис\-ле\-ния гистограмм 
на\-прав\-лен\-ных градиентов (ГНГ) по нескольким круглым скользящим окнам 
и~\mbox{пирамидальному} разложению изображения~[15, 16]. Для работы с~особыми 
признаками используется сле\-ду\-ющая схема.
  \begin{enumerate}[1.]
  \item  Вычисление ГНГ на изображениях.
  \item  Сопоставление между особыми точками для выбранных подмножеств.
  \item  Отбрасывание некоторых пар особых точек для поиска~[17].
  \item Решение вариационной задачи регистрации данных для визуально 
связанных характеристик изображения.
  \end{enumerate}
  
  Рассмотрим более подробно пп.~2 и~4. В~работе используется 
корреляционный оператор, при помощи которого осуществляется процедура 
со\-по\-став\-ле\-ния данных из различных RGB-D-кад\-ров. Введем формулу для 
определения нормализованной центрированной ГНГ эталонного изображения:
  \begin{equation*}
  \overline{\mathrm{HOG}_i^R}(\alpha) =\fr{\mathrm{HOG}_i^R(\alpha) -
  \mathrm{Mean}^R}{\sqrt{\mathrm{Var}^R}}\,,
  %\label{e1-voh}
  \end{equation*}
где $\mathrm{Mean}^R$~--- среднее значение ГНГ; $\mathrm{Var}^R$~--- дисперсия ГНГ. Тогда 
для каж\-до\-го $i$-го медианного фильт\-ра (МФ) в~позиции~$k$ можно 
определить корреляционную функцию:
\begin{multline*}
C_i^k(\alpha) ={}\\
{}=\mathrm{IFT} \left[ \fr{\mathrm{HS}_i^k(\omega) \mathrm{HR}_i^*(\omega)} {\sqrt{Q 
\sum\nolimits_{q=0}^{Q-1}\left( \mathrm{HOG}_i^k(q)\right)^2- \left( 
\mathrm{HS}_i^k(0)\right)^2}}\right],\hspace*{-0.56015pt}
%\label{e2-voh}
\end{multline*}
где $\mathrm{HS}_i^k(\omega)$~--- преобразование Фурье ГНГ внутри $i$-го МФ 
входной сцены; $\mathrm{HR}_i(\omega)$~--- преобразование Фурье 
$\overline{\mathrm{HOG}_i^R}(\alpha)$; $*$ для $i$-го преобразования Фурье обозначает 
комплексное сопряжение. Для определения подобия двух ГНГ применяется 
корреляционный пик $P_i^k\hm= \max\nolimits_\alpha \left\{ 
C_i^k(\alpha)\right\}$.
  
  Решение вариационной задачи для визуально связанных характеристик 
изоб\-ра\-же\-ния может быть пред\-став\-ле\-но в~виде:
  \begin{equation*}
  J(\mathrm{RV}\,)=\fr{1}{\left\vert {A}_f \right\vert} 
  \sum\limits^n_{i\in A_f} {w}_i\left\| 
M\left(\mathrm{RV}\,F_x^i\right) - M\left(F_y^i\right)\right\|^2\,.
  %\label{e3-voh}
  \end{equation*}
Здесь $\mathrm{RV}$~--- матрица аффинного преобразования для визуально связанных 
характеристик сцены; ${w}_i$~--- весовые характеристики данных; 
$F_x^i$ и~$F_y^i$~--- визуально связанные характеристики сцены в~исходном и~целевом кадре соответственно:
\begin{equation*}
F_x^i=\left( x^i_{1f}, x^i_{2f}, x^i_{3f}\right)^{\mathrm{T}}\,;\enskip
F_y^i= \left( y^i_{1f}, y^i_{2f}, y^i_{3f}\right)^{\mathrm{T}}\,.
%\label{e4-voh}
\end{equation*}
  
  Функция~$M$ осуществляет преобразование координат точек~$F_x^i$ 
и~$F_y^i\hm\in \mathbb{R}^3$ из трехмерной сис\-те\-мы координат относительно 
камеры в~систему координат камеры $C^i\hm= \left( C_x^i, C_y^i, 
D^i\right)\hm\in \mathbb{R}^3$, где $C_x^i$ и~$C_y^i$~--- соответствующие 
координаты точек в~пиксельном пространстве; $D^i$~--- значение глубины 
в~пиксельном про\-стран\-стве:
  \begin{align*}
  C_x^i&=\fr{f}{x^i_{3f}}\,x^i_{1f}+O_x\,;\\
  C_y^i&=\fr{f}{x^i_{3f}}\,x^i_{2f}+O_y\,;\\
  D^i&=\sqrt{{x^i_{1f}}^2 +{x^i_{2f}}^2+{x^i_{3f}}^2}\,.
  \end{align*}
%  \label{e5-voh}
 Здесь $O_x$ и~$O_y$~--- координаты центра изображения в~пиксельном 
пространстве; $f$~--- фокус камеры. Аналогичным образом могут быть 
определены координаты точек в~сис\-те\-ме координат камеры для точек~$F_y^i$.

\section{Решение задачи точка--точка метода ICP для~аффинных 
преобразований}

  Задача регистрации трехмерных данных (данных кадра о глубине) на основе 
алгоритма ICP со\-сто\-ит из следующих шагов.
  \begin{enumerate}[1.]
\item Формирование разреженных подмножеств точек из двух плот\-ных 
трехмерных облаков точек.
\item Определение соответствующих точек в~каждом из разреженных 
подмножеств.
\item Определение весовых коэффициентов для каж\-дой полученной пары.
\item Отбрасывание некоторых пар в~облаках точек (RANSAC-ме\-тод или 
аналог).
\item Выбор метрики ошибки для пар точек.
\item Решение вариационной задачи на основе минимизации функции ошибки.
\end{enumerate}
  
  Обозначим через $X\hm=\{x_1, \ldots , x_n\}$ множество точек в~исходном 
RGB-кадре и~через $Y\hm= \{ y_1, \ldots , y_m\}$ множество точек в~целевом 
RGB-D-кад\-ре в~$\mathbb{R}^3$. Предположим, что отношения между 
точками в~кадрах~$X$ и~$Y$ такие, что для каждой точки в~$x_i$ можно 
вычислить со\-от\-вет\-ст\-ву\-ющую точ\-ку в~$y_i$. Тогда алгоритм ICP можно 
рассмотреть как гео\-мет\-ри\-че\-ское преобразование из~$X$ в~$Y$ сле\-ду\-юще\-го 
вида:
  \begin{equation*}
  Rx_i+T\,,
  %\label{e6-voh}
  \end{equation*}
где $R$~--- матрица поворота; $T$~--- вектор параллельного переноса; 
$i\hm=1,\ldots , n$. Тогда аффинное преобра\-зо\-ва\-ние в~$\mathbb{R}^3$ можно 
представить в~виде функции от двенадцати переменных и~решить 
вариационную задачу алгоритма регистрации кад\-ров для произвольного 
преобразования. 

Пусть $J(R,T)$~--- функция вида:
\begin{equation*}
J(R,T)=\sum\limits_{i=1}^n \| R x_i+T-y_i\|^2\,,
%\label{e7-voh}
\end{equation*}
 где
 $$
x_i=\begin{pmatrix}
x_{1i}\\ x_{2i}\\ x_{3i}
\end{pmatrix}\,;\enskip
y_i=\begin{pmatrix}
y_{1i}\\ y_{2i}\\ y_{3i}
\end{pmatrix}\,.
$$
  %
  Тогда вариационная задача ICP может быть определена как
%  \begin{equation*}
 $\mathop{\mathrm{arg\,min}}\limits_{ R,T} J(R,T)$,
 %  \end{equation*}
где 
$$
R=\begin{pmatrix}
r_{11} & r_{12} & r_{13}\\
r_{21} & r_{22} & r_{23}\\
r_{31} & r_{32} & r_{33}\end{pmatrix}\,;\enskip
T=\begin{pmatrix}
t_1\\ t_2\\ t_3
\end{pmatrix}\,.
\vspace*{6pt}
$$

  
  Можно заметить, что
  \begin{multline}
  J(R,T)={}\\
  {}=\sum\limits_{i=1}^n  \left( 
r_{11}x_{1i}+r_{12}x_{2i}+r_{13}x_{3i}+t_1-y_{1i}\right)^2+{}\\
  {}+ \left( r_{21}x_{1i}+r_{22}x_{2i}+r_{23}x_{3i}+t_2 -y_{2i}\right)^2+{}\\
  {}+ \left( r_{31}x_{1i}+r_{32}x_{2i}+r_{33}x_{3i}+t_3-y_{3i}\right)^2\,.
  \label{e9-voh}
  \end{multline}
  
  Применим к~множеству точек~$X$ преобразование переноса сле\-ду\-юще\-го 
вида:

\noindent
  \begin{multline*}
  \left( x^t_{1i}=x_{1i}-\fr{1}{n}\sum\limits^n_{j=1}x_{1j}\,,
  x^t_{2i}=x_{2i}-\fr{1}{n}\sum\limits^n_{j=1} x_{2j}\,,\right.\\
  \left. x^t_{3i}=x_{3i}-\fr{1}{n}\sum\limits^n_{j=1}x_{3j}\right)\,.
 % \label{e10-voh}
  \end{multline*}
  
  Далее выполним аналогичное преобразование\linebreak для облака точек~$Y$ 
и~подставим новые координаты в~выражение~(\ref{e9-voh}). Очевидно, что 
функционал~$J(R,T)$ не зависит от элементов вектора параллельного переноса. 
Решение вариационной \mbox{задачи} ICP может быть найдено методом Хор\-на 
в~общем случае. Распространим метод Хорна на случай аффинных 
преобразований. Решение задачи для вы\-рож\-ден\-ных случаев представлено 
в~работе~\cite{17-voh}. Положим, что 
$$
\sum\limits^n_{i=1} x^2_{1i}\not=0\,;\enskip 
  \sum\limits^n_{i=1} x^2_{2i}\not=0\,;\enskip
  \sum\nolimits^n_{i=1}  x^2_{3i}\not=0\,.
  $$
  
   Тогда можно решить вариационную задачу относительно 
мат\-ри\-цы~${R}$:

\noindent
  \begin{multline*}
  \fr{\partial J(R)}{\partial r_{1k}} ={}\\
  {}=\sum\limits^n_{i=1} 2\left( 
r_{11}x_{1i}+r_{12} x_{2i}+r_{13}x_{3i}-y_{1i}\right) x_{ki} =0\,,\\
  k=1,2,3\,;
 % \label{e11-voh}
  \end{multline*}
  
\vspace*{-14pt}

  \noindent
  \begin{multline}
  \sum\limits^n_{i=1} \left( 
r_{1{m}}x_{{m}i}+r_{1{n}}x_{{n}i} -
y_{1i}\right) x_{ki} +r_{1k} \sum\limits^n_{i=1} x^2_{ki}=0\,,\\
  k,m,n=1,2,3\,;\ m,n\not= k\,.
  \label{e12-voh}
  \end{multline}
  
  Из выражения~(\ref{e12-voh}) можем получить значения параметров 
мат\-ри\-цы поворота:
  \begin{equation}
  r_{1k}=-\fr{\sum\nolimits^n_{i=1} (r_{1{m}} 
x_{{m}i}+r_{1{n}}x_{{n}i}-
y_{1i})x_{ki}}{\sum\nolimits^n_{i=1} x^2_{ki}}\,.
  \label{e13-voh}
  \end{equation}
  
  Учитывая выражение~(\ref{e13-voh}), можем пред\-ста\-вить $J(R)$ 
в~сле\-ду\-ющем виде:

\noindent
  \begin{multline}
  J(R)= \sum\limits^n_{i=1} \left(
  \vphantom{\fr{\sum\nolimits^n_{j=1} (r_{1{m}}x_{{m}j}+r_{1{n}} 
x_{{n}j}-y_{1j})x_{kj}}{\sum\nolimits^n_{j=1} x^2_{kj}}}
 r_{1{m}}x_{{m}i}-{}\right.\\
  {}-
\fr{\sum\nolimits^n_{j=1} (r_{1{m}}x_{{m}j}+r_{1{n}} 
x_{{n}j}-y_{1j})x_{kj}}{\sum\nolimits^n_{j=1} x^2_{kj}}\,x_{ki} 
+r_{1{n}} x_{{n}i} -{}\\
\left.{}-y_{1i}
  \vphantom{\fr{\sum\nolimits^n_{j=1} (r_{1\mathrm{m}}x_{\mathrm{m}j}+r_{1\mathrm{n}} 
x_{\mathrm{n}j}-y_{1j})x_{kj}}{\sum\nolimits^n_{j=1} x^2_{kj}}}
\right)^2+  \left( r_{21}x_{1i}+r_{22}x_{2i}+r_{23}x_{3i}-y_{2i}\right)^2+{}\\
  {}+ \left(  r_{31}x_{1i} +r_{32}x_{2i}+r_{33} x_{3i} -y_{3i}\right)^2\,.
  \label{e14-voh}
  \end{multline}
  
  Рассмотрим более подробно первое сла\-га\-емое в~выражении~(\ref{e14-voh}) 
и~раскроем скоб\-ки под знаком суммы. Если подставить полученное выражение 
в~функционал~$J(R)$, то имеем
  \begin{multline}
  J(R)=\sum\limits^n_{i=1}\left( r_{1m}\left(
  x_{mi}-x_{ki}\fr{\sum\nolimits^n_{j=1} x_{mj} x_{kj}}
  {\sum\nolimits^n_{j=1} x^2_{kj}}\right)
  +{}\right.\\
  {}+r_{1n}\left( x_{ni}-x_{ki}\fr{\sum\nolimits^n_{j=1} x_{nj} x_{kj}}
  {\sum\nolimits^n_{j=1} x^2_{kj}}\right)-{}\\
 \left. {}-\left( y_{1i}-x_{ki}
  \fr{\sum\nolimits^n_{j=1} y_{1j} x_{kj}}
  {\sum\nolimits^n_{j=1} x^2_{kj}}\right)\right)^2+{}\\
  {}+
 \left( r_{21}x_{1i}+r_{22} x_{2i}+r_{23}x_{3i}-
y_{2i}\right)^2 +{}\\
  {}+\left( r_{31}x_{1i}+r_{32} x_{2i}+r_{33}x_{3i}-y_{3i}\right)^2\,.
  \label{e15-voh}
  \end{multline}
  
  Введем следующие обозначения для упрощения вида  
выражения~(\ref{e15-voh}):
  \begin{align*}
  G_{mi}&=x_{{m}i}-x_{ki} \fr{\sum\nolimits^n_{j=1} x_{mj} x_{kj}} 
{\sum\nolimits^n_{j=1} x^2_{kj}}\,;\\
  G_{pi}&=x_{ni}- x_{ki} \fr{\sum\nolimits^n_{j=1} x_{nj} x_{kj}} 
{\sum\nolimits^n_{j=1} x^2_{kj}}\,;\\
  G_{ki}&=y_{1i}- x_{ki} \fr{\sum\nolimits^n_{j=1} y_{1j} x_{kj}} 
{\sum\nolimits^n_{j=1} x^2_{kj}}\,.
  \end{align*}
Тогда с~учетом обозначений можем переписать выражение~(\ref{e15-voh}) как
\begin{multline}
J(R)=\sum\limits^n_{i=1}\left( r_{1{m}} G_{mi}+r_{1{n}}G_{pi} 
-G_{ki}\right)^2+{}\\
{}+ \left( r_{21}x_{1i} +r_{22}x_{2i} +r_{23}x_{3i}-y_{2i}\right)^2+{}\\
{}+ \left( r_{31}x_{1i} +r_{32}x_{2i} +r_{33}x_{3i}-y_{3i}\right)^2\,.
\label{e16-voh}
\end{multline}
  
  Определим частную производную~$J(R)$ относительно~$r_{1m}$:
  \begin{equation*}
  \hspace*{-1.63478pt}\fr{\partial J(R)}{\partial r_{1m}} =2\sum\limits^n_{i=1} \left( 
r_{1{m}}G_{mi} +r_{1{n}}G_{pi} -G_{ki}\right) G_{mi}=0.
%  \label{e17-voh}
  \end{equation*}
  Тогда найдем~$r_{1{m}}$:
  $$
  r_{1{m}}= -\fr{r_{1n}\sum\nolimits^n_{i=1} G_{mi} G_{pi}-
\sum\nolimits^n_{i=1} G_{mi} G_{ki}}{\sum\nolimits^n_{i=1} G_{mi}^2}\,.
  $$
  Подставим полученное решение в~функционал~$J(R)$:
  \begin{multline}
  J(R)={}\\
  {}=\sum\limits^n_{i=1} \!\left(\! -\fr{r_{1{n}}\sum\nolimits^n_{j=1} 
G_{mj} G_{pj} \!-\!\sum\nolimits^n_{j=1} G_{mj} G_{kj}} {\sum\nolimits^n_{j=1} 
G_{mj}^2}\,G_{mi} +\right.\\
\left.{}+r_{1n}G_{pi} -G_{ki}
\vphantom{\fr{r_{1\mathrm{n}}\sum\nolimits^n_{j=1} 
G_{mj} G_{pj} -\sum\nolimits^n_{j=1} G_{mj} G_{kj}} {\sum\nolimits^n_{j=1} 
G_{mj}^2}}
\right)^2+
 \left( r_{21}x_{1i} +r_{22}x_{2i} + r_{23}x_{3i}-{}\right.\\
\left. {}-y_{2i}\right)^2+ \left( r_{31}x_{1i} +r_{32}x_{2i} +r_{33}x_{3i}-y_{3i}\right)^2\,.
  \label{e18-voh}
  \end{multline}
  
  Произведем некоторые преобразования в~первом сла\-га\-емом 
выражения~(\ref{e18-voh}):
  \begin{multline*}
  J(R)=\sum\limits^n_{i=1}
  \left(  \left(
  - r_{1{n}} G_{mi} 
\sum\limits^n_{j=1} G_{mj} G_{pj} -{}\right.\right.\\
\left.\left.{}-G_{mi} \sum\limits^n_{j=1} G_{mj} 
G_{kj}\!\right) \!\Bigg/\!  \sum\limits^n_{j=1} G_{mj}^2+
r_{1n} G_{pi} -G_{ki}\!\right)^2\!\!+\\
  {}+
  \left( r_{21}x_{1i} +r_{22}x_{2i} +r_{23}x_{3i}-y_{2i}\right)^2+{}\\
  {}+ \left( r_{31}x_{1i} +r_{32}x_{2i} +r_{33}x_{3i}-y_{3i}\right)^2\,.
  %\label{e19-voh}
  \end{multline*}
  %
  Затем сделаем группировку слагаемых сле\-ду\-юще\-го вида:
  \begin{multline}
  J(R)= \sum\limits^n_{i=1} \left(\! r_{1n}\left(\! G_{pi} -\fr{G_{mi} 
\sum\nolimits^n_{j=1} G_{mj} G_{pj}} {\sum\nolimits^n_{j=1} G_{mj}^2} 
\!\right)-{}\right.\\[2pt]
  \left.{}-\left( G_{ki} -\fr{ G_{mi} \sum\nolimits^n_{j=1} G_{mj} G_{kj}}{ 
\sum\nolimits^n_{j=1} G_{mj}^2}\right) \right)^2+{}\\[2pt]
  {}+ \left( r_{21}x_{1i} +r_{22}x_{2i} +r_{23}x_{3i}-y_{2i}\right)^2+{}\\[2pt]
  {}+ \left( r_{31}x_{1i} +r_{32}x_{2i} +r_{33}x_{3i}-y_{3i}\right)^2\,.
  \label{e20-voh}
  \end{multline}
  
  Введем следующие обозначения:
  \begin{equation}
  \left.
  \begin{array}{rl}
  \Omega_1 &= G_{pi} -\fr{G_{mi} \sum\nolimits^n_{j=1} G_{mj}G_{pj}} 
{\sum\nolimits^n_{j=1} G_{mj}^2}\,;\\[16pt]
  \Omega_2 &= G_{ki} -\fr{ G_{mi} \sum\nolimits^n_{j=1} G_{mj}G_{kj}} 
{\sum\nolimits^n_{j=1} G_{mj}^2}\,.
  \end{array}
  \right\}
  \label{e21-voh}
  \end{equation}
  
  С учетом~(\ref{e20-voh}) и~(\ref{e21-voh}) выражение~(\ref{e16-voh}) может 
быть пред\-став\-ле\-но в~виде:
  \begin{multline*}
  J(R)=\sum\limits^n_{i=1} \left( r_{1n} \Omega_1-\Omega_2\right)^2+{}\\
  {}+
  \left( r_{21}x_{1i} +r_{22}x_{2i} +r_{23}x_{3i}-y_{2i}\right)^2+{}\\
  {}+ \left( r_{31}x_{1i} +r_{32}x_{2i} +r_{33}x_{3i}-y_{3i}\right)^2\,.
 % \label{e222-voh}
  \end{multline*}
  
  Теперь определим част\-ную производную~$J(R)$ относительно~$r_{1n}$:
  \begin{equation*}
  \fr{\partial J(R)}{\partial r_{1{n}}} =2\sum\limits^n_{j=1} \left( r_{1n} 
\Omega_1 -\Omega_2\right) \Omega_1=0\,.
%  \label{e23-voh}
  \end{equation*}
  %
  Тогда 
  $$
  r_{1n}=\fr{\sum\nolimits^n_{k=1} \Omega_1\Omega_2}{\sum\nolimits^n_{k=1} 
\Omega_1^2}\,.
  $$
  
  При $\sum\nolimits^n_{k=1} \Omega_1^2\not=0$ можем определить 
па\-ра\-мет\-ры мат\-ри\-цы поворота. Тогда первая строка мат\-ри\-цы будет выглядеть 
сле\-ду\-ющим образом:
  \begin{align*}
  r_{1{m}}& =-\fr{r_{1{n}}\sum\nolimits^n_{i=1} G_{mi} G_{pi} 
-\sum\nolimits^n_{i=1} G_{mi} G_{ki}} {\sum\nolimits^n_{i=1} G_{mi}^2}\,;\\ 
r_{1n} &=\fr{\sum\nolimits^n_{k=1} \Omega_1\Omega_2}{\sum\nolimits^n_{k=1} 
\Omega_1^2}\,;\\
  r_{1k} &=-\fr{\sum\nolimits^n_{i=1} \left( r_{1{m}} x_{{m}i} 
+r_{1{n}} x_{{n}i} -y_{1i}\right) x_{ki}} {\sum\nolimits^n_{i=1} 
x^2_{ki}}\,.
  \end{align*}
  %
  Вторая строка параметров мат\-ри\-цы поворота:
  \begin{align*}
  r_{2{m}} &=-\fr{r_{2{n}}\sum\nolimits^n_{i=1} G_{mi} G_{pi} 
-\sum\nolimits^n_{i=1} G_{mi} G_{ki}} {\sum\nolimits^n_{i=1} G_{mi}^2}\,;\\ 
r_{2n} &=\fr{\sum\nolimits^n_{k=1} \Omega_1\Omega_2}{\sum\nolimits^n_{k=1} 
\Omega_1^2}\,;\\
  r_{2k} &=-\fr{\sum\nolimits^n_{i=1} \left( r_{2{m}} x_{{m}i} 
+r_{2{n}} x_{{n}i} -y_{2i}\right) x_{ki}} {\sum\nolimits^n_{i=1} 
x^2_{ki}}\,.
  \end{align*}
  %
  Третья строка параметров мат\-ри\-цы поворота:
  \begin{align*}
  r_{3{m}}& =-\fr{r_{3{n}}\sum\nolimits^n_{i=1} G_{mi} G_{pi} 
-\sum\nolimits^n_{i=1} G_{mi} G_{ki}} {\sum\nolimits^n_{i=1} G_{mi}^2}\,;\\ 
r_{3n} &=\fr{\sum\nolimits^n_{k=1} \Omega_1\Omega_2}{\sum\nolimits^n_{k=1} 
\Omega_1^2}\,;\\
  r_{3k} &=-\fr{\sum\nolimits^n_{i=1} \left( r_{3{m}} x_{{m}i} 
+r_{3{n}} x_{{n}i} -y_{3i}\right) x_{ki}} {\sum\nolimits^n_{i=1} 
x^2_{ki}}\,.
  \end{align*}
  
  Определим элементы вектора параллельного переноса~$T$ через элементы 
мат\-ри\-цы поворота:
  \begin{multline*}
  t_k=\fr{1}{n}\sum\limits^n_{i=1} \left( y_{ki}-\left( r_{k1}x_{1i} 
+r_{k2}x_{2i} +r_{k3} x_{3i}\right)\right)=0\,,\\
  k=1,2,3\,.
 % \label{e24-voh}
  \end{multline*}
  
  \vspace*{-6pt}

\section{Комбинированное решение задачи точка--точка 
для~аффинных преобразований в~трехмерном пространстве}

  Вариационную задачу точ\-ка--точ\-ка для визуально связанных характеристик 
сцены и~данных глубины мож\-но представить в~виде:
  $\mathop{\mathrm{arg\,min}}\limits_{  \mathrm{RV}, \mathrm{RD}}
  J\left(\mathrm{RV}, \mathrm{RD}\right)$, где 
  
  \noindent
  \begin{multline*}
  J\left(\mathrm{RV}, \mathrm{RD}\right)={}\\
  {}= \alpha \fr{1}{{W}} \,\fr{1}{\vert 
{A}_{{f}} \vert} \sum\limits^m_{i\in A_f} {w}_i\left\|
  M\left( \mathrm{RV}\,F_x^i\right) -M\left( F_y^i\right)\right\|^2+{}\\
  {}+(1-\alpha) \fr{1}{{W}}\,
  \fr{1}{\vert {A}_{{d}}\vert } \sum\limits^n_{j\in A_d} {w}_j\left\| 
\mathrm{RD}\,x_j +T-y_j\right\|^2\,,
%  \label{e25-voh}
  \end{multline*}
где $\mathrm{RV}$ и~$\mathrm{RD}$~--- мат\-ри\-цы аффинного преобразования для визуально 
связанных характеристик сцены и~для данных о~глубине сцены соответственно; 
$\alpha$ и~${W}$~--- набор параметров для нормировки данных, 
подбираемый эмпирическим путем; ${w}_i$ и~${w}_j$~--- 
весовые характеристики данных, связанные с~семантической маркировкой 
про\-стран\-ст\-ва; ${A}_{{f}}$~--- подмножество, которое 
содержит связи между особыми точками в~двух кадрах; 
${A}_{{d}}$~--- содержит связи между соответствующими 
точками~$x_j$ и~$y_j$ в~трехмерных облаках точек в~двух кадрах данных; 
$F_x^i$ и~$F_y^i$~--- визуально связанные характеристики сцены. 

В~общем 
случае $\mathrm{RV}\not= \mathrm{RD}$, в~данной работе находится совместное решение 
вариационной задачи для част\-но\-го случая, когда $\mathrm{RV}\hm= \mathrm{RD}
\hm= \mathrm{RT}$. Пусть~${T}_{km}$~--- начальная оценка для ICP с~использованием 
кинематической модели движения камеры; ${k}_{\max}$  
и~$\varepsilon$~--- пороги алгоритма ICP по числу шагов и~по величине 
ошибки~${E}$ соответственно; RT$^*$~--- лучшее преобразование на 
$i$-м шаге метода; $\delta$~--- порог для точек ин\-лай\-не\-ров. Общая схема 
комбинированного метода регистрации может быть представлена в~сле\-ду\-ющем 
виде.
\begin{description}
\item[Шаг~1.] Вычисление ГНГ на 
изображениях.
\item[Шаг~2.] Сопоставление между особыми точками~$F_x^i$ и~$F_y^i$ для 
выбранных подмножеств. Решение вариационной задачи регистрации данных 
для визуально связанных характеристик изоб\-ра\-же\-ния. Получим RT$^*$ 
и~${A}_{{f}}$.
\item[Шаг~3.] Проверка: если $\vert {A}_{{f}} \vert \hm <\delta$, 
то $\mathrm{RT}^* \hm={T}_{{km}}$ 
и~${A}_{{f}}\hm=\emptyset$. Положим $i\hm = 1$.
\item[Шаг~4.] Определение соответствующих 
точек~${A}_{{d}}$ в~исходном и~целевом облаке точек 
с~использованием метода ближайших соседей. Определение весовых 
коэффициентов для каж\-дой полученной пары~${A}_{{d}}$.
\item[Шаг~5.] Решение комбинированной вариационной задачи относительно 
RT$^*$, ${A}_{{f}}$ и~${A}_{d}$.
\item[Шаг~6.] Проверка: $({E}(\mathrm{RT}^*_i)-
{E}(\mathrm{RT}^*_{i+1})\leq \varepsilon)$ или (Номер  
ите\-ра\-ции\;$>$\;$\mathrm{k}_{\max}$). Если условие неверно, то возврат на 
шаг~4 и~$i \hm= i \hm+ 1$, иначе получено искомое преобразование RT$^*$.
\end{description}
  
  В качестве продолжения направления исследований представляет интерес 
разработка метода решения комбинированной вариационной задачи  
точ\-ка--плос\-кость в~замкнутой форме для аффинных и~ортогональных 
преобразований. Обозначим: 
  \begin{equation*}
  \eta_1=\alpha\fr{1}{{W}}\,\fr{1}{\vert 
{A}_{{f}}\vert}\,,\
  \eta_2=(1-\alpha) \fr{1}{{W}}\,\fr{1}{\vert 
{A}_{d}\vert}.
\end{equation*}
 Пусть~$n^i$ есть унитарная нормаль 
к~касательной плос\-кости~$T({y}^i)$ к~поверхности~$S(Y)$ в~точке~$y^i$; $\mathrm{RD}^M$~--- комбинированная матрица аффинных 
преобразований, содержащая компоненты параллельного переноса и~поворота. 
Тогда
  \begin{multline*}
  \hspace*{-2.90392pt}J\left( \mathrm{RD}^M\right) =\sum\limits^m_{i=1} \eta_1{w}_i \left( M\left( 
\mathrm{RD}^M F_x^i\right) -M\left( F_y^i\right)\right)^2+{}\\
  {}+\sum\limits_{j=1}^n \eta_2 {w}_j \left( \left\langle \mathrm{RD}^M 
x^{\prime\,j} -y^{\prime\,j}, n^j\right\rangle \right)^2\,,
  %\label{e26-voh}
  \end{multline*}
где 
$$
x^{\prime\,j}=\left( x_1^{\prime\,j}, x_2^{\prime\,j},
x_3^{\prime\,j},1\right)^{\mathrm{T}}\,;\ 
y^{\prime j}= \left(y_1^{\prime\,j}, y_2^{\prime\,j}, 
y_3^{\prime\,j},1\right)^{\mathrm{T}}.
$$
Предполагается, что 
метод будет основан на проецировании элемента многообразия мат\-риц 
линейных преобразований на подмногообразие ортогональных мат\-риц.

  \begin{figure*}[b] %fig1
  \vspace*{1pt}
    \begin{center}  
  \mbox{%
 \epsfxsize=163mm 
 \epsfbox{voh-1.eps}
 }
\end{center}
\vspace*{-10pt}
  \Caption{Тестовый набор данных Classrooms (классная комната) из NYU Depth Dataset: 
(\textit{а})~визуальные данные RGB-D-кад\-ра; 
(\textit{б})~данные глубины RGB-D-кад\-ра}
  \end{figure*}

\section{Компьютерное моделирование}

  В данном разделе представлены результаты компьютерного моделирования. 
Для проведения компьютерного моделирования были выбраны четыре набора 
данных из базы данных NYU Depth Dataset~V2~\cite{18-voh, 19-voh}, 
содержащих фрагменты крупномасштабных сцен помещений: Classrooms, 
Living Rooms~(1/4), Offices~(1/2) и~Offices~(2/2). Компьютерное моделирование 
проводилось на персональном компьютере Intel Core~i7 с~многоядерным 
графическим процессором. Для экспериментальных исследований 
использовалась камера Kinect~2.0 в~качестве RGB-D-сен\-со\-ра. Каждый из 
четырех наборов данных из тестовой базы содержит файлы дампа, изображение 
в~формате ppm (рис.~1,\,\textit{а}), данные о~глубине в~формате pgm 
(рис.~1,\,\textit{б}). Данные в~RGB-D-кад\-рах получены из разных положений 
камеры на сцене. Например, набор данных Classrooms содержит~688~ключевых 
кад\-ров размером $640\times 480$.
  

  
  Для повышения точности решения вариационной задачи точка--точка 
в~предложенном методе используются данные о~визуально связанных 
характеристиках изображений, а~значит решение вариационной задачи в~целом 
зависит от точности сопоставления данных на изображениях в~RGB-D-кад\-ре. 
Проведем сравнительный анализ известных дескрипторных методов (SIFT
(scale-invariant feature transform), 
SURF (speeded-up robust features) и~ORB (oriented fast and rotated brief)) 
и~предложенного метода со\-по\-став\-ле\-ния особых точек 
(HOGs, histogram oriented gradients)~\cite{15-voh, 16-voh}. 

Будем оценивать точность методов по чис\-лу 
правильных сопоставлений при разных значениях угла поворота и~изменения 
мас\-шта\-ба. Были получены следующие результаты зависимости точности 
реконструкции трехмерной комбинированной карты и~производительности от 
выбранного типа двумерного дескриптора (см.\ таб\-ли\-цу)~\cite{20-voh}. 
В~таб\-ли\-це приведены средние значения точности для разного типа аффинных 
преобразований (угол поворота и~масштаб) для различных наборов данных из 
тес\-то\-вой базы данных и~различных двумерных дескрипторов.
  
  \begin{table*}\small
  \begin{center}
  \begin{tabular}{|l|c|c|c|c|}
  \multicolumn{5}{p{118mm}}{Зависимость точности/производительности метода 
реконструкции трехмерной комбинированной плотной карты от типа двумерного 
дескриптора}\\
  \multicolumn{5}{c}{\ }\\[-6pt]
  \hline
\multicolumn{1}{|c|}{Метод}&\tabcolsep=0pt\begin{tabular}{c}Набор данных\\ Classrooms\end{tabular}&
\tabcolsep=0pt\begin{tabular}{c}Набор данных\\ Living Rooms (1/4)\end{tabular}&
\tabcolsep=0pt\begin{tabular}{c}Набор данных\\ Offices (1/2)\end{tabular}&
\tabcolsep=0pt\begin{tabular}{c}Набор данных\\ Offices (2/2)\end{tabular}\\
\hline
&\multicolumn{4}{c|}{Точность, \%}\\
\hline
SIFT&93&93&94&95\\
SURF&57&61&63&59\\
ORB&78&82&75&75\\
HOGs&93&96&97&96\\
\hline
&\multicolumn{4}{c|}{Производительность, с}\\
\hline
SIFT&17,2&26,21\hphantom{9}&12,82\hphantom{9}&18,9\hphantom{99}\\
SURF&0,23&0,7\hphantom{9}&0,4\hphantom{9}&0,44\\
ORB&3,34&3,42&5,15&3,33\\
HOGs&0,79&1,11&0,68&0,81\\
\hline
\end{tabular}
\end{center}
\vspace*{-2pt}
\end{table*}

\begin{figure*}[b] %fig2
   \vspace*{1pt}
    \begin{center}  
  \mbox{%
 \epsfxsize=163mm 
 \epsfbox{voh-2.eps}
 }
\end{center}
\vspace*{-10pt}
  \Caption{Сравнение скорости сходимости в~зависимости 
  от ошибки метрики и~условий 
наблюдения: (\textit{а})~изменение ско\-рости сходимости в~контролируемых 
условиях; 
(\textit{б})~изменение скорости сходимости в~неконтролируемых условиях;
  \textit{1}~--- точ\-ка--точ\-ка; \textit{2}~--- точ\-ка--точ\-ка с~экстраполяцией; \textit{3}~--- 
точ\-ка--плос\-кость; \textit{4}~--- комбинированный подход}
  \end{figure*}
  
  В работе проведен сравнительный анализ скорости сходимости 
предложенного метода регистрации данных и~известных методов регистрации 
на основе итеративного алгоритма ближайших точек\linebreak
 в~зависимости от выбора 
метрики ошибки пар точек: точ\-ка--точ\-ка~\cite{6-voh}, точ\-ка--точ\-ка 
с~экстраполяцией~\cite{8-voh}, точ\-ка--плос\-кость~\cite{14-voh} в~терминах 
среднеквадратичной ошибки. Для проведения \mbox{компьютерного} моделирования 
было выбрано два набора данных из базы данных NYU Depth Dataset: 
в~контролируемых условиях (Offices~(1/2), сцена~1) и~в~неконтролируемых 
условиях (Offices~(1/2), сцена~2). 

В~результате серии тестов установлены 
зависимости скорости сходимости методов от выбора\linebreak ошибки метрики 
и~условий проведения экс\-пе\-ри\-мен\-тов (рис.~2). Было установлено, что 
в~конт\-ро\-ли\-ру\-емых условиях предложенный метод регистрации данных имеет 
сходимость, близкую \mbox{к~обеспечиваемой} методом, который использует метрику 
точ\-ка--плос\-кость (рис.~2,\,\textit{а}). В~неконтролируемых условиях (при 
неравномерном освещении) предложенный метод показывает лучшую 
сходимость, чем указанные выше методы регистрации данных 
(рис.~2,\,\textit{б}). Дополнительно было установлено, что точность метода 
реконструкции зависит от числа особых точек в~RGB-D-кад\-ре нелинейным 
образом~--- в~виде функции с~одним ярко выраженным пиком для всех типов 
дескрипторов~\cite{21-voh, 22-voh}.
  
  
  
  Как известно, дескриптор HOG инвариантен к~неравномерному освещению 
и~фотометрическим преобразованиям, которые проявляются на больших 
сценах~\cite{15-voh}. Был проведен сравнительный анализ за\-ви\-си\-мости 
ско\-рости схо\-ди\-мости методов регистрации от выбора типа дескриптора 
и~условий проведения экспериментов (рис.~3). Было установлено, что 
в~конт\-ро\-ли\-ру\-емых условиях тип используемого дескриптора имеет 
ограниченное влияние на схо\-ди\-мость метода регистрации данных\linebreak 
(рис.~3,\,\textit{а}). В~неконтролируемых условиях ис\-поль\-зование дескриптора 
HOGs позволяет получить\linebreak лучшую схо\-ди\-мость в~сравнении с~другими 
дескрипторными методами: предложенный метод регистрации сходится уже 
после 11-й итерации, тогда как при использовании дескриптора ORB метод 
сходится только после 16-й итерации (рис.~3,\,\textit{б}).

 На рис.~4 показаны 
результаты применения предложенного метода регистрации для 
по\-сле\-до\-ва\-тель\-ности ключевых RGB-D-кадров из базы 
дан-\linebreak\vspace*{-10pt}

\pagebreak

\end{multicols}

\begin{figure*} %fig3
   \vspace*{1pt}
    \begin{center}  
  \mbox{%
 \epsfxsize=163.096mm 
 \epsfbox{voh-3.eps}
 }
\end{center}
\vspace*{-10pt}
  \Caption{Сравнение скорости сходимости в~зависимости от типа дескриптора и~условий 
наблюдения: (\textit{а})~изменение скорости сходимости в~контролируемых условиях; 
(\textit{б})~изменение скорости сходимости в~неконтролируемых условиях; \textit{1}~--- 
SIFT; \textit{2}~--- SURF; \textit{3}~--- ORB; \textit{4}~--- HOGs}
  %\end{figure*}
 % 
 % \begin{figure*} %fig4
   \vspace*{15pt}
    \begin{center}  
  \mbox{%
 \epsfxsize=163mm 
 \epsfbox{voh-4.eps}
 }
\end{center}
\vspace*{-10pt}
  \Caption{Реконструкция трехмерного объекта на основе предложенного метода 
регистрации данных с~разных точек обзора}
  \end{figure*}
  

\begin{multicols}{2}

\noindent
ных NYU Depth 
Dataset: исходное облако точек визуально совмещено с~целевым облаком точек 
на примере статуи головы человека. Недостаток классического метода 
ICP~\cite{6-voh}~--- большая вычислительная слож\-ность. Проекционные 
методы могут сократить вы\-чис\-ли\-тель\-ную слож\-ность метода регистрации ICP 
с~$O(N_S\log(N_T))$ для метода ICP с~k-D-де\-ре\-вом до~$O(N_S)$ для метода 
ICP c~ограничением в~виде сферы или треугольника.
  
  
  
  Для ускорения процедуры сопоставления данных в~работе используется 
пирамидальный подход, основанный на низкочастотной фильт\-ра\-ции 
и~по\-сле\-ду\-ющей дискретизации результирующего изоб\-ра\-же\-ния/об\-ла\-ка 
точек. Вы\-чис\-ли\-тель\-ная слож\-ность пред\-ла\-га\-емо\-го метода регистрации может 
быть оценена сле\-ду\-ющим образом: $n_1+ n_2O_1/F^1+\cdots\linebreak \cdots +  
n_kO_1/F^{k-1}$,  где $k$~--- чис\-ло шагов дискретизации; $n_i$~--- чис\-ло 
итераций в~методе регистрации, выполненных на шаге~$k$; $O_1$~--- 
вы\-чис\-ли\-тель\-ная слож\-ность первого шага алгоритма; $F$~--- параметр, 
опре\-де\-ля\-ющий разбиение RGB-D-кадра.

\vspace*{-6pt}

\section{Выводы}

  В работе предложен комбинированной метод решения вариационной задачи 
точка--точка в~замк\-ну\-той форме для аффинных преобразований, который 
создает основу для распространения метода Хорна на случай с~неригидными 
объектами на сцене. Было проведено сравнение пред\-ла\-га\-емо\-го метода 
регистрации данных с~методом Хорна для метрики точ\-ка--точ\-ка 
с~экстраполяцией и~без экстраполяции, а~также с~методом регистрации для\linebreak 
метрики точ\-ка--плос\-кость. В~результате компьютерного моделирования было 
установлено, что применение визуально связанных характеристик для решения 
вариационной задачи алгоритма ICP позво\-ля\-ет улучшить сходимость метода 
в~неконтролируемых условиях. Использование визуально\linebreak связанных 
характеристик изображений поз\-во\-ля\-ет решить проб\-ле\-му за\-ви\-си\-мости 
результата решения вариационной задачи от пра\-виль\-ности выбора начальных 
значений. Двумерный дескриптор HOGs обладает лучшими характеристиками 
по сравнению с~известными дескрипторами при малых поворотах в~об\-ласти 
сцены. Предложенный метод используется для регистрации облаков точек 
с~произвольным пространственным разрешением и~масштабом относительно 
друг друга, дает точные оценки для слож\-ных крупномасштабных сцен.
  
{\small\frenchspacing
 {\baselineskip=11.5pt
 \addcontentsline{toc}{section}{References}
 \begin{thebibliography}{99}
\bibitem{1-voh}
\Au{Vidal-Calleja T.\,A., Berger~C., Sola~J., Lacroix~S.} Large scale multiple 
robot visual mapping with heterogeneous landmarks in semi-structured terrain~// 
J.~Robotics Autonomous Systems, 2011. Vol.~59. Iss.~9. P.~654--674. doi: 
10.1016/j.robot.2011.05.008.
\bibitem{2-voh}
\Au{Vokmintsev A., Timchenko~M., Yakovlev~K.} Simultaneous localization and 
mapping in unknown environment using dynamic matching of images and 
registration of point clouds~// 2nd Conference (\mbox{International}) on Industrial 
Engineering, Applications and Manufacturing.~--- IEEE, 2017. Art. ID 7910967. 
6~p. doi: 10.1109/ICIEAM.2016.7910967.
\bibitem{3-voh}
\Au{Bokovoy A., Yakovlev~K.} Sparse 3D point-cloud map upsampling and noise 
removal as a~vSLAM post-processing step: Experimental evaluation~// 
Interactive collaborative robotics~/ 
Eds. A.~Ronzhin, G.~Rigoll, R.~Meshcheryakov.~--- Lecture 
notes in computer science ser.~--- Springer, 2018. Vol.~11097. P.~23--33.

\bibitem{5-voh}
\Au{Tam G., Cheng~Z.-Q., Lai~Y.-K., Langbein~F., Liu~Y., Marshall~D., 
Martin~R., Rosin~P.} Registration of 3D point clouds and meshes: A~survey 
from rigid to nonrigid~// IEEE~T. Vis. Comput. Gr., 2013. Vol.~19. Iss.~7. 
P.~1199--1217. doi: 10.1109/tvcg.2012.310.

\bibitem{4-voh}
\Au{Picos K., Diaz-Ramirez~V.\,H., Kober~V., Montemayor~A.\,S., 
Pantrigo~J.\,J.} Accurate three-dimensional pose recognition from monocular 
images using template matched filtering~// Opt. Eng., 2016. Vol.~55. 
Iss.~6. Art. ID 063102. doi: 10.1117/1.oe.55.6.063102.

\bibitem{6-voh}
\Au{Besl P., McKay~N.} A~method for registration of \mbox{3-D} shapes~// 
IEEE~T. Pattern Anal., 1992. Vol.~14. Iss.~2. P.~239--256. doi: 
10.1109/34.121791.
\bibitem{7-voh}
\Au{Cheng~S., Marras~I., Zafeiriou~S., Pantic~M.} Statistical non-rigid ICP 
algorithm and its application to 3D face alignment~// Image Vision Comput., 2017. 
Vol.~58. P.~3--12. doi: 10.1016/j.imavis.2016.10.007.
\bibitem{8-voh}
\Au{Horn B.} Closed-form solution of absolute orientation using unit 
quaternions~// J.~Opt. Soc. Am.~A, 1987. Vol.~4. Iss.~4.  
P.~629--642. doi: 10.1364/josaa.4.000629.
\bibitem{9-voh}
\Au{Horn B., Hilden H., Negahdaripour~S.} Closed-form solution of absolute 
orientation using orthonormal matrices~// J.~Opt. Soc. Am.~A, 1988. 
Vol.~5. Iss.~7. P.~1127--1135. doi: 10.1364/josaa.5.001127.
\bibitem{10-voh}
\Au{Khoshelham K.} Closed-form solutions for estimating a~rigid motion from 
plane correspondences extracted from point clouds~// ISPRS J.~Photogramm., 
2016. Vol.~114. Р.~78--91. doi: 10.1016/j.isprsjprs.2016.01.010.
\bibitem{11-voh}
\Au{Du S., Liu J., Zhang~C., Zhu~J., Li~K.} Probability iterative closest point 
algorithm for m-D point set registration with noise~// Neurocomputing, 2015. 
Vol.~157. Iss.~1. Р.~187--198. doi: 10.1016/j.neucom.2015.01.019.
\bibitem{12-voh}
\Au{Cheng S., Marras I., Zafeiriou~S.} Active nonrigid ICP algorithm~// 11th 
IEEE Conference (International) and Workshops on Automatic Face and Gesture 
Recognition Proceedings.~--- IEEE, 2015. Art. ID 7163161. 8~p. doi: 
10.1109/FG.2015.7163161.
\bibitem{13-voh}
\Au{Echeagaray-Patron B.\,A., Kober~V., Karnaukhov~V., Kuznetsov~V.} 
A~method of face recognition using 3D facial surfaces~// J.~Commun. 
Technol. El., 2017. Vol.~62. Iss.~6. Р.~648--652. doi: 
10.1134/s1064226917060067.
\bibitem{14-voh}
\Au{Low K.\,L.} Linear least-squares optimization for point-to-plane ICP surface 
registration.~--- Chapel Hill, NC, USA: University of 
North Carolina at Chapel Hill, Department of Computer Science, 2004. 
 Technical Report TTR04-004.
 {\sf 
https://www.comp.nus.edu.sg/$\sim$lowkl/\linebreak publications/lowk\_point-to-plane\_icp\_techrep.pdf}.
\bibitem{15-voh}
\Au{Вохминцев А.\,В., Соченков~И.\,В., Кузнецов~В.\,В., Тихоньких~Д.\,В.} 
Распознавание лиц на основе алгоритма сопоставления изображений 
с~рекурсивным вычислением гистограмм направленных градиентов~// 
Докл.  Акад. наук, 2016. Т.~466. №\,3. 
С.~261. doi: 10.7868/ S0869565216030087.


\bibitem{16-voh}
\Au{Diaz-Escobar J., Kober~V.} A~robust HOG-based descriptor for pattern 
recognition~// Proc. SPIE, 2016. Vol.~9971.
 Art. ID 99712A. doi: 10.1117/12.2237963.
\bibitem{17-voh}
\Au{Vokhmintcev A., Yakovlev~K.} A~real-time algorithm for mobile robot 
mapping based on rotation-invariant descriptors and ICP~// Comm. 
Comp. Inf. Sc., 2016. Vol.~661. P.~357--369.
\bibitem{18-voh}
\Au{Silberman N., Kohli~P., Hoiem~D., Fergus~R.} NYU Depth Dataset~V2. {\sf 
https://cs.nyu.edu/$\sim$silberman/datasets/\linebreak  nyu\_depth\_v2.html}.
\bibitem{19-voh}
\Au{Silberman N., Hoiem~D., Kohli~P., Fergus~R.} Indoor segmentation and 
support inference from RGBD images~// Computer 
vision~/ Eds. A.\,W.~Fitzgibbon, S.~Lazebnik, P.~Perona,
\textit{et al.}~--- Lecture notes in computer science ser.~--- 2012. Vol.~7576.  
P.~746--760.
\bibitem{20-voh}
\Au{Vokhmintcev A., Botova~T., Sochenkov~I., Sochenkova~A., Makovetskii~A.} 
Robot mapping algorithm based on Kalman filtering and symbolic tags~// Proc. 
SPIE, 2017. Vol.~10396. Art. 
ID~103962I. doi: 10.1117/12.2273562.
\bibitem{21-voh}
\Au{Vokhmintcev A., Timchenko~M., Melnikov~A., Kozko~A., Makovetskii~A.} 
Robot path planning algorithm based on symbolic tags in dynamic environment~// 
Proc. SPIE, 2017. Vol.~10396. Art. 
ID~103962E. doi: 10.1117/ 12.2273279.
\bibitem{22-voh}
\Au{Sochenkov I., Vokhmintsev~A.} Visual duplicates image search for  
a~non-cooperative person recognition at a~distance~// Procedia Engineer., 
2015. Vol.~129. Р.~440--445. doi: 10.1016/j.proeng.2015.12.147.
 \end{thebibliography}

 }
 }

\end{multicols}

\vspace*{-3pt}

\hfill{\small\textit{Поступила в~редакцию 25.02.19}}

\vspace*{8pt}

%\pagebreak

%\newpage

%\vspace*{-28pt}

\hrule

\vspace*{2pt}

\hrule

%\vspace*{-2pt}

\def\tit{SIMULTANEOUS LOCALIZATION AND~MAPPING 
METHOD IN~THREE-DIMENSIONAL SPACE BASED 
ON~THE~COMBINED SOLUTION OF~THE~POINT--POINT VARIATION 
PROBLEM ICP FOR~AN~AFFINE TRANSFORMATION}


\def\titkol{Simultaneous localization and~mapping 
method in~three-dimensional space based 
on~the~combined solution}
% of~the~point--point variation 
%problem ICP for~an~affine transformation}

\def\aut{A.\,V.~Vokhmintcev$^{1,2}$, A.\,V.~Melnikov$^2$, and~S.\,A.~Pachganov$^2$}

\def\autkol{A.\,V.~Vokhmintcev, A.\,V.~Melnikov, and~S.\,A.~Pachganov}

\titel{\tit}{\aut}{\autkol}{\titkol}

\vspace*{-11pt}


\noindent
$^1$Chelyabinsk State University, 129 Br.~Kashirinyh Str., Chelyabinsk 454001, 
Russian Federation


\noindent
$^2$Ugra State University, 16~Chekhov Str., Khanty-Mansiysk 628012, Russian 
Federation

\def\leftfootline{\small{\textbf{\thepage}
\hfill INFORMATIKA I EE PRIMENENIYA~--- INFORMATICS AND
APPLICATIONS\ \ \ 2020\ \ \ volume~14\ \ \ issue\ 1}
}%
 \def\rightfootline{\small{INFORMATIKA I EE PRIMENENIYA~---
INFORMATICS AND APPLICATIONS\ \ \ 2020\ \ \ volume~14\ \ \ issue\ 1
\hfill \textbf{\thepage}}}

\vspace*{3pt} 
     


\Abste{Simultaneous localization and mapping is a~problem in which frame data are 
used as the only source of external information to define the position of a~moving 
camera in space and at the same time, to reconstruct a~map of the study area. 
Nowadays, this problem is considered solved for the construction of two-dimensional 
maps for small static scenes using range sensors such as lasers or sonar. However, 
for dynamic, complex, and large-scale scenes, the construction of an accurate  
three-dimensional map of the surrounding space is an active area of research. To 
solve this problem, the authors propose a~solution of the point--point problem 
for  an affine transformation and develop a~fast iterative algorithm 
for point  clouds registering in three-dimensional space. The performance and computational complexity 
of the proposed method are presented and discussed by an example of reference data. 
The results can be applied  for navigation tasks of a~mobile robot
in real-time.}

\KWE{registration problem; localization; simultaneous localization and mapping; 
affine transformation; two-dimensional descriptors; iterative closest point}



\DOI{10.14357/19922264200114} 

\vspace*{-6pt}

\Ack
\noindent
This work was partially supported by the Russian Foundation for Basic Research 
(grant 18-37-20032) and by the Russian Science Foundation (project  
No.\,15-19-10010).

\vspace*{9pt}

  \begin{multicols}{2}

\renewcommand{\bibname}{\protect\rmfamily References}
%\renewcommand{\bibname}{\large\protect\rm References}

{\small\frenchspacing
 {%\baselineskip=10.8pt
 \addcontentsline{toc}{section}{References}
 \begin{thebibliography}{99}
 
 %\vspace*{-3pt}
\bibitem{1-voh-1}
\Aue{Vidal-Calleja, T.\,A., C.~Berger, J.~Sola, and S.~Lacroix.} 2011. Large 
scale multiple robot visual mapping with heterogeneous landmarks in semi-
structured terrain. \textit{J.~Robotics Autonomous Systems} 59(9):654--674. doi: 
10.1016/j.robot.2011.05.008.
\bibitem{2-voh-1}
\Aue{Vokmintsev, A., M.~Timchenko, and K.~Yakovlev.} 2017. Simultaneous 
localization and mapping in unknown\linebreak environment using dynamic matching of 
images and registration of point clouds. \textit{2nd  Conference 
(International) on Industrial Engineering, Applications and Manufacturing}. 
IEEE. Art. 
ID 7910967. 6~p. doi: 10.1109/ ICIEAM.2016.7910967.
\bibitem{3-voh-1}
\Aue{Bokovoy, A., and K.~Yakovlev.} 2018. Sparse 3D point-cloud map 
upsampling and noise removal as a~vSLAM post-processing step: Experimental 
evaluation. \textit{Interactive collaborative 
robotics}. Eds. A.~Ronzhin, G.~Rigoll, and R.~Meshcheryakov.
Lecture notes in computer science ser. Springer. 11097:23--33.

\bibitem{5-voh-1}
\Aue{Tam, G., Z.-Q.~Cheng, Y.-K.~Lai, F.~Langbein, Y.~Liu, D.~Marshall, 
R.~Martin, and P.~Rosin.} 2013. Registration of 3D point clouds and meshes: 
A~survey from rigid to nonrigid. \textit{IEEE~T. Vis. Comput. Gr.}  
19(7):1199--1217. doi: 10.1109/tvcg.2012.310.

\bibitem{4-voh-1}
\Aue{Picos, K., V.\,H.~Diaz-Ramirez, V.~Kober, A.\,S.~Montemayor, and 
J.\,J.~Pantrigo.} 2016. Accurate three-dimensional pose recognition from 
monocular images using template matched filtering. \textit{Opt. 
Eng.} 55(6):063102. doi: 10.1117/1.oe.55.6.063102.


\bibitem{6-voh-1}
\Aue{Besl, P., and N.~McKay.} 1992. A~method for registration of 3-D shapes. 
\textit{IEEE~T. Pattern Anal.} 14(2):239--256. 
doi: 10.1109/34.121791.
\bibitem{7-voh-1}
\Aue{Cheng, S., I.~Marras, S.~Zafeiriou, and M.~Pantic.} 2017. Statistical 
non-rigid ICP algorithm and its application to 3D face alignment. \textit{IEEE 
Image Vision Comput.} 58:3--12. doi: 10.1016/j.imavis.2016.10.007.
\bibitem{8-voh-1}
\Aue{Horn, B.} 1987. Closed-form solution of absolute orientation using unit 
quaternions. \textit{J.~Opt. Soc. Am.~A} 4(4):629--642. doi: 
10.1364/josaa.4.000629.
\bibitem{9-voh-1}
\Aue{Horn, B., H.~Hilden, and S.~Negahdaripour.} 1988. Closed-form solution 
of absolute orientation using orthonormal matrices. \textit{J.~Opt. Soc. 
Am.~A} 5(7):1127--1135. doi: 10.1364/JOSAA.5.001127.
\bibitem{10-voh-1}
\Aue{Khoshelham, K.} 2016. Closed-form solutions for estimating a rigid motion 
from plane correspondences extracted from point clouds. \textit{J.~ISPRS 
Photogramm.} 114:78--91. doi: 
10.1016/j.isprsjprs.2016.01.010.
\bibitem{11-voh-1}
\Aue{Du, S., J.~Liu, C.~Zhang, J.~Zhu, and K.~Li.} 2015. Probability iterative 
closest point algorithm for m-D point set registration with noise. 
\textit{Neurocomputing} 157(1):187--198. doi: 10.1016/j.neucom.2015.01.019.
\bibitem{12-voh-1}
\Aue{Cheng, S., I.~Marras, and S.~Zafeiriou.} 2015. Active nonrigid ICP 
algorithm. \textit{IEEE 11th  Conference (International) and Workshops on 
Automatic Face and Gesture Recognition Proceedings}. Art. ID 7163161. 8~p.
doi: 10.1109/FG.2015.7163161.
\bibitem{13-voh-1}
\Aue{Echeagaray-Patron, B.\,A., V.~Kober, V.~Karnaukhov, and V.~Kuznetsov.} 
2017. A~method of face recognition using 3D facial surfaces. 
\textit{J.~Commun. Technol. El.} 62(6):648--652. doi: 
10.1134/s1064226917060067.
\bibitem{14-voh-1}
\Aue{Low, K.\,L.} 2004. Linear least-squares optimization for point-to-plane ICP 
surface registration.  Chapel Hill, NC: University of 
North Carolina at Chapel Hill, Department of Computer Science. 
Technical Report TTR04-004. Available at: 
{\sf  
https://www.comp.nus.edu.sg/$\sim$lowkl/\linebreak publications/lowk\_point-to-plane\_icp\_techrep.pdf} (accessed December~17, 2019).
\bibitem{15-voh-1}
\Aue{Vokhmintcev, A.\,V., I.\,V.~Sochenkov, V.\,V.~Kuznetsov, and 
D.\,V.~Tikhonkikh.} 2016. Face recognition based on a~matching algorithm with 
recursive calculation of oriented gradient histograms. \textit{Doklady 
Mathematics}  
93(1):37--41. doi: 10.1134/s1064562416010178.
\bibitem{16-voh-1}
\Aue{Diaz-Escobar, J., and V.~Kober.} 2016. A~robust HOG-based descriptor 
for pattern recognition. \textit{Proc. SPIE} 9971:99712A. doi: 10.1117/12.2237963.
\bibitem{17-voh-1}
\Aue{Vokhmintcev, A., and K.~Yakovlev.} 2016. A~real-time algorithm for 
mobile robot mapping based on rotation-invariant descriptors and ICP. 
\textit{Comm. Comp. Inf. Sc.} 661:357--369. 
\bibitem{18-voh-1}
\Aue{Silberman, N., P.~Kohli, D.~Hoiem, and R.~Fergus}. NYU depth dataset 
V2. Available at: {\sf 
https://cs.nyu.edu/\linebreak $\sim$silberman/datasets/nyu\_depth\_v2.html} (accessed 
December~17, 2019).
\bibitem{19-voh-1}
\Aue{Silberman, N., D.~Hoiem, P.~Kohli, and R.~Fergus.} 2012. Indoor 
segmentation and support inference from RGBD Images. \textit{Computer vision}. 
Eds. A.\,W.~Fitzgibbon, S.~Lazebnik, P.~Perona,
\textit{et al.}
Lecture notes in computer science ser.  
7576:746--760. 
\bibitem{20-voh-1}
\Aue{Vokhmintcev, A., T.~Botova, I.~Sochenkov, A.~Sochenkova, and 
A.~Makovetskii.} 2017. Robot mapping algorithm based on Kalman filtering and 
symbolic tags. \textit{Proc. SPIE} 10396:103962I. doi: 10.1117/12.2273562.
\bibitem{21-voh-1}
\Aue{Vokhmintcev, A., M.~Timchenko, A.~Melnikov, A.~Kozko, and 
A.~Makovetskii.} 2017. Robot path planning algorithm based on symbolic tags in 
dynamic environment. \textit{Proc. SPIE} 10396:103962E. doi: 10.1117/12.2273279.
\bibitem{22-voh-1}
\Aue{Sochenkov, I., and A.~Vokhmintsev.} 2015. Visual duplicates image search 
for a~non-cooperative person recognition at a~distance. \textit{Procedia 
Engineer.} 129:440--445. doi: 10.1016/j.proeng.2015.12.147.
\end{thebibliography}

 }
 }

\end{multicols}

%\vspace*{-7pt}

\hfill{\small\textit{Received February 25, 2019}}

%\pagebreak

%\vspace*{-22pt}

\Contr

\noindent
\textbf{Vokhmintcev Alexander V.} (b.\ 1978)~--- 
Candidate of Science (PhD) in technology, associate professor; 
head of laboratory, Chelyabinsk State University, 
129~Br.~Kashirinyh Str., Chelyabinsk 454001, Russian Federation; 
associate professor, Ugra State University, 16~Chekhov Str.,
 Khanty-Mansiysk, 628012, Russian Federation; 
\mbox{vav@csu.ru}

\vspace*{3pt}

\noindent
\textbf{Melnikov Andrey V.} (b.\ 1956)~--- Doctor of Science in technology, 
professor, Ugra State University, 16~Chekhov Str., Khanty-Mansiysk 628012, 
Russian Federation; \mbox{melnikovav@uriit.ru}

\vspace*{3pt}

\noindent
\textbf{Pachganov Stepan A.} (b.\ 1994)~---
 PhD student, Ugra State University, 16~Chekhov Str., Khanty-Mansiysk 628012, 
Russian Federation; \mbox{pachganovsa@uriit.ru}
     
\label{end\stat}

\renewcommand{\bibname}{\protect\rm Литература} 