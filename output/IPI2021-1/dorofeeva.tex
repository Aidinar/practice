
\def\stat{dorofeeva}

\def\tit{О ТОЧНОСТИ НОРМАЛЬНОЙ АППРОКСИМАЦИИ
ПРИ~ОТСУТСТВИИ НОРМАЛЬНОЙ СХОДИМОСТИ$^*$}

\def\titkol{О точности нормальной аппроксимации
при~отсутствии нормальной сходимости}

\def\aut{В.\,Ю.~Королев$^1$,  А.\,В.~Дорофеева$^2$}

\def\autkol{В.\,Ю.~Королев,  А.\,В.~Дорофеева}

\titel{\tit}{\aut}{\autkol}{\titkol}

\index{Королев В.\,Ю.}
\index{Дорофеева А.\,В.}
\index{Korolev V.\,Yu.}
\index{Dorofeeva A.\,V.}

{\renewcommand{\thefootnote}{\fnsymbol{footnote}} \footnotetext[1]
{Работа выполнена при поддержке РФФИ (проект 18-07-01405).
  }}

\renewcommand{\thefootnote}{\arabic{footnote}}
\footnotetext[1]{Факультет вычислительной математики и~кибернетики Московского государственного университета имени
 М.\,В.~Ломоносова; Институт проб\-лем информатики Федерального исследовательского цент\-ра 
 <<Информатика и~управ\-ле\-ние>> Российской академии наук, \mbox{vkorolev@cs.msu.ru}}
\footnotetext[2]{Факультет вычислительной математики и~кибернетики Московского государственного 
 университета имени М.\,В.~Ломоносова, \mbox{alex.dorofeyeva@gmail.com}}

%\vspace*{-12pt}

  

\Abst{При решении прикладных задач в~самых разных областях принято использовать 
нормальное распределение в~качестве модели статистических закономерностей в~наблюдаемых 
данных с~аддитивной структурой. В~качестве критерия степени адекватности такой 
модели можно использовать оценки ско\-рости сходимости в~центральной предельной теореме (ЦПТ) 
тео\-рии вероятностей, устанавливающей, что при определенных условиях (например, 
при условии Линдеберга) суммарное воздействие большого числа случайных факторов проявляется в~виде 
случайной величины с~нормальным распределением. 
Классические оценки скорости сходимости в~ЦПТ 
типа неравенства Бер\-ри--Эс\-се\-ена доказаны при условии конечности третьих моментов слагаемых.
 Известны также оценки скорости сходимости при существовании моментов порядка $2\hm+\delta$ с~$0<\delta\hm<1$. 
 Если существуют моменты лишь второго порядка, то сходимость в~ЦПТ может быть как угодно медленной. 
 Если же у слагаемых моменты второго порядка не существуют, то сходимость распределений сумм 
 независимых случайных величин к~нормальному закону не имеет места. 
 Условия, гарантирующие 
 справедливость ЦПТ, практически невозможно достоверно проверить 
 при ограниченном объеме наблюдаемой выборки. Поэтому вопрос о том, какой может быть 
 реальная точ\-ность нормальной аппроксимации, когда она теоретически не применима, 
 но используется в~практических вычислениях, представляет большой интерес. Более того, в~некоторых 
 ситуациях при имитационном моделировании, когда распределения слагаемых принадлежат об\-ласти 
 притяжения устойчивого закона с~характеристическим показателем, меньшим двух, при увеличении числа 
 слагаемых сначала наблюдается убывание расстояния между распределением нормированной суммы и~нормальным 
 законом и~лишь при довольно большом числе слагаемых это расстояние начинает увеличиваться. 
 В~данной заметке предпринята попытка дать ответ на сформулированный выше вопрос и~привести 
 некоторые теоретические объяснения указанному эффекту.}

\KW{центральная предельная теорема; точ\-ность нормальной аппроксимации; тяжелые хвос\-ты; 
равномерное расстояние}

\DOI{10.14357/19922264210116}

\vspace*{3pt}

\vskip 10pt plus 9pt minus 6pt

\thispagestyle{headings}

\begin{multicols}{2}

\label{st\stat}

\section{Введение}


При решении прикладных задач в~самых разных областях принято использовать нормальное 
распределение в~качестве модели статистических закономерностей в~наблюдаемых данных с~аддитивной структурой. 
В~качестве критерия степени адекватности такой модели можно использовать оценки ско\-рости 
сходимости в~ЦПТ тео\-рии вероятностей, устанавливающей, 
что при определенных условиях (например, при условии Линдеберга) суммарное воздействие 
большого числа случайных факторов проявляется в~виде случайной величины с~нормальным распределением. 

Классические оценки скорости сходимости в~ЦПТ типа неравенства Бер\-ри--Эс\-се\-ена доказаны при 
условии конечности третьих моментов сла\-га\-емых. 

Известны также оценки ско\-рости сходимости при 
существовании моментов порядка $2\hm+\delta$ с~$0\hm<\delta\hm<1$ (см.\ подробный обзор в~[1]).
 Если существуют моменты лишь второго порядка, то сходимость в~ЦПТ может быть как угодно медленной~[2, 3]. 
 Если же у слагаемых моменты второго порядка не существуют, то сходимость распределений 
 сумм независимых случайных величин к~нормальному закону не имеет места. 
 
 Условия, 
 гарантирующие справедливость ЦПТ, практически 
 невозможно достоверно проверить при ограниченном объеме наблюдаемой выборки. 
 В~част\-ности, гистограмма, построенная по выборке из имеющего очень тяжелые хвосты 
 распределения Коши (у~которого отсутствует даже математическое ожидание), при умеренном 
 объеме выборки может быть визуально практически неотличимой от нормального распределения. 
 Поэтому вопрос о том, какой может быть реальная точность нормальной аппроксимации, когда 
 она теоретически не применима, но используется в~практических вычислениях, представляет 
 большой интерес. Более того, в~некоторых ситуациях при имитационном моделировании, 
 когда распределения слагаемых принадлежат области притяжения устойчивого закона с~характеристическим 
 показателем, меньшим двух, при увеличении числа слагаемых сначала наблюдается убывание 
 расстояния между распределением нормированной суммы и~нормальным законом и~лишь при довольно 
 большом числе слагаемых это расстояние начинает увеличиваться. 
 
 В~данной заметке предпринята 
 попытка дать ответ на сформулированный выше вопрос и~привести некоторые теоретические объяснения 
 указанному эффекту.


\section{Обозначения и~вспомогательные результаты}


Пусть $n\hm\in\mathbb{N}$, $\xi_1,\ldots,\xi_n$~--- независимые необязательно одинаково 
распределенные случайные величины, заданные на вероятностном пространстве 
$(\Omega,\mathfrak{A},{\sf P})$. Обозначим $F_j(x)\hm={\sf P}(\xi_j<x)$, $x\hm\in\mathbb{R}$, 
$j\hm\in\mathbb{N}$. Без существенного ограничения общ\-ности для удобства будем считать, 
что все функции распределения~$F_j(x)$ непрерывны.

Обозначим $S_n=\xi_1+\cdots+\xi_n$. Индикатор множества (события)~$A$ обозначим $\mathbb{I}(A)$. 
Пусть $u\hm>0$. Очевидно, 
$$
\xi_j=\xi_j\mathbb{I}\left(|\xi_j|\le u\right)+\xi_j\mathbb{I}\left(|\xi_j|> u\right)\,.
$$
 Тогда
\begin{multline*}
S_n=\sum\limits_{j=1}^n \xi_j\mathbb{I}(|\xi_j|\le u)+\sum\limits_{j=1}^n \xi_j\mathbb{I}(|\xi_j|> u)\equiv{}\\
{}\equiv
S_n^{(\le u)}+S_n^{(> u)}.
\end{multline*}
Если условиться считать равенство единице индикатора $\mathbb{I}(|\xi_j|\hm\le u)$ <<успехом>>, 
а~противоположное событие~--- <<неудачей>>, то число~$N_n(u)$ ненулевых слагаемых в~сумме 
$S_n^{(\le u)}$ будет случайной величиной, имеющей пуас\-сон-би\-но\-ми\-аль\-ное распределение 
с~па\-ра\-мет\-ра\-ми~$n$ и~$p_j\hm=p_j(u)\hm={\sf P}(|\xi_j|\hm\le u)\hm=F_j(u)\hm-F_j(-u)$, 
$j\hm=1,\ldots,n$. Заметим, что при неограниченном увеличении~$u$ параметры~$p_j$ стремятся к~единице.

\smallskip

\noindent
\textbf{Лемма~1.}\ \textit{Пусть $A,B\hm\in\mathfrak{A}$. Тогда} 
$$
{\sf P}(A\bigcap B)\ge{\sf P}(A) -{\sf P}\left(\overline{B}\right).
$$

%\smallskip

\noindent
Д\,о\,к\,а\,з\,а\,т\,е\,л\,ь\,с\,т\,в\,о\  элементарно.

\smallskip

Равномерное расстояние между функциями распределения~$F_{\xi}$ и~$F_{\eta}$ случайных величин~$\xi$ 
и~$\eta$ будем обозначать $\rho(F_{\xi},\,F_{\eta})$: 
$$
\rho(F_{\xi},\,F_{\eta})= \sup\limits_x\left\vert F_{\xi}(x)-F_{\eta}(x)\right\vert\,.
$$

 Нормальную функцию распределения со средним $a\hm\in\mathbb{R}$ и~дисперсией 
$\sigma^2\hm>0$ обозначим~$\Phi_{a,\sigma}$:
\begin{multline*}
\Phi_{a,\sigma}(x)=\fr{1}{\sigma\sqrt{2\pi}}
\int\limits_{-\infty}^{x}\exp\left\{-\fr{(z-a)^2}{2\sigma^2}\right\}dz={}\\
{}=
\Phi_{0,1}\left(\fr{x-a}{\sigma}\right)=
\Phi_{0,\sigma}(x-a)\,,\enskip x\in\mathbb{R}\,.
\end{multline*}

\smallskip

\noindent
\textbf{Лемма~2.}\ \textit{Для любых $a\hm\in\mathbb{R}$, $\sigma\hm>0$, $b\hm\in\mathbb{R}$}
$$
\rho\left(\Phi_{a+b,\sigma},\,\Phi_{a,\sigma}\right)= 
2\Phi_{0,\sigma}\left(\fr{|b|}{2}\right)-1\,.
$$

\smallskip

\noindent
Д\,о\,к\,а\,з\,а\,т\,е\,л\,ь\,с\,т\,в\,о\,.\ \
 Заметим, что если $H(x)$ и~$G(x)$~--- дифференцируемые функции распределения, то $\rho(H,G)$ 
 реализуется (т.\,е.\ точная верхняя грань $\sup_x|H(x)\hm-G(x)|$ по~$x$ достигается) в~одной из точек~$x$, 
 где $F'(x)\hm=G'(x)$. Действительно,
\begin{multline*}
\rho(H,G)=\sup\limits_x|H(x)-G(x)|={}\\
{}=\max\left\{\max\limits_x\left[H(x)-G(x)\right], 
\max\limits_x\left[G(x)-H(x)\right]\right\}
\end{multline*}
и экстремум каждого из выражений в~фигурных скобках достигается в~такой точке, где производная 
соответствующего выражения равна нулю, что равносильно равенству производных функций 
распределения~$H$ и~$G$, т.\,е.\ равенству соответствующих плотностей. 
В~рассматриваемом случае нормальных плотностей последнее условие эквивалентно тому, что
\begin{multline*}
\fr{1}{\sigma\sqrt{2\pi}}\exp\left\{-\fr{1}{2}\left(\fr{x-a-b}{\sigma}\right)^2\right\}={}\\
{}=
\fr{1}{\sigma\sqrt{2\pi}}\exp\left\{-\fr{1}{2}\left(\fr{x-a}{\sigma}\right)^2\right\},
\end{multline*}
или $\big(x-(a+b)\big)^2\hm=(x-a)^2$. Решая данное уравнение, получаем $x\hm-a\hm={b}/{2}$, 
откуда с~учетом соотношения $\Phi_{0,\sigma}(-|b|)\hm=1-\Phi_{0,\sigma}(|b|)$ вытекает требуемое утверждение.

\smallskip

Используя формулу Лагранжа, из леммы~2 легко получить известное неравенство:
$$
\rho\left(\Phi_{a+b,\sigma},\Phi_{a,\sigma}\right)\le \fr{|b|}{\sigma\sqrt{2\pi}}
$$
(см., например, неравенство~(3.4) в~книге~[4]).

\smallskip

\noindent
\textbf{Лемма~3.}\ \textit{Пусть $n\hm\in\mathbb{N}$, $\xi_1,\ldots,\xi_n$~--- случайные величины, 
$a_1,\ldots,a_n$~--- положительные числа, такие что
$a_1+\cdots+a_n\hm=1$. Тогда для любого $x\hm>0$
$$
{\sf P}\left(\left|\sum\limits_{j=1}^n\xi_j\right|\ge x\right)\le\sum\limits_{j=1}^n{\sf P}\left(|\xi_j|\ge a_jx\right).
$$
Если дополнительно случайные величины $\xi_1,\ldots,\xi_n$ одинаково распределены, то}
$$
{\sf P}\left(\left|\sum\limits_{j=1}^n\xi_j\right|\ge x\right)\le n{\sf P}\left(\left\vert \xi_1\right\vert \ge 
\fr{x}{n}\right).
$$


\smallskip

\noindent
Д\,о\,к\,а\,з\,а\,т\,е\,л\,ь\,с\,т\,в\,о\,.\ \ Заметим, что
$$
{\sf P}\left(\left|\sum\limits_{j=1}^n\xi_j\right|\ge x\right)\le{\sf P}\left(\sum\limits_{j=1}^n|\xi_j|
\ge x\right).
$$
Из геометрических соображений вытекает, что
\begin{multline*}
\left\{\omega:\, \sum\limits_{j=1}^n|\xi_j(\omega)|\ge x\right\}\subseteq
\left\{\omega:\,\left\vert \xi_1(\omega)\right\vert \ge a_1x\right\}\bigcup{}\\
{}\bigcup
\left\{\omega:\, \sum\limits_{j=2}^n \left\vert \xi_j(\omega)\right\vert \ge 
\left(1-a_1\right)x\right\}\subseteq
{}\\
{}
\subseteq\left\{\omega:\,\left\vert \xi_1(\omega)\right\vert \ge a_1x\right\}\bigcup
\left\{\omega:\,\left\vert \xi_2(\omega)\right\vert \ge a_2x\right\}\bigcup{}\\
{}\bigcup
\left\{\omega:\, \sum\limits_{j=3}^n\left\vert \xi_j(\omega)\right\vert
\ge \left(1-a_1-a_2\right)x\right\}\subseteq\cdots
\\
\cdots\subseteq\bigcup\limits_{j=1}^n\left\{\omega:\,\left\vert \xi_j(\omega)\right\vert
\ge a_jx\right\}.
\end{multline*}
Поэтому
\begin{multline*}
{\sf P}\left(\left\vert \sum\nolimits_{j=1}^n\xi_j\right\vert \ge x\right)
\le{\sf P}\left(\sum\limits_{j=1}^n\left\vert \xi_j\right\vert \ge x\right)\le
{}\\
{}
\le{\sf P}\!\!\left(\bigcup\limits_{j=1}^n\left\{\omega:\,\left\vert \xi_j(\omega)\right\vert
\ge a_jx\right\}\!\right)\!
\le\!\sum\limits_{j=1}^n{\sf P}\left(\left\vert \xi_j\right\vert \ge a_jx\right).\hspace*{-0.93279pt}
\end{multline*}
Лемма доказана.


\section{Основные результаты}


Рассмотрим оценку равномерного расстояния между распределением суммы $S_n\hm=S_n^{(\le u)}
\hm+S_n^{(>u)}$ и~нормальным законом с~соответствующими математическим ожиданием 
$a\hm\in\mathbb{R}$ и~дисперсией $\sigma\hm>0$, конкретный выбор которых прокомментируем \mbox{ниже}.

\smallskip

\noindent
\textbf{Теорема~1.} \textit{ Пусть $\epsilon\hm>0$, $u\hm>0$~--- произвольны. 
Тогда для любых $a\hm\in\mathbb{R}$, $\sigma\hm>0$}
\begin{multline}
\rho\left(F_{S_n},\Phi_{a,\,\sigma}\right)\le
\rho\left(F_{S_n^{(\le u)}},\Phi_{a,\,\sigma}\right)+{}\\
{}+
\sum\limits_{j=1}^n\left[F_j(-u)+1-F_j(u)\right].
\label{e1-dor}
\end{multline}

\smallskip

\noindent
Д\,о\,к\,а\,з\,а\,т\,е\,л\,ь\,с\,т\,в\,о\,.\ \
 Пусть $\epsilon\hm>0$ произвольно. На основании леммы~1 имеем:
\begin{multline}
{\sf P}\left(S_n<x\right)={\sf P}\left(S_n<x;\,\left\vert S_n^{(>u)}\right\vert
\le\epsilon\right)+{}\\
{}+{\sf P}\left(S_n<x;\,\left\vert S_n^{(>u)}\right\vert >\epsilon\right)\ge{}
\\
{}\ge{\sf P}\left(S_n^{(\le u)}<x-S_n^{(>u)};\,|S_n^{(>u)}|\le\epsilon\right)\ge{}\\
{}\ge
{\sf P}\left(S_n^{(\le u)}<x-\epsilon;\,\left\vert S_n^{(>u)}\right\vert \le\epsilon\right)\ge
{}\\
{}\ge{\sf P}\left(S_n^{(\le u)}<x-\epsilon\right)-{\sf P}\left(
\left\vert S_n^{(>u)}\right\vert \ge\epsilon\right).
\label{e2-dor}
\end{multline}
С другой стороны, очевидно:
\begin{multline}
{\sf P}\left(S_n<x\right)={\sf P}\!\left(S_n^{(\le u)}<x-S_n^{(>u)};\,\left\vert S_n^{(>u)}\right\vert
\le\epsilon\right)+{}\\
{}+
{\sf P}\left(S_n<x;\,\left\vert S_n^{(>u)}\right\vert >\epsilon\right)\le
{}\\
{}
\le{\sf P}\left(S_n^{(\le u)}<x+\epsilon;\,\left\vert S_n^{(>u)}\right\vert \le\epsilon\right)+{}\\
{}+
{\sf P}(S_n<x;\,\left\vert S_n^{(>u)}\right\vert >\epsilon)\le
{}\\
{}\le{\sf P}\left(S_n^{(\le u)}<x+\epsilon\right)+{\sf P}\left(\left\vert S_n^{(>u)}\right\vert
>\epsilon\right).
\label{e3-dor}
\end{multline}
Легко видеть, что
\begin{multline}
\hspace*{-9.96pt}\left\vert {\sf P}\left(S_n<x\right)-\Phi_{a,\sigma}(x)\right\vert\!=\!
\max\left\{{\sf P}\left(S_n<x\right)-\Phi_{a,\sigma}(x),\right.\\
\left.\Phi_{a,\sigma}(x)-{\sf P}\left(S_n<x\right)\right\}.
\label{e4-dor}
\end{multline}
Используя~(\ref{e3-dor}) и~лемму~2, получим:
\begin{multline}
{\sf P}\left(S_n<x\right)-\Phi_{a,\sigma}(x)\le 
{\sf P}\left(\left\vert S_n^{(>u)}\right\vert>\epsilon\right)+{}\\
{}+
\left[{\sf P}\left(S_n^{(\le u)}<x+\epsilon\right)-\Phi_{a,\sigma}(x+\epsilon)\right]+{}
\\
{}+\left[\Phi_{a,\sigma}(x+\epsilon)-\Phi_{a,\sigma}(x)\right]\le
{\sf P}\left(\left\vert S_n^{(>u)}\right\vert >\epsilon\right)+{}\\
{}+
\rho\left(F_{S_n^{(\le u)}},\Phi_{a,\,\sigma}\right)+
\left[2\Phi_{0,\sigma}\left({\fr{\epsilon}{2}}\right)-1\right].
\label{e5-dor}
\end{multline}
Используя~(\ref{e2-dor}) и~лемму~2, получим
\begin{multline}
\Phi_{a,\sigma}(x)-{\sf P}\left(S_n<x\right)\le 
\Phi_{a,\sigma}(x)-{}\\
{}-{\sf P}\left(S_n^{(\le u)}<x-\epsilon\right)+{\sf P}\left(\left\vert S_n^{(>u)}\right\vert
>\epsilon\right)={}
\\
{}=\left[\Phi_{a,\sigma}(x)-\Phi_{a,\sigma}(x-\epsilon)\right]-
\left[{\sf P}\left(S_n^{(\le u)}<x-\epsilon\right)-{}\right.\\
\left.{}-\Phi_{a,\sigma}(x-\epsilon)
\vphantom{\left(S_n^{(\le u)}<x-\epsilon\right)}
\right]+ 
{\sf P}\left(\left\vert S_n^{(>u)}\right\vert>\epsilon\right) \le
{}\\
{}\le\left[2\Phi_{0,\sigma}\left({\fr{\epsilon}{2}}\right)-1\right]+
\rho(F_{S_n^{(\le u)}},\Phi_{a,\sigma})+{}\\
{}+{\sf P}\left(\left\vert S_n^{(>u)}\right\vert>\epsilon\right).
\label{e6-dor}
\end{multline}
Подставив оценки~(\ref{e5-dor}) и~(\ref{e6-dor}) в~(\ref{e4-dor}), получим
\begin{multline}
\rho(F_{S_n},\Phi_{a,\sigma})\le\rho\left(F_{S_n^{(\le u)}},\Phi_{a,\sigma}\right)+{}\\
{}+
\left[2\Phi_{0,\sigma}\left({\fr{\epsilon}{2}}\right)-1\right]+
{\sf P}\left(\left\vert S_n^{(>u)}\right\vert >\epsilon\right).
\label{e7-dor}
\end{multline}
Рассмотрим третье слагаемое в~правой части~(\ref{e7-dor}). На основании леммы~3 
по формуле полной ве\-ро\-ят\-ности, принимая во внимание тот факт, что 
${\epsilon}/{n}\hm>0$ и~$|\xi_j(\omega)|\mathbb{I}(|\xi_j(\omega)|>u)\hm=0$ для тех~$\omega$, 
для которых $|\xi_j(\omega)|\hm\le u$, $j\hm=1,\ldots,n$, имеем:
\begin{multline}
{\sf P}\left(\left\vert S_n^{(>u)}\right\vert > \epsilon\right)=
{\sf P}\left(\left\vert \sum\limits_{j=1}^n\xi_j\mathbb{I}
\left(\left\vert \xi_j\right\vert >u\right)\right\vert>\epsilon\right)\le{}\\
{}\le
\sum\limits_{j=1}^n{\sf P}\left(
\left\vert \xi_j\right\vert \mathbb{I}\left(\left\vert \xi_j\right\vert >u\right)>
{\fr{\epsilon}{n}}\right)=
{}\\
{}=\!\sum\limits_{j=1}^n\!\left[ {\sf P}\left(\!|\xi_j|\mathbb{I}(|\xi_j|>u)>
{\fr{\epsilon}{n}}\!\left\vert\!
\vphantom{\fr{\epsilon}{n}} \right.
|\xi_j|>u\!\right)\!{\sf P}(|\xi_j|>u)+{}\right.
\\
\left.{}+{\sf P}\!\left(\left\vert \xi_j\right\vert
\mathbb{I}\left(\left\vert \xi_j\right\vert >u\right)>{\fr{\epsilon}{n}}\!\left\vert\!
\vphantom{\fr{\epsilon}{n}}\right.\,\left\vert \xi_j\right\vert 
\le u\right)
{\sf P}\left(\left\vert \xi_j\right\vert \le u\right)\right]={}
\\
{}=\sum\limits_{j=1}^n\left[1-p_j(u)\right]
{\sf P}\!\left(\!|\xi_j|\mathbb{I}\left(\left\vert \xi_j\right\vert>u\right)>
\left.{\fr{\epsilon}{n}}
\right\vert
\left\vert \xi_j\right\vert >u\right)\le{}\\
{}\le
\sum\limits_{j=1}^n\left[F_j(-u)+1-F_j(u)\right].
\label{e8-dor}
\end{multline}
Подставив~(\ref{e8-dor}) в~(\ref{e7-dor}) и~устремив~$\epsilon$ к~нулю, получим требуемое. Теорема доказана.

\smallskip


На практике в~качестве параметров~$a$ и~$\sigma$ можно брать, например,
\begin{multline*}
a=a(u)={\sf E}S_n^{(\le u)}={\sf E}\sum\limits_{j=1}^n \xi_j\mathbb{I}\left(\left\vert \xi_j\right\vert 
\le u\right)={}\\
{}= \sum\limits_{j=1}^n {\sf E}\left[\xi_j\mathbb{I}\left(\left\vert \xi_j\right\vert
\le u\right)\right]=\sum\limits_{j=1}^np_j(u)\int\limits_{-u}^{u}x\,dF_j(x);
\end{multline*}

%\vspace*{-12pt}

\noindent
\begin{multline*}
\sigma^2=\sigma^2(u)={\sf D}S_n^{(\le u)}={\sf D}\sum\limits_{j=1}^n 
\xi_j\mathbb{I}\left(\left\vert \xi_j\right\vert \le u\right)={}\\
{}=
\sum\limits_{j=1}^n {\sf D}\left[\xi_j\mathbb{I}\left(\left\vert \xi_j\right\vert \le u\right)\right]=
{}\\
{}=\sum\limits_{j=1}^n\left[p_j(u)\!\int\limits_{-u}^{u}\!x^2\,dF_j(x)-
\left(\! p_j(u)\!\int\limits_{-u}^{u}\!x\,dF_j(x)\!
\right)^{\!\!2\,}\!\right].
\end{multline*}

При фиксированном $u\hm>0$ первое слагаемое в~правой части~(\ref{e1-dor}) 
убывает с~увеличением~$n$, тогда как второе возрастает. При этом существует $n_0\hm\ge1$ такое, что при
 $1\hm\le n\hm\le n_0$ вся правая часть~(\ref{e1-dor}) убывает, а~при $n\hm\ge n_0$ воз\-рас\-та\-ет. 
 В~случае одинаково распределенных слагаемых второе слагаемое воз\-рас\-та\-ет как~$kn$, где $k\hm>0$. 
 При этом за счет выбора очень большого~$u$ можно добиться произвольной малости коэффициента~$k$ 
 и,~как следствие, очень медленного роста второго слагаемого. Поэтому параметр~$n_0$ может 
 принимать довольно большие значения.

Задачу определения указанного~$n_0$ конкретизируем для частного случая. Предположим, что
\begin{equation}
\rho\left(F_{S_n^{(\le u)}},\Phi_{a,\sigma}\right)\le C n^{-\gamma}, \label{e9-dor}
\end{equation}
где $\gamma>0$, а коэффициент $C\hm=C(u)$ определяется свойствами распределений слагаемых 
в~сумме $S_n^{(\le u)}$. Некоторые критерии справедливости~(\ref{e9-dor}) 
приведены, например, в~[5]. Предположим также, что
\begin{equation}
\sum\limits_{j=1}^n\left[F_j(-u)+1-F_j(u)\right]\le kn\,,
\label{e10-dor}\end{equation}
где $k=k(u)\hm>0$. Несложно убедиться, что минимум функции
$$
g(x)=\fr{C}{x^{\gamma}}+kx
$$
достигается в~точке
$$
x_0=\left(\fr{C\gamma}{k}\right)^{{1}/({1+\gamma})},
$$
причем
\begin{equation}
\min\limits_{x\ge0}g(x)=g(x_0)=
(\gamma+1)\left(\fr{Ck^{\gamma}}{\gamma^{\gamma}}\right)^{{1}/({\gamma+1})}.
\label{e11-dor}
\end{equation}
Таким образом, в~качестве~$n_0$ выступает либо $[x_0]$, либо $[x_0]+1$. 
При этом правую часть~(\ref{e11-dor}) можно рассматривать как приближенное значение наилучшей 
верхней оценки точности нормальной аппроксимации при справедливости условий~(\ref{e9-dor}) и~(\ref{e10-dor}).

{\small\frenchspacing
{%\baselineskip=10.8pt
%\addcontentsline{toc}{section}{References}
\begin{thebibliography}{9}
\bibitem{Shevtsova2016}
\Au{Шевцова И.\,Г.} Точность нормальной аппроксимации: 
методы оценивания и~новые результаты.~--- М.: Аргамак-Медиа, 2016. 380~с.

\bibitem{Mackevicius1983}
\Au{Мацкявичюс В.\,К.} О~нижней оценке скорости сходимости в~центральной предельной теореме~// 
Теория вероятностей и~ее применения,
1983. Т.~28. Вып.~3. С.~565--569.

\bibitem{KorolevDorofeeva2017}
\Au{Korolev V.\,Yu., Dorofeeva~A.\,V.} Bounds of the accuracy of 
the normal approximation to the distributions of random sums under relaxed moment conditions~// 
Lith. Math.~J., 2017. Vol.~57. No.\,1. P.~38--58.

\bibitem{Petrov1972}
\Au{Петров В.\,В.} Суммы независимых случайных величин.~--- М.: Наука, 1972. 414~с.

%\bibitem{Klebanov1999}
%{\it L. B. Klebanov, S. T. Rachev, G. J. Szekely.} Pre-limit theorems and their applications // Acta Applicandae Mathematicae, 1999. Vol 58. P. %159--174.

\bibitem{Ibragimov1966}
\Au{Ибрагимов И.\,А.} О~точ\-ности аппроксимации функций распределения сумм 
независимых случайных величин нормальным распределением~// Теория вероятностей и~ее применения, 1966. 
Т.~11. Вып.~4. С.~632--655.
\end{thebibliography}

}
}

\end{multicols}

\vspace*{-3pt}

\hfill{\small\textit{Поступила в~редакцию 13.10.2020}}

\vspace*{8pt}

%\pagebreak

%\newpage

%\vspace*{-28pt}

\hrule

\vspace*{2pt}

\hrule

%\vspace*{-2pt}

\def\tit{ON THE ACCURACY OF~THE~NORMAL APPROXIMATION UNDER~THE~VIOLATION OF~THE~NORMAL CONVERGENCE}

\def\titkol{On the accuracy of~the~normal approximation under~the~violation of~the~normal convergence}

\def\aut{V.\,Yu.~Korolev$^{1,2}$ and~A.\,V.~Dorofeeva$^1$}

\def\autkol{V.\,Yu.~Korolev and~A.\,V.~Dorofeeva}

\titel{\tit}{\aut}{\autkol}{\titkol}

\vspace*{-11pt}


\noindent
$^1$Faculty of Computational Mathematics and Cybernetics, 
M.\,V.~Lomonosov Moscow State University, GSP-1,\linebreak
$\hphantom{^1}$Leninskie Gory, Moscow 119991, Russian Federation

\noindent
$^2$Institute of Informatics Problems, Federal Research Center ``Computer Sciences and Control'' 
of the Russian\linebreak
$\hphantom{^1}$Academy of Sciences; 44-2~Vavilov Str., Moscow 119133, Russian Federation

\def\leftfootline{\small{\textbf{\thepage}
\hfill INFORMATIKA I EE PRIMENENIYA~--- INFORMATICS AND
APPLICATIONS\ \ \ 2021\ \ \ volume~15\ \ \ issue\ 1}
}%
\def\rightfootline{\small{INFORMATIKA I EE PRIMENENIYA~---
INFORMATICS AND APPLICATIONS\ \ \ 2021\ \ \ volume~15\ \ \ issue\ 1
\hfill \textbf{\thepage}}}

\vspace*{6pt}





\Abste{When solving applied problems in various fields, it is conventional 
to use the normal approximation to the distribution of data with additive structure. 
As a~criterion of the adequacy of such a~model, it is possible to use bounds for the convergence 
rate in the central limit theorem (CLT) of the probability theory stating that under certain 
conditions (say, under the Lindeberg condition), the total effect of very many random factors acts as 
a~random variable with the normal distribution. The classical bounds for the convergence 
rate in the CLT such as the Berry--Esseen inequality are proved under the condition 
that the third moments of the summands exist. Also, bounds are known that require the
 existence of the moments of orders $2+\delta$ with $0< \delta <1$. 
 If only the moments of the second order exist, then the convergence in the CLT can 
 be arbitrarily slow. But if the moments of the summands of the second order do 
 not exist, then the convergence of the distributions of sums of independent 
 random variables to the normal law does not take place. It is practically impossible 
 to reliably check the conditions of the central limit theorem with the limited size 
 of the available sample. Therefore, the question of what is the real accuracy 
 of the normal approximation if it is theoretically impossible is of great interest. 
 Moreover, in some situations, in computer simulation of sums of 
 random variables whose distributions belong to the domain of attraction of the stable 
 distribution with the characteristic exponent less than two, as the number of summands grows, 
 first, the distance between the distribution of the normalized sum and the normal law 
 decreases and starts to increase only when the number of summands becomes sufficiently large. 
 In this paper, an attempt is undertaken to give some theoretical explanation of this 
 effect and to give an answer to the question posed above.}

       
\KWE{central limit theorem; accuracy of normal approximation; heavy tails; uniform distance}



\DOI{10.14357/19922264210116}

%\vspace*{-15pt}

\Ack
\noindent
Research supported by the Russian Foundation for Basic Research, project 18-07-01405.

\vspace*{4pt}

  \begin{multicols}{2}

\renewcommand{\bibname}{\protect\rmfamily References}
%\renewcommand{\bibname}{\large\protect\rm References}

{\small\frenchspacing
 {%\baselineskip=10.8pt
 \addcontentsline{toc}{section}{References}
 \begin{thebibliography}{9}
\bibitem{1-dor}
\Aue{Shevtsova, I.\,G.} 2016. \textit{Tochnost' normal'noy approksimatsii: metody otsenivaniya 
i~novye resul'taty} [The accuracy of the normal approximation: 
Methods of estimation and new results]. Moscow: Argamak-Media. 380~p.
{\looseness=1

}

\columnbreak


\bibitem{2-dor}
\Aue{Mackevicius, V.\,K.} 1984. A~lower bound for the convergence rate in the central limit theorem. 
\textit{Theor. Probab. Appl.} 28(3):596--601.

\vspace*{-2pt}

\bibitem{3-dor}
\Aue{Korolev, V.\,Yu., and A.\,V.~Dorofeeva.}
 2017. Bounds of the accuracy of the normal approximation to the distributions of random 
 sums under relaxed moment conditions. \textit{Lith. Math.~J.} 57(1):38--58.
\bibitem{4-dor}
\Aue{Petrov, V.\,V.} 1972. \textit{Summy nezavisimykh sluchaynykh velichin} 
[Sums of independent random variables]. Moscow: Nauka. 416~p.
\bibitem{5-dor}
\Aue{Ibragimov, I.\,A.} 1966. On the accuracy of Gaussian approximation to the distribution functions 
of sums of independent variables. \textit{Theor. Probab. Appl.} 11(4):559--579.
 \end{thebibliography}

 }
 }

\end{multicols}

\vspace*{-3pt}

  \hfill{\small\textit{Received October~13, 2020}}


%\pagebreak

%\vspace*{-8pt}     

\Contr

\noindent
\textbf{Korolev Victor Yu.} (b.\ 1954)~--- 
Doctor of Science in physics and mathematics, professor, Head of Department, Faculty of 
Computational Mathematics and Cybernetics, M.\,V.~Lomonosov Moscow State University, 
GSP-1, Leninskie Gory, Moscow 119991, Russian Federation; leading scientist, 
Federal Research Center ``Computer Science and Control'' 
of the Russian Academy of Sciences, 44-2~Vavilov Str.,Moscow 119333, Russian Federation; 
\mbox{vkorolev@cs.msu.ru}

\vspace*{6pt}

\noindent
\textbf{Dorofeeva Alexandra V.} (b.\ 1991)~--- 
graduate PhD student, Faculty of Computational Mathematics and Cybernetics, 
M.\,V.~Lomonosov Moscow State University, GSP-1, Leninskie Gory, Moscow 119991, 
Russian Federation; \mbox{alex.dorofeyeva@gmail.com}

\label{end\stat}

\renewcommand{\bibname}{\protect\rm Литература}