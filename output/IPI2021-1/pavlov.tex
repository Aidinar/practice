%\def\M{\mathop{\kern\z@\mbox{\bfseries\sffamily\upshape E}}\nolimits}
%\def\P{\mathop{\kern\z@\mbox{\bfseries\sffamily\upshape P}}\nolimits}

\def\stat{pavlov}

\def\tit{СВЯЗНОСТЬ КОНФИГУРАЦИОННЫХ ГРАФОВ\\ В~МОДЕЛЯХ СЛОЖНЫХ СЕТЕЙ$^*$}

\def\titkol{Связность конфигурационных графов в~моделях сложных сетей}

\def\aut{Ю.\,Л.~Павлов$^1$}

\def\autkol{Ю.\,Л.~Павлов}

\titel{\tit}{\aut}{\autkol}{\titkol}

\index{Павлов Ю.\,Л.}
\index{Pavlov Yu.\,L.}

{\renewcommand{\thefootnote}{\fnsymbol{footnote}} \footnotetext[1]
{Финансовое обеспечение исследований осуществлялось из средств федерального бюджета на 
выполнение государственного задания Карельского научного центра Российской академии наук 
(Институт прикладных математических исследований КарНЦ РАН).}}

\renewcommand{\thefootnote}{\arabic{footnote}}
\footnotetext[1]{Институт прикладных математических исследований КарНЦ РАН, 
ФИЦ <<Карельский научный центр РАН>>, \mbox{pavlov@krc.karelia.ru}}


%\vspace*{-12pt}





\Abst{Рассматриваются конфигурационные графы, степени вершин которых являются независимыми 
случайными величинами, одинаково распределенными по обобщенному дискретному степенному закону. 
Связи между вершинами формируются равновероятно в~соответствии со степенями вершин. Эти 
случайные графы часто используются для моделирования сложных сетей коммуникаций, таких как 
интернет и~социальные сети. В статье предполагается, что распределение степеней вершин неизвестно, 
поскольку зависит от медленно меняющейся функции с~неизвестными свойствами. При стремлении числа 
вершин к~бесконечности найдены условия, при выполнении которых граф становится асимптотически 
достоверно связным. При этих условиях получены оценки скорости сходимости к~нулю вероятности 
того, что граф не связен. Для доказательства результатов статьи использовались свойства устойчивых 
распределений и~медленно меняющихся функций.}


\KW{случайные графы; конфигурационные графы; случайные степени вершин; связность графа}

\DOI{10.14357/19922264210103}


\vskip 10pt plus 9pt minus 6pt

\thispagestyle{headings}

\begin{multicols}{2}

\label{st\stat}

\section{Введение}
\label{SC:1}

Случайные графы широко используются при моделировании сложных сетей коммуникаций, таких как 
интернет, транспортные, телефонные, социальные сети и~т.\,д.~\cite{Hof}. 
Такие модели обычно соответствуют известным свойствам реальных сетей, обнаруженным в~ходе 
многочисленных эмпирических исследований и~подробно описанным в~различных публикациях 
(см., например,~\cite{Fal}). Наблюдения показали, что в~графах, описывающих топологию сетей, 
степени вершин можно считать независимыми одинаково распределенными случайными величинами. 
Одно из важнейших свойств, присущих сетям коммуникаций различной природы, состоит в~том, 
что число вершин степени, не меньшей чем~$k,$ при больших~$k$ пропорционально~$k^{-\tau},$ при 
этом значения положительного параметра~$\tau$ могут отличаться для разных сетей. 
Поэтому во многих работах предполагается, что распределение случайной величины~$\xi,$ равной 
степени любой вершины графа, можно задать следующим образом:
\begin{equation}
\label{EQ:1}
\mathsf{P}\{\xi \geqslant k\} = \fr{h(k)}{k^\tau}\,,\enskip k=1,2,\ldots,
\end{equation}
где $h(k)$~--- медленно меняющаяся функция.

В настоящее время одной из наиболее популярных моделей сложных сетей коммуникаций 
служит так называемый конфигурационный граф, определение которого было дано в~статье~\cite{Bol}. 

Обозначим через~$N$ число вершин графа. Степени вершин задаются~$N$~независимыми реализациями 
случайной величины~$\xi$, и~они равны числу инцидентных каждой вершине полуребер, т.\,е.\ 
ребер, для которых смежные вершины еще не определены. Все полуребра различимы (занумерованы).
 Граф строится путем попарного равновероятного соединения полуребер друг с~другом для образования ребер. 
 Разумеется, сумма степеней всех вершин любого графа должна быть четной, в~случае нечетной суммы в~граф 
 вводится вспомогательная вершина единичной степени. В~статье~\cite{RN1} было замечено, что эта 
 вспомогательная вершина не влияет на основные асимптотические свойства графа при $N\hm\rightarrow \infty.$ 
 Поэтому ниже не будем учитывать эту дополнительную вершину, что фактически соответствует 
 предположению о~чет\-ности суммы степеней. Если же это предположение неверно, то, как легко 
 убедиться, следуя доказательству полученных в~данной \mbox{статье} результатов, дополнительная вершина 
 не влияет на эти результаты. Заметим еще, что описанная конструкция конфигурационных графов 
 допускает появление петель и~кратных ребер.

Асимптотические свойства случайных конфигурационных графов при $N\hm\rightarrow \infty$ изучались 
многими авторами, наиболее полные обзоры таких работ можно найти в~\cite{Hof,Dur}. Известно, что 
если $\tau \hm\in (1, 2),$ то асимптотически достоверно, т.\,е.\ с~ве\-ро\-ят\-ностью, стремящейся к~единице, граф содержит единственную гигантскую компоненту связности, число вершин в~которой 
пропорционально~$N,$ в~то время как объемы других компонент имеют порядок~$o(N).$

В~статье~\cite{RN1} рассматривался простейший вариант распределения~(\ref{EQ:1}) 
в~предположении, что $h(k)\hm\equiv 1.$ Для таких графов даны оценки объема 
и~диа\-мет\-ра гигантской компоненты связности и~\mbox{подробно} исследована ее структура.

В \cite{PavCh} впервые рассматривались условные конфигурационные графы с~распределением~(\ref{EQ:1}) 
степеней вершин в~случае $h(k)\hm\equiv 1$ при условии, что число ребер известно. 
В~\cite{PavKh} такие графы исследовались уже при условии, что число ребер ограничено сверху. 
Далее, в~\cite{Pav} в~распределении~(\ref{EQ:1}) степеней вершин условных графов медленно меняющаяся 
функция~$h(k)$ уже не предполагалась известной, но задавалась асимптотика вероятностей 
стремящихся к~бесконечности значений степеней вершин:
\begin{equation}
\label{EQ:2}
\mathsf{P}\{\xi = k\} \sim \fr{d}{k^g (\ln k)^\eta}\,,
\end{equation}
где $k\rightarrow \infty$, $g\hm>1$, $\eta \hm\geqslant 0.$ Нетрудно проверить, что 
распределение~(\ref{EQ:1}) удовлетворяет условию~(\ref{EQ:2}), если $d\hm=\tau$, 
$g\hm=\tau \hm+1$ и~$h(k)\hm=(\ln k)^{-\eta}.$
Хотя значения~$h(k)$ в~(\ref{EQ:1}) известны не для всех~$k$, знание асимптотического поведения 
вероятностей~(\ref{EQ:2}) существенно облегчило  доказательство полученных в~\cite{Pav} результатов.

Обозначим $A_N$ событие, состоящее в~том, что граф не связен.
В~теореме~4.15 книги~\cite{Hof2} показано, что если $\mathsf{P}\{\xi \hm= 1\} \hm= \mathsf{P}\{\xi\hm = 2\}
\hm =0,$ то при $N\hm\rightarrow \infty$ асимптотически достоверно граф состоит из 
единственной компоненты связности, содержащей все~$N$~вершин. 
В~\cite{Hof2} установлено также, что в~этом случае 
$\mathsf{P}\{A_N\}\hm=O(1/N).$ В~статье~\cite{Pav2} были найдены условия асимптотически достоверной 
связ\-ности конфигурационных графов, степени вершин которых обладают свойством~(\ref{EQ:2}), в~том числе 
и~в~случаях $\mathsf{P}\{\xi \hm= 1\}\hm>0$, $\mathsf{P}\{\xi \hm= 2\}>0.$

Таким образом, осталась нерешенной задача нахождения условий, при выполнении которых 
асимптотически достоверно связен случайный конфигурационный граф, 
степени вершин которого заданы распределением~(\ref{EQ:1}) с~неизвестной медленно меняющейся функцией~$h(k),$ 
необязательно обеспечивающей выполнение соотношения~(\ref{EQ:2}). Основным результатом настоящей статьи 
является доказанная ниже тео\-ре\-ма, в~которой найдены такие условия и~даны оценки скорости сходимости 
к~нулю вероятности того, что граф не связен. 
Согласно теореме~4.4~\cite{Hof2}, если 
$\mathsf{M} \xi(\xi\hm-1)/\mathsf{M} \xi\hm >1,$ 
то граф асимптотически достоверно содержит больше одной компоненты связности. 
Нетрудно видеть, что это имеет место в~случае $\tau\hm >1,$ поэтому в~теореме 
рассматривается только случай $0\hm <\tau \hm\leqslant 1.$

Доказательство теоремы основано на идеях, изложенных в~доказательстве теоремы~4.15~\cite{Hof2}, но 
с~существенными изменениями, связанными с~использованием, как и~в~\cite{Pav2}, 
методов исследования локальной сходимости распределений сумм независимых случайных величин к~устойчивым 
законам~\cite{IL}. Кроме того, впервые при решении подобных задач используются общие свойства 
медленно\linebreak меняющихся функций, изложение со\-от\-вет\-ст\-ву\-ющей теории можно найти, например, в~\cite{Bin}.

В следующем разделе в~виде теоремы сформулирован основной результат статьи, доказательство 
этой теоремы приводится в~разд.~3.


\section{Основной результат}
\label{SC:2}

Пусть конфигурационный граф содержит~$N$~вершин, степени которых независимы и~одинаково 
распределены в~соответствии с~(\ref{EQ:1}), где $h(k)$, $k\hm\geqslant 1,$~--- 
измеримая медленно меняющаяся функция. Предположим для простоты, что максимальный шаг 
распределения случайной величины~$\xi$ равен единице. Введем последовательность~$B_N$, 
$N\hm=1,2,\ldots,$ при $N\hm\rightarrow \infty$ удовлетворяющую условию
\begin{equation}
\label{EQ:3}
B_N\sim \left(Nh\left([B_N]\right)\right)^{1/\tau},
\end{equation}
где $[x]$ означает целую часть числа~$x.$ Отсюда и~из свойств медленно меняющихся функций 
очевидным образом вытекает, что $B_N\hm\rightarrow \infty.$ Примером 
построения такой последовательности для распределения, имеющего свойство~(\ref{EQ:2}), 
могут служить величины вида $B_N\hm=N^{1/\tau},$ если $\eta\hm=0,$ а если $0\hm<\eta \hm<1,$ то
\begin{equation}
\label{EQ:4}
B_N=\left(N\left(\fr{\tau}{\ln N}\right)^\eta\right)^{1/\tau}.
\end{equation}

Теперь можно сформулировать основной результат статьи.

\smallskip

\noindent
\textbf{Теорема.}\ \textit{Пусть $N\hm\rightarrow \infty$ и~$\mathsf{P}\{\xi \hm= 2\}\hm>0.$ 
Тогда с~вероятностью, сколь угодно близкой к~единице,
справедливы следующие утверждения}.
\begin{enumerate}
\item \textit{Если $\mathsf{P}\{\xi = 1\}\hm=0$ и~$\tau\hm =1,$ то $\mathsf{P}\{A_N\}\hm=O(1/\ln N)$}.
\item \textit{Если $\mathsf{P}\{\xi = 1\}\hm=0$ и~$0<\tau \hm<1,$ то $\mathsf{P}\{A_N\}\hm=O(N/B_N)$}.
\item \textit{Если $\mathsf{P}\{\xi = 1\}\hm>0$ и~$0<\tau\hm <1/2,$ то $\mathsf{P}\{A_N\}\hm=O(N^2/B_N)$}.
\end{enumerate}

\section{Доказательство теоремы} 
\label{SC:3}

Пусть произошло событие~$A_N.$ Тогда множество~$V$ всех вершин графа можно представить в~виде 
$V\hm=V_1\bigcup V_2,$ где $V_1$ и~$V_2$~--- 
непересекающиеся множества вершин такие, что не существует ребер, соединяющих вершины из~$V_1$ 
с~вершинами из~$V_2.$ Можем считать, что $|V_1| \hm \leqslant |V_2|,$ где $|V_1|$ и~$|V_2|$~--- 
мощности множеств~$V_1$ и~$V_2$ соответственно. Тогда $|V_1| \hm \leqslant N/2.$ 
Обозначим~$\Omega$ множество всех возможных разбиений~$V$ на~$V_1$ и~$V_2.$ 
Пусть случайные величины $\xi_1,\ldots, \xi_N$ равны степеням вершин $1,\ldots, N$ 
соответственно. Понятно, что распределения этих независимых случайных величин совпадают с~(\ref{EQ:1}). 
Представляет интерес предельное поведение их суммы
\begin{equation}
\label{EQ:5}
\zeta_N=\xi_1+\cdots +\xi_N.
\end{equation}
Обозначим также
\begin{equation}
\label{EQ:6}
\zeta_N^{(1)}=\sum\limits_{i\in V_1} \xi_i, \quad  \zeta_N^{(2)}=\sum\limits_{i\in V_2} \xi_i.
\end{equation}

Ясно, что общее число различных графов с~суммой степеней вершин~(\ref{EQ:5}) равно 
$(\zeta_N\hm-1)!!$ Учитывая равновероятность соединения полуребер при образовании ребер, находим, что
$$
\mathsf{P}\{A_N\}\leqslant \sum\limits_{V_1,V_2\in \Omega} 
\fr{(\zeta_N^{(1)}-1)!!(\zeta_N^{(2)}-1)!!}{(\zeta_N-1)!!}\,.
$$
Отсюда следует оценка вероятности того, что граф не связен:
\begin{equation*}
%\label{EQ:7}
\mathsf{P}\{A_N\} \leqslant \sum\limits_{V_1\in \Omega} \prod\limits_{j=1}^{\zeta_N^{(1)}/2} 
\fr{\zeta_N^{(1)}-2j+1}{\zeta_N-2j+1}\,.
\end{equation*}
Верхний предел произведения в~этом выражении равен~$\zeta_N^{(1)}/2,$ 
поскольку, очевидно, $\zeta_N^{(1)},$ как и~$\zeta_N^{(2)},$ принимает четные значения.

Пусть $N\rightarrow \infty.$ Обозначим~$F(x)$ функцию распределения случайной величины~$\xi.$ 
Из~(\ref{EQ:1}) следует, что
\begin{equation}
\label{EQ:8}
F(x)=
\begin{cases}
0\,, & \!\!\!\mbox{если} x<0;\\
1-\fr{h(x)}{x^\tau}\left(1 +o(1)\right)& \!\!\!\mbox{при } x\rightarrow \infty.
\end{cases}
\end{equation}
В силу (\ref{EQ:8}) функция~$F(x)$ удовлетворяет условиям теоремы~2.6.1 книги~\cite{IL}, 
следовательно, она\linebreak принадлежит области притяжения устойчивого\linebreak закона~$G(x)$ с~показателем~$\tau.$ 
Применив тео\-ре\-му~2.2.2~\cite{IL}, находим, что если $0\hm<\tau \hm<1,$ то логарифм 
характеристической функции~$\varphi_G(t)$ устойчивого закона~$G(x)$ имеет вид:
\begin{equation}
\label{EQ:10}
\ln \varphi_G(t)=i\gamma t-c|t|^\tau \left(1+i\beta \fr{t}{|t|}\tan \frac{\pi \tau}{2}\right),
\end{equation}
где $\gamma$~--- некоторая постоянная, 
а~значения па\-ра\-мет\-ров~$\beta$ и~$c$ с~по\-мощью~(\ref{EQ:8}) 
%и~(\ref{EQ:9}) 
определены в~ходе доказательства теоремы~2.2.2~\cite{IL}: $\beta \hm=-1;$
\begin{equation}
\label{EQ:11}
c=\Gamma (1-\tau)\cos \fr{\pi \tau}{2}\,,
\end{equation}
где $\Gamma (x)$~--- гам\-ма-функ\-ция.

Вид логарифма характеристической функции~$\varphi_G(t)$ при $\tau \hm=1$ определяется аналогичным образом:
\begin{equation}
\label{EQ:12}
\ln \varphi_G(t)=i\gamma t-\fr{\pi}{2}\left\vert t\right\vert 
\left(1+i\fr{t}{|t|}\,\fr{2}{\pi}\,\ln |t|\right).
\end{equation}
В выражениях~(\ref{EQ:10}) и~(\ref{EQ:12}) константу~$\gamma$ можно сделать равной нулю,
 подобрав нужным образом нормирующие множители в~определении области притяжения закона~$G(x)$~\cite{IL}. 
 Обозначим~$\varphi (t)$ характеристическую функцию случайной величины~$\xi.$ 
 Из~(\ref{EQ:10}), (\ref{EQ:12}) и~теоремы~2.6.5~\cite{IL} следует, что в~окрестности нуля
\begin{equation}
\label{EQ:13}
\ln \varphi(t)=-c|t|^\tau l(t)\left(1+i\beta\fr{t}{|t|}\,\omega (t,\tau)\right),
\end{equation}
где $\beta \hm=-1,$ а~$c$ определено в~(\ref{EQ:11}) в~случае $0\hm<\tau \hm<1,$ 
а~если $\tau \hm=1,$ то $\beta \hm=1$, $c\hm=\pi/2,$ функция~$l(t)$ является медленно меняющейся 
и
\begin{equation}
\label{14}
\omega (t,\tau)=\begin{cases}
\tan \fr{\pi \tau}{2},& 0<\tau<1\,;\\
\fr{2}{\pi}\ln |t|,& \tau =1\,.
\end{cases}
 \end{equation}
В доказательстве теоремы~2.6.8~[11, с.~104--108] 
показано, что для распределения~(\ref{EQ:1}) при $t\hm\rightarrow 0$
\begin{equation}
\label{EQ:15}
l(t)=h\left(\fr{1}{|t|}\right)\left(1+o(1)\right)\,.
\end{equation}

Очевидно, что $\varphi (0)\hm=1.$ Рассмотрим $\varphi (t)$ при фиксированных $t\hm\neq 0.$ 
Пусть $0\hm<\tau \hm<1.$ Используя~(\ref{EQ:3}), (\ref{EQ:11}), (\ref{EQ:13})--(\ref{EQ:15}) 
и~проводя простые вычисления, находим, что если $N\hm\rightarrow \infty,$ то
$$
\varphi^N\left(\fr{t}{B_N}\right) \rightarrow 
\exp \left\{-c|t|^\tau \left(1-i\fr{t}{|t|}\tan \fr{\pi \tau}{2}\right)\right\}.
$$
Это соотношение показывает, что распределения сумм~$\zeta_N$ слабо сходятся к~устойчивому 
закону с~показателем~$\tau.$ Заметим, что, согласно теореме~4.2.1~\cite{IL}, 
на самом деле имеет место и~локальная сходимость. Это значит, что при достаточно больших~$N$ 
и~любом сколь угодно малом $\varepsilon \hm>0$ существует положительная константа~$L$ такая, что
\begin{equation}
\label{EQ:16}
\mathsf{P}\left\{\fr{B_N}{L}\leqslant \zeta_N \leqslant LB_N\right \}>1-\varepsilon\,.
\end{equation}

Пусть $\tau =1.$ С~по\-мощью~(\ref{EQ:3}), (\ref{EQ:13})--(\ref{EQ:15}) и~известных свойств
 медленно меняющихся функций~\cite{Bin} нетрудно вывести, что при $N\hm\rightarrow \infty$ 
 существует стремящаяся к~нулю последовательность~$q(N)$ такая, что при любом фиксированном $t\hm\neq 0$
\begin{multline}
\label{EQ:17}
\varphi^N\left(\fr{t}{B_N}\right) \exp 
\{-itN(\ln N)(1+q(N))\}\rightarrow{}\\
{}\rightarrow \exp 
\left\{-\fr{\pi}{2}\left\vert t\right\vert
\left(1+i\fr{t}{|t|}\,\fr{2}{\pi}\ln |t|\right)\right\}.
\end{multline}
Из~(\ref{EQ:17}) и~локальной предельной теоремы~4.2.1~\cite{IL} 
теперь вытекает, что асимптотически достоверно
\begin{equation}
\label{EQ:18}
\zeta_N\sim N\ln N\,.
\end{equation}

Рассмотрим предельное поведение~$\zeta_N^{(1)}.$ Если объем множества~$V_1$ конечен, то из~(\ref{EQ:1}) 
и~(\ref{EQ:6}) нетрудно получить, что сумма~$\zeta_N^{(1)}$ асимптотически достоверно конечна.
 Пусть $|V_1|\hm\rightarrow \infty.$ Если $0\hm<\tau \hm<1,$ то по аналогии с~(\ref{EQ:3}) 
 и~(\ref{EQ:16}) находим, что при достаточно большом~$|V_1|$ и~достаточно малом $\varepsilon \hm>0$ 
 существует положительная константа~$L_1$ такая, что
\begin{equation}
\label{EQ:19}
\mathsf{P}\left\{\fr{B_N^{(1)}}{L_1}\leqslant \zeta_N^{(1)} \leqslant L_1B_N^{(1)}\right\}>1-\varepsilon\,,
\end{equation}
где
$$
B_N^{(1)}=(|V_1|h([B_N^{(1)}]))^{1/\tau}.
$$
Если же $\tau \hm=1,$ то замечаем, что, подобно~(\ref{EQ:18}),
\begin{equation}
\label{EQ:20}
\zeta_N^{(1)}\sim \left\vert V_1\right\vert \ln \left\vert V_1\right\vert.
\end{equation}

Теперь для того, чтобы получить утверж\-де\-ния~\mbox{1--3}, достаточно повторить доказательство 
тео\-ре\-мы статьи~\cite{Pav2}, в~котором~(\ref{EQ:4}) заменить на~(\ref{EQ:3}) и~для 
оценки предельного поведения сумм~$\zeta_N$ и~$\zeta_N^{(1)}$ использовать соотношения~(\ref{EQ:16}), 
(\ref{EQ:18})--(\ref{EQ:20}).



{\small\frenchspacing
{%\baselineskip=10.8pt
%\addcontentsline{toc}{section}{References}
\begin{thebibliography}{99}

%\vspace*{-2pt}
\bibitem{Hof}
\Au{Hofstad R.} Random graphs and complex networks.~--- Cambridge:
Cambridge University Press, 2017.  Vol.~1. 337~p.

\bibitem{Fal}
\Au{Faloutsos C., Faloutsos~P., Faloutsos~M.} 
On power-law relationships of the Internet topology~// Comput. Commun. Rev., 1999. Vol.~29. P.~251--262.

\bibitem{Bol}
\Au{Bollobas B.\,A.} A~probabilistic proof of an asymptotic formula
for the number of labelled regular graphs~// Eur. J.~Combin., 1980. Vol.~1. P.~311--316.

\bibitem{RN1}
\Au{Reittu H., Norros~I.} On the power-law random graph model of massive
data networks~// Perform. Evaluation, 2004. Vol.~55. P.~3--23.

\bibitem{Dur}
\Au{Durrett R.} Random graph dynamics.~--- Cambridge:
Cambridge University Press, 2007. 212~p.

\bibitem{PavCh}
\Au{Павлов Ю.\,Л., Чеплюкова~И.\,А.} Случайные графы Ин\-тер\-нет-ти\-па 
и~обобщенная схема размещения~// Дискретная математика, 2008. Т.~20. Вып.~3. С.~3--18.

\bibitem{PavKh}
\Au{Павлов Ю.\,Л., Хворостянская~Е.\,В.} 
О~предельных распределениях степеней вершин конфигурационных графов с~ограниченным числом ребер~// 
Математический сборник, 2016. Т.~207. Вып.~3. С.~93--110.

\bibitem{Pav}
\Au{Павлов Ю.\,Л.} Условные конфигурационные графы со случайным параметром степенного распределения степеней~// 
Математический сборник, 2018. Т.~209. Вып.~2. С.~120--137.

\bibitem{Hof2}
\Au{Hofstad R.} Random graphs and complex networks. Vol.~2~// 
Notes RGCNII, November~16, 2020. 341~p. {\sf https://www.win.tue.nl/$\sim$rhofstad/NotesRGCNII.pdf}.


\bibitem{Pav2}
\Au{Павлов Ю.\,Л.} О~связности конфигурационных графов~// Дискретная математика, 2019. Т.~31. Вып.~2. С.~115--123.

\bibitem{IL}
\Au{Ибрагимов И.\,А., Линник~Ю.\,В.} 
Независимые и~стационарно связанные величины.~--- М: Наука, 1965. 524~с.

\bibitem{Bin}
\Au{Bigham N.\,H., Goldie~C.\,M., Teugels~J.\,L.} 
Regular variations. Encyclopedia of mathematics and its applications. ~--- 
Cambridge: Cambridge University Press, 1987. Vol.~27. 513~p.
\end{thebibliography}

}
}

\end{multicols}

\vspace*{-6pt}

\hfill{\small\textit{Поступила в~редакцию 15.04.2020}}

\vspace*{6pt}

%\pagebreak

%\newpage

%\vspace*{-28pt}

\hrule

\vspace*{2pt}

\hrule

\vspace*{-2pt}

\def\tit{CONNECTIVITY OF CONFIGURATION GRAPHS IN~COMPLEX NETWORK MODELS}

\def\titkol{Connectivity of configuration graphs in~complex network models}

\def\aut{Yu.\,L.~Pavlov}

\def\autkol{Yu.\,L.~Pavlov}

\titel{\tit}{\aut}{\autkol}{\titkol}

\vspace*{-11pt}


\noindent
Institute of Applied Mathematical Research of the Karelian Research Centre
 of the Russian Academy of Sciences, 11~Pushkinskaya Str., Petrozavodsk 185910, Russian Federation

\def\leftfootline{\small{\textbf{\thepage}
\hfill INFORMATIKA I EE PRIMENENIYA~--- INFORMATICS AND
APPLICATIONS\ \ \ 2021\ \ \ volume~15\ \ \ issue\ 1}
}%
\def\rightfootline{\small{INFORMATIKA I EE PRIMENENIYA~---
INFORMATICS AND APPLICATIONS\ \ \ 2021\ \ \ volume~15\ \ \ issue\ 1
\hfill \textbf{\thepage}}}

\vspace*{3pt}




\Abste{The author considers configuration graphs whose degrees of vertices are independent 
and identically distributed according to the generalized power-law distribution. 
Connections between vertices are equiprobably\linebreak\vspace*{-12pt}}

\Abstend{formed in compliance with their degrees. 
Such random graphs are often used for modeling complex communication networks like the 
Internet and social networks. It is assumed that the distribution of vertex degrees 
is unknown because it depends on a slowly varying function with unknown properties. 
The conditions are found under which a~graph is asymptotically almost surely connected
 as the number of vertices tends to infinity. Under these conditions, estimates 
 of the convergence rate to zero of the probability that the graph is not connected 
 are obtained. The results in the present paper are proved using the properties of 
 stable distributions and slowly varying functions.}


\KWE{random graphs; configuration graphs; random vertex degrees; graph connectivity}

\DOI{10.14357/19922264210103}

%\vspace*{-15pt}

\Ack
\noindent
The study was carried out under state order to the Karelian Research Centre 
of the Russian Academy of Sciences
(Institute of Applied Mathematical Research KarRC RAS).

\vspace*{12pt}

  \begin{multicols}{2}

\renewcommand{\bibname}{\protect\rmfamily References}
%\renewcommand{\bibname}{\large\protect\rm References}

{\small\frenchspacing
 {%\baselineskip=10.8pt
 \addcontentsline{toc}{section}{References}
 \begin{thebibliography}{99}

\bibitem{1-pav}
\Aue{Hofstad, R.} 2017. 
\textit{Random graphs and complex networks.} Cambridge:
Cambridge University Press.  Vol.~1. 337~p.

\bibitem{2-pav}
\Aue{Faloutsos, C., P.~Faloutsos, and M.~Faloutsos.}
 1999. On power-law relationships of the Internet topology. 
 \textit{Comput. Commun. Rev.} 29:251--262.

\bibitem{3-pav}
\Aue{Bollobas, B.\,A.} 1980. A~probabilistic proof of an asymptotic formula for the number
of labelled regular graphs. \textit{Eur. J.~Combin.} 1:311--316.

\bibitem{4-pav}
\Aue{Reittu, H., and I.~Norros.} 2004. On the power-law random graph model of massive data
networks. \textit{Perform. Evaluation} 55:3--23.

\bibitem{5-pav}
\Aue{Durrett, R.} 2007. \textit{Random graph dynamics.} Cambridge:
Cambridge University Press. 212~p.

\bibitem{6-pav}
\Aue{Pavlov, Yu.\,L., and I.\,A.~Cheplyukova.} 2008. 
Random graphs of Internet type and the generalised allocation scheme.
\textit{Discrete Mathematics Applications} 18(5):447--463.

\bibitem{7-pav}
\Aue{Pavlov, Yu.\,L., and E.\,V.~Khvorostyanskaya.} 
2016. On the limit distributions of the degrees of vertices in configuration graphs with 
a~bounded number of edges. 
\textit{Sb. Math.} 207(3):400--417.

\bibitem{8-pav}
\Aue{Pavlov, Yu.\,L.} 2018. 
Conditional configuration graphs with discrete power-law distribution of vertex degrees. 
\textit{Sb. Math.} 209(2):258--275.

\bibitem{9-pav}
\Aue{Hofstad, R.} 2020. Random graphs and complex Networks. Vol.~2. 
\textit{Notes RGCNII} Available at: 
{\sf https:// www.win.tue.nl/$\sim$rhofstad/NotesRGCNII.pdf} (accessed January~11, 2021).

\bibitem{10-pav}
\Aue{Pavlov, Yu.\,L.} 2019. O~svyaznosti konfiguratsionnykh grafov 
[On connectivity of configuration graphs]. \textit{Discrete Mathematics Applications} 31(2):115--123.

\bibitem{11-pav}
\Aue{Ibragimov, I.\,A., and Yu.\,V.~Linnik.} 1965.  
\textit{Nezavisimye i~statsionarno svyazannye velichiny} 
[Independent and stationary sequences of random variables]. Moscow: Nauka. 524~p.

\bibitem{12-pav}
\Aue{Bigham, N.\,H., C.\,M.~Goldie, and J.\,L.~Teugels.}
 1987. \textit{Regular variations. Encyclopedia of mathematics and its applications.} 
 Cambridge: Cambridge University Press.  Vol.~27. 513~p.



\end{thebibliography}

 }
 }

\end{multicols}

\vspace*{-3pt}

  \hfill{\small\textit{Received April~15, 2020}}


%\pagebreak

%\vspace*{-8pt}


\Contrl

\noindent
\textbf{Pavlov Yuri L.} (b.\ 1949)~--- 
Doctor of Science in physics and mathematics, principal scientist, Institute 
of Applied Mathematical Research of the Karelian Research Centre 
of the Russian Academy of Sciences, 11~Pushkinskaya Str., Petrozavodsk 185910, Russian Federation; 
\mbox{pavlov@krc.karelia.ru}


\label{end\stat}

\renewcommand{\bibname}{\protect\rm Литература} 
     