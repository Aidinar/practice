\def\stat{grinchenko}

\def\tit{О СИСТЕМНОЙ ИЕРАРХИИ ИСКУССТВЕННОГО ИНТЕЛЛЕКТА}

\def\titkol{О системной иерархии искусственного интеллекта}

\def\aut{С.\,Н.~Гринченко$^1$}

\def\autkol{С.\,Н.~Гринченко}

\titel{\tit}{\aut}{\autkol}{\titkol}

\index{Гринченко С.\,Н.}
\index{Grinchenko S.\,N.}

%{\renewcommand{\thefootnote}{\fnsymbol{footnote}} \footnotetext[1]
%{Работа выполнена при частичной поддержке РФФИ (проект 19-07-00187-A).}}

\renewcommand{\thefootnote}{\arabic{footnote}}
\footnotetext[1]{Институт проблем информатики Федерального исследовательского центра <<Информатика 
и~управление>> Российской академии наук, \mbox{sgrinchenko@ipiran.ru}}

%\vspace*{-12pt}

 



  
  \Abst{Искусственный интеллект (ИИ) рассмотрен с позиций  
ин\-фор\-ма\-ти\-ко-ки\-бер\-не\-ти\-че\-ско\-го моделирования (ИКМ) процесса развития 
самоуправляющейся ие\-рар\-хо\-се\-те\-вой сис\-те\-мы Человечества как природный феномен, 
теснейшим образом сопряженный с понятиями <<когнитивные функции человека>> 
и~<<интеллектуальная деятельность человека>>. Опираясь на  
ин\-фор\-ма\-ци\-он\-но-ком\-му\-ни\-ка\-ци\-он\-но-ин\-фра\-струк\-тур\-ную составляющую 
определения ИИ и на ИКМ, понятие <<че\-ло\-ве\-ко-ап\-па\-ра\-тур\-ной интеллектуальной 
единицы>> естественным образом обобщено на \textit{все} уров\-ни/яру\-сы сис\-те\-мы 
Человечества, расположенные в ее иерархии выше и ниже относительно уров\-ня/яру\-са 
<<че\-ло\-век/лич\-ность>>. Как следствие, феномен <<личностного  
ес\-тест\-вен\-но-ис\-кус\-ст\-вен\-но\-го интеллекта>> дополняется феноменом 
<<иерархического ИИ>>. Его формирование стало возможным начиная примерно 
с~1946~г.~--- с~возникновением базисной информационной технологии (БИТ) 
компьютеров~--- и~приняло взрывной характер примерно с~1979~г.~--- с~возникновением 
БИТ те\-ле\-ком\-му\-ни\-ка\-ций/се\-тей. Приводятся типичные размеры ареалов  
уров\-ней/яру\-сов в иерархии ИИ сис\-те\-мы Человечества (указанные даты и~размеры~--- 
результат модельного расчета).}
  
  \KW{искусственный интеллект; информационные технологии; информатико-ки\-бер\-не\-ти\-че\-ская модель; 
  самоуправляющаяся иерархосетевая система Человечества;  
че\-ло\-ве\-ко-ап\-па\-ра\-тур\-ная интеллектуальная единица}

\DOI{10.14357/19922264210115}

%\vspace*{2pt}

\vskip 10pt plus 9pt minus 6pt

\thispagestyle{headings}

\begin{multicols}{2}

\label{st\stat}
      
  Что такое \textit{искусственный интеллект}? Будем исходить из 
следующей формулировки: это <<комплекс технологических решений, 
поз\-во\-ля\-ющий имитировать когнитивные функции человека (включая 
самообучение и поиск решений без заранее заданного алгоритма) и получать 
при выполнении конкретных задач результаты, сопоставимые, как минимум, 
с~результатами интеллектуальной деятельности человека. Комплекс 
технологических решений включает в себя  
ин\-фор\-ма\-ци\-он\-но-ком\-му\-ни\-ка\-ци\-он\-ную инфраструктуру, 
программное обеспечение (в~том числе то, в~котором используются методы 
машинного обучения), процессы и сервисы по обработке данных и поиску 
решений>>~[1].
  
  Из определения ИИ следует, что это понятие теснейшим образам сопрягается 
с~понятиями <<когнитивные функции человека>> и <<интеллектуальная 
деятельность человека>>. Конкретизируем этот факт.
  
  Тезис~1 (А.\,А.~Зиновьева): <<Человеческий интеллект есть не просто 
интеллект отдельно взятых людей. Это~--- интеллект человечества, 
аккумулирующийся в языковой практике человечества, включая практику 
научного познания$\ldots$ Разгадку тайны человеческого сознания надо искать не 
в~головах отдельно взятых людей, а в интеллектуальной деятельности всего 
человечества, осуществляемой в языке>>~[2]. 
  
  Тезис 2 (авторский) состоит в рассмотрении Человечества~--- в~контексте 
его ИКМ~--- не как 
суммы собственно людей как таковых, но как со\-во\-куп\-ности  
<<\textbf{че\-ло\-ве\-ко-ап\-па\-ра\-тур\-ных интеллектуальных единиц}>>, 
характерных для соответствующей <<про\-дви\-ну\-той>> фазы его  
лич\-ност\-но-про\-из\-вод\-ст\-вен\-но-со\-ци\-аль\-ной 
метаэволюции\footnote[2]{Метаэволюция~--- процесс последовательного наращивания чис\-ла 
уровней/ярусов иерархической сис\-те\-мы в ходе ее формирования как таковой.}. При этом 
<<\textit{аппаратурная составляющая} интеллектуальной единицы в~общем 
случае содержит комбинацию <<софтвера>> (software), <<хардвера>> 
(hardware) и <<брэйнвера>>\linebreak (brainware), иначе говоря: \textit{памяти} 
(\textit{баз данных})\;+\;\textit{аппаратной основы}\;+\;\textit{алгоритмов 
преобразования информации}; аналогичным образом в ее 
<<\textit{человеческой}>> \textit{со\-став\-ля\-ющей} (по крайней мере, для 
человека, находящегося в~сознании) целесообразно выделить подобную же 
комбинацию, а~именно: \textit{продуктивную личностную па\-мять}\;+\; 
\textit{ра\-бо\-то\-спо\-соб\-ный мозг}\;+\;\textit{при\-ем\-ле\-мый уровень 
мыслительной деятельности}>>~[3, с.~92; 4, 5].
  
  Тезис~3 (авторский) заключается в формализации понятия <<интеллект>>~--- 
с~ин\-фор\-ма\-ти\-ко-ки\-бер\-не\-ти\-че\-ских позиций~--- как проявления 
деятельности имманентно присущего системам живой\linebreak  
и~<<че\-ло\-ве\-ко-ис\-кус\-ствен\-ной>> природы механизма иерархической 
адаптивной поисковой оптимизации (целевых критериев энергетического 
характера), определяющего их приспособительное поведение~[6, 7]. Введение 
такого модельного взгляда при рассмотрении основных структурных элементов 
окружающего мира эквивалентно переходу от их трактовки как 
\textit{пассивных} и~<<\textit{косных}>>, безропотно допускающих любые 
воздействия на себя извне~--- к~представлению о~них как об \textit{активных}, 
перманентно стремящихся к~достижению \textit{собственных целей} (т.\,е.\ 
к~энергетически оптимальным их состояниям), парирующих существенную 
часть подобных воздействий. 
  
  Представление об \textit{аппаратурной составляющей} 
 <<че\-ло\-ве\-ко-ап\-па\-ра\-тур\-ной интеллектуальной единицы>>  
в~ие\-рар\-хо\-се\-те\-вой сис\-те\-ме Человечества вполне соответствует 
вышеприведенному понятию ИИ в локальном его смысле, соотносящемся 
с~отдельным \textit{че\-ло\-ве\-ком/лич\-ностью}. При этом, опираясь на 
ин\-фор\-ма\-ци\-он\-но-ком\-му\-ни\-ка\-ци\-он\-но-ин\-фра\-струк\-тур\-ную со\-став\-ля\-ющую 
определения ИИ и на ИКМ, понятие <<че\-ло\-ве\-ко-ап\-па\-ра\-тур\-ной 
интеллектуальной единицы>> естественным образом обобщается на \textit{все}  
уров\-ни/яру\-сы са\-мо\-управ\-ля\-ющей\-ся ие\-рар\-хо\-се\-те\-вой сис\-те\-мы 
Человечества, расположенные в ее иерархии выше и ниже относительно  
уров\-ня/яру\-са <<че\-ло\-век/лич\-ность>>. Как следствие, феномен 
<<личностного ес\-тест\-вен\-но-ис\-кус\-ст\-вен\-но\-го интеллекта>> (или  
<<че\-ло\-ве\-ко-ап\-па\-ра\-тур\-ной интеллектуальной единицы>>) 
дополняется феноменом <<иерархического искусственного интеллекта>> (или 
иерархии <<со\-ци\-ум-ап\-па\-ра\-тур\-ных интеллектуальных общностей>>). 
  
  <<Человеко-аппаратурная интеллектуальная единица>>~--- элементарное 
звено са\-мо\-управ\-ля\-ющей\-ся ие\-рар\-хо\-се\-те\-вой сис\-те\-мы Человечества~--- 
в~своем достаточно полном виде, т.\,е.\ в~форме триады  
<<софтвер>> (па\-мять)\,--\,<<хард\-вер>> (но\-си\-те\-ли)\,--\,<<брэйн\-вер>> 
(алгоритмы), возникает начиная с момента создания: 
  \begin{itemize}
\item БИТ компьютеров (начиная примерно с~1946~г.),
\item БИТ телекоммуникаций/сетей (начиная примерно с~1979~г.),
\item перспективной нано-БИТ (начиная примерно с~1981~г.)\ 
и~т.\,д.\footnote{Временн$\acute{\mbox{ы}}$е и пространственные количественные 
параметры ИКМ базируются на геометрической прогрессии со знаменателем 
$e^e\hm=15{,}15426\ldots$, выявленной А.\,В.~Жирмунским и~В.\,И.~Кузьминым при 
исследовании критических уровней в развитии биосистем~\cite{8-gr}.}
\end{itemize}

  Их предтечей были БИТ, не обладающие треть\-ей составляющей 
вышеприведенной интеллектуальной триады~--- автоматическими 
\textit{алгоритмами} преобразования информации, но в ка\-кой-то степени 
реа\-ли\-зо\-вы\-ва\-ющие первую и вторую ее со\-став\-ля\-ющие~--- \textit{память} на 
основе внешних (по отношению к человеку) ее пассивных \textit{носителей}, 
рет\-ро\-спек\-тивно: 
  \begin{itemize}
\item БИТ тиражирования текстов/книгопечатания (начиная примерно с~1446~г.~--- <<носитель>> как  
кни\-га/жур\-нал/га\-зе\-та/лис\-тов\-ка/ин\-струк\-тив\-ный материал 
и~т.\,п.);
\item БИТ письменности/чте\-ния (возникшая примерно 8,1~тыс.\ лет назад~--- 
<<носитель>> как рукопись).
\end{itemize}

  Наконец, еще более ранняя в истории Человечества БИТ ре\-чи/язы\-ка 
(возникшая примерно 123~тыс.\ лет назад) в некоторой степени 
реализовывала лишь первую со\-став\-ля\-ющую указанной интеллектуальной 
комбинации~--- память (чис\-то биологическую, управляемую сознанием 
человека, с~носителем~--- корой головного мозга). 
  
  Согласно данной связке ИКМ и ИИ, инфраструктура ИИ пространственно 
структурируется на сле\-ду\-ющие типичные иерархические компоненты 
(рис.~1): 
  \begin{enumerate}[Я+7)]
\item[(Я0)] искусственная составляющая <<личностного  
ес\-тест\-вен\-но-ис\-кус\-ст\-вен\-но\-го интеллекта>> (в~пределах 
пространства, соразмерного  
че\-ло\-ве\-ку/лич\-ности/ин\-ди\-ви\-ду~--- т.\,е.\ личного 
психологического пространства человека с~характерным размером $\sim 
4{,}2$~м~--- радиусом круга вокруг него);
\item[(Я+1)] ИИ, ориентированный на решение задач уровня  
семьи\,--\,дво\-ра\,--\,ма\-ло\-го коллектива (с~характерным размером  
со\-ци\-ума/инфра\-струк\-ту\-ры до $\sim64$~м~--- радиусом круга той 
же площади);
\item[(Я+2)] ИИ, ориентированный на решение задач уровня  
по\-се\-ле\-ния\,--\,круп\-но\-го коллектива (с~характерным размером  
со\-ци\-ума/инфра\-струк\-ту\-ры до $\sim 1$~км);
\item[(Я+3)] ИИ, ориентированный на решение задач уровня округи 
(с~характерным размером со\-ци\-ума/инфра\-струк\-ту\-ры до $\sim 
15$~км);
\item[(Я+4)] ИИ, ориентированный на решение задач уровня <<сверхрайона>> 
(с~характерным размером со\-ци\-ума/инфра\-струк\-ту\-ры 
\mbox{до~$\sim 222$~км});
\item[(Я+5)] ИИ, ориентированный на решение задач уровня 
<<сверхстраны>> (с~характерным размером социума и инфраструктуры 
\mbox{до~$\sim 3370$~км});
\end{enumerate}

\end{multicols}

\begin{figure*} %fig1
\vspace*{1pt}
\begin{center}
\mbox{%
\epsfxsize=162.758mm 
\epsfbox{gri-1.eps}
}
\end{center}
\vspace*{-11pt}
\Caption{Подсистемы иерархосетевой системы Человечества на этапе развития: 
(\textit{а})~компьютерном~--- примерно с~1946~г.; (\textit{б})~телекоммуникационном~--- 
примерно с~1979~г.;
СТ~--- социальные технологии; ИПТ~--- инфраструктурные 
производственные технологии; ПТ~--- производственные технологии; Я~--- ярусы.
Восходящие стрелки (имеющие структуру <<многие~--- к~одному>>) отражают первую 
из трех основных составляющих контура поисковой оптимизации~--- \textit{поисковую 
активность} представителей соответствующих ярусов в иерархии. Нисходящие 
сплошные (имеющие структуру <<один~--- ко многим>>) стрелки отражают вторую 
составляющую~--- \textit{целевые} критерии поисковой оптимизации энергетики системы 
Человечества. Нисходящие пунктирные (<<один~--- ко многим>>) стрелки отражают 
третью со\-став\-ля\-ющую~--- \textit{оптимизационную системную память} 
 лич\-ност\-но-про\-из\-вод\-ст\-вен\-но-со\-ци\-аль\-но\-го (результат адаптивных влияний 
представителей вышележащих иерархических ярусов на структуру вложенных в них 
нижележащих)}
%\vspace*{-6pt}
\end{figure*}


\begin{multicols}{2}

\noindent
\begin{enumerate}
\item[(Я+6)] ИИ, ориентированный на решение задач уровня Планетарного 
Человечества (с~характерным размером  
со\-ци\-ума/инфра\-струк\-ту\-ры до $\sim 51$~тыс.\ км);
\item[(Я+7)] ИИ, ориентированный на решение задач уровня Человечества 
Околоземного Космоса (с~характерным размером  
со\-ци\-ума/инфра\-струк\-ту\-ры в объеме, эквивалентном шару 
радиусом $\sim 770$~тыс.\ км), и~т.\,д.
     \end{enumerate}
     
     Производственные технологии (ПТ), которые должны обеспечивать 
реализацию субстрата ИИ указанных уров\-ней/яру\-сов в иерархии системы 
Человечества, соответственно:
     \begin{enumerate}[Я--7)]
\item[(Я--1)] ПТ с точностью до $\sim28$~см;
\item[(Я--2)] ПТ с точностью до $\sim 1{,}8$~см;
\item[(Я--3)] ПТ с точностью до $\sim 1{,}2$~мм;
\item[(Я--4)] ПТ с точностью до $\sim 80$~мкм;
\item[(Я--5)] ПТ с точностью до $\sim 5$~мкм;
\item[(Я--6)] ПТ с точностью до $\sim 0{,}35$~мкм;
\item[(Я--7)] ПТ с точностью до $\sim 23$~нм и т.\,д.
\end{enumerate}
  
  На первом этапе формирования иерархического ИИ (рис.~1,\,\textit{а}) 
возможны лишь начальные попытки этого~--- на базе возникших тогда 
прообразов локальных компьютерных сетей (ярус Я+1). Полностью этот 
процесс обеспечила БИТ телекоммуникаций (рис.~1,\,\textit{б}).
  
  Из определения <<человеко-аппаратурной интеллектуальной единицы>> 
следует, что <<искусственный интеллект>>~--- не автономная сущность, 
а~неотъемлемый элемент триады~[9] <<\textit{естественный\linebreak  
ин\-тел\-лект}\;--\;\textit{ин\-тер\-фейс}\,--\,\textit{ис\-кус\-ст\-вен\-ный 
интеллект}>>, предназначение которой~--- выполнение творческих функций. 
Этим она отличается от \textit{интеллектуальных роботов}, функция которых 
существенно шире и~совпадает с основной функцией человека~--- формировать 
вокруг себя \textit{вторую природу}. Именно в~активности деятельности 
интеллектуальных роботов~--- промежуточного результата технологического 
развития Человечества~--- и~заключается экзистенциальная~---  
гуманитарная!~--- опасность для последнего.
  
\section*{Выводы}

\noindent
\begin{enumerate}[1.]
  \item Высказывание С.~Лема о том, что <<как существуют двигатели разной 
мощности, так же\linebreak могут быть разной силы искусственные интеллекты>>~[10], 
с~позиций ин\-фор\-ма\-ти\-ко-ки\-бер\-не\-ти\-че\-ско\-го подхода выглядит вполне 
оправданным.




  \item Человек с его \textit{естественным интеллектом}~--- как центральный 
субъект в системной про\-стран\-ст\-вен\-но-вре\-мен\-н$\acute{\mbox{о}}$й 
иерархии лич\-ност\-но-про\-из\-вод\-ст\-вен\-но-со\-ци\-аль\-ной природы 
(\mbox{Человечества}) и~как элемент в системной иерархии живой природы~--- есть 
природный феномен. В~ходе своего эволюционного саморазвития он 
естественным образом формирует окружающие его сущности, к~каковым, 
несомненно, должен быть отнесен и~<<искусственный интеллект>> как 
<<феномен формирующейся второй природы>>. 
  \item Позиционирование ИТ локальных компьютеров и ИТ 
телекоммуникаций в качестве не\-отъемлемых составляющих совокупности 
монотонно усложняющихся в ходе \mbox{цивилизационного} развития 
информационных технологий позволяет рассматривать их появление 
и~функционирование в широком контексте единой исторической 
ретроспективы и перспективы, давая возможность делать не только 
теоретические, но и практические выводы.
  \end{enumerate}
  
{\small\frenchspacing
{%\baselineskip=10.8pt
%\addcontentsline{toc}{section}{References}
\begin{thebibliography}{99}

%\vspace*{-2pt}
\bibitem{1-gr}
%Национальная стратегия развития искусственного интеллекта на период до 2030~года. 
%Утверждена Указом Президента Российской Федерации от 10~октября 2019~г.\ №\,490. 
О~развитии искусственного 
интеллекта в Российской Федерации: Указ Президента Российской Федерации от 
10.10.2019 №\,490. {\sf 
http://publication.\linebreak pravo.gov.ru/Document/View/0001201910110003}.
\bibitem{2-gr}
\Au{Зиновьев А.\,А.} Логический интеллект.~--- М.: МосГУ, 2006. 282~с.
\bibitem{3-gr}
\Au{Гринченко С.\,Н.} Метаэволюция (систем неживой, живой и социально-технологической 
природы).~--- М.: ИПИ РАН, 2007. 456~с.
\bibitem{4-gr}
\Au{Гринченко С.\,Н.} <<Че\-ло\-ве\-ко-ап\-па\-ра\-тур\-ная интеллектуальная единица>> как 
элемент информационного общества: кибернетический взгляд~// Большая Евразия: Развитие, 
безопасность, сотрудничество.~--- М.: ИНИОН РАН, 2018. Вып.~2. Ч.~2. С.~685--688.
\bibitem{5-gr}
\Au{Гринченко С.\,Н.} О~генезисе информационного общества:  
ин\-фор\-ма\-ти\-ко-ки\-бер\-не\-ти\-че\-ское модельное представление~// Информатика и её 
применения, 2019. Т.~13. Вып.~2. С.~100--108.
\bibitem{6-gr}
\Au{Гринченко С.\,Н.} Интеллект и <<поиско\-во-оп\-ти\-ми\-за\-ци\-он\-ная>> картина 
мира~// Открытое образование, 2005. №\,2(49). С.~39--42.
\bibitem{7-gr}
\Au{Гринченко С.\,Н.} О~пространственном структурировании феномена <<искусственный 
интеллект>>~// \mbox{ИТНОУ}: Информационные технологии в науке, образовании и управлении, 
2019. №\,4(14). С.~10--16.
\bibitem{8-gr}
\Au{Жирмунский А.\,В., Кузьмин~В.\,И.} Критические уровни в процессах развития 
биологических сис\-тем.~--- М.: Наука, 1982. 179~с.
\bibitem{9-gr}
\Au{Гринченко С.\,Н.} Биполярность и триадичность: кибернетический взгляд на проблему~// 
Biocosmology~--- neo-Aristotelism, 2016. Vol.~6. No.\,1. P.~166--175.
\bibitem{10-gr}
\Au{Лем С.} Artificial servility~// Мегабитовая бомба~/
Пер. с~польск.~--- 2002.
(\Au{Lem~S.} Artificial servility~// Bomba megabitowa.~--- 
Krakow, Poland: Wydawnictwo Literackie, 1999. 228~p.) 

\end{thebibliography}

}
}

\end{multicols}

\vspace*{-8pt}

\hfill{\small\textit{Поступила в~редакцию 10.02.2020}}

%\vspace*{8pt}

%\pagebreak

\newpage

\vspace*{-28pt}

%\hrule

%\vspace*{2pt}

%\hrule

%\vspace*{-2pt}


\def\tit{ON THE SYSTEM HIERARCHY OF~ARTIFICIAL INTELLIGENCE}

\def\titkol{On the system hierarchy of~artificial intelligence}

\def\aut{S.\,N.~Grinchenko}

\def\autkol{S.\,N.~Grinchenko}

\titel{\tit}{\aut}{\autkol}{\titkol}

\vspace*{-11pt}

\noindent
Institute of Informatics Problems of the Federal Research Center 
``Informatics and Control'' of the Russian Academy of Sciences,  
44-2~Vavilov Str., Moscow 119333, Russian Federation


\def\leftfootline{\small{\textbf{\thepage}
\hfill INFORMATIKA I EE PRIMENENIYA~--- INFORMATICS AND
APPLICATIONS\ \ \ 2021\ \ \ volume~15\ \ \ issue\ 1}
}%
\def\rightfootline{\small{INFORMATIKA I EE PRIMENENIYA~---
INFORMATICS AND APPLICATIONS\ \ \ 2021\ \ \ volume~15\ \ \ issue\ 1
\hfill \textbf{\thepage}}}

\vspace*{3pt} 





\Abste{Artificial intelligence (AI) is considered from the point of view of 
informatics-cybernetic modeling (ICM) of the development process of the 
self-controlling hierarchical-network system of Humankind as a~natural 
phenomenon, closely associated with the concepts of ``cognitive functions 
of man'' and ``intellectual activity of\linebreak\vspace*{-12pt}}

\Abstend{man.'' Based on the information-communication-infrastructural 
component of the AI definition and on 
ICM, the concept of ``human-hardware intellectual unit''' is naturally 
generalized to all levels/tiers of the Humanity system located in its 
hierarchy above and below relative to the level/tier of ``personality.'' As 
a~result, the phenomenon of ``personal natural-artificial intelligence'' is 
supplemented by the phenomenon of ``hierarchical AI.'' Its formation 
became possible starting from $\sim 1946$~--- with the advent of basic 
information technology (BIT) of computers and took on an explosive 
character from $\sim 1979$~--- with the advent of 
telecommunication/network BIT. Typical sizes of the ranges of levels/tiers 
in the hierarchy of the AI of the Humankind system are given (indicated 
dates and sizes are the result of a model calculation).}

\KWE{artificial intelligence; information technology;  
informatics-cybernetic model; self-controlling hierarchical-network system 
of Humankind; human-hardware intellectual unit}


\DOI{10.14357/19922264210115}

\vspace*{6pt}

\begin{multicols}{2}

\renewcommand{\bibname}{\protect\rmfamily References}
%\renewcommand{\bibname}{\large\protect\rm References}

{\small\frenchspacing
{%\baselineskip=10.8pt
\addcontentsline{toc}{section}{References}
\begin{thebibliography}{99}
\bibitem{1-gr-1}
O razvitii iskusstvennogo intellekta v~Rossiyskoy Federatsii: Ukaz 
Prezidenta ot 10.10.2019 No.\,490 [About strategy of scientific and 
technological development of the Russian Federation. Presidential Decree 
No.\,490 dated 10.10.2019]. Available at: {\sf 
http://static.kremlin.ru/\linebreak media/events/files/ru/AH4x6HgKWANwVtMOfPDhcb\linebreak Rpvd1HCCsv.pdf} (accessed 
December~8, 2020).
\bibitem{2-gr-1}
\Aue{Zinov'ev, A.\,A.} 2006. \textit{Logicheskiy intellect} [The logical 
intellect]. Moscow: MosGU. 282~p.
\bibitem{3gr-1}
\Aue{Grinchenko, S.\,N.} 2007. \textit{Metaevolyutsiya (sistem nezhivoy, 
zhivoy i~sotsial'no-tekhnologicheskoy prirody}) [Metaevolution (of 
inanimate, animate, and sociotechnological nature systems)]. Moscow: 
IPIRAN. 456~p.
\bibitem{4-gr-1}
\Aue{Grinchenko, S.\,N.} 2019. ``Cheloveko-apparaturnaya\linebreak 
intellektual'naya edinitsa'' kak element informatsionnogo obshchestva: 
kiberneticheskiy vzglyad [<<Human-apparatus intellectual unit>> as an 
element of the information society: Cybernetic view]. \textit{Bol'shaya 
Evraziya: Razvitiye, bezopasnost', sotrudnichestvo} [Greater Eurasia: 
Development, security, cooperation] 2(2):685--688.
\bibitem{5-gr-1}
\Aue{Grinchenko, S.\,N.} 2019. O genezise informatsionnogo 
obshchestva: informatiko-kiberneticheskoe model'noe predstavlenie 
[On the genesis of the information society: Informatics-cybernetic model 
representation]. \textit{Informatika i~ee Primeneniya~--- Inform. Appl.} 
13(2):100--108.
\bibitem{6-gr-1}
\Aue{Grinchenko, S.\,N.} 2005. Intellekt  
i~``poiskovo-optimizatsionnaya'' kartina mira [Intelligence and the 
``search-optimization'' picture of the World]. \textit{Otkrytoe obrazovanie} 
[Open Education] 2:39--42.
\bibitem{7-gr-1}
\Aue{Grinchenko, S.\,N.} 2019. O prostranstvennom strukturirovanii 
fenomena ``iskusstvennyy intellekt'' [On the spatial structuring of the 
phenomenon of ``artificial intelligence'']. \textit{ITNOU: Informatsionnye 
tekhnologii v~nauke, obrazovanii i~upravlenii} [ITNOU: Information 
Technologies in Science, Education and Conrol] 14(4):10--16.
\bibitem{8-gr-1}
\Aue{Zhirmunskiy, A.\,V., and V.\,I.~Kuz'min.} 1982. \textit{Kriticheskie 
urovni v~protsessakh razvitiya biologicheskikh sistem} [Critical levels in 
the development of biological systems]. Moscow: Nauka. 179~p.
\bibitem{9-gr-1}
\Aue{Grinchenko, S.\,N.} 2016. Bipolyarnost' i~triadichnost': 
kiberneticheskiy vzglyad na problemu [Bipolarity and triad: A~cybernetic 
view of the problem]. \textit{Biokosmologiya~--- neo-Aristotelizm} 
6(1):166--175.
\bibitem{10-gr-1}
\Aue{Lem, S.} 1999. Artificial servility. \textit{Bomba megabitowa}. 
Krakow, Poland: Wydawnictwo Literackie. 228~p.
\end{thebibliography}

}
}

\end{multicols}

\vspace*{-3pt}

\hfill{\small\textit{Received February~10, 2020}}


%\pagebreak

%\vspace*{-18pt}

\Contrl

\noindent
\textbf{Grinchenko Sergey N.} (b.\ 1946)~--- Doctor of Science in 
technology, professor, principal scientist, Institute of Informatics 
Problems, Federal Research Center ``Computer Science and Control'' of 
the Russian Academy of Sciences, 44-2~Vavilov Str., Moscow 119333, 
Russian Federation; \mbox{sgrinchenko@ipiran.ru}

\label{end\stat}

\renewcommand{\bibname}{\protect\rm Литература} 