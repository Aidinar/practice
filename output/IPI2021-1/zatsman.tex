\def\stat{zatsman}

\def\tit{ПРОБЛЕМНО-ОРИЕНТИРОВАННАЯ АКТУАЛИЗАЦИЯ СЛОВАРНЫХ СТАТЕЙ 
ДВУЯЗЫЧНЫХ СЛОВАРЕЙ И~МЕДИЦИНСКОЙ ТЕРМИНОЛОГИИ:\\ 
СОПОСТАВИТЕЛЬНЫЙ АНАЛИЗ$^*$}

\def\titkol{Проблемно-ориентированная актуализация словарных статей 
двуязычных словарей и~медицинской терминологии} %:  сопоставительный анализ}

\def\aut{И.\,М.~Зацман$^1$}

\def\autkol{И.\,М.~Зацман}

\titel{\tit}{\aut}{\autkol}{\titkol}

\index{И.\,М.~Зацман$^1$}
\index{I.\,M.~Zatsman}

{\renewcommand{\thefootnote}{\fnsymbol{footnote}} \footnotetext[1]
{Исследование выполнено при поддержке РФФИ (проект 20-012-00166) и~по гранту РФФИ 
и~Государственного фонда естественных наук Китая №\,21-57-53018.
  }}

\renewcommand{\thefootnote}{\arabic{footnote}}
\footnotetext[1]{Институт проблем информатики Федерального исследовательского центра <<Информатика 
и~управ\-ле\-ние>> Российской академии наук, \mbox{izatsman@yandex.ru}}


%\vspace*{-12pt}



  \Abst{Сопоставляются два подхода к~целенаправленному извлечению нового знания из 
текстовых данных. Первый подход относится к~предметной области лексикографии 
и~ориентирован на извлечение новых значений слов из текстов в~интересах пополнения 
словарных статей двуязычных словарей. Второй подход относится к~медицинской науке 
и~ориентирован на извлечение новых значений терминов в~интересах актуализации 
терминологического портрета болезни, включающего дефиниции терминов с~отражением 
их динамики во времени, отношения между терминами, контексты их использования 
и~ссылки на источники контекстов. Эти подходы сравниваются по следующим позициям: 
\textit{проблема}, для решения которой извлекается новое знание, \textit{цель} его 
извлечения, \textit{источник} концептов нового знания, \textit{эталон}, сравнение 
с~которым используется как критерий новизны, \textit{связь} концепта с~его источником 
и~\textit{динамика} концептов. Цель статьи состоит в~описании результатов 
сопоставительного анализа обоих подходов. Результаты анализа предлагается 
позиционировать как исходные данные для создания концепции системы  
ис\-кус\-ст\-вен\-но-ес\-тест\-вен\-но\-го интеллекта (ИЕИ-сис\-те\-мы) для целенаправленного извлечения 
нового знания из данных большого объема, применимой в~разных предметных областях.}
  
  \KW{генерация нового знания; извлечение знания из текстов; искусственный 
интеллект; система искусственно-естественного интеллекта}

\DOI{10.14357/19922264210113}


\vskip 10pt plus 9pt minus 6pt

\thispagestyle{headings}

\begin{multicols}{2}

\label{st\stat}
  
\section{Введение}

  Сопоставляются два подхода к~извлечению нового знания из текстовых 
данных. Первый подход относится к~предметной области лексикографии 
и~ориентирован на извлечение новых значений языковых единиц (ЯЕ) из 
параллельных текстов художественных произведений в~интересах 
пополнения словарных статей двуязычных словарей. Второй подход 
относится к~медицинской науке и~ориентирован на извлечение новых 
значений терминов в~интересах актуализации терминологического портрета 
болезни\footnote[2]{В рамках второго подхода портрет болезни~--- это множество аннотаций 
терминов, используемых при описании болезни. Аннотация содержит: (1)~структурированную 
дефиницию значения термина, в~которой могут быть выделены его подзначения; 
(2)~ретроспективу ее изменений во времени; (3)~отношения с~другими терминами; (4)~контексты 
использования термина; (5)~его синонимы; (6)~ссылки на источники контекстов.}.
  
  Разработка первого подхода для целенаправленного извлечения новых 
значений ЯЕ и~их переводов из параллельных текстов 
в~интересах создания новых версий словарных статей о~немецких модальных 
глаголах (НМГ) является целью проекта по гранту РФФИ  
№\,20-012-00166, который в~настоящее время выполняется в~Институте 
проб\-лем информатики ФИЦ ИУ РАН~[1, 2]. Предлагаемый подход, 
описанию которого посвящен второй раздел статьи, предполагает 
применение двух методов. 

Первый метод служит для нахождения лакун 
в~типологии значений НМГ и~их переводов на русский язык (см. структуру 
типологии на рис.~3 в~[3]). Для нахождения лакун применяется процедура 
лингвистического аннотирования предложений с~НМГ~[4, 5]. 

Второй метод 
предназначен для заполнения лакун, обнаруженных с~помощью первого 
метода, что обеспечивает целенаправленное пополнение существующей 
типологии~\cite{5-z, 6-z}.

\begin{table*}[b]\small %tabl1
\begin{center}

\vspace*{-14pt}

\Caption{Примеры предложений немецко-русских параллельных текстов}
\vspace*{2ex}

\begin{tabular}{|p{75mm}|p{75mm}|}
\hline 
\textbf{Sollte} jetzt etwa eine Predigt stattfinden?\newline
[Franz Kafka. Der Prozess (1914)]&Неужели сейчас кто-то будет читать проповедь?\newline
[Франц Кафка. Процесс (Р.~Райт-Ковалева, 1965)]\\
\hline
Warum \textbf{mu{\!\!\ptb{\ss}}\,te} er diese Demoiselle St$\ddot{\mbox{u}}$wing heiraten 
und den$\ldots$ Laden$\ldots$\newline
[Thomas Mann. Buddenbrooks (1896--1900)]&Зачем ему понадобилось жениться на этой 
мадемуазель Штювинг с~ее$\ldots$ лавкой?\newline
[Томас Манн. Будденброки (Н.~Ман, 1953)]\\
\hline
Er \textbf{durfte} nun eine Weile lang guten Gewissens ruhen.\newline
[Patrick S$\ddot{\mbox{u}}$skind. Das Parfum: Die Geschichte eines 
M$\ddot{\mbox{o}}$rders (1985)]&Теперь он имел право некоторое время 
отдыхать.\newline
[Патрик Зюскинд. Парфюмер: История одного убийцы (Э.~Венгерова, 1992)]\\
\hline
\end{tabular}
\end{center}
\end{table*}
  
  
  Разработка второго подхода и~его применение для извлечения из текстов 
новых значений медицинских терминов в~интересах актуализации 
терминологического портрета исследуемой болезни является одной из задач 
проекта по гранту РФФИ и~Государственного фонда естественных наук 
Китая №\,21-57-53018. Цель проекта состоит в~разработке теоретических 
основ создания интеллектуальной системы извлечения из текстовых данных 
знания о~проб\-ле\-ме раннего выявления и~прогнозирования заболеваний 
с~использованием регулярно актуализируемых терминологических 
портретов болезней, содержащих аннотации терминов.
  
  Второй подход, описанию которого посвящен\linebreak третий раздел статьи, 
основан на обработке научных текстов по медицинской науке с~целью 
формирования и~итерационного обновления терминологического портрета 
болезни. До начала первой\linebreak итерации экспертами формируется исходная 
версия терминологического портрета на основе существующих словарей 
медицинских терминов и~их дефиниций~[7--9]. На первой итерации исходная\linebreak 
версия портрета используется для программного поиска и~извлечения новых 
терминов из коллекции текстов большого объема в~целях его обновления,\linebreak 
в~результате которого формируется первая версии портрета. На второй 
и~последующих итерациях используются уже обновленные портреты для 
извлечения новых терминов. Итерационный процесс завершается 
экспертами, если прекращается значимое с~их точки зрения обновление 
портрета. Источником новых терминов служат коллекции,\linebreak 
содержащие 
аннотации научных статей по медицинской науке и~их полные тексты. 
В~качестве\linebreak одного из наиболее известных примеров такой коллекции можно 
привести базу данных PubMed,\linebreak содержащую более 30~млн записей 
и~обес\-пе\-чи\-ва\-ющую развитые возможности поиска и~формирования 
статистических данных, вклю\-чая получение распределения статей 
с~заданным термином по\linebreak годам их пуб\-ли\-ка\-ции~[10].
{ %\looseness=1

}
  
  Цель статьи состоит в~описании результатов сопоставительного анализа 
двух подходов, исполь\-зу\-емых в~лингвистике и~медицинской науке. 
Результаты сопоставления предлагается позиционировать\linebreak как исходные 
данные для создания концепции ИЕИ-сис\-те\-мы  
для целенаправленного 
извлечения нового знания из данных большого объема.
  

\section{Первый подход}

\vspace*{-17pt}

  Основная идея целенаправленного извлечения новых значений 
ЯЕ и~их переводов из параллельных текстов заключается 
в~сле\-ду\-ющем. Сначала формулируется проблема, например обновление 
двуязычных словарей за счет извлечения новых значений ЯЕ некоторого 
класса. В~проекте по гранту РФФИ №\,20-012-00166 исследуемыми ЯЕ 
служат~6~НМГ (d$\ddot{\mbox{u}}$rfen, k$\ddot{\mbox{o}}$nnen, 
m$\ddot{\mbox{o}}$gen, m$\ddot{\mbox{u}}$ssen, sollen и~wollen), новыми 
извлеченными значениями которых пополняется не\-мец\-ко-рус\-ский 
словарь~[11]. Вариант этого же словаря по состоянию на начало реализации 
проекта был выбран как эталон, отражающий современный уровень знания 
о~НМГ. Информация его словарных статей о значениях НМГ и~их переводах 
была представлена в~виде исходной версии лингвистической типологии, 
структура которой приведена на рис.~3 в~\cite{3-z}.
  
  После формулировки проблемы (обновление двуязычного словаря), 
определения эталона и~формирования исходной версии лингвистической 
типологии экспертами выбирается массив параллельных текстов как 
потенциальных источников нового\linebreak знания о~ЯЕ исследуемого класса. Для 
поиска новых значений НМГ использовались тексты, полученные из 
параллельного немецкого подкорпуса Национального корпуса русского 
языка~[12]. Эти\linebreak текс\-ты включают 2,6~млн словоупотреблений: 1,4~млн 
сло\-во\-употреб\-ле\-ний в~оригинальных текстах на немецком языке, включая 
16\,268~употреб\-ле\-ний НМГ, и~1,2~млн словоупотреблений в~их переводах 
на русский язык (см.\ табл.~1 с~примерами предложений и~их переводами, 
которые содержат~3 из~6~исследуемых НМГ, выделенных полужирным 
шрифтом).
  
  Пополнение исходной версии типологии происходит в~процессе обработки 
параллельных текстов с~использованием упомянутых в~первом разделе 
методов и~разработанной на их основе информационной технологии поиска 
новых значений НМГ~[13]. Сначала в~параллельных текстах программно 
осуществляется поиск предложений, которые содержат исследуемые ЯЕ.
  

  
  Для предложений и~их переводов, найденных по леммам НМГ, 
заполняются поля структурированных аннотаций (см.\ примеры двух 
аннотаций в~табл.~2 и~3 в~\cite{14-z}), включая поле значения исследуемой 
ЯЕ в~найденном предложении. В~целях его заполнения лингвист 
сопоставляет смысловое содержание ЯЕ в~найденном предложении 
с~дефинициями значений ЯЕ, уже имеющимися в~текущей версии 
типологии. Если такая дефиниция в~ней найдена, то лингвист проставляет 
в~аннотации код этой дефиниции (см.\ код <<sollen-3>> в~табл.~2  
в~\cite{14-z}). Если она отсутствует, то в~этом поле аннотации проставляется 
специальный код с~меткой~<<x>> (см.\ код <<sollen-х>> в~табл.~3  
в~\cite{14-z}), говорящий о том, что обнаружена лакуна в~текущей версии 
типологии и~поэтому процесс формирования аннотации не может быть 
завершен без пополнения типологии. Коды с~литерой~<<х>> помечают 
неполные (=\;не\-за\-вер\-шен\-ные) аннотации относительно \textit{текущего 
состояния типологии}.
  
  На основе сформулированных положений по извлечению новых значений 
ЯЕ были разработаны два метода~\cite{15-z}:
  \begin{enumerate}[(1)]
  \item для обнаружения понятийных лакун в~текущей версии типологии 
значений НМГ;
  \item  для целенаправленного заполнения обнаруженных лакун.
  \end{enumerate}
  
  Каждая итерация первого метода включает следующие шаги (итерация 
выполняется для одного употребления исследуемой ЯЕ):
  \begin{itemize}
\item в~параллельных текстах осуществляется поиск предложения, которое 
содержит ис\-сле\-ду\-емую~ЯЕ;
\item в~найденном предложении с~ЯЕ (см. их примеры, найденные по леммам 
НМГ sollen, m$\ddot{\mbox{u}}$ssen и~d$\ddot{\mbox{u}}$rfen, в~первом 
столбце табл.~1) в~результате семантического анализа лингвистами 
определяется значение ЯЕ в~контексте этого предложения;
\item для найденного предложения и~его перевода формируется 
структурированная аннотация контекста НМГ, которая содержит код 
дефиниции значения ЯЕ (по текущей версии типологии) или код 
с~меткой~<<x>>, который обозначает аннотацию, процесс формирования 
которой не завершен из-за отсутствия в~типологии обнаруженного значения 
ЯЕ, т.\,е.\ в~ней обнаружена понятийная лакуна (см. табл.~2 и~3 в~\cite{14-z} 
соответственно), а также гиперссылка на источник контекста 
с~найденной~ЯЕ;
\item если сформирована незавершенная аннотация с~кодом с~меткой~<<x>>, 
то она передается на дальнейшую обработку с~помощью второго метода;
\item аннотированное предложение помечается как обработанное первым 
методом и~осуществляется переход к~первому шагу.
\end{itemize}

  Незавершенные аннотации обрабатываются экспертами высокой 
квалификации, имеющими, как правило, степень доктора филологических 
наук, с~помощью второго метода заполнения обнаруженных понятийных 
лакун в~типологии. Обработка незавершенных аннотаций, которых, как 
правило, на три порядка меньше, чем число употреблений ЯЕ, включает 
следующие шаги:
  \begin{itemize}
\item для нового значения ЯЕ (т.\,е.\ отсутствующего в~текущей версии 
типологии) лингвистами формулируется новая дефиниция или увеличивается 
объем значения уже существующей дефиниции~\cite{16-z};\\[-13pt]
\item если формулируется новая дефиниция, то она обозначается новым 
кодом и~добавляется в~текущую версию типологии;\\[-13pt]
\item в~незавершенной аннотации новый код заменяет код с~меткой~<<x>>, 
и~она становится завершенной;\\[-13pt]
\item если увеличивается объем значения уже существующей дефиниции, то 
ее код проставляется в~незавершенной аннотации;\\[-13pt]
\item вариант перевода нового значения добавляется в~типологию.
\end{itemize}

  В заключение описания первого подхода и~двух его методов отметим, что 
семантический анализ\linebreak незавершенных аннотаций придает 
це\-ле\-на\-прав\-лен\-ность процессу заполнения лакун в~лингвистической 
типологии. Это обеспечивается тем, что сначала фиксируются понятийные 
лакуны в~типологии с~помощью простановки в~незавершенных аннотациях 
кодов с~меткой~<<x>>, а затем формируются дефиниции новых значений 
исследуемых ЯЕ, что и~заполняет лакуны в~типологии \textit{согласно цели 
проекта}.

\vspace*{-6pt}
  
\section{Второй подход}

\vspace*{-2pt}

  Второй подход к~обнаружению концептов нового знания относится 
к~медицинской науке и~ориентирован на извлечение концептов нового 
знания из научных текстов в~этой предметной области. Сначала 
формулируется проблема, например раннее выявление и~прогнозирование 
заболеваний с~использованием регулярно актуализируемых 
терминологических портретов болезней. Затем выбирается эталонный 
словарь, отражающий современный уровень знания о терминах медицинской 
науки, например~\cite{8-z}. После формулировки проблемы и~выбора 
эталонного словаря определяется коллекция научных текстов как 
потенциальных источников нового знания для актуализации портрета, 
например PubMed~\cite{10-z}. Второй подход основан на итерационном 
выполнении четырех стадий обработки научных текстов коллекции большого 
объема. Первые две стадии предназначены для автоматической их обработки, 
которая существенно сужает круг потенциальных источников нового знания 
для актуализации портрета экспертами. Существенное сужение необходимо 
обеспечить на первых двух стадиях из-за последующего подключения 
экспертов к~извлечению нового знания на каждой итерации. Эксперты 
выполняют две последние стадии семантического анализа потенциальных 
источников и~актуализацию терминологического портрета болезни.
  
  До начала первой итерации автоматической обработки научных текстов 
коллекции экспертами в~области медицинской науки формируется исходная 
версия терминологического портрета болезни, который включает аннотации, 
состоящие из~6~компонентов, перечисленных в~сноске первого раздела 
статьи. Эксперты используют, с~одной стороны, термины словарей 
и~классификаций~\cite{7-z, 8-z, 9-z, 17-z}, с~другой стороны, термины и~их 
контексты из экспертной выборки статей об исследуемой болезни. 
  
  Как было отмечено в~разд.~1, на первой итерации исходная версия 
портрета используется для извлечения из коллекции потенциально новых 
терминов (далее~--- тер\-ми\-ны-кан\-ди\-да\-ты) и~их контекстов в~целях 
пополнения портрета. Потом эксперты выполняют семантический анализ 
контекстов тер\-ми\-нов-кан\-ди\-да\-тов и~отбирают новые термины. Они 
добавляются в~исходную версию портрета, и~в~конце первой итерации 
эксперты формируют его первую версию. На второй и~последующих 
итерациях используются уже пополненные портреты для извлечения новых 
тер\-ми\-нов-кан\-ди\-да\-тов и~их контекстов. 

Рассмотрим подробнее четыре стадии 
каждой итерации.
  
  Цель первой стадии автоматической обработки коллекции состоит 
в~создании пред\-мет\-но-ори\-ен\-ти\-ро\-ван\-но\-го массива (ПОМ) 
научных текстов, которые описывают исследуемую болезнь на всех стадиях 
заболевания, включая ранние. Эти тексты позиционируются как 
потенциальные источники нового знания об этой болезни, т.\,е.\ 
неотраженного в~медицинских классификациях, словарях и~справочниках, 
используемых экспертами проекта. Пред\-мет\-но-ори\-ен\-ти\-ро\-ван\-ный массив в~прос\-тей\-шем случае может быть 
сформирован в~результате поиска по названию болезни и~его синонимам. 
Например, если в~коллекции PubMed, содержащей более 30~млн 
записей~\cite{10-z}, задать поиск по запросу <<covid-19>>, то будут найдены 
87\,737~текстов (по состоянию на 16.01.2021), а~если поисковый запрос 
расширить за счет синонима: (<<covid-19>>) OR (<<coronavirus disease 
2019>>), то поисковая выдача увеличится до 88\,365~текстов.
  
  На второй стадии каждой итерации из ПОМ научных текстов извлекаются 
тер\-ми\-ны-кан\-ди\-да\-ты. Для их поиска в~ПОМ используются термины 
текущей версии портрета. На основе \textit{поисковых запросов с~этими 
терминами} сначала из ПОМ извлекаются тексты. Затем с~по\-мощью 
лингвистического процессора~[18--20] из этих текс\-тов извлекаются те 
термины, которых нет в~текущей версии портрета (т.\,е.\  
тер\-ми\-ны-кан\-ди\-да\-ты), каждый с~одним или несколькими контекстами 
и~гиперссылками на текс\-ты-ис\-точ\-ни\-ки этих контекстов. Из этих же 
текс\-тов дополнительно извлекаются новые кон\-текс\-ты терминов портрета, 
которых нет в~его текущей версии, и~гиперссылки на текс\-ты-ис\-точ\-ни\-ки 
этих контекстов. В~конце второй стадии каж\-дой итерации  
тер\-ми\-ны-кан\-ди\-да\-ты объединяются с~терминами текущей версии 
портрета в~рамках единого массива терминов (ЕМТ) вмес\-те с~их контекстами 
и~гиперссылками. Такое объединение позволяет в~по\-сле\-ду\-ющей обработке 
актуализировать значения терминов текущей версии портрета, если были 
извлечены их новые контексты из ПОМ.

\begin{table*}[b]\small %tabl2
\begin{center}
\Caption{Сравнительный анализ двух подходов}
\vspace*{2ex}

\begin{tabular}{|c|p{45mm}|p{44mm}|p{45mm}|}
\hline
\multicolumn{1}{|c|}{\tabcolsep=0pt\begin{tabular}{c}Позиции\\ сравнения\end{tabular}}&\multicolumn{1}{c|}{Проект по 
лексикографии}&\multicolumn{1}{c|}{Проект по медицинской 
науке}&\multicolumn{1}{c|}{\tabcolsep=0pt\begin{tabular}{c}Исходные положения\\ как обобщение\end{tabular}}\\
\hline
\raisebox{-6pt}[0pt][0pt]{Проблема}&Обновление двуязычных словарей&Раннее выявление и~прогнозирование 
заболеваний&Формулировка проблемы\\
\hline
\raisebox{-11pt}[0pt][0pt]{Цель}&
Актуализация описаний значений НМГ&Актуализация системы терминов и~их 
дефиниций для описания болезни&Формулировка цели извлечения концептов нового 
знания\\
\hline
\raisebox{-6pt}[0pt][0pt]{Источник}&Параллельные тексты художественных произведений&Коллекция научных 
текстов по медицинской науке&Определение потенциального источника нового знания\\
\hline
\raisebox{-11pt}[0pt][0pt]{Эталон}&Немецко-русский словарь с~дефинициями значений НМГ&Медицинский словарь 
с~дефинициями терминов&Определение эталона, который используется как критерий 
новизны\\
\hline
\raisebox{-6pt}[0pt][0pt]{Связь}&Гиперссылки на тексты художественных произведений&Гиперссылки на тексты 
по медицинской науке&Определение связей концептов с~их источниками\\
\hline
\raisebox{-18pt}[0pt][0pt]{Динамика}&Создается новый вариант существующей дефиниции значения НМГ c 
сохранением ее предыду\-щих вариантов&Создается новый вариант существующей 
дефиниции термина c~сохранением ее предыду\-щих вариантов&Отслеживание динамики 
концептов и~создание ретроспективы вариантов их дефиниций\\
\hline
\end{tabular}
\end{center}
\end{table*}
  
  На третьей стадии выполняется экспертный анализ контекстов каждого 
термина ЕМТ, чтобы сформировать четыре группы терминов и~разделить 
ЕМТ на четыре соответствующих им подмассива ЕМТ1--ЕМТ4 на основе 
четырех критериев (первый критерий для первой группы и~т.\,д.):
  \begin{enumerate}[(1)]
\item термин не имеет отношения к~предметной области исследуемой болезни 
или является общим для терминологических портретов разных болезней;
\item контексты термина говорят о новом синониме одного из терминов 
текущей версии терминологического портрета болезни и/или выбранного 
эталона;
\item новые контексты термина текущей версии портрета, включенные 
в~ЕМТ, говорят о необходимости актуализировать дефиницию его значения;
\item термин в~контекстах выражает концепт нового знания, т.\,е.\ это его 
значение не представлено ни в~текущей версии портрета болезни, ни 
в~выбранном эталоне.
\end{enumerate}

  На этой стадии объектами экспертного анализа служат контексты 
терминов ЕМТ, извлеченные на второй стадии из ПОМ, сформированного на 
первой стадии каждой итерации. Если в~процессе анализа экспертам 
требуется расширить некоторый контекст, то используется гиперссылка на 
текст-ис\-точ\-ник этого контекста.
  
  Цель четвертой стадии каждой итерации состоит в~создании экспертами 
новой версии терминологического портрета болезни добавлением новых или 
обновлением уже существующих аннотаций его текущей версии. При 
создании новой версии портрета:
  \begin{itemize}
  \item термин второй группы добавляется как синоним в~соответствующую 
аннотацию, \textit{если она уже есть в~портрете}; в~противном случае из 
эталонного словаря берется термин с~дефиницией и~на ее основе создается 
новая аннотация термина с~добавлением обнаруженного синонима и~его 
контекстов;
  \item на основе контекстов термина третьей группы экспертами создается 
новый вариант его дефиниции, что обновляет существующую аннотацию 
этого термина;
  \item для каждого термина четвертой группы создается его аннотация.
  \end{itemize}
  
  В~заключение этого раздела отметим, что \mbox{третья} стадия может быть 
автоматизирована (но не на начальных итерациях), если спроектировать 
искусственную нейронную сеть для распределения терминов ЕМТ по 
четырем группам. Предварительно необходимо создать представительный 
набор подмассивов ЕМТ1--ЕМТ4, который может быть использован для 
обучения нейронной сети, чтобы потом применить ее на последующих 
итерациях для автоматического формирования групп терминов 
и~подмассивов.
  
\section{Сопоставление подходов}

\vspace*{-12pt}

  Рассмотренные подходы сопоставим по следующим позициям: 
\textit{проблема}, для решения которой извлекается новое знание; 
\textit{цель} его извлечения; \textit{источник} концептов нового знания; 
\textit{эталон}, сравнение с~которым используется как критерий новизны; 
\textit{связь} концепта с~его источником; \textit{динамика} концептов 
(изменение их смыслового содержания во времени) (табл.~2).
  

  
  В последнюю колонку табл.~2 включены исходные положения, на основе 
которых планируется создать концепцию системы, которую предлагается 
назвать \textit{системой ис\-кус\-ст\-вен\-но-ес\-тест\-вен\-но\-го 
интеллекта}, для целенаправленного извлечения нового 
знания из данных большого объема с~подключением экспертов к~его 
извлечению. Основная идея концепции состоит в~том, чтобы эту систему 
можно было адаптировать для применения в~целом ряде предметных 
областей с~учетом их специфики.
  
  Подключение экспертов к~извлечению нового знания~\cite{21-z, 22-z} 
предполагает, что функционирование ИЕИ-сис\-те\-мы будет охватывать 
ментальную, информационную и~цифровую среды, включая границы между 
ними с~интерфейсами второго и~треть\-его порядка~\cite{23-z}. Последние 
описывают взаимодействие сущностей всех трех сред с~по\-мощью 
полисредовых объектов ИЕИ-сис\-тем, объеди\-ня\-ющих: (1)~концепт 
ментальной среды знаний экспертов; (2)~выражающее его слово как 
последовательность литер, принадлежащей информационной среде; 
(3)~компьютерный код пары (концепт, литеры) цифровой среды. Понятие 
полисредовых объектов может служить теоретической основой разработки 
кодовых таблиц для пар (концепт, литеры). Они обеспечат  
в~ИЕИ-сис\-те\-мах присвоение разных компьютерных кодов парам, 
содержащим один и~тот же термин с~идентификаторами разных его значений 
(=\,кон\-цеп\-тов), выраженных их дефинициями, с~учетом их изменений. Это 
отличает ИЕИ-сис\-те\-мы от систем искусственного интеллекта, в~которых, 
как правило, используются традиционные кодовые таб\-ли\-цы для цифрового 
кодирования только самих слов как последовательностей литер без 
одновременного учета кодами значения слова и~его изменения во времени.
  

\section{Заключение}

  Существующие качественные подходы к~описанию процессов генерации 
нового знания~\cite{24-z} носят объяснительный характер и~не являются 
ал\-го\-рит\-мо-ори\-ен\-ти\-ро\-ван\-ны\-ми. Рассмотренные два подхода таковыми 
являются, что проверено при проектировании на основе первого подхода 
алгоритмов, программ и~технологии целенаправленного извлечения 
экспертами концептов нового знания о~значениях НМГ~\cite{3-z, 5-z, 6-z}. 
Сохранение свойства ал\-го\-рит\-мо-ори\-ен\-ти\-ро\-ван\-ности~--- ключевое 
требование к~созданию концепции ИЕИ-сис\-тем на основе исходных 
положений, перечисленных в~табл.~2.
  
{\small\frenchspacing
{%\baselineskip=10.8pt
%\addcontentsline{toc}{section}{References}
\begin{thebibliography}{99}
\bibitem{1-z}
\Au{Добровольский Д.\,О.} Немецкие модальные глаголы в~параллельном корпусе и~задачи 
двуязычной лексикографии~// Германские языки: текст, корпус, перевод.~--- М.: Институт 
языкознания РАН, 2020. С.~103--116.
\bibitem{2-z}
\Au{Добровольский Д.\,О., Зализняк~Анна~А.} Русские конструкции с~потенциально 
модальным значением по данным параллельных корпусов~// Труды Института русского 
языка им.\ В.\,В.~Виноградова, 2020. №\,3. С.~35--49.
\bibitem{3-z}
\Au{Zatsman I.} Finding and filling lacunas in knowledge systems~// 20th European Conference 
on Knowledge Management Proceedings.~--- Reading: Academic Publishing International Ltd., 
2019. Vol.~2. P.~1143--1151.
\bibitem{4-z}
Handbook of linguistic annotation~/ Eds. N.~Ide, J.~Pustejovsky.~--- Dordrecht, The 
Netherlands: Springer Science\;+\;Business Media, 2017. 1468~p.
\bibitem{5-z}
\Au{Зацман И.\,М.} Целенаправленное развитие систем лингвистических знаний: 
выявление и~заполнение лакун~// Информатика и~её применения, 2019. Т.~13. Вып.~1. 
С.~91--98.

\columnbreak

\bibitem{6-z}
\Au{Зацман И.\,М.} Проб\-лем\-но-ори\-ен\-ти\-ро\-ван\-ная верификация полноты 
темпоральных онтологий и~заполнение понятийных лакун~// Информатика и~её 
применения, 2020. Т.~14. Вып.~3. С.~119--128.
\bibitem{7-z}
Medical dictionary of health terms.~--- Harvard Health Publishing, 2011. {\sf 
https://www.health.harvard.edu/\linebreak a-through-c}.
\bibitem{8-z}
Dorland's illustrated medical dictionary.~--- 33rd ed.~--- Philadelphia, PA, USA: Elsevier Saunders, 2019.\linebreak 
2144~p.
\bibitem{9-z}
MedTerms medical dictionary. {\sf  
https://www.\linebreak medicinenet.com/medterms-medical-dictionary/article.\linebreak htm}.
\bibitem{10-z}
PubMed. National Library of Medicine. {\sf https://pubmed. ncbi.nlm.nih.gov}.
\bibitem{11-z}
Немецко-русский словарь актуальной лексики~/ Под ред. Д.\,О.~Добровольского.~--- М.: 
Лексрус, 2021 (в~печати).
\bibitem{12-z}
Параллельный немецкий подкорпус Национального корпуса русского языка. {\sf 
http://www.ruscorpora. ru/search-para-de.html}.
\bibitem{13-z}
\Au{Zatsman I.} Finding and filling lacunas in linguistic typologies~// 15th  Forum 
(International)  on Knowledge Asset Dynamics Proceedings.~--- Matera: Institute of 
Knowledge Asset Management, 2020. P.~780--793.
\bibitem{14-z}
\Au{Зацман И.\,М.} Стадии целенаправленного извлечения знаний, имплицированных 
в~параллельных текстах~// Системы и~средства информатики, 2018. Т.~28. №\,3.  
С.~175--188.
\bibitem{15-z}
\Au{Zatsman I.} Three-dimensional encoding of emerging meanings in AI-systems~// 21st 
European Conference on Knowledge Management Proceedings.~--- Reading: Academic 
Publishing International Ltd., 2020. P.~878--887.
\bibitem{16-z}
\Au{Гончаров А.\,А., Зацман~И.\,М., Кружков~М.\,Г.} Эволюция классификаций 
в~надкорпусных базах данных~// Информатика и~её применения, 2020. Т.~14. Вып.~4. 
С.~108--116.
\bibitem{17-z}
Международная статистическая классификация болезней
и~проб\-лем, связанных со здоровьем. Десятый пересмотр.
Т.~1. Ч.~1. {\sf 
https://apps.who.int/iris/\linebreak bitstream/handle/10665/85974/5225032680.pdf?seque\linebreak nce=1\&isAllowed=y}.


\bibitem{19-z} %18
\Au{V$\grave{\mbox{a}}$zquez M., Oliver~A.} Improving term candidates' selection using 
terminological tokens~// Terminology, 2018. Vol.~24. No.\,1. P.~122--147.

\bibitem{18-z} %19
\Au{Khakimova A., Yang~X., Zolotarev~O., Berberova~M., Charnine~M.} Tracking knowledge 
evolution based on the terminology dynamics in 4P-medicine~// Int. J.~Env. Res.
Pub. He., 2020. Vol.~17. Iss.~20. Art. ID: 7444.

\bibitem{20-z}
\Au{Oliver~A., V$\grave{\mbox{a}}$zquez~M.} TermEval~2020: Using TSR filtering method to 
improve automatic term extraction~// 6th Workshop (International) on Computational 
Terminology Proceedings.~--- Paris: European Language Resources 
Association, 2020. P.~106--113.

\pagebreak

\bibitem{21-z}
\Au{Jarrahi~M.\,H.} Artificial intelligence and the future of work: Human-AI symbiosis in 
organizational decision making~// Bus. Horizons, 2018. Vol.~61. No.\,4. P.~577--586.



\bibitem{22-z}
\Au{Trunk A., Birkel~H., Hartmann~E.} On the current state of combining human and artificial 
intelligence for strategic organizational decision making~// Bus. Res., 2020. Vol.~13. 
No.\,3. P.~875--919.
\bibitem{23-z}
\Au{Зацман И.\,М.} Интерфейсы третьего порядка в~информатике~// Информатика и~её 
применения, 2019. Т.~13. Вып.~3. С.~82--89.
\bibitem{24-z}
\Au{Wierzbicki~A.\,P., Nakamori~Y.} Basic dimensions of creative space~// Creative space: 
Models of creative processes for knowledge civilization age~/ Eds. A.\,P.~Wierzbicki, 
Y.~Nakamori.~--- Berlin: Springer Verlag, 2006. P.~59--90.
\end{thebibliography}

}
}

\end{multicols}

\vspace*{-3pt}

\hfill{\small\textit{Поступила в~редакцию 16.01.2021}}

\vspace*{8pt}

%\pagebreak

%\newpage

%\vspace*{-28pt}

\hrule

\vspace*{2pt}

\hrule

%\vspace*{-2pt}

\def\tit{PROBLEM-ORIENTED UPDATING OF DICTIONARY ENTRIES OF~BILINGUAL 
DICTIONARIES AND~MEDICAL TERMINOLOGY: COMPARATIVE ANALYSIS}

\def\titkol{Problem-oriented updating of dictionary entries of~bilingual 
dictionaries and~medical terminology: Comparative analysis}

\def\aut{I.\,M.~Zatsman}

\def\autkol{I.\,M.~Zatsman}

\titel{\tit}{\aut}{\autkol}{\titkol}

\vspace*{-11pt}


\noindent
Institute of Informatics Problems, Federal Research Center ``Computer Science and
Control'' of the Russian Academy of Sciences, 44-2~Vavilov Str., Moscow 119333,
Russian Federation

\def\leftfootline{\small{\textbf{\thepage}
\hfill INFORMATIKA I EE PRIMENENIYA~--- INFORMATICS AND
APPLICATIONS\ \ \ 2021\ \ \ volume~15\ \ \ issue\ 1}
}%
\def\rightfootline{\small{INFORMATIKA I EE PRIMENENIYA~---
INFORMATICS AND APPLICATIONS\ \ \ 2021\ \ \ volume~15\ \ \ issue\ 1
\hfill \textbf{\thepage}}}

\vspace*{3pt}


\Abste{Two approaches to the goal-oriented discovery of new knowledge from text data are 
compared. The first approach relates to the subject domain of lexicography. It focuses on 
extracting new meanings of linguistic units from texts to replenish the dictionary entries of 
bilingual dictionaries. The second approach relates to medical science and focuses on 
discovering new meanings of terms to update a disease's description in the form of its 
terminological portrait. The portrait includes definitions of terms with reflecting their dynamics 
over time, relationships between terms, contexts of their use, and links to sources of contexts. 
These approaches are compared in the following positions: the problem for the solution of which 
new knowledge is discovered, the purpose of its discovery, sources of concepts of new 
knowledge, the standard, comparison with which uses as the criterion of concepts' novelty, 
concept-source linkages, and concept dynamics. The purpose of the paper is to describe the 
outcomes of the comparative analysis of the approaches. It is proposed to position analysis 
outcomes as initial data for creating the conception of a human-artificial intelligence system for 
goal-oriented discovery of new knowledge from big data which is applicable in different subject 
domains.}

\KWE{new knowledge generation; discovering knowledge from texts; artificial intelligence; 
human-artificial intelligence system}



\DOI{10.14357/19922264210113}

\vspace*{-14pt}

\Ack

\vspace*{-2pt}

\noindent
The reported study was funded by RFBR, project number 20-012-00166, and
 by RFBR and NSFC, project number 21-57-53018.

\vspace*{5pt}

  \begin{multicols}{2}

\renewcommand{\bibname}{\protect\rmfamily References}
%\renewcommand{\bibname}{\large\protect\rm References}

{\small\frenchspacing
 {%\baselineskip=10.8pt
 \addcontentsline{toc}{section}{References}
 \begin{thebibliography}{99}
 
 \vspace*{-3pt}
 
\bibitem{1-z-1}
\Aue{Dobrovol'skiy, D.\,O.} 2020. Nemetskie modal'nye glagoly 
v~parallel'nom korpuse i~zadachi dvuyazychnoy leksikografii 
[German modal verbs in a~parallel corpus and bilingual 
lexicography tasks]. \textit{Germanskie yazyki: tekst, korpus, perevod} [German languages: 
Text, corpus, translation]. 103--116.
\bibitem{2-z-1}
\Aue{Dobrovol'skiy, D.\,O., and Anna A.~Zaliznyak.} 2020. Russkie konstruktsii s~potentsial'no 
modal'nym znacheniem po dannym parallel'nykh korpusov [Russian constructions with 
potentially modal meanings: An analysis based on parallel corpus data]. \textit{Trudy Instituta 
russkogo yazyka im.\ V.\,V.~Vinogradova} [V.\,V.~Vinogradov Russian Language Institute 
Proceedings]. 3:35--49.
\bibitem{3-z-1}
\Aue{Zatsman, I.} 2019. Finding and filling lacunas in knowledge systems. \textit{20th 
European Conference on Knowledge Management Proceedings}. Reading: Academic Publishing 
International Ltd. 2:1143--1151.
\bibitem{4-z-1}
Ide, N., and J.~Pustejovsky, eds. 2017. \textit{Handbook of linguistic annotation}. Dordrecht, 
The Netherlands: Springer Science\;+\;Business Media. 1468~p.
\bibitem{5-z-1}
\Aue{Zatsman, I.\,M.} 2019. Tselenapravlennoe razvitie sistem lingvisticheskikh znaniy: 
vyyavlenie i~zapolnenie lakun [Goal-oriented development of linguistic knowledge systems: 
Identifying and filling of lacunae]. \textit{Informatika i~ee Primeneniya~--- Inform. Appl.} 
13(1):91--98.
\bibitem{6-z-1}
\Aue{Zatsman, I.} 2020.Problemno-orientirovannaya ve\-ri\-fi\-ka\-tsiya polnoty temporal'nykh 
ontologiy i~zapolnenie po\-nya\-tiy\-nykh lakun [Problem-oriented verifying the completeness of 
temporal ontologies and filling conceptual lacunas]. \textit{Informatika i~ee Primeneniya~--- 
Inform. Appl.} 14(3):119--128.
\bibitem{7-z-1}
Medical dictionary of health terms. Available at: {\sf 
 https://www.health.harvard.edu/a-through-c} (accessed January~14, 2021).
\bibitem{8-z-1}
 \textit{Dorland's illustrated medical dictionary}.  2019. 33rd ed. Philadelphia, PA: Elsevier 
Saunders. 2144~p.
\bibitem{9-z-1}
MedTerms medical dictionary. Available at: {\sf  
https://\linebreak www.medicinenet.com/medterms-medical-dictionary/article.htm} (accessed January~14, 
2021).
\bibitem{10-z-1}
National library of medicine. Available at: {\sf https:// pubmed.ncbi.nlm.nih.gov} (accessed 
January~14, 2021).
\bibitem{11-z-1}
Dobrovol'skiy, D.\,O., ed. 2021 (in press). \textit{Nemetsko-russkiy slovar' aktual'noy leksiki} 
[German--Russian dictionary: Actual vocabulary]. Moscow: Leksrus.
\bibitem{12-z-1}
Parallel'nyy nemetskiy podkorpus Natsional'nogo    kor\-pu\-sa russkogo yazyka
[Parallel German subcorpus  of the National Russian corpus]. Available at: {\sf 
http://www. ruscorpora.ru/searchpara-de.html} (accessed January~14, 2021).
\bibitem{13-z-1}
\Aue{Zatsman, I.} 2020. Finding and filling lacunas in linguistic typologies. \textit{15th Forum 
(International) on Knowledge Asset Dynamics Proceedings}. Matera: Institute of Knowledge 
Asset Management. 780--793.
\bibitem{14-z-1}
\Aue{Zatsman, I.} 2018. Stadii tselenapravlennogo izvlecheniya znaniy, implitsirovannykh 
v~parallel'nykh tekstakh [Stages of goal-oriented discovery of knowledge implied in parallel 
texts]. \textit{Sistemy i~Sredstva Informatiki~--- Systems and Means of Informatics} 
28(3):175--188.
\bibitem{15-z-1}
\Aue{Zatsman, I.} 2020. Three-dimensional encoding of emerging meanings in AI-systems. 
\textit{21st European Conference on Knowledge Management Proceedings}. Reading: 
Academic Publishing International Ltd. 878--887.
\bibitem{16-z-1}
\Aue{Goncharov, A.\,A., I.\,M.~Zatsman, and M.\,G.~Kruzhkov.} 2020. Evolyutsiya 
klassifikatsiy v nadkorpusnykh ba\-zakh dannykh [Evolution of classifications in supracorpora 
databases]. \textit{Informatika i~ee Primeneniya~--- Inform. \mbox{Appl.}} 14(4):108--116.
\bibitem{17-z-1}
\textit{Mezhdunarodnaya statisticheskaya klassifikatsiya bo\-lez\-ney i~problem, svyazannykh so zdorov'em. 
Desyatyy peresmotr} [International 
statistical classification of diseases and related health problems. Tenth Revision]. 1(1).
Available\linebreak at: {\sf 
https://apps.who.int/iris/bitstream/handle/10665/\linebreak 85974/5225032680.pdf?sequence=1\&isAllowed=y} (accessed 
January~14, 2021).

\bibitem{19-z-1}
\Aue{V$\grave{\mbox{a}}$zquez, M., and A.~Oliver.} 2018. Improving term candidates' 
selection using terminological tokens. \textit{Terminology} 24(1):122--147.

\bibitem{18-z-1} %19
\Aue{Khakimova, A., X.~Yang, O.~Zolotarev, M.~Berberova, and M.~Charnine.} 2020. 
Tracking knowledge evolution based on the terminology dynamics in 4P-Medicine. \textit{Int. 
J.~Env. Res. Pub. He.} 17(20):7444. 19~p.

\bibitem{20-z-1}
\Aue{Oliver, A., and M.~V$\grave{\mbox{a}}$zquez.} 2020. TermEval 2020: Using TSR 
filtering method to improve automatic term extraction. \textit{6th  Workshop (International) on 
Computational Terminology Proceedings}. Paris: European Language Resources Association.  
106--113.
\bibitem{21-z-1}
\Aue{Jarrahi, M.\,H.} 2018. Artificial intelligence and the future of work: Human-AI symbiosis 
in organizational decision making. \textit{Bus. Horizons} 61(4):577--586.
\bibitem{22-z-1}
\Aue{Trunk, A., H.~Birkel, and E.~Hartmann.} 2020. On the current state of combining human 
and artificial intelligence for strategic organizational decision making. \textit{Bus. 
Res.} 13(3):875--919.
\bibitem{23-z-1}
\Aue{Zatsman, I.\,M.} 2019. Interfeysy tret'ego poryadka v~informatike [Third-order interfaces 
in informatics]. \textit{Informatika i~ee Primeneniya~--- Inform. Appl.} 13(3):82--89.
\bibitem{24-z-1}
\Aue{Wierzbicki, A.\,P., and Y.~Nakamori.} 2006. Basic dimensions of creative space. 
\textit{Creative space: Models of creative processes for knowledge civilization age}. Eds. 
A.\,P.~Wierzbicki and Y.~Nakamori. Berlin: Springer Verlag. 59--90. 
\end{thebibliography}

 }
 }

\end{multicols}

\vspace*{-3pt}

  \hfill{\small\textit{Received January~16, 2021}}


%\pagebreak

%\vspace*{-8pt}     

\Contrl

\noindent
\textbf{Zatsman Igor M.} (b.\ 1952)~--- Doctor of Science in technology, Head of Department, 
Institute of Informatics Problems, Federal Research Center ``Computer Science and Control'' of 
the Russian Academy of Sciences, 44-2~Vavilov Str., Moscow 119333, Russian Federation; 
\mbox{izatsman@yandex.ru}

\label{end\stat}

\renewcommand{\bibname}{\protect\rm Литература}