\def\stat{stup-br}

\def\tit{АРХИТЕКТУРА РАСПРЕДЕЛЕННОГО РЕШЕНИЯ ЗАДАЧ АНАЛИЗА ДАННЫХ 
В~ОБЛАСТИ НЕЙРОФИЗИОЛОГИИ$^*$}

\def\titkol{Архитектура распределенного решения задач анализа данных 
в~области нейрофизиологии}

\def\aut{Д.\,О.~Брюхов$^1$, С.\,А.~Ступников$^2$, Д.\,Ю.~Ковалёв$^3$, 
И.\,А.~Шанин$^4$}

\def\autkol{Д.\,О.~Брюхов, С.\,А.~Ступников, Д.\,Ю.~Ковалёв, 
И.\,А.~Шанин}

\titel{\tit}{\aut}{\autkol}{\titkol}

\index{Брюхов Д.\,О.}
\index{Ступников С.\,А.}
\index{Ковалёв Д.\,Ю.}
\index{Шанин И.\,А.}
\index{Briukhov D.\,O.}
\index{Stupnikov S.\,A.}
\index{Kovalev D.\,Yu.}
\index{Shanin I.\,A.}

{\renewcommand{\thefootnote}{\fnsymbol{footnote}} \footnotetext[1]
{Работа выполнена при финансовой поддержке РФФИ (проект 18-29-22096).
  }}

\renewcommand{\thefootnote}{\arabic{footnote}}
\footnotetext[1]{Институт проблем информатики Федерального исследовательского центра <<Информатика и~управление>> Российской 
академии наук, \mbox{dbriukhov@ipiran.ru}}
\footnotetext[2]{Институт проблем информатики Федерального исследовательского центра <<Информатика и~управление>> Российской 
академии наук, \mbox{sstupnikov@ipiran.ru}}
\footnotetext[3]{Институт проблем информатики Федерального исследовательского центра <<Информатика и~управление>> Российской 
академии наук, \mbox{dm.kovalev@gmail.com}}
\footnotetext[4]{Институт проблем информатики Федерального исследовательского центра <<Информатика и~управление>> Российской 
академии наук, \mbox{ivan.shanin@gmail.com}}



%\vspace*{-12pt}




\Abst{С ростом объема и~разнообразия нейрофизиологических данных происходит и~рост 
интереса к~применению методов информатики, таких как статистический анализ, 
машинное обучение, нейронные сети, для анализа этих данных. Появляется потребность 
в~создании инфраструктур, обеспечивающих как хранение большого объема данных 
в~области нейрофизиологии, так и~их распределенную обработку и~анализ. В~данной 
статье предлагается архитектура средств решения задач на основе технологий 
распределенного хранения и~анализа больших данных Hadoop и~высокопроизводительных 
вычислений с~применением графических ускорителей.}

\KW{нейрофизиология; нейроинформатика; интенсивное использование данных; 
инфраструктуры решения задач; анализ данных}

\DOI{10.14357/19922264210111}

\vspace*{8pt}

\vskip 10pt plus 9pt minus 6pt

\thispagestyle{headings}

\begin{multicols}{2}

\label{st\stat}

\section{Введение}

%\vspace*{-2pt}

    Понимание работы человеческого мозга становится одной из основных 
научных задач в~на\-сто\-ящее время. Рас\-тет количество и~качество 
оборудования в~области нейрофизиологии. В~связи с~этим растет и~объем 
получаемых данных. 

Традиционные методы анализа данных не справляются 
с~обработкой большого объема нейрофизиологических данных. В~связи 
с~этим происходит рост интереса к~использованию методов информатики, 
таких как статистический анализ, машинное обуче\-ние (и,~в~част\-ности, 
нейронные сети глубокого обуче\-ния). Появляется потребность в~создании 
инфраструктур, позволяющих как хранить большой объем данных, так 
и~обрабатывать и~анализировать эти данные.
    
    Первоначально такие инфраструктуры пред\-став\-ля\-ли собой веб-ка\-та\-ло\-ги 
данных и~программных средств, где ученые могли делиться своими данными с~другими учеными, 
находить и~использовать данные и~программы в~своих 
исследованиях. В~последнее время стали развиваться инфраструктуры, 
предоставляющие исследователям средства высокопроизводительных 
вычислений для обработки и~анализа данных в~области нейрофизиологии. 

    
    В данной статье предлагается архитектура распределенного решения 
задач в~области нейрофизиологии, объединяющая технологии 
распределенного хранения и~анализа больших данных Hadoop и~технологии 
высокопроизводительных вычислений с~применением графических 
ускорителей. Hadoop позволяет хранить данные большого объема (например, 
нейроизображения) и~выполнять распределенную параллельную обработку 
этих данных на вычислительном кластере. 

Ряд методов, включающих 
сложные вычисления, требует использования графических ускорителей для 
эффективного выполнения (например, построение моделей в~методах 
машинного обучения). Такие вычисления выполняются на отдельных 
серверах с~мощными графическими ускорителями.

\begin{table*}[b]\small
\begin{center}
\tabcolsep=3pt
\begin{tabular}{|l|c|c|c|c|c|}
\multicolumn{6}{c}{Сравнение инфраструктур решения задач в~области 
нейрофизиологии}\\
\multicolumn{6}{c}{\ }\\[-6pt]
\hline
\multicolumn{1}{|c|}{Инфраструктура}&XNAT&NTRC&MIRF&HBP ICT&
\tabcolsep=0pt\begin{tabular}{c}Предлагаемая\\ архитектура\end{tabular}\\
\hline
Вид&\tabcolsep=0pt\begin{tabular}{c}Веб-\\приложение\end{tabular}&
\tabcolsep=0pt\begin{tabular}{c}Набор\\ программ\end{tabular}&
\tabcolsep=0pt\begin{tabular}{c}Вычислитель-\\ная среда\end{tabular}&
\tabcolsep=0pt\begin{tabular}{c}Вычислитель-\\ная среда\end{tabular}&
\tabcolsep=0pt\begin{tabular}{c}Вычислитель-\\ная среда\end{tabular}\\
\hline
Распределенные технологии&&Облако&&Облако, HPC&Hadoop, HPC\\
\hline
Способ хранения данных&Файлы&Файлы&Файлы&Файлы&
\tabcolsep=0pt\begin{tabular}{c}Файлы,\\ базы данных\end{tabular}\\
\hline
Распределенное хранение&&&&&Да (HDFS)\\
\hline
Распределенные вычисления&&&&Да&Да\\
\hline
Графические ускорители&Да&Да&Да&Да&Да\\
\hline
Набор встроенных библиотек&Да&Да&Да&Да&Да\\
\hline
Атласы&&&&Да&Да\\
\hline
Обработка изображений&Да&Да&Да&Да&Да\\
\hline
Визуализация изображений&Да&Да&Да&Да&\\
\hline
Анализ изображений&Да&Да&Да&Да&Да\\
\hline
Методы машинного обучения&Да&Да&Да&Да&Да\\
\hline
Программная архитектура&
\tabcolsep=0pt\begin{tabular}{c}Отдельные\\ программы\end{tabular}&
\tabcolsep=0pt\begin{tabular}{c}Отдельные\\ программы\end{tabular}&Конвейеры&
\tabcolsep=0pt\begin{tabular}{c}Программа\\ организуется\\ как сервис\\ (SaaS)\end{tabular}&Модули\\
\hline
\tabcolsep=0pt\begin{tabular}{l}Взаимодействие с~другими\\ программами\end{tabular}&
\tabcolsep=0pt\begin{tabular}{c}Расширения\\ (плагины)\end{tabular}&
\tabcolsep=0pt\begin{tabular}{c}Общие\\ входные/выходные\\ данные\end{tabular}&
\tabcolsep=0pt\begin{tabular}{c}Вызов 
через\\ API программы\end{tabular}&
\tabcolsep=0pt\begin{tabular}{c}Вызов как\\ сервис (SaaS)\end{tabular}&
\tabcolsep=0pt\begin{tabular}{c}Вызов\\ через API\\ программы\end{tabular}\\
\hline
Поддержка баз данных&&&&&Да\\
\hline
\end{tabular}
\end{center}
\end{table*} 
    
    В рамках данной статьи был проведен анализ существующих 
инфраструктур в~области нейрофизиологии, предложена архитектура средств 
распределенной обработки нейрофизиологических данных, приведен пример 
использования этой архитектуры при решении задачи поиска значимых 
различий нелинейной функциональной связности головного мозга для 
мужчин и~женщин.
{\looseness=1

}
    
\section{Инфраструктуры решения задач в~области 
нейрофизиологии}

    В последнее время стали активно разрабатываться инфраструктуры для 
запуска различных программ обработки и~анализа нейрофизиологических 
данных в~одной среде, позволяющие встраивать существующие программы 
и~библиотеки в~создаваемые пользователями программы, решающие 
конкретные задачи. Ниже рассмотрены основные современные 
инфраструктуры для решения задач в~области нейрофизиологии.
    
    XNAT~[1]~--- это открытая информационная платформа для работы 
с~нейроизображениями, разработанная исследовательской группой по 
нейроинформатике в~Вашингтонском университете. Она облегчает общие 
задачи управления, обеспечения производительности и~качества обработки 
нейроизображений и~связанных данных. XNAT позволяет программировать 
сложные потоки работ с~несколькими уровнями автоматизации.
    
    NeuroImaging Tools \& Resources Collaboratory (NITRC)~\cite{2-st}~---  
это бесплатный веб-ре\-сурс, который предлагает информацию о постоянно 
расширяющемся наборе программного обеспечения и~данных для 
нейроинформатики. Он состоит из трех компонентов: реестра ресурсов, 
репозитория изображений и~вычислительной среды~--- виртуальной 
машины, содержащей предустановленный набор программных средств для 
работы с~нейроизображениями.
    
    Программная платформа MIRF (Medical Images Research 
Framework)~\cite{4-st}~---  это платформа с~открытым исходным кодом для 
быстрой разработки приложений для обработки медицинских изображений. 
Основным сценарием использования MIRF является создание пользователем 
конвейеров~--- последовательности обработчиков исходных данных. 
    
    В рамках проекта Европейского Союза Human Brain Project 
(HBP)~\cite{5-st} разрабатывается архитектура технологии хранения 
информации и~коммуникации с~исследователями ICT (Information and 
Communication Technology)~\cite{3-st}. Основные компоненты ICT~--- 
платформа нейроинформатики NIP и~высокопроизводительная аналитическая и~вы\-чис\-ли\-тель\-ная 
платформа HPAC. Компонент \mbox{COLLAB} обеспечивает 
взаимодействие как между компонентами ICT, так и~взаимодействие 
исследователей с~ICT.
    
    В таблице приведено сравнение существующих инфраструктур решения 
задач в~области нейрофизиологии.


    
    Можно отметить, что развивающиеся в~настоящее время 
инфраструктуры ориентированы на предостав\-ле\-ние возможности повторного 
использования написанных программ и~библиотек для решения новых задач в~области нейрофизиологии. К~основным недостаткам рассмотренных 
инфраструктур можно отнести отсутствие поддержки распределенного 
хранения большого объема данных и~распределенной параллельной 
обработки этих данных. Предлагаемая в~данной работе (см.\ разд.~3) 
архитектура решения задач нацелена на преодоление этих недостатков. Она 
также нацелена на повторное использование уже реализованных методов 
анализа нейрофизиологических данных. В~архитектуре предусмотрено 
использование различных баз данных для эффективного хранения часто 
используемых типов данных, например временн$\acute{\mbox{ы}}$х рядов с~данными по 
регионам мозга.

\section{Программная архитектура распределенного решения задач~анализа данных в~области~нейрофизиологии}

    Задача исследования работы головного мозга становится все более 
актуальной. Растет объем исследований головного мозга, развиваются 
методы анализа получаемых данных, включая ней\-ро\-изоб\-ра\-же\-ния. 
В~работе~\cite{6-st} было продемонстрировано многообразие форматов 
представления ней\-ро\-изоб\-ра\-же\-ний, программных средств обработки 
и~анализа ней\-ро\-изоб\-ра\-же\-ний. Активно развиваются методы глубокого 
обучения, появляется все больше исследований, использующих нейронные 
сети для анализа ней\-ро\-изоб\-ра\-же\-ний. В~связи с~этим встает вопрос 
о~создании архитектуры, поддерживающей хранение больших объемов 
данных и~методы распределенной обработки этих данных. 
    
    Предлагаемая в~рамках данной работы архитектура основана на 
технологиях распределенного хранения и~анализа больших данных Hadoop 
и~высокопроизводительных вычислений с~применением графических 
ускорителей. В~качестве платформы распределенных вычислений в~Hadoop 
был выбран фреймворк Spark. В~зависимости от конкретной задачи анализ 
нейрофизиологических данных может выполняться как на одной из этих 
платформ, так и~совместно на обеих платформах. 
    
    Spark обычно применяется в~случаях, когда требуется параллельная 
обработка большого чис\-ла файлов с~данными в~об\-ласти нейрофизиологии. 
При распределенном анализе рекомендуется использовать методы 
статистического анализа, в~част\-ности регрессионный анализ, проверку 
статистических гипотез, детектирование аномалий. На Spark также можно 
реализовывать методы машинного обучения для небольших моделей 
с~малым числом параметров (признаков). Если для анализа данных 
применяются сложные методы, тре\-бу\-ющие большого объема вы\-чис\-ле\-ний, то 
эти методы рекомендуется выполнять на графических ускорителях. К~таким 
методам относятся методы глубокого обуче\-ния, включая сверточные 
и~рекуррентные нейронные сети, автокодировщики и~другие эффективные 
методы машинного обучения, такие как градиентный бустинг и~метод 
опорных векторов. Для решения ряда задач может применяться комбинация 
этих подходов, например модели в~методах машинного обучения или 
в~нейронных сетях могут строиться на данных из обучающей выборки на 
отдельном сервере с~графическими картами, а~дальнейшее применение этих 
моделей и~оценка качества построенных моделей осуществляется в~Spark.
    
    Программы анализа данных могут быть реализованы на любых языках 
программирования, поддерживающих Spark, например Python, Java, Scala, 
Julia. В реализации предлагаемой в~этой статье архитектуры распределенного 
решения задач будет использован язык Python, также будут использованы 
библиотеки методов обработки и~анализа данных, написанные на Python.
    
    На рис.~1 представлена схема архитектуры решения задач в~области 
нейрофизиологии. В~архитектуре выделяются три уровня: уровень данных, 
уровень поддержки приложений и~уровень приложений.
    
    На \textit{уровне данных} обеспечивается надежное хранение большого 
объема нейрофизиологических данных, включая исходные данные 
(например, нейро\-изоб\-ра\-же\-ния), промежуточные данные (обработанные 
данные, сгенерированные модели) и~результаты работы модулей анализа 
данных (предсказания, результаты проверки гипотез, результаты 
классификации данных). Данные могут храниться как в~виде файлов 
в~файловой системе, так и~в~базах данных. Исходные данные загружаются 
и~хранятся в~распределенной файловой системе HDFS (Hadoop distributed file system). HDFS автоматически 
создает несколько копий данных, что обеспечивает надежность их хранения 
на случай отказа оборудования. Для хранения данных, представленных 
в~виде многомерных временн$\acute{\mbox{ы}}$х рядов (например, данные, извлеченные по 
регионам мозга) используется база данных OpenTSDB. 
    


    Часть данных, необходимых для выполнения программ 
с~использованием графических ускорителей, хранится в~локальной файловой 
сис\-те\-ме этого сервера. На \textit{уровне поддержки приложений} 
уста\-нав\-ли\-ва\-ют\-ся программы и~биб\-лио\-те\-ки, ис\-поль\-зу\-емые\linebreak\vspace*{-12pt}

\pagebreak

\end{multicols}

\begin{figure*} %fig1
\vspace*{1pt}
 \begin{center}
 \mbox{%
 \epsfxsize=161.936mm  
 \epsfbox{stu-1.eps}
 }
  \end{center}
\vspace*{-6pt}
\Caption{Общая архитектура распределенного решения задач анализа данных в~об\-ласти 
нейрофизиологии}
\end{figure*}

\begin{multicols}{2}

\noindent
 для обработки 
и~анализа нейрофизиологических данных. Разработанные программы или их 
части могут быть оформлены в~виде отдельных биб\-лио\-тек и~затем повторно 
использованы при разработке новых программных средств. Такие программы 
формируют \textit{пользовательские библиотеки}. Примерами таких модулей 
могут служить: извлечение регионов мозга по заданному атласу, выявление 
и~классификация артефактов в~электроэнцефалограмме (ЭЭГ).
    
    \textit{Библиотеки обработки данных} включают биб\-лио\-те\-ки по работе 
(чтению и~записи) с~различными форматами нейрофизиологических 
изображений и~простыми операциями над ними. \mbox{Примерами} таких биб\-лио\-тек 
служат NiPy и~Nibabel для языка Python. \textit{Библиотеки работы 
с~атласами} используются для извлечения регионов мозга на основе\linebreak 
существующих атласов. Для работы с~атласами можно использовать как 
специализированные\linebreak биб\-лио\-те\-ки (например, AtlasReader), так и~\mbox{общие} 
биб\-лио\-те\-ки по анализу нейрофизиологических {данных}, под\-дер\-жи\-ва\-ющих 
работу с~атласами (например, Nilearn). \textit{Библиотеки доступа к~данным} 
используются для работы с~базами данных, в~которых хранятся 
промежуточные данные и~результаты работы программ. Например, 
библиотека \mbox{opentsdb-py} используется для организации доступа из 
программы, написанной на языке Python, к~базе данных OpenTSDB. 
Библиотека \mbox{happybase} используется для связи с~базой данных HBase, для 
хранения вы\-чис\-лен\-ных нелинейных функций за\-ви\-си\-мости одних регионов 
мозга от других. \textit{Модуль копирования данных} предназначен для 
поддержки копирования данных между распределенной файловой сис\-те\-мы 
HDFS и~локальной файловой сис\-те\-мой сервера с~графическим ускорителем.
    
    \textit{Библиотеки статистических методов, методов машинного 
обучения и~нейронных сетей} используются для выполнения анализа 
нейрофизиологических данных. Примерами таких биб\-лио\-тек могут служить: 
Nitime (для анализа вре\-мен\-н$\acute{\mbox{ы}}$х рядов), \mbox{Nilearn} (для статистического 
исследования данных нейровизуализации), MNE-Python (для статистического 
анализа и~методов машинного обуче\-ния над данными маг\-нит\-но-ре\-зо\-нанс\-ной 
томографии (МРТ) и~ЭЭГ), \mbox{tsfresh} 
(для анализа временн$\acute{\mbox{ы}}$х рядов), scikit-learn (для методов машинного 
обучения и~нейронных сетей), \mbox{Keras} (для нейронных сетей), \mbox{gplearn} (для 
методов генетического программирования), \mbox{XGBoost} (для метода 
градиентного бус\-тин\-га на деревьях), \mbox{CatBoost} (для метода градиентного 
бус\-тин\-га на деревьях). Все биб\-лио\-те\-ки, используемые в~Spark, 
устанавливаются на каж\-дый узел клас\-тера.
    
    \textit{Уровень приложений} содержит пользовательские программы для 
обработки и~анализа нейрофизиологических данных. Программы могут быть 
реализованы либо только на Spark, либо только на графических ускорителях, 
либо в~общей архитектуре. Программы имеют модульную структуру. 
Цент\-раль\-ным является \textit{управляющий модуль}, который\linebreak отвечает за 
организацию процесса обработки и~анализа данных, взаимодействуя 
с~остальными модулями. Данные могут быть загружены в~систему до начала 
работы программы или может быть использован \textit{модуль загрузки 
данных} для динамической загрузки данных из внешних ресурсов.
    
    \textit{Модуль предобработки данных} производит предварительную 
обработку данных, включая очистку данных, преобразование к~формату, 
используемому при дальнейшем анализе этих данных, разделение данных по 
заданному критерию (например, для разных групп субъектов), отбор 
значимых признаков. Особенно следует выделить \textit{подмодуль 
выявления регионов}, осуществляющий преобразование исходных 4-мер\-ных 
фМРТ-изоб\-ра\-же\-ний (фМРТ~--- функциональная  
МРТ) в~набор многомерных временн$\acute{\mbox{ы}}$х 
рядов для каждого региона мозга. Регионы мозга определяются на основании 
применения атласов регионов (например, Harvard--Oxford, Automated 
Anatomical Labeling). Полученные временн$\acute{\mbox{ы}}$е ряды сохраняются в~базе 
данных OpenTSDB. Подмодуль \textit{извлечения признаков} выявляет 
значимые признаки в~исходных данных, включая статистики для описания 
временных рядов. Отобранные признаки подаются на вход методов анализа 
данных, включая методы машинного обучения и~нейронные сети.
    
    \textit{Модуль анализа данных} использует методы информатики, такие 
как статистический анализ, машинное обучение, нейронные сети, для анализа 
нейрофизиологических данных. Модуль может выполняться как на 
платформе Spark для методов параллельной обработки входных данных 
(например, параллельная обработка данных по разным объектам 
исследования), так и~на сервере с~графическим ускорителем для 
вычислительно сложных методов, хорошо распараллеливаемых на 
графических ускорителях (например, нейронных сетей). Связь между 
модулями, работающими на Spark и~на сервере с~графическим ускорителем, 
осуществляется посредством удаленного вызова скрипта на сервере 
с~графическим ускорителем, запускающего необходимый программный 
модуль анализа данных. Программа может содержать несколько модулей 
предобработки и~анализа данных, например, для сравнения результатов 
применения различных методов или использования ансамбля методов для 
получения более точных результатов. 
    
    В \textit{модуле оценки качества} происходит вычисление известных 
метрик, таких как доля правильно классифицированных объектов, точ\-ность 
и~полнота, для построенных моделей.

\section{Пример задачи анализа данных в~области  
нейрофизиологии}

    В качестве примера задачи, реализуемой в~рамках представленной 
архитектуры, рассмотрим задачу поиска значимых различий нелинейной 
функциональной связности головного мозга для мужчин и~женщин 
в~состоянии покоя. Изучение гендерных различий в~работе мозга~--- одна из 
важных областей когнитивного анализа мозга. Понимание роли гендерных 
эффектов может помочь в~развитии специализированных методов лечения, 
раз\-ли\-ча\-ющих\-ся для мужчин и~женщин. В~работе~\cite{7-st} было 
рассмотрено применение моделирования нелинейной функциональной 
связности головного мозга для поиска различий между мужчинами 
и~женщинами. 

\begin{figure*} %fig2
\vspace*{1pt}
 \begin{center}
 \mbox{%
 \epsfxsize=149.855mm  
 \epsfbox{stu-2.eps}
 }
  \end{center}
\vspace*{-9pt}
\Caption{Архитектура средств поиска значимых различий нелинейной функциональной 
связности головного мозга для мужчин и~женщин}
\end{figure*}

    
    На рис.~2 представлена предлагаемая в~работе программная архитектура 
средств реализации этой задачи, соответствующая общей архитектуре 
(см.\ рис.~1). Архитектура включает следующие основные компоненты, 
соответствующие этапам решения задачи.
    
    \textit{Извлечение регионов головного мозга}. На данном этапе 
происходит преобразование 4-мерного изоб\-ра\-же\-ния головного мозга 
в~двумерный массив, где одним из измерений выступают регионы головного 
мозга, а другим~--- время. Извлечение регионов происходит 
с~использованием вероятностного атласа Harvard--Oxford~\cite{8-st}. 
Полученные данные сохраняются в~виде временн$\acute{\mbox{ы}}$х рядов в~базе данных 
OpenTSDB.
    
    \textit{Построение аналитических функций с~применением метода 
генетического программирования}~\cite{9-st}. Генетическое 
программирование~--- это метод, при помощи которого можно восстановить 
нелинейную функциональную связь. Метод основан на идее биологической 
эволюции. Существенный недостаток этого метода заключается в~том, что 
его вычислительная стоимость растет экспоненциально с~увеличением 
размерности, поэтому данный этап выполняется на компьютере 
с~графическим ускорителем. Входные данные берутся из базы данных 
OpenTSDB. Вычисленные нелинейные функции зависимости одних регионов 
мозга от других как для мужчин, так и~для женщин сохраняются в~базу 
данных HBase.
    
    \textit{Проведение статистического анализа для сравнения полученных 
функций} и~вывод о схожести или различии того или иного региона 
головного мозга для мужчин и~женщин. Включает в~себя по\-стро\-ение 
предсказаний с~использованием функций, построенных отдельно для мужчин 
и~отдельно для женщин, и~проверку гипотезы о~том, что ошибка между 
предсказаниями функций рав\-на нулю. Входные данные для выполнения 
этого этапа берутся из базы данных HBase.
    
    На основе этой архитектуры был разработан прототип поиска значимых 
различий нелинейной функциональной связности головного мозга для 
мужчин и~женщин. Основная часть прототипа была реализована на языке 
Python в~среде Spark с~применением библиотеки pyspark, а~метод 
генетического программирования был реализован на компьютере 
с~графическим ускорителем. Для извлечения регионов мозга на основе 
атласа Harvard-Oxford использовалась библиотека nileran. Для обработки 
полученных данных по регионам мозга использовались библиотеки numpy 
и~pandas. Полученные данные в~виде временн$\acute{\mbox{ы}}$х рядов сохранялись в~базу 
данных OpenTSDB при помощи библиотеки opentsdb-py. При анализе этих 
временн$\acute{\mbox{ы}}$х рядов с~помощью методов генетического программирования 
использовались библиотеки gplearn и~sklearn. Данные по полученным 
в~результате анализа аналитическим функциям записывались в~базу данных 
HBase при помощи библиотеки happybase. Разработанный прототип был 
опробован на данных из проекта Human Connectome Project  
(HCP)~\cite{10-st}.

\section{Заключение}

    Рост объема данных в~области нейрофизиологии приводит к~тому, что 
использование одного компьютера уже недостаточно для их обработки 
и~анализа. Анализ существующих систем выявляет потребность создания 
архитектуры, позволяющей как хранить большой объем данных, так 
и~обрабатывать\linebreak эти данные с~применением распределенных технологий. 
В~статье предложена про\-грам\-мная архитектура распределенного решения 
задач в~области нейрофизиологии над данными больших объемов, 
\mbox{объединяющая} технологии распределенного хранения и~анализа больших 
данных Hadoop и~технологии высокопроизводительных вычислений 
с~применением графических ускорителей. Хранение данных осуществляется 
в~распределенной файловой системе HDFS. Параллельная распределенная 
обработка и~анализ нейрофизиологических данных выполняются на 
Hadoop/Spark-клас\-те\-ре. Выполнение вычислительно сложных алгоритмов, 
требующих графических ускорителей, осуществляется на отдельных 
серверах, содержащих графические карты.
    
    Предложенная архитектура была использована при решении задач по 
обработке нейрофизиологических изображений, включая анализ 
функциональной связности отделов мозга на основе данных фМРТ состояния 
покоя, выявления шаблонов гендерных различий в~работе мозга на основе 
данных фМРТ состояния покоя, определения активности человека 
с~использованием фМРТ действия, обнаружения артефактов в~ЭЭГ.
    
{\small\frenchspacing
{%\baselineskip=10.8pt
%\addcontentsline{toc}{section}{References}
\begin{thebibliography}{99}
  \bibitem{1-st}
  \Au{Marcus D., Olsen T.\,R., Ramaratnam~M., Buckner~R.\,L.} The extensible neuroimaging 
archive toolkit (XNAT): An informatics platform for managing, exploring, and sharing 
neuroimaging data~// Neuroinformatics, 2007. Vol.~5. P.~11--34.
  \bibitem{2-st}
  NITRC. {\sf https://www.nitrc.org}.
  
  \bibitem{4-st} %3
  \Au{Musatian S., Lomakin~A., Chizhova~A.} Medical images research framework~// CEUR 
Workshop Procee., 2019. Vol.~2372. P.~60--66. 
  \bibitem{5-st} %4
  Human brain project. {\sf https://www.humanbrainproject. eu}.
  \bibitem{3-st} %5
  \Au{Amunts K., Ebell~C., Muller~J., Telefont~M., Knoll~A., Lippert~L.} The human brain 
project: Creating a~European research infrastructure to decode the human brain~// Neuron, 
2016. Vol.~92. P.~574--581.
  \bibitem{6-st}
  \Au{Брюхов Д.\,О., Ступников~С.\,А., Ковалёв~Д.\,Ю., Шанин~И.\,А.} 
Нейрофизиология как предметная область для решения задач с~интенсивным 
использованием данных~// Информатика и~её применения, 2020. Т.~14. Вып.~1. С.~40--47.
  \bibitem{7-st}
  \Au{Kovalev D., Tirikov~E., Sergeev~D., Ponomareva~N.\,V.} Methods and tools for 
analyzing human brain signals based on functional magnetic resonance imaging data~// CEUR 
Workshop Procee., 2020. Vol.~2790. P.~214--229.
  \bibitem{8-st}
  \Au{Desikan R.\,S., Segonne~F., Fischl~B., \textit{et al.}} An automated labeling system for 
subdividing the human cerebral cortex on MRI scans into gyral based regions of interest~// 
NeuroImage, 2006. Vol.~31. No.\,3. P.~968--980.
  \bibitem{9-st}
  \Au{Allgaier N., Banaschewski~T., Barker~G., \textit{et al.}} Nonlinear functional mapping 
of the human brain~// arXiv.org, 2015. arXiv:1510.03765. 21~p.
  \bibitem{10-st}
  \Au{Elam J.\,S., Van Essen~D.} Human connectome project~// Encyclopedia of 
computational neuroscience~/ Eds. D.~Jaeger, R.~Jung.~--- New York, NY, USA: Springer, 
2013. doi: 10.1007/978-1-4614-7320-6\_592-1.
\end{thebibliography}

}
}

\end{multicols}

\vspace*{-6pt}

\hfill{\small\textit{Поступила в~редакцию 27.12.2020}}

\vspace*{8pt}

%\pagebreak

%\newpage

%\vspace*{-28pt}

\hrule

\vspace*{2pt}

\hrule

%\vspace*{-2pt}

\def\tit{AN ARCHITECTURE FOR~DISTRIBUTED DATA ANALYSIS PROBLEM 
SOLVING IN~NEUROPHYSIOLOGY}

\def\titkol{An architecture for~distributed data analysis problem 
solving in~neurophysiology}

\def\aut{D.\,O.~Briukhov, S.\,A.~Stupnikov, D.\,Yu.~Kovalev, and~I.\,A.~Shanin}

\def\autkol{D.\,O.~Briukhov, S.\,A.~Stupnikov, D.\,Yu.~Kovalev, and~I.\,A.~Shanin}

\titel{\tit}{\aut}{\autkol}{\titkol}

\vspace*{-11pt}


\noindent
Institute of Informatics Problems, Federal Research Center ``Computer Science and
Control'' of the Russian Academy of Sciences, 44-2~Vavilov Str., Moscow 119333,
Russian Federation

\def\leftfootline{\small{\textbf{\thepage}
\hfill INFORMATIKA I EE PRIMENENIYA~--- INFORMATICS AND
APPLICATIONS\ \ \ 2021\ \ \ volume~15\ \ \ issue\ 1}
}%
\def\rightfootline{\small{INFORMATIKA I EE PRIMENENIYA~---
INFORMATICS AND APPLICATIONS\ \ \ 2021\ \ \ volume~15\ \ \ issue\ 1
\hfill \textbf{\thepage}}}

\vspace*{3pt}
  

\Abste{The growth of volume and variety of data in the field of neurophysiology increases the 
need of the application of computer science methods such as statistical analysis, machine 
learning, and neural networks for the data analysis. Infrastructures providing storage of a large 
volume of data in neurophysiology as well as data distributed processing and analysis are 
required. This article proposes a software architecture for the problem solving based on the 
Hadoop distributed storage and analysis framework and GPU-assisted high-performance 
computing technologies.}

\KWE{neurophysiology; neurophysiological resources; neuroinformatics; data intensive 
research; problem solving infrastructure; analysis of neurophysiological data}



\DOI{10.14357/19922264210111}

\vspace*{-18pt}

\Ack

\vspace*{-4pt}

\noindent
This research was financially supported by the Russian Foundation for Basic Research (project 
18-29-22096).

\vspace*{3pt}

  \begin{multicols}{2}

\renewcommand{\bibname}{\protect\rmfamily References}
%\renewcommand{\bibname}{\large\protect\rm References}

{\small\frenchspacing
 {%\baselineskip=10.8pt
 \addcontentsline{toc}{section}{References}
 \begin{thebibliography}{99}
 
 \vspace*{-3pt}
 
\bibitem{1-st-1}
\Aue{Marcus, D., T.\,R.~Olsen, M.~Ramaratnam, and R.\,L.~Buckner.} 2007. The 
extensible neuroimaging archive toolkit (XNAT): An informatics platform for 
managing, exploring, and sharing neuroimaging data. \textit{Neuroinformatics} 
5:11--34.
\bibitem{2-st-1}
NITRC. Available at: {\sf https://www.nitrc.org/} (accessed 
January~14, 2021)

\bibitem{4-st-1} %3
\Aue{Musatian, S., A.~Lomakin, and A.~Chizhova.} 2019. Medical images 
research framework. \textit{CEUR Workshop Procee.} 2372:60--66. 
\bibitem{5-st-1} %4
Human brain project. Available at: {\sf 
https://www.\linebreak humanbrainproject.eu} (accessed January~14, 2021).

\bibitem{3-st-1} %5
\Aue{Amunts, K., C.~Ebell, J.~Muller, M.~Telefont, A.~Knoll, and L.~Lippert.} 
2016. The human brain project: Creating a~European research infrastructure to 
decode the human brain. \textit{Neuron} 92:574--581.

\bibitem{6-st-1}
\Aue{Bryukhov, D.\,O., S.\,A.~Stupnikov, D.\,Yu.~Kovalev, and I.\,A.~Shanin}. 
2020. Neyrofiziologiya kak predmetnaya oblast' dlya resheniya zadach 
s~intensivnym ispol'zovaniem dannykh [Neurophysiology as a subject domain for 
data intensive problem solving]. \textit{Informatika i~ee Primeneniya~--- Inform. 
Appl.} 14(1):40--47.

\columnbreak


\bibitem{7-st-1}
\Aue{Kovalev, D., D.~Sergeev, E.~Tirikov, and N.~Ponomareva.} 2020. Methods 
and tools for analyzing human brain signals based on functional magnetic 
resonance imaging data. \textit{CEUR Workshop Procee.} 2790:214--229. 
\bibitem{8-st-1}
\Aue{Desikan, R.\,S., F.~Segonne, B.~Fischl, \textit{et al.}} 2006. An automated 
labeling system for subdividing the human cerebral cortex on MRI scans into gyral 
based regions of interest. \textit{NeuroImage} 31(3):968--980.
\bibitem{9-st-1}
\Aue{Allgaier, N., T.~Banaschewski, G.~Barker, \textit{et al.}} 2015. Nonlinear 
functional mapping of the human brain. 21~p. Available at: {\sf 
https://arxiv.org/abs/1510.03765} (accessed January~14, 2021).
\bibitem{10-st-1}
\Aue{Elam, J.\,S., and D.~Van Essen.} 2013. Human connectome project. \textit{Encyclopedia of 
computational neuroscience}. Eds. D.~Jaeger and R.~Jung. New York, NY: Springer. 
doi: 10.1007/978-1-4614-7320-6\_592-1.

\end{thebibliography}

 }
 }

\end{multicols}

\vspace*{-3pt}

  \hfill{\small\textit{Received December~27, 2020}}


%\pagebreak

%\vspace*{-8pt}     

\Contr

\noindent
\textbf{Briukhov Dmitry O.} (b.\ 1971)~--- Candidate of Science (PhD) in technology, senior 
scientist, Institute of Informatics Problems, Federal Research Center ``Computer Science and 
Control'' of the Russian Academy of Sciences, 44-2~Vavilov Str., Moscow 119333, Russian 
Federation; \mbox{dbriukhov@ipiran.ru}

\vspace*{3pt}

\noindent
\textbf{Stupnikov Sergey A.} (b.\ 1978)~--- Candidate of Science (PhD) in technology, leading 
scientist, Institute of Informatics Problems, Federal Research Center ``Computer Science and 
Control'' of the Russian Academy of Sciences, 44-2~Vavilov Str., Moscow 119333, Russian 
Federation; \mbox{sstupnikov@ipiran.ru}
    
\vspace*{3pt}

\noindent
    \textbf{Kovalev Dmitry Y.} (b.\ 1988)~--- scientist, Institute of Informatics 
Problems, Federal Research Center ``Computer Science and Control'' of the Russian 
Academy of Sciences, 44-2~Vavilov Str.,  Moscow 119333, Russian Federation; 
\mbox{dkovalev@ipiran.ru}
    
    
\vspace*{3pt}

\noindent
    \textbf{Shanin Ivan A.} (b.\ 1991)~--- junior scientist, Institute of Informatics 
Problems, Federal Research Center ``Computer Science and Control'' of the 
Russian Academy of Sciences, 44-2~Vavilov Str., Moscow 119333, Russian 
Federation; \mbox{v08shanin@gmail.com}


\label{end\stat}

\renewcommand{\bibname}{\protect\rm Литература}