\def\stat{bosov}

\def\tit{О НЕКОТОРЫХ ЧАСТНЫХ СЛУЧАЯХ В~ЗАДАЧЕ УПРАВЛЕНИЯ ВЫХОДОМ 
СТОХАСТИЧЕСКОЙ ДИФФЕРЕНЦИАЛЬНОЙ СИСТЕМЫ ПО~КВАДРАТИЧНОМУ 
КРИТЕРИЮ$^*$}

\def\titkol{О некоторых частных случаях в~задаче управления выходом 
стохастической дифференциальной системы} % по~квадратичному  критерию}

\def\aut{А.\,В.~Босов$^1$}

\def\autkol{А.\,В.~Босов}

\titel{\tit}{\aut}{\autkol}{\titkol}

\index{Босов А.\,В.}
\index{Bosov A.\,V.}

{\renewcommand{\thefootnote}{\fnsymbol{footnote}} \footnotetext[1]
{Работа выполнена при частичной поддержке РФФИ (проект 19-07-00187-A).
  }}

\renewcommand{\thefootnote}{\arabic{footnote}}
\footnotetext[1]{Институт проблем информатики 
Федерального исследовательского центра <<Информатика 
и~управление>> Российской академии наук, \mbox{ABosov@frccsc.ru}}


\vspace*{-12pt}




\Abst{Общее исследование задачи оптимального управления для диффузионного 
процесса Ито и~линейного управ\-ля\-емо\-го выхода с~квадратичным критерием 
качества выполнено в~предыдущих работах автора
(совместно с~А.\,И.~Стефановичем). Анализ имеющихся результатов позволяет 
выделить некоторые модели, носящие частный характер в~отношении общей 
постановки, но имеющие особую практическую значимость. В~данной статье 
рассмотрены две такие частные модели. Первая модель определяется 
предположением о линейном сносе в~уравнении состояния при сохранении 
нелинейной диффузии. Показано, что такая модель обеспечивает линейность 
оптимальному управлению и~отсутствие необходимости для его реализации 
решать параболическое уравнение. При этом квадратичной функции Беллмана 
в~задаче не возникает, соответствующее выражение описывается, как и~в общем 
случае, решением параболического уравнения и~сохраняет содержательную 
стохастическую интерпретацию, выражаемую формулой Фейн\-ма\-на--Ка\-ца. 
Вторая модель реализует предположение о~за\-ви\-си\-мости возмущений 
в~уравнениях состояния и~выхода. Модифицированное уравнение динамического 
программирования решается тем же способом, что и~в~общем случае, 
рассмотренном в~предыдущих работах, в~том чис\-ле в~рамках объединенной 
модели, включающей оба представленных случая. Данная модель окажется особо 
востребованной в~задачах с~неполной информацией, когда предположение 
о~наличии полной информации о~состоянии и~выходе будет заменяться 
описанием сис\-те\-мы наблюдения, в~которой выход интерпретируется как 
косвенные наблюдения за со\-сто\-яни\-ем. Кратко обсуждается чис\-лен\-ный пример, 
детально исследованный в~предыдущих работах автора (совместно с~А.\,И.~Стефановичем), так как оказывается, что он 
удовлетворяет предположению о~линейном сносе в~уравнении со\-сто\-яния 
и,~соответственно, полученные ранее приближенные решения удается уточнить.}

\KW{стохастическое дифференциальное уравнение; оптимальное управление; 
управление выходом; дифференциальные сис\-те\-мы с~мультипликативными 
и~зависимыми возмущениями}

\DOI{10.14357/19922264210102}

\vspace*{-4pt}


\vskip 10pt plus 9pt minus 6pt

\thispagestyle{headings}

\begin{multicols}{2}

\label{st\stat}


\section{Введение}

     Общая постановка задачи управления линейным выходом 
стохастической дифференциальной системы по квадратичному критерию 
качества, предложенная в~\cite{1-bos}, использует в~качестве модели для 
состояния~$y_t$ управляемой по выходу системы стохастическое 
дифференциальное уравнение Ито с~винеровским процессом~$v_t$ (как 
и~в~цитируемых работах, в~качестве непринципиального упрощения далее 
рассматривается скалярный случай):

\vspace*{2pt}

\noindent
     \begin{equation}
     dy_t=A_t(y_t)\,dt+\Sigma_t(y_t)\,dv_t\,,\enskip y_0=Y\,.
     \label{e1-bos}
     \end{equation}
     
     \vspace*{-2pt}
     
     Предположения в~отношении нелинейных функций $A_t(y)$ 
и~$\Sigma_t(y)$ ограничены лишь требованием существования единственного 
решения.
     
     Моделью связанного с~состоянием~$y_t$ линейного выхода~$z_t$ 
также является уравнение Ито с~независимым винеровским 
процессом~$w_t$, линейное по~$y_t, z_t$ и~управлению~$u_t$:

\noindent
     \begin{multline}
     dz_t=a_t y_t\,dt +b_t z_t\,dt +c_t u_t\,dt+\sigma_t\,dw_t\,,\\
     z_0=Z\,.
     \label{e2-bos}
     \end{multline}
     
     Функции $a_t$, $b_t$, $c_t$ и~$\sigma_t$ предполагаются 
ограниченными, процесс управления~$u_t$~--- допустимый с~полной 
обратной связью $u_t\hm= u_t(y,z)$~\cite{5-bos}, что обеспечивает 
существование и~единственность решения уравнения~(\ref{e2-bos}). 

Для 
оптимизации $U_0^{{T}}\hm= \{u_t, \ 0\hm\leq t\hm\leq T\}$ используется 
квадратичный целевой функционал:

\noindent
     \begin{multline}
     \hspace*{-5pt}J(U_0^{{T}})=E\left\{ 
     \int\limits_0^{{T}}\left( S_t\left( s_t y_t-g_t z_t -h_t u_t\right)^2 +G_t z_t^2+ {}\right.\right.\\
\left.\left.     {}+
H_t u_t^2
\vphantom{\left( \right)^2}
\right)dt + S_T\left( s_T y_T-g_T z_T\right)^2 +G_T z_T^2
\vphantom{\int\limits_0^{{T}}}
\right\}\,,
     \label{e3-bos}
     \end{multline}
где $S_t$, $G_t$ и~$H_t$~--- неотрицательные ограниченные функции.

\pagebreak

     Реализация полученных в~\cite{1-bos} соотношений для оптимального 
управления $(U^*)_0^{{T}}\hm= \{u_t^*,\ 0\hm\leq t\hm\leq T\}$, доставляющего 
минимум~$J(U_0^{{T}})$, требует решения дифференциальных уравнений 
в~частных производных параболического типа. В~\cite{2-bos} с~этой \mbox{целью} 
используется традиционный метод сеток, в~\cite{3-bos} к~свойствам 
оптимального управления в~части, связанной с~параболическим уравнением, 
добавляется формула Фейн\-ма\-на--Ка\-ца, на основании которой  
в~\cite{4-bos} реализуется приближенное решение на основании 
имитационного моделирования и~метода Мон\-те-Карло.
     
     В данной работе по отношению к~модели~(\ref{e1-bos}), (\ref{e2-bos}) 
сделаны два частных предположения, име\-ющих особое прикладное значение. 
Во-пер\-вых, уравнение~(\ref{e1-bos}) упрощается предположением 
о~линейности сноса~$A_t(y)$ при сохранении нелинейной 
диффузии~$\Sigma_t(y)$. Такие модели широко известны как линейные 
модели с~мультипликативными по состоянию возмущениями, их 
исследованию посвящено множество работ, обзор которых может 
представлять самостоятельный исследовательский интерес. В~качестве 
близкой к~тематике данной статьи можно указать, например,  
работу~\cite{6-bos}, где задача управления ставится в~терминах  
$H_\infty$-ро\-баст\-ности или, точнее, $H_2/H_\infty$-управ\-ле\-ния. Таких 
работ очень много, что связано как с~практической востребованностью, так 
и~с~наличием успешно применимых к~этим моделям исследовательских 
инструментов, в~отличие от моделей полностью нелинейных.

     
     Второе уточнение в~исходную модель вносит отказ от предположения о 
независимости возмущений~$v_t$ и~$w_t$. Такое уточнение уместно  
 с~учетом формирования линейной модели управляемой системы 
с~мультипликативными возмущениями. В~условиях полностью 
наблюдаемого состояния и~выхода присутствие в~уравнениях для~$y_t$ 
и~$z_t$ одного и~того же возмущения вполне возможно. Второе полезное 
приложение зависимые возмущения дадут при решении задачи с~неполной 
информацией, когда состояние~$y_t$ неизвестно, а~выход~$z_t$ 
интерпретируется как косвенные наблюдения. В~таких приложениях на 
первый план выходят задачи стохастической фильтрации. Общие постановки 
такого характера детально исследованы в~монографии~\cite{7-bos}.

%\vspace*{-12pt}
     
\section{Частная модель <<состояние--выход>> и~решение задачи 
оптимизации}

     Для формальной постановки задачи воспользуемся теми же 
уравнениями~(\ref{e1-bos}), (\ref{e2-bos}), внеся в~их описание следующие 
дополнения. Во-пер\-вых, будем предполагать, что функция сноса  
в~(\ref{e1-bos}) имеет вид:
     \begin{equation}
     A_t=A_t(y)=A_t^a y+A_t^b\,,
     \label{e4-bos}
\end{equation}
т.\,е.\ является линейной по аргументу~$y$. Неслучайные функции~$A_t^a$ 
и~$A_t^b$ предполагаются ограниченными, что обеспечивает выполнение 
условий Ито для существования решения~(\ref{e1-bos}).

     Второе уточнение состоит в~отказе от независимости винеровских 
процессов~$v_t$ и~$w_t$ в~(\ref{e1-bos}), (\ref{e2-bos}). Для описания этих 
возмущений введем два независимых стандартных винеровских процесса 
$\xi_t$ и~$\eta_t$ и~будем считать, что~$v_t$ и~$w_t$ порождаются следующим 
линейным преобразованием:
     \begin{equation}
     \begin{pmatrix}
     v_t\\ w_t \end{pmatrix} = \begin{pmatrix}
     1-\Lambda_t & \Lambda_t\\
     0& 1
     \end{pmatrix}
      \begin{pmatrix}
     \xi_t \\ \eta_t
     \end{pmatrix}\,,
     \label{e5-bos}
     \end{equation}
где функция $\Lambda_t$ предполагается ограниченной: $\vert \Lambda_t\vert 
\hm\leq 1$, $0\hm\leq t\hm\leq T$. Такое преобразование обеспечивает 
коррелированность возмущений: $E\{v_tw_t\}\hm= \Lambda_t t$, в~том числе 
допускается возможность не только зависимости, но и~равенства возмущений 
$v_t\hm=w_t$ при $\Lambda_t\hm=1$.

     Решение задачи, как и~в~общем случае~\cite{1-bos}, дает метод 
динамического программирования. Обозначив функцию Беллмана
     \begin{multline*}
     V_t= V_t(y,z) ={}\\
     {}= \mathop{\mathrm{inf}}\limits_{U_t^T} E \left \{
     \int\limits_0^{{T}} \!\!\left( S_\tau\left( s_\tau y_\tau -g_\tau z_\tau -h_\tau 
u_\tau \right)^2 +G_\tau z_\tau^2 +{}\right.\right.\hspace*{-2.35788pt}\\
\left.\left.{}+H_\tau u_\tau^2
\vphantom{\left( \right)^2}
\right)d\tau +
     S_T\left( s_T y_T -g_T z_T\right)^2 +G_T 
z_T^2\big\vert {\mathcal{F}_t^{y,z}}
\vphantom{\int\limits_0^{{T}}}
\right\}\!, \hspace*{-4.30814pt}
     \end{multline*}
где $\mathcal{F}_t^{y,z}$~--- $\sigma$-ал\-геб\-ра, порожденная $y_\tau, 
z_\tau$, $0\hm\leq \tau \hm\leq t$, $E\{\cdot \vert \mathcal{F}\}$~--- оператор 
условного математического ожидания относительно~$\mathcal{F}$, 
и~опуская аргументы у~функций~$A_t(y)$ и~$\Sigma_t(y)$, записываем 
уравнение динамического программирования
\begin{multline}
\fr{\partial V_t}{\partial t}+\fr{1}{2}\left( \left( \left( 1-\Lambda_t\right)^2 
+\Lambda_t^2\right)\Sigma_t^2\fr{\partial^2 V_t}{\partial y^2} +\sigma_t^2 
\fr{\partial^2 V_t}{\partial z^2}\right) +{}\\
{}+\Lambda_t\Sigma_t \sigma_t\fr{\partial^2 
V_t}{\partial y\partial z}+{}\\
{}+
\min\limits_u \left[ A_t\fr{\partial V_t}{\partial y} +\left( a_t y +b_t z+c_t u\right)
\fr{\partial V_t}{\partial z}+{}\right.\\
\left.{}+S_t\left( s_t y -g_t z-h_t u\right)^2+G_t z^2 +H_t 
u^2
\vphantom{\fr{\partial V_t}{\partial y}}
\right]=0\,,
\\
V_T(y,z)=S_T\left( s_T y-g_T z\right)^2+G_T z^2\,.
\label{e6-bos}
\end{multline}
     
    \noindent
     Отличие этого уравнения от полученного для случая независимых 
возмущений $v_t$ и~$w_t$ состоит в~до-\linebreak\vspace*{-12pt}

\pagebreak

\noindent
полнительном слагаемом $\Lambda_t 
\Sigma_t \sigma_t \partial^2 V_t/(\partial y\partial z$). При этом сохраняется 
возможность реализовать главное предположение, используемое для 
решения, а~именно: пред\-став\-ле\-ние функ\-ции Беллмана в~виде
     \begin{equation}
     V_t=\alpha_t z^2 +\beta_t(y) z+\gamma_t (y)\,,
     \label{e7-bos}
\end{equation}
так как $\partial^2 V_t/(\partial y\partial z)\hm= \partial \beta_t/\partial y$, что 
дает ту же структуру решаемого уравнения, что и~в~\cite{1-bos}, и~сводит 
задачу оптимизации к~поиску достаточно гладких функций~$\alpha_t$, 
$\beta_t(y)$ и~$\gamma_t(y)$, обеспечивающих решение  
уравнения~(\ref{e6-bos}). При этом оптимальное управ\-ле\-ние, доставляющее 
минимум~(\ref{e3-bos}), имеет тот же вид:
\begin{multline}
u_t^*=u_t^*(y_t,z_t)={}\\
{}=-\fr{1}{2}\left( S_t h_t^2+H_t\right)^{-1} \left( c_t \left( 
2\alpha_t z_t +\beta_t(y_t)\right)+{}\right.\\
\left.{}+2S_t \left( s_t y_t -g_t z_t\right)h_t\right)
\label{e8-bos}
\end{multline}
с учетом предположения $S_t h_t^2\hm+H_t\hm>0$ и~равенства $\partial 
V_t/\partial z \hm= 2\alpha_t z_t+\beta_t(y)$. 

     Уточнение уравнений, определяющих искомые коэффициенты 
$\alpha_t$, $\beta_t(y)$ и~$\gamma_t(y)$, в~связи  
с~предположением~(\ref{e5-bos}) о~коррелированности возмущений дает
     \begin{multline}
     \fr{d\alpha_t}{dt}+2\alpha_t \left( b_t -\left( S_t h_t^2 +H_t\right)^{-1} c_t 
S_t h_t g_t\right) +{}\\
     {}+\left( S_t -\left( S_t h_t^2 + H_t\right)^{-1} S_t^2 h_t^2\right) g_t^2+G_t-{}\\
     {} -\left( S_t h_t^2 +H_t\right)^{-1} c_t^2 \alpha_t^2 = 0\,,\enskip 
\alpha_T=S_T g_t^2+G_T\,;
     \label{e9-bos}
     \end{multline}
     
     \vspace*{-12pt}
     
     \noindent
     \begin{multline}
     \fr{\partial \beta_t}{\partial t} +A_t\fr{\partial \beta_t}{\partial y} 
+\fr{1}{2}\left( \left( 1-\Lambda_t\right)^2 +\Lambda_t^2\right) 
\Sigma_t^2 \fr{\partial^2\beta_t}{\partial y^2}+{}\\
     {}+M_t y+N_t\beta_t =0\,,\enskip \beta_T=-2S_T s_T g_T y\,;
     \label{e10-bos}
     \end{multline}
     
     \vspace*{-12pt}
     
     \noindent
     \begin{multline}
     \fr{\partial \gamma_t}{\partial t}+A_t\fr{\partial \gamma_t}{\partial y} 
+\fr{1}{2}\left( \left(1- \Lambda_t\right)^2 +\Lambda_t^2\right) \Sigma_t^2 
\fr{\partial^2 \gamma_t}{\partial y^2}+L_t=0\,,\\
     \gamma_T=S_T s_T^2 y^2\,,
     \label{e11-bos}
\end{multline}
где
\begin{align*}
M_t&=2\left(\alpha_t\left( a_t+\left( S_t h_t^2+H_t\right)^{-1} c_t S_t h_t s_t\right) 
- {}\right.\\
&\hspace*{10mm}\left.{}-
\left( S_t -\left( S_t h_t^2+H_t\right)^{-1} S_t^2 h_t^2\right) s_t g_t\right)\,;
\\
N_t&=b_t-\left( S_t h_t^2 +H_t\right)^{-1} c_t S_t h_t g_t -{}\\
&\hspace*{33mm}{}-\left( S_t 
h_t^2+H_t\right)^{-1} c_t^2 \alpha_t\,;\\
L_t&= L_t(y)=\Lambda_t \Sigma_t\sigma_t \fr{\partial\beta_t}{\partial y} 
+\sigma_t^2\alpha_t +{}\\
&\hspace*{10mm}{}+\beta_t\left(  a_t +\left( S_t h_t^2+H_t\right)^{-1} c_t S_t h_t s_t\right) y+{}\\
&\hspace*{15mm}{}+ \left( S_t- \left( S_t h_t^2+H_t\right)^{-1} S_t^2 h_t^2\right)s_t^2 y^2-{}\\
&\hspace*{35mm}{}-\fr{1}{4}\left( 
S_t h_t^2 +H_t\right)^{-1} c_t^2 \beta_t^2\,.
\end{align*}
     
     Остается учесть частное условие на модель~(\ref{e4-bos}). На 
уравнение Риккати~(\ref{e9-bos}) это предположение никак не влияет. Также 
не претерпит изменений параболическое уравнение~(\ref{e11-bos}).  
Принципиально изменится только уравнение~(\ref{e10-bos}). Действительно, 
поскольку в~этом уравнении коэффициенты~$M_t$ и~$N_t$ не зависят 
от~$y$, а~в~силу~(\ref{e4-bos}) $A_t(y)$ линейно по~$y$, то 
решение~(\ref{e10-bos}) можно искать в~виде 
$$
\beta_t= \beta_t(y)= 
\beta_t^a y + \beta_t^b\,,
$$
 т.\,е.\ в~виде линейной функции. Для 
уравнения~(\ref{e11-bos}) линейность~$A_t(y)$ дает единственное 
небольшое изменение~--- это явное представление $\partial\beta_t/\partial 
y\hm= \beta_t^a$ в~соотношении, определяющем свободный член~$L_t$. 
В~уравнении~(\ref{e10-bos}), прежде всего, устраняется лапласиан и~оно 
фактически перестает быть уравнением в~частных производных, 
преобразовываясь в~систему обыкновенных дифференциальных уравнений:
   \begin{multline*}
     \fr{d\beta_t^a}{dt}\,y+\fr{d\beta_t^b}{dt}+{}\\
     {}+\left( A_t^a 
y+A_t^b\right)\beta_t^a +M_t y+ N_t\left( \beta_t^a y+\beta_t^b\right) =0,
     \end{multline*}
откуда для~$\beta_t^a$ и~$\beta_t^b$ получается
\begin{equation}
\left.
\begin{array}{rl}
\fr{d\beta_t^a}{dt}+\left( A_t^a+N_t\right) \beta_t^a +M_t&=0\,,\\
&\hspace*{-7mm} \beta_T^a=-2S_T s_T g_T\,;\\[6pt]
\fr{d\beta_t^b}{dt}+A_t^b \beta_t^a +N_t \beta_t^b &=0\,, \ \beta_T^b=0\,,
\end{array}
\right\}
\label{e12-bos}
\end{equation}
т.\,е.\ система из двух линейных уравнений.

     Оптимальное управление~(\ref{e8-bos}) принимает вид:
     \begin{multline}
     u_t^*=-\fr{1}{2}\left( S_t h_t^2+H_t\right)^{-1} \left( 2 \left( c_t\alpha_t - 
S_t g_t h_t\right) z_t +{}\right.\\
\left.{}+\left( c_t\beta_t^a +2S_t s_t h_t\right) y_t +c_t 
\beta_t^b\right)\,.
     \label{e13-bos}
     \end{multline}
     
     Полученное решение, т.\,е.\ соотношения~(\ref{e12-bos}),  
(\ref{e13-bos}), характерно тем, что оптимальное управление 
в~рассмотренном частном случае может быть найде\-но аналитически~--- 
без численного решения параболического уравнения~(\ref{e11-bos}). 
Очевидно, что к~такому же результату приводит и~другой частный  
случай~--- классическая ли\-ней\-но-гаус\-сов\-ская постановка. В~ней 
функция Беллмана~(\ref{e7-bos}) является квадратичной, что означает 
и~линейность коэффициента~$\beta_t(y)$, и~квадратичность 
коэффициента~$\gamma_t(y)$ по~$y$, т.\,е.\ кроме того, что 
решение~(\ref{e10-bos}) может быть найдено в~виде $\beta_t\hm= \beta_t^a y 
\hm+ \beta_t^b$, еще и~решение~(\ref{e11-bos}) представляется в~виде 
\begin{equation}
\gamma_t= \gamma_t^a y^2 + \gamma_t^b y + \gamma_t^c\,. 
\label{e14a-bos}
\end{equation}

\pagebreak
     
     Рассмотренная постановка представляет более интересный частный 
вариант. Оказалось, что для линейности коэффициента~$\beta_t(y)$ 
достаточно линейного сноса~$A_t(y)$ в~(\ref{e1-bos}), а~линейность 
диффузии~$\Sigma_t(y)$ не требуется. В~отличие от  
ли\-ней\-но-гаус\-сов\-ско\-го случая функция Беллмана при этом не является 
квадратичной, решение уравнения~(\ref{e11-bos}) в~виде~(\ref{e14a-bos})
 не представляется 
именно из-за нелинейности~$\Sigma_t(y)$, но простое линейное 
представление оптимального управления~(\ref{e13-bos}) в~этом практически 
важном частном случае получается.

\vspace*{-5pt}
     
\section{Численный пример}

     Для иллюстрации рассмотренного частного случая линейного сноса 
в~модели состояния~(\ref{e1-bos}) оказались применимы выполненные 
в~работах~\cite{2-bos, 4-bos} эксперименты с~математической моделью для 
показателя RTT (Round-Trip Time) сетевого протокола TCP (Transmission 
Control Protocol), предложенной в~\cite{8-bos}, следующего вида:

\vspace*{-3pt}

\noindent
     \begin{multline}
     dy_t=\left( 1-0{,}1 y_t\right) dt +0{,}5\sqrt{y_t}\,dv_t\,,\\
      y_0=Y\sim 
N(15{,}9)\,.
     \label{e14-bos}
     \end{multline}
     
     \vspace*{-3pt}
     
     \noindent
Здесь $N(M,D)$~--- нормальное распределение со средним~$M$ и~дисперсией~$D$.
     
     Выход для~(\ref{e14-bos}) задается уравнением
     
     \vspace*{-3pt}
     
     \noindent
     \begin{multline}
     dz_t=0{,}1 y_t\,dt-z_t\,dt+u_t\,dt+dw_t\,,\\
      z_0=Z\sim N(9{,}9)\,,
     \label{e15-bos}
\end{multline}

\vspace*{-3pt}

\noindent
целевой функционал

\vspace*{-3pt}

\noindent
\begin{multline}
J\left(U_0^{{T}}\right) =E\left\{ \int\limits_0^{{T}} \left( \left( y_t-z_t\right)^2 +z_t^2 
+u_t^2\right)dt+{}\right.\\[-4pt]
\left.{}+\left( y_T - z_T\right)^2 +z_T^2
\vphantom{\int\limits_0^{{T}}}
\right\}.
\label{e16-bos}
\end{multline}

\vspace*{-3pt}
     
     Поскольку в~работе~\cite{2-bos} эксперименты с~данным примером 
(приближенное вычисление коэффициента~$\beta_t(y)$, оптимального 
управления~(\ref{e8-bos}) и~его качества~$J(U_0^t)$) проведены 
традиционным методом сеток, а~в~\cite{4-bos}~--- методом Мон\-те-Кар\-ло 
на основе формулы Фейн\-ма\-на--Ка\-ца, то здесь можно получить эталонное 
решение, вычислив~$\beta_t(y)$ по формулам~(\ref{e12-bos}), а~оптимальное 
управление~--- по формуле~(\ref{e13-bos}). Для этого вначале решается уравнение 
Риккати~(\ref{e9-bos}) для~$\alpha_t$. С~учетом того, что в~модели 
выхода~(\ref{e15-bos}) и~целевом функционале~(\ref{e16-bos}) все 
коэффициенты постоянны, это уравнение решается аналитически, а именно:
     \begin{equation*}
     \alpha_t= \fr{C_\alpha e^{2\sqrt{3}\,x}(1+\sqrt{3})-1+\sqrt{3}}{1-
C_\alpha e^{2\sqrt{3}\,x}}\,,
\end{equation*}

\noindent
где

\vspace*{-3pt}

\noindent
     \begin{equation*}
     C_\alpha= \fr{3-\sqrt{3}}{3+\sqrt{3}}\,e^{-10\sqrt{3}}\,.
     \end{equation*}
     
     Далее решалась система~(\ref{e12-bos}) линейных уравнений 
для~$\beta_t^a$ и~$\beta_t^b$:
     \begin{align*}
     \fr{d\beta_t^a}{dt}-\left( 1{,}1-\alpha_t\right)\beta_t^a 
+2\alpha_t+2&=0\,;\\
     \fr{d\beta_t^b}{dt} -\left( 1+\alpha_t\right) \beta_t^b+\beta_t^a&=0\,.
     \end{align*}
Для этого был использован неявный метод Эйлера с~шагом $\delta_t\hm= 
0{,}001$. Это тот же шаг, что использовался в~сеточном методе для расчета 
$\beta_t(y)$ в~\cite{2-bos} и~для дискретизации и~моделирования 
решений~(\ref{e14-bos}), (\ref{e15-bos}) в~\cite{4-bos}.

     Выполненный расчет иллюстрируется рис.~1, на котором изображен 
фрагмент поверхности~$\beta_t(y)$, и~рис.~2, на котором представлены 
отдельные сечения этой поверхности в~точках $t\hm=0{,}0$; 1,0; 2,0; 3,0; 4,0; 
4,1; 4,2; 4,3; 4,4; 4,5; 4,6; 4,7; 4,8; 4,9; 4,95; 5,0, т.\,е.\ образующие поверхность 
прямые $\beta_t^a y \hm+ \beta_t^b$.





     
     Визуальное сравнение поверхности на рис.~2 с~аппроксимациями, 
полученными в~\cite{2-bos, 4-bos}, пока-\linebreak\vspace*{-12pt}

{ \begin{center}  %fig1
 \vspace*{9pt}
    \mbox{%
 \epsfxsize=79.002mm 
 \epsfbox{bos-1.eps}
 }


\vspace*{6pt}

\noindent
{{\figurename~1}\ \ \small{
Поверхность $\beta_t(y)$
}}
\end{center}
}

%\vspace*{6pt}

{ \begin{center}  %fig2
 \vspace*{9pt}
    \mbox{%
\epsfxsize=73.519mm
\epsfbox{bos-2.eps}
}
\end{center}

\noindent
{{\figurename~2}\ \ \small{
Сечения поверхности $\beta_t(y)$:
      \textit{1}~--- $t\hm=0{,}0$; \textit{2}~--- 1,0; \textit{3}~--- 2,0;
      \textit{4}~--- 3,0; \textit{5}~--- 4,0;  \textit{6}~--- 4,1; \textit{7}~--- 4,2; 
      \textit{8}~--- 4,3; \textit{9}~--- 4,4; \textit{10}~--- 4,5;  \textit{11}~--- 4,6; 
      \textit{12}~--- 4,7; \textit{13}~--- 4,8; \textit{14}~--- 4,9; \textit{15}~--- 4,95; 
      \textit{16}~--- $t\hm=5{,}0$
}}
}
     
\noindent
зывает их полное сходство, 
относительные ошибки в~большинстве точек $(t,y)$ составляют 1\%--2\%, 
в~отдельных, особенно близких к~границе, достигают 8\%--12\%. Это 
свидетельствует о достаточно высокой точности методов, представленных 
в~\cite{2-bos, 4-bos}, так как в~данном расчете точность ограничена только 
шагом решения линейного дифференциального уравнения методом Эйлера, 
поэтому результат можно обоснованно считать эталонным и~использовать для 
сравнения с~действительно приближенными решениями  
из~\cite{2-bos, 4-bos}. Наконец, окончательный вывод о качестве 
аппроксимаций в~сравнении с~точным решением дает обеспечиваемое 
методом расчета оптимального управления значение целевого 
функционала~(\ref{e16-bos}). Заметим, что в~\cite{4-bos} показано 
преимущество метода Монте-Карло в~сравнении с~сеточной аппроксимацией, 
обеспечившего меньшее значение~$J(U_0^{{T}})$ в~терминальной точке на 
величину порядка~5\%. Вычисленное здесь точное оптимальное управление 
дает по критерию преимущество еще порядка~3\%.


     

\section*{Заключение}

     В рамках исследования задачи оптимизации линейного выхода 
нелинейной дифференциальной системы по квадратичному критерию, 
выполненному в~работах~\cite{1-bos, 2-bos, 3-bos, 4-bos}, получено решение задачи в~общем случае 
нелинейного уравнения Ито, описывающего состояние оптимизируемой 
системы. Для практической реализации оптимального управления 
приходится решать дифференциальное уравнение в~частных производных 
параболического типа, для чего имеются вполне эффективные чис\-лен\-ные 
методы, причем технически разные, что полезно, например, для контроля 
правильности выполненного расчета. Но открытым при этом остается вопрос 
точности аппроксимаций, т.\,е.\ бли\-зости приближенного решения 
к~точному. Для ответа на него требуется эталонная модель, пример, 
в~котором решение можно было бы вычислить точ\-но. Именно эта цель 
и~достигнута в~данной работе. Частный случай модели состояния, 
интерпретируемый как модель с~мультипликативными возмущениями, дал 
возможность решить действительно нелинейную задачу фактически 
аналитически и~не апеллировать к~классической  
ли\-ней\-но-гаус\-сов\-ской постановке. Таким образом, хотя бы на 
модельном примере удалось продемонстрировать эффективность 
предложенных ранее аппроксимаций рассматриваемой задачи.
     
     Также полезным для дальнейших исследований представляется 
и~возможность использования в~модели <<со\-сто\-яние--вы\-ход>> 
зависимых возмущений. Уточнение уравнений для оптимального решения 
станет в~дальнейшем основанием для рассмотрения постановок, 
предполагающих отказ от предположения о~наблюдаемости со\-сто\-яния, 
понимания выхода как косвенных наблюдений и~поиска управ\-ле\-ния как 
функции этих наблюдений.

{\small\frenchspacing
{%\baselineskip=10.8pt
%\addcontentsline{toc}{section}{References}
\begin{thebibliography}{9}
\bibitem{1-bos}
\Au{Босов А.\,В., Стефанович~А.\,И.} Управление выходом 
стохастической дифференциальной системы по квадратичному критерию. 
I.~Оптимальное решение методом динамического программирования~// 
Информатика и~её применения, 2018. Т.~12. Вып.~3. С.~99--106.

\bibitem{5-bos} %2
\Au{Флеминг У., Ришел~Р.} Оптимальное управление 
детерминированными и~стохастическими системами~/ Пер. с~англ.~--- М.: 
Мир, 1978. 316~с. (\Au{Fleming~W.\,H., Rishel~R.\,W.} Deterministic and 
stochastic optimal control.~--- New York, NY, USA: Springer-Verlag, 1975. 
222~p.)
\bibitem{2-bos} %3
\Au{Босов А.\,В., Стефанович~А.\,И.} Управление выходом 
стохастической дифференциальной системы по квадратичному критерию. 
II.~Численное решение уравнений динамического программирования~// 
Информатика и~её применения, 2019. Т.~13. Вып.~1. С.~9--15.
\bibitem{3-bos} %4
\Au{Босов А.\,В., Стефанович~А.\,И.} Управление выходом 
стохастической дифференциальной системы по квадратичному критерию. 
III.~Анализ свойств оптимального управления~// Информатика и~её 
применения, 2019. Т.~13. Вып.~3. С.~41--49.
\bibitem{4-bos} %5
\Au{Босов А.\,В., Стефанович~А.\,И.} Управление выходом 
стохастической дифференциальной системы по квадратичному критерию. 
IV.~Альтернативное численное решение~// Информатика и~её 
применения, 2020. Т.~14. Вып.~1. С.~24--30.

\bibitem{6-bos}
\Au{Chen B.\,S., Zhang~W.} Stochastic $H_2/H_1$ control with state-dependent 
noise~// IEEE T.~Automat. Contr., 2004. Vol.~49. Iss.~1. P.~45--56.
\bibitem{7-bos}
\Au{Липцер Р.\,Ш., Ширяев~А.\,Н.} Статистика случайных процессов 
(нелинейная фильтрация и~смежные вопросы).~--- М.: Наука, 1974. 696~с.
\bibitem{8-bos}
\Au{Bohacek S., Rozovskii~B.} A~diffusion model of roundtrip time~// 
Comput. Stat. Data An., 2004. Vol.~45. Iss.~1. P.~25--50.
\end{thebibliography}
}
}
\end{multicols}

\vspace*{-3pt}

\hfill{\small\textit{Поступила в~редакцию 12.12.2020}}

%\vspace*{8pt}

%\pagebreak

\newpage

\vspace*{-30pt}

%\hrule

%\vspace*{2pt}

%\hrule

%\vspace*{-2pt}

\def\tit{ON SOME SPECIAL CASES IN~THE~PROBLEM\\ OF~STOCHASTIC 
DIFFERENTIAL SYSTEM OUTPUT CONTROL\\ BY~THE~QUADRATIC CRITERION}

\def\titkol{On some special cases in~the~problem of~stochastic 
differential system output control by~the~quadratic criterion}

\def\aut{A.\,V.~Bosov}

\def\autkol{A.\,V.~Bosov}

\titel{\tit}{\aut}{\autkol}{\titkol}

\vspace*{-9pt}


\noindent
Institute of Informatics Problems, Federal Research Center ``Computer Science and
Control'' of the Russian Academy of Sciences, 44-2~Vavilov Str., Moscow 119333,
Russian Federation

\def\leftfootline{\small{\textbf{\thepage}
\hfill INFORMATIKA I EE PRIMENENIYA~--- INFORMATICS AND
APPLICATIONS\ \ \ 2021\ \ \ volume~15\ \ \ issue\ 1}
}%
\def\rightfootline{\small{INFORMATIKA I EE PRIMENENIYA~---
INFORMATICS AND APPLICATIONS\ \ \ 2021\ \ \ volume~15\ \ \ issue\ 1
\hfill \textbf{\thepage}}}

\vspace*{6pt}
     
      
     
      
      \Abste{A general study of the optimal control problem for the Ito diffusion process and 
linear controlled output with the quadratic quality criterion was carried out in
the author's previous works (coauthored by A.\,I.~Stefanovich). An analysis 
of the available results allows to single out some models that are of a particular nature in relation 
to the general setting but have special practical significance. This article examines two such 
particular models. The first model is determined by the assumption of linear drift in the equation 
of state while maintaining nonlinear diffusion. It is shown that such a model provides linearity to 
the optimal control and the absence of the need to solve a parabolic equation for its 
implementation. But in this case, the quadratic Bellman function does not appear in the problem, 
the corresponding expression, as in the general case, is described by the solution of a parabolic 
equation and retains a meaningful stochastic interpretation expressed by the Feynman--Katz 
formula. The second model implements the assumption about the dependence of disturbances in 
the equations of state and output. The modified dynamic programming equation is solved in the 
same way as in the general case considered in previous works, including and within the framework of 
a combined model involving both cases presented. This model will be especially useful in the
problems with incomplete information, when the assumption of the presence of complete 
information about the state and output will be replaced by a description of the observation 
system, in which the output is interpreted as indirect observations of the state. A~numerical 
example, studied in detail in the author's previous works
 (coauthored by A.\,I.~Stefanovich), is briefly discussed, since it turns out that 
it satisfies the assumption of linear drift in the equation of state and, accordingly, the previously 
obtained approximate solutions can be refined.}
      
      \KWE{stochastic differential equation; optimal control; system output control; stochastic 
differential systems with multiplicative and dependent disturbances}
      
      
      
\DOI{10.14357/19922264210102}

%\vspace*{-15pt}

\Ack
      \noindent
      This work was partially supported by the Russian Foundation for Basic Research (grant 
19-07-00187-A).

\vspace*{6pt}

  \begin{multicols}{2}

\renewcommand{\bibname}{\protect\rmfamily References}
%\renewcommand{\bibname}{\large\protect\rm References}

{\small\frenchspacing
 {%\baselineskip=10.8pt
 \addcontentsline{toc}{section}{References}
 \begin{thebibliography}{9}
\bibitem{1-bos-1}
     \Aue{Bosov, A.\,V., and A.\,I.~Stefanovich.} 2018. Upravlenie vykhodom 
stokhasticheskoy differentsial'noy sistemy po kvadratichnomu kriteriyu. I.~Optimal'noe reshenie 
metodom dinamicheskogo programmirovaniya [Stochastic differential system output control by 
the quadratic criterion. I.~Dynamic programming optimal solution]. \textit{Informatika i~ee 
Primeneniya~--- Inform. Appl.} 12(3):99--106.
{\looseness=1

}

\bibitem{5-bos-1}
     \Aue{Fleming, W.\,H., and R.\,W.~Rishel.} 1975. \textit{Deterministic and stochastic 
optimal control.} New York, NY: Springer-Verlag. 222~p.

\bibitem{2-bos-1} %3
     \Aue{Bosov, A.\,V., and A.\,I.~Stefanovich.} 2019. Upravlenie vykhodom 
stokhasticheskoy differentsial'noy sistemy po kvadratichnomu kriteriyu. II.~Chislennoe reshenie 
uravneniy dinamicheskogo programmirovaniya [Stochastic differential system output control by 
the quadratic criterion. II.~Dynamic programming equations numerical solution]. 
\textit{Informatika i~ee Primeneniya~--- Inform. Appl.} 13(1):9--15.
{\looseness=1

}
\bibitem{3-bos-1} %4
     \Aue{Bosov, A.\,V., and A.\,I.~Stefanovich.} 2019. Upravlenie vykhodom 
stokhasticheskoy differentsial'noy sistemy po kvadratichnomu kriteriyu. III.~Analiz svoystv 
optimal'nogo upravleniya [Stochastic differential system output control by the quadratic 
criterion. III.~Optimal control properties analysis]. \textit{Informatika i~ee Primeneniya~--- 
Inform. Appl.} 13(3):41--49.
\bibitem{4-bos-1} %5
     \Aue{Bosov, A.\,V., and A.\,I.~Stefanovich.} 2020. Upravlenie vykhodom 
stokhasticheskoy differentsial'noy sistemy po kvadratichnomu kriteriyu. IV.~Al'ternativnoe 
chislennoe reshenie [Stochastic differential system output control by the quadratic criterion. 
IV.~Alternative numerical decision]. \textit{Informatika i~ee Primeneniya~--- Inform. Appl.} 
14(1):24--30.
{\looseness=1

}

\bibitem{6-bos-1}
     \Aue{Chen, B.\,S., and W.~Zhang.} 2004. Stochastic $H_2/H_1$ control with  
state-dependent noise. \textit{IEEE T.~Automat. Contr.} 49(1):45--56.

\pagebreak

\bibitem{7-bos-1}
     \Aue{Liptser, R.\,S., and A.\,N.~Shiryaev.} 2001. \textit{Statistics of random processes 
II.~Applications.} Berlin: Springer-Verlag. 402~p.
\bibitem{8-bos-1}
     \Aue{Bohacek, S., and B.~Rozovskii.} 2004. A~diffusion model of roundtrip time. 
\textit{Comput. Stat. Data An.} 45(1):25--50.
 \end{thebibliography}

 }
 }

\end{multicols}

\vspace*{-3pt}

  \hfill{\small\textit{Received December~12, 2020}}


%\pagebreak

%\vspace*{-24pt}     
      
      \Contrl
      
      \noindent
      \textbf{Bosov Alexey V.} (b.\ 1969)~--- Doctor of Science in technology, principal 
scientist, Institute of Informatics Problems, Federal Research Center ``Computer Science and 
Control'' of the Russian Academy of Sciences, 44-2~Vavilov Str., Moscow 119333, Russian 
Federation; \mbox{AVBosov@ipiran.ru}
     
\label{end\stat}

\renewcommand{\bibname}{\protect\rm Литература}

      