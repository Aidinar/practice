%есть таблица и рисунки

\def\stat{abgaryan}

\def\tit{ПРИМЕНЕНИЕ МНОГОМАСШТАБНОГО ПОДХОДА\\ 
И~МЕТОДОВ АНАЛИЗА ДАННЫХ ДЛЯ~МОДЕЛИРОВАНИЯ 
ТЕПЛОПРОВОДНОСТИ В~СЛОИСТЫХ СТРУКТУРАХ$^*$}

\def\titkol{Применение многомасштабного подхода и~методов анализа 
данных для~моделирования теплопроводности} % в~слоистых структурах}

\def\aut{К.\,К.~Абгарян$^1$, И.\,С.~Колбин$^2$}

\def\autkol{К.\,К.~Абгарян, И.\,С.~Колбин}

\titel{\tit}{\aut}{\autkol}{\titkol}

\index{Абгарян К.\,К.}
\index{Колбин И.\,С.}
\index{Abgaryan K.\,K.}
\index{Kolbin I.\,S.}


{\renewcommand{\thefootnote}{\fnsymbol{footnote}} \footnotetext[1]
{Работа выполнена при частичной финансовой поддержке РФФИ (проекты 19-29-03051~мк 
и~19-08-01191~A).}}


\renewcommand{\thefootnote}{\arabic{footnote}}
\footnotetext[1]{Вычислительный центр им.\ А.\,А.~Дородницына Федерального 
исследовательского центра <<Информатика и управление>> Российской академии наук; 
Московский авиационный институт (национальный исследовательский университет), 
\mbox{kristal83@mail.ru}}
\footnotetext[2]{Вычислительный центр им.\ А.\,А.~Дородницына Федерального 
исследовательского центра <<Информатика и управление>> Российской академии наук; 
Московский авиационный институт (национальный исследовательский университет), 
\mbox{eugavrilov@gmail.com}}

\vspace*{-12pt}

     \Abst{Моделирование тепловых свойств слоистых структур в настоящее время стало 
востребованным направлением научных исследований. Это связано с постоянно растущей 
скоростью работы микроэлектронных элементов на основе слоистых структур, выделяющих при 
работе все большее количество энергии в виде тепла, которое требуется отводить, чтобы 
избежать перегрева и потери функциональных свойств устройств. В~работе представлен 
интеграционный подход, позволяющий объединить методы многомасштабного моделирования 
и анализа данных. Применение данного подхода дает возможность получить новое качество 
при решении задачи построения модели теплопереноса в~двухслойной структуре GaAs/AlAs. 
Показана эффективность применения методов машинного обучения для анализа за\-ви\-си\-мости 
эффективного коэффициента теплопроводности слоистых материалов от структурных 
особенностей и~внеш\-них факторов. Развитие предложенного подхода сможет обеспечить 
формирование информации для обоснованного подбора материалов слоистых структур для 
микроэлектронных устройств.}
     
     \KW{многомасштабное моделирование; интеграционный подход; слоистые структуры; 
предсказательное моделирование; кинетическое уравнение Больцмана; коэффициент 
теплопроводности; методы анализа данных}
     
\DOI{10.14357/19922264200413} 
  
%\vspace*{9pt}


\vskip 10pt plus 9pt minus 6pt

\thispagestyle{headings}

\begin{multicols}{2}

\label{st\stat}
     
\section{Введение}

    В настоящее время стремительное развитие машинного обучения в качестве 
мощного метода интеграции данных с множественной точ\-ностью и~выявления 
корреляций между взаимосвязанными явлениями дает возможность за 
ограниченное время находить решение сложных задач в~разных предметных 
областях. Технологии машинного обуче\-ния получили существенный импульс 
в~развитии за последнее десятилетие. В~настоящее время ведутся актив\-ные 
исследования в~об\-ласти применения алгоритмов машинного обучения в~задачах 
материаловедения~[1], например при исследовании \mbox{критических} явлений или 
внутренних корреляций квантовой сис\-те\-мы методом искусственных нейронных 
сетей, а~также при определении па\-ра\-мет\-ров потенциалов межатомного 
взаимодействия и~моделировании динамики решеток с~использованием  
ма\-шин\-но-обучен\-ных (полу)эмпирических потенциалов~[2]. Однако 
классические методы машинного обучения час\-то игнорируют фундаментальные 
законы физики, что приводит к~некорректным задачам или нефизичным 
решениям.
    
Сегодня можно говорить о том, что многомасштабное моделирование~--- 
    это успешная стратегия интеграции мультимасштабных, многофизических 
данных, которая позволяет раскрыть механизмы, объясняющие появление 
функциональных зависимостей при изучении физических явлений 
и~процессов~[1]. Применение технологии математического многомасштабного 
моделирования~[3, 4], согласно которой расчеты на каждом уровне проводятся 
с~использованием соответствующих математических моделей и~вычислительных 
алгоритмов, позволяет:\\[-15pt]
    \begin{itemize}
     \item объяснить многие явления и свойства объектов, включая 
исследование структурных особенностей физических явлений и процессов на 
нескольких масштабах;\\[-15pt]
     \item получать качественно новые результаты в~об\-ласти прогнозирования 
свойств новых объектов;\\[-15pt]
     \item решать задачи оптимизации состава и структуры многомасштабных 
объектов, выстраивать взаимосвязи между структурой и свойствами, что дает 
возможность синтезировать компози-\linebreak\vspace*{-12pt}
\end{itemize}

\pagebreak

\begin{itemize}
\item[ ] 
ционные структуры, обладающие заданным 
набором свойств.
    \end{itemize}
        При этом необходимо учитывать, что одно многомасштабное 
моделирование часто не дает возможности эффективно комбинировать большие 
наборы данных из разных источников и~с~разных \mbox{масштабных} уровней. Авторы 
последних научных пуб\-ли\-ка\-ций, включая~[1], показывают, как машинное 
обучение и~многомасштабное моделирование могут дополнять друг друга, 
создавая надежные прогностические модели. Они базируются на подходах, 
основанных на теоретическом фи\-зи\-ко-ма\-те\-ма\-ти\-че\-ском моделировании 
с~применением \mbox{математического} аппарата обыкновенных дифференциальных 
уравнений, уравнений в~част\-ных производных, а~также на подходах, 
основанных на методах анализа данных. Для достижения поставленных целей 
используется опыт, накопленный в~прикладной математике, информатике, 
вычислительной биологии, биохимии, биофизике, медицине и~в~других 
областях.
    
    Таким образом, как отмечается в~[1], меж\-дис\-цип\-ли\-нар\-ный подход 
предполагает, что интеграция машинного обучения и многомасштабного 
моделирования может дать новое представление о механизмах создания новых 
материалов, процессах разработки новых лекарств, о причинах заболеваний, 
помочь в определении новых целей и~стратегий лечения, а также в принятии 
решений на благо здоровья человека. 
    
    Использование интеграции методов машинного обучения и технологий 
многомасштабного моделирования (интеграционный подход) стало на сегодня 
одним из наиболее перспективных и быстроразвивающихся трендов в~области 
материаловедения. 
    
    Необходимо отметить, что при физико-ма\-те\-ма\-ти\-че\-ском моделировании 
структурных свойств материалов необходимо получить максимально точные 
характеристики, используя при этом \mbox{минимальный} объем эмпирических данных. 
Подобными свойствами обладают кван\-то\-во-ме\-ха\-ни\-че\-ские расчеты из 
первых принципов~[5]. Однако они отличаются существенной вычислительной 
сложностью, в~связи с~чем рассматриваемые сис\-те\-мы, как правило, ограничены 
размером в~сотни,\linebreak иногда тысячи атомов. В~данной работе 
продемонстрировано, что подходы машинного обучения\linebreak дают возможность 
проводить целенаправленные расчеты с~точ\-ностью, приближенной 
к~первопринципным расчетам для систем из миллионов атомов на существующем 
в~настоящий момент оборудовании за приемлемое время~[6].
    
    В данной работе для решения задачи о~теплопереносе в~слоистых 
структурах была разработана многомасштабная модель и~на ее основе 
многомасштабная композиция, в~которой были задействованы методы 
машинного обучения. Показано, что\linebreak такой интеграционный подход позволяет 
существенно ускорить вычислительно затратные расчеты по получению 
функциональных зависимостей между внеш\-ни\-ми па\-ра\-мет\-ра\-ми среды, а~так\-же 
внут\-рен\-ни\-ми параметрами сло\-истых структур и~значением эффективного 
коэффициента теп\-ло\-про\-вод\-ности.

\vspace*{-6pt}

\section{Модель теплопереноса в~слоистых структурах}

\vspace*{-4pt}

    Моделирование тепловых свойств слоистых структур в настоящее время 
стало востребованным направлением научных исследований, что связано 
с~постоянно растущей скоростью работы микроэлектронных элементов на основе 
слоистых структур, выделяющих все большее количество энергии в~виде тепла, 
которое требуется отводить, \mbox{чтобы} избежать перегрева и~потери 
функциональных свойств устройств~[7].
     
    Цель настоящей работы заключается в применении интеграционного 
подхода к задаче моделирования зависимости эффективного коэффициента 
теплопроводности слоистых структур, а именно сверхрешеток, от варьируемых 
параметров материала и~внеш\-ней среды. Для этого применялось прямое 
многомасштабное 
 фи\-зи\-ко-ма\-те\-ма\-ти\-че\-ское моделирование, проводилась генерация 
выборки на основе модели модального подавления~[8] с~дальнейшим обучением 
нейронных сетей на ней. На выходе алгоритма были получены компактные 
нейросетевые модели, которые сравнивались между собой по точ\-ности работы 
на тестовом наборе данных для определения лучшей сети.
    
    Основными переносчиками тепла в полупроводниках и диэлектриках 
служат квази\-час\-ти\-цы-фо\-но\-ны, описывающие колебания кристаллической 
ячейки. Стоит отметить, что применение классических подходов к~решению 
задач теплопроводности на основе закона Фурье для рас\-смат\-ри\-ва\-емых слоистых 
структур дает неудовлетворительные результаты, так как  при подобном подходе 
игнорируются  
кван\-то\-во-ме\-ха\-ни\-че\-ские эффекты в материалах, что дает сильное 
рассогласование с экспериментальными данными~[7]. Для построения моделей 
теплопереноса в слоистых структурах свою эффективность показали методы на 
основе решения кинетического уравнения Больцмана для фононов. При наличии 
теплового градиента распределение фононов может быть описано с помощью 
кинетического уравнения Больцмана:

\noindent
    \begin{equation*}
    \fr{df}{dt}= 
\fr{df}{dt}\left(\mathrm{diffusion}\right)+\fr{df}{dt}\left(\mathrm{scattering}\right)=0\,,
\end{equation*}
где
$$
    \fr{df}{dt}\left( \mathrm{diffusion}\right)=\nabla T v \fr{df}{dT}\,.
  $$
    
    Кинетическое уравнение Больцмана относится к сложным 
    ин\-тег\-ро\-диф\-фе\-рен\-ци\-аль\-ным уравнениям. Для достаточно небольшого температурного 
градиента распределение фононов может быть выражено в~приближении 
времени релаксации:
    \begin{equation*}
    \fr{f-f_0}{\tau^0}=-\nabla T v \fr{df_0}{dT}\,,\enskip
    f_0(\omega, T)=\fr{1}{e^{\hbar\omega/(kT)}}-1\,.
    \end{equation*}
 % 
    При этом одна из проблем при построении вычислительных алгоритмов 
связана с учетом рассеяния фононов. Однако во многих случаях при решении 
данного уравнения достаточно учитывать лишь приближения времени 
релаксации, что существенно упрощает задачу~[9]. В~такой постановке 
необходимо решить вопрос, связанный с~расчетом данных по параметрам 
релаксации. Ранее они вычислялись полуэмпирически с~учетом согласования 
модельных расчетов с~результатами экспериментов. 

Несмотря на определенные 
успехи данный подход весьма трудоемок. В~связи с этим его в основном 
применяли для моделирования структур однокомпонентных материалов (как 
правило, кремния и~германия)~[10, 11]. 

Значительный прорыв в~данном вопросе 
произошел при комбинировании методов с~использованием кинетического 
уравнения с~первопринципными  
расчетами~\cite{8-ab, 9-ab, 12-ab}. При этом требуемые\linebreak \mbox{характеристики} фононов 
могут быть получены не из аппроксимации экспериментальных %\linebreak
 данных, а~из 
первопринципных кван\-то\-во-ме\-ха\-ни\-че\-ских расчетов, что значительно 
повышает точность\linebreak вычислений, открывает возможности эффективного 
пред\-ска\-за\-ния свойств моделируемых {материалов}, минимизируя 
различные допущения. \mbox{Данные}, которые получают,~--- это координаты 
базисных атомов кристаллической ячейки, межатомные силовые константы для 
моделирования двухфононных взаимодействий, межатомные силовые 
константы для моделирования трехфононных \mbox{взаимодействий}, диэлектрический 
тензор и~эффективные заряды Борна для неаналитического поправочного члена.
    
    Более точное моделирование можно получить с использованием методов 
молекулярной динамики, однако это сопряжено с высокой вычислительной 
сложностью и нетривиальной задачей подбора оптимального потенциала~[13, 
14] и значений его параметров под конкретный материал. 
    
    Рассмотрим вычисление эффективного коэффициента теплопроводности 
бинарной гетероструктуры на примере сверхрешетки GaAs/AlAs. Для этого 
применим подход, описанный в работе~\cite{8-ab}. Расчет ведется в 
приближении виртуального кристалла~[15, 16]: 
    $$
    \fr{1}{\tau^0_\lambda}=\fr{1}{\tau_\lambda^{3\mathrm{ph}+}}+\fr{1} 
{\tau_\lambda^{3\mathrm{ph}-}}+\fr{1}{\tau^a_\lambda} +\fr{1}{\tau^b_\lambda}\,.
    $$
    Здесь $\lambda$~--- обобщенный индекс (фононная мода), который 
объединяет информацию по поляризации фононов~$p$ и по волновому 
вектору~$\mathbf{q}$~\cite{16-ab}; $\tau_\lambda^{3\mathrm{ph}+}$ описывает процессы 
адсорбции: один фонон из двух падающих~\cite{15-ab, 16-ab}; 
$\tau_\lambda^{3\mathrm{ph}-}$ описывает процессы эмиссии, в ходе которых один 
падающий фонон разделяется на два~\cite{15-ab, 16-ab}, $\tau_\lambda^a$~--- 
сплавной член, зависит от послойного распределения компонентов~[17]; 
$\tau_\lambda^b$~--- барьерный член~[17].
    
    Рассматривается послойное распределение материала. Для моделирования 
распределения материалов при росте сверхрешетки использовалась модель 
Мураки~[18]:
    $$
    \overset{\smile}{X}(i)=\begin{cases}
    1\,, & i<1\,;\\
    1-\phi\left( 1-R^i\right)\,, & 1\leq i\leq n_{\mathrm{ML}}\,;\\
    1-\phi\left(1-R^{n_{\mathrm{ML}}}\right)R^{i-n_{\mathrm{ML}}}\,,& i\geq n_{\mathrm{ML}}\,.
    \end{cases}
    $$
    %
    Здесь $R$~--- варьируемый параметр модели; $n_{\mathrm{ML}}$~--- число монослоев 
первого материала в периоде сверхрешетки.
    
    Схема расчета параметров релаксации пред\-став\-ле\-на на рис.~1.
    
     \begin{figure*} %fig1
     \vspace*{1pt}
    \begin{center}  
  \mbox{%
 \epsfxsize=144.603mm 
 \epsfbox{abg-1.eps}
 }
\end{center}
\vspace*{-11pt}
     \Caption{Схема расчета параметров релаксации сверхрешетки}
     \end{figure*}
     
    Для расчета эффективного коэффициента теп\-ло\-про\-вод\-ности 
использовалась модель модального подавления~[18]:
    $$
    \kappa(L)=\sum\limits_\lambda S_\lambda C_\lambda \| v_\lambda \| 
\Lambda_\lambda \cos^2\left(\theta_\lambda\right)\,.
    $$
Здесь 
$$
S_\lambda=\fr{1}{1+2K_\lambda}\,;\enskip 
\Lambda_\lambda= \| v_\lambda \| \tau^0_\lambda\,;\enskip 
K_\lambda=\fr{ \Lambda_\lambda \left(\cos(\theta_\lambda)\right)}{L}\,;
$$ 
\begin{equation*}
C_\lambda=\fr{k_B}{N\Omega}\,\fr{\hbar \omega_\lambda}{k_BT}\, 
f_0(f_0\hm+1),\quad
f_0\hm=f_0(\omega_\lambda, T);
\end{equation*} 
$\theta_\lambda$~--- угол между групповой скоростью~$v_\lambda$ 
фононной моды~$\lambda$ и осью теплопереноса.
    
    В отсутствие температурного градиента и иных термодинамических сил 
система находится в тепловом равновесии и распределение фононов подчинено 
закону Бо\-зе--Эйн\-штейна.

\section{Формирование обучающей выборки}

    Обучающая выборка сформирована по результатам расчетов эффективного 
коэффициента теплопроводности в~пакете Alma BTE с~варьированием 
следующих параметров:
    \begin{description}
    \item[\,] $R$~--- параметр модели Мураки, отвечающий за послойное 
распределение материалов в периоде сверхрешетки, варьировался от~0 до~0,9;
\item[\,] $x$~--- число монослоев первого материала (GaAs), варьировалось 
от~1 до~20;
\item[\,] $y$~--- число монослоев второго материала (AlAs), варьировалось 
от~1 до~20;
\item[\,] $T$~--- температура окружающей среды, варьировалась от~100 
до~500~K;
\item[\,] $L$~--- толщина сверхрешетки, варьировалась от~1~нм до~100~мкм. 
\end{description}

\section{Многомасштабная композиция для~моделирования 
теплопроводности в~слоистых~структурах}

    В данной работе при построении многомасштабной композиции для 
моделирования теплопроводности в~слоистых структурах используются 
подходы, разработанные ранее и представленные в~\cite{3-ab, 4-ab}.
     
    Покажем, как с помощью данной технологии при решении задачи 
    о~применении интеграционных подходов к~моделированию теплопроводности 
    в~слоистых структурах из конкретных базовых  
мо\-де\-лей-ком\-по\-зи\-ций (БК) составляются многомасштабные композиции 
(МК)~--- вычислительные аналоги многомасштабных моделей, передающие 
информационную сущность многомасштабных вычислительных  
процессов~\cite{3-ab, 4-ab}. При описании многомасштабной композиции и~базовых  
мо\-де\-лей-ком\-по\-зи\-ций, например $\mathbf{MC}_{1,1}^{A^i_{a_i}}: 
\left\{ V_0^{A^i_{a_i}}, X_0^{A^i_{a_i}}, \mathrm{MA}_0^{A^i_{a_i}}\right\}$, 
представляющих собой однопараметрическое семейство множеств разного 
структурного типа, объединенных в~общем вы\-чис\-ли\-тель\-ном процессе, где 
$V_0^{A^i_{a_i}}$~--- множество входных данных; 
$X_0^{A^i_{a_i}}$~--- множество выходных данных; $\mathrm{MA}_0^{A^i_{a_i}}$~--- 
множество, состоящее из моделей и алгоритмов, используем упрощенную 
форму записи: 
    $$
    \mathbf{MC}_{1,1}^{A^i_{a_i}}: \left\{ V_0^{A^i_{a_i}}, 
X_0^{A^i_{a_i}}, \mathrm{MA}_0^{A^i_{a_i}}\right\}\hm= 
\mathbf{MC}_{1,1}^{A^i_{a_i}}.
$$
    
    Рассмотрим расчет теплопроводности бинарной гетероструктуры 
GaAs/AlAs для разных периодов сверхрешетки. При построении 
многомасштабной композиции 
$$
\mathbf{MK}_{0,1,2}^{(A^i_{a_i}A^j_{a_j}/A^k_{a_k}A^j_{a_j})}\hm= 
\mathbf{MK}_{0,1,2}^{(\mathrm{GaAs/AlAs})}
$$ 
выделим три масштабных 
уровня $(0,\,1,\,2)$ и используем следующие обозначения.

 \begin{table*}[b]\small
     \begin{center}
     \tabcolsep=1pt
     \begin{tabular}{|c|l|c|}
     \multicolumn{3}{c}{Базовые модели-композиции, входящие в состав
    $\mathbf{MK}_{0,1,2}^{(A^i_{a_i}A^j_{a_j}/A^k_{a_k}A^j_{a_j})}$}\\
    \multicolumn{3}{c}{\ }\\[-6pt]
    \hline
    \tabcolsep=0pt\begin{tabular}{c} №\\ масштаб-\\ ного\\
уровня\end{tabular}&\multicolumn{1}{c|}{Обозначение и название БК}&
\tabcolsep=0pt\begin{tabular}{c}Название масштабного\\ уровня\end{tabular}\\
\hline
0&$\mathbf{MC}_0^{A^i_{a_i}}=\mathbf{MC}_0$ <<АТОМ $A_0^i$>>&
\tabcolsep=0pt\begin{tabular}{c}Уровень\\ 
химических элементов\end{tabular}\\
\hline
&&\\[-9pt]
1&\tabcolsep=0pt\begin{tabular}{l}
$\mathbf{MC}_{1,1}^{A^i_{a_i}A^j_{a_j}}=\mathbf{MC}_1^1$\ <<КРИСТАЛЛОХИМИЧЕСКАЯ ФОРМУЛА>>\\
$\mathbf{MC}_{1,2}^{A^i_{a_i}A^j_{a_j}}=\mathbf{MC}_1^2$\ <<КВАНТОВО-МЕХАНИЧЕСКАЯ ЯЧЕЙКА>>\end{tabular}&
\tabcolsep=0pt\begin{tabular}{c} Квантово-механический\\ уровень\end{tabular}\\
\hline
&&\\[-9pt]
2&\tabcolsep=0pt\begin{tabular}{l}
$\mathbf{MC}_{2,3}^{\mathrm{GaAs/AlAs}}=
\mathbf{MC}_{2,3}^{A^i_{a_i}A^j_{a_j}/A^k_{s_k}A^j_{a_j}}=
\mathbf{MC}_2^3$\ <<УРАВНЕНИЕ БОЛЬЦМАНА>>\\
$\boldsymbol{K}_{2,4}^{\mathrm{GaAs/AlAs}}=
\boldsymbol{K}_{2,4}^{A^i_{a_i}A^j_{a_j}/A^k_{a_k}A^j_{a_j}}=\boldsymbol{K}_2^4$ \ <<АППРОКСИМАЦИОННАЯ МОДЕЛЬ>>
\end{tabular}&Наноуровень\\
\hline
\end{tabular}
\end{center}
\end{table*}
    
    Базовая модель-ком\-по\-зи\-ция~--- $\mathbf{MC}_0^{A^i_{a_i}}$ 
($i$~--- номер элемента в таблице Менделеева), поставим ей в соответствие:
    \begin{description}
    \item[\,] $\mathbf{El}_0^{31}$~--- экземпляр для описания характеристик 
химического элемента Ga ($i\hm =31$);
     \item [\,] $\mathbf{El}_0^{33}$~--- экземпляр для описания характеристик химического 
элемента As ($i\hm=33$);
     \item[\,] $\mathbf{El}_0^{13}$~--- экземпляр для описания характеристик химического 
элемента Al ($i\hm=13$);
     \item[\,] $\mathbf{MC}_{1,1}^{\mathrm{GaAs}}$~--- экземпляр 
соответствующей БК для расчета кристаллохимической структуры GaAs; 
     \item[\,] $\mathbf{MC}_{1,1}^{\mathrm{AlAs}}$~--- экземпляр БК для расчета 
кристаллохимической структуры AlAs; 
    \item[\,] $\mathbf{MC}_{1,2}^{\mathrm{GaAs}}$~--- экземпляр БК для  
кван\-то\-во-ме\-ха\-ни\-че\-ско\-го расчета GaAs; 
     \item[\,] $\mathbf{MC}_{1,2}^{\mathrm{AlAs}}$~--- экземпляр БК для  
кван\-то\-во-ме\-ха\-ни\-че\-ско\-го расчета AlAs;
     \item[\,] $\mathbf{MC}_{2,3}^{\mathrm{GaAs/AlAs}}$~--- экземпляр БК для 
решения уравнения Больцмана;
     \item[\,] $\boldsymbol {K}_{2,4}^{\mathrm{GaAs/AlAs}}$~--- экземпляр композиции 
(формируется за счет объединения БК с~одного масштабного уровня, в~данном случае~--- 2-го) 
для определения регрессионной функции.
     \end{description}
    
     Общее представление о распределении БК 
     по масштабным уровням, задействованным в~вы\-чис\-ли\-тель\-ном процессе, 
показано в~таблице.
     
    
     
    Покажем, как организован вычислительный процесс.
    
    На нулевом масштабном уровне с помощью БК
    $\mathbf{MC}_0^1$ <<АТОМ  >> задаются основные данные по 
химическим элементам, входящим в~состав соединений, участвующих 
в~вы\-чис\-ли\-тель\-ном процессе (атомный номер химического элемента, масса атома, 
заряд ядра, радиус атома, электронная конфигурация, структура решетки и~др.). 
В~данном случае формируются экземпляры БК $\mathbf{MC}_0$ <<АТОМ~GA$_0^{31}$>>, 
$\mathbf{MC}_0$ <<АТОМ As$_0^{33}$>> и~$\mathbf{MC}_0$ <<АТОМ Al$_0^{13}$>>.
     
    К первому масштабному уровню отнесены БК 
    $\mathbf{MC}_1^1$ <<КРИСТАЛЛОХИМИЧЕСКАЯ 
ФОРМУЛА>> и~$\mathbf{MC}_1^2$ <<КВАН\-ТО\-ВО-МЕ\-ХА\-НИ\-ЧЕ\-СКАЯ ЯЧЕЙКА>>. 
C~их по\-мощью данные, полученные с~нулевого масштабного уровня из 
$\mathbf{MC}_0$<<АТОМ Ga$_0^{31}$>>, $\mathbf{MC}_0$<<АТОМ As$_0^{33}$>>
и~$\mathbf{MC}_0$<<АТОМ Al$_0^{13}$>>, передаются на  
кван\-то\-во-ме\-ха\-ни\-че\-ский уровень, первоначально в~$\mathbf{MC}_1^1$ 
<<КРИСТАЛЛОХИМИЧЕСКАЯ ФОРМУЛА>>, где, используя знания о~химическом 
составе и~крис\-тал\-ло\-гра\-фи\-че\-ской структуре,\linebreak определяется крис\-тал\-ло\-хи\-ми\-че\-ская 
структура соединения (мет\-ри\-че\-ские параметры кристалличе\-ской решетки, 
координаты базисных атомов и~др.). Данная БК программно реализована в~двух 
рас\-чет\-ных модулях (модель <<Плотная упаковка>> и~программный комплекс 
Materials Studio ({\sf  
https://www.\linebreak 3dsbiovia.com/products/collaborative-science/biovia-materials-studio})).
    
    Далее полученные в ходе вычислительного процесса данные передаются 
в~БК $\mathbf{MC}_1^2$ <<КВАН\-ТО\-ВО-МЕ\-ХА\-НИ\-ЧЕ\-СКАЯ ЯЧЕЙКА>>. 
Данная БК программно реализована в~двух расчетных модулях 
(программный комплекс VASP ({\sf https://www.\linebreak vasp.at}) и пакет программ 
с~открытым кодом Quantum Espresso ({\sf https://www.quantum-espresso.org})). 
Здесь на базе  
кван\-то\-во-ме\-ха\-ни\-че\-ской теории по каждому слою бинарной 
гетероструктуры уточняются параметры кристаллической решетки, 
рассчитывается электронная плот\-ность~\cite{7-ab}, полная энергия для заданной 
конфигурации базисных атомов, значения диэлектрических тензоров, 
эффективные заряды Борна, силовые константы 2-го и~3-го порядка 
рассчитываются как производные со\-от\-вет\-ст\-ву\-ющих порядков от энергии.
        
    На втором масштабном уровне, используя данные, полученные на 
предыдущем уровне, применяется модель <<виртуального кристалла>>, 
в~которой двухслойная структура рассматривается как сплав, однако при этом 
дополнительно учитывается послойное распределение материалов (модель 
Мураки)~\cite{17-ab}. Для расчета теплопроводности используется уравнение 
Больцмана с добавлением барьерного члена для учета приближения времени 
релаксации. Базовая  
мо\-дель-ком\-по\-зи\-ция $\mathbf{MC}_2^3$ <<УРАВНЕНИЕ БОЛЬЦМАНА>> 
программно реализована в расчетном модуле, в основе которого лежит 
комплекс AlmaBTE ({\sf www//https://almabte.bitbucket.io}).

     \begin{figure*} %fig2
     \vspace*{1pt}
    \begin{center}  
  \mbox{%
 \epsfxsize=154.934mm 
 \epsfbox{abg-2.eps}
 }
\end{center}
\vspace*{-6pt}
     \Caption{Схематическое представление многомасштабной композиции 
$\mathbf{MK}_{0,1,2}^{(A^i_{a_i}A^j_{a_j}/A^k_{a_k}A^j_{a_j})}$ для моделирования 
теп\-ло\-пе\-ре\-но\-са в~слоистой структуре GaAs/AlAs}
\vspace*{3pt}
     \end{figure*}
    
     В результате расчетов получаем значения эффективного коэффициента 
теплопроводности для двухслойной структуры в зависимости от периода (числа 
слоев GaAs и AlAs, т.\,е.\ распределения вещества) бинарной структуры, от 
ширины образца, от температуры внешней среды. Формируем набор расчетных 
данных, варьируя внешнюю температуру (например, от 100 до~500~K), меняем 
размеры периода, распределение веществ в~наногетероструктуре GaAs/AlAs, 
общую ширину сверхрешетки. Расчеты проводятся во вложенных циклах, таким 
образом собирается выборка. 
    
    Полученные наборы данных передаются в~экземпляр композиции 
$\boldsymbol{K}_{24}^{\mathrm{GaAs/AlAs}}$, которая формируется за счет 
объединения БК с~одного масштабного уровня, в~данном случае \mbox{2-го}, 
и~служит для определения регрессионной функции, описывающей 
функциональную за\-ви\-си\-мость изменения эффективного коэффициента 
теплопроводности бинарной структуры от варь\-и\-ру\-емых па\-ра\-мет\-ров. 
Определяется об\-ласть до\-пус\-ти\-мых значений, которые может принимать 
искомая функция. Используются программные модули,\linebreak в~основе которых лежат 
пакеты \mbox{TenzorFlow} ({\sf https://aws.amazon.com/ru/tensorflow}) и~\mbox{Pytorch} 
(https://pytorch.org). На рис.~2 пред\-став\-ле\-на архитектурная схема расчетных 
модулей многомасштабной композиции 
$\mathbf{MK}_{0,1,2}^{(A^i_{a_i}A^j_{a_j}/A^k_{a_k}A^j_{a_j})}$ 
и~основные потоки данных.
     

     
\section{Результаты вычислений, выводы}

\vspace*{-16pt}

    Были построены нейросетевые модели для расчета эффективного 
коэффициента теплопроводности для слоистых структур~--- сверхрешеток 
GaAs/AlAs с разными периодами слоев. Данные для обучения были 
сгенерированы в программном пакете \mbox{AlmaBTE}~1.3.2, параметры 
материалов получены из открытой базы данных проекта. Выборка 
формировалась для различных комбинаций содержания GaAs и~AlAs, толщин 
пленок, периодов сверхрешетки. Полученный массив данных был разделен на 
3~части: 60\% для обучения нейросетей, 20\% для валидации (во избежание 
переобучения) и 20\% как тестовая выборка для оценки результирующей 
точности обученных моделей. Оптимизация нейросетей велась с 
использованием алгоритма \mbox{RMSprop} с шагом~0,0001 в среде 
\mbox{Tensorflow}~2.3.
{\looseness=1

}
    
    В качестве модели для расчета использовались многослойные нейронные 
сети прямого распространения. В~работе были рассмотрены сети с~различным 
числом скрытых слоев, также проведены сравнения для варьирующего числа 
нейронов и~распространенных активационных функций. Полученные 
в~результате обучения сети сравнивались по среднеквадратичной ошибке. 
    %
    Относительная среднеквадратичная ошибка при этом составила  
порядка 3\%--5\%, что позволяет говорить о~достаточно хорошей точности 
выбранного подхода.
{ %\looseness=1

}
    
    Разработанные подходы могут быть использованы при решении обратных 
задач для предсказательного моделирования структурных характеристик 
слоистых материалов с~заданными значениями эффективного коэффициента 
теплопроводности.
    
{\small\frenchspacing
 {%\baselineskip=10.8pt
 %\addcontentsline{toc}{section}{References}
 \begin{thebibliography}{99}
     
     \bibitem{1-ab}
     \Au{Mark A., Tepole A.\,B., Cannon~W.\,R., \textit{et al.}} Integrating machine learning and 
multiscale modeling-perspectives, challenges, and opportunities in the biological, biomedical, and behavioral 
sciences~// NPJ Digital Medicine, 2019. Vol.~2. Iss.~1. Art. No.\,115. 11~p. doi:  
10.1038/s41746-019-0193-y.
     \bibitem{2-ab}
     \Au{Ladygin V., Korotaev P., Yanilkin~A., Shapeev~A.} Lattice dynamics simulation using machine 
learning interatomic potentials~// Comp. Mater. Sci., 2020. Vol.~172. Art. ID: 109333.
     \bibitem{3-ab}
\Au{Абгарян К.\,К}. Многомасштабное моделирование в задачах структурного  
материаловедения.~--- М.: МАКС Пресс, 2017. 284~с.
     \bibitem{4-ab}
     \Au{Абгарян К.\,К.} Информационная технология по\-стро\-ения многомасштабных моделей 
     в~задачах вычислительного материаловедения~// Системы высокой доступности, 2018. Т.~15. 
№\,2. С.~9--15.
     \bibitem{5-ab}
     \Au{Kohn W., Sham L.\,J.} Self-consistent equations including exchange and correlation effects~// 
Phys. Rev., 1965. Vol.~140. Iss.~4A. P.~A1133--A1138. doi: 10.1103/\mbox{PhysRev}.140.A1133.
     \bibitem{6-ab}
     \Au{Jia Weile, Wang Han, Chen Mohan, \textit{et al.}} 
     Pushing the limit of molecular dynamics with \textit{ab initio} 
accuracy to 100 million atoms with machine learning.~--- Cornell University, 2020. arXiv:2005.00223 
[physics.comp-ph]. {\sf 
https://arxiv.org/pdf/2005.00223.pdf}.
     \bibitem{7-ab}
     \Au{Xвесюк В.\,И., Скрябин~А.\,С.} Теплопроводность наноструктур~// Теплофизика 
высоких температур, 2017. Т.~55. №\,3. С.~446--471. doi: 10.7868/ S0040364417030127.
     \bibitem{8-ab}
     \Au{Vermeersch B., Carrete~J., Mingo~N.} Cross-plane heat conduction in thin films with \textit{ab-initio} 
phonon dispersions and scattering rates~// Appl. Phys. Lett., 2016. Vol.~108. Iss.~19. Art. ID: 193104. doi: 
10.1063/1.4948968.
     \bibitem{9-ab}
\Au{Carrete J., Vermeersch~B., Katre~A., Roekeghem~A., Wang~T., Madsen~G., Mingo~N.} 
АlmaBTE: A~solver of the space--time dependent Boltzmann transport equation for phonons in structured 
materials~// Comput. Phys. Commun., 2017. Vol.~220C. P.~351--362. doi: 10.1016/j.cpc.2017.06.023.
     
     \bibitem{11-ab} %10
\Au{Chung J.\,D., McGaughey~A.\,J.\,H., Kaviany~M.} Role of phonon dispersion in lattice thermal 
conductivity modeling~// J.~Heat Transf., 2004. Vol.~126. Iss.~3. P.~376--380. doi: 10.1115/1.1723469.

\bibitem{10-ab} %11
\Au{Loy J.\,M., Murthy~J.\,Y., Singh~D.\,A.} Fast hybrid Fourier--Boltzmann transport equation solver for 
nongray phonon transport~// J.~Heat Transf., 2012. Vol.~135. Iss.~1. Art. ID: 011008. doi: 
10.1115/1.4007654.

     \bibitem{12-ab}
\Au{Broido D.\,A., Malorny~M., Birner~G., Mingo~N., Stewart~D.\,A.} Intrinsic lattice thermal 
conductivity of semiconductors from first principles~// Appl. Phys. Lett., 2007. Vol.~91. Iss.~23. Art. 
ID: 231922. doi: 10.1063/1.2822891.
     \bibitem{13-ab}
\Au{Powell D.} Elasticity, lattice dynamics and parameterisation techniques for the Tersoff potential applied 
to elemental and type III--V semiconductors.~--- Sheffield, U.K.: University of Sheffield, 2006. PhD Thesis. 
252~p.
     \bibitem{14-ab}
     \Au{Abgaryan K.\,K., Mutigullin~I.\,V., Uvarov~S.\,I., Uvarova~O.\,V.} Multiscale modeling of 
clusters of point defects in semiconductor structures~// CEUR Workshop Proceedings~/
Eds. S.\,I.~Smagin, A.\,A.~Zatsarinnyy.~--- Khabarovsk, 2019. Vol.~2426. P.~43--51.
     \bibitem{15-ab}
     \Au{Li W., Lindsay L., Broido~D.\,A., Stewart~D.\,A., Mingo~N.} Thermal conductivity of bulk and 
nanowire Mg$_2$Si$_х$Sn$_{1-х}$ alloys from first principles~// Phys. Rev.~B, 2012. Vol.~86. Iss.~17. Art. 
ID: 174307. doi: 10.1103/\mbox{PhysRevB}.\linebreak 86.174307.
     \bibitem{16-ab}
     \Au{Li W., Carrete~J., Katcho~N.\,A., Mingo~N.} A~solver of the Boltzmann transport equation for 
phonons~// Comput. Phys. Commun., 2014. Vol.~185. Iss.~6. P.~1747--1758. doi: 
10.1016/j.cpc.2014.02.015.
     \bibitem{17-ab}
     \Au{Carrete J., Vermeersh~B., Thumfart~L., Kakodkar~R.\,R., Trevisi~G., Frigeri~P., 
Seravalli~L., Feser~J.\,P., Rastelli~A., Mingo~N.} Predictive design and experimental realization of 
InAs/GaAs superlatices with tailored thermal conductivity~// J.~Phys. Chem.~C, 2018. Vol.~122. Iss.~7.  
P.~4054--4062. doi: 10.1021/acs.jpcc.7b11133.
     \bibitem{18-ab}
\Au{Muraki K., Fukatsu~S., Shiraki~Y., Ito~R.} Surface segregation of In atoms during molecular beam 
epitaxy and its influence on the energy levels in InGaAs/GaAs quantum wells~// Appl. Phys. Lett., 1992. 
Vol.~61. Iss.~5. P.~557--559. doi: 10.1063/1.107835.
\end{thebibliography}

 }
 }

\end{multicols}

\vspace*{-3pt}

\hfill{\small\textit{Поступила в~редакцию 05.10.20}}

%\vspace*{8pt}

%\pagebreak

\newpage

\vspace*{-28pt}

%\hrule

%\vspace*{2pt}

%\hrule

%\vspace*{-2pt}

\def\tit{APPLICATION OF~MULTISCALE APPROACH AND~DATA SCIENCES 
FOR~MODELING THERMAL CONDUCTIVITY IN~LAYERED STRUCTURES}


\def\titkol{Application of multiscale approach and data sciences for modeling 
thermal conductivity in layered structures}


\def\aut{K.\,K.~Abgaryan$^{1,2}$ and I.\,S.~Kolbin$^{1,2}$}

\def\autkol{K.\,K.~Abgaryan and I.\,S.~Kolbin}

\titel{\tit}{\aut}{\autkol}{\titkol}

\vspace*{-11pt}


\noindent
$^1$A.\,A.~Dorodnicyn Computing Center, Federal Research Center ``Computer Science and Control'' of 
the Russian\linebreak
$\hphantom{^1}$Academy of Sciences, 40~Vavilov Str., Moscow 119333, Russian Federation

\noindent
$^2$Moscow Aviation Institute (National Research University), 4~Volokolamskoe Shosse, Moscow 
125080, Russian\linebreak
$\hphantom{^1}$Federation

\def\leftfootline{\small{\textbf{\thepage}
\hfill INFORMATIKA I EE PRIMENENIYA~--- INFORMATICS AND
APPLICATIONS\ \ \ 2021\ \ \ volume~15\ \ \ issue\ 1}
}%
 \def\rightfootline{\small{INFORMATIKA I EE PRIMENENIYA~---
INFORMATICS AND APPLICATIONS\ \ \ 2021\ \ \ volume~15\ \ \ issue\ 1
\hfill \textbf{\thepage}}}

     \vspace*{3pt}
    
\noindent

\Abste{Modeling thermal properties of layered structures is currently 
a~popular area of scientific research. This is due to the constantly growing speed 
of operation of microelectronic elements often based on layered structures 
that release more and more energy during operation in the form of heat which must 
be removed to avoid overheating and loss of functional properties of devices.
 The paper presents an integration approach that allows one to combine the methods of 
 multiscale modeling and data analysis. It is shown that application of this approach 
 makes it possible to obtain a~new quality when solving the problem of constructing 
 a~model of heat transfer in a~two-layer GaAs/AlAs structure. The effectiveness of use of
  machine learning methods for analyzing the dependence of the effective thermal conductivity 
  coefficient of laminated materials on structural features and external factors is shown. 
  The development of the proposed approach will be able to provide formation of information 
for reasonable selection of materials for layered structures for microelectronic devices.}

\KWE{multiscale modeling; integration approach; layered structures; predictive modeling; kinetic Boltzmann 
equation, thermal conductivity coefficient; data analysis methods}

\DOI{10.14357/19922264200413} 

%\vspace*{-20pt}

\Ack
\noindent
The work was partially supported by the Russian Foundation for Basic Research, projects 19-29-03051~mk 
and 19-08-01191~A).


%\vspace*{6pt}

  \begin{multicols}{2}

\renewcommand{\bibname}{\protect\rmfamily References}
%\renewcommand{\bibname}{\large\protect\rm References}

{\small\frenchspacing
 {%\baselineskip=10.8pt
 \addcontentsline{toc}{section}{References}
 \begin{thebibliography}{99}

\bibitem{1-ab-1}
  \Aue{Mark, A., A.\,B.~Tepole, W.\,R.~Cannon, \textit{et al.}} 2019. Integrating machine learning and 
multiscale modeling-perspectives, challenges, and opportunities in the biological, biomedical, and behavioral 
sciences. \textit{NPJ Digital Medicine} 2(1):115. 11~p. doi: 10.1038/s41746-019-0193-y.
\bibitem{2-ab-1}
\Aue{Ladygin, V., P.~Korotaev, A.~Yanilkin, and A.~Shapeev.} 2020. Lattice dynamics simulation using 
machine learning interatomic potentials. \textit{Comp. Mater. Sci.} 172:109333.
\bibitem{3-ab-1}
\Aue{Abgaryan, K.\,K.} 2017. \textit{Mnogomasshtabnoe modelirovanie v~zadachakh strukturnogo 
materialovedeniya} [Multiscale modeling in material science problems]. Moscow: MAKS Press. 284 p.
\bibitem{4-ab-1}
  \Aue{Abgaryan, K.\,K.} 2018. Informatsionnaya tekhnologiya postroeniya mnogomasshtabnykh modeley 
v~zadachakh vychislitel'nogo materialovedeniya [Information technology in the construction of multiscale models 
in problems of computational materials science]. \textit{Sistemy vysokoy do\-stup\-nosti} [High Availability 
Systems] 14(2):9--15.
\bibitem{5-ab-1}
  \Aue{Kohn, W., and L.\,J.~Sham.} 1965. Self-consistent equations including exchange and correlation 
effects. \textit{Phys. Rev.~A} 140(4A):A1133--A1138. doi: 10.1103/\mbox{PhysRev}.140.A1133.
\bibitem{6-ab-1}
  \Aue{Jia, Weile, Han~Wang, Mohan Chen, \textit{el al.}} 2020. Pushing the limit of molecular dynamics with 
  \textit{ab initio} 
accuracy to 100~million atoms with machine learning. 
Cornell University. arXiv:2005.00223 [physics.comp-ph]. Available at: {\sf 
https://arxiv.org/pdf/2005.00223.pdf} (accessed November~9, 2020).
\bibitem{7-ab-1}
  \Aue{Khvesyuk, V.\,I., and A.\,S.~Skryabin.} 2017. Heat 
conduction in nanostructures. \textit{High 
Temp.} 55(3):434--456. doi: 10.1134/S0018151X17030129.
\bibitem{8-ab-1}
  \Aue{Vermeersch, B., J.~Carrete, and N.~Mingo.} 2016. Cross-plane heat conduction in thin films with 
  \textit{ab-initio} phonon dispersions and scattering rates. \textit{Appl. Phys. Lett.} 108(19):193104. doi: 
10.1063/1.4948968.
\bibitem{9-ab-1}
  \Aue{Carrete, J., B. Vermeersch, A.~Katre, A.~Roekeghem, T.~Wang, G.~Madsen, and N.~Mingo.} 
2017. АlmaBTE: A~solver of the space-time dependent Boltzmann transport equation for phonons in structured 
materials. \textit{Comput. Phys. Commun.} 220C:351--362. doi: 10.1016/j.cpc.2017.06.023.

\bibitem{11-ab-1}
  \Aue{Chung, J.\,D., A.\,J.\,H.~McGaughey, and M.~Kaviany.} 2004. Role of phonon dispersion in lattice 
thermal conductivity modeling. \textit{J.~Heat Transf.} 126(3):376--380. doi: 10.1115/1.1723469.

\bibitem{10-ab-1}
  \Aue{Loy, J.\,M., J.\,Y.~Murthy, and D.\,A.~Singh.} 2012. Fast hybrid Fourier--Boltzmann transport 
equation solver for nongray phonon transport. \textit{J.~Heat Transf.} 135(1):011008. doi: 
10.1115/1.4007654.

\bibitem{12-ab-1}
  \Aue{Broido, D.\,A., M.~Malorny, G.~Birner, N.~Mingo, and D.\,A.~Stewart.} 2007. Intrinsic lattice 
thermal conductivity of semiconductors from first principles. \textit{Appl. Phys. Lett.} 91(23):231922. doi: 
10.1063/1.2822891. 
\bibitem{13-ab-1}
  \Aue{Powell, D.} 2006. Elasticity, lattice dynamics and parameterization techniques for the Tersoff potential 
applied to elemental and type III--V semiconductors. Sheffield: University of Sheffield. PhD Thesis. 252~p.
\bibitem{14-ab-1}
\Aue{Abgaryan, K.\,K., I.\,V.~Mutigullin, S.\,I.~Uvarov, and O.\,V.~Uvarova.} 2019. Multiscale modeling of 
clusters of point defects in semiconductor structures. \textit{CEUR Workshop Proceedings}.  
Eds. S.\,I.~Smagin and A.\,A.~Zatsarinnyy. Khabarovsk. 2426:43--51.
\bibitem{15-ab-1}
  \Aue{Li, W., L. Lindsay, D.\,A.~Broido, D.\,A.~Stewart, and N.~Mingo.} 2012. Thermal conductivity of 
bulk and nanowire Mg$_2$Si$_х$Sn$_{1-х}$ alloys from first principles. \textit{Phys. Rev.~B} 
86(17):174307. doi: 10.1103/PhysRevB.86.174307.
\bibitem{16-ab-1}
  \Aue{Li, W., J. Carrete, N.\,A.~Katcho, and N.~Mingo.} 2014. A~solver of the Boltzmann transport 
equation for phonons. \textit{Comput. Phys. Commun.} 185(6):1747--1758. doi: 
10.1016/j.cpc.2014.02.015.
\bibitem{17-ab-1}
  \Aue{Carrete, J., B.~Vermeersh, L.~Thumfart, R.\,R.~Kakodkar, G.~Trevisi, P.~Frigeri, L.~Seravalli, 
J.\,P.~Feser, A.~Rastelli, and N.~Mingo.} 2018. Predictive design and experimental realization of InAs/GaAs 
superlatices with tailored thermal conductivity. \textit{J.~Phys. Chem.~C} 
122(7):4054--4062. doi: 10.1021/acs.jpcc.7b11133.
\bibitem{18-ab-1}
  \Aue{Muraki, K., S.~Fukatsu, Y.~Shiraki, and R.~Ito.} 1992. Surface segregation of In atoms during 
molecular beam epitaxy and its influence on the energy levels in \mbox{InGaAs}/\mbox{GaAs} quantum wells. \textit{Appl. Phys. 
Lett.} 61(5):557--559. doi: 10.1063/1.107835.
\end{thebibliography}

 }
 }

\end{multicols}

\vspace*{-3pt}

\hfill{\small\textit{Received October 15, 2020}}

%\pagebreak

%\vspace*{-24pt}

\Contr

  \noindent
  \textbf{Abgaryan Karine K.} (b.\ 1963)~---
  Doctor of Science in physics and mathematics, principal scientist, 
A.\,A.~Dorodnicyn Computing Center, Federal Research Center ``Computer Science and Control'' of the 
Russian Academy of Sciences, 40~Vavilov Str., Moscow 119333, Russian Federation; Head of Department, 
Moscow Aviation Institute (National Research University), 4~Volokolamskoe Shosse, Moscow 125080, 
Russian Federation; \mbox{kristal83@mail.ru}
  
  \vspace*{6pt}
  
  \noindent
  \textbf{Kolbin Ilya S.} (b.\ 1986)~--- Candidate of Science (PhD) in physics and mathematics, scientist, 
A.\,A.~Dorodnicyn Computing Center, Federal Research Center ``Computer Science and Control'' of the 
Russian Academy of Sciences, 40~Vavilov Str., Moscow 119333, Russian Federation; senior lecturer, Moscow 
Aviation Institute (National Research University), 4~Volokolamskoe Shosse, Moscow 125080, Russian 
Federation; \mbox{eugavrilov@gmail.com}
\label{end\stat}

\renewcommand{\bibname}{\protect\rm Литература} 
    